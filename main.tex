\documentclass[12pt]{article}

%%%%%%%%%%%%%%%%%%%%%%%% Balíky %%%%%%%%%%%%%%%%%%%%%%%%%%%%%%%%%%%%%
\usepackage[utf8]{inputenc}
\usepackage[T1]{fontenc}
\usepackage[czech]{babel} 
\usepackage{amsmath, amsfonts, amssymb, amsthm}
\usepackage{graphicx}
\usepackage{booktabs} % Pre profesionálne vyzerajúce tabuľky
\usepackage{geometry}
\geometry{a4paper, margin=2.5cm}
\usepackage[colorlinks=true, linkcolor=black, citecolor=red]{hyperref}
\usepackage{titlesec}
\titleformat{\section}{\large\bfseries}{\thesection}{1em}{}
\usepackage{lmodern} 
%\usepackage{newtxtext,newtxmath}
%\usepackage{microtype}
\usepackage{pgfplots}
\usepackage{float}
\usepackage{multirow}
\usepackage{subcaption}
\usepackage{verbatim}
\usepackage{float}


\pgfplotsset{compat=1.18}
\usetikzlibrary{arrows.meta}

%%%%%%%%%%%%%%%%%%%%%%% Obsah %%%%%%%%%%%%%%%%%%%%%%%%%%%%%%%%%%%%%%%
\usepackage{tocloft}
\renewcommand{\cftsecdotsep}{\cftdotsep}
\renewcommand{\cfttoctitlefont}{\Large\bfseries}
\renewcommand{\cftaftertoctitle}{\vspace{0.5cm}}
\setlength{\cftsecindent}{0pt}      % Sekcia bez odsadenia
\setlength{\cftsubsecindent}{1.5em} % Podsekcia odsadená
\setlength{\cftsubsubsecindent}{3em} % Podpodsekcia odsadená viac
\renewcommand{\cftsecpagefont}{\normalfont}
\renewcommand{\cftsubsecpagefont}{\normalfont}




%%%%%%%%%%%%%%%%%%%%% Okraje a Odsadeni %%%%%%%%%%%%%%%%%%%%%%%%%%%%%

\geometry{
    a4paper,
    total={170mm,257mm},
    left=2.5cm,
    right=2.5cm,
    top=5cm,      % Toto spraví ten veľký odskok zhora ako v PDF
    bottom=5cm    % Spodný okraj pre číslovanie
}

%%%%%%%%%%%%%%%%%%%%%%%%%%%%%%%%%%%%%%%%%%%%%%%%%%%%%%%%%%%%%%%%%%%%%

\begin{document}

\begin{titlepage}
    \centering
    \vspace*{2cm} % Odsadenie zhora, aby bol hlavný názov približne v strede hornej polovice

    % Hlavný názov
    {\huge Projekt – numerické metódy v $\mathbb{R}$ a $\mathbb{R}^n$ \par}
    
    \vspace{0.8cm} % Medzera medzi názvom a predmetom
    
    % Názov predmetu - stredne veľké písmo
    {\Large Matematické programovanie (M5170) \par}
    
    \vspace{0.8cm} % Väčšia medzera pred menami autorov
    
    % Mená autorov 
    {\large Ján Húska, Radim Čech \par}
    
    \vfill % Vyplnenie miesta, aby sa číslo strany dostalo na spodok

    % Číslo strany na titulnom liste 
    \vfill % Toto vytlačí zvyšok textu úplne na spodok strany
    
    % Dátum a číslo strany
    {\large január 2026 \par}
    \vspace{0.5cm}
    {\large 1 \par}
\end{titlepage}


\pagenumbering{arabic}
\setcounter{page}{2}
%%%%%%%%%%%%%%%%%%%%%%%%%%%%%%%%%%%%%%%%%%%%%%%%%%%%%%%%%%%%%%%%%%%%%

%Obsah 
\tableofcontents
\newpage


%%%%%%%%%%%%%%%%%%%%%%%%%%%%%%%%%%%%%%%%%%%%%%%%%%%%%%%%%%%%%%%%%%%%%
\section{Analýza numerických metód pre funkciu jednej premennej}

Na analýzu správania numerických metód pre funkciu jednej premennej sme si zvolili funkciu
$$ f(x) = (x-2)^2 + \frac{1}{2}\sin(3x) - \frac{1}{3}\cos(2x) + 1, $$
ktorá je spojitá na celom $\mathbb{R}$. Svoje „presné“ minimum táto funkcia nadobúda v bode
$$ x^* \doteq 1{,}74261110151914722 $$
s funkčnou hodnotou
$$ f(x^*) \doteq 0{,}9450058166282782830. $$
Zvolená funkcia pozostáva z kvadratického člena a goniometrických zložiek, čo jej dodáva charakteristický priebeh v tvare vlnitej paraboly. Dominantný vplyv kvadratickej časti zabezpečuje, že oscilácie trigonometrických členov nenarúšajú celkový trend a funkcia si v širšom okolí minima zachováva unimodalitu. Tiež vidíme, že pre jej deriváciu platí $f'(x) = 2(x - 2) + \frac{3}{2}\cos(3x) + \frac{2}{3}\sin(2x)$. Keďže rovnica $f'(x) = 0$ je transcendentná a nemá analytické riešenie, predstavuje táto funkcia vhodný testovací model pre porovnanie efektívnosti numerických metód.
Na nasledujúcej strane je vykreslený graf tejto funkcie pre obmedzený rozsah hodnôt $x$ a $y$.

Pre tento projekt sme vybrali \textbf{metódu rozpoľovania intervalu} a \textbf{metódu zlatého rezu}, ktorých správanie budeme na tejto funkcii skúmať.\footnote{Na analýzu numerických metód v $\mathbf{R}$ používame MapleCloud skripty z https://www.math.muni.cz/~zemanekp/vyuka.html}



\newpage
%%%%%%%%%%%%%%%%%%%%%%%%%%%%%%%%%%%%%%%%%%%%%%%%%%%%%%%%%%%%%%%%%%%%%
% --- 4. STRANA: GRAF FUNKCIE 
\thispagestyle{plain}
\vspace*{\fill} 

\begin{figure}[h!]
    \centering
    \begin{tikzpicture}
        \begin{axis}[
            width=16cm, height=12cm,    % ZVÄČŠENIE GRAFU na maximum strany
            axis lines=middle,
            xlabel={$x$}, ylabel={$f(x)$},
            xlabel style={at={(ticklabel* cs:1)}, anchor=north west},
            ylabel style={at={(ticklabel* cs:1)}, anchor=south west},
            xmin=-2.5, xmax=6.5,
            ymin=-2, ymax=18,
            grid=major,
            grid style={dotted, gray!50},
            domain=-2.2:6.2,
            samples=200,
            trig format plots=rad,
            legend style={at={(0.95,0.95)}, anchor=north east, draw=none, fill=none},
            tick label style={font=\footnotesize}
        ]
            
            % 1. Funkcia
            \addplot[color=blue!80!black, thick] 
            { (x-2)^2 + 0.5*sin(3*x) - (1/3)*cos(2*x) + 1 };
            \addlegendentry{$f(x)$}

            % 2. Minimum (bodka)
            \addplot[mark=*, color=red, mark size=2.5pt, only marks] 
            coordinates {(1.7876, 0.9959)};

            % 3. Vodiace čiary
            \draw[dashed, gray] (axis cs:1.7876, 0) -- (axis cs:1.7876, 0.9959);
            \draw[dashed, gray] (axis cs:0, 0.9959) -- (axis cs:1.7876, 0.9959);

            % 4. Popis minima 
            \node[
                anchor=west,           % Zarovnanie textu
                fill=white,            % Biele pozadie (aby bol text čitateľný)
                font=\footnotesize,
                inner sep=2pt,         % Odsadenie textu od okraja
                draw=gray!30,          % Jemný rámček (voliteľné)
                rounded corners        % Zaoblené rohy rámčeka
            ] (popis) at (axis cs:2.5, 4) {Minimum $[1{,}7426; 0{,}9450]$};

            % Šípka od popisu k bodu (aby bolo jasné, kam patrí)
            \draw[->, gray, thick, shorten >=2pt] (popis.south west) -- (axis cs:1.85, 1.1);

        \end{axis}
    \end{tikzpicture}
    \caption{Graf funkcie $ f(x) = (x-2)^2 + \frac{1}{2}\sin(3x) - \frac{1}{3}\cos(2x) + 1. $}
\end{figure}

\vspace*{\fill}
\newpage
%%%%%%%%%%%%%%%%%%%%%%%%%%%%%%%%%%%%%%%%%%%%%%%%%%%%%%%%%%%%%%%%%%%%%

\subsection{Metóda rozpoľovania intervalu}

\subsubsection{Analýza správania metódy pri rôznych hodnotách počtu vyčíslení}

Najprv preskúmame vplyv počtu vyčíslení na presnosť riešenia. Konkrétne zvolíme $N \in \{4, 10, 20, 50, 100\}$, keďže použitá metóda je definovaná pre párny počet funkčných hodnôt.


V tejto časti analýzy sa zameriame na to, ako voľba počtu vyčíslení $N$ ovplyvňuje výslednú presnosť riešenia. Aby sme mohli objektívne sledovať vplyv parametra $N$, pre túto sériu výpočtov zafixujeme počiatočný interval. Vychádzame z faktu, že hľadané minimum sa nachádza v bode $x^* \doteq 1{,}7456125$, preto volíme interval $I_0 = [-6; 10]$. Tento rozsah je voči bodu minima približne symetrický. Z metodického hľadiska je dôležité, aby bod optima neležal v bezprostrednej blízkosti krajov intervalu. Ak by sme totiž zvolili počiatočné ohraničenie s krajným bodom blízko $x^*$, algoritmus by vykonával takmer výlučne jednostrannú kontrakciu intervalu, čo by mohlo skresliť sledované správanie metódy. Navyše, v praxi presnú polohu minima vopred nepoznáme, preto je voľba intervalu, kde sa riešenie nachádza hlbšie v jeho vnútri, realistickejším predpokladom. 


Pre korektné fungovanie algoritmu je potrebné definovať aj parameter $\delta$, určujúci vzdialenosť testovacích bodov od stredu intervalu. Teoretická podmienka $\delta < \frac{1}{2}(b-a) = 8$ je v našom prípade splnená voľbou $\delta = 0{,}01$. Ak by sme zvolili parameter $\delta$ príliš malý, hrozilo by, že funkčné hodnoty v testovacích bodoch budú v rámci numerickej presnosti takmer identické. To by mohlo skomplikovať rozhodovanie o smere ďalšieho kroku. Na druhej strane je naším cieľom držať $\delta$ na čo najnižšej úrovni, aby bola metóda efektívna a výsledný interval lokalizácie minima (ILM) bol čo najužší. Práve preto sme zvolili hodnotu $\delta = 0{,}01$. Považujeme ju za rozumný kompromis – je dostatočne malá na to, aby výsledný interval čo najtesnejšie ohraničoval minimum, no zároveň poskytuje dostatočný odstup na to, aby algoritmus vedel spoľahlivo určiť smer ďalšieho kroku. Pri tomto nastavení môžeme v nasledujúcej časti objektívne porovnať, ako sa mení presnosť výsledkov pri rôznom počte vyčíslení $N$, pričom všetky ostatné vstupné parametre zostávajú rovnaké.

\vspace{0.5cm}

\noindent \underline{\textbf{N = 4}}

\vspace{0.5cm}
Pre $N=4$ vyčíslení (2 kroky) dostávame výsledný ILM približne ako $I_2 = [-2{,}005; 2{,}01]$ a apriórny odhad maximálnej chyby je $\epsilon_{max} \doteq 2{,}0075$. Aproximáciu minima dostávame ako bod $\tilde{x} \doteq 0{,}0025$ s funkčnou hodnotou $f(\tilde{x}) \doteq 4{,}660427048$. Skutočná chyba je potom $\epsilon \doteq 1{,}740111102$. Vidíme, že po iba dvoch krokoch metódy je chyba stále značná, čo je vzhľadom na šírku pôvodného intervalu očakávateľné. 

\newpage

\noindent \underline{\textbf{N = 10}}

\vspace{0.5cm}

Pre $N=10$ vyčíslení dostávame výsledný ILM približne ako $I_5 = [1{,}490625; 2{,}01]$ a apriórny odhad maximálnej chyby je $\epsilon_{max} \doteq 0{,}2596875$. Aproximáciu minima dostávame ako bod $\tilde{x} \doteq 1{,}7503125$ s funkčnou hodnotou $f(\tilde{x}) \doteq 0{,}945195918$. Skutočná chyba je potom $\epsilon \doteq 0{,}007701398$. 

Zaujímavým zistením je rádový pokles skutočnej chyby medzi $N=4$ a $N=10$. Zatiaľ čo pri dvoch krokoch ($N=4$) sa aproximácia nachádzala ešte v časti intervalu s chybou v rádoch jednotiek ($\epsilon \approx 1{,}74$), po piatich krokoch ($N=10$) klesla skutočná chyba o tri rády na úroveň tisícin ($\epsilon \approx 0{,}0077$). Tento vyše 200-násobný nárast presnosti demonštruje, že po počiatočnej fáze hrubej lokalizácie začína metóda veľmi efektívne konvergovať k stacionárnemu bodu. 

\vspace{0.5cm}

\noindent \underline{\textbf{N = 20}}

\vspace{0.5cm}
Pre $N=20$ vyčíslení dostávame výsledný ILM ako $I_{10} = [1{,}724707031; 1{,}7603125]$ a apriórny odhad maximálnej chyby klesá pod $0{,}0178027$. Aproximáciu minima dostávame ako bod $\tilde{x} \doteq 1{,}742509766$ s funkčnou hodnotou $f(\tilde{x}) \doteq 0{,}9959$. Skutočná chyba je $\epsilon \doteq 0{,}000101336$. 

Vidíme, že algoritmus už prekonal fázu „hľadania“ v širokom počiatočnom intervale a plne sa stabilizoval v konvexnej oblasti funkcie. Zaujímavým javom je, že hoci sa dĺžka intervalu stále zmenšuje podľa teoretického predpokladu, začíname narážať na spodnú hranicu presnosti určenú parametrom $\delta = 0{,}01$.
\vspace{0.5cm}

\noindent \underline{\textbf{N = 50}}

\vspace{0.5cm}

Pre $N=50$ vyčíslení dostávame výsledný interval približne ako 
$I_{25} = [1{,}732631684; \allowbreak 1{,}752632160]$ a apriórny odhad maximálnej chyby je $\epsilon_{max} \doteq 0{,}010000023812$. Aproximáciu minima dostávame ako bod $\tilde{x} \doteq 1{,}742631922$ s funkčnou hodnotou $f(\tilde{x}) \doteq 0{,}9450581673$. Skutočná chyba je potom $\epsilon \doteq 0{,}000020820480852$. 

Vidíme, že aproximácia sa spresnila na úroveň piatich desatinných miest ($10^{-5}$). Hoci je odhad chyby $\epsilon_{max}$ stále na úrovni $0{,}01$, reálna poloha minima $\tilde{x} \doteq 1{,}74263$ je už veľmi blízko skutočnému minimu funkcie. Funkčná hodnota $f(\tilde{x}) \doteq 0{,}945058$ taktiež zaznamenala pokles a ustálila sa v tesnej blízkosti globálneho minima.
Možno konštatovať, že akékoľvek ďalšie navyšovanie počtu vyčíslení pri zachovaní $\delta = 0{,}01$ už nepovedie k zmenšeniu apriórnej chyby, pretože tá je limitovaná samotnou konštrukciou metódy, nie počtom iterácií. Tento stav demonštruje vysokú efektivitu výpočtu, no zároveň odhaľuje jeho presnostný strop. Ďalej sa naskytuje otázka, či je vôbec potrebné učiniť väčšie vyčíslenia, pretože si tým už pravdepodobne moc neprilepšíme.
\vspace{0.5cm}

\noindent \underline{\textbf{N = 100}}

\vspace{0.5cm}
Pre $N=100$ dostávame výsledný ILM približne ako $I_{50} = [1{,}732631689; 1{,}752631690]$ a apriórny odhad maximálnej chyby je $\epsilon_{max} \doteq 0{,}0100000000$. Aproximáciu minima dostávame ako bod $\tilde{x} \doteq 1{,}742631690$ s funkčnou hodnotou $f(\tilde{x}) \doteq 0{,}9450581673$. Skutočná chyba je potom $\epsilon \doteq 0{,}00002058848085$. 

Výsledky pre 100 vyčíslení vykazujú takmer úplnú identitu s predchádzajúcim meraním. Apriórny odhad maximálnej chyby dosiahol svoju limitnú hodnotu $\epsilon_{max} \doteq 0{,}01$, čo exaktne zodpovedá zvolenému parametru $\delta$. Skutočná chyba $\epsilon \doteq 0{,}00002058$ sa oproti $N=50$ zmenila až na deviatom desatinnom mieste, čo je z hľadiska praktickej numerickej analýzy zanedbateľný posun. Z hľadiska efektivity výpočtu môžeme konštatovať, že pre danú funkciu a konfiguráciu parametrov je $N=100$ už neefektívnym „overkillom“. 

V nadväznosti na doterajšie pozorovania sa v ďalšej analýze obmedzíme na dve reprezentatívne hodnoty $N=10$ a $N=50$. Tento výber nám umožňuje sledovať dostatočný kontrast v presnosti bez zbytočnej redundancie výpočtov. Zatiaľ čo $N=10$ reprezentuje fázu úspešnej počiatočnej lokalizácie minima, hodnota $N=50$ už predstavuje dosiahnutie praktického limitu presnosti algoritmu pri zvolenej hodnote $\delta$.

Pre zaujímovosť (bez slovnej analýzy) ešte uvádzame tabuľku pre rôzne hodnoty parametru delta $\delta:$

\begin{figure}[H]
    \centering
    \includegraphics[width=0.65\textwidth]{grafy/vhyna.png}
    \label{fig:moj_obrazok}
\end{figure}




%%%%%%%%%%%%%%%%%%%%%%%%%%%%%%%%%%%%%%%%%%%%%%%%%%%%%%%%%%%%%%%%%%%%%

\subsubsection{Analýza správania metódy pri rôznych východiskových intervaloch lokalizácie minima}

V nasledujúcej časti zanalyzujeme vplyv šírky počiatočného intervalu na celkovú presnosť lokalizácie optima pri $\delta=0,01$. Náš pôvodný interval $I_0 = [-6; 10]$ budeme rôzne modifikovať. Najskôr ho zúžime na polovicu (t.j. $[-3; 5]$) a následne ho v dvakrát rozšírime (t.j. $[-12; 20]$). Motiváciou pre tento experiment je štruktúra apriórneho odhadu chyby, ktorý vykazuje priamu závislosť od vstupnej dĺžky intervalu $(b-a)$. Zo vzťahu pre dĺžku výsledného intervalu po $k$ krokoch $$l_k = \frac{b-a}{2^k} + \delta \frac{2^k - 1}{2^{k-1}}$$je zrejmé, že redukcia šírky počiatočného intervalu by mala pri konštantnom $N$ viesť k vyššej presnosti, zatiaľ čo jeho rozšírenie bude mať za následok nárast chyby aproximácie. 

Ďalej sa pozrieme na ďalšie iné zmeny, ako je napríklad nesymetrický interval s minimom pri okraji $I_{asym1} = [ 1{,}7; 15 ]$, nesymetrickým intervalom s minimom mimo okraja $I_{asym2} = [ -6; 21 ]$, mikro intervalom $I_{micro} = [ 1{,}73; 1{,}76 ]$ a obrovským intervalom $I_{giant} = [ -100; 100 ]$.
\newpage

\noindent \underline{\textbf{Kratší počiatočný interval}}
\vspace{0.5cm}

Pre východiskový interval $I_0 = [-3; 5]$ a $N=10$ vyčíslení dostávame výsledný ILM približne ako $I_5 = [1{,}48875; 1{,}7584125]$. Apriórny odhad maximálnej chyby je $\epsilon_{max} \doteq 0{,}1346875$. Aproximáciu minima dostávame ako bod $\tilde{x} \doteq 1{,}62334375$ s funkčnou hodnotou $f(\tilde{x}) \doteq 0{,}9795089721$. Skutočná chyba je v tomto prípade $\epsilon \doteq 0{,}1191736015$.

\vspace{0.3cm}

Pre východiskový interval $I_0 = [-3; 5]$ a $N=50$ vyčíslení dostávame výsledný ILM približne ako $I_5 = [1{,}732631536; 1{,}752631774]$. Apriórny odhad maximálnej chyby je $\epsilon_{max} \doteq 0{,}0100001$. Aproximáciu minima dostávame ako bod $\tilde{x} \doteq 1{,}742631655$ s funkčnou hodnotou $f(\tilde{x}) \doteq 0{,}9450581673$. Skutočná chyba je v tomto prípade $\epsilon \doteq 0{,}000020553$.

\vspace{0.3cm}

Ak porovnáme výsledky pre $N = 10$, vidíme, že zmenšenie intervalu na polovicu nám znížilo apriórnu chybu o cca $0{,}125$, čo je pri takom malom počte krokov naozaj výrazný posun k lepšiemu. Naopak pri $N = 50$ je už rozdiel medzi pôvodným a týmto kratším intervalom prakticky nulový (rozdiel je len v rádoch $10^{-8}$), pretože v tomto bode už vplyv počiatočnej šírky intervalu vo vzorci pre $\ell_k$ takmer úplne zaniká v prospech zvolenej delty. Z týchto meraní je teda jasné, že snažiť sa o čo najužší počiatočný interval má zmysel hlavne vtedy, keď chceme minimum nájsť rýchlo na pár krokov, zatiaľ čo pri dlhšom výpočte sa táto počiatočná výhoda postupne úplne vytráca.

\vspace{0.5cm}
\noindent \underline{\textbf{Dlhší počiatočný interval}}
\vspace{0.5cm}


Pre východiskový interval $I_0 = [-12; 20]$ a $N=10$ vyčíslení dostávame výsledný ILM približne ako $I_5 = [0{,}991875; 2{,}01125]$. Apriórny odhad maximálnej chyby je $\epsilon_{max} \doteq 0{,}5096875$. Aproximáciu minima dostávame ako bod $\tilde{x} \doteq 1{,}5015625$ s funkčnou hodnotou $f(\tilde{x}) \doteq 1{,}089329089$. Skutočná chyba je v tomto prípade $\epsilon \doteq 0{,}241048601$.


\vspace{0.3cm}

Pre východiskový interval $I_0 = [-12; 20]$ a $N=50$ vyčíslení dostávame výsledný ILM približne ako $I_5 = [1{,}732631257; 1{,}752632210]$. Apriórny odhad maximálnej chyby je $\epsilon_{max} \doteq 0{,}0100005$. Aproximáciu minima dostávame ako bod $\tilde{x} \doteq 1{,}742631734$ s funkčnou hodnotou $f(\tilde{x}) \doteq 0{,}9450581672$. Skutočná chyba je v tomto prípade $\epsilon \doteq 0{,}000020632$.

\vspace{0.3cm}

To potvrdzuje teoretický predpoklad, že ak o polohe minima nič nevieme a musíme zvoliť široký interval, zaplatíme za to nižšou presnosťou alebo nutnosťou vyššieho počtu iterácií.
Pre N=10 je výsledný interval neistoty podstatne širší v porovnaní s kratším počiatočným intervalom a skutočná chyba nadobúda vyššie hodnoty, hoci stále zostáva menšia než apriórny odhad maximálnej chyby. Pri zvýšení počtu vyčíslení na N=50 však dochádza k výraznému zlepšeniu presnosti a výsledný interval sa prakticky zhoduje s výsledkami získanými pre kratší počiatočný interval.

\vspace{0.5cm}
\noindent \underline{\textbf{Nesymetrický východiskový interval s minimom pri okraji}}
\vspace{0.5cm}


Zameráme sa teraz na interval, ktorého jeden z krajných bodov nie je od skutočného bodu minima príliš ďaleko, teda na $I_0 = [1.7; 15]$. V úvode sme spomínali, že takýto interval sa bude pri metóde prevažne skracovať z jednej strany.

\vspace{0.3cm}

Pre východiskový interval $I_0 = [1.7; 15]$ a $N=10$ vyčíslení dostávame výsledný ILM približne ako $I_5 = [1{,}7; 2{,}135]$. Apriórny odhad maximálnej chyby je $\epsilon_{max} \doteq 0{,}2175$. Aproximáciu minima dostávame ako bod $\tilde{x} \doteq 1{,}9175$ s funkčnou hodnotou $f(\tilde{x}) \doteq 1{,}010101541$. Skutočná chyba je v tomto prípade $\epsilon \doteq 0{,}174888984793$.

\vspace{0.3cm}

Pre východiskový interval $I_0 = [1.7; 15]$ a $N=50$ vyčíslení dostávame výsledný ILM približne ako $I_5 = [1{,}732631636; 1{,}752632032]$. Apriórny odhad maximálnej chyby je $\epsilon_{max} \doteq 0{,}0100001$. Aproximáciu minima dostávame ako bod $\tilde{x} \doteq 1{,}742631834$ s funkčnou hodnotou $f(\tilde{x}) \doteq 0{,}9450581673$. Skutočná chyba je v tomto prípade $\epsilon \doteq 0{,}0000207324$.

\vspace{0.3cm}

Hoci bol ľavý krajný bod už od začiatku vynikajúcim odhadom, pri $N=10$ sme stále skončili s chybou okolo 0,22. Je to jasný dôkaz toho, že algoritmus o skutočnej polohe minima na začiatku „netuší“ a musí sa k nemu aj tak postupne prepracovať cez postupné skracovanie dlhej pravej strany intervalu. Opäť sa nám však potvrdilo, že pri $N=50$ sme narazili na ten istý presnostný strop 0,01 ako v predchádzajúcich prípadoch. Ukazuje sa teda, že ani takáto výrazná nápoveda na štarte nepomôže metóde prekonať jej vlastné limity. Bez ohľadu na to, ako veľmi asymetricky začneme, metóda sa pri dostatočnom počte vyčíslení vždy ustáli na rovnakom výslednom ILM, ktorý je pevne daný našou hodnotou delta.

\vspace{0.5cm}
\noindent \underline{\textbf{Nesymetrický východiskový interval s minimom ďalej od okraja} }
\vspace{0.5cm}


Pre východiskový interval $I_0 = [-6; 21]$ a $N=10$ vyčíslení dostávame výsledný ILM približne ako $I_5 = [1{,}588125; 2{,}45125]$. Apriórny odhad maximálnej chyby je $\epsilon_{max} \doteq 0{,}4315625$. Aproximáciu minima dostávame ako bod $\tilde{x} \doteq 2{,}0196875$ s funkčnou hodnotou $f(\tilde{x}) \doteq 1{,}097043863$. Skutočná chyba je v tomto prípade $\epsilon \doteq 0{,}27707639848$.

\vspace{0.3cm}

Pre východiskový interval $I_0 = [-6; 21]$ a $N=50$ vyčíslení dostávame výsledný ILM približne ako $I_5 = [1{,}732631056; 1{,}752631860]$. Apriórny odhad maximálnej chyby je $\epsilon_{max} \doteq 0{,}0100004$. Aproximáciu minima dostávame ako bod $\tilde{x} \doteq 1{,}742631458$ s funkčnou hodnotou $f(\tilde{x}) \doteq 0{,}9450581672$. Skutočná chyba je v tomto prípade $\epsilon \doteq 0{,}00002035648$.

\vspace{0.3cm}

Pri tomto meraní sme interval natiahli na šírku 27 a posunuli ho asymetricky doprava. Z výsledkov pre $N=10$ jasne vidíme, že čím je ten počiatočný rozsah širší, tým horšie sme na tom s presnosťou v prvých krokoch. Napriek tomu, že minimum tentokrát neleží v strede, algoritmus stále efektívne pracuje a vďaka dostatočnému počtu krokov pri $N=50$ nás opäť dostal na našu hranicu $0{,}01$. Je to ďalší dôkaz toho, že táto metóda je mimoriadne robustná. Aj keď na začiatku zvolíme interval trochu náhodne alebo nesymetricky, s pribúdajúcimi vyčísleniami sa tieto počiatočné nedostatky úplne vymažú a výsledok skončí v podstate v rovnakej kvalite ako pri ideálnom symetrickom zadaní.

\vspace{0.5cm}
\noindent \underline{\textbf{Mikro interval} }
\vspace{0.5cm}

Pre východiskový interval $I_{micro} = [ 1{,}73; 1{,}76 ]$ a $N=10$ vyčíslení dostávame výsledný ILM približne ako $I_5 = [1{,}7325; 1{,}7528125]$. Apriórny odhad maximálnej chyby je $\epsilon_{max} \doteq 0{,}01015625$. Aproximáciu minima dostávame ako bod $\tilde{x} \doteq 1{,}74265625$ s funkčnou hodnotou $f(\tilde{x}) \doteq 0{,}9450581711$. Skutočná chyba je v tomto prípade $\epsilon \doteq 0{,}000045162313$.

\vspace{0.3cm}

Pre východiskový interval $I_{micro} = [ 1{,}73; 1{,}76 ]$ a $N=50$ vyčíslení dostávame výsledný ILM približne ako $I_5 = [1{,}732631689; 1{,}752631690]$. Apriórny odhad maximálnej chyby je $\epsilon_{max} \doteq 0{,}0100000$. Aproximáciu minima dostávame ako bod $\tilde{x} \doteq 1{,}742631690$ s funkčnou hodnotou $f(\tilde{x}) \doteq 0{,}9450581673$. Skutočná chyba je v tomto prípade $\epsilon \doteq 0{,}000020602313$.

\vspace{0.3cm}

Je veľmi zaujímavé sledovať, že pri takto úzkom štarte sme sa s apriórnou chybou dostali na úroveň našej delty ($0{,}01$) už pri $N=10$. V podstate to znamená, že metóda „vyriešila“ svoju úlohu hneď v prvých pár krokoch a zvyšných 40 vyčíslení pri $N=50$ už bolo úplne zbytočných, keďže nepriniesli žiadne reálne zlepšenie odhadu chyby.

\vspace{0.5cm}
\noindent \underline{\textbf{Obrovský interval}}
\vspace{0.5cm}

Pre východiskový interval $I_0 = [-100; 100]$ a $N=10$ vyčíslení dostávame výsledný ILM približne ako $I_5 = [-0{,}01; 6{,}259375]$. Apriórny odhad maximálnej chyby je $\epsilon_{max} \doteq 3{,}1346875$. Aproximáciu minima dostávame ako bod $\tilde{x} \doteq 3{,}1246875$ s funkčnou hodnotou $f(\tilde{x}) \doteq 1{,}957126$. Skutočná chyba je v tomto prípade $\epsilon \doteq 1{,}382076398$.

\vspace{0.3cm}

Pre východiskový interval $I_0 = [-100; 100]$ a $N=50$ vyčíslení dostávame výsledný ILM približne ako $I_5 = [1{,}732629768; 1{,}752635728]$. Apriórny odhad maximálnej chyby je $\epsilon_{max} \doteq 0{,}0100002$. Aproximáciu minima dostávame ako bod $\tilde{x} \doteq 1{,}742632748$ s funkčnou hodnotou $f(\tilde{x}) \doteq 0{,}9450581674$. Skutočná chyba je v tomto prípade $\epsilon \doteq 0{,}0000216464$.

\vspace{0.3cm}

Pri $N=10$ je zrejmé, že nízky počet krokov na takýto široký interval nestačí, čo potvrdzuje aj pomerne vysoká apriórna chyba presahujúca tri jednotky. Stred intervalu sa v tejto fáze nachádzal len v okolí hodnoty $3{,}12$. Zaujímavý posun však nastáva pri $N=50$. Ukazuje sa, že dĺžka počiatočného intervalu nie je až taká podstatná, pretože pri dostatočnom počte vyčíslení metóda vždy spoľahlivo dosiahne presnosť určenú parametrom $\delta$.

\vspace{0.3cm}

Ďalej si môžeme naše namerané hodnoty zaznamenať do tabuľky pre lepšiu prehľadnosť. 
Dôležité sú najmä hodnoty pre $N=10$, keďže pre $N=50$ boli všetky takmer rovnaké.

\begin{figure}[h]
    \centering
    \includegraphics[width=0.8\textwidth]{grafy/analyza vplyvu.png}
    \label{fig:moj_obrazok}
\end{figure}
Tabuľka potvrdzuje, že čím širší je počiatočný interval, tým nižšiu presnosť pri rovnakom počte vyčíslení dosahujeme. Zatiaľ čo pri malých rozsahoch je metóda vysoko efektívna, pri extrémnom prípade „Makro“ už narážame na limity algoritmu kvôli nedostatočnému počtu krokov.

%%%%%%%%%%%%%%%%%%%%%%%%%%%%%%%%%%%%%%%%%%%%%%%%%%%%%%%%%%%%%%%%%%%%%
% --- METÓDA ZLATÉHO REZU ---
\newpage
\subsection{Metóda zlatého rezu}

Postup pri analýze presnosti zostáva podobný ako v predchádzajúcom prípade. Opäť budeme sledovať, ako voľba počtu vyčíslení $N$ (kde $N \in \{4, 10, 20, 50, 100\}$) ovplyvňuje výsledok, pričom zachováme počiatočný interval $I_0 = [-6; 10]$. Tu už nepracujeme s parametrom $\delta$.

Princíp tejto metódy vychádza práve z metódy polenia intervalu, avšak vhodnou voľbou bodov budeme okrem prvého kroku vyčísľovať vždy iba jeden nový bod. Teda v prípade rovnakého počtu vyčíslení $N$ by sme mali dostať $N-1$ nových intervalov, zatiaľ čo pri MPI dostaneme iba $N/2$.

\subsubsection{Analýza správania metódy pri rôznych hodnotách počtu vyčíslení}

\vspace{0.5cm}
\noindent \underline{\textbf{N=4}}
\vspace{0.5cm}

Pre $N = 4$ vyčíslenia dostávame výsledný ILM ako $I_3 = [0{,}111456165; 3{,}888543802]$ a apriórny odhad maximálnej chyby je $\epsilon_{max} \approx 1{,}888543813$. Aproximáciu minima dostávame ako bod $\tilde{x} \approx 1{,}999999984$ s funkčnou hodnotou $f(\tilde{x}) \approx 1{,}078173443$. Skutočná chyba je potom $\epsilon \approx 0{,}2573888824$.

\vspace{0.3cm}

Zaujímavé je, že náš teoretický odhad apriórnej chyby je podobný ako pri metóde polenia, avšak naša aproximácia minima $\tilde{x} \approx 1{,}9999$ je v tomto prípade až prekvapivo blízko k bodu $2$. To sa odrazilo aj na skutočnej chybe, ktorá je $\epsilon \approx 0{,}257$, čo je v porovnaní s predchádzajúcimi výsledkami výrazné zlepšenie. Ďalším porovnaním dostávame, že pri pri metóde zlatého rezu (MZR) o jeden interval viac oproti MPI (konkrétne 3 oproti 2). Pri takto malých hodnotách vyčíslení každý nový krok prináša zlepšenie.


\vspace{0.5cm}
\noindent \underline{\textbf{N=10}} 
\vspace{0.5cm}

Pre $N = 10$ vyčíslenia dostávame výsledný ILM ako $I_9 = [1{,}684265159; 1{,}894755039]$ a apriórny odhad maximálnej chyby je $\epsilon_{max} \approx 0{,}1052449388$. Aproximáciu minima dostávame ako bod $\tilde{x} \approx 1{,}789510$ s funkčnou hodnotou $f(\tilde{x}) \approx 0{,}95007804$. Skutočná chyba je potom $\epsilon \approx 0{,}046898997$.

\vspace{0.3cm}

Pri desiatich vyčísleniach metóda zlatého rezu výrazne pokročila. Apriórny odhad maximálnej chyby klesol na $\epsilon_{max} \approx 0{,}105$, čo je v porovnaní s počiatočným stavom ($N=4$) takmer 18-násobné spresnenie. Aproximácia minima $\tilde{x} \approx 1{,}7895$ sa už nachádza vo veľmi tesnej blízkosti skutočného minima.
Pri desiatich vyčísleniach vidíme, že zlatý rez má teoreticky navrch, pretože vďaka recyklovaniu bodov stihol urobiť až deväť krokov, zatiaľ čo metóda polenia len päť. Zaujímavé však je, že hoci má zlatý rez lepší apriórny odhad, naša skutočná chyba vyšla pri metóde polenia o niečo lepšie, čo zrejme súvisí s tým, že metóda polenia v každom kroku osekáva interval agresívnejšie a v našom konkrétnom prípade nám to vďaka symetrii vyšlo presnejšie. Zlatý rez je teda síce teoreticky efektívnejší pri využívaní počtu meraní, ale metóda polenia dokáže byť vďaka rýchlejšiemu deleniu intervalu v praxi veľmi silným konkurentom.


\vspace{0.5cm}
\noindent \underline{\textbf{N=20}} 
\vspace{0.5cm}

Pre $N = 20$ vyčíslenia dostávame výsledný ILM ako $I_{19} = [1{,}742262452; 1{,}7436973865]$ a apriórny odhad maximálnej chyby je $\epsilon_{max} \approx 0{,}0008557$. Aproximáciu minima dostávame ako bod $\tilde{x} \approx 1{,}743118158$ s funkčnou hodnotou $f(\tilde{x}) \approx 0{,}9450587651$. Skutočná chyba je potom $\epsilon \approx 0{,}0005070564808$.

\vspace{0.3cm}

Pri dvadsiatich vyčísleniach už metóda zlatého rezu v teoretickej presnosti úplne dominuje, keďže jej apriórny odhad chyby klesol pod hranicu $0{,}001$, čo je v porovnaní s hodnotou $0{,}0178$ pri metóde polenia zjavný rozdiel. Tento náskok je spôsobený tým, že zlatý rez stihol za rovnaký počet meraní urobiť až 19 krokov, kým metóda polenia len 10, vďaka čomu je výsledný interval zlatého rezu nepomerne užší. Aj keď pri skutočnej chybe bola metóda polenia doteraz o niečo lepšia a triafala sa bližšie k optimu, pri $N=20$ už zlatý rez garantuje dostatočnú presnosť na to, aby sme výsledok mohli považovať za veľmi verný odhad skutočného minima.

\vspace{0.5cm}
\noindent \underline{\textbf{N=50}} 
\vspace{0.5cm}

Pre $N = 50$ vyčíslenia dostávame výsledný ILM ako $I_{49} = [1{,}742605919; 1{,}742605920]$ a apriórny odhad maximálnej chyby je $\epsilon_{max} \approx 4{,}599 \times10^{-10}$. Aproximáciu minima dostávame ako bod $\tilde{x} \approx 1{,}742605920$ s funkčnou hodnotou $f(\tilde{x}) \approx 0{,}9450581663$. Skutočná chyba je potom $\epsilon \approx 5{,}1815 \times10^{-6}$.

\vspace{0.3cm}

Pri päťdesiatich vyčísleniach sa naplno prejavila najväčšia slabina metódy polenia v porovnaní so zlatým rezom. Zatiaľ čo metóda polenia pri tomto počte krokov úplne narazila na svoj presnostný strop kvôli parametru delta a jej apriórna chyba sa zasekla na hodnote $0,01$, zlatý rez pokračoval v zmenšovaní intervalu bez akéhokoľvek obmedzenia až na úroveň miliardtín. Tento veľký rozdiel v teoretickej presnosti jasne ukazuje, že zlatý rez je pri vyššom počte meraní neporovnateľne efektívnejší, pretože dokáže z každého jedného vyčíslenia vyťažiť maximum informácie.

\vspace{0.5cm}
\noindent \underline{\textbf{N=100}} 
\vspace{0.5cm}

Pre $N = 100$ vyčíslenia dostávame výsledný ILM ako $I_{99} = [1{,}742605919; 1{,}742605920]$ a apriórny odhad maximálnej chyby je $\epsilon_{max} \approx 1{,}634 \times10^{-20}$. Aproximáciu minima dostávame ako bod $\tilde{x} \approx 1{,}742605920$ s funkčnou hodnotou $f(\tilde{x}) \approx 0{,}9450581663$. Skutočná chyba je potom $\epsilon \approx 5{,}1815 \times10^{-6}$.

\vspace{0.3cm}

Pri sto vyčísleniach sme sa pri zlatom reze dostali k apriórnemu odhadu chyby v rádoch $10^{-20}$, čo je z praktického hľadiska v podstate dokonalá presnosť. Keď si spomenieme, že metóda polenia pri $N=100$ stále beznádejne stála na hodnote $0{,}01$ kvôli obmedzeniu parametrom delta, je jasné, že zlatý rez je pre vysokú presnosť neporovnateľne lepšou voľbou. Je však zaujímavé, že hoci teoretický odhad extrémne klesol, naša skutočná chyba $\epsilon$ sa od merania s $N=50$ už takmer nepohla. To nám napovedá, že sme zrejme narazili na hranicu numerickej presnosti samotného softvéru alebo referenčnej hodnoty minima, ktorú používame na porovnanie.
%%%%%%%%%%%%%%%%%%%%%%%%%%%%%%%%%%%%%%%%%%%%%%%%%%%%%%%%%%%%%%%%%%%%%

\subsubsection{Analýza správania metódy pri rôznych východiskových intervaloch lokalizácie minima}

Rovnako ako pri metóde polenia intervalu, aj tu predpokladáme, že dĺžka počiatočného intervalu ILM bude mať priamy vplyv na presnosť výsledkov. Očakávame teda, že pri kratšom intervale dosiahneme lepšie výsledky než pri tom dlhšom, čo si následne overíme na porovnaní pre $N=4$ a $N=20$ vyčíslení. Vyberáme si tieto hodnoty vyčíslenia, pretože na nich bude najlepšie vidieť správanie metódy pri rôznych východiskových, bohužiaľ, nebudeme môct už vhodne porovnávať MPI, keďže tam sme vybrali $N=10$ a $N=50$.

\vspace{0.5cm}
\noindent \underline{\textbf{Kratší počiatočný interval}}
\vspace{0.5cm}

Pre východiskový interval $I_0 = [-3; 5]$ a $N=4$ vyčíslení dostávame výsledný ILM približne ako $I_3 = [1{,}222912349; 3{,}111456173]$. Apriórny odhad maximálnej chyby je $\epsilon_{max} \doteq 0{,}9442719065$. Aproximáciu minima dostávame ako bod $\tilde{x} \doteq 2{,}167184260$ s funkčnou hodnotou $f(\tilde{x}) \doteq 1{,}259295792$. Skutočná chyba je v tomto prípade $\epsilon \doteq 0{,}4245731584$.

\vspace{0.3cm}


Pre východiskový interval $I_0 = [-3; 5]$ a $N=20$ vyčíslení dostávame výsledný ILM približne ako $I_{19} = [1{,}742416242; 1{,}743271948]$. Apriórny odhad maximálnej chyby je $\epsilon_{max} \doteq 0{,}000427853$. Aproximáciu minima dostávame ako bod $\tilde{x} \doteq 1{,}742844095$ s funkčnou hodnotou $f(\tilde{x}) \doteq 0{,}945058292$. Skutočná chyba je v tomto prípade $\epsilon \doteq 0{,}000232993$.

\vspace{0.3cm}

Pri kratšom intervale zlatý rez okamžite profituje z lepšieho štartu, čo vidno na polovičnej chybe hneď pri štyroch vyčísleniach. To v podstate znamená, že kombinácia úzkeho intervalu a zlatého rezu je najefektívnejšia cesta, ako sa bez zbytočných výpočtových bariér dostať k veľmi presnému cieľu.

\vspace{0.5cm}
\noindent \underline{\textbf{Dlhší počiatočný interval}}
\vspace{0.5cm}

Pre východiskový interval $I_0 = [-12; 20]$ a $N=4$ vyčíslení dostávame výsledný ILM približne ako $I_3 = [0{,}222912326; 7{,}77708762]$. Apriórny odhad maximálnej chyby je $\epsilon_{max} \doteq 3{,}777087626$. Aproximáciu minima dostávame ako bod $\tilde{x} \doteq 3{,}9999999$ s funkčnou hodnotou $f(\tilde{x}) \doteq 4{,}780213392$. Skutočná chyba je v tomto prípade $\epsilon \doteq 2{,}25738871$.

\vspace{0.3cm}

Pre východiskový interval $I_0 = [-12; 20]$ a $N=20$ vyčíslení dostávame výsledný ILM približne ako $I_{19} = [1{,}740493152; 1{,}743915977]$. Apriórny odhad maximálnej chyby je $\epsilon_{max} \doteq 0{,}001711412926$. Aproximáciu minima dostávame ako bod $\tilde{x} \doteq 1{,}742204564$ s funkčnou hodnotou $f(\tilde{x}) \doteq 0{,}9450585514$. Skutočná chyba je v tomto prípade $\epsilon \doteq 0{,}000406537519147$.

\vspace{0.3cm}

Aj pri takto širokom počiatočnom intervale (šírka 32) vidíme, že hoci je odhad pri $N=4$ ešte pomerne hrubý, s pribúdajúcimi krokmi sa metóda dokáže s touto neistotou vysporiadať veľmi efektívne. Už pri $N=20$ sme dosiahli apriórnu chybu $0{,}0017$, čo predstavuje takmer desaťkrát lepší výsledok, než aký vykazovala metóda polenia pri polovičnom (a teda jednoduchšom) intervale. Opäť sa nám tak potvrdilo, že zlatý rez pracuje pri veľkých rozsahoch oveľa výkonnejšie a z každého vykonaného vyčíslenia dokáže vyťažiť maximum informácií o polohe minima.

\vspace{0.5cm}
\noindent \underline{\textbf{Nesymetrický východiskový interval s minimom pri okraji}} 
\vspace{0.5cm}

Pre východiskový interval $I_0 = [1.7; 15]$ a $N=4$ vyčíslení dostávame výsledný ILM približne ako $I_3 = [1{,}7; 4{,}839704089]$. Apriórny odhad maximálnej chyby je $\epsilon_{max} \doteq 1{,}569852044$. Aproximáciu minima dostávame ako bod $\tilde{x} \doteq 3{,}269852044$ s funkčnou hodnotou $f(\tilde{x}) \doteq 2{,}102421088$. Skutočná chyba je v tomto prípade $\epsilon \doteq 1{,}527240942$.

\vspace{0.3cm}

Pre východiskový interval $I_0 = [1.7; 15]$ a $N=20$ vyčíslení dostávame výsledný ILM približne ako $I_{19} = [1{,}742183968; 1{,}743606580]$. Apriórny odhad maximálnej chyby je $\epsilon_{max} \doteq 0{,}000711306$. Aproximáciu minima dostávame ako bod $\tilde{x} \doteq 1{,}742895274$ s funkčnou hodnotou $f(\tilde{x}) \doteq 0{,}9450583544$. Skutočná chyba je v tomto prípade $\epsilon \doteq 0{,}0002841724793$.

\vspace{0.3cm}

Pri nesymetrickom intervale s minimom blízko okraja sa pri malom počte vyčíslení výrazne znižuje presnosť aproximácie a výsledný interval zostáva široký. Zvýšením počtu vyčíslení sa však metóda rýchlo stabilizuje a poskytuje veľmi presný odhad minima. Tento prípad poukazuje na citlivosť metódy na polohu minima v počiatočnom intervale.

\vspace{0.5cm}
\noindent \underline{\textbf{Nesymetrický východiskový interval s minimom ďalej od okraja}}
\vspace{0.5cm}

Pre východiskový interval $I_0 = [-6; 21]$ a $N=4$ vyčíslení dostávame výsledný ILM približne ako $I_3 = [-2{,}060753108; 4{,}31308228]$. Apriórny odhad maximálnej chyby je $\epsilon_{max} \doteq 3{,}1869178684$. Aproximáciu minima dostávame ako bod $\tilde{x} \doteq 1{,}126164586$ s funkčnou hodnotou $f(\tilde{x}) \doteq 1{,}856237351$. Skutočná chyba je v tomto prípade $\epsilon \doteq 0{,}6164465$.

\vspace{0.3cm}

Pre východiskový interval $I_0 = [-6; 21]$ a $N=20$ vyčíslení dostávame výsledný ILM približne ako $I_{19} = [1{,}741033938; 1{,}743921948]$. Apriórny odhad maximálnej chyby je $\epsilon_{max} \doteq 0{,}001444004$. Aproximáciu minima dostávame ako bod $\tilde{x} \doteq 1{,}742477943$ s funkčnou hodnotou $f(\tilde{x}) \doteq 0{,}09450582075$. Skutočná chyba je v tomto prípade $\epsilon \doteq 0{,}0001331585$.

\vspace{0.3cm}

Pri nesymetrickom intervale, v ktorom sa minimum nachádza ďalej od okraja, je už pri malom počte vyčíslení dosiahnutá lepšia presnosť v porovnaní s prípadom minima na hranici intervalu. So zvyšujúcim sa počtom vyčíslení dochádza k výraznému zúženiu výsledného intervalu a k rýchlej konvergencii k presnej hodnote minima. Tento výsledok potvrdzuje, že poloha minima v rámci počiatočného intervalu má významný vplyv na efektivitu metódy.

\vspace{0.5cm}
\noindent \underline{\textbf{Mikro interval}}
\vspace{0.5cm}

Pre východiskový interval $I_{micro} = [ 1{,}73; 1{,}76 ]$ a $N=4$ vyčíslení dostávame výsledný ILM približne ako $I_3 = [1{,}73708039; 1{,}744164079]$. Apriórny odhad maximálnej chyby je $\epsilon_{max} \doteq 0{,}003541019$. Aproximáciu minima dostávame ako bod $\tilde{x} \doteq 1{,}740623059$ s funkčnou hodnotou $f(\tilde{x}) \doteq 0{,}9450673825$. Skutočná chyba je v tomto prípade $\epsilon \doteq 0{,}0019880286$.

\vspace{0.3cm}

Pre východiskový interval $I_{micro} = [ 1{,}73; 1{,}76 ]$ a $N=20$ vyčíslení dostávame výsledný ILM približne ako $I_{19} = [1{,}742606174; 1{,}742609383]$. Apriórny odhad maximálnej chyby je $\epsilon_{max} \doteq 1{,}6044 \times10^{-6}$. Aproximáciu minima dostávame ako bod $\tilde{x} \doteq 1{,}742607778$ s funkčnou hodnotou $f(\tilde{x}) \doteq 0{,}9450581663$. Skutočná chyba je v tomto prípade $\epsilon \doteq 3{,}3096869 \times10^{-6}$.

\vspace{0.5cm}
\noindent \underline{\textbf{Obrovský interval}} 
\vspace{0.5cm}

Pre východiskový interval $I_0 = [-100; 100]$ a $N=4$ vyčíslení dostávame výsledný ILM približne ako $I_3 = [-23{,}60679794; 23{,}60679750]$. Apriórny odhad maximálnej chyby je $\epsilon_{max} \doteq 23{,}60679766$. Aproximáciu minima dostávame ako bod $\tilde{x} \doteq -2{,}2 \times10^{-7}$ s funkčnou hodnotou $f(\tilde{x}) \doteq 4{,}666667217$. Skutočná chyba je v tomto prípade $\epsilon \doteq 1{,}74261132151$.

\vspace{0.3cm}

Pre východiskový interval $I_0 = [-100; 100]$ a $N=20$ vyčíslení dostávame výsledný ILM približne ako $I_{19} = [1{,}734051255; 1{,}755443918]$. Apriórny odhad maximálnej chyby je $\epsilon_{max} \doteq 0{,}01069633079$. Aproximáciu minima dostávame ako bod $\tilde{x} \doteq 1{,}744747586$ s funkčnou hodnotou $f(\tilde{x}) \doteq 0{,}9450687923$. Skutočná chyba je v tomto prípade $\epsilon \doteq 0{,}0021356484408$.

\vspace{0.3cm}

Aj pri tomto obrovskom rozsahu šírky 200 sa ukázalo, že zlatý rez je extrémne efektívny, pretože na chybu $0{,}01$ mu stačilo len 20 meraní. Pre porovnanie, metóda polenia potrebovala na dosiahnutie tejto presnosti pri makro intervale až 50 vyčíslení. Zlatý rez teda vďaka vyššiemu počtu krokov spracuje aj takto veľké intervaly oveľa rýchlejšie a bez zbytočného plytvania.

\begin{figure}[h]
    \centering
    \includegraphics[width=0.8\textwidth]{grafy/tabulka2.png}
    \label{fig:moj_obrazok}
\end{figure}

\vspace{0.3cm}

Z celého nášho testovania jasne vyplýva, že metóda zlatého rezu je efektívnejšia, pretože vďaka lepšiemu využitiu bodov dokáže urobiť takmer dvojnásobok krokov pri rovnakom počte meraní. Najväčší rozdiel sme videli pri vyššom počte vyčíslení, kde metóda polenia úplne narazila na svoj strop kvôli parametru delta, zatiaľ čo zlatý rez bez problémov pokračoval v spresňovaní až k astronomicky malým hodnotám. Ukázalo sa tiež, že hoci šírka počiatočného intervalu výrazne ovplyvňuje presnosť v úvode, zlatý rez dokáže aj obrovské intervaly spracovať oveľa rýchlejšie a s lepšou matematickou istotou než metóda polenia. Celkovo teda môžeme povedať, že ak nepotrebujeme extrémne vysokú presnosť a máme minimum v strede, metódy sú porovnateľné, ale pre akékoľvek náročnejšie výpočty je zlatý rez jednoznačne lepšou voľbou.


\newpage
%%%%%%%%%%%%%%%%%%%%%%%%%%%%%%%%%%%%%%%%%%%%%%%%%%%%%%%%%%%%%%%%%%%%%

\section{Analýza numerických metód pre funkciu viac premenných}

Na analýzu numerických metód pre funkciu viac premenných sme si zvolili funkciu
$$ f(x,y) = e^{-x} + e^{y} + (x-y^2)^2 + x, $$
ktorá je definovaná a spojitá na celom priestore $\mathbb{R}^2$. Svoje „presné“ minimum táto funkcia nadobúda v bode
$$ [x^*; y^*] \doteq [0.388128831; -0.740924056] $$
s funkčnou hodnotou
$$ f(x^*; y^*) \doteq 1.568996403. $$
Na nasledujúcich troch stranách sú potom vykreslené vrstevnicový graf a 3D graf tejto funkcie z rôznych pohľadov, ktoré ilustrujú jej tvar a globálne vlastnosti.

Pre tento projekt sme zvolili \textbf{Newtonovu metódu (NM)} a \textbf{metódu združených gradientov (MSG)}, ktorých správanie a rýchlosť konvergencie budeme pre túto funkciu analyzovať.\footnote{Používame Python, použité skripty možno nájsť na https://github.com/RadimCech/mat\_prog\_projekt}

\newpage

% --- 10. STRANA: VRSTEVNICOVÝ GRAF ---
\newpage

\begin{figure}[H]
   \centering
   \resizebox{0.8\textwidth}{!}{
   %% Creator: Matplotlib, PGF backend
%%
%% To include the figure in your LaTeX document, write
%%   \input{<filename>.pgf}
%%
%% Make sure the required packages are loaded in your preamble
%%   \usepackage{pgf}
%%
%% Also ensure that all the required font packages are loaded; for instance,
%% the lmodern package is sometimes necessary when using math font.
%%   \usepackage{lmodern}
%%
%% Figures using additional raster images can only be included by \input if
%% they are in the same directory as the main LaTeX file. For loading figures
%% from other directories you can use the `import` package
%%   \usepackage{import}
%%
%% and then include the figures with
%%   \import{<path to file>}{<filename>.pgf}
%%
%% Matplotlib used the following preamble
%%   
%%   \usepackage{fontspec}
%%   \setmainfont{DejaVuSerif.ttf}[Path=\detokenize{/home/radimek/Documents/projekt_mat_prog/mat_prog_kernel/lib/python3.12/site-packages/matplotlib/mpl-data/fonts/ttf/}]
%%   \setsansfont{DejaVuSans.ttf}[Path=\detokenize{/home/radimek/Documents/projekt_mat_prog/mat_prog_kernel/lib/python3.12/site-packages/matplotlib/mpl-data/fonts/ttf/}]
%%   \setmonofont{DejaVuSansMono.ttf}[Path=\detokenize{/home/radimek/Documents/projekt_mat_prog/mat_prog_kernel/lib/python3.12/site-packages/matplotlib/mpl-data/fonts/ttf/}]
%%   \makeatletter\@ifpackageloaded{underscore}{}{\usepackage[strings]{underscore}}\makeatother
%%
\begingroup%
\makeatletter%
\begin{pgfpicture}%
\pgfpathrectangle{\pgfpointorigin}{\pgfqpoint{7.000000in}{6.000000in}}%
\pgfusepath{use as bounding box, clip}%
\begin{pgfscope}%
\pgfsetbuttcap%
\pgfsetmiterjoin%
\definecolor{currentfill}{rgb}{1.000000,1.000000,1.000000}%
\pgfsetfillcolor{currentfill}%
\pgfsetlinewidth{0.000000pt}%
\definecolor{currentstroke}{rgb}{1.000000,1.000000,1.000000}%
\pgfsetstrokecolor{currentstroke}%
\pgfsetdash{}{0pt}%
\pgfpathmoveto{\pgfqpoint{0.000000in}{0.000000in}}%
\pgfpathlineto{\pgfqpoint{7.000000in}{0.000000in}}%
\pgfpathlineto{\pgfqpoint{7.000000in}{6.000000in}}%
\pgfpathlineto{\pgfqpoint{0.000000in}{6.000000in}}%
\pgfpathlineto{\pgfqpoint{0.000000in}{0.000000in}}%
\pgfpathclose%
\pgfusepath{fill}%
\end{pgfscope}%
\begin{pgfscope}%
\pgfsetbuttcap%
\pgfsetmiterjoin%
\definecolor{currentfill}{rgb}{1.000000,1.000000,1.000000}%
\pgfsetfillcolor{currentfill}%
\pgfsetlinewidth{0.000000pt}%
\definecolor{currentstroke}{rgb}{0.000000,0.000000,0.000000}%
\pgfsetstrokecolor{currentstroke}%
\pgfsetstrokeopacity{0.000000}%
\pgfsetdash{}{0pt}%
\pgfpathmoveto{\pgfqpoint{0.854460in}{0.571603in}}%
\pgfpathlineto{\pgfqpoint{6.739560in}{0.571603in}}%
\pgfpathlineto{\pgfqpoint{6.739560in}{5.797238in}}%
\pgfpathlineto{\pgfqpoint{0.854460in}{5.797238in}}%
\pgfpathlineto{\pgfqpoint{0.854460in}{0.571603in}}%
\pgfpathclose%
\pgfusepath{fill}%
\end{pgfscope}%
\begin{pgfscope}%
\pgfsetbuttcap%
\pgfsetroundjoin%
\definecolor{currentfill}{rgb}{0.000000,0.000000,0.000000}%
\pgfsetfillcolor{currentfill}%
\pgfsetlinewidth{0.803000pt}%
\definecolor{currentstroke}{rgb}{0.000000,0.000000,0.000000}%
\pgfsetstrokecolor{currentstroke}%
\pgfsetdash{}{0pt}%
\pgfsys@defobject{currentmarker}{\pgfqpoint{0.000000in}{-0.048611in}}{\pgfqpoint{0.000000in}{0.000000in}}{%
\pgfpathmoveto{\pgfqpoint{0.000000in}{0.000000in}}%
\pgfpathlineto{\pgfqpoint{0.000000in}{-0.048611in}}%
\pgfusepath{stroke,fill}%
}%
\begin{pgfscope}%
\pgfsys@transformshift{0.854460in}{0.571603in}%
\pgfsys@useobject{currentmarker}{}%
\end{pgfscope}%
\end{pgfscope}%
\begin{pgfscope}%
\definecolor{textcolor}{rgb}{0.000000,0.000000,0.000000}%
\pgfsetstrokecolor{textcolor}%
\pgfsetfillcolor{textcolor}%
\pgftext[x=0.854460in,y=0.474381in,,top]{\color{textcolor}\sffamily\fontsize{10.000000}{12.000000}\selectfont \ensuremath{-}1.0}%
\end{pgfscope}%
\begin{pgfscope}%
\pgfsetbuttcap%
\pgfsetroundjoin%
\definecolor{currentfill}{rgb}{0.000000,0.000000,0.000000}%
\pgfsetfillcolor{currentfill}%
\pgfsetlinewidth{0.803000pt}%
\definecolor{currentstroke}{rgb}{0.000000,0.000000,0.000000}%
\pgfsetstrokecolor{currentstroke}%
\pgfsetdash{}{0pt}%
\pgfsys@defobject{currentmarker}{\pgfqpoint{0.000000in}{-0.048611in}}{\pgfqpoint{0.000000in}{0.000000in}}{%
\pgfpathmoveto{\pgfqpoint{0.000000in}{0.000000in}}%
\pgfpathlineto{\pgfqpoint{0.000000in}{-0.048611in}}%
\pgfusepath{stroke,fill}%
}%
\begin{pgfscope}%
\pgfsys@transformshift{1.835310in}{0.571603in}%
\pgfsys@useobject{currentmarker}{}%
\end{pgfscope}%
\end{pgfscope}%
\begin{pgfscope}%
\definecolor{textcolor}{rgb}{0.000000,0.000000,0.000000}%
\pgfsetstrokecolor{textcolor}%
\pgfsetfillcolor{textcolor}%
\pgftext[x=1.835310in,y=0.474381in,,top]{\color{textcolor}\sffamily\fontsize{10.000000}{12.000000}\selectfont \ensuremath{-}0.5}%
\end{pgfscope}%
\begin{pgfscope}%
\pgfsetbuttcap%
\pgfsetroundjoin%
\definecolor{currentfill}{rgb}{0.000000,0.000000,0.000000}%
\pgfsetfillcolor{currentfill}%
\pgfsetlinewidth{0.803000pt}%
\definecolor{currentstroke}{rgb}{0.000000,0.000000,0.000000}%
\pgfsetstrokecolor{currentstroke}%
\pgfsetdash{}{0pt}%
\pgfsys@defobject{currentmarker}{\pgfqpoint{0.000000in}{-0.048611in}}{\pgfqpoint{0.000000in}{0.000000in}}{%
\pgfpathmoveto{\pgfqpoint{0.000000in}{0.000000in}}%
\pgfpathlineto{\pgfqpoint{0.000000in}{-0.048611in}}%
\pgfusepath{stroke,fill}%
}%
\begin{pgfscope}%
\pgfsys@transformshift{2.816160in}{0.571603in}%
\pgfsys@useobject{currentmarker}{}%
\end{pgfscope}%
\end{pgfscope}%
\begin{pgfscope}%
\definecolor{textcolor}{rgb}{0.000000,0.000000,0.000000}%
\pgfsetstrokecolor{textcolor}%
\pgfsetfillcolor{textcolor}%
\pgftext[x=2.816160in,y=0.474381in,,top]{\color{textcolor}\sffamily\fontsize{10.000000}{12.000000}\selectfont 0.0}%
\end{pgfscope}%
\begin{pgfscope}%
\pgfsetbuttcap%
\pgfsetroundjoin%
\definecolor{currentfill}{rgb}{0.000000,0.000000,0.000000}%
\pgfsetfillcolor{currentfill}%
\pgfsetlinewidth{0.803000pt}%
\definecolor{currentstroke}{rgb}{0.000000,0.000000,0.000000}%
\pgfsetstrokecolor{currentstroke}%
\pgfsetdash{}{0pt}%
\pgfsys@defobject{currentmarker}{\pgfqpoint{0.000000in}{-0.048611in}}{\pgfqpoint{0.000000in}{0.000000in}}{%
\pgfpathmoveto{\pgfqpoint{0.000000in}{0.000000in}}%
\pgfpathlineto{\pgfqpoint{0.000000in}{-0.048611in}}%
\pgfusepath{stroke,fill}%
}%
\begin{pgfscope}%
\pgfsys@transformshift{3.797010in}{0.571603in}%
\pgfsys@useobject{currentmarker}{}%
\end{pgfscope}%
\end{pgfscope}%
\begin{pgfscope}%
\definecolor{textcolor}{rgb}{0.000000,0.000000,0.000000}%
\pgfsetstrokecolor{textcolor}%
\pgfsetfillcolor{textcolor}%
\pgftext[x=3.797010in,y=0.474381in,,top]{\color{textcolor}\sffamily\fontsize{10.000000}{12.000000}\selectfont 0.5}%
\end{pgfscope}%
\begin{pgfscope}%
\pgfsetbuttcap%
\pgfsetroundjoin%
\definecolor{currentfill}{rgb}{0.000000,0.000000,0.000000}%
\pgfsetfillcolor{currentfill}%
\pgfsetlinewidth{0.803000pt}%
\definecolor{currentstroke}{rgb}{0.000000,0.000000,0.000000}%
\pgfsetstrokecolor{currentstroke}%
\pgfsetdash{}{0pt}%
\pgfsys@defobject{currentmarker}{\pgfqpoint{0.000000in}{-0.048611in}}{\pgfqpoint{0.000000in}{0.000000in}}{%
\pgfpathmoveto{\pgfqpoint{0.000000in}{0.000000in}}%
\pgfpathlineto{\pgfqpoint{0.000000in}{-0.048611in}}%
\pgfusepath{stroke,fill}%
}%
\begin{pgfscope}%
\pgfsys@transformshift{4.777860in}{0.571603in}%
\pgfsys@useobject{currentmarker}{}%
\end{pgfscope}%
\end{pgfscope}%
\begin{pgfscope}%
\definecolor{textcolor}{rgb}{0.000000,0.000000,0.000000}%
\pgfsetstrokecolor{textcolor}%
\pgfsetfillcolor{textcolor}%
\pgftext[x=4.777860in,y=0.474381in,,top]{\color{textcolor}\sffamily\fontsize{10.000000}{12.000000}\selectfont 1.0}%
\end{pgfscope}%
\begin{pgfscope}%
\pgfsetbuttcap%
\pgfsetroundjoin%
\definecolor{currentfill}{rgb}{0.000000,0.000000,0.000000}%
\pgfsetfillcolor{currentfill}%
\pgfsetlinewidth{0.803000pt}%
\definecolor{currentstroke}{rgb}{0.000000,0.000000,0.000000}%
\pgfsetstrokecolor{currentstroke}%
\pgfsetdash{}{0pt}%
\pgfsys@defobject{currentmarker}{\pgfqpoint{0.000000in}{-0.048611in}}{\pgfqpoint{0.000000in}{0.000000in}}{%
\pgfpathmoveto{\pgfqpoint{0.000000in}{0.000000in}}%
\pgfpathlineto{\pgfqpoint{0.000000in}{-0.048611in}}%
\pgfusepath{stroke,fill}%
}%
\begin{pgfscope}%
\pgfsys@transformshift{5.758710in}{0.571603in}%
\pgfsys@useobject{currentmarker}{}%
\end{pgfscope}%
\end{pgfscope}%
\begin{pgfscope}%
\definecolor{textcolor}{rgb}{0.000000,0.000000,0.000000}%
\pgfsetstrokecolor{textcolor}%
\pgfsetfillcolor{textcolor}%
\pgftext[x=5.758710in,y=0.474381in,,top]{\color{textcolor}\sffamily\fontsize{10.000000}{12.000000}\selectfont 1.5}%
\end{pgfscope}%
\begin{pgfscope}%
\pgfsetbuttcap%
\pgfsetroundjoin%
\definecolor{currentfill}{rgb}{0.000000,0.000000,0.000000}%
\pgfsetfillcolor{currentfill}%
\pgfsetlinewidth{0.803000pt}%
\definecolor{currentstroke}{rgb}{0.000000,0.000000,0.000000}%
\pgfsetstrokecolor{currentstroke}%
\pgfsetdash{}{0pt}%
\pgfsys@defobject{currentmarker}{\pgfqpoint{0.000000in}{-0.048611in}}{\pgfqpoint{0.000000in}{0.000000in}}{%
\pgfpathmoveto{\pgfqpoint{0.000000in}{0.000000in}}%
\pgfpathlineto{\pgfqpoint{0.000000in}{-0.048611in}}%
\pgfusepath{stroke,fill}%
}%
\begin{pgfscope}%
\pgfsys@transformshift{6.739560in}{0.571603in}%
\pgfsys@useobject{currentmarker}{}%
\end{pgfscope}%
\end{pgfscope}%
\begin{pgfscope}%
\definecolor{textcolor}{rgb}{0.000000,0.000000,0.000000}%
\pgfsetstrokecolor{textcolor}%
\pgfsetfillcolor{textcolor}%
\pgftext[x=6.739560in,y=0.474381in,,top]{\color{textcolor}\sffamily\fontsize{10.000000}{12.000000}\selectfont 2.0}%
\end{pgfscope}%
\begin{pgfscope}%
\definecolor{textcolor}{rgb}{0.000000,0.000000,0.000000}%
\pgfsetstrokecolor{textcolor}%
\pgfsetfillcolor{textcolor}%
\pgftext[x=3.797010in,y=0.284413in,,top]{\color{textcolor}\sffamily\fontsize{10.000000}{12.000000}\selectfont x}%
\end{pgfscope}%
\begin{pgfscope}%
\pgfsetbuttcap%
\pgfsetroundjoin%
\definecolor{currentfill}{rgb}{0.000000,0.000000,0.000000}%
\pgfsetfillcolor{currentfill}%
\pgfsetlinewidth{0.803000pt}%
\definecolor{currentstroke}{rgb}{0.000000,0.000000,0.000000}%
\pgfsetstrokecolor{currentstroke}%
\pgfsetdash{}{0pt}%
\pgfsys@defobject{currentmarker}{\pgfqpoint{-0.048611in}{0.000000in}}{\pgfqpoint{-0.000000in}{0.000000in}}{%
\pgfpathmoveto{\pgfqpoint{-0.000000in}{0.000000in}}%
\pgfpathlineto{\pgfqpoint{-0.048611in}{0.000000in}}%
\pgfusepath{stroke,fill}%
}%
\begin{pgfscope}%
\pgfsys@transformshift{0.854460in}{0.571603in}%
\pgfsys@useobject{currentmarker}{}%
\end{pgfscope}%
\end{pgfscope}%
\begin{pgfscope}%
\definecolor{textcolor}{rgb}{0.000000,0.000000,0.000000}%
\pgfsetstrokecolor{textcolor}%
\pgfsetfillcolor{textcolor}%
\pgftext[x=0.339968in, y=0.518842in, left, base]{\color{textcolor}\sffamily\fontsize{10.000000}{12.000000}\selectfont \ensuremath{-}1.00}%
\end{pgfscope}%
\begin{pgfscope}%
\pgfsetbuttcap%
\pgfsetroundjoin%
\definecolor{currentfill}{rgb}{0.000000,0.000000,0.000000}%
\pgfsetfillcolor{currentfill}%
\pgfsetlinewidth{0.803000pt}%
\definecolor{currentstroke}{rgb}{0.000000,0.000000,0.000000}%
\pgfsetstrokecolor{currentstroke}%
\pgfsetdash{}{0pt}%
\pgfsys@defobject{currentmarker}{\pgfqpoint{-0.048611in}{0.000000in}}{\pgfqpoint{-0.000000in}{0.000000in}}{%
\pgfpathmoveto{\pgfqpoint{-0.000000in}{0.000000in}}%
\pgfpathlineto{\pgfqpoint{-0.048611in}{0.000000in}}%
\pgfusepath{stroke,fill}%
}%
\begin{pgfscope}%
\pgfsys@transformshift{0.854460in}{1.224808in}%
\pgfsys@useobject{currentmarker}{}%
\end{pgfscope}%
\end{pgfscope}%
\begin{pgfscope}%
\definecolor{textcolor}{rgb}{0.000000,0.000000,0.000000}%
\pgfsetstrokecolor{textcolor}%
\pgfsetfillcolor{textcolor}%
\pgftext[x=0.339968in, y=1.172046in, left, base]{\color{textcolor}\sffamily\fontsize{10.000000}{12.000000}\selectfont \ensuremath{-}0.75}%
\end{pgfscope}%
\begin{pgfscope}%
\pgfsetbuttcap%
\pgfsetroundjoin%
\definecolor{currentfill}{rgb}{0.000000,0.000000,0.000000}%
\pgfsetfillcolor{currentfill}%
\pgfsetlinewidth{0.803000pt}%
\definecolor{currentstroke}{rgb}{0.000000,0.000000,0.000000}%
\pgfsetstrokecolor{currentstroke}%
\pgfsetdash{}{0pt}%
\pgfsys@defobject{currentmarker}{\pgfqpoint{-0.048611in}{0.000000in}}{\pgfqpoint{-0.000000in}{0.000000in}}{%
\pgfpathmoveto{\pgfqpoint{-0.000000in}{0.000000in}}%
\pgfpathlineto{\pgfqpoint{-0.048611in}{0.000000in}}%
\pgfusepath{stroke,fill}%
}%
\begin{pgfscope}%
\pgfsys@transformshift{0.854460in}{1.878012in}%
\pgfsys@useobject{currentmarker}{}%
\end{pgfscope}%
\end{pgfscope}%
\begin{pgfscope}%
\definecolor{textcolor}{rgb}{0.000000,0.000000,0.000000}%
\pgfsetstrokecolor{textcolor}%
\pgfsetfillcolor{textcolor}%
\pgftext[x=0.339968in, y=1.825251in, left, base]{\color{textcolor}\sffamily\fontsize{10.000000}{12.000000}\selectfont \ensuremath{-}0.50}%
\end{pgfscope}%
\begin{pgfscope}%
\pgfsetbuttcap%
\pgfsetroundjoin%
\definecolor{currentfill}{rgb}{0.000000,0.000000,0.000000}%
\pgfsetfillcolor{currentfill}%
\pgfsetlinewidth{0.803000pt}%
\definecolor{currentstroke}{rgb}{0.000000,0.000000,0.000000}%
\pgfsetstrokecolor{currentstroke}%
\pgfsetdash{}{0pt}%
\pgfsys@defobject{currentmarker}{\pgfqpoint{-0.048611in}{0.000000in}}{\pgfqpoint{-0.000000in}{0.000000in}}{%
\pgfpathmoveto{\pgfqpoint{-0.000000in}{0.000000in}}%
\pgfpathlineto{\pgfqpoint{-0.048611in}{0.000000in}}%
\pgfusepath{stroke,fill}%
}%
\begin{pgfscope}%
\pgfsys@transformshift{0.854460in}{2.531217in}%
\pgfsys@useobject{currentmarker}{}%
\end{pgfscope}%
\end{pgfscope}%
\begin{pgfscope}%
\definecolor{textcolor}{rgb}{0.000000,0.000000,0.000000}%
\pgfsetstrokecolor{textcolor}%
\pgfsetfillcolor{textcolor}%
\pgftext[x=0.339968in, y=2.478455in, left, base]{\color{textcolor}\sffamily\fontsize{10.000000}{12.000000}\selectfont \ensuremath{-}0.25}%
\end{pgfscope}%
\begin{pgfscope}%
\pgfsetbuttcap%
\pgfsetroundjoin%
\definecolor{currentfill}{rgb}{0.000000,0.000000,0.000000}%
\pgfsetfillcolor{currentfill}%
\pgfsetlinewidth{0.803000pt}%
\definecolor{currentstroke}{rgb}{0.000000,0.000000,0.000000}%
\pgfsetstrokecolor{currentstroke}%
\pgfsetdash{}{0pt}%
\pgfsys@defobject{currentmarker}{\pgfqpoint{-0.048611in}{0.000000in}}{\pgfqpoint{-0.000000in}{0.000000in}}{%
\pgfpathmoveto{\pgfqpoint{-0.000000in}{0.000000in}}%
\pgfpathlineto{\pgfqpoint{-0.048611in}{0.000000in}}%
\pgfusepath{stroke,fill}%
}%
\begin{pgfscope}%
\pgfsys@transformshift{0.854460in}{3.184421in}%
\pgfsys@useobject{currentmarker}{}%
\end{pgfscope}%
\end{pgfscope}%
\begin{pgfscope}%
\definecolor{textcolor}{rgb}{0.000000,0.000000,0.000000}%
\pgfsetstrokecolor{textcolor}%
\pgfsetfillcolor{textcolor}%
\pgftext[x=0.447993in, y=3.131659in, left, base]{\color{textcolor}\sffamily\fontsize{10.000000}{12.000000}\selectfont 0.00}%
\end{pgfscope}%
\begin{pgfscope}%
\pgfsetbuttcap%
\pgfsetroundjoin%
\definecolor{currentfill}{rgb}{0.000000,0.000000,0.000000}%
\pgfsetfillcolor{currentfill}%
\pgfsetlinewidth{0.803000pt}%
\definecolor{currentstroke}{rgb}{0.000000,0.000000,0.000000}%
\pgfsetstrokecolor{currentstroke}%
\pgfsetdash{}{0pt}%
\pgfsys@defobject{currentmarker}{\pgfqpoint{-0.048611in}{0.000000in}}{\pgfqpoint{-0.000000in}{0.000000in}}{%
\pgfpathmoveto{\pgfqpoint{-0.000000in}{0.000000in}}%
\pgfpathlineto{\pgfqpoint{-0.048611in}{0.000000in}}%
\pgfusepath{stroke,fill}%
}%
\begin{pgfscope}%
\pgfsys@transformshift{0.854460in}{3.837625in}%
\pgfsys@useobject{currentmarker}{}%
\end{pgfscope}%
\end{pgfscope}%
\begin{pgfscope}%
\definecolor{textcolor}{rgb}{0.000000,0.000000,0.000000}%
\pgfsetstrokecolor{textcolor}%
\pgfsetfillcolor{textcolor}%
\pgftext[x=0.447993in, y=3.784864in, left, base]{\color{textcolor}\sffamily\fontsize{10.000000}{12.000000}\selectfont 0.25}%
\end{pgfscope}%
\begin{pgfscope}%
\pgfsetbuttcap%
\pgfsetroundjoin%
\definecolor{currentfill}{rgb}{0.000000,0.000000,0.000000}%
\pgfsetfillcolor{currentfill}%
\pgfsetlinewidth{0.803000pt}%
\definecolor{currentstroke}{rgb}{0.000000,0.000000,0.000000}%
\pgfsetstrokecolor{currentstroke}%
\pgfsetdash{}{0pt}%
\pgfsys@defobject{currentmarker}{\pgfqpoint{-0.048611in}{0.000000in}}{\pgfqpoint{-0.000000in}{0.000000in}}{%
\pgfpathmoveto{\pgfqpoint{-0.000000in}{0.000000in}}%
\pgfpathlineto{\pgfqpoint{-0.048611in}{0.000000in}}%
\pgfusepath{stroke,fill}%
}%
\begin{pgfscope}%
\pgfsys@transformshift{0.854460in}{4.490830in}%
\pgfsys@useobject{currentmarker}{}%
\end{pgfscope}%
\end{pgfscope}%
\begin{pgfscope}%
\definecolor{textcolor}{rgb}{0.000000,0.000000,0.000000}%
\pgfsetstrokecolor{textcolor}%
\pgfsetfillcolor{textcolor}%
\pgftext[x=0.447993in, y=4.438068in, left, base]{\color{textcolor}\sffamily\fontsize{10.000000}{12.000000}\selectfont 0.50}%
\end{pgfscope}%
\begin{pgfscope}%
\pgfsetbuttcap%
\pgfsetroundjoin%
\definecolor{currentfill}{rgb}{0.000000,0.000000,0.000000}%
\pgfsetfillcolor{currentfill}%
\pgfsetlinewidth{0.803000pt}%
\definecolor{currentstroke}{rgb}{0.000000,0.000000,0.000000}%
\pgfsetstrokecolor{currentstroke}%
\pgfsetdash{}{0pt}%
\pgfsys@defobject{currentmarker}{\pgfqpoint{-0.048611in}{0.000000in}}{\pgfqpoint{-0.000000in}{0.000000in}}{%
\pgfpathmoveto{\pgfqpoint{-0.000000in}{0.000000in}}%
\pgfpathlineto{\pgfqpoint{-0.048611in}{0.000000in}}%
\pgfusepath{stroke,fill}%
}%
\begin{pgfscope}%
\pgfsys@transformshift{0.854460in}{5.144034in}%
\pgfsys@useobject{currentmarker}{}%
\end{pgfscope}%
\end{pgfscope}%
\begin{pgfscope}%
\definecolor{textcolor}{rgb}{0.000000,0.000000,0.000000}%
\pgfsetstrokecolor{textcolor}%
\pgfsetfillcolor{textcolor}%
\pgftext[x=0.447993in, y=5.091273in, left, base]{\color{textcolor}\sffamily\fontsize{10.000000}{12.000000}\selectfont 0.75}%
\end{pgfscope}%
\begin{pgfscope}%
\pgfsetbuttcap%
\pgfsetroundjoin%
\definecolor{currentfill}{rgb}{0.000000,0.000000,0.000000}%
\pgfsetfillcolor{currentfill}%
\pgfsetlinewidth{0.803000pt}%
\definecolor{currentstroke}{rgb}{0.000000,0.000000,0.000000}%
\pgfsetstrokecolor{currentstroke}%
\pgfsetdash{}{0pt}%
\pgfsys@defobject{currentmarker}{\pgfqpoint{-0.048611in}{0.000000in}}{\pgfqpoint{-0.000000in}{0.000000in}}{%
\pgfpathmoveto{\pgfqpoint{-0.000000in}{0.000000in}}%
\pgfpathlineto{\pgfqpoint{-0.048611in}{0.000000in}}%
\pgfusepath{stroke,fill}%
}%
\begin{pgfscope}%
\pgfsys@transformshift{0.854460in}{5.797238in}%
\pgfsys@useobject{currentmarker}{}%
\end{pgfscope}%
\end{pgfscope}%
\begin{pgfscope}%
\definecolor{textcolor}{rgb}{0.000000,0.000000,0.000000}%
\pgfsetstrokecolor{textcolor}%
\pgfsetfillcolor{textcolor}%
\pgftext[x=0.447993in, y=5.744477in, left, base]{\color{textcolor}\sffamily\fontsize{10.000000}{12.000000}\selectfont 1.00}%
\end{pgfscope}%
\begin{pgfscope}%
\definecolor{textcolor}{rgb}{0.000000,0.000000,0.000000}%
\pgfsetstrokecolor{textcolor}%
\pgfsetfillcolor{textcolor}%
\pgftext[x=0.284413in,y=3.184421in,,bottom,rotate=90.000000]{\color{textcolor}\sffamily\fontsize{10.000000}{12.000000}\selectfont y}%
\end{pgfscope}%
\begin{pgfscope}%
\pgfpathrectangle{\pgfqpoint{0.854460in}{0.571603in}}{\pgfqpoint{5.885100in}{5.225635in}}%
\pgfusepath{clip}%
\pgfsetbuttcap%
\pgfsetroundjoin%
\pgfsetlinewidth{1.505625pt}%
\definecolor{currentstroke}{rgb}{0.273809,0.031497,0.358853}%
\pgfsetstrokecolor{currentstroke}%
\pgfsetdash{}{0pt}%
\pgfpathmoveto{\pgfqpoint{4.070867in}{1.241281in}}%
\pgfpathlineto{\pgfqpoint{4.125782in}{1.175571in}}%
\pgfpathlineto{\pgfqpoint{4.168246in}{1.123052in}}%
\pgfpathlineto{\pgfqpoint{4.228536in}{1.044274in}}%
\pgfpathlineto{\pgfqpoint{4.284278in}{0.965495in}}%
\pgfpathlineto{\pgfqpoint{4.318017in}{0.912976in}}%
\pgfpathlineto{\pgfqpoint{4.348371in}{0.860458in}}%
\pgfpathlineto{\pgfqpoint{4.374350in}{0.807939in}}%
\pgfpathlineto{\pgfqpoint{4.384843in}{0.781679in}}%
\pgfpathlineto{\pgfqpoint{4.393739in}{0.755420in}}%
\pgfpathlineto{\pgfqpoint{4.400495in}{0.729160in}}%
\pgfpathlineto{\pgfqpoint{4.404221in}{0.702901in}}%
\pgfpathlineto{\pgfqpoint{4.403264in}{0.673127in}}%
\pgfpathlineto{\pgfqpoint{4.397683in}{0.650382in}}%
\pgfpathlineto{\pgfqpoint{4.382086in}{0.624122in}}%
\pgfpathlineto{\pgfqpoint{4.373691in}{0.615959in}}%
\pgfpathlineto{\pgfqpoint{4.344118in}{0.597011in}}%
\pgfpathlineto{\pgfqpoint{4.314544in}{0.588172in}}%
\pgfpathlineto{\pgfqpoint{4.284971in}{0.583610in}}%
\pgfpathlineto{\pgfqpoint{4.255398in}{0.582147in}}%
\pgfpathlineto{\pgfqpoint{4.225824in}{0.583004in}}%
\pgfpathlineto{\pgfqpoint{4.196251in}{0.585645in}}%
\pgfpathlineto{\pgfqpoint{4.166677in}{0.589688in}}%
\pgfpathlineto{\pgfqpoint{4.122964in}{0.597863in}}%
\pgfpathlineto{\pgfqpoint{4.077957in}{0.608820in}}%
\pgfpathlineto{\pgfqpoint{4.018811in}{0.625812in}}%
\pgfpathlineto{\pgfqpoint{3.948066in}{0.650382in}}%
\pgfpathlineto{\pgfqpoint{3.881868in}{0.676641in}}%
\pgfpathlineto{\pgfqpoint{3.822205in}{0.702901in}}%
\pgfpathlineto{\pgfqpoint{3.767335in}{0.729160in}}%
\pgfpathlineto{\pgfqpoint{3.716076in}{0.755420in}}%
\pgfpathlineto{\pgfqpoint{3.634357in}{0.801055in}}%
\pgfpathlineto{\pgfqpoint{3.575210in}{0.836798in}}%
\pgfpathlineto{\pgfqpoint{3.499258in}{0.886717in}}%
\pgfpathlineto{\pgfqpoint{3.456917in}{0.916419in}}%
\pgfpathlineto{\pgfqpoint{3.391535in}{0.965495in}}%
\pgfpathlineto{\pgfqpoint{3.327111in}{1.018014in}}%
\pgfpathlineto{\pgfqpoint{3.267775in}{1.070533in}}%
\pgfpathlineto{\pgfqpoint{3.213147in}{1.123052in}}%
\pgfpathlineto{\pgfqpoint{3.161183in}{1.177492in}}%
\pgfpathlineto{\pgfqpoint{3.117242in}{1.228090in}}%
\pgfpathlineto{\pgfqpoint{3.075310in}{1.280609in}}%
\pgfpathlineto{\pgfqpoint{3.042890in}{1.325448in}}%
\pgfpathlineto{\pgfqpoint{3.020151in}{1.359388in}}%
\pgfpathlineto{\pgfqpoint{2.988135in}{1.411906in}}%
\pgfpathlineto{\pgfqpoint{2.973705in}{1.438166in}}%
\pgfpathlineto{\pgfqpoint{2.947533in}{1.490685in}}%
\pgfpathlineto{\pgfqpoint{2.924596in}{1.544954in}}%
\pgfpathlineto{\pgfqpoint{2.907476in}{1.595723in}}%
\pgfpathlineto{\pgfqpoint{2.893639in}{1.648242in}}%
\pgfpathlineto{\pgfqpoint{2.888732in}{1.674501in}}%
\pgfpathlineto{\pgfqpoint{2.885009in}{1.700761in}}%
\pgfpathlineto{\pgfqpoint{2.882585in}{1.727020in}}%
\pgfpathlineto{\pgfqpoint{2.881588in}{1.753280in}}%
\pgfpathlineto{\pgfqpoint{2.882167in}{1.779539in}}%
\pgfpathlineto{\pgfqpoint{2.884491in}{1.805799in}}%
\pgfpathlineto{\pgfqpoint{2.888754in}{1.832058in}}%
\pgfpathlineto{\pgfqpoint{2.895198in}{1.858318in}}%
\pgfpathlineto{\pgfqpoint{2.905163in}{1.884577in}}%
\pgfpathlineto{\pgfqpoint{2.924596in}{1.920928in}}%
\pgfpathlineto{\pgfqpoint{2.936985in}{1.937096in}}%
\pgfpathlineto{\pgfqpoint{2.954169in}{1.955505in}}%
\pgfpathlineto{\pgfqpoint{2.963745in}{1.963355in}}%
\pgfpathlineto{\pgfqpoint{2.983743in}{1.977341in}}%
\pgfpathlineto{\pgfqpoint{3.013316in}{1.991481in}}%
\pgfpathlineto{\pgfqpoint{3.042890in}{1.999641in}}%
\pgfpathlineto{\pgfqpoint{3.072463in}{2.003499in}}%
\pgfpathlineto{\pgfqpoint{3.102036in}{2.003735in}}%
\pgfpathlineto{\pgfqpoint{3.131610in}{2.000894in}}%
\pgfpathlineto{\pgfqpoint{3.161183in}{1.995420in}}%
\pgfpathlineto{\pgfqpoint{3.190756in}{1.987599in}}%
\pgfpathlineto{\pgfqpoint{3.220330in}{1.977529in}}%
\pgfpathlineto{\pgfqpoint{3.255016in}{1.963355in}}%
\pgfpathlineto{\pgfqpoint{3.279476in}{1.952053in}}%
\pgfpathlineto{\pgfqpoint{3.309050in}{1.936987in}}%
\pgfpathlineto{\pgfqpoint{3.368197in}{1.902303in}}%
\pgfpathlineto{\pgfqpoint{3.427343in}{1.862827in}}%
\pgfpathlineto{\pgfqpoint{3.486490in}{1.819041in}}%
\pgfpathlineto{\pgfqpoint{3.545637in}{1.771672in}}%
\pgfpathlineto{\pgfqpoint{3.604783in}{1.721129in}}%
\pgfpathlineto{\pgfqpoint{3.684793in}{1.648242in}}%
\pgfpathlineto{\pgfqpoint{3.739690in}{1.595723in}}%
\pgfpathlineto{\pgfqpoint{3.792774in}{1.543204in}}%
\pgfpathlineto{\pgfqpoint{3.810573in}{1.525316in}}%
\pgfpathlineto{\pgfqpoint{3.810573in}{1.525316in}}%
\pgfusepath{stroke}%
\end{pgfscope}%
\begin{pgfscope}%
\pgfpathrectangle{\pgfqpoint{0.854460in}{0.571603in}}{\pgfqpoint{5.885100in}{5.225635in}}%
\pgfusepath{clip}%
\pgfsetbuttcap%
\pgfsetroundjoin%
\pgfsetlinewidth{1.505625pt}%
\definecolor{currentstroke}{rgb}{0.278791,0.062145,0.386592}%
\pgfsetstrokecolor{currentstroke}%
\pgfsetdash{}{0pt}%
\pgfpathmoveto{\pgfqpoint{3.596928in}{0.571603in}}%
\pgfpathlineto{\pgfqpoint{3.507962in}{0.624122in}}%
\pgfpathlineto{\pgfqpoint{3.424344in}{0.676641in}}%
\pgfpathlineto{\pgfqpoint{3.338623in}{0.734146in}}%
\pgfpathlineto{\pgfqpoint{3.271786in}{0.781679in}}%
\pgfpathlineto{\pgfqpoint{3.190756in}{0.843047in}}%
\pgfpathlineto{\pgfqpoint{3.131610in}{0.890565in}}%
\pgfpathlineto{\pgfqpoint{3.072463in}{0.940814in}}%
\pgfpathlineto{\pgfqpoint{3.013316in}{0.994202in}}%
\pgfpathlineto{\pgfqpoint{2.954169in}{1.051179in}}%
\pgfpathlineto{\pgfqpoint{2.895023in}{1.112244in}}%
\pgfpathlineto{\pgfqpoint{2.861102in}{1.149312in}}%
\pgfpathlineto{\pgfqpoint{2.815844in}{1.201831in}}%
\pgfpathlineto{\pgfqpoint{2.773448in}{1.254350in}}%
\pgfpathlineto{\pgfqpoint{2.734109in}{1.306869in}}%
\pgfpathlineto{\pgfqpoint{2.697497in}{1.359388in}}%
\pgfpathlineto{\pgfqpoint{2.663625in}{1.411906in}}%
\pgfpathlineto{\pgfqpoint{2.628862in}{1.470916in}}%
\pgfpathlineto{\pgfqpoint{2.599289in}{1.526338in}}%
\pgfpathlineto{\pgfqpoint{2.578304in}{1.569463in}}%
\pgfpathlineto{\pgfqpoint{2.555142in}{1.621982in}}%
\pgfpathlineto{\pgfqpoint{2.534494in}{1.674501in}}%
\pgfpathlineto{\pgfqpoint{2.516507in}{1.727020in}}%
\pgfpathlineto{\pgfqpoint{2.501163in}{1.779539in}}%
\pgfpathlineto{\pgfqpoint{2.488396in}{1.832058in}}%
\pgfpathlineto{\pgfqpoint{2.478251in}{1.884577in}}%
\pgfpathlineto{\pgfqpoint{2.471010in}{1.937096in}}%
\pgfpathlineto{\pgfqpoint{2.466512in}{1.989615in}}%
\pgfpathlineto{\pgfqpoint{2.464931in}{2.042134in}}%
\pgfpathlineto{\pgfqpoint{2.466447in}{2.094653in}}%
\pgfpathlineto{\pgfqpoint{2.471242in}{2.147172in}}%
\pgfpathlineto{\pgfqpoint{2.479501in}{2.199691in}}%
\pgfpathlineto{\pgfqpoint{2.485234in}{2.225950in}}%
\pgfpathlineto{\pgfqpoint{2.499924in}{2.278469in}}%
\pgfpathlineto{\pgfqpoint{2.510569in}{2.309328in}}%
\pgfpathlineto{\pgfqpoint{2.531363in}{2.357248in}}%
\pgfpathlineto{\pgfqpoint{2.544884in}{2.383507in}}%
\pgfpathlineto{\pgfqpoint{2.569716in}{2.424058in}}%
\pgfpathlineto{\pgfqpoint{2.578291in}{2.436026in}}%
\pgfpathlineto{\pgfqpoint{2.599289in}{2.463158in}}%
\pgfpathlineto{\pgfqpoint{2.628862in}{2.494724in}}%
\pgfpathlineto{\pgfqpoint{2.658436in}{2.520641in}}%
\pgfpathlineto{\pgfqpoint{2.688009in}{2.541897in}}%
\pgfpathlineto{\pgfqpoint{2.717582in}{2.559018in}}%
\pgfpathlineto{\pgfqpoint{2.747156in}{2.572694in}}%
\pgfpathlineto{\pgfqpoint{2.776729in}{2.583170in}}%
\pgfpathlineto{\pgfqpoint{2.806303in}{2.590825in}}%
\pgfpathlineto{\pgfqpoint{2.835876in}{2.595789in}}%
\pgfpathlineto{\pgfqpoint{2.865449in}{2.598257in}}%
\pgfpathlineto{\pgfqpoint{2.895023in}{2.598418in}}%
\pgfpathlineto{\pgfqpoint{2.924596in}{2.596395in}}%
\pgfpathlineto{\pgfqpoint{2.954169in}{2.592289in}}%
\pgfpathlineto{\pgfqpoint{2.983743in}{2.586165in}}%
\pgfpathlineto{\pgfqpoint{3.013316in}{2.578175in}}%
\pgfpathlineto{\pgfqpoint{3.045689in}{2.567323in}}%
\pgfpathlineto{\pgfqpoint{3.072463in}{2.556883in}}%
\pgfpathlineto{\pgfqpoint{3.107404in}{2.541064in}}%
\pgfpathlineto{\pgfqpoint{3.131610in}{2.529015in}}%
\pgfpathlineto{\pgfqpoint{3.161183in}{2.512831in}}%
\pgfpathlineto{\pgfqpoint{3.201057in}{2.488545in}}%
\pgfpathlineto{\pgfqpoint{3.249903in}{2.455987in}}%
\pgfpathlineto{\pgfqpoint{3.311829in}{2.409766in}}%
\pgfpathlineto{\pgfqpoint{3.368197in}{2.363794in}}%
\pgfpathlineto{\pgfqpoint{3.435131in}{2.304729in}}%
\pgfpathlineto{\pgfqpoint{3.491374in}{2.252210in}}%
\pgfpathlineto{\pgfqpoint{3.545637in}{2.199381in}}%
\pgfpathlineto{\pgfqpoint{3.634357in}{2.109002in}}%
\pgfpathlineto{\pgfqpoint{3.752650in}{1.983160in}}%
\pgfpathlineto{\pgfqpoint{3.913574in}{1.805799in}}%
\pgfpathlineto{\pgfqpoint{4.048384in}{1.654649in}}%
\pgfpathlineto{\pgfqpoint{4.216859in}{1.463870in}}%
\pgfpathlineto{\pgfqpoint{4.216859in}{1.463870in}}%
\pgfusepath{stroke}%
\end{pgfscope}%
\begin{pgfscope}%
\pgfpathrectangle{\pgfqpoint{0.854460in}{0.571603in}}{\pgfqpoint{5.885100in}{5.225635in}}%
\pgfusepath{clip}%
\pgfsetbuttcap%
\pgfsetroundjoin%
\pgfsetlinewidth{1.505625pt}%
\definecolor{currentstroke}{rgb}{0.278791,0.062145,0.386592}%
\pgfsetstrokecolor{currentstroke}%
\pgfsetdash{}{0pt}%
\pgfpathmoveto{\pgfqpoint{4.470801in}{1.173541in}}%
\pgfpathlineto{\pgfqpoint{4.491834in}{1.149312in}}%
\pgfpathlineto{\pgfqpoint{4.491984in}{1.149137in}}%
\pgfpathlineto{\pgfqpoint{4.514386in}{1.123052in}}%
\pgfpathlineto{\pgfqpoint{4.521558in}{1.114719in}}%
\pgfpathlineto{\pgfqpoint{4.536919in}{1.096793in}}%
\pgfpathlineto{\pgfqpoint{4.551131in}{1.080234in}}%
\pgfpathlineto{\pgfqpoint{4.559426in}{1.070533in}}%
\pgfpathlineto{\pgfqpoint{4.580705in}{1.045672in}}%
\pgfpathlineto{\pgfqpoint{4.581897in}{1.044274in}}%
\pgfpathlineto{\pgfqpoint{4.604115in}{1.018014in}}%
\pgfpathlineto{\pgfqpoint{4.610278in}{1.010713in}}%
\pgfpathlineto{\pgfqpoint{4.626239in}{0.991755in}}%
\pgfpathlineto{\pgfqpoint{4.639851in}{0.975564in}}%
\pgfpathlineto{\pgfqpoint{4.648299in}{0.965495in}}%
\pgfpathlineto{\pgfqpoint{4.669425in}{0.940263in}}%
\pgfpathlineto{\pgfqpoint{4.670283in}{0.939236in}}%
\pgfpathlineto{\pgfqpoint{4.691938in}{0.912976in}}%
\pgfpathlineto{\pgfqpoint{4.698998in}{0.904367in}}%
\pgfpathlineto{\pgfqpoint{4.713461in}{0.886717in}}%
\pgfpathlineto{\pgfqpoint{4.728571in}{0.868183in}}%
\pgfpathlineto{\pgfqpoint{4.734868in}{0.860458in}}%
\pgfpathlineto{\pgfqpoint{4.756072in}{0.834198in}}%
\pgfpathlineto{\pgfqpoint{4.758145in}{0.831582in}}%
\pgfpathlineto{\pgfqpoint{4.776894in}{0.807939in}}%
\pgfpathlineto{\pgfqpoint{4.787718in}{0.794169in}}%
\pgfpathlineto{\pgfqpoint{4.797548in}{0.781679in}}%
\pgfpathlineto{\pgfqpoint{4.817291in}{0.756347in}}%
\pgfpathlineto{\pgfqpoint{4.818015in}{0.755420in}}%
\pgfpathlineto{\pgfqpoint{4.837949in}{0.729160in}}%
\pgfpathlineto{\pgfqpoint{4.846865in}{0.717241in}}%
\pgfpathlineto{\pgfqpoint{4.857619in}{0.702901in}}%
\pgfpathlineto{\pgfqpoint{4.876438in}{0.677437in}}%
\pgfpathlineto{\pgfqpoint{4.877028in}{0.676641in}}%
\pgfpathlineto{\pgfqpoint{4.895758in}{0.650382in}}%
\pgfpathlineto{\pgfqpoint{4.906012in}{0.635703in}}%
\pgfpathlineto{\pgfqpoint{4.914136in}{0.624122in}}%
\pgfpathlineto{\pgfqpoint{4.932012in}{0.597863in}}%
\pgfpathlineto{\pgfqpoint{4.935585in}{0.592358in}}%
\pgfpathlineto{\pgfqpoint{4.949135in}{0.571603in}}%
\pgfusepath{stroke}%
\end{pgfscope}%
\begin{pgfscope}%
\pgfpathrectangle{\pgfqpoint{0.854460in}{0.571603in}}{\pgfqpoint{5.885100in}{5.225635in}}%
\pgfusepath{clip}%
\pgfsetbuttcap%
\pgfsetroundjoin%
\pgfsetlinewidth{1.505625pt}%
\definecolor{currentstroke}{rgb}{0.282327,0.094955,0.417331}%
\pgfsetstrokecolor{currentstroke}%
\pgfsetdash{}{0pt}%
\pgfpathmoveto{\pgfqpoint{3.320897in}{0.571603in}}%
\pgfpathlineto{\pgfqpoint{3.241386in}{0.624122in}}%
\pgfpathlineto{\pgfqpoint{3.161183in}{0.679973in}}%
\pgfpathlineto{\pgfqpoint{3.094084in}{0.729160in}}%
\pgfpathlineto{\pgfqpoint{3.013316in}{0.791741in}}%
\pgfpathlineto{\pgfqpoint{2.954169in}{0.840053in}}%
\pgfpathlineto{\pgfqpoint{2.895023in}{0.890818in}}%
\pgfpathlineto{\pgfqpoint{2.835876in}{0.944361in}}%
\pgfpathlineto{\pgfqpoint{2.776729in}{1.001038in}}%
\pgfpathlineto{\pgfqpoint{2.717582in}{1.061240in}}%
\pgfpathlineto{\pgfqpoint{2.684320in}{1.096793in}}%
\pgfpathlineto{\pgfqpoint{2.628862in}{1.159588in}}%
\pgfpathlineto{\pgfqpoint{2.593597in}{1.201831in}}%
\pgfpathlineto{\pgfqpoint{2.552204in}{1.254350in}}%
\pgfpathlineto{\pgfqpoint{2.510569in}{1.310614in}}%
\pgfpathlineto{\pgfqpoint{2.476815in}{1.359388in}}%
\pgfpathlineto{\pgfqpoint{2.442837in}{1.411906in}}%
\pgfpathlineto{\pgfqpoint{2.411190in}{1.464425in}}%
\pgfpathlineto{\pgfqpoint{2.381843in}{1.516944in}}%
\pgfpathlineto{\pgfqpoint{2.354752in}{1.569463in}}%
\pgfpathlineto{\pgfqpoint{2.329861in}{1.621982in}}%
\pgfpathlineto{\pgfqpoint{2.303555in}{1.683593in}}%
\pgfpathlineto{\pgfqpoint{2.286822in}{1.727020in}}%
\pgfpathlineto{\pgfqpoint{2.268455in}{1.779539in}}%
\pgfpathlineto{\pgfqpoint{2.252290in}{1.832058in}}%
\pgfpathlineto{\pgfqpoint{2.238208in}{1.884577in}}%
\pgfpathlineto{\pgfqpoint{2.226295in}{1.937096in}}%
\pgfpathlineto{\pgfqpoint{2.214835in}{1.999222in}}%
\pgfpathlineto{\pgfqpoint{2.208712in}{2.042134in}}%
\pgfpathlineto{\pgfqpoint{2.203184in}{2.094653in}}%
\pgfpathlineto{\pgfqpoint{2.199788in}{2.147172in}}%
\pgfpathlineto{\pgfqpoint{2.198587in}{2.199691in}}%
\pgfpathlineto{\pgfqpoint{2.199638in}{2.252210in}}%
\pgfpathlineto{\pgfqpoint{2.202997in}{2.304729in}}%
\pgfpathlineto{\pgfqpoint{2.208708in}{2.357248in}}%
\pgfpathlineto{\pgfqpoint{2.216899in}{2.409766in}}%
\pgfpathlineto{\pgfqpoint{2.227905in}{2.462285in}}%
\pgfpathlineto{\pgfqpoint{2.244409in}{2.524783in}}%
\pgfpathlineto{\pgfqpoint{2.258291in}{2.567323in}}%
\pgfpathlineto{\pgfqpoint{2.278191in}{2.619842in}}%
\pgfpathlineto{\pgfqpoint{2.303555in}{2.675954in}}%
\pgfpathlineto{\pgfqpoint{2.333129in}{2.730852in}}%
\pgfpathlineto{\pgfqpoint{2.362702in}{2.777810in}}%
\pgfpathlineto{\pgfqpoint{2.392275in}{2.818450in}}%
\pgfpathlineto{\pgfqpoint{2.423604in}{2.856177in}}%
\pgfpathlineto{\pgfqpoint{2.451422in}{2.885754in}}%
\pgfpathlineto{\pgfqpoint{2.480996in}{2.913618in}}%
\pgfpathlineto{\pgfqpoint{2.510569in}{2.938265in}}%
\pgfpathlineto{\pgfqpoint{2.542015in}{2.961215in}}%
\pgfpathlineto{\pgfqpoint{2.584806in}{2.987475in}}%
\pgfpathlineto{\pgfqpoint{2.599289in}{2.995471in}}%
\pgfpathlineto{\pgfqpoint{2.638922in}{3.013734in}}%
\pgfpathlineto{\pgfqpoint{2.658436in}{3.021531in}}%
\pgfpathlineto{\pgfqpoint{2.688009in}{3.031290in}}%
\pgfpathlineto{\pgfqpoint{2.722650in}{3.039994in}}%
\pgfpathlineto{\pgfqpoint{2.747156in}{3.044671in}}%
\pgfpathlineto{\pgfqpoint{2.776729in}{3.048378in}}%
\pgfpathlineto{\pgfqpoint{2.806303in}{3.050171in}}%
\pgfpathlineto{\pgfqpoint{2.835876in}{3.050081in}}%
\pgfpathlineto{\pgfqpoint{2.865449in}{3.048132in}}%
\pgfpathlineto{\pgfqpoint{2.895023in}{3.044353in}}%
\pgfpathlineto{\pgfqpoint{2.924596in}{3.038760in}}%
\pgfpathlineto{\pgfqpoint{2.954169in}{3.031344in}}%
\pgfpathlineto{\pgfqpoint{2.983743in}{3.022173in}}%
\pgfpathlineto{\pgfqpoint{3.013316in}{3.011260in}}%
\pgfpathlineto{\pgfqpoint{3.042890in}{2.998596in}}%
\pgfpathlineto{\pgfqpoint{3.072463in}{2.984253in}}%
\pgfpathlineto{\pgfqpoint{3.113769in}{2.961215in}}%
\pgfpathlineto{\pgfqpoint{3.131610in}{2.950569in}}%
\pgfpathlineto{\pgfqpoint{3.161183in}{2.931330in}}%
\pgfpathlineto{\pgfqpoint{3.193200in}{2.908696in}}%
\pgfpathlineto{\pgfqpoint{3.227570in}{2.882437in}}%
\pgfpathlineto{\pgfqpoint{3.279476in}{2.839487in}}%
\pgfpathlineto{\pgfqpoint{3.338623in}{2.785546in}}%
\pgfpathlineto{\pgfqpoint{3.399841in}{2.724880in}}%
\pgfpathlineto{\pgfqpoint{3.456917in}{2.664673in}}%
\pgfpathlineto{\pgfqpoint{3.545637in}{2.565274in}}%
\pgfpathlineto{\pgfqpoint{3.634357in}{2.460792in}}%
\pgfpathlineto{\pgfqpoint{3.783692in}{2.278469in}}%
\pgfpathlineto{\pgfqpoint{4.060099in}{1.937096in}}%
\pgfpathlineto{\pgfqpoint{4.255874in}{1.700761in}}%
\pgfpathlineto{\pgfqpoint{4.389314in}{1.543204in}}%
\pgfpathlineto{\pgfqpoint{4.580705in}{1.322359in}}%
\pgfpathlineto{\pgfqpoint{4.722109in}{1.162052in}}%
\pgfpathlineto{\pgfqpoint{4.722109in}{1.162052in}}%
\pgfusepath{stroke}%
\end{pgfscope}%
\begin{pgfscope}%
\pgfpathrectangle{\pgfqpoint{0.854460in}{0.571603in}}{\pgfqpoint{5.885100in}{5.225635in}}%
\pgfusepath{clip}%
\pgfsetbuttcap%
\pgfsetroundjoin%
\pgfsetlinewidth{1.505625pt}%
\definecolor{currentstroke}{rgb}{0.282327,0.094955,0.417331}%
\pgfsetstrokecolor{currentstroke}%
\pgfsetdash{}{0pt}%
\pgfpathmoveto{\pgfqpoint{4.978414in}{0.873986in}}%
\pgfpathlineto{\pgfqpoint{4.990401in}{0.860458in}}%
\pgfpathlineto{\pgfqpoint{4.994732in}{0.855554in}}%
\pgfpathlineto{\pgfqpoint{5.013603in}{0.834198in}}%
\pgfpathlineto{\pgfqpoint{5.024305in}{0.822067in}}%
\pgfpathlineto{\pgfqpoint{5.036783in}{0.807939in}}%
\pgfpathlineto{\pgfqpoint{5.053878in}{0.788537in}}%
\pgfpathlineto{\pgfqpoint{5.059929in}{0.781679in}}%
\pgfpathlineto{\pgfqpoint{5.083020in}{0.755420in}}%
\pgfpathlineto{\pgfqpoint{5.083452in}{0.754923in}}%
\pgfpathlineto{\pgfqpoint{5.105885in}{0.729160in}}%
\pgfpathlineto{\pgfqpoint{5.113025in}{0.720921in}}%
\pgfpathlineto{\pgfqpoint{5.128680in}{0.702901in}}%
\pgfpathlineto{\pgfqpoint{5.142599in}{0.686790in}}%
\pgfpathlineto{\pgfqpoint{5.151392in}{0.676641in}}%
\pgfpathlineto{\pgfqpoint{5.172172in}{0.652504in}}%
\pgfpathlineto{\pgfqpoint{5.174005in}{0.650382in}}%
\pgfpathlineto{\pgfqpoint{5.196355in}{0.624122in}}%
\pgfpathlineto{\pgfqpoint{5.201745in}{0.617715in}}%
\pgfpathlineto{\pgfqpoint{5.218518in}{0.597863in}}%
\pgfpathlineto{\pgfqpoint{5.231319in}{0.582563in}}%
\pgfpathlineto{\pgfqpoint{5.240530in}{0.571603in}}%
\pgfusepath{stroke}%
\end{pgfscope}%
\begin{pgfscope}%
\pgfpathrectangle{\pgfqpoint{0.854460in}{0.571603in}}{\pgfqpoint{5.885100in}{5.225635in}}%
\pgfusepath{clip}%
\pgfsetbuttcap%
\pgfsetroundjoin%
\pgfsetlinewidth{1.505625pt}%
\definecolor{currentstroke}{rgb}{0.283229,0.120777,0.440584}%
\pgfsetstrokecolor{currentstroke}%
\pgfsetdash{}{0pt}%
\pgfpathmoveto{\pgfqpoint{3.111874in}{0.571603in}}%
\pgfpathlineto{\pgfqpoint{3.036974in}{0.624122in}}%
\pgfpathlineto{\pgfqpoint{2.954169in}{0.685284in}}%
\pgfpathlineto{\pgfqpoint{2.895023in}{0.730992in}}%
\pgfpathlineto{\pgfqpoint{2.832351in}{0.781679in}}%
\pgfpathlineto{\pgfqpoint{2.770439in}{0.834198in}}%
\pgfpathlineto{\pgfqpoint{2.711451in}{0.886717in}}%
\pgfpathlineto{\pgfqpoint{2.655289in}{0.939236in}}%
\pgfpathlineto{\pgfqpoint{2.599289in}{0.994422in}}%
\pgfpathlineto{\pgfqpoint{2.540142in}{1.056223in}}%
\pgfpathlineto{\pgfqpoint{2.503165in}{1.096793in}}%
\pgfpathlineto{\pgfqpoint{2.451422in}{1.156674in}}%
\pgfpathlineto{\pgfqpoint{2.414477in}{1.201831in}}%
\pgfpathlineto{\pgfqpoint{2.362702in}{1.269182in}}%
\pgfpathlineto{\pgfqpoint{2.333129in}{1.309959in}}%
\pgfpathlineto{\pgfqpoint{2.299196in}{1.359388in}}%
\pgfpathlineto{\pgfqpoint{2.265319in}{1.411906in}}%
\pgfpathlineto{\pgfqpoint{2.233586in}{1.464425in}}%
\pgfpathlineto{\pgfqpoint{2.203964in}{1.516944in}}%
\pgfpathlineto{\pgfqpoint{2.176410in}{1.569463in}}%
\pgfpathlineto{\pgfqpoint{2.150874in}{1.621982in}}%
\pgfpathlineto{\pgfqpoint{2.126115in}{1.677374in}}%
\pgfpathlineto{\pgfqpoint{2.095749in}{1.753280in}}%
\pgfpathlineto{\pgfqpoint{2.068520in}{1.832058in}}%
\pgfpathlineto{\pgfqpoint{2.052817in}{1.884577in}}%
\pgfpathlineto{\pgfqpoint{2.037395in}{1.942959in}}%
\pgfpathlineto{\pgfqpoint{2.026824in}{1.989615in}}%
\pgfpathlineto{\pgfqpoint{2.016594in}{2.042134in}}%
\pgfpathlineto{\pgfqpoint{2.007822in}{2.096853in}}%
\pgfpathlineto{\pgfqpoint{2.001639in}{2.147172in}}%
\pgfpathlineto{\pgfqpoint{1.996955in}{2.199691in}}%
\pgfpathlineto{\pgfqpoint{1.994088in}{2.252210in}}%
\pgfpathlineto{\pgfqpoint{1.993068in}{2.304729in}}%
\pgfpathlineto{\pgfqpoint{1.993917in}{2.357248in}}%
\pgfpathlineto{\pgfqpoint{1.996657in}{2.409766in}}%
\pgfpathlineto{\pgfqpoint{2.001300in}{2.462285in}}%
\pgfpathlineto{\pgfqpoint{2.007856in}{2.514804in}}%
\pgfpathlineto{\pgfqpoint{2.016639in}{2.567323in}}%
\pgfpathlineto{\pgfqpoint{2.027408in}{2.619842in}}%
\pgfpathlineto{\pgfqpoint{2.040259in}{2.672361in}}%
\pgfpathlineto{\pgfqpoint{2.055541in}{2.724880in}}%
\pgfpathlineto{\pgfqpoint{2.073079in}{2.777399in}}%
\pgfpathlineto{\pgfqpoint{2.096542in}{2.838179in}}%
\pgfpathlineto{\pgfqpoint{2.116068in}{2.882437in}}%
\pgfpathlineto{\pgfqpoint{2.141871in}{2.934956in}}%
\pgfpathlineto{\pgfqpoint{2.170909in}{2.987475in}}%
\pgfpathlineto{\pgfqpoint{2.186612in}{3.013734in}}%
\pgfpathlineto{\pgfqpoint{2.221387in}{3.066253in}}%
\pgfpathlineto{\pgfqpoint{2.260633in}{3.118772in}}%
\pgfpathlineto{\pgfqpoint{2.282138in}{3.145032in}}%
\pgfpathlineto{\pgfqpoint{2.305035in}{3.171291in}}%
\pgfpathlineto{\pgfqpoint{2.333129in}{3.201024in}}%
\pgfpathlineto{\pgfqpoint{2.362702in}{3.229777in}}%
\pgfpathlineto{\pgfqpoint{2.392275in}{3.256177in}}%
\pgfpathlineto{\pgfqpoint{2.421849in}{3.280394in}}%
\pgfpathlineto{\pgfqpoint{2.451447in}{3.302589in}}%
\pgfpathlineto{\pgfqpoint{2.490766in}{3.328848in}}%
\pgfpathlineto{\pgfqpoint{2.510569in}{3.341074in}}%
\pgfpathlineto{\pgfqpoint{2.540142in}{3.357678in}}%
\pgfpathlineto{\pgfqpoint{2.589522in}{3.381367in}}%
\pgfpathlineto{\pgfqpoint{2.599289in}{3.385693in}}%
\pgfpathlineto{\pgfqpoint{2.628862in}{3.397209in}}%
\pgfpathlineto{\pgfqpoint{2.660050in}{3.407626in}}%
\pgfpathlineto{\pgfqpoint{2.688009in}{3.415462in}}%
\pgfpathlineto{\pgfqpoint{2.717582in}{3.422196in}}%
\pgfpathlineto{\pgfqpoint{2.747156in}{3.427364in}}%
\pgfpathlineto{\pgfqpoint{2.776729in}{3.430951in}}%
\pgfpathlineto{\pgfqpoint{2.806303in}{3.432943in}}%
\pgfpathlineto{\pgfqpoint{2.835876in}{3.433324in}}%
\pgfpathlineto{\pgfqpoint{2.865449in}{3.432078in}}%
\pgfpathlineto{\pgfqpoint{2.895023in}{3.429188in}}%
\pgfpathlineto{\pgfqpoint{2.924596in}{3.424637in}}%
\pgfpathlineto{\pgfqpoint{2.954169in}{3.418408in}}%
\pgfpathlineto{\pgfqpoint{2.992540in}{3.407626in}}%
\pgfpathlineto{\pgfqpoint{3.013316in}{3.400794in}}%
\pgfpathlineto{\pgfqpoint{3.060808in}{3.381367in}}%
\pgfpathlineto{\pgfqpoint{3.072463in}{3.376120in}}%
\pgfpathlineto{\pgfqpoint{3.112557in}{3.355107in}}%
\pgfpathlineto{\pgfqpoint{3.131610in}{3.344204in}}%
\pgfpathlineto{\pgfqpoint{3.161183in}{3.325509in}}%
\pgfpathlineto{\pgfqpoint{3.193920in}{3.302589in}}%
\pgfpathlineto{\pgfqpoint{3.228022in}{3.276329in}}%
\pgfpathlineto{\pgfqpoint{3.259437in}{3.250070in}}%
\pgfpathlineto{\pgfqpoint{3.288731in}{3.223810in}}%
\pgfpathlineto{\pgfqpoint{3.316328in}{3.197551in}}%
\pgfpathlineto{\pgfqpoint{3.342552in}{3.171291in}}%
\pgfpathlineto{\pgfqpoint{3.397770in}{3.111951in}}%
\pgfpathlineto{\pgfqpoint{3.459181in}{3.039994in}}%
\pgfpathlineto{\pgfqpoint{3.521898in}{2.961215in}}%
\pgfpathlineto{\pgfqpoint{3.581555in}{2.882437in}}%
\pgfpathlineto{\pgfqpoint{3.663930in}{2.769436in}}%
\pgfpathlineto{\pgfqpoint{3.824952in}{2.541543in}}%
\pgfpathlineto{\pgfqpoint{3.824952in}{2.541543in}}%
\pgfusepath{stroke}%
\end{pgfscope}%
\begin{pgfscope}%
\pgfpathrectangle{\pgfqpoint{0.854460in}{0.571603in}}{\pgfqpoint{5.885100in}{5.225635in}}%
\pgfusepath{clip}%
\pgfsetbuttcap%
\pgfsetroundjoin%
\pgfsetlinewidth{1.505625pt}%
\definecolor{currentstroke}{rgb}{0.283229,0.120777,0.440584}%
\pgfsetstrokecolor{currentstroke}%
\pgfsetdash{}{0pt}%
\pgfpathmoveto{\pgfqpoint{4.050252in}{2.226569in}}%
\pgfpathlineto{\pgfqpoint{4.050698in}{2.225950in}}%
\pgfpathlineto{\pgfqpoint{4.069901in}{2.199691in}}%
\pgfpathlineto{\pgfqpoint{4.077957in}{2.188846in}}%
\pgfpathlineto{\pgfqpoint{4.089233in}{2.173431in}}%
\pgfpathlineto{\pgfqpoint{4.107531in}{2.148836in}}%
\pgfpathlineto{\pgfqpoint{4.108750in}{2.147172in}}%
\pgfpathlineto{\pgfqpoint{4.128230in}{2.120912in}}%
\pgfpathlineto{\pgfqpoint{4.137104in}{2.109138in}}%
\pgfpathlineto{\pgfqpoint{4.147868in}{2.094653in}}%
\pgfpathlineto{\pgfqpoint{4.166677in}{2.069756in}}%
\pgfpathlineto{\pgfqpoint{4.167693in}{2.068393in}}%
\pgfpathlineto{\pgfqpoint{4.187488in}{2.042134in}}%
\pgfpathlineto{\pgfqpoint{4.196251in}{2.030690in}}%
\pgfpathlineto{\pgfqpoint{4.207448in}{2.015874in}}%
\pgfpathlineto{\pgfqpoint{4.225824in}{1.991948in}}%
\pgfpathlineto{\pgfqpoint{4.227594in}{1.989615in}}%
\pgfpathlineto{\pgfqpoint{4.247739in}{1.963355in}}%
\pgfpathlineto{\pgfqpoint{4.255398in}{1.953518in}}%
\pgfpathlineto{\pgfqpoint{4.268030in}{1.937096in}}%
\pgfpathlineto{\pgfqpoint{4.284971in}{1.915412in}}%
\pgfpathlineto{\pgfqpoint{4.288505in}{1.910836in}}%
\pgfpathlineto{\pgfqpoint{4.309031in}{1.884577in}}%
\pgfpathlineto{\pgfqpoint{4.314544in}{1.877614in}}%
\pgfpathlineto{\pgfqpoint{4.329657in}{1.858318in}}%
\pgfpathlineto{\pgfqpoint{4.344118in}{1.840127in}}%
\pgfpathlineto{\pgfqpoint{4.350466in}{1.832058in}}%
\pgfpathlineto{\pgfqpoint{4.371399in}{1.805799in}}%
\pgfpathlineto{\pgfqpoint{4.373691in}{1.802953in}}%
\pgfpathlineto{\pgfqpoint{4.392363in}{1.779539in}}%
\pgfpathlineto{\pgfqpoint{4.403264in}{1.766061in}}%
\pgfpathlineto{\pgfqpoint{4.413505in}{1.753280in}}%
\pgfpathlineto{\pgfqpoint{4.432838in}{1.729485in}}%
\pgfpathlineto{\pgfqpoint{4.434822in}{1.727020in}}%
\pgfpathlineto{\pgfqpoint{4.456168in}{1.700761in}}%
\pgfpathlineto{\pgfqpoint{4.462411in}{1.693173in}}%
\pgfpathlineto{\pgfqpoint{4.477643in}{1.674501in}}%
\pgfpathlineto{\pgfqpoint{4.491984in}{1.657149in}}%
\pgfpathlineto{\pgfqpoint{4.499287in}{1.648242in}}%
\pgfpathlineto{\pgfqpoint{4.521087in}{1.621982in}}%
\pgfpathlineto{\pgfqpoint{4.521558in}{1.621419in}}%
\pgfpathlineto{\pgfqpoint{4.542887in}{1.595723in}}%
\pgfpathlineto{\pgfqpoint{4.551131in}{1.585911in}}%
\pgfpathlineto{\pgfqpoint{4.564852in}{1.569463in}}%
\pgfpathlineto{\pgfqpoint{4.580705in}{1.550686in}}%
\pgfpathlineto{\pgfqpoint{4.586978in}{1.543204in}}%
\pgfpathlineto{\pgfqpoint{4.609238in}{1.516944in}}%
\pgfpathlineto{\pgfqpoint{4.610278in}{1.515726in}}%
\pgfpathlineto{\pgfqpoint{4.631514in}{1.490685in}}%
\pgfpathlineto{\pgfqpoint{4.639851in}{1.480962in}}%
\pgfpathlineto{\pgfqpoint{4.653945in}{1.464425in}}%
\pgfpathlineto{\pgfqpoint{4.669425in}{1.446458in}}%
\pgfpathlineto{\pgfqpoint{4.676527in}{1.438166in}}%
\pgfpathlineto{\pgfqpoint{4.698998in}{1.412207in}}%
\pgfpathlineto{\pgfqpoint{4.699256in}{1.411906in}}%
\pgfpathlineto{\pgfqpoint{4.721985in}{1.385647in}}%
\pgfpathlineto{\pgfqpoint{4.728571in}{1.378109in}}%
\pgfpathlineto{\pgfqpoint{4.744851in}{1.359388in}}%
\pgfpathlineto{\pgfqpoint{4.758145in}{1.344243in}}%
\pgfpathlineto{\pgfqpoint{4.767857in}{1.333128in}}%
\pgfpathlineto{\pgfqpoint{4.787718in}{1.310604in}}%
\pgfpathlineto{\pgfqpoint{4.790998in}{1.306869in}}%
\pgfpathlineto{\pgfqpoint{4.814203in}{1.280609in}}%
\pgfpathlineto{\pgfqpoint{4.817291in}{1.277134in}}%
\pgfpathlineto{\pgfqpoint{4.837466in}{1.254350in}}%
\pgfpathlineto{\pgfqpoint{4.846865in}{1.243819in}}%
\pgfpathlineto{\pgfqpoint{4.860855in}{1.228090in}}%
\pgfpathlineto{\pgfqpoint{4.876438in}{1.210704in}}%
\pgfpathlineto{\pgfqpoint{4.884366in}{1.201831in}}%
\pgfpathlineto{\pgfqpoint{4.906012in}{1.177781in}}%
\pgfpathlineto{\pgfqpoint{4.907995in}{1.175571in}}%
\pgfpathlineto{\pgfqpoint{4.931650in}{1.149312in}}%
\pgfpathlineto{\pgfqpoint{4.935585in}{1.144965in}}%
\pgfpathlineto{\pgfqpoint{4.955373in}{1.123052in}}%
\pgfpathlineto{\pgfqpoint{4.965158in}{1.112282in}}%
\pgfpathlineto{\pgfqpoint{4.979202in}{1.096793in}}%
\pgfpathlineto{\pgfqpoint{4.994732in}{1.079762in}}%
\pgfpathlineto{\pgfqpoint{5.003133in}{1.070533in}}%
\pgfpathlineto{\pgfqpoint{5.024305in}{1.047397in}}%
\pgfpathlineto{\pgfqpoint{5.027159in}{1.044274in}}%
\pgfpathlineto{\pgfqpoint{5.051217in}{1.018014in}}%
\pgfpathlineto{\pgfqpoint{5.053878in}{1.015115in}}%
\pgfpathlineto{\pgfqpoint{5.075300in}{0.991755in}}%
\pgfpathlineto{\pgfqpoint{5.083452in}{0.982900in}}%
\pgfpathlineto{\pgfqpoint{5.099464in}{0.965495in}}%
\pgfpathlineto{\pgfqpoint{5.113025in}{0.950807in}}%
\pgfpathlineto{\pgfqpoint{5.123704in}{0.939236in}}%
\pgfpathlineto{\pgfqpoint{5.142599in}{0.918826in}}%
\pgfpathlineto{\pgfqpoint{5.148014in}{0.912976in}}%
\pgfpathlineto{\pgfqpoint{5.172172in}{0.886948in}}%
\pgfpathlineto{\pgfqpoint{5.172387in}{0.886717in}}%
\pgfpathlineto{\pgfqpoint{5.196705in}{0.860458in}}%
\pgfpathlineto{\pgfqpoint{5.201745in}{0.855021in}}%
\pgfpathlineto{\pgfqpoint{5.221069in}{0.834198in}}%
\pgfpathlineto{\pgfqpoint{5.231319in}{0.823164in}}%
\pgfpathlineto{\pgfqpoint{5.245477in}{0.807939in}}%
\pgfpathlineto{\pgfqpoint{5.260892in}{0.791370in}}%
\pgfpathlineto{\pgfqpoint{5.269921in}{0.781679in}}%
\pgfpathlineto{\pgfqpoint{5.290465in}{0.759629in}}%
\pgfpathlineto{\pgfqpoint{5.294394in}{0.755420in}}%
\pgfpathlineto{\pgfqpoint{5.318860in}{0.729160in}}%
\pgfpathlineto{\pgfqpoint{5.320039in}{0.727888in}}%
\pgfpathlineto{\pgfqpoint{5.343246in}{0.702901in}}%
\pgfpathlineto{\pgfqpoint{5.349612in}{0.696033in}}%
\pgfpathlineto{\pgfqpoint{5.367635in}{0.676641in}}%
\pgfpathlineto{\pgfqpoint{5.379185in}{0.664181in}}%
\pgfpathlineto{\pgfqpoint{5.392017in}{0.650382in}}%
\pgfpathlineto{\pgfqpoint{5.408759in}{0.632318in}}%
\pgfpathlineto{\pgfqpoint{5.416380in}{0.624122in}}%
\pgfpathlineto{\pgfqpoint{5.438332in}{0.600426in}}%
\pgfpathlineto{\pgfqpoint{5.440715in}{0.597863in}}%
\pgfpathlineto{\pgfqpoint{5.464941in}{0.571603in}}%
\pgfusepath{stroke}%
\end{pgfscope}%
\begin{pgfscope}%
\pgfpathrectangle{\pgfqpoint{0.854460in}{0.571603in}}{\pgfqpoint{5.885100in}{5.225635in}}%
\pgfusepath{clip}%
\pgfsetbuttcap%
\pgfsetroundjoin%
\pgfsetlinewidth{1.505625pt}%
\definecolor{currentstroke}{rgb}{0.281887,0.150881,0.465405}%
\pgfsetstrokecolor{currentstroke}%
\pgfsetdash{}{0pt}%
\pgfpathmoveto{\pgfqpoint{2.937449in}{0.571603in}}%
\pgfpathlineto{\pgfqpoint{2.865449in}{0.624148in}}%
\pgfpathlineto{\pgfqpoint{2.776729in}{0.692488in}}%
\pgfpathlineto{\pgfqpoint{2.717582in}{0.740298in}}%
\pgfpathlineto{\pgfqpoint{2.658436in}{0.790164in}}%
\pgfpathlineto{\pgfqpoint{2.599289in}{0.842331in}}%
\pgfpathlineto{\pgfqpoint{2.540142in}{0.897064in}}%
\pgfpathlineto{\pgfqpoint{2.480996in}{0.954654in}}%
\pgfpathlineto{\pgfqpoint{2.421849in}{1.015418in}}%
\pgfpathlineto{\pgfqpoint{2.392275in}{1.047209in}}%
\pgfpathlineto{\pgfqpoint{2.333129in}{1.113940in}}%
\pgfpathlineto{\pgfqpoint{2.303200in}{1.149312in}}%
\pgfpathlineto{\pgfqpoint{2.244409in}{1.223053in}}%
\pgfpathlineto{\pgfqpoint{2.214835in}{1.262397in}}%
\pgfpathlineto{\pgfqpoint{2.182802in}{1.306869in}}%
\pgfpathlineto{\pgfqpoint{2.147058in}{1.359388in}}%
\pgfpathlineto{\pgfqpoint{2.113369in}{1.411906in}}%
\pgfpathlineto{\pgfqpoint{2.081708in}{1.464425in}}%
\pgfpathlineto{\pgfqpoint{2.052039in}{1.516944in}}%
\pgfpathlineto{\pgfqpoint{2.024323in}{1.569463in}}%
\pgfpathlineto{\pgfqpoint{1.998509in}{1.621982in}}%
\pgfpathlineto{\pgfqpoint{1.974542in}{1.674501in}}%
\pgfpathlineto{\pgfqpoint{1.948675in}{1.736538in}}%
\pgfpathlineto{\pgfqpoint{1.922716in}{1.805799in}}%
\pgfpathlineto{\pgfqpoint{1.905159in}{1.858318in}}%
\pgfpathlineto{\pgfqpoint{1.889196in}{1.910836in}}%
\pgfpathlineto{\pgfqpoint{1.868691in}{1.989615in}}%
\pgfpathlineto{\pgfqpoint{1.857039in}{2.042134in}}%
\pgfpathlineto{\pgfqpoint{1.847176in}{2.094653in}}%
\pgfpathlineto{\pgfqpoint{1.838897in}{2.147172in}}%
\pgfpathlineto{\pgfqpoint{1.830381in}{2.217546in}}%
\pgfpathlineto{\pgfqpoint{1.827285in}{2.252210in}}%
\pgfpathlineto{\pgfqpoint{1.824006in}{2.304729in}}%
\pgfpathlineto{\pgfqpoint{1.822355in}{2.357248in}}%
\pgfpathlineto{\pgfqpoint{1.822344in}{2.409766in}}%
\pgfpathlineto{\pgfqpoint{1.823983in}{2.462285in}}%
\pgfpathlineto{\pgfqpoint{1.827276in}{2.514804in}}%
\pgfpathlineto{\pgfqpoint{1.832281in}{2.567323in}}%
\pgfpathlineto{\pgfqpoint{1.839089in}{2.619842in}}%
\pgfpathlineto{\pgfqpoint{1.847595in}{2.672361in}}%
\pgfpathlineto{\pgfqpoint{1.859955in}{2.734863in}}%
\pgfpathlineto{\pgfqpoint{1.869968in}{2.777399in}}%
\pgfpathlineto{\pgfqpoint{1.889528in}{2.849272in}}%
\pgfpathlineto{\pgfqpoint{1.899907in}{2.882437in}}%
\pgfpathlineto{\pgfqpoint{1.919102in}{2.938585in}}%
\pgfpathlineto{\pgfqpoint{1.937976in}{2.987475in}}%
\pgfpathlineto{\pgfqpoint{1.960306in}{3.039994in}}%
\pgfpathlineto{\pgfqpoint{1.984961in}{3.092513in}}%
\pgfpathlineto{\pgfqpoint{2.012147in}{3.145032in}}%
\pgfpathlineto{\pgfqpoint{2.042089in}{3.197551in}}%
\pgfpathlineto{\pgfqpoint{2.075035in}{3.250070in}}%
\pgfpathlineto{\pgfqpoint{2.111252in}{3.302589in}}%
\pgfpathlineto{\pgfqpoint{2.130609in}{3.328848in}}%
\pgfpathlineto{\pgfqpoint{2.172569in}{3.381367in}}%
\pgfpathlineto{\pgfqpoint{2.195145in}{3.407626in}}%
\pgfpathlineto{\pgfqpoint{2.218884in}{3.433886in}}%
\pgfpathlineto{\pgfqpoint{2.244409in}{3.460607in}}%
\pgfpathlineto{\pgfqpoint{2.273982in}{3.489617in}}%
\pgfpathlineto{\pgfqpoint{2.303555in}{3.516816in}}%
\pgfpathlineto{\pgfqpoint{2.333129in}{3.542315in}}%
\pgfpathlineto{\pgfqpoint{2.362702in}{3.566208in}}%
\pgfpathlineto{\pgfqpoint{2.396372in}{3.591443in}}%
\pgfpathlineto{\pgfqpoint{2.434455in}{3.617702in}}%
\pgfpathlineto{\pgfqpoint{2.480996in}{3.646920in}}%
\pgfpathlineto{\pgfqpoint{2.523110in}{3.670221in}}%
\pgfpathlineto{\pgfqpoint{2.569716in}{3.693264in}}%
\pgfpathlineto{\pgfqpoint{2.599289in}{3.706134in}}%
\pgfpathlineto{\pgfqpoint{2.642880in}{3.722740in}}%
\pgfpathlineto{\pgfqpoint{2.658436in}{3.728192in}}%
\pgfpathlineto{\pgfqpoint{2.688009in}{3.737328in}}%
\pgfpathlineto{\pgfqpoint{2.734196in}{3.749000in}}%
\pgfpathlineto{\pgfqpoint{2.747156in}{3.751905in}}%
\pgfpathlineto{\pgfqpoint{2.776729in}{3.757253in}}%
\pgfpathlineto{\pgfqpoint{2.806303in}{3.761327in}}%
\pgfpathlineto{\pgfqpoint{2.835876in}{3.764093in}}%
\pgfpathlineto{\pgfqpoint{2.865449in}{3.765518in}}%
\pgfpathlineto{\pgfqpoint{2.895023in}{3.765566in}}%
\pgfpathlineto{\pgfqpoint{2.924596in}{3.764199in}}%
\pgfpathlineto{\pgfqpoint{2.954169in}{3.761378in}}%
\pgfpathlineto{\pgfqpoint{2.983743in}{3.757063in}}%
\pgfpathlineto{\pgfqpoint{3.022177in}{3.749000in}}%
\pgfpathlineto{\pgfqpoint{3.042890in}{3.743715in}}%
\pgfpathlineto{\pgfqpoint{3.072463in}{3.734552in}}%
\pgfpathlineto{\pgfqpoint{3.104313in}{3.722740in}}%
\pgfpathlineto{\pgfqpoint{3.131610in}{3.710994in}}%
\pgfpathlineto{\pgfqpoint{3.161186in}{3.696481in}}%
\pgfpathlineto{\pgfqpoint{3.206521in}{3.670221in}}%
\pgfpathlineto{\pgfqpoint{3.220330in}{3.661515in}}%
\pgfpathlineto{\pgfqpoint{3.249903in}{3.640997in}}%
\pgfpathlineto{\pgfqpoint{3.280268in}{3.617702in}}%
\pgfpathlineto{\pgfqpoint{3.311382in}{3.591443in}}%
\pgfpathlineto{\pgfqpoint{3.340018in}{3.565183in}}%
\pgfpathlineto{\pgfqpoint{3.368197in}{3.537333in}}%
\pgfpathlineto{\pgfqpoint{3.414998in}{3.486405in}}%
\pgfpathlineto{\pgfqpoint{3.437464in}{3.460145in}}%
\pgfpathlineto{\pgfqpoint{3.479636in}{3.407626in}}%
\pgfpathlineto{\pgfqpoint{3.499539in}{3.381367in}}%
\pgfpathlineto{\pgfqpoint{3.545637in}{3.317728in}}%
\pgfpathlineto{\pgfqpoint{3.608856in}{3.223810in}}%
\pgfpathlineto{\pgfqpoint{3.675582in}{3.118772in}}%
\pgfpathlineto{\pgfqpoint{3.752650in}{2.993144in}}%
\pgfpathlineto{\pgfqpoint{3.851039in}{2.829918in}}%
\pgfpathlineto{\pgfqpoint{3.979633in}{2.619842in}}%
\pgfpathlineto{\pgfqpoint{4.096572in}{2.436026in}}%
\pgfpathlineto{\pgfqpoint{4.201375in}{2.278469in}}%
\pgfpathlineto{\pgfqpoint{4.284971in}{2.157724in}}%
\pgfpathlineto{\pgfqpoint{4.348668in}{2.068393in}}%
\pgfpathlineto{\pgfqpoint{4.432838in}{1.953993in}}%
\pgfpathlineto{\pgfqpoint{4.505314in}{1.858318in}}%
\pgfpathlineto{\pgfqpoint{4.610278in}{1.724457in}}%
\pgfpathlineto{\pgfqpoint{4.698998in}{1.614956in}}%
\pgfpathlineto{\pgfqpoint{4.787718in}{1.508522in}}%
\pgfpathlineto{\pgfqpoint{4.876438in}{1.404842in}}%
\pgfpathlineto{\pgfqpoint{4.965158in}{1.303629in}}%
\pgfpathlineto{\pgfqpoint{5.056368in}{1.201831in}}%
\pgfpathlineto{\pgfqpoint{5.176738in}{1.070533in}}%
\pgfpathlineto{\pgfqpoint{5.317428in}{0.920721in}}%
\pgfpathlineto{\pgfqpoint{5.317428in}{0.920721in}}%
\pgfusepath{stroke}%
\end{pgfscope}%
\begin{pgfscope}%
\pgfpathrectangle{\pgfqpoint{0.854460in}{0.571603in}}{\pgfqpoint{5.885100in}{5.225635in}}%
\pgfusepath{clip}%
\pgfsetbuttcap%
\pgfsetroundjoin%
\pgfsetlinewidth{1.505625pt}%
\definecolor{currentstroke}{rgb}{0.281887,0.150881,0.465405}%
\pgfsetstrokecolor{currentstroke}%
\pgfsetdash{}{0pt}%
\pgfpathmoveto{\pgfqpoint{5.584550in}{0.643548in}}%
\pgfpathlineto{\pgfqpoint{5.586199in}{0.641851in}}%
\pgfpathlineto{\pgfqpoint{5.603483in}{0.624122in}}%
\pgfpathlineto{\pgfqpoint{5.615772in}{0.611496in}}%
\pgfpathlineto{\pgfqpoint{5.629088in}{0.597863in}}%
\pgfpathlineto{\pgfqpoint{5.645346in}{0.581183in}}%
\pgfpathlineto{\pgfqpoint{5.654718in}{0.571603in}}%
\pgfusepath{stroke}%
\end{pgfscope}%
\begin{pgfscope}%
\pgfpathrectangle{\pgfqpoint{0.854460in}{0.571603in}}{\pgfqpoint{5.885100in}{5.225635in}}%
\pgfusepath{clip}%
\pgfsetbuttcap%
\pgfsetroundjoin%
\pgfsetlinewidth{1.505625pt}%
\definecolor{currentstroke}{rgb}{0.278826,0.175490,0.483397}%
\pgfsetstrokecolor{currentstroke}%
\pgfsetdash{}{0pt}%
\pgfpathmoveto{\pgfqpoint{2.785093in}{0.571603in}}%
\pgfpathlineto{\pgfqpoint{2.715383in}{0.624122in}}%
\pgfpathlineto{\pgfqpoint{2.628862in}{0.692749in}}%
\pgfpathlineto{\pgfqpoint{2.569716in}{0.741958in}}%
\pgfpathlineto{\pgfqpoint{2.510569in}{0.793289in}}%
\pgfpathlineto{\pgfqpoint{2.451422in}{0.846979in}}%
\pgfpathlineto{\pgfqpoint{2.392275in}{0.903287in}}%
\pgfpathlineto{\pgfqpoint{2.333129in}{0.962497in}}%
\pgfpathlineto{\pgfqpoint{2.303555in}{0.993342in}}%
\pgfpathlineto{\pgfqpoint{2.244409in}{1.057921in}}%
\pgfpathlineto{\pgfqpoint{2.188127in}{1.123052in}}%
\pgfpathlineto{\pgfqpoint{2.145356in}{1.175571in}}%
\pgfpathlineto{\pgfqpoint{2.096542in}{1.239009in}}%
\pgfpathlineto{\pgfqpoint{2.047697in}{1.306869in}}%
\pgfpathlineto{\pgfqpoint{2.007822in}{1.366145in}}%
\pgfpathlineto{\pgfqpoint{1.962806in}{1.438166in}}%
\pgfpathlineto{\pgfqpoint{1.932252in}{1.490685in}}%
\pgfpathlineto{\pgfqpoint{1.903583in}{1.543204in}}%
\pgfpathlineto{\pgfqpoint{1.876752in}{1.595723in}}%
\pgfpathlineto{\pgfqpoint{1.851712in}{1.648242in}}%
\pgfpathlineto{\pgfqpoint{1.828407in}{1.700761in}}%
\pgfpathlineto{\pgfqpoint{1.796785in}{1.779539in}}%
\pgfpathlineto{\pgfqpoint{1.768930in}{1.858318in}}%
\pgfpathlineto{\pgfqpoint{1.744809in}{1.937096in}}%
\pgfpathlineto{\pgfqpoint{1.730795in}{1.989615in}}%
\pgfpathlineto{\pgfqpoint{1.712599in}{2.068393in}}%
\pgfpathlineto{\pgfqpoint{1.702558in}{2.120912in}}%
\pgfpathlineto{\pgfqpoint{1.690277in}{2.199691in}}%
\pgfpathlineto{\pgfqpoint{1.682515in}{2.266866in}}%
\pgfpathlineto{\pgfqpoint{1.679240in}{2.304729in}}%
\pgfpathlineto{\pgfqpoint{1.676049in}{2.357248in}}%
\pgfpathlineto{\pgfqpoint{1.674356in}{2.409766in}}%
\pgfpathlineto{\pgfqpoint{1.674170in}{2.462285in}}%
\pgfpathlineto{\pgfqpoint{1.675492in}{2.514804in}}%
\pgfpathlineto{\pgfqpoint{1.678324in}{2.567323in}}%
\pgfpathlineto{\pgfqpoint{1.682666in}{2.619842in}}%
\pgfpathlineto{\pgfqpoint{1.688672in}{2.672361in}}%
\pgfpathlineto{\pgfqpoint{1.696214in}{2.724880in}}%
\pgfpathlineto{\pgfqpoint{1.705280in}{2.777399in}}%
\pgfpathlineto{\pgfqpoint{1.715965in}{2.829918in}}%
\pgfpathlineto{\pgfqpoint{1.735165in}{2.908696in}}%
\pgfpathlineto{\pgfqpoint{1.750092in}{2.961215in}}%
\pgfpathlineto{\pgfqpoint{1.771235in}{3.027067in}}%
\pgfpathlineto{\pgfqpoint{1.795185in}{3.092513in}}%
\pgfpathlineto{\pgfqpoint{1.816583in}{3.145032in}}%
\pgfpathlineto{\pgfqpoint{1.839933in}{3.197551in}}%
\pgfpathlineto{\pgfqpoint{1.865373in}{3.250070in}}%
\pgfpathlineto{\pgfqpoint{1.893055in}{3.302589in}}%
\pgfpathlineto{\pgfqpoint{1.923135in}{3.355107in}}%
\pgfpathlineto{\pgfqpoint{1.955780in}{3.407626in}}%
\pgfpathlineto{\pgfqpoint{1.991163in}{3.460145in}}%
\pgfpathlineto{\pgfqpoint{2.029464in}{3.512664in}}%
\pgfpathlineto{\pgfqpoint{2.066968in}{3.560290in}}%
\pgfpathlineto{\pgfqpoint{2.096542in}{3.595375in}}%
\pgfpathlineto{\pgfqpoint{2.140567in}{3.643962in}}%
\pgfpathlineto{\pgfqpoint{2.185262in}{3.689702in}}%
\pgfpathlineto{\pgfqpoint{2.220068in}{3.722740in}}%
\pgfpathlineto{\pgfqpoint{2.273982in}{3.770194in}}%
\pgfpathlineto{\pgfqpoint{2.312775in}{3.801519in}}%
\pgfpathlineto{\pgfqpoint{2.362702in}{3.838957in}}%
\pgfpathlineto{\pgfqpoint{2.423815in}{3.880297in}}%
\pgfpathlineto{\pgfqpoint{2.480996in}{3.914781in}}%
\pgfpathlineto{\pgfqpoint{2.540142in}{3.946416in}}%
\pgfpathlineto{\pgfqpoint{2.599289in}{3.974072in}}%
\pgfpathlineto{\pgfqpoint{2.658436in}{3.997801in}}%
\pgfpathlineto{\pgfqpoint{2.717582in}{4.017630in}}%
\pgfpathlineto{\pgfqpoint{2.776729in}{4.033485in}}%
\pgfpathlineto{\pgfqpoint{2.835876in}{4.045247in}}%
\pgfpathlineto{\pgfqpoint{2.895023in}{4.052781in}}%
\pgfpathlineto{\pgfqpoint{2.924596in}{4.054900in}}%
\pgfpathlineto{\pgfqpoint{2.954169in}{4.055868in}}%
\pgfpathlineto{\pgfqpoint{2.983743in}{4.055642in}}%
\pgfpathlineto{\pgfqpoint{3.013316in}{4.054178in}}%
\pgfpathlineto{\pgfqpoint{3.042890in}{4.051426in}}%
\pgfpathlineto{\pgfqpoint{3.072463in}{4.047337in}}%
\pgfpathlineto{\pgfqpoint{3.119233in}{4.037854in}}%
\pgfpathlineto{\pgfqpoint{3.131610in}{4.034883in}}%
\pgfpathlineto{\pgfqpoint{3.161183in}{4.026318in}}%
\pgfpathlineto{\pgfqpoint{3.202287in}{4.011594in}}%
\pgfpathlineto{\pgfqpoint{3.220330in}{4.004236in}}%
\pgfpathlineto{\pgfqpoint{3.259804in}{3.985335in}}%
\pgfpathlineto{\pgfqpoint{3.279476in}{3.974786in}}%
\pgfpathlineto{\pgfqpoint{3.309050in}{3.957060in}}%
\pgfpathlineto{\pgfqpoint{3.344457in}{3.932816in}}%
\pgfpathlineto{\pgfqpoint{3.378316in}{3.906556in}}%
\pgfpathlineto{\pgfqpoint{3.408674in}{3.880297in}}%
\pgfpathlineto{\pgfqpoint{3.436309in}{3.854037in}}%
\pgfpathlineto{\pgfqpoint{3.461792in}{3.827778in}}%
\pgfpathlineto{\pgfqpoint{3.486490in}{3.800410in}}%
\pgfpathlineto{\pgfqpoint{3.528436in}{3.749000in}}%
\pgfpathlineto{\pgfqpoint{3.567260in}{3.696481in}}%
\pgfpathlineto{\pgfqpoint{3.585399in}{3.670221in}}%
\pgfpathlineto{\pgfqpoint{3.619737in}{3.617702in}}%
\pgfpathlineto{\pgfqpoint{3.652007in}{3.565183in}}%
\pgfpathlineto{\pgfqpoint{3.693504in}{3.493790in}}%
\pgfpathlineto{\pgfqpoint{3.751765in}{3.386975in}}%
\pgfpathlineto{\pgfqpoint{3.751765in}{3.386975in}}%
\pgfusepath{stroke}%
\end{pgfscope}%
\begin{pgfscope}%
\pgfpathrectangle{\pgfqpoint{0.854460in}{0.571603in}}{\pgfqpoint{5.885100in}{5.225635in}}%
\pgfusepath{clip}%
\pgfsetbuttcap%
\pgfsetroundjoin%
\pgfsetlinewidth{1.505625pt}%
\definecolor{currentstroke}{rgb}{0.278826,0.175490,0.483397}%
\pgfsetstrokecolor{currentstroke}%
\pgfsetdash{}{0pt}%
\pgfpathmoveto{\pgfqpoint{3.929964in}{3.040749in}}%
\pgfpathlineto{\pgfqpoint{4.026504in}{2.856177in}}%
\pgfpathlineto{\pgfqpoint{4.098026in}{2.724880in}}%
\pgfpathlineto{\pgfqpoint{4.187928in}{2.567323in}}%
\pgfpathlineto{\pgfqpoint{4.266724in}{2.436026in}}%
\pgfpathlineto{\pgfqpoint{4.344118in}{2.313056in}}%
\pgfpathlineto{\pgfqpoint{4.383608in}{2.252210in}}%
\pgfpathlineto{\pgfqpoint{4.462411in}{2.135025in}}%
\pgfpathlineto{\pgfqpoint{4.527177in}{2.042134in}}%
\pgfpathlineto{\pgfqpoint{4.610278in}{1.927384in}}%
\pgfpathlineto{\pgfqpoint{4.681731in}{1.832058in}}%
\pgfpathlineto{\pgfqpoint{4.763219in}{1.727020in}}%
\pgfpathlineto{\pgfqpoint{4.847472in}{1.621982in}}%
\pgfpathlineto{\pgfqpoint{4.935585in}{1.515556in}}%
\pgfpathlineto{\pgfqpoint{5.024305in}{1.411586in}}%
\pgfpathlineto{\pgfqpoint{5.116201in}{1.306869in}}%
\pgfpathlineto{\pgfqpoint{5.231319in}{1.179522in}}%
\pgfpathlineto{\pgfqpoint{5.332531in}{1.070533in}}%
\pgfpathlineto{\pgfqpoint{5.467906in}{0.928601in}}%
\pgfpathlineto{\pgfqpoint{5.586199in}{0.807544in}}%
\pgfpathlineto{\pgfqpoint{5.742861in}{0.650382in}}%
\pgfpathlineto{\pgfqpoint{5.822537in}{0.571603in}}%
\pgfpathlineto{\pgfqpoint{5.822537in}{0.571603in}}%
\pgfusepath{stroke}%
\end{pgfscope}%
\begin{pgfscope}%
\pgfpathrectangle{\pgfqpoint{0.854460in}{0.571603in}}{\pgfqpoint{5.885100in}{5.225635in}}%
\pgfusepath{clip}%
\pgfsetbuttcap%
\pgfsetroundjoin%
\pgfsetlinewidth{1.505625pt}%
\definecolor{currentstroke}{rgb}{0.273006,0.204520,0.501721}%
\pgfsetstrokecolor{currentstroke}%
\pgfsetdash{}{0pt}%
\pgfpathmoveto{\pgfqpoint{2.648461in}{0.571603in}}%
\pgfpathlineto{\pgfqpoint{2.569716in}{0.632750in}}%
\pgfpathlineto{\pgfqpoint{2.510569in}{0.680656in}}%
\pgfpathlineto{\pgfqpoint{2.451422in}{0.730504in}}%
\pgfpathlineto{\pgfqpoint{2.392275in}{0.782508in}}%
\pgfpathlineto{\pgfqpoint{2.333129in}{0.836903in}}%
\pgfpathlineto{\pgfqpoint{2.273982in}{0.893944in}}%
\pgfpathlineto{\pgfqpoint{2.214835in}{0.953909in}}%
\pgfpathlineto{\pgfqpoint{2.154856in}{1.018014in}}%
\pgfpathlineto{\pgfqpoint{2.096542in}{1.084327in}}%
\pgfpathlineto{\pgfqpoint{2.042639in}{1.149312in}}%
\pgfpathlineto{\pgfqpoint{2.001497in}{1.201831in}}%
\pgfpathlineto{\pgfqpoint{1.948675in}{1.273505in}}%
\pgfpathlineto{\pgfqpoint{1.907566in}{1.333128in}}%
\pgfpathlineto{\pgfqpoint{1.873426in}{1.385647in}}%
\pgfpathlineto{\pgfqpoint{1.830381in}{1.456433in}}%
\pgfpathlineto{\pgfqpoint{1.796225in}{1.516944in}}%
\pgfpathlineto{\pgfqpoint{1.768513in}{1.569463in}}%
\pgfpathlineto{\pgfqpoint{1.730293in}{1.648242in}}%
\pgfpathlineto{\pgfqpoint{1.706880in}{1.700761in}}%
\pgfpathlineto{\pgfqpoint{1.674934in}{1.779539in}}%
\pgfpathlineto{\pgfqpoint{1.646612in}{1.858318in}}%
\pgfpathlineto{\pgfqpoint{1.621809in}{1.937096in}}%
\pgfpathlineto{\pgfqpoint{1.600559in}{2.015874in}}%
\pgfpathlineto{\pgfqpoint{1.588247in}{2.068393in}}%
\pgfpathlineto{\pgfqpoint{1.572598in}{2.147172in}}%
\pgfpathlineto{\pgfqpoint{1.563922in}{2.199691in}}%
\pgfpathlineto{\pgfqpoint{1.553806in}{2.278469in}}%
\pgfpathlineto{\pgfqpoint{1.546851in}{2.357248in}}%
\pgfpathlineto{\pgfqpoint{1.543083in}{2.436026in}}%
\pgfpathlineto{\pgfqpoint{1.542519in}{2.514804in}}%
\pgfpathlineto{\pgfqpoint{1.545165in}{2.593583in}}%
\pgfpathlineto{\pgfqpoint{1.551013in}{2.672361in}}%
\pgfpathlineto{\pgfqpoint{1.560041in}{2.751140in}}%
\pgfpathlineto{\pgfqpoint{1.567906in}{2.803659in}}%
\pgfpathlineto{\pgfqpoint{1.582535in}{2.882437in}}%
\pgfpathlineto{\pgfqpoint{1.594046in}{2.934956in}}%
\pgfpathlineto{\pgfqpoint{1.614425in}{3.013734in}}%
\pgfpathlineto{\pgfqpoint{1.629920in}{3.066253in}}%
\pgfpathlineto{\pgfqpoint{1.656227in}{3.145032in}}%
\pgfpathlineto{\pgfqpoint{1.686409in}{3.223810in}}%
\pgfpathlineto{\pgfqpoint{1.712088in}{3.283837in}}%
\pgfpathlineto{\pgfqpoint{1.745810in}{3.355107in}}%
\pgfpathlineto{\pgfqpoint{1.772956in}{3.407626in}}%
\pgfpathlineto{\pgfqpoint{1.802225in}{3.460145in}}%
\pgfpathlineto{\pgfqpoint{1.833741in}{3.512664in}}%
\pgfpathlineto{\pgfqpoint{1.867632in}{3.565183in}}%
\pgfpathlineto{\pgfqpoint{1.904024in}{3.617702in}}%
\pgfpathlineto{\pgfqpoint{1.943046in}{3.670221in}}%
\pgfpathlineto{\pgfqpoint{1.978248in}{3.714528in}}%
\pgfpathlineto{\pgfqpoint{2.007822in}{3.749758in}}%
\pgfpathlineto{\pgfqpoint{2.054260in}{3.801519in}}%
\pgfpathlineto{\pgfqpoint{2.096542in}{3.845658in}}%
\pgfpathlineto{\pgfqpoint{2.131872in}{3.880297in}}%
\pgfpathlineto{\pgfqpoint{2.189212in}{3.932816in}}%
\pgfpathlineto{\pgfqpoint{2.244409in}{3.979321in}}%
\pgfpathlineto{\pgfqpoint{2.303555in}{4.025087in}}%
\pgfpathlineto{\pgfqpoint{2.362702in}{4.067050in}}%
\pgfpathlineto{\pgfqpoint{2.421849in}{4.105319in}}%
\pgfpathlineto{\pgfqpoint{2.485828in}{4.142892in}}%
\pgfpathlineto{\pgfqpoint{2.540142in}{4.171814in}}%
\pgfpathlineto{\pgfqpoint{2.599289in}{4.200178in}}%
\pgfpathlineto{\pgfqpoint{2.658436in}{4.225418in}}%
\pgfpathlineto{\pgfqpoint{2.718717in}{4.247930in}}%
\pgfpathlineto{\pgfqpoint{2.776729in}{4.266456in}}%
\pgfpathlineto{\pgfqpoint{2.835876in}{4.282206in}}%
\pgfpathlineto{\pgfqpoint{2.895023in}{4.294666in}}%
\pgfpathlineto{\pgfqpoint{2.954169in}{4.303684in}}%
\pgfpathlineto{\pgfqpoint{3.013316in}{4.309014in}}%
\pgfpathlineto{\pgfqpoint{3.072463in}{4.310471in}}%
\pgfpathlineto{\pgfqpoint{3.131610in}{4.307713in}}%
\pgfpathlineto{\pgfqpoint{3.161183in}{4.304633in}}%
\pgfpathlineto{\pgfqpoint{3.190756in}{4.300348in}}%
\pgfpathlineto{\pgfqpoint{3.220330in}{4.294694in}}%
\pgfpathlineto{\pgfqpoint{3.249903in}{4.287692in}}%
\pgfpathlineto{\pgfqpoint{3.294791in}{4.274189in}}%
\pgfpathlineto{\pgfqpoint{3.309050in}{4.269277in}}%
\pgfpathlineto{\pgfqpoint{3.360115in}{4.247930in}}%
\pgfpathlineto{\pgfqpoint{3.368197in}{4.244161in}}%
\pgfpathlineto{\pgfqpoint{3.410022in}{4.221670in}}%
\pgfpathlineto{\pgfqpoint{3.427343in}{4.211280in}}%
\pgfpathlineto{\pgfqpoint{3.456917in}{4.191586in}}%
\pgfpathlineto{\pgfqpoint{3.486860in}{4.169151in}}%
\pgfpathlineto{\pgfqpoint{3.517996in}{4.142892in}}%
\pgfpathlineto{\pgfqpoint{3.545997in}{4.116632in}}%
\pgfpathlineto{\pgfqpoint{3.575210in}{4.086234in}}%
\pgfpathlineto{\pgfqpoint{3.616258in}{4.037854in}}%
\pgfpathlineto{\pgfqpoint{3.636564in}{4.011594in}}%
\pgfpathlineto{\pgfqpoint{3.673376in}{3.959075in}}%
\pgfpathlineto{\pgfqpoint{3.706598in}{3.906556in}}%
\pgfpathlineto{\pgfqpoint{3.737052in}{3.854037in}}%
\pgfpathlineto{\pgfqpoint{3.765369in}{3.801519in}}%
\pgfpathlineto{\pgfqpoint{3.804883in}{3.722740in}}%
\pgfpathlineto{\pgfqpoint{3.829709in}{3.670221in}}%
\pgfpathlineto{\pgfqpoint{3.870944in}{3.579318in}}%
\pgfpathlineto{\pgfqpoint{3.967515in}{3.355107in}}%
\pgfpathlineto{\pgfqpoint{4.048384in}{3.168100in}}%
\pgfpathlineto{\pgfqpoint{4.093758in}{3.066253in}}%
\pgfpathlineto{\pgfqpoint{4.154695in}{2.934956in}}%
\pgfpathlineto{\pgfqpoint{4.218835in}{2.803659in}}%
\pgfpathlineto{\pgfqpoint{4.286657in}{2.672361in}}%
\pgfpathlineto{\pgfqpoint{4.344118in}{2.566727in}}%
\pgfpathlineto{\pgfqpoint{4.388328in}{2.488545in}}%
\pgfpathlineto{\pgfqpoint{4.450274in}{2.383507in}}%
\pgfpathlineto{\pgfqpoint{4.521558in}{2.268405in}}%
\pgfpathlineto{\pgfqpoint{4.565681in}{2.199691in}}%
\pgfpathlineto{\pgfqpoint{4.639851in}{2.088720in}}%
\pgfpathlineto{\pgfqpoint{4.698998in}{2.003576in}}%
\pgfpathlineto{\pgfqpoint{4.758145in}{1.921168in}}%
\pgfpathlineto{\pgfqpoint{4.817291in}{1.841256in}}%
\pgfpathlineto{\pgfqpoint{4.884393in}{1.753280in}}%
\pgfpathlineto{\pgfqpoint{4.967374in}{1.648242in}}%
\pgfpathlineto{\pgfqpoint{5.053878in}{1.542541in}}%
\pgfpathlineto{\pgfqpoint{5.133820in}{1.447933in}}%
\pgfpathlineto{\pgfqpoint{5.133820in}{1.447933in}}%
\pgfusepath{stroke}%
\end{pgfscope}%
\begin{pgfscope}%
\pgfpathrectangle{\pgfqpoint{0.854460in}{0.571603in}}{\pgfqpoint{5.885100in}{5.225635in}}%
\pgfusepath{clip}%
\pgfsetbuttcap%
\pgfsetroundjoin%
\pgfsetlinewidth{1.505625pt}%
\definecolor{currentstroke}{rgb}{0.273006,0.204520,0.501721}%
\pgfsetstrokecolor{currentstroke}%
\pgfsetdash{}{0pt}%
\pgfpathmoveto{\pgfqpoint{5.390741in}{1.160502in}}%
\pgfpathlineto{\pgfqpoint{5.401122in}{1.149312in}}%
\pgfpathlineto{\pgfqpoint{5.408759in}{1.141148in}}%
\pgfpathlineto{\pgfqpoint{5.425664in}{1.123052in}}%
\pgfpathlineto{\pgfqpoint{5.438332in}{1.109600in}}%
\pgfpathlineto{\pgfqpoint{5.450381in}{1.096793in}}%
\pgfpathlineto{\pgfqpoint{5.467906in}{1.078306in}}%
\pgfpathlineto{\pgfqpoint{5.475268in}{1.070533in}}%
\pgfpathlineto{\pgfqpoint{5.497479in}{1.047256in}}%
\pgfpathlineto{\pgfqpoint{5.500323in}{1.044274in}}%
\pgfpathlineto{\pgfqpoint{5.525518in}{1.018014in}}%
\pgfpathlineto{\pgfqpoint{5.527052in}{1.016423in}}%
\pgfpathlineto{\pgfqpoint{5.550831in}{0.991755in}}%
\pgfpathlineto{\pgfqpoint{5.556626in}{0.985781in}}%
\pgfpathlineto{\pgfqpoint{5.576303in}{0.965495in}}%
\pgfpathlineto{\pgfqpoint{5.586199in}{0.955355in}}%
\pgfpathlineto{\pgfqpoint{5.601932in}{0.939236in}}%
\pgfpathlineto{\pgfqpoint{5.615772in}{0.925137in}}%
\pgfpathlineto{\pgfqpoint{5.627714in}{0.912976in}}%
\pgfpathlineto{\pgfqpoint{5.645346in}{0.895118in}}%
\pgfpathlineto{\pgfqpoint{5.653645in}{0.886717in}}%
\pgfpathlineto{\pgfqpoint{5.674919in}{0.865290in}}%
\pgfpathlineto{\pgfqpoint{5.679722in}{0.860458in}}%
\pgfpathlineto{\pgfqpoint{5.704492in}{0.835646in}}%
\pgfpathlineto{\pgfqpoint{5.705939in}{0.834198in}}%
\pgfpathlineto{\pgfqpoint{5.732266in}{0.807939in}}%
\pgfpathlineto{\pgfqpoint{5.734066in}{0.806149in}}%
\pgfpathlineto{\pgfqpoint{5.758706in}{0.781679in}}%
\pgfpathlineto{\pgfqpoint{5.763639in}{0.776797in}}%
\pgfpathlineto{\pgfqpoint{5.785277in}{0.755420in}}%
\pgfpathlineto{\pgfqpoint{5.793213in}{0.747604in}}%
\pgfpathlineto{\pgfqpoint{5.811976in}{0.729160in}}%
\pgfpathlineto{\pgfqpoint{5.822786in}{0.718563in}}%
\pgfpathlineto{\pgfqpoint{5.838798in}{0.702901in}}%
\pgfpathlineto{\pgfqpoint{5.852359in}{0.689667in}}%
\pgfpathlineto{\pgfqpoint{5.865739in}{0.676641in}}%
\pgfpathlineto{\pgfqpoint{5.881933in}{0.660907in}}%
\pgfpathlineto{\pgfqpoint{5.892795in}{0.650382in}}%
\pgfpathlineto{\pgfqpoint{5.911506in}{0.632278in}}%
\pgfpathlineto{\pgfqpoint{5.919959in}{0.624122in}}%
\pgfpathlineto{\pgfqpoint{5.941079in}{0.603770in}}%
\pgfpathlineto{\pgfqpoint{5.947229in}{0.597863in}}%
\pgfpathlineto{\pgfqpoint{5.970653in}{0.575378in}}%
\pgfpathlineto{\pgfqpoint{5.974598in}{0.571603in}}%
\pgfusepath{stroke}%
\end{pgfscope}%
\begin{pgfscope}%
\pgfpathrectangle{\pgfqpoint{0.854460in}{0.571603in}}{\pgfqpoint{5.885100in}{5.225635in}}%
\pgfusepath{clip}%
\pgfsetbuttcap%
\pgfsetroundjoin%
\pgfsetlinewidth{1.505625pt}%
\definecolor{currentstroke}{rgb}{0.266580,0.228262,0.514349}%
\pgfsetstrokecolor{currentstroke}%
\pgfsetdash{}{0pt}%
\pgfpathmoveto{\pgfqpoint{2.523741in}{0.571603in}}%
\pgfpathlineto{\pgfqpoint{2.451422in}{0.628877in}}%
\pgfpathlineto{\pgfqpoint{2.392275in}{0.677661in}}%
\pgfpathlineto{\pgfqpoint{2.332323in}{0.729160in}}%
\pgfpathlineto{\pgfqpoint{2.273700in}{0.781679in}}%
\pgfpathlineto{\pgfqpoint{2.214835in}{0.836828in}}%
\pgfpathlineto{\pgfqpoint{2.155689in}{0.894930in}}%
\pgfpathlineto{\pgfqpoint{2.096542in}{0.955995in}}%
\pgfpathlineto{\pgfqpoint{2.039514in}{1.018014in}}%
\pgfpathlineto{\pgfqpoint{1.993724in}{1.070533in}}%
\pgfpathlineto{\pgfqpoint{1.948675in}{1.124630in}}%
\pgfpathlineto{\pgfqpoint{1.888250in}{1.201831in}}%
\pgfpathlineto{\pgfqpoint{1.830381in}{1.281560in}}%
\pgfpathlineto{\pgfqpoint{1.778215in}{1.359388in}}%
\pgfpathlineto{\pgfqpoint{1.741661in}{1.417885in}}%
\pgfpathlineto{\pgfqpoint{1.699324in}{1.490685in}}%
\pgfpathlineto{\pgfqpoint{1.670859in}{1.543204in}}%
\pgfpathlineto{\pgfqpoint{1.644100in}{1.595723in}}%
\pgfpathlineto{\pgfqpoint{1.619002in}{1.648242in}}%
\pgfpathlineto{\pgfqpoint{1.584491in}{1.727020in}}%
\pgfpathlineto{\pgfqpoint{1.553544in}{1.805799in}}%
\pgfpathlineto{\pgfqpoint{1.526073in}{1.884577in}}%
\pgfpathlineto{\pgfqpoint{1.501972in}{1.963355in}}%
\pgfpathlineto{\pgfqpoint{1.481244in}{2.042134in}}%
\pgfpathlineto{\pgfqpoint{1.469223in}{2.094653in}}%
\pgfpathlineto{\pgfqpoint{1.453860in}{2.173431in}}%
\pgfpathlineto{\pgfqpoint{1.441611in}{2.252210in}}%
\pgfpathlineto{\pgfqpoint{1.432530in}{2.330988in}}%
\pgfpathlineto{\pgfqpoint{1.426468in}{2.409766in}}%
\pgfpathlineto{\pgfqpoint{1.423441in}{2.488545in}}%
\pgfpathlineto{\pgfqpoint{1.423456in}{2.567323in}}%
\pgfpathlineto{\pgfqpoint{1.426511in}{2.646102in}}%
\pgfpathlineto{\pgfqpoint{1.432594in}{2.724880in}}%
\pgfpathlineto{\pgfqpoint{1.441680in}{2.803659in}}%
\pgfpathlineto{\pgfqpoint{1.449476in}{2.856177in}}%
\pgfpathlineto{\pgfqpoint{1.463840in}{2.934956in}}%
\pgfpathlineto{\pgfqpoint{1.481311in}{3.013734in}}%
\pgfpathlineto{\pgfqpoint{1.502055in}{3.092513in}}%
\pgfpathlineto{\pgfqpoint{1.517825in}{3.145032in}}%
\pgfpathlineto{\pgfqpoint{1.534977in}{3.197551in}}%
\pgfpathlineto{\pgfqpoint{1.564221in}{3.277397in}}%
\pgfpathlineto{\pgfqpoint{1.596357in}{3.355107in}}%
\pgfpathlineto{\pgfqpoint{1.632833in}{3.433886in}}%
\pgfpathlineto{\pgfqpoint{1.659367in}{3.486405in}}%
\pgfpathlineto{\pgfqpoint{1.687824in}{3.538924in}}%
\pgfpathlineto{\pgfqpoint{1.718303in}{3.591443in}}%
\pgfpathlineto{\pgfqpoint{1.750902in}{3.643962in}}%
\pgfpathlineto{\pgfqpoint{1.785723in}{3.696481in}}%
\pgfpathlineto{\pgfqpoint{1.830381in}{3.759265in}}%
\pgfpathlineto{\pgfqpoint{1.862502in}{3.801519in}}%
\pgfpathlineto{\pgfqpoint{1.904983in}{3.854037in}}%
\pgfpathlineto{\pgfqpoint{1.950213in}{3.906556in}}%
\pgfpathlineto{\pgfqpoint{2.007822in}{3.968486in}}%
\pgfpathlineto{\pgfqpoint{2.066968in}{4.027172in}}%
\pgfpathlineto{\pgfqpoint{2.126115in}{4.081422in}}%
\pgfpathlineto{\pgfqpoint{2.185262in}{4.131657in}}%
\pgfpathlineto{\pgfqpoint{2.244409in}{4.178244in}}%
\pgfpathlineto{\pgfqpoint{2.303790in}{4.221670in}}%
\pgfpathlineto{\pgfqpoint{2.362702in}{4.261490in}}%
\pgfpathlineto{\pgfqpoint{2.424939in}{4.300449in}}%
\pgfpathlineto{\pgfqpoint{2.480996in}{4.332823in}}%
\pgfpathlineto{\pgfqpoint{2.540142in}{4.364320in}}%
\pgfpathlineto{\pgfqpoint{2.599289in}{4.393192in}}%
\pgfpathlineto{\pgfqpoint{2.658436in}{4.419474in}}%
\pgfpathlineto{\pgfqpoint{2.717582in}{4.443195in}}%
\pgfpathlineto{\pgfqpoint{2.776729in}{4.464376in}}%
\pgfpathlineto{\pgfqpoint{2.840371in}{4.484265in}}%
\pgfpathlineto{\pgfqpoint{2.895023in}{4.498899in}}%
\pgfpathlineto{\pgfqpoint{2.954169in}{4.512204in}}%
\pgfpathlineto{\pgfqpoint{3.013316in}{4.522583in}}%
\pgfpathlineto{\pgfqpoint{3.072463in}{4.530069in}}%
\pgfpathlineto{\pgfqpoint{3.131610in}{4.534409in}}%
\pgfpathlineto{\pgfqpoint{3.190756in}{4.535316in}}%
\pgfpathlineto{\pgfqpoint{3.249903in}{4.532462in}}%
\pgfpathlineto{\pgfqpoint{3.309050in}{4.525470in}}%
\pgfpathlineto{\pgfqpoint{3.338623in}{4.520288in}}%
\pgfpathlineto{\pgfqpoint{3.381388in}{4.510524in}}%
\pgfpathlineto{\pgfqpoint{3.397770in}{4.506150in}}%
\pgfpathlineto{\pgfqpoint{3.427343in}{4.496950in}}%
\pgfpathlineto{\pgfqpoint{3.461941in}{4.484265in}}%
\pgfpathlineto{\pgfqpoint{3.486490in}{4.473886in}}%
\pgfpathlineto{\pgfqpoint{3.519392in}{4.458005in}}%
\pgfpathlineto{\pgfqpoint{3.565002in}{4.431746in}}%
\pgfpathlineto{\pgfqpoint{3.575210in}{4.425246in}}%
\pgfpathlineto{\pgfqpoint{3.604783in}{4.404572in}}%
\pgfpathlineto{\pgfqpoint{3.636667in}{4.379227in}}%
\pgfpathlineto{\pgfqpoint{3.665953in}{4.352967in}}%
\pgfpathlineto{\pgfqpoint{3.693504in}{4.325360in}}%
\pgfpathlineto{\pgfqpoint{3.723077in}{4.292142in}}%
\pgfpathlineto{\pgfqpoint{3.757812in}{4.247930in}}%
\pgfpathlineto{\pgfqpoint{3.793851in}{4.195411in}}%
\pgfpathlineto{\pgfqpoint{3.825657in}{4.142892in}}%
\pgfpathlineto{\pgfqpoint{3.854213in}{4.090373in}}%
\pgfpathlineto{\pgfqpoint{3.880260in}{4.037854in}}%
\pgfpathlineto{\pgfqpoint{3.904370in}{3.985335in}}%
\pgfpathlineto{\pgfqpoint{3.937655in}{3.906556in}}%
\pgfpathlineto{\pgfqpoint{3.968576in}{3.827778in}}%
\pgfpathlineto{\pgfqpoint{4.016975in}{3.696481in}}%
\pgfpathlineto{\pgfqpoint{4.054229in}{3.591443in}}%
\pgfpathlineto{\pgfqpoint{4.072686in}{3.539146in}}%
\pgfpathlineto{\pgfqpoint{4.072686in}{3.539146in}}%
\pgfusepath{stroke}%
\end{pgfscope}%
\begin{pgfscope}%
\pgfpathrectangle{\pgfqpoint{0.854460in}{0.571603in}}{\pgfqpoint{5.885100in}{5.225635in}}%
\pgfusepath{clip}%
\pgfsetbuttcap%
\pgfsetroundjoin%
\pgfsetlinewidth{1.505625pt}%
\definecolor{currentstroke}{rgb}{0.266580,0.228262,0.514349}%
\pgfsetstrokecolor{currentstroke}%
\pgfsetdash{}{0pt}%
\pgfpathmoveto{\pgfqpoint{4.207778in}{3.172350in}}%
\pgfpathlineto{\pgfqpoint{4.239728in}{3.092513in}}%
\pgfpathlineto{\pgfqpoint{4.284971in}{2.984687in}}%
\pgfpathlineto{\pgfqpoint{4.330071in}{2.882437in}}%
\pgfpathlineto{\pgfqpoint{4.379087in}{2.777399in}}%
\pgfpathlineto{\pgfqpoint{4.430876in}{2.672361in}}%
\pgfpathlineto{\pgfqpoint{4.491984in}{2.555403in}}%
\pgfpathlineto{\pgfqpoint{4.528506in}{2.488545in}}%
\pgfpathlineto{\pgfqpoint{4.588514in}{2.383507in}}%
\pgfpathlineto{\pgfqpoint{4.651661in}{2.278469in}}%
\pgfpathlineto{\pgfqpoint{4.718018in}{2.173431in}}%
\pgfpathlineto{\pgfqpoint{4.787718in}{2.068300in}}%
\pgfpathlineto{\pgfqpoint{4.846865in}{1.982657in}}%
\pgfpathlineto{\pgfqpoint{4.906012in}{1.899973in}}%
\pgfpathlineto{\pgfqpoint{4.965158in}{1.819963in}}%
\pgfpathlineto{\pgfqpoint{5.036166in}{1.727020in}}%
\pgfpathlineto{\pgfqpoint{5.119525in}{1.621982in}}%
\pgfpathlineto{\pgfqpoint{5.206055in}{1.516944in}}%
\pgfpathlineto{\pgfqpoint{5.295698in}{1.411906in}}%
\pgfpathlineto{\pgfqpoint{5.388396in}{1.306869in}}%
\pgfpathlineto{\pgfqpoint{5.497479in}{1.187427in}}%
\pgfpathlineto{\pgfqpoint{5.586199in}{1.093162in}}%
\pgfpathlineto{\pgfqpoint{5.684106in}{0.991755in}}%
\pgfpathlineto{\pgfqpoint{5.793213in}{0.881718in}}%
\pgfpathlineto{\pgfqpoint{5.921782in}{0.755420in}}%
\pgfpathlineto{\pgfqpoint{6.059373in}{0.623802in}}%
\pgfpathlineto{\pgfqpoint{6.114810in}{0.571603in}}%
\pgfpathlineto{\pgfqpoint{6.114810in}{0.571603in}}%
\pgfusepath{stroke}%
\end{pgfscope}%
\begin{pgfscope}%
\pgfpathrectangle{\pgfqpoint{0.854460in}{0.571603in}}{\pgfqpoint{5.885100in}{5.225635in}}%
\pgfusepath{clip}%
\pgfsetbuttcap%
\pgfsetroundjoin%
\pgfsetlinewidth{1.505625pt}%
\definecolor{currentstroke}{rgb}{0.257322,0.256130,0.526563}%
\pgfsetstrokecolor{currentstroke}%
\pgfsetdash{}{0pt}%
\pgfpathmoveto{\pgfqpoint{2.408459in}{0.571603in}}%
\pgfpathlineto{\pgfqpoint{2.333129in}{0.632527in}}%
\pgfpathlineto{\pgfqpoint{2.273982in}{0.682415in}}%
\pgfpathlineto{\pgfqpoint{2.214835in}{0.734351in}}%
\pgfpathlineto{\pgfqpoint{2.155689in}{0.788548in}}%
\pgfpathlineto{\pgfqpoint{2.096542in}{0.845235in}}%
\pgfpathlineto{\pgfqpoint{2.029326in}{0.912976in}}%
\pgfpathlineto{\pgfqpoint{1.978248in}{0.967162in}}%
\pgfpathlineto{\pgfqpoint{1.919102in}{1.033349in}}%
\pgfpathlineto{\pgfqpoint{1.865440in}{1.096793in}}%
\pgfpathlineto{\pgfqpoint{1.823323in}{1.149312in}}%
\pgfpathlineto{\pgfqpoint{1.771235in}{1.218021in}}%
\pgfpathlineto{\pgfqpoint{1.726633in}{1.280609in}}%
\pgfpathlineto{\pgfqpoint{1.682515in}{1.346560in}}%
\pgfpathlineto{\pgfqpoint{1.641616in}{1.411906in}}%
\pgfpathlineto{\pgfqpoint{1.610735in}{1.464425in}}%
\pgfpathlineto{\pgfqpoint{1.581573in}{1.516944in}}%
\pgfpathlineto{\pgfqpoint{1.554090in}{1.569463in}}%
\pgfpathlineto{\pgfqpoint{1.528242in}{1.621982in}}%
\pgfpathlineto{\pgfqpoint{1.492531in}{1.700761in}}%
\pgfpathlineto{\pgfqpoint{1.470637in}{1.753280in}}%
\pgfpathlineto{\pgfqpoint{1.440711in}{1.832058in}}%
\pgfpathlineto{\pgfqpoint{1.414113in}{1.910836in}}%
\pgfpathlineto{\pgfqpoint{1.390820in}{1.989615in}}%
\pgfpathlineto{\pgfqpoint{1.377088in}{2.042134in}}%
\pgfpathlineto{\pgfqpoint{1.359004in}{2.120912in}}%
\pgfpathlineto{\pgfqpoint{1.348741in}{2.173431in}}%
\pgfpathlineto{\pgfqpoint{1.335805in}{2.252210in}}%
\pgfpathlineto{\pgfqpoint{1.325816in}{2.330988in}}%
\pgfpathlineto{\pgfqpoint{1.318875in}{2.409766in}}%
\pgfpathlineto{\pgfqpoint{1.314819in}{2.488545in}}%
\pgfpathlineto{\pgfqpoint{1.313656in}{2.567323in}}%
\pgfpathlineto{\pgfqpoint{1.315382in}{2.646102in}}%
\pgfpathlineto{\pgfqpoint{1.319987in}{2.724880in}}%
\pgfpathlineto{\pgfqpoint{1.327634in}{2.805204in}}%
\pgfpathlineto{\pgfqpoint{1.337984in}{2.882437in}}%
\pgfpathlineto{\pgfqpoint{1.351383in}{2.961215in}}%
\pgfpathlineto{\pgfqpoint{1.361996in}{3.013734in}}%
\pgfpathlineto{\pgfqpoint{1.380493in}{3.092513in}}%
\pgfpathlineto{\pgfqpoint{1.394562in}{3.145032in}}%
\pgfpathlineto{\pgfqpoint{1.418274in}{3.223810in}}%
\pgfpathlineto{\pgfqpoint{1.445928in}{3.304244in}}%
\pgfpathlineto{\pgfqpoint{1.475808in}{3.381367in}}%
\pgfpathlineto{\pgfqpoint{1.509930in}{3.460145in}}%
\pgfpathlineto{\pgfqpoint{1.547812in}{3.538924in}}%
\pgfpathlineto{\pgfqpoint{1.575228in}{3.591443in}}%
\pgfpathlineto{\pgfqpoint{1.604490in}{3.643962in}}%
\pgfpathlineto{\pgfqpoint{1.635680in}{3.696481in}}%
\pgfpathlineto{\pgfqpoint{1.668884in}{3.749000in}}%
\pgfpathlineto{\pgfqpoint{1.712088in}{3.812886in}}%
\pgfpathlineto{\pgfqpoint{1.761415in}{3.880297in}}%
\pgfpathlineto{\pgfqpoint{1.802519in}{3.932816in}}%
\pgfpathlineto{\pgfqpoint{1.859955in}{4.001016in}}%
\pgfpathlineto{\pgfqpoint{1.892930in}{4.037854in}}%
\pgfpathlineto{\pgfqpoint{1.948675in}{4.096609in}}%
\pgfpathlineto{\pgfqpoint{2.007822in}{4.154608in}}%
\pgfpathlineto{\pgfqpoint{2.066968in}{4.208636in}}%
\pgfpathlineto{\pgfqpoint{2.126115in}{4.259062in}}%
\pgfpathlineto{\pgfqpoint{2.185262in}{4.306209in}}%
\pgfpathlineto{\pgfqpoint{2.248121in}{4.352967in}}%
\pgfpathlineto{\pgfqpoint{2.303555in}{4.391524in}}%
\pgfpathlineto{\pgfqpoint{2.365231in}{4.431746in}}%
\pgfpathlineto{\pgfqpoint{2.421849in}{4.466193in}}%
\pgfpathlineto{\pgfqpoint{2.480996in}{4.499852in}}%
\pgfpathlineto{\pgfqpoint{2.551170in}{4.536784in}}%
\pgfpathlineto{\pgfqpoint{2.605036in}{4.563043in}}%
\pgfpathlineto{\pgfqpoint{2.663107in}{4.589303in}}%
\pgfpathlineto{\pgfqpoint{2.726489in}{4.615562in}}%
\pgfpathlineto{\pgfqpoint{2.796708in}{4.641822in}}%
\pgfpathlineto{\pgfqpoint{2.835876in}{4.655169in}}%
\pgfpathlineto{\pgfqpoint{2.895023in}{4.673550in}}%
\pgfpathlineto{\pgfqpoint{2.954169in}{4.689723in}}%
\pgfpathlineto{\pgfqpoint{3.013316in}{4.703618in}}%
\pgfpathlineto{\pgfqpoint{3.072463in}{4.715230in}}%
\pgfpathlineto{\pgfqpoint{3.131610in}{4.724395in}}%
\pgfpathlineto{\pgfqpoint{3.190756in}{4.730932in}}%
\pgfpathlineto{\pgfqpoint{3.249903in}{4.734743in}}%
\pgfpathlineto{\pgfqpoint{3.309050in}{4.735570in}}%
\pgfpathlineto{\pgfqpoint{3.368197in}{4.733118in}}%
\pgfpathlineto{\pgfqpoint{3.427343in}{4.727045in}}%
\pgfpathlineto{\pgfqpoint{3.467291in}{4.720600in}}%
\pgfpathlineto{\pgfqpoint{3.516063in}{4.710000in}}%
\pgfpathlineto{\pgfqpoint{3.569663in}{4.694341in}}%
\pgfpathlineto{\pgfqpoint{3.575210in}{4.692507in}}%
\pgfpathlineto{\pgfqpoint{3.604783in}{4.681493in}}%
\pgfpathlineto{\pgfqpoint{3.636289in}{4.668081in}}%
\pgfpathlineto{\pgfqpoint{3.663930in}{4.654537in}}%
\pgfpathlineto{\pgfqpoint{3.693504in}{4.638227in}}%
\pgfpathlineto{\pgfqpoint{3.729076in}{4.615562in}}%
\pgfpathlineto{\pgfqpoint{3.764577in}{4.589303in}}%
\pgfpathlineto{\pgfqpoint{3.795522in}{4.563043in}}%
\pgfpathlineto{\pgfqpoint{3.822915in}{4.536784in}}%
\pgfpathlineto{\pgfqpoint{3.847496in}{4.510524in}}%
\pgfpathlineto{\pgfqpoint{3.870944in}{4.482823in}}%
\pgfpathlineto{\pgfqpoint{3.900517in}{4.443433in}}%
\pgfpathlineto{\pgfqpoint{3.930090in}{4.398676in}}%
\pgfpathlineto{\pgfqpoint{3.959664in}{4.347642in}}%
\pgfpathlineto{\pgfqpoint{3.989237in}{4.289426in}}%
\pgfpathlineto{\pgfqpoint{4.018811in}{4.223240in}}%
\pgfpathlineto{\pgfqpoint{4.040435in}{4.169151in}}%
\pgfpathlineto{\pgfqpoint{4.059863in}{4.116632in}}%
\pgfpathlineto{\pgfqpoint{4.086762in}{4.037854in}}%
\pgfpathlineto{\pgfqpoint{4.119682in}{3.932816in}}%
\pgfpathlineto{\pgfqpoint{4.166677in}{3.772576in}}%
\pgfpathlineto{\pgfqpoint{4.226620in}{3.565183in}}%
\pgfpathlineto{\pgfqpoint{4.266281in}{3.433886in}}%
\pgfpathlineto{\pgfqpoint{4.308516in}{3.302589in}}%
\pgfpathlineto{\pgfqpoint{4.353909in}{3.171291in}}%
\pgfpathlineto{\pgfqpoint{4.392902in}{3.066253in}}%
\pgfpathlineto{\pgfqpoint{4.434557in}{2.961215in}}%
\pgfpathlineto{\pgfqpoint{4.467572in}{2.882437in}}%
\pgfpathlineto{\pgfqpoint{4.514173in}{2.777399in}}%
\pgfpathlineto{\pgfqpoint{4.563818in}{2.672361in}}%
\pgfpathlineto{\pgfqpoint{4.616645in}{2.567323in}}%
\pgfpathlineto{\pgfqpoint{4.669425in}{2.468340in}}%
\pgfpathlineto{\pgfqpoint{4.701965in}{2.409766in}}%
\pgfpathlineto{\pgfqpoint{4.758145in}{2.312910in}}%
\pgfpathlineto{\pgfqpoint{4.794731in}{2.252210in}}%
\pgfpathlineto{\pgfqpoint{4.860797in}{2.147172in}}%
\pgfpathlineto{\pgfqpoint{4.930307in}{2.042134in}}%
\pgfpathlineto{\pgfqpoint{5.003198in}{1.937096in}}%
\pgfpathlineto{\pgfqpoint{5.083452in}{1.826827in}}%
\pgfpathlineto{\pgfqpoint{5.142599in}{1.748630in}}%
\pgfpathlineto{\pgfqpoint{5.201745in}{1.672799in}}%
\pgfpathlineto{\pgfqpoint{5.285134in}{1.569463in}}%
\pgfpathlineto{\pgfqpoint{5.379185in}{1.457483in}}%
\pgfpathlineto{\pgfqpoint{5.464590in}{1.359388in}}%
\pgfpathlineto{\pgfqpoint{5.559161in}{1.254350in}}%
\pgfpathlineto{\pgfqpoint{5.656831in}{1.149312in}}%
\pgfpathlineto{\pgfqpoint{5.700715in}{1.103224in}}%
\pgfpathlineto{\pgfqpoint{5.700715in}{1.103224in}}%
\pgfusepath{stroke}%
\end{pgfscope}%
\begin{pgfscope}%
\pgfpathrectangle{\pgfqpoint{0.854460in}{0.571603in}}{\pgfqpoint{5.885100in}{5.225635in}}%
\pgfusepath{clip}%
\pgfsetbuttcap%
\pgfsetroundjoin%
\pgfsetlinewidth{1.505625pt}%
\definecolor{currentstroke}{rgb}{0.257322,0.256130,0.526563}%
\pgfsetstrokecolor{currentstroke}%
\pgfsetdash{}{0pt}%
\pgfpathmoveto{\pgfqpoint{5.971831in}{0.830327in}}%
\pgfpathlineto{\pgfqpoint{5.994893in}{0.807939in}}%
\pgfpathlineto{\pgfqpoint{6.000226in}{0.802787in}}%
\pgfpathlineto{\pgfqpoint{6.022110in}{0.781679in}}%
\pgfpathlineto{\pgfqpoint{6.029799in}{0.774296in}}%
\pgfpathlineto{\pgfqpoint{6.049494in}{0.755420in}}%
\pgfpathlineto{\pgfqpoint{6.059373in}{0.745991in}}%
\pgfpathlineto{\pgfqpoint{6.077041in}{0.729160in}}%
\pgfpathlineto{\pgfqpoint{6.088946in}{0.717864in}}%
\pgfpathlineto{\pgfqpoint{6.104749in}{0.702901in}}%
\pgfpathlineto{\pgfqpoint{6.118520in}{0.689908in}}%
\pgfpathlineto{\pgfqpoint{6.132614in}{0.676641in}}%
\pgfpathlineto{\pgfqpoint{6.148093in}{0.662119in}}%
\pgfpathlineto{\pgfqpoint{6.160634in}{0.650382in}}%
\pgfpathlineto{\pgfqpoint{6.177666in}{0.634488in}}%
\pgfpathlineto{\pgfqpoint{6.188805in}{0.624122in}}%
\pgfpathlineto{\pgfqpoint{6.207240in}{0.607011in}}%
\pgfpathlineto{\pgfqpoint{6.217123in}{0.597863in}}%
\pgfpathlineto{\pgfqpoint{6.236813in}{0.579680in}}%
\pgfpathlineto{\pgfqpoint{6.245586in}{0.571603in}}%
\pgfusepath{stroke}%
\end{pgfscope}%
\begin{pgfscope}%
\pgfpathrectangle{\pgfqpoint{0.854460in}{0.571603in}}{\pgfqpoint{5.885100in}{5.225635in}}%
\pgfusepath{clip}%
\pgfsetbuttcap%
\pgfsetroundjoin%
\pgfsetlinewidth{1.505625pt}%
\definecolor{currentstroke}{rgb}{0.248629,0.278775,0.534556}%
\pgfsetstrokecolor{currentstroke}%
\pgfsetdash{}{0pt}%
\pgfpathmoveto{\pgfqpoint{2.300862in}{0.571603in}}%
\pgfpathlineto{\pgfqpoint{2.214835in}{0.642685in}}%
\pgfpathlineto{\pgfqpoint{2.145448in}{0.702901in}}%
\pgfpathlineto{\pgfqpoint{2.066968in}{0.774661in}}%
\pgfpathlineto{\pgfqpoint{2.005096in}{0.834198in}}%
\pgfpathlineto{\pgfqpoint{1.948675in}{0.891202in}}%
\pgfpathlineto{\pgfqpoint{1.878991in}{0.965495in}}%
\pgfpathlineto{\pgfqpoint{1.830381in}{1.020181in}}%
\pgfpathlineto{\pgfqpoint{1.766099in}{1.096793in}}%
\pgfpathlineto{\pgfqpoint{1.712088in}{1.165492in}}%
\pgfpathlineto{\pgfqpoint{1.665705in}{1.228090in}}%
\pgfpathlineto{\pgfqpoint{1.623368in}{1.288646in}}%
\pgfpathlineto{\pgfqpoint{1.576976in}{1.359388in}}%
\pgfpathlineto{\pgfqpoint{1.534648in}{1.428631in}}%
\pgfpathlineto{\pgfqpoint{1.499224in}{1.490685in}}%
\pgfpathlineto{\pgfqpoint{1.457630in}{1.569463in}}%
\pgfpathlineto{\pgfqpoint{1.431903in}{1.621982in}}%
\pgfpathlineto{\pgfqpoint{1.407732in}{1.674501in}}%
\pgfpathlineto{\pgfqpoint{1.374381in}{1.753280in}}%
\pgfpathlineto{\pgfqpoint{1.344379in}{1.832058in}}%
\pgfpathlineto{\pgfqpoint{1.317635in}{1.910836in}}%
\pgfpathlineto{\pgfqpoint{1.294045in}{1.989615in}}%
\pgfpathlineto{\pgfqpoint{1.273593in}{2.068393in}}%
\pgfpathlineto{\pgfqpoint{1.261653in}{2.120912in}}%
\pgfpathlineto{\pgfqpoint{1.246213in}{2.199691in}}%
\pgfpathlineto{\pgfqpoint{1.233692in}{2.278469in}}%
\pgfpathlineto{\pgfqpoint{1.224100in}{2.357248in}}%
\pgfpathlineto{\pgfqpoint{1.217308in}{2.436026in}}%
\pgfpathlineto{\pgfqpoint{1.213324in}{2.514804in}}%
\pgfpathlineto{\pgfqpoint{1.212153in}{2.593583in}}%
\pgfpathlineto{\pgfqpoint{1.213789in}{2.672361in}}%
\pgfpathlineto{\pgfqpoint{1.218224in}{2.751140in}}%
\pgfpathlineto{\pgfqpoint{1.225437in}{2.829918in}}%
\pgfpathlineto{\pgfqpoint{1.235403in}{2.908696in}}%
\pgfpathlineto{\pgfqpoint{1.248296in}{2.987475in}}%
\pgfpathlineto{\pgfqpoint{1.264009in}{3.066253in}}%
\pgfpathlineto{\pgfqpoint{1.276157in}{3.118772in}}%
\pgfpathlineto{\pgfqpoint{1.298061in}{3.202016in}}%
\pgfpathlineto{\pgfqpoint{1.320592in}{3.276329in}}%
\pgfpathlineto{\pgfqpoint{1.338197in}{3.328848in}}%
\pgfpathlineto{\pgfqpoint{1.367289in}{3.407626in}}%
\pgfpathlineto{\pgfqpoint{1.399761in}{3.486405in}}%
\pgfpathlineto{\pgfqpoint{1.423327in}{3.538924in}}%
\pgfpathlineto{\pgfqpoint{1.461784in}{3.617702in}}%
\pgfpathlineto{\pgfqpoint{1.489523in}{3.670221in}}%
\pgfpathlineto{\pgfqpoint{1.519033in}{3.722740in}}%
\pgfpathlineto{\pgfqpoint{1.550387in}{3.775259in}}%
\pgfpathlineto{\pgfqpoint{1.593795in}{3.843232in}}%
\pgfpathlineto{\pgfqpoint{1.637408in}{3.906556in}}%
\pgfpathlineto{\pgfqpoint{1.682515in}{3.967815in}}%
\pgfpathlineto{\pgfqpoint{1.738024in}{4.037854in}}%
\pgfpathlineto{\pgfqpoint{1.782622in}{4.090373in}}%
\pgfpathlineto{\pgfqpoint{1.830381in}{4.143559in}}%
\pgfpathlineto{\pgfqpoint{1.889528in}{4.205113in}}%
\pgfpathlineto{\pgfqpoint{1.948675in}{4.262641in}}%
\pgfpathlineto{\pgfqpoint{2.007822in}{4.316514in}}%
\pgfpathlineto{\pgfqpoint{2.066968in}{4.367064in}}%
\pgfpathlineto{\pgfqpoint{2.126115in}{4.414587in}}%
\pgfpathlineto{\pgfqpoint{2.185262in}{4.459353in}}%
\pgfpathlineto{\pgfqpoint{2.257714in}{4.510524in}}%
\pgfpathlineto{\pgfqpoint{2.303555in}{4.541116in}}%
\pgfpathlineto{\pgfqpoint{2.380565in}{4.589303in}}%
\pgfpathlineto{\pgfqpoint{2.425097in}{4.615562in}}%
\pgfpathlineto{\pgfqpoint{2.480996in}{4.646787in}}%
\pgfpathlineto{\pgfqpoint{2.540142in}{4.677868in}}%
\pgfpathlineto{\pgfqpoint{2.599289in}{4.707027in}}%
\pgfpathlineto{\pgfqpoint{2.658436in}{4.734304in}}%
\pgfpathlineto{\pgfqpoint{2.717582in}{4.759729in}}%
\pgfpathlineto{\pgfqpoint{2.776729in}{4.783334in}}%
\pgfpathlineto{\pgfqpoint{2.835876in}{4.805144in}}%
\pgfpathlineto{\pgfqpoint{2.896511in}{4.825638in}}%
\pgfpathlineto{\pgfqpoint{2.954169in}{4.843258in}}%
\pgfpathlineto{\pgfqpoint{3.013316in}{4.859561in}}%
\pgfpathlineto{\pgfqpoint{3.091390in}{4.878157in}}%
\pgfpathlineto{\pgfqpoint{3.131610in}{4.886412in}}%
\pgfpathlineto{\pgfqpoint{3.190756in}{4.896831in}}%
\pgfpathlineto{\pgfqpoint{3.249903in}{4.905169in}}%
\pgfpathlineto{\pgfqpoint{3.309050in}{4.911129in}}%
\pgfpathlineto{\pgfqpoint{3.368197in}{4.914700in}}%
\pgfpathlineto{\pgfqpoint{3.427343in}{4.915654in}}%
\pgfpathlineto{\pgfqpoint{3.486490in}{4.913729in}}%
\pgfpathlineto{\pgfqpoint{3.545637in}{4.908626in}}%
\pgfpathlineto{\pgfqpoint{3.604783in}{4.899871in}}%
\pgfpathlineto{\pgfqpoint{3.663930in}{4.886945in}}%
\pgfpathlineto{\pgfqpoint{3.695608in}{4.878157in}}%
\pgfpathlineto{\pgfqpoint{3.723077in}{4.869214in}}%
\pgfpathlineto{\pgfqpoint{3.768080in}{4.851897in}}%
\pgfpathlineto{\pgfqpoint{3.782224in}{4.845744in}}%
\pgfpathlineto{\pgfqpoint{3.822770in}{4.825638in}}%
\pgfpathlineto{\pgfqpoint{3.841370in}{4.815217in}}%
\pgfpathlineto{\pgfqpoint{3.870944in}{4.796864in}}%
\pgfpathlineto{\pgfqpoint{3.904254in}{4.773119in}}%
\pgfpathlineto{\pgfqpoint{3.936265in}{4.746860in}}%
\pgfpathlineto{\pgfqpoint{3.964398in}{4.720600in}}%
\pgfpathlineto{\pgfqpoint{3.989449in}{4.694341in}}%
\pgfpathlineto{\pgfqpoint{4.018811in}{4.659283in}}%
\pgfpathlineto{\pgfqpoint{4.050572in}{4.615562in}}%
\pgfpathlineto{\pgfqpoint{4.083043in}{4.563043in}}%
\pgfpathlineto{\pgfqpoint{4.110873in}{4.510524in}}%
\pgfpathlineto{\pgfqpoint{4.137104in}{4.453284in}}%
\pgfpathlineto{\pgfqpoint{4.156342in}{4.405486in}}%
\pgfpathlineto{\pgfqpoint{4.175426in}{4.352967in}}%
\pgfpathlineto{\pgfqpoint{4.200808in}{4.274189in}}%
\pgfpathlineto{\pgfqpoint{4.225824in}{4.186073in}}%
\pgfpathlineto{\pgfqpoint{4.250214in}{4.090373in}}%
\pgfpathlineto{\pgfqpoint{4.270759in}{4.003979in}}%
\pgfpathlineto{\pgfqpoint{4.270759in}{4.003979in}}%
\pgfusepath{stroke}%
\end{pgfscope}%
\begin{pgfscope}%
\pgfpathrectangle{\pgfqpoint{0.854460in}{0.571603in}}{\pgfqpoint{5.885100in}{5.225635in}}%
\pgfusepath{clip}%
\pgfsetbuttcap%
\pgfsetroundjoin%
\pgfsetlinewidth{1.505625pt}%
\definecolor{currentstroke}{rgb}{0.248629,0.278775,0.534556}%
\pgfsetstrokecolor{currentstroke}%
\pgfsetdash{}{0pt}%
\pgfpathmoveto{\pgfqpoint{4.359434in}{3.622081in}}%
\pgfpathlineto{\pgfqpoint{4.387134in}{3.512664in}}%
\pgfpathlineto{\pgfqpoint{4.415603in}{3.407626in}}%
\pgfpathlineto{\pgfqpoint{4.446234in}{3.302589in}}%
\pgfpathlineto{\pgfqpoint{4.479304in}{3.197551in}}%
\pgfpathlineto{\pgfqpoint{4.515054in}{3.092513in}}%
\pgfpathlineto{\pgfqpoint{4.553661in}{2.987475in}}%
\pgfpathlineto{\pgfqpoint{4.584560in}{2.908696in}}%
\pgfpathlineto{\pgfqpoint{4.617202in}{2.829918in}}%
\pgfpathlineto{\pgfqpoint{4.651645in}{2.751140in}}%
\pgfpathlineto{\pgfqpoint{4.700500in}{2.646102in}}%
\pgfpathlineto{\pgfqpoint{4.739268in}{2.567323in}}%
\pgfpathlineto{\pgfqpoint{4.794010in}{2.462285in}}%
\pgfpathlineto{\pgfqpoint{4.846865in}{2.366718in}}%
\pgfpathlineto{\pgfqpoint{4.882606in}{2.304729in}}%
\pgfpathlineto{\pgfqpoint{4.935585in}{2.216716in}}%
\pgfpathlineto{\pgfqpoint{4.979094in}{2.147172in}}%
\pgfpathlineto{\pgfqpoint{5.047879in}{2.042134in}}%
\pgfpathlineto{\pgfqpoint{5.120193in}{1.937096in}}%
\pgfpathlineto{\pgfqpoint{5.196066in}{1.832058in}}%
\pgfpathlineto{\pgfqpoint{5.275448in}{1.727020in}}%
\pgfpathlineto{\pgfqpoint{5.358370in}{1.621982in}}%
\pgfpathlineto{\pgfqpoint{5.444771in}{1.516944in}}%
\pgfpathlineto{\pgfqpoint{5.534601in}{1.411906in}}%
\pgfpathlineto{\pgfqpoint{5.627812in}{1.306869in}}%
\pgfpathlineto{\pgfqpoint{5.734066in}{1.191503in}}%
\pgfpathlineto{\pgfqpoint{5.824208in}{1.096793in}}%
\pgfpathlineto{\pgfqpoint{5.927151in}{0.991755in}}%
\pgfpathlineto{\pgfqpoint{6.033240in}{0.886717in}}%
\pgfpathlineto{\pgfqpoint{6.148093in}{0.776141in}}%
\pgfpathlineto{\pgfqpoint{6.266386in}{0.665275in}}%
\pgfpathlineto{\pgfqpoint{6.368678in}{0.571603in}}%
\pgfpathlineto{\pgfqpoint{6.368678in}{0.571603in}}%
\pgfusepath{stroke}%
\end{pgfscope}%
\begin{pgfscope}%
\pgfpathrectangle{\pgfqpoint{0.854460in}{0.571603in}}{\pgfqpoint{5.885100in}{5.225635in}}%
\pgfusepath{clip}%
\pgfsetbuttcap%
\pgfsetroundjoin%
\pgfsetlinewidth{1.505625pt}%
\definecolor{currentstroke}{rgb}{0.239346,0.300855,0.540844}%
\pgfsetstrokecolor{currentstroke}%
\pgfsetdash{}{0pt}%
\pgfpathmoveto{\pgfqpoint{2.199834in}{0.571603in}}%
\pgfpathlineto{\pgfqpoint{2.126115in}{0.633238in}}%
\pgfpathlineto{\pgfqpoint{2.066968in}{0.684827in}}%
\pgfpathlineto{\pgfqpoint{2.007822in}{0.738559in}}%
\pgfpathlineto{\pgfqpoint{1.935002in}{0.807939in}}%
\pgfpathlineto{\pgfqpoint{1.882494in}{0.860458in}}%
\pgfpathlineto{\pgfqpoint{1.830381in}{0.914880in}}%
\pgfpathlineto{\pgfqpoint{1.760715in}{0.991755in}}%
\pgfpathlineto{\pgfqpoint{1.712088in}{1.048483in}}%
\pgfpathlineto{\pgfqpoint{1.651680in}{1.123052in}}%
\pgfpathlineto{\pgfqpoint{1.592106in}{1.201831in}}%
\pgfpathlineto{\pgfqpoint{1.536696in}{1.280609in}}%
\pgfpathlineto{\pgfqpoint{1.502001in}{1.333128in}}%
\pgfpathlineto{\pgfqpoint{1.453222in}{1.411906in}}%
\pgfpathlineto{\pgfqpoint{1.416354in}{1.475945in}}%
\pgfpathlineto{\pgfqpoint{1.380218in}{1.543204in}}%
\pgfpathlineto{\pgfqpoint{1.341235in}{1.621982in}}%
\pgfpathlineto{\pgfqpoint{1.317154in}{1.674501in}}%
\pgfpathlineto{\pgfqpoint{1.283855in}{1.753280in}}%
\pgfpathlineto{\pgfqpoint{1.253832in}{1.832058in}}%
\pgfpathlineto{\pgfqpoint{1.226994in}{1.910836in}}%
\pgfpathlineto{\pgfqpoint{1.203240in}{1.989615in}}%
\pgfpathlineto{\pgfqpoint{1.182508in}{2.068393in}}%
\pgfpathlineto{\pgfqpoint{1.170386in}{2.120912in}}%
\pgfpathlineto{\pgfqpoint{1.154541in}{2.199691in}}%
\pgfpathlineto{\pgfqpoint{1.141602in}{2.278469in}}%
\pgfpathlineto{\pgfqpoint{1.131455in}{2.357248in}}%
\pgfpathlineto{\pgfqpoint{1.124030in}{2.436026in}}%
\pgfpathlineto{\pgfqpoint{1.119360in}{2.514804in}}%
\pgfpathlineto{\pgfqpoint{1.117439in}{2.593583in}}%
\pgfpathlineto{\pgfqpoint{1.118192in}{2.672361in}}%
\pgfpathlineto{\pgfqpoint{1.121631in}{2.751140in}}%
\pgfpathlineto{\pgfqpoint{1.127827in}{2.829918in}}%
\pgfpathlineto{\pgfqpoint{1.136707in}{2.908696in}}%
\pgfpathlineto{\pgfqpoint{1.148241in}{2.987475in}}%
\pgfpathlineto{\pgfqpoint{1.157546in}{3.039994in}}%
\pgfpathlineto{\pgfqpoint{1.173771in}{3.118772in}}%
\pgfpathlineto{\pgfqpoint{1.186178in}{3.171291in}}%
\pgfpathlineto{\pgfqpoint{1.209341in}{3.257593in}}%
\pgfpathlineto{\pgfqpoint{1.231212in}{3.328848in}}%
\pgfpathlineto{\pgfqpoint{1.248919in}{3.381367in}}%
\pgfpathlineto{\pgfqpoint{1.278088in}{3.460145in}}%
\pgfpathlineto{\pgfqpoint{1.310523in}{3.538924in}}%
\pgfpathlineto{\pgfqpoint{1.333993in}{3.591443in}}%
\pgfpathlineto{\pgfqpoint{1.372200in}{3.670221in}}%
\pgfpathlineto{\pgfqpoint{1.399685in}{3.722740in}}%
\pgfpathlineto{\pgfqpoint{1.428867in}{3.775259in}}%
\pgfpathlineto{\pgfqpoint{1.459814in}{3.827778in}}%
\pgfpathlineto{\pgfqpoint{1.492593in}{3.880297in}}%
\pgfpathlineto{\pgfqpoint{1.534648in}{3.943640in}}%
\pgfpathlineto{\pgfqpoint{1.583121in}{4.011594in}}%
\pgfpathlineto{\pgfqpoint{1.623368in}{4.064663in}}%
\pgfpathlineto{\pgfqpoint{1.682515in}{4.137440in}}%
\pgfpathlineto{\pgfqpoint{1.733009in}{4.195411in}}%
\pgfpathlineto{\pgfqpoint{1.781493in}{4.247930in}}%
\pgfpathlineto{\pgfqpoint{1.832725in}{4.300449in}}%
\pgfpathlineto{\pgfqpoint{1.889528in}{4.355306in}}%
\pgfpathlineto{\pgfqpoint{1.948675in}{4.409063in}}%
\pgfpathlineto{\pgfqpoint{2.007822in}{4.459742in}}%
\pgfpathlineto{\pgfqpoint{2.070772in}{4.510524in}}%
\pgfpathlineto{\pgfqpoint{2.140132in}{4.563043in}}%
\pgfpathlineto{\pgfqpoint{2.185262in}{4.595492in}}%
\pgfpathlineto{\pgfqpoint{2.253239in}{4.641822in}}%
\pgfpathlineto{\pgfqpoint{2.333129in}{4.692772in}}%
\pgfpathlineto{\pgfqpoint{2.392275in}{4.728081in}}%
\pgfpathlineto{\pgfqpoint{2.472648in}{4.773119in}}%
\pgfpathlineto{\pgfqpoint{2.540142in}{4.808393in}}%
\pgfpathlineto{\pgfqpoint{2.599289in}{4.837469in}}%
\pgfpathlineto{\pgfqpoint{2.658436in}{4.864892in}}%
\pgfpathlineto{\pgfqpoint{2.717582in}{4.890693in}}%
\pgfpathlineto{\pgfqpoint{2.776729in}{4.914904in}}%
\pgfpathlineto{\pgfqpoint{2.835876in}{4.937553in}}%
\pgfpathlineto{\pgfqpoint{2.924596in}{4.968558in}}%
\pgfpathlineto{\pgfqpoint{2.983743in}{4.987318in}}%
\pgfpathlineto{\pgfqpoint{3.072463in}{5.012445in}}%
\pgfpathlineto{\pgfqpoint{3.161183in}{5.033920in}}%
\pgfpathlineto{\pgfqpoint{3.220330in}{5.046064in}}%
\pgfpathlineto{\pgfqpoint{3.279476in}{5.056533in}}%
\pgfpathlineto{\pgfqpoint{3.338623in}{5.065154in}}%
\pgfpathlineto{\pgfqpoint{3.397770in}{5.071765in}}%
\pgfpathlineto{\pgfqpoint{3.456917in}{5.076339in}}%
\pgfpathlineto{\pgfqpoint{3.516063in}{5.078686in}}%
\pgfpathlineto{\pgfqpoint{3.575210in}{5.078591in}}%
\pgfpathlineto{\pgfqpoint{3.634357in}{5.075804in}}%
\pgfpathlineto{\pgfqpoint{3.693504in}{5.070037in}}%
\pgfpathlineto{\pgfqpoint{3.752650in}{5.060927in}}%
\pgfpathlineto{\pgfqpoint{3.811797in}{5.047752in}}%
\pgfpathlineto{\pgfqpoint{3.853806in}{5.035714in}}%
\pgfpathlineto{\pgfqpoint{3.870944in}{5.030097in}}%
\pgfpathlineto{\pgfqpoint{3.924283in}{5.009454in}}%
\pgfpathlineto{\pgfqpoint{3.930090in}{5.006932in}}%
\pgfpathlineto{\pgfqpoint{3.977959in}{4.983195in}}%
\pgfpathlineto{\pgfqpoint{3.989237in}{4.976910in}}%
\pgfpathlineto{\pgfqpoint{4.021760in}{4.956935in}}%
\pgfpathlineto{\pgfqpoint{4.058440in}{4.930676in}}%
\pgfpathlineto{\pgfqpoint{4.090096in}{4.904416in}}%
\pgfpathlineto{\pgfqpoint{4.117805in}{4.878157in}}%
\pgfpathlineto{\pgfqpoint{4.142367in}{4.851897in}}%
\pgfpathlineto{\pgfqpoint{4.166677in}{4.822621in}}%
\pgfpathlineto{\pgfqpoint{4.196251in}{4.781607in}}%
\pgfpathlineto{\pgfqpoint{4.218060in}{4.746860in}}%
\pgfpathlineto{\pgfqpoint{4.232928in}{4.720600in}}%
\pgfpathlineto{\pgfqpoint{4.259187in}{4.668081in}}%
\pgfpathlineto{\pgfqpoint{4.284971in}{4.607037in}}%
\pgfpathlineto{\pgfqpoint{4.301007in}{4.563043in}}%
\pgfpathlineto{\pgfqpoint{4.318101in}{4.510524in}}%
\pgfpathlineto{\pgfqpoint{4.340179in}{4.431746in}}%
\pgfpathlineto{\pgfqpoint{4.353074in}{4.379227in}}%
\pgfpathlineto{\pgfqpoint{4.373691in}{4.285165in}}%
\pgfpathlineto{\pgfqpoint{4.391102in}{4.195411in}}%
\pgfpathlineto{\pgfqpoint{4.419208in}{4.037854in}}%
\pgfpathlineto{\pgfqpoint{4.466010in}{3.775259in}}%
\pgfpathlineto{\pgfqpoint{4.491984in}{3.643777in}}%
\pgfpathlineto{\pgfqpoint{4.514644in}{3.538924in}}%
\pgfpathlineto{\pgfqpoint{4.539528in}{3.433886in}}%
\pgfpathlineto{\pgfqpoint{4.566859in}{3.328848in}}%
\pgfpathlineto{\pgfqpoint{4.596864in}{3.223810in}}%
\pgfpathlineto{\pgfqpoint{4.629743in}{3.118772in}}%
\pgfpathlineto{\pgfqpoint{4.665675in}{3.013734in}}%
\pgfpathlineto{\pgfqpoint{4.704746in}{2.908696in}}%
\pgfpathlineto{\pgfqpoint{4.736204in}{2.829918in}}%
\pgfpathlineto{\pgfqpoint{4.769553in}{2.751140in}}%
\pgfpathlineto{\pgfqpoint{4.804840in}{2.672361in}}%
\pgfpathlineto{\pgfqpoint{4.854954in}{2.567323in}}%
\pgfpathlineto{\pgfqpoint{4.894848in}{2.488545in}}%
\pgfpathlineto{\pgfqpoint{4.951184in}{2.383507in}}%
\pgfpathlineto{\pgfqpoint{5.011142in}{2.278469in}}%
\pgfpathlineto{\pgfqpoint{5.074752in}{2.173431in}}%
\pgfpathlineto{\pgfqpoint{5.142599in}{2.067570in}}%
\pgfpathlineto{\pgfqpoint{5.201745in}{1.979665in}}%
\pgfpathlineto{\pgfqpoint{5.260892in}{1.895327in}}%
\pgfpathlineto{\pgfqpoint{5.320039in}{1.814163in}}%
\pgfpathlineto{\pgfqpoint{5.379185in}{1.735835in}}%
\pgfpathlineto{\pgfqpoint{5.438332in}{1.660051in}}%
\pgfpathlineto{\pgfqpoint{5.497479in}{1.586566in}}%
\pgfpathlineto{\pgfqpoint{5.556626in}{1.515170in}}%
\pgfpathlineto{\pgfqpoint{5.622235in}{1.438166in}}%
\pgfpathlineto{\pgfqpoint{5.714815in}{1.333128in}}%
\pgfpathlineto{\pgfqpoint{5.810879in}{1.228090in}}%
\pgfpathlineto{\pgfqpoint{5.911506in}{1.121905in}}%
\pgfpathlineto{\pgfqpoint{6.013183in}{1.018014in}}%
\pgfpathlineto{\pgfqpoint{6.036192in}{0.994973in}}%
\pgfpathlineto{\pgfqpoint{6.036192in}{0.994973in}}%
\pgfusepath{stroke}%
\end{pgfscope}%
\begin{pgfscope}%
\pgfpathrectangle{\pgfqpoint{0.854460in}{0.571603in}}{\pgfqpoint{5.885100in}{5.225635in}}%
\pgfusepath{clip}%
\pgfsetbuttcap%
\pgfsetroundjoin%
\pgfsetlinewidth{1.505625pt}%
\definecolor{currentstroke}{rgb}{0.239346,0.300855,0.540844}%
\pgfsetstrokecolor{currentstroke}%
\pgfsetdash{}{0pt}%
\pgfpathmoveto{\pgfqpoint{6.313150in}{0.728529in}}%
\pgfpathlineto{\pgfqpoint{6.325533in}{0.717050in}}%
\pgfpathlineto{\pgfqpoint{6.340827in}{0.702901in}}%
\pgfpathlineto{\pgfqpoint{6.355107in}{0.689748in}}%
\pgfpathlineto{\pgfqpoint{6.369368in}{0.676641in}}%
\pgfpathlineto{\pgfqpoint{6.384680in}{0.662627in}}%
\pgfpathlineto{\pgfqpoint{6.398091in}{0.650382in}}%
\pgfpathlineto{\pgfqpoint{6.414253in}{0.635681in}}%
\pgfpathlineto{\pgfqpoint{6.426993in}{0.624122in}}%
\pgfpathlineto{\pgfqpoint{6.443827in}{0.608904in}}%
\pgfpathlineto{\pgfqpoint{6.456073in}{0.597863in}}%
\pgfpathlineto{\pgfqpoint{6.473400in}{0.582291in}}%
\pgfpathlineto{\pgfqpoint{6.485327in}{0.571603in}}%
\pgfusepath{stroke}%
\end{pgfscope}%
\begin{pgfscope}%
\pgfpathrectangle{\pgfqpoint{0.854460in}{0.571603in}}{\pgfqpoint{5.885100in}{5.225635in}}%
\pgfusepath{clip}%
\pgfsetbuttcap%
\pgfsetroundjoin%
\pgfsetlinewidth{1.505625pt}%
\definecolor{currentstroke}{rgb}{0.227802,0.326594,0.546532}%
\pgfsetstrokecolor{currentstroke}%
\pgfsetdash{}{0pt}%
\pgfpathmoveto{\pgfqpoint{2.104286in}{0.571603in}}%
\pgfpathlineto{\pgfqpoint{2.037395in}{0.628224in}}%
\pgfpathlineto{\pgfqpoint{1.978248in}{0.680367in}}%
\pgfpathlineto{\pgfqpoint{1.919102in}{0.734689in}}%
\pgfpathlineto{\pgfqpoint{1.859955in}{0.791401in}}%
\pgfpathlineto{\pgfqpoint{1.791350in}{0.860458in}}%
\pgfpathlineto{\pgfqpoint{1.741628in}{0.912976in}}%
\pgfpathlineto{\pgfqpoint{1.671078in}{0.991755in}}%
\pgfpathlineto{\pgfqpoint{1.623368in}{1.048057in}}%
\pgfpathlineto{\pgfqpoint{1.563287in}{1.123052in}}%
\pgfpathlineto{\pgfqpoint{1.504338in}{1.201831in}}%
\pgfpathlineto{\pgfqpoint{1.449491in}{1.280609in}}%
\pgfpathlineto{\pgfqpoint{1.415088in}{1.333128in}}%
\pgfpathlineto{\pgfqpoint{1.366741in}{1.411906in}}%
\pgfpathlineto{\pgfqpoint{1.327634in}{1.480483in}}%
\pgfpathlineto{\pgfqpoint{1.294223in}{1.543204in}}%
\pgfpathlineto{\pgfqpoint{1.255463in}{1.621982in}}%
\pgfpathlineto{\pgfqpoint{1.220046in}{1.700761in}}%
\pgfpathlineto{\pgfqpoint{1.198266in}{1.753280in}}%
\pgfpathlineto{\pgfqpoint{1.168264in}{1.832058in}}%
\pgfpathlineto{\pgfqpoint{1.141372in}{1.910836in}}%
\pgfpathlineto{\pgfqpoint{1.117491in}{1.989615in}}%
\pgfpathlineto{\pgfqpoint{1.096614in}{2.068393in}}%
\pgfpathlineto{\pgfqpoint{1.078636in}{2.147172in}}%
\pgfpathlineto{\pgfqpoint{1.063407in}{2.225950in}}%
\pgfpathlineto{\pgfqpoint{1.051049in}{2.304729in}}%
\pgfpathlineto{\pgfqpoint{1.041366in}{2.383507in}}%
\pgfpathlineto{\pgfqpoint{1.034333in}{2.462285in}}%
\pgfpathlineto{\pgfqpoint{1.029993in}{2.541064in}}%
\pgfpathlineto{\pgfqpoint{1.028305in}{2.619842in}}%
\pgfpathlineto{\pgfqpoint{1.029217in}{2.698621in}}%
\pgfpathlineto{\pgfqpoint{1.032734in}{2.777399in}}%
\pgfpathlineto{\pgfqpoint{1.038933in}{2.856177in}}%
\pgfpathlineto{\pgfqpoint{1.047737in}{2.934956in}}%
\pgfpathlineto{\pgfqpoint{1.059120in}{3.013734in}}%
\pgfpathlineto{\pgfqpoint{1.073293in}{3.092513in}}%
\pgfpathlineto{\pgfqpoint{1.091047in}{3.175419in}}%
\pgfpathlineto{\pgfqpoint{1.109776in}{3.250070in}}%
\pgfpathlineto{\pgfqpoint{1.132262in}{3.328848in}}%
\pgfpathlineto{\pgfqpoint{1.157660in}{3.407626in}}%
\pgfpathlineto{\pgfqpoint{1.186092in}{3.486405in}}%
\pgfpathlineto{\pgfqpoint{1.217675in}{3.565183in}}%
\pgfpathlineto{\pgfqpoint{1.252523in}{3.643962in}}%
\pgfpathlineto{\pgfqpoint{1.277617in}{3.696481in}}%
\pgfpathlineto{\pgfqpoint{1.318222in}{3.775259in}}%
\pgfpathlineto{\pgfqpoint{1.357208in}{3.844980in}}%
\pgfpathlineto{\pgfqpoint{1.394215in}{3.906556in}}%
\pgfpathlineto{\pgfqpoint{1.427737in}{3.959075in}}%
\pgfpathlineto{\pgfqpoint{1.475501in}{4.029356in}}%
\pgfpathlineto{\pgfqpoint{1.519941in}{4.090373in}}%
\pgfpathlineto{\pgfqpoint{1.564221in}{4.147729in}}%
\pgfpathlineto{\pgfqpoint{1.623368in}{4.219402in}}%
\pgfpathlineto{\pgfqpoint{1.671661in}{4.274189in}}%
\pgfpathlineto{\pgfqpoint{1.720487in}{4.326708in}}%
\pgfpathlineto{\pgfqpoint{1.771971in}{4.379227in}}%
\pgfpathlineto{\pgfqpoint{1.830381in}{4.435432in}}%
\pgfpathlineto{\pgfqpoint{1.889528in}{4.489145in}}%
\pgfpathlineto{\pgfqpoint{1.948675in}{4.539935in}}%
\pgfpathlineto{\pgfqpoint{2.009432in}{4.589303in}}%
\pgfpathlineto{\pgfqpoint{2.078015in}{4.641822in}}%
\pgfpathlineto{\pgfqpoint{2.150852in}{4.694341in}}%
\pgfpathlineto{\pgfqpoint{2.214835in}{4.737794in}}%
\pgfpathlineto{\pgfqpoint{2.273982in}{4.775973in}}%
\pgfpathlineto{\pgfqpoint{2.362702in}{4.829729in}}%
\pgfpathlineto{\pgfqpoint{2.451422in}{4.879643in}}%
\pgfpathlineto{\pgfqpoint{2.540142in}{4.925874in}}%
\pgfpathlineto{\pgfqpoint{2.603777in}{4.956935in}}%
\pgfpathlineto{\pgfqpoint{2.688009in}{4.995371in}}%
\pgfpathlineto{\pgfqpoint{2.776729in}{5.032811in}}%
\pgfpathlineto{\pgfqpoint{2.851808in}{5.061973in}}%
\pgfpathlineto{\pgfqpoint{2.924607in}{5.088233in}}%
\pgfpathlineto{\pgfqpoint{3.013316in}{5.117327in}}%
\pgfpathlineto{\pgfqpoint{3.102036in}{5.143369in}}%
\pgfpathlineto{\pgfqpoint{3.193847in}{5.167011in}}%
\pgfpathlineto{\pgfqpoint{3.279476in}{5.185880in}}%
\pgfpathlineto{\pgfqpoint{3.338623in}{5.197130in}}%
\pgfpathlineto{\pgfqpoint{3.397770in}{5.206742in}}%
\pgfpathlineto{\pgfqpoint{3.456917in}{5.214785in}}%
\pgfpathlineto{\pgfqpoint{3.516063in}{5.221085in}}%
\pgfpathlineto{\pgfqpoint{3.575210in}{5.225452in}}%
\pgfpathlineto{\pgfqpoint{3.634357in}{5.227848in}}%
\pgfpathlineto{\pgfqpoint{3.693504in}{5.228084in}}%
\pgfpathlineto{\pgfqpoint{3.752650in}{5.225945in}}%
\pgfpathlineto{\pgfqpoint{3.811797in}{5.221181in}}%
\pgfpathlineto{\pgfqpoint{3.870944in}{5.213317in}}%
\pgfpathlineto{\pgfqpoint{3.930090in}{5.202039in}}%
\pgfpathlineto{\pgfqpoint{3.966184in}{5.193271in}}%
\pgfpathlineto{\pgfqpoint{4.018811in}{5.177377in}}%
\pgfpathlineto{\pgfqpoint{4.048384in}{5.166789in}}%
\pgfpathlineto{\pgfqpoint{4.077957in}{5.154539in}}%
\pgfpathlineto{\pgfqpoint{4.107787in}{5.140752in}}%
\pgfpathlineto{\pgfqpoint{4.155426in}{5.114492in}}%
\pgfpathlineto{\pgfqpoint{4.166677in}{5.107517in}}%
\pgfpathlineto{\pgfqpoint{4.196251in}{5.087528in}}%
\pgfpathlineto{\pgfqpoint{4.229038in}{5.061973in}}%
\pgfpathlineto{\pgfqpoint{4.258340in}{5.035714in}}%
\pgfpathlineto{\pgfqpoint{4.284971in}{5.008484in}}%
\pgfpathlineto{\pgfqpoint{4.314544in}{4.973504in}}%
\pgfpathlineto{\pgfqpoint{4.344118in}{4.932669in}}%
\pgfpathlineto{\pgfqpoint{4.361842in}{4.904416in}}%
\pgfpathlineto{\pgfqpoint{4.376852in}{4.878157in}}%
\pgfpathlineto{\pgfqpoint{4.403264in}{4.824843in}}%
\pgfpathlineto{\pgfqpoint{4.424655in}{4.773119in}}%
\pgfpathlineto{\pgfqpoint{4.443142in}{4.720600in}}%
\pgfpathlineto{\pgfqpoint{4.462411in}{4.655731in}}%
\pgfpathlineto{\pgfqpoint{4.472658in}{4.615562in}}%
\pgfpathlineto{\pgfqpoint{4.490084in}{4.536784in}}%
\pgfpathlineto{\pgfqpoint{4.499991in}{4.484265in}}%
\pgfpathlineto{\pgfqpoint{4.513101in}{4.405486in}}%
\pgfpathlineto{\pgfqpoint{4.528253in}{4.300449in}}%
\pgfpathlineto{\pgfqpoint{4.551739in}{4.116632in}}%
\pgfpathlineto{\pgfqpoint{4.575227in}{3.932816in}}%
\pgfpathlineto{\pgfqpoint{4.593905in}{3.801519in}}%
\pgfpathlineto{\pgfqpoint{4.610752in}{3.696481in}}%
\pgfpathlineto{\pgfqpoint{4.629524in}{3.591443in}}%
\pgfpathlineto{\pgfqpoint{4.650674in}{3.486405in}}%
\pgfpathlineto{\pgfqpoint{4.674431in}{3.381367in}}%
\pgfpathlineto{\pgfqpoint{4.698998in}{3.283955in}}%
\pgfpathlineto{\pgfqpoint{4.715330in}{3.223810in}}%
\pgfpathlineto{\pgfqpoint{4.746413in}{3.118772in}}%
\pgfpathlineto{\pgfqpoint{4.780720in}{3.013734in}}%
\pgfpathlineto{\pgfqpoint{4.818378in}{2.908696in}}%
\pgfpathlineto{\pgfqpoint{4.848845in}{2.829918in}}%
\pgfpathlineto{\pgfqpoint{4.881272in}{2.751140in}}%
\pgfpathlineto{\pgfqpoint{4.915700in}{2.672361in}}%
\pgfpathlineto{\pgfqpoint{4.952168in}{2.593583in}}%
\pgfpathlineto{\pgfqpoint{5.004000in}{2.488545in}}%
\pgfpathlineto{\pgfqpoint{5.053878in}{2.393981in}}%
\pgfpathlineto{\pgfqpoint{5.088685in}{2.330988in}}%
\pgfpathlineto{\pgfqpoint{5.142599in}{2.238080in}}%
\pgfpathlineto{\pgfqpoint{5.181750in}{2.173431in}}%
\pgfpathlineto{\pgfqpoint{5.231500in}{2.094653in}}%
\pgfpathlineto{\pgfqpoint{5.290465in}{2.005300in}}%
\pgfpathlineto{\pgfqpoint{5.349612in}{1.919527in}}%
\pgfpathlineto{\pgfqpoint{5.393299in}{1.858318in}}%
\pgfpathlineto{\pgfqpoint{5.471361in}{1.753280in}}%
\pgfpathlineto{\pgfqpoint{5.532333in}{1.674501in}}%
\pgfpathlineto{\pgfqpoint{5.580965in}{1.613555in}}%
\pgfpathlineto{\pgfqpoint{5.580965in}{1.613555in}}%
\pgfusepath{stroke}%
\end{pgfscope}%
\begin{pgfscope}%
\pgfpathrectangle{\pgfqpoint{0.854460in}{0.571603in}}{\pgfqpoint{5.885100in}{5.225635in}}%
\pgfusepath{clip}%
\pgfsetbuttcap%
\pgfsetroundjoin%
\pgfsetlinewidth{1.505625pt}%
\definecolor{currentstroke}{rgb}{0.227802,0.326594,0.546532}%
\pgfsetstrokecolor{currentstroke}%
\pgfsetdash{}{0pt}%
\pgfpathmoveto{\pgfqpoint{5.831969in}{1.320530in}}%
\pgfpathlineto{\pgfqpoint{5.844318in}{1.306869in}}%
\pgfpathlineto{\pgfqpoint{5.852359in}{1.298071in}}%
\pgfpathlineto{\pgfqpoint{5.868295in}{1.280609in}}%
\pgfpathlineto{\pgfqpoint{5.881933in}{1.265827in}}%
\pgfpathlineto{\pgfqpoint{5.892507in}{1.254350in}}%
\pgfpathlineto{\pgfqpoint{5.911506in}{1.233946in}}%
\pgfpathlineto{\pgfqpoint{5.916952in}{1.228090in}}%
\pgfpathlineto{\pgfqpoint{5.941079in}{1.202413in}}%
\pgfpathlineto{\pgfqpoint{5.941626in}{1.201831in}}%
\pgfpathlineto{\pgfqpoint{5.966480in}{1.175571in}}%
\pgfpathlineto{\pgfqpoint{5.970653in}{1.171204in}}%
\pgfpathlineto{\pgfqpoint{5.991556in}{1.149312in}}%
\pgfpathlineto{\pgfqpoint{6.000226in}{1.140316in}}%
\pgfpathlineto{\pgfqpoint{6.016857in}{1.123052in}}%
\pgfpathlineto{\pgfqpoint{6.029799in}{1.109741in}}%
\pgfpathlineto{\pgfqpoint{6.042383in}{1.096793in}}%
\pgfpathlineto{\pgfqpoint{6.059373in}{1.079467in}}%
\pgfpathlineto{\pgfqpoint{6.068131in}{1.070533in}}%
\pgfpathlineto{\pgfqpoint{6.088946in}{1.049484in}}%
\pgfpathlineto{\pgfqpoint{6.094098in}{1.044274in}}%
\pgfpathlineto{\pgfqpoint{6.118520in}{1.019783in}}%
\pgfpathlineto{\pgfqpoint{6.120283in}{1.018014in}}%
\pgfpathlineto{\pgfqpoint{6.146667in}{0.991755in}}%
\pgfpathlineto{\pgfqpoint{6.148093in}{0.990345in}}%
\pgfpathlineto{\pgfqpoint{6.173246in}{0.965495in}}%
\pgfpathlineto{\pgfqpoint{6.177666in}{0.961161in}}%
\pgfpathlineto{\pgfqpoint{6.200039in}{0.939236in}}%
\pgfpathlineto{\pgfqpoint{6.207240in}{0.932230in}}%
\pgfpathlineto{\pgfqpoint{6.227044in}{0.912976in}}%
\pgfpathlineto{\pgfqpoint{6.236813in}{0.903545in}}%
\pgfpathlineto{\pgfqpoint{6.254259in}{0.886717in}}%
\pgfpathlineto{\pgfqpoint{6.266386in}{0.875097in}}%
\pgfpathlineto{\pgfqpoint{6.281681in}{0.860458in}}%
\pgfpathlineto{\pgfqpoint{6.295960in}{0.846879in}}%
\pgfpathlineto{\pgfqpoint{6.309310in}{0.834198in}}%
\pgfpathlineto{\pgfqpoint{6.325533in}{0.818883in}}%
\pgfpathlineto{\pgfqpoint{6.337143in}{0.807939in}}%
\pgfpathlineto{\pgfqpoint{6.355107in}{0.791103in}}%
\pgfpathlineto{\pgfqpoint{6.365177in}{0.781679in}}%
\pgfpathlineto{\pgfqpoint{6.384680in}{0.763531in}}%
\pgfpathlineto{\pgfqpoint{6.393411in}{0.755420in}}%
\pgfpathlineto{\pgfqpoint{6.414253in}{0.736161in}}%
\pgfpathlineto{\pgfqpoint{6.421844in}{0.729160in}}%
\pgfpathlineto{\pgfqpoint{6.443827in}{0.708987in}}%
\pgfpathlineto{\pgfqpoint{6.450472in}{0.702901in}}%
\pgfpathlineto{\pgfqpoint{6.473400in}{0.682002in}}%
\pgfpathlineto{\pgfqpoint{6.479294in}{0.676641in}}%
\pgfpathlineto{\pgfqpoint{6.502973in}{0.655201in}}%
\pgfpathlineto{\pgfqpoint{6.508309in}{0.650382in}}%
\pgfpathlineto{\pgfqpoint{6.532547in}{0.628578in}}%
\pgfpathlineto{\pgfqpoint{6.537513in}{0.624122in}}%
\pgfpathlineto{\pgfqpoint{6.562120in}{0.602128in}}%
\pgfpathlineto{\pgfqpoint{6.566905in}{0.597863in}}%
\pgfpathlineto{\pgfqpoint{6.591693in}{0.575845in}}%
\pgfpathlineto{\pgfqpoint{6.596482in}{0.571603in}}%
\pgfusepath{stroke}%
\end{pgfscope}%
\begin{pgfscope}%
\pgfpathrectangle{\pgfqpoint{0.854460in}{0.571603in}}{\pgfqpoint{5.885100in}{5.225635in}}%
\pgfusepath{clip}%
\pgfsetbuttcap%
\pgfsetroundjoin%
\pgfsetlinewidth{1.505625pt}%
\definecolor{currentstroke}{rgb}{0.218130,0.347432,0.550038}%
\pgfsetstrokecolor{currentstroke}%
\pgfsetdash{}{0pt}%
\pgfpathmoveto{\pgfqpoint{2.013550in}{0.571603in}}%
\pgfpathlineto{\pgfqpoint{1.948675in}{0.627249in}}%
\pgfpathlineto{\pgfqpoint{1.889528in}{0.680074in}}%
\pgfpathlineto{\pgfqpoint{1.830381in}{0.735120in}}%
\pgfpathlineto{\pgfqpoint{1.755856in}{0.807939in}}%
\pgfpathlineto{\pgfqpoint{1.704674in}{0.860458in}}%
\pgfpathlineto{\pgfqpoint{1.652941in}{0.915876in}}%
\pgfpathlineto{\pgfqpoint{1.585818in}{0.991755in}}%
\pgfpathlineto{\pgfqpoint{1.534648in}{1.053002in}}%
\pgfpathlineto{\pgfqpoint{1.479239in}{1.123052in}}%
\pgfpathlineto{\pgfqpoint{1.439902in}{1.175571in}}%
\pgfpathlineto{\pgfqpoint{1.384219in}{1.254350in}}%
\pgfpathlineto{\pgfqpoint{1.332476in}{1.333128in}}%
\pgfpathlineto{\pgfqpoint{1.298061in}{1.388984in}}%
\pgfpathlineto{\pgfqpoint{1.254544in}{1.464425in}}%
\pgfpathlineto{\pgfqpoint{1.226152in}{1.516944in}}%
\pgfpathlineto{\pgfqpoint{1.186426in}{1.595723in}}%
\pgfpathlineto{\pgfqpoint{1.161842in}{1.648242in}}%
\pgfpathlineto{\pgfqpoint{1.138700in}{1.700761in}}%
\pgfpathlineto{\pgfqpoint{1.106659in}{1.779539in}}%
\pgfpathlineto{\pgfqpoint{1.077734in}{1.858318in}}%
\pgfpathlineto{\pgfqpoint{1.051830in}{1.937096in}}%
\pgfpathlineto{\pgfqpoint{1.028848in}{2.015874in}}%
\pgfpathlineto{\pgfqpoint{1.008794in}{2.094653in}}%
\pgfpathlineto{\pgfqpoint{0.991544in}{2.173431in}}%
\pgfpathlineto{\pgfqpoint{0.976994in}{2.252210in}}%
\pgfpathlineto{\pgfqpoint{0.965190in}{2.330988in}}%
\pgfpathlineto{\pgfqpoint{0.956035in}{2.409766in}}%
\pgfpathlineto{\pgfqpoint{0.949458in}{2.488545in}}%
\pgfpathlineto{\pgfqpoint{0.945462in}{2.567323in}}%
\pgfpathlineto{\pgfqpoint{0.944046in}{2.646102in}}%
\pgfpathlineto{\pgfqpoint{0.945205in}{2.724880in}}%
\pgfpathlineto{\pgfqpoint{0.948927in}{2.803659in}}%
\pgfpathlineto{\pgfqpoint{0.955198in}{2.882437in}}%
\pgfpathlineto{\pgfqpoint{0.963998in}{2.961215in}}%
\pgfpathlineto{\pgfqpoint{0.975354in}{3.039994in}}%
\pgfpathlineto{\pgfqpoint{0.989412in}{3.118772in}}%
\pgfpathlineto{\pgfqpoint{1.006036in}{3.197551in}}%
\pgfpathlineto{\pgfqpoint{1.025427in}{3.276329in}}%
\pgfpathlineto{\pgfqpoint{1.047595in}{3.355107in}}%
\pgfpathlineto{\pgfqpoint{1.072580in}{3.433886in}}%
\pgfpathlineto{\pgfqpoint{1.100498in}{3.512664in}}%
\pgfpathlineto{\pgfqpoint{1.131463in}{3.591443in}}%
\pgfpathlineto{\pgfqpoint{1.165583in}{3.670221in}}%
\pgfpathlineto{\pgfqpoint{1.190138in}{3.722740in}}%
\pgfpathlineto{\pgfqpoint{1.229814in}{3.801519in}}%
\pgfpathlineto{\pgfqpoint{1.268488in}{3.872339in}}%
\pgfpathlineto{\pgfqpoint{1.303931in}{3.932816in}}%
\pgfpathlineto{\pgfqpoint{1.353521in}{4.011594in}}%
\pgfpathlineto{\pgfqpoint{1.388855in}{4.064113in}}%
\pgfpathlineto{\pgfqpoint{1.445928in}{4.143499in}}%
\pgfpathlineto{\pgfqpoint{1.506726in}{4.221670in}}%
\pgfpathlineto{\pgfqpoint{1.564221in}{4.290401in}}%
\pgfpathlineto{\pgfqpoint{1.619995in}{4.352967in}}%
\pgfpathlineto{\pgfqpoint{1.669622in}{4.405486in}}%
\pgfpathlineto{\pgfqpoint{1.721849in}{4.458005in}}%
\pgfpathlineto{\pgfqpoint{1.776889in}{4.510524in}}%
\pgfpathlineto{\pgfqpoint{1.834963in}{4.563043in}}%
\pgfpathlineto{\pgfqpoint{1.896301in}{4.615562in}}%
\pgfpathlineto{\pgfqpoint{1.961135in}{4.668081in}}%
\pgfpathlineto{\pgfqpoint{2.029706in}{4.720600in}}%
\pgfpathlineto{\pgfqpoint{2.096542in}{4.769004in}}%
\pgfpathlineto{\pgfqpoint{2.155689in}{4.809660in}}%
\pgfpathlineto{\pgfqpoint{2.220144in}{4.851897in}}%
\pgfpathlineto{\pgfqpoint{2.305151in}{4.904416in}}%
\pgfpathlineto{\pgfqpoint{2.396281in}{4.956935in}}%
\pgfpathlineto{\pgfqpoint{2.480996in}{5.002505in}}%
\pgfpathlineto{\pgfqpoint{2.546417in}{5.035714in}}%
\pgfpathlineto{\pgfqpoint{2.628862in}{5.075115in}}%
\pgfpathlineto{\pgfqpoint{2.717582in}{5.114758in}}%
\pgfpathlineto{\pgfqpoint{2.806303in}{5.151393in}}%
\pgfpathlineto{\pgfqpoint{2.895023in}{5.185320in}}%
\pgfpathlineto{\pgfqpoint{2.983743in}{5.216533in}}%
\pgfpathlineto{\pgfqpoint{3.075024in}{5.245790in}}%
\pgfpathlineto{\pgfqpoint{3.165999in}{5.272049in}}%
\pgfpathlineto{\pgfqpoint{3.249903in}{5.293683in}}%
\pgfpathlineto{\pgfqpoint{3.309050in}{5.307395in}}%
\pgfpathlineto{\pgfqpoint{3.397770in}{5.325631in}}%
\pgfpathlineto{\pgfqpoint{3.486490in}{5.340673in}}%
\pgfpathlineto{\pgfqpoint{3.560583in}{5.350827in}}%
\pgfpathlineto{\pgfqpoint{3.604783in}{5.355730in}}%
\pgfpathlineto{\pgfqpoint{3.663930in}{5.360830in}}%
\pgfpathlineto{\pgfqpoint{3.723077in}{5.364221in}}%
\pgfpathlineto{\pgfqpoint{3.782224in}{5.365746in}}%
\pgfpathlineto{\pgfqpoint{3.841370in}{5.365225in}}%
\pgfpathlineto{\pgfqpoint{3.900517in}{5.362451in}}%
\pgfpathlineto{\pgfqpoint{3.959664in}{5.357186in}}%
\pgfpathlineto{\pgfqpoint{4.018811in}{5.349100in}}%
\pgfpathlineto{\pgfqpoint{4.077957in}{5.337608in}}%
\pgfpathlineto{\pgfqpoint{4.137104in}{5.322442in}}%
\pgfpathlineto{\pgfqpoint{4.196251in}{5.302590in}}%
\pgfpathlineto{\pgfqpoint{4.225824in}{5.290618in}}%
\pgfpathlineto{\pgfqpoint{4.265634in}{5.272049in}}%
\pgfpathlineto{\pgfqpoint{4.284971in}{5.261824in}}%
\pgfpathlineto{\pgfqpoint{4.314544in}{5.244629in}}%
\pgfpathlineto{\pgfqpoint{4.351651in}{5.219530in}}%
\pgfpathlineto{\pgfqpoint{4.384920in}{5.193271in}}%
\pgfpathlineto{\pgfqpoint{4.413745in}{5.167011in}}%
\pgfpathlineto{\pgfqpoint{4.439011in}{5.140752in}}%
\pgfpathlineto{\pgfqpoint{4.462411in}{5.113161in}}%
\pgfpathlineto{\pgfqpoint{4.491984in}{5.072419in}}%
\pgfpathlineto{\pgfqpoint{4.514721in}{5.035714in}}%
\pgfpathlineto{\pgfqpoint{4.529049in}{5.009454in}}%
\pgfpathlineto{\pgfqpoint{4.553787in}{4.956935in}}%
\pgfpathlineto{\pgfqpoint{4.574124in}{4.904416in}}%
\pgfpathlineto{\pgfqpoint{4.591037in}{4.851897in}}%
\pgfpathlineto{\pgfqpoint{4.605228in}{4.799378in}}%
\pgfpathlineto{\pgfqpoint{4.617135in}{4.746860in}}%
\pgfpathlineto{\pgfqpoint{4.627209in}{4.694341in}}%
\pgfpathlineto{\pgfqpoint{4.639851in}{4.614696in}}%
\pgfpathlineto{\pgfqpoint{4.649695in}{4.536784in}}%
\pgfpathlineto{\pgfqpoint{4.660514in}{4.431746in}}%
\pgfpathlineto{\pgfqpoint{4.673836in}{4.274189in}}%
\pgfpathlineto{\pgfqpoint{4.679091in}{4.207309in}}%
\pgfpathlineto{\pgfqpoint{4.679091in}{4.207309in}}%
\pgfusepath{stroke}%
\end{pgfscope}%
\begin{pgfscope}%
\pgfpathrectangle{\pgfqpoint{0.854460in}{0.571603in}}{\pgfqpoint{5.885100in}{5.225635in}}%
\pgfusepath{clip}%
\pgfsetbuttcap%
\pgfsetroundjoin%
\pgfsetlinewidth{1.505625pt}%
\definecolor{currentstroke}{rgb}{0.218130,0.347432,0.550038}%
\pgfsetstrokecolor{currentstroke}%
\pgfsetdash{}{0pt}%
\pgfpathmoveto{\pgfqpoint{4.716136in}{3.816309in}}%
\pgfpathlineto{\pgfqpoint{4.728571in}{3.720952in}}%
\pgfpathlineto{\pgfqpoint{4.744042in}{3.617702in}}%
\pgfpathlineto{\pgfqpoint{4.762373in}{3.512664in}}%
\pgfpathlineto{\pgfqpoint{4.777880in}{3.433886in}}%
\pgfpathlineto{\pgfqpoint{4.801135in}{3.328848in}}%
\pgfpathlineto{\pgfqpoint{4.820623in}{3.250070in}}%
\pgfpathlineto{\pgfqpoint{4.846865in}{3.153906in}}%
\pgfpathlineto{\pgfqpoint{4.864967in}{3.092513in}}%
\pgfpathlineto{\pgfqpoint{4.889983in}{3.013734in}}%
\pgfpathlineto{\pgfqpoint{4.916955in}{2.934956in}}%
\pgfpathlineto{\pgfqpoint{4.945914in}{2.856177in}}%
\pgfpathlineto{\pgfqpoint{4.976893in}{2.777399in}}%
\pgfpathlineto{\pgfqpoint{5.009927in}{2.698621in}}%
\pgfpathlineto{\pgfqpoint{5.053878in}{2.600870in}}%
\pgfpathlineto{\pgfqpoint{5.083452in}{2.538718in}}%
\pgfpathlineto{\pgfqpoint{5.121601in}{2.462285in}}%
\pgfpathlineto{\pgfqpoint{5.172172in}{2.366855in}}%
\pgfpathlineto{\pgfqpoint{5.206701in}{2.304729in}}%
\pgfpathlineto{\pgfqpoint{5.260892in}{2.211934in}}%
\pgfpathlineto{\pgfqpoint{5.300397in}{2.147172in}}%
\pgfpathlineto{\pgfqpoint{5.350533in}{2.068393in}}%
\pgfpathlineto{\pgfqpoint{5.408759in}{1.980872in}}%
\pgfpathlineto{\pgfqpoint{5.467906in}{1.895789in}}%
\pgfpathlineto{\pgfqpoint{5.527052in}{1.814116in}}%
\pgfpathlineto{\pgfqpoint{5.586199in}{1.735472in}}%
\pgfpathlineto{\pgfqpoint{5.645346in}{1.659531in}}%
\pgfpathlineto{\pgfqpoint{5.704492in}{1.586020in}}%
\pgfpathlineto{\pgfqpoint{5.763639in}{1.514708in}}%
\pgfpathlineto{\pgfqpoint{5.829029in}{1.438166in}}%
\pgfpathlineto{\pgfqpoint{5.922018in}{1.333128in}}%
\pgfpathlineto{\pgfqpoint{6.018707in}{1.228090in}}%
\pgfpathlineto{\pgfqpoint{6.119074in}{1.123052in}}%
\pgfpathlineto{\pgfqpoint{6.222950in}{1.018014in}}%
\pgfpathlineto{\pgfqpoint{6.330398in}{0.912976in}}%
\pgfpathlineto{\pgfqpoint{6.443827in}{0.805565in}}%
\pgfpathlineto{\pgfqpoint{6.562120in}{0.696898in}}%
\pgfpathlineto{\pgfqpoint{6.680414in}{0.591301in}}%
\pgfpathlineto{\pgfqpoint{6.702832in}{0.571603in}}%
\pgfpathlineto{\pgfqpoint{6.702832in}{0.571603in}}%
\pgfusepath{stroke}%
\end{pgfscope}%
\begin{pgfscope}%
\pgfpathrectangle{\pgfqpoint{0.854460in}{0.571603in}}{\pgfqpoint{5.885100in}{5.225635in}}%
\pgfusepath{clip}%
\pgfsetbuttcap%
\pgfsetroundjoin%
\pgfsetlinewidth{1.505625pt}%
\definecolor{currentstroke}{rgb}{0.206756,0.371758,0.553117}%
\pgfsetstrokecolor{currentstroke}%
\pgfsetdash{}{0pt}%
\pgfpathmoveto{\pgfqpoint{1.927055in}{0.571603in}}%
\pgfpathlineto{\pgfqpoint{1.859955in}{0.629974in}}%
\pgfpathlineto{\pgfqpoint{1.780068in}{0.702901in}}%
\pgfpathlineto{\pgfqpoint{1.712088in}{0.768349in}}%
\pgfpathlineto{\pgfqpoint{1.646978in}{0.834198in}}%
\pgfpathlineto{\pgfqpoint{1.593795in}{0.890682in}}%
\pgfpathlineto{\pgfqpoint{1.526955in}{0.965495in}}%
\pgfpathlineto{\pgfqpoint{1.475501in}{1.026398in}}%
\pgfpathlineto{\pgfqpoint{1.416354in}{1.100402in}}%
\pgfpathlineto{\pgfqpoint{1.357208in}{1.179482in}}%
\pgfpathlineto{\pgfqpoint{1.304912in}{1.254350in}}%
\pgfpathlineto{\pgfqpoint{1.268488in}{1.309653in}}%
\pgfpathlineto{\pgfqpoint{1.221495in}{1.385647in}}%
\pgfpathlineto{\pgfqpoint{1.179767in}{1.458108in}}%
\pgfpathlineto{\pgfqpoint{1.148054in}{1.516944in}}%
\pgfpathlineto{\pgfqpoint{1.108637in}{1.595723in}}%
\pgfpathlineto{\pgfqpoint{1.072470in}{1.674501in}}%
\pgfpathlineto{\pgfqpoint{1.050140in}{1.727020in}}%
\pgfpathlineto{\pgfqpoint{1.019239in}{1.805799in}}%
\pgfpathlineto{\pgfqpoint{0.991365in}{1.884577in}}%
\pgfpathlineto{\pgfqpoint{0.966422in}{1.963355in}}%
\pgfpathlineto{\pgfqpoint{0.943181in}{2.046592in}}%
\pgfpathlineto{\pgfqpoint{0.925123in}{2.120912in}}%
\pgfpathlineto{\pgfqpoint{0.908570in}{2.199691in}}%
\pgfpathlineto{\pgfqpoint{0.894743in}{2.278469in}}%
\pgfpathlineto{\pgfqpoint{0.883463in}{2.357248in}}%
\pgfpathlineto{\pgfqpoint{0.874877in}{2.436026in}}%
\pgfpathlineto{\pgfqpoint{0.868800in}{2.514804in}}%
\pgfpathlineto{\pgfqpoint{0.865230in}{2.593583in}}%
\pgfpathlineto{\pgfqpoint{0.864167in}{2.672361in}}%
\pgfpathlineto{\pgfqpoint{0.865601in}{2.751140in}}%
\pgfpathlineto{\pgfqpoint{0.869523in}{2.829918in}}%
\pgfpathlineto{\pgfqpoint{0.875918in}{2.908696in}}%
\pgfpathlineto{\pgfqpoint{0.884781in}{2.987475in}}%
\pgfpathlineto{\pgfqpoint{0.896280in}{3.066253in}}%
\pgfpathlineto{\pgfqpoint{0.910232in}{3.145032in}}%
\pgfpathlineto{\pgfqpoint{0.926860in}{3.223810in}}%
\pgfpathlineto{\pgfqpoint{0.946045in}{3.302589in}}%
\pgfpathlineto{\pgfqpoint{0.967994in}{3.381367in}}%
\pgfpathlineto{\pgfqpoint{0.984185in}{3.433886in}}%
\pgfpathlineto{\pgfqpoint{1.010784in}{3.512664in}}%
\pgfpathlineto{\pgfqpoint{1.040316in}{3.591443in}}%
\pgfpathlineto{\pgfqpoint{1.072887in}{3.670221in}}%
\pgfpathlineto{\pgfqpoint{1.108602in}{3.749000in}}%
\pgfpathlineto{\pgfqpoint{1.134230in}{3.801519in}}%
\pgfpathlineto{\pgfqpoint{1.161360in}{3.854037in}}%
\pgfpathlineto{\pgfqpoint{1.190046in}{3.906556in}}%
\pgfpathlineto{\pgfqpoint{1.220347in}{3.959075in}}%
\pgfpathlineto{\pgfqpoint{1.252318in}{4.011594in}}%
\pgfpathlineto{\pgfqpoint{1.298061in}{4.082296in}}%
\pgfpathlineto{\pgfqpoint{1.339954in}{4.142892in}}%
\pgfpathlineto{\pgfqpoint{1.386781in}{4.206723in}}%
\pgfpathlineto{\pgfqpoint{1.439543in}{4.274189in}}%
\pgfpathlineto{\pgfqpoint{1.483055in}{4.326708in}}%
\pgfpathlineto{\pgfqpoint{1.534648in}{4.385763in}}%
\pgfpathlineto{\pgfqpoint{1.593795in}{4.449560in}}%
\pgfpathlineto{\pgfqpoint{1.653753in}{4.510524in}}%
\pgfpathlineto{\pgfqpoint{1.712088in}{4.566499in}}%
\pgfpathlineto{\pgfqpoint{1.771235in}{4.620294in}}%
\pgfpathlineto{\pgfqpoint{1.830381in}{4.671381in}}%
\pgfpathlineto{\pgfqpoint{1.890302in}{4.720600in}}%
\pgfpathlineto{\pgfqpoint{1.957769in}{4.773119in}}%
\pgfpathlineto{\pgfqpoint{2.037395in}{4.831650in}}%
\pgfpathlineto{\pgfqpoint{2.104372in}{4.878157in}}%
\pgfpathlineto{\pgfqpoint{2.185262in}{4.931319in}}%
\pgfpathlineto{\pgfqpoint{2.273982in}{4.986008in}}%
\pgfpathlineto{\pgfqpoint{2.362702in}{5.037290in}}%
\pgfpathlineto{\pgfqpoint{2.456952in}{5.088233in}}%
\pgfpathlineto{\pgfqpoint{2.540142in}{5.130362in}}%
\pgfpathlineto{\pgfqpoint{2.628862in}{5.172597in}}%
\pgfpathlineto{\pgfqpoint{2.717582in}{5.212087in}}%
\pgfpathlineto{\pgfqpoint{2.806303in}{5.249008in}}%
\pgfpathlineto{\pgfqpoint{2.895023in}{5.283328in}}%
\pgfpathlineto{\pgfqpoint{2.983743in}{5.315225in}}%
\pgfpathlineto{\pgfqpoint{3.072463in}{5.344695in}}%
\pgfpathlineto{\pgfqpoint{3.161183in}{5.371730in}}%
\pgfpathlineto{\pgfqpoint{3.249903in}{5.396310in}}%
\pgfpathlineto{\pgfqpoint{3.338623in}{5.418410in}}%
\pgfpathlineto{\pgfqpoint{3.427343in}{5.437993in}}%
\pgfpathlineto{\pgfqpoint{3.521176in}{5.455865in}}%
\pgfpathlineto{\pgfqpoint{3.604783in}{5.469069in}}%
\pgfpathlineto{\pgfqpoint{3.693504in}{5.480323in}}%
\pgfpathlineto{\pgfqpoint{3.752650in}{5.485989in}}%
\pgfpathlineto{\pgfqpoint{3.811797in}{5.490125in}}%
\pgfpathlineto{\pgfqpoint{3.870944in}{5.492648in}}%
\pgfpathlineto{\pgfqpoint{3.930090in}{5.493410in}}%
\pgfpathlineto{\pgfqpoint{3.989237in}{5.492242in}}%
\pgfpathlineto{\pgfqpoint{4.048384in}{5.488948in}}%
\pgfpathlineto{\pgfqpoint{4.116872in}{5.482125in}}%
\pgfpathlineto{\pgfqpoint{4.166677in}{5.474813in}}%
\pgfpathlineto{\pgfqpoint{4.225824in}{5.463256in}}%
\pgfpathlineto{\pgfqpoint{4.256774in}{5.455865in}}%
\pgfpathlineto{\pgfqpoint{4.314544in}{5.438804in}}%
\pgfpathlineto{\pgfqpoint{4.344118in}{5.428472in}}%
\pgfpathlineto{\pgfqpoint{4.373691in}{5.416600in}}%
\pgfpathlineto{\pgfqpoint{4.403459in}{5.403346in}}%
\pgfpathlineto{\pgfqpoint{4.452909in}{5.377087in}}%
\pgfpathlineto{\pgfqpoint{4.462411in}{5.371403in}}%
\pgfpathlineto{\pgfqpoint{4.494010in}{5.350827in}}%
\pgfpathlineto{\pgfqpoint{4.528663in}{5.324568in}}%
\pgfpathlineto{\pgfqpoint{4.558542in}{5.298308in}}%
\pgfpathlineto{\pgfqpoint{4.584594in}{5.272049in}}%
\pgfpathlineto{\pgfqpoint{4.610278in}{5.242279in}}%
\pgfpathlineto{\pgfqpoint{4.639851in}{5.201918in}}%
\pgfpathlineto{\pgfqpoint{4.645565in}{5.193271in}}%
\pgfpathlineto{\pgfqpoint{4.669425in}{5.152920in}}%
\pgfpathlineto{\pgfqpoint{4.688728in}{5.114492in}}%
\pgfpathlineto{\pgfqpoint{4.700349in}{5.088233in}}%
\pgfpathlineto{\pgfqpoint{4.720066in}{5.035714in}}%
\pgfpathlineto{\pgfqpoint{4.736157in}{4.983195in}}%
\pgfpathlineto{\pgfqpoint{4.749260in}{4.930676in}}%
\pgfpathlineto{\pgfqpoint{4.759995in}{4.878157in}}%
\pgfpathlineto{\pgfqpoint{4.768657in}{4.825638in}}%
\pgfpathlineto{\pgfqpoint{4.778797in}{4.746860in}}%
\pgfpathlineto{\pgfqpoint{4.786301in}{4.668081in}}%
\pgfpathlineto{\pgfqpoint{4.791822in}{4.589303in}}%
\pgfpathlineto{\pgfqpoint{4.798351in}{4.458005in}}%
\pgfpathlineto{\pgfqpoint{4.816272in}{4.037854in}}%
\pgfpathlineto{\pgfqpoint{4.825305in}{3.906556in}}%
\pgfpathlineto{\pgfqpoint{4.834760in}{3.801519in}}%
\pgfpathlineto{\pgfqpoint{4.846865in}{3.694214in}}%
\pgfpathlineto{\pgfqpoint{4.860864in}{3.591443in}}%
\pgfpathlineto{\pgfqpoint{4.878034in}{3.486405in}}%
\pgfpathlineto{\pgfqpoint{4.892798in}{3.407626in}}%
\pgfpathlineto{\pgfqpoint{4.909381in}{3.328848in}}%
\pgfpathlineto{\pgfqpoint{4.927738in}{3.250070in}}%
\pgfpathlineto{\pgfqpoint{4.947980in}{3.171291in}}%
\pgfpathlineto{\pgfqpoint{4.970176in}{3.092513in}}%
\pgfpathlineto{\pgfqpoint{4.994732in}{3.012562in}}%
\pgfpathlineto{\pgfqpoint{5.024305in}{2.924111in}}%
\pgfpathlineto{\pgfqpoint{5.048701in}{2.856177in}}%
\pgfpathlineto{\pgfqpoint{5.083452in}{2.766376in}}%
\pgfpathlineto{\pgfqpoint{5.113025in}{2.694869in}}%
\pgfpathlineto{\pgfqpoint{5.145926in}{2.619842in}}%
\pgfpathlineto{\pgfqpoint{5.182583in}{2.541064in}}%
\pgfpathlineto{\pgfqpoint{5.221429in}{2.462285in}}%
\pgfpathlineto{\pgfqpoint{5.262485in}{2.383507in}}%
\pgfpathlineto{\pgfqpoint{5.305670in}{2.304729in}}%
\pgfpathlineto{\pgfqpoint{5.351123in}{2.225950in}}%
\pgfpathlineto{\pgfqpoint{5.398719in}{2.147172in}}%
\pgfpathlineto{\pgfqpoint{5.448554in}{2.068393in}}%
\pgfpathlineto{\pgfqpoint{5.500622in}{1.989615in}}%
\pgfpathlineto{\pgfqpoint{5.556626in}{1.908382in}}%
\pgfpathlineto{\pgfqpoint{5.615772in}{1.826041in}}%
\pgfpathlineto{\pgfqpoint{5.674919in}{1.746851in}}%
\pgfpathlineto{\pgfqpoint{5.734066in}{1.670472in}}%
\pgfpathlineto{\pgfqpoint{5.793934in}{1.595723in}}%
\pgfpathlineto{\pgfqpoint{5.859136in}{1.516944in}}%
\pgfpathlineto{\pgfqpoint{5.949465in}{1.411906in}}%
\pgfpathlineto{\pgfqpoint{5.988268in}{1.368147in}}%
\pgfpathlineto{\pgfqpoint{5.988268in}{1.368147in}}%
\pgfusepath{stroke}%
\end{pgfscope}%
\begin{pgfscope}%
\pgfpathrectangle{\pgfqpoint{0.854460in}{0.571603in}}{\pgfqpoint{5.885100in}{5.225635in}}%
\pgfusepath{clip}%
\pgfsetbuttcap%
\pgfsetroundjoin%
\pgfsetlinewidth{1.505625pt}%
\definecolor{currentstroke}{rgb}{0.206756,0.371758,0.553117}%
\pgfsetstrokecolor{currentstroke}%
\pgfsetdash{}{0pt}%
\pgfpathmoveto{\pgfqpoint{6.252171in}{1.087699in}}%
\pgfpathlineto{\pgfqpoint{6.266386in}{1.073341in}}%
\pgfpathlineto{\pgfqpoint{6.269166in}{1.070533in}}%
\pgfpathlineto{\pgfqpoint{6.295390in}{1.044274in}}%
\pgfpathlineto{\pgfqpoint{6.295960in}{1.043707in}}%
\pgfpathlineto{\pgfqpoint{6.321816in}{1.018014in}}%
\pgfpathlineto{\pgfqpoint{6.325533in}{1.014351in}}%
\pgfpathlineto{\pgfqpoint{6.348473in}{0.991755in}}%
\pgfpathlineto{\pgfqpoint{6.355107in}{0.985273in}}%
\pgfpathlineto{\pgfqpoint{6.375358in}{0.965495in}}%
\pgfpathlineto{\pgfqpoint{6.384680in}{0.956464in}}%
\pgfpathlineto{\pgfqpoint{6.402471in}{0.939236in}}%
\pgfpathlineto{\pgfqpoint{6.414253in}{0.927914in}}%
\pgfpathlineto{\pgfqpoint{6.429810in}{0.912976in}}%
\pgfpathlineto{\pgfqpoint{6.443827in}{0.899616in}}%
\pgfpathlineto{\pgfqpoint{6.457372in}{0.886717in}}%
\pgfpathlineto{\pgfqpoint{6.473400in}{0.871562in}}%
\pgfpathlineto{\pgfqpoint{6.485156in}{0.860458in}}%
\pgfpathlineto{\pgfqpoint{6.502973in}{0.843744in}}%
\pgfpathlineto{\pgfqpoint{6.513161in}{0.834198in}}%
\pgfpathlineto{\pgfqpoint{6.532547in}{0.816155in}}%
\pgfpathlineto{\pgfqpoint{6.541385in}{0.807939in}}%
\pgfpathlineto{\pgfqpoint{6.562120in}{0.788787in}}%
\pgfpathlineto{\pgfqpoint{6.569827in}{0.781679in}}%
\pgfpathlineto{\pgfqpoint{6.591693in}{0.761634in}}%
\pgfpathlineto{\pgfqpoint{6.598484in}{0.755420in}}%
\pgfpathlineto{\pgfqpoint{6.621267in}{0.734690in}}%
\pgfpathlineto{\pgfqpoint{6.627356in}{0.729160in}}%
\pgfpathlineto{\pgfqpoint{6.650840in}{0.707949in}}%
\pgfpathlineto{\pgfqpoint{6.656440in}{0.702901in}}%
\pgfpathlineto{\pgfqpoint{6.680414in}{0.681403in}}%
\pgfpathlineto{\pgfqpoint{6.685735in}{0.676641in}}%
\pgfpathlineto{\pgfqpoint{6.709987in}{0.655049in}}%
\pgfpathlineto{\pgfqpoint{6.715240in}{0.650382in}}%
\pgfpathlineto{\pgfqpoint{6.739560in}{0.628879in}}%
\pgfusepath{stroke}%
\end{pgfscope}%
\begin{pgfscope}%
\pgfpathrectangle{\pgfqpoint{0.854460in}{0.571603in}}{\pgfqpoint{5.885100in}{5.225635in}}%
\pgfusepath{clip}%
\pgfsetbuttcap%
\pgfsetroundjoin%
\pgfsetlinewidth{1.505625pt}%
\definecolor{currentstroke}{rgb}{0.197636,0.391528,0.554969}%
\pgfsetstrokecolor{currentstroke}%
\pgfsetdash{}{0pt}%
\pgfpathmoveto{\pgfqpoint{1.844312in}{0.571603in}}%
\pgfpathlineto{\pgfqpoint{1.830381in}{0.583696in}}%
\pgfpathlineto{\pgfqpoint{1.814129in}{0.597863in}}%
\pgfpathlineto{\pgfqpoint{1.812832in}{0.599010in}}%
\pgfusepath{stroke}%
\end{pgfscope}%
\begin{pgfscope}%
\pgfpathrectangle{\pgfqpoint{0.854460in}{0.571603in}}{\pgfqpoint{5.885100in}{5.225635in}}%
\pgfusepath{clip}%
\pgfsetbuttcap%
\pgfsetroundjoin%
\pgfsetlinewidth{1.505625pt}%
\definecolor{currentstroke}{rgb}{0.197636,0.391528,0.554969}%
\pgfsetstrokecolor{currentstroke}%
\pgfsetdash{}{0pt}%
\pgfpathmoveto{\pgfqpoint{1.537085in}{0.866600in}}%
\pgfpathlineto{\pgfqpoint{1.534648in}{0.869231in}}%
\pgfpathlineto{\pgfqpoint{1.518549in}{0.886717in}}%
\pgfpathlineto{\pgfqpoint{1.505074in}{0.901579in}}%
\pgfpathlineto{\pgfqpoint{1.494807in}{0.912976in}}%
\pgfpathlineto{\pgfqpoint{1.475501in}{0.934739in}}%
\pgfpathlineto{\pgfqpoint{1.471538in}{0.939236in}}%
\pgfpathlineto{\pgfqpoint{1.448775in}{0.965495in}}%
\pgfpathlineto{\pgfqpoint{1.445928in}{0.968837in}}%
\pgfpathlineto{\pgfqpoint{1.426533in}{0.991755in}}%
\pgfpathlineto{\pgfqpoint{1.416354in}{1.003971in}}%
\pgfpathlineto{\pgfqpoint{1.404737in}{1.018014in}}%
\pgfpathlineto{\pgfqpoint{1.386781in}{1.040061in}}%
\pgfpathlineto{\pgfqpoint{1.383375in}{1.044274in}}%
\pgfpathlineto{\pgfqpoint{1.362518in}{1.070533in}}%
\pgfpathlineto{\pgfqpoint{1.357208in}{1.077336in}}%
\pgfpathlineto{\pgfqpoint{1.342133in}{1.096793in}}%
\pgfpathlineto{\pgfqpoint{1.327634in}{1.115807in}}%
\pgfpathlineto{\pgfqpoint{1.322152in}{1.123052in}}%
\pgfpathlineto{\pgfqpoint{1.302636in}{1.149312in}}%
\pgfpathlineto{\pgfqpoint{1.298061in}{1.155582in}}%
\pgfpathlineto{\pgfqpoint{1.283593in}{1.175571in}}%
\pgfpathlineto{\pgfqpoint{1.268488in}{1.196783in}}%
\pgfpathlineto{\pgfqpoint{1.264923in}{1.201831in}}%
\pgfpathlineto{\pgfqpoint{1.246738in}{1.228090in}}%
\pgfpathlineto{\pgfqpoint{1.238914in}{1.239592in}}%
\pgfpathlineto{\pgfqpoint{1.228963in}{1.254350in}}%
\pgfpathlineto{\pgfqpoint{1.211563in}{1.280609in}}%
\pgfpathlineto{\pgfqpoint{1.209341in}{1.284034in}}%
\pgfpathlineto{\pgfqpoint{1.194658in}{1.306869in}}%
\pgfpathlineto{\pgfqpoint{1.179767in}{1.330421in}}%
\pgfpathlineto{\pgfqpoint{1.178071in}{1.333128in}}%
\pgfpathlineto{\pgfqpoint{1.161980in}{1.359388in}}%
\pgfpathlineto{\pgfqpoint{1.150194in}{1.378972in}}%
\pgfpathlineto{\pgfqpoint{1.146215in}{1.385647in}}%
\pgfpathlineto{\pgfqpoint{1.130900in}{1.411906in}}%
\pgfpathlineto{\pgfqpoint{1.120621in}{1.429878in}}%
\pgfpathlineto{\pgfqpoint{1.115926in}{1.438166in}}%
\pgfpathlineto{\pgfqpoint{1.101382in}{1.464425in}}%
\pgfpathlineto{\pgfqpoint{1.091047in}{1.483468in}}%
\pgfpathlineto{\pgfqpoint{1.087169in}{1.490685in}}%
\pgfpathlineto{\pgfqpoint{1.073392in}{1.516944in}}%
\pgfpathlineto{\pgfqpoint{1.061474in}{1.540127in}}%
\pgfpathlineto{\pgfqpoint{1.059908in}{1.543204in}}%
\pgfpathlineto{\pgfqpoint{1.046889in}{1.569463in}}%
\pgfpathlineto{\pgfqpoint{1.034140in}{1.595723in}}%
\pgfpathlineto{\pgfqpoint{1.031901in}{1.600460in}}%
\pgfpathlineto{\pgfqpoint{1.021834in}{1.621982in}}%
\pgfpathlineto{\pgfqpoint{1.009837in}{1.648242in}}%
\pgfpathlineto{\pgfqpoint{1.002327in}{1.665097in}}%
\pgfpathlineto{\pgfqpoint{0.998184in}{1.674501in}}%
\pgfpathlineto{\pgfqpoint{0.986928in}{1.700761in}}%
\pgfpathlineto{\pgfqpoint{0.975946in}{1.727020in}}%
\pgfpathlineto{\pgfqpoint{0.972754in}{1.734881in}}%
\pgfpathlineto{\pgfqpoint{0.965367in}{1.753280in}}%
\pgfpathlineto{\pgfqpoint{0.955117in}{1.779539in}}%
\pgfpathlineto{\pgfqpoint{0.945142in}{1.805799in}}%
\pgfpathlineto{\pgfqpoint{0.943181in}{1.811131in}}%
\pgfpathlineto{\pgfqpoint{0.935576in}{1.832058in}}%
\pgfpathlineto{\pgfqpoint{0.926318in}{1.858318in}}%
\pgfpathlineto{\pgfqpoint{0.917338in}{1.884577in}}%
\pgfpathlineto{\pgfqpoint{0.913607in}{1.895866in}}%
\pgfpathlineto{\pgfqpoint{0.908720in}{1.910836in}}%
\pgfpathlineto{\pgfqpoint{0.900442in}{1.937096in}}%
\pgfpathlineto{\pgfqpoint{0.892441in}{1.963355in}}%
\pgfpathlineto{\pgfqpoint{0.884718in}{1.989615in}}%
\pgfpathlineto{\pgfqpoint{0.884034in}{1.992037in}}%
\pgfpathlineto{\pgfqpoint{0.877390in}{2.015874in}}%
\pgfpathlineto{\pgfqpoint{0.870350in}{2.042134in}}%
\pgfpathlineto{\pgfqpoint{0.863588in}{2.068393in}}%
\pgfpathlineto{\pgfqpoint{0.857103in}{2.094653in}}%
\pgfpathlineto{\pgfqpoint{0.854460in}{2.105859in}}%
\pgfusepath{stroke}%
\end{pgfscope}%
\begin{pgfscope}%
\pgfpathrectangle{\pgfqpoint{0.854460in}{0.571603in}}{\pgfqpoint{5.885100in}{5.225635in}}%
\pgfusepath{clip}%
\pgfsetbuttcap%
\pgfsetroundjoin%
\pgfsetlinewidth{1.505625pt}%
\definecolor{currentstroke}{rgb}{0.197636,0.391528,0.554969}%
\pgfsetstrokecolor{currentstroke}%
\pgfsetdash{}{0pt}%
\pgfpathmoveto{\pgfqpoint{0.854460in}{3.261442in}}%
\pgfpathlineto{\pgfqpoint{0.871070in}{3.328848in}}%
\pgfpathlineto{\pgfqpoint{0.892890in}{3.407626in}}%
\pgfpathlineto{\pgfqpoint{0.917409in}{3.486405in}}%
\pgfpathlineto{\pgfqpoint{0.944736in}{3.565183in}}%
\pgfpathlineto{\pgfqpoint{0.974979in}{3.643962in}}%
\pgfpathlineto{\pgfqpoint{1.008241in}{3.722740in}}%
\pgfpathlineto{\pgfqpoint{1.044619in}{3.801519in}}%
\pgfpathlineto{\pgfqpoint{1.084208in}{3.880297in}}%
\pgfpathlineto{\pgfqpoint{1.120621in}{3.947404in}}%
\pgfpathlineto{\pgfqpoint{1.157885in}{4.011594in}}%
\pgfpathlineto{\pgfqpoint{1.206943in}{4.090373in}}%
\pgfpathlineto{\pgfqpoint{1.241838in}{4.142892in}}%
\pgfpathlineto{\pgfqpoint{1.298061in}{4.222338in}}%
\pgfpathlineto{\pgfqpoint{1.357653in}{4.300449in}}%
\pgfpathlineto{\pgfqpoint{1.416354in}{4.372106in}}%
\pgfpathlineto{\pgfqpoint{1.468280in}{4.431746in}}%
\pgfpathlineto{\pgfqpoint{1.516473in}{4.484265in}}%
\pgfpathlineto{\pgfqpoint{1.567031in}{4.536784in}}%
\pgfpathlineto{\pgfqpoint{1.623368in}{4.592349in}}%
\pgfpathlineto{\pgfqpoint{1.682515in}{4.647684in}}%
\pgfpathlineto{\pgfqpoint{1.741661in}{4.700257in}}%
\pgfpathlineto{\pgfqpoint{1.800808in}{4.750305in}}%
\pgfpathlineto{\pgfqpoint{1.861687in}{4.799378in}}%
\pgfpathlineto{\pgfqpoint{1.930308in}{4.851897in}}%
\pgfpathlineto{\pgfqpoint{2.007822in}{4.908104in}}%
\pgfpathlineto{\pgfqpoint{2.079026in}{4.956935in}}%
\pgfpathlineto{\pgfqpoint{2.159813in}{5.009454in}}%
\pgfpathlineto{\pgfqpoint{2.245485in}{5.061973in}}%
\pgfpathlineto{\pgfqpoint{2.336615in}{5.114492in}}%
\pgfpathlineto{\pgfqpoint{2.433832in}{5.167011in}}%
\pgfpathlineto{\pgfqpoint{2.510569in}{5.206078in}}%
\pgfpathlineto{\pgfqpoint{2.599289in}{5.248832in}}%
\pgfpathlineto{\pgfqpoint{2.688009in}{5.288968in}}%
\pgfpathlineto{\pgfqpoint{2.776729in}{5.326758in}}%
\pgfpathlineto{\pgfqpoint{2.865449in}{5.362111in}}%
\pgfpathlineto{\pgfqpoint{2.954169in}{5.395250in}}%
\pgfpathlineto{\pgfqpoint{3.053137in}{5.429606in}}%
\pgfpathlineto{\pgfqpoint{3.134672in}{5.455865in}}%
\pgfpathlineto{\pgfqpoint{3.222709in}{5.482125in}}%
\pgfpathlineto{\pgfqpoint{3.319161in}{5.508384in}}%
\pgfpathlineto{\pgfqpoint{3.397770in}{5.527847in}}%
\pgfpathlineto{\pgfqpoint{3.486490in}{5.547673in}}%
\pgfpathlineto{\pgfqpoint{3.575210in}{5.565213in}}%
\pgfpathlineto{\pgfqpoint{3.663930in}{5.580241in}}%
\pgfpathlineto{\pgfqpoint{3.752650in}{5.592701in}}%
\pgfpathlineto{\pgfqpoint{3.841370in}{5.602335in}}%
\pgfpathlineto{\pgfqpoint{3.930090in}{5.609001in}}%
\pgfpathlineto{\pgfqpoint{3.989237in}{5.611604in}}%
\pgfpathlineto{\pgfqpoint{4.048384in}{5.612577in}}%
\pgfpathlineto{\pgfqpoint{4.107531in}{5.611764in}}%
\pgfpathlineto{\pgfqpoint{4.166677in}{5.608987in}}%
\pgfpathlineto{\pgfqpoint{4.225824in}{5.604040in}}%
\pgfpathlineto{\pgfqpoint{4.284971in}{5.596687in}}%
\pgfpathlineto{\pgfqpoint{4.344118in}{5.586635in}}%
\pgfpathlineto{\pgfqpoint{4.403264in}{5.573122in}}%
\pgfpathlineto{\pgfqpoint{4.462411in}{5.555890in}}%
\pgfpathlineto{\pgfqpoint{4.521558in}{5.533967in}}%
\pgfpathlineto{\pgfqpoint{4.576240in}{5.508384in}}%
\pgfpathlineto{\pgfqpoint{4.610278in}{5.489276in}}%
\pgfpathlineto{\pgfqpoint{4.639851in}{5.470324in}}%
\pgfpathlineto{\pgfqpoint{4.669425in}{5.448803in}}%
\pgfpathlineto{\pgfqpoint{4.698998in}{5.424110in}}%
\pgfpathlineto{\pgfqpoint{4.728571in}{5.395479in}}%
\pgfpathlineto{\pgfqpoint{4.758145in}{5.361914in}}%
\pgfpathlineto{\pgfqpoint{4.766937in}{5.350827in}}%
\pgfpathlineto{\pgfqpoint{4.787718in}{5.321971in}}%
\pgfpathlineto{\pgfqpoint{4.802723in}{5.298308in}}%
\pgfpathlineto{\pgfqpoint{4.817754in}{5.272049in}}%
\pgfpathlineto{\pgfqpoint{4.830972in}{5.245790in}}%
\pgfpathlineto{\pgfqpoint{4.846865in}{5.209901in}}%
\pgfpathlineto{\pgfqpoint{4.862875in}{5.167011in}}%
\pgfpathlineto{\pgfqpoint{4.878890in}{5.114492in}}%
\pgfpathlineto{\pgfqpoint{4.891506in}{5.061973in}}%
\pgfpathlineto{\pgfqpoint{4.901533in}{5.009454in}}%
\pgfpathlineto{\pgfqpoint{4.909337in}{4.956935in}}%
\pgfpathlineto{\pgfqpoint{4.915300in}{4.904416in}}%
\pgfpathlineto{\pgfqpoint{4.921600in}{4.825638in}}%
\pgfpathlineto{\pgfqpoint{4.925464in}{4.746860in}}%
\pgfpathlineto{\pgfqpoint{4.927987in}{4.641822in}}%
\pgfpathlineto{\pgfqpoint{4.928650in}{4.496916in}}%
\pgfpathlineto{\pgfqpoint{4.928650in}{4.496916in}}%
\pgfusepath{stroke}%
\end{pgfscope}%
\begin{pgfscope}%
\pgfpathrectangle{\pgfqpoint{0.854460in}{0.571603in}}{\pgfqpoint{5.885100in}{5.225635in}}%
\pgfusepath{clip}%
\pgfsetbuttcap%
\pgfsetroundjoin%
\pgfsetlinewidth{1.505625pt}%
\definecolor{currentstroke}{rgb}{0.197636,0.391528,0.554969}%
\pgfsetstrokecolor{currentstroke}%
\pgfsetdash{}{0pt}%
\pgfpathmoveto{\pgfqpoint{4.930491in}{4.103998in}}%
\pgfpathlineto{\pgfqpoint{4.933295in}{4.011594in}}%
\pgfpathlineto{\pgfqpoint{4.938317in}{3.906556in}}%
\pgfpathlineto{\pgfqpoint{4.945574in}{3.801519in}}%
\pgfpathlineto{\pgfqpoint{4.955389in}{3.696481in}}%
\pgfpathlineto{\pgfqpoint{4.967993in}{3.591443in}}%
\pgfpathlineto{\pgfqpoint{4.979334in}{3.512664in}}%
\pgfpathlineto{\pgfqpoint{4.994732in}{3.421582in}}%
\pgfpathlineto{\pgfqpoint{5.007375in}{3.355107in}}%
\pgfpathlineto{\pgfqpoint{5.024305in}{3.275940in}}%
\pgfpathlineto{\pgfqpoint{5.042897in}{3.197551in}}%
\pgfpathlineto{\pgfqpoint{5.063595in}{3.118772in}}%
\pgfpathlineto{\pgfqpoint{5.086330in}{3.039994in}}%
\pgfpathlineto{\pgfqpoint{5.113025in}{2.955458in}}%
\pgfpathlineto{\pgfqpoint{5.137947in}{2.882437in}}%
\pgfpathlineto{\pgfqpoint{5.172172in}{2.790114in}}%
\pgfpathlineto{\pgfqpoint{5.201745in}{2.715976in}}%
\pgfpathlineto{\pgfqpoint{5.231354in}{2.646102in}}%
\pgfpathlineto{\pgfqpoint{5.266804in}{2.567323in}}%
\pgfpathlineto{\pgfqpoint{5.304469in}{2.488545in}}%
\pgfpathlineto{\pgfqpoint{5.349612in}{2.399851in}}%
\pgfpathlineto{\pgfqpoint{5.386504in}{2.330988in}}%
\pgfpathlineto{\pgfqpoint{5.438332in}{2.239428in}}%
\pgfpathlineto{\pgfqpoint{5.477471in}{2.173431in}}%
\pgfpathlineto{\pgfqpoint{5.527052in}{2.093570in}}%
\pgfpathlineto{\pgfqpoint{5.586199in}{2.002780in}}%
\pgfpathlineto{\pgfqpoint{5.630782in}{1.937096in}}%
\pgfpathlineto{\pgfqpoint{5.704492in}{1.833385in}}%
\pgfpathlineto{\pgfqpoint{5.744235in}{1.779539in}}%
\pgfpathlineto{\pgfqpoint{5.824870in}{1.674501in}}%
\pgfpathlineto{\pgfqpoint{5.887912in}{1.595723in}}%
\pgfpathlineto{\pgfqpoint{5.975441in}{1.490685in}}%
\pgfpathlineto{\pgfqpoint{6.066872in}{1.385647in}}%
\pgfpathlineto{\pgfqpoint{6.162185in}{1.280609in}}%
\pgfpathlineto{\pgfqpoint{6.261365in}{1.175571in}}%
\pgfpathlineto{\pgfqpoint{6.364315in}{1.070533in}}%
\pgfpathlineto{\pgfqpoint{6.473400in}{0.963222in}}%
\pgfpathlineto{\pgfqpoint{6.581411in}{0.860458in}}%
\pgfpathlineto{\pgfqpoint{6.695420in}{0.755420in}}%
\pgfpathlineto{\pgfqpoint{6.739560in}{0.715624in}}%
\pgfpathlineto{\pgfqpoint{6.739560in}{0.715624in}}%
\pgfusepath{stroke}%
\end{pgfscope}%
\begin{pgfscope}%
\pgfpathrectangle{\pgfqpoint{0.854460in}{0.571603in}}{\pgfqpoint{5.885100in}{5.225635in}}%
\pgfusepath{clip}%
\pgfsetbuttcap%
\pgfsetroundjoin%
\pgfsetlinewidth{1.505625pt}%
\definecolor{currentstroke}{rgb}{0.187231,0.414746,0.556547}%
\pgfsetstrokecolor{currentstroke}%
\pgfsetdash{}{0pt}%
\pgfpathmoveto{\pgfqpoint{1.764898in}{0.571603in}}%
\pgfpathlineto{\pgfqpoint{1.741661in}{0.592038in}}%
\pgfpathlineto{\pgfqpoint{1.735067in}{0.597863in}}%
\pgfpathlineto{\pgfqpoint{1.712088in}{0.618460in}}%
\pgfpathlineto{\pgfqpoint{1.705799in}{0.624122in}}%
\pgfpathlineto{\pgfqpoint{1.682515in}{0.645400in}}%
\pgfpathlineto{\pgfqpoint{1.677089in}{0.650382in}}%
\pgfpathlineto{\pgfqpoint{1.652941in}{0.672882in}}%
\pgfpathlineto{\pgfqpoint{1.648927in}{0.676641in}}%
\pgfpathlineto{\pgfqpoint{1.623368in}{0.700931in}}%
\pgfpathlineto{\pgfqpoint{1.621306in}{0.702901in}}%
\pgfpathlineto{\pgfqpoint{1.594224in}{0.729160in}}%
\pgfpathlineto{\pgfqpoint{1.593795in}{0.729584in}}%
\pgfpathlineto{\pgfqpoint{1.567705in}{0.755420in}}%
\pgfpathlineto{\pgfqpoint{1.564221in}{0.758922in}}%
\pgfpathlineto{\pgfqpoint{1.541709in}{0.781679in}}%
\pgfpathlineto{\pgfqpoint{1.534648in}{0.788925in}}%
\pgfpathlineto{\pgfqpoint{1.516225in}{0.807939in}}%
\pgfpathlineto{\pgfqpoint{1.505074in}{0.819622in}}%
\pgfpathlineto{\pgfqpoint{1.491244in}{0.834198in}}%
\pgfpathlineto{\pgfqpoint{1.475501in}{0.851044in}}%
\pgfpathlineto{\pgfqpoint{1.466757in}{0.860458in}}%
\pgfpathlineto{\pgfqpoint{1.445928in}{0.883223in}}%
\pgfpathlineto{\pgfqpoint{1.442751in}{0.886717in}}%
\pgfpathlineto{\pgfqpoint{1.419262in}{0.912976in}}%
\pgfpathlineto{\pgfqpoint{1.416354in}{0.916281in}}%
\pgfpathlineto{\pgfqpoint{1.396290in}{0.939236in}}%
\pgfpathlineto{\pgfqpoint{1.390182in}{0.946332in}}%
\pgfusepath{stroke}%
\end{pgfscope}%
\begin{pgfscope}%
\pgfpathrectangle{\pgfqpoint{0.854460in}{0.571603in}}{\pgfqpoint{5.885100in}{5.225635in}}%
\pgfusepath{clip}%
\pgfsetbuttcap%
\pgfsetroundjoin%
\pgfsetlinewidth{1.505625pt}%
\definecolor{currentstroke}{rgb}{0.187231,0.414746,0.556547}%
\pgfsetstrokecolor{currentstroke}%
\pgfsetdash{}{0pt}%
\pgfpathmoveto{\pgfqpoint{1.156225in}{1.254043in}}%
\pgfpathlineto{\pgfqpoint{1.156020in}{1.254350in}}%
\pgfpathlineto{\pgfqpoint{1.150194in}{1.263196in}}%
\pgfpathlineto{\pgfqpoint{1.138827in}{1.280609in}}%
\pgfpathlineto{\pgfqpoint{1.121977in}{1.306869in}}%
\pgfpathlineto{\pgfqpoint{1.120621in}{1.309029in}}%
\pgfpathlineto{\pgfqpoint{1.105625in}{1.333128in}}%
\pgfpathlineto{\pgfqpoint{1.091047in}{1.356957in}}%
\pgfpathlineto{\pgfqpoint{1.089574in}{1.359388in}}%
\pgfpathlineto{\pgfqpoint{1.074010in}{1.385647in}}%
\pgfpathlineto{\pgfqpoint{1.061474in}{1.407183in}}%
\pgfpathlineto{\pgfqpoint{1.058751in}{1.411906in}}%
\pgfpathlineto{\pgfqpoint{1.043950in}{1.438166in}}%
\pgfpathlineto{\pgfqpoint{1.031901in}{1.459955in}}%
\pgfpathlineto{\pgfqpoint{1.029452in}{1.464425in}}%
\pgfpathlineto{\pgfqpoint{1.015410in}{1.490685in}}%
\pgfpathlineto{\pgfqpoint{1.002327in}{1.515627in}}%
\pgfpathlineto{\pgfqpoint{1.001643in}{1.516944in}}%
\pgfpathlineto{\pgfqpoint{0.988351in}{1.543204in}}%
\pgfpathlineto{\pgfqpoint{0.975326in}{1.569463in}}%
\pgfpathlineto{\pgfqpoint{0.972754in}{1.574783in}}%
\pgfpathlineto{\pgfqpoint{0.962735in}{1.595723in}}%
\pgfpathlineto{\pgfqpoint{0.950454in}{1.621982in}}%
\pgfpathlineto{\pgfqpoint{0.943181in}{1.637919in}}%
\pgfpathlineto{\pgfqpoint{0.938520in}{1.648242in}}%
\pgfpathlineto{\pgfqpoint{0.926974in}{1.674501in}}%
\pgfpathlineto{\pgfqpoint{0.915697in}{1.700761in}}%
\pgfpathlineto{\pgfqpoint{0.913607in}{1.705770in}}%
\pgfpathlineto{\pgfqpoint{0.904840in}{1.727020in}}%
\pgfpathlineto{\pgfqpoint{0.894288in}{1.753280in}}%
\pgfpathlineto{\pgfqpoint{0.884034in}{1.779472in}}%
\pgfpathlineto{\pgfqpoint{0.884008in}{1.779539in}}%
\pgfpathlineto{\pgfqpoint{0.874167in}{1.805799in}}%
\pgfpathlineto{\pgfqpoint{0.864598in}{1.832058in}}%
\pgfpathlineto{\pgfqpoint{0.855302in}{1.858318in}}%
\pgfpathlineto{\pgfqpoint{0.854460in}{1.860779in}}%
\pgfusepath{stroke}%
\end{pgfscope}%
\begin{pgfscope}%
\pgfpathrectangle{\pgfqpoint{0.854460in}{0.571603in}}{\pgfqpoint{5.885100in}{5.225635in}}%
\pgfusepath{clip}%
\pgfsetbuttcap%
\pgfsetroundjoin%
\pgfsetlinewidth{1.505625pt}%
\definecolor{currentstroke}{rgb}{0.187231,0.414746,0.556547}%
\pgfsetstrokecolor{currentstroke}%
\pgfsetdash{}{0pt}%
\pgfpathmoveto{\pgfqpoint{0.854460in}{3.537749in}}%
\pgfpathlineto{\pgfqpoint{0.873372in}{3.591443in}}%
\pgfpathlineto{\pgfqpoint{0.903426in}{3.670221in}}%
\pgfpathlineto{\pgfqpoint{0.936405in}{3.749000in}}%
\pgfpathlineto{\pgfqpoint{0.960110in}{3.801519in}}%
\pgfpathlineto{\pgfqpoint{0.985190in}{3.854037in}}%
\pgfpathlineto{\pgfqpoint{1.025496in}{3.932816in}}%
\pgfpathlineto{\pgfqpoint{1.061474in}{3.998110in}}%
\pgfpathlineto{\pgfqpoint{1.100327in}{4.064113in}}%
\pgfpathlineto{\pgfqpoint{1.150194in}{4.143173in}}%
\pgfpathlineto{\pgfqpoint{1.203682in}{4.221670in}}%
\pgfpathlineto{\pgfqpoint{1.241686in}{4.274189in}}%
\pgfpathlineto{\pgfqpoint{1.298061in}{4.347604in}}%
\pgfpathlineto{\pgfqpoint{1.345307in}{4.405486in}}%
\pgfpathlineto{\pgfqpoint{1.390329in}{4.458005in}}%
\pgfpathlineto{\pgfqpoint{1.445928in}{4.519478in}}%
\pgfpathlineto{\pgfqpoint{1.505074in}{4.581312in}}%
\pgfpathlineto{\pgfqpoint{1.566263in}{4.641822in}}%
\pgfpathlineto{\pgfqpoint{1.623368in}{4.695374in}}%
\pgfpathlineto{\pgfqpoint{1.682515in}{4.748154in}}%
\pgfpathlineto{\pgfqpoint{1.742781in}{4.799378in}}%
\pgfpathlineto{\pgfqpoint{1.807741in}{4.851897in}}%
\pgfpathlineto{\pgfqpoint{1.889528in}{4.914471in}}%
\pgfpathlineto{\pgfqpoint{1.948675in}{4.957513in}}%
\pgfpathlineto{\pgfqpoint{2.037395in}{5.018638in}}%
\pgfpathlineto{\pgfqpoint{2.103665in}{5.061973in}}%
\pgfpathlineto{\pgfqpoint{2.188176in}{5.114492in}}%
\pgfpathlineto{\pgfqpoint{2.277741in}{5.167011in}}%
\pgfpathlineto{\pgfqpoint{2.372889in}{5.219530in}}%
\pgfpathlineto{\pgfqpoint{2.451422in}{5.260509in}}%
\pgfpathlineto{\pgfqpoint{2.540142in}{5.304433in}}%
\pgfpathlineto{\pgfqpoint{2.639721in}{5.350827in}}%
\pgfpathlineto{\pgfqpoint{2.717582in}{5.385048in}}%
\pgfpathlineto{\pgfqpoint{2.825233in}{5.429606in}}%
\pgfpathlineto{\pgfqpoint{2.895023in}{5.456830in}}%
\pgfpathlineto{\pgfqpoint{3.013316in}{5.499942in}}%
\pgfpathlineto{\pgfqpoint{3.102036in}{5.529981in}}%
\pgfpathlineto{\pgfqpoint{3.200278in}{5.560903in}}%
\pgfpathlineto{\pgfqpoint{3.309050in}{5.592287in}}%
\pgfpathlineto{\pgfqpoint{3.397770in}{5.615708in}}%
\pgfpathlineto{\pgfqpoint{3.497928in}{5.639682in}}%
\pgfpathlineto{\pgfqpoint{3.575210in}{5.656355in}}%
\pgfpathlineto{\pgfqpoint{3.663930in}{5.673553in}}%
\pgfpathlineto{\pgfqpoint{3.752650in}{5.688552in}}%
\pgfpathlineto{\pgfqpoint{3.841370in}{5.701136in}}%
\pgfpathlineto{\pgfqpoint{3.930090in}{5.711261in}}%
\pgfpathlineto{\pgfqpoint{4.018811in}{5.718734in}}%
\pgfpathlineto{\pgfqpoint{4.107531in}{5.723096in}}%
\pgfpathlineto{\pgfqpoint{4.166677in}{5.724199in}}%
\pgfpathlineto{\pgfqpoint{4.225824in}{5.723702in}}%
\pgfpathlineto{\pgfqpoint{4.284971in}{5.721453in}}%
\pgfpathlineto{\pgfqpoint{4.344118in}{5.717235in}}%
\pgfpathlineto{\pgfqpoint{4.403264in}{5.710706in}}%
\pgfpathlineto{\pgfqpoint{4.462411in}{5.701717in}}%
\pgfpathlineto{\pgfqpoint{4.521558in}{5.689896in}}%
\pgfpathlineto{\pgfqpoint{4.580705in}{5.674500in}}%
\pgfpathlineto{\pgfqpoint{4.610278in}{5.665439in}}%
\pgfpathlineto{\pgfqpoint{4.669425in}{5.643504in}}%
\pgfpathlineto{\pgfqpoint{4.698998in}{5.630435in}}%
\pgfpathlineto{\pgfqpoint{4.733190in}{5.613422in}}%
\pgfpathlineto{\pgfqpoint{4.777773in}{5.587163in}}%
\pgfpathlineto{\pgfqpoint{4.787718in}{5.580573in}}%
\pgfpathlineto{\pgfqpoint{4.817291in}{5.559343in}}%
\pgfpathlineto{\pgfqpoint{4.847183in}{5.534644in}}%
\pgfpathlineto{\pgfqpoint{4.876438in}{5.506547in}}%
\pgfpathlineto{\pgfqpoint{4.906012in}{5.473104in}}%
\pgfpathlineto{\pgfqpoint{4.919488in}{5.455865in}}%
\pgfpathlineto{\pgfqpoint{4.937987in}{5.429606in}}%
\pgfpathlineto{\pgfqpoint{4.965158in}{5.383604in}}%
\pgfpathlineto{\pgfqpoint{4.981282in}{5.350827in}}%
\pgfpathlineto{\pgfqpoint{4.994732in}{5.319152in}}%
\pgfpathlineto{\pgfqpoint{5.011351in}{5.272049in}}%
\pgfpathlineto{\pgfqpoint{5.026062in}{5.219530in}}%
\pgfpathlineto{\pgfqpoint{5.037303in}{5.167011in}}%
\pgfpathlineto{\pgfqpoint{5.045871in}{5.114492in}}%
\pgfpathlineto{\pgfqpoint{5.052208in}{5.061973in}}%
\pgfpathlineto{\pgfqpoint{5.056640in}{5.009454in}}%
\pgfpathlineto{\pgfqpoint{5.060495in}{4.930676in}}%
\pgfpathlineto{\pgfqpoint{5.061808in}{4.851897in}}%
\pgfpathlineto{\pgfqpoint{5.060792in}{4.746860in}}%
\pgfpathlineto{\pgfqpoint{5.056856in}{4.615562in}}%
\pgfpathlineto{\pgfqpoint{5.043871in}{4.247930in}}%
\pgfpathlineto{\pgfqpoint{5.042193in}{4.116632in}}%
\pgfpathlineto{\pgfqpoint{5.042923in}{4.011594in}}%
\pgfpathlineto{\pgfqpoint{5.045854in}{3.906556in}}%
\pgfpathlineto{\pgfqpoint{5.051277in}{3.801519in}}%
\pgfpathlineto{\pgfqpoint{5.059383in}{3.696481in}}%
\pgfpathlineto{\pgfqpoint{5.067367in}{3.617702in}}%
\pgfpathlineto{\pgfqpoint{5.077108in}{3.538924in}}%
\pgfpathlineto{\pgfqpoint{5.088635in}{3.460145in}}%
\pgfpathlineto{\pgfqpoint{5.102002in}{3.381367in}}%
\pgfpathlineto{\pgfqpoint{5.117319in}{3.302589in}}%
\pgfpathlineto{\pgfqpoint{5.134573in}{3.223810in}}%
\pgfpathlineto{\pgfqpoint{5.153847in}{3.145032in}}%
\pgfpathlineto{\pgfqpoint{5.175207in}{3.066253in}}%
\pgfpathlineto{\pgfqpoint{5.201745in}{2.977657in}}%
\pgfpathlineto{\pgfqpoint{5.224170in}{2.908696in}}%
\pgfpathlineto{\pgfqpoint{5.251871in}{2.829918in}}%
\pgfpathlineto{\pgfqpoint{5.281756in}{2.751140in}}%
\pgfpathlineto{\pgfqpoint{5.320039in}{2.657889in}}%
\pgfpathlineto{\pgfqpoint{5.349612in}{2.590428in}}%
\pgfpathlineto{\pgfqpoint{5.384698in}{2.514804in}}%
\pgfpathlineto{\pgfqpoint{5.423475in}{2.436026in}}%
\pgfpathlineto{\pgfqpoint{5.467906in}{2.351065in}}%
\pgfpathlineto{\pgfqpoint{5.507858in}{2.278469in}}%
\pgfpathlineto{\pgfqpoint{5.556626in}{2.194449in}}%
\pgfpathlineto{\pgfqpoint{5.601340in}{2.120912in}}%
\pgfpathlineto{\pgfqpoint{5.626516in}{2.081004in}}%
\pgfpathlineto{\pgfqpoint{5.626516in}{2.081004in}}%
\pgfusepath{stroke}%
\end{pgfscope}%
\begin{pgfscope}%
\pgfpathrectangle{\pgfqpoint{0.854460in}{0.571603in}}{\pgfqpoint{5.885100in}{5.225635in}}%
\pgfusepath{clip}%
\pgfsetbuttcap%
\pgfsetroundjoin%
\pgfsetlinewidth{1.505625pt}%
\definecolor{currentstroke}{rgb}{0.187231,0.414746,0.556547}%
\pgfsetstrokecolor{currentstroke}%
\pgfsetdash{}{0pt}%
\pgfpathmoveto{\pgfqpoint{5.847701in}{1.762996in}}%
\pgfpathlineto{\pgfqpoint{5.852359in}{1.756816in}}%
\pgfpathlineto{\pgfqpoint{5.855016in}{1.753280in}}%
\pgfpathlineto{\pgfqpoint{5.875019in}{1.727020in}}%
\pgfpathlineto{\pgfqpoint{5.881933in}{1.718078in}}%
\pgfpathlineto{\pgfqpoint{5.895278in}{1.700761in}}%
\pgfpathlineto{\pgfqpoint{5.911506in}{1.680027in}}%
\pgfpathlineto{\pgfqpoint{5.915818in}{1.674501in}}%
\pgfpathlineto{\pgfqpoint{5.936590in}{1.648242in}}%
\pgfpathlineto{\pgfqpoint{5.941079in}{1.642642in}}%
\pgfpathlineto{\pgfqpoint{5.957597in}{1.621982in}}%
\pgfpathlineto{\pgfqpoint{5.970653in}{1.605890in}}%
\pgfpathlineto{\pgfqpoint{5.978879in}{1.595723in}}%
\pgfpathlineto{\pgfqpoint{6.000226in}{1.569717in}}%
\pgfpathlineto{\pgfqpoint{6.000434in}{1.569463in}}%
\pgfpathlineto{\pgfqpoint{6.022183in}{1.543204in}}%
\pgfpathlineto{\pgfqpoint{6.029799in}{1.534133in}}%
\pgfpathlineto{\pgfqpoint{6.044200in}{1.516944in}}%
\pgfpathlineto{\pgfqpoint{6.059373in}{1.499079in}}%
\pgfpathlineto{\pgfqpoint{6.066486in}{1.490685in}}%
\pgfpathlineto{\pgfqpoint{6.088946in}{1.464532in}}%
\pgfpathlineto{\pgfqpoint{6.089037in}{1.464425in}}%
\pgfpathlineto{\pgfqpoint{6.111785in}{1.438166in}}%
\pgfpathlineto{\pgfqpoint{6.118520in}{1.430490in}}%
\pgfpathlineto{\pgfqpoint{6.134796in}{1.411906in}}%
\pgfpathlineto{\pgfqpoint{6.148093in}{1.396914in}}%
\pgfpathlineto{\pgfqpoint{6.158069in}{1.385647in}}%
\pgfpathlineto{\pgfqpoint{6.177666in}{1.363784in}}%
\pgfpathlineto{\pgfqpoint{6.181601in}{1.359388in}}%
\pgfpathlineto{\pgfqpoint{6.205372in}{1.333128in}}%
\pgfpathlineto{\pgfqpoint{6.207240in}{1.331086in}}%
\pgfpathlineto{\pgfqpoint{6.229362in}{1.306869in}}%
\pgfpathlineto{\pgfqpoint{6.236813in}{1.298805in}}%
\pgfpathlineto{\pgfqpoint{6.253608in}{1.280609in}}%
\pgfpathlineto{\pgfqpoint{6.266386in}{1.266920in}}%
\pgfpathlineto{\pgfqpoint{6.278108in}{1.254350in}}%
\pgfpathlineto{\pgfqpoint{6.295960in}{1.235416in}}%
\pgfpathlineto{\pgfqpoint{6.302861in}{1.228090in}}%
\pgfpathlineto{\pgfqpoint{6.325533in}{1.204280in}}%
\pgfpathlineto{\pgfqpoint{6.327863in}{1.201831in}}%
\pgfpathlineto{\pgfqpoint{6.353095in}{1.175571in}}%
\pgfpathlineto{\pgfqpoint{6.355107in}{1.173497in}}%
\pgfpathlineto{\pgfqpoint{6.378552in}{1.149312in}}%
\pgfpathlineto{\pgfqpoint{6.384680in}{1.143054in}}%
\pgfpathlineto{\pgfqpoint{6.404257in}{1.123052in}}%
\pgfpathlineto{\pgfqpoint{6.414253in}{1.112939in}}%
\pgfpathlineto{\pgfqpoint{6.430208in}{1.096793in}}%
\pgfpathlineto{\pgfqpoint{6.443827in}{1.083142in}}%
\pgfpathlineto{\pgfqpoint{6.456404in}{1.070533in}}%
\pgfpathlineto{\pgfqpoint{6.473400in}{1.053652in}}%
\pgfpathlineto{\pgfqpoint{6.482842in}{1.044274in}}%
\pgfpathlineto{\pgfqpoint{6.502973in}{1.024458in}}%
\pgfpathlineto{\pgfqpoint{6.509521in}{1.018014in}}%
\pgfpathlineto{\pgfqpoint{6.532547in}{0.995552in}}%
\pgfpathlineto{\pgfqpoint{6.536440in}{0.991755in}}%
\pgfpathlineto{\pgfqpoint{6.562120in}{0.966923in}}%
\pgfpathlineto{\pgfqpoint{6.563597in}{0.965495in}}%
\pgfpathlineto{\pgfqpoint{6.590984in}{0.939236in}}%
\pgfpathlineto{\pgfqpoint{6.591693in}{0.938560in}}%
\pgfpathlineto{\pgfqpoint{6.618593in}{0.912976in}}%
\pgfpathlineto{\pgfqpoint{6.621267in}{0.910453in}}%
\pgfpathlineto{\pgfqpoint{6.646439in}{0.886717in}}%
\pgfpathlineto{\pgfqpoint{6.650840in}{0.882597in}}%
\pgfpathlineto{\pgfqpoint{6.674519in}{0.860458in}}%
\pgfpathlineto{\pgfqpoint{6.680414in}{0.854986in}}%
\pgfpathlineto{\pgfqpoint{6.702833in}{0.834198in}}%
\pgfpathlineto{\pgfqpoint{6.709987in}{0.827611in}}%
\pgfpathlineto{\pgfqpoint{6.731379in}{0.807939in}}%
\pgfpathlineto{\pgfqpoint{6.739560in}{0.800465in}}%
\pgfusepath{stroke}%
\end{pgfscope}%
\begin{pgfscope}%
\pgfpathrectangle{\pgfqpoint{0.854460in}{0.571603in}}{\pgfqpoint{5.885100in}{5.225635in}}%
\pgfusepath{clip}%
\pgfsetbuttcap%
\pgfsetroundjoin%
\pgfsetlinewidth{1.505625pt}%
\definecolor{currentstroke}{rgb}{0.179019,0.433756,0.557430}%
\pgfsetstrokecolor{currentstroke}%
\pgfsetdash{}{0pt}%
\pgfpathmoveto{\pgfqpoint{1.688535in}{0.571603in}}%
\pgfpathlineto{\pgfqpoint{1.682515in}{0.576944in}}%
\pgfpathlineto{\pgfqpoint{1.659034in}{0.597863in}}%
\pgfpathlineto{\pgfqpoint{1.652941in}{0.603371in}}%
\pgfpathlineto{\pgfqpoint{1.630094in}{0.624122in}}%
\pgfpathlineto{\pgfqpoint{1.623368in}{0.630321in}}%
\pgfpathlineto{\pgfqpoint{1.601706in}{0.650382in}}%
\pgfpathlineto{\pgfqpoint{1.593795in}{0.657817in}}%
\pgfpathlineto{\pgfqpoint{1.573864in}{0.676641in}}%
\pgfpathlineto{\pgfqpoint{1.564221in}{0.685884in}}%
\pgfpathlineto{\pgfqpoint{1.546558in}{0.702901in}}%
\pgfpathlineto{\pgfqpoint{1.534648in}{0.714547in}}%
\pgfpathlineto{\pgfqpoint{1.519781in}{0.729160in}}%
\pgfpathlineto{\pgfqpoint{1.505074in}{0.743832in}}%
\pgfpathlineto{\pgfqpoint{1.493523in}{0.755420in}}%
\pgfpathlineto{\pgfqpoint{1.475501in}{0.773768in}}%
\pgfpathlineto{\pgfqpoint{1.467775in}{0.781679in}}%
\pgfpathlineto{\pgfqpoint{1.445928in}{0.804384in}}%
\pgfpathlineto{\pgfqpoint{1.442527in}{0.807939in}}%
\pgfpathlineto{\pgfqpoint{1.417792in}{0.834198in}}%
\pgfpathlineto{\pgfqpoint{1.416354in}{0.835749in}}%
\pgfpathlineto{\pgfqpoint{1.393595in}{0.860458in}}%
\pgfpathlineto{\pgfqpoint{1.386781in}{0.867967in}}%
\pgfpathlineto{\pgfqpoint{1.369876in}{0.886717in}}%
\pgfpathlineto{\pgfqpoint{1.357208in}{0.900982in}}%
\pgfpathlineto{\pgfqpoint{1.346624in}{0.912976in}}%
\pgfpathlineto{\pgfqpoint{1.327634in}{0.934827in}}%
\pgfpathlineto{\pgfqpoint{1.323828in}{0.939236in}}%
\pgfpathlineto{\pgfqpoint{1.301529in}{0.965495in}}%
\pgfpathlineto{\pgfqpoint{1.298061in}{0.969649in}}%
\pgfpathlineto{\pgfqpoint{1.279729in}{0.991755in}}%
\pgfpathlineto{\pgfqpoint{1.268488in}{1.005521in}}%
\pgfpathlineto{\pgfqpoint{1.258358in}{1.018014in}}%
\pgfpathlineto{\pgfqpoint{1.238914in}{1.042370in}}%
\pgfpathlineto{\pgfqpoint{1.237405in}{1.044274in}}%
\pgfpathlineto{\pgfqpoint{1.216973in}{1.070533in}}%
\pgfpathlineto{\pgfqpoint{1.209341in}{1.080503in}}%
\pgfpathlineto{\pgfqpoint{1.196965in}{1.096793in}}%
\pgfpathlineto{\pgfqpoint{1.179767in}{1.119788in}}%
\pgfpathlineto{\pgfqpoint{1.177345in}{1.123052in}}%
\pgfpathlineto{\pgfqpoint{1.158224in}{1.149312in}}%
\pgfpathlineto{\pgfqpoint{1.150194in}{1.160527in}}%
\pgfpathlineto{\pgfqpoint{1.139510in}{1.175571in}}%
\pgfpathlineto{\pgfqpoint{1.121162in}{1.201831in}}%
\pgfpathlineto{\pgfqpoint{1.120621in}{1.202621in}}%
\pgfpathlineto{\pgfqpoint{1.103330in}{1.228090in}}%
\pgfpathlineto{\pgfqpoint{1.091047in}{1.246481in}}%
\pgfpathlineto{\pgfqpoint{1.085837in}{1.254350in}}%
\pgfpathlineto{\pgfqpoint{1.068782in}{1.280609in}}%
\pgfpathlineto{\pgfqpoint{1.061474in}{1.292073in}}%
\pgfpathlineto{\pgfqpoint{1.052126in}{1.306869in}}%
\pgfpathlineto{\pgfqpoint{1.042764in}{1.321960in}}%
\pgfusepath{stroke}%
\end{pgfscope}%
\begin{pgfscope}%
\pgfpathrectangle{\pgfqpoint{0.854460in}{0.571603in}}{\pgfqpoint{5.885100in}{5.225635in}}%
\pgfusepath{clip}%
\pgfsetbuttcap%
\pgfsetroundjoin%
\pgfsetlinewidth{1.505625pt}%
\definecolor{currentstroke}{rgb}{0.179019,0.433756,0.557430}%
\pgfsetstrokecolor{currentstroke}%
\pgfsetdash{}{0pt}%
\pgfpathmoveto{\pgfqpoint{0.862013in}{1.666348in}}%
\pgfpathlineto{\pgfqpoint{0.858429in}{1.674501in}}%
\pgfpathlineto{\pgfqpoint{0.854460in}{1.683776in}}%
\pgfusepath{stroke}%
\end{pgfscope}%
\begin{pgfscope}%
\pgfpathrectangle{\pgfqpoint{0.854460in}{0.571603in}}{\pgfqpoint{5.885100in}{5.225635in}}%
\pgfusepath{clip}%
\pgfsetbuttcap%
\pgfsetroundjoin%
\pgfsetlinewidth{1.505625pt}%
\definecolor{currentstroke}{rgb}{0.179019,0.433756,0.557430}%
\pgfsetstrokecolor{currentstroke}%
\pgfsetdash{}{0pt}%
\pgfpathmoveto{\pgfqpoint{0.854460in}{3.742772in}}%
\pgfpathlineto{\pgfqpoint{0.857082in}{3.749000in}}%
\pgfpathlineto{\pgfqpoint{0.868405in}{3.775259in}}%
\pgfpathlineto{\pgfqpoint{0.879979in}{3.801519in}}%
\pgfpathlineto{\pgfqpoint{0.884034in}{3.810508in}}%
\pgfpathlineto{\pgfqpoint{0.891949in}{3.827778in}}%
\pgfpathlineto{\pgfqpoint{0.904246in}{3.854037in}}%
\pgfpathlineto{\pgfqpoint{0.913607in}{3.873619in}}%
\pgfpathlineto{\pgfqpoint{0.916851in}{3.880297in}}%
\pgfpathlineto{\pgfqpoint{0.929884in}{3.906556in}}%
\pgfpathlineto{\pgfqpoint{0.943163in}{3.932816in}}%
\pgfpathlineto{\pgfqpoint{0.943181in}{3.932850in}}%
\pgfpathlineto{\pgfqpoint{0.956944in}{3.959075in}}%
\pgfpathlineto{\pgfqpoint{0.970970in}{3.985335in}}%
\pgfpathlineto{\pgfqpoint{0.972754in}{3.988607in}}%
\pgfpathlineto{\pgfqpoint{0.985478in}{4.011594in}}%
\pgfpathlineto{\pgfqpoint{1.000263in}{4.037854in}}%
\pgfpathlineto{\pgfqpoint{1.002327in}{4.041450in}}%
\pgfpathlineto{\pgfqpoint{1.015536in}{4.064113in}}%
\pgfpathlineto{\pgfqpoint{1.031090in}{4.090373in}}%
\pgfpathlineto{\pgfqpoint{1.031901in}{4.091716in}}%
\pgfpathlineto{\pgfqpoint{1.047169in}{4.116632in}}%
\pgfpathlineto{\pgfqpoint{1.061474in}{4.139622in}}%
\pgfpathlineto{\pgfqpoint{1.063539in}{4.142892in}}%
\pgfpathlineto{\pgfqpoint{1.080423in}{4.169151in}}%
\pgfpathlineto{\pgfqpoint{1.091047in}{4.185422in}}%
\pgfpathlineto{\pgfqpoint{1.097666in}{4.195411in}}%
\pgfpathlineto{\pgfqpoint{1.115347in}{4.221670in}}%
\pgfpathlineto{\pgfqpoint{1.120621in}{4.229377in}}%
\pgfpathlineto{\pgfqpoint{1.133502in}{4.247930in}}%
\pgfpathlineto{\pgfqpoint{1.150194in}{4.271632in}}%
\pgfpathlineto{\pgfqpoint{1.152021in}{4.274189in}}%
\pgfpathlineto{\pgfqpoint{1.171091in}{4.300449in}}%
\pgfpathlineto{\pgfqpoint{1.179767in}{4.312225in}}%
\pgfpathlineto{\pgfqpoint{1.190590in}{4.326708in}}%
\pgfpathlineto{\pgfqpoint{1.209341in}{4.351461in}}%
\pgfpathlineto{\pgfqpoint{1.210498in}{4.352967in}}%
\pgfpathlineto{\pgfqpoint{1.230991in}{4.379227in}}%
\pgfpathlineto{\pgfqpoint{1.238914in}{4.389241in}}%
\pgfpathlineto{\pgfqpoint{1.251946in}{4.405486in}}%
\pgfpathlineto{\pgfqpoint{1.268488in}{4.425835in}}%
\pgfpathlineto{\pgfqpoint{1.273359in}{4.431746in}}%
\pgfpathlineto{\pgfqpoint{1.295302in}{4.458005in}}%
\pgfpathlineto{\pgfqpoint{1.298061in}{4.461259in}}%
\pgfpathlineto{\pgfqpoint{1.317827in}{4.484265in}}%
\pgfpathlineto{\pgfqpoint{1.327634in}{4.495534in}}%
\pgfpathlineto{\pgfqpoint{1.340856in}{4.510524in}}%
\pgfpathlineto{\pgfqpoint{1.357208in}{4.528828in}}%
\pgfpathlineto{\pgfqpoint{1.364410in}{4.536784in}}%
\pgfpathlineto{\pgfqpoint{1.386781in}{4.561185in}}%
\pgfpathlineto{\pgfqpoint{1.388507in}{4.563043in}}%
\pgfpathlineto{\pgfqpoint{1.413230in}{4.589303in}}%
\pgfpathlineto{\pgfqpoint{1.416354in}{4.592579in}}%
\pgfpathlineto{\pgfqpoint{1.438558in}{4.615562in}}%
\pgfpathlineto{\pgfqpoint{1.445928in}{4.623096in}}%
\pgfpathlineto{\pgfqpoint{1.464478in}{4.641822in}}%
\pgfpathlineto{\pgfqpoint{1.475501in}{4.652812in}}%
\pgfpathlineto{\pgfqpoint{1.491009in}{4.668081in}}%
\pgfpathlineto{\pgfqpoint{1.505074in}{4.681761in}}%
\pgfpathlineto{\pgfqpoint{1.518170in}{4.694341in}}%
\pgfpathlineto{\pgfqpoint{1.534648in}{4.709977in}}%
\pgfpathlineto{\pgfqpoint{1.545981in}{4.720600in}}%
\pgfpathlineto{\pgfqpoint{1.564221in}{4.737491in}}%
\pgfpathlineto{\pgfqpoint{1.574462in}{4.746860in}}%
\pgfpathlineto{\pgfqpoint{1.593795in}{4.764333in}}%
\pgfpathlineto{\pgfqpoint{1.603633in}{4.773119in}}%
\pgfpathlineto{\pgfqpoint{1.623368in}{4.790532in}}%
\pgfpathlineto{\pgfqpoint{1.633514in}{4.799378in}}%
\pgfpathlineto{\pgfqpoint{1.652941in}{4.816116in}}%
\pgfpathlineto{\pgfqpoint{1.664124in}{4.825638in}}%
\pgfpathlineto{\pgfqpoint{1.682515in}{4.841111in}}%
\pgfpathlineto{\pgfqpoint{1.695484in}{4.851897in}}%
\pgfpathlineto{\pgfqpoint{1.712088in}{4.865542in}}%
\pgfpathlineto{\pgfqpoint{1.727613in}{4.878157in}}%
\pgfpathlineto{\pgfqpoint{1.741661in}{4.889436in}}%
\pgfpathlineto{\pgfqpoint{1.760531in}{4.904416in}}%
\pgfpathlineto{\pgfqpoint{1.771235in}{4.912814in}}%
\pgfpathlineto{\pgfqpoint{1.794255in}{4.930676in}}%
\pgfpathlineto{\pgfqpoint{1.800808in}{4.935700in}}%
\pgfpathlineto{\pgfqpoint{1.828807in}{4.956935in}}%
\pgfpathlineto{\pgfqpoint{1.830381in}{4.958115in}}%
\pgfpathlineto{\pgfqpoint{1.859955in}{4.980016in}}%
\pgfpathlineto{\pgfqpoint{1.864295in}{4.983195in}}%
\pgfpathlineto{\pgfqpoint{1.889528in}{5.001454in}}%
\pgfpathlineto{\pgfqpoint{1.900700in}{5.009454in}}%
\pgfpathlineto{\pgfqpoint{1.919102in}{5.022475in}}%
\pgfpathlineto{\pgfqpoint{1.938006in}{5.035714in}}%
\pgfpathlineto{\pgfqpoint{1.948675in}{5.043097in}}%
\pgfpathlineto{\pgfqpoint{1.976231in}{5.061973in}}%
\pgfpathlineto{\pgfqpoint{1.978248in}{5.063339in}}%
\pgfpathlineto{\pgfqpoint{2.007822in}{5.083114in}}%
\pgfpathlineto{\pgfqpoint{2.015559in}{5.088233in}}%
\pgfpathlineto{\pgfqpoint{2.037395in}{5.102508in}}%
\pgfpathlineto{\pgfqpoint{2.055907in}{5.114492in}}%
\pgfpathlineto{\pgfqpoint{2.066968in}{5.121568in}}%
\pgfpathlineto{\pgfqpoint{2.096542in}{5.140302in}}%
\pgfpathlineto{\pgfqpoint{2.097261in}{5.140752in}}%
\pgfpathlineto{\pgfqpoint{2.126115in}{5.158577in}}%
\pgfpathlineto{\pgfqpoint{2.139896in}{5.167011in}}%
\pgfpathlineto{\pgfqpoint{2.155689in}{5.176561in}}%
\pgfpathlineto{\pgfqpoint{2.183575in}{5.193271in}}%
\pgfpathlineto{\pgfqpoint{2.185262in}{5.194269in}}%
\pgfpathlineto{\pgfqpoint{2.214835in}{5.211548in}}%
\pgfpathlineto{\pgfqpoint{2.228628in}{5.219530in}}%
\pgfpathlineto{\pgfqpoint{2.244409in}{5.228553in}}%
\pgfpathlineto{\pgfqpoint{2.273982in}{5.245312in}}%
\pgfpathlineto{\pgfqpoint{2.274837in}{5.245790in}}%
\pgfpathlineto{\pgfqpoint{2.303555in}{5.261646in}}%
\pgfpathlineto{\pgfqpoint{2.322559in}{5.272049in}}%
\pgfpathlineto{\pgfqpoint{2.333129in}{5.277766in}}%
\pgfpathlineto{\pgfqpoint{2.362702in}{5.293584in}}%
\pgfpathlineto{\pgfqpoint{2.371641in}{5.298308in}}%
\pgfpathlineto{\pgfqpoint{2.392275in}{5.309084in}}%
\pgfpathlineto{\pgfqpoint{2.421849in}{5.324402in}}%
\pgfpathlineto{\pgfqpoint{2.422174in}{5.324568in}}%
\pgfpathlineto{\pgfqpoint{2.451422in}{5.339312in}}%
\pgfpathlineto{\pgfqpoint{2.474446in}{5.350827in}}%
\pgfpathlineto{\pgfqpoint{2.480996in}{5.354063in}}%
\pgfpathlineto{\pgfqpoint{2.510569in}{5.368482in}}%
\pgfpathlineto{\pgfqpoint{2.528393in}{5.377087in}}%
\pgfpathlineto{\pgfqpoint{2.540142in}{5.382690in}}%
\pgfpathlineto{\pgfqpoint{2.569716in}{5.396625in}}%
\pgfpathlineto{\pgfqpoint{2.584142in}{5.403346in}}%
\pgfpathlineto{\pgfqpoint{2.599289in}{5.410317in}}%
\pgfpathlineto{\pgfqpoint{2.602810in}{5.411919in}}%
\pgfusepath{stroke}%
\end{pgfscope}%
\begin{pgfscope}%
\pgfpathrectangle{\pgfqpoint{0.854460in}{0.571603in}}{\pgfqpoint{5.885100in}{5.225635in}}%
\pgfusepath{clip}%
\pgfsetbuttcap%
\pgfsetroundjoin%
\pgfsetlinewidth{1.505625pt}%
\definecolor{currentstroke}{rgb}{0.179019,0.433756,0.557430}%
\pgfsetstrokecolor{currentstroke}%
\pgfsetdash{}{0pt}%
\pgfpathmoveto{\pgfqpoint{2.953235in}{5.556238in}}%
\pgfpathlineto{\pgfqpoint{2.954169in}{5.556586in}}%
\pgfpathlineto{\pgfqpoint{2.965953in}{5.560903in}}%
\pgfpathlineto{\pgfqpoint{2.983743in}{5.567338in}}%
\pgfpathlineto{\pgfqpoint{3.013316in}{5.577882in}}%
\pgfpathlineto{\pgfqpoint{3.039648in}{5.587163in}}%
\pgfpathlineto{\pgfqpoint{3.042890in}{5.588291in}}%
\pgfpathlineto{\pgfqpoint{3.072463in}{5.598372in}}%
\pgfpathlineto{\pgfqpoint{3.102036in}{5.608340in}}%
\pgfpathlineto{\pgfqpoint{3.117372in}{5.613422in}}%
\pgfpathlineto{\pgfqpoint{3.131610in}{5.618079in}}%
\pgfpathlineto{\pgfqpoint{3.161183in}{5.627581in}}%
\pgfpathlineto{\pgfqpoint{3.190756in}{5.636962in}}%
\pgfpathlineto{\pgfqpoint{3.199512in}{5.639682in}}%
\pgfpathlineto{\pgfqpoint{3.220330in}{5.646063in}}%
\pgfpathlineto{\pgfqpoint{3.249903in}{5.654974in}}%
\pgfpathlineto{\pgfqpoint{3.279476in}{5.663757in}}%
\pgfpathlineto{\pgfqpoint{3.287007in}{5.665941in}}%
\pgfpathlineto{\pgfqpoint{3.309050in}{5.672250in}}%
\pgfpathlineto{\pgfqpoint{3.338623in}{5.680558in}}%
\pgfpathlineto{\pgfqpoint{3.368197in}{5.688728in}}%
\pgfpathlineto{\pgfqpoint{3.381072in}{5.692201in}}%
\pgfpathlineto{\pgfqpoint{3.397770in}{5.696644in}}%
\pgfpathlineto{\pgfqpoint{3.427343in}{5.704333in}}%
\pgfpathlineto{\pgfqpoint{3.456917in}{5.711874in}}%
\pgfpathlineto{\pgfqpoint{3.483302in}{5.718460in}}%
\pgfpathlineto{\pgfqpoint{3.486490in}{5.719245in}}%
\pgfpathlineto{\pgfqpoint{3.516063in}{5.726298in}}%
\pgfpathlineto{\pgfqpoint{3.545637in}{5.733193in}}%
\pgfpathlineto{\pgfqpoint{3.575210in}{5.739926in}}%
\pgfpathlineto{\pgfqpoint{3.596873in}{5.744720in}}%
\pgfpathlineto{\pgfqpoint{3.604783in}{5.746446in}}%
\pgfpathlineto{\pgfqpoint{3.634357in}{5.752674in}}%
\pgfpathlineto{\pgfqpoint{3.663930in}{5.758727in}}%
\pgfpathlineto{\pgfqpoint{3.693504in}{5.764600in}}%
\pgfpathlineto{\pgfqpoint{3.723077in}{5.770287in}}%
\pgfpathlineto{\pgfqpoint{3.726842in}{5.770979in}}%
\pgfpathlineto{\pgfqpoint{3.752650in}{5.775655in}}%
\pgfpathlineto{\pgfqpoint{3.782224in}{5.780809in}}%
\pgfpathlineto{\pgfqpoint{3.811797in}{5.785762in}}%
\pgfpathlineto{\pgfqpoint{3.841370in}{5.790506in}}%
\pgfpathlineto{\pgfqpoint{3.870944in}{5.795035in}}%
\pgfpathlineto{\pgfqpoint{3.886176in}{5.797238in}}%
\pgfusepath{stroke}%
\end{pgfscope}%
\begin{pgfscope}%
\pgfpathrectangle{\pgfqpoint{0.854460in}{0.571603in}}{\pgfqpoint{5.885100in}{5.225635in}}%
\pgfusepath{clip}%
\pgfsetbuttcap%
\pgfsetroundjoin%
\pgfsetlinewidth{1.505625pt}%
\definecolor{currentstroke}{rgb}{0.179019,0.433756,0.557430}%
\pgfsetstrokecolor{currentstroke}%
\pgfsetdash{}{0pt}%
\pgfpathmoveto{\pgfqpoint{4.649720in}{5.797238in}}%
\pgfpathlineto{\pgfqpoint{4.669425in}{5.792848in}}%
\pgfpathlineto{\pgfqpoint{4.698998in}{5.785523in}}%
\pgfpathlineto{\pgfqpoint{4.728571in}{5.777368in}}%
\pgfpathlineto{\pgfqpoint{4.749608in}{5.770979in}}%
\pgfpathlineto{\pgfqpoint{4.758145in}{5.768222in}}%
\pgfpathlineto{\pgfqpoint{4.787718in}{5.757827in}}%
\pgfpathlineto{\pgfqpoint{4.817291in}{5.746368in}}%
\pgfpathlineto{\pgfqpoint{4.821242in}{5.744720in}}%
\pgfpathlineto{\pgfqpoint{4.846865in}{5.733291in}}%
\pgfpathlineto{\pgfqpoint{4.876438in}{5.718842in}}%
\pgfpathlineto{\pgfqpoint{4.877169in}{5.718460in}}%
\pgfpathlineto{\pgfqpoint{4.906012in}{5.702274in}}%
\pgfpathlineto{\pgfqpoint{4.922579in}{5.692201in}}%
\pgfpathlineto{\pgfqpoint{4.935585in}{5.683676in}}%
\pgfpathlineto{\pgfqpoint{4.960719in}{5.665941in}}%
\pgfpathlineto{\pgfqpoint{4.965158in}{5.662551in}}%
\pgfpathlineto{\pgfqpoint{4.993131in}{5.639682in}}%
\pgfpathlineto{\pgfqpoint{4.994732in}{5.638259in}}%
\pgfpathlineto{\pgfqpoint{5.020957in}{5.613422in}}%
\pgfpathlineto{\pgfqpoint{5.024305in}{5.609958in}}%
\pgfpathlineto{\pgfqpoint{5.045064in}{5.587163in}}%
\pgfpathlineto{\pgfqpoint{5.053878in}{5.576525in}}%
\pgfpathlineto{\pgfqpoint{5.066120in}{5.560903in}}%
\pgfpathlineto{\pgfqpoint{5.083452in}{5.536426in}}%
\pgfpathlineto{\pgfqpoint{5.084649in}{5.534644in}}%
\pgfpathlineto{\pgfqpoint{5.100776in}{5.508384in}}%
\pgfpathlineto{\pgfqpoint{5.113025in}{5.486066in}}%
\pgfpathlineto{\pgfqpoint{5.115087in}{5.482125in}}%
\pgfpathlineto{\pgfqpoint{5.127560in}{5.455865in}}%
\pgfpathlineto{\pgfqpoint{5.138628in}{5.429606in}}%
\pgfpathlineto{\pgfqpoint{5.142599in}{5.419096in}}%
\pgfpathlineto{\pgfqpoint{5.148298in}{5.403346in}}%
\pgfpathlineto{\pgfqpoint{5.156753in}{5.377087in}}%
\pgfpathlineto{\pgfqpoint{5.164179in}{5.350827in}}%
\pgfpathlineto{\pgfqpoint{5.170667in}{5.324568in}}%
\pgfpathlineto{\pgfqpoint{5.172172in}{5.317643in}}%
\pgfpathlineto{\pgfqpoint{5.176218in}{5.298308in}}%
\pgfpathlineto{\pgfqpoint{5.180976in}{5.272049in}}%
\pgfpathlineto{\pgfqpoint{5.185036in}{5.245790in}}%
\pgfpathlineto{\pgfqpoint{5.188457in}{5.219530in}}%
\pgfpathlineto{\pgfqpoint{5.191294in}{5.193271in}}%
\pgfpathlineto{\pgfqpoint{5.193598in}{5.167011in}}%
\pgfpathlineto{\pgfqpoint{5.195416in}{5.140752in}}%
\pgfpathlineto{\pgfqpoint{5.196789in}{5.114492in}}%
\pgfpathlineto{\pgfqpoint{5.197758in}{5.088233in}}%
\pgfpathlineto{\pgfqpoint{5.198358in}{5.061973in}}%
\pgfpathlineto{\pgfqpoint{5.198625in}{5.035714in}}%
\pgfpathlineto{\pgfqpoint{5.198588in}{5.009454in}}%
\pgfpathlineto{\pgfqpoint{5.198279in}{4.983195in}}%
\pgfpathlineto{\pgfqpoint{5.197722in}{4.956935in}}%
\pgfpathlineto{\pgfqpoint{5.196944in}{4.930676in}}%
\pgfpathlineto{\pgfqpoint{5.195969in}{4.904416in}}%
\pgfpathlineto{\pgfqpoint{5.195355in}{4.890407in}}%
\pgfusepath{stroke}%
\end{pgfscope}%
\begin{pgfscope}%
\pgfpathrectangle{\pgfqpoint{0.854460in}{0.571603in}}{\pgfqpoint{5.885100in}{5.225635in}}%
\pgfusepath{clip}%
\pgfsetbuttcap%
\pgfsetroundjoin%
\pgfsetlinewidth{1.505625pt}%
\definecolor{currentstroke}{rgb}{0.179019,0.433756,0.557430}%
\pgfsetstrokecolor{currentstroke}%
\pgfsetdash{}{0pt}%
\pgfpathmoveto{\pgfqpoint{5.168913in}{4.498456in}}%
\pgfpathlineto{\pgfqpoint{5.159124in}{4.352967in}}%
\pgfpathlineto{\pgfqpoint{5.152381in}{4.221670in}}%
\pgfpathlineto{\pgfqpoint{5.148959in}{4.116632in}}%
\pgfpathlineto{\pgfqpoint{5.147649in}{4.011594in}}%
\pgfpathlineto{\pgfqpoint{5.148737in}{3.906556in}}%
\pgfpathlineto{\pgfqpoint{5.152479in}{3.801519in}}%
\pgfpathlineto{\pgfqpoint{5.157166in}{3.722740in}}%
\pgfpathlineto{\pgfqpoint{5.163565in}{3.643962in}}%
\pgfpathlineto{\pgfqpoint{5.172172in}{3.561788in}}%
\pgfpathlineto{\pgfqpoint{5.181738in}{3.486405in}}%
\pgfpathlineto{\pgfqpoint{5.193651in}{3.407626in}}%
\pgfpathlineto{\pgfqpoint{5.207522in}{3.328848in}}%
\pgfpathlineto{\pgfqpoint{5.223386in}{3.250070in}}%
\pgfpathlineto{\pgfqpoint{5.241300in}{3.171291in}}%
\pgfpathlineto{\pgfqpoint{5.261341in}{3.092513in}}%
\pgfpathlineto{\pgfqpoint{5.283446in}{3.013734in}}%
\pgfpathlineto{\pgfqpoint{5.307724in}{2.934956in}}%
\pgfpathlineto{\pgfqpoint{5.334194in}{2.856177in}}%
\pgfpathlineto{\pgfqpoint{5.362875in}{2.777399in}}%
\pgfpathlineto{\pgfqpoint{5.393786in}{2.698621in}}%
\pgfpathlineto{\pgfqpoint{5.426949in}{2.619842in}}%
\pgfpathlineto{\pgfqpoint{5.467906in}{2.529343in}}%
\pgfpathlineto{\pgfqpoint{5.500105in}{2.462285in}}%
\pgfpathlineto{\pgfqpoint{5.540078in}{2.383507in}}%
\pgfpathlineto{\pgfqpoint{5.586199in}{2.297915in}}%
\pgfpathlineto{\pgfqpoint{5.626982in}{2.225950in}}%
\pgfpathlineto{\pgfqpoint{5.674919in}{2.145567in}}%
\pgfpathlineto{\pgfqpoint{5.723132in}{2.068393in}}%
\pgfpathlineto{\pgfqpoint{5.774681in}{1.989615in}}%
\pgfpathlineto{\pgfqpoint{5.828558in}{1.910836in}}%
\pgfpathlineto{\pgfqpoint{5.884748in}{1.832058in}}%
\pgfpathlineto{\pgfqpoint{5.943241in}{1.753280in}}%
\pgfpathlineto{\pgfqpoint{6.004030in}{1.674501in}}%
\pgfpathlineto{\pgfqpoint{6.067113in}{1.595723in}}%
\pgfpathlineto{\pgfqpoint{6.154802in}{1.490685in}}%
\pgfpathlineto{\pgfqpoint{6.246531in}{1.385647in}}%
\pgfpathlineto{\pgfqpoint{6.342282in}{1.280609in}}%
\pgfpathlineto{\pgfqpoint{6.442049in}{1.175571in}}%
\pgfpathlineto{\pgfqpoint{6.545710in}{1.070533in}}%
\pgfpathlineto{\pgfqpoint{6.653326in}{0.965495in}}%
\pgfpathlineto{\pgfqpoint{6.739560in}{0.883914in}}%
\pgfpathlineto{\pgfqpoint{6.739560in}{0.883914in}}%
\pgfusepath{stroke}%
\end{pgfscope}%
\begin{pgfscope}%
\pgfpathrectangle{\pgfqpoint{0.854460in}{0.571603in}}{\pgfqpoint{5.885100in}{5.225635in}}%
\pgfusepath{clip}%
\pgfsetbuttcap%
\pgfsetroundjoin%
\pgfsetlinewidth{1.505625pt}%
\definecolor{currentstroke}{rgb}{0.169646,0.456262,0.558030}%
\pgfsetstrokecolor{currentstroke}%
\pgfsetdash{}{0pt}%
\pgfpathmoveto{\pgfqpoint{1.614939in}{0.571603in}}%
\pgfpathlineto{\pgfqpoint{1.593795in}{0.590603in}}%
\pgfpathlineto{\pgfqpoint{1.585752in}{0.597863in}}%
\pgfpathlineto{\pgfqpoint{1.584965in}{0.598583in}}%
\pgfusepath{stroke}%
\end{pgfscope}%
\begin{pgfscope}%
\pgfpathrectangle{\pgfqpoint{0.854460in}{0.571603in}}{\pgfqpoint{5.885100in}{5.225635in}}%
\pgfusepath{clip}%
\pgfsetbuttcap%
\pgfsetroundjoin%
\pgfsetlinewidth{1.505625pt}%
\definecolor{currentstroke}{rgb}{0.169646,0.456262,0.558030}%
\pgfsetstrokecolor{currentstroke}%
\pgfsetdash{}{0pt}%
\pgfpathmoveto{\pgfqpoint{1.313602in}{0.870997in}}%
\pgfpathlineto{\pgfqpoint{1.299525in}{0.886717in}}%
\pgfpathlineto{\pgfqpoint{1.298061in}{0.888379in}}%
\pgfpathlineto{\pgfqpoint{1.276538in}{0.912976in}}%
\pgfpathlineto{\pgfqpoint{1.268488in}{0.922317in}}%
\pgfpathlineto{\pgfqpoint{1.254002in}{0.939236in}}%
\pgfpathlineto{\pgfqpoint{1.238914in}{0.957128in}}%
\pgfpathlineto{\pgfqpoint{1.231906in}{0.965495in}}%
\pgfpathlineto{\pgfqpoint{1.210252in}{0.991755in}}%
\pgfpathlineto{\pgfqpoint{1.209341in}{0.992880in}}%
\pgfpathlineto{\pgfqpoint{1.189128in}{1.018014in}}%
\pgfpathlineto{\pgfqpoint{1.179767in}{1.029834in}}%
\pgfpathlineto{\pgfqpoint{1.168416in}{1.044274in}}%
\pgfpathlineto{\pgfqpoint{1.150194in}{1.067814in}}%
\pgfpathlineto{\pgfqpoint{1.148105in}{1.070533in}}%
\pgfpathlineto{\pgfqpoint{1.128297in}{1.096793in}}%
\pgfpathlineto{\pgfqpoint{1.120621in}{1.107138in}}%
\pgfpathlineto{\pgfqpoint{1.108904in}{1.123052in}}%
\pgfpathlineto{\pgfqpoint{1.091047in}{1.147691in}}%
\pgfpathlineto{\pgfqpoint{1.089882in}{1.149312in}}%
\pgfpathlineto{\pgfqpoint{1.071368in}{1.175571in}}%
\pgfpathlineto{\pgfqpoint{1.061474in}{1.189837in}}%
\pgfpathlineto{\pgfqpoint{1.053224in}{1.201831in}}%
\pgfpathlineto{\pgfqpoint{1.035474in}{1.228090in}}%
\pgfpathlineto{\pgfqpoint{1.031901in}{1.233481in}}%
\pgfpathlineto{\pgfqpoint{1.018184in}{1.254350in}}%
\pgfpathlineto{\pgfqpoint{1.002327in}{1.278873in}}%
\pgfpathlineto{\pgfqpoint{1.001215in}{1.280609in}}%
\pgfpathlineto{\pgfqpoint{0.984736in}{1.306869in}}%
\pgfpathlineto{\pgfqpoint{0.972754in}{1.326292in}}%
\pgfpathlineto{\pgfqpoint{0.968574in}{1.333128in}}%
\pgfpathlineto{\pgfqpoint{0.952849in}{1.359388in}}%
\pgfpathlineto{\pgfqpoint{0.943181in}{1.375839in}}%
\pgfpathlineto{\pgfqpoint{0.937469in}{1.385647in}}%
\pgfpathlineto{\pgfqpoint{0.922493in}{1.411906in}}%
\pgfpathlineto{\pgfqpoint{0.913607in}{1.427799in}}%
\pgfpathlineto{\pgfqpoint{0.907865in}{1.438166in}}%
\pgfpathlineto{\pgfqpoint{0.893633in}{1.464425in}}%
\pgfpathlineto{\pgfqpoint{0.884034in}{1.482498in}}%
\pgfpathlineto{\pgfqpoint{0.879728in}{1.490685in}}%
\pgfpathlineto{\pgfqpoint{0.866232in}{1.516944in}}%
\pgfpathlineto{\pgfqpoint{0.854460in}{1.540315in}}%
\pgfusepath{stroke}%
\end{pgfscope}%
\begin{pgfscope}%
\pgfpathrectangle{\pgfqpoint{0.854460in}{0.571603in}}{\pgfqpoint{5.885100in}{5.225635in}}%
\pgfusepath{clip}%
\pgfsetbuttcap%
\pgfsetroundjoin%
\pgfsetlinewidth{1.505625pt}%
\definecolor{currentstroke}{rgb}{0.169646,0.456262,0.558030}%
\pgfsetstrokecolor{currentstroke}%
\pgfsetdash{}{0pt}%
\pgfpathmoveto{\pgfqpoint{0.854460in}{3.911451in}}%
\pgfpathlineto{\pgfqpoint{0.865051in}{3.932816in}}%
\pgfpathlineto{\pgfqpoint{0.878320in}{3.959075in}}%
\pgfpathlineto{\pgfqpoint{0.884034in}{3.970159in}}%
\pgfpathlineto{\pgfqpoint{0.891977in}{3.985335in}}%
\pgfpathlineto{\pgfqpoint{0.905985in}{4.011594in}}%
\pgfpathlineto{\pgfqpoint{0.913607in}{4.025618in}}%
\pgfpathlineto{\pgfqpoint{0.920358in}{4.037854in}}%
\pgfpathlineto{\pgfqpoint{0.935116in}{4.064113in}}%
\pgfpathlineto{\pgfqpoint{0.943181in}{4.078209in}}%
\pgfpathlineto{\pgfqpoint{0.950243in}{4.090373in}}%
\pgfpathlineto{\pgfqpoint{0.965761in}{4.116632in}}%
\pgfpathlineto{\pgfqpoint{0.972754in}{4.128261in}}%
\pgfpathlineto{\pgfqpoint{0.981681in}{4.142892in}}%
\pgfpathlineto{\pgfqpoint{0.997969in}{4.169151in}}%
\pgfpathlineto{\pgfqpoint{1.002327in}{4.176057in}}%
\pgfpathlineto{\pgfqpoint{1.014719in}{4.195411in}}%
\pgfpathlineto{\pgfqpoint{1.031784in}{4.221670in}}%
\pgfpathlineto{\pgfqpoint{1.031901in}{4.221846in}}%
\pgfpathlineto{\pgfqpoint{1.049403in}{4.247930in}}%
\pgfpathlineto{\pgfqpoint{1.061474in}{4.265662in}}%
\pgfpathlineto{\pgfqpoint{1.067362in}{4.274189in}}%
\pgfpathlineto{\pgfqpoint{1.085777in}{4.300449in}}%
\pgfpathlineto{\pgfqpoint{1.091047in}{4.307848in}}%
\pgfpathlineto{\pgfqpoint{1.104669in}{4.326708in}}%
\pgfpathlineto{\pgfqpoint{1.120621in}{4.348489in}}%
\pgfpathlineto{\pgfqpoint{1.123946in}{4.352967in}}%
\pgfpathlineto{\pgfqpoint{1.143747in}{4.379227in}}%
\pgfpathlineto{\pgfqpoint{1.150194in}{4.387654in}}%
\pgfpathlineto{\pgfqpoint{1.164022in}{4.405486in}}%
\pgfpathlineto{\pgfqpoint{1.179767in}{4.425521in}}%
\pgfpathlineto{\pgfqpoint{1.184726in}{4.431746in}}%
\pgfpathlineto{\pgfqpoint{1.205942in}{4.458005in}}%
\pgfpathlineto{\pgfqpoint{1.209341in}{4.462152in}}%
\pgfpathlineto{\pgfqpoint{1.227708in}{4.484265in}}%
\pgfpathlineto{\pgfqpoint{1.238914in}{4.497582in}}%
\pgfpathlineto{\pgfqpoint{1.249948in}{4.510524in}}%
\pgfpathlineto{\pgfqpoint{1.268488in}{4.531991in}}%
\pgfpathlineto{\pgfqpoint{1.272681in}{4.536784in}}%
\pgfpathlineto{\pgfqpoint{1.295965in}{4.563043in}}%
\pgfpathlineto{\pgfqpoint{1.298061in}{4.565374in}}%
\pgfpathlineto{\pgfqpoint{1.319848in}{4.589303in}}%
\pgfpathlineto{\pgfqpoint{1.327634in}{4.597746in}}%
\pgfpathlineto{\pgfqpoint{1.344270in}{4.615562in}}%
\pgfpathlineto{\pgfqpoint{1.357208in}{4.629245in}}%
\pgfpathlineto{\pgfqpoint{1.369248in}{4.641822in}}%
\pgfpathlineto{\pgfqpoint{1.386781in}{4.659909in}}%
\pgfpathlineto{\pgfqpoint{1.394801in}{4.668081in}}%
\pgfpathlineto{\pgfqpoint{1.416354in}{4.689773in}}%
\pgfpathlineto{\pgfqpoint{1.420948in}{4.694341in}}%
\pgfpathlineto{\pgfqpoint{1.445928in}{4.718873in}}%
\pgfpathlineto{\pgfqpoint{1.447707in}{4.720600in}}%
\pgfpathlineto{\pgfqpoint{1.475106in}{4.746860in}}%
\pgfpathlineto{\pgfqpoint{1.475501in}{4.747234in}}%
\pgfpathlineto{\pgfqpoint{1.503177in}{4.773119in}}%
\pgfpathlineto{\pgfqpoint{1.505074in}{4.774872in}}%
\pgfpathlineto{\pgfqpoint{1.531904in}{4.799378in}}%
\pgfpathlineto{\pgfqpoint{1.534648in}{4.801854in}}%
\pgfpathlineto{\pgfqpoint{1.561307in}{4.825638in}}%
\pgfpathlineto{\pgfqpoint{1.564221in}{4.828207in}}%
\pgfpathlineto{\pgfqpoint{1.591403in}{4.851897in}}%
\pgfpathlineto{\pgfqpoint{1.593795in}{4.853957in}}%
\pgfpathlineto{\pgfqpoint{1.622210in}{4.878157in}}%
\pgfpathlineto{\pgfqpoint{1.623368in}{4.879131in}}%
\pgfpathlineto{\pgfqpoint{1.652941in}{4.903740in}}%
\pgfpathlineto{\pgfqpoint{1.653763in}{4.904416in}}%
\pgfpathlineto{\pgfqpoint{1.682515in}{4.927790in}}%
\pgfpathlineto{\pgfqpoint{1.686103in}{4.930676in}}%
\pgfpathlineto{\pgfqpoint{1.712088in}{4.951325in}}%
\pgfpathlineto{\pgfqpoint{1.719223in}{4.956935in}}%
\pgfpathlineto{\pgfqpoint{1.741661in}{4.974367in}}%
\pgfpathlineto{\pgfqpoint{1.753143in}{4.983195in}}%
\pgfpathlineto{\pgfqpoint{1.771235in}{4.996938in}}%
\pgfpathlineto{\pgfqpoint{1.787880in}{5.009454in}}%
\pgfpathlineto{\pgfqpoint{1.800808in}{5.019060in}}%
\pgfpathlineto{\pgfqpoint{1.823450in}{5.035714in}}%
\pgfpathlineto{\pgfqpoint{1.830381in}{5.040751in}}%
\pgfpathlineto{\pgfqpoint{1.859871in}{5.061973in}}%
\pgfpathlineto{\pgfqpoint{1.859955in}{5.062033in}}%
\pgfpathlineto{\pgfqpoint{1.889528in}{5.082812in}}%
\pgfpathlineto{\pgfqpoint{1.897319in}{5.088233in}}%
\pgfpathlineto{\pgfqpoint{1.919102in}{5.103209in}}%
\pgfpathlineto{\pgfqpoint{1.935670in}{5.114492in}}%
\pgfpathlineto{\pgfqpoint{1.948675in}{5.123243in}}%
\pgfpathlineto{\pgfqpoint{1.974938in}{5.140752in}}%
\pgfpathlineto{\pgfqpoint{1.978248in}{5.142932in}}%
\pgfpathlineto{\pgfqpoint{2.007822in}{5.162193in}}%
\pgfpathlineto{\pgfqpoint{2.015296in}{5.167011in}}%
\pgfpathlineto{\pgfqpoint{2.037395in}{5.181088in}}%
\pgfpathlineto{\pgfqpoint{2.056691in}{5.193271in}}%
\pgfpathlineto{\pgfqpoint{2.066968in}{5.199681in}}%
\pgfpathlineto{\pgfqpoint{2.096542in}{5.217955in}}%
\pgfpathlineto{\pgfqpoint{2.099122in}{5.219530in}}%
\pgfpathlineto{\pgfqpoint{2.126115in}{5.235814in}}%
\pgfpathlineto{\pgfqpoint{2.142793in}{5.245790in}}%
\pgfpathlineto{\pgfqpoint{2.155689in}{5.253410in}}%
\pgfpathlineto{\pgfqpoint{2.185262in}{5.270730in}}%
\pgfpathlineto{\pgfqpoint{2.187542in}{5.272049in}}%
\pgfpathlineto{\pgfqpoint{2.214835in}{5.287647in}}%
\pgfpathlineto{\pgfqpoint{2.224866in}{5.293334in}}%
\pgfusepath{stroke}%
\end{pgfscope}%
\begin{pgfscope}%
\pgfpathrectangle{\pgfqpoint{0.854460in}{0.571603in}}{\pgfqpoint{5.885100in}{5.225635in}}%
\pgfusepath{clip}%
\pgfsetbuttcap%
\pgfsetroundjoin%
\pgfsetlinewidth{1.505625pt}%
\definecolor{currentstroke}{rgb}{0.169646,0.456262,0.558030}%
\pgfsetstrokecolor{currentstroke}%
\pgfsetdash{}{0pt}%
\pgfpathmoveto{\pgfqpoint{2.562937in}{5.466735in}}%
\pgfpathlineto{\pgfqpoint{2.569716in}{5.469882in}}%
\pgfpathlineto{\pgfqpoint{2.596260in}{5.482125in}}%
\pgfpathlineto{\pgfqpoint{2.599289in}{5.483504in}}%
\pgfpathlineto{\pgfqpoint{2.628862in}{5.496783in}}%
\pgfpathlineto{\pgfqpoint{2.654882in}{5.508384in}}%
\pgfpathlineto{\pgfqpoint{2.658436in}{5.509949in}}%
\pgfpathlineto{\pgfqpoint{2.688009in}{5.522782in}}%
\pgfpathlineto{\pgfqpoint{2.715535in}{5.534644in}}%
\pgfpathlineto{\pgfqpoint{2.717582in}{5.535515in}}%
\pgfpathlineto{\pgfqpoint{2.747156in}{5.547905in}}%
\pgfpathlineto{\pgfqpoint{2.776729in}{5.560214in}}%
\pgfpathlineto{\pgfqpoint{2.778412in}{5.560903in}}%
\pgfpathlineto{\pgfqpoint{2.806303in}{5.572182in}}%
\pgfpathlineto{\pgfqpoint{2.835876in}{5.584049in}}%
\pgfpathlineto{\pgfqpoint{2.843749in}{5.587163in}}%
\pgfpathlineto{\pgfqpoint{2.865449in}{5.595636in}}%
\pgfpathlineto{\pgfqpoint{2.895023in}{5.607065in}}%
\pgfpathlineto{\pgfqpoint{2.911673in}{5.613422in}}%
\pgfpathlineto{\pgfqpoint{2.924596in}{5.618293in}}%
\pgfpathlineto{\pgfqpoint{2.954169in}{5.629287in}}%
\pgfpathlineto{\pgfqpoint{2.982379in}{5.639682in}}%
\pgfpathlineto{\pgfqpoint{2.983743in}{5.640178in}}%
\pgfpathlineto{\pgfqpoint{3.013316in}{5.650739in}}%
\pgfpathlineto{\pgfqpoint{3.042890in}{5.661205in}}%
\pgfpathlineto{\pgfqpoint{3.056473in}{5.665941in}}%
\pgfpathlineto{\pgfqpoint{3.072463in}{5.671444in}}%
\pgfpathlineto{\pgfqpoint{3.102036in}{5.681476in}}%
\pgfpathlineto{\pgfqpoint{3.131610in}{5.691408in}}%
\pgfpathlineto{\pgfqpoint{3.134017in}{5.692201in}}%
\pgfpathlineto{\pgfqpoint{3.161183in}{5.701026in}}%
\pgfpathlineto{\pgfqpoint{3.190756in}{5.710521in}}%
\pgfpathlineto{\pgfqpoint{3.215801in}{5.718460in}}%
\pgfpathlineto{\pgfqpoint{3.220330in}{5.719876in}}%
\pgfpathlineto{\pgfqpoint{3.249903in}{5.728937in}}%
\pgfpathlineto{\pgfqpoint{3.279476in}{5.737887in}}%
\pgfpathlineto{\pgfqpoint{3.302396in}{5.744720in}}%
\pgfpathlineto{\pgfqpoint{3.309050in}{5.746677in}}%
\pgfpathlineto{\pgfqpoint{3.338623in}{5.755188in}}%
\pgfpathlineto{\pgfqpoint{3.368197in}{5.763583in}}%
\pgfpathlineto{\pgfqpoint{3.394661in}{5.770979in}}%
\pgfpathlineto{\pgfqpoint{3.397770in}{5.771836in}}%
\pgfpathlineto{\pgfqpoint{3.427343in}{5.779788in}}%
\pgfpathlineto{\pgfqpoint{3.456917in}{5.787615in}}%
\pgfpathlineto{\pgfqpoint{3.486490in}{5.795314in}}%
\pgfpathlineto{\pgfqpoint{3.494076in}{5.797238in}}%
\pgfusepath{stroke}%
\end{pgfscope}%
\begin{pgfscope}%
\pgfpathrectangle{\pgfqpoint{0.854460in}{0.571603in}}{\pgfqpoint{5.885100in}{5.225635in}}%
\pgfusepath{clip}%
\pgfsetbuttcap%
\pgfsetroundjoin%
\pgfsetlinewidth{1.505625pt}%
\definecolor{currentstroke}{rgb}{0.169646,0.456262,0.558030}%
\pgfsetstrokecolor{currentstroke}%
\pgfsetdash{}{0pt}%
\pgfpathmoveto{\pgfqpoint{5.057130in}{5.797238in}}%
\pgfpathlineto{\pgfqpoint{5.097244in}{5.770979in}}%
\pgfpathlineto{\pgfqpoint{5.131244in}{5.744720in}}%
\pgfpathlineto{\pgfqpoint{5.160345in}{5.718460in}}%
\pgfpathlineto{\pgfqpoint{5.185468in}{5.692201in}}%
\pgfpathlineto{\pgfqpoint{5.207326in}{5.665941in}}%
\pgfpathlineto{\pgfqpoint{5.231319in}{5.632101in}}%
\pgfpathlineto{\pgfqpoint{5.242946in}{5.613422in}}%
\pgfpathlineto{\pgfqpoint{5.260892in}{5.580461in}}%
\pgfpathlineto{\pgfqpoint{5.270228in}{5.560903in}}%
\pgfpathlineto{\pgfqpoint{5.281351in}{5.534644in}}%
\pgfpathlineto{\pgfqpoint{5.291106in}{5.508384in}}%
\pgfpathlineto{\pgfqpoint{5.306748in}{5.455865in}}%
\pgfpathlineto{\pgfqpoint{5.318399in}{5.403346in}}%
\pgfpathlineto{\pgfqpoint{5.326602in}{5.350827in}}%
\pgfpathlineto{\pgfqpoint{5.332056in}{5.298308in}}%
\pgfpathlineto{\pgfqpoint{5.335238in}{5.245790in}}%
\pgfpathlineto{\pgfqpoint{5.336523in}{5.193271in}}%
\pgfpathlineto{\pgfqpoint{5.335585in}{5.114492in}}%
\pgfpathlineto{\pgfqpoint{5.331986in}{5.035714in}}%
\pgfpathlineto{\pgfqpoint{5.324268in}{4.930676in}}%
\pgfpathlineto{\pgfqpoint{5.311646in}{4.799378in}}%
\pgfpathlineto{\pgfqpoint{5.268436in}{4.379227in}}%
\pgfpathlineto{\pgfqpoint{5.258302in}{4.247930in}}%
\pgfpathlineto{\pgfqpoint{5.252307in}{4.142892in}}%
\pgfpathlineto{\pgfqpoint{5.248580in}{4.037854in}}%
\pgfpathlineto{\pgfqpoint{5.247385in}{3.932816in}}%
\pgfpathlineto{\pgfqpoint{5.248952in}{3.827778in}}%
\pgfpathlineto{\pgfqpoint{5.252069in}{3.749000in}}%
\pgfpathlineto{\pgfqpoint{5.256942in}{3.670221in}}%
\pgfpathlineto{\pgfqpoint{5.263623in}{3.591443in}}%
\pgfpathlineto{\pgfqpoint{5.272155in}{3.512664in}}%
\pgfpathlineto{\pgfqpoint{5.282655in}{3.433886in}}%
\pgfpathlineto{\pgfqpoint{5.295149in}{3.355107in}}%
\pgfpathlineto{\pgfqpoint{5.309656in}{3.276329in}}%
\pgfpathlineto{\pgfqpoint{5.326267in}{3.197551in}}%
\pgfpathlineto{\pgfqpoint{5.344993in}{3.118772in}}%
\pgfpathlineto{\pgfqpoint{5.365848in}{3.039994in}}%
\pgfpathlineto{\pgfqpoint{5.388909in}{2.961215in}}%
\pgfpathlineto{\pgfqpoint{5.414187in}{2.882437in}}%
\pgfpathlineto{\pgfqpoint{5.441698in}{2.803659in}}%
\pgfpathlineto{\pgfqpoint{5.471459in}{2.724880in}}%
\pgfpathlineto{\pgfqpoint{5.503487in}{2.646102in}}%
\pgfpathlineto{\pgfqpoint{5.537806in}{2.567323in}}%
\pgfpathlineto{\pgfqpoint{5.574441in}{2.488545in}}%
\pgfpathlineto{\pgfqpoint{5.615772in}{2.405202in}}%
\pgfpathlineto{\pgfqpoint{5.654677in}{2.330988in}}%
\pgfpathlineto{\pgfqpoint{5.704492in}{2.241439in}}%
\pgfpathlineto{\pgfqpoint{5.744244in}{2.173431in}}%
\pgfpathlineto{\pgfqpoint{5.793213in}{2.093642in}}%
\pgfpathlineto{\pgfqpoint{5.852359in}{2.002065in}}%
\pgfpathlineto{\pgfqpoint{5.896172in}{1.937096in}}%
\pgfpathlineto{\pgfqpoint{5.951509in}{1.858318in}}%
\pgfpathlineto{\pgfqpoint{6.009188in}{1.779539in}}%
\pgfpathlineto{\pgfqpoint{6.069200in}{1.700761in}}%
\pgfpathlineto{\pgfqpoint{6.148093in}{1.601602in}}%
\pgfpathlineto{\pgfqpoint{6.207240in}{1.530100in}}%
\pgfpathlineto{\pgfqpoint{6.266386in}{1.460787in}}%
\pgfpathlineto{\pgfqpoint{6.347093in}{1.369430in}}%
\pgfpathlineto{\pgfqpoint{6.347093in}{1.369430in}}%
\pgfusepath{stroke}%
\end{pgfscope}%
\begin{pgfscope}%
\pgfpathrectangle{\pgfqpoint{0.854460in}{0.571603in}}{\pgfqpoint{5.885100in}{5.225635in}}%
\pgfusepath{clip}%
\pgfsetbuttcap%
\pgfsetroundjoin%
\pgfsetlinewidth{1.505625pt}%
\definecolor{currentstroke}{rgb}{0.169646,0.456262,0.558030}%
\pgfsetstrokecolor{currentstroke}%
\pgfsetdash{}{0pt}%
\pgfpathmoveto{\pgfqpoint{6.612442in}{1.090477in}}%
\pgfpathlineto{\pgfqpoint{6.621267in}{1.081693in}}%
\pgfpathlineto{\pgfqpoint{6.632478in}{1.070533in}}%
\pgfpathlineto{\pgfqpoint{6.650840in}{1.052428in}}%
\pgfpathlineto{\pgfqpoint{6.659110in}{1.044274in}}%
\pgfpathlineto{\pgfqpoint{6.680414in}{1.023463in}}%
\pgfpathlineto{\pgfqpoint{6.685992in}{1.018014in}}%
\pgfpathlineto{\pgfqpoint{6.709987in}{0.994789in}}%
\pgfpathlineto{\pgfqpoint{6.713123in}{0.991755in}}%
\pgfpathlineto{\pgfqpoint{6.739560in}{0.966396in}}%
\pgfusepath{stroke}%
\end{pgfscope}%
\begin{pgfscope}%
\pgfpathrectangle{\pgfqpoint{0.854460in}{0.571603in}}{\pgfqpoint{5.885100in}{5.225635in}}%
\pgfusepath{clip}%
\pgfsetbuttcap%
\pgfsetroundjoin%
\pgfsetlinewidth{1.505625pt}%
\definecolor{currentstroke}{rgb}{0.162142,0.474838,0.558140}%
\pgfsetstrokecolor{currentstroke}%
\pgfsetdash{}{0pt}%
\pgfpathmoveto{\pgfqpoint{1.543857in}{0.571603in}}%
\pgfpathlineto{\pgfqpoint{1.534648in}{0.579950in}}%
\pgfpathlineto{\pgfqpoint{1.517055in}{0.595968in}}%
\pgfusepath{stroke}%
\end{pgfscope}%
\begin{pgfscope}%
\pgfpathrectangle{\pgfqpoint{0.854460in}{0.571603in}}{\pgfqpoint{5.885100in}{5.225635in}}%
\pgfusepath{clip}%
\pgfsetbuttcap%
\pgfsetroundjoin%
\pgfsetlinewidth{1.505625pt}%
\definecolor{currentstroke}{rgb}{0.162142,0.474838,0.558140}%
\pgfsetstrokecolor{currentstroke}%
\pgfsetdash{}{0pt}%
\pgfpathmoveto{\pgfqpoint{1.246806in}{0.869582in}}%
\pgfpathlineto{\pgfqpoint{1.238914in}{0.878465in}}%
\pgfpathlineto{\pgfqpoint{1.231629in}{0.886717in}}%
\pgfpathlineto{\pgfqpoint{1.209341in}{0.912343in}}%
\pgfpathlineto{\pgfqpoint{1.208794in}{0.912976in}}%
\pgfpathlineto{\pgfqpoint{1.186494in}{0.939236in}}%
\pgfpathlineto{\pgfqpoint{1.179767in}{0.947279in}}%
\pgfpathlineto{\pgfqpoint{1.164638in}{0.965495in}}%
\pgfpathlineto{\pgfqpoint{1.150194in}{0.983151in}}%
\pgfpathlineto{\pgfqpoint{1.143204in}{0.991755in}}%
\pgfpathlineto{\pgfqpoint{1.122206in}{1.018014in}}%
\pgfpathlineto{\pgfqpoint{1.120621in}{1.020033in}}%
\pgfpathlineto{\pgfqpoint{1.101719in}{1.044274in}}%
\pgfpathlineto{\pgfqpoint{1.091047in}{1.058172in}}%
\pgfpathlineto{\pgfqpoint{1.081627in}{1.070533in}}%
\pgfpathlineto{\pgfqpoint{1.061925in}{1.096793in}}%
\pgfpathlineto{\pgfqpoint{1.061474in}{1.097405in}}%
\pgfpathlineto{\pgfqpoint{1.042742in}{1.123052in}}%
\pgfpathlineto{\pgfqpoint{1.031901in}{1.138129in}}%
\pgfpathlineto{\pgfqpoint{1.023924in}{1.149312in}}%
\pgfpathlineto{\pgfqpoint{1.005507in}{1.175571in}}%
\pgfpathlineto{\pgfqpoint{1.002327in}{1.180192in}}%
\pgfpathlineto{\pgfqpoint{0.987558in}{1.201831in}}%
\pgfpathlineto{\pgfqpoint{0.972754in}{1.223871in}}%
\pgfpathlineto{\pgfqpoint{0.969944in}{1.228090in}}%
\pgfpathlineto{\pgfqpoint{0.952796in}{1.254350in}}%
\pgfpathlineto{\pgfqpoint{0.943181in}{1.269332in}}%
\pgfpathlineto{\pgfqpoint{0.936006in}{1.280609in}}%
\pgfpathlineto{\pgfqpoint{0.919610in}{1.306869in}}%
\pgfpathlineto{\pgfqpoint{0.913607in}{1.316671in}}%
\pgfpathlineto{\pgfqpoint{0.903619in}{1.333128in}}%
\pgfpathlineto{\pgfqpoint{0.887971in}{1.359388in}}%
\pgfpathlineto{\pgfqpoint{0.884034in}{1.366134in}}%
\pgfpathlineto{\pgfqpoint{0.872751in}{1.385647in}}%
\pgfpathlineto{\pgfqpoint{0.857846in}{1.411906in}}%
\pgfpathlineto{\pgfqpoint{0.854460in}{1.418003in}}%
\pgfusepath{stroke}%
\end{pgfscope}%
\begin{pgfscope}%
\pgfpathrectangle{\pgfqpoint{0.854460in}{0.571603in}}{\pgfqpoint{5.885100in}{5.225635in}}%
\pgfusepath{clip}%
\pgfsetbuttcap%
\pgfsetroundjoin%
\pgfsetlinewidth{1.505625pt}%
\definecolor{currentstroke}{rgb}{0.162142,0.474838,0.558140}%
\pgfsetstrokecolor{currentstroke}%
\pgfsetdash{}{0pt}%
\pgfpathmoveto{\pgfqpoint{0.854460in}{4.056693in}}%
\pgfpathlineto{\pgfqpoint{0.858544in}{4.064113in}}%
\pgfpathlineto{\pgfqpoint{0.873273in}{4.090373in}}%
\pgfpathlineto{\pgfqpoint{0.884034in}{4.109232in}}%
\pgfpathlineto{\pgfqpoint{0.888318in}{4.116632in}}%
\pgfpathlineto{\pgfqpoint{0.903798in}{4.142892in}}%
\pgfpathlineto{\pgfqpoint{0.913607in}{4.159256in}}%
\pgfpathlineto{\pgfqpoint{0.919623in}{4.169151in}}%
\pgfpathlineto{\pgfqpoint{0.935865in}{4.195411in}}%
\pgfpathlineto{\pgfqpoint{0.943181in}{4.207044in}}%
\pgfpathlineto{\pgfqpoint{0.952508in}{4.221670in}}%
\pgfpathlineto{\pgfqpoint{0.969518in}{4.247930in}}%
\pgfpathlineto{\pgfqpoint{0.972754in}{4.252841in}}%
\pgfpathlineto{\pgfqpoint{0.987016in}{4.274189in}}%
\pgfpathlineto{\pgfqpoint{1.002327in}{4.296781in}}%
\pgfpathlineto{\pgfqpoint{1.004848in}{4.300449in}}%
\pgfpathlineto{\pgfqpoint{1.023191in}{4.326708in}}%
\pgfpathlineto{\pgfqpoint{1.031901in}{4.338994in}}%
\pgfpathlineto{\pgfqpoint{1.041941in}{4.352967in}}%
\pgfpathlineto{\pgfqpoint{1.061075in}{4.379227in}}%
\pgfpathlineto{\pgfqpoint{1.061474in}{4.379766in}}%
\pgfpathlineto{\pgfqpoint{1.080778in}{4.405486in}}%
\pgfpathlineto{\pgfqpoint{1.091047in}{4.418984in}}%
\pgfpathlineto{\pgfqpoint{1.100886in}{4.431746in}}%
\pgfpathlineto{\pgfqpoint{1.120621in}{4.457004in}}%
\pgfpathlineto{\pgfqpoint{1.121413in}{4.458005in}}%
\pgfpathlineto{\pgfqpoint{1.142522in}{4.484265in}}%
\pgfpathlineto{\pgfqpoint{1.150194in}{4.493682in}}%
\pgfpathlineto{\pgfqpoint{1.164092in}{4.510524in}}%
\pgfpathlineto{\pgfqpoint{1.179767in}{4.529274in}}%
\pgfpathlineto{\pgfqpoint{1.186126in}{4.536784in}}%
\pgfpathlineto{\pgfqpoint{1.208654in}{4.563043in}}%
\pgfpathlineto{\pgfqpoint{1.209341in}{4.563832in}}%
\pgfpathlineto{\pgfqpoint{1.231790in}{4.589303in}}%
\pgfpathlineto{\pgfqpoint{1.238914in}{4.597282in}}%
\pgfpathlineto{\pgfqpoint{1.255434in}{4.615562in}}%
\pgfpathlineto{\pgfqpoint{1.268488in}{4.629823in}}%
\pgfpathlineto{\pgfqpoint{1.279604in}{4.641822in}}%
\pgfpathlineto{\pgfqpoint{1.298061in}{4.661493in}}%
\pgfpathlineto{\pgfqpoint{1.304317in}{4.668081in}}%
\pgfpathlineto{\pgfqpoint{1.327634in}{4.692330in}}%
\pgfpathlineto{\pgfqpoint{1.329590in}{4.694341in}}%
\pgfpathlineto{\pgfqpoint{1.355476in}{4.720600in}}%
\pgfpathlineto{\pgfqpoint{1.357208in}{4.722334in}}%
\pgfpathlineto{\pgfqpoint{1.381984in}{4.746860in}}%
\pgfpathlineto{\pgfqpoint{1.386781in}{4.751549in}}%
\pgfpathlineto{\pgfqpoint{1.409096in}{4.773119in}}%
\pgfpathlineto{\pgfqpoint{1.416354in}{4.780049in}}%
\pgfpathlineto{\pgfqpoint{1.436830in}{4.799378in}}%
\pgfpathlineto{\pgfqpoint{1.445928in}{4.807862in}}%
\pgfpathlineto{\pgfqpoint{1.465203in}{4.825638in}}%
\pgfpathlineto{\pgfqpoint{1.475501in}{4.835020in}}%
\pgfpathlineto{\pgfqpoint{1.494233in}{4.851897in}}%
\pgfpathlineto{\pgfqpoint{1.505074in}{4.861548in}}%
\pgfpathlineto{\pgfqpoint{1.523937in}{4.878157in}}%
\pgfpathlineto{\pgfqpoint{1.534648in}{4.887475in}}%
\pgfpathlineto{\pgfqpoint{1.554332in}{4.904416in}}%
\pgfpathlineto{\pgfqpoint{1.564221in}{4.912825in}}%
\pgfpathlineto{\pgfqpoint{1.585437in}{4.930676in}}%
\pgfpathlineto{\pgfqpoint{1.593795in}{4.937623in}}%
\pgfpathlineto{\pgfqpoint{1.617268in}{4.956935in}}%
\pgfpathlineto{\pgfqpoint{1.623368in}{4.961893in}}%
\pgfpathlineto{\pgfqpoint{1.649842in}{4.983195in}}%
\pgfpathlineto{\pgfqpoint{1.652941in}{4.985658in}}%
\pgfpathlineto{\pgfqpoint{1.682515in}{5.008929in}}%
\pgfpathlineto{\pgfqpoint{1.683189in}{5.009454in}}%
\pgfpathlineto{\pgfqpoint{1.712088in}{5.031678in}}%
\pgfpathlineto{\pgfqpoint{1.717389in}{5.035714in}}%
\pgfpathlineto{\pgfqpoint{1.741661in}{5.053975in}}%
\pgfpathlineto{\pgfqpoint{1.752396in}{5.061973in}}%
\pgfpathlineto{\pgfqpoint{1.771235in}{5.075842in}}%
\pgfpathlineto{\pgfqpoint{1.788227in}{5.088233in}}%
\pgfpathlineto{\pgfqpoint{1.800808in}{5.097297in}}%
\pgfpathlineto{\pgfqpoint{1.824898in}{5.114492in}}%
\pgfpathlineto{\pgfqpoint{1.830381in}{5.118359in}}%
\pgfpathlineto{\pgfqpoint{1.848788in}{5.131214in}}%
\pgfusepath{stroke}%
\end{pgfscope}%
\begin{pgfscope}%
\pgfpathrectangle{\pgfqpoint{0.854460in}{0.571603in}}{\pgfqpoint{5.885100in}{5.225635in}}%
\pgfusepath{clip}%
\pgfsetbuttcap%
\pgfsetroundjoin%
\pgfsetlinewidth{1.505625pt}%
\definecolor{currentstroke}{rgb}{0.162142,0.474838,0.558140}%
\pgfsetstrokecolor{currentstroke}%
\pgfsetdash{}{0pt}%
\pgfpathmoveto{\pgfqpoint{2.171091in}{5.334606in}}%
\pgfpathlineto{\pgfqpoint{2.185262in}{5.342697in}}%
\pgfpathlineto{\pgfqpoint{2.199619in}{5.350827in}}%
\pgfpathlineto{\pgfqpoint{2.214835in}{5.359340in}}%
\pgfpathlineto{\pgfqpoint{2.244409in}{5.375753in}}%
\pgfpathlineto{\pgfqpoint{2.246841in}{5.377087in}}%
\pgfpathlineto{\pgfqpoint{2.273982in}{5.391790in}}%
\pgfpathlineto{\pgfqpoint{2.295464in}{5.403346in}}%
\pgfpathlineto{\pgfqpoint{2.303555in}{5.407646in}}%
\pgfpathlineto{\pgfqpoint{2.333129in}{5.423198in}}%
\pgfpathlineto{\pgfqpoint{2.345428in}{5.429606in}}%
\pgfpathlineto{\pgfqpoint{2.362702in}{5.438495in}}%
\pgfpathlineto{\pgfqpoint{2.392275in}{5.453595in}}%
\pgfpathlineto{\pgfqpoint{2.396774in}{5.455865in}}%
\pgfpathlineto{\pgfqpoint{2.421849in}{5.468361in}}%
\pgfpathlineto{\pgfqpoint{2.449641in}{5.482125in}}%
\pgfpathlineto{\pgfqpoint{2.451422in}{5.482996in}}%
\pgfpathlineto{\pgfqpoint{2.480996in}{5.497274in}}%
\pgfpathlineto{\pgfqpoint{2.504160in}{5.508384in}}%
\pgfpathlineto{\pgfqpoint{2.510569in}{5.511420in}}%
\pgfpathlineto{\pgfqpoint{2.540142in}{5.525264in}}%
\pgfpathlineto{\pgfqpoint{2.560336in}{5.534644in}}%
\pgfpathlineto{\pgfqpoint{2.569716in}{5.538946in}}%
\pgfpathlineto{\pgfqpoint{2.599289in}{5.552360in}}%
\pgfpathlineto{\pgfqpoint{2.618284in}{5.560903in}}%
\pgfpathlineto{\pgfqpoint{2.628862in}{5.565602in}}%
\pgfpathlineto{\pgfqpoint{2.658436in}{5.578590in}}%
\pgfpathlineto{\pgfqpoint{2.678122in}{5.587163in}}%
\pgfpathlineto{\pgfqpoint{2.688009in}{5.591414in}}%
\pgfpathlineto{\pgfqpoint{2.717582in}{5.603981in}}%
\pgfpathlineto{\pgfqpoint{2.739978in}{5.613422in}}%
\pgfpathlineto{\pgfqpoint{2.747156in}{5.616410in}}%
\pgfpathlineto{\pgfqpoint{2.776729in}{5.628559in}}%
\pgfpathlineto{\pgfqpoint{2.803986in}{5.639682in}}%
\pgfpathlineto{\pgfqpoint{2.806303in}{5.640615in}}%
\pgfpathlineto{\pgfqpoint{2.835876in}{5.652350in}}%
\pgfpathlineto{\pgfqpoint{2.865449in}{5.664011in}}%
\pgfpathlineto{\pgfqpoint{2.870417in}{5.665941in}}%
\pgfpathlineto{\pgfqpoint{2.895023in}{5.675379in}}%
\pgfpathlineto{\pgfqpoint{2.924596in}{5.686627in}}%
\pgfpathlineto{\pgfqpoint{2.939428in}{5.692201in}}%
\pgfpathlineto{\pgfqpoint{2.954169in}{5.697669in}}%
\pgfpathlineto{\pgfqpoint{2.983743in}{5.708507in}}%
\pgfpathlineto{\pgfqpoint{3.011129in}{5.718460in}}%
\pgfpathlineto{\pgfqpoint{3.013316in}{5.719245in}}%
\pgfpathlineto{\pgfqpoint{3.042890in}{5.729674in}}%
\pgfpathlineto{\pgfqpoint{3.072463in}{5.740020in}}%
\pgfpathlineto{\pgfqpoint{3.086081in}{5.744720in}}%
\pgfpathlineto{\pgfqpoint{3.102036in}{5.750153in}}%
\pgfpathlineto{\pgfqpoint{3.131610in}{5.760091in}}%
\pgfpathlineto{\pgfqpoint{3.161183in}{5.769940in}}%
\pgfpathlineto{\pgfqpoint{3.164360in}{5.770979in}}%
\pgfpathlineto{\pgfqpoint{3.190756in}{5.779496in}}%
\pgfpathlineto{\pgfqpoint{3.220330in}{5.788936in}}%
\pgfpathlineto{\pgfqpoint{3.246629in}{5.797238in}}%
\pgfusepath{stroke}%
\end{pgfscope}%
\begin{pgfscope}%
\pgfpathrectangle{\pgfqpoint{0.854460in}{0.571603in}}{\pgfqpoint{5.885100in}{5.225635in}}%
\pgfusepath{clip}%
\pgfsetbuttcap%
\pgfsetroundjoin%
\pgfsetlinewidth{1.505625pt}%
\definecolor{currentstroke}{rgb}{0.162142,0.474838,0.558140}%
\pgfsetstrokecolor{currentstroke}%
\pgfsetdash{}{0pt}%
\pgfpathmoveto{\pgfqpoint{5.319889in}{5.797238in}}%
\pgfpathlineto{\pgfqpoint{5.343080in}{5.770979in}}%
\pgfpathlineto{\pgfqpoint{5.363234in}{5.744720in}}%
\pgfpathlineto{\pgfqpoint{5.380860in}{5.718460in}}%
\pgfpathlineto{\pgfqpoint{5.396085in}{5.692201in}}%
\pgfpathlineto{\pgfqpoint{5.409467in}{5.665941in}}%
\pgfpathlineto{\pgfqpoint{5.420967in}{5.639682in}}%
\pgfpathlineto{\pgfqpoint{5.438332in}{5.591459in}}%
\pgfpathlineto{\pgfqpoint{5.447080in}{5.560903in}}%
\pgfpathlineto{\pgfqpoint{5.458713in}{5.508384in}}%
\pgfpathlineto{\pgfqpoint{5.466774in}{5.455865in}}%
\pgfpathlineto{\pgfqpoint{5.471784in}{5.403346in}}%
\pgfpathlineto{\pgfqpoint{5.474329in}{5.350827in}}%
\pgfpathlineto{\pgfqpoint{5.474831in}{5.298308in}}%
\pgfpathlineto{\pgfqpoint{5.473630in}{5.245790in}}%
\pgfpathlineto{\pgfqpoint{5.469261in}{5.167011in}}%
\pgfpathlineto{\pgfqpoint{5.462445in}{5.088233in}}%
\pgfpathlineto{\pgfqpoint{5.450740in}{4.983195in}}%
\pgfpathlineto{\pgfqpoint{5.433594in}{4.851897in}}%
\pgfpathlineto{\pgfqpoint{5.383680in}{4.484265in}}%
\pgfpathlineto{\pgfqpoint{5.368694in}{4.352967in}}%
\pgfpathlineto{\pgfqpoint{5.358744in}{4.247930in}}%
\pgfpathlineto{\pgfqpoint{5.350973in}{4.142892in}}%
\pgfpathlineto{\pgfqpoint{5.345599in}{4.037854in}}%
\pgfpathlineto{\pgfqpoint{5.342913in}{3.932816in}}%
\pgfpathlineto{\pgfqpoint{5.342799in}{3.854037in}}%
\pgfpathlineto{\pgfqpoint{5.344404in}{3.775259in}}%
\pgfpathlineto{\pgfqpoint{5.347804in}{3.696481in}}%
\pgfpathlineto{\pgfqpoint{5.353038in}{3.617702in}}%
\pgfpathlineto{\pgfqpoint{5.360175in}{3.538924in}}%
\pgfpathlineto{\pgfqpoint{5.369303in}{3.460145in}}%
\pgfpathlineto{\pgfqpoint{5.380478in}{3.381367in}}%
\pgfpathlineto{\pgfqpoint{5.393661in}{3.302589in}}%
\pgfpathlineto{\pgfqpoint{5.409018in}{3.223810in}}%
\pgfpathlineto{\pgfqpoint{5.426450in}{3.145032in}}%
\pgfpathlineto{\pgfqpoint{5.446103in}{3.066253in}}%
\pgfpathlineto{\pgfqpoint{5.467985in}{2.987475in}}%
\pgfpathlineto{\pgfqpoint{5.492053in}{2.908696in}}%
\pgfpathlineto{\pgfqpoint{5.518394in}{2.829918in}}%
\pgfpathlineto{\pgfqpoint{5.547023in}{2.751140in}}%
\pgfpathlineto{\pgfqpoint{5.577957in}{2.672361in}}%
\pgfpathlineto{\pgfqpoint{5.615772in}{2.583299in}}%
\pgfpathlineto{\pgfqpoint{5.646802in}{2.514804in}}%
\pgfpathlineto{\pgfqpoint{5.684696in}{2.436026in}}%
\pgfpathlineto{\pgfqpoint{5.724967in}{2.357248in}}%
\pgfpathlineto{\pgfqpoint{5.767609in}{2.278469in}}%
\pgfpathlineto{\pgfqpoint{5.812588in}{2.199691in}}%
\pgfpathlineto{\pgfqpoint{5.859957in}{2.120912in}}%
\pgfpathlineto{\pgfqpoint{5.911506in}{2.039385in}}%
\pgfpathlineto{\pgfqpoint{5.970653in}{1.950390in}}%
\pgfpathlineto{\pgfqpoint{6.016274in}{1.884577in}}%
\pgfpathlineto{\pgfqpoint{6.088946in}{1.784502in}}%
\pgfpathlineto{\pgfqpoint{6.132344in}{1.727020in}}%
\pgfpathlineto{\pgfqpoint{6.145535in}{1.709951in}}%
\pgfpathlineto{\pgfqpoint{6.145535in}{1.709951in}}%
\pgfusepath{stroke}%
\end{pgfscope}%
\begin{pgfscope}%
\pgfpathrectangle{\pgfqpoint{0.854460in}{0.571603in}}{\pgfqpoint{5.885100in}{5.225635in}}%
\pgfusepath{clip}%
\pgfsetbuttcap%
\pgfsetroundjoin%
\pgfsetlinewidth{1.505625pt}%
\definecolor{currentstroke}{rgb}{0.162142,0.474838,0.558140}%
\pgfsetstrokecolor{currentstroke}%
\pgfsetdash{}{0pt}%
\pgfpathmoveto{\pgfqpoint{6.391998in}{1.412793in}}%
\pgfpathlineto{\pgfqpoint{6.392778in}{1.411906in}}%
\pgfpathlineto{\pgfqpoint{6.414253in}{1.387822in}}%
\pgfpathlineto{\pgfqpoint{6.416190in}{1.385647in}}%
\pgfpathlineto{\pgfqpoint{6.439834in}{1.359388in}}%
\pgfpathlineto{\pgfqpoint{6.443827in}{1.355005in}}%
\pgfpathlineto{\pgfqpoint{6.463730in}{1.333128in}}%
\pgfpathlineto{\pgfqpoint{6.473400in}{1.322626in}}%
\pgfpathlineto{\pgfqpoint{6.487894in}{1.306869in}}%
\pgfpathlineto{\pgfqpoint{6.502973in}{1.290665in}}%
\pgfpathlineto{\pgfqpoint{6.512323in}{1.280609in}}%
\pgfpathlineto{\pgfqpoint{6.532547in}{1.259106in}}%
\pgfpathlineto{\pgfqpoint{6.537016in}{1.254350in}}%
\pgfpathlineto{\pgfqpoint{6.561971in}{1.228090in}}%
\pgfpathlineto{\pgfqpoint{6.562120in}{1.227934in}}%
\pgfpathlineto{\pgfqpoint{6.587147in}{1.201831in}}%
\pgfpathlineto{\pgfqpoint{6.591693in}{1.197139in}}%
\pgfpathlineto{\pgfqpoint{6.612585in}{1.175571in}}%
\pgfpathlineto{\pgfqpoint{6.621267in}{1.166702in}}%
\pgfpathlineto{\pgfqpoint{6.638283in}{1.149312in}}%
\pgfpathlineto{\pgfqpoint{6.650840in}{1.136610in}}%
\pgfpathlineto{\pgfqpoint{6.664240in}{1.123052in}}%
\pgfpathlineto{\pgfqpoint{6.680414in}{1.106851in}}%
\pgfpathlineto{\pgfqpoint{6.690454in}{1.096793in}}%
\pgfpathlineto{\pgfqpoint{6.709987in}{1.077415in}}%
\pgfpathlineto{\pgfqpoint{6.716924in}{1.070533in}}%
\pgfpathlineto{\pgfqpoint{6.739560in}{1.048290in}}%
\pgfusepath{stroke}%
\end{pgfscope}%
\begin{pgfscope}%
\pgfpathrectangle{\pgfqpoint{0.854460in}{0.571603in}}{\pgfqpoint{5.885100in}{5.225635in}}%
\pgfusepath{clip}%
\pgfsetbuttcap%
\pgfsetroundjoin%
\pgfsetlinewidth{1.505625pt}%
\definecolor{currentstroke}{rgb}{0.154815,0.493313,0.557840}%
\pgfsetstrokecolor{currentstroke}%
\pgfsetdash{}{0pt}%
\pgfpathmoveto{\pgfqpoint{1.475048in}{0.571603in}}%
\pgfpathlineto{\pgfqpoint{1.446453in}{0.597863in}}%
\pgfpathlineto{\pgfqpoint{1.445928in}{0.598352in}}%
\pgfpathlineto{\pgfqpoint{1.435295in}{0.608307in}}%
\pgfusepath{stroke}%
\end{pgfscope}%
\begin{pgfscope}%
\pgfpathrectangle{\pgfqpoint{0.854460in}{0.571603in}}{\pgfqpoint{5.885100in}{5.225635in}}%
\pgfusepath{clip}%
\pgfsetbuttcap%
\pgfsetroundjoin%
\pgfsetlinewidth{1.505625pt}%
\definecolor{currentstroke}{rgb}{0.154815,0.493313,0.557840}%
\pgfsetstrokecolor{currentstroke}%
\pgfsetdash{}{0pt}%
\pgfpathmoveto{\pgfqpoint{1.167670in}{0.884705in}}%
\pgfpathlineto{\pgfqpoint{1.165909in}{0.886717in}}%
\pgfpathlineto{\pgfqpoint{1.150194in}{0.904939in}}%
\pgfpathlineto{\pgfqpoint{1.143308in}{0.912976in}}%
\pgfpathlineto{\pgfqpoint{1.121147in}{0.939236in}}%
\pgfpathlineto{\pgfqpoint{1.120621in}{0.939871in}}%
\pgfpathlineto{\pgfqpoint{1.099515in}{0.965495in}}%
\pgfpathlineto{\pgfqpoint{1.091047in}{0.975932in}}%
\pgfpathlineto{\pgfqpoint{1.078301in}{0.991755in}}%
\pgfpathlineto{\pgfqpoint{1.061474in}{1.012961in}}%
\pgfpathlineto{\pgfqpoint{1.057493in}{1.018014in}}%
\pgfpathlineto{\pgfqpoint{1.037156in}{1.044274in}}%
\pgfpathlineto{\pgfqpoint{1.031901in}{1.051175in}}%
\pgfpathlineto{\pgfqpoint{1.017268in}{1.070533in}}%
\pgfpathlineto{\pgfqpoint{1.002327in}{1.090606in}}%
\pgfpathlineto{\pgfqpoint{0.997757in}{1.096793in}}%
\pgfpathlineto{\pgfqpoint{0.978700in}{1.123052in}}%
\pgfpathlineto{\pgfqpoint{0.972754in}{1.131388in}}%
\pgfpathlineto{\pgfqpoint{0.960071in}{1.149312in}}%
\pgfpathlineto{\pgfqpoint{0.943181in}{1.173558in}}%
\pgfpathlineto{\pgfqpoint{0.941790in}{1.175571in}}%
\pgfpathlineto{\pgfqpoint{0.923998in}{1.201831in}}%
\pgfpathlineto{\pgfqpoint{0.913607in}{1.217421in}}%
\pgfpathlineto{\pgfqpoint{0.906556in}{1.228090in}}%
\pgfpathlineto{\pgfqpoint{0.889513in}{1.254350in}}%
\pgfpathlineto{\pgfqpoint{0.884034in}{1.262952in}}%
\pgfpathlineto{\pgfqpoint{0.872886in}{1.280609in}}%
\pgfpathlineto{\pgfqpoint{0.856588in}{1.306869in}}%
\pgfpathlineto{\pgfqpoint{0.854460in}{1.310369in}}%
\pgfusepath{stroke}%
\end{pgfscope}%
\begin{pgfscope}%
\pgfpathrectangle{\pgfqpoint{0.854460in}{0.571603in}}{\pgfqpoint{5.885100in}{5.225635in}}%
\pgfusepath{clip}%
\pgfsetbuttcap%
\pgfsetroundjoin%
\pgfsetlinewidth{1.505625pt}%
\definecolor{currentstroke}{rgb}{0.154815,0.493313,0.557840}%
\pgfsetstrokecolor{currentstroke}%
\pgfsetdash{}{0pt}%
\pgfpathmoveto{\pgfqpoint{0.854460in}{4.185177in}}%
\pgfpathlineto{\pgfqpoint{0.860665in}{4.195411in}}%
\pgfpathlineto{\pgfqpoint{0.876858in}{4.221670in}}%
\pgfpathlineto{\pgfqpoint{0.884034in}{4.233116in}}%
\pgfpathlineto{\pgfqpoint{0.893449in}{4.247930in}}%
\pgfpathlineto{\pgfqpoint{0.910402in}{4.274189in}}%
\pgfpathlineto{\pgfqpoint{0.913607in}{4.279070in}}%
\pgfpathlineto{\pgfqpoint{0.927836in}{4.300449in}}%
\pgfpathlineto{\pgfqpoint{0.943181in}{4.323174in}}%
\pgfpathlineto{\pgfqpoint{0.945599in}{4.326708in}}%
\pgfpathlineto{\pgfqpoint{0.963867in}{4.352967in}}%
\pgfpathlineto{\pgfqpoint{0.972754in}{4.365556in}}%
\pgfpathlineto{\pgfqpoint{0.982533in}{4.379227in}}%
\pgfpathlineto{\pgfqpoint{1.001583in}{4.405486in}}%
\pgfpathlineto{\pgfqpoint{1.002327in}{4.406495in}}%
\pgfpathlineto{\pgfqpoint{1.021188in}{4.431746in}}%
\pgfpathlineto{\pgfqpoint{1.031901in}{4.445894in}}%
\pgfpathlineto{\pgfqpoint{1.041189in}{4.458005in}}%
\pgfpathlineto{\pgfqpoint{1.061474in}{4.484101in}}%
\pgfpathlineto{\pgfqpoint{1.061603in}{4.484265in}}%
\pgfpathlineto{\pgfqpoint{1.082600in}{4.510524in}}%
\pgfpathlineto{\pgfqpoint{1.091047in}{4.520950in}}%
\pgfpathlineto{\pgfqpoint{1.104038in}{4.536784in}}%
\pgfpathlineto{\pgfqpoint{1.120621in}{4.556733in}}%
\pgfpathlineto{\pgfqpoint{1.125931in}{4.563043in}}%
\pgfpathlineto{\pgfqpoint{1.148330in}{4.589303in}}%
\pgfpathlineto{\pgfqpoint{1.150194in}{4.591456in}}%
\pgfpathlineto{\pgfqpoint{1.171305in}{4.615562in}}%
\pgfpathlineto{\pgfqpoint{1.179767in}{4.625101in}}%
\pgfpathlineto{\pgfqpoint{1.194778in}{4.641822in}}%
\pgfpathlineto{\pgfqpoint{1.209341in}{4.657838in}}%
\pgfpathlineto{\pgfqpoint{1.218765in}{4.668081in}}%
\pgfpathlineto{\pgfqpoint{1.238914in}{4.689705in}}%
\pgfpathlineto{\pgfqpoint{1.243284in}{4.694341in}}%
\pgfpathlineto{\pgfqpoint{1.268354in}{4.720600in}}%
\pgfpathlineto{\pgfqpoint{1.268488in}{4.720739in}}%
\pgfpathlineto{\pgfqpoint{1.294058in}{4.746860in}}%
\pgfpathlineto{\pgfqpoint{1.298061in}{4.750898in}}%
\pgfpathlineto{\pgfqpoint{1.320336in}{4.773119in}}%
\pgfpathlineto{\pgfqpoint{1.327634in}{4.780310in}}%
\pgfpathlineto{\pgfqpoint{1.347203in}{4.799378in}}%
\pgfpathlineto{\pgfqpoint{1.357208in}{4.809007in}}%
\pgfpathlineto{\pgfqpoint{1.374677in}{4.825638in}}%
\pgfpathlineto{\pgfqpoint{1.386781in}{4.837019in}}%
\pgfpathlineto{\pgfqpoint{1.402775in}{4.851897in}}%
\pgfpathlineto{\pgfqpoint{1.416354in}{4.864375in}}%
\pgfpathlineto{\pgfqpoint{1.431514in}{4.878157in}}%
\pgfpathlineto{\pgfqpoint{1.445928in}{4.891101in}}%
\pgfpathlineto{\pgfqpoint{1.460910in}{4.904416in}}%
\pgfpathlineto{\pgfqpoint{1.475501in}{4.917226in}}%
\pgfpathlineto{\pgfqpoint{1.490981in}{4.930676in}}%
\pgfpathlineto{\pgfqpoint{1.503351in}{4.941294in}}%
\pgfusepath{stroke}%
\end{pgfscope}%
\begin{pgfscope}%
\pgfpathrectangle{\pgfqpoint{0.854460in}{0.571603in}}{\pgfqpoint{5.885100in}{5.225635in}}%
\pgfusepath{clip}%
\pgfsetbuttcap%
\pgfsetroundjoin%
\pgfsetlinewidth{1.505625pt}%
\definecolor{currentstroke}{rgb}{0.154815,0.493313,0.557840}%
\pgfsetstrokecolor{currentstroke}%
\pgfsetdash{}{0pt}%
\pgfpathmoveto{\pgfqpoint{1.806253in}{5.174806in}}%
\pgfpathlineto{\pgfqpoint{1.830381in}{5.191519in}}%
\pgfpathlineto{\pgfqpoint{1.832936in}{5.193271in}}%
\pgfpathlineto{\pgfqpoint{1.859955in}{5.211571in}}%
\pgfpathlineto{\pgfqpoint{1.871805in}{5.219530in}}%
\pgfpathlineto{\pgfqpoint{1.889528in}{5.231291in}}%
\pgfpathlineto{\pgfqpoint{1.911558in}{5.245790in}}%
\pgfpathlineto{\pgfqpoint{1.919102in}{5.250694in}}%
\pgfpathlineto{\pgfqpoint{1.948675in}{5.269750in}}%
\pgfpathlineto{\pgfqpoint{1.952279in}{5.272049in}}%
\pgfpathlineto{\pgfqpoint{1.978248in}{5.288412in}}%
\pgfpathlineto{\pgfqpoint{1.994077in}{5.298308in}}%
\pgfpathlineto{\pgfqpoint{2.007822in}{5.306798in}}%
\pgfpathlineto{\pgfqpoint{2.036811in}{5.324568in}}%
\pgfpathlineto{\pgfqpoint{2.037395in}{5.324922in}}%
\pgfpathlineto{\pgfqpoint{2.066968in}{5.342629in}}%
\pgfpathlineto{\pgfqpoint{2.080765in}{5.350827in}}%
\pgfpathlineto{\pgfqpoint{2.096542in}{5.360089in}}%
\pgfpathlineto{\pgfqpoint{2.125706in}{5.377087in}}%
\pgfpathlineto{\pgfqpoint{2.126115in}{5.377322in}}%
\pgfpathlineto{\pgfqpoint{2.155689in}{5.394153in}}%
\pgfpathlineto{\pgfqpoint{2.171957in}{5.403346in}}%
\pgfpathlineto{\pgfqpoint{2.185262in}{5.410773in}}%
\pgfpathlineto{\pgfqpoint{2.214835in}{5.427150in}}%
\pgfpathlineto{\pgfqpoint{2.219317in}{5.429606in}}%
\pgfpathlineto{\pgfqpoint{2.244409in}{5.443185in}}%
\pgfpathlineto{\pgfqpoint{2.267990in}{5.455865in}}%
\pgfpathlineto{\pgfqpoint{2.273982in}{5.459048in}}%
\pgfpathlineto{\pgfqpoint{2.303555in}{5.474591in}}%
\pgfpathlineto{\pgfqpoint{2.318003in}{5.482125in}}%
\pgfpathlineto{\pgfqpoint{2.333129in}{5.489914in}}%
\pgfpathlineto{\pgfqpoint{2.362702in}{5.505024in}}%
\pgfpathlineto{\pgfqpoint{2.369348in}{5.508384in}}%
\pgfpathlineto{\pgfqpoint{2.392275in}{5.519833in}}%
\pgfpathlineto{\pgfqpoint{2.421849in}{5.534515in}}%
\pgfpathlineto{\pgfqpoint{2.422111in}{5.534644in}}%
\pgfpathlineto{\pgfqpoint{2.451422in}{5.548834in}}%
\pgfpathlineto{\pgfqpoint{2.476487in}{5.560903in}}%
\pgfpathlineto{\pgfqpoint{2.480996in}{5.563047in}}%
\pgfpathlineto{\pgfqpoint{2.510569in}{5.576946in}}%
\pgfpathlineto{\pgfqpoint{2.532447in}{5.587163in}}%
\pgfpathlineto{\pgfqpoint{2.540142in}{5.590711in}}%
\pgfpathlineto{\pgfqpoint{2.569716in}{5.604199in}}%
\pgfpathlineto{\pgfqpoint{2.590084in}{5.613422in}}%
\pgfpathlineto{\pgfqpoint{2.599289in}{5.617538in}}%
\pgfpathlineto{\pgfqpoint{2.628862in}{5.630619in}}%
\pgfpathlineto{\pgfqpoint{2.649502in}{5.639682in}}%
\pgfpathlineto{\pgfqpoint{2.658436in}{5.643554in}}%
\pgfpathlineto{\pgfqpoint{2.688009in}{5.656233in}}%
\pgfpathlineto{\pgfqpoint{2.710812in}{5.665941in}}%
\pgfpathlineto{\pgfqpoint{2.717582in}{5.668787in}}%
\pgfpathlineto{\pgfqpoint{2.747156in}{5.681067in}}%
\pgfpathlineto{\pgfqpoint{2.774130in}{5.692201in}}%
\pgfpathlineto{\pgfqpoint{2.776729in}{5.693260in}}%
\pgfpathlineto{\pgfqpoint{2.806303in}{5.705145in}}%
\pgfpathlineto{\pgfqpoint{2.835876in}{5.716965in}}%
\pgfpathlineto{\pgfqpoint{2.839667in}{5.718460in}}%
\pgfpathlineto{\pgfqpoint{2.865449in}{5.728493in}}%
\pgfpathlineto{\pgfqpoint{2.895023in}{5.739920in}}%
\pgfpathlineto{\pgfqpoint{2.907587in}{5.744720in}}%
\pgfpathlineto{\pgfqpoint{2.924596in}{5.751133in}}%
\pgfpathlineto{\pgfqpoint{2.954169in}{5.762170in}}%
\pgfpathlineto{\pgfqpoint{2.977966in}{5.770979in}}%
\pgfpathlineto{\pgfqpoint{2.983743in}{5.773089in}}%
\pgfpathlineto{\pgfqpoint{3.013316in}{5.783740in}}%
\pgfpathlineto{\pgfqpoint{3.042890in}{5.794318in}}%
\pgfpathlineto{\pgfqpoint{3.051169in}{5.797238in}}%
\pgfusepath{stroke}%
\end{pgfscope}%
\begin{pgfscope}%
\pgfpathrectangle{\pgfqpoint{0.854460in}{0.571603in}}{\pgfqpoint{5.885100in}{5.225635in}}%
\pgfusepath{clip}%
\pgfsetbuttcap%
\pgfsetroundjoin%
\pgfsetlinewidth{1.505625pt}%
\definecolor{currentstroke}{rgb}{0.154815,0.493313,0.557840}%
\pgfsetstrokecolor{currentstroke}%
\pgfsetdash{}{0pt}%
\pgfpathmoveto{\pgfqpoint{5.530714in}{5.797238in}}%
\pgfpathlineto{\pgfqpoint{5.545001in}{5.770979in}}%
\pgfpathlineto{\pgfqpoint{5.556626in}{5.746573in}}%
\pgfpathlineto{\pgfqpoint{5.557469in}{5.744720in}}%
\pgfpathlineto{\pgfqpoint{5.568107in}{5.718460in}}%
\pgfpathlineto{\pgfqpoint{5.577306in}{5.692201in}}%
\pgfpathlineto{\pgfqpoint{5.585196in}{5.665941in}}%
\pgfpathlineto{\pgfqpoint{5.586199in}{5.662081in}}%
\pgfpathlineto{\pgfqpoint{5.591790in}{5.639682in}}%
\pgfpathlineto{\pgfqpoint{5.597305in}{5.613422in}}%
\pgfpathlineto{\pgfqpoint{5.601851in}{5.587163in}}%
\pgfpathlineto{\pgfqpoint{5.605513in}{5.560903in}}%
\pgfpathlineto{\pgfqpoint{5.608367in}{5.534644in}}%
\pgfpathlineto{\pgfqpoint{5.610481in}{5.508384in}}%
\pgfpathlineto{\pgfqpoint{5.611920in}{5.482125in}}%
\pgfpathlineto{\pgfqpoint{5.612741in}{5.455865in}}%
\pgfpathlineto{\pgfqpoint{5.612997in}{5.429606in}}%
\pgfpathlineto{\pgfqpoint{5.612738in}{5.403346in}}%
\pgfpathlineto{\pgfqpoint{5.612008in}{5.377087in}}%
\pgfpathlineto{\pgfqpoint{5.610849in}{5.350827in}}%
\pgfpathlineto{\pgfqpoint{5.609298in}{5.324568in}}%
\pgfpathlineto{\pgfqpoint{5.607392in}{5.298308in}}%
\pgfpathlineto{\pgfqpoint{5.605162in}{5.272049in}}%
\pgfpathlineto{\pgfqpoint{5.602640in}{5.245790in}}%
\pgfpathlineto{\pgfqpoint{5.599853in}{5.219530in}}%
\pgfpathlineto{\pgfqpoint{5.596829in}{5.193271in}}%
\pgfpathlineto{\pgfqpoint{5.593590in}{5.167011in}}%
\pgfpathlineto{\pgfqpoint{5.590161in}{5.140752in}}%
\pgfpathlineto{\pgfqpoint{5.587949in}{5.124606in}}%
\pgfusepath{stroke}%
\end{pgfscope}%
\begin{pgfscope}%
\pgfpathrectangle{\pgfqpoint{0.854460in}{0.571603in}}{\pgfqpoint{5.885100in}{5.225635in}}%
\pgfusepath{clip}%
\pgfsetbuttcap%
\pgfsetroundjoin%
\pgfsetlinewidth{1.505625pt}%
\definecolor{currentstroke}{rgb}{0.154815,0.493313,0.557840}%
\pgfsetstrokecolor{currentstroke}%
\pgfsetdash{}{0pt}%
\pgfpathmoveto{\pgfqpoint{5.524927in}{4.737221in}}%
\pgfpathlineto{\pgfqpoint{5.497479in}{4.568906in}}%
\pgfpathlineto{\pgfqpoint{5.481014in}{4.458005in}}%
\pgfpathlineto{\pgfqpoint{5.467239in}{4.352967in}}%
\pgfpathlineto{\pgfqpoint{5.455432in}{4.247930in}}%
\pgfpathlineto{\pgfqpoint{5.446008in}{4.142892in}}%
\pgfpathlineto{\pgfqpoint{5.439200in}{4.037854in}}%
\pgfpathlineto{\pgfqpoint{5.435181in}{3.932816in}}%
\pgfpathlineto{\pgfqpoint{5.434144in}{3.854037in}}%
\pgfpathlineto{\pgfqpoint{5.434891in}{3.775259in}}%
\pgfpathlineto{\pgfqpoint{5.438332in}{3.679404in}}%
\pgfpathlineto{\pgfqpoint{5.441977in}{3.617702in}}%
\pgfpathlineto{\pgfqpoint{5.448421in}{3.538924in}}%
\pgfpathlineto{\pgfqpoint{5.456897in}{3.460145in}}%
\pgfpathlineto{\pgfqpoint{5.467906in}{3.378514in}}%
\pgfpathlineto{\pgfqpoint{5.480084in}{3.302589in}}%
\pgfpathlineto{\pgfqpoint{5.497479in}{3.211361in}}%
\pgfpathlineto{\pgfqpoint{5.511847in}{3.145032in}}%
\pgfpathlineto{\pgfqpoint{5.531050in}{3.066253in}}%
\pgfpathlineto{\pgfqpoint{5.552474in}{2.987475in}}%
\pgfpathlineto{\pgfqpoint{5.576151in}{2.908696in}}%
\pgfpathlineto{\pgfqpoint{5.602127in}{2.829918in}}%
\pgfpathlineto{\pgfqpoint{5.630416in}{2.751140in}}%
\pgfpathlineto{\pgfqpoint{5.661033in}{2.672361in}}%
\pgfpathlineto{\pgfqpoint{5.693995in}{2.593583in}}%
\pgfpathlineto{\pgfqpoint{5.734066in}{2.504710in}}%
\pgfpathlineto{\pgfqpoint{5.767010in}{2.436026in}}%
\pgfpathlineto{\pgfqpoint{5.807049in}{2.357248in}}%
\pgfpathlineto{\pgfqpoint{5.852359in}{2.273394in}}%
\pgfpathlineto{\pgfqpoint{5.894326in}{2.199691in}}%
\pgfpathlineto{\pgfqpoint{5.941584in}{2.120912in}}%
\pgfpathlineto{\pgfqpoint{5.991182in}{2.042134in}}%
\pgfpathlineto{\pgfqpoint{6.043199in}{1.963355in}}%
\pgfpathlineto{\pgfqpoint{6.097624in}{1.884577in}}%
\pgfpathlineto{\pgfqpoint{6.154445in}{1.805799in}}%
\pgfpathlineto{\pgfqpoint{6.213653in}{1.727020in}}%
\pgfpathlineto{\pgfqpoint{6.275244in}{1.648242in}}%
\pgfpathlineto{\pgfqpoint{6.355107in}{1.550394in}}%
\pgfpathlineto{\pgfqpoint{6.414253in}{1.480634in}}%
\pgfpathlineto{\pgfqpoint{6.474325in}{1.411906in}}%
\pgfpathlineto{\pgfqpoint{6.562120in}{1.314986in}}%
\pgfpathlineto{\pgfqpoint{6.621267in}{1.251791in}}%
\pgfpathlineto{\pgfqpoint{6.709987in}{1.159871in}}%
\pgfpathlineto{\pgfqpoint{6.739560in}{1.129938in}}%
\pgfpathlineto{\pgfqpoint{6.739560in}{1.129938in}}%
\pgfusepath{stroke}%
\end{pgfscope}%
\begin{pgfscope}%
\pgfpathrectangle{\pgfqpoint{0.854460in}{0.571603in}}{\pgfqpoint{5.885100in}{5.225635in}}%
\pgfusepath{clip}%
\pgfsetbuttcap%
\pgfsetroundjoin%
\pgfsetlinewidth{1.505625pt}%
\definecolor{currentstroke}{rgb}{0.146180,0.515413,0.556823}%
\pgfsetstrokecolor{currentstroke}%
\pgfsetdash{}{0pt}%
\pgfpathmoveto{\pgfqpoint{1.408457in}{0.571603in}}%
\pgfpathlineto{\pgfqpoint{1.386781in}{0.591677in}}%
\pgfpathlineto{\pgfqpoint{1.380134in}{0.597863in}}%
\pgfpathlineto{\pgfqpoint{1.357208in}{0.619503in}}%
\pgfpathlineto{\pgfqpoint{1.352338in}{0.624122in}}%
\pgfpathlineto{\pgfqpoint{1.327634in}{0.647893in}}%
\pgfpathlineto{\pgfqpoint{1.325061in}{0.650382in}}%
\pgfpathlineto{\pgfqpoint{1.298298in}{0.676641in}}%
\pgfpathlineto{\pgfqpoint{1.298061in}{0.676878in}}%
\pgfpathlineto{\pgfqpoint{1.272081in}{0.702901in}}%
\pgfpathlineto{\pgfqpoint{1.268488in}{0.706552in}}%
\pgfpathlineto{\pgfqpoint{1.246364in}{0.729160in}}%
\pgfpathlineto{\pgfqpoint{1.238914in}{0.736884in}}%
\pgfpathlineto{\pgfqpoint{1.221138in}{0.755420in}}%
\pgfpathlineto{\pgfqpoint{1.209341in}{0.767901in}}%
\pgfpathlineto{\pgfqpoint{1.196395in}{0.781679in}}%
\pgfpathlineto{\pgfqpoint{1.179767in}{0.799634in}}%
\pgfpathlineto{\pgfqpoint{1.172124in}{0.807939in}}%
\pgfpathlineto{\pgfqpoint{1.150194in}{0.832114in}}%
\pgfpathlineto{\pgfqpoint{1.148315in}{0.834198in}}%
\pgfpathlineto{\pgfqpoint{1.125020in}{0.860458in}}%
\pgfpathlineto{\pgfqpoint{1.120621in}{0.865494in}}%
\pgfpathlineto{\pgfqpoint{1.102201in}{0.886717in}}%
\pgfpathlineto{\pgfqpoint{1.091047in}{0.899760in}}%
\pgfpathlineto{\pgfqpoint{1.079820in}{0.912976in}}%
\pgfpathlineto{\pgfqpoint{1.061474in}{0.934896in}}%
\pgfpathlineto{\pgfqpoint{1.057866in}{0.939236in}}%
\pgfpathlineto{\pgfqpoint{1.053968in}{0.944003in}}%
\pgfusepath{stroke}%
\end{pgfscope}%
\begin{pgfscope}%
\pgfpathrectangle{\pgfqpoint{0.854460in}{0.571603in}}{\pgfqpoint{5.885100in}{5.225635in}}%
\pgfusepath{clip}%
\pgfsetbuttcap%
\pgfsetroundjoin%
\pgfsetlinewidth{1.505625pt}%
\definecolor{currentstroke}{rgb}{0.146180,0.515413,0.556823}%
\pgfsetstrokecolor{currentstroke}%
\pgfsetdash{}{0pt}%
\pgfpathmoveto{\pgfqpoint{0.854460in}{4.300944in}}%
\pgfpathlineto{\pgfqpoint{0.871553in}{4.326708in}}%
\pgfpathlineto{\pgfqpoint{0.884034in}{4.345251in}}%
\pgfpathlineto{\pgfqpoint{0.889296in}{4.352967in}}%
\pgfpathlineto{\pgfqpoint{0.907487in}{4.379227in}}%
\pgfpathlineto{\pgfqpoint{0.913607in}{4.387926in}}%
\pgfpathlineto{\pgfqpoint{0.926120in}{4.405486in}}%
\pgfpathlineto{\pgfqpoint{0.943181in}{4.429099in}}%
\pgfpathlineto{\pgfqpoint{0.945118in}{4.431746in}}%
\pgfpathlineto{\pgfqpoint{0.964640in}{4.458005in}}%
\pgfpathlineto{\pgfqpoint{0.972754in}{4.468766in}}%
\pgfpathlineto{\pgfqpoint{0.984587in}{4.484265in}}%
\pgfpathlineto{\pgfqpoint{1.002327in}{4.507190in}}%
\pgfpathlineto{\pgfqpoint{1.004940in}{4.510524in}}%
\pgfpathlineto{\pgfqpoint{1.025823in}{4.536784in}}%
\pgfpathlineto{\pgfqpoint{1.031901in}{4.544321in}}%
\pgfpathlineto{\pgfqpoint{1.047183in}{4.563043in}}%
\pgfpathlineto{\pgfqpoint{1.061474in}{4.580323in}}%
\pgfpathlineto{\pgfqpoint{1.068990in}{4.589303in}}%
\pgfpathlineto{\pgfqpoint{1.091047in}{4.615314in}}%
\pgfpathlineto{\pgfqpoint{1.091261in}{4.615562in}}%
\pgfpathlineto{\pgfqpoint{1.114127in}{4.641822in}}%
\pgfpathlineto{\pgfqpoint{1.120621in}{4.649183in}}%
\pgfpathlineto{\pgfqpoint{1.137486in}{4.668081in}}%
\pgfpathlineto{\pgfqpoint{1.150194in}{4.682140in}}%
\pgfpathlineto{\pgfqpoint{1.161349in}{4.694341in}}%
\pgfpathlineto{\pgfqpoint{1.179767in}{4.714230in}}%
\pgfpathlineto{\pgfqpoint{1.180814in}{4.715348in}}%
\pgfusepath{stroke}%
\end{pgfscope}%
\begin{pgfscope}%
\pgfpathrectangle{\pgfqpoint{0.854460in}{0.571603in}}{\pgfqpoint{5.885100in}{5.225635in}}%
\pgfusepath{clip}%
\pgfsetbuttcap%
\pgfsetroundjoin%
\pgfsetlinewidth{1.505625pt}%
\definecolor{currentstroke}{rgb}{0.146180,0.515413,0.556823}%
\pgfsetstrokecolor{currentstroke}%
\pgfsetdash{}{0pt}%
\pgfpathmoveto{\pgfqpoint{1.458807in}{4.980346in}}%
\pgfpathlineto{\pgfqpoint{1.462130in}{4.983195in}}%
\pgfpathlineto{\pgfqpoint{1.475501in}{4.994518in}}%
\pgfpathlineto{\pgfqpoint{1.493312in}{5.009454in}}%
\pgfpathlineto{\pgfqpoint{1.505074in}{5.019200in}}%
\pgfpathlineto{\pgfqpoint{1.525198in}{5.035714in}}%
\pgfpathlineto{\pgfqpoint{1.534648in}{5.043374in}}%
\pgfpathlineto{\pgfqpoint{1.557806in}{5.061973in}}%
\pgfpathlineto{\pgfqpoint{1.564221in}{5.067064in}}%
\pgfpathlineto{\pgfqpoint{1.591148in}{5.088233in}}%
\pgfpathlineto{\pgfqpoint{1.593795in}{5.090288in}}%
\pgfpathlineto{\pgfqpoint{1.623368in}{5.113039in}}%
\pgfpathlineto{\pgfqpoint{1.625276in}{5.114492in}}%
\pgfpathlineto{\pgfqpoint{1.652941in}{5.135314in}}%
\pgfpathlineto{\pgfqpoint{1.660230in}{5.140752in}}%
\pgfpathlineto{\pgfqpoint{1.682515in}{5.157175in}}%
\pgfpathlineto{\pgfqpoint{1.695978in}{5.167011in}}%
\pgfpathlineto{\pgfqpoint{1.712088in}{5.178640in}}%
\pgfpathlineto{\pgfqpoint{1.732531in}{5.193271in}}%
\pgfpathlineto{\pgfqpoint{1.741661in}{5.199726in}}%
\pgfpathlineto{\pgfqpoint{1.769903in}{5.219530in}}%
\pgfpathlineto{\pgfqpoint{1.771235in}{5.220452in}}%
\pgfpathlineto{\pgfqpoint{1.800808in}{5.240733in}}%
\pgfpathlineto{\pgfqpoint{1.808245in}{5.245790in}}%
\pgfpathlineto{\pgfqpoint{1.830381in}{5.260660in}}%
\pgfpathlineto{\pgfqpoint{1.847472in}{5.272049in}}%
\pgfpathlineto{\pgfqpoint{1.859955in}{5.280268in}}%
\pgfpathlineto{\pgfqpoint{1.887569in}{5.298308in}}%
\pgfpathlineto{\pgfqpoint{1.889528in}{5.299573in}}%
\pgfpathlineto{\pgfqpoint{1.919102in}{5.318467in}}%
\pgfpathlineto{\pgfqpoint{1.928728in}{5.324568in}}%
\pgfpathlineto{\pgfqpoint{1.948675in}{5.337056in}}%
\pgfpathlineto{\pgfqpoint{1.970834in}{5.350827in}}%
\pgfpathlineto{\pgfqpoint{1.978248in}{5.355380in}}%
\pgfpathlineto{\pgfqpoint{2.007822in}{5.373377in}}%
\pgfpathlineto{\pgfqpoint{2.013974in}{5.377087in}}%
\pgfpathlineto{\pgfqpoint{2.037395in}{5.391037in}}%
\pgfpathlineto{\pgfqpoint{2.058203in}{5.403346in}}%
\pgfpathlineto{\pgfqpoint{2.066968in}{5.408468in}}%
\pgfpathlineto{\pgfqpoint{2.096542in}{5.425601in}}%
\pgfpathlineto{\pgfqpoint{2.103518in}{5.429606in}}%
\pgfpathlineto{\pgfqpoint{2.126115in}{5.442419in}}%
\pgfpathlineto{\pgfqpoint{2.149982in}{5.455865in}}%
\pgfpathlineto{\pgfqpoint{2.155689in}{5.459041in}}%
\pgfpathlineto{\pgfqpoint{2.185262in}{5.475339in}}%
\pgfpathlineto{\pgfqpoint{2.197671in}{5.482125in}}%
\pgfpathlineto{\pgfqpoint{2.214835in}{5.491395in}}%
\pgfpathlineto{\pgfqpoint{2.244409in}{5.507263in}}%
\pgfpathlineto{\pgfqpoint{2.246521in}{5.508384in}}%
\pgfpathlineto{\pgfqpoint{2.273982in}{5.522779in}}%
\pgfpathlineto{\pgfqpoint{2.296747in}{5.534644in}}%
\pgfpathlineto{\pgfqpoint{2.303555in}{5.538148in}}%
\pgfpathlineto{\pgfqpoint{2.333129in}{5.553223in}}%
\pgfpathlineto{\pgfqpoint{2.348308in}{5.560903in}}%
\pgfpathlineto{\pgfqpoint{2.362702in}{5.568095in}}%
\pgfpathlineto{\pgfqpoint{2.392275in}{5.582759in}}%
\pgfpathlineto{\pgfqpoint{2.401242in}{5.587163in}}%
\pgfpathlineto{\pgfqpoint{2.421849in}{5.597158in}}%
\pgfpathlineto{\pgfqpoint{2.451422in}{5.611415in}}%
\pgfpathlineto{\pgfqpoint{2.455632in}{5.613422in}}%
\pgfpathlineto{\pgfqpoint{2.480996in}{5.625364in}}%
\pgfpathlineto{\pgfqpoint{2.510569in}{5.639220in}}%
\pgfpathlineto{\pgfqpoint{2.511566in}{5.639682in}}%
\pgfpathlineto{\pgfqpoint{2.540142in}{5.652742in}}%
\pgfpathlineto{\pgfqpoint{2.569147in}{5.665941in}}%
\pgfpathlineto{\pgfqpoint{2.569716in}{5.666197in}}%
\pgfpathlineto{\pgfqpoint{2.599289in}{5.679318in}}%
\pgfpathlineto{\pgfqpoint{2.628444in}{5.692201in}}%
\pgfpathlineto{\pgfqpoint{2.628862in}{5.692383in}}%
\pgfpathlineto{\pgfqpoint{2.658436in}{5.705119in}}%
\pgfpathlineto{\pgfqpoint{2.688009in}{5.717801in}}%
\pgfpathlineto{\pgfqpoint{2.689565in}{5.718460in}}%
\pgfpathlineto{\pgfqpoint{2.717582in}{5.730170in}}%
\pgfpathlineto{\pgfqpoint{2.747156in}{5.742469in}}%
\pgfpathlineto{\pgfqpoint{2.752632in}{5.744720in}}%
\pgfpathlineto{\pgfqpoint{2.776729in}{5.754495in}}%
\pgfpathlineto{\pgfqpoint{2.806303in}{5.766415in}}%
\pgfpathlineto{\pgfqpoint{2.817744in}{5.770979in}}%
\pgfpathlineto{\pgfqpoint{2.835876in}{5.778118in}}%
\pgfpathlineto{\pgfqpoint{2.865449in}{5.789663in}}%
\pgfpathlineto{\pgfqpoint{2.885014in}{5.797238in}}%
\pgfusepath{stroke}%
\end{pgfscope}%
\begin{pgfscope}%
\pgfpathrectangle{\pgfqpoint{0.854460in}{0.571603in}}{\pgfqpoint{5.885100in}{5.225635in}}%
\pgfusepath{clip}%
\pgfsetbuttcap%
\pgfsetroundjoin%
\pgfsetlinewidth{1.505625pt}%
\definecolor{currentstroke}{rgb}{0.146180,0.515413,0.556823}%
\pgfsetstrokecolor{currentstroke}%
\pgfsetdash{}{0pt}%
\pgfpathmoveto{\pgfqpoint{5.712248in}{5.797238in}}%
\pgfpathlineto{\pgfqpoint{5.720805in}{5.770979in}}%
\pgfpathlineto{\pgfqpoint{5.728054in}{5.744720in}}%
\pgfpathlineto{\pgfqpoint{5.734110in}{5.718460in}}%
\pgfpathlineto{\pgfqpoint{5.742908in}{5.665941in}}%
\pgfpathlineto{\pgfqpoint{5.748150in}{5.613422in}}%
\pgfpathlineto{\pgfqpoint{5.750421in}{5.560903in}}%
\pgfpathlineto{\pgfqpoint{5.750206in}{5.508384in}}%
\pgfpathlineto{\pgfqpoint{5.747915in}{5.455865in}}%
\pgfpathlineto{\pgfqpoint{5.743893in}{5.403346in}}%
\pgfpathlineto{\pgfqpoint{5.735248in}{5.324568in}}%
\pgfpathlineto{\pgfqpoint{5.724056in}{5.245790in}}%
\pgfpathlineto{\pgfqpoint{5.706472in}{5.140752in}}%
\pgfpathlineto{\pgfqpoint{5.681692in}{5.009454in}}%
\pgfpathlineto{\pgfqpoint{5.595356in}{4.563043in}}%
\pgfpathlineto{\pgfqpoint{5.573702in}{4.431746in}}%
\pgfpathlineto{\pgfqpoint{5.558628in}{4.326708in}}%
\pgfpathlineto{\pgfqpoint{5.545798in}{4.221670in}}%
\pgfpathlineto{\pgfqpoint{5.535562in}{4.116632in}}%
\pgfpathlineto{\pgfqpoint{5.528140in}{4.011594in}}%
\pgfpathlineto{\pgfqpoint{5.524504in}{3.932816in}}%
\pgfpathlineto{\pgfqpoint{5.522630in}{3.854037in}}%
\pgfpathlineto{\pgfqpoint{5.522596in}{3.775259in}}%
\pgfpathlineto{\pgfqpoint{5.524466in}{3.696481in}}%
\pgfpathlineto{\pgfqpoint{5.528291in}{3.617702in}}%
\pgfpathlineto{\pgfqpoint{5.534097in}{3.538924in}}%
\pgfpathlineto{\pgfqpoint{5.541972in}{3.460145in}}%
\pgfpathlineto{\pgfqpoint{5.551975in}{3.381367in}}%
\pgfpathlineto{\pgfqpoint{5.564095in}{3.302589in}}%
\pgfpathlineto{\pgfqpoint{5.578401in}{3.223810in}}%
\pgfpathlineto{\pgfqpoint{5.594919in}{3.145032in}}%
\pgfpathlineto{\pgfqpoint{5.615772in}{3.058240in}}%
\pgfpathlineto{\pgfqpoint{5.634710in}{2.987475in}}%
\pgfpathlineto{\pgfqpoint{5.658033in}{2.908696in}}%
\pgfpathlineto{\pgfqpoint{5.683680in}{2.829918in}}%
\pgfpathlineto{\pgfqpoint{5.711661in}{2.751140in}}%
\pgfpathlineto{\pgfqpoint{5.741988in}{2.672361in}}%
\pgfpathlineto{\pgfqpoint{5.774680in}{2.593583in}}%
\pgfpathlineto{\pgfqpoint{5.809755in}{2.514804in}}%
\pgfpathlineto{\pgfqpoint{5.852359in}{2.425726in}}%
\pgfpathlineto{\pgfqpoint{5.887104in}{2.357248in}}%
\pgfpathlineto{\pgfqpoint{5.929366in}{2.278469in}}%
\pgfpathlineto{\pgfqpoint{5.974072in}{2.199691in}}%
\pgfpathlineto{\pgfqpoint{6.021164in}{2.120912in}}%
\pgfpathlineto{\pgfqpoint{6.068671in}{2.045293in}}%
\pgfpathlineto{\pgfqpoint{6.068671in}{2.045293in}}%
\pgfusepath{stroke}%
\end{pgfscope}%
\begin{pgfscope}%
\pgfpathrectangle{\pgfqpoint{0.854460in}{0.571603in}}{\pgfqpoint{5.885100in}{5.225635in}}%
\pgfusepath{clip}%
\pgfsetbuttcap%
\pgfsetroundjoin%
\pgfsetlinewidth{1.505625pt}%
\definecolor{currentstroke}{rgb}{0.146180,0.515413,0.556823}%
\pgfsetstrokecolor{currentstroke}%
\pgfsetdash{}{0pt}%
\pgfpathmoveto{\pgfqpoint{6.291718in}{1.728712in}}%
\pgfpathlineto{\pgfqpoint{6.293007in}{1.727020in}}%
\pgfpathlineto{\pgfqpoint{6.295960in}{1.723200in}}%
\pgfpathlineto{\pgfqpoint{6.313258in}{1.700761in}}%
\pgfpathlineto{\pgfqpoint{6.325533in}{1.685087in}}%
\pgfpathlineto{\pgfqpoint{6.333802in}{1.674501in}}%
\pgfpathlineto{\pgfqpoint{6.354633in}{1.648242in}}%
\pgfpathlineto{\pgfqpoint{6.355107in}{1.647652in}}%
\pgfpathlineto{\pgfqpoint{6.375682in}{1.621982in}}%
\pgfpathlineto{\pgfqpoint{6.384680in}{1.610922in}}%
\pgfpathlineto{\pgfqpoint{6.397019in}{1.595723in}}%
\pgfpathlineto{\pgfqpoint{6.414253in}{1.574802in}}%
\pgfpathlineto{\pgfqpoint{6.418642in}{1.569463in}}%
\pgfpathlineto{\pgfqpoint{6.440521in}{1.543204in}}%
\pgfpathlineto{\pgfqpoint{6.443827in}{1.539287in}}%
\pgfpathlineto{\pgfqpoint{6.462646in}{1.516944in}}%
\pgfpathlineto{\pgfqpoint{6.473400in}{1.504353in}}%
\pgfpathlineto{\pgfqpoint{6.485053in}{1.490685in}}%
\pgfpathlineto{\pgfqpoint{6.502973in}{1.469951in}}%
\pgfpathlineto{\pgfqpoint{6.507741in}{1.464425in}}%
\pgfpathlineto{\pgfqpoint{6.530690in}{1.438166in}}%
\pgfpathlineto{\pgfqpoint{6.532547in}{1.436066in}}%
\pgfpathlineto{\pgfqpoint{6.553877in}{1.411906in}}%
\pgfpathlineto{\pgfqpoint{6.562120in}{1.402690in}}%
\pgfpathlineto{\pgfqpoint{6.577341in}{1.385647in}}%
\pgfpathlineto{\pgfqpoint{6.591693in}{1.369779in}}%
\pgfpathlineto{\pgfqpoint{6.601081in}{1.359388in}}%
\pgfpathlineto{\pgfqpoint{6.621267in}{1.337316in}}%
\pgfpathlineto{\pgfqpoint{6.625093in}{1.333128in}}%
\pgfpathlineto{\pgfqpoint{6.649365in}{1.306869in}}%
\pgfpathlineto{\pgfqpoint{6.650840in}{1.305289in}}%
\pgfpathlineto{\pgfqpoint{6.673876in}{1.280609in}}%
\pgfpathlineto{\pgfqpoint{6.680414in}{1.273685in}}%
\pgfpathlineto{\pgfqpoint{6.698658in}{1.254350in}}%
\pgfpathlineto{\pgfqpoint{6.709987in}{1.242478in}}%
\pgfpathlineto{\pgfqpoint{6.723709in}{1.228090in}}%
\pgfpathlineto{\pgfqpoint{6.739560in}{1.211653in}}%
\pgfusepath{stroke}%
\end{pgfscope}%
\begin{pgfscope}%
\pgfpathrectangle{\pgfqpoint{0.854460in}{0.571603in}}{\pgfqpoint{5.885100in}{5.225635in}}%
\pgfusepath{clip}%
\pgfsetbuttcap%
\pgfsetroundjoin%
\pgfsetlinewidth{1.505625pt}%
\definecolor{currentstroke}{rgb}{0.139147,0.533812,0.555298}%
\pgfsetstrokecolor{currentstroke}%
\pgfsetdash{}{0pt}%
\pgfpathmoveto{\pgfqpoint{1.343822in}{0.571603in}}%
\pgfpathlineto{\pgfqpoint{1.327634in}{0.586725in}}%
\pgfpathlineto{\pgfqpoint{1.316287in}{0.597376in}}%
\pgfusepath{stroke}%
\end{pgfscope}%
\begin{pgfscope}%
\pgfpathrectangle{\pgfqpoint{0.854460in}{0.571603in}}{\pgfqpoint{5.885100in}{5.225635in}}%
\pgfusepath{clip}%
\pgfsetbuttcap%
\pgfsetroundjoin%
\pgfsetlinewidth{1.505625pt}%
\definecolor{currentstroke}{rgb}{0.139147,0.533812,0.555298}%
\pgfsetstrokecolor{currentstroke}%
\pgfsetdash{}{0pt}%
\pgfpathmoveto{\pgfqpoint{1.050159in}{0.875342in}}%
\pgfpathlineto{\pgfqpoint{1.040371in}{0.886717in}}%
\pgfpathlineto{\pgfqpoint{1.031901in}{0.896707in}}%
\pgfpathlineto{\pgfqpoint{1.018198in}{0.912976in}}%
\pgfpathlineto{\pgfqpoint{1.002327in}{0.932101in}}%
\pgfpathlineto{\pgfqpoint{0.996447in}{0.939236in}}%
\pgfpathlineto{\pgfqpoint{0.975140in}{0.965495in}}%
\pgfpathlineto{\pgfqpoint{0.972754in}{0.968487in}}%
\pgfpathlineto{\pgfqpoint{0.954323in}{0.991755in}}%
\pgfpathlineto{\pgfqpoint{0.943181in}{1.006034in}}%
\pgfpathlineto{\pgfqpoint{0.933900in}{1.018014in}}%
\pgfpathlineto{\pgfqpoint{0.913865in}{1.044274in}}%
\pgfpathlineto{\pgfqpoint{0.913607in}{1.044619in}}%
\pgfpathlineto{\pgfqpoint{0.894341in}{1.070533in}}%
\pgfpathlineto{\pgfqpoint{0.884034in}{1.084612in}}%
\pgfpathlineto{\pgfqpoint{0.875184in}{1.096793in}}%
\pgfpathlineto{\pgfqpoint{0.856410in}{1.123052in}}%
\pgfpathlineto{\pgfqpoint{0.854460in}{1.125830in}}%
\pgfusepath{stroke}%
\end{pgfscope}%
\begin{pgfscope}%
\pgfpathrectangle{\pgfqpoint{0.854460in}{0.571603in}}{\pgfqpoint{5.885100in}{5.225635in}}%
\pgfusepath{clip}%
\pgfsetbuttcap%
\pgfsetroundjoin%
\pgfsetlinewidth{1.505625pt}%
\definecolor{currentstroke}{rgb}{0.139147,0.533812,0.555298}%
\pgfsetstrokecolor{currentstroke}%
\pgfsetdash{}{0pt}%
\pgfpathmoveto{\pgfqpoint{0.854460in}{4.406469in}}%
\pgfpathlineto{\pgfqpoint{0.872412in}{4.431746in}}%
\pgfpathlineto{\pgfqpoint{0.883859in}{4.447642in}}%
\pgfusepath{stroke}%
\end{pgfscope}%
\begin{pgfscope}%
\pgfpathrectangle{\pgfqpoint{0.854460in}{0.571603in}}{\pgfqpoint{5.885100in}{5.225635in}}%
\pgfusepath{clip}%
\pgfsetbuttcap%
\pgfsetroundjoin%
\pgfsetlinewidth{1.505625pt}%
\definecolor{currentstroke}{rgb}{0.139147,0.533812,0.555298}%
\pgfsetstrokecolor{currentstroke}%
\pgfsetdash{}{0pt}%
\pgfpathmoveto{\pgfqpoint{1.129692in}{4.745129in}}%
\pgfpathlineto{\pgfqpoint{1.131303in}{4.746860in}}%
\pgfpathlineto{\pgfqpoint{1.150194in}{4.766893in}}%
\pgfpathlineto{\pgfqpoint{1.156130in}{4.773119in}}%
\pgfpathlineto{\pgfqpoint{1.179767in}{4.797602in}}%
\pgfpathlineto{\pgfqpoint{1.181502in}{4.799378in}}%
\pgfpathlineto{\pgfqpoint{1.207465in}{4.825638in}}%
\pgfpathlineto{\pgfqpoint{1.209341in}{4.827510in}}%
\pgfpathlineto{\pgfqpoint{1.234025in}{4.851897in}}%
\pgfpathlineto{\pgfqpoint{1.238914in}{4.856668in}}%
\pgfpathlineto{\pgfqpoint{1.261166in}{4.878157in}}%
\pgfpathlineto{\pgfqpoint{1.268488in}{4.885140in}}%
\pgfpathlineto{\pgfqpoint{1.288905in}{4.904416in}}%
\pgfpathlineto{\pgfqpoint{1.298061in}{4.912954in}}%
\pgfpathlineto{\pgfqpoint{1.317257in}{4.930676in}}%
\pgfpathlineto{\pgfqpoint{1.327634in}{4.940139in}}%
\pgfpathlineto{\pgfqpoint{1.346238in}{4.956935in}}%
\pgfpathlineto{\pgfqpoint{1.357208in}{4.966719in}}%
\pgfpathlineto{\pgfqpoint{1.375862in}{4.983195in}}%
\pgfpathlineto{\pgfqpoint{1.386781in}{4.992721in}}%
\pgfpathlineto{\pgfqpoint{1.406146in}{5.009454in}}%
\pgfpathlineto{\pgfqpoint{1.416354in}{5.018168in}}%
\pgfpathlineto{\pgfqpoint{1.437104in}{5.035714in}}%
\pgfpathlineto{\pgfqpoint{1.445928in}{5.043085in}}%
\pgfpathlineto{\pgfqpoint{1.468751in}{5.061973in}}%
\pgfpathlineto{\pgfqpoint{1.475501in}{5.067492in}}%
\pgfpathlineto{\pgfqpoint{1.501101in}{5.088233in}}%
\pgfpathlineto{\pgfqpoint{1.505074in}{5.091413in}}%
\pgfpathlineto{\pgfqpoint{1.534170in}{5.114492in}}%
\pgfpathlineto{\pgfqpoint{1.534648in}{5.114867in}}%
\pgfpathlineto{\pgfqpoint{1.564221in}{5.137816in}}%
\pgfpathlineto{\pgfqpoint{1.568038in}{5.140752in}}%
\pgfpathlineto{\pgfqpoint{1.593795in}{5.160323in}}%
\pgfpathlineto{\pgfqpoint{1.602673in}{5.167011in}}%
\pgfpathlineto{\pgfqpoint{1.623368in}{5.182413in}}%
\pgfpathlineto{\pgfqpoint{1.638081in}{5.193271in}}%
\pgfpathlineto{\pgfqpoint{1.652941in}{5.204105in}}%
\pgfpathlineto{\pgfqpoint{1.674274in}{5.219530in}}%
\pgfpathlineto{\pgfqpoint{1.682515in}{5.225417in}}%
\pgfpathlineto{\pgfqpoint{1.711265in}{5.245790in}}%
\pgfpathlineto{\pgfqpoint{1.712088in}{5.246366in}}%
\pgfpathlineto{\pgfqpoint{1.741661in}{5.266865in}}%
\pgfpathlineto{\pgfqpoint{1.749201in}{5.272049in}}%
\pgfpathlineto{\pgfqpoint{1.771235in}{5.287016in}}%
\pgfpathlineto{\pgfqpoint{1.787988in}{5.298308in}}%
\pgfpathlineto{\pgfqpoint{1.800808in}{5.306845in}}%
\pgfpathlineto{\pgfqpoint{1.827622in}{5.324568in}}%
\pgfpathlineto{\pgfqpoint{1.830381in}{5.326370in}}%
\pgfpathlineto{\pgfqpoint{1.859955in}{5.345497in}}%
\pgfpathlineto{\pgfqpoint{1.868264in}{5.350827in}}%
\pgfpathlineto{\pgfqpoint{1.889528in}{5.364304in}}%
\pgfpathlineto{\pgfqpoint{1.909840in}{5.377087in}}%
\pgfpathlineto{\pgfqpoint{1.919102in}{5.382845in}}%
\pgfpathlineto{\pgfqpoint{1.948675in}{5.401086in}}%
\pgfpathlineto{\pgfqpoint{1.952375in}{5.403346in}}%
\pgfpathlineto{\pgfqpoint{1.978248in}{5.418963in}}%
\pgfpathlineto{\pgfqpoint{1.996000in}{5.429606in}}%
\pgfpathlineto{\pgfqpoint{2.007822in}{5.436608in}}%
\pgfpathlineto{\pgfqpoint{2.037395in}{5.453995in}}%
\pgfpathlineto{\pgfqpoint{2.040607in}{5.455865in}}%
\pgfpathlineto{\pgfqpoint{2.066968in}{5.471026in}}%
\pgfpathlineto{\pgfqpoint{2.086386in}{5.482125in}}%
\pgfpathlineto{\pgfqpoint{2.096542in}{5.487858in}}%
\pgfpathlineto{\pgfqpoint{2.126115in}{5.504423in}}%
\pgfpathlineto{\pgfqpoint{2.133251in}{5.508384in}}%
\pgfpathlineto{\pgfqpoint{2.155689in}{5.520688in}}%
\pgfpathlineto{\pgfqpoint{2.181285in}{5.534644in}}%
\pgfpathlineto{\pgfqpoint{2.185262in}{5.536785in}}%
\pgfpathlineto{\pgfqpoint{2.214835in}{5.552555in}}%
\pgfpathlineto{\pgfqpoint{2.230596in}{5.560903in}}%
\pgfpathlineto{\pgfqpoint{2.244409in}{5.568130in}}%
\pgfpathlineto{\pgfqpoint{2.273982in}{5.583490in}}%
\pgfpathlineto{\pgfqpoint{2.281118in}{5.587163in}}%
\pgfpathlineto{\pgfqpoint{2.303555in}{5.598567in}}%
\pgfpathlineto{\pgfqpoint{2.332928in}{5.613422in}}%
\pgfpathlineto{\pgfqpoint{2.333129in}{5.613523in}}%
\pgfpathlineto{\pgfqpoint{2.362702in}{5.628127in}}%
\pgfpathlineto{\pgfqpoint{2.386210in}{5.639682in}}%
\pgfpathlineto{\pgfqpoint{2.392275in}{5.642625in}}%
\pgfpathlineto{\pgfqpoint{2.421849in}{5.656839in}}%
\pgfpathlineto{\pgfqpoint{2.440904in}{5.665941in}}%
\pgfpathlineto{\pgfqpoint{2.451422in}{5.670902in}}%
\pgfpathlineto{\pgfqpoint{2.480996in}{5.684730in}}%
\pgfpathlineto{\pgfqpoint{2.497088in}{5.692201in}}%
\pgfpathlineto{\pgfqpoint{2.510569in}{5.698380in}}%
\pgfpathlineto{\pgfqpoint{2.540142in}{5.711827in}}%
\pgfpathlineto{\pgfqpoint{2.554842in}{5.718460in}}%
\pgfpathlineto{\pgfqpoint{2.569716in}{5.725086in}}%
\pgfpathlineto{\pgfqpoint{2.599289in}{5.738157in}}%
\pgfpathlineto{\pgfqpoint{2.614253in}{5.744720in}}%
\pgfpathlineto{\pgfqpoint{2.628862in}{5.751044in}}%
\pgfpathlineto{\pgfqpoint{2.658436in}{5.763744in}}%
\pgfpathlineto{\pgfqpoint{2.675409in}{5.770979in}}%
\pgfpathlineto{\pgfqpoint{2.688009in}{5.776281in}}%
\pgfpathlineto{\pgfqpoint{2.717582in}{5.788614in}}%
\pgfpathlineto{\pgfqpoint{2.738400in}{5.797238in}}%
\pgfusepath{stroke}%
\end{pgfscope}%
\begin{pgfscope}%
\pgfpathrectangle{\pgfqpoint{0.854460in}{0.571603in}}{\pgfqpoint{5.885100in}{5.225635in}}%
\pgfusepath{clip}%
\pgfsetbuttcap%
\pgfsetroundjoin%
\pgfsetlinewidth{1.505625pt}%
\definecolor{currentstroke}{rgb}{0.139147,0.533812,0.555298}%
\pgfsetstrokecolor{currentstroke}%
\pgfsetdash{}{0pt}%
\pgfpathmoveto{\pgfqpoint{5.874289in}{5.797238in}}%
\pgfpathlineto{\pgfqpoint{5.878766in}{5.770979in}}%
\pgfpathlineto{\pgfqpoint{5.881933in}{5.747098in}}%
\pgfpathlineto{\pgfqpoint{5.882237in}{5.744720in}}%
\pgfpathlineto{\pgfqpoint{5.884756in}{5.718460in}}%
\pgfpathlineto{\pgfqpoint{5.886455in}{5.692201in}}%
\pgfpathlineto{\pgfqpoint{5.887402in}{5.665941in}}%
\pgfpathlineto{\pgfqpoint{5.887663in}{5.639682in}}%
\pgfpathlineto{\pgfqpoint{5.887295in}{5.613422in}}%
\pgfpathlineto{\pgfqpoint{5.886352in}{5.587163in}}%
\pgfpathlineto{\pgfqpoint{5.884883in}{5.560903in}}%
\pgfpathlineto{\pgfqpoint{5.882934in}{5.534644in}}%
\pgfpathlineto{\pgfqpoint{5.881933in}{5.523728in}}%
\pgfpathlineto{\pgfqpoint{5.880527in}{5.508384in}}%
\pgfpathlineto{\pgfqpoint{5.877700in}{5.482125in}}%
\pgfpathlineto{\pgfqpoint{5.874505in}{5.455865in}}%
\pgfpathlineto{\pgfqpoint{5.870975in}{5.429606in}}%
\pgfpathlineto{\pgfqpoint{5.867141in}{5.403346in}}%
\pgfpathlineto{\pgfqpoint{5.863034in}{5.377087in}}%
\pgfpathlineto{\pgfqpoint{5.858678in}{5.350827in}}%
\pgfpathlineto{\pgfqpoint{5.854101in}{5.324568in}}%
\pgfpathlineto{\pgfqpoint{5.852359in}{5.315073in}}%
\pgfpathlineto{\pgfqpoint{5.849284in}{5.298308in}}%
\pgfpathlineto{\pgfqpoint{5.844266in}{5.272049in}}%
\pgfpathlineto{\pgfqpoint{5.839091in}{5.245790in}}%
\pgfpathlineto{\pgfqpoint{5.833778in}{5.219530in}}%
\pgfpathlineto{\pgfqpoint{5.828344in}{5.193271in}}%
\pgfpathlineto{\pgfqpoint{5.822808in}{5.167011in}}%
\pgfpathlineto{\pgfqpoint{5.822786in}{5.166911in}}%
\pgfpathlineto{\pgfqpoint{5.817115in}{5.140752in}}%
\pgfpathlineto{\pgfqpoint{5.811349in}{5.114492in}}%
\pgfpathlineto{\pgfqpoint{5.805527in}{5.088233in}}%
\pgfpathlineto{\pgfqpoint{5.799660in}{5.061973in}}%
\pgfpathlineto{\pgfqpoint{5.793761in}{5.035714in}}%
\pgfpathlineto{\pgfqpoint{5.793213in}{5.033304in}}%
\pgfpathlineto{\pgfqpoint{5.787778in}{5.009454in}}%
\pgfpathlineto{\pgfqpoint{5.781782in}{4.983195in}}%
\pgfpathlineto{\pgfqpoint{5.775788in}{4.956935in}}%
\pgfpathlineto{\pgfqpoint{5.769807in}{4.930676in}}%
\pgfpathlineto{\pgfqpoint{5.763849in}{4.904416in}}%
\pgfpathlineto{\pgfqpoint{5.763639in}{4.903493in}}%
\pgfpathlineto{\pgfqpoint{5.757857in}{4.878157in}}%
\pgfpathlineto{\pgfqpoint{5.751904in}{4.851897in}}%
\pgfpathlineto{\pgfqpoint{5.746001in}{4.825638in}}%
\pgfpathlineto{\pgfqpoint{5.740156in}{4.799378in}}%
\pgfpathlineto{\pgfqpoint{5.734374in}{4.773119in}}%
\pgfpathlineto{\pgfqpoint{5.734066in}{4.771710in}}%
\pgfpathlineto{\pgfqpoint{5.728605in}{4.746860in}}%
\pgfpathlineto{\pgfqpoint{5.722912in}{4.720600in}}%
\pgfpathlineto{\pgfqpoint{5.717306in}{4.694341in}}%
\pgfpathlineto{\pgfqpoint{5.711792in}{4.668081in}}%
\pgfpathlineto{\pgfqpoint{5.708198in}{4.650654in}}%
\pgfusepath{stroke}%
\end{pgfscope}%
\begin{pgfscope}%
\pgfpathrectangle{\pgfqpoint{0.854460in}{0.571603in}}{\pgfqpoint{5.885100in}{5.225635in}}%
\pgfusepath{clip}%
\pgfsetbuttcap%
\pgfsetroundjoin%
\pgfsetlinewidth{1.505625pt}%
\definecolor{currentstroke}{rgb}{0.139147,0.533812,0.555298}%
\pgfsetstrokecolor{currentstroke}%
\pgfsetdash{}{0pt}%
\pgfpathmoveto{\pgfqpoint{5.641245in}{4.264062in}}%
\pgfpathlineto{\pgfqpoint{5.629644in}{4.169151in}}%
\pgfpathlineto{\pgfqpoint{5.619550in}{4.064113in}}%
\pgfpathlineto{\pgfqpoint{5.613938in}{3.985335in}}%
\pgfpathlineto{\pgfqpoint{5.610075in}{3.906556in}}%
\pgfpathlineto{\pgfqpoint{5.608061in}{3.827778in}}%
\pgfpathlineto{\pgfqpoint{5.607956in}{3.749000in}}%
\pgfpathlineto{\pgfqpoint{5.609819in}{3.670221in}}%
\pgfpathlineto{\pgfqpoint{5.613705in}{3.591443in}}%
\pgfpathlineto{\pgfqpoint{5.619637in}{3.512664in}}%
\pgfpathlineto{\pgfqpoint{5.627666in}{3.433886in}}%
\pgfpathlineto{\pgfqpoint{5.637870in}{3.355107in}}%
\pgfpathlineto{\pgfqpoint{5.650261in}{3.276329in}}%
\pgfpathlineto{\pgfqpoint{5.664855in}{3.197551in}}%
\pgfpathlineto{\pgfqpoint{5.681718in}{3.118772in}}%
\pgfpathlineto{\pgfqpoint{5.700864in}{3.039994in}}%
\pgfpathlineto{\pgfqpoint{5.722287in}{2.961215in}}%
\pgfpathlineto{\pgfqpoint{5.746055in}{2.882437in}}%
\pgfpathlineto{\pgfqpoint{5.772175in}{2.803659in}}%
\pgfpathlineto{\pgfqpoint{5.800658in}{2.724880in}}%
\pgfpathlineto{\pgfqpoint{5.831516in}{2.646102in}}%
\pgfpathlineto{\pgfqpoint{5.864767in}{2.567323in}}%
\pgfpathlineto{\pgfqpoint{5.900430in}{2.488545in}}%
\pgfpathlineto{\pgfqpoint{5.941079in}{2.404717in}}%
\pgfpathlineto{\pgfqpoint{5.979014in}{2.330988in}}%
\pgfpathlineto{\pgfqpoint{6.029799in}{2.238361in}}%
\pgfpathlineto{\pgfqpoint{6.067317in}{2.173431in}}%
\pgfpathlineto{\pgfqpoint{6.118520in}{2.089275in}}%
\pgfpathlineto{\pgfqpoint{6.165371in}{2.015874in}}%
\pgfpathlineto{\pgfqpoint{6.218066in}{1.937096in}}%
\pgfpathlineto{\pgfqpoint{6.273209in}{1.858318in}}%
\pgfpathlineto{\pgfqpoint{6.330790in}{1.779539in}}%
\pgfpathlineto{\pgfqpoint{6.390803in}{1.700761in}}%
\pgfpathlineto{\pgfqpoint{6.453246in}{1.621982in}}%
\pgfpathlineto{\pgfqpoint{6.532547in}{1.526130in}}%
\pgfpathlineto{\pgfqpoint{6.591693in}{1.457274in}}%
\pgfpathlineto{\pgfqpoint{6.655156in}{1.385647in}}%
\pgfpathlineto{\pgfqpoint{6.739560in}{1.293737in}}%
\pgfpathlineto{\pgfqpoint{6.739560in}{1.293737in}}%
\pgfusepath{stroke}%
\end{pgfscope}%
\begin{pgfscope}%
\pgfpathrectangle{\pgfqpoint{0.854460in}{0.571603in}}{\pgfqpoint{5.885100in}{5.225635in}}%
\pgfusepath{clip}%
\pgfsetbuttcap%
\pgfsetroundjoin%
\pgfsetlinewidth{1.505625pt}%
\definecolor{currentstroke}{rgb}{0.131172,0.555899,0.552459}%
\pgfsetstrokecolor{currentstroke}%
\pgfsetdash{}{0pt}%
\pgfpathmoveto{\pgfqpoint{1.281021in}{0.571603in}}%
\pgfpathlineto{\pgfqpoint{1.268488in}{0.583413in}}%
\pgfpathlineto{\pgfqpoint{1.256686in}{0.594588in}}%
\pgfusepath{stroke}%
\end{pgfscope}%
\begin{pgfscope}%
\pgfpathrectangle{\pgfqpoint{0.854460in}{0.571603in}}{\pgfqpoint{5.885100in}{5.225635in}}%
\pgfusepath{clip}%
\pgfsetbuttcap%
\pgfsetroundjoin%
\pgfsetlinewidth{1.505625pt}%
\definecolor{currentstroke}{rgb}{0.131172,0.555899,0.552459}%
\pgfsetstrokecolor{currentstroke}%
\pgfsetdash{}{0pt}%
\pgfpathmoveto{\pgfqpoint{0.991526in}{0.873556in}}%
\pgfpathlineto{\pgfqpoint{0.980299in}{0.886717in}}%
\pgfpathlineto{\pgfqpoint{0.972754in}{0.895692in}}%
\pgfpathlineto{\pgfqpoint{0.958322in}{0.912976in}}%
\pgfpathlineto{\pgfqpoint{0.943181in}{0.931379in}}%
\pgfpathlineto{\pgfqpoint{0.936761in}{0.939236in}}%
\pgfpathlineto{\pgfqpoint{0.915633in}{0.965495in}}%
\pgfpathlineto{\pgfqpoint{0.913607in}{0.968056in}}%
\pgfpathlineto{\pgfqpoint{0.894995in}{0.991755in}}%
\pgfpathlineto{\pgfqpoint{0.884034in}{1.005922in}}%
\pgfpathlineto{\pgfqpoint{0.874746in}{1.018014in}}%
\pgfpathlineto{\pgfqpoint{0.854881in}{1.044274in}}%
\pgfpathlineto{\pgfqpoint{0.854460in}{1.044840in}}%
\pgfusepath{stroke}%
\end{pgfscope}%
\begin{pgfscope}%
\pgfpathrectangle{\pgfqpoint{0.854460in}{0.571603in}}{\pgfqpoint{5.885100in}{5.225635in}}%
\pgfusepath{clip}%
\pgfsetbuttcap%
\pgfsetroundjoin%
\pgfsetlinewidth{1.505625pt}%
\definecolor{currentstroke}{rgb}{0.131172,0.555899,0.552459}%
\pgfsetstrokecolor{currentstroke}%
\pgfsetdash{}{0pt}%
\pgfpathmoveto{\pgfqpoint{0.854460in}{4.503619in}}%
\pgfpathlineto{\pgfqpoint{0.859611in}{4.510524in}}%
\pgfpathlineto{\pgfqpoint{0.879488in}{4.536784in}}%
\pgfpathlineto{\pgfqpoint{0.884034in}{4.542702in}}%
\pgfpathlineto{\pgfqpoint{0.899846in}{4.563043in}}%
\pgfpathlineto{\pgfqpoint{0.913607in}{4.580512in}}%
\pgfpathlineto{\pgfqpoint{0.920614in}{4.589303in}}%
\pgfpathlineto{\pgfqpoint{0.941830in}{4.615562in}}%
\pgfpathlineto{\pgfqpoint{0.943181in}{4.617208in}}%
\pgfpathlineto{\pgfqpoint{0.963598in}{4.641822in}}%
\pgfpathlineto{\pgfqpoint{0.972754in}{4.652716in}}%
\pgfpathlineto{\pgfqpoint{0.985814in}{4.668081in}}%
\pgfpathlineto{\pgfqpoint{1.002327in}{4.687258in}}%
\pgfpathlineto{\pgfqpoint{1.008495in}{4.694341in}}%
\pgfpathlineto{\pgfqpoint{1.031660in}{4.720600in}}%
\pgfpathlineto{\pgfqpoint{1.031901in}{4.720869in}}%
\pgfpathlineto{\pgfqpoint{1.055416in}{4.746860in}}%
\pgfpathlineto{\pgfqpoint{1.061474in}{4.753470in}}%
\pgfpathlineto{\pgfqpoint{1.079675in}{4.773119in}}%
\pgfpathlineto{\pgfqpoint{1.091047in}{4.785242in}}%
\pgfpathlineto{\pgfqpoint{1.104452in}{4.799378in}}%
\pgfpathlineto{\pgfqpoint{1.120621in}{4.816217in}}%
\pgfpathlineto{\pgfqpoint{1.129763in}{4.825638in}}%
\pgfpathlineto{\pgfqpoint{1.150194in}{4.846429in}}%
\pgfpathlineto{\pgfqpoint{1.155624in}{4.851897in}}%
\pgfpathlineto{\pgfqpoint{1.179767in}{4.875911in}}%
\pgfpathlineto{\pgfqpoint{1.182049in}{4.878157in}}%
\pgfpathlineto{\pgfqpoint{1.209060in}{4.904416in}}%
\pgfpathlineto{\pgfqpoint{1.209341in}{4.904686in}}%
\pgfpathlineto{\pgfqpoint{1.236695in}{4.930676in}}%
\pgfpathlineto{\pgfqpoint{1.238914in}{4.932758in}}%
\pgfpathlineto{\pgfqpoint{1.264931in}{4.956935in}}%
\pgfpathlineto{\pgfqpoint{1.268488in}{4.960200in}}%
\pgfpathlineto{\pgfqpoint{1.293783in}{4.983195in}}%
\pgfpathlineto{\pgfqpoint{1.298061in}{4.987036in}}%
\pgfpathlineto{\pgfqpoint{1.323266in}{5.009454in}}%
\pgfpathlineto{\pgfqpoint{1.327634in}{5.013292in}}%
\pgfpathlineto{\pgfqpoint{1.353394in}{5.035714in}}%
\pgfpathlineto{\pgfqpoint{1.357208in}{5.038992in}}%
\pgfpathlineto{\pgfqpoint{1.384183in}{5.061973in}}%
\pgfpathlineto{\pgfqpoint{1.386781in}{5.064160in}}%
\pgfpathlineto{\pgfqpoint{1.415645in}{5.088233in}}%
\pgfpathlineto{\pgfqpoint{1.416354in}{5.088817in}}%
\pgfpathlineto{\pgfqpoint{1.445928in}{5.112955in}}%
\pgfpathlineto{\pgfqpoint{1.447828in}{5.114492in}}%
\pgfpathlineto{\pgfqpoint{1.475501in}{5.136603in}}%
\pgfpathlineto{\pgfqpoint{1.480738in}{5.140752in}}%
\pgfpathlineto{\pgfqpoint{1.505074in}{5.159795in}}%
\pgfpathlineto{\pgfqpoint{1.514376in}{5.167011in}}%
\pgfpathlineto{\pgfqpoint{1.534648in}{5.182549in}}%
\pgfpathlineto{\pgfqpoint{1.548755in}{5.193271in}}%
\pgfpathlineto{\pgfqpoint{1.564221in}{5.204884in}}%
\pgfpathlineto{\pgfqpoint{1.583887in}{5.219530in}}%
\pgfpathlineto{\pgfqpoint{1.593795in}{5.226819in}}%
\pgfpathlineto{\pgfqpoint{1.619787in}{5.245790in}}%
\pgfpathlineto{\pgfqpoint{1.623368in}{5.248372in}}%
\pgfpathlineto{\pgfqpoint{1.652941in}{5.269509in}}%
\pgfpathlineto{\pgfqpoint{1.656526in}{5.272049in}}%
\pgfpathlineto{\pgfqpoint{1.682515in}{5.290238in}}%
\pgfpathlineto{\pgfqpoint{1.694133in}{5.298308in}}%
\pgfpathlineto{\pgfqpoint{1.712088in}{5.310629in}}%
\pgfpathlineto{\pgfqpoint{1.732555in}{5.324568in}}%
\pgfpathlineto{\pgfqpoint{1.741661in}{5.330695in}}%
\pgfpathlineto{\pgfqpoint{1.771235in}{5.350446in}}%
\pgfpathlineto{\pgfqpoint{1.771812in}{5.350827in}}%
\pgfpathlineto{\pgfqpoint{1.800808in}{5.369773in}}%
\pgfpathlineto{\pgfqpoint{1.812082in}{5.377087in}}%
\pgfpathlineto{\pgfqpoint{1.830381in}{5.388815in}}%
\pgfpathlineto{\pgfqpoint{1.853210in}{5.403346in}}%
\pgfpathlineto{\pgfqpoint{1.859955in}{5.407587in}}%
\pgfpathlineto{\pgfqpoint{1.889528in}{5.426031in}}%
\pgfpathlineto{\pgfqpoint{1.895309in}{5.429606in}}%
\pgfpathlineto{\pgfqpoint{1.919102in}{5.444140in}}%
\pgfpathlineto{\pgfqpoint{1.938421in}{5.455865in}}%
\pgfpathlineto{\pgfqpoint{1.948675in}{5.462013in}}%
\pgfpathlineto{\pgfqpoint{1.978248in}{5.479613in}}%
\pgfpathlineto{\pgfqpoint{1.982507in}{5.482125in}}%
\pgfpathlineto{\pgfqpoint{2.007822in}{5.496873in}}%
\pgfpathlineto{\pgfqpoint{2.027700in}{5.508384in}}%
\pgfpathlineto{\pgfqpoint{2.037395in}{5.513930in}}%
\pgfpathlineto{\pgfqpoint{2.066968in}{5.530717in}}%
\pgfpathlineto{\pgfqpoint{2.073944in}{5.534644in}}%
\pgfpathlineto{\pgfqpoint{2.096542in}{5.547207in}}%
\pgfpathlineto{\pgfqpoint{2.121313in}{5.560903in}}%
\pgfpathlineto{\pgfqpoint{2.126115in}{5.563526in}}%
\pgfpathlineto{\pgfqpoint{2.155689in}{5.579529in}}%
\pgfpathlineto{\pgfqpoint{2.169890in}{5.587163in}}%
\pgfpathlineto{\pgfqpoint{2.185262in}{5.595324in}}%
\pgfpathlineto{\pgfqpoint{2.214835in}{5.610926in}}%
\pgfpathlineto{\pgfqpoint{2.219612in}{5.613422in}}%
\pgfpathlineto{\pgfqpoint{2.244409in}{5.626222in}}%
\pgfpathlineto{\pgfqpoint{2.270609in}{5.639682in}}%
\pgfpathlineto{\pgfqpoint{2.273982in}{5.641393in}}%
\pgfpathlineto{\pgfqpoint{2.303555in}{5.656249in}}%
\pgfpathlineto{\pgfqpoint{2.322953in}{5.665941in}}%
\pgfpathlineto{\pgfqpoint{2.333129in}{5.670962in}}%
\pgfpathlineto{\pgfqpoint{2.362702in}{5.685435in}}%
\pgfpathlineto{\pgfqpoint{2.376624in}{5.692201in}}%
\pgfpathlineto{\pgfqpoint{2.392275in}{5.699711in}}%
\pgfpathlineto{\pgfqpoint{2.421849in}{5.713807in}}%
\pgfpathlineto{\pgfqpoint{2.426481in}{5.715997in}}%
\pgfusepath{stroke}%
\end{pgfscope}%
\begin{pgfscope}%
\pgfpathrectangle{\pgfqpoint{0.854460in}{0.571603in}}{\pgfqpoint{5.885100in}{5.225635in}}%
\pgfusepath{clip}%
\pgfsetbuttcap%
\pgfsetroundjoin%
\pgfsetlinewidth{1.505625pt}%
\definecolor{currentstroke}{rgb}{0.131172,0.555899,0.552459}%
\pgfsetstrokecolor{currentstroke}%
\pgfsetdash{}{0pt}%
\pgfpathmoveto{\pgfqpoint{6.022114in}{5.797238in}}%
\pgfpathlineto{\pgfqpoint{6.023430in}{5.770979in}}%
\pgfpathlineto{\pgfqpoint{6.023966in}{5.744720in}}%
\pgfpathlineto{\pgfqpoint{6.023790in}{5.718460in}}%
\pgfpathlineto{\pgfqpoint{6.022961in}{5.692201in}}%
\pgfpathlineto{\pgfqpoint{6.021536in}{5.665941in}}%
\pgfpathlineto{\pgfqpoint{6.019566in}{5.639682in}}%
\pgfpathlineto{\pgfqpoint{6.017098in}{5.613422in}}%
\pgfpathlineto{\pgfqpoint{6.014176in}{5.587163in}}%
\pgfpathlineto{\pgfqpoint{6.010839in}{5.560903in}}%
\pgfpathlineto{\pgfqpoint{6.007124in}{5.534644in}}%
\pgfpathlineto{\pgfqpoint{6.003066in}{5.508384in}}%
\pgfpathlineto{\pgfqpoint{6.000226in}{5.491406in}}%
\pgfpathlineto{\pgfqpoint{5.998675in}{5.482125in}}%
\pgfpathlineto{\pgfqpoint{5.993962in}{5.455865in}}%
\pgfpathlineto{\pgfqpoint{5.988991in}{5.429606in}}%
\pgfpathlineto{\pgfqpoint{5.983790in}{5.403346in}}%
\pgfpathlineto{\pgfqpoint{5.978382in}{5.377087in}}%
\pgfpathlineto{\pgfqpoint{5.972789in}{5.350827in}}%
\pgfpathlineto{\pgfqpoint{5.970653in}{5.341160in}}%
\pgfpathlineto{\pgfqpoint{5.966987in}{5.324568in}}%
\pgfpathlineto{\pgfqpoint{5.961014in}{5.298308in}}%
\pgfpathlineto{\pgfqpoint{5.954915in}{5.272049in}}%
\pgfpathlineto{\pgfqpoint{5.948707in}{5.245790in}}%
\pgfpathlineto{\pgfqpoint{5.942408in}{5.219530in}}%
\pgfpathlineto{\pgfqpoint{5.941079in}{5.214106in}}%
\pgfpathlineto{\pgfqpoint{5.935971in}{5.193271in}}%
\pgfpathlineto{\pgfqpoint{5.930078in}{5.169513in}}%
\pgfusepath{stroke}%
\end{pgfscope}%
\begin{pgfscope}%
\pgfpathrectangle{\pgfqpoint{0.854460in}{0.571603in}}{\pgfqpoint{5.885100in}{5.225635in}}%
\pgfusepath{clip}%
\pgfsetbuttcap%
\pgfsetroundjoin%
\pgfsetlinewidth{1.505625pt}%
\definecolor{currentstroke}{rgb}{0.131172,0.555899,0.552459}%
\pgfsetstrokecolor{currentstroke}%
\pgfsetdash{}{0pt}%
\pgfpathmoveto{\pgfqpoint{5.834333in}{4.789472in}}%
\pgfpathlineto{\pgfqpoint{5.805733in}{4.668081in}}%
\pgfpathlineto{\pgfqpoint{5.777265in}{4.536784in}}%
\pgfpathlineto{\pgfqpoint{5.756775in}{4.431746in}}%
\pgfpathlineto{\pgfqpoint{5.738623in}{4.326708in}}%
\pgfpathlineto{\pgfqpoint{5.723045in}{4.221670in}}%
\pgfpathlineto{\pgfqpoint{5.710316in}{4.116632in}}%
\pgfpathlineto{\pgfqpoint{5.702748in}{4.037854in}}%
\pgfpathlineto{\pgfqpoint{5.696923in}{3.959075in}}%
\pgfpathlineto{\pgfqpoint{5.692956in}{3.880297in}}%
\pgfpathlineto{\pgfqpoint{5.690903in}{3.801519in}}%
\pgfpathlineto{\pgfqpoint{5.690819in}{3.722740in}}%
\pgfpathlineto{\pgfqpoint{5.692758in}{3.643962in}}%
\pgfpathlineto{\pgfqpoint{5.696771in}{3.565183in}}%
\pgfpathlineto{\pgfqpoint{5.702912in}{3.486405in}}%
\pgfpathlineto{\pgfqpoint{5.711173in}{3.407626in}}%
\pgfpathlineto{\pgfqpoint{5.721636in}{3.328848in}}%
\pgfpathlineto{\pgfqpoint{5.734366in}{3.250070in}}%
\pgfpathlineto{\pgfqpoint{5.749293in}{3.171291in}}%
\pgfpathlineto{\pgfqpoint{5.766561in}{3.092513in}}%
\pgfpathlineto{\pgfqpoint{5.786108in}{3.013734in}}%
\pgfpathlineto{\pgfqpoint{5.807996in}{2.934956in}}%
\pgfpathlineto{\pgfqpoint{5.832254in}{2.856177in}}%
\pgfpathlineto{\pgfqpoint{5.858889in}{2.777399in}}%
\pgfpathlineto{\pgfqpoint{5.887913in}{2.698621in}}%
\pgfpathlineto{\pgfqpoint{5.919338in}{2.619842in}}%
\pgfpathlineto{\pgfqpoint{5.953180in}{2.541064in}}%
\pgfpathlineto{\pgfqpoint{5.989460in}{2.462285in}}%
\pgfpathlineto{\pgfqpoint{6.029799in}{2.380396in}}%
\pgfpathlineto{\pgfqpoint{6.069345in}{2.304729in}}%
\pgfpathlineto{\pgfqpoint{6.118520in}{2.216317in}}%
\pgfpathlineto{\pgfqpoint{6.159038in}{2.147172in}}%
\pgfpathlineto{\pgfqpoint{6.207603in}{2.068393in}}%
\pgfpathlineto{\pgfqpoint{6.266386in}{1.977953in}}%
\pgfpathlineto{\pgfqpoint{6.312036in}{1.910836in}}%
\pgfpathlineto{\pgfqpoint{6.367962in}{1.832058in}}%
\pgfpathlineto{\pgfqpoint{6.426349in}{1.753280in}}%
\pgfpathlineto{\pgfqpoint{6.502973in}{1.654630in}}%
\pgfpathlineto{\pgfqpoint{6.562120in}{1.581632in}}%
\pgfpathlineto{\pgfqpoint{6.621267in}{1.511089in}}%
\pgfpathlineto{\pgfqpoint{6.684450in}{1.438166in}}%
\pgfpathlineto{\pgfqpoint{6.739560in}{1.376493in}}%
\pgfpathlineto{\pgfqpoint{6.739560in}{1.376493in}}%
\pgfusepath{stroke}%
\end{pgfscope}%
\begin{pgfscope}%
\pgfpathrectangle{\pgfqpoint{0.854460in}{0.571603in}}{\pgfqpoint{5.885100in}{5.225635in}}%
\pgfusepath{clip}%
\pgfsetbuttcap%
\pgfsetroundjoin%
\pgfsetlinewidth{1.505625pt}%
\definecolor{currentstroke}{rgb}{0.125394,0.574318,0.549086}%
\pgfsetstrokecolor{currentstroke}%
\pgfsetdash{}{0pt}%
\pgfpathmoveto{\pgfqpoint{1.219941in}{0.571603in}}%
\pgfpathlineto{\pgfqpoint{1.209341in}{0.581679in}}%
\pgfpathlineto{\pgfqpoint{1.197001in}{0.593467in}}%
\pgfusepath{stroke}%
\end{pgfscope}%
\begin{pgfscope}%
\pgfpathrectangle{\pgfqpoint{0.854460in}{0.571603in}}{\pgfqpoint{5.885100in}{5.225635in}}%
\pgfusepath{clip}%
\pgfsetbuttcap%
\pgfsetroundjoin%
\pgfsetlinewidth{1.505625pt}%
\definecolor{currentstroke}{rgb}{0.125394,0.574318,0.549086}%
\pgfsetstrokecolor{currentstroke}%
\pgfsetdash{}{0pt}%
\pgfpathmoveto{\pgfqpoint{0.932968in}{0.873593in}}%
\pgfpathlineto{\pgfqpoint{0.921870in}{0.886717in}}%
\pgfpathlineto{\pgfqpoint{0.913607in}{0.896632in}}%
\pgfpathlineto{\pgfqpoint{0.900078in}{0.912976in}}%
\pgfpathlineto{\pgfqpoint{0.884034in}{0.932646in}}%
\pgfpathlineto{\pgfqpoint{0.878696in}{0.939236in}}%
\pgfpathlineto{\pgfqpoint{0.857758in}{0.965495in}}%
\pgfpathlineto{\pgfqpoint{0.854460in}{0.969700in}}%
\pgfusepath{stroke}%
\end{pgfscope}%
\begin{pgfscope}%
\pgfpathrectangle{\pgfqpoint{0.854460in}{0.571603in}}{\pgfqpoint{5.885100in}{5.225635in}}%
\pgfusepath{clip}%
\pgfsetbuttcap%
\pgfsetroundjoin%
\pgfsetlinewidth{1.505625pt}%
\definecolor{currentstroke}{rgb}{0.125394,0.574318,0.549086}%
\pgfsetstrokecolor{currentstroke}%
\pgfsetdash{}{0pt}%
\pgfpathmoveto{\pgfqpoint{0.854460in}{4.593842in}}%
\pgfpathlineto{\pgfqpoint{0.871713in}{4.615562in}}%
\pgfpathlineto{\pgfqpoint{0.884034in}{4.630870in}}%
\pgfpathlineto{\pgfqpoint{0.892949in}{4.641822in}}%
\pgfpathlineto{\pgfqpoint{0.913607in}{4.666866in}}%
\pgfpathlineto{\pgfqpoint{0.914620in}{4.668081in}}%
\pgfpathlineto{\pgfqpoint{0.936847in}{4.694341in}}%
\pgfpathlineto{\pgfqpoint{0.943181in}{4.701725in}}%
\pgfpathlineto{\pgfqpoint{0.959548in}{4.720600in}}%
\pgfpathlineto{\pgfqpoint{0.972754in}{4.735633in}}%
\pgfpathlineto{\pgfqpoint{0.982722in}{4.746860in}}%
\pgfpathlineto{\pgfqpoint{1.002327in}{4.768657in}}%
\pgfpathlineto{\pgfqpoint{1.006384in}{4.773119in}}%
\pgfpathlineto{\pgfqpoint{1.030571in}{4.799378in}}%
\pgfpathlineto{\pgfqpoint{1.031901in}{4.800802in}}%
\pgfpathlineto{\pgfqpoint{1.055335in}{4.825638in}}%
\pgfpathlineto{\pgfqpoint{1.061474in}{4.832062in}}%
\pgfpathlineto{\pgfqpoint{1.080624in}{4.851897in}}%
\pgfpathlineto{\pgfqpoint{1.091047in}{4.862559in}}%
\pgfpathlineto{\pgfqpoint{1.106453in}{4.878157in}}%
\pgfpathlineto{\pgfqpoint{1.120621in}{4.892324in}}%
\pgfpathlineto{\pgfqpoint{1.132836in}{4.904416in}}%
\pgfpathlineto{\pgfqpoint{1.150194in}{4.921386in}}%
\pgfpathlineto{\pgfqpoint{1.159790in}{4.930676in}}%
\pgfpathlineto{\pgfqpoint{1.179767in}{4.949776in}}%
\pgfpathlineto{\pgfqpoint{1.187329in}{4.956935in}}%
\pgfpathlineto{\pgfqpoint{1.209341in}{4.977520in}}%
\pgfpathlineto{\pgfqpoint{1.215467in}{4.983195in}}%
\pgfpathlineto{\pgfqpoint{1.238914in}{5.004646in}}%
\pgfpathlineto{\pgfqpoint{1.244220in}{5.009454in}}%
\pgfpathlineto{\pgfqpoint{1.268488in}{5.031178in}}%
\pgfpathlineto{\pgfqpoint{1.273602in}{5.035714in}}%
\pgfpathlineto{\pgfqpoint{1.298061in}{5.057141in}}%
\pgfpathlineto{\pgfqpoint{1.303627in}{5.061973in}}%
\pgfpathlineto{\pgfqpoint{1.327634in}{5.082560in}}%
\pgfpathlineto{\pgfqpoint{1.334310in}{5.088233in}}%
\pgfpathlineto{\pgfqpoint{1.357208in}{5.107455in}}%
\pgfpathlineto{\pgfqpoint{1.365664in}{5.114492in}}%
\pgfpathlineto{\pgfqpoint{1.386781in}{5.131850in}}%
\pgfpathlineto{\pgfqpoint{1.397704in}{5.140752in}}%
\pgfpathlineto{\pgfqpoint{1.416354in}{5.155765in}}%
\pgfpathlineto{\pgfqpoint{1.430443in}{5.167011in}}%
\pgfpathlineto{\pgfqpoint{1.445928in}{5.179221in}}%
\pgfpathlineto{\pgfqpoint{1.463894in}{5.193271in}}%
\pgfpathlineto{\pgfqpoint{1.475501in}{5.202237in}}%
\pgfpathlineto{\pgfqpoint{1.498069in}{5.219530in}}%
\pgfpathlineto{\pgfqpoint{1.505074in}{5.224833in}}%
\pgfpathlineto{\pgfqpoint{1.532982in}{5.245790in}}%
\pgfpathlineto{\pgfqpoint{1.534648in}{5.247026in}}%
\pgfpathlineto{\pgfqpoint{1.564221in}{5.268769in}}%
\pgfpathlineto{\pgfqpoint{1.568719in}{5.272049in}}%
\pgfpathlineto{\pgfqpoint{1.593795in}{5.290113in}}%
\pgfpathlineto{\pgfqpoint{1.605258in}{5.298308in}}%
\pgfpathlineto{\pgfqpoint{1.623368in}{5.311099in}}%
\pgfpathlineto{\pgfqpoint{1.642582in}{5.324568in}}%
\pgfpathlineto{\pgfqpoint{1.652941in}{5.331742in}}%
\pgfpathlineto{\pgfqpoint{1.680701in}{5.350827in}}%
\pgfpathlineto{\pgfqpoint{1.682515in}{5.352059in}}%
\pgfpathlineto{\pgfqpoint{1.712088in}{5.371965in}}%
\pgfpathlineto{\pgfqpoint{1.719756in}{5.377087in}}%
\pgfpathlineto{\pgfqpoint{1.741661in}{5.391543in}}%
\pgfpathlineto{\pgfqpoint{1.759671in}{5.403346in}}%
\pgfpathlineto{\pgfqpoint{1.771235in}{5.410834in}}%
\pgfpathlineto{\pgfqpoint{1.800423in}{5.429606in}}%
\pgfpathlineto{\pgfqpoint{1.800808in}{5.429851in}}%
\pgfpathlineto{\pgfqpoint{1.830381in}{5.448461in}}%
\pgfpathlineto{\pgfqpoint{1.842224in}{5.455865in}}%
\pgfpathlineto{\pgfqpoint{1.859955in}{5.466815in}}%
\pgfpathlineto{\pgfqpoint{1.884900in}{5.482125in}}%
\pgfpathlineto{\pgfqpoint{1.889528in}{5.484931in}}%
\pgfpathlineto{\pgfqpoint{1.919102in}{5.502708in}}%
\pgfpathlineto{\pgfqpoint{1.928614in}{5.508384in}}%
\pgfpathlineto{\pgfqpoint{1.948675in}{5.520209in}}%
\pgfpathlineto{\pgfqpoint{1.973306in}{5.534644in}}%
\pgfpathlineto{\pgfqpoint{1.978248in}{5.537505in}}%
\pgfpathlineto{\pgfqpoint{1.996419in}{5.547933in}}%
\pgfusepath{stroke}%
\end{pgfscope}%
\begin{pgfscope}%
\pgfpathrectangle{\pgfqpoint{0.854460in}{0.571603in}}{\pgfqpoint{5.885100in}{5.225635in}}%
\pgfusepath{clip}%
\pgfsetbuttcap%
\pgfsetroundjoin%
\pgfsetlinewidth{1.505625pt}%
\definecolor{currentstroke}{rgb}{0.125394,0.574318,0.549086}%
\pgfsetstrokecolor{currentstroke}%
\pgfsetdash{}{0pt}%
\pgfpathmoveto{\pgfqpoint{2.332474in}{5.725605in}}%
\pgfpathlineto{\pgfqpoint{2.333129in}{5.725924in}}%
\pgfpathlineto{\pgfqpoint{2.362702in}{5.740247in}}%
\pgfpathlineto{\pgfqpoint{2.372011in}{5.744720in}}%
\pgfpathlineto{\pgfqpoint{2.392275in}{5.754333in}}%
\pgfpathlineto{\pgfqpoint{2.421849in}{5.768292in}}%
\pgfpathlineto{\pgfqpoint{2.427594in}{5.770979in}}%
\pgfpathlineto{\pgfqpoint{2.451422in}{5.781982in}}%
\pgfpathlineto{\pgfqpoint{2.480996in}{5.795582in}}%
\pgfpathlineto{\pgfqpoint{2.484635in}{5.797238in}}%
\pgfusepath{stroke}%
\end{pgfscope}%
\begin{pgfscope}%
\pgfpathrectangle{\pgfqpoint{0.854460in}{0.571603in}}{\pgfqpoint{5.885100in}{5.225635in}}%
\pgfusepath{clip}%
\pgfsetbuttcap%
\pgfsetroundjoin%
\pgfsetlinewidth{1.505625pt}%
\definecolor{currentstroke}{rgb}{0.125394,0.574318,0.549086}%
\pgfsetstrokecolor{currentstroke}%
\pgfsetdash{}{0pt}%
\pgfpathmoveto{\pgfqpoint{6.158977in}{5.797238in}}%
\pgfpathlineto{\pgfqpoint{6.155981in}{5.744720in}}%
\pgfpathlineto{\pgfqpoint{6.150734in}{5.692201in}}%
\pgfpathlineto{\pgfqpoint{6.143547in}{5.639682in}}%
\pgfpathlineto{\pgfqpoint{6.129836in}{5.560903in}}%
\pgfpathlineto{\pgfqpoint{6.113391in}{5.482125in}}%
\pgfpathlineto{\pgfqpoint{6.088313in}{5.377087in}}%
\pgfpathlineto{\pgfqpoint{6.053614in}{5.245790in}}%
\pgfpathlineto{\pgfqpoint{5.987486in}{5.009454in}}%
\pgfpathlineto{\pgfqpoint{5.937015in}{4.825638in}}%
\pgfpathlineto{\pgfqpoint{5.903129in}{4.694341in}}%
\pgfpathlineto{\pgfqpoint{5.877958in}{4.589303in}}%
\pgfpathlineto{\pgfqpoint{5.854844in}{4.484265in}}%
\pgfpathlineto{\pgfqpoint{5.834054in}{4.379227in}}%
\pgfpathlineto{\pgfqpoint{5.815902in}{4.274189in}}%
\pgfpathlineto{\pgfqpoint{5.800570in}{4.169151in}}%
\pgfpathlineto{\pgfqpoint{5.791066in}{4.090373in}}%
\pgfpathlineto{\pgfqpoint{5.783304in}{4.011594in}}%
\pgfpathlineto{\pgfqpoint{5.777409in}{3.932816in}}%
\pgfpathlineto{\pgfqpoint{5.773434in}{3.854037in}}%
\pgfpathlineto{\pgfqpoint{5.771429in}{3.775259in}}%
\pgfpathlineto{\pgfqpoint{5.771446in}{3.696481in}}%
\pgfpathlineto{\pgfqpoint{5.773532in}{3.617702in}}%
\pgfpathlineto{\pgfqpoint{5.777737in}{3.538924in}}%
\pgfpathlineto{\pgfqpoint{5.784112in}{3.460145in}}%
\pgfpathlineto{\pgfqpoint{5.793213in}{3.377415in}}%
\pgfpathlineto{\pgfqpoint{5.803480in}{3.302589in}}%
\pgfpathlineto{\pgfqpoint{5.816555in}{3.223810in}}%
\pgfpathlineto{\pgfqpoint{5.831910in}{3.145032in}}%
\pgfpathlineto{\pgfqpoint{5.849608in}{3.066253in}}%
\pgfpathlineto{\pgfqpoint{5.869614in}{2.987475in}}%
\pgfpathlineto{\pgfqpoint{5.892009in}{2.908696in}}%
\pgfpathlineto{\pgfqpoint{5.916796in}{2.829918in}}%
\pgfpathlineto{\pgfqpoint{5.943984in}{2.751140in}}%
\pgfpathlineto{\pgfqpoint{5.973582in}{2.672361in}}%
\pgfpathlineto{\pgfqpoint{6.005604in}{2.593583in}}%
\pgfpathlineto{\pgfqpoint{6.033868in}{2.528678in}}%
\pgfpathlineto{\pgfqpoint{6.033868in}{2.528678in}}%
\pgfusepath{stroke}%
\end{pgfscope}%
\begin{pgfscope}%
\pgfpathrectangle{\pgfqpoint{0.854460in}{0.571603in}}{\pgfqpoint{5.885100in}{5.225635in}}%
\pgfusepath{clip}%
\pgfsetbuttcap%
\pgfsetroundjoin%
\pgfsetlinewidth{1.505625pt}%
\definecolor{currentstroke}{rgb}{0.125394,0.574318,0.549086}%
\pgfsetstrokecolor{currentstroke}%
\pgfsetdash{}{0pt}%
\pgfpathmoveto{\pgfqpoint{6.212394in}{2.182862in}}%
\pgfpathlineto{\pgfqpoint{6.217887in}{2.173431in}}%
\pgfpathlineto{\pgfqpoint{6.233505in}{2.147172in}}%
\pgfpathlineto{\pgfqpoint{6.236813in}{2.141713in}}%
\pgfpathlineto{\pgfqpoint{6.249365in}{2.120912in}}%
\pgfpathlineto{\pgfqpoint{6.265544in}{2.094653in}}%
\pgfpathlineto{\pgfqpoint{6.266386in}{2.093310in}}%
\pgfpathlineto{\pgfqpoint{6.281946in}{2.068393in}}%
\pgfpathlineto{\pgfqpoint{6.295960in}{2.046404in}}%
\pgfpathlineto{\pgfqpoint{6.298671in}{2.042134in}}%
\pgfpathlineto{\pgfqpoint{6.315635in}{2.015874in}}%
\pgfpathlineto{\pgfqpoint{6.325533in}{2.000838in}}%
\pgfpathlineto{\pgfqpoint{6.332895in}{1.989615in}}%
\pgfpathlineto{\pgfqpoint{6.350434in}{1.963355in}}%
\pgfpathlineto{\pgfqpoint{6.355107in}{1.956477in}}%
\pgfpathlineto{\pgfqpoint{6.368228in}{1.937096in}}%
\pgfpathlineto{\pgfqpoint{6.384680in}{1.913244in}}%
\pgfpathlineto{\pgfqpoint{6.386335in}{1.910836in}}%
\pgfpathlineto{\pgfqpoint{6.404675in}{1.884577in}}%
\pgfpathlineto{\pgfqpoint{6.414253in}{1.871101in}}%
\pgfpathlineto{\pgfqpoint{6.423311in}{1.858318in}}%
\pgfpathlineto{\pgfqpoint{6.442242in}{1.832058in}}%
\pgfpathlineto{\pgfqpoint{6.443827in}{1.829892in}}%
\pgfpathlineto{\pgfqpoint{6.461406in}{1.805799in}}%
\pgfpathlineto{\pgfqpoint{6.473400in}{1.789638in}}%
\pgfpathlineto{\pgfqpoint{6.480874in}{1.779539in}}%
\pgfpathlineto{\pgfqpoint{6.500626in}{1.753280in}}%
\pgfpathlineto{\pgfqpoint{6.502973in}{1.750204in}}%
\pgfpathlineto{\pgfqpoint{6.520619in}{1.727020in}}%
\pgfpathlineto{\pgfqpoint{6.532547in}{1.711601in}}%
\pgfpathlineto{\pgfqpoint{6.540912in}{1.700761in}}%
\pgfpathlineto{\pgfqpoint{6.561498in}{1.674501in}}%
\pgfpathlineto{\pgfqpoint{6.562120in}{1.673718in}}%
\pgfpathlineto{\pgfqpoint{6.582313in}{1.648242in}}%
\pgfpathlineto{\pgfqpoint{6.591693in}{1.636587in}}%
\pgfpathlineto{\pgfqpoint{6.603423in}{1.621982in}}%
\pgfpathlineto{\pgfqpoint{6.621267in}{1.600098in}}%
\pgfpathlineto{\pgfqpoint{6.624827in}{1.595723in}}%
\pgfpathlineto{\pgfqpoint{6.646488in}{1.569463in}}%
\pgfpathlineto{\pgfqpoint{6.650840in}{1.564257in}}%
\pgfpathlineto{\pgfqpoint{6.668411in}{1.543204in}}%
\pgfpathlineto{\pgfqpoint{6.680414in}{1.529025in}}%
\pgfpathlineto{\pgfqpoint{6.690623in}{1.516944in}}%
\pgfpathlineto{\pgfqpoint{6.709987in}{1.494351in}}%
\pgfpathlineto{\pgfqpoint{6.713123in}{1.490685in}}%
\pgfpathlineto{\pgfqpoint{6.735880in}{1.464425in}}%
\pgfpathlineto{\pgfqpoint{6.739560in}{1.460232in}}%
\pgfusepath{stroke}%
\end{pgfscope}%
\begin{pgfscope}%
\pgfpathrectangle{\pgfqpoint{0.854460in}{0.571603in}}{\pgfqpoint{5.885100in}{5.225635in}}%
\pgfusepath{clip}%
\pgfsetbuttcap%
\pgfsetroundjoin%
\pgfsetlinewidth{1.505625pt}%
\definecolor{currentstroke}{rgb}{0.120565,0.596422,0.543611}%
\pgfsetstrokecolor{currentstroke}%
\pgfsetdash{}{0pt}%
\pgfpathmoveto{\pgfqpoint{1.160477in}{0.571603in}}%
\pgfpathlineto{\pgfqpoint{1.150194in}{0.581464in}}%
\pgfpathlineto{\pgfqpoint{1.133178in}{0.597863in}}%
\pgfpathlineto{\pgfqpoint{1.120621in}{0.610136in}}%
\pgfpathlineto{\pgfqpoint{1.106385in}{0.624122in}}%
\pgfpathlineto{\pgfqpoint{1.091047in}{0.639404in}}%
\pgfpathlineto{\pgfqpoint{1.080089in}{0.650382in}}%
\pgfpathlineto{\pgfqpoint{1.061474in}{0.669293in}}%
\pgfpathlineto{\pgfqpoint{1.054281in}{0.676641in}}%
\pgfpathlineto{\pgfqpoint{1.031901in}{0.699830in}}%
\pgfpathlineto{\pgfqpoint{1.028953in}{0.702901in}}%
\pgfpathlineto{\pgfqpoint{1.004120in}{0.729160in}}%
\pgfpathlineto{\pgfqpoint{1.002327in}{0.731084in}}%
\pgfpathlineto{\pgfqpoint{0.979793in}{0.755420in}}%
\pgfpathlineto{\pgfqpoint{0.972754in}{0.763131in}}%
\pgfpathlineto{\pgfqpoint{0.955925in}{0.781679in}}%
\pgfpathlineto{\pgfqpoint{0.943181in}{0.795929in}}%
\pgfpathlineto{\pgfqpoint{0.932506in}{0.807939in}}%
\pgfpathlineto{\pgfqpoint{0.913607in}{0.829509in}}%
\pgfpathlineto{\pgfqpoint{0.909525in}{0.834198in}}%
\pgfpathlineto{\pgfqpoint{0.887013in}{0.860458in}}%
\pgfpathlineto{\pgfqpoint{0.884034in}{0.863988in}}%
\pgfpathlineto{\pgfqpoint{0.864981in}{0.886717in}}%
\pgfpathlineto{\pgfqpoint{0.854460in}{0.899452in}}%
\pgfusepath{stroke}%
\end{pgfscope}%
\begin{pgfscope}%
\pgfpathrectangle{\pgfqpoint{0.854460in}{0.571603in}}{\pgfqpoint{5.885100in}{5.225635in}}%
\pgfusepath{clip}%
\pgfsetbuttcap%
\pgfsetroundjoin%
\pgfsetlinewidth{1.505625pt}%
\definecolor{currentstroke}{rgb}{0.120565,0.596422,0.543611}%
\pgfsetstrokecolor{currentstroke}%
\pgfsetdash{}{0pt}%
\pgfpathmoveto{\pgfqpoint{0.854460in}{4.678071in}}%
\pgfpathlineto{\pgfqpoint{0.867984in}{4.694341in}}%
\pgfpathlineto{\pgfqpoint{0.884034in}{4.713398in}}%
\pgfpathlineto{\pgfqpoint{0.890165in}{4.720600in}}%
\pgfpathlineto{\pgfqpoint{0.912817in}{4.746860in}}%
\pgfpathlineto{\pgfqpoint{0.913607in}{4.747762in}}%
\pgfpathlineto{\pgfqpoint{0.936035in}{4.773119in}}%
\pgfpathlineto{\pgfqpoint{0.943181in}{4.781096in}}%
\pgfpathlineto{\pgfqpoint{0.959732in}{4.799378in}}%
\pgfpathlineto{\pgfqpoint{0.972754in}{4.813580in}}%
\pgfpathlineto{\pgfqpoint{0.983926in}{4.825638in}}%
\pgfpathlineto{\pgfqpoint{1.002327in}{4.845249in}}%
\pgfpathlineto{\pgfqpoint{1.008629in}{4.851897in}}%
\pgfpathlineto{\pgfqpoint{1.031901in}{4.876139in}}%
\pgfpathlineto{\pgfqpoint{1.033858in}{4.878157in}}%
\pgfpathlineto{\pgfqpoint{1.059656in}{4.904416in}}%
\pgfpathlineto{\pgfqpoint{1.061474in}{4.906243in}}%
\pgfpathlineto{\pgfqpoint{1.086032in}{4.930676in}}%
\pgfpathlineto{\pgfqpoint{1.091047in}{4.935604in}}%
\pgfpathlineto{\pgfqpoint{1.112968in}{4.956935in}}%
\pgfpathlineto{\pgfqpoint{1.120621in}{4.964290in}}%
\pgfpathlineto{\pgfqpoint{1.140479in}{4.983195in}}%
\pgfpathlineto{\pgfqpoint{1.150194in}{4.992330in}}%
\pgfpathlineto{\pgfqpoint{1.168578in}{5.009454in}}%
\pgfpathlineto{\pgfqpoint{1.179767in}{5.019749in}}%
\pgfpathlineto{\pgfqpoint{1.197281in}{5.035714in}}%
\pgfpathlineto{\pgfqpoint{1.209341in}{5.046572in}}%
\pgfpathlineto{\pgfqpoint{1.226602in}{5.061973in}}%
\pgfpathlineto{\pgfqpoint{1.238914in}{5.072825in}}%
\pgfpathlineto{\pgfqpoint{1.256553in}{5.088233in}}%
\pgfpathlineto{\pgfqpoint{1.268488in}{5.098531in}}%
\pgfpathlineto{\pgfqpoint{1.287149in}{5.114492in}}%
\pgfpathlineto{\pgfqpoint{1.298061in}{5.123711in}}%
\pgfpathlineto{\pgfqpoint{1.318404in}{5.140752in}}%
\pgfpathlineto{\pgfqpoint{1.327634in}{5.148389in}}%
\pgfpathlineto{\pgfqpoint{1.350330in}{5.167011in}}%
\pgfpathlineto{\pgfqpoint{1.357208in}{5.172585in}}%
\pgfpathlineto{\pgfqpoint{1.382941in}{5.193271in}}%
\pgfpathlineto{\pgfqpoint{1.386781in}{5.196320in}}%
\pgfpathlineto{\pgfqpoint{1.416249in}{5.219530in}}%
\pgfpathlineto{\pgfqpoint{1.416354in}{5.219612in}}%
\pgfpathlineto{\pgfqpoint{1.445928in}{5.242415in}}%
\pgfpathlineto{\pgfqpoint{1.450339in}{5.245790in}}%
\pgfpathlineto{\pgfqpoint{1.475501in}{5.264804in}}%
\pgfpathlineto{\pgfqpoint{1.485164in}{5.272049in}}%
\pgfpathlineto{\pgfqpoint{1.505074in}{5.286797in}}%
\pgfpathlineto{\pgfqpoint{1.520734in}{5.298308in}}%
\pgfpathlineto{\pgfqpoint{1.534648in}{5.308413in}}%
\pgfpathlineto{\pgfqpoint{1.557060in}{5.324568in}}%
\pgfpathlineto{\pgfqpoint{1.564221in}{5.329668in}}%
\pgfpathlineto{\pgfqpoint{1.592626in}{5.349747in}}%
\pgfusepath{stroke}%
\end{pgfscope}%
\begin{pgfscope}%
\pgfpathrectangle{\pgfqpoint{0.854460in}{0.571603in}}{\pgfqpoint{5.885100in}{5.225635in}}%
\pgfusepath{clip}%
\pgfsetbuttcap%
\pgfsetroundjoin%
\pgfsetlinewidth{1.505625pt}%
\definecolor{currentstroke}{rgb}{0.120565,0.596422,0.543611}%
\pgfsetstrokecolor{currentstroke}%
\pgfsetdash{}{0pt}%
\pgfpathmoveto{\pgfqpoint{1.913581in}{5.555498in}}%
\pgfpathlineto{\pgfqpoint{1.919102in}{5.558765in}}%
\pgfpathlineto{\pgfqpoint{1.922744in}{5.560903in}}%
\pgfpathlineto{\pgfqpoint{1.948675in}{5.575934in}}%
\pgfpathlineto{\pgfqpoint{1.968151in}{5.587163in}}%
\pgfpathlineto{\pgfqpoint{1.978248in}{5.592913in}}%
\pgfpathlineto{\pgfqpoint{2.007822in}{5.609637in}}%
\pgfpathlineto{\pgfqpoint{2.014567in}{5.613422in}}%
\pgfpathlineto{\pgfqpoint{2.037395in}{5.626073in}}%
\pgfpathlineto{\pgfqpoint{2.062073in}{5.639682in}}%
\pgfpathlineto{\pgfqpoint{2.066968in}{5.642348in}}%
\pgfpathlineto{\pgfqpoint{2.096542in}{5.658320in}}%
\pgfpathlineto{\pgfqpoint{2.110737in}{5.665941in}}%
\pgfpathlineto{\pgfqpoint{2.126115in}{5.674095in}}%
\pgfpathlineto{\pgfqpoint{2.155689in}{5.689685in}}%
\pgfpathlineto{\pgfqpoint{2.160500in}{5.692201in}}%
\pgfpathlineto{\pgfqpoint{2.185262in}{5.704983in}}%
\pgfpathlineto{\pgfqpoint{2.211481in}{5.718460in}}%
\pgfpathlineto{\pgfqpoint{2.214835in}{5.720163in}}%
\pgfpathlineto{\pgfqpoint{2.244409in}{5.735040in}}%
\pgfpathlineto{\pgfqpoint{2.263744in}{5.744720in}}%
\pgfpathlineto{\pgfqpoint{2.273982in}{5.749781in}}%
\pgfpathlineto{\pgfqpoint{2.303555in}{5.764295in}}%
\pgfpathlineto{\pgfqpoint{2.317263in}{5.770979in}}%
\pgfpathlineto{\pgfqpoint{2.333129in}{5.778618in}}%
\pgfpathlineto{\pgfqpoint{2.362702in}{5.792774in}}%
\pgfpathlineto{\pgfqpoint{2.372101in}{5.797238in}}%
\pgfusepath{stroke}%
\end{pgfscope}%
\begin{pgfscope}%
\pgfpathrectangle{\pgfqpoint{0.854460in}{0.571603in}}{\pgfqpoint{5.885100in}{5.225635in}}%
\pgfusepath{clip}%
\pgfsetbuttcap%
\pgfsetroundjoin%
\pgfsetlinewidth{1.505625pt}%
\definecolor{currentstroke}{rgb}{0.120565,0.596422,0.543611}%
\pgfsetstrokecolor{currentstroke}%
\pgfsetdash{}{0pt}%
\pgfpathmoveto{\pgfqpoint{6.287072in}{5.797238in}}%
\pgfpathlineto{\pgfqpoint{6.283813in}{5.770979in}}%
\pgfpathlineto{\pgfqpoint{6.280061in}{5.744720in}}%
\pgfpathlineto{\pgfqpoint{6.275859in}{5.718460in}}%
\pgfpathlineto{\pgfqpoint{6.271247in}{5.692201in}}%
\pgfpathlineto{\pgfqpoint{6.266386in}{5.666612in}}%
\pgfpathlineto{\pgfqpoint{6.266259in}{5.665941in}}%
\pgfpathlineto{\pgfqpoint{6.260865in}{5.639682in}}%
\pgfpathlineto{\pgfqpoint{6.255161in}{5.613422in}}%
\pgfpathlineto{\pgfqpoint{6.249176in}{5.587163in}}%
\pgfpathlineto{\pgfqpoint{6.242937in}{5.560903in}}%
\pgfpathlineto{\pgfqpoint{6.236813in}{5.536057in}}%
\pgfpathlineto{\pgfqpoint{6.236465in}{5.534644in}}%
\pgfpathlineto{\pgfqpoint{6.229713in}{5.508384in}}%
\pgfpathlineto{\pgfqpoint{6.222779in}{5.482125in}}%
\pgfpathlineto{\pgfqpoint{6.215683in}{5.455865in}}%
\pgfpathlineto{\pgfqpoint{6.208445in}{5.429606in}}%
\pgfpathlineto{\pgfqpoint{6.207240in}{5.425350in}}%
\pgfpathlineto{\pgfqpoint{6.201012in}{5.403346in}}%
\pgfpathlineto{\pgfqpoint{6.193459in}{5.377087in}}%
\pgfpathlineto{\pgfqpoint{6.185816in}{5.350827in}}%
\pgfpathlineto{\pgfqpoint{6.178098in}{5.324568in}}%
\pgfpathlineto{\pgfqpoint{6.177666in}{5.323124in}}%
\pgfpathlineto{\pgfqpoint{6.170238in}{5.298308in}}%
\pgfpathlineto{\pgfqpoint{6.162327in}{5.272049in}}%
\pgfpathlineto{\pgfqpoint{6.154383in}{5.245790in}}%
\pgfpathlineto{\pgfqpoint{6.148093in}{5.225097in}}%
\pgfpathlineto{\pgfqpoint{6.146400in}{5.219530in}}%
\pgfpathlineto{\pgfqpoint{6.138338in}{5.193271in}}%
\pgfpathlineto{\pgfqpoint{6.130279in}{5.167011in}}%
\pgfpathlineto{\pgfqpoint{6.122233in}{5.140752in}}%
\pgfpathlineto{\pgfqpoint{6.118520in}{5.128664in}}%
\pgfpathlineto{\pgfqpoint{6.114162in}{5.114492in}}%
\pgfpathlineto{\pgfqpoint{6.106084in}{5.088233in}}%
\pgfpathlineto{\pgfqpoint{6.098047in}{5.061973in}}%
\pgfpathlineto{\pgfqpoint{6.090059in}{5.035714in}}%
\pgfpathlineto{\pgfqpoint{6.088946in}{5.032057in}}%
\pgfpathlineto{\pgfqpoint{6.082057in}{5.009454in}}%
\pgfpathlineto{\pgfqpoint{6.074110in}{4.983195in}}%
\pgfpathlineto{\pgfqpoint{6.066236in}{4.956935in}}%
\pgfpathlineto{\pgfqpoint{6.059373in}{4.933839in}}%
\pgfpathlineto{\pgfqpoint{6.058431in}{4.930676in}}%
\pgfpathlineto{\pgfqpoint{6.050644in}{4.904416in}}%
\pgfpathlineto{\pgfqpoint{6.042950in}{4.878157in}}%
\pgfpathlineto{\pgfqpoint{6.035356in}{4.851897in}}%
\pgfpathlineto{\pgfqpoint{6.029799in}{4.832455in}}%
\pgfpathlineto{\pgfqpoint{6.027847in}{4.825638in}}%
\pgfpathlineto{\pgfqpoint{6.020394in}{4.799378in}}%
\pgfpathlineto{\pgfqpoint{6.013057in}{4.773119in}}%
\pgfpathlineto{\pgfqpoint{6.005842in}{4.746860in}}%
\pgfpathlineto{\pgfqpoint{6.000226in}{4.726087in}}%
\pgfpathlineto{\pgfqpoint{5.998739in}{4.720600in}}%
\pgfpathlineto{\pgfqpoint{5.991711in}{4.694341in}}%
\pgfpathlineto{\pgfqpoint{5.984821in}{4.668081in}}%
\pgfpathlineto{\pgfqpoint{5.978070in}{4.641822in}}%
\pgfpathlineto{\pgfqpoint{5.971464in}{4.615562in}}%
\pgfpathlineto{\pgfqpoint{5.970653in}{4.612283in}}%
\pgfpathlineto{\pgfqpoint{5.964950in}{4.589303in}}%
\pgfpathlineto{\pgfqpoint{5.958582in}{4.563043in}}%
\pgfpathlineto{\pgfqpoint{5.952370in}{4.536784in}}%
\pgfpathlineto{\pgfqpoint{5.946318in}{4.510524in}}%
\pgfpathlineto{\pgfqpoint{5.941079in}{4.487183in}}%
\pgfpathlineto{\pgfqpoint{5.940422in}{4.484265in}}%
\pgfpathlineto{\pgfqpoint{5.934644in}{4.458005in}}%
\pgfpathlineto{\pgfqpoint{5.929036in}{4.431746in}}%
\pgfpathlineto{\pgfqpoint{5.923601in}{4.405486in}}%
\pgfpathlineto{\pgfqpoint{5.918343in}{4.379227in}}%
\pgfpathlineto{\pgfqpoint{5.913263in}{4.352967in}}%
\pgfpathlineto{\pgfqpoint{5.911506in}{4.343583in}}%
\pgfpathlineto{\pgfqpoint{5.908334in}{4.326708in}}%
\pgfpathlineto{\pgfqpoint{5.903574in}{4.300449in}}%
\pgfpathlineto{\pgfqpoint{5.899002in}{4.274189in}}%
\pgfpathlineto{\pgfqpoint{5.894621in}{4.247930in}}%
\pgfpathlineto{\pgfqpoint{5.890431in}{4.221670in}}%
\pgfpathlineto{\pgfqpoint{5.886436in}{4.195411in}}%
\pgfpathlineto{\pgfqpoint{5.882637in}{4.169151in}}%
\pgfpathlineto{\pgfqpoint{5.881933in}{4.164027in}}%
\pgfpathlineto{\pgfqpoint{5.879011in}{4.142892in}}%
\pgfpathlineto{\pgfqpoint{5.875580in}{4.116632in}}%
\pgfpathlineto{\pgfqpoint{5.872355in}{4.090373in}}%
\pgfpathlineto{\pgfqpoint{5.869335in}{4.064113in}}%
\pgfpathlineto{\pgfqpoint{5.866524in}{4.037854in}}%
\pgfpathlineto{\pgfqpoint{5.863924in}{4.011594in}}%
\pgfpathlineto{\pgfqpoint{5.861535in}{3.985335in}}%
\pgfpathlineto{\pgfqpoint{5.859361in}{3.959075in}}%
\pgfpathlineto{\pgfqpoint{5.857402in}{3.932816in}}%
\pgfpathlineto{\pgfqpoint{5.855661in}{3.906556in}}%
\pgfpathlineto{\pgfqpoint{5.854139in}{3.880297in}}%
\pgfpathlineto{\pgfqpoint{5.852838in}{3.854037in}}%
\pgfpathlineto{\pgfqpoint{5.852359in}{3.842403in}}%
\pgfpathlineto{\pgfqpoint{5.851754in}{3.827778in}}%
\pgfpathlineto{\pgfqpoint{5.850893in}{3.801519in}}%
\pgfpathlineto{\pgfqpoint{5.850259in}{3.775259in}}%
\pgfpathlineto{\pgfqpoint{5.849856in}{3.749000in}}%
\pgfpathlineto{\pgfqpoint{5.849684in}{3.722740in}}%
\pgfpathlineto{\pgfqpoint{5.849746in}{3.696481in}}%
\pgfpathlineto{\pgfqpoint{5.850042in}{3.670221in}}%
\pgfpathlineto{\pgfqpoint{5.850575in}{3.643962in}}%
\pgfpathlineto{\pgfqpoint{5.851347in}{3.617702in}}%
\pgfpathlineto{\pgfqpoint{5.852359in}{3.591443in}}%
\pgfpathlineto{\pgfqpoint{5.852359in}{3.591434in}}%
\pgfpathlineto{\pgfqpoint{5.853602in}{3.565183in}}%
\pgfpathlineto{\pgfqpoint{5.855086in}{3.538924in}}%
\pgfpathlineto{\pgfqpoint{5.856814in}{3.512664in}}%
\pgfpathlineto{\pgfqpoint{5.858786in}{3.486405in}}%
\pgfpathlineto{\pgfqpoint{5.861005in}{3.460145in}}%
\pgfpathlineto{\pgfqpoint{5.863473in}{3.433886in}}%
\pgfpathlineto{\pgfqpoint{5.866192in}{3.407626in}}%
\pgfpathlineto{\pgfqpoint{5.869162in}{3.381367in}}%
\pgfpathlineto{\pgfqpoint{5.872386in}{3.355107in}}%
\pgfpathlineto{\pgfqpoint{5.875867in}{3.328848in}}%
\pgfpathlineto{\pgfqpoint{5.879605in}{3.302589in}}%
\pgfpathlineto{\pgfqpoint{5.881933in}{3.287299in}}%
\pgfpathlineto{\pgfqpoint{5.883588in}{3.276329in}}%
\pgfpathlineto{\pgfqpoint{5.887811in}{3.250070in}}%
\pgfpathlineto{\pgfqpoint{5.892296in}{3.223810in}}%
\pgfpathlineto{\pgfqpoint{5.897044in}{3.197551in}}%
\pgfpathlineto{\pgfqpoint{5.902057in}{3.171291in}}%
\pgfpathlineto{\pgfqpoint{5.907338in}{3.145032in}}%
\pgfpathlineto{\pgfqpoint{5.911506in}{3.125310in}}%
\pgfpathlineto{\pgfqpoint{5.912876in}{3.118772in}}%
\pgfpathlineto{\pgfqpoint{5.918649in}{3.092513in}}%
\pgfpathlineto{\pgfqpoint{5.924694in}{3.066253in}}%
\pgfpathlineto{\pgfqpoint{5.929870in}{3.044742in}}%
\pgfusepath{stroke}%
\end{pgfscope}%
\begin{pgfscope}%
\pgfpathrectangle{\pgfqpoint{0.854460in}{0.571603in}}{\pgfqpoint{5.885100in}{5.225635in}}%
\pgfusepath{clip}%
\pgfsetbuttcap%
\pgfsetroundjoin%
\pgfsetlinewidth{1.505625pt}%
\definecolor{currentstroke}{rgb}{0.120565,0.596422,0.543611}%
\pgfsetstrokecolor{currentstroke}%
\pgfsetdash{}{0pt}%
\pgfpathmoveto{\pgfqpoint{6.047648in}{2.671808in}}%
\pgfpathlineto{\pgfqpoint{6.057793in}{2.646102in}}%
\pgfpathlineto{\pgfqpoint{6.059373in}{2.642205in}}%
\pgfpathlineto{\pgfqpoint{6.068386in}{2.619842in}}%
\pgfpathlineto{\pgfqpoint{6.079271in}{2.593583in}}%
\pgfpathlineto{\pgfqpoint{6.088946in}{2.570877in}}%
\pgfpathlineto{\pgfqpoint{6.090452in}{2.567323in}}%
\pgfpathlineto{\pgfqpoint{6.101861in}{2.541064in}}%
\pgfpathlineto{\pgfqpoint{6.113582in}{2.514804in}}%
\pgfpathlineto{\pgfqpoint{6.118520in}{2.504014in}}%
\pgfpathlineto{\pgfqpoint{6.125559in}{2.488545in}}%
\pgfpathlineto{\pgfqpoint{6.137811in}{2.462285in}}%
\pgfpathlineto{\pgfqpoint{6.148093in}{2.440793in}}%
\pgfpathlineto{\pgfqpoint{6.150362in}{2.436026in}}%
\pgfpathlineto{\pgfqpoint{6.163148in}{2.409766in}}%
\pgfpathlineto{\pgfqpoint{6.176256in}{2.383507in}}%
\pgfpathlineto{\pgfqpoint{6.177666in}{2.380743in}}%
\pgfpathlineto{\pgfqpoint{6.189592in}{2.357248in}}%
\pgfpathlineto{\pgfqpoint{6.203243in}{2.330988in}}%
\pgfpathlineto{\pgfqpoint{6.207240in}{2.323463in}}%
\pgfpathlineto{\pgfqpoint{6.217143in}{2.304729in}}%
\pgfpathlineto{\pgfqpoint{6.231340in}{2.278469in}}%
\pgfpathlineto{\pgfqpoint{6.236813in}{2.268558in}}%
\pgfpathlineto{\pgfqpoint{6.245799in}{2.252210in}}%
\pgfpathlineto{\pgfqpoint{6.260548in}{2.225950in}}%
\pgfpathlineto{\pgfqpoint{6.266386in}{2.215765in}}%
\pgfpathlineto{\pgfqpoint{6.275561in}{2.199691in}}%
\pgfpathlineto{\pgfqpoint{6.290867in}{2.173431in}}%
\pgfpathlineto{\pgfqpoint{6.295960in}{2.164863in}}%
\pgfpathlineto{\pgfqpoint{6.306431in}{2.147172in}}%
\pgfpathlineto{\pgfqpoint{6.322299in}{2.120912in}}%
\pgfpathlineto{\pgfqpoint{6.325533in}{2.115658in}}%
\pgfpathlineto{\pgfqpoint{6.338412in}{2.094653in}}%
\pgfpathlineto{\pgfqpoint{6.354847in}{2.068393in}}%
\pgfpathlineto{\pgfqpoint{6.355107in}{2.067986in}}%
\pgfpathlineto{\pgfqpoint{6.371504in}{2.042134in}}%
\pgfpathlineto{\pgfqpoint{6.384680in}{2.021773in}}%
\pgfpathlineto{\pgfqpoint{6.388483in}{2.015874in}}%
\pgfpathlineto{\pgfqpoint{6.405713in}{1.989615in}}%
\pgfpathlineto{\pgfqpoint{6.414253in}{1.976836in}}%
\pgfpathlineto{\pgfqpoint{6.423232in}{1.963355in}}%
\pgfpathlineto{\pgfqpoint{6.441043in}{1.937096in}}%
\pgfpathlineto{\pgfqpoint{6.443827in}{1.933059in}}%
\pgfpathlineto{\pgfqpoint{6.459098in}{1.910836in}}%
\pgfpathlineto{\pgfqpoint{6.473400in}{1.890404in}}%
\pgfpathlineto{\pgfqpoint{6.477466in}{1.884577in}}%
\pgfpathlineto{\pgfqpoint{6.496090in}{1.858318in}}%
\pgfpathlineto{\pgfqpoint{6.502973in}{1.848771in}}%
\pgfpathlineto{\pgfqpoint{6.514990in}{1.832058in}}%
\pgfpathlineto{\pgfqpoint{6.532547in}{1.808062in}}%
\pgfpathlineto{\pgfqpoint{6.534198in}{1.805799in}}%
\pgfpathlineto{\pgfqpoint{6.553644in}{1.779539in}}%
\pgfpathlineto{\pgfqpoint{6.562120in}{1.768279in}}%
\pgfpathlineto{\pgfqpoint{6.573381in}{1.753280in}}%
\pgfpathlineto{\pgfqpoint{6.591693in}{1.729289in}}%
\pgfpathlineto{\pgfqpoint{6.593421in}{1.727020in}}%
\pgfpathlineto{\pgfqpoint{6.613701in}{1.700761in}}%
\pgfpathlineto{\pgfqpoint{6.621267in}{1.691114in}}%
\pgfpathlineto{\pgfqpoint{6.634267in}{1.674501in}}%
\pgfpathlineto{\pgfqpoint{6.650840in}{1.653652in}}%
\pgfpathlineto{\pgfqpoint{6.655132in}{1.648242in}}%
\pgfpathlineto{\pgfqpoint{6.676259in}{1.621982in}}%
\pgfpathlineto{\pgfqpoint{6.680414in}{1.616890in}}%
\pgfpathlineto{\pgfqpoint{6.697649in}{1.595723in}}%
\pgfpathlineto{\pgfqpoint{6.709987in}{1.580793in}}%
\pgfpathlineto{\pgfqpoint{6.719333in}{1.569463in}}%
\pgfpathlineto{\pgfqpoint{6.739560in}{1.545296in}}%
\pgfusepath{stroke}%
\end{pgfscope}%
\begin{pgfscope}%
\pgfpathrectangle{\pgfqpoint{0.854460in}{0.571603in}}{\pgfqpoint{5.885100in}{5.225635in}}%
\pgfusepath{clip}%
\pgfsetbuttcap%
\pgfsetroundjoin%
\pgfsetlinewidth{1.505625pt}%
\definecolor{currentstroke}{rgb}{0.119483,0.614817,0.537692}%
\pgfsetstrokecolor{currentstroke}%
\pgfsetdash{}{0pt}%
\pgfpathmoveto{\pgfqpoint{1.102532in}{0.571603in}}%
\pgfpathlineto{\pgfqpoint{1.091047in}{0.582713in}}%
\pgfpathlineto{\pgfqpoint{1.075466in}{0.597863in}}%
\pgfpathlineto{\pgfqpoint{1.061474in}{0.611659in}}%
\pgfpathlineto{\pgfqpoint{1.048901in}{0.624122in}}%
\pgfpathlineto{\pgfqpoint{1.031901in}{0.641211in}}%
\pgfpathlineto{\pgfqpoint{1.022827in}{0.650382in}}%
\pgfpathlineto{\pgfqpoint{1.002327in}{0.671392in}}%
\pgfpathlineto{\pgfqpoint{0.997235in}{0.676641in}}%
\pgfpathlineto{\pgfqpoint{0.972754in}{0.702231in}}%
\pgfpathlineto{\pgfqpoint{0.972117in}{0.702901in}}%
\pgfpathlineto{\pgfqpoint{0.947520in}{0.729160in}}%
\pgfpathlineto{\pgfqpoint{0.943181in}{0.733860in}}%
\pgfpathlineto{\pgfqpoint{0.923394in}{0.755420in}}%
\pgfpathlineto{\pgfqpoint{0.913607in}{0.766235in}}%
\pgfpathlineto{\pgfqpoint{0.899719in}{0.781679in}}%
\pgfpathlineto{\pgfqpoint{0.884034in}{0.799372in}}%
\pgfpathlineto{\pgfqpoint{0.876487in}{0.807939in}}%
\pgfpathlineto{\pgfqpoint{0.854460in}{0.833302in}}%
\pgfusepath{stroke}%
\end{pgfscope}%
\begin{pgfscope}%
\pgfpathrectangle{\pgfqpoint{0.854460in}{0.571603in}}{\pgfqpoint{5.885100in}{5.225635in}}%
\pgfusepath{clip}%
\pgfsetbuttcap%
\pgfsetroundjoin%
\pgfsetlinewidth{1.505625pt}%
\definecolor{currentstroke}{rgb}{0.119483,0.614817,0.537692}%
\pgfsetstrokecolor{currentstroke}%
\pgfsetdash{}{0pt}%
\pgfpathmoveto{\pgfqpoint{0.854460in}{4.757175in}}%
\pgfpathlineto{\pgfqpoint{0.868311in}{4.773119in}}%
\pgfpathlineto{\pgfqpoint{0.884034in}{4.790986in}}%
\pgfpathlineto{\pgfqpoint{0.891496in}{4.799378in}}%
\pgfpathlineto{\pgfqpoint{0.913607in}{4.823928in}}%
\pgfpathlineto{\pgfqpoint{0.915163in}{4.825638in}}%
\pgfpathlineto{\pgfqpoint{0.939387in}{4.851897in}}%
\pgfpathlineto{\pgfqpoint{0.943181in}{4.855957in}}%
\pgfpathlineto{\pgfqpoint{0.964138in}{4.878157in}}%
\pgfpathlineto{\pgfqpoint{0.972754in}{4.887169in}}%
\pgfpathlineto{\pgfqpoint{0.989407in}{4.904416in}}%
\pgfpathlineto{\pgfqpoint{1.002327in}{4.917631in}}%
\pgfpathlineto{\pgfqpoint{1.015206in}{4.930676in}}%
\pgfpathlineto{\pgfqpoint{1.031901in}{4.947374in}}%
\pgfpathlineto{\pgfqpoint{1.041552in}{4.956935in}}%
\pgfpathlineto{\pgfqpoint{1.061474in}{4.976427in}}%
\pgfpathlineto{\pgfqpoint{1.068457in}{4.983195in}}%
\pgfpathlineto{\pgfqpoint{1.091047in}{5.004817in}}%
\pgfpathlineto{\pgfqpoint{1.095937in}{5.009454in}}%
\pgfpathlineto{\pgfqpoint{1.120621in}{5.032573in}}%
\pgfpathlineto{\pgfqpoint{1.124005in}{5.035714in}}%
\pgfpathlineto{\pgfqpoint{1.150194in}{5.059719in}}%
\pgfpathlineto{\pgfqpoint{1.152676in}{5.061973in}}%
\pgfpathlineto{\pgfqpoint{1.179767in}{5.086281in}}%
\pgfpathlineto{\pgfqpoint{1.181962in}{5.088233in}}%
\pgfpathlineto{\pgfqpoint{1.209341in}{5.112283in}}%
\pgfpathlineto{\pgfqpoint{1.211878in}{5.114492in}}%
\pgfpathlineto{\pgfqpoint{1.223297in}{5.124314in}}%
\pgfusepath{stroke}%
\end{pgfscope}%
\begin{pgfscope}%
\pgfpathrectangle{\pgfqpoint{0.854460in}{0.571603in}}{\pgfqpoint{5.885100in}{5.225635in}}%
\pgfusepath{clip}%
\pgfsetbuttcap%
\pgfsetroundjoin%
\pgfsetlinewidth{1.505625pt}%
\definecolor{currentstroke}{rgb}{0.119483,0.614817,0.537692}%
\pgfsetstrokecolor{currentstroke}%
\pgfsetdash{}{0pt}%
\pgfpathmoveto{\pgfqpoint{1.524648in}{5.360032in}}%
\pgfpathlineto{\pgfqpoint{1.534648in}{5.367143in}}%
\pgfpathlineto{\pgfqpoint{1.548731in}{5.377087in}}%
\pgfpathlineto{\pgfqpoint{1.564221in}{5.387892in}}%
\pgfpathlineto{\pgfqpoint{1.586530in}{5.403346in}}%
\pgfpathlineto{\pgfqpoint{1.593795in}{5.408318in}}%
\pgfpathlineto{\pgfqpoint{1.623368in}{5.428412in}}%
\pgfpathlineto{\pgfqpoint{1.625140in}{5.429606in}}%
\pgfpathlineto{\pgfqpoint{1.652941in}{5.448108in}}%
\pgfpathlineto{\pgfqpoint{1.664673in}{5.455865in}}%
\pgfpathlineto{\pgfqpoint{1.682515in}{5.467519in}}%
\pgfpathlineto{\pgfqpoint{1.705018in}{5.482125in}}%
\pgfpathlineto{\pgfqpoint{1.712088in}{5.486658in}}%
\pgfpathlineto{\pgfqpoint{1.741661in}{5.505482in}}%
\pgfpathlineto{\pgfqpoint{1.746257in}{5.508384in}}%
\pgfpathlineto{\pgfqpoint{1.771235in}{5.523966in}}%
\pgfpathlineto{\pgfqpoint{1.788453in}{5.534644in}}%
\pgfpathlineto{\pgfqpoint{1.800808in}{5.542212in}}%
\pgfpathlineto{\pgfqpoint{1.830381in}{5.560220in}}%
\pgfpathlineto{\pgfqpoint{1.831514in}{5.560903in}}%
\pgfpathlineto{\pgfqpoint{1.859955in}{5.577859in}}%
\pgfpathlineto{\pgfqpoint{1.875646in}{5.587163in}}%
\pgfpathlineto{\pgfqpoint{1.889528in}{5.595293in}}%
\pgfpathlineto{\pgfqpoint{1.919102in}{5.612515in}}%
\pgfpathlineto{\pgfqpoint{1.920673in}{5.613422in}}%
\pgfpathlineto{\pgfqpoint{1.948675in}{5.629387in}}%
\pgfpathlineto{\pgfqpoint{1.966821in}{5.639682in}}%
\pgfpathlineto{\pgfqpoint{1.978248in}{5.646084in}}%
\pgfpathlineto{\pgfqpoint{2.007822in}{5.662549in}}%
\pgfpathlineto{\pgfqpoint{2.013962in}{5.665941in}}%
\pgfpathlineto{\pgfqpoint{2.037395in}{5.678727in}}%
\pgfpathlineto{\pgfqpoint{2.062201in}{5.692201in}}%
\pgfpathlineto{\pgfqpoint{2.066968in}{5.694758in}}%
\pgfpathlineto{\pgfqpoint{2.096542in}{5.710493in}}%
\pgfpathlineto{\pgfqpoint{2.111600in}{5.718460in}}%
\pgfpathlineto{\pgfqpoint{2.126115in}{5.726045in}}%
\pgfpathlineto{\pgfqpoint{2.155689in}{5.741410in}}%
\pgfpathlineto{\pgfqpoint{2.162109in}{5.744720in}}%
\pgfpathlineto{\pgfqpoint{2.185262in}{5.756506in}}%
\pgfpathlineto{\pgfqpoint{2.213800in}{5.770979in}}%
\pgfpathlineto{\pgfqpoint{2.214835in}{5.771497in}}%
\pgfpathlineto{\pgfqpoint{2.244409in}{5.786170in}}%
\pgfpathlineto{\pgfqpoint{2.266804in}{5.797238in}}%
\pgfusepath{stroke}%
\end{pgfscope}%
\begin{pgfscope}%
\pgfpathrectangle{\pgfqpoint{0.854460in}{0.571603in}}{\pgfqpoint{5.885100in}{5.225635in}}%
\pgfusepath{clip}%
\pgfsetbuttcap%
\pgfsetroundjoin%
\pgfsetlinewidth{1.505625pt}%
\definecolor{currentstroke}{rgb}{0.119483,0.614817,0.537692}%
\pgfsetstrokecolor{currentstroke}%
\pgfsetdash{}{0pt}%
\pgfpathmoveto{\pgfqpoint{6.407925in}{5.797238in}}%
\pgfpathlineto{\pgfqpoint{6.402925in}{5.770979in}}%
\pgfpathlineto{\pgfqpoint{6.397530in}{5.744720in}}%
\pgfpathlineto{\pgfqpoint{6.391775in}{5.718460in}}%
\pgfpathlineto{\pgfqpoint{6.385694in}{5.692201in}}%
\pgfpathlineto{\pgfqpoint{6.384680in}{5.688076in}}%
\pgfpathlineto{\pgfqpoint{6.379250in}{5.665941in}}%
\pgfpathlineto{\pgfqpoint{6.372525in}{5.639682in}}%
\pgfpathlineto{\pgfqpoint{6.365558in}{5.613422in}}%
\pgfpathlineto{\pgfqpoint{6.358373in}{5.587163in}}%
\pgfpathlineto{\pgfqpoint{6.355107in}{5.575626in}}%
\pgfpathlineto{\pgfqpoint{6.350945in}{5.560903in}}%
\pgfpathlineto{\pgfqpoint{6.343305in}{5.534644in}}%
\pgfpathlineto{\pgfqpoint{6.335514in}{5.508384in}}%
\pgfpathlineto{\pgfqpoint{6.327588in}{5.482125in}}%
\pgfpathlineto{\pgfqpoint{6.325533in}{5.475474in}}%
\pgfpathlineto{\pgfqpoint{6.319480in}{5.455865in}}%
\pgfpathlineto{\pgfqpoint{6.311249in}{5.429606in}}%
\pgfpathlineto{\pgfqpoint{6.302937in}{5.403346in}}%
\pgfpathlineto{\pgfqpoint{6.295960in}{5.381526in}}%
\pgfpathlineto{\pgfqpoint{6.294541in}{5.377087in}}%
\pgfpathlineto{\pgfqpoint{6.286015in}{5.350827in}}%
\pgfpathlineto{\pgfqpoint{6.277450in}{5.324568in}}%
\pgfpathlineto{\pgfqpoint{6.268858in}{5.298308in}}%
\pgfpathlineto{\pgfqpoint{6.266386in}{5.290826in}}%
\pgfpathlineto{\pgfqpoint{6.260184in}{5.272049in}}%
\pgfpathlineto{\pgfqpoint{6.251481in}{5.245790in}}%
\pgfpathlineto{\pgfqpoint{6.242786in}{5.219530in}}%
\pgfpathlineto{\pgfqpoint{6.236813in}{5.201512in}}%
\pgfpathlineto{\pgfqpoint{6.234079in}{5.193271in}}%
\pgfpathlineto{\pgfqpoint{6.225338in}{5.167011in}}%
\pgfpathlineto{\pgfqpoint{6.216634in}{5.140752in}}%
\pgfpathlineto{\pgfqpoint{6.207976in}{5.114492in}}%
\pgfpathlineto{\pgfqpoint{6.207240in}{5.112265in}}%
\pgfpathlineto{\pgfqpoint{6.199290in}{5.088233in}}%
\pgfpathlineto{\pgfqpoint{6.190659in}{5.061973in}}%
\pgfpathlineto{\pgfqpoint{6.182099in}{5.035714in}}%
\pgfpathlineto{\pgfqpoint{6.177666in}{5.022052in}}%
\pgfpathlineto{\pgfqpoint{6.173573in}{5.009454in}}%
\pgfpathlineto{\pgfqpoint{6.165088in}{4.983195in}}%
\pgfpathlineto{\pgfqpoint{6.156694in}{4.956935in}}%
\pgfpathlineto{\pgfqpoint{6.148396in}{4.930676in}}%
\pgfpathlineto{\pgfqpoint{6.148093in}{4.929711in}}%
\pgfpathlineto{\pgfqpoint{6.140123in}{4.904416in}}%
\pgfpathlineto{\pgfqpoint{6.131956in}{4.878157in}}%
\pgfpathlineto{\pgfqpoint{6.123904in}{4.851897in}}%
\pgfpathlineto{\pgfqpoint{6.118520in}{4.834118in}}%
\pgfpathlineto{\pgfqpoint{6.115946in}{4.825638in}}%
\pgfpathlineto{\pgfqpoint{6.108061in}{4.799378in}}%
\pgfpathlineto{\pgfqpoint{6.100306in}{4.773119in}}%
\pgfpathlineto{\pgfqpoint{6.092685in}{4.746860in}}%
\pgfpathlineto{\pgfqpoint{6.088946in}{4.733784in}}%
\pgfpathlineto{\pgfqpoint{6.086734in}{4.726070in}}%
\pgfusepath{stroke}%
\end{pgfscope}%
\begin{pgfscope}%
\pgfpathrectangle{\pgfqpoint{0.854460in}{0.571603in}}{\pgfqpoint{5.885100in}{5.225635in}}%
\pgfusepath{clip}%
\pgfsetbuttcap%
\pgfsetroundjoin%
\pgfsetlinewidth{1.505625pt}%
\definecolor{currentstroke}{rgb}{0.119483,0.614817,0.537692}%
\pgfsetstrokecolor{currentstroke}%
\pgfsetdash{}{0pt}%
\pgfpathmoveto{\pgfqpoint{5.993560in}{4.345453in}}%
\pgfpathlineto{\pgfqpoint{5.980004in}{4.274189in}}%
\pgfpathlineto{\pgfqpoint{5.962623in}{4.169151in}}%
\pgfpathlineto{\pgfqpoint{5.951678in}{4.090373in}}%
\pgfpathlineto{\pgfqpoint{5.941079in}{3.995821in}}%
\pgfpathlineto{\pgfqpoint{5.935487in}{3.932816in}}%
\pgfpathlineto{\pgfqpoint{5.930336in}{3.854037in}}%
\pgfpathlineto{\pgfqpoint{5.927236in}{3.775259in}}%
\pgfpathlineto{\pgfqpoint{5.926232in}{3.696481in}}%
\pgfpathlineto{\pgfqpoint{5.927366in}{3.617702in}}%
\pgfpathlineto{\pgfqpoint{5.930681in}{3.538924in}}%
\pgfpathlineto{\pgfqpoint{5.936222in}{3.460145in}}%
\pgfpathlineto{\pgfqpoint{5.944009in}{3.381367in}}%
\pgfpathlineto{\pgfqpoint{5.954054in}{3.302589in}}%
\pgfpathlineto{\pgfqpoint{5.966448in}{3.223810in}}%
\pgfpathlineto{\pgfqpoint{5.981152in}{3.145032in}}%
\pgfpathlineto{\pgfqpoint{6.000226in}{3.057968in}}%
\pgfpathlineto{\pgfqpoint{6.017701in}{2.987475in}}%
\pgfpathlineto{\pgfqpoint{6.039579in}{2.908696in}}%
\pgfpathlineto{\pgfqpoint{6.063889in}{2.829918in}}%
\pgfpathlineto{\pgfqpoint{6.090638in}{2.751140in}}%
\pgfpathlineto{\pgfqpoint{6.119834in}{2.672361in}}%
\pgfpathlineto{\pgfqpoint{6.151490in}{2.593583in}}%
\pgfpathlineto{\pgfqpoint{6.185620in}{2.514804in}}%
\pgfpathlineto{\pgfqpoint{6.222243in}{2.436026in}}%
\pgfpathlineto{\pgfqpoint{6.266386in}{2.347606in}}%
\pgfpathlineto{\pgfqpoint{6.303006in}{2.278469in}}%
\pgfpathlineto{\pgfqpoint{6.347141in}{2.199691in}}%
\pgfpathlineto{\pgfqpoint{6.393789in}{2.120912in}}%
\pgfpathlineto{\pgfqpoint{6.443827in}{2.040817in}}%
\pgfpathlineto{\pgfqpoint{6.502973in}{1.951069in}}%
\pgfpathlineto{\pgfqpoint{6.548833in}{1.884577in}}%
\pgfpathlineto{\pgfqpoint{6.605551in}{1.805799in}}%
\pgfpathlineto{\pgfqpoint{6.664786in}{1.727020in}}%
\pgfpathlineto{\pgfqpoint{6.739560in}{1.632079in}}%
\pgfpathlineto{\pgfqpoint{6.739560in}{1.632079in}}%
\pgfusepath{stroke}%
\end{pgfscope}%
\begin{pgfscope}%
\pgfpathrectangle{\pgfqpoint{0.854460in}{0.571603in}}{\pgfqpoint{5.885100in}{5.225635in}}%
\pgfusepath{clip}%
\pgfsetbuttcap%
\pgfsetroundjoin%
\pgfsetlinewidth{1.505625pt}%
\definecolor{currentstroke}{rgb}{0.123444,0.636809,0.528763}%
\pgfsetstrokecolor{currentstroke}%
\pgfsetdash{}{0pt}%
\pgfpathmoveto{\pgfqpoint{1.046014in}{0.571603in}}%
\pgfpathlineto{\pgfqpoint{1.031901in}{0.585377in}}%
\pgfpathlineto{\pgfqpoint{1.019173in}{0.597863in}}%
\pgfpathlineto{\pgfqpoint{1.002327in}{0.614621in}}%
\pgfpathlineto{\pgfqpoint{0.992827in}{0.624122in}}%
\pgfpathlineto{\pgfqpoint{0.972754in}{0.644479in}}%
\pgfpathlineto{\pgfqpoint{0.966966in}{0.650382in}}%
\pgfpathlineto{\pgfqpoint{0.943181in}{0.674978in}}%
\pgfpathlineto{\pgfqpoint{0.941581in}{0.676641in}}%
\pgfpathlineto{\pgfqpoint{0.916704in}{0.702901in}}%
\pgfpathlineto{\pgfqpoint{0.913607in}{0.706217in}}%
\pgfpathlineto{\pgfqpoint{0.892313in}{0.729160in}}%
\pgfpathlineto{\pgfqpoint{0.884034in}{0.738206in}}%
\pgfpathlineto{\pgfqpoint{0.868376in}{0.755420in}}%
\pgfpathlineto{\pgfqpoint{0.854460in}{0.770936in}}%
\pgfusepath{stroke}%
\end{pgfscope}%
\begin{pgfscope}%
\pgfpathrectangle{\pgfqpoint{0.854460in}{0.571603in}}{\pgfqpoint{5.885100in}{5.225635in}}%
\pgfusepath{clip}%
\pgfsetbuttcap%
\pgfsetroundjoin%
\pgfsetlinewidth{1.505625pt}%
\definecolor{currentstroke}{rgb}{0.123444,0.636809,0.528763}%
\pgfsetstrokecolor{currentstroke}%
\pgfsetdash{}{0pt}%
\pgfpathmoveto{\pgfqpoint{0.854460in}{4.831793in}}%
\pgfpathlineto{\pgfqpoint{0.872690in}{4.851897in}}%
\pgfpathlineto{\pgfqpoint{0.884034in}{4.864249in}}%
\pgfpathlineto{\pgfqpoint{0.891351in}{4.872138in}}%
\pgfusepath{stroke}%
\end{pgfscope}%
\begin{pgfscope}%
\pgfpathrectangle{\pgfqpoint{0.854460in}{0.571603in}}{\pgfqpoint{5.885100in}{5.225635in}}%
\pgfusepath{clip}%
\pgfsetbuttcap%
\pgfsetroundjoin%
\pgfsetlinewidth{1.505625pt}%
\definecolor{currentstroke}{rgb}{0.123444,0.636809,0.528763}%
\pgfsetstrokecolor{currentstroke}%
\pgfsetdash{}{0pt}%
\pgfpathmoveto{\pgfqpoint{1.167645in}{5.139087in}}%
\pgfpathlineto{\pgfqpoint{1.169547in}{5.140752in}}%
\pgfpathlineto{\pgfqpoint{1.179767in}{5.149590in}}%
\pgfpathlineto{\pgfqpoint{1.200083in}{5.167011in}}%
\pgfpathlineto{\pgfqpoint{1.209341in}{5.174854in}}%
\pgfpathlineto{\pgfqpoint{1.231263in}{5.193271in}}%
\pgfpathlineto{\pgfqpoint{1.238914in}{5.199620in}}%
\pgfpathlineto{\pgfqpoint{1.263100in}{5.219530in}}%
\pgfpathlineto{\pgfqpoint{1.268488in}{5.223911in}}%
\pgfpathlineto{\pgfqpoint{1.295606in}{5.245790in}}%
\pgfpathlineto{\pgfqpoint{1.298061in}{5.247746in}}%
\pgfpathlineto{\pgfqpoint{1.327634in}{5.271125in}}%
\pgfpathlineto{\pgfqpoint{1.328812in}{5.272049in}}%
\pgfpathlineto{\pgfqpoint{1.357208in}{5.294040in}}%
\pgfpathlineto{\pgfqpoint{1.362761in}{5.298308in}}%
\pgfpathlineto{\pgfqpoint{1.386781in}{5.316546in}}%
\pgfpathlineto{\pgfqpoint{1.397426in}{5.324568in}}%
\pgfpathlineto{\pgfqpoint{1.416354in}{5.338661in}}%
\pgfpathlineto{\pgfqpoint{1.432816in}{5.350827in}}%
\pgfpathlineto{\pgfqpoint{1.445928in}{5.360401in}}%
\pgfpathlineto{\pgfqpoint{1.468943in}{5.377087in}}%
\pgfpathlineto{\pgfqpoint{1.475501in}{5.381784in}}%
\pgfpathlineto{\pgfqpoint{1.505074in}{5.402815in}}%
\pgfpathlineto{\pgfqpoint{1.505828in}{5.403346in}}%
\pgfpathlineto{\pgfqpoint{1.534648in}{5.423421in}}%
\pgfpathlineto{\pgfqpoint{1.543587in}{5.429606in}}%
\pgfpathlineto{\pgfqpoint{1.564221in}{5.443709in}}%
\pgfpathlineto{\pgfqpoint{1.582123in}{5.455865in}}%
\pgfpathlineto{\pgfqpoint{1.593795in}{5.463695in}}%
\pgfpathlineto{\pgfqpoint{1.621445in}{5.482125in}}%
\pgfpathlineto{\pgfqpoint{1.623368in}{5.483391in}}%
\pgfpathlineto{\pgfqpoint{1.652941in}{5.502703in}}%
\pgfpathlineto{\pgfqpoint{1.661700in}{5.508384in}}%
\pgfpathlineto{\pgfqpoint{1.682515in}{5.521722in}}%
\pgfpathlineto{\pgfqpoint{1.702799in}{5.534644in}}%
\pgfpathlineto{\pgfqpoint{1.712088in}{5.540489in}}%
\pgfpathlineto{\pgfqpoint{1.741661in}{5.558976in}}%
\pgfpathlineto{\pgfqpoint{1.744769in}{5.560903in}}%
\pgfpathlineto{\pgfqpoint{1.771235in}{5.577115in}}%
\pgfpathlineto{\pgfqpoint{1.787729in}{5.587163in}}%
\pgfpathlineto{\pgfqpoint{1.800808in}{5.595033in}}%
\pgfpathlineto{\pgfqpoint{1.830381in}{5.612728in}}%
\pgfpathlineto{\pgfqpoint{1.831551in}{5.613422in}}%
\pgfpathlineto{\pgfqpoint{1.859955in}{5.630065in}}%
\pgfpathlineto{\pgfqpoint{1.876450in}{5.639682in}}%
\pgfpathlineto{\pgfqpoint{1.889528in}{5.647212in}}%
\pgfpathlineto{\pgfqpoint{1.919102in}{5.664144in}}%
\pgfpathlineto{\pgfqpoint{1.922265in}{5.665941in}}%
\pgfpathlineto{\pgfqpoint{1.948675in}{5.680754in}}%
\pgfpathlineto{\pgfqpoint{1.969178in}{5.692201in}}%
\pgfpathlineto{\pgfqpoint{1.978248in}{5.697202in}}%
\pgfpathlineto{\pgfqpoint{2.007822in}{5.713399in}}%
\pgfpathlineto{\pgfqpoint{2.017125in}{5.718460in}}%
\pgfpathlineto{\pgfqpoint{2.037395in}{5.729351in}}%
\pgfpathlineto{\pgfqpoint{2.066119in}{5.744720in}}%
\pgfpathlineto{\pgfqpoint{2.066968in}{5.745168in}}%
\pgfpathlineto{\pgfqpoint{2.096542in}{5.760657in}}%
\pgfpathlineto{\pgfqpoint{2.116334in}{5.770979in}}%
\pgfpathlineto{\pgfqpoint{2.126115in}{5.776017in}}%
\pgfpathlineto{\pgfqpoint{2.155689in}{5.791147in}}%
\pgfpathlineto{\pgfqpoint{2.167670in}{5.797238in}}%
\pgfusepath{stroke}%
\end{pgfscope}%
\begin{pgfscope}%
\pgfpathrectangle{\pgfqpoint{0.854460in}{0.571603in}}{\pgfqpoint{5.885100in}{5.225635in}}%
\pgfusepath{clip}%
\pgfsetbuttcap%
\pgfsetroundjoin%
\pgfsetlinewidth{1.505625pt}%
\definecolor{currentstroke}{rgb}{0.123444,0.636809,0.528763}%
\pgfsetstrokecolor{currentstroke}%
\pgfsetdash{}{0pt}%
\pgfpathmoveto{\pgfqpoint{6.522648in}{5.797238in}}%
\pgfpathlineto{\pgfqpoint{6.516168in}{5.770979in}}%
\pgfpathlineto{\pgfqpoint{6.509373in}{5.744720in}}%
\pgfpathlineto{\pgfqpoint{6.502973in}{5.721018in}}%
\pgfpathlineto{\pgfqpoint{6.502284in}{5.718460in}}%
\pgfpathlineto{\pgfqpoint{6.494859in}{5.692201in}}%
\pgfpathlineto{\pgfqpoint{6.487200in}{5.665941in}}%
\pgfpathlineto{\pgfqpoint{6.479333in}{5.639682in}}%
\pgfpathlineto{\pgfqpoint{6.473400in}{5.620402in}}%
\pgfpathlineto{\pgfqpoint{6.471256in}{5.613422in}}%
\pgfpathlineto{\pgfqpoint{6.462945in}{5.587163in}}%
\pgfpathlineto{\pgfqpoint{6.454489in}{5.560903in}}%
\pgfpathlineto{\pgfqpoint{6.445907in}{5.534644in}}%
\pgfpathlineto{\pgfqpoint{6.443827in}{5.528415in}}%
\pgfpathlineto{\pgfqpoint{6.437144in}{5.508384in}}%
\pgfpathlineto{\pgfqpoint{6.428265in}{5.482125in}}%
\pgfpathlineto{\pgfqpoint{6.419311in}{5.455865in}}%
\pgfpathlineto{\pgfqpoint{6.414253in}{5.441208in}}%
\pgfpathlineto{\pgfqpoint{6.410252in}{5.429606in}}%
\pgfpathlineto{\pgfqpoint{6.401092in}{5.403346in}}%
\pgfpathlineto{\pgfqpoint{6.391897in}{5.377087in}}%
\pgfpathlineto{\pgfqpoint{6.384680in}{5.356570in}}%
\pgfpathlineto{\pgfqpoint{6.382660in}{5.350827in}}%
\pgfpathlineto{\pgfqpoint{6.373337in}{5.324568in}}%
\pgfpathlineto{\pgfqpoint{6.364016in}{5.298308in}}%
\pgfpathlineto{\pgfqpoint{6.355107in}{5.273186in}}%
\pgfpathlineto{\pgfqpoint{6.354703in}{5.272049in}}%
\pgfpathlineto{\pgfqpoint{6.345518in}{5.246342in}}%
\pgfusepath{stroke}%
\end{pgfscope}%
\begin{pgfscope}%
\pgfpathrectangle{\pgfqpoint{0.854460in}{0.571603in}}{\pgfqpoint{5.885100in}{5.225635in}}%
\pgfusepath{clip}%
\pgfsetbuttcap%
\pgfsetroundjoin%
\pgfsetlinewidth{1.505625pt}%
\definecolor{currentstroke}{rgb}{0.123444,0.636809,0.528763}%
\pgfsetstrokecolor{currentstroke}%
\pgfsetdash{}{0pt}%
\pgfpathmoveto{\pgfqpoint{6.217917in}{4.876630in}}%
\pgfpathlineto{\pgfqpoint{6.177153in}{4.746860in}}%
\pgfpathlineto{\pgfqpoint{6.146427in}{4.641822in}}%
\pgfpathlineto{\pgfqpoint{6.118073in}{4.536784in}}%
\pgfpathlineto{\pgfqpoint{6.092313in}{4.431746in}}%
\pgfpathlineto{\pgfqpoint{6.069381in}{4.326708in}}%
\pgfpathlineto{\pgfqpoint{6.049495in}{4.221670in}}%
\pgfpathlineto{\pgfqpoint{6.036682in}{4.142892in}}%
\pgfpathlineto{\pgfqpoint{6.025742in}{4.064113in}}%
\pgfpathlineto{\pgfqpoint{6.016711in}{3.985335in}}%
\pgfpathlineto{\pgfqpoint{6.009692in}{3.906556in}}%
\pgfpathlineto{\pgfqpoint{6.004724in}{3.827778in}}%
\pgfpathlineto{\pgfqpoint{6.001848in}{3.749000in}}%
\pgfpathlineto{\pgfqpoint{6.001103in}{3.670221in}}%
\pgfpathlineto{\pgfqpoint{6.002530in}{3.591443in}}%
\pgfpathlineto{\pgfqpoint{6.006169in}{3.512664in}}%
\pgfpathlineto{\pgfqpoint{6.012064in}{3.433886in}}%
\pgfpathlineto{\pgfqpoint{6.020257in}{3.355107in}}%
\pgfpathlineto{\pgfqpoint{6.030785in}{3.276329in}}%
\pgfpathlineto{\pgfqpoint{6.043610in}{3.197551in}}%
\pgfpathlineto{\pgfqpoint{6.059373in}{3.116360in}}%
\pgfpathlineto{\pgfqpoint{6.076439in}{3.039994in}}%
\pgfpathlineto{\pgfqpoint{6.096468in}{2.961215in}}%
\pgfpathlineto{\pgfqpoint{6.118941in}{2.882437in}}%
\pgfpathlineto{\pgfqpoint{6.143830in}{2.803659in}}%
\pgfpathlineto{\pgfqpoint{6.171188in}{2.724880in}}%
\pgfpathlineto{\pgfqpoint{6.201029in}{2.646102in}}%
\pgfpathlineto{\pgfqpoint{6.236813in}{2.559353in}}%
\pgfpathlineto{\pgfqpoint{6.268196in}{2.488545in}}%
\pgfpathlineto{\pgfqpoint{6.305511in}{2.409766in}}%
\pgfpathlineto{\pgfqpoint{6.345360in}{2.330988in}}%
\pgfpathlineto{\pgfqpoint{6.387744in}{2.252210in}}%
\pgfpathlineto{\pgfqpoint{6.432623in}{2.173431in}}%
\pgfpathlineto{\pgfqpoint{6.480055in}{2.094653in}}%
\pgfpathlineto{\pgfqpoint{6.532547in}{2.012032in}}%
\pgfpathlineto{\pgfqpoint{6.582499in}{1.937096in}}%
\pgfpathlineto{\pgfqpoint{6.637521in}{1.858318in}}%
\pgfpathlineto{\pgfqpoint{6.695081in}{1.779539in}}%
\pgfpathlineto{\pgfqpoint{6.739560in}{1.720969in}}%
\pgfpathlineto{\pgfqpoint{6.739560in}{1.720969in}}%
\pgfusepath{stroke}%
\end{pgfscope}%
\begin{pgfscope}%
\pgfpathrectangle{\pgfqpoint{0.854460in}{0.571603in}}{\pgfqpoint{5.885100in}{5.225635in}}%
\pgfusepath{clip}%
\pgfsetbuttcap%
\pgfsetroundjoin%
\pgfsetlinewidth{1.505625pt}%
\definecolor{currentstroke}{rgb}{0.132268,0.655014,0.519661}%
\pgfsetstrokecolor{currentstroke}%
\pgfsetdash{}{0pt}%
\pgfpathmoveto{\pgfqpoint{0.990838in}{0.571603in}}%
\pgfpathlineto{\pgfqpoint{0.972754in}{0.589410in}}%
\pgfpathlineto{\pgfqpoint{0.964215in}{0.597863in}}%
\pgfpathlineto{\pgfqpoint{0.943181in}{0.618974in}}%
\pgfpathlineto{\pgfqpoint{0.938079in}{0.624122in}}%
\pgfpathlineto{\pgfqpoint{0.913607in}{0.649163in}}%
\pgfpathlineto{\pgfqpoint{0.912423in}{0.650382in}}%
\pgfpathlineto{\pgfqpoint{0.887278in}{0.676641in}}%
\pgfpathlineto{\pgfqpoint{0.884034in}{0.680078in}}%
\pgfpathlineto{\pgfqpoint{0.862616in}{0.702901in}}%
\pgfpathlineto{\pgfqpoint{0.854460in}{0.711713in}}%
\pgfusepath{stroke}%
\end{pgfscope}%
\begin{pgfscope}%
\pgfpathrectangle{\pgfqpoint{0.854460in}{0.571603in}}{\pgfqpoint{5.885100in}{5.225635in}}%
\pgfusepath{clip}%
\pgfsetbuttcap%
\pgfsetroundjoin%
\pgfsetlinewidth{1.505625pt}%
\definecolor{currentstroke}{rgb}{0.132268,0.655014,0.519661}%
\pgfsetstrokecolor{currentstroke}%
\pgfsetdash{}{0pt}%
\pgfpathmoveto{\pgfqpoint{0.854460in}{4.902448in}}%
\pgfpathlineto{\pgfqpoint{0.856296in}{4.904416in}}%
\pgfpathlineto{\pgfqpoint{0.881110in}{4.930676in}}%
\pgfpathlineto{\pgfqpoint{0.884034in}{4.933730in}}%
\pgfpathlineto{\pgfqpoint{0.906462in}{4.956935in}}%
\pgfpathlineto{\pgfqpoint{0.913607in}{4.964236in}}%
\pgfpathlineto{\pgfqpoint{0.932338in}{4.983195in}}%
\pgfpathlineto{\pgfqpoint{0.943181in}{4.994033in}}%
\pgfpathlineto{\pgfqpoint{0.958751in}{5.009454in}}%
\pgfpathlineto{\pgfqpoint{0.972754in}{5.023150in}}%
\pgfpathlineto{\pgfqpoint{0.985716in}{5.035714in}}%
\pgfpathlineto{\pgfqpoint{1.002327in}{5.051615in}}%
\pgfpathlineto{\pgfqpoint{1.013246in}{5.061973in}}%
\pgfpathlineto{\pgfqpoint{1.031901in}{5.079453in}}%
\pgfpathlineto{\pgfqpoint{1.041354in}{5.088233in}}%
\pgfpathlineto{\pgfqpoint{1.061474in}{5.106690in}}%
\pgfpathlineto{\pgfqpoint{1.070054in}{5.114492in}}%
\pgfpathlineto{\pgfqpoint{1.091047in}{5.133350in}}%
\pgfpathlineto{\pgfqpoint{1.099358in}{5.140752in}}%
\pgfpathlineto{\pgfqpoint{1.120621in}{5.159457in}}%
\pgfpathlineto{\pgfqpoint{1.129280in}{5.167011in}}%
\pgfpathlineto{\pgfqpoint{1.150194in}{5.185033in}}%
\pgfpathlineto{\pgfqpoint{1.159833in}{5.193271in}}%
\pgfpathlineto{\pgfqpoint{1.179767in}{5.210100in}}%
\pgfpathlineto{\pgfqpoint{1.191029in}{5.219530in}}%
\pgfpathlineto{\pgfqpoint{1.209341in}{5.234678in}}%
\pgfpathlineto{\pgfqpoint{1.222880in}{5.245790in}}%
\pgfpathlineto{\pgfqpoint{1.238914in}{5.258789in}}%
\pgfpathlineto{\pgfqpoint{1.255398in}{5.272049in}}%
\pgfpathlineto{\pgfqpoint{1.268488in}{5.282451in}}%
\pgfpathlineto{\pgfqpoint{1.288595in}{5.298308in}}%
\pgfpathlineto{\pgfqpoint{1.298061in}{5.305683in}}%
\pgfpathlineto{\pgfqpoint{1.322483in}{5.324568in}}%
\pgfpathlineto{\pgfqpoint{1.327634in}{5.328503in}}%
\pgfpathlineto{\pgfqpoint{1.357072in}{5.350827in}}%
\pgfpathlineto{\pgfqpoint{1.357208in}{5.350929in}}%
\pgfpathlineto{\pgfqpoint{1.386781in}{5.372898in}}%
\pgfpathlineto{\pgfqpoint{1.392460in}{5.377087in}}%
\pgfpathlineto{\pgfqpoint{1.416354in}{5.394497in}}%
\pgfpathlineto{\pgfqpoint{1.428584in}{5.403346in}}%
\pgfpathlineto{\pgfqpoint{1.445928in}{5.415744in}}%
\pgfpathlineto{\pgfqpoint{1.465450in}{5.429606in}}%
\pgfpathlineto{\pgfqpoint{1.475501in}{5.436655in}}%
\pgfpathlineto{\pgfqpoint{1.503069in}{5.455865in}}%
\pgfpathlineto{\pgfqpoint{1.505074in}{5.457246in}}%
\pgfpathlineto{\pgfqpoint{1.534648in}{5.477440in}}%
\pgfpathlineto{\pgfqpoint{1.541556in}{5.482125in}}%
\pgfpathlineto{\pgfqpoint{1.564221in}{5.497309in}}%
\pgfpathlineto{\pgfqpoint{1.580856in}{5.508384in}}%
\pgfpathlineto{\pgfqpoint{1.593795in}{5.516894in}}%
\pgfpathlineto{\pgfqpoint{1.620945in}{5.534644in}}%
\pgfpathlineto{\pgfqpoint{1.623368in}{5.536209in}}%
\pgfpathlineto{\pgfqpoint{1.628346in}{5.539398in}}%
\pgfusepath{stroke}%
\end{pgfscope}%
\begin{pgfscope}%
\pgfpathrectangle{\pgfqpoint{0.854460in}{0.571603in}}{\pgfqpoint{5.885100in}{5.225635in}}%
\pgfusepath{clip}%
\pgfsetbuttcap%
\pgfsetroundjoin%
\pgfsetlinewidth{1.505625pt}%
\definecolor{currentstroke}{rgb}{0.132268,0.655014,0.519661}%
\pgfsetstrokecolor{currentstroke}%
\pgfsetdash{}{0pt}%
\pgfpathmoveto{\pgfqpoint{1.955637in}{5.734033in}}%
\pgfpathlineto{\pgfqpoint{1.975084in}{5.744720in}}%
\pgfpathlineto{\pgfqpoint{1.978248in}{5.746437in}}%
\pgfpathlineto{\pgfqpoint{2.007822in}{5.762359in}}%
\pgfpathlineto{\pgfqpoint{2.023909in}{5.770979in}}%
\pgfpathlineto{\pgfqpoint{2.037395in}{5.778115in}}%
\pgfpathlineto{\pgfqpoint{2.066968in}{5.793678in}}%
\pgfpathlineto{\pgfqpoint{2.073783in}{5.797238in}}%
\pgfusepath{stroke}%
\end{pgfscope}%
\begin{pgfscope}%
\pgfpathrectangle{\pgfqpoint{0.854460in}{0.571603in}}{\pgfqpoint{5.885100in}{5.225635in}}%
\pgfusepath{clip}%
\pgfsetbuttcap%
\pgfsetroundjoin%
\pgfsetlinewidth{1.505625pt}%
\definecolor{currentstroke}{rgb}{0.132268,0.655014,0.519661}%
\pgfsetstrokecolor{currentstroke}%
\pgfsetdash{}{0pt}%
\pgfpathmoveto{\pgfqpoint{6.632130in}{5.797238in}}%
\pgfpathlineto{\pgfqpoint{6.624383in}{5.770979in}}%
\pgfpathlineto{\pgfqpoint{6.621267in}{5.760835in}}%
\pgfpathlineto{\pgfqpoint{6.616777in}{5.746181in}}%
\pgfusepath{stroke}%
\end{pgfscope}%
\begin{pgfscope}%
\pgfpathrectangle{\pgfqpoint{0.854460in}{0.571603in}}{\pgfqpoint{5.885100in}{5.225635in}}%
\pgfusepath{clip}%
\pgfsetbuttcap%
\pgfsetroundjoin%
\pgfsetlinewidth{1.505625pt}%
\definecolor{currentstroke}{rgb}{0.132268,0.655014,0.519661}%
\pgfsetstrokecolor{currentstroke}%
\pgfsetdash{}{0pt}%
\pgfpathmoveto{\pgfqpoint{6.486197in}{5.377655in}}%
\pgfpathlineto{\pgfqpoint{6.358027in}{5.035714in}}%
\pgfpathlineto{\pgfqpoint{6.311512in}{4.904416in}}%
\pgfpathlineto{\pgfqpoint{6.276261in}{4.799378in}}%
\pgfpathlineto{\pgfqpoint{6.243087in}{4.694341in}}%
\pgfpathlineto{\pgfqpoint{6.212267in}{4.589303in}}%
\pgfpathlineto{\pgfqpoint{6.184045in}{4.484265in}}%
\pgfpathlineto{\pgfqpoint{6.158647in}{4.379227in}}%
\pgfpathlineto{\pgfqpoint{6.136280in}{4.274189in}}%
\pgfpathlineto{\pgfqpoint{6.118520in}{4.177219in}}%
\pgfpathlineto{\pgfqpoint{6.108787in}{4.116632in}}%
\pgfpathlineto{\pgfqpoint{6.097906in}{4.037854in}}%
\pgfpathlineto{\pgfqpoint{6.088946in}{3.958099in}}%
\pgfpathlineto{\pgfqpoint{6.082168in}{3.880297in}}%
\pgfpathlineto{\pgfqpoint{6.077403in}{3.801519in}}%
\pgfpathlineto{\pgfqpoint{6.074781in}{3.722740in}}%
\pgfpathlineto{\pgfqpoint{6.074341in}{3.643962in}}%
\pgfpathlineto{\pgfqpoint{6.076119in}{3.565183in}}%
\pgfpathlineto{\pgfqpoint{6.080156in}{3.486405in}}%
\pgfpathlineto{\pgfqpoint{6.086492in}{3.407626in}}%
\pgfpathlineto{\pgfqpoint{6.095120in}{3.328848in}}%
\pgfpathlineto{\pgfqpoint{6.106096in}{3.250070in}}%
\pgfpathlineto{\pgfqpoint{6.119477in}{3.171291in}}%
\pgfpathlineto{\pgfqpoint{6.135204in}{3.092513in}}%
\pgfpathlineto{\pgfqpoint{6.153385in}{3.013734in}}%
\pgfpathlineto{\pgfqpoint{6.173994in}{2.934956in}}%
\pgfpathlineto{\pgfqpoint{6.197040in}{2.856177in}}%
\pgfpathlineto{\pgfqpoint{6.222566in}{2.777399in}}%
\pgfpathlineto{\pgfqpoint{6.250578in}{2.698621in}}%
\pgfpathlineto{\pgfqpoint{6.281087in}{2.619842in}}%
\pgfpathlineto{\pgfqpoint{6.314104in}{2.541064in}}%
\pgfpathlineto{\pgfqpoint{6.355107in}{2.450758in}}%
\pgfpathlineto{\pgfqpoint{6.387712in}{2.383507in}}%
\pgfpathlineto{\pgfqpoint{6.428285in}{2.304729in}}%
\pgfpathlineto{\pgfqpoint{6.473400in}{2.222502in}}%
\pgfpathlineto{\pgfqpoint{6.517084in}{2.147172in}}%
\pgfpathlineto{\pgfqpoint{6.565318in}{2.068393in}}%
\pgfpathlineto{\pgfqpoint{6.621267in}{1.981835in}}%
\pgfpathlineto{\pgfqpoint{6.669382in}{1.910836in}}%
\pgfpathlineto{\pgfqpoint{6.725245in}{1.832058in}}%
\pgfpathlineto{\pgfqpoint{6.739560in}{1.812493in}}%
\pgfpathlineto{\pgfqpoint{6.739560in}{1.812493in}}%
\pgfusepath{stroke}%
\end{pgfscope}%
\begin{pgfscope}%
\pgfpathrectangle{\pgfqpoint{0.854460in}{0.571603in}}{\pgfqpoint{5.885100in}{5.225635in}}%
\pgfusepath{clip}%
\pgfsetbuttcap%
\pgfsetroundjoin%
\pgfsetlinewidth{1.505625pt}%
\definecolor{currentstroke}{rgb}{0.146616,0.673050,0.508936}%
\pgfsetstrokecolor{currentstroke}%
\pgfsetdash{}{0pt}%
\pgfpathmoveto{\pgfqpoint{0.936926in}{0.571603in}}%
\pgfpathlineto{\pgfqpoint{0.913607in}{0.594771in}}%
\pgfpathlineto{\pgfqpoint{0.910512in}{0.597863in}}%
\pgfpathlineto{\pgfqpoint{0.884585in}{0.624122in}}%
\pgfpathlineto{\pgfqpoint{0.884034in}{0.624689in}}%
\pgfpathlineto{\pgfqpoint{0.859178in}{0.650382in}}%
\pgfpathlineto{\pgfqpoint{0.854460in}{0.655326in}}%
\pgfusepath{stroke}%
\end{pgfscope}%
\begin{pgfscope}%
\pgfpathrectangle{\pgfqpoint{0.854460in}{0.571603in}}{\pgfqpoint{5.885100in}{5.225635in}}%
\pgfusepath{clip}%
\pgfsetbuttcap%
\pgfsetroundjoin%
\pgfsetlinewidth{1.505625pt}%
\definecolor{currentstroke}{rgb}{0.146616,0.673050,0.508936}%
\pgfsetstrokecolor{currentstroke}%
\pgfsetdash{}{0pt}%
\pgfpathmoveto{\pgfqpoint{0.854460in}{4.969488in}}%
\pgfpathlineto{\pgfqpoint{0.867775in}{4.983195in}}%
\pgfpathlineto{\pgfqpoint{0.884034in}{4.999724in}}%
\pgfpathlineto{\pgfqpoint{0.893694in}{5.009454in}}%
\pgfpathlineto{\pgfqpoint{0.913607in}{5.029264in}}%
\pgfpathlineto{\pgfqpoint{0.920150in}{5.035714in}}%
\pgfpathlineto{\pgfqpoint{0.943181in}{5.058136in}}%
\pgfpathlineto{\pgfqpoint{0.947158in}{5.061973in}}%
\pgfpathlineto{\pgfqpoint{0.972754in}{5.086366in}}%
\pgfpathlineto{\pgfqpoint{0.974730in}{5.088233in}}%
\pgfpathlineto{\pgfqpoint{1.002327in}{5.113980in}}%
\pgfpathlineto{\pgfqpoint{1.002881in}{5.114492in}}%
\pgfpathlineto{\pgfqpoint{1.031627in}{5.140752in}}%
\pgfpathlineto{\pgfqpoint{1.031901in}{5.140998in}}%
\pgfpathlineto{\pgfqpoint{1.060976in}{5.167011in}}%
\pgfpathlineto{\pgfqpoint{1.061474in}{5.167451in}}%
\pgfpathlineto{\pgfqpoint{1.090933in}{5.193271in}}%
\pgfpathlineto{\pgfqpoint{1.091047in}{5.193369in}}%
\pgfpathlineto{\pgfqpoint{1.120621in}{5.218761in}}%
\pgfpathlineto{\pgfqpoint{1.121524in}{5.219530in}}%
\pgfpathlineto{\pgfqpoint{1.150194in}{5.243650in}}%
\pgfpathlineto{\pgfqpoint{1.152758in}{5.245790in}}%
\pgfpathlineto{\pgfqpoint{1.179767in}{5.268059in}}%
\pgfpathlineto{\pgfqpoint{1.184645in}{5.272049in}}%
\pgfpathlineto{\pgfqpoint{1.209341in}{5.292007in}}%
\pgfpathlineto{\pgfqpoint{1.217198in}{5.298308in}}%
\pgfpathlineto{\pgfqpoint{1.237679in}{5.314534in}}%
\pgfusepath{stroke}%
\end{pgfscope}%
\begin{pgfscope}%
\pgfpathrectangle{\pgfqpoint{0.854460in}{0.571603in}}{\pgfqpoint{5.885100in}{5.225635in}}%
\pgfusepath{clip}%
\pgfsetbuttcap%
\pgfsetroundjoin%
\pgfsetlinewidth{1.505625pt}%
\definecolor{currentstroke}{rgb}{0.146616,0.673050,0.508936}%
\pgfsetstrokecolor{currentstroke}%
\pgfsetdash{}{0pt}%
\pgfpathmoveto{\pgfqpoint{1.547574in}{5.537931in}}%
\pgfpathlineto{\pgfqpoint{1.564221in}{5.548866in}}%
\pgfpathlineto{\pgfqpoint{1.582655in}{5.560903in}}%
\pgfpathlineto{\pgfqpoint{1.593795in}{5.568089in}}%
\pgfpathlineto{\pgfqpoint{1.623368in}{5.587057in}}%
\pgfpathlineto{\pgfqpoint{1.623534in}{5.587163in}}%
\pgfpathlineto{\pgfqpoint{1.652941in}{5.605639in}}%
\pgfpathlineto{\pgfqpoint{1.665396in}{5.613422in}}%
\pgfpathlineto{\pgfqpoint{1.682515in}{5.623989in}}%
\pgfpathlineto{\pgfqpoint{1.708073in}{5.639682in}}%
\pgfpathlineto{\pgfqpoint{1.712088in}{5.642117in}}%
\pgfpathlineto{\pgfqpoint{1.741661in}{5.659921in}}%
\pgfpathlineto{\pgfqpoint{1.751722in}{5.665941in}}%
\pgfpathlineto{\pgfqpoint{1.771235in}{5.677474in}}%
\pgfpathlineto{\pgfqpoint{1.796270in}{5.692201in}}%
\pgfpathlineto{\pgfqpoint{1.800808in}{5.694837in}}%
\pgfpathlineto{\pgfqpoint{1.830381in}{5.711893in}}%
\pgfpathlineto{\pgfqpoint{1.841833in}{5.718460in}}%
\pgfpathlineto{\pgfqpoint{1.859955in}{5.728724in}}%
\pgfpathlineto{\pgfqpoint{1.888320in}{5.744720in}}%
\pgfpathlineto{\pgfqpoint{1.889528in}{5.745393in}}%
\pgfpathlineto{\pgfqpoint{1.919102in}{5.761729in}}%
\pgfpathlineto{\pgfqpoint{1.935922in}{5.770979in}}%
\pgfpathlineto{\pgfqpoint{1.948675in}{5.777906in}}%
\pgfpathlineto{\pgfqpoint{1.978248in}{5.793881in}}%
\pgfpathlineto{\pgfqpoint{1.984508in}{5.797238in}}%
\pgfusepath{stroke}%
\end{pgfscope}%
\begin{pgfscope}%
\pgfpathrectangle{\pgfqpoint{0.854460in}{0.571603in}}{\pgfqpoint{5.885100in}{5.225635in}}%
\pgfusepath{clip}%
\pgfsetbuttcap%
\pgfsetroundjoin%
\pgfsetlinewidth{1.505625pt}%
\definecolor{currentstroke}{rgb}{0.146616,0.673050,0.508936}%
\pgfsetstrokecolor{currentstroke}%
\pgfsetdash{}{0pt}%
\pgfpathmoveto{\pgfqpoint{6.737069in}{5.797238in}}%
\pgfpathlineto{\pgfqpoint{6.728128in}{5.770979in}}%
\pgfpathlineto{\pgfqpoint{6.718992in}{5.744720in}}%
\pgfpathlineto{\pgfqpoint{6.709987in}{5.719330in}}%
\pgfpathlineto{\pgfqpoint{6.709679in}{5.718460in}}%
\pgfpathlineto{\pgfqpoint{6.700114in}{5.692201in}}%
\pgfpathlineto{\pgfqpoint{6.690416in}{5.665941in}}%
\pgfpathlineto{\pgfqpoint{6.680603in}{5.639682in}}%
\pgfpathlineto{\pgfqpoint{6.680414in}{5.639186in}}%
\pgfpathlineto{\pgfqpoint{6.670588in}{5.613422in}}%
\pgfpathlineto{\pgfqpoint{6.660491in}{5.587163in}}%
\pgfpathlineto{\pgfqpoint{6.650840in}{5.562244in}}%
\pgfpathlineto{\pgfqpoint{6.650322in}{5.560903in}}%
\pgfpathlineto{\pgfqpoint{6.640000in}{5.534644in}}%
\pgfpathlineto{\pgfqpoint{6.629640in}{5.508384in}}%
\pgfpathlineto{\pgfqpoint{6.621267in}{5.487257in}}%
\pgfpathlineto{\pgfqpoint{6.619235in}{5.482125in}}%
\pgfpathlineto{\pgfqpoint{6.608731in}{5.455865in}}%
\pgfpathlineto{\pgfqpoint{6.598227in}{5.429606in}}%
\pgfpathlineto{\pgfqpoint{6.591693in}{5.413325in}}%
\pgfpathlineto{\pgfqpoint{6.587692in}{5.403346in}}%
\pgfpathlineto{\pgfqpoint{6.577112in}{5.377087in}}%
\pgfpathlineto{\pgfqpoint{6.566563in}{5.350827in}}%
\pgfpathlineto{\pgfqpoint{6.562120in}{5.339791in}}%
\pgfpathlineto{\pgfqpoint{6.555993in}{5.324568in}}%
\pgfpathlineto{\pgfqpoint{6.545429in}{5.298308in}}%
\pgfpathlineto{\pgfqpoint{6.534921in}{5.272049in}}%
\pgfpathlineto{\pgfqpoint{6.532547in}{5.266121in}}%
\pgfpathlineto{\pgfqpoint{6.524401in}{5.245790in}}%
\pgfpathlineto{\pgfqpoint{6.513931in}{5.219530in}}%
\pgfpathlineto{\pgfqpoint{6.503541in}{5.193271in}}%
\pgfpathlineto{\pgfqpoint{6.502973in}{5.191836in}}%
\pgfpathlineto{\pgfqpoint{6.493145in}{5.167011in}}%
\pgfpathlineto{\pgfqpoint{6.482839in}{5.140752in}}%
\pgfpathlineto{\pgfqpoint{6.473400in}{5.116483in}}%
\pgfpathlineto{\pgfqpoint{6.472625in}{5.114492in}}%
\pgfpathlineto{\pgfqpoint{6.462430in}{5.088233in}}%
\pgfpathlineto{\pgfqpoint{6.452348in}{5.061973in}}%
\pgfpathlineto{\pgfqpoint{6.443827in}{5.039545in}}%
\pgfpathlineto{\pgfqpoint{6.442369in}{5.035714in}}%
\pgfpathlineto{\pgfqpoint{6.432436in}{5.009454in}}%
\pgfpathlineto{\pgfqpoint{6.422632in}{4.983195in}}%
\pgfpathlineto{\pgfqpoint{6.414253in}{4.960469in}}%
\pgfpathlineto{\pgfqpoint{6.412948in}{4.956935in}}%
\pgfpathlineto{\pgfqpoint{6.403327in}{4.930676in}}%
\pgfpathlineto{\pgfqpoint{6.393849in}{4.904416in}}%
\pgfpathlineto{\pgfqpoint{6.384680in}{4.878621in}}%
\pgfpathlineto{\pgfqpoint{6.384515in}{4.878157in}}%
\pgfpathlineto{\pgfqpoint{6.375250in}{4.851897in}}%
\pgfpathlineto{\pgfqpoint{6.366140in}{4.825638in}}%
\pgfpathlineto{\pgfqpoint{6.357189in}{4.799378in}}%
\pgfpathlineto{\pgfqpoint{6.355107in}{4.793189in}}%
\pgfpathlineto{\pgfqpoint{6.348339in}{4.773119in}}%
\pgfpathlineto{\pgfqpoint{6.339638in}{4.746860in}}%
\pgfpathlineto{\pgfqpoint{6.331105in}{4.720600in}}%
\pgfpathlineto{\pgfqpoint{6.325533in}{4.703138in}}%
\pgfpathlineto{\pgfqpoint{6.322719in}{4.694341in}}%
\pgfpathlineto{\pgfqpoint{6.314461in}{4.668081in}}%
\pgfpathlineto{\pgfqpoint{6.306381in}{4.641822in}}%
\pgfpathlineto{\pgfqpoint{6.298482in}{4.615562in}}%
\pgfpathlineto{\pgfqpoint{6.295960in}{4.607011in}}%
\pgfpathlineto{\pgfqpoint{6.290721in}{4.589303in}}%
\pgfpathlineto{\pgfqpoint{6.283126in}{4.563043in}}%
\pgfpathlineto{\pgfqpoint{6.275721in}{4.536784in}}%
\pgfpathlineto{\pgfqpoint{6.268507in}{4.510524in}}%
\pgfpathlineto{\pgfqpoint{6.266386in}{4.502622in}}%
\pgfpathlineto{\pgfqpoint{6.261444in}{4.484265in}}%
\pgfpathlineto{\pgfqpoint{6.254561in}{4.458005in}}%
\pgfpathlineto{\pgfqpoint{6.247877in}{4.431746in}}%
\pgfpathlineto{\pgfqpoint{6.241394in}{4.405486in}}%
\pgfpathlineto{\pgfqpoint{6.236813in}{4.386361in}}%
\pgfpathlineto{\pgfqpoint{6.235098in}{4.379227in}}%
\pgfpathlineto{\pgfqpoint{6.228971in}{4.352967in}}%
\pgfpathlineto{\pgfqpoint{6.223051in}{4.326708in}}%
\pgfpathlineto{\pgfqpoint{6.217340in}{4.300449in}}%
\pgfpathlineto{\pgfqpoint{6.211841in}{4.274189in}}%
\pgfpathlineto{\pgfqpoint{6.211678in}{4.273380in}}%
\pgfusepath{stroke}%
\end{pgfscope}%
\begin{pgfscope}%
\pgfpathrectangle{\pgfqpoint{0.854460in}{0.571603in}}{\pgfqpoint{5.885100in}{5.225635in}}%
\pgfusepath{clip}%
\pgfsetbuttcap%
\pgfsetroundjoin%
\pgfsetlinewidth{1.505625pt}%
\definecolor{currentstroke}{rgb}{0.146616,0.673050,0.508936}%
\pgfsetstrokecolor{currentstroke}%
\pgfsetdash{}{0pt}%
\pgfpathmoveto{\pgfqpoint{6.155420in}{3.885115in}}%
\pgfpathlineto{\pgfqpoint{6.155016in}{3.880297in}}%
\pgfpathlineto{\pgfqpoint{6.153049in}{3.854037in}}%
\pgfpathlineto{\pgfqpoint{6.151320in}{3.827778in}}%
\pgfpathlineto{\pgfqpoint{6.149829in}{3.801519in}}%
\pgfpathlineto{\pgfqpoint{6.148579in}{3.775259in}}%
\pgfpathlineto{\pgfqpoint{6.148093in}{3.762624in}}%
\pgfpathlineto{\pgfqpoint{6.147566in}{3.749000in}}%
\pgfpathlineto{\pgfqpoint{6.146794in}{3.722740in}}%
\pgfpathlineto{\pgfqpoint{6.146268in}{3.696481in}}%
\pgfpathlineto{\pgfqpoint{6.145989in}{3.670221in}}%
\pgfpathlineto{\pgfqpoint{6.145959in}{3.643962in}}%
\pgfpathlineto{\pgfqpoint{6.146179in}{3.617702in}}%
\pgfpathlineto{\pgfqpoint{6.146651in}{3.591443in}}%
\pgfpathlineto{\pgfqpoint{6.147375in}{3.565183in}}%
\pgfpathlineto{\pgfqpoint{6.148093in}{3.545930in}}%
\pgfpathlineto{\pgfqpoint{6.148352in}{3.538924in}}%
\pgfpathlineto{\pgfqpoint{6.149576in}{3.512664in}}%
\pgfpathlineto{\pgfqpoint{6.151056in}{3.486405in}}%
\pgfpathlineto{\pgfqpoint{6.152792in}{3.460145in}}%
\pgfpathlineto{\pgfqpoint{6.154786in}{3.433886in}}%
\pgfpathlineto{\pgfqpoint{6.157039in}{3.407626in}}%
\pgfpathlineto{\pgfqpoint{6.159554in}{3.381367in}}%
\pgfpathlineto{\pgfqpoint{6.162331in}{3.355107in}}%
\pgfpathlineto{\pgfqpoint{6.165372in}{3.328848in}}%
\pgfpathlineto{\pgfqpoint{6.168679in}{3.302589in}}%
\pgfpathlineto{\pgfqpoint{6.172253in}{3.276329in}}%
\pgfpathlineto{\pgfqpoint{6.176096in}{3.250070in}}%
\pgfpathlineto{\pgfqpoint{6.177666in}{3.240048in}}%
\pgfpathlineto{\pgfqpoint{6.180190in}{3.223810in}}%
\pgfpathlineto{\pgfqpoint{6.184543in}{3.197551in}}%
\pgfpathlineto{\pgfqpoint{6.189167in}{3.171291in}}%
\pgfpathlineto{\pgfqpoint{6.194066in}{3.145032in}}%
\pgfpathlineto{\pgfqpoint{6.199240in}{3.118772in}}%
\pgfpathlineto{\pgfqpoint{6.204691in}{3.092513in}}%
\pgfpathlineto{\pgfqpoint{6.207240in}{3.080834in}}%
\pgfpathlineto{\pgfqpoint{6.210398in}{3.066253in}}%
\pgfpathlineto{\pgfqpoint{6.216364in}{3.039994in}}%
\pgfpathlineto{\pgfqpoint{6.222612in}{3.013734in}}%
\pgfpathlineto{\pgfqpoint{6.229143in}{2.987475in}}%
\pgfpathlineto{\pgfqpoint{6.235959in}{2.961215in}}%
\pgfpathlineto{\pgfqpoint{6.236813in}{2.958055in}}%
\pgfpathlineto{\pgfqpoint{6.243014in}{2.934956in}}%
\pgfpathlineto{\pgfqpoint{6.250351in}{2.908696in}}%
\pgfpathlineto{\pgfqpoint{6.257977in}{2.882437in}}%
\pgfpathlineto{\pgfqpoint{6.265894in}{2.856177in}}%
\pgfpathlineto{\pgfqpoint{6.266386in}{2.854601in}}%
\pgfpathlineto{\pgfqpoint{6.274047in}{2.829918in}}%
\pgfpathlineto{\pgfqpoint{6.282490in}{2.803659in}}%
\pgfpathlineto{\pgfqpoint{6.291229in}{2.777399in}}%
\pgfpathlineto{\pgfqpoint{6.295960in}{2.763645in}}%
\pgfpathlineto{\pgfqpoint{6.300235in}{2.751140in}}%
\pgfpathlineto{\pgfqpoint{6.309503in}{2.724880in}}%
\pgfpathlineto{\pgfqpoint{6.319073in}{2.698621in}}%
\pgfpathlineto{\pgfqpoint{6.325533in}{2.681428in}}%
\pgfpathlineto{\pgfqpoint{6.328920in}{2.672361in}}%
\pgfpathlineto{\pgfqpoint{6.339024in}{2.646102in}}%
\pgfpathlineto{\pgfqpoint{6.349433in}{2.619842in}}%
\pgfpathlineto{\pgfqpoint{6.355107in}{2.605931in}}%
\pgfpathlineto{\pgfqpoint{6.360115in}{2.593583in}}%
\pgfpathlineto{\pgfqpoint{6.371063in}{2.567323in}}%
\pgfpathlineto{\pgfqpoint{6.382324in}{2.541064in}}%
\pgfpathlineto{\pgfqpoint{6.384680in}{2.535709in}}%
\pgfpathlineto{\pgfqpoint{6.393831in}{2.514804in}}%
\pgfpathlineto{\pgfqpoint{6.405636in}{2.488545in}}%
\pgfpathlineto{\pgfqpoint{6.414253in}{2.469866in}}%
\pgfpathlineto{\pgfqpoint{6.417733in}{2.462285in}}%
\pgfpathlineto{\pgfqpoint{6.430086in}{2.436026in}}%
\pgfpathlineto{\pgfqpoint{6.442760in}{2.409766in}}%
\pgfpathlineto{\pgfqpoint{6.443827in}{2.407606in}}%
\pgfpathlineto{\pgfqpoint{6.455671in}{2.383507in}}%
\pgfpathlineto{\pgfqpoint{6.468900in}{2.357248in}}%
\pgfpathlineto{\pgfqpoint{6.473400in}{2.348514in}}%
\pgfpathlineto{\pgfqpoint{6.482390in}{2.330988in}}%
\pgfpathlineto{\pgfqpoint{6.496177in}{2.304729in}}%
\pgfpathlineto{\pgfqpoint{6.502973in}{2.292069in}}%
\pgfpathlineto{\pgfqpoint{6.510243in}{2.278469in}}%
\pgfpathlineto{\pgfqpoint{6.524592in}{2.252210in}}%
\pgfpathlineto{\pgfqpoint{6.532547in}{2.237966in}}%
\pgfpathlineto{\pgfqpoint{6.539229in}{2.225950in}}%
\pgfpathlineto{\pgfqpoint{6.554145in}{2.199691in}}%
\pgfpathlineto{\pgfqpoint{6.562120in}{2.185943in}}%
\pgfpathlineto{\pgfqpoint{6.569350in}{2.173431in}}%
\pgfpathlineto{\pgfqpoint{6.584838in}{2.147172in}}%
\pgfpathlineto{\pgfqpoint{6.591693in}{2.135778in}}%
\pgfpathlineto{\pgfqpoint{6.600606in}{2.120912in}}%
\pgfpathlineto{\pgfqpoint{6.616671in}{2.094653in}}%
\pgfpathlineto{\pgfqpoint{6.621267in}{2.087280in}}%
\pgfpathlineto{\pgfqpoint{6.633000in}{2.068393in}}%
\pgfpathlineto{\pgfqpoint{6.649648in}{2.042134in}}%
\pgfpathlineto{\pgfqpoint{6.650840in}{2.040285in}}%
\pgfpathlineto{\pgfqpoint{6.666536in}{2.015874in}}%
\pgfpathlineto{\pgfqpoint{6.680414in}{1.994714in}}%
\pgfpathlineto{\pgfqpoint{6.683747in}{1.989615in}}%
\pgfpathlineto{\pgfqpoint{6.701216in}{1.963355in}}%
\pgfpathlineto{\pgfqpoint{6.709987in}{1.950410in}}%
\pgfpathlineto{\pgfqpoint{6.718981in}{1.937096in}}%
\pgfpathlineto{\pgfqpoint{6.737046in}{1.910836in}}%
\pgfpathlineto{\pgfqpoint{6.739560in}{1.907241in}}%
\pgfusepath{stroke}%
\end{pgfscope}%
\begin{pgfscope}%
\pgfpathrectangle{\pgfqpoint{0.854460in}{0.571603in}}{\pgfqpoint{5.885100in}{5.225635in}}%
\pgfusepath{clip}%
\pgfsetbuttcap%
\pgfsetroundjoin%
\pgfsetlinewidth{1.505625pt}%
\definecolor{currentstroke}{rgb}{0.170948,0.694384,0.493803}%
\pgfsetstrokecolor{currentstroke}%
\pgfsetdash{}{0pt}%
\pgfpathmoveto{\pgfqpoint{0.884205in}{0.571603in}}%
\pgfpathlineto{\pgfqpoint{0.884034in}{0.571774in}}%
\pgfpathlineto{\pgfqpoint{0.858034in}{0.597863in}}%
\pgfpathlineto{\pgfqpoint{0.854460in}{0.601498in}}%
\pgfusepath{stroke}%
\end{pgfscope}%
\begin{pgfscope}%
\pgfpathrectangle{\pgfqpoint{0.854460in}{0.571603in}}{\pgfqpoint{5.885100in}{5.225635in}}%
\pgfusepath{clip}%
\pgfsetbuttcap%
\pgfsetroundjoin%
\pgfsetlinewidth{1.505625pt}%
\definecolor{currentstroke}{rgb}{0.170948,0.694384,0.493803}%
\pgfsetstrokecolor{currentstroke}%
\pgfsetdash{}{0pt}%
\pgfpathmoveto{\pgfqpoint{0.854460in}{5.033443in}}%
\pgfpathlineto{\pgfqpoint{0.856726in}{5.035714in}}%
\pgfpathlineto{\pgfqpoint{0.883257in}{5.061973in}}%
\pgfpathlineto{\pgfqpoint{0.884034in}{5.062732in}}%
\pgfpathlineto{\pgfqpoint{0.910365in}{5.088233in}}%
\pgfpathlineto{\pgfqpoint{0.913607in}{5.091334in}}%
\pgfpathlineto{\pgfqpoint{0.938030in}{5.114492in}}%
\pgfpathlineto{\pgfqpoint{0.943181in}{5.119316in}}%
\pgfpathlineto{\pgfqpoint{0.966264in}{5.140752in}}%
\pgfpathlineto{\pgfqpoint{0.972754in}{5.146704in}}%
\pgfpathlineto{\pgfqpoint{0.995080in}{5.167011in}}%
\pgfpathlineto{\pgfqpoint{1.002327in}{5.173523in}}%
\pgfpathlineto{\pgfqpoint{1.024490in}{5.193271in}}%
\pgfpathlineto{\pgfqpoint{1.031901in}{5.199793in}}%
\pgfpathlineto{\pgfqpoint{1.054507in}{5.219530in}}%
\pgfpathlineto{\pgfqpoint{1.061474in}{5.225538in}}%
\pgfpathlineto{\pgfqpoint{1.085143in}{5.245790in}}%
\pgfpathlineto{\pgfqpoint{1.091047in}{5.250780in}}%
\pgfpathlineto{\pgfqpoint{1.116410in}{5.272049in}}%
\pgfpathlineto{\pgfqpoint{1.120621in}{5.275537in}}%
\pgfpathlineto{\pgfqpoint{1.148319in}{5.298308in}}%
\pgfpathlineto{\pgfqpoint{1.150194in}{5.299831in}}%
\pgfpathlineto{\pgfqpoint{1.179767in}{5.323663in}}%
\pgfpathlineto{\pgfqpoint{1.180900in}{5.324568in}}%
\pgfpathlineto{\pgfqpoint{1.209341in}{5.347031in}}%
\pgfpathlineto{\pgfqpoint{1.214183in}{5.350827in}}%
\pgfpathlineto{\pgfqpoint{1.237296in}{5.368729in}}%
\pgfusepath{stroke}%
\end{pgfscope}%
\begin{pgfscope}%
\pgfpathrectangle{\pgfqpoint{0.854460in}{0.571603in}}{\pgfqpoint{5.885100in}{5.225635in}}%
\pgfusepath{clip}%
\pgfsetbuttcap%
\pgfsetroundjoin%
\pgfsetlinewidth{1.505625pt}%
\definecolor{currentstroke}{rgb}{0.170948,0.694384,0.493803}%
\pgfsetstrokecolor{currentstroke}%
\pgfsetdash{}{0pt}%
\pgfpathmoveto{\pgfqpoint{1.549295in}{5.588930in}}%
\pgfpathlineto{\pgfqpoint{1.564221in}{5.598546in}}%
\pgfpathlineto{\pgfqpoint{1.587443in}{5.613422in}}%
\pgfpathlineto{\pgfqpoint{1.593795in}{5.617441in}}%
\pgfpathlineto{\pgfqpoint{1.623368in}{5.636031in}}%
\pgfpathlineto{\pgfqpoint{1.629215in}{5.639682in}}%
\pgfpathlineto{\pgfqpoint{1.652941in}{5.654314in}}%
\pgfpathlineto{\pgfqpoint{1.671893in}{5.665941in}}%
\pgfpathlineto{\pgfqpoint{1.682515in}{5.672379in}}%
\pgfpathlineto{\pgfqpoint{1.712088in}{5.690199in}}%
\pgfpathlineto{\pgfqpoint{1.715434in}{5.692201in}}%
\pgfpathlineto{\pgfqpoint{1.741661in}{5.707696in}}%
\pgfpathlineto{\pgfqpoint{1.759967in}{5.718460in}}%
\pgfpathlineto{\pgfqpoint{1.771235in}{5.725005in}}%
\pgfpathlineto{\pgfqpoint{1.800808in}{5.742086in}}%
\pgfpathlineto{\pgfqpoint{1.805401in}{5.744720in}}%
\pgfpathlineto{\pgfqpoint{1.830381in}{5.758869in}}%
\pgfpathlineto{\pgfqpoint{1.851854in}{5.770979in}}%
\pgfpathlineto{\pgfqpoint{1.859955in}{5.775491in}}%
\pgfpathlineto{\pgfqpoint{1.889528in}{5.791860in}}%
\pgfpathlineto{\pgfqpoint{1.899303in}{5.797238in}}%
\pgfusepath{stroke}%
\end{pgfscope}%
\begin{pgfscope}%
\pgfpathrectangle{\pgfqpoint{0.854460in}{0.571603in}}{\pgfqpoint{5.885100in}{5.225635in}}%
\pgfusepath{clip}%
\pgfsetbuttcap%
\pgfsetroundjoin%
\pgfsetlinewidth{1.505625pt}%
\definecolor{currentstroke}{rgb}{0.170948,0.694384,0.493803}%
\pgfsetstrokecolor{currentstroke}%
\pgfsetdash{}{0pt}%
\pgfpathmoveto{\pgfqpoint{6.739560in}{5.551949in}}%
\pgfpathlineto{\pgfqpoint{6.532547in}{5.055914in}}%
\pgfpathlineto{\pgfqpoint{6.483999in}{4.930676in}}%
\pgfpathlineto{\pgfqpoint{6.443827in}{4.820983in}}%
\pgfpathlineto{\pgfqpoint{6.409186in}{4.720600in}}%
\pgfpathlineto{\pgfqpoint{6.375428in}{4.615562in}}%
\pgfpathlineto{\pgfqpoint{6.344433in}{4.510524in}}%
\pgfpathlineto{\pgfqpoint{6.316403in}{4.405486in}}%
\pgfpathlineto{\pgfqpoint{6.295960in}{4.320078in}}%
\pgfpathlineto{\pgfqpoint{6.280296in}{4.247930in}}%
\pgfpathlineto{\pgfqpoint{6.265092in}{4.169151in}}%
\pgfpathlineto{\pgfqpoint{6.251802in}{4.090373in}}%
\pgfpathlineto{\pgfqpoint{6.240572in}{4.011594in}}%
\pgfpathlineto{\pgfqpoint{6.231391in}{3.932816in}}%
\pgfpathlineto{\pgfqpoint{6.224323in}{3.854037in}}%
\pgfpathlineto{\pgfqpoint{6.219429in}{3.775259in}}%
\pgfpathlineto{\pgfqpoint{6.216740in}{3.696481in}}%
\pgfpathlineto{\pgfqpoint{6.216291in}{3.617702in}}%
\pgfpathlineto{\pgfqpoint{6.218116in}{3.538924in}}%
\pgfpathlineto{\pgfqpoint{6.222251in}{3.460145in}}%
\pgfpathlineto{\pgfqpoint{6.228732in}{3.381367in}}%
\pgfpathlineto{\pgfqpoint{6.237594in}{3.302589in}}%
\pgfpathlineto{\pgfqpoint{6.248805in}{3.223810in}}%
\pgfpathlineto{\pgfqpoint{6.262470in}{3.145032in}}%
\pgfpathlineto{\pgfqpoint{6.278544in}{3.066253in}}%
\pgfpathlineto{\pgfqpoint{6.297112in}{2.987475in}}%
\pgfpathlineto{\pgfqpoint{6.318114in}{2.908696in}}%
\pgfpathlineto{\pgfqpoint{6.341618in}{2.829918in}}%
\pgfpathlineto{\pgfqpoint{6.367633in}{2.751140in}}%
\pgfpathlineto{\pgfqpoint{6.396167in}{2.672361in}}%
\pgfpathlineto{\pgfqpoint{6.427229in}{2.593583in}}%
\pgfpathlineto{\pgfqpoint{6.460831in}{2.514804in}}%
\pgfpathlineto{\pgfqpoint{6.502513in}{2.424560in}}%
\pgfpathlineto{\pgfqpoint{6.502513in}{2.424560in}}%
\pgfusepath{stroke}%
\end{pgfscope}%
\begin{pgfscope}%
\pgfpathrectangle{\pgfqpoint{0.854460in}{0.571603in}}{\pgfqpoint{5.885100in}{5.225635in}}%
\pgfusepath{clip}%
\pgfsetbuttcap%
\pgfsetroundjoin%
\pgfsetlinewidth{1.505625pt}%
\definecolor{currentstroke}{rgb}{0.170948,0.694384,0.493803}%
\pgfsetstrokecolor{currentstroke}%
\pgfsetdash{}{0pt}%
\pgfpathmoveto{\pgfqpoint{6.689942in}{2.083902in}}%
\pgfpathlineto{\pgfqpoint{6.699576in}{2.068393in}}%
\pgfpathlineto{\pgfqpoint{6.709987in}{2.051966in}}%
\pgfpathlineto{\pgfqpoint{6.716197in}{2.042134in}}%
\pgfpathlineto{\pgfqpoint{6.733098in}{2.015874in}}%
\pgfpathlineto{\pgfqpoint{6.739560in}{2.006016in}}%
\pgfusepath{stroke}%
\end{pgfscope}%
\begin{pgfscope}%
\pgfpathrectangle{\pgfqpoint{0.854460in}{0.571603in}}{\pgfqpoint{5.885100in}{5.225635in}}%
\pgfusepath{clip}%
\pgfsetbuttcap%
\pgfsetroundjoin%
\pgfsetlinewidth{1.505625pt}%
\definecolor{currentstroke}{rgb}{0.196571,0.711827,0.479221}%
\pgfsetstrokecolor{currentstroke}%
\pgfsetdash{}{0pt}%
\pgfpathmoveto{\pgfqpoint{0.854460in}{5.094464in}}%
\pgfpathlineto{\pgfqpoint{0.875240in}{5.114492in}}%
\pgfpathlineto{\pgfqpoint{0.884034in}{5.122864in}}%
\pgfpathlineto{\pgfqpoint{0.902984in}{5.140752in}}%
\pgfpathlineto{\pgfqpoint{0.913607in}{5.150656in}}%
\pgfpathlineto{\pgfqpoint{0.931298in}{5.167011in}}%
\pgfpathlineto{\pgfqpoint{0.943181in}{5.177862in}}%
\pgfpathlineto{\pgfqpoint{0.960193in}{5.193271in}}%
\pgfpathlineto{\pgfqpoint{0.972754in}{5.204508in}}%
\pgfpathlineto{\pgfqpoint{0.989683in}{5.219530in}}%
\pgfpathlineto{\pgfqpoint{1.002327in}{5.230614in}}%
\pgfpathlineto{\pgfqpoint{1.019778in}{5.245790in}}%
\pgfpathlineto{\pgfqpoint{1.031901in}{5.256203in}}%
\pgfpathlineto{\pgfqpoint{1.050492in}{5.272049in}}%
\pgfpathlineto{\pgfqpoint{1.061474in}{5.281296in}}%
\pgfpathlineto{\pgfqpoint{1.081835in}{5.298308in}}%
\pgfpathlineto{\pgfqpoint{1.091047in}{5.305912in}}%
\pgfpathlineto{\pgfqpoint{1.113820in}{5.324568in}}%
\pgfpathlineto{\pgfqpoint{1.120621in}{5.330072in}}%
\pgfpathlineto{\pgfqpoint{1.146457in}{5.350827in}}%
\pgfpathlineto{\pgfqpoint{1.150194in}{5.353793in}}%
\pgfpathlineto{\pgfqpoint{1.179757in}{5.377087in}}%
\pgfpathlineto{\pgfqpoint{1.179767in}{5.377095in}}%
\pgfpathlineto{\pgfqpoint{1.209341in}{5.399930in}}%
\pgfpathlineto{\pgfqpoint{1.213796in}{5.403346in}}%
\pgfpathlineto{\pgfqpoint{1.238914in}{5.422372in}}%
\pgfpathlineto{\pgfqpoint{1.248529in}{5.429606in}}%
\pgfpathlineto{\pgfqpoint{1.260747in}{5.438686in}}%
\pgfusepath{stroke}%
\end{pgfscope}%
\begin{pgfscope}%
\pgfpathrectangle{\pgfqpoint{0.854460in}{0.571603in}}{\pgfqpoint{5.885100in}{5.225635in}}%
\pgfusepath{clip}%
\pgfsetbuttcap%
\pgfsetroundjoin%
\pgfsetlinewidth{1.505625pt}%
\definecolor{currentstroke}{rgb}{0.196571,0.711827,0.479221}%
\pgfsetstrokecolor{currentstroke}%
\pgfsetdash{}{0pt}%
\pgfpathmoveto{\pgfqpoint{1.575956in}{5.653874in}}%
\pgfpathlineto{\pgfqpoint{1.593795in}{5.665085in}}%
\pgfpathlineto{\pgfqpoint{1.595168in}{5.665941in}}%
\pgfpathlineto{\pgfqpoint{1.623368in}{5.683313in}}%
\pgfpathlineto{\pgfqpoint{1.637867in}{5.692201in}}%
\pgfpathlineto{\pgfqpoint{1.652941in}{5.701327in}}%
\pgfpathlineto{\pgfqpoint{1.681374in}{5.718460in}}%
\pgfpathlineto{\pgfqpoint{1.682515in}{5.719139in}}%
\pgfpathlineto{\pgfqpoint{1.712088in}{5.736604in}}%
\pgfpathlineto{\pgfqpoint{1.725897in}{5.744720in}}%
\pgfpathlineto{\pgfqpoint{1.741661in}{5.753871in}}%
\pgfpathlineto{\pgfqpoint{1.771235in}{5.770964in}}%
\pgfpathlineto{\pgfqpoint{1.771262in}{5.770979in}}%
\pgfpathlineto{\pgfqpoint{1.800808in}{5.787711in}}%
\pgfpathlineto{\pgfqpoint{1.817704in}{5.797238in}}%
\pgfusepath{stroke}%
\end{pgfscope}%
\begin{pgfscope}%
\pgfpathrectangle{\pgfqpoint{0.854460in}{0.571603in}}{\pgfqpoint{5.885100in}{5.225635in}}%
\pgfusepath{clip}%
\pgfsetbuttcap%
\pgfsetroundjoin%
\pgfsetlinewidth{1.505625pt}%
\definecolor{currentstroke}{rgb}{0.196571,0.711827,0.479221}%
\pgfsetstrokecolor{currentstroke}%
\pgfsetdash{}{0pt}%
\pgfpathmoveto{\pgfqpoint{6.739560in}{5.350069in}}%
\pgfpathlineto{\pgfqpoint{6.728236in}{5.324568in}}%
\pgfpathlineto{\pgfqpoint{6.716658in}{5.298308in}}%
\pgfpathlineto{\pgfqpoint{6.714626in}{5.293682in}}%
\pgfusepath{stroke}%
\end{pgfscope}%
\begin{pgfscope}%
\pgfpathrectangle{\pgfqpoint{0.854460in}{0.571603in}}{\pgfqpoint{5.885100in}{5.225635in}}%
\pgfusepath{clip}%
\pgfsetbuttcap%
\pgfsetroundjoin%
\pgfsetlinewidth{1.505625pt}%
\definecolor{currentstroke}{rgb}{0.196571,0.711827,0.479221}%
\pgfsetstrokecolor{currentstroke}%
\pgfsetdash{}{0pt}%
\pgfpathmoveto{\pgfqpoint{6.563934in}{4.933562in}}%
\pgfpathlineto{\pgfqpoint{6.532547in}{4.851404in}}%
\pgfpathlineto{\pgfqpoint{6.494655in}{4.746860in}}%
\pgfpathlineto{\pgfqpoint{6.459174in}{4.641822in}}%
\pgfpathlineto{\pgfqpoint{6.426496in}{4.536784in}}%
\pgfpathlineto{\pgfqpoint{6.396813in}{4.431746in}}%
\pgfpathlineto{\pgfqpoint{6.376631in}{4.352967in}}%
\pgfpathlineto{\pgfqpoint{6.355107in}{4.259362in}}%
\pgfpathlineto{\pgfqpoint{6.341880in}{4.195411in}}%
\pgfpathlineto{\pgfqpoint{6.325533in}{4.104974in}}%
\pgfpathlineto{\pgfqpoint{6.315043in}{4.037854in}}%
\pgfpathlineto{\pgfqpoint{6.304725in}{3.959075in}}%
\pgfpathlineto{\pgfqpoint{6.295960in}{3.873313in}}%
\pgfpathlineto{\pgfqpoint{6.290525in}{3.801519in}}%
\pgfpathlineto{\pgfqpoint{6.286716in}{3.722740in}}%
\pgfpathlineto{\pgfqpoint{6.285164in}{3.643962in}}%
\pgfpathlineto{\pgfqpoint{6.285901in}{3.565183in}}%
\pgfpathlineto{\pgfqpoint{6.288961in}{3.486405in}}%
\pgfpathlineto{\pgfqpoint{6.294380in}{3.407626in}}%
\pgfpathlineto{\pgfqpoint{6.302147in}{3.328848in}}%
\pgfpathlineto{\pgfqpoint{6.312320in}{3.250070in}}%
\pgfpathlineto{\pgfqpoint{6.325533in}{3.168089in}}%
\pgfpathlineto{\pgfqpoint{6.339980in}{3.092513in}}%
\pgfpathlineto{\pgfqpoint{6.357519in}{3.013734in}}%
\pgfpathlineto{\pgfqpoint{6.377510in}{2.934956in}}%
\pgfpathlineto{\pgfqpoint{6.400006in}{2.856177in}}%
\pgfpathlineto{\pgfqpoint{6.425027in}{2.777399in}}%
\pgfpathlineto{\pgfqpoint{6.452575in}{2.698621in}}%
\pgfpathlineto{\pgfqpoint{6.482660in}{2.619842in}}%
\pgfpathlineto{\pgfqpoint{6.515292in}{2.541064in}}%
\pgfpathlineto{\pgfqpoint{6.550487in}{2.462285in}}%
\pgfpathlineto{\pgfqpoint{6.591693in}{2.376677in}}%
\pgfpathlineto{\pgfqpoint{6.628590in}{2.304729in}}%
\pgfpathlineto{\pgfqpoint{6.671502in}{2.225950in}}%
\pgfpathlineto{\pgfqpoint{6.717003in}{2.147172in}}%
\pgfpathlineto{\pgfqpoint{6.739560in}{2.109766in}}%
\pgfpathlineto{\pgfqpoint{6.739560in}{2.109766in}}%
\pgfusepath{stroke}%
\end{pgfscope}%
\begin{pgfscope}%
\pgfpathrectangle{\pgfqpoint{0.854460in}{0.571603in}}{\pgfqpoint{5.885100in}{5.225635in}}%
\pgfusepath{clip}%
\pgfsetbuttcap%
\pgfsetroundjoin%
\pgfsetlinewidth{1.505625pt}%
\definecolor{currentstroke}{rgb}{0.232815,0.732247,0.459277}%
\pgfsetstrokecolor{currentstroke}%
\pgfsetdash{}{0pt}%
\pgfpathmoveto{\pgfqpoint{0.854460in}{5.152895in}}%
\pgfpathlineto{\pgfqpoint{0.869485in}{5.167011in}}%
\pgfpathlineto{\pgfqpoint{0.884034in}{5.180513in}}%
\pgfpathlineto{\pgfqpoint{0.897895in}{5.193271in}}%
\pgfpathlineto{\pgfqpoint{0.913607in}{5.207555in}}%
\pgfpathlineto{\pgfqpoint{0.926887in}{5.219530in}}%
\pgfpathlineto{\pgfqpoint{0.943181in}{5.234044in}}%
\pgfpathlineto{\pgfqpoint{0.956472in}{5.245790in}}%
\pgfpathlineto{\pgfqpoint{0.972754in}{5.260003in}}%
\pgfpathlineto{\pgfqpoint{0.986662in}{5.272049in}}%
\pgfpathlineto{\pgfqpoint{1.002327in}{5.285452in}}%
\pgfpathlineto{\pgfqpoint{1.017470in}{5.298308in}}%
\pgfpathlineto{\pgfqpoint{1.031901in}{5.310411in}}%
\pgfpathlineto{\pgfqpoint{1.048907in}{5.324568in}}%
\pgfpathlineto{\pgfqpoint{1.061474in}{5.334902in}}%
\pgfpathlineto{\pgfqpoint{1.080983in}{5.350827in}}%
\pgfpathlineto{\pgfqpoint{1.091047in}{5.358943in}}%
\pgfpathlineto{\pgfqpoint{1.113711in}{5.377087in}}%
\pgfpathlineto{\pgfqpoint{1.120621in}{5.382552in}}%
\pgfpathlineto{\pgfqpoint{1.147100in}{5.403346in}}%
\pgfpathlineto{\pgfqpoint{1.150194in}{5.405747in}}%
\pgfpathlineto{\pgfqpoint{1.179767in}{5.428526in}}%
\pgfpathlineto{\pgfqpoint{1.181181in}{5.429606in}}%
\pgfpathlineto{\pgfqpoint{1.209341in}{5.450872in}}%
\pgfpathlineto{\pgfqpoint{1.215998in}{5.455865in}}%
\pgfpathlineto{\pgfqpoint{1.238914in}{5.472847in}}%
\pgfpathlineto{\pgfqpoint{1.251517in}{5.482125in}}%
\pgfpathlineto{\pgfqpoint{1.259804in}{5.488152in}}%
\pgfusepath{stroke}%
\end{pgfscope}%
\begin{pgfscope}%
\pgfpathrectangle{\pgfqpoint{0.854460in}{0.571603in}}{\pgfqpoint{5.885100in}{5.225635in}}%
\pgfusepath{clip}%
\pgfsetbuttcap%
\pgfsetroundjoin%
\pgfsetlinewidth{1.505625pt}%
\definecolor{currentstroke}{rgb}{0.232815,0.732247,0.459277}%
\pgfsetstrokecolor{currentstroke}%
\pgfsetdash{}{0pt}%
\pgfpathmoveto{\pgfqpoint{1.576740in}{5.700573in}}%
\pgfpathlineto{\pgfqpoint{1.593795in}{5.711069in}}%
\pgfpathlineto{\pgfqpoint{1.605867in}{5.718460in}}%
\pgfpathlineto{\pgfqpoint{1.623368in}{5.729044in}}%
\pgfpathlineto{\pgfqpoint{1.649410in}{5.744720in}}%
\pgfpathlineto{\pgfqpoint{1.652941in}{5.746819in}}%
\pgfpathlineto{\pgfqpoint{1.682515in}{5.764279in}}%
\pgfpathlineto{\pgfqpoint{1.693923in}{5.770979in}}%
\pgfpathlineto{\pgfqpoint{1.712088in}{5.781516in}}%
\pgfpathlineto{\pgfqpoint{1.739307in}{5.797238in}}%
\pgfusepath{stroke}%
\end{pgfscope}%
\begin{pgfscope}%
\pgfpathrectangle{\pgfqpoint{0.854460in}{0.571603in}}{\pgfqpoint{5.885100in}{5.225635in}}%
\pgfusepath{clip}%
\pgfsetbuttcap%
\pgfsetroundjoin%
\pgfsetlinewidth{1.505625pt}%
\definecolor{currentstroke}{rgb}{0.232815,0.732247,0.459277}%
\pgfsetstrokecolor{currentstroke}%
\pgfsetdash{}{0pt}%
\pgfpathmoveto{\pgfqpoint{6.739560in}{5.166380in}}%
\pgfpathlineto{\pgfqpoint{6.728245in}{5.140752in}}%
\pgfpathlineto{\pgfqpoint{6.716788in}{5.114492in}}%
\pgfpathlineto{\pgfqpoint{6.709987in}{5.098755in}}%
\pgfpathlineto{\pgfqpoint{6.705436in}{5.088233in}}%
\pgfpathlineto{\pgfqpoint{6.694176in}{5.061973in}}%
\pgfpathlineto{\pgfqpoint{6.683068in}{5.035714in}}%
\pgfpathlineto{\pgfqpoint{6.680414in}{5.029385in}}%
\pgfpathlineto{\pgfqpoint{6.672046in}{5.009454in}}%
\pgfpathlineto{\pgfqpoint{6.661163in}{4.983195in}}%
\pgfpathlineto{\pgfqpoint{6.650840in}{4.957913in}}%
\pgfpathlineto{\pgfqpoint{6.650440in}{4.956935in}}%
\pgfpathlineto{\pgfqpoint{6.639800in}{4.930676in}}%
\pgfpathlineto{\pgfqpoint{6.629331in}{4.904416in}}%
\pgfpathlineto{\pgfqpoint{6.621267in}{4.883874in}}%
\pgfpathlineto{\pgfqpoint{6.619019in}{4.878157in}}%
\pgfpathlineto{\pgfqpoint{6.608817in}{4.851897in}}%
\pgfpathlineto{\pgfqpoint{6.598797in}{4.825638in}}%
\pgfpathlineto{\pgfqpoint{6.591693in}{4.806708in}}%
\pgfpathlineto{\pgfqpoint{6.588938in}{4.799378in}}%
\pgfpathlineto{\pgfqpoint{6.579207in}{4.773119in}}%
\pgfpathlineto{\pgfqpoint{6.569666in}{4.746860in}}%
\pgfpathlineto{\pgfqpoint{6.562120in}{4.725691in}}%
\pgfpathlineto{\pgfqpoint{6.560301in}{4.720600in}}%
\pgfpathlineto{\pgfqpoint{6.551070in}{4.694341in}}%
\pgfpathlineto{\pgfqpoint{6.542037in}{4.668081in}}%
\pgfpathlineto{\pgfqpoint{6.533202in}{4.641822in}}%
\pgfpathlineto{\pgfqpoint{6.532547in}{4.639840in}}%
\pgfpathlineto{\pgfqpoint{6.530733in}{4.634367in}}%
\pgfusepath{stroke}%
\end{pgfscope}%
\begin{pgfscope}%
\pgfpathrectangle{\pgfqpoint{0.854460in}{0.571603in}}{\pgfqpoint{5.885100in}{5.225635in}}%
\pgfusepath{clip}%
\pgfsetbuttcap%
\pgfsetroundjoin%
\pgfsetlinewidth{1.505625pt}%
\definecolor{currentstroke}{rgb}{0.232815,0.732247,0.459277}%
\pgfsetstrokecolor{currentstroke}%
\pgfsetdash{}{0pt}%
\pgfpathmoveto{\pgfqpoint{6.425615in}{4.257219in}}%
\pgfpathlineto{\pgfqpoint{6.423542in}{4.247930in}}%
\pgfpathlineto{\pgfqpoint{6.417911in}{4.221670in}}%
\pgfpathlineto{\pgfqpoint{6.414253in}{4.203910in}}%
\pgfpathlineto{\pgfqpoint{6.412495in}{4.195411in}}%
\pgfpathlineto{\pgfqpoint{6.407283in}{4.169151in}}%
\pgfpathlineto{\pgfqpoint{6.402303in}{4.142892in}}%
\pgfpathlineto{\pgfqpoint{6.397556in}{4.116632in}}%
\pgfpathlineto{\pgfqpoint{6.393044in}{4.090373in}}%
\pgfpathlineto{\pgfqpoint{6.388768in}{4.064113in}}%
\pgfpathlineto{\pgfqpoint{6.384728in}{4.037854in}}%
\pgfpathlineto{\pgfqpoint{6.384680in}{4.037525in}}%
\pgfpathlineto{\pgfqpoint{6.380897in}{4.011594in}}%
\pgfpathlineto{\pgfqpoint{6.377306in}{3.985335in}}%
\pgfpathlineto{\pgfqpoint{6.373956in}{3.959075in}}%
\pgfpathlineto{\pgfqpoint{6.370849in}{3.932816in}}%
\pgfpathlineto{\pgfqpoint{6.367986in}{3.906556in}}%
\pgfpathlineto{\pgfqpoint{6.365367in}{3.880297in}}%
\pgfpathlineto{\pgfqpoint{6.362994in}{3.854037in}}%
\pgfpathlineto{\pgfqpoint{6.360867in}{3.827778in}}%
\pgfpathlineto{\pgfqpoint{6.358988in}{3.801519in}}%
\pgfpathlineto{\pgfqpoint{6.357357in}{3.775259in}}%
\pgfpathlineto{\pgfqpoint{6.355976in}{3.749000in}}%
\pgfpathlineto{\pgfqpoint{6.355107in}{3.728801in}}%
\pgfpathlineto{\pgfqpoint{6.354844in}{3.722740in}}%
\pgfpathlineto{\pgfqpoint{6.353959in}{3.696481in}}%
\pgfpathlineto{\pgfqpoint{6.353330in}{3.670221in}}%
\pgfpathlineto{\pgfqpoint{6.352956in}{3.643962in}}%
\pgfpathlineto{\pgfqpoint{6.352839in}{3.617702in}}%
\pgfpathlineto{\pgfqpoint{6.352981in}{3.591443in}}%
\pgfpathlineto{\pgfqpoint{6.353382in}{3.565183in}}%
\pgfpathlineto{\pgfqpoint{6.354044in}{3.538924in}}%
\pgfpathlineto{\pgfqpoint{6.354968in}{3.512664in}}%
\pgfpathlineto{\pgfqpoint{6.355107in}{3.509605in}}%
\pgfpathlineto{\pgfqpoint{6.356147in}{3.486405in}}%
\pgfpathlineto{\pgfqpoint{6.357588in}{3.460145in}}%
\pgfpathlineto{\pgfqpoint{6.359293in}{3.433886in}}%
\pgfpathlineto{\pgfqpoint{6.361263in}{3.407626in}}%
\pgfpathlineto{\pgfqpoint{6.363500in}{3.381367in}}%
\pgfpathlineto{\pgfqpoint{6.366004in}{3.355107in}}%
\pgfpathlineto{\pgfqpoint{6.368778in}{3.328848in}}%
\pgfpathlineto{\pgfqpoint{6.371822in}{3.302589in}}%
\pgfpathlineto{\pgfqpoint{6.375139in}{3.276329in}}%
\pgfpathlineto{\pgfqpoint{6.378729in}{3.250070in}}%
\pgfpathlineto{\pgfqpoint{6.382595in}{3.223810in}}%
\pgfpathlineto{\pgfqpoint{6.384680in}{3.210595in}}%
\pgfpathlineto{\pgfqpoint{6.386723in}{3.197551in}}%
\pgfpathlineto{\pgfqpoint{6.391111in}{3.171291in}}%
\pgfpathlineto{\pgfqpoint{6.395778in}{3.145032in}}%
\pgfpathlineto{\pgfqpoint{6.400725in}{3.118772in}}%
\pgfpathlineto{\pgfqpoint{6.405953in}{3.092513in}}%
\pgfpathlineto{\pgfqpoint{6.411464in}{3.066253in}}%
\pgfpathlineto{\pgfqpoint{6.414253in}{3.053613in}}%
\pgfpathlineto{\pgfqpoint{6.417238in}{3.039994in}}%
\pgfpathlineto{\pgfqpoint{6.423277in}{3.013734in}}%
\pgfpathlineto{\pgfqpoint{6.429603in}{2.987475in}}%
\pgfpathlineto{\pgfqpoint{6.436217in}{2.961215in}}%
\pgfpathlineto{\pgfqpoint{6.443121in}{2.934956in}}%
\pgfpathlineto{\pgfqpoint{6.443827in}{2.932380in}}%
\pgfpathlineto{\pgfqpoint{6.450272in}{2.908696in}}%
\pgfpathlineto{\pgfqpoint{6.457709in}{2.882437in}}%
\pgfpathlineto{\pgfqpoint{6.465441in}{2.856177in}}%
\pgfpathlineto{\pgfqpoint{6.473400in}{2.830143in}}%
\pgfpathlineto{\pgfqpoint{6.473468in}{2.829918in}}%
\pgfpathlineto{\pgfqpoint{6.481736in}{2.803659in}}%
\pgfpathlineto{\pgfqpoint{6.490302in}{2.777399in}}%
\pgfpathlineto{\pgfqpoint{6.499169in}{2.751140in}}%
\pgfpathlineto{\pgfqpoint{6.502973in}{2.740235in}}%
\pgfpathlineto{\pgfqpoint{6.508301in}{2.724880in}}%
\pgfpathlineto{\pgfqpoint{6.517709in}{2.698621in}}%
\pgfpathlineto{\pgfqpoint{6.527423in}{2.672361in}}%
\pgfpathlineto{\pgfqpoint{6.532547in}{2.658923in}}%
\pgfpathlineto{\pgfqpoint{6.537410in}{2.646102in}}%
\pgfpathlineto{\pgfqpoint{6.547669in}{2.619842in}}%
\pgfpathlineto{\pgfqpoint{6.558239in}{2.593583in}}%
\pgfpathlineto{\pgfqpoint{6.562120in}{2.584206in}}%
\pgfpathlineto{\pgfqpoint{6.569073in}{2.567323in}}%
\pgfpathlineto{\pgfqpoint{6.580193in}{2.541064in}}%
\pgfpathlineto{\pgfqpoint{6.591630in}{2.514804in}}%
\pgfpathlineto{\pgfqpoint{6.591693in}{2.514663in}}%
\pgfpathlineto{\pgfqpoint{6.603304in}{2.488545in}}%
\pgfpathlineto{\pgfqpoint{6.615296in}{2.462285in}}%
\pgfpathlineto{\pgfqpoint{6.621267in}{2.449538in}}%
\pgfpathlineto{\pgfqpoint{6.627567in}{2.436026in}}%
\pgfpathlineto{\pgfqpoint{6.640118in}{2.409766in}}%
\pgfpathlineto{\pgfqpoint{6.650840in}{2.387893in}}%
\pgfpathlineto{\pgfqpoint{6.652981in}{2.383507in}}%
\pgfpathlineto{\pgfqpoint{6.666094in}{2.357248in}}%
\pgfpathlineto{\pgfqpoint{6.679538in}{2.330988in}}%
\pgfpathlineto{\pgfqpoint{6.680414in}{2.329314in}}%
\pgfpathlineto{\pgfqpoint{6.693223in}{2.304729in}}%
\pgfpathlineto{\pgfqpoint{6.707236in}{2.278469in}}%
\pgfpathlineto{\pgfqpoint{6.709987in}{2.273422in}}%
\pgfpathlineto{\pgfqpoint{6.721504in}{2.252210in}}%
\pgfpathlineto{\pgfqpoint{6.736091in}{2.225950in}}%
\pgfpathlineto{\pgfqpoint{6.739560in}{2.219832in}}%
\pgfusepath{stroke}%
\end{pgfscope}%
\begin{pgfscope}%
\pgfpathrectangle{\pgfqpoint{0.854460in}{0.571603in}}{\pgfqpoint{5.885100in}{5.225635in}}%
\pgfusepath{clip}%
\pgfsetbuttcap%
\pgfsetroundjoin%
\pgfsetlinewidth{1.505625pt}%
\definecolor{currentstroke}{rgb}{0.266941,0.748751,0.440573}%
\pgfsetstrokecolor{currentstroke}%
\pgfsetdash{}{0pt}%
\pgfpathmoveto{\pgfqpoint{6.739560in}{4.988824in}}%
\pgfpathlineto{\pgfqpoint{6.737162in}{4.983195in}}%
\pgfpathlineto{\pgfqpoint{6.726088in}{4.956935in}}%
\pgfpathlineto{\pgfqpoint{6.722791in}{4.948990in}}%
\pgfusepath{stroke}%
\end{pgfscope}%
\begin{pgfscope}%
\pgfpathrectangle{\pgfqpoint{0.854460in}{0.571603in}}{\pgfqpoint{5.885100in}{5.225635in}}%
\pgfusepath{clip}%
\pgfsetbuttcap%
\pgfsetroundjoin%
\pgfsetlinewidth{1.505625pt}%
\definecolor{currentstroke}{rgb}{0.266941,0.748751,0.440573}%
\pgfsetstrokecolor{currentstroke}%
\pgfsetdash{}{0pt}%
\pgfpathmoveto{\pgfqpoint{6.586045in}{4.582953in}}%
\pgfpathlineto{\pgfqpoint{6.579572in}{4.563043in}}%
\pgfpathlineto{\pgfqpoint{6.571246in}{4.536784in}}%
\pgfpathlineto{\pgfqpoint{6.563136in}{4.510524in}}%
\pgfpathlineto{\pgfqpoint{6.562120in}{4.507161in}}%
\pgfpathlineto{\pgfqpoint{6.555187in}{4.484265in}}%
\pgfpathlineto{\pgfqpoint{6.547448in}{4.458005in}}%
\pgfpathlineto{\pgfqpoint{6.539930in}{4.431746in}}%
\pgfpathlineto{\pgfqpoint{6.532633in}{4.405486in}}%
\pgfpathlineto{\pgfqpoint{6.532547in}{4.405169in}}%
\pgfpathlineto{\pgfqpoint{6.525503in}{4.379227in}}%
\pgfpathlineto{\pgfqpoint{6.518598in}{4.352967in}}%
\pgfpathlineto{\pgfqpoint{6.511920in}{4.326708in}}%
\pgfpathlineto{\pgfqpoint{6.505468in}{4.300449in}}%
\pgfpathlineto{\pgfqpoint{6.502973in}{4.289949in}}%
\pgfpathlineto{\pgfqpoint{6.499216in}{4.274189in}}%
\pgfpathlineto{\pgfqpoint{6.493175in}{4.247930in}}%
\pgfpathlineto{\pgfqpoint{6.487367in}{4.221670in}}%
\pgfpathlineto{\pgfqpoint{6.481791in}{4.195411in}}%
\pgfpathlineto{\pgfqpoint{6.476450in}{4.169151in}}%
\pgfpathlineto{\pgfqpoint{6.473400in}{4.153492in}}%
\pgfpathlineto{\pgfqpoint{6.471327in}{4.142892in}}%
\pgfpathlineto{\pgfqpoint{6.466419in}{4.116632in}}%
\pgfpathlineto{\pgfqpoint{6.461749in}{4.090373in}}%
\pgfpathlineto{\pgfqpoint{6.457319in}{4.064113in}}%
\pgfpathlineto{\pgfqpoint{6.453129in}{4.037854in}}%
\pgfpathlineto{\pgfqpoint{6.449181in}{4.011594in}}%
\pgfpathlineto{\pgfqpoint{6.445474in}{3.985335in}}%
\pgfpathlineto{\pgfqpoint{6.443827in}{3.972859in}}%
\pgfpathlineto{\pgfqpoint{6.441998in}{3.959075in}}%
\pgfpathlineto{\pgfqpoint{6.438754in}{3.932816in}}%
\pgfpathlineto{\pgfqpoint{6.435758in}{3.906556in}}%
\pgfpathlineto{\pgfqpoint{6.433009in}{3.880297in}}%
\pgfpathlineto{\pgfqpoint{6.430509in}{3.854037in}}%
\pgfpathlineto{\pgfqpoint{6.428259in}{3.827778in}}%
\pgfpathlineto{\pgfqpoint{6.426259in}{3.801519in}}%
\pgfpathlineto{\pgfqpoint{6.424512in}{3.775259in}}%
\pgfpathlineto{\pgfqpoint{6.423016in}{3.749000in}}%
\pgfpathlineto{\pgfqpoint{6.421775in}{3.722740in}}%
\pgfpathlineto{\pgfqpoint{6.420788in}{3.696481in}}%
\pgfpathlineto{\pgfqpoint{6.420057in}{3.670221in}}%
\pgfpathlineto{\pgfqpoint{6.419582in}{3.643962in}}%
\pgfpathlineto{\pgfqpoint{6.419366in}{3.617702in}}%
\pgfpathlineto{\pgfqpoint{6.419409in}{3.591443in}}%
\pgfpathlineto{\pgfqpoint{6.419711in}{3.565183in}}%
\pgfpathlineto{\pgfqpoint{6.420275in}{3.538924in}}%
\pgfpathlineto{\pgfqpoint{6.421102in}{3.512664in}}%
\pgfpathlineto{\pgfqpoint{6.422192in}{3.486405in}}%
\pgfpathlineto{\pgfqpoint{6.423547in}{3.460145in}}%
\pgfpathlineto{\pgfqpoint{6.425168in}{3.433886in}}%
\pgfpathlineto{\pgfqpoint{6.427056in}{3.407626in}}%
\pgfpathlineto{\pgfqpoint{6.429213in}{3.381367in}}%
\pgfpathlineto{\pgfqpoint{6.431640in}{3.355107in}}%
\pgfpathlineto{\pgfqpoint{6.434339in}{3.328848in}}%
\pgfpathlineto{\pgfqpoint{6.437310in}{3.302589in}}%
\pgfpathlineto{\pgfqpoint{6.440556in}{3.276329in}}%
\pgfpathlineto{\pgfqpoint{6.443827in}{3.251936in}}%
\pgfpathlineto{\pgfqpoint{6.444075in}{3.250070in}}%
\pgfpathlineto{\pgfqpoint{6.447846in}{3.223810in}}%
\pgfpathlineto{\pgfqpoint{6.451894in}{3.197551in}}%
\pgfpathlineto{\pgfqpoint{6.456220in}{3.171291in}}%
\pgfpathlineto{\pgfqpoint{6.460826in}{3.145032in}}%
\pgfpathlineto{\pgfqpoint{6.465714in}{3.118772in}}%
\pgfpathlineto{\pgfqpoint{6.470884in}{3.092513in}}%
\pgfpathlineto{\pgfqpoint{6.473400in}{3.080401in}}%
\pgfpathlineto{\pgfqpoint{6.476319in}{3.066253in}}%
\pgfpathlineto{\pgfqpoint{6.482020in}{3.039994in}}%
\pgfpathlineto{\pgfqpoint{6.488008in}{3.013734in}}%
\pgfpathlineto{\pgfqpoint{6.494283in}{2.987475in}}%
\pgfpathlineto{\pgfqpoint{6.500849in}{2.961215in}}%
\pgfpathlineto{\pgfqpoint{6.502973in}{2.953076in}}%
\pgfpathlineto{\pgfqpoint{6.507673in}{2.934956in}}%
\pgfpathlineto{\pgfqpoint{6.514774in}{2.908696in}}%
\pgfpathlineto{\pgfqpoint{6.522169in}{2.882437in}}%
\pgfpathlineto{\pgfqpoint{6.529859in}{2.856177in}}%
\pgfpathlineto{\pgfqpoint{6.532547in}{2.847337in}}%
\pgfpathlineto{\pgfqpoint{6.537811in}{2.829918in}}%
\pgfpathlineto{\pgfqpoint{6.546041in}{2.803659in}}%
\pgfpathlineto{\pgfqpoint{6.554571in}{2.777399in}}%
\pgfpathlineto{\pgfqpoint{6.562120in}{2.754949in}}%
\pgfpathlineto{\pgfqpoint{6.563394in}{2.751140in}}%
\pgfpathlineto{\pgfqpoint{6.572466in}{2.724880in}}%
\pgfpathlineto{\pgfqpoint{6.581844in}{2.698621in}}%
\pgfpathlineto{\pgfqpoint{6.591528in}{2.672361in}}%
\pgfpathlineto{\pgfqpoint{6.591693in}{2.671925in}}%
\pgfpathlineto{\pgfqpoint{6.601452in}{2.646102in}}%
\pgfpathlineto{\pgfqpoint{6.611686in}{2.619842in}}%
\pgfpathlineto{\pgfqpoint{6.621267in}{2.595979in}}%
\pgfpathlineto{\pgfqpoint{6.622224in}{2.593583in}}%
\pgfpathlineto{\pgfqpoint{6.633008in}{2.567323in}}%
\pgfpathlineto{\pgfqpoint{6.644107in}{2.541064in}}%
\pgfpathlineto{\pgfqpoint{6.650840in}{2.525562in}}%
\pgfpathlineto{\pgfqpoint{6.655491in}{2.514804in}}%
\pgfpathlineto{\pgfqpoint{6.667146in}{2.488545in}}%
\pgfpathlineto{\pgfqpoint{6.679122in}{2.462285in}}%
\pgfpathlineto{\pgfqpoint{6.680414in}{2.459521in}}%
\pgfpathlineto{\pgfqpoint{6.691345in}{2.436026in}}%
\pgfpathlineto{\pgfqpoint{6.703883in}{2.409766in}}%
\pgfpathlineto{\pgfqpoint{6.709987in}{2.397290in}}%
\pgfpathlineto{\pgfqpoint{6.716701in}{2.383507in}}%
\pgfpathlineto{\pgfqpoint{6.729804in}{2.357248in}}%
\pgfpathlineto{\pgfqpoint{6.739560in}{2.338163in}}%
\pgfusepath{stroke}%
\end{pgfscope}%
\begin{pgfscope}%
\pgfpathrectangle{\pgfqpoint{0.854460in}{0.571603in}}{\pgfqpoint{5.885100in}{5.225635in}}%
\pgfusepath{clip}%
\pgfsetbuttcap%
\pgfsetroundjoin%
\pgfsetlinewidth{1.505625pt}%
\definecolor{currentstroke}{rgb}{0.266941,0.748751,0.440573}%
\pgfsetstrokecolor{currentstroke}%
\pgfsetdash{}{0pt}%
\pgfpathmoveto{\pgfqpoint{0.854460in}{5.208975in}}%
\pgfpathlineto{\pgfqpoint{0.865981in}{5.219530in}}%
\pgfpathlineto{\pgfqpoint{0.884034in}{5.235869in}}%
\pgfpathlineto{\pgfqpoint{0.886302in}{5.237905in}}%
\pgfusepath{stroke}%
\end{pgfscope}%
\begin{pgfscope}%
\pgfpathrectangle{\pgfqpoint{0.854460in}{0.571603in}}{\pgfqpoint{5.885100in}{5.225635in}}%
\pgfusepath{clip}%
\pgfsetbuttcap%
\pgfsetroundjoin%
\pgfsetlinewidth{1.505625pt}%
\definecolor{currentstroke}{rgb}{0.266941,0.748751,0.440573}%
\pgfsetstrokecolor{currentstroke}%
\pgfsetdash{}{0pt}%
\pgfpathmoveto{\pgfqpoint{1.182729in}{5.480327in}}%
\pgfpathlineto{\pgfqpoint{1.185132in}{5.482125in}}%
\pgfpathlineto{\pgfqpoint{1.209341in}{5.500010in}}%
\pgfpathlineto{\pgfqpoint{1.220749in}{5.508384in}}%
\pgfpathlineto{\pgfqpoint{1.238914in}{5.521558in}}%
\pgfpathlineto{\pgfqpoint{1.257073in}{5.534644in}}%
\pgfpathlineto{\pgfqpoint{1.268488in}{5.542771in}}%
\pgfpathlineto{\pgfqpoint{1.294112in}{5.560903in}}%
\pgfpathlineto{\pgfqpoint{1.298061in}{5.563664in}}%
\pgfpathlineto{\pgfqpoint{1.327634in}{5.584196in}}%
\pgfpathlineto{\pgfqpoint{1.331937in}{5.587163in}}%
\pgfpathlineto{\pgfqpoint{1.357208in}{5.604378in}}%
\pgfpathlineto{\pgfqpoint{1.370560in}{5.613422in}}%
\pgfpathlineto{\pgfqpoint{1.386781in}{5.624276in}}%
\pgfpathlineto{\pgfqpoint{1.409933in}{5.639682in}}%
\pgfpathlineto{\pgfqpoint{1.416354in}{5.643903in}}%
\pgfpathlineto{\pgfqpoint{1.445928in}{5.663221in}}%
\pgfpathlineto{\pgfqpoint{1.450120in}{5.665941in}}%
\pgfpathlineto{\pgfqpoint{1.475501in}{5.682209in}}%
\pgfpathlineto{\pgfqpoint{1.491171in}{5.692201in}}%
\pgfpathlineto{\pgfqpoint{1.505074in}{5.700958in}}%
\pgfpathlineto{\pgfqpoint{1.533000in}{5.718460in}}%
\pgfpathlineto{\pgfqpoint{1.534648in}{5.719481in}}%
\pgfpathlineto{\pgfqpoint{1.564221in}{5.737656in}}%
\pgfpathlineto{\pgfqpoint{1.575774in}{5.744720in}}%
\pgfpathlineto{\pgfqpoint{1.593795in}{5.755604in}}%
\pgfpathlineto{\pgfqpoint{1.619369in}{5.770979in}}%
\pgfpathlineto{\pgfqpoint{1.623368in}{5.773354in}}%
\pgfpathlineto{\pgfqpoint{1.652941in}{5.790797in}}%
\pgfpathlineto{\pgfqpoint{1.663920in}{5.797238in}}%
\pgfusepath{stroke}%
\end{pgfscope}%
\begin{pgfscope}%
\pgfpathrectangle{\pgfqpoint{0.854460in}{0.571603in}}{\pgfqpoint{5.885100in}{5.225635in}}%
\pgfusepath{clip}%
\pgfsetbuttcap%
\pgfsetroundjoin%
\pgfsetlinewidth{1.505625pt}%
\definecolor{currentstroke}{rgb}{0.311925,0.767822,0.415586}%
\pgfsetstrokecolor{currentstroke}%
\pgfsetdash{}{0pt}%
\pgfpathmoveto{\pgfqpoint{6.739560in}{4.809727in}}%
\pgfpathlineto{\pgfqpoint{6.735466in}{4.799378in}}%
\pgfpathlineto{\pgfqpoint{6.725239in}{4.773119in}}%
\pgfpathlineto{\pgfqpoint{6.715214in}{4.746860in}}%
\pgfpathlineto{\pgfqpoint{6.709987in}{4.732922in}}%
\pgfpathlineto{\pgfqpoint{6.705357in}{4.720600in}}%
\pgfpathlineto{\pgfqpoint{6.695666in}{4.694341in}}%
\pgfpathlineto{\pgfqpoint{6.686184in}{4.668081in}}%
\pgfpathlineto{\pgfqpoint{6.680414in}{4.651773in}}%
\pgfpathlineto{\pgfqpoint{6.676884in}{4.641822in}}%
\pgfpathlineto{\pgfqpoint{6.667753in}{4.615562in}}%
\pgfpathlineto{\pgfqpoint{6.658837in}{4.589303in}}%
\pgfpathlineto{\pgfqpoint{6.650840in}{4.565178in}}%
\pgfpathlineto{\pgfqpoint{6.650131in}{4.563043in}}%
\pgfpathlineto{\pgfqpoint{6.641581in}{4.536784in}}%
\pgfpathlineto{\pgfqpoint{6.633252in}{4.510524in}}%
\pgfpathlineto{\pgfqpoint{6.625144in}{4.484265in}}%
\pgfpathlineto{\pgfqpoint{6.621267in}{4.471385in}}%
\pgfpathlineto{\pgfqpoint{6.617227in}{4.458005in}}%
\pgfpathlineto{\pgfqpoint{6.609506in}{4.431746in}}%
\pgfpathlineto{\pgfqpoint{6.602010in}{4.405486in}}%
\pgfpathlineto{\pgfqpoint{6.594741in}{4.379227in}}%
\pgfpathlineto{\pgfqpoint{6.591693in}{4.367887in}}%
\pgfpathlineto{\pgfqpoint{6.587670in}{4.352967in}}%
\pgfpathlineto{\pgfqpoint{6.580807in}{4.326708in}}%
\pgfpathlineto{\pgfqpoint{6.574175in}{4.300449in}}%
\pgfpathlineto{\pgfqpoint{6.567775in}{4.274189in}}%
\pgfpathlineto{\pgfqpoint{6.563578in}{4.256322in}}%
\pgfusepath{stroke}%
\end{pgfscope}%
\begin{pgfscope}%
\pgfpathrectangle{\pgfqpoint{0.854460in}{0.571603in}}{\pgfqpoint{5.885100in}{5.225635in}}%
\pgfusepath{clip}%
\pgfsetbuttcap%
\pgfsetroundjoin%
\pgfsetlinewidth{1.505625pt}%
\definecolor{currentstroke}{rgb}{0.311925,0.767822,0.415586}%
\pgfsetstrokecolor{currentstroke}%
\pgfsetdash{}{0pt}%
\pgfpathmoveto{\pgfqpoint{6.498470in}{3.869602in}}%
\pgfpathlineto{\pgfqpoint{6.496917in}{3.854037in}}%
\pgfpathlineto{\pgfqpoint{6.494550in}{3.827778in}}%
\pgfpathlineto{\pgfqpoint{6.492436in}{3.801519in}}%
\pgfpathlineto{\pgfqpoint{6.490576in}{3.775259in}}%
\pgfpathlineto{\pgfqpoint{6.488972in}{3.749000in}}%
\pgfpathlineto{\pgfqpoint{6.487624in}{3.722740in}}%
\pgfpathlineto{\pgfqpoint{6.486533in}{3.696481in}}%
\pgfpathlineto{\pgfqpoint{6.485701in}{3.670221in}}%
\pgfpathlineto{\pgfqpoint{6.485128in}{3.643962in}}%
\pgfpathlineto{\pgfqpoint{6.484816in}{3.617702in}}%
\pgfpathlineto{\pgfqpoint{6.484764in}{3.591443in}}%
\pgfpathlineto{\pgfqpoint{6.484976in}{3.565183in}}%
\pgfpathlineto{\pgfqpoint{6.485450in}{3.538924in}}%
\pgfpathlineto{\pgfqpoint{6.486190in}{3.512664in}}%
\pgfpathlineto{\pgfqpoint{6.487195in}{3.486405in}}%
\pgfpathlineto{\pgfqpoint{6.488468in}{3.460145in}}%
\pgfpathlineto{\pgfqpoint{6.490008in}{3.433886in}}%
\pgfpathlineto{\pgfqpoint{6.491818in}{3.407626in}}%
\pgfpathlineto{\pgfqpoint{6.493899in}{3.381367in}}%
\pgfpathlineto{\pgfqpoint{6.496252in}{3.355107in}}%
\pgfpathlineto{\pgfqpoint{6.498878in}{3.328848in}}%
\pgfpathlineto{\pgfqpoint{6.501779in}{3.302589in}}%
\pgfpathlineto{\pgfqpoint{6.502973in}{3.292716in}}%
\pgfpathlineto{\pgfqpoint{6.504942in}{3.276329in}}%
\pgfpathlineto{\pgfqpoint{6.508372in}{3.250070in}}%
\pgfpathlineto{\pgfqpoint{6.512079in}{3.223810in}}%
\pgfpathlineto{\pgfqpoint{6.516064in}{3.197551in}}%
\pgfpathlineto{\pgfqpoint{6.520329in}{3.171291in}}%
\pgfpathlineto{\pgfqpoint{6.524876in}{3.145032in}}%
\pgfpathlineto{\pgfqpoint{6.529706in}{3.118772in}}%
\pgfpathlineto{\pgfqpoint{6.532547in}{3.104187in}}%
\pgfpathlineto{\pgfqpoint{6.534805in}{3.092513in}}%
\pgfpathlineto{\pgfqpoint{6.540168in}{3.066253in}}%
\pgfpathlineto{\pgfqpoint{6.545818in}{3.039994in}}%
\pgfpathlineto{\pgfqpoint{6.551755in}{3.013734in}}%
\pgfpathlineto{\pgfqpoint{6.557982in}{2.987475in}}%
\pgfpathlineto{\pgfqpoint{6.562120in}{2.970800in}}%
\pgfpathlineto{\pgfqpoint{6.564484in}{2.961215in}}%
\pgfpathlineto{\pgfqpoint{6.571248in}{2.934956in}}%
\pgfpathlineto{\pgfqpoint{6.578306in}{2.908696in}}%
\pgfpathlineto{\pgfqpoint{6.585658in}{2.882437in}}%
\pgfpathlineto{\pgfqpoint{6.591693in}{2.861714in}}%
\pgfpathlineto{\pgfqpoint{6.593296in}{2.856177in}}%
\pgfpathlineto{\pgfqpoint{6.601190in}{2.829918in}}%
\pgfpathlineto{\pgfqpoint{6.609383in}{2.803659in}}%
\pgfpathlineto{\pgfqpoint{6.617877in}{2.777399in}}%
\pgfpathlineto{\pgfqpoint{6.621267in}{2.767272in}}%
\pgfpathlineto{\pgfqpoint{6.626637in}{2.751140in}}%
\pgfpathlineto{\pgfqpoint{6.635677in}{2.724880in}}%
\pgfpathlineto{\pgfqpoint{6.645023in}{2.698621in}}%
\pgfpathlineto{\pgfqpoint{6.650840in}{2.682787in}}%
\pgfpathlineto{\pgfqpoint{6.654650in}{2.672361in}}%
\pgfpathlineto{\pgfqpoint{6.664546in}{2.646102in}}%
\pgfpathlineto{\pgfqpoint{6.674753in}{2.619842in}}%
\pgfpathlineto{\pgfqpoint{6.680414in}{2.605698in}}%
\pgfpathlineto{\pgfqpoint{6.685239in}{2.593583in}}%
\pgfpathlineto{\pgfqpoint{6.695999in}{2.567323in}}%
\pgfpathlineto{\pgfqpoint{6.707076in}{2.541064in}}%
\pgfpathlineto{\pgfqpoint{6.709987in}{2.534344in}}%
\pgfpathlineto{\pgfqpoint{6.718413in}{2.514804in}}%
\pgfpathlineto{\pgfqpoint{6.730049in}{2.488545in}}%
\pgfpathlineto{\pgfqpoint{6.739560in}{2.467646in}}%
\pgfusepath{stroke}%
\end{pgfscope}%
\begin{pgfscope}%
\pgfpathrectangle{\pgfqpoint{0.854460in}{0.571603in}}{\pgfqpoint{5.885100in}{5.225635in}}%
\pgfusepath{clip}%
\pgfsetbuttcap%
\pgfsetroundjoin%
\pgfsetlinewidth{1.505625pt}%
\definecolor{currentstroke}{rgb}{0.311925,0.767822,0.415586}%
\pgfsetstrokecolor{currentstroke}%
\pgfsetdash{}{0pt}%
\pgfpathmoveto{\pgfqpoint{0.854460in}{5.262885in}}%
\pgfpathlineto{\pgfqpoint{0.864712in}{5.272049in}}%
\pgfpathlineto{\pgfqpoint{0.884034in}{5.289110in}}%
\pgfpathlineto{\pgfqpoint{0.884966in}{5.289926in}}%
\pgfusepath{stroke}%
\end{pgfscope}%
\begin{pgfscope}%
\pgfpathrectangle{\pgfqpoint{0.854460in}{0.571603in}}{\pgfqpoint{5.885100in}{5.225635in}}%
\pgfusepath{clip}%
\pgfsetbuttcap%
\pgfsetroundjoin%
\pgfsetlinewidth{1.505625pt}%
\definecolor{currentstroke}{rgb}{0.311925,0.767822,0.415586}%
\pgfsetstrokecolor{currentstroke}%
\pgfsetdash{}{0pt}%
\pgfpathmoveto{\pgfqpoint{1.183880in}{5.529013in}}%
\pgfpathlineto{\pgfqpoint{1.191573in}{5.534644in}}%
\pgfpathlineto{\pgfqpoint{1.209341in}{5.547490in}}%
\pgfpathlineto{\pgfqpoint{1.228006in}{5.560903in}}%
\pgfpathlineto{\pgfqpoint{1.238914in}{5.568647in}}%
\pgfpathlineto{\pgfqpoint{1.265151in}{5.587163in}}%
\pgfpathlineto{\pgfqpoint{1.268488in}{5.589489in}}%
\pgfpathlineto{\pgfqpoint{1.298061in}{5.609964in}}%
\pgfpathlineto{\pgfqpoint{1.303088in}{5.613422in}}%
\pgfpathlineto{\pgfqpoint{1.327634in}{5.630102in}}%
\pgfpathlineto{\pgfqpoint{1.341810in}{5.639682in}}%
\pgfpathlineto{\pgfqpoint{1.357208in}{5.649960in}}%
\pgfpathlineto{\pgfqpoint{1.381277in}{5.665941in}}%
\pgfpathlineto{\pgfqpoint{1.386781in}{5.669551in}}%
\pgfpathlineto{\pgfqpoint{1.416354in}{5.688825in}}%
\pgfpathlineto{\pgfqpoint{1.421568in}{5.692201in}}%
\pgfpathlineto{\pgfqpoint{1.445928in}{5.707783in}}%
\pgfpathlineto{\pgfqpoint{1.462705in}{5.718460in}}%
\pgfpathlineto{\pgfqpoint{1.475501in}{5.726505in}}%
\pgfpathlineto{\pgfqpoint{1.504613in}{5.744720in}}%
\pgfpathlineto{\pgfqpoint{1.505074in}{5.745004in}}%
\pgfpathlineto{\pgfqpoint{1.534648in}{5.763146in}}%
\pgfpathlineto{\pgfqpoint{1.547478in}{5.770979in}}%
\pgfpathlineto{\pgfqpoint{1.564221in}{5.781076in}}%
\pgfpathlineto{\pgfqpoint{1.591141in}{5.797238in}}%
\pgfusepath{stroke}%
\end{pgfscope}%
\begin{pgfscope}%
\pgfpathrectangle{\pgfqpoint{0.854460in}{0.571603in}}{\pgfqpoint{5.885100in}{5.225635in}}%
\pgfusepath{clip}%
\pgfsetbuttcap%
\pgfsetroundjoin%
\pgfsetlinewidth{1.505625pt}%
\definecolor{currentstroke}{rgb}{0.352360,0.783011,0.392636}%
\pgfsetstrokecolor{currentstroke}%
\pgfsetdash{}{0pt}%
\pgfpathmoveto{\pgfqpoint{6.739560in}{4.621564in}}%
\pgfpathlineto{\pgfqpoint{6.737430in}{4.615562in}}%
\pgfpathlineto{\pgfqpoint{6.728292in}{4.589303in}}%
\pgfpathlineto{\pgfqpoint{6.719376in}{4.563043in}}%
\pgfpathlineto{\pgfqpoint{6.710681in}{4.536784in}}%
\pgfpathlineto{\pgfqpoint{6.709987in}{4.534642in}}%
\pgfpathlineto{\pgfqpoint{6.702151in}{4.510524in}}%
\pgfpathlineto{\pgfqpoint{6.693842in}{4.484265in}}%
\pgfpathlineto{\pgfqpoint{6.685759in}{4.458005in}}%
\pgfpathlineto{\pgfqpoint{6.680414in}{4.440159in}}%
\pgfpathlineto{\pgfqpoint{6.677886in}{4.431746in}}%
\pgfpathlineto{\pgfqpoint{6.670203in}{4.405486in}}%
\pgfpathlineto{\pgfqpoint{6.662751in}{4.379227in}}%
\pgfpathlineto{\pgfqpoint{6.655532in}{4.352967in}}%
\pgfpathlineto{\pgfqpoint{6.650840in}{4.335356in}}%
\pgfpathlineto{\pgfqpoint{6.648529in}{4.326708in}}%
\pgfpathlineto{\pgfqpoint{6.641726in}{4.300449in}}%
\pgfpathlineto{\pgfqpoint{6.635161in}{4.274189in}}%
\pgfpathlineto{\pgfqpoint{6.628832in}{4.247930in}}%
\pgfpathlineto{\pgfqpoint{6.622741in}{4.221670in}}%
\pgfpathlineto{\pgfqpoint{6.621267in}{4.215070in}}%
\pgfpathlineto{\pgfqpoint{6.616858in}{4.195411in}}%
\pgfpathlineto{\pgfqpoint{6.611204in}{4.169151in}}%
\pgfpathlineto{\pgfqpoint{6.605794in}{4.142892in}}%
\pgfpathlineto{\pgfqpoint{6.600626in}{4.116632in}}%
\pgfpathlineto{\pgfqpoint{6.595701in}{4.090373in}}%
\pgfpathlineto{\pgfqpoint{6.591693in}{4.067894in}}%
\pgfpathlineto{\pgfqpoint{6.591017in}{4.064113in}}%
\pgfpathlineto{\pgfqpoint{6.586550in}{4.037854in}}%
\pgfpathlineto{\pgfqpoint{6.582331in}{4.011594in}}%
\pgfpathlineto{\pgfqpoint{6.578361in}{3.985335in}}%
\pgfpathlineto{\pgfqpoint{6.574640in}{3.959075in}}%
\pgfpathlineto{\pgfqpoint{6.571169in}{3.932816in}}%
\pgfpathlineto{\pgfqpoint{6.567950in}{3.906556in}}%
\pgfpathlineto{\pgfqpoint{6.564982in}{3.880297in}}%
\pgfpathlineto{\pgfqpoint{6.562267in}{3.854037in}}%
\pgfpathlineto{\pgfqpoint{6.562120in}{3.852475in}}%
\pgfpathlineto{\pgfqpoint{6.559789in}{3.827778in}}%
\pgfpathlineto{\pgfqpoint{6.557566in}{3.801519in}}%
\pgfpathlineto{\pgfqpoint{6.555599in}{3.775259in}}%
\pgfpathlineto{\pgfqpoint{6.553891in}{3.749000in}}%
\pgfpathlineto{\pgfqpoint{6.552442in}{3.722740in}}%
\pgfpathlineto{\pgfqpoint{6.551253in}{3.696481in}}%
\pgfpathlineto{\pgfqpoint{6.550324in}{3.670221in}}%
\pgfpathlineto{\pgfqpoint{6.549657in}{3.643962in}}%
\pgfpathlineto{\pgfqpoint{6.549252in}{3.617702in}}%
\pgfpathlineto{\pgfqpoint{6.549111in}{3.591443in}}%
\pgfpathlineto{\pgfqpoint{6.549235in}{3.565183in}}%
\pgfpathlineto{\pgfqpoint{6.549624in}{3.538924in}}%
\pgfpathlineto{\pgfqpoint{6.550281in}{3.512664in}}%
\pgfpathlineto{\pgfqpoint{6.551205in}{3.486405in}}%
\pgfpathlineto{\pgfqpoint{6.552398in}{3.460145in}}%
\pgfpathlineto{\pgfqpoint{6.553212in}{3.445537in}}%
\pgfusepath{stroke}%
\end{pgfscope}%
\begin{pgfscope}%
\pgfpathrectangle{\pgfqpoint{0.854460in}{0.571603in}}{\pgfqpoint{5.885100in}{5.225635in}}%
\pgfusepath{clip}%
\pgfsetbuttcap%
\pgfsetroundjoin%
\pgfsetlinewidth{1.505625pt}%
\definecolor{currentstroke}{rgb}{0.352360,0.783011,0.392636}%
\pgfsetstrokecolor{currentstroke}%
\pgfsetdash{}{0pt}%
\pgfpathmoveto{\pgfqpoint{6.605099in}{3.056763in}}%
\pgfpathlineto{\pgfqpoint{6.608675in}{3.039994in}}%
\pgfpathlineto{\pgfqpoint{6.614563in}{3.013734in}}%
\pgfpathlineto{\pgfqpoint{6.620742in}{2.987475in}}%
\pgfpathlineto{\pgfqpoint{6.621267in}{2.985342in}}%
\pgfpathlineto{\pgfqpoint{6.627173in}{2.961215in}}%
\pgfpathlineto{\pgfqpoint{6.633893in}{2.934956in}}%
\pgfpathlineto{\pgfqpoint{6.640907in}{2.908696in}}%
\pgfpathlineto{\pgfqpoint{6.648218in}{2.882437in}}%
\pgfpathlineto{\pgfqpoint{6.650840in}{2.873381in}}%
\pgfpathlineto{\pgfqpoint{6.655793in}{2.856177in}}%
\pgfpathlineto{\pgfqpoint{6.663649in}{2.829918in}}%
\pgfpathlineto{\pgfqpoint{6.671804in}{2.803659in}}%
\pgfpathlineto{\pgfqpoint{6.680262in}{2.777399in}}%
\pgfpathlineto{\pgfqpoint{6.680414in}{2.776945in}}%
\pgfpathlineto{\pgfqpoint{6.688966in}{2.751140in}}%
\pgfpathlineto{\pgfqpoint{6.697974in}{2.724880in}}%
\pgfpathlineto{\pgfqpoint{6.707288in}{2.698621in}}%
\pgfpathlineto{\pgfqpoint{6.709987in}{2.691247in}}%
\pgfpathlineto{\pgfqpoint{6.716864in}{2.672361in}}%
\pgfpathlineto{\pgfqpoint{6.726732in}{2.646102in}}%
\pgfpathlineto{\pgfqpoint{6.736911in}{2.619842in}}%
\pgfpathlineto{\pgfqpoint{6.739560in}{2.613202in}}%
\pgfusepath{stroke}%
\end{pgfscope}%
\begin{pgfscope}%
\pgfpathrectangle{\pgfqpoint{0.854460in}{0.571603in}}{\pgfqpoint{5.885100in}{5.225635in}}%
\pgfusepath{clip}%
\pgfsetbuttcap%
\pgfsetroundjoin%
\pgfsetlinewidth{1.505625pt}%
\definecolor{currentstroke}{rgb}{0.352360,0.783011,0.392636}%
\pgfsetstrokecolor{currentstroke}%
\pgfsetdash{}{0pt}%
\pgfpathmoveto{\pgfqpoint{0.854460in}{5.314795in}}%
\pgfpathlineto{\pgfqpoint{0.865663in}{5.324568in}}%
\pgfpathlineto{\pgfqpoint{0.883727in}{5.340135in}}%
\pgfusepath{stroke}%
\end{pgfscope}%
\begin{pgfscope}%
\pgfpathrectangle{\pgfqpoint{0.854460in}{0.571603in}}{\pgfqpoint{5.885100in}{5.225635in}}%
\pgfusepath{clip}%
\pgfsetbuttcap%
\pgfsetroundjoin%
\pgfsetlinewidth{1.505625pt}%
\definecolor{currentstroke}{rgb}{0.352360,0.783011,0.392636}%
\pgfsetstrokecolor{currentstroke}%
\pgfsetdash{}{0pt}%
\pgfpathmoveto{\pgfqpoint{1.184956in}{5.576052in}}%
\pgfpathlineto{\pgfqpoint{1.200463in}{5.587163in}}%
\pgfpathlineto{\pgfqpoint{1.209341in}{5.593447in}}%
\pgfpathlineto{\pgfqpoint{1.237727in}{5.613422in}}%
\pgfpathlineto{\pgfqpoint{1.238914in}{5.614247in}}%
\pgfpathlineto{\pgfqpoint{1.268488in}{5.634658in}}%
\pgfpathlineto{\pgfqpoint{1.275810in}{5.639682in}}%
\pgfpathlineto{\pgfqpoint{1.298061in}{5.654763in}}%
\pgfpathlineto{\pgfqpoint{1.314644in}{5.665941in}}%
\pgfpathlineto{\pgfqpoint{1.327634in}{5.674592in}}%
\pgfpathlineto{\pgfqpoint{1.354219in}{5.692201in}}%
\pgfpathlineto{\pgfqpoint{1.357208in}{5.694156in}}%
\pgfpathlineto{\pgfqpoint{1.386781in}{5.713376in}}%
\pgfpathlineto{\pgfqpoint{1.394649in}{5.718460in}}%
\pgfpathlineto{\pgfqpoint{1.416354in}{5.732315in}}%
\pgfpathlineto{\pgfqpoint{1.435884in}{5.744720in}}%
\pgfpathlineto{\pgfqpoint{1.445928in}{5.751022in}}%
\pgfpathlineto{\pgfqpoint{1.475501in}{5.769480in}}%
\pgfpathlineto{\pgfqpoint{1.477920in}{5.770979in}}%
\pgfpathlineto{\pgfqpoint{1.505074in}{5.787606in}}%
\pgfpathlineto{\pgfqpoint{1.520878in}{5.797238in}}%
\pgfusepath{stroke}%
\end{pgfscope}%
\begin{pgfscope}%
\pgfpathrectangle{\pgfqpoint{0.854460in}{0.571603in}}{\pgfqpoint{5.885100in}{5.225635in}}%
\pgfusepath{clip}%
\pgfsetbuttcap%
\pgfsetroundjoin%
\pgfsetlinewidth{1.505625pt}%
\definecolor{currentstroke}{rgb}{0.404001,0.800275,0.362552}%
\pgfsetstrokecolor{currentstroke}%
\pgfsetdash{}{0pt}%
\pgfpathmoveto{\pgfqpoint{6.739560in}{4.413175in}}%
\pgfpathlineto{\pgfqpoint{6.737266in}{4.405486in}}%
\pgfpathlineto{\pgfqpoint{6.729642in}{4.379227in}}%
\pgfpathlineto{\pgfqpoint{6.722254in}{4.352967in}}%
\pgfpathlineto{\pgfqpoint{6.715102in}{4.326708in}}%
\pgfpathlineto{\pgfqpoint{6.709987in}{4.307297in}}%
\pgfpathlineto{\pgfqpoint{6.708176in}{4.300449in}}%
\pgfpathlineto{\pgfqpoint{6.701453in}{4.274189in}}%
\pgfpathlineto{\pgfqpoint{6.694970in}{4.247930in}}%
\pgfpathlineto{\pgfqpoint{6.688730in}{4.221670in}}%
\pgfpathlineto{\pgfqpoint{6.682731in}{4.195411in}}%
\pgfpathlineto{\pgfqpoint{6.680414in}{4.184858in}}%
\pgfpathlineto{\pgfqpoint{6.676951in}{4.169151in}}%
\pgfpathlineto{\pgfqpoint{6.671400in}{4.142892in}}%
\pgfpathlineto{\pgfqpoint{6.666095in}{4.116632in}}%
\pgfpathlineto{\pgfqpoint{6.661037in}{4.090373in}}%
\pgfpathlineto{\pgfqpoint{6.656226in}{4.064113in}}%
\pgfpathlineto{\pgfqpoint{6.651664in}{4.037854in}}%
\pgfpathlineto{\pgfqpoint{6.650840in}{4.032849in}}%
\pgfpathlineto{\pgfqpoint{6.647326in}{4.011594in}}%
\pgfpathlineto{\pgfqpoint{6.643234in}{3.985335in}}%
\pgfpathlineto{\pgfqpoint{6.639394in}{3.959075in}}%
\pgfpathlineto{\pgfqpoint{6.635807in}{3.932816in}}%
\pgfpathlineto{\pgfqpoint{6.632474in}{3.906556in}}%
\pgfpathlineto{\pgfqpoint{6.629395in}{3.880297in}}%
\pgfpathlineto{\pgfqpoint{6.626571in}{3.854037in}}%
\pgfpathlineto{\pgfqpoint{6.624003in}{3.827778in}}%
\pgfpathlineto{\pgfqpoint{6.621693in}{3.801519in}}%
\pgfpathlineto{\pgfqpoint{6.621267in}{3.796081in}}%
\pgfpathlineto{\pgfqpoint{6.620606in}{3.787678in}}%
\pgfusepath{stroke}%
\end{pgfscope}%
\begin{pgfscope}%
\pgfpathrectangle{\pgfqpoint{0.854460in}{0.571603in}}{\pgfqpoint{5.885100in}{5.225635in}}%
\pgfusepath{clip}%
\pgfsetbuttcap%
\pgfsetroundjoin%
\pgfsetlinewidth{1.505625pt}%
\definecolor{currentstroke}{rgb}{0.404001,0.800275,0.362552}%
\pgfsetstrokecolor{currentstroke}%
\pgfsetdash{}{0pt}%
\pgfpathmoveto{\pgfqpoint{6.619351in}{3.395158in}}%
\pgfpathlineto{\pgfqpoint{6.620368in}{3.381367in}}%
\pgfpathlineto{\pgfqpoint{6.621267in}{3.370701in}}%
\pgfpathlineto{\pgfqpoint{6.622571in}{3.355107in}}%
\pgfpathlineto{\pgfqpoint{6.625043in}{3.328848in}}%
\pgfpathlineto{\pgfqpoint{6.627791in}{3.302589in}}%
\pgfpathlineto{\pgfqpoint{6.630817in}{3.276329in}}%
\pgfpathlineto{\pgfqpoint{6.634121in}{3.250070in}}%
\pgfpathlineto{\pgfqpoint{6.637706in}{3.223810in}}%
\pgfpathlineto{\pgfqpoint{6.641572in}{3.197551in}}%
\pgfpathlineto{\pgfqpoint{6.645721in}{3.171291in}}%
\pgfpathlineto{\pgfqpoint{6.650155in}{3.145032in}}%
\pgfpathlineto{\pgfqpoint{6.650840in}{3.141217in}}%
\pgfpathlineto{\pgfqpoint{6.654847in}{3.118772in}}%
\pgfpathlineto{\pgfqpoint{6.659820in}{3.092513in}}%
\pgfpathlineto{\pgfqpoint{6.665081in}{3.066253in}}%
\pgfpathlineto{\pgfqpoint{6.670630in}{3.039994in}}%
\pgfpathlineto{\pgfqpoint{6.676470in}{3.013734in}}%
\pgfpathlineto{\pgfqpoint{6.680414in}{2.996844in}}%
\pgfpathlineto{\pgfqpoint{6.682587in}{2.987475in}}%
\pgfpathlineto{\pgfqpoint{6.688970in}{2.961215in}}%
\pgfpathlineto{\pgfqpoint{6.695647in}{2.934956in}}%
\pgfpathlineto{\pgfqpoint{6.702619in}{2.908696in}}%
\pgfpathlineto{\pgfqpoint{6.709888in}{2.882437in}}%
\pgfpathlineto{\pgfqpoint{6.709987in}{2.882095in}}%
\pgfpathlineto{\pgfqpoint{6.717407in}{2.856177in}}%
\pgfpathlineto{\pgfqpoint{6.725225in}{2.829918in}}%
\pgfpathlineto{\pgfqpoint{6.733343in}{2.803659in}}%
\pgfpathlineto{\pgfqpoint{6.739560in}{2.784268in}}%
\pgfusepath{stroke}%
\end{pgfscope}%
\begin{pgfscope}%
\pgfpathrectangle{\pgfqpoint{0.854460in}{0.571603in}}{\pgfqpoint{5.885100in}{5.225635in}}%
\pgfusepath{clip}%
\pgfsetbuttcap%
\pgfsetroundjoin%
\pgfsetlinewidth{1.505625pt}%
\definecolor{currentstroke}{rgb}{0.404001,0.800275,0.362552}%
\pgfsetstrokecolor{currentstroke}%
\pgfsetdash{}{0pt}%
\pgfpathmoveto{\pgfqpoint{0.854460in}{5.364861in}}%
\pgfpathlineto{\pgfqpoint{0.868815in}{5.377087in}}%
\pgfpathlineto{\pgfqpoint{0.882580in}{5.388669in}}%
\pgfusepath{stroke}%
\end{pgfscope}%
\begin{pgfscope}%
\pgfpathrectangle{\pgfqpoint{0.854460in}{0.571603in}}{\pgfqpoint{5.885100in}{5.225635in}}%
\pgfusepath{clip}%
\pgfsetbuttcap%
\pgfsetroundjoin%
\pgfsetlinewidth{1.505625pt}%
\definecolor{currentstroke}{rgb}{0.404001,0.800275,0.362552}%
\pgfsetstrokecolor{currentstroke}%
\pgfsetdash{}{0pt}%
\pgfpathmoveto{\pgfqpoint{1.185961in}{5.621578in}}%
\pgfpathlineto{\pgfqpoint{1.209341in}{5.637973in}}%
\pgfpathlineto{\pgfqpoint{1.211794in}{5.639682in}}%
\pgfpathlineto{\pgfqpoint{1.238914in}{5.658342in}}%
\pgfpathlineto{\pgfqpoint{1.250021in}{5.665941in}}%
\pgfpathlineto{\pgfqpoint{1.268488in}{5.678424in}}%
\pgfpathlineto{\pgfqpoint{1.288978in}{5.692201in}}%
\pgfpathlineto{\pgfqpoint{1.298061in}{5.698233in}}%
\pgfpathlineto{\pgfqpoint{1.327634in}{5.717769in}}%
\pgfpathlineto{\pgfqpoint{1.328687in}{5.718460in}}%
\pgfpathlineto{\pgfqpoint{1.357208in}{5.736940in}}%
\pgfpathlineto{\pgfqpoint{1.369275in}{5.744720in}}%
\pgfpathlineto{\pgfqpoint{1.386781in}{5.755869in}}%
\pgfpathlineto{\pgfqpoint{1.410620in}{5.770979in}}%
\pgfpathlineto{\pgfqpoint{1.416354in}{5.774569in}}%
\pgfpathlineto{\pgfqpoint{1.445928in}{5.792973in}}%
\pgfpathlineto{\pgfqpoint{1.452822in}{5.797238in}}%
\pgfusepath{stroke}%
\end{pgfscope}%
\begin{pgfscope}%
\pgfpathrectangle{\pgfqpoint{0.854460in}{0.571603in}}{\pgfqpoint{5.885100in}{5.225635in}}%
\pgfusepath{clip}%
\pgfsetbuttcap%
\pgfsetroundjoin%
\pgfsetlinewidth{1.505625pt}%
\definecolor{currentstroke}{rgb}{0.449368,0.813768,0.335384}%
\pgfsetstrokecolor{currentstroke}%
\pgfsetdash{}{0pt}%
\pgfpathmoveto{\pgfqpoint{6.739560in}{4.159402in}}%
\pgfpathlineto{\pgfqpoint{6.735991in}{4.142892in}}%
\pgfpathlineto{\pgfqpoint{6.730556in}{4.116632in}}%
\pgfpathlineto{\pgfqpoint{6.725371in}{4.090373in}}%
\pgfpathlineto{\pgfqpoint{6.722350in}{4.074299in}}%
\pgfusepath{stroke}%
\end{pgfscope}%
\begin{pgfscope}%
\pgfpathrectangle{\pgfqpoint{0.854460in}{0.571603in}}{\pgfqpoint{5.885100in}{5.225635in}}%
\pgfusepath{clip}%
\pgfsetbuttcap%
\pgfsetroundjoin%
\pgfsetlinewidth{1.505625pt}%
\definecolor{currentstroke}{rgb}{0.449368,0.813768,0.335384}%
\pgfsetstrokecolor{currentstroke}%
\pgfsetdash{}{0pt}%
\pgfpathmoveto{\pgfqpoint{6.677286in}{3.684548in}}%
\pgfpathlineto{\pgfqpoint{6.676681in}{3.670221in}}%
\pgfpathlineto{\pgfqpoint{6.675838in}{3.643962in}}%
\pgfpathlineto{\pgfqpoint{6.675261in}{3.617702in}}%
\pgfpathlineto{\pgfqpoint{6.674952in}{3.591443in}}%
\pgfpathlineto{\pgfqpoint{6.674911in}{3.565183in}}%
\pgfpathlineto{\pgfqpoint{6.675140in}{3.538924in}}%
\pgfpathlineto{\pgfqpoint{6.675640in}{3.512664in}}%
\pgfpathlineto{\pgfqpoint{6.676411in}{3.486405in}}%
\pgfpathlineto{\pgfqpoint{6.677455in}{3.460145in}}%
\pgfpathlineto{\pgfqpoint{6.678773in}{3.433886in}}%
\pgfpathlineto{\pgfqpoint{6.680365in}{3.407626in}}%
\pgfpathlineto{\pgfqpoint{6.680414in}{3.406948in}}%
\pgfpathlineto{\pgfqpoint{6.682221in}{3.381367in}}%
\pgfpathlineto{\pgfqpoint{6.684353in}{3.355107in}}%
\pgfpathlineto{\pgfqpoint{6.686760in}{3.328848in}}%
\pgfpathlineto{\pgfqpoint{6.689446in}{3.302589in}}%
\pgfpathlineto{\pgfqpoint{6.692410in}{3.276329in}}%
\pgfpathlineto{\pgfqpoint{6.695655in}{3.250070in}}%
\pgfpathlineto{\pgfqpoint{6.699181in}{3.223810in}}%
\pgfpathlineto{\pgfqpoint{6.702990in}{3.197551in}}%
\pgfpathlineto{\pgfqpoint{6.707083in}{3.171291in}}%
\pgfpathlineto{\pgfqpoint{6.709987in}{3.153881in}}%
\pgfpathlineto{\pgfqpoint{6.711453in}{3.145032in}}%
\pgfpathlineto{\pgfqpoint{6.716089in}{3.118772in}}%
\pgfpathlineto{\pgfqpoint{6.721011in}{3.092513in}}%
\pgfpathlineto{\pgfqpoint{6.726222in}{3.066253in}}%
\pgfpathlineto{\pgfqpoint{6.731723in}{3.039994in}}%
\pgfpathlineto{\pgfqpoint{6.737516in}{3.013734in}}%
\pgfpathlineto{\pgfqpoint{6.739560in}{3.004909in}}%
\pgfusepath{stroke}%
\end{pgfscope}%
\begin{pgfscope}%
\pgfpathrectangle{\pgfqpoint{0.854460in}{0.571603in}}{\pgfqpoint{5.885100in}{5.225635in}}%
\pgfusepath{clip}%
\pgfsetbuttcap%
\pgfsetroundjoin%
\pgfsetlinewidth{1.505625pt}%
\definecolor{currentstroke}{rgb}{0.449368,0.813768,0.335384}%
\pgfsetstrokecolor{currentstroke}%
\pgfsetdash{}{0pt}%
\pgfpathmoveto{\pgfqpoint{0.854460in}{5.413227in}}%
\pgfpathlineto{\pgfqpoint{0.874148in}{5.429606in}}%
\pgfpathlineto{\pgfqpoint{0.884034in}{5.437731in}}%
\pgfpathlineto{\pgfqpoint{0.890509in}{5.443016in}}%
\pgfusepath{stroke}%
\end{pgfscope}%
\begin{pgfscope}%
\pgfpathrectangle{\pgfqpoint{0.854460in}{0.571603in}}{\pgfqpoint{5.885100in}{5.225635in}}%
\pgfusepath{clip}%
\pgfsetbuttcap%
\pgfsetroundjoin%
\pgfsetlinewidth{1.505625pt}%
\definecolor{currentstroke}{rgb}{0.449368,0.813768,0.335384}%
\pgfsetstrokecolor{currentstroke}%
\pgfsetdash{}{0pt}%
\pgfpathmoveto{\pgfqpoint{1.196465in}{5.672243in}}%
\pgfpathlineto{\pgfqpoint{1.209341in}{5.681077in}}%
\pgfpathlineto{\pgfqpoint{1.225644in}{5.692201in}}%
\pgfpathlineto{\pgfqpoint{1.238914in}{5.701146in}}%
\pgfpathlineto{\pgfqpoint{1.264735in}{5.718460in}}%
\pgfpathlineto{\pgfqpoint{1.268488in}{5.720946in}}%
\pgfpathlineto{\pgfqpoint{1.298061in}{5.740409in}}%
\pgfpathlineto{\pgfqpoint{1.304649in}{5.744720in}}%
\pgfpathlineto{\pgfqpoint{1.327634in}{5.759576in}}%
\pgfpathlineto{\pgfqpoint{1.345363in}{5.770979in}}%
\pgfpathlineto{\pgfqpoint{1.357208in}{5.778506in}}%
\pgfpathlineto{\pgfqpoint{1.386781in}{5.797208in}}%
\pgfpathlineto{\pgfqpoint{1.386830in}{5.797238in}}%
\pgfusepath{stroke}%
\end{pgfscope}%
\begin{pgfscope}%
\pgfpathrectangle{\pgfqpoint{0.854460in}{0.571603in}}{\pgfqpoint{5.885100in}{5.225635in}}%
\pgfusepath{clip}%
\pgfsetbuttcap%
\pgfsetroundjoin%
\pgfsetlinewidth{1.505625pt}%
\definecolor{currentstroke}{rgb}{0.496615,0.826376,0.306377}%
\pgfsetstrokecolor{currentstroke}%
\pgfsetdash{}{0pt}%
\pgfpathmoveto{\pgfqpoint{6.739560in}{3.693595in}}%
\pgfpathlineto{\pgfqpoint{6.738498in}{3.670221in}}%
\pgfpathlineto{\pgfqpoint{6.737572in}{3.643962in}}%
\pgfpathlineto{\pgfqpoint{6.736914in}{3.617702in}}%
\pgfpathlineto{\pgfqpoint{6.736526in}{3.591443in}}%
\pgfpathlineto{\pgfqpoint{6.736409in}{3.565183in}}%
\pgfpathlineto{\pgfqpoint{6.736562in}{3.538924in}}%
\pgfpathlineto{\pgfqpoint{6.736988in}{3.512664in}}%
\pgfpathlineto{\pgfqpoint{6.737688in}{3.486405in}}%
\pgfpathlineto{\pgfqpoint{6.738661in}{3.460145in}}%
\pgfpathlineto{\pgfqpoint{6.739560in}{3.441246in}}%
\pgfusepath{stroke}%
\end{pgfscope}%
\begin{pgfscope}%
\pgfpathrectangle{\pgfqpoint{0.854460in}{0.571603in}}{\pgfqpoint{5.885100in}{5.225635in}}%
\pgfusepath{clip}%
\pgfsetbuttcap%
\pgfsetroundjoin%
\pgfsetlinewidth{1.505625pt}%
\definecolor{currentstroke}{rgb}{0.496615,0.826376,0.306377}%
\pgfsetstrokecolor{currentstroke}%
\pgfsetdash{}{0pt}%
\pgfpathmoveto{\pgfqpoint{0.854460in}{5.460030in}}%
\pgfpathlineto{\pgfqpoint{0.881642in}{5.482125in}}%
\pgfpathlineto{\pgfqpoint{0.884034in}{5.484046in}}%
\pgfpathlineto{\pgfqpoint{0.900856in}{5.497463in}}%
\pgfusepath{stroke}%
\end{pgfscope}%
\begin{pgfscope}%
\pgfpathrectangle{\pgfqpoint{0.854460in}{0.571603in}}{\pgfqpoint{5.885100in}{5.225635in}}%
\pgfusepath{clip}%
\pgfsetbuttcap%
\pgfsetroundjoin%
\pgfsetlinewidth{1.505625pt}%
\definecolor{currentstroke}{rgb}{0.496615,0.826376,0.306377}%
\pgfsetstrokecolor{currentstroke}%
\pgfsetdash{}{0pt}%
\pgfpathmoveto{\pgfqpoint{1.209342in}{5.722987in}}%
\pgfpathlineto{\pgfqpoint{1.238914in}{5.742749in}}%
\pgfpathlineto{\pgfqpoint{1.241882in}{5.744720in}}%
\pgfpathlineto{\pgfqpoint{1.268488in}{5.762169in}}%
\pgfpathlineto{\pgfqpoint{1.281987in}{5.770979in}}%
\pgfpathlineto{\pgfqpoint{1.298061in}{5.781342in}}%
\pgfpathlineto{\pgfqpoint{1.322837in}{5.797238in}}%
\pgfusepath{stroke}%
\end{pgfscope}%
\begin{pgfscope}%
\pgfpathrectangle{\pgfqpoint{0.854460in}{0.571603in}}{\pgfqpoint{5.885100in}{5.225635in}}%
\pgfusepath{clip}%
\pgfsetbuttcap%
\pgfsetroundjoin%
\pgfsetlinewidth{1.505625pt}%
\definecolor{currentstroke}{rgb}{0.555484,0.840254,0.269281}%
\pgfsetstrokecolor{currentstroke}%
\pgfsetdash{}{0pt}%
\pgfpathmoveto{\pgfqpoint{0.854460in}{5.505341in}}%
\pgfpathlineto{\pgfqpoint{0.858247in}{5.508384in}}%
\pgfpathlineto{\pgfqpoint{0.879483in}{5.525244in}}%
\pgfusepath{stroke}%
\end{pgfscope}%
\begin{pgfscope}%
\pgfpathrectangle{\pgfqpoint{0.854460in}{0.571603in}}{\pgfqpoint{5.885100in}{5.225635in}}%
\pgfusepath{clip}%
\pgfsetbuttcap%
\pgfsetroundjoin%
\pgfsetlinewidth{1.505625pt}%
\definecolor{currentstroke}{rgb}{0.555484,0.840254,0.269281}%
\pgfsetstrokecolor{currentstroke}%
\pgfsetdash{}{0pt}%
\pgfpathmoveto{\pgfqpoint{1.188581in}{5.749859in}}%
\pgfpathlineto{\pgfqpoint{1.209341in}{5.763673in}}%
\pgfpathlineto{\pgfqpoint{1.220377in}{5.770979in}}%
\pgfpathlineto{\pgfqpoint{1.238914in}{5.783104in}}%
\pgfpathlineto{\pgfqpoint{1.260631in}{5.797238in}}%
\pgfusepath{stroke}%
\end{pgfscope}%
\begin{pgfscope}%
\pgfpathrectangle{\pgfqpoint{0.854460in}{0.571603in}}{\pgfqpoint{5.885100in}{5.225635in}}%
\pgfusepath{clip}%
\pgfsetbuttcap%
\pgfsetroundjoin%
\pgfsetlinewidth{1.505625pt}%
\definecolor{currentstroke}{rgb}{0.606045,0.850733,0.236712}%
\pgfsetstrokecolor{currentstroke}%
\pgfsetdash{}{0pt}%
\pgfpathmoveto{\pgfqpoint{0.854460in}{5.549232in}}%
\pgfpathlineto{\pgfqpoint{0.869317in}{5.560903in}}%
\pgfpathlineto{\pgfqpoint{0.878551in}{5.568070in}}%
\pgfusepath{stroke}%
\end{pgfscope}%
\begin{pgfscope}%
\pgfpathrectangle{\pgfqpoint{0.854460in}{0.571603in}}{\pgfqpoint{5.885100in}{5.225635in}}%
\pgfusepath{clip}%
\pgfsetbuttcap%
\pgfsetroundjoin%
\pgfsetlinewidth{1.505625pt}%
\definecolor{currentstroke}{rgb}{0.606045,0.850733,0.236712}%
\pgfsetstrokecolor{currentstroke}%
\pgfsetdash{}{0pt}%
\pgfpathmoveto{\pgfqpoint{1.189343in}{5.790135in}}%
\pgfpathlineto{\pgfqpoint{1.200103in}{5.797238in}}%
\pgfusepath{stroke}%
\end{pgfscope}%
\begin{pgfscope}%
\pgfpathrectangle{\pgfqpoint{0.854460in}{0.571603in}}{\pgfqpoint{5.885100in}{5.225635in}}%
\pgfusepath{clip}%
\pgfsetbuttcap%
\pgfsetroundjoin%
\pgfsetlinewidth{1.505625pt}%
\definecolor{currentstroke}{rgb}{0.668054,0.861999,0.196293}%
\pgfsetstrokecolor{currentstroke}%
\pgfsetdash{}{0pt}%
\pgfpathmoveto{\pgfqpoint{0.854460in}{5.591890in}}%
\pgfpathlineto{\pgfqpoint{0.882488in}{5.613422in}}%
\pgfpathlineto{\pgfqpoint{0.884034in}{5.614596in}}%
\pgfpathlineto{\pgfqpoint{0.913607in}{5.636891in}}%
\pgfpathlineto{\pgfqpoint{0.917332in}{5.639682in}}%
\pgfpathlineto{\pgfqpoint{0.943181in}{5.658813in}}%
\pgfpathlineto{\pgfqpoint{0.952869in}{5.665941in}}%
\pgfpathlineto{\pgfqpoint{0.972754in}{5.680397in}}%
\pgfpathlineto{\pgfqpoint{0.989085in}{5.692201in}}%
\pgfpathlineto{\pgfqpoint{1.002327in}{5.701658in}}%
\pgfpathlineto{\pgfqpoint{1.025987in}{5.718460in}}%
\pgfpathlineto{\pgfqpoint{1.031901in}{5.722609in}}%
\pgfpathlineto{\pgfqpoint{1.061474in}{5.743238in}}%
\pgfpathlineto{\pgfqpoint{1.063612in}{5.744720in}}%
\pgfpathlineto{\pgfqpoint{1.091047in}{5.763504in}}%
\pgfpathlineto{\pgfqpoint{1.102023in}{5.770979in}}%
\pgfpathlineto{\pgfqpoint{1.120621in}{5.783494in}}%
\pgfpathlineto{\pgfqpoint{1.141151in}{5.797238in}}%
\pgfusepath{stroke}%
\end{pgfscope}%
\begin{pgfscope}%
\pgfpathrectangle{\pgfqpoint{0.854460in}{0.571603in}}{\pgfqpoint{5.885100in}{5.225635in}}%
\pgfusepath{clip}%
\pgfsetbuttcap%
\pgfsetroundjoin%
\pgfsetlinewidth{1.505625pt}%
\definecolor{currentstroke}{rgb}{0.720391,0.870350,0.162603}%
\pgfsetstrokecolor{currentstroke}%
\pgfsetdash{}{0pt}%
\pgfpathmoveto{\pgfqpoint{0.854460in}{5.633306in}}%
\pgfpathlineto{\pgfqpoint{0.862851in}{5.639682in}}%
\pgfpathlineto{\pgfqpoint{0.884034in}{5.655585in}}%
\pgfpathlineto{\pgfqpoint{0.897911in}{5.665941in}}%
\pgfpathlineto{\pgfqpoint{0.913607in}{5.677515in}}%
\pgfpathlineto{\pgfqpoint{0.933640in}{5.692201in}}%
\pgfpathlineto{\pgfqpoint{0.943181in}{5.699111in}}%
\pgfpathlineto{\pgfqpoint{0.970046in}{5.718460in}}%
\pgfpathlineto{\pgfqpoint{0.972754in}{5.720387in}}%
\pgfpathlineto{\pgfqpoint{1.002327in}{5.741296in}}%
\pgfpathlineto{\pgfqpoint{1.007198in}{5.744720in}}%
\pgfpathlineto{\pgfqpoint{1.031901in}{5.761872in}}%
\pgfpathlineto{\pgfqpoint{1.045086in}{5.770979in}}%
\pgfpathlineto{\pgfqpoint{1.061474in}{5.782163in}}%
\pgfpathlineto{\pgfqpoint{1.083680in}{5.797238in}}%
\pgfusepath{stroke}%
\end{pgfscope}%
\begin{pgfscope}%
\pgfpathrectangle{\pgfqpoint{0.854460in}{0.571603in}}{\pgfqpoint{5.885100in}{5.225635in}}%
\pgfusepath{clip}%
\pgfsetbuttcap%
\pgfsetroundjoin%
\pgfsetlinewidth{1.505625pt}%
\definecolor{currentstroke}{rgb}{0.783315,0.879285,0.125405}%
\pgfsetstrokecolor{currentstroke}%
\pgfsetdash{}{0pt}%
\pgfpathmoveto{\pgfqpoint{0.854460in}{5.673587in}}%
\pgfpathlineto{\pgfqpoint{0.879497in}{5.692201in}}%
\pgfpathlineto{\pgfqpoint{0.884034in}{5.695533in}}%
\pgfpathlineto{\pgfqpoint{0.913607in}{5.717125in}}%
\pgfpathlineto{\pgfqpoint{0.915448in}{5.718460in}}%
\pgfpathlineto{\pgfqpoint{0.943181in}{5.738334in}}%
\pgfpathlineto{\pgfqpoint{0.952140in}{5.744720in}}%
\pgfpathlineto{\pgfqpoint{0.972754in}{5.759235in}}%
\pgfpathlineto{\pgfqpoint{0.989523in}{5.770979in}}%
\pgfpathlineto{\pgfqpoint{1.002327in}{5.779840in}}%
\pgfpathlineto{\pgfqpoint{1.027602in}{5.797238in}}%
\pgfusepath{stroke}%
\end{pgfscope}%
\begin{pgfscope}%
\pgfpathrectangle{\pgfqpoint{0.854460in}{0.571603in}}{\pgfqpoint{5.885100in}{5.225635in}}%
\pgfusepath{clip}%
\pgfsetbuttcap%
\pgfsetroundjoin%
\pgfsetlinewidth{1.505625pt}%
\definecolor{currentstroke}{rgb}{0.835270,0.886029,0.102646}%
\pgfsetstrokecolor{currentstroke}%
\pgfsetdash{}{0pt}%
\pgfpathmoveto{\pgfqpoint{0.854460in}{5.712811in}}%
\pgfpathlineto{\pgfqpoint{0.862142in}{5.718460in}}%
\pgfpathlineto{\pgfqpoint{0.884034in}{5.734370in}}%
\pgfpathlineto{\pgfqpoint{0.898357in}{5.744720in}}%
\pgfpathlineto{\pgfqpoint{0.913607in}{5.755609in}}%
\pgfpathlineto{\pgfqpoint{0.935252in}{5.770979in}}%
\pgfpathlineto{\pgfqpoint{0.943181in}{5.776542in}}%
\pgfpathlineto{\pgfqpoint{0.972754in}{5.797182in}}%
\pgfpathlineto{\pgfqpoint{0.972835in}{5.797238in}}%
\pgfusepath{stroke}%
\end{pgfscope}%
\begin{pgfscope}%
\pgfpathrectangle{\pgfqpoint{0.854460in}{0.571603in}}{\pgfqpoint{5.885100in}{5.225635in}}%
\pgfusepath{clip}%
\pgfsetbuttcap%
\pgfsetroundjoin%
\pgfsetlinewidth{1.505625pt}%
\definecolor{currentstroke}{rgb}{0.896320,0.893616,0.096335}%
\pgfsetstrokecolor{currentstroke}%
\pgfsetdash{}{0pt}%
\pgfpathmoveto{\pgfqpoint{0.854460in}{5.751010in}}%
\pgfpathlineto{\pgfqpoint{0.882197in}{5.770979in}}%
\pgfpathlineto{\pgfqpoint{0.884034in}{5.772286in}}%
\pgfpathlineto{\pgfqpoint{0.913607in}{5.793188in}}%
\pgfpathlineto{\pgfqpoint{0.919371in}{5.797238in}}%
\pgfusepath{stroke}%
\end{pgfscope}%
\begin{pgfscope}%
\pgfpathrectangle{\pgfqpoint{0.854460in}{0.571603in}}{\pgfqpoint{5.885100in}{5.225635in}}%
\pgfusepath{clip}%
\pgfsetbuttcap%
\pgfsetroundjoin%
\pgfsetlinewidth{1.505625pt}%
\definecolor{currentstroke}{rgb}{0.945636,0.899815,0.112838}%
\pgfsetstrokecolor{currentstroke}%
\pgfsetdash{}{0pt}%
\pgfpathmoveto{\pgfqpoint{0.854460in}{5.788247in}}%
\pgfpathlineto{\pgfqpoint{0.867082in}{5.797238in}}%
\pgfusepath{stroke}%
\end{pgfscope}%
\begin{pgfscope}%
\pgfsetrectcap%
\pgfsetmiterjoin%
\pgfsetlinewidth{0.803000pt}%
\definecolor{currentstroke}{rgb}{0.000000,0.000000,0.000000}%
\pgfsetstrokecolor{currentstroke}%
\pgfsetdash{}{0pt}%
\pgfpathmoveto{\pgfqpoint{0.854460in}{0.571603in}}%
\pgfpathlineto{\pgfqpoint{0.854460in}{5.797238in}}%
\pgfusepath{stroke}%
\end{pgfscope}%
\begin{pgfscope}%
\pgfsetrectcap%
\pgfsetmiterjoin%
\pgfsetlinewidth{0.803000pt}%
\definecolor{currentstroke}{rgb}{0.000000,0.000000,0.000000}%
\pgfsetstrokecolor{currentstroke}%
\pgfsetdash{}{0pt}%
\pgfpathmoveto{\pgfqpoint{6.739560in}{0.571603in}}%
\pgfpathlineto{\pgfqpoint{6.739560in}{5.797238in}}%
\pgfusepath{stroke}%
\end{pgfscope}%
\begin{pgfscope}%
\pgfsetrectcap%
\pgfsetmiterjoin%
\pgfsetlinewidth{0.803000pt}%
\definecolor{currentstroke}{rgb}{0.000000,0.000000,0.000000}%
\pgfsetstrokecolor{currentstroke}%
\pgfsetdash{}{0pt}%
\pgfpathmoveto{\pgfqpoint{0.854460in}{0.571603in}}%
\pgfpathlineto{\pgfqpoint{6.739560in}{0.571603in}}%
\pgfusepath{stroke}%
\end{pgfscope}%
\begin{pgfscope}%
\pgfsetrectcap%
\pgfsetmiterjoin%
\pgfsetlinewidth{0.803000pt}%
\definecolor{currentstroke}{rgb}{0.000000,0.000000,0.000000}%
\pgfsetstrokecolor{currentstroke}%
\pgfsetdash{}{0pt}%
\pgfpathmoveto{\pgfqpoint{0.854460in}{5.797238in}}%
\pgfpathlineto{\pgfqpoint{6.739560in}{5.797238in}}%
\pgfusepath{stroke}%
\end{pgfscope}%
\begin{pgfscope}%
\definecolor{textcolor}{rgb}{0.273809,0.031497,0.358853}%
\pgfsetstrokecolor{textcolor}%
\pgfsetfillcolor{textcolor}%
\pgftext[x=3.835467in, y=1.453917in, left, base,rotate=313.690413]{\color{textcolor}\sffamily\fontsize{8.000000}{9.600000}\selectfont 1.65}%
\end{pgfscope}%
\begin{pgfscope}%
\definecolor{textcolor}{rgb}{0.278791,0.062145,0.386592}%
\pgfsetstrokecolor{textcolor}%
\pgfsetfillcolor{textcolor}%
\pgftext[x=4.238110in, y=1.389713in, left, base,rotate=312.368660]{\color{textcolor}\sffamily\fontsize{8.000000}{9.600000}\selectfont 1.80}%
\end{pgfscope}%
\begin{pgfscope}%
\definecolor{textcolor}{rgb}{0.282327,0.094955,0.417331}%
\pgfsetstrokecolor{textcolor}%
\pgfsetfillcolor{textcolor}%
\pgftext[x=4.743645in, y=1.087834in, left, base,rotate=312.861190]{\color{textcolor}\sffamily\fontsize{8.000000}{9.600000}\selectfont 1.95}%
\end{pgfscope}%
\begin{pgfscope}%
\definecolor{textcolor}{rgb}{0.283229,0.120777,0.440584}%
\pgfsetstrokecolor{textcolor}%
\pgfsetfillcolor{textcolor}%
\pgftext[x=3.838407in, y=2.464382in, left, base,rotate=306.690648]{\color{textcolor}\sffamily\fontsize{8.000000}{9.600000}\selectfont 2.10}%
\end{pgfscope}%
\begin{pgfscope}%
\definecolor{textcolor}{rgb}{0.281887,0.150881,0.465405}%
\pgfsetstrokecolor{textcolor}%
\pgfsetfillcolor{textcolor}%
\pgftext[x=5.341234in, y=0.847176in, left, base,rotate=315.154933]{\color{textcolor}\sffamily\fontsize{8.000000}{9.600000}\selectfont 2.25}%
\end{pgfscope}%
\begin{pgfscope}%
\definecolor{textcolor}{rgb}{0.278826,0.175490,0.483397}%
\pgfsetstrokecolor{textcolor}%
\pgfsetfillcolor{textcolor}%
\pgftext[x=3.756018in, y=3.308780in, left, base,rotate=298.129815]{\color{textcolor}\sffamily\fontsize{8.000000}{9.600000}\selectfont 2.40}%
\end{pgfscope}%
\begin{pgfscope}%
\definecolor{textcolor}{rgb}{0.273006,0.204520,0.501721}%
\pgfsetstrokecolor{textcolor}%
\pgfsetfillcolor{textcolor}%
\pgftext[x=5.154117in, y=1.372439in, left, base,rotate=312.995477]{\color{textcolor}\sffamily\fontsize{8.000000}{9.600000}\selectfont 2.55}%
\end{pgfscope}%
\begin{pgfscope}%
\definecolor{textcolor}{rgb}{0.266580,0.228262,0.514349}%
\pgfsetstrokecolor{textcolor}%
\pgfsetfillcolor{textcolor}%
\pgftext[x=4.065769in, y=3.459660in, left, base,rotate=290.956508]{\color{textcolor}\sffamily\fontsize{8.000000}{9.600000}\selectfont 2.70}%
\end{pgfscope}%
\begin{pgfscope}%
\definecolor{textcolor}{rgb}{0.257322,0.256130,0.526563}%
\pgfsetstrokecolor{textcolor}%
\pgfsetfillcolor{textcolor}%
\pgftext[x=5.724790in, y=1.029333in, left, base,rotate=316.020956]{\color{textcolor}\sffamily\fontsize{8.000000}{9.600000}\selectfont 2.85}%
\end{pgfscope}%
\begin{pgfscope}%
\definecolor{textcolor}{rgb}{0.248629,0.278775,0.534556}%
\pgfsetstrokecolor{textcolor}%
\pgfsetfillcolor{textcolor}%
\pgftext[x=4.255841in, y=3.926004in, left, base,rotate=283.510037]{\color{textcolor}\sffamily\fontsize{8.000000}{9.600000}\selectfont 3.00}%
\end{pgfscope}%
\begin{pgfscope}%
\definecolor{textcolor}{rgb}{0.239346,0.300855,0.540844}%
\pgfsetstrokecolor{textcolor}%
\pgfsetfillcolor{textcolor}%
\pgftext[x=6.061820in, y=0.921798in, left, base,rotate=317.309850]{\color{textcolor}\sffamily\fontsize{8.000000}{9.600000}\selectfont 3.15}%
\end{pgfscope}%
\begin{pgfscope}%
\definecolor{textcolor}{rgb}{0.227802,0.326594,0.546532}%
\pgfsetstrokecolor{textcolor}%
\pgfsetfillcolor{textcolor}%
\pgftext[x=5.599219in, y=1.537026in, left, base,rotate=311.778721]{\color{textcolor}\sffamily\fontsize{8.000000}{9.600000}\selectfont 3.30}%
\end{pgfscope}%
\begin{pgfscope}%
\definecolor{textcolor}{rgb}{0.218130,0.347432,0.550038}%
\pgfsetstrokecolor{textcolor}%
\pgfsetfillcolor{textcolor}%
\pgftext[x=4.653047in, y=4.131735in, left, base,rotate=275.567428]{\color{textcolor}\sffamily\fontsize{8.000000}{9.600000}\selectfont 3.45}%
\end{pgfscope}%
\begin{pgfscope}%
\definecolor{textcolor}{rgb}{0.206756,0.371758,0.553117}%
\pgfsetstrokecolor{textcolor}%
\pgfsetfillcolor{textcolor}%
\pgftext[x=6.009996in, y=1.292993in, left, base,rotate=314.465189]{\color{textcolor}\sffamily\fontsize{8.000000}{9.600000}\selectfont 3.60}%
\end{pgfscope}%
\begin{pgfscope}%
\definecolor{textcolor}{rgb}{0.197636,0.391528,0.554969}%
\pgfsetstrokecolor{textcolor}%
\pgfsetfillcolor{textcolor}%
\pgftext[x=1.560240in, y=0.790944in, left, base,rotate=317.082108]{\color{textcolor}\sffamily\fontsize{8.000000}{9.600000}\selectfont 3.75}%
\end{pgfscope}%
\begin{pgfscope}%
\definecolor{textcolor}{rgb}{0.197636,0.391528,0.554969}%
\pgfsetstrokecolor{textcolor}%
\pgfsetfillcolor{textcolor}%
\pgftext[x=4.897248in, y=4.424045in, left, base,rotate=270.187789]{\color{textcolor}\sffamily\fontsize{8.000000}{9.600000}\selectfont 3.75}%
\end{pgfscope}%
\begin{pgfscope}%
\definecolor{textcolor}{rgb}{0.187231,0.414746,0.556547}%
\pgfsetstrokecolor{textcolor}%
\pgfsetfillcolor{textcolor}%
\pgftext[x=1.167604in, y=1.174159in, left, base,rotate=308.419345]{\color{textcolor}\sffamily\fontsize{8.000000}{9.600000}\selectfont 3.90}%
\end{pgfscope}%
\begin{pgfscope}%
\definecolor{textcolor}{rgb}{0.187231,0.414746,0.556547}%
\pgfsetstrokecolor{textcolor}%
\pgfsetfillcolor{textcolor}%
\pgftext[x=5.636613in, y=2.001820in, left, base,rotate=305.962539]{\color{textcolor}\sffamily\fontsize{8.000000}{9.600000}\selectfont 3.90}%
\end{pgfscope}%
\begin{pgfscope}%
\definecolor{textcolor}{rgb}{0.179019,0.433756,0.557430}%
\pgfsetstrokecolor{textcolor}%
\pgfsetfillcolor{textcolor}%
\pgftext[x=0.859998in, y=1.584449in, left, base,rotate=298.711312]{\color{textcolor}\sffamily\fontsize{8.000000}{9.600000}\selectfont 4.05}%
\end{pgfscope}%
\begin{pgfscope}%
\definecolor{textcolor}{rgb}{0.179019,0.433756,0.557430}%
\pgfsetstrokecolor{textcolor}%
\pgfsetfillcolor{textcolor}%
\pgftext[x=2.672938in, y=5.413545in, left, base,rotate=21.550188]{\color{textcolor}\sffamily\fontsize{8.000000}{9.600000}\selectfont 4.05}%
\end{pgfscope}%
\begin{pgfscope}%
\definecolor{textcolor}{rgb}{0.179019,0.433756,0.557430}%
\pgfsetstrokecolor{textcolor}%
\pgfsetfillcolor{textcolor}%
\pgftext[x=5.205035in, y=4.568747in, left, base,rotate=85.845100]{\color{textcolor}\sffamily\fontsize{8.000000}{9.600000}\selectfont 4.05}%
\end{pgfscope}%
\begin{pgfscope}%
\definecolor{textcolor}{rgb}{0.169646,0.456262,0.558030}%
\pgfsetstrokecolor{textcolor}%
\pgfsetfillcolor{textcolor}%
\pgftext[x=1.335526in, y=0.794816in, left, base,rotate=316.114333]{\color{textcolor}\sffamily\fontsize{8.000000}{9.600000}\selectfont 4.20}%
\end{pgfscope}%
\begin{pgfscope}%
\definecolor{textcolor}{rgb}{0.169646,0.456262,0.558030}%
\pgfsetstrokecolor{textcolor}%
\pgfsetfillcolor{textcolor}%
\pgftext[x=2.294789in, y=5.301400in, left, base,rotate=26.177995]{\color{textcolor}\sffamily\fontsize{8.000000}{9.600000}\selectfont 4.20}%
\end{pgfscope}%
\begin{pgfscope}%
\definecolor{textcolor}{rgb}{0.169646,0.456262,0.558030}%
\pgfsetstrokecolor{textcolor}%
\pgfsetfillcolor{textcolor}%
\pgftext[x=6.369078in, y=1.294305in, left, base,rotate=314.777759]{\color{textcolor}\sffamily\fontsize{8.000000}{9.600000}\selectfont 4.20}%
\end{pgfscope}%
\begin{pgfscope}%
\definecolor{textcolor}{rgb}{0.162142,0.474838,0.558140}%
\pgfsetstrokecolor{textcolor}%
\pgfsetfillcolor{textcolor}%
\pgftext[x=1.268435in, y=0.793294in, left, base,rotate=315.866959]{\color{textcolor}\sffamily\fontsize{8.000000}{9.600000}\selectfont 4.35}%
\end{pgfscope}%
\begin{pgfscope}%
\definecolor{textcolor}{rgb}{0.162142,0.474838,0.558140}%
\pgfsetstrokecolor{textcolor}%
\pgfsetfillcolor{textcolor}%
\pgftext[x=1.917844in, y=5.146308in, left, base,rotate=31.167936]{\color{textcolor}\sffamily\fontsize{8.000000}{9.600000}\selectfont 4.35}%
\end{pgfscope}%
\begin{pgfscope}%
\definecolor{textcolor}{rgb}{0.162142,0.474838,0.558140}%
\pgfsetstrokecolor{textcolor}%
\pgfsetfillcolor{textcolor}%
\pgftext[x=6.162264in, y=1.632708in, left, base,rotate=310.870830]{\color{textcolor}\sffamily\fontsize{8.000000}{9.600000}\selectfont 4.35}%
\end{pgfscope}%
\begin{pgfscope}%
\definecolor{textcolor}{rgb}{0.154815,0.493313,0.557840}%
\pgfsetstrokecolor{textcolor}%
\pgfsetfillcolor{textcolor}%
\pgftext[x=1.188584in, y=0.808143in, left, base,rotate=315.291812]{\color{textcolor}\sffamily\fontsize{8.000000}{9.600000}\selectfont 4.50}%
\end{pgfscope}%
\begin{pgfscope}%
\definecolor{textcolor}{rgb}{0.154815,0.493313,0.557840}%
\pgfsetstrokecolor{textcolor}%
\pgfsetfillcolor{textcolor}%
\pgftext[x=1.570753in, y=4.963806in, left, base,rotate=36.463096]{\color{textcolor}\sffamily\fontsize{8.000000}{9.600000}\selectfont 4.50}%
\end{pgfscope}%
\begin{pgfscope}%
\definecolor{textcolor}{rgb}{0.154815,0.493313,0.557840}%
\pgfsetstrokecolor{textcolor}%
\pgfsetfillcolor{textcolor}%
\pgftext[x=5.567048in, y=4.803573in, left, base,rotate=80.244277]{\color{textcolor}\sffamily\fontsize{8.000000}{9.600000}\selectfont 4.50}%
\end{pgfscope}%
\begin{pgfscope}%
\definecolor{textcolor}{rgb}{0.146180,0.515413,0.556823}%
\pgfsetstrokecolor{textcolor}%
\pgfsetfillcolor{textcolor}%
\pgftext[x=0.836395in, y=1.176842in, left, base,rotate=307.169580]{\color{textcolor}\sffamily\fontsize{8.000000}{9.600000}\selectfont 4.65}%
\end{pgfscope}%
\begin{pgfscope}%
\definecolor{textcolor}{rgb}{0.146180,0.515413,0.556823}%
\pgfsetstrokecolor{textcolor}%
\pgfsetfillcolor{textcolor}%
\pgftext[x=1.245619in, y=4.745825in, left, base,rotate=42.421140]{\color{textcolor}\sffamily\fontsize{8.000000}{9.600000}\selectfont 4.65}%
\end{pgfscope}%
\begin{pgfscope}%
\definecolor{textcolor}{rgb}{0.146180,0.515413,0.556823}%
\pgfsetstrokecolor{textcolor}%
\pgfsetfillcolor{textcolor}%
\pgftext[x=6.079067in, y=1.966092in, left, base,rotate=306.317042]{\color{textcolor}\sffamily\fontsize{8.000000}{9.600000}\selectfont 4.65}%
\end{pgfscope}%
\begin{pgfscope}%
\definecolor{textcolor}{rgb}{0.139147,0.533812,0.555298}%
\pgfsetstrokecolor{textcolor}%
\pgfsetfillcolor{textcolor}%
\pgftext[x=1.070645in, y=0.798591in, left, base,rotate=314.982643]{\color{textcolor}\sffamily\fontsize{8.000000}{9.600000}\selectfont 4.80}%
\end{pgfscope}%
\begin{pgfscope}%
\definecolor{textcolor}{rgb}{0.139147,0.533812,0.555298}%
\pgfsetstrokecolor{textcolor}%
\pgfsetfillcolor{textcolor}%
\pgftext[x=0.944780in, y=4.486673in, left, base,rotate=49.237941]{\color{textcolor}\sffamily\fontsize{8.000000}{9.600000}\selectfont 4.80}%
\end{pgfscope}%
\begin{pgfscope}%
\definecolor{textcolor}{rgb}{0.139147,0.533812,0.555298}%
\pgfsetstrokecolor{textcolor}%
\pgfsetfillcolor{textcolor}%
\pgftext[x=5.679344in, y=4.330822in, left, base,rotate=79.716205]{\color{textcolor}\sffamily\fontsize{8.000000}{9.600000}\selectfont 4.80}%
\end{pgfscope}%
\begin{pgfscope}%
\definecolor{textcolor}{rgb}{0.131172,0.555899,0.552459}%
\pgfsetstrokecolor{textcolor}%
\pgfsetfillcolor{textcolor}%
\pgftext[x=1.011743in, y=0.796703in, left, base,rotate=314.770236]{\color{textcolor}\sffamily\fontsize{8.000000}{9.600000}\selectfont 4.95}%
\end{pgfscope}%
\begin{pgfscope}%
\definecolor{textcolor}{rgb}{0.131172,0.555899,0.552459}%
\pgfsetstrokecolor{textcolor}%
\pgfsetfillcolor{textcolor}%
\pgftext[x=2.497697in, y=5.717366in, left, base,rotate=23.220107]{\color{textcolor}\sffamily\fontsize{8.000000}{9.600000}\selectfont 4.95}%
\end{pgfscope}%
\begin{pgfscope}%
\definecolor{textcolor}{rgb}{0.131172,0.555899,0.552459}%
\pgfsetstrokecolor{textcolor}%
\pgfsetfillcolor{textcolor}%
\pgftext[x=5.879960in, y=4.852142in, left, base,rotate=75.195251]{\color{textcolor}\sffamily\fontsize{8.000000}{9.600000}\selectfont 4.95}%
\end{pgfscope}%
\begin{pgfscope}%
\definecolor{textcolor}{rgb}{0.125394,0.574318,0.549086}%
\pgfsetstrokecolor{textcolor}%
\pgfsetfillcolor{textcolor}%
\pgftext[x=0.952876in, y=0.796624in, left, base,rotate=314.528502]{\color{textcolor}\sffamily\fontsize{8.000000}{9.600000}\selectfont 5.10}%
\end{pgfscope}%
\begin{pgfscope}%
\definecolor{textcolor}{rgb}{0.125394,0.574318,0.549086}%
\pgfsetstrokecolor{textcolor}%
\pgfsetfillcolor{textcolor}%
\pgftext[x=2.066548in, y=5.556422in, left, base,rotate=26.875090]{\color{textcolor}\sffamily\fontsize{8.000000}{9.600000}\selectfont 5.10}%
\end{pgfscope}%
\begin{pgfscope}%
\definecolor{textcolor}{rgb}{0.125394,0.574318,0.549086}%
\pgfsetstrokecolor{textcolor}%
\pgfsetfillcolor{textcolor}%
\pgftext[x=6.032868in, y=2.447549in, left, base,rotate=298.311033]{\color{textcolor}\sffamily\fontsize{8.000000}{9.600000}\selectfont 5.10}%
\end{pgfscope}%
\begin{pgfscope}%
\definecolor{textcolor}{rgb}{0.120565,0.596422,0.543611}%
\pgfsetstrokecolor{textcolor}%
\pgfsetfillcolor{textcolor}%
\pgftext[x=1.661852in, y=5.364984in, left, base,rotate=31.572470]{\color{textcolor}\sffamily\fontsize{8.000000}{9.600000}\selectfont 5.25}%
\end{pgfscope}%
\begin{pgfscope}%
\definecolor{textcolor}{rgb}{0.120565,0.596422,0.543611}%
\pgfsetstrokecolor{textcolor}%
\pgfsetfillcolor{textcolor}%
\pgftext[x=5.914806in, y=2.964062in, left, base,rotate=288.243380]{\color{textcolor}\sffamily\fontsize{8.000000}{9.600000}\selectfont 5.25}%
\end{pgfscope}%
\begin{pgfscope}%
\definecolor{textcolor}{rgb}{0.119483,0.614817,0.537692}%
\pgfsetstrokecolor{textcolor}%
\pgfsetfillcolor{textcolor}%
\pgftext[x=1.290793in, y=5.147056in, left, base,rotate=36.864246]{\color{textcolor}\sffamily\fontsize{8.000000}{9.600000}\selectfont 5.40}%
\end{pgfscope}%
\begin{pgfscope}%
\definecolor{textcolor}{rgb}{0.119483,0.614817,0.537692}%
\pgfsetstrokecolor{textcolor}%
\pgfsetfillcolor{textcolor}%
\pgftext[x=6.035320in, y=4.409348in, left, base,rotate=75.643307]{\color{textcolor}\sffamily\fontsize{8.000000}{9.600000}\selectfont 5.40}%
\end{pgfscope}%
\begin{pgfscope}%
\definecolor{textcolor}{rgb}{0.123444,0.636809,0.528763}%
\pgfsetstrokecolor{textcolor}%
\pgfsetfillcolor{textcolor}%
\pgftext[x=0.956149in, y=4.902916in, left, base,rotate=42.801183]{\color{textcolor}\sffamily\fontsize{8.000000}{9.600000}\selectfont 5.55}%
\end{pgfscope}%
\begin{pgfscope}%
\definecolor{textcolor}{rgb}{0.123444,0.636809,0.528763}%
\pgfsetstrokecolor{textcolor}%
\pgfsetfillcolor{textcolor}%
\pgftext[x=6.267239in, y=4.935217in, left, base,rotate=70.147409]{\color{textcolor}\sffamily\fontsize{8.000000}{9.600000}\selectfont 5.55}%
\end{pgfscope}%
\begin{pgfscope}%
\definecolor{textcolor}{rgb}{0.132268,0.655014,0.519661}%
\pgfsetstrokecolor{textcolor}%
\pgfsetfillcolor{textcolor}%
\pgftext[x=1.698091in, y=5.551786in, left, base,rotate=29.688537]{\color{textcolor}\sffamily\fontsize{8.000000}{9.600000}\selectfont 5.70}%
\end{pgfscope}%
\begin{pgfscope}%
\definecolor{textcolor}{rgb}{0.132268,0.655014,0.519661}%
\pgfsetstrokecolor{textcolor}%
\pgfsetfillcolor{textcolor}%
\pgftext[x=6.539746in, y=5.434279in, left, base,rotate=69.596688]{\color{textcolor}\sffamily\fontsize{8.000000}{9.600000}\selectfont 5.70}%
\end{pgfscope}%
\begin{pgfscope}%
\definecolor{textcolor}{rgb}{0.146616,0.673050,0.508936}%
\pgfsetstrokecolor{textcolor}%
\pgfsetfillcolor{textcolor}%
\pgftext[x=1.306055in, y=5.334061in, left, base,rotate=34.648669]{\color{textcolor}\sffamily\fontsize{8.000000}{9.600000}\selectfont 5.85}%
\end{pgfscope}%
\begin{pgfscope}%
\definecolor{textcolor}{rgb}{0.146616,0.673050,0.508936}%
\pgfsetstrokecolor{textcolor}%
\pgfsetfillcolor{textcolor}%
\pgftext[x=6.189433in, y=3.953276in, left, base,rotate=81.379693]{\color{textcolor}\sffamily\fontsize{8.000000}{9.600000}\selectfont 5.85}%
\end{pgfscope}%
\begin{pgfscope}%
\definecolor{textcolor}{rgb}{0.170948,0.694384,0.493803}%
\pgfsetstrokecolor{textcolor}%
\pgfsetfillcolor{textcolor}%
\pgftext[x=1.305882in, y=5.387419in, left, base,rotate=34.082045]{\color{textcolor}\sffamily\fontsize{8.000000}{9.600000}\selectfont 6.00}%
\end{pgfscope}%
\begin{pgfscope}%
\definecolor{textcolor}{rgb}{0.170948,0.694384,0.493803}%
\pgfsetstrokecolor{textcolor}%
\pgfsetfillcolor{textcolor}%
\pgftext[x=6.503512in, y=2.343526in, left, base,rotate=299.866712]{\color{textcolor}\sffamily\fontsize{8.000000}{9.600000}\selectfont 6.00}%
\end{pgfscope}%
\begin{pgfscope}%
\definecolor{textcolor}{rgb}{0.196571,0.711827,0.479221}%
\pgfsetstrokecolor{textcolor}%
\pgfsetfillcolor{textcolor}%
\pgftext[x=1.329632in, y=5.456078in, left, base,rotate=33.199178]{\color{textcolor}\sffamily\fontsize{8.000000}{9.600000}\selectfont 6.15}%
\end{pgfscope}%
\begin{pgfscope}%
\definecolor{textcolor}{rgb}{0.196571,0.711827,0.479221}%
\pgfsetstrokecolor{textcolor}%
\pgfsetfillcolor{textcolor}%
\pgftext[x=6.615890in, y=4.988875in, left, base,rotate=66.381696]{\color{textcolor}\sffamily\fontsize{8.000000}{9.600000}\selectfont 6.15}%
\end{pgfscope}%
\begin{pgfscope}%
\definecolor{textcolor}{rgb}{0.232815,0.732247,0.459277}%
\pgfsetstrokecolor{textcolor}%
\pgfsetfillcolor{textcolor}%
\pgftext[x=1.328853in, y=5.504819in, left, base,rotate=32.719600]{\color{textcolor}\sffamily\fontsize{8.000000}{9.600000}\selectfont 6.30}%
\end{pgfscope}%
\begin{pgfscope}%
\definecolor{textcolor}{rgb}{0.232815,0.732247,0.459277}%
\pgfsetstrokecolor{textcolor}%
\pgfsetfillcolor{textcolor}%
\pgftext[x=6.468250in, y=4.319901in, left, base,rotate=73.766077]{\color{textcolor}\sffamily\fontsize{8.000000}{9.600000}\selectfont 6.30}%
\end{pgfscope}%
\begin{pgfscope}%
\definecolor{textcolor}{rgb}{0.266941,0.748751,0.440573}%
\pgfsetstrokecolor{textcolor}%
\pgfsetfillcolor{textcolor}%
\pgftext[x=6.634421in, y=4.641045in, left, base,rotate=68.679374]{\color{textcolor}\sffamily\fontsize{8.000000}{9.600000}\selectfont 6.45}%
\end{pgfscope}%
\begin{pgfscope}%
\definecolor{textcolor}{rgb}{0.266941,0.748751,0.440573}%
\pgfsetstrokecolor{textcolor}%
\pgfsetfillcolor{textcolor}%
\pgftext[x=0.953453in, y=5.262157in, left, base,rotate=38.088586]{\color{textcolor}\sffamily\fontsize{8.000000}{9.600000}\selectfont 6.45}%
\end{pgfscope}%
\begin{pgfscope}%
\definecolor{textcolor}{rgb}{0.311925,0.767822,0.415586}%
\pgfsetstrokecolor{textcolor}%
\pgfsetfillcolor{textcolor}%
\pgftext[x=6.533473in, y=3.936975in, left, base,rotate=80.021325]{\color{textcolor}\sffamily\fontsize{8.000000}{9.600000}\selectfont 6.60}%
\end{pgfscope}%
\begin{pgfscope}%
\definecolor{textcolor}{rgb}{0.311925,0.767822,0.415586}%
\pgfsetstrokecolor{textcolor}%
\pgfsetfillcolor{textcolor}%
\pgftext[x=0.952380in, y=5.313311in, left, base,rotate=37.473849]{\color{textcolor}\sffamily\fontsize{8.000000}{9.600000}\selectfont 6.60}%
\end{pgfscope}%
\begin{pgfscope}%
\definecolor{textcolor}{rgb}{0.352360,0.783011,0.392636}%
\pgfsetstrokecolor{textcolor}%
\pgfsetfillcolor{textcolor}%
\pgftext[x=6.524283in, y=3.368356in, left, base,rotate=277.928949]{\color{textcolor}\sffamily\fontsize{8.000000}{9.600000}\selectfont 6.75}%
\end{pgfscope}%
\begin{pgfscope}%
\definecolor{textcolor}{rgb}{0.352360,0.783011,0.392636}%
\pgfsetstrokecolor{textcolor}%
\pgfsetfillcolor{textcolor}%
\pgftext[x=0.951374in, y=5.362706in, left, base,rotate=36.893021]{\color{textcolor}\sffamily\fontsize{8.000000}{9.600000}\selectfont 6.75}%
\end{pgfscope}%
\begin{pgfscope}%
\definecolor{textcolor}{rgb}{0.404001,0.800275,0.362552}%
\pgfsetstrokecolor{textcolor}%
\pgfsetfillcolor{textcolor}%
\pgftext[x=6.642711in, y=3.467638in, left, base,rotate=89.798304]{\color{textcolor}\sffamily\fontsize{8.000000}{9.600000}\selectfont 6.90}%
\end{pgfscope}%
\begin{pgfscope}%
\definecolor{textcolor}{rgb}{0.404001,0.800275,0.362552}%
\pgfsetstrokecolor{textcolor}%
\pgfsetfillcolor{textcolor}%
\pgftext[x=0.950442in, y=5.410471in, left, base,rotate=36.349940]{\color{textcolor}\sffamily\fontsize{8.000000}{9.600000}\selectfont 6.90}%
\end{pgfscope}%
\begin{pgfscope}%
\definecolor{textcolor}{rgb}{0.449368,0.813768,0.335384}%
\pgfsetstrokecolor{textcolor}%
\pgfsetfillcolor{textcolor}%
\pgftext[x=6.708421in, y=3.753813in, left, base,rotate=83.103022]{\color{textcolor}\sffamily\fontsize{8.000000}{9.600000}\selectfont 7.05}%
\end{pgfscope}%
\begin{pgfscope}%
\definecolor{textcolor}{rgb}{0.449368,0.813768,0.335384}%
\pgfsetstrokecolor{textcolor}%
\pgfsetfillcolor{textcolor}%
\pgftext[x=0.958646in, y=5.463843in, left, base,rotate=35.689435]{\color{textcolor}\sffamily\fontsize{8.000000}{9.600000}\selectfont 7.05}%
\end{pgfscope}%
\begin{pgfscope}%
\definecolor{textcolor}{rgb}{0.496615,0.826376,0.306377}%
\pgfsetstrokecolor{textcolor}%
\pgfsetfillcolor{textcolor}%
\pgftext[x=0.969230in, y=5.517338in, left, base,rotate=35.016807]{\color{textcolor}\sffamily\fontsize{8.000000}{9.600000}\selectfont 7.20}%
\end{pgfscope}%
\begin{pgfscope}%
\definecolor{textcolor}{rgb}{0.555484,0.840254,0.269281}%
\pgfsetstrokecolor{textcolor}%
\pgfsetfillcolor{textcolor}%
\pgftext[x=0.947920in, y=5.544883in, left, base,rotate=34.859578]{\color{textcolor}\sffamily\fontsize{8.000000}{9.600000}\selectfont 7.35}%
\end{pgfscope}%
\begin{pgfscope}%
\definecolor{textcolor}{rgb}{0.606045,0.850733,0.236712}%
\pgfsetstrokecolor{textcolor}%
\pgfsetfillcolor{textcolor}%
\pgftext[x=0.947167in, y=5.587025in, left, base,rotate=34.407845]{\color{textcolor}\sffamily\fontsize{8.000000}{9.600000}\selectfont 7.50}%
\end{pgfscope}%
\end{pgfpicture}%
\makeatother%
\endgroup%
}
   \caption{Vrstevnicový graf funkcie $f(x,y)$.}
   \label{fig:graph_contour}
\end{figure}

\vspace*{\fill}

\newpage
\thispagestyle{plain}
\vspace*{\fill}
\begin{figure}[H]
   \centering
   \resizebox{1\textwidth}{!}{
   %% Creator: Matplotlib, PGF backend
%%
%% To include the figure in your LaTeX document, write
%%   \input{<filename>.pgf}
%%
%% Make sure the required packages are loaded in your preamble
%%   \usepackage{pgf}
%%
%% Also ensure that all the required font packages are loaded; for instance,
%% the lmodern package is sometimes necessary when using math font.
%%   \usepackage{lmodern}
%%
%% Figures using additional raster images can only be included by \input if
%% they are in the same directory as the main LaTeX file. For loading figures
%% from other directories you can use the `import` package
%%   \usepackage{import}
%%
%% and then include the figures with
%%   \import{<path to file>}{<filename>.pgf}
%%
%% Matplotlib used the following preamble
%%   
%%   \usepackage{fontspec}
%%   \setmainfont{DejaVuSerif.ttf}[Path=\detokenize{/home/radimek/Documents/projekt_mat_prog/mat_prog_kernel/lib/python3.12/site-packages/matplotlib/mpl-data/fonts/ttf/}]
%%   \setsansfont{DejaVuSans.ttf}[Path=\detokenize{/home/radimek/Documents/projekt_mat_prog/mat_prog_kernel/lib/python3.12/site-packages/matplotlib/mpl-data/fonts/ttf/}]
%%   \setmonofont{DejaVuSansMono.ttf}[Path=\detokenize{/home/radimek/Documents/projekt_mat_prog/mat_prog_kernel/lib/python3.12/site-packages/matplotlib/mpl-data/fonts/ttf/}]
%%   \makeatletter\@ifpackageloaded{underscore}{}{\usepackage[strings]{underscore}}\makeatother
%%
\begingroup%
\makeatletter%
\begin{pgfpicture}%
\pgfpathrectangle{\pgfpointorigin}{\pgfqpoint{8.000000in}{6.000000in}}%
\pgfusepath{use as bounding box, clip}%
\begin{pgfscope}%
\pgfsetbuttcap%
\pgfsetmiterjoin%
\definecolor{currentfill}{rgb}{1.000000,1.000000,1.000000}%
\pgfsetfillcolor{currentfill}%
\pgfsetlinewidth{0.000000pt}%
\definecolor{currentstroke}{rgb}{1.000000,1.000000,1.000000}%
\pgfsetstrokecolor{currentstroke}%
\pgfsetdash{}{0pt}%
\pgfpathmoveto{\pgfqpoint{0.000000in}{0.000000in}}%
\pgfpathlineto{\pgfqpoint{8.000000in}{0.000000in}}%
\pgfpathlineto{\pgfqpoint{8.000000in}{6.000000in}}%
\pgfpathlineto{\pgfqpoint{0.000000in}{6.000000in}}%
\pgfpathlineto{\pgfqpoint{0.000000in}{0.000000in}}%
\pgfpathclose%
\pgfusepath{fill}%
\end{pgfscope}%
\begin{pgfscope}%
\pgfsetbuttcap%
\pgfsetmiterjoin%
\definecolor{currentfill}{rgb}{1.000000,1.000000,1.000000}%
\pgfsetfillcolor{currentfill}%
\pgfsetlinewidth{0.000000pt}%
\definecolor{currentstroke}{rgb}{0.000000,0.000000,0.000000}%
\pgfsetstrokecolor{currentstroke}%
\pgfsetstrokeopacity{0.000000}%
\pgfsetdash{}{0pt}%
\pgfpathmoveto{\pgfqpoint{1.150000in}{0.150000in}}%
\pgfpathlineto{\pgfqpoint{6.850000in}{0.150000in}}%
\pgfpathlineto{\pgfqpoint{6.850000in}{5.850000in}}%
\pgfpathlineto{\pgfqpoint{1.150000in}{5.850000in}}%
\pgfpathlineto{\pgfqpoint{1.150000in}{0.150000in}}%
\pgfpathclose%
\pgfusepath{fill}%
\end{pgfscope}%
\begin{pgfscope}%
\pgfsetbuttcap%
\pgfsetmiterjoin%
\definecolor{currentfill}{rgb}{0.950000,0.950000,0.950000}%
\pgfsetfillcolor{currentfill}%
\pgfsetfillopacity{0.500000}%
\pgfsetlinewidth{1.003750pt}%
\definecolor{currentstroke}{rgb}{0.950000,0.950000,0.950000}%
\pgfsetstrokecolor{currentstroke}%
\pgfsetstrokeopacity{0.500000}%
\pgfsetdash{}{0pt}%
\pgfpathmoveto{\pgfqpoint{1.580389in}{1.555437in}}%
\pgfpathlineto{\pgfqpoint{3.462715in}{3.133240in}}%
\pgfpathlineto{\pgfqpoint{3.436549in}{5.408715in}}%
\pgfpathlineto{\pgfqpoint{1.464144in}{3.969343in}}%
\pgfusepath{stroke,fill}%
\end{pgfscope}%
\begin{pgfscope}%
\pgfsetbuttcap%
\pgfsetmiterjoin%
\definecolor{currentfill}{rgb}{0.900000,0.900000,0.900000}%
\pgfsetfillcolor{currentfill}%
\pgfsetfillopacity{0.500000}%
\pgfsetlinewidth{1.003750pt}%
\definecolor{currentstroke}{rgb}{0.900000,0.900000,0.900000}%
\pgfsetstrokecolor{currentstroke}%
\pgfsetstrokeopacity{0.500000}%
\pgfsetdash{}{0pt}%
\pgfpathmoveto{\pgfqpoint{3.462715in}{3.133240in}}%
\pgfpathlineto{\pgfqpoint{6.483177in}{2.255311in}}%
\pgfpathlineto{\pgfqpoint{6.590967in}{4.609162in}}%
\pgfpathlineto{\pgfqpoint{3.436549in}{5.408715in}}%
\pgfusepath{stroke,fill}%
\end{pgfscope}%
\begin{pgfscope}%
\pgfsetbuttcap%
\pgfsetmiterjoin%
\definecolor{currentfill}{rgb}{0.925000,0.925000,0.925000}%
\pgfsetfillcolor{currentfill}%
\pgfsetfillopacity{0.500000}%
\pgfsetlinewidth{1.003750pt}%
\definecolor{currentstroke}{rgb}{0.925000,0.925000,0.925000}%
\pgfsetstrokecolor{currentstroke}%
\pgfsetstrokeopacity{0.500000}%
\pgfsetdash{}{0pt}%
\pgfpathmoveto{\pgfqpoint{1.580389in}{1.555437in}}%
\pgfpathlineto{\pgfqpoint{4.782226in}{0.509717in}}%
\pgfpathlineto{\pgfqpoint{6.483177in}{2.255311in}}%
\pgfpathlineto{\pgfqpoint{3.462715in}{3.133240in}}%
\pgfusepath{stroke,fill}%
\end{pgfscope}%
\begin{pgfscope}%
\pgfsetrectcap%
\pgfsetroundjoin%
\pgfsetlinewidth{0.803000pt}%
\definecolor{currentstroke}{rgb}{0.000000,0.000000,0.000000}%
\pgfsetstrokecolor{currentstroke}%
\pgfsetdash{}{0pt}%
\pgfpathmoveto{\pgfqpoint{1.580389in}{1.555437in}}%
\pgfpathlineto{\pgfqpoint{4.782226in}{0.509717in}}%
\pgfusepath{stroke}%
\end{pgfscope}%
\begin{pgfscope}%
\definecolor{textcolor}{rgb}{0.000000,0.000000,0.000000}%
\pgfsetstrokecolor{textcolor}%
\pgfsetfillcolor{textcolor}%
\pgftext[x=2.913491in,y=0.557898in,,]{\color{textcolor}\sffamily\fontsize{10.000000}{12.000000}\selectfont x}%
\end{pgfscope}%
\begin{pgfscope}%
\pgfsetbuttcap%
\pgfsetroundjoin%
\pgfsetlinewidth{0.803000pt}%
\definecolor{currentstroke}{rgb}{0.690196,0.690196,0.690196}%
\pgfsetstrokecolor{currentstroke}%
\pgfsetdash{}{0pt}%
\pgfpathmoveto{\pgfqpoint{1.774309in}{1.492103in}}%
\pgfpathlineto{\pgfqpoint{3.646411in}{3.079847in}}%
\pgfpathlineto{\pgfqpoint{3.628011in}{5.360185in}}%
\pgfusepath{stroke}%
\end{pgfscope}%
\begin{pgfscope}%
\pgfsetbuttcap%
\pgfsetroundjoin%
\pgfsetlinewidth{0.803000pt}%
\definecolor{currentstroke}{rgb}{0.690196,0.690196,0.690196}%
\pgfsetstrokecolor{currentstroke}%
\pgfsetdash{}{0pt}%
\pgfpathmoveto{\pgfqpoint{2.222368in}{1.345767in}}%
\pgfpathlineto{\pgfqpoint{4.070468in}{2.956591in}}%
\pgfpathlineto{\pgfqpoint{4.070186in}{5.248106in}}%
\pgfusepath{stroke}%
\end{pgfscope}%
\begin{pgfscope}%
\pgfsetbuttcap%
\pgfsetroundjoin%
\pgfsetlinewidth{0.803000pt}%
\definecolor{currentstroke}{rgb}{0.690196,0.690196,0.690196}%
\pgfsetstrokecolor{currentstroke}%
\pgfsetdash{}{0pt}%
\pgfpathmoveto{\pgfqpoint{2.677247in}{1.197204in}}%
\pgfpathlineto{\pgfqpoint{4.500444in}{2.831614in}}%
\pgfpathlineto{\pgfqpoint{4.518800in}{5.134396in}}%
\pgfusepath{stroke}%
\end{pgfscope}%
\begin{pgfscope}%
\pgfsetbuttcap%
\pgfsetroundjoin%
\pgfsetlinewidth{0.803000pt}%
\definecolor{currentstroke}{rgb}{0.690196,0.690196,0.690196}%
\pgfsetstrokecolor{currentstroke}%
\pgfsetdash{}{0pt}%
\pgfpathmoveto{\pgfqpoint{3.139103in}{1.046362in}}%
\pgfpathlineto{\pgfqpoint{4.936464in}{2.704880in}}%
\pgfpathlineto{\pgfqpoint{4.973994in}{5.019017in}}%
\pgfusepath{stroke}%
\end{pgfscope}%
\begin{pgfscope}%
\pgfsetbuttcap%
\pgfsetroundjoin%
\pgfsetlinewidth{0.803000pt}%
\definecolor{currentstroke}{rgb}{0.690196,0.690196,0.690196}%
\pgfsetstrokecolor{currentstroke}%
\pgfsetdash{}{0pt}%
\pgfpathmoveto{\pgfqpoint{3.608098in}{0.893188in}}%
\pgfpathlineto{\pgfqpoint{5.378655in}{2.576352in}}%
\pgfpathlineto{\pgfqpoint{5.435914in}{4.901934in}}%
\pgfusepath{stroke}%
\end{pgfscope}%
\begin{pgfscope}%
\pgfsetbuttcap%
\pgfsetroundjoin%
\pgfsetlinewidth{0.803000pt}%
\definecolor{currentstroke}{rgb}{0.690196,0.690196,0.690196}%
\pgfsetstrokecolor{currentstroke}%
\pgfsetdash{}{0pt}%
\pgfpathmoveto{\pgfqpoint{4.084398in}{0.737628in}}%
\pgfpathlineto{\pgfqpoint{5.827149in}{2.445993in}}%
\pgfpathlineto{\pgfqpoint{5.904712in}{4.783107in}}%
\pgfusepath{stroke}%
\end{pgfscope}%
\begin{pgfscope}%
\pgfsetbuttcap%
\pgfsetroundjoin%
\pgfsetlinewidth{0.803000pt}%
\definecolor{currentstroke}{rgb}{0.690196,0.690196,0.690196}%
\pgfsetstrokecolor{currentstroke}%
\pgfsetdash{}{0pt}%
\pgfpathmoveto{\pgfqpoint{4.568177in}{0.579626in}}%
\pgfpathlineto{\pgfqpoint{6.282083in}{2.313762in}}%
\pgfpathlineto{\pgfqpoint{6.380540in}{4.662499in}}%
\pgfusepath{stroke}%
\end{pgfscope}%
\begin{pgfscope}%
\pgfsetrectcap%
\pgfsetroundjoin%
\pgfsetlinewidth{0.803000pt}%
\definecolor{currentstroke}{rgb}{0.000000,0.000000,0.000000}%
\pgfsetstrokecolor{currentstroke}%
\pgfsetdash{}{0pt}%
\pgfpathmoveto{\pgfqpoint{1.790612in}{1.505929in}}%
\pgfpathlineto{\pgfqpoint{1.741635in}{1.464392in}}%
\pgfusepath{stroke}%
\end{pgfscope}%
\begin{pgfscope}%
\definecolor{textcolor}{rgb}{0.000000,0.000000,0.000000}%
\pgfsetstrokecolor{textcolor}%
\pgfsetfillcolor{textcolor}%
\pgftext[x=1.669876in,y=1.274184in,,top]{\color{textcolor}\sffamily\fontsize{10.000000}{12.000000}\selectfont \ensuremath{-}1.0}%
\end{pgfscope}%
\begin{pgfscope}%
\pgfsetrectcap%
\pgfsetroundjoin%
\pgfsetlinewidth{0.803000pt}%
\definecolor{currentstroke}{rgb}{0.000000,0.000000,0.000000}%
\pgfsetstrokecolor{currentstroke}%
\pgfsetdash{}{0pt}%
\pgfpathmoveto{\pgfqpoint{2.238471in}{1.359803in}}%
\pgfpathlineto{\pgfqpoint{2.190092in}{1.317635in}}%
\pgfusepath{stroke}%
\end{pgfscope}%
\begin{pgfscope}%
\definecolor{textcolor}{rgb}{0.000000,0.000000,0.000000}%
\pgfsetstrokecolor{textcolor}%
\pgfsetfillcolor{textcolor}%
\pgftext[x=2.118230in,y=1.125843in,,top]{\color{textcolor}\sffamily\fontsize{10.000000}{12.000000}\selectfont \ensuremath{-}0.5}%
\end{pgfscope}%
\begin{pgfscope}%
\pgfsetrectcap%
\pgfsetroundjoin%
\pgfsetlinewidth{0.803000pt}%
\definecolor{currentstroke}{rgb}{0.000000,0.000000,0.000000}%
\pgfsetstrokecolor{currentstroke}%
\pgfsetdash{}{0pt}%
\pgfpathmoveto{\pgfqpoint{2.693143in}{1.211454in}}%
\pgfpathlineto{\pgfqpoint{2.645386in}{1.168642in}}%
\pgfusepath{stroke}%
\end{pgfscope}%
\begin{pgfscope}%
\definecolor{textcolor}{rgb}{0.000000,0.000000,0.000000}%
\pgfsetstrokecolor{textcolor}%
\pgfsetfillcolor{textcolor}%
\pgftext[x=2.573426in,y=0.975238in,,top]{\color{textcolor}\sffamily\fontsize{10.000000}{12.000000}\selectfont 0.0}%
\end{pgfscope}%
\begin{pgfscope}%
\pgfsetrectcap%
\pgfsetroundjoin%
\pgfsetlinewidth{0.803000pt}%
\definecolor{currentstroke}{rgb}{0.000000,0.000000,0.000000}%
\pgfsetstrokecolor{currentstroke}%
\pgfsetdash{}{0pt}%
\pgfpathmoveto{\pgfqpoint{3.154783in}{1.060831in}}%
\pgfpathlineto{\pgfqpoint{3.107673in}{1.017359in}}%
\pgfusepath{stroke}%
\end{pgfscope}%
\begin{pgfscope}%
\definecolor{textcolor}{rgb}{0.000000,0.000000,0.000000}%
\pgfsetstrokecolor{textcolor}%
\pgfsetfillcolor{textcolor}%
\pgftext[x=3.035622in,y=0.822318in,,top]{\color{textcolor}\sffamily\fontsize{10.000000}{12.000000}\selectfont 0.5}%
\end{pgfscope}%
\begin{pgfscope}%
\pgfsetrectcap%
\pgfsetroundjoin%
\pgfsetlinewidth{0.803000pt}%
\definecolor{currentstroke}{rgb}{0.000000,0.000000,0.000000}%
\pgfsetstrokecolor{currentstroke}%
\pgfsetdash{}{0pt}%
\pgfpathmoveto{\pgfqpoint{3.623554in}{0.907881in}}%
\pgfpathlineto{\pgfqpoint{3.577116in}{0.863735in}}%
\pgfusepath{stroke}%
\end{pgfscope}%
\begin{pgfscope}%
\definecolor{textcolor}{rgb}{0.000000,0.000000,0.000000}%
\pgfsetstrokecolor{textcolor}%
\pgfsetfillcolor{textcolor}%
\pgftext[x=3.504979in,y=0.667028in,,top]{\color{textcolor}\sffamily\fontsize{10.000000}{12.000000}\selectfont 1.0}%
\end{pgfscope}%
\begin{pgfscope}%
\pgfsetrectcap%
\pgfsetroundjoin%
\pgfsetlinewidth{0.803000pt}%
\definecolor{currentstroke}{rgb}{0.000000,0.000000,0.000000}%
\pgfsetstrokecolor{currentstroke}%
\pgfsetdash{}{0pt}%
\pgfpathmoveto{\pgfqpoint{4.099622in}{0.752551in}}%
\pgfpathlineto{\pgfqpoint{4.053883in}{0.707715in}}%
\pgfusepath{stroke}%
\end{pgfscope}%
\begin{pgfscope}%
\definecolor{textcolor}{rgb}{0.000000,0.000000,0.000000}%
\pgfsetstrokecolor{textcolor}%
\pgfsetfillcolor{textcolor}%
\pgftext[x=3.981665in,y=0.509314in,,top]{\color{textcolor}\sffamily\fontsize{10.000000}{12.000000}\selectfont 1.5}%
\end{pgfscope}%
\begin{pgfscope}%
\pgfsetrectcap%
\pgfsetroundjoin%
\pgfsetlinewidth{0.803000pt}%
\definecolor{currentstroke}{rgb}{0.000000,0.000000,0.000000}%
\pgfsetstrokecolor{currentstroke}%
\pgfsetdash{}{0pt}%
\pgfpathmoveto{\pgfqpoint{4.583158in}{0.594784in}}%
\pgfpathlineto{\pgfqpoint{4.538146in}{0.549241in}}%
\pgfusepath{stroke}%
\end{pgfscope}%
\begin{pgfscope}%
\definecolor{textcolor}{rgb}{0.000000,0.000000,0.000000}%
\pgfsetstrokecolor{textcolor}%
\pgfsetfillcolor{textcolor}%
\pgftext[x=4.465855in,y=0.349116in,,top]{\color{textcolor}\sffamily\fontsize{10.000000}{12.000000}\selectfont 2.0}%
\end{pgfscope}%
\begin{pgfscope}%
\pgfsetrectcap%
\pgfsetroundjoin%
\pgfsetlinewidth{0.803000pt}%
\definecolor{currentstroke}{rgb}{0.000000,0.000000,0.000000}%
\pgfsetstrokecolor{currentstroke}%
\pgfsetdash{}{0pt}%
\pgfpathmoveto{\pgfqpoint{6.483177in}{2.255311in}}%
\pgfpathlineto{\pgfqpoint{4.782226in}{0.509717in}}%
\pgfusepath{stroke}%
\end{pgfscope}%
\begin{pgfscope}%
\definecolor{textcolor}{rgb}{0.000000,0.000000,0.000000}%
\pgfsetstrokecolor{textcolor}%
\pgfsetfillcolor{textcolor}%
\pgftext[x=6.045209in,y=1.032725in,,]{\color{textcolor}\sffamily\fontsize{10.000000}{12.000000}\selectfont y}%
\end{pgfscope}%
\begin{pgfscope}%
\pgfsetbuttcap%
\pgfsetroundjoin%
\pgfsetlinewidth{0.803000pt}%
\definecolor{currentstroke}{rgb}{0.690196,0.690196,0.690196}%
\pgfsetstrokecolor{currentstroke}%
\pgfsetdash{}{0pt}%
\pgfpathmoveto{\pgfqpoint{1.600541in}{4.068879in}}%
\pgfpathlineto{\pgfqpoint{1.710097in}{1.664161in}}%
\pgfpathlineto{\pgfqpoint{4.899919in}{0.630499in}}%
\pgfusepath{stroke}%
\end{pgfscope}%
\begin{pgfscope}%
\pgfsetbuttcap%
\pgfsetroundjoin%
\pgfsetlinewidth{0.803000pt}%
\definecolor{currentstroke}{rgb}{0.690196,0.690196,0.690196}%
\pgfsetstrokecolor{currentstroke}%
\pgfsetdash{}{0pt}%
\pgfpathmoveto{\pgfqpoint{1.830662in}{4.236811in}}%
\pgfpathlineto{\pgfqpoint{1.929089in}{1.847724in}}%
\pgfpathlineto{\pgfqpoint{5.098461in}{0.834252in}}%
\pgfusepath{stroke}%
\end{pgfscope}%
\begin{pgfscope}%
\pgfsetbuttcap%
\pgfsetroundjoin%
\pgfsetlinewidth{0.803000pt}%
\definecolor{currentstroke}{rgb}{0.690196,0.690196,0.690196}%
\pgfsetstrokecolor{currentstroke}%
\pgfsetdash{}{0pt}%
\pgfpathmoveto{\pgfqpoint{2.056228in}{4.401419in}}%
\pgfpathlineto{\pgfqpoint{2.143933in}{2.027810in}}%
\pgfpathlineto{\pgfqpoint{5.293045in}{1.033943in}}%
\pgfusepath{stroke}%
\end{pgfscope}%
\begin{pgfscope}%
\pgfsetbuttcap%
\pgfsetroundjoin%
\pgfsetlinewidth{0.803000pt}%
\definecolor{currentstroke}{rgb}{0.690196,0.690196,0.690196}%
\pgfsetstrokecolor{currentstroke}%
\pgfsetdash{}{0pt}%
\pgfpathmoveto{\pgfqpoint{2.277372in}{4.562799in}}%
\pgfpathlineto{\pgfqpoint{2.354746in}{2.204518in}}%
\pgfpathlineto{\pgfqpoint{5.483787in}{1.229691in}}%
\pgfusepath{stroke}%
\end{pgfscope}%
\begin{pgfscope}%
\pgfsetbuttcap%
\pgfsetroundjoin%
\pgfsetlinewidth{0.803000pt}%
\definecolor{currentstroke}{rgb}{0.690196,0.690196,0.690196}%
\pgfsetstrokecolor{currentstroke}%
\pgfsetdash{}{0pt}%
\pgfpathmoveto{\pgfqpoint{2.494222in}{4.721047in}}%
\pgfpathlineto{\pgfqpoint{2.561641in}{2.377942in}}%
\pgfpathlineto{\pgfqpoint{5.670801in}{1.421614in}}%
\pgfusepath{stroke}%
\end{pgfscope}%
\begin{pgfscope}%
\pgfsetbuttcap%
\pgfsetroundjoin%
\pgfsetlinewidth{0.803000pt}%
\definecolor{currentstroke}{rgb}{0.690196,0.690196,0.690196}%
\pgfsetstrokecolor{currentstroke}%
\pgfsetdash{}{0pt}%
\pgfpathmoveto{\pgfqpoint{2.706903in}{4.876252in}}%
\pgfpathlineto{\pgfqpoint{2.764725in}{2.548171in}}%
\pgfpathlineto{\pgfqpoint{5.854194in}{1.609820in}}%
\pgfusepath{stroke}%
\end{pgfscope}%
\begin{pgfscope}%
\pgfsetbuttcap%
\pgfsetroundjoin%
\pgfsetlinewidth{0.803000pt}%
\definecolor{currentstroke}{rgb}{0.690196,0.690196,0.690196}%
\pgfsetstrokecolor{currentstroke}%
\pgfsetdash{}{0pt}%
\pgfpathmoveto{\pgfqpoint{2.915534in}{5.028501in}}%
\pgfpathlineto{\pgfqpoint{2.964104in}{2.715295in}}%
\pgfpathlineto{\pgfqpoint{6.034072in}{1.794419in}}%
\pgfusepath{stroke}%
\end{pgfscope}%
\begin{pgfscope}%
\pgfsetbuttcap%
\pgfsetroundjoin%
\pgfsetlinewidth{0.803000pt}%
\definecolor{currentstroke}{rgb}{0.690196,0.690196,0.690196}%
\pgfsetstrokecolor{currentstroke}%
\pgfsetdash{}{0pt}%
\pgfpathmoveto{\pgfqpoint{3.120229in}{5.177879in}}%
\pgfpathlineto{\pgfqpoint{3.159878in}{2.879396in}}%
\pgfpathlineto{\pgfqpoint{6.210533in}{1.975511in}}%
\pgfusepath{stroke}%
\end{pgfscope}%
\begin{pgfscope}%
\pgfsetbuttcap%
\pgfsetroundjoin%
\pgfsetlinewidth{0.803000pt}%
\definecolor{currentstroke}{rgb}{0.690196,0.690196,0.690196}%
\pgfsetstrokecolor{currentstroke}%
\pgfsetdash{}{0pt}%
\pgfpathmoveto{\pgfqpoint{3.321099in}{5.324464in}}%
\pgfpathlineto{\pgfqpoint{3.352144in}{3.040557in}}%
\pgfpathlineto{\pgfqpoint{6.383674in}{2.153197in}}%
\pgfusepath{stroke}%
\end{pgfscope}%
\begin{pgfscope}%
\pgfsetrectcap%
\pgfsetroundjoin%
\pgfsetlinewidth{0.803000pt}%
\definecolor{currentstroke}{rgb}{0.000000,0.000000,0.000000}%
\pgfsetstrokecolor{currentstroke}%
\pgfsetdash{}{0pt}%
\pgfpathmoveto{\pgfqpoint{4.873038in}{0.639210in}}%
\pgfpathlineto{\pgfqpoint{4.953750in}{0.613055in}}%
\pgfusepath{stroke}%
\end{pgfscope}%
\begin{pgfscope}%
\definecolor{textcolor}{rgb}{0.000000,0.000000,0.000000}%
\pgfsetstrokecolor{textcolor}%
\pgfsetfillcolor{textcolor}%
\pgftext[x=5.078779in,y=0.444104in,,top]{\color{textcolor}\sffamily\fontsize{10.000000}{12.000000}\selectfont \ensuremath{-}1.00}%
\end{pgfscope}%
\begin{pgfscope}%
\pgfsetrectcap%
\pgfsetroundjoin%
\pgfsetlinewidth{0.803000pt}%
\definecolor{currentstroke}{rgb}{0.000000,0.000000,0.000000}%
\pgfsetstrokecolor{currentstroke}%
\pgfsetdash{}{0pt}%
\pgfpathmoveto{\pgfqpoint{5.071766in}{0.842788in}}%
\pgfpathlineto{\pgfqpoint{5.151919in}{0.817158in}}%
\pgfusepath{stroke}%
\end{pgfscope}%
\begin{pgfscope}%
\definecolor{textcolor}{rgb}{0.000000,0.000000,0.000000}%
\pgfsetstrokecolor{textcolor}%
\pgfsetfillcolor{textcolor}%
\pgftext[x=5.275596in,y=0.649813in,,top]{\color{textcolor}\sffamily\fontsize{10.000000}{12.000000}\selectfont \ensuremath{-}0.75}%
\end{pgfscope}%
\begin{pgfscope}%
\pgfsetrectcap%
\pgfsetroundjoin%
\pgfsetlinewidth{0.803000pt}%
\definecolor{currentstroke}{rgb}{0.000000,0.000000,0.000000}%
\pgfsetstrokecolor{currentstroke}%
\pgfsetdash{}{0pt}%
\pgfpathmoveto{\pgfqpoint{5.266534in}{1.042310in}}%
\pgfpathlineto{\pgfqpoint{5.346134in}{1.017188in}}%
\pgfusepath{stroke}%
\end{pgfscope}%
\begin{pgfscope}%
\definecolor{textcolor}{rgb}{0.000000,0.000000,0.000000}%
\pgfsetstrokecolor{textcolor}%
\pgfsetfillcolor{textcolor}%
\pgftext[x=5.468487in,y=0.851419in,,top]{\color{textcolor}\sffamily\fontsize{10.000000}{12.000000}\selectfont \ensuremath{-}0.50}%
\end{pgfscope}%
\begin{pgfscope}%
\pgfsetrectcap%
\pgfsetroundjoin%
\pgfsetlinewidth{0.803000pt}%
\definecolor{currentstroke}{rgb}{0.000000,0.000000,0.000000}%
\pgfsetstrokecolor{currentstroke}%
\pgfsetdash{}{0pt}%
\pgfpathmoveto{\pgfqpoint{5.457458in}{1.237894in}}%
\pgfpathlineto{\pgfqpoint{5.536511in}{1.213266in}}%
\pgfusepath{stroke}%
\end{pgfscope}%
\begin{pgfscope}%
\definecolor{textcolor}{rgb}{0.000000,0.000000,0.000000}%
\pgfsetstrokecolor{textcolor}%
\pgfsetfillcolor{textcolor}%
\pgftext[x=5.657569in,y=1.049043in,,top]{\color{textcolor}\sffamily\fontsize{10.000000}{12.000000}\selectfont \ensuremath{-}0.25}%
\end{pgfscope}%
\begin{pgfscope}%
\pgfsetrectcap%
\pgfsetroundjoin%
\pgfsetlinewidth{0.803000pt}%
\definecolor{currentstroke}{rgb}{0.000000,0.000000,0.000000}%
\pgfsetstrokecolor{currentstroke}%
\pgfsetdash{}{0pt}%
\pgfpathmoveto{\pgfqpoint{5.644652in}{1.429657in}}%
\pgfpathlineto{\pgfqpoint{5.723164in}{1.405507in}}%
\pgfusepath{stroke}%
\end{pgfscope}%
\begin{pgfscope}%
\definecolor{textcolor}{rgb}{0.000000,0.000000,0.000000}%
\pgfsetstrokecolor{textcolor}%
\pgfsetfillcolor{textcolor}%
\pgftext[x=5.842953in,y=1.242803in,,top]{\color{textcolor}\sffamily\fontsize{10.000000}{12.000000}\selectfont 0.00}%
\end{pgfscope}%
\begin{pgfscope}%
\pgfsetrectcap%
\pgfsetroundjoin%
\pgfsetlinewidth{0.803000pt}%
\definecolor{currentstroke}{rgb}{0.000000,0.000000,0.000000}%
\pgfsetstrokecolor{currentstroke}%
\pgfsetdash{}{0pt}%
\pgfpathmoveto{\pgfqpoint{5.828223in}{1.617708in}}%
\pgfpathlineto{\pgfqpoint{5.906201in}{1.594025in}}%
\pgfusepath{stroke}%
\end{pgfscope}%
\begin{pgfscope}%
\definecolor{textcolor}{rgb}{0.000000,0.000000,0.000000}%
\pgfsetstrokecolor{textcolor}%
\pgfsetfillcolor{textcolor}%
\pgftext[x=6.024748in,y=1.432811in,,top]{\color{textcolor}\sffamily\fontsize{10.000000}{12.000000}\selectfont 0.25}%
\end{pgfscope}%
\begin{pgfscope}%
\pgfsetrectcap%
\pgfsetroundjoin%
\pgfsetlinewidth{0.803000pt}%
\definecolor{currentstroke}{rgb}{0.000000,0.000000,0.000000}%
\pgfsetstrokecolor{currentstroke}%
\pgfsetdash{}{0pt}%
\pgfpathmoveto{\pgfqpoint{6.008276in}{1.802156in}}%
\pgfpathlineto{\pgfqpoint{6.085725in}{1.778924in}}%
\pgfusepath{stroke}%
\end{pgfscope}%
\begin{pgfscope}%
\definecolor{textcolor}{rgb}{0.000000,0.000000,0.000000}%
\pgfsetstrokecolor{textcolor}%
\pgfsetfillcolor{textcolor}%
\pgftext[x=6.203055in,y=1.619174in,,top]{\color{textcolor}\sffamily\fontsize{10.000000}{12.000000}\selectfont 0.50}%
\end{pgfscope}%
\begin{pgfscope}%
\pgfsetrectcap%
\pgfsetroundjoin%
\pgfsetlinewidth{0.803000pt}%
\definecolor{currentstroke}{rgb}{0.000000,0.000000,0.000000}%
\pgfsetstrokecolor{currentstroke}%
\pgfsetdash{}{0pt}%
\pgfpathmoveto{\pgfqpoint{6.184911in}{1.983102in}}%
\pgfpathlineto{\pgfqpoint{6.261837in}{1.960310in}}%
\pgfusepath{stroke}%
\end{pgfscope}%
\begin{pgfscope}%
\definecolor{textcolor}{rgb}{0.000000,0.000000,0.000000}%
\pgfsetstrokecolor{textcolor}%
\pgfsetfillcolor{textcolor}%
\pgftext[x=6.377975in,y=1.801997in,,top]{\color{textcolor}\sffamily\fontsize{10.000000}{12.000000}\selectfont 0.75}%
\end{pgfscope}%
\begin{pgfscope}%
\pgfsetrectcap%
\pgfsetroundjoin%
\pgfsetlinewidth{0.803000pt}%
\definecolor{currentstroke}{rgb}{0.000000,0.000000,0.000000}%
\pgfsetstrokecolor{currentstroke}%
\pgfsetdash{}{0pt}%
\pgfpathmoveto{\pgfqpoint{6.358225in}{2.160646in}}%
\pgfpathlineto{\pgfqpoint{6.434634in}{2.138280in}}%
\pgfusepath{stroke}%
\end{pgfscope}%
\begin{pgfscope}%
\definecolor{textcolor}{rgb}{0.000000,0.000000,0.000000}%
\pgfsetstrokecolor{textcolor}%
\pgfsetfillcolor{textcolor}%
\pgftext[x=6.549603in,y=1.981379in,,top]{\color{textcolor}\sffamily\fontsize{10.000000}{12.000000}\selectfont 1.00}%
\end{pgfscope}%
\begin{pgfscope}%
\pgfsetrectcap%
\pgfsetroundjoin%
\pgfsetlinewidth{0.803000pt}%
\definecolor{currentstroke}{rgb}{0.000000,0.000000,0.000000}%
\pgfsetstrokecolor{currentstroke}%
\pgfsetdash{}{0pt}%
\pgfpathmoveto{\pgfqpoint{6.483177in}{2.255311in}}%
\pgfpathlineto{\pgfqpoint{6.590967in}{4.609162in}}%
\pgfusepath{stroke}%
\end{pgfscope}%
\begin{pgfscope}%
\definecolor{textcolor}{rgb}{0.000000,0.000000,0.000000}%
\pgfsetstrokecolor{textcolor}%
\pgfsetfillcolor{textcolor}%
\pgftext[x=7.097978in,y=3.481758in,,,rotate=87.378092]{\color{textcolor}\sffamily\fontsize{10.000000}{12.000000}\selectfont f(x, y)}%
\end{pgfscope}%
\begin{pgfscope}%
\pgfsetbuttcap%
\pgfsetroundjoin%
\pgfsetlinewidth{0.803000pt}%
\definecolor{currentstroke}{rgb}{0.690196,0.690196,0.690196}%
\pgfsetstrokecolor{currentstroke}%
\pgfsetdash{}{0pt}%
\pgfpathmoveto{\pgfqpoint{6.491485in}{2.436728in}}%
\pgfpathlineto{\pgfqpoint{3.460695in}{3.308959in}}%
\pgfpathlineto{\pgfqpoint{1.571444in}{1.741193in}}%
\pgfusepath{stroke}%
\end{pgfscope}%
\begin{pgfscope}%
\pgfsetbuttcap%
\pgfsetroundjoin%
\pgfsetlinewidth{0.803000pt}%
\definecolor{currentstroke}{rgb}{0.690196,0.690196,0.690196}%
\pgfsetstrokecolor{currentstroke}%
\pgfsetdash{}{0pt}%
\pgfpathmoveto{\pgfqpoint{6.506094in}{2.755756in}}%
\pgfpathlineto{\pgfqpoint{3.457143in}{3.617827in}}%
\pgfpathlineto{\pgfqpoint{1.555708in}{2.067969in}}%
\pgfusepath{stroke}%
\end{pgfscope}%
\begin{pgfscope}%
\pgfsetbuttcap%
\pgfsetroundjoin%
\pgfsetlinewidth{0.803000pt}%
\definecolor{currentstroke}{rgb}{0.690196,0.690196,0.690196}%
\pgfsetstrokecolor{currentstroke}%
\pgfsetdash{}{0pt}%
\pgfpathmoveto{\pgfqpoint{6.520881in}{3.078669in}}%
\pgfpathlineto{\pgfqpoint{3.453550in}{3.930275in}}%
\pgfpathlineto{\pgfqpoint{1.539772in}{2.398876in}}%
\pgfusepath{stroke}%
\end{pgfscope}%
\begin{pgfscope}%
\pgfsetbuttcap%
\pgfsetroundjoin%
\pgfsetlinewidth{0.803000pt}%
\definecolor{currentstroke}{rgb}{0.690196,0.690196,0.690196}%
\pgfsetstrokecolor{currentstroke}%
\pgfsetdash{}{0pt}%
\pgfpathmoveto{\pgfqpoint{6.535849in}{3.405537in}}%
\pgfpathlineto{\pgfqpoint{3.449915in}{4.246367in}}%
\pgfpathlineto{\pgfqpoint{1.523634in}{2.733992in}}%
\pgfusepath{stroke}%
\end{pgfscope}%
\begin{pgfscope}%
\pgfsetbuttcap%
\pgfsetroundjoin%
\pgfsetlinewidth{0.803000pt}%
\definecolor{currentstroke}{rgb}{0.690196,0.690196,0.690196}%
\pgfsetstrokecolor{currentstroke}%
\pgfsetdash{}{0pt}%
\pgfpathmoveto{\pgfqpoint{6.551002in}{3.736434in}}%
\pgfpathlineto{\pgfqpoint{3.446238in}{4.566166in}}%
\pgfpathlineto{\pgfqpoint{1.507290in}{3.073399in}}%
\pgfusepath{stroke}%
\end{pgfscope}%
\begin{pgfscope}%
\pgfsetbuttcap%
\pgfsetroundjoin%
\pgfsetlinewidth{0.803000pt}%
\definecolor{currentstroke}{rgb}{0.690196,0.690196,0.690196}%
\pgfsetstrokecolor{currentstroke}%
\pgfsetdash{}{0pt}%
\pgfpathmoveto{\pgfqpoint{6.566343in}{4.071434in}}%
\pgfpathlineto{\pgfqpoint{3.442517in}{4.889738in}}%
\pgfpathlineto{\pgfqpoint{1.490734in}{3.417178in}}%
\pgfusepath{stroke}%
\end{pgfscope}%
\begin{pgfscope}%
\pgfsetbuttcap%
\pgfsetroundjoin%
\pgfsetlinewidth{0.803000pt}%
\definecolor{currentstroke}{rgb}{0.690196,0.690196,0.690196}%
\pgfsetstrokecolor{currentstroke}%
\pgfsetdash{}{0pt}%
\pgfpathmoveto{\pgfqpoint{6.581875in}{4.410615in}}%
\pgfpathlineto{\pgfqpoint{3.438752in}{5.217149in}}%
\pgfpathlineto{\pgfqpoint{1.473965in}{3.765416in}}%
\pgfusepath{stroke}%
\end{pgfscope}%
\begin{pgfscope}%
\pgfsetrectcap%
\pgfsetroundjoin%
\pgfsetlinewidth{0.803000pt}%
\definecolor{currentstroke}{rgb}{0.000000,0.000000,0.000000}%
\pgfsetstrokecolor{currentstroke}%
\pgfsetdash{}{0pt}%
\pgfpathmoveto{\pgfqpoint{6.466044in}{2.444050in}}%
\pgfpathlineto{\pgfqpoint{6.542427in}{2.422067in}}%
\pgfusepath{stroke}%
\end{pgfscope}%
\begin{pgfscope}%
\definecolor{textcolor}{rgb}{0.000000,0.000000,0.000000}%
\pgfsetstrokecolor{textcolor}%
\pgfsetfillcolor{textcolor}%
\pgftext[x=6.746064in,y=2.472717in,,top]{\color{textcolor}\sffamily\fontsize{10.000000}{12.000000}\selectfont 2}%
\end{pgfscope}%
\begin{pgfscope}%
\pgfsetrectcap%
\pgfsetroundjoin%
\pgfsetlinewidth{0.803000pt}%
\definecolor{currentstroke}{rgb}{0.000000,0.000000,0.000000}%
\pgfsetstrokecolor{currentstroke}%
\pgfsetdash{}{0pt}%
\pgfpathmoveto{\pgfqpoint{6.480494in}{2.762995in}}%
\pgfpathlineto{\pgfqpoint{6.557357in}{2.741262in}}%
\pgfusepath{stroke}%
\end{pgfscope}%
\begin{pgfscope}%
\definecolor{textcolor}{rgb}{0.000000,0.000000,0.000000}%
\pgfsetstrokecolor{textcolor}%
\pgfsetfillcolor{textcolor}%
\pgftext[x=6.762182in,y=2.791336in,,top]{\color{textcolor}\sffamily\fontsize{10.000000}{12.000000}\selectfont 3}%
\end{pgfscope}%
\begin{pgfscope}%
\pgfsetrectcap%
\pgfsetroundjoin%
\pgfsetlinewidth{0.803000pt}%
\definecolor{currentstroke}{rgb}{0.000000,0.000000,0.000000}%
\pgfsetstrokecolor{currentstroke}%
\pgfsetdash{}{0pt}%
\pgfpathmoveto{\pgfqpoint{6.495119in}{3.085821in}}%
\pgfpathlineto{\pgfqpoint{6.572468in}{3.064346in}}%
\pgfusepath{stroke}%
\end{pgfscope}%
\begin{pgfscope}%
\definecolor{textcolor}{rgb}{0.000000,0.000000,0.000000}%
\pgfsetstrokecolor{textcolor}%
\pgfsetfillcolor{textcolor}%
\pgftext[x=6.778495in,y=3.113826in,,top]{\color{textcolor}\sffamily\fontsize{10.000000}{12.000000}\selectfont 4}%
\end{pgfscope}%
\begin{pgfscope}%
\pgfsetrectcap%
\pgfsetroundjoin%
\pgfsetlinewidth{0.803000pt}%
\definecolor{currentstroke}{rgb}{0.000000,0.000000,0.000000}%
\pgfsetstrokecolor{currentstroke}%
\pgfsetdash{}{0pt}%
\pgfpathmoveto{\pgfqpoint{6.509923in}{3.412601in}}%
\pgfpathlineto{\pgfqpoint{6.587765in}{3.391391in}}%
\pgfusepath{stroke}%
\end{pgfscope}%
\begin{pgfscope}%
\definecolor{textcolor}{rgb}{0.000000,0.000000,0.000000}%
\pgfsetstrokecolor{textcolor}%
\pgfsetfillcolor{textcolor}%
\pgftext[x=6.795009in,y=3.440259in,,top]{\color{textcolor}\sffamily\fontsize{10.000000}{12.000000}\selectfont 5}%
\end{pgfscope}%
\begin{pgfscope}%
\pgfsetrectcap%
\pgfsetroundjoin%
\pgfsetlinewidth{0.803000pt}%
\definecolor{currentstroke}{rgb}{0.000000,0.000000,0.000000}%
\pgfsetstrokecolor{currentstroke}%
\pgfsetdash{}{0pt}%
\pgfpathmoveto{\pgfqpoint{6.524910in}{3.743407in}}%
\pgfpathlineto{\pgfqpoint{6.603250in}{3.722471in}}%
\pgfusepath{stroke}%
\end{pgfscope}%
\begin{pgfscope}%
\definecolor{textcolor}{rgb}{0.000000,0.000000,0.000000}%
\pgfsetstrokecolor{textcolor}%
\pgfsetfillcolor{textcolor}%
\pgftext[x=6.811725in,y=3.770708in,,top]{\color{textcolor}\sffamily\fontsize{10.000000}{12.000000}\selectfont 6}%
\end{pgfscope}%
\begin{pgfscope}%
\pgfsetrectcap%
\pgfsetroundjoin%
\pgfsetlinewidth{0.803000pt}%
\definecolor{currentstroke}{rgb}{0.000000,0.000000,0.000000}%
\pgfsetstrokecolor{currentstroke}%
\pgfsetdash{}{0pt}%
\pgfpathmoveto{\pgfqpoint{6.540083in}{4.078313in}}%
\pgfpathlineto{\pgfqpoint{6.618927in}{4.057659in}}%
\pgfusepath{stroke}%
\end{pgfscope}%
\begin{pgfscope}%
\definecolor{textcolor}{rgb}{0.000000,0.000000,0.000000}%
\pgfsetstrokecolor{textcolor}%
\pgfsetfillcolor{textcolor}%
\pgftext[x=6.828648in,y=4.105246in,,top]{\color{textcolor}\sffamily\fontsize{10.000000}{12.000000}\selectfont 7}%
\end{pgfscope}%
\begin{pgfscope}%
\pgfsetrectcap%
\pgfsetroundjoin%
\pgfsetlinewidth{0.803000pt}%
\definecolor{currentstroke}{rgb}{0.000000,0.000000,0.000000}%
\pgfsetstrokecolor{currentstroke}%
\pgfsetdash{}{0pt}%
\pgfpathmoveto{\pgfqpoint{6.555445in}{4.417397in}}%
\pgfpathlineto{\pgfqpoint{6.634801in}{4.397034in}}%
\pgfusepath{stroke}%
\end{pgfscope}%
\begin{pgfscope}%
\definecolor{textcolor}{rgb}{0.000000,0.000000,0.000000}%
\pgfsetstrokecolor{textcolor}%
\pgfsetfillcolor{textcolor}%
\pgftext[x=6.845782in,y=4.443950in,,top]{\color{textcolor}\sffamily\fontsize{10.000000}{12.000000}\selectfont 8}%
\end{pgfscope}%
\begin{pgfscope}%
\pgfpathrectangle{\pgfqpoint{1.150000in}{0.150000in}}{\pgfqpoint{5.700000in}{5.700000in}}%
\pgfusepath{clip}%
\pgfsetbuttcap%
\pgfsetroundjoin%
\definecolor{currentfill}{rgb}{0.136408,0.541173,0.554483}%
\pgfsetfillcolor{currentfill}%
\pgfsetfillopacity{0.800000}%
\pgfsetlinewidth{0.000000pt}%
\definecolor{currentstroke}{rgb}{0.000000,0.000000,0.000000}%
\pgfsetstrokecolor{currentstroke}%
\pgfsetdash{}{0pt}%
\pgfpathmoveto{\pgfqpoint{4.071654in}{3.739275in}}%
\pgfpathlineto{\pgfqpoint{4.084938in}{3.723813in}}%
\pgfpathlineto{\pgfqpoint{4.098221in}{3.708593in}}%
\pgfpathlineto{\pgfqpoint{4.111505in}{3.693611in}}%
\pgfpathlineto{\pgfqpoint{4.124790in}{3.678867in}}%
\pgfpathlineto{\pgfqpoint{4.132459in}{3.704914in}}%
\pgfpathlineto{\pgfqpoint{4.140128in}{3.731384in}}%
\pgfpathlineto{\pgfqpoint{4.147798in}{3.758285in}}%
\pgfpathlineto{\pgfqpoint{4.155469in}{3.785625in}}%
\pgfpathlineto{\pgfqpoint{4.142185in}{3.801078in}}%
\pgfpathlineto{\pgfqpoint{4.128901in}{3.816769in}}%
\pgfpathlineto{\pgfqpoint{4.115617in}{3.832700in}}%
\pgfpathlineto{\pgfqpoint{4.102333in}{3.848873in}}%
\pgfpathlineto{\pgfqpoint{4.094663in}{3.820809in}}%
\pgfpathlineto{\pgfqpoint{4.086993in}{3.793193in}}%
\pgfpathlineto{\pgfqpoint{4.079323in}{3.766018in}}%
\pgfpathlineto{\pgfqpoint{4.071654in}{3.739275in}}%
\pgfpathclose%
\pgfusepath{fill}%
\end{pgfscope}%
\begin{pgfscope}%
\pgfpathrectangle{\pgfqpoint{1.150000in}{0.150000in}}{\pgfqpoint{5.700000in}{5.700000in}}%
\pgfusepath{clip}%
\pgfsetbuttcap%
\pgfsetroundjoin%
\definecolor{currentfill}{rgb}{0.141935,0.526453,0.555991}%
\pgfsetfillcolor{currentfill}%
\pgfsetfillopacity{0.800000}%
\pgfsetlinewidth{0.000000pt}%
\definecolor{currentstroke}{rgb}{0.000000,0.000000,0.000000}%
\pgfsetstrokecolor{currentstroke}%
\pgfsetdash{}{0pt}%
\pgfpathmoveto{\pgfqpoint{3.987835in}{3.698031in}}%
\pgfpathlineto{\pgfqpoint{4.001120in}{3.682267in}}%
\pgfpathlineto{\pgfqpoint{4.014405in}{3.666750in}}%
\pgfpathlineto{\pgfqpoint{4.027689in}{3.651479in}}%
\pgfpathlineto{\pgfqpoint{4.040974in}{3.636451in}}%
\pgfpathlineto{\pgfqpoint{4.048644in}{3.661551in}}%
\pgfpathlineto{\pgfqpoint{4.056315in}{3.687050in}}%
\pgfpathlineto{\pgfqpoint{4.063984in}{3.712955in}}%
\pgfpathlineto{\pgfqpoint{4.071654in}{3.739275in}}%
\pgfpathlineto{\pgfqpoint{4.058370in}{3.754979in}}%
\pgfpathlineto{\pgfqpoint{4.045086in}{3.770928in}}%
\pgfpathlineto{\pgfqpoint{4.031801in}{3.787123in}}%
\pgfpathlineto{\pgfqpoint{4.018515in}{3.803565in}}%
\pgfpathlineto{\pgfqpoint{4.010846in}{3.776554in}}%
\pgfpathlineto{\pgfqpoint{4.003176in}{3.749966in}}%
\pgfpathlineto{\pgfqpoint{3.995506in}{3.723795in}}%
\pgfpathlineto{\pgfqpoint{3.987835in}{3.698031in}}%
\pgfpathclose%
\pgfusepath{fill}%
\end{pgfscope}%
\begin{pgfscope}%
\pgfpathrectangle{\pgfqpoint{1.150000in}{0.150000in}}{\pgfqpoint{5.700000in}{5.700000in}}%
\pgfusepath{clip}%
\pgfsetbuttcap%
\pgfsetroundjoin%
\definecolor{currentfill}{rgb}{0.144759,0.519093,0.556572}%
\pgfsetfillcolor{currentfill}%
\pgfsetfillopacity{0.800000}%
\pgfsetlinewidth{0.000000pt}%
\definecolor{currentstroke}{rgb}{0.000000,0.000000,0.000000}%
\pgfsetstrokecolor{currentstroke}%
\pgfsetdash{}{0pt}%
\pgfpathmoveto{\pgfqpoint{4.124790in}{3.678867in}}%
\pgfpathlineto{\pgfqpoint{4.138075in}{3.664360in}}%
\pgfpathlineto{\pgfqpoint{4.151360in}{3.650087in}}%
\pgfpathlineto{\pgfqpoint{4.164647in}{3.636047in}}%
\pgfpathlineto{\pgfqpoint{4.177935in}{3.622239in}}%
\pgfpathlineto{\pgfqpoint{4.185603in}{3.647593in}}%
\pgfpathlineto{\pgfqpoint{4.193271in}{3.673361in}}%
\pgfpathlineto{\pgfqpoint{4.200940in}{3.699550in}}%
\pgfpathlineto{\pgfqpoint{4.208610in}{3.726171in}}%
\pgfpathlineto{\pgfqpoint{4.195324in}{3.740684in}}%
\pgfpathlineto{\pgfqpoint{4.182038in}{3.755430in}}%
\pgfpathlineto{\pgfqpoint{4.168753in}{3.770410in}}%
\pgfpathlineto{\pgfqpoint{4.155469in}{3.785625in}}%
\pgfpathlineto{\pgfqpoint{4.147798in}{3.758285in}}%
\pgfpathlineto{\pgfqpoint{4.140128in}{3.731384in}}%
\pgfpathlineto{\pgfqpoint{4.132459in}{3.704914in}}%
\pgfpathlineto{\pgfqpoint{4.124790in}{3.678867in}}%
\pgfpathclose%
\pgfusepath{fill}%
\end{pgfscope}%
\begin{pgfscope}%
\pgfpathrectangle{\pgfqpoint{1.150000in}{0.150000in}}{\pgfqpoint{5.700000in}{5.700000in}}%
\pgfusepath{clip}%
\pgfsetbuttcap%
\pgfsetroundjoin%
\definecolor{currentfill}{rgb}{0.150476,0.504369,0.557430}%
\pgfsetfillcolor{currentfill}%
\pgfsetfillopacity{0.800000}%
\pgfsetlinewidth{0.000000pt}%
\definecolor{currentstroke}{rgb}{0.000000,0.000000,0.000000}%
\pgfsetstrokecolor{currentstroke}%
\pgfsetdash{}{0pt}%
\pgfpathmoveto{\pgfqpoint{4.040974in}{3.636451in}}%
\pgfpathlineto{\pgfqpoint{4.054258in}{3.621664in}}%
\pgfpathlineto{\pgfqpoint{4.067543in}{3.607118in}}%
\pgfpathlineto{\pgfqpoint{4.080828in}{3.592811in}}%
\pgfpathlineto{\pgfqpoint{4.094114in}{3.578740in}}%
\pgfpathlineto{\pgfqpoint{4.101783in}{3.603179in}}%
\pgfpathlineto{\pgfqpoint{4.109452in}{3.628008in}}%
\pgfpathlineto{\pgfqpoint{4.117121in}{3.653235in}}%
\pgfpathlineto{\pgfqpoint{4.124790in}{3.678867in}}%
\pgfpathlineto{\pgfqpoint{4.111505in}{3.693611in}}%
\pgfpathlineto{\pgfqpoint{4.098221in}{3.708593in}}%
\pgfpathlineto{\pgfqpoint{4.084938in}{3.723813in}}%
\pgfpathlineto{\pgfqpoint{4.071654in}{3.739275in}}%
\pgfpathlineto{\pgfqpoint{4.063984in}{3.712955in}}%
\pgfpathlineto{\pgfqpoint{4.056315in}{3.687050in}}%
\pgfpathlineto{\pgfqpoint{4.048644in}{3.661551in}}%
\pgfpathlineto{\pgfqpoint{4.040974in}{3.636451in}}%
\pgfpathclose%
\pgfusepath{fill}%
\end{pgfscope}%
\begin{pgfscope}%
\pgfpathrectangle{\pgfqpoint{1.150000in}{0.150000in}}{\pgfqpoint{5.700000in}{5.700000in}}%
\pgfusepath{clip}%
\pgfsetbuttcap%
\pgfsetroundjoin%
\definecolor{currentfill}{rgb}{0.128729,0.563265,0.551229}%
\pgfsetfillcolor{currentfill}%
\pgfsetfillopacity{0.800000}%
\pgfsetlinewidth{0.000000pt}%
\definecolor{currentstroke}{rgb}{0.000000,0.000000,0.000000}%
\pgfsetstrokecolor{currentstroke}%
\pgfsetdash{}{0pt}%
\pgfpathmoveto{\pgfqpoint{4.018515in}{3.803565in}}%
\pgfpathlineto{\pgfqpoint{4.031801in}{3.787123in}}%
\pgfpathlineto{\pgfqpoint{4.045086in}{3.770928in}}%
\pgfpathlineto{\pgfqpoint{4.058370in}{3.754979in}}%
\pgfpathlineto{\pgfqpoint{4.071654in}{3.739275in}}%
\pgfpathlineto{\pgfqpoint{4.079323in}{3.766018in}}%
\pgfpathlineto{\pgfqpoint{4.086993in}{3.793193in}}%
\pgfpathlineto{\pgfqpoint{4.094663in}{3.820809in}}%
\pgfpathlineto{\pgfqpoint{4.102333in}{3.848873in}}%
\pgfpathlineto{\pgfqpoint{4.089049in}{3.865290in}}%
\pgfpathlineto{\pgfqpoint{4.075764in}{3.881951in}}%
\pgfpathlineto{\pgfqpoint{4.062478in}{3.898860in}}%
\pgfpathlineto{\pgfqpoint{4.049191in}{3.916018in}}%
\pgfpathlineto{\pgfqpoint{4.041522in}{3.887226in}}%
\pgfpathlineto{\pgfqpoint{4.033853in}{3.858892in}}%
\pgfpathlineto{\pgfqpoint{4.026184in}{3.831008in}}%
\pgfpathlineto{\pgfqpoint{4.018515in}{3.803565in}}%
\pgfpathclose%
\pgfusepath{fill}%
\end{pgfscope}%
\begin{pgfscope}%
\pgfpathrectangle{\pgfqpoint{1.150000in}{0.150000in}}{\pgfqpoint{5.700000in}{5.700000in}}%
\pgfusepath{clip}%
\pgfsetbuttcap%
\pgfsetroundjoin%
\definecolor{currentfill}{rgb}{0.131172,0.555899,0.552459}%
\pgfsetfillcolor{currentfill}%
\pgfsetfillopacity{0.800000}%
\pgfsetlinewidth{0.000000pt}%
\definecolor{currentstroke}{rgb}{0.000000,0.000000,0.000000}%
\pgfsetstrokecolor{currentstroke}%
\pgfsetdash{}{0pt}%
\pgfpathmoveto{\pgfqpoint{4.155469in}{3.785625in}}%
\pgfpathlineto{\pgfqpoint{4.168753in}{3.770410in}}%
\pgfpathlineto{\pgfqpoint{4.182038in}{3.755430in}}%
\pgfpathlineto{\pgfqpoint{4.195324in}{3.740684in}}%
\pgfpathlineto{\pgfqpoint{4.208610in}{3.726171in}}%
\pgfpathlineto{\pgfqpoint{4.216281in}{3.753231in}}%
\pgfpathlineto{\pgfqpoint{4.223954in}{3.780739in}}%
\pgfpathlineto{\pgfqpoint{4.231628in}{3.808705in}}%
\pgfpathlineto{\pgfqpoint{4.239304in}{3.837137in}}%
\pgfpathlineto{\pgfqpoint{4.226018in}{3.852392in}}%
\pgfpathlineto{\pgfqpoint{4.212732in}{3.867880in}}%
\pgfpathlineto{\pgfqpoint{4.199447in}{3.883603in}}%
\pgfpathlineto{\pgfqpoint{4.186163in}{3.899563in}}%
\pgfpathlineto{\pgfqpoint{4.178487in}{3.870374in}}%
\pgfpathlineto{\pgfqpoint{4.170813in}{3.841661in}}%
\pgfpathlineto{\pgfqpoint{4.163140in}{3.813414in}}%
\pgfpathlineto{\pgfqpoint{4.155469in}{3.785625in}}%
\pgfpathclose%
\pgfusepath{fill}%
\end{pgfscope}%
\begin{pgfscope}%
\pgfpathrectangle{\pgfqpoint{1.150000in}{0.150000in}}{\pgfqpoint{5.700000in}{5.700000in}}%
\pgfusepath{clip}%
\pgfsetbuttcap%
\pgfsetroundjoin%
\definecolor{currentfill}{rgb}{0.133743,0.548535,0.553541}%
\pgfsetfillcolor{currentfill}%
\pgfsetfillopacity{0.800000}%
\pgfsetlinewidth{0.000000pt}%
\definecolor{currentstroke}{rgb}{0.000000,0.000000,0.000000}%
\pgfsetstrokecolor{currentstroke}%
\pgfsetdash{}{0pt}%
\pgfpathmoveto{\pgfqpoint{3.934685in}{3.763589in}}%
\pgfpathlineto{\pgfqpoint{3.947974in}{3.746820in}}%
\pgfpathlineto{\pgfqpoint{3.961262in}{3.730305in}}%
\pgfpathlineto{\pgfqpoint{3.974549in}{3.714043in}}%
\pgfpathlineto{\pgfqpoint{3.987835in}{3.698031in}}%
\pgfpathlineto{\pgfqpoint{3.995506in}{3.723795in}}%
\pgfpathlineto{\pgfqpoint{4.003176in}{3.749966in}}%
\pgfpathlineto{\pgfqpoint{4.010846in}{3.776554in}}%
\pgfpathlineto{\pgfqpoint{4.018515in}{3.803565in}}%
\pgfpathlineto{\pgfqpoint{4.005229in}{3.820257in}}%
\pgfpathlineto{\pgfqpoint{3.991942in}{3.837200in}}%
\pgfpathlineto{\pgfqpoint{3.978653in}{3.854396in}}%
\pgfpathlineto{\pgfqpoint{3.965363in}{3.871848in}}%
\pgfpathlineto{\pgfqpoint{3.957695in}{3.844142in}}%
\pgfpathlineto{\pgfqpoint{3.950026in}{3.816869in}}%
\pgfpathlineto{\pgfqpoint{3.942356in}{3.790021in}}%
\pgfpathlineto{\pgfqpoint{3.934685in}{3.763589in}}%
\pgfpathclose%
\pgfusepath{fill}%
\end{pgfscope}%
\begin{pgfscope}%
\pgfpathrectangle{\pgfqpoint{1.150000in}{0.150000in}}{\pgfqpoint{5.700000in}{5.700000in}}%
\pgfusepath{clip}%
\pgfsetbuttcap%
\pgfsetroundjoin%
\definecolor{currentfill}{rgb}{0.124395,0.578002,0.548287}%
\pgfsetfillcolor{currentfill}%
\pgfsetfillopacity{0.800000}%
\pgfsetlinewidth{0.000000pt}%
\definecolor{currentstroke}{rgb}{0.000000,0.000000,0.000000}%
\pgfsetstrokecolor{currentstroke}%
\pgfsetdash{}{0pt}%
\pgfpathmoveto{\pgfqpoint{4.102333in}{3.848873in}}%
\pgfpathlineto{\pgfqpoint{4.115617in}{3.832700in}}%
\pgfpathlineto{\pgfqpoint{4.128901in}{3.816769in}}%
\pgfpathlineto{\pgfqpoint{4.142185in}{3.801078in}}%
\pgfpathlineto{\pgfqpoint{4.155469in}{3.785625in}}%
\pgfpathlineto{\pgfqpoint{4.163140in}{3.813414in}}%
\pgfpathlineto{\pgfqpoint{4.170813in}{3.841661in}}%
\pgfpathlineto{\pgfqpoint{4.178487in}{3.870374in}}%
\pgfpathlineto{\pgfqpoint{4.186163in}{3.899563in}}%
\pgfpathlineto{\pgfqpoint{4.172878in}{3.915760in}}%
\pgfpathlineto{\pgfqpoint{4.159593in}{3.932197in}}%
\pgfpathlineto{\pgfqpoint{4.146309in}{3.948876in}}%
\pgfpathlineto{\pgfqpoint{4.133023in}{3.965797in}}%
\pgfpathlineto{\pgfqpoint{4.125349in}{3.935848in}}%
\pgfpathlineto{\pgfqpoint{4.117676in}{3.906383in}}%
\pgfpathlineto{\pgfqpoint{4.110004in}{3.877395in}}%
\pgfpathlineto{\pgfqpoint{4.102333in}{3.848873in}}%
\pgfpathclose%
\pgfusepath{fill}%
\end{pgfscope}%
\begin{pgfscope}%
\pgfpathrectangle{\pgfqpoint{1.150000in}{0.150000in}}{\pgfqpoint{5.700000in}{5.700000in}}%
\pgfusepath{clip}%
\pgfsetbuttcap%
\pgfsetroundjoin%
\definecolor{currentfill}{rgb}{0.157729,0.485932,0.558013}%
\pgfsetfillcolor{currentfill}%
\pgfsetfillopacity{0.800000}%
\pgfsetlinewidth{0.000000pt}%
\definecolor{currentstroke}{rgb}{0.000000,0.000000,0.000000}%
\pgfsetstrokecolor{currentstroke}%
\pgfsetdash{}{0pt}%
\pgfpathmoveto{\pgfqpoint{4.094114in}{3.578740in}}%
\pgfpathlineto{\pgfqpoint{4.107400in}{3.564905in}}%
\pgfpathlineto{\pgfqpoint{4.120687in}{3.551303in}}%
\pgfpathlineto{\pgfqpoint{4.133976in}{3.537934in}}%
\pgfpathlineto{\pgfqpoint{4.147265in}{3.524797in}}%
\pgfpathlineto{\pgfqpoint{4.154933in}{3.548578in}}%
\pgfpathlineto{\pgfqpoint{4.162600in}{3.572740in}}%
\pgfpathlineto{\pgfqpoint{4.170267in}{3.597291in}}%
\pgfpathlineto{\pgfqpoint{4.177935in}{3.622239in}}%
\pgfpathlineto{\pgfqpoint{4.164647in}{3.636047in}}%
\pgfpathlineto{\pgfqpoint{4.151360in}{3.650087in}}%
\pgfpathlineto{\pgfqpoint{4.138075in}{3.664360in}}%
\pgfpathlineto{\pgfqpoint{4.124790in}{3.678867in}}%
\pgfpathlineto{\pgfqpoint{4.117121in}{3.653235in}}%
\pgfpathlineto{\pgfqpoint{4.109452in}{3.628008in}}%
\pgfpathlineto{\pgfqpoint{4.101783in}{3.603179in}}%
\pgfpathlineto{\pgfqpoint{4.094114in}{3.578740in}}%
\pgfpathclose%
\pgfusepath{fill}%
\end{pgfscope}%
\begin{pgfscope}%
\pgfpathrectangle{\pgfqpoint{1.150000in}{0.150000in}}{\pgfqpoint{5.700000in}{5.700000in}}%
\pgfusepath{clip}%
\pgfsetbuttcap%
\pgfsetroundjoin%
\definecolor{currentfill}{rgb}{0.137770,0.537492,0.554906}%
\pgfsetfillcolor{currentfill}%
\pgfsetfillopacity{0.800000}%
\pgfsetlinewidth{0.000000pt}%
\definecolor{currentstroke}{rgb}{0.000000,0.000000,0.000000}%
\pgfsetstrokecolor{currentstroke}%
\pgfsetdash{}{0pt}%
\pgfpathmoveto{\pgfqpoint{4.208610in}{3.726171in}}%
\pgfpathlineto{\pgfqpoint{4.221897in}{3.711888in}}%
\pgfpathlineto{\pgfqpoint{4.235186in}{3.697836in}}%
\pgfpathlineto{\pgfqpoint{4.248476in}{3.684011in}}%
\pgfpathlineto{\pgfqpoint{4.261768in}{3.670413in}}%
\pgfpathlineto{\pgfqpoint{4.269437in}{3.696747in}}%
\pgfpathlineto{\pgfqpoint{4.277109in}{3.723521in}}%
\pgfpathlineto{\pgfqpoint{4.284782in}{3.750742in}}%
\pgfpathlineto{\pgfqpoint{4.292457in}{3.778422in}}%
\pgfpathlineto{\pgfqpoint{4.279167in}{3.792758in}}%
\pgfpathlineto{\pgfqpoint{4.265878in}{3.807321in}}%
\pgfpathlineto{\pgfqpoint{4.252590in}{3.822114in}}%
\pgfpathlineto{\pgfqpoint{4.239304in}{3.837137in}}%
\pgfpathlineto{\pgfqpoint{4.231628in}{3.808705in}}%
\pgfpathlineto{\pgfqpoint{4.223954in}{3.780739in}}%
\pgfpathlineto{\pgfqpoint{4.216281in}{3.753231in}}%
\pgfpathlineto{\pgfqpoint{4.208610in}{3.726171in}}%
\pgfpathclose%
\pgfusepath{fill}%
\end{pgfscope}%
\begin{pgfscope}%
\pgfpathrectangle{\pgfqpoint{1.150000in}{0.150000in}}{\pgfqpoint{5.700000in}{5.700000in}}%
\pgfusepath{clip}%
\pgfsetbuttcap%
\pgfsetroundjoin%
\definecolor{currentfill}{rgb}{0.154815,0.493313,0.557840}%
\pgfsetfillcolor{currentfill}%
\pgfsetfillopacity{0.800000}%
\pgfsetlinewidth{0.000000pt}%
\definecolor{currentstroke}{rgb}{0.000000,0.000000,0.000000}%
\pgfsetstrokecolor{currentstroke}%
\pgfsetdash{}{0pt}%
\pgfpathmoveto{\pgfqpoint{3.957139in}{3.598891in}}%
\pgfpathlineto{\pgfqpoint{3.970425in}{3.583770in}}%
\pgfpathlineto{\pgfqpoint{3.983711in}{3.568896in}}%
\pgfpathlineto{\pgfqpoint{3.996996in}{3.554266in}}%
\pgfpathlineto{\pgfqpoint{4.010282in}{3.539879in}}%
\pgfpathlineto{\pgfqpoint{4.017957in}{3.563463in}}%
\pgfpathlineto{\pgfqpoint{4.025630in}{3.587415in}}%
\pgfpathlineto{\pgfqpoint{4.033302in}{3.611741in}}%
\pgfpathlineto{\pgfqpoint{4.040974in}{3.636451in}}%
\pgfpathlineto{\pgfqpoint{4.027689in}{3.651479in}}%
\pgfpathlineto{\pgfqpoint{4.014405in}{3.666750in}}%
\pgfpathlineto{\pgfqpoint{4.001120in}{3.682267in}}%
\pgfpathlineto{\pgfqpoint{3.987835in}{3.698031in}}%
\pgfpathlineto{\pgfqpoint{3.980163in}{3.672666in}}%
\pgfpathlineto{\pgfqpoint{3.972489in}{3.647693in}}%
\pgfpathlineto{\pgfqpoint{3.964815in}{3.623104in}}%
\pgfpathlineto{\pgfqpoint{3.957139in}{3.598891in}}%
\pgfpathclose%
\pgfusepath{fill}%
\end{pgfscope}%
\begin{pgfscope}%
\pgfpathrectangle{\pgfqpoint{1.150000in}{0.150000in}}{\pgfqpoint{5.700000in}{5.700000in}}%
\pgfusepath{clip}%
\pgfsetbuttcap%
\pgfsetroundjoin%
\definecolor{currentfill}{rgb}{0.151918,0.500685,0.557587}%
\pgfsetfillcolor{currentfill}%
\pgfsetfillopacity{0.800000}%
\pgfsetlinewidth{0.000000pt}%
\definecolor{currentstroke}{rgb}{0.000000,0.000000,0.000000}%
\pgfsetstrokecolor{currentstroke}%
\pgfsetdash{}{0pt}%
\pgfpathmoveto{\pgfqpoint{4.177935in}{3.622239in}}%
\pgfpathlineto{\pgfqpoint{4.191224in}{3.608661in}}%
\pgfpathlineto{\pgfqpoint{4.204514in}{3.595312in}}%
\pgfpathlineto{\pgfqpoint{4.217806in}{3.582191in}}%
\pgfpathlineto{\pgfqpoint{4.231100in}{3.569295in}}%
\pgfpathlineto{\pgfqpoint{4.238765in}{3.593958in}}%
\pgfpathlineto{\pgfqpoint{4.246432in}{3.619027in}}%
\pgfpathlineto{\pgfqpoint{4.254099in}{3.644509in}}%
\pgfpathlineto{\pgfqpoint{4.261768in}{3.670413in}}%
\pgfpathlineto{\pgfqpoint{4.248476in}{3.684011in}}%
\pgfpathlineto{\pgfqpoint{4.235186in}{3.697836in}}%
\pgfpathlineto{\pgfqpoint{4.221897in}{3.711888in}}%
\pgfpathlineto{\pgfqpoint{4.208610in}{3.726171in}}%
\pgfpathlineto{\pgfqpoint{4.200940in}{3.699550in}}%
\pgfpathlineto{\pgfqpoint{4.193271in}{3.673361in}}%
\pgfpathlineto{\pgfqpoint{4.185603in}{3.647593in}}%
\pgfpathlineto{\pgfqpoint{4.177935in}{3.622239in}}%
\pgfpathclose%
\pgfusepath{fill}%
\end{pgfscope}%
\begin{pgfscope}%
\pgfpathrectangle{\pgfqpoint{1.150000in}{0.150000in}}{\pgfqpoint{5.700000in}{5.700000in}}%
\pgfusepath{clip}%
\pgfsetbuttcap%
\pgfsetroundjoin%
\definecolor{currentfill}{rgb}{0.147607,0.511733,0.557049}%
\pgfsetfillcolor{currentfill}%
\pgfsetfillopacity{0.800000}%
\pgfsetlinewidth{0.000000pt}%
\definecolor{currentstroke}{rgb}{0.000000,0.000000,0.000000}%
\pgfsetstrokecolor{currentstroke}%
\pgfsetdash{}{0pt}%
\pgfpathmoveto{\pgfqpoint{3.903988in}{3.661868in}}%
\pgfpathlineto{\pgfqpoint{3.917277in}{3.645746in}}%
\pgfpathlineto{\pgfqpoint{3.930565in}{3.629877in}}%
\pgfpathlineto{\pgfqpoint{3.943853in}{3.614259in}}%
\pgfpathlineto{\pgfqpoint{3.957139in}{3.598891in}}%
\pgfpathlineto{\pgfqpoint{3.964815in}{3.623104in}}%
\pgfpathlineto{\pgfqpoint{3.972489in}{3.647693in}}%
\pgfpathlineto{\pgfqpoint{3.980163in}{3.672666in}}%
\pgfpathlineto{\pgfqpoint{3.987835in}{3.698031in}}%
\pgfpathlineto{\pgfqpoint{3.974549in}{3.714043in}}%
\pgfpathlineto{\pgfqpoint{3.961262in}{3.730305in}}%
\pgfpathlineto{\pgfqpoint{3.947974in}{3.746820in}}%
\pgfpathlineto{\pgfqpoint{3.934685in}{3.763589in}}%
\pgfpathlineto{\pgfqpoint{3.927013in}{3.737566in}}%
\pgfpathlineto{\pgfqpoint{3.919339in}{3.711943in}}%
\pgfpathlineto{\pgfqpoint{3.911665in}{3.686713in}}%
\pgfpathlineto{\pgfqpoint{3.903988in}{3.661868in}}%
\pgfpathclose%
\pgfusepath{fill}%
\end{pgfscope}%
\begin{pgfscope}%
\pgfpathrectangle{\pgfqpoint{1.150000in}{0.150000in}}{\pgfqpoint{5.700000in}{5.700000in}}%
\pgfusepath{clip}%
\pgfsetbuttcap%
\pgfsetroundjoin%
\definecolor{currentfill}{rgb}{0.122606,0.585371,0.546557}%
\pgfsetfillcolor{currentfill}%
\pgfsetfillopacity{0.800000}%
\pgfsetlinewidth{0.000000pt}%
\definecolor{currentstroke}{rgb}{0.000000,0.000000,0.000000}%
\pgfsetstrokecolor{currentstroke}%
\pgfsetdash{}{0pt}%
\pgfpathmoveto{\pgfqpoint{3.965363in}{3.871848in}}%
\pgfpathlineto{\pgfqpoint{3.978653in}{3.854396in}}%
\pgfpathlineto{\pgfqpoint{3.991942in}{3.837200in}}%
\pgfpathlineto{\pgfqpoint{4.005229in}{3.820257in}}%
\pgfpathlineto{\pgfqpoint{4.018515in}{3.803565in}}%
\pgfpathlineto{\pgfqpoint{4.026184in}{3.831008in}}%
\pgfpathlineto{\pgfqpoint{4.033853in}{3.858892in}}%
\pgfpathlineto{\pgfqpoint{4.041522in}{3.887226in}}%
\pgfpathlineto{\pgfqpoint{4.049191in}{3.916018in}}%
\pgfpathlineto{\pgfqpoint{4.035904in}{3.933426in}}%
\pgfpathlineto{\pgfqpoint{4.022615in}{3.951086in}}%
\pgfpathlineto{\pgfqpoint{4.009325in}{3.969001in}}%
\pgfpathlineto{\pgfqpoint{3.996033in}{3.987172in}}%
\pgfpathlineto{\pgfqpoint{3.988365in}{3.957648in}}%
\pgfpathlineto{\pgfqpoint{3.980698in}{3.928592in}}%
\pgfpathlineto{\pgfqpoint{3.973031in}{3.899995in}}%
\pgfpathlineto{\pgfqpoint{3.965363in}{3.871848in}}%
\pgfpathclose%
\pgfusepath{fill}%
\end{pgfscope}%
\begin{pgfscope}%
\pgfpathrectangle{\pgfqpoint{1.150000in}{0.150000in}}{\pgfqpoint{5.700000in}{5.700000in}}%
\pgfusepath{clip}%
\pgfsetbuttcap%
\pgfsetroundjoin%
\definecolor{currentfill}{rgb}{0.163625,0.471133,0.558148}%
\pgfsetfillcolor{currentfill}%
\pgfsetfillopacity{0.800000}%
\pgfsetlinewidth{0.000000pt}%
\definecolor{currentstroke}{rgb}{0.000000,0.000000,0.000000}%
\pgfsetstrokecolor{currentstroke}%
\pgfsetdash{}{0pt}%
\pgfpathmoveto{\pgfqpoint{4.010282in}{3.539879in}}%
\pgfpathlineto{\pgfqpoint{4.023568in}{3.525733in}}%
\pgfpathlineto{\pgfqpoint{4.036854in}{3.511826in}}%
\pgfpathlineto{\pgfqpoint{4.050141in}{3.498157in}}%
\pgfpathlineto{\pgfqpoint{4.063429in}{3.484725in}}%
\pgfpathlineto{\pgfqpoint{4.071101in}{3.507683in}}%
\pgfpathlineto{\pgfqpoint{4.078773in}{3.530999in}}%
\pgfpathlineto{\pgfqpoint{4.086444in}{3.554683in}}%
\pgfpathlineto{\pgfqpoint{4.094114in}{3.578740in}}%
\pgfpathlineto{\pgfqpoint{4.080828in}{3.592811in}}%
\pgfpathlineto{\pgfqpoint{4.067543in}{3.607118in}}%
\pgfpathlineto{\pgfqpoint{4.054258in}{3.621664in}}%
\pgfpathlineto{\pgfqpoint{4.040974in}{3.636451in}}%
\pgfpathlineto{\pgfqpoint{4.033302in}{3.611741in}}%
\pgfpathlineto{\pgfqpoint{4.025630in}{3.587415in}}%
\pgfpathlineto{\pgfqpoint{4.017957in}{3.563463in}}%
\pgfpathlineto{\pgfqpoint{4.010282in}{3.539879in}}%
\pgfpathclose%
\pgfusepath{fill}%
\end{pgfscope}%
\begin{pgfscope}%
\pgfpathrectangle{\pgfqpoint{1.150000in}{0.150000in}}{\pgfqpoint{5.700000in}{5.700000in}}%
\pgfusepath{clip}%
\pgfsetbuttcap%
\pgfsetroundjoin%
\definecolor{currentfill}{rgb}{0.119738,0.603785,0.541400}%
\pgfsetfillcolor{currentfill}%
\pgfsetfillopacity{0.800000}%
\pgfsetlinewidth{0.000000pt}%
\definecolor{currentstroke}{rgb}{0.000000,0.000000,0.000000}%
\pgfsetstrokecolor{currentstroke}%
\pgfsetdash{}{0pt}%
\pgfpathmoveto{\pgfqpoint{4.049191in}{3.916018in}}%
\pgfpathlineto{\pgfqpoint{4.062478in}{3.898860in}}%
\pgfpathlineto{\pgfqpoint{4.075764in}{3.881951in}}%
\pgfpathlineto{\pgfqpoint{4.089049in}{3.865290in}}%
\pgfpathlineto{\pgfqpoint{4.102333in}{3.848873in}}%
\pgfpathlineto{\pgfqpoint{4.110004in}{3.877395in}}%
\pgfpathlineto{\pgfqpoint{4.117676in}{3.906383in}}%
\pgfpathlineto{\pgfqpoint{4.125349in}{3.935848in}}%
\pgfpathlineto{\pgfqpoint{4.133023in}{3.965797in}}%
\pgfpathlineto{\pgfqpoint{4.119738in}{3.982963in}}%
\pgfpathlineto{\pgfqpoint{4.106451in}{4.000375in}}%
\pgfpathlineto{\pgfqpoint{4.093164in}{4.018035in}}%
\pgfpathlineto{\pgfqpoint{4.079875in}{4.035946in}}%
\pgfpathlineto{\pgfqpoint{4.072203in}{4.005231in}}%
\pgfpathlineto{\pgfqpoint{4.064531in}{3.975011in}}%
\pgfpathlineto{\pgfqpoint{4.056861in}{3.945277in}}%
\pgfpathlineto{\pgfqpoint{4.049191in}{3.916018in}}%
\pgfpathclose%
\pgfusepath{fill}%
\end{pgfscope}%
\begin{pgfscope}%
\pgfpathrectangle{\pgfqpoint{1.150000in}{0.150000in}}{\pgfqpoint{5.700000in}{5.700000in}}%
\pgfusepath{clip}%
\pgfsetbuttcap%
\pgfsetroundjoin%
\definecolor{currentfill}{rgb}{0.126453,0.570633,0.549841}%
\pgfsetfillcolor{currentfill}%
\pgfsetfillopacity{0.800000}%
\pgfsetlinewidth{0.000000pt}%
\definecolor{currentstroke}{rgb}{0.000000,0.000000,0.000000}%
\pgfsetstrokecolor{currentstroke}%
\pgfsetdash{}{0pt}%
\pgfpathmoveto{\pgfqpoint{4.239304in}{3.837137in}}%
\pgfpathlineto{\pgfqpoint{4.252590in}{3.822114in}}%
\pgfpathlineto{\pgfqpoint{4.265878in}{3.807321in}}%
\pgfpathlineto{\pgfqpoint{4.279167in}{3.792758in}}%
\pgfpathlineto{\pgfqpoint{4.292457in}{3.778422in}}%
\pgfpathlineto{\pgfqpoint{4.300135in}{3.806567in}}%
\pgfpathlineto{\pgfqpoint{4.307815in}{3.835188in}}%
\pgfpathlineto{\pgfqpoint{4.315497in}{3.864295in}}%
\pgfpathlineto{\pgfqpoint{4.302207in}{3.879208in}}%
\pgfpathlineto{\pgfqpoint{4.288919in}{3.894350in}}%
\pgfpathlineto{\pgfqpoint{4.275631in}{3.909722in}}%
\pgfpathlineto{\pgfqpoint{4.262344in}{3.925324in}}%
\pgfpathlineto{\pgfqpoint{4.254662in}{3.895437in}}%
\pgfpathlineto{\pgfqpoint{4.246981in}{3.866045in}}%
\pgfpathlineto{\pgfqpoint{4.239304in}{3.837137in}}%
\pgfpathclose%
\pgfusepath{fill}%
\end{pgfscope}%
\begin{pgfscope}%
\pgfpathrectangle{\pgfqpoint{1.150000in}{0.150000in}}{\pgfqpoint{5.700000in}{5.700000in}}%
\pgfusepath{clip}%
\pgfsetbuttcap%
\pgfsetroundjoin%
\definecolor{currentfill}{rgb}{0.121148,0.592739,0.544641}%
\pgfsetfillcolor{currentfill}%
\pgfsetfillopacity{0.800000}%
\pgfsetlinewidth{0.000000pt}%
\definecolor{currentstroke}{rgb}{0.000000,0.000000,0.000000}%
\pgfsetstrokecolor{currentstroke}%
\pgfsetdash{}{0pt}%
\pgfpathmoveto{\pgfqpoint{4.186163in}{3.899563in}}%
\pgfpathlineto{\pgfqpoint{4.199447in}{3.883603in}}%
\pgfpathlineto{\pgfqpoint{4.212732in}{3.867880in}}%
\pgfpathlineto{\pgfqpoint{4.226018in}{3.852392in}}%
\pgfpathlineto{\pgfqpoint{4.239304in}{3.837137in}}%
\pgfpathlineto{\pgfqpoint{4.246981in}{3.866045in}}%
\pgfpathlineto{\pgfqpoint{4.254662in}{3.895437in}}%
\pgfpathlineto{\pgfqpoint{4.262344in}{3.925324in}}%
\pgfpathlineto{\pgfqpoint{4.249058in}{3.941159in}}%
\pgfpathlineto{\pgfqpoint{4.235772in}{3.957228in}}%
\pgfpathlineto{\pgfqpoint{4.222486in}{3.973533in}}%
\pgfpathlineto{\pgfqpoint{4.209201in}{3.990076in}}%
\pgfpathlineto{\pgfqpoint{4.201519in}{3.959404in}}%
\pgfpathlineto{\pgfqpoint{4.193840in}{3.929236in}}%
\pgfpathlineto{\pgfqpoint{4.186163in}{3.899563in}}%
\pgfpathclose%
\pgfusepath{fill}%
\end{pgfscope}%
\begin{pgfscope}%
\pgfpathrectangle{\pgfqpoint{1.150000in}{0.150000in}}{\pgfqpoint{5.700000in}{5.700000in}}%
\pgfusepath{clip}%
\pgfsetbuttcap%
\pgfsetroundjoin%
\definecolor{currentfill}{rgb}{0.137770,0.537492,0.554906}%
\pgfsetfillcolor{currentfill}%
\pgfsetfillopacity{0.800000}%
\pgfsetlinewidth{0.000000pt}%
\definecolor{currentstroke}{rgb}{0.000000,0.000000,0.000000}%
\pgfsetstrokecolor{currentstroke}%
\pgfsetdash{}{0pt}%
\pgfpathmoveto{\pgfqpoint{3.850817in}{3.728927in}}%
\pgfpathlineto{\pgfqpoint{3.864112in}{3.711773in}}%
\pgfpathlineto{\pgfqpoint{3.877406in}{3.694880in}}%
\pgfpathlineto{\pgfqpoint{3.890698in}{3.678245in}}%
\pgfpathlineto{\pgfqpoint{3.903988in}{3.661868in}}%
\pgfpathlineto{\pgfqpoint{3.911665in}{3.686713in}}%
\pgfpathlineto{\pgfqpoint{3.919339in}{3.711943in}}%
\pgfpathlineto{\pgfqpoint{3.927013in}{3.737566in}}%
\pgfpathlineto{\pgfqpoint{3.934685in}{3.763589in}}%
\pgfpathlineto{\pgfqpoint{3.921394in}{3.780614in}}%
\pgfpathlineto{\pgfqpoint{3.908102in}{3.797896in}}%
\pgfpathlineto{\pgfqpoint{3.894808in}{3.815439in}}%
\pgfpathlineto{\pgfqpoint{3.881512in}{3.833243in}}%
\pgfpathlineto{\pgfqpoint{3.873841in}{3.806557in}}%
\pgfpathlineto{\pgfqpoint{3.866168in}{3.780281in}}%
\pgfpathlineto{\pgfqpoint{3.858493in}{3.754407in}}%
\pgfpathlineto{\pgfqpoint{3.850817in}{3.728927in}}%
\pgfpathclose%
\pgfusepath{fill}%
\end{pgfscope}%
\begin{pgfscope}%
\pgfpathrectangle{\pgfqpoint{1.150000in}{0.150000in}}{\pgfqpoint{5.700000in}{5.700000in}}%
\pgfusepath{clip}%
\pgfsetbuttcap%
\pgfsetroundjoin%
\definecolor{currentfill}{rgb}{0.125394,0.574318,0.549086}%
\pgfsetfillcolor{currentfill}%
\pgfsetfillopacity{0.800000}%
\pgfsetlinewidth{0.000000pt}%
\definecolor{currentstroke}{rgb}{0.000000,0.000000,0.000000}%
\pgfsetstrokecolor{currentstroke}%
\pgfsetdash{}{0pt}%
\pgfpathmoveto{\pgfqpoint{3.881512in}{3.833243in}}%
\pgfpathlineto{\pgfqpoint{3.894808in}{3.815439in}}%
\pgfpathlineto{\pgfqpoint{3.908102in}{3.797896in}}%
\pgfpathlineto{\pgfqpoint{3.921394in}{3.780614in}}%
\pgfpathlineto{\pgfqpoint{3.934685in}{3.763589in}}%
\pgfpathlineto{\pgfqpoint{3.942356in}{3.790021in}}%
\pgfpathlineto{\pgfqpoint{3.950026in}{3.816869in}}%
\pgfpathlineto{\pgfqpoint{3.957695in}{3.844142in}}%
\pgfpathlineto{\pgfqpoint{3.965363in}{3.871848in}}%
\pgfpathlineto{\pgfqpoint{3.952071in}{3.889556in}}%
\pgfpathlineto{\pgfqpoint{3.938778in}{3.907523in}}%
\pgfpathlineto{\pgfqpoint{3.925482in}{3.925751in}}%
\pgfpathlineto{\pgfqpoint{3.912184in}{3.944242in}}%
\pgfpathlineto{\pgfqpoint{3.904518in}{3.915837in}}%
\pgfpathlineto{\pgfqpoint{3.896850in}{3.887874in}}%
\pgfpathlineto{\pgfqpoint{3.889182in}{3.860346in}}%
\pgfpathlineto{\pgfqpoint{3.881512in}{3.833243in}}%
\pgfpathclose%
\pgfusepath{fill}%
\end{pgfscope}%
\begin{pgfscope}%
\pgfpathrectangle{\pgfqpoint{1.150000in}{0.150000in}}{\pgfqpoint{5.700000in}{5.700000in}}%
\pgfusepath{clip}%
\pgfsetbuttcap%
\pgfsetroundjoin%
\definecolor{currentfill}{rgb}{0.144759,0.519093,0.556572}%
\pgfsetfillcolor{currentfill}%
\pgfsetfillopacity{0.800000}%
\pgfsetlinewidth{0.000000pt}%
\definecolor{currentstroke}{rgb}{0.000000,0.000000,0.000000}%
\pgfsetstrokecolor{currentstroke}%
\pgfsetdash{}{0pt}%
\pgfpathmoveto{\pgfqpoint{4.261768in}{3.670413in}}%
\pgfpathlineto{\pgfqpoint{4.275061in}{3.657040in}}%
\pgfpathlineto{\pgfqpoint{4.288356in}{3.643891in}}%
\pgfpathlineto{\pgfqpoint{4.301653in}{3.630965in}}%
\pgfpathlineto{\pgfqpoint{4.314952in}{3.618260in}}%
\pgfpathlineto{\pgfqpoint{4.322620in}{3.643872in}}%
\pgfpathlineto{\pgfqpoint{4.330289in}{3.669914in}}%
\pgfpathlineto{\pgfqpoint{4.337961in}{3.696395in}}%
\pgfpathlineto{\pgfqpoint{4.345634in}{3.723325in}}%
\pgfpathlineto{\pgfqpoint{4.332337in}{3.736765in}}%
\pgfpathlineto{\pgfqpoint{4.319042in}{3.750426in}}%
\pgfpathlineto{\pgfqpoint{4.305749in}{3.764312in}}%
\pgfpathlineto{\pgfqpoint{4.292457in}{3.778422in}}%
\pgfpathlineto{\pgfqpoint{4.284782in}{3.750742in}}%
\pgfpathlineto{\pgfqpoint{4.277109in}{3.723521in}}%
\pgfpathlineto{\pgfqpoint{4.269437in}{3.696747in}}%
\pgfpathlineto{\pgfqpoint{4.261768in}{3.670413in}}%
\pgfpathclose%
\pgfusepath{fill}%
\end{pgfscope}%
\begin{pgfscope}%
\pgfpathrectangle{\pgfqpoint{1.150000in}{0.150000in}}{\pgfqpoint{5.700000in}{5.700000in}}%
\pgfusepath{clip}%
\pgfsetbuttcap%
\pgfsetroundjoin%
\definecolor{currentfill}{rgb}{0.165117,0.467423,0.558141}%
\pgfsetfillcolor{currentfill}%
\pgfsetfillopacity{0.800000}%
\pgfsetlinewidth{0.000000pt}%
\definecolor{currentstroke}{rgb}{0.000000,0.000000,0.000000}%
\pgfsetstrokecolor{currentstroke}%
\pgfsetdash{}{0pt}%
\pgfpathmoveto{\pgfqpoint{4.147265in}{3.524797in}}%
\pgfpathlineto{\pgfqpoint{4.160557in}{3.511888in}}%
\pgfpathlineto{\pgfqpoint{4.173849in}{3.499208in}}%
\pgfpathlineto{\pgfqpoint{4.187144in}{3.486754in}}%
\pgfpathlineto{\pgfqpoint{4.200441in}{3.474526in}}%
\pgfpathlineto{\pgfqpoint{4.208105in}{3.497652in}}%
\pgfpathlineto{\pgfqpoint{4.215770in}{3.521150in}}%
\pgfpathlineto{\pgfqpoint{4.223435in}{3.545028in}}%
\pgfpathlineto{\pgfqpoint{4.231100in}{3.569295in}}%
\pgfpathlineto{\pgfqpoint{4.217806in}{3.582191in}}%
\pgfpathlineto{\pgfqpoint{4.204514in}{3.595312in}}%
\pgfpathlineto{\pgfqpoint{4.191224in}{3.608661in}}%
\pgfpathlineto{\pgfqpoint{4.177935in}{3.622239in}}%
\pgfpathlineto{\pgfqpoint{4.170267in}{3.597291in}}%
\pgfpathlineto{\pgfqpoint{4.162600in}{3.572740in}}%
\pgfpathlineto{\pgfqpoint{4.154933in}{3.548578in}}%
\pgfpathlineto{\pgfqpoint{4.147265in}{3.524797in}}%
\pgfpathclose%
\pgfusepath{fill}%
\end{pgfscope}%
\begin{pgfscope}%
\pgfpathrectangle{\pgfqpoint{1.150000in}{0.150000in}}{\pgfqpoint{5.700000in}{5.700000in}}%
\pgfusepath{clip}%
\pgfsetbuttcap%
\pgfsetroundjoin%
\definecolor{currentfill}{rgb}{0.132444,0.552216,0.553018}%
\pgfsetfillcolor{currentfill}%
\pgfsetfillopacity{0.800000}%
\pgfsetlinewidth{0.000000pt}%
\definecolor{currentstroke}{rgb}{0.000000,0.000000,0.000000}%
\pgfsetstrokecolor{currentstroke}%
\pgfsetdash{}{0pt}%
\pgfpathmoveto{\pgfqpoint{4.292457in}{3.778422in}}%
\pgfpathlineto{\pgfqpoint{4.305749in}{3.764312in}}%
\pgfpathlineto{\pgfqpoint{4.319042in}{3.750426in}}%
\pgfpathlineto{\pgfqpoint{4.332337in}{3.736765in}}%
\pgfpathlineto{\pgfqpoint{4.345634in}{3.723325in}}%
\pgfpathlineto{\pgfqpoint{4.353310in}{3.750712in}}%
\pgfpathlineto{\pgfqpoint{4.360989in}{3.778566in}}%
\pgfpathlineto{\pgfqpoint{4.368670in}{3.806895in}}%
\pgfpathlineto{\pgfqpoint{4.355375in}{3.820910in}}%
\pgfpathlineto{\pgfqpoint{4.342081in}{3.835147in}}%
\pgfpathlineto{\pgfqpoint{4.328788in}{3.849608in}}%
\pgfpathlineto{\pgfqpoint{4.315497in}{3.864295in}}%
\pgfpathlineto{\pgfqpoint{4.307815in}{3.835188in}}%
\pgfpathlineto{\pgfqpoint{4.300135in}{3.806567in}}%
\pgfpathlineto{\pgfqpoint{4.292457in}{3.778422in}}%
\pgfpathclose%
\pgfusepath{fill}%
\end{pgfscope}%
\begin{pgfscope}%
\pgfpathrectangle{\pgfqpoint{1.150000in}{0.150000in}}{\pgfqpoint{5.700000in}{5.700000in}}%
\pgfusepath{clip}%
\pgfsetbuttcap%
\pgfsetroundjoin%
\definecolor{currentfill}{rgb}{0.157729,0.485932,0.558013}%
\pgfsetfillcolor{currentfill}%
\pgfsetfillopacity{0.800000}%
\pgfsetlinewidth{0.000000pt}%
\definecolor{currentstroke}{rgb}{0.000000,0.000000,0.000000}%
\pgfsetstrokecolor{currentstroke}%
\pgfsetdash{}{0pt}%
\pgfpathmoveto{\pgfqpoint{4.231100in}{3.569295in}}%
\pgfpathlineto{\pgfqpoint{4.244395in}{3.556624in}}%
\pgfpathlineto{\pgfqpoint{4.257693in}{3.544176in}}%
\pgfpathlineto{\pgfqpoint{4.270993in}{3.531951in}}%
\pgfpathlineto{\pgfqpoint{4.284295in}{3.519946in}}%
\pgfpathlineto{\pgfqpoint{4.291958in}{3.543921in}}%
\pgfpathlineto{\pgfqpoint{4.299622in}{3.568294in}}%
\pgfpathlineto{\pgfqpoint{4.307286in}{3.593070in}}%
\pgfpathlineto{\pgfqpoint{4.314952in}{3.618260in}}%
\pgfpathlineto{\pgfqpoint{4.301653in}{3.630965in}}%
\pgfpathlineto{\pgfqpoint{4.288356in}{3.643891in}}%
\pgfpathlineto{\pgfqpoint{4.275061in}{3.657040in}}%
\pgfpathlineto{\pgfqpoint{4.261768in}{3.670413in}}%
\pgfpathlineto{\pgfqpoint{4.254099in}{3.644509in}}%
\pgfpathlineto{\pgfqpoint{4.246432in}{3.619027in}}%
\pgfpathlineto{\pgfqpoint{4.238765in}{3.593958in}}%
\pgfpathlineto{\pgfqpoint{4.231100in}{3.569295in}}%
\pgfpathclose%
\pgfusepath{fill}%
\end{pgfscope}%
\begin{pgfscope}%
\pgfpathrectangle{\pgfqpoint{1.150000in}{0.150000in}}{\pgfqpoint{5.700000in}{5.700000in}}%
\pgfusepath{clip}%
\pgfsetbuttcap%
\pgfsetroundjoin%
\definecolor{currentfill}{rgb}{0.119483,0.614817,0.537692}%
\pgfsetfillcolor{currentfill}%
\pgfsetfillopacity{0.800000}%
\pgfsetlinewidth{0.000000pt}%
\definecolor{currentstroke}{rgb}{0.000000,0.000000,0.000000}%
\pgfsetstrokecolor{currentstroke}%
\pgfsetdash{}{0pt}%
\pgfpathmoveto{\pgfqpoint{4.133023in}{3.965797in}}%
\pgfpathlineto{\pgfqpoint{4.146309in}{3.948876in}}%
\pgfpathlineto{\pgfqpoint{4.159593in}{3.932197in}}%
\pgfpathlineto{\pgfqpoint{4.172878in}{3.915760in}}%
\pgfpathlineto{\pgfqpoint{4.186163in}{3.899563in}}%
\pgfpathlineto{\pgfqpoint{4.193840in}{3.929236in}}%
\pgfpathlineto{\pgfqpoint{4.201519in}{3.959404in}}%
\pgfpathlineto{\pgfqpoint{4.209201in}{3.990076in}}%
\pgfpathlineto{\pgfqpoint{4.195915in}{4.006857in}}%
\pgfpathlineto{\pgfqpoint{4.182630in}{4.023878in}}%
\pgfpathlineto{\pgfqpoint{4.169343in}{4.041141in}}%
\pgfpathlineto{\pgfqpoint{4.156057in}{4.058649in}}%
\pgfpathlineto{\pgfqpoint{4.148377in}{4.027188in}}%
\pgfpathlineto{\pgfqpoint{4.140699in}{3.996241in}}%
\pgfpathlineto{\pgfqpoint{4.133023in}{3.965797in}}%
\pgfpathclose%
\pgfusepath{fill}%
\end{pgfscope}%
\begin{pgfscope}%
\pgfpathrectangle{\pgfqpoint{1.150000in}{0.150000in}}{\pgfqpoint{5.700000in}{5.700000in}}%
\pgfusepath{clip}%
\pgfsetbuttcap%
\pgfsetroundjoin%
\definecolor{currentfill}{rgb}{0.159194,0.482237,0.558073}%
\pgfsetfillcolor{currentfill}%
\pgfsetfillopacity{0.800000}%
\pgfsetlinewidth{0.000000pt}%
\definecolor{currentstroke}{rgb}{0.000000,0.000000,0.000000}%
\pgfsetstrokecolor{currentstroke}%
\pgfsetdash{}{0pt}%
\pgfpathmoveto{\pgfqpoint{3.873263in}{3.566187in}}%
\pgfpathlineto{\pgfqpoint{3.886553in}{3.550676in}}%
\pgfpathlineto{\pgfqpoint{3.899842in}{3.535417in}}%
\pgfpathlineto{\pgfqpoint{3.913131in}{3.520409in}}%
\pgfpathlineto{\pgfqpoint{3.926419in}{3.505650in}}%
\pgfpathlineto{\pgfqpoint{3.934102in}{3.528433in}}%
\pgfpathlineto{\pgfqpoint{3.941783in}{3.551563in}}%
\pgfpathlineto{\pgfqpoint{3.949462in}{3.575046in}}%
\pgfpathlineto{\pgfqpoint{3.957139in}{3.598891in}}%
\pgfpathlineto{\pgfqpoint{3.943853in}{3.614259in}}%
\pgfpathlineto{\pgfqpoint{3.930565in}{3.629877in}}%
\pgfpathlineto{\pgfqpoint{3.917277in}{3.645746in}}%
\pgfpathlineto{\pgfqpoint{3.903988in}{3.661868in}}%
\pgfpathlineto{\pgfqpoint{3.896310in}{3.637400in}}%
\pgfpathlineto{\pgfqpoint{3.888630in}{3.613302in}}%
\pgfpathlineto{\pgfqpoint{3.880948in}{3.589567in}}%
\pgfpathlineto{\pgfqpoint{3.873263in}{3.566187in}}%
\pgfpathclose%
\pgfusepath{fill}%
\end{pgfscope}%
\begin{pgfscope}%
\pgfpathrectangle{\pgfqpoint{1.150000in}{0.150000in}}{\pgfqpoint{5.700000in}{5.700000in}}%
\pgfusepath{clip}%
\pgfsetbuttcap%
\pgfsetroundjoin%
\definecolor{currentfill}{rgb}{0.171176,0.452530,0.557965}%
\pgfsetfillcolor{currentfill}%
\pgfsetfillopacity{0.800000}%
\pgfsetlinewidth{0.000000pt}%
\definecolor{currentstroke}{rgb}{0.000000,0.000000,0.000000}%
\pgfsetstrokecolor{currentstroke}%
\pgfsetdash{}{0pt}%
\pgfpathmoveto{\pgfqpoint{4.063429in}{3.484725in}}%
\pgfpathlineto{\pgfqpoint{4.076717in}{3.471527in}}%
\pgfpathlineto{\pgfqpoint{4.090007in}{3.458562in}}%
\pgfpathlineto{\pgfqpoint{4.103298in}{3.445830in}}%
\pgfpathlineto{\pgfqpoint{4.116590in}{3.433327in}}%
\pgfpathlineto{\pgfqpoint{4.124260in}{3.455662in}}%
\pgfpathlineto{\pgfqpoint{4.131929in}{3.478346in}}%
\pgfpathlineto{\pgfqpoint{4.139598in}{3.501389in}}%
\pgfpathlineto{\pgfqpoint{4.147265in}{3.524797in}}%
\pgfpathlineto{\pgfqpoint{4.133976in}{3.537934in}}%
\pgfpathlineto{\pgfqpoint{4.120687in}{3.551303in}}%
\pgfpathlineto{\pgfqpoint{4.107400in}{3.564905in}}%
\pgfpathlineto{\pgfqpoint{4.094114in}{3.578740in}}%
\pgfpathlineto{\pgfqpoint{4.086444in}{3.554683in}}%
\pgfpathlineto{\pgfqpoint{4.078773in}{3.530999in}}%
\pgfpathlineto{\pgfqpoint{4.071101in}{3.507683in}}%
\pgfpathlineto{\pgfqpoint{4.063429in}{3.484725in}}%
\pgfpathclose%
\pgfusepath{fill}%
\end{pgfscope}%
\begin{pgfscope}%
\pgfpathrectangle{\pgfqpoint{1.150000in}{0.150000in}}{\pgfqpoint{5.700000in}{5.700000in}}%
\pgfusepath{clip}%
\pgfsetbuttcap%
\pgfsetroundjoin%
\definecolor{currentfill}{rgb}{0.168126,0.459988,0.558082}%
\pgfsetfillcolor{currentfill}%
\pgfsetfillopacity{0.800000}%
\pgfsetlinewidth{0.000000pt}%
\definecolor{currentstroke}{rgb}{0.000000,0.000000,0.000000}%
\pgfsetstrokecolor{currentstroke}%
\pgfsetdash{}{0pt}%
\pgfpathmoveto{\pgfqpoint{3.926419in}{3.505650in}}%
\pgfpathlineto{\pgfqpoint{3.939706in}{3.491138in}}%
\pgfpathlineto{\pgfqpoint{3.952994in}{3.476871in}}%
\pgfpathlineto{\pgfqpoint{3.966281in}{3.462848in}}%
\pgfpathlineto{\pgfqpoint{3.979569in}{3.449067in}}%
\pgfpathlineto{\pgfqpoint{3.987250in}{3.471256in}}%
\pgfpathlineto{\pgfqpoint{3.994929in}{3.493783in}}%
\pgfpathlineto{\pgfqpoint{4.002606in}{3.516655in}}%
\pgfpathlineto{\pgfqpoint{4.010282in}{3.539879in}}%
\pgfpathlineto{\pgfqpoint{3.996996in}{3.554266in}}%
\pgfpathlineto{\pgfqpoint{3.983711in}{3.568896in}}%
\pgfpathlineto{\pgfqpoint{3.970425in}{3.583770in}}%
\pgfpathlineto{\pgfqpoint{3.957139in}{3.598891in}}%
\pgfpathlineto{\pgfqpoint{3.949462in}{3.575046in}}%
\pgfpathlineto{\pgfqpoint{3.941783in}{3.551563in}}%
\pgfpathlineto{\pgfqpoint{3.934102in}{3.528433in}}%
\pgfpathlineto{\pgfqpoint{3.926419in}{3.505650in}}%
\pgfpathclose%
\pgfusepath{fill}%
\end{pgfscope}%
\begin{pgfscope}%
\pgfpathrectangle{\pgfqpoint{1.150000in}{0.150000in}}{\pgfqpoint{5.700000in}{5.700000in}}%
\pgfusepath{clip}%
\pgfsetbuttcap%
\pgfsetroundjoin%
\definecolor{currentfill}{rgb}{0.150476,0.504369,0.557430}%
\pgfsetfillcolor{currentfill}%
\pgfsetfillopacity{0.800000}%
\pgfsetlinewidth{0.000000pt}%
\definecolor{currentstroke}{rgb}{0.000000,0.000000,0.000000}%
\pgfsetstrokecolor{currentstroke}%
\pgfsetdash{}{0pt}%
\pgfpathmoveto{\pgfqpoint{3.820090in}{3.630793in}}%
\pgfpathlineto{\pgfqpoint{3.833386in}{3.614254in}}%
\pgfpathlineto{\pgfqpoint{3.846680in}{3.597974in}}%
\pgfpathlineto{\pgfqpoint{3.859972in}{3.581953in}}%
\pgfpathlineto{\pgfqpoint{3.873263in}{3.566187in}}%
\pgfpathlineto{\pgfqpoint{3.880948in}{3.589567in}}%
\pgfpathlineto{\pgfqpoint{3.888630in}{3.613302in}}%
\pgfpathlineto{\pgfqpoint{3.896310in}{3.637400in}}%
\pgfpathlineto{\pgfqpoint{3.903988in}{3.661868in}}%
\pgfpathlineto{\pgfqpoint{3.890698in}{3.678245in}}%
\pgfpathlineto{\pgfqpoint{3.877406in}{3.694880in}}%
\pgfpathlineto{\pgfqpoint{3.864112in}{3.711773in}}%
\pgfpathlineto{\pgfqpoint{3.850817in}{3.728927in}}%
\pgfpathlineto{\pgfqpoint{3.843138in}{3.703833in}}%
\pgfpathlineto{\pgfqpoint{3.835458in}{3.679117in}}%
\pgfpathlineto{\pgfqpoint{3.827775in}{3.654773in}}%
\pgfpathlineto{\pgfqpoint{3.820090in}{3.630793in}}%
\pgfpathclose%
\pgfusepath{fill}%
\end{pgfscope}%
\begin{pgfscope}%
\pgfpathrectangle{\pgfqpoint{1.150000in}{0.150000in}}{\pgfqpoint{5.700000in}{5.700000in}}%
\pgfusepath{clip}%
\pgfsetbuttcap%
\pgfsetroundjoin%
\definecolor{currentfill}{rgb}{0.119423,0.611141,0.538982}%
\pgfsetfillcolor{currentfill}%
\pgfsetfillopacity{0.800000}%
\pgfsetlinewidth{0.000000pt}%
\definecolor{currentstroke}{rgb}{0.000000,0.000000,0.000000}%
\pgfsetstrokecolor{currentstroke}%
\pgfsetdash{}{0pt}%
\pgfpathmoveto{\pgfqpoint{3.912184in}{3.944242in}}%
\pgfpathlineto{\pgfqpoint{3.925482in}{3.925751in}}%
\pgfpathlineto{\pgfqpoint{3.938778in}{3.907523in}}%
\pgfpathlineto{\pgfqpoint{3.952071in}{3.889556in}}%
\pgfpathlineto{\pgfqpoint{3.965363in}{3.871848in}}%
\pgfpathlineto{\pgfqpoint{3.973031in}{3.899995in}}%
\pgfpathlineto{\pgfqpoint{3.980698in}{3.928592in}}%
\pgfpathlineto{\pgfqpoint{3.988365in}{3.957648in}}%
\pgfpathlineto{\pgfqpoint{3.996033in}{3.987172in}}%
\pgfpathlineto{\pgfqpoint{3.982739in}{4.005601in}}%
\pgfpathlineto{\pgfqpoint{3.969443in}{4.024290in}}%
\pgfpathlineto{\pgfqpoint{3.956145in}{4.043241in}}%
\pgfpathlineto{\pgfqpoint{3.942845in}{4.062456in}}%
\pgfpathlineto{\pgfqpoint{3.935180in}{4.032196in}}%
\pgfpathlineto{\pgfqpoint{3.927515in}{4.002413in}}%
\pgfpathlineto{\pgfqpoint{3.919850in}{3.973098in}}%
\pgfpathlineto{\pgfqpoint{3.912184in}{3.944242in}}%
\pgfpathclose%
\pgfusepath{fill}%
\end{pgfscope}%
\begin{pgfscope}%
\pgfpathrectangle{\pgfqpoint{1.150000in}{0.150000in}}{\pgfqpoint{5.700000in}{5.700000in}}%
\pgfusepath{clip}%
\pgfsetbuttcap%
\pgfsetroundjoin%
\definecolor{currentfill}{rgb}{0.120638,0.625828,0.533488}%
\pgfsetfillcolor{currentfill}%
\pgfsetfillopacity{0.800000}%
\pgfsetlinewidth{0.000000pt}%
\definecolor{currentstroke}{rgb}{0.000000,0.000000,0.000000}%
\pgfsetstrokecolor{currentstroke}%
\pgfsetdash{}{0pt}%
\pgfpathmoveto{\pgfqpoint{3.996033in}{3.987172in}}%
\pgfpathlineto{\pgfqpoint{4.009325in}{3.969001in}}%
\pgfpathlineto{\pgfqpoint{4.022615in}{3.951086in}}%
\pgfpathlineto{\pgfqpoint{4.035904in}{3.933426in}}%
\pgfpathlineto{\pgfqpoint{4.049191in}{3.916018in}}%
\pgfpathlineto{\pgfqpoint{4.056861in}{3.945277in}}%
\pgfpathlineto{\pgfqpoint{4.064531in}{3.975011in}}%
\pgfpathlineto{\pgfqpoint{4.072203in}{4.005231in}}%
\pgfpathlineto{\pgfqpoint{4.079875in}{4.035946in}}%
\pgfpathlineto{\pgfqpoint{4.066585in}{4.054108in}}%
\pgfpathlineto{\pgfqpoint{4.053294in}{4.072523in}}%
\pgfpathlineto{\pgfqpoint{4.040001in}{4.091194in}}%
\pgfpathlineto{\pgfqpoint{4.026706in}{4.110122in}}%
\pgfpathlineto{\pgfqpoint{4.019036in}{4.078638in}}%
\pgfpathlineto{\pgfqpoint{4.011368in}{4.047658in}}%
\pgfpathlineto{\pgfqpoint{4.003700in}{4.017172in}}%
\pgfpathlineto{\pgfqpoint{3.996033in}{3.987172in}}%
\pgfpathclose%
\pgfusepath{fill}%
\end{pgfscope}%
\begin{pgfscope}%
\pgfpathrectangle{\pgfqpoint{1.150000in}{0.150000in}}{\pgfqpoint{5.700000in}{5.700000in}}%
\pgfusepath{clip}%
\pgfsetbuttcap%
\pgfsetroundjoin%
\definecolor{currentfill}{rgb}{0.129933,0.559582,0.551864}%
\pgfsetfillcolor{currentfill}%
\pgfsetfillopacity{0.800000}%
\pgfsetlinewidth{0.000000pt}%
\definecolor{currentstroke}{rgb}{0.000000,0.000000,0.000000}%
\pgfsetstrokecolor{currentstroke}%
\pgfsetdash{}{0pt}%
\pgfpathmoveto{\pgfqpoint{3.797613in}{3.800192in}}%
\pgfpathlineto{\pgfqpoint{3.810917in}{3.781974in}}%
\pgfpathlineto{\pgfqpoint{3.824220in}{3.764025in}}%
\pgfpathlineto{\pgfqpoint{3.837519in}{3.746344in}}%
\pgfpathlineto{\pgfqpoint{3.850817in}{3.728927in}}%
\pgfpathlineto{\pgfqpoint{3.858493in}{3.754407in}}%
\pgfpathlineto{\pgfqpoint{3.866168in}{3.780281in}}%
\pgfpathlineto{\pgfqpoint{3.873841in}{3.806557in}}%
\pgfpathlineto{\pgfqpoint{3.881512in}{3.833243in}}%
\pgfpathlineto{\pgfqpoint{3.868214in}{3.851311in}}%
\pgfpathlineto{\pgfqpoint{3.854913in}{3.869645in}}%
\pgfpathlineto{\pgfqpoint{3.841610in}{3.888246in}}%
\pgfpathlineto{\pgfqpoint{3.828303in}{3.907118in}}%
\pgfpathlineto{\pgfqpoint{3.820634in}{3.879767in}}%
\pgfpathlineto{\pgfqpoint{3.812962in}{3.852833in}}%
\pgfpathlineto{\pgfqpoint{3.805288in}{3.826311in}}%
\pgfpathlineto{\pgfqpoint{3.797613in}{3.800192in}}%
\pgfpathclose%
\pgfusepath{fill}%
\end{pgfscope}%
\begin{pgfscope}%
\pgfpathrectangle{\pgfqpoint{1.150000in}{0.150000in}}{\pgfqpoint{5.700000in}{5.700000in}}%
\pgfusepath{clip}%
\pgfsetbuttcap%
\pgfsetroundjoin%
\definecolor{currentfill}{rgb}{0.150476,0.504369,0.557430}%
\pgfsetfillcolor{currentfill}%
\pgfsetfillopacity{0.800000}%
\pgfsetlinewidth{0.000000pt}%
\definecolor{currentstroke}{rgb}{0.000000,0.000000,0.000000}%
\pgfsetstrokecolor{currentstroke}%
\pgfsetdash{}{0pt}%
\pgfpathmoveto{\pgfqpoint{4.314952in}{3.618260in}}%
\pgfpathlineto{\pgfqpoint{4.328254in}{3.605776in}}%
\pgfpathlineto{\pgfqpoint{4.341558in}{3.593510in}}%
\pgfpathlineto{\pgfqpoint{4.354865in}{3.581462in}}%
\pgfpathlineto{\pgfqpoint{4.368174in}{3.569630in}}%
\pgfpathlineto{\pgfqpoint{4.375839in}{3.594522in}}%
\pgfpathlineto{\pgfqpoint{4.383505in}{3.619835in}}%
\pgfpathlineto{\pgfqpoint{4.391174in}{3.645579in}}%
\pgfpathlineto{\pgfqpoint{4.398845in}{3.671762in}}%
\pgfpathlineto{\pgfqpoint{4.385538in}{3.684326in}}%
\pgfpathlineto{\pgfqpoint{4.372235in}{3.697107in}}%
\pgfpathlineto{\pgfqpoint{4.358933in}{3.710106in}}%
\pgfpathlineto{\pgfqpoint{4.345634in}{3.723325in}}%
\pgfpathlineto{\pgfqpoint{4.337961in}{3.696395in}}%
\pgfpathlineto{\pgfqpoint{4.330289in}{3.669914in}}%
\pgfpathlineto{\pgfqpoint{4.322620in}{3.643872in}}%
\pgfpathlineto{\pgfqpoint{4.314952in}{3.618260in}}%
\pgfpathclose%
\pgfusepath{fill}%
\end{pgfscope}%
\begin{pgfscope}%
\pgfpathrectangle{\pgfqpoint{1.150000in}{0.150000in}}{\pgfqpoint{5.700000in}{5.700000in}}%
\pgfusepath{clip}%
\pgfsetbuttcap%
\pgfsetroundjoin%
\definecolor{currentfill}{rgb}{0.139147,0.533812,0.555298}%
\pgfsetfillcolor{currentfill}%
\pgfsetfillopacity{0.800000}%
\pgfsetlinewidth{0.000000pt}%
\definecolor{currentstroke}{rgb}{0.000000,0.000000,0.000000}%
\pgfsetstrokecolor{currentstroke}%
\pgfsetdash{}{0pt}%
\pgfpathmoveto{\pgfqpoint{4.345634in}{3.723325in}}%
\pgfpathlineto{\pgfqpoint{4.358933in}{3.710106in}}%
\pgfpathlineto{\pgfqpoint{4.372235in}{3.697107in}}%
\pgfpathlineto{\pgfqpoint{4.385538in}{3.684326in}}%
\pgfpathlineto{\pgfqpoint{4.398845in}{3.671762in}}%
\pgfpathlineto{\pgfqpoint{4.406518in}{3.698394in}}%
\pgfpathlineto{\pgfqpoint{4.414194in}{3.725483in}}%
\pgfpathlineto{\pgfqpoint{4.421874in}{3.753039in}}%
\pgfpathlineto{\pgfqpoint{4.408570in}{3.766175in}}%
\pgfpathlineto{\pgfqpoint{4.395268in}{3.779529in}}%
\pgfpathlineto{\pgfqpoint{4.381968in}{3.793102in}}%
\pgfpathlineto{\pgfqpoint{4.368670in}{3.806895in}}%
\pgfpathlineto{\pgfqpoint{4.360989in}{3.778566in}}%
\pgfpathlineto{\pgfqpoint{4.353310in}{3.750712in}}%
\pgfpathlineto{\pgfqpoint{4.345634in}{3.723325in}}%
\pgfpathclose%
\pgfusepath{fill}%
\end{pgfscope}%
\begin{pgfscope}%
\pgfpathrectangle{\pgfqpoint{1.150000in}{0.150000in}}{\pgfqpoint{5.700000in}{5.700000in}}%
\pgfusepath{clip}%
\pgfsetbuttcap%
\pgfsetroundjoin%
\definecolor{currentfill}{rgb}{0.175841,0.441290,0.557685}%
\pgfsetfillcolor{currentfill}%
\pgfsetfillopacity{0.800000}%
\pgfsetlinewidth{0.000000pt}%
\definecolor{currentstroke}{rgb}{0.000000,0.000000,0.000000}%
\pgfsetstrokecolor{currentstroke}%
\pgfsetdash{}{0pt}%
\pgfpathmoveto{\pgfqpoint{3.979569in}{3.449067in}}%
\pgfpathlineto{\pgfqpoint{3.992857in}{3.435527in}}%
\pgfpathlineto{\pgfqpoint{4.006146in}{3.422225in}}%
\pgfpathlineto{\pgfqpoint{4.019435in}{3.409161in}}%
\pgfpathlineto{\pgfqpoint{4.032725in}{3.396332in}}%
\pgfpathlineto{\pgfqpoint{4.040403in}{3.417929in}}%
\pgfpathlineto{\pgfqpoint{4.048080in}{3.439855in}}%
\pgfpathlineto{\pgfqpoint{4.055755in}{3.462118in}}%
\pgfpathlineto{\pgfqpoint{4.063429in}{3.484725in}}%
\pgfpathlineto{\pgfqpoint{4.050141in}{3.498157in}}%
\pgfpathlineto{\pgfqpoint{4.036854in}{3.511826in}}%
\pgfpathlineto{\pgfqpoint{4.023568in}{3.525733in}}%
\pgfpathlineto{\pgfqpoint{4.010282in}{3.539879in}}%
\pgfpathlineto{\pgfqpoint{4.002606in}{3.516655in}}%
\pgfpathlineto{\pgfqpoint{3.994929in}{3.493783in}}%
\pgfpathlineto{\pgfqpoint{3.987250in}{3.471256in}}%
\pgfpathlineto{\pgfqpoint{3.979569in}{3.449067in}}%
\pgfpathclose%
\pgfusepath{fill}%
\end{pgfscope}%
\begin{pgfscope}%
\pgfpathrectangle{\pgfqpoint{1.150000in}{0.150000in}}{\pgfqpoint{5.700000in}{5.700000in}}%
\pgfusepath{clip}%
\pgfsetbuttcap%
\pgfsetroundjoin%
\definecolor{currentfill}{rgb}{0.120092,0.600104,0.542530}%
\pgfsetfillcolor{currentfill}%
\pgfsetfillopacity{0.800000}%
\pgfsetlinewidth{0.000000pt}%
\definecolor{currentstroke}{rgb}{0.000000,0.000000,0.000000}%
\pgfsetstrokecolor{currentstroke}%
\pgfsetdash{}{0pt}%
\pgfpathmoveto{\pgfqpoint{3.828303in}{3.907118in}}%
\pgfpathlineto{\pgfqpoint{3.841610in}{3.888246in}}%
\pgfpathlineto{\pgfqpoint{3.854913in}{3.869645in}}%
\pgfpathlineto{\pgfqpoint{3.868214in}{3.851311in}}%
\pgfpathlineto{\pgfqpoint{3.881512in}{3.833243in}}%
\pgfpathlineto{\pgfqpoint{3.889182in}{3.860346in}}%
\pgfpathlineto{\pgfqpoint{3.896850in}{3.887874in}}%
\pgfpathlineto{\pgfqpoint{3.904518in}{3.915837in}}%
\pgfpathlineto{\pgfqpoint{3.912184in}{3.944242in}}%
\pgfpathlineto{\pgfqpoint{3.898884in}{3.962998in}}%
\pgfpathlineto{\pgfqpoint{3.885582in}{3.982020in}}%
\pgfpathlineto{\pgfqpoint{3.872276in}{4.001312in}}%
\pgfpathlineto{\pgfqpoint{3.858967in}{4.020876in}}%
\pgfpathlineto{\pgfqpoint{3.851303in}{3.991767in}}%
\pgfpathlineto{\pgfqpoint{3.843638in}{3.963110in}}%
\pgfpathlineto{\pgfqpoint{3.835972in}{3.934897in}}%
\pgfpathlineto{\pgfqpoint{3.828303in}{3.907118in}}%
\pgfpathclose%
\pgfusepath{fill}%
\end{pgfscope}%
\begin{pgfscope}%
\pgfpathrectangle{\pgfqpoint{1.150000in}{0.150000in}}{\pgfqpoint{5.700000in}{5.700000in}}%
\pgfusepath{clip}%
\pgfsetbuttcap%
\pgfsetroundjoin%
\definecolor{currentfill}{rgb}{0.124780,0.640461,0.527068}%
\pgfsetfillcolor{currentfill}%
\pgfsetfillopacity{0.800000}%
\pgfsetlinewidth{0.000000pt}%
\definecolor{currentstroke}{rgb}{0.000000,0.000000,0.000000}%
\pgfsetstrokecolor{currentstroke}%
\pgfsetdash{}{0pt}%
\pgfpathmoveto{\pgfqpoint{4.079875in}{4.035946in}}%
\pgfpathlineto{\pgfqpoint{4.093164in}{4.018035in}}%
\pgfpathlineto{\pgfqpoint{4.106451in}{4.000375in}}%
\pgfpathlineto{\pgfqpoint{4.119738in}{3.982963in}}%
\pgfpathlineto{\pgfqpoint{4.133023in}{3.965797in}}%
\pgfpathlineto{\pgfqpoint{4.140699in}{3.996241in}}%
\pgfpathlineto{\pgfqpoint{4.148377in}{4.027188in}}%
\pgfpathlineto{\pgfqpoint{4.156057in}{4.058649in}}%
\pgfpathlineto{\pgfqpoint{4.142769in}{4.076402in}}%
\pgfpathlineto{\pgfqpoint{4.129481in}{4.094402in}}%
\pgfpathlineto{\pgfqpoint{4.116191in}{4.112651in}}%
\pgfpathlineto{\pgfqpoint{4.102900in}{4.131151in}}%
\pgfpathlineto{\pgfqpoint{4.095223in}{4.098896in}}%
\pgfpathlineto{\pgfqpoint{4.087548in}{4.067164in}}%
\pgfpathlineto{\pgfqpoint{4.079875in}{4.035946in}}%
\pgfpathclose%
\pgfusepath{fill}%
\end{pgfscope}%
\begin{pgfscope}%
\pgfpathrectangle{\pgfqpoint{1.150000in}{0.150000in}}{\pgfqpoint{5.700000in}{5.700000in}}%
\pgfusepath{clip}%
\pgfsetbuttcap%
\pgfsetroundjoin%
\definecolor{currentfill}{rgb}{0.141935,0.526453,0.555991}%
\pgfsetfillcolor{currentfill}%
\pgfsetfillopacity{0.800000}%
\pgfsetlinewidth{0.000000pt}%
\definecolor{currentstroke}{rgb}{0.000000,0.000000,0.000000}%
\pgfsetstrokecolor{currentstroke}%
\pgfsetdash{}{0pt}%
\pgfpathmoveto{\pgfqpoint{3.766886in}{3.699592in}}%
\pgfpathlineto{\pgfqpoint{3.780191in}{3.681992in}}%
\pgfpathlineto{\pgfqpoint{3.793493in}{3.664660in}}%
\pgfpathlineto{\pgfqpoint{3.806792in}{3.647595in}}%
\pgfpathlineto{\pgfqpoint{3.820090in}{3.630793in}}%
\pgfpathlineto{\pgfqpoint{3.827775in}{3.654773in}}%
\pgfpathlineto{\pgfqpoint{3.835458in}{3.679117in}}%
\pgfpathlineto{\pgfqpoint{3.843138in}{3.703833in}}%
\pgfpathlineto{\pgfqpoint{3.850817in}{3.728927in}}%
\pgfpathlineto{\pgfqpoint{3.837519in}{3.746344in}}%
\pgfpathlineto{\pgfqpoint{3.824220in}{3.764025in}}%
\pgfpathlineto{\pgfqpoint{3.810917in}{3.781974in}}%
\pgfpathlineto{\pgfqpoint{3.797613in}{3.800192in}}%
\pgfpathlineto{\pgfqpoint{3.789935in}{3.774468in}}%
\pgfpathlineto{\pgfqpoint{3.782254in}{3.749131in}}%
\pgfpathlineto{\pgfqpoint{3.774572in}{3.724175in}}%
\pgfpathlineto{\pgfqpoint{3.766886in}{3.699592in}}%
\pgfpathclose%
\pgfusepath{fill}%
\end{pgfscope}%
\begin{pgfscope}%
\pgfpathrectangle{\pgfqpoint{1.150000in}{0.150000in}}{\pgfqpoint{5.700000in}{5.700000in}}%
\pgfusepath{clip}%
\pgfsetbuttcap%
\pgfsetroundjoin%
\definecolor{currentfill}{rgb}{0.171176,0.452530,0.557965}%
\pgfsetfillcolor{currentfill}%
\pgfsetfillopacity{0.800000}%
\pgfsetlinewidth{0.000000pt}%
\definecolor{currentstroke}{rgb}{0.000000,0.000000,0.000000}%
\pgfsetstrokecolor{currentstroke}%
\pgfsetdash{}{0pt}%
\pgfpathmoveto{\pgfqpoint{4.200441in}{3.474526in}}%
\pgfpathlineto{\pgfqpoint{4.213739in}{3.462522in}}%
\pgfpathlineto{\pgfqpoint{4.227040in}{3.450741in}}%
\pgfpathlineto{\pgfqpoint{4.240343in}{3.439181in}}%
\pgfpathlineto{\pgfqpoint{4.253649in}{3.427842in}}%
\pgfpathlineto{\pgfqpoint{4.261310in}{3.450314in}}%
\pgfpathlineto{\pgfqpoint{4.268972in}{3.473150in}}%
\pgfpathlineto{\pgfqpoint{4.276633in}{3.496358in}}%
\pgfpathlineto{\pgfqpoint{4.284295in}{3.519946in}}%
\pgfpathlineto{\pgfqpoint{4.270993in}{3.531951in}}%
\pgfpathlineto{\pgfqpoint{4.257693in}{3.544176in}}%
\pgfpathlineto{\pgfqpoint{4.244395in}{3.556624in}}%
\pgfpathlineto{\pgfqpoint{4.231100in}{3.569295in}}%
\pgfpathlineto{\pgfqpoint{4.223435in}{3.545028in}}%
\pgfpathlineto{\pgfqpoint{4.215770in}{3.521150in}}%
\pgfpathlineto{\pgfqpoint{4.208105in}{3.497652in}}%
\pgfpathlineto{\pgfqpoint{4.200441in}{3.474526in}}%
\pgfpathclose%
\pgfusepath{fill}%
\end{pgfscope}%
\begin{pgfscope}%
\pgfpathrectangle{\pgfqpoint{1.150000in}{0.150000in}}{\pgfqpoint{5.700000in}{5.700000in}}%
\pgfusepath{clip}%
\pgfsetbuttcap%
\pgfsetroundjoin%
\definecolor{currentfill}{rgb}{0.177423,0.437527,0.557565}%
\pgfsetfillcolor{currentfill}%
\pgfsetfillopacity{0.800000}%
\pgfsetlinewidth{0.000000pt}%
\definecolor{currentstroke}{rgb}{0.000000,0.000000,0.000000}%
\pgfsetstrokecolor{currentstroke}%
\pgfsetdash{}{0pt}%
\pgfpathmoveto{\pgfqpoint{4.116590in}{3.433327in}}%
\pgfpathlineto{\pgfqpoint{4.129885in}{3.421054in}}%
\pgfpathlineto{\pgfqpoint{4.143180in}{3.409008in}}%
\pgfpathlineto{\pgfqpoint{4.156478in}{3.397189in}}%
\pgfpathlineto{\pgfqpoint{4.169778in}{3.385594in}}%
\pgfpathlineto{\pgfqpoint{4.177445in}{3.407307in}}%
\pgfpathlineto{\pgfqpoint{4.185110in}{3.429362in}}%
\pgfpathlineto{\pgfqpoint{4.192776in}{3.451766in}}%
\pgfpathlineto{\pgfqpoint{4.200441in}{3.474526in}}%
\pgfpathlineto{\pgfqpoint{4.187144in}{3.486754in}}%
\pgfpathlineto{\pgfqpoint{4.173849in}{3.499208in}}%
\pgfpathlineto{\pgfqpoint{4.160557in}{3.511888in}}%
\pgfpathlineto{\pgfqpoint{4.147265in}{3.524797in}}%
\pgfpathlineto{\pgfqpoint{4.139598in}{3.501389in}}%
\pgfpathlineto{\pgfqpoint{4.131929in}{3.478346in}}%
\pgfpathlineto{\pgfqpoint{4.124260in}{3.455662in}}%
\pgfpathlineto{\pgfqpoint{4.116590in}{3.433327in}}%
\pgfpathclose%
\pgfusepath{fill}%
\end{pgfscope}%
\begin{pgfscope}%
\pgfpathrectangle{\pgfqpoint{1.150000in}{0.150000in}}{\pgfqpoint{5.700000in}{5.700000in}}%
\pgfusepath{clip}%
\pgfsetbuttcap%
\pgfsetroundjoin%
\definecolor{currentfill}{rgb}{0.163625,0.471133,0.558148}%
\pgfsetfillcolor{currentfill}%
\pgfsetfillopacity{0.800000}%
\pgfsetlinewidth{0.000000pt}%
\definecolor{currentstroke}{rgb}{0.000000,0.000000,0.000000}%
\pgfsetstrokecolor{currentstroke}%
\pgfsetdash{}{0pt}%
\pgfpathmoveto{\pgfqpoint{4.284295in}{3.519946in}}%
\pgfpathlineto{\pgfqpoint{4.297600in}{3.508160in}}%
\pgfpathlineto{\pgfqpoint{4.310908in}{3.496593in}}%
\pgfpathlineto{\pgfqpoint{4.324218in}{3.485243in}}%
\pgfpathlineto{\pgfqpoint{4.337532in}{3.474109in}}%
\pgfpathlineto{\pgfqpoint{4.345190in}{3.497399in}}%
\pgfpathlineto{\pgfqpoint{4.352850in}{3.521077in}}%
\pgfpathlineto{\pgfqpoint{4.360512in}{3.545152in}}%
\pgfpathlineto{\pgfqpoint{4.368174in}{3.569630in}}%
\pgfpathlineto{\pgfqpoint{4.354865in}{3.581462in}}%
\pgfpathlineto{\pgfqpoint{4.341558in}{3.593510in}}%
\pgfpathlineto{\pgfqpoint{4.328254in}{3.605776in}}%
\pgfpathlineto{\pgfqpoint{4.314952in}{3.618260in}}%
\pgfpathlineto{\pgfqpoint{4.307286in}{3.593070in}}%
\pgfpathlineto{\pgfqpoint{4.299622in}{3.568294in}}%
\pgfpathlineto{\pgfqpoint{4.291958in}{3.543921in}}%
\pgfpathlineto{\pgfqpoint{4.284295in}{3.519946in}}%
\pgfpathclose%
\pgfusepath{fill}%
\end{pgfscope}%
\begin{pgfscope}%
\pgfpathrectangle{\pgfqpoint{1.150000in}{0.150000in}}{\pgfqpoint{5.700000in}{5.700000in}}%
\pgfusepath{clip}%
\pgfsetbuttcap%
\pgfsetroundjoin%
\definecolor{currentfill}{rgb}{0.171176,0.452530,0.557965}%
\pgfsetfillcolor{currentfill}%
\pgfsetfillopacity{0.800000}%
\pgfsetlinewidth{0.000000pt}%
\definecolor{currentstroke}{rgb}{0.000000,0.000000,0.000000}%
\pgfsetstrokecolor{currentstroke}%
\pgfsetdash{}{0pt}%
\pgfpathmoveto{\pgfqpoint{3.842502in}{3.476078in}}%
\pgfpathlineto{\pgfqpoint{3.855794in}{3.461144in}}%
\pgfpathlineto{\pgfqpoint{3.869085in}{3.446461in}}%
\pgfpathlineto{\pgfqpoint{3.882375in}{3.432028in}}%
\pgfpathlineto{\pgfqpoint{3.895665in}{3.417844in}}%
\pgfpathlineto{\pgfqpoint{3.903357in}{3.439310in}}%
\pgfpathlineto{\pgfqpoint{3.911047in}{3.461096in}}%
\pgfpathlineto{\pgfqpoint{3.918734in}{3.483207in}}%
\pgfpathlineto{\pgfqpoint{3.926419in}{3.505650in}}%
\pgfpathlineto{\pgfqpoint{3.913131in}{3.520409in}}%
\pgfpathlineto{\pgfqpoint{3.899842in}{3.535417in}}%
\pgfpathlineto{\pgfqpoint{3.886553in}{3.550676in}}%
\pgfpathlineto{\pgfqpoint{3.873263in}{3.566187in}}%
\pgfpathlineto{\pgfqpoint{3.865577in}{3.543155in}}%
\pgfpathlineto{\pgfqpoint{3.857888in}{3.520464in}}%
\pgfpathlineto{\pgfqpoint{3.850196in}{3.498108in}}%
\pgfpathlineto{\pgfqpoint{3.842502in}{3.476078in}}%
\pgfpathclose%
\pgfusepath{fill}%
\end{pgfscope}%
\begin{pgfscope}%
\pgfpathrectangle{\pgfqpoint{1.150000in}{0.150000in}}{\pgfqpoint{5.700000in}{5.700000in}}%
\pgfusepath{clip}%
\pgfsetbuttcap%
\pgfsetroundjoin%
\definecolor{currentfill}{rgb}{0.163625,0.471133,0.558148}%
\pgfsetfillcolor{currentfill}%
\pgfsetfillopacity{0.800000}%
\pgfsetlinewidth{0.000000pt}%
\definecolor{currentstroke}{rgb}{0.000000,0.000000,0.000000}%
\pgfsetstrokecolor{currentstroke}%
\pgfsetdash{}{0pt}%
\pgfpathmoveto{\pgfqpoint{3.789323in}{3.538371in}}%
\pgfpathlineto{\pgfqpoint{3.802620in}{3.522411in}}%
\pgfpathlineto{\pgfqpoint{3.815915in}{3.506710in}}%
\pgfpathlineto{\pgfqpoint{3.829209in}{3.491266in}}%
\pgfpathlineto{\pgfqpoint{3.842502in}{3.476078in}}%
\pgfpathlineto{\pgfqpoint{3.850196in}{3.498108in}}%
\pgfpathlineto{\pgfqpoint{3.857888in}{3.520464in}}%
\pgfpathlineto{\pgfqpoint{3.865577in}{3.543155in}}%
\pgfpathlineto{\pgfqpoint{3.873263in}{3.566187in}}%
\pgfpathlineto{\pgfqpoint{3.859972in}{3.581953in}}%
\pgfpathlineto{\pgfqpoint{3.846680in}{3.597974in}}%
\pgfpathlineto{\pgfqpoint{3.833386in}{3.614254in}}%
\pgfpathlineto{\pgfqpoint{3.820090in}{3.630793in}}%
\pgfpathlineto{\pgfqpoint{3.812402in}{3.607170in}}%
\pgfpathlineto{\pgfqpoint{3.804712in}{3.583896in}}%
\pgfpathlineto{\pgfqpoint{3.797019in}{3.560966in}}%
\pgfpathlineto{\pgfqpoint{3.789323in}{3.538371in}}%
\pgfpathclose%
\pgfusepath{fill}%
\end{pgfscope}%
\begin{pgfscope}%
\pgfpathrectangle{\pgfqpoint{1.150000in}{0.150000in}}{\pgfqpoint{5.700000in}{5.700000in}}%
\pgfusepath{clip}%
\pgfsetbuttcap%
\pgfsetroundjoin%
\definecolor{currentfill}{rgb}{0.144759,0.519093,0.556572}%
\pgfsetfillcolor{currentfill}%
\pgfsetfillopacity{0.800000}%
\pgfsetlinewidth{0.000000pt}%
\definecolor{currentstroke}{rgb}{0.000000,0.000000,0.000000}%
\pgfsetstrokecolor{currentstroke}%
\pgfsetdash{}{0pt}%
\pgfpathmoveto{\pgfqpoint{4.398845in}{3.671762in}}%
\pgfpathlineto{\pgfqpoint{4.412154in}{3.659414in}}%
\pgfpathlineto{\pgfqpoint{4.425465in}{3.647281in}}%
\pgfpathlineto{\pgfqpoint{4.438780in}{3.635361in}}%
\pgfpathlineto{\pgfqpoint{4.452099in}{3.623654in}}%
\pgfpathlineto{\pgfqpoint{4.459769in}{3.649533in}}%
\pgfpathlineto{\pgfqpoint{4.467442in}{3.675861in}}%
\pgfpathlineto{\pgfqpoint{4.475118in}{3.702646in}}%
\pgfpathlineto{\pgfqpoint{4.461803in}{3.714924in}}%
\pgfpathlineto{\pgfqpoint{4.448490in}{3.727415in}}%
\pgfpathlineto{\pgfqpoint{4.435181in}{3.740119in}}%
\pgfpathlineto{\pgfqpoint{4.421874in}{3.753039in}}%
\pgfpathlineto{\pgfqpoint{4.414194in}{3.725483in}}%
\pgfpathlineto{\pgfqpoint{4.406518in}{3.698394in}}%
\pgfpathlineto{\pgfqpoint{4.398845in}{3.671762in}}%
\pgfpathclose%
\pgfusepath{fill}%
\end{pgfscope}%
\begin{pgfscope}%
\pgfpathrectangle{\pgfqpoint{1.150000in}{0.150000in}}{\pgfqpoint{5.700000in}{5.700000in}}%
\pgfusepath{clip}%
\pgfsetbuttcap%
\pgfsetroundjoin%
\definecolor{currentfill}{rgb}{0.182256,0.426184,0.557120}%
\pgfsetfillcolor{currentfill}%
\pgfsetfillopacity{0.800000}%
\pgfsetlinewidth{0.000000pt}%
\definecolor{currentstroke}{rgb}{0.000000,0.000000,0.000000}%
\pgfsetstrokecolor{currentstroke}%
\pgfsetdash{}{0pt}%
\pgfpathmoveto{\pgfqpoint{4.032725in}{3.396332in}}%
\pgfpathlineto{\pgfqpoint{4.046017in}{3.383738in}}%
\pgfpathlineto{\pgfqpoint{4.059309in}{3.371376in}}%
\pgfpathlineto{\pgfqpoint{4.072604in}{3.359246in}}%
\pgfpathlineto{\pgfqpoint{4.085899in}{3.347345in}}%
\pgfpathlineto{\pgfqpoint{4.093574in}{3.368352in}}%
\pgfpathlineto{\pgfqpoint{4.101248in}{3.389680in}}%
\pgfpathlineto{\pgfqpoint{4.108920in}{3.411336in}}%
\pgfpathlineto{\pgfqpoint{4.116590in}{3.433327in}}%
\pgfpathlineto{\pgfqpoint{4.103298in}{3.445830in}}%
\pgfpathlineto{\pgfqpoint{4.090007in}{3.458562in}}%
\pgfpathlineto{\pgfqpoint{4.076717in}{3.471527in}}%
\pgfpathlineto{\pgfqpoint{4.063429in}{3.484725in}}%
\pgfpathlineto{\pgfqpoint{4.055755in}{3.462118in}}%
\pgfpathlineto{\pgfqpoint{4.048080in}{3.439855in}}%
\pgfpathlineto{\pgfqpoint{4.040403in}{3.417929in}}%
\pgfpathlineto{\pgfqpoint{4.032725in}{3.396332in}}%
\pgfpathclose%
\pgfusepath{fill}%
\end{pgfscope}%
\begin{pgfscope}%
\pgfpathrectangle{\pgfqpoint{1.150000in}{0.150000in}}{\pgfqpoint{5.700000in}{5.700000in}}%
\pgfusepath{clip}%
\pgfsetbuttcap%
\pgfsetroundjoin%
\definecolor{currentfill}{rgb}{0.132268,0.655014,0.519661}%
\pgfsetfillcolor{currentfill}%
\pgfsetfillopacity{0.800000}%
\pgfsetlinewidth{0.000000pt}%
\definecolor{currentstroke}{rgb}{0.000000,0.000000,0.000000}%
\pgfsetstrokecolor{currentstroke}%
\pgfsetdash{}{0pt}%
\pgfpathmoveto{\pgfqpoint{3.942845in}{4.062456in}}%
\pgfpathlineto{\pgfqpoint{3.956145in}{4.043241in}}%
\pgfpathlineto{\pgfqpoint{3.969443in}{4.024290in}}%
\pgfpathlineto{\pgfqpoint{3.982739in}{4.005601in}}%
\pgfpathlineto{\pgfqpoint{3.996033in}{3.987172in}}%
\pgfpathlineto{\pgfqpoint{4.003700in}{4.017172in}}%
\pgfpathlineto{\pgfqpoint{4.011368in}{4.047658in}}%
\pgfpathlineto{\pgfqpoint{4.019036in}{4.078638in}}%
\pgfpathlineto{\pgfqpoint{4.026706in}{4.110122in}}%
\pgfpathlineto{\pgfqpoint{4.013409in}{4.129310in}}%
\pgfpathlineto{\pgfqpoint{4.000109in}{4.148759in}}%
\pgfpathlineto{\pgfqpoint{3.986808in}{4.168471in}}%
\pgfpathlineto{\pgfqpoint{3.973503in}{4.188449in}}%
\pgfpathlineto{\pgfqpoint{3.965838in}{4.156189in}}%
\pgfpathlineto{\pgfqpoint{3.958173in}{4.124444in}}%
\pgfpathlineto{\pgfqpoint{3.950509in}{4.093202in}}%
\pgfpathlineto{\pgfqpoint{3.942845in}{4.062456in}}%
\pgfpathclose%
\pgfusepath{fill}%
\end{pgfscope}%
\begin{pgfscope}%
\pgfpathrectangle{\pgfqpoint{1.150000in}{0.150000in}}{\pgfqpoint{5.700000in}{5.700000in}}%
\pgfusepath{clip}%
\pgfsetbuttcap%
\pgfsetroundjoin%
\definecolor{currentfill}{rgb}{0.124780,0.640461,0.527068}%
\pgfsetfillcolor{currentfill}%
\pgfsetfillopacity{0.800000}%
\pgfsetlinewidth{0.000000pt}%
\definecolor{currentstroke}{rgb}{0.000000,0.000000,0.000000}%
\pgfsetstrokecolor{currentstroke}%
\pgfsetdash{}{0pt}%
\pgfpathmoveto{\pgfqpoint{3.858967in}{4.020876in}}%
\pgfpathlineto{\pgfqpoint{3.872276in}{4.001312in}}%
\pgfpathlineto{\pgfqpoint{3.885582in}{3.982020in}}%
\pgfpathlineto{\pgfqpoint{3.898884in}{3.962998in}}%
\pgfpathlineto{\pgfqpoint{3.912184in}{3.944242in}}%
\pgfpathlineto{\pgfqpoint{3.919850in}{3.973098in}}%
\pgfpathlineto{\pgfqpoint{3.927515in}{4.002413in}}%
\pgfpathlineto{\pgfqpoint{3.935180in}{4.032196in}}%
\pgfpathlineto{\pgfqpoint{3.942845in}{4.062456in}}%
\pgfpathlineto{\pgfqpoint{3.929541in}{4.081937in}}%
\pgfpathlineto{\pgfqpoint{3.916236in}{4.101687in}}%
\pgfpathlineto{\pgfqpoint{3.902927in}{4.121706in}}%
\pgfpathlineto{\pgfqpoint{3.889615in}{4.141999in}}%
\pgfpathlineto{\pgfqpoint{3.881954in}{4.110997in}}%
\pgfpathlineto{\pgfqpoint{3.874293in}{4.080482in}}%
\pgfpathlineto{\pgfqpoint{3.866630in}{4.050444in}}%
\pgfpathlineto{\pgfqpoint{3.858967in}{4.020876in}}%
\pgfpathclose%
\pgfusepath{fill}%
\end{pgfscope}%
\begin{pgfscope}%
\pgfpathrectangle{\pgfqpoint{1.150000in}{0.150000in}}{\pgfqpoint{5.700000in}{5.700000in}}%
\pgfusepath{clip}%
\pgfsetbuttcap%
\pgfsetroundjoin%
\definecolor{currentfill}{rgb}{0.121831,0.589055,0.545623}%
\pgfsetfillcolor{currentfill}%
\pgfsetfillopacity{0.800000}%
\pgfsetlinewidth{0.000000pt}%
\definecolor{currentstroke}{rgb}{0.000000,0.000000,0.000000}%
\pgfsetstrokecolor{currentstroke}%
\pgfsetdash{}{0pt}%
\pgfpathmoveto{\pgfqpoint{3.744363in}{3.875797in}}%
\pgfpathlineto{\pgfqpoint{3.757680in}{3.856481in}}%
\pgfpathlineto{\pgfqpoint{3.770994in}{3.837443in}}%
\pgfpathlineto{\pgfqpoint{3.784305in}{3.818681in}}%
\pgfpathlineto{\pgfqpoint{3.797613in}{3.800192in}}%
\pgfpathlineto{\pgfqpoint{3.805288in}{3.826311in}}%
\pgfpathlineto{\pgfqpoint{3.812962in}{3.852833in}}%
\pgfpathlineto{\pgfqpoint{3.820634in}{3.879767in}}%
\pgfpathlineto{\pgfqpoint{3.828303in}{3.907118in}}%
\pgfpathlineto{\pgfqpoint{3.814994in}{3.926263in}}%
\pgfpathlineto{\pgfqpoint{3.801682in}{3.945681in}}%
\pgfpathlineto{\pgfqpoint{3.788366in}{3.965377in}}%
\pgfpathlineto{\pgfqpoint{3.775046in}{3.985351in}}%
\pgfpathlineto{\pgfqpoint{3.767379in}{3.957329in}}%
\pgfpathlineto{\pgfqpoint{3.759709in}{3.929734in}}%
\pgfpathlineto{\pgfqpoint{3.752037in}{3.902560in}}%
\pgfpathlineto{\pgfqpoint{3.744363in}{3.875797in}}%
\pgfpathclose%
\pgfusepath{fill}%
\end{pgfscope}%
\begin{pgfscope}%
\pgfpathrectangle{\pgfqpoint{1.150000in}{0.150000in}}{\pgfqpoint{5.700000in}{5.700000in}}%
\pgfusepath{clip}%
\pgfsetbuttcap%
\pgfsetroundjoin%
\definecolor{currentfill}{rgb}{0.156270,0.489624,0.557936}%
\pgfsetfillcolor{currentfill}%
\pgfsetfillopacity{0.800000}%
\pgfsetlinewidth{0.000000pt}%
\definecolor{currentstroke}{rgb}{0.000000,0.000000,0.000000}%
\pgfsetstrokecolor{currentstroke}%
\pgfsetdash{}{0pt}%
\pgfpathmoveto{\pgfqpoint{4.368174in}{3.569630in}}%
\pgfpathlineto{\pgfqpoint{4.381487in}{3.558014in}}%
\pgfpathlineto{\pgfqpoint{4.394803in}{3.546611in}}%
\pgfpathlineto{\pgfqpoint{4.408121in}{3.535422in}}%
\pgfpathlineto{\pgfqpoint{4.421444in}{3.524445in}}%
\pgfpathlineto{\pgfqpoint{4.429104in}{3.548619in}}%
\pgfpathlineto{\pgfqpoint{4.436767in}{3.573206in}}%
\pgfpathlineto{\pgfqpoint{4.444431in}{3.598214in}}%
\pgfpathlineto{\pgfqpoint{4.452099in}{3.623654in}}%
\pgfpathlineto{\pgfqpoint{4.438780in}{3.635361in}}%
\pgfpathlineto{\pgfqpoint{4.425465in}{3.647281in}}%
\pgfpathlineto{\pgfqpoint{4.412154in}{3.659414in}}%
\pgfpathlineto{\pgfqpoint{4.398845in}{3.671762in}}%
\pgfpathlineto{\pgfqpoint{4.391174in}{3.645579in}}%
\pgfpathlineto{\pgfqpoint{4.383505in}{3.619835in}}%
\pgfpathlineto{\pgfqpoint{4.375839in}{3.594522in}}%
\pgfpathlineto{\pgfqpoint{4.368174in}{3.569630in}}%
\pgfpathclose%
\pgfusepath{fill}%
\end{pgfscope}%
\begin{pgfscope}%
\pgfpathrectangle{\pgfqpoint{1.150000in}{0.150000in}}{\pgfqpoint{5.700000in}{5.700000in}}%
\pgfusepath{clip}%
\pgfsetbuttcap%
\pgfsetroundjoin%
\definecolor{currentfill}{rgb}{0.180629,0.429975,0.557282}%
\pgfsetfillcolor{currentfill}%
\pgfsetfillopacity{0.800000}%
\pgfsetlinewidth{0.000000pt}%
\definecolor{currentstroke}{rgb}{0.000000,0.000000,0.000000}%
\pgfsetstrokecolor{currentstroke}%
\pgfsetdash{}{0pt}%
\pgfpathmoveto{\pgfqpoint{3.895665in}{3.417844in}}%
\pgfpathlineto{\pgfqpoint{3.908955in}{3.403906in}}%
\pgfpathlineto{\pgfqpoint{3.922245in}{3.390212in}}%
\pgfpathlineto{\pgfqpoint{3.935535in}{3.376762in}}%
\pgfpathlineto{\pgfqpoint{3.948825in}{3.363554in}}%
\pgfpathlineto{\pgfqpoint{3.956514in}{3.384460in}}%
\pgfpathlineto{\pgfqpoint{3.964201in}{3.405676in}}%
\pgfpathlineto{\pgfqpoint{3.971886in}{3.427210in}}%
\pgfpathlineto{\pgfqpoint{3.979569in}{3.449067in}}%
\pgfpathlineto{\pgfqpoint{3.966281in}{3.462848in}}%
\pgfpathlineto{\pgfqpoint{3.952994in}{3.476871in}}%
\pgfpathlineto{\pgfqpoint{3.939706in}{3.491138in}}%
\pgfpathlineto{\pgfqpoint{3.926419in}{3.505650in}}%
\pgfpathlineto{\pgfqpoint{3.918734in}{3.483207in}}%
\pgfpathlineto{\pgfqpoint{3.911047in}{3.461096in}}%
\pgfpathlineto{\pgfqpoint{3.903357in}{3.439310in}}%
\pgfpathlineto{\pgfqpoint{3.895665in}{3.417844in}}%
\pgfpathclose%
\pgfusepath{fill}%
\end{pgfscope}%
\begin{pgfscope}%
\pgfpathrectangle{\pgfqpoint{1.150000in}{0.150000in}}{\pgfqpoint{5.700000in}{5.700000in}}%
\pgfusepath{clip}%
\pgfsetbuttcap%
\pgfsetroundjoin%
\definecolor{currentfill}{rgb}{0.154815,0.493313,0.557840}%
\pgfsetfillcolor{currentfill}%
\pgfsetfillopacity{0.800000}%
\pgfsetlinewidth{0.000000pt}%
\definecolor{currentstroke}{rgb}{0.000000,0.000000,0.000000}%
\pgfsetstrokecolor{currentstroke}%
\pgfsetdash{}{0pt}%
\pgfpathmoveto{\pgfqpoint{3.736116in}{3.604844in}}%
\pgfpathlineto{\pgfqpoint{3.749421in}{3.587827in}}%
\pgfpathlineto{\pgfqpoint{3.762724in}{3.571077in}}%
\pgfpathlineto{\pgfqpoint{3.776024in}{3.554592in}}%
\pgfpathlineto{\pgfqpoint{3.789323in}{3.538371in}}%
\pgfpathlineto{\pgfqpoint{3.797019in}{3.560966in}}%
\pgfpathlineto{\pgfqpoint{3.804712in}{3.583896in}}%
\pgfpathlineto{\pgfqpoint{3.812402in}{3.607170in}}%
\pgfpathlineto{\pgfqpoint{3.820090in}{3.630793in}}%
\pgfpathlineto{\pgfqpoint{3.806792in}{3.647595in}}%
\pgfpathlineto{\pgfqpoint{3.793493in}{3.664660in}}%
\pgfpathlineto{\pgfqpoint{3.780191in}{3.681992in}}%
\pgfpathlineto{\pgfqpoint{3.766886in}{3.699592in}}%
\pgfpathlineto{\pgfqpoint{3.759198in}{3.675374in}}%
\pgfpathlineto{\pgfqpoint{3.751507in}{3.651515in}}%
\pgfpathlineto{\pgfqpoint{3.743813in}{3.628007in}}%
\pgfpathlineto{\pgfqpoint{3.736116in}{3.604844in}}%
\pgfpathclose%
\pgfusepath{fill}%
\end{pgfscope}%
\begin{pgfscope}%
\pgfpathrectangle{\pgfqpoint{1.150000in}{0.150000in}}{\pgfqpoint{5.700000in}{5.700000in}}%
\pgfusepath{clip}%
\pgfsetbuttcap%
\pgfsetroundjoin%
\definecolor{currentfill}{rgb}{0.140210,0.665859,0.513427}%
\pgfsetfillcolor{currentfill}%
\pgfsetfillopacity{0.800000}%
\pgfsetlinewidth{0.000000pt}%
\definecolor{currentstroke}{rgb}{0.000000,0.000000,0.000000}%
\pgfsetstrokecolor{currentstroke}%
\pgfsetdash{}{0pt}%
\pgfpathmoveto{\pgfqpoint{4.026706in}{4.110122in}}%
\pgfpathlineto{\pgfqpoint{4.040001in}{4.091194in}}%
\pgfpathlineto{\pgfqpoint{4.053294in}{4.072523in}}%
\pgfpathlineto{\pgfqpoint{4.066585in}{4.054108in}}%
\pgfpathlineto{\pgfqpoint{4.079875in}{4.035946in}}%
\pgfpathlineto{\pgfqpoint{4.087548in}{4.067164in}}%
\pgfpathlineto{\pgfqpoint{4.095223in}{4.098896in}}%
\pgfpathlineto{\pgfqpoint{4.102900in}{4.131151in}}%
\pgfpathlineto{\pgfqpoint{4.089608in}{4.149903in}}%
\pgfpathlineto{\pgfqpoint{4.076314in}{4.168910in}}%
\pgfpathlineto{\pgfqpoint{4.063018in}{4.188174in}}%
\pgfpathlineto{\pgfqpoint{4.049720in}{4.207696in}}%
\pgfpathlineto{\pgfqpoint{4.042047in}{4.174642in}}%
\pgfpathlineto{\pgfqpoint{4.034376in}{4.142120in}}%
\pgfpathlineto{\pgfqpoint{4.026706in}{4.110122in}}%
\pgfpathclose%
\pgfusepath{fill}%
\end{pgfscope}%
\begin{pgfscope}%
\pgfpathrectangle{\pgfqpoint{1.150000in}{0.150000in}}{\pgfqpoint{5.700000in}{5.700000in}}%
\pgfusepath{clip}%
\pgfsetbuttcap%
\pgfsetroundjoin%
\definecolor{currentfill}{rgb}{0.132444,0.552216,0.553018}%
\pgfsetfillcolor{currentfill}%
\pgfsetfillopacity{0.800000}%
\pgfsetlinewidth{0.000000pt}%
\definecolor{currentstroke}{rgb}{0.000000,0.000000,0.000000}%
\pgfsetstrokecolor{currentstroke}%
\pgfsetdash{}{0pt}%
\pgfpathmoveto{\pgfqpoint{3.713639in}{3.772715in}}%
\pgfpathlineto{\pgfqpoint{3.726956in}{3.754021in}}%
\pgfpathlineto{\pgfqpoint{3.740269in}{3.735604in}}%
\pgfpathlineto{\pgfqpoint{3.753579in}{3.717462in}}%
\pgfpathlineto{\pgfqpoint{3.766886in}{3.699592in}}%
\pgfpathlineto{\pgfqpoint{3.774572in}{3.724175in}}%
\pgfpathlineto{\pgfqpoint{3.782254in}{3.749131in}}%
\pgfpathlineto{\pgfqpoint{3.789935in}{3.774468in}}%
\pgfpathlineto{\pgfqpoint{3.797613in}{3.800192in}}%
\pgfpathlineto{\pgfqpoint{3.784305in}{3.818681in}}%
\pgfpathlineto{\pgfqpoint{3.770994in}{3.837443in}}%
\pgfpathlineto{\pgfqpoint{3.757680in}{3.856481in}}%
\pgfpathlineto{\pgfqpoint{3.744363in}{3.875797in}}%
\pgfpathlineto{\pgfqpoint{3.736686in}{3.849439in}}%
\pgfpathlineto{\pgfqpoint{3.729006in}{3.823478in}}%
\pgfpathlineto{\pgfqpoint{3.721324in}{3.797906in}}%
\pgfpathlineto{\pgfqpoint{3.713639in}{3.772715in}}%
\pgfpathclose%
\pgfusepath{fill}%
\end{pgfscope}%
\begin{pgfscope}%
\pgfpathrectangle{\pgfqpoint{1.150000in}{0.150000in}}{\pgfqpoint{5.700000in}{5.700000in}}%
\pgfusepath{clip}%
\pgfsetbuttcap%
\pgfsetroundjoin%
\definecolor{currentfill}{rgb}{0.120638,0.625828,0.533488}%
\pgfsetfillcolor{currentfill}%
\pgfsetfillopacity{0.800000}%
\pgfsetlinewidth{0.000000pt}%
\definecolor{currentstroke}{rgb}{0.000000,0.000000,0.000000}%
\pgfsetstrokecolor{currentstroke}%
\pgfsetdash{}{0pt}%
\pgfpathmoveto{\pgfqpoint{3.775046in}{3.985351in}}%
\pgfpathlineto{\pgfqpoint{3.788366in}{3.965377in}}%
\pgfpathlineto{\pgfqpoint{3.801682in}{3.945681in}}%
\pgfpathlineto{\pgfqpoint{3.814994in}{3.926263in}}%
\pgfpathlineto{\pgfqpoint{3.828303in}{3.907118in}}%
\pgfpathlineto{\pgfqpoint{3.835972in}{3.934897in}}%
\pgfpathlineto{\pgfqpoint{3.843638in}{3.963110in}}%
\pgfpathlineto{\pgfqpoint{3.851303in}{3.991767in}}%
\pgfpathlineto{\pgfqpoint{3.858967in}{4.020876in}}%
\pgfpathlineto{\pgfqpoint{3.845656in}{4.040712in}}%
\pgfpathlineto{\pgfqpoint{3.832341in}{4.060825in}}%
\pgfpathlineto{\pgfqpoint{3.819022in}{4.081215in}}%
\pgfpathlineto{\pgfqpoint{3.805699in}{4.101886in}}%
\pgfpathlineto{\pgfqpoint{3.798038in}{4.072069in}}%
\pgfpathlineto{\pgfqpoint{3.790376in}{4.042713in}}%
\pgfpathlineto{\pgfqpoint{3.782712in}{4.013810in}}%
\pgfpathlineto{\pgfqpoint{3.775046in}{3.985351in}}%
\pgfpathclose%
\pgfusepath{fill}%
\end{pgfscope}%
\begin{pgfscope}%
\pgfpathrectangle{\pgfqpoint{1.150000in}{0.150000in}}{\pgfqpoint{5.700000in}{5.700000in}}%
\pgfusepath{clip}%
\pgfsetbuttcap%
\pgfsetroundjoin%
\definecolor{currentfill}{rgb}{0.177423,0.437527,0.557565}%
\pgfsetfillcolor{currentfill}%
\pgfsetfillopacity{0.800000}%
\pgfsetlinewidth{0.000000pt}%
\definecolor{currentstroke}{rgb}{0.000000,0.000000,0.000000}%
\pgfsetstrokecolor{currentstroke}%
\pgfsetdash{}{0pt}%
\pgfpathmoveto{\pgfqpoint{4.253649in}{3.427842in}}%
\pgfpathlineto{\pgfqpoint{4.266958in}{3.416721in}}%
\pgfpathlineto{\pgfqpoint{4.280270in}{3.405818in}}%
\pgfpathlineto{\pgfqpoint{4.293585in}{3.395132in}}%
\pgfpathlineto{\pgfqpoint{4.306903in}{3.384661in}}%
\pgfpathlineto{\pgfqpoint{4.314559in}{3.406482in}}%
\pgfpathlineto{\pgfqpoint{4.322216in}{3.428658in}}%
\pgfpathlineto{\pgfqpoint{4.329873in}{3.451198in}}%
\pgfpathlineto{\pgfqpoint{4.337532in}{3.474109in}}%
\pgfpathlineto{\pgfqpoint{4.324218in}{3.485243in}}%
\pgfpathlineto{\pgfqpoint{4.310908in}{3.496593in}}%
\pgfpathlineto{\pgfqpoint{4.297600in}{3.508160in}}%
\pgfpathlineto{\pgfqpoint{4.284295in}{3.519946in}}%
\pgfpathlineto{\pgfqpoint{4.276633in}{3.496358in}}%
\pgfpathlineto{\pgfqpoint{4.268972in}{3.473150in}}%
\pgfpathlineto{\pgfqpoint{4.261310in}{3.450314in}}%
\pgfpathlineto{\pgfqpoint{4.253649in}{3.427842in}}%
\pgfpathclose%
\pgfusepath{fill}%
\end{pgfscope}%
\begin{pgfscope}%
\pgfpathrectangle{\pgfqpoint{1.150000in}{0.150000in}}{\pgfqpoint{5.700000in}{5.700000in}}%
\pgfusepath{clip}%
\pgfsetbuttcap%
\pgfsetroundjoin%
\definecolor{currentfill}{rgb}{0.183898,0.422383,0.556944}%
\pgfsetfillcolor{currentfill}%
\pgfsetfillopacity{0.800000}%
\pgfsetlinewidth{0.000000pt}%
\definecolor{currentstroke}{rgb}{0.000000,0.000000,0.000000}%
\pgfsetstrokecolor{currentstroke}%
\pgfsetdash{}{0pt}%
\pgfpathmoveto{\pgfqpoint{4.169778in}{3.385594in}}%
\pgfpathlineto{\pgfqpoint{4.183081in}{3.374223in}}%
\pgfpathlineto{\pgfqpoint{4.196385in}{3.363074in}}%
\pgfpathlineto{\pgfqpoint{4.209693in}{3.352147in}}%
\pgfpathlineto{\pgfqpoint{4.223003in}{3.341439in}}%
\pgfpathlineto{\pgfqpoint{4.230665in}{3.362532in}}%
\pgfpathlineto{\pgfqpoint{4.238327in}{3.383958in}}%
\pgfpathlineto{\pgfqpoint{4.245988in}{3.405726in}}%
\pgfpathlineto{\pgfqpoint{4.253649in}{3.427842in}}%
\pgfpathlineto{\pgfqpoint{4.240343in}{3.439181in}}%
\pgfpathlineto{\pgfqpoint{4.227040in}{3.450741in}}%
\pgfpathlineto{\pgfqpoint{4.213739in}{3.462522in}}%
\pgfpathlineto{\pgfqpoint{4.200441in}{3.474526in}}%
\pgfpathlineto{\pgfqpoint{4.192776in}{3.451766in}}%
\pgfpathlineto{\pgfqpoint{4.185110in}{3.429362in}}%
\pgfpathlineto{\pgfqpoint{4.177445in}{3.407307in}}%
\pgfpathlineto{\pgfqpoint{4.169778in}{3.385594in}}%
\pgfpathclose%
\pgfusepath{fill}%
\end{pgfscope}%
\begin{pgfscope}%
\pgfpathrectangle{\pgfqpoint{1.150000in}{0.150000in}}{\pgfqpoint{5.700000in}{5.700000in}}%
\pgfusepath{clip}%
\pgfsetbuttcap%
\pgfsetroundjoin%
\definecolor{currentfill}{rgb}{0.187231,0.414746,0.556547}%
\pgfsetfillcolor{currentfill}%
\pgfsetfillopacity{0.800000}%
\pgfsetlinewidth{0.000000pt}%
\definecolor{currentstroke}{rgb}{0.000000,0.000000,0.000000}%
\pgfsetstrokecolor{currentstroke}%
\pgfsetdash{}{0pt}%
\pgfpathmoveto{\pgfqpoint{3.948825in}{3.363554in}}%
\pgfpathlineto{\pgfqpoint{3.962116in}{3.350585in}}%
\pgfpathlineto{\pgfqpoint{3.975408in}{3.337854in}}%
\pgfpathlineto{\pgfqpoint{3.988700in}{3.325361in}}%
\pgfpathlineto{\pgfqpoint{4.001994in}{3.313102in}}%
\pgfpathlineto{\pgfqpoint{4.009680in}{3.333450in}}%
\pgfpathlineto{\pgfqpoint{4.017364in}{3.354100in}}%
\pgfpathlineto{\pgfqpoint{4.025045in}{3.375058in}}%
\pgfpathlineto{\pgfqpoint{4.032725in}{3.396332in}}%
\pgfpathlineto{\pgfqpoint{4.019435in}{3.409161in}}%
\pgfpathlineto{\pgfqpoint{4.006146in}{3.422225in}}%
\pgfpathlineto{\pgfqpoint{3.992857in}{3.435527in}}%
\pgfpathlineto{\pgfqpoint{3.979569in}{3.449067in}}%
\pgfpathlineto{\pgfqpoint{3.971886in}{3.427210in}}%
\pgfpathlineto{\pgfqpoint{3.964201in}{3.405676in}}%
\pgfpathlineto{\pgfqpoint{3.956514in}{3.384460in}}%
\pgfpathlineto{\pgfqpoint{3.948825in}{3.363554in}}%
\pgfpathclose%
\pgfusepath{fill}%
\end{pgfscope}%
\begin{pgfscope}%
\pgfpathrectangle{\pgfqpoint{1.150000in}{0.150000in}}{\pgfqpoint{5.700000in}{5.700000in}}%
\pgfusepath{clip}%
\pgfsetbuttcap%
\pgfsetroundjoin%
\definecolor{currentfill}{rgb}{0.144759,0.519093,0.556572}%
\pgfsetfillcolor{currentfill}%
\pgfsetfillopacity{0.800000}%
\pgfsetlinewidth{0.000000pt}%
\definecolor{currentstroke}{rgb}{0.000000,0.000000,0.000000}%
\pgfsetstrokecolor{currentstroke}%
\pgfsetdash{}{0pt}%
\pgfpathmoveto{\pgfqpoint{3.682868in}{3.675629in}}%
\pgfpathlineto{\pgfqpoint{3.696184in}{3.657521in}}%
\pgfpathlineto{\pgfqpoint{3.709498in}{3.639689in}}%
\pgfpathlineto{\pgfqpoint{3.722808in}{3.622131in}}%
\pgfpathlineto{\pgfqpoint{3.736116in}{3.604844in}}%
\pgfpathlineto{\pgfqpoint{3.743813in}{3.628007in}}%
\pgfpathlineto{\pgfqpoint{3.751507in}{3.651515in}}%
\pgfpathlineto{\pgfqpoint{3.759198in}{3.675374in}}%
\pgfpathlineto{\pgfqpoint{3.766886in}{3.699592in}}%
\pgfpathlineto{\pgfqpoint{3.753579in}{3.717462in}}%
\pgfpathlineto{\pgfqpoint{3.740269in}{3.735604in}}%
\pgfpathlineto{\pgfqpoint{3.726956in}{3.754021in}}%
\pgfpathlineto{\pgfqpoint{3.713639in}{3.772715in}}%
\pgfpathlineto{\pgfqpoint{3.705951in}{3.747900in}}%
\pgfpathlineto{\pgfqpoint{3.698260in}{3.723451in}}%
\pgfpathlineto{\pgfqpoint{3.690565in}{3.699363in}}%
\pgfpathlineto{\pgfqpoint{3.682868in}{3.675629in}}%
\pgfpathclose%
\pgfusepath{fill}%
\end{pgfscope}%
\begin{pgfscope}%
\pgfpathrectangle{\pgfqpoint{1.150000in}{0.150000in}}{\pgfqpoint{5.700000in}{5.700000in}}%
\pgfusepath{clip}%
\pgfsetbuttcap%
\pgfsetroundjoin%
\definecolor{currentfill}{rgb}{0.169646,0.456262,0.558030}%
\pgfsetfillcolor{currentfill}%
\pgfsetfillopacity{0.800000}%
\pgfsetlinewidth{0.000000pt}%
\definecolor{currentstroke}{rgb}{0.000000,0.000000,0.000000}%
\pgfsetstrokecolor{currentstroke}%
\pgfsetdash{}{0pt}%
\pgfpathmoveto{\pgfqpoint{4.337532in}{3.474109in}}%
\pgfpathlineto{\pgfqpoint{4.350848in}{3.463189in}}%
\pgfpathlineto{\pgfqpoint{4.364168in}{3.452483in}}%
\pgfpathlineto{\pgfqpoint{4.377492in}{3.441990in}}%
\pgfpathlineto{\pgfqpoint{4.390819in}{3.431708in}}%
\pgfpathlineto{\pgfqpoint{4.398473in}{3.454315in}}%
\pgfpathlineto{\pgfqpoint{4.406129in}{3.477301in}}%
\pgfpathlineto{\pgfqpoint{4.413785in}{3.500675in}}%
\pgfpathlineto{\pgfqpoint{4.421444in}{3.524445in}}%
\pgfpathlineto{\pgfqpoint{4.408121in}{3.535422in}}%
\pgfpathlineto{\pgfqpoint{4.394803in}{3.546611in}}%
\pgfpathlineto{\pgfqpoint{4.381487in}{3.558014in}}%
\pgfpathlineto{\pgfqpoint{4.368174in}{3.569630in}}%
\pgfpathlineto{\pgfqpoint{4.360512in}{3.545152in}}%
\pgfpathlineto{\pgfqpoint{4.352850in}{3.521077in}}%
\pgfpathlineto{\pgfqpoint{4.345190in}{3.497399in}}%
\pgfpathlineto{\pgfqpoint{4.337532in}{3.474109in}}%
\pgfpathclose%
\pgfusepath{fill}%
\end{pgfscope}%
\begin{pgfscope}%
\pgfpathrectangle{\pgfqpoint{1.150000in}{0.150000in}}{\pgfqpoint{5.700000in}{5.700000in}}%
\pgfusepath{clip}%
\pgfsetbuttcap%
\pgfsetroundjoin%
\definecolor{currentfill}{rgb}{0.188923,0.410910,0.556326}%
\pgfsetfillcolor{currentfill}%
\pgfsetfillopacity{0.800000}%
\pgfsetlinewidth{0.000000pt}%
\definecolor{currentstroke}{rgb}{0.000000,0.000000,0.000000}%
\pgfsetstrokecolor{currentstroke}%
\pgfsetdash{}{0pt}%
\pgfpathmoveto{\pgfqpoint{4.085899in}{3.347345in}}%
\pgfpathlineto{\pgfqpoint{4.099197in}{3.335673in}}%
\pgfpathlineto{\pgfqpoint{4.112497in}{3.324229in}}%
\pgfpathlineto{\pgfqpoint{4.125798in}{3.313009in}}%
\pgfpathlineto{\pgfqpoint{4.139102in}{3.302015in}}%
\pgfpathlineto{\pgfqpoint{4.146773in}{3.322433in}}%
\pgfpathlineto{\pgfqpoint{4.154443in}{3.343165in}}%
\pgfpathlineto{\pgfqpoint{4.162111in}{3.364216in}}%
\pgfpathlineto{\pgfqpoint{4.169778in}{3.385594in}}%
\pgfpathlineto{\pgfqpoint{4.156478in}{3.397189in}}%
\pgfpathlineto{\pgfqpoint{4.143180in}{3.409008in}}%
\pgfpathlineto{\pgfqpoint{4.129885in}{3.421054in}}%
\pgfpathlineto{\pgfqpoint{4.116590in}{3.433327in}}%
\pgfpathlineto{\pgfqpoint{4.108920in}{3.411336in}}%
\pgfpathlineto{\pgfqpoint{4.101248in}{3.389680in}}%
\pgfpathlineto{\pgfqpoint{4.093574in}{3.368352in}}%
\pgfpathlineto{\pgfqpoint{4.085899in}{3.347345in}}%
\pgfpathclose%
\pgfusepath{fill}%
\end{pgfscope}%
\begin{pgfscope}%
\pgfpathrectangle{\pgfqpoint{1.150000in}{0.150000in}}{\pgfqpoint{5.700000in}{5.700000in}}%
\pgfusepath{clip}%
\pgfsetbuttcap%
\pgfsetroundjoin%
\definecolor{currentfill}{rgb}{0.150476,0.504369,0.557430}%
\pgfsetfillcolor{currentfill}%
\pgfsetfillopacity{0.800000}%
\pgfsetlinewidth{0.000000pt}%
\definecolor{currentstroke}{rgb}{0.000000,0.000000,0.000000}%
\pgfsetstrokecolor{currentstroke}%
\pgfsetdash{}{0pt}%
\pgfpathmoveto{\pgfqpoint{4.452099in}{3.623654in}}%
\pgfpathlineto{\pgfqpoint{4.465420in}{3.612158in}}%
\pgfpathlineto{\pgfqpoint{4.478745in}{3.600872in}}%
\pgfpathlineto{\pgfqpoint{4.492073in}{3.589796in}}%
\pgfpathlineto{\pgfqpoint{4.505406in}{3.578927in}}%
\pgfpathlineto{\pgfqpoint{4.513071in}{3.604056in}}%
\pgfpathlineto{\pgfqpoint{4.520740in}{3.629625in}}%
\pgfpathlineto{\pgfqpoint{4.528413in}{3.655643in}}%
\pgfpathlineto{\pgfqpoint{4.515084in}{3.667080in}}%
\pgfpathlineto{\pgfqpoint{4.501758in}{3.678725in}}%
\pgfpathlineto{\pgfqpoint{4.488437in}{3.690581in}}%
\pgfpathlineto{\pgfqpoint{4.475118in}{3.702646in}}%
\pgfpathlineto{\pgfqpoint{4.467442in}{3.675861in}}%
\pgfpathlineto{\pgfqpoint{4.459769in}{3.649533in}}%
\pgfpathlineto{\pgfqpoint{4.452099in}{3.623654in}}%
\pgfpathclose%
\pgfusepath{fill}%
\end{pgfscope}%
\begin{pgfscope}%
\pgfpathrectangle{\pgfqpoint{1.150000in}{0.150000in}}{\pgfqpoint{5.700000in}{5.700000in}}%
\pgfusepath{clip}%
\pgfsetbuttcap%
\pgfsetroundjoin%
\definecolor{currentfill}{rgb}{0.174274,0.445044,0.557792}%
\pgfsetfillcolor{currentfill}%
\pgfsetfillopacity{0.800000}%
\pgfsetlinewidth{0.000000pt}%
\definecolor{currentstroke}{rgb}{0.000000,0.000000,0.000000}%
\pgfsetstrokecolor{currentstroke}%
\pgfsetdash{}{0pt}%
\pgfpathmoveto{\pgfqpoint{3.758509in}{3.451217in}}%
\pgfpathlineto{\pgfqpoint{3.771808in}{3.435802in}}%
\pgfpathlineto{\pgfqpoint{3.785105in}{3.420645in}}%
\pgfpathlineto{\pgfqpoint{3.798401in}{3.405745in}}%
\pgfpathlineto{\pgfqpoint{3.811696in}{3.391100in}}%
\pgfpathlineto{\pgfqpoint{3.819402in}{3.411887in}}%
\pgfpathlineto{\pgfqpoint{3.827105in}{3.432974in}}%
\pgfpathlineto{\pgfqpoint{3.834805in}{3.454369in}}%
\pgfpathlineto{\pgfqpoint{3.842502in}{3.476078in}}%
\pgfpathlineto{\pgfqpoint{3.829209in}{3.491266in}}%
\pgfpathlineto{\pgfqpoint{3.815915in}{3.506710in}}%
\pgfpathlineto{\pgfqpoint{3.802620in}{3.522411in}}%
\pgfpathlineto{\pgfqpoint{3.789323in}{3.538371in}}%
\pgfpathlineto{\pgfqpoint{3.781624in}{3.516105in}}%
\pgfpathlineto{\pgfqpoint{3.773922in}{3.494162in}}%
\pgfpathlineto{\pgfqpoint{3.766217in}{3.472535in}}%
\pgfpathlineto{\pgfqpoint{3.758509in}{3.451217in}}%
\pgfpathclose%
\pgfusepath{fill}%
\end{pgfscope}%
\begin{pgfscope}%
\pgfpathrectangle{\pgfqpoint{1.150000in}{0.150000in}}{\pgfqpoint{5.700000in}{5.700000in}}%
\pgfusepath{clip}%
\pgfsetbuttcap%
\pgfsetroundjoin%
\definecolor{currentfill}{rgb}{0.165117,0.467423,0.558141}%
\pgfsetfillcolor{currentfill}%
\pgfsetfillopacity{0.800000}%
\pgfsetlinewidth{0.000000pt}%
\definecolor{currentstroke}{rgb}{0.000000,0.000000,0.000000}%
\pgfsetstrokecolor{currentstroke}%
\pgfsetdash{}{0pt}%
\pgfpathmoveto{\pgfqpoint{3.705295in}{3.515503in}}%
\pgfpathlineto{\pgfqpoint{3.718602in}{3.499034in}}%
\pgfpathlineto{\pgfqpoint{3.731906in}{3.482831in}}%
\pgfpathlineto{\pgfqpoint{3.745208in}{3.466893in}}%
\pgfpathlineto{\pgfqpoint{3.758509in}{3.451217in}}%
\pgfpathlineto{\pgfqpoint{3.766217in}{3.472535in}}%
\pgfpathlineto{\pgfqpoint{3.773922in}{3.494162in}}%
\pgfpathlineto{\pgfqpoint{3.781624in}{3.516105in}}%
\pgfpathlineto{\pgfqpoint{3.789323in}{3.538371in}}%
\pgfpathlineto{\pgfqpoint{3.776024in}{3.554592in}}%
\pgfpathlineto{\pgfqpoint{3.762724in}{3.571077in}}%
\pgfpathlineto{\pgfqpoint{3.749421in}{3.587827in}}%
\pgfpathlineto{\pgfqpoint{3.736116in}{3.604844in}}%
\pgfpathlineto{\pgfqpoint{3.728416in}{3.582019in}}%
\pgfpathlineto{\pgfqpoint{3.720712in}{3.559524in}}%
\pgfpathlineto{\pgfqpoint{3.713005in}{3.537355in}}%
\pgfpathlineto{\pgfqpoint{3.705295in}{3.515503in}}%
\pgfpathclose%
\pgfusepath{fill}%
\end{pgfscope}%
\begin{pgfscope}%
\pgfpathrectangle{\pgfqpoint{1.150000in}{0.150000in}}{\pgfqpoint{5.700000in}{5.700000in}}%
\pgfusepath{clip}%
\pgfsetbuttcap%
\pgfsetroundjoin%
\definecolor{currentfill}{rgb}{0.157851,0.683765,0.501686}%
\pgfsetfillcolor{currentfill}%
\pgfsetfillopacity{0.800000}%
\pgfsetlinewidth{0.000000pt}%
\definecolor{currentstroke}{rgb}{0.000000,0.000000,0.000000}%
\pgfsetstrokecolor{currentstroke}%
\pgfsetdash{}{0pt}%
\pgfpathmoveto{\pgfqpoint{3.889615in}{4.141999in}}%
\pgfpathlineto{\pgfqpoint{3.902927in}{4.121706in}}%
\pgfpathlineto{\pgfqpoint{3.916236in}{4.101687in}}%
\pgfpathlineto{\pgfqpoint{3.929541in}{4.081937in}}%
\pgfpathlineto{\pgfqpoint{3.942845in}{4.062456in}}%
\pgfpathlineto{\pgfqpoint{3.950509in}{4.093202in}}%
\pgfpathlineto{\pgfqpoint{3.958173in}{4.124444in}}%
\pgfpathlineto{\pgfqpoint{3.965838in}{4.156189in}}%
\pgfpathlineto{\pgfqpoint{3.973503in}{4.188449in}}%
\pgfpathlineto{\pgfqpoint{3.960196in}{4.208694in}}%
\pgfpathlineto{\pgfqpoint{3.946886in}{4.229209in}}%
\pgfpathlineto{\pgfqpoint{3.933573in}{4.249996in}}%
\pgfpathlineto{\pgfqpoint{3.920256in}{4.271057in}}%
\pgfpathlineto{\pgfqpoint{3.912596in}{4.238016in}}%
\pgfpathlineto{\pgfqpoint{3.904935in}{4.205499in}}%
\pgfpathlineto{\pgfqpoint{3.897275in}{4.173496in}}%
\pgfpathlineto{\pgfqpoint{3.889615in}{4.141999in}}%
\pgfpathclose%
\pgfusepath{fill}%
\end{pgfscope}%
\begin{pgfscope}%
\pgfpathrectangle{\pgfqpoint{1.150000in}{0.150000in}}{\pgfqpoint{5.700000in}{5.700000in}}%
\pgfusepath{clip}%
\pgfsetbuttcap%
\pgfsetroundjoin%
\definecolor{currentfill}{rgb}{0.183898,0.422383,0.556944}%
\pgfsetfillcolor{currentfill}%
\pgfsetfillopacity{0.800000}%
\pgfsetlinewidth{0.000000pt}%
\definecolor{currentstroke}{rgb}{0.000000,0.000000,0.000000}%
\pgfsetstrokecolor{currentstroke}%
\pgfsetdash{}{0pt}%
\pgfpathmoveto{\pgfqpoint{3.811696in}{3.391100in}}%
\pgfpathlineto{\pgfqpoint{3.824990in}{3.376708in}}%
\pgfpathlineto{\pgfqpoint{3.838284in}{3.362568in}}%
\pgfpathlineto{\pgfqpoint{3.851577in}{3.348676in}}%
\pgfpathlineto{\pgfqpoint{3.864870in}{3.335033in}}%
\pgfpathlineto{\pgfqpoint{3.872573in}{3.355290in}}%
\pgfpathlineto{\pgfqpoint{3.880273in}{3.375840in}}%
\pgfpathlineto{\pgfqpoint{3.887970in}{3.396689in}}%
\pgfpathlineto{\pgfqpoint{3.895665in}{3.417844in}}%
\pgfpathlineto{\pgfqpoint{3.882375in}{3.432028in}}%
\pgfpathlineto{\pgfqpoint{3.869085in}{3.446461in}}%
\pgfpathlineto{\pgfqpoint{3.855794in}{3.461144in}}%
\pgfpathlineto{\pgfqpoint{3.842502in}{3.476078in}}%
\pgfpathlineto{\pgfqpoint{3.834805in}{3.454369in}}%
\pgfpathlineto{\pgfqpoint{3.827105in}{3.432974in}}%
\pgfpathlineto{\pgfqpoint{3.819402in}{3.411887in}}%
\pgfpathlineto{\pgfqpoint{3.811696in}{3.391100in}}%
\pgfpathclose%
\pgfusepath{fill}%
\end{pgfscope}%
\begin{pgfscope}%
\pgfpathrectangle{\pgfqpoint{1.150000in}{0.150000in}}{\pgfqpoint{5.700000in}{5.700000in}}%
\pgfusepath{clip}%
\pgfsetbuttcap%
\pgfsetroundjoin%
\definecolor{currentfill}{rgb}{0.123463,0.581687,0.547445}%
\pgfsetfillcolor{currentfill}%
\pgfsetfillopacity{0.800000}%
\pgfsetlinewidth{0.000000pt}%
\definecolor{currentstroke}{rgb}{0.000000,0.000000,0.000000}%
\pgfsetstrokecolor{currentstroke}%
\pgfsetdash{}{0pt}%
\pgfpathmoveto{\pgfqpoint{3.660334in}{3.850307in}}%
\pgfpathlineto{\pgfqpoint{3.673667in}{3.830482in}}%
\pgfpathlineto{\pgfqpoint{3.686995in}{3.810943in}}%
\pgfpathlineto{\pgfqpoint{3.700319in}{3.791689in}}%
\pgfpathlineto{\pgfqpoint{3.713639in}{3.772715in}}%
\pgfpathlineto{\pgfqpoint{3.721324in}{3.797906in}}%
\pgfpathlineto{\pgfqpoint{3.729006in}{3.823478in}}%
\pgfpathlineto{\pgfqpoint{3.736686in}{3.849439in}}%
\pgfpathlineto{\pgfqpoint{3.744363in}{3.875797in}}%
\pgfpathlineto{\pgfqpoint{3.731041in}{3.895393in}}%
\pgfpathlineto{\pgfqpoint{3.717716in}{3.915272in}}%
\pgfpathlineto{\pgfqpoint{3.704387in}{3.935436in}}%
\pgfpathlineto{\pgfqpoint{3.691053in}{3.955887in}}%
\pgfpathlineto{\pgfqpoint{3.683378in}{3.928891in}}%
\pgfpathlineto{\pgfqpoint{3.675700in}{3.902300in}}%
\pgfpathlineto{\pgfqpoint{3.668019in}{3.876108in}}%
\pgfpathlineto{\pgfqpoint{3.660334in}{3.850307in}}%
\pgfpathclose%
\pgfusepath{fill}%
\end{pgfscope}%
\begin{pgfscope}%
\pgfpathrectangle{\pgfqpoint{1.150000in}{0.150000in}}{\pgfqpoint{5.700000in}{5.700000in}}%
\pgfusepath{clip}%
\pgfsetbuttcap%
\pgfsetroundjoin%
\definecolor{currentfill}{rgb}{0.162142,0.474838,0.558140}%
\pgfsetfillcolor{currentfill}%
\pgfsetfillopacity{0.800000}%
\pgfsetlinewidth{0.000000pt}%
\definecolor{currentstroke}{rgb}{0.000000,0.000000,0.000000}%
\pgfsetstrokecolor{currentstroke}%
\pgfsetdash{}{0pt}%
\pgfpathmoveto{\pgfqpoint{4.421444in}{3.524445in}}%
\pgfpathlineto{\pgfqpoint{4.434770in}{3.513678in}}%
\pgfpathlineto{\pgfqpoint{4.448099in}{3.503121in}}%
\pgfpathlineto{\pgfqpoint{4.461433in}{3.492772in}}%
\pgfpathlineto{\pgfqpoint{4.474770in}{3.482631in}}%
\pgfpathlineto{\pgfqpoint{4.482426in}{3.506090in}}%
\pgfpathlineto{\pgfqpoint{4.490083in}{3.529953in}}%
\pgfpathlineto{\pgfqpoint{4.497743in}{3.554229in}}%
\pgfpathlineto{\pgfqpoint{4.505406in}{3.578927in}}%
\pgfpathlineto{\pgfqpoint{4.492073in}{3.589796in}}%
\pgfpathlineto{\pgfqpoint{4.478745in}{3.600872in}}%
\pgfpathlineto{\pgfqpoint{4.465420in}{3.612158in}}%
\pgfpathlineto{\pgfqpoint{4.452099in}{3.623654in}}%
\pgfpathlineto{\pgfqpoint{4.444431in}{3.598214in}}%
\pgfpathlineto{\pgfqpoint{4.436767in}{3.573206in}}%
\pgfpathlineto{\pgfqpoint{4.429104in}{3.548619in}}%
\pgfpathlineto{\pgfqpoint{4.421444in}{3.524445in}}%
\pgfpathclose%
\pgfusepath{fill}%
\end{pgfscope}%
\begin{pgfscope}%
\pgfpathrectangle{\pgfqpoint{1.150000in}{0.150000in}}{\pgfqpoint{5.700000in}{5.700000in}}%
\pgfusepath{clip}%
\pgfsetbuttcap%
\pgfsetroundjoin%
\definecolor{currentfill}{rgb}{0.170948,0.694384,0.493803}%
\pgfsetfillcolor{currentfill}%
\pgfsetfillopacity{0.800000}%
\pgfsetlinewidth{0.000000pt}%
\definecolor{currentstroke}{rgb}{0.000000,0.000000,0.000000}%
\pgfsetstrokecolor{currentstroke}%
\pgfsetdash{}{0pt}%
\pgfpathmoveto{\pgfqpoint{3.973503in}{4.188449in}}%
\pgfpathlineto{\pgfqpoint{3.986808in}{4.168471in}}%
\pgfpathlineto{\pgfqpoint{4.000109in}{4.148759in}}%
\pgfpathlineto{\pgfqpoint{4.013409in}{4.129310in}}%
\pgfpathlineto{\pgfqpoint{4.026706in}{4.110122in}}%
\pgfpathlineto{\pgfqpoint{4.034376in}{4.142120in}}%
\pgfpathlineto{\pgfqpoint{4.042047in}{4.174642in}}%
\pgfpathlineto{\pgfqpoint{4.049720in}{4.207696in}}%
\pgfpathlineto{\pgfqpoint{4.036420in}{4.227478in}}%
\pgfpathlineto{\pgfqpoint{4.023117in}{4.247523in}}%
\pgfpathlineto{\pgfqpoint{4.009812in}{4.267832in}}%
\pgfpathlineto{\pgfqpoint{3.996504in}{4.288408in}}%
\pgfpathlineto{\pgfqpoint{3.988836in}{4.254548in}}%
\pgfpathlineto{\pgfqpoint{3.981169in}{4.221232in}}%
\pgfpathlineto{\pgfqpoint{3.973503in}{4.188449in}}%
\pgfpathclose%
\pgfusepath{fill}%
\end{pgfscope}%
\begin{pgfscope}%
\pgfpathrectangle{\pgfqpoint{1.150000in}{0.150000in}}{\pgfqpoint{5.700000in}{5.700000in}}%
\pgfusepath{clip}%
\pgfsetbuttcap%
\pgfsetroundjoin%
\definecolor{currentfill}{rgb}{0.119699,0.618490,0.536347}%
\pgfsetfillcolor{currentfill}%
\pgfsetfillopacity{0.800000}%
\pgfsetlinewidth{0.000000pt}%
\definecolor{currentstroke}{rgb}{0.000000,0.000000,0.000000}%
\pgfsetstrokecolor{currentstroke}%
\pgfsetdash{}{0pt}%
\pgfpathmoveto{\pgfqpoint{3.691053in}{3.955887in}}%
\pgfpathlineto{\pgfqpoint{3.704387in}{3.935436in}}%
\pgfpathlineto{\pgfqpoint{3.717716in}{3.915272in}}%
\pgfpathlineto{\pgfqpoint{3.731041in}{3.895393in}}%
\pgfpathlineto{\pgfqpoint{3.744363in}{3.875797in}}%
\pgfpathlineto{\pgfqpoint{3.752037in}{3.902560in}}%
\pgfpathlineto{\pgfqpoint{3.759709in}{3.929734in}}%
\pgfpathlineto{\pgfqpoint{3.767379in}{3.957329in}}%
\pgfpathlineto{\pgfqpoint{3.775046in}{3.985351in}}%
\pgfpathlineto{\pgfqpoint{3.761723in}{4.005607in}}%
\pgfpathlineto{\pgfqpoint{3.748395in}{4.026147in}}%
\pgfpathlineto{\pgfqpoint{3.735063in}{4.046973in}}%
\pgfpathlineto{\pgfqpoint{3.721727in}{4.068087in}}%
\pgfpathlineto{\pgfqpoint{3.714062in}{4.039389in}}%
\pgfpathlineto{\pgfqpoint{3.706395in}{4.011128in}}%
\pgfpathlineto{\pgfqpoint{3.698725in}{3.983297in}}%
\pgfpathlineto{\pgfqpoint{3.691053in}{3.955887in}}%
\pgfpathclose%
\pgfusepath{fill}%
\end{pgfscope}%
\begin{pgfscope}%
\pgfpathrectangle{\pgfqpoint{1.150000in}{0.150000in}}{\pgfqpoint{5.700000in}{5.700000in}}%
\pgfusepath{clip}%
\pgfsetbuttcap%
\pgfsetroundjoin%
\definecolor{currentfill}{rgb}{0.143303,0.669459,0.511215}%
\pgfsetfillcolor{currentfill}%
\pgfsetfillopacity{0.800000}%
\pgfsetlinewidth{0.000000pt}%
\definecolor{currentstroke}{rgb}{0.000000,0.000000,0.000000}%
\pgfsetstrokecolor{currentstroke}%
\pgfsetdash{}{0pt}%
\pgfpathmoveto{\pgfqpoint{3.805699in}{4.101886in}}%
\pgfpathlineto{\pgfqpoint{3.819022in}{4.081215in}}%
\pgfpathlineto{\pgfqpoint{3.832341in}{4.060825in}}%
\pgfpathlineto{\pgfqpoint{3.845656in}{4.040712in}}%
\pgfpathlineto{\pgfqpoint{3.858967in}{4.020876in}}%
\pgfpathlineto{\pgfqpoint{3.866630in}{4.050444in}}%
\pgfpathlineto{\pgfqpoint{3.874293in}{4.080482in}}%
\pgfpathlineto{\pgfqpoint{3.881954in}{4.110997in}}%
\pgfpathlineto{\pgfqpoint{3.889615in}{4.141999in}}%
\pgfpathlineto{\pgfqpoint{3.876299in}{4.162566in}}%
\pgfpathlineto{\pgfqpoint{3.862980in}{4.183410in}}%
\pgfpathlineto{\pgfqpoint{3.849657in}{4.204534in}}%
\pgfpathlineto{\pgfqpoint{3.836330in}{4.225939in}}%
\pgfpathlineto{\pgfqpoint{3.828674in}{4.194190in}}%
\pgfpathlineto{\pgfqpoint{3.821017in}{4.162938in}}%
\pgfpathlineto{\pgfqpoint{3.813358in}{4.132172in}}%
\pgfpathlineto{\pgfqpoint{3.805699in}{4.101886in}}%
\pgfpathclose%
\pgfusepath{fill}%
\end{pgfscope}%
\begin{pgfscope}%
\pgfpathrectangle{\pgfqpoint{1.150000in}{0.150000in}}{\pgfqpoint{5.700000in}{5.700000in}}%
\pgfusepath{clip}%
\pgfsetbuttcap%
\pgfsetroundjoin%
\definecolor{currentfill}{rgb}{0.194100,0.399323,0.555565}%
\pgfsetfillcolor{currentfill}%
\pgfsetfillopacity{0.800000}%
\pgfsetlinewidth{0.000000pt}%
\definecolor{currentstroke}{rgb}{0.000000,0.000000,0.000000}%
\pgfsetstrokecolor{currentstroke}%
\pgfsetdash{}{0pt}%
\pgfpathmoveto{\pgfqpoint{4.001994in}{3.313102in}}%
\pgfpathlineto{\pgfqpoint{4.015289in}{3.301078in}}%
\pgfpathlineto{\pgfqpoint{4.028585in}{3.289285in}}%
\pgfpathlineto{\pgfqpoint{4.041883in}{3.277724in}}%
\pgfpathlineto{\pgfqpoint{4.055183in}{3.266392in}}%
\pgfpathlineto{\pgfqpoint{4.062865in}{3.286183in}}%
\pgfpathlineto{\pgfqpoint{4.070545in}{3.306268in}}%
\pgfpathlineto{\pgfqpoint{4.078223in}{3.326653in}}%
\pgfpathlineto{\pgfqpoint{4.085899in}{3.347345in}}%
\pgfpathlineto{\pgfqpoint{4.072604in}{3.359246in}}%
\pgfpathlineto{\pgfqpoint{4.059309in}{3.371376in}}%
\pgfpathlineto{\pgfqpoint{4.046017in}{3.383738in}}%
\pgfpathlineto{\pgfqpoint{4.032725in}{3.396332in}}%
\pgfpathlineto{\pgfqpoint{4.025045in}{3.375058in}}%
\pgfpathlineto{\pgfqpoint{4.017364in}{3.354100in}}%
\pgfpathlineto{\pgfqpoint{4.009680in}{3.333450in}}%
\pgfpathlineto{\pgfqpoint{4.001994in}{3.313102in}}%
\pgfpathclose%
\pgfusepath{fill}%
\end{pgfscope}%
\begin{pgfscope}%
\pgfpathrectangle{\pgfqpoint{1.150000in}{0.150000in}}{\pgfqpoint{5.700000in}{5.700000in}}%
\pgfusepath{clip}%
\pgfsetbuttcap%
\pgfsetroundjoin%
\definecolor{currentfill}{rgb}{0.135066,0.544853,0.554029}%
\pgfsetfillcolor{currentfill}%
\pgfsetfillopacity{0.800000}%
\pgfsetlinewidth{0.000000pt}%
\definecolor{currentstroke}{rgb}{0.000000,0.000000,0.000000}%
\pgfsetstrokecolor{currentstroke}%
\pgfsetdash{}{0pt}%
\pgfpathmoveto{\pgfqpoint{3.629564in}{3.750867in}}%
\pgfpathlineto{\pgfqpoint{3.642896in}{3.731632in}}%
\pgfpathlineto{\pgfqpoint{3.656224in}{3.712682in}}%
\pgfpathlineto{\pgfqpoint{3.669548in}{3.694015in}}%
\pgfpathlineto{\pgfqpoint{3.682868in}{3.675629in}}%
\pgfpathlineto{\pgfqpoint{3.690565in}{3.699363in}}%
\pgfpathlineto{\pgfqpoint{3.698260in}{3.723451in}}%
\pgfpathlineto{\pgfqpoint{3.705951in}{3.747900in}}%
\pgfpathlineto{\pgfqpoint{3.713639in}{3.772715in}}%
\pgfpathlineto{\pgfqpoint{3.700319in}{3.791689in}}%
\pgfpathlineto{\pgfqpoint{3.686995in}{3.810943in}}%
\pgfpathlineto{\pgfqpoint{3.673667in}{3.830482in}}%
\pgfpathlineto{\pgfqpoint{3.660334in}{3.850307in}}%
\pgfpathlineto{\pgfqpoint{3.652647in}{3.824890in}}%
\pgfpathlineto{\pgfqpoint{3.644956in}{3.799848in}}%
\pgfpathlineto{\pgfqpoint{3.637262in}{3.775177in}}%
\pgfpathlineto{\pgfqpoint{3.629564in}{3.750867in}}%
\pgfpathclose%
\pgfusepath{fill}%
\end{pgfscope}%
\begin{pgfscope}%
\pgfpathrectangle{\pgfqpoint{1.150000in}{0.150000in}}{\pgfqpoint{5.700000in}{5.700000in}}%
\pgfusepath{clip}%
\pgfsetbuttcap%
\pgfsetroundjoin%
\definecolor{currentfill}{rgb}{0.156270,0.489624,0.557936}%
\pgfsetfillcolor{currentfill}%
\pgfsetfillopacity{0.800000}%
\pgfsetlinewidth{0.000000pt}%
\definecolor{currentstroke}{rgb}{0.000000,0.000000,0.000000}%
\pgfsetstrokecolor{currentstroke}%
\pgfsetdash{}{0pt}%
\pgfpathmoveto{\pgfqpoint{3.652042in}{3.584090in}}%
\pgfpathlineto{\pgfqpoint{3.665360in}{3.566532in}}%
\pgfpathlineto{\pgfqpoint{3.678674in}{3.549250in}}%
\pgfpathlineto{\pgfqpoint{3.691986in}{3.532241in}}%
\pgfpathlineto{\pgfqpoint{3.705295in}{3.515503in}}%
\pgfpathlineto{\pgfqpoint{3.713005in}{3.537355in}}%
\pgfpathlineto{\pgfqpoint{3.720712in}{3.559524in}}%
\pgfpathlineto{\pgfqpoint{3.728416in}{3.582019in}}%
\pgfpathlineto{\pgfqpoint{3.736116in}{3.604844in}}%
\pgfpathlineto{\pgfqpoint{3.722808in}{3.622131in}}%
\pgfpathlineto{\pgfqpoint{3.709498in}{3.639689in}}%
\pgfpathlineto{\pgfqpoint{3.696184in}{3.657521in}}%
\pgfpathlineto{\pgfqpoint{3.682868in}{3.675629in}}%
\pgfpathlineto{\pgfqpoint{3.675167in}{3.652241in}}%
\pgfpathlineto{\pgfqpoint{3.667462in}{3.629193in}}%
\pgfpathlineto{\pgfqpoint{3.659754in}{3.606478in}}%
\pgfpathlineto{\pgfqpoint{3.652042in}{3.584090in}}%
\pgfpathclose%
\pgfusepath{fill}%
\end{pgfscope}%
\begin{pgfscope}%
\pgfpathrectangle{\pgfqpoint{1.150000in}{0.150000in}}{\pgfqpoint{5.700000in}{5.700000in}}%
\pgfusepath{clip}%
\pgfsetbuttcap%
\pgfsetroundjoin%
\definecolor{currentfill}{rgb}{0.192357,0.403199,0.555836}%
\pgfsetfillcolor{currentfill}%
\pgfsetfillopacity{0.800000}%
\pgfsetlinewidth{0.000000pt}%
\definecolor{currentstroke}{rgb}{0.000000,0.000000,0.000000}%
\pgfsetstrokecolor{currentstroke}%
\pgfsetdash{}{0pt}%
\pgfpathmoveto{\pgfqpoint{3.864870in}{3.335033in}}%
\pgfpathlineto{\pgfqpoint{3.878163in}{3.321635in}}%
\pgfpathlineto{\pgfqpoint{3.891456in}{3.308481in}}%
\pgfpathlineto{\pgfqpoint{3.904749in}{3.295571in}}%
\pgfpathlineto{\pgfqpoint{3.918043in}{3.282901in}}%
\pgfpathlineto{\pgfqpoint{3.925743in}{3.302631in}}%
\pgfpathlineto{\pgfqpoint{3.933439in}{3.322646in}}%
\pgfpathlineto{\pgfqpoint{3.941134in}{3.342951in}}%
\pgfpathlineto{\pgfqpoint{3.948825in}{3.363554in}}%
\pgfpathlineto{\pgfqpoint{3.935535in}{3.376762in}}%
\pgfpathlineto{\pgfqpoint{3.922245in}{3.390212in}}%
\pgfpathlineto{\pgfqpoint{3.908955in}{3.403906in}}%
\pgfpathlineto{\pgfqpoint{3.895665in}{3.417844in}}%
\pgfpathlineto{\pgfqpoint{3.887970in}{3.396689in}}%
\pgfpathlineto{\pgfqpoint{3.880273in}{3.375840in}}%
\pgfpathlineto{\pgfqpoint{3.872573in}{3.355290in}}%
\pgfpathlineto{\pgfqpoint{3.864870in}{3.335033in}}%
\pgfpathclose%
\pgfusepath{fill}%
\end{pgfscope}%
\begin{pgfscope}%
\pgfpathrectangle{\pgfqpoint{1.150000in}{0.150000in}}{\pgfqpoint{5.700000in}{5.700000in}}%
\pgfusepath{clip}%
\pgfsetbuttcap%
\pgfsetroundjoin%
\definecolor{currentfill}{rgb}{0.188923,0.410910,0.556326}%
\pgfsetfillcolor{currentfill}%
\pgfsetfillopacity{0.800000}%
\pgfsetlinewidth{0.000000pt}%
\definecolor{currentstroke}{rgb}{0.000000,0.000000,0.000000}%
\pgfsetstrokecolor{currentstroke}%
\pgfsetdash{}{0pt}%
\pgfpathmoveto{\pgfqpoint{4.223003in}{3.341439in}}%
\pgfpathlineto{\pgfqpoint{4.236316in}{3.330949in}}%
\pgfpathlineto{\pgfqpoint{4.249633in}{3.320677in}}%
\pgfpathlineto{\pgfqpoint{4.262952in}{3.310621in}}%
\pgfpathlineto{\pgfqpoint{4.276275in}{3.300780in}}%
\pgfpathlineto{\pgfqpoint{4.283932in}{3.321255in}}%
\pgfpathlineto{\pgfqpoint{4.291589in}{3.342056in}}%
\pgfpathlineto{\pgfqpoint{4.299246in}{3.363188in}}%
\pgfpathlineto{\pgfqpoint{4.306903in}{3.384661in}}%
\pgfpathlineto{\pgfqpoint{4.293585in}{3.395132in}}%
\pgfpathlineto{\pgfqpoint{4.280270in}{3.405818in}}%
\pgfpathlineto{\pgfqpoint{4.266958in}{3.416721in}}%
\pgfpathlineto{\pgfqpoint{4.253649in}{3.427842in}}%
\pgfpathlineto{\pgfqpoint{4.245988in}{3.405726in}}%
\pgfpathlineto{\pgfqpoint{4.238327in}{3.383958in}}%
\pgfpathlineto{\pgfqpoint{4.230665in}{3.362532in}}%
\pgfpathlineto{\pgfqpoint{4.223003in}{3.341439in}}%
\pgfpathclose%
\pgfusepath{fill}%
\end{pgfscope}%
\begin{pgfscope}%
\pgfpathrectangle{\pgfqpoint{1.150000in}{0.150000in}}{\pgfqpoint{5.700000in}{5.700000in}}%
\pgfusepath{clip}%
\pgfsetbuttcap%
\pgfsetroundjoin%
\definecolor{currentfill}{rgb}{0.182256,0.426184,0.557120}%
\pgfsetfillcolor{currentfill}%
\pgfsetfillopacity{0.800000}%
\pgfsetlinewidth{0.000000pt}%
\definecolor{currentstroke}{rgb}{0.000000,0.000000,0.000000}%
\pgfsetstrokecolor{currentstroke}%
\pgfsetdash{}{0pt}%
\pgfpathmoveto{\pgfqpoint{4.306903in}{3.384661in}}%
\pgfpathlineto{\pgfqpoint{4.320224in}{3.374405in}}%
\pgfpathlineto{\pgfqpoint{4.333549in}{3.364361in}}%
\pgfpathlineto{\pgfqpoint{4.346878in}{3.354530in}}%
\pgfpathlineto{\pgfqpoint{4.360210in}{3.344909in}}%
\pgfpathlineto{\pgfqpoint{4.367861in}{3.366080in}}%
\pgfpathlineto{\pgfqpoint{4.375513in}{3.387598in}}%
\pgfpathlineto{\pgfqpoint{4.383166in}{3.409472in}}%
\pgfpathlineto{\pgfqpoint{4.390819in}{3.431708in}}%
\pgfpathlineto{\pgfqpoint{4.377492in}{3.441990in}}%
\pgfpathlineto{\pgfqpoint{4.364168in}{3.452483in}}%
\pgfpathlineto{\pgfqpoint{4.350848in}{3.463189in}}%
\pgfpathlineto{\pgfqpoint{4.337532in}{3.474109in}}%
\pgfpathlineto{\pgfqpoint{4.329873in}{3.451198in}}%
\pgfpathlineto{\pgfqpoint{4.322216in}{3.428658in}}%
\pgfpathlineto{\pgfqpoint{4.314559in}{3.406482in}}%
\pgfpathlineto{\pgfqpoint{4.306903in}{3.384661in}}%
\pgfpathclose%
\pgfusepath{fill}%
\end{pgfscope}%
\begin{pgfscope}%
\pgfpathrectangle{\pgfqpoint{1.150000in}{0.150000in}}{\pgfqpoint{5.700000in}{5.700000in}}%
\pgfusepath{clip}%
\pgfsetbuttcap%
\pgfsetroundjoin%
\definecolor{currentfill}{rgb}{0.132268,0.655014,0.519661}%
\pgfsetfillcolor{currentfill}%
\pgfsetfillopacity{0.800000}%
\pgfsetlinewidth{0.000000pt}%
\definecolor{currentstroke}{rgb}{0.000000,0.000000,0.000000}%
\pgfsetstrokecolor{currentstroke}%
\pgfsetdash{}{0pt}%
\pgfpathmoveto{\pgfqpoint{3.721727in}{4.068087in}}%
\pgfpathlineto{\pgfqpoint{3.735063in}{4.046973in}}%
\pgfpathlineto{\pgfqpoint{3.748395in}{4.026147in}}%
\pgfpathlineto{\pgfqpoint{3.761723in}{4.005607in}}%
\pgfpathlineto{\pgfqpoint{3.775046in}{3.985351in}}%
\pgfpathlineto{\pgfqpoint{3.782712in}{4.013810in}}%
\pgfpathlineto{\pgfqpoint{3.790376in}{4.042713in}}%
\pgfpathlineto{\pgfqpoint{3.798038in}{4.072069in}}%
\pgfpathlineto{\pgfqpoint{3.805699in}{4.101886in}}%
\pgfpathlineto{\pgfqpoint{3.792372in}{4.122839in}}%
\pgfpathlineto{\pgfqpoint{3.779041in}{4.144078in}}%
\pgfpathlineto{\pgfqpoint{3.765706in}{4.165603in}}%
\pgfpathlineto{\pgfqpoint{3.752365in}{4.187419in}}%
\pgfpathlineto{\pgfqpoint{3.744709in}{4.156889in}}%
\pgfpathlineto{\pgfqpoint{3.737050in}{4.126829in}}%
\pgfpathlineto{\pgfqpoint{3.729390in}{4.097231in}}%
\pgfpathlineto{\pgfqpoint{3.721727in}{4.068087in}}%
\pgfpathclose%
\pgfusepath{fill}%
\end{pgfscope}%
\begin{pgfscope}%
\pgfpathrectangle{\pgfqpoint{1.150000in}{0.150000in}}{\pgfqpoint{5.700000in}{5.700000in}}%
\pgfusepath{clip}%
\pgfsetbuttcap%
\pgfsetroundjoin%
\definecolor{currentfill}{rgb}{0.195860,0.395433,0.555276}%
\pgfsetfillcolor{currentfill}%
\pgfsetfillopacity{0.800000}%
\pgfsetlinewidth{0.000000pt}%
\definecolor{currentstroke}{rgb}{0.000000,0.000000,0.000000}%
\pgfsetstrokecolor{currentstroke}%
\pgfsetdash{}{0pt}%
\pgfpathmoveto{\pgfqpoint{4.139102in}{3.302015in}}%
\pgfpathlineto{\pgfqpoint{4.152409in}{3.291243in}}%
\pgfpathlineto{\pgfqpoint{4.165718in}{3.280694in}}%
\pgfpathlineto{\pgfqpoint{4.179030in}{3.270365in}}%
\pgfpathlineto{\pgfqpoint{4.192345in}{3.260255in}}%
\pgfpathlineto{\pgfqpoint{4.200011in}{3.280087in}}%
\pgfpathlineto{\pgfqpoint{4.207676in}{3.300224in}}%
\pgfpathlineto{\pgfqpoint{4.215340in}{3.320672in}}%
\pgfpathlineto{\pgfqpoint{4.223003in}{3.341439in}}%
\pgfpathlineto{\pgfqpoint{4.209693in}{3.352147in}}%
\pgfpathlineto{\pgfqpoint{4.196385in}{3.363074in}}%
\pgfpathlineto{\pgfqpoint{4.183081in}{3.374223in}}%
\pgfpathlineto{\pgfqpoint{4.169778in}{3.385594in}}%
\pgfpathlineto{\pgfqpoint{4.162111in}{3.364216in}}%
\pgfpathlineto{\pgfqpoint{4.154443in}{3.343165in}}%
\pgfpathlineto{\pgfqpoint{4.146773in}{3.322433in}}%
\pgfpathlineto{\pgfqpoint{4.139102in}{3.302015in}}%
\pgfpathclose%
\pgfusepath{fill}%
\end{pgfscope}%
\begin{pgfscope}%
\pgfpathrectangle{\pgfqpoint{1.150000in}{0.150000in}}{\pgfqpoint{5.700000in}{5.700000in}}%
\pgfusepath{clip}%
\pgfsetbuttcap%
\pgfsetroundjoin%
\definecolor{currentfill}{rgb}{0.174274,0.445044,0.557792}%
\pgfsetfillcolor{currentfill}%
\pgfsetfillopacity{0.800000}%
\pgfsetlinewidth{0.000000pt}%
\definecolor{currentstroke}{rgb}{0.000000,0.000000,0.000000}%
\pgfsetstrokecolor{currentstroke}%
\pgfsetdash{}{0pt}%
\pgfpathmoveto{\pgfqpoint{4.390819in}{3.431708in}}%
\pgfpathlineto{\pgfqpoint{4.404150in}{3.421636in}}%
\pgfpathlineto{\pgfqpoint{4.417485in}{3.411773in}}%
\pgfpathlineto{\pgfqpoint{4.430824in}{3.402119in}}%
\pgfpathlineto{\pgfqpoint{4.444167in}{3.392671in}}%
\pgfpathlineto{\pgfqpoint{4.451816in}{3.414597in}}%
\pgfpathlineto{\pgfqpoint{4.459466in}{3.436893in}}%
\pgfpathlineto{\pgfqpoint{4.467117in}{3.459569in}}%
\pgfpathlineto{\pgfqpoint{4.474770in}{3.482631in}}%
\pgfpathlineto{\pgfqpoint{4.461433in}{3.492772in}}%
\pgfpathlineto{\pgfqpoint{4.448099in}{3.503121in}}%
\pgfpathlineto{\pgfqpoint{4.434770in}{3.513678in}}%
\pgfpathlineto{\pgfqpoint{4.421444in}{3.524445in}}%
\pgfpathlineto{\pgfqpoint{4.413785in}{3.500675in}}%
\pgfpathlineto{\pgfqpoint{4.406129in}{3.477301in}}%
\pgfpathlineto{\pgfqpoint{4.398473in}{3.454315in}}%
\pgfpathlineto{\pgfqpoint{4.390819in}{3.431708in}}%
\pgfpathclose%
\pgfusepath{fill}%
\end{pgfscope}%
\begin{pgfscope}%
\pgfpathrectangle{\pgfqpoint{1.150000in}{0.150000in}}{\pgfqpoint{5.700000in}{5.700000in}}%
\pgfusepath{clip}%
\pgfsetbuttcap%
\pgfsetroundjoin%
\definecolor{currentfill}{rgb}{0.156270,0.489624,0.557936}%
\pgfsetfillcolor{currentfill}%
\pgfsetfillopacity{0.800000}%
\pgfsetlinewidth{0.000000pt}%
\definecolor{currentstroke}{rgb}{0.000000,0.000000,0.000000}%
\pgfsetstrokecolor{currentstroke}%
\pgfsetdash{}{0pt}%
\pgfpathmoveto{\pgfqpoint{4.505406in}{3.578927in}}%
\pgfpathlineto{\pgfqpoint{4.518742in}{3.568266in}}%
\pgfpathlineto{\pgfqpoint{4.532082in}{3.557810in}}%
\pgfpathlineto{\pgfqpoint{4.545427in}{3.547560in}}%
\pgfpathlineto{\pgfqpoint{4.558776in}{3.537513in}}%
\pgfpathlineto{\pgfqpoint{4.566436in}{3.561895in}}%
\pgfpathlineto{\pgfqpoint{4.574100in}{3.586707in}}%
\pgfpathlineto{\pgfqpoint{4.581767in}{3.611960in}}%
\pgfpathlineto{\pgfqpoint{4.568422in}{3.622573in}}%
\pgfpathlineto{\pgfqpoint{4.555082in}{3.633390in}}%
\pgfpathlineto{\pgfqpoint{4.541745in}{3.644413in}}%
\pgfpathlineto{\pgfqpoint{4.528413in}{3.655643in}}%
\pgfpathlineto{\pgfqpoint{4.520740in}{3.629625in}}%
\pgfpathlineto{\pgfqpoint{4.513071in}{3.604056in}}%
\pgfpathlineto{\pgfqpoint{4.505406in}{3.578927in}}%
\pgfpathclose%
\pgfusepath{fill}%
\end{pgfscope}%
\begin{pgfscope}%
\pgfpathrectangle{\pgfqpoint{1.150000in}{0.150000in}}{\pgfqpoint{5.700000in}{5.700000in}}%
\pgfusepath{clip}%
\pgfsetbuttcap%
\pgfsetroundjoin%
\definecolor{currentfill}{rgb}{0.146180,0.515413,0.556823}%
\pgfsetfillcolor{currentfill}%
\pgfsetfillopacity{0.800000}%
\pgfsetlinewidth{0.000000pt}%
\definecolor{currentstroke}{rgb}{0.000000,0.000000,0.000000}%
\pgfsetstrokecolor{currentstroke}%
\pgfsetdash{}{0pt}%
\pgfpathmoveto{\pgfqpoint{3.598735in}{3.657116in}}%
\pgfpathlineto{\pgfqpoint{3.612068in}{3.638435in}}%
\pgfpathlineto{\pgfqpoint{3.625396in}{3.620039in}}%
\pgfpathlineto{\pgfqpoint{3.638721in}{3.601924in}}%
\pgfpathlineto{\pgfqpoint{3.652042in}{3.584090in}}%
\pgfpathlineto{\pgfqpoint{3.659754in}{3.606478in}}%
\pgfpathlineto{\pgfqpoint{3.667462in}{3.629193in}}%
\pgfpathlineto{\pgfqpoint{3.675167in}{3.652241in}}%
\pgfpathlineto{\pgfqpoint{3.682868in}{3.675629in}}%
\pgfpathlineto{\pgfqpoint{3.669548in}{3.694015in}}%
\pgfpathlineto{\pgfqpoint{3.656224in}{3.712682in}}%
\pgfpathlineto{\pgfqpoint{3.642896in}{3.731632in}}%
\pgfpathlineto{\pgfqpoint{3.629564in}{3.750867in}}%
\pgfpathlineto{\pgfqpoint{3.621863in}{3.726913in}}%
\pgfpathlineto{\pgfqpoint{3.614157in}{3.703308in}}%
\pgfpathlineto{\pgfqpoint{3.606448in}{3.680044in}}%
\pgfpathlineto{\pgfqpoint{3.598735in}{3.657116in}}%
\pgfpathclose%
\pgfusepath{fill}%
\end{pgfscope}%
\begin{pgfscope}%
\pgfpathrectangle{\pgfqpoint{1.150000in}{0.150000in}}{\pgfqpoint{5.700000in}{5.700000in}}%
\pgfusepath{clip}%
\pgfsetbuttcap%
\pgfsetroundjoin%
\definecolor{currentfill}{rgb}{0.199430,0.387607,0.554642}%
\pgfsetfillcolor{currentfill}%
\pgfsetfillopacity{0.800000}%
\pgfsetlinewidth{0.000000pt}%
\definecolor{currentstroke}{rgb}{0.000000,0.000000,0.000000}%
\pgfsetstrokecolor{currentstroke}%
\pgfsetdash{}{0pt}%
\pgfpathmoveto{\pgfqpoint{3.918043in}{3.282901in}}%
\pgfpathlineto{\pgfqpoint{3.931337in}{3.270471in}}%
\pgfpathlineto{\pgfqpoint{3.944633in}{3.258278in}}%
\pgfpathlineto{\pgfqpoint{3.957929in}{3.246322in}}%
\pgfpathlineto{\pgfqpoint{3.971227in}{3.234601in}}%
\pgfpathlineto{\pgfqpoint{3.978922in}{3.253806in}}%
\pgfpathlineto{\pgfqpoint{3.986615in}{3.273287in}}%
\pgfpathlineto{\pgfqpoint{3.994306in}{3.293050in}}%
\pgfpathlineto{\pgfqpoint{4.001994in}{3.313102in}}%
\pgfpathlineto{\pgfqpoint{3.988700in}{3.325361in}}%
\pgfpathlineto{\pgfqpoint{3.975408in}{3.337854in}}%
\pgfpathlineto{\pgfqpoint{3.962116in}{3.350585in}}%
\pgfpathlineto{\pgfqpoint{3.948825in}{3.363554in}}%
\pgfpathlineto{\pgfqpoint{3.941134in}{3.342951in}}%
\pgfpathlineto{\pgfqpoint{3.933439in}{3.322646in}}%
\pgfpathlineto{\pgfqpoint{3.925743in}{3.302631in}}%
\pgfpathlineto{\pgfqpoint{3.918043in}{3.282901in}}%
\pgfpathclose%
\pgfusepath{fill}%
\end{pgfscope}%
\begin{pgfscope}%
\pgfpathrectangle{\pgfqpoint{1.150000in}{0.150000in}}{\pgfqpoint{5.700000in}{5.700000in}}%
\pgfusepath{clip}%
\pgfsetbuttcap%
\pgfsetroundjoin%
\definecolor{currentfill}{rgb}{0.201239,0.383670,0.554294}%
\pgfsetfillcolor{currentfill}%
\pgfsetfillopacity{0.800000}%
\pgfsetlinewidth{0.000000pt}%
\definecolor{currentstroke}{rgb}{0.000000,0.000000,0.000000}%
\pgfsetstrokecolor{currentstroke}%
\pgfsetdash{}{0pt}%
\pgfpathmoveto{\pgfqpoint{4.055183in}{3.266392in}}%
\pgfpathlineto{\pgfqpoint{4.068485in}{3.255288in}}%
\pgfpathlineto{\pgfqpoint{4.081789in}{3.244411in}}%
\pgfpathlineto{\pgfqpoint{4.095095in}{3.233759in}}%
\pgfpathlineto{\pgfqpoint{4.108403in}{3.223331in}}%
\pgfpathlineto{\pgfqpoint{4.116081in}{3.242567in}}%
\pgfpathlineto{\pgfqpoint{4.123756in}{3.262089in}}%
\pgfpathlineto{\pgfqpoint{4.131430in}{3.281902in}}%
\pgfpathlineto{\pgfqpoint{4.139102in}{3.302015in}}%
\pgfpathlineto{\pgfqpoint{4.125798in}{3.313009in}}%
\pgfpathlineto{\pgfqpoint{4.112497in}{3.324229in}}%
\pgfpathlineto{\pgfqpoint{4.099197in}{3.335673in}}%
\pgfpathlineto{\pgfqpoint{4.085899in}{3.347345in}}%
\pgfpathlineto{\pgfqpoint{4.078223in}{3.326653in}}%
\pgfpathlineto{\pgfqpoint{4.070545in}{3.306268in}}%
\pgfpathlineto{\pgfqpoint{4.062865in}{3.286183in}}%
\pgfpathlineto{\pgfqpoint{4.055183in}{3.266392in}}%
\pgfpathclose%
\pgfusepath{fill}%
\end{pgfscope}%
\begin{pgfscope}%
\pgfpathrectangle{\pgfqpoint{1.150000in}{0.150000in}}{\pgfqpoint{5.700000in}{5.700000in}}%
\pgfusepath{clip}%
\pgfsetbuttcap%
\pgfsetroundjoin%
\definecolor{currentfill}{rgb}{0.119423,0.611141,0.538982}%
\pgfsetfillcolor{currentfill}%
\pgfsetfillopacity{0.800000}%
\pgfsetlinewidth{0.000000pt}%
\definecolor{currentstroke}{rgb}{0.000000,0.000000,0.000000}%
\pgfsetstrokecolor{currentstroke}%
\pgfsetdash{}{0pt}%
\pgfpathmoveto{\pgfqpoint{3.606958in}{3.932520in}}%
\pgfpathlineto{\pgfqpoint{3.620310in}{3.911525in}}%
\pgfpathlineto{\pgfqpoint{3.633656in}{3.890826in}}%
\pgfpathlineto{\pgfqpoint{3.646998in}{3.870421in}}%
\pgfpathlineto{\pgfqpoint{3.660334in}{3.850307in}}%
\pgfpathlineto{\pgfqpoint{3.668019in}{3.876108in}}%
\pgfpathlineto{\pgfqpoint{3.675700in}{3.902300in}}%
\pgfpathlineto{\pgfqpoint{3.683378in}{3.928891in}}%
\pgfpathlineto{\pgfqpoint{3.691053in}{3.955887in}}%
\pgfpathlineto{\pgfqpoint{3.677714in}{3.976628in}}%
\pgfpathlineto{\pgfqpoint{3.664371in}{3.997661in}}%
\pgfpathlineto{\pgfqpoint{3.651023in}{4.018990in}}%
\pgfpathlineto{\pgfqpoint{3.637669in}{4.040616in}}%
\pgfpathlineto{\pgfqpoint{3.629996in}{4.012977in}}%
\pgfpathlineto{\pgfqpoint{3.622320in}{3.985753in}}%
\pgfpathlineto{\pgfqpoint{3.614641in}{3.958936in}}%
\pgfpathlineto{\pgfqpoint{3.606958in}{3.932520in}}%
\pgfpathclose%
\pgfusepath{fill}%
\end{pgfscope}%
\begin{pgfscope}%
\pgfpathrectangle{\pgfqpoint{1.150000in}{0.150000in}}{\pgfqpoint{5.700000in}{5.700000in}}%
\pgfusepath{clip}%
\pgfsetbuttcap%
\pgfsetroundjoin%
\definecolor{currentfill}{rgb}{0.166617,0.463708,0.558119}%
\pgfsetfillcolor{currentfill}%
\pgfsetfillopacity{0.800000}%
\pgfsetlinewidth{0.000000pt}%
\definecolor{currentstroke}{rgb}{0.000000,0.000000,0.000000}%
\pgfsetstrokecolor{currentstroke}%
\pgfsetdash{}{0pt}%
\pgfpathmoveto{\pgfqpoint{4.474770in}{3.482631in}}%
\pgfpathlineto{\pgfqpoint{4.488112in}{3.472697in}}%
\pgfpathlineto{\pgfqpoint{4.501458in}{3.462968in}}%
\pgfpathlineto{\pgfqpoint{4.514809in}{3.453444in}}%
\pgfpathlineto{\pgfqpoint{4.528164in}{3.444124in}}%
\pgfpathlineto{\pgfqpoint{4.535813in}{3.466868in}}%
\pgfpathlineto{\pgfqpoint{4.543465in}{3.490009in}}%
\pgfpathlineto{\pgfqpoint{4.551119in}{3.513555in}}%
\pgfpathlineto{\pgfqpoint{4.558776in}{3.537513in}}%
\pgfpathlineto{\pgfqpoint{4.545427in}{3.547560in}}%
\pgfpathlineto{\pgfqpoint{4.532082in}{3.557810in}}%
\pgfpathlineto{\pgfqpoint{4.518742in}{3.568266in}}%
\pgfpathlineto{\pgfqpoint{4.505406in}{3.578927in}}%
\pgfpathlineto{\pgfqpoint{4.497743in}{3.554229in}}%
\pgfpathlineto{\pgfqpoint{4.490083in}{3.529953in}}%
\pgfpathlineto{\pgfqpoint{4.482426in}{3.506090in}}%
\pgfpathlineto{\pgfqpoint{4.474770in}{3.482631in}}%
\pgfpathclose%
\pgfusepath{fill}%
\end{pgfscope}%
\begin{pgfscope}%
\pgfpathrectangle{\pgfqpoint{1.150000in}{0.150000in}}{\pgfqpoint{5.700000in}{5.700000in}}%
\pgfusepath{clip}%
\pgfsetbuttcap%
\pgfsetroundjoin%
\definecolor{currentfill}{rgb}{0.177423,0.437527,0.557565}%
\pgfsetfillcolor{currentfill}%
\pgfsetfillopacity{0.800000}%
\pgfsetlinewidth{0.000000pt}%
\definecolor{currentstroke}{rgb}{0.000000,0.000000,0.000000}%
\pgfsetstrokecolor{currentstroke}%
\pgfsetdash{}{0pt}%
\pgfpathmoveto{\pgfqpoint{3.674417in}{3.431151in}}%
\pgfpathlineto{\pgfqpoint{3.687726in}{3.415195in}}%
\pgfpathlineto{\pgfqpoint{3.701032in}{3.399506in}}%
\pgfpathlineto{\pgfqpoint{3.714337in}{3.384080in}}%
\pgfpathlineto{\pgfqpoint{3.727640in}{3.368916in}}%
\pgfpathlineto{\pgfqpoint{3.735363in}{3.389058in}}%
\pgfpathlineto{\pgfqpoint{3.743082in}{3.409485in}}%
\pgfpathlineto{\pgfqpoint{3.750797in}{3.430202in}}%
\pgfpathlineto{\pgfqpoint{3.758509in}{3.451217in}}%
\pgfpathlineto{\pgfqpoint{3.745208in}{3.466893in}}%
\pgfpathlineto{\pgfqpoint{3.731906in}{3.482831in}}%
\pgfpathlineto{\pgfqpoint{3.718602in}{3.499034in}}%
\pgfpathlineto{\pgfqpoint{3.705295in}{3.515503in}}%
\pgfpathlineto{\pgfqpoint{3.697581in}{3.493963in}}%
\pgfpathlineto{\pgfqpoint{3.689863in}{3.472729in}}%
\pgfpathlineto{\pgfqpoint{3.682142in}{3.451793in}}%
\pgfpathlineto{\pgfqpoint{3.674417in}{3.431151in}}%
\pgfpathclose%
\pgfusepath{fill}%
\end{pgfscope}%
\begin{pgfscope}%
\pgfpathrectangle{\pgfqpoint{1.150000in}{0.150000in}}{\pgfqpoint{5.700000in}{5.700000in}}%
\pgfusepath{clip}%
\pgfsetbuttcap%
\pgfsetroundjoin%
\definecolor{currentfill}{rgb}{0.185556,0.418570,0.556753}%
\pgfsetfillcolor{currentfill}%
\pgfsetfillopacity{0.800000}%
\pgfsetlinewidth{0.000000pt}%
\definecolor{currentstroke}{rgb}{0.000000,0.000000,0.000000}%
\pgfsetstrokecolor{currentstroke}%
\pgfsetdash{}{0pt}%
\pgfpathmoveto{\pgfqpoint{3.727640in}{3.368916in}}%
\pgfpathlineto{\pgfqpoint{3.740942in}{3.354012in}}%
\pgfpathlineto{\pgfqpoint{3.754242in}{3.339366in}}%
\pgfpathlineto{\pgfqpoint{3.767541in}{3.324976in}}%
\pgfpathlineto{\pgfqpoint{3.780839in}{3.310840in}}%
\pgfpathlineto{\pgfqpoint{3.788559in}{3.330485in}}%
\pgfpathlineto{\pgfqpoint{3.796275in}{3.350406in}}%
\pgfpathlineto{\pgfqpoint{3.803987in}{3.370609in}}%
\pgfpathlineto{\pgfqpoint{3.811696in}{3.391100in}}%
\pgfpathlineto{\pgfqpoint{3.798401in}{3.405745in}}%
\pgfpathlineto{\pgfqpoint{3.785105in}{3.420645in}}%
\pgfpathlineto{\pgfqpoint{3.771808in}{3.435802in}}%
\pgfpathlineto{\pgfqpoint{3.758509in}{3.451217in}}%
\pgfpathlineto{\pgfqpoint{3.750797in}{3.430202in}}%
\pgfpathlineto{\pgfqpoint{3.743082in}{3.409485in}}%
\pgfpathlineto{\pgfqpoint{3.735363in}{3.389058in}}%
\pgfpathlineto{\pgfqpoint{3.727640in}{3.368916in}}%
\pgfpathclose%
\pgfusepath{fill}%
\end{pgfscope}%
\begin{pgfscope}%
\pgfpathrectangle{\pgfqpoint{1.150000in}{0.150000in}}{\pgfqpoint{5.700000in}{5.700000in}}%
\pgfusepath{clip}%
\pgfsetbuttcap%
\pgfsetroundjoin%
\definecolor{currentfill}{rgb}{0.214000,0.722114,0.469588}%
\pgfsetfillcolor{currentfill}%
\pgfsetfillopacity{0.800000}%
\pgfsetlinewidth{0.000000pt}%
\definecolor{currentstroke}{rgb}{0.000000,0.000000,0.000000}%
\pgfsetstrokecolor{currentstroke}%
\pgfsetdash{}{0pt}%
\pgfpathmoveto{\pgfqpoint{3.920256in}{4.271057in}}%
\pgfpathlineto{\pgfqpoint{3.933573in}{4.249996in}}%
\pgfpathlineto{\pgfqpoint{3.946886in}{4.229209in}}%
\pgfpathlineto{\pgfqpoint{3.960196in}{4.208694in}}%
\pgfpathlineto{\pgfqpoint{3.973503in}{4.188449in}}%
\pgfpathlineto{\pgfqpoint{3.981169in}{4.221232in}}%
\pgfpathlineto{\pgfqpoint{3.988836in}{4.254548in}}%
\pgfpathlineto{\pgfqpoint{3.996504in}{4.288408in}}%
\pgfpathlineto{\pgfqpoint{3.983193in}{4.309252in}}%
\pgfpathlineto{\pgfqpoint{3.969879in}{4.330367in}}%
\pgfpathlineto{\pgfqpoint{3.956561in}{4.351755in}}%
\pgfpathlineto{\pgfqpoint{3.943240in}{4.373419in}}%
\pgfpathlineto{\pgfqpoint{3.935578in}{4.338748in}}%
\pgfpathlineto{\pgfqpoint{3.927917in}{4.304631in}}%
\pgfpathlineto{\pgfqpoint{3.920256in}{4.271057in}}%
\pgfpathclose%
\pgfusepath{fill}%
\end{pgfscope}%
\begin{pgfscope}%
\pgfpathrectangle{\pgfqpoint{1.150000in}{0.150000in}}{\pgfqpoint{5.700000in}{5.700000in}}%
\pgfusepath{clip}%
\pgfsetbuttcap%
\pgfsetroundjoin%
\definecolor{currentfill}{rgb}{0.196571,0.711827,0.479221}%
\pgfsetfillcolor{currentfill}%
\pgfsetfillopacity{0.800000}%
\pgfsetlinewidth{0.000000pt}%
\definecolor{currentstroke}{rgb}{0.000000,0.000000,0.000000}%
\pgfsetstrokecolor{currentstroke}%
\pgfsetdash{}{0pt}%
\pgfpathmoveto{\pgfqpoint{3.836330in}{4.225939in}}%
\pgfpathlineto{\pgfqpoint{3.849657in}{4.204534in}}%
\pgfpathlineto{\pgfqpoint{3.862980in}{4.183410in}}%
\pgfpathlineto{\pgfqpoint{3.876299in}{4.162566in}}%
\pgfpathlineto{\pgfqpoint{3.889615in}{4.141999in}}%
\pgfpathlineto{\pgfqpoint{3.897275in}{4.173496in}}%
\pgfpathlineto{\pgfqpoint{3.904935in}{4.205499in}}%
\pgfpathlineto{\pgfqpoint{3.912596in}{4.238016in}}%
\pgfpathlineto{\pgfqpoint{3.920256in}{4.271057in}}%
\pgfpathlineto{\pgfqpoint{3.906935in}{4.292394in}}%
\pgfpathlineto{\pgfqpoint{3.893611in}{4.314009in}}%
\pgfpathlineto{\pgfqpoint{3.880283in}{4.335906in}}%
\pgfpathlineto{\pgfqpoint{3.866950in}{4.358086in}}%
\pgfpathlineto{\pgfqpoint{3.859296in}{4.324258in}}%
\pgfpathlineto{\pgfqpoint{3.851641in}{4.290964in}}%
\pgfpathlineto{\pgfqpoint{3.843986in}{4.258194in}}%
\pgfpathlineto{\pgfqpoint{3.836330in}{4.225939in}}%
\pgfpathclose%
\pgfusepath{fill}%
\end{pgfscope}%
\begin{pgfscope}%
\pgfpathrectangle{\pgfqpoint{1.150000in}{0.150000in}}{\pgfqpoint{5.700000in}{5.700000in}}%
\pgfusepath{clip}%
\pgfsetbuttcap%
\pgfsetroundjoin%
\definecolor{currentfill}{rgb}{0.125394,0.574318,0.549086}%
\pgfsetfillcolor{currentfill}%
\pgfsetfillopacity{0.800000}%
\pgfsetlinewidth{0.000000pt}%
\definecolor{currentstroke}{rgb}{0.000000,0.000000,0.000000}%
\pgfsetstrokecolor{currentstroke}%
\pgfsetdash{}{0pt}%
\pgfpathmoveto{\pgfqpoint{3.576191in}{3.830710in}}%
\pgfpathlineto{\pgfqpoint{3.589542in}{3.810309in}}%
\pgfpathlineto{\pgfqpoint{3.602887in}{3.790203in}}%
\pgfpathlineto{\pgfqpoint{3.616228in}{3.770390in}}%
\pgfpathlineto{\pgfqpoint{3.629564in}{3.750867in}}%
\pgfpathlineto{\pgfqpoint{3.637262in}{3.775177in}}%
\pgfpathlineto{\pgfqpoint{3.644956in}{3.799848in}}%
\pgfpathlineto{\pgfqpoint{3.652647in}{3.824890in}}%
\pgfpathlineto{\pgfqpoint{3.660334in}{3.850307in}}%
\pgfpathlineto{\pgfqpoint{3.646998in}{3.870421in}}%
\pgfpathlineto{\pgfqpoint{3.633656in}{3.890826in}}%
\pgfpathlineto{\pgfqpoint{3.620310in}{3.911525in}}%
\pgfpathlineto{\pgfqpoint{3.606958in}{3.932520in}}%
\pgfpathlineto{\pgfqpoint{3.599272in}{3.906496in}}%
\pgfpathlineto{\pgfqpoint{3.591582in}{3.880858in}}%
\pgfpathlineto{\pgfqpoint{3.583889in}{3.855599in}}%
\pgfpathlineto{\pgfqpoint{3.576191in}{3.830710in}}%
\pgfpathclose%
\pgfusepath{fill}%
\end{pgfscope}%
\begin{pgfscope}%
\pgfpathrectangle{\pgfqpoint{1.150000in}{0.150000in}}{\pgfqpoint{5.700000in}{5.700000in}}%
\pgfusepath{clip}%
\pgfsetbuttcap%
\pgfsetroundjoin%
\definecolor{currentfill}{rgb}{0.128087,0.647749,0.523491}%
\pgfsetfillcolor{currentfill}%
\pgfsetfillopacity{0.800000}%
\pgfsetlinewidth{0.000000pt}%
\definecolor{currentstroke}{rgb}{0.000000,0.000000,0.000000}%
\pgfsetstrokecolor{currentstroke}%
\pgfsetdash{}{0pt}%
\pgfpathmoveto{\pgfqpoint{3.637669in}{4.040616in}}%
\pgfpathlineto{\pgfqpoint{3.651023in}{4.018990in}}%
\pgfpathlineto{\pgfqpoint{3.664371in}{3.997661in}}%
\pgfpathlineto{\pgfqpoint{3.677714in}{3.976628in}}%
\pgfpathlineto{\pgfqpoint{3.691053in}{3.955887in}}%
\pgfpathlineto{\pgfqpoint{3.698725in}{3.983297in}}%
\pgfpathlineto{\pgfqpoint{3.706395in}{4.011128in}}%
\pgfpathlineto{\pgfqpoint{3.714062in}{4.039389in}}%
\pgfpathlineto{\pgfqpoint{3.721727in}{4.068087in}}%
\pgfpathlineto{\pgfqpoint{3.708386in}{4.089493in}}%
\pgfpathlineto{\pgfqpoint{3.695040in}{4.111192in}}%
\pgfpathlineto{\pgfqpoint{3.681688in}{4.133188in}}%
\pgfpathlineto{\pgfqpoint{3.668331in}{4.155483in}}%
\pgfpathlineto{\pgfqpoint{3.660670in}{4.126104in}}%
\pgfpathlineto{\pgfqpoint{3.653006in}{4.097172in}}%
\pgfpathlineto{\pgfqpoint{3.645339in}{4.068678in}}%
\pgfpathlineto{\pgfqpoint{3.637669in}{4.040616in}}%
\pgfpathclose%
\pgfusepath{fill}%
\end{pgfscope}%
\begin{pgfscope}%
\pgfpathrectangle{\pgfqpoint{1.150000in}{0.150000in}}{\pgfqpoint{5.700000in}{5.700000in}}%
\pgfusepath{clip}%
\pgfsetbuttcap%
\pgfsetroundjoin%
\definecolor{currentfill}{rgb}{0.168126,0.459988,0.558082}%
\pgfsetfillcolor{currentfill}%
\pgfsetfillopacity{0.800000}%
\pgfsetlinewidth{0.000000pt}%
\definecolor{currentstroke}{rgb}{0.000000,0.000000,0.000000}%
\pgfsetstrokecolor{currentstroke}%
\pgfsetdash{}{0pt}%
\pgfpathmoveto{\pgfqpoint{3.621155in}{3.497677in}}%
\pgfpathlineto{\pgfqpoint{3.634475in}{3.480636in}}%
\pgfpathlineto{\pgfqpoint{3.647792in}{3.463869in}}%
\pgfpathlineto{\pgfqpoint{3.661105in}{3.447375in}}%
\pgfpathlineto{\pgfqpoint{3.674417in}{3.431151in}}%
\pgfpathlineto{\pgfqpoint{3.682142in}{3.451793in}}%
\pgfpathlineto{\pgfqpoint{3.689863in}{3.472729in}}%
\pgfpathlineto{\pgfqpoint{3.697581in}{3.493963in}}%
\pgfpathlineto{\pgfqpoint{3.705295in}{3.515503in}}%
\pgfpathlineto{\pgfqpoint{3.691986in}{3.532241in}}%
\pgfpathlineto{\pgfqpoint{3.678674in}{3.549250in}}%
\pgfpathlineto{\pgfqpoint{3.665360in}{3.566532in}}%
\pgfpathlineto{\pgfqpoint{3.652042in}{3.584090in}}%
\pgfpathlineto{\pgfqpoint{3.644326in}{3.562022in}}%
\pgfpathlineto{\pgfqpoint{3.636607in}{3.540268in}}%
\pgfpathlineto{\pgfqpoint{3.628883in}{3.518821in}}%
\pgfpathlineto{\pgfqpoint{3.621155in}{3.497677in}}%
\pgfpathclose%
\pgfusepath{fill}%
\end{pgfscope}%
\begin{pgfscope}%
\pgfpathrectangle{\pgfqpoint{1.150000in}{0.150000in}}{\pgfqpoint{5.700000in}{5.700000in}}%
\pgfusepath{clip}%
\pgfsetbuttcap%
\pgfsetroundjoin%
\definecolor{currentfill}{rgb}{0.175707,0.697900,0.491033}%
\pgfsetfillcolor{currentfill}%
\pgfsetfillopacity{0.800000}%
\pgfsetlinewidth{0.000000pt}%
\definecolor{currentstroke}{rgb}{0.000000,0.000000,0.000000}%
\pgfsetstrokecolor{currentstroke}%
\pgfsetdash{}{0pt}%
\pgfpathmoveto{\pgfqpoint{3.752365in}{4.187419in}}%
\pgfpathlineto{\pgfqpoint{3.765706in}{4.165603in}}%
\pgfpathlineto{\pgfqpoint{3.779041in}{4.144078in}}%
\pgfpathlineto{\pgfqpoint{3.792372in}{4.122839in}}%
\pgfpathlineto{\pgfqpoint{3.805699in}{4.101886in}}%
\pgfpathlineto{\pgfqpoint{3.813358in}{4.132172in}}%
\pgfpathlineto{\pgfqpoint{3.821017in}{4.162938in}}%
\pgfpathlineto{\pgfqpoint{3.828674in}{4.194190in}}%
\pgfpathlineto{\pgfqpoint{3.836330in}{4.225939in}}%
\pgfpathlineto{\pgfqpoint{3.822999in}{4.247629in}}%
\pgfpathlineto{\pgfqpoint{3.809663in}{4.269605in}}%
\pgfpathlineto{\pgfqpoint{3.796323in}{4.291870in}}%
\pgfpathlineto{\pgfqpoint{3.782977in}{4.314427in}}%
\pgfpathlineto{\pgfqpoint{3.775326in}{4.281924in}}%
\pgfpathlineto{\pgfqpoint{3.767674in}{4.249928in}}%
\pgfpathlineto{\pgfqpoint{3.760021in}{4.218430in}}%
\pgfpathlineto{\pgfqpoint{3.752365in}{4.187419in}}%
\pgfpathclose%
\pgfusepath{fill}%
\end{pgfscope}%
\begin{pgfscope}%
\pgfpathrectangle{\pgfqpoint{1.150000in}{0.150000in}}{\pgfqpoint{5.700000in}{5.700000in}}%
\pgfusepath{clip}%
\pgfsetbuttcap%
\pgfsetroundjoin%
\definecolor{currentfill}{rgb}{0.195860,0.395433,0.555276}%
\pgfsetfillcolor{currentfill}%
\pgfsetfillopacity{0.800000}%
\pgfsetlinewidth{0.000000pt}%
\definecolor{currentstroke}{rgb}{0.000000,0.000000,0.000000}%
\pgfsetstrokecolor{currentstroke}%
\pgfsetdash{}{0pt}%
\pgfpathmoveto{\pgfqpoint{3.780839in}{3.310840in}}%
\pgfpathlineto{\pgfqpoint{3.794137in}{3.296957in}}%
\pgfpathlineto{\pgfqpoint{3.807434in}{3.283325in}}%
\pgfpathlineto{\pgfqpoint{3.820730in}{3.269942in}}%
\pgfpathlineto{\pgfqpoint{3.834027in}{3.256806in}}%
\pgfpathlineto{\pgfqpoint{3.841743in}{3.275954in}}%
\pgfpathlineto{\pgfqpoint{3.849455in}{3.295371in}}%
\pgfpathlineto{\pgfqpoint{3.857164in}{3.315062in}}%
\pgfpathlineto{\pgfqpoint{3.864870in}{3.335033in}}%
\pgfpathlineto{\pgfqpoint{3.851577in}{3.348676in}}%
\pgfpathlineto{\pgfqpoint{3.838284in}{3.362568in}}%
\pgfpathlineto{\pgfqpoint{3.824990in}{3.376708in}}%
\pgfpathlineto{\pgfqpoint{3.811696in}{3.391100in}}%
\pgfpathlineto{\pgfqpoint{3.803987in}{3.370609in}}%
\pgfpathlineto{\pgfqpoint{3.796275in}{3.350406in}}%
\pgfpathlineto{\pgfqpoint{3.788559in}{3.330485in}}%
\pgfpathlineto{\pgfqpoint{3.780839in}{3.310840in}}%
\pgfpathclose%
\pgfusepath{fill}%
\end{pgfscope}%
\begin{pgfscope}%
\pgfpathrectangle{\pgfqpoint{1.150000in}{0.150000in}}{\pgfqpoint{5.700000in}{5.700000in}}%
\pgfusepath{clip}%
\pgfsetbuttcap%
\pgfsetroundjoin%
\definecolor{currentfill}{rgb}{0.194100,0.399323,0.555565}%
\pgfsetfillcolor{currentfill}%
\pgfsetfillopacity{0.800000}%
\pgfsetlinewidth{0.000000pt}%
\definecolor{currentstroke}{rgb}{0.000000,0.000000,0.000000}%
\pgfsetstrokecolor{currentstroke}%
\pgfsetdash{}{0pt}%
\pgfpathmoveto{\pgfqpoint{4.276275in}{3.300780in}}%
\pgfpathlineto{\pgfqpoint{4.289602in}{3.291153in}}%
\pgfpathlineto{\pgfqpoint{4.302932in}{3.281739in}}%
\pgfpathlineto{\pgfqpoint{4.316266in}{3.272537in}}%
\pgfpathlineto{\pgfqpoint{4.329605in}{3.263545in}}%
\pgfpathlineto{\pgfqpoint{4.337256in}{3.283403in}}%
\pgfpathlineto{\pgfqpoint{4.344908in}{3.303578in}}%
\pgfpathlineto{\pgfqpoint{4.352559in}{3.324078in}}%
\pgfpathlineto{\pgfqpoint{4.360210in}{3.344909in}}%
\pgfpathlineto{\pgfqpoint{4.346878in}{3.354530in}}%
\pgfpathlineto{\pgfqpoint{4.333549in}{3.364361in}}%
\pgfpathlineto{\pgfqpoint{4.320224in}{3.374405in}}%
\pgfpathlineto{\pgfqpoint{4.306903in}{3.384661in}}%
\pgfpathlineto{\pgfqpoint{4.299246in}{3.363188in}}%
\pgfpathlineto{\pgfqpoint{4.291589in}{3.342056in}}%
\pgfpathlineto{\pgfqpoint{4.283932in}{3.321255in}}%
\pgfpathlineto{\pgfqpoint{4.276275in}{3.300780in}}%
\pgfpathclose%
\pgfusepath{fill}%
\end{pgfscope}%
\begin{pgfscope}%
\pgfpathrectangle{\pgfqpoint{1.150000in}{0.150000in}}{\pgfqpoint{5.700000in}{5.700000in}}%
\pgfusepath{clip}%
\pgfsetbuttcap%
\pgfsetroundjoin%
\definecolor{currentfill}{rgb}{0.187231,0.414746,0.556547}%
\pgfsetfillcolor{currentfill}%
\pgfsetfillopacity{0.800000}%
\pgfsetlinewidth{0.000000pt}%
\definecolor{currentstroke}{rgb}{0.000000,0.000000,0.000000}%
\pgfsetstrokecolor{currentstroke}%
\pgfsetdash{}{0pt}%
\pgfpathmoveto{\pgfqpoint{4.360210in}{3.344909in}}%
\pgfpathlineto{\pgfqpoint{4.373547in}{3.335499in}}%
\pgfpathlineto{\pgfqpoint{4.386887in}{3.326297in}}%
\pgfpathlineto{\pgfqpoint{4.400232in}{3.317303in}}%
\pgfpathlineto{\pgfqpoint{4.413582in}{3.308517in}}%
\pgfpathlineto{\pgfqpoint{4.421227in}{3.329039in}}%
\pgfpathlineto{\pgfqpoint{4.428873in}{3.349900in}}%
\pgfpathlineto{\pgfqpoint{4.436520in}{3.371109in}}%
\pgfpathlineto{\pgfqpoint{4.444167in}{3.392671in}}%
\pgfpathlineto{\pgfqpoint{4.430824in}{3.402119in}}%
\pgfpathlineto{\pgfqpoint{4.417485in}{3.411773in}}%
\pgfpathlineto{\pgfqpoint{4.404150in}{3.421636in}}%
\pgfpathlineto{\pgfqpoint{4.390819in}{3.431708in}}%
\pgfpathlineto{\pgfqpoint{4.383166in}{3.409472in}}%
\pgfpathlineto{\pgfqpoint{4.375513in}{3.387598in}}%
\pgfpathlineto{\pgfqpoint{4.367861in}{3.366080in}}%
\pgfpathlineto{\pgfqpoint{4.360210in}{3.344909in}}%
\pgfpathclose%
\pgfusepath{fill}%
\end{pgfscope}%
\begin{pgfscope}%
\pgfpathrectangle{\pgfqpoint{1.150000in}{0.150000in}}{\pgfqpoint{5.700000in}{5.700000in}}%
\pgfusepath{clip}%
\pgfsetbuttcap%
\pgfsetroundjoin%
\definecolor{currentfill}{rgb}{0.206756,0.371758,0.553117}%
\pgfsetfillcolor{currentfill}%
\pgfsetfillopacity{0.800000}%
\pgfsetlinewidth{0.000000pt}%
\definecolor{currentstroke}{rgb}{0.000000,0.000000,0.000000}%
\pgfsetstrokecolor{currentstroke}%
\pgfsetdash{}{0pt}%
\pgfpathmoveto{\pgfqpoint{3.971227in}{3.234601in}}%
\pgfpathlineto{\pgfqpoint{3.984526in}{3.223113in}}%
\pgfpathlineto{\pgfqpoint{3.997826in}{3.211857in}}%
\pgfpathlineto{\pgfqpoint{4.011129in}{3.200832in}}%
\pgfpathlineto{\pgfqpoint{4.024433in}{3.190035in}}%
\pgfpathlineto{\pgfqpoint{4.032124in}{3.208716in}}%
\pgfpathlineto{\pgfqpoint{4.039813in}{3.227665in}}%
\pgfpathlineto{\pgfqpoint{4.047499in}{3.246888in}}%
\pgfpathlineto{\pgfqpoint{4.055183in}{3.266392in}}%
\pgfpathlineto{\pgfqpoint{4.041883in}{3.277724in}}%
\pgfpathlineto{\pgfqpoint{4.028585in}{3.289285in}}%
\pgfpathlineto{\pgfqpoint{4.015289in}{3.301078in}}%
\pgfpathlineto{\pgfqpoint{4.001994in}{3.313102in}}%
\pgfpathlineto{\pgfqpoint{3.994306in}{3.293050in}}%
\pgfpathlineto{\pgfqpoint{3.986615in}{3.273287in}}%
\pgfpathlineto{\pgfqpoint{3.978922in}{3.253806in}}%
\pgfpathlineto{\pgfqpoint{3.971227in}{3.234601in}}%
\pgfpathclose%
\pgfusepath{fill}%
\end{pgfscope}%
\begin{pgfscope}%
\pgfpathrectangle{\pgfqpoint{1.150000in}{0.150000in}}{\pgfqpoint{5.700000in}{5.700000in}}%
\pgfusepath{clip}%
\pgfsetbuttcap%
\pgfsetroundjoin%
\definecolor{currentfill}{rgb}{0.136408,0.541173,0.554483}%
\pgfsetfillcolor{currentfill}%
\pgfsetfillopacity{0.800000}%
\pgfsetlinewidth{0.000000pt}%
\definecolor{currentstroke}{rgb}{0.000000,0.000000,0.000000}%
\pgfsetstrokecolor{currentstroke}%
\pgfsetdash{}{0pt}%
\pgfpathmoveto{\pgfqpoint{3.545361in}{3.734733in}}%
\pgfpathlineto{\pgfqpoint{3.558712in}{3.714890in}}%
\pgfpathlineto{\pgfqpoint{3.572058in}{3.695341in}}%
\pgfpathlineto{\pgfqpoint{3.585399in}{3.676084in}}%
\pgfpathlineto{\pgfqpoint{3.598735in}{3.657116in}}%
\pgfpathlineto{\pgfqpoint{3.606448in}{3.680044in}}%
\pgfpathlineto{\pgfqpoint{3.614157in}{3.703308in}}%
\pgfpathlineto{\pgfqpoint{3.621863in}{3.726913in}}%
\pgfpathlineto{\pgfqpoint{3.629564in}{3.750867in}}%
\pgfpathlineto{\pgfqpoint{3.616228in}{3.770390in}}%
\pgfpathlineto{\pgfqpoint{3.602887in}{3.790203in}}%
\pgfpathlineto{\pgfqpoint{3.589542in}{3.810309in}}%
\pgfpathlineto{\pgfqpoint{3.576191in}{3.830710in}}%
\pgfpathlineto{\pgfqpoint{3.568490in}{3.806186in}}%
\pgfpathlineto{\pgfqpoint{3.560784in}{3.782020in}}%
\pgfpathlineto{\pgfqpoint{3.553075in}{3.758205in}}%
\pgfpathlineto{\pgfqpoint{3.545361in}{3.734733in}}%
\pgfpathclose%
\pgfusepath{fill}%
\end{pgfscope}%
\begin{pgfscope}%
\pgfpathrectangle{\pgfqpoint{1.150000in}{0.150000in}}{\pgfqpoint{5.700000in}{5.700000in}}%
\pgfusepath{clip}%
\pgfsetbuttcap%
\pgfsetroundjoin%
\definecolor{currentfill}{rgb}{0.201239,0.383670,0.554294}%
\pgfsetfillcolor{currentfill}%
\pgfsetfillopacity{0.800000}%
\pgfsetlinewidth{0.000000pt}%
\definecolor{currentstroke}{rgb}{0.000000,0.000000,0.000000}%
\pgfsetstrokecolor{currentstroke}%
\pgfsetdash{}{0pt}%
\pgfpathmoveto{\pgfqpoint{4.192345in}{3.260255in}}%
\pgfpathlineto{\pgfqpoint{4.205663in}{3.250364in}}%
\pgfpathlineto{\pgfqpoint{4.218985in}{3.240689in}}%
\pgfpathlineto{\pgfqpoint{4.232310in}{3.231231in}}%
\pgfpathlineto{\pgfqpoint{4.245638in}{3.221987in}}%
\pgfpathlineto{\pgfqpoint{4.253299in}{3.241234in}}%
\pgfpathlineto{\pgfqpoint{4.260959in}{3.260777in}}%
\pgfpathlineto{\pgfqpoint{4.268617in}{3.280623in}}%
\pgfpathlineto{\pgfqpoint{4.276275in}{3.300780in}}%
\pgfpathlineto{\pgfqpoint{4.262952in}{3.310621in}}%
\pgfpathlineto{\pgfqpoint{4.249633in}{3.320677in}}%
\pgfpathlineto{\pgfqpoint{4.236316in}{3.330949in}}%
\pgfpathlineto{\pgfqpoint{4.223003in}{3.341439in}}%
\pgfpathlineto{\pgfqpoint{4.215340in}{3.320672in}}%
\pgfpathlineto{\pgfqpoint{4.207676in}{3.300224in}}%
\pgfpathlineto{\pgfqpoint{4.200011in}{3.280087in}}%
\pgfpathlineto{\pgfqpoint{4.192345in}{3.260255in}}%
\pgfpathclose%
\pgfusepath{fill}%
\end{pgfscope}%
\begin{pgfscope}%
\pgfpathrectangle{\pgfqpoint{1.150000in}{0.150000in}}{\pgfqpoint{5.700000in}{5.700000in}}%
\pgfusepath{clip}%
\pgfsetbuttcap%
\pgfsetroundjoin%
\definecolor{currentfill}{rgb}{0.157729,0.485932,0.558013}%
\pgfsetfillcolor{currentfill}%
\pgfsetfillopacity{0.800000}%
\pgfsetlinewidth{0.000000pt}%
\definecolor{currentstroke}{rgb}{0.000000,0.000000,0.000000}%
\pgfsetstrokecolor{currentstroke}%
\pgfsetdash{}{0pt}%
\pgfpathmoveto{\pgfqpoint{3.567842in}{3.568631in}}%
\pgfpathlineto{\pgfqpoint{3.581176in}{3.550469in}}%
\pgfpathlineto{\pgfqpoint{3.594506in}{3.532591in}}%
\pgfpathlineto{\pgfqpoint{3.607832in}{3.514995in}}%
\pgfpathlineto{\pgfqpoint{3.621155in}{3.497677in}}%
\pgfpathlineto{\pgfqpoint{3.628883in}{3.518821in}}%
\pgfpathlineto{\pgfqpoint{3.636607in}{3.540268in}}%
\pgfpathlineto{\pgfqpoint{3.644326in}{3.562022in}}%
\pgfpathlineto{\pgfqpoint{3.652042in}{3.584090in}}%
\pgfpathlineto{\pgfqpoint{3.638721in}{3.601924in}}%
\pgfpathlineto{\pgfqpoint{3.625396in}{3.620039in}}%
\pgfpathlineto{\pgfqpoint{3.612068in}{3.638435in}}%
\pgfpathlineto{\pgfqpoint{3.598735in}{3.657116in}}%
\pgfpathlineto{\pgfqpoint{3.591018in}{3.634517in}}%
\pgfpathlineto{\pgfqpoint{3.583297in}{3.612241in}}%
\pgfpathlineto{\pgfqpoint{3.575572in}{3.590281in}}%
\pgfpathlineto{\pgfqpoint{3.567842in}{3.568631in}}%
\pgfpathclose%
\pgfusepath{fill}%
\end{pgfscope}%
\begin{pgfscope}%
\pgfpathrectangle{\pgfqpoint{1.150000in}{0.150000in}}{\pgfqpoint{5.700000in}{5.700000in}}%
\pgfusepath{clip}%
\pgfsetbuttcap%
\pgfsetroundjoin%
\definecolor{currentfill}{rgb}{0.160665,0.478540,0.558115}%
\pgfsetfillcolor{currentfill}%
\pgfsetfillopacity{0.800000}%
\pgfsetlinewidth{0.000000pt}%
\definecolor{currentstroke}{rgb}{0.000000,0.000000,0.000000}%
\pgfsetstrokecolor{currentstroke}%
\pgfsetdash{}{0pt}%
\pgfpathmoveto{\pgfqpoint{4.558776in}{3.537513in}}%
\pgfpathlineto{\pgfqpoint{4.572129in}{3.527670in}}%
\pgfpathlineto{\pgfqpoint{4.585488in}{3.518029in}}%
\pgfpathlineto{\pgfqpoint{4.598851in}{3.508590in}}%
\pgfpathlineto{\pgfqpoint{4.612218in}{3.499350in}}%
\pgfpathlineto{\pgfqpoint{4.619872in}{3.522985in}}%
\pgfpathlineto{\pgfqpoint{4.627529in}{3.547043in}}%
\pgfpathlineto{\pgfqpoint{4.635190in}{3.571533in}}%
\pgfpathlineto{\pgfqpoint{4.621827in}{3.581337in}}%
\pgfpathlineto{\pgfqpoint{4.608469in}{3.591343in}}%
\pgfpathlineto{\pgfqpoint{4.595116in}{3.601550in}}%
\pgfpathlineto{\pgfqpoint{4.581767in}{3.611960in}}%
\pgfpathlineto{\pgfqpoint{4.574100in}{3.586707in}}%
\pgfpathlineto{\pgfqpoint{4.566436in}{3.561895in}}%
\pgfpathlineto{\pgfqpoint{4.558776in}{3.537513in}}%
\pgfpathclose%
\pgfusepath{fill}%
\end{pgfscope}%
\begin{pgfscope}%
\pgfpathrectangle{\pgfqpoint{1.150000in}{0.150000in}}{\pgfqpoint{5.700000in}{5.700000in}}%
\pgfusepath{clip}%
\pgfsetbuttcap%
\pgfsetroundjoin%
\definecolor{currentfill}{rgb}{0.203063,0.379716,0.553925}%
\pgfsetfillcolor{currentfill}%
\pgfsetfillopacity{0.800000}%
\pgfsetlinewidth{0.000000pt}%
\definecolor{currentstroke}{rgb}{0.000000,0.000000,0.000000}%
\pgfsetstrokecolor{currentstroke}%
\pgfsetdash{}{0pt}%
\pgfpathmoveto{\pgfqpoint{3.834027in}{3.256806in}}%
\pgfpathlineto{\pgfqpoint{3.847323in}{3.243915in}}%
\pgfpathlineto{\pgfqpoint{3.860620in}{3.231268in}}%
\pgfpathlineto{\pgfqpoint{3.873917in}{3.218864in}}%
\pgfpathlineto{\pgfqpoint{3.887215in}{3.206700in}}%
\pgfpathlineto{\pgfqpoint{3.894927in}{3.225354in}}%
\pgfpathlineto{\pgfqpoint{3.902635in}{3.244269in}}%
\pgfpathlineto{\pgfqpoint{3.910341in}{3.263449in}}%
\pgfpathlineto{\pgfqpoint{3.918043in}{3.282901in}}%
\pgfpathlineto{\pgfqpoint{3.904749in}{3.295571in}}%
\pgfpathlineto{\pgfqpoint{3.891456in}{3.308481in}}%
\pgfpathlineto{\pgfqpoint{3.878163in}{3.321635in}}%
\pgfpathlineto{\pgfqpoint{3.864870in}{3.335033in}}%
\pgfpathlineto{\pgfqpoint{3.857164in}{3.315062in}}%
\pgfpathlineto{\pgfqpoint{3.849455in}{3.295371in}}%
\pgfpathlineto{\pgfqpoint{3.841743in}{3.275954in}}%
\pgfpathlineto{\pgfqpoint{3.834027in}{3.256806in}}%
\pgfpathclose%
\pgfusepath{fill}%
\end{pgfscope}%
\begin{pgfscope}%
\pgfpathrectangle{\pgfqpoint{1.150000in}{0.150000in}}{\pgfqpoint{5.700000in}{5.700000in}}%
\pgfusepath{clip}%
\pgfsetbuttcap%
\pgfsetroundjoin%
\definecolor{currentfill}{rgb}{0.162016,0.687316,0.499129}%
\pgfsetfillcolor{currentfill}%
\pgfsetfillopacity{0.800000}%
\pgfsetlinewidth{0.000000pt}%
\definecolor{currentstroke}{rgb}{0.000000,0.000000,0.000000}%
\pgfsetstrokecolor{currentstroke}%
\pgfsetdash{}{0pt}%
\pgfpathmoveto{\pgfqpoint{3.668331in}{4.155483in}}%
\pgfpathlineto{\pgfqpoint{3.681688in}{4.133188in}}%
\pgfpathlineto{\pgfqpoint{3.695040in}{4.111192in}}%
\pgfpathlineto{\pgfqpoint{3.708386in}{4.089493in}}%
\pgfpathlineto{\pgfqpoint{3.721727in}{4.068087in}}%
\pgfpathlineto{\pgfqpoint{3.729390in}{4.097231in}}%
\pgfpathlineto{\pgfqpoint{3.737050in}{4.126829in}}%
\pgfpathlineto{\pgfqpoint{3.744709in}{4.156889in}}%
\pgfpathlineto{\pgfqpoint{3.752365in}{4.187419in}}%
\pgfpathlineto{\pgfqpoint{3.739020in}{4.209528in}}%
\pgfpathlineto{\pgfqpoint{3.725670in}{4.231932in}}%
\pgfpathlineto{\pgfqpoint{3.712314in}{4.254633in}}%
\pgfpathlineto{\pgfqpoint{3.698953in}{4.277635in}}%
\pgfpathlineto{\pgfqpoint{3.691301in}{4.246385in}}%
\pgfpathlineto{\pgfqpoint{3.683647in}{4.215615in}}%
\pgfpathlineto{\pgfqpoint{3.675990in}{4.185317in}}%
\pgfpathlineto{\pgfqpoint{3.668331in}{4.155483in}}%
\pgfpathclose%
\pgfusepath{fill}%
\end{pgfscope}%
\begin{pgfscope}%
\pgfpathrectangle{\pgfqpoint{1.150000in}{0.150000in}}{\pgfqpoint{5.700000in}{5.700000in}}%
\pgfusepath{clip}%
\pgfsetbuttcap%
\pgfsetroundjoin%
\definecolor{currentfill}{rgb}{0.179019,0.433756,0.557430}%
\pgfsetfillcolor{currentfill}%
\pgfsetfillopacity{0.800000}%
\pgfsetlinewidth{0.000000pt}%
\definecolor{currentstroke}{rgb}{0.000000,0.000000,0.000000}%
\pgfsetstrokecolor{currentstroke}%
\pgfsetdash{}{0pt}%
\pgfpathmoveto{\pgfqpoint{4.444167in}{3.392671in}}%
\pgfpathlineto{\pgfqpoint{4.457515in}{3.383430in}}%
\pgfpathlineto{\pgfqpoint{4.470867in}{3.374395in}}%
\pgfpathlineto{\pgfqpoint{4.484224in}{3.365563in}}%
\pgfpathlineto{\pgfqpoint{4.497586in}{3.356935in}}%
\pgfpathlineto{\pgfqpoint{4.505228in}{3.378180in}}%
\pgfpathlineto{\pgfqpoint{4.512872in}{3.399788in}}%
\pgfpathlineto{\pgfqpoint{4.520517in}{3.421766in}}%
\pgfpathlineto{\pgfqpoint{4.528164in}{3.444124in}}%
\pgfpathlineto{\pgfqpoint{4.514809in}{3.453444in}}%
\pgfpathlineto{\pgfqpoint{4.501458in}{3.462968in}}%
\pgfpathlineto{\pgfqpoint{4.488112in}{3.472697in}}%
\pgfpathlineto{\pgfqpoint{4.474770in}{3.482631in}}%
\pgfpathlineto{\pgfqpoint{4.467117in}{3.459569in}}%
\pgfpathlineto{\pgfqpoint{4.459466in}{3.436893in}}%
\pgfpathlineto{\pgfqpoint{4.451816in}{3.414597in}}%
\pgfpathlineto{\pgfqpoint{4.444167in}{3.392671in}}%
\pgfpathclose%
\pgfusepath{fill}%
\end{pgfscope}%
\begin{pgfscope}%
\pgfpathrectangle{\pgfqpoint{1.150000in}{0.150000in}}{\pgfqpoint{5.700000in}{5.700000in}}%
\pgfusepath{clip}%
\pgfsetbuttcap%
\pgfsetroundjoin%
\definecolor{currentfill}{rgb}{0.208623,0.367752,0.552675}%
\pgfsetfillcolor{currentfill}%
\pgfsetfillopacity{0.800000}%
\pgfsetlinewidth{0.000000pt}%
\definecolor{currentstroke}{rgb}{0.000000,0.000000,0.000000}%
\pgfsetstrokecolor{currentstroke}%
\pgfsetdash{}{0pt}%
\pgfpathmoveto{\pgfqpoint{4.108403in}{3.223331in}}%
\pgfpathlineto{\pgfqpoint{4.121715in}{3.213127in}}%
\pgfpathlineto{\pgfqpoint{4.135029in}{3.203143in}}%
\pgfpathlineto{\pgfqpoint{4.148346in}{3.193380in}}%
\pgfpathlineto{\pgfqpoint{4.161666in}{3.183836in}}%
\pgfpathlineto{\pgfqpoint{4.169338in}{3.202518in}}%
\pgfpathlineto{\pgfqpoint{4.177009in}{3.221477in}}%
\pgfpathlineto{\pgfqpoint{4.184678in}{3.240721in}}%
\pgfpathlineto{\pgfqpoint{4.192345in}{3.260255in}}%
\pgfpathlineto{\pgfqpoint{4.179030in}{3.270365in}}%
\pgfpathlineto{\pgfqpoint{4.165718in}{3.280694in}}%
\pgfpathlineto{\pgfqpoint{4.152409in}{3.291243in}}%
\pgfpathlineto{\pgfqpoint{4.139102in}{3.302015in}}%
\pgfpathlineto{\pgfqpoint{4.131430in}{3.281902in}}%
\pgfpathlineto{\pgfqpoint{4.123756in}{3.262089in}}%
\pgfpathlineto{\pgfqpoint{4.116081in}{3.242567in}}%
\pgfpathlineto{\pgfqpoint{4.108403in}{3.223331in}}%
\pgfpathclose%
\pgfusepath{fill}%
\end{pgfscope}%
\begin{pgfscope}%
\pgfpathrectangle{\pgfqpoint{1.150000in}{0.150000in}}{\pgfqpoint{5.700000in}{5.700000in}}%
\pgfusepath{clip}%
\pgfsetbuttcap%
\pgfsetroundjoin%
\definecolor{currentfill}{rgb}{0.171176,0.452530,0.557965}%
\pgfsetfillcolor{currentfill}%
\pgfsetfillopacity{0.800000}%
\pgfsetlinewidth{0.000000pt}%
\definecolor{currentstroke}{rgb}{0.000000,0.000000,0.000000}%
\pgfsetstrokecolor{currentstroke}%
\pgfsetdash{}{0pt}%
\pgfpathmoveto{\pgfqpoint{4.528164in}{3.444124in}}%
\pgfpathlineto{\pgfqpoint{4.541524in}{3.435006in}}%
\pgfpathlineto{\pgfqpoint{4.554889in}{3.426090in}}%
\pgfpathlineto{\pgfqpoint{4.568259in}{3.417375in}}%
\pgfpathlineto{\pgfqpoint{4.581634in}{3.408860in}}%
\pgfpathlineto{\pgfqpoint{4.589276in}{3.430892in}}%
\pgfpathlineto{\pgfqpoint{4.596921in}{3.453313in}}%
\pgfpathlineto{\pgfqpoint{4.604568in}{3.476129in}}%
\pgfpathlineto{\pgfqpoint{4.612218in}{3.499350in}}%
\pgfpathlineto{\pgfqpoint{4.598851in}{3.508590in}}%
\pgfpathlineto{\pgfqpoint{4.585488in}{3.518029in}}%
\pgfpathlineto{\pgfqpoint{4.572129in}{3.527670in}}%
\pgfpathlineto{\pgfqpoint{4.558776in}{3.537513in}}%
\pgfpathlineto{\pgfqpoint{4.551119in}{3.513555in}}%
\pgfpathlineto{\pgfqpoint{4.543465in}{3.490009in}}%
\pgfpathlineto{\pgfqpoint{4.535813in}{3.466868in}}%
\pgfpathlineto{\pgfqpoint{4.528164in}{3.444124in}}%
\pgfpathclose%
\pgfusepath{fill}%
\end{pgfscope}%
\begin{pgfscope}%
\pgfpathrectangle{\pgfqpoint{1.150000in}{0.150000in}}{\pgfqpoint{5.700000in}{5.700000in}}%
\pgfusepath{clip}%
\pgfsetbuttcap%
\pgfsetroundjoin%
\definecolor{currentfill}{rgb}{0.124780,0.640461,0.527068}%
\pgfsetfillcolor{currentfill}%
\pgfsetfillopacity{0.800000}%
\pgfsetlinewidth{0.000000pt}%
\definecolor{currentstroke}{rgb}{0.000000,0.000000,0.000000}%
\pgfsetstrokecolor{currentstroke}%
\pgfsetdash{}{0pt}%
\pgfpathmoveto{\pgfqpoint{3.553496in}{4.019519in}}%
\pgfpathlineto{\pgfqpoint{3.566870in}{3.997311in}}%
\pgfpathlineto{\pgfqpoint{3.580239in}{3.975410in}}%
\pgfpathlineto{\pgfqpoint{3.593601in}{3.953814in}}%
\pgfpathlineto{\pgfqpoint{3.606958in}{3.932520in}}%
\pgfpathlineto{\pgfqpoint{3.614641in}{3.958936in}}%
\pgfpathlineto{\pgfqpoint{3.622320in}{3.985753in}}%
\pgfpathlineto{\pgfqpoint{3.629996in}{4.012977in}}%
\pgfpathlineto{\pgfqpoint{3.637669in}{4.040616in}}%
\pgfpathlineto{\pgfqpoint{3.624310in}{4.062542in}}%
\pgfpathlineto{\pgfqpoint{3.610945in}{4.084772in}}%
\pgfpathlineto{\pgfqpoint{3.597574in}{4.107307in}}%
\pgfpathlineto{\pgfqpoint{3.584197in}{4.130151in}}%
\pgfpathlineto{\pgfqpoint{3.576527in}{4.101863in}}%
\pgfpathlineto{\pgfqpoint{3.568854in}{4.074001in}}%
\pgfpathlineto{\pgfqpoint{3.561176in}{4.046555in}}%
\pgfpathlineto{\pgfqpoint{3.553496in}{4.019519in}}%
\pgfpathclose%
\pgfusepath{fill}%
\end{pgfscope}%
\begin{pgfscope}%
\pgfpathrectangle{\pgfqpoint{1.150000in}{0.150000in}}{\pgfqpoint{5.700000in}{5.700000in}}%
\pgfusepath{clip}%
\pgfsetbuttcap%
\pgfsetroundjoin%
\definecolor{currentfill}{rgb}{0.119512,0.607464,0.540218}%
\pgfsetfillcolor{currentfill}%
\pgfsetfillopacity{0.800000}%
\pgfsetlinewidth{0.000000pt}%
\definecolor{currentstroke}{rgb}{0.000000,0.000000,0.000000}%
\pgfsetstrokecolor{currentstroke}%
\pgfsetdash{}{0pt}%
\pgfpathmoveto{\pgfqpoint{3.522733in}{3.915322in}}%
\pgfpathlineto{\pgfqpoint{3.536107in}{3.893713in}}%
\pgfpathlineto{\pgfqpoint{3.549474in}{3.872409in}}%
\pgfpathlineto{\pgfqpoint{3.562835in}{3.851410in}}%
\pgfpathlineto{\pgfqpoint{3.576191in}{3.830710in}}%
\pgfpathlineto{\pgfqpoint{3.583889in}{3.855599in}}%
\pgfpathlineto{\pgfqpoint{3.591582in}{3.880858in}}%
\pgfpathlineto{\pgfqpoint{3.599272in}{3.906496in}}%
\pgfpathlineto{\pgfqpoint{3.606958in}{3.932520in}}%
\pgfpathlineto{\pgfqpoint{3.593601in}{3.953814in}}%
\pgfpathlineto{\pgfqpoint{3.580239in}{3.975410in}}%
\pgfpathlineto{\pgfqpoint{3.566870in}{3.997311in}}%
\pgfpathlineto{\pgfqpoint{3.553496in}{4.019519in}}%
\pgfpathlineto{\pgfqpoint{3.545811in}{3.992884in}}%
\pgfpathlineto{\pgfqpoint{3.538123in}{3.966645in}}%
\pgfpathlineto{\pgfqpoint{3.530430in}{3.940793in}}%
\pgfpathlineto{\pgfqpoint{3.522733in}{3.915322in}}%
\pgfpathclose%
\pgfusepath{fill}%
\end{pgfscope}%
\begin{pgfscope}%
\pgfpathrectangle{\pgfqpoint{1.150000in}{0.150000in}}{\pgfqpoint{5.700000in}{5.700000in}}%
\pgfusepath{clip}%
\pgfsetbuttcap%
\pgfsetroundjoin%
\definecolor{currentfill}{rgb}{0.266941,0.748751,0.440573}%
\pgfsetfillcolor{currentfill}%
\pgfsetfillopacity{0.800000}%
\pgfsetlinewidth{0.000000pt}%
\definecolor{currentstroke}{rgb}{0.000000,0.000000,0.000000}%
\pgfsetstrokecolor{currentstroke}%
\pgfsetdash{}{0pt}%
\pgfpathmoveto{\pgfqpoint{3.866950in}{4.358086in}}%
\pgfpathlineto{\pgfqpoint{3.880283in}{4.335906in}}%
\pgfpathlineto{\pgfqpoint{3.893611in}{4.314009in}}%
\pgfpathlineto{\pgfqpoint{3.906935in}{4.292394in}}%
\pgfpathlineto{\pgfqpoint{3.920256in}{4.271057in}}%
\pgfpathlineto{\pgfqpoint{3.927917in}{4.304631in}}%
\pgfpathlineto{\pgfqpoint{3.935578in}{4.338748in}}%
\pgfpathlineto{\pgfqpoint{3.943240in}{4.373419in}}%
\pgfpathlineto{\pgfqpoint{3.929915in}{4.395359in}}%
\pgfpathlineto{\pgfqpoint{3.916586in}{4.417580in}}%
\pgfpathlineto{\pgfqpoint{3.903253in}{4.440083in}}%
\pgfpathlineto{\pgfqpoint{3.889915in}{4.462870in}}%
\pgfpathlineto{\pgfqpoint{3.882260in}{4.427382in}}%
\pgfpathlineto{\pgfqpoint{3.874605in}{4.392457in}}%
\pgfpathlineto{\pgfqpoint{3.866950in}{4.358086in}}%
\pgfpathclose%
\pgfusepath{fill}%
\end{pgfscope}%
\begin{pgfscope}%
\pgfpathrectangle{\pgfqpoint{1.150000in}{0.150000in}}{\pgfqpoint{5.700000in}{5.700000in}}%
\pgfusepath{clip}%
\pgfsetbuttcap%
\pgfsetroundjoin%
\definecolor{currentfill}{rgb}{0.147607,0.511733,0.557049}%
\pgfsetfillcolor{currentfill}%
\pgfsetfillopacity{0.800000}%
\pgfsetlinewidth{0.000000pt}%
\definecolor{currentstroke}{rgb}{0.000000,0.000000,0.000000}%
\pgfsetstrokecolor{currentstroke}%
\pgfsetdash{}{0pt}%
\pgfpathmoveto{\pgfqpoint{3.514462in}{3.644163in}}%
\pgfpathlineto{\pgfqpoint{3.527814in}{3.624843in}}%
\pgfpathlineto{\pgfqpoint{3.541161in}{3.605815in}}%
\pgfpathlineto{\pgfqpoint{3.554504in}{3.587079in}}%
\pgfpathlineto{\pgfqpoint{3.567842in}{3.568631in}}%
\pgfpathlineto{\pgfqpoint{3.575572in}{3.590281in}}%
\pgfpathlineto{\pgfqpoint{3.583297in}{3.612241in}}%
\pgfpathlineto{\pgfqpoint{3.591018in}{3.634517in}}%
\pgfpathlineto{\pgfqpoint{3.598735in}{3.657116in}}%
\pgfpathlineto{\pgfqpoint{3.585399in}{3.676084in}}%
\pgfpathlineto{\pgfqpoint{3.572058in}{3.695341in}}%
\pgfpathlineto{\pgfqpoint{3.558712in}{3.714890in}}%
\pgfpathlineto{\pgfqpoint{3.545361in}{3.734733in}}%
\pgfpathlineto{\pgfqpoint{3.537643in}{3.711600in}}%
\pgfpathlineto{\pgfqpoint{3.529921in}{3.688798in}}%
\pgfpathlineto{\pgfqpoint{3.522194in}{3.666321in}}%
\pgfpathlineto{\pgfqpoint{3.514462in}{3.644163in}}%
\pgfpathclose%
\pgfusepath{fill}%
\end{pgfscope}%
\begin{pgfscope}%
\pgfpathrectangle{\pgfqpoint{1.150000in}{0.150000in}}{\pgfqpoint{5.700000in}{5.700000in}}%
\pgfusepath{clip}%
\pgfsetbuttcap%
\pgfsetroundjoin%
\definecolor{currentfill}{rgb}{0.210503,0.363727,0.552206}%
\pgfsetfillcolor{currentfill}%
\pgfsetfillopacity{0.800000}%
\pgfsetlinewidth{0.000000pt}%
\definecolor{currentstroke}{rgb}{0.000000,0.000000,0.000000}%
\pgfsetstrokecolor{currentstroke}%
\pgfsetdash{}{0pt}%
\pgfpathmoveto{\pgfqpoint{3.887215in}{3.206700in}}%
\pgfpathlineto{\pgfqpoint{3.900514in}{3.194775in}}%
\pgfpathlineto{\pgfqpoint{3.913813in}{3.183088in}}%
\pgfpathlineto{\pgfqpoint{3.927114in}{3.171637in}}%
\pgfpathlineto{\pgfqpoint{3.940416in}{3.160420in}}%
\pgfpathlineto{\pgfqpoint{3.948123in}{3.178581in}}%
\pgfpathlineto{\pgfqpoint{3.955827in}{3.196995in}}%
\pgfpathlineto{\pgfqpoint{3.963528in}{3.215666in}}%
\pgfpathlineto{\pgfqpoint{3.971227in}{3.234601in}}%
\pgfpathlineto{\pgfqpoint{3.957929in}{3.246322in}}%
\pgfpathlineto{\pgfqpoint{3.944633in}{3.258278in}}%
\pgfpathlineto{\pgfqpoint{3.931337in}{3.270471in}}%
\pgfpathlineto{\pgfqpoint{3.918043in}{3.282901in}}%
\pgfpathlineto{\pgfqpoint{3.910341in}{3.263449in}}%
\pgfpathlineto{\pgfqpoint{3.902635in}{3.244269in}}%
\pgfpathlineto{\pgfqpoint{3.894927in}{3.225354in}}%
\pgfpathlineto{\pgfqpoint{3.887215in}{3.206700in}}%
\pgfpathclose%
\pgfusepath{fill}%
\end{pgfscope}%
\begin{pgfscope}%
\pgfpathrectangle{\pgfqpoint{1.150000in}{0.150000in}}{\pgfqpoint{5.700000in}{5.700000in}}%
\pgfusepath{clip}%
\pgfsetbuttcap%
\pgfsetroundjoin%
\definecolor{currentfill}{rgb}{0.187231,0.414746,0.556547}%
\pgfsetfillcolor{currentfill}%
\pgfsetfillopacity{0.800000}%
\pgfsetlinewidth{0.000000pt}%
\definecolor{currentstroke}{rgb}{0.000000,0.000000,0.000000}%
\pgfsetstrokecolor{currentstroke}%
\pgfsetdash{}{0pt}%
\pgfpathmoveto{\pgfqpoint{3.643475in}{3.351397in}}%
\pgfpathlineto{\pgfqpoint{3.656787in}{3.335921in}}%
\pgfpathlineto{\pgfqpoint{3.670097in}{3.320710in}}%
\pgfpathlineto{\pgfqpoint{3.683406in}{3.305763in}}%
\pgfpathlineto{\pgfqpoint{3.696712in}{3.291077in}}%
\pgfpathlineto{\pgfqpoint{3.704450in}{3.310139in}}%
\pgfpathlineto{\pgfqpoint{3.712184in}{3.329462in}}%
\pgfpathlineto{\pgfqpoint{3.719914in}{3.349052in}}%
\pgfpathlineto{\pgfqpoint{3.727640in}{3.368916in}}%
\pgfpathlineto{\pgfqpoint{3.714337in}{3.384080in}}%
\pgfpathlineto{\pgfqpoint{3.701032in}{3.399506in}}%
\pgfpathlineto{\pgfqpoint{3.687726in}{3.415195in}}%
\pgfpathlineto{\pgfqpoint{3.674417in}{3.431151in}}%
\pgfpathlineto{\pgfqpoint{3.666688in}{3.410796in}}%
\pgfpathlineto{\pgfqpoint{3.658954in}{3.390723in}}%
\pgfpathlineto{\pgfqpoint{3.651217in}{3.370925in}}%
\pgfpathlineto{\pgfqpoint{3.643475in}{3.351397in}}%
\pgfpathclose%
\pgfusepath{fill}%
\end{pgfscope}%
\begin{pgfscope}%
\pgfpathrectangle{\pgfqpoint{1.150000in}{0.150000in}}{\pgfqpoint{5.700000in}{5.700000in}}%
\pgfusepath{clip}%
\pgfsetbuttcap%
\pgfsetroundjoin%
\definecolor{currentfill}{rgb}{0.252899,0.742211,0.448284}%
\pgfsetfillcolor{currentfill}%
\pgfsetfillopacity{0.800000}%
\pgfsetlinewidth{0.000000pt}%
\definecolor{currentstroke}{rgb}{0.000000,0.000000,0.000000}%
\pgfsetstrokecolor{currentstroke}%
\pgfsetdash{}{0pt}%
\pgfpathmoveto{\pgfqpoint{3.782977in}{4.314427in}}%
\pgfpathlineto{\pgfqpoint{3.796323in}{4.291870in}}%
\pgfpathlineto{\pgfqpoint{3.809663in}{4.269605in}}%
\pgfpathlineto{\pgfqpoint{3.822999in}{4.247629in}}%
\pgfpathlineto{\pgfqpoint{3.836330in}{4.225939in}}%
\pgfpathlineto{\pgfqpoint{3.843986in}{4.258194in}}%
\pgfpathlineto{\pgfqpoint{3.851641in}{4.290964in}}%
\pgfpathlineto{\pgfqpoint{3.859296in}{4.324258in}}%
\pgfpathlineto{\pgfqpoint{3.866950in}{4.358086in}}%
\pgfpathlineto{\pgfqpoint{3.853613in}{4.380551in}}%
\pgfpathlineto{\pgfqpoint{3.840272in}{4.403305in}}%
\pgfpathlineto{\pgfqpoint{3.826925in}{4.426349in}}%
\pgfpathlineto{\pgfqpoint{3.813573in}{4.449687in}}%
\pgfpathlineto{\pgfqpoint{3.805925in}{4.415065in}}%
\pgfpathlineto{\pgfqpoint{3.798277in}{4.380988in}}%
\pgfpathlineto{\pgfqpoint{3.790627in}{4.347445in}}%
\pgfpathlineto{\pgfqpoint{3.782977in}{4.314427in}}%
\pgfpathclose%
\pgfusepath{fill}%
\end{pgfscope}%
\begin{pgfscope}%
\pgfpathrectangle{\pgfqpoint{1.150000in}{0.150000in}}{\pgfqpoint{5.700000in}{5.700000in}}%
\pgfusepath{clip}%
\pgfsetbuttcap%
\pgfsetroundjoin%
\definecolor{currentfill}{rgb}{0.179019,0.433756,0.557430}%
\pgfsetfillcolor{currentfill}%
\pgfsetfillopacity{0.800000}%
\pgfsetlinewidth{0.000000pt}%
\definecolor{currentstroke}{rgb}{0.000000,0.000000,0.000000}%
\pgfsetstrokecolor{currentstroke}%
\pgfsetdash{}{0pt}%
\pgfpathmoveto{\pgfqpoint{3.590202in}{3.415997in}}%
\pgfpathlineto{\pgfqpoint{3.603525in}{3.399438in}}%
\pgfpathlineto{\pgfqpoint{3.616844in}{3.383154in}}%
\pgfpathlineto{\pgfqpoint{3.630161in}{3.367140in}}%
\pgfpathlineto{\pgfqpoint{3.643475in}{3.351397in}}%
\pgfpathlineto{\pgfqpoint{3.651217in}{3.370925in}}%
\pgfpathlineto{\pgfqpoint{3.658954in}{3.390723in}}%
\pgfpathlineto{\pgfqpoint{3.666688in}{3.410796in}}%
\pgfpathlineto{\pgfqpoint{3.674417in}{3.431151in}}%
\pgfpathlineto{\pgfqpoint{3.661105in}{3.447375in}}%
\pgfpathlineto{\pgfqpoint{3.647792in}{3.463869in}}%
\pgfpathlineto{\pgfqpoint{3.634475in}{3.480636in}}%
\pgfpathlineto{\pgfqpoint{3.621155in}{3.497677in}}%
\pgfpathlineto{\pgfqpoint{3.613423in}{3.476828in}}%
\pgfpathlineto{\pgfqpoint{3.605687in}{3.456269in}}%
\pgfpathlineto{\pgfqpoint{3.597947in}{3.435994in}}%
\pgfpathlineto{\pgfqpoint{3.590202in}{3.415997in}}%
\pgfpathclose%
\pgfusepath{fill}%
\end{pgfscope}%
\begin{pgfscope}%
\pgfpathrectangle{\pgfqpoint{1.150000in}{0.150000in}}{\pgfqpoint{5.700000in}{5.700000in}}%
\pgfusepath{clip}%
\pgfsetbuttcap%
\pgfsetroundjoin%
\definecolor{currentfill}{rgb}{0.197636,0.391528,0.554969}%
\pgfsetfillcolor{currentfill}%
\pgfsetfillopacity{0.800000}%
\pgfsetlinewidth{0.000000pt}%
\definecolor{currentstroke}{rgb}{0.000000,0.000000,0.000000}%
\pgfsetstrokecolor{currentstroke}%
\pgfsetdash{}{0pt}%
\pgfpathmoveto{\pgfqpoint{3.696712in}{3.291077in}}%
\pgfpathlineto{\pgfqpoint{3.710017in}{3.276651in}}%
\pgfpathlineto{\pgfqpoint{3.723321in}{3.262482in}}%
\pgfpathlineto{\pgfqpoint{3.736624in}{3.248569in}}%
\pgfpathlineto{\pgfqpoint{3.749926in}{3.234910in}}%
\pgfpathlineto{\pgfqpoint{3.757660in}{3.253507in}}%
\pgfpathlineto{\pgfqpoint{3.765390in}{3.272357in}}%
\pgfpathlineto{\pgfqpoint{3.773117in}{3.291467in}}%
\pgfpathlineto{\pgfqpoint{3.780839in}{3.310840in}}%
\pgfpathlineto{\pgfqpoint{3.767541in}{3.324976in}}%
\pgfpathlineto{\pgfqpoint{3.754242in}{3.339366in}}%
\pgfpathlineto{\pgfqpoint{3.740942in}{3.354012in}}%
\pgfpathlineto{\pgfqpoint{3.727640in}{3.368916in}}%
\pgfpathlineto{\pgfqpoint{3.719914in}{3.349052in}}%
\pgfpathlineto{\pgfqpoint{3.712184in}{3.329462in}}%
\pgfpathlineto{\pgfqpoint{3.704450in}{3.310139in}}%
\pgfpathlineto{\pgfqpoint{3.696712in}{3.291077in}}%
\pgfpathclose%
\pgfusepath{fill}%
\end{pgfscope}%
\begin{pgfscope}%
\pgfpathrectangle{\pgfqpoint{1.150000in}{0.150000in}}{\pgfqpoint{5.700000in}{5.700000in}}%
\pgfusepath{clip}%
\pgfsetbuttcap%
\pgfsetroundjoin%
\definecolor{currentfill}{rgb}{0.214298,0.355619,0.551184}%
\pgfsetfillcolor{currentfill}%
\pgfsetfillopacity{0.800000}%
\pgfsetlinewidth{0.000000pt}%
\definecolor{currentstroke}{rgb}{0.000000,0.000000,0.000000}%
\pgfsetstrokecolor{currentstroke}%
\pgfsetdash{}{0pt}%
\pgfpathmoveto{\pgfqpoint{4.024433in}{3.190035in}}%
\pgfpathlineto{\pgfqpoint{4.037739in}{3.179467in}}%
\pgfpathlineto{\pgfqpoint{4.051048in}{3.169124in}}%
\pgfpathlineto{\pgfqpoint{4.064359in}{3.159007in}}%
\pgfpathlineto{\pgfqpoint{4.077673in}{3.149114in}}%
\pgfpathlineto{\pgfqpoint{4.085359in}{3.167272in}}%
\pgfpathlineto{\pgfqpoint{4.093043in}{3.185690in}}%
\pgfpathlineto{\pgfqpoint{4.100724in}{3.204375in}}%
\pgfpathlineto{\pgfqpoint{4.108403in}{3.223331in}}%
\pgfpathlineto{\pgfqpoint{4.095095in}{3.233759in}}%
\pgfpathlineto{\pgfqpoint{4.081789in}{3.244411in}}%
\pgfpathlineto{\pgfqpoint{4.068485in}{3.255288in}}%
\pgfpathlineto{\pgfqpoint{4.055183in}{3.266392in}}%
\pgfpathlineto{\pgfqpoint{4.047499in}{3.246888in}}%
\pgfpathlineto{\pgfqpoint{4.039813in}{3.227665in}}%
\pgfpathlineto{\pgfqpoint{4.032124in}{3.208716in}}%
\pgfpathlineto{\pgfqpoint{4.024433in}{3.190035in}}%
\pgfpathclose%
\pgfusepath{fill}%
\end{pgfscope}%
\begin{pgfscope}%
\pgfpathrectangle{\pgfqpoint{1.150000in}{0.150000in}}{\pgfqpoint{5.700000in}{5.700000in}}%
\pgfusepath{clip}%
\pgfsetbuttcap%
\pgfsetroundjoin%
\definecolor{currentfill}{rgb}{0.153894,0.680203,0.504172}%
\pgfsetfillcolor{currentfill}%
\pgfsetfillopacity{0.800000}%
\pgfsetlinewidth{0.000000pt}%
\definecolor{currentstroke}{rgb}{0.000000,0.000000,0.000000}%
\pgfsetstrokecolor{currentstroke}%
\pgfsetdash{}{0pt}%
\pgfpathmoveto{\pgfqpoint{3.584197in}{4.130151in}}%
\pgfpathlineto{\pgfqpoint{3.597574in}{4.107307in}}%
\pgfpathlineto{\pgfqpoint{3.610945in}{4.084772in}}%
\pgfpathlineto{\pgfqpoint{3.624310in}{4.062542in}}%
\pgfpathlineto{\pgfqpoint{3.637669in}{4.040616in}}%
\pgfpathlineto{\pgfqpoint{3.645339in}{4.068678in}}%
\pgfpathlineto{\pgfqpoint{3.653006in}{4.097172in}}%
\pgfpathlineto{\pgfqpoint{3.660670in}{4.126104in}}%
\pgfpathlineto{\pgfqpoint{3.668331in}{4.155483in}}%
\pgfpathlineto{\pgfqpoint{3.654969in}{4.178079in}}%
\pgfpathlineto{\pgfqpoint{3.641600in}{4.200980in}}%
\pgfpathlineto{\pgfqpoint{3.628226in}{4.224188in}}%
\pgfpathlineto{\pgfqpoint{3.614844in}{4.247706in}}%
\pgfpathlineto{\pgfqpoint{3.607187in}{4.217640in}}%
\pgfpathlineto{\pgfqpoint{3.599527in}{4.188031in}}%
\pgfpathlineto{\pgfqpoint{3.591864in}{4.158871in}}%
\pgfpathlineto{\pgfqpoint{3.584197in}{4.130151in}}%
\pgfpathclose%
\pgfusepath{fill}%
\end{pgfscope}%
\begin{pgfscope}%
\pgfpathrectangle{\pgfqpoint{1.150000in}{0.150000in}}{\pgfqpoint{5.700000in}{5.700000in}}%
\pgfusepath{clip}%
\pgfsetbuttcap%
\pgfsetroundjoin%
\definecolor{currentfill}{rgb}{0.126453,0.570633,0.549841}%
\pgfsetfillcolor{currentfill}%
\pgfsetfillopacity{0.800000}%
\pgfsetlinewidth{0.000000pt}%
\definecolor{currentstroke}{rgb}{0.000000,0.000000,0.000000}%
\pgfsetstrokecolor{currentstroke}%
\pgfsetdash{}{0pt}%
\pgfpathmoveto{\pgfqpoint{3.491904in}{3.817103in}}%
\pgfpathlineto{\pgfqpoint{3.505277in}{3.796056in}}%
\pgfpathlineto{\pgfqpoint{3.518644in}{3.775314in}}%
\pgfpathlineto{\pgfqpoint{3.532005in}{3.754874in}}%
\pgfpathlineto{\pgfqpoint{3.545361in}{3.734733in}}%
\pgfpathlineto{\pgfqpoint{3.553075in}{3.758205in}}%
\pgfpathlineto{\pgfqpoint{3.560784in}{3.782020in}}%
\pgfpathlineto{\pgfqpoint{3.568490in}{3.806186in}}%
\pgfpathlineto{\pgfqpoint{3.576191in}{3.830710in}}%
\pgfpathlineto{\pgfqpoint{3.562835in}{3.851410in}}%
\pgfpathlineto{\pgfqpoint{3.549474in}{3.872409in}}%
\pgfpathlineto{\pgfqpoint{3.536107in}{3.893713in}}%
\pgfpathlineto{\pgfqpoint{3.522733in}{3.915322in}}%
\pgfpathlineto{\pgfqpoint{3.515033in}{3.890224in}}%
\pgfpathlineto{\pgfqpoint{3.507327in}{3.865493in}}%
\pgfpathlineto{\pgfqpoint{3.499618in}{3.841121in}}%
\pgfpathlineto{\pgfqpoint{3.491904in}{3.817103in}}%
\pgfpathclose%
\pgfusepath{fill}%
\end{pgfscope}%
\begin{pgfscope}%
\pgfpathrectangle{\pgfqpoint{1.150000in}{0.150000in}}{\pgfqpoint{5.700000in}{5.700000in}}%
\pgfusepath{clip}%
\pgfsetbuttcap%
\pgfsetroundjoin%
\definecolor{currentfill}{rgb}{0.226397,0.728888,0.462789}%
\pgfsetfillcolor{currentfill}%
\pgfsetfillopacity{0.800000}%
\pgfsetlinewidth{0.000000pt}%
\definecolor{currentstroke}{rgb}{0.000000,0.000000,0.000000}%
\pgfsetstrokecolor{currentstroke}%
\pgfsetdash{}{0pt}%
\pgfpathmoveto{\pgfqpoint{3.698953in}{4.277635in}}%
\pgfpathlineto{\pgfqpoint{3.712314in}{4.254633in}}%
\pgfpathlineto{\pgfqpoint{3.725670in}{4.231932in}}%
\pgfpathlineto{\pgfqpoint{3.739020in}{4.209528in}}%
\pgfpathlineto{\pgfqpoint{3.752365in}{4.187419in}}%
\pgfpathlineto{\pgfqpoint{3.760021in}{4.218430in}}%
\pgfpathlineto{\pgfqpoint{3.767674in}{4.249928in}}%
\pgfpathlineto{\pgfqpoint{3.775326in}{4.281924in}}%
\pgfpathlineto{\pgfqpoint{3.782977in}{4.314427in}}%
\pgfpathlineto{\pgfqpoint{3.769627in}{4.337278in}}%
\pgfpathlineto{\pgfqpoint{3.756271in}{4.360425in}}%
\pgfpathlineto{\pgfqpoint{3.742910in}{4.383872in}}%
\pgfpathlineto{\pgfqpoint{3.729542in}{4.407621in}}%
\pgfpathlineto{\pgfqpoint{3.721897in}{4.374359in}}%
\pgfpathlineto{\pgfqpoint{3.714251in}{4.341613in}}%
\pgfpathlineto{\pgfqpoint{3.706603in}{4.309375in}}%
\pgfpathlineto{\pgfqpoint{3.698953in}{4.277635in}}%
\pgfpathclose%
\pgfusepath{fill}%
\end{pgfscope}%
\begin{pgfscope}%
\pgfpathrectangle{\pgfqpoint{1.150000in}{0.150000in}}{\pgfqpoint{5.700000in}{5.700000in}}%
\pgfusepath{clip}%
\pgfsetbuttcap%
\pgfsetroundjoin%
\definecolor{currentfill}{rgb}{0.168126,0.459988,0.558082}%
\pgfsetfillcolor{currentfill}%
\pgfsetfillopacity{0.800000}%
\pgfsetlinewidth{0.000000pt}%
\definecolor{currentstroke}{rgb}{0.000000,0.000000,0.000000}%
\pgfsetstrokecolor{currentstroke}%
\pgfsetdash{}{0pt}%
\pgfpathmoveto{\pgfqpoint{3.536878in}{3.485016in}}%
\pgfpathlineto{\pgfqpoint{3.550215in}{3.467339in}}%
\pgfpathlineto{\pgfqpoint{3.563547in}{3.449945in}}%
\pgfpathlineto{\pgfqpoint{3.576876in}{3.432832in}}%
\pgfpathlineto{\pgfqpoint{3.590202in}{3.415997in}}%
\pgfpathlineto{\pgfqpoint{3.597947in}{3.435994in}}%
\pgfpathlineto{\pgfqpoint{3.605687in}{3.456269in}}%
\pgfpathlineto{\pgfqpoint{3.613423in}{3.476828in}}%
\pgfpathlineto{\pgfqpoint{3.621155in}{3.497677in}}%
\pgfpathlineto{\pgfqpoint{3.607832in}{3.514995in}}%
\pgfpathlineto{\pgfqpoint{3.594506in}{3.532591in}}%
\pgfpathlineto{\pgfqpoint{3.581176in}{3.550469in}}%
\pgfpathlineto{\pgfqpoint{3.567842in}{3.568631in}}%
\pgfpathlineto{\pgfqpoint{3.560108in}{3.547286in}}%
\pgfpathlineto{\pgfqpoint{3.552369in}{3.526239in}}%
\pgfpathlineto{\pgfqpoint{3.544626in}{3.505484in}}%
\pgfpathlineto{\pgfqpoint{3.536878in}{3.485016in}}%
\pgfpathclose%
\pgfusepath{fill}%
\end{pgfscope}%
\begin{pgfscope}%
\pgfpathrectangle{\pgfqpoint{1.150000in}{0.150000in}}{\pgfqpoint{5.700000in}{5.700000in}}%
\pgfusepath{clip}%
\pgfsetbuttcap%
\pgfsetroundjoin%
\definecolor{currentfill}{rgb}{0.199430,0.387607,0.554642}%
\pgfsetfillcolor{currentfill}%
\pgfsetfillopacity{0.800000}%
\pgfsetlinewidth{0.000000pt}%
\definecolor{currentstroke}{rgb}{0.000000,0.000000,0.000000}%
\pgfsetstrokecolor{currentstroke}%
\pgfsetdash{}{0pt}%
\pgfpathmoveto{\pgfqpoint{4.329605in}{3.263545in}}%
\pgfpathlineto{\pgfqpoint{4.342947in}{3.254763in}}%
\pgfpathlineto{\pgfqpoint{4.356294in}{3.246189in}}%
\pgfpathlineto{\pgfqpoint{4.369646in}{3.237823in}}%
\pgfpathlineto{\pgfqpoint{4.383002in}{3.229664in}}%
\pgfpathlineto{\pgfqpoint{4.390647in}{3.248906in}}%
\pgfpathlineto{\pgfqpoint{4.398292in}{3.268458in}}%
\pgfpathlineto{\pgfqpoint{4.405937in}{3.288325in}}%
\pgfpathlineto{\pgfqpoint{4.413582in}{3.308517in}}%
\pgfpathlineto{\pgfqpoint{4.400232in}{3.317303in}}%
\pgfpathlineto{\pgfqpoint{4.386887in}{3.326297in}}%
\pgfpathlineto{\pgfqpoint{4.373547in}{3.335499in}}%
\pgfpathlineto{\pgfqpoint{4.360210in}{3.344909in}}%
\pgfpathlineto{\pgfqpoint{4.352559in}{3.324078in}}%
\pgfpathlineto{\pgfqpoint{4.344908in}{3.303578in}}%
\pgfpathlineto{\pgfqpoint{4.337256in}{3.283403in}}%
\pgfpathlineto{\pgfqpoint{4.329605in}{3.263545in}}%
\pgfpathclose%
\pgfusepath{fill}%
\end{pgfscope}%
\begin{pgfscope}%
\pgfpathrectangle{\pgfqpoint{1.150000in}{0.150000in}}{\pgfqpoint{5.700000in}{5.700000in}}%
\pgfusepath{clip}%
\pgfsetbuttcap%
\pgfsetroundjoin%
\definecolor{currentfill}{rgb}{0.206756,0.371758,0.553117}%
\pgfsetfillcolor{currentfill}%
\pgfsetfillopacity{0.800000}%
\pgfsetlinewidth{0.000000pt}%
\definecolor{currentstroke}{rgb}{0.000000,0.000000,0.000000}%
\pgfsetstrokecolor{currentstroke}%
\pgfsetdash{}{0pt}%
\pgfpathmoveto{\pgfqpoint{3.749926in}{3.234910in}}%
\pgfpathlineto{\pgfqpoint{3.763227in}{3.221503in}}%
\pgfpathlineto{\pgfqpoint{3.776528in}{3.208347in}}%
\pgfpathlineto{\pgfqpoint{3.789829in}{3.195438in}}%
\pgfpathlineto{\pgfqpoint{3.803129in}{3.182777in}}%
\pgfpathlineto{\pgfqpoint{3.810859in}{3.200911in}}%
\pgfpathlineto{\pgfqpoint{3.818585in}{3.219290in}}%
\pgfpathlineto{\pgfqpoint{3.826308in}{3.237919in}}%
\pgfpathlineto{\pgfqpoint{3.834027in}{3.256806in}}%
\pgfpathlineto{\pgfqpoint{3.820730in}{3.269942in}}%
\pgfpathlineto{\pgfqpoint{3.807434in}{3.283325in}}%
\pgfpathlineto{\pgfqpoint{3.794137in}{3.296957in}}%
\pgfpathlineto{\pgfqpoint{3.780839in}{3.310840in}}%
\pgfpathlineto{\pgfqpoint{3.773117in}{3.291467in}}%
\pgfpathlineto{\pgfqpoint{3.765390in}{3.272357in}}%
\pgfpathlineto{\pgfqpoint{3.757660in}{3.253507in}}%
\pgfpathlineto{\pgfqpoint{3.749926in}{3.234910in}}%
\pgfpathclose%
\pgfusepath{fill}%
\end{pgfscope}%
\begin{pgfscope}%
\pgfpathrectangle{\pgfqpoint{1.150000in}{0.150000in}}{\pgfqpoint{5.700000in}{5.700000in}}%
\pgfusepath{clip}%
\pgfsetbuttcap%
\pgfsetroundjoin%
\definecolor{currentfill}{rgb}{0.192357,0.403199,0.555836}%
\pgfsetfillcolor{currentfill}%
\pgfsetfillopacity{0.800000}%
\pgfsetlinewidth{0.000000pt}%
\definecolor{currentstroke}{rgb}{0.000000,0.000000,0.000000}%
\pgfsetstrokecolor{currentstroke}%
\pgfsetdash{}{0pt}%
\pgfpathmoveto{\pgfqpoint{4.413582in}{3.308517in}}%
\pgfpathlineto{\pgfqpoint{4.426936in}{3.299935in}}%
\pgfpathlineto{\pgfqpoint{4.440295in}{3.291559in}}%
\pgfpathlineto{\pgfqpoint{4.453659in}{3.283387in}}%
\pgfpathlineto{\pgfqpoint{4.467028in}{3.275418in}}%
\pgfpathlineto{\pgfqpoint{4.474667in}{3.295293in}}%
\pgfpathlineto{\pgfqpoint{4.482306in}{3.315499in}}%
\pgfpathlineto{\pgfqpoint{4.489945in}{3.336044in}}%
\pgfpathlineto{\pgfqpoint{4.497586in}{3.356935in}}%
\pgfpathlineto{\pgfqpoint{4.484224in}{3.365563in}}%
\pgfpathlineto{\pgfqpoint{4.470867in}{3.374395in}}%
\pgfpathlineto{\pgfqpoint{4.457515in}{3.383430in}}%
\pgfpathlineto{\pgfqpoint{4.444167in}{3.392671in}}%
\pgfpathlineto{\pgfqpoint{4.436520in}{3.371109in}}%
\pgfpathlineto{\pgfqpoint{4.428873in}{3.349900in}}%
\pgfpathlineto{\pgfqpoint{4.421227in}{3.329039in}}%
\pgfpathlineto{\pgfqpoint{4.413582in}{3.308517in}}%
\pgfpathclose%
\pgfusepath{fill}%
\end{pgfscope}%
\begin{pgfscope}%
\pgfpathrectangle{\pgfqpoint{1.150000in}{0.150000in}}{\pgfqpoint{5.700000in}{5.700000in}}%
\pgfusepath{clip}%
\pgfsetbuttcap%
\pgfsetroundjoin%
\definecolor{currentfill}{rgb}{0.206756,0.371758,0.553117}%
\pgfsetfillcolor{currentfill}%
\pgfsetfillopacity{0.800000}%
\pgfsetlinewidth{0.000000pt}%
\definecolor{currentstroke}{rgb}{0.000000,0.000000,0.000000}%
\pgfsetstrokecolor{currentstroke}%
\pgfsetdash{}{0pt}%
\pgfpathmoveto{\pgfqpoint{4.245638in}{3.221987in}}%
\pgfpathlineto{\pgfqpoint{4.258971in}{3.212957in}}%
\pgfpathlineto{\pgfqpoint{4.272307in}{3.204139in}}%
\pgfpathlineto{\pgfqpoint{4.285647in}{3.195533in}}%
\pgfpathlineto{\pgfqpoint{4.298992in}{3.187137in}}%
\pgfpathlineto{\pgfqpoint{4.306646in}{3.205800in}}%
\pgfpathlineto{\pgfqpoint{4.314300in}{3.224750in}}%
\pgfpathlineto{\pgfqpoint{4.321953in}{3.243996in}}%
\pgfpathlineto{\pgfqpoint{4.329605in}{3.263545in}}%
\pgfpathlineto{\pgfqpoint{4.316266in}{3.272537in}}%
\pgfpathlineto{\pgfqpoint{4.302932in}{3.281739in}}%
\pgfpathlineto{\pgfqpoint{4.289602in}{3.291153in}}%
\pgfpathlineto{\pgfqpoint{4.276275in}{3.300780in}}%
\pgfpathlineto{\pgfqpoint{4.268617in}{3.280623in}}%
\pgfpathlineto{\pgfqpoint{4.260959in}{3.260777in}}%
\pgfpathlineto{\pgfqpoint{4.253299in}{3.241234in}}%
\pgfpathlineto{\pgfqpoint{4.245638in}{3.221987in}}%
\pgfpathclose%
\pgfusepath{fill}%
\end{pgfscope}%
\begin{pgfscope}%
\pgfpathrectangle{\pgfqpoint{1.150000in}{0.150000in}}{\pgfqpoint{5.700000in}{5.700000in}}%
\pgfusepath{clip}%
\pgfsetbuttcap%
\pgfsetroundjoin%
\definecolor{currentfill}{rgb}{0.165117,0.467423,0.558141}%
\pgfsetfillcolor{currentfill}%
\pgfsetfillopacity{0.800000}%
\pgfsetlinewidth{0.000000pt}%
\definecolor{currentstroke}{rgb}{0.000000,0.000000,0.000000}%
\pgfsetstrokecolor{currentstroke}%
\pgfsetdash{}{0pt}%
\pgfpathmoveto{\pgfqpoint{4.612218in}{3.499350in}}%
\pgfpathlineto{\pgfqpoint{4.625592in}{3.490310in}}%
\pgfpathlineto{\pgfqpoint{4.638970in}{3.481469in}}%
\pgfpathlineto{\pgfqpoint{4.652353in}{3.472826in}}%
\pgfpathlineto{\pgfqpoint{4.665743in}{3.464379in}}%
\pgfpathlineto{\pgfqpoint{4.673389in}{3.487270in}}%
\pgfpathlineto{\pgfqpoint{4.681038in}{3.510575in}}%
\pgfpathlineto{\pgfqpoint{4.688692in}{3.534303in}}%
\pgfpathlineto{\pgfqpoint{4.675308in}{3.543314in}}%
\pgfpathlineto{\pgfqpoint{4.661930in}{3.552522in}}%
\pgfpathlineto{\pgfqpoint{4.648558in}{3.561928in}}%
\pgfpathlineto{\pgfqpoint{4.635190in}{3.571533in}}%
\pgfpathlineto{\pgfqpoint{4.627529in}{3.547043in}}%
\pgfpathlineto{\pgfqpoint{4.619872in}{3.522985in}}%
\pgfpathlineto{\pgfqpoint{4.612218in}{3.499350in}}%
\pgfpathclose%
\pgfusepath{fill}%
\end{pgfscope}%
\begin{pgfscope}%
\pgfpathrectangle{\pgfqpoint{1.150000in}{0.150000in}}{\pgfqpoint{5.700000in}{5.700000in}}%
\pgfusepath{clip}%
\pgfsetbuttcap%
\pgfsetroundjoin%
\definecolor{currentfill}{rgb}{0.214298,0.355619,0.551184}%
\pgfsetfillcolor{currentfill}%
\pgfsetfillopacity{0.800000}%
\pgfsetlinewidth{0.000000pt}%
\definecolor{currentstroke}{rgb}{0.000000,0.000000,0.000000}%
\pgfsetstrokecolor{currentstroke}%
\pgfsetdash{}{0pt}%
\pgfpathmoveto{\pgfqpoint{4.161666in}{3.183836in}}%
\pgfpathlineto{\pgfqpoint{4.174990in}{3.174510in}}%
\pgfpathlineto{\pgfqpoint{4.188317in}{3.165401in}}%
\pgfpathlineto{\pgfqpoint{4.201647in}{3.156507in}}%
\pgfpathlineto{\pgfqpoint{4.214982in}{3.147828in}}%
\pgfpathlineto{\pgfqpoint{4.222648in}{3.165957in}}%
\pgfpathlineto{\pgfqpoint{4.230313in}{3.184356in}}%
\pgfpathlineto{\pgfqpoint{4.237976in}{3.203030in}}%
\pgfpathlineto{\pgfqpoint{4.245638in}{3.221987in}}%
\pgfpathlineto{\pgfqpoint{4.232310in}{3.231231in}}%
\pgfpathlineto{\pgfqpoint{4.218985in}{3.240689in}}%
\pgfpathlineto{\pgfqpoint{4.205663in}{3.250364in}}%
\pgfpathlineto{\pgfqpoint{4.192345in}{3.260255in}}%
\pgfpathlineto{\pgfqpoint{4.184678in}{3.240721in}}%
\pgfpathlineto{\pgfqpoint{4.177009in}{3.221477in}}%
\pgfpathlineto{\pgfqpoint{4.169338in}{3.202518in}}%
\pgfpathlineto{\pgfqpoint{4.161666in}{3.183836in}}%
\pgfpathclose%
\pgfusepath{fill}%
\end{pgfscope}%
\begin{pgfscope}%
\pgfpathrectangle{\pgfqpoint{1.150000in}{0.150000in}}{\pgfqpoint{5.700000in}{5.700000in}}%
\pgfusepath{clip}%
\pgfsetbuttcap%
\pgfsetroundjoin%
\definecolor{currentfill}{rgb}{0.183898,0.422383,0.556944}%
\pgfsetfillcolor{currentfill}%
\pgfsetfillopacity{0.800000}%
\pgfsetlinewidth{0.000000pt}%
\definecolor{currentstroke}{rgb}{0.000000,0.000000,0.000000}%
\pgfsetstrokecolor{currentstroke}%
\pgfsetdash{}{0pt}%
\pgfpathmoveto{\pgfqpoint{4.497586in}{3.356935in}}%
\pgfpathlineto{\pgfqpoint{4.510953in}{3.348509in}}%
\pgfpathlineto{\pgfqpoint{4.524325in}{3.340284in}}%
\pgfpathlineto{\pgfqpoint{4.537703in}{3.332260in}}%
\pgfpathlineto{\pgfqpoint{4.551085in}{3.324436in}}%
\pgfpathlineto{\pgfqpoint{4.558720in}{3.345002in}}%
\pgfpathlineto{\pgfqpoint{4.566356in}{3.365923in}}%
\pgfpathlineto{\pgfqpoint{4.573994in}{3.387206in}}%
\pgfpathlineto{\pgfqpoint{4.581634in}{3.408860in}}%
\pgfpathlineto{\pgfqpoint{4.568259in}{3.417375in}}%
\pgfpathlineto{\pgfqpoint{4.554889in}{3.426090in}}%
\pgfpathlineto{\pgfqpoint{4.541524in}{3.435006in}}%
\pgfpathlineto{\pgfqpoint{4.528164in}{3.444124in}}%
\pgfpathlineto{\pgfqpoint{4.520517in}{3.421766in}}%
\pgfpathlineto{\pgfqpoint{4.512872in}{3.399788in}}%
\pgfpathlineto{\pgfqpoint{4.505228in}{3.378180in}}%
\pgfpathlineto{\pgfqpoint{4.497586in}{3.356935in}}%
\pgfpathclose%
\pgfusepath{fill}%
\end{pgfscope}%
\begin{pgfscope}%
\pgfpathrectangle{\pgfqpoint{1.150000in}{0.150000in}}{\pgfqpoint{5.700000in}{5.700000in}}%
\pgfusepath{clip}%
\pgfsetbuttcap%
\pgfsetroundjoin%
\definecolor{currentfill}{rgb}{0.218130,0.347432,0.550038}%
\pgfsetfillcolor{currentfill}%
\pgfsetfillopacity{0.800000}%
\pgfsetlinewidth{0.000000pt}%
\definecolor{currentstroke}{rgb}{0.000000,0.000000,0.000000}%
\pgfsetstrokecolor{currentstroke}%
\pgfsetdash{}{0pt}%
\pgfpathmoveto{\pgfqpoint{3.940416in}{3.160420in}}%
\pgfpathlineto{\pgfqpoint{3.953720in}{3.149436in}}%
\pgfpathlineto{\pgfqpoint{3.967025in}{3.138684in}}%
\pgfpathlineto{\pgfqpoint{3.980332in}{3.128162in}}%
\pgfpathlineto{\pgfqpoint{3.993641in}{3.117869in}}%
\pgfpathlineto{\pgfqpoint{4.001344in}{3.135539in}}%
\pgfpathlineto{\pgfqpoint{4.009043in}{3.153453in}}%
\pgfpathlineto{\pgfqpoint{4.016739in}{3.171616in}}%
\pgfpathlineto{\pgfqpoint{4.024433in}{3.190035in}}%
\pgfpathlineto{\pgfqpoint{4.011129in}{3.200832in}}%
\pgfpathlineto{\pgfqpoint{3.997826in}{3.211857in}}%
\pgfpathlineto{\pgfqpoint{3.984526in}{3.223113in}}%
\pgfpathlineto{\pgfqpoint{3.971227in}{3.234601in}}%
\pgfpathlineto{\pgfqpoint{3.963528in}{3.215666in}}%
\pgfpathlineto{\pgfqpoint{3.955827in}{3.196995in}}%
\pgfpathlineto{\pgfqpoint{3.948123in}{3.178581in}}%
\pgfpathlineto{\pgfqpoint{3.940416in}{3.160420in}}%
\pgfpathclose%
\pgfusepath{fill}%
\end{pgfscope}%
\begin{pgfscope}%
\pgfpathrectangle{\pgfqpoint{1.150000in}{0.150000in}}{\pgfqpoint{5.700000in}{5.700000in}}%
\pgfusepath{clip}%
\pgfsetbuttcap%
\pgfsetroundjoin%
\definecolor{currentfill}{rgb}{0.136408,0.541173,0.554483}%
\pgfsetfillcolor{currentfill}%
\pgfsetfillopacity{0.800000}%
\pgfsetlinewidth{0.000000pt}%
\definecolor{currentstroke}{rgb}{0.000000,0.000000,0.000000}%
\pgfsetstrokecolor{currentstroke}%
\pgfsetdash{}{0pt}%
\pgfpathmoveto{\pgfqpoint{3.461001in}{3.724434in}}%
\pgfpathlineto{\pgfqpoint{3.474375in}{3.703913in}}%
\pgfpathlineto{\pgfqpoint{3.487743in}{3.683696in}}%
\pgfpathlineto{\pgfqpoint{3.501105in}{3.663780in}}%
\pgfpathlineto{\pgfqpoint{3.514462in}{3.644163in}}%
\pgfpathlineto{\pgfqpoint{3.522194in}{3.666321in}}%
\pgfpathlineto{\pgfqpoint{3.529921in}{3.688798in}}%
\pgfpathlineto{\pgfqpoint{3.537643in}{3.711600in}}%
\pgfpathlineto{\pgfqpoint{3.545361in}{3.734733in}}%
\pgfpathlineto{\pgfqpoint{3.532005in}{3.754874in}}%
\pgfpathlineto{\pgfqpoint{3.518644in}{3.775314in}}%
\pgfpathlineto{\pgfqpoint{3.505277in}{3.796056in}}%
\pgfpathlineto{\pgfqpoint{3.491904in}{3.817103in}}%
\pgfpathlineto{\pgfqpoint{3.484185in}{3.793432in}}%
\pgfpathlineto{\pgfqpoint{3.476462in}{3.770101in}}%
\pgfpathlineto{\pgfqpoint{3.468734in}{3.747104in}}%
\pgfpathlineto{\pgfqpoint{3.461001in}{3.724434in}}%
\pgfpathclose%
\pgfusepath{fill}%
\end{pgfscope}%
\begin{pgfscope}%
\pgfpathrectangle{\pgfqpoint{1.150000in}{0.150000in}}{\pgfqpoint{5.700000in}{5.700000in}}%
\pgfusepath{clip}%
\pgfsetbuttcap%
\pgfsetroundjoin%
\definecolor{currentfill}{rgb}{0.208030,0.718701,0.472873}%
\pgfsetfillcolor{currentfill}%
\pgfsetfillopacity{0.800000}%
\pgfsetlinewidth{0.000000pt}%
\definecolor{currentstroke}{rgb}{0.000000,0.000000,0.000000}%
\pgfsetstrokecolor{currentstroke}%
\pgfsetdash{}{0pt}%
\pgfpathmoveto{\pgfqpoint{3.614844in}{4.247706in}}%
\pgfpathlineto{\pgfqpoint{3.628226in}{4.224188in}}%
\pgfpathlineto{\pgfqpoint{3.641600in}{4.200980in}}%
\pgfpathlineto{\pgfqpoint{3.654969in}{4.178079in}}%
\pgfpathlineto{\pgfqpoint{3.668331in}{4.155483in}}%
\pgfpathlineto{\pgfqpoint{3.675990in}{4.185317in}}%
\pgfpathlineto{\pgfqpoint{3.683647in}{4.215615in}}%
\pgfpathlineto{\pgfqpoint{3.691301in}{4.246385in}}%
\pgfpathlineto{\pgfqpoint{3.698953in}{4.277635in}}%
\pgfpathlineto{\pgfqpoint{3.685585in}{4.300940in}}%
\pgfpathlineto{\pgfqpoint{3.672212in}{4.324551in}}%
\pgfpathlineto{\pgfqpoint{3.658832in}{4.348471in}}%
\pgfpathlineto{\pgfqpoint{3.645446in}{4.372703in}}%
\pgfpathlineto{\pgfqpoint{3.637799in}{4.340726in}}%
\pgfpathlineto{\pgfqpoint{3.630150in}{4.309240in}}%
\pgfpathlineto{\pgfqpoint{3.622499in}{4.278236in}}%
\pgfpathlineto{\pgfqpoint{3.614844in}{4.247706in}}%
\pgfpathclose%
\pgfusepath{fill}%
\end{pgfscope}%
\begin{pgfscope}%
\pgfpathrectangle{\pgfqpoint{1.150000in}{0.150000in}}{\pgfqpoint{5.700000in}{5.700000in}}%
\pgfusepath{clip}%
\pgfsetbuttcap%
\pgfsetroundjoin%
\definecolor{currentfill}{rgb}{0.157729,0.485932,0.558013}%
\pgfsetfillcolor{currentfill}%
\pgfsetfillopacity{0.800000}%
\pgfsetlinewidth{0.000000pt}%
\definecolor{currentstroke}{rgb}{0.000000,0.000000,0.000000}%
\pgfsetstrokecolor{currentstroke}%
\pgfsetdash{}{0pt}%
\pgfpathmoveto{\pgfqpoint{3.483489in}{3.558602in}}%
\pgfpathlineto{\pgfqpoint{3.496844in}{3.539769in}}%
\pgfpathlineto{\pgfqpoint{3.510193in}{3.521229in}}%
\pgfpathlineto{\pgfqpoint{3.523538in}{3.502979in}}%
\pgfpathlineto{\pgfqpoint{3.536878in}{3.485016in}}%
\pgfpathlineto{\pgfqpoint{3.544626in}{3.505484in}}%
\pgfpathlineto{\pgfqpoint{3.552369in}{3.526239in}}%
\pgfpathlineto{\pgfqpoint{3.560108in}{3.547286in}}%
\pgfpathlineto{\pgfqpoint{3.567842in}{3.568631in}}%
\pgfpathlineto{\pgfqpoint{3.554504in}{3.587079in}}%
\pgfpathlineto{\pgfqpoint{3.541161in}{3.605815in}}%
\pgfpathlineto{\pgfqpoint{3.527814in}{3.624843in}}%
\pgfpathlineto{\pgfqpoint{3.514462in}{3.644163in}}%
\pgfpathlineto{\pgfqpoint{3.506726in}{3.622318in}}%
\pgfpathlineto{\pgfqpoint{3.498985in}{3.600781in}}%
\pgfpathlineto{\pgfqpoint{3.491240in}{3.579544in}}%
\pgfpathlineto{\pgfqpoint{3.483489in}{3.558602in}}%
\pgfpathclose%
\pgfusepath{fill}%
\end{pgfscope}%
\begin{pgfscope}%
\pgfpathrectangle{\pgfqpoint{1.150000in}{0.150000in}}{\pgfqpoint{5.700000in}{5.700000in}}%
\pgfusepath{clip}%
\pgfsetbuttcap%
\pgfsetroundjoin%
\definecolor{currentfill}{rgb}{0.214298,0.355619,0.551184}%
\pgfsetfillcolor{currentfill}%
\pgfsetfillopacity{0.800000}%
\pgfsetlinewidth{0.000000pt}%
\definecolor{currentstroke}{rgb}{0.000000,0.000000,0.000000}%
\pgfsetstrokecolor{currentstroke}%
\pgfsetdash{}{0pt}%
\pgfpathmoveto{\pgfqpoint{3.803129in}{3.182777in}}%
\pgfpathlineto{\pgfqpoint{3.816430in}{3.170361in}}%
\pgfpathlineto{\pgfqpoint{3.829731in}{3.158188in}}%
\pgfpathlineto{\pgfqpoint{3.843033in}{3.146257in}}%
\pgfpathlineto{\pgfqpoint{3.856335in}{3.134566in}}%
\pgfpathlineto{\pgfqpoint{3.864060in}{3.152239in}}%
\pgfpathlineto{\pgfqpoint{3.871782in}{3.170148in}}%
\pgfpathlineto{\pgfqpoint{3.879500in}{3.188300in}}%
\pgfpathlineto{\pgfqpoint{3.887215in}{3.206700in}}%
\pgfpathlineto{\pgfqpoint{3.873917in}{3.218864in}}%
\pgfpathlineto{\pgfqpoint{3.860620in}{3.231268in}}%
\pgfpathlineto{\pgfqpoint{3.847323in}{3.243915in}}%
\pgfpathlineto{\pgfqpoint{3.834027in}{3.256806in}}%
\pgfpathlineto{\pgfqpoint{3.826308in}{3.237919in}}%
\pgfpathlineto{\pgfqpoint{3.818585in}{3.219290in}}%
\pgfpathlineto{\pgfqpoint{3.810859in}{3.200911in}}%
\pgfpathlineto{\pgfqpoint{3.803129in}{3.182777in}}%
\pgfpathclose%
\pgfusepath{fill}%
\end{pgfscope}%
\begin{pgfscope}%
\pgfpathrectangle{\pgfqpoint{1.150000in}{0.150000in}}{\pgfqpoint{5.700000in}{5.700000in}}%
\pgfusepath{clip}%
\pgfsetbuttcap%
\pgfsetroundjoin%
\definecolor{currentfill}{rgb}{0.220057,0.343307,0.549413}%
\pgfsetfillcolor{currentfill}%
\pgfsetfillopacity{0.800000}%
\pgfsetlinewidth{0.000000pt}%
\definecolor{currentstroke}{rgb}{0.000000,0.000000,0.000000}%
\pgfsetstrokecolor{currentstroke}%
\pgfsetdash{}{0pt}%
\pgfpathmoveto{\pgfqpoint{4.077673in}{3.149114in}}%
\pgfpathlineto{\pgfqpoint{4.090989in}{3.139443in}}%
\pgfpathlineto{\pgfqpoint{4.104309in}{3.129994in}}%
\pgfpathlineto{\pgfqpoint{4.117631in}{3.120764in}}%
\pgfpathlineto{\pgfqpoint{4.130957in}{3.111754in}}%
\pgfpathlineto{\pgfqpoint{4.138638in}{3.129390in}}%
\pgfpathlineto{\pgfqpoint{4.146316in}{3.147279in}}%
\pgfpathlineto{\pgfqpoint{4.153992in}{3.165425in}}%
\pgfpathlineto{\pgfqpoint{4.161666in}{3.183836in}}%
\pgfpathlineto{\pgfqpoint{4.148346in}{3.193380in}}%
\pgfpathlineto{\pgfqpoint{4.135029in}{3.203143in}}%
\pgfpathlineto{\pgfqpoint{4.121715in}{3.213127in}}%
\pgfpathlineto{\pgfqpoint{4.108403in}{3.223331in}}%
\pgfpathlineto{\pgfqpoint{4.100724in}{3.204375in}}%
\pgfpathlineto{\pgfqpoint{4.093043in}{3.185690in}}%
\pgfpathlineto{\pgfqpoint{4.085359in}{3.167272in}}%
\pgfpathlineto{\pgfqpoint{4.077673in}{3.149114in}}%
\pgfpathclose%
\pgfusepath{fill}%
\end{pgfscope}%
\begin{pgfscope}%
\pgfpathrectangle{\pgfqpoint{1.150000in}{0.150000in}}{\pgfqpoint{5.700000in}{5.700000in}}%
\pgfusepath{clip}%
\pgfsetbuttcap%
\pgfsetroundjoin%
\definecolor{currentfill}{rgb}{0.175841,0.441290,0.557685}%
\pgfsetfillcolor{currentfill}%
\pgfsetfillopacity{0.800000}%
\pgfsetlinewidth{0.000000pt}%
\definecolor{currentstroke}{rgb}{0.000000,0.000000,0.000000}%
\pgfsetstrokecolor{currentstroke}%
\pgfsetdash{}{0pt}%
\pgfpathmoveto{\pgfqpoint{4.581634in}{3.408860in}}%
\pgfpathlineto{\pgfqpoint{4.595015in}{3.400544in}}%
\pgfpathlineto{\pgfqpoint{4.608401in}{3.392426in}}%
\pgfpathlineto{\pgfqpoint{4.621792in}{3.384506in}}%
\pgfpathlineto{\pgfqpoint{4.635190in}{3.376782in}}%
\pgfpathlineto{\pgfqpoint{4.642824in}{3.398104in}}%
\pgfpathlineto{\pgfqpoint{4.650460in}{3.419805in}}%
\pgfpathlineto{\pgfqpoint{4.658100in}{3.441894in}}%
\pgfpathlineto{\pgfqpoint{4.665743in}{3.464379in}}%
\pgfpathlineto{\pgfqpoint{4.652353in}{3.472826in}}%
\pgfpathlineto{\pgfqpoint{4.638970in}{3.481469in}}%
\pgfpathlineto{\pgfqpoint{4.625592in}{3.490310in}}%
\pgfpathlineto{\pgfqpoint{4.612218in}{3.499350in}}%
\pgfpathlineto{\pgfqpoint{4.604568in}{3.476129in}}%
\pgfpathlineto{\pgfqpoint{4.596921in}{3.453313in}}%
\pgfpathlineto{\pgfqpoint{4.589276in}{3.430892in}}%
\pgfpathlineto{\pgfqpoint{4.581634in}{3.408860in}}%
\pgfpathclose%
\pgfusepath{fill}%
\end{pgfscope}%
\begin{pgfscope}%
\pgfpathrectangle{\pgfqpoint{1.150000in}{0.150000in}}{\pgfqpoint{5.700000in}{5.700000in}}%
\pgfusepath{clip}%
\pgfsetbuttcap%
\pgfsetroundjoin%
\definecolor{currentfill}{rgb}{0.123444,0.636809,0.528763}%
\pgfsetfillcolor{currentfill}%
\pgfsetfillopacity{0.800000}%
\pgfsetlinewidth{0.000000pt}%
\definecolor{currentstroke}{rgb}{0.000000,0.000000,0.000000}%
\pgfsetstrokecolor{currentstroke}%
\pgfsetdash{}{0pt}%
\pgfpathmoveto{\pgfqpoint{3.469176in}{4.004877in}}%
\pgfpathlineto{\pgfqpoint{3.482575in}{3.982015in}}%
\pgfpathlineto{\pgfqpoint{3.495968in}{3.959470in}}%
\pgfpathlineto{\pgfqpoint{3.509354in}{3.937240in}}%
\pgfpathlineto{\pgfqpoint{3.522733in}{3.915322in}}%
\pgfpathlineto{\pgfqpoint{3.530430in}{3.940793in}}%
\pgfpathlineto{\pgfqpoint{3.538123in}{3.966645in}}%
\pgfpathlineto{\pgfqpoint{3.545811in}{3.992884in}}%
\pgfpathlineto{\pgfqpoint{3.553496in}{4.019519in}}%
\pgfpathlineto{\pgfqpoint{3.540115in}{4.042037in}}%
\pgfpathlineto{\pgfqpoint{3.526727in}{4.064868in}}%
\pgfpathlineto{\pgfqpoint{3.513333in}{4.088014in}}%
\pgfpathlineto{\pgfqpoint{3.499931in}{4.111480in}}%
\pgfpathlineto{\pgfqpoint{3.492249in}{4.084231in}}%
\pgfpathlineto{\pgfqpoint{3.484562in}{4.057385in}}%
\pgfpathlineto{\pgfqpoint{3.476871in}{4.030936in}}%
\pgfpathlineto{\pgfqpoint{3.469176in}{4.004877in}}%
\pgfpathclose%
\pgfusepath{fill}%
\end{pgfscope}%
\begin{pgfscope}%
\pgfpathrectangle{\pgfqpoint{1.150000in}{0.150000in}}{\pgfqpoint{5.700000in}{5.700000in}}%
\pgfusepath{clip}%
\pgfsetbuttcap%
\pgfsetroundjoin%
\definecolor{currentfill}{rgb}{0.150148,0.676631,0.506589}%
\pgfsetfillcolor{currentfill}%
\pgfsetfillopacity{0.800000}%
\pgfsetlinewidth{0.000000pt}%
\definecolor{currentstroke}{rgb}{0.000000,0.000000,0.000000}%
\pgfsetstrokecolor{currentstroke}%
\pgfsetdash{}{0pt}%
\pgfpathmoveto{\pgfqpoint{3.499931in}{4.111480in}}%
\pgfpathlineto{\pgfqpoint{3.513333in}{4.088014in}}%
\pgfpathlineto{\pgfqpoint{3.526727in}{4.064868in}}%
\pgfpathlineto{\pgfqpoint{3.540115in}{4.042037in}}%
\pgfpathlineto{\pgfqpoint{3.553496in}{4.019519in}}%
\pgfpathlineto{\pgfqpoint{3.561176in}{4.046555in}}%
\pgfpathlineto{\pgfqpoint{3.568854in}{4.074001in}}%
\pgfpathlineto{\pgfqpoint{3.576527in}{4.101863in}}%
\pgfpathlineto{\pgfqpoint{3.584197in}{4.130151in}}%
\pgfpathlineto{\pgfqpoint{3.570813in}{4.153306in}}%
\pgfpathlineto{\pgfqpoint{3.557423in}{4.176775in}}%
\pgfpathlineto{\pgfqpoint{3.544025in}{4.200562in}}%
\pgfpathlineto{\pgfqpoint{3.530620in}{4.224670in}}%
\pgfpathlineto{\pgfqpoint{3.522954in}{4.195729in}}%
\pgfpathlineto{\pgfqpoint{3.515284in}{4.167222in}}%
\pgfpathlineto{\pgfqpoint{3.507609in}{4.139142in}}%
\pgfpathlineto{\pgfqpoint{3.499931in}{4.111480in}}%
\pgfpathclose%
\pgfusepath{fill}%
\end{pgfscope}%
\begin{pgfscope}%
\pgfpathrectangle{\pgfqpoint{1.150000in}{0.150000in}}{\pgfqpoint{5.700000in}{5.700000in}}%
\pgfusepath{clip}%
\pgfsetbuttcap%
\pgfsetroundjoin%
\definecolor{currentfill}{rgb}{0.335885,0.777018,0.402049}%
\pgfsetfillcolor{currentfill}%
\pgfsetfillopacity{0.800000}%
\pgfsetlinewidth{0.000000pt}%
\definecolor{currentstroke}{rgb}{0.000000,0.000000,0.000000}%
\pgfsetstrokecolor{currentstroke}%
\pgfsetdash{}{0pt}%
\pgfpathmoveto{\pgfqpoint{3.813573in}{4.449687in}}%
\pgfpathlineto{\pgfqpoint{3.826925in}{4.426349in}}%
\pgfpathlineto{\pgfqpoint{3.840272in}{4.403305in}}%
\pgfpathlineto{\pgfqpoint{3.853613in}{4.380551in}}%
\pgfpathlineto{\pgfqpoint{3.866950in}{4.358086in}}%
\pgfpathlineto{\pgfqpoint{3.874605in}{4.392457in}}%
\pgfpathlineto{\pgfqpoint{3.882260in}{4.427382in}}%
\pgfpathlineto{\pgfqpoint{3.889915in}{4.462870in}}%
\pgfpathlineto{\pgfqpoint{3.876573in}{4.485945in}}%
\pgfpathlineto{\pgfqpoint{3.863226in}{4.509309in}}%
\pgfpathlineto{\pgfqpoint{3.849874in}{4.532965in}}%
\pgfpathlineto{\pgfqpoint{3.836516in}{4.556916in}}%
\pgfpathlineto{\pgfqpoint{3.828868in}{4.520602in}}%
\pgfpathlineto{\pgfqpoint{3.821221in}{4.484863in}}%
\pgfpathlineto{\pgfqpoint{3.813573in}{4.449687in}}%
\pgfpathclose%
\pgfusepath{fill}%
\end{pgfscope}%
\begin{pgfscope}%
\pgfpathrectangle{\pgfqpoint{1.150000in}{0.150000in}}{\pgfqpoint{5.700000in}{5.700000in}}%
\pgfusepath{clip}%
\pgfsetbuttcap%
\pgfsetroundjoin%
\definecolor{currentfill}{rgb}{0.119738,0.603785,0.541400}%
\pgfsetfillcolor{currentfill}%
\pgfsetfillopacity{0.800000}%
\pgfsetlinewidth{0.000000pt}%
\definecolor{currentstroke}{rgb}{0.000000,0.000000,0.000000}%
\pgfsetstrokecolor{currentstroke}%
\pgfsetdash{}{0pt}%
\pgfpathmoveto{\pgfqpoint{3.438348in}{3.904400in}}%
\pgfpathlineto{\pgfqpoint{3.451747in}{3.882104in}}%
\pgfpathlineto{\pgfqpoint{3.465139in}{3.860124in}}%
\pgfpathlineto{\pgfqpoint{3.478525in}{3.838458in}}%
\pgfpathlineto{\pgfqpoint{3.491904in}{3.817103in}}%
\pgfpathlineto{\pgfqpoint{3.499618in}{3.841121in}}%
\pgfpathlineto{\pgfqpoint{3.507327in}{3.865493in}}%
\pgfpathlineto{\pgfqpoint{3.515033in}{3.890224in}}%
\pgfpathlineto{\pgfqpoint{3.522733in}{3.915322in}}%
\pgfpathlineto{\pgfqpoint{3.509354in}{3.937240in}}%
\pgfpathlineto{\pgfqpoint{3.495968in}{3.959470in}}%
\pgfpathlineto{\pgfqpoint{3.482575in}{3.982015in}}%
\pgfpathlineto{\pgfqpoint{3.469176in}{4.004877in}}%
\pgfpathlineto{\pgfqpoint{3.461476in}{3.979201in}}%
\pgfpathlineto{\pgfqpoint{3.453771in}{3.953900in}}%
\pgfpathlineto{\pgfqpoint{3.446062in}{3.928969in}}%
\pgfpathlineto{\pgfqpoint{3.438348in}{3.904400in}}%
\pgfpathclose%
\pgfusepath{fill}%
\end{pgfscope}%
\begin{pgfscope}%
\pgfpathrectangle{\pgfqpoint{1.150000in}{0.150000in}}{\pgfqpoint{5.700000in}{5.700000in}}%
\pgfusepath{clip}%
\pgfsetbuttcap%
\pgfsetroundjoin%
\definecolor{currentfill}{rgb}{0.199430,0.387607,0.554642}%
\pgfsetfillcolor{currentfill}%
\pgfsetfillopacity{0.800000}%
\pgfsetlinewidth{0.000000pt}%
\definecolor{currentstroke}{rgb}{0.000000,0.000000,0.000000}%
\pgfsetstrokecolor{currentstroke}%
\pgfsetdash{}{0pt}%
\pgfpathmoveto{\pgfqpoint{3.612466in}{3.275874in}}%
\pgfpathlineto{\pgfqpoint{3.625782in}{3.260844in}}%
\pgfpathlineto{\pgfqpoint{3.639096in}{3.246080in}}%
\pgfpathlineto{\pgfqpoint{3.652408in}{3.231578in}}%
\pgfpathlineto{\pgfqpoint{3.665719in}{3.217338in}}%
\pgfpathlineto{\pgfqpoint{3.673473in}{3.235407in}}%
\pgfpathlineto{\pgfqpoint{3.681224in}{3.253717in}}%
\pgfpathlineto{\pgfqpoint{3.688970in}{3.272272in}}%
\pgfpathlineto{\pgfqpoint{3.696712in}{3.291077in}}%
\pgfpathlineto{\pgfqpoint{3.683406in}{3.305763in}}%
\pgfpathlineto{\pgfqpoint{3.670097in}{3.320710in}}%
\pgfpathlineto{\pgfqpoint{3.656787in}{3.335921in}}%
\pgfpathlineto{\pgfqpoint{3.643475in}{3.351397in}}%
\pgfpathlineto{\pgfqpoint{3.635730in}{3.332133in}}%
\pgfpathlineto{\pgfqpoint{3.627979in}{3.313128in}}%
\pgfpathlineto{\pgfqpoint{3.620225in}{3.294377in}}%
\pgfpathlineto{\pgfqpoint{3.612466in}{3.275874in}}%
\pgfpathclose%
\pgfusepath{fill}%
\end{pgfscope}%
\begin{pgfscope}%
\pgfpathrectangle{\pgfqpoint{1.150000in}{0.150000in}}{\pgfqpoint{5.700000in}{5.700000in}}%
\pgfusepath{clip}%
\pgfsetbuttcap%
\pgfsetroundjoin%
\definecolor{currentfill}{rgb}{0.188923,0.410910,0.556326}%
\pgfsetfillcolor{currentfill}%
\pgfsetfillopacity{0.800000}%
\pgfsetlinewidth{0.000000pt}%
\definecolor{currentstroke}{rgb}{0.000000,0.000000,0.000000}%
\pgfsetstrokecolor{currentstroke}%
\pgfsetdash{}{0pt}%
\pgfpathmoveto{\pgfqpoint{3.559177in}{3.338683in}}%
\pgfpathlineto{\pgfqpoint{3.572504in}{3.322573in}}%
\pgfpathlineto{\pgfqpoint{3.585827in}{3.306736in}}%
\pgfpathlineto{\pgfqpoint{3.599148in}{3.291170in}}%
\pgfpathlineto{\pgfqpoint{3.612466in}{3.275874in}}%
\pgfpathlineto{\pgfqpoint{3.620225in}{3.294377in}}%
\pgfpathlineto{\pgfqpoint{3.627979in}{3.313128in}}%
\pgfpathlineto{\pgfqpoint{3.635730in}{3.332133in}}%
\pgfpathlineto{\pgfqpoint{3.643475in}{3.351397in}}%
\pgfpathlineto{\pgfqpoint{3.630161in}{3.367140in}}%
\pgfpathlineto{\pgfqpoint{3.616844in}{3.383154in}}%
\pgfpathlineto{\pgfqpoint{3.603525in}{3.399438in}}%
\pgfpathlineto{\pgfqpoint{3.590202in}{3.415997in}}%
\pgfpathlineto{\pgfqpoint{3.582453in}{3.396273in}}%
\pgfpathlineto{\pgfqpoint{3.574699in}{3.376816in}}%
\pgfpathlineto{\pgfqpoint{3.566941in}{3.357621in}}%
\pgfpathlineto{\pgfqpoint{3.559177in}{3.338683in}}%
\pgfpathclose%
\pgfusepath{fill}%
\end{pgfscope}%
\begin{pgfscope}%
\pgfpathrectangle{\pgfqpoint{1.150000in}{0.150000in}}{\pgfqpoint{5.700000in}{5.700000in}}%
\pgfusepath{clip}%
\pgfsetbuttcap%
\pgfsetroundjoin%
\definecolor{currentfill}{rgb}{0.319809,0.770914,0.411152}%
\pgfsetfillcolor{currentfill}%
\pgfsetfillopacity{0.800000}%
\pgfsetlinewidth{0.000000pt}%
\definecolor{currentstroke}{rgb}{0.000000,0.000000,0.000000}%
\pgfsetstrokecolor{currentstroke}%
\pgfsetdash{}{0pt}%
\pgfpathmoveto{\pgfqpoint{3.729542in}{4.407621in}}%
\pgfpathlineto{\pgfqpoint{3.742910in}{4.383872in}}%
\pgfpathlineto{\pgfqpoint{3.756271in}{4.360425in}}%
\pgfpathlineto{\pgfqpoint{3.769627in}{4.337278in}}%
\pgfpathlineto{\pgfqpoint{3.782977in}{4.314427in}}%
\pgfpathlineto{\pgfqpoint{3.790627in}{4.347445in}}%
\pgfpathlineto{\pgfqpoint{3.798277in}{4.380988in}}%
\pgfpathlineto{\pgfqpoint{3.805925in}{4.415065in}}%
\pgfpathlineto{\pgfqpoint{3.813573in}{4.449687in}}%
\pgfpathlineto{\pgfqpoint{3.800216in}{4.473320in}}%
\pgfpathlineto{\pgfqpoint{3.786853in}{4.497252in}}%
\pgfpathlineto{\pgfqpoint{3.773485in}{4.521486in}}%
\pgfpathlineto{\pgfqpoint{3.760110in}{4.546023in}}%
\pgfpathlineto{\pgfqpoint{3.752470in}{4.510600in}}%
\pgfpathlineto{\pgfqpoint{3.744828in}{4.475732in}}%
\pgfpathlineto{\pgfqpoint{3.737186in}{4.441409in}}%
\pgfpathlineto{\pgfqpoint{3.729542in}{4.407621in}}%
\pgfpathclose%
\pgfusepath{fill}%
\end{pgfscope}%
\begin{pgfscope}%
\pgfpathrectangle{\pgfqpoint{1.150000in}{0.150000in}}{\pgfqpoint{5.700000in}{5.700000in}}%
\pgfusepath{clip}%
\pgfsetbuttcap%
\pgfsetroundjoin%
\definecolor{currentfill}{rgb}{0.221989,0.339161,0.548752}%
\pgfsetfillcolor{currentfill}%
\pgfsetfillopacity{0.800000}%
\pgfsetlinewidth{0.000000pt}%
\definecolor{currentstroke}{rgb}{0.000000,0.000000,0.000000}%
\pgfsetstrokecolor{currentstroke}%
\pgfsetdash{}{0pt}%
\pgfpathmoveto{\pgfqpoint{3.856335in}{3.134566in}}%
\pgfpathlineto{\pgfqpoint{3.869638in}{3.123114in}}%
\pgfpathlineto{\pgfqpoint{3.882943in}{3.111900in}}%
\pgfpathlineto{\pgfqpoint{3.896248in}{3.100921in}}%
\pgfpathlineto{\pgfqpoint{3.909556in}{3.090176in}}%
\pgfpathlineto{\pgfqpoint{3.917276in}{3.107388in}}%
\pgfpathlineto{\pgfqpoint{3.924992in}{3.124829in}}%
\pgfpathlineto{\pgfqpoint{3.932706in}{3.142504in}}%
\pgfpathlineto{\pgfqpoint{3.940416in}{3.160420in}}%
\pgfpathlineto{\pgfqpoint{3.927114in}{3.171637in}}%
\pgfpathlineto{\pgfqpoint{3.913813in}{3.183088in}}%
\pgfpathlineto{\pgfqpoint{3.900514in}{3.194775in}}%
\pgfpathlineto{\pgfqpoint{3.887215in}{3.206700in}}%
\pgfpathlineto{\pgfqpoint{3.879500in}{3.188300in}}%
\pgfpathlineto{\pgfqpoint{3.871782in}{3.170148in}}%
\pgfpathlineto{\pgfqpoint{3.864060in}{3.152239in}}%
\pgfpathlineto{\pgfqpoint{3.856335in}{3.134566in}}%
\pgfpathclose%
\pgfusepath{fill}%
\end{pgfscope}%
\begin{pgfscope}%
\pgfpathrectangle{\pgfqpoint{1.150000in}{0.150000in}}{\pgfqpoint{5.700000in}{5.700000in}}%
\pgfusepath{clip}%
\pgfsetbuttcap%
\pgfsetroundjoin%
\definecolor{currentfill}{rgb}{0.208623,0.367752,0.552675}%
\pgfsetfillcolor{currentfill}%
\pgfsetfillopacity{0.800000}%
\pgfsetlinewidth{0.000000pt}%
\definecolor{currentstroke}{rgb}{0.000000,0.000000,0.000000}%
\pgfsetstrokecolor{currentstroke}%
\pgfsetdash{}{0pt}%
\pgfpathmoveto{\pgfqpoint{3.665719in}{3.217338in}}%
\pgfpathlineto{\pgfqpoint{3.679028in}{3.203356in}}%
\pgfpathlineto{\pgfqpoint{3.692336in}{3.189632in}}%
\pgfpathlineto{\pgfqpoint{3.705643in}{3.176163in}}%
\pgfpathlineto{\pgfqpoint{3.718950in}{3.162948in}}%
\pgfpathlineto{\pgfqpoint{3.726700in}{3.180585in}}%
\pgfpathlineto{\pgfqpoint{3.734446in}{3.198455in}}%
\pgfpathlineto{\pgfqpoint{3.742188in}{3.216561in}}%
\pgfpathlineto{\pgfqpoint{3.749926in}{3.234910in}}%
\pgfpathlineto{\pgfqpoint{3.736624in}{3.248569in}}%
\pgfpathlineto{\pgfqpoint{3.723321in}{3.262482in}}%
\pgfpathlineto{\pgfqpoint{3.710017in}{3.276651in}}%
\pgfpathlineto{\pgfqpoint{3.696712in}{3.291077in}}%
\pgfpathlineto{\pgfqpoint{3.688970in}{3.272272in}}%
\pgfpathlineto{\pgfqpoint{3.681224in}{3.253717in}}%
\pgfpathlineto{\pgfqpoint{3.673473in}{3.235407in}}%
\pgfpathlineto{\pgfqpoint{3.665719in}{3.217338in}}%
\pgfpathclose%
\pgfusepath{fill}%
\end{pgfscope}%
\begin{pgfscope}%
\pgfpathrectangle{\pgfqpoint{1.150000in}{0.150000in}}{\pgfqpoint{5.700000in}{5.700000in}}%
\pgfusepath{clip}%
\pgfsetbuttcap%
\pgfsetroundjoin%
\definecolor{currentfill}{rgb}{0.179019,0.433756,0.557430}%
\pgfsetfillcolor{currentfill}%
\pgfsetfillopacity{0.800000}%
\pgfsetlinewidth{0.000000pt}%
\definecolor{currentstroke}{rgb}{0.000000,0.000000,0.000000}%
\pgfsetstrokecolor{currentstroke}%
\pgfsetdash{}{0pt}%
\pgfpathmoveto{\pgfqpoint{3.505840in}{3.405902in}}%
\pgfpathlineto{\pgfqpoint{3.519180in}{3.388676in}}%
\pgfpathlineto{\pgfqpoint{3.532516in}{3.371732in}}%
\pgfpathlineto{\pgfqpoint{3.545848in}{3.355069in}}%
\pgfpathlineto{\pgfqpoint{3.559177in}{3.338683in}}%
\pgfpathlineto{\pgfqpoint{3.566941in}{3.357621in}}%
\pgfpathlineto{\pgfqpoint{3.574699in}{3.376816in}}%
\pgfpathlineto{\pgfqpoint{3.582453in}{3.396273in}}%
\pgfpathlineto{\pgfqpoint{3.590202in}{3.415997in}}%
\pgfpathlineto{\pgfqpoint{3.576876in}{3.432832in}}%
\pgfpathlineto{\pgfqpoint{3.563547in}{3.449945in}}%
\pgfpathlineto{\pgfqpoint{3.550215in}{3.467339in}}%
\pgfpathlineto{\pgfqpoint{3.536878in}{3.485016in}}%
\pgfpathlineto{\pgfqpoint{3.529126in}{3.464830in}}%
\pgfpathlineto{\pgfqpoint{3.521369in}{3.444919in}}%
\pgfpathlineto{\pgfqpoint{3.513607in}{3.425278in}}%
\pgfpathlineto{\pgfqpoint{3.505840in}{3.405902in}}%
\pgfpathclose%
\pgfusepath{fill}%
\end{pgfscope}%
\begin{pgfscope}%
\pgfpathrectangle{\pgfqpoint{1.150000in}{0.150000in}}{\pgfqpoint{5.700000in}{5.700000in}}%
\pgfusepath{clip}%
\pgfsetbuttcap%
\pgfsetroundjoin%
\definecolor{currentfill}{rgb}{0.147607,0.511733,0.557049}%
\pgfsetfillcolor{currentfill}%
\pgfsetfillopacity{0.800000}%
\pgfsetlinewidth{0.000000pt}%
\definecolor{currentstroke}{rgb}{0.000000,0.000000,0.000000}%
\pgfsetstrokecolor{currentstroke}%
\pgfsetdash{}{0pt}%
\pgfpathmoveto{\pgfqpoint{3.430020in}{3.636915in}}%
\pgfpathlineto{\pgfqpoint{3.443396in}{3.616884in}}%
\pgfpathlineto{\pgfqpoint{3.456766in}{3.597157in}}%
\pgfpathlineto{\pgfqpoint{3.470130in}{3.577731in}}%
\pgfpathlineto{\pgfqpoint{3.483489in}{3.558602in}}%
\pgfpathlineto{\pgfqpoint{3.491240in}{3.579544in}}%
\pgfpathlineto{\pgfqpoint{3.498985in}{3.600781in}}%
\pgfpathlineto{\pgfqpoint{3.506726in}{3.622318in}}%
\pgfpathlineto{\pgfqpoint{3.514462in}{3.644163in}}%
\pgfpathlineto{\pgfqpoint{3.501105in}{3.663780in}}%
\pgfpathlineto{\pgfqpoint{3.487743in}{3.683696in}}%
\pgfpathlineto{\pgfqpoint{3.474375in}{3.703913in}}%
\pgfpathlineto{\pgfqpoint{3.461001in}{3.724434in}}%
\pgfpathlineto{\pgfqpoint{3.453263in}{3.702087in}}%
\pgfpathlineto{\pgfqpoint{3.445521in}{3.680055in}}%
\pgfpathlineto{\pgfqpoint{3.437773in}{3.658333in}}%
\pgfpathlineto{\pgfqpoint{3.430020in}{3.636915in}}%
\pgfpathclose%
\pgfusepath{fill}%
\end{pgfscope}%
\begin{pgfscope}%
\pgfpathrectangle{\pgfqpoint{1.150000in}{0.150000in}}{\pgfqpoint{5.700000in}{5.700000in}}%
\pgfusepath{clip}%
\pgfsetbuttcap%
\pgfsetroundjoin%
\definecolor{currentfill}{rgb}{0.196571,0.711827,0.479221}%
\pgfsetfillcolor{currentfill}%
\pgfsetfillopacity{0.800000}%
\pgfsetlinewidth{0.000000pt}%
\definecolor{currentstroke}{rgb}{0.000000,0.000000,0.000000}%
\pgfsetstrokecolor{currentstroke}%
\pgfsetdash{}{0pt}%
\pgfpathmoveto{\pgfqpoint{3.530620in}{4.224670in}}%
\pgfpathlineto{\pgfqpoint{3.544025in}{4.200562in}}%
\pgfpathlineto{\pgfqpoint{3.557423in}{4.176775in}}%
\pgfpathlineto{\pgfqpoint{3.570813in}{4.153306in}}%
\pgfpathlineto{\pgfqpoint{3.584197in}{4.130151in}}%
\pgfpathlineto{\pgfqpoint{3.591864in}{4.158871in}}%
\pgfpathlineto{\pgfqpoint{3.599527in}{4.188031in}}%
\pgfpathlineto{\pgfqpoint{3.607187in}{4.217640in}}%
\pgfpathlineto{\pgfqpoint{3.614844in}{4.247706in}}%
\pgfpathlineto{\pgfqpoint{3.601457in}{4.271537in}}%
\pgfpathlineto{\pgfqpoint{3.588062in}{4.295683in}}%
\pgfpathlineto{\pgfqpoint{3.574660in}{4.320149in}}%
\pgfpathlineto{\pgfqpoint{3.561251in}{4.344937in}}%
\pgfpathlineto{\pgfqpoint{3.553598in}{4.314178in}}%
\pgfpathlineto{\pgfqpoint{3.545943in}{4.283886in}}%
\pgfpathlineto{\pgfqpoint{3.538283in}{4.254053in}}%
\pgfpathlineto{\pgfqpoint{3.530620in}{4.224670in}}%
\pgfpathclose%
\pgfusepath{fill}%
\end{pgfscope}%
\begin{pgfscope}%
\pgfpathrectangle{\pgfqpoint{1.150000in}{0.150000in}}{\pgfqpoint{5.700000in}{5.700000in}}%
\pgfusepath{clip}%
\pgfsetbuttcap%
\pgfsetroundjoin%
\definecolor{currentfill}{rgb}{0.223925,0.334994,0.548053}%
\pgfsetfillcolor{currentfill}%
\pgfsetfillopacity{0.800000}%
\pgfsetlinewidth{0.000000pt}%
\definecolor{currentstroke}{rgb}{0.000000,0.000000,0.000000}%
\pgfsetstrokecolor{currentstroke}%
\pgfsetdash{}{0pt}%
\pgfpathmoveto{\pgfqpoint{3.993641in}{3.117869in}}%
\pgfpathlineto{\pgfqpoint{4.006953in}{3.107803in}}%
\pgfpathlineto{\pgfqpoint{4.020267in}{3.097963in}}%
\pgfpathlineto{\pgfqpoint{4.033583in}{3.088349in}}%
\pgfpathlineto{\pgfqpoint{4.046902in}{3.078957in}}%
\pgfpathlineto{\pgfqpoint{4.054599in}{3.096137in}}%
\pgfpathlineto{\pgfqpoint{4.062293in}{3.113552in}}%
\pgfpathlineto{\pgfqpoint{4.069984in}{3.131209in}}%
\pgfpathlineto{\pgfqpoint{4.077673in}{3.149114in}}%
\pgfpathlineto{\pgfqpoint{4.064359in}{3.159007in}}%
\pgfpathlineto{\pgfqpoint{4.051048in}{3.169124in}}%
\pgfpathlineto{\pgfqpoint{4.037739in}{3.179467in}}%
\pgfpathlineto{\pgfqpoint{4.024433in}{3.190035in}}%
\pgfpathlineto{\pgfqpoint{4.016739in}{3.171616in}}%
\pgfpathlineto{\pgfqpoint{4.009043in}{3.153453in}}%
\pgfpathlineto{\pgfqpoint{4.001344in}{3.135539in}}%
\pgfpathlineto{\pgfqpoint{3.993641in}{3.117869in}}%
\pgfpathclose%
\pgfusepath{fill}%
\end{pgfscope}%
\begin{pgfscope}%
\pgfpathrectangle{\pgfqpoint{1.150000in}{0.150000in}}{\pgfqpoint{5.700000in}{5.700000in}}%
\pgfusepath{clip}%
\pgfsetbuttcap%
\pgfsetroundjoin%
\definecolor{currentfill}{rgb}{0.203063,0.379716,0.553925}%
\pgfsetfillcolor{currentfill}%
\pgfsetfillopacity{0.800000}%
\pgfsetlinewidth{0.000000pt}%
\definecolor{currentstroke}{rgb}{0.000000,0.000000,0.000000}%
\pgfsetstrokecolor{currentstroke}%
\pgfsetdash{}{0pt}%
\pgfpathmoveto{\pgfqpoint{4.383002in}{3.229664in}}%
\pgfpathlineto{\pgfqpoint{4.396363in}{3.221710in}}%
\pgfpathlineto{\pgfqpoint{4.409729in}{3.213961in}}%
\pgfpathlineto{\pgfqpoint{4.423100in}{3.206415in}}%
\pgfpathlineto{\pgfqpoint{4.436477in}{3.199073in}}%
\pgfpathlineto{\pgfqpoint{4.444115in}{3.217701in}}%
\pgfpathlineto{\pgfqpoint{4.451753in}{3.236629in}}%
\pgfpathlineto{\pgfqpoint{4.459390in}{3.255866in}}%
\pgfpathlineto{\pgfqpoint{4.467028in}{3.275418in}}%
\pgfpathlineto{\pgfqpoint{4.453659in}{3.283387in}}%
\pgfpathlineto{\pgfqpoint{4.440295in}{3.291559in}}%
\pgfpathlineto{\pgfqpoint{4.426936in}{3.299935in}}%
\pgfpathlineto{\pgfqpoint{4.413582in}{3.308517in}}%
\pgfpathlineto{\pgfqpoint{4.405937in}{3.288325in}}%
\pgfpathlineto{\pgfqpoint{4.398292in}{3.268458in}}%
\pgfpathlineto{\pgfqpoint{4.390647in}{3.248906in}}%
\pgfpathlineto{\pgfqpoint{4.383002in}{3.229664in}}%
\pgfpathclose%
\pgfusepath{fill}%
\end{pgfscope}%
\begin{pgfscope}%
\pgfpathrectangle{\pgfqpoint{1.150000in}{0.150000in}}{\pgfqpoint{5.700000in}{5.700000in}}%
\pgfusepath{clip}%
\pgfsetbuttcap%
\pgfsetroundjoin%
\definecolor{currentfill}{rgb}{0.210503,0.363727,0.552206}%
\pgfsetfillcolor{currentfill}%
\pgfsetfillopacity{0.800000}%
\pgfsetlinewidth{0.000000pt}%
\definecolor{currentstroke}{rgb}{0.000000,0.000000,0.000000}%
\pgfsetstrokecolor{currentstroke}%
\pgfsetdash{}{0pt}%
\pgfpathmoveto{\pgfqpoint{4.298992in}{3.187137in}}%
\pgfpathlineto{\pgfqpoint{4.312341in}{3.178951in}}%
\pgfpathlineto{\pgfqpoint{4.325694in}{3.170973in}}%
\pgfpathlineto{\pgfqpoint{4.339053in}{3.163202in}}%
\pgfpathlineto{\pgfqpoint{4.352416in}{3.155638in}}%
\pgfpathlineto{\pgfqpoint{4.360064in}{3.173717in}}%
\pgfpathlineto{\pgfqpoint{4.367710in}{3.192076in}}%
\pgfpathlineto{\pgfqpoint{4.375357in}{3.210723in}}%
\pgfpathlineto{\pgfqpoint{4.383002in}{3.229664in}}%
\pgfpathlineto{\pgfqpoint{4.369646in}{3.237823in}}%
\pgfpathlineto{\pgfqpoint{4.356294in}{3.246189in}}%
\pgfpathlineto{\pgfqpoint{4.342947in}{3.254763in}}%
\pgfpathlineto{\pgfqpoint{4.329605in}{3.263545in}}%
\pgfpathlineto{\pgfqpoint{4.321953in}{3.243996in}}%
\pgfpathlineto{\pgfqpoint{4.314300in}{3.224750in}}%
\pgfpathlineto{\pgfqpoint{4.306646in}{3.205800in}}%
\pgfpathlineto{\pgfqpoint{4.298992in}{3.187137in}}%
\pgfpathclose%
\pgfusepath{fill}%
\end{pgfscope}%
\begin{pgfscope}%
\pgfpathrectangle{\pgfqpoint{1.150000in}{0.150000in}}{\pgfqpoint{5.700000in}{5.700000in}}%
\pgfusepath{clip}%
\pgfsetbuttcap%
\pgfsetroundjoin%
\definecolor{currentfill}{rgb}{0.168126,0.459988,0.558082}%
\pgfsetfillcolor{currentfill}%
\pgfsetfillopacity{0.800000}%
\pgfsetlinewidth{0.000000pt}%
\definecolor{currentstroke}{rgb}{0.000000,0.000000,0.000000}%
\pgfsetstrokecolor{currentstroke}%
\pgfsetdash{}{0pt}%
\pgfpathmoveto{\pgfqpoint{4.665743in}{3.464379in}}%
\pgfpathlineto{\pgfqpoint{4.679137in}{3.456128in}}%
\pgfpathlineto{\pgfqpoint{4.692538in}{3.448073in}}%
\pgfpathlineto{\pgfqpoint{4.705945in}{3.440213in}}%
\pgfpathlineto{\pgfqpoint{4.719358in}{3.432546in}}%
\pgfpathlineto{\pgfqpoint{4.726995in}{3.454693in}}%
\pgfpathlineto{\pgfqpoint{4.734636in}{3.477247in}}%
\pgfpathlineto{\pgfqpoint{4.742281in}{3.500215in}}%
\pgfpathlineto{\pgfqpoint{4.728875in}{3.508445in}}%
\pgfpathlineto{\pgfqpoint{4.715475in}{3.516869in}}%
\pgfpathlineto{\pgfqpoint{4.702080in}{3.525488in}}%
\pgfpathlineto{\pgfqpoint{4.688692in}{3.534303in}}%
\pgfpathlineto{\pgfqpoint{4.681038in}{3.510575in}}%
\pgfpathlineto{\pgfqpoint{4.673389in}{3.487270in}}%
\pgfpathlineto{\pgfqpoint{4.665743in}{3.464379in}}%
\pgfpathclose%
\pgfusepath{fill}%
\end{pgfscope}%
\begin{pgfscope}%
\pgfpathrectangle{\pgfqpoint{1.150000in}{0.150000in}}{\pgfqpoint{5.700000in}{5.700000in}}%
\pgfusepath{clip}%
\pgfsetbuttcap%
\pgfsetroundjoin%
\definecolor{currentfill}{rgb}{0.296479,0.761561,0.424223}%
\pgfsetfillcolor{currentfill}%
\pgfsetfillopacity{0.800000}%
\pgfsetlinewidth{0.000000pt}%
\definecolor{currentstroke}{rgb}{0.000000,0.000000,0.000000}%
\pgfsetstrokecolor{currentstroke}%
\pgfsetdash{}{0pt}%
\pgfpathmoveto{\pgfqpoint{3.645446in}{4.372703in}}%
\pgfpathlineto{\pgfqpoint{3.658832in}{4.348471in}}%
\pgfpathlineto{\pgfqpoint{3.672212in}{4.324551in}}%
\pgfpathlineto{\pgfqpoint{3.685585in}{4.300940in}}%
\pgfpathlineto{\pgfqpoint{3.698953in}{4.277635in}}%
\pgfpathlineto{\pgfqpoint{3.706603in}{4.309375in}}%
\pgfpathlineto{\pgfqpoint{3.714251in}{4.341613in}}%
\pgfpathlineto{\pgfqpoint{3.721897in}{4.374359in}}%
\pgfpathlineto{\pgfqpoint{3.729542in}{4.407621in}}%
\pgfpathlineto{\pgfqpoint{3.716169in}{4.431675in}}%
\pgfpathlineto{\pgfqpoint{3.702789in}{4.456036in}}%
\pgfpathlineto{\pgfqpoint{3.689403in}{4.480708in}}%
\pgfpathlineto{\pgfqpoint{3.676010in}{4.505694in}}%
\pgfpathlineto{\pgfqpoint{3.668372in}{4.471665in}}%
\pgfpathlineto{\pgfqpoint{3.660732in}{4.438163in}}%
\pgfpathlineto{\pgfqpoint{3.653090in}{4.405179in}}%
\pgfpathlineto{\pgfqpoint{3.645446in}{4.372703in}}%
\pgfpathclose%
\pgfusepath{fill}%
\end{pgfscope}%
\begin{pgfscope}%
\pgfpathrectangle{\pgfqpoint{1.150000in}{0.150000in}}{\pgfqpoint{5.700000in}{5.700000in}}%
\pgfusepath{clip}%
\pgfsetbuttcap%
\pgfsetroundjoin%
\definecolor{currentfill}{rgb}{0.195860,0.395433,0.555276}%
\pgfsetfillcolor{currentfill}%
\pgfsetfillopacity{0.800000}%
\pgfsetlinewidth{0.000000pt}%
\definecolor{currentstroke}{rgb}{0.000000,0.000000,0.000000}%
\pgfsetstrokecolor{currentstroke}%
\pgfsetdash{}{0pt}%
\pgfpathmoveto{\pgfqpoint{4.467028in}{3.275418in}}%
\pgfpathlineto{\pgfqpoint{4.480402in}{3.267651in}}%
\pgfpathlineto{\pgfqpoint{4.493782in}{3.260085in}}%
\pgfpathlineto{\pgfqpoint{4.507167in}{3.252720in}}%
\pgfpathlineto{\pgfqpoint{4.520558in}{3.245554in}}%
\pgfpathlineto{\pgfqpoint{4.528189in}{3.264783in}}%
\pgfpathlineto{\pgfqpoint{4.535820in}{3.284334in}}%
\pgfpathlineto{\pgfqpoint{4.543452in}{3.304216in}}%
\pgfpathlineto{\pgfqpoint{4.551085in}{3.324436in}}%
\pgfpathlineto{\pgfqpoint{4.537703in}{3.332260in}}%
\pgfpathlineto{\pgfqpoint{4.524325in}{3.340284in}}%
\pgfpathlineto{\pgfqpoint{4.510953in}{3.348509in}}%
\pgfpathlineto{\pgfqpoint{4.497586in}{3.356935in}}%
\pgfpathlineto{\pgfqpoint{4.489945in}{3.336044in}}%
\pgfpathlineto{\pgfqpoint{4.482306in}{3.315499in}}%
\pgfpathlineto{\pgfqpoint{4.474667in}{3.295293in}}%
\pgfpathlineto{\pgfqpoint{4.467028in}{3.275418in}}%
\pgfpathclose%
\pgfusepath{fill}%
\end{pgfscope}%
\begin{pgfscope}%
\pgfpathrectangle{\pgfqpoint{1.150000in}{0.150000in}}{\pgfqpoint{5.700000in}{5.700000in}}%
\pgfusepath{clip}%
\pgfsetbuttcap%
\pgfsetroundjoin%
\definecolor{currentfill}{rgb}{0.126453,0.570633,0.549841}%
\pgfsetfillcolor{currentfill}%
\pgfsetfillopacity{0.800000}%
\pgfsetlinewidth{0.000000pt}%
\definecolor{currentstroke}{rgb}{0.000000,0.000000,0.000000}%
\pgfsetstrokecolor{currentstroke}%
\pgfsetdash{}{0pt}%
\pgfpathmoveto{\pgfqpoint{3.407443in}{3.809617in}}%
\pgfpathlineto{\pgfqpoint{3.420842in}{3.787851in}}%
\pgfpathlineto{\pgfqpoint{3.434235in}{3.766400in}}%
\pgfpathlineto{\pgfqpoint{3.447621in}{3.745263in}}%
\pgfpathlineto{\pgfqpoint{3.461001in}{3.724434in}}%
\pgfpathlineto{\pgfqpoint{3.468734in}{3.747104in}}%
\pgfpathlineto{\pgfqpoint{3.476462in}{3.770101in}}%
\pgfpathlineto{\pgfqpoint{3.484185in}{3.793432in}}%
\pgfpathlineto{\pgfqpoint{3.491904in}{3.817103in}}%
\pgfpathlineto{\pgfqpoint{3.478525in}{3.838458in}}%
\pgfpathlineto{\pgfqpoint{3.465139in}{3.860124in}}%
\pgfpathlineto{\pgfqpoint{3.451747in}{3.882104in}}%
\pgfpathlineto{\pgfqpoint{3.438348in}{3.904400in}}%
\pgfpathlineto{\pgfqpoint{3.430629in}{3.880186in}}%
\pgfpathlineto{\pgfqpoint{3.422905in}{3.856322in}}%
\pgfpathlineto{\pgfqpoint{3.415177in}{3.832801in}}%
\pgfpathlineto{\pgfqpoint{3.407443in}{3.809617in}}%
\pgfpathclose%
\pgfusepath{fill}%
\end{pgfscope}%
\begin{pgfscope}%
\pgfpathrectangle{\pgfqpoint{1.150000in}{0.150000in}}{\pgfqpoint{5.700000in}{5.700000in}}%
\pgfusepath{clip}%
\pgfsetbuttcap%
\pgfsetroundjoin%
\definecolor{currentfill}{rgb}{0.218130,0.347432,0.550038}%
\pgfsetfillcolor{currentfill}%
\pgfsetfillopacity{0.800000}%
\pgfsetlinewidth{0.000000pt}%
\definecolor{currentstroke}{rgb}{0.000000,0.000000,0.000000}%
\pgfsetstrokecolor{currentstroke}%
\pgfsetdash{}{0pt}%
\pgfpathmoveto{\pgfqpoint{4.214982in}{3.147828in}}%
\pgfpathlineto{\pgfqpoint{4.228320in}{3.139362in}}%
\pgfpathlineto{\pgfqpoint{4.241663in}{3.131109in}}%
\pgfpathlineto{\pgfqpoint{4.255009in}{3.123067in}}%
\pgfpathlineto{\pgfqpoint{4.268361in}{3.115235in}}%
\pgfpathlineto{\pgfqpoint{4.276021in}{3.132812in}}%
\pgfpathlineto{\pgfqpoint{4.283679in}{3.150650in}}%
\pgfpathlineto{\pgfqpoint{4.291336in}{3.168756in}}%
\pgfpathlineto{\pgfqpoint{4.298992in}{3.187137in}}%
\pgfpathlineto{\pgfqpoint{4.285647in}{3.195533in}}%
\pgfpathlineto{\pgfqpoint{4.272307in}{3.204139in}}%
\pgfpathlineto{\pgfqpoint{4.258971in}{3.212957in}}%
\pgfpathlineto{\pgfqpoint{4.245638in}{3.221987in}}%
\pgfpathlineto{\pgfqpoint{4.237976in}{3.203030in}}%
\pgfpathlineto{\pgfqpoint{4.230313in}{3.184356in}}%
\pgfpathlineto{\pgfqpoint{4.222648in}{3.165957in}}%
\pgfpathlineto{\pgfqpoint{4.214982in}{3.147828in}}%
\pgfpathclose%
\pgfusepath{fill}%
\end{pgfscope}%
\begin{pgfscope}%
\pgfpathrectangle{\pgfqpoint{1.150000in}{0.150000in}}{\pgfqpoint{5.700000in}{5.700000in}}%
\pgfusepath{clip}%
\pgfsetbuttcap%
\pgfsetroundjoin%
\definecolor{currentfill}{rgb}{0.216210,0.351535,0.550627}%
\pgfsetfillcolor{currentfill}%
\pgfsetfillopacity{0.800000}%
\pgfsetlinewidth{0.000000pt}%
\definecolor{currentstroke}{rgb}{0.000000,0.000000,0.000000}%
\pgfsetstrokecolor{currentstroke}%
\pgfsetdash{}{0pt}%
\pgfpathmoveto{\pgfqpoint{3.718950in}{3.162948in}}%
\pgfpathlineto{\pgfqpoint{3.732256in}{3.149984in}}%
\pgfpathlineto{\pgfqpoint{3.745561in}{3.137270in}}%
\pgfpathlineto{\pgfqpoint{3.758866in}{3.124804in}}%
\pgfpathlineto{\pgfqpoint{3.772172in}{3.112585in}}%
\pgfpathlineto{\pgfqpoint{3.779917in}{3.129792in}}%
\pgfpathlineto{\pgfqpoint{3.787658in}{3.147223in}}%
\pgfpathlineto{\pgfqpoint{3.795396in}{3.164883in}}%
\pgfpathlineto{\pgfqpoint{3.803129in}{3.182777in}}%
\pgfpathlineto{\pgfqpoint{3.789829in}{3.195438in}}%
\pgfpathlineto{\pgfqpoint{3.776528in}{3.208347in}}%
\pgfpathlineto{\pgfqpoint{3.763227in}{3.221503in}}%
\pgfpathlineto{\pgfqpoint{3.749926in}{3.234910in}}%
\pgfpathlineto{\pgfqpoint{3.742188in}{3.216561in}}%
\pgfpathlineto{\pgfqpoint{3.734446in}{3.198455in}}%
\pgfpathlineto{\pgfqpoint{3.726700in}{3.180585in}}%
\pgfpathlineto{\pgfqpoint{3.718950in}{3.162948in}}%
\pgfpathclose%
\pgfusepath{fill}%
\end{pgfscope}%
\begin{pgfscope}%
\pgfpathrectangle{\pgfqpoint{1.150000in}{0.150000in}}{\pgfqpoint{5.700000in}{5.700000in}}%
\pgfusepath{clip}%
\pgfsetbuttcap%
\pgfsetroundjoin%
\definecolor{currentfill}{rgb}{0.168126,0.459988,0.558082}%
\pgfsetfillcolor{currentfill}%
\pgfsetfillopacity{0.800000}%
\pgfsetlinewidth{0.000000pt}%
\definecolor{currentstroke}{rgb}{0.000000,0.000000,0.000000}%
\pgfsetstrokecolor{currentstroke}%
\pgfsetdash{}{0pt}%
\pgfpathmoveto{\pgfqpoint{3.452438in}{3.477678in}}%
\pgfpathlineto{\pgfqpoint{3.465795in}{3.459299in}}%
\pgfpathlineto{\pgfqpoint{3.479148in}{3.441211in}}%
\pgfpathlineto{\pgfqpoint{3.492496in}{3.423413in}}%
\pgfpathlineto{\pgfqpoint{3.505840in}{3.405902in}}%
\pgfpathlineto{\pgfqpoint{3.513607in}{3.425278in}}%
\pgfpathlineto{\pgfqpoint{3.521369in}{3.444919in}}%
\pgfpathlineto{\pgfqpoint{3.529126in}{3.464830in}}%
\pgfpathlineto{\pgfqpoint{3.536878in}{3.485016in}}%
\pgfpathlineto{\pgfqpoint{3.523538in}{3.502979in}}%
\pgfpathlineto{\pgfqpoint{3.510193in}{3.521229in}}%
\pgfpathlineto{\pgfqpoint{3.496844in}{3.539769in}}%
\pgfpathlineto{\pgfqpoint{3.483489in}{3.558602in}}%
\pgfpathlineto{\pgfqpoint{3.475734in}{3.537951in}}%
\pgfpathlineto{\pgfqpoint{3.467974in}{3.517583in}}%
\pgfpathlineto{\pgfqpoint{3.460208in}{3.497494in}}%
\pgfpathlineto{\pgfqpoint{3.452438in}{3.477678in}}%
\pgfpathclose%
\pgfusepath{fill}%
\end{pgfscope}%
\begin{pgfscope}%
\pgfpathrectangle{\pgfqpoint{1.150000in}{0.150000in}}{\pgfqpoint{5.700000in}{5.700000in}}%
\pgfusepath{clip}%
\pgfsetbuttcap%
\pgfsetroundjoin%
\definecolor{currentfill}{rgb}{0.187231,0.414746,0.556547}%
\pgfsetfillcolor{currentfill}%
\pgfsetfillopacity{0.800000}%
\pgfsetlinewidth{0.000000pt}%
\definecolor{currentstroke}{rgb}{0.000000,0.000000,0.000000}%
\pgfsetstrokecolor{currentstroke}%
\pgfsetdash{}{0pt}%
\pgfpathmoveto{\pgfqpoint{4.551085in}{3.324436in}}%
\pgfpathlineto{\pgfqpoint{4.564474in}{3.316811in}}%
\pgfpathlineto{\pgfqpoint{4.577868in}{3.309384in}}%
\pgfpathlineto{\pgfqpoint{4.591268in}{3.302154in}}%
\pgfpathlineto{\pgfqpoint{4.604674in}{3.295120in}}%
\pgfpathlineto{\pgfqpoint{4.612300in}{3.315008in}}%
\pgfpathlineto{\pgfqpoint{4.619928in}{3.335243in}}%
\pgfpathlineto{\pgfqpoint{4.627558in}{3.355831in}}%
\pgfpathlineto{\pgfqpoint{4.635190in}{3.376782in}}%
\pgfpathlineto{\pgfqpoint{4.621792in}{3.384506in}}%
\pgfpathlineto{\pgfqpoint{4.608401in}{3.392426in}}%
\pgfpathlineto{\pgfqpoint{4.595015in}{3.400544in}}%
\pgfpathlineto{\pgfqpoint{4.581634in}{3.408860in}}%
\pgfpathlineto{\pgfqpoint{4.573994in}{3.387206in}}%
\pgfpathlineto{\pgfqpoint{4.566356in}{3.365923in}}%
\pgfpathlineto{\pgfqpoint{4.558720in}{3.345002in}}%
\pgfpathlineto{\pgfqpoint{4.551085in}{3.324436in}}%
\pgfpathclose%
\pgfusepath{fill}%
\end{pgfscope}%
\begin{pgfscope}%
\pgfpathrectangle{\pgfqpoint{1.150000in}{0.150000in}}{\pgfqpoint{5.700000in}{5.700000in}}%
\pgfusepath{clip}%
\pgfsetbuttcap%
\pgfsetroundjoin%
\definecolor{currentfill}{rgb}{0.223925,0.334994,0.548053}%
\pgfsetfillcolor{currentfill}%
\pgfsetfillopacity{0.800000}%
\pgfsetlinewidth{0.000000pt}%
\definecolor{currentstroke}{rgb}{0.000000,0.000000,0.000000}%
\pgfsetstrokecolor{currentstroke}%
\pgfsetdash{}{0pt}%
\pgfpathmoveto{\pgfqpoint{4.130957in}{3.111754in}}%
\pgfpathlineto{\pgfqpoint{4.144287in}{3.102961in}}%
\pgfpathlineto{\pgfqpoint{4.157619in}{3.094384in}}%
\pgfpathlineto{\pgfqpoint{4.170956in}{3.086024in}}%
\pgfpathlineto{\pgfqpoint{4.184297in}{3.077877in}}%
\pgfpathlineto{\pgfqpoint{4.191971in}{3.094993in}}%
\pgfpathlineto{\pgfqpoint{4.199643in}{3.112352in}}%
\pgfpathlineto{\pgfqpoint{4.207314in}{3.129962in}}%
\pgfpathlineto{\pgfqpoint{4.214982in}{3.147828in}}%
\pgfpathlineto{\pgfqpoint{4.201647in}{3.156507in}}%
\pgfpathlineto{\pgfqpoint{4.188317in}{3.165401in}}%
\pgfpathlineto{\pgfqpoint{4.174990in}{3.174510in}}%
\pgfpathlineto{\pgfqpoint{4.161666in}{3.183836in}}%
\pgfpathlineto{\pgfqpoint{4.153992in}{3.165425in}}%
\pgfpathlineto{\pgfqpoint{4.146316in}{3.147279in}}%
\pgfpathlineto{\pgfqpoint{4.138638in}{3.129390in}}%
\pgfpathlineto{\pgfqpoint{4.130957in}{3.111754in}}%
\pgfpathclose%
\pgfusepath{fill}%
\end{pgfscope}%
\begin{pgfscope}%
\pgfpathrectangle{\pgfqpoint{1.150000in}{0.150000in}}{\pgfqpoint{5.700000in}{5.700000in}}%
\pgfusepath{clip}%
\pgfsetbuttcap%
\pgfsetroundjoin%
\definecolor{currentfill}{rgb}{0.229739,0.322361,0.545706}%
\pgfsetfillcolor{currentfill}%
\pgfsetfillopacity{0.800000}%
\pgfsetlinewidth{0.000000pt}%
\definecolor{currentstroke}{rgb}{0.000000,0.000000,0.000000}%
\pgfsetstrokecolor{currentstroke}%
\pgfsetdash{}{0pt}%
\pgfpathmoveto{\pgfqpoint{3.909556in}{3.090176in}}%
\pgfpathlineto{\pgfqpoint{3.922864in}{3.079664in}}%
\pgfpathlineto{\pgfqpoint{3.936175in}{3.069383in}}%
\pgfpathlineto{\pgfqpoint{3.949487in}{3.059333in}}%
\pgfpathlineto{\pgfqpoint{3.962802in}{3.049511in}}%
\pgfpathlineto{\pgfqpoint{3.970517in}{3.066263in}}%
\pgfpathlineto{\pgfqpoint{3.978228in}{3.083236in}}%
\pgfpathlineto{\pgfqpoint{3.985936in}{3.100436in}}%
\pgfpathlineto{\pgfqpoint{3.993641in}{3.117869in}}%
\pgfpathlineto{\pgfqpoint{3.980332in}{3.128162in}}%
\pgfpathlineto{\pgfqpoint{3.967025in}{3.138684in}}%
\pgfpathlineto{\pgfqpoint{3.953720in}{3.149436in}}%
\pgfpathlineto{\pgfqpoint{3.940416in}{3.160420in}}%
\pgfpathlineto{\pgfqpoint{3.932706in}{3.142504in}}%
\pgfpathlineto{\pgfqpoint{3.924992in}{3.124829in}}%
\pgfpathlineto{\pgfqpoint{3.917276in}{3.107388in}}%
\pgfpathlineto{\pgfqpoint{3.909556in}{3.090176in}}%
\pgfpathclose%
\pgfusepath{fill}%
\end{pgfscope}%
\begin{pgfscope}%
\pgfpathrectangle{\pgfqpoint{1.150000in}{0.150000in}}{\pgfqpoint{5.700000in}{5.700000in}}%
\pgfusepath{clip}%
\pgfsetbuttcap%
\pgfsetroundjoin%
\definecolor{currentfill}{rgb}{0.274149,0.751988,0.436601}%
\pgfsetfillcolor{currentfill}%
\pgfsetfillopacity{0.800000}%
\pgfsetlinewidth{0.000000pt}%
\definecolor{currentstroke}{rgb}{0.000000,0.000000,0.000000}%
\pgfsetstrokecolor{currentstroke}%
\pgfsetdash{}{0pt}%
\pgfpathmoveto{\pgfqpoint{3.561251in}{4.344937in}}%
\pgfpathlineto{\pgfqpoint{3.574660in}{4.320149in}}%
\pgfpathlineto{\pgfqpoint{3.588062in}{4.295683in}}%
\pgfpathlineto{\pgfqpoint{3.601457in}{4.271537in}}%
\pgfpathlineto{\pgfqpoint{3.614844in}{4.247706in}}%
\pgfpathlineto{\pgfqpoint{3.622499in}{4.278236in}}%
\pgfpathlineto{\pgfqpoint{3.630150in}{4.309240in}}%
\pgfpathlineto{\pgfqpoint{3.637799in}{4.340726in}}%
\pgfpathlineto{\pgfqpoint{3.645446in}{4.372703in}}%
\pgfpathlineto{\pgfqpoint{3.632053in}{4.397249in}}%
\pgfpathlineto{\pgfqpoint{3.618653in}{4.422112in}}%
\pgfpathlineto{\pgfqpoint{3.605245in}{4.447296in}}%
\pgfpathlineto{\pgfqpoint{3.591830in}{4.472804in}}%
\pgfpathlineto{\pgfqpoint{3.584190in}{4.440095in}}%
\pgfpathlineto{\pgfqpoint{3.576546in}{4.407886in}}%
\pgfpathlineto{\pgfqpoint{3.568900in}{4.376170in}}%
\pgfpathlineto{\pgfqpoint{3.561251in}{4.344937in}}%
\pgfpathclose%
\pgfusepath{fill}%
\end{pgfscope}%
\begin{pgfscope}%
\pgfpathrectangle{\pgfqpoint{1.150000in}{0.150000in}}{\pgfqpoint{5.700000in}{5.700000in}}%
\pgfusepath{clip}%
\pgfsetbuttcap%
\pgfsetroundjoin%
\definecolor{currentfill}{rgb}{0.179019,0.433756,0.557430}%
\pgfsetfillcolor{currentfill}%
\pgfsetfillopacity{0.800000}%
\pgfsetlinewidth{0.000000pt}%
\definecolor{currentstroke}{rgb}{0.000000,0.000000,0.000000}%
\pgfsetstrokecolor{currentstroke}%
\pgfsetdash{}{0pt}%
\pgfpathmoveto{\pgfqpoint{4.635190in}{3.376782in}}%
\pgfpathlineto{\pgfqpoint{4.648593in}{3.369254in}}%
\pgfpathlineto{\pgfqpoint{4.662003in}{3.361922in}}%
\pgfpathlineto{\pgfqpoint{4.675418in}{3.354783in}}%
\pgfpathlineto{\pgfqpoint{4.688840in}{3.347838in}}%
\pgfpathlineto{\pgfqpoint{4.696465in}{3.368450in}}%
\pgfpathlineto{\pgfqpoint{4.704093in}{3.389433in}}%
\pgfpathlineto{\pgfqpoint{4.711724in}{3.410795in}}%
\pgfpathlineto{\pgfqpoint{4.719358in}{3.432546in}}%
\pgfpathlineto{\pgfqpoint{4.705945in}{3.440213in}}%
\pgfpathlineto{\pgfqpoint{4.692538in}{3.448073in}}%
\pgfpathlineto{\pgfqpoint{4.679137in}{3.456128in}}%
\pgfpathlineto{\pgfqpoint{4.665743in}{3.464379in}}%
\pgfpathlineto{\pgfqpoint{4.658100in}{3.441894in}}%
\pgfpathlineto{\pgfqpoint{4.650460in}{3.419805in}}%
\pgfpathlineto{\pgfqpoint{4.642824in}{3.398104in}}%
\pgfpathlineto{\pgfqpoint{4.635190in}{3.376782in}}%
\pgfpathclose%
\pgfusepath{fill}%
\end{pgfscope}%
\begin{pgfscope}%
\pgfpathrectangle{\pgfqpoint{1.150000in}{0.150000in}}{\pgfqpoint{5.700000in}{5.700000in}}%
\pgfusepath{clip}%
\pgfsetbuttcap%
\pgfsetroundjoin%
\definecolor{currentfill}{rgb}{0.135066,0.544853,0.554029}%
\pgfsetfillcolor{currentfill}%
\pgfsetfillopacity{0.800000}%
\pgfsetlinewidth{0.000000pt}%
\definecolor{currentstroke}{rgb}{0.000000,0.000000,0.000000}%
\pgfsetstrokecolor{currentstroke}%
\pgfsetdash{}{0pt}%
\pgfpathmoveto{\pgfqpoint{3.376455in}{3.720125in}}%
\pgfpathlineto{\pgfqpoint{3.389856in}{3.698853in}}%
\pgfpathlineto{\pgfqpoint{3.403251in}{3.677896in}}%
\pgfpathlineto{\pgfqpoint{3.416639in}{3.657251in}}%
\pgfpathlineto{\pgfqpoint{3.430020in}{3.636915in}}%
\pgfpathlineto{\pgfqpoint{3.437773in}{3.658333in}}%
\pgfpathlineto{\pgfqpoint{3.445521in}{3.680055in}}%
\pgfpathlineto{\pgfqpoint{3.453263in}{3.702087in}}%
\pgfpathlineto{\pgfqpoint{3.461001in}{3.724434in}}%
\pgfpathlineto{\pgfqpoint{3.447621in}{3.745263in}}%
\pgfpathlineto{\pgfqpoint{3.434235in}{3.766400in}}%
\pgfpathlineto{\pgfqpoint{3.420842in}{3.787851in}}%
\pgfpathlineto{\pgfqpoint{3.407443in}{3.809617in}}%
\pgfpathlineto{\pgfqpoint{3.399704in}{3.786763in}}%
\pgfpathlineto{\pgfqpoint{3.391959in}{3.764234in}}%
\pgfpathlineto{\pgfqpoint{3.384210in}{3.742023in}}%
\pgfpathlineto{\pgfqpoint{3.376455in}{3.720125in}}%
\pgfpathclose%
\pgfusepath{fill}%
\end{pgfscope}%
\begin{pgfscope}%
\pgfpathrectangle{\pgfqpoint{1.150000in}{0.150000in}}{\pgfqpoint{5.700000in}{5.700000in}}%
\pgfusepath{clip}%
\pgfsetbuttcap%
\pgfsetroundjoin%
\definecolor{currentfill}{rgb}{0.225863,0.330805,0.547314}%
\pgfsetfillcolor{currentfill}%
\pgfsetfillopacity{0.800000}%
\pgfsetlinewidth{0.000000pt}%
\definecolor{currentstroke}{rgb}{0.000000,0.000000,0.000000}%
\pgfsetstrokecolor{currentstroke}%
\pgfsetdash{}{0pt}%
\pgfpathmoveto{\pgfqpoint{3.772172in}{3.112585in}}%
\pgfpathlineto{\pgfqpoint{3.785477in}{3.100611in}}%
\pgfpathlineto{\pgfqpoint{3.798784in}{3.088879in}}%
\pgfpathlineto{\pgfqpoint{3.812090in}{3.077390in}}%
\pgfpathlineto{\pgfqpoint{3.825398in}{3.066140in}}%
\pgfpathlineto{\pgfqpoint{3.833138in}{3.082918in}}%
\pgfpathlineto{\pgfqpoint{3.840874in}{3.099911in}}%
\pgfpathlineto{\pgfqpoint{3.848606in}{3.117126in}}%
\pgfpathlineto{\pgfqpoint{3.856335in}{3.134566in}}%
\pgfpathlineto{\pgfqpoint{3.843033in}{3.146257in}}%
\pgfpathlineto{\pgfqpoint{3.829731in}{3.158188in}}%
\pgfpathlineto{\pgfqpoint{3.816430in}{3.170361in}}%
\pgfpathlineto{\pgfqpoint{3.803129in}{3.182777in}}%
\pgfpathlineto{\pgfqpoint{3.795396in}{3.164883in}}%
\pgfpathlineto{\pgfqpoint{3.787658in}{3.147223in}}%
\pgfpathlineto{\pgfqpoint{3.779917in}{3.129792in}}%
\pgfpathlineto{\pgfqpoint{3.772172in}{3.112585in}}%
\pgfpathclose%
\pgfusepath{fill}%
\end{pgfscope}%
\begin{pgfscope}%
\pgfpathrectangle{\pgfqpoint{1.150000in}{0.150000in}}{\pgfqpoint{5.700000in}{5.700000in}}%
\pgfusepath{clip}%
\pgfsetbuttcap%
\pgfsetroundjoin%
\definecolor{currentfill}{rgb}{0.146616,0.673050,0.508936}%
\pgfsetfillcolor{currentfill}%
\pgfsetfillopacity{0.800000}%
\pgfsetlinewidth{0.000000pt}%
\definecolor{currentstroke}{rgb}{0.000000,0.000000,0.000000}%
\pgfsetstrokecolor{currentstroke}%
\pgfsetdash{}{0pt}%
\pgfpathmoveto{\pgfqpoint{3.415501in}{4.099565in}}%
\pgfpathlineto{\pgfqpoint{3.428931in}{4.075401in}}%
\pgfpathlineto{\pgfqpoint{3.442354in}{4.051567in}}%
\pgfpathlineto{\pgfqpoint{3.455768in}{4.028060in}}%
\pgfpathlineto{\pgfqpoint{3.469176in}{4.004877in}}%
\pgfpathlineto{\pgfqpoint{3.476871in}{4.030936in}}%
\pgfpathlineto{\pgfqpoint{3.484562in}{4.057385in}}%
\pgfpathlineto{\pgfqpoint{3.492249in}{4.084231in}}%
\pgfpathlineto{\pgfqpoint{3.499931in}{4.111480in}}%
\pgfpathlineto{\pgfqpoint{3.486522in}{4.135268in}}%
\pgfpathlineto{\pgfqpoint{3.473105in}{4.159381in}}%
\pgfpathlineto{\pgfqpoint{3.459681in}{4.183822in}}%
\pgfpathlineto{\pgfqpoint{3.446248in}{4.208595in}}%
\pgfpathlineto{\pgfqpoint{3.438568in}{4.180724in}}%
\pgfpathlineto{\pgfqpoint{3.430884in}{4.153267in}}%
\pgfpathlineto{\pgfqpoint{3.423195in}{4.126217in}}%
\pgfpathlineto{\pgfqpoint{3.415501in}{4.099565in}}%
\pgfpathclose%
\pgfusepath{fill}%
\end{pgfscope}%
\begin{pgfscope}%
\pgfpathrectangle{\pgfqpoint{1.150000in}{0.150000in}}{\pgfqpoint{5.700000in}{5.700000in}}%
\pgfusepath{clip}%
\pgfsetbuttcap%
\pgfsetroundjoin%
\definecolor{currentfill}{rgb}{0.157729,0.485932,0.558013}%
\pgfsetfillcolor{currentfill}%
\pgfsetfillopacity{0.800000}%
\pgfsetlinewidth{0.000000pt}%
\definecolor{currentstroke}{rgb}{0.000000,0.000000,0.000000}%
\pgfsetstrokecolor{currentstroke}%
\pgfsetdash{}{0pt}%
\pgfpathmoveto{\pgfqpoint{3.398957in}{3.554170in}}%
\pgfpathlineto{\pgfqpoint{3.412335in}{3.534596in}}%
\pgfpathlineto{\pgfqpoint{3.425708in}{3.515325in}}%
\pgfpathlineto{\pgfqpoint{3.439076in}{3.496353in}}%
\pgfpathlineto{\pgfqpoint{3.452438in}{3.477678in}}%
\pgfpathlineto{\pgfqpoint{3.460208in}{3.497494in}}%
\pgfpathlineto{\pgfqpoint{3.467974in}{3.517583in}}%
\pgfpathlineto{\pgfqpoint{3.475734in}{3.537951in}}%
\pgfpathlineto{\pgfqpoint{3.483489in}{3.558602in}}%
\pgfpathlineto{\pgfqpoint{3.470130in}{3.577731in}}%
\pgfpathlineto{\pgfqpoint{3.456766in}{3.597157in}}%
\pgfpathlineto{\pgfqpoint{3.443396in}{3.616884in}}%
\pgfpathlineto{\pgfqpoint{3.430020in}{3.636915in}}%
\pgfpathlineto{\pgfqpoint{3.422262in}{3.615795in}}%
\pgfpathlineto{\pgfqpoint{3.414499in}{3.594968in}}%
\pgfpathlineto{\pgfqpoint{3.406731in}{3.574428in}}%
\pgfpathlineto{\pgfqpoint{3.398957in}{3.554170in}}%
\pgfpathclose%
\pgfusepath{fill}%
\end{pgfscope}%
\begin{pgfscope}%
\pgfpathrectangle{\pgfqpoint{1.150000in}{0.150000in}}{\pgfqpoint{5.700000in}{5.700000in}}%
\pgfusepath{clip}%
\pgfsetbuttcap%
\pgfsetroundjoin%
\definecolor{currentfill}{rgb}{0.123444,0.636809,0.528763}%
\pgfsetfillcolor{currentfill}%
\pgfsetfillopacity{0.800000}%
\pgfsetlinewidth{0.000000pt}%
\definecolor{currentstroke}{rgb}{0.000000,0.000000,0.000000}%
\pgfsetstrokecolor{currentstroke}%
\pgfsetdash{}{0pt}%
\pgfpathmoveto{\pgfqpoint{3.384677in}{3.996810in}}%
\pgfpathlineto{\pgfqpoint{3.398106in}{3.973217in}}%
\pgfpathlineto{\pgfqpoint{3.411528in}{3.949954in}}%
\pgfpathlineto{\pgfqpoint{3.424942in}{3.927015in}}%
\pgfpathlineto{\pgfqpoint{3.438348in}{3.904400in}}%
\pgfpathlineto{\pgfqpoint{3.446062in}{3.928969in}}%
\pgfpathlineto{\pgfqpoint{3.453771in}{3.953900in}}%
\pgfpathlineto{\pgfqpoint{3.461476in}{3.979201in}}%
\pgfpathlineto{\pgfqpoint{3.469176in}{4.004877in}}%
\pgfpathlineto{\pgfqpoint{3.455768in}{4.028060in}}%
\pgfpathlineto{\pgfqpoint{3.442354in}{4.051567in}}%
\pgfpathlineto{\pgfqpoint{3.428931in}{4.075401in}}%
\pgfpathlineto{\pgfqpoint{3.415501in}{4.099565in}}%
\pgfpathlineto{\pgfqpoint{3.407802in}{4.073305in}}%
\pgfpathlineto{\pgfqpoint{3.400099in}{4.047431in}}%
\pgfpathlineto{\pgfqpoint{3.392390in}{4.021935in}}%
\pgfpathlineto{\pgfqpoint{3.384677in}{3.996810in}}%
\pgfpathclose%
\pgfusepath{fill}%
\end{pgfscope}%
\begin{pgfscope}%
\pgfpathrectangle{\pgfqpoint{1.150000in}{0.150000in}}{\pgfqpoint{5.700000in}{5.700000in}}%
\pgfusepath{clip}%
\pgfsetbuttcap%
\pgfsetroundjoin%
\definecolor{currentfill}{rgb}{0.229739,0.322361,0.545706}%
\pgfsetfillcolor{currentfill}%
\pgfsetfillopacity{0.800000}%
\pgfsetlinewidth{0.000000pt}%
\definecolor{currentstroke}{rgb}{0.000000,0.000000,0.000000}%
\pgfsetstrokecolor{currentstroke}%
\pgfsetdash{}{0pt}%
\pgfpathmoveto{\pgfqpoint{4.046902in}{3.078957in}}%
\pgfpathlineto{\pgfqpoint{4.060225in}{3.069789in}}%
\pgfpathlineto{\pgfqpoint{4.073550in}{3.060841in}}%
\pgfpathlineto{\pgfqpoint{4.086878in}{3.052113in}}%
\pgfpathlineto{\pgfqpoint{4.100210in}{3.043604in}}%
\pgfpathlineto{\pgfqpoint{4.107901in}{3.060294in}}%
\pgfpathlineto{\pgfqpoint{4.115589in}{3.077211in}}%
\pgfpathlineto{\pgfqpoint{4.123274in}{3.094363in}}%
\pgfpathlineto{\pgfqpoint{4.130957in}{3.111754in}}%
\pgfpathlineto{\pgfqpoint{4.117631in}{3.120764in}}%
\pgfpathlineto{\pgfqpoint{4.104309in}{3.129994in}}%
\pgfpathlineto{\pgfqpoint{4.090989in}{3.139443in}}%
\pgfpathlineto{\pgfqpoint{4.077673in}{3.149114in}}%
\pgfpathlineto{\pgfqpoint{4.069984in}{3.131209in}}%
\pgfpathlineto{\pgfqpoint{4.062293in}{3.113552in}}%
\pgfpathlineto{\pgfqpoint{4.054599in}{3.096137in}}%
\pgfpathlineto{\pgfqpoint{4.046902in}{3.078957in}}%
\pgfpathclose%
\pgfusepath{fill}%
\end{pgfscope}%
\begin{pgfscope}%
\pgfpathrectangle{\pgfqpoint{1.150000in}{0.150000in}}{\pgfqpoint{5.700000in}{5.700000in}}%
\pgfusepath{clip}%
\pgfsetbuttcap%
\pgfsetroundjoin%
\definecolor{currentfill}{rgb}{0.421908,0.805774,0.351910}%
\pgfsetfillcolor{currentfill}%
\pgfsetfillopacity{0.800000}%
\pgfsetlinewidth{0.000000pt}%
\definecolor{currentstroke}{rgb}{0.000000,0.000000,0.000000}%
\pgfsetstrokecolor{currentstroke}%
\pgfsetdash{}{0pt}%
\pgfpathmoveto{\pgfqpoint{3.760110in}{4.546023in}}%
\pgfpathlineto{\pgfqpoint{3.773485in}{4.521486in}}%
\pgfpathlineto{\pgfqpoint{3.786853in}{4.497252in}}%
\pgfpathlineto{\pgfqpoint{3.800216in}{4.473320in}}%
\pgfpathlineto{\pgfqpoint{3.813573in}{4.449687in}}%
\pgfpathlineto{\pgfqpoint{3.821221in}{4.484863in}}%
\pgfpathlineto{\pgfqpoint{3.828868in}{4.520602in}}%
\pgfpathlineto{\pgfqpoint{3.836516in}{4.556916in}}%
\pgfpathlineto{\pgfqpoint{3.823153in}{4.581164in}}%
\pgfpathlineto{\pgfqpoint{3.809784in}{4.605712in}}%
\pgfpathlineto{\pgfqpoint{3.796410in}{4.630562in}}%
\pgfpathlineto{\pgfqpoint{3.783029in}{4.655718in}}%
\pgfpathlineto{\pgfqpoint{3.775390in}{4.618571in}}%
\pgfpathlineto{\pgfqpoint{3.767750in}{4.582010in}}%
\pgfpathlineto{\pgfqpoint{3.760110in}{4.546023in}}%
\pgfpathclose%
\pgfusepath{fill}%
\end{pgfscope}%
\begin{pgfscope}%
\pgfpathrectangle{\pgfqpoint{1.150000in}{0.150000in}}{\pgfqpoint{5.700000in}{5.700000in}}%
\pgfusepath{clip}%
\pgfsetbuttcap%
\pgfsetroundjoin%
\definecolor{currentfill}{rgb}{0.191090,0.708366,0.482284}%
\pgfsetfillcolor{currentfill}%
\pgfsetfillopacity{0.800000}%
\pgfsetlinewidth{0.000000pt}%
\definecolor{currentstroke}{rgb}{0.000000,0.000000,0.000000}%
\pgfsetstrokecolor{currentstroke}%
\pgfsetdash{}{0pt}%
\pgfpathmoveto{\pgfqpoint{3.446248in}{4.208595in}}%
\pgfpathlineto{\pgfqpoint{3.459681in}{4.183822in}}%
\pgfpathlineto{\pgfqpoint{3.473105in}{4.159381in}}%
\pgfpathlineto{\pgfqpoint{3.486522in}{4.135268in}}%
\pgfpathlineto{\pgfqpoint{3.499931in}{4.111480in}}%
\pgfpathlineto{\pgfqpoint{3.507609in}{4.139142in}}%
\pgfpathlineto{\pgfqpoint{3.515284in}{4.167222in}}%
\pgfpathlineto{\pgfqpoint{3.522954in}{4.195729in}}%
\pgfpathlineto{\pgfqpoint{3.530620in}{4.224670in}}%
\pgfpathlineto{\pgfqpoint{3.517208in}{4.249100in}}%
\pgfpathlineto{\pgfqpoint{3.503788in}{4.273858in}}%
\pgfpathlineto{\pgfqpoint{3.490360in}{4.298945in}}%
\pgfpathlineto{\pgfqpoint{3.476923in}{4.324365in}}%
\pgfpathlineto{\pgfqpoint{3.469261in}{4.294764in}}%
\pgfpathlineto{\pgfqpoint{3.461594in}{4.265607in}}%
\pgfpathlineto{\pgfqpoint{3.453923in}{4.236887in}}%
\pgfpathlineto{\pgfqpoint{3.446248in}{4.208595in}}%
\pgfpathclose%
\pgfusepath{fill}%
\end{pgfscope}%
\begin{pgfscope}%
\pgfpathrectangle{\pgfqpoint{1.150000in}{0.150000in}}{\pgfqpoint{5.700000in}{5.700000in}}%
\pgfusepath{clip}%
\pgfsetbuttcap%
\pgfsetroundjoin%
\definecolor{currentfill}{rgb}{0.199430,0.387607,0.554642}%
\pgfsetfillcolor{currentfill}%
\pgfsetfillopacity{0.800000}%
\pgfsetlinewidth{0.000000pt}%
\definecolor{currentstroke}{rgb}{0.000000,0.000000,0.000000}%
\pgfsetstrokecolor{currentstroke}%
\pgfsetdash{}{0pt}%
\pgfpathmoveto{\pgfqpoint{3.528077in}{3.265392in}}%
\pgfpathlineto{\pgfqpoint{3.541408in}{3.249697in}}%
\pgfpathlineto{\pgfqpoint{3.554735in}{3.234275in}}%
\pgfpathlineto{\pgfqpoint{3.568061in}{3.219124in}}%
\pgfpathlineto{\pgfqpoint{3.581383in}{3.204241in}}%
\pgfpathlineto{\pgfqpoint{3.589161in}{3.221803in}}%
\pgfpathlineto{\pgfqpoint{3.596934in}{3.239592in}}%
\pgfpathlineto{\pgfqpoint{3.604702in}{3.257614in}}%
\pgfpathlineto{\pgfqpoint{3.612466in}{3.275874in}}%
\pgfpathlineto{\pgfqpoint{3.599148in}{3.291170in}}%
\pgfpathlineto{\pgfqpoint{3.585827in}{3.306736in}}%
\pgfpathlineto{\pgfqpoint{3.572504in}{3.322573in}}%
\pgfpathlineto{\pgfqpoint{3.559177in}{3.338683in}}%
\pgfpathlineto{\pgfqpoint{3.551410in}{3.319996in}}%
\pgfpathlineto{\pgfqpoint{3.543637in}{3.301555in}}%
\pgfpathlineto{\pgfqpoint{3.535859in}{3.283355in}}%
\pgfpathlineto{\pgfqpoint{3.528077in}{3.265392in}}%
\pgfpathclose%
\pgfusepath{fill}%
\end{pgfscope}%
\begin{pgfscope}%
\pgfpathrectangle{\pgfqpoint{1.150000in}{0.150000in}}{\pgfqpoint{5.700000in}{5.700000in}}%
\pgfusepath{clip}%
\pgfsetbuttcap%
\pgfsetroundjoin%
\definecolor{currentfill}{rgb}{0.208623,0.367752,0.552675}%
\pgfsetfillcolor{currentfill}%
\pgfsetfillopacity{0.800000}%
\pgfsetlinewidth{0.000000pt}%
\definecolor{currentstroke}{rgb}{0.000000,0.000000,0.000000}%
\pgfsetstrokecolor{currentstroke}%
\pgfsetdash{}{0pt}%
\pgfpathmoveto{\pgfqpoint{3.581383in}{3.204241in}}%
\pgfpathlineto{\pgfqpoint{3.594704in}{3.189626in}}%
\pgfpathlineto{\pgfqpoint{3.608023in}{3.175274in}}%
\pgfpathlineto{\pgfqpoint{3.621340in}{3.161186in}}%
\pgfpathlineto{\pgfqpoint{3.634655in}{3.147358in}}%
\pgfpathlineto{\pgfqpoint{3.642428in}{3.164518in}}%
\pgfpathlineto{\pgfqpoint{3.650196in}{3.181898in}}%
\pgfpathlineto{\pgfqpoint{3.657960in}{3.199503in}}%
\pgfpathlineto{\pgfqpoint{3.665719in}{3.217338in}}%
\pgfpathlineto{\pgfqpoint{3.652408in}{3.231578in}}%
\pgfpathlineto{\pgfqpoint{3.639096in}{3.246080in}}%
\pgfpathlineto{\pgfqpoint{3.625782in}{3.260844in}}%
\pgfpathlineto{\pgfqpoint{3.612466in}{3.275874in}}%
\pgfpathlineto{\pgfqpoint{3.604702in}{3.257614in}}%
\pgfpathlineto{\pgfqpoint{3.596934in}{3.239592in}}%
\pgfpathlineto{\pgfqpoint{3.589161in}{3.221803in}}%
\pgfpathlineto{\pgfqpoint{3.581383in}{3.204241in}}%
\pgfpathclose%
\pgfusepath{fill}%
\end{pgfscope}%
\begin{pgfscope}%
\pgfpathrectangle{\pgfqpoint{1.150000in}{0.150000in}}{\pgfqpoint{5.700000in}{5.700000in}}%
\pgfusepath{clip}%
\pgfsetbuttcap%
\pgfsetroundjoin%
\definecolor{currentfill}{rgb}{0.188923,0.410910,0.556326}%
\pgfsetfillcolor{currentfill}%
\pgfsetfillopacity{0.800000}%
\pgfsetlinewidth{0.000000pt}%
\definecolor{currentstroke}{rgb}{0.000000,0.000000,0.000000}%
\pgfsetstrokecolor{currentstroke}%
\pgfsetdash{}{0pt}%
\pgfpathmoveto{\pgfqpoint{3.474722in}{3.330944in}}%
\pgfpathlineto{\pgfqpoint{3.488066in}{3.314136in}}%
\pgfpathlineto{\pgfqpoint{3.501406in}{3.297609in}}%
\pgfpathlineto{\pgfqpoint{3.514743in}{3.281362in}}%
\pgfpathlineto{\pgfqpoint{3.528077in}{3.265392in}}%
\pgfpathlineto{\pgfqpoint{3.535859in}{3.283355in}}%
\pgfpathlineto{\pgfqpoint{3.543637in}{3.301555in}}%
\pgfpathlineto{\pgfqpoint{3.551410in}{3.319996in}}%
\pgfpathlineto{\pgfqpoint{3.559177in}{3.338683in}}%
\pgfpathlineto{\pgfqpoint{3.545848in}{3.355069in}}%
\pgfpathlineto{\pgfqpoint{3.532516in}{3.371732in}}%
\pgfpathlineto{\pgfqpoint{3.519180in}{3.388676in}}%
\pgfpathlineto{\pgfqpoint{3.505840in}{3.405902in}}%
\pgfpathlineto{\pgfqpoint{3.498068in}{3.386786in}}%
\pgfpathlineto{\pgfqpoint{3.490291in}{3.367924in}}%
\pgfpathlineto{\pgfqpoint{3.482509in}{3.349312in}}%
\pgfpathlineto{\pgfqpoint{3.474722in}{3.330944in}}%
\pgfpathclose%
\pgfusepath{fill}%
\end{pgfscope}%
\begin{pgfscope}%
\pgfpathrectangle{\pgfqpoint{1.150000in}{0.150000in}}{\pgfqpoint{5.700000in}{5.700000in}}%
\pgfusepath{clip}%
\pgfsetbuttcap%
\pgfsetroundjoin%
\definecolor{currentfill}{rgb}{0.119738,0.603785,0.541400}%
\pgfsetfillcolor{currentfill}%
\pgfsetfillopacity{0.800000}%
\pgfsetlinewidth{0.000000pt}%
\definecolor{currentstroke}{rgb}{0.000000,0.000000,0.000000}%
\pgfsetstrokecolor{currentstroke}%
\pgfsetdash{}{0pt}%
\pgfpathmoveto{\pgfqpoint{3.353770in}{3.899896in}}%
\pgfpathlineto{\pgfqpoint{3.367200in}{3.876838in}}%
\pgfpathlineto{\pgfqpoint{3.380622in}{3.854107in}}%
\pgfpathlineto{\pgfqpoint{3.394036in}{3.831701in}}%
\pgfpathlineto{\pgfqpoint{3.407443in}{3.809617in}}%
\pgfpathlineto{\pgfqpoint{3.415177in}{3.832801in}}%
\pgfpathlineto{\pgfqpoint{3.422905in}{3.856322in}}%
\pgfpathlineto{\pgfqpoint{3.430629in}{3.880186in}}%
\pgfpathlineto{\pgfqpoint{3.438348in}{3.904400in}}%
\pgfpathlineto{\pgfqpoint{3.424942in}{3.927015in}}%
\pgfpathlineto{\pgfqpoint{3.411528in}{3.949954in}}%
\pgfpathlineto{\pgfqpoint{3.398106in}{3.973217in}}%
\pgfpathlineto{\pgfqpoint{3.384677in}{3.996810in}}%
\pgfpathlineto{\pgfqpoint{3.376958in}{3.972050in}}%
\pgfpathlineto{\pgfqpoint{3.369234in}{3.947649in}}%
\pgfpathlineto{\pgfqpoint{3.361505in}{3.923600in}}%
\pgfpathlineto{\pgfqpoint{3.353770in}{3.899896in}}%
\pgfpathclose%
\pgfusepath{fill}%
\end{pgfscope}%
\begin{pgfscope}%
\pgfpathrectangle{\pgfqpoint{1.150000in}{0.150000in}}{\pgfqpoint{5.700000in}{5.700000in}}%
\pgfusepath{clip}%
\pgfsetbuttcap%
\pgfsetroundjoin%
\definecolor{currentfill}{rgb}{0.404001,0.800275,0.362552}%
\pgfsetfillcolor{currentfill}%
\pgfsetfillopacity{0.800000}%
\pgfsetlinewidth{0.000000pt}%
\definecolor{currentstroke}{rgb}{0.000000,0.000000,0.000000}%
\pgfsetstrokecolor{currentstroke}%
\pgfsetdash{}{0pt}%
\pgfpathmoveto{\pgfqpoint{3.676010in}{4.505694in}}%
\pgfpathlineto{\pgfqpoint{3.689403in}{4.480708in}}%
\pgfpathlineto{\pgfqpoint{3.702789in}{4.456036in}}%
\pgfpathlineto{\pgfqpoint{3.716169in}{4.431675in}}%
\pgfpathlineto{\pgfqpoint{3.729542in}{4.407621in}}%
\pgfpathlineto{\pgfqpoint{3.737186in}{4.441409in}}%
\pgfpathlineto{\pgfqpoint{3.744828in}{4.475732in}}%
\pgfpathlineto{\pgfqpoint{3.752470in}{4.510600in}}%
\pgfpathlineto{\pgfqpoint{3.760110in}{4.546023in}}%
\pgfpathlineto{\pgfqpoint{3.746730in}{4.570866in}}%
\pgfpathlineto{\pgfqpoint{3.733342in}{4.596020in}}%
\pgfpathlineto{\pgfqpoint{3.719948in}{4.621486in}}%
\pgfpathlineto{\pgfqpoint{3.706547in}{4.647267in}}%
\pgfpathlineto{\pgfqpoint{3.698915in}{4.611035in}}%
\pgfpathlineto{\pgfqpoint{3.691282in}{4.575369in}}%
\pgfpathlineto{\pgfqpoint{3.683647in}{4.540259in}}%
\pgfpathlineto{\pgfqpoint{3.676010in}{4.505694in}}%
\pgfpathclose%
\pgfusepath{fill}%
\end{pgfscope}%
\begin{pgfscope}%
\pgfpathrectangle{\pgfqpoint{1.150000in}{0.150000in}}{\pgfqpoint{5.700000in}{5.700000in}}%
\pgfusepath{clip}%
\pgfsetbuttcap%
\pgfsetroundjoin%
\definecolor{currentfill}{rgb}{0.206756,0.371758,0.553117}%
\pgfsetfillcolor{currentfill}%
\pgfsetfillopacity{0.800000}%
\pgfsetlinewidth{0.000000pt}%
\definecolor{currentstroke}{rgb}{0.000000,0.000000,0.000000}%
\pgfsetstrokecolor{currentstroke}%
\pgfsetdash{}{0pt}%
\pgfpathmoveto{\pgfqpoint{4.436477in}{3.199073in}}%
\pgfpathlineto{\pgfqpoint{4.449859in}{3.191932in}}%
\pgfpathlineto{\pgfqpoint{4.463246in}{3.184993in}}%
\pgfpathlineto{\pgfqpoint{4.476640in}{3.178254in}}%
\pgfpathlineto{\pgfqpoint{4.490039in}{3.171714in}}%
\pgfpathlineto{\pgfqpoint{4.497669in}{3.189727in}}%
\pgfpathlineto{\pgfqpoint{4.505298in}{3.208034in}}%
\pgfpathlineto{\pgfqpoint{4.512928in}{3.226640in}}%
\pgfpathlineto{\pgfqpoint{4.520558in}{3.245554in}}%
\pgfpathlineto{\pgfqpoint{4.507167in}{3.252720in}}%
\pgfpathlineto{\pgfqpoint{4.493782in}{3.260085in}}%
\pgfpathlineto{\pgfqpoint{4.480402in}{3.267651in}}%
\pgfpathlineto{\pgfqpoint{4.467028in}{3.275418in}}%
\pgfpathlineto{\pgfqpoint{4.459390in}{3.255866in}}%
\pgfpathlineto{\pgfqpoint{4.451753in}{3.236629in}}%
\pgfpathlineto{\pgfqpoint{4.444115in}{3.217701in}}%
\pgfpathlineto{\pgfqpoint{4.436477in}{3.199073in}}%
\pgfpathclose%
\pgfusepath{fill}%
\end{pgfscope}%
\begin{pgfscope}%
\pgfpathrectangle{\pgfqpoint{1.150000in}{0.150000in}}{\pgfqpoint{5.700000in}{5.700000in}}%
\pgfusepath{clip}%
\pgfsetbuttcap%
\pgfsetroundjoin%
\definecolor{currentfill}{rgb}{0.214298,0.355619,0.551184}%
\pgfsetfillcolor{currentfill}%
\pgfsetfillopacity{0.800000}%
\pgfsetlinewidth{0.000000pt}%
\definecolor{currentstroke}{rgb}{0.000000,0.000000,0.000000}%
\pgfsetstrokecolor{currentstroke}%
\pgfsetdash{}{0pt}%
\pgfpathmoveto{\pgfqpoint{4.352416in}{3.155638in}}%
\pgfpathlineto{\pgfqpoint{4.365784in}{3.148279in}}%
\pgfpathlineto{\pgfqpoint{4.379158in}{3.141125in}}%
\pgfpathlineto{\pgfqpoint{4.392536in}{3.134174in}}%
\pgfpathlineto{\pgfqpoint{4.405921in}{3.127426in}}%
\pgfpathlineto{\pgfqpoint{4.413561in}{3.144922in}}%
\pgfpathlineto{\pgfqpoint{4.421200in}{3.162691in}}%
\pgfpathlineto{\pgfqpoint{4.428839in}{3.180739in}}%
\pgfpathlineto{\pgfqpoint{4.436477in}{3.199073in}}%
\pgfpathlineto{\pgfqpoint{4.423100in}{3.206415in}}%
\pgfpathlineto{\pgfqpoint{4.409729in}{3.213961in}}%
\pgfpathlineto{\pgfqpoint{4.396363in}{3.221710in}}%
\pgfpathlineto{\pgfqpoint{4.383002in}{3.229664in}}%
\pgfpathlineto{\pgfqpoint{4.375357in}{3.210723in}}%
\pgfpathlineto{\pgfqpoint{4.367710in}{3.192076in}}%
\pgfpathlineto{\pgfqpoint{4.360064in}{3.173717in}}%
\pgfpathlineto{\pgfqpoint{4.352416in}{3.155638in}}%
\pgfpathclose%
\pgfusepath{fill}%
\end{pgfscope}%
\begin{pgfscope}%
\pgfpathrectangle{\pgfqpoint{1.150000in}{0.150000in}}{\pgfqpoint{5.700000in}{5.700000in}}%
\pgfusepath{clip}%
\pgfsetbuttcap%
\pgfsetroundjoin%
\definecolor{currentfill}{rgb}{0.171176,0.452530,0.557965}%
\pgfsetfillcolor{currentfill}%
\pgfsetfillopacity{0.800000}%
\pgfsetlinewidth{0.000000pt}%
\definecolor{currentstroke}{rgb}{0.000000,0.000000,0.000000}%
\pgfsetstrokecolor{currentstroke}%
\pgfsetdash{}{0pt}%
\pgfpathmoveto{\pgfqpoint{4.719358in}{3.432546in}}%
\pgfpathlineto{\pgfqpoint{4.732777in}{3.425072in}}%
\pgfpathlineto{\pgfqpoint{4.746202in}{3.417790in}}%
\pgfpathlineto{\pgfqpoint{4.759634in}{3.410700in}}%
\pgfpathlineto{\pgfqpoint{4.773072in}{3.403801in}}%
\pgfpathlineto{\pgfqpoint{4.780700in}{3.425207in}}%
\pgfpathlineto{\pgfqpoint{4.788331in}{3.447010in}}%
\pgfpathlineto{\pgfqpoint{4.795967in}{3.469219in}}%
\pgfpathlineto{\pgfqpoint{4.782536in}{3.476681in}}%
\pgfpathlineto{\pgfqpoint{4.769111in}{3.484334in}}%
\pgfpathlineto{\pgfqpoint{4.755693in}{3.492178in}}%
\pgfpathlineto{\pgfqpoint{4.742281in}{3.500215in}}%
\pgfpathlineto{\pgfqpoint{4.734636in}{3.477247in}}%
\pgfpathlineto{\pgfqpoint{4.726995in}{3.454693in}}%
\pgfpathlineto{\pgfqpoint{4.719358in}{3.432546in}}%
\pgfpathclose%
\pgfusepath{fill}%
\end{pgfscope}%
\begin{pgfscope}%
\pgfpathrectangle{\pgfqpoint{1.150000in}{0.150000in}}{\pgfqpoint{5.700000in}{5.700000in}}%
\pgfusepath{clip}%
\pgfsetbuttcap%
\pgfsetroundjoin%
\definecolor{currentfill}{rgb}{0.218130,0.347432,0.550038}%
\pgfsetfillcolor{currentfill}%
\pgfsetfillopacity{0.800000}%
\pgfsetlinewidth{0.000000pt}%
\definecolor{currentstroke}{rgb}{0.000000,0.000000,0.000000}%
\pgfsetstrokecolor{currentstroke}%
\pgfsetdash{}{0pt}%
\pgfpathmoveto{\pgfqpoint{3.634655in}{3.147358in}}%
\pgfpathlineto{\pgfqpoint{3.647970in}{3.133788in}}%
\pgfpathlineto{\pgfqpoint{3.661283in}{3.120476in}}%
\pgfpathlineto{\pgfqpoint{3.674595in}{3.107418in}}%
\pgfpathlineto{\pgfqpoint{3.687906in}{3.094614in}}%
\pgfpathlineto{\pgfqpoint{3.695674in}{3.111375in}}%
\pgfpathlineto{\pgfqpoint{3.703437in}{3.128347in}}%
\pgfpathlineto{\pgfqpoint{3.711195in}{3.145537in}}%
\pgfpathlineto{\pgfqpoint{3.718950in}{3.162948in}}%
\pgfpathlineto{\pgfqpoint{3.705643in}{3.176163in}}%
\pgfpathlineto{\pgfqpoint{3.692336in}{3.189632in}}%
\pgfpathlineto{\pgfqpoint{3.679028in}{3.203356in}}%
\pgfpathlineto{\pgfqpoint{3.665719in}{3.217338in}}%
\pgfpathlineto{\pgfqpoint{3.657960in}{3.199503in}}%
\pgfpathlineto{\pgfqpoint{3.650196in}{3.181898in}}%
\pgfpathlineto{\pgfqpoint{3.642428in}{3.164518in}}%
\pgfpathlineto{\pgfqpoint{3.634655in}{3.147358in}}%
\pgfpathclose%
\pgfusepath{fill}%
\end{pgfscope}%
\begin{pgfscope}%
\pgfpathrectangle{\pgfqpoint{1.150000in}{0.150000in}}{\pgfqpoint{5.700000in}{5.700000in}}%
\pgfusepath{clip}%
\pgfsetbuttcap%
\pgfsetroundjoin%
\definecolor{currentfill}{rgb}{0.231674,0.318106,0.544834}%
\pgfsetfillcolor{currentfill}%
\pgfsetfillopacity{0.800000}%
\pgfsetlinewidth{0.000000pt}%
\definecolor{currentstroke}{rgb}{0.000000,0.000000,0.000000}%
\pgfsetstrokecolor{currentstroke}%
\pgfsetdash{}{0pt}%
\pgfpathmoveto{\pgfqpoint{3.825398in}{3.066140in}}%
\pgfpathlineto{\pgfqpoint{3.838706in}{3.055129in}}%
\pgfpathlineto{\pgfqpoint{3.852016in}{3.044355in}}%
\pgfpathlineto{\pgfqpoint{3.865327in}{3.033816in}}%
\pgfpathlineto{\pgfqpoint{3.878639in}{3.023511in}}%
\pgfpathlineto{\pgfqpoint{3.886374in}{3.039861in}}%
\pgfpathlineto{\pgfqpoint{3.894105in}{3.056418in}}%
\pgfpathlineto{\pgfqpoint{3.901832in}{3.073188in}}%
\pgfpathlineto{\pgfqpoint{3.909556in}{3.090176in}}%
\pgfpathlineto{\pgfqpoint{3.896248in}{3.100921in}}%
\pgfpathlineto{\pgfqpoint{3.882943in}{3.111900in}}%
\pgfpathlineto{\pgfqpoint{3.869638in}{3.123114in}}%
\pgfpathlineto{\pgfqpoint{3.856335in}{3.134566in}}%
\pgfpathlineto{\pgfqpoint{3.848606in}{3.117126in}}%
\pgfpathlineto{\pgfqpoint{3.840874in}{3.099911in}}%
\pgfpathlineto{\pgfqpoint{3.833138in}{3.082918in}}%
\pgfpathlineto{\pgfqpoint{3.825398in}{3.066140in}}%
\pgfpathclose%
\pgfusepath{fill}%
\end{pgfscope}%
\begin{pgfscope}%
\pgfpathrectangle{\pgfqpoint{1.150000in}{0.150000in}}{\pgfqpoint{5.700000in}{5.700000in}}%
\pgfusepath{clip}%
\pgfsetbuttcap%
\pgfsetroundjoin%
\definecolor{currentfill}{rgb}{0.199430,0.387607,0.554642}%
\pgfsetfillcolor{currentfill}%
\pgfsetfillopacity{0.800000}%
\pgfsetlinewidth{0.000000pt}%
\definecolor{currentstroke}{rgb}{0.000000,0.000000,0.000000}%
\pgfsetstrokecolor{currentstroke}%
\pgfsetdash{}{0pt}%
\pgfpathmoveto{\pgfqpoint{4.520558in}{3.245554in}}%
\pgfpathlineto{\pgfqpoint{4.533955in}{3.238587in}}%
\pgfpathlineto{\pgfqpoint{4.547357in}{3.231817in}}%
\pgfpathlineto{\pgfqpoint{4.560766in}{3.225245in}}%
\pgfpathlineto{\pgfqpoint{4.574181in}{3.218869in}}%
\pgfpathlineto{\pgfqpoint{4.581803in}{3.237452in}}%
\pgfpathlineto{\pgfqpoint{4.589426in}{3.256350in}}%
\pgfpathlineto{\pgfqpoint{4.597049in}{3.275570in}}%
\pgfpathlineto{\pgfqpoint{4.604674in}{3.295120in}}%
\pgfpathlineto{\pgfqpoint{4.591268in}{3.302154in}}%
\pgfpathlineto{\pgfqpoint{4.577868in}{3.309384in}}%
\pgfpathlineto{\pgfqpoint{4.564474in}{3.316811in}}%
\pgfpathlineto{\pgfqpoint{4.551085in}{3.324436in}}%
\pgfpathlineto{\pgfqpoint{4.543452in}{3.304216in}}%
\pgfpathlineto{\pgfqpoint{4.535820in}{3.284334in}}%
\pgfpathlineto{\pgfqpoint{4.528189in}{3.264783in}}%
\pgfpathlineto{\pgfqpoint{4.520558in}{3.245554in}}%
\pgfpathclose%
\pgfusepath{fill}%
\end{pgfscope}%
\begin{pgfscope}%
\pgfpathrectangle{\pgfqpoint{1.150000in}{0.150000in}}{\pgfqpoint{5.700000in}{5.700000in}}%
\pgfusepath{clip}%
\pgfsetbuttcap%
\pgfsetroundjoin%
\definecolor{currentfill}{rgb}{0.221989,0.339161,0.548752}%
\pgfsetfillcolor{currentfill}%
\pgfsetfillopacity{0.800000}%
\pgfsetlinewidth{0.000000pt}%
\definecolor{currentstroke}{rgb}{0.000000,0.000000,0.000000}%
\pgfsetstrokecolor{currentstroke}%
\pgfsetdash{}{0pt}%
\pgfpathmoveto{\pgfqpoint{4.268361in}{3.115235in}}%
\pgfpathlineto{\pgfqpoint{4.281716in}{3.107612in}}%
\pgfpathlineto{\pgfqpoint{4.295077in}{3.100197in}}%
\pgfpathlineto{\pgfqpoint{4.308442in}{3.092990in}}%
\pgfpathlineto{\pgfqpoint{4.321813in}{3.085989in}}%
\pgfpathlineto{\pgfqpoint{4.329466in}{3.103014in}}%
\pgfpathlineto{\pgfqpoint{4.337117in}{3.120293in}}%
\pgfpathlineto{\pgfqpoint{4.344767in}{3.137832in}}%
\pgfpathlineto{\pgfqpoint{4.352416in}{3.155638in}}%
\pgfpathlineto{\pgfqpoint{4.339053in}{3.163202in}}%
\pgfpathlineto{\pgfqpoint{4.325694in}{3.170973in}}%
\pgfpathlineto{\pgfqpoint{4.312341in}{3.178951in}}%
\pgfpathlineto{\pgfqpoint{4.298992in}{3.187137in}}%
\pgfpathlineto{\pgfqpoint{4.291336in}{3.168756in}}%
\pgfpathlineto{\pgfqpoint{4.283679in}{3.150650in}}%
\pgfpathlineto{\pgfqpoint{4.276021in}{3.132812in}}%
\pgfpathlineto{\pgfqpoint{4.268361in}{3.115235in}}%
\pgfpathclose%
\pgfusepath{fill}%
\end{pgfscope}%
\begin{pgfscope}%
\pgfpathrectangle{\pgfqpoint{1.150000in}{0.150000in}}{\pgfqpoint{5.700000in}{5.700000in}}%
\pgfusepath{clip}%
\pgfsetbuttcap%
\pgfsetroundjoin%
\definecolor{currentfill}{rgb}{0.177423,0.437527,0.557565}%
\pgfsetfillcolor{currentfill}%
\pgfsetfillopacity{0.800000}%
\pgfsetlinewidth{0.000000pt}%
\definecolor{currentstroke}{rgb}{0.000000,0.000000,0.000000}%
\pgfsetstrokecolor{currentstroke}%
\pgfsetdash{}{0pt}%
\pgfpathmoveto{\pgfqpoint{3.421303in}{3.401046in}}%
\pgfpathlineto{\pgfqpoint{3.434665in}{3.383085in}}%
\pgfpathlineto{\pgfqpoint{3.448022in}{3.365417in}}%
\pgfpathlineto{\pgfqpoint{3.461374in}{3.348037in}}%
\pgfpathlineto{\pgfqpoint{3.474722in}{3.330944in}}%
\pgfpathlineto{\pgfqpoint{3.482509in}{3.349312in}}%
\pgfpathlineto{\pgfqpoint{3.490291in}{3.367924in}}%
\pgfpathlineto{\pgfqpoint{3.498068in}{3.386786in}}%
\pgfpathlineto{\pgfqpoint{3.505840in}{3.405902in}}%
\pgfpathlineto{\pgfqpoint{3.492496in}{3.423413in}}%
\pgfpathlineto{\pgfqpoint{3.479148in}{3.441211in}}%
\pgfpathlineto{\pgfqpoint{3.465795in}{3.459299in}}%
\pgfpathlineto{\pgfqpoint{3.452438in}{3.477678in}}%
\pgfpathlineto{\pgfqpoint{3.444662in}{3.458131in}}%
\pgfpathlineto{\pgfqpoint{3.436881in}{3.438846in}}%
\pgfpathlineto{\pgfqpoint{3.429095in}{3.419819in}}%
\pgfpathlineto{\pgfqpoint{3.421303in}{3.401046in}}%
\pgfpathclose%
\pgfusepath{fill}%
\end{pgfscope}%
\begin{pgfscope}%
\pgfpathrectangle{\pgfqpoint{1.150000in}{0.150000in}}{\pgfqpoint{5.700000in}{5.700000in}}%
\pgfusepath{clip}%
\pgfsetbuttcap%
\pgfsetroundjoin%
\definecolor{currentfill}{rgb}{0.266941,0.748751,0.440573}%
\pgfsetfillcolor{currentfill}%
\pgfsetfillopacity{0.800000}%
\pgfsetlinewidth{0.000000pt}%
\definecolor{currentstroke}{rgb}{0.000000,0.000000,0.000000}%
\pgfsetstrokecolor{currentstroke}%
\pgfsetdash{}{0pt}%
\pgfpathmoveto{\pgfqpoint{3.476923in}{4.324365in}}%
\pgfpathlineto{\pgfqpoint{3.490360in}{4.298945in}}%
\pgfpathlineto{\pgfqpoint{3.503788in}{4.273858in}}%
\pgfpathlineto{\pgfqpoint{3.517208in}{4.249100in}}%
\pgfpathlineto{\pgfqpoint{3.530620in}{4.224670in}}%
\pgfpathlineto{\pgfqpoint{3.538283in}{4.254053in}}%
\pgfpathlineto{\pgfqpoint{3.545943in}{4.283886in}}%
\pgfpathlineto{\pgfqpoint{3.553598in}{4.314178in}}%
\pgfpathlineto{\pgfqpoint{3.561251in}{4.344937in}}%
\pgfpathlineto{\pgfqpoint{3.547834in}{4.370049in}}%
\pgfpathlineto{\pgfqpoint{3.534409in}{4.395490in}}%
\pgfpathlineto{\pgfqpoint{3.520976in}{4.421262in}}%
\pgfpathlineto{\pgfqpoint{3.507534in}{4.447369in}}%
\pgfpathlineto{\pgfqpoint{3.499887in}{4.415912in}}%
\pgfpathlineto{\pgfqpoint{3.492236in}{4.384930in}}%
\pgfpathlineto{\pgfqpoint{3.484582in}{4.354418in}}%
\pgfpathlineto{\pgfqpoint{3.476923in}{4.324365in}}%
\pgfpathclose%
\pgfusepath{fill}%
\end{pgfscope}%
\begin{pgfscope}%
\pgfpathrectangle{\pgfqpoint{1.150000in}{0.150000in}}{\pgfqpoint{5.700000in}{5.700000in}}%
\pgfusepath{clip}%
\pgfsetbuttcap%
\pgfsetroundjoin%
\definecolor{currentfill}{rgb}{0.235526,0.309527,0.542944}%
\pgfsetfillcolor{currentfill}%
\pgfsetfillopacity{0.800000}%
\pgfsetlinewidth{0.000000pt}%
\definecolor{currentstroke}{rgb}{0.000000,0.000000,0.000000}%
\pgfsetstrokecolor{currentstroke}%
\pgfsetdash{}{0pt}%
\pgfpathmoveto{\pgfqpoint{3.962802in}{3.049511in}}%
\pgfpathlineto{\pgfqpoint{3.976119in}{3.039916in}}%
\pgfpathlineto{\pgfqpoint{3.989439in}{3.030547in}}%
\pgfpathlineto{\pgfqpoint{4.002761in}{3.021403in}}%
\pgfpathlineto{\pgfqpoint{4.016086in}{3.012482in}}%
\pgfpathlineto{\pgfqpoint{4.023795in}{3.028776in}}%
\pgfpathlineto{\pgfqpoint{4.031500in}{3.045282in}}%
\pgfpathlineto{\pgfqpoint{4.039203in}{3.062008in}}%
\pgfpathlineto{\pgfqpoint{4.046902in}{3.078957in}}%
\pgfpathlineto{\pgfqpoint{4.033583in}{3.088349in}}%
\pgfpathlineto{\pgfqpoint{4.020267in}{3.097963in}}%
\pgfpathlineto{\pgfqpoint{4.006953in}{3.107803in}}%
\pgfpathlineto{\pgfqpoint{3.993641in}{3.117869in}}%
\pgfpathlineto{\pgfqpoint{3.985936in}{3.100436in}}%
\pgfpathlineto{\pgfqpoint{3.978228in}{3.083236in}}%
\pgfpathlineto{\pgfqpoint{3.970517in}{3.066263in}}%
\pgfpathlineto{\pgfqpoint{3.962802in}{3.049511in}}%
\pgfpathclose%
\pgfusepath{fill}%
\end{pgfscope}%
\begin{pgfscope}%
\pgfpathrectangle{\pgfqpoint{1.150000in}{0.150000in}}{\pgfqpoint{5.700000in}{5.700000in}}%
\pgfusepath{clip}%
\pgfsetbuttcap%
\pgfsetroundjoin%
\definecolor{currentfill}{rgb}{0.146180,0.515413,0.556823}%
\pgfsetfillcolor{currentfill}%
\pgfsetfillopacity{0.800000}%
\pgfsetlinewidth{0.000000pt}%
\definecolor{currentstroke}{rgb}{0.000000,0.000000,0.000000}%
\pgfsetstrokecolor{currentstroke}%
\pgfsetdash{}{0pt}%
\pgfpathmoveto{\pgfqpoint{3.345380in}{3.635548in}}%
\pgfpathlineto{\pgfqpoint{3.358784in}{3.614736in}}%
\pgfpathlineto{\pgfqpoint{3.372181in}{3.594237in}}%
\pgfpathlineto{\pgfqpoint{3.385572in}{3.574050in}}%
\pgfpathlineto{\pgfqpoint{3.398957in}{3.554170in}}%
\pgfpathlineto{\pgfqpoint{3.406731in}{3.574428in}}%
\pgfpathlineto{\pgfqpoint{3.414499in}{3.594968in}}%
\pgfpathlineto{\pgfqpoint{3.422262in}{3.615795in}}%
\pgfpathlineto{\pgfqpoint{3.430020in}{3.636915in}}%
\pgfpathlineto{\pgfqpoint{3.416639in}{3.657251in}}%
\pgfpathlineto{\pgfqpoint{3.403251in}{3.677896in}}%
\pgfpathlineto{\pgfqpoint{3.389856in}{3.698853in}}%
\pgfpathlineto{\pgfqpoint{3.376455in}{3.720125in}}%
\pgfpathlineto{\pgfqpoint{3.368694in}{3.698534in}}%
\pgfpathlineto{\pgfqpoint{3.360929in}{3.677245in}}%
\pgfpathlineto{\pgfqpoint{3.353157in}{3.656251in}}%
\pgfpathlineto{\pgfqpoint{3.345380in}{3.635548in}}%
\pgfpathclose%
\pgfusepath{fill}%
\end{pgfscope}%
\begin{pgfscope}%
\pgfpathrectangle{\pgfqpoint{1.150000in}{0.150000in}}{\pgfqpoint{5.700000in}{5.700000in}}%
\pgfusepath{clip}%
\pgfsetbuttcap%
\pgfsetroundjoin%
\definecolor{currentfill}{rgb}{0.377779,0.791781,0.377939}%
\pgfsetfillcolor{currentfill}%
\pgfsetfillopacity{0.800000}%
\pgfsetlinewidth{0.000000pt}%
\definecolor{currentstroke}{rgb}{0.000000,0.000000,0.000000}%
\pgfsetstrokecolor{currentstroke}%
\pgfsetdash{}{0pt}%
\pgfpathmoveto{\pgfqpoint{3.591830in}{4.472804in}}%
\pgfpathlineto{\pgfqpoint{3.605245in}{4.447296in}}%
\pgfpathlineto{\pgfqpoint{3.618653in}{4.422112in}}%
\pgfpathlineto{\pgfqpoint{3.632053in}{4.397249in}}%
\pgfpathlineto{\pgfqpoint{3.645446in}{4.372703in}}%
\pgfpathlineto{\pgfqpoint{3.653090in}{4.405179in}}%
\pgfpathlineto{\pgfqpoint{3.660732in}{4.438163in}}%
\pgfpathlineto{\pgfqpoint{3.668372in}{4.471665in}}%
\pgfpathlineto{\pgfqpoint{3.676010in}{4.505694in}}%
\pgfpathlineto{\pgfqpoint{3.662610in}{4.530996in}}%
\pgfpathlineto{\pgfqpoint{3.649203in}{4.556617in}}%
\pgfpathlineto{\pgfqpoint{3.635788in}{4.582560in}}%
\pgfpathlineto{\pgfqpoint{3.622366in}{4.608830in}}%
\pgfpathlineto{\pgfqpoint{3.614736in}{4.574027in}}%
\pgfpathlineto{\pgfqpoint{3.607103in}{4.539761in}}%
\pgfpathlineto{\pgfqpoint{3.599468in}{4.506023in}}%
\pgfpathlineto{\pgfqpoint{3.591830in}{4.472804in}}%
\pgfpathclose%
\pgfusepath{fill}%
\end{pgfscope}%
\begin{pgfscope}%
\pgfpathrectangle{\pgfqpoint{1.150000in}{0.150000in}}{\pgfqpoint{5.700000in}{5.700000in}}%
\pgfusepath{clip}%
\pgfsetbuttcap%
\pgfsetroundjoin%
\definecolor{currentfill}{rgb}{0.190631,0.407061,0.556089}%
\pgfsetfillcolor{currentfill}%
\pgfsetfillopacity{0.800000}%
\pgfsetlinewidth{0.000000pt}%
\definecolor{currentstroke}{rgb}{0.000000,0.000000,0.000000}%
\pgfsetstrokecolor{currentstroke}%
\pgfsetdash{}{0pt}%
\pgfpathmoveto{\pgfqpoint{4.604674in}{3.295120in}}%
\pgfpathlineto{\pgfqpoint{4.618086in}{3.288282in}}%
\pgfpathlineto{\pgfqpoint{4.631505in}{3.281639in}}%
\pgfpathlineto{\pgfqpoint{4.644930in}{3.275190in}}%
\pgfpathlineto{\pgfqpoint{4.658362in}{3.268935in}}%
\pgfpathlineto{\pgfqpoint{4.665979in}{3.288145in}}%
\pgfpathlineto{\pgfqpoint{4.673597in}{3.307694in}}%
\pgfpathlineto{\pgfqpoint{4.681218in}{3.327589in}}%
\pgfpathlineto{\pgfqpoint{4.688840in}{3.347838in}}%
\pgfpathlineto{\pgfqpoint{4.675418in}{3.354783in}}%
\pgfpathlineto{\pgfqpoint{4.662003in}{3.361922in}}%
\pgfpathlineto{\pgfqpoint{4.648593in}{3.369254in}}%
\pgfpathlineto{\pgfqpoint{4.635190in}{3.376782in}}%
\pgfpathlineto{\pgfqpoint{4.627558in}{3.355831in}}%
\pgfpathlineto{\pgfqpoint{4.619928in}{3.335243in}}%
\pgfpathlineto{\pgfqpoint{4.612300in}{3.315008in}}%
\pgfpathlineto{\pgfqpoint{4.604674in}{3.295120in}}%
\pgfpathclose%
\pgfusepath{fill}%
\end{pgfscope}%
\begin{pgfscope}%
\pgfpathrectangle{\pgfqpoint{1.150000in}{0.150000in}}{\pgfqpoint{5.700000in}{5.700000in}}%
\pgfusepath{clip}%
\pgfsetbuttcap%
\pgfsetroundjoin%
\definecolor{currentfill}{rgb}{0.227802,0.326594,0.546532}%
\pgfsetfillcolor{currentfill}%
\pgfsetfillopacity{0.800000}%
\pgfsetlinewidth{0.000000pt}%
\definecolor{currentstroke}{rgb}{0.000000,0.000000,0.000000}%
\pgfsetstrokecolor{currentstroke}%
\pgfsetdash{}{0pt}%
\pgfpathmoveto{\pgfqpoint{4.184297in}{3.077877in}}%
\pgfpathlineto{\pgfqpoint{4.197642in}{3.069944in}}%
\pgfpathlineto{\pgfqpoint{4.210991in}{3.062222in}}%
\pgfpathlineto{\pgfqpoint{4.224344in}{3.054712in}}%
\pgfpathlineto{\pgfqpoint{4.237702in}{3.047412in}}%
\pgfpathlineto{\pgfqpoint{4.245370in}{3.064008in}}%
\pgfpathlineto{\pgfqpoint{4.253035in}{3.080839in}}%
\pgfpathlineto{\pgfqpoint{4.260699in}{3.097913in}}%
\pgfpathlineto{\pgfqpoint{4.268361in}{3.115235in}}%
\pgfpathlineto{\pgfqpoint{4.255009in}{3.123067in}}%
\pgfpathlineto{\pgfqpoint{4.241663in}{3.131109in}}%
\pgfpathlineto{\pgfqpoint{4.228320in}{3.139362in}}%
\pgfpathlineto{\pgfqpoint{4.214982in}{3.147828in}}%
\pgfpathlineto{\pgfqpoint{4.207314in}{3.129962in}}%
\pgfpathlineto{\pgfqpoint{4.199643in}{3.112352in}}%
\pgfpathlineto{\pgfqpoint{4.191971in}{3.094993in}}%
\pgfpathlineto{\pgfqpoint{4.184297in}{3.077877in}}%
\pgfpathclose%
\pgfusepath{fill}%
\end{pgfscope}%
\begin{pgfscope}%
\pgfpathrectangle{\pgfqpoint{1.150000in}{0.150000in}}{\pgfqpoint{5.700000in}{5.700000in}}%
\pgfusepath{clip}%
\pgfsetbuttcap%
\pgfsetroundjoin%
\definecolor{currentfill}{rgb}{0.125394,0.574318,0.549086}%
\pgfsetfillcolor{currentfill}%
\pgfsetfillopacity{0.800000}%
\pgfsetlinewidth{0.000000pt}%
\definecolor{currentstroke}{rgb}{0.000000,0.000000,0.000000}%
\pgfsetstrokecolor{currentstroke}%
\pgfsetdash{}{0pt}%
\pgfpathmoveto{\pgfqpoint{3.322777in}{3.808417in}}%
\pgfpathlineto{\pgfqpoint{3.336208in}{3.785857in}}%
\pgfpathlineto{\pgfqpoint{3.349631in}{3.763624in}}%
\pgfpathlineto{\pgfqpoint{3.363047in}{3.741714in}}%
\pgfpathlineto{\pgfqpoint{3.376455in}{3.720125in}}%
\pgfpathlineto{\pgfqpoint{3.384210in}{3.742023in}}%
\pgfpathlineto{\pgfqpoint{3.391959in}{3.764234in}}%
\pgfpathlineto{\pgfqpoint{3.399704in}{3.786763in}}%
\pgfpathlineto{\pgfqpoint{3.407443in}{3.809617in}}%
\pgfpathlineto{\pgfqpoint{3.394036in}{3.831701in}}%
\pgfpathlineto{\pgfqpoint{3.380622in}{3.854107in}}%
\pgfpathlineto{\pgfqpoint{3.367200in}{3.876838in}}%
\pgfpathlineto{\pgfqpoint{3.353770in}{3.899896in}}%
\pgfpathlineto{\pgfqpoint{3.346030in}{3.876532in}}%
\pgfpathlineto{\pgfqpoint{3.338285in}{3.853502in}}%
\pgfpathlineto{\pgfqpoint{3.330533in}{3.830799in}}%
\pgfpathlineto{\pgfqpoint{3.322777in}{3.808417in}}%
\pgfpathclose%
\pgfusepath{fill}%
\end{pgfscope}%
\begin{pgfscope}%
\pgfpathrectangle{\pgfqpoint{1.150000in}{0.150000in}}{\pgfqpoint{5.700000in}{5.700000in}}%
\pgfusepath{clip}%
\pgfsetbuttcap%
\pgfsetroundjoin%
\definecolor{currentfill}{rgb}{0.227802,0.326594,0.546532}%
\pgfsetfillcolor{currentfill}%
\pgfsetfillopacity{0.800000}%
\pgfsetlinewidth{0.000000pt}%
\definecolor{currentstroke}{rgb}{0.000000,0.000000,0.000000}%
\pgfsetstrokecolor{currentstroke}%
\pgfsetdash{}{0pt}%
\pgfpathmoveto{\pgfqpoint{3.687906in}{3.094614in}}%
\pgfpathlineto{\pgfqpoint{3.701217in}{3.082061in}}%
\pgfpathlineto{\pgfqpoint{3.714528in}{3.069758in}}%
\pgfpathlineto{\pgfqpoint{3.727839in}{3.057702in}}%
\pgfpathlineto{\pgfqpoint{3.741149in}{3.045893in}}%
\pgfpathlineto{\pgfqpoint{3.748911in}{3.062255in}}%
\pgfpathlineto{\pgfqpoint{3.756669in}{3.078822in}}%
\pgfpathlineto{\pgfqpoint{3.764422in}{3.095596in}}%
\pgfpathlineto{\pgfqpoint{3.772172in}{3.112585in}}%
\pgfpathlineto{\pgfqpoint{3.758866in}{3.124804in}}%
\pgfpathlineto{\pgfqpoint{3.745561in}{3.137270in}}%
\pgfpathlineto{\pgfqpoint{3.732256in}{3.149984in}}%
\pgfpathlineto{\pgfqpoint{3.718950in}{3.162948in}}%
\pgfpathlineto{\pgfqpoint{3.711195in}{3.145537in}}%
\pgfpathlineto{\pgfqpoint{3.703437in}{3.128347in}}%
\pgfpathlineto{\pgfqpoint{3.695674in}{3.111375in}}%
\pgfpathlineto{\pgfqpoint{3.687906in}{3.094614in}}%
\pgfpathclose%
\pgfusepath{fill}%
\end{pgfscope}%
\begin{pgfscope}%
\pgfpathrectangle{\pgfqpoint{1.150000in}{0.150000in}}{\pgfqpoint{5.700000in}{5.700000in}}%
\pgfusepath{clip}%
\pgfsetbuttcap%
\pgfsetroundjoin%
\definecolor{currentfill}{rgb}{0.166617,0.463708,0.558119}%
\pgfsetfillcolor{currentfill}%
\pgfsetfillopacity{0.800000}%
\pgfsetlinewidth{0.000000pt}%
\definecolor{currentstroke}{rgb}{0.000000,0.000000,0.000000}%
\pgfsetstrokecolor{currentstroke}%
\pgfsetdash{}{0pt}%
\pgfpathmoveto{\pgfqpoint{3.367806in}{3.475853in}}%
\pgfpathlineto{\pgfqpoint{3.381189in}{3.456701in}}%
\pgfpathlineto{\pgfqpoint{3.394566in}{3.437851in}}%
\pgfpathlineto{\pgfqpoint{3.407937in}{3.419300in}}%
\pgfpathlineto{\pgfqpoint{3.421303in}{3.401046in}}%
\pgfpathlineto{\pgfqpoint{3.429095in}{3.419819in}}%
\pgfpathlineto{\pgfqpoint{3.436881in}{3.438846in}}%
\pgfpathlineto{\pgfqpoint{3.444662in}{3.458131in}}%
\pgfpathlineto{\pgfqpoint{3.452438in}{3.477678in}}%
\pgfpathlineto{\pgfqpoint{3.439076in}{3.496353in}}%
\pgfpathlineto{\pgfqpoint{3.425708in}{3.515325in}}%
\pgfpathlineto{\pgfqpoint{3.412335in}{3.534596in}}%
\pgfpathlineto{\pgfqpoint{3.398957in}{3.554170in}}%
\pgfpathlineto{\pgfqpoint{3.391177in}{3.534189in}}%
\pgfpathlineto{\pgfqpoint{3.383392in}{3.514479in}}%
\pgfpathlineto{\pgfqpoint{3.375602in}{3.495035in}}%
\pgfpathlineto{\pgfqpoint{3.367806in}{3.475853in}}%
\pgfpathclose%
\pgfusepath{fill}%
\end{pgfscope}%
\begin{pgfscope}%
\pgfpathrectangle{\pgfqpoint{1.150000in}{0.150000in}}{\pgfqpoint{5.700000in}{5.700000in}}%
\pgfusepath{clip}%
\pgfsetbuttcap%
\pgfsetroundjoin%
\definecolor{currentfill}{rgb}{0.180629,0.429975,0.557282}%
\pgfsetfillcolor{currentfill}%
\pgfsetfillopacity{0.800000}%
\pgfsetlinewidth{0.000000pt}%
\definecolor{currentstroke}{rgb}{0.000000,0.000000,0.000000}%
\pgfsetstrokecolor{currentstroke}%
\pgfsetdash{}{0pt}%
\pgfpathmoveto{\pgfqpoint{4.688840in}{3.347838in}}%
\pgfpathlineto{\pgfqpoint{4.702269in}{3.341086in}}%
\pgfpathlineto{\pgfqpoint{4.715704in}{3.334526in}}%
\pgfpathlineto{\pgfqpoint{4.729146in}{3.328157in}}%
\pgfpathlineto{\pgfqpoint{4.742595in}{3.321980in}}%
\pgfpathlineto{\pgfqpoint{4.750210in}{3.341882in}}%
\pgfpathlineto{\pgfqpoint{4.757828in}{3.362147in}}%
\pgfpathlineto{\pgfqpoint{4.765448in}{3.382784in}}%
\pgfpathlineto{\pgfqpoint{4.773072in}{3.403801in}}%
\pgfpathlineto{\pgfqpoint{4.759634in}{3.410700in}}%
\pgfpathlineto{\pgfqpoint{4.746202in}{3.417790in}}%
\pgfpathlineto{\pgfqpoint{4.732777in}{3.425072in}}%
\pgfpathlineto{\pgfqpoint{4.719358in}{3.432546in}}%
\pgfpathlineto{\pgfqpoint{4.711724in}{3.410795in}}%
\pgfpathlineto{\pgfqpoint{4.704093in}{3.389433in}}%
\pgfpathlineto{\pgfqpoint{4.696465in}{3.368450in}}%
\pgfpathlineto{\pgfqpoint{4.688840in}{3.347838in}}%
\pgfpathclose%
\pgfusepath{fill}%
\end{pgfscope}%
\begin{pgfscope}%
\pgfpathrectangle{\pgfqpoint{1.150000in}{0.150000in}}{\pgfqpoint{5.700000in}{5.700000in}}%
\pgfusepath{clip}%
\pgfsetbuttcap%
\pgfsetroundjoin%
\definecolor{currentfill}{rgb}{0.233603,0.313828,0.543914}%
\pgfsetfillcolor{currentfill}%
\pgfsetfillopacity{0.800000}%
\pgfsetlinewidth{0.000000pt}%
\definecolor{currentstroke}{rgb}{0.000000,0.000000,0.000000}%
\pgfsetstrokecolor{currentstroke}%
\pgfsetdash{}{0pt}%
\pgfpathmoveto{\pgfqpoint{4.100210in}{3.043604in}}%
\pgfpathlineto{\pgfqpoint{4.113546in}{3.035312in}}%
\pgfpathlineto{\pgfqpoint{4.126885in}{3.027237in}}%
\pgfpathlineto{\pgfqpoint{4.140228in}{3.019377in}}%
\pgfpathlineto{\pgfqpoint{4.153575in}{3.011731in}}%
\pgfpathlineto{\pgfqpoint{4.161260in}{3.027932in}}%
\pgfpathlineto{\pgfqpoint{4.168941in}{3.044353in}}%
\pgfpathlineto{\pgfqpoint{4.176620in}{3.060999in}}%
\pgfpathlineto{\pgfqpoint{4.184297in}{3.077877in}}%
\pgfpathlineto{\pgfqpoint{4.170956in}{3.086024in}}%
\pgfpathlineto{\pgfqpoint{4.157619in}{3.094384in}}%
\pgfpathlineto{\pgfqpoint{4.144287in}{3.102961in}}%
\pgfpathlineto{\pgfqpoint{4.130957in}{3.111754in}}%
\pgfpathlineto{\pgfqpoint{4.123274in}{3.094363in}}%
\pgfpathlineto{\pgfqpoint{4.115589in}{3.077211in}}%
\pgfpathlineto{\pgfqpoint{4.107901in}{3.060294in}}%
\pgfpathlineto{\pgfqpoint{4.100210in}{3.043604in}}%
\pgfpathclose%
\pgfusepath{fill}%
\end{pgfscope}%
\begin{pgfscope}%
\pgfpathrectangle{\pgfqpoint{1.150000in}{0.150000in}}{\pgfqpoint{5.700000in}{5.700000in}}%
\pgfusepath{clip}%
\pgfsetbuttcap%
\pgfsetroundjoin%
\definecolor{currentfill}{rgb}{0.360741,0.785964,0.387814}%
\pgfsetfillcolor{currentfill}%
\pgfsetfillopacity{0.800000}%
\pgfsetlinewidth{0.000000pt}%
\definecolor{currentstroke}{rgb}{0.000000,0.000000,0.000000}%
\pgfsetstrokecolor{currentstroke}%
\pgfsetdash{}{0pt}%
\pgfpathmoveto{\pgfqpoint{3.507534in}{4.447369in}}%
\pgfpathlineto{\pgfqpoint{3.520976in}{4.421262in}}%
\pgfpathlineto{\pgfqpoint{3.534409in}{4.395490in}}%
\pgfpathlineto{\pgfqpoint{3.547834in}{4.370049in}}%
\pgfpathlineto{\pgfqpoint{3.561251in}{4.344937in}}%
\pgfpathlineto{\pgfqpoint{3.568900in}{4.376170in}}%
\pgfpathlineto{\pgfqpoint{3.576546in}{4.407886in}}%
\pgfpathlineto{\pgfqpoint{3.584190in}{4.440095in}}%
\pgfpathlineto{\pgfqpoint{3.591830in}{4.472804in}}%
\pgfpathlineto{\pgfqpoint{3.578407in}{4.498639in}}%
\pgfpathlineto{\pgfqpoint{3.564976in}{4.524803in}}%
\pgfpathlineto{\pgfqpoint{3.551536in}{4.551301in}}%
\pgfpathlineto{\pgfqpoint{3.538088in}{4.578136in}}%
\pgfpathlineto{\pgfqpoint{3.530455in}{4.544686in}}%
\pgfpathlineto{\pgfqpoint{3.522818in}{4.511748in}}%
\pgfpathlineto{\pgfqpoint{3.515178in}{4.479312in}}%
\pgfpathlineto{\pgfqpoint{3.507534in}{4.447369in}}%
\pgfpathclose%
\pgfusepath{fill}%
\end{pgfscope}%
\begin{pgfscope}%
\pgfpathrectangle{\pgfqpoint{1.150000in}{0.150000in}}{\pgfqpoint{5.700000in}{5.700000in}}%
\pgfusepath{clip}%
\pgfsetbuttcap%
\pgfsetroundjoin%
\definecolor{currentfill}{rgb}{0.239346,0.300855,0.540844}%
\pgfsetfillcolor{currentfill}%
\pgfsetfillopacity{0.800000}%
\pgfsetlinewidth{0.000000pt}%
\definecolor{currentstroke}{rgb}{0.000000,0.000000,0.000000}%
\pgfsetstrokecolor{currentstroke}%
\pgfsetdash{}{0pt}%
\pgfpathmoveto{\pgfqpoint{3.878639in}{3.023511in}}%
\pgfpathlineto{\pgfqpoint{3.891954in}{3.013439in}}%
\pgfpathlineto{\pgfqpoint{3.905270in}{3.003598in}}%
\pgfpathlineto{\pgfqpoint{3.918588in}{2.993987in}}%
\pgfpathlineto{\pgfqpoint{3.931909in}{2.984604in}}%
\pgfpathlineto{\pgfqpoint{3.939637in}{3.000526in}}%
\pgfpathlineto{\pgfqpoint{3.947362in}{3.016647in}}%
\pgfpathlineto{\pgfqpoint{3.955084in}{3.032974in}}%
\pgfpathlineto{\pgfqpoint{3.962802in}{3.049511in}}%
\pgfpathlineto{\pgfqpoint{3.949487in}{3.059333in}}%
\pgfpathlineto{\pgfqpoint{3.936175in}{3.069383in}}%
\pgfpathlineto{\pgfqpoint{3.922864in}{3.079664in}}%
\pgfpathlineto{\pgfqpoint{3.909556in}{3.090176in}}%
\pgfpathlineto{\pgfqpoint{3.901832in}{3.073188in}}%
\pgfpathlineto{\pgfqpoint{3.894105in}{3.056418in}}%
\pgfpathlineto{\pgfqpoint{3.886374in}{3.039861in}}%
\pgfpathlineto{\pgfqpoint{3.878639in}{3.023511in}}%
\pgfpathclose%
\pgfusepath{fill}%
\end{pgfscope}%
\begin{pgfscope}%
\pgfpathrectangle{\pgfqpoint{1.150000in}{0.150000in}}{\pgfqpoint{5.700000in}{5.700000in}}%
\pgfusepath{clip}%
\pgfsetbuttcap%
\pgfsetroundjoin%
\definecolor{currentfill}{rgb}{0.146616,0.673050,0.508936}%
\pgfsetfillcolor{currentfill}%
\pgfsetfillopacity{0.800000}%
\pgfsetlinewidth{0.000000pt}%
\definecolor{currentstroke}{rgb}{0.000000,0.000000,0.000000}%
\pgfsetstrokecolor{currentstroke}%
\pgfsetdash{}{0pt}%
\pgfpathmoveto{\pgfqpoint{3.330873in}{4.094535in}}%
\pgfpathlineto{\pgfqpoint{3.344338in}{4.069594in}}%
\pgfpathlineto{\pgfqpoint{3.357793in}{4.044995in}}%
\pgfpathlineto{\pgfqpoint{3.371239in}{4.020735in}}%
\pgfpathlineto{\pgfqpoint{3.384677in}{3.996810in}}%
\pgfpathlineto{\pgfqpoint{3.392390in}{4.021935in}}%
\pgfpathlineto{\pgfqpoint{3.400099in}{4.047431in}}%
\pgfpathlineto{\pgfqpoint{3.407802in}{4.073305in}}%
\pgfpathlineto{\pgfqpoint{3.415501in}{4.099565in}}%
\pgfpathlineto{\pgfqpoint{3.402062in}{4.124062in}}%
\pgfpathlineto{\pgfqpoint{3.388614in}{4.148896in}}%
\pgfpathlineto{\pgfqpoint{3.375158in}{4.174070in}}%
\pgfpathlineto{\pgfqpoint{3.361692in}{4.199588in}}%
\pgfpathlineto{\pgfqpoint{3.353995in}{4.172740in}}%
\pgfpathlineto{\pgfqpoint{3.346293in}{4.146286in}}%
\pgfpathlineto{\pgfqpoint{3.338586in}{4.120220in}}%
\pgfpathlineto{\pgfqpoint{3.330873in}{4.094535in}}%
\pgfpathclose%
\pgfusepath{fill}%
\end{pgfscope}%
\begin{pgfscope}%
\pgfpathrectangle{\pgfqpoint{1.150000in}{0.150000in}}{\pgfqpoint{5.700000in}{5.700000in}}%
\pgfusepath{clip}%
\pgfsetbuttcap%
\pgfsetroundjoin%
\definecolor{currentfill}{rgb}{0.191090,0.708366,0.482284}%
\pgfsetfillcolor{currentfill}%
\pgfsetfillopacity{0.800000}%
\pgfsetlinewidth{0.000000pt}%
\definecolor{currentstroke}{rgb}{0.000000,0.000000,0.000000}%
\pgfsetstrokecolor{currentstroke}%
\pgfsetdash{}{0pt}%
\pgfpathmoveto{\pgfqpoint{3.361692in}{4.199588in}}%
\pgfpathlineto{\pgfqpoint{3.375158in}{4.174070in}}%
\pgfpathlineto{\pgfqpoint{3.388614in}{4.148896in}}%
\pgfpathlineto{\pgfqpoint{3.402062in}{4.124062in}}%
\pgfpathlineto{\pgfqpoint{3.415501in}{4.099565in}}%
\pgfpathlineto{\pgfqpoint{3.423195in}{4.126217in}}%
\pgfpathlineto{\pgfqpoint{3.430884in}{4.153267in}}%
\pgfpathlineto{\pgfqpoint{3.438568in}{4.180724in}}%
\pgfpathlineto{\pgfqpoint{3.446248in}{4.208595in}}%
\pgfpathlineto{\pgfqpoint{3.432806in}{4.233703in}}%
\pgfpathlineto{\pgfqpoint{3.419356in}{4.259148in}}%
\pgfpathlineto{\pgfqpoint{3.405897in}{4.284935in}}%
\pgfpathlineto{\pgfqpoint{3.392429in}{4.311068in}}%
\pgfpathlineto{\pgfqpoint{3.384752in}{4.282570in}}%
\pgfpathlineto{\pgfqpoint{3.377070in}{4.254496in}}%
\pgfpathlineto{\pgfqpoint{3.369384in}{4.226837in}}%
\pgfpathlineto{\pgfqpoint{3.361692in}{4.199588in}}%
\pgfpathclose%
\pgfusepath{fill}%
\end{pgfscope}%
\begin{pgfscope}%
\pgfpathrectangle{\pgfqpoint{1.150000in}{0.150000in}}{\pgfqpoint{5.700000in}{5.700000in}}%
\pgfusepath{clip}%
\pgfsetbuttcap%
\pgfsetroundjoin%
\definecolor{currentfill}{rgb}{0.235526,0.309527,0.542944}%
\pgfsetfillcolor{currentfill}%
\pgfsetfillopacity{0.800000}%
\pgfsetlinewidth{0.000000pt}%
\definecolor{currentstroke}{rgb}{0.000000,0.000000,0.000000}%
\pgfsetstrokecolor{currentstroke}%
\pgfsetdash{}{0pt}%
\pgfpathmoveto{\pgfqpoint{3.741149in}{3.045893in}}%
\pgfpathlineto{\pgfqpoint{3.754461in}{3.034328in}}%
\pgfpathlineto{\pgfqpoint{3.767772in}{3.023006in}}%
\pgfpathlineto{\pgfqpoint{3.781084in}{3.011926in}}%
\pgfpathlineto{\pgfqpoint{3.794397in}{3.001086in}}%
\pgfpathlineto{\pgfqpoint{3.802154in}{3.017051in}}%
\pgfpathlineto{\pgfqpoint{3.809906in}{3.033211in}}%
\pgfpathlineto{\pgfqpoint{3.817654in}{3.049573in}}%
\pgfpathlineto{\pgfqpoint{3.825398in}{3.066140in}}%
\pgfpathlineto{\pgfqpoint{3.812090in}{3.077390in}}%
\pgfpathlineto{\pgfqpoint{3.798784in}{3.088879in}}%
\pgfpathlineto{\pgfqpoint{3.785477in}{3.100611in}}%
\pgfpathlineto{\pgfqpoint{3.772172in}{3.112585in}}%
\pgfpathlineto{\pgfqpoint{3.764422in}{3.095596in}}%
\pgfpathlineto{\pgfqpoint{3.756669in}{3.078822in}}%
\pgfpathlineto{\pgfqpoint{3.748911in}{3.062255in}}%
\pgfpathlineto{\pgfqpoint{3.741149in}{3.045893in}}%
\pgfpathclose%
\pgfusepath{fill}%
\end{pgfscope}%
\begin{pgfscope}%
\pgfpathrectangle{\pgfqpoint{1.150000in}{0.150000in}}{\pgfqpoint{5.700000in}{5.700000in}}%
\pgfusepath{clip}%
\pgfsetbuttcap%
\pgfsetroundjoin%
\definecolor{currentfill}{rgb}{0.133743,0.548535,0.553541}%
\pgfsetfillcolor{currentfill}%
\pgfsetfillopacity{0.800000}%
\pgfsetlinewidth{0.000000pt}%
\definecolor{currentstroke}{rgb}{0.000000,0.000000,0.000000}%
\pgfsetstrokecolor{currentstroke}%
\pgfsetdash{}{0pt}%
\pgfpathmoveto{\pgfqpoint{3.291691in}{3.721995in}}%
\pgfpathlineto{\pgfqpoint{3.305125in}{3.699897in}}%
\pgfpathlineto{\pgfqpoint{3.318551in}{3.678126in}}%
\pgfpathlineto{\pgfqpoint{3.331969in}{3.656677in}}%
\pgfpathlineto{\pgfqpoint{3.345380in}{3.635548in}}%
\pgfpathlineto{\pgfqpoint{3.353157in}{3.656251in}}%
\pgfpathlineto{\pgfqpoint{3.360929in}{3.677245in}}%
\pgfpathlineto{\pgfqpoint{3.368694in}{3.698534in}}%
\pgfpathlineto{\pgfqpoint{3.376455in}{3.720125in}}%
\pgfpathlineto{\pgfqpoint{3.363047in}{3.741714in}}%
\pgfpathlineto{\pgfqpoint{3.349631in}{3.763624in}}%
\pgfpathlineto{\pgfqpoint{3.336208in}{3.785857in}}%
\pgfpathlineto{\pgfqpoint{3.322777in}{3.808417in}}%
\pgfpathlineto{\pgfqpoint{3.315014in}{3.786352in}}%
\pgfpathlineto{\pgfqpoint{3.307245in}{3.764597in}}%
\pgfpathlineto{\pgfqpoint{3.299471in}{3.743146in}}%
\pgfpathlineto{\pgfqpoint{3.291691in}{3.721995in}}%
\pgfpathclose%
\pgfusepath{fill}%
\end{pgfscope}%
\begin{pgfscope}%
\pgfpathrectangle{\pgfqpoint{1.150000in}{0.150000in}}{\pgfqpoint{5.700000in}{5.700000in}}%
\pgfusepath{clip}%
\pgfsetbuttcap%
\pgfsetroundjoin%
\definecolor{currentfill}{rgb}{0.208623,0.367752,0.552675}%
\pgfsetfillcolor{currentfill}%
\pgfsetfillopacity{0.800000}%
\pgfsetlinewidth{0.000000pt}%
\definecolor{currentstroke}{rgb}{0.000000,0.000000,0.000000}%
\pgfsetstrokecolor{currentstroke}%
\pgfsetdash{}{0pt}%
\pgfpathmoveto{\pgfqpoint{3.496897in}{3.195805in}}%
\pgfpathlineto{\pgfqpoint{3.510232in}{3.180493in}}%
\pgfpathlineto{\pgfqpoint{3.523566in}{3.165453in}}%
\pgfpathlineto{\pgfqpoint{3.536896in}{3.150684in}}%
\pgfpathlineto{\pgfqpoint{3.550224in}{3.136183in}}%
\pgfpathlineto{\pgfqpoint{3.558021in}{3.152879in}}%
\pgfpathlineto{\pgfqpoint{3.565814in}{3.169784in}}%
\pgfpathlineto{\pgfqpoint{3.573601in}{3.186904in}}%
\pgfpathlineto{\pgfqpoint{3.581383in}{3.204241in}}%
\pgfpathlineto{\pgfqpoint{3.568061in}{3.219124in}}%
\pgfpathlineto{\pgfqpoint{3.554735in}{3.234275in}}%
\pgfpathlineto{\pgfqpoint{3.541408in}{3.249697in}}%
\pgfpathlineto{\pgfqpoint{3.528077in}{3.265392in}}%
\pgfpathlineto{\pgfqpoint{3.520290in}{3.247659in}}%
\pgfpathlineto{\pgfqpoint{3.512497in}{3.230154in}}%
\pgfpathlineto{\pgfqpoint{3.504699in}{3.212871in}}%
\pgfpathlineto{\pgfqpoint{3.496897in}{3.195805in}}%
\pgfpathclose%
\pgfusepath{fill}%
\end{pgfscope}%
\begin{pgfscope}%
\pgfpathrectangle{\pgfqpoint{1.150000in}{0.150000in}}{\pgfqpoint{5.700000in}{5.700000in}}%
\pgfusepath{clip}%
\pgfsetbuttcap%
\pgfsetroundjoin%
\definecolor{currentfill}{rgb}{0.124780,0.640461,0.527068}%
\pgfsetfillcolor{currentfill}%
\pgfsetfillopacity{0.800000}%
\pgfsetlinewidth{0.000000pt}%
\definecolor{currentstroke}{rgb}{0.000000,0.000000,0.000000}%
\pgfsetstrokecolor{currentstroke}%
\pgfsetdash{}{0pt}%
\pgfpathmoveto{\pgfqpoint{3.299967in}{3.995472in}}%
\pgfpathlineto{\pgfqpoint{3.313431in}{3.971070in}}%
\pgfpathlineto{\pgfqpoint{3.326886in}{3.947009in}}%
\pgfpathlineto{\pgfqpoint{3.340332in}{3.923286in}}%
\pgfpathlineto{\pgfqpoint{3.353770in}{3.899896in}}%
\pgfpathlineto{\pgfqpoint{3.361505in}{3.923600in}}%
\pgfpathlineto{\pgfqpoint{3.369234in}{3.947649in}}%
\pgfpathlineto{\pgfqpoint{3.376958in}{3.972050in}}%
\pgfpathlineto{\pgfqpoint{3.384677in}{3.996810in}}%
\pgfpathlineto{\pgfqpoint{3.371239in}{4.020735in}}%
\pgfpathlineto{\pgfqpoint{3.357793in}{4.044995in}}%
\pgfpathlineto{\pgfqpoint{3.344338in}{4.069594in}}%
\pgfpathlineto{\pgfqpoint{3.330873in}{4.094535in}}%
\pgfpathlineto{\pgfqpoint{3.323155in}{4.069224in}}%
\pgfpathlineto{\pgfqpoint{3.315431in}{4.044281in}}%
\pgfpathlineto{\pgfqpoint{3.307702in}{4.019699in}}%
\pgfpathlineto{\pgfqpoint{3.299967in}{3.995472in}}%
\pgfpathclose%
\pgfusepath{fill}%
\end{pgfscope}%
\begin{pgfscope}%
\pgfpathrectangle{\pgfqpoint{1.150000in}{0.150000in}}{\pgfqpoint{5.700000in}{5.700000in}}%
\pgfusepath{clip}%
\pgfsetbuttcap%
\pgfsetroundjoin%
\definecolor{currentfill}{rgb}{0.515992,0.831158,0.294279}%
\pgfsetfillcolor{currentfill}%
\pgfsetfillopacity{0.800000}%
\pgfsetlinewidth{0.000000pt}%
\definecolor{currentstroke}{rgb}{0.000000,0.000000,0.000000}%
\pgfsetstrokecolor{currentstroke}%
\pgfsetdash{}{0pt}%
\pgfpathmoveto{\pgfqpoint{3.706547in}{4.647267in}}%
\pgfpathlineto{\pgfqpoint{3.719948in}{4.621486in}}%
\pgfpathlineto{\pgfqpoint{3.733342in}{4.596020in}}%
\pgfpathlineto{\pgfqpoint{3.746730in}{4.570866in}}%
\pgfpathlineto{\pgfqpoint{3.760110in}{4.546023in}}%
\pgfpathlineto{\pgfqpoint{3.767750in}{4.582010in}}%
\pgfpathlineto{\pgfqpoint{3.775390in}{4.618571in}}%
\pgfpathlineto{\pgfqpoint{3.783029in}{4.655718in}}%
\pgfpathlineto{\pgfqpoint{3.769641in}{4.681182in}}%
\pgfpathlineto{\pgfqpoint{3.756247in}{4.706957in}}%
\pgfpathlineto{\pgfqpoint{3.742846in}{4.733046in}}%
\pgfpathlineto{\pgfqpoint{3.729438in}{4.759453in}}%
\pgfpathlineto{\pgfqpoint{3.721809in}{4.721465in}}%
\pgfpathlineto{\pgfqpoint{3.714178in}{4.684073in}}%
\pgfpathlineto{\pgfqpoint{3.706547in}{4.647267in}}%
\pgfpathclose%
\pgfusepath{fill}%
\end{pgfscope}%
\begin{pgfscope}%
\pgfpathrectangle{\pgfqpoint{1.150000in}{0.150000in}}{\pgfqpoint{5.700000in}{5.700000in}}%
\pgfusepath{clip}%
\pgfsetbuttcap%
\pgfsetroundjoin%
\definecolor{currentfill}{rgb}{0.239346,0.300855,0.540844}%
\pgfsetfillcolor{currentfill}%
\pgfsetfillopacity{0.800000}%
\pgfsetlinewidth{0.000000pt}%
\definecolor{currentstroke}{rgb}{0.000000,0.000000,0.000000}%
\pgfsetstrokecolor{currentstroke}%
\pgfsetdash{}{0pt}%
\pgfpathmoveto{\pgfqpoint{4.016086in}{3.012482in}}%
\pgfpathlineto{\pgfqpoint{4.029414in}{3.003783in}}%
\pgfpathlineto{\pgfqpoint{4.042745in}{2.995305in}}%
\pgfpathlineto{\pgfqpoint{4.056080in}{2.987047in}}%
\pgfpathlineto{\pgfqpoint{4.069418in}{2.979008in}}%
\pgfpathlineto{\pgfqpoint{4.077121in}{2.994843in}}%
\pgfpathlineto{\pgfqpoint{4.084820in}{3.010884in}}%
\pgfpathlineto{\pgfqpoint{4.092517in}{3.027136in}}%
\pgfpathlineto{\pgfqpoint{4.100210in}{3.043604in}}%
\pgfpathlineto{\pgfqpoint{4.086878in}{3.052113in}}%
\pgfpathlineto{\pgfqpoint{4.073550in}{3.060841in}}%
\pgfpathlineto{\pgfqpoint{4.060225in}{3.069789in}}%
\pgfpathlineto{\pgfqpoint{4.046902in}{3.078957in}}%
\pgfpathlineto{\pgfqpoint{4.039203in}{3.062008in}}%
\pgfpathlineto{\pgfqpoint{4.031500in}{3.045282in}}%
\pgfpathlineto{\pgfqpoint{4.023795in}{3.028776in}}%
\pgfpathlineto{\pgfqpoint{4.016086in}{3.012482in}}%
\pgfpathclose%
\pgfusepath{fill}%
\end{pgfscope}%
\begin{pgfscope}%
\pgfpathrectangle{\pgfqpoint{1.150000in}{0.150000in}}{\pgfqpoint{5.700000in}{5.700000in}}%
\pgfusepath{clip}%
\pgfsetbuttcap%
\pgfsetroundjoin%
\definecolor{currentfill}{rgb}{0.197636,0.391528,0.554969}%
\pgfsetfillcolor{currentfill}%
\pgfsetfillopacity{0.800000}%
\pgfsetlinewidth{0.000000pt}%
\definecolor{currentstroke}{rgb}{0.000000,0.000000,0.000000}%
\pgfsetstrokecolor{currentstroke}%
\pgfsetdash{}{0pt}%
\pgfpathmoveto{\pgfqpoint{3.443521in}{3.259823in}}%
\pgfpathlineto{\pgfqpoint{3.456870in}{3.243399in}}%
\pgfpathlineto{\pgfqpoint{3.470216in}{3.227256in}}%
\pgfpathlineto{\pgfqpoint{3.483558in}{3.211392in}}%
\pgfpathlineto{\pgfqpoint{3.496897in}{3.195805in}}%
\pgfpathlineto{\pgfqpoint{3.504699in}{3.212871in}}%
\pgfpathlineto{\pgfqpoint{3.512497in}{3.230154in}}%
\pgfpathlineto{\pgfqpoint{3.520290in}{3.247659in}}%
\pgfpathlineto{\pgfqpoint{3.528077in}{3.265392in}}%
\pgfpathlineto{\pgfqpoint{3.514743in}{3.281362in}}%
\pgfpathlineto{\pgfqpoint{3.501406in}{3.297609in}}%
\pgfpathlineto{\pgfqpoint{3.488066in}{3.314136in}}%
\pgfpathlineto{\pgfqpoint{3.474722in}{3.330944in}}%
\pgfpathlineto{\pgfqpoint{3.466930in}{3.312816in}}%
\pgfpathlineto{\pgfqpoint{3.459132in}{3.294923in}}%
\pgfpathlineto{\pgfqpoint{3.451329in}{3.277260in}}%
\pgfpathlineto{\pgfqpoint{3.443521in}{3.259823in}}%
\pgfpathclose%
\pgfusepath{fill}%
\end{pgfscope}%
\begin{pgfscope}%
\pgfpathrectangle{\pgfqpoint{1.150000in}{0.150000in}}{\pgfqpoint{5.700000in}{5.700000in}}%
\pgfusepath{clip}%
\pgfsetbuttcap%
\pgfsetroundjoin%
\definecolor{currentfill}{rgb}{0.259857,0.745492,0.444467}%
\pgfsetfillcolor{currentfill}%
\pgfsetfillopacity{0.800000}%
\pgfsetlinewidth{0.000000pt}%
\definecolor{currentstroke}{rgb}{0.000000,0.000000,0.000000}%
\pgfsetstrokecolor{currentstroke}%
\pgfsetdash{}{0pt}%
\pgfpathmoveto{\pgfqpoint{3.392429in}{4.311068in}}%
\pgfpathlineto{\pgfqpoint{3.405897in}{4.284935in}}%
\pgfpathlineto{\pgfqpoint{3.419356in}{4.259148in}}%
\pgfpathlineto{\pgfqpoint{3.432806in}{4.233703in}}%
\pgfpathlineto{\pgfqpoint{3.446248in}{4.208595in}}%
\pgfpathlineto{\pgfqpoint{3.453923in}{4.236887in}}%
\pgfpathlineto{\pgfqpoint{3.461594in}{4.265607in}}%
\pgfpathlineto{\pgfqpoint{3.469261in}{4.294764in}}%
\pgfpathlineto{\pgfqpoint{3.476923in}{4.324365in}}%
\pgfpathlineto{\pgfqpoint{3.463478in}{4.350121in}}%
\pgfpathlineto{\pgfqpoint{3.450024in}{4.376217in}}%
\pgfpathlineto{\pgfqpoint{3.436561in}{4.402657in}}%
\pgfpathlineto{\pgfqpoint{3.423088in}{4.429443in}}%
\pgfpathlineto{\pgfqpoint{3.415430in}{4.399176in}}%
\pgfpathlineto{\pgfqpoint{3.407768in}{4.369363in}}%
\pgfpathlineto{\pgfqpoint{3.400101in}{4.339996in}}%
\pgfpathlineto{\pgfqpoint{3.392429in}{4.311068in}}%
\pgfpathclose%
\pgfusepath{fill}%
\end{pgfscope}%
\begin{pgfscope}%
\pgfpathrectangle{\pgfqpoint{1.150000in}{0.150000in}}{\pgfqpoint{5.700000in}{5.700000in}}%
\pgfusepath{clip}%
\pgfsetbuttcap%
\pgfsetroundjoin%
\definecolor{currentfill}{rgb}{0.156270,0.489624,0.557936}%
\pgfsetfillcolor{currentfill}%
\pgfsetfillopacity{0.800000}%
\pgfsetlinewidth{0.000000pt}%
\definecolor{currentstroke}{rgb}{0.000000,0.000000,0.000000}%
\pgfsetstrokecolor{currentstroke}%
\pgfsetdash{}{0pt}%
\pgfpathmoveto{\pgfqpoint{3.314214in}{3.555536in}}%
\pgfpathlineto{\pgfqpoint{3.327622in}{3.535149in}}%
\pgfpathlineto{\pgfqpoint{3.341023in}{3.515074in}}%
\pgfpathlineto{\pgfqpoint{3.354418in}{3.495310in}}%
\pgfpathlineto{\pgfqpoint{3.367806in}{3.475853in}}%
\pgfpathlineto{\pgfqpoint{3.375602in}{3.495035in}}%
\pgfpathlineto{\pgfqpoint{3.383392in}{3.514479in}}%
\pgfpathlineto{\pgfqpoint{3.391177in}{3.534189in}}%
\pgfpathlineto{\pgfqpoint{3.398957in}{3.554170in}}%
\pgfpathlineto{\pgfqpoint{3.385572in}{3.574050in}}%
\pgfpathlineto{\pgfqpoint{3.372181in}{3.594237in}}%
\pgfpathlineto{\pgfqpoint{3.358784in}{3.614736in}}%
\pgfpathlineto{\pgfqpoint{3.345380in}{3.635548in}}%
\pgfpathlineto{\pgfqpoint{3.337597in}{3.615130in}}%
\pgfpathlineto{\pgfqpoint{3.329809in}{3.594992in}}%
\pgfpathlineto{\pgfqpoint{3.322014in}{3.575129in}}%
\pgfpathlineto{\pgfqpoint{3.314214in}{3.555536in}}%
\pgfpathclose%
\pgfusepath{fill}%
\end{pgfscope}%
\begin{pgfscope}%
\pgfpathrectangle{\pgfqpoint{1.150000in}{0.150000in}}{\pgfqpoint{5.700000in}{5.700000in}}%
\pgfusepath{clip}%
\pgfsetbuttcap%
\pgfsetroundjoin%
\definecolor{currentfill}{rgb}{0.174274,0.445044,0.557792}%
\pgfsetfillcolor{currentfill}%
\pgfsetfillopacity{0.800000}%
\pgfsetlinewidth{0.000000pt}%
\definecolor{currentstroke}{rgb}{0.000000,0.000000,0.000000}%
\pgfsetstrokecolor{currentstroke}%
\pgfsetdash{}{0pt}%
\pgfpathmoveto{\pgfqpoint{4.773072in}{3.403801in}}%
\pgfpathlineto{\pgfqpoint{4.786518in}{3.397092in}}%
\pgfpathlineto{\pgfqpoint{4.799970in}{3.390573in}}%
\pgfpathlineto{\pgfqpoint{4.813429in}{3.384242in}}%
\pgfpathlineto{\pgfqpoint{4.826896in}{3.378099in}}%
\pgfpathlineto{\pgfqpoint{4.834512in}{3.398763in}}%
\pgfpathlineto{\pgfqpoint{4.842133in}{3.419817in}}%
\pgfpathlineto{\pgfqpoint{4.849758in}{3.441269in}}%
\pgfpathlineto{\pgfqpoint{4.836300in}{3.447974in}}%
\pgfpathlineto{\pgfqpoint{4.822848in}{3.454866in}}%
\pgfpathlineto{\pgfqpoint{4.809404in}{3.461948in}}%
\pgfpathlineto{\pgfqpoint{4.795967in}{3.469219in}}%
\pgfpathlineto{\pgfqpoint{4.788331in}{3.447010in}}%
\pgfpathlineto{\pgfqpoint{4.780700in}{3.425207in}}%
\pgfpathlineto{\pgfqpoint{4.773072in}{3.403801in}}%
\pgfpathclose%
\pgfusepath{fill}%
\end{pgfscope}%
\begin{pgfscope}%
\pgfpathrectangle{\pgfqpoint{1.150000in}{0.150000in}}{\pgfqpoint{5.700000in}{5.700000in}}%
\pgfusepath{clip}%
\pgfsetbuttcap%
\pgfsetroundjoin%
\definecolor{currentfill}{rgb}{0.220057,0.343307,0.549413}%
\pgfsetfillcolor{currentfill}%
\pgfsetfillopacity{0.800000}%
\pgfsetlinewidth{0.000000pt}%
\definecolor{currentstroke}{rgb}{0.000000,0.000000,0.000000}%
\pgfsetstrokecolor{currentstroke}%
\pgfsetdash{}{0pt}%
\pgfpathmoveto{\pgfqpoint{3.550224in}{3.136183in}}%
\pgfpathlineto{\pgfqpoint{3.563550in}{3.121948in}}%
\pgfpathlineto{\pgfqpoint{3.576875in}{3.107977in}}%
\pgfpathlineto{\pgfqpoint{3.590197in}{3.094269in}}%
\pgfpathlineto{\pgfqpoint{3.603518in}{3.080821in}}%
\pgfpathlineto{\pgfqpoint{3.611309in}{3.097149in}}%
\pgfpathlineto{\pgfqpoint{3.619096in}{3.113678in}}%
\pgfpathlineto{\pgfqpoint{3.626878in}{3.130412in}}%
\pgfpathlineto{\pgfqpoint{3.634655in}{3.147358in}}%
\pgfpathlineto{\pgfqpoint{3.621340in}{3.161186in}}%
\pgfpathlineto{\pgfqpoint{3.608023in}{3.175274in}}%
\pgfpathlineto{\pgfqpoint{3.594704in}{3.189626in}}%
\pgfpathlineto{\pgfqpoint{3.581383in}{3.204241in}}%
\pgfpathlineto{\pgfqpoint{3.573601in}{3.186904in}}%
\pgfpathlineto{\pgfqpoint{3.565814in}{3.169784in}}%
\pgfpathlineto{\pgfqpoint{3.558021in}{3.152879in}}%
\pgfpathlineto{\pgfqpoint{3.550224in}{3.136183in}}%
\pgfpathclose%
\pgfusepath{fill}%
\end{pgfscope}%
\begin{pgfscope}%
\pgfpathrectangle{\pgfqpoint{1.150000in}{0.150000in}}{\pgfqpoint{5.700000in}{5.700000in}}%
\pgfusepath{clip}%
\pgfsetbuttcap%
\pgfsetroundjoin%
\definecolor{currentfill}{rgb}{0.218130,0.347432,0.550038}%
\pgfsetfillcolor{currentfill}%
\pgfsetfillopacity{0.800000}%
\pgfsetlinewidth{0.000000pt}%
\definecolor{currentstroke}{rgb}{0.000000,0.000000,0.000000}%
\pgfsetstrokecolor{currentstroke}%
\pgfsetdash{}{0pt}%
\pgfpathmoveto{\pgfqpoint{4.405921in}{3.127426in}}%
\pgfpathlineto{\pgfqpoint{4.419311in}{3.120880in}}%
\pgfpathlineto{\pgfqpoint{4.432706in}{3.114535in}}%
\pgfpathlineto{\pgfqpoint{4.446108in}{3.108390in}}%
\pgfpathlineto{\pgfqpoint{4.459516in}{3.102444in}}%
\pgfpathlineto{\pgfqpoint{4.467147in}{3.119358in}}%
\pgfpathlineto{\pgfqpoint{4.474778in}{3.136536in}}%
\pgfpathlineto{\pgfqpoint{4.482409in}{3.153986in}}%
\pgfpathlineto{\pgfqpoint{4.490039in}{3.171714in}}%
\pgfpathlineto{\pgfqpoint{4.476640in}{3.178254in}}%
\pgfpathlineto{\pgfqpoint{4.463246in}{3.184993in}}%
\pgfpathlineto{\pgfqpoint{4.449859in}{3.191932in}}%
\pgfpathlineto{\pgfqpoint{4.436477in}{3.199073in}}%
\pgfpathlineto{\pgfqpoint{4.428839in}{3.180739in}}%
\pgfpathlineto{\pgfqpoint{4.421200in}{3.162691in}}%
\pgfpathlineto{\pgfqpoint{4.413561in}{3.144922in}}%
\pgfpathlineto{\pgfqpoint{4.405921in}{3.127426in}}%
\pgfpathclose%
\pgfusepath{fill}%
\end{pgfscope}%
\begin{pgfscope}%
\pgfpathrectangle{\pgfqpoint{1.150000in}{0.150000in}}{\pgfqpoint{5.700000in}{5.700000in}}%
\pgfusepath{clip}%
\pgfsetbuttcap%
\pgfsetroundjoin%
\definecolor{currentfill}{rgb}{0.210503,0.363727,0.552206}%
\pgfsetfillcolor{currentfill}%
\pgfsetfillopacity{0.800000}%
\pgfsetlinewidth{0.000000pt}%
\definecolor{currentstroke}{rgb}{0.000000,0.000000,0.000000}%
\pgfsetstrokecolor{currentstroke}%
\pgfsetdash{}{0pt}%
\pgfpathmoveto{\pgfqpoint{4.490039in}{3.171714in}}%
\pgfpathlineto{\pgfqpoint{4.503444in}{3.165373in}}%
\pgfpathlineto{\pgfqpoint{4.516855in}{3.159229in}}%
\pgfpathlineto{\pgfqpoint{4.530273in}{3.153283in}}%
\pgfpathlineto{\pgfqpoint{4.543697in}{3.147532in}}%
\pgfpathlineto{\pgfqpoint{4.551318in}{3.164932in}}%
\pgfpathlineto{\pgfqpoint{4.558939in}{3.182616in}}%
\pgfpathlineto{\pgfqpoint{4.566560in}{3.200593in}}%
\pgfpathlineto{\pgfqpoint{4.574181in}{3.218869in}}%
\pgfpathlineto{\pgfqpoint{4.560766in}{3.225245in}}%
\pgfpathlineto{\pgfqpoint{4.547357in}{3.231817in}}%
\pgfpathlineto{\pgfqpoint{4.533955in}{3.238587in}}%
\pgfpathlineto{\pgfqpoint{4.520558in}{3.245554in}}%
\pgfpathlineto{\pgfqpoint{4.512928in}{3.226640in}}%
\pgfpathlineto{\pgfqpoint{4.505298in}{3.208034in}}%
\pgfpathlineto{\pgfqpoint{4.497669in}{3.189727in}}%
\pgfpathlineto{\pgfqpoint{4.490039in}{3.171714in}}%
\pgfpathclose%
\pgfusepath{fill}%
\end{pgfscope}%
\begin{pgfscope}%
\pgfpathrectangle{\pgfqpoint{1.150000in}{0.150000in}}{\pgfqpoint{5.700000in}{5.700000in}}%
\pgfusepath{clip}%
\pgfsetbuttcap%
\pgfsetroundjoin%
\definecolor{currentfill}{rgb}{0.225863,0.330805,0.547314}%
\pgfsetfillcolor{currentfill}%
\pgfsetfillopacity{0.800000}%
\pgfsetlinewidth{0.000000pt}%
\definecolor{currentstroke}{rgb}{0.000000,0.000000,0.000000}%
\pgfsetstrokecolor{currentstroke}%
\pgfsetdash{}{0pt}%
\pgfpathmoveto{\pgfqpoint{4.321813in}{3.085989in}}%
\pgfpathlineto{\pgfqpoint{4.335189in}{3.079193in}}%
\pgfpathlineto{\pgfqpoint{4.348570in}{3.072602in}}%
\pgfpathlineto{\pgfqpoint{4.361956in}{3.066214in}}%
\pgfpathlineto{\pgfqpoint{4.375349in}{3.060029in}}%
\pgfpathlineto{\pgfqpoint{4.382994in}{3.076504in}}%
\pgfpathlineto{\pgfqpoint{4.390637in}{3.093223in}}%
\pgfpathlineto{\pgfqpoint{4.398279in}{3.110195in}}%
\pgfpathlineto{\pgfqpoint{4.405921in}{3.127426in}}%
\pgfpathlineto{\pgfqpoint{4.392536in}{3.134174in}}%
\pgfpathlineto{\pgfqpoint{4.379158in}{3.141125in}}%
\pgfpathlineto{\pgfqpoint{4.365784in}{3.148279in}}%
\pgfpathlineto{\pgfqpoint{4.352416in}{3.155638in}}%
\pgfpathlineto{\pgfqpoint{4.344767in}{3.137832in}}%
\pgfpathlineto{\pgfqpoint{4.337117in}{3.120293in}}%
\pgfpathlineto{\pgfqpoint{4.329466in}{3.103014in}}%
\pgfpathlineto{\pgfqpoint{4.321813in}{3.085989in}}%
\pgfpathclose%
\pgfusepath{fill}%
\end{pgfscope}%
\begin{pgfscope}%
\pgfpathrectangle{\pgfqpoint{1.150000in}{0.150000in}}{\pgfqpoint{5.700000in}{5.700000in}}%
\pgfusepath{clip}%
\pgfsetbuttcap%
\pgfsetroundjoin%
\definecolor{currentfill}{rgb}{0.187231,0.414746,0.556547}%
\pgfsetfillcolor{currentfill}%
\pgfsetfillopacity{0.800000}%
\pgfsetlinewidth{0.000000pt}%
\definecolor{currentstroke}{rgb}{0.000000,0.000000,0.000000}%
\pgfsetstrokecolor{currentstroke}%
\pgfsetdash{}{0pt}%
\pgfpathmoveto{\pgfqpoint{3.390082in}{3.328383in}}%
\pgfpathlineto{\pgfqpoint{3.403449in}{3.310809in}}%
\pgfpathlineto{\pgfqpoint{3.416810in}{3.293526in}}%
\pgfpathlineto{\pgfqpoint{3.430168in}{3.276531in}}%
\pgfpathlineto{\pgfqpoint{3.443521in}{3.259823in}}%
\pgfpathlineto{\pgfqpoint{3.451329in}{3.277260in}}%
\pgfpathlineto{\pgfqpoint{3.459132in}{3.294923in}}%
\pgfpathlineto{\pgfqpoint{3.466930in}{3.312816in}}%
\pgfpathlineto{\pgfqpoint{3.474722in}{3.330944in}}%
\pgfpathlineto{\pgfqpoint{3.461374in}{3.348037in}}%
\pgfpathlineto{\pgfqpoint{3.448022in}{3.365417in}}%
\pgfpathlineto{\pgfqpoint{3.434665in}{3.383085in}}%
\pgfpathlineto{\pgfqpoint{3.421303in}{3.401046in}}%
\pgfpathlineto{\pgfqpoint{3.413506in}{3.382520in}}%
\pgfpathlineto{\pgfqpoint{3.405704in}{3.364237in}}%
\pgfpathlineto{\pgfqpoint{3.397896in}{3.346193in}}%
\pgfpathlineto{\pgfqpoint{3.390082in}{3.328383in}}%
\pgfpathclose%
\pgfusepath{fill}%
\end{pgfscope}%
\begin{pgfscope}%
\pgfpathrectangle{\pgfqpoint{1.150000in}{0.150000in}}{\pgfqpoint{5.700000in}{5.700000in}}%
\pgfusepath{clip}%
\pgfsetbuttcap%
\pgfsetroundjoin%
\definecolor{currentfill}{rgb}{0.201239,0.383670,0.554294}%
\pgfsetfillcolor{currentfill}%
\pgfsetfillopacity{0.800000}%
\pgfsetlinewidth{0.000000pt}%
\definecolor{currentstroke}{rgb}{0.000000,0.000000,0.000000}%
\pgfsetstrokecolor{currentstroke}%
\pgfsetdash{}{0pt}%
\pgfpathmoveto{\pgfqpoint{4.574181in}{3.218869in}}%
\pgfpathlineto{\pgfqpoint{4.587603in}{3.212689in}}%
\pgfpathlineto{\pgfqpoint{4.601031in}{3.206703in}}%
\pgfpathlineto{\pgfqpoint{4.614465in}{3.200911in}}%
\pgfpathlineto{\pgfqpoint{4.627907in}{3.195313in}}%
\pgfpathlineto{\pgfqpoint{4.635519in}{3.213251in}}%
\pgfpathlineto{\pgfqpoint{4.643132in}{3.231495in}}%
\pgfpathlineto{\pgfqpoint{4.650746in}{3.250054in}}%
\pgfpathlineto{\pgfqpoint{4.658362in}{3.268935in}}%
\pgfpathlineto{\pgfqpoint{4.644930in}{3.275190in}}%
\pgfpathlineto{\pgfqpoint{4.631505in}{3.281639in}}%
\pgfpathlineto{\pgfqpoint{4.618086in}{3.288282in}}%
\pgfpathlineto{\pgfqpoint{4.604674in}{3.295120in}}%
\pgfpathlineto{\pgfqpoint{4.597049in}{3.275570in}}%
\pgfpathlineto{\pgfqpoint{4.589426in}{3.256350in}}%
\pgfpathlineto{\pgfqpoint{4.581803in}{3.237452in}}%
\pgfpathlineto{\pgfqpoint{4.574181in}{3.218869in}}%
\pgfpathclose%
\pgfusepath{fill}%
\end{pgfscope}%
\begin{pgfscope}%
\pgfpathrectangle{\pgfqpoint{1.150000in}{0.150000in}}{\pgfqpoint{5.700000in}{5.700000in}}%
\pgfusepath{clip}%
\pgfsetbuttcap%
\pgfsetroundjoin%
\definecolor{currentfill}{rgb}{0.506271,0.828786,0.300362}%
\pgfsetfillcolor{currentfill}%
\pgfsetfillopacity{0.800000}%
\pgfsetlinewidth{0.000000pt}%
\definecolor{currentstroke}{rgb}{0.000000,0.000000,0.000000}%
\pgfsetstrokecolor{currentstroke}%
\pgfsetdash{}{0pt}%
\pgfpathmoveto{\pgfqpoint{3.622366in}{4.608830in}}%
\pgfpathlineto{\pgfqpoint{3.635788in}{4.582560in}}%
\pgfpathlineto{\pgfqpoint{3.649203in}{4.556617in}}%
\pgfpathlineto{\pgfqpoint{3.662610in}{4.530996in}}%
\pgfpathlineto{\pgfqpoint{3.676010in}{4.505694in}}%
\pgfpathlineto{\pgfqpoint{3.683647in}{4.540259in}}%
\pgfpathlineto{\pgfqpoint{3.691282in}{4.575369in}}%
\pgfpathlineto{\pgfqpoint{3.698915in}{4.611035in}}%
\pgfpathlineto{\pgfqpoint{3.706547in}{4.647267in}}%
\pgfpathlineto{\pgfqpoint{3.693139in}{4.673366in}}%
\pgfpathlineto{\pgfqpoint{3.679723in}{4.699787in}}%
\pgfpathlineto{\pgfqpoint{3.666300in}{4.726533in}}%
\pgfpathlineto{\pgfqpoint{3.652868in}{4.753606in}}%
\pgfpathlineto{\pgfqpoint{3.645245in}{4.716558in}}%
\pgfpathlineto{\pgfqpoint{3.637621in}{4.680085in}}%
\pgfpathlineto{\pgfqpoint{3.629994in}{4.644179in}}%
\pgfpathlineto{\pgfqpoint{3.622366in}{4.608830in}}%
\pgfpathclose%
\pgfusepath{fill}%
\end{pgfscope}%
\begin{pgfscope}%
\pgfpathrectangle{\pgfqpoint{1.150000in}{0.150000in}}{\pgfqpoint{5.700000in}{5.700000in}}%
\pgfusepath{clip}%
\pgfsetbuttcap%
\pgfsetroundjoin%
\definecolor{currentfill}{rgb}{0.227802,0.326594,0.546532}%
\pgfsetfillcolor{currentfill}%
\pgfsetfillopacity{0.800000}%
\pgfsetlinewidth{0.000000pt}%
\definecolor{currentstroke}{rgb}{0.000000,0.000000,0.000000}%
\pgfsetstrokecolor{currentstroke}%
\pgfsetdash{}{0pt}%
\pgfpathmoveto{\pgfqpoint{3.603518in}{3.080821in}}%
\pgfpathlineto{\pgfqpoint{3.616838in}{3.067632in}}%
\pgfpathlineto{\pgfqpoint{3.630156in}{3.054699in}}%
\pgfpathlineto{\pgfqpoint{3.643474in}{3.042020in}}%
\pgfpathlineto{\pgfqpoint{3.656792in}{3.029595in}}%
\pgfpathlineto{\pgfqpoint{3.664577in}{3.045555in}}%
\pgfpathlineto{\pgfqpoint{3.672358in}{3.061709in}}%
\pgfpathlineto{\pgfqpoint{3.680134in}{3.078060in}}%
\pgfpathlineto{\pgfqpoint{3.687906in}{3.094614in}}%
\pgfpathlineto{\pgfqpoint{3.674595in}{3.107418in}}%
\pgfpathlineto{\pgfqpoint{3.661283in}{3.120476in}}%
\pgfpathlineto{\pgfqpoint{3.647970in}{3.133788in}}%
\pgfpathlineto{\pgfqpoint{3.634655in}{3.147358in}}%
\pgfpathlineto{\pgfqpoint{3.626878in}{3.130412in}}%
\pgfpathlineto{\pgfqpoint{3.619096in}{3.113678in}}%
\pgfpathlineto{\pgfqpoint{3.611309in}{3.097149in}}%
\pgfpathlineto{\pgfqpoint{3.603518in}{3.080821in}}%
\pgfpathclose%
\pgfusepath{fill}%
\end{pgfscope}%
\begin{pgfscope}%
\pgfpathrectangle{\pgfqpoint{1.150000in}{0.150000in}}{\pgfqpoint{5.700000in}{5.700000in}}%
\pgfusepath{clip}%
\pgfsetbuttcap%
\pgfsetroundjoin%
\definecolor{currentfill}{rgb}{0.231674,0.318106,0.544834}%
\pgfsetfillcolor{currentfill}%
\pgfsetfillopacity{0.800000}%
\pgfsetlinewidth{0.000000pt}%
\definecolor{currentstroke}{rgb}{0.000000,0.000000,0.000000}%
\pgfsetstrokecolor{currentstroke}%
\pgfsetdash{}{0pt}%
\pgfpathmoveto{\pgfqpoint{4.237702in}{3.047412in}}%
\pgfpathlineto{\pgfqpoint{4.251065in}{3.040321in}}%
\pgfpathlineto{\pgfqpoint{4.264433in}{3.033439in}}%
\pgfpathlineto{\pgfqpoint{4.277806in}{3.026763in}}%
\pgfpathlineto{\pgfqpoint{4.291184in}{3.020293in}}%
\pgfpathlineto{\pgfqpoint{4.298844in}{3.036369in}}%
\pgfpathlineto{\pgfqpoint{4.306502in}{3.052673in}}%
\pgfpathlineto{\pgfqpoint{4.314158in}{3.069211in}}%
\pgfpathlineto{\pgfqpoint{4.321813in}{3.085989in}}%
\pgfpathlineto{\pgfqpoint{4.308442in}{3.092990in}}%
\pgfpathlineto{\pgfqpoint{4.295077in}{3.100197in}}%
\pgfpathlineto{\pgfqpoint{4.281716in}{3.107612in}}%
\pgfpathlineto{\pgfqpoint{4.268361in}{3.115235in}}%
\pgfpathlineto{\pgfqpoint{4.260699in}{3.097913in}}%
\pgfpathlineto{\pgfqpoint{4.253035in}{3.080839in}}%
\pgfpathlineto{\pgfqpoint{4.245370in}{3.064008in}}%
\pgfpathlineto{\pgfqpoint{4.237702in}{3.047412in}}%
\pgfpathclose%
\pgfusepath{fill}%
\end{pgfscope}%
\begin{pgfscope}%
\pgfpathrectangle{\pgfqpoint{1.150000in}{0.150000in}}{\pgfqpoint{5.700000in}{5.700000in}}%
\pgfusepath{clip}%
\pgfsetbuttcap%
\pgfsetroundjoin%
\definecolor{currentfill}{rgb}{0.119423,0.611141,0.538982}%
\pgfsetfillcolor{currentfill}%
\pgfsetfillopacity{0.800000}%
\pgfsetlinewidth{0.000000pt}%
\definecolor{currentstroke}{rgb}{0.000000,0.000000,0.000000}%
\pgfsetstrokecolor{currentstroke}%
\pgfsetdash{}{0pt}%
\pgfpathmoveto{\pgfqpoint{3.268968in}{3.901990in}}%
\pgfpathlineto{\pgfqpoint{3.282433in}{3.878091in}}%
\pgfpathlineto{\pgfqpoint{3.295890in}{3.854531in}}%
\pgfpathlineto{\pgfqpoint{3.309337in}{3.831307in}}%
\pgfpathlineto{\pgfqpoint{3.322777in}{3.808417in}}%
\pgfpathlineto{\pgfqpoint{3.330533in}{3.830799in}}%
\pgfpathlineto{\pgfqpoint{3.338285in}{3.853502in}}%
\pgfpathlineto{\pgfqpoint{3.346030in}{3.876532in}}%
\pgfpathlineto{\pgfqpoint{3.353770in}{3.899896in}}%
\pgfpathlineto{\pgfqpoint{3.340332in}{3.923286in}}%
\pgfpathlineto{\pgfqpoint{3.326886in}{3.947009in}}%
\pgfpathlineto{\pgfqpoint{3.313431in}{3.971070in}}%
\pgfpathlineto{\pgfqpoint{3.299967in}{3.995472in}}%
\pgfpathlineto{\pgfqpoint{3.292226in}{3.971594in}}%
\pgfpathlineto{\pgfqpoint{3.284479in}{3.948058in}}%
\pgfpathlineto{\pgfqpoint{3.276726in}{3.924859in}}%
\pgfpathlineto{\pgfqpoint{3.268968in}{3.901990in}}%
\pgfpathclose%
\pgfusepath{fill}%
\end{pgfscope}%
\begin{pgfscope}%
\pgfpathrectangle{\pgfqpoint{1.150000in}{0.150000in}}{\pgfqpoint{5.700000in}{5.700000in}}%
\pgfusepath{clip}%
\pgfsetbuttcap%
\pgfsetroundjoin%
\definecolor{currentfill}{rgb}{0.241237,0.296485,0.539709}%
\pgfsetfillcolor{currentfill}%
\pgfsetfillopacity{0.800000}%
\pgfsetlinewidth{0.000000pt}%
\definecolor{currentstroke}{rgb}{0.000000,0.000000,0.000000}%
\pgfsetstrokecolor{currentstroke}%
\pgfsetdash{}{0pt}%
\pgfpathmoveto{\pgfqpoint{3.794397in}{3.001086in}}%
\pgfpathlineto{\pgfqpoint{3.807712in}{2.990483in}}%
\pgfpathlineto{\pgfqpoint{3.821027in}{2.980118in}}%
\pgfpathlineto{\pgfqpoint{3.834344in}{2.969987in}}%
\pgfpathlineto{\pgfqpoint{3.847662in}{2.960091in}}%
\pgfpathlineto{\pgfqpoint{3.855413in}{2.975659in}}%
\pgfpathlineto{\pgfqpoint{3.863159in}{2.991416in}}%
\pgfpathlineto{\pgfqpoint{3.870901in}{3.007365in}}%
\pgfpathlineto{\pgfqpoint{3.878639in}{3.023511in}}%
\pgfpathlineto{\pgfqpoint{3.865327in}{3.033816in}}%
\pgfpathlineto{\pgfqpoint{3.852016in}{3.044355in}}%
\pgfpathlineto{\pgfqpoint{3.838706in}{3.055129in}}%
\pgfpathlineto{\pgfqpoint{3.825398in}{3.066140in}}%
\pgfpathlineto{\pgfqpoint{3.817654in}{3.049573in}}%
\pgfpathlineto{\pgfqpoint{3.809906in}{3.033211in}}%
\pgfpathlineto{\pgfqpoint{3.802154in}{3.017051in}}%
\pgfpathlineto{\pgfqpoint{3.794397in}{3.001086in}}%
\pgfpathclose%
\pgfusepath{fill}%
\end{pgfscope}%
\begin{pgfscope}%
\pgfpathrectangle{\pgfqpoint{1.150000in}{0.150000in}}{\pgfqpoint{5.700000in}{5.700000in}}%
\pgfusepath{clip}%
\pgfsetbuttcap%
\pgfsetroundjoin%
\definecolor{currentfill}{rgb}{0.192357,0.403199,0.555836}%
\pgfsetfillcolor{currentfill}%
\pgfsetfillopacity{0.800000}%
\pgfsetlinewidth{0.000000pt}%
\definecolor{currentstroke}{rgb}{0.000000,0.000000,0.000000}%
\pgfsetstrokecolor{currentstroke}%
\pgfsetdash{}{0pt}%
\pgfpathmoveto{\pgfqpoint{4.658362in}{3.268935in}}%
\pgfpathlineto{\pgfqpoint{4.671800in}{3.262872in}}%
\pgfpathlineto{\pgfqpoint{4.685246in}{3.257001in}}%
\pgfpathlineto{\pgfqpoint{4.698698in}{3.251321in}}%
\pgfpathlineto{\pgfqpoint{4.712157in}{3.245832in}}%
\pgfpathlineto{\pgfqpoint{4.719764in}{3.264366in}}%
\pgfpathlineto{\pgfqpoint{4.727372in}{3.283230in}}%
\pgfpathlineto{\pgfqpoint{4.734982in}{3.302432in}}%
\pgfpathlineto{\pgfqpoint{4.742595in}{3.321980in}}%
\pgfpathlineto{\pgfqpoint{4.729146in}{3.328157in}}%
\pgfpathlineto{\pgfqpoint{4.715704in}{3.334526in}}%
\pgfpathlineto{\pgfqpoint{4.702269in}{3.341086in}}%
\pgfpathlineto{\pgfqpoint{4.688840in}{3.347838in}}%
\pgfpathlineto{\pgfqpoint{4.681218in}{3.327589in}}%
\pgfpathlineto{\pgfqpoint{4.673597in}{3.307694in}}%
\pgfpathlineto{\pgfqpoint{4.665979in}{3.288145in}}%
\pgfpathlineto{\pgfqpoint{4.658362in}{3.268935in}}%
\pgfpathclose%
\pgfusepath{fill}%
\end{pgfscope}%
\begin{pgfscope}%
\pgfpathrectangle{\pgfqpoint{1.150000in}{0.150000in}}{\pgfqpoint{5.700000in}{5.700000in}}%
\pgfusepath{clip}%
\pgfsetbuttcap%
\pgfsetroundjoin%
\definecolor{currentfill}{rgb}{0.344074,0.780029,0.397381}%
\pgfsetfillcolor{currentfill}%
\pgfsetfillopacity{0.800000}%
\pgfsetlinewidth{0.000000pt}%
\definecolor{currentstroke}{rgb}{0.000000,0.000000,0.000000}%
\pgfsetstrokecolor{currentstroke}%
\pgfsetdash{}{0pt}%
\pgfpathmoveto{\pgfqpoint{3.423088in}{4.429443in}}%
\pgfpathlineto{\pgfqpoint{3.436561in}{4.402657in}}%
\pgfpathlineto{\pgfqpoint{3.450024in}{4.376217in}}%
\pgfpathlineto{\pgfqpoint{3.463478in}{4.350121in}}%
\pgfpathlineto{\pgfqpoint{3.476923in}{4.324365in}}%
\pgfpathlineto{\pgfqpoint{3.484582in}{4.354418in}}%
\pgfpathlineto{\pgfqpoint{3.492236in}{4.384930in}}%
\pgfpathlineto{\pgfqpoint{3.499887in}{4.415912in}}%
\pgfpathlineto{\pgfqpoint{3.507534in}{4.447369in}}%
\pgfpathlineto{\pgfqpoint{3.494084in}{4.473814in}}%
\pgfpathlineto{\pgfqpoint{3.480625in}{4.500601in}}%
\pgfpathlineto{\pgfqpoint{3.467156in}{4.527732in}}%
\pgfpathlineto{\pgfqpoint{3.453678in}{4.555212in}}%
\pgfpathlineto{\pgfqpoint{3.446037in}{4.523048in}}%
\pgfpathlineto{\pgfqpoint{3.438391in}{4.491370in}}%
\pgfpathlineto{\pgfqpoint{3.430742in}{4.460171in}}%
\pgfpathlineto{\pgfqpoint{3.423088in}{4.429443in}}%
\pgfpathclose%
\pgfusepath{fill}%
\end{pgfscope}%
\begin{pgfscope}%
\pgfpathrectangle{\pgfqpoint{1.150000in}{0.150000in}}{\pgfqpoint{5.700000in}{5.700000in}}%
\pgfusepath{clip}%
\pgfsetbuttcap%
\pgfsetroundjoin%
\definecolor{currentfill}{rgb}{0.243113,0.292092,0.538516}%
\pgfsetfillcolor{currentfill}%
\pgfsetfillopacity{0.800000}%
\pgfsetlinewidth{0.000000pt}%
\definecolor{currentstroke}{rgb}{0.000000,0.000000,0.000000}%
\pgfsetstrokecolor{currentstroke}%
\pgfsetdash{}{0pt}%
\pgfpathmoveto{\pgfqpoint{3.931909in}{2.984604in}}%
\pgfpathlineto{\pgfqpoint{3.945232in}{2.975448in}}%
\pgfpathlineto{\pgfqpoint{3.958557in}{2.966518in}}%
\pgfpathlineto{\pgfqpoint{3.971886in}{2.957813in}}%
\pgfpathlineto{\pgfqpoint{3.985217in}{2.949331in}}%
\pgfpathlineto{\pgfqpoint{3.992939in}{2.964825in}}%
\pgfpathlineto{\pgfqpoint{4.000658in}{2.980512in}}%
\pgfpathlineto{\pgfqpoint{4.008374in}{2.996396in}}%
\pgfpathlineto{\pgfqpoint{4.016086in}{3.012482in}}%
\pgfpathlineto{\pgfqpoint{4.002761in}{3.021403in}}%
\pgfpathlineto{\pgfqpoint{3.989439in}{3.030547in}}%
\pgfpathlineto{\pgfqpoint{3.976119in}{3.039916in}}%
\pgfpathlineto{\pgfqpoint{3.962802in}{3.049511in}}%
\pgfpathlineto{\pgfqpoint{3.955084in}{3.032974in}}%
\pgfpathlineto{\pgfqpoint{3.947362in}{3.016647in}}%
\pgfpathlineto{\pgfqpoint{3.939637in}{3.000526in}}%
\pgfpathlineto{\pgfqpoint{3.931909in}{2.984604in}}%
\pgfpathclose%
\pgfusepath{fill}%
\end{pgfscope}%
\begin{pgfscope}%
\pgfpathrectangle{\pgfqpoint{1.150000in}{0.150000in}}{\pgfqpoint{5.700000in}{5.700000in}}%
\pgfusepath{clip}%
\pgfsetbuttcap%
\pgfsetroundjoin%
\definecolor{currentfill}{rgb}{0.175841,0.441290,0.557685}%
\pgfsetfillcolor{currentfill}%
\pgfsetfillopacity{0.800000}%
\pgfsetlinewidth{0.000000pt}%
\definecolor{currentstroke}{rgb}{0.000000,0.000000,0.000000}%
\pgfsetstrokecolor{currentstroke}%
\pgfsetdash{}{0pt}%
\pgfpathmoveto{\pgfqpoint{3.336565in}{3.401641in}}%
\pgfpathlineto{\pgfqpoint{3.349953in}{3.382877in}}%
\pgfpathlineto{\pgfqpoint{3.363334in}{3.364414in}}%
\pgfpathlineto{\pgfqpoint{3.376711in}{3.346250in}}%
\pgfpathlineto{\pgfqpoint{3.390082in}{3.328383in}}%
\pgfpathlineto{\pgfqpoint{3.397896in}{3.346193in}}%
\pgfpathlineto{\pgfqpoint{3.405704in}{3.364237in}}%
\pgfpathlineto{\pgfqpoint{3.413506in}{3.382520in}}%
\pgfpathlineto{\pgfqpoint{3.421303in}{3.401046in}}%
\pgfpathlineto{\pgfqpoint{3.407937in}{3.419300in}}%
\pgfpathlineto{\pgfqpoint{3.394566in}{3.437851in}}%
\pgfpathlineto{\pgfqpoint{3.381189in}{3.456701in}}%
\pgfpathlineto{\pgfqpoint{3.367806in}{3.475853in}}%
\pgfpathlineto{\pgfqpoint{3.360004in}{3.456927in}}%
\pgfpathlineto{\pgfqpoint{3.352197in}{3.438253in}}%
\pgfpathlineto{\pgfqpoint{3.344384in}{3.419826in}}%
\pgfpathlineto{\pgfqpoint{3.336565in}{3.401641in}}%
\pgfpathclose%
\pgfusepath{fill}%
\end{pgfscope}%
\begin{pgfscope}%
\pgfpathrectangle{\pgfqpoint{1.150000in}{0.150000in}}{\pgfqpoint{5.700000in}{5.700000in}}%
\pgfusepath{clip}%
\pgfsetbuttcap%
\pgfsetroundjoin%
\definecolor{currentfill}{rgb}{0.477504,0.821444,0.318195}%
\pgfsetfillcolor{currentfill}%
\pgfsetfillopacity{0.800000}%
\pgfsetlinewidth{0.000000pt}%
\definecolor{currentstroke}{rgb}{0.000000,0.000000,0.000000}%
\pgfsetstrokecolor{currentstroke}%
\pgfsetdash{}{0pt}%
\pgfpathmoveto{\pgfqpoint{3.538088in}{4.578136in}}%
\pgfpathlineto{\pgfqpoint{3.551536in}{4.551301in}}%
\pgfpathlineto{\pgfqpoint{3.564976in}{4.524803in}}%
\pgfpathlineto{\pgfqpoint{3.578407in}{4.498639in}}%
\pgfpathlineto{\pgfqpoint{3.591830in}{4.472804in}}%
\pgfpathlineto{\pgfqpoint{3.599468in}{4.506023in}}%
\pgfpathlineto{\pgfqpoint{3.607103in}{4.539761in}}%
\pgfpathlineto{\pgfqpoint{3.614736in}{4.574027in}}%
\pgfpathlineto{\pgfqpoint{3.622366in}{4.608830in}}%
\pgfpathlineto{\pgfqpoint{3.608935in}{4.635428in}}%
\pgfpathlineto{\pgfqpoint{3.595497in}{4.662358in}}%
\pgfpathlineto{\pgfqpoint{3.582049in}{4.689623in}}%
\pgfpathlineto{\pgfqpoint{3.568593in}{4.717227in}}%
\pgfpathlineto{\pgfqpoint{3.560971in}{4.681641in}}%
\pgfpathlineto{\pgfqpoint{3.553346in}{4.646604in}}%
\pgfpathlineto{\pgfqpoint{3.545719in}{4.612105in}}%
\pgfpathlineto{\pgfqpoint{3.538088in}{4.578136in}}%
\pgfpathclose%
\pgfusepath{fill}%
\end{pgfscope}%
\begin{pgfscope}%
\pgfpathrectangle{\pgfqpoint{1.150000in}{0.150000in}}{\pgfqpoint{5.700000in}{5.700000in}}%
\pgfusepath{clip}%
\pgfsetbuttcap%
\pgfsetroundjoin%
\definecolor{currentfill}{rgb}{0.237441,0.305202,0.541921}%
\pgfsetfillcolor{currentfill}%
\pgfsetfillopacity{0.800000}%
\pgfsetlinewidth{0.000000pt}%
\definecolor{currentstroke}{rgb}{0.000000,0.000000,0.000000}%
\pgfsetstrokecolor{currentstroke}%
\pgfsetdash{}{0pt}%
\pgfpathmoveto{\pgfqpoint{4.153575in}{3.011731in}}%
\pgfpathlineto{\pgfqpoint{4.166927in}{3.004298in}}%
\pgfpathlineto{\pgfqpoint{4.180283in}{2.997078in}}%
\pgfpathlineto{\pgfqpoint{4.193643in}{2.990068in}}%
\pgfpathlineto{\pgfqpoint{4.207009in}{2.983269in}}%
\pgfpathlineto{\pgfqpoint{4.214686in}{2.998981in}}%
\pgfpathlineto{\pgfqpoint{4.222360in}{3.014905in}}%
\pgfpathlineto{\pgfqpoint{4.230033in}{3.031047in}}%
\pgfpathlineto{\pgfqpoint{4.237702in}{3.047412in}}%
\pgfpathlineto{\pgfqpoint{4.224344in}{3.054712in}}%
\pgfpathlineto{\pgfqpoint{4.210991in}{3.062222in}}%
\pgfpathlineto{\pgfqpoint{4.197642in}{3.069944in}}%
\pgfpathlineto{\pgfqpoint{4.184297in}{3.077877in}}%
\pgfpathlineto{\pgfqpoint{4.176620in}{3.060999in}}%
\pgfpathlineto{\pgfqpoint{4.168941in}{3.044353in}}%
\pgfpathlineto{\pgfqpoint{4.161260in}{3.027932in}}%
\pgfpathlineto{\pgfqpoint{4.153575in}{3.011731in}}%
\pgfpathclose%
\pgfusepath{fill}%
\end{pgfscope}%
\begin{pgfscope}%
\pgfpathrectangle{\pgfqpoint{1.150000in}{0.150000in}}{\pgfqpoint{5.700000in}{5.700000in}}%
\pgfusepath{clip}%
\pgfsetbuttcap%
\pgfsetroundjoin%
\definecolor{currentfill}{rgb}{0.237441,0.305202,0.541921}%
\pgfsetfillcolor{currentfill}%
\pgfsetfillopacity{0.800000}%
\pgfsetlinewidth{0.000000pt}%
\definecolor{currentstroke}{rgb}{0.000000,0.000000,0.000000}%
\pgfsetstrokecolor{currentstroke}%
\pgfsetdash{}{0pt}%
\pgfpathmoveto{\pgfqpoint{3.656792in}{3.029595in}}%
\pgfpathlineto{\pgfqpoint{3.670108in}{3.017421in}}%
\pgfpathlineto{\pgfqpoint{3.683425in}{3.005496in}}%
\pgfpathlineto{\pgfqpoint{3.696741in}{2.993819in}}%
\pgfpathlineto{\pgfqpoint{3.710058in}{2.982387in}}%
\pgfpathlineto{\pgfqpoint{3.717838in}{2.997981in}}%
\pgfpathlineto{\pgfqpoint{3.725613in}{3.013760in}}%
\pgfpathlineto{\pgfqpoint{3.733383in}{3.029729in}}%
\pgfpathlineto{\pgfqpoint{3.741149in}{3.045893in}}%
\pgfpathlineto{\pgfqpoint{3.727839in}{3.057702in}}%
\pgfpathlineto{\pgfqpoint{3.714528in}{3.069758in}}%
\pgfpathlineto{\pgfqpoint{3.701217in}{3.082061in}}%
\pgfpathlineto{\pgfqpoint{3.687906in}{3.094614in}}%
\pgfpathlineto{\pgfqpoint{3.680134in}{3.078060in}}%
\pgfpathlineto{\pgfqpoint{3.672358in}{3.061709in}}%
\pgfpathlineto{\pgfqpoint{3.664577in}{3.045555in}}%
\pgfpathlineto{\pgfqpoint{3.656792in}{3.029595in}}%
\pgfpathclose%
\pgfusepath{fill}%
\end{pgfscope}%
\begin{pgfscope}%
\pgfpathrectangle{\pgfqpoint{1.150000in}{0.150000in}}{\pgfqpoint{5.700000in}{5.700000in}}%
\pgfusepath{clip}%
\pgfsetbuttcap%
\pgfsetroundjoin%
\definecolor{currentfill}{rgb}{0.143343,0.522773,0.556295}%
\pgfsetfillcolor{currentfill}%
\pgfsetfillopacity{0.800000}%
\pgfsetlinewidth{0.000000pt}%
\definecolor{currentstroke}{rgb}{0.000000,0.000000,0.000000}%
\pgfsetstrokecolor{currentstroke}%
\pgfsetdash{}{0pt}%
\pgfpathmoveto{\pgfqpoint{3.260510in}{3.640277in}}%
\pgfpathlineto{\pgfqpoint{3.273947in}{3.618607in}}%
\pgfpathlineto{\pgfqpoint{3.287377in}{3.597262in}}%
\pgfpathlineto{\pgfqpoint{3.300799in}{3.576240in}}%
\pgfpathlineto{\pgfqpoint{3.314214in}{3.555536in}}%
\pgfpathlineto{\pgfqpoint{3.322014in}{3.575129in}}%
\pgfpathlineto{\pgfqpoint{3.329809in}{3.594992in}}%
\pgfpathlineto{\pgfqpoint{3.337597in}{3.615130in}}%
\pgfpathlineto{\pgfqpoint{3.345380in}{3.635548in}}%
\pgfpathlineto{\pgfqpoint{3.331969in}{3.656677in}}%
\pgfpathlineto{\pgfqpoint{3.318551in}{3.678126in}}%
\pgfpathlineto{\pgfqpoint{3.305125in}{3.699897in}}%
\pgfpathlineto{\pgfqpoint{3.291691in}{3.721995in}}%
\pgfpathlineto{\pgfqpoint{3.283905in}{3.701137in}}%
\pgfpathlineto{\pgfqpoint{3.276113in}{3.680569in}}%
\pgfpathlineto{\pgfqpoint{3.268314in}{3.660283in}}%
\pgfpathlineto{\pgfqpoint{3.260510in}{3.640277in}}%
\pgfpathclose%
\pgfusepath{fill}%
\end{pgfscope}%
\begin{pgfscope}%
\pgfpathrectangle{\pgfqpoint{1.150000in}{0.150000in}}{\pgfqpoint{5.700000in}{5.700000in}}%
\pgfusepath{clip}%
\pgfsetbuttcap%
\pgfsetroundjoin%
\definecolor{currentfill}{rgb}{0.183898,0.422383,0.556944}%
\pgfsetfillcolor{currentfill}%
\pgfsetfillopacity{0.800000}%
\pgfsetlinewidth{0.000000pt}%
\definecolor{currentstroke}{rgb}{0.000000,0.000000,0.000000}%
\pgfsetstrokecolor{currentstroke}%
\pgfsetdash{}{0pt}%
\pgfpathmoveto{\pgfqpoint{4.742595in}{3.321980in}}%
\pgfpathlineto{\pgfqpoint{4.756051in}{3.315992in}}%
\pgfpathlineto{\pgfqpoint{4.769514in}{3.310193in}}%
\pgfpathlineto{\pgfqpoint{4.782984in}{3.304583in}}%
\pgfpathlineto{\pgfqpoint{4.796462in}{3.299161in}}%
\pgfpathlineto{\pgfqpoint{4.804066in}{3.318355in}}%
\pgfpathlineto{\pgfqpoint{4.811673in}{3.337904in}}%
\pgfpathlineto{\pgfqpoint{4.819283in}{3.357815in}}%
\pgfpathlineto{\pgfqpoint{4.826896in}{3.378099in}}%
\pgfpathlineto{\pgfqpoint{4.813429in}{3.384242in}}%
\pgfpathlineto{\pgfqpoint{4.799970in}{3.390573in}}%
\pgfpathlineto{\pgfqpoint{4.786518in}{3.397092in}}%
\pgfpathlineto{\pgfqpoint{4.773072in}{3.403801in}}%
\pgfpathlineto{\pgfqpoint{4.765448in}{3.382784in}}%
\pgfpathlineto{\pgfqpoint{4.757828in}{3.362147in}}%
\pgfpathlineto{\pgfqpoint{4.750210in}{3.341882in}}%
\pgfpathlineto{\pgfqpoint{4.742595in}{3.321980in}}%
\pgfpathclose%
\pgfusepath{fill}%
\end{pgfscope}%
\begin{pgfscope}%
\pgfpathrectangle{\pgfqpoint{1.150000in}{0.150000in}}{\pgfqpoint{5.700000in}{5.700000in}}%
\pgfusepath{clip}%
\pgfsetbuttcap%
\pgfsetroundjoin%
\definecolor{currentfill}{rgb}{0.123463,0.581687,0.547445}%
\pgfsetfillcolor{currentfill}%
\pgfsetfillopacity{0.800000}%
\pgfsetlinewidth{0.000000pt}%
\definecolor{currentstroke}{rgb}{0.000000,0.000000,0.000000}%
\pgfsetstrokecolor{currentstroke}%
\pgfsetdash{}{0pt}%
\pgfpathmoveto{\pgfqpoint{3.237872in}{3.813708in}}%
\pgfpathlineto{\pgfqpoint{3.251340in}{3.790275in}}%
\pgfpathlineto{\pgfqpoint{3.264799in}{3.767180in}}%
\pgfpathlineto{\pgfqpoint{3.278249in}{3.744421in}}%
\pgfpathlineto{\pgfqpoint{3.291691in}{3.721995in}}%
\pgfpathlineto{\pgfqpoint{3.299471in}{3.743146in}}%
\pgfpathlineto{\pgfqpoint{3.307245in}{3.764597in}}%
\pgfpathlineto{\pgfqpoint{3.315014in}{3.786352in}}%
\pgfpathlineto{\pgfqpoint{3.322777in}{3.808417in}}%
\pgfpathlineto{\pgfqpoint{3.309337in}{3.831307in}}%
\pgfpathlineto{\pgfqpoint{3.295890in}{3.854531in}}%
\pgfpathlineto{\pgfqpoint{3.282433in}{3.878091in}}%
\pgfpathlineto{\pgfqpoint{3.268968in}{3.901990in}}%
\pgfpathlineto{\pgfqpoint{3.261203in}{3.879446in}}%
\pgfpathlineto{\pgfqpoint{3.253432in}{3.857222in}}%
\pgfpathlineto{\pgfqpoint{3.245655in}{3.835311in}}%
\pgfpathlineto{\pgfqpoint{3.237872in}{3.813708in}}%
\pgfpathclose%
\pgfusepath{fill}%
\end{pgfscope}%
\begin{pgfscope}%
\pgfpathrectangle{\pgfqpoint{1.150000in}{0.150000in}}{\pgfqpoint{5.700000in}{5.700000in}}%
\pgfusepath{clip}%
\pgfsetbuttcap%
\pgfsetroundjoin%
\definecolor{currentfill}{rgb}{0.243113,0.292092,0.538516}%
\pgfsetfillcolor{currentfill}%
\pgfsetfillopacity{0.800000}%
\pgfsetlinewidth{0.000000pt}%
\definecolor{currentstroke}{rgb}{0.000000,0.000000,0.000000}%
\pgfsetstrokecolor{currentstroke}%
\pgfsetdash{}{0pt}%
\pgfpathmoveto{\pgfqpoint{4.069418in}{2.979008in}}%
\pgfpathlineto{\pgfqpoint{4.082760in}{2.971186in}}%
\pgfpathlineto{\pgfqpoint{4.096106in}{2.963580in}}%
\pgfpathlineto{\pgfqpoint{4.109456in}{2.956189in}}%
\pgfpathlineto{\pgfqpoint{4.122810in}{2.949013in}}%
\pgfpathlineto{\pgfqpoint{4.130506in}{2.964391in}}%
\pgfpathlineto{\pgfqpoint{4.138199in}{2.979966in}}%
\pgfpathlineto{\pgfqpoint{4.145889in}{2.995744in}}%
\pgfpathlineto{\pgfqpoint{4.153575in}{3.011731in}}%
\pgfpathlineto{\pgfqpoint{4.140228in}{3.019377in}}%
\pgfpathlineto{\pgfqpoint{4.126885in}{3.027237in}}%
\pgfpathlineto{\pgfqpoint{4.113546in}{3.035312in}}%
\pgfpathlineto{\pgfqpoint{4.100210in}{3.043604in}}%
\pgfpathlineto{\pgfqpoint{4.092517in}{3.027136in}}%
\pgfpathlineto{\pgfqpoint{4.084820in}{3.010884in}}%
\pgfpathlineto{\pgfqpoint{4.077121in}{2.994843in}}%
\pgfpathlineto{\pgfqpoint{4.069418in}{2.979008in}}%
\pgfpathclose%
\pgfusepath{fill}%
\end{pgfscope}%
\begin{pgfscope}%
\pgfpathrectangle{\pgfqpoint{1.150000in}{0.150000in}}{\pgfqpoint{5.700000in}{5.700000in}}%
\pgfusepath{clip}%
\pgfsetbuttcap%
\pgfsetroundjoin%
\definecolor{currentfill}{rgb}{0.165117,0.467423,0.558141}%
\pgfsetfillcolor{currentfill}%
\pgfsetfillopacity{0.800000}%
\pgfsetlinewidth{0.000000pt}%
\definecolor{currentstroke}{rgb}{0.000000,0.000000,0.000000}%
\pgfsetstrokecolor{currentstroke}%
\pgfsetdash{}{0pt}%
\pgfpathmoveto{\pgfqpoint{3.282953in}{3.479765in}}%
\pgfpathlineto{\pgfqpoint{3.296366in}{3.459768in}}%
\pgfpathlineto{\pgfqpoint{3.309772in}{3.440083in}}%
\pgfpathlineto{\pgfqpoint{3.323172in}{3.420709in}}%
\pgfpathlineto{\pgfqpoint{3.336565in}{3.401641in}}%
\pgfpathlineto{\pgfqpoint{3.344384in}{3.419826in}}%
\pgfpathlineto{\pgfqpoint{3.352197in}{3.438253in}}%
\pgfpathlineto{\pgfqpoint{3.360004in}{3.456927in}}%
\pgfpathlineto{\pgfqpoint{3.367806in}{3.475853in}}%
\pgfpathlineto{\pgfqpoint{3.354418in}{3.495310in}}%
\pgfpathlineto{\pgfqpoint{3.341023in}{3.515074in}}%
\pgfpathlineto{\pgfqpoint{3.327622in}{3.535149in}}%
\pgfpathlineto{\pgfqpoint{3.314214in}{3.555536in}}%
\pgfpathlineto{\pgfqpoint{3.306408in}{3.536208in}}%
\pgfpathlineto{\pgfqpoint{3.298596in}{3.517140in}}%
\pgfpathlineto{\pgfqpoint{3.290778in}{3.498327in}}%
\pgfpathlineto{\pgfqpoint{3.282953in}{3.479765in}}%
\pgfpathclose%
\pgfusepath{fill}%
\end{pgfscope}%
\begin{pgfscope}%
\pgfpathrectangle{\pgfqpoint{1.150000in}{0.150000in}}{\pgfqpoint{5.700000in}{5.700000in}}%
\pgfusepath{clip}%
\pgfsetbuttcap%
\pgfsetroundjoin%
\definecolor{currentfill}{rgb}{0.196571,0.711827,0.479221}%
\pgfsetfillcolor{currentfill}%
\pgfsetfillopacity{0.800000}%
\pgfsetlinewidth{0.000000pt}%
\definecolor{currentstroke}{rgb}{0.000000,0.000000,0.000000}%
\pgfsetstrokecolor{currentstroke}%
\pgfsetdash{}{0pt}%
\pgfpathmoveto{\pgfqpoint{3.276919in}{4.197791in}}%
\pgfpathlineto{\pgfqpoint{3.290422in}{4.171446in}}%
\pgfpathlineto{\pgfqpoint{3.303916in}{4.145458in}}%
\pgfpathlineto{\pgfqpoint{3.317399in}{4.119822in}}%
\pgfpathlineto{\pgfqpoint{3.330873in}{4.094535in}}%
\pgfpathlineto{\pgfqpoint{3.338586in}{4.120220in}}%
\pgfpathlineto{\pgfqpoint{3.346293in}{4.146286in}}%
\pgfpathlineto{\pgfqpoint{3.353995in}{4.172740in}}%
\pgfpathlineto{\pgfqpoint{3.361692in}{4.199588in}}%
\pgfpathlineto{\pgfqpoint{3.348217in}{4.225452in}}%
\pgfpathlineto{\pgfqpoint{3.334732in}{4.251667in}}%
\pgfpathlineto{\pgfqpoint{3.321237in}{4.278236in}}%
\pgfpathlineto{\pgfqpoint{3.307731in}{4.305163in}}%
\pgfpathlineto{\pgfqpoint{3.300036in}{4.277721in}}%
\pgfpathlineto{\pgfqpoint{3.292336in}{4.250683in}}%
\pgfpathlineto{\pgfqpoint{3.284630in}{4.224042in}}%
\pgfpathlineto{\pgfqpoint{3.276919in}{4.197791in}}%
\pgfpathclose%
\pgfusepath{fill}%
\end{pgfscope}%
\begin{pgfscope}%
\pgfpathrectangle{\pgfqpoint{1.150000in}{0.150000in}}{\pgfqpoint{5.700000in}{5.700000in}}%
\pgfusepath{clip}%
\pgfsetbuttcap%
\pgfsetroundjoin%
\definecolor{currentfill}{rgb}{0.246811,0.283237,0.535941}%
\pgfsetfillcolor{currentfill}%
\pgfsetfillopacity{0.800000}%
\pgfsetlinewidth{0.000000pt}%
\definecolor{currentstroke}{rgb}{0.000000,0.000000,0.000000}%
\pgfsetstrokecolor{currentstroke}%
\pgfsetdash{}{0pt}%
\pgfpathmoveto{\pgfqpoint{3.847662in}{2.960091in}}%
\pgfpathlineto{\pgfqpoint{3.860983in}{2.950427in}}%
\pgfpathlineto{\pgfqpoint{3.874305in}{2.940994in}}%
\pgfpathlineto{\pgfqpoint{3.887629in}{2.931790in}}%
\pgfpathlineto{\pgfqpoint{3.900956in}{2.922815in}}%
\pgfpathlineto{\pgfqpoint{3.908700in}{2.937987in}}%
\pgfpathlineto{\pgfqpoint{3.916440in}{2.953340in}}%
\pgfpathlineto{\pgfqpoint{3.924176in}{2.968877in}}%
\pgfpathlineto{\pgfqpoint{3.931909in}{2.984604in}}%
\pgfpathlineto{\pgfqpoint{3.918588in}{2.993987in}}%
\pgfpathlineto{\pgfqpoint{3.905270in}{3.003598in}}%
\pgfpathlineto{\pgfqpoint{3.891954in}{3.013439in}}%
\pgfpathlineto{\pgfqpoint{3.878639in}{3.023511in}}%
\pgfpathlineto{\pgfqpoint{3.870901in}{3.007365in}}%
\pgfpathlineto{\pgfqpoint{3.863159in}{2.991416in}}%
\pgfpathlineto{\pgfqpoint{3.855413in}{2.975659in}}%
\pgfpathlineto{\pgfqpoint{3.847662in}{2.960091in}}%
\pgfpathclose%
\pgfusepath{fill}%
\end{pgfscope}%
\begin{pgfscope}%
\pgfpathrectangle{\pgfqpoint{1.150000in}{0.150000in}}{\pgfqpoint{5.700000in}{5.700000in}}%
\pgfusepath{clip}%
\pgfsetbuttcap%
\pgfsetroundjoin%
\definecolor{currentfill}{rgb}{0.259857,0.745492,0.444467}%
\pgfsetfillcolor{currentfill}%
\pgfsetfillopacity{0.800000}%
\pgfsetlinewidth{0.000000pt}%
\definecolor{currentstroke}{rgb}{0.000000,0.000000,0.000000}%
\pgfsetstrokecolor{currentstroke}%
\pgfsetdash{}{0pt}%
\pgfpathmoveto{\pgfqpoint{3.307731in}{4.305163in}}%
\pgfpathlineto{\pgfqpoint{3.321237in}{4.278236in}}%
\pgfpathlineto{\pgfqpoint{3.334732in}{4.251667in}}%
\pgfpathlineto{\pgfqpoint{3.348217in}{4.225452in}}%
\pgfpathlineto{\pgfqpoint{3.361692in}{4.199588in}}%
\pgfpathlineto{\pgfqpoint{3.369384in}{4.226837in}}%
\pgfpathlineto{\pgfqpoint{3.377070in}{4.254496in}}%
\pgfpathlineto{\pgfqpoint{3.384752in}{4.282570in}}%
\pgfpathlineto{\pgfqpoint{3.392429in}{4.311068in}}%
\pgfpathlineto{\pgfqpoint{3.378951in}{4.337548in}}%
\pgfpathlineto{\pgfqpoint{3.365463in}{4.364381in}}%
\pgfpathlineto{\pgfqpoint{3.351964in}{4.391569in}}%
\pgfpathlineto{\pgfqpoint{3.338456in}{4.419117in}}%
\pgfpathlineto{\pgfqpoint{3.330782in}{4.389987in}}%
\pgfpathlineto{\pgfqpoint{3.323104in}{4.361289in}}%
\pgfpathlineto{\pgfqpoint{3.315420in}{4.333017in}}%
\pgfpathlineto{\pgfqpoint{3.307731in}{4.305163in}}%
\pgfpathclose%
\pgfusepath{fill}%
\end{pgfscope}%
\begin{pgfscope}%
\pgfpathrectangle{\pgfqpoint{1.150000in}{0.150000in}}{\pgfqpoint{5.700000in}{5.700000in}}%
\pgfusepath{clip}%
\pgfsetbuttcap%
\pgfsetroundjoin%
\definecolor{currentfill}{rgb}{0.458674,0.816363,0.329727}%
\pgfsetfillcolor{currentfill}%
\pgfsetfillopacity{0.800000}%
\pgfsetlinewidth{0.000000pt}%
\definecolor{currentstroke}{rgb}{0.000000,0.000000,0.000000}%
\pgfsetstrokecolor{currentstroke}%
\pgfsetdash{}{0pt}%
\pgfpathmoveto{\pgfqpoint{3.453678in}{4.555212in}}%
\pgfpathlineto{\pgfqpoint{3.467156in}{4.527732in}}%
\pgfpathlineto{\pgfqpoint{3.480625in}{4.500601in}}%
\pgfpathlineto{\pgfqpoint{3.494084in}{4.473814in}}%
\pgfpathlineto{\pgfqpoint{3.507534in}{4.447369in}}%
\pgfpathlineto{\pgfqpoint{3.515178in}{4.479312in}}%
\pgfpathlineto{\pgfqpoint{3.522818in}{4.511748in}}%
\pgfpathlineto{\pgfqpoint{3.530455in}{4.544686in}}%
\pgfpathlineto{\pgfqpoint{3.538088in}{4.578136in}}%
\pgfpathlineto{\pgfqpoint{3.524631in}{4.605310in}}%
\pgfpathlineto{\pgfqpoint{3.511165in}{4.632828in}}%
\pgfpathlineto{\pgfqpoint{3.497689in}{4.660693in}}%
\pgfpathlineto{\pgfqpoint{3.484204in}{4.688908in}}%
\pgfpathlineto{\pgfqpoint{3.476578in}{4.654710in}}%
\pgfpathlineto{\pgfqpoint{3.468948in}{4.621035in}}%
\pgfpathlineto{\pgfqpoint{3.461315in}{4.587871in}}%
\pgfpathlineto{\pgfqpoint{3.453678in}{4.555212in}}%
\pgfpathclose%
\pgfusepath{fill}%
\end{pgfscope}%
\begin{pgfscope}%
\pgfpathrectangle{\pgfqpoint{1.150000in}{0.150000in}}{\pgfqpoint{5.700000in}{5.700000in}}%
\pgfusepath{clip}%
\pgfsetbuttcap%
\pgfsetroundjoin%
\definecolor{currentfill}{rgb}{0.150148,0.676631,0.506589}%
\pgfsetfillcolor{currentfill}%
\pgfsetfillopacity{0.800000}%
\pgfsetlinewidth{0.000000pt}%
\definecolor{currentstroke}{rgb}{0.000000,0.000000,0.000000}%
\pgfsetstrokecolor{currentstroke}%
\pgfsetdash{}{0pt}%
\pgfpathmoveto{\pgfqpoint{3.246013in}{4.096560in}}%
\pgfpathlineto{\pgfqpoint{3.259516in}{4.070759in}}%
\pgfpathlineto{\pgfqpoint{3.273010in}{4.045313in}}%
\pgfpathlineto{\pgfqpoint{3.286493in}{4.020219in}}%
\pgfpathlineto{\pgfqpoint{3.299967in}{3.995472in}}%
\pgfpathlineto{\pgfqpoint{3.307702in}{4.019699in}}%
\pgfpathlineto{\pgfqpoint{3.315431in}{4.044281in}}%
\pgfpathlineto{\pgfqpoint{3.323155in}{4.069224in}}%
\pgfpathlineto{\pgfqpoint{3.330873in}{4.094535in}}%
\pgfpathlineto{\pgfqpoint{3.317399in}{4.119822in}}%
\pgfpathlineto{\pgfqpoint{3.303916in}{4.145458in}}%
\pgfpathlineto{\pgfqpoint{3.290422in}{4.171446in}}%
\pgfpathlineto{\pgfqpoint{3.276919in}{4.197791in}}%
\pgfpathlineto{\pgfqpoint{3.269201in}{4.171924in}}%
\pgfpathlineto{\pgfqpoint{3.261478in}{4.146434in}}%
\pgfpathlineto{\pgfqpoint{3.253748in}{4.121315in}}%
\pgfpathlineto{\pgfqpoint{3.246013in}{4.096560in}}%
\pgfpathclose%
\pgfusepath{fill}%
\end{pgfscope}%
\begin{pgfscope}%
\pgfpathrectangle{\pgfqpoint{1.150000in}{0.150000in}}{\pgfqpoint{5.700000in}{5.700000in}}%
\pgfusepath{clip}%
\pgfsetbuttcap%
\pgfsetroundjoin%
\definecolor{currentfill}{rgb}{0.243113,0.292092,0.538516}%
\pgfsetfillcolor{currentfill}%
\pgfsetfillopacity{0.800000}%
\pgfsetlinewidth{0.000000pt}%
\definecolor{currentstroke}{rgb}{0.000000,0.000000,0.000000}%
\pgfsetstrokecolor{currentstroke}%
\pgfsetdash{}{0pt}%
\pgfpathmoveto{\pgfqpoint{3.710058in}{2.982387in}}%
\pgfpathlineto{\pgfqpoint{3.723375in}{2.971201in}}%
\pgfpathlineto{\pgfqpoint{3.736693in}{2.960256in}}%
\pgfpathlineto{\pgfqpoint{3.750011in}{2.949553in}}%
\pgfpathlineto{\pgfqpoint{3.763330in}{2.939090in}}%
\pgfpathlineto{\pgfqpoint{3.771103in}{2.954318in}}%
\pgfpathlineto{\pgfqpoint{3.778872in}{2.969724in}}%
\pgfpathlineto{\pgfqpoint{3.786637in}{2.985312in}}%
\pgfpathlineto{\pgfqpoint{3.794397in}{3.001086in}}%
\pgfpathlineto{\pgfqpoint{3.781084in}{3.011926in}}%
\pgfpathlineto{\pgfqpoint{3.767772in}{3.023006in}}%
\pgfpathlineto{\pgfqpoint{3.754461in}{3.034328in}}%
\pgfpathlineto{\pgfqpoint{3.741149in}{3.045893in}}%
\pgfpathlineto{\pgfqpoint{3.733383in}{3.029729in}}%
\pgfpathlineto{\pgfqpoint{3.725613in}{3.013760in}}%
\pgfpathlineto{\pgfqpoint{3.717838in}{2.997981in}}%
\pgfpathlineto{\pgfqpoint{3.710058in}{2.982387in}}%
\pgfpathclose%
\pgfusepath{fill}%
\end{pgfscope}%
\begin{pgfscope}%
\pgfpathrectangle{\pgfqpoint{1.150000in}{0.150000in}}{\pgfqpoint{5.700000in}{5.700000in}}%
\pgfusepath{clip}%
\pgfsetbuttcap%
\pgfsetroundjoin%
\definecolor{currentfill}{rgb}{0.208623,0.367752,0.552675}%
\pgfsetfillcolor{currentfill}%
\pgfsetfillopacity{0.800000}%
\pgfsetlinewidth{0.000000pt}%
\definecolor{currentstroke}{rgb}{0.000000,0.000000,0.000000}%
\pgfsetstrokecolor{currentstroke}%
\pgfsetdash{}{0pt}%
\pgfpathmoveto{\pgfqpoint{3.412233in}{3.192243in}}%
\pgfpathlineto{\pgfqpoint{3.425588in}{3.176170in}}%
\pgfpathlineto{\pgfqpoint{3.438940in}{3.160378in}}%
\pgfpathlineto{\pgfqpoint{3.452288in}{3.144865in}}%
\pgfpathlineto{\pgfqpoint{3.465633in}{3.129628in}}%
\pgfpathlineto{\pgfqpoint{3.473457in}{3.145868in}}%
\pgfpathlineto{\pgfqpoint{3.481275in}{3.162308in}}%
\pgfpathlineto{\pgfqpoint{3.489089in}{3.178952in}}%
\pgfpathlineto{\pgfqpoint{3.496897in}{3.195805in}}%
\pgfpathlineto{\pgfqpoint{3.483558in}{3.211392in}}%
\pgfpathlineto{\pgfqpoint{3.470216in}{3.227256in}}%
\pgfpathlineto{\pgfqpoint{3.456870in}{3.243399in}}%
\pgfpathlineto{\pgfqpoint{3.443521in}{3.259823in}}%
\pgfpathlineto{\pgfqpoint{3.435707in}{3.242607in}}%
\pgfpathlineto{\pgfqpoint{3.427888in}{3.225608in}}%
\pgfpathlineto{\pgfqpoint{3.420063in}{3.208821in}}%
\pgfpathlineto{\pgfqpoint{3.412233in}{3.192243in}}%
\pgfpathclose%
\pgfusepath{fill}%
\end{pgfscope}%
\begin{pgfscope}%
\pgfpathrectangle{\pgfqpoint{1.150000in}{0.150000in}}{\pgfqpoint{5.700000in}{5.700000in}}%
\pgfusepath{clip}%
\pgfsetbuttcap%
\pgfsetroundjoin%
\definecolor{currentfill}{rgb}{0.175841,0.441290,0.557685}%
\pgfsetfillcolor{currentfill}%
\pgfsetfillopacity{0.800000}%
\pgfsetlinewidth{0.000000pt}%
\definecolor{currentstroke}{rgb}{0.000000,0.000000,0.000000}%
\pgfsetstrokecolor{currentstroke}%
\pgfsetdash{}{0pt}%
\pgfpathmoveto{\pgfqpoint{4.826896in}{3.378099in}}%
\pgfpathlineto{\pgfqpoint{4.840369in}{3.372144in}}%
\pgfpathlineto{\pgfqpoint{4.853851in}{3.366376in}}%
\pgfpathlineto{\pgfqpoint{4.867340in}{3.360794in}}%
\pgfpathlineto{\pgfqpoint{4.880837in}{3.355398in}}%
\pgfpathlineto{\pgfqpoint{4.888441in}{3.375322in}}%
\pgfpathlineto{\pgfqpoint{4.896050in}{3.395626in}}%
\pgfpathlineto{\pgfqpoint{4.903663in}{3.416321in}}%
\pgfpathlineto{\pgfqpoint{4.890175in}{3.422279in}}%
\pgfpathlineto{\pgfqpoint{4.876695in}{3.428422in}}%
\pgfpathlineto{\pgfqpoint{4.863223in}{3.434752in}}%
\pgfpathlineto{\pgfqpoint{4.849758in}{3.441269in}}%
\pgfpathlineto{\pgfqpoint{4.842133in}{3.419817in}}%
\pgfpathlineto{\pgfqpoint{4.834512in}{3.398763in}}%
\pgfpathlineto{\pgfqpoint{4.826896in}{3.378099in}}%
\pgfpathclose%
\pgfusepath{fill}%
\end{pgfscope}%
\begin{pgfscope}%
\pgfpathrectangle{\pgfqpoint{1.150000in}{0.150000in}}{\pgfqpoint{5.700000in}{5.700000in}}%
\pgfusepath{clip}%
\pgfsetbuttcap%
\pgfsetroundjoin%
\definecolor{currentfill}{rgb}{0.218130,0.347432,0.550038}%
\pgfsetfillcolor{currentfill}%
\pgfsetfillopacity{0.800000}%
\pgfsetlinewidth{0.000000pt}%
\definecolor{currentstroke}{rgb}{0.000000,0.000000,0.000000}%
\pgfsetstrokecolor{currentstroke}%
\pgfsetdash{}{0pt}%
\pgfpathmoveto{\pgfqpoint{3.465633in}{3.129628in}}%
\pgfpathlineto{\pgfqpoint{3.478975in}{3.114666in}}%
\pgfpathlineto{\pgfqpoint{3.492314in}{3.099976in}}%
\pgfpathlineto{\pgfqpoint{3.505650in}{3.085556in}}%
\pgfpathlineto{\pgfqpoint{3.518985in}{3.071404in}}%
\pgfpathlineto{\pgfqpoint{3.526802in}{3.087306in}}%
\pgfpathlineto{\pgfqpoint{3.534615in}{3.103401in}}%
\pgfpathlineto{\pgfqpoint{3.542422in}{3.119691in}}%
\pgfpathlineto{\pgfqpoint{3.550224in}{3.136183in}}%
\pgfpathlineto{\pgfqpoint{3.536896in}{3.150684in}}%
\pgfpathlineto{\pgfqpoint{3.523566in}{3.165453in}}%
\pgfpathlineto{\pgfqpoint{3.510232in}{3.180493in}}%
\pgfpathlineto{\pgfqpoint{3.496897in}{3.195805in}}%
\pgfpathlineto{\pgfqpoint{3.489089in}{3.178952in}}%
\pgfpathlineto{\pgfqpoint{3.481275in}{3.162308in}}%
\pgfpathlineto{\pgfqpoint{3.473457in}{3.145868in}}%
\pgfpathlineto{\pgfqpoint{3.465633in}{3.129628in}}%
\pgfpathclose%
\pgfusepath{fill}%
\end{pgfscope}%
\begin{pgfscope}%
\pgfpathrectangle{\pgfqpoint{1.150000in}{0.150000in}}{\pgfqpoint{5.700000in}{5.700000in}}%
\pgfusepath{clip}%
\pgfsetbuttcap%
\pgfsetroundjoin%
\definecolor{currentfill}{rgb}{0.220057,0.343307,0.549413}%
\pgfsetfillcolor{currentfill}%
\pgfsetfillopacity{0.800000}%
\pgfsetlinewidth{0.000000pt}%
\definecolor{currentstroke}{rgb}{0.000000,0.000000,0.000000}%
\pgfsetstrokecolor{currentstroke}%
\pgfsetdash{}{0pt}%
\pgfpathmoveto{\pgfqpoint{4.459516in}{3.102444in}}%
\pgfpathlineto{\pgfqpoint{4.472930in}{3.096697in}}%
\pgfpathlineto{\pgfqpoint{4.486350in}{3.091148in}}%
\pgfpathlineto{\pgfqpoint{4.499777in}{3.085795in}}%
\pgfpathlineto{\pgfqpoint{4.513210in}{3.080639in}}%
\pgfpathlineto{\pgfqpoint{4.520833in}{3.096971in}}%
\pgfpathlineto{\pgfqpoint{4.528455in}{3.113559in}}%
\pgfpathlineto{\pgfqpoint{4.536076in}{3.130410in}}%
\pgfpathlineto{\pgfqpoint{4.543697in}{3.147532in}}%
\pgfpathlineto{\pgfqpoint{4.530273in}{3.153283in}}%
\pgfpathlineto{\pgfqpoint{4.516855in}{3.159229in}}%
\pgfpathlineto{\pgfqpoint{4.503444in}{3.165373in}}%
\pgfpathlineto{\pgfqpoint{4.490039in}{3.171714in}}%
\pgfpathlineto{\pgfqpoint{4.482409in}{3.153986in}}%
\pgfpathlineto{\pgfqpoint{4.474778in}{3.136536in}}%
\pgfpathlineto{\pgfqpoint{4.467147in}{3.119358in}}%
\pgfpathlineto{\pgfqpoint{4.459516in}{3.102444in}}%
\pgfpathclose%
\pgfusepath{fill}%
\end{pgfscope}%
\begin{pgfscope}%
\pgfpathrectangle{\pgfqpoint{1.150000in}{0.150000in}}{\pgfqpoint{5.700000in}{5.700000in}}%
\pgfusepath{clip}%
\pgfsetbuttcap%
\pgfsetroundjoin%
\definecolor{currentfill}{rgb}{0.212395,0.359683,0.551710}%
\pgfsetfillcolor{currentfill}%
\pgfsetfillopacity{0.800000}%
\pgfsetlinewidth{0.000000pt}%
\definecolor{currentstroke}{rgb}{0.000000,0.000000,0.000000}%
\pgfsetstrokecolor{currentstroke}%
\pgfsetdash{}{0pt}%
\pgfpathmoveto{\pgfqpoint{4.543697in}{3.147532in}}%
\pgfpathlineto{\pgfqpoint{4.557128in}{3.141977in}}%
\pgfpathlineto{\pgfqpoint{4.570566in}{3.136617in}}%
\pgfpathlineto{\pgfqpoint{4.584010in}{3.131451in}}%
\pgfpathlineto{\pgfqpoint{4.597462in}{3.126479in}}%
\pgfpathlineto{\pgfqpoint{4.605073in}{3.143264in}}%
\pgfpathlineto{\pgfqpoint{4.612684in}{3.160327in}}%
\pgfpathlineto{\pgfqpoint{4.620295in}{3.177674in}}%
\pgfpathlineto{\pgfqpoint{4.627907in}{3.195313in}}%
\pgfpathlineto{\pgfqpoint{4.614465in}{3.200911in}}%
\pgfpathlineto{\pgfqpoint{4.601031in}{3.206703in}}%
\pgfpathlineto{\pgfqpoint{4.587603in}{3.212689in}}%
\pgfpathlineto{\pgfqpoint{4.574181in}{3.218869in}}%
\pgfpathlineto{\pgfqpoint{4.566560in}{3.200593in}}%
\pgfpathlineto{\pgfqpoint{4.558939in}{3.182616in}}%
\pgfpathlineto{\pgfqpoint{4.551318in}{3.164932in}}%
\pgfpathlineto{\pgfqpoint{4.543697in}{3.147532in}}%
\pgfpathclose%
\pgfusepath{fill}%
\end{pgfscope}%
\begin{pgfscope}%
\pgfpathrectangle{\pgfqpoint{1.150000in}{0.150000in}}{\pgfqpoint{5.700000in}{5.700000in}}%
\pgfusepath{clip}%
\pgfsetbuttcap%
\pgfsetroundjoin%
\definecolor{currentfill}{rgb}{0.227802,0.326594,0.546532}%
\pgfsetfillcolor{currentfill}%
\pgfsetfillopacity{0.800000}%
\pgfsetlinewidth{0.000000pt}%
\definecolor{currentstroke}{rgb}{0.000000,0.000000,0.000000}%
\pgfsetstrokecolor{currentstroke}%
\pgfsetdash{}{0pt}%
\pgfpathmoveto{\pgfqpoint{4.375349in}{3.060029in}}%
\pgfpathlineto{\pgfqpoint{4.388747in}{3.054046in}}%
\pgfpathlineto{\pgfqpoint{4.402151in}{3.048264in}}%
\pgfpathlineto{\pgfqpoint{4.415561in}{3.042682in}}%
\pgfpathlineto{\pgfqpoint{4.428977in}{3.037299in}}%
\pgfpathlineto{\pgfqpoint{4.436614in}{3.053222in}}%
\pgfpathlineto{\pgfqpoint{4.444249in}{3.069383in}}%
\pgfpathlineto{\pgfqpoint{4.451883in}{3.085788in}}%
\pgfpathlineto{\pgfqpoint{4.459516in}{3.102444in}}%
\pgfpathlineto{\pgfqpoint{4.446108in}{3.108390in}}%
\pgfpathlineto{\pgfqpoint{4.432706in}{3.114535in}}%
\pgfpathlineto{\pgfqpoint{4.419311in}{3.120880in}}%
\pgfpathlineto{\pgfqpoint{4.405921in}{3.127426in}}%
\pgfpathlineto{\pgfqpoint{4.398279in}{3.110195in}}%
\pgfpathlineto{\pgfqpoint{4.390637in}{3.093223in}}%
\pgfpathlineto{\pgfqpoint{4.382994in}{3.076504in}}%
\pgfpathlineto{\pgfqpoint{4.375349in}{3.060029in}}%
\pgfpathclose%
\pgfusepath{fill}%
\end{pgfscope}%
\begin{pgfscope}%
\pgfpathrectangle{\pgfqpoint{1.150000in}{0.150000in}}{\pgfqpoint{5.700000in}{5.700000in}}%
\pgfusepath{clip}%
\pgfsetbuttcap%
\pgfsetroundjoin%
\definecolor{currentfill}{rgb}{0.626579,0.854645,0.223353}%
\pgfsetfillcolor{currentfill}%
\pgfsetfillopacity{0.800000}%
\pgfsetlinewidth{0.000000pt}%
\definecolor{currentstroke}{rgb}{0.000000,0.000000,0.000000}%
\pgfsetstrokecolor{currentstroke}%
\pgfsetdash{}{0pt}%
\pgfpathmoveto{\pgfqpoint{3.652868in}{4.753606in}}%
\pgfpathlineto{\pgfqpoint{3.666300in}{4.726533in}}%
\pgfpathlineto{\pgfqpoint{3.679723in}{4.699787in}}%
\pgfpathlineto{\pgfqpoint{3.693139in}{4.673366in}}%
\pgfpathlineto{\pgfqpoint{3.706547in}{4.647267in}}%
\pgfpathlineto{\pgfqpoint{3.714178in}{4.684073in}}%
\pgfpathlineto{\pgfqpoint{3.721809in}{4.721465in}}%
\pgfpathlineto{\pgfqpoint{3.729438in}{4.759453in}}%
\pgfpathlineto{\pgfqpoint{3.716023in}{4.786179in}}%
\pgfpathlineto{\pgfqpoint{3.702600in}{4.813228in}}%
\pgfpathlineto{\pgfqpoint{3.689169in}{4.840604in}}%
\pgfpathlineto{\pgfqpoint{3.675730in}{4.868308in}}%
\pgfpathlineto{\pgfqpoint{3.668110in}{4.829471in}}%
\pgfpathlineto{\pgfqpoint{3.660490in}{4.791240in}}%
\pgfpathlineto{\pgfqpoint{3.652868in}{4.753606in}}%
\pgfpathclose%
\pgfusepath{fill}%
\end{pgfscope}%
\begin{pgfscope}%
\pgfpathrectangle{\pgfqpoint{1.150000in}{0.150000in}}{\pgfqpoint{5.700000in}{5.700000in}}%
\pgfusepath{clip}%
\pgfsetbuttcap%
\pgfsetroundjoin%
\definecolor{currentfill}{rgb}{0.195860,0.395433,0.555276}%
\pgfsetfillcolor{currentfill}%
\pgfsetfillopacity{0.800000}%
\pgfsetlinewidth{0.000000pt}%
\definecolor{currentstroke}{rgb}{0.000000,0.000000,0.000000}%
\pgfsetstrokecolor{currentstroke}%
\pgfsetdash{}{0pt}%
\pgfpathmoveto{\pgfqpoint{3.358771in}{3.259393in}}%
\pgfpathlineto{\pgfqpoint{3.372143in}{3.242172in}}%
\pgfpathlineto{\pgfqpoint{3.385511in}{3.225242in}}%
\pgfpathlineto{\pgfqpoint{3.398874in}{3.208599in}}%
\pgfpathlineto{\pgfqpoint{3.412233in}{3.192243in}}%
\pgfpathlineto{\pgfqpoint{3.420063in}{3.208821in}}%
\pgfpathlineto{\pgfqpoint{3.427888in}{3.225608in}}%
\pgfpathlineto{\pgfqpoint{3.435707in}{3.242607in}}%
\pgfpathlineto{\pgfqpoint{3.443521in}{3.259823in}}%
\pgfpathlineto{\pgfqpoint{3.430168in}{3.276531in}}%
\pgfpathlineto{\pgfqpoint{3.416810in}{3.293526in}}%
\pgfpathlineto{\pgfqpoint{3.403449in}{3.310809in}}%
\pgfpathlineto{\pgfqpoint{3.390082in}{3.328383in}}%
\pgfpathlineto{\pgfqpoint{3.382263in}{3.310802in}}%
\pgfpathlineto{\pgfqpoint{3.374438in}{3.293446in}}%
\pgfpathlineto{\pgfqpoint{3.366607in}{3.276311in}}%
\pgfpathlineto{\pgfqpoint{3.358771in}{3.259393in}}%
\pgfpathclose%
\pgfusepath{fill}%
\end{pgfscope}%
\begin{pgfscope}%
\pgfpathrectangle{\pgfqpoint{1.150000in}{0.150000in}}{\pgfqpoint{5.700000in}{5.700000in}}%
\pgfusepath{clip}%
\pgfsetbuttcap%
\pgfsetroundjoin%
\definecolor{currentfill}{rgb}{0.248629,0.278775,0.534556}%
\pgfsetfillcolor{currentfill}%
\pgfsetfillopacity{0.800000}%
\pgfsetlinewidth{0.000000pt}%
\definecolor{currentstroke}{rgb}{0.000000,0.000000,0.000000}%
\pgfsetstrokecolor{currentstroke}%
\pgfsetdash{}{0pt}%
\pgfpathmoveto{\pgfqpoint{3.985217in}{2.949331in}}%
\pgfpathlineto{\pgfqpoint{3.998552in}{2.941070in}}%
\pgfpathlineto{\pgfqpoint{4.011889in}{2.933031in}}%
\pgfpathlineto{\pgfqpoint{4.025231in}{2.925211in}}%
\pgfpathlineto{\pgfqpoint{4.038576in}{2.917610in}}%
\pgfpathlineto{\pgfqpoint{4.046291in}{2.932678in}}%
\pgfpathlineto{\pgfqpoint{4.054004in}{2.947930in}}%
\pgfpathlineto{\pgfqpoint{4.061713in}{2.963372in}}%
\pgfpathlineto{\pgfqpoint{4.069418in}{2.979008in}}%
\pgfpathlineto{\pgfqpoint{4.056080in}{2.987047in}}%
\pgfpathlineto{\pgfqpoint{4.042745in}{2.995305in}}%
\pgfpathlineto{\pgfqpoint{4.029414in}{3.003783in}}%
\pgfpathlineto{\pgfqpoint{4.016086in}{3.012482in}}%
\pgfpathlineto{\pgfqpoint{4.008374in}{2.996396in}}%
\pgfpathlineto{\pgfqpoint{4.000658in}{2.980512in}}%
\pgfpathlineto{\pgfqpoint{3.992939in}{2.964825in}}%
\pgfpathlineto{\pgfqpoint{3.985217in}{2.949331in}}%
\pgfpathclose%
\pgfusepath{fill}%
\end{pgfscope}%
\begin{pgfscope}%
\pgfpathrectangle{\pgfqpoint{1.150000in}{0.150000in}}{\pgfqpoint{5.700000in}{5.700000in}}%
\pgfusepath{clip}%
\pgfsetbuttcap%
\pgfsetroundjoin%
\definecolor{currentfill}{rgb}{0.227802,0.326594,0.546532}%
\pgfsetfillcolor{currentfill}%
\pgfsetfillopacity{0.800000}%
\pgfsetlinewidth{0.000000pt}%
\definecolor{currentstroke}{rgb}{0.000000,0.000000,0.000000}%
\pgfsetstrokecolor{currentstroke}%
\pgfsetdash{}{0pt}%
\pgfpathmoveto{\pgfqpoint{3.518985in}{3.071404in}}%
\pgfpathlineto{\pgfqpoint{3.532317in}{3.057518in}}%
\pgfpathlineto{\pgfqpoint{3.545647in}{3.043896in}}%
\pgfpathlineto{\pgfqpoint{3.558976in}{3.030536in}}%
\pgfpathlineto{\pgfqpoint{3.572303in}{3.017436in}}%
\pgfpathlineto{\pgfqpoint{3.580114in}{3.033002in}}%
\pgfpathlineto{\pgfqpoint{3.587920in}{3.048752in}}%
\pgfpathlineto{\pgfqpoint{3.595722in}{3.064690in}}%
\pgfpathlineto{\pgfqpoint{3.603518in}{3.080821in}}%
\pgfpathlineto{\pgfqpoint{3.590197in}{3.094269in}}%
\pgfpathlineto{\pgfqpoint{3.576875in}{3.107977in}}%
\pgfpathlineto{\pgfqpoint{3.563550in}{3.121948in}}%
\pgfpathlineto{\pgfqpoint{3.550224in}{3.136183in}}%
\pgfpathlineto{\pgfqpoint{3.542422in}{3.119691in}}%
\pgfpathlineto{\pgfqpoint{3.534615in}{3.103401in}}%
\pgfpathlineto{\pgfqpoint{3.526802in}{3.087306in}}%
\pgfpathlineto{\pgfqpoint{3.518985in}{3.071404in}}%
\pgfpathclose%
\pgfusepath{fill}%
\end{pgfscope}%
\begin{pgfscope}%
\pgfpathrectangle{\pgfqpoint{1.150000in}{0.150000in}}{\pgfqpoint{5.700000in}{5.700000in}}%
\pgfusepath{clip}%
\pgfsetbuttcap%
\pgfsetroundjoin%
\definecolor{currentfill}{rgb}{0.344074,0.780029,0.397381}%
\pgfsetfillcolor{currentfill}%
\pgfsetfillopacity{0.800000}%
\pgfsetlinewidth{0.000000pt}%
\definecolor{currentstroke}{rgb}{0.000000,0.000000,0.000000}%
\pgfsetstrokecolor{currentstroke}%
\pgfsetdash{}{0pt}%
\pgfpathmoveto{\pgfqpoint{3.338456in}{4.419117in}}%
\pgfpathlineto{\pgfqpoint{3.351964in}{4.391569in}}%
\pgfpathlineto{\pgfqpoint{3.365463in}{4.364381in}}%
\pgfpathlineto{\pgfqpoint{3.378951in}{4.337548in}}%
\pgfpathlineto{\pgfqpoint{3.392429in}{4.311068in}}%
\pgfpathlineto{\pgfqpoint{3.400101in}{4.339996in}}%
\pgfpathlineto{\pgfqpoint{3.407768in}{4.369363in}}%
\pgfpathlineto{\pgfqpoint{3.415430in}{4.399176in}}%
\pgfpathlineto{\pgfqpoint{3.423088in}{4.429443in}}%
\pgfpathlineto{\pgfqpoint{3.409606in}{4.456579in}}%
\pgfpathlineto{\pgfqpoint{3.396114in}{4.484069in}}%
\pgfpathlineto{\pgfqpoint{3.382611in}{4.511916in}}%
\pgfpathlineto{\pgfqpoint{3.369097in}{4.540125in}}%
\pgfpathlineto{\pgfqpoint{3.361444in}{4.509185in}}%
\pgfpathlineto{\pgfqpoint{3.353786in}{4.478709in}}%
\pgfpathlineto{\pgfqpoint{3.346124in}{4.448689in}}%
\pgfpathlineto{\pgfqpoint{3.338456in}{4.419117in}}%
\pgfpathclose%
\pgfusepath{fill}%
\end{pgfscope}%
\begin{pgfscope}%
\pgfpathrectangle{\pgfqpoint{1.150000in}{0.150000in}}{\pgfqpoint{5.700000in}{5.700000in}}%
\pgfusepath{clip}%
\pgfsetbuttcap%
\pgfsetroundjoin%
\definecolor{currentfill}{rgb}{0.235526,0.309527,0.542944}%
\pgfsetfillcolor{currentfill}%
\pgfsetfillopacity{0.800000}%
\pgfsetlinewidth{0.000000pt}%
\definecolor{currentstroke}{rgb}{0.000000,0.000000,0.000000}%
\pgfsetstrokecolor{currentstroke}%
\pgfsetdash{}{0pt}%
\pgfpathmoveto{\pgfqpoint{4.291184in}{3.020293in}}%
\pgfpathlineto{\pgfqpoint{4.304567in}{3.014029in}}%
\pgfpathlineto{\pgfqpoint{4.317956in}{3.007970in}}%
\pgfpathlineto{\pgfqpoint{4.331351in}{3.002113in}}%
\pgfpathlineto{\pgfqpoint{4.344751in}{2.996460in}}%
\pgfpathlineto{\pgfqpoint{4.352403in}{3.012016in}}%
\pgfpathlineto{\pgfqpoint{4.360054in}{3.027792in}}%
\pgfpathlineto{\pgfqpoint{4.367702in}{3.043794in}}%
\pgfpathlineto{\pgfqpoint{4.375349in}{3.060029in}}%
\pgfpathlineto{\pgfqpoint{4.361956in}{3.066214in}}%
\pgfpathlineto{\pgfqpoint{4.348570in}{3.072602in}}%
\pgfpathlineto{\pgfqpoint{4.335189in}{3.079193in}}%
\pgfpathlineto{\pgfqpoint{4.321813in}{3.085989in}}%
\pgfpathlineto{\pgfqpoint{4.314158in}{3.069211in}}%
\pgfpathlineto{\pgfqpoint{4.306502in}{3.052673in}}%
\pgfpathlineto{\pgfqpoint{4.298844in}{3.036369in}}%
\pgfpathlineto{\pgfqpoint{4.291184in}{3.020293in}}%
\pgfpathclose%
\pgfusepath{fill}%
\end{pgfscope}%
\begin{pgfscope}%
\pgfpathrectangle{\pgfqpoint{1.150000in}{0.150000in}}{\pgfqpoint{5.700000in}{5.700000in}}%
\pgfusepath{clip}%
\pgfsetbuttcap%
\pgfsetroundjoin%
\definecolor{currentfill}{rgb}{0.203063,0.379716,0.553925}%
\pgfsetfillcolor{currentfill}%
\pgfsetfillopacity{0.800000}%
\pgfsetlinewidth{0.000000pt}%
\definecolor{currentstroke}{rgb}{0.000000,0.000000,0.000000}%
\pgfsetstrokecolor{currentstroke}%
\pgfsetdash{}{0pt}%
\pgfpathmoveto{\pgfqpoint{4.627907in}{3.195313in}}%
\pgfpathlineto{\pgfqpoint{4.641356in}{3.189907in}}%
\pgfpathlineto{\pgfqpoint{4.654811in}{3.184694in}}%
\pgfpathlineto{\pgfqpoint{4.668274in}{3.179671in}}%
\pgfpathlineto{\pgfqpoint{4.681744in}{3.174839in}}%
\pgfpathlineto{\pgfqpoint{4.689346in}{3.192132in}}%
\pgfpathlineto{\pgfqpoint{4.696949in}{3.209723in}}%
\pgfpathlineto{\pgfqpoint{4.704552in}{3.227620in}}%
\pgfpathlineto{\pgfqpoint{4.712157in}{3.245832in}}%
\pgfpathlineto{\pgfqpoint{4.698698in}{3.251321in}}%
\pgfpathlineto{\pgfqpoint{4.685246in}{3.257001in}}%
\pgfpathlineto{\pgfqpoint{4.671800in}{3.262872in}}%
\pgfpathlineto{\pgfqpoint{4.658362in}{3.268935in}}%
\pgfpathlineto{\pgfqpoint{4.650746in}{3.250054in}}%
\pgfpathlineto{\pgfqpoint{4.643132in}{3.231495in}}%
\pgfpathlineto{\pgfqpoint{4.635519in}{3.213251in}}%
\pgfpathlineto{\pgfqpoint{4.627907in}{3.195313in}}%
\pgfpathclose%
\pgfusepath{fill}%
\end{pgfscope}%
\begin{pgfscope}%
\pgfpathrectangle{\pgfqpoint{1.150000in}{0.150000in}}{\pgfqpoint{5.700000in}{5.700000in}}%
\pgfusepath{clip}%
\pgfsetbuttcap%
\pgfsetroundjoin%
\definecolor{currentfill}{rgb}{0.128087,0.647749,0.523491}%
\pgfsetfillcolor{currentfill}%
\pgfsetfillopacity{0.800000}%
\pgfsetlinewidth{0.000000pt}%
\definecolor{currentstroke}{rgb}{0.000000,0.000000,0.000000}%
\pgfsetstrokecolor{currentstroke}%
\pgfsetdash{}{0pt}%
\pgfpathmoveto{\pgfqpoint{3.215010in}{4.001057in}}%
\pgfpathlineto{\pgfqpoint{3.228514in}{3.975763in}}%
\pgfpathlineto{\pgfqpoint{3.242008in}{3.950823in}}%
\pgfpathlineto{\pgfqpoint{3.255493in}{3.926233in}}%
\pgfpathlineto{\pgfqpoint{3.268968in}{3.901990in}}%
\pgfpathlineto{\pgfqpoint{3.276726in}{3.924859in}}%
\pgfpathlineto{\pgfqpoint{3.284479in}{3.948058in}}%
\pgfpathlineto{\pgfqpoint{3.292226in}{3.971594in}}%
\pgfpathlineto{\pgfqpoint{3.299967in}{3.995472in}}%
\pgfpathlineto{\pgfqpoint{3.286493in}{4.020219in}}%
\pgfpathlineto{\pgfqpoint{3.273010in}{4.045313in}}%
\pgfpathlineto{\pgfqpoint{3.259516in}{4.070759in}}%
\pgfpathlineto{\pgfqpoint{3.246013in}{4.096560in}}%
\pgfpathlineto{\pgfqpoint{3.238271in}{4.072162in}}%
\pgfpathlineto{\pgfqpoint{3.230524in}{4.048117in}}%
\pgfpathlineto{\pgfqpoint{3.222770in}{4.024417in}}%
\pgfpathlineto{\pgfqpoint{3.215010in}{4.001057in}}%
\pgfpathclose%
\pgfusepath{fill}%
\end{pgfscope}%
\begin{pgfscope}%
\pgfpathrectangle{\pgfqpoint{1.150000in}{0.150000in}}{\pgfqpoint{5.700000in}{5.700000in}}%
\pgfusepath{clip}%
\pgfsetbuttcap%
\pgfsetroundjoin%
\definecolor{currentfill}{rgb}{0.131172,0.555899,0.552459}%
\pgfsetfillcolor{currentfill}%
\pgfsetfillopacity{0.800000}%
\pgfsetlinewidth{0.000000pt}%
\definecolor{currentstroke}{rgb}{0.000000,0.000000,0.000000}%
\pgfsetstrokecolor{currentstroke}%
\pgfsetdash{}{0pt}%
\pgfpathmoveto{\pgfqpoint{3.206676in}{3.730271in}}%
\pgfpathlineto{\pgfqpoint{3.220148in}{3.707268in}}%
\pgfpathlineto{\pgfqpoint{3.233611in}{3.684604in}}%
\pgfpathlineto{\pgfqpoint{3.247064in}{3.662275in}}%
\pgfpathlineto{\pgfqpoint{3.260510in}{3.640277in}}%
\pgfpathlineto{\pgfqpoint{3.268314in}{3.660283in}}%
\pgfpathlineto{\pgfqpoint{3.276113in}{3.680569in}}%
\pgfpathlineto{\pgfqpoint{3.283905in}{3.701137in}}%
\pgfpathlineto{\pgfqpoint{3.291691in}{3.721995in}}%
\pgfpathlineto{\pgfqpoint{3.278249in}{3.744421in}}%
\pgfpathlineto{\pgfqpoint{3.264799in}{3.767180in}}%
\pgfpathlineto{\pgfqpoint{3.251340in}{3.790275in}}%
\pgfpathlineto{\pgfqpoint{3.237872in}{3.813708in}}%
\pgfpathlineto{\pgfqpoint{3.230083in}{3.792407in}}%
\pgfpathlineto{\pgfqpoint{3.222287in}{3.771405in}}%
\pgfpathlineto{\pgfqpoint{3.214485in}{3.750694in}}%
\pgfpathlineto{\pgfqpoint{3.206676in}{3.730271in}}%
\pgfpathclose%
\pgfusepath{fill}%
\end{pgfscope}%
\begin{pgfscope}%
\pgfpathrectangle{\pgfqpoint{1.150000in}{0.150000in}}{\pgfqpoint{5.700000in}{5.700000in}}%
\pgfusepath{clip}%
\pgfsetbuttcap%
\pgfsetroundjoin%
\definecolor{currentfill}{rgb}{0.153364,0.497000,0.557724}%
\pgfsetfillcolor{currentfill}%
\pgfsetfillopacity{0.800000}%
\pgfsetlinewidth{0.000000pt}%
\definecolor{currentstroke}{rgb}{0.000000,0.000000,0.000000}%
\pgfsetstrokecolor{currentstroke}%
\pgfsetdash{}{0pt}%
\pgfpathmoveto{\pgfqpoint{3.229230in}{3.562936in}}%
\pgfpathlineto{\pgfqpoint{3.242672in}{3.541660in}}%
\pgfpathlineto{\pgfqpoint{3.256107in}{3.520708in}}%
\pgfpathlineto{\pgfqpoint{3.269534in}{3.500077in}}%
\pgfpathlineto{\pgfqpoint{3.282953in}{3.479765in}}%
\pgfpathlineto{\pgfqpoint{3.290778in}{3.498327in}}%
\pgfpathlineto{\pgfqpoint{3.298596in}{3.517140in}}%
\pgfpathlineto{\pgfqpoint{3.306408in}{3.536208in}}%
\pgfpathlineto{\pgfqpoint{3.314214in}{3.555536in}}%
\pgfpathlineto{\pgfqpoint{3.300799in}{3.576240in}}%
\pgfpathlineto{\pgfqpoint{3.287377in}{3.597262in}}%
\pgfpathlineto{\pgfqpoint{3.273947in}{3.618607in}}%
\pgfpathlineto{\pgfqpoint{3.260510in}{3.640277in}}%
\pgfpathlineto{\pgfqpoint{3.252699in}{3.620543in}}%
\pgfpathlineto{\pgfqpoint{3.244883in}{3.601079in}}%
\pgfpathlineto{\pgfqpoint{3.237059in}{3.581878in}}%
\pgfpathlineto{\pgfqpoint{3.229230in}{3.562936in}}%
\pgfpathclose%
\pgfusepath{fill}%
\end{pgfscope}%
\begin{pgfscope}%
\pgfpathrectangle{\pgfqpoint{1.150000in}{0.150000in}}{\pgfqpoint{5.700000in}{5.700000in}}%
\pgfusepath{clip}%
\pgfsetbuttcap%
\pgfsetroundjoin%
\definecolor{currentfill}{rgb}{0.185556,0.418570,0.556753}%
\pgfsetfillcolor{currentfill}%
\pgfsetfillopacity{0.800000}%
\pgfsetlinewidth{0.000000pt}%
\definecolor{currentstroke}{rgb}{0.000000,0.000000,0.000000}%
\pgfsetstrokecolor{currentstroke}%
\pgfsetdash{}{0pt}%
\pgfpathmoveto{\pgfqpoint{3.305230in}{3.331234in}}%
\pgfpathlineto{\pgfqpoint{3.318623in}{3.312825in}}%
\pgfpathlineto{\pgfqpoint{3.332011in}{3.294717in}}%
\pgfpathlineto{\pgfqpoint{3.345394in}{3.276907in}}%
\pgfpathlineto{\pgfqpoint{3.358771in}{3.259393in}}%
\pgfpathlineto{\pgfqpoint{3.366607in}{3.276311in}}%
\pgfpathlineto{\pgfqpoint{3.374438in}{3.293446in}}%
\pgfpathlineto{\pgfqpoint{3.382263in}{3.310802in}}%
\pgfpathlineto{\pgfqpoint{3.390082in}{3.328383in}}%
\pgfpathlineto{\pgfqpoint{3.376711in}{3.346250in}}%
\pgfpathlineto{\pgfqpoint{3.363334in}{3.364414in}}%
\pgfpathlineto{\pgfqpoint{3.349953in}{3.382877in}}%
\pgfpathlineto{\pgfqpoint{3.336565in}{3.401641in}}%
\pgfpathlineto{\pgfqpoint{3.328740in}{3.383693in}}%
\pgfpathlineto{\pgfqpoint{3.320909in}{3.365979in}}%
\pgfpathlineto{\pgfqpoint{3.313073in}{3.348494in}}%
\pgfpathlineto{\pgfqpoint{3.305230in}{3.331234in}}%
\pgfpathclose%
\pgfusepath{fill}%
\end{pgfscope}%
\begin{pgfscope}%
\pgfpathrectangle{\pgfqpoint{1.150000in}{0.150000in}}{\pgfqpoint{5.700000in}{5.700000in}}%
\pgfusepath{clip}%
\pgfsetbuttcap%
\pgfsetroundjoin%
\definecolor{currentfill}{rgb}{0.241237,0.296485,0.539709}%
\pgfsetfillcolor{currentfill}%
\pgfsetfillopacity{0.800000}%
\pgfsetlinewidth{0.000000pt}%
\definecolor{currentstroke}{rgb}{0.000000,0.000000,0.000000}%
\pgfsetstrokecolor{currentstroke}%
\pgfsetdash{}{0pt}%
\pgfpathmoveto{\pgfqpoint{4.207009in}{2.983269in}}%
\pgfpathlineto{\pgfqpoint{4.220379in}{2.976678in}}%
\pgfpathlineto{\pgfqpoint{4.233754in}{2.970296in}}%
\pgfpathlineto{\pgfqpoint{4.247134in}{2.964120in}}%
\pgfpathlineto{\pgfqpoint{4.260520in}{2.958151in}}%
\pgfpathlineto{\pgfqpoint{4.268190in}{2.973375in}}%
\pgfpathlineto{\pgfqpoint{4.275857in}{2.988802in}}%
\pgfpathlineto{\pgfqpoint{4.283521in}{3.004440in}}%
\pgfpathlineto{\pgfqpoint{4.291184in}{3.020293in}}%
\pgfpathlineto{\pgfqpoint{4.277806in}{3.026763in}}%
\pgfpathlineto{\pgfqpoint{4.264433in}{3.033439in}}%
\pgfpathlineto{\pgfqpoint{4.251065in}{3.040321in}}%
\pgfpathlineto{\pgfqpoint{4.237702in}{3.047412in}}%
\pgfpathlineto{\pgfqpoint{4.230033in}{3.031047in}}%
\pgfpathlineto{\pgfqpoint{4.222360in}{3.014905in}}%
\pgfpathlineto{\pgfqpoint{4.214686in}{2.998981in}}%
\pgfpathlineto{\pgfqpoint{4.207009in}{2.983269in}}%
\pgfpathclose%
\pgfusepath{fill}%
\end{pgfscope}%
\begin{pgfscope}%
\pgfpathrectangle{\pgfqpoint{1.150000in}{0.150000in}}{\pgfqpoint{5.700000in}{5.700000in}}%
\pgfusepath{clip}%
\pgfsetbuttcap%
\pgfsetroundjoin%
\definecolor{currentfill}{rgb}{0.616293,0.852709,0.230052}%
\pgfsetfillcolor{currentfill}%
\pgfsetfillopacity{0.800000}%
\pgfsetlinewidth{0.000000pt}%
\definecolor{currentstroke}{rgb}{0.000000,0.000000,0.000000}%
\pgfsetstrokecolor{currentstroke}%
\pgfsetdash{}{0pt}%
\pgfpathmoveto{\pgfqpoint{3.568593in}{4.717227in}}%
\pgfpathlineto{\pgfqpoint{3.582049in}{4.689623in}}%
\pgfpathlineto{\pgfqpoint{3.595497in}{4.662358in}}%
\pgfpathlineto{\pgfqpoint{3.608935in}{4.635428in}}%
\pgfpathlineto{\pgfqpoint{3.622366in}{4.608830in}}%
\pgfpathlineto{\pgfqpoint{3.629994in}{4.644179in}}%
\pgfpathlineto{\pgfqpoint{3.637621in}{4.680085in}}%
\pgfpathlineto{\pgfqpoint{3.645245in}{4.716558in}}%
\pgfpathlineto{\pgfqpoint{3.652868in}{4.753606in}}%
\pgfpathlineto{\pgfqpoint{3.639429in}{4.781010in}}%
\pgfpathlineto{\pgfqpoint{3.625981in}{4.808748in}}%
\pgfpathlineto{\pgfqpoint{3.612524in}{4.836824in}}%
\pgfpathlineto{\pgfqpoint{3.599058in}{4.865240in}}%
\pgfpathlineto{\pgfqpoint{3.591445in}{4.827366in}}%
\pgfpathlineto{\pgfqpoint{3.583830in}{4.790079in}}%
\pgfpathlineto{\pgfqpoint{3.576213in}{4.753369in}}%
\pgfpathlineto{\pgfqpoint{3.568593in}{4.717227in}}%
\pgfpathclose%
\pgfusepath{fill}%
\end{pgfscope}%
\begin{pgfscope}%
\pgfpathrectangle{\pgfqpoint{1.150000in}{0.150000in}}{\pgfqpoint{5.700000in}{5.700000in}}%
\pgfusepath{clip}%
\pgfsetbuttcap%
\pgfsetroundjoin%
\definecolor{currentfill}{rgb}{0.194100,0.399323,0.555565}%
\pgfsetfillcolor{currentfill}%
\pgfsetfillopacity{0.800000}%
\pgfsetlinewidth{0.000000pt}%
\definecolor{currentstroke}{rgb}{0.000000,0.000000,0.000000}%
\pgfsetstrokecolor{currentstroke}%
\pgfsetdash{}{0pt}%
\pgfpathmoveto{\pgfqpoint{4.712157in}{3.245832in}}%
\pgfpathlineto{\pgfqpoint{4.725624in}{3.240533in}}%
\pgfpathlineto{\pgfqpoint{4.739098in}{3.235423in}}%
\pgfpathlineto{\pgfqpoint{4.752580in}{3.230502in}}%
\pgfpathlineto{\pgfqpoint{4.766070in}{3.225769in}}%
\pgfpathlineto{\pgfqpoint{4.773665in}{3.243626in}}%
\pgfpathlineto{\pgfqpoint{4.781262in}{3.261805in}}%
\pgfpathlineto{\pgfqpoint{4.788861in}{3.280314in}}%
\pgfpathlineto{\pgfqpoint{4.796462in}{3.299161in}}%
\pgfpathlineto{\pgfqpoint{4.782984in}{3.304583in}}%
\pgfpathlineto{\pgfqpoint{4.769514in}{3.310193in}}%
\pgfpathlineto{\pgfqpoint{4.756051in}{3.315992in}}%
\pgfpathlineto{\pgfqpoint{4.742595in}{3.321980in}}%
\pgfpathlineto{\pgfqpoint{4.734982in}{3.302432in}}%
\pgfpathlineto{\pgfqpoint{4.727372in}{3.283230in}}%
\pgfpathlineto{\pgfqpoint{4.719764in}{3.264366in}}%
\pgfpathlineto{\pgfqpoint{4.712157in}{3.245832in}}%
\pgfpathclose%
\pgfusepath{fill}%
\end{pgfscope}%
\begin{pgfscope}%
\pgfpathrectangle{\pgfqpoint{1.150000in}{0.150000in}}{\pgfqpoint{5.700000in}{5.700000in}}%
\pgfusepath{clip}%
\pgfsetbuttcap%
\pgfsetroundjoin%
\definecolor{currentfill}{rgb}{0.237441,0.305202,0.541921}%
\pgfsetfillcolor{currentfill}%
\pgfsetfillopacity{0.800000}%
\pgfsetlinewidth{0.000000pt}%
\definecolor{currentstroke}{rgb}{0.000000,0.000000,0.000000}%
\pgfsetstrokecolor{currentstroke}%
\pgfsetdash{}{0pt}%
\pgfpathmoveto{\pgfqpoint{3.572303in}{3.017436in}}%
\pgfpathlineto{\pgfqpoint{3.585629in}{3.004594in}}%
\pgfpathlineto{\pgfqpoint{3.598954in}{2.992009in}}%
\pgfpathlineto{\pgfqpoint{3.612278in}{2.979678in}}%
\pgfpathlineto{\pgfqpoint{3.625602in}{2.967599in}}%
\pgfpathlineto{\pgfqpoint{3.633406in}{2.982830in}}%
\pgfpathlineto{\pgfqpoint{3.641206in}{2.998237in}}%
\pgfpathlineto{\pgfqpoint{3.649001in}{3.013823in}}%
\pgfpathlineto{\pgfqpoint{3.656792in}{3.029595in}}%
\pgfpathlineto{\pgfqpoint{3.643474in}{3.042020in}}%
\pgfpathlineto{\pgfqpoint{3.630156in}{3.054699in}}%
\pgfpathlineto{\pgfqpoint{3.616838in}{3.067632in}}%
\pgfpathlineto{\pgfqpoint{3.603518in}{3.080821in}}%
\pgfpathlineto{\pgfqpoint{3.595722in}{3.064690in}}%
\pgfpathlineto{\pgfqpoint{3.587920in}{3.048752in}}%
\pgfpathlineto{\pgfqpoint{3.580114in}{3.033002in}}%
\pgfpathlineto{\pgfqpoint{3.572303in}{3.017436in}}%
\pgfpathclose%
\pgfusepath{fill}%
\end{pgfscope}%
\begin{pgfscope}%
\pgfpathrectangle{\pgfqpoint{1.150000in}{0.150000in}}{\pgfqpoint{5.700000in}{5.700000in}}%
\pgfusepath{clip}%
\pgfsetbuttcap%
\pgfsetroundjoin%
\definecolor{currentfill}{rgb}{0.250425,0.274290,0.533103}%
\pgfsetfillcolor{currentfill}%
\pgfsetfillopacity{0.800000}%
\pgfsetlinewidth{0.000000pt}%
\definecolor{currentstroke}{rgb}{0.000000,0.000000,0.000000}%
\pgfsetstrokecolor{currentstroke}%
\pgfsetdash{}{0pt}%
\pgfpathmoveto{\pgfqpoint{3.763330in}{2.939090in}}%
\pgfpathlineto{\pgfqpoint{3.776650in}{2.928865in}}%
\pgfpathlineto{\pgfqpoint{3.789972in}{2.918876in}}%
\pgfpathlineto{\pgfqpoint{3.803295in}{2.909123in}}%
\pgfpathlineto{\pgfqpoint{3.816620in}{2.899603in}}%
\pgfpathlineto{\pgfqpoint{3.824387in}{2.914466in}}%
\pgfpathlineto{\pgfqpoint{3.832150in}{2.929499in}}%
\pgfpathlineto{\pgfqpoint{3.839908in}{2.944706in}}%
\pgfpathlineto{\pgfqpoint{3.847662in}{2.960091in}}%
\pgfpathlineto{\pgfqpoint{3.834344in}{2.969987in}}%
\pgfpathlineto{\pgfqpoint{3.821027in}{2.980118in}}%
\pgfpathlineto{\pgfqpoint{3.807712in}{2.990483in}}%
\pgfpathlineto{\pgfqpoint{3.794397in}{3.001086in}}%
\pgfpathlineto{\pgfqpoint{3.786637in}{2.985312in}}%
\pgfpathlineto{\pgfqpoint{3.778872in}{2.969724in}}%
\pgfpathlineto{\pgfqpoint{3.771103in}{2.954318in}}%
\pgfpathlineto{\pgfqpoint{3.763330in}{2.939090in}}%
\pgfpathclose%
\pgfusepath{fill}%
\end{pgfscope}%
\begin{pgfscope}%
\pgfpathrectangle{\pgfqpoint{1.150000in}{0.150000in}}{\pgfqpoint{5.700000in}{5.700000in}}%
\pgfusepath{clip}%
\pgfsetbuttcap%
\pgfsetroundjoin%
\definecolor{currentfill}{rgb}{0.252194,0.269783,0.531579}%
\pgfsetfillcolor{currentfill}%
\pgfsetfillopacity{0.800000}%
\pgfsetlinewidth{0.000000pt}%
\definecolor{currentstroke}{rgb}{0.000000,0.000000,0.000000}%
\pgfsetstrokecolor{currentstroke}%
\pgfsetdash{}{0pt}%
\pgfpathmoveto{\pgfqpoint{3.900956in}{2.922815in}}%
\pgfpathlineto{\pgfqpoint{3.914286in}{2.914067in}}%
\pgfpathlineto{\pgfqpoint{3.927618in}{2.905544in}}%
\pgfpathlineto{\pgfqpoint{3.940952in}{2.897246in}}%
\pgfpathlineto{\pgfqpoint{3.954290in}{2.889171in}}%
\pgfpathlineto{\pgfqpoint{3.962028in}{2.903948in}}%
\pgfpathlineto{\pgfqpoint{3.969761in}{2.918897in}}%
\pgfpathlineto{\pgfqpoint{3.977491in}{2.934023in}}%
\pgfpathlineto{\pgfqpoint{3.985217in}{2.949331in}}%
\pgfpathlineto{\pgfqpoint{3.971886in}{2.957813in}}%
\pgfpathlineto{\pgfqpoint{3.958557in}{2.966518in}}%
\pgfpathlineto{\pgfqpoint{3.945232in}{2.975448in}}%
\pgfpathlineto{\pgfqpoint{3.931909in}{2.984604in}}%
\pgfpathlineto{\pgfqpoint{3.924176in}{2.968877in}}%
\pgfpathlineto{\pgfqpoint{3.916440in}{2.953340in}}%
\pgfpathlineto{\pgfqpoint{3.908700in}{2.937987in}}%
\pgfpathlineto{\pgfqpoint{3.900956in}{2.922815in}}%
\pgfpathclose%
\pgfusepath{fill}%
\end{pgfscope}%
\begin{pgfscope}%
\pgfpathrectangle{\pgfqpoint{1.150000in}{0.150000in}}{\pgfqpoint{5.700000in}{5.700000in}}%
\pgfusepath{clip}%
\pgfsetbuttcap%
\pgfsetroundjoin%
\definecolor{currentfill}{rgb}{0.119699,0.618490,0.536347}%
\pgfsetfillcolor{currentfill}%
\pgfsetfillopacity{0.800000}%
\pgfsetlinewidth{0.000000pt}%
\definecolor{currentstroke}{rgb}{0.000000,0.000000,0.000000}%
\pgfsetstrokecolor{currentstroke}%
\pgfsetdash{}{0pt}%
\pgfpathmoveto{\pgfqpoint{3.183905in}{3.910898in}}%
\pgfpathlineto{\pgfqpoint{3.197412in}{3.886075in}}%
\pgfpathlineto{\pgfqpoint{3.210908in}{3.861605in}}%
\pgfpathlineto{\pgfqpoint{3.224395in}{3.837483in}}%
\pgfpathlineto{\pgfqpoint{3.237872in}{3.813708in}}%
\pgfpathlineto{\pgfqpoint{3.245655in}{3.835311in}}%
\pgfpathlineto{\pgfqpoint{3.253432in}{3.857222in}}%
\pgfpathlineto{\pgfqpoint{3.261203in}{3.879446in}}%
\pgfpathlineto{\pgfqpoint{3.268968in}{3.901990in}}%
\pgfpathlineto{\pgfqpoint{3.255493in}{3.926233in}}%
\pgfpathlineto{\pgfqpoint{3.242008in}{3.950823in}}%
\pgfpathlineto{\pgfqpoint{3.228514in}{3.975763in}}%
\pgfpathlineto{\pgfqpoint{3.215010in}{4.001057in}}%
\pgfpathlineto{\pgfqpoint{3.207243in}{3.978030in}}%
\pgfpathlineto{\pgfqpoint{3.199470in}{3.955332in}}%
\pgfpathlineto{\pgfqpoint{3.191691in}{3.932957in}}%
\pgfpathlineto{\pgfqpoint{3.183905in}{3.910898in}}%
\pgfpathclose%
\pgfusepath{fill}%
\end{pgfscope}%
\begin{pgfscope}%
\pgfpathrectangle{\pgfqpoint{1.150000in}{0.150000in}}{\pgfqpoint{5.700000in}{5.700000in}}%
\pgfusepath{clip}%
\pgfsetbuttcap%
\pgfsetroundjoin%
\definecolor{currentfill}{rgb}{0.449368,0.813768,0.335384}%
\pgfsetfillcolor{currentfill}%
\pgfsetfillopacity{0.800000}%
\pgfsetlinewidth{0.000000pt}%
\definecolor{currentstroke}{rgb}{0.000000,0.000000,0.000000}%
\pgfsetstrokecolor{currentstroke}%
\pgfsetdash{}{0pt}%
\pgfpathmoveto{\pgfqpoint{3.369097in}{4.540125in}}%
\pgfpathlineto{\pgfqpoint{3.382611in}{4.511916in}}%
\pgfpathlineto{\pgfqpoint{3.396114in}{4.484069in}}%
\pgfpathlineto{\pgfqpoint{3.409606in}{4.456579in}}%
\pgfpathlineto{\pgfqpoint{3.423088in}{4.429443in}}%
\pgfpathlineto{\pgfqpoint{3.430742in}{4.460171in}}%
\pgfpathlineto{\pgfqpoint{3.438391in}{4.491370in}}%
\pgfpathlineto{\pgfqpoint{3.446037in}{4.523048in}}%
\pgfpathlineto{\pgfqpoint{3.453678in}{4.555212in}}%
\pgfpathlineto{\pgfqpoint{3.440190in}{4.583044in}}%
\pgfpathlineto{\pgfqpoint{3.426691in}{4.611231in}}%
\pgfpathlineto{\pgfqpoint{3.413182in}{4.639779in}}%
\pgfpathlineto{\pgfqpoint{3.399663in}{4.668689in}}%
\pgfpathlineto{\pgfqpoint{3.392028in}{4.635811in}}%
\pgfpathlineto{\pgfqpoint{3.384389in}{4.603429in}}%
\pgfpathlineto{\pgfqpoint{3.376746in}{4.571537in}}%
\pgfpathlineto{\pgfqpoint{3.369097in}{4.540125in}}%
\pgfpathclose%
\pgfusepath{fill}%
\end{pgfscope}%
\begin{pgfscope}%
\pgfpathrectangle{\pgfqpoint{1.150000in}{0.150000in}}{\pgfqpoint{5.700000in}{5.700000in}}%
\pgfusepath{clip}%
\pgfsetbuttcap%
\pgfsetroundjoin%
\definecolor{currentfill}{rgb}{0.246811,0.283237,0.535941}%
\pgfsetfillcolor{currentfill}%
\pgfsetfillopacity{0.800000}%
\pgfsetlinewidth{0.000000pt}%
\definecolor{currentstroke}{rgb}{0.000000,0.000000,0.000000}%
\pgfsetstrokecolor{currentstroke}%
\pgfsetdash{}{0pt}%
\pgfpathmoveto{\pgfqpoint{4.122810in}{2.949013in}}%
\pgfpathlineto{\pgfqpoint{4.136169in}{2.942049in}}%
\pgfpathlineto{\pgfqpoint{4.149532in}{2.935298in}}%
\pgfpathlineto{\pgfqpoint{4.162900in}{2.928758in}}%
\pgfpathlineto{\pgfqpoint{4.176272in}{2.922427in}}%
\pgfpathlineto{\pgfqpoint{4.183961in}{2.937348in}}%
\pgfpathlineto{\pgfqpoint{4.191646in}{2.952458in}}%
\pgfpathlineto{\pgfqpoint{4.199329in}{2.967763in}}%
\pgfpathlineto{\pgfqpoint{4.207009in}{2.983269in}}%
\pgfpathlineto{\pgfqpoint{4.193643in}{2.990068in}}%
\pgfpathlineto{\pgfqpoint{4.180283in}{2.997078in}}%
\pgfpathlineto{\pgfqpoint{4.166927in}{3.004298in}}%
\pgfpathlineto{\pgfqpoint{4.153575in}{3.011731in}}%
\pgfpathlineto{\pgfqpoint{4.145889in}{2.995744in}}%
\pgfpathlineto{\pgfqpoint{4.138199in}{2.979966in}}%
\pgfpathlineto{\pgfqpoint{4.130506in}{2.964391in}}%
\pgfpathlineto{\pgfqpoint{4.122810in}{2.949013in}}%
\pgfpathclose%
\pgfusepath{fill}%
\end{pgfscope}%
\begin{pgfscope}%
\pgfpathrectangle{\pgfqpoint{1.150000in}{0.150000in}}{\pgfqpoint{5.700000in}{5.700000in}}%
\pgfusepath{clip}%
\pgfsetbuttcap%
\pgfsetroundjoin%
\definecolor{currentfill}{rgb}{0.185556,0.418570,0.556753}%
\pgfsetfillcolor{currentfill}%
\pgfsetfillopacity{0.800000}%
\pgfsetlinewidth{0.000000pt}%
\definecolor{currentstroke}{rgb}{0.000000,0.000000,0.000000}%
\pgfsetstrokecolor{currentstroke}%
\pgfsetdash{}{0pt}%
\pgfpathmoveto{\pgfqpoint{4.796462in}{3.299161in}}%
\pgfpathlineto{\pgfqpoint{4.809948in}{3.293927in}}%
\pgfpathlineto{\pgfqpoint{4.823441in}{3.288879in}}%
\pgfpathlineto{\pgfqpoint{4.836943in}{3.284018in}}%
\pgfpathlineto{\pgfqpoint{4.850452in}{3.279343in}}%
\pgfpathlineto{\pgfqpoint{4.858043in}{3.297828in}}%
\pgfpathlineto{\pgfqpoint{4.865638in}{3.316660in}}%
\pgfpathlineto{\pgfqpoint{4.873235in}{3.335847in}}%
\pgfpathlineto{\pgfqpoint{4.880837in}{3.355398in}}%
\pgfpathlineto{\pgfqpoint{4.867340in}{3.360794in}}%
\pgfpathlineto{\pgfqpoint{4.853851in}{3.366376in}}%
\pgfpathlineto{\pgfqpoint{4.840369in}{3.372144in}}%
\pgfpathlineto{\pgfqpoint{4.826896in}{3.378099in}}%
\pgfpathlineto{\pgfqpoint{4.819283in}{3.357815in}}%
\pgfpathlineto{\pgfqpoint{4.811673in}{3.337904in}}%
\pgfpathlineto{\pgfqpoint{4.804066in}{3.318355in}}%
\pgfpathlineto{\pgfqpoint{4.796462in}{3.299161in}}%
\pgfpathclose%
\pgfusepath{fill}%
\end{pgfscope}%
\begin{pgfscope}%
\pgfpathrectangle{\pgfqpoint{1.150000in}{0.150000in}}{\pgfqpoint{5.700000in}{5.700000in}}%
\pgfusepath{clip}%
\pgfsetbuttcap%
\pgfsetroundjoin%
\definecolor{currentfill}{rgb}{0.172719,0.448791,0.557885}%
\pgfsetfillcolor{currentfill}%
\pgfsetfillopacity{0.800000}%
\pgfsetlinewidth{0.000000pt}%
\definecolor{currentstroke}{rgb}{0.000000,0.000000,0.000000}%
\pgfsetstrokecolor{currentstroke}%
\pgfsetdash{}{0pt}%
\pgfpathmoveto{\pgfqpoint{3.251595in}{3.407934in}}%
\pgfpathlineto{\pgfqpoint{3.265013in}{3.388294in}}%
\pgfpathlineto{\pgfqpoint{3.278425in}{3.368966in}}%
\pgfpathlineto{\pgfqpoint{3.291831in}{3.349947in}}%
\pgfpathlineto{\pgfqpoint{3.305230in}{3.331234in}}%
\pgfpathlineto{\pgfqpoint{3.313073in}{3.348494in}}%
\pgfpathlineto{\pgfqpoint{3.320909in}{3.365979in}}%
\pgfpathlineto{\pgfqpoint{3.328740in}{3.383693in}}%
\pgfpathlineto{\pgfqpoint{3.336565in}{3.401641in}}%
\pgfpathlineto{\pgfqpoint{3.323172in}{3.420709in}}%
\pgfpathlineto{\pgfqpoint{3.309772in}{3.440083in}}%
\pgfpathlineto{\pgfqpoint{3.296366in}{3.459768in}}%
\pgfpathlineto{\pgfqpoint{3.282953in}{3.479765in}}%
\pgfpathlineto{\pgfqpoint{3.275123in}{3.461449in}}%
\pgfpathlineto{\pgfqpoint{3.267286in}{3.443375in}}%
\pgfpathlineto{\pgfqpoint{3.259444in}{3.425538in}}%
\pgfpathlineto{\pgfqpoint{3.251595in}{3.407934in}}%
\pgfpathclose%
\pgfusepath{fill}%
\end{pgfscope}%
\begin{pgfscope}%
\pgfpathrectangle{\pgfqpoint{1.150000in}{0.150000in}}{\pgfqpoint{5.700000in}{5.700000in}}%
\pgfusepath{clip}%
\pgfsetbuttcap%
\pgfsetroundjoin%
\definecolor{currentfill}{rgb}{0.585678,0.846661,0.249897}%
\pgfsetfillcolor{currentfill}%
\pgfsetfillopacity{0.800000}%
\pgfsetlinewidth{0.000000pt}%
\definecolor{currentstroke}{rgb}{0.000000,0.000000,0.000000}%
\pgfsetstrokecolor{currentstroke}%
\pgfsetdash{}{0pt}%
\pgfpathmoveto{\pgfqpoint{3.484204in}{4.688908in}}%
\pgfpathlineto{\pgfqpoint{3.497689in}{4.660693in}}%
\pgfpathlineto{\pgfqpoint{3.511165in}{4.632828in}}%
\pgfpathlineto{\pgfqpoint{3.524631in}{4.605310in}}%
\pgfpathlineto{\pgfqpoint{3.538088in}{4.578136in}}%
\pgfpathlineto{\pgfqpoint{3.545719in}{4.612105in}}%
\pgfpathlineto{\pgfqpoint{3.553346in}{4.646604in}}%
\pgfpathlineto{\pgfqpoint{3.560971in}{4.681641in}}%
\pgfpathlineto{\pgfqpoint{3.568593in}{4.717227in}}%
\pgfpathlineto{\pgfqpoint{3.555128in}{4.745173in}}%
\pgfpathlineto{\pgfqpoint{3.541653in}{4.773464in}}%
\pgfpathlineto{\pgfqpoint{3.528169in}{4.802105in}}%
\pgfpathlineto{\pgfqpoint{3.514674in}{4.831098in}}%
\pgfpathlineto{\pgfqpoint{3.507061in}{4.794722in}}%
\pgfpathlineto{\pgfqpoint{3.499445in}{4.758904in}}%
\pgfpathlineto{\pgfqpoint{3.491826in}{4.723636in}}%
\pgfpathlineto{\pgfqpoint{3.484204in}{4.688908in}}%
\pgfpathclose%
\pgfusepath{fill}%
\end{pgfscope}%
\begin{pgfscope}%
\pgfpathrectangle{\pgfqpoint{1.150000in}{0.150000in}}{\pgfqpoint{5.700000in}{5.700000in}}%
\pgfusepath{clip}%
\pgfsetbuttcap%
\pgfsetroundjoin%
\definecolor{currentfill}{rgb}{0.244972,0.287675,0.537260}%
\pgfsetfillcolor{currentfill}%
\pgfsetfillopacity{0.800000}%
\pgfsetlinewidth{0.000000pt}%
\definecolor{currentstroke}{rgb}{0.000000,0.000000,0.000000}%
\pgfsetstrokecolor{currentstroke}%
\pgfsetdash{}{0pt}%
\pgfpathmoveto{\pgfqpoint{3.625602in}{2.967599in}}%
\pgfpathlineto{\pgfqpoint{3.638925in}{2.955772in}}%
\pgfpathlineto{\pgfqpoint{3.652248in}{2.944194in}}%
\pgfpathlineto{\pgfqpoint{3.665571in}{2.932863in}}%
\pgfpathlineto{\pgfqpoint{3.678894in}{2.921778in}}%
\pgfpathlineto{\pgfqpoint{3.686692in}{2.936674in}}%
\pgfpathlineto{\pgfqpoint{3.694485in}{2.951738in}}%
\pgfpathlineto{\pgfqpoint{3.702274in}{2.966974in}}%
\pgfpathlineto{\pgfqpoint{3.710058in}{2.982387in}}%
\pgfpathlineto{\pgfqpoint{3.696741in}{2.993819in}}%
\pgfpathlineto{\pgfqpoint{3.683425in}{3.005496in}}%
\pgfpathlineto{\pgfqpoint{3.670108in}{3.017421in}}%
\pgfpathlineto{\pgfqpoint{3.656792in}{3.029595in}}%
\pgfpathlineto{\pgfqpoint{3.649001in}{3.013823in}}%
\pgfpathlineto{\pgfqpoint{3.641206in}{2.998237in}}%
\pgfpathlineto{\pgfqpoint{3.633406in}{2.982830in}}%
\pgfpathlineto{\pgfqpoint{3.625602in}{2.967599in}}%
\pgfpathclose%
\pgfusepath{fill}%
\end{pgfscope}%
\begin{pgfscope}%
\pgfpathrectangle{\pgfqpoint{1.150000in}{0.150000in}}{\pgfqpoint{5.700000in}{5.700000in}}%
\pgfusepath{clip}%
\pgfsetbuttcap%
\pgfsetroundjoin%
\definecolor{currentfill}{rgb}{0.252194,0.269783,0.531579}%
\pgfsetfillcolor{currentfill}%
\pgfsetfillopacity{0.800000}%
\pgfsetlinewidth{0.000000pt}%
\definecolor{currentstroke}{rgb}{0.000000,0.000000,0.000000}%
\pgfsetstrokecolor{currentstroke}%
\pgfsetdash{}{0pt}%
\pgfpathmoveto{\pgfqpoint{4.038576in}{2.917610in}}%
\pgfpathlineto{\pgfqpoint{4.051924in}{2.910226in}}%
\pgfpathlineto{\pgfqpoint{4.065277in}{2.903059in}}%
\pgfpathlineto{\pgfqpoint{4.078634in}{2.896106in}}%
\pgfpathlineto{\pgfqpoint{4.091995in}{2.889368in}}%
\pgfpathlineto{\pgfqpoint{4.099704in}{2.904009in}}%
\pgfpathlineto{\pgfqpoint{4.107409in}{2.918827in}}%
\pgfpathlineto{\pgfqpoint{4.115111in}{2.933827in}}%
\pgfpathlineto{\pgfqpoint{4.122810in}{2.949013in}}%
\pgfpathlineto{\pgfqpoint{4.109456in}{2.956189in}}%
\pgfpathlineto{\pgfqpoint{4.096106in}{2.963580in}}%
\pgfpathlineto{\pgfqpoint{4.082760in}{2.971186in}}%
\pgfpathlineto{\pgfqpoint{4.069418in}{2.979008in}}%
\pgfpathlineto{\pgfqpoint{4.061713in}{2.963372in}}%
\pgfpathlineto{\pgfqpoint{4.054004in}{2.947930in}}%
\pgfpathlineto{\pgfqpoint{4.046291in}{2.932678in}}%
\pgfpathlineto{\pgfqpoint{4.038576in}{2.917610in}}%
\pgfpathclose%
\pgfusepath{fill}%
\end{pgfscope}%
\begin{pgfscope}%
\pgfpathrectangle{\pgfqpoint{1.150000in}{0.150000in}}{\pgfqpoint{5.700000in}{5.700000in}}%
\pgfusepath{clip}%
\pgfsetbuttcap%
\pgfsetroundjoin%
\definecolor{currentfill}{rgb}{0.140536,0.530132,0.555659}%
\pgfsetfillcolor{currentfill}%
\pgfsetfillopacity{0.800000}%
\pgfsetlinewidth{0.000000pt}%
\definecolor{currentstroke}{rgb}{0.000000,0.000000,0.000000}%
\pgfsetstrokecolor{currentstroke}%
\pgfsetdash{}{0pt}%
\pgfpathmoveto{\pgfqpoint{3.175377in}{3.651350in}}%
\pgfpathlineto{\pgfqpoint{3.188853in}{3.628744in}}%
\pgfpathlineto{\pgfqpoint{3.202321in}{3.606475in}}%
\pgfpathlineto{\pgfqpoint{3.215780in}{3.584540in}}%
\pgfpathlineto{\pgfqpoint{3.229230in}{3.562936in}}%
\pgfpathlineto{\pgfqpoint{3.237059in}{3.581878in}}%
\pgfpathlineto{\pgfqpoint{3.244883in}{3.601079in}}%
\pgfpathlineto{\pgfqpoint{3.252699in}{3.620543in}}%
\pgfpathlineto{\pgfqpoint{3.260510in}{3.640277in}}%
\pgfpathlineto{\pgfqpoint{3.247064in}{3.662275in}}%
\pgfpathlineto{\pgfqpoint{3.233611in}{3.684604in}}%
\pgfpathlineto{\pgfqpoint{3.220148in}{3.707268in}}%
\pgfpathlineto{\pgfqpoint{3.206676in}{3.730271in}}%
\pgfpathlineto{\pgfqpoint{3.198861in}{3.710129in}}%
\pgfpathlineto{\pgfqpoint{3.191040in}{3.690265in}}%
\pgfpathlineto{\pgfqpoint{3.183212in}{3.670674in}}%
\pgfpathlineto{\pgfqpoint{3.175377in}{3.651350in}}%
\pgfpathclose%
\pgfusepath{fill}%
\end{pgfscope}%
\begin{pgfscope}%
\pgfpathrectangle{\pgfqpoint{1.150000in}{0.150000in}}{\pgfqpoint{5.700000in}{5.700000in}}%
\pgfusepath{clip}%
\pgfsetbuttcap%
\pgfsetroundjoin%
\definecolor{currentfill}{rgb}{0.177423,0.437527,0.557565}%
\pgfsetfillcolor{currentfill}%
\pgfsetfillopacity{0.800000}%
\pgfsetlinewidth{0.000000pt}%
\definecolor{currentstroke}{rgb}{0.000000,0.000000,0.000000}%
\pgfsetstrokecolor{currentstroke}%
\pgfsetdash{}{0pt}%
\pgfpathmoveto{\pgfqpoint{4.880837in}{3.355398in}}%
\pgfpathlineto{\pgfqpoint{4.894341in}{3.350187in}}%
\pgfpathlineto{\pgfqpoint{4.907854in}{3.345161in}}%
\pgfpathlineto{\pgfqpoint{4.921375in}{3.340318in}}%
\pgfpathlineto{\pgfqpoint{4.934904in}{3.335659in}}%
\pgfpathlineto{\pgfqpoint{4.942495in}{3.354842in}}%
\pgfpathlineto{\pgfqpoint{4.950091in}{3.374398in}}%
\pgfpathlineto{\pgfqpoint{4.957691in}{3.394337in}}%
\pgfpathlineto{\pgfqpoint{4.944172in}{3.399557in}}%
\pgfpathlineto{\pgfqpoint{4.930661in}{3.404961in}}%
\pgfpathlineto{\pgfqpoint{4.917158in}{3.410549in}}%
\pgfpathlineto{\pgfqpoint{4.903663in}{3.416321in}}%
\pgfpathlineto{\pgfqpoint{4.896050in}{3.395626in}}%
\pgfpathlineto{\pgfqpoint{4.888441in}{3.375322in}}%
\pgfpathlineto{\pgfqpoint{4.880837in}{3.355398in}}%
\pgfpathclose%
\pgfusepath{fill}%
\end{pgfscope}%
\begin{pgfscope}%
\pgfpathrectangle{\pgfqpoint{1.150000in}{0.150000in}}{\pgfqpoint{5.700000in}{5.700000in}}%
\pgfusepath{clip}%
\pgfsetbuttcap%
\pgfsetroundjoin%
\definecolor{currentfill}{rgb}{0.216210,0.351535,0.550627}%
\pgfsetfillcolor{currentfill}%
\pgfsetfillopacity{0.800000}%
\pgfsetlinewidth{0.000000pt}%
\definecolor{currentstroke}{rgb}{0.000000,0.000000,0.000000}%
\pgfsetstrokecolor{currentstroke}%
\pgfsetdash{}{0pt}%
\pgfpathmoveto{\pgfqpoint{3.380856in}{3.127932in}}%
\pgfpathlineto{\pgfqpoint{3.394218in}{3.112178in}}%
\pgfpathlineto{\pgfqpoint{3.407576in}{3.096704in}}%
\pgfpathlineto{\pgfqpoint{3.420931in}{3.081509in}}%
\pgfpathlineto{\pgfqpoint{3.434283in}{3.066591in}}%
\pgfpathlineto{\pgfqpoint{3.442128in}{3.082070in}}%
\pgfpathlineto{\pgfqpoint{3.449969in}{3.097733in}}%
\pgfpathlineto{\pgfqpoint{3.457804in}{3.113585in}}%
\pgfpathlineto{\pgfqpoint{3.465633in}{3.129628in}}%
\pgfpathlineto{\pgfqpoint{3.452288in}{3.144865in}}%
\pgfpathlineto{\pgfqpoint{3.438940in}{3.160378in}}%
\pgfpathlineto{\pgfqpoint{3.425588in}{3.176170in}}%
\pgfpathlineto{\pgfqpoint{3.412233in}{3.192243in}}%
\pgfpathlineto{\pgfqpoint{3.404397in}{3.175869in}}%
\pgfpathlineto{\pgfqpoint{3.396556in}{3.159695in}}%
\pgfpathlineto{\pgfqpoint{3.388709in}{3.143717in}}%
\pgfpathlineto{\pgfqpoint{3.380856in}{3.127932in}}%
\pgfpathclose%
\pgfusepath{fill}%
\end{pgfscope}%
\begin{pgfscope}%
\pgfpathrectangle{\pgfqpoint{1.150000in}{0.150000in}}{\pgfqpoint{5.700000in}{5.700000in}}%
\pgfusepath{clip}%
\pgfsetbuttcap%
\pgfsetroundjoin%
\definecolor{currentfill}{rgb}{0.121831,0.589055,0.545623}%
\pgfsetfillcolor{currentfill}%
\pgfsetfillopacity{0.800000}%
\pgfsetlinewidth{0.000000pt}%
\definecolor{currentstroke}{rgb}{0.000000,0.000000,0.000000}%
\pgfsetstrokecolor{currentstroke}%
\pgfsetdash{}{0pt}%
\pgfpathmoveto{\pgfqpoint{3.152695in}{3.825728in}}%
\pgfpathlineto{\pgfqpoint{3.166205in}{3.801340in}}%
\pgfpathlineto{\pgfqpoint{3.179705in}{3.777303in}}%
\pgfpathlineto{\pgfqpoint{3.193196in}{3.753614in}}%
\pgfpathlineto{\pgfqpoint{3.206676in}{3.730271in}}%
\pgfpathlineto{\pgfqpoint{3.214485in}{3.750694in}}%
\pgfpathlineto{\pgfqpoint{3.222287in}{3.771405in}}%
\pgfpathlineto{\pgfqpoint{3.230083in}{3.792407in}}%
\pgfpathlineto{\pgfqpoint{3.237872in}{3.813708in}}%
\pgfpathlineto{\pgfqpoint{3.224395in}{3.837483in}}%
\pgfpathlineto{\pgfqpoint{3.210908in}{3.861605in}}%
\pgfpathlineto{\pgfqpoint{3.197412in}{3.886075in}}%
\pgfpathlineto{\pgfqpoint{3.183905in}{3.910898in}}%
\pgfpathlineto{\pgfqpoint{3.176112in}{3.889152in}}%
\pgfpathlineto{\pgfqpoint{3.168313in}{3.867711in}}%
\pgfpathlineto{\pgfqpoint{3.160507in}{3.846572in}}%
\pgfpathlineto{\pgfqpoint{3.152695in}{3.825728in}}%
\pgfpathclose%
\pgfusepath{fill}%
\end{pgfscope}%
\begin{pgfscope}%
\pgfpathrectangle{\pgfqpoint{1.150000in}{0.150000in}}{\pgfqpoint{5.700000in}{5.700000in}}%
\pgfusepath{clip}%
\pgfsetbuttcap%
\pgfsetroundjoin%
\definecolor{currentfill}{rgb}{0.221989,0.339161,0.548752}%
\pgfsetfillcolor{currentfill}%
\pgfsetfillopacity{0.800000}%
\pgfsetlinewidth{0.000000pt}%
\definecolor{currentstroke}{rgb}{0.000000,0.000000,0.000000}%
\pgfsetstrokecolor{currentstroke}%
\pgfsetdash{}{0pt}%
\pgfpathmoveto{\pgfqpoint{4.513210in}{3.080639in}}%
\pgfpathlineto{\pgfqpoint{4.526651in}{3.075679in}}%
\pgfpathlineto{\pgfqpoint{4.540098in}{3.070913in}}%
\pgfpathlineto{\pgfqpoint{4.553553in}{3.066341in}}%
\pgfpathlineto{\pgfqpoint{4.567014in}{3.061962in}}%
\pgfpathlineto{\pgfqpoint{4.574627in}{3.077711in}}%
\pgfpathlineto{\pgfqpoint{4.582239in}{3.093709in}}%
\pgfpathlineto{\pgfqpoint{4.589851in}{3.109962in}}%
\pgfpathlineto{\pgfqpoint{4.597462in}{3.126479in}}%
\pgfpathlineto{\pgfqpoint{4.584010in}{3.131451in}}%
\pgfpathlineto{\pgfqpoint{4.570566in}{3.136617in}}%
\pgfpathlineto{\pgfqpoint{4.557128in}{3.141977in}}%
\pgfpathlineto{\pgfqpoint{4.543697in}{3.147532in}}%
\pgfpathlineto{\pgfqpoint{4.536076in}{3.130410in}}%
\pgfpathlineto{\pgfqpoint{4.528455in}{3.113559in}}%
\pgfpathlineto{\pgfqpoint{4.520833in}{3.096971in}}%
\pgfpathlineto{\pgfqpoint{4.513210in}{3.080639in}}%
\pgfpathclose%
\pgfusepath{fill}%
\end{pgfscope}%
\begin{pgfscope}%
\pgfpathrectangle{\pgfqpoint{1.150000in}{0.150000in}}{\pgfqpoint{5.700000in}{5.700000in}}%
\pgfusepath{clip}%
\pgfsetbuttcap%
\pgfsetroundjoin%
\definecolor{currentfill}{rgb}{0.253935,0.265254,0.529983}%
\pgfsetfillcolor{currentfill}%
\pgfsetfillopacity{0.800000}%
\pgfsetlinewidth{0.000000pt}%
\definecolor{currentstroke}{rgb}{0.000000,0.000000,0.000000}%
\pgfsetstrokecolor{currentstroke}%
\pgfsetdash{}{0pt}%
\pgfpathmoveto{\pgfqpoint{3.816620in}{2.899603in}}%
\pgfpathlineto{\pgfqpoint{3.829947in}{2.890315in}}%
\pgfpathlineto{\pgfqpoint{3.843276in}{2.881259in}}%
\pgfpathlineto{\pgfqpoint{3.856607in}{2.872431in}}%
\pgfpathlineto{\pgfqpoint{3.869940in}{2.863832in}}%
\pgfpathlineto{\pgfqpoint{3.877700in}{2.878331in}}%
\pgfpathlineto{\pgfqpoint{3.885456in}{2.892991in}}%
\pgfpathlineto{\pgfqpoint{3.893208in}{2.907818in}}%
\pgfpathlineto{\pgfqpoint{3.900956in}{2.922815in}}%
\pgfpathlineto{\pgfqpoint{3.887629in}{2.931790in}}%
\pgfpathlineto{\pgfqpoint{3.874305in}{2.940994in}}%
\pgfpathlineto{\pgfqpoint{3.860983in}{2.950427in}}%
\pgfpathlineto{\pgfqpoint{3.847662in}{2.960091in}}%
\pgfpathlineto{\pgfqpoint{3.839908in}{2.944706in}}%
\pgfpathlineto{\pgfqpoint{3.832150in}{2.929499in}}%
\pgfpathlineto{\pgfqpoint{3.824387in}{2.914466in}}%
\pgfpathlineto{\pgfqpoint{3.816620in}{2.899603in}}%
\pgfpathclose%
\pgfusepath{fill}%
\end{pgfscope}%
\begin{pgfscope}%
\pgfpathrectangle{\pgfqpoint{1.150000in}{0.150000in}}{\pgfqpoint{5.700000in}{5.700000in}}%
\pgfusepath{clip}%
\pgfsetbuttcap%
\pgfsetroundjoin%
\definecolor{currentfill}{rgb}{0.229739,0.322361,0.545706}%
\pgfsetfillcolor{currentfill}%
\pgfsetfillopacity{0.800000}%
\pgfsetlinewidth{0.000000pt}%
\definecolor{currentstroke}{rgb}{0.000000,0.000000,0.000000}%
\pgfsetstrokecolor{currentstroke}%
\pgfsetdash{}{0pt}%
\pgfpathmoveto{\pgfqpoint{4.428977in}{3.037299in}}%
\pgfpathlineto{\pgfqpoint{4.442400in}{3.032114in}}%
\pgfpathlineto{\pgfqpoint{4.455830in}{3.027128in}}%
\pgfpathlineto{\pgfqpoint{4.469266in}{3.022338in}}%
\pgfpathlineto{\pgfqpoint{4.482709in}{3.017745in}}%
\pgfpathlineto{\pgfqpoint{4.490336in}{3.033117in}}%
\pgfpathlineto{\pgfqpoint{4.497962in}{3.048719in}}%
\pgfpathlineto{\pgfqpoint{4.505587in}{3.064558in}}%
\pgfpathlineto{\pgfqpoint{4.513210in}{3.080639in}}%
\pgfpathlineto{\pgfqpoint{4.499777in}{3.085795in}}%
\pgfpathlineto{\pgfqpoint{4.486350in}{3.091148in}}%
\pgfpathlineto{\pgfqpoint{4.472930in}{3.096697in}}%
\pgfpathlineto{\pgfqpoint{4.459516in}{3.102444in}}%
\pgfpathlineto{\pgfqpoint{4.451883in}{3.085788in}}%
\pgfpathlineto{\pgfqpoint{4.444249in}{3.069383in}}%
\pgfpathlineto{\pgfqpoint{4.436614in}{3.053222in}}%
\pgfpathlineto{\pgfqpoint{4.428977in}{3.037299in}}%
\pgfpathclose%
\pgfusepath{fill}%
\end{pgfscope}%
\begin{pgfscope}%
\pgfpathrectangle{\pgfqpoint{1.150000in}{0.150000in}}{\pgfqpoint{5.700000in}{5.700000in}}%
\pgfusepath{clip}%
\pgfsetbuttcap%
\pgfsetroundjoin%
\definecolor{currentfill}{rgb}{0.227802,0.326594,0.546532}%
\pgfsetfillcolor{currentfill}%
\pgfsetfillopacity{0.800000}%
\pgfsetlinewidth{0.000000pt}%
\definecolor{currentstroke}{rgb}{0.000000,0.000000,0.000000}%
\pgfsetstrokecolor{currentstroke}%
\pgfsetdash{}{0pt}%
\pgfpathmoveto{\pgfqpoint{3.434283in}{3.066591in}}%
\pgfpathlineto{\pgfqpoint{3.447631in}{3.051946in}}%
\pgfpathlineto{\pgfqpoint{3.460977in}{3.037574in}}%
\pgfpathlineto{\pgfqpoint{3.474321in}{3.023471in}}%
\pgfpathlineto{\pgfqpoint{3.487662in}{3.009636in}}%
\pgfpathlineto{\pgfqpoint{3.495500in}{3.024810in}}%
\pgfpathlineto{\pgfqpoint{3.503334in}{3.040160in}}%
\pgfpathlineto{\pgfqpoint{3.511162in}{3.055690in}}%
\pgfpathlineto{\pgfqpoint{3.518985in}{3.071404in}}%
\pgfpathlineto{\pgfqpoint{3.505650in}{3.085556in}}%
\pgfpathlineto{\pgfqpoint{3.492314in}{3.099976in}}%
\pgfpathlineto{\pgfqpoint{3.478975in}{3.114666in}}%
\pgfpathlineto{\pgfqpoint{3.465633in}{3.129628in}}%
\pgfpathlineto{\pgfqpoint{3.457804in}{3.113585in}}%
\pgfpathlineto{\pgfqpoint{3.449969in}{3.097733in}}%
\pgfpathlineto{\pgfqpoint{3.442128in}{3.082070in}}%
\pgfpathlineto{\pgfqpoint{3.434283in}{3.066591in}}%
\pgfpathclose%
\pgfusepath{fill}%
\end{pgfscope}%
\begin{pgfscope}%
\pgfpathrectangle{\pgfqpoint{1.150000in}{0.150000in}}{\pgfqpoint{5.700000in}{5.700000in}}%
\pgfusepath{clip}%
\pgfsetbuttcap%
\pgfsetroundjoin%
\definecolor{currentfill}{rgb}{0.162142,0.474838,0.558140}%
\pgfsetfillcolor{currentfill}%
\pgfsetfillopacity{0.800000}%
\pgfsetlinewidth{0.000000pt}%
\definecolor{currentstroke}{rgb}{0.000000,0.000000,0.000000}%
\pgfsetstrokecolor{currentstroke}%
\pgfsetdash{}{0pt}%
\pgfpathmoveto{\pgfqpoint{3.197848in}{3.489673in}}%
\pgfpathlineto{\pgfqpoint{3.211296in}{3.468755in}}%
\pgfpathlineto{\pgfqpoint{3.224736in}{3.448162in}}%
\pgfpathlineto{\pgfqpoint{3.238169in}{3.427889in}}%
\pgfpathlineto{\pgfqpoint{3.251595in}{3.407934in}}%
\pgfpathlineto{\pgfqpoint{3.259444in}{3.425538in}}%
\pgfpathlineto{\pgfqpoint{3.267286in}{3.443375in}}%
\pgfpathlineto{\pgfqpoint{3.275123in}{3.461449in}}%
\pgfpathlineto{\pgfqpoint{3.282953in}{3.479765in}}%
\pgfpathlineto{\pgfqpoint{3.269534in}{3.500077in}}%
\pgfpathlineto{\pgfqpoint{3.256107in}{3.520708in}}%
\pgfpathlineto{\pgfqpoint{3.242672in}{3.541660in}}%
\pgfpathlineto{\pgfqpoint{3.229230in}{3.562936in}}%
\pgfpathlineto{\pgfqpoint{3.221394in}{3.544250in}}%
\pgfpathlineto{\pgfqpoint{3.213552in}{3.525813in}}%
\pgfpathlineto{\pgfqpoint{3.205703in}{3.507622in}}%
\pgfpathlineto{\pgfqpoint{3.197848in}{3.489673in}}%
\pgfpathclose%
\pgfusepath{fill}%
\end{pgfscope}%
\begin{pgfscope}%
\pgfpathrectangle{\pgfqpoint{1.150000in}{0.150000in}}{\pgfqpoint{5.700000in}{5.700000in}}%
\pgfusepath{clip}%
\pgfsetbuttcap%
\pgfsetroundjoin%
\definecolor{currentfill}{rgb}{0.212395,0.359683,0.551710}%
\pgfsetfillcolor{currentfill}%
\pgfsetfillopacity{0.800000}%
\pgfsetlinewidth{0.000000pt}%
\definecolor{currentstroke}{rgb}{0.000000,0.000000,0.000000}%
\pgfsetstrokecolor{currentstroke}%
\pgfsetdash{}{0pt}%
\pgfpathmoveto{\pgfqpoint{4.597462in}{3.126479in}}%
\pgfpathlineto{\pgfqpoint{4.610921in}{3.121698in}}%
\pgfpathlineto{\pgfqpoint{4.624387in}{3.117110in}}%
\pgfpathlineto{\pgfqpoint{4.637861in}{3.112713in}}%
\pgfpathlineto{\pgfqpoint{4.651342in}{3.108507in}}%
\pgfpathlineto{\pgfqpoint{4.658942in}{3.124679in}}%
\pgfpathlineto{\pgfqpoint{4.666543in}{3.141120in}}%
\pgfpathlineto{\pgfqpoint{4.674143in}{3.157838in}}%
\pgfpathlineto{\pgfqpoint{4.681744in}{3.174839in}}%
\pgfpathlineto{\pgfqpoint{4.668274in}{3.179671in}}%
\pgfpathlineto{\pgfqpoint{4.654811in}{3.184694in}}%
\pgfpathlineto{\pgfqpoint{4.641356in}{3.189907in}}%
\pgfpathlineto{\pgfqpoint{4.627907in}{3.195313in}}%
\pgfpathlineto{\pgfqpoint{4.620295in}{3.177674in}}%
\pgfpathlineto{\pgfqpoint{4.612684in}{3.160327in}}%
\pgfpathlineto{\pgfqpoint{4.605073in}{3.143264in}}%
\pgfpathlineto{\pgfqpoint{4.597462in}{3.126479in}}%
\pgfpathclose%
\pgfusepath{fill}%
\end{pgfscope}%
\begin{pgfscope}%
\pgfpathrectangle{\pgfqpoint{1.150000in}{0.150000in}}{\pgfqpoint{5.700000in}{5.700000in}}%
\pgfusepath{clip}%
\pgfsetbuttcap%
\pgfsetroundjoin%
\definecolor{currentfill}{rgb}{0.204903,0.375746,0.553533}%
\pgfsetfillcolor{currentfill}%
\pgfsetfillopacity{0.800000}%
\pgfsetlinewidth{0.000000pt}%
\definecolor{currentstroke}{rgb}{0.000000,0.000000,0.000000}%
\pgfsetstrokecolor{currentstroke}%
\pgfsetdash{}{0pt}%
\pgfpathmoveto{\pgfqpoint{3.327366in}{3.193803in}}%
\pgfpathlineto{\pgfqpoint{3.340745in}{3.176902in}}%
\pgfpathlineto{\pgfqpoint{3.354120in}{3.160292in}}%
\pgfpathlineto{\pgfqpoint{3.367490in}{3.143969in}}%
\pgfpathlineto{\pgfqpoint{3.380856in}{3.127932in}}%
\pgfpathlineto{\pgfqpoint{3.388709in}{3.143717in}}%
\pgfpathlineto{\pgfqpoint{3.396556in}{3.159695in}}%
\pgfpathlineto{\pgfqpoint{3.404397in}{3.175869in}}%
\pgfpathlineto{\pgfqpoint{3.412233in}{3.192243in}}%
\pgfpathlineto{\pgfqpoint{3.398874in}{3.208599in}}%
\pgfpathlineto{\pgfqpoint{3.385511in}{3.225242in}}%
\pgfpathlineto{\pgfqpoint{3.372143in}{3.242172in}}%
\pgfpathlineto{\pgfqpoint{3.358771in}{3.259393in}}%
\pgfpathlineto{\pgfqpoint{3.350928in}{3.242687in}}%
\pgfpathlineto{\pgfqpoint{3.343080in}{3.226189in}}%
\pgfpathlineto{\pgfqpoint{3.335226in}{3.209896in}}%
\pgfpathlineto{\pgfqpoint{3.327366in}{3.193803in}}%
\pgfpathclose%
\pgfusepath{fill}%
\end{pgfscope}%
\begin{pgfscope}%
\pgfpathrectangle{\pgfqpoint{1.150000in}{0.150000in}}{\pgfqpoint{5.700000in}{5.700000in}}%
\pgfusepath{clip}%
\pgfsetbuttcap%
\pgfsetroundjoin%
\definecolor{currentfill}{rgb}{0.252194,0.269783,0.531579}%
\pgfsetfillcolor{currentfill}%
\pgfsetfillopacity{0.800000}%
\pgfsetlinewidth{0.000000pt}%
\definecolor{currentstroke}{rgb}{0.000000,0.000000,0.000000}%
\pgfsetstrokecolor{currentstroke}%
\pgfsetdash{}{0pt}%
\pgfpathmoveto{\pgfqpoint{3.678894in}{2.921778in}}%
\pgfpathlineto{\pgfqpoint{3.692217in}{2.910937in}}%
\pgfpathlineto{\pgfqpoint{3.705541in}{2.900339in}}%
\pgfpathlineto{\pgfqpoint{3.718866in}{2.889982in}}%
\pgfpathlineto{\pgfqpoint{3.732192in}{2.879864in}}%
\pgfpathlineto{\pgfqpoint{3.739983in}{2.894426in}}%
\pgfpathlineto{\pgfqpoint{3.747770in}{2.909148in}}%
\pgfpathlineto{\pgfqpoint{3.755552in}{2.924035in}}%
\pgfpathlineto{\pgfqpoint{3.763330in}{2.939090in}}%
\pgfpathlineto{\pgfqpoint{3.750011in}{2.949553in}}%
\pgfpathlineto{\pgfqpoint{3.736693in}{2.960256in}}%
\pgfpathlineto{\pgfqpoint{3.723375in}{2.971201in}}%
\pgfpathlineto{\pgfqpoint{3.710058in}{2.982387in}}%
\pgfpathlineto{\pgfqpoint{3.702274in}{2.966974in}}%
\pgfpathlineto{\pgfqpoint{3.694485in}{2.951738in}}%
\pgfpathlineto{\pgfqpoint{3.686692in}{2.936674in}}%
\pgfpathlineto{\pgfqpoint{3.678894in}{2.921778in}}%
\pgfpathclose%
\pgfusepath{fill}%
\end{pgfscope}%
\begin{pgfscope}%
\pgfpathrectangle{\pgfqpoint{1.150000in}{0.150000in}}{\pgfqpoint{5.700000in}{5.700000in}}%
\pgfusepath{clip}%
\pgfsetbuttcap%
\pgfsetroundjoin%
\definecolor{currentfill}{rgb}{0.237441,0.305202,0.541921}%
\pgfsetfillcolor{currentfill}%
\pgfsetfillopacity{0.800000}%
\pgfsetlinewidth{0.000000pt}%
\definecolor{currentstroke}{rgb}{0.000000,0.000000,0.000000}%
\pgfsetstrokecolor{currentstroke}%
\pgfsetdash{}{0pt}%
\pgfpathmoveto{\pgfqpoint{4.344751in}{2.996460in}}%
\pgfpathlineto{\pgfqpoint{4.358158in}{2.991008in}}%
\pgfpathlineto{\pgfqpoint{4.371570in}{2.985757in}}%
\pgfpathlineto{\pgfqpoint{4.384989in}{2.980706in}}%
\pgfpathlineto{\pgfqpoint{4.398414in}{2.975855in}}%
\pgfpathlineto{\pgfqpoint{4.406058in}{2.990891in}}%
\pgfpathlineto{\pgfqpoint{4.413699in}{3.006140in}}%
\pgfpathlineto{\pgfqpoint{4.421339in}{3.021607in}}%
\pgfpathlineto{\pgfqpoint{4.428977in}{3.037299in}}%
\pgfpathlineto{\pgfqpoint{4.415561in}{3.042682in}}%
\pgfpathlineto{\pgfqpoint{4.402151in}{3.048264in}}%
\pgfpathlineto{\pgfqpoint{4.388747in}{3.054046in}}%
\pgfpathlineto{\pgfqpoint{4.375349in}{3.060029in}}%
\pgfpathlineto{\pgfqpoint{4.367702in}{3.043794in}}%
\pgfpathlineto{\pgfqpoint{4.360054in}{3.027792in}}%
\pgfpathlineto{\pgfqpoint{4.352403in}{3.012016in}}%
\pgfpathlineto{\pgfqpoint{4.344751in}{2.996460in}}%
\pgfpathclose%
\pgfusepath{fill}%
\end{pgfscope}%
\begin{pgfscope}%
\pgfpathrectangle{\pgfqpoint{1.150000in}{0.150000in}}{\pgfqpoint{5.700000in}{5.700000in}}%
\pgfusepath{clip}%
\pgfsetbuttcap%
\pgfsetroundjoin%
\definecolor{currentfill}{rgb}{0.565498,0.842430,0.262877}%
\pgfsetfillcolor{currentfill}%
\pgfsetfillopacity{0.800000}%
\pgfsetlinewidth{0.000000pt}%
\definecolor{currentstroke}{rgb}{0.000000,0.000000,0.000000}%
\pgfsetstrokecolor{currentstroke}%
\pgfsetdash{}{0pt}%
\pgfpathmoveto{\pgfqpoint{3.399663in}{4.668689in}}%
\pgfpathlineto{\pgfqpoint{3.413182in}{4.639779in}}%
\pgfpathlineto{\pgfqpoint{3.426691in}{4.611231in}}%
\pgfpathlineto{\pgfqpoint{3.440190in}{4.583044in}}%
\pgfpathlineto{\pgfqpoint{3.453678in}{4.555212in}}%
\pgfpathlineto{\pgfqpoint{3.461315in}{4.587871in}}%
\pgfpathlineto{\pgfqpoint{3.468948in}{4.621035in}}%
\pgfpathlineto{\pgfqpoint{3.476578in}{4.654710in}}%
\pgfpathlineto{\pgfqpoint{3.484204in}{4.688908in}}%
\pgfpathlineto{\pgfqpoint{3.470708in}{4.717477in}}%
\pgfpathlineto{\pgfqpoint{3.457202in}{4.746404in}}%
\pgfpathlineto{\pgfqpoint{3.443686in}{4.775693in}}%
\pgfpathlineto{\pgfqpoint{3.430159in}{4.805347in}}%
\pgfpathlineto{\pgfqpoint{3.422541in}{4.770393in}}%
\pgfpathlineto{\pgfqpoint{3.414919in}{4.735971in}}%
\pgfpathlineto{\pgfqpoint{3.407293in}{4.702073in}}%
\pgfpathlineto{\pgfqpoint{3.399663in}{4.668689in}}%
\pgfpathclose%
\pgfusepath{fill}%
\end{pgfscope}%
\begin{pgfscope}%
\pgfpathrectangle{\pgfqpoint{1.150000in}{0.150000in}}{\pgfqpoint{5.700000in}{5.700000in}}%
\pgfusepath{clip}%
\pgfsetbuttcap%
\pgfsetroundjoin%
\definecolor{currentfill}{rgb}{0.204903,0.375746,0.553533}%
\pgfsetfillcolor{currentfill}%
\pgfsetfillopacity{0.800000}%
\pgfsetlinewidth{0.000000pt}%
\definecolor{currentstroke}{rgb}{0.000000,0.000000,0.000000}%
\pgfsetstrokecolor{currentstroke}%
\pgfsetdash{}{0pt}%
\pgfpathmoveto{\pgfqpoint{4.681744in}{3.174839in}}%
\pgfpathlineto{\pgfqpoint{4.695222in}{3.170197in}}%
\pgfpathlineto{\pgfqpoint{4.708708in}{3.165744in}}%
\pgfpathlineto{\pgfqpoint{4.722201in}{3.161480in}}%
\pgfpathlineto{\pgfqpoint{4.735703in}{3.157405in}}%
\pgfpathlineto{\pgfqpoint{4.743293in}{3.174052in}}%
\pgfpathlineto{\pgfqpoint{4.750884in}{3.190990in}}%
\pgfpathlineto{\pgfqpoint{4.758476in}{3.208226in}}%
\pgfpathlineto{\pgfqpoint{4.766070in}{3.225769in}}%
\pgfpathlineto{\pgfqpoint{4.752580in}{3.230502in}}%
\pgfpathlineto{\pgfqpoint{4.739098in}{3.235423in}}%
\pgfpathlineto{\pgfqpoint{4.725624in}{3.240533in}}%
\pgfpathlineto{\pgfqpoint{4.712157in}{3.245832in}}%
\pgfpathlineto{\pgfqpoint{4.704552in}{3.227620in}}%
\pgfpathlineto{\pgfqpoint{4.696949in}{3.209723in}}%
\pgfpathlineto{\pgfqpoint{4.689346in}{3.192132in}}%
\pgfpathlineto{\pgfqpoint{4.681744in}{3.174839in}}%
\pgfpathclose%
\pgfusepath{fill}%
\end{pgfscope}%
\begin{pgfscope}%
\pgfpathrectangle{\pgfqpoint{1.150000in}{0.150000in}}{\pgfqpoint{5.700000in}{5.700000in}}%
\pgfusepath{clip}%
\pgfsetbuttcap%
\pgfsetroundjoin%
\definecolor{currentfill}{rgb}{0.237441,0.305202,0.541921}%
\pgfsetfillcolor{currentfill}%
\pgfsetfillopacity{0.800000}%
\pgfsetlinewidth{0.000000pt}%
\definecolor{currentstroke}{rgb}{0.000000,0.000000,0.000000}%
\pgfsetstrokecolor{currentstroke}%
\pgfsetdash{}{0pt}%
\pgfpathmoveto{\pgfqpoint{3.487662in}{3.009636in}}%
\pgfpathlineto{\pgfqpoint{3.501001in}{2.996067in}}%
\pgfpathlineto{\pgfqpoint{3.514338in}{2.982761in}}%
\pgfpathlineto{\pgfqpoint{3.527673in}{2.969717in}}%
\pgfpathlineto{\pgfqpoint{3.541007in}{2.956933in}}%
\pgfpathlineto{\pgfqpoint{3.548839in}{2.971803in}}%
\pgfpathlineto{\pgfqpoint{3.556665in}{2.986840in}}%
\pgfpathlineto{\pgfqpoint{3.564487in}{3.002050in}}%
\pgfpathlineto{\pgfqpoint{3.572303in}{3.017436in}}%
\pgfpathlineto{\pgfqpoint{3.558976in}{3.030536in}}%
\pgfpathlineto{\pgfqpoint{3.545647in}{3.043896in}}%
\pgfpathlineto{\pgfqpoint{3.532317in}{3.057518in}}%
\pgfpathlineto{\pgfqpoint{3.518985in}{3.071404in}}%
\pgfpathlineto{\pgfqpoint{3.511162in}{3.055690in}}%
\pgfpathlineto{\pgfqpoint{3.503334in}{3.040160in}}%
\pgfpathlineto{\pgfqpoint{3.495500in}{3.024810in}}%
\pgfpathlineto{\pgfqpoint{3.487662in}{3.009636in}}%
\pgfpathclose%
\pgfusepath{fill}%
\end{pgfscope}%
\begin{pgfscope}%
\pgfpathrectangle{\pgfqpoint{1.150000in}{0.150000in}}{\pgfqpoint{5.700000in}{5.700000in}}%
\pgfusepath{clip}%
\pgfsetbuttcap%
\pgfsetroundjoin%
\definecolor{currentfill}{rgb}{0.751884,0.874951,0.143228}%
\pgfsetfillcolor{currentfill}%
\pgfsetfillopacity{0.800000}%
\pgfsetlinewidth{0.000000pt}%
\definecolor{currentstroke}{rgb}{0.000000,0.000000,0.000000}%
\pgfsetstrokecolor{currentstroke}%
\pgfsetdash{}{0pt}%
\pgfpathmoveto{\pgfqpoint{3.599058in}{4.865240in}}%
\pgfpathlineto{\pgfqpoint{3.612524in}{4.836824in}}%
\pgfpathlineto{\pgfqpoint{3.625981in}{4.808748in}}%
\pgfpathlineto{\pgfqpoint{3.639429in}{4.781010in}}%
\pgfpathlineto{\pgfqpoint{3.652868in}{4.753606in}}%
\pgfpathlineto{\pgfqpoint{3.660490in}{4.791240in}}%
\pgfpathlineto{\pgfqpoint{3.668110in}{4.829471in}}%
\pgfpathlineto{\pgfqpoint{3.675730in}{4.868308in}}%
\pgfpathlineto{\pgfqpoint{3.662282in}{4.896346in}}%
\pgfpathlineto{\pgfqpoint{3.648826in}{4.924719in}}%
\pgfpathlineto{\pgfqpoint{3.635361in}{4.953432in}}%
\pgfpathlineto{\pgfqpoint{3.621886in}{4.982488in}}%
\pgfpathlineto{\pgfqpoint{3.614278in}{4.942790in}}%
\pgfpathlineto{\pgfqpoint{3.606669in}{4.903712in}}%
\pgfpathlineto{\pgfqpoint{3.599058in}{4.865240in}}%
\pgfpathclose%
\pgfusepath{fill}%
\end{pgfscope}%
\begin{pgfscope}%
\pgfpathrectangle{\pgfqpoint{1.150000in}{0.150000in}}{\pgfqpoint{5.700000in}{5.700000in}}%
\pgfusepath{clip}%
\pgfsetbuttcap%
\pgfsetroundjoin%
\definecolor{currentfill}{rgb}{0.255645,0.260703,0.528312}%
\pgfsetfillcolor{currentfill}%
\pgfsetfillopacity{0.800000}%
\pgfsetlinewidth{0.000000pt}%
\definecolor{currentstroke}{rgb}{0.000000,0.000000,0.000000}%
\pgfsetstrokecolor{currentstroke}%
\pgfsetdash{}{0pt}%
\pgfpathmoveto{\pgfqpoint{3.954290in}{2.889171in}}%
\pgfpathlineto{\pgfqpoint{3.967632in}{2.881318in}}%
\pgfpathlineto{\pgfqpoint{3.980976in}{2.873686in}}%
\pgfpathlineto{\pgfqpoint{3.994324in}{2.866273in}}%
\pgfpathlineto{\pgfqpoint{4.007676in}{2.859079in}}%
\pgfpathlineto{\pgfqpoint{4.015407in}{2.873460in}}%
\pgfpathlineto{\pgfqpoint{4.023133in}{2.888006in}}%
\pgfpathlineto{\pgfqpoint{4.030856in}{2.902721in}}%
\pgfpathlineto{\pgfqpoint{4.038576in}{2.917610in}}%
\pgfpathlineto{\pgfqpoint{4.025231in}{2.925211in}}%
\pgfpathlineto{\pgfqpoint{4.011889in}{2.933031in}}%
\pgfpathlineto{\pgfqpoint{3.998552in}{2.941070in}}%
\pgfpathlineto{\pgfqpoint{3.985217in}{2.949331in}}%
\pgfpathlineto{\pgfqpoint{3.977491in}{2.934023in}}%
\pgfpathlineto{\pgfqpoint{3.969761in}{2.918897in}}%
\pgfpathlineto{\pgfqpoint{3.962028in}{2.903948in}}%
\pgfpathlineto{\pgfqpoint{3.954290in}{2.889171in}}%
\pgfpathclose%
\pgfusepath{fill}%
\end{pgfscope}%
\begin{pgfscope}%
\pgfpathrectangle{\pgfqpoint{1.150000in}{0.150000in}}{\pgfqpoint{5.700000in}{5.700000in}}%
\pgfusepath{clip}%
\pgfsetbuttcap%
\pgfsetroundjoin%
\definecolor{currentfill}{rgb}{0.243113,0.292092,0.538516}%
\pgfsetfillcolor{currentfill}%
\pgfsetfillopacity{0.800000}%
\pgfsetlinewidth{0.000000pt}%
\definecolor{currentstroke}{rgb}{0.000000,0.000000,0.000000}%
\pgfsetstrokecolor{currentstroke}%
\pgfsetdash{}{0pt}%
\pgfpathmoveto{\pgfqpoint{4.260520in}{2.958151in}}%
\pgfpathlineto{\pgfqpoint{4.273911in}{2.952387in}}%
\pgfpathlineto{\pgfqpoint{4.287308in}{2.946828in}}%
\pgfpathlineto{\pgfqpoint{4.300711in}{2.941472in}}%
\pgfpathlineto{\pgfqpoint{4.314120in}{2.936318in}}%
\pgfpathlineto{\pgfqpoint{4.321781in}{2.951053in}}%
\pgfpathlineto{\pgfqpoint{4.329440in}{2.965985in}}%
\pgfpathlineto{\pgfqpoint{4.337097in}{2.981118in}}%
\pgfpathlineto{\pgfqpoint{4.344751in}{2.996460in}}%
\pgfpathlineto{\pgfqpoint{4.331351in}{3.002113in}}%
\pgfpathlineto{\pgfqpoint{4.317956in}{3.007970in}}%
\pgfpathlineto{\pgfqpoint{4.304567in}{3.014029in}}%
\pgfpathlineto{\pgfqpoint{4.291184in}{3.020293in}}%
\pgfpathlineto{\pgfqpoint{4.283521in}{3.004440in}}%
\pgfpathlineto{\pgfqpoint{4.275857in}{2.988802in}}%
\pgfpathlineto{\pgfqpoint{4.268190in}{2.973375in}}%
\pgfpathlineto{\pgfqpoint{4.260520in}{2.958151in}}%
\pgfpathclose%
\pgfusepath{fill}%
\end{pgfscope}%
\begin{pgfscope}%
\pgfpathrectangle{\pgfqpoint{1.150000in}{0.150000in}}{\pgfqpoint{5.700000in}{5.700000in}}%
\pgfusepath{clip}%
\pgfsetbuttcap%
\pgfsetroundjoin%
\definecolor{currentfill}{rgb}{0.194100,0.399323,0.555565}%
\pgfsetfillcolor{currentfill}%
\pgfsetfillopacity{0.800000}%
\pgfsetlinewidth{0.000000pt}%
\definecolor{currentstroke}{rgb}{0.000000,0.000000,0.000000}%
\pgfsetstrokecolor{currentstroke}%
\pgfsetdash{}{0pt}%
\pgfpathmoveto{\pgfqpoint{3.273798in}{3.264360in}}%
\pgfpathlineto{\pgfqpoint{3.287198in}{3.246272in}}%
\pgfpathlineto{\pgfqpoint{3.300593in}{3.228486in}}%
\pgfpathlineto{\pgfqpoint{3.313982in}{3.210997in}}%
\pgfpathlineto{\pgfqpoint{3.327366in}{3.193803in}}%
\pgfpathlineto{\pgfqpoint{3.335226in}{3.209896in}}%
\pgfpathlineto{\pgfqpoint{3.343080in}{3.226189in}}%
\pgfpathlineto{\pgfqpoint{3.350928in}{3.242687in}}%
\pgfpathlineto{\pgfqpoint{3.358771in}{3.259393in}}%
\pgfpathlineto{\pgfqpoint{3.345394in}{3.276907in}}%
\pgfpathlineto{\pgfqpoint{3.332011in}{3.294717in}}%
\pgfpathlineto{\pgfqpoint{3.318623in}{3.312825in}}%
\pgfpathlineto{\pgfqpoint{3.305230in}{3.331234in}}%
\pgfpathlineto{\pgfqpoint{3.297381in}{3.314195in}}%
\pgfpathlineto{\pgfqpoint{3.289526in}{3.297372in}}%
\pgfpathlineto{\pgfqpoint{3.281665in}{3.280761in}}%
\pgfpathlineto{\pgfqpoint{3.273798in}{3.264360in}}%
\pgfpathclose%
\pgfusepath{fill}%
\end{pgfscope}%
\begin{pgfscope}%
\pgfpathrectangle{\pgfqpoint{1.150000in}{0.150000in}}{\pgfqpoint{5.700000in}{5.700000in}}%
\pgfusepath{clip}%
\pgfsetbuttcap%
\pgfsetroundjoin%
\definecolor{currentfill}{rgb}{0.195860,0.395433,0.555276}%
\pgfsetfillcolor{currentfill}%
\pgfsetfillopacity{0.800000}%
\pgfsetlinewidth{0.000000pt}%
\definecolor{currentstroke}{rgb}{0.000000,0.000000,0.000000}%
\pgfsetstrokecolor{currentstroke}%
\pgfsetdash{}{0pt}%
\pgfpathmoveto{\pgfqpoint{4.766070in}{3.225769in}}%
\pgfpathlineto{\pgfqpoint{4.779567in}{3.221224in}}%
\pgfpathlineto{\pgfqpoint{4.793073in}{3.216865in}}%
\pgfpathlineto{\pgfqpoint{4.806587in}{3.212692in}}%
\pgfpathlineto{\pgfqpoint{4.820109in}{3.208706in}}%
\pgfpathlineto{\pgfqpoint{4.827691in}{3.225886in}}%
\pgfpathlineto{\pgfqpoint{4.835276in}{3.243380in}}%
\pgfpathlineto{\pgfqpoint{4.842863in}{3.261196in}}%
\pgfpathlineto{\pgfqpoint{4.850452in}{3.279343in}}%
\pgfpathlineto{\pgfqpoint{4.836943in}{3.284018in}}%
\pgfpathlineto{\pgfqpoint{4.823441in}{3.288879in}}%
\pgfpathlineto{\pgfqpoint{4.809948in}{3.293927in}}%
\pgfpathlineto{\pgfqpoint{4.796462in}{3.299161in}}%
\pgfpathlineto{\pgfqpoint{4.788861in}{3.280314in}}%
\pgfpathlineto{\pgfqpoint{4.781262in}{3.261805in}}%
\pgfpathlineto{\pgfqpoint{4.773665in}{3.243626in}}%
\pgfpathlineto{\pgfqpoint{4.766070in}{3.225769in}}%
\pgfpathclose%
\pgfusepath{fill}%
\end{pgfscope}%
\begin{pgfscope}%
\pgfpathrectangle{\pgfqpoint{1.150000in}{0.150000in}}{\pgfqpoint{5.700000in}{5.700000in}}%
\pgfusepath{clip}%
\pgfsetbuttcap%
\pgfsetroundjoin%
\definecolor{currentfill}{rgb}{0.248629,0.278775,0.534556}%
\pgfsetfillcolor{currentfill}%
\pgfsetfillopacity{0.800000}%
\pgfsetlinewidth{0.000000pt}%
\definecolor{currentstroke}{rgb}{0.000000,0.000000,0.000000}%
\pgfsetstrokecolor{currentstroke}%
\pgfsetdash{}{0pt}%
\pgfpathmoveto{\pgfqpoint{4.176272in}{2.922427in}}%
\pgfpathlineto{\pgfqpoint{4.189650in}{2.916306in}}%
\pgfpathlineto{\pgfqpoint{4.203033in}{2.910392in}}%
\pgfpathlineto{\pgfqpoint{4.216421in}{2.904686in}}%
\pgfpathlineto{\pgfqpoint{4.229814in}{2.899186in}}%
\pgfpathlineto{\pgfqpoint{4.237495in}{2.913649in}}%
\pgfpathlineto{\pgfqpoint{4.245173in}{2.928294in}}%
\pgfpathlineto{\pgfqpoint{4.252848in}{2.943126in}}%
\pgfpathlineto{\pgfqpoint{4.260520in}{2.958151in}}%
\pgfpathlineto{\pgfqpoint{4.247134in}{2.964120in}}%
\pgfpathlineto{\pgfqpoint{4.233754in}{2.970296in}}%
\pgfpathlineto{\pgfqpoint{4.220379in}{2.976678in}}%
\pgfpathlineto{\pgfqpoint{4.207009in}{2.983269in}}%
\pgfpathlineto{\pgfqpoint{4.199329in}{2.967763in}}%
\pgfpathlineto{\pgfqpoint{4.191646in}{2.952458in}}%
\pgfpathlineto{\pgfqpoint{4.183961in}{2.937348in}}%
\pgfpathlineto{\pgfqpoint{4.176272in}{2.922427in}}%
\pgfpathclose%
\pgfusepath{fill}%
\end{pgfscope}%
\begin{pgfscope}%
\pgfpathrectangle{\pgfqpoint{1.150000in}{0.150000in}}{\pgfqpoint{5.700000in}{5.700000in}}%
\pgfusepath{clip}%
\pgfsetbuttcap%
\pgfsetroundjoin%
\definecolor{currentfill}{rgb}{0.244972,0.287675,0.537260}%
\pgfsetfillcolor{currentfill}%
\pgfsetfillopacity{0.800000}%
\pgfsetlinewidth{0.000000pt}%
\definecolor{currentstroke}{rgb}{0.000000,0.000000,0.000000}%
\pgfsetstrokecolor{currentstroke}%
\pgfsetdash{}{0pt}%
\pgfpathmoveto{\pgfqpoint{3.541007in}{2.956933in}}%
\pgfpathlineto{\pgfqpoint{3.554340in}{2.944407in}}%
\pgfpathlineto{\pgfqpoint{3.567672in}{2.932137in}}%
\pgfpathlineto{\pgfqpoint{3.581003in}{2.920122in}}%
\pgfpathlineto{\pgfqpoint{3.594333in}{2.908359in}}%
\pgfpathlineto{\pgfqpoint{3.602158in}{2.922924in}}%
\pgfpathlineto{\pgfqpoint{3.609977in}{2.937651in}}%
\pgfpathlineto{\pgfqpoint{3.617792in}{2.952541in}}%
\pgfpathlineto{\pgfqpoint{3.625602in}{2.967599in}}%
\pgfpathlineto{\pgfqpoint{3.612278in}{2.979678in}}%
\pgfpathlineto{\pgfqpoint{3.598954in}{2.992009in}}%
\pgfpathlineto{\pgfqpoint{3.585629in}{3.004594in}}%
\pgfpathlineto{\pgfqpoint{3.572303in}{3.017436in}}%
\pgfpathlineto{\pgfqpoint{3.564487in}{3.002050in}}%
\pgfpathlineto{\pgfqpoint{3.556665in}{2.986840in}}%
\pgfpathlineto{\pgfqpoint{3.548839in}{2.971803in}}%
\pgfpathlineto{\pgfqpoint{3.541007in}{2.956933in}}%
\pgfpathclose%
\pgfusepath{fill}%
\end{pgfscope}%
\begin{pgfscope}%
\pgfpathrectangle{\pgfqpoint{1.150000in}{0.150000in}}{\pgfqpoint{5.700000in}{5.700000in}}%
\pgfusepath{clip}%
\pgfsetbuttcap%
\pgfsetroundjoin%
\definecolor{currentfill}{rgb}{0.182256,0.426184,0.557120}%
\pgfsetfillcolor{currentfill}%
\pgfsetfillopacity{0.800000}%
\pgfsetlinewidth{0.000000pt}%
\definecolor{currentstroke}{rgb}{0.000000,0.000000,0.000000}%
\pgfsetstrokecolor{currentstroke}%
\pgfsetdash{}{0pt}%
\pgfpathmoveto{\pgfqpoint{3.220135in}{3.339768in}}%
\pgfpathlineto{\pgfqpoint{3.233561in}{3.320452in}}%
\pgfpathlineto{\pgfqpoint{3.246980in}{3.301446in}}%
\pgfpathlineto{\pgfqpoint{3.260392in}{3.282750in}}%
\pgfpathlineto{\pgfqpoint{3.273798in}{3.264360in}}%
\pgfpathlineto{\pgfqpoint{3.281665in}{3.280761in}}%
\pgfpathlineto{\pgfqpoint{3.289526in}{3.297372in}}%
\pgfpathlineto{\pgfqpoint{3.297381in}{3.314195in}}%
\pgfpathlineto{\pgfqpoint{3.305230in}{3.331234in}}%
\pgfpathlineto{\pgfqpoint{3.291831in}{3.349947in}}%
\pgfpathlineto{\pgfqpoint{3.278425in}{3.368966in}}%
\pgfpathlineto{\pgfqpoint{3.265013in}{3.388294in}}%
\pgfpathlineto{\pgfqpoint{3.251595in}{3.407934in}}%
\pgfpathlineto{\pgfqpoint{3.243739in}{3.390559in}}%
\pgfpathlineto{\pgfqpoint{3.235878in}{3.373409in}}%
\pgfpathlineto{\pgfqpoint{3.228010in}{3.356480in}}%
\pgfpathlineto{\pgfqpoint{3.220135in}{3.339768in}}%
\pgfpathclose%
\pgfusepath{fill}%
\end{pgfscope}%
\begin{pgfscope}%
\pgfpathrectangle{\pgfqpoint{1.150000in}{0.150000in}}{\pgfqpoint{5.700000in}{5.700000in}}%
\pgfusepath{clip}%
\pgfsetbuttcap%
\pgfsetroundjoin%
\definecolor{currentfill}{rgb}{0.730889,0.871916,0.156029}%
\pgfsetfillcolor{currentfill}%
\pgfsetfillopacity{0.800000}%
\pgfsetlinewidth{0.000000pt}%
\definecolor{currentstroke}{rgb}{0.000000,0.000000,0.000000}%
\pgfsetstrokecolor{currentstroke}%
\pgfsetdash{}{0pt}%
\pgfpathmoveto{\pgfqpoint{3.514674in}{4.831098in}}%
\pgfpathlineto{\pgfqpoint{3.528169in}{4.802105in}}%
\pgfpathlineto{\pgfqpoint{3.541653in}{4.773464in}}%
\pgfpathlineto{\pgfqpoint{3.555128in}{4.745173in}}%
\pgfpathlineto{\pgfqpoint{3.568593in}{4.717227in}}%
\pgfpathlineto{\pgfqpoint{3.576213in}{4.753369in}}%
\pgfpathlineto{\pgfqpoint{3.583830in}{4.790079in}}%
\pgfpathlineto{\pgfqpoint{3.591445in}{4.827366in}}%
\pgfpathlineto{\pgfqpoint{3.599058in}{4.865240in}}%
\pgfpathlineto{\pgfqpoint{3.585583in}{4.894001in}}%
\pgfpathlineto{\pgfqpoint{3.572098in}{4.923110in}}%
\pgfpathlineto{\pgfqpoint{3.558603in}{4.952570in}}%
\pgfpathlineto{\pgfqpoint{3.545098in}{4.982386in}}%
\pgfpathlineto{\pgfqpoint{3.537496in}{4.943676in}}%
\pgfpathlineto{\pgfqpoint{3.529892in}{4.905565in}}%
\pgfpathlineto{\pgfqpoint{3.522284in}{4.868042in}}%
\pgfpathlineto{\pgfqpoint{3.514674in}{4.831098in}}%
\pgfpathclose%
\pgfusepath{fill}%
\end{pgfscope}%
\begin{pgfscope}%
\pgfpathrectangle{\pgfqpoint{1.150000in}{0.150000in}}{\pgfqpoint{5.700000in}{5.700000in}}%
\pgfusepath{clip}%
\pgfsetbuttcap%
\pgfsetroundjoin%
\definecolor{currentfill}{rgb}{0.185556,0.418570,0.556753}%
\pgfsetfillcolor{currentfill}%
\pgfsetfillopacity{0.800000}%
\pgfsetlinewidth{0.000000pt}%
\definecolor{currentstroke}{rgb}{0.000000,0.000000,0.000000}%
\pgfsetstrokecolor{currentstroke}%
\pgfsetdash{}{0pt}%
\pgfpathmoveto{\pgfqpoint{4.850452in}{3.279343in}}%
\pgfpathlineto{\pgfqpoint{4.863969in}{3.274852in}}%
\pgfpathlineto{\pgfqpoint{4.877495in}{3.270546in}}%
\pgfpathlineto{\pgfqpoint{4.891029in}{3.266424in}}%
\pgfpathlineto{\pgfqpoint{4.904572in}{3.262486in}}%
\pgfpathlineto{\pgfqpoint{4.912150in}{3.280263in}}%
\pgfpathlineto{\pgfqpoint{4.919731in}{3.298378in}}%
\pgfpathlineto{\pgfqpoint{4.927316in}{3.316841in}}%
\pgfpathlineto{\pgfqpoint{4.934904in}{3.335659in}}%
\pgfpathlineto{\pgfqpoint{4.921375in}{3.340318in}}%
\pgfpathlineto{\pgfqpoint{4.907854in}{3.345161in}}%
\pgfpathlineto{\pgfqpoint{4.894341in}{3.350187in}}%
\pgfpathlineto{\pgfqpoint{4.880837in}{3.355398in}}%
\pgfpathlineto{\pgfqpoint{4.873235in}{3.335847in}}%
\pgfpathlineto{\pgfqpoint{4.865638in}{3.316660in}}%
\pgfpathlineto{\pgfqpoint{4.858043in}{3.297828in}}%
\pgfpathlineto{\pgfqpoint{4.850452in}{3.279343in}}%
\pgfpathclose%
\pgfusepath{fill}%
\end{pgfscope}%
\begin{pgfscope}%
\pgfpathrectangle{\pgfqpoint{1.150000in}{0.150000in}}{\pgfqpoint{5.700000in}{5.700000in}}%
\pgfusepath{clip}%
\pgfsetbuttcap%
\pgfsetroundjoin%
\definecolor{currentfill}{rgb}{0.257322,0.256130,0.526563}%
\pgfsetfillcolor{currentfill}%
\pgfsetfillopacity{0.800000}%
\pgfsetlinewidth{0.000000pt}%
\definecolor{currentstroke}{rgb}{0.000000,0.000000,0.000000}%
\pgfsetstrokecolor{currentstroke}%
\pgfsetdash{}{0pt}%
\pgfpathmoveto{\pgfqpoint{3.732192in}{2.879864in}}%
\pgfpathlineto{\pgfqpoint{3.745519in}{2.869985in}}%
\pgfpathlineto{\pgfqpoint{3.758847in}{2.860341in}}%
\pgfpathlineto{\pgfqpoint{3.772177in}{2.850933in}}%
\pgfpathlineto{\pgfqpoint{3.785509in}{2.841759in}}%
\pgfpathlineto{\pgfqpoint{3.793293in}{2.855987in}}%
\pgfpathlineto{\pgfqpoint{3.801073in}{2.870367in}}%
\pgfpathlineto{\pgfqpoint{3.808849in}{2.884905in}}%
\pgfpathlineto{\pgfqpoint{3.816620in}{2.899603in}}%
\pgfpathlineto{\pgfqpoint{3.803295in}{2.909123in}}%
\pgfpathlineto{\pgfqpoint{3.789972in}{2.918876in}}%
\pgfpathlineto{\pgfqpoint{3.776650in}{2.928865in}}%
\pgfpathlineto{\pgfqpoint{3.763330in}{2.939090in}}%
\pgfpathlineto{\pgfqpoint{3.755552in}{2.924035in}}%
\pgfpathlineto{\pgfqpoint{3.747770in}{2.909148in}}%
\pgfpathlineto{\pgfqpoint{3.739983in}{2.894426in}}%
\pgfpathlineto{\pgfqpoint{3.732192in}{2.879864in}}%
\pgfpathclose%
\pgfusepath{fill}%
\end{pgfscope}%
\begin{pgfscope}%
\pgfpathrectangle{\pgfqpoint{1.150000in}{0.150000in}}{\pgfqpoint{5.700000in}{5.700000in}}%
\pgfusepath{clip}%
\pgfsetbuttcap%
\pgfsetroundjoin%
\definecolor{currentfill}{rgb}{0.128729,0.563265,0.551229}%
\pgfsetfillcolor{currentfill}%
\pgfsetfillopacity{0.800000}%
\pgfsetlinewidth{0.000000pt}%
\definecolor{currentstroke}{rgb}{0.000000,0.000000,0.000000}%
\pgfsetstrokecolor{currentstroke}%
\pgfsetdash{}{0pt}%
\pgfpathmoveto{\pgfqpoint{3.121376in}{3.745216in}}%
\pgfpathlineto{\pgfqpoint{3.134891in}{3.721226in}}%
\pgfpathlineto{\pgfqpoint{3.148396in}{3.697588in}}%
\pgfpathlineto{\pgfqpoint{3.161891in}{3.674297in}}%
\pgfpathlineto{\pgfqpoint{3.175377in}{3.651350in}}%
\pgfpathlineto{\pgfqpoint{3.183212in}{3.670674in}}%
\pgfpathlineto{\pgfqpoint{3.191040in}{3.690265in}}%
\pgfpathlineto{\pgfqpoint{3.198861in}{3.710129in}}%
\pgfpathlineto{\pgfqpoint{3.206676in}{3.730271in}}%
\pgfpathlineto{\pgfqpoint{3.193196in}{3.753614in}}%
\pgfpathlineto{\pgfqpoint{3.179705in}{3.777303in}}%
\pgfpathlineto{\pgfqpoint{3.166205in}{3.801340in}}%
\pgfpathlineto{\pgfqpoint{3.152695in}{3.825728in}}%
\pgfpathlineto{\pgfqpoint{3.144875in}{3.805176in}}%
\pgfpathlineto{\pgfqpoint{3.137049in}{3.784910in}}%
\pgfpathlineto{\pgfqpoint{3.129216in}{3.764924in}}%
\pgfpathlineto{\pgfqpoint{3.121376in}{3.745216in}}%
\pgfpathclose%
\pgfusepath{fill}%
\end{pgfscope}%
\begin{pgfscope}%
\pgfpathrectangle{\pgfqpoint{1.150000in}{0.150000in}}{\pgfqpoint{5.700000in}{5.700000in}}%
\pgfusepath{clip}%
\pgfsetbuttcap%
\pgfsetroundjoin%
\definecolor{currentfill}{rgb}{0.149039,0.508051,0.557250}%
\pgfsetfillcolor{currentfill}%
\pgfsetfillopacity{0.800000}%
\pgfsetlinewidth{0.000000pt}%
\definecolor{currentstroke}{rgb}{0.000000,0.000000,0.000000}%
\pgfsetstrokecolor{currentstroke}%
\pgfsetdash{}{0pt}%
\pgfpathmoveto{\pgfqpoint{3.143971in}{3.576644in}}%
\pgfpathlineto{\pgfqpoint{3.157453in}{3.554399in}}%
\pgfpathlineto{\pgfqpoint{3.170927in}{3.532491in}}%
\pgfpathlineto{\pgfqpoint{3.184391in}{3.510917in}}%
\pgfpathlineto{\pgfqpoint{3.197848in}{3.489673in}}%
\pgfpathlineto{\pgfqpoint{3.205703in}{3.507622in}}%
\pgfpathlineto{\pgfqpoint{3.213552in}{3.525813in}}%
\pgfpathlineto{\pgfqpoint{3.221394in}{3.544250in}}%
\pgfpathlineto{\pgfqpoint{3.229230in}{3.562936in}}%
\pgfpathlineto{\pgfqpoint{3.215780in}{3.584540in}}%
\pgfpathlineto{\pgfqpoint{3.202321in}{3.606475in}}%
\pgfpathlineto{\pgfqpoint{3.188853in}{3.628744in}}%
\pgfpathlineto{\pgfqpoint{3.175377in}{3.651350in}}%
\pgfpathlineto{\pgfqpoint{3.167536in}{3.632290in}}%
\pgfpathlineto{\pgfqpoint{3.159688in}{3.613488in}}%
\pgfpathlineto{\pgfqpoint{3.151833in}{3.594941in}}%
\pgfpathlineto{\pgfqpoint{3.143971in}{3.576644in}}%
\pgfpathclose%
\pgfusepath{fill}%
\end{pgfscope}%
\begin{pgfscope}%
\pgfpathrectangle{\pgfqpoint{1.150000in}{0.150000in}}{\pgfqpoint{5.700000in}{5.700000in}}%
\pgfusepath{clip}%
\pgfsetbuttcap%
\pgfsetroundjoin%
\definecolor{currentfill}{rgb}{0.258965,0.251537,0.524736}%
\pgfsetfillcolor{currentfill}%
\pgfsetfillopacity{0.800000}%
\pgfsetlinewidth{0.000000pt}%
\definecolor{currentstroke}{rgb}{0.000000,0.000000,0.000000}%
\pgfsetstrokecolor{currentstroke}%
\pgfsetdash{}{0pt}%
\pgfpathmoveto{\pgfqpoint{3.869940in}{2.863832in}}%
\pgfpathlineto{\pgfqpoint{3.883276in}{2.855460in}}%
\pgfpathlineto{\pgfqpoint{3.896615in}{2.847314in}}%
\pgfpathlineto{\pgfqpoint{3.909957in}{2.839392in}}%
\pgfpathlineto{\pgfqpoint{3.923301in}{2.831693in}}%
\pgfpathlineto{\pgfqpoint{3.931055in}{2.845827in}}%
\pgfpathlineto{\pgfqpoint{3.938804in}{2.860115in}}%
\pgfpathlineto{\pgfqpoint{3.946549in}{2.874562in}}%
\pgfpathlineto{\pgfqpoint{3.954290in}{2.889171in}}%
\pgfpathlineto{\pgfqpoint{3.940952in}{2.897246in}}%
\pgfpathlineto{\pgfqpoint{3.927618in}{2.905544in}}%
\pgfpathlineto{\pgfqpoint{3.914286in}{2.914067in}}%
\pgfpathlineto{\pgfqpoint{3.900956in}{2.922815in}}%
\pgfpathlineto{\pgfqpoint{3.893208in}{2.907818in}}%
\pgfpathlineto{\pgfqpoint{3.885456in}{2.892991in}}%
\pgfpathlineto{\pgfqpoint{3.877700in}{2.878331in}}%
\pgfpathlineto{\pgfqpoint{3.869940in}{2.863832in}}%
\pgfpathclose%
\pgfusepath{fill}%
\end{pgfscope}%
\begin{pgfscope}%
\pgfpathrectangle{\pgfqpoint{1.150000in}{0.150000in}}{\pgfqpoint{5.700000in}{5.700000in}}%
\pgfusepath{clip}%
\pgfsetbuttcap%
\pgfsetroundjoin%
\definecolor{currentfill}{rgb}{0.253935,0.265254,0.529983}%
\pgfsetfillcolor{currentfill}%
\pgfsetfillopacity{0.800000}%
\pgfsetlinewidth{0.000000pt}%
\definecolor{currentstroke}{rgb}{0.000000,0.000000,0.000000}%
\pgfsetstrokecolor{currentstroke}%
\pgfsetdash{}{0pt}%
\pgfpathmoveto{\pgfqpoint{4.091995in}{2.889368in}}%
\pgfpathlineto{\pgfqpoint{4.105361in}{2.882842in}}%
\pgfpathlineto{\pgfqpoint{4.118731in}{2.876529in}}%
\pgfpathlineto{\pgfqpoint{4.132106in}{2.870426in}}%
\pgfpathlineto{\pgfqpoint{4.145486in}{2.864534in}}%
\pgfpathlineto{\pgfqpoint{4.153188in}{2.878749in}}%
\pgfpathlineto{\pgfqpoint{4.160886in}{2.893133in}}%
\pgfpathlineto{\pgfqpoint{4.168581in}{2.907691in}}%
\pgfpathlineto{\pgfqpoint{4.176272in}{2.922427in}}%
\pgfpathlineto{\pgfqpoint{4.162900in}{2.928758in}}%
\pgfpathlineto{\pgfqpoint{4.149532in}{2.935298in}}%
\pgfpathlineto{\pgfqpoint{4.136169in}{2.942049in}}%
\pgfpathlineto{\pgfqpoint{4.122810in}{2.949013in}}%
\pgfpathlineto{\pgfqpoint{4.115111in}{2.933827in}}%
\pgfpathlineto{\pgfqpoint{4.107409in}{2.918827in}}%
\pgfpathlineto{\pgfqpoint{4.099704in}{2.904009in}}%
\pgfpathlineto{\pgfqpoint{4.091995in}{2.889368in}}%
\pgfpathclose%
\pgfusepath{fill}%
\end{pgfscope}%
\begin{pgfscope}%
\pgfpathrectangle{\pgfqpoint{1.150000in}{0.150000in}}{\pgfqpoint{5.700000in}{5.700000in}}%
\pgfusepath{clip}%
\pgfsetbuttcap%
\pgfsetroundjoin%
\definecolor{currentfill}{rgb}{0.252194,0.269783,0.531579}%
\pgfsetfillcolor{currentfill}%
\pgfsetfillopacity{0.800000}%
\pgfsetlinewidth{0.000000pt}%
\definecolor{currentstroke}{rgb}{0.000000,0.000000,0.000000}%
\pgfsetstrokecolor{currentstroke}%
\pgfsetdash{}{0pt}%
\pgfpathmoveto{\pgfqpoint{3.594333in}{2.908359in}}%
\pgfpathlineto{\pgfqpoint{3.607663in}{2.896846in}}%
\pgfpathlineto{\pgfqpoint{3.620993in}{2.885583in}}%
\pgfpathlineto{\pgfqpoint{3.634323in}{2.874567in}}%
\pgfpathlineto{\pgfqpoint{3.647653in}{2.863797in}}%
\pgfpathlineto{\pgfqpoint{3.655470in}{2.878060in}}%
\pgfpathlineto{\pgfqpoint{3.663283in}{2.892475in}}%
\pgfpathlineto{\pgfqpoint{3.671091in}{2.907046in}}%
\pgfpathlineto{\pgfqpoint{3.678894in}{2.921778in}}%
\pgfpathlineto{\pgfqpoint{3.665571in}{2.932863in}}%
\pgfpathlineto{\pgfqpoint{3.652248in}{2.944194in}}%
\pgfpathlineto{\pgfqpoint{3.638925in}{2.955772in}}%
\pgfpathlineto{\pgfqpoint{3.625602in}{2.967599in}}%
\pgfpathlineto{\pgfqpoint{3.617792in}{2.952541in}}%
\pgfpathlineto{\pgfqpoint{3.609977in}{2.937651in}}%
\pgfpathlineto{\pgfqpoint{3.602158in}{2.922924in}}%
\pgfpathlineto{\pgfqpoint{3.594333in}{2.908359in}}%
\pgfpathclose%
\pgfusepath{fill}%
\end{pgfscope}%
\begin{pgfscope}%
\pgfpathrectangle{\pgfqpoint{1.150000in}{0.150000in}}{\pgfqpoint{5.700000in}{5.700000in}}%
\pgfusepath{clip}%
\pgfsetbuttcap%
\pgfsetroundjoin%
\definecolor{currentfill}{rgb}{0.709898,0.868751,0.169257}%
\pgfsetfillcolor{currentfill}%
\pgfsetfillopacity{0.800000}%
\pgfsetlinewidth{0.000000pt}%
\definecolor{currentstroke}{rgb}{0.000000,0.000000,0.000000}%
\pgfsetstrokecolor{currentstroke}%
\pgfsetdash{}{0pt}%
\pgfpathmoveto{\pgfqpoint{3.430159in}{4.805347in}}%
\pgfpathlineto{\pgfqpoint{3.443686in}{4.775693in}}%
\pgfpathlineto{\pgfqpoint{3.457202in}{4.746404in}}%
\pgfpathlineto{\pgfqpoint{3.470708in}{4.717477in}}%
\pgfpathlineto{\pgfqpoint{3.484204in}{4.688908in}}%
\pgfpathlineto{\pgfqpoint{3.491826in}{4.723636in}}%
\pgfpathlineto{\pgfqpoint{3.499445in}{4.758904in}}%
\pgfpathlineto{\pgfqpoint{3.507061in}{4.794722in}}%
\pgfpathlineto{\pgfqpoint{3.514674in}{4.831098in}}%
\pgfpathlineto{\pgfqpoint{3.501170in}{4.860447in}}%
\pgfpathlineto{\pgfqpoint{3.487655in}{4.890157in}}%
\pgfpathlineto{\pgfqpoint{3.474129in}{4.920231in}}%
\pgfpathlineto{\pgfqpoint{3.460592in}{4.950672in}}%
\pgfpathlineto{\pgfqpoint{3.452989in}{4.913495in}}%
\pgfpathlineto{\pgfqpoint{3.445383in}{4.876889in}}%
\pgfpathlineto{\pgfqpoint{3.437772in}{4.840843in}}%
\pgfpathlineto{\pgfqpoint{3.430159in}{4.805347in}}%
\pgfpathclose%
\pgfusepath{fill}%
\end{pgfscope}%
\begin{pgfscope}%
\pgfpathrectangle{\pgfqpoint{1.150000in}{0.150000in}}{\pgfqpoint{5.700000in}{5.700000in}}%
\pgfusepath{clip}%
\pgfsetbuttcap%
\pgfsetroundjoin%
\definecolor{currentfill}{rgb}{0.169646,0.456262,0.558030}%
\pgfsetfillcolor{currentfill}%
\pgfsetfillopacity{0.800000}%
\pgfsetlinewidth{0.000000pt}%
\definecolor{currentstroke}{rgb}{0.000000,0.000000,0.000000}%
\pgfsetstrokecolor{currentstroke}%
\pgfsetdash{}{0pt}%
\pgfpathmoveto{\pgfqpoint{3.166360in}{3.420208in}}%
\pgfpathlineto{\pgfqpoint{3.179815in}{3.399616in}}%
\pgfpathlineto{\pgfqpoint{3.193263in}{3.379347in}}%
\pgfpathlineto{\pgfqpoint{3.206703in}{3.359399in}}%
\pgfpathlineto{\pgfqpoint{3.220135in}{3.339768in}}%
\pgfpathlineto{\pgfqpoint{3.228010in}{3.356480in}}%
\pgfpathlineto{\pgfqpoint{3.235878in}{3.373409in}}%
\pgfpathlineto{\pgfqpoint{3.243739in}{3.390559in}}%
\pgfpathlineto{\pgfqpoint{3.251595in}{3.407934in}}%
\pgfpathlineto{\pgfqpoint{3.238169in}{3.427889in}}%
\pgfpathlineto{\pgfqpoint{3.224736in}{3.448162in}}%
\pgfpathlineto{\pgfqpoint{3.211296in}{3.468755in}}%
\pgfpathlineto{\pgfqpoint{3.197848in}{3.489673in}}%
\pgfpathlineto{\pgfqpoint{3.189986in}{3.471960in}}%
\pgfpathlineto{\pgfqpoint{3.182117in}{3.454482in}}%
\pgfpathlineto{\pgfqpoint{3.174242in}{3.437232in}}%
\pgfpathlineto{\pgfqpoint{3.166360in}{3.420208in}}%
\pgfpathclose%
\pgfusepath{fill}%
\end{pgfscope}%
\begin{pgfscope}%
\pgfpathrectangle{\pgfqpoint{1.150000in}{0.150000in}}{\pgfqpoint{5.700000in}{5.700000in}}%
\pgfusepath{clip}%
\pgfsetbuttcap%
\pgfsetroundjoin%
\definecolor{currentfill}{rgb}{0.177423,0.437527,0.557565}%
\pgfsetfillcolor{currentfill}%
\pgfsetfillopacity{0.800000}%
\pgfsetlinewidth{0.000000pt}%
\definecolor{currentstroke}{rgb}{0.000000,0.000000,0.000000}%
\pgfsetstrokecolor{currentstroke}%
\pgfsetdash{}{0pt}%
\pgfpathmoveto{\pgfqpoint{4.934904in}{3.335659in}}%
\pgfpathlineto{\pgfqpoint{4.948441in}{3.331183in}}%
\pgfpathlineto{\pgfqpoint{4.961987in}{3.326889in}}%
\pgfpathlineto{\pgfqpoint{4.975542in}{3.322778in}}%
\pgfpathlineto{\pgfqpoint{4.989106in}{3.318848in}}%
\pgfpathlineto{\pgfqpoint{4.996683in}{3.337290in}}%
\pgfpathlineto{\pgfqpoint{5.004265in}{3.356098in}}%
\pgfpathlineto{\pgfqpoint{5.011850in}{3.375279in}}%
\pgfpathlineto{\pgfqpoint{4.998298in}{3.379771in}}%
\pgfpathlineto{\pgfqpoint{4.984754in}{3.384444in}}%
\pgfpathlineto{\pgfqpoint{4.971218in}{3.389299in}}%
\pgfpathlineto{\pgfqpoint{4.957691in}{3.394337in}}%
\pgfpathlineto{\pgfqpoint{4.950091in}{3.374398in}}%
\pgfpathlineto{\pgfqpoint{4.942495in}{3.354842in}}%
\pgfpathlineto{\pgfqpoint{4.934904in}{3.335659in}}%
\pgfpathclose%
\pgfusepath{fill}%
\end{pgfscope}%
\begin{pgfscope}%
\pgfpathrectangle{\pgfqpoint{1.150000in}{0.150000in}}{\pgfqpoint{5.700000in}{5.700000in}}%
\pgfusepath{clip}%
\pgfsetbuttcap%
\pgfsetroundjoin%
\definecolor{currentfill}{rgb}{0.257322,0.256130,0.526563}%
\pgfsetfillcolor{currentfill}%
\pgfsetfillopacity{0.800000}%
\pgfsetlinewidth{0.000000pt}%
\definecolor{currentstroke}{rgb}{0.000000,0.000000,0.000000}%
\pgfsetstrokecolor{currentstroke}%
\pgfsetdash{}{0pt}%
\pgfpathmoveto{\pgfqpoint{4.007676in}{2.859079in}}%
\pgfpathlineto{\pgfqpoint{4.021032in}{2.852102in}}%
\pgfpathlineto{\pgfqpoint{4.034392in}{2.845341in}}%
\pgfpathlineto{\pgfqpoint{4.047756in}{2.838795in}}%
\pgfpathlineto{\pgfqpoint{4.061124in}{2.832464in}}%
\pgfpathlineto{\pgfqpoint{4.068848in}{2.846450in}}%
\pgfpathlineto{\pgfqpoint{4.076567in}{2.860593in}}%
\pgfpathlineto{\pgfqpoint{4.084283in}{2.874897in}}%
\pgfpathlineto{\pgfqpoint{4.091995in}{2.889368in}}%
\pgfpathlineto{\pgfqpoint{4.078634in}{2.896106in}}%
\pgfpathlineto{\pgfqpoint{4.065277in}{2.903059in}}%
\pgfpathlineto{\pgfqpoint{4.051924in}{2.910226in}}%
\pgfpathlineto{\pgfqpoint{4.038576in}{2.917610in}}%
\pgfpathlineto{\pgfqpoint{4.030856in}{2.902721in}}%
\pgfpathlineto{\pgfqpoint{4.023133in}{2.888006in}}%
\pgfpathlineto{\pgfqpoint{4.015407in}{2.873460in}}%
\pgfpathlineto{\pgfqpoint{4.007676in}{2.859079in}}%
\pgfpathclose%
\pgfusepath{fill}%
\end{pgfscope}%
\begin{pgfscope}%
\pgfpathrectangle{\pgfqpoint{1.150000in}{0.150000in}}{\pgfqpoint{5.700000in}{5.700000in}}%
\pgfusepath{clip}%
\pgfsetbuttcap%
\pgfsetroundjoin%
\definecolor{currentfill}{rgb}{0.229739,0.322361,0.545706}%
\pgfsetfillcolor{currentfill}%
\pgfsetfillopacity{0.800000}%
\pgfsetlinewidth{0.000000pt}%
\definecolor{currentstroke}{rgb}{0.000000,0.000000,0.000000}%
\pgfsetstrokecolor{currentstroke}%
\pgfsetdash{}{0pt}%
\pgfpathmoveto{\pgfqpoint{4.482709in}{3.017745in}}%
\pgfpathlineto{\pgfqpoint{4.496159in}{3.013347in}}%
\pgfpathlineto{\pgfqpoint{4.509616in}{3.009144in}}%
\pgfpathlineto{\pgfqpoint{4.523081in}{3.005135in}}%
\pgfpathlineto{\pgfqpoint{4.536553in}{3.001319in}}%
\pgfpathlineto{\pgfqpoint{4.544170in}{3.016140in}}%
\pgfpathlineto{\pgfqpoint{4.551786in}{3.031183in}}%
\pgfpathlineto{\pgfqpoint{4.559401in}{3.046455in}}%
\pgfpathlineto{\pgfqpoint{4.567014in}{3.061962in}}%
\pgfpathlineto{\pgfqpoint{4.553553in}{3.066341in}}%
\pgfpathlineto{\pgfqpoint{4.540098in}{3.070913in}}%
\pgfpathlineto{\pgfqpoint{4.526651in}{3.075679in}}%
\pgfpathlineto{\pgfqpoint{4.513210in}{3.080639in}}%
\pgfpathlineto{\pgfqpoint{4.505587in}{3.064558in}}%
\pgfpathlineto{\pgfqpoint{4.497962in}{3.048719in}}%
\pgfpathlineto{\pgfqpoint{4.490336in}{3.033117in}}%
\pgfpathlineto{\pgfqpoint{4.482709in}{3.017745in}}%
\pgfpathclose%
\pgfusepath{fill}%
\end{pgfscope}%
\begin{pgfscope}%
\pgfpathrectangle{\pgfqpoint{1.150000in}{0.150000in}}{\pgfqpoint{5.700000in}{5.700000in}}%
\pgfusepath{clip}%
\pgfsetbuttcap%
\pgfsetroundjoin%
\definecolor{currentfill}{rgb}{0.221989,0.339161,0.548752}%
\pgfsetfillcolor{currentfill}%
\pgfsetfillopacity{0.800000}%
\pgfsetlinewidth{0.000000pt}%
\definecolor{currentstroke}{rgb}{0.000000,0.000000,0.000000}%
\pgfsetstrokecolor{currentstroke}%
\pgfsetdash{}{0pt}%
\pgfpathmoveto{\pgfqpoint{4.567014in}{3.061962in}}%
\pgfpathlineto{\pgfqpoint{4.580484in}{3.057776in}}%
\pgfpathlineto{\pgfqpoint{4.593960in}{3.053782in}}%
\pgfpathlineto{\pgfqpoint{4.607445in}{3.049979in}}%
\pgfpathlineto{\pgfqpoint{4.620937in}{3.046368in}}%
\pgfpathlineto{\pgfqpoint{4.628539in}{3.061534in}}%
\pgfpathlineto{\pgfqpoint{4.636141in}{3.076941in}}%
\pgfpathlineto{\pgfqpoint{4.643741in}{3.092597in}}%
\pgfpathlineto{\pgfqpoint{4.651342in}{3.108507in}}%
\pgfpathlineto{\pgfqpoint{4.637861in}{3.112713in}}%
\pgfpathlineto{\pgfqpoint{4.624387in}{3.117110in}}%
\pgfpathlineto{\pgfqpoint{4.610921in}{3.121698in}}%
\pgfpathlineto{\pgfqpoint{4.597462in}{3.126479in}}%
\pgfpathlineto{\pgfqpoint{4.589851in}{3.109962in}}%
\pgfpathlineto{\pgfqpoint{4.582239in}{3.093709in}}%
\pgfpathlineto{\pgfqpoint{4.574627in}{3.077711in}}%
\pgfpathlineto{\pgfqpoint{4.567014in}{3.061962in}}%
\pgfpathclose%
\pgfusepath{fill}%
\end{pgfscope}%
\begin{pgfscope}%
\pgfpathrectangle{\pgfqpoint{1.150000in}{0.150000in}}{\pgfqpoint{5.700000in}{5.700000in}}%
\pgfusepath{clip}%
\pgfsetbuttcap%
\pgfsetroundjoin%
\definecolor{currentfill}{rgb}{0.225863,0.330805,0.547314}%
\pgfsetfillcolor{currentfill}%
\pgfsetfillopacity{0.800000}%
\pgfsetlinewidth{0.000000pt}%
\definecolor{currentstroke}{rgb}{0.000000,0.000000,0.000000}%
\pgfsetstrokecolor{currentstroke}%
\pgfsetdash{}{0pt}%
\pgfpathmoveto{\pgfqpoint{3.349386in}{3.066641in}}%
\pgfpathlineto{\pgfqpoint{3.362755in}{3.051173in}}%
\pgfpathlineto{\pgfqpoint{3.376121in}{3.035986in}}%
\pgfpathlineto{\pgfqpoint{3.389484in}{3.021077in}}%
\pgfpathlineto{\pgfqpoint{3.402843in}{3.006444in}}%
\pgfpathlineto{\pgfqpoint{3.410711in}{3.021223in}}%
\pgfpathlineto{\pgfqpoint{3.418574in}{3.036171in}}%
\pgfpathlineto{\pgfqpoint{3.426431in}{3.051292in}}%
\pgfpathlineto{\pgfqpoint{3.434283in}{3.066591in}}%
\pgfpathlineto{\pgfqpoint{3.420931in}{3.081509in}}%
\pgfpathlineto{\pgfqpoint{3.407576in}{3.096704in}}%
\pgfpathlineto{\pgfqpoint{3.394218in}{3.112178in}}%
\pgfpathlineto{\pgfqpoint{3.380856in}{3.127932in}}%
\pgfpathlineto{\pgfqpoint{3.372997in}{3.112335in}}%
\pgfpathlineto{\pgfqpoint{3.365133in}{3.096923in}}%
\pgfpathlineto{\pgfqpoint{3.357262in}{3.081693in}}%
\pgfpathlineto{\pgfqpoint{3.349386in}{3.066641in}}%
\pgfpathclose%
\pgfusepath{fill}%
\end{pgfscope}%
\begin{pgfscope}%
\pgfpathrectangle{\pgfqpoint{1.150000in}{0.150000in}}{\pgfqpoint{5.700000in}{5.700000in}}%
\pgfusepath{clip}%
\pgfsetbuttcap%
\pgfsetroundjoin%
\definecolor{currentfill}{rgb}{0.237441,0.305202,0.541921}%
\pgfsetfillcolor{currentfill}%
\pgfsetfillopacity{0.800000}%
\pgfsetlinewidth{0.000000pt}%
\definecolor{currentstroke}{rgb}{0.000000,0.000000,0.000000}%
\pgfsetstrokecolor{currentstroke}%
\pgfsetdash{}{0pt}%
\pgfpathmoveto{\pgfqpoint{4.398414in}{2.975855in}}%
\pgfpathlineto{\pgfqpoint{4.411846in}{2.971202in}}%
\pgfpathlineto{\pgfqpoint{4.425285in}{2.966747in}}%
\pgfpathlineto{\pgfqpoint{4.438730in}{2.962489in}}%
\pgfpathlineto{\pgfqpoint{4.452183in}{2.958427in}}%
\pgfpathlineto{\pgfqpoint{4.459817in}{2.972943in}}%
\pgfpathlineto{\pgfqpoint{4.467449in}{2.987664in}}%
\pgfpathlineto{\pgfqpoint{4.475080in}{3.002596in}}%
\pgfpathlineto{\pgfqpoint{4.482709in}{3.017745in}}%
\pgfpathlineto{\pgfqpoint{4.469266in}{3.022338in}}%
\pgfpathlineto{\pgfqpoint{4.455830in}{3.027128in}}%
\pgfpathlineto{\pgfqpoint{4.442400in}{3.032114in}}%
\pgfpathlineto{\pgfqpoint{4.428977in}{3.037299in}}%
\pgfpathlineto{\pgfqpoint{4.421339in}{3.021607in}}%
\pgfpathlineto{\pgfqpoint{4.413699in}{3.006140in}}%
\pgfpathlineto{\pgfqpoint{4.406058in}{2.990891in}}%
\pgfpathlineto{\pgfqpoint{4.398414in}{2.975855in}}%
\pgfpathclose%
\pgfusepath{fill}%
\end{pgfscope}%
\begin{pgfscope}%
\pgfpathrectangle{\pgfqpoint{1.150000in}{0.150000in}}{\pgfqpoint{5.700000in}{5.700000in}}%
\pgfusepath{clip}%
\pgfsetbuttcap%
\pgfsetroundjoin%
\definecolor{currentfill}{rgb}{0.214298,0.355619,0.551184}%
\pgfsetfillcolor{currentfill}%
\pgfsetfillopacity{0.800000}%
\pgfsetlinewidth{0.000000pt}%
\definecolor{currentstroke}{rgb}{0.000000,0.000000,0.000000}%
\pgfsetstrokecolor{currentstroke}%
\pgfsetdash{}{0pt}%
\pgfpathmoveto{\pgfqpoint{3.295866in}{3.131363in}}%
\pgfpathlineto{\pgfqpoint{3.309252in}{3.114750in}}%
\pgfpathlineto{\pgfqpoint{3.322635in}{3.098427in}}%
\pgfpathlineto{\pgfqpoint{3.336012in}{3.082391in}}%
\pgfpathlineto{\pgfqpoint{3.349386in}{3.066641in}}%
\pgfpathlineto{\pgfqpoint{3.357262in}{3.081693in}}%
\pgfpathlineto{\pgfqpoint{3.365133in}{3.096923in}}%
\pgfpathlineto{\pgfqpoint{3.372997in}{3.112335in}}%
\pgfpathlineto{\pgfqpoint{3.380856in}{3.127932in}}%
\pgfpathlineto{\pgfqpoint{3.367490in}{3.143969in}}%
\pgfpathlineto{\pgfqpoint{3.354120in}{3.160292in}}%
\pgfpathlineto{\pgfqpoint{3.340745in}{3.176902in}}%
\pgfpathlineto{\pgfqpoint{3.327366in}{3.193803in}}%
\pgfpathlineto{\pgfqpoint{3.319500in}{3.177907in}}%
\pgfpathlineto{\pgfqpoint{3.311628in}{3.162204in}}%
\pgfpathlineto{\pgfqpoint{3.303750in}{3.146691in}}%
\pgfpathlineto{\pgfqpoint{3.295866in}{3.131363in}}%
\pgfpathclose%
\pgfusepath{fill}%
\end{pgfscope}%
\begin{pgfscope}%
\pgfpathrectangle{\pgfqpoint{1.150000in}{0.150000in}}{\pgfqpoint{5.700000in}{5.700000in}}%
\pgfusepath{clip}%
\pgfsetbuttcap%
\pgfsetroundjoin%
\definecolor{currentfill}{rgb}{0.214298,0.355619,0.551184}%
\pgfsetfillcolor{currentfill}%
\pgfsetfillopacity{0.800000}%
\pgfsetlinewidth{0.000000pt}%
\definecolor{currentstroke}{rgb}{0.000000,0.000000,0.000000}%
\pgfsetstrokecolor{currentstroke}%
\pgfsetdash{}{0pt}%
\pgfpathmoveto{\pgfqpoint{4.651342in}{3.108507in}}%
\pgfpathlineto{\pgfqpoint{4.664831in}{3.104491in}}%
\pgfpathlineto{\pgfqpoint{4.678328in}{3.100664in}}%
\pgfpathlineto{\pgfqpoint{4.691833in}{3.097025in}}%
\pgfpathlineto{\pgfqpoint{4.705346in}{3.093576in}}%
\pgfpathlineto{\pgfqpoint{4.712935in}{3.109134in}}%
\pgfpathlineto{\pgfqpoint{4.720524in}{3.124953in}}%
\pgfpathlineto{\pgfqpoint{4.728113in}{3.141041in}}%
\pgfpathlineto{\pgfqpoint{4.735703in}{3.157405in}}%
\pgfpathlineto{\pgfqpoint{4.722201in}{3.161480in}}%
\pgfpathlineto{\pgfqpoint{4.708708in}{3.165744in}}%
\pgfpathlineto{\pgfqpoint{4.695222in}{3.170197in}}%
\pgfpathlineto{\pgfqpoint{4.681744in}{3.174839in}}%
\pgfpathlineto{\pgfqpoint{4.674143in}{3.157838in}}%
\pgfpathlineto{\pgfqpoint{4.666543in}{3.141120in}}%
\pgfpathlineto{\pgfqpoint{4.658942in}{3.124679in}}%
\pgfpathlineto{\pgfqpoint{4.651342in}{3.108507in}}%
\pgfpathclose%
\pgfusepath{fill}%
\end{pgfscope}%
\begin{pgfscope}%
\pgfpathrectangle{\pgfqpoint{1.150000in}{0.150000in}}{\pgfqpoint{5.700000in}{5.700000in}}%
\pgfusepath{clip}%
\pgfsetbuttcap%
\pgfsetroundjoin%
\definecolor{currentfill}{rgb}{0.235526,0.309527,0.542944}%
\pgfsetfillcolor{currentfill}%
\pgfsetfillopacity{0.800000}%
\pgfsetlinewidth{0.000000pt}%
\definecolor{currentstroke}{rgb}{0.000000,0.000000,0.000000}%
\pgfsetstrokecolor{currentstroke}%
\pgfsetdash{}{0pt}%
\pgfpathmoveto{\pgfqpoint{3.402843in}{3.006444in}}%
\pgfpathlineto{\pgfqpoint{3.416199in}{2.992085in}}%
\pgfpathlineto{\pgfqpoint{3.429553in}{2.977998in}}%
\pgfpathlineto{\pgfqpoint{3.442904in}{2.964181in}}%
\pgfpathlineto{\pgfqpoint{3.456252in}{2.950631in}}%
\pgfpathlineto{\pgfqpoint{3.464113in}{2.965136in}}%
\pgfpathlineto{\pgfqpoint{3.471968in}{2.979802in}}%
\pgfpathlineto{\pgfqpoint{3.479817in}{2.994635in}}%
\pgfpathlineto{\pgfqpoint{3.487662in}{3.009636in}}%
\pgfpathlineto{\pgfqpoint{3.474321in}{3.023471in}}%
\pgfpathlineto{\pgfqpoint{3.460977in}{3.037574in}}%
\pgfpathlineto{\pgfqpoint{3.447631in}{3.051946in}}%
\pgfpathlineto{\pgfqpoint{3.434283in}{3.066591in}}%
\pgfpathlineto{\pgfqpoint{3.426431in}{3.051292in}}%
\pgfpathlineto{\pgfqpoint{3.418574in}{3.036171in}}%
\pgfpathlineto{\pgfqpoint{3.410711in}{3.021223in}}%
\pgfpathlineto{\pgfqpoint{3.402843in}{3.006444in}}%
\pgfpathclose%
\pgfusepath{fill}%
\end{pgfscope}%
\begin{pgfscope}%
\pgfpathrectangle{\pgfqpoint{1.150000in}{0.150000in}}{\pgfqpoint{5.700000in}{5.700000in}}%
\pgfusepath{clip}%
\pgfsetbuttcap%
\pgfsetroundjoin%
\definecolor{currentfill}{rgb}{0.260571,0.246922,0.522828}%
\pgfsetfillcolor{currentfill}%
\pgfsetfillopacity{0.800000}%
\pgfsetlinewidth{0.000000pt}%
\definecolor{currentstroke}{rgb}{0.000000,0.000000,0.000000}%
\pgfsetstrokecolor{currentstroke}%
\pgfsetdash{}{0pt}%
\pgfpathmoveto{\pgfqpoint{3.785509in}{2.841759in}}%
\pgfpathlineto{\pgfqpoint{3.798842in}{2.832816in}}%
\pgfpathlineto{\pgfqpoint{3.812178in}{2.824105in}}%
\pgfpathlineto{\pgfqpoint{3.825516in}{2.815623in}}%
\pgfpathlineto{\pgfqpoint{3.838856in}{2.807369in}}%
\pgfpathlineto{\pgfqpoint{3.846634in}{2.821263in}}%
\pgfpathlineto{\pgfqpoint{3.854407in}{2.835303in}}%
\pgfpathlineto{\pgfqpoint{3.862176in}{2.849491in}}%
\pgfpathlineto{\pgfqpoint{3.869940in}{2.863832in}}%
\pgfpathlineto{\pgfqpoint{3.856607in}{2.872431in}}%
\pgfpathlineto{\pgfqpoint{3.843276in}{2.881259in}}%
\pgfpathlineto{\pgfqpoint{3.829947in}{2.890315in}}%
\pgfpathlineto{\pgfqpoint{3.816620in}{2.899603in}}%
\pgfpathlineto{\pgfqpoint{3.808849in}{2.884905in}}%
\pgfpathlineto{\pgfqpoint{3.801073in}{2.870367in}}%
\pgfpathlineto{\pgfqpoint{3.793293in}{2.855987in}}%
\pgfpathlineto{\pgfqpoint{3.785509in}{2.841759in}}%
\pgfpathclose%
\pgfusepath{fill}%
\end{pgfscope}%
\begin{pgfscope}%
\pgfpathrectangle{\pgfqpoint{1.150000in}{0.150000in}}{\pgfqpoint{5.700000in}{5.700000in}}%
\pgfusepath{clip}%
\pgfsetbuttcap%
\pgfsetroundjoin%
\definecolor{currentfill}{rgb}{0.244972,0.287675,0.537260}%
\pgfsetfillcolor{currentfill}%
\pgfsetfillopacity{0.800000}%
\pgfsetlinewidth{0.000000pt}%
\definecolor{currentstroke}{rgb}{0.000000,0.000000,0.000000}%
\pgfsetstrokecolor{currentstroke}%
\pgfsetdash{}{0pt}%
\pgfpathmoveto{\pgfqpoint{4.314120in}{2.936318in}}%
\pgfpathlineto{\pgfqpoint{4.327535in}{2.931367in}}%
\pgfpathlineto{\pgfqpoint{4.340956in}{2.926616in}}%
\pgfpathlineto{\pgfqpoint{4.354383in}{2.922066in}}%
\pgfpathlineto{\pgfqpoint{4.367818in}{2.917715in}}%
\pgfpathlineto{\pgfqpoint{4.375470in}{2.931961in}}%
\pgfpathlineto{\pgfqpoint{4.383121in}{2.946396in}}%
\pgfpathlineto{\pgfqpoint{4.390769in}{2.961025in}}%
\pgfpathlineto{\pgfqpoint{4.398414in}{2.975855in}}%
\pgfpathlineto{\pgfqpoint{4.384989in}{2.980706in}}%
\pgfpathlineto{\pgfqpoint{4.371570in}{2.985757in}}%
\pgfpathlineto{\pgfqpoint{4.358158in}{2.991008in}}%
\pgfpathlineto{\pgfqpoint{4.344751in}{2.996460in}}%
\pgfpathlineto{\pgfqpoint{4.337097in}{2.981118in}}%
\pgfpathlineto{\pgfqpoint{4.329440in}{2.965985in}}%
\pgfpathlineto{\pgfqpoint{4.321781in}{2.951053in}}%
\pgfpathlineto{\pgfqpoint{4.314120in}{2.936318in}}%
\pgfpathclose%
\pgfusepath{fill}%
\end{pgfscope}%
\begin{pgfscope}%
\pgfpathrectangle{\pgfqpoint{1.150000in}{0.150000in}}{\pgfqpoint{5.700000in}{5.700000in}}%
\pgfusepath{clip}%
\pgfsetbuttcap%
\pgfsetroundjoin%
\definecolor{currentfill}{rgb}{0.204903,0.375746,0.553533}%
\pgfsetfillcolor{currentfill}%
\pgfsetfillopacity{0.800000}%
\pgfsetlinewidth{0.000000pt}%
\definecolor{currentstroke}{rgb}{0.000000,0.000000,0.000000}%
\pgfsetstrokecolor{currentstroke}%
\pgfsetdash{}{0pt}%
\pgfpathmoveto{\pgfqpoint{4.735703in}{3.157405in}}%
\pgfpathlineto{\pgfqpoint{4.749212in}{3.153516in}}%
\pgfpathlineto{\pgfqpoint{4.762730in}{3.149815in}}%
\pgfpathlineto{\pgfqpoint{4.776256in}{3.146300in}}%
\pgfpathlineto{\pgfqpoint{4.789791in}{3.142970in}}%
\pgfpathlineto{\pgfqpoint{4.797369in}{3.158972in}}%
\pgfpathlineto{\pgfqpoint{4.804948in}{3.175257in}}%
\pgfpathlineto{\pgfqpoint{4.812527in}{3.191832in}}%
\pgfpathlineto{\pgfqpoint{4.820109in}{3.208706in}}%
\pgfpathlineto{\pgfqpoint{4.806587in}{3.212692in}}%
\pgfpathlineto{\pgfqpoint{4.793073in}{3.216865in}}%
\pgfpathlineto{\pgfqpoint{4.779567in}{3.221224in}}%
\pgfpathlineto{\pgfqpoint{4.766070in}{3.225769in}}%
\pgfpathlineto{\pgfqpoint{4.758476in}{3.208226in}}%
\pgfpathlineto{\pgfqpoint{4.750884in}{3.190990in}}%
\pgfpathlineto{\pgfqpoint{4.743293in}{3.174052in}}%
\pgfpathlineto{\pgfqpoint{4.735703in}{3.157405in}}%
\pgfpathclose%
\pgfusepath{fill}%
\end{pgfscope}%
\begin{pgfscope}%
\pgfpathrectangle{\pgfqpoint{1.150000in}{0.150000in}}{\pgfqpoint{5.700000in}{5.700000in}}%
\pgfusepath{clip}%
\pgfsetbuttcap%
\pgfsetroundjoin%
\definecolor{currentfill}{rgb}{0.257322,0.256130,0.526563}%
\pgfsetfillcolor{currentfill}%
\pgfsetfillopacity{0.800000}%
\pgfsetlinewidth{0.000000pt}%
\definecolor{currentstroke}{rgb}{0.000000,0.000000,0.000000}%
\pgfsetstrokecolor{currentstroke}%
\pgfsetdash{}{0pt}%
\pgfpathmoveto{\pgfqpoint{3.647653in}{2.863797in}}%
\pgfpathlineto{\pgfqpoint{3.660983in}{2.853271in}}%
\pgfpathlineto{\pgfqpoint{3.674315in}{2.842987in}}%
\pgfpathlineto{\pgfqpoint{3.687646in}{2.832945in}}%
\pgfpathlineto{\pgfqpoint{3.700979in}{2.823141in}}%
\pgfpathlineto{\pgfqpoint{3.708790in}{2.837101in}}%
\pgfpathlineto{\pgfqpoint{3.716595in}{2.851206in}}%
\pgfpathlineto{\pgfqpoint{3.724396in}{2.865459in}}%
\pgfpathlineto{\pgfqpoint{3.732192in}{2.879864in}}%
\pgfpathlineto{\pgfqpoint{3.718866in}{2.889982in}}%
\pgfpathlineto{\pgfqpoint{3.705541in}{2.900339in}}%
\pgfpathlineto{\pgfqpoint{3.692217in}{2.910937in}}%
\pgfpathlineto{\pgfqpoint{3.678894in}{2.921778in}}%
\pgfpathlineto{\pgfqpoint{3.671091in}{2.907046in}}%
\pgfpathlineto{\pgfqpoint{3.663283in}{2.892475in}}%
\pgfpathlineto{\pgfqpoint{3.655470in}{2.878060in}}%
\pgfpathlineto{\pgfqpoint{3.647653in}{2.863797in}}%
\pgfpathclose%
\pgfusepath{fill}%
\end{pgfscope}%
\begin{pgfscope}%
\pgfpathrectangle{\pgfqpoint{1.150000in}{0.150000in}}{\pgfqpoint{5.700000in}{5.700000in}}%
\pgfusepath{clip}%
\pgfsetbuttcap%
\pgfsetroundjoin%
\definecolor{currentfill}{rgb}{0.203063,0.379716,0.553925}%
\pgfsetfillcolor{currentfill}%
\pgfsetfillopacity{0.800000}%
\pgfsetlinewidth{0.000000pt}%
\definecolor{currentstroke}{rgb}{0.000000,0.000000,0.000000}%
\pgfsetstrokecolor{currentstroke}%
\pgfsetdash{}{0pt}%
\pgfpathmoveto{\pgfqpoint{3.242267in}{3.200766in}}%
\pgfpathlineto{\pgfqpoint{3.255675in}{3.182968in}}%
\pgfpathlineto{\pgfqpoint{3.269077in}{3.165469in}}%
\pgfpathlineto{\pgfqpoint{3.282474in}{3.148269in}}%
\pgfpathlineto{\pgfqpoint{3.295866in}{3.131363in}}%
\pgfpathlineto{\pgfqpoint{3.303750in}{3.146691in}}%
\pgfpathlineto{\pgfqpoint{3.311628in}{3.162204in}}%
\pgfpathlineto{\pgfqpoint{3.319500in}{3.177907in}}%
\pgfpathlineto{\pgfqpoint{3.327366in}{3.193803in}}%
\pgfpathlineto{\pgfqpoint{3.313982in}{3.210997in}}%
\pgfpathlineto{\pgfqpoint{3.300593in}{3.228486in}}%
\pgfpathlineto{\pgfqpoint{3.287198in}{3.246272in}}%
\pgfpathlineto{\pgfqpoint{3.273798in}{3.264360in}}%
\pgfpathlineto{\pgfqpoint{3.265925in}{3.248163in}}%
\pgfpathlineto{\pgfqpoint{3.258045in}{3.232167in}}%
\pgfpathlineto{\pgfqpoint{3.250159in}{3.216370in}}%
\pgfpathlineto{\pgfqpoint{3.242267in}{3.200766in}}%
\pgfpathclose%
\pgfusepath{fill}%
\end{pgfscope}%
\begin{pgfscope}%
\pgfpathrectangle{\pgfqpoint{1.150000in}{0.150000in}}{\pgfqpoint{5.700000in}{5.700000in}}%
\pgfusepath{clip}%
\pgfsetbuttcap%
\pgfsetroundjoin%
\definecolor{currentfill}{rgb}{0.136408,0.541173,0.554483}%
\pgfsetfillcolor{currentfill}%
\pgfsetfillopacity{0.800000}%
\pgfsetlinewidth{0.000000pt}%
\definecolor{currentstroke}{rgb}{0.000000,0.000000,0.000000}%
\pgfsetstrokecolor{currentstroke}%
\pgfsetdash{}{0pt}%
\pgfpathmoveto{\pgfqpoint{3.089945in}{3.669056in}}%
\pgfpathlineto{\pgfqpoint{3.103467in}{3.645431in}}%
\pgfpathlineto{\pgfqpoint{3.116978in}{3.622156in}}%
\pgfpathlineto{\pgfqpoint{3.130479in}{3.599228in}}%
\pgfpathlineto{\pgfqpoint{3.143971in}{3.576644in}}%
\pgfpathlineto{\pgfqpoint{3.151833in}{3.594941in}}%
\pgfpathlineto{\pgfqpoint{3.159688in}{3.613488in}}%
\pgfpathlineto{\pgfqpoint{3.167536in}{3.632290in}}%
\pgfpathlineto{\pgfqpoint{3.175377in}{3.651350in}}%
\pgfpathlineto{\pgfqpoint{3.161891in}{3.674297in}}%
\pgfpathlineto{\pgfqpoint{3.148396in}{3.697588in}}%
\pgfpathlineto{\pgfqpoint{3.134891in}{3.721226in}}%
\pgfpathlineto{\pgfqpoint{3.121376in}{3.745216in}}%
\pgfpathlineto{\pgfqpoint{3.113529in}{3.725779in}}%
\pgfpathlineto{\pgfqpoint{3.105675in}{3.706610in}}%
\pgfpathlineto{\pgfqpoint{3.097814in}{3.687704in}}%
\pgfpathlineto{\pgfqpoint{3.089945in}{3.669056in}}%
\pgfpathclose%
\pgfusepath{fill}%
\end{pgfscope}%
\begin{pgfscope}%
\pgfpathrectangle{\pgfqpoint{1.150000in}{0.150000in}}{\pgfqpoint{5.700000in}{5.700000in}}%
\pgfusepath{clip}%
\pgfsetbuttcap%
\pgfsetroundjoin%
\definecolor{currentfill}{rgb}{0.244972,0.287675,0.537260}%
\pgfsetfillcolor{currentfill}%
\pgfsetfillopacity{0.800000}%
\pgfsetlinewidth{0.000000pt}%
\definecolor{currentstroke}{rgb}{0.000000,0.000000,0.000000}%
\pgfsetstrokecolor{currentstroke}%
\pgfsetdash{}{0pt}%
\pgfpathmoveto{\pgfqpoint{3.456252in}{2.950631in}}%
\pgfpathlineto{\pgfqpoint{3.469599in}{2.937346in}}%
\pgfpathlineto{\pgfqpoint{3.482943in}{2.924325in}}%
\pgfpathlineto{\pgfqpoint{3.496286in}{2.911566in}}%
\pgfpathlineto{\pgfqpoint{3.509628in}{2.899066in}}%
\pgfpathlineto{\pgfqpoint{3.517481in}{2.913298in}}%
\pgfpathlineto{\pgfqpoint{3.525328in}{2.927685in}}%
\pgfpathlineto{\pgfqpoint{3.533170in}{2.942228in}}%
\pgfpathlineto{\pgfqpoint{3.541007in}{2.956933in}}%
\pgfpathlineto{\pgfqpoint{3.527673in}{2.969717in}}%
\pgfpathlineto{\pgfqpoint{3.514338in}{2.982761in}}%
\pgfpathlineto{\pgfqpoint{3.501001in}{2.996067in}}%
\pgfpathlineto{\pgfqpoint{3.487662in}{3.009636in}}%
\pgfpathlineto{\pgfqpoint{3.479817in}{2.994635in}}%
\pgfpathlineto{\pgfqpoint{3.471968in}{2.979802in}}%
\pgfpathlineto{\pgfqpoint{3.464113in}{2.965136in}}%
\pgfpathlineto{\pgfqpoint{3.456252in}{2.950631in}}%
\pgfpathclose%
\pgfusepath{fill}%
\end{pgfscope}%
\begin{pgfscope}%
\pgfpathrectangle{\pgfqpoint{1.150000in}{0.150000in}}{\pgfqpoint{5.700000in}{5.700000in}}%
\pgfusepath{clip}%
\pgfsetbuttcap%
\pgfsetroundjoin%
\definecolor{currentfill}{rgb}{0.250425,0.274290,0.533103}%
\pgfsetfillcolor{currentfill}%
\pgfsetfillopacity{0.800000}%
\pgfsetlinewidth{0.000000pt}%
\definecolor{currentstroke}{rgb}{0.000000,0.000000,0.000000}%
\pgfsetstrokecolor{currentstroke}%
\pgfsetdash{}{0pt}%
\pgfpathmoveto{\pgfqpoint{4.229814in}{2.899186in}}%
\pgfpathlineto{\pgfqpoint{4.243214in}{2.893891in}}%
\pgfpathlineto{\pgfqpoint{4.256619in}{2.888801in}}%
\pgfpathlineto{\pgfqpoint{4.270030in}{2.883914in}}%
\pgfpathlineto{\pgfqpoint{4.283447in}{2.879230in}}%
\pgfpathlineto{\pgfqpoint{4.291119in}{2.893236in}}%
\pgfpathlineto{\pgfqpoint{4.298789in}{2.907415in}}%
\pgfpathlineto{\pgfqpoint{4.306456in}{2.921774in}}%
\pgfpathlineto{\pgfqpoint{4.314120in}{2.936318in}}%
\pgfpathlineto{\pgfqpoint{4.300711in}{2.941472in}}%
\pgfpathlineto{\pgfqpoint{4.287308in}{2.946828in}}%
\pgfpathlineto{\pgfqpoint{4.273911in}{2.952387in}}%
\pgfpathlineto{\pgfqpoint{4.260520in}{2.958151in}}%
\pgfpathlineto{\pgfqpoint{4.252848in}{2.943126in}}%
\pgfpathlineto{\pgfqpoint{4.245173in}{2.928294in}}%
\pgfpathlineto{\pgfqpoint{4.237495in}{2.913649in}}%
\pgfpathlineto{\pgfqpoint{4.229814in}{2.899186in}}%
\pgfpathclose%
\pgfusepath{fill}%
\end{pgfscope}%
\begin{pgfscope}%
\pgfpathrectangle{\pgfqpoint{1.150000in}{0.150000in}}{\pgfqpoint{5.700000in}{5.700000in}}%
\pgfusepath{clip}%
\pgfsetbuttcap%
\pgfsetroundjoin%
\definecolor{currentfill}{rgb}{0.262138,0.242286,0.520837}%
\pgfsetfillcolor{currentfill}%
\pgfsetfillopacity{0.800000}%
\pgfsetlinewidth{0.000000pt}%
\definecolor{currentstroke}{rgb}{0.000000,0.000000,0.000000}%
\pgfsetstrokecolor{currentstroke}%
\pgfsetdash{}{0pt}%
\pgfpathmoveto{\pgfqpoint{3.923301in}{2.831693in}}%
\pgfpathlineto{\pgfqpoint{3.936650in}{2.824216in}}%
\pgfpathlineto{\pgfqpoint{3.950001in}{2.816959in}}%
\pgfpathlineto{\pgfqpoint{3.963356in}{2.809922in}}%
\pgfpathlineto{\pgfqpoint{3.976715in}{2.803104in}}%
\pgfpathlineto{\pgfqpoint{3.984462in}{2.816874in}}%
\pgfpathlineto{\pgfqpoint{3.992204in}{2.830790in}}%
\pgfpathlineto{\pgfqpoint{3.999942in}{2.844857in}}%
\pgfpathlineto{\pgfqpoint{4.007676in}{2.859079in}}%
\pgfpathlineto{\pgfqpoint{3.994324in}{2.866273in}}%
\pgfpathlineto{\pgfqpoint{3.980976in}{2.873686in}}%
\pgfpathlineto{\pgfqpoint{3.967632in}{2.881318in}}%
\pgfpathlineto{\pgfqpoint{3.954290in}{2.889171in}}%
\pgfpathlineto{\pgfqpoint{3.946549in}{2.874562in}}%
\pgfpathlineto{\pgfqpoint{3.938804in}{2.860115in}}%
\pgfpathlineto{\pgfqpoint{3.931055in}{2.845827in}}%
\pgfpathlineto{\pgfqpoint{3.923301in}{2.831693in}}%
\pgfpathclose%
\pgfusepath{fill}%
\end{pgfscope}%
\begin{pgfscope}%
\pgfpathrectangle{\pgfqpoint{1.150000in}{0.150000in}}{\pgfqpoint{5.700000in}{5.700000in}}%
\pgfusepath{clip}%
\pgfsetbuttcap%
\pgfsetroundjoin%
\definecolor{currentfill}{rgb}{0.195860,0.395433,0.555276}%
\pgfsetfillcolor{currentfill}%
\pgfsetfillopacity{0.800000}%
\pgfsetlinewidth{0.000000pt}%
\definecolor{currentstroke}{rgb}{0.000000,0.000000,0.000000}%
\pgfsetstrokecolor{currentstroke}%
\pgfsetdash{}{0pt}%
\pgfpathmoveto{\pgfqpoint{4.820109in}{3.208706in}}%
\pgfpathlineto{\pgfqpoint{4.833639in}{3.204904in}}%
\pgfpathlineto{\pgfqpoint{4.847178in}{3.201287in}}%
\pgfpathlineto{\pgfqpoint{4.860726in}{3.197854in}}%
\pgfpathlineto{\pgfqpoint{4.874282in}{3.194605in}}%
\pgfpathlineto{\pgfqpoint{4.881852in}{3.211108in}}%
\pgfpathlineto{\pgfqpoint{4.889423in}{3.227917in}}%
\pgfpathlineto{\pgfqpoint{4.896996in}{3.245040in}}%
\pgfpathlineto{\pgfqpoint{4.904572in}{3.262486in}}%
\pgfpathlineto{\pgfqpoint{4.891029in}{3.266424in}}%
\pgfpathlineto{\pgfqpoint{4.877495in}{3.270546in}}%
\pgfpathlineto{\pgfqpoint{4.863969in}{3.274852in}}%
\pgfpathlineto{\pgfqpoint{4.850452in}{3.279343in}}%
\pgfpathlineto{\pgfqpoint{4.842863in}{3.261196in}}%
\pgfpathlineto{\pgfqpoint{4.835276in}{3.243380in}}%
\pgfpathlineto{\pgfqpoint{4.827691in}{3.225886in}}%
\pgfpathlineto{\pgfqpoint{4.820109in}{3.208706in}}%
\pgfpathclose%
\pgfusepath{fill}%
\end{pgfscope}%
\begin{pgfscope}%
\pgfpathrectangle{\pgfqpoint{1.150000in}{0.150000in}}{\pgfqpoint{5.700000in}{5.700000in}}%
\pgfusepath{clip}%
\pgfsetbuttcap%
\pgfsetroundjoin%
\definecolor{currentfill}{rgb}{0.157729,0.485932,0.558013}%
\pgfsetfillcolor{currentfill}%
\pgfsetfillopacity{0.800000}%
\pgfsetlinewidth{0.000000pt}%
\definecolor{currentstroke}{rgb}{0.000000,0.000000,0.000000}%
\pgfsetstrokecolor{currentstroke}%
\pgfsetdash{}{0pt}%
\pgfpathmoveto{\pgfqpoint{3.112455in}{3.505872in}}%
\pgfpathlineto{\pgfqpoint{3.125945in}{3.483955in}}%
\pgfpathlineto{\pgfqpoint{3.139425in}{3.462374in}}%
\pgfpathlineto{\pgfqpoint{3.152897in}{3.441126in}}%
\pgfpathlineto{\pgfqpoint{3.166360in}{3.420208in}}%
\pgfpathlineto{\pgfqpoint{3.174242in}{3.437232in}}%
\pgfpathlineto{\pgfqpoint{3.182117in}{3.454482in}}%
\pgfpathlineto{\pgfqpoint{3.189986in}{3.471960in}}%
\pgfpathlineto{\pgfqpoint{3.197848in}{3.489673in}}%
\pgfpathlineto{\pgfqpoint{3.184391in}{3.510917in}}%
\pgfpathlineto{\pgfqpoint{3.170927in}{3.532491in}}%
\pgfpathlineto{\pgfqpoint{3.157453in}{3.554399in}}%
\pgfpathlineto{\pgfqpoint{3.143971in}{3.576644in}}%
\pgfpathlineto{\pgfqpoint{3.136102in}{3.558592in}}%
\pgfpathlineto{\pgfqpoint{3.128227in}{3.540782in}}%
\pgfpathlineto{\pgfqpoint{3.120344in}{3.523210in}}%
\pgfpathlineto{\pgfqpoint{3.112455in}{3.505872in}}%
\pgfpathclose%
\pgfusepath{fill}%
\end{pgfscope}%
\begin{pgfscope}%
\pgfpathrectangle{\pgfqpoint{1.150000in}{0.150000in}}{\pgfqpoint{5.700000in}{5.700000in}}%
\pgfusepath{clip}%
\pgfsetbuttcap%
\pgfsetroundjoin%
\definecolor{currentfill}{rgb}{0.876168,0.891125,0.095250}%
\pgfsetfillcolor{currentfill}%
\pgfsetfillopacity{0.800000}%
\pgfsetlinewidth{0.000000pt}%
\definecolor{currentstroke}{rgb}{0.000000,0.000000,0.000000}%
\pgfsetstrokecolor{currentstroke}%
\pgfsetdash{}{0pt}%
\pgfpathmoveto{\pgfqpoint{3.545098in}{4.982386in}}%
\pgfpathlineto{\pgfqpoint{3.558603in}{4.952570in}}%
\pgfpathlineto{\pgfqpoint{3.572098in}{4.923110in}}%
\pgfpathlineto{\pgfqpoint{3.585583in}{4.894001in}}%
\pgfpathlineto{\pgfqpoint{3.599058in}{4.865240in}}%
\pgfpathlineto{\pgfqpoint{3.606669in}{4.903712in}}%
\pgfpathlineto{\pgfqpoint{3.614278in}{4.942790in}}%
\pgfpathlineto{\pgfqpoint{3.621886in}{4.982488in}}%
\pgfpathlineto{\pgfqpoint{3.608403in}{5.011889in}}%
\pgfpathlineto{\pgfqpoint{3.594909in}{5.041641in}}%
\pgfpathlineto{\pgfqpoint{3.581406in}{5.071745in}}%
\pgfpathlineto{\pgfqpoint{3.567892in}{5.102207in}}%
\pgfpathlineto{\pgfqpoint{3.560296in}{5.061641in}}%
\pgfpathlineto{\pgfqpoint{3.552698in}{5.021704in}}%
\pgfpathlineto{\pgfqpoint{3.545098in}{4.982386in}}%
\pgfpathclose%
\pgfusepath{fill}%
\end{pgfscope}%
\begin{pgfscope}%
\pgfpathrectangle{\pgfqpoint{1.150000in}{0.150000in}}{\pgfqpoint{5.700000in}{5.700000in}}%
\pgfusepath{clip}%
\pgfsetbuttcap%
\pgfsetroundjoin%
\definecolor{currentfill}{rgb}{0.190631,0.407061,0.556089}%
\pgfsetfillcolor{currentfill}%
\pgfsetfillopacity{0.800000}%
\pgfsetlinewidth{0.000000pt}%
\definecolor{currentstroke}{rgb}{0.000000,0.000000,0.000000}%
\pgfsetstrokecolor{currentstroke}%
\pgfsetdash{}{0pt}%
\pgfpathmoveto{\pgfqpoint{3.188572in}{3.275015in}}%
\pgfpathlineto{\pgfqpoint{3.202006in}{3.255989in}}%
\pgfpathlineto{\pgfqpoint{3.215432in}{3.237274in}}%
\pgfpathlineto{\pgfqpoint{3.228853in}{3.218867in}}%
\pgfpathlineto{\pgfqpoint{3.242267in}{3.200766in}}%
\pgfpathlineto{\pgfqpoint{3.250159in}{3.216370in}}%
\pgfpathlineto{\pgfqpoint{3.258045in}{3.232167in}}%
\pgfpathlineto{\pgfqpoint{3.265925in}{3.248163in}}%
\pgfpathlineto{\pgfqpoint{3.273798in}{3.264360in}}%
\pgfpathlineto{\pgfqpoint{3.260392in}{3.282750in}}%
\pgfpathlineto{\pgfqpoint{3.246980in}{3.301446in}}%
\pgfpathlineto{\pgfqpoint{3.233561in}{3.320452in}}%
\pgfpathlineto{\pgfqpoint{3.220135in}{3.339768in}}%
\pgfpathlineto{\pgfqpoint{3.212254in}{3.323269in}}%
\pgfpathlineto{\pgfqpoint{3.204367in}{3.306980in}}%
\pgfpathlineto{\pgfqpoint{3.196473in}{3.290896in}}%
\pgfpathlineto{\pgfqpoint{3.188572in}{3.275015in}}%
\pgfpathclose%
\pgfusepath{fill}%
\end{pgfscope}%
\begin{pgfscope}%
\pgfpathrectangle{\pgfqpoint{1.150000in}{0.150000in}}{\pgfqpoint{5.700000in}{5.700000in}}%
\pgfusepath{clip}%
\pgfsetbuttcap%
\pgfsetroundjoin%
\definecolor{currentfill}{rgb}{0.255645,0.260703,0.528312}%
\pgfsetfillcolor{currentfill}%
\pgfsetfillopacity{0.800000}%
\pgfsetlinewidth{0.000000pt}%
\definecolor{currentstroke}{rgb}{0.000000,0.000000,0.000000}%
\pgfsetstrokecolor{currentstroke}%
\pgfsetdash{}{0pt}%
\pgfpathmoveto{\pgfqpoint{4.145486in}{2.864534in}}%
\pgfpathlineto{\pgfqpoint{4.158872in}{2.858850in}}%
\pgfpathlineto{\pgfqpoint{4.172262in}{2.853375in}}%
\pgfpathlineto{\pgfqpoint{4.185658in}{2.848107in}}%
\pgfpathlineto{\pgfqpoint{4.199060in}{2.843045in}}%
\pgfpathlineto{\pgfqpoint{4.206753in}{2.856834in}}%
\pgfpathlineto{\pgfqpoint{4.214444in}{2.870784in}}%
\pgfpathlineto{\pgfqpoint{4.222130in}{2.884899in}}%
\pgfpathlineto{\pgfqpoint{4.229814in}{2.899186in}}%
\pgfpathlineto{\pgfqpoint{4.216421in}{2.904686in}}%
\pgfpathlineto{\pgfqpoint{4.203033in}{2.910392in}}%
\pgfpathlineto{\pgfqpoint{4.189650in}{2.916306in}}%
\pgfpathlineto{\pgfqpoint{4.176272in}{2.922427in}}%
\pgfpathlineto{\pgfqpoint{4.168581in}{2.907691in}}%
\pgfpathlineto{\pgfqpoint{4.160886in}{2.893133in}}%
\pgfpathlineto{\pgfqpoint{4.153188in}{2.878749in}}%
\pgfpathlineto{\pgfqpoint{4.145486in}{2.864534in}}%
\pgfpathclose%
\pgfusepath{fill}%
\end{pgfscope}%
\begin{pgfscope}%
\pgfpathrectangle{\pgfqpoint{1.150000in}{0.150000in}}{\pgfqpoint{5.700000in}{5.700000in}}%
\pgfusepath{clip}%
\pgfsetbuttcap%
\pgfsetroundjoin%
\definecolor{currentfill}{rgb}{0.252194,0.269783,0.531579}%
\pgfsetfillcolor{currentfill}%
\pgfsetfillopacity{0.800000}%
\pgfsetlinewidth{0.000000pt}%
\definecolor{currentstroke}{rgb}{0.000000,0.000000,0.000000}%
\pgfsetstrokecolor{currentstroke}%
\pgfsetdash{}{0pt}%
\pgfpathmoveto{\pgfqpoint{3.509628in}{2.899066in}}%
\pgfpathlineto{\pgfqpoint{3.522968in}{2.886824in}}%
\pgfpathlineto{\pgfqpoint{3.536307in}{2.874838in}}%
\pgfpathlineto{\pgfqpoint{3.549646in}{2.863107in}}%
\pgfpathlineto{\pgfqpoint{3.562984in}{2.851627in}}%
\pgfpathlineto{\pgfqpoint{3.570829in}{2.865588in}}%
\pgfpathlineto{\pgfqpoint{3.578669in}{2.879694in}}%
\pgfpathlineto{\pgfqpoint{3.586504in}{2.893950in}}%
\pgfpathlineto{\pgfqpoint{3.594333in}{2.908359in}}%
\pgfpathlineto{\pgfqpoint{3.581003in}{2.920122in}}%
\pgfpathlineto{\pgfqpoint{3.567672in}{2.932137in}}%
\pgfpathlineto{\pgfqpoint{3.554340in}{2.944407in}}%
\pgfpathlineto{\pgfqpoint{3.541007in}{2.956933in}}%
\pgfpathlineto{\pgfqpoint{3.533170in}{2.942228in}}%
\pgfpathlineto{\pgfqpoint{3.525328in}{2.927685in}}%
\pgfpathlineto{\pgfqpoint{3.517481in}{2.913298in}}%
\pgfpathlineto{\pgfqpoint{3.509628in}{2.899066in}}%
\pgfpathclose%
\pgfusepath{fill}%
\end{pgfscope}%
\begin{pgfscope}%
\pgfpathrectangle{\pgfqpoint{1.150000in}{0.150000in}}{\pgfqpoint{5.700000in}{5.700000in}}%
\pgfusepath{clip}%
\pgfsetbuttcap%
\pgfsetroundjoin%
\definecolor{currentfill}{rgb}{0.185556,0.418570,0.556753}%
\pgfsetfillcolor{currentfill}%
\pgfsetfillopacity{0.800000}%
\pgfsetlinewidth{0.000000pt}%
\definecolor{currentstroke}{rgb}{0.000000,0.000000,0.000000}%
\pgfsetstrokecolor{currentstroke}%
\pgfsetdash{}{0pt}%
\pgfpathmoveto{\pgfqpoint{4.904572in}{3.262486in}}%
\pgfpathlineto{\pgfqpoint{4.918123in}{3.258731in}}%
\pgfpathlineto{\pgfqpoint{4.931684in}{3.255158in}}%
\pgfpathlineto{\pgfqpoint{4.945253in}{3.251767in}}%
\pgfpathlineto{\pgfqpoint{4.958831in}{3.248558in}}%
\pgfpathlineto{\pgfqpoint{4.966395in}{3.265625in}}%
\pgfpathlineto{\pgfqpoint{4.973962in}{3.283024in}}%
\pgfpathlineto{\pgfqpoint{4.981532in}{3.300762in}}%
\pgfpathlineto{\pgfqpoint{4.989106in}{3.318848in}}%
\pgfpathlineto{\pgfqpoint{4.975542in}{3.322778in}}%
\pgfpathlineto{\pgfqpoint{4.961987in}{3.326889in}}%
\pgfpathlineto{\pgfqpoint{4.948441in}{3.331183in}}%
\pgfpathlineto{\pgfqpoint{4.934904in}{3.335659in}}%
\pgfpathlineto{\pgfqpoint{4.927316in}{3.316841in}}%
\pgfpathlineto{\pgfqpoint{4.919731in}{3.298378in}}%
\pgfpathlineto{\pgfqpoint{4.912150in}{3.280263in}}%
\pgfpathlineto{\pgfqpoint{4.904572in}{3.262486in}}%
\pgfpathclose%
\pgfusepath{fill}%
\end{pgfscope}%
\begin{pgfscope}%
\pgfpathrectangle{\pgfqpoint{1.150000in}{0.150000in}}{\pgfqpoint{5.700000in}{5.700000in}}%
\pgfusepath{clip}%
\pgfsetbuttcap%
\pgfsetroundjoin%
\definecolor{currentfill}{rgb}{0.262138,0.242286,0.520837}%
\pgfsetfillcolor{currentfill}%
\pgfsetfillopacity{0.800000}%
\pgfsetlinewidth{0.000000pt}%
\definecolor{currentstroke}{rgb}{0.000000,0.000000,0.000000}%
\pgfsetstrokecolor{currentstroke}%
\pgfsetdash{}{0pt}%
\pgfpathmoveto{\pgfqpoint{3.700979in}{2.823141in}}%
\pgfpathlineto{\pgfqpoint{3.714313in}{2.813576in}}%
\pgfpathlineto{\pgfqpoint{3.727649in}{2.804247in}}%
\pgfpathlineto{\pgfqpoint{3.740986in}{2.795152in}}%
\pgfpathlineto{\pgfqpoint{3.754325in}{2.786292in}}%
\pgfpathlineto{\pgfqpoint{3.762128in}{2.799949in}}%
\pgfpathlineto{\pgfqpoint{3.769926in}{2.813744in}}%
\pgfpathlineto{\pgfqpoint{3.777720in}{2.827679in}}%
\pgfpathlineto{\pgfqpoint{3.785509in}{2.841759in}}%
\pgfpathlineto{\pgfqpoint{3.772177in}{2.850933in}}%
\pgfpathlineto{\pgfqpoint{3.758847in}{2.860341in}}%
\pgfpathlineto{\pgfqpoint{3.745519in}{2.869985in}}%
\pgfpathlineto{\pgfqpoint{3.732192in}{2.879864in}}%
\pgfpathlineto{\pgfqpoint{3.724396in}{2.865459in}}%
\pgfpathlineto{\pgfqpoint{3.716595in}{2.851206in}}%
\pgfpathlineto{\pgfqpoint{3.708790in}{2.837101in}}%
\pgfpathlineto{\pgfqpoint{3.700979in}{2.823141in}}%
\pgfpathclose%
\pgfusepath{fill}%
\end{pgfscope}%
\begin{pgfscope}%
\pgfpathrectangle{\pgfqpoint{1.150000in}{0.150000in}}{\pgfqpoint{5.700000in}{5.700000in}}%
\pgfusepath{clip}%
\pgfsetbuttcap%
\pgfsetroundjoin%
\definecolor{currentfill}{rgb}{0.866013,0.889868,0.095953}%
\pgfsetfillcolor{currentfill}%
\pgfsetfillopacity{0.800000}%
\pgfsetlinewidth{0.000000pt}%
\definecolor{currentstroke}{rgb}{0.000000,0.000000,0.000000}%
\pgfsetstrokecolor{currentstroke}%
\pgfsetdash{}{0pt}%
\pgfpathmoveto{\pgfqpoint{3.460592in}{4.950672in}}%
\pgfpathlineto{\pgfqpoint{3.474129in}{4.920231in}}%
\pgfpathlineto{\pgfqpoint{3.487655in}{4.890157in}}%
\pgfpathlineto{\pgfqpoint{3.501170in}{4.860447in}}%
\pgfpathlineto{\pgfqpoint{3.514674in}{4.831098in}}%
\pgfpathlineto{\pgfqpoint{3.522284in}{4.868042in}}%
\pgfpathlineto{\pgfqpoint{3.529892in}{4.905565in}}%
\pgfpathlineto{\pgfqpoint{3.537496in}{4.943676in}}%
\pgfpathlineto{\pgfqpoint{3.545098in}{4.982386in}}%
\pgfpathlineto{\pgfqpoint{3.531583in}{5.012560in}}%
\pgfpathlineto{\pgfqpoint{3.518057in}{5.043097in}}%
\pgfpathlineto{\pgfqpoint{3.504521in}{5.074000in}}%
\pgfpathlineto{\pgfqpoint{3.490972in}{5.105273in}}%
\pgfpathlineto{\pgfqpoint{3.483382in}{5.065718in}}%
\pgfpathlineto{\pgfqpoint{3.475788in}{5.026773in}}%
\pgfpathlineto{\pgfqpoint{3.468192in}{4.988428in}}%
\pgfpathlineto{\pgfqpoint{3.460592in}{4.950672in}}%
\pgfpathclose%
\pgfusepath{fill}%
\end{pgfscope}%
\begin{pgfscope}%
\pgfpathrectangle{\pgfqpoint{1.150000in}{0.150000in}}{\pgfqpoint{5.700000in}{5.700000in}}%
\pgfusepath{clip}%
\pgfsetbuttcap%
\pgfsetroundjoin%
\definecolor{currentfill}{rgb}{0.263663,0.237631,0.518762}%
\pgfsetfillcolor{currentfill}%
\pgfsetfillopacity{0.800000}%
\pgfsetlinewidth{0.000000pt}%
\definecolor{currentstroke}{rgb}{0.000000,0.000000,0.000000}%
\pgfsetstrokecolor{currentstroke}%
\pgfsetdash{}{0pt}%
\pgfpathmoveto{\pgfqpoint{3.838856in}{2.807369in}}%
\pgfpathlineto{\pgfqpoint{3.852199in}{2.799341in}}%
\pgfpathlineto{\pgfqpoint{3.865545in}{2.791540in}}%
\pgfpathlineto{\pgfqpoint{3.878894in}{2.783963in}}%
\pgfpathlineto{\pgfqpoint{3.892246in}{2.776608in}}%
\pgfpathlineto{\pgfqpoint{3.900016in}{2.790170in}}%
\pgfpathlineto{\pgfqpoint{3.907782in}{2.803868in}}%
\pgfpathlineto{\pgfqpoint{3.915544in}{2.817708in}}%
\pgfpathlineto{\pgfqpoint{3.923301in}{2.831693in}}%
\pgfpathlineto{\pgfqpoint{3.909957in}{2.839392in}}%
\pgfpathlineto{\pgfqpoint{3.896615in}{2.847314in}}%
\pgfpathlineto{\pgfqpoint{3.883276in}{2.855460in}}%
\pgfpathlineto{\pgfqpoint{3.869940in}{2.863832in}}%
\pgfpathlineto{\pgfqpoint{3.862176in}{2.849491in}}%
\pgfpathlineto{\pgfqpoint{3.854407in}{2.835303in}}%
\pgfpathlineto{\pgfqpoint{3.846634in}{2.821263in}}%
\pgfpathlineto{\pgfqpoint{3.838856in}{2.807369in}}%
\pgfpathclose%
\pgfusepath{fill}%
\end{pgfscope}%
\begin{pgfscope}%
\pgfpathrectangle{\pgfqpoint{1.150000in}{0.150000in}}{\pgfqpoint{5.700000in}{5.700000in}}%
\pgfusepath{clip}%
\pgfsetbuttcap%
\pgfsetroundjoin%
\definecolor{currentfill}{rgb}{0.177423,0.437527,0.557565}%
\pgfsetfillcolor{currentfill}%
\pgfsetfillopacity{0.800000}%
\pgfsetlinewidth{0.000000pt}%
\definecolor{currentstroke}{rgb}{0.000000,0.000000,0.000000}%
\pgfsetstrokecolor{currentstroke}%
\pgfsetdash{}{0pt}%
\pgfpathmoveto{\pgfqpoint{3.134765in}{3.354289in}}%
\pgfpathlineto{\pgfqpoint{3.148228in}{3.333989in}}%
\pgfpathlineto{\pgfqpoint{3.161684in}{3.314012in}}%
\pgfpathlineto{\pgfqpoint{3.175131in}{3.294355in}}%
\pgfpathlineto{\pgfqpoint{3.188572in}{3.275015in}}%
\pgfpathlineto{\pgfqpoint{3.196473in}{3.290896in}}%
\pgfpathlineto{\pgfqpoint{3.204367in}{3.306980in}}%
\pgfpathlineto{\pgfqpoint{3.212254in}{3.323269in}}%
\pgfpathlineto{\pgfqpoint{3.220135in}{3.339768in}}%
\pgfpathlineto{\pgfqpoint{3.206703in}{3.359399in}}%
\pgfpathlineto{\pgfqpoint{3.193263in}{3.379347in}}%
\pgfpathlineto{\pgfqpoint{3.179815in}{3.399616in}}%
\pgfpathlineto{\pgfqpoint{3.166360in}{3.420208in}}%
\pgfpathlineto{\pgfqpoint{3.158472in}{3.403405in}}%
\pgfpathlineto{\pgfqpoint{3.150576in}{3.386820in}}%
\pgfpathlineto{\pgfqpoint{3.142674in}{3.370450in}}%
\pgfpathlineto{\pgfqpoint{3.134765in}{3.354289in}}%
\pgfpathclose%
\pgfusepath{fill}%
\end{pgfscope}%
\begin{pgfscope}%
\pgfpathrectangle{\pgfqpoint{1.150000in}{0.150000in}}{\pgfqpoint{5.700000in}{5.700000in}}%
\pgfusepath{clip}%
\pgfsetbuttcap%
\pgfsetroundjoin%
\definecolor{currentfill}{rgb}{0.260571,0.246922,0.522828}%
\pgfsetfillcolor{currentfill}%
\pgfsetfillopacity{0.800000}%
\pgfsetlinewidth{0.000000pt}%
\definecolor{currentstroke}{rgb}{0.000000,0.000000,0.000000}%
\pgfsetstrokecolor{currentstroke}%
\pgfsetdash{}{0pt}%
\pgfpathmoveto{\pgfqpoint{4.061124in}{2.832464in}}%
\pgfpathlineto{\pgfqpoint{4.074498in}{2.826345in}}%
\pgfpathlineto{\pgfqpoint{4.087875in}{2.820439in}}%
\pgfpathlineto{\pgfqpoint{4.101258in}{2.814743in}}%
\pgfpathlineto{\pgfqpoint{4.114646in}{2.809258in}}%
\pgfpathlineto{\pgfqpoint{4.122361in}{2.822849in}}%
\pgfpathlineto{\pgfqpoint{4.130073in}{2.836588in}}%
\pgfpathlineto{\pgfqpoint{4.137782in}{2.850482in}}%
\pgfpathlineto{\pgfqpoint{4.145486in}{2.864534in}}%
\pgfpathlineto{\pgfqpoint{4.132106in}{2.870426in}}%
\pgfpathlineto{\pgfqpoint{4.118731in}{2.876529in}}%
\pgfpathlineto{\pgfqpoint{4.105361in}{2.882842in}}%
\pgfpathlineto{\pgfqpoint{4.091995in}{2.889368in}}%
\pgfpathlineto{\pgfqpoint{4.084283in}{2.874897in}}%
\pgfpathlineto{\pgfqpoint{4.076567in}{2.860593in}}%
\pgfpathlineto{\pgfqpoint{4.068848in}{2.846450in}}%
\pgfpathlineto{\pgfqpoint{4.061124in}{2.832464in}}%
\pgfpathclose%
\pgfusepath{fill}%
\end{pgfscope}%
\begin{pgfscope}%
\pgfpathrectangle{\pgfqpoint{1.150000in}{0.150000in}}{\pgfqpoint{5.700000in}{5.700000in}}%
\pgfusepath{clip}%
\pgfsetbuttcap%
\pgfsetroundjoin%
\definecolor{currentfill}{rgb}{0.258965,0.251537,0.524736}%
\pgfsetfillcolor{currentfill}%
\pgfsetfillopacity{0.800000}%
\pgfsetlinewidth{0.000000pt}%
\definecolor{currentstroke}{rgb}{0.000000,0.000000,0.000000}%
\pgfsetstrokecolor{currentstroke}%
\pgfsetdash{}{0pt}%
\pgfpathmoveto{\pgfqpoint{3.562984in}{2.851627in}}%
\pgfpathlineto{\pgfqpoint{3.576321in}{2.840399in}}%
\pgfpathlineto{\pgfqpoint{3.589658in}{2.829419in}}%
\pgfpathlineto{\pgfqpoint{3.602996in}{2.818686in}}%
\pgfpathlineto{\pgfqpoint{3.616333in}{2.808199in}}%
\pgfpathlineto{\pgfqpoint{3.624171in}{2.821888in}}%
\pgfpathlineto{\pgfqpoint{3.632003in}{2.835715in}}%
\pgfpathlineto{\pgfqpoint{3.639831in}{2.849683in}}%
\pgfpathlineto{\pgfqpoint{3.647653in}{2.863797in}}%
\pgfpathlineto{\pgfqpoint{3.634323in}{2.874567in}}%
\pgfpathlineto{\pgfqpoint{3.620993in}{2.885583in}}%
\pgfpathlineto{\pgfqpoint{3.607663in}{2.896846in}}%
\pgfpathlineto{\pgfqpoint{3.594333in}{2.908359in}}%
\pgfpathlineto{\pgfqpoint{3.586504in}{2.893950in}}%
\pgfpathlineto{\pgfqpoint{3.578669in}{2.879694in}}%
\pgfpathlineto{\pgfqpoint{3.570829in}{2.865588in}}%
\pgfpathlineto{\pgfqpoint{3.562984in}{2.851627in}}%
\pgfpathclose%
\pgfusepath{fill}%
\end{pgfscope}%
\begin{pgfscope}%
\pgfpathrectangle{\pgfqpoint{1.150000in}{0.150000in}}{\pgfqpoint{5.700000in}{5.700000in}}%
\pgfusepath{clip}%
\pgfsetbuttcap%
\pgfsetroundjoin%
\definecolor{currentfill}{rgb}{0.177423,0.437527,0.557565}%
\pgfsetfillcolor{currentfill}%
\pgfsetfillopacity{0.800000}%
\pgfsetlinewidth{0.000000pt}%
\definecolor{currentstroke}{rgb}{0.000000,0.000000,0.000000}%
\pgfsetstrokecolor{currentstroke}%
\pgfsetdash{}{0pt}%
\pgfpathmoveto{\pgfqpoint{4.989106in}{3.318848in}}%
\pgfpathlineto{\pgfqpoint{5.002679in}{3.315098in}}%
\pgfpathlineto{\pgfqpoint{5.016260in}{3.311529in}}%
\pgfpathlineto{\pgfqpoint{5.029851in}{3.308141in}}%
\pgfpathlineto{\pgfqpoint{5.043452in}{3.304931in}}%
\pgfpathlineto{\pgfqpoint{5.051014in}{3.322633in}}%
\pgfpathlineto{\pgfqpoint{5.058579in}{3.340692in}}%
\pgfpathlineto{\pgfqpoint{5.066150in}{3.359117in}}%
\pgfpathlineto{\pgfqpoint{5.052561in}{3.362888in}}%
\pgfpathlineto{\pgfqpoint{5.038982in}{3.366838in}}%
\pgfpathlineto{\pgfqpoint{5.025412in}{3.370969in}}%
\pgfpathlineto{\pgfqpoint{5.011850in}{3.375279in}}%
\pgfpathlineto{\pgfqpoint{5.004265in}{3.356098in}}%
\pgfpathlineto{\pgfqpoint{4.996683in}{3.337290in}}%
\pgfpathlineto{\pgfqpoint{4.989106in}{3.318848in}}%
\pgfpathclose%
\pgfusepath{fill}%
\end{pgfscope}%
\begin{pgfscope}%
\pgfpathrectangle{\pgfqpoint{1.150000in}{0.150000in}}{\pgfqpoint{5.700000in}{5.700000in}}%
\pgfusepath{clip}%
\pgfsetbuttcap%
\pgfsetroundjoin%
\definecolor{currentfill}{rgb}{0.144759,0.519093,0.556572}%
\pgfsetfillcolor{currentfill}%
\pgfsetfillopacity{0.800000}%
\pgfsetlinewidth{0.000000pt}%
\definecolor{currentstroke}{rgb}{0.000000,0.000000,0.000000}%
\pgfsetstrokecolor{currentstroke}%
\pgfsetdash{}{0pt}%
\pgfpathmoveto{\pgfqpoint{3.058401in}{3.596968in}}%
\pgfpathlineto{\pgfqpoint{3.071929in}{3.573673in}}%
\pgfpathlineto{\pgfqpoint{3.085448in}{3.550728in}}%
\pgfpathlineto{\pgfqpoint{3.098956in}{3.528128in}}%
\pgfpathlineto{\pgfqpoint{3.112455in}{3.505872in}}%
\pgfpathlineto{\pgfqpoint{3.120344in}{3.523210in}}%
\pgfpathlineto{\pgfqpoint{3.128227in}{3.540782in}}%
\pgfpathlineto{\pgfqpoint{3.136102in}{3.558592in}}%
\pgfpathlineto{\pgfqpoint{3.143971in}{3.576644in}}%
\pgfpathlineto{\pgfqpoint{3.130479in}{3.599228in}}%
\pgfpathlineto{\pgfqpoint{3.116978in}{3.622156in}}%
\pgfpathlineto{\pgfqpoint{3.103467in}{3.645431in}}%
\pgfpathlineto{\pgfqpoint{3.089945in}{3.669056in}}%
\pgfpathlineto{\pgfqpoint{3.082070in}{3.650663in}}%
\pgfpathlineto{\pgfqpoint{3.074188in}{3.632520in}}%
\pgfpathlineto{\pgfqpoint{3.066298in}{3.614623in}}%
\pgfpathlineto{\pgfqpoint{3.058401in}{3.596968in}}%
\pgfpathclose%
\pgfusepath{fill}%
\end{pgfscope}%
\begin{pgfscope}%
\pgfpathrectangle{\pgfqpoint{1.150000in}{0.150000in}}{\pgfqpoint{5.700000in}{5.700000in}}%
\pgfusepath{clip}%
\pgfsetbuttcap%
\pgfsetroundjoin%
\definecolor{currentfill}{rgb}{0.229739,0.322361,0.545706}%
\pgfsetfillcolor{currentfill}%
\pgfsetfillopacity{0.800000}%
\pgfsetlinewidth{0.000000pt}%
\definecolor{currentstroke}{rgb}{0.000000,0.000000,0.000000}%
\pgfsetstrokecolor{currentstroke}%
\pgfsetdash{}{0pt}%
\pgfpathmoveto{\pgfqpoint{4.536553in}{3.001319in}}%
\pgfpathlineto{\pgfqpoint{4.550032in}{2.997696in}}%
\pgfpathlineto{\pgfqpoint{4.563520in}{2.994266in}}%
\pgfpathlineto{\pgfqpoint{4.577015in}{2.991027in}}%
\pgfpathlineto{\pgfqpoint{4.590518in}{2.987978in}}%
\pgfpathlineto{\pgfqpoint{4.598125in}{3.002247in}}%
\pgfpathlineto{\pgfqpoint{4.605730in}{3.016731in}}%
\pgfpathlineto{\pgfqpoint{4.613334in}{3.031436in}}%
\pgfpathlineto{\pgfqpoint{4.620937in}{3.046368in}}%
\pgfpathlineto{\pgfqpoint{4.607445in}{3.049979in}}%
\pgfpathlineto{\pgfqpoint{4.593960in}{3.053782in}}%
\pgfpathlineto{\pgfqpoint{4.580484in}{3.057776in}}%
\pgfpathlineto{\pgfqpoint{4.567014in}{3.061962in}}%
\pgfpathlineto{\pgfqpoint{4.559401in}{3.046455in}}%
\pgfpathlineto{\pgfqpoint{4.551786in}{3.031183in}}%
\pgfpathlineto{\pgfqpoint{4.544170in}{3.016140in}}%
\pgfpathlineto{\pgfqpoint{4.536553in}{3.001319in}}%
\pgfpathclose%
\pgfusepath{fill}%
\end{pgfscope}%
\begin{pgfscope}%
\pgfpathrectangle{\pgfqpoint{1.150000in}{0.150000in}}{\pgfqpoint{5.700000in}{5.700000in}}%
\pgfusepath{clip}%
\pgfsetbuttcap%
\pgfsetroundjoin%
\definecolor{currentfill}{rgb}{0.221989,0.339161,0.548752}%
\pgfsetfillcolor{currentfill}%
\pgfsetfillopacity{0.800000}%
\pgfsetlinewidth{0.000000pt}%
\definecolor{currentstroke}{rgb}{0.000000,0.000000,0.000000}%
\pgfsetstrokecolor{currentstroke}%
\pgfsetdash{}{0pt}%
\pgfpathmoveto{\pgfqpoint{4.620937in}{3.046368in}}%
\pgfpathlineto{\pgfqpoint{4.634437in}{3.042946in}}%
\pgfpathlineto{\pgfqpoint{4.647946in}{3.039714in}}%
\pgfpathlineto{\pgfqpoint{4.661462in}{3.036670in}}%
\pgfpathlineto{\pgfqpoint{4.674988in}{3.033815in}}%
\pgfpathlineto{\pgfqpoint{4.682578in}{3.048399in}}%
\pgfpathlineto{\pgfqpoint{4.690168in}{3.063215in}}%
\pgfpathlineto{\pgfqpoint{4.697757in}{3.078272in}}%
\pgfpathlineto{\pgfqpoint{4.705346in}{3.093576in}}%
\pgfpathlineto{\pgfqpoint{4.691833in}{3.097025in}}%
\pgfpathlineto{\pgfqpoint{4.678328in}{3.100664in}}%
\pgfpathlineto{\pgfqpoint{4.664831in}{3.104491in}}%
\pgfpathlineto{\pgfqpoint{4.651342in}{3.108507in}}%
\pgfpathlineto{\pgfqpoint{4.643741in}{3.092597in}}%
\pgfpathlineto{\pgfqpoint{4.636141in}{3.076941in}}%
\pgfpathlineto{\pgfqpoint{4.628539in}{3.061534in}}%
\pgfpathlineto{\pgfqpoint{4.620937in}{3.046368in}}%
\pgfpathclose%
\pgfusepath{fill}%
\end{pgfscope}%
\begin{pgfscope}%
\pgfpathrectangle{\pgfqpoint{1.150000in}{0.150000in}}{\pgfqpoint{5.700000in}{5.700000in}}%
\pgfusepath{clip}%
\pgfsetbuttcap%
\pgfsetroundjoin%
\definecolor{currentfill}{rgb}{0.237441,0.305202,0.541921}%
\pgfsetfillcolor{currentfill}%
\pgfsetfillopacity{0.800000}%
\pgfsetlinewidth{0.000000pt}%
\definecolor{currentstroke}{rgb}{0.000000,0.000000,0.000000}%
\pgfsetstrokecolor{currentstroke}%
\pgfsetdash{}{0pt}%
\pgfpathmoveto{\pgfqpoint{4.452183in}{2.958427in}}%
\pgfpathlineto{\pgfqpoint{4.465643in}{2.954561in}}%
\pgfpathlineto{\pgfqpoint{4.479110in}{2.950890in}}%
\pgfpathlineto{\pgfqpoint{4.492584in}{2.947413in}}%
\pgfpathlineto{\pgfqpoint{4.506067in}{2.944130in}}%
\pgfpathlineto{\pgfqpoint{4.513691in}{2.958126in}}%
\pgfpathlineto{\pgfqpoint{4.521313in}{2.972318in}}%
\pgfpathlineto{\pgfqpoint{4.528934in}{2.986714in}}%
\pgfpathlineto{\pgfqpoint{4.536553in}{3.001319in}}%
\pgfpathlineto{\pgfqpoint{4.523081in}{3.005135in}}%
\pgfpathlineto{\pgfqpoint{4.509616in}{3.009144in}}%
\pgfpathlineto{\pgfqpoint{4.496159in}{3.013347in}}%
\pgfpathlineto{\pgfqpoint{4.482709in}{3.017745in}}%
\pgfpathlineto{\pgfqpoint{4.475080in}{3.002596in}}%
\pgfpathlineto{\pgfqpoint{4.467449in}{2.987664in}}%
\pgfpathlineto{\pgfqpoint{4.459817in}{2.972943in}}%
\pgfpathlineto{\pgfqpoint{4.452183in}{2.958427in}}%
\pgfpathclose%
\pgfusepath{fill}%
\end{pgfscope}%
\begin{pgfscope}%
\pgfpathrectangle{\pgfqpoint{1.150000in}{0.150000in}}{\pgfqpoint{5.700000in}{5.700000in}}%
\pgfusepath{clip}%
\pgfsetbuttcap%
\pgfsetroundjoin%
\definecolor{currentfill}{rgb}{0.233603,0.313828,0.543914}%
\pgfsetfillcolor{currentfill}%
\pgfsetfillopacity{0.800000}%
\pgfsetlinewidth{0.000000pt}%
\definecolor{currentstroke}{rgb}{0.000000,0.000000,0.000000}%
\pgfsetstrokecolor{currentstroke}%
\pgfsetdash{}{0pt}%
\pgfpathmoveto{\pgfqpoint{3.317821in}{3.008143in}}%
\pgfpathlineto{\pgfqpoint{3.331199in}{2.992930in}}%
\pgfpathlineto{\pgfqpoint{3.344573in}{2.977997in}}%
\pgfpathlineto{\pgfqpoint{3.357944in}{2.963342in}}%
\pgfpathlineto{\pgfqpoint{3.371311in}{2.948963in}}%
\pgfpathlineto{\pgfqpoint{3.379203in}{2.963095in}}%
\pgfpathlineto{\pgfqpoint{3.387089in}{2.977384in}}%
\pgfpathlineto{\pgfqpoint{3.394969in}{2.991833in}}%
\pgfpathlineto{\pgfqpoint{3.402843in}{3.006444in}}%
\pgfpathlineto{\pgfqpoint{3.389484in}{3.021077in}}%
\pgfpathlineto{\pgfqpoint{3.376121in}{3.035986in}}%
\pgfpathlineto{\pgfqpoint{3.362755in}{3.051173in}}%
\pgfpathlineto{\pgfqpoint{3.349386in}{3.066641in}}%
\pgfpathlineto{\pgfqpoint{3.341504in}{3.051763in}}%
\pgfpathlineto{\pgfqpoint{3.333615in}{3.037056in}}%
\pgfpathlineto{\pgfqpoint{3.325721in}{3.022517in}}%
\pgfpathlineto{\pgfqpoint{3.317821in}{3.008143in}}%
\pgfpathclose%
\pgfusepath{fill}%
\end{pgfscope}%
\begin{pgfscope}%
\pgfpathrectangle{\pgfqpoint{1.150000in}{0.150000in}}{\pgfqpoint{5.700000in}{5.700000in}}%
\pgfusepath{clip}%
\pgfsetbuttcap%
\pgfsetroundjoin%
\definecolor{currentfill}{rgb}{0.221989,0.339161,0.548752}%
\pgfsetfillcolor{currentfill}%
\pgfsetfillopacity{0.800000}%
\pgfsetlinewidth{0.000000pt}%
\definecolor{currentstroke}{rgb}{0.000000,0.000000,0.000000}%
\pgfsetstrokecolor{currentstroke}%
\pgfsetdash{}{0pt}%
\pgfpathmoveto{\pgfqpoint{3.264266in}{3.071846in}}%
\pgfpathlineto{\pgfqpoint{3.277662in}{3.055488in}}%
\pgfpathlineto{\pgfqpoint{3.291053in}{3.039420in}}%
\pgfpathlineto{\pgfqpoint{3.304439in}{3.023639in}}%
\pgfpathlineto{\pgfqpoint{3.317821in}{3.008143in}}%
\pgfpathlineto{\pgfqpoint{3.325721in}{3.022517in}}%
\pgfpathlineto{\pgfqpoint{3.333615in}{3.037056in}}%
\pgfpathlineto{\pgfqpoint{3.341504in}{3.051763in}}%
\pgfpathlineto{\pgfqpoint{3.349386in}{3.066641in}}%
\pgfpathlineto{\pgfqpoint{3.336012in}{3.082391in}}%
\pgfpathlineto{\pgfqpoint{3.322635in}{3.098427in}}%
\pgfpathlineto{\pgfqpoint{3.309252in}{3.114750in}}%
\pgfpathlineto{\pgfqpoint{3.295866in}{3.131363in}}%
\pgfpathlineto{\pgfqpoint{3.287975in}{3.116218in}}%
\pgfpathlineto{\pgfqpoint{3.280079in}{3.101252in}}%
\pgfpathlineto{\pgfqpoint{3.272176in}{3.086463in}}%
\pgfpathlineto{\pgfqpoint{3.264266in}{3.071846in}}%
\pgfpathclose%
\pgfusepath{fill}%
\end{pgfscope}%
\begin{pgfscope}%
\pgfpathrectangle{\pgfqpoint{1.150000in}{0.150000in}}{\pgfqpoint{5.700000in}{5.700000in}}%
\pgfusepath{clip}%
\pgfsetbuttcap%
\pgfsetroundjoin%
\definecolor{currentfill}{rgb}{0.263663,0.237631,0.518762}%
\pgfsetfillcolor{currentfill}%
\pgfsetfillopacity{0.800000}%
\pgfsetlinewidth{0.000000pt}%
\definecolor{currentstroke}{rgb}{0.000000,0.000000,0.000000}%
\pgfsetstrokecolor{currentstroke}%
\pgfsetdash{}{0pt}%
\pgfpathmoveto{\pgfqpoint{3.976715in}{2.803104in}}%
\pgfpathlineto{\pgfqpoint{3.990078in}{2.796503in}}%
\pgfpathlineto{\pgfqpoint{4.003446in}{2.790118in}}%
\pgfpathlineto{\pgfqpoint{4.016817in}{2.783948in}}%
\pgfpathlineto{\pgfqpoint{4.030193in}{2.777993in}}%
\pgfpathlineto{\pgfqpoint{4.037932in}{2.791398in}}%
\pgfpathlineto{\pgfqpoint{4.045667in}{2.804943in}}%
\pgfpathlineto{\pgfqpoint{4.053397in}{2.818630in}}%
\pgfpathlineto{\pgfqpoint{4.061124in}{2.832464in}}%
\pgfpathlineto{\pgfqpoint{4.047756in}{2.838795in}}%
\pgfpathlineto{\pgfqpoint{4.034392in}{2.845341in}}%
\pgfpathlineto{\pgfqpoint{4.021032in}{2.852102in}}%
\pgfpathlineto{\pgfqpoint{4.007676in}{2.859079in}}%
\pgfpathlineto{\pgfqpoint{3.999942in}{2.844857in}}%
\pgfpathlineto{\pgfqpoint{3.992204in}{2.830790in}}%
\pgfpathlineto{\pgfqpoint{3.984462in}{2.816874in}}%
\pgfpathlineto{\pgfqpoint{3.976715in}{2.803104in}}%
\pgfpathclose%
\pgfusepath{fill}%
\end{pgfscope}%
\begin{pgfscope}%
\pgfpathrectangle{\pgfqpoint{1.150000in}{0.150000in}}{\pgfqpoint{5.700000in}{5.700000in}}%
\pgfusepath{clip}%
\pgfsetbuttcap%
\pgfsetroundjoin%
\definecolor{currentfill}{rgb}{0.212395,0.359683,0.551710}%
\pgfsetfillcolor{currentfill}%
\pgfsetfillopacity{0.800000}%
\pgfsetlinewidth{0.000000pt}%
\definecolor{currentstroke}{rgb}{0.000000,0.000000,0.000000}%
\pgfsetstrokecolor{currentstroke}%
\pgfsetdash{}{0pt}%
\pgfpathmoveto{\pgfqpoint{4.705346in}{3.093576in}}%
\pgfpathlineto{\pgfqpoint{4.718868in}{3.090313in}}%
\pgfpathlineto{\pgfqpoint{4.732398in}{3.087238in}}%
\pgfpathlineto{\pgfqpoint{4.745937in}{3.084349in}}%
\pgfpathlineto{\pgfqpoint{4.759484in}{3.081646in}}%
\pgfpathlineto{\pgfqpoint{4.767061in}{3.096590in}}%
\pgfpathlineto{\pgfqpoint{4.774637in}{3.111787in}}%
\pgfpathlineto{\pgfqpoint{4.782214in}{3.127244in}}%
\pgfpathlineto{\pgfqpoint{4.789791in}{3.142970in}}%
\pgfpathlineto{\pgfqpoint{4.776256in}{3.146300in}}%
\pgfpathlineto{\pgfqpoint{4.762730in}{3.149815in}}%
\pgfpathlineto{\pgfqpoint{4.749212in}{3.153516in}}%
\pgfpathlineto{\pgfqpoint{4.735703in}{3.157405in}}%
\pgfpathlineto{\pgfqpoint{4.728113in}{3.141041in}}%
\pgfpathlineto{\pgfqpoint{4.720524in}{3.124953in}}%
\pgfpathlineto{\pgfqpoint{4.712935in}{3.109134in}}%
\pgfpathlineto{\pgfqpoint{4.705346in}{3.093576in}}%
\pgfpathclose%
\pgfusepath{fill}%
\end{pgfscope}%
\begin{pgfscope}%
\pgfpathrectangle{\pgfqpoint{1.150000in}{0.150000in}}{\pgfqpoint{5.700000in}{5.700000in}}%
\pgfusepath{clip}%
\pgfsetbuttcap%
\pgfsetroundjoin%
\definecolor{currentfill}{rgb}{0.244972,0.287675,0.537260}%
\pgfsetfillcolor{currentfill}%
\pgfsetfillopacity{0.800000}%
\pgfsetlinewidth{0.000000pt}%
\definecolor{currentstroke}{rgb}{0.000000,0.000000,0.000000}%
\pgfsetstrokecolor{currentstroke}%
\pgfsetdash{}{0pt}%
\pgfpathmoveto{\pgfqpoint{4.367818in}{2.917715in}}%
\pgfpathlineto{\pgfqpoint{4.381259in}{2.913563in}}%
\pgfpathlineto{\pgfqpoint{4.394706in}{2.909609in}}%
\pgfpathlineto{\pgfqpoint{4.408161in}{2.905851in}}%
\pgfpathlineto{\pgfqpoint{4.421624in}{2.902291in}}%
\pgfpathlineto{\pgfqpoint{4.429267in}{2.916047in}}%
\pgfpathlineto{\pgfqpoint{4.436908in}{2.929985in}}%
\pgfpathlineto{\pgfqpoint{4.444546in}{2.944110in}}%
\pgfpathlineto{\pgfqpoint{4.452183in}{2.958427in}}%
\pgfpathlineto{\pgfqpoint{4.438730in}{2.962489in}}%
\pgfpathlineto{\pgfqpoint{4.425285in}{2.966747in}}%
\pgfpathlineto{\pgfqpoint{4.411846in}{2.971202in}}%
\pgfpathlineto{\pgfqpoint{4.398414in}{2.975855in}}%
\pgfpathlineto{\pgfqpoint{4.390769in}{2.961025in}}%
\pgfpathlineto{\pgfqpoint{4.383121in}{2.946396in}}%
\pgfpathlineto{\pgfqpoint{4.375470in}{2.931961in}}%
\pgfpathlineto{\pgfqpoint{4.367818in}{2.917715in}}%
\pgfpathclose%
\pgfusepath{fill}%
\end{pgfscope}%
\begin{pgfscope}%
\pgfpathrectangle{\pgfqpoint{1.150000in}{0.150000in}}{\pgfqpoint{5.700000in}{5.700000in}}%
\pgfusepath{clip}%
\pgfsetbuttcap%
\pgfsetroundjoin%
\definecolor{currentfill}{rgb}{0.243113,0.292092,0.538516}%
\pgfsetfillcolor{currentfill}%
\pgfsetfillopacity{0.800000}%
\pgfsetlinewidth{0.000000pt}%
\definecolor{currentstroke}{rgb}{0.000000,0.000000,0.000000}%
\pgfsetstrokecolor{currentstroke}%
\pgfsetdash{}{0pt}%
\pgfpathmoveto{\pgfqpoint{3.371311in}{2.948963in}}%
\pgfpathlineto{\pgfqpoint{3.384676in}{2.934858in}}%
\pgfpathlineto{\pgfqpoint{3.398038in}{2.921024in}}%
\pgfpathlineto{\pgfqpoint{3.411397in}{2.907460in}}%
\pgfpathlineto{\pgfqpoint{3.424754in}{2.894163in}}%
\pgfpathlineto{\pgfqpoint{3.432637in}{2.908054in}}%
\pgfpathlineto{\pgfqpoint{3.440514in}{2.922093in}}%
\pgfpathlineto{\pgfqpoint{3.448386in}{2.936284in}}%
\pgfpathlineto{\pgfqpoint{3.456252in}{2.950631in}}%
\pgfpathlineto{\pgfqpoint{3.442904in}{2.964181in}}%
\pgfpathlineto{\pgfqpoint{3.429553in}{2.977998in}}%
\pgfpathlineto{\pgfqpoint{3.416199in}{2.992085in}}%
\pgfpathlineto{\pgfqpoint{3.402843in}{3.006444in}}%
\pgfpathlineto{\pgfqpoint{3.394969in}{2.991833in}}%
\pgfpathlineto{\pgfqpoint{3.387089in}{2.977384in}}%
\pgfpathlineto{\pgfqpoint{3.379203in}{2.963095in}}%
\pgfpathlineto{\pgfqpoint{3.371311in}{2.948963in}}%
\pgfpathclose%
\pgfusepath{fill}%
\end{pgfscope}%
\begin{pgfscope}%
\pgfpathrectangle{\pgfqpoint{1.150000in}{0.150000in}}{\pgfqpoint{5.700000in}{5.700000in}}%
\pgfusepath{clip}%
\pgfsetbuttcap%
\pgfsetroundjoin%
\definecolor{currentfill}{rgb}{0.210503,0.363727,0.552206}%
\pgfsetfillcolor{currentfill}%
\pgfsetfillopacity{0.800000}%
\pgfsetlinewidth{0.000000pt}%
\definecolor{currentstroke}{rgb}{0.000000,0.000000,0.000000}%
\pgfsetstrokecolor{currentstroke}%
\pgfsetdash{}{0pt}%
\pgfpathmoveto{\pgfqpoint{3.210633in}{3.140225in}}%
\pgfpathlineto{\pgfqpoint{3.224049in}{3.122683in}}%
\pgfpathlineto{\pgfqpoint{3.237461in}{3.105441in}}%
\pgfpathlineto{\pgfqpoint{3.250866in}{3.088496in}}%
\pgfpathlineto{\pgfqpoint{3.264266in}{3.071846in}}%
\pgfpathlineto{\pgfqpoint{3.272176in}{3.086463in}}%
\pgfpathlineto{\pgfqpoint{3.280079in}{3.101252in}}%
\pgfpathlineto{\pgfqpoint{3.287975in}{3.116218in}}%
\pgfpathlineto{\pgfqpoint{3.295866in}{3.131363in}}%
\pgfpathlineto{\pgfqpoint{3.282474in}{3.148269in}}%
\pgfpathlineto{\pgfqpoint{3.269077in}{3.165469in}}%
\pgfpathlineto{\pgfqpoint{3.255675in}{3.182968in}}%
\pgfpathlineto{\pgfqpoint{3.242267in}{3.200766in}}%
\pgfpathlineto{\pgfqpoint{3.234368in}{3.185353in}}%
\pgfpathlineto{\pgfqpoint{3.226463in}{3.170127in}}%
\pgfpathlineto{\pgfqpoint{3.218551in}{3.155086in}}%
\pgfpathlineto{\pgfqpoint{3.210633in}{3.140225in}}%
\pgfpathclose%
\pgfusepath{fill}%
\end{pgfscope}%
\begin{pgfscope}%
\pgfpathrectangle{\pgfqpoint{1.150000in}{0.150000in}}{\pgfqpoint{5.700000in}{5.700000in}}%
\pgfusepath{clip}%
\pgfsetbuttcap%
\pgfsetroundjoin%
\definecolor{currentfill}{rgb}{0.165117,0.467423,0.558141}%
\pgfsetfillcolor{currentfill}%
\pgfsetfillopacity{0.800000}%
\pgfsetlinewidth{0.000000pt}%
\definecolor{currentstroke}{rgb}{0.000000,0.000000,0.000000}%
\pgfsetstrokecolor{currentstroke}%
\pgfsetdash{}{0pt}%
\pgfpathmoveto{\pgfqpoint{3.080827in}{3.438781in}}%
\pgfpathlineto{\pgfqpoint{3.094325in}{3.417158in}}%
\pgfpathlineto{\pgfqpoint{3.107813in}{3.395870in}}%
\pgfpathlineto{\pgfqpoint{3.121293in}{3.374915in}}%
\pgfpathlineto{\pgfqpoint{3.134765in}{3.354289in}}%
\pgfpathlineto{\pgfqpoint{3.142674in}{3.370450in}}%
\pgfpathlineto{\pgfqpoint{3.150576in}{3.386820in}}%
\pgfpathlineto{\pgfqpoint{3.158472in}{3.403405in}}%
\pgfpathlineto{\pgfqpoint{3.166360in}{3.420208in}}%
\pgfpathlineto{\pgfqpoint{3.152897in}{3.441126in}}%
\pgfpathlineto{\pgfqpoint{3.139425in}{3.462374in}}%
\pgfpathlineto{\pgfqpoint{3.125945in}{3.483955in}}%
\pgfpathlineto{\pgfqpoint{3.112455in}{3.505872in}}%
\pgfpathlineto{\pgfqpoint{3.104559in}{3.488764in}}%
\pgfpathlineto{\pgfqpoint{3.096655in}{3.471882in}}%
\pgfpathlineto{\pgfqpoint{3.088744in}{3.455222in}}%
\pgfpathlineto{\pgfqpoint{3.080827in}{3.438781in}}%
\pgfpathclose%
\pgfusepath{fill}%
\end{pgfscope}%
\begin{pgfscope}%
\pgfpathrectangle{\pgfqpoint{1.150000in}{0.150000in}}{\pgfqpoint{5.700000in}{5.700000in}}%
\pgfusepath{clip}%
\pgfsetbuttcap%
\pgfsetroundjoin%
\definecolor{currentfill}{rgb}{0.204903,0.375746,0.553533}%
\pgfsetfillcolor{currentfill}%
\pgfsetfillopacity{0.800000}%
\pgfsetlinewidth{0.000000pt}%
\definecolor{currentstroke}{rgb}{0.000000,0.000000,0.000000}%
\pgfsetstrokecolor{currentstroke}%
\pgfsetdash{}{0pt}%
\pgfpathmoveto{\pgfqpoint{4.789791in}{3.142970in}}%
\pgfpathlineto{\pgfqpoint{4.803334in}{3.139826in}}%
\pgfpathlineto{\pgfqpoint{4.816887in}{3.136867in}}%
\pgfpathlineto{\pgfqpoint{4.830448in}{3.134092in}}%
\pgfpathlineto{\pgfqpoint{4.844018in}{3.131501in}}%
\pgfpathlineto{\pgfqpoint{4.851582in}{3.146856in}}%
\pgfpathlineto{\pgfqpoint{4.859148in}{3.162487in}}%
\pgfpathlineto{\pgfqpoint{4.866714in}{3.178401in}}%
\pgfpathlineto{\pgfqpoint{4.874282in}{3.194605in}}%
\pgfpathlineto{\pgfqpoint{4.860726in}{3.197854in}}%
\pgfpathlineto{\pgfqpoint{4.847178in}{3.201287in}}%
\pgfpathlineto{\pgfqpoint{4.833639in}{3.204904in}}%
\pgfpathlineto{\pgfqpoint{4.820109in}{3.208706in}}%
\pgfpathlineto{\pgfqpoint{4.812527in}{3.191832in}}%
\pgfpathlineto{\pgfqpoint{4.804948in}{3.175257in}}%
\pgfpathlineto{\pgfqpoint{4.797369in}{3.158972in}}%
\pgfpathlineto{\pgfqpoint{4.789791in}{3.142970in}}%
\pgfpathclose%
\pgfusepath{fill}%
\end{pgfscope}%
\begin{pgfscope}%
\pgfpathrectangle{\pgfqpoint{1.150000in}{0.150000in}}{\pgfqpoint{5.700000in}{5.700000in}}%
\pgfusepath{clip}%
\pgfsetbuttcap%
\pgfsetroundjoin%
\definecolor{currentfill}{rgb}{0.250425,0.274290,0.533103}%
\pgfsetfillcolor{currentfill}%
\pgfsetfillopacity{0.800000}%
\pgfsetlinewidth{0.000000pt}%
\definecolor{currentstroke}{rgb}{0.000000,0.000000,0.000000}%
\pgfsetstrokecolor{currentstroke}%
\pgfsetdash{}{0pt}%
\pgfpathmoveto{\pgfqpoint{4.283447in}{2.879230in}}%
\pgfpathlineto{\pgfqpoint{4.296870in}{2.874748in}}%
\pgfpathlineto{\pgfqpoint{4.310300in}{2.870467in}}%
\pgfpathlineto{\pgfqpoint{4.323737in}{2.866386in}}%
\pgfpathlineto{\pgfqpoint{4.337180in}{2.862505in}}%
\pgfpathlineto{\pgfqpoint{4.344843in}{2.876052in}}%
\pgfpathlineto{\pgfqpoint{4.352504in}{2.889766in}}%
\pgfpathlineto{\pgfqpoint{4.360162in}{2.903652in}}%
\pgfpathlineto{\pgfqpoint{4.367818in}{2.917715in}}%
\pgfpathlineto{\pgfqpoint{4.354383in}{2.922066in}}%
\pgfpathlineto{\pgfqpoint{4.340956in}{2.926616in}}%
\pgfpathlineto{\pgfqpoint{4.327535in}{2.931367in}}%
\pgfpathlineto{\pgfqpoint{4.314120in}{2.936318in}}%
\pgfpathlineto{\pgfqpoint{4.306456in}{2.921774in}}%
\pgfpathlineto{\pgfqpoint{4.298789in}{2.907415in}}%
\pgfpathlineto{\pgfqpoint{4.291119in}{2.893236in}}%
\pgfpathlineto{\pgfqpoint{4.283447in}{2.879230in}}%
\pgfpathclose%
\pgfusepath{fill}%
\end{pgfscope}%
\begin{pgfscope}%
\pgfpathrectangle{\pgfqpoint{1.150000in}{0.150000in}}{\pgfqpoint{5.700000in}{5.700000in}}%
\pgfusepath{clip}%
\pgfsetbuttcap%
\pgfsetroundjoin%
\definecolor{currentfill}{rgb}{0.266580,0.228262,0.514349}%
\pgfsetfillcolor{currentfill}%
\pgfsetfillopacity{0.800000}%
\pgfsetlinewidth{0.000000pt}%
\definecolor{currentstroke}{rgb}{0.000000,0.000000,0.000000}%
\pgfsetstrokecolor{currentstroke}%
\pgfsetdash{}{0pt}%
\pgfpathmoveto{\pgfqpoint{3.754325in}{2.786292in}}%
\pgfpathlineto{\pgfqpoint{3.767666in}{2.777663in}}%
\pgfpathlineto{\pgfqpoint{3.781008in}{2.769266in}}%
\pgfpathlineto{\pgfqpoint{3.794354in}{2.761097in}}%
\pgfpathlineto{\pgfqpoint{3.807701in}{2.753157in}}%
\pgfpathlineto{\pgfqpoint{3.815497in}{2.766513in}}%
\pgfpathlineto{\pgfqpoint{3.823288in}{2.779997in}}%
\pgfpathlineto{\pgfqpoint{3.831074in}{2.793614in}}%
\pgfpathlineto{\pgfqpoint{3.838856in}{2.807369in}}%
\pgfpathlineto{\pgfqpoint{3.825516in}{2.815623in}}%
\pgfpathlineto{\pgfqpoint{3.812178in}{2.824105in}}%
\pgfpathlineto{\pgfqpoint{3.798842in}{2.832816in}}%
\pgfpathlineto{\pgfqpoint{3.785509in}{2.841759in}}%
\pgfpathlineto{\pgfqpoint{3.777720in}{2.827679in}}%
\pgfpathlineto{\pgfqpoint{3.769926in}{2.813744in}}%
\pgfpathlineto{\pgfqpoint{3.762128in}{2.799949in}}%
\pgfpathlineto{\pgfqpoint{3.754325in}{2.786292in}}%
\pgfpathclose%
\pgfusepath{fill}%
\end{pgfscope}%
\begin{pgfscope}%
\pgfpathrectangle{\pgfqpoint{1.150000in}{0.150000in}}{\pgfqpoint{5.700000in}{5.700000in}}%
\pgfusepath{clip}%
\pgfsetbuttcap%
\pgfsetroundjoin%
\definecolor{currentfill}{rgb}{0.263663,0.237631,0.518762}%
\pgfsetfillcolor{currentfill}%
\pgfsetfillopacity{0.800000}%
\pgfsetlinewidth{0.000000pt}%
\definecolor{currentstroke}{rgb}{0.000000,0.000000,0.000000}%
\pgfsetstrokecolor{currentstroke}%
\pgfsetdash{}{0pt}%
\pgfpathmoveto{\pgfqpoint{3.616333in}{2.808199in}}%
\pgfpathlineto{\pgfqpoint{3.629671in}{2.797956in}}%
\pgfpathlineto{\pgfqpoint{3.643010in}{2.787956in}}%
\pgfpathlineto{\pgfqpoint{3.656349in}{2.778196in}}%
\pgfpathlineto{\pgfqpoint{3.669690in}{2.768676in}}%
\pgfpathlineto{\pgfqpoint{3.677519in}{2.782093in}}%
\pgfpathlineto{\pgfqpoint{3.685344in}{2.795641in}}%
\pgfpathlineto{\pgfqpoint{3.693164in}{2.809322in}}%
\pgfpathlineto{\pgfqpoint{3.700979in}{2.823141in}}%
\pgfpathlineto{\pgfqpoint{3.687646in}{2.832945in}}%
\pgfpathlineto{\pgfqpoint{3.674315in}{2.842987in}}%
\pgfpathlineto{\pgfqpoint{3.660983in}{2.853271in}}%
\pgfpathlineto{\pgfqpoint{3.647653in}{2.863797in}}%
\pgfpathlineto{\pgfqpoint{3.639831in}{2.849683in}}%
\pgfpathlineto{\pgfqpoint{3.632003in}{2.835715in}}%
\pgfpathlineto{\pgfqpoint{3.624171in}{2.821888in}}%
\pgfpathlineto{\pgfqpoint{3.616333in}{2.808199in}}%
\pgfpathclose%
\pgfusepath{fill}%
\end{pgfscope}%
\begin{pgfscope}%
\pgfpathrectangle{\pgfqpoint{1.150000in}{0.150000in}}{\pgfqpoint{5.700000in}{5.700000in}}%
\pgfusepath{clip}%
\pgfsetbuttcap%
\pgfsetroundjoin%
\definecolor{currentfill}{rgb}{0.250425,0.274290,0.533103}%
\pgfsetfillcolor{currentfill}%
\pgfsetfillopacity{0.800000}%
\pgfsetlinewidth{0.000000pt}%
\definecolor{currentstroke}{rgb}{0.000000,0.000000,0.000000}%
\pgfsetstrokecolor{currentstroke}%
\pgfsetdash{}{0pt}%
\pgfpathmoveto{\pgfqpoint{3.424754in}{2.894163in}}%
\pgfpathlineto{\pgfqpoint{3.438108in}{2.881132in}}%
\pgfpathlineto{\pgfqpoint{3.451461in}{2.868364in}}%
\pgfpathlineto{\pgfqpoint{3.464812in}{2.855857in}}%
\pgfpathlineto{\pgfqpoint{3.478162in}{2.843610in}}%
\pgfpathlineto{\pgfqpoint{3.486036in}{2.857260in}}%
\pgfpathlineto{\pgfqpoint{3.493906in}{2.871050in}}%
\pgfpathlineto{\pgfqpoint{3.501769in}{2.884984in}}%
\pgfpathlineto{\pgfqpoint{3.509628in}{2.899066in}}%
\pgfpathlineto{\pgfqpoint{3.496286in}{2.911566in}}%
\pgfpathlineto{\pgfqpoint{3.482943in}{2.924325in}}%
\pgfpathlineto{\pgfqpoint{3.469599in}{2.937346in}}%
\pgfpathlineto{\pgfqpoint{3.456252in}{2.950631in}}%
\pgfpathlineto{\pgfqpoint{3.448386in}{2.936284in}}%
\pgfpathlineto{\pgfqpoint{3.440514in}{2.922093in}}%
\pgfpathlineto{\pgfqpoint{3.432637in}{2.908054in}}%
\pgfpathlineto{\pgfqpoint{3.424754in}{2.894163in}}%
\pgfpathclose%
\pgfusepath{fill}%
\end{pgfscope}%
\begin{pgfscope}%
\pgfpathrectangle{\pgfqpoint{1.150000in}{0.150000in}}{\pgfqpoint{5.700000in}{5.700000in}}%
\pgfusepath{clip}%
\pgfsetbuttcap%
\pgfsetroundjoin%
\definecolor{currentfill}{rgb}{0.195860,0.395433,0.555276}%
\pgfsetfillcolor{currentfill}%
\pgfsetfillopacity{0.800000}%
\pgfsetlinewidth{0.000000pt}%
\definecolor{currentstroke}{rgb}{0.000000,0.000000,0.000000}%
\pgfsetstrokecolor{currentstroke}%
\pgfsetdash{}{0pt}%
\pgfpathmoveto{\pgfqpoint{4.874282in}{3.194605in}}%
\pgfpathlineto{\pgfqpoint{4.887848in}{3.191539in}}%
\pgfpathlineto{\pgfqpoint{4.901422in}{3.188655in}}%
\pgfpathlineto{\pgfqpoint{4.915006in}{3.185954in}}%
\pgfpathlineto{\pgfqpoint{4.928599in}{3.183434in}}%
\pgfpathlineto{\pgfqpoint{4.936154in}{3.199259in}}%
\pgfpathlineto{\pgfqpoint{4.943711in}{3.215383in}}%
\pgfpathlineto{\pgfqpoint{4.951270in}{3.231813in}}%
\pgfpathlineto{\pgfqpoint{4.958831in}{3.248558in}}%
\pgfpathlineto{\pgfqpoint{4.945253in}{3.251767in}}%
\pgfpathlineto{\pgfqpoint{4.931684in}{3.255158in}}%
\pgfpathlineto{\pgfqpoint{4.918123in}{3.258731in}}%
\pgfpathlineto{\pgfqpoint{4.904572in}{3.262486in}}%
\pgfpathlineto{\pgfqpoint{4.896996in}{3.245040in}}%
\pgfpathlineto{\pgfqpoint{4.889423in}{3.227917in}}%
\pgfpathlineto{\pgfqpoint{4.881852in}{3.211108in}}%
\pgfpathlineto{\pgfqpoint{4.874282in}{3.194605in}}%
\pgfpathclose%
\pgfusepath{fill}%
\end{pgfscope}%
\begin{pgfscope}%
\pgfpathrectangle{\pgfqpoint{1.150000in}{0.150000in}}{\pgfqpoint{5.700000in}{5.700000in}}%
\pgfusepath{clip}%
\pgfsetbuttcap%
\pgfsetroundjoin%
\definecolor{currentfill}{rgb}{0.197636,0.391528,0.554969}%
\pgfsetfillcolor{currentfill}%
\pgfsetfillopacity{0.800000}%
\pgfsetlinewidth{0.000000pt}%
\definecolor{currentstroke}{rgb}{0.000000,0.000000,0.000000}%
\pgfsetstrokecolor{currentstroke}%
\pgfsetdash{}{0pt}%
\pgfpathmoveto{\pgfqpoint{3.156903in}{3.213446in}}%
\pgfpathlineto{\pgfqpoint{3.170345in}{3.194677in}}%
\pgfpathlineto{\pgfqpoint{3.183781in}{3.176219in}}%
\pgfpathlineto{\pgfqpoint{3.197210in}{3.158069in}}%
\pgfpathlineto{\pgfqpoint{3.210633in}{3.140225in}}%
\pgfpathlineto{\pgfqpoint{3.218551in}{3.155086in}}%
\pgfpathlineto{\pgfqpoint{3.226463in}{3.170127in}}%
\pgfpathlineto{\pgfqpoint{3.234368in}{3.185353in}}%
\pgfpathlineto{\pgfqpoint{3.242267in}{3.200766in}}%
\pgfpathlineto{\pgfqpoint{3.228853in}{3.218867in}}%
\pgfpathlineto{\pgfqpoint{3.215432in}{3.237274in}}%
\pgfpathlineto{\pgfqpoint{3.202006in}{3.255989in}}%
\pgfpathlineto{\pgfqpoint{3.188572in}{3.275015in}}%
\pgfpathlineto{\pgfqpoint{3.180665in}{3.259333in}}%
\pgfpathlineto{\pgfqpoint{3.172751in}{3.243846in}}%
\pgfpathlineto{\pgfqpoint{3.164830in}{3.228551in}}%
\pgfpathlineto{\pgfqpoint{3.156903in}{3.213446in}}%
\pgfpathclose%
\pgfusepath{fill}%
\end{pgfscope}%
\begin{pgfscope}%
\pgfpathrectangle{\pgfqpoint{1.150000in}{0.150000in}}{\pgfqpoint{5.700000in}{5.700000in}}%
\pgfusepath{clip}%
\pgfsetbuttcap%
\pgfsetroundjoin%
\definecolor{currentfill}{rgb}{0.257322,0.256130,0.526563}%
\pgfsetfillcolor{currentfill}%
\pgfsetfillopacity{0.800000}%
\pgfsetlinewidth{0.000000pt}%
\definecolor{currentstroke}{rgb}{0.000000,0.000000,0.000000}%
\pgfsetstrokecolor{currentstroke}%
\pgfsetdash{}{0pt}%
\pgfpathmoveto{\pgfqpoint{4.199060in}{2.843045in}}%
\pgfpathlineto{\pgfqpoint{4.212467in}{2.838188in}}%
\pgfpathlineto{\pgfqpoint{4.225881in}{2.833536in}}%
\pgfpathlineto{\pgfqpoint{4.239300in}{2.829088in}}%
\pgfpathlineto{\pgfqpoint{4.252726in}{2.824842in}}%
\pgfpathlineto{\pgfqpoint{4.260411in}{2.838204in}}%
\pgfpathlineto{\pgfqpoint{4.268093in}{2.851720in}}%
\pgfpathlineto{\pgfqpoint{4.275771in}{2.865393in}}%
\pgfpathlineto{\pgfqpoint{4.283447in}{2.879230in}}%
\pgfpathlineto{\pgfqpoint{4.270030in}{2.883914in}}%
\pgfpathlineto{\pgfqpoint{4.256619in}{2.888801in}}%
\pgfpathlineto{\pgfqpoint{4.243214in}{2.893891in}}%
\pgfpathlineto{\pgfqpoint{4.229814in}{2.899186in}}%
\pgfpathlineto{\pgfqpoint{4.222130in}{2.884899in}}%
\pgfpathlineto{\pgfqpoint{4.214444in}{2.870784in}}%
\pgfpathlineto{\pgfqpoint{4.206753in}{2.856834in}}%
\pgfpathlineto{\pgfqpoint{4.199060in}{2.843045in}}%
\pgfpathclose%
\pgfusepath{fill}%
\end{pgfscope}%
\begin{pgfscope}%
\pgfpathrectangle{\pgfqpoint{1.150000in}{0.150000in}}{\pgfqpoint{5.700000in}{5.700000in}}%
\pgfusepath{clip}%
\pgfsetbuttcap%
\pgfsetroundjoin%
\definecolor{currentfill}{rgb}{0.266580,0.228262,0.514349}%
\pgfsetfillcolor{currentfill}%
\pgfsetfillopacity{0.800000}%
\pgfsetlinewidth{0.000000pt}%
\definecolor{currentstroke}{rgb}{0.000000,0.000000,0.000000}%
\pgfsetstrokecolor{currentstroke}%
\pgfsetdash{}{0pt}%
\pgfpathmoveto{\pgfqpoint{3.892246in}{2.776608in}}%
\pgfpathlineto{\pgfqpoint{3.905601in}{2.769476in}}%
\pgfpathlineto{\pgfqpoint{3.918960in}{2.762564in}}%
\pgfpathlineto{\pgfqpoint{3.932323in}{2.755872in}}%
\pgfpathlineto{\pgfqpoint{3.945689in}{2.749399in}}%
\pgfpathlineto{\pgfqpoint{3.953452in}{2.762627in}}%
\pgfpathlineto{\pgfqpoint{3.961211in}{2.775985in}}%
\pgfpathlineto{\pgfqpoint{3.968965in}{2.789476in}}%
\pgfpathlineto{\pgfqpoint{3.976715in}{2.803104in}}%
\pgfpathlineto{\pgfqpoint{3.963356in}{2.809922in}}%
\pgfpathlineto{\pgfqpoint{3.950001in}{2.816959in}}%
\pgfpathlineto{\pgfqpoint{3.936650in}{2.824216in}}%
\pgfpathlineto{\pgfqpoint{3.923301in}{2.831693in}}%
\pgfpathlineto{\pgfqpoint{3.915544in}{2.817708in}}%
\pgfpathlineto{\pgfqpoint{3.907782in}{2.803868in}}%
\pgfpathlineto{\pgfqpoint{3.900016in}{2.790170in}}%
\pgfpathlineto{\pgfqpoint{3.892246in}{2.776608in}}%
\pgfpathclose%
\pgfusepath{fill}%
\end{pgfscope}%
\begin{pgfscope}%
\pgfpathrectangle{\pgfqpoint{1.150000in}{0.150000in}}{\pgfqpoint{5.700000in}{5.700000in}}%
\pgfusepath{clip}%
\pgfsetbuttcap%
\pgfsetroundjoin%
\definecolor{currentfill}{rgb}{0.185556,0.418570,0.556753}%
\pgfsetfillcolor{currentfill}%
\pgfsetfillopacity{0.800000}%
\pgfsetlinewidth{0.000000pt}%
\definecolor{currentstroke}{rgb}{0.000000,0.000000,0.000000}%
\pgfsetstrokecolor{currentstroke}%
\pgfsetdash{}{0pt}%
\pgfpathmoveto{\pgfqpoint{4.958831in}{3.248558in}}%
\pgfpathlineto{\pgfqpoint{4.972419in}{3.245529in}}%
\pgfpathlineto{\pgfqpoint{4.986016in}{3.242682in}}%
\pgfpathlineto{\pgfqpoint{4.999623in}{3.240014in}}%
\pgfpathlineto{\pgfqpoint{5.013239in}{3.237526in}}%
\pgfpathlineto{\pgfqpoint{5.020788in}{3.253884in}}%
\pgfpathlineto{\pgfqpoint{5.028339in}{3.270565in}}%
\pgfpathlineto{\pgfqpoint{5.035894in}{3.287578in}}%
\pgfpathlineto{\pgfqpoint{5.043452in}{3.304931in}}%
\pgfpathlineto{\pgfqpoint{5.029851in}{3.308141in}}%
\pgfpathlineto{\pgfqpoint{5.016260in}{3.311529in}}%
\pgfpathlineto{\pgfqpoint{5.002679in}{3.315098in}}%
\pgfpathlineto{\pgfqpoint{4.989106in}{3.318848in}}%
\pgfpathlineto{\pgfqpoint{4.981532in}{3.300762in}}%
\pgfpathlineto{\pgfqpoint{4.973962in}{3.283024in}}%
\pgfpathlineto{\pgfqpoint{4.966395in}{3.265625in}}%
\pgfpathlineto{\pgfqpoint{4.958831in}{3.248558in}}%
\pgfpathclose%
\pgfusepath{fill}%
\end{pgfscope}%
\begin{pgfscope}%
\pgfpathrectangle{\pgfqpoint{1.150000in}{0.150000in}}{\pgfqpoint{5.700000in}{5.700000in}}%
\pgfusepath{clip}%
\pgfsetbuttcap%
\pgfsetroundjoin%
\definecolor{currentfill}{rgb}{0.262138,0.242286,0.520837}%
\pgfsetfillcolor{currentfill}%
\pgfsetfillopacity{0.800000}%
\pgfsetlinewidth{0.000000pt}%
\definecolor{currentstroke}{rgb}{0.000000,0.000000,0.000000}%
\pgfsetstrokecolor{currentstroke}%
\pgfsetdash{}{0pt}%
\pgfpathmoveto{\pgfqpoint{4.114646in}{2.809258in}}%
\pgfpathlineto{\pgfqpoint{4.128039in}{2.803981in}}%
\pgfpathlineto{\pgfqpoint{4.141437in}{2.798913in}}%
\pgfpathlineto{\pgfqpoint{4.154841in}{2.794052in}}%
\pgfpathlineto{\pgfqpoint{4.168251in}{2.789397in}}%
\pgfpathlineto{\pgfqpoint{4.175959in}{2.802593in}}%
\pgfpathlineto{\pgfqpoint{4.183663in}{2.815929in}}%
\pgfpathlineto{\pgfqpoint{4.191363in}{2.829411in}}%
\pgfpathlineto{\pgfqpoint{4.199060in}{2.843045in}}%
\pgfpathlineto{\pgfqpoint{4.185658in}{2.848107in}}%
\pgfpathlineto{\pgfqpoint{4.172262in}{2.853375in}}%
\pgfpathlineto{\pgfqpoint{4.158872in}{2.858850in}}%
\pgfpathlineto{\pgfqpoint{4.145486in}{2.864534in}}%
\pgfpathlineto{\pgfqpoint{4.137782in}{2.850482in}}%
\pgfpathlineto{\pgfqpoint{4.130073in}{2.836588in}}%
\pgfpathlineto{\pgfqpoint{4.122361in}{2.822849in}}%
\pgfpathlineto{\pgfqpoint{4.114646in}{2.809258in}}%
\pgfpathclose%
\pgfusepath{fill}%
\end{pgfscope}%
\begin{pgfscope}%
\pgfpathrectangle{\pgfqpoint{1.150000in}{0.150000in}}{\pgfqpoint{5.700000in}{5.700000in}}%
\pgfusepath{clip}%
\pgfsetbuttcap%
\pgfsetroundjoin%
\definecolor{currentfill}{rgb}{0.258965,0.251537,0.524736}%
\pgfsetfillcolor{currentfill}%
\pgfsetfillopacity{0.800000}%
\pgfsetlinewidth{0.000000pt}%
\definecolor{currentstroke}{rgb}{0.000000,0.000000,0.000000}%
\pgfsetstrokecolor{currentstroke}%
\pgfsetdash{}{0pt}%
\pgfpathmoveto{\pgfqpoint{3.478162in}{2.843610in}}%
\pgfpathlineto{\pgfqpoint{3.491510in}{2.831621in}}%
\pgfpathlineto{\pgfqpoint{3.504857in}{2.819887in}}%
\pgfpathlineto{\pgfqpoint{3.518204in}{2.808408in}}%
\pgfpathlineto{\pgfqpoint{3.531550in}{2.797181in}}%
\pgfpathlineto{\pgfqpoint{3.539416in}{2.810590in}}%
\pgfpathlineto{\pgfqpoint{3.547277in}{2.824132in}}%
\pgfpathlineto{\pgfqpoint{3.555133in}{2.837810in}}%
\pgfpathlineto{\pgfqpoint{3.562984in}{2.851627in}}%
\pgfpathlineto{\pgfqpoint{3.549646in}{2.863107in}}%
\pgfpathlineto{\pgfqpoint{3.536307in}{2.874838in}}%
\pgfpathlineto{\pgfqpoint{3.522968in}{2.886824in}}%
\pgfpathlineto{\pgfqpoint{3.509628in}{2.899066in}}%
\pgfpathlineto{\pgfqpoint{3.501769in}{2.884984in}}%
\pgfpathlineto{\pgfqpoint{3.493906in}{2.871050in}}%
\pgfpathlineto{\pgfqpoint{3.486036in}{2.857260in}}%
\pgfpathlineto{\pgfqpoint{3.478162in}{2.843610in}}%
\pgfpathclose%
\pgfusepath{fill}%
\end{pgfscope}%
\begin{pgfscope}%
\pgfpathrectangle{\pgfqpoint{1.150000in}{0.150000in}}{\pgfqpoint{5.700000in}{5.700000in}}%
\pgfusepath{clip}%
\pgfsetbuttcap%
\pgfsetroundjoin%
\definecolor{currentfill}{rgb}{0.993248,0.906157,0.143936}%
\pgfsetfillcolor{currentfill}%
\pgfsetfillopacity{0.800000}%
\pgfsetlinewidth{0.000000pt}%
\definecolor{currentstroke}{rgb}{0.000000,0.000000,0.000000}%
\pgfsetstrokecolor{currentstroke}%
\pgfsetdash{}{0pt}%
\pgfpathmoveto{\pgfqpoint{3.490972in}{5.105273in}}%
\pgfpathlineto{\pgfqpoint{3.504521in}{5.074000in}}%
\pgfpathlineto{\pgfqpoint{3.518057in}{5.043097in}}%
\pgfpathlineto{\pgfqpoint{3.531583in}{5.012560in}}%
\pgfpathlineto{\pgfqpoint{3.545098in}{4.982386in}}%
\pgfpathlineto{\pgfqpoint{3.552698in}{5.021704in}}%
\pgfpathlineto{\pgfqpoint{3.560296in}{5.061641in}}%
\pgfpathlineto{\pgfqpoint{3.567892in}{5.102207in}}%
\pgfpathlineto{\pgfqpoint{3.554367in}{5.133030in}}%
\pgfpathlineto{\pgfqpoint{3.540832in}{5.164217in}}%
\pgfpathlineto{\pgfqpoint{3.527286in}{5.195773in}}%
\pgfpathlineto{\pgfqpoint{3.513728in}{5.227701in}}%
\pgfpathlineto{\pgfqpoint{3.506145in}{5.186254in}}%
\pgfpathlineto{\pgfqpoint{3.498560in}{5.145448in}}%
\pgfpathlineto{\pgfqpoint{3.490972in}{5.105273in}}%
\pgfpathclose%
\pgfusepath{fill}%
\end{pgfscope}%
\begin{pgfscope}%
\pgfpathrectangle{\pgfqpoint{1.150000in}{0.150000in}}{\pgfqpoint{5.700000in}{5.700000in}}%
\pgfusepath{clip}%
\pgfsetbuttcap%
\pgfsetroundjoin%
\definecolor{currentfill}{rgb}{0.185556,0.418570,0.556753}%
\pgfsetfillcolor{currentfill}%
\pgfsetfillopacity{0.800000}%
\pgfsetlinewidth{0.000000pt}%
\definecolor{currentstroke}{rgb}{0.000000,0.000000,0.000000}%
\pgfsetstrokecolor{currentstroke}%
\pgfsetdash{}{0pt}%
\pgfpathmoveto{\pgfqpoint{3.103059in}{3.291687in}}%
\pgfpathlineto{\pgfqpoint{3.116531in}{3.271646in}}%
\pgfpathlineto{\pgfqpoint{3.129996in}{3.251927in}}%
\pgfpathlineto{\pgfqpoint{3.143453in}{3.232528in}}%
\pgfpathlineto{\pgfqpoint{3.156903in}{3.213446in}}%
\pgfpathlineto{\pgfqpoint{3.164830in}{3.228551in}}%
\pgfpathlineto{\pgfqpoint{3.172751in}{3.243846in}}%
\pgfpathlineto{\pgfqpoint{3.180665in}{3.259333in}}%
\pgfpathlineto{\pgfqpoint{3.188572in}{3.275015in}}%
\pgfpathlineto{\pgfqpoint{3.175131in}{3.294355in}}%
\pgfpathlineto{\pgfqpoint{3.161684in}{3.314012in}}%
\pgfpathlineto{\pgfqpoint{3.148228in}{3.333989in}}%
\pgfpathlineto{\pgfqpoint{3.134765in}{3.354289in}}%
\pgfpathlineto{\pgfqpoint{3.126849in}{3.338336in}}%
\pgfpathlineto{\pgfqpoint{3.118926in}{3.322587in}}%
\pgfpathlineto{\pgfqpoint{3.110996in}{3.307038in}}%
\pgfpathlineto{\pgfqpoint{3.103059in}{3.291687in}}%
\pgfpathclose%
\pgfusepath{fill}%
\end{pgfscope}%
\begin{pgfscope}%
\pgfpathrectangle{\pgfqpoint{1.150000in}{0.150000in}}{\pgfqpoint{5.700000in}{5.700000in}}%
\pgfusepath{clip}%
\pgfsetbuttcap%
\pgfsetroundjoin%
\definecolor{currentfill}{rgb}{0.151918,0.500685,0.557587}%
\pgfsetfillcolor{currentfill}%
\pgfsetfillopacity{0.800000}%
\pgfsetlinewidth{0.000000pt}%
\definecolor{currentstroke}{rgb}{0.000000,0.000000,0.000000}%
\pgfsetstrokecolor{currentstroke}%
\pgfsetdash{}{0pt}%
\pgfpathmoveto{\pgfqpoint{3.026738in}{3.528697in}}%
\pgfpathlineto{\pgfqpoint{3.040276in}{3.505698in}}%
\pgfpathlineto{\pgfqpoint{3.053803in}{3.483047in}}%
\pgfpathlineto{\pgfqpoint{3.067319in}{3.460743in}}%
\pgfpathlineto{\pgfqpoint{3.080827in}{3.438781in}}%
\pgfpathlineto{\pgfqpoint{3.088744in}{3.455222in}}%
\pgfpathlineto{\pgfqpoint{3.096655in}{3.471882in}}%
\pgfpathlineto{\pgfqpoint{3.104559in}{3.488764in}}%
\pgfpathlineto{\pgfqpoint{3.112455in}{3.505872in}}%
\pgfpathlineto{\pgfqpoint{3.098956in}{3.528128in}}%
\pgfpathlineto{\pgfqpoint{3.085448in}{3.550728in}}%
\pgfpathlineto{\pgfqpoint{3.071929in}{3.573673in}}%
\pgfpathlineto{\pgfqpoint{3.058401in}{3.596968in}}%
\pgfpathlineto{\pgfqpoint{3.050496in}{3.579552in}}%
\pgfpathlineto{\pgfqpoint{3.042584in}{3.562371in}}%
\pgfpathlineto{\pgfqpoint{3.034665in}{3.545420in}}%
\pgfpathlineto{\pgfqpoint{3.026738in}{3.528697in}}%
\pgfpathclose%
\pgfusepath{fill}%
\end{pgfscope}%
\begin{pgfscope}%
\pgfpathrectangle{\pgfqpoint{1.150000in}{0.150000in}}{\pgfqpoint{5.700000in}{5.700000in}}%
\pgfusepath{clip}%
\pgfsetbuttcap%
\pgfsetroundjoin%
\definecolor{currentfill}{rgb}{0.267968,0.223549,0.512008}%
\pgfsetfillcolor{currentfill}%
\pgfsetfillopacity{0.800000}%
\pgfsetlinewidth{0.000000pt}%
\definecolor{currentstroke}{rgb}{0.000000,0.000000,0.000000}%
\pgfsetstrokecolor{currentstroke}%
\pgfsetdash{}{0pt}%
\pgfpathmoveto{\pgfqpoint{3.669690in}{2.768676in}}%
\pgfpathlineto{\pgfqpoint{3.683031in}{2.759393in}}%
\pgfpathlineto{\pgfqpoint{3.696374in}{2.750347in}}%
\pgfpathlineto{\pgfqpoint{3.709719in}{2.741536in}}%
\pgfpathlineto{\pgfqpoint{3.723065in}{2.732958in}}%
\pgfpathlineto{\pgfqpoint{3.730887in}{2.746104in}}%
\pgfpathlineto{\pgfqpoint{3.738705in}{2.759373in}}%
\pgfpathlineto{\pgfqpoint{3.746517in}{2.772767in}}%
\pgfpathlineto{\pgfqpoint{3.754325in}{2.786292in}}%
\pgfpathlineto{\pgfqpoint{3.740986in}{2.795152in}}%
\pgfpathlineto{\pgfqpoint{3.727649in}{2.804247in}}%
\pgfpathlineto{\pgfqpoint{3.714313in}{2.813576in}}%
\pgfpathlineto{\pgfqpoint{3.700979in}{2.823141in}}%
\pgfpathlineto{\pgfqpoint{3.693164in}{2.809322in}}%
\pgfpathlineto{\pgfqpoint{3.685344in}{2.795641in}}%
\pgfpathlineto{\pgfqpoint{3.677519in}{2.782093in}}%
\pgfpathlineto{\pgfqpoint{3.669690in}{2.768676in}}%
\pgfpathclose%
\pgfusepath{fill}%
\end{pgfscope}%
\begin{pgfscope}%
\pgfpathrectangle{\pgfqpoint{1.150000in}{0.150000in}}{\pgfqpoint{5.700000in}{5.700000in}}%
\pgfusepath{clip}%
\pgfsetbuttcap%
\pgfsetroundjoin%
\definecolor{currentfill}{rgb}{0.269308,0.218818,0.509577}%
\pgfsetfillcolor{currentfill}%
\pgfsetfillopacity{0.800000}%
\pgfsetlinewidth{0.000000pt}%
\definecolor{currentstroke}{rgb}{0.000000,0.000000,0.000000}%
\pgfsetstrokecolor{currentstroke}%
\pgfsetdash{}{0pt}%
\pgfpathmoveto{\pgfqpoint{3.807701in}{2.753157in}}%
\pgfpathlineto{\pgfqpoint{3.821052in}{2.745444in}}%
\pgfpathlineto{\pgfqpoint{3.834405in}{2.737956in}}%
\pgfpathlineto{\pgfqpoint{3.847761in}{2.730693in}}%
\pgfpathlineto{\pgfqpoint{3.861121in}{2.723652in}}%
\pgfpathlineto{\pgfqpoint{3.868909in}{2.736705in}}%
\pgfpathlineto{\pgfqpoint{3.876692in}{2.749880in}}%
\pgfpathlineto{\pgfqpoint{3.884471in}{2.763180in}}%
\pgfpathlineto{\pgfqpoint{3.892246in}{2.776608in}}%
\pgfpathlineto{\pgfqpoint{3.878894in}{2.783963in}}%
\pgfpathlineto{\pgfqpoint{3.865545in}{2.791540in}}%
\pgfpathlineto{\pgfqpoint{3.852199in}{2.799341in}}%
\pgfpathlineto{\pgfqpoint{3.838856in}{2.807369in}}%
\pgfpathlineto{\pgfqpoint{3.831074in}{2.793614in}}%
\pgfpathlineto{\pgfqpoint{3.823288in}{2.779997in}}%
\pgfpathlineto{\pgfqpoint{3.815497in}{2.766513in}}%
\pgfpathlineto{\pgfqpoint{3.807701in}{2.753157in}}%
\pgfpathclose%
\pgfusepath{fill}%
\end{pgfscope}%
\begin{pgfscope}%
\pgfpathrectangle{\pgfqpoint{1.150000in}{0.150000in}}{\pgfqpoint{5.700000in}{5.700000in}}%
\pgfusepath{clip}%
\pgfsetbuttcap%
\pgfsetroundjoin%
\definecolor{currentfill}{rgb}{0.265145,0.232956,0.516599}%
\pgfsetfillcolor{currentfill}%
\pgfsetfillopacity{0.800000}%
\pgfsetlinewidth{0.000000pt}%
\definecolor{currentstroke}{rgb}{0.000000,0.000000,0.000000}%
\pgfsetstrokecolor{currentstroke}%
\pgfsetdash{}{0pt}%
\pgfpathmoveto{\pgfqpoint{4.030193in}{2.777993in}}%
\pgfpathlineto{\pgfqpoint{4.043574in}{2.772250in}}%
\pgfpathlineto{\pgfqpoint{4.056959in}{2.766719in}}%
\pgfpathlineto{\pgfqpoint{4.070350in}{2.761400in}}%
\pgfpathlineto{\pgfqpoint{4.083745in}{2.756291in}}%
\pgfpathlineto{\pgfqpoint{4.091476in}{2.769332in}}%
\pgfpathlineto{\pgfqpoint{4.099203in}{2.782504in}}%
\pgfpathlineto{\pgfqpoint{4.106927in}{2.795811in}}%
\pgfpathlineto{\pgfqpoint{4.114646in}{2.809258in}}%
\pgfpathlineto{\pgfqpoint{4.101258in}{2.814743in}}%
\pgfpathlineto{\pgfqpoint{4.087875in}{2.820439in}}%
\pgfpathlineto{\pgfqpoint{4.074498in}{2.826345in}}%
\pgfpathlineto{\pgfqpoint{4.061124in}{2.832464in}}%
\pgfpathlineto{\pgfqpoint{4.053397in}{2.818630in}}%
\pgfpathlineto{\pgfqpoint{4.045667in}{2.804943in}}%
\pgfpathlineto{\pgfqpoint{4.037932in}{2.791398in}}%
\pgfpathlineto{\pgfqpoint{4.030193in}{2.777993in}}%
\pgfpathclose%
\pgfusepath{fill}%
\end{pgfscope}%
\begin{pgfscope}%
\pgfpathrectangle{\pgfqpoint{1.150000in}{0.150000in}}{\pgfqpoint{5.700000in}{5.700000in}}%
\pgfusepath{clip}%
\pgfsetbuttcap%
\pgfsetroundjoin%
\definecolor{currentfill}{rgb}{0.177423,0.437527,0.557565}%
\pgfsetfillcolor{currentfill}%
\pgfsetfillopacity{0.800000}%
\pgfsetlinewidth{0.000000pt}%
\definecolor{currentstroke}{rgb}{0.000000,0.000000,0.000000}%
\pgfsetstrokecolor{currentstroke}%
\pgfsetdash{}{0pt}%
\pgfpathmoveto{\pgfqpoint{5.043452in}{3.304931in}}%
\pgfpathlineto{\pgfqpoint{5.057062in}{3.301901in}}%
\pgfpathlineto{\pgfqpoint{5.070681in}{3.299050in}}%
\pgfpathlineto{\pgfqpoint{5.084311in}{3.296377in}}%
\pgfpathlineto{\pgfqpoint{5.097950in}{3.293881in}}%
\pgfpathlineto{\pgfqpoint{5.105495in}{3.310842in}}%
\pgfpathlineto{\pgfqpoint{5.113044in}{3.328152in}}%
\pgfpathlineto{\pgfqpoint{5.120598in}{3.345820in}}%
\pgfpathlineto{\pgfqpoint{5.106972in}{3.348877in}}%
\pgfpathlineto{\pgfqpoint{5.093355in}{3.352112in}}%
\pgfpathlineto{\pgfqpoint{5.079748in}{3.355525in}}%
\pgfpathlineto{\pgfqpoint{5.066150in}{3.359117in}}%
\pgfpathlineto{\pgfqpoint{5.058579in}{3.340692in}}%
\pgfpathlineto{\pgfqpoint{5.051014in}{3.322633in}}%
\pgfpathlineto{\pgfqpoint{5.043452in}{3.304931in}}%
\pgfpathclose%
\pgfusepath{fill}%
\end{pgfscope}%
\begin{pgfscope}%
\pgfpathrectangle{\pgfqpoint{1.150000in}{0.150000in}}{\pgfqpoint{5.700000in}{5.700000in}}%
\pgfusepath{clip}%
\pgfsetbuttcap%
\pgfsetroundjoin%
\definecolor{currentfill}{rgb}{0.263663,0.237631,0.518762}%
\pgfsetfillcolor{currentfill}%
\pgfsetfillopacity{0.800000}%
\pgfsetlinewidth{0.000000pt}%
\definecolor{currentstroke}{rgb}{0.000000,0.000000,0.000000}%
\pgfsetstrokecolor{currentstroke}%
\pgfsetdash{}{0pt}%
\pgfpathmoveto{\pgfqpoint{3.531550in}{2.797181in}}%
\pgfpathlineto{\pgfqpoint{3.544895in}{2.786204in}}%
\pgfpathlineto{\pgfqpoint{3.558241in}{2.775477in}}%
\pgfpathlineto{\pgfqpoint{3.571586in}{2.764996in}}%
\pgfpathlineto{\pgfqpoint{3.584932in}{2.754761in}}%
\pgfpathlineto{\pgfqpoint{3.592790in}{2.767930in}}%
\pgfpathlineto{\pgfqpoint{3.600643in}{2.781223in}}%
\pgfpathlineto{\pgfqpoint{3.608491in}{2.794646in}}%
\pgfpathlineto{\pgfqpoint{3.616333in}{2.808199in}}%
\pgfpathlineto{\pgfqpoint{3.602996in}{2.818686in}}%
\pgfpathlineto{\pgfqpoint{3.589658in}{2.829419in}}%
\pgfpathlineto{\pgfqpoint{3.576321in}{2.840399in}}%
\pgfpathlineto{\pgfqpoint{3.562984in}{2.851627in}}%
\pgfpathlineto{\pgfqpoint{3.555133in}{2.837810in}}%
\pgfpathlineto{\pgfqpoint{3.547277in}{2.824132in}}%
\pgfpathlineto{\pgfqpoint{3.539416in}{2.810590in}}%
\pgfpathlineto{\pgfqpoint{3.531550in}{2.797181in}}%
\pgfpathclose%
\pgfusepath{fill}%
\end{pgfscope}%
\begin{pgfscope}%
\pgfpathrectangle{\pgfqpoint{1.150000in}{0.150000in}}{\pgfqpoint{5.700000in}{5.700000in}}%
\pgfusepath{clip}%
\pgfsetbuttcap%
\pgfsetroundjoin%
\definecolor{currentfill}{rgb}{0.229739,0.322361,0.545706}%
\pgfsetfillcolor{currentfill}%
\pgfsetfillopacity{0.800000}%
\pgfsetlinewidth{0.000000pt}%
\definecolor{currentstroke}{rgb}{0.000000,0.000000,0.000000}%
\pgfsetstrokecolor{currentstroke}%
\pgfsetdash{}{0pt}%
\pgfpathmoveto{\pgfqpoint{4.590518in}{2.987978in}}%
\pgfpathlineto{\pgfqpoint{4.604030in}{2.985120in}}%
\pgfpathlineto{\pgfqpoint{4.617549in}{2.982452in}}%
\pgfpathlineto{\pgfqpoint{4.631078in}{2.979973in}}%
\pgfpathlineto{\pgfqpoint{4.644615in}{2.977682in}}%
\pgfpathlineto{\pgfqpoint{4.652210in}{2.991398in}}%
\pgfpathlineto{\pgfqpoint{4.659804in}{3.005322in}}%
\pgfpathlineto{\pgfqpoint{4.667396in}{3.019459in}}%
\pgfpathlineto{\pgfqpoint{4.674988in}{3.033815in}}%
\pgfpathlineto{\pgfqpoint{4.661462in}{3.036670in}}%
\pgfpathlineto{\pgfqpoint{4.647946in}{3.039714in}}%
\pgfpathlineto{\pgfqpoint{4.634437in}{3.042946in}}%
\pgfpathlineto{\pgfqpoint{4.620937in}{3.046368in}}%
\pgfpathlineto{\pgfqpoint{4.613334in}{3.031436in}}%
\pgfpathlineto{\pgfqpoint{4.605730in}{3.016731in}}%
\pgfpathlineto{\pgfqpoint{4.598125in}{3.002247in}}%
\pgfpathlineto{\pgfqpoint{4.590518in}{2.987978in}}%
\pgfpathclose%
\pgfusepath{fill}%
\end{pgfscope}%
\begin{pgfscope}%
\pgfpathrectangle{\pgfqpoint{1.150000in}{0.150000in}}{\pgfqpoint{5.700000in}{5.700000in}}%
\pgfusepath{clip}%
\pgfsetbuttcap%
\pgfsetroundjoin%
\definecolor{currentfill}{rgb}{0.237441,0.305202,0.541921}%
\pgfsetfillcolor{currentfill}%
\pgfsetfillopacity{0.800000}%
\pgfsetlinewidth{0.000000pt}%
\definecolor{currentstroke}{rgb}{0.000000,0.000000,0.000000}%
\pgfsetstrokecolor{currentstroke}%
\pgfsetdash{}{0pt}%
\pgfpathmoveto{\pgfqpoint{4.506067in}{2.944130in}}%
\pgfpathlineto{\pgfqpoint{4.519557in}{2.941040in}}%
\pgfpathlineto{\pgfqpoint{4.533055in}{2.938141in}}%
\pgfpathlineto{\pgfqpoint{4.546561in}{2.935435in}}%
\pgfpathlineto{\pgfqpoint{4.560075in}{2.932919in}}%
\pgfpathlineto{\pgfqpoint{4.567688in}{2.946394in}}%
\pgfpathlineto{\pgfqpoint{4.575300in}{2.960058in}}%
\pgfpathlineto{\pgfqpoint{4.582910in}{2.973917in}}%
\pgfpathlineto{\pgfqpoint{4.590518in}{2.987978in}}%
\pgfpathlineto{\pgfqpoint{4.577015in}{2.991027in}}%
\pgfpathlineto{\pgfqpoint{4.563520in}{2.994266in}}%
\pgfpathlineto{\pgfqpoint{4.550032in}{2.997696in}}%
\pgfpathlineto{\pgfqpoint{4.536553in}{3.001319in}}%
\pgfpathlineto{\pgfqpoint{4.528934in}{2.986714in}}%
\pgfpathlineto{\pgfqpoint{4.521313in}{2.972318in}}%
\pgfpathlineto{\pgfqpoint{4.513691in}{2.958126in}}%
\pgfpathlineto{\pgfqpoint{4.506067in}{2.944130in}}%
\pgfpathclose%
\pgfusepath{fill}%
\end{pgfscope}%
\begin{pgfscope}%
\pgfpathrectangle{\pgfqpoint{1.150000in}{0.150000in}}{\pgfqpoint{5.700000in}{5.700000in}}%
\pgfusepath{clip}%
\pgfsetbuttcap%
\pgfsetroundjoin%
\definecolor{currentfill}{rgb}{0.221989,0.339161,0.548752}%
\pgfsetfillcolor{currentfill}%
\pgfsetfillopacity{0.800000}%
\pgfsetlinewidth{0.000000pt}%
\definecolor{currentstroke}{rgb}{0.000000,0.000000,0.000000}%
\pgfsetstrokecolor{currentstroke}%
\pgfsetdash{}{0pt}%
\pgfpathmoveto{\pgfqpoint{4.674988in}{3.033815in}}%
\pgfpathlineto{\pgfqpoint{4.688521in}{3.031148in}}%
\pgfpathlineto{\pgfqpoint{4.702064in}{3.028668in}}%
\pgfpathlineto{\pgfqpoint{4.715615in}{3.026375in}}%
\pgfpathlineto{\pgfqpoint{4.729175in}{3.024268in}}%
\pgfpathlineto{\pgfqpoint{4.736753in}{3.038267in}}%
\pgfpathlineto{\pgfqpoint{4.744331in}{3.052492in}}%
\pgfpathlineto{\pgfqpoint{4.751908in}{3.066949in}}%
\pgfpathlineto{\pgfqpoint{4.759484in}{3.081646in}}%
\pgfpathlineto{\pgfqpoint{4.745937in}{3.084349in}}%
\pgfpathlineto{\pgfqpoint{4.732398in}{3.087238in}}%
\pgfpathlineto{\pgfqpoint{4.718868in}{3.090313in}}%
\pgfpathlineto{\pgfqpoint{4.705346in}{3.093576in}}%
\pgfpathlineto{\pgfqpoint{4.697757in}{3.078272in}}%
\pgfpathlineto{\pgfqpoint{4.690168in}{3.063215in}}%
\pgfpathlineto{\pgfqpoint{4.682578in}{3.048399in}}%
\pgfpathlineto{\pgfqpoint{4.674988in}{3.033815in}}%
\pgfpathclose%
\pgfusepath{fill}%
\end{pgfscope}%
\begin{pgfscope}%
\pgfpathrectangle{\pgfqpoint{1.150000in}{0.150000in}}{\pgfqpoint{5.700000in}{5.700000in}}%
\pgfusepath{clip}%
\pgfsetbuttcap%
\pgfsetroundjoin%
\definecolor{currentfill}{rgb}{0.171176,0.452530,0.557965}%
\pgfsetfillcolor{currentfill}%
\pgfsetfillopacity{0.800000}%
\pgfsetlinewidth{0.000000pt}%
\definecolor{currentstroke}{rgb}{0.000000,0.000000,0.000000}%
\pgfsetstrokecolor{currentstroke}%
\pgfsetdash{}{0pt}%
\pgfpathmoveto{\pgfqpoint{3.049083in}{3.375139in}}%
\pgfpathlineto{\pgfqpoint{3.062591in}{3.353777in}}%
\pgfpathlineto{\pgfqpoint{3.076089in}{3.332749in}}%
\pgfpathlineto{\pgfqpoint{3.089578in}{3.312053in}}%
\pgfpathlineto{\pgfqpoint{3.103059in}{3.291687in}}%
\pgfpathlineto{\pgfqpoint{3.110996in}{3.307038in}}%
\pgfpathlineto{\pgfqpoint{3.118926in}{3.322587in}}%
\pgfpathlineto{\pgfqpoint{3.126849in}{3.338336in}}%
\pgfpathlineto{\pgfqpoint{3.134765in}{3.354289in}}%
\pgfpathlineto{\pgfqpoint{3.121293in}{3.374915in}}%
\pgfpathlineto{\pgfqpoint{3.107813in}{3.395870in}}%
\pgfpathlineto{\pgfqpoint{3.094325in}{3.417158in}}%
\pgfpathlineto{\pgfqpoint{3.080827in}{3.438781in}}%
\pgfpathlineto{\pgfqpoint{3.072902in}{3.422556in}}%
\pgfpathlineto{\pgfqpoint{3.064970in}{3.406543in}}%
\pgfpathlineto{\pgfqpoint{3.057030in}{3.390738in}}%
\pgfpathlineto{\pgfqpoint{3.049083in}{3.375139in}}%
\pgfpathclose%
\pgfusepath{fill}%
\end{pgfscope}%
\begin{pgfscope}%
\pgfpathrectangle{\pgfqpoint{1.150000in}{0.150000in}}{\pgfqpoint{5.700000in}{5.700000in}}%
\pgfusepath{clip}%
\pgfsetbuttcap%
\pgfsetroundjoin%
\definecolor{currentfill}{rgb}{0.244972,0.287675,0.537260}%
\pgfsetfillcolor{currentfill}%
\pgfsetfillopacity{0.800000}%
\pgfsetlinewidth{0.000000pt}%
\definecolor{currentstroke}{rgb}{0.000000,0.000000,0.000000}%
\pgfsetstrokecolor{currentstroke}%
\pgfsetdash{}{0pt}%
\pgfpathmoveto{\pgfqpoint{4.421624in}{2.902291in}}%
\pgfpathlineto{\pgfqpoint{4.435093in}{2.898925in}}%
\pgfpathlineto{\pgfqpoint{4.448570in}{2.895755in}}%
\pgfpathlineto{\pgfqpoint{4.462055in}{2.892780in}}%
\pgfpathlineto{\pgfqpoint{4.475547in}{2.889998in}}%
\pgfpathlineto{\pgfqpoint{4.483181in}{2.903265in}}%
\pgfpathlineto{\pgfqpoint{4.490812in}{2.916706in}}%
\pgfpathlineto{\pgfqpoint{4.498440in}{2.930325in}}%
\pgfpathlineto{\pgfqpoint{4.506067in}{2.944130in}}%
\pgfpathlineto{\pgfqpoint{4.492584in}{2.947413in}}%
\pgfpathlineto{\pgfqpoint{4.479110in}{2.950890in}}%
\pgfpathlineto{\pgfqpoint{4.465643in}{2.954561in}}%
\pgfpathlineto{\pgfqpoint{4.452183in}{2.958427in}}%
\pgfpathlineto{\pgfqpoint{4.444546in}{2.944110in}}%
\pgfpathlineto{\pgfqpoint{4.436908in}{2.929985in}}%
\pgfpathlineto{\pgfqpoint{4.429267in}{2.916047in}}%
\pgfpathlineto{\pgfqpoint{4.421624in}{2.902291in}}%
\pgfpathclose%
\pgfusepath{fill}%
\end{pgfscope}%
\begin{pgfscope}%
\pgfpathrectangle{\pgfqpoint{1.150000in}{0.150000in}}{\pgfqpoint{5.700000in}{5.700000in}}%
\pgfusepath{clip}%
\pgfsetbuttcap%
\pgfsetroundjoin%
\definecolor{currentfill}{rgb}{0.229739,0.322361,0.545706}%
\pgfsetfillcolor{currentfill}%
\pgfsetfillopacity{0.800000}%
\pgfsetlinewidth{0.000000pt}%
\definecolor{currentstroke}{rgb}{0.000000,0.000000,0.000000}%
\pgfsetstrokecolor{currentstroke}%
\pgfsetdash{}{0pt}%
\pgfpathmoveto{\pgfqpoint{3.232566in}{3.015047in}}%
\pgfpathlineto{\pgfqpoint{3.245971in}{2.998912in}}%
\pgfpathlineto{\pgfqpoint{3.259371in}{2.983067in}}%
\pgfpathlineto{\pgfqpoint{3.272767in}{2.967509in}}%
\pgfpathlineto{\pgfqpoint{3.286158in}{2.952236in}}%
\pgfpathlineto{\pgfqpoint{3.294083in}{2.965980in}}%
\pgfpathlineto{\pgfqpoint{3.302002in}{2.979878in}}%
\pgfpathlineto{\pgfqpoint{3.309914in}{2.993931in}}%
\pgfpathlineto{\pgfqpoint{3.317821in}{3.008143in}}%
\pgfpathlineto{\pgfqpoint{3.304439in}{3.023639in}}%
\pgfpathlineto{\pgfqpoint{3.291053in}{3.039420in}}%
\pgfpathlineto{\pgfqpoint{3.277662in}{3.055488in}}%
\pgfpathlineto{\pgfqpoint{3.264266in}{3.071846in}}%
\pgfpathlineto{\pgfqpoint{3.256351in}{3.057399in}}%
\pgfpathlineto{\pgfqpoint{3.248429in}{3.043118in}}%
\pgfpathlineto{\pgfqpoint{3.240501in}{3.029002in}}%
\pgfpathlineto{\pgfqpoint{3.232566in}{3.015047in}}%
\pgfpathclose%
\pgfusepath{fill}%
\end{pgfscope}%
\begin{pgfscope}%
\pgfpathrectangle{\pgfqpoint{1.150000in}{0.150000in}}{\pgfqpoint{5.700000in}{5.700000in}}%
\pgfusepath{clip}%
\pgfsetbuttcap%
\pgfsetroundjoin%
\definecolor{currentfill}{rgb}{0.239346,0.300855,0.540844}%
\pgfsetfillcolor{currentfill}%
\pgfsetfillopacity{0.800000}%
\pgfsetlinewidth{0.000000pt}%
\definecolor{currentstroke}{rgb}{0.000000,0.000000,0.000000}%
\pgfsetstrokecolor{currentstroke}%
\pgfsetdash{}{0pt}%
\pgfpathmoveto{\pgfqpoint{3.286158in}{2.952236in}}%
\pgfpathlineto{\pgfqpoint{3.299545in}{2.937245in}}%
\pgfpathlineto{\pgfqpoint{3.312929in}{2.922534in}}%
\pgfpathlineto{\pgfqpoint{3.326309in}{2.908102in}}%
\pgfpathlineto{\pgfqpoint{3.339686in}{2.893945in}}%
\pgfpathlineto{\pgfqpoint{3.347601in}{2.907479in}}%
\pgfpathlineto{\pgfqpoint{3.355510in}{2.921158in}}%
\pgfpathlineto{\pgfqpoint{3.363414in}{2.934985in}}%
\pgfpathlineto{\pgfqpoint{3.371311in}{2.948963in}}%
\pgfpathlineto{\pgfqpoint{3.357944in}{2.963342in}}%
\pgfpathlineto{\pgfqpoint{3.344573in}{2.977997in}}%
\pgfpathlineto{\pgfqpoint{3.331199in}{2.992930in}}%
\pgfpathlineto{\pgfqpoint{3.317821in}{3.008143in}}%
\pgfpathlineto{\pgfqpoint{3.309914in}{2.993931in}}%
\pgfpathlineto{\pgfqpoint{3.302002in}{2.979878in}}%
\pgfpathlineto{\pgfqpoint{3.294083in}{2.965980in}}%
\pgfpathlineto{\pgfqpoint{3.286158in}{2.952236in}}%
\pgfpathclose%
\pgfusepath{fill}%
\end{pgfscope}%
\begin{pgfscope}%
\pgfpathrectangle{\pgfqpoint{1.150000in}{0.150000in}}{\pgfqpoint{5.700000in}{5.700000in}}%
\pgfusepath{clip}%
\pgfsetbuttcap%
\pgfsetroundjoin%
\definecolor{currentfill}{rgb}{0.212395,0.359683,0.551710}%
\pgfsetfillcolor{currentfill}%
\pgfsetfillopacity{0.800000}%
\pgfsetlinewidth{0.000000pt}%
\definecolor{currentstroke}{rgb}{0.000000,0.000000,0.000000}%
\pgfsetstrokecolor{currentstroke}%
\pgfsetdash{}{0pt}%
\pgfpathmoveto{\pgfqpoint{4.759484in}{3.081646in}}%
\pgfpathlineto{\pgfqpoint{4.773041in}{3.079129in}}%
\pgfpathlineto{\pgfqpoint{4.786606in}{3.076797in}}%
\pgfpathlineto{\pgfqpoint{4.800181in}{3.074649in}}%
\pgfpathlineto{\pgfqpoint{4.813765in}{3.072685in}}%
\pgfpathlineto{\pgfqpoint{4.821328in}{3.087013in}}%
\pgfpathlineto{\pgfqpoint{4.828891in}{3.101587in}}%
\pgfpathlineto{\pgfqpoint{4.836454in}{3.116414in}}%
\pgfpathlineto{\pgfqpoint{4.844018in}{3.131501in}}%
\pgfpathlineto{\pgfqpoint{4.830448in}{3.134092in}}%
\pgfpathlineto{\pgfqpoint{4.816887in}{3.136867in}}%
\pgfpathlineto{\pgfqpoint{4.803334in}{3.139826in}}%
\pgfpathlineto{\pgfqpoint{4.789791in}{3.142970in}}%
\pgfpathlineto{\pgfqpoint{4.782214in}{3.127244in}}%
\pgfpathlineto{\pgfqpoint{4.774637in}{3.111787in}}%
\pgfpathlineto{\pgfqpoint{4.767061in}{3.096590in}}%
\pgfpathlineto{\pgfqpoint{4.759484in}{3.081646in}}%
\pgfpathclose%
\pgfusepath{fill}%
\end{pgfscope}%
\begin{pgfscope}%
\pgfpathrectangle{\pgfqpoint{1.150000in}{0.150000in}}{\pgfqpoint{5.700000in}{5.700000in}}%
\pgfusepath{clip}%
\pgfsetbuttcap%
\pgfsetroundjoin%
\definecolor{currentfill}{rgb}{0.252194,0.269783,0.531579}%
\pgfsetfillcolor{currentfill}%
\pgfsetfillopacity{0.800000}%
\pgfsetlinewidth{0.000000pt}%
\definecolor{currentstroke}{rgb}{0.000000,0.000000,0.000000}%
\pgfsetstrokecolor{currentstroke}%
\pgfsetdash{}{0pt}%
\pgfpathmoveto{\pgfqpoint{4.337180in}{2.862505in}}%
\pgfpathlineto{\pgfqpoint{4.350630in}{2.858822in}}%
\pgfpathlineto{\pgfqpoint{4.364087in}{2.855338in}}%
\pgfpathlineto{\pgfqpoint{4.377551in}{2.852050in}}%
\pgfpathlineto{\pgfqpoint{4.391023in}{2.848960in}}%
\pgfpathlineto{\pgfqpoint{4.398677in}{2.862049in}}%
\pgfpathlineto{\pgfqpoint{4.406329in}{2.875297in}}%
\pgfpathlineto{\pgfqpoint{4.413978in}{2.888709in}}%
\pgfpathlineto{\pgfqpoint{4.421624in}{2.902291in}}%
\pgfpathlineto{\pgfqpoint{4.408161in}{2.905851in}}%
\pgfpathlineto{\pgfqpoint{4.394706in}{2.909609in}}%
\pgfpathlineto{\pgfqpoint{4.381259in}{2.913563in}}%
\pgfpathlineto{\pgfqpoint{4.367818in}{2.917715in}}%
\pgfpathlineto{\pgfqpoint{4.360162in}{2.903652in}}%
\pgfpathlineto{\pgfqpoint{4.352504in}{2.889766in}}%
\pgfpathlineto{\pgfqpoint{4.344843in}{2.876052in}}%
\pgfpathlineto{\pgfqpoint{4.337180in}{2.862505in}}%
\pgfpathclose%
\pgfusepath{fill}%
\end{pgfscope}%
\begin{pgfscope}%
\pgfpathrectangle{\pgfqpoint{1.150000in}{0.150000in}}{\pgfqpoint{5.700000in}{5.700000in}}%
\pgfusepath{clip}%
\pgfsetbuttcap%
\pgfsetroundjoin%
\definecolor{currentfill}{rgb}{0.218130,0.347432,0.550038}%
\pgfsetfillcolor{currentfill}%
\pgfsetfillopacity{0.800000}%
\pgfsetlinewidth{0.000000pt}%
\definecolor{currentstroke}{rgb}{0.000000,0.000000,0.000000}%
\pgfsetstrokecolor{currentstroke}%
\pgfsetdash{}{0pt}%
\pgfpathmoveto{\pgfqpoint{3.178894in}{3.082531in}}%
\pgfpathlineto{\pgfqpoint{3.192320in}{3.065213in}}%
\pgfpathlineto{\pgfqpoint{3.205741in}{3.048195in}}%
\pgfpathlineto{\pgfqpoint{3.219156in}{3.031473in}}%
\pgfpathlineto{\pgfqpoint{3.232566in}{3.015047in}}%
\pgfpathlineto{\pgfqpoint{3.240501in}{3.029002in}}%
\pgfpathlineto{\pgfqpoint{3.248429in}{3.043118in}}%
\pgfpathlineto{\pgfqpoint{3.256351in}{3.057399in}}%
\pgfpathlineto{\pgfqpoint{3.264266in}{3.071846in}}%
\pgfpathlineto{\pgfqpoint{3.250866in}{3.088496in}}%
\pgfpathlineto{\pgfqpoint{3.237461in}{3.105441in}}%
\pgfpathlineto{\pgfqpoint{3.224049in}{3.122683in}}%
\pgfpathlineto{\pgfqpoint{3.210633in}{3.140225in}}%
\pgfpathlineto{\pgfqpoint{3.202708in}{3.125542in}}%
\pgfpathlineto{\pgfqpoint{3.194777in}{3.111034in}}%
\pgfpathlineto{\pgfqpoint{3.186839in}{3.096698in}}%
\pgfpathlineto{\pgfqpoint{3.178894in}{3.082531in}}%
\pgfpathclose%
\pgfusepath{fill}%
\end{pgfscope}%
\begin{pgfscope}%
\pgfpathrectangle{\pgfqpoint{1.150000in}{0.150000in}}{\pgfqpoint{5.700000in}{5.700000in}}%
\pgfusepath{clip}%
\pgfsetbuttcap%
\pgfsetroundjoin%
\definecolor{currentfill}{rgb}{0.267968,0.223549,0.512008}%
\pgfsetfillcolor{currentfill}%
\pgfsetfillopacity{0.800000}%
\pgfsetlinewidth{0.000000pt}%
\definecolor{currentstroke}{rgb}{0.000000,0.000000,0.000000}%
\pgfsetstrokecolor{currentstroke}%
\pgfsetdash{}{0pt}%
\pgfpathmoveto{\pgfqpoint{3.945689in}{2.749399in}}%
\pgfpathlineto{\pgfqpoint{3.959060in}{2.743143in}}%
\pgfpathlineto{\pgfqpoint{3.972435in}{2.737103in}}%
\pgfpathlineto{\pgfqpoint{3.985814in}{2.731278in}}%
\pgfpathlineto{\pgfqpoint{3.999197in}{2.725667in}}%
\pgfpathlineto{\pgfqpoint{4.006953in}{2.738562in}}%
\pgfpathlineto{\pgfqpoint{4.014704in}{2.751579in}}%
\pgfpathlineto{\pgfqpoint{4.022450in}{2.764721in}}%
\pgfpathlineto{\pgfqpoint{4.030193in}{2.777993in}}%
\pgfpathlineto{\pgfqpoint{4.016817in}{2.783948in}}%
\pgfpathlineto{\pgfqpoint{4.003446in}{2.790118in}}%
\pgfpathlineto{\pgfqpoint{3.990078in}{2.796503in}}%
\pgfpathlineto{\pgfqpoint{3.976715in}{2.803104in}}%
\pgfpathlineto{\pgfqpoint{3.968965in}{2.789476in}}%
\pgfpathlineto{\pgfqpoint{3.961211in}{2.775985in}}%
\pgfpathlineto{\pgfqpoint{3.953452in}{2.762627in}}%
\pgfpathlineto{\pgfqpoint{3.945689in}{2.749399in}}%
\pgfpathclose%
\pgfusepath{fill}%
\end{pgfscope}%
\begin{pgfscope}%
\pgfpathrectangle{\pgfqpoint{1.150000in}{0.150000in}}{\pgfqpoint{5.700000in}{5.700000in}}%
\pgfusepath{clip}%
\pgfsetbuttcap%
\pgfsetroundjoin%
\definecolor{currentfill}{rgb}{0.248629,0.278775,0.534556}%
\pgfsetfillcolor{currentfill}%
\pgfsetfillopacity{0.800000}%
\pgfsetlinewidth{0.000000pt}%
\definecolor{currentstroke}{rgb}{0.000000,0.000000,0.000000}%
\pgfsetstrokecolor{currentstroke}%
\pgfsetdash{}{0pt}%
\pgfpathmoveto{\pgfqpoint{3.339686in}{2.893945in}}%
\pgfpathlineto{\pgfqpoint{3.353059in}{2.880061in}}%
\pgfpathlineto{\pgfqpoint{3.366430in}{2.866449in}}%
\pgfpathlineto{\pgfqpoint{3.379798in}{2.853107in}}%
\pgfpathlineto{\pgfqpoint{3.393164in}{2.840031in}}%
\pgfpathlineto{\pgfqpoint{3.401070in}{2.853356in}}%
\pgfpathlineto{\pgfqpoint{3.408970in}{2.866817in}}%
\pgfpathlineto{\pgfqpoint{3.416865in}{2.880419in}}%
\pgfpathlineto{\pgfqpoint{3.424754in}{2.894163in}}%
\pgfpathlineto{\pgfqpoint{3.411397in}{2.907460in}}%
\pgfpathlineto{\pgfqpoint{3.398038in}{2.921024in}}%
\pgfpathlineto{\pgfqpoint{3.384676in}{2.934858in}}%
\pgfpathlineto{\pgfqpoint{3.371311in}{2.948963in}}%
\pgfpathlineto{\pgfqpoint{3.363414in}{2.934985in}}%
\pgfpathlineto{\pgfqpoint{3.355510in}{2.921158in}}%
\pgfpathlineto{\pgfqpoint{3.347601in}{2.907479in}}%
\pgfpathlineto{\pgfqpoint{3.339686in}{2.893945in}}%
\pgfpathclose%
\pgfusepath{fill}%
\end{pgfscope}%
\begin{pgfscope}%
\pgfpathrectangle{\pgfqpoint{1.150000in}{0.150000in}}{\pgfqpoint{5.700000in}{5.700000in}}%
\pgfusepath{clip}%
\pgfsetbuttcap%
\pgfsetroundjoin%
\definecolor{currentfill}{rgb}{0.203063,0.379716,0.553925}%
\pgfsetfillcolor{currentfill}%
\pgfsetfillopacity{0.800000}%
\pgfsetlinewidth{0.000000pt}%
\definecolor{currentstroke}{rgb}{0.000000,0.000000,0.000000}%
\pgfsetstrokecolor{currentstroke}%
\pgfsetdash{}{0pt}%
\pgfpathmoveto{\pgfqpoint{4.844018in}{3.131501in}}%
\pgfpathlineto{\pgfqpoint{4.857598in}{3.129093in}}%
\pgfpathlineto{\pgfqpoint{4.871186in}{3.126868in}}%
\pgfpathlineto{\pgfqpoint{4.884785in}{3.124825in}}%
\pgfpathlineto{\pgfqpoint{4.898393in}{3.122964in}}%
\pgfpathlineto{\pgfqpoint{4.905943in}{3.137673in}}%
\pgfpathlineto{\pgfqpoint{4.913494in}{3.152649in}}%
\pgfpathlineto{\pgfqpoint{4.921046in}{3.167900in}}%
\pgfpathlineto{\pgfqpoint{4.928599in}{3.183434in}}%
\pgfpathlineto{\pgfqpoint{4.915006in}{3.185954in}}%
\pgfpathlineto{\pgfqpoint{4.901422in}{3.188655in}}%
\pgfpathlineto{\pgfqpoint{4.887848in}{3.191539in}}%
\pgfpathlineto{\pgfqpoint{4.874282in}{3.194605in}}%
\pgfpathlineto{\pgfqpoint{4.866714in}{3.178401in}}%
\pgfpathlineto{\pgfqpoint{4.859148in}{3.162487in}}%
\pgfpathlineto{\pgfqpoint{4.851582in}{3.146856in}}%
\pgfpathlineto{\pgfqpoint{4.844018in}{3.131501in}}%
\pgfpathclose%
\pgfusepath{fill}%
\end{pgfscope}%
\begin{pgfscope}%
\pgfpathrectangle{\pgfqpoint{1.150000in}{0.150000in}}{\pgfqpoint{5.700000in}{5.700000in}}%
\pgfusepath{clip}%
\pgfsetbuttcap%
\pgfsetroundjoin%
\definecolor{currentfill}{rgb}{0.257322,0.256130,0.526563}%
\pgfsetfillcolor{currentfill}%
\pgfsetfillopacity{0.800000}%
\pgfsetlinewidth{0.000000pt}%
\definecolor{currentstroke}{rgb}{0.000000,0.000000,0.000000}%
\pgfsetstrokecolor{currentstroke}%
\pgfsetdash{}{0pt}%
\pgfpathmoveto{\pgfqpoint{4.252726in}{2.824842in}}%
\pgfpathlineto{\pgfqpoint{4.266158in}{2.820798in}}%
\pgfpathlineto{\pgfqpoint{4.279596in}{2.816956in}}%
\pgfpathlineto{\pgfqpoint{4.293041in}{2.813314in}}%
\pgfpathlineto{\pgfqpoint{4.306494in}{2.809871in}}%
\pgfpathlineto{\pgfqpoint{4.314170in}{2.822806in}}%
\pgfpathlineto{\pgfqpoint{4.321843in}{2.835887in}}%
\pgfpathlineto{\pgfqpoint{4.329513in}{2.849118in}}%
\pgfpathlineto{\pgfqpoint{4.337180in}{2.862505in}}%
\pgfpathlineto{\pgfqpoint{4.323737in}{2.866386in}}%
\pgfpathlineto{\pgfqpoint{4.310300in}{2.870467in}}%
\pgfpathlineto{\pgfqpoint{4.296870in}{2.874748in}}%
\pgfpathlineto{\pgfqpoint{4.283447in}{2.879230in}}%
\pgfpathlineto{\pgfqpoint{4.275771in}{2.865393in}}%
\pgfpathlineto{\pgfqpoint{4.268093in}{2.851720in}}%
\pgfpathlineto{\pgfqpoint{4.260411in}{2.838204in}}%
\pgfpathlineto{\pgfqpoint{4.252726in}{2.824842in}}%
\pgfpathclose%
\pgfusepath{fill}%
\end{pgfscope}%
\begin{pgfscope}%
\pgfpathrectangle{\pgfqpoint{1.150000in}{0.150000in}}{\pgfqpoint{5.700000in}{5.700000in}}%
\pgfusepath{clip}%
\pgfsetbuttcap%
\pgfsetroundjoin%
\definecolor{currentfill}{rgb}{0.270595,0.214069,0.507052}%
\pgfsetfillcolor{currentfill}%
\pgfsetfillopacity{0.800000}%
\pgfsetlinewidth{0.000000pt}%
\definecolor{currentstroke}{rgb}{0.000000,0.000000,0.000000}%
\pgfsetstrokecolor{currentstroke}%
\pgfsetdash{}{0pt}%
\pgfpathmoveto{\pgfqpoint{3.723065in}{2.732958in}}%
\pgfpathlineto{\pgfqpoint{3.736414in}{2.724612in}}%
\pgfpathlineto{\pgfqpoint{3.749764in}{2.716497in}}%
\pgfpathlineto{\pgfqpoint{3.763117in}{2.708612in}}%
\pgfpathlineto{\pgfqpoint{3.776472in}{2.700954in}}%
\pgfpathlineto{\pgfqpoint{3.784287in}{2.713829in}}%
\pgfpathlineto{\pgfqpoint{3.792096in}{2.726819in}}%
\pgfpathlineto{\pgfqpoint{3.799901in}{2.739927in}}%
\pgfpathlineto{\pgfqpoint{3.807701in}{2.753157in}}%
\pgfpathlineto{\pgfqpoint{3.794354in}{2.761097in}}%
\pgfpathlineto{\pgfqpoint{3.781008in}{2.769266in}}%
\pgfpathlineto{\pgfqpoint{3.767666in}{2.777663in}}%
\pgfpathlineto{\pgfqpoint{3.754325in}{2.786292in}}%
\pgfpathlineto{\pgfqpoint{3.746517in}{2.772767in}}%
\pgfpathlineto{\pgfqpoint{3.738705in}{2.759373in}}%
\pgfpathlineto{\pgfqpoint{3.730887in}{2.746104in}}%
\pgfpathlineto{\pgfqpoint{3.723065in}{2.732958in}}%
\pgfpathclose%
\pgfusepath{fill}%
\end{pgfscope}%
\begin{pgfscope}%
\pgfpathrectangle{\pgfqpoint{1.150000in}{0.150000in}}{\pgfqpoint{5.700000in}{5.700000in}}%
\pgfusepath{clip}%
\pgfsetbuttcap%
\pgfsetroundjoin%
\definecolor{currentfill}{rgb}{0.204903,0.375746,0.553533}%
\pgfsetfillcolor{currentfill}%
\pgfsetfillopacity{0.800000}%
\pgfsetlinewidth{0.000000pt}%
\definecolor{currentstroke}{rgb}{0.000000,0.000000,0.000000}%
\pgfsetstrokecolor{currentstroke}%
\pgfsetdash{}{0pt}%
\pgfpathmoveto{\pgfqpoint{3.125124in}{3.154854in}}%
\pgfpathlineto{\pgfqpoint{3.138577in}{3.136310in}}%
\pgfpathlineto{\pgfqpoint{3.152022in}{3.118077in}}%
\pgfpathlineto{\pgfqpoint{3.165461in}{3.100152in}}%
\pgfpathlineto{\pgfqpoint{3.178894in}{3.082531in}}%
\pgfpathlineto{\pgfqpoint{3.186839in}{3.096698in}}%
\pgfpathlineto{\pgfqpoint{3.194777in}{3.111034in}}%
\pgfpathlineto{\pgfqpoint{3.202708in}{3.125542in}}%
\pgfpathlineto{\pgfqpoint{3.210633in}{3.140225in}}%
\pgfpathlineto{\pgfqpoint{3.197210in}{3.158069in}}%
\pgfpathlineto{\pgfqpoint{3.183781in}{3.176219in}}%
\pgfpathlineto{\pgfqpoint{3.170345in}{3.194677in}}%
\pgfpathlineto{\pgfqpoint{3.156903in}{3.213446in}}%
\pgfpathlineto{\pgfqpoint{3.148968in}{3.198526in}}%
\pgfpathlineto{\pgfqpoint{3.141027in}{3.183790in}}%
\pgfpathlineto{\pgfqpoint{3.133079in}{3.169233in}}%
\pgfpathlineto{\pgfqpoint{3.125124in}{3.154854in}}%
\pgfpathclose%
\pgfusepath{fill}%
\end{pgfscope}%
\begin{pgfscope}%
\pgfpathrectangle{\pgfqpoint{1.150000in}{0.150000in}}{\pgfqpoint{5.700000in}{5.700000in}}%
\pgfusepath{clip}%
\pgfsetbuttcap%
\pgfsetroundjoin%
\definecolor{currentfill}{rgb}{0.267968,0.223549,0.512008}%
\pgfsetfillcolor{currentfill}%
\pgfsetfillopacity{0.800000}%
\pgfsetlinewidth{0.000000pt}%
\definecolor{currentstroke}{rgb}{0.000000,0.000000,0.000000}%
\pgfsetstrokecolor{currentstroke}%
\pgfsetdash{}{0pt}%
\pgfpathmoveto{\pgfqpoint{3.584932in}{2.754761in}}%
\pgfpathlineto{\pgfqpoint{3.598278in}{2.744770in}}%
\pgfpathlineto{\pgfqpoint{3.611624in}{2.735022in}}%
\pgfpathlineto{\pgfqpoint{3.624972in}{2.725514in}}%
\pgfpathlineto{\pgfqpoint{3.638320in}{2.716245in}}%
\pgfpathlineto{\pgfqpoint{3.646170in}{2.729173in}}%
\pgfpathlineto{\pgfqpoint{3.654015in}{2.742219in}}%
\pgfpathlineto{\pgfqpoint{3.661855in}{2.755386in}}%
\pgfpathlineto{\pgfqpoint{3.669690in}{2.768676in}}%
\pgfpathlineto{\pgfqpoint{3.656349in}{2.778196in}}%
\pgfpathlineto{\pgfqpoint{3.643010in}{2.787956in}}%
\pgfpathlineto{\pgfqpoint{3.629671in}{2.797956in}}%
\pgfpathlineto{\pgfqpoint{3.616333in}{2.808199in}}%
\pgfpathlineto{\pgfqpoint{3.608491in}{2.794646in}}%
\pgfpathlineto{\pgfqpoint{3.600643in}{2.781223in}}%
\pgfpathlineto{\pgfqpoint{3.592790in}{2.767930in}}%
\pgfpathlineto{\pgfqpoint{3.584932in}{2.754761in}}%
\pgfpathclose%
\pgfusepath{fill}%
\end{pgfscope}%
\begin{pgfscope}%
\pgfpathrectangle{\pgfqpoint{1.150000in}{0.150000in}}{\pgfqpoint{5.700000in}{5.700000in}}%
\pgfusepath{clip}%
\pgfsetbuttcap%
\pgfsetroundjoin%
\definecolor{currentfill}{rgb}{0.194100,0.399323,0.555565}%
\pgfsetfillcolor{currentfill}%
\pgfsetfillopacity{0.800000}%
\pgfsetlinewidth{0.000000pt}%
\definecolor{currentstroke}{rgb}{0.000000,0.000000,0.000000}%
\pgfsetstrokecolor{currentstroke}%
\pgfsetdash{}{0pt}%
\pgfpathmoveto{\pgfqpoint{4.928599in}{3.183434in}}%
\pgfpathlineto{\pgfqpoint{4.942202in}{3.181096in}}%
\pgfpathlineto{\pgfqpoint{4.955815in}{3.178938in}}%
\pgfpathlineto{\pgfqpoint{4.969437in}{3.176961in}}%
\pgfpathlineto{\pgfqpoint{4.983069in}{3.175164in}}%
\pgfpathlineto{\pgfqpoint{4.990609in}{3.190310in}}%
\pgfpathlineto{\pgfqpoint{4.998150in}{3.205747in}}%
\pgfpathlineto{\pgfqpoint{5.005693in}{3.221483in}}%
\pgfpathlineto{\pgfqpoint{5.013239in}{3.237526in}}%
\pgfpathlineto{\pgfqpoint{4.999623in}{3.240014in}}%
\pgfpathlineto{\pgfqpoint{4.986016in}{3.242682in}}%
\pgfpathlineto{\pgfqpoint{4.972419in}{3.245529in}}%
\pgfpathlineto{\pgfqpoint{4.958831in}{3.248558in}}%
\pgfpathlineto{\pgfqpoint{4.951270in}{3.231813in}}%
\pgfpathlineto{\pgfqpoint{4.943711in}{3.215383in}}%
\pgfpathlineto{\pgfqpoint{4.936154in}{3.199259in}}%
\pgfpathlineto{\pgfqpoint{4.928599in}{3.183434in}}%
\pgfpathclose%
\pgfusepath{fill}%
\end{pgfscope}%
\begin{pgfscope}%
\pgfpathrectangle{\pgfqpoint{1.150000in}{0.150000in}}{\pgfqpoint{5.700000in}{5.700000in}}%
\pgfusepath{clip}%
\pgfsetbuttcap%
\pgfsetroundjoin%
\definecolor{currentfill}{rgb}{0.257322,0.256130,0.526563}%
\pgfsetfillcolor{currentfill}%
\pgfsetfillopacity{0.800000}%
\pgfsetlinewidth{0.000000pt}%
\definecolor{currentstroke}{rgb}{0.000000,0.000000,0.000000}%
\pgfsetstrokecolor{currentstroke}%
\pgfsetdash{}{0pt}%
\pgfpathmoveto{\pgfqpoint{3.393164in}{2.840031in}}%
\pgfpathlineto{\pgfqpoint{3.406527in}{2.827221in}}%
\pgfpathlineto{\pgfqpoint{3.419889in}{2.814674in}}%
\pgfpathlineto{\pgfqpoint{3.433249in}{2.802389in}}%
\pgfpathlineto{\pgfqpoint{3.446607in}{2.790363in}}%
\pgfpathlineto{\pgfqpoint{3.454504in}{2.803478in}}%
\pgfpathlineto{\pgfqpoint{3.462396in}{2.816722in}}%
\pgfpathlineto{\pgfqpoint{3.470281in}{2.830099in}}%
\pgfpathlineto{\pgfqpoint{3.478162in}{2.843610in}}%
\pgfpathlineto{\pgfqpoint{3.464812in}{2.855857in}}%
\pgfpathlineto{\pgfqpoint{3.451461in}{2.868364in}}%
\pgfpathlineto{\pgfqpoint{3.438108in}{2.881132in}}%
\pgfpathlineto{\pgfqpoint{3.424754in}{2.894163in}}%
\pgfpathlineto{\pgfqpoint{3.416865in}{2.880419in}}%
\pgfpathlineto{\pgfqpoint{3.408970in}{2.866817in}}%
\pgfpathlineto{\pgfqpoint{3.401070in}{2.853356in}}%
\pgfpathlineto{\pgfqpoint{3.393164in}{2.840031in}}%
\pgfpathclose%
\pgfusepath{fill}%
\end{pgfscope}%
\begin{pgfscope}%
\pgfpathrectangle{\pgfqpoint{1.150000in}{0.150000in}}{\pgfqpoint{5.700000in}{5.700000in}}%
\pgfusepath{clip}%
\pgfsetbuttcap%
\pgfsetroundjoin%
\definecolor{currentfill}{rgb}{0.262138,0.242286,0.520837}%
\pgfsetfillcolor{currentfill}%
\pgfsetfillopacity{0.800000}%
\pgfsetlinewidth{0.000000pt}%
\definecolor{currentstroke}{rgb}{0.000000,0.000000,0.000000}%
\pgfsetstrokecolor{currentstroke}%
\pgfsetdash{}{0pt}%
\pgfpathmoveto{\pgfqpoint{4.168251in}{2.789397in}}%
\pgfpathlineto{\pgfqpoint{4.181667in}{2.784948in}}%
\pgfpathlineto{\pgfqpoint{4.195088in}{2.780703in}}%
\pgfpathlineto{\pgfqpoint{4.208516in}{2.776662in}}%
\pgfpathlineto{\pgfqpoint{4.221950in}{2.772824in}}%
\pgfpathlineto{\pgfqpoint{4.229649in}{2.785624in}}%
\pgfpathlineto{\pgfqpoint{4.237345in}{2.798557in}}%
\pgfpathlineto{\pgfqpoint{4.245037in}{2.811628in}}%
\pgfpathlineto{\pgfqpoint{4.252726in}{2.824842in}}%
\pgfpathlineto{\pgfqpoint{4.239300in}{2.829088in}}%
\pgfpathlineto{\pgfqpoint{4.225881in}{2.833536in}}%
\pgfpathlineto{\pgfqpoint{4.212467in}{2.838188in}}%
\pgfpathlineto{\pgfqpoint{4.199060in}{2.843045in}}%
\pgfpathlineto{\pgfqpoint{4.191363in}{2.829411in}}%
\pgfpathlineto{\pgfqpoint{4.183663in}{2.815929in}}%
\pgfpathlineto{\pgfqpoint{4.175959in}{2.802593in}}%
\pgfpathlineto{\pgfqpoint{4.168251in}{2.789397in}}%
\pgfpathclose%
\pgfusepath{fill}%
\end{pgfscope}%
\begin{pgfscope}%
\pgfpathrectangle{\pgfqpoint{1.150000in}{0.150000in}}{\pgfqpoint{5.700000in}{5.700000in}}%
\pgfusepath{clip}%
\pgfsetbuttcap%
\pgfsetroundjoin%
\definecolor{currentfill}{rgb}{0.270595,0.214069,0.507052}%
\pgfsetfillcolor{currentfill}%
\pgfsetfillopacity{0.800000}%
\pgfsetlinewidth{0.000000pt}%
\definecolor{currentstroke}{rgb}{0.000000,0.000000,0.000000}%
\pgfsetstrokecolor{currentstroke}%
\pgfsetdash{}{0pt}%
\pgfpathmoveto{\pgfqpoint{3.861121in}{2.723652in}}%
\pgfpathlineto{\pgfqpoint{3.874484in}{2.716834in}}%
\pgfpathlineto{\pgfqpoint{3.887850in}{2.710236in}}%
\pgfpathlineto{\pgfqpoint{3.901220in}{2.703858in}}%
\pgfpathlineto{\pgfqpoint{3.914594in}{2.697698in}}%
\pgfpathlineto{\pgfqpoint{3.922375in}{2.710449in}}%
\pgfpathlineto{\pgfqpoint{3.930151in}{2.723314in}}%
\pgfpathlineto{\pgfqpoint{3.937922in}{2.736296in}}%
\pgfpathlineto{\pgfqpoint{3.945689in}{2.749399in}}%
\pgfpathlineto{\pgfqpoint{3.932323in}{2.755872in}}%
\pgfpathlineto{\pgfqpoint{3.918960in}{2.762564in}}%
\pgfpathlineto{\pgfqpoint{3.905601in}{2.769476in}}%
\pgfpathlineto{\pgfqpoint{3.892246in}{2.776608in}}%
\pgfpathlineto{\pgfqpoint{3.884471in}{2.763180in}}%
\pgfpathlineto{\pgfqpoint{3.876692in}{2.749880in}}%
\pgfpathlineto{\pgfqpoint{3.868909in}{2.736705in}}%
\pgfpathlineto{\pgfqpoint{3.861121in}{2.723652in}}%
\pgfpathclose%
\pgfusepath{fill}%
\end{pgfscope}%
\begin{pgfscope}%
\pgfpathrectangle{\pgfqpoint{1.150000in}{0.150000in}}{\pgfqpoint{5.700000in}{5.700000in}}%
\pgfusepath{clip}%
\pgfsetbuttcap%
\pgfsetroundjoin%
\definecolor{currentfill}{rgb}{0.159194,0.482237,0.558073}%
\pgfsetfillcolor{currentfill}%
\pgfsetfillopacity{0.800000}%
\pgfsetlinewidth{0.000000pt}%
\definecolor{currentstroke}{rgb}{0.000000,0.000000,0.000000}%
\pgfsetstrokecolor{currentstroke}%
\pgfsetdash{}{0pt}%
\pgfpathmoveto{\pgfqpoint{2.994957in}{3.464010in}}%
\pgfpathlineto{\pgfqpoint{3.008503in}{3.441273in}}%
\pgfpathlineto{\pgfqpoint{3.022040in}{3.418884in}}%
\pgfpathlineto{\pgfqpoint{3.035567in}{3.396841in}}%
\pgfpathlineto{\pgfqpoint{3.049083in}{3.375139in}}%
\pgfpathlineto{\pgfqpoint{3.057030in}{3.390738in}}%
\pgfpathlineto{\pgfqpoint{3.064970in}{3.406543in}}%
\pgfpathlineto{\pgfqpoint{3.072902in}{3.422556in}}%
\pgfpathlineto{\pgfqpoint{3.080827in}{3.438781in}}%
\pgfpathlineto{\pgfqpoint{3.067319in}{3.460743in}}%
\pgfpathlineto{\pgfqpoint{3.053803in}{3.483047in}}%
\pgfpathlineto{\pgfqpoint{3.040276in}{3.505698in}}%
\pgfpathlineto{\pgfqpoint{3.026738in}{3.528697in}}%
\pgfpathlineto{\pgfqpoint{3.018804in}{3.512198in}}%
\pgfpathlineto{\pgfqpoint{3.010863in}{3.495919in}}%
\pgfpathlineto{\pgfqpoint{3.002913in}{3.479858in}}%
\pgfpathlineto{\pgfqpoint{2.994957in}{3.464010in}}%
\pgfpathclose%
\pgfusepath{fill}%
\end{pgfscope}%
\begin{pgfscope}%
\pgfpathrectangle{\pgfqpoint{1.150000in}{0.150000in}}{\pgfqpoint{5.700000in}{5.700000in}}%
\pgfusepath{clip}%
\pgfsetbuttcap%
\pgfsetroundjoin%
\definecolor{currentfill}{rgb}{0.185556,0.418570,0.556753}%
\pgfsetfillcolor{currentfill}%
\pgfsetfillopacity{0.800000}%
\pgfsetlinewidth{0.000000pt}%
\definecolor{currentstroke}{rgb}{0.000000,0.000000,0.000000}%
\pgfsetstrokecolor{currentstroke}%
\pgfsetdash{}{0pt}%
\pgfpathmoveto{\pgfqpoint{5.013239in}{3.237526in}}%
\pgfpathlineto{\pgfqpoint{5.026865in}{3.235218in}}%
\pgfpathlineto{\pgfqpoint{5.040501in}{3.233088in}}%
\pgfpathlineto{\pgfqpoint{5.054148in}{3.231137in}}%
\pgfpathlineto{\pgfqpoint{5.067804in}{3.229364in}}%
\pgfpathlineto{\pgfqpoint{5.075336in}{3.245012in}}%
\pgfpathlineto{\pgfqpoint{5.082870in}{3.260975in}}%
\pgfpathlineto{\pgfqpoint{5.090408in}{3.277262in}}%
\pgfpathlineto{\pgfqpoint{5.097950in}{3.293881in}}%
\pgfpathlineto{\pgfqpoint{5.084311in}{3.296377in}}%
\pgfpathlineto{\pgfqpoint{5.070681in}{3.299050in}}%
\pgfpathlineto{\pgfqpoint{5.057062in}{3.301901in}}%
\pgfpathlineto{\pgfqpoint{5.043452in}{3.304931in}}%
\pgfpathlineto{\pgfqpoint{5.035894in}{3.287578in}}%
\pgfpathlineto{\pgfqpoint{5.028339in}{3.270565in}}%
\pgfpathlineto{\pgfqpoint{5.020788in}{3.253884in}}%
\pgfpathlineto{\pgfqpoint{5.013239in}{3.237526in}}%
\pgfpathclose%
\pgfusepath{fill}%
\end{pgfscope}%
\begin{pgfscope}%
\pgfpathrectangle{\pgfqpoint{1.150000in}{0.150000in}}{\pgfqpoint{5.700000in}{5.700000in}}%
\pgfusepath{clip}%
\pgfsetbuttcap%
\pgfsetroundjoin%
\definecolor{currentfill}{rgb}{0.192357,0.403199,0.555836}%
\pgfsetfillcolor{currentfill}%
\pgfsetfillopacity{0.800000}%
\pgfsetlinewidth{0.000000pt}%
\definecolor{currentstroke}{rgb}{0.000000,0.000000,0.000000}%
\pgfsetstrokecolor{currentstroke}%
\pgfsetdash{}{0pt}%
\pgfpathmoveto{\pgfqpoint{3.071240in}{3.232193in}}%
\pgfpathlineto{\pgfqpoint{3.084723in}{3.212378in}}%
\pgfpathlineto{\pgfqpoint{3.098198in}{3.192885in}}%
\pgfpathlineto{\pgfqpoint{3.111665in}{3.173711in}}%
\pgfpathlineto{\pgfqpoint{3.125124in}{3.154854in}}%
\pgfpathlineto{\pgfqpoint{3.133079in}{3.169233in}}%
\pgfpathlineto{\pgfqpoint{3.141027in}{3.183790in}}%
\pgfpathlineto{\pgfqpoint{3.148968in}{3.198526in}}%
\pgfpathlineto{\pgfqpoint{3.156903in}{3.213446in}}%
\pgfpathlineto{\pgfqpoint{3.143453in}{3.232528in}}%
\pgfpathlineto{\pgfqpoint{3.129996in}{3.251927in}}%
\pgfpathlineto{\pgfqpoint{3.116531in}{3.271646in}}%
\pgfpathlineto{\pgfqpoint{3.103059in}{3.291687in}}%
\pgfpathlineto{\pgfqpoint{3.095115in}{3.276529in}}%
\pgfpathlineto{\pgfqpoint{3.087164in}{3.261563in}}%
\pgfpathlineto{\pgfqpoint{3.079205in}{3.246786in}}%
\pgfpathlineto{\pgfqpoint{3.071240in}{3.232193in}}%
\pgfpathclose%
\pgfusepath{fill}%
\end{pgfscope}%
\begin{pgfscope}%
\pgfpathrectangle{\pgfqpoint{1.150000in}{0.150000in}}{\pgfqpoint{5.700000in}{5.700000in}}%
\pgfusepath{clip}%
\pgfsetbuttcap%
\pgfsetroundjoin%
\definecolor{currentfill}{rgb}{0.263663,0.237631,0.518762}%
\pgfsetfillcolor{currentfill}%
\pgfsetfillopacity{0.800000}%
\pgfsetlinewidth{0.000000pt}%
\definecolor{currentstroke}{rgb}{0.000000,0.000000,0.000000}%
\pgfsetstrokecolor{currentstroke}%
\pgfsetdash{}{0pt}%
\pgfpathmoveto{\pgfqpoint{3.446607in}{2.790363in}}%
\pgfpathlineto{\pgfqpoint{3.459964in}{2.778595in}}%
\pgfpathlineto{\pgfqpoint{3.473320in}{2.767083in}}%
\pgfpathlineto{\pgfqpoint{3.486676in}{2.755824in}}%
\pgfpathlineto{\pgfqpoint{3.500030in}{2.744818in}}%
\pgfpathlineto{\pgfqpoint{3.507918in}{2.757724in}}%
\pgfpathlineto{\pgfqpoint{3.515801in}{2.770751in}}%
\pgfpathlineto{\pgfqpoint{3.523678in}{2.783902in}}%
\pgfpathlineto{\pgfqpoint{3.531550in}{2.797181in}}%
\pgfpathlineto{\pgfqpoint{3.518204in}{2.808408in}}%
\pgfpathlineto{\pgfqpoint{3.504857in}{2.819887in}}%
\pgfpathlineto{\pgfqpoint{3.491510in}{2.831621in}}%
\pgfpathlineto{\pgfqpoint{3.478162in}{2.843610in}}%
\pgfpathlineto{\pgfqpoint{3.470281in}{2.830099in}}%
\pgfpathlineto{\pgfqpoint{3.462396in}{2.816722in}}%
\pgfpathlineto{\pgfqpoint{3.454504in}{2.803478in}}%
\pgfpathlineto{\pgfqpoint{3.446607in}{2.790363in}}%
\pgfpathclose%
\pgfusepath{fill}%
\end{pgfscope}%
\begin{pgfscope}%
\pgfpathrectangle{\pgfqpoint{1.150000in}{0.150000in}}{\pgfqpoint{5.700000in}{5.700000in}}%
\pgfusepath{clip}%
\pgfsetbuttcap%
\pgfsetroundjoin%
\definecolor{currentfill}{rgb}{0.266580,0.228262,0.514349}%
\pgfsetfillcolor{currentfill}%
\pgfsetfillopacity{0.800000}%
\pgfsetlinewidth{0.000000pt}%
\definecolor{currentstroke}{rgb}{0.000000,0.000000,0.000000}%
\pgfsetstrokecolor{currentstroke}%
\pgfsetdash{}{0pt}%
\pgfpathmoveto{\pgfqpoint{4.083745in}{2.756291in}}%
\pgfpathlineto{\pgfqpoint{4.097146in}{2.751390in}}%
\pgfpathlineto{\pgfqpoint{4.110553in}{2.746698in}}%
\pgfpathlineto{\pgfqpoint{4.123965in}{2.742213in}}%
\pgfpathlineto{\pgfqpoint{4.137383in}{2.737935in}}%
\pgfpathlineto{\pgfqpoint{4.145106in}{2.750612in}}%
\pgfpathlineto{\pgfqpoint{4.152825in}{2.763411in}}%
\pgfpathlineto{\pgfqpoint{4.160540in}{2.776338in}}%
\pgfpathlineto{\pgfqpoint{4.168251in}{2.789397in}}%
\pgfpathlineto{\pgfqpoint{4.154841in}{2.794052in}}%
\pgfpathlineto{\pgfqpoint{4.141437in}{2.798913in}}%
\pgfpathlineto{\pgfqpoint{4.128039in}{2.803981in}}%
\pgfpathlineto{\pgfqpoint{4.114646in}{2.809258in}}%
\pgfpathlineto{\pgfqpoint{4.106927in}{2.795811in}}%
\pgfpathlineto{\pgfqpoint{4.099203in}{2.782504in}}%
\pgfpathlineto{\pgfqpoint{4.091476in}{2.769332in}}%
\pgfpathlineto{\pgfqpoint{4.083745in}{2.756291in}}%
\pgfpathclose%
\pgfusepath{fill}%
\end{pgfscope}%
\begin{pgfscope}%
\pgfpathrectangle{\pgfqpoint{1.150000in}{0.150000in}}{\pgfqpoint{5.700000in}{5.700000in}}%
\pgfusepath{clip}%
\pgfsetbuttcap%
\pgfsetroundjoin%
\definecolor{currentfill}{rgb}{0.271828,0.209303,0.504434}%
\pgfsetfillcolor{currentfill}%
\pgfsetfillopacity{0.800000}%
\pgfsetlinewidth{0.000000pt}%
\definecolor{currentstroke}{rgb}{0.000000,0.000000,0.000000}%
\pgfsetstrokecolor{currentstroke}%
\pgfsetdash{}{0pt}%
\pgfpathmoveto{\pgfqpoint{3.638320in}{2.716245in}}%
\pgfpathlineto{\pgfqpoint{3.651670in}{2.707214in}}%
\pgfpathlineto{\pgfqpoint{3.665021in}{2.698420in}}%
\pgfpathlineto{\pgfqpoint{3.678373in}{2.689860in}}%
\pgfpathlineto{\pgfqpoint{3.691728in}{2.681534in}}%
\pgfpathlineto{\pgfqpoint{3.699569in}{2.694222in}}%
\pgfpathlineto{\pgfqpoint{3.707406in}{2.707020in}}%
\pgfpathlineto{\pgfqpoint{3.715238in}{2.719931in}}%
\pgfpathlineto{\pgfqpoint{3.723065in}{2.732958in}}%
\pgfpathlineto{\pgfqpoint{3.709719in}{2.741536in}}%
\pgfpathlineto{\pgfqpoint{3.696374in}{2.750347in}}%
\pgfpathlineto{\pgfqpoint{3.683031in}{2.759393in}}%
\pgfpathlineto{\pgfqpoint{3.669690in}{2.768676in}}%
\pgfpathlineto{\pgfqpoint{3.661855in}{2.755386in}}%
\pgfpathlineto{\pgfqpoint{3.654015in}{2.742219in}}%
\pgfpathlineto{\pgfqpoint{3.646170in}{2.729173in}}%
\pgfpathlineto{\pgfqpoint{3.638320in}{2.716245in}}%
\pgfpathclose%
\pgfusepath{fill}%
\end{pgfscope}%
\begin{pgfscope}%
\pgfpathrectangle{\pgfqpoint{1.150000in}{0.150000in}}{\pgfqpoint{5.700000in}{5.700000in}}%
\pgfusepath{clip}%
\pgfsetbuttcap%
\pgfsetroundjoin%
\definecolor{currentfill}{rgb}{0.177423,0.437527,0.557565}%
\pgfsetfillcolor{currentfill}%
\pgfsetfillopacity{0.800000}%
\pgfsetlinewidth{0.000000pt}%
\definecolor{currentstroke}{rgb}{0.000000,0.000000,0.000000}%
\pgfsetstrokecolor{currentstroke}%
\pgfsetdash{}{0pt}%
\pgfpathmoveto{\pgfqpoint{5.097950in}{3.293881in}}%
\pgfpathlineto{\pgfqpoint{5.111599in}{3.291564in}}%
\pgfpathlineto{\pgfqpoint{5.125259in}{3.289423in}}%
\pgfpathlineto{\pgfqpoint{5.138929in}{3.287459in}}%
\pgfpathlineto{\pgfqpoint{5.152609in}{3.285672in}}%
\pgfpathlineto{\pgfqpoint{5.160136in}{3.301890in}}%
\pgfpathlineto{\pgfqpoint{5.167668in}{3.318450in}}%
\pgfpathlineto{\pgfqpoint{5.175204in}{3.335361in}}%
\pgfpathlineto{\pgfqpoint{5.161537in}{3.337711in}}%
\pgfpathlineto{\pgfqpoint{5.147881in}{3.340237in}}%
\pgfpathlineto{\pgfqpoint{5.134234in}{3.342940in}}%
\pgfpathlineto{\pgfqpoint{5.120598in}{3.345820in}}%
\pgfpathlineto{\pgfqpoint{5.113044in}{3.328152in}}%
\pgfpathlineto{\pgfqpoint{5.105495in}{3.310842in}}%
\pgfpathlineto{\pgfqpoint{5.097950in}{3.293881in}}%
\pgfpathclose%
\pgfusepath{fill}%
\end{pgfscope}%
\begin{pgfscope}%
\pgfpathrectangle{\pgfqpoint{1.150000in}{0.150000in}}{\pgfqpoint{5.700000in}{5.700000in}}%
\pgfusepath{clip}%
\pgfsetbuttcap%
\pgfsetroundjoin%
\definecolor{currentfill}{rgb}{0.273006,0.204520,0.501721}%
\pgfsetfillcolor{currentfill}%
\pgfsetfillopacity{0.800000}%
\pgfsetlinewidth{0.000000pt}%
\definecolor{currentstroke}{rgb}{0.000000,0.000000,0.000000}%
\pgfsetstrokecolor{currentstroke}%
\pgfsetdash{}{0pt}%
\pgfpathmoveto{\pgfqpoint{3.776472in}{2.700954in}}%
\pgfpathlineto{\pgfqpoint{3.789830in}{2.693524in}}%
\pgfpathlineto{\pgfqpoint{3.803191in}{2.686319in}}%
\pgfpathlineto{\pgfqpoint{3.816555in}{2.679338in}}%
\pgfpathlineto{\pgfqpoint{3.829922in}{2.672580in}}%
\pgfpathlineto{\pgfqpoint{3.837729in}{2.685184in}}%
\pgfpathlineto{\pgfqpoint{3.845531in}{2.697895in}}%
\pgfpathlineto{\pgfqpoint{3.853328in}{2.710717in}}%
\pgfpathlineto{\pgfqpoint{3.861121in}{2.723652in}}%
\pgfpathlineto{\pgfqpoint{3.847761in}{2.730693in}}%
\pgfpathlineto{\pgfqpoint{3.834405in}{2.737956in}}%
\pgfpathlineto{\pgfqpoint{3.821052in}{2.745444in}}%
\pgfpathlineto{\pgfqpoint{3.807701in}{2.753157in}}%
\pgfpathlineto{\pgfqpoint{3.799901in}{2.739927in}}%
\pgfpathlineto{\pgfqpoint{3.792096in}{2.726819in}}%
\pgfpathlineto{\pgfqpoint{3.784287in}{2.713829in}}%
\pgfpathlineto{\pgfqpoint{3.776472in}{2.700954in}}%
\pgfpathclose%
\pgfusepath{fill}%
\end{pgfscope}%
\begin{pgfscope}%
\pgfpathrectangle{\pgfqpoint{1.150000in}{0.150000in}}{\pgfqpoint{5.700000in}{5.700000in}}%
\pgfusepath{clip}%
\pgfsetbuttcap%
\pgfsetroundjoin%
\definecolor{currentfill}{rgb}{0.269308,0.218818,0.509577}%
\pgfsetfillcolor{currentfill}%
\pgfsetfillopacity{0.800000}%
\pgfsetlinewidth{0.000000pt}%
\definecolor{currentstroke}{rgb}{0.000000,0.000000,0.000000}%
\pgfsetstrokecolor{currentstroke}%
\pgfsetdash{}{0pt}%
\pgfpathmoveto{\pgfqpoint{3.999197in}{2.725667in}}%
\pgfpathlineto{\pgfqpoint{4.012586in}{2.720270in}}%
\pgfpathlineto{\pgfqpoint{4.025979in}{2.715084in}}%
\pgfpathlineto{\pgfqpoint{4.039377in}{2.710110in}}%
\pgfpathlineto{\pgfqpoint{4.052781in}{2.705346in}}%
\pgfpathlineto{\pgfqpoint{4.060528in}{2.717907in}}%
\pgfpathlineto{\pgfqpoint{4.068271in}{2.730582in}}%
\pgfpathlineto{\pgfqpoint{4.076010in}{2.743375in}}%
\pgfpathlineto{\pgfqpoint{4.083745in}{2.756291in}}%
\pgfpathlineto{\pgfqpoint{4.070350in}{2.761400in}}%
\pgfpathlineto{\pgfqpoint{4.056959in}{2.766719in}}%
\pgfpathlineto{\pgfqpoint{4.043574in}{2.772250in}}%
\pgfpathlineto{\pgfqpoint{4.030193in}{2.777993in}}%
\pgfpathlineto{\pgfqpoint{4.022450in}{2.764721in}}%
\pgfpathlineto{\pgfqpoint{4.014704in}{2.751579in}}%
\pgfpathlineto{\pgfqpoint{4.006953in}{2.738562in}}%
\pgfpathlineto{\pgfqpoint{3.999197in}{2.725667in}}%
\pgfpathclose%
\pgfusepath{fill}%
\end{pgfscope}%
\begin{pgfscope}%
\pgfpathrectangle{\pgfqpoint{1.150000in}{0.150000in}}{\pgfqpoint{5.700000in}{5.700000in}}%
\pgfusepath{clip}%
\pgfsetbuttcap%
\pgfsetroundjoin%
\definecolor{currentfill}{rgb}{0.179019,0.433756,0.557430}%
\pgfsetfillcolor{currentfill}%
\pgfsetfillopacity{0.800000}%
\pgfsetlinewidth{0.000000pt}%
\definecolor{currentstroke}{rgb}{0.000000,0.000000,0.000000}%
\pgfsetstrokecolor{currentstroke}%
\pgfsetdash{}{0pt}%
\pgfpathmoveto{\pgfqpoint{3.017222in}{3.314739in}}%
\pgfpathlineto{\pgfqpoint{3.030740in}{3.293604in}}%
\pgfpathlineto{\pgfqpoint{3.044249in}{3.272803in}}%
\pgfpathlineto{\pgfqpoint{3.057749in}{3.252334in}}%
\pgfpathlineto{\pgfqpoint{3.071240in}{3.232193in}}%
\pgfpathlineto{\pgfqpoint{3.079205in}{3.246786in}}%
\pgfpathlineto{\pgfqpoint{3.087164in}{3.261563in}}%
\pgfpathlineto{\pgfqpoint{3.095115in}{3.276529in}}%
\pgfpathlineto{\pgfqpoint{3.103059in}{3.291687in}}%
\pgfpathlineto{\pgfqpoint{3.089578in}{3.312053in}}%
\pgfpathlineto{\pgfqpoint{3.076089in}{3.332749in}}%
\pgfpathlineto{\pgfqpoint{3.062591in}{3.353777in}}%
\pgfpathlineto{\pgfqpoint{3.049083in}{3.375139in}}%
\pgfpathlineto{\pgfqpoint{3.041129in}{3.359743in}}%
\pgfpathlineto{\pgfqpoint{3.033168in}{3.344546in}}%
\pgfpathlineto{\pgfqpoint{3.025199in}{3.329546in}}%
\pgfpathlineto{\pgfqpoint{3.017222in}{3.314739in}}%
\pgfpathclose%
\pgfusepath{fill}%
\end{pgfscope}%
\begin{pgfscope}%
\pgfpathrectangle{\pgfqpoint{1.150000in}{0.150000in}}{\pgfqpoint{5.700000in}{5.700000in}}%
\pgfusepath{clip}%
\pgfsetbuttcap%
\pgfsetroundjoin%
\definecolor{currentfill}{rgb}{0.267968,0.223549,0.512008}%
\pgfsetfillcolor{currentfill}%
\pgfsetfillopacity{0.800000}%
\pgfsetlinewidth{0.000000pt}%
\definecolor{currentstroke}{rgb}{0.000000,0.000000,0.000000}%
\pgfsetstrokecolor{currentstroke}%
\pgfsetdash{}{0pt}%
\pgfpathmoveto{\pgfqpoint{3.500030in}{2.744818in}}%
\pgfpathlineto{\pgfqpoint{3.513384in}{2.734062in}}%
\pgfpathlineto{\pgfqpoint{3.526738in}{2.723555in}}%
\pgfpathlineto{\pgfqpoint{3.540092in}{2.713296in}}%
\pgfpathlineto{\pgfqpoint{3.553446in}{2.703281in}}%
\pgfpathlineto{\pgfqpoint{3.561326in}{2.715978in}}%
\pgfpathlineto{\pgfqpoint{3.569200in}{2.728788in}}%
\pgfpathlineto{\pgfqpoint{3.577068in}{2.741715in}}%
\pgfpathlineto{\pgfqpoint{3.584932in}{2.754761in}}%
\pgfpathlineto{\pgfqpoint{3.571586in}{2.764996in}}%
\pgfpathlineto{\pgfqpoint{3.558241in}{2.775477in}}%
\pgfpathlineto{\pgfqpoint{3.544895in}{2.786204in}}%
\pgfpathlineto{\pgfqpoint{3.531550in}{2.797181in}}%
\pgfpathlineto{\pgfqpoint{3.523678in}{2.783902in}}%
\pgfpathlineto{\pgfqpoint{3.515801in}{2.770751in}}%
\pgfpathlineto{\pgfqpoint{3.507918in}{2.757724in}}%
\pgfpathlineto{\pgfqpoint{3.500030in}{2.744818in}}%
\pgfpathclose%
\pgfusepath{fill}%
\end{pgfscope}%
\begin{pgfscope}%
\pgfpathrectangle{\pgfqpoint{1.150000in}{0.150000in}}{\pgfqpoint{5.700000in}{5.700000in}}%
\pgfusepath{clip}%
\pgfsetbuttcap%
\pgfsetroundjoin%
\definecolor{currentfill}{rgb}{0.235526,0.309527,0.542944}%
\pgfsetfillcolor{currentfill}%
\pgfsetfillopacity{0.800000}%
\pgfsetlinewidth{0.000000pt}%
\definecolor{currentstroke}{rgb}{0.000000,0.000000,0.000000}%
\pgfsetstrokecolor{currentstroke}%
\pgfsetdash{}{0pt}%
\pgfpathmoveto{\pgfqpoint{4.560075in}{2.932919in}}%
\pgfpathlineto{\pgfqpoint{4.573597in}{2.930594in}}%
\pgfpathlineto{\pgfqpoint{4.587129in}{2.928459in}}%
\pgfpathlineto{\pgfqpoint{4.600668in}{2.926513in}}%
\pgfpathlineto{\pgfqpoint{4.614217in}{2.924756in}}%
\pgfpathlineto{\pgfqpoint{4.621819in}{2.937709in}}%
\pgfpathlineto{\pgfqpoint{4.629420in}{2.950843in}}%
\pgfpathlineto{\pgfqpoint{4.637018in}{2.964165in}}%
\pgfpathlineto{\pgfqpoint{4.644615in}{2.977682in}}%
\pgfpathlineto{\pgfqpoint{4.631078in}{2.979973in}}%
\pgfpathlineto{\pgfqpoint{4.617549in}{2.982452in}}%
\pgfpathlineto{\pgfqpoint{4.604030in}{2.985120in}}%
\pgfpathlineto{\pgfqpoint{4.590518in}{2.987978in}}%
\pgfpathlineto{\pgfqpoint{4.582910in}{2.973917in}}%
\pgfpathlineto{\pgfqpoint{4.575300in}{2.960058in}}%
\pgfpathlineto{\pgfqpoint{4.567688in}{2.946394in}}%
\pgfpathlineto{\pgfqpoint{4.560075in}{2.932919in}}%
\pgfpathclose%
\pgfusepath{fill}%
\end{pgfscope}%
\begin{pgfscope}%
\pgfpathrectangle{\pgfqpoint{1.150000in}{0.150000in}}{\pgfqpoint{5.700000in}{5.700000in}}%
\pgfusepath{clip}%
\pgfsetbuttcap%
\pgfsetroundjoin%
\definecolor{currentfill}{rgb}{0.227802,0.326594,0.546532}%
\pgfsetfillcolor{currentfill}%
\pgfsetfillopacity{0.800000}%
\pgfsetlinewidth{0.000000pt}%
\definecolor{currentstroke}{rgb}{0.000000,0.000000,0.000000}%
\pgfsetstrokecolor{currentstroke}%
\pgfsetdash{}{0pt}%
\pgfpathmoveto{\pgfqpoint{4.644615in}{2.977682in}}%
\pgfpathlineto{\pgfqpoint{4.658160in}{2.975579in}}%
\pgfpathlineto{\pgfqpoint{4.671715in}{2.973664in}}%
\pgfpathlineto{\pgfqpoint{4.685279in}{2.971935in}}%
\pgfpathlineto{\pgfqpoint{4.698851in}{2.970393in}}%
\pgfpathlineto{\pgfqpoint{4.706434in}{2.983557in}}%
\pgfpathlineto{\pgfqpoint{4.714016in}{2.996919in}}%
\pgfpathlineto{\pgfqpoint{4.721596in}{3.010487in}}%
\pgfpathlineto{\pgfqpoint{4.729175in}{3.024268in}}%
\pgfpathlineto{\pgfqpoint{4.715615in}{3.026375in}}%
\pgfpathlineto{\pgfqpoint{4.702064in}{3.028668in}}%
\pgfpathlineto{\pgfqpoint{4.688521in}{3.031148in}}%
\pgfpathlineto{\pgfqpoint{4.674988in}{3.033815in}}%
\pgfpathlineto{\pgfqpoint{4.667396in}{3.019459in}}%
\pgfpathlineto{\pgfqpoint{4.659804in}{3.005322in}}%
\pgfpathlineto{\pgfqpoint{4.652210in}{2.991398in}}%
\pgfpathlineto{\pgfqpoint{4.644615in}{2.977682in}}%
\pgfpathclose%
\pgfusepath{fill}%
\end{pgfscope}%
\begin{pgfscope}%
\pgfpathrectangle{\pgfqpoint{1.150000in}{0.150000in}}{\pgfqpoint{5.700000in}{5.700000in}}%
\pgfusepath{clip}%
\pgfsetbuttcap%
\pgfsetroundjoin%
\definecolor{currentfill}{rgb}{0.243113,0.292092,0.538516}%
\pgfsetfillcolor{currentfill}%
\pgfsetfillopacity{0.800000}%
\pgfsetlinewidth{0.000000pt}%
\definecolor{currentstroke}{rgb}{0.000000,0.000000,0.000000}%
\pgfsetstrokecolor{currentstroke}%
\pgfsetdash{}{0pt}%
\pgfpathmoveto{\pgfqpoint{4.475547in}{2.889998in}}%
\pgfpathlineto{\pgfqpoint{4.489047in}{2.887409in}}%
\pgfpathlineto{\pgfqpoint{4.502556in}{2.885013in}}%
\pgfpathlineto{\pgfqpoint{4.516073in}{2.882808in}}%
\pgfpathlineto{\pgfqpoint{4.529598in}{2.880795in}}%
\pgfpathlineto{\pgfqpoint{4.537221in}{2.893572in}}%
\pgfpathlineto{\pgfqpoint{4.544841in}{2.906514in}}%
\pgfpathlineto{\pgfqpoint{4.552459in}{2.919628in}}%
\pgfpathlineto{\pgfqpoint{4.560075in}{2.932919in}}%
\pgfpathlineto{\pgfqpoint{4.546561in}{2.935435in}}%
\pgfpathlineto{\pgfqpoint{4.533055in}{2.938141in}}%
\pgfpathlineto{\pgfqpoint{4.519557in}{2.941040in}}%
\pgfpathlineto{\pgfqpoint{4.506067in}{2.944130in}}%
\pgfpathlineto{\pgfqpoint{4.498440in}{2.930325in}}%
\pgfpathlineto{\pgfqpoint{4.490812in}{2.916706in}}%
\pgfpathlineto{\pgfqpoint{4.483181in}{2.903265in}}%
\pgfpathlineto{\pgfqpoint{4.475547in}{2.889998in}}%
\pgfpathclose%
\pgfusepath{fill}%
\end{pgfscope}%
\begin{pgfscope}%
\pgfpathrectangle{\pgfqpoint{1.150000in}{0.150000in}}{\pgfqpoint{5.700000in}{5.700000in}}%
\pgfusepath{clip}%
\pgfsetbuttcap%
\pgfsetroundjoin%
\definecolor{currentfill}{rgb}{0.220057,0.343307,0.549413}%
\pgfsetfillcolor{currentfill}%
\pgfsetfillopacity{0.800000}%
\pgfsetlinewidth{0.000000pt}%
\definecolor{currentstroke}{rgb}{0.000000,0.000000,0.000000}%
\pgfsetstrokecolor{currentstroke}%
\pgfsetdash{}{0pt}%
\pgfpathmoveto{\pgfqpoint{4.729175in}{3.024268in}}%
\pgfpathlineto{\pgfqpoint{4.742744in}{3.022346in}}%
\pgfpathlineto{\pgfqpoint{4.756323in}{3.020610in}}%
\pgfpathlineto{\pgfqpoint{4.769911in}{3.019058in}}%
\pgfpathlineto{\pgfqpoint{4.783509in}{3.017691in}}%
\pgfpathlineto{\pgfqpoint{4.791074in}{3.031106in}}%
\pgfpathlineto{\pgfqpoint{4.798638in}{3.044738in}}%
\pgfpathlineto{\pgfqpoint{4.806201in}{3.058595in}}%
\pgfpathlineto{\pgfqpoint{4.813765in}{3.072685in}}%
\pgfpathlineto{\pgfqpoint{4.800181in}{3.074649in}}%
\pgfpathlineto{\pgfqpoint{4.786606in}{3.076797in}}%
\pgfpathlineto{\pgfqpoint{4.773041in}{3.079129in}}%
\pgfpathlineto{\pgfqpoint{4.759484in}{3.081646in}}%
\pgfpathlineto{\pgfqpoint{4.751908in}{3.066949in}}%
\pgfpathlineto{\pgfqpoint{4.744331in}{3.052492in}}%
\pgfpathlineto{\pgfqpoint{4.736753in}{3.038267in}}%
\pgfpathlineto{\pgfqpoint{4.729175in}{3.024268in}}%
\pgfpathclose%
\pgfusepath{fill}%
\end{pgfscope}%
\begin{pgfscope}%
\pgfpathrectangle{\pgfqpoint{1.150000in}{0.150000in}}{\pgfqpoint{5.700000in}{5.700000in}}%
\pgfusepath{clip}%
\pgfsetbuttcap%
\pgfsetroundjoin%
\definecolor{currentfill}{rgb}{0.235526,0.309527,0.542944}%
\pgfsetfillcolor{currentfill}%
\pgfsetfillopacity{0.800000}%
\pgfsetlinewidth{0.000000pt}%
\definecolor{currentstroke}{rgb}{0.000000,0.000000,0.000000}%
\pgfsetstrokecolor{currentstroke}%
\pgfsetdash{}{0pt}%
\pgfpathmoveto{\pgfqpoint{3.200763in}{2.960783in}}%
\pgfpathlineto{\pgfqpoint{3.214178in}{2.944840in}}%
\pgfpathlineto{\pgfqpoint{3.227588in}{2.929185in}}%
\pgfpathlineto{\pgfqpoint{3.240994in}{2.913818in}}%
\pgfpathlineto{\pgfqpoint{3.254396in}{2.898736in}}%
\pgfpathlineto{\pgfqpoint{3.262346in}{2.911894in}}%
\pgfpathlineto{\pgfqpoint{3.270290in}{2.925195in}}%
\pgfpathlineto{\pgfqpoint{3.278227in}{2.938641in}}%
\pgfpathlineto{\pgfqpoint{3.286158in}{2.952236in}}%
\pgfpathlineto{\pgfqpoint{3.272767in}{2.967509in}}%
\pgfpathlineto{\pgfqpoint{3.259371in}{2.983067in}}%
\pgfpathlineto{\pgfqpoint{3.245971in}{2.998912in}}%
\pgfpathlineto{\pgfqpoint{3.232566in}{3.015047in}}%
\pgfpathlineto{\pgfqpoint{3.224625in}{3.001250in}}%
\pgfpathlineto{\pgfqpoint{3.216678in}{2.987609in}}%
\pgfpathlineto{\pgfqpoint{3.208723in}{2.974120in}}%
\pgfpathlineto{\pgfqpoint{3.200763in}{2.960783in}}%
\pgfpathclose%
\pgfusepath{fill}%
\end{pgfscope}%
\begin{pgfscope}%
\pgfpathrectangle{\pgfqpoint{1.150000in}{0.150000in}}{\pgfqpoint{5.700000in}{5.700000in}}%
\pgfusepath{clip}%
\pgfsetbuttcap%
\pgfsetroundjoin%
\definecolor{currentfill}{rgb}{0.250425,0.274290,0.533103}%
\pgfsetfillcolor{currentfill}%
\pgfsetfillopacity{0.800000}%
\pgfsetlinewidth{0.000000pt}%
\definecolor{currentstroke}{rgb}{0.000000,0.000000,0.000000}%
\pgfsetstrokecolor{currentstroke}%
\pgfsetdash{}{0pt}%
\pgfpathmoveto{\pgfqpoint{4.391023in}{2.848960in}}%
\pgfpathlineto{\pgfqpoint{4.404502in}{2.846065in}}%
\pgfpathlineto{\pgfqpoint{4.417989in}{2.843365in}}%
\pgfpathlineto{\pgfqpoint{4.431483in}{2.840860in}}%
\pgfpathlineto{\pgfqpoint{4.444986in}{2.838549in}}%
\pgfpathlineto{\pgfqpoint{4.452631in}{2.851179in}}%
\pgfpathlineto{\pgfqpoint{4.460272in}{2.863961in}}%
\pgfpathlineto{\pgfqpoint{4.467911in}{2.876898in}}%
\pgfpathlineto{\pgfqpoint{4.475547in}{2.889998in}}%
\pgfpathlineto{\pgfqpoint{4.462055in}{2.892780in}}%
\pgfpathlineto{\pgfqpoint{4.448570in}{2.895755in}}%
\pgfpathlineto{\pgfqpoint{4.435093in}{2.898925in}}%
\pgfpathlineto{\pgfqpoint{4.421624in}{2.902291in}}%
\pgfpathlineto{\pgfqpoint{4.413978in}{2.888709in}}%
\pgfpathlineto{\pgfqpoint{4.406329in}{2.875297in}}%
\pgfpathlineto{\pgfqpoint{4.398677in}{2.862049in}}%
\pgfpathlineto{\pgfqpoint{4.391023in}{2.848960in}}%
\pgfpathclose%
\pgfusepath{fill}%
\end{pgfscope}%
\begin{pgfscope}%
\pgfpathrectangle{\pgfqpoint{1.150000in}{0.150000in}}{\pgfqpoint{5.700000in}{5.700000in}}%
\pgfusepath{clip}%
\pgfsetbuttcap%
\pgfsetroundjoin%
\definecolor{currentfill}{rgb}{0.246811,0.283237,0.535941}%
\pgfsetfillcolor{currentfill}%
\pgfsetfillopacity{0.800000}%
\pgfsetlinewidth{0.000000pt}%
\definecolor{currentstroke}{rgb}{0.000000,0.000000,0.000000}%
\pgfsetstrokecolor{currentstroke}%
\pgfsetdash{}{0pt}%
\pgfpathmoveto{\pgfqpoint{3.254396in}{2.898736in}}%
\pgfpathlineto{\pgfqpoint{3.267793in}{2.883935in}}%
\pgfpathlineto{\pgfqpoint{3.281187in}{2.869415in}}%
\pgfpathlineto{\pgfqpoint{3.294577in}{2.855173in}}%
\pgfpathlineto{\pgfqpoint{3.307963in}{2.841206in}}%
\pgfpathlineto{\pgfqpoint{3.315903in}{2.854186in}}%
\pgfpathlineto{\pgfqpoint{3.323836in}{2.867301in}}%
\pgfpathlineto{\pgfqpoint{3.331764in}{2.880553in}}%
\pgfpathlineto{\pgfqpoint{3.339686in}{2.893945in}}%
\pgfpathlineto{\pgfqpoint{3.326309in}{2.908102in}}%
\pgfpathlineto{\pgfqpoint{3.312929in}{2.922534in}}%
\pgfpathlineto{\pgfqpoint{3.299545in}{2.937245in}}%
\pgfpathlineto{\pgfqpoint{3.286158in}{2.952236in}}%
\pgfpathlineto{\pgfqpoint{3.278227in}{2.938641in}}%
\pgfpathlineto{\pgfqpoint{3.270290in}{2.925195in}}%
\pgfpathlineto{\pgfqpoint{3.262346in}{2.911894in}}%
\pgfpathlineto{\pgfqpoint{3.254396in}{2.898736in}}%
\pgfpathclose%
\pgfusepath{fill}%
\end{pgfscope}%
\begin{pgfscope}%
\pgfpathrectangle{\pgfqpoint{1.150000in}{0.150000in}}{\pgfqpoint{5.700000in}{5.700000in}}%
\pgfusepath{clip}%
\pgfsetbuttcap%
\pgfsetroundjoin%
\definecolor{currentfill}{rgb}{0.210503,0.363727,0.552206}%
\pgfsetfillcolor{currentfill}%
\pgfsetfillopacity{0.800000}%
\pgfsetlinewidth{0.000000pt}%
\definecolor{currentstroke}{rgb}{0.000000,0.000000,0.000000}%
\pgfsetstrokecolor{currentstroke}%
\pgfsetdash{}{0pt}%
\pgfpathmoveto{\pgfqpoint{4.813765in}{3.072685in}}%
\pgfpathlineto{\pgfqpoint{4.827358in}{3.070904in}}%
\pgfpathlineto{\pgfqpoint{4.840961in}{3.069307in}}%
\pgfpathlineto{\pgfqpoint{4.854574in}{3.067892in}}%
\pgfpathlineto{\pgfqpoint{4.868197in}{3.066659in}}%
\pgfpathlineto{\pgfqpoint{4.875746in}{3.080371in}}%
\pgfpathlineto{\pgfqpoint{4.883294in}{3.094321in}}%
\pgfpathlineto{\pgfqpoint{4.890843in}{3.108516in}}%
\pgfpathlineto{\pgfqpoint{4.898393in}{3.122964in}}%
\pgfpathlineto{\pgfqpoint{4.884785in}{3.124825in}}%
\pgfpathlineto{\pgfqpoint{4.871186in}{3.126868in}}%
\pgfpathlineto{\pgfqpoint{4.857598in}{3.129093in}}%
\pgfpathlineto{\pgfqpoint{4.844018in}{3.131501in}}%
\pgfpathlineto{\pgfqpoint{4.836454in}{3.116414in}}%
\pgfpathlineto{\pgfqpoint{4.828891in}{3.101587in}}%
\pgfpathlineto{\pgfqpoint{4.821328in}{3.087013in}}%
\pgfpathlineto{\pgfqpoint{4.813765in}{3.072685in}}%
\pgfpathclose%
\pgfusepath{fill}%
\end{pgfscope}%
\begin{pgfscope}%
\pgfpathrectangle{\pgfqpoint{1.150000in}{0.150000in}}{\pgfqpoint{5.700000in}{5.700000in}}%
\pgfusepath{clip}%
\pgfsetbuttcap%
\pgfsetroundjoin%
\definecolor{currentfill}{rgb}{0.223925,0.334994,0.548053}%
\pgfsetfillcolor{currentfill}%
\pgfsetfillopacity{0.800000}%
\pgfsetlinewidth{0.000000pt}%
\definecolor{currentstroke}{rgb}{0.000000,0.000000,0.000000}%
\pgfsetstrokecolor{currentstroke}%
\pgfsetdash{}{0pt}%
\pgfpathmoveto{\pgfqpoint{3.147048in}{3.027502in}}%
\pgfpathlineto{\pgfqpoint{3.160485in}{3.010376in}}%
\pgfpathlineto{\pgfqpoint{3.173917in}{2.993549in}}%
\pgfpathlineto{\pgfqpoint{3.187342in}{2.977019in}}%
\pgfpathlineto{\pgfqpoint{3.200763in}{2.960783in}}%
\pgfpathlineto{\pgfqpoint{3.208723in}{2.974120in}}%
\pgfpathlineto{\pgfqpoint{3.216678in}{2.987609in}}%
\pgfpathlineto{\pgfqpoint{3.224625in}{3.001250in}}%
\pgfpathlineto{\pgfqpoint{3.232566in}{3.015047in}}%
\pgfpathlineto{\pgfqpoint{3.219156in}{3.031473in}}%
\pgfpathlineto{\pgfqpoint{3.205741in}{3.048195in}}%
\pgfpathlineto{\pgfqpoint{3.192320in}{3.065213in}}%
\pgfpathlineto{\pgfqpoint{3.178894in}{3.082531in}}%
\pgfpathlineto{\pgfqpoint{3.170943in}{3.068531in}}%
\pgfpathlineto{\pgfqpoint{3.162985in}{3.054694in}}%
\pgfpathlineto{\pgfqpoint{3.155020in}{3.041019in}}%
\pgfpathlineto{\pgfqpoint{3.147048in}{3.027502in}}%
\pgfpathclose%
\pgfusepath{fill}%
\end{pgfscope}%
\begin{pgfscope}%
\pgfpathrectangle{\pgfqpoint{1.150000in}{0.150000in}}{\pgfqpoint{5.700000in}{5.700000in}}%
\pgfusepath{clip}%
\pgfsetbuttcap%
\pgfsetroundjoin%
\definecolor{currentfill}{rgb}{0.257322,0.256130,0.526563}%
\pgfsetfillcolor{currentfill}%
\pgfsetfillopacity{0.800000}%
\pgfsetlinewidth{0.000000pt}%
\definecolor{currentstroke}{rgb}{0.000000,0.000000,0.000000}%
\pgfsetstrokecolor{currentstroke}%
\pgfsetdash{}{0pt}%
\pgfpathmoveto{\pgfqpoint{4.306494in}{2.809871in}}%
\pgfpathlineto{\pgfqpoint{4.319953in}{2.806628in}}%
\pgfpathlineto{\pgfqpoint{4.333419in}{2.803582in}}%
\pgfpathlineto{\pgfqpoint{4.346893in}{2.800734in}}%
\pgfpathlineto{\pgfqpoint{4.360374in}{2.798083in}}%
\pgfpathlineto{\pgfqpoint{4.368041in}{2.810591in}}%
\pgfpathlineto{\pgfqpoint{4.375705in}{2.823236in}}%
\pgfpathlineto{\pgfqpoint{4.383365in}{2.836024in}}%
\pgfpathlineto{\pgfqpoint{4.391023in}{2.848960in}}%
\pgfpathlineto{\pgfqpoint{4.377551in}{2.852050in}}%
\pgfpathlineto{\pgfqpoint{4.364087in}{2.855338in}}%
\pgfpathlineto{\pgfqpoint{4.350630in}{2.858822in}}%
\pgfpathlineto{\pgfqpoint{4.337180in}{2.862505in}}%
\pgfpathlineto{\pgfqpoint{4.329513in}{2.849118in}}%
\pgfpathlineto{\pgfqpoint{4.321843in}{2.835887in}}%
\pgfpathlineto{\pgfqpoint{4.314170in}{2.822806in}}%
\pgfpathlineto{\pgfqpoint{4.306494in}{2.809871in}}%
\pgfpathclose%
\pgfusepath{fill}%
\end{pgfscope}%
\begin{pgfscope}%
\pgfpathrectangle{\pgfqpoint{1.150000in}{0.150000in}}{\pgfqpoint{5.700000in}{5.700000in}}%
\pgfusepath{clip}%
\pgfsetbuttcap%
\pgfsetroundjoin%
\definecolor{currentfill}{rgb}{0.271828,0.209303,0.504434}%
\pgfsetfillcolor{currentfill}%
\pgfsetfillopacity{0.800000}%
\pgfsetlinewidth{0.000000pt}%
\definecolor{currentstroke}{rgb}{0.000000,0.000000,0.000000}%
\pgfsetstrokecolor{currentstroke}%
\pgfsetdash{}{0pt}%
\pgfpathmoveto{\pgfqpoint{3.914594in}{2.697698in}}%
\pgfpathlineto{\pgfqpoint{3.927972in}{2.691756in}}%
\pgfpathlineto{\pgfqpoint{3.941355in}{2.686030in}}%
\pgfpathlineto{\pgfqpoint{3.954742in}{2.680519in}}%
\pgfpathlineto{\pgfqpoint{3.968133in}{2.675222in}}%
\pgfpathlineto{\pgfqpoint{3.975906in}{2.687671in}}%
\pgfpathlineto{\pgfqpoint{3.983674in}{2.700225in}}%
\pgfpathlineto{\pgfqpoint{3.991438in}{2.712890in}}%
\pgfpathlineto{\pgfqpoint{3.999197in}{2.725667in}}%
\pgfpathlineto{\pgfqpoint{3.985814in}{2.731278in}}%
\pgfpathlineto{\pgfqpoint{3.972435in}{2.737103in}}%
\pgfpathlineto{\pgfqpoint{3.959060in}{2.743143in}}%
\pgfpathlineto{\pgfqpoint{3.945689in}{2.749399in}}%
\pgfpathlineto{\pgfqpoint{3.937922in}{2.736296in}}%
\pgfpathlineto{\pgfqpoint{3.930151in}{2.723314in}}%
\pgfpathlineto{\pgfqpoint{3.922375in}{2.710449in}}%
\pgfpathlineto{\pgfqpoint{3.914594in}{2.697698in}}%
\pgfpathclose%
\pgfusepath{fill}%
\end{pgfscope}%
\begin{pgfscope}%
\pgfpathrectangle{\pgfqpoint{1.150000in}{0.150000in}}{\pgfqpoint{5.700000in}{5.700000in}}%
\pgfusepath{clip}%
\pgfsetbuttcap%
\pgfsetroundjoin%
\definecolor{currentfill}{rgb}{0.255645,0.260703,0.528312}%
\pgfsetfillcolor{currentfill}%
\pgfsetfillopacity{0.800000}%
\pgfsetlinewidth{0.000000pt}%
\definecolor{currentstroke}{rgb}{0.000000,0.000000,0.000000}%
\pgfsetstrokecolor{currentstroke}%
\pgfsetdash{}{0pt}%
\pgfpathmoveto{\pgfqpoint{3.307963in}{2.841206in}}%
\pgfpathlineto{\pgfqpoint{3.321346in}{2.827513in}}%
\pgfpathlineto{\pgfqpoint{3.334727in}{2.814091in}}%
\pgfpathlineto{\pgfqpoint{3.348105in}{2.800939in}}%
\pgfpathlineto{\pgfqpoint{3.361480in}{2.788053in}}%
\pgfpathlineto{\pgfqpoint{3.369410in}{2.800855in}}%
\pgfpathlineto{\pgfqpoint{3.377334in}{2.813783in}}%
\pgfpathlineto{\pgfqpoint{3.385252in}{2.826841in}}%
\pgfpathlineto{\pgfqpoint{3.393164in}{2.840031in}}%
\pgfpathlineto{\pgfqpoint{3.379798in}{2.853107in}}%
\pgfpathlineto{\pgfqpoint{3.366430in}{2.866449in}}%
\pgfpathlineto{\pgfqpoint{3.353059in}{2.880061in}}%
\pgfpathlineto{\pgfqpoint{3.339686in}{2.893945in}}%
\pgfpathlineto{\pgfqpoint{3.331764in}{2.880553in}}%
\pgfpathlineto{\pgfqpoint{3.323836in}{2.867301in}}%
\pgfpathlineto{\pgfqpoint{3.315903in}{2.854186in}}%
\pgfpathlineto{\pgfqpoint{3.307963in}{2.841206in}}%
\pgfpathclose%
\pgfusepath{fill}%
\end{pgfscope}%
\begin{pgfscope}%
\pgfpathrectangle{\pgfqpoint{1.150000in}{0.150000in}}{\pgfqpoint{5.700000in}{5.700000in}}%
\pgfusepath{clip}%
\pgfsetbuttcap%
\pgfsetroundjoin%
\definecolor{currentfill}{rgb}{0.201239,0.383670,0.554294}%
\pgfsetfillcolor{currentfill}%
\pgfsetfillopacity{0.800000}%
\pgfsetlinewidth{0.000000pt}%
\definecolor{currentstroke}{rgb}{0.000000,0.000000,0.000000}%
\pgfsetstrokecolor{currentstroke}%
\pgfsetdash{}{0pt}%
\pgfpathmoveto{\pgfqpoint{4.898393in}{3.122964in}}%
\pgfpathlineto{\pgfqpoint{4.912011in}{3.121285in}}%
\pgfpathlineto{\pgfqpoint{4.925638in}{3.119786in}}%
\pgfpathlineto{\pgfqpoint{4.939276in}{3.118469in}}%
\pgfpathlineto{\pgfqpoint{4.952924in}{3.117331in}}%
\pgfpathlineto{\pgfqpoint{4.960459in}{3.131392in}}%
\pgfpathlineto{\pgfqpoint{4.967995in}{3.145712in}}%
\pgfpathlineto{\pgfqpoint{4.975531in}{3.160300in}}%
\pgfpathlineto{\pgfqpoint{4.983069in}{3.175164in}}%
\pgfpathlineto{\pgfqpoint{4.969437in}{3.176961in}}%
\pgfpathlineto{\pgfqpoint{4.955815in}{3.178938in}}%
\pgfpathlineto{\pgfqpoint{4.942202in}{3.181096in}}%
\pgfpathlineto{\pgfqpoint{4.928599in}{3.183434in}}%
\pgfpathlineto{\pgfqpoint{4.921046in}{3.167900in}}%
\pgfpathlineto{\pgfqpoint{4.913494in}{3.152649in}}%
\pgfpathlineto{\pgfqpoint{4.905943in}{3.137673in}}%
\pgfpathlineto{\pgfqpoint{4.898393in}{3.122964in}}%
\pgfpathclose%
\pgfusepath{fill}%
\end{pgfscope}%
\begin{pgfscope}%
\pgfpathrectangle{\pgfqpoint{1.150000in}{0.150000in}}{\pgfqpoint{5.700000in}{5.700000in}}%
\pgfusepath{clip}%
\pgfsetbuttcap%
\pgfsetroundjoin%
\definecolor{currentfill}{rgb}{0.212395,0.359683,0.551710}%
\pgfsetfillcolor{currentfill}%
\pgfsetfillopacity{0.800000}%
\pgfsetlinewidth{0.000000pt}%
\definecolor{currentstroke}{rgb}{0.000000,0.000000,0.000000}%
\pgfsetstrokecolor{currentstroke}%
\pgfsetdash{}{0pt}%
\pgfpathmoveto{\pgfqpoint{3.093235in}{3.099057in}}%
\pgfpathlineto{\pgfqpoint{3.106698in}{3.080705in}}%
\pgfpathlineto{\pgfqpoint{3.120155in}{3.062664in}}%
\pgfpathlineto{\pgfqpoint{3.133605in}{3.044931in}}%
\pgfpathlineto{\pgfqpoint{3.147048in}{3.027502in}}%
\pgfpathlineto{\pgfqpoint{3.155020in}{3.041019in}}%
\pgfpathlineto{\pgfqpoint{3.162985in}{3.054694in}}%
\pgfpathlineto{\pgfqpoint{3.170943in}{3.068531in}}%
\pgfpathlineto{\pgfqpoint{3.178894in}{3.082531in}}%
\pgfpathlineto{\pgfqpoint{3.165461in}{3.100152in}}%
\pgfpathlineto{\pgfqpoint{3.152022in}{3.118077in}}%
\pgfpathlineto{\pgfqpoint{3.138577in}{3.136310in}}%
\pgfpathlineto{\pgfqpoint{3.125124in}{3.154854in}}%
\pgfpathlineto{\pgfqpoint{3.117162in}{3.140650in}}%
\pgfpathlineto{\pgfqpoint{3.109194in}{3.126617in}}%
\pgfpathlineto{\pgfqpoint{3.101218in}{3.112753in}}%
\pgfpathlineto{\pgfqpoint{3.093235in}{3.099057in}}%
\pgfpathclose%
\pgfusepath{fill}%
\end{pgfscope}%
\begin{pgfscope}%
\pgfpathrectangle{\pgfqpoint{1.150000in}{0.150000in}}{\pgfqpoint{5.700000in}{5.700000in}}%
\pgfusepath{clip}%
\pgfsetbuttcap%
\pgfsetroundjoin%
\definecolor{currentfill}{rgb}{0.262138,0.242286,0.520837}%
\pgfsetfillcolor{currentfill}%
\pgfsetfillopacity{0.800000}%
\pgfsetlinewidth{0.000000pt}%
\definecolor{currentstroke}{rgb}{0.000000,0.000000,0.000000}%
\pgfsetstrokecolor{currentstroke}%
\pgfsetdash{}{0pt}%
\pgfpathmoveto{\pgfqpoint{4.221950in}{2.772824in}}%
\pgfpathlineto{\pgfqpoint{4.235391in}{2.769188in}}%
\pgfpathlineto{\pgfqpoint{4.248838in}{2.765754in}}%
\pgfpathlineto{\pgfqpoint{4.262292in}{2.762520in}}%
\pgfpathlineto{\pgfqpoint{4.275753in}{2.759485in}}%
\pgfpathlineto{\pgfqpoint{4.283444in}{2.771888in}}%
\pgfpathlineto{\pgfqpoint{4.291131in}{2.784417in}}%
\pgfpathlineto{\pgfqpoint{4.298814in}{2.797076in}}%
\pgfpathlineto{\pgfqpoint{4.306494in}{2.809871in}}%
\pgfpathlineto{\pgfqpoint{4.293041in}{2.813314in}}%
\pgfpathlineto{\pgfqpoint{4.279596in}{2.816956in}}%
\pgfpathlineto{\pgfqpoint{4.266158in}{2.820798in}}%
\pgfpathlineto{\pgfqpoint{4.252726in}{2.824842in}}%
\pgfpathlineto{\pgfqpoint{4.245037in}{2.811628in}}%
\pgfpathlineto{\pgfqpoint{4.237345in}{2.798557in}}%
\pgfpathlineto{\pgfqpoint{4.229649in}{2.785624in}}%
\pgfpathlineto{\pgfqpoint{4.221950in}{2.772824in}}%
\pgfpathclose%
\pgfusepath{fill}%
\end{pgfscope}%
\begin{pgfscope}%
\pgfpathrectangle{\pgfqpoint{1.150000in}{0.150000in}}{\pgfqpoint{5.700000in}{5.700000in}}%
\pgfusepath{clip}%
\pgfsetbuttcap%
\pgfsetroundjoin%
\definecolor{currentfill}{rgb}{0.165117,0.467423,0.558141}%
\pgfsetfillcolor{currentfill}%
\pgfsetfillopacity{0.800000}%
\pgfsetlinewidth{0.000000pt}%
\definecolor{currentstroke}{rgb}{0.000000,0.000000,0.000000}%
\pgfsetstrokecolor{currentstroke}%
\pgfsetdash{}{0pt}%
\pgfpathmoveto{\pgfqpoint{2.963052in}{3.402698in}}%
\pgfpathlineto{\pgfqpoint{2.976610in}{3.380189in}}%
\pgfpathlineto{\pgfqpoint{2.990158in}{3.358029in}}%
\pgfpathlineto{\pgfqpoint{3.003695in}{3.336213in}}%
\pgfpathlineto{\pgfqpoint{3.017222in}{3.314739in}}%
\pgfpathlineto{\pgfqpoint{3.025199in}{3.329546in}}%
\pgfpathlineto{\pgfqpoint{3.033168in}{3.344546in}}%
\pgfpathlineto{\pgfqpoint{3.041129in}{3.359743in}}%
\pgfpathlineto{\pgfqpoint{3.049083in}{3.375139in}}%
\pgfpathlineto{\pgfqpoint{3.035567in}{3.396841in}}%
\pgfpathlineto{\pgfqpoint{3.022040in}{3.418884in}}%
\pgfpathlineto{\pgfqpoint{3.008503in}{3.441273in}}%
\pgfpathlineto{\pgfqpoint{2.994957in}{3.464010in}}%
\pgfpathlineto{\pgfqpoint{2.986992in}{3.448373in}}%
\pgfpathlineto{\pgfqpoint{2.979020in}{3.432944in}}%
\pgfpathlineto{\pgfqpoint{2.971040in}{3.417720in}}%
\pgfpathlineto{\pgfqpoint{2.963052in}{3.402698in}}%
\pgfpathclose%
\pgfusepath{fill}%
\end{pgfscope}%
\begin{pgfscope}%
\pgfpathrectangle{\pgfqpoint{1.150000in}{0.150000in}}{\pgfqpoint{5.700000in}{5.700000in}}%
\pgfusepath{clip}%
\pgfsetbuttcap%
\pgfsetroundjoin%
\definecolor{currentfill}{rgb}{0.274128,0.199721,0.498911}%
\pgfsetfillcolor{currentfill}%
\pgfsetfillopacity{0.800000}%
\pgfsetlinewidth{0.000000pt}%
\definecolor{currentstroke}{rgb}{0.000000,0.000000,0.000000}%
\pgfsetstrokecolor{currentstroke}%
\pgfsetdash{}{0pt}%
\pgfpathmoveto{\pgfqpoint{3.691728in}{2.681534in}}%
\pgfpathlineto{\pgfqpoint{3.705084in}{2.673440in}}%
\pgfpathlineto{\pgfqpoint{3.718442in}{2.665577in}}%
\pgfpathlineto{\pgfqpoint{3.731803in}{2.657943in}}%
\pgfpathlineto{\pgfqpoint{3.745166in}{2.650537in}}%
\pgfpathlineto{\pgfqpoint{3.753000in}{2.662986in}}%
\pgfpathlineto{\pgfqpoint{3.760829in}{2.675536in}}%
\pgfpathlineto{\pgfqpoint{3.768653in}{2.688191in}}%
\pgfpathlineto{\pgfqpoint{3.776472in}{2.700954in}}%
\pgfpathlineto{\pgfqpoint{3.763117in}{2.708612in}}%
\pgfpathlineto{\pgfqpoint{3.749764in}{2.716497in}}%
\pgfpathlineto{\pgfqpoint{3.736414in}{2.724612in}}%
\pgfpathlineto{\pgfqpoint{3.723065in}{2.732958in}}%
\pgfpathlineto{\pgfqpoint{3.715238in}{2.719931in}}%
\pgfpathlineto{\pgfqpoint{3.707406in}{2.707020in}}%
\pgfpathlineto{\pgfqpoint{3.699569in}{2.694222in}}%
\pgfpathlineto{\pgfqpoint{3.691728in}{2.681534in}}%
\pgfpathclose%
\pgfusepath{fill}%
\end{pgfscope}%
\begin{pgfscope}%
\pgfpathrectangle{\pgfqpoint{1.150000in}{0.150000in}}{\pgfqpoint{5.700000in}{5.700000in}}%
\pgfusepath{clip}%
\pgfsetbuttcap%
\pgfsetroundjoin%
\definecolor{currentfill}{rgb}{0.192357,0.403199,0.555836}%
\pgfsetfillcolor{currentfill}%
\pgfsetfillopacity{0.800000}%
\pgfsetlinewidth{0.000000pt}%
\definecolor{currentstroke}{rgb}{0.000000,0.000000,0.000000}%
\pgfsetstrokecolor{currentstroke}%
\pgfsetdash{}{0pt}%
\pgfpathmoveto{\pgfqpoint{4.983069in}{3.175164in}}%
\pgfpathlineto{\pgfqpoint{4.996711in}{3.173546in}}%
\pgfpathlineto{\pgfqpoint{5.010364in}{3.172107in}}%
\pgfpathlineto{\pgfqpoint{5.024027in}{3.170847in}}%
\pgfpathlineto{\pgfqpoint{5.037700in}{3.169766in}}%
\pgfpathlineto{\pgfqpoint{5.045223in}{3.184232in}}%
\pgfpathlineto{\pgfqpoint{5.052748in}{3.198982in}}%
\pgfpathlineto{\pgfqpoint{5.060275in}{3.214024in}}%
\pgfpathlineto{\pgfqpoint{5.067804in}{3.229364in}}%
\pgfpathlineto{\pgfqpoint{5.054148in}{3.231137in}}%
\pgfpathlineto{\pgfqpoint{5.040501in}{3.233088in}}%
\pgfpathlineto{\pgfqpoint{5.026865in}{3.235218in}}%
\pgfpathlineto{\pgfqpoint{5.013239in}{3.237526in}}%
\pgfpathlineto{\pgfqpoint{5.005693in}{3.221483in}}%
\pgfpathlineto{\pgfqpoint{4.998150in}{3.205747in}}%
\pgfpathlineto{\pgfqpoint{4.990609in}{3.190310in}}%
\pgfpathlineto{\pgfqpoint{4.983069in}{3.175164in}}%
\pgfpathclose%
\pgfusepath{fill}%
\end{pgfscope}%
\begin{pgfscope}%
\pgfpathrectangle{\pgfqpoint{1.150000in}{0.150000in}}{\pgfqpoint{5.700000in}{5.700000in}}%
\pgfusepath{clip}%
\pgfsetbuttcap%
\pgfsetroundjoin%
\definecolor{currentfill}{rgb}{0.271828,0.209303,0.504434}%
\pgfsetfillcolor{currentfill}%
\pgfsetfillopacity{0.800000}%
\pgfsetlinewidth{0.000000pt}%
\definecolor{currentstroke}{rgb}{0.000000,0.000000,0.000000}%
\pgfsetstrokecolor{currentstroke}%
\pgfsetdash{}{0pt}%
\pgfpathmoveto{\pgfqpoint{3.553446in}{2.703281in}}%
\pgfpathlineto{\pgfqpoint{3.566801in}{2.693511in}}%
\pgfpathlineto{\pgfqpoint{3.580156in}{2.683983in}}%
\pgfpathlineto{\pgfqpoint{3.593512in}{2.674696in}}%
\pgfpathlineto{\pgfqpoint{3.606868in}{2.665648in}}%
\pgfpathlineto{\pgfqpoint{3.614739in}{2.678135in}}%
\pgfpathlineto{\pgfqpoint{3.622605in}{2.690729in}}%
\pgfpathlineto{\pgfqpoint{3.630465in}{2.703431in}}%
\pgfpathlineto{\pgfqpoint{3.638320in}{2.716245in}}%
\pgfpathlineto{\pgfqpoint{3.624972in}{2.725514in}}%
\pgfpathlineto{\pgfqpoint{3.611624in}{2.735022in}}%
\pgfpathlineto{\pgfqpoint{3.598278in}{2.744770in}}%
\pgfpathlineto{\pgfqpoint{3.584932in}{2.754761in}}%
\pgfpathlineto{\pgfqpoint{3.577068in}{2.741715in}}%
\pgfpathlineto{\pgfqpoint{3.569200in}{2.728788in}}%
\pgfpathlineto{\pgfqpoint{3.561326in}{2.715978in}}%
\pgfpathlineto{\pgfqpoint{3.553446in}{2.703281in}}%
\pgfpathclose%
\pgfusepath{fill}%
\end{pgfscope}%
\begin{pgfscope}%
\pgfpathrectangle{\pgfqpoint{1.150000in}{0.150000in}}{\pgfqpoint{5.700000in}{5.700000in}}%
\pgfusepath{clip}%
\pgfsetbuttcap%
\pgfsetroundjoin%
\definecolor{currentfill}{rgb}{0.262138,0.242286,0.520837}%
\pgfsetfillcolor{currentfill}%
\pgfsetfillopacity{0.800000}%
\pgfsetlinewidth{0.000000pt}%
\definecolor{currentstroke}{rgb}{0.000000,0.000000,0.000000}%
\pgfsetstrokecolor{currentstroke}%
\pgfsetdash{}{0pt}%
\pgfpathmoveto{\pgfqpoint{3.361480in}{2.788053in}}%
\pgfpathlineto{\pgfqpoint{3.374853in}{2.775433in}}%
\pgfpathlineto{\pgfqpoint{3.388224in}{2.763076in}}%
\pgfpathlineto{\pgfqpoint{3.401594in}{2.750981in}}%
\pgfpathlineto{\pgfqpoint{3.414962in}{2.739144in}}%
\pgfpathlineto{\pgfqpoint{3.422882in}{2.751768in}}%
\pgfpathlineto{\pgfqpoint{3.430796in}{2.764511in}}%
\pgfpathlineto{\pgfqpoint{3.438704in}{2.777375in}}%
\pgfpathlineto{\pgfqpoint{3.446607in}{2.790363in}}%
\pgfpathlineto{\pgfqpoint{3.433249in}{2.802389in}}%
\pgfpathlineto{\pgfqpoint{3.419889in}{2.814674in}}%
\pgfpathlineto{\pgfqpoint{3.406527in}{2.827221in}}%
\pgfpathlineto{\pgfqpoint{3.393164in}{2.840031in}}%
\pgfpathlineto{\pgfqpoint{3.385252in}{2.826841in}}%
\pgfpathlineto{\pgfqpoint{3.377334in}{2.813783in}}%
\pgfpathlineto{\pgfqpoint{3.369410in}{2.800855in}}%
\pgfpathlineto{\pgfqpoint{3.361480in}{2.788053in}}%
\pgfpathclose%
\pgfusepath{fill}%
\end{pgfscope}%
\begin{pgfscope}%
\pgfpathrectangle{\pgfqpoint{1.150000in}{0.150000in}}{\pgfqpoint{5.700000in}{5.700000in}}%
\pgfusepath{clip}%
\pgfsetbuttcap%
\pgfsetroundjoin%
\definecolor{currentfill}{rgb}{0.266580,0.228262,0.514349}%
\pgfsetfillcolor{currentfill}%
\pgfsetfillopacity{0.800000}%
\pgfsetlinewidth{0.000000pt}%
\definecolor{currentstroke}{rgb}{0.000000,0.000000,0.000000}%
\pgfsetstrokecolor{currentstroke}%
\pgfsetdash{}{0pt}%
\pgfpathmoveto{\pgfqpoint{4.137383in}{2.737935in}}%
\pgfpathlineto{\pgfqpoint{4.150806in}{2.733862in}}%
\pgfpathlineto{\pgfqpoint{4.164236in}{2.729994in}}%
\pgfpathlineto{\pgfqpoint{4.177673in}{2.726330in}}%
\pgfpathlineto{\pgfqpoint{4.191115in}{2.722869in}}%
\pgfpathlineto{\pgfqpoint{4.198830in}{2.735181in}}%
\pgfpathlineto{\pgfqpoint{4.206540in}{2.747607in}}%
\pgfpathlineto{\pgfqpoint{4.214247in}{2.760154in}}%
\pgfpathlineto{\pgfqpoint{4.221950in}{2.772824in}}%
\pgfpathlineto{\pgfqpoint{4.208516in}{2.776662in}}%
\pgfpathlineto{\pgfqpoint{4.195088in}{2.780703in}}%
\pgfpathlineto{\pgfqpoint{4.181667in}{2.784948in}}%
\pgfpathlineto{\pgfqpoint{4.168251in}{2.789397in}}%
\pgfpathlineto{\pgfqpoint{4.160540in}{2.776338in}}%
\pgfpathlineto{\pgfqpoint{4.152825in}{2.763411in}}%
\pgfpathlineto{\pgfqpoint{4.145106in}{2.750612in}}%
\pgfpathlineto{\pgfqpoint{4.137383in}{2.737935in}}%
\pgfpathclose%
\pgfusepath{fill}%
\end{pgfscope}%
\begin{pgfscope}%
\pgfpathrectangle{\pgfqpoint{1.150000in}{0.150000in}}{\pgfqpoint{5.700000in}{5.700000in}}%
\pgfusepath{clip}%
\pgfsetbuttcap%
\pgfsetroundjoin%
\definecolor{currentfill}{rgb}{0.197636,0.391528,0.554969}%
\pgfsetfillcolor{currentfill}%
\pgfsetfillopacity{0.800000}%
\pgfsetlinewidth{0.000000pt}%
\definecolor{currentstroke}{rgb}{0.000000,0.000000,0.000000}%
\pgfsetstrokecolor{currentstroke}%
\pgfsetdash{}{0pt}%
\pgfpathmoveto{\pgfqpoint{3.039305in}{3.175624in}}%
\pgfpathlineto{\pgfqpoint{3.052800in}{3.156002in}}%
\pgfpathlineto{\pgfqpoint{3.066286in}{3.136702in}}%
\pgfpathlineto{\pgfqpoint{3.079764in}{3.117721in}}%
\pgfpathlineto{\pgfqpoint{3.093235in}{3.099057in}}%
\pgfpathlineto{\pgfqpoint{3.101218in}{3.112753in}}%
\pgfpathlineto{\pgfqpoint{3.109194in}{3.126617in}}%
\pgfpathlineto{\pgfqpoint{3.117162in}{3.140650in}}%
\pgfpathlineto{\pgfqpoint{3.125124in}{3.154854in}}%
\pgfpathlineto{\pgfqpoint{3.111665in}{3.173711in}}%
\pgfpathlineto{\pgfqpoint{3.098198in}{3.192885in}}%
\pgfpathlineto{\pgfqpoint{3.084723in}{3.212378in}}%
\pgfpathlineto{\pgfqpoint{3.071240in}{3.232193in}}%
\pgfpathlineto{\pgfqpoint{3.063267in}{3.217783in}}%
\pgfpathlineto{\pgfqpoint{3.055287in}{3.203554in}}%
\pgfpathlineto{\pgfqpoint{3.047300in}{3.189502in}}%
\pgfpathlineto{\pgfqpoint{3.039305in}{3.175624in}}%
\pgfpathclose%
\pgfusepath{fill}%
\end{pgfscope}%
\begin{pgfscope}%
\pgfpathrectangle{\pgfqpoint{1.150000in}{0.150000in}}{\pgfqpoint{5.700000in}{5.700000in}}%
\pgfusepath{clip}%
\pgfsetbuttcap%
\pgfsetroundjoin%
\definecolor{currentfill}{rgb}{0.183898,0.422383,0.556944}%
\pgfsetfillcolor{currentfill}%
\pgfsetfillopacity{0.800000}%
\pgfsetlinewidth{0.000000pt}%
\definecolor{currentstroke}{rgb}{0.000000,0.000000,0.000000}%
\pgfsetstrokecolor{currentstroke}%
\pgfsetdash{}{0pt}%
\pgfpathmoveto{\pgfqpoint{5.067804in}{3.229364in}}%
\pgfpathlineto{\pgfqpoint{5.081471in}{3.227769in}}%
\pgfpathlineto{\pgfqpoint{5.095148in}{3.226351in}}%
\pgfpathlineto{\pgfqpoint{5.108836in}{3.225110in}}%
\pgfpathlineto{\pgfqpoint{5.122534in}{3.224046in}}%
\pgfpathlineto{\pgfqpoint{5.130048in}{3.238982in}}%
\pgfpathlineto{\pgfqpoint{5.137565in}{3.254226in}}%
\pgfpathlineto{\pgfqpoint{5.145085in}{3.269787in}}%
\pgfpathlineto{\pgfqpoint{5.152609in}{3.285672in}}%
\pgfpathlineto{\pgfqpoint{5.138929in}{3.287459in}}%
\pgfpathlineto{\pgfqpoint{5.125259in}{3.289423in}}%
\pgfpathlineto{\pgfqpoint{5.111599in}{3.291564in}}%
\pgfpathlineto{\pgfqpoint{5.097950in}{3.293881in}}%
\pgfpathlineto{\pgfqpoint{5.090408in}{3.277262in}}%
\pgfpathlineto{\pgfqpoint{5.082870in}{3.260975in}}%
\pgfpathlineto{\pgfqpoint{5.075336in}{3.245012in}}%
\pgfpathlineto{\pgfqpoint{5.067804in}{3.229364in}}%
\pgfpathclose%
\pgfusepath{fill}%
\end{pgfscope}%
\begin{pgfscope}%
\pgfpathrectangle{\pgfqpoint{1.150000in}{0.150000in}}{\pgfqpoint{5.700000in}{5.700000in}}%
\pgfusepath{clip}%
\pgfsetbuttcap%
\pgfsetroundjoin%
\definecolor{currentfill}{rgb}{0.274128,0.199721,0.498911}%
\pgfsetfillcolor{currentfill}%
\pgfsetfillopacity{0.800000}%
\pgfsetlinewidth{0.000000pt}%
\definecolor{currentstroke}{rgb}{0.000000,0.000000,0.000000}%
\pgfsetstrokecolor{currentstroke}%
\pgfsetdash{}{0pt}%
\pgfpathmoveto{\pgfqpoint{3.829922in}{2.672580in}}%
\pgfpathlineto{\pgfqpoint{3.843293in}{2.666045in}}%
\pgfpathlineto{\pgfqpoint{3.856667in}{2.659730in}}%
\pgfpathlineto{\pgfqpoint{3.870045in}{2.653635in}}%
\pgfpathlineto{\pgfqpoint{3.883427in}{2.647758in}}%
\pgfpathlineto{\pgfqpoint{3.891226in}{2.660091in}}%
\pgfpathlineto{\pgfqpoint{3.899020in}{2.672522in}}%
\pgfpathlineto{\pgfqpoint{3.906809in}{2.685057in}}%
\pgfpathlineto{\pgfqpoint{3.914594in}{2.697698in}}%
\pgfpathlineto{\pgfqpoint{3.901220in}{2.703858in}}%
\pgfpathlineto{\pgfqpoint{3.887850in}{2.710236in}}%
\pgfpathlineto{\pgfqpoint{3.874484in}{2.716834in}}%
\pgfpathlineto{\pgfqpoint{3.861121in}{2.723652in}}%
\pgfpathlineto{\pgfqpoint{3.853328in}{2.710717in}}%
\pgfpathlineto{\pgfqpoint{3.845531in}{2.697895in}}%
\pgfpathlineto{\pgfqpoint{3.837729in}{2.685184in}}%
\pgfpathlineto{\pgfqpoint{3.829922in}{2.672580in}}%
\pgfpathclose%
\pgfusepath{fill}%
\end{pgfscope}%
\begin{pgfscope}%
\pgfpathrectangle{\pgfqpoint{1.150000in}{0.150000in}}{\pgfqpoint{5.700000in}{5.700000in}}%
\pgfusepath{clip}%
\pgfsetbuttcap%
\pgfsetroundjoin%
\definecolor{currentfill}{rgb}{0.267968,0.223549,0.512008}%
\pgfsetfillcolor{currentfill}%
\pgfsetfillopacity{0.800000}%
\pgfsetlinewidth{0.000000pt}%
\definecolor{currentstroke}{rgb}{0.000000,0.000000,0.000000}%
\pgfsetstrokecolor{currentstroke}%
\pgfsetdash{}{0pt}%
\pgfpathmoveto{\pgfqpoint{3.414962in}{2.739144in}}%
\pgfpathlineto{\pgfqpoint{3.428328in}{2.727566in}}%
\pgfpathlineto{\pgfqpoint{3.441694in}{2.716243in}}%
\pgfpathlineto{\pgfqpoint{3.455058in}{2.705174in}}%
\pgfpathlineto{\pgfqpoint{3.468422in}{2.694358in}}%
\pgfpathlineto{\pgfqpoint{3.476332in}{2.706803in}}%
\pgfpathlineto{\pgfqpoint{3.484237in}{2.719360in}}%
\pgfpathlineto{\pgfqpoint{3.492136in}{2.732031in}}%
\pgfpathlineto{\pgfqpoint{3.500030in}{2.744818in}}%
\pgfpathlineto{\pgfqpoint{3.486676in}{2.755824in}}%
\pgfpathlineto{\pgfqpoint{3.473320in}{2.767083in}}%
\pgfpathlineto{\pgfqpoint{3.459964in}{2.778595in}}%
\pgfpathlineto{\pgfqpoint{3.446607in}{2.790363in}}%
\pgfpathlineto{\pgfqpoint{3.438704in}{2.777375in}}%
\pgfpathlineto{\pgfqpoint{3.430796in}{2.764511in}}%
\pgfpathlineto{\pgfqpoint{3.422882in}{2.751768in}}%
\pgfpathlineto{\pgfqpoint{3.414962in}{2.739144in}}%
\pgfpathclose%
\pgfusepath{fill}%
\end{pgfscope}%
\begin{pgfscope}%
\pgfpathrectangle{\pgfqpoint{1.150000in}{0.150000in}}{\pgfqpoint{5.700000in}{5.700000in}}%
\pgfusepath{clip}%
\pgfsetbuttcap%
\pgfsetroundjoin%
\definecolor{currentfill}{rgb}{0.270595,0.214069,0.507052}%
\pgfsetfillcolor{currentfill}%
\pgfsetfillopacity{0.800000}%
\pgfsetlinewidth{0.000000pt}%
\definecolor{currentstroke}{rgb}{0.000000,0.000000,0.000000}%
\pgfsetstrokecolor{currentstroke}%
\pgfsetdash{}{0pt}%
\pgfpathmoveto{\pgfqpoint{4.052781in}{2.705346in}}%
\pgfpathlineto{\pgfqpoint{4.066190in}{2.700791in}}%
\pgfpathlineto{\pgfqpoint{4.079604in}{2.696444in}}%
\pgfpathlineto{\pgfqpoint{4.093024in}{2.692305in}}%
\pgfpathlineto{\pgfqpoint{4.106450in}{2.688372in}}%
\pgfpathlineto{\pgfqpoint{4.114190in}{2.700600in}}%
\pgfpathlineto{\pgfqpoint{4.121925in}{2.712933in}}%
\pgfpathlineto{\pgfqpoint{4.129656in}{2.725377in}}%
\pgfpathlineto{\pgfqpoint{4.137383in}{2.737935in}}%
\pgfpathlineto{\pgfqpoint{4.123965in}{2.742213in}}%
\pgfpathlineto{\pgfqpoint{4.110553in}{2.746698in}}%
\pgfpathlineto{\pgfqpoint{4.097146in}{2.751390in}}%
\pgfpathlineto{\pgfqpoint{4.083745in}{2.756291in}}%
\pgfpathlineto{\pgfqpoint{4.076010in}{2.743375in}}%
\pgfpathlineto{\pgfqpoint{4.068271in}{2.730582in}}%
\pgfpathlineto{\pgfqpoint{4.060528in}{2.717907in}}%
\pgfpathlineto{\pgfqpoint{4.052781in}{2.705346in}}%
\pgfpathclose%
\pgfusepath{fill}%
\end{pgfscope}%
\begin{pgfscope}%
\pgfpathrectangle{\pgfqpoint{1.150000in}{0.150000in}}{\pgfqpoint{5.700000in}{5.700000in}}%
\pgfusepath{clip}%
\pgfsetbuttcap%
\pgfsetroundjoin%
\definecolor{currentfill}{rgb}{0.185556,0.418570,0.556753}%
\pgfsetfillcolor{currentfill}%
\pgfsetfillopacity{0.800000}%
\pgfsetlinewidth{0.000000pt}%
\definecolor{currentstroke}{rgb}{0.000000,0.000000,0.000000}%
\pgfsetstrokecolor{currentstroke}%
\pgfsetdash{}{0pt}%
\pgfpathmoveto{\pgfqpoint{2.985241in}{3.257395in}}%
\pgfpathlineto{\pgfqpoint{2.998771in}{3.236454in}}%
\pgfpathlineto{\pgfqpoint{3.012291in}{3.215847in}}%
\pgfpathlineto{\pgfqpoint{3.025803in}{3.195571in}}%
\pgfpathlineto{\pgfqpoint{3.039305in}{3.175624in}}%
\pgfpathlineto{\pgfqpoint{3.047300in}{3.189502in}}%
\pgfpathlineto{\pgfqpoint{3.055287in}{3.203554in}}%
\pgfpathlineto{\pgfqpoint{3.063267in}{3.217783in}}%
\pgfpathlineto{\pgfqpoint{3.071240in}{3.232193in}}%
\pgfpathlineto{\pgfqpoint{3.057749in}{3.252334in}}%
\pgfpathlineto{\pgfqpoint{3.044249in}{3.272803in}}%
\pgfpathlineto{\pgfqpoint{3.030740in}{3.293604in}}%
\pgfpathlineto{\pgfqpoint{3.017222in}{3.314739in}}%
\pgfpathlineto{\pgfqpoint{3.009238in}{3.300123in}}%
\pgfpathlineto{\pgfqpoint{3.001247in}{3.285696in}}%
\pgfpathlineto{\pgfqpoint{2.993248in}{3.271454in}}%
\pgfpathlineto{\pgfqpoint{2.985241in}{3.257395in}}%
\pgfpathclose%
\pgfusepath{fill}%
\end{pgfscope}%
\begin{pgfscope}%
\pgfpathrectangle{\pgfqpoint{1.150000in}{0.150000in}}{\pgfqpoint{5.700000in}{5.700000in}}%
\pgfusepath{clip}%
\pgfsetbuttcap%
\pgfsetroundjoin%
\definecolor{currentfill}{rgb}{0.175841,0.441290,0.557685}%
\pgfsetfillcolor{currentfill}%
\pgfsetfillopacity{0.800000}%
\pgfsetlinewidth{0.000000pt}%
\definecolor{currentstroke}{rgb}{0.000000,0.000000,0.000000}%
\pgfsetstrokecolor{currentstroke}%
\pgfsetdash{}{0pt}%
\pgfpathmoveto{\pgfqpoint{5.152609in}{3.285672in}}%
\pgfpathlineto{\pgfqpoint{5.166300in}{3.284061in}}%
\pgfpathlineto{\pgfqpoint{5.180001in}{3.282625in}}%
\pgfpathlineto{\pgfqpoint{5.193714in}{3.281365in}}%
\pgfpathlineto{\pgfqpoint{5.207437in}{3.280280in}}%
\pgfpathlineto{\pgfqpoint{5.214946in}{3.295755in}}%
\pgfpathlineto{\pgfqpoint{5.222458in}{3.311564in}}%
\pgfpathlineto{\pgfqpoint{5.229975in}{3.327716in}}%
\pgfpathlineto{\pgfqpoint{5.216266in}{3.329365in}}%
\pgfpathlineto{\pgfqpoint{5.202568in}{3.331188in}}%
\pgfpathlineto{\pgfqpoint{5.188881in}{3.333187in}}%
\pgfpathlineto{\pgfqpoint{5.175204in}{3.335361in}}%
\pgfpathlineto{\pgfqpoint{5.167668in}{3.318450in}}%
\pgfpathlineto{\pgfqpoint{5.160136in}{3.301890in}}%
\pgfpathlineto{\pgfqpoint{5.152609in}{3.285672in}}%
\pgfpathclose%
\pgfusepath{fill}%
\end{pgfscope}%
\begin{pgfscope}%
\pgfpathrectangle{\pgfqpoint{1.150000in}{0.150000in}}{\pgfqpoint{5.700000in}{5.700000in}}%
\pgfusepath{clip}%
\pgfsetbuttcap%
\pgfsetroundjoin%
\definecolor{currentfill}{rgb}{0.275191,0.194905,0.496005}%
\pgfsetfillcolor{currentfill}%
\pgfsetfillopacity{0.800000}%
\pgfsetlinewidth{0.000000pt}%
\definecolor{currentstroke}{rgb}{0.000000,0.000000,0.000000}%
\pgfsetstrokecolor{currentstroke}%
\pgfsetdash{}{0pt}%
\pgfpathmoveto{\pgfqpoint{3.606868in}{2.665648in}}%
\pgfpathlineto{\pgfqpoint{3.620226in}{2.656838in}}%
\pgfpathlineto{\pgfqpoint{3.633586in}{2.648264in}}%
\pgfpathlineto{\pgfqpoint{3.646947in}{2.639925in}}%
\pgfpathlineto{\pgfqpoint{3.660309in}{2.631819in}}%
\pgfpathlineto{\pgfqpoint{3.668172in}{2.644098in}}%
\pgfpathlineto{\pgfqpoint{3.676029in}{2.656475in}}%
\pgfpathlineto{\pgfqpoint{3.683881in}{2.668953in}}%
\pgfpathlineto{\pgfqpoint{3.691728in}{2.681534in}}%
\pgfpathlineto{\pgfqpoint{3.678373in}{2.689860in}}%
\pgfpathlineto{\pgfqpoint{3.665021in}{2.698420in}}%
\pgfpathlineto{\pgfqpoint{3.651670in}{2.707214in}}%
\pgfpathlineto{\pgfqpoint{3.638320in}{2.716245in}}%
\pgfpathlineto{\pgfqpoint{3.630465in}{2.703431in}}%
\pgfpathlineto{\pgfqpoint{3.622605in}{2.690729in}}%
\pgfpathlineto{\pgfqpoint{3.614739in}{2.678135in}}%
\pgfpathlineto{\pgfqpoint{3.606868in}{2.665648in}}%
\pgfpathclose%
\pgfusepath{fill}%
\end{pgfscope}%
\begin{pgfscope}%
\pgfpathrectangle{\pgfqpoint{1.150000in}{0.150000in}}{\pgfqpoint{5.700000in}{5.700000in}}%
\pgfusepath{clip}%
\pgfsetbuttcap%
\pgfsetroundjoin%
\definecolor{currentfill}{rgb}{0.276194,0.190074,0.493001}%
\pgfsetfillcolor{currentfill}%
\pgfsetfillopacity{0.800000}%
\pgfsetlinewidth{0.000000pt}%
\definecolor{currentstroke}{rgb}{0.000000,0.000000,0.000000}%
\pgfsetstrokecolor{currentstroke}%
\pgfsetdash{}{0pt}%
\pgfpathmoveto{\pgfqpoint{3.745166in}{2.650537in}}%
\pgfpathlineto{\pgfqpoint{3.758532in}{2.643359in}}%
\pgfpathlineto{\pgfqpoint{3.771901in}{2.636405in}}%
\pgfpathlineto{\pgfqpoint{3.785273in}{2.629677in}}%
\pgfpathlineto{\pgfqpoint{3.798649in}{2.623171in}}%
\pgfpathlineto{\pgfqpoint{3.806474in}{2.635379in}}%
\pgfpathlineto{\pgfqpoint{3.814295in}{2.647681in}}%
\pgfpathlineto{\pgfqpoint{3.822111in}{2.660080in}}%
\pgfpathlineto{\pgfqpoint{3.829922in}{2.672580in}}%
\pgfpathlineto{\pgfqpoint{3.816555in}{2.679338in}}%
\pgfpathlineto{\pgfqpoint{3.803191in}{2.686319in}}%
\pgfpathlineto{\pgfqpoint{3.789830in}{2.693524in}}%
\pgfpathlineto{\pgfqpoint{3.776472in}{2.700954in}}%
\pgfpathlineto{\pgfqpoint{3.768653in}{2.688191in}}%
\pgfpathlineto{\pgfqpoint{3.760829in}{2.675536in}}%
\pgfpathlineto{\pgfqpoint{3.753000in}{2.662986in}}%
\pgfpathlineto{\pgfqpoint{3.745166in}{2.650537in}}%
\pgfpathclose%
\pgfusepath{fill}%
\end{pgfscope}%
\begin{pgfscope}%
\pgfpathrectangle{\pgfqpoint{1.150000in}{0.150000in}}{\pgfqpoint{5.700000in}{5.700000in}}%
\pgfusepath{clip}%
\pgfsetbuttcap%
\pgfsetroundjoin%
\definecolor{currentfill}{rgb}{0.273006,0.204520,0.501721}%
\pgfsetfillcolor{currentfill}%
\pgfsetfillopacity{0.800000}%
\pgfsetlinewidth{0.000000pt}%
\definecolor{currentstroke}{rgb}{0.000000,0.000000,0.000000}%
\pgfsetstrokecolor{currentstroke}%
\pgfsetdash{}{0pt}%
\pgfpathmoveto{\pgfqpoint{3.968133in}{2.675222in}}%
\pgfpathlineto{\pgfqpoint{3.981529in}{2.670139in}}%
\pgfpathlineto{\pgfqpoint{3.994931in}{2.665268in}}%
\pgfpathlineto{\pgfqpoint{4.008337in}{2.660608in}}%
\pgfpathlineto{\pgfqpoint{4.021748in}{2.656158in}}%
\pgfpathlineto{\pgfqpoint{4.029513in}{2.668304in}}%
\pgfpathlineto{\pgfqpoint{4.037273in}{2.680548in}}%
\pgfpathlineto{\pgfqpoint{4.045029in}{2.692894in}}%
\pgfpathlineto{\pgfqpoint{4.052781in}{2.705346in}}%
\pgfpathlineto{\pgfqpoint{4.039377in}{2.710110in}}%
\pgfpathlineto{\pgfqpoint{4.025979in}{2.715084in}}%
\pgfpathlineto{\pgfqpoint{4.012586in}{2.720270in}}%
\pgfpathlineto{\pgfqpoint{3.999197in}{2.725667in}}%
\pgfpathlineto{\pgfqpoint{3.991438in}{2.712890in}}%
\pgfpathlineto{\pgfqpoint{3.983674in}{2.700225in}}%
\pgfpathlineto{\pgfqpoint{3.975906in}{2.687671in}}%
\pgfpathlineto{\pgfqpoint{3.968133in}{2.675222in}}%
\pgfpathclose%
\pgfusepath{fill}%
\end{pgfscope}%
\begin{pgfscope}%
\pgfpathrectangle{\pgfqpoint{1.150000in}{0.150000in}}{\pgfqpoint{5.700000in}{5.700000in}}%
\pgfusepath{clip}%
\pgfsetbuttcap%
\pgfsetroundjoin%
\definecolor{currentfill}{rgb}{0.233603,0.313828,0.543914}%
\pgfsetfillcolor{currentfill}%
\pgfsetfillopacity{0.800000}%
\pgfsetlinewidth{0.000000pt}%
\definecolor{currentstroke}{rgb}{0.000000,0.000000,0.000000}%
\pgfsetstrokecolor{currentstroke}%
\pgfsetdash{}{0pt}%
\pgfpathmoveto{\pgfqpoint{4.614217in}{2.924756in}}%
\pgfpathlineto{\pgfqpoint{4.627775in}{2.923187in}}%
\pgfpathlineto{\pgfqpoint{4.641341in}{2.921806in}}%
\pgfpathlineto{\pgfqpoint{4.654917in}{2.920612in}}%
\pgfpathlineto{\pgfqpoint{4.668502in}{2.919605in}}%
\pgfpathlineto{\pgfqpoint{4.676093in}{2.932035in}}%
\pgfpathlineto{\pgfqpoint{4.683681in}{2.944638in}}%
\pgfpathlineto{\pgfqpoint{4.691267in}{2.957423in}}%
\pgfpathlineto{\pgfqpoint{4.698851in}{2.970393in}}%
\pgfpathlineto{\pgfqpoint{4.685279in}{2.971935in}}%
\pgfpathlineto{\pgfqpoint{4.671715in}{2.973664in}}%
\pgfpathlineto{\pgfqpoint{4.658160in}{2.975579in}}%
\pgfpathlineto{\pgfqpoint{4.644615in}{2.977682in}}%
\pgfpathlineto{\pgfqpoint{4.637018in}{2.964165in}}%
\pgfpathlineto{\pgfqpoint{4.629420in}{2.950843in}}%
\pgfpathlineto{\pgfqpoint{4.621819in}{2.937709in}}%
\pgfpathlineto{\pgfqpoint{4.614217in}{2.924756in}}%
\pgfpathclose%
\pgfusepath{fill}%
\end{pgfscope}%
\begin{pgfscope}%
\pgfpathrectangle{\pgfqpoint{1.150000in}{0.150000in}}{\pgfqpoint{5.700000in}{5.700000in}}%
\pgfusepath{clip}%
\pgfsetbuttcap%
\pgfsetroundjoin%
\definecolor{currentfill}{rgb}{0.243113,0.292092,0.538516}%
\pgfsetfillcolor{currentfill}%
\pgfsetfillopacity{0.800000}%
\pgfsetlinewidth{0.000000pt}%
\definecolor{currentstroke}{rgb}{0.000000,0.000000,0.000000}%
\pgfsetstrokecolor{currentstroke}%
\pgfsetdash{}{0pt}%
\pgfpathmoveto{\pgfqpoint{4.529598in}{2.880795in}}%
\pgfpathlineto{\pgfqpoint{4.543131in}{2.878972in}}%
\pgfpathlineto{\pgfqpoint{4.556674in}{2.877340in}}%
\pgfpathlineto{\pgfqpoint{4.570225in}{2.875897in}}%
\pgfpathlineto{\pgfqpoint{4.583785in}{2.874643in}}%
\pgfpathlineto{\pgfqpoint{4.591397in}{2.886928in}}%
\pgfpathlineto{\pgfqpoint{4.599006in}{2.899372in}}%
\pgfpathlineto{\pgfqpoint{4.606613in}{2.911979in}}%
\pgfpathlineto{\pgfqpoint{4.614217in}{2.924756in}}%
\pgfpathlineto{\pgfqpoint{4.600668in}{2.926513in}}%
\pgfpathlineto{\pgfqpoint{4.587129in}{2.928459in}}%
\pgfpathlineto{\pgfqpoint{4.573597in}{2.930594in}}%
\pgfpathlineto{\pgfqpoint{4.560075in}{2.932919in}}%
\pgfpathlineto{\pgfqpoint{4.552459in}{2.919628in}}%
\pgfpathlineto{\pgfqpoint{4.544841in}{2.906514in}}%
\pgfpathlineto{\pgfqpoint{4.537221in}{2.893572in}}%
\pgfpathlineto{\pgfqpoint{4.529598in}{2.880795in}}%
\pgfpathclose%
\pgfusepath{fill}%
\end{pgfscope}%
\begin{pgfscope}%
\pgfpathrectangle{\pgfqpoint{1.150000in}{0.150000in}}{\pgfqpoint{5.700000in}{5.700000in}}%
\pgfusepath{clip}%
\pgfsetbuttcap%
\pgfsetroundjoin%
\definecolor{currentfill}{rgb}{0.225863,0.330805,0.547314}%
\pgfsetfillcolor{currentfill}%
\pgfsetfillopacity{0.800000}%
\pgfsetlinewidth{0.000000pt}%
\definecolor{currentstroke}{rgb}{0.000000,0.000000,0.000000}%
\pgfsetstrokecolor{currentstroke}%
\pgfsetdash{}{0pt}%
\pgfpathmoveto{\pgfqpoint{4.698851in}{2.970393in}}%
\pgfpathlineto{\pgfqpoint{4.712434in}{2.969037in}}%
\pgfpathlineto{\pgfqpoint{4.726025in}{2.967867in}}%
\pgfpathlineto{\pgfqpoint{4.739626in}{2.966881in}}%
\pgfpathlineto{\pgfqpoint{4.753237in}{2.966080in}}%
\pgfpathlineto{\pgfqpoint{4.760807in}{2.978689in}}%
\pgfpathlineto{\pgfqpoint{4.768376in}{2.991489in}}%
\pgfpathlineto{\pgfqpoint{4.775943in}{3.004488in}}%
\pgfpathlineto{\pgfqpoint{4.783509in}{3.017691in}}%
\pgfpathlineto{\pgfqpoint{4.769911in}{3.019058in}}%
\pgfpathlineto{\pgfqpoint{4.756323in}{3.020610in}}%
\pgfpathlineto{\pgfqpoint{4.742744in}{3.022346in}}%
\pgfpathlineto{\pgfqpoint{4.729175in}{3.024268in}}%
\pgfpathlineto{\pgfqpoint{4.721596in}{3.010487in}}%
\pgfpathlineto{\pgfqpoint{4.714016in}{2.996919in}}%
\pgfpathlineto{\pgfqpoint{4.706434in}{2.983557in}}%
\pgfpathlineto{\pgfqpoint{4.698851in}{2.970393in}}%
\pgfpathclose%
\pgfusepath{fill}%
\end{pgfscope}%
\begin{pgfscope}%
\pgfpathrectangle{\pgfqpoint{1.150000in}{0.150000in}}{\pgfqpoint{5.700000in}{5.700000in}}%
\pgfusepath{clip}%
\pgfsetbuttcap%
\pgfsetroundjoin%
\definecolor{currentfill}{rgb}{0.271828,0.209303,0.504434}%
\pgfsetfillcolor{currentfill}%
\pgfsetfillopacity{0.800000}%
\pgfsetlinewidth{0.000000pt}%
\definecolor{currentstroke}{rgb}{0.000000,0.000000,0.000000}%
\pgfsetstrokecolor{currentstroke}%
\pgfsetdash{}{0pt}%
\pgfpathmoveto{\pgfqpoint{3.468422in}{2.694358in}}%
\pgfpathlineto{\pgfqpoint{3.481785in}{2.683791in}}%
\pgfpathlineto{\pgfqpoint{3.495148in}{2.673474in}}%
\pgfpathlineto{\pgfqpoint{3.508511in}{2.663404in}}%
\pgfpathlineto{\pgfqpoint{3.521875in}{2.653579in}}%
\pgfpathlineto{\pgfqpoint{3.529776in}{2.665847in}}%
\pgfpathlineto{\pgfqpoint{3.537671in}{2.678218in}}%
\pgfpathlineto{\pgfqpoint{3.545561in}{2.690696in}}%
\pgfpathlineto{\pgfqpoint{3.553446in}{2.703281in}}%
\pgfpathlineto{\pgfqpoint{3.540092in}{2.713296in}}%
\pgfpathlineto{\pgfqpoint{3.526738in}{2.723555in}}%
\pgfpathlineto{\pgfqpoint{3.513384in}{2.734062in}}%
\pgfpathlineto{\pgfqpoint{3.500030in}{2.744818in}}%
\pgfpathlineto{\pgfqpoint{3.492136in}{2.732031in}}%
\pgfpathlineto{\pgfqpoint{3.484237in}{2.719360in}}%
\pgfpathlineto{\pgfqpoint{3.476332in}{2.706803in}}%
\pgfpathlineto{\pgfqpoint{3.468422in}{2.694358in}}%
\pgfpathclose%
\pgfusepath{fill}%
\end{pgfscope}%
\begin{pgfscope}%
\pgfpathrectangle{\pgfqpoint{1.150000in}{0.150000in}}{\pgfqpoint{5.700000in}{5.700000in}}%
\pgfusepath{clip}%
\pgfsetbuttcap%
\pgfsetroundjoin%
\definecolor{currentfill}{rgb}{0.248629,0.278775,0.534556}%
\pgfsetfillcolor{currentfill}%
\pgfsetfillopacity{0.800000}%
\pgfsetlinewidth{0.000000pt}%
\definecolor{currentstroke}{rgb}{0.000000,0.000000,0.000000}%
\pgfsetstrokecolor{currentstroke}%
\pgfsetdash{}{0pt}%
\pgfpathmoveto{\pgfqpoint{4.444986in}{2.838549in}}%
\pgfpathlineto{\pgfqpoint{4.458497in}{2.836432in}}%
\pgfpathlineto{\pgfqpoint{4.472016in}{2.834506in}}%
\pgfpathlineto{\pgfqpoint{4.485543in}{2.832773in}}%
\pgfpathlineto{\pgfqpoint{4.499079in}{2.831232in}}%
\pgfpathlineto{\pgfqpoint{4.506713in}{2.843402in}}%
\pgfpathlineto{\pgfqpoint{4.514344in}{2.855716in}}%
\pgfpathlineto{\pgfqpoint{4.521972in}{2.868178in}}%
\pgfpathlineto{\pgfqpoint{4.529598in}{2.880795in}}%
\pgfpathlineto{\pgfqpoint{4.516073in}{2.882808in}}%
\pgfpathlineto{\pgfqpoint{4.502556in}{2.885013in}}%
\pgfpathlineto{\pgfqpoint{4.489047in}{2.887409in}}%
\pgfpathlineto{\pgfqpoint{4.475547in}{2.889998in}}%
\pgfpathlineto{\pgfqpoint{4.467911in}{2.876898in}}%
\pgfpathlineto{\pgfqpoint{4.460272in}{2.863961in}}%
\pgfpathlineto{\pgfqpoint{4.452631in}{2.851179in}}%
\pgfpathlineto{\pgfqpoint{4.444986in}{2.838549in}}%
\pgfpathclose%
\pgfusepath{fill}%
\end{pgfscope}%
\begin{pgfscope}%
\pgfpathrectangle{\pgfqpoint{1.150000in}{0.150000in}}{\pgfqpoint{5.700000in}{5.700000in}}%
\pgfusepath{clip}%
\pgfsetbuttcap%
\pgfsetroundjoin%
\definecolor{currentfill}{rgb}{0.218130,0.347432,0.550038}%
\pgfsetfillcolor{currentfill}%
\pgfsetfillopacity{0.800000}%
\pgfsetlinewidth{0.000000pt}%
\definecolor{currentstroke}{rgb}{0.000000,0.000000,0.000000}%
\pgfsetstrokecolor{currentstroke}%
\pgfsetdash{}{0pt}%
\pgfpathmoveto{\pgfqpoint{4.783509in}{3.017691in}}%
\pgfpathlineto{\pgfqpoint{4.797116in}{3.016508in}}%
\pgfpathlineto{\pgfqpoint{4.810733in}{3.015507in}}%
\pgfpathlineto{\pgfqpoint{4.824360in}{3.014690in}}%
\pgfpathlineto{\pgfqpoint{4.837997in}{3.014055in}}%
\pgfpathlineto{\pgfqpoint{4.845548in}{3.026883in}}%
\pgfpathlineto{\pgfqpoint{4.853098in}{3.039922in}}%
\pgfpathlineto{\pgfqpoint{4.860648in}{3.053179in}}%
\pgfpathlineto{\pgfqpoint{4.868197in}{3.066659in}}%
\pgfpathlineto{\pgfqpoint{4.854574in}{3.067892in}}%
\pgfpathlineto{\pgfqpoint{4.840961in}{3.069307in}}%
\pgfpathlineto{\pgfqpoint{4.827358in}{3.070904in}}%
\pgfpathlineto{\pgfqpoint{4.813765in}{3.072685in}}%
\pgfpathlineto{\pgfqpoint{4.806201in}{3.058595in}}%
\pgfpathlineto{\pgfqpoint{4.798638in}{3.044738in}}%
\pgfpathlineto{\pgfqpoint{4.791074in}{3.031106in}}%
\pgfpathlineto{\pgfqpoint{4.783509in}{3.017691in}}%
\pgfpathclose%
\pgfusepath{fill}%
\end{pgfscope}%
\begin{pgfscope}%
\pgfpathrectangle{\pgfqpoint{1.150000in}{0.150000in}}{\pgfqpoint{5.700000in}{5.700000in}}%
\pgfusepath{clip}%
\pgfsetbuttcap%
\pgfsetroundjoin%
\definecolor{currentfill}{rgb}{0.241237,0.296485,0.539709}%
\pgfsetfillcolor{currentfill}%
\pgfsetfillopacity{0.800000}%
\pgfsetlinewidth{0.000000pt}%
\definecolor{currentstroke}{rgb}{0.000000,0.000000,0.000000}%
\pgfsetstrokecolor{currentstroke}%
\pgfsetdash{}{0pt}%
\pgfpathmoveto{\pgfqpoint{3.168853in}{2.908895in}}%
\pgfpathlineto{\pgfqpoint{3.182280in}{2.893110in}}%
\pgfpathlineto{\pgfqpoint{3.195702in}{2.877615in}}%
\pgfpathlineto{\pgfqpoint{3.209119in}{2.862407in}}%
\pgfpathlineto{\pgfqpoint{3.222531in}{2.847483in}}%
\pgfpathlineto{\pgfqpoint{3.230507in}{2.860094in}}%
\pgfpathlineto{\pgfqpoint{3.238476in}{2.872838in}}%
\pgfpathlineto{\pgfqpoint{3.246439in}{2.885718in}}%
\pgfpathlineto{\pgfqpoint{3.254396in}{2.898736in}}%
\pgfpathlineto{\pgfqpoint{3.240994in}{2.913818in}}%
\pgfpathlineto{\pgfqpoint{3.227588in}{2.929185in}}%
\pgfpathlineto{\pgfqpoint{3.214178in}{2.944840in}}%
\pgfpathlineto{\pgfqpoint{3.200763in}{2.960783in}}%
\pgfpathlineto{\pgfqpoint{3.192795in}{2.947594in}}%
\pgfpathlineto{\pgfqpoint{3.184821in}{2.934551in}}%
\pgfpathlineto{\pgfqpoint{3.176841in}{2.921652in}}%
\pgfpathlineto{\pgfqpoint{3.168853in}{2.908895in}}%
\pgfpathclose%
\pgfusepath{fill}%
\end{pgfscope}%
\begin{pgfscope}%
\pgfpathrectangle{\pgfqpoint{1.150000in}{0.150000in}}{\pgfqpoint{5.700000in}{5.700000in}}%
\pgfusepath{clip}%
\pgfsetbuttcap%
\pgfsetroundjoin%
\definecolor{currentfill}{rgb}{0.229739,0.322361,0.545706}%
\pgfsetfillcolor{currentfill}%
\pgfsetfillopacity{0.800000}%
\pgfsetlinewidth{0.000000pt}%
\definecolor{currentstroke}{rgb}{0.000000,0.000000,0.000000}%
\pgfsetstrokecolor{currentstroke}%
\pgfsetdash{}{0pt}%
\pgfpathmoveto{\pgfqpoint{3.115093in}{2.974976in}}%
\pgfpathlineto{\pgfqpoint{3.128542in}{2.958009in}}%
\pgfpathlineto{\pgfqpoint{3.141984in}{2.941342in}}%
\pgfpathlineto{\pgfqpoint{3.155422in}{2.924971in}}%
\pgfpathlineto{\pgfqpoint{3.168853in}{2.908895in}}%
\pgfpathlineto{\pgfqpoint{3.176841in}{2.921652in}}%
\pgfpathlineto{\pgfqpoint{3.184821in}{2.934551in}}%
\pgfpathlineto{\pgfqpoint{3.192795in}{2.947594in}}%
\pgfpathlineto{\pgfqpoint{3.200763in}{2.960783in}}%
\pgfpathlineto{\pgfqpoint{3.187342in}{2.977019in}}%
\pgfpathlineto{\pgfqpoint{3.173917in}{2.993549in}}%
\pgfpathlineto{\pgfqpoint{3.160485in}{3.010376in}}%
\pgfpathlineto{\pgfqpoint{3.147048in}{3.027502in}}%
\pgfpathlineto{\pgfqpoint{3.139070in}{3.014142in}}%
\pgfpathlineto{\pgfqpoint{3.131084in}{3.000935in}}%
\pgfpathlineto{\pgfqpoint{3.123092in}{2.987881in}}%
\pgfpathlineto{\pgfqpoint{3.115093in}{2.974976in}}%
\pgfpathclose%
\pgfusepath{fill}%
\end{pgfscope}%
\begin{pgfscope}%
\pgfpathrectangle{\pgfqpoint{1.150000in}{0.150000in}}{\pgfqpoint{5.700000in}{5.700000in}}%
\pgfusepath{clip}%
\pgfsetbuttcap%
\pgfsetroundjoin%
\definecolor{currentfill}{rgb}{0.255645,0.260703,0.528312}%
\pgfsetfillcolor{currentfill}%
\pgfsetfillopacity{0.800000}%
\pgfsetlinewidth{0.000000pt}%
\definecolor{currentstroke}{rgb}{0.000000,0.000000,0.000000}%
\pgfsetstrokecolor{currentstroke}%
\pgfsetdash{}{0pt}%
\pgfpathmoveto{\pgfqpoint{4.360374in}{2.798083in}}%
\pgfpathlineto{\pgfqpoint{4.373863in}{2.795628in}}%
\pgfpathlineto{\pgfqpoint{4.387359in}{2.793368in}}%
\pgfpathlineto{\pgfqpoint{4.400864in}{2.791304in}}%
\pgfpathlineto{\pgfqpoint{4.414376in}{2.789433in}}%
\pgfpathlineto{\pgfqpoint{4.422034in}{2.801512in}}%
\pgfpathlineto{\pgfqpoint{4.429688in}{2.813721in}}%
\pgfpathlineto{\pgfqpoint{4.437338in}{2.826065in}}%
\pgfpathlineto{\pgfqpoint{4.444986in}{2.838549in}}%
\pgfpathlineto{\pgfqpoint{4.431483in}{2.840860in}}%
\pgfpathlineto{\pgfqpoint{4.417989in}{2.843365in}}%
\pgfpathlineto{\pgfqpoint{4.404502in}{2.846065in}}%
\pgfpathlineto{\pgfqpoint{4.391023in}{2.848960in}}%
\pgfpathlineto{\pgfqpoint{4.383365in}{2.836024in}}%
\pgfpathlineto{\pgfqpoint{4.375705in}{2.823236in}}%
\pgfpathlineto{\pgfqpoint{4.368041in}{2.810591in}}%
\pgfpathlineto{\pgfqpoint{4.360374in}{2.798083in}}%
\pgfpathclose%
\pgfusepath{fill}%
\end{pgfscope}%
\begin{pgfscope}%
\pgfpathrectangle{\pgfqpoint{1.150000in}{0.150000in}}{\pgfqpoint{5.700000in}{5.700000in}}%
\pgfusepath{clip}%
\pgfsetbuttcap%
\pgfsetroundjoin%
\definecolor{currentfill}{rgb}{0.252194,0.269783,0.531579}%
\pgfsetfillcolor{currentfill}%
\pgfsetfillopacity{0.800000}%
\pgfsetlinewidth{0.000000pt}%
\definecolor{currentstroke}{rgb}{0.000000,0.000000,0.000000}%
\pgfsetstrokecolor{currentstroke}%
\pgfsetdash{}{0pt}%
\pgfpathmoveto{\pgfqpoint{3.222531in}{2.847483in}}%
\pgfpathlineto{\pgfqpoint{3.235939in}{2.832842in}}%
\pgfpathlineto{\pgfqpoint{3.249344in}{2.818480in}}%
\pgfpathlineto{\pgfqpoint{3.262745in}{2.804397in}}%
\pgfpathlineto{\pgfqpoint{3.276142in}{2.790589in}}%
\pgfpathlineto{\pgfqpoint{3.284106in}{2.803052in}}%
\pgfpathlineto{\pgfqpoint{3.292065in}{2.815642in}}%
\pgfpathlineto{\pgfqpoint{3.300017in}{2.828359in}}%
\pgfpathlineto{\pgfqpoint{3.307963in}{2.841206in}}%
\pgfpathlineto{\pgfqpoint{3.294577in}{2.855173in}}%
\pgfpathlineto{\pgfqpoint{3.281187in}{2.869415in}}%
\pgfpathlineto{\pgfqpoint{3.267793in}{2.883935in}}%
\pgfpathlineto{\pgfqpoint{3.254396in}{2.898736in}}%
\pgfpathlineto{\pgfqpoint{3.246439in}{2.885718in}}%
\pgfpathlineto{\pgfqpoint{3.238476in}{2.872838in}}%
\pgfpathlineto{\pgfqpoint{3.230507in}{2.860094in}}%
\pgfpathlineto{\pgfqpoint{3.222531in}{2.847483in}}%
\pgfpathclose%
\pgfusepath{fill}%
\end{pgfscope}%
\begin{pgfscope}%
\pgfpathrectangle{\pgfqpoint{1.150000in}{0.150000in}}{\pgfqpoint{5.700000in}{5.700000in}}%
\pgfusepath{clip}%
\pgfsetbuttcap%
\pgfsetroundjoin%
\definecolor{currentfill}{rgb}{0.208623,0.367752,0.552675}%
\pgfsetfillcolor{currentfill}%
\pgfsetfillopacity{0.800000}%
\pgfsetlinewidth{0.000000pt}%
\definecolor{currentstroke}{rgb}{0.000000,0.000000,0.000000}%
\pgfsetstrokecolor{currentstroke}%
\pgfsetdash{}{0pt}%
\pgfpathmoveto{\pgfqpoint{4.868197in}{3.066659in}}%
\pgfpathlineto{\pgfqpoint{4.881830in}{3.065608in}}%
\pgfpathlineto{\pgfqpoint{4.895473in}{3.064738in}}%
\pgfpathlineto{\pgfqpoint{4.909126in}{3.064050in}}%
\pgfpathlineto{\pgfqpoint{4.922789in}{3.063542in}}%
\pgfpathlineto{\pgfqpoint{4.930323in}{3.076636in}}%
\pgfpathlineto{\pgfqpoint{4.937857in}{3.089960in}}%
\pgfpathlineto{\pgfqpoint{4.945390in}{3.103523in}}%
\pgfpathlineto{\pgfqpoint{4.952924in}{3.117331in}}%
\pgfpathlineto{\pgfqpoint{4.939276in}{3.118469in}}%
\pgfpathlineto{\pgfqpoint{4.925638in}{3.119786in}}%
\pgfpathlineto{\pgfqpoint{4.912011in}{3.121285in}}%
\pgfpathlineto{\pgfqpoint{4.898393in}{3.122964in}}%
\pgfpathlineto{\pgfqpoint{4.890843in}{3.108516in}}%
\pgfpathlineto{\pgfqpoint{4.883294in}{3.094321in}}%
\pgfpathlineto{\pgfqpoint{4.875746in}{3.080371in}}%
\pgfpathlineto{\pgfqpoint{4.868197in}{3.066659in}}%
\pgfpathclose%
\pgfusepath{fill}%
\end{pgfscope}%
\begin{pgfscope}%
\pgfpathrectangle{\pgfqpoint{1.150000in}{0.150000in}}{\pgfqpoint{5.700000in}{5.700000in}}%
\pgfusepath{clip}%
\pgfsetbuttcap%
\pgfsetroundjoin%
\definecolor{currentfill}{rgb}{0.262138,0.242286,0.520837}%
\pgfsetfillcolor{currentfill}%
\pgfsetfillopacity{0.800000}%
\pgfsetlinewidth{0.000000pt}%
\definecolor{currentstroke}{rgb}{0.000000,0.000000,0.000000}%
\pgfsetstrokecolor{currentstroke}%
\pgfsetdash{}{0pt}%
\pgfpathmoveto{\pgfqpoint{4.275753in}{2.759485in}}%
\pgfpathlineto{\pgfqpoint{4.289221in}{2.756650in}}%
\pgfpathlineto{\pgfqpoint{4.302697in}{2.754013in}}%
\pgfpathlineto{\pgfqpoint{4.316180in}{2.751574in}}%
\pgfpathlineto{\pgfqpoint{4.329670in}{2.749332in}}%
\pgfpathlineto{\pgfqpoint{4.337352in}{2.761338in}}%
\pgfpathlineto{\pgfqpoint{4.345029in}{2.773462in}}%
\pgfpathlineto{\pgfqpoint{4.352703in}{2.785709in}}%
\pgfpathlineto{\pgfqpoint{4.360374in}{2.798083in}}%
\pgfpathlineto{\pgfqpoint{4.346893in}{2.800734in}}%
\pgfpathlineto{\pgfqpoint{4.333419in}{2.803582in}}%
\pgfpathlineto{\pgfqpoint{4.319953in}{2.806628in}}%
\pgfpathlineto{\pgfqpoint{4.306494in}{2.809871in}}%
\pgfpathlineto{\pgfqpoint{4.298814in}{2.797076in}}%
\pgfpathlineto{\pgfqpoint{4.291131in}{2.784417in}}%
\pgfpathlineto{\pgfqpoint{4.283444in}{2.771888in}}%
\pgfpathlineto{\pgfqpoint{4.275753in}{2.759485in}}%
\pgfpathclose%
\pgfusepath{fill}%
\end{pgfscope}%
\begin{pgfscope}%
\pgfpathrectangle{\pgfqpoint{1.150000in}{0.150000in}}{\pgfqpoint{5.700000in}{5.700000in}}%
\pgfusepath{clip}%
\pgfsetbuttcap%
\pgfsetroundjoin%
\definecolor{currentfill}{rgb}{0.171176,0.452530,0.557965}%
\pgfsetfillcolor{currentfill}%
\pgfsetfillopacity{0.800000}%
\pgfsetlinewidth{0.000000pt}%
\definecolor{currentstroke}{rgb}{0.000000,0.000000,0.000000}%
\pgfsetstrokecolor{currentstroke}%
\pgfsetdash{}{0pt}%
\pgfpathmoveto{\pgfqpoint{2.931023in}{3.344576in}}%
\pgfpathlineto{\pgfqpoint{2.944593in}{3.322262in}}%
\pgfpathlineto{\pgfqpoint{2.958153in}{3.300296in}}%
\pgfpathlineto{\pgfqpoint{2.971702in}{3.278675in}}%
\pgfpathlineto{\pgfqpoint{2.985241in}{3.257395in}}%
\pgfpathlineto{\pgfqpoint{2.993248in}{3.271454in}}%
\pgfpathlineto{\pgfqpoint{3.001247in}{3.285696in}}%
\pgfpathlineto{\pgfqpoint{3.009238in}{3.300123in}}%
\pgfpathlineto{\pgfqpoint{3.017222in}{3.314739in}}%
\pgfpathlineto{\pgfqpoint{3.003695in}{3.336213in}}%
\pgfpathlineto{\pgfqpoint{2.990158in}{3.358029in}}%
\pgfpathlineto{\pgfqpoint{2.976610in}{3.380189in}}%
\pgfpathlineto{\pgfqpoint{2.963052in}{3.402698in}}%
\pgfpathlineto{\pgfqpoint{2.955057in}{3.387875in}}%
\pgfpathlineto{\pgfqpoint{2.947053in}{3.373249in}}%
\pgfpathlineto{\pgfqpoint{2.939042in}{3.358816in}}%
\pgfpathlineto{\pgfqpoint{2.931023in}{3.344576in}}%
\pgfpathclose%
\pgfusepath{fill}%
\end{pgfscope}%
\begin{pgfscope}%
\pgfpathrectangle{\pgfqpoint{1.150000in}{0.150000in}}{\pgfqpoint{5.700000in}{5.700000in}}%
\pgfusepath{clip}%
\pgfsetbuttcap%
\pgfsetroundjoin%
\definecolor{currentfill}{rgb}{0.218130,0.347432,0.550038}%
\pgfsetfillcolor{currentfill}%
\pgfsetfillopacity{0.800000}%
\pgfsetlinewidth{0.000000pt}%
\definecolor{currentstroke}{rgb}{0.000000,0.000000,0.000000}%
\pgfsetstrokecolor{currentstroke}%
\pgfsetdash{}{0pt}%
\pgfpathmoveto{\pgfqpoint{3.061232in}{3.045892in}}%
\pgfpathlineto{\pgfqpoint{3.074707in}{3.027700in}}%
\pgfpathlineto{\pgfqpoint{3.088176in}{3.009819in}}%
\pgfpathlineto{\pgfqpoint{3.101637in}{2.992245in}}%
\pgfpathlineto{\pgfqpoint{3.115093in}{2.974976in}}%
\pgfpathlineto{\pgfqpoint{3.123092in}{2.987881in}}%
\pgfpathlineto{\pgfqpoint{3.131084in}{3.000935in}}%
\pgfpathlineto{\pgfqpoint{3.139070in}{3.014142in}}%
\pgfpathlineto{\pgfqpoint{3.147048in}{3.027502in}}%
\pgfpathlineto{\pgfqpoint{3.133605in}{3.044931in}}%
\pgfpathlineto{\pgfqpoint{3.120155in}{3.062664in}}%
\pgfpathlineto{\pgfqpoint{3.106698in}{3.080705in}}%
\pgfpathlineto{\pgfqpoint{3.093235in}{3.099057in}}%
\pgfpathlineto{\pgfqpoint{3.085245in}{3.085524in}}%
\pgfpathlineto{\pgfqpoint{3.077248in}{3.072154in}}%
\pgfpathlineto{\pgfqpoint{3.069243in}{3.058944in}}%
\pgfpathlineto{\pgfqpoint{3.061232in}{3.045892in}}%
\pgfpathclose%
\pgfusepath{fill}%
\end{pgfscope}%
\begin{pgfscope}%
\pgfpathrectangle{\pgfqpoint{1.150000in}{0.150000in}}{\pgfqpoint{5.700000in}{5.700000in}}%
\pgfusepath{clip}%
\pgfsetbuttcap%
\pgfsetroundjoin%
\definecolor{currentfill}{rgb}{0.260571,0.246922,0.522828}%
\pgfsetfillcolor{currentfill}%
\pgfsetfillopacity{0.800000}%
\pgfsetlinewidth{0.000000pt}%
\definecolor{currentstroke}{rgb}{0.000000,0.000000,0.000000}%
\pgfsetstrokecolor{currentstroke}%
\pgfsetdash{}{0pt}%
\pgfpathmoveto{\pgfqpoint{3.276142in}{2.790589in}}%
\pgfpathlineto{\pgfqpoint{3.289536in}{2.777054in}}%
\pgfpathlineto{\pgfqpoint{3.302927in}{2.763791in}}%
\pgfpathlineto{\pgfqpoint{3.316315in}{2.750797in}}%
\pgfpathlineto{\pgfqpoint{3.329701in}{2.738070in}}%
\pgfpathlineto{\pgfqpoint{3.337655in}{2.750387in}}%
\pgfpathlineto{\pgfqpoint{3.345603in}{2.762821in}}%
\pgfpathlineto{\pgfqpoint{3.353544in}{2.775376in}}%
\pgfpathlineto{\pgfqpoint{3.361480in}{2.788053in}}%
\pgfpathlineto{\pgfqpoint{3.348105in}{2.800939in}}%
\pgfpathlineto{\pgfqpoint{3.334727in}{2.814091in}}%
\pgfpathlineto{\pgfqpoint{3.321346in}{2.827513in}}%
\pgfpathlineto{\pgfqpoint{3.307963in}{2.841206in}}%
\pgfpathlineto{\pgfqpoint{3.300017in}{2.828359in}}%
\pgfpathlineto{\pgfqpoint{3.292065in}{2.815642in}}%
\pgfpathlineto{\pgfqpoint{3.284106in}{2.803052in}}%
\pgfpathlineto{\pgfqpoint{3.276142in}{2.790589in}}%
\pgfpathclose%
\pgfusepath{fill}%
\end{pgfscope}%
\begin{pgfscope}%
\pgfpathrectangle{\pgfqpoint{1.150000in}{0.150000in}}{\pgfqpoint{5.700000in}{5.700000in}}%
\pgfusepath{clip}%
\pgfsetbuttcap%
\pgfsetroundjoin%
\definecolor{currentfill}{rgb}{0.199430,0.387607,0.554642}%
\pgfsetfillcolor{currentfill}%
\pgfsetfillopacity{0.800000}%
\pgfsetlinewidth{0.000000pt}%
\definecolor{currentstroke}{rgb}{0.000000,0.000000,0.000000}%
\pgfsetstrokecolor{currentstroke}%
\pgfsetdash{}{0pt}%
\pgfpathmoveto{\pgfqpoint{4.952924in}{3.117331in}}%
\pgfpathlineto{\pgfqpoint{4.966583in}{3.116373in}}%
\pgfpathlineto{\pgfqpoint{4.980252in}{3.115595in}}%
\pgfpathlineto{\pgfqpoint{4.993931in}{3.114996in}}%
\pgfpathlineto{\pgfqpoint{5.007621in}{3.114575in}}%
\pgfpathlineto{\pgfqpoint{5.015140in}{3.127987in}}%
\pgfpathlineto{\pgfqpoint{5.022659in}{3.141650in}}%
\pgfpathlineto{\pgfqpoint{5.030179in}{3.155574in}}%
\pgfpathlineto{\pgfqpoint{5.037700in}{3.169766in}}%
\pgfpathlineto{\pgfqpoint{5.024027in}{3.170847in}}%
\pgfpathlineto{\pgfqpoint{5.010364in}{3.172107in}}%
\pgfpathlineto{\pgfqpoint{4.996711in}{3.173546in}}%
\pgfpathlineto{\pgfqpoint{4.983069in}{3.175164in}}%
\pgfpathlineto{\pgfqpoint{4.975531in}{3.160300in}}%
\pgfpathlineto{\pgfqpoint{4.967995in}{3.145712in}}%
\pgfpathlineto{\pgfqpoint{4.960459in}{3.131392in}}%
\pgfpathlineto{\pgfqpoint{4.952924in}{3.117331in}}%
\pgfpathclose%
\pgfusepath{fill}%
\end{pgfscope}%
\begin{pgfscope}%
\pgfpathrectangle{\pgfqpoint{1.150000in}{0.150000in}}{\pgfqpoint{5.700000in}{5.700000in}}%
\pgfusepath{clip}%
\pgfsetbuttcap%
\pgfsetroundjoin%
\definecolor{currentfill}{rgb}{0.275191,0.194905,0.496005}%
\pgfsetfillcolor{currentfill}%
\pgfsetfillopacity{0.800000}%
\pgfsetlinewidth{0.000000pt}%
\definecolor{currentstroke}{rgb}{0.000000,0.000000,0.000000}%
\pgfsetstrokecolor{currentstroke}%
\pgfsetdash{}{0pt}%
\pgfpathmoveto{\pgfqpoint{3.883427in}{2.647758in}}%
\pgfpathlineto{\pgfqpoint{3.896813in}{2.642099in}}%
\pgfpathlineto{\pgfqpoint{3.910204in}{2.636656in}}%
\pgfpathlineto{\pgfqpoint{3.923598in}{2.631428in}}%
\pgfpathlineto{\pgfqpoint{3.936998in}{2.626415in}}%
\pgfpathlineto{\pgfqpoint{3.944788in}{2.638476in}}%
\pgfpathlineto{\pgfqpoint{3.952575in}{2.650629in}}%
\pgfpathlineto{\pgfqpoint{3.960356in}{2.662876in}}%
\pgfpathlineto{\pgfqpoint{3.968133in}{2.675222in}}%
\pgfpathlineto{\pgfqpoint{3.954742in}{2.680519in}}%
\pgfpathlineto{\pgfqpoint{3.941355in}{2.686030in}}%
\pgfpathlineto{\pgfqpoint{3.927972in}{2.691756in}}%
\pgfpathlineto{\pgfqpoint{3.914594in}{2.697698in}}%
\pgfpathlineto{\pgfqpoint{3.906809in}{2.685057in}}%
\pgfpathlineto{\pgfqpoint{3.899020in}{2.672522in}}%
\pgfpathlineto{\pgfqpoint{3.891226in}{2.660091in}}%
\pgfpathlineto{\pgfqpoint{3.883427in}{2.647758in}}%
\pgfpathclose%
\pgfusepath{fill}%
\end{pgfscope}%
\begin{pgfscope}%
\pgfpathrectangle{\pgfqpoint{1.150000in}{0.150000in}}{\pgfqpoint{5.700000in}{5.700000in}}%
\pgfusepath{clip}%
\pgfsetbuttcap%
\pgfsetroundjoin%
\definecolor{currentfill}{rgb}{0.266580,0.228262,0.514349}%
\pgfsetfillcolor{currentfill}%
\pgfsetfillopacity{0.800000}%
\pgfsetlinewidth{0.000000pt}%
\definecolor{currentstroke}{rgb}{0.000000,0.000000,0.000000}%
\pgfsetstrokecolor{currentstroke}%
\pgfsetdash{}{0pt}%
\pgfpathmoveto{\pgfqpoint{4.191115in}{2.722869in}}%
\pgfpathlineto{\pgfqpoint{4.204564in}{2.719610in}}%
\pgfpathlineto{\pgfqpoint{4.218020in}{2.716553in}}%
\pgfpathlineto{\pgfqpoint{4.231483in}{2.713696in}}%
\pgfpathlineto{\pgfqpoint{4.244953in}{2.711039in}}%
\pgfpathlineto{\pgfqpoint{4.252659in}{2.722985in}}%
\pgfpathlineto{\pgfqpoint{4.260361in}{2.735038in}}%
\pgfpathlineto{\pgfqpoint{4.268059in}{2.747204in}}%
\pgfpathlineto{\pgfqpoint{4.275753in}{2.759485in}}%
\pgfpathlineto{\pgfqpoint{4.262292in}{2.762520in}}%
\pgfpathlineto{\pgfqpoint{4.248838in}{2.765754in}}%
\pgfpathlineto{\pgfqpoint{4.235391in}{2.769188in}}%
\pgfpathlineto{\pgfqpoint{4.221950in}{2.772824in}}%
\pgfpathlineto{\pgfqpoint{4.214247in}{2.760154in}}%
\pgfpathlineto{\pgfqpoint{4.206540in}{2.747607in}}%
\pgfpathlineto{\pgfqpoint{4.198830in}{2.735181in}}%
\pgfpathlineto{\pgfqpoint{4.191115in}{2.722869in}}%
\pgfpathclose%
\pgfusepath{fill}%
\end{pgfscope}%
\begin{pgfscope}%
\pgfpathrectangle{\pgfqpoint{1.150000in}{0.150000in}}{\pgfqpoint{5.700000in}{5.700000in}}%
\pgfusepath{clip}%
\pgfsetbuttcap%
\pgfsetroundjoin%
\definecolor{currentfill}{rgb}{0.190631,0.407061,0.556089}%
\pgfsetfillcolor{currentfill}%
\pgfsetfillopacity{0.800000}%
\pgfsetlinewidth{0.000000pt}%
\definecolor{currentstroke}{rgb}{0.000000,0.000000,0.000000}%
\pgfsetstrokecolor{currentstroke}%
\pgfsetdash{}{0pt}%
\pgfpathmoveto{\pgfqpoint{5.037700in}{3.169766in}}%
\pgfpathlineto{\pgfqpoint{5.051384in}{3.168862in}}%
\pgfpathlineto{\pgfqpoint{5.065079in}{3.168136in}}%
\pgfpathlineto{\pgfqpoint{5.078785in}{3.167588in}}%
\pgfpathlineto{\pgfqpoint{5.092501in}{3.167216in}}%
\pgfpathlineto{\pgfqpoint{5.100007in}{3.181002in}}%
\pgfpathlineto{\pgfqpoint{5.107514in}{3.195064in}}%
\pgfpathlineto{\pgfqpoint{5.115023in}{3.209409in}}%
\pgfpathlineto{\pgfqpoint{5.122534in}{3.224046in}}%
\pgfpathlineto{\pgfqpoint{5.108836in}{3.225110in}}%
\pgfpathlineto{\pgfqpoint{5.095148in}{3.226351in}}%
\pgfpathlineto{\pgfqpoint{5.081471in}{3.227769in}}%
\pgfpathlineto{\pgfqpoint{5.067804in}{3.229364in}}%
\pgfpathlineto{\pgfqpoint{5.060275in}{3.214024in}}%
\pgfpathlineto{\pgfqpoint{5.052748in}{3.198982in}}%
\pgfpathlineto{\pgfqpoint{5.045223in}{3.184232in}}%
\pgfpathlineto{\pgfqpoint{5.037700in}{3.169766in}}%
\pgfpathclose%
\pgfusepath{fill}%
\end{pgfscope}%
\begin{pgfscope}%
\pgfpathrectangle{\pgfqpoint{1.150000in}{0.150000in}}{\pgfqpoint{5.700000in}{5.700000in}}%
\pgfusepath{clip}%
\pgfsetbuttcap%
\pgfsetroundjoin%
\definecolor{currentfill}{rgb}{0.277134,0.185228,0.489898}%
\pgfsetfillcolor{currentfill}%
\pgfsetfillopacity{0.800000}%
\pgfsetlinewidth{0.000000pt}%
\definecolor{currentstroke}{rgb}{0.000000,0.000000,0.000000}%
\pgfsetstrokecolor{currentstroke}%
\pgfsetdash{}{0pt}%
\pgfpathmoveto{\pgfqpoint{3.660309in}{2.631819in}}%
\pgfpathlineto{\pgfqpoint{3.673674in}{2.623946in}}%
\pgfpathlineto{\pgfqpoint{3.687041in}{2.616304in}}%
\pgfpathlineto{\pgfqpoint{3.700410in}{2.608891in}}%
\pgfpathlineto{\pgfqpoint{3.713782in}{2.601706in}}%
\pgfpathlineto{\pgfqpoint{3.721635in}{2.613775in}}%
\pgfpathlineto{\pgfqpoint{3.729484in}{2.625935in}}%
\pgfpathlineto{\pgfqpoint{3.737328in}{2.638188in}}%
\pgfpathlineto{\pgfqpoint{3.745166in}{2.650537in}}%
\pgfpathlineto{\pgfqpoint{3.731803in}{2.657943in}}%
\pgfpathlineto{\pgfqpoint{3.718442in}{2.665577in}}%
\pgfpathlineto{\pgfqpoint{3.705084in}{2.673440in}}%
\pgfpathlineto{\pgfqpoint{3.691728in}{2.681534in}}%
\pgfpathlineto{\pgfqpoint{3.683881in}{2.668953in}}%
\pgfpathlineto{\pgfqpoint{3.676029in}{2.656475in}}%
\pgfpathlineto{\pgfqpoint{3.668172in}{2.644098in}}%
\pgfpathlineto{\pgfqpoint{3.660309in}{2.631819in}}%
\pgfpathclose%
\pgfusepath{fill}%
\end{pgfscope}%
\begin{pgfscope}%
\pgfpathrectangle{\pgfqpoint{1.150000in}{0.150000in}}{\pgfqpoint{5.700000in}{5.700000in}}%
\pgfusepath{clip}%
\pgfsetbuttcap%
\pgfsetroundjoin%
\definecolor{currentfill}{rgb}{0.275191,0.194905,0.496005}%
\pgfsetfillcolor{currentfill}%
\pgfsetfillopacity{0.800000}%
\pgfsetlinewidth{0.000000pt}%
\definecolor{currentstroke}{rgb}{0.000000,0.000000,0.000000}%
\pgfsetstrokecolor{currentstroke}%
\pgfsetdash{}{0pt}%
\pgfpathmoveto{\pgfqpoint{3.521875in}{2.653579in}}%
\pgfpathlineto{\pgfqpoint{3.535238in}{2.643998in}}%
\pgfpathlineto{\pgfqpoint{3.548602in}{2.634660in}}%
\pgfpathlineto{\pgfqpoint{3.561967in}{2.625562in}}%
\pgfpathlineto{\pgfqpoint{3.575333in}{2.616704in}}%
\pgfpathlineto{\pgfqpoint{3.583225in}{2.628794in}}%
\pgfpathlineto{\pgfqpoint{3.591111in}{2.640979in}}%
\pgfpathlineto{\pgfqpoint{3.598992in}{2.653263in}}%
\pgfpathlineto{\pgfqpoint{3.606868in}{2.665648in}}%
\pgfpathlineto{\pgfqpoint{3.593512in}{2.674696in}}%
\pgfpathlineto{\pgfqpoint{3.580156in}{2.683983in}}%
\pgfpathlineto{\pgfqpoint{3.566801in}{2.693511in}}%
\pgfpathlineto{\pgfqpoint{3.553446in}{2.703281in}}%
\pgfpathlineto{\pgfqpoint{3.545561in}{2.690696in}}%
\pgfpathlineto{\pgfqpoint{3.537671in}{2.678218in}}%
\pgfpathlineto{\pgfqpoint{3.529776in}{2.665847in}}%
\pgfpathlineto{\pgfqpoint{3.521875in}{2.653579in}}%
\pgfpathclose%
\pgfusepath{fill}%
\end{pgfscope}%
\begin{pgfscope}%
\pgfpathrectangle{\pgfqpoint{1.150000in}{0.150000in}}{\pgfqpoint{5.700000in}{5.700000in}}%
\pgfusepath{clip}%
\pgfsetbuttcap%
\pgfsetroundjoin%
\definecolor{currentfill}{rgb}{0.204903,0.375746,0.553533}%
\pgfsetfillcolor{currentfill}%
\pgfsetfillopacity{0.800000}%
\pgfsetlinewidth{0.000000pt}%
\definecolor{currentstroke}{rgb}{0.000000,0.000000,0.000000}%
\pgfsetstrokecolor{currentstroke}%
\pgfsetdash{}{0pt}%
\pgfpathmoveto{\pgfqpoint{3.007253in}{3.121818in}}%
\pgfpathlineto{\pgfqpoint{3.020760in}{3.102357in}}%
\pgfpathlineto{\pgfqpoint{3.034258in}{3.083217in}}%
\pgfpathlineto{\pgfqpoint{3.047749in}{3.064396in}}%
\pgfpathlineto{\pgfqpoint{3.061232in}{3.045892in}}%
\pgfpathlineto{\pgfqpoint{3.069243in}{3.058944in}}%
\pgfpathlineto{\pgfqpoint{3.077248in}{3.072154in}}%
\pgfpathlineto{\pgfqpoint{3.085245in}{3.085524in}}%
\pgfpathlineto{\pgfqpoint{3.093235in}{3.099057in}}%
\pgfpathlineto{\pgfqpoint{3.079764in}{3.117721in}}%
\pgfpathlineto{\pgfqpoint{3.066286in}{3.136702in}}%
\pgfpathlineto{\pgfqpoint{3.052800in}{3.156002in}}%
\pgfpathlineto{\pgfqpoint{3.039305in}{3.175624in}}%
\pgfpathlineto{\pgfqpoint{3.031303in}{3.161920in}}%
\pgfpathlineto{\pgfqpoint{3.023294in}{3.148385in}}%
\pgfpathlineto{\pgfqpoint{3.015277in}{3.135019in}}%
\pgfpathlineto{\pgfqpoint{3.007253in}{3.121818in}}%
\pgfpathclose%
\pgfusepath{fill}%
\end{pgfscope}%
\begin{pgfscope}%
\pgfpathrectangle{\pgfqpoint{1.150000in}{0.150000in}}{\pgfqpoint{5.700000in}{5.700000in}}%
\pgfusepath{clip}%
\pgfsetbuttcap%
\pgfsetroundjoin%
\definecolor{currentfill}{rgb}{0.266580,0.228262,0.514349}%
\pgfsetfillcolor{currentfill}%
\pgfsetfillopacity{0.800000}%
\pgfsetlinewidth{0.000000pt}%
\definecolor{currentstroke}{rgb}{0.000000,0.000000,0.000000}%
\pgfsetstrokecolor{currentstroke}%
\pgfsetdash{}{0pt}%
\pgfpathmoveto{\pgfqpoint{3.329701in}{2.738070in}}%
\pgfpathlineto{\pgfqpoint{3.343084in}{2.725608in}}%
\pgfpathlineto{\pgfqpoint{3.356466in}{2.713410in}}%
\pgfpathlineto{\pgfqpoint{3.369845in}{2.701472in}}%
\pgfpathlineto{\pgfqpoint{3.383223in}{2.689795in}}%
\pgfpathlineto{\pgfqpoint{3.391167in}{2.701965in}}%
\pgfpathlineto{\pgfqpoint{3.399104in}{2.714245in}}%
\pgfpathlineto{\pgfqpoint{3.407036in}{2.726638in}}%
\pgfpathlineto{\pgfqpoint{3.414962in}{2.739144in}}%
\pgfpathlineto{\pgfqpoint{3.401594in}{2.750981in}}%
\pgfpathlineto{\pgfqpoint{3.388224in}{2.763076in}}%
\pgfpathlineto{\pgfqpoint{3.374853in}{2.775433in}}%
\pgfpathlineto{\pgfqpoint{3.361480in}{2.788053in}}%
\pgfpathlineto{\pgfqpoint{3.353544in}{2.775376in}}%
\pgfpathlineto{\pgfqpoint{3.345603in}{2.762821in}}%
\pgfpathlineto{\pgfqpoint{3.337655in}{2.750387in}}%
\pgfpathlineto{\pgfqpoint{3.329701in}{2.738070in}}%
\pgfpathclose%
\pgfusepath{fill}%
\end{pgfscope}%
\begin{pgfscope}%
\pgfpathrectangle{\pgfqpoint{1.150000in}{0.150000in}}{\pgfqpoint{5.700000in}{5.700000in}}%
\pgfusepath{clip}%
\pgfsetbuttcap%
\pgfsetroundjoin%
\definecolor{currentfill}{rgb}{0.270595,0.214069,0.507052}%
\pgfsetfillcolor{currentfill}%
\pgfsetfillopacity{0.800000}%
\pgfsetlinewidth{0.000000pt}%
\definecolor{currentstroke}{rgb}{0.000000,0.000000,0.000000}%
\pgfsetstrokecolor{currentstroke}%
\pgfsetdash{}{0pt}%
\pgfpathmoveto{\pgfqpoint{4.106450in}{2.688372in}}%
\pgfpathlineto{\pgfqpoint{4.119882in}{2.684645in}}%
\pgfpathlineto{\pgfqpoint{4.133321in}{2.681123in}}%
\pgfpathlineto{\pgfqpoint{4.146765in}{2.677805in}}%
\pgfpathlineto{\pgfqpoint{4.160216in}{2.674690in}}%
\pgfpathlineto{\pgfqpoint{4.167947in}{2.686583in}}%
\pgfpathlineto{\pgfqpoint{4.175674in}{2.698575in}}%
\pgfpathlineto{\pgfqpoint{4.183397in}{2.710669in}}%
\pgfpathlineto{\pgfqpoint{4.191115in}{2.722869in}}%
\pgfpathlineto{\pgfqpoint{4.177673in}{2.726330in}}%
\pgfpathlineto{\pgfqpoint{4.164236in}{2.729994in}}%
\pgfpathlineto{\pgfqpoint{4.150806in}{2.733862in}}%
\pgfpathlineto{\pgfqpoint{4.137383in}{2.737935in}}%
\pgfpathlineto{\pgfqpoint{4.129656in}{2.725377in}}%
\pgfpathlineto{\pgfqpoint{4.121925in}{2.712933in}}%
\pgfpathlineto{\pgfqpoint{4.114190in}{2.700600in}}%
\pgfpathlineto{\pgfqpoint{4.106450in}{2.688372in}}%
\pgfpathclose%
\pgfusepath{fill}%
\end{pgfscope}%
\begin{pgfscope}%
\pgfpathrectangle{\pgfqpoint{1.150000in}{0.150000in}}{\pgfqpoint{5.700000in}{5.700000in}}%
\pgfusepath{clip}%
\pgfsetbuttcap%
\pgfsetroundjoin%
\definecolor{currentfill}{rgb}{0.182256,0.426184,0.557120}%
\pgfsetfillcolor{currentfill}%
\pgfsetfillopacity{0.800000}%
\pgfsetlinewidth{0.000000pt}%
\definecolor{currentstroke}{rgb}{0.000000,0.000000,0.000000}%
\pgfsetstrokecolor{currentstroke}%
\pgfsetdash{}{0pt}%
\pgfpathmoveto{\pgfqpoint{5.122534in}{3.224046in}}%
\pgfpathlineto{\pgfqpoint{5.136244in}{3.223158in}}%
\pgfpathlineto{\pgfqpoint{5.149964in}{3.222446in}}%
\pgfpathlineto{\pgfqpoint{5.163696in}{3.221910in}}%
\pgfpathlineto{\pgfqpoint{5.177439in}{3.221549in}}%
\pgfpathlineto{\pgfqpoint{5.184934in}{3.235773in}}%
\pgfpathlineto{\pgfqpoint{5.192432in}{3.250297in}}%
\pgfpathlineto{\pgfqpoint{5.199933in}{3.265130in}}%
\pgfpathlineto{\pgfqpoint{5.207437in}{3.280280in}}%
\pgfpathlineto{\pgfqpoint{5.193714in}{3.281365in}}%
\pgfpathlineto{\pgfqpoint{5.180001in}{3.282625in}}%
\pgfpathlineto{\pgfqpoint{5.166300in}{3.284061in}}%
\pgfpathlineto{\pgfqpoint{5.152609in}{3.285672in}}%
\pgfpathlineto{\pgfqpoint{5.145085in}{3.269787in}}%
\pgfpathlineto{\pgfqpoint{5.137565in}{3.254226in}}%
\pgfpathlineto{\pgfqpoint{5.130048in}{3.238982in}}%
\pgfpathlineto{\pgfqpoint{5.122534in}{3.224046in}}%
\pgfpathclose%
\pgfusepath{fill}%
\end{pgfscope}%
\begin{pgfscope}%
\pgfpathrectangle{\pgfqpoint{1.150000in}{0.150000in}}{\pgfqpoint{5.700000in}{5.700000in}}%
\pgfusepath{clip}%
\pgfsetbuttcap%
\pgfsetroundjoin%
\definecolor{currentfill}{rgb}{0.277134,0.185228,0.489898}%
\pgfsetfillcolor{currentfill}%
\pgfsetfillopacity{0.800000}%
\pgfsetlinewidth{0.000000pt}%
\definecolor{currentstroke}{rgb}{0.000000,0.000000,0.000000}%
\pgfsetstrokecolor{currentstroke}%
\pgfsetdash{}{0pt}%
\pgfpathmoveto{\pgfqpoint{3.798649in}{2.623171in}}%
\pgfpathlineto{\pgfqpoint{3.812027in}{2.616887in}}%
\pgfpathlineto{\pgfqpoint{3.825409in}{2.610824in}}%
\pgfpathlineto{\pgfqpoint{3.838795in}{2.604981in}}%
\pgfpathlineto{\pgfqpoint{3.852185in}{2.599357in}}%
\pgfpathlineto{\pgfqpoint{3.860003in}{2.611324in}}%
\pgfpathlineto{\pgfqpoint{3.867816in}{2.623378in}}%
\pgfpathlineto{\pgfqpoint{3.875624in}{2.635522in}}%
\pgfpathlineto{\pgfqpoint{3.883427in}{2.647758in}}%
\pgfpathlineto{\pgfqpoint{3.870045in}{2.653635in}}%
\pgfpathlineto{\pgfqpoint{3.856667in}{2.659730in}}%
\pgfpathlineto{\pgfqpoint{3.843293in}{2.666045in}}%
\pgfpathlineto{\pgfqpoint{3.829922in}{2.672580in}}%
\pgfpathlineto{\pgfqpoint{3.822111in}{2.660080in}}%
\pgfpathlineto{\pgfqpoint{3.814295in}{2.647681in}}%
\pgfpathlineto{\pgfqpoint{3.806474in}{2.635379in}}%
\pgfpathlineto{\pgfqpoint{3.798649in}{2.623171in}}%
\pgfpathclose%
\pgfusepath{fill}%
\end{pgfscope}%
\begin{pgfscope}%
\pgfpathrectangle{\pgfqpoint{1.150000in}{0.150000in}}{\pgfqpoint{5.700000in}{5.700000in}}%
\pgfusepath{clip}%
\pgfsetbuttcap%
\pgfsetroundjoin%
\definecolor{currentfill}{rgb}{0.271828,0.209303,0.504434}%
\pgfsetfillcolor{currentfill}%
\pgfsetfillopacity{0.800000}%
\pgfsetlinewidth{0.000000pt}%
\definecolor{currentstroke}{rgb}{0.000000,0.000000,0.000000}%
\pgfsetstrokecolor{currentstroke}%
\pgfsetdash{}{0pt}%
\pgfpathmoveto{\pgfqpoint{3.383223in}{2.689795in}}%
\pgfpathlineto{\pgfqpoint{3.396600in}{2.678374in}}%
\pgfpathlineto{\pgfqpoint{3.409975in}{2.667210in}}%
\pgfpathlineto{\pgfqpoint{3.423350in}{2.656300in}}%
\pgfpathlineto{\pgfqpoint{3.436723in}{2.645641in}}%
\pgfpathlineto{\pgfqpoint{3.444657in}{2.657665in}}%
\pgfpathlineto{\pgfqpoint{3.452584in}{2.669791in}}%
\pgfpathlineto{\pgfqpoint{3.460506in}{2.682021in}}%
\pgfpathlineto{\pgfqpoint{3.468422in}{2.694358in}}%
\pgfpathlineto{\pgfqpoint{3.455058in}{2.705174in}}%
\pgfpathlineto{\pgfqpoint{3.441694in}{2.716243in}}%
\pgfpathlineto{\pgfqpoint{3.428328in}{2.727566in}}%
\pgfpathlineto{\pgfqpoint{3.414962in}{2.739144in}}%
\pgfpathlineto{\pgfqpoint{3.407036in}{2.726638in}}%
\pgfpathlineto{\pgfqpoint{3.399104in}{2.714245in}}%
\pgfpathlineto{\pgfqpoint{3.391167in}{2.701965in}}%
\pgfpathlineto{\pgfqpoint{3.383223in}{2.689795in}}%
\pgfpathclose%
\pgfusepath{fill}%
\end{pgfscope}%
\begin{pgfscope}%
\pgfpathrectangle{\pgfqpoint{1.150000in}{0.150000in}}{\pgfqpoint{5.700000in}{5.700000in}}%
\pgfusepath{clip}%
\pgfsetbuttcap%
\pgfsetroundjoin%
\definecolor{currentfill}{rgb}{0.190631,0.407061,0.556089}%
\pgfsetfillcolor{currentfill}%
\pgfsetfillopacity{0.800000}%
\pgfsetlinewidth{0.000000pt}%
\definecolor{currentstroke}{rgb}{0.000000,0.000000,0.000000}%
\pgfsetstrokecolor{currentstroke}%
\pgfsetdash{}{0pt}%
\pgfpathmoveto{\pgfqpoint{2.953137in}{3.202946in}}%
\pgfpathlineto{\pgfqpoint{2.966680in}{3.182165in}}%
\pgfpathlineto{\pgfqpoint{2.980213in}{3.161719in}}%
\pgfpathlineto{\pgfqpoint{2.993737in}{3.141605in}}%
\pgfpathlineto{\pgfqpoint{3.007253in}{3.121818in}}%
\pgfpathlineto{\pgfqpoint{3.015277in}{3.135019in}}%
\pgfpathlineto{\pgfqpoint{3.023294in}{3.148385in}}%
\pgfpathlineto{\pgfqpoint{3.031303in}{3.161920in}}%
\pgfpathlineto{\pgfqpoint{3.039305in}{3.175624in}}%
\pgfpathlineto{\pgfqpoint{3.025803in}{3.195571in}}%
\pgfpathlineto{\pgfqpoint{3.012291in}{3.215847in}}%
\pgfpathlineto{\pgfqpoint{2.998771in}{3.236454in}}%
\pgfpathlineto{\pgfqpoint{2.985241in}{3.257395in}}%
\pgfpathlineto{\pgfqpoint{2.977227in}{3.243518in}}%
\pgfpathlineto{\pgfqpoint{2.969205in}{3.229818in}}%
\pgfpathlineto{\pgfqpoint{2.961175in}{3.216295in}}%
\pgfpathlineto{\pgfqpoint{2.953137in}{3.202946in}}%
\pgfpathclose%
\pgfusepath{fill}%
\end{pgfscope}%
\begin{pgfscope}%
\pgfpathrectangle{\pgfqpoint{1.150000in}{0.150000in}}{\pgfqpoint{5.700000in}{5.700000in}}%
\pgfusepath{clip}%
\pgfsetbuttcap%
\pgfsetroundjoin%
\definecolor{currentfill}{rgb}{0.273006,0.204520,0.501721}%
\pgfsetfillcolor{currentfill}%
\pgfsetfillopacity{0.800000}%
\pgfsetlinewidth{0.000000pt}%
\definecolor{currentstroke}{rgb}{0.000000,0.000000,0.000000}%
\pgfsetstrokecolor{currentstroke}%
\pgfsetdash{}{0pt}%
\pgfpathmoveto{\pgfqpoint{4.021748in}{2.656158in}}%
\pgfpathlineto{\pgfqpoint{4.035165in}{2.651918in}}%
\pgfpathlineto{\pgfqpoint{4.048588in}{2.647886in}}%
\pgfpathlineto{\pgfqpoint{4.062016in}{2.644062in}}%
\pgfpathlineto{\pgfqpoint{4.075451in}{2.640444in}}%
\pgfpathlineto{\pgfqpoint{4.083207in}{2.652287in}}%
\pgfpathlineto{\pgfqpoint{4.090959in}{2.664220in}}%
\pgfpathlineto{\pgfqpoint{4.098707in}{2.676247in}}%
\pgfpathlineto{\pgfqpoint{4.106450in}{2.688372in}}%
\pgfpathlineto{\pgfqpoint{4.093024in}{2.692305in}}%
\pgfpathlineto{\pgfqpoint{4.079604in}{2.696444in}}%
\pgfpathlineto{\pgfqpoint{4.066190in}{2.700791in}}%
\pgfpathlineto{\pgfqpoint{4.052781in}{2.705346in}}%
\pgfpathlineto{\pgfqpoint{4.045029in}{2.692894in}}%
\pgfpathlineto{\pgfqpoint{4.037273in}{2.680548in}}%
\pgfpathlineto{\pgfqpoint{4.029513in}{2.668304in}}%
\pgfpathlineto{\pgfqpoint{4.021748in}{2.656158in}}%
\pgfpathclose%
\pgfusepath{fill}%
\end{pgfscope}%
\begin{pgfscope}%
\pgfpathrectangle{\pgfqpoint{1.150000in}{0.150000in}}{\pgfqpoint{5.700000in}{5.700000in}}%
\pgfusepath{clip}%
\pgfsetbuttcap%
\pgfsetroundjoin%
\definecolor{currentfill}{rgb}{0.174274,0.445044,0.557792}%
\pgfsetfillcolor{currentfill}%
\pgfsetfillopacity{0.800000}%
\pgfsetlinewidth{0.000000pt}%
\definecolor{currentstroke}{rgb}{0.000000,0.000000,0.000000}%
\pgfsetstrokecolor{currentstroke}%
\pgfsetdash{}{0pt}%
\pgfpathmoveto{\pgfqpoint{5.207437in}{3.280280in}}%
\pgfpathlineto{\pgfqpoint{5.221172in}{3.279369in}}%
\pgfpathlineto{\pgfqpoint{5.234918in}{3.278633in}}%
\pgfpathlineto{\pgfqpoint{5.248675in}{3.278072in}}%
\pgfpathlineto{\pgfqpoint{5.262444in}{3.277684in}}%
\pgfpathlineto{\pgfqpoint{5.269932in}{3.292415in}}%
\pgfpathlineto{\pgfqpoint{5.277424in}{3.307472in}}%
\pgfpathlineto{\pgfqpoint{5.284921in}{3.322864in}}%
\pgfpathlineto{\pgfqpoint{5.271167in}{3.323816in}}%
\pgfpathlineto{\pgfqpoint{5.257425in}{3.324942in}}%
\pgfpathlineto{\pgfqpoint{5.243695in}{3.326242in}}%
\pgfpathlineto{\pgfqpoint{5.229975in}{3.327716in}}%
\pgfpathlineto{\pgfqpoint{5.222458in}{3.311564in}}%
\pgfpathlineto{\pgfqpoint{5.214946in}{3.295755in}}%
\pgfpathlineto{\pgfqpoint{5.207437in}{3.280280in}}%
\pgfpathclose%
\pgfusepath{fill}%
\end{pgfscope}%
\begin{pgfscope}%
\pgfpathrectangle{\pgfqpoint{1.150000in}{0.150000in}}{\pgfqpoint{5.700000in}{5.700000in}}%
\pgfusepath{clip}%
\pgfsetbuttcap%
\pgfsetroundjoin%
\definecolor{currentfill}{rgb}{0.278012,0.180367,0.486697}%
\pgfsetfillcolor{currentfill}%
\pgfsetfillopacity{0.800000}%
\pgfsetlinewidth{0.000000pt}%
\definecolor{currentstroke}{rgb}{0.000000,0.000000,0.000000}%
\pgfsetstrokecolor{currentstroke}%
\pgfsetdash{}{0pt}%
\pgfpathmoveto{\pgfqpoint{3.575333in}{2.616704in}}%
\pgfpathlineto{\pgfqpoint{3.588700in}{2.608083in}}%
\pgfpathlineto{\pgfqpoint{3.602068in}{2.599699in}}%
\pgfpathlineto{\pgfqpoint{3.615438in}{2.591550in}}%
\pgfpathlineto{\pgfqpoint{3.628809in}{2.583634in}}%
\pgfpathlineto{\pgfqpoint{3.636692in}{2.595546in}}%
\pgfpathlineto{\pgfqpoint{3.644570in}{2.607546in}}%
\pgfpathlineto{\pgfqpoint{3.652442in}{2.619636in}}%
\pgfpathlineto{\pgfqpoint{3.660309in}{2.631819in}}%
\pgfpathlineto{\pgfqpoint{3.646947in}{2.639925in}}%
\pgfpathlineto{\pgfqpoint{3.633586in}{2.648264in}}%
\pgfpathlineto{\pgfqpoint{3.620226in}{2.656838in}}%
\pgfpathlineto{\pgfqpoint{3.606868in}{2.665648in}}%
\pgfpathlineto{\pgfqpoint{3.598992in}{2.653263in}}%
\pgfpathlineto{\pgfqpoint{3.591111in}{2.640979in}}%
\pgfpathlineto{\pgfqpoint{3.583225in}{2.628794in}}%
\pgfpathlineto{\pgfqpoint{3.575333in}{2.616704in}}%
\pgfpathclose%
\pgfusepath{fill}%
\end{pgfscope}%
\begin{pgfscope}%
\pgfpathrectangle{\pgfqpoint{1.150000in}{0.150000in}}{\pgfqpoint{5.700000in}{5.700000in}}%
\pgfusepath{clip}%
\pgfsetbuttcap%
\pgfsetroundjoin%
\definecolor{currentfill}{rgb}{0.278012,0.180367,0.486697}%
\pgfsetfillcolor{currentfill}%
\pgfsetfillopacity{0.800000}%
\pgfsetlinewidth{0.000000pt}%
\definecolor{currentstroke}{rgb}{0.000000,0.000000,0.000000}%
\pgfsetstrokecolor{currentstroke}%
\pgfsetdash{}{0pt}%
\pgfpathmoveto{\pgfqpoint{3.713782in}{2.601706in}}%
\pgfpathlineto{\pgfqpoint{3.727156in}{2.594748in}}%
\pgfpathlineto{\pgfqpoint{3.740533in}{2.588015in}}%
\pgfpathlineto{\pgfqpoint{3.753913in}{2.581508in}}%
\pgfpathlineto{\pgfqpoint{3.767297in}{2.575223in}}%
\pgfpathlineto{\pgfqpoint{3.775142in}{2.587083in}}%
\pgfpathlineto{\pgfqpoint{3.782983in}{2.599026in}}%
\pgfpathlineto{\pgfqpoint{3.790818in}{2.611054in}}%
\pgfpathlineto{\pgfqpoint{3.798649in}{2.623171in}}%
\pgfpathlineto{\pgfqpoint{3.785273in}{2.629677in}}%
\pgfpathlineto{\pgfqpoint{3.771901in}{2.636405in}}%
\pgfpathlineto{\pgfqpoint{3.758532in}{2.643359in}}%
\pgfpathlineto{\pgfqpoint{3.745166in}{2.650537in}}%
\pgfpathlineto{\pgfqpoint{3.737328in}{2.638188in}}%
\pgfpathlineto{\pgfqpoint{3.729484in}{2.625935in}}%
\pgfpathlineto{\pgfqpoint{3.721635in}{2.613775in}}%
\pgfpathlineto{\pgfqpoint{3.713782in}{2.601706in}}%
\pgfpathclose%
\pgfusepath{fill}%
\end{pgfscope}%
\begin{pgfscope}%
\pgfpathrectangle{\pgfqpoint{1.150000in}{0.150000in}}{\pgfqpoint{5.700000in}{5.700000in}}%
\pgfusepath{clip}%
\pgfsetbuttcap%
\pgfsetroundjoin%
\definecolor{currentfill}{rgb}{0.276194,0.190074,0.493001}%
\pgfsetfillcolor{currentfill}%
\pgfsetfillopacity{0.800000}%
\pgfsetlinewidth{0.000000pt}%
\definecolor{currentstroke}{rgb}{0.000000,0.000000,0.000000}%
\pgfsetstrokecolor{currentstroke}%
\pgfsetdash{}{0pt}%
\pgfpathmoveto{\pgfqpoint{3.936998in}{2.626415in}}%
\pgfpathlineto{\pgfqpoint{3.950402in}{2.621615in}}%
\pgfpathlineto{\pgfqpoint{3.963811in}{2.617028in}}%
\pgfpathlineto{\pgfqpoint{3.977226in}{2.612651in}}%
\pgfpathlineto{\pgfqpoint{3.990645in}{2.608485in}}%
\pgfpathlineto{\pgfqpoint{3.998428in}{2.620274in}}%
\pgfpathlineto{\pgfqpoint{4.006206in}{2.632147in}}%
\pgfpathlineto{\pgfqpoint{4.013979in}{2.644107in}}%
\pgfpathlineto{\pgfqpoint{4.021748in}{2.656158in}}%
\pgfpathlineto{\pgfqpoint{4.008337in}{2.660608in}}%
\pgfpathlineto{\pgfqpoint{3.994931in}{2.665268in}}%
\pgfpathlineto{\pgfqpoint{3.981529in}{2.670139in}}%
\pgfpathlineto{\pgfqpoint{3.968133in}{2.675222in}}%
\pgfpathlineto{\pgfqpoint{3.960356in}{2.662876in}}%
\pgfpathlineto{\pgfqpoint{3.952575in}{2.650629in}}%
\pgfpathlineto{\pgfqpoint{3.944788in}{2.638476in}}%
\pgfpathlineto{\pgfqpoint{3.936998in}{2.626415in}}%
\pgfpathclose%
\pgfusepath{fill}%
\end{pgfscope}%
\begin{pgfscope}%
\pgfpathrectangle{\pgfqpoint{1.150000in}{0.150000in}}{\pgfqpoint{5.700000in}{5.700000in}}%
\pgfusepath{clip}%
\pgfsetbuttcap%
\pgfsetroundjoin%
\definecolor{currentfill}{rgb}{0.239346,0.300855,0.540844}%
\pgfsetfillcolor{currentfill}%
\pgfsetfillopacity{0.800000}%
\pgfsetlinewidth{0.000000pt}%
\definecolor{currentstroke}{rgb}{0.000000,0.000000,0.000000}%
\pgfsetstrokecolor{currentstroke}%
\pgfsetdash{}{0pt}%
\pgfpathmoveto{\pgfqpoint{4.583785in}{2.874643in}}%
\pgfpathlineto{\pgfqpoint{4.597354in}{2.873577in}}%
\pgfpathlineto{\pgfqpoint{4.610933in}{2.872700in}}%
\pgfpathlineto{\pgfqpoint{4.624521in}{2.872010in}}%
\pgfpathlineto{\pgfqpoint{4.638118in}{2.871507in}}%
\pgfpathlineto{\pgfqpoint{4.645718in}{2.883300in}}%
\pgfpathlineto{\pgfqpoint{4.653315in}{2.895243in}}%
\pgfpathlineto{\pgfqpoint{4.660910in}{2.907343in}}%
\pgfpathlineto{\pgfqpoint{4.668502in}{2.919605in}}%
\pgfpathlineto{\pgfqpoint{4.654917in}{2.920612in}}%
\pgfpathlineto{\pgfqpoint{4.641341in}{2.921806in}}%
\pgfpathlineto{\pgfqpoint{4.627775in}{2.923187in}}%
\pgfpathlineto{\pgfqpoint{4.614217in}{2.924756in}}%
\pgfpathlineto{\pgfqpoint{4.606613in}{2.911979in}}%
\pgfpathlineto{\pgfqpoint{4.599006in}{2.899372in}}%
\pgfpathlineto{\pgfqpoint{4.591397in}{2.886928in}}%
\pgfpathlineto{\pgfqpoint{4.583785in}{2.874643in}}%
\pgfpathclose%
\pgfusepath{fill}%
\end{pgfscope}%
\begin{pgfscope}%
\pgfpathrectangle{\pgfqpoint{1.150000in}{0.150000in}}{\pgfqpoint{5.700000in}{5.700000in}}%
\pgfusepath{clip}%
\pgfsetbuttcap%
\pgfsetroundjoin%
\definecolor{currentfill}{rgb}{0.231674,0.318106,0.544834}%
\pgfsetfillcolor{currentfill}%
\pgfsetfillopacity{0.800000}%
\pgfsetlinewidth{0.000000pt}%
\definecolor{currentstroke}{rgb}{0.000000,0.000000,0.000000}%
\pgfsetstrokecolor{currentstroke}%
\pgfsetdash{}{0pt}%
\pgfpathmoveto{\pgfqpoint{4.668502in}{2.919605in}}%
\pgfpathlineto{\pgfqpoint{4.682097in}{2.918784in}}%
\pgfpathlineto{\pgfqpoint{4.695702in}{2.918148in}}%
\pgfpathlineto{\pgfqpoint{4.709316in}{2.917698in}}%
\pgfpathlineto{\pgfqpoint{4.722940in}{2.917433in}}%
\pgfpathlineto{\pgfqpoint{4.730518in}{2.929339in}}%
\pgfpathlineto{\pgfqpoint{4.738093in}{2.941411in}}%
\pgfpathlineto{\pgfqpoint{4.745666in}{2.953656in}}%
\pgfpathlineto{\pgfqpoint{4.753237in}{2.966080in}}%
\pgfpathlineto{\pgfqpoint{4.739626in}{2.966881in}}%
\pgfpathlineto{\pgfqpoint{4.726025in}{2.967867in}}%
\pgfpathlineto{\pgfqpoint{4.712434in}{2.969037in}}%
\pgfpathlineto{\pgfqpoint{4.698851in}{2.970393in}}%
\pgfpathlineto{\pgfqpoint{4.691267in}{2.957423in}}%
\pgfpathlineto{\pgfqpoint{4.683681in}{2.944638in}}%
\pgfpathlineto{\pgfqpoint{4.676093in}{2.932035in}}%
\pgfpathlineto{\pgfqpoint{4.668502in}{2.919605in}}%
\pgfpathclose%
\pgfusepath{fill}%
\end{pgfscope}%
\begin{pgfscope}%
\pgfpathrectangle{\pgfqpoint{1.150000in}{0.150000in}}{\pgfqpoint{5.700000in}{5.700000in}}%
\pgfusepath{clip}%
\pgfsetbuttcap%
\pgfsetroundjoin%
\definecolor{currentfill}{rgb}{0.246811,0.283237,0.535941}%
\pgfsetfillcolor{currentfill}%
\pgfsetfillopacity{0.800000}%
\pgfsetlinewidth{0.000000pt}%
\definecolor{currentstroke}{rgb}{0.000000,0.000000,0.000000}%
\pgfsetstrokecolor{currentstroke}%
\pgfsetdash{}{0pt}%
\pgfpathmoveto{\pgfqpoint{4.499079in}{2.831232in}}%
\pgfpathlineto{\pgfqpoint{4.512623in}{2.829881in}}%
\pgfpathlineto{\pgfqpoint{4.526177in}{2.828721in}}%
\pgfpathlineto{\pgfqpoint{4.539739in}{2.827750in}}%
\pgfpathlineto{\pgfqpoint{4.553310in}{2.826969in}}%
\pgfpathlineto{\pgfqpoint{4.560933in}{2.838678in}}%
\pgfpathlineto{\pgfqpoint{4.568554in}{2.850524in}}%
\pgfpathlineto{\pgfqpoint{4.576171in}{2.862510in}}%
\pgfpathlineto{\pgfqpoint{4.583785in}{2.874643in}}%
\pgfpathlineto{\pgfqpoint{4.570225in}{2.875897in}}%
\pgfpathlineto{\pgfqpoint{4.556674in}{2.877340in}}%
\pgfpathlineto{\pgfqpoint{4.543131in}{2.878972in}}%
\pgfpathlineto{\pgfqpoint{4.529598in}{2.880795in}}%
\pgfpathlineto{\pgfqpoint{4.521972in}{2.868178in}}%
\pgfpathlineto{\pgfqpoint{4.514344in}{2.855716in}}%
\pgfpathlineto{\pgfqpoint{4.506713in}{2.843402in}}%
\pgfpathlineto{\pgfqpoint{4.499079in}{2.831232in}}%
\pgfpathclose%
\pgfusepath{fill}%
\end{pgfscope}%
\begin{pgfscope}%
\pgfpathrectangle{\pgfqpoint{1.150000in}{0.150000in}}{\pgfqpoint{5.700000in}{5.700000in}}%
\pgfusepath{clip}%
\pgfsetbuttcap%
\pgfsetroundjoin%
\definecolor{currentfill}{rgb}{0.223925,0.334994,0.548053}%
\pgfsetfillcolor{currentfill}%
\pgfsetfillopacity{0.800000}%
\pgfsetlinewidth{0.000000pt}%
\definecolor{currentstroke}{rgb}{0.000000,0.000000,0.000000}%
\pgfsetstrokecolor{currentstroke}%
\pgfsetdash{}{0pt}%
\pgfpathmoveto{\pgfqpoint{4.753237in}{2.966080in}}%
\pgfpathlineto{\pgfqpoint{4.766858in}{2.965463in}}%
\pgfpathlineto{\pgfqpoint{4.780489in}{2.965029in}}%
\pgfpathlineto{\pgfqpoint{4.794131in}{2.964779in}}%
\pgfpathlineto{\pgfqpoint{4.807782in}{2.964711in}}%
\pgfpathlineto{\pgfqpoint{4.815338in}{2.976765in}}%
\pgfpathlineto{\pgfqpoint{4.822893in}{2.989002in}}%
\pgfpathlineto{\pgfqpoint{4.830446in}{3.001430in}}%
\pgfpathlineto{\pgfqpoint{4.837997in}{3.014055in}}%
\pgfpathlineto{\pgfqpoint{4.824360in}{3.014690in}}%
\pgfpathlineto{\pgfqpoint{4.810733in}{3.015507in}}%
\pgfpathlineto{\pgfqpoint{4.797116in}{3.016508in}}%
\pgfpathlineto{\pgfqpoint{4.783509in}{3.017691in}}%
\pgfpathlineto{\pgfqpoint{4.775943in}{3.004488in}}%
\pgfpathlineto{\pgfqpoint{4.768376in}{2.991489in}}%
\pgfpathlineto{\pgfqpoint{4.760807in}{2.978689in}}%
\pgfpathlineto{\pgfqpoint{4.753237in}{2.966080in}}%
\pgfpathclose%
\pgfusepath{fill}%
\end{pgfscope}%
\begin{pgfscope}%
\pgfpathrectangle{\pgfqpoint{1.150000in}{0.150000in}}{\pgfqpoint{5.700000in}{5.700000in}}%
\pgfusepath{clip}%
\pgfsetbuttcap%
\pgfsetroundjoin%
\definecolor{currentfill}{rgb}{0.275191,0.194905,0.496005}%
\pgfsetfillcolor{currentfill}%
\pgfsetfillopacity{0.800000}%
\pgfsetlinewidth{0.000000pt}%
\definecolor{currentstroke}{rgb}{0.000000,0.000000,0.000000}%
\pgfsetstrokecolor{currentstroke}%
\pgfsetdash{}{0pt}%
\pgfpathmoveto{\pgfqpoint{3.436723in}{2.645641in}}%
\pgfpathlineto{\pgfqpoint{3.450097in}{2.635233in}}%
\pgfpathlineto{\pgfqpoint{3.463469in}{2.625074in}}%
\pgfpathlineto{\pgfqpoint{3.476842in}{2.615163in}}%
\pgfpathlineto{\pgfqpoint{3.490215in}{2.605496in}}%
\pgfpathlineto{\pgfqpoint{3.498138in}{2.617373in}}%
\pgfpathlineto{\pgfqpoint{3.506056in}{2.629344in}}%
\pgfpathlineto{\pgfqpoint{3.513968in}{2.641412in}}%
\pgfpathlineto{\pgfqpoint{3.521875in}{2.653579in}}%
\pgfpathlineto{\pgfqpoint{3.508511in}{2.663404in}}%
\pgfpathlineto{\pgfqpoint{3.495148in}{2.673474in}}%
\pgfpathlineto{\pgfqpoint{3.481785in}{2.683791in}}%
\pgfpathlineto{\pgfqpoint{3.468422in}{2.694358in}}%
\pgfpathlineto{\pgfqpoint{3.460506in}{2.682021in}}%
\pgfpathlineto{\pgfqpoint{3.452584in}{2.669791in}}%
\pgfpathlineto{\pgfqpoint{3.444657in}{2.657665in}}%
\pgfpathlineto{\pgfqpoint{3.436723in}{2.645641in}}%
\pgfpathclose%
\pgfusepath{fill}%
\end{pgfscope}%
\begin{pgfscope}%
\pgfpathrectangle{\pgfqpoint{1.150000in}{0.150000in}}{\pgfqpoint{5.700000in}{5.700000in}}%
\pgfusepath{clip}%
\pgfsetbuttcap%
\pgfsetroundjoin%
\definecolor{currentfill}{rgb}{0.253935,0.265254,0.529983}%
\pgfsetfillcolor{currentfill}%
\pgfsetfillopacity{0.800000}%
\pgfsetlinewidth{0.000000pt}%
\definecolor{currentstroke}{rgb}{0.000000,0.000000,0.000000}%
\pgfsetstrokecolor{currentstroke}%
\pgfsetdash{}{0pt}%
\pgfpathmoveto{\pgfqpoint{4.414376in}{2.789433in}}%
\pgfpathlineto{\pgfqpoint{4.427897in}{2.787755in}}%
\pgfpathlineto{\pgfqpoint{4.441426in}{2.786271in}}%
\pgfpathlineto{\pgfqpoint{4.454964in}{2.784979in}}%
\pgfpathlineto{\pgfqpoint{4.468510in}{2.783879in}}%
\pgfpathlineto{\pgfqpoint{4.476157in}{2.795528in}}%
\pgfpathlineto{\pgfqpoint{4.483801in}{2.807300in}}%
\pgfpathlineto{\pgfqpoint{4.491441in}{2.819200in}}%
\pgfpathlineto{\pgfqpoint{4.499079in}{2.831232in}}%
\pgfpathlineto{\pgfqpoint{4.485543in}{2.832773in}}%
\pgfpathlineto{\pgfqpoint{4.472016in}{2.834506in}}%
\pgfpathlineto{\pgfqpoint{4.458497in}{2.836432in}}%
\pgfpathlineto{\pgfqpoint{4.444986in}{2.838549in}}%
\pgfpathlineto{\pgfqpoint{4.437338in}{2.826065in}}%
\pgfpathlineto{\pgfqpoint{4.429688in}{2.813721in}}%
\pgfpathlineto{\pgfqpoint{4.422034in}{2.801512in}}%
\pgfpathlineto{\pgfqpoint{4.414376in}{2.789433in}}%
\pgfpathclose%
\pgfusepath{fill}%
\end{pgfscope}%
\begin{pgfscope}%
\pgfpathrectangle{\pgfqpoint{1.150000in}{0.150000in}}{\pgfqpoint{5.700000in}{5.700000in}}%
\pgfusepath{clip}%
\pgfsetbuttcap%
\pgfsetroundjoin%
\definecolor{currentfill}{rgb}{0.246811,0.283237,0.535941}%
\pgfsetfillcolor{currentfill}%
\pgfsetfillopacity{0.800000}%
\pgfsetlinewidth{0.000000pt}%
\definecolor{currentstroke}{rgb}{0.000000,0.000000,0.000000}%
\pgfsetstrokecolor{currentstroke}%
\pgfsetdash{}{0pt}%
\pgfpathmoveto{\pgfqpoint{3.136836in}{2.859243in}}%
\pgfpathlineto{\pgfqpoint{3.150275in}{2.843586in}}%
\pgfpathlineto{\pgfqpoint{3.163709in}{2.828218in}}%
\pgfpathlineto{\pgfqpoint{3.177138in}{2.813137in}}%
\pgfpathlineto{\pgfqpoint{3.190562in}{2.798340in}}%
\pgfpathlineto{\pgfqpoint{3.198564in}{2.810435in}}%
\pgfpathlineto{\pgfqpoint{3.206560in}{2.822656in}}%
\pgfpathlineto{\pgfqpoint{3.214549in}{2.835005in}}%
\pgfpathlineto{\pgfqpoint{3.222531in}{2.847483in}}%
\pgfpathlineto{\pgfqpoint{3.209119in}{2.862407in}}%
\pgfpathlineto{\pgfqpoint{3.195702in}{2.877615in}}%
\pgfpathlineto{\pgfqpoint{3.182280in}{2.893110in}}%
\pgfpathlineto{\pgfqpoint{3.168853in}{2.908895in}}%
\pgfpathlineto{\pgfqpoint{3.160859in}{2.896277in}}%
\pgfpathlineto{\pgfqpoint{3.152858in}{2.883797in}}%
\pgfpathlineto{\pgfqpoint{3.144851in}{2.871453in}}%
\pgfpathlineto{\pgfqpoint{3.136836in}{2.859243in}}%
\pgfpathclose%
\pgfusepath{fill}%
\end{pgfscope}%
\begin{pgfscope}%
\pgfpathrectangle{\pgfqpoint{1.150000in}{0.150000in}}{\pgfqpoint{5.700000in}{5.700000in}}%
\pgfusepath{clip}%
\pgfsetbuttcap%
\pgfsetroundjoin%
\definecolor{currentfill}{rgb}{0.214298,0.355619,0.551184}%
\pgfsetfillcolor{currentfill}%
\pgfsetfillopacity{0.800000}%
\pgfsetlinewidth{0.000000pt}%
\definecolor{currentstroke}{rgb}{0.000000,0.000000,0.000000}%
\pgfsetstrokecolor{currentstroke}%
\pgfsetdash{}{0pt}%
\pgfpathmoveto{\pgfqpoint{4.837997in}{3.014055in}}%
\pgfpathlineto{\pgfqpoint{4.851645in}{3.013602in}}%
\pgfpathlineto{\pgfqpoint{4.865303in}{3.013331in}}%
\pgfpathlineto{\pgfqpoint{4.878971in}{3.013241in}}%
\pgfpathlineto{\pgfqpoint{4.892650in}{3.013332in}}%
\pgfpathlineto{\pgfqpoint{4.900186in}{3.025573in}}%
\pgfpathlineto{\pgfqpoint{4.907721in}{3.038017in}}%
\pgfpathlineto{\pgfqpoint{4.915256in}{3.050671in}}%
\pgfpathlineto{\pgfqpoint{4.922789in}{3.063542in}}%
\pgfpathlineto{\pgfqpoint{4.909126in}{3.064050in}}%
\pgfpathlineto{\pgfqpoint{4.895473in}{3.064738in}}%
\pgfpathlineto{\pgfqpoint{4.881830in}{3.065608in}}%
\pgfpathlineto{\pgfqpoint{4.868197in}{3.066659in}}%
\pgfpathlineto{\pgfqpoint{4.860648in}{3.053179in}}%
\pgfpathlineto{\pgfqpoint{4.853098in}{3.039922in}}%
\pgfpathlineto{\pgfqpoint{4.845548in}{3.026883in}}%
\pgfpathlineto{\pgfqpoint{4.837997in}{3.014055in}}%
\pgfpathclose%
\pgfusepath{fill}%
\end{pgfscope}%
\begin{pgfscope}%
\pgfpathrectangle{\pgfqpoint{1.150000in}{0.150000in}}{\pgfqpoint{5.700000in}{5.700000in}}%
\pgfusepath{clip}%
\pgfsetbuttcap%
\pgfsetroundjoin%
\definecolor{currentfill}{rgb}{0.235526,0.309527,0.542944}%
\pgfsetfillcolor{currentfill}%
\pgfsetfillopacity{0.800000}%
\pgfsetlinewidth{0.000000pt}%
\definecolor{currentstroke}{rgb}{0.000000,0.000000,0.000000}%
\pgfsetstrokecolor{currentstroke}%
\pgfsetdash{}{0pt}%
\pgfpathmoveto{\pgfqpoint{3.083025in}{2.924815in}}%
\pgfpathlineto{\pgfqpoint{3.096487in}{2.907975in}}%
\pgfpathlineto{\pgfqpoint{3.109942in}{2.891435in}}%
\pgfpathlineto{\pgfqpoint{3.123392in}{2.875192in}}%
\pgfpathlineto{\pgfqpoint{3.136836in}{2.859243in}}%
\pgfpathlineto{\pgfqpoint{3.144851in}{2.871453in}}%
\pgfpathlineto{\pgfqpoint{3.152858in}{2.883797in}}%
\pgfpathlineto{\pgfqpoint{3.160859in}{2.896277in}}%
\pgfpathlineto{\pgfqpoint{3.168853in}{2.908895in}}%
\pgfpathlineto{\pgfqpoint{3.155422in}{2.924971in}}%
\pgfpathlineto{\pgfqpoint{3.141984in}{2.941342in}}%
\pgfpathlineto{\pgfqpoint{3.128542in}{2.958009in}}%
\pgfpathlineto{\pgfqpoint{3.115093in}{2.974976in}}%
\pgfpathlineto{\pgfqpoint{3.107086in}{2.962219in}}%
\pgfpathlineto{\pgfqpoint{3.099073in}{2.949608in}}%
\pgfpathlineto{\pgfqpoint{3.091053in}{2.937140in}}%
\pgfpathlineto{\pgfqpoint{3.083025in}{2.924815in}}%
\pgfpathclose%
\pgfusepath{fill}%
\end{pgfscope}%
\begin{pgfscope}%
\pgfpathrectangle{\pgfqpoint{1.150000in}{0.150000in}}{\pgfqpoint{5.700000in}{5.700000in}}%
\pgfusepath{clip}%
\pgfsetbuttcap%
\pgfsetroundjoin%
\definecolor{currentfill}{rgb}{0.177423,0.437527,0.557565}%
\pgfsetfillcolor{currentfill}%
\pgfsetfillopacity{0.800000}%
\pgfsetlinewidth{0.000000pt}%
\definecolor{currentstroke}{rgb}{0.000000,0.000000,0.000000}%
\pgfsetstrokecolor{currentstroke}%
\pgfsetdash{}{0pt}%
\pgfpathmoveto{\pgfqpoint{2.898866in}{3.289479in}}%
\pgfpathlineto{\pgfqpoint{2.912449in}{3.267327in}}%
\pgfpathlineto{\pgfqpoint{2.926022in}{3.245523in}}%
\pgfpathlineto{\pgfqpoint{2.939585in}{3.224064in}}%
\pgfpathlineto{\pgfqpoint{2.953137in}{3.202946in}}%
\pgfpathlineto{\pgfqpoint{2.961175in}{3.216295in}}%
\pgfpathlineto{\pgfqpoint{2.969205in}{3.229818in}}%
\pgfpathlineto{\pgfqpoint{2.977227in}{3.243518in}}%
\pgfpathlineto{\pgfqpoint{2.985241in}{3.257395in}}%
\pgfpathlineto{\pgfqpoint{2.971702in}{3.278675in}}%
\pgfpathlineto{\pgfqpoint{2.958153in}{3.300296in}}%
\pgfpathlineto{\pgfqpoint{2.944593in}{3.322262in}}%
\pgfpathlineto{\pgfqpoint{2.931023in}{3.344576in}}%
\pgfpathlineto{\pgfqpoint{2.922996in}{3.330524in}}%
\pgfpathlineto{\pgfqpoint{2.914960in}{3.316658in}}%
\pgfpathlineto{\pgfqpoint{2.906917in}{3.302978in}}%
\pgfpathlineto{\pgfqpoint{2.898866in}{3.289479in}}%
\pgfpathclose%
\pgfusepath{fill}%
\end{pgfscope}%
\begin{pgfscope}%
\pgfpathrectangle{\pgfqpoint{1.150000in}{0.150000in}}{\pgfqpoint{5.700000in}{5.700000in}}%
\pgfusepath{clip}%
\pgfsetbuttcap%
\pgfsetroundjoin%
\definecolor{currentfill}{rgb}{0.257322,0.256130,0.526563}%
\pgfsetfillcolor{currentfill}%
\pgfsetfillopacity{0.800000}%
\pgfsetlinewidth{0.000000pt}%
\definecolor{currentstroke}{rgb}{0.000000,0.000000,0.000000}%
\pgfsetstrokecolor{currentstroke}%
\pgfsetdash{}{0pt}%
\pgfpathmoveto{\pgfqpoint{3.190562in}{2.798340in}}%
\pgfpathlineto{\pgfqpoint{3.203982in}{2.783826in}}%
\pgfpathlineto{\pgfqpoint{3.217399in}{2.769592in}}%
\pgfpathlineto{\pgfqpoint{3.230811in}{2.755635in}}%
\pgfpathlineto{\pgfqpoint{3.244220in}{2.741954in}}%
\pgfpathlineto{\pgfqpoint{3.252210in}{2.753934in}}%
\pgfpathlineto{\pgfqpoint{3.260193in}{2.766031in}}%
\pgfpathlineto{\pgfqpoint{3.268171in}{2.778249in}}%
\pgfpathlineto{\pgfqpoint{3.276142in}{2.790589in}}%
\pgfpathlineto{\pgfqpoint{3.262745in}{2.804397in}}%
\pgfpathlineto{\pgfqpoint{3.249344in}{2.818480in}}%
\pgfpathlineto{\pgfqpoint{3.235939in}{2.832842in}}%
\pgfpathlineto{\pgfqpoint{3.222531in}{2.847483in}}%
\pgfpathlineto{\pgfqpoint{3.214549in}{2.835005in}}%
\pgfpathlineto{\pgfqpoint{3.206560in}{2.822656in}}%
\pgfpathlineto{\pgfqpoint{3.198564in}{2.810435in}}%
\pgfpathlineto{\pgfqpoint{3.190562in}{2.798340in}}%
\pgfpathclose%
\pgfusepath{fill}%
\end{pgfscope}%
\begin{pgfscope}%
\pgfpathrectangle{\pgfqpoint{1.150000in}{0.150000in}}{\pgfqpoint{5.700000in}{5.700000in}}%
\pgfusepath{clip}%
\pgfsetbuttcap%
\pgfsetroundjoin%
\definecolor{currentfill}{rgb}{0.260571,0.246922,0.522828}%
\pgfsetfillcolor{currentfill}%
\pgfsetfillopacity{0.800000}%
\pgfsetlinewidth{0.000000pt}%
\definecolor{currentstroke}{rgb}{0.000000,0.000000,0.000000}%
\pgfsetstrokecolor{currentstroke}%
\pgfsetdash{}{0pt}%
\pgfpathmoveto{\pgfqpoint{4.329670in}{2.749332in}}%
\pgfpathlineto{\pgfqpoint{4.343169in}{2.747286in}}%
\pgfpathlineto{\pgfqpoint{4.356675in}{2.745436in}}%
\pgfpathlineto{\pgfqpoint{4.370189in}{2.743780in}}%
\pgfpathlineto{\pgfqpoint{4.383711in}{2.742319in}}%
\pgfpathlineto{\pgfqpoint{4.391383in}{2.753927in}}%
\pgfpathlineto{\pgfqpoint{4.399051in}{2.765645in}}%
\pgfpathlineto{\pgfqpoint{4.406715in}{2.777479in}}%
\pgfpathlineto{\pgfqpoint{4.414376in}{2.789433in}}%
\pgfpathlineto{\pgfqpoint{4.400864in}{2.791304in}}%
\pgfpathlineto{\pgfqpoint{4.387359in}{2.793368in}}%
\pgfpathlineto{\pgfqpoint{4.373863in}{2.795628in}}%
\pgfpathlineto{\pgfqpoint{4.360374in}{2.798083in}}%
\pgfpathlineto{\pgfqpoint{4.352703in}{2.785709in}}%
\pgfpathlineto{\pgfqpoint{4.345029in}{2.773462in}}%
\pgfpathlineto{\pgfqpoint{4.337352in}{2.761338in}}%
\pgfpathlineto{\pgfqpoint{4.329670in}{2.749332in}}%
\pgfpathclose%
\pgfusepath{fill}%
\end{pgfscope}%
\begin{pgfscope}%
\pgfpathrectangle{\pgfqpoint{1.150000in}{0.150000in}}{\pgfqpoint{5.700000in}{5.700000in}}%
\pgfusepath{clip}%
\pgfsetbuttcap%
\pgfsetroundjoin%
\definecolor{currentfill}{rgb}{0.204903,0.375746,0.553533}%
\pgfsetfillcolor{currentfill}%
\pgfsetfillopacity{0.800000}%
\pgfsetlinewidth{0.000000pt}%
\definecolor{currentstroke}{rgb}{0.000000,0.000000,0.000000}%
\pgfsetstrokecolor{currentstroke}%
\pgfsetdash{}{0pt}%
\pgfpathmoveto{\pgfqpoint{4.922789in}{3.063542in}}%
\pgfpathlineto{\pgfqpoint{4.936464in}{3.063214in}}%
\pgfpathlineto{\pgfqpoint{4.950149in}{3.063065in}}%
\pgfpathlineto{\pgfqpoint{4.963845in}{3.063096in}}%
\pgfpathlineto{\pgfqpoint{4.977552in}{3.063307in}}%
\pgfpathlineto{\pgfqpoint{4.985069in}{3.075782in}}%
\pgfpathlineto{\pgfqpoint{4.992586in}{3.088480in}}%
\pgfpathlineto{\pgfqpoint{5.000104in}{3.101409in}}%
\pgfpathlineto{\pgfqpoint{5.007621in}{3.114575in}}%
\pgfpathlineto{\pgfqpoint{4.993931in}{3.114996in}}%
\pgfpathlineto{\pgfqpoint{4.980252in}{3.115595in}}%
\pgfpathlineto{\pgfqpoint{4.966583in}{3.116373in}}%
\pgfpathlineto{\pgfqpoint{4.952924in}{3.117331in}}%
\pgfpathlineto{\pgfqpoint{4.945390in}{3.103523in}}%
\pgfpathlineto{\pgfqpoint{4.937857in}{3.089960in}}%
\pgfpathlineto{\pgfqpoint{4.930323in}{3.076636in}}%
\pgfpathlineto{\pgfqpoint{4.922789in}{3.063542in}}%
\pgfpathclose%
\pgfusepath{fill}%
\end{pgfscope}%
\begin{pgfscope}%
\pgfpathrectangle{\pgfqpoint{1.150000in}{0.150000in}}{\pgfqpoint{5.700000in}{5.700000in}}%
\pgfusepath{clip}%
\pgfsetbuttcap%
\pgfsetroundjoin%
\definecolor{currentfill}{rgb}{0.223925,0.334994,0.548053}%
\pgfsetfillcolor{currentfill}%
\pgfsetfillopacity{0.800000}%
\pgfsetlinewidth{0.000000pt}%
\definecolor{currentstroke}{rgb}{0.000000,0.000000,0.000000}%
\pgfsetstrokecolor{currentstroke}%
\pgfsetdash{}{0pt}%
\pgfpathmoveto{\pgfqpoint{3.029112in}{2.995220in}}%
\pgfpathlineto{\pgfqpoint{3.042601in}{2.977156in}}%
\pgfpathlineto{\pgfqpoint{3.056083in}{2.959403in}}%
\pgfpathlineto{\pgfqpoint{3.069557in}{2.941956in}}%
\pgfpathlineto{\pgfqpoint{3.083025in}{2.924815in}}%
\pgfpathlineto{\pgfqpoint{3.091053in}{2.937140in}}%
\pgfpathlineto{\pgfqpoint{3.099073in}{2.949608in}}%
\pgfpathlineto{\pgfqpoint{3.107086in}{2.962219in}}%
\pgfpathlineto{\pgfqpoint{3.115093in}{2.974976in}}%
\pgfpathlineto{\pgfqpoint{3.101637in}{2.992245in}}%
\pgfpathlineto{\pgfqpoint{3.088176in}{3.009819in}}%
\pgfpathlineto{\pgfqpoint{3.074707in}{3.027700in}}%
\pgfpathlineto{\pgfqpoint{3.061232in}{3.045892in}}%
\pgfpathlineto{\pgfqpoint{3.053213in}{3.032995in}}%
\pgfpathlineto{\pgfqpoint{3.045187in}{3.020252in}}%
\pgfpathlineto{\pgfqpoint{3.037153in}{3.007661in}}%
\pgfpathlineto{\pgfqpoint{3.029112in}{2.995220in}}%
\pgfpathclose%
\pgfusepath{fill}%
\end{pgfscope}%
\begin{pgfscope}%
\pgfpathrectangle{\pgfqpoint{1.150000in}{0.150000in}}{\pgfqpoint{5.700000in}{5.700000in}}%
\pgfusepath{clip}%
\pgfsetbuttcap%
\pgfsetroundjoin%
\definecolor{currentfill}{rgb}{0.265145,0.232956,0.516599}%
\pgfsetfillcolor{currentfill}%
\pgfsetfillopacity{0.800000}%
\pgfsetlinewidth{0.000000pt}%
\definecolor{currentstroke}{rgb}{0.000000,0.000000,0.000000}%
\pgfsetstrokecolor{currentstroke}%
\pgfsetdash{}{0pt}%
\pgfpathmoveto{\pgfqpoint{4.244953in}{2.711039in}}%
\pgfpathlineto{\pgfqpoint{4.258430in}{2.708582in}}%
\pgfpathlineto{\pgfqpoint{4.271915in}{2.706323in}}%
\pgfpathlineto{\pgfqpoint{4.285407in}{2.704262in}}%
\pgfpathlineto{\pgfqpoint{4.298907in}{2.702398in}}%
\pgfpathlineto{\pgfqpoint{4.306604in}{2.713977in}}%
\pgfpathlineto{\pgfqpoint{4.314296in}{2.725656in}}%
\pgfpathlineto{\pgfqpoint{4.321985in}{2.737440in}}%
\pgfpathlineto{\pgfqpoint{4.329670in}{2.749332in}}%
\pgfpathlineto{\pgfqpoint{4.316180in}{2.751574in}}%
\pgfpathlineto{\pgfqpoint{4.302697in}{2.754013in}}%
\pgfpathlineto{\pgfqpoint{4.289221in}{2.756650in}}%
\pgfpathlineto{\pgfqpoint{4.275753in}{2.759485in}}%
\pgfpathlineto{\pgfqpoint{4.268059in}{2.747204in}}%
\pgfpathlineto{\pgfqpoint{4.260361in}{2.735038in}}%
\pgfpathlineto{\pgfqpoint{4.252659in}{2.722985in}}%
\pgfpathlineto{\pgfqpoint{4.244953in}{2.711039in}}%
\pgfpathclose%
\pgfusepath{fill}%
\end{pgfscope}%
\begin{pgfscope}%
\pgfpathrectangle{\pgfqpoint{1.150000in}{0.150000in}}{\pgfqpoint{5.700000in}{5.700000in}}%
\pgfusepath{clip}%
\pgfsetbuttcap%
\pgfsetroundjoin%
\definecolor{currentfill}{rgb}{0.263663,0.237631,0.518762}%
\pgfsetfillcolor{currentfill}%
\pgfsetfillopacity{0.800000}%
\pgfsetlinewidth{0.000000pt}%
\definecolor{currentstroke}{rgb}{0.000000,0.000000,0.000000}%
\pgfsetstrokecolor{currentstroke}%
\pgfsetdash{}{0pt}%
\pgfpathmoveto{\pgfqpoint{3.244220in}{2.741954in}}%
\pgfpathlineto{\pgfqpoint{3.257625in}{2.728546in}}%
\pgfpathlineto{\pgfqpoint{3.271028in}{2.715410in}}%
\pgfpathlineto{\pgfqpoint{3.284427in}{2.702543in}}%
\pgfpathlineto{\pgfqpoint{3.297824in}{2.689943in}}%
\pgfpathlineto{\pgfqpoint{3.305803in}{2.701808in}}%
\pgfpathlineto{\pgfqpoint{3.313775in}{2.713782in}}%
\pgfpathlineto{\pgfqpoint{3.321741in}{2.725869in}}%
\pgfpathlineto{\pgfqpoint{3.329701in}{2.738070in}}%
\pgfpathlineto{\pgfqpoint{3.316315in}{2.750797in}}%
\pgfpathlineto{\pgfqpoint{3.302927in}{2.763791in}}%
\pgfpathlineto{\pgfqpoint{3.289536in}{2.777054in}}%
\pgfpathlineto{\pgfqpoint{3.276142in}{2.790589in}}%
\pgfpathlineto{\pgfqpoint{3.268171in}{2.778249in}}%
\pgfpathlineto{\pgfqpoint{3.260193in}{2.766031in}}%
\pgfpathlineto{\pgfqpoint{3.252210in}{2.753934in}}%
\pgfpathlineto{\pgfqpoint{3.244220in}{2.741954in}}%
\pgfpathclose%
\pgfusepath{fill}%
\end{pgfscope}%
\begin{pgfscope}%
\pgfpathrectangle{\pgfqpoint{1.150000in}{0.150000in}}{\pgfqpoint{5.700000in}{5.700000in}}%
\pgfusepath{clip}%
\pgfsetbuttcap%
\pgfsetroundjoin%
\definecolor{currentfill}{rgb}{0.197636,0.391528,0.554969}%
\pgfsetfillcolor{currentfill}%
\pgfsetfillopacity{0.800000}%
\pgfsetlinewidth{0.000000pt}%
\definecolor{currentstroke}{rgb}{0.000000,0.000000,0.000000}%
\pgfsetstrokecolor{currentstroke}%
\pgfsetdash{}{0pt}%
\pgfpathmoveto{\pgfqpoint{5.007621in}{3.114575in}}%
\pgfpathlineto{\pgfqpoint{5.021322in}{3.114333in}}%
\pgfpathlineto{\pgfqpoint{5.035035in}{3.114269in}}%
\pgfpathlineto{\pgfqpoint{5.048758in}{3.114383in}}%
\pgfpathlineto{\pgfqpoint{5.062493in}{3.114674in}}%
\pgfpathlineto{\pgfqpoint{5.069993in}{3.127435in}}%
\pgfpathlineto{\pgfqpoint{5.077495in}{3.140440in}}%
\pgfpathlineto{\pgfqpoint{5.084997in}{3.153698in}}%
\pgfpathlineto{\pgfqpoint{5.092501in}{3.167216in}}%
\pgfpathlineto{\pgfqpoint{5.078785in}{3.167588in}}%
\pgfpathlineto{\pgfqpoint{5.065079in}{3.168136in}}%
\pgfpathlineto{\pgfqpoint{5.051384in}{3.168862in}}%
\pgfpathlineto{\pgfqpoint{5.037700in}{3.169766in}}%
\pgfpathlineto{\pgfqpoint{5.030179in}{3.155574in}}%
\pgfpathlineto{\pgfqpoint{5.022659in}{3.141650in}}%
\pgfpathlineto{\pgfqpoint{5.015140in}{3.127987in}}%
\pgfpathlineto{\pgfqpoint{5.007621in}{3.114575in}}%
\pgfpathclose%
\pgfusepath{fill}%
\end{pgfscope}%
\begin{pgfscope}%
\pgfpathrectangle{\pgfqpoint{1.150000in}{0.150000in}}{\pgfqpoint{5.700000in}{5.700000in}}%
\pgfusepath{clip}%
\pgfsetbuttcap%
\pgfsetroundjoin%
\definecolor{currentfill}{rgb}{0.278012,0.180367,0.486697}%
\pgfsetfillcolor{currentfill}%
\pgfsetfillopacity{0.800000}%
\pgfsetlinewidth{0.000000pt}%
\definecolor{currentstroke}{rgb}{0.000000,0.000000,0.000000}%
\pgfsetstrokecolor{currentstroke}%
\pgfsetdash{}{0pt}%
\pgfpathmoveto{\pgfqpoint{3.852185in}{2.599357in}}%
\pgfpathlineto{\pgfqpoint{3.865580in}{2.593950in}}%
\pgfpathlineto{\pgfqpoint{3.878978in}{2.588759in}}%
\pgfpathlineto{\pgfqpoint{3.892381in}{2.583784in}}%
\pgfpathlineto{\pgfqpoint{3.905788in}{2.579023in}}%
\pgfpathlineto{\pgfqpoint{3.913598in}{2.590750in}}%
\pgfpathlineto{\pgfqpoint{3.921402in}{2.602555in}}%
\pgfpathlineto{\pgfqpoint{3.929202in}{2.614443in}}%
\pgfpathlineto{\pgfqpoint{3.936998in}{2.626415in}}%
\pgfpathlineto{\pgfqpoint{3.923598in}{2.631428in}}%
\pgfpathlineto{\pgfqpoint{3.910204in}{2.636656in}}%
\pgfpathlineto{\pgfqpoint{3.896813in}{2.642099in}}%
\pgfpathlineto{\pgfqpoint{3.883427in}{2.647758in}}%
\pgfpathlineto{\pgfqpoint{3.875624in}{2.635522in}}%
\pgfpathlineto{\pgfqpoint{3.867816in}{2.623378in}}%
\pgfpathlineto{\pgfqpoint{3.860003in}{2.611324in}}%
\pgfpathlineto{\pgfqpoint{3.852185in}{2.599357in}}%
\pgfpathclose%
\pgfusepath{fill}%
\end{pgfscope}%
\begin{pgfscope}%
\pgfpathrectangle{\pgfqpoint{1.150000in}{0.150000in}}{\pgfqpoint{5.700000in}{5.700000in}}%
\pgfusepath{clip}%
\pgfsetbuttcap%
\pgfsetroundjoin%
\definecolor{currentfill}{rgb}{0.269308,0.218818,0.509577}%
\pgfsetfillcolor{currentfill}%
\pgfsetfillopacity{0.800000}%
\pgfsetlinewidth{0.000000pt}%
\definecolor{currentstroke}{rgb}{0.000000,0.000000,0.000000}%
\pgfsetstrokecolor{currentstroke}%
\pgfsetdash{}{0pt}%
\pgfpathmoveto{\pgfqpoint{4.160216in}{2.674690in}}%
\pgfpathlineto{\pgfqpoint{4.173674in}{2.671778in}}%
\pgfpathlineto{\pgfqpoint{4.187139in}{2.669067in}}%
\pgfpathlineto{\pgfqpoint{4.200610in}{2.666557in}}%
\pgfpathlineto{\pgfqpoint{4.214089in}{2.664247in}}%
\pgfpathlineto{\pgfqpoint{4.221811in}{2.675805in}}%
\pgfpathlineto{\pgfqpoint{4.229529in}{2.687454in}}%
\pgfpathlineto{\pgfqpoint{4.237243in}{2.699197in}}%
\pgfpathlineto{\pgfqpoint{4.244953in}{2.711039in}}%
\pgfpathlineto{\pgfqpoint{4.231483in}{2.713696in}}%
\pgfpathlineto{\pgfqpoint{4.218020in}{2.716553in}}%
\pgfpathlineto{\pgfqpoint{4.204564in}{2.719610in}}%
\pgfpathlineto{\pgfqpoint{4.191115in}{2.722869in}}%
\pgfpathlineto{\pgfqpoint{4.183397in}{2.710669in}}%
\pgfpathlineto{\pgfqpoint{4.175674in}{2.698575in}}%
\pgfpathlineto{\pgfqpoint{4.167947in}{2.686583in}}%
\pgfpathlineto{\pgfqpoint{4.160216in}{2.674690in}}%
\pgfpathclose%
\pgfusepath{fill}%
\end{pgfscope}%
\begin{pgfscope}%
\pgfpathrectangle{\pgfqpoint{1.150000in}{0.150000in}}{\pgfqpoint{5.700000in}{5.700000in}}%
\pgfusepath{clip}%
\pgfsetbuttcap%
\pgfsetroundjoin%
\definecolor{currentfill}{rgb}{0.210503,0.363727,0.552206}%
\pgfsetfillcolor{currentfill}%
\pgfsetfillopacity{0.800000}%
\pgfsetlinewidth{0.000000pt}%
\definecolor{currentstroke}{rgb}{0.000000,0.000000,0.000000}%
\pgfsetstrokecolor{currentstroke}%
\pgfsetdash{}{0pt}%
\pgfpathmoveto{\pgfqpoint{2.975079in}{3.070635in}}%
\pgfpathlineto{\pgfqpoint{2.988600in}{3.051301in}}%
\pgfpathlineto{\pgfqpoint{3.002112in}{3.032290in}}%
\pgfpathlineto{\pgfqpoint{3.015616in}{3.013597in}}%
\pgfpathlineto{\pgfqpoint{3.029112in}{2.995220in}}%
\pgfpathlineto{\pgfqpoint{3.037153in}{3.007661in}}%
\pgfpathlineto{\pgfqpoint{3.045187in}{3.020252in}}%
\pgfpathlineto{\pgfqpoint{3.053213in}{3.032995in}}%
\pgfpathlineto{\pgfqpoint{3.061232in}{3.045892in}}%
\pgfpathlineto{\pgfqpoint{3.047749in}{3.064396in}}%
\pgfpathlineto{\pgfqpoint{3.034258in}{3.083217in}}%
\pgfpathlineto{\pgfqpoint{3.020760in}{3.102357in}}%
\pgfpathlineto{\pgfqpoint{3.007253in}{3.121818in}}%
\pgfpathlineto{\pgfqpoint{2.999221in}{3.108781in}}%
\pgfpathlineto{\pgfqpoint{2.991181in}{3.095906in}}%
\pgfpathlineto{\pgfqpoint{2.983134in}{3.083192in}}%
\pgfpathlineto{\pgfqpoint{2.975079in}{3.070635in}}%
\pgfpathclose%
\pgfusepath{fill}%
\end{pgfscope}%
\begin{pgfscope}%
\pgfpathrectangle{\pgfqpoint{1.150000in}{0.150000in}}{\pgfqpoint{5.700000in}{5.700000in}}%
\pgfusepath{clip}%
\pgfsetbuttcap%
\pgfsetroundjoin%
\definecolor{currentfill}{rgb}{0.187231,0.414746,0.556547}%
\pgfsetfillcolor{currentfill}%
\pgfsetfillopacity{0.800000}%
\pgfsetlinewidth{0.000000pt}%
\definecolor{currentstroke}{rgb}{0.000000,0.000000,0.000000}%
\pgfsetstrokecolor{currentstroke}%
\pgfsetdash{}{0pt}%
\pgfpathmoveto{\pgfqpoint{5.092501in}{3.167216in}}%
\pgfpathlineto{\pgfqpoint{5.106229in}{3.167022in}}%
\pgfpathlineto{\pgfqpoint{5.119968in}{3.167003in}}%
\pgfpathlineto{\pgfqpoint{5.133719in}{3.167161in}}%
\pgfpathlineto{\pgfqpoint{5.147481in}{3.167495in}}%
\pgfpathlineto{\pgfqpoint{5.154968in}{3.180598in}}%
\pgfpathlineto{\pgfqpoint{5.162456in}{3.193970in}}%
\pgfpathlineto{\pgfqpoint{5.169946in}{3.207618in}}%
\pgfpathlineto{\pgfqpoint{5.177439in}{3.221549in}}%
\pgfpathlineto{\pgfqpoint{5.163696in}{3.221910in}}%
\pgfpathlineto{\pgfqpoint{5.149964in}{3.222446in}}%
\pgfpathlineto{\pgfqpoint{5.136244in}{3.223158in}}%
\pgfpathlineto{\pgfqpoint{5.122534in}{3.224046in}}%
\pgfpathlineto{\pgfqpoint{5.115023in}{3.209409in}}%
\pgfpathlineto{\pgfqpoint{5.107514in}{3.195064in}}%
\pgfpathlineto{\pgfqpoint{5.100007in}{3.181002in}}%
\pgfpathlineto{\pgfqpoint{5.092501in}{3.167216in}}%
\pgfpathclose%
\pgfusepath{fill}%
\end{pgfscope}%
\begin{pgfscope}%
\pgfpathrectangle{\pgfqpoint{1.150000in}{0.150000in}}{\pgfqpoint{5.700000in}{5.700000in}}%
\pgfusepath{clip}%
\pgfsetbuttcap%
\pgfsetroundjoin%
\definecolor{currentfill}{rgb}{0.278826,0.175490,0.483397}%
\pgfsetfillcolor{currentfill}%
\pgfsetfillopacity{0.800000}%
\pgfsetlinewidth{0.000000pt}%
\definecolor{currentstroke}{rgb}{0.000000,0.000000,0.000000}%
\pgfsetstrokecolor{currentstroke}%
\pgfsetdash{}{0pt}%
\pgfpathmoveto{\pgfqpoint{3.628809in}{2.583634in}}%
\pgfpathlineto{\pgfqpoint{3.642182in}{2.575950in}}%
\pgfpathlineto{\pgfqpoint{3.655558in}{2.568498in}}%
\pgfpathlineto{\pgfqpoint{3.668936in}{2.561274in}}%
\pgfpathlineto{\pgfqpoint{3.682316in}{2.554279in}}%
\pgfpathlineto{\pgfqpoint{3.690190in}{2.566013in}}%
\pgfpathlineto{\pgfqpoint{3.698059in}{2.577827in}}%
\pgfpathlineto{\pgfqpoint{3.705923in}{2.589724in}}%
\pgfpathlineto{\pgfqpoint{3.713782in}{2.601706in}}%
\pgfpathlineto{\pgfqpoint{3.700410in}{2.608891in}}%
\pgfpathlineto{\pgfqpoint{3.687041in}{2.616304in}}%
\pgfpathlineto{\pgfqpoint{3.673674in}{2.623946in}}%
\pgfpathlineto{\pgfqpoint{3.660309in}{2.631819in}}%
\pgfpathlineto{\pgfqpoint{3.652442in}{2.619636in}}%
\pgfpathlineto{\pgfqpoint{3.644570in}{2.607546in}}%
\pgfpathlineto{\pgfqpoint{3.636692in}{2.595546in}}%
\pgfpathlineto{\pgfqpoint{3.628809in}{2.583634in}}%
\pgfpathclose%
\pgfusepath{fill}%
\end{pgfscope}%
\begin{pgfscope}%
\pgfpathrectangle{\pgfqpoint{1.150000in}{0.150000in}}{\pgfqpoint{5.700000in}{5.700000in}}%
\pgfusepath{clip}%
\pgfsetbuttcap%
\pgfsetroundjoin%
\definecolor{currentfill}{rgb}{0.278012,0.180367,0.486697}%
\pgfsetfillcolor{currentfill}%
\pgfsetfillopacity{0.800000}%
\pgfsetlinewidth{0.000000pt}%
\definecolor{currentstroke}{rgb}{0.000000,0.000000,0.000000}%
\pgfsetstrokecolor{currentstroke}%
\pgfsetdash{}{0pt}%
\pgfpathmoveto{\pgfqpoint{3.490215in}{2.605496in}}%
\pgfpathlineto{\pgfqpoint{3.503588in}{2.596074in}}%
\pgfpathlineto{\pgfqpoint{3.516962in}{2.586894in}}%
\pgfpathlineto{\pgfqpoint{3.530336in}{2.577955in}}%
\pgfpathlineto{\pgfqpoint{3.543711in}{2.569255in}}%
\pgfpathlineto{\pgfqpoint{3.551625in}{2.580985in}}%
\pgfpathlineto{\pgfqpoint{3.559533in}{2.592801in}}%
\pgfpathlineto{\pgfqpoint{3.567435in}{2.604707in}}%
\pgfpathlineto{\pgfqpoint{3.575333in}{2.616704in}}%
\pgfpathlineto{\pgfqpoint{3.561967in}{2.625562in}}%
\pgfpathlineto{\pgfqpoint{3.548602in}{2.634660in}}%
\pgfpathlineto{\pgfqpoint{3.535238in}{2.643998in}}%
\pgfpathlineto{\pgfqpoint{3.521875in}{2.653579in}}%
\pgfpathlineto{\pgfqpoint{3.513968in}{2.641412in}}%
\pgfpathlineto{\pgfqpoint{3.506056in}{2.629344in}}%
\pgfpathlineto{\pgfqpoint{3.498138in}{2.617373in}}%
\pgfpathlineto{\pgfqpoint{3.490215in}{2.605496in}}%
\pgfpathclose%
\pgfusepath{fill}%
\end{pgfscope}%
\begin{pgfscope}%
\pgfpathrectangle{\pgfqpoint{1.150000in}{0.150000in}}{\pgfqpoint{5.700000in}{5.700000in}}%
\pgfusepath{clip}%
\pgfsetbuttcap%
\pgfsetroundjoin%
\definecolor{currentfill}{rgb}{0.269308,0.218818,0.509577}%
\pgfsetfillcolor{currentfill}%
\pgfsetfillopacity{0.800000}%
\pgfsetlinewidth{0.000000pt}%
\definecolor{currentstroke}{rgb}{0.000000,0.000000,0.000000}%
\pgfsetstrokecolor{currentstroke}%
\pgfsetdash{}{0pt}%
\pgfpathmoveto{\pgfqpoint{3.297824in}{2.689943in}}%
\pgfpathlineto{\pgfqpoint{3.311219in}{2.677608in}}%
\pgfpathlineto{\pgfqpoint{3.324611in}{2.665537in}}%
\pgfpathlineto{\pgfqpoint{3.338002in}{2.653727in}}%
\pgfpathlineto{\pgfqpoint{3.351390in}{2.642176in}}%
\pgfpathlineto{\pgfqpoint{3.359358in}{2.653925in}}%
\pgfpathlineto{\pgfqpoint{3.367319in}{2.665777in}}%
\pgfpathlineto{\pgfqpoint{3.375274in}{2.677733in}}%
\pgfpathlineto{\pgfqpoint{3.383223in}{2.689795in}}%
\pgfpathlineto{\pgfqpoint{3.369845in}{2.701472in}}%
\pgfpathlineto{\pgfqpoint{3.356466in}{2.713410in}}%
\pgfpathlineto{\pgfqpoint{3.343084in}{2.725608in}}%
\pgfpathlineto{\pgfqpoint{3.329701in}{2.738070in}}%
\pgfpathlineto{\pgfqpoint{3.321741in}{2.725869in}}%
\pgfpathlineto{\pgfqpoint{3.313775in}{2.713782in}}%
\pgfpathlineto{\pgfqpoint{3.305803in}{2.701808in}}%
\pgfpathlineto{\pgfqpoint{3.297824in}{2.689943in}}%
\pgfpathclose%
\pgfusepath{fill}%
\end{pgfscope}%
\begin{pgfscope}%
\pgfpathrectangle{\pgfqpoint{1.150000in}{0.150000in}}{\pgfqpoint{5.700000in}{5.700000in}}%
\pgfusepath{clip}%
\pgfsetbuttcap%
\pgfsetroundjoin%
\definecolor{currentfill}{rgb}{0.273006,0.204520,0.501721}%
\pgfsetfillcolor{currentfill}%
\pgfsetfillopacity{0.800000}%
\pgfsetlinewidth{0.000000pt}%
\definecolor{currentstroke}{rgb}{0.000000,0.000000,0.000000}%
\pgfsetstrokecolor{currentstroke}%
\pgfsetdash{}{0pt}%
\pgfpathmoveto{\pgfqpoint{4.075451in}{2.640444in}}%
\pgfpathlineto{\pgfqpoint{4.088891in}{2.637032in}}%
\pgfpathlineto{\pgfqpoint{4.102338in}{2.633825in}}%
\pgfpathlineto{\pgfqpoint{4.115791in}{2.630823in}}%
\pgfpathlineto{\pgfqpoint{4.129250in}{2.628023in}}%
\pgfpathlineto{\pgfqpoint{4.136998in}{2.639562in}}%
\pgfpathlineto{\pgfqpoint{4.144742in}{2.651184in}}%
\pgfpathlineto{\pgfqpoint{4.152481in}{2.662892in}}%
\pgfpathlineto{\pgfqpoint{4.160216in}{2.674690in}}%
\pgfpathlineto{\pgfqpoint{4.146765in}{2.677805in}}%
\pgfpathlineto{\pgfqpoint{4.133321in}{2.681123in}}%
\pgfpathlineto{\pgfqpoint{4.119882in}{2.684645in}}%
\pgfpathlineto{\pgfqpoint{4.106450in}{2.688372in}}%
\pgfpathlineto{\pgfqpoint{4.098707in}{2.676247in}}%
\pgfpathlineto{\pgfqpoint{4.090959in}{2.664220in}}%
\pgfpathlineto{\pgfqpoint{4.083207in}{2.652287in}}%
\pgfpathlineto{\pgfqpoint{4.075451in}{2.640444in}}%
\pgfpathclose%
\pgfusepath{fill}%
\end{pgfscope}%
\begin{pgfscope}%
\pgfpathrectangle{\pgfqpoint{1.150000in}{0.150000in}}{\pgfqpoint{5.700000in}{5.700000in}}%
\pgfusepath{clip}%
\pgfsetbuttcap%
\pgfsetroundjoin%
\definecolor{currentfill}{rgb}{0.179019,0.433756,0.557430}%
\pgfsetfillcolor{currentfill}%
\pgfsetfillopacity{0.800000}%
\pgfsetlinewidth{0.000000pt}%
\definecolor{currentstroke}{rgb}{0.000000,0.000000,0.000000}%
\pgfsetstrokecolor{currentstroke}%
\pgfsetdash{}{0pt}%
\pgfpathmoveto{\pgfqpoint{5.177439in}{3.221549in}}%
\pgfpathlineto{\pgfqpoint{5.191193in}{3.221364in}}%
\pgfpathlineto{\pgfqpoint{5.204959in}{3.221353in}}%
\pgfpathlineto{\pgfqpoint{5.218737in}{3.221517in}}%
\pgfpathlineto{\pgfqpoint{5.232526in}{3.221855in}}%
\pgfpathlineto{\pgfqpoint{5.240001in}{3.235365in}}%
\pgfpathlineto{\pgfqpoint{5.247479in}{3.249167in}}%
\pgfpathlineto{\pgfqpoint{5.254960in}{3.263271in}}%
\pgfpathlineto{\pgfqpoint{5.262444in}{3.277684in}}%
\pgfpathlineto{\pgfqpoint{5.248675in}{3.278072in}}%
\pgfpathlineto{\pgfqpoint{5.234918in}{3.278633in}}%
\pgfpathlineto{\pgfqpoint{5.221172in}{3.279369in}}%
\pgfpathlineto{\pgfqpoint{5.207437in}{3.280280in}}%
\pgfpathlineto{\pgfqpoint{5.199933in}{3.265130in}}%
\pgfpathlineto{\pgfqpoint{5.192432in}{3.250297in}}%
\pgfpathlineto{\pgfqpoint{5.184934in}{3.235773in}}%
\pgfpathlineto{\pgfqpoint{5.177439in}{3.221549in}}%
\pgfpathclose%
\pgfusepath{fill}%
\end{pgfscope}%
\begin{pgfscope}%
\pgfpathrectangle{\pgfqpoint{1.150000in}{0.150000in}}{\pgfqpoint{5.700000in}{5.700000in}}%
\pgfusepath{clip}%
\pgfsetbuttcap%
\pgfsetroundjoin%
\definecolor{currentfill}{rgb}{0.279574,0.170599,0.479997}%
\pgfsetfillcolor{currentfill}%
\pgfsetfillopacity{0.800000}%
\pgfsetlinewidth{0.000000pt}%
\definecolor{currentstroke}{rgb}{0.000000,0.000000,0.000000}%
\pgfsetstrokecolor{currentstroke}%
\pgfsetdash{}{0pt}%
\pgfpathmoveto{\pgfqpoint{3.767297in}{2.575223in}}%
\pgfpathlineto{\pgfqpoint{3.780684in}{2.569160in}}%
\pgfpathlineto{\pgfqpoint{3.794074in}{2.563319in}}%
\pgfpathlineto{\pgfqpoint{3.807469in}{2.557697in}}%
\pgfpathlineto{\pgfqpoint{3.820867in}{2.552294in}}%
\pgfpathlineto{\pgfqpoint{3.828704in}{2.563944in}}%
\pgfpathlineto{\pgfqpoint{3.836536in}{2.575670in}}%
\pgfpathlineto{\pgfqpoint{3.844363in}{2.587473in}}%
\pgfpathlineto{\pgfqpoint{3.852185in}{2.599357in}}%
\pgfpathlineto{\pgfqpoint{3.838795in}{2.604981in}}%
\pgfpathlineto{\pgfqpoint{3.825409in}{2.610824in}}%
\pgfpathlineto{\pgfqpoint{3.812027in}{2.616887in}}%
\pgfpathlineto{\pgfqpoint{3.798649in}{2.623171in}}%
\pgfpathlineto{\pgfqpoint{3.790818in}{2.611054in}}%
\pgfpathlineto{\pgfqpoint{3.782983in}{2.599026in}}%
\pgfpathlineto{\pgfqpoint{3.775142in}{2.587083in}}%
\pgfpathlineto{\pgfqpoint{3.767297in}{2.575223in}}%
\pgfpathclose%
\pgfusepath{fill}%
\end{pgfscope}%
\begin{pgfscope}%
\pgfpathrectangle{\pgfqpoint{1.150000in}{0.150000in}}{\pgfqpoint{5.700000in}{5.700000in}}%
\pgfusepath{clip}%
\pgfsetbuttcap%
\pgfsetroundjoin%
\definecolor{currentfill}{rgb}{0.195860,0.395433,0.555276}%
\pgfsetfillcolor{currentfill}%
\pgfsetfillopacity{0.800000}%
\pgfsetlinewidth{0.000000pt}%
\definecolor{currentstroke}{rgb}{0.000000,0.000000,0.000000}%
\pgfsetstrokecolor{currentstroke}%
\pgfsetdash{}{0pt}%
\pgfpathmoveto{\pgfqpoint{2.920908in}{3.151249in}}%
\pgfpathlineto{\pgfqpoint{2.934465in}{3.130597in}}%
\pgfpathlineto{\pgfqpoint{2.948012in}{3.110280in}}%
\pgfpathlineto{\pgfqpoint{2.961550in}{3.090293in}}%
\pgfpathlineto{\pgfqpoint{2.975079in}{3.070635in}}%
\pgfpathlineto{\pgfqpoint{2.983134in}{3.083192in}}%
\pgfpathlineto{\pgfqpoint{2.991181in}{3.095906in}}%
\pgfpathlineto{\pgfqpoint{2.999221in}{3.108781in}}%
\pgfpathlineto{\pgfqpoint{3.007253in}{3.121818in}}%
\pgfpathlineto{\pgfqpoint{2.993737in}{3.141605in}}%
\pgfpathlineto{\pgfqpoint{2.980213in}{3.161719in}}%
\pgfpathlineto{\pgfqpoint{2.966680in}{3.182165in}}%
\pgfpathlineto{\pgfqpoint{2.953137in}{3.202946in}}%
\pgfpathlineto{\pgfqpoint{2.945092in}{3.189768in}}%
\pgfpathlineto{\pgfqpoint{2.937038in}{3.176761in}}%
\pgfpathlineto{\pgfqpoint{2.928977in}{3.163922in}}%
\pgfpathlineto{\pgfqpoint{2.920908in}{3.151249in}}%
\pgfpathclose%
\pgfusepath{fill}%
\end{pgfscope}%
\begin{pgfscope}%
\pgfpathrectangle{\pgfqpoint{1.150000in}{0.150000in}}{\pgfqpoint{5.700000in}{5.700000in}}%
\pgfusepath{clip}%
\pgfsetbuttcap%
\pgfsetroundjoin%
\definecolor{currentfill}{rgb}{0.274128,0.199721,0.498911}%
\pgfsetfillcolor{currentfill}%
\pgfsetfillopacity{0.800000}%
\pgfsetlinewidth{0.000000pt}%
\definecolor{currentstroke}{rgb}{0.000000,0.000000,0.000000}%
\pgfsetstrokecolor{currentstroke}%
\pgfsetdash{}{0pt}%
\pgfpathmoveto{\pgfqpoint{3.351390in}{2.642176in}}%
\pgfpathlineto{\pgfqpoint{3.364778in}{2.630883in}}%
\pgfpathlineto{\pgfqpoint{3.378164in}{2.619846in}}%
\pgfpathlineto{\pgfqpoint{3.391549in}{2.609062in}}%
\pgfpathlineto{\pgfqpoint{3.404933in}{2.598531in}}%
\pgfpathlineto{\pgfqpoint{3.412889in}{2.610165in}}%
\pgfpathlineto{\pgfqpoint{3.420840in}{2.621893in}}%
\pgfpathlineto{\pgfqpoint{3.428785in}{2.633718in}}%
\pgfpathlineto{\pgfqpoint{3.436723in}{2.645641in}}%
\pgfpathlineto{\pgfqpoint{3.423350in}{2.656300in}}%
\pgfpathlineto{\pgfqpoint{3.409975in}{2.667210in}}%
\pgfpathlineto{\pgfqpoint{3.396600in}{2.678374in}}%
\pgfpathlineto{\pgfqpoint{3.383223in}{2.689795in}}%
\pgfpathlineto{\pgfqpoint{3.375274in}{2.677733in}}%
\pgfpathlineto{\pgfqpoint{3.367319in}{2.665777in}}%
\pgfpathlineto{\pgfqpoint{3.359358in}{2.653925in}}%
\pgfpathlineto{\pgfqpoint{3.351390in}{2.642176in}}%
\pgfpathclose%
\pgfusepath{fill}%
\end{pgfscope}%
\begin{pgfscope}%
\pgfpathrectangle{\pgfqpoint{1.150000in}{0.150000in}}{\pgfqpoint{5.700000in}{5.700000in}}%
\pgfusepath{clip}%
\pgfsetbuttcap%
\pgfsetroundjoin%
\definecolor{currentfill}{rgb}{0.276194,0.190074,0.493001}%
\pgfsetfillcolor{currentfill}%
\pgfsetfillopacity{0.800000}%
\pgfsetlinewidth{0.000000pt}%
\definecolor{currentstroke}{rgb}{0.000000,0.000000,0.000000}%
\pgfsetstrokecolor{currentstroke}%
\pgfsetdash{}{0pt}%
\pgfpathmoveto{\pgfqpoint{3.990645in}{2.608485in}}%
\pgfpathlineto{\pgfqpoint{4.004070in}{2.604529in}}%
\pgfpathlineto{\pgfqpoint{4.017501in}{2.600781in}}%
\pgfpathlineto{\pgfqpoint{4.030938in}{2.597241in}}%
\pgfpathlineto{\pgfqpoint{4.044380in}{2.593907in}}%
\pgfpathlineto{\pgfqpoint{4.052154in}{2.605424in}}%
\pgfpathlineto{\pgfqpoint{4.059924in}{2.617016in}}%
\pgfpathlineto{\pgfqpoint{4.067690in}{2.628689in}}%
\pgfpathlineto{\pgfqpoint{4.075451in}{2.640444in}}%
\pgfpathlineto{\pgfqpoint{4.062016in}{2.644062in}}%
\pgfpathlineto{\pgfqpoint{4.048588in}{2.647886in}}%
\pgfpathlineto{\pgfqpoint{4.035165in}{2.651918in}}%
\pgfpathlineto{\pgfqpoint{4.021748in}{2.656158in}}%
\pgfpathlineto{\pgfqpoint{4.013979in}{2.644107in}}%
\pgfpathlineto{\pgfqpoint{4.006206in}{2.632147in}}%
\pgfpathlineto{\pgfqpoint{3.998428in}{2.620274in}}%
\pgfpathlineto{\pgfqpoint{3.990645in}{2.608485in}}%
\pgfpathclose%
\pgfusepath{fill}%
\end{pgfscope}%
\begin{pgfscope}%
\pgfpathrectangle{\pgfqpoint{1.150000in}{0.150000in}}{\pgfqpoint{5.700000in}{5.700000in}}%
\pgfusepath{clip}%
\pgfsetbuttcap%
\pgfsetroundjoin%
\definecolor{currentfill}{rgb}{0.171176,0.452530,0.557965}%
\pgfsetfillcolor{currentfill}%
\pgfsetfillopacity{0.800000}%
\pgfsetlinewidth{0.000000pt}%
\definecolor{currentstroke}{rgb}{0.000000,0.000000,0.000000}%
\pgfsetstrokecolor{currentstroke}%
\pgfsetdash{}{0pt}%
\pgfpathmoveto{\pgfqpoint{5.262444in}{3.277684in}}%
\pgfpathlineto{\pgfqpoint{5.276224in}{3.277470in}}%
\pgfpathlineto{\pgfqpoint{5.290017in}{3.277429in}}%
\pgfpathlineto{\pgfqpoint{5.303821in}{3.277561in}}%
\pgfpathlineto{\pgfqpoint{5.317637in}{3.277866in}}%
\pgfpathlineto{\pgfqpoint{5.325104in}{3.291851in}}%
\pgfpathlineto{\pgfqpoint{5.332574in}{3.306155in}}%
\pgfpathlineto{\pgfqpoint{5.340049in}{3.320786in}}%
\pgfpathlineto{\pgfqpoint{5.326249in}{3.321047in}}%
\pgfpathlineto{\pgfqpoint{5.312461in}{3.321480in}}%
\pgfpathlineto{\pgfqpoint{5.298685in}{3.322085in}}%
\pgfpathlineto{\pgfqpoint{5.284921in}{3.322864in}}%
\pgfpathlineto{\pgfqpoint{5.277424in}{3.307472in}}%
\pgfpathlineto{\pgfqpoint{5.269932in}{3.292415in}}%
\pgfpathlineto{\pgfqpoint{5.262444in}{3.277684in}}%
\pgfpathclose%
\pgfusepath{fill}%
\end{pgfscope}%
\begin{pgfscope}%
\pgfpathrectangle{\pgfqpoint{1.150000in}{0.150000in}}{\pgfqpoint{5.700000in}{5.700000in}}%
\pgfusepath{clip}%
\pgfsetbuttcap%
\pgfsetroundjoin%
\definecolor{currentfill}{rgb}{0.279574,0.170599,0.479997}%
\pgfsetfillcolor{currentfill}%
\pgfsetfillopacity{0.800000}%
\pgfsetlinewidth{0.000000pt}%
\definecolor{currentstroke}{rgb}{0.000000,0.000000,0.000000}%
\pgfsetstrokecolor{currentstroke}%
\pgfsetdash{}{0pt}%
\pgfpathmoveto{\pgfqpoint{3.543711in}{2.569255in}}%
\pgfpathlineto{\pgfqpoint{3.557087in}{2.560793in}}%
\pgfpathlineto{\pgfqpoint{3.570465in}{2.552567in}}%
\pgfpathlineto{\pgfqpoint{3.583844in}{2.544576in}}%
\pgfpathlineto{\pgfqpoint{3.597224in}{2.536819in}}%
\pgfpathlineto{\pgfqpoint{3.605129in}{2.548402in}}%
\pgfpathlineto{\pgfqpoint{3.613027in}{2.560064in}}%
\pgfpathlineto{\pgfqpoint{3.620921in}{2.571808in}}%
\pgfpathlineto{\pgfqpoint{3.628809in}{2.583634in}}%
\pgfpathlineto{\pgfqpoint{3.615438in}{2.591550in}}%
\pgfpathlineto{\pgfqpoint{3.602068in}{2.599699in}}%
\pgfpathlineto{\pgfqpoint{3.588700in}{2.608083in}}%
\pgfpathlineto{\pgfqpoint{3.575333in}{2.616704in}}%
\pgfpathlineto{\pgfqpoint{3.567435in}{2.604707in}}%
\pgfpathlineto{\pgfqpoint{3.559533in}{2.592801in}}%
\pgfpathlineto{\pgfqpoint{3.551625in}{2.580985in}}%
\pgfpathlineto{\pgfqpoint{3.543711in}{2.569255in}}%
\pgfpathclose%
\pgfusepath{fill}%
\end{pgfscope}%
\begin{pgfscope}%
\pgfpathrectangle{\pgfqpoint{1.150000in}{0.150000in}}{\pgfqpoint{5.700000in}{5.700000in}}%
\pgfusepath{clip}%
\pgfsetbuttcap%
\pgfsetroundjoin%
\definecolor{currentfill}{rgb}{0.237441,0.305202,0.541921}%
\pgfsetfillcolor{currentfill}%
\pgfsetfillopacity{0.800000}%
\pgfsetlinewidth{0.000000pt}%
\definecolor{currentstroke}{rgb}{0.000000,0.000000,0.000000}%
\pgfsetstrokecolor{currentstroke}%
\pgfsetdash{}{0pt}%
\pgfpathmoveto{\pgfqpoint{4.638118in}{2.871507in}}%
\pgfpathlineto{\pgfqpoint{4.651725in}{2.871190in}}%
\pgfpathlineto{\pgfqpoint{4.665343in}{2.871060in}}%
\pgfpathlineto{\pgfqpoint{4.678970in}{2.871115in}}%
\pgfpathlineto{\pgfqpoint{4.692607in}{2.871355in}}%
\pgfpathlineto{\pgfqpoint{4.700194in}{2.882654in}}%
\pgfpathlineto{\pgfqpoint{4.707779in}{2.894097in}}%
\pgfpathlineto{\pgfqpoint{4.715361in}{2.905687in}}%
\pgfpathlineto{\pgfqpoint{4.722940in}{2.917433in}}%
\pgfpathlineto{\pgfqpoint{4.709316in}{2.917698in}}%
\pgfpathlineto{\pgfqpoint{4.695702in}{2.918148in}}%
\pgfpathlineto{\pgfqpoint{4.682097in}{2.918784in}}%
\pgfpathlineto{\pgfqpoint{4.668502in}{2.919605in}}%
\pgfpathlineto{\pgfqpoint{4.660910in}{2.907343in}}%
\pgfpathlineto{\pgfqpoint{4.653315in}{2.895243in}}%
\pgfpathlineto{\pgfqpoint{4.645718in}{2.883300in}}%
\pgfpathlineto{\pgfqpoint{4.638118in}{2.871507in}}%
\pgfpathclose%
\pgfusepath{fill}%
\end{pgfscope}%
\begin{pgfscope}%
\pgfpathrectangle{\pgfqpoint{1.150000in}{0.150000in}}{\pgfqpoint{5.700000in}{5.700000in}}%
\pgfusepath{clip}%
\pgfsetbuttcap%
\pgfsetroundjoin%
\definecolor{currentfill}{rgb}{0.244972,0.287675,0.537260}%
\pgfsetfillcolor{currentfill}%
\pgfsetfillopacity{0.800000}%
\pgfsetlinewidth{0.000000pt}%
\definecolor{currentstroke}{rgb}{0.000000,0.000000,0.000000}%
\pgfsetstrokecolor{currentstroke}%
\pgfsetdash{}{0pt}%
\pgfpathmoveto{\pgfqpoint{4.553310in}{2.826969in}}%
\pgfpathlineto{\pgfqpoint{4.566891in}{2.826377in}}%
\pgfpathlineto{\pgfqpoint{4.580481in}{2.825972in}}%
\pgfpathlineto{\pgfqpoint{4.594081in}{2.825756in}}%
\pgfpathlineto{\pgfqpoint{4.607690in}{2.825727in}}%
\pgfpathlineto{\pgfqpoint{4.615302in}{2.836974in}}%
\pgfpathlineto{\pgfqpoint{4.622910in}{2.848350in}}%
\pgfpathlineto{\pgfqpoint{4.630516in}{2.859859in}}%
\pgfpathlineto{\pgfqpoint{4.638118in}{2.871507in}}%
\pgfpathlineto{\pgfqpoint{4.624521in}{2.872010in}}%
\pgfpathlineto{\pgfqpoint{4.610933in}{2.872700in}}%
\pgfpathlineto{\pgfqpoint{4.597354in}{2.873577in}}%
\pgfpathlineto{\pgfqpoint{4.583785in}{2.874643in}}%
\pgfpathlineto{\pgfqpoint{4.576171in}{2.862510in}}%
\pgfpathlineto{\pgfqpoint{4.568554in}{2.850524in}}%
\pgfpathlineto{\pgfqpoint{4.560933in}{2.838678in}}%
\pgfpathlineto{\pgfqpoint{4.553310in}{2.826969in}}%
\pgfpathclose%
\pgfusepath{fill}%
\end{pgfscope}%
\begin{pgfscope}%
\pgfpathrectangle{\pgfqpoint{1.150000in}{0.150000in}}{\pgfqpoint{5.700000in}{5.700000in}}%
\pgfusepath{clip}%
\pgfsetbuttcap%
\pgfsetroundjoin%
\definecolor{currentfill}{rgb}{0.280255,0.165693,0.476498}%
\pgfsetfillcolor{currentfill}%
\pgfsetfillopacity{0.800000}%
\pgfsetlinewidth{0.000000pt}%
\definecolor{currentstroke}{rgb}{0.000000,0.000000,0.000000}%
\pgfsetstrokecolor{currentstroke}%
\pgfsetdash{}{0pt}%
\pgfpathmoveto{\pgfqpoint{3.682316in}{2.554279in}}%
\pgfpathlineto{\pgfqpoint{3.695699in}{2.547511in}}%
\pgfpathlineto{\pgfqpoint{3.709085in}{2.540969in}}%
\pgfpathlineto{\pgfqpoint{3.722474in}{2.534651in}}%
\pgfpathlineto{\pgfqpoint{3.735866in}{2.528557in}}%
\pgfpathlineto{\pgfqpoint{3.743731in}{2.540112in}}%
\pgfpathlineto{\pgfqpoint{3.751591in}{2.551740in}}%
\pgfpathlineto{\pgfqpoint{3.759447in}{2.563443in}}%
\pgfpathlineto{\pgfqpoint{3.767297in}{2.575223in}}%
\pgfpathlineto{\pgfqpoint{3.753913in}{2.581508in}}%
\pgfpathlineto{\pgfqpoint{3.740533in}{2.588015in}}%
\pgfpathlineto{\pgfqpoint{3.727156in}{2.594748in}}%
\pgfpathlineto{\pgfqpoint{3.713782in}{2.601706in}}%
\pgfpathlineto{\pgfqpoint{3.705923in}{2.589724in}}%
\pgfpathlineto{\pgfqpoint{3.698059in}{2.577827in}}%
\pgfpathlineto{\pgfqpoint{3.690190in}{2.566013in}}%
\pgfpathlineto{\pgfqpoint{3.682316in}{2.554279in}}%
\pgfpathclose%
\pgfusepath{fill}%
\end{pgfscope}%
\begin{pgfscope}%
\pgfpathrectangle{\pgfqpoint{1.150000in}{0.150000in}}{\pgfqpoint{5.700000in}{5.700000in}}%
\pgfusepath{clip}%
\pgfsetbuttcap%
\pgfsetroundjoin%
\definecolor{currentfill}{rgb}{0.278012,0.180367,0.486697}%
\pgfsetfillcolor{currentfill}%
\pgfsetfillopacity{0.800000}%
\pgfsetlinewidth{0.000000pt}%
\definecolor{currentstroke}{rgb}{0.000000,0.000000,0.000000}%
\pgfsetstrokecolor{currentstroke}%
\pgfsetdash{}{0pt}%
\pgfpathmoveto{\pgfqpoint{3.905788in}{2.579023in}}%
\pgfpathlineto{\pgfqpoint{3.919201in}{2.574476in}}%
\pgfpathlineto{\pgfqpoint{3.932618in}{2.570141in}}%
\pgfpathlineto{\pgfqpoint{3.946041in}{2.566018in}}%
\pgfpathlineto{\pgfqpoint{3.959468in}{2.562105in}}%
\pgfpathlineto{\pgfqpoint{3.967270in}{2.573590in}}%
\pgfpathlineto{\pgfqpoint{3.975066in}{2.585147in}}%
\pgfpathlineto{\pgfqpoint{3.982858in}{2.596777in}}%
\pgfpathlineto{\pgfqpoint{3.990645in}{2.608485in}}%
\pgfpathlineto{\pgfqpoint{3.977226in}{2.612651in}}%
\pgfpathlineto{\pgfqpoint{3.963811in}{2.617028in}}%
\pgfpathlineto{\pgfqpoint{3.950402in}{2.621615in}}%
\pgfpathlineto{\pgfqpoint{3.936998in}{2.626415in}}%
\pgfpathlineto{\pgfqpoint{3.929202in}{2.614443in}}%
\pgfpathlineto{\pgfqpoint{3.921402in}{2.602555in}}%
\pgfpathlineto{\pgfqpoint{3.913598in}{2.590750in}}%
\pgfpathlineto{\pgfqpoint{3.905788in}{2.579023in}}%
\pgfpathclose%
\pgfusepath{fill}%
\end{pgfscope}%
\begin{pgfscope}%
\pgfpathrectangle{\pgfqpoint{1.150000in}{0.150000in}}{\pgfqpoint{5.700000in}{5.700000in}}%
\pgfusepath{clip}%
\pgfsetbuttcap%
\pgfsetroundjoin%
\definecolor{currentfill}{rgb}{0.227802,0.326594,0.546532}%
\pgfsetfillcolor{currentfill}%
\pgfsetfillopacity{0.800000}%
\pgfsetlinewidth{0.000000pt}%
\definecolor{currentstroke}{rgb}{0.000000,0.000000,0.000000}%
\pgfsetstrokecolor{currentstroke}%
\pgfsetdash{}{0pt}%
\pgfpathmoveto{\pgfqpoint{4.722940in}{2.917433in}}%
\pgfpathlineto{\pgfqpoint{4.736575in}{2.917352in}}%
\pgfpathlineto{\pgfqpoint{4.750219in}{2.917455in}}%
\pgfpathlineto{\pgfqpoint{4.763874in}{2.917741in}}%
\pgfpathlineto{\pgfqpoint{4.777540in}{2.918211in}}%
\pgfpathlineto{\pgfqpoint{4.785103in}{2.929592in}}%
\pgfpathlineto{\pgfqpoint{4.792665in}{2.941131in}}%
\pgfpathlineto{\pgfqpoint{4.800225in}{2.952836in}}%
\pgfpathlineto{\pgfqpoint{4.807782in}{2.964711in}}%
\pgfpathlineto{\pgfqpoint{4.794131in}{2.964779in}}%
\pgfpathlineto{\pgfqpoint{4.780489in}{2.965029in}}%
\pgfpathlineto{\pgfqpoint{4.766858in}{2.965463in}}%
\pgfpathlineto{\pgfqpoint{4.753237in}{2.966080in}}%
\pgfpathlineto{\pgfqpoint{4.745666in}{2.953656in}}%
\pgfpathlineto{\pgfqpoint{4.738093in}{2.941411in}}%
\pgfpathlineto{\pgfqpoint{4.730518in}{2.929339in}}%
\pgfpathlineto{\pgfqpoint{4.722940in}{2.917433in}}%
\pgfpathclose%
\pgfusepath{fill}%
\end{pgfscope}%
\begin{pgfscope}%
\pgfpathrectangle{\pgfqpoint{1.150000in}{0.150000in}}{\pgfqpoint{5.700000in}{5.700000in}}%
\pgfusepath{clip}%
\pgfsetbuttcap%
\pgfsetroundjoin%
\definecolor{currentfill}{rgb}{0.182256,0.426184,0.557120}%
\pgfsetfillcolor{currentfill}%
\pgfsetfillopacity{0.800000}%
\pgfsetlinewidth{0.000000pt}%
\definecolor{currentstroke}{rgb}{0.000000,0.000000,0.000000}%
\pgfsetstrokecolor{currentstroke}%
\pgfsetdash{}{0pt}%
\pgfpathmoveto{\pgfqpoint{2.866578in}{3.237268in}}%
\pgfpathlineto{\pgfqpoint{2.880176in}{3.215245in}}%
\pgfpathlineto{\pgfqpoint{2.893764in}{3.193570in}}%
\pgfpathlineto{\pgfqpoint{2.907341in}{3.172239in}}%
\pgfpathlineto{\pgfqpoint{2.920908in}{3.151249in}}%
\pgfpathlineto{\pgfqpoint{2.928977in}{3.163922in}}%
\pgfpathlineto{\pgfqpoint{2.937038in}{3.176761in}}%
\pgfpathlineto{\pgfqpoint{2.945092in}{3.189768in}}%
\pgfpathlineto{\pgfqpoint{2.953137in}{3.202946in}}%
\pgfpathlineto{\pgfqpoint{2.939585in}{3.224064in}}%
\pgfpathlineto{\pgfqpoint{2.926022in}{3.245523in}}%
\pgfpathlineto{\pgfqpoint{2.912449in}{3.267327in}}%
\pgfpathlineto{\pgfqpoint{2.898866in}{3.289479in}}%
\pgfpathlineto{\pgfqpoint{2.890806in}{3.276161in}}%
\pgfpathlineto{\pgfqpoint{2.882738in}{3.263021in}}%
\pgfpathlineto{\pgfqpoint{2.874662in}{3.250057in}}%
\pgfpathlineto{\pgfqpoint{2.866578in}{3.237268in}}%
\pgfpathclose%
\pgfusepath{fill}%
\end{pgfscope}%
\begin{pgfscope}%
\pgfpathrectangle{\pgfqpoint{1.150000in}{0.150000in}}{\pgfqpoint{5.700000in}{5.700000in}}%
\pgfusepath{clip}%
\pgfsetbuttcap%
\pgfsetroundjoin%
\definecolor{currentfill}{rgb}{0.252194,0.269783,0.531579}%
\pgfsetfillcolor{currentfill}%
\pgfsetfillopacity{0.800000}%
\pgfsetlinewidth{0.000000pt}%
\definecolor{currentstroke}{rgb}{0.000000,0.000000,0.000000}%
\pgfsetstrokecolor{currentstroke}%
\pgfsetdash{}{0pt}%
\pgfpathmoveto{\pgfqpoint{4.468510in}{2.783879in}}%
\pgfpathlineto{\pgfqpoint{4.482065in}{2.782969in}}%
\pgfpathlineto{\pgfqpoint{4.495629in}{2.782251in}}%
\pgfpathlineto{\pgfqpoint{4.509203in}{2.781722in}}%
\pgfpathlineto{\pgfqpoint{4.522785in}{2.781384in}}%
\pgfpathlineto{\pgfqpoint{4.530422in}{2.792603in}}%
\pgfpathlineto{\pgfqpoint{4.538055in}{2.803936in}}%
\pgfpathlineto{\pgfqpoint{4.545684in}{2.815390in}}%
\pgfpathlineto{\pgfqpoint{4.553310in}{2.826969in}}%
\pgfpathlineto{\pgfqpoint{4.539739in}{2.827750in}}%
\pgfpathlineto{\pgfqpoint{4.526177in}{2.828721in}}%
\pgfpathlineto{\pgfqpoint{4.512623in}{2.829881in}}%
\pgfpathlineto{\pgfqpoint{4.499079in}{2.831232in}}%
\pgfpathlineto{\pgfqpoint{4.491441in}{2.819200in}}%
\pgfpathlineto{\pgfqpoint{4.483801in}{2.807300in}}%
\pgfpathlineto{\pgfqpoint{4.476157in}{2.795528in}}%
\pgfpathlineto{\pgfqpoint{4.468510in}{2.783879in}}%
\pgfpathclose%
\pgfusepath{fill}%
\end{pgfscope}%
\begin{pgfscope}%
\pgfpathrectangle{\pgfqpoint{1.150000in}{0.150000in}}{\pgfqpoint{5.700000in}{5.700000in}}%
\pgfusepath{clip}%
\pgfsetbuttcap%
\pgfsetroundjoin%
\definecolor{currentfill}{rgb}{0.220057,0.343307,0.549413}%
\pgfsetfillcolor{currentfill}%
\pgfsetfillopacity{0.800000}%
\pgfsetlinewidth{0.000000pt}%
\definecolor{currentstroke}{rgb}{0.000000,0.000000,0.000000}%
\pgfsetstrokecolor{currentstroke}%
\pgfsetdash{}{0pt}%
\pgfpathmoveto{\pgfqpoint{4.807782in}{2.964711in}}%
\pgfpathlineto{\pgfqpoint{4.821444in}{2.964826in}}%
\pgfpathlineto{\pgfqpoint{4.835117in}{2.965123in}}%
\pgfpathlineto{\pgfqpoint{4.848800in}{2.965602in}}%
\pgfpathlineto{\pgfqpoint{4.862494in}{2.966262in}}%
\pgfpathlineto{\pgfqpoint{4.870036in}{2.977758in}}%
\pgfpathlineto{\pgfqpoint{4.877575in}{2.989431in}}%
\pgfpathlineto{\pgfqpoint{4.885113in}{3.001286in}}%
\pgfpathlineto{\pgfqpoint{4.892650in}{3.013332in}}%
\pgfpathlineto{\pgfqpoint{4.878971in}{3.013241in}}%
\pgfpathlineto{\pgfqpoint{4.865303in}{3.013331in}}%
\pgfpathlineto{\pgfqpoint{4.851645in}{3.013602in}}%
\pgfpathlineto{\pgfqpoint{4.837997in}{3.014055in}}%
\pgfpathlineto{\pgfqpoint{4.830446in}{3.001430in}}%
\pgfpathlineto{\pgfqpoint{4.822893in}{2.989002in}}%
\pgfpathlineto{\pgfqpoint{4.815338in}{2.976765in}}%
\pgfpathlineto{\pgfqpoint{4.807782in}{2.964711in}}%
\pgfpathclose%
\pgfusepath{fill}%
\end{pgfscope}%
\begin{pgfscope}%
\pgfpathrectangle{\pgfqpoint{1.150000in}{0.150000in}}{\pgfqpoint{5.700000in}{5.700000in}}%
\pgfusepath{clip}%
\pgfsetbuttcap%
\pgfsetroundjoin%
\definecolor{currentfill}{rgb}{0.277134,0.185228,0.489898}%
\pgfsetfillcolor{currentfill}%
\pgfsetfillopacity{0.800000}%
\pgfsetlinewidth{0.000000pt}%
\definecolor{currentstroke}{rgb}{0.000000,0.000000,0.000000}%
\pgfsetstrokecolor{currentstroke}%
\pgfsetdash{}{0pt}%
\pgfpathmoveto{\pgfqpoint{3.404933in}{2.598531in}}%
\pgfpathlineto{\pgfqpoint{3.418316in}{2.588250in}}%
\pgfpathlineto{\pgfqpoint{3.431700in}{2.578219in}}%
\pgfpathlineto{\pgfqpoint{3.445083in}{2.568434in}}%
\pgfpathlineto{\pgfqpoint{3.458466in}{2.558895in}}%
\pgfpathlineto{\pgfqpoint{3.466411in}{2.570413in}}%
\pgfpathlineto{\pgfqpoint{3.474352in}{2.582018in}}%
\pgfpathlineto{\pgfqpoint{3.482286in}{2.593712in}}%
\pgfpathlineto{\pgfqpoint{3.490215in}{2.605496in}}%
\pgfpathlineto{\pgfqpoint{3.476842in}{2.615163in}}%
\pgfpathlineto{\pgfqpoint{3.463469in}{2.625074in}}%
\pgfpathlineto{\pgfqpoint{3.450097in}{2.635233in}}%
\pgfpathlineto{\pgfqpoint{3.436723in}{2.645641in}}%
\pgfpathlineto{\pgfqpoint{3.428785in}{2.633718in}}%
\pgfpathlineto{\pgfqpoint{3.420840in}{2.621893in}}%
\pgfpathlineto{\pgfqpoint{3.412889in}{2.610165in}}%
\pgfpathlineto{\pgfqpoint{3.404933in}{2.598531in}}%
\pgfpathclose%
\pgfusepath{fill}%
\end{pgfscope}%
\begin{pgfscope}%
\pgfpathrectangle{\pgfqpoint{1.150000in}{0.150000in}}{\pgfqpoint{5.700000in}{5.700000in}}%
\pgfusepath{clip}%
\pgfsetbuttcap%
\pgfsetroundjoin%
\definecolor{currentfill}{rgb}{0.252194,0.269783,0.531579}%
\pgfsetfillcolor{currentfill}%
\pgfsetfillopacity{0.800000}%
\pgfsetlinewidth{0.000000pt}%
\definecolor{currentstroke}{rgb}{0.000000,0.000000,0.000000}%
\pgfsetstrokecolor{currentstroke}%
\pgfsetdash{}{0pt}%
\pgfpathmoveto{\pgfqpoint{3.104708in}{2.811711in}}%
\pgfpathlineto{\pgfqpoint{3.118160in}{2.796149in}}%
\pgfpathlineto{\pgfqpoint{3.131607in}{2.780876in}}%
\pgfpathlineto{\pgfqpoint{3.145049in}{2.765891in}}%
\pgfpathlineto{\pgfqpoint{3.158487in}{2.751189in}}%
\pgfpathlineto{\pgfqpoint{3.166516in}{2.762796in}}%
\pgfpathlineto{\pgfqpoint{3.174538in}{2.774522in}}%
\pgfpathlineto{\pgfqpoint{3.182553in}{2.786370in}}%
\pgfpathlineto{\pgfqpoint{3.190562in}{2.798340in}}%
\pgfpathlineto{\pgfqpoint{3.177138in}{2.813137in}}%
\pgfpathlineto{\pgfqpoint{3.163709in}{2.828218in}}%
\pgfpathlineto{\pgfqpoint{3.150275in}{2.843586in}}%
\pgfpathlineto{\pgfqpoint{3.136836in}{2.859243in}}%
\pgfpathlineto{\pgfqpoint{3.128815in}{2.847165in}}%
\pgfpathlineto{\pgfqpoint{3.120786in}{2.835218in}}%
\pgfpathlineto{\pgfqpoint{3.112751in}{2.823401in}}%
\pgfpathlineto{\pgfqpoint{3.104708in}{2.811711in}}%
\pgfpathclose%
\pgfusepath{fill}%
\end{pgfscope}%
\begin{pgfscope}%
\pgfpathrectangle{\pgfqpoint{1.150000in}{0.150000in}}{\pgfqpoint{5.700000in}{5.700000in}}%
\pgfusepath{clip}%
\pgfsetbuttcap%
\pgfsetroundjoin%
\definecolor{currentfill}{rgb}{0.258965,0.251537,0.524736}%
\pgfsetfillcolor{currentfill}%
\pgfsetfillopacity{0.800000}%
\pgfsetlinewidth{0.000000pt}%
\definecolor{currentstroke}{rgb}{0.000000,0.000000,0.000000}%
\pgfsetstrokecolor{currentstroke}%
\pgfsetdash{}{0pt}%
\pgfpathmoveto{\pgfqpoint{4.383711in}{2.742319in}}%
\pgfpathlineto{\pgfqpoint{4.397242in}{2.741052in}}%
\pgfpathlineto{\pgfqpoint{4.410781in}{2.739978in}}%
\pgfpathlineto{\pgfqpoint{4.424329in}{2.739096in}}%
\pgfpathlineto{\pgfqpoint{4.437886in}{2.738406in}}%
\pgfpathlineto{\pgfqpoint{4.445547in}{2.749615in}}%
\pgfpathlineto{\pgfqpoint{4.453205in}{2.760927in}}%
\pgfpathlineto{\pgfqpoint{4.460860in}{2.772347in}}%
\pgfpathlineto{\pgfqpoint{4.468510in}{2.783879in}}%
\pgfpathlineto{\pgfqpoint{4.454964in}{2.784979in}}%
\pgfpathlineto{\pgfqpoint{4.441426in}{2.786271in}}%
\pgfpathlineto{\pgfqpoint{4.427897in}{2.787755in}}%
\pgfpathlineto{\pgfqpoint{4.414376in}{2.789433in}}%
\pgfpathlineto{\pgfqpoint{4.406715in}{2.777479in}}%
\pgfpathlineto{\pgfqpoint{4.399051in}{2.765645in}}%
\pgfpathlineto{\pgfqpoint{4.391383in}{2.753927in}}%
\pgfpathlineto{\pgfqpoint{4.383711in}{2.742319in}}%
\pgfpathclose%
\pgfusepath{fill}%
\end{pgfscope}%
\begin{pgfscope}%
\pgfpathrectangle{\pgfqpoint{1.150000in}{0.150000in}}{\pgfqpoint{5.700000in}{5.700000in}}%
\pgfusepath{clip}%
\pgfsetbuttcap%
\pgfsetroundjoin%
\definecolor{currentfill}{rgb}{0.241237,0.296485,0.539709}%
\pgfsetfillcolor{currentfill}%
\pgfsetfillopacity{0.800000}%
\pgfsetlinewidth{0.000000pt}%
\definecolor{currentstroke}{rgb}{0.000000,0.000000,0.000000}%
\pgfsetstrokecolor{currentstroke}%
\pgfsetdash{}{0pt}%
\pgfpathmoveto{\pgfqpoint{3.050843in}{2.876901in}}%
\pgfpathlineto{\pgfqpoint{3.064319in}{2.860157in}}%
\pgfpathlineto{\pgfqpoint{3.077788in}{2.843712in}}%
\pgfpathlineto{\pgfqpoint{3.091251in}{2.827564in}}%
\pgfpathlineto{\pgfqpoint{3.104708in}{2.811711in}}%
\pgfpathlineto{\pgfqpoint{3.112751in}{2.823401in}}%
\pgfpathlineto{\pgfqpoint{3.120786in}{2.835218in}}%
\pgfpathlineto{\pgfqpoint{3.128815in}{2.847165in}}%
\pgfpathlineto{\pgfqpoint{3.136836in}{2.859243in}}%
\pgfpathlineto{\pgfqpoint{3.123392in}{2.875192in}}%
\pgfpathlineto{\pgfqpoint{3.109942in}{2.891435in}}%
\pgfpathlineto{\pgfqpoint{3.096487in}{2.907975in}}%
\pgfpathlineto{\pgfqpoint{3.083025in}{2.924815in}}%
\pgfpathlineto{\pgfqpoint{3.074991in}{2.912630in}}%
\pgfpathlineto{\pgfqpoint{3.066949in}{2.900583in}}%
\pgfpathlineto{\pgfqpoint{3.058900in}{2.888674in}}%
\pgfpathlineto{\pgfqpoint{3.050843in}{2.876901in}}%
\pgfpathclose%
\pgfusepath{fill}%
\end{pgfscope}%
\begin{pgfscope}%
\pgfpathrectangle{\pgfqpoint{1.150000in}{0.150000in}}{\pgfqpoint{5.700000in}{5.700000in}}%
\pgfusepath{clip}%
\pgfsetbuttcap%
\pgfsetroundjoin%
\definecolor{currentfill}{rgb}{0.210503,0.363727,0.552206}%
\pgfsetfillcolor{currentfill}%
\pgfsetfillopacity{0.800000}%
\pgfsetlinewidth{0.000000pt}%
\definecolor{currentstroke}{rgb}{0.000000,0.000000,0.000000}%
\pgfsetstrokecolor{currentstroke}%
\pgfsetdash{}{0pt}%
\pgfpathmoveto{\pgfqpoint{4.892650in}{3.013332in}}%
\pgfpathlineto{\pgfqpoint{4.906340in}{3.013603in}}%
\pgfpathlineto{\pgfqpoint{4.920041in}{3.014055in}}%
\pgfpathlineto{\pgfqpoint{4.933753in}{3.014686in}}%
\pgfpathlineto{\pgfqpoint{4.947477in}{3.015497in}}%
\pgfpathlineto{\pgfqpoint{4.954997in}{3.027150in}}%
\pgfpathlineto{\pgfqpoint{4.962516in}{3.038998in}}%
\pgfpathlineto{\pgfqpoint{4.970034in}{3.051048in}}%
\pgfpathlineto{\pgfqpoint{4.977552in}{3.063307in}}%
\pgfpathlineto{\pgfqpoint{4.963845in}{3.063096in}}%
\pgfpathlineto{\pgfqpoint{4.950149in}{3.063065in}}%
\pgfpathlineto{\pgfqpoint{4.936464in}{3.063214in}}%
\pgfpathlineto{\pgfqpoint{4.922789in}{3.063542in}}%
\pgfpathlineto{\pgfqpoint{4.915256in}{3.050671in}}%
\pgfpathlineto{\pgfqpoint{4.907721in}{3.038017in}}%
\pgfpathlineto{\pgfqpoint{4.900186in}{3.025573in}}%
\pgfpathlineto{\pgfqpoint{4.892650in}{3.013332in}}%
\pgfpathclose%
\pgfusepath{fill}%
\end{pgfscope}%
\begin{pgfscope}%
\pgfpathrectangle{\pgfqpoint{1.150000in}{0.150000in}}{\pgfqpoint{5.700000in}{5.700000in}}%
\pgfusepath{clip}%
\pgfsetbuttcap%
\pgfsetroundjoin%
\definecolor{currentfill}{rgb}{0.260571,0.246922,0.522828}%
\pgfsetfillcolor{currentfill}%
\pgfsetfillopacity{0.800000}%
\pgfsetlinewidth{0.000000pt}%
\definecolor{currentstroke}{rgb}{0.000000,0.000000,0.000000}%
\pgfsetstrokecolor{currentstroke}%
\pgfsetdash{}{0pt}%
\pgfpathmoveto{\pgfqpoint{3.158487in}{2.751189in}}%
\pgfpathlineto{\pgfqpoint{3.171920in}{2.736771in}}%
\pgfpathlineto{\pgfqpoint{3.185348in}{2.722632in}}%
\pgfpathlineto{\pgfqpoint{3.198773in}{2.708771in}}%
\pgfpathlineto{\pgfqpoint{3.212194in}{2.695185in}}%
\pgfpathlineto{\pgfqpoint{3.220210in}{2.706708in}}%
\pgfpathlineto{\pgfqpoint{3.228220in}{2.718343in}}%
\pgfpathlineto{\pgfqpoint{3.236223in}{2.730091in}}%
\pgfpathlineto{\pgfqpoint{3.244220in}{2.741954in}}%
\pgfpathlineto{\pgfqpoint{3.230811in}{2.755635in}}%
\pgfpathlineto{\pgfqpoint{3.217399in}{2.769592in}}%
\pgfpathlineto{\pgfqpoint{3.203982in}{2.783826in}}%
\pgfpathlineto{\pgfqpoint{3.190562in}{2.798340in}}%
\pgfpathlineto{\pgfqpoint{3.182553in}{2.786370in}}%
\pgfpathlineto{\pgfqpoint{3.174538in}{2.774522in}}%
\pgfpathlineto{\pgfqpoint{3.166516in}{2.762796in}}%
\pgfpathlineto{\pgfqpoint{3.158487in}{2.751189in}}%
\pgfpathclose%
\pgfusepath{fill}%
\end{pgfscope}%
\begin{pgfscope}%
\pgfpathrectangle{\pgfqpoint{1.150000in}{0.150000in}}{\pgfqpoint{5.700000in}{5.700000in}}%
\pgfusepath{clip}%
\pgfsetbuttcap%
\pgfsetroundjoin%
\definecolor{currentfill}{rgb}{0.263663,0.237631,0.518762}%
\pgfsetfillcolor{currentfill}%
\pgfsetfillopacity{0.800000}%
\pgfsetlinewidth{0.000000pt}%
\definecolor{currentstroke}{rgb}{0.000000,0.000000,0.000000}%
\pgfsetstrokecolor{currentstroke}%
\pgfsetdash{}{0pt}%
\pgfpathmoveto{\pgfqpoint{4.298907in}{2.702398in}}%
\pgfpathlineto{\pgfqpoint{4.312414in}{2.700731in}}%
\pgfpathlineto{\pgfqpoint{4.325930in}{2.699259in}}%
\pgfpathlineto{\pgfqpoint{4.339454in}{2.697982in}}%
\pgfpathlineto{\pgfqpoint{4.352986in}{2.696901in}}%
\pgfpathlineto{\pgfqpoint{4.360673in}{2.708112in}}%
\pgfpathlineto{\pgfqpoint{4.368356in}{2.719416in}}%
\pgfpathlineto{\pgfqpoint{4.376036in}{2.730817in}}%
\pgfpathlineto{\pgfqpoint{4.383711in}{2.742319in}}%
\pgfpathlineto{\pgfqpoint{4.370189in}{2.743780in}}%
\pgfpathlineto{\pgfqpoint{4.356675in}{2.745436in}}%
\pgfpathlineto{\pgfqpoint{4.343169in}{2.747286in}}%
\pgfpathlineto{\pgfqpoint{4.329670in}{2.749332in}}%
\pgfpathlineto{\pgfqpoint{4.321985in}{2.737440in}}%
\pgfpathlineto{\pgfqpoint{4.314296in}{2.725656in}}%
\pgfpathlineto{\pgfqpoint{4.306604in}{2.713977in}}%
\pgfpathlineto{\pgfqpoint{4.298907in}{2.702398in}}%
\pgfpathclose%
\pgfusepath{fill}%
\end{pgfscope}%
\begin{pgfscope}%
\pgfpathrectangle{\pgfqpoint{1.150000in}{0.150000in}}{\pgfqpoint{5.700000in}{5.700000in}}%
\pgfusepath{clip}%
\pgfsetbuttcap%
\pgfsetroundjoin%
\definecolor{currentfill}{rgb}{0.229739,0.322361,0.545706}%
\pgfsetfillcolor{currentfill}%
\pgfsetfillopacity{0.800000}%
\pgfsetlinewidth{0.000000pt}%
\definecolor{currentstroke}{rgb}{0.000000,0.000000,0.000000}%
\pgfsetstrokecolor{currentstroke}%
\pgfsetdash{}{0pt}%
\pgfpathmoveto{\pgfqpoint{2.996874in}{2.946924in}}%
\pgfpathlineto{\pgfqpoint{3.010377in}{2.928956in}}%
\pgfpathlineto{\pgfqpoint{3.023873in}{2.911297in}}%
\pgfpathlineto{\pgfqpoint{3.037361in}{2.893947in}}%
\pgfpathlineto{\pgfqpoint{3.050843in}{2.876901in}}%
\pgfpathlineto{\pgfqpoint{3.058900in}{2.888674in}}%
\pgfpathlineto{\pgfqpoint{3.066949in}{2.900583in}}%
\pgfpathlineto{\pgfqpoint{3.074991in}{2.912630in}}%
\pgfpathlineto{\pgfqpoint{3.083025in}{2.924815in}}%
\pgfpathlineto{\pgfqpoint{3.069557in}{2.941956in}}%
\pgfpathlineto{\pgfqpoint{3.056083in}{2.959403in}}%
\pgfpathlineto{\pgfqpoint{3.042601in}{2.977156in}}%
\pgfpathlineto{\pgfqpoint{3.029112in}{2.995220in}}%
\pgfpathlineto{\pgfqpoint{3.021064in}{2.982927in}}%
\pgfpathlineto{\pgfqpoint{3.013008in}{2.970782in}}%
\pgfpathlineto{\pgfqpoint{3.004945in}{2.958781in}}%
\pgfpathlineto{\pgfqpoint{2.996874in}{2.946924in}}%
\pgfpathclose%
\pgfusepath{fill}%
\end{pgfscope}%
\begin{pgfscope}%
\pgfpathrectangle{\pgfqpoint{1.150000in}{0.150000in}}{\pgfqpoint{5.700000in}{5.700000in}}%
\pgfusepath{clip}%
\pgfsetbuttcap%
\pgfsetroundjoin%
\definecolor{currentfill}{rgb}{0.203063,0.379716,0.553925}%
\pgfsetfillcolor{currentfill}%
\pgfsetfillopacity{0.800000}%
\pgfsetlinewidth{0.000000pt}%
\definecolor{currentstroke}{rgb}{0.000000,0.000000,0.000000}%
\pgfsetstrokecolor{currentstroke}%
\pgfsetdash{}{0pt}%
\pgfpathmoveto{\pgfqpoint{4.977552in}{3.063307in}}%
\pgfpathlineto{\pgfqpoint{4.991270in}{3.063696in}}%
\pgfpathlineto{\pgfqpoint{5.004999in}{3.064263in}}%
\pgfpathlineto{\pgfqpoint{5.018740in}{3.065009in}}%
\pgfpathlineto{\pgfqpoint{5.032492in}{3.065932in}}%
\pgfpathlineto{\pgfqpoint{5.039992in}{3.077787in}}%
\pgfpathlineto{\pgfqpoint{5.047492in}{3.089857in}}%
\pgfpathlineto{\pgfqpoint{5.054992in}{3.102151in}}%
\pgfpathlineto{\pgfqpoint{5.062493in}{3.114674in}}%
\pgfpathlineto{\pgfqpoint{5.048758in}{3.114383in}}%
\pgfpathlineto{\pgfqpoint{5.035035in}{3.114269in}}%
\pgfpathlineto{\pgfqpoint{5.021322in}{3.114333in}}%
\pgfpathlineto{\pgfqpoint{5.007621in}{3.114575in}}%
\pgfpathlineto{\pgfqpoint{5.000104in}{3.101409in}}%
\pgfpathlineto{\pgfqpoint{4.992586in}{3.088480in}}%
\pgfpathlineto{\pgfqpoint{4.985069in}{3.075782in}}%
\pgfpathlineto{\pgfqpoint{4.977552in}{3.063307in}}%
\pgfpathclose%
\pgfusepath{fill}%
\end{pgfscope}%
\begin{pgfscope}%
\pgfpathrectangle{\pgfqpoint{1.150000in}{0.150000in}}{\pgfqpoint{5.700000in}{5.700000in}}%
\pgfusepath{clip}%
\pgfsetbuttcap%
\pgfsetroundjoin%
\definecolor{currentfill}{rgb}{0.269308,0.218818,0.509577}%
\pgfsetfillcolor{currentfill}%
\pgfsetfillopacity{0.800000}%
\pgfsetlinewidth{0.000000pt}%
\definecolor{currentstroke}{rgb}{0.000000,0.000000,0.000000}%
\pgfsetstrokecolor{currentstroke}%
\pgfsetdash{}{0pt}%
\pgfpathmoveto{\pgfqpoint{4.214089in}{2.664247in}}%
\pgfpathlineto{\pgfqpoint{4.227575in}{2.662137in}}%
\pgfpathlineto{\pgfqpoint{4.241069in}{2.660225in}}%
\pgfpathlineto{\pgfqpoint{4.254570in}{2.658512in}}%
\pgfpathlineto{\pgfqpoint{4.268079in}{2.656996in}}%
\pgfpathlineto{\pgfqpoint{4.275792in}{2.668217in}}%
\pgfpathlineto{\pgfqpoint{4.283501in}{2.679522in}}%
\pgfpathlineto{\pgfqpoint{4.291206in}{2.690914in}}%
\pgfpathlineto{\pgfqpoint{4.298907in}{2.702398in}}%
\pgfpathlineto{\pgfqpoint{4.285407in}{2.704262in}}%
\pgfpathlineto{\pgfqpoint{4.271915in}{2.706323in}}%
\pgfpathlineto{\pgfqpoint{4.258430in}{2.708582in}}%
\pgfpathlineto{\pgfqpoint{4.244953in}{2.711039in}}%
\pgfpathlineto{\pgfqpoint{4.237243in}{2.699197in}}%
\pgfpathlineto{\pgfqpoint{4.229529in}{2.687454in}}%
\pgfpathlineto{\pgfqpoint{4.221811in}{2.675805in}}%
\pgfpathlineto{\pgfqpoint{4.214089in}{2.664247in}}%
\pgfpathclose%
\pgfusepath{fill}%
\end{pgfscope}%
\begin{pgfscope}%
\pgfpathrectangle{\pgfqpoint{1.150000in}{0.150000in}}{\pgfqpoint{5.700000in}{5.700000in}}%
\pgfusepath{clip}%
\pgfsetbuttcap%
\pgfsetroundjoin%
\definecolor{currentfill}{rgb}{0.267968,0.223549,0.512008}%
\pgfsetfillcolor{currentfill}%
\pgfsetfillopacity{0.800000}%
\pgfsetlinewidth{0.000000pt}%
\definecolor{currentstroke}{rgb}{0.000000,0.000000,0.000000}%
\pgfsetstrokecolor{currentstroke}%
\pgfsetdash{}{0pt}%
\pgfpathmoveto{\pgfqpoint{3.212194in}{2.695185in}}%
\pgfpathlineto{\pgfqpoint{3.225612in}{2.681873in}}%
\pgfpathlineto{\pgfqpoint{3.239027in}{2.668832in}}%
\pgfpathlineto{\pgfqpoint{3.252438in}{2.656061in}}%
\pgfpathlineto{\pgfqpoint{3.265847in}{2.643557in}}%
\pgfpathlineto{\pgfqpoint{3.273851in}{2.654996in}}%
\pgfpathlineto{\pgfqpoint{3.281848in}{2.666539in}}%
\pgfpathlineto{\pgfqpoint{3.289839in}{2.678188in}}%
\pgfpathlineto{\pgfqpoint{3.297824in}{2.689943in}}%
\pgfpathlineto{\pgfqpoint{3.284427in}{2.702543in}}%
\pgfpathlineto{\pgfqpoint{3.271028in}{2.715410in}}%
\pgfpathlineto{\pgfqpoint{3.257625in}{2.728546in}}%
\pgfpathlineto{\pgfqpoint{3.244220in}{2.741954in}}%
\pgfpathlineto{\pgfqpoint{3.236223in}{2.730091in}}%
\pgfpathlineto{\pgfqpoint{3.228220in}{2.718343in}}%
\pgfpathlineto{\pgfqpoint{3.220210in}{2.706708in}}%
\pgfpathlineto{\pgfqpoint{3.212194in}{2.695185in}}%
\pgfpathclose%
\pgfusepath{fill}%
\end{pgfscope}%
\begin{pgfscope}%
\pgfpathrectangle{\pgfqpoint{1.150000in}{0.150000in}}{\pgfqpoint{5.700000in}{5.700000in}}%
\pgfusepath{clip}%
\pgfsetbuttcap%
\pgfsetroundjoin%
\definecolor{currentfill}{rgb}{0.279574,0.170599,0.479997}%
\pgfsetfillcolor{currentfill}%
\pgfsetfillopacity{0.800000}%
\pgfsetlinewidth{0.000000pt}%
\definecolor{currentstroke}{rgb}{0.000000,0.000000,0.000000}%
\pgfsetstrokecolor{currentstroke}%
\pgfsetdash{}{0pt}%
\pgfpathmoveto{\pgfqpoint{3.820867in}{2.552294in}}%
\pgfpathlineto{\pgfqpoint{3.834269in}{2.547108in}}%
\pgfpathlineto{\pgfqpoint{3.847676in}{2.542139in}}%
\pgfpathlineto{\pgfqpoint{3.861087in}{2.537385in}}%
\pgfpathlineto{\pgfqpoint{3.874503in}{2.532846in}}%
\pgfpathlineto{\pgfqpoint{3.882331in}{2.544287in}}%
\pgfpathlineto{\pgfqpoint{3.890155in}{2.555795in}}%
\pgfpathlineto{\pgfqpoint{3.897974in}{2.567373in}}%
\pgfpathlineto{\pgfqpoint{3.905788in}{2.579023in}}%
\pgfpathlineto{\pgfqpoint{3.892381in}{2.583784in}}%
\pgfpathlineto{\pgfqpoint{3.878978in}{2.588759in}}%
\pgfpathlineto{\pgfqpoint{3.865580in}{2.593950in}}%
\pgfpathlineto{\pgfqpoint{3.852185in}{2.599357in}}%
\pgfpathlineto{\pgfqpoint{3.844363in}{2.587473in}}%
\pgfpathlineto{\pgfqpoint{3.836536in}{2.575670in}}%
\pgfpathlineto{\pgfqpoint{3.828704in}{2.563944in}}%
\pgfpathlineto{\pgfqpoint{3.820867in}{2.552294in}}%
\pgfpathclose%
\pgfusepath{fill}%
\end{pgfscope}%
\begin{pgfscope}%
\pgfpathrectangle{\pgfqpoint{1.150000in}{0.150000in}}{\pgfqpoint{5.700000in}{5.700000in}}%
\pgfusepath{clip}%
\pgfsetbuttcap%
\pgfsetroundjoin%
\definecolor{currentfill}{rgb}{0.194100,0.399323,0.555565}%
\pgfsetfillcolor{currentfill}%
\pgfsetfillopacity{0.800000}%
\pgfsetlinewidth{0.000000pt}%
\definecolor{currentstroke}{rgb}{0.000000,0.000000,0.000000}%
\pgfsetstrokecolor{currentstroke}%
\pgfsetdash{}{0pt}%
\pgfpathmoveto{\pgfqpoint{5.062493in}{3.114674in}}%
\pgfpathlineto{\pgfqpoint{5.076239in}{3.115142in}}%
\pgfpathlineto{\pgfqpoint{5.089996in}{3.115787in}}%
\pgfpathlineto{\pgfqpoint{5.103766in}{3.116608in}}%
\pgfpathlineto{\pgfqpoint{5.117547in}{3.117606in}}%
\pgfpathlineto{\pgfqpoint{5.125029in}{3.129714in}}%
\pgfpathlineto{\pgfqpoint{5.132512in}{3.142060in}}%
\pgfpathlineto{\pgfqpoint{5.139996in}{3.154651in}}%
\pgfpathlineto{\pgfqpoint{5.147481in}{3.167495in}}%
\pgfpathlineto{\pgfqpoint{5.133719in}{3.167161in}}%
\pgfpathlineto{\pgfqpoint{5.119968in}{3.167003in}}%
\pgfpathlineto{\pgfqpoint{5.106229in}{3.167022in}}%
\pgfpathlineto{\pgfqpoint{5.092501in}{3.167216in}}%
\pgfpathlineto{\pgfqpoint{5.084997in}{3.153698in}}%
\pgfpathlineto{\pgfqpoint{5.077495in}{3.140440in}}%
\pgfpathlineto{\pgfqpoint{5.069993in}{3.127435in}}%
\pgfpathlineto{\pgfqpoint{5.062493in}{3.114674in}}%
\pgfpathclose%
\pgfusepath{fill}%
\end{pgfscope}%
\begin{pgfscope}%
\pgfpathrectangle{\pgfqpoint{1.150000in}{0.150000in}}{\pgfqpoint{5.700000in}{5.700000in}}%
\pgfusepath{clip}%
\pgfsetbuttcap%
\pgfsetroundjoin%
\definecolor{currentfill}{rgb}{0.216210,0.351535,0.550627}%
\pgfsetfillcolor{currentfill}%
\pgfsetfillopacity{0.800000}%
\pgfsetlinewidth{0.000000pt}%
\definecolor{currentstroke}{rgb}{0.000000,0.000000,0.000000}%
\pgfsetstrokecolor{currentstroke}%
\pgfsetdash{}{0pt}%
\pgfpathmoveto{\pgfqpoint{2.942782in}{3.021956in}}%
\pgfpathlineto{\pgfqpoint{2.956318in}{3.002719in}}%
\pgfpathlineto{\pgfqpoint{2.969845in}{2.983802in}}%
\pgfpathlineto{\pgfqpoint{2.983363in}{2.965205in}}%
\pgfpathlineto{\pgfqpoint{2.996874in}{2.946924in}}%
\pgfpathlineto{\pgfqpoint{3.004945in}{2.958781in}}%
\pgfpathlineto{\pgfqpoint{3.013008in}{2.970782in}}%
\pgfpathlineto{\pgfqpoint{3.021064in}{2.982927in}}%
\pgfpathlineto{\pgfqpoint{3.029112in}{2.995220in}}%
\pgfpathlineto{\pgfqpoint{3.015616in}{3.013597in}}%
\pgfpathlineto{\pgfqpoint{3.002112in}{3.032290in}}%
\pgfpathlineto{\pgfqpoint{2.988600in}{3.051301in}}%
\pgfpathlineto{\pgfqpoint{2.975079in}{3.070635in}}%
\pgfpathlineto{\pgfqpoint{2.967017in}{3.058234in}}%
\pgfpathlineto{\pgfqpoint{2.958947in}{3.045989in}}%
\pgfpathlineto{\pgfqpoint{2.950868in}{3.033897in}}%
\pgfpathlineto{\pgfqpoint{2.942782in}{3.021956in}}%
\pgfpathclose%
\pgfusepath{fill}%
\end{pgfscope}%
\begin{pgfscope}%
\pgfpathrectangle{\pgfqpoint{1.150000in}{0.150000in}}{\pgfqpoint{5.700000in}{5.700000in}}%
\pgfusepath{clip}%
\pgfsetbuttcap%
\pgfsetroundjoin%
\definecolor{currentfill}{rgb}{0.273006,0.204520,0.501721}%
\pgfsetfillcolor{currentfill}%
\pgfsetfillopacity{0.800000}%
\pgfsetlinewidth{0.000000pt}%
\definecolor{currentstroke}{rgb}{0.000000,0.000000,0.000000}%
\pgfsetstrokecolor{currentstroke}%
\pgfsetdash{}{0pt}%
\pgfpathmoveto{\pgfqpoint{4.129250in}{2.628023in}}%
\pgfpathlineto{\pgfqpoint{4.142716in}{2.625427in}}%
\pgfpathlineto{\pgfqpoint{4.156190in}{2.623032in}}%
\pgfpathlineto{\pgfqpoint{4.169670in}{2.620838in}}%
\pgfpathlineto{\pgfqpoint{4.183157in}{2.618844in}}%
\pgfpathlineto{\pgfqpoint{4.190897in}{2.630079in}}%
\pgfpathlineto{\pgfqpoint{4.198632in}{2.641388in}}%
\pgfpathlineto{\pgfqpoint{4.206363in}{2.652776in}}%
\pgfpathlineto{\pgfqpoint{4.214089in}{2.664247in}}%
\pgfpathlineto{\pgfqpoint{4.200610in}{2.666557in}}%
\pgfpathlineto{\pgfqpoint{4.187139in}{2.669067in}}%
\pgfpathlineto{\pgfqpoint{4.173674in}{2.671778in}}%
\pgfpathlineto{\pgfqpoint{4.160216in}{2.674690in}}%
\pgfpathlineto{\pgfqpoint{4.152481in}{2.662892in}}%
\pgfpathlineto{\pgfqpoint{4.144742in}{2.651184in}}%
\pgfpathlineto{\pgfqpoint{4.136998in}{2.639562in}}%
\pgfpathlineto{\pgfqpoint{4.129250in}{2.628023in}}%
\pgfpathclose%
\pgfusepath{fill}%
\end{pgfscope}%
\begin{pgfscope}%
\pgfpathrectangle{\pgfqpoint{1.150000in}{0.150000in}}{\pgfqpoint{5.700000in}{5.700000in}}%
\pgfusepath{clip}%
\pgfsetbuttcap%
\pgfsetroundjoin%
\definecolor{currentfill}{rgb}{0.280868,0.160771,0.472899}%
\pgfsetfillcolor{currentfill}%
\pgfsetfillopacity{0.800000}%
\pgfsetlinewidth{0.000000pt}%
\definecolor{currentstroke}{rgb}{0.000000,0.000000,0.000000}%
\pgfsetstrokecolor{currentstroke}%
\pgfsetdash{}{0pt}%
\pgfpathmoveto{\pgfqpoint{3.597224in}{2.536819in}}%
\pgfpathlineto{\pgfqpoint{3.610607in}{2.529294in}}%
\pgfpathlineto{\pgfqpoint{3.623992in}{2.522000in}}%
\pgfpathlineto{\pgfqpoint{3.637379in}{2.514936in}}%
\pgfpathlineto{\pgfqpoint{3.650768in}{2.508100in}}%
\pgfpathlineto{\pgfqpoint{3.658663in}{2.519536in}}%
\pgfpathlineto{\pgfqpoint{3.666552in}{2.531043in}}%
\pgfpathlineto{\pgfqpoint{3.674437in}{2.542623in}}%
\pgfpathlineto{\pgfqpoint{3.682316in}{2.554279in}}%
\pgfpathlineto{\pgfqpoint{3.668936in}{2.561274in}}%
\pgfpathlineto{\pgfqpoint{3.655558in}{2.568498in}}%
\pgfpathlineto{\pgfqpoint{3.642182in}{2.575950in}}%
\pgfpathlineto{\pgfqpoint{3.628809in}{2.583634in}}%
\pgfpathlineto{\pgfqpoint{3.620921in}{2.571808in}}%
\pgfpathlineto{\pgfqpoint{3.613027in}{2.560064in}}%
\pgfpathlineto{\pgfqpoint{3.605129in}{2.548402in}}%
\pgfpathlineto{\pgfqpoint{3.597224in}{2.536819in}}%
\pgfpathclose%
\pgfusepath{fill}%
\end{pgfscope}%
\begin{pgfscope}%
\pgfpathrectangle{\pgfqpoint{1.150000in}{0.150000in}}{\pgfqpoint{5.700000in}{5.700000in}}%
\pgfusepath{clip}%
\pgfsetbuttcap%
\pgfsetroundjoin%
\definecolor{currentfill}{rgb}{0.183898,0.422383,0.556944}%
\pgfsetfillcolor{currentfill}%
\pgfsetfillopacity{0.800000}%
\pgfsetlinewidth{0.000000pt}%
\definecolor{currentstroke}{rgb}{0.000000,0.000000,0.000000}%
\pgfsetstrokecolor{currentstroke}%
\pgfsetdash{}{0pt}%
\pgfpathmoveto{\pgfqpoint{5.147481in}{3.167495in}}%
\pgfpathlineto{\pgfqpoint{5.161255in}{3.168004in}}%
\pgfpathlineto{\pgfqpoint{5.175041in}{3.168688in}}%
\pgfpathlineto{\pgfqpoint{5.188839in}{3.169547in}}%
\pgfpathlineto{\pgfqpoint{5.202649in}{3.170581in}}%
\pgfpathlineto{\pgfqpoint{5.210115in}{3.183001in}}%
\pgfpathlineto{\pgfqpoint{5.217584in}{3.195681in}}%
\pgfpathlineto{\pgfqpoint{5.225054in}{3.208630in}}%
\pgfpathlineto{\pgfqpoint{5.232526in}{3.221855in}}%
\pgfpathlineto{\pgfqpoint{5.218737in}{3.221517in}}%
\pgfpathlineto{\pgfqpoint{5.204959in}{3.221353in}}%
\pgfpathlineto{\pgfqpoint{5.191193in}{3.221364in}}%
\pgfpathlineto{\pgfqpoint{5.177439in}{3.221549in}}%
\pgfpathlineto{\pgfqpoint{5.169946in}{3.207618in}}%
\pgfpathlineto{\pgfqpoint{5.162456in}{3.193970in}}%
\pgfpathlineto{\pgfqpoint{5.154968in}{3.180598in}}%
\pgfpathlineto{\pgfqpoint{5.147481in}{3.167495in}}%
\pgfpathclose%
\pgfusepath{fill}%
\end{pgfscope}%
\begin{pgfscope}%
\pgfpathrectangle{\pgfqpoint{1.150000in}{0.150000in}}{\pgfqpoint{5.700000in}{5.700000in}}%
\pgfusepath{clip}%
\pgfsetbuttcap%
\pgfsetroundjoin%
\definecolor{currentfill}{rgb}{0.273006,0.204520,0.501721}%
\pgfsetfillcolor{currentfill}%
\pgfsetfillopacity{0.800000}%
\pgfsetlinewidth{0.000000pt}%
\definecolor{currentstroke}{rgb}{0.000000,0.000000,0.000000}%
\pgfsetstrokecolor{currentstroke}%
\pgfsetdash{}{0pt}%
\pgfpathmoveto{\pgfqpoint{3.265847in}{2.643557in}}%
\pgfpathlineto{\pgfqpoint{3.279254in}{2.631318in}}%
\pgfpathlineto{\pgfqpoint{3.292658in}{2.619342in}}%
\pgfpathlineto{\pgfqpoint{3.306060in}{2.607627in}}%
\pgfpathlineto{\pgfqpoint{3.319460in}{2.596172in}}%
\pgfpathlineto{\pgfqpoint{3.327452in}{2.607528in}}%
\pgfpathlineto{\pgfqpoint{3.335438in}{2.618979in}}%
\pgfpathlineto{\pgfqpoint{3.343417in}{2.630528in}}%
\pgfpathlineto{\pgfqpoint{3.351390in}{2.642176in}}%
\pgfpathlineto{\pgfqpoint{3.338002in}{2.653727in}}%
\pgfpathlineto{\pgfqpoint{3.324611in}{2.665537in}}%
\pgfpathlineto{\pgfqpoint{3.311219in}{2.677608in}}%
\pgfpathlineto{\pgfqpoint{3.297824in}{2.689943in}}%
\pgfpathlineto{\pgfqpoint{3.289839in}{2.678188in}}%
\pgfpathlineto{\pgfqpoint{3.281848in}{2.666539in}}%
\pgfpathlineto{\pgfqpoint{3.273851in}{2.654996in}}%
\pgfpathlineto{\pgfqpoint{3.265847in}{2.643557in}}%
\pgfpathclose%
\pgfusepath{fill}%
\end{pgfscope}%
\begin{pgfscope}%
\pgfpathrectangle{\pgfqpoint{1.150000in}{0.150000in}}{\pgfqpoint{5.700000in}{5.700000in}}%
\pgfusepath{clip}%
\pgfsetbuttcap%
\pgfsetroundjoin%
\definecolor{currentfill}{rgb}{0.279574,0.170599,0.479997}%
\pgfsetfillcolor{currentfill}%
\pgfsetfillopacity{0.800000}%
\pgfsetlinewidth{0.000000pt}%
\definecolor{currentstroke}{rgb}{0.000000,0.000000,0.000000}%
\pgfsetstrokecolor{currentstroke}%
\pgfsetdash{}{0pt}%
\pgfpathmoveto{\pgfqpoint{3.458466in}{2.558895in}}%
\pgfpathlineto{\pgfqpoint{3.471849in}{2.549600in}}%
\pgfpathlineto{\pgfqpoint{3.485232in}{2.540547in}}%
\pgfpathlineto{\pgfqpoint{3.498617in}{2.531735in}}%
\pgfpathlineto{\pgfqpoint{3.512002in}{2.523163in}}%
\pgfpathlineto{\pgfqpoint{3.519937in}{2.534565in}}%
\pgfpathlineto{\pgfqpoint{3.527867in}{2.546047in}}%
\pgfpathlineto{\pgfqpoint{3.535792in}{2.557609in}}%
\pgfpathlineto{\pgfqpoint{3.543711in}{2.569255in}}%
\pgfpathlineto{\pgfqpoint{3.530336in}{2.577955in}}%
\pgfpathlineto{\pgfqpoint{3.516962in}{2.586894in}}%
\pgfpathlineto{\pgfqpoint{3.503588in}{2.596074in}}%
\pgfpathlineto{\pgfqpoint{3.490215in}{2.605496in}}%
\pgfpathlineto{\pgfqpoint{3.482286in}{2.593712in}}%
\pgfpathlineto{\pgfqpoint{3.474352in}{2.582018in}}%
\pgfpathlineto{\pgfqpoint{3.466411in}{2.570413in}}%
\pgfpathlineto{\pgfqpoint{3.458466in}{2.558895in}}%
\pgfpathclose%
\pgfusepath{fill}%
\end{pgfscope}%
\begin{pgfscope}%
\pgfpathrectangle{\pgfqpoint{1.150000in}{0.150000in}}{\pgfqpoint{5.700000in}{5.700000in}}%
\pgfusepath{clip}%
\pgfsetbuttcap%
\pgfsetroundjoin%
\definecolor{currentfill}{rgb}{0.276194,0.190074,0.493001}%
\pgfsetfillcolor{currentfill}%
\pgfsetfillopacity{0.800000}%
\pgfsetlinewidth{0.000000pt}%
\definecolor{currentstroke}{rgb}{0.000000,0.000000,0.000000}%
\pgfsetstrokecolor{currentstroke}%
\pgfsetdash{}{0pt}%
\pgfpathmoveto{\pgfqpoint{4.044380in}{2.593907in}}%
\pgfpathlineto{\pgfqpoint{4.057829in}{2.590780in}}%
\pgfpathlineto{\pgfqpoint{4.071284in}{2.587858in}}%
\pgfpathlineto{\pgfqpoint{4.084745in}{2.585140in}}%
\pgfpathlineto{\pgfqpoint{4.098213in}{2.582625in}}%
\pgfpathlineto{\pgfqpoint{4.105979in}{2.593868in}}%
\pgfpathlineto{\pgfqpoint{4.113741in}{2.605180in}}%
\pgfpathlineto{\pgfqpoint{4.121498in}{2.616564in}}%
\pgfpathlineto{\pgfqpoint{4.129250in}{2.628023in}}%
\pgfpathlineto{\pgfqpoint{4.115791in}{2.630823in}}%
\pgfpathlineto{\pgfqpoint{4.102338in}{2.633825in}}%
\pgfpathlineto{\pgfqpoint{4.088891in}{2.637032in}}%
\pgfpathlineto{\pgfqpoint{4.075451in}{2.640444in}}%
\pgfpathlineto{\pgfqpoint{4.067690in}{2.628689in}}%
\pgfpathlineto{\pgfqpoint{4.059924in}{2.617016in}}%
\pgfpathlineto{\pgfqpoint{4.052154in}{2.605424in}}%
\pgfpathlineto{\pgfqpoint{4.044380in}{2.593907in}}%
\pgfpathclose%
\pgfusepath{fill}%
\end{pgfscope}%
\begin{pgfscope}%
\pgfpathrectangle{\pgfqpoint{1.150000in}{0.150000in}}{\pgfqpoint{5.700000in}{5.700000in}}%
\pgfusepath{clip}%
\pgfsetbuttcap%
\pgfsetroundjoin%
\definecolor{currentfill}{rgb}{0.201239,0.383670,0.554294}%
\pgfsetfillcolor{currentfill}%
\pgfsetfillopacity{0.800000}%
\pgfsetlinewidth{0.000000pt}%
\definecolor{currentstroke}{rgb}{0.000000,0.000000,0.000000}%
\pgfsetstrokecolor{currentstroke}%
\pgfsetdash{}{0pt}%
\pgfpathmoveto{\pgfqpoint{2.888550in}{3.102188in}}%
\pgfpathlineto{\pgfqpoint{2.902122in}{3.081632in}}%
\pgfpathlineto{\pgfqpoint{2.915685in}{3.061410in}}%
\pgfpathlineto{\pgfqpoint{2.929238in}{3.041519in}}%
\pgfpathlineto{\pgfqpoint{2.942782in}{3.021956in}}%
\pgfpathlineto{\pgfqpoint{2.950868in}{3.033897in}}%
\pgfpathlineto{\pgfqpoint{2.958947in}{3.045989in}}%
\pgfpathlineto{\pgfqpoint{2.967017in}{3.058234in}}%
\pgfpathlineto{\pgfqpoint{2.975079in}{3.070635in}}%
\pgfpathlineto{\pgfqpoint{2.961550in}{3.090293in}}%
\pgfpathlineto{\pgfqpoint{2.948012in}{3.110280in}}%
\pgfpathlineto{\pgfqpoint{2.934465in}{3.130597in}}%
\pgfpathlineto{\pgfqpoint{2.920908in}{3.151249in}}%
\pgfpathlineto{\pgfqpoint{2.912831in}{3.138741in}}%
\pgfpathlineto{\pgfqpoint{2.904745in}{3.126396in}}%
\pgfpathlineto{\pgfqpoint{2.896652in}{3.114212in}}%
\pgfpathlineto{\pgfqpoint{2.888550in}{3.102188in}}%
\pgfpathclose%
\pgfusepath{fill}%
\end{pgfscope}%
\begin{pgfscope}%
\pgfpathrectangle{\pgfqpoint{1.150000in}{0.150000in}}{\pgfqpoint{5.700000in}{5.700000in}}%
\pgfusepath{clip}%
\pgfsetbuttcap%
\pgfsetroundjoin%
\definecolor{currentfill}{rgb}{0.175841,0.441290,0.557685}%
\pgfsetfillcolor{currentfill}%
\pgfsetfillopacity{0.800000}%
\pgfsetlinewidth{0.000000pt}%
\definecolor{currentstroke}{rgb}{0.000000,0.000000,0.000000}%
\pgfsetstrokecolor{currentstroke}%
\pgfsetdash{}{0pt}%
\pgfpathmoveto{\pgfqpoint{5.232526in}{3.221855in}}%
\pgfpathlineto{\pgfqpoint{5.246328in}{3.222367in}}%
\pgfpathlineto{\pgfqpoint{5.260141in}{3.223053in}}%
\pgfpathlineto{\pgfqpoint{5.273967in}{3.223913in}}%
\pgfpathlineto{\pgfqpoint{5.287806in}{3.224946in}}%
\pgfpathlineto{\pgfqpoint{5.295259in}{3.237740in}}%
\pgfpathlineto{\pgfqpoint{5.302715in}{3.250819in}}%
\pgfpathlineto{\pgfqpoint{5.310175in}{3.264192in}}%
\pgfpathlineto{\pgfqpoint{5.317637in}{3.277866in}}%
\pgfpathlineto{\pgfqpoint{5.303821in}{3.277561in}}%
\pgfpathlineto{\pgfqpoint{5.290017in}{3.277429in}}%
\pgfpathlineto{\pgfqpoint{5.276224in}{3.277470in}}%
\pgfpathlineto{\pgfqpoint{5.262444in}{3.277684in}}%
\pgfpathlineto{\pgfqpoint{5.254960in}{3.263271in}}%
\pgfpathlineto{\pgfqpoint{5.247479in}{3.249167in}}%
\pgfpathlineto{\pgfqpoint{5.240001in}{3.235365in}}%
\pgfpathlineto{\pgfqpoint{5.232526in}{3.221855in}}%
\pgfpathclose%
\pgfusepath{fill}%
\end{pgfscope}%
\begin{pgfscope}%
\pgfpathrectangle{\pgfqpoint{1.150000in}{0.150000in}}{\pgfqpoint{5.700000in}{5.700000in}}%
\pgfusepath{clip}%
\pgfsetbuttcap%
\pgfsetroundjoin%
\definecolor{currentfill}{rgb}{0.280868,0.160771,0.472899}%
\pgfsetfillcolor{currentfill}%
\pgfsetfillopacity{0.800000}%
\pgfsetlinewidth{0.000000pt}%
\definecolor{currentstroke}{rgb}{0.000000,0.000000,0.000000}%
\pgfsetstrokecolor{currentstroke}%
\pgfsetdash{}{0pt}%
\pgfpathmoveto{\pgfqpoint{3.735866in}{2.528557in}}%
\pgfpathlineto{\pgfqpoint{3.749261in}{2.522684in}}%
\pgfpathlineto{\pgfqpoint{3.762660in}{2.517033in}}%
\pgfpathlineto{\pgfqpoint{3.776063in}{2.511601in}}%
\pgfpathlineto{\pgfqpoint{3.789470in}{2.506389in}}%
\pgfpathlineto{\pgfqpoint{3.797326in}{2.517765in}}%
\pgfpathlineto{\pgfqpoint{3.805178in}{2.529207in}}%
\pgfpathlineto{\pgfqpoint{3.813025in}{2.540715in}}%
\pgfpathlineto{\pgfqpoint{3.820867in}{2.552294in}}%
\pgfpathlineto{\pgfqpoint{3.807469in}{2.557697in}}%
\pgfpathlineto{\pgfqpoint{3.794074in}{2.563319in}}%
\pgfpathlineto{\pgfqpoint{3.780684in}{2.569160in}}%
\pgfpathlineto{\pgfqpoint{3.767297in}{2.575223in}}%
\pgfpathlineto{\pgfqpoint{3.759447in}{2.563443in}}%
\pgfpathlineto{\pgfqpoint{3.751591in}{2.551740in}}%
\pgfpathlineto{\pgfqpoint{3.743731in}{2.540112in}}%
\pgfpathlineto{\pgfqpoint{3.735866in}{2.528557in}}%
\pgfpathclose%
\pgfusepath{fill}%
\end{pgfscope}%
\begin{pgfscope}%
\pgfpathrectangle{\pgfqpoint{1.150000in}{0.150000in}}{\pgfqpoint{5.700000in}{5.700000in}}%
\pgfusepath{clip}%
\pgfsetbuttcap%
\pgfsetroundjoin%
\definecolor{currentfill}{rgb}{0.277134,0.185228,0.489898}%
\pgfsetfillcolor{currentfill}%
\pgfsetfillopacity{0.800000}%
\pgfsetlinewidth{0.000000pt}%
\definecolor{currentstroke}{rgb}{0.000000,0.000000,0.000000}%
\pgfsetstrokecolor{currentstroke}%
\pgfsetdash{}{0pt}%
\pgfpathmoveto{\pgfqpoint{3.319460in}{2.596172in}}%
\pgfpathlineto{\pgfqpoint{3.332859in}{2.584975in}}%
\pgfpathlineto{\pgfqpoint{3.346257in}{2.574034in}}%
\pgfpathlineto{\pgfqpoint{3.359653in}{2.563346in}}%
\pgfpathlineto{\pgfqpoint{3.373048in}{2.552911in}}%
\pgfpathlineto{\pgfqpoint{3.381028in}{2.564182in}}%
\pgfpathlineto{\pgfqpoint{3.389002in}{2.575541in}}%
\pgfpathlineto{\pgfqpoint{3.396971in}{2.586990in}}%
\pgfpathlineto{\pgfqpoint{3.404933in}{2.598531in}}%
\pgfpathlineto{\pgfqpoint{3.391549in}{2.609062in}}%
\pgfpathlineto{\pgfqpoint{3.378164in}{2.619846in}}%
\pgfpathlineto{\pgfqpoint{3.364778in}{2.630883in}}%
\pgfpathlineto{\pgfqpoint{3.351390in}{2.642176in}}%
\pgfpathlineto{\pgfqpoint{3.343417in}{2.630528in}}%
\pgfpathlineto{\pgfqpoint{3.335438in}{2.618979in}}%
\pgfpathlineto{\pgfqpoint{3.327452in}{2.607528in}}%
\pgfpathlineto{\pgfqpoint{3.319460in}{2.596172in}}%
\pgfpathclose%
\pgfusepath{fill}%
\end{pgfscope}%
\begin{pgfscope}%
\pgfpathrectangle{\pgfqpoint{1.150000in}{0.150000in}}{\pgfqpoint{5.700000in}{5.700000in}}%
\pgfusepath{clip}%
\pgfsetbuttcap%
\pgfsetroundjoin%
\definecolor{currentfill}{rgb}{0.278012,0.180367,0.486697}%
\pgfsetfillcolor{currentfill}%
\pgfsetfillopacity{0.800000}%
\pgfsetlinewidth{0.000000pt}%
\definecolor{currentstroke}{rgb}{0.000000,0.000000,0.000000}%
\pgfsetstrokecolor{currentstroke}%
\pgfsetdash{}{0pt}%
\pgfpathmoveto{\pgfqpoint{3.959468in}{2.562105in}}%
\pgfpathlineto{\pgfqpoint{3.972902in}{2.558401in}}%
\pgfpathlineto{\pgfqpoint{3.986341in}{2.554907in}}%
\pgfpathlineto{\pgfqpoint{3.999785in}{2.551620in}}%
\pgfpathlineto{\pgfqpoint{4.013236in}{2.548540in}}%
\pgfpathlineto{\pgfqpoint{4.021029in}{2.559783in}}%
\pgfpathlineto{\pgfqpoint{4.028817in}{2.571091in}}%
\pgfpathlineto{\pgfqpoint{4.036601in}{2.582464in}}%
\pgfpathlineto{\pgfqpoint{4.044380in}{2.593907in}}%
\pgfpathlineto{\pgfqpoint{4.030938in}{2.597241in}}%
\pgfpathlineto{\pgfqpoint{4.017501in}{2.600781in}}%
\pgfpathlineto{\pgfqpoint{4.004070in}{2.604529in}}%
\pgfpathlineto{\pgfqpoint{3.990645in}{2.608485in}}%
\pgfpathlineto{\pgfqpoint{3.982858in}{2.596777in}}%
\pgfpathlineto{\pgfqpoint{3.975066in}{2.585147in}}%
\pgfpathlineto{\pgfqpoint{3.967270in}{2.573590in}}%
\pgfpathlineto{\pgfqpoint{3.959468in}{2.562105in}}%
\pgfpathclose%
\pgfusepath{fill}%
\end{pgfscope}%
\begin{pgfscope}%
\pgfpathrectangle{\pgfqpoint{1.150000in}{0.150000in}}{\pgfqpoint{5.700000in}{5.700000in}}%
\pgfusepath{clip}%
\pgfsetbuttcap%
\pgfsetroundjoin%
\definecolor{currentfill}{rgb}{0.168126,0.459988,0.558082}%
\pgfsetfillcolor{currentfill}%
\pgfsetfillopacity{0.800000}%
\pgfsetlinewidth{0.000000pt}%
\definecolor{currentstroke}{rgb}{0.000000,0.000000,0.000000}%
\pgfsetstrokecolor{currentstroke}%
\pgfsetdash{}{0pt}%
\pgfpathmoveto{\pgfqpoint{5.317637in}{3.277866in}}%
\pgfpathlineto{\pgfqpoint{5.331466in}{3.278344in}}%
\pgfpathlineto{\pgfqpoint{5.345307in}{3.278994in}}%
\pgfpathlineto{\pgfqpoint{5.359160in}{3.279817in}}%
\pgfpathlineto{\pgfqpoint{5.373026in}{3.280811in}}%
\pgfpathlineto{\pgfqpoint{5.380470in}{3.294048in}}%
\pgfpathlineto{\pgfqpoint{5.387917in}{3.307597in}}%
\pgfpathlineto{\pgfqpoint{5.395369in}{3.321466in}}%
\pgfpathlineto{\pgfqpoint{5.381521in}{3.321038in}}%
\pgfpathlineto{\pgfqpoint{5.367685in}{3.320782in}}%
\pgfpathlineto{\pgfqpoint{5.353861in}{3.320698in}}%
\pgfpathlineto{\pgfqpoint{5.340049in}{3.320786in}}%
\pgfpathlineto{\pgfqpoint{5.332574in}{3.306155in}}%
\pgfpathlineto{\pgfqpoint{5.325104in}{3.291851in}}%
\pgfpathlineto{\pgfqpoint{5.317637in}{3.277866in}}%
\pgfpathclose%
\pgfusepath{fill}%
\end{pgfscope}%
\begin{pgfscope}%
\pgfpathrectangle{\pgfqpoint{1.150000in}{0.150000in}}{\pgfqpoint{5.700000in}{5.700000in}}%
\pgfusepath{clip}%
\pgfsetbuttcap%
\pgfsetroundjoin%
\definecolor{currentfill}{rgb}{0.280868,0.160771,0.472899}%
\pgfsetfillcolor{currentfill}%
\pgfsetfillopacity{0.800000}%
\pgfsetlinewidth{0.000000pt}%
\definecolor{currentstroke}{rgb}{0.000000,0.000000,0.000000}%
\pgfsetstrokecolor{currentstroke}%
\pgfsetdash{}{0pt}%
\pgfpathmoveto{\pgfqpoint{3.512002in}{2.523163in}}%
\pgfpathlineto{\pgfqpoint{3.525388in}{2.514828in}}%
\pgfpathlineto{\pgfqpoint{3.538775in}{2.506730in}}%
\pgfpathlineto{\pgfqpoint{3.552164in}{2.498867in}}%
\pgfpathlineto{\pgfqpoint{3.565554in}{2.491238in}}%
\pgfpathlineto{\pgfqpoint{3.573480in}{2.502524in}}%
\pgfpathlineto{\pgfqpoint{3.581400in}{2.513882in}}%
\pgfpathlineto{\pgfqpoint{3.589315in}{2.525313in}}%
\pgfpathlineto{\pgfqpoint{3.597224in}{2.536819in}}%
\pgfpathlineto{\pgfqpoint{3.583844in}{2.544576in}}%
\pgfpathlineto{\pgfqpoint{3.570465in}{2.552567in}}%
\pgfpathlineto{\pgfqpoint{3.557087in}{2.560793in}}%
\pgfpathlineto{\pgfqpoint{3.543711in}{2.569255in}}%
\pgfpathlineto{\pgfqpoint{3.535792in}{2.557609in}}%
\pgfpathlineto{\pgfqpoint{3.527867in}{2.546047in}}%
\pgfpathlineto{\pgfqpoint{3.519937in}{2.534565in}}%
\pgfpathlineto{\pgfqpoint{3.512002in}{2.523163in}}%
\pgfpathclose%
\pgfusepath{fill}%
\end{pgfscope}%
\begin{pgfscope}%
\pgfpathrectangle{\pgfqpoint{1.150000in}{0.150000in}}{\pgfqpoint{5.700000in}{5.700000in}}%
\pgfusepath{clip}%
\pgfsetbuttcap%
\pgfsetroundjoin%
\definecolor{currentfill}{rgb}{0.187231,0.414746,0.556547}%
\pgfsetfillcolor{currentfill}%
\pgfsetfillopacity{0.800000}%
\pgfsetlinewidth{0.000000pt}%
\definecolor{currentstroke}{rgb}{0.000000,0.000000,0.000000}%
\pgfsetstrokecolor{currentstroke}%
\pgfsetdash{}{0pt}%
\pgfpathmoveto{\pgfqpoint{2.834156in}{3.187823in}}%
\pgfpathlineto{\pgfqpoint{2.847771in}{3.165896in}}%
\pgfpathlineto{\pgfqpoint{2.861375in}{3.144317in}}%
\pgfpathlineto{\pgfqpoint{2.874967in}{3.123082in}}%
\pgfpathlineto{\pgfqpoint{2.888550in}{3.102188in}}%
\pgfpathlineto{\pgfqpoint{2.896652in}{3.114212in}}%
\pgfpathlineto{\pgfqpoint{2.904745in}{3.126396in}}%
\pgfpathlineto{\pgfqpoint{2.912831in}{3.138741in}}%
\pgfpathlineto{\pgfqpoint{2.920908in}{3.151249in}}%
\pgfpathlineto{\pgfqpoint{2.907341in}{3.172239in}}%
\pgfpathlineto{\pgfqpoint{2.893764in}{3.193570in}}%
\pgfpathlineto{\pgfqpoint{2.880176in}{3.215245in}}%
\pgfpathlineto{\pgfqpoint{2.866578in}{3.237268in}}%
\pgfpathlineto{\pgfqpoint{2.858485in}{3.224652in}}%
\pgfpathlineto{\pgfqpoint{2.850384in}{3.212206in}}%
\pgfpathlineto{\pgfqpoint{2.842274in}{3.199931in}}%
\pgfpathlineto{\pgfqpoint{2.834156in}{3.187823in}}%
\pgfpathclose%
\pgfusepath{fill}%
\end{pgfscope}%
\begin{pgfscope}%
\pgfpathrectangle{\pgfqpoint{1.150000in}{0.150000in}}{\pgfqpoint{5.700000in}{5.700000in}}%
\pgfusepath{clip}%
\pgfsetbuttcap%
\pgfsetroundjoin%
\definecolor{currentfill}{rgb}{0.241237,0.296485,0.539709}%
\pgfsetfillcolor{currentfill}%
\pgfsetfillopacity{0.800000}%
\pgfsetlinewidth{0.000000pt}%
\definecolor{currentstroke}{rgb}{0.000000,0.000000,0.000000}%
\pgfsetstrokecolor{currentstroke}%
\pgfsetdash{}{0pt}%
\pgfpathmoveto{\pgfqpoint{4.607690in}{2.825727in}}%
\pgfpathlineto{\pgfqpoint{4.621310in}{2.825885in}}%
\pgfpathlineto{\pgfqpoint{4.634939in}{2.826229in}}%
\pgfpathlineto{\pgfqpoint{4.648579in}{2.826759in}}%
\pgfpathlineto{\pgfqpoint{4.662228in}{2.827474in}}%
\pgfpathlineto{\pgfqpoint{4.669828in}{2.838258in}}%
\pgfpathlineto{\pgfqpoint{4.677424in}{2.849163in}}%
\pgfpathlineto{\pgfqpoint{4.685017in}{2.860193in}}%
\pgfpathlineto{\pgfqpoint{4.692607in}{2.871355in}}%
\pgfpathlineto{\pgfqpoint{4.678970in}{2.871115in}}%
\pgfpathlineto{\pgfqpoint{4.665343in}{2.871060in}}%
\pgfpathlineto{\pgfqpoint{4.651725in}{2.871190in}}%
\pgfpathlineto{\pgfqpoint{4.638118in}{2.871507in}}%
\pgfpathlineto{\pgfqpoint{4.630516in}{2.859859in}}%
\pgfpathlineto{\pgfqpoint{4.622910in}{2.848350in}}%
\pgfpathlineto{\pgfqpoint{4.615302in}{2.836974in}}%
\pgfpathlineto{\pgfqpoint{4.607690in}{2.825727in}}%
\pgfpathclose%
\pgfusepath{fill}%
\end{pgfscope}%
\begin{pgfscope}%
\pgfpathrectangle{\pgfqpoint{1.150000in}{0.150000in}}{\pgfqpoint{5.700000in}{5.700000in}}%
\pgfusepath{clip}%
\pgfsetbuttcap%
\pgfsetroundjoin%
\definecolor{currentfill}{rgb}{0.281412,0.155834,0.469201}%
\pgfsetfillcolor{currentfill}%
\pgfsetfillopacity{0.800000}%
\pgfsetlinewidth{0.000000pt}%
\definecolor{currentstroke}{rgb}{0.000000,0.000000,0.000000}%
\pgfsetstrokecolor{currentstroke}%
\pgfsetdash{}{0pt}%
\pgfpathmoveto{\pgfqpoint{3.650768in}{2.508100in}}%
\pgfpathlineto{\pgfqpoint{3.664160in}{2.501491in}}%
\pgfpathlineto{\pgfqpoint{3.677554in}{2.495108in}}%
\pgfpathlineto{\pgfqpoint{3.690952in}{2.488949in}}%
\pgfpathlineto{\pgfqpoint{3.704353in}{2.483014in}}%
\pgfpathlineto{\pgfqpoint{3.712239in}{2.494302in}}%
\pgfpathlineto{\pgfqpoint{3.720120in}{2.505654in}}%
\pgfpathlineto{\pgfqpoint{3.727995in}{2.517071in}}%
\pgfpathlineto{\pgfqpoint{3.735866in}{2.528557in}}%
\pgfpathlineto{\pgfqpoint{3.722474in}{2.534651in}}%
\pgfpathlineto{\pgfqpoint{3.709085in}{2.540969in}}%
\pgfpathlineto{\pgfqpoint{3.695699in}{2.547511in}}%
\pgfpathlineto{\pgfqpoint{3.682316in}{2.554279in}}%
\pgfpathlineto{\pgfqpoint{3.674437in}{2.542623in}}%
\pgfpathlineto{\pgfqpoint{3.666552in}{2.531043in}}%
\pgfpathlineto{\pgfqpoint{3.658663in}{2.519536in}}%
\pgfpathlineto{\pgfqpoint{3.650768in}{2.508100in}}%
\pgfpathclose%
\pgfusepath{fill}%
\end{pgfscope}%
\begin{pgfscope}%
\pgfpathrectangle{\pgfqpoint{1.150000in}{0.150000in}}{\pgfqpoint{5.700000in}{5.700000in}}%
\pgfusepath{clip}%
\pgfsetbuttcap%
\pgfsetroundjoin%
\definecolor{currentfill}{rgb}{0.248629,0.278775,0.534556}%
\pgfsetfillcolor{currentfill}%
\pgfsetfillopacity{0.800000}%
\pgfsetlinewidth{0.000000pt}%
\definecolor{currentstroke}{rgb}{0.000000,0.000000,0.000000}%
\pgfsetstrokecolor{currentstroke}%
\pgfsetdash{}{0pt}%
\pgfpathmoveto{\pgfqpoint{4.522785in}{2.781384in}}%
\pgfpathlineto{\pgfqpoint{4.536377in}{2.781234in}}%
\pgfpathlineto{\pgfqpoint{4.549979in}{2.781273in}}%
\pgfpathlineto{\pgfqpoint{4.563590in}{2.781499in}}%
\pgfpathlineto{\pgfqpoint{4.577211in}{2.781914in}}%
\pgfpathlineto{\pgfqpoint{4.584836in}{2.792701in}}%
\pgfpathlineto{\pgfqpoint{4.592458in}{2.803596in}}%
\pgfpathlineto{\pgfqpoint{4.600076in}{2.814603in}}%
\pgfpathlineto{\pgfqpoint{4.607690in}{2.825727in}}%
\pgfpathlineto{\pgfqpoint{4.594081in}{2.825756in}}%
\pgfpathlineto{\pgfqpoint{4.580481in}{2.825972in}}%
\pgfpathlineto{\pgfqpoint{4.566891in}{2.826377in}}%
\pgfpathlineto{\pgfqpoint{4.553310in}{2.826969in}}%
\pgfpathlineto{\pgfqpoint{4.545684in}{2.815390in}}%
\pgfpathlineto{\pgfqpoint{4.538055in}{2.803936in}}%
\pgfpathlineto{\pgfqpoint{4.530422in}{2.792603in}}%
\pgfpathlineto{\pgfqpoint{4.522785in}{2.781384in}}%
\pgfpathclose%
\pgfusepath{fill}%
\end{pgfscope}%
\begin{pgfscope}%
\pgfpathrectangle{\pgfqpoint{1.150000in}{0.150000in}}{\pgfqpoint{5.700000in}{5.700000in}}%
\pgfusepath{clip}%
\pgfsetbuttcap%
\pgfsetroundjoin%
\definecolor{currentfill}{rgb}{0.280255,0.165693,0.476498}%
\pgfsetfillcolor{currentfill}%
\pgfsetfillopacity{0.800000}%
\pgfsetlinewidth{0.000000pt}%
\definecolor{currentstroke}{rgb}{0.000000,0.000000,0.000000}%
\pgfsetstrokecolor{currentstroke}%
\pgfsetdash{}{0pt}%
\pgfpathmoveto{\pgfqpoint{3.874503in}{2.532846in}}%
\pgfpathlineto{\pgfqpoint{3.887923in}{2.528521in}}%
\pgfpathlineto{\pgfqpoint{3.901349in}{2.524408in}}%
\pgfpathlineto{\pgfqpoint{3.914780in}{2.520506in}}%
\pgfpathlineto{\pgfqpoint{3.928216in}{2.516816in}}%
\pgfpathlineto{\pgfqpoint{3.936036in}{2.528046in}}%
\pgfpathlineto{\pgfqpoint{3.943852in}{2.539335in}}%
\pgfpathlineto{\pgfqpoint{3.951662in}{2.550687in}}%
\pgfpathlineto{\pgfqpoint{3.959468in}{2.562105in}}%
\pgfpathlineto{\pgfqpoint{3.946041in}{2.566018in}}%
\pgfpathlineto{\pgfqpoint{3.932618in}{2.570141in}}%
\pgfpathlineto{\pgfqpoint{3.919201in}{2.574476in}}%
\pgfpathlineto{\pgfqpoint{3.905788in}{2.579023in}}%
\pgfpathlineto{\pgfqpoint{3.897974in}{2.567373in}}%
\pgfpathlineto{\pgfqpoint{3.890155in}{2.555795in}}%
\pgfpathlineto{\pgfqpoint{3.882331in}{2.544287in}}%
\pgfpathlineto{\pgfqpoint{3.874503in}{2.532846in}}%
\pgfpathclose%
\pgfusepath{fill}%
\end{pgfscope}%
\begin{pgfscope}%
\pgfpathrectangle{\pgfqpoint{1.150000in}{0.150000in}}{\pgfqpoint{5.700000in}{5.700000in}}%
\pgfusepath{clip}%
\pgfsetbuttcap%
\pgfsetroundjoin%
\definecolor{currentfill}{rgb}{0.233603,0.313828,0.543914}%
\pgfsetfillcolor{currentfill}%
\pgfsetfillopacity{0.800000}%
\pgfsetlinewidth{0.000000pt}%
\definecolor{currentstroke}{rgb}{0.000000,0.000000,0.000000}%
\pgfsetstrokecolor{currentstroke}%
\pgfsetdash{}{0pt}%
\pgfpathmoveto{\pgfqpoint{4.692607in}{2.871355in}}%
\pgfpathlineto{\pgfqpoint{4.706254in}{2.871780in}}%
\pgfpathlineto{\pgfqpoint{4.719912in}{2.872390in}}%
\pgfpathlineto{\pgfqpoint{4.733581in}{2.873183in}}%
\pgfpathlineto{\pgfqpoint{4.747260in}{2.874160in}}%
\pgfpathlineto{\pgfqpoint{4.754834in}{2.884964in}}%
\pgfpathlineto{\pgfqpoint{4.762405in}{2.895903in}}%
\pgfpathlineto{\pgfqpoint{4.769974in}{2.906984in}}%
\pgfpathlineto{\pgfqpoint{4.777540in}{2.918211in}}%
\pgfpathlineto{\pgfqpoint{4.763874in}{2.917741in}}%
\pgfpathlineto{\pgfqpoint{4.750219in}{2.917455in}}%
\pgfpathlineto{\pgfqpoint{4.736575in}{2.917352in}}%
\pgfpathlineto{\pgfqpoint{4.722940in}{2.917433in}}%
\pgfpathlineto{\pgfqpoint{4.715361in}{2.905687in}}%
\pgfpathlineto{\pgfqpoint{4.707779in}{2.894097in}}%
\pgfpathlineto{\pgfqpoint{4.700194in}{2.882654in}}%
\pgfpathlineto{\pgfqpoint{4.692607in}{2.871355in}}%
\pgfpathclose%
\pgfusepath{fill}%
\end{pgfscope}%
\begin{pgfscope}%
\pgfpathrectangle{\pgfqpoint{1.150000in}{0.150000in}}{\pgfqpoint{5.700000in}{5.700000in}}%
\pgfusepath{clip}%
\pgfsetbuttcap%
\pgfsetroundjoin%
\definecolor{currentfill}{rgb}{0.255645,0.260703,0.528312}%
\pgfsetfillcolor{currentfill}%
\pgfsetfillopacity{0.800000}%
\pgfsetlinewidth{0.000000pt}%
\definecolor{currentstroke}{rgb}{0.000000,0.000000,0.000000}%
\pgfsetstrokecolor{currentstroke}%
\pgfsetdash{}{0pt}%
\pgfpathmoveto{\pgfqpoint{4.437886in}{2.738406in}}%
\pgfpathlineto{\pgfqpoint{4.451451in}{2.737908in}}%
\pgfpathlineto{\pgfqpoint{4.465026in}{2.737601in}}%
\pgfpathlineto{\pgfqpoint{4.478610in}{2.737484in}}%
\pgfpathlineto{\pgfqpoint{4.492204in}{2.737557in}}%
\pgfpathlineto{\pgfqpoint{4.499855in}{2.748366in}}%
\pgfpathlineto{\pgfqpoint{4.507502in}{2.759270in}}%
\pgfpathlineto{\pgfqpoint{4.515145in}{2.770274in}}%
\pgfpathlineto{\pgfqpoint{4.522785in}{2.781384in}}%
\pgfpathlineto{\pgfqpoint{4.509203in}{2.781722in}}%
\pgfpathlineto{\pgfqpoint{4.495629in}{2.782251in}}%
\pgfpathlineto{\pgfqpoint{4.482065in}{2.782969in}}%
\pgfpathlineto{\pgfqpoint{4.468510in}{2.783879in}}%
\pgfpathlineto{\pgfqpoint{4.460860in}{2.772347in}}%
\pgfpathlineto{\pgfqpoint{4.453205in}{2.760927in}}%
\pgfpathlineto{\pgfqpoint{4.445547in}{2.749615in}}%
\pgfpathlineto{\pgfqpoint{4.437886in}{2.738406in}}%
\pgfpathclose%
\pgfusepath{fill}%
\end{pgfscope}%
\begin{pgfscope}%
\pgfpathrectangle{\pgfqpoint{1.150000in}{0.150000in}}{\pgfqpoint{5.700000in}{5.700000in}}%
\pgfusepath{clip}%
\pgfsetbuttcap%
\pgfsetroundjoin%
\definecolor{currentfill}{rgb}{0.223925,0.334994,0.548053}%
\pgfsetfillcolor{currentfill}%
\pgfsetfillopacity{0.800000}%
\pgfsetlinewidth{0.000000pt}%
\definecolor{currentstroke}{rgb}{0.000000,0.000000,0.000000}%
\pgfsetstrokecolor{currentstroke}%
\pgfsetdash{}{0pt}%
\pgfpathmoveto{\pgfqpoint{4.777540in}{2.918211in}}%
\pgfpathlineto{\pgfqpoint{4.791216in}{2.918864in}}%
\pgfpathlineto{\pgfqpoint{4.804903in}{2.919699in}}%
\pgfpathlineto{\pgfqpoint{4.818601in}{2.920716in}}%
\pgfpathlineto{\pgfqpoint{4.832310in}{2.921914in}}%
\pgfpathlineto{\pgfqpoint{4.839859in}{2.932768in}}%
\pgfpathlineto{\pgfqpoint{4.847406in}{2.943773in}}%
\pgfpathlineto{\pgfqpoint{4.854951in}{2.954935in}}%
\pgfpathlineto{\pgfqpoint{4.862494in}{2.966262in}}%
\pgfpathlineto{\pgfqpoint{4.848800in}{2.965602in}}%
\pgfpathlineto{\pgfqpoint{4.835117in}{2.965123in}}%
\pgfpathlineto{\pgfqpoint{4.821444in}{2.964826in}}%
\pgfpathlineto{\pgfqpoint{4.807782in}{2.964711in}}%
\pgfpathlineto{\pgfqpoint{4.800225in}{2.952836in}}%
\pgfpathlineto{\pgfqpoint{4.792665in}{2.941131in}}%
\pgfpathlineto{\pgfqpoint{4.785103in}{2.929592in}}%
\pgfpathlineto{\pgfqpoint{4.777540in}{2.918211in}}%
\pgfpathclose%
\pgfusepath{fill}%
\end{pgfscope}%
\begin{pgfscope}%
\pgfpathrectangle{\pgfqpoint{1.150000in}{0.150000in}}{\pgfqpoint{5.700000in}{5.700000in}}%
\pgfusepath{clip}%
\pgfsetbuttcap%
\pgfsetroundjoin%
\definecolor{currentfill}{rgb}{0.255645,0.260703,0.528312}%
\pgfsetfillcolor{currentfill}%
\pgfsetfillopacity{0.800000}%
\pgfsetlinewidth{0.000000pt}%
\definecolor{currentstroke}{rgb}{0.000000,0.000000,0.000000}%
\pgfsetstrokecolor{currentstroke}%
\pgfsetdash{}{0pt}%
\pgfpathmoveto{\pgfqpoint{3.072468in}{2.766202in}}%
\pgfpathlineto{\pgfqpoint{3.085934in}{2.750704in}}%
\pgfpathlineto{\pgfqpoint{3.099395in}{2.735495in}}%
\pgfpathlineto{\pgfqpoint{3.112851in}{2.720573in}}%
\pgfpathlineto{\pgfqpoint{3.126302in}{2.705936in}}%
\pgfpathlineto{\pgfqpoint{3.134358in}{2.717076in}}%
\pgfpathlineto{\pgfqpoint{3.142408in}{2.728331in}}%
\pgfpathlineto{\pgfqpoint{3.150451in}{2.739701in}}%
\pgfpathlineto{\pgfqpoint{3.158487in}{2.751189in}}%
\pgfpathlineto{\pgfqpoint{3.145049in}{2.765891in}}%
\pgfpathlineto{\pgfqpoint{3.131607in}{2.780876in}}%
\pgfpathlineto{\pgfqpoint{3.118160in}{2.796149in}}%
\pgfpathlineto{\pgfqpoint{3.104708in}{2.811711in}}%
\pgfpathlineto{\pgfqpoint{3.096659in}{2.800147in}}%
\pgfpathlineto{\pgfqpoint{3.088602in}{2.788708in}}%
\pgfpathlineto{\pgfqpoint{3.080538in}{2.777394in}}%
\pgfpathlineto{\pgfqpoint{3.072468in}{2.766202in}}%
\pgfpathclose%
\pgfusepath{fill}%
\end{pgfscope}%
\begin{pgfscope}%
\pgfpathrectangle{\pgfqpoint{1.150000in}{0.150000in}}{\pgfqpoint{5.700000in}{5.700000in}}%
\pgfusepath{clip}%
\pgfsetbuttcap%
\pgfsetroundjoin%
\definecolor{currentfill}{rgb}{0.279574,0.170599,0.479997}%
\pgfsetfillcolor{currentfill}%
\pgfsetfillopacity{0.800000}%
\pgfsetlinewidth{0.000000pt}%
\definecolor{currentstroke}{rgb}{0.000000,0.000000,0.000000}%
\pgfsetstrokecolor{currentstroke}%
\pgfsetdash{}{0pt}%
\pgfpathmoveto{\pgfqpoint{3.373048in}{2.552911in}}%
\pgfpathlineto{\pgfqpoint{3.386443in}{2.542726in}}%
\pgfpathlineto{\pgfqpoint{3.399837in}{2.532790in}}%
\pgfpathlineto{\pgfqpoint{3.413231in}{2.523102in}}%
\pgfpathlineto{\pgfqpoint{3.426625in}{2.513659in}}%
\pgfpathlineto{\pgfqpoint{3.434594in}{2.524846in}}%
\pgfpathlineto{\pgfqpoint{3.442557in}{2.536113in}}%
\pgfpathlineto{\pgfqpoint{3.450514in}{2.547462in}}%
\pgfpathlineto{\pgfqpoint{3.458466in}{2.558895in}}%
\pgfpathlineto{\pgfqpoint{3.445083in}{2.568434in}}%
\pgfpathlineto{\pgfqpoint{3.431700in}{2.578219in}}%
\pgfpathlineto{\pgfqpoint{3.418316in}{2.588250in}}%
\pgfpathlineto{\pgfqpoint{3.404933in}{2.598531in}}%
\pgfpathlineto{\pgfqpoint{3.396971in}{2.586990in}}%
\pgfpathlineto{\pgfqpoint{3.389002in}{2.575541in}}%
\pgfpathlineto{\pgfqpoint{3.381028in}{2.564182in}}%
\pgfpathlineto{\pgfqpoint{3.373048in}{2.552911in}}%
\pgfpathclose%
\pgfusepath{fill}%
\end{pgfscope}%
\begin{pgfscope}%
\pgfpathrectangle{\pgfqpoint{1.150000in}{0.150000in}}{\pgfqpoint{5.700000in}{5.700000in}}%
\pgfusepath{clip}%
\pgfsetbuttcap%
\pgfsetroundjoin%
\definecolor{currentfill}{rgb}{0.244972,0.287675,0.537260}%
\pgfsetfillcolor{currentfill}%
\pgfsetfillopacity{0.800000}%
\pgfsetlinewidth{0.000000pt}%
\definecolor{currentstroke}{rgb}{0.000000,0.000000,0.000000}%
\pgfsetstrokecolor{currentstroke}%
\pgfsetdash{}{0pt}%
\pgfpathmoveto{\pgfqpoint{3.018544in}{2.831138in}}%
\pgfpathlineto{\pgfqpoint{3.032034in}{2.814457in}}%
\pgfpathlineto{\pgfqpoint{3.045518in}{2.798076in}}%
\pgfpathlineto{\pgfqpoint{3.058996in}{2.781992in}}%
\pgfpathlineto{\pgfqpoint{3.072468in}{2.766202in}}%
\pgfpathlineto{\pgfqpoint{3.080538in}{2.777394in}}%
\pgfpathlineto{\pgfqpoint{3.088602in}{2.788708in}}%
\pgfpathlineto{\pgfqpoint{3.096659in}{2.800147in}}%
\pgfpathlineto{\pgfqpoint{3.104708in}{2.811711in}}%
\pgfpathlineto{\pgfqpoint{3.091251in}{2.827564in}}%
\pgfpathlineto{\pgfqpoint{3.077788in}{2.843712in}}%
\pgfpathlineto{\pgfqpoint{3.064319in}{2.860157in}}%
\pgfpathlineto{\pgfqpoint{3.050843in}{2.876901in}}%
\pgfpathlineto{\pgfqpoint{3.042779in}{2.865262in}}%
\pgfpathlineto{\pgfqpoint{3.034708in}{2.853756in}}%
\pgfpathlineto{\pgfqpoint{3.026630in}{2.842381in}}%
\pgfpathlineto{\pgfqpoint{3.018544in}{2.831138in}}%
\pgfpathclose%
\pgfusepath{fill}%
\end{pgfscope}%
\begin{pgfscope}%
\pgfpathrectangle{\pgfqpoint{1.150000in}{0.150000in}}{\pgfqpoint{5.700000in}{5.700000in}}%
\pgfusepath{clip}%
\pgfsetbuttcap%
\pgfsetroundjoin%
\definecolor{currentfill}{rgb}{0.262138,0.242286,0.520837}%
\pgfsetfillcolor{currentfill}%
\pgfsetfillopacity{0.800000}%
\pgfsetlinewidth{0.000000pt}%
\definecolor{currentstroke}{rgb}{0.000000,0.000000,0.000000}%
\pgfsetstrokecolor{currentstroke}%
\pgfsetdash{}{0pt}%
\pgfpathmoveto{\pgfqpoint{4.352986in}{2.696901in}}%
\pgfpathlineto{\pgfqpoint{4.366526in}{2.696013in}}%
\pgfpathlineto{\pgfqpoint{4.380075in}{2.695318in}}%
\pgfpathlineto{\pgfqpoint{4.393633in}{2.694817in}}%
\pgfpathlineto{\pgfqpoint{4.407200in}{2.694508in}}%
\pgfpathlineto{\pgfqpoint{4.414878in}{2.705351in}}%
\pgfpathlineto{\pgfqpoint{4.422551in}{2.716279in}}%
\pgfpathlineto{\pgfqpoint{4.430220in}{2.727296in}}%
\pgfpathlineto{\pgfqpoint{4.437886in}{2.738406in}}%
\pgfpathlineto{\pgfqpoint{4.424329in}{2.739096in}}%
\pgfpathlineto{\pgfqpoint{4.410781in}{2.739978in}}%
\pgfpathlineto{\pgfqpoint{4.397242in}{2.741052in}}%
\pgfpathlineto{\pgfqpoint{4.383711in}{2.742319in}}%
\pgfpathlineto{\pgfqpoint{4.376036in}{2.730817in}}%
\pgfpathlineto{\pgfqpoint{4.368356in}{2.719416in}}%
\pgfpathlineto{\pgfqpoint{4.360673in}{2.708112in}}%
\pgfpathlineto{\pgfqpoint{4.352986in}{2.696901in}}%
\pgfpathclose%
\pgfusepath{fill}%
\end{pgfscope}%
\begin{pgfscope}%
\pgfpathrectangle{\pgfqpoint{1.150000in}{0.150000in}}{\pgfqpoint{5.700000in}{5.700000in}}%
\pgfusepath{clip}%
\pgfsetbuttcap%
\pgfsetroundjoin%
\definecolor{currentfill}{rgb}{0.216210,0.351535,0.550627}%
\pgfsetfillcolor{currentfill}%
\pgfsetfillopacity{0.800000}%
\pgfsetlinewidth{0.000000pt}%
\definecolor{currentstroke}{rgb}{0.000000,0.000000,0.000000}%
\pgfsetstrokecolor{currentstroke}%
\pgfsetdash{}{0pt}%
\pgfpathmoveto{\pgfqpoint{4.862494in}{2.966262in}}%
\pgfpathlineto{\pgfqpoint{4.876200in}{2.967102in}}%
\pgfpathlineto{\pgfqpoint{4.889916in}{2.968124in}}%
\pgfpathlineto{\pgfqpoint{4.903644in}{2.969325in}}%
\pgfpathlineto{\pgfqpoint{4.917383in}{2.970707in}}%
\pgfpathlineto{\pgfqpoint{4.924909in}{2.981644in}}%
\pgfpathlineto{\pgfqpoint{4.932433in}{2.992751in}}%
\pgfpathlineto{\pgfqpoint{4.939956in}{3.004033in}}%
\pgfpathlineto{\pgfqpoint{4.947477in}{3.015497in}}%
\pgfpathlineto{\pgfqpoint{4.933753in}{3.014686in}}%
\pgfpathlineto{\pgfqpoint{4.920041in}{3.014055in}}%
\pgfpathlineto{\pgfqpoint{4.906340in}{3.013603in}}%
\pgfpathlineto{\pgfqpoint{4.892650in}{3.013332in}}%
\pgfpathlineto{\pgfqpoint{4.885113in}{3.001286in}}%
\pgfpathlineto{\pgfqpoint{4.877575in}{2.989431in}}%
\pgfpathlineto{\pgfqpoint{4.870036in}{2.977758in}}%
\pgfpathlineto{\pgfqpoint{4.862494in}{2.966262in}}%
\pgfpathclose%
\pgfusepath{fill}%
\end{pgfscope}%
\begin{pgfscope}%
\pgfpathrectangle{\pgfqpoint{1.150000in}{0.150000in}}{\pgfqpoint{5.700000in}{5.700000in}}%
\pgfusepath{clip}%
\pgfsetbuttcap%
\pgfsetroundjoin%
\definecolor{currentfill}{rgb}{0.263663,0.237631,0.518762}%
\pgfsetfillcolor{currentfill}%
\pgfsetfillopacity{0.800000}%
\pgfsetlinewidth{0.000000pt}%
\definecolor{currentstroke}{rgb}{0.000000,0.000000,0.000000}%
\pgfsetstrokecolor{currentstroke}%
\pgfsetdash{}{0pt}%
\pgfpathmoveto{\pgfqpoint{3.126302in}{2.705936in}}%
\pgfpathlineto{\pgfqpoint{3.139749in}{2.691581in}}%
\pgfpathlineto{\pgfqpoint{3.153191in}{2.677506in}}%
\pgfpathlineto{\pgfqpoint{3.166629in}{2.663709in}}%
\pgfpathlineto{\pgfqpoint{3.180064in}{2.650187in}}%
\pgfpathlineto{\pgfqpoint{3.188106in}{2.661275in}}%
\pgfpathlineto{\pgfqpoint{3.196142in}{2.672470in}}%
\pgfpathlineto{\pgfqpoint{3.204172in}{2.683773in}}%
\pgfpathlineto{\pgfqpoint{3.212194in}{2.695185in}}%
\pgfpathlineto{\pgfqpoint{3.198773in}{2.708771in}}%
\pgfpathlineto{\pgfqpoint{3.185348in}{2.722632in}}%
\pgfpathlineto{\pgfqpoint{3.171920in}{2.736771in}}%
\pgfpathlineto{\pgfqpoint{3.158487in}{2.751189in}}%
\pgfpathlineto{\pgfqpoint{3.150451in}{2.739701in}}%
\pgfpathlineto{\pgfqpoint{3.142408in}{2.728331in}}%
\pgfpathlineto{\pgfqpoint{3.134358in}{2.717076in}}%
\pgfpathlineto{\pgfqpoint{3.126302in}{2.705936in}}%
\pgfpathclose%
\pgfusepath{fill}%
\end{pgfscope}%
\begin{pgfscope}%
\pgfpathrectangle{\pgfqpoint{1.150000in}{0.150000in}}{\pgfqpoint{5.700000in}{5.700000in}}%
\pgfusepath{clip}%
\pgfsetbuttcap%
\pgfsetroundjoin%
\definecolor{currentfill}{rgb}{0.233603,0.313828,0.543914}%
\pgfsetfillcolor{currentfill}%
\pgfsetfillopacity{0.800000}%
\pgfsetlinewidth{0.000000pt}%
\definecolor{currentstroke}{rgb}{0.000000,0.000000,0.000000}%
\pgfsetstrokecolor{currentstroke}%
\pgfsetdash{}{0pt}%
\pgfpathmoveto{\pgfqpoint{2.964514in}{2.900908in}}%
\pgfpathlineto{\pgfqpoint{2.978032in}{2.883003in}}%
\pgfpathlineto{\pgfqpoint{2.991543in}{2.865408in}}%
\pgfpathlineto{\pgfqpoint{3.005047in}{2.848121in}}%
\pgfpathlineto{\pgfqpoint{3.018544in}{2.831138in}}%
\pgfpathlineto{\pgfqpoint{3.026630in}{2.842381in}}%
\pgfpathlineto{\pgfqpoint{3.034708in}{2.853756in}}%
\pgfpathlineto{\pgfqpoint{3.042779in}{2.865262in}}%
\pgfpathlineto{\pgfqpoint{3.050843in}{2.876901in}}%
\pgfpathlineto{\pgfqpoint{3.037361in}{2.893947in}}%
\pgfpathlineto{\pgfqpoint{3.023873in}{2.911297in}}%
\pgfpathlineto{\pgfqpoint{3.010377in}{2.928956in}}%
\pgfpathlineto{\pgfqpoint{2.996874in}{2.946924in}}%
\pgfpathlineto{\pgfqpoint{2.988795in}{2.935209in}}%
\pgfpathlineto{\pgfqpoint{2.980709in}{2.923636in}}%
\pgfpathlineto{\pgfqpoint{2.972615in}{2.912202in}}%
\pgfpathlineto{\pgfqpoint{2.964514in}{2.900908in}}%
\pgfpathclose%
\pgfusepath{fill}%
\end{pgfscope}%
\begin{pgfscope}%
\pgfpathrectangle{\pgfqpoint{1.150000in}{0.150000in}}{\pgfqpoint{5.700000in}{5.700000in}}%
\pgfusepath{clip}%
\pgfsetbuttcap%
\pgfsetroundjoin%
\definecolor{currentfill}{rgb}{0.206756,0.371758,0.553117}%
\pgfsetfillcolor{currentfill}%
\pgfsetfillopacity{0.800000}%
\pgfsetlinewidth{0.000000pt}%
\definecolor{currentstroke}{rgb}{0.000000,0.000000,0.000000}%
\pgfsetstrokecolor{currentstroke}%
\pgfsetdash{}{0pt}%
\pgfpathmoveto{\pgfqpoint{4.947477in}{3.015497in}}%
\pgfpathlineto{\pgfqpoint{4.961211in}{3.016487in}}%
\pgfpathlineto{\pgfqpoint{4.974957in}{3.017656in}}%
\pgfpathlineto{\pgfqpoint{4.988715in}{3.019004in}}%
\pgfpathlineto{\pgfqpoint{5.002485in}{3.020529in}}%
\pgfpathlineto{\pgfqpoint{5.009988in}{3.031592in}}%
\pgfpathlineto{\pgfqpoint{5.017490in}{3.042841in}}%
\pgfpathlineto{\pgfqpoint{5.024991in}{3.054286in}}%
\pgfpathlineto{\pgfqpoint{5.032492in}{3.065932in}}%
\pgfpathlineto{\pgfqpoint{5.018740in}{3.065009in}}%
\pgfpathlineto{\pgfqpoint{5.004999in}{3.064263in}}%
\pgfpathlineto{\pgfqpoint{4.991270in}{3.063696in}}%
\pgfpathlineto{\pgfqpoint{4.977552in}{3.063307in}}%
\pgfpathlineto{\pgfqpoint{4.970034in}{3.051048in}}%
\pgfpathlineto{\pgfqpoint{4.962516in}{3.038998in}}%
\pgfpathlineto{\pgfqpoint{4.954997in}{3.027150in}}%
\pgfpathlineto{\pgfqpoint{4.947477in}{3.015497in}}%
\pgfpathclose%
\pgfusepath{fill}%
\end{pgfscope}%
\begin{pgfscope}%
\pgfpathrectangle{\pgfqpoint{1.150000in}{0.150000in}}{\pgfqpoint{5.700000in}{5.700000in}}%
\pgfusepath{clip}%
\pgfsetbuttcap%
\pgfsetroundjoin%
\definecolor{currentfill}{rgb}{0.266580,0.228262,0.514349}%
\pgfsetfillcolor{currentfill}%
\pgfsetfillopacity{0.800000}%
\pgfsetlinewidth{0.000000pt}%
\definecolor{currentstroke}{rgb}{0.000000,0.000000,0.000000}%
\pgfsetstrokecolor{currentstroke}%
\pgfsetdash{}{0pt}%
\pgfpathmoveto{\pgfqpoint{4.268079in}{2.656996in}}%
\pgfpathlineto{\pgfqpoint{4.281596in}{2.655676in}}%
\pgfpathlineto{\pgfqpoint{4.295121in}{2.654553in}}%
\pgfpathlineto{\pgfqpoint{4.308654in}{2.653625in}}%
\pgfpathlineto{\pgfqpoint{4.322195in}{2.652891in}}%
\pgfpathlineto{\pgfqpoint{4.329899in}{2.663776in}}%
\pgfpathlineto{\pgfqpoint{4.337599in}{2.674737in}}%
\pgfpathlineto{\pgfqpoint{4.345294in}{2.685777in}}%
\pgfpathlineto{\pgfqpoint{4.352986in}{2.696901in}}%
\pgfpathlineto{\pgfqpoint{4.339454in}{2.697982in}}%
\pgfpathlineto{\pgfqpoint{4.325930in}{2.699259in}}%
\pgfpathlineto{\pgfqpoint{4.312414in}{2.700731in}}%
\pgfpathlineto{\pgfqpoint{4.298907in}{2.702398in}}%
\pgfpathlineto{\pgfqpoint{4.291206in}{2.690914in}}%
\pgfpathlineto{\pgfqpoint{4.283501in}{2.679522in}}%
\pgfpathlineto{\pgfqpoint{4.275792in}{2.668217in}}%
\pgfpathlineto{\pgfqpoint{4.268079in}{2.656996in}}%
\pgfpathclose%
\pgfusepath{fill}%
\end{pgfscope}%
\begin{pgfscope}%
\pgfpathrectangle{\pgfqpoint{1.150000in}{0.150000in}}{\pgfqpoint{5.700000in}{5.700000in}}%
\pgfusepath{clip}%
\pgfsetbuttcap%
\pgfsetroundjoin%
\definecolor{currentfill}{rgb}{0.197636,0.391528,0.554969}%
\pgfsetfillcolor{currentfill}%
\pgfsetfillopacity{0.800000}%
\pgfsetlinewidth{0.000000pt}%
\definecolor{currentstroke}{rgb}{0.000000,0.000000,0.000000}%
\pgfsetstrokecolor{currentstroke}%
\pgfsetdash{}{0pt}%
\pgfpathmoveto{\pgfqpoint{5.032492in}{3.065932in}}%
\pgfpathlineto{\pgfqpoint{5.046256in}{3.067033in}}%
\pgfpathlineto{\pgfqpoint{5.060032in}{3.068311in}}%
\pgfpathlineto{\pgfqpoint{5.073820in}{3.069766in}}%
\pgfpathlineto{\pgfqpoint{5.087620in}{3.071398in}}%
\pgfpathlineto{\pgfqpoint{5.095102in}{3.082631in}}%
\pgfpathlineto{\pgfqpoint{5.102584in}{3.094071in}}%
\pgfpathlineto{\pgfqpoint{5.110065in}{3.105727in}}%
\pgfpathlineto{\pgfqpoint{5.117547in}{3.117606in}}%
\pgfpathlineto{\pgfqpoint{5.103766in}{3.116608in}}%
\pgfpathlineto{\pgfqpoint{5.089996in}{3.115787in}}%
\pgfpathlineto{\pgfqpoint{5.076239in}{3.115142in}}%
\pgfpathlineto{\pgfqpoint{5.062493in}{3.114674in}}%
\pgfpathlineto{\pgfqpoint{5.054992in}{3.102151in}}%
\pgfpathlineto{\pgfqpoint{5.047492in}{3.089857in}}%
\pgfpathlineto{\pgfqpoint{5.039992in}{3.077787in}}%
\pgfpathlineto{\pgfqpoint{5.032492in}{3.065932in}}%
\pgfpathclose%
\pgfusepath{fill}%
\end{pgfscope}%
\begin{pgfscope}%
\pgfpathrectangle{\pgfqpoint{1.150000in}{0.150000in}}{\pgfqpoint{5.700000in}{5.700000in}}%
\pgfusepath{clip}%
\pgfsetbuttcap%
\pgfsetroundjoin%
\definecolor{currentfill}{rgb}{0.270595,0.214069,0.507052}%
\pgfsetfillcolor{currentfill}%
\pgfsetfillopacity{0.800000}%
\pgfsetlinewidth{0.000000pt}%
\definecolor{currentstroke}{rgb}{0.000000,0.000000,0.000000}%
\pgfsetstrokecolor{currentstroke}%
\pgfsetdash{}{0pt}%
\pgfpathmoveto{\pgfqpoint{3.180064in}{2.650187in}}%
\pgfpathlineto{\pgfqpoint{3.193495in}{2.636939in}}%
\pgfpathlineto{\pgfqpoint{3.206922in}{2.623962in}}%
\pgfpathlineto{\pgfqpoint{3.220347in}{2.611255in}}%
\pgfpathlineto{\pgfqpoint{3.233769in}{2.598815in}}%
\pgfpathlineto{\pgfqpoint{3.241798in}{2.609851in}}%
\pgfpathlineto{\pgfqpoint{3.249821in}{2.620985in}}%
\pgfpathlineto{\pgfqpoint{3.257837in}{2.632220in}}%
\pgfpathlineto{\pgfqpoint{3.265847in}{2.643557in}}%
\pgfpathlineto{\pgfqpoint{3.252438in}{2.656061in}}%
\pgfpathlineto{\pgfqpoint{3.239027in}{2.668832in}}%
\pgfpathlineto{\pgfqpoint{3.225612in}{2.681873in}}%
\pgfpathlineto{\pgfqpoint{3.212194in}{2.695185in}}%
\pgfpathlineto{\pgfqpoint{3.204172in}{2.683773in}}%
\pgfpathlineto{\pgfqpoint{3.196142in}{2.672470in}}%
\pgfpathlineto{\pgfqpoint{3.188106in}{2.661275in}}%
\pgfpathlineto{\pgfqpoint{3.180064in}{2.650187in}}%
\pgfpathclose%
\pgfusepath{fill}%
\end{pgfscope}%
\begin{pgfscope}%
\pgfpathrectangle{\pgfqpoint{1.150000in}{0.150000in}}{\pgfqpoint{5.700000in}{5.700000in}}%
\pgfusepath{clip}%
\pgfsetbuttcap%
\pgfsetroundjoin%
\definecolor{currentfill}{rgb}{0.271828,0.209303,0.504434}%
\pgfsetfillcolor{currentfill}%
\pgfsetfillopacity{0.800000}%
\pgfsetlinewidth{0.000000pt}%
\definecolor{currentstroke}{rgb}{0.000000,0.000000,0.000000}%
\pgfsetstrokecolor{currentstroke}%
\pgfsetdash{}{0pt}%
\pgfpathmoveto{\pgfqpoint{4.183157in}{2.618844in}}%
\pgfpathlineto{\pgfqpoint{4.196652in}{2.617051in}}%
\pgfpathlineto{\pgfqpoint{4.210155in}{2.615456in}}%
\pgfpathlineto{\pgfqpoint{4.223665in}{2.614059in}}%
\pgfpathlineto{\pgfqpoint{4.237182in}{2.612860in}}%
\pgfpathlineto{\pgfqpoint{4.244913in}{2.623789in}}%
\pgfpathlineto{\pgfqpoint{4.252639in}{2.634785in}}%
\pgfpathlineto{\pgfqpoint{4.260361in}{2.645853in}}%
\pgfpathlineto{\pgfqpoint{4.268079in}{2.656996in}}%
\pgfpathlineto{\pgfqpoint{4.254570in}{2.658512in}}%
\pgfpathlineto{\pgfqpoint{4.241069in}{2.660225in}}%
\pgfpathlineto{\pgfqpoint{4.227575in}{2.662137in}}%
\pgfpathlineto{\pgfqpoint{4.214089in}{2.664247in}}%
\pgfpathlineto{\pgfqpoint{4.206363in}{2.652776in}}%
\pgfpathlineto{\pgfqpoint{4.198632in}{2.641388in}}%
\pgfpathlineto{\pgfqpoint{4.190897in}{2.630079in}}%
\pgfpathlineto{\pgfqpoint{4.183157in}{2.618844in}}%
\pgfpathclose%
\pgfusepath{fill}%
\end{pgfscope}%
\begin{pgfscope}%
\pgfpathrectangle{\pgfqpoint{1.150000in}{0.150000in}}{\pgfqpoint{5.700000in}{5.700000in}}%
\pgfusepath{clip}%
\pgfsetbuttcap%
\pgfsetroundjoin%
\definecolor{currentfill}{rgb}{0.281412,0.155834,0.469201}%
\pgfsetfillcolor{currentfill}%
\pgfsetfillopacity{0.800000}%
\pgfsetlinewidth{0.000000pt}%
\definecolor{currentstroke}{rgb}{0.000000,0.000000,0.000000}%
\pgfsetstrokecolor{currentstroke}%
\pgfsetdash{}{0pt}%
\pgfpathmoveto{\pgfqpoint{3.789470in}{2.506389in}}%
\pgfpathlineto{\pgfqpoint{3.802880in}{2.501394in}}%
\pgfpathlineto{\pgfqpoint{3.816296in}{2.496615in}}%
\pgfpathlineto{\pgfqpoint{3.829715in}{2.492053in}}%
\pgfpathlineto{\pgfqpoint{3.843139in}{2.487704in}}%
\pgfpathlineto{\pgfqpoint{3.850988in}{2.498902in}}%
\pgfpathlineto{\pgfqpoint{3.858831in}{2.510156in}}%
\pgfpathlineto{\pgfqpoint{3.866669in}{2.521470in}}%
\pgfpathlineto{\pgfqpoint{3.874503in}{2.532846in}}%
\pgfpathlineto{\pgfqpoint{3.861087in}{2.537385in}}%
\pgfpathlineto{\pgfqpoint{3.847676in}{2.542139in}}%
\pgfpathlineto{\pgfqpoint{3.834269in}{2.547108in}}%
\pgfpathlineto{\pgfqpoint{3.820867in}{2.552294in}}%
\pgfpathlineto{\pgfqpoint{3.813025in}{2.540715in}}%
\pgfpathlineto{\pgfqpoint{3.805178in}{2.529207in}}%
\pgfpathlineto{\pgfqpoint{3.797326in}{2.517765in}}%
\pgfpathlineto{\pgfqpoint{3.789470in}{2.506389in}}%
\pgfpathclose%
\pgfusepath{fill}%
\end{pgfscope}%
\begin{pgfscope}%
\pgfpathrectangle{\pgfqpoint{1.150000in}{0.150000in}}{\pgfqpoint{5.700000in}{5.700000in}}%
\pgfusepath{clip}%
\pgfsetbuttcap%
\pgfsetroundjoin%
\definecolor{currentfill}{rgb}{0.220057,0.343307,0.549413}%
\pgfsetfillcolor{currentfill}%
\pgfsetfillopacity{0.800000}%
\pgfsetlinewidth{0.000000pt}%
\definecolor{currentstroke}{rgb}{0.000000,0.000000,0.000000}%
\pgfsetstrokecolor{currentstroke}%
\pgfsetdash{}{0pt}%
\pgfpathmoveto{\pgfqpoint{2.910359in}{2.975687in}}%
\pgfpathlineto{\pgfqpoint{2.923910in}{2.956512in}}%
\pgfpathlineto{\pgfqpoint{2.937453in}{2.937660in}}%
\pgfpathlineto{\pgfqpoint{2.950987in}{2.919126in}}%
\pgfpathlineto{\pgfqpoint{2.964514in}{2.900908in}}%
\pgfpathlineto{\pgfqpoint{2.972615in}{2.912202in}}%
\pgfpathlineto{\pgfqpoint{2.980709in}{2.923636in}}%
\pgfpathlineto{\pgfqpoint{2.988795in}{2.935209in}}%
\pgfpathlineto{\pgfqpoint{2.996874in}{2.946924in}}%
\pgfpathlineto{\pgfqpoint{2.983363in}{2.965205in}}%
\pgfpathlineto{\pgfqpoint{2.969845in}{2.983802in}}%
\pgfpathlineto{\pgfqpoint{2.956318in}{3.002719in}}%
\pgfpathlineto{\pgfqpoint{2.942782in}{3.021956in}}%
\pgfpathlineto{\pgfqpoint{2.934689in}{3.010167in}}%
\pgfpathlineto{\pgfqpoint{2.926587in}{2.998526in}}%
\pgfpathlineto{\pgfqpoint{2.918477in}{2.987033in}}%
\pgfpathlineto{\pgfqpoint{2.910359in}{2.975687in}}%
\pgfpathclose%
\pgfusepath{fill}%
\end{pgfscope}%
\begin{pgfscope}%
\pgfpathrectangle{\pgfqpoint{1.150000in}{0.150000in}}{\pgfqpoint{5.700000in}{5.700000in}}%
\pgfusepath{clip}%
\pgfsetbuttcap%
\pgfsetroundjoin%
\definecolor{currentfill}{rgb}{0.188923,0.410910,0.556326}%
\pgfsetfillcolor{currentfill}%
\pgfsetfillopacity{0.800000}%
\pgfsetlinewidth{0.000000pt}%
\definecolor{currentstroke}{rgb}{0.000000,0.000000,0.000000}%
\pgfsetstrokecolor{currentstroke}%
\pgfsetdash{}{0pt}%
\pgfpathmoveto{\pgfqpoint{5.117547in}{3.117606in}}%
\pgfpathlineto{\pgfqpoint{5.131340in}{3.118780in}}%
\pgfpathlineto{\pgfqpoint{5.145146in}{3.120129in}}%
\pgfpathlineto{\pgfqpoint{5.158963in}{3.121654in}}%
\pgfpathlineto{\pgfqpoint{5.172794in}{3.123354in}}%
\pgfpathlineto{\pgfqpoint{5.180256in}{3.134808in}}%
\pgfpathlineto{\pgfqpoint{5.187720in}{3.146493in}}%
\pgfpathlineto{\pgfqpoint{5.195184in}{3.158414in}}%
\pgfpathlineto{\pgfqpoint{5.202649in}{3.170581in}}%
\pgfpathlineto{\pgfqpoint{5.188839in}{3.169547in}}%
\pgfpathlineto{\pgfqpoint{5.175041in}{3.168688in}}%
\pgfpathlineto{\pgfqpoint{5.161255in}{3.168004in}}%
\pgfpathlineto{\pgfqpoint{5.147481in}{3.167495in}}%
\pgfpathlineto{\pgfqpoint{5.139996in}{3.154651in}}%
\pgfpathlineto{\pgfqpoint{5.132512in}{3.142060in}}%
\pgfpathlineto{\pgfqpoint{5.125029in}{3.129714in}}%
\pgfpathlineto{\pgfqpoint{5.117547in}{3.117606in}}%
\pgfpathclose%
\pgfusepath{fill}%
\end{pgfscope}%
\begin{pgfscope}%
\pgfpathrectangle{\pgfqpoint{1.150000in}{0.150000in}}{\pgfqpoint{5.700000in}{5.700000in}}%
\pgfusepath{clip}%
\pgfsetbuttcap%
\pgfsetroundjoin%
\definecolor{currentfill}{rgb}{0.275191,0.194905,0.496005}%
\pgfsetfillcolor{currentfill}%
\pgfsetfillopacity{0.800000}%
\pgfsetlinewidth{0.000000pt}%
\definecolor{currentstroke}{rgb}{0.000000,0.000000,0.000000}%
\pgfsetstrokecolor{currentstroke}%
\pgfsetdash{}{0pt}%
\pgfpathmoveto{\pgfqpoint{4.098213in}{2.582625in}}%
\pgfpathlineto{\pgfqpoint{4.111688in}{2.580313in}}%
\pgfpathlineto{\pgfqpoint{4.125170in}{2.578204in}}%
\pgfpathlineto{\pgfqpoint{4.138658in}{2.576295in}}%
\pgfpathlineto{\pgfqpoint{4.152154in}{2.574587in}}%
\pgfpathlineto{\pgfqpoint{4.159912in}{2.585557in}}%
\pgfpathlineto{\pgfqpoint{4.167665in}{2.596587in}}%
\pgfpathlineto{\pgfqpoint{4.175413in}{2.607682in}}%
\pgfpathlineto{\pgfqpoint{4.183157in}{2.618844in}}%
\pgfpathlineto{\pgfqpoint{4.169670in}{2.620838in}}%
\pgfpathlineto{\pgfqpoint{4.156190in}{2.623032in}}%
\pgfpathlineto{\pgfqpoint{4.142716in}{2.625427in}}%
\pgfpathlineto{\pgfqpoint{4.129250in}{2.628023in}}%
\pgfpathlineto{\pgfqpoint{4.121498in}{2.616564in}}%
\pgfpathlineto{\pgfqpoint{4.113741in}{2.605180in}}%
\pgfpathlineto{\pgfqpoint{4.105979in}{2.593868in}}%
\pgfpathlineto{\pgfqpoint{4.098213in}{2.582625in}}%
\pgfpathclose%
\pgfusepath{fill}%
\end{pgfscope}%
\begin{pgfscope}%
\pgfpathrectangle{\pgfqpoint{1.150000in}{0.150000in}}{\pgfqpoint{5.700000in}{5.700000in}}%
\pgfusepath{clip}%
\pgfsetbuttcap%
\pgfsetroundjoin%
\definecolor{currentfill}{rgb}{0.281887,0.150881,0.465405}%
\pgfsetfillcolor{currentfill}%
\pgfsetfillopacity{0.800000}%
\pgfsetlinewidth{0.000000pt}%
\definecolor{currentstroke}{rgb}{0.000000,0.000000,0.000000}%
\pgfsetstrokecolor{currentstroke}%
\pgfsetdash{}{0pt}%
\pgfpathmoveto{\pgfqpoint{3.565554in}{2.491238in}}%
\pgfpathlineto{\pgfqpoint{3.578946in}{2.483841in}}%
\pgfpathlineto{\pgfqpoint{3.592340in}{2.476675in}}%
\pgfpathlineto{\pgfqpoint{3.605737in}{2.469738in}}%
\pgfpathlineto{\pgfqpoint{3.619135in}{2.463030in}}%
\pgfpathlineto{\pgfqpoint{3.627052in}{2.474200in}}%
\pgfpathlineto{\pgfqpoint{3.634962in}{2.485434in}}%
\pgfpathlineto{\pgfqpoint{3.642868in}{2.496733in}}%
\pgfpathlineto{\pgfqpoint{3.650768in}{2.508100in}}%
\pgfpathlineto{\pgfqpoint{3.637379in}{2.514936in}}%
\pgfpathlineto{\pgfqpoint{3.623992in}{2.522000in}}%
\pgfpathlineto{\pgfqpoint{3.610607in}{2.529294in}}%
\pgfpathlineto{\pgfqpoint{3.597224in}{2.536819in}}%
\pgfpathlineto{\pgfqpoint{3.589315in}{2.525313in}}%
\pgfpathlineto{\pgfqpoint{3.581400in}{2.513882in}}%
\pgfpathlineto{\pgfqpoint{3.573480in}{2.502524in}}%
\pgfpathlineto{\pgfqpoint{3.565554in}{2.491238in}}%
\pgfpathclose%
\pgfusepath{fill}%
\end{pgfscope}%
\begin{pgfscope}%
\pgfpathrectangle{\pgfqpoint{1.150000in}{0.150000in}}{\pgfqpoint{5.700000in}{5.700000in}}%
\pgfusepath{clip}%
\pgfsetbuttcap%
\pgfsetroundjoin%
\definecolor{currentfill}{rgb}{0.280868,0.160771,0.472899}%
\pgfsetfillcolor{currentfill}%
\pgfsetfillopacity{0.800000}%
\pgfsetlinewidth{0.000000pt}%
\definecolor{currentstroke}{rgb}{0.000000,0.000000,0.000000}%
\pgfsetstrokecolor{currentstroke}%
\pgfsetdash{}{0pt}%
\pgfpathmoveto{\pgfqpoint{3.426625in}{2.513659in}}%
\pgfpathlineto{\pgfqpoint{3.440019in}{2.504460in}}%
\pgfpathlineto{\pgfqpoint{3.453413in}{2.495504in}}%
\pgfpathlineto{\pgfqpoint{3.466808in}{2.486788in}}%
\pgfpathlineto{\pgfqpoint{3.480203in}{2.478312in}}%
\pgfpathlineto{\pgfqpoint{3.488161in}{2.489414in}}%
\pgfpathlineto{\pgfqpoint{3.496114in}{2.500589in}}%
\pgfpathlineto{\pgfqpoint{3.504060in}{2.511838in}}%
\pgfpathlineto{\pgfqpoint{3.512002in}{2.523163in}}%
\pgfpathlineto{\pgfqpoint{3.498617in}{2.531735in}}%
\pgfpathlineto{\pgfqpoint{3.485232in}{2.540547in}}%
\pgfpathlineto{\pgfqpoint{3.471849in}{2.549600in}}%
\pgfpathlineto{\pgfqpoint{3.458466in}{2.558895in}}%
\pgfpathlineto{\pgfqpoint{3.450514in}{2.547462in}}%
\pgfpathlineto{\pgfqpoint{3.442557in}{2.536113in}}%
\pgfpathlineto{\pgfqpoint{3.434594in}{2.524846in}}%
\pgfpathlineto{\pgfqpoint{3.426625in}{2.513659in}}%
\pgfpathclose%
\pgfusepath{fill}%
\end{pgfscope}%
\begin{pgfscope}%
\pgfpathrectangle{\pgfqpoint{1.150000in}{0.150000in}}{\pgfqpoint{5.700000in}{5.700000in}}%
\pgfusepath{clip}%
\pgfsetbuttcap%
\pgfsetroundjoin%
\definecolor{currentfill}{rgb}{0.275191,0.194905,0.496005}%
\pgfsetfillcolor{currentfill}%
\pgfsetfillopacity{0.800000}%
\pgfsetlinewidth{0.000000pt}%
\definecolor{currentstroke}{rgb}{0.000000,0.000000,0.000000}%
\pgfsetstrokecolor{currentstroke}%
\pgfsetdash{}{0pt}%
\pgfpathmoveto{\pgfqpoint{3.233769in}{2.598815in}}%
\pgfpathlineto{\pgfqpoint{3.247188in}{2.586640in}}%
\pgfpathlineto{\pgfqpoint{3.260604in}{2.574729in}}%
\pgfpathlineto{\pgfqpoint{3.274019in}{2.563079in}}%
\pgfpathlineto{\pgfqpoint{3.287432in}{2.551688in}}%
\pgfpathlineto{\pgfqpoint{3.295448in}{2.562671in}}%
\pgfpathlineto{\pgfqpoint{3.303459in}{2.573746in}}%
\pgfpathlineto{\pgfqpoint{3.311463in}{2.584912in}}%
\pgfpathlineto{\pgfqpoint{3.319460in}{2.596172in}}%
\pgfpathlineto{\pgfqpoint{3.306060in}{2.607627in}}%
\pgfpathlineto{\pgfqpoint{3.292658in}{2.619342in}}%
\pgfpathlineto{\pgfqpoint{3.279254in}{2.631318in}}%
\pgfpathlineto{\pgfqpoint{3.265847in}{2.643557in}}%
\pgfpathlineto{\pgfqpoint{3.257837in}{2.632220in}}%
\pgfpathlineto{\pgfqpoint{3.249821in}{2.620985in}}%
\pgfpathlineto{\pgfqpoint{3.241798in}{2.609851in}}%
\pgfpathlineto{\pgfqpoint{3.233769in}{2.598815in}}%
\pgfpathclose%
\pgfusepath{fill}%
\end{pgfscope}%
\begin{pgfscope}%
\pgfpathrectangle{\pgfqpoint{1.150000in}{0.150000in}}{\pgfqpoint{5.700000in}{5.700000in}}%
\pgfusepath{clip}%
\pgfsetbuttcap%
\pgfsetroundjoin%
\definecolor{currentfill}{rgb}{0.180629,0.429975,0.557282}%
\pgfsetfillcolor{currentfill}%
\pgfsetfillopacity{0.800000}%
\pgfsetlinewidth{0.000000pt}%
\definecolor{currentstroke}{rgb}{0.000000,0.000000,0.000000}%
\pgfsetstrokecolor{currentstroke}%
\pgfsetdash{}{0pt}%
\pgfpathmoveto{\pgfqpoint{5.202649in}{3.170581in}}%
\pgfpathlineto{\pgfqpoint{5.216471in}{3.171790in}}%
\pgfpathlineto{\pgfqpoint{5.230306in}{3.173173in}}%
\pgfpathlineto{\pgfqpoint{5.244153in}{3.174730in}}%
\pgfpathlineto{\pgfqpoint{5.258013in}{3.176460in}}%
\pgfpathlineto{\pgfqpoint{5.265458in}{3.188194in}}%
\pgfpathlineto{\pgfqpoint{5.272906in}{3.200181in}}%
\pgfpathlineto{\pgfqpoint{5.280354in}{3.212429in}}%
\pgfpathlineto{\pgfqpoint{5.287806in}{3.224946in}}%
\pgfpathlineto{\pgfqpoint{5.273967in}{3.223913in}}%
\pgfpathlineto{\pgfqpoint{5.260141in}{3.223053in}}%
\pgfpathlineto{\pgfqpoint{5.246328in}{3.222367in}}%
\pgfpathlineto{\pgfqpoint{5.232526in}{3.221855in}}%
\pgfpathlineto{\pgfqpoint{5.225054in}{3.208630in}}%
\pgfpathlineto{\pgfqpoint{5.217584in}{3.195681in}}%
\pgfpathlineto{\pgfqpoint{5.210115in}{3.183001in}}%
\pgfpathlineto{\pgfqpoint{5.202649in}{3.170581in}}%
\pgfpathclose%
\pgfusepath{fill}%
\end{pgfscope}%
\begin{pgfscope}%
\pgfpathrectangle{\pgfqpoint{1.150000in}{0.150000in}}{\pgfqpoint{5.700000in}{5.700000in}}%
\pgfusepath{clip}%
\pgfsetbuttcap%
\pgfsetroundjoin%
\definecolor{currentfill}{rgb}{0.206756,0.371758,0.553117}%
\pgfsetfillcolor{currentfill}%
\pgfsetfillopacity{0.800000}%
\pgfsetlinewidth{0.000000pt}%
\definecolor{currentstroke}{rgb}{0.000000,0.000000,0.000000}%
\pgfsetstrokecolor{currentstroke}%
\pgfsetdash{}{0pt}%
\pgfpathmoveto{\pgfqpoint{2.856060in}{3.055666in}}%
\pgfpathlineto{\pgfqpoint{2.869650in}{3.035173in}}%
\pgfpathlineto{\pgfqpoint{2.883229in}{3.015014in}}%
\pgfpathlineto{\pgfqpoint{2.896798in}{2.995187in}}%
\pgfpathlineto{\pgfqpoint{2.910359in}{2.975687in}}%
\pgfpathlineto{\pgfqpoint{2.918477in}{2.987033in}}%
\pgfpathlineto{\pgfqpoint{2.926587in}{2.998526in}}%
\pgfpathlineto{\pgfqpoint{2.934689in}{3.010167in}}%
\pgfpathlineto{\pgfqpoint{2.942782in}{3.021956in}}%
\pgfpathlineto{\pgfqpoint{2.929238in}{3.041519in}}%
\pgfpathlineto{\pgfqpoint{2.915685in}{3.061410in}}%
\pgfpathlineto{\pgfqpoint{2.902122in}{3.081632in}}%
\pgfpathlineto{\pgfqpoint{2.888550in}{3.102188in}}%
\pgfpathlineto{\pgfqpoint{2.880440in}{3.090323in}}%
\pgfpathlineto{\pgfqpoint{2.872322in}{3.078615in}}%
\pgfpathlineto{\pgfqpoint{2.864195in}{3.067063in}}%
\pgfpathlineto{\pgfqpoint{2.856060in}{3.055666in}}%
\pgfpathclose%
\pgfusepath{fill}%
\end{pgfscope}%
\begin{pgfscope}%
\pgfpathrectangle{\pgfqpoint{1.150000in}{0.150000in}}{\pgfqpoint{5.700000in}{5.700000in}}%
\pgfusepath{clip}%
\pgfsetbuttcap%
\pgfsetroundjoin%
\definecolor{currentfill}{rgb}{0.278012,0.180367,0.486697}%
\pgfsetfillcolor{currentfill}%
\pgfsetfillopacity{0.800000}%
\pgfsetlinewidth{0.000000pt}%
\definecolor{currentstroke}{rgb}{0.000000,0.000000,0.000000}%
\pgfsetstrokecolor{currentstroke}%
\pgfsetdash{}{0pt}%
\pgfpathmoveto{\pgfqpoint{4.013236in}{2.548540in}}%
\pgfpathlineto{\pgfqpoint{4.026693in}{2.545666in}}%
\pgfpathlineto{\pgfqpoint{4.040156in}{2.542998in}}%
\pgfpathlineto{\pgfqpoint{4.053626in}{2.540534in}}%
\pgfpathlineto{\pgfqpoint{4.067103in}{2.538273in}}%
\pgfpathlineto{\pgfqpoint{4.074887in}{2.549275in}}%
\pgfpathlineto{\pgfqpoint{4.082667in}{2.560331in}}%
\pgfpathlineto{\pgfqpoint{4.090442in}{2.571447in}}%
\pgfpathlineto{\pgfqpoint{4.098213in}{2.582625in}}%
\pgfpathlineto{\pgfqpoint{4.084745in}{2.585140in}}%
\pgfpathlineto{\pgfqpoint{4.071284in}{2.587858in}}%
\pgfpathlineto{\pgfqpoint{4.057829in}{2.590780in}}%
\pgfpathlineto{\pgfqpoint{4.044380in}{2.593907in}}%
\pgfpathlineto{\pgfqpoint{4.036601in}{2.582464in}}%
\pgfpathlineto{\pgfqpoint{4.028817in}{2.571091in}}%
\pgfpathlineto{\pgfqpoint{4.021029in}{2.559783in}}%
\pgfpathlineto{\pgfqpoint{4.013236in}{2.548540in}}%
\pgfpathclose%
\pgfusepath{fill}%
\end{pgfscope}%
\begin{pgfscope}%
\pgfpathrectangle{\pgfqpoint{1.150000in}{0.150000in}}{\pgfqpoint{5.700000in}{5.700000in}}%
\pgfusepath{clip}%
\pgfsetbuttcap%
\pgfsetroundjoin%
\definecolor{currentfill}{rgb}{0.281887,0.150881,0.465405}%
\pgfsetfillcolor{currentfill}%
\pgfsetfillopacity{0.800000}%
\pgfsetlinewidth{0.000000pt}%
\definecolor{currentstroke}{rgb}{0.000000,0.000000,0.000000}%
\pgfsetstrokecolor{currentstroke}%
\pgfsetdash{}{0pt}%
\pgfpathmoveto{\pgfqpoint{3.704353in}{2.483014in}}%
\pgfpathlineto{\pgfqpoint{3.717757in}{2.477301in}}%
\pgfpathlineto{\pgfqpoint{3.731165in}{2.471809in}}%
\pgfpathlineto{\pgfqpoint{3.744577in}{2.466537in}}%
\pgfpathlineto{\pgfqpoint{3.757992in}{2.461484in}}%
\pgfpathlineto{\pgfqpoint{3.765869in}{2.472624in}}%
\pgfpathlineto{\pgfqpoint{3.773741in}{2.483820in}}%
\pgfpathlineto{\pgfqpoint{3.781608in}{2.495074in}}%
\pgfpathlineto{\pgfqpoint{3.789470in}{2.506389in}}%
\pgfpathlineto{\pgfqpoint{3.776063in}{2.511601in}}%
\pgfpathlineto{\pgfqpoint{3.762660in}{2.517033in}}%
\pgfpathlineto{\pgfqpoint{3.749261in}{2.522684in}}%
\pgfpathlineto{\pgfqpoint{3.735866in}{2.528557in}}%
\pgfpathlineto{\pgfqpoint{3.727995in}{2.517071in}}%
\pgfpathlineto{\pgfqpoint{3.720120in}{2.505654in}}%
\pgfpathlineto{\pgfqpoint{3.712239in}{2.494302in}}%
\pgfpathlineto{\pgfqpoint{3.704353in}{2.483014in}}%
\pgfpathclose%
\pgfusepath{fill}%
\end{pgfscope}%
\begin{pgfscope}%
\pgfpathrectangle{\pgfqpoint{1.150000in}{0.150000in}}{\pgfqpoint{5.700000in}{5.700000in}}%
\pgfusepath{clip}%
\pgfsetbuttcap%
\pgfsetroundjoin%
\definecolor{currentfill}{rgb}{0.172719,0.448791,0.557885}%
\pgfsetfillcolor{currentfill}%
\pgfsetfillopacity{0.800000}%
\pgfsetlinewidth{0.000000pt}%
\definecolor{currentstroke}{rgb}{0.000000,0.000000,0.000000}%
\pgfsetstrokecolor{currentstroke}%
\pgfsetdash{}{0pt}%
\pgfpathmoveto{\pgfqpoint{5.287806in}{3.224946in}}%
\pgfpathlineto{\pgfqpoint{5.301656in}{3.226152in}}%
\pgfpathlineto{\pgfqpoint{5.315520in}{3.227531in}}%
\pgfpathlineto{\pgfqpoint{5.329396in}{3.229082in}}%
\pgfpathlineto{\pgfqpoint{5.343285in}{3.230806in}}%
\pgfpathlineto{\pgfqpoint{5.350716in}{3.242882in}}%
\pgfpathlineto{\pgfqpoint{5.358150in}{3.255236in}}%
\pgfpathlineto{\pgfqpoint{5.365586in}{3.267876in}}%
\pgfpathlineto{\pgfqpoint{5.373026in}{3.280811in}}%
\pgfpathlineto{\pgfqpoint{5.359160in}{3.279817in}}%
\pgfpathlineto{\pgfqpoint{5.345307in}{3.278994in}}%
\pgfpathlineto{\pgfqpoint{5.331466in}{3.278344in}}%
\pgfpathlineto{\pgfqpoint{5.317637in}{3.277866in}}%
\pgfpathlineto{\pgfqpoint{5.310175in}{3.264192in}}%
\pgfpathlineto{\pgfqpoint{5.302715in}{3.250819in}}%
\pgfpathlineto{\pgfqpoint{5.295259in}{3.237740in}}%
\pgfpathlineto{\pgfqpoint{5.287806in}{3.224946in}}%
\pgfpathclose%
\pgfusepath{fill}%
\end{pgfscope}%
\begin{pgfscope}%
\pgfpathrectangle{\pgfqpoint{1.150000in}{0.150000in}}{\pgfqpoint{5.700000in}{5.700000in}}%
\pgfusepath{clip}%
\pgfsetbuttcap%
\pgfsetroundjoin%
\definecolor{currentfill}{rgb}{0.278826,0.175490,0.483397}%
\pgfsetfillcolor{currentfill}%
\pgfsetfillopacity{0.800000}%
\pgfsetlinewidth{0.000000pt}%
\definecolor{currentstroke}{rgb}{0.000000,0.000000,0.000000}%
\pgfsetstrokecolor{currentstroke}%
\pgfsetdash{}{0pt}%
\pgfpathmoveto{\pgfqpoint{3.287432in}{2.551688in}}%
\pgfpathlineto{\pgfqpoint{3.300843in}{2.540555in}}%
\pgfpathlineto{\pgfqpoint{3.314252in}{2.529679in}}%
\pgfpathlineto{\pgfqpoint{3.327660in}{2.519056in}}%
\pgfpathlineto{\pgfqpoint{3.341068in}{2.508685in}}%
\pgfpathlineto{\pgfqpoint{3.349072in}{2.519615in}}%
\pgfpathlineto{\pgfqpoint{3.357070in}{2.530629in}}%
\pgfpathlineto{\pgfqpoint{3.365062in}{2.541727in}}%
\pgfpathlineto{\pgfqpoint{3.373048in}{2.552911in}}%
\pgfpathlineto{\pgfqpoint{3.359653in}{2.563346in}}%
\pgfpathlineto{\pgfqpoint{3.346257in}{2.574034in}}%
\pgfpathlineto{\pgfqpoint{3.332859in}{2.584975in}}%
\pgfpathlineto{\pgfqpoint{3.319460in}{2.596172in}}%
\pgfpathlineto{\pgfqpoint{3.311463in}{2.584912in}}%
\pgfpathlineto{\pgfqpoint{3.303459in}{2.573746in}}%
\pgfpathlineto{\pgfqpoint{3.295448in}{2.562671in}}%
\pgfpathlineto{\pgfqpoint{3.287432in}{2.551688in}}%
\pgfpathclose%
\pgfusepath{fill}%
\end{pgfscope}%
\begin{pgfscope}%
\pgfpathrectangle{\pgfqpoint{1.150000in}{0.150000in}}{\pgfqpoint{5.700000in}{5.700000in}}%
\pgfusepath{clip}%
\pgfsetbuttcap%
\pgfsetroundjoin%
\definecolor{currentfill}{rgb}{0.279574,0.170599,0.479997}%
\pgfsetfillcolor{currentfill}%
\pgfsetfillopacity{0.800000}%
\pgfsetlinewidth{0.000000pt}%
\definecolor{currentstroke}{rgb}{0.000000,0.000000,0.000000}%
\pgfsetstrokecolor{currentstroke}%
\pgfsetdash{}{0pt}%
\pgfpathmoveto{\pgfqpoint{3.928216in}{2.516816in}}%
\pgfpathlineto{\pgfqpoint{3.941657in}{2.513335in}}%
\pgfpathlineto{\pgfqpoint{3.955105in}{2.510062in}}%
\pgfpathlineto{\pgfqpoint{3.968558in}{2.506998in}}%
\pgfpathlineto{\pgfqpoint{3.982017in}{2.504141in}}%
\pgfpathlineto{\pgfqpoint{3.989829in}{2.515160in}}%
\pgfpathlineto{\pgfqpoint{3.997636in}{2.526231in}}%
\pgfpathlineto{\pgfqpoint{4.005439in}{2.537357in}}%
\pgfpathlineto{\pgfqpoint{4.013236in}{2.548540in}}%
\pgfpathlineto{\pgfqpoint{3.999785in}{2.551620in}}%
\pgfpathlineto{\pgfqpoint{3.986341in}{2.554907in}}%
\pgfpathlineto{\pgfqpoint{3.972902in}{2.558401in}}%
\pgfpathlineto{\pgfqpoint{3.959468in}{2.562105in}}%
\pgfpathlineto{\pgfqpoint{3.951662in}{2.550687in}}%
\pgfpathlineto{\pgfqpoint{3.943852in}{2.539335in}}%
\pgfpathlineto{\pgfqpoint{3.936036in}{2.528046in}}%
\pgfpathlineto{\pgfqpoint{3.928216in}{2.516816in}}%
\pgfpathclose%
\pgfusepath{fill}%
\end{pgfscope}%
\begin{pgfscope}%
\pgfpathrectangle{\pgfqpoint{1.150000in}{0.150000in}}{\pgfqpoint{5.700000in}{5.700000in}}%
\pgfusepath{clip}%
\pgfsetbuttcap%
\pgfsetroundjoin%
\definecolor{currentfill}{rgb}{0.190631,0.407061,0.556089}%
\pgfsetfillcolor{currentfill}%
\pgfsetfillopacity{0.800000}%
\pgfsetlinewidth{0.000000pt}%
\definecolor{currentstroke}{rgb}{0.000000,0.000000,0.000000}%
\pgfsetstrokecolor{currentstroke}%
\pgfsetdash{}{0pt}%
\pgfpathmoveto{\pgfqpoint{2.801598in}{3.141049in}}%
\pgfpathlineto{\pgfqpoint{2.815230in}{3.119185in}}%
\pgfpathlineto{\pgfqpoint{2.828851in}{3.097669in}}%
\pgfpathlineto{\pgfqpoint{2.842461in}{3.076497in}}%
\pgfpathlineto{\pgfqpoint{2.856060in}{3.055666in}}%
\pgfpathlineto{\pgfqpoint{2.864195in}{3.067063in}}%
\pgfpathlineto{\pgfqpoint{2.872322in}{3.078615in}}%
\pgfpathlineto{\pgfqpoint{2.880440in}{3.090323in}}%
\pgfpathlineto{\pgfqpoint{2.888550in}{3.102188in}}%
\pgfpathlineto{\pgfqpoint{2.874967in}{3.123082in}}%
\pgfpathlineto{\pgfqpoint{2.861375in}{3.144317in}}%
\pgfpathlineto{\pgfqpoint{2.847771in}{3.165896in}}%
\pgfpathlineto{\pgfqpoint{2.834156in}{3.187823in}}%
\pgfpathlineto{\pgfqpoint{2.826030in}{3.175883in}}%
\pgfpathlineto{\pgfqpoint{2.817894in}{3.164108in}}%
\pgfpathlineto{\pgfqpoint{2.809751in}{3.152497in}}%
\pgfpathlineto{\pgfqpoint{2.801598in}{3.141049in}}%
\pgfpathclose%
\pgfusepath{fill}%
\end{pgfscope}%
\begin{pgfscope}%
\pgfpathrectangle{\pgfqpoint{1.150000in}{0.150000in}}{\pgfqpoint{5.700000in}{5.700000in}}%
\pgfusepath{clip}%
\pgfsetbuttcap%
\pgfsetroundjoin%
\definecolor{currentfill}{rgb}{0.165117,0.467423,0.558141}%
\pgfsetfillcolor{currentfill}%
\pgfsetfillopacity{0.800000}%
\pgfsetlinewidth{0.000000pt}%
\definecolor{currentstroke}{rgb}{0.000000,0.000000,0.000000}%
\pgfsetstrokecolor{currentstroke}%
\pgfsetdash{}{0pt}%
\pgfpathmoveto{\pgfqpoint{5.373026in}{3.280811in}}%
\pgfpathlineto{\pgfqpoint{5.386905in}{3.281977in}}%
\pgfpathlineto{\pgfqpoint{5.400797in}{3.283315in}}%
\pgfpathlineto{\pgfqpoint{5.414701in}{3.284824in}}%
\pgfpathlineto{\pgfqpoint{5.428619in}{3.286504in}}%
\pgfpathlineto{\pgfqpoint{5.436038in}{3.298992in}}%
\pgfpathlineto{\pgfqpoint{5.443462in}{3.311784in}}%
\pgfpathlineto{\pgfqpoint{5.450889in}{3.324888in}}%
\pgfpathlineto{\pgfqpoint{5.436990in}{3.323776in}}%
\pgfpathlineto{\pgfqpoint{5.423104in}{3.322835in}}%
\pgfpathlineto{\pgfqpoint{5.409230in}{3.322065in}}%
\pgfpathlineto{\pgfqpoint{5.395369in}{3.321466in}}%
\pgfpathlineto{\pgfqpoint{5.387917in}{3.307597in}}%
\pgfpathlineto{\pgfqpoint{5.380470in}{3.294048in}}%
\pgfpathlineto{\pgfqpoint{5.373026in}{3.280811in}}%
\pgfpathclose%
\pgfusepath{fill}%
\end{pgfscope}%
\begin{pgfscope}%
\pgfpathrectangle{\pgfqpoint{1.150000in}{0.150000in}}{\pgfqpoint{5.700000in}{5.700000in}}%
\pgfusepath{clip}%
\pgfsetbuttcap%
\pgfsetroundjoin%
\definecolor{currentfill}{rgb}{0.281887,0.150881,0.465405}%
\pgfsetfillcolor{currentfill}%
\pgfsetfillopacity{0.800000}%
\pgfsetlinewidth{0.000000pt}%
\definecolor{currentstroke}{rgb}{0.000000,0.000000,0.000000}%
\pgfsetstrokecolor{currentstroke}%
\pgfsetdash{}{0pt}%
\pgfpathmoveto{\pgfqpoint{3.480203in}{2.478312in}}%
\pgfpathlineto{\pgfqpoint{3.493600in}{2.470074in}}%
\pgfpathlineto{\pgfqpoint{3.506997in}{2.462073in}}%
\pgfpathlineto{\pgfqpoint{3.520396in}{2.454306in}}%
\pgfpathlineto{\pgfqpoint{3.533797in}{2.446773in}}%
\pgfpathlineto{\pgfqpoint{3.541744in}{2.457790in}}%
\pgfpathlineto{\pgfqpoint{3.549686in}{2.468873in}}%
\pgfpathlineto{\pgfqpoint{3.557623in}{2.480021in}}%
\pgfpathlineto{\pgfqpoint{3.565554in}{2.491238in}}%
\pgfpathlineto{\pgfqpoint{3.552164in}{2.498867in}}%
\pgfpathlineto{\pgfqpoint{3.538775in}{2.506730in}}%
\pgfpathlineto{\pgfqpoint{3.525388in}{2.514828in}}%
\pgfpathlineto{\pgfqpoint{3.512002in}{2.523163in}}%
\pgfpathlineto{\pgfqpoint{3.504060in}{2.511838in}}%
\pgfpathlineto{\pgfqpoint{3.496114in}{2.500589in}}%
\pgfpathlineto{\pgfqpoint{3.488161in}{2.489414in}}%
\pgfpathlineto{\pgfqpoint{3.480203in}{2.478312in}}%
\pgfpathclose%
\pgfusepath{fill}%
\end{pgfscope}%
\begin{pgfscope}%
\pgfpathrectangle{\pgfqpoint{1.150000in}{0.150000in}}{\pgfqpoint{5.700000in}{5.700000in}}%
\pgfusepath{clip}%
\pgfsetbuttcap%
\pgfsetroundjoin%
\definecolor{currentfill}{rgb}{0.282290,0.145912,0.461510}%
\pgfsetfillcolor{currentfill}%
\pgfsetfillopacity{0.800000}%
\pgfsetlinewidth{0.000000pt}%
\definecolor{currentstroke}{rgb}{0.000000,0.000000,0.000000}%
\pgfsetstrokecolor{currentstroke}%
\pgfsetdash{}{0pt}%
\pgfpathmoveto{\pgfqpoint{3.619135in}{2.463030in}}%
\pgfpathlineto{\pgfqpoint{3.632537in}{2.456549in}}%
\pgfpathlineto{\pgfqpoint{3.645941in}{2.450294in}}%
\pgfpathlineto{\pgfqpoint{3.659348in}{2.444264in}}%
\pgfpathlineto{\pgfqpoint{3.672758in}{2.438457in}}%
\pgfpathlineto{\pgfqpoint{3.680665in}{2.449510in}}%
\pgfpathlineto{\pgfqpoint{3.688566in}{2.460620in}}%
\pgfpathlineto{\pgfqpoint{3.696462in}{2.471787in}}%
\pgfpathlineto{\pgfqpoint{3.704353in}{2.483014in}}%
\pgfpathlineto{\pgfqpoint{3.690952in}{2.488949in}}%
\pgfpathlineto{\pgfqpoint{3.677554in}{2.495108in}}%
\pgfpathlineto{\pgfqpoint{3.664160in}{2.501491in}}%
\pgfpathlineto{\pgfqpoint{3.650768in}{2.508100in}}%
\pgfpathlineto{\pgfqpoint{3.642868in}{2.496733in}}%
\pgfpathlineto{\pgfqpoint{3.634962in}{2.485434in}}%
\pgfpathlineto{\pgfqpoint{3.627052in}{2.474200in}}%
\pgfpathlineto{\pgfqpoint{3.619135in}{2.463030in}}%
\pgfpathclose%
\pgfusepath{fill}%
\end{pgfscope}%
\begin{pgfscope}%
\pgfpathrectangle{\pgfqpoint{1.150000in}{0.150000in}}{\pgfqpoint{5.700000in}{5.700000in}}%
\pgfusepath{clip}%
\pgfsetbuttcap%
\pgfsetroundjoin%
\definecolor{currentfill}{rgb}{0.281412,0.155834,0.469201}%
\pgfsetfillcolor{currentfill}%
\pgfsetfillopacity{0.800000}%
\pgfsetlinewidth{0.000000pt}%
\definecolor{currentstroke}{rgb}{0.000000,0.000000,0.000000}%
\pgfsetstrokecolor{currentstroke}%
\pgfsetdash{}{0pt}%
\pgfpathmoveto{\pgfqpoint{3.843139in}{2.487704in}}%
\pgfpathlineto{\pgfqpoint{3.856568in}{2.483570in}}%
\pgfpathlineto{\pgfqpoint{3.870002in}{2.479648in}}%
\pgfpathlineto{\pgfqpoint{3.883441in}{2.475938in}}%
\pgfpathlineto{\pgfqpoint{3.896886in}{2.472438in}}%
\pgfpathlineto{\pgfqpoint{3.904726in}{2.483456in}}%
\pgfpathlineto{\pgfqpoint{3.912561in}{2.494523in}}%
\pgfpathlineto{\pgfqpoint{3.920391in}{2.505642in}}%
\pgfpathlineto{\pgfqpoint{3.928216in}{2.516816in}}%
\pgfpathlineto{\pgfqpoint{3.914780in}{2.520506in}}%
\pgfpathlineto{\pgfqpoint{3.901349in}{2.524408in}}%
\pgfpathlineto{\pgfqpoint{3.887923in}{2.528521in}}%
\pgfpathlineto{\pgfqpoint{3.874503in}{2.532846in}}%
\pgfpathlineto{\pgfqpoint{3.866669in}{2.521470in}}%
\pgfpathlineto{\pgfqpoint{3.858831in}{2.510156in}}%
\pgfpathlineto{\pgfqpoint{3.850988in}{2.498902in}}%
\pgfpathlineto{\pgfqpoint{3.843139in}{2.487704in}}%
\pgfpathclose%
\pgfusepath{fill}%
\end{pgfscope}%
\begin{pgfscope}%
\pgfpathrectangle{\pgfqpoint{1.150000in}{0.150000in}}{\pgfqpoint{5.700000in}{5.700000in}}%
\pgfusepath{clip}%
\pgfsetbuttcap%
\pgfsetroundjoin%
\definecolor{currentfill}{rgb}{0.244972,0.287675,0.537260}%
\pgfsetfillcolor{currentfill}%
\pgfsetfillopacity{0.800000}%
\pgfsetlinewidth{0.000000pt}%
\definecolor{currentstroke}{rgb}{0.000000,0.000000,0.000000}%
\pgfsetstrokecolor{currentstroke}%
\pgfsetdash{}{0pt}%
\pgfpathmoveto{\pgfqpoint{4.577211in}{2.781914in}}%
\pgfpathlineto{\pgfqpoint{4.590842in}{2.782516in}}%
\pgfpathlineto{\pgfqpoint{4.604484in}{2.783304in}}%
\pgfpathlineto{\pgfqpoint{4.618135in}{2.784279in}}%
\pgfpathlineto{\pgfqpoint{4.631797in}{2.785439in}}%
\pgfpathlineto{\pgfqpoint{4.639411in}{2.795793in}}%
\pgfpathlineto{\pgfqpoint{4.647020in}{2.806247in}}%
\pgfpathlineto{\pgfqpoint{4.654626in}{2.816806in}}%
\pgfpathlineto{\pgfqpoint{4.662228in}{2.827474in}}%
\pgfpathlineto{\pgfqpoint{4.648579in}{2.826759in}}%
\pgfpathlineto{\pgfqpoint{4.634939in}{2.826229in}}%
\pgfpathlineto{\pgfqpoint{4.621310in}{2.825885in}}%
\pgfpathlineto{\pgfqpoint{4.607690in}{2.825727in}}%
\pgfpathlineto{\pgfqpoint{4.600076in}{2.814603in}}%
\pgfpathlineto{\pgfqpoint{4.592458in}{2.803596in}}%
\pgfpathlineto{\pgfqpoint{4.584836in}{2.792701in}}%
\pgfpathlineto{\pgfqpoint{4.577211in}{2.781914in}}%
\pgfpathclose%
\pgfusepath{fill}%
\end{pgfscope}%
\begin{pgfscope}%
\pgfpathrectangle{\pgfqpoint{1.150000in}{0.150000in}}{\pgfqpoint{5.700000in}{5.700000in}}%
\pgfusepath{clip}%
\pgfsetbuttcap%
\pgfsetroundjoin%
\definecolor{currentfill}{rgb}{0.252194,0.269783,0.531579}%
\pgfsetfillcolor{currentfill}%
\pgfsetfillopacity{0.800000}%
\pgfsetlinewidth{0.000000pt}%
\definecolor{currentstroke}{rgb}{0.000000,0.000000,0.000000}%
\pgfsetstrokecolor{currentstroke}%
\pgfsetdash{}{0pt}%
\pgfpathmoveto{\pgfqpoint{4.492204in}{2.737557in}}%
\pgfpathlineto{\pgfqpoint{4.505806in}{2.737820in}}%
\pgfpathlineto{\pgfqpoint{4.519419in}{2.738271in}}%
\pgfpathlineto{\pgfqpoint{4.533042in}{2.738911in}}%
\pgfpathlineto{\pgfqpoint{4.546674in}{2.739738in}}%
\pgfpathlineto{\pgfqpoint{4.554314in}{2.750146in}}%
\pgfpathlineto{\pgfqpoint{4.561950in}{2.760641in}}%
\pgfpathlineto{\pgfqpoint{4.569583in}{2.771229in}}%
\pgfpathlineto{\pgfqpoint{4.577211in}{2.781914in}}%
\pgfpathlineto{\pgfqpoint{4.563590in}{2.781499in}}%
\pgfpathlineto{\pgfqpoint{4.549979in}{2.781273in}}%
\pgfpathlineto{\pgfqpoint{4.536377in}{2.781234in}}%
\pgfpathlineto{\pgfqpoint{4.522785in}{2.781384in}}%
\pgfpathlineto{\pgfqpoint{4.515145in}{2.770274in}}%
\pgfpathlineto{\pgfqpoint{4.507502in}{2.759270in}}%
\pgfpathlineto{\pgfqpoint{4.499855in}{2.748366in}}%
\pgfpathlineto{\pgfqpoint{4.492204in}{2.737557in}}%
\pgfpathclose%
\pgfusepath{fill}%
\end{pgfscope}%
\begin{pgfscope}%
\pgfpathrectangle{\pgfqpoint{1.150000in}{0.150000in}}{\pgfqpoint{5.700000in}{5.700000in}}%
\pgfusepath{clip}%
\pgfsetbuttcap%
\pgfsetroundjoin%
\definecolor{currentfill}{rgb}{0.237441,0.305202,0.541921}%
\pgfsetfillcolor{currentfill}%
\pgfsetfillopacity{0.800000}%
\pgfsetlinewidth{0.000000pt}%
\definecolor{currentstroke}{rgb}{0.000000,0.000000,0.000000}%
\pgfsetstrokecolor{currentstroke}%
\pgfsetdash{}{0pt}%
\pgfpathmoveto{\pgfqpoint{4.662228in}{2.827474in}}%
\pgfpathlineto{\pgfqpoint{4.675889in}{2.828375in}}%
\pgfpathlineto{\pgfqpoint{4.689560in}{2.829461in}}%
\pgfpathlineto{\pgfqpoint{4.703241in}{2.830730in}}%
\pgfpathlineto{\pgfqpoint{4.716934in}{2.832184in}}%
\pgfpathlineto{\pgfqpoint{4.724520in}{2.842503in}}%
\pgfpathlineto{\pgfqpoint{4.732103in}{2.852935in}}%
\pgfpathlineto{\pgfqpoint{4.739683in}{2.863485in}}%
\pgfpathlineto{\pgfqpoint{4.747260in}{2.874160in}}%
\pgfpathlineto{\pgfqpoint{4.733581in}{2.873183in}}%
\pgfpathlineto{\pgfqpoint{4.719912in}{2.872390in}}%
\pgfpathlineto{\pgfqpoint{4.706254in}{2.871780in}}%
\pgfpathlineto{\pgfqpoint{4.692607in}{2.871355in}}%
\pgfpathlineto{\pgfqpoint{4.685017in}{2.860193in}}%
\pgfpathlineto{\pgfqpoint{4.677424in}{2.849163in}}%
\pgfpathlineto{\pgfqpoint{4.669828in}{2.838258in}}%
\pgfpathlineto{\pgfqpoint{4.662228in}{2.827474in}}%
\pgfpathclose%
\pgfusepath{fill}%
\end{pgfscope}%
\begin{pgfscope}%
\pgfpathrectangle{\pgfqpoint{1.150000in}{0.150000in}}{\pgfqpoint{5.700000in}{5.700000in}}%
\pgfusepath{clip}%
\pgfsetbuttcap%
\pgfsetroundjoin%
\definecolor{currentfill}{rgb}{0.227802,0.326594,0.546532}%
\pgfsetfillcolor{currentfill}%
\pgfsetfillopacity{0.800000}%
\pgfsetlinewidth{0.000000pt}%
\definecolor{currentstroke}{rgb}{0.000000,0.000000,0.000000}%
\pgfsetstrokecolor{currentstroke}%
\pgfsetdash{}{0pt}%
\pgfpathmoveto{\pgfqpoint{4.747260in}{2.874160in}}%
\pgfpathlineto{\pgfqpoint{4.760950in}{2.875320in}}%
\pgfpathlineto{\pgfqpoint{4.774651in}{2.876662in}}%
\pgfpathlineto{\pgfqpoint{4.788363in}{2.878188in}}%
\pgfpathlineto{\pgfqpoint{4.802087in}{2.879895in}}%
\pgfpathlineto{\pgfqpoint{4.809647in}{2.890202in}}%
\pgfpathlineto{\pgfqpoint{4.817204in}{2.900638in}}%
\pgfpathlineto{\pgfqpoint{4.824758in}{2.911206in}}%
\pgfpathlineto{\pgfqpoint{4.832310in}{2.921914in}}%
\pgfpathlineto{\pgfqpoint{4.818601in}{2.920716in}}%
\pgfpathlineto{\pgfqpoint{4.804903in}{2.919699in}}%
\pgfpathlineto{\pgfqpoint{4.791216in}{2.918864in}}%
\pgfpathlineto{\pgfqpoint{4.777540in}{2.918211in}}%
\pgfpathlineto{\pgfqpoint{4.769974in}{2.906984in}}%
\pgfpathlineto{\pgfqpoint{4.762405in}{2.895903in}}%
\pgfpathlineto{\pgfqpoint{4.754834in}{2.884964in}}%
\pgfpathlineto{\pgfqpoint{4.747260in}{2.874160in}}%
\pgfpathclose%
\pgfusepath{fill}%
\end{pgfscope}%
\begin{pgfscope}%
\pgfpathrectangle{\pgfqpoint{1.150000in}{0.150000in}}{\pgfqpoint{5.700000in}{5.700000in}}%
\pgfusepath{clip}%
\pgfsetbuttcap%
\pgfsetroundjoin%
\definecolor{currentfill}{rgb}{0.258965,0.251537,0.524736}%
\pgfsetfillcolor{currentfill}%
\pgfsetfillopacity{0.800000}%
\pgfsetlinewidth{0.000000pt}%
\definecolor{currentstroke}{rgb}{0.000000,0.000000,0.000000}%
\pgfsetstrokecolor{currentstroke}%
\pgfsetdash{}{0pt}%
\pgfpathmoveto{\pgfqpoint{4.407200in}{2.694508in}}%
\pgfpathlineto{\pgfqpoint{4.420776in}{2.694390in}}%
\pgfpathlineto{\pgfqpoint{4.434361in}{2.694464in}}%
\pgfpathlineto{\pgfqpoint{4.447955in}{2.694728in}}%
\pgfpathlineto{\pgfqpoint{4.461559in}{2.695182in}}%
\pgfpathlineto{\pgfqpoint{4.469226in}{2.705656in}}%
\pgfpathlineto{\pgfqpoint{4.476889in}{2.716207in}}%
\pgfpathlineto{\pgfqpoint{4.484548in}{2.726839in}}%
\pgfpathlineto{\pgfqpoint{4.492204in}{2.737557in}}%
\pgfpathlineto{\pgfqpoint{4.478610in}{2.737484in}}%
\pgfpathlineto{\pgfqpoint{4.465026in}{2.737601in}}%
\pgfpathlineto{\pgfqpoint{4.451451in}{2.737908in}}%
\pgfpathlineto{\pgfqpoint{4.437886in}{2.738406in}}%
\pgfpathlineto{\pgfqpoint{4.430220in}{2.727296in}}%
\pgfpathlineto{\pgfqpoint{4.422551in}{2.716279in}}%
\pgfpathlineto{\pgfqpoint{4.414878in}{2.705351in}}%
\pgfpathlineto{\pgfqpoint{4.407200in}{2.694508in}}%
\pgfpathclose%
\pgfusepath{fill}%
\end{pgfscope}%
\begin{pgfscope}%
\pgfpathrectangle{\pgfqpoint{1.150000in}{0.150000in}}{\pgfqpoint{5.700000in}{5.700000in}}%
\pgfusepath{clip}%
\pgfsetbuttcap%
\pgfsetroundjoin%
\definecolor{currentfill}{rgb}{0.258965,0.251537,0.524736}%
\pgfsetfillcolor{currentfill}%
\pgfsetfillopacity{0.800000}%
\pgfsetlinewidth{0.000000pt}%
\definecolor{currentstroke}{rgb}{0.000000,0.000000,0.000000}%
\pgfsetstrokecolor{currentstroke}%
\pgfsetdash{}{0pt}%
\pgfpathmoveto{\pgfqpoint{3.040111in}{2.722643in}}%
\pgfpathlineto{\pgfqpoint{3.053592in}{2.707177in}}%
\pgfpathlineto{\pgfqpoint{3.067069in}{2.692000in}}%
\pgfpathlineto{\pgfqpoint{3.080540in}{2.677110in}}%
\pgfpathlineto{\pgfqpoint{3.094006in}{2.662505in}}%
\pgfpathlineto{\pgfqpoint{3.102090in}{2.673195in}}%
\pgfpathlineto{\pgfqpoint{3.110168in}{2.683996in}}%
\pgfpathlineto{\pgfqpoint{3.118238in}{2.694909in}}%
\pgfpathlineto{\pgfqpoint{3.126302in}{2.705936in}}%
\pgfpathlineto{\pgfqpoint{3.112851in}{2.720573in}}%
\pgfpathlineto{\pgfqpoint{3.099395in}{2.735495in}}%
\pgfpathlineto{\pgfqpoint{3.085934in}{2.750704in}}%
\pgfpathlineto{\pgfqpoint{3.072468in}{2.766202in}}%
\pgfpathlineto{\pgfqpoint{3.064389in}{2.755132in}}%
\pgfpathlineto{\pgfqpoint{3.056304in}{2.744183in}}%
\pgfpathlineto{\pgfqpoint{3.048211in}{2.733353in}}%
\pgfpathlineto{\pgfqpoint{3.040111in}{2.722643in}}%
\pgfpathclose%
\pgfusepath{fill}%
\end{pgfscope}%
\begin{pgfscope}%
\pgfpathrectangle{\pgfqpoint{1.150000in}{0.150000in}}{\pgfqpoint{5.700000in}{5.700000in}}%
\pgfusepath{clip}%
\pgfsetbuttcap%
\pgfsetroundjoin%
\definecolor{currentfill}{rgb}{0.250425,0.274290,0.533103}%
\pgfsetfillcolor{currentfill}%
\pgfsetfillopacity{0.800000}%
\pgfsetlinewidth{0.000000pt}%
\definecolor{currentstroke}{rgb}{0.000000,0.000000,0.000000}%
\pgfsetstrokecolor{currentstroke}%
\pgfsetdash{}{0pt}%
\pgfpathmoveto{\pgfqpoint{2.986124in}{2.787452in}}%
\pgfpathlineto{\pgfqpoint{2.999630in}{2.770803in}}%
\pgfpathlineto{\pgfqpoint{3.013130in}{2.754454in}}%
\pgfpathlineto{\pgfqpoint{3.026623in}{2.738401in}}%
\pgfpathlineto{\pgfqpoint{3.040111in}{2.722643in}}%
\pgfpathlineto{\pgfqpoint{3.048211in}{2.733353in}}%
\pgfpathlineto{\pgfqpoint{3.056304in}{2.744183in}}%
\pgfpathlineto{\pgfqpoint{3.064389in}{2.755132in}}%
\pgfpathlineto{\pgfqpoint{3.072468in}{2.766202in}}%
\pgfpathlineto{\pgfqpoint{3.058996in}{2.781992in}}%
\pgfpathlineto{\pgfqpoint{3.045518in}{2.798076in}}%
\pgfpathlineto{\pgfqpoint{3.032034in}{2.814457in}}%
\pgfpathlineto{\pgfqpoint{3.018544in}{2.831138in}}%
\pgfpathlineto{\pgfqpoint{3.010450in}{2.820024in}}%
\pgfpathlineto{\pgfqpoint{3.002349in}{2.809039in}}%
\pgfpathlineto{\pgfqpoint{2.994241in}{2.798182in}}%
\pgfpathlineto{\pgfqpoint{2.986124in}{2.787452in}}%
\pgfpathclose%
\pgfusepath{fill}%
\end{pgfscope}%
\begin{pgfscope}%
\pgfpathrectangle{\pgfqpoint{1.150000in}{0.150000in}}{\pgfqpoint{5.700000in}{5.700000in}}%
\pgfusepath{clip}%
\pgfsetbuttcap%
\pgfsetroundjoin%
\definecolor{currentfill}{rgb}{0.280868,0.160771,0.472899}%
\pgfsetfillcolor{currentfill}%
\pgfsetfillopacity{0.800000}%
\pgfsetlinewidth{0.000000pt}%
\definecolor{currentstroke}{rgb}{0.000000,0.000000,0.000000}%
\pgfsetstrokecolor{currentstroke}%
\pgfsetdash{}{0pt}%
\pgfpathmoveto{\pgfqpoint{3.341068in}{2.508685in}}%
\pgfpathlineto{\pgfqpoint{3.354474in}{2.498565in}}%
\pgfpathlineto{\pgfqpoint{3.367880in}{2.488694in}}%
\pgfpathlineto{\pgfqpoint{3.381285in}{2.479071in}}%
\pgfpathlineto{\pgfqpoint{3.394691in}{2.469693in}}%
\pgfpathlineto{\pgfqpoint{3.402683in}{2.480570in}}%
\pgfpathlineto{\pgfqpoint{3.410669in}{2.491522in}}%
\pgfpathlineto{\pgfqpoint{3.418650in}{2.502552in}}%
\pgfpathlineto{\pgfqpoint{3.426625in}{2.513659in}}%
\pgfpathlineto{\pgfqpoint{3.413231in}{2.523102in}}%
\pgfpathlineto{\pgfqpoint{3.399837in}{2.532790in}}%
\pgfpathlineto{\pgfqpoint{3.386443in}{2.542726in}}%
\pgfpathlineto{\pgfqpoint{3.373048in}{2.552911in}}%
\pgfpathlineto{\pgfqpoint{3.365062in}{2.541727in}}%
\pgfpathlineto{\pgfqpoint{3.357070in}{2.530629in}}%
\pgfpathlineto{\pgfqpoint{3.349072in}{2.519615in}}%
\pgfpathlineto{\pgfqpoint{3.341068in}{2.508685in}}%
\pgfpathclose%
\pgfusepath{fill}%
\end{pgfscope}%
\begin{pgfscope}%
\pgfpathrectangle{\pgfqpoint{1.150000in}{0.150000in}}{\pgfqpoint{5.700000in}{5.700000in}}%
\pgfusepath{clip}%
\pgfsetbuttcap%
\pgfsetroundjoin%
\definecolor{currentfill}{rgb}{0.220057,0.343307,0.549413}%
\pgfsetfillcolor{currentfill}%
\pgfsetfillopacity{0.800000}%
\pgfsetlinewidth{0.000000pt}%
\definecolor{currentstroke}{rgb}{0.000000,0.000000,0.000000}%
\pgfsetstrokecolor{currentstroke}%
\pgfsetdash{}{0pt}%
\pgfpathmoveto{\pgfqpoint{4.832310in}{2.921914in}}%
\pgfpathlineto{\pgfqpoint{4.846030in}{2.923294in}}%
\pgfpathlineto{\pgfqpoint{4.859762in}{2.924855in}}%
\pgfpathlineto{\pgfqpoint{4.873505in}{2.926597in}}%
\pgfpathlineto{\pgfqpoint{4.887260in}{2.928519in}}%
\pgfpathlineto{\pgfqpoint{4.894794in}{2.938844in}}%
\pgfpathlineto{\pgfqpoint{4.902326in}{2.949313in}}%
\pgfpathlineto{\pgfqpoint{4.909856in}{2.959932in}}%
\pgfpathlineto{\pgfqpoint{4.917383in}{2.970707in}}%
\pgfpathlineto{\pgfqpoint{4.903644in}{2.969325in}}%
\pgfpathlineto{\pgfqpoint{4.889916in}{2.968124in}}%
\pgfpathlineto{\pgfqpoint{4.876200in}{2.967102in}}%
\pgfpathlineto{\pgfqpoint{4.862494in}{2.966262in}}%
\pgfpathlineto{\pgfqpoint{4.854951in}{2.954935in}}%
\pgfpathlineto{\pgfqpoint{4.847406in}{2.943773in}}%
\pgfpathlineto{\pgfqpoint{4.839859in}{2.932768in}}%
\pgfpathlineto{\pgfqpoint{4.832310in}{2.921914in}}%
\pgfpathclose%
\pgfusepath{fill}%
\end{pgfscope}%
\begin{pgfscope}%
\pgfpathrectangle{\pgfqpoint{1.150000in}{0.150000in}}{\pgfqpoint{5.700000in}{5.700000in}}%
\pgfusepath{clip}%
\pgfsetbuttcap%
\pgfsetroundjoin%
\definecolor{currentfill}{rgb}{0.265145,0.232956,0.516599}%
\pgfsetfillcolor{currentfill}%
\pgfsetfillopacity{0.800000}%
\pgfsetlinewidth{0.000000pt}%
\definecolor{currentstroke}{rgb}{0.000000,0.000000,0.000000}%
\pgfsetstrokecolor{currentstroke}%
\pgfsetdash{}{0pt}%
\pgfpathmoveto{\pgfqpoint{4.322195in}{2.652891in}}%
\pgfpathlineto{\pgfqpoint{4.335745in}{2.652353in}}%
\pgfpathlineto{\pgfqpoint{4.349304in}{2.652007in}}%
\pgfpathlineto{\pgfqpoint{4.362872in}{2.651855in}}%
\pgfpathlineto{\pgfqpoint{4.376448in}{2.651896in}}%
\pgfpathlineto{\pgfqpoint{4.384143in}{2.662443in}}%
\pgfpathlineto{\pgfqpoint{4.391833in}{2.673058in}}%
\pgfpathlineto{\pgfqpoint{4.399518in}{2.683745in}}%
\pgfpathlineto{\pgfqpoint{4.407200in}{2.694508in}}%
\pgfpathlineto{\pgfqpoint{4.393633in}{2.694817in}}%
\pgfpathlineto{\pgfqpoint{4.380075in}{2.695318in}}%
\pgfpathlineto{\pgfqpoint{4.366526in}{2.696013in}}%
\pgfpathlineto{\pgfqpoint{4.352986in}{2.696901in}}%
\pgfpathlineto{\pgfqpoint{4.345294in}{2.685777in}}%
\pgfpathlineto{\pgfqpoint{4.337599in}{2.674737in}}%
\pgfpathlineto{\pgfqpoint{4.329899in}{2.663776in}}%
\pgfpathlineto{\pgfqpoint{4.322195in}{2.652891in}}%
\pgfpathclose%
\pgfusepath{fill}%
\end{pgfscope}%
\begin{pgfscope}%
\pgfpathrectangle{\pgfqpoint{1.150000in}{0.150000in}}{\pgfqpoint{5.700000in}{5.700000in}}%
\pgfusepath{clip}%
\pgfsetbuttcap%
\pgfsetroundjoin%
\definecolor{currentfill}{rgb}{0.266580,0.228262,0.514349}%
\pgfsetfillcolor{currentfill}%
\pgfsetfillopacity{0.800000}%
\pgfsetlinewidth{0.000000pt}%
\definecolor{currentstroke}{rgb}{0.000000,0.000000,0.000000}%
\pgfsetstrokecolor{currentstroke}%
\pgfsetdash{}{0pt}%
\pgfpathmoveto{\pgfqpoint{3.094006in}{2.662505in}}%
\pgfpathlineto{\pgfqpoint{3.107467in}{2.648182in}}%
\pgfpathlineto{\pgfqpoint{3.120924in}{2.634139in}}%
\pgfpathlineto{\pgfqpoint{3.134376in}{2.620374in}}%
\pgfpathlineto{\pgfqpoint{3.147825in}{2.606885in}}%
\pgfpathlineto{\pgfqpoint{3.155895in}{2.617555in}}%
\pgfpathlineto{\pgfqpoint{3.163958in}{2.628328in}}%
\pgfpathlineto{\pgfqpoint{3.172014in}{2.639205in}}%
\pgfpathlineto{\pgfqpoint{3.180064in}{2.650187in}}%
\pgfpathlineto{\pgfqpoint{3.166629in}{2.663709in}}%
\pgfpathlineto{\pgfqpoint{3.153191in}{2.677506in}}%
\pgfpathlineto{\pgfqpoint{3.139749in}{2.691581in}}%
\pgfpathlineto{\pgfqpoint{3.126302in}{2.705936in}}%
\pgfpathlineto{\pgfqpoint{3.118238in}{2.694909in}}%
\pgfpathlineto{\pgfqpoint{3.110168in}{2.683996in}}%
\pgfpathlineto{\pgfqpoint{3.102090in}{2.673195in}}%
\pgfpathlineto{\pgfqpoint{3.094006in}{2.662505in}}%
\pgfpathclose%
\pgfusepath{fill}%
\end{pgfscope}%
\begin{pgfscope}%
\pgfpathrectangle{\pgfqpoint{1.150000in}{0.150000in}}{\pgfqpoint{5.700000in}{5.700000in}}%
\pgfusepath{clip}%
\pgfsetbuttcap%
\pgfsetroundjoin%
\definecolor{currentfill}{rgb}{0.210503,0.363727,0.552206}%
\pgfsetfillcolor{currentfill}%
\pgfsetfillopacity{0.800000}%
\pgfsetlinewidth{0.000000pt}%
\definecolor{currentstroke}{rgb}{0.000000,0.000000,0.000000}%
\pgfsetstrokecolor{currentstroke}%
\pgfsetdash{}{0pt}%
\pgfpathmoveto{\pgfqpoint{4.917383in}{2.970707in}}%
\pgfpathlineto{\pgfqpoint{4.931134in}{2.972268in}}%
\pgfpathlineto{\pgfqpoint{4.944897in}{2.974008in}}%
\pgfpathlineto{\pgfqpoint{4.958672in}{2.975928in}}%
\pgfpathlineto{\pgfqpoint{4.972458in}{2.978026in}}%
\pgfpathlineto{\pgfqpoint{4.979967in}{2.988403in}}%
\pgfpathlineto{\pgfqpoint{4.987475in}{2.998942in}}%
\pgfpathlineto{\pgfqpoint{4.994981in}{3.009648in}}%
\pgfpathlineto{\pgfqpoint{5.002485in}{3.020529in}}%
\pgfpathlineto{\pgfqpoint{4.988715in}{3.019004in}}%
\pgfpathlineto{\pgfqpoint{4.974957in}{3.017656in}}%
\pgfpathlineto{\pgfqpoint{4.961211in}{3.016487in}}%
\pgfpathlineto{\pgfqpoint{4.947477in}{3.015497in}}%
\pgfpathlineto{\pgfqpoint{4.939956in}{3.004033in}}%
\pgfpathlineto{\pgfqpoint{4.932433in}{2.992751in}}%
\pgfpathlineto{\pgfqpoint{4.924909in}{2.981644in}}%
\pgfpathlineto{\pgfqpoint{4.917383in}{2.970707in}}%
\pgfpathclose%
\pgfusepath{fill}%
\end{pgfscope}%
\begin{pgfscope}%
\pgfpathrectangle{\pgfqpoint{1.150000in}{0.150000in}}{\pgfqpoint{5.700000in}{5.700000in}}%
\pgfusepath{clip}%
\pgfsetbuttcap%
\pgfsetroundjoin%
\definecolor{currentfill}{rgb}{0.237441,0.305202,0.541921}%
\pgfsetfillcolor{currentfill}%
\pgfsetfillopacity{0.800000}%
\pgfsetlinewidth{0.000000pt}%
\definecolor{currentstroke}{rgb}{0.000000,0.000000,0.000000}%
\pgfsetstrokecolor{currentstroke}%
\pgfsetdash{}{0pt}%
\pgfpathmoveto{\pgfqpoint{2.932029in}{2.857097in}}%
\pgfpathlineto{\pgfqpoint{2.945564in}{2.839223in}}%
\pgfpathlineto{\pgfqpoint{2.959091in}{2.821659in}}%
\pgfpathlineto{\pgfqpoint{2.972611in}{2.804403in}}%
\pgfpathlineto{\pgfqpoint{2.986124in}{2.787452in}}%
\pgfpathlineto{\pgfqpoint{2.994241in}{2.798182in}}%
\pgfpathlineto{\pgfqpoint{3.002349in}{2.809039in}}%
\pgfpathlineto{\pgfqpoint{3.010450in}{2.820024in}}%
\pgfpathlineto{\pgfqpoint{3.018544in}{2.831138in}}%
\pgfpathlineto{\pgfqpoint{3.005047in}{2.848121in}}%
\pgfpathlineto{\pgfqpoint{2.991543in}{2.865408in}}%
\pgfpathlineto{\pgfqpoint{2.978032in}{2.883003in}}%
\pgfpathlineto{\pgfqpoint{2.964514in}{2.900908in}}%
\pgfpathlineto{\pgfqpoint{2.956404in}{2.889750in}}%
\pgfpathlineto{\pgfqpoint{2.948287in}{2.878730in}}%
\pgfpathlineto{\pgfqpoint{2.940162in}{2.867846in}}%
\pgfpathlineto{\pgfqpoint{2.932029in}{2.857097in}}%
\pgfpathclose%
\pgfusepath{fill}%
\end{pgfscope}%
\begin{pgfscope}%
\pgfpathrectangle{\pgfqpoint{1.150000in}{0.150000in}}{\pgfqpoint{5.700000in}{5.700000in}}%
\pgfusepath{clip}%
\pgfsetbuttcap%
\pgfsetroundjoin%
\definecolor{currentfill}{rgb}{0.269308,0.218818,0.509577}%
\pgfsetfillcolor{currentfill}%
\pgfsetfillopacity{0.800000}%
\pgfsetlinewidth{0.000000pt}%
\definecolor{currentstroke}{rgb}{0.000000,0.000000,0.000000}%
\pgfsetstrokecolor{currentstroke}%
\pgfsetdash{}{0pt}%
\pgfpathmoveto{\pgfqpoint{4.237182in}{2.612860in}}%
\pgfpathlineto{\pgfqpoint{4.250708in}{2.611858in}}%
\pgfpathlineto{\pgfqpoint{4.264242in}{2.611052in}}%
\pgfpathlineto{\pgfqpoint{4.277785in}{2.610442in}}%
\pgfpathlineto{\pgfqpoint{4.291335in}{2.610027in}}%
\pgfpathlineto{\pgfqpoint{4.299057in}{2.620649in}}%
\pgfpathlineto{\pgfqpoint{4.306774in}{2.631332in}}%
\pgfpathlineto{\pgfqpoint{4.314487in}{2.642078in}}%
\pgfpathlineto{\pgfqpoint{4.322195in}{2.652891in}}%
\pgfpathlineto{\pgfqpoint{4.308654in}{2.653625in}}%
\pgfpathlineto{\pgfqpoint{4.295121in}{2.654553in}}%
\pgfpathlineto{\pgfqpoint{4.281596in}{2.655676in}}%
\pgfpathlineto{\pgfqpoint{4.268079in}{2.656996in}}%
\pgfpathlineto{\pgfqpoint{4.260361in}{2.645853in}}%
\pgfpathlineto{\pgfqpoint{4.252639in}{2.634785in}}%
\pgfpathlineto{\pgfqpoint{4.244913in}{2.623789in}}%
\pgfpathlineto{\pgfqpoint{4.237182in}{2.612860in}}%
\pgfpathclose%
\pgfusepath{fill}%
\end{pgfscope}%
\begin{pgfscope}%
\pgfpathrectangle{\pgfqpoint{1.150000in}{0.150000in}}{\pgfqpoint{5.700000in}{5.700000in}}%
\pgfusepath{clip}%
\pgfsetbuttcap%
\pgfsetroundjoin%
\definecolor{currentfill}{rgb}{0.203063,0.379716,0.553925}%
\pgfsetfillcolor{currentfill}%
\pgfsetfillopacity{0.800000}%
\pgfsetlinewidth{0.000000pt}%
\definecolor{currentstroke}{rgb}{0.000000,0.000000,0.000000}%
\pgfsetstrokecolor{currentstroke}%
\pgfsetdash{}{0pt}%
\pgfpathmoveto{\pgfqpoint{5.002485in}{3.020529in}}%
\pgfpathlineto{\pgfqpoint{5.016267in}{3.022233in}}%
\pgfpathlineto{\pgfqpoint{5.030060in}{3.024115in}}%
\pgfpathlineto{\pgfqpoint{5.043866in}{3.026174in}}%
\pgfpathlineto{\pgfqpoint{5.057685in}{3.028410in}}%
\pgfpathlineto{\pgfqpoint{5.065170in}{3.038880in}}%
\pgfpathlineto{\pgfqpoint{5.072654in}{3.049530in}}%
\pgfpathlineto{\pgfqpoint{5.080137in}{3.060367in}}%
\pgfpathlineto{\pgfqpoint{5.087620in}{3.071398in}}%
\pgfpathlineto{\pgfqpoint{5.073820in}{3.069766in}}%
\pgfpathlineto{\pgfqpoint{5.060032in}{3.068311in}}%
\pgfpathlineto{\pgfqpoint{5.046256in}{3.067033in}}%
\pgfpathlineto{\pgfqpoint{5.032492in}{3.065932in}}%
\pgfpathlineto{\pgfqpoint{5.024991in}{3.054286in}}%
\pgfpathlineto{\pgfqpoint{5.017490in}{3.042841in}}%
\pgfpathlineto{\pgfqpoint{5.009988in}{3.031592in}}%
\pgfpathlineto{\pgfqpoint{5.002485in}{3.020529in}}%
\pgfpathclose%
\pgfusepath{fill}%
\end{pgfscope}%
\begin{pgfscope}%
\pgfpathrectangle{\pgfqpoint{1.150000in}{0.150000in}}{\pgfqpoint{5.700000in}{5.700000in}}%
\pgfusepath{clip}%
\pgfsetbuttcap%
\pgfsetroundjoin%
\definecolor{currentfill}{rgb}{0.273006,0.204520,0.501721}%
\pgfsetfillcolor{currentfill}%
\pgfsetfillopacity{0.800000}%
\pgfsetlinewidth{0.000000pt}%
\definecolor{currentstroke}{rgb}{0.000000,0.000000,0.000000}%
\pgfsetstrokecolor{currentstroke}%
\pgfsetdash{}{0pt}%
\pgfpathmoveto{\pgfqpoint{3.147825in}{2.606885in}}%
\pgfpathlineto{\pgfqpoint{3.161270in}{2.593670in}}%
\pgfpathlineto{\pgfqpoint{3.174712in}{2.580726in}}%
\pgfpathlineto{\pgfqpoint{3.188150in}{2.568051in}}%
\pgfpathlineto{\pgfqpoint{3.201586in}{2.555644in}}%
\pgfpathlineto{\pgfqpoint{3.209641in}{2.566293in}}%
\pgfpathlineto{\pgfqpoint{3.217690in}{2.577037in}}%
\pgfpathlineto{\pgfqpoint{3.225733in}{2.587877in}}%
\pgfpathlineto{\pgfqpoint{3.233769in}{2.598815in}}%
\pgfpathlineto{\pgfqpoint{3.220347in}{2.611255in}}%
\pgfpathlineto{\pgfqpoint{3.206922in}{2.623962in}}%
\pgfpathlineto{\pgfqpoint{3.193495in}{2.636939in}}%
\pgfpathlineto{\pgfqpoint{3.180064in}{2.650187in}}%
\pgfpathlineto{\pgfqpoint{3.172014in}{2.639205in}}%
\pgfpathlineto{\pgfqpoint{3.163958in}{2.628328in}}%
\pgfpathlineto{\pgfqpoint{3.155895in}{2.617555in}}%
\pgfpathlineto{\pgfqpoint{3.147825in}{2.606885in}}%
\pgfpathclose%
\pgfusepath{fill}%
\end{pgfscope}%
\begin{pgfscope}%
\pgfpathrectangle{\pgfqpoint{1.150000in}{0.150000in}}{\pgfqpoint{5.700000in}{5.700000in}}%
\pgfusepath{clip}%
\pgfsetbuttcap%
\pgfsetroundjoin%
\definecolor{currentfill}{rgb}{0.274128,0.199721,0.498911}%
\pgfsetfillcolor{currentfill}%
\pgfsetfillopacity{0.800000}%
\pgfsetlinewidth{0.000000pt}%
\definecolor{currentstroke}{rgb}{0.000000,0.000000,0.000000}%
\pgfsetstrokecolor{currentstroke}%
\pgfsetdash{}{0pt}%
\pgfpathmoveto{\pgfqpoint{4.152154in}{2.574587in}}%
\pgfpathlineto{\pgfqpoint{4.165658in}{2.573079in}}%
\pgfpathlineto{\pgfqpoint{4.179169in}{2.571770in}}%
\pgfpathlineto{\pgfqpoint{4.192688in}{2.570660in}}%
\pgfpathlineto{\pgfqpoint{4.206214in}{2.569747in}}%
\pgfpathlineto{\pgfqpoint{4.213963in}{2.580442in}}%
\pgfpathlineto{\pgfqpoint{4.221708in}{2.591190in}}%
\pgfpathlineto{\pgfqpoint{4.229447in}{2.601995in}}%
\pgfpathlineto{\pgfqpoint{4.237182in}{2.612860in}}%
\pgfpathlineto{\pgfqpoint{4.223665in}{2.614059in}}%
\pgfpathlineto{\pgfqpoint{4.210155in}{2.615456in}}%
\pgfpathlineto{\pgfqpoint{4.196652in}{2.617051in}}%
\pgfpathlineto{\pgfqpoint{4.183157in}{2.618844in}}%
\pgfpathlineto{\pgfqpoint{4.175413in}{2.607682in}}%
\pgfpathlineto{\pgfqpoint{4.167665in}{2.596587in}}%
\pgfpathlineto{\pgfqpoint{4.159912in}{2.585557in}}%
\pgfpathlineto{\pgfqpoint{4.152154in}{2.574587in}}%
\pgfpathclose%
\pgfusepath{fill}%
\end{pgfscope}%
\begin{pgfscope}%
\pgfpathrectangle{\pgfqpoint{1.150000in}{0.150000in}}{\pgfqpoint{5.700000in}{5.700000in}}%
\pgfusepath{clip}%
\pgfsetbuttcap%
\pgfsetroundjoin%
\definecolor{currentfill}{rgb}{0.282290,0.145912,0.461510}%
\pgfsetfillcolor{currentfill}%
\pgfsetfillopacity{0.800000}%
\pgfsetlinewidth{0.000000pt}%
\definecolor{currentstroke}{rgb}{0.000000,0.000000,0.000000}%
\pgfsetstrokecolor{currentstroke}%
\pgfsetdash{}{0pt}%
\pgfpathmoveto{\pgfqpoint{3.757992in}{2.461484in}}%
\pgfpathlineto{\pgfqpoint{3.771412in}{2.456648in}}%
\pgfpathlineto{\pgfqpoint{3.784835in}{2.452030in}}%
\pgfpathlineto{\pgfqpoint{3.798264in}{2.447627in}}%
\pgfpathlineto{\pgfqpoint{3.811696in}{2.443439in}}%
\pgfpathlineto{\pgfqpoint{3.819565in}{2.454431in}}%
\pgfpathlineto{\pgfqpoint{3.827428in}{2.465471in}}%
\pgfpathlineto{\pgfqpoint{3.835286in}{2.476562in}}%
\pgfpathlineto{\pgfqpoint{3.843139in}{2.487704in}}%
\pgfpathlineto{\pgfqpoint{3.829715in}{2.492053in}}%
\pgfpathlineto{\pgfqpoint{3.816296in}{2.496615in}}%
\pgfpathlineto{\pgfqpoint{3.802880in}{2.501394in}}%
\pgfpathlineto{\pgfqpoint{3.789470in}{2.506389in}}%
\pgfpathlineto{\pgfqpoint{3.781608in}{2.495074in}}%
\pgfpathlineto{\pgfqpoint{3.773741in}{2.483820in}}%
\pgfpathlineto{\pgfqpoint{3.765869in}{2.472624in}}%
\pgfpathlineto{\pgfqpoint{3.757992in}{2.461484in}}%
\pgfpathclose%
\pgfusepath{fill}%
\end{pgfscope}%
\begin{pgfscope}%
\pgfpathrectangle{\pgfqpoint{1.150000in}{0.150000in}}{\pgfqpoint{5.700000in}{5.700000in}}%
\pgfusepath{clip}%
\pgfsetbuttcap%
\pgfsetroundjoin%
\definecolor{currentfill}{rgb}{0.223925,0.334994,0.548053}%
\pgfsetfillcolor{currentfill}%
\pgfsetfillopacity{0.800000}%
\pgfsetlinewidth{0.000000pt}%
\definecolor{currentstroke}{rgb}{0.000000,0.000000,0.000000}%
\pgfsetstrokecolor{currentstroke}%
\pgfsetdash{}{0pt}%
\pgfpathmoveto{\pgfqpoint{2.877805in}{2.931752in}}%
\pgfpathlineto{\pgfqpoint{2.891374in}{2.912608in}}%
\pgfpathlineto{\pgfqpoint{2.904934in}{2.893786in}}%
\pgfpathlineto{\pgfqpoint{2.918485in}{2.875284in}}%
\pgfpathlineto{\pgfqpoint{2.932029in}{2.857097in}}%
\pgfpathlineto{\pgfqpoint{2.940162in}{2.867846in}}%
\pgfpathlineto{\pgfqpoint{2.948287in}{2.878730in}}%
\pgfpathlineto{\pgfqpoint{2.956404in}{2.889750in}}%
\pgfpathlineto{\pgfqpoint{2.964514in}{2.900908in}}%
\pgfpathlineto{\pgfqpoint{2.950987in}{2.919126in}}%
\pgfpathlineto{\pgfqpoint{2.937453in}{2.937660in}}%
\pgfpathlineto{\pgfqpoint{2.923910in}{2.956512in}}%
\pgfpathlineto{\pgfqpoint{2.910359in}{2.975687in}}%
\pgfpathlineto{\pgfqpoint{2.902233in}{2.964487in}}%
\pgfpathlineto{\pgfqpoint{2.894099in}{2.953432in}}%
\pgfpathlineto{\pgfqpoint{2.885956in}{2.942520in}}%
\pgfpathlineto{\pgfqpoint{2.877805in}{2.931752in}}%
\pgfpathclose%
\pgfusepath{fill}%
\end{pgfscope}%
\begin{pgfscope}%
\pgfpathrectangle{\pgfqpoint{1.150000in}{0.150000in}}{\pgfqpoint{5.700000in}{5.700000in}}%
\pgfusepath{clip}%
\pgfsetbuttcap%
\pgfsetroundjoin%
\definecolor{currentfill}{rgb}{0.194100,0.399323,0.555565}%
\pgfsetfillcolor{currentfill}%
\pgfsetfillopacity{0.800000}%
\pgfsetlinewidth{0.000000pt}%
\definecolor{currentstroke}{rgb}{0.000000,0.000000,0.000000}%
\pgfsetstrokecolor{currentstroke}%
\pgfsetdash{}{0pt}%
\pgfpathmoveto{\pgfqpoint{5.087620in}{3.071398in}}%
\pgfpathlineto{\pgfqpoint{5.101432in}{3.073207in}}%
\pgfpathlineto{\pgfqpoint{5.115257in}{3.075191in}}%
\pgfpathlineto{\pgfqpoint{5.129094in}{3.077352in}}%
\pgfpathlineto{\pgfqpoint{5.142944in}{3.079688in}}%
\pgfpathlineto{\pgfqpoint{5.150407in}{3.090296in}}%
\pgfpathlineto{\pgfqpoint{5.157869in}{3.101105in}}%
\pgfpathlineto{\pgfqpoint{5.165331in}{3.112122in}}%
\pgfpathlineto{\pgfqpoint{5.172794in}{3.123354in}}%
\pgfpathlineto{\pgfqpoint{5.158963in}{3.121654in}}%
\pgfpathlineto{\pgfqpoint{5.145146in}{3.120129in}}%
\pgfpathlineto{\pgfqpoint{5.131340in}{3.118780in}}%
\pgfpathlineto{\pgfqpoint{5.117547in}{3.117606in}}%
\pgfpathlineto{\pgfqpoint{5.110065in}{3.105727in}}%
\pgfpathlineto{\pgfqpoint{5.102584in}{3.094071in}}%
\pgfpathlineto{\pgfqpoint{5.095102in}{3.082631in}}%
\pgfpathlineto{\pgfqpoint{5.087620in}{3.071398in}}%
\pgfpathclose%
\pgfusepath{fill}%
\end{pgfscope}%
\begin{pgfscope}%
\pgfpathrectangle{\pgfqpoint{1.150000in}{0.150000in}}{\pgfqpoint{5.700000in}{5.700000in}}%
\pgfusepath{clip}%
\pgfsetbuttcap%
\pgfsetroundjoin%
\definecolor{currentfill}{rgb}{0.277134,0.185228,0.489898}%
\pgfsetfillcolor{currentfill}%
\pgfsetfillopacity{0.800000}%
\pgfsetlinewidth{0.000000pt}%
\definecolor{currentstroke}{rgb}{0.000000,0.000000,0.000000}%
\pgfsetstrokecolor{currentstroke}%
\pgfsetdash{}{0pt}%
\pgfpathmoveto{\pgfqpoint{4.067103in}{2.538273in}}%
\pgfpathlineto{\pgfqpoint{4.080586in}{2.536216in}}%
\pgfpathlineto{\pgfqpoint{4.094076in}{2.534361in}}%
\pgfpathlineto{\pgfqpoint{4.107573in}{2.532707in}}%
\pgfpathlineto{\pgfqpoint{4.121078in}{2.531254in}}%
\pgfpathlineto{\pgfqpoint{4.128854in}{2.542012in}}%
\pgfpathlineto{\pgfqpoint{4.136625in}{2.552818in}}%
\pgfpathlineto{\pgfqpoint{4.144392in}{2.563676in}}%
\pgfpathlineto{\pgfqpoint{4.152154in}{2.574587in}}%
\pgfpathlineto{\pgfqpoint{4.138658in}{2.576295in}}%
\pgfpathlineto{\pgfqpoint{4.125170in}{2.578204in}}%
\pgfpathlineto{\pgfqpoint{4.111688in}{2.580313in}}%
\pgfpathlineto{\pgfqpoint{4.098213in}{2.582625in}}%
\pgfpathlineto{\pgfqpoint{4.090442in}{2.571447in}}%
\pgfpathlineto{\pgfqpoint{4.082667in}{2.560331in}}%
\pgfpathlineto{\pgfqpoint{4.074887in}{2.549275in}}%
\pgfpathlineto{\pgfqpoint{4.067103in}{2.538273in}}%
\pgfpathclose%
\pgfusepath{fill}%
\end{pgfscope}%
\begin{pgfscope}%
\pgfpathrectangle{\pgfqpoint{1.150000in}{0.150000in}}{\pgfqpoint{5.700000in}{5.700000in}}%
\pgfusepath{clip}%
\pgfsetbuttcap%
\pgfsetroundjoin%
\definecolor{currentfill}{rgb}{0.185556,0.418570,0.556753}%
\pgfsetfillcolor{currentfill}%
\pgfsetfillopacity{0.800000}%
\pgfsetlinewidth{0.000000pt}%
\definecolor{currentstroke}{rgb}{0.000000,0.000000,0.000000}%
\pgfsetstrokecolor{currentstroke}%
\pgfsetdash{}{0pt}%
\pgfpathmoveto{\pgfqpoint{5.172794in}{3.123354in}}%
\pgfpathlineto{\pgfqpoint{5.186636in}{3.125229in}}%
\pgfpathlineto{\pgfqpoint{5.200492in}{3.127279in}}%
\pgfpathlineto{\pgfqpoint{5.214360in}{3.129503in}}%
\pgfpathlineto{\pgfqpoint{5.228241in}{3.131902in}}%
\pgfpathlineto{\pgfqpoint{5.235683in}{3.142700in}}%
\pgfpathlineto{\pgfqpoint{5.243125in}{3.153721in}}%
\pgfpathlineto{\pgfqpoint{5.250569in}{3.164972in}}%
\pgfpathlineto{\pgfqpoint{5.258013in}{3.176460in}}%
\pgfpathlineto{\pgfqpoint{5.244153in}{3.174730in}}%
\pgfpathlineto{\pgfqpoint{5.230306in}{3.173173in}}%
\pgfpathlineto{\pgfqpoint{5.216471in}{3.171790in}}%
\pgfpathlineto{\pgfqpoint{5.202649in}{3.170581in}}%
\pgfpathlineto{\pgfqpoint{5.195184in}{3.158414in}}%
\pgfpathlineto{\pgfqpoint{5.187720in}{3.146493in}}%
\pgfpathlineto{\pgfqpoint{5.180256in}{3.134808in}}%
\pgfpathlineto{\pgfqpoint{5.172794in}{3.123354in}}%
\pgfpathclose%
\pgfusepath{fill}%
\end{pgfscope}%
\begin{pgfscope}%
\pgfpathrectangle{\pgfqpoint{1.150000in}{0.150000in}}{\pgfqpoint{5.700000in}{5.700000in}}%
\pgfusepath{clip}%
\pgfsetbuttcap%
\pgfsetroundjoin%
\definecolor{currentfill}{rgb}{0.282623,0.140926,0.457517}%
\pgfsetfillcolor{currentfill}%
\pgfsetfillopacity{0.800000}%
\pgfsetlinewidth{0.000000pt}%
\definecolor{currentstroke}{rgb}{0.000000,0.000000,0.000000}%
\pgfsetstrokecolor{currentstroke}%
\pgfsetdash{}{0pt}%
\pgfpathmoveto{\pgfqpoint{3.533797in}{2.446773in}}%
\pgfpathlineto{\pgfqpoint{3.547199in}{2.439473in}}%
\pgfpathlineto{\pgfqpoint{3.560603in}{2.432404in}}%
\pgfpathlineto{\pgfqpoint{3.574009in}{2.425564in}}%
\pgfpathlineto{\pgfqpoint{3.587418in}{2.418953in}}%
\pgfpathlineto{\pgfqpoint{3.595355in}{2.429885in}}%
\pgfpathlineto{\pgfqpoint{3.603287in}{2.440874in}}%
\pgfpathlineto{\pgfqpoint{3.611214in}{2.451922in}}%
\pgfpathlineto{\pgfqpoint{3.619135in}{2.463030in}}%
\pgfpathlineto{\pgfqpoint{3.605737in}{2.469738in}}%
\pgfpathlineto{\pgfqpoint{3.592340in}{2.476675in}}%
\pgfpathlineto{\pgfqpoint{3.578946in}{2.483841in}}%
\pgfpathlineto{\pgfqpoint{3.565554in}{2.491238in}}%
\pgfpathlineto{\pgfqpoint{3.557623in}{2.480021in}}%
\pgfpathlineto{\pgfqpoint{3.549686in}{2.468873in}}%
\pgfpathlineto{\pgfqpoint{3.541744in}{2.457790in}}%
\pgfpathlineto{\pgfqpoint{3.533797in}{2.446773in}}%
\pgfpathclose%
\pgfusepath{fill}%
\end{pgfscope}%
\begin{pgfscope}%
\pgfpathrectangle{\pgfqpoint{1.150000in}{0.150000in}}{\pgfqpoint{5.700000in}{5.700000in}}%
\pgfusepath{clip}%
\pgfsetbuttcap%
\pgfsetroundjoin%
\definecolor{currentfill}{rgb}{0.281887,0.150881,0.465405}%
\pgfsetfillcolor{currentfill}%
\pgfsetfillopacity{0.800000}%
\pgfsetlinewidth{0.000000pt}%
\definecolor{currentstroke}{rgb}{0.000000,0.000000,0.000000}%
\pgfsetstrokecolor{currentstroke}%
\pgfsetdash{}{0pt}%
\pgfpathmoveto{\pgfqpoint{3.394691in}{2.469693in}}%
\pgfpathlineto{\pgfqpoint{3.408096in}{2.460559in}}%
\pgfpathlineto{\pgfqpoint{3.421501in}{2.451668in}}%
\pgfpathlineto{\pgfqpoint{3.434907in}{2.443018in}}%
\pgfpathlineto{\pgfqpoint{3.448314in}{2.434607in}}%
\pgfpathlineto{\pgfqpoint{3.456295in}{2.445430in}}%
\pgfpathlineto{\pgfqpoint{3.464270in}{2.456321in}}%
\pgfpathlineto{\pgfqpoint{3.472239in}{2.467282in}}%
\pgfpathlineto{\pgfqpoint{3.480203in}{2.478312in}}%
\pgfpathlineto{\pgfqpoint{3.466808in}{2.486788in}}%
\pgfpathlineto{\pgfqpoint{3.453413in}{2.495504in}}%
\pgfpathlineto{\pgfqpoint{3.440019in}{2.504460in}}%
\pgfpathlineto{\pgfqpoint{3.426625in}{2.513659in}}%
\pgfpathlineto{\pgfqpoint{3.418650in}{2.502552in}}%
\pgfpathlineto{\pgfqpoint{3.410669in}{2.491522in}}%
\pgfpathlineto{\pgfqpoint{3.402683in}{2.480570in}}%
\pgfpathlineto{\pgfqpoint{3.394691in}{2.469693in}}%
\pgfpathclose%
\pgfusepath{fill}%
\end{pgfscope}%
\begin{pgfscope}%
\pgfpathrectangle{\pgfqpoint{1.150000in}{0.150000in}}{\pgfqpoint{5.700000in}{5.700000in}}%
\pgfusepath{clip}%
\pgfsetbuttcap%
\pgfsetroundjoin%
\definecolor{currentfill}{rgb}{0.277134,0.185228,0.489898}%
\pgfsetfillcolor{currentfill}%
\pgfsetfillopacity{0.800000}%
\pgfsetlinewidth{0.000000pt}%
\definecolor{currentstroke}{rgb}{0.000000,0.000000,0.000000}%
\pgfsetstrokecolor{currentstroke}%
\pgfsetdash{}{0pt}%
\pgfpathmoveto{\pgfqpoint{3.201586in}{2.555644in}}%
\pgfpathlineto{\pgfqpoint{3.215018in}{2.543502in}}%
\pgfpathlineto{\pgfqpoint{3.228448in}{2.531624in}}%
\pgfpathlineto{\pgfqpoint{3.241876in}{2.520007in}}%
\pgfpathlineto{\pgfqpoint{3.255302in}{2.508649in}}%
\pgfpathlineto{\pgfqpoint{3.263344in}{2.519277in}}%
\pgfpathlineto{\pgfqpoint{3.271380in}{2.529992in}}%
\pgfpathlineto{\pgfqpoint{3.279409in}{2.540795in}}%
\pgfpathlineto{\pgfqpoint{3.287432in}{2.551688in}}%
\pgfpathlineto{\pgfqpoint{3.274019in}{2.563079in}}%
\pgfpathlineto{\pgfqpoint{3.260604in}{2.574729in}}%
\pgfpathlineto{\pgfqpoint{3.247188in}{2.586640in}}%
\pgfpathlineto{\pgfqpoint{3.233769in}{2.598815in}}%
\pgfpathlineto{\pgfqpoint{3.225733in}{2.587877in}}%
\pgfpathlineto{\pgfqpoint{3.217690in}{2.577037in}}%
\pgfpathlineto{\pgfqpoint{3.209641in}{2.566293in}}%
\pgfpathlineto{\pgfqpoint{3.201586in}{2.555644in}}%
\pgfpathclose%
\pgfusepath{fill}%
\end{pgfscope}%
\begin{pgfscope}%
\pgfpathrectangle{\pgfqpoint{1.150000in}{0.150000in}}{\pgfqpoint{5.700000in}{5.700000in}}%
\pgfusepath{clip}%
\pgfsetbuttcap%
\pgfsetroundjoin%
\definecolor{currentfill}{rgb}{0.210503,0.363727,0.552206}%
\pgfsetfillcolor{currentfill}%
\pgfsetfillopacity{0.800000}%
\pgfsetlinewidth{0.000000pt}%
\definecolor{currentstroke}{rgb}{0.000000,0.000000,0.000000}%
\pgfsetstrokecolor{currentstroke}%
\pgfsetdash{}{0pt}%
\pgfpathmoveto{\pgfqpoint{2.823436in}{3.011609in}}%
\pgfpathlineto{\pgfqpoint{2.837043in}{2.991146in}}%
\pgfpathlineto{\pgfqpoint{2.850640in}{2.971018in}}%
\pgfpathlineto{\pgfqpoint{2.864227in}{2.951221in}}%
\pgfpathlineto{\pgfqpoint{2.877805in}{2.931752in}}%
\pgfpathlineto{\pgfqpoint{2.885956in}{2.942520in}}%
\pgfpathlineto{\pgfqpoint{2.894099in}{2.953432in}}%
\pgfpathlineto{\pgfqpoint{2.902233in}{2.964487in}}%
\pgfpathlineto{\pgfqpoint{2.910359in}{2.975687in}}%
\pgfpathlineto{\pgfqpoint{2.896798in}{2.995187in}}%
\pgfpathlineto{\pgfqpoint{2.883229in}{3.015014in}}%
\pgfpathlineto{\pgfqpoint{2.869650in}{3.035173in}}%
\pgfpathlineto{\pgfqpoint{2.856060in}{3.055666in}}%
\pgfpathlineto{\pgfqpoint{2.847917in}{3.044423in}}%
\pgfpathlineto{\pgfqpoint{2.839765in}{3.033333in}}%
\pgfpathlineto{\pgfqpoint{2.831605in}{3.022395in}}%
\pgfpathlineto{\pgfqpoint{2.823436in}{3.011609in}}%
\pgfpathclose%
\pgfusepath{fill}%
\end{pgfscope}%
\begin{pgfscope}%
\pgfpathrectangle{\pgfqpoint{1.150000in}{0.150000in}}{\pgfqpoint{5.700000in}{5.700000in}}%
\pgfusepath{clip}%
\pgfsetbuttcap%
\pgfsetroundjoin%
\definecolor{currentfill}{rgb}{0.177423,0.437527,0.557565}%
\pgfsetfillcolor{currentfill}%
\pgfsetfillopacity{0.800000}%
\pgfsetlinewidth{0.000000pt}%
\definecolor{currentstroke}{rgb}{0.000000,0.000000,0.000000}%
\pgfsetstrokecolor{currentstroke}%
\pgfsetdash{}{0pt}%
\pgfpathmoveto{\pgfqpoint{5.258013in}{3.176460in}}%
\pgfpathlineto{\pgfqpoint{5.271886in}{3.178365in}}%
\pgfpathlineto{\pgfqpoint{5.285771in}{3.180442in}}%
\pgfpathlineto{\pgfqpoint{5.299670in}{3.182693in}}%
\pgfpathlineto{\pgfqpoint{5.313582in}{3.185117in}}%
\pgfpathlineto{\pgfqpoint{5.321005in}{3.196163in}}%
\pgfpathlineto{\pgfqpoint{5.328430in}{3.207455in}}%
\pgfpathlineto{\pgfqpoint{5.335857in}{3.219000in}}%
\pgfpathlineto{\pgfqpoint{5.343285in}{3.230806in}}%
\pgfpathlineto{\pgfqpoint{5.329396in}{3.229082in}}%
\pgfpathlineto{\pgfqpoint{5.315520in}{3.227531in}}%
\pgfpathlineto{\pgfqpoint{5.301656in}{3.226152in}}%
\pgfpathlineto{\pgfqpoint{5.287806in}{3.224946in}}%
\pgfpathlineto{\pgfqpoint{5.280354in}{3.212429in}}%
\pgfpathlineto{\pgfqpoint{5.272906in}{3.200181in}}%
\pgfpathlineto{\pgfqpoint{5.265458in}{3.188194in}}%
\pgfpathlineto{\pgfqpoint{5.258013in}{3.176460in}}%
\pgfpathclose%
\pgfusepath{fill}%
\end{pgfscope}%
\begin{pgfscope}%
\pgfpathrectangle{\pgfqpoint{1.150000in}{0.150000in}}{\pgfqpoint{5.700000in}{5.700000in}}%
\pgfusepath{clip}%
\pgfsetbuttcap%
\pgfsetroundjoin%
\definecolor{currentfill}{rgb}{0.279574,0.170599,0.479997}%
\pgfsetfillcolor{currentfill}%
\pgfsetfillopacity{0.800000}%
\pgfsetlinewidth{0.000000pt}%
\definecolor{currentstroke}{rgb}{0.000000,0.000000,0.000000}%
\pgfsetstrokecolor{currentstroke}%
\pgfsetdash{}{0pt}%
\pgfpathmoveto{\pgfqpoint{3.982017in}{2.504141in}}%
\pgfpathlineto{\pgfqpoint{3.995482in}{2.501490in}}%
\pgfpathlineto{\pgfqpoint{4.008954in}{2.499045in}}%
\pgfpathlineto{\pgfqpoint{4.022432in}{2.496804in}}%
\pgfpathlineto{\pgfqpoint{4.035916in}{2.494767in}}%
\pgfpathlineto{\pgfqpoint{4.043720in}{2.505575in}}%
\pgfpathlineto{\pgfqpoint{4.051519in}{2.516426in}}%
\pgfpathlineto{\pgfqpoint{4.059313in}{2.527325in}}%
\pgfpathlineto{\pgfqpoint{4.067103in}{2.538273in}}%
\pgfpathlineto{\pgfqpoint{4.053626in}{2.540534in}}%
\pgfpathlineto{\pgfqpoint{4.040156in}{2.542998in}}%
\pgfpathlineto{\pgfqpoint{4.026693in}{2.545666in}}%
\pgfpathlineto{\pgfqpoint{4.013236in}{2.548540in}}%
\pgfpathlineto{\pgfqpoint{4.005439in}{2.537357in}}%
\pgfpathlineto{\pgfqpoint{3.997636in}{2.526231in}}%
\pgfpathlineto{\pgfqpoint{3.989829in}{2.515160in}}%
\pgfpathlineto{\pgfqpoint{3.982017in}{2.504141in}}%
\pgfpathclose%
\pgfusepath{fill}%
\end{pgfscope}%
\begin{pgfscope}%
\pgfpathrectangle{\pgfqpoint{1.150000in}{0.150000in}}{\pgfqpoint{5.700000in}{5.700000in}}%
\pgfusepath{clip}%
\pgfsetbuttcap%
\pgfsetroundjoin%
\definecolor{currentfill}{rgb}{0.282623,0.140926,0.457517}%
\pgfsetfillcolor{currentfill}%
\pgfsetfillopacity{0.800000}%
\pgfsetlinewidth{0.000000pt}%
\definecolor{currentstroke}{rgb}{0.000000,0.000000,0.000000}%
\pgfsetstrokecolor{currentstroke}%
\pgfsetdash{}{0pt}%
\pgfpathmoveto{\pgfqpoint{3.672758in}{2.438457in}}%
\pgfpathlineto{\pgfqpoint{3.686171in}{2.432872in}}%
\pgfpathlineto{\pgfqpoint{3.699588in}{2.427509in}}%
\pgfpathlineto{\pgfqpoint{3.713009in}{2.422365in}}%
\pgfpathlineto{\pgfqpoint{3.726433in}{2.417441in}}%
\pgfpathlineto{\pgfqpoint{3.734331in}{2.428377in}}%
\pgfpathlineto{\pgfqpoint{3.742223in}{2.439362in}}%
\pgfpathlineto{\pgfqpoint{3.750110in}{2.450397in}}%
\pgfpathlineto{\pgfqpoint{3.757992in}{2.461484in}}%
\pgfpathlineto{\pgfqpoint{3.744577in}{2.466537in}}%
\pgfpathlineto{\pgfqpoint{3.731165in}{2.471809in}}%
\pgfpathlineto{\pgfqpoint{3.717757in}{2.477301in}}%
\pgfpathlineto{\pgfqpoint{3.704353in}{2.483014in}}%
\pgfpathlineto{\pgfqpoint{3.696462in}{2.471787in}}%
\pgfpathlineto{\pgfqpoint{3.688566in}{2.460620in}}%
\pgfpathlineto{\pgfqpoint{3.680665in}{2.449510in}}%
\pgfpathlineto{\pgfqpoint{3.672758in}{2.438457in}}%
\pgfpathclose%
\pgfusepath{fill}%
\end{pgfscope}%
\begin{pgfscope}%
\pgfpathrectangle{\pgfqpoint{1.150000in}{0.150000in}}{\pgfqpoint{5.700000in}{5.700000in}}%
\pgfusepath{clip}%
\pgfsetbuttcap%
\pgfsetroundjoin%
\definecolor{currentfill}{rgb}{0.168126,0.459988,0.558082}%
\pgfsetfillcolor{currentfill}%
\pgfsetfillopacity{0.800000}%
\pgfsetlinewidth{0.000000pt}%
\definecolor{currentstroke}{rgb}{0.000000,0.000000,0.000000}%
\pgfsetstrokecolor{currentstroke}%
\pgfsetdash{}{0pt}%
\pgfpathmoveto{\pgfqpoint{5.343285in}{3.230806in}}%
\pgfpathlineto{\pgfqpoint{5.357187in}{3.232703in}}%
\pgfpathlineto{\pgfqpoint{5.371103in}{3.234771in}}%
\pgfpathlineto{\pgfqpoint{5.385032in}{3.237012in}}%
\pgfpathlineto{\pgfqpoint{5.398974in}{3.239424in}}%
\pgfpathlineto{\pgfqpoint{5.406381in}{3.250780in}}%
\pgfpathlineto{\pgfqpoint{5.413791in}{3.262407in}}%
\pgfpathlineto{\pgfqpoint{5.421203in}{3.274312in}}%
\pgfpathlineto{\pgfqpoint{5.428619in}{3.286504in}}%
\pgfpathlineto{\pgfqpoint{5.414701in}{3.284824in}}%
\pgfpathlineto{\pgfqpoint{5.400797in}{3.283315in}}%
\pgfpathlineto{\pgfqpoint{5.386905in}{3.281977in}}%
\pgfpathlineto{\pgfqpoint{5.373026in}{3.280811in}}%
\pgfpathlineto{\pgfqpoint{5.365586in}{3.267876in}}%
\pgfpathlineto{\pgfqpoint{5.358150in}{3.255236in}}%
\pgfpathlineto{\pgfqpoint{5.350716in}{3.242882in}}%
\pgfpathlineto{\pgfqpoint{5.343285in}{3.230806in}}%
\pgfpathclose%
\pgfusepath{fill}%
\end{pgfscope}%
\begin{pgfscope}%
\pgfpathrectangle{\pgfqpoint{1.150000in}{0.150000in}}{\pgfqpoint{5.700000in}{5.700000in}}%
\pgfusepath{clip}%
\pgfsetbuttcap%
\pgfsetroundjoin%
\definecolor{currentfill}{rgb}{0.280255,0.165693,0.476498}%
\pgfsetfillcolor{currentfill}%
\pgfsetfillopacity{0.800000}%
\pgfsetlinewidth{0.000000pt}%
\definecolor{currentstroke}{rgb}{0.000000,0.000000,0.000000}%
\pgfsetstrokecolor{currentstroke}%
\pgfsetdash{}{0pt}%
\pgfpathmoveto{\pgfqpoint{3.255302in}{2.508649in}}%
\pgfpathlineto{\pgfqpoint{3.268726in}{2.497550in}}%
\pgfpathlineto{\pgfqpoint{3.282148in}{2.486706in}}%
\pgfpathlineto{\pgfqpoint{3.295569in}{2.476117in}}%
\pgfpathlineto{\pgfqpoint{3.308989in}{2.465780in}}%
\pgfpathlineto{\pgfqpoint{3.317018in}{2.476386in}}%
\pgfpathlineto{\pgfqpoint{3.325041in}{2.487071in}}%
\pgfpathlineto{\pgfqpoint{3.333057in}{2.497837in}}%
\pgfpathlineto{\pgfqpoint{3.341068in}{2.508685in}}%
\pgfpathlineto{\pgfqpoint{3.327660in}{2.519056in}}%
\pgfpathlineto{\pgfqpoint{3.314252in}{2.529679in}}%
\pgfpathlineto{\pgfqpoint{3.300843in}{2.540555in}}%
\pgfpathlineto{\pgfqpoint{3.287432in}{2.551688in}}%
\pgfpathlineto{\pgfqpoint{3.279409in}{2.540795in}}%
\pgfpathlineto{\pgfqpoint{3.271380in}{2.529992in}}%
\pgfpathlineto{\pgfqpoint{3.263344in}{2.519277in}}%
\pgfpathlineto{\pgfqpoint{3.255302in}{2.508649in}}%
\pgfpathclose%
\pgfusepath{fill}%
\end{pgfscope}%
\begin{pgfscope}%
\pgfpathrectangle{\pgfqpoint{1.150000in}{0.150000in}}{\pgfqpoint{5.700000in}{5.700000in}}%
\pgfusepath{clip}%
\pgfsetbuttcap%
\pgfsetroundjoin%
\definecolor{currentfill}{rgb}{0.280868,0.160771,0.472899}%
\pgfsetfillcolor{currentfill}%
\pgfsetfillopacity{0.800000}%
\pgfsetlinewidth{0.000000pt}%
\definecolor{currentstroke}{rgb}{0.000000,0.000000,0.000000}%
\pgfsetstrokecolor{currentstroke}%
\pgfsetdash{}{0pt}%
\pgfpathmoveto{\pgfqpoint{3.896886in}{2.472438in}}%
\pgfpathlineto{\pgfqpoint{3.910336in}{2.469149in}}%
\pgfpathlineto{\pgfqpoint{3.923791in}{2.466068in}}%
\pgfpathlineto{\pgfqpoint{3.937253in}{2.463196in}}%
\pgfpathlineto{\pgfqpoint{3.950720in}{2.460531in}}%
\pgfpathlineto{\pgfqpoint{3.958552in}{2.471368in}}%
\pgfpathlineto{\pgfqpoint{3.966378in}{2.482248in}}%
\pgfpathlineto{\pgfqpoint{3.974200in}{2.493171in}}%
\pgfpathlineto{\pgfqpoint{3.982017in}{2.504141in}}%
\pgfpathlineto{\pgfqpoint{3.968558in}{2.506998in}}%
\pgfpathlineto{\pgfqpoint{3.955105in}{2.510062in}}%
\pgfpathlineto{\pgfqpoint{3.941657in}{2.513335in}}%
\pgfpathlineto{\pgfqpoint{3.928216in}{2.516816in}}%
\pgfpathlineto{\pgfqpoint{3.920391in}{2.505642in}}%
\pgfpathlineto{\pgfqpoint{3.912561in}{2.494523in}}%
\pgfpathlineto{\pgfqpoint{3.904726in}{2.483456in}}%
\pgfpathlineto{\pgfqpoint{3.896886in}{2.472438in}}%
\pgfpathclose%
\pgfusepath{fill}%
\end{pgfscope}%
\begin{pgfscope}%
\pgfpathrectangle{\pgfqpoint{1.150000in}{0.150000in}}{\pgfqpoint{5.700000in}{5.700000in}}%
\pgfusepath{clip}%
\pgfsetbuttcap%
\pgfsetroundjoin%
\definecolor{currentfill}{rgb}{0.195860,0.395433,0.555276}%
\pgfsetfillcolor{currentfill}%
\pgfsetfillopacity{0.800000}%
\pgfsetlinewidth{0.000000pt}%
\definecolor{currentstroke}{rgb}{0.000000,0.000000,0.000000}%
\pgfsetstrokecolor{currentstroke}%
\pgfsetdash{}{0pt}%
\pgfpathmoveto{\pgfqpoint{2.768899in}{3.096870in}}%
\pgfpathlineto{\pgfqpoint{2.782550in}{3.075036in}}%
\pgfpathlineto{\pgfqpoint{2.796189in}{3.053550in}}%
\pgfpathlineto{\pgfqpoint{2.809818in}{3.032409in}}%
\pgfpathlineto{\pgfqpoint{2.823436in}{3.011609in}}%
\pgfpathlineto{\pgfqpoint{2.831605in}{3.022395in}}%
\pgfpathlineto{\pgfqpoint{2.839765in}{3.033333in}}%
\pgfpathlineto{\pgfqpoint{2.847917in}{3.044423in}}%
\pgfpathlineto{\pgfqpoint{2.856060in}{3.055666in}}%
\pgfpathlineto{\pgfqpoint{2.842461in}{3.076497in}}%
\pgfpathlineto{\pgfqpoint{2.828851in}{3.097669in}}%
\pgfpathlineto{\pgfqpoint{2.815230in}{3.119185in}}%
\pgfpathlineto{\pgfqpoint{2.801598in}{3.141049in}}%
\pgfpathlineto{\pgfqpoint{2.793436in}{3.129763in}}%
\pgfpathlineto{\pgfqpoint{2.785266in}{3.118639in}}%
\pgfpathlineto{\pgfqpoint{2.777087in}{3.107675in}}%
\pgfpathlineto{\pgfqpoint{2.768899in}{3.096870in}}%
\pgfpathclose%
\pgfusepath{fill}%
\end{pgfscope}%
\begin{pgfscope}%
\pgfpathrectangle{\pgfqpoint{1.150000in}{0.150000in}}{\pgfqpoint{5.700000in}{5.700000in}}%
\pgfusepath{clip}%
\pgfsetbuttcap%
\pgfsetroundjoin%
\definecolor{currentfill}{rgb}{0.282623,0.140926,0.457517}%
\pgfsetfillcolor{currentfill}%
\pgfsetfillopacity{0.800000}%
\pgfsetlinewidth{0.000000pt}%
\definecolor{currentstroke}{rgb}{0.000000,0.000000,0.000000}%
\pgfsetstrokecolor{currentstroke}%
\pgfsetdash{}{0pt}%
\pgfpathmoveto{\pgfqpoint{3.448314in}{2.434607in}}%
\pgfpathlineto{\pgfqpoint{3.461721in}{2.426434in}}%
\pgfpathlineto{\pgfqpoint{3.475130in}{2.418498in}}%
\pgfpathlineto{\pgfqpoint{3.488539in}{2.410797in}}%
\pgfpathlineto{\pgfqpoint{3.501950in}{2.403330in}}%
\pgfpathlineto{\pgfqpoint{3.509920in}{2.414100in}}%
\pgfpathlineto{\pgfqpoint{3.517885in}{2.424929in}}%
\pgfpathlineto{\pgfqpoint{3.525843in}{2.435820in}}%
\pgfpathlineto{\pgfqpoint{3.533797in}{2.446773in}}%
\pgfpathlineto{\pgfqpoint{3.520396in}{2.454306in}}%
\pgfpathlineto{\pgfqpoint{3.506997in}{2.462073in}}%
\pgfpathlineto{\pgfqpoint{3.493600in}{2.470074in}}%
\pgfpathlineto{\pgfqpoint{3.480203in}{2.478312in}}%
\pgfpathlineto{\pgfqpoint{3.472239in}{2.467282in}}%
\pgfpathlineto{\pgfqpoint{3.464270in}{2.456321in}}%
\pgfpathlineto{\pgfqpoint{3.456295in}{2.445430in}}%
\pgfpathlineto{\pgfqpoint{3.448314in}{2.434607in}}%
\pgfpathclose%
\pgfusepath{fill}%
\end{pgfscope}%
\begin{pgfscope}%
\pgfpathrectangle{\pgfqpoint{1.150000in}{0.150000in}}{\pgfqpoint{5.700000in}{5.700000in}}%
\pgfusepath{clip}%
\pgfsetbuttcap%
\pgfsetroundjoin%
\definecolor{currentfill}{rgb}{0.160665,0.478540,0.558115}%
\pgfsetfillcolor{currentfill}%
\pgfsetfillopacity{0.800000}%
\pgfsetlinewidth{0.000000pt}%
\definecolor{currentstroke}{rgb}{0.000000,0.000000,0.000000}%
\pgfsetstrokecolor{currentstroke}%
\pgfsetdash{}{0pt}%
\pgfpathmoveto{\pgfqpoint{5.428619in}{3.286504in}}%
\pgfpathlineto{\pgfqpoint{5.442550in}{3.288356in}}%
\pgfpathlineto{\pgfqpoint{5.456495in}{3.290378in}}%
\pgfpathlineto{\pgfqpoint{5.470453in}{3.292572in}}%
\pgfpathlineto{\pgfqpoint{5.484425in}{3.294936in}}%
\pgfpathlineto{\pgfqpoint{5.491819in}{3.306671in}}%
\pgfpathlineto{\pgfqpoint{5.499216in}{3.318704in}}%
\pgfpathlineto{\pgfqpoint{5.506618in}{3.331041in}}%
\pgfpathlineto{\pgfqpoint{5.492666in}{3.329248in}}%
\pgfpathlineto{\pgfqpoint{5.478727in}{3.327624in}}%
\pgfpathlineto{\pgfqpoint{5.464802in}{3.326171in}}%
\pgfpathlineto{\pgfqpoint{5.450889in}{3.324888in}}%
\pgfpathlineto{\pgfqpoint{5.443462in}{3.311784in}}%
\pgfpathlineto{\pgfqpoint{5.436038in}{3.298992in}}%
\pgfpathlineto{\pgfqpoint{5.428619in}{3.286504in}}%
\pgfpathclose%
\pgfusepath{fill}%
\end{pgfscope}%
\begin{pgfscope}%
\pgfpathrectangle{\pgfqpoint{1.150000in}{0.150000in}}{\pgfqpoint{5.700000in}{5.700000in}}%
\pgfusepath{clip}%
\pgfsetbuttcap%
\pgfsetroundjoin%
\definecolor{currentfill}{rgb}{0.282884,0.135920,0.453427}%
\pgfsetfillcolor{currentfill}%
\pgfsetfillopacity{0.800000}%
\pgfsetlinewidth{0.000000pt}%
\definecolor{currentstroke}{rgb}{0.000000,0.000000,0.000000}%
\pgfsetstrokecolor{currentstroke}%
\pgfsetdash{}{0pt}%
\pgfpathmoveto{\pgfqpoint{3.587418in}{2.418953in}}%
\pgfpathlineto{\pgfqpoint{3.600829in}{2.412569in}}%
\pgfpathlineto{\pgfqpoint{3.614242in}{2.406411in}}%
\pgfpathlineto{\pgfqpoint{3.627659in}{2.400478in}}%
\pgfpathlineto{\pgfqpoint{3.641079in}{2.394769in}}%
\pgfpathlineto{\pgfqpoint{3.649006in}{2.405615in}}%
\pgfpathlineto{\pgfqpoint{3.656929in}{2.416510in}}%
\pgfpathlineto{\pgfqpoint{3.664846in}{2.427457in}}%
\pgfpathlineto{\pgfqpoint{3.672758in}{2.438457in}}%
\pgfpathlineto{\pgfqpoint{3.659348in}{2.444264in}}%
\pgfpathlineto{\pgfqpoint{3.645941in}{2.450294in}}%
\pgfpathlineto{\pgfqpoint{3.632537in}{2.456549in}}%
\pgfpathlineto{\pgfqpoint{3.619135in}{2.463030in}}%
\pgfpathlineto{\pgfqpoint{3.611214in}{2.451922in}}%
\pgfpathlineto{\pgfqpoint{3.603287in}{2.440874in}}%
\pgfpathlineto{\pgfqpoint{3.595355in}{2.429885in}}%
\pgfpathlineto{\pgfqpoint{3.587418in}{2.418953in}}%
\pgfpathclose%
\pgfusepath{fill}%
\end{pgfscope}%
\begin{pgfscope}%
\pgfpathrectangle{\pgfqpoint{1.150000in}{0.150000in}}{\pgfqpoint{5.700000in}{5.700000in}}%
\pgfusepath{clip}%
\pgfsetbuttcap%
\pgfsetroundjoin%
\definecolor{currentfill}{rgb}{0.282290,0.145912,0.461510}%
\pgfsetfillcolor{currentfill}%
\pgfsetfillopacity{0.800000}%
\pgfsetlinewidth{0.000000pt}%
\definecolor{currentstroke}{rgb}{0.000000,0.000000,0.000000}%
\pgfsetstrokecolor{currentstroke}%
\pgfsetdash{}{0pt}%
\pgfpathmoveto{\pgfqpoint{3.811696in}{2.443439in}}%
\pgfpathlineto{\pgfqpoint{3.825134in}{2.439464in}}%
\pgfpathlineto{\pgfqpoint{3.838576in}{2.435703in}}%
\pgfpathlineto{\pgfqpoint{3.852024in}{2.432153in}}%
\pgfpathlineto{\pgfqpoint{3.865477in}{2.428814in}}%
\pgfpathlineto{\pgfqpoint{3.873337in}{2.439658in}}%
\pgfpathlineto{\pgfqpoint{3.881191in}{2.450541in}}%
\pgfpathlineto{\pgfqpoint{3.889041in}{2.461467in}}%
\pgfpathlineto{\pgfqpoint{3.896886in}{2.472438in}}%
\pgfpathlineto{\pgfqpoint{3.883441in}{2.475938in}}%
\pgfpathlineto{\pgfqpoint{3.870002in}{2.479648in}}%
\pgfpathlineto{\pgfqpoint{3.856568in}{2.483570in}}%
\pgfpathlineto{\pgfqpoint{3.843139in}{2.487704in}}%
\pgfpathlineto{\pgfqpoint{3.835286in}{2.476562in}}%
\pgfpathlineto{\pgfqpoint{3.827428in}{2.465471in}}%
\pgfpathlineto{\pgfqpoint{3.819565in}{2.454431in}}%
\pgfpathlineto{\pgfqpoint{3.811696in}{2.443439in}}%
\pgfpathclose%
\pgfusepath{fill}%
\end{pgfscope}%
\begin{pgfscope}%
\pgfpathrectangle{\pgfqpoint{1.150000in}{0.150000in}}{\pgfqpoint{5.700000in}{5.700000in}}%
\pgfusepath{clip}%
\pgfsetbuttcap%
\pgfsetroundjoin%
\definecolor{currentfill}{rgb}{0.248629,0.278775,0.534556}%
\pgfsetfillcolor{currentfill}%
\pgfsetfillopacity{0.800000}%
\pgfsetlinewidth{0.000000pt}%
\definecolor{currentstroke}{rgb}{0.000000,0.000000,0.000000}%
\pgfsetstrokecolor{currentstroke}%
\pgfsetdash{}{0pt}%
\pgfpathmoveto{\pgfqpoint{4.546674in}{2.739738in}}%
\pgfpathlineto{\pgfqpoint{4.560317in}{2.740754in}}%
\pgfpathlineto{\pgfqpoint{4.573970in}{2.741956in}}%
\pgfpathlineto{\pgfqpoint{4.587633in}{2.743345in}}%
\pgfpathlineto{\pgfqpoint{4.601307in}{2.744920in}}%
\pgfpathlineto{\pgfqpoint{4.608935in}{2.754925in}}%
\pgfpathlineto{\pgfqpoint{4.616560in}{2.765010in}}%
\pgfpathlineto{\pgfqpoint{4.624181in}{2.775180in}}%
\pgfpathlineto{\pgfqpoint{4.631797in}{2.785439in}}%
\pgfpathlineto{\pgfqpoint{4.618135in}{2.784279in}}%
\pgfpathlineto{\pgfqpoint{4.604484in}{2.783304in}}%
\pgfpathlineto{\pgfqpoint{4.590842in}{2.782516in}}%
\pgfpathlineto{\pgfqpoint{4.577211in}{2.781914in}}%
\pgfpathlineto{\pgfqpoint{4.569583in}{2.771229in}}%
\pgfpathlineto{\pgfqpoint{4.561950in}{2.760641in}}%
\pgfpathlineto{\pgfqpoint{4.554314in}{2.750146in}}%
\pgfpathlineto{\pgfqpoint{4.546674in}{2.739738in}}%
\pgfpathclose%
\pgfusepath{fill}%
\end{pgfscope}%
\begin{pgfscope}%
\pgfpathrectangle{\pgfqpoint{1.150000in}{0.150000in}}{\pgfqpoint{5.700000in}{5.700000in}}%
\pgfusepath{clip}%
\pgfsetbuttcap%
\pgfsetroundjoin%
\definecolor{currentfill}{rgb}{0.255645,0.260703,0.528312}%
\pgfsetfillcolor{currentfill}%
\pgfsetfillopacity{0.800000}%
\pgfsetlinewidth{0.000000pt}%
\definecolor{currentstroke}{rgb}{0.000000,0.000000,0.000000}%
\pgfsetstrokecolor{currentstroke}%
\pgfsetdash{}{0pt}%
\pgfpathmoveto{\pgfqpoint{4.461559in}{2.695182in}}%
\pgfpathlineto{\pgfqpoint{4.475173in}{2.695827in}}%
\pgfpathlineto{\pgfqpoint{4.488796in}{2.696660in}}%
\pgfpathlineto{\pgfqpoint{4.502429in}{2.697682in}}%
\pgfpathlineto{\pgfqpoint{4.516073in}{2.698893in}}%
\pgfpathlineto{\pgfqpoint{4.523729in}{2.708995in}}%
\pgfpathlineto{\pgfqpoint{4.531382in}{2.719168in}}%
\pgfpathlineto{\pgfqpoint{4.539030in}{2.729414in}}%
\pgfpathlineto{\pgfqpoint{4.546674in}{2.739738in}}%
\pgfpathlineto{\pgfqpoint{4.533042in}{2.738911in}}%
\pgfpathlineto{\pgfqpoint{4.519419in}{2.738271in}}%
\pgfpathlineto{\pgfqpoint{4.505806in}{2.737820in}}%
\pgfpathlineto{\pgfqpoint{4.492204in}{2.737557in}}%
\pgfpathlineto{\pgfqpoint{4.484548in}{2.726839in}}%
\pgfpathlineto{\pgfqpoint{4.476889in}{2.716207in}}%
\pgfpathlineto{\pgfqpoint{4.469226in}{2.705656in}}%
\pgfpathlineto{\pgfqpoint{4.461559in}{2.695182in}}%
\pgfpathclose%
\pgfusepath{fill}%
\end{pgfscope}%
\begin{pgfscope}%
\pgfpathrectangle{\pgfqpoint{1.150000in}{0.150000in}}{\pgfqpoint{5.700000in}{5.700000in}}%
\pgfusepath{clip}%
\pgfsetbuttcap%
\pgfsetroundjoin%
\definecolor{currentfill}{rgb}{0.241237,0.296485,0.539709}%
\pgfsetfillcolor{currentfill}%
\pgfsetfillopacity{0.800000}%
\pgfsetlinewidth{0.000000pt}%
\definecolor{currentstroke}{rgb}{0.000000,0.000000,0.000000}%
\pgfsetstrokecolor{currentstroke}%
\pgfsetdash{}{0pt}%
\pgfpathmoveto{\pgfqpoint{4.631797in}{2.785439in}}%
\pgfpathlineto{\pgfqpoint{4.645470in}{2.786785in}}%
\pgfpathlineto{\pgfqpoint{4.659154in}{2.788317in}}%
\pgfpathlineto{\pgfqpoint{4.672848in}{2.790032in}}%
\pgfpathlineto{\pgfqpoint{4.686553in}{2.791933in}}%
\pgfpathlineto{\pgfqpoint{4.694154in}{2.801852in}}%
\pgfpathlineto{\pgfqpoint{4.701751in}{2.811864in}}%
\pgfpathlineto{\pgfqpoint{4.709344in}{2.821973in}}%
\pgfpathlineto{\pgfqpoint{4.716934in}{2.832184in}}%
\pgfpathlineto{\pgfqpoint{4.703241in}{2.830730in}}%
\pgfpathlineto{\pgfqpoint{4.689560in}{2.829461in}}%
\pgfpathlineto{\pgfqpoint{4.675889in}{2.828375in}}%
\pgfpathlineto{\pgfqpoint{4.662228in}{2.827474in}}%
\pgfpathlineto{\pgfqpoint{4.654626in}{2.816806in}}%
\pgfpathlineto{\pgfqpoint{4.647020in}{2.806247in}}%
\pgfpathlineto{\pgfqpoint{4.639411in}{2.795793in}}%
\pgfpathlineto{\pgfqpoint{4.631797in}{2.785439in}}%
\pgfpathclose%
\pgfusepath{fill}%
\end{pgfscope}%
\begin{pgfscope}%
\pgfpathrectangle{\pgfqpoint{1.150000in}{0.150000in}}{\pgfqpoint{5.700000in}{5.700000in}}%
\pgfusepath{clip}%
\pgfsetbuttcap%
\pgfsetroundjoin%
\definecolor{currentfill}{rgb}{0.262138,0.242286,0.520837}%
\pgfsetfillcolor{currentfill}%
\pgfsetfillopacity{0.800000}%
\pgfsetlinewidth{0.000000pt}%
\definecolor{currentstroke}{rgb}{0.000000,0.000000,0.000000}%
\pgfsetstrokecolor{currentstroke}%
\pgfsetdash{}{0pt}%
\pgfpathmoveto{\pgfqpoint{4.376448in}{2.651896in}}%
\pgfpathlineto{\pgfqpoint{4.390034in}{2.652128in}}%
\pgfpathlineto{\pgfqpoint{4.403629in}{2.652552in}}%
\pgfpathlineto{\pgfqpoint{4.417233in}{2.653167in}}%
\pgfpathlineto{\pgfqpoint{4.430847in}{2.653972in}}%
\pgfpathlineto{\pgfqpoint{4.438532in}{2.664180in}}%
\pgfpathlineto{\pgfqpoint{4.446212in}{2.674449in}}%
\pgfpathlineto{\pgfqpoint{4.453888in}{2.684781in}}%
\pgfpathlineto{\pgfqpoint{4.461559in}{2.695182in}}%
\pgfpathlineto{\pgfqpoint{4.447955in}{2.694728in}}%
\pgfpathlineto{\pgfqpoint{4.434361in}{2.694464in}}%
\pgfpathlineto{\pgfqpoint{4.420776in}{2.694390in}}%
\pgfpathlineto{\pgfqpoint{4.407200in}{2.694508in}}%
\pgfpathlineto{\pgfqpoint{4.399518in}{2.683745in}}%
\pgfpathlineto{\pgfqpoint{4.391833in}{2.673058in}}%
\pgfpathlineto{\pgfqpoint{4.384143in}{2.662443in}}%
\pgfpathlineto{\pgfqpoint{4.376448in}{2.651896in}}%
\pgfpathclose%
\pgfusepath{fill}%
\end{pgfscope}%
\begin{pgfscope}%
\pgfpathrectangle{\pgfqpoint{1.150000in}{0.150000in}}{\pgfqpoint{5.700000in}{5.700000in}}%
\pgfusepath{clip}%
\pgfsetbuttcap%
\pgfsetroundjoin%
\definecolor{currentfill}{rgb}{0.231674,0.318106,0.544834}%
\pgfsetfillcolor{currentfill}%
\pgfsetfillopacity{0.800000}%
\pgfsetlinewidth{0.000000pt}%
\definecolor{currentstroke}{rgb}{0.000000,0.000000,0.000000}%
\pgfsetstrokecolor{currentstroke}%
\pgfsetdash{}{0pt}%
\pgfpathmoveto{\pgfqpoint{4.716934in}{2.832184in}}%
\pgfpathlineto{\pgfqpoint{4.730637in}{2.833821in}}%
\pgfpathlineto{\pgfqpoint{4.744352in}{2.835642in}}%
\pgfpathlineto{\pgfqpoint{4.758078in}{2.837645in}}%
\pgfpathlineto{\pgfqpoint{4.771815in}{2.839831in}}%
\pgfpathlineto{\pgfqpoint{4.779388in}{2.849683in}}%
\pgfpathlineto{\pgfqpoint{4.786958in}{2.859641in}}%
\pgfpathlineto{\pgfqpoint{4.794524in}{2.869710in}}%
\pgfpathlineto{\pgfqpoint{4.802087in}{2.879895in}}%
\pgfpathlineto{\pgfqpoint{4.788363in}{2.878188in}}%
\pgfpathlineto{\pgfqpoint{4.774651in}{2.876662in}}%
\pgfpathlineto{\pgfqpoint{4.760950in}{2.875320in}}%
\pgfpathlineto{\pgfqpoint{4.747260in}{2.874160in}}%
\pgfpathlineto{\pgfqpoint{4.739683in}{2.863485in}}%
\pgfpathlineto{\pgfqpoint{4.732103in}{2.852935in}}%
\pgfpathlineto{\pgfqpoint{4.724520in}{2.842503in}}%
\pgfpathlineto{\pgfqpoint{4.716934in}{2.832184in}}%
\pgfpathclose%
\pgfusepath{fill}%
\end{pgfscope}%
\begin{pgfscope}%
\pgfpathrectangle{\pgfqpoint{1.150000in}{0.150000in}}{\pgfqpoint{5.700000in}{5.700000in}}%
\pgfusepath{clip}%
\pgfsetbuttcap%
\pgfsetroundjoin%
\definecolor{currentfill}{rgb}{0.281887,0.150881,0.465405}%
\pgfsetfillcolor{currentfill}%
\pgfsetfillopacity{0.800000}%
\pgfsetlinewidth{0.000000pt}%
\definecolor{currentstroke}{rgb}{0.000000,0.000000,0.000000}%
\pgfsetstrokecolor{currentstroke}%
\pgfsetdash{}{0pt}%
\pgfpathmoveto{\pgfqpoint{3.308989in}{2.465780in}}%
\pgfpathlineto{\pgfqpoint{3.322408in}{2.455693in}}%
\pgfpathlineto{\pgfqpoint{3.335826in}{2.445856in}}%
\pgfpathlineto{\pgfqpoint{3.349244in}{2.436266in}}%
\pgfpathlineto{\pgfqpoint{3.362661in}{2.426922in}}%
\pgfpathlineto{\pgfqpoint{3.370678in}{2.437506in}}%
\pgfpathlineto{\pgfqpoint{3.378688in}{2.448162in}}%
\pgfpathlineto{\pgfqpoint{3.386692in}{2.458891in}}%
\pgfpathlineto{\pgfqpoint{3.394691in}{2.469693in}}%
\pgfpathlineto{\pgfqpoint{3.381285in}{2.479071in}}%
\pgfpathlineto{\pgfqpoint{3.367880in}{2.488694in}}%
\pgfpathlineto{\pgfqpoint{3.354474in}{2.498565in}}%
\pgfpathlineto{\pgfqpoint{3.341068in}{2.508685in}}%
\pgfpathlineto{\pgfqpoint{3.333057in}{2.497837in}}%
\pgfpathlineto{\pgfqpoint{3.325041in}{2.487071in}}%
\pgfpathlineto{\pgfqpoint{3.317018in}{2.476386in}}%
\pgfpathlineto{\pgfqpoint{3.308989in}{2.465780in}}%
\pgfpathclose%
\pgfusepath{fill}%
\end{pgfscope}%
\begin{pgfscope}%
\pgfpathrectangle{\pgfqpoint{1.150000in}{0.150000in}}{\pgfqpoint{5.700000in}{5.700000in}}%
\pgfusepath{clip}%
\pgfsetbuttcap%
\pgfsetroundjoin%
\definecolor{currentfill}{rgb}{0.262138,0.242286,0.520837}%
\pgfsetfillcolor{currentfill}%
\pgfsetfillopacity{0.800000}%
\pgfsetlinewidth{0.000000pt}%
\definecolor{currentstroke}{rgb}{0.000000,0.000000,0.000000}%
\pgfsetstrokecolor{currentstroke}%
\pgfsetdash{}{0pt}%
\pgfpathmoveto{\pgfqpoint{3.007635in}{2.680981in}}%
\pgfpathlineto{\pgfqpoint{3.021133in}{2.665514in}}%
\pgfpathlineto{\pgfqpoint{3.034626in}{2.650338in}}%
\pgfpathlineto{\pgfqpoint{3.048113in}{2.635448in}}%
\pgfpathlineto{\pgfqpoint{3.061594in}{2.620843in}}%
\pgfpathlineto{\pgfqpoint{3.069708in}{2.631095in}}%
\pgfpathlineto{\pgfqpoint{3.077815in}{2.641455in}}%
\pgfpathlineto{\pgfqpoint{3.085914in}{2.651925in}}%
\pgfpathlineto{\pgfqpoint{3.094006in}{2.662505in}}%
\pgfpathlineto{\pgfqpoint{3.080540in}{2.677110in}}%
\pgfpathlineto{\pgfqpoint{3.067069in}{2.692000in}}%
\pgfpathlineto{\pgfqpoint{3.053592in}{2.707177in}}%
\pgfpathlineto{\pgfqpoint{3.040111in}{2.722643in}}%
\pgfpathlineto{\pgfqpoint{3.032003in}{2.712051in}}%
\pgfpathlineto{\pgfqpoint{3.023888in}{2.701577in}}%
\pgfpathlineto{\pgfqpoint{3.015765in}{2.691221in}}%
\pgfpathlineto{\pgfqpoint{3.007635in}{2.680981in}}%
\pgfpathclose%
\pgfusepath{fill}%
\end{pgfscope}%
\begin{pgfscope}%
\pgfpathrectangle{\pgfqpoint{1.150000in}{0.150000in}}{\pgfqpoint{5.700000in}{5.700000in}}%
\pgfusepath{clip}%
\pgfsetbuttcap%
\pgfsetroundjoin%
\definecolor{currentfill}{rgb}{0.252194,0.269783,0.531579}%
\pgfsetfillcolor{currentfill}%
\pgfsetfillopacity{0.800000}%
\pgfsetlinewidth{0.000000pt}%
\definecolor{currentstroke}{rgb}{0.000000,0.000000,0.000000}%
\pgfsetstrokecolor{currentstroke}%
\pgfsetdash{}{0pt}%
\pgfpathmoveto{\pgfqpoint{2.953581in}{2.745791in}}%
\pgfpathlineto{\pgfqpoint{2.967104in}{2.729142in}}%
\pgfpathlineto{\pgfqpoint{2.980621in}{2.712792in}}%
\pgfpathlineto{\pgfqpoint{2.994131in}{2.696739in}}%
\pgfpathlineto{\pgfqpoint{3.007635in}{2.680981in}}%
\pgfpathlineto{\pgfqpoint{3.015765in}{2.691221in}}%
\pgfpathlineto{\pgfqpoint{3.023888in}{2.701577in}}%
\pgfpathlineto{\pgfqpoint{3.032003in}{2.712051in}}%
\pgfpathlineto{\pgfqpoint{3.040111in}{2.722643in}}%
\pgfpathlineto{\pgfqpoint{3.026623in}{2.738401in}}%
\pgfpathlineto{\pgfqpoint{3.013130in}{2.754454in}}%
\pgfpathlineto{\pgfqpoint{2.999630in}{2.770803in}}%
\pgfpathlineto{\pgfqpoint{2.986124in}{2.787452in}}%
\pgfpathlineto{\pgfqpoint{2.978000in}{2.776849in}}%
\pgfpathlineto{\pgfqpoint{2.969868in}{2.766371in}}%
\pgfpathlineto{\pgfqpoint{2.961729in}{2.756019in}}%
\pgfpathlineto{\pgfqpoint{2.953581in}{2.745791in}}%
\pgfpathclose%
\pgfusepath{fill}%
\end{pgfscope}%
\begin{pgfscope}%
\pgfpathrectangle{\pgfqpoint{1.150000in}{0.150000in}}{\pgfqpoint{5.700000in}{5.700000in}}%
\pgfusepath{clip}%
\pgfsetbuttcap%
\pgfsetroundjoin%
\definecolor{currentfill}{rgb}{0.223925,0.334994,0.548053}%
\pgfsetfillcolor{currentfill}%
\pgfsetfillopacity{0.800000}%
\pgfsetlinewidth{0.000000pt}%
\definecolor{currentstroke}{rgb}{0.000000,0.000000,0.000000}%
\pgfsetstrokecolor{currentstroke}%
\pgfsetdash{}{0pt}%
\pgfpathmoveto{\pgfqpoint{4.802087in}{2.879895in}}%
\pgfpathlineto{\pgfqpoint{4.815822in}{2.881784in}}%
\pgfpathlineto{\pgfqpoint{4.829568in}{2.883855in}}%
\pgfpathlineto{\pgfqpoint{4.843326in}{2.886107in}}%
\pgfpathlineto{\pgfqpoint{4.857097in}{2.888539in}}%
\pgfpathlineto{\pgfqpoint{4.864642in}{2.898348in}}%
\pgfpathlineto{\pgfqpoint{4.872184in}{2.908277in}}%
\pgfpathlineto{\pgfqpoint{4.879724in}{2.918332in}}%
\pgfpathlineto{\pgfqpoint{4.887260in}{2.928519in}}%
\pgfpathlineto{\pgfqpoint{4.873505in}{2.926597in}}%
\pgfpathlineto{\pgfqpoint{4.859762in}{2.924855in}}%
\pgfpathlineto{\pgfqpoint{4.846030in}{2.923294in}}%
\pgfpathlineto{\pgfqpoint{4.832310in}{2.921914in}}%
\pgfpathlineto{\pgfqpoint{4.824758in}{2.911206in}}%
\pgfpathlineto{\pgfqpoint{4.817204in}{2.900638in}}%
\pgfpathlineto{\pgfqpoint{4.809647in}{2.890202in}}%
\pgfpathlineto{\pgfqpoint{4.802087in}{2.879895in}}%
\pgfpathclose%
\pgfusepath{fill}%
\end{pgfscope}%
\begin{pgfscope}%
\pgfpathrectangle{\pgfqpoint{1.150000in}{0.150000in}}{\pgfqpoint{5.700000in}{5.700000in}}%
\pgfusepath{clip}%
\pgfsetbuttcap%
\pgfsetroundjoin%
\definecolor{currentfill}{rgb}{0.267968,0.223549,0.512008}%
\pgfsetfillcolor{currentfill}%
\pgfsetfillopacity{0.800000}%
\pgfsetlinewidth{0.000000pt}%
\definecolor{currentstroke}{rgb}{0.000000,0.000000,0.000000}%
\pgfsetstrokecolor{currentstroke}%
\pgfsetdash{}{0pt}%
\pgfpathmoveto{\pgfqpoint{4.291335in}{2.610027in}}%
\pgfpathlineto{\pgfqpoint{4.304895in}{2.609806in}}%
\pgfpathlineto{\pgfqpoint{4.318463in}{2.609779in}}%
\pgfpathlineto{\pgfqpoint{4.332040in}{2.609946in}}%
\pgfpathlineto{\pgfqpoint{4.345626in}{2.610306in}}%
\pgfpathlineto{\pgfqpoint{4.353338in}{2.620621in}}%
\pgfpathlineto{\pgfqpoint{4.361046in}{2.630989in}}%
\pgfpathlineto{\pgfqpoint{4.368749in}{2.641412in}}%
\pgfpathlineto{\pgfqpoint{4.376448in}{2.651896in}}%
\pgfpathlineto{\pgfqpoint{4.362872in}{2.651855in}}%
\pgfpathlineto{\pgfqpoint{4.349304in}{2.652007in}}%
\pgfpathlineto{\pgfqpoint{4.335745in}{2.652353in}}%
\pgfpathlineto{\pgfqpoint{4.322195in}{2.652891in}}%
\pgfpathlineto{\pgfqpoint{4.314487in}{2.642078in}}%
\pgfpathlineto{\pgfqpoint{4.306774in}{2.631332in}}%
\pgfpathlineto{\pgfqpoint{4.299057in}{2.620649in}}%
\pgfpathlineto{\pgfqpoint{4.291335in}{2.610027in}}%
\pgfpathclose%
\pgfusepath{fill}%
\end{pgfscope}%
\begin{pgfscope}%
\pgfpathrectangle{\pgfqpoint{1.150000in}{0.150000in}}{\pgfqpoint{5.700000in}{5.700000in}}%
\pgfusepath{clip}%
\pgfsetbuttcap%
\pgfsetroundjoin%
\definecolor{currentfill}{rgb}{0.269308,0.218818,0.509577}%
\pgfsetfillcolor{currentfill}%
\pgfsetfillopacity{0.800000}%
\pgfsetlinewidth{0.000000pt}%
\definecolor{currentstroke}{rgb}{0.000000,0.000000,0.000000}%
\pgfsetstrokecolor{currentstroke}%
\pgfsetdash{}{0pt}%
\pgfpathmoveto{\pgfqpoint{3.061594in}{2.620843in}}%
\pgfpathlineto{\pgfqpoint{3.075071in}{2.606521in}}%
\pgfpathlineto{\pgfqpoint{3.088544in}{2.592479in}}%
\pgfpathlineto{\pgfqpoint{3.102012in}{2.578715in}}%
\pgfpathlineto{\pgfqpoint{3.115476in}{2.565227in}}%
\pgfpathlineto{\pgfqpoint{3.123574in}{2.575490in}}%
\pgfpathlineto{\pgfqpoint{3.131665in}{2.585853in}}%
\pgfpathlineto{\pgfqpoint{3.139748in}{2.596318in}}%
\pgfpathlineto{\pgfqpoint{3.147825in}{2.606885in}}%
\pgfpathlineto{\pgfqpoint{3.134376in}{2.620374in}}%
\pgfpathlineto{\pgfqpoint{3.120924in}{2.634139in}}%
\pgfpathlineto{\pgfqpoint{3.107467in}{2.648182in}}%
\pgfpathlineto{\pgfqpoint{3.094006in}{2.662505in}}%
\pgfpathlineto{\pgfqpoint{3.085914in}{2.651925in}}%
\pgfpathlineto{\pgfqpoint{3.077815in}{2.641455in}}%
\pgfpathlineto{\pgfqpoint{3.069708in}{2.631095in}}%
\pgfpathlineto{\pgfqpoint{3.061594in}{2.620843in}}%
\pgfpathclose%
\pgfusepath{fill}%
\end{pgfscope}%
\begin{pgfscope}%
\pgfpathrectangle{\pgfqpoint{1.150000in}{0.150000in}}{\pgfqpoint{5.700000in}{5.700000in}}%
\pgfusepath{clip}%
\pgfsetbuttcap%
\pgfsetroundjoin%
\definecolor{currentfill}{rgb}{0.241237,0.296485,0.539709}%
\pgfsetfillcolor{currentfill}%
\pgfsetfillopacity{0.800000}%
\pgfsetlinewidth{0.000000pt}%
\definecolor{currentstroke}{rgb}{0.000000,0.000000,0.000000}%
\pgfsetstrokecolor{currentstroke}%
\pgfsetdash{}{0pt}%
\pgfpathmoveto{\pgfqpoint{2.899415in}{2.815438in}}%
\pgfpathlineto{\pgfqpoint{2.912968in}{2.797564in}}%
\pgfpathlineto{\pgfqpoint{2.926513in}{2.779999in}}%
\pgfpathlineto{\pgfqpoint{2.940051in}{2.762743in}}%
\pgfpathlineto{\pgfqpoint{2.953581in}{2.745791in}}%
\pgfpathlineto{\pgfqpoint{2.961729in}{2.756019in}}%
\pgfpathlineto{\pgfqpoint{2.969868in}{2.766371in}}%
\pgfpathlineto{\pgfqpoint{2.978000in}{2.776849in}}%
\pgfpathlineto{\pgfqpoint{2.986124in}{2.787452in}}%
\pgfpathlineto{\pgfqpoint{2.972611in}{2.804403in}}%
\pgfpathlineto{\pgfqpoint{2.959091in}{2.821659in}}%
\pgfpathlineto{\pgfqpoint{2.945564in}{2.839223in}}%
\pgfpathlineto{\pgfqpoint{2.932029in}{2.857097in}}%
\pgfpathlineto{\pgfqpoint{2.923887in}{2.846482in}}%
\pgfpathlineto{\pgfqpoint{2.915738in}{2.836001in}}%
\pgfpathlineto{\pgfqpoint{2.907581in}{2.825653in}}%
\pgfpathlineto{\pgfqpoint{2.899415in}{2.815438in}}%
\pgfpathclose%
\pgfusepath{fill}%
\end{pgfscope}%
\begin{pgfscope}%
\pgfpathrectangle{\pgfqpoint{1.150000in}{0.150000in}}{\pgfqpoint{5.700000in}{5.700000in}}%
\pgfusepath{clip}%
\pgfsetbuttcap%
\pgfsetroundjoin%
\definecolor{currentfill}{rgb}{0.214298,0.355619,0.551184}%
\pgfsetfillcolor{currentfill}%
\pgfsetfillopacity{0.800000}%
\pgfsetlinewidth{0.000000pt}%
\definecolor{currentstroke}{rgb}{0.000000,0.000000,0.000000}%
\pgfsetstrokecolor{currentstroke}%
\pgfsetdash{}{0pt}%
\pgfpathmoveto{\pgfqpoint{4.887260in}{2.928519in}}%
\pgfpathlineto{\pgfqpoint{4.901027in}{2.930621in}}%
\pgfpathlineto{\pgfqpoint{4.914806in}{2.932903in}}%
\pgfpathlineto{\pgfqpoint{4.928597in}{2.935365in}}%
\pgfpathlineto{\pgfqpoint{4.942400in}{2.938006in}}%
\pgfpathlineto{\pgfqpoint{4.949918in}{2.947800in}}%
\pgfpathlineto{\pgfqpoint{4.957434in}{2.957731in}}%
\pgfpathlineto{\pgfqpoint{4.964947in}{2.967804in}}%
\pgfpathlineto{\pgfqpoint{4.972458in}{2.978026in}}%
\pgfpathlineto{\pgfqpoint{4.958672in}{2.975928in}}%
\pgfpathlineto{\pgfqpoint{4.944897in}{2.974008in}}%
\pgfpathlineto{\pgfqpoint{4.931134in}{2.972268in}}%
\pgfpathlineto{\pgfqpoint{4.917383in}{2.970707in}}%
\pgfpathlineto{\pgfqpoint{4.909856in}{2.959932in}}%
\pgfpathlineto{\pgfqpoint{4.902326in}{2.949313in}}%
\pgfpathlineto{\pgfqpoint{4.894794in}{2.938844in}}%
\pgfpathlineto{\pgfqpoint{4.887260in}{2.928519in}}%
\pgfpathclose%
\pgfusepath{fill}%
\end{pgfscope}%
\begin{pgfscope}%
\pgfpathrectangle{\pgfqpoint{1.150000in}{0.150000in}}{\pgfqpoint{5.700000in}{5.700000in}}%
\pgfusepath{clip}%
\pgfsetbuttcap%
\pgfsetroundjoin%
\definecolor{currentfill}{rgb}{0.271828,0.209303,0.504434}%
\pgfsetfillcolor{currentfill}%
\pgfsetfillopacity{0.800000}%
\pgfsetlinewidth{0.000000pt}%
\definecolor{currentstroke}{rgb}{0.000000,0.000000,0.000000}%
\pgfsetstrokecolor{currentstroke}%
\pgfsetdash{}{0pt}%
\pgfpathmoveto{\pgfqpoint{4.206214in}{2.569747in}}%
\pgfpathlineto{\pgfqpoint{4.219749in}{2.569032in}}%
\pgfpathlineto{\pgfqpoint{4.233292in}{2.568513in}}%
\pgfpathlineto{\pgfqpoint{4.246843in}{2.568190in}}%
\pgfpathlineto{\pgfqpoint{4.260403in}{2.568062in}}%
\pgfpathlineto{\pgfqpoint{4.268143in}{2.578481in}}%
\pgfpathlineto{\pgfqpoint{4.275878in}{2.588946in}}%
\pgfpathlineto{\pgfqpoint{4.283609in}{2.599460in}}%
\pgfpathlineto{\pgfqpoint{4.291335in}{2.610027in}}%
\pgfpathlineto{\pgfqpoint{4.277785in}{2.610442in}}%
\pgfpathlineto{\pgfqpoint{4.264242in}{2.611052in}}%
\pgfpathlineto{\pgfqpoint{4.250708in}{2.611858in}}%
\pgfpathlineto{\pgfqpoint{4.237182in}{2.612860in}}%
\pgfpathlineto{\pgfqpoint{4.229447in}{2.601995in}}%
\pgfpathlineto{\pgfqpoint{4.221708in}{2.591190in}}%
\pgfpathlineto{\pgfqpoint{4.213963in}{2.580442in}}%
\pgfpathlineto{\pgfqpoint{4.206214in}{2.569747in}}%
\pgfpathclose%
\pgfusepath{fill}%
\end{pgfscope}%
\begin{pgfscope}%
\pgfpathrectangle{\pgfqpoint{1.150000in}{0.150000in}}{\pgfqpoint{5.700000in}{5.700000in}}%
\pgfusepath{clip}%
\pgfsetbuttcap%
\pgfsetroundjoin%
\definecolor{currentfill}{rgb}{0.206756,0.371758,0.553117}%
\pgfsetfillcolor{currentfill}%
\pgfsetfillopacity{0.800000}%
\pgfsetlinewidth{0.000000pt}%
\definecolor{currentstroke}{rgb}{0.000000,0.000000,0.000000}%
\pgfsetstrokecolor{currentstroke}%
\pgfsetdash{}{0pt}%
\pgfpathmoveto{\pgfqpoint{4.972458in}{2.978026in}}%
\pgfpathlineto{\pgfqpoint{4.986257in}{2.980303in}}%
\pgfpathlineto{\pgfqpoint{5.000068in}{2.982758in}}%
\pgfpathlineto{\pgfqpoint{5.013892in}{2.985391in}}%
\pgfpathlineto{\pgfqpoint{5.027728in}{2.988202in}}%
\pgfpathlineto{\pgfqpoint{5.035220in}{2.998016in}}%
\pgfpathlineto{\pgfqpoint{5.042710in}{3.007985in}}%
\pgfpathlineto{\pgfqpoint{5.050198in}{3.018114in}}%
\pgfpathlineto{\pgfqpoint{5.057685in}{3.028410in}}%
\pgfpathlineto{\pgfqpoint{5.043866in}{3.026174in}}%
\pgfpathlineto{\pgfqpoint{5.030060in}{3.024115in}}%
\pgfpathlineto{\pgfqpoint{5.016267in}{3.022233in}}%
\pgfpathlineto{\pgfqpoint{5.002485in}{3.020529in}}%
\pgfpathlineto{\pgfqpoint{4.994981in}{3.009648in}}%
\pgfpathlineto{\pgfqpoint{4.987475in}{2.998942in}}%
\pgfpathlineto{\pgfqpoint{4.979967in}{2.988403in}}%
\pgfpathlineto{\pgfqpoint{4.972458in}{2.978026in}}%
\pgfpathclose%
\pgfusepath{fill}%
\end{pgfscope}%
\begin{pgfscope}%
\pgfpathrectangle{\pgfqpoint{1.150000in}{0.150000in}}{\pgfqpoint{5.700000in}{5.700000in}}%
\pgfusepath{clip}%
\pgfsetbuttcap%
\pgfsetroundjoin%
\definecolor{currentfill}{rgb}{0.282884,0.135920,0.453427}%
\pgfsetfillcolor{currentfill}%
\pgfsetfillopacity{0.800000}%
\pgfsetlinewidth{0.000000pt}%
\definecolor{currentstroke}{rgb}{0.000000,0.000000,0.000000}%
\pgfsetstrokecolor{currentstroke}%
\pgfsetdash{}{0pt}%
\pgfpathmoveto{\pgfqpoint{3.726433in}{2.417441in}}%
\pgfpathlineto{\pgfqpoint{3.739862in}{2.412734in}}%
\pgfpathlineto{\pgfqpoint{3.753294in}{2.408245in}}%
\pgfpathlineto{\pgfqpoint{3.766731in}{2.403971in}}%
\pgfpathlineto{\pgfqpoint{3.780173in}{2.399912in}}%
\pgfpathlineto{\pgfqpoint{3.788061in}{2.410731in}}%
\pgfpathlineto{\pgfqpoint{3.795945in}{2.421591in}}%
\pgfpathlineto{\pgfqpoint{3.803823in}{2.432493in}}%
\pgfpathlineto{\pgfqpoint{3.811696in}{2.443439in}}%
\pgfpathlineto{\pgfqpoint{3.798264in}{2.447627in}}%
\pgfpathlineto{\pgfqpoint{3.784835in}{2.452030in}}%
\pgfpathlineto{\pgfqpoint{3.771412in}{2.456648in}}%
\pgfpathlineto{\pgfqpoint{3.757992in}{2.461484in}}%
\pgfpathlineto{\pgfqpoint{3.750110in}{2.450397in}}%
\pgfpathlineto{\pgfqpoint{3.742223in}{2.439362in}}%
\pgfpathlineto{\pgfqpoint{3.734331in}{2.428377in}}%
\pgfpathlineto{\pgfqpoint{3.726433in}{2.417441in}}%
\pgfpathclose%
\pgfusepath{fill}%
\end{pgfscope}%
\begin{pgfscope}%
\pgfpathrectangle{\pgfqpoint{1.150000in}{0.150000in}}{\pgfqpoint{5.700000in}{5.700000in}}%
\pgfusepath{clip}%
\pgfsetbuttcap%
\pgfsetroundjoin%
\definecolor{currentfill}{rgb}{0.276194,0.190074,0.493001}%
\pgfsetfillcolor{currentfill}%
\pgfsetfillopacity{0.800000}%
\pgfsetlinewidth{0.000000pt}%
\definecolor{currentstroke}{rgb}{0.000000,0.000000,0.000000}%
\pgfsetstrokecolor{currentstroke}%
\pgfsetdash{}{0pt}%
\pgfpathmoveto{\pgfqpoint{4.121078in}{2.531254in}}%
\pgfpathlineto{\pgfqpoint{4.134590in}{2.530001in}}%
\pgfpathlineto{\pgfqpoint{4.148109in}{2.528948in}}%
\pgfpathlineto{\pgfqpoint{4.161637in}{2.528093in}}%
\pgfpathlineto{\pgfqpoint{4.175172in}{2.527436in}}%
\pgfpathlineto{\pgfqpoint{4.182940in}{2.537950in}}%
\pgfpathlineto{\pgfqpoint{4.190703in}{2.548505in}}%
\pgfpathlineto{\pgfqpoint{4.198461in}{2.559103in}}%
\pgfpathlineto{\pgfqpoint{4.206214in}{2.569747in}}%
\pgfpathlineto{\pgfqpoint{4.192688in}{2.570660in}}%
\pgfpathlineto{\pgfqpoint{4.179169in}{2.571770in}}%
\pgfpathlineto{\pgfqpoint{4.165658in}{2.573079in}}%
\pgfpathlineto{\pgfqpoint{4.152154in}{2.574587in}}%
\pgfpathlineto{\pgfqpoint{4.144392in}{2.563676in}}%
\pgfpathlineto{\pgfqpoint{4.136625in}{2.552818in}}%
\pgfpathlineto{\pgfqpoint{4.128854in}{2.542012in}}%
\pgfpathlineto{\pgfqpoint{4.121078in}{2.531254in}}%
\pgfpathclose%
\pgfusepath{fill}%
\end{pgfscope}%
\begin{pgfscope}%
\pgfpathrectangle{\pgfqpoint{1.150000in}{0.150000in}}{\pgfqpoint{5.700000in}{5.700000in}}%
\pgfusepath{clip}%
\pgfsetbuttcap%
\pgfsetroundjoin%
\definecolor{currentfill}{rgb}{0.275191,0.194905,0.496005}%
\pgfsetfillcolor{currentfill}%
\pgfsetfillopacity{0.800000}%
\pgfsetlinewidth{0.000000pt}%
\definecolor{currentstroke}{rgb}{0.000000,0.000000,0.000000}%
\pgfsetstrokecolor{currentstroke}%
\pgfsetdash{}{0pt}%
\pgfpathmoveto{\pgfqpoint{3.115476in}{2.565227in}}%
\pgfpathlineto{\pgfqpoint{3.128936in}{2.552013in}}%
\pgfpathlineto{\pgfqpoint{3.142393in}{2.539070in}}%
\pgfpathlineto{\pgfqpoint{3.155846in}{2.526397in}}%
\pgfpathlineto{\pgfqpoint{3.169296in}{2.513991in}}%
\pgfpathlineto{\pgfqpoint{3.177378in}{2.524264in}}%
\pgfpathlineto{\pgfqpoint{3.185454in}{2.534630in}}%
\pgfpathlineto{\pgfqpoint{3.193523in}{2.545090in}}%
\pgfpathlineto{\pgfqpoint{3.201586in}{2.555644in}}%
\pgfpathlineto{\pgfqpoint{3.188150in}{2.568051in}}%
\pgfpathlineto{\pgfqpoint{3.174712in}{2.580726in}}%
\pgfpathlineto{\pgfqpoint{3.161270in}{2.593670in}}%
\pgfpathlineto{\pgfqpoint{3.147825in}{2.606885in}}%
\pgfpathlineto{\pgfqpoint{3.139748in}{2.596318in}}%
\pgfpathlineto{\pgfqpoint{3.131665in}{2.585853in}}%
\pgfpathlineto{\pgfqpoint{3.123574in}{2.575490in}}%
\pgfpathlineto{\pgfqpoint{3.115476in}{2.565227in}}%
\pgfpathclose%
\pgfusepath{fill}%
\end{pgfscope}%
\begin{pgfscope}%
\pgfpathrectangle{\pgfqpoint{1.150000in}{0.150000in}}{\pgfqpoint{5.700000in}{5.700000in}}%
\pgfusepath{clip}%
\pgfsetbuttcap%
\pgfsetroundjoin%
\definecolor{currentfill}{rgb}{0.197636,0.391528,0.554969}%
\pgfsetfillcolor{currentfill}%
\pgfsetfillopacity{0.800000}%
\pgfsetlinewidth{0.000000pt}%
\definecolor{currentstroke}{rgb}{0.000000,0.000000,0.000000}%
\pgfsetstrokecolor{currentstroke}%
\pgfsetdash{}{0pt}%
\pgfpathmoveto{\pgfqpoint{5.057685in}{3.028410in}}%
\pgfpathlineto{\pgfqpoint{5.071515in}{3.030823in}}%
\pgfpathlineto{\pgfqpoint{5.085359in}{3.033413in}}%
\pgfpathlineto{\pgfqpoint{5.099215in}{3.036180in}}%
\pgfpathlineto{\pgfqpoint{5.113084in}{3.039122in}}%
\pgfpathlineto{\pgfqpoint{5.120551in}{3.048998in}}%
\pgfpathlineto{\pgfqpoint{5.128016in}{3.059046in}}%
\pgfpathlineto{\pgfqpoint{5.135480in}{3.069274in}}%
\pgfpathlineto{\pgfqpoint{5.142944in}{3.079688in}}%
\pgfpathlineto{\pgfqpoint{5.129094in}{3.077352in}}%
\pgfpathlineto{\pgfqpoint{5.115257in}{3.075191in}}%
\pgfpathlineto{\pgfqpoint{5.101432in}{3.073207in}}%
\pgfpathlineto{\pgfqpoint{5.087620in}{3.071398in}}%
\pgfpathlineto{\pgfqpoint{5.080137in}{3.060367in}}%
\pgfpathlineto{\pgfqpoint{5.072654in}{3.049530in}}%
\pgfpathlineto{\pgfqpoint{5.065170in}{3.038880in}}%
\pgfpathlineto{\pgfqpoint{5.057685in}{3.028410in}}%
\pgfpathclose%
\pgfusepath{fill}%
\end{pgfscope}%
\begin{pgfscope}%
\pgfpathrectangle{\pgfqpoint{1.150000in}{0.150000in}}{\pgfqpoint{5.700000in}{5.700000in}}%
\pgfusepath{clip}%
\pgfsetbuttcap%
\pgfsetroundjoin%
\definecolor{currentfill}{rgb}{0.227802,0.326594,0.546532}%
\pgfsetfillcolor{currentfill}%
\pgfsetfillopacity{0.800000}%
\pgfsetlinewidth{0.000000pt}%
\definecolor{currentstroke}{rgb}{0.000000,0.000000,0.000000}%
\pgfsetstrokecolor{currentstroke}%
\pgfsetdash{}{0pt}%
\pgfpathmoveto{\pgfqpoint{2.845118in}{2.890099in}}%
\pgfpathlineto{\pgfqpoint{2.858706in}{2.870954in}}%
\pgfpathlineto{\pgfqpoint{2.872284in}{2.852131in}}%
\pgfpathlineto{\pgfqpoint{2.885854in}{2.833626in}}%
\pgfpathlineto{\pgfqpoint{2.899415in}{2.815438in}}%
\pgfpathlineto{\pgfqpoint{2.907581in}{2.825653in}}%
\pgfpathlineto{\pgfqpoint{2.915738in}{2.836001in}}%
\pgfpathlineto{\pgfqpoint{2.923887in}{2.846482in}}%
\pgfpathlineto{\pgfqpoint{2.932029in}{2.857097in}}%
\pgfpathlineto{\pgfqpoint{2.918485in}{2.875284in}}%
\pgfpathlineto{\pgfqpoint{2.904934in}{2.893786in}}%
\pgfpathlineto{\pgfqpoint{2.891374in}{2.912608in}}%
\pgfpathlineto{\pgfqpoint{2.877805in}{2.931752in}}%
\pgfpathlineto{\pgfqpoint{2.869646in}{2.921127in}}%
\pgfpathlineto{\pgfqpoint{2.861479in}{2.910643in}}%
\pgfpathlineto{\pgfqpoint{2.853303in}{2.900301in}}%
\pgfpathlineto{\pgfqpoint{2.845118in}{2.890099in}}%
\pgfpathclose%
\pgfusepath{fill}%
\end{pgfscope}%
\begin{pgfscope}%
\pgfpathrectangle{\pgfqpoint{1.150000in}{0.150000in}}{\pgfqpoint{5.700000in}{5.700000in}}%
\pgfusepath{clip}%
\pgfsetbuttcap%
\pgfsetroundjoin%
\definecolor{currentfill}{rgb}{0.188923,0.410910,0.556326}%
\pgfsetfillcolor{currentfill}%
\pgfsetfillopacity{0.800000}%
\pgfsetlinewidth{0.000000pt}%
\definecolor{currentstroke}{rgb}{0.000000,0.000000,0.000000}%
\pgfsetstrokecolor{currentstroke}%
\pgfsetdash{}{0pt}%
\pgfpathmoveto{\pgfqpoint{5.142944in}{3.079688in}}%
\pgfpathlineto{\pgfqpoint{5.156807in}{3.082200in}}%
\pgfpathlineto{\pgfqpoint{5.170682in}{3.084887in}}%
\pgfpathlineto{\pgfqpoint{5.184571in}{3.087749in}}%
\pgfpathlineto{\pgfqpoint{5.198473in}{3.090787in}}%
\pgfpathlineto{\pgfqpoint{5.205916in}{3.100768in}}%
\pgfpathlineto{\pgfqpoint{5.213358in}{3.110943in}}%
\pgfpathlineto{\pgfqpoint{5.220799in}{3.121319in}}%
\pgfpathlineto{\pgfqpoint{5.228241in}{3.131902in}}%
\pgfpathlineto{\pgfqpoint{5.214360in}{3.129503in}}%
\pgfpathlineto{\pgfqpoint{5.200492in}{3.127279in}}%
\pgfpathlineto{\pgfqpoint{5.186636in}{3.125229in}}%
\pgfpathlineto{\pgfqpoint{5.172794in}{3.123354in}}%
\pgfpathlineto{\pgfqpoint{5.165331in}{3.112122in}}%
\pgfpathlineto{\pgfqpoint{5.157869in}{3.101105in}}%
\pgfpathlineto{\pgfqpoint{5.150407in}{3.090296in}}%
\pgfpathlineto{\pgfqpoint{5.142944in}{3.079688in}}%
\pgfpathclose%
\pgfusepath{fill}%
\end{pgfscope}%
\begin{pgfscope}%
\pgfpathrectangle{\pgfqpoint{1.150000in}{0.150000in}}{\pgfqpoint{5.700000in}{5.700000in}}%
\pgfusepath{clip}%
\pgfsetbuttcap%
\pgfsetroundjoin%
\definecolor{currentfill}{rgb}{0.278826,0.175490,0.483397}%
\pgfsetfillcolor{currentfill}%
\pgfsetfillopacity{0.800000}%
\pgfsetlinewidth{0.000000pt}%
\definecolor{currentstroke}{rgb}{0.000000,0.000000,0.000000}%
\pgfsetstrokecolor{currentstroke}%
\pgfsetdash{}{0pt}%
\pgfpathmoveto{\pgfqpoint{4.035916in}{2.494767in}}%
\pgfpathlineto{\pgfqpoint{4.049408in}{2.492933in}}%
\pgfpathlineto{\pgfqpoint{4.062906in}{2.491302in}}%
\pgfpathlineto{\pgfqpoint{4.076412in}{2.489872in}}%
\pgfpathlineto{\pgfqpoint{4.089925in}{2.488643in}}%
\pgfpathlineto{\pgfqpoint{4.097720in}{2.499238in}}%
\pgfpathlineto{\pgfqpoint{4.105511in}{2.509870in}}%
\pgfpathlineto{\pgfqpoint{4.113297in}{2.520541in}}%
\pgfpathlineto{\pgfqpoint{4.121078in}{2.531254in}}%
\pgfpathlineto{\pgfqpoint{4.107573in}{2.532707in}}%
\pgfpathlineto{\pgfqpoint{4.094076in}{2.534361in}}%
\pgfpathlineto{\pgfqpoint{4.080586in}{2.536216in}}%
\pgfpathlineto{\pgfqpoint{4.067103in}{2.538273in}}%
\pgfpathlineto{\pgfqpoint{4.059313in}{2.527325in}}%
\pgfpathlineto{\pgfqpoint{4.051519in}{2.516426in}}%
\pgfpathlineto{\pgfqpoint{4.043720in}{2.505575in}}%
\pgfpathlineto{\pgfqpoint{4.035916in}{2.494767in}}%
\pgfpathclose%
\pgfusepath{fill}%
\end{pgfscope}%
\begin{pgfscope}%
\pgfpathrectangle{\pgfqpoint{1.150000in}{0.150000in}}{\pgfqpoint{5.700000in}{5.700000in}}%
\pgfusepath{clip}%
\pgfsetbuttcap%
\pgfsetroundjoin%
\definecolor{currentfill}{rgb}{0.283072,0.130895,0.449241}%
\pgfsetfillcolor{currentfill}%
\pgfsetfillopacity{0.800000}%
\pgfsetlinewidth{0.000000pt}%
\definecolor{currentstroke}{rgb}{0.000000,0.000000,0.000000}%
\pgfsetstrokecolor{currentstroke}%
\pgfsetdash{}{0pt}%
\pgfpathmoveto{\pgfqpoint{3.501950in}{2.403330in}}%
\pgfpathlineto{\pgfqpoint{3.515363in}{2.396096in}}%
\pgfpathlineto{\pgfqpoint{3.528778in}{2.389092in}}%
\pgfpathlineto{\pgfqpoint{3.542194in}{2.382319in}}%
\pgfpathlineto{\pgfqpoint{3.555613in}{2.375774in}}%
\pgfpathlineto{\pgfqpoint{3.563572in}{2.386489in}}%
\pgfpathlineto{\pgfqpoint{3.571526in}{2.397257in}}%
\pgfpathlineto{\pgfqpoint{3.579475in}{2.408078in}}%
\pgfpathlineto{\pgfqpoint{3.587418in}{2.418953in}}%
\pgfpathlineto{\pgfqpoint{3.574009in}{2.425564in}}%
\pgfpathlineto{\pgfqpoint{3.560603in}{2.432404in}}%
\pgfpathlineto{\pgfqpoint{3.547199in}{2.439473in}}%
\pgfpathlineto{\pgfqpoint{3.533797in}{2.446773in}}%
\pgfpathlineto{\pgfqpoint{3.525843in}{2.435820in}}%
\pgfpathlineto{\pgfqpoint{3.517885in}{2.424929in}}%
\pgfpathlineto{\pgfqpoint{3.509920in}{2.414100in}}%
\pgfpathlineto{\pgfqpoint{3.501950in}{2.403330in}}%
\pgfpathclose%
\pgfusepath{fill}%
\end{pgfscope}%
\begin{pgfscope}%
\pgfpathrectangle{\pgfqpoint{1.150000in}{0.150000in}}{\pgfqpoint{5.700000in}{5.700000in}}%
\pgfusepath{clip}%
\pgfsetbuttcap%
\pgfsetroundjoin%
\definecolor{currentfill}{rgb}{0.282623,0.140926,0.457517}%
\pgfsetfillcolor{currentfill}%
\pgfsetfillopacity{0.800000}%
\pgfsetlinewidth{0.000000pt}%
\definecolor{currentstroke}{rgb}{0.000000,0.000000,0.000000}%
\pgfsetstrokecolor{currentstroke}%
\pgfsetdash{}{0pt}%
\pgfpathmoveto{\pgfqpoint{3.362661in}{2.426922in}}%
\pgfpathlineto{\pgfqpoint{3.376079in}{2.417823in}}%
\pgfpathlineto{\pgfqpoint{3.389496in}{2.408965in}}%
\pgfpathlineto{\pgfqpoint{3.402914in}{2.400349in}}%
\pgfpathlineto{\pgfqpoint{3.416332in}{2.391973in}}%
\pgfpathlineto{\pgfqpoint{3.424336in}{2.402534in}}%
\pgfpathlineto{\pgfqpoint{3.432334in}{2.413160in}}%
\pgfpathlineto{\pgfqpoint{3.440327in}{2.423850in}}%
\pgfpathlineto{\pgfqpoint{3.448314in}{2.434607in}}%
\pgfpathlineto{\pgfqpoint{3.434907in}{2.443018in}}%
\pgfpathlineto{\pgfqpoint{3.421501in}{2.451668in}}%
\pgfpathlineto{\pgfqpoint{3.408096in}{2.460559in}}%
\pgfpathlineto{\pgfqpoint{3.394691in}{2.469693in}}%
\pgfpathlineto{\pgfqpoint{3.386692in}{2.458891in}}%
\pgfpathlineto{\pgfqpoint{3.378688in}{2.448162in}}%
\pgfpathlineto{\pgfqpoint{3.370678in}{2.437506in}}%
\pgfpathlineto{\pgfqpoint{3.362661in}{2.426922in}}%
\pgfpathclose%
\pgfusepath{fill}%
\end{pgfscope}%
\begin{pgfscope}%
\pgfpathrectangle{\pgfqpoint{1.150000in}{0.150000in}}{\pgfqpoint{5.700000in}{5.700000in}}%
\pgfusepath{clip}%
\pgfsetbuttcap%
\pgfsetroundjoin%
\definecolor{currentfill}{rgb}{0.278826,0.175490,0.483397}%
\pgfsetfillcolor{currentfill}%
\pgfsetfillopacity{0.800000}%
\pgfsetlinewidth{0.000000pt}%
\definecolor{currentstroke}{rgb}{0.000000,0.000000,0.000000}%
\pgfsetstrokecolor{currentstroke}%
\pgfsetdash{}{0pt}%
\pgfpathmoveto{\pgfqpoint{3.169296in}{2.513991in}}%
\pgfpathlineto{\pgfqpoint{3.182743in}{2.501850in}}%
\pgfpathlineto{\pgfqpoint{3.196187in}{2.489973in}}%
\pgfpathlineto{\pgfqpoint{3.209629in}{2.478358in}}%
\pgfpathlineto{\pgfqpoint{3.223069in}{2.467003in}}%
\pgfpathlineto{\pgfqpoint{3.231137in}{2.477286in}}%
\pgfpathlineto{\pgfqpoint{3.239199in}{2.487654in}}%
\pgfpathlineto{\pgfqpoint{3.247254in}{2.498109in}}%
\pgfpathlineto{\pgfqpoint{3.255302in}{2.508649in}}%
\pgfpathlineto{\pgfqpoint{3.241876in}{2.520007in}}%
\pgfpathlineto{\pgfqpoint{3.228448in}{2.531624in}}%
\pgfpathlineto{\pgfqpoint{3.215018in}{2.543502in}}%
\pgfpathlineto{\pgfqpoint{3.201586in}{2.555644in}}%
\pgfpathlineto{\pgfqpoint{3.193523in}{2.545090in}}%
\pgfpathlineto{\pgfqpoint{3.185454in}{2.534630in}}%
\pgfpathlineto{\pgfqpoint{3.177378in}{2.524264in}}%
\pgfpathlineto{\pgfqpoint{3.169296in}{2.513991in}}%
\pgfpathclose%
\pgfusepath{fill}%
\end{pgfscope}%
\begin{pgfscope}%
\pgfpathrectangle{\pgfqpoint{1.150000in}{0.150000in}}{\pgfqpoint{5.700000in}{5.700000in}}%
\pgfusepath{clip}%
\pgfsetbuttcap%
\pgfsetroundjoin%
\definecolor{currentfill}{rgb}{0.180629,0.429975,0.557282}%
\pgfsetfillcolor{currentfill}%
\pgfsetfillopacity{0.800000}%
\pgfsetlinewidth{0.000000pt}%
\definecolor{currentstroke}{rgb}{0.000000,0.000000,0.000000}%
\pgfsetstrokecolor{currentstroke}%
\pgfsetdash{}{0pt}%
\pgfpathmoveto{\pgfqpoint{5.228241in}{3.131902in}}%
\pgfpathlineto{\pgfqpoint{5.242135in}{3.134475in}}%
\pgfpathlineto{\pgfqpoint{5.256043in}{3.137222in}}%
\pgfpathlineto{\pgfqpoint{5.269964in}{3.140143in}}%
\pgfpathlineto{\pgfqpoint{5.283898in}{3.143237in}}%
\pgfpathlineto{\pgfqpoint{5.291318in}{3.153377in}}%
\pgfpathlineto{\pgfqpoint{5.298739in}{3.163732in}}%
\pgfpathlineto{\pgfqpoint{5.306160in}{3.174310in}}%
\pgfpathlineto{\pgfqpoint{5.313582in}{3.185117in}}%
\pgfpathlineto{\pgfqpoint{5.299670in}{3.182693in}}%
\pgfpathlineto{\pgfqpoint{5.285771in}{3.180442in}}%
\pgfpathlineto{\pgfqpoint{5.271886in}{3.178365in}}%
\pgfpathlineto{\pgfqpoint{5.258013in}{3.176460in}}%
\pgfpathlineto{\pgfqpoint{5.250569in}{3.164972in}}%
\pgfpathlineto{\pgfqpoint{5.243125in}{3.153721in}}%
\pgfpathlineto{\pgfqpoint{5.235683in}{3.142700in}}%
\pgfpathlineto{\pgfqpoint{5.228241in}{3.131902in}}%
\pgfpathclose%
\pgfusepath{fill}%
\end{pgfscope}%
\begin{pgfscope}%
\pgfpathrectangle{\pgfqpoint{1.150000in}{0.150000in}}{\pgfqpoint{5.700000in}{5.700000in}}%
\pgfusepath{clip}%
\pgfsetbuttcap%
\pgfsetroundjoin%
\definecolor{currentfill}{rgb}{0.214298,0.355619,0.551184}%
\pgfsetfillcolor{currentfill}%
\pgfsetfillopacity{0.800000}%
\pgfsetlinewidth{0.000000pt}%
\definecolor{currentstroke}{rgb}{0.000000,0.000000,0.000000}%
\pgfsetstrokecolor{currentstroke}%
\pgfsetdash{}{0pt}%
\pgfpathmoveto{\pgfqpoint{2.790672in}{2.969963in}}%
\pgfpathlineto{\pgfqpoint{2.804299in}{2.949498in}}%
\pgfpathlineto{\pgfqpoint{2.817915in}{2.929368in}}%
\pgfpathlineto{\pgfqpoint{2.831522in}{2.909570in}}%
\pgfpathlineto{\pgfqpoint{2.845118in}{2.890099in}}%
\pgfpathlineto{\pgfqpoint{2.853303in}{2.900301in}}%
\pgfpathlineto{\pgfqpoint{2.861479in}{2.910643in}}%
\pgfpathlineto{\pgfqpoint{2.869646in}{2.921127in}}%
\pgfpathlineto{\pgfqpoint{2.877805in}{2.931752in}}%
\pgfpathlineto{\pgfqpoint{2.864227in}{2.951221in}}%
\pgfpathlineto{\pgfqpoint{2.850640in}{2.971018in}}%
\pgfpathlineto{\pgfqpoint{2.837043in}{2.991146in}}%
\pgfpathlineto{\pgfqpoint{2.823436in}{3.011609in}}%
\pgfpathlineto{\pgfqpoint{2.815258in}{3.000973in}}%
\pgfpathlineto{\pgfqpoint{2.807071in}{2.990487in}}%
\pgfpathlineto{\pgfqpoint{2.798876in}{2.980150in}}%
\pgfpathlineto{\pgfqpoint{2.790672in}{2.969963in}}%
\pgfpathclose%
\pgfusepath{fill}%
\end{pgfscope}%
\begin{pgfscope}%
\pgfpathrectangle{\pgfqpoint{1.150000in}{0.150000in}}{\pgfqpoint{5.700000in}{5.700000in}}%
\pgfusepath{clip}%
\pgfsetbuttcap%
\pgfsetroundjoin%
\definecolor{currentfill}{rgb}{0.280868,0.160771,0.472899}%
\pgfsetfillcolor{currentfill}%
\pgfsetfillopacity{0.800000}%
\pgfsetlinewidth{0.000000pt}%
\definecolor{currentstroke}{rgb}{0.000000,0.000000,0.000000}%
\pgfsetstrokecolor{currentstroke}%
\pgfsetdash{}{0pt}%
\pgfpathmoveto{\pgfqpoint{3.950720in}{2.460531in}}%
\pgfpathlineto{\pgfqpoint{3.964194in}{2.458072in}}%
\pgfpathlineto{\pgfqpoint{3.977674in}{2.455819in}}%
\pgfpathlineto{\pgfqpoint{3.991160in}{2.453770in}}%
\pgfpathlineto{\pgfqpoint{4.004653in}{2.451926in}}%
\pgfpathlineto{\pgfqpoint{4.012476in}{2.462583in}}%
\pgfpathlineto{\pgfqpoint{4.020294in}{2.473274in}}%
\pgfpathlineto{\pgfqpoint{4.028108in}{2.484001in}}%
\pgfpathlineto{\pgfqpoint{4.035916in}{2.494767in}}%
\pgfpathlineto{\pgfqpoint{4.022432in}{2.496804in}}%
\pgfpathlineto{\pgfqpoint{4.008954in}{2.499045in}}%
\pgfpathlineto{\pgfqpoint{3.995482in}{2.501490in}}%
\pgfpathlineto{\pgfqpoint{3.982017in}{2.504141in}}%
\pgfpathlineto{\pgfqpoint{3.974200in}{2.493171in}}%
\pgfpathlineto{\pgfqpoint{3.966378in}{2.482248in}}%
\pgfpathlineto{\pgfqpoint{3.958552in}{2.471368in}}%
\pgfpathlineto{\pgfqpoint{3.950720in}{2.460531in}}%
\pgfpathclose%
\pgfusepath{fill}%
\end{pgfscope}%
\begin{pgfscope}%
\pgfpathrectangle{\pgfqpoint{1.150000in}{0.150000in}}{\pgfqpoint{5.700000in}{5.700000in}}%
\pgfusepath{clip}%
\pgfsetbuttcap%
\pgfsetroundjoin%
\definecolor{currentfill}{rgb}{0.283072,0.130895,0.449241}%
\pgfsetfillcolor{currentfill}%
\pgfsetfillopacity{0.800000}%
\pgfsetlinewidth{0.000000pt}%
\definecolor{currentstroke}{rgb}{0.000000,0.000000,0.000000}%
\pgfsetstrokecolor{currentstroke}%
\pgfsetdash{}{0pt}%
\pgfpathmoveto{\pgfqpoint{3.641079in}{2.394769in}}%
\pgfpathlineto{\pgfqpoint{3.654501in}{2.389282in}}%
\pgfpathlineto{\pgfqpoint{3.667928in}{2.384016in}}%
\pgfpathlineto{\pgfqpoint{3.681358in}{2.378970in}}%
\pgfpathlineto{\pgfqpoint{3.694791in}{2.374143in}}%
\pgfpathlineto{\pgfqpoint{3.702710in}{2.384903in}}%
\pgfpathlineto{\pgfqpoint{3.710623in}{2.395705in}}%
\pgfpathlineto{\pgfqpoint{3.718530in}{2.406551in}}%
\pgfpathlineto{\pgfqpoint{3.726433in}{2.417441in}}%
\pgfpathlineto{\pgfqpoint{3.713009in}{2.422365in}}%
\pgfpathlineto{\pgfqpoint{3.699588in}{2.427509in}}%
\pgfpathlineto{\pgfqpoint{3.686171in}{2.432872in}}%
\pgfpathlineto{\pgfqpoint{3.672758in}{2.438457in}}%
\pgfpathlineto{\pgfqpoint{3.664846in}{2.427457in}}%
\pgfpathlineto{\pgfqpoint{3.656929in}{2.416510in}}%
\pgfpathlineto{\pgfqpoint{3.649006in}{2.405615in}}%
\pgfpathlineto{\pgfqpoint{3.641079in}{2.394769in}}%
\pgfpathclose%
\pgfusepath{fill}%
\end{pgfscope}%
\begin{pgfscope}%
\pgfpathrectangle{\pgfqpoint{1.150000in}{0.150000in}}{\pgfqpoint{5.700000in}{5.700000in}}%
\pgfusepath{clip}%
\pgfsetbuttcap%
\pgfsetroundjoin%
\definecolor{currentfill}{rgb}{0.172719,0.448791,0.557885}%
\pgfsetfillcolor{currentfill}%
\pgfsetfillopacity{0.800000}%
\pgfsetlinewidth{0.000000pt}%
\definecolor{currentstroke}{rgb}{0.000000,0.000000,0.000000}%
\pgfsetstrokecolor{currentstroke}%
\pgfsetdash{}{0pt}%
\pgfpathmoveto{\pgfqpoint{5.313582in}{3.185117in}}%
\pgfpathlineto{\pgfqpoint{5.327508in}{3.187714in}}%
\pgfpathlineto{\pgfqpoint{5.341447in}{3.190484in}}%
\pgfpathlineto{\pgfqpoint{5.355399in}{3.193426in}}%
\pgfpathlineto{\pgfqpoint{5.369366in}{3.196541in}}%
\pgfpathlineto{\pgfqpoint{5.376765in}{3.206896in}}%
\pgfpathlineto{\pgfqpoint{5.384166in}{3.217490in}}%
\pgfpathlineto{\pgfqpoint{5.391569in}{3.228330in}}%
\pgfpathlineto{\pgfqpoint{5.398974in}{3.239424in}}%
\pgfpathlineto{\pgfqpoint{5.385032in}{3.237012in}}%
\pgfpathlineto{\pgfqpoint{5.371103in}{3.234771in}}%
\pgfpathlineto{\pgfqpoint{5.357187in}{3.232703in}}%
\pgfpathlineto{\pgfqpoint{5.343285in}{3.230806in}}%
\pgfpathlineto{\pgfqpoint{5.335857in}{3.219000in}}%
\pgfpathlineto{\pgfqpoint{5.328430in}{3.207455in}}%
\pgfpathlineto{\pgfqpoint{5.321005in}{3.196163in}}%
\pgfpathlineto{\pgfqpoint{5.313582in}{3.185117in}}%
\pgfpathclose%
\pgfusepath{fill}%
\end{pgfscope}%
\begin{pgfscope}%
\pgfpathrectangle{\pgfqpoint{1.150000in}{0.150000in}}{\pgfqpoint{5.700000in}{5.700000in}}%
\pgfusepath{clip}%
\pgfsetbuttcap%
\pgfsetroundjoin%
\definecolor{currentfill}{rgb}{0.281887,0.150881,0.465405}%
\pgfsetfillcolor{currentfill}%
\pgfsetfillopacity{0.800000}%
\pgfsetlinewidth{0.000000pt}%
\definecolor{currentstroke}{rgb}{0.000000,0.000000,0.000000}%
\pgfsetstrokecolor{currentstroke}%
\pgfsetdash{}{0pt}%
\pgfpathmoveto{\pgfqpoint{3.865477in}{2.428814in}}%
\pgfpathlineto{\pgfqpoint{3.878935in}{2.425685in}}%
\pgfpathlineto{\pgfqpoint{3.892399in}{2.422766in}}%
\pgfpathlineto{\pgfqpoint{3.905869in}{2.420054in}}%
\pgfpathlineto{\pgfqpoint{3.919345in}{2.417550in}}%
\pgfpathlineto{\pgfqpoint{3.927196in}{2.428244in}}%
\pgfpathlineto{\pgfqpoint{3.935042in}{2.438971in}}%
\pgfpathlineto{\pgfqpoint{3.942884in}{2.449732in}}%
\pgfpathlineto{\pgfqpoint{3.950720in}{2.460531in}}%
\pgfpathlineto{\pgfqpoint{3.937253in}{2.463196in}}%
\pgfpathlineto{\pgfqpoint{3.923791in}{2.466068in}}%
\pgfpathlineto{\pgfqpoint{3.910336in}{2.469149in}}%
\pgfpathlineto{\pgfqpoint{3.896886in}{2.472438in}}%
\pgfpathlineto{\pgfqpoint{3.889041in}{2.461467in}}%
\pgfpathlineto{\pgfqpoint{3.881191in}{2.450541in}}%
\pgfpathlineto{\pgfqpoint{3.873337in}{2.439658in}}%
\pgfpathlineto{\pgfqpoint{3.865477in}{2.428814in}}%
\pgfpathclose%
\pgfusepath{fill}%
\end{pgfscope}%
\begin{pgfscope}%
\pgfpathrectangle{\pgfqpoint{1.150000in}{0.150000in}}{\pgfqpoint{5.700000in}{5.700000in}}%
\pgfusepath{clip}%
\pgfsetbuttcap%
\pgfsetroundjoin%
\definecolor{currentfill}{rgb}{0.280868,0.160771,0.472899}%
\pgfsetfillcolor{currentfill}%
\pgfsetfillopacity{0.800000}%
\pgfsetlinewidth{0.000000pt}%
\definecolor{currentstroke}{rgb}{0.000000,0.000000,0.000000}%
\pgfsetstrokecolor{currentstroke}%
\pgfsetdash{}{0pt}%
\pgfpathmoveto{\pgfqpoint{3.223069in}{2.467003in}}%
\pgfpathlineto{\pgfqpoint{3.236507in}{2.455905in}}%
\pgfpathlineto{\pgfqpoint{3.249943in}{2.445063in}}%
\pgfpathlineto{\pgfqpoint{3.263378in}{2.434476in}}%
\pgfpathlineto{\pgfqpoint{3.276811in}{2.424141in}}%
\pgfpathlineto{\pgfqpoint{3.284865in}{2.434434in}}%
\pgfpathlineto{\pgfqpoint{3.292913in}{2.444805in}}%
\pgfpathlineto{\pgfqpoint{3.300954in}{2.455253in}}%
\pgfpathlineto{\pgfqpoint{3.308989in}{2.465780in}}%
\pgfpathlineto{\pgfqpoint{3.295569in}{2.476117in}}%
\pgfpathlineto{\pgfqpoint{3.282148in}{2.486706in}}%
\pgfpathlineto{\pgfqpoint{3.268726in}{2.497550in}}%
\pgfpathlineto{\pgfqpoint{3.255302in}{2.508649in}}%
\pgfpathlineto{\pgfqpoint{3.247254in}{2.498109in}}%
\pgfpathlineto{\pgfqpoint{3.239199in}{2.487654in}}%
\pgfpathlineto{\pgfqpoint{3.231137in}{2.477286in}}%
\pgfpathlineto{\pgfqpoint{3.223069in}{2.467003in}}%
\pgfpathclose%
\pgfusepath{fill}%
\end{pgfscope}%
\begin{pgfscope}%
\pgfpathrectangle{\pgfqpoint{1.150000in}{0.150000in}}{\pgfqpoint{5.700000in}{5.700000in}}%
\pgfusepath{clip}%
\pgfsetbuttcap%
\pgfsetroundjoin%
\definecolor{currentfill}{rgb}{0.163625,0.471133,0.558148}%
\pgfsetfillcolor{currentfill}%
\pgfsetfillopacity{0.800000}%
\pgfsetlinewidth{0.000000pt}%
\definecolor{currentstroke}{rgb}{0.000000,0.000000,0.000000}%
\pgfsetstrokecolor{currentstroke}%
\pgfsetdash{}{0pt}%
\pgfpathmoveto{\pgfqpoint{5.398974in}{3.239424in}}%
\pgfpathlineto{\pgfqpoint{5.412930in}{3.242008in}}%
\pgfpathlineto{\pgfqpoint{5.426900in}{3.244763in}}%
\pgfpathlineto{\pgfqpoint{5.440883in}{3.247690in}}%
\pgfpathlineto{\pgfqpoint{5.454881in}{3.250789in}}%
\pgfpathlineto{\pgfqpoint{5.462263in}{3.261422in}}%
\pgfpathlineto{\pgfqpoint{5.469647in}{3.272319in}}%
\pgfpathlineto{\pgfqpoint{5.477034in}{3.283488in}}%
\pgfpathlineto{\pgfqpoint{5.484425in}{3.294936in}}%
\pgfpathlineto{\pgfqpoint{5.470453in}{3.292572in}}%
\pgfpathlineto{\pgfqpoint{5.456495in}{3.290378in}}%
\pgfpathlineto{\pgfqpoint{5.442550in}{3.288356in}}%
\pgfpathlineto{\pgfqpoint{5.428619in}{3.286504in}}%
\pgfpathlineto{\pgfqpoint{5.421203in}{3.274312in}}%
\pgfpathlineto{\pgfqpoint{5.413791in}{3.262407in}}%
\pgfpathlineto{\pgfqpoint{5.406381in}{3.250780in}}%
\pgfpathlineto{\pgfqpoint{5.398974in}{3.239424in}}%
\pgfpathclose%
\pgfusepath{fill}%
\end{pgfscope}%
\begin{pgfscope}%
\pgfpathrectangle{\pgfqpoint{1.150000in}{0.150000in}}{\pgfqpoint{5.700000in}{5.700000in}}%
\pgfusepath{clip}%
\pgfsetbuttcap%
\pgfsetroundjoin%
\definecolor{currentfill}{rgb}{0.199430,0.387607,0.554642}%
\pgfsetfillcolor{currentfill}%
\pgfsetfillopacity{0.800000}%
\pgfsetlinewidth{0.000000pt}%
\definecolor{currentstroke}{rgb}{0.000000,0.000000,0.000000}%
\pgfsetstrokecolor{currentstroke}%
\pgfsetdash{}{0pt}%
\pgfpathmoveto{\pgfqpoint{2.736055in}{3.055234in}}%
\pgfpathlineto{\pgfqpoint{2.749726in}{3.033398in}}%
\pgfpathlineto{\pgfqpoint{2.763386in}{3.011909in}}%
\pgfpathlineto{\pgfqpoint{2.777034in}{2.990766in}}%
\pgfpathlineto{\pgfqpoint{2.790672in}{2.969963in}}%
\pgfpathlineto{\pgfqpoint{2.798876in}{2.980150in}}%
\pgfpathlineto{\pgfqpoint{2.807071in}{2.990487in}}%
\pgfpathlineto{\pgfqpoint{2.815258in}{3.000973in}}%
\pgfpathlineto{\pgfqpoint{2.823436in}{3.011609in}}%
\pgfpathlineto{\pgfqpoint{2.809818in}{3.032409in}}%
\pgfpathlineto{\pgfqpoint{2.796189in}{3.053550in}}%
\pgfpathlineto{\pgfqpoint{2.782550in}{3.075036in}}%
\pgfpathlineto{\pgfqpoint{2.768899in}{3.096870in}}%
\pgfpathlineto{\pgfqpoint{2.760701in}{3.086224in}}%
\pgfpathlineto{\pgfqpoint{2.752495in}{3.075737in}}%
\pgfpathlineto{\pgfqpoint{2.744280in}{3.065407in}}%
\pgfpathlineto{\pgfqpoint{2.736055in}{3.055234in}}%
\pgfpathclose%
\pgfusepath{fill}%
\end{pgfscope}%
\begin{pgfscope}%
\pgfpathrectangle{\pgfqpoint{1.150000in}{0.150000in}}{\pgfqpoint{5.700000in}{5.700000in}}%
\pgfusepath{clip}%
\pgfsetbuttcap%
\pgfsetroundjoin%
\definecolor{currentfill}{rgb}{0.283072,0.130895,0.449241}%
\pgfsetfillcolor{currentfill}%
\pgfsetfillopacity{0.800000}%
\pgfsetlinewidth{0.000000pt}%
\definecolor{currentstroke}{rgb}{0.000000,0.000000,0.000000}%
\pgfsetstrokecolor{currentstroke}%
\pgfsetdash{}{0pt}%
\pgfpathmoveto{\pgfqpoint{3.416332in}{2.391973in}}%
\pgfpathlineto{\pgfqpoint{3.429751in}{2.383835in}}%
\pgfpathlineto{\pgfqpoint{3.443171in}{2.375933in}}%
\pgfpathlineto{\pgfqpoint{3.456592in}{2.368267in}}%
\pgfpathlineto{\pgfqpoint{3.470014in}{2.360834in}}%
\pgfpathlineto{\pgfqpoint{3.478006in}{2.371373in}}%
\pgfpathlineto{\pgfqpoint{3.485993in}{2.381968in}}%
\pgfpathlineto{\pgfqpoint{3.493975in}{2.392620in}}%
\pgfpathlineto{\pgfqpoint{3.501950in}{2.403330in}}%
\pgfpathlineto{\pgfqpoint{3.488539in}{2.410797in}}%
\pgfpathlineto{\pgfqpoint{3.475130in}{2.418498in}}%
\pgfpathlineto{\pgfqpoint{3.461721in}{2.426434in}}%
\pgfpathlineto{\pgfqpoint{3.448314in}{2.434607in}}%
\pgfpathlineto{\pgfqpoint{3.440327in}{2.423850in}}%
\pgfpathlineto{\pgfqpoint{3.432334in}{2.413160in}}%
\pgfpathlineto{\pgfqpoint{3.424336in}{2.402534in}}%
\pgfpathlineto{\pgfqpoint{3.416332in}{2.391973in}}%
\pgfpathclose%
\pgfusepath{fill}%
\end{pgfscope}%
\begin{pgfscope}%
\pgfpathrectangle{\pgfqpoint{1.150000in}{0.150000in}}{\pgfqpoint{5.700000in}{5.700000in}}%
\pgfusepath{clip}%
\pgfsetbuttcap%
\pgfsetroundjoin%
\definecolor{currentfill}{rgb}{0.283187,0.125848,0.444960}%
\pgfsetfillcolor{currentfill}%
\pgfsetfillopacity{0.800000}%
\pgfsetlinewidth{0.000000pt}%
\definecolor{currentstroke}{rgb}{0.000000,0.000000,0.000000}%
\pgfsetstrokecolor{currentstroke}%
\pgfsetdash{}{0pt}%
\pgfpathmoveto{\pgfqpoint{3.555613in}{2.375774in}}%
\pgfpathlineto{\pgfqpoint{3.569034in}{2.369456in}}%
\pgfpathlineto{\pgfqpoint{3.582458in}{2.363364in}}%
\pgfpathlineto{\pgfqpoint{3.595885in}{2.357498in}}%
\pgfpathlineto{\pgfqpoint{3.609314in}{2.351854in}}%
\pgfpathlineto{\pgfqpoint{3.617263in}{2.362515in}}%
\pgfpathlineto{\pgfqpoint{3.625207in}{2.373220in}}%
\pgfpathlineto{\pgfqpoint{3.633145in}{2.383971in}}%
\pgfpathlineto{\pgfqpoint{3.641079in}{2.394769in}}%
\pgfpathlineto{\pgfqpoint{3.627659in}{2.400478in}}%
\pgfpathlineto{\pgfqpoint{3.614242in}{2.406411in}}%
\pgfpathlineto{\pgfqpoint{3.600829in}{2.412569in}}%
\pgfpathlineto{\pgfqpoint{3.587418in}{2.418953in}}%
\pgfpathlineto{\pgfqpoint{3.579475in}{2.408078in}}%
\pgfpathlineto{\pgfqpoint{3.571526in}{2.397257in}}%
\pgfpathlineto{\pgfqpoint{3.563572in}{2.386489in}}%
\pgfpathlineto{\pgfqpoint{3.555613in}{2.375774in}}%
\pgfpathclose%
\pgfusepath{fill}%
\end{pgfscope}%
\begin{pgfscope}%
\pgfpathrectangle{\pgfqpoint{1.150000in}{0.150000in}}{\pgfqpoint{5.700000in}{5.700000in}}%
\pgfusepath{clip}%
\pgfsetbuttcap%
\pgfsetroundjoin%
\definecolor{currentfill}{rgb}{0.282884,0.135920,0.453427}%
\pgfsetfillcolor{currentfill}%
\pgfsetfillopacity{0.800000}%
\pgfsetlinewidth{0.000000pt}%
\definecolor{currentstroke}{rgb}{0.000000,0.000000,0.000000}%
\pgfsetstrokecolor{currentstroke}%
\pgfsetdash{}{0pt}%
\pgfpathmoveto{\pgfqpoint{3.780173in}{2.399912in}}%
\pgfpathlineto{\pgfqpoint{3.793619in}{2.396067in}}%
\pgfpathlineto{\pgfqpoint{3.807070in}{2.392435in}}%
\pgfpathlineto{\pgfqpoint{3.820526in}{2.389014in}}%
\pgfpathlineto{\pgfqpoint{3.833988in}{2.385805in}}%
\pgfpathlineto{\pgfqpoint{3.841868in}{2.396506in}}%
\pgfpathlineto{\pgfqpoint{3.849743in}{2.407241in}}%
\pgfpathlineto{\pgfqpoint{3.857612in}{2.418009in}}%
\pgfpathlineto{\pgfqpoint{3.865477in}{2.428814in}}%
\pgfpathlineto{\pgfqpoint{3.852024in}{2.432153in}}%
\pgfpathlineto{\pgfqpoint{3.838576in}{2.435703in}}%
\pgfpathlineto{\pgfqpoint{3.825134in}{2.439464in}}%
\pgfpathlineto{\pgfqpoint{3.811696in}{2.443439in}}%
\pgfpathlineto{\pgfqpoint{3.803823in}{2.432493in}}%
\pgfpathlineto{\pgfqpoint{3.795945in}{2.421591in}}%
\pgfpathlineto{\pgfqpoint{3.788061in}{2.410731in}}%
\pgfpathlineto{\pgfqpoint{3.780173in}{2.399912in}}%
\pgfpathclose%
\pgfusepath{fill}%
\end{pgfscope}%
\begin{pgfscope}%
\pgfpathrectangle{\pgfqpoint{1.150000in}{0.150000in}}{\pgfqpoint{5.700000in}{5.700000in}}%
\pgfusepath{clip}%
\pgfsetbuttcap%
\pgfsetroundjoin%
\definecolor{currentfill}{rgb}{0.156270,0.489624,0.557936}%
\pgfsetfillcolor{currentfill}%
\pgfsetfillopacity{0.800000}%
\pgfsetlinewidth{0.000000pt}%
\definecolor{currentstroke}{rgb}{0.000000,0.000000,0.000000}%
\pgfsetstrokecolor{currentstroke}%
\pgfsetdash{}{0pt}%
\pgfpathmoveto{\pgfqpoint{5.484425in}{3.294936in}}%
\pgfpathlineto{\pgfqpoint{5.498410in}{3.297470in}}%
\pgfpathlineto{\pgfqpoint{5.512410in}{3.300175in}}%
\pgfpathlineto{\pgfqpoint{5.526424in}{3.303050in}}%
\pgfpathlineto{\pgfqpoint{5.540452in}{3.306095in}}%
\pgfpathlineto{\pgfqpoint{5.547819in}{3.317076in}}%
\pgfpathlineto{\pgfqpoint{5.555190in}{3.328347in}}%
\pgfpathlineto{\pgfqpoint{5.562564in}{3.339916in}}%
\pgfpathlineto{\pgfqpoint{5.548557in}{3.337443in}}%
\pgfpathlineto{\pgfqpoint{5.534564in}{3.335139in}}%
\pgfpathlineto{\pgfqpoint{5.520584in}{3.333005in}}%
\pgfpathlineto{\pgfqpoint{5.506618in}{3.331041in}}%
\pgfpathlineto{\pgfqpoint{5.499216in}{3.318704in}}%
\pgfpathlineto{\pgfqpoint{5.491819in}{3.306671in}}%
\pgfpathlineto{\pgfqpoint{5.484425in}{3.294936in}}%
\pgfpathclose%
\pgfusepath{fill}%
\end{pgfscope}%
\begin{pgfscope}%
\pgfpathrectangle{\pgfqpoint{1.150000in}{0.150000in}}{\pgfqpoint{5.700000in}{5.700000in}}%
\pgfusepath{clip}%
\pgfsetbuttcap%
\pgfsetroundjoin%
\definecolor{currentfill}{rgb}{0.250425,0.274290,0.533103}%
\pgfsetfillcolor{currentfill}%
\pgfsetfillopacity{0.800000}%
\pgfsetlinewidth{0.000000pt}%
\definecolor{currentstroke}{rgb}{0.000000,0.000000,0.000000}%
\pgfsetstrokecolor{currentstroke}%
\pgfsetdash{}{0pt}%
\pgfpathmoveto{\pgfqpoint{4.516073in}{2.698893in}}%
\pgfpathlineto{\pgfqpoint{4.529727in}{2.700291in}}%
\pgfpathlineto{\pgfqpoint{4.543391in}{2.701877in}}%
\pgfpathlineto{\pgfqpoint{4.557065in}{2.703650in}}%
\pgfpathlineto{\pgfqpoint{4.570751in}{2.705609in}}%
\pgfpathlineto{\pgfqpoint{4.578396in}{2.715340in}}%
\pgfpathlineto{\pgfqpoint{4.586037in}{2.725132in}}%
\pgfpathlineto{\pgfqpoint{4.593674in}{2.734991in}}%
\pgfpathlineto{\pgfqpoint{4.601307in}{2.744920in}}%
\pgfpathlineto{\pgfqpoint{4.587633in}{2.743345in}}%
\pgfpathlineto{\pgfqpoint{4.573970in}{2.741956in}}%
\pgfpathlineto{\pgfqpoint{4.560317in}{2.740754in}}%
\pgfpathlineto{\pgfqpoint{4.546674in}{2.739738in}}%
\pgfpathlineto{\pgfqpoint{4.539030in}{2.729414in}}%
\pgfpathlineto{\pgfqpoint{4.531382in}{2.719168in}}%
\pgfpathlineto{\pgfqpoint{4.523729in}{2.708995in}}%
\pgfpathlineto{\pgfqpoint{4.516073in}{2.698893in}}%
\pgfpathclose%
\pgfusepath{fill}%
\end{pgfscope}%
\begin{pgfscope}%
\pgfpathrectangle{\pgfqpoint{1.150000in}{0.150000in}}{\pgfqpoint{5.700000in}{5.700000in}}%
\pgfusepath{clip}%
\pgfsetbuttcap%
\pgfsetroundjoin%
\definecolor{currentfill}{rgb}{0.258965,0.251537,0.524736}%
\pgfsetfillcolor{currentfill}%
\pgfsetfillopacity{0.800000}%
\pgfsetlinewidth{0.000000pt}%
\definecolor{currentstroke}{rgb}{0.000000,0.000000,0.000000}%
\pgfsetstrokecolor{currentstroke}%
\pgfsetdash{}{0pt}%
\pgfpathmoveto{\pgfqpoint{4.430847in}{2.653972in}}%
\pgfpathlineto{\pgfqpoint{4.444471in}{2.654968in}}%
\pgfpathlineto{\pgfqpoint{4.458105in}{2.656153in}}%
\pgfpathlineto{\pgfqpoint{4.471749in}{2.657527in}}%
\pgfpathlineto{\pgfqpoint{4.485403in}{2.659090in}}%
\pgfpathlineto{\pgfqpoint{4.493077in}{2.668957in}}%
\pgfpathlineto{\pgfqpoint{4.500747in}{2.678878in}}%
\pgfpathlineto{\pgfqpoint{4.508412in}{2.688855in}}%
\pgfpathlineto{\pgfqpoint{4.516073in}{2.698893in}}%
\pgfpathlineto{\pgfqpoint{4.502429in}{2.697682in}}%
\pgfpathlineto{\pgfqpoint{4.488796in}{2.696660in}}%
\pgfpathlineto{\pgfqpoint{4.475173in}{2.695827in}}%
\pgfpathlineto{\pgfqpoint{4.461559in}{2.695182in}}%
\pgfpathlineto{\pgfqpoint{4.453888in}{2.684781in}}%
\pgfpathlineto{\pgfqpoint{4.446212in}{2.674449in}}%
\pgfpathlineto{\pgfqpoint{4.438532in}{2.664180in}}%
\pgfpathlineto{\pgfqpoint{4.430847in}{2.653972in}}%
\pgfpathclose%
\pgfusepath{fill}%
\end{pgfscope}%
\begin{pgfscope}%
\pgfpathrectangle{\pgfqpoint{1.150000in}{0.150000in}}{\pgfqpoint{5.700000in}{5.700000in}}%
\pgfusepath{clip}%
\pgfsetbuttcap%
\pgfsetroundjoin%
\definecolor{currentfill}{rgb}{0.243113,0.292092,0.538516}%
\pgfsetfillcolor{currentfill}%
\pgfsetfillopacity{0.800000}%
\pgfsetlinewidth{0.000000pt}%
\definecolor{currentstroke}{rgb}{0.000000,0.000000,0.000000}%
\pgfsetstrokecolor{currentstroke}%
\pgfsetdash{}{0pt}%
\pgfpathmoveto{\pgfqpoint{4.601307in}{2.744920in}}%
\pgfpathlineto{\pgfqpoint{4.614991in}{2.746681in}}%
\pgfpathlineto{\pgfqpoint{4.628687in}{2.748628in}}%
\pgfpathlineto{\pgfqpoint{4.642394in}{2.750760in}}%
\pgfpathlineto{\pgfqpoint{4.656111in}{2.753076in}}%
\pgfpathlineto{\pgfqpoint{4.663728in}{2.762677in}}%
\pgfpathlineto{\pgfqpoint{4.671340in}{2.772350in}}%
\pgfpathlineto{\pgfqpoint{4.678949in}{2.782100in}}%
\pgfpathlineto{\pgfqpoint{4.686553in}{2.791933in}}%
\pgfpathlineto{\pgfqpoint{4.672848in}{2.790032in}}%
\pgfpathlineto{\pgfqpoint{4.659154in}{2.788317in}}%
\pgfpathlineto{\pgfqpoint{4.645470in}{2.786785in}}%
\pgfpathlineto{\pgfqpoint{4.631797in}{2.785439in}}%
\pgfpathlineto{\pgfqpoint{4.624181in}{2.775180in}}%
\pgfpathlineto{\pgfqpoint{4.616560in}{2.765010in}}%
\pgfpathlineto{\pgfqpoint{4.608935in}{2.754925in}}%
\pgfpathlineto{\pgfqpoint{4.601307in}{2.744920in}}%
\pgfpathclose%
\pgfusepath{fill}%
\end{pgfscope}%
\begin{pgfscope}%
\pgfpathrectangle{\pgfqpoint{1.150000in}{0.150000in}}{\pgfqpoint{5.700000in}{5.700000in}}%
\pgfusepath{clip}%
\pgfsetbuttcap%
\pgfsetroundjoin%
\definecolor{currentfill}{rgb}{0.265145,0.232956,0.516599}%
\pgfsetfillcolor{currentfill}%
\pgfsetfillopacity{0.800000}%
\pgfsetlinewidth{0.000000pt}%
\definecolor{currentstroke}{rgb}{0.000000,0.000000,0.000000}%
\pgfsetstrokecolor{currentstroke}%
\pgfsetdash{}{0pt}%
\pgfpathmoveto{\pgfqpoint{4.345626in}{2.610306in}}%
\pgfpathlineto{\pgfqpoint{4.359221in}{2.610858in}}%
\pgfpathlineto{\pgfqpoint{4.372826in}{2.611601in}}%
\pgfpathlineto{\pgfqpoint{4.386440in}{2.612536in}}%
\pgfpathlineto{\pgfqpoint{4.400064in}{2.613662in}}%
\pgfpathlineto{\pgfqpoint{4.407767in}{2.623669in}}%
\pgfpathlineto{\pgfqpoint{4.415465in}{2.633720in}}%
\pgfpathlineto{\pgfqpoint{4.423159in}{2.643820in}}%
\pgfpathlineto{\pgfqpoint{4.430847in}{2.653972in}}%
\pgfpathlineto{\pgfqpoint{4.417233in}{2.653167in}}%
\pgfpathlineto{\pgfqpoint{4.403629in}{2.652552in}}%
\pgfpathlineto{\pgfqpoint{4.390034in}{2.652128in}}%
\pgfpathlineto{\pgfqpoint{4.376448in}{2.651896in}}%
\pgfpathlineto{\pgfqpoint{4.368749in}{2.641412in}}%
\pgfpathlineto{\pgfqpoint{4.361046in}{2.630989in}}%
\pgfpathlineto{\pgfqpoint{4.353338in}{2.620621in}}%
\pgfpathlineto{\pgfqpoint{4.345626in}{2.610306in}}%
\pgfpathclose%
\pgfusepath{fill}%
\end{pgfscope}%
\begin{pgfscope}%
\pgfpathrectangle{\pgfqpoint{1.150000in}{0.150000in}}{\pgfqpoint{5.700000in}{5.700000in}}%
\pgfusepath{clip}%
\pgfsetbuttcap%
\pgfsetroundjoin%
\definecolor{currentfill}{rgb}{0.282290,0.145912,0.461510}%
\pgfsetfillcolor{currentfill}%
\pgfsetfillopacity{0.800000}%
\pgfsetlinewidth{0.000000pt}%
\definecolor{currentstroke}{rgb}{0.000000,0.000000,0.000000}%
\pgfsetstrokecolor{currentstroke}%
\pgfsetdash{}{0pt}%
\pgfpathmoveto{\pgfqpoint{3.276811in}{2.424141in}}%
\pgfpathlineto{\pgfqpoint{3.290243in}{2.414057in}}%
\pgfpathlineto{\pgfqpoint{3.303674in}{2.404223in}}%
\pgfpathlineto{\pgfqpoint{3.317105in}{2.394636in}}%
\pgfpathlineto{\pgfqpoint{3.330535in}{2.385294in}}%
\pgfpathlineto{\pgfqpoint{3.338576in}{2.395596in}}%
\pgfpathlineto{\pgfqpoint{3.346610in}{2.405968in}}%
\pgfpathlineto{\pgfqpoint{3.354639in}{2.416410in}}%
\pgfpathlineto{\pgfqpoint{3.362661in}{2.426922in}}%
\pgfpathlineto{\pgfqpoint{3.349244in}{2.436266in}}%
\pgfpathlineto{\pgfqpoint{3.335826in}{2.445856in}}%
\pgfpathlineto{\pgfqpoint{3.322408in}{2.455693in}}%
\pgfpathlineto{\pgfqpoint{3.308989in}{2.465780in}}%
\pgfpathlineto{\pgfqpoint{3.300954in}{2.455253in}}%
\pgfpathlineto{\pgfqpoint{3.292913in}{2.444805in}}%
\pgfpathlineto{\pgfqpoint{3.284865in}{2.434434in}}%
\pgfpathlineto{\pgfqpoint{3.276811in}{2.424141in}}%
\pgfpathclose%
\pgfusepath{fill}%
\end{pgfscope}%
\begin{pgfscope}%
\pgfpathrectangle{\pgfqpoint{1.150000in}{0.150000in}}{\pgfqpoint{5.700000in}{5.700000in}}%
\pgfusepath{clip}%
\pgfsetbuttcap%
\pgfsetroundjoin%
\definecolor{currentfill}{rgb}{0.235526,0.309527,0.542944}%
\pgfsetfillcolor{currentfill}%
\pgfsetfillopacity{0.800000}%
\pgfsetlinewidth{0.000000pt}%
\definecolor{currentstroke}{rgb}{0.000000,0.000000,0.000000}%
\pgfsetstrokecolor{currentstroke}%
\pgfsetdash{}{0pt}%
\pgfpathmoveto{\pgfqpoint{4.686553in}{2.791933in}}%
\pgfpathlineto{\pgfqpoint{4.700270in}{2.794017in}}%
\pgfpathlineto{\pgfqpoint{4.713998in}{2.796285in}}%
\pgfpathlineto{\pgfqpoint{4.727737in}{2.798736in}}%
\pgfpathlineto{\pgfqpoint{4.741488in}{2.801370in}}%
\pgfpathlineto{\pgfqpoint{4.749075in}{2.810854in}}%
\pgfpathlineto{\pgfqpoint{4.756659in}{2.820421in}}%
\pgfpathlineto{\pgfqpoint{4.764239in}{2.830079in}}%
\pgfpathlineto{\pgfqpoint{4.771815in}{2.839831in}}%
\pgfpathlineto{\pgfqpoint{4.758078in}{2.837645in}}%
\pgfpathlineto{\pgfqpoint{4.744352in}{2.835642in}}%
\pgfpathlineto{\pgfqpoint{4.730637in}{2.833821in}}%
\pgfpathlineto{\pgfqpoint{4.716934in}{2.832184in}}%
\pgfpathlineto{\pgfqpoint{4.709344in}{2.821973in}}%
\pgfpathlineto{\pgfqpoint{4.701751in}{2.811864in}}%
\pgfpathlineto{\pgfqpoint{4.694154in}{2.801852in}}%
\pgfpathlineto{\pgfqpoint{4.686553in}{2.791933in}}%
\pgfpathclose%
\pgfusepath{fill}%
\end{pgfscope}%
\begin{pgfscope}%
\pgfpathrectangle{\pgfqpoint{1.150000in}{0.150000in}}{\pgfqpoint{5.700000in}{5.700000in}}%
\pgfusepath{clip}%
\pgfsetbuttcap%
\pgfsetroundjoin%
\definecolor{currentfill}{rgb}{0.265145,0.232956,0.516599}%
\pgfsetfillcolor{currentfill}%
\pgfsetfillopacity{0.800000}%
\pgfsetlinewidth{0.000000pt}%
\definecolor{currentstroke}{rgb}{0.000000,0.000000,0.000000}%
\pgfsetstrokecolor{currentstroke}%
\pgfsetdash{}{0pt}%
\pgfpathmoveto{\pgfqpoint{2.975037in}{2.641184in}}%
\pgfpathlineto{\pgfqpoint{2.988553in}{2.625686in}}%
\pgfpathlineto{\pgfqpoint{3.002062in}{2.610478in}}%
\pgfpathlineto{\pgfqpoint{3.015567in}{2.595557in}}%
\pgfpathlineto{\pgfqpoint{3.029066in}{2.580921in}}%
\pgfpathlineto{\pgfqpoint{3.037209in}{2.590739in}}%
\pgfpathlineto{\pgfqpoint{3.045345in}{2.600666in}}%
\pgfpathlineto{\pgfqpoint{3.053473in}{2.610701in}}%
\pgfpathlineto{\pgfqpoint{3.061594in}{2.620843in}}%
\pgfpathlineto{\pgfqpoint{3.048113in}{2.635448in}}%
\pgfpathlineto{\pgfqpoint{3.034626in}{2.650338in}}%
\pgfpathlineto{\pgfqpoint{3.021133in}{2.665514in}}%
\pgfpathlineto{\pgfqpoint{3.007635in}{2.680981in}}%
\pgfpathlineto{\pgfqpoint{2.999497in}{2.670857in}}%
\pgfpathlineto{\pgfqpoint{2.991352in}{2.660850in}}%
\pgfpathlineto{\pgfqpoint{2.983198in}{2.650959in}}%
\pgfpathlineto{\pgfqpoint{2.975037in}{2.641184in}}%
\pgfpathclose%
\pgfusepath{fill}%
\end{pgfscope}%
\begin{pgfscope}%
\pgfpathrectangle{\pgfqpoint{1.150000in}{0.150000in}}{\pgfqpoint{5.700000in}{5.700000in}}%
\pgfusepath{clip}%
\pgfsetbuttcap%
\pgfsetroundjoin%
\definecolor{currentfill}{rgb}{0.255645,0.260703,0.528312}%
\pgfsetfillcolor{currentfill}%
\pgfsetfillopacity{0.800000}%
\pgfsetlinewidth{0.000000pt}%
\definecolor{currentstroke}{rgb}{0.000000,0.000000,0.000000}%
\pgfsetstrokecolor{currentstroke}%
\pgfsetdash{}{0pt}%
\pgfpathmoveto{\pgfqpoint{2.920911in}{2.706123in}}%
\pgfpathlineto{\pgfqpoint{2.934452in}{2.689441in}}%
\pgfpathlineto{\pgfqpoint{2.947987in}{2.673059in}}%
\pgfpathlineto{\pgfqpoint{2.961515in}{2.656974in}}%
\pgfpathlineto{\pgfqpoint{2.975037in}{2.641184in}}%
\pgfpathlineto{\pgfqpoint{2.983198in}{2.650959in}}%
\pgfpathlineto{\pgfqpoint{2.991352in}{2.660850in}}%
\pgfpathlineto{\pgfqpoint{2.999497in}{2.670857in}}%
\pgfpathlineto{\pgfqpoint{3.007635in}{2.680981in}}%
\pgfpathlineto{\pgfqpoint{2.994131in}{2.696739in}}%
\pgfpathlineto{\pgfqpoint{2.980621in}{2.712792in}}%
\pgfpathlineto{\pgfqpoint{2.967104in}{2.729142in}}%
\pgfpathlineto{\pgfqpoint{2.953581in}{2.745791in}}%
\pgfpathlineto{\pgfqpoint{2.945426in}{2.735688in}}%
\pgfpathlineto{\pgfqpoint{2.937262in}{2.725709in}}%
\pgfpathlineto{\pgfqpoint{2.929090in}{2.715854in}}%
\pgfpathlineto{\pgfqpoint{2.920911in}{2.706123in}}%
\pgfpathclose%
\pgfusepath{fill}%
\end{pgfscope}%
\begin{pgfscope}%
\pgfpathrectangle{\pgfqpoint{1.150000in}{0.150000in}}{\pgfqpoint{5.700000in}{5.700000in}}%
\pgfusepath{clip}%
\pgfsetbuttcap%
\pgfsetroundjoin%
\definecolor{currentfill}{rgb}{0.269308,0.218818,0.509577}%
\pgfsetfillcolor{currentfill}%
\pgfsetfillopacity{0.800000}%
\pgfsetlinewidth{0.000000pt}%
\definecolor{currentstroke}{rgb}{0.000000,0.000000,0.000000}%
\pgfsetstrokecolor{currentstroke}%
\pgfsetdash{}{0pt}%
\pgfpathmoveto{\pgfqpoint{4.260403in}{2.568062in}}%
\pgfpathlineto{\pgfqpoint{4.273971in}{2.568129in}}%
\pgfpathlineto{\pgfqpoint{4.287549in}{2.568391in}}%
\pgfpathlineto{\pgfqpoint{4.301135in}{2.568846in}}%
\pgfpathlineto{\pgfqpoint{4.314730in}{2.569494in}}%
\pgfpathlineto{\pgfqpoint{4.322461in}{2.579636in}}%
\pgfpathlineto{\pgfqpoint{4.330188in}{2.589817in}}%
\pgfpathlineto{\pgfqpoint{4.337909in}{2.600039in}}%
\pgfpathlineto{\pgfqpoint{4.345626in}{2.610306in}}%
\pgfpathlineto{\pgfqpoint{4.332040in}{2.609946in}}%
\pgfpathlineto{\pgfqpoint{4.318463in}{2.609779in}}%
\pgfpathlineto{\pgfqpoint{4.304895in}{2.609806in}}%
\pgfpathlineto{\pgfqpoint{4.291335in}{2.610027in}}%
\pgfpathlineto{\pgfqpoint{4.283609in}{2.599460in}}%
\pgfpathlineto{\pgfqpoint{4.275878in}{2.588946in}}%
\pgfpathlineto{\pgfqpoint{4.268143in}{2.578481in}}%
\pgfpathlineto{\pgfqpoint{4.260403in}{2.568062in}}%
\pgfpathclose%
\pgfusepath{fill}%
\end{pgfscope}%
\begin{pgfscope}%
\pgfpathrectangle{\pgfqpoint{1.150000in}{0.150000in}}{\pgfqpoint{5.700000in}{5.700000in}}%
\pgfusepath{clip}%
\pgfsetbuttcap%
\pgfsetroundjoin%
\definecolor{currentfill}{rgb}{0.225863,0.330805,0.547314}%
\pgfsetfillcolor{currentfill}%
\pgfsetfillopacity{0.800000}%
\pgfsetlinewidth{0.000000pt}%
\definecolor{currentstroke}{rgb}{0.000000,0.000000,0.000000}%
\pgfsetstrokecolor{currentstroke}%
\pgfsetdash{}{0pt}%
\pgfpathmoveto{\pgfqpoint{4.771815in}{2.839831in}}%
\pgfpathlineto{\pgfqpoint{4.785564in}{2.842199in}}%
\pgfpathlineto{\pgfqpoint{4.799325in}{2.844749in}}%
\pgfpathlineto{\pgfqpoint{4.813098in}{2.847481in}}%
\pgfpathlineto{\pgfqpoint{4.826883in}{2.850394in}}%
\pgfpathlineto{\pgfqpoint{4.834441in}{2.859778in}}%
\pgfpathlineto{\pgfqpoint{4.841996in}{2.869260in}}%
\pgfpathlineto{\pgfqpoint{4.849548in}{2.878845in}}%
\pgfpathlineto{\pgfqpoint{4.857097in}{2.888539in}}%
\pgfpathlineto{\pgfqpoint{4.843326in}{2.886107in}}%
\pgfpathlineto{\pgfqpoint{4.829568in}{2.883855in}}%
\pgfpathlineto{\pgfqpoint{4.815822in}{2.881784in}}%
\pgfpathlineto{\pgfqpoint{4.802087in}{2.879895in}}%
\pgfpathlineto{\pgfqpoint{4.794524in}{2.869710in}}%
\pgfpathlineto{\pgfqpoint{4.786958in}{2.859641in}}%
\pgfpathlineto{\pgfqpoint{4.779388in}{2.849683in}}%
\pgfpathlineto{\pgfqpoint{4.771815in}{2.839831in}}%
\pgfpathclose%
\pgfusepath{fill}%
\end{pgfscope}%
\begin{pgfscope}%
\pgfpathrectangle{\pgfqpoint{1.150000in}{0.150000in}}{\pgfqpoint{5.700000in}{5.700000in}}%
\pgfusepath{clip}%
\pgfsetbuttcap%
\pgfsetroundjoin%
\definecolor{currentfill}{rgb}{0.271828,0.209303,0.504434}%
\pgfsetfillcolor{currentfill}%
\pgfsetfillopacity{0.800000}%
\pgfsetlinewidth{0.000000pt}%
\definecolor{currentstroke}{rgb}{0.000000,0.000000,0.000000}%
\pgfsetstrokecolor{currentstroke}%
\pgfsetdash{}{0pt}%
\pgfpathmoveto{\pgfqpoint{3.029066in}{2.580921in}}%
\pgfpathlineto{\pgfqpoint{3.042559in}{2.566567in}}%
\pgfpathlineto{\pgfqpoint{3.056048in}{2.552494in}}%
\pgfpathlineto{\pgfqpoint{3.069533in}{2.538699in}}%
\pgfpathlineto{\pgfqpoint{3.083013in}{2.525181in}}%
\pgfpathlineto{\pgfqpoint{3.091140in}{2.535042in}}%
\pgfpathlineto{\pgfqpoint{3.099259in}{2.545004in}}%
\pgfpathlineto{\pgfqpoint{3.107371in}{2.555065in}}%
\pgfpathlineto{\pgfqpoint{3.115476in}{2.565227in}}%
\pgfpathlineto{\pgfqpoint{3.102012in}{2.578715in}}%
\pgfpathlineto{\pgfqpoint{3.088544in}{2.592479in}}%
\pgfpathlineto{\pgfqpoint{3.075071in}{2.606521in}}%
\pgfpathlineto{\pgfqpoint{3.061594in}{2.620843in}}%
\pgfpathlineto{\pgfqpoint{3.053473in}{2.610701in}}%
\pgfpathlineto{\pgfqpoint{3.045345in}{2.600666in}}%
\pgfpathlineto{\pgfqpoint{3.037209in}{2.590739in}}%
\pgfpathlineto{\pgfqpoint{3.029066in}{2.580921in}}%
\pgfpathclose%
\pgfusepath{fill}%
\end{pgfscope}%
\begin{pgfscope}%
\pgfpathrectangle{\pgfqpoint{1.150000in}{0.150000in}}{\pgfqpoint{5.700000in}{5.700000in}}%
\pgfusepath{clip}%
\pgfsetbuttcap%
\pgfsetroundjoin%
\definecolor{currentfill}{rgb}{0.244972,0.287675,0.537260}%
\pgfsetfillcolor{currentfill}%
\pgfsetfillopacity{0.800000}%
\pgfsetlinewidth{0.000000pt}%
\definecolor{currentstroke}{rgb}{0.000000,0.000000,0.000000}%
\pgfsetstrokecolor{currentstroke}%
\pgfsetdash{}{0pt}%
\pgfpathmoveto{\pgfqpoint{2.866669in}{2.775902in}}%
\pgfpathlineto{\pgfqpoint{2.880241in}{2.757994in}}%
\pgfpathlineto{\pgfqpoint{2.893805in}{2.740397in}}%
\pgfpathlineto{\pgfqpoint{2.907362in}{2.723107in}}%
\pgfpathlineto{\pgfqpoint{2.920911in}{2.706123in}}%
\pgfpathlineto{\pgfqpoint{2.929090in}{2.715854in}}%
\pgfpathlineto{\pgfqpoint{2.937262in}{2.725709in}}%
\pgfpathlineto{\pgfqpoint{2.945426in}{2.735688in}}%
\pgfpathlineto{\pgfqpoint{2.953581in}{2.745791in}}%
\pgfpathlineto{\pgfqpoint{2.940051in}{2.762743in}}%
\pgfpathlineto{\pgfqpoint{2.926513in}{2.779999in}}%
\pgfpathlineto{\pgfqpoint{2.912968in}{2.797564in}}%
\pgfpathlineto{\pgfqpoint{2.899415in}{2.815438in}}%
\pgfpathlineto{\pgfqpoint{2.891241in}{2.805356in}}%
\pgfpathlineto{\pgfqpoint{2.883059in}{2.795406in}}%
\pgfpathlineto{\pgfqpoint{2.874868in}{2.785588in}}%
\pgfpathlineto{\pgfqpoint{2.866669in}{2.775902in}}%
\pgfpathclose%
\pgfusepath{fill}%
\end{pgfscope}%
\begin{pgfscope}%
\pgfpathrectangle{\pgfqpoint{1.150000in}{0.150000in}}{\pgfqpoint{5.700000in}{5.700000in}}%
\pgfusepath{clip}%
\pgfsetbuttcap%
\pgfsetroundjoin%
\definecolor{currentfill}{rgb}{0.274128,0.199721,0.498911}%
\pgfsetfillcolor{currentfill}%
\pgfsetfillopacity{0.800000}%
\pgfsetlinewidth{0.000000pt}%
\definecolor{currentstroke}{rgb}{0.000000,0.000000,0.000000}%
\pgfsetstrokecolor{currentstroke}%
\pgfsetdash{}{0pt}%
\pgfpathmoveto{\pgfqpoint{4.175172in}{2.527436in}}%
\pgfpathlineto{\pgfqpoint{4.188715in}{2.526976in}}%
\pgfpathlineto{\pgfqpoint{4.202267in}{2.526713in}}%
\pgfpathlineto{\pgfqpoint{4.215827in}{2.526647in}}%
\pgfpathlineto{\pgfqpoint{4.229396in}{2.526776in}}%
\pgfpathlineto{\pgfqpoint{4.237155in}{2.537045in}}%
\pgfpathlineto{\pgfqpoint{4.244909in}{2.547347in}}%
\pgfpathlineto{\pgfqpoint{4.252658in}{2.557685in}}%
\pgfpathlineto{\pgfqpoint{4.260403in}{2.568062in}}%
\pgfpathlineto{\pgfqpoint{4.246843in}{2.568190in}}%
\pgfpathlineto{\pgfqpoint{4.233292in}{2.568513in}}%
\pgfpathlineto{\pgfqpoint{4.219749in}{2.569032in}}%
\pgfpathlineto{\pgfqpoint{4.206214in}{2.569747in}}%
\pgfpathlineto{\pgfqpoint{4.198461in}{2.559103in}}%
\pgfpathlineto{\pgfqpoint{4.190703in}{2.548505in}}%
\pgfpathlineto{\pgfqpoint{4.182940in}{2.537950in}}%
\pgfpathlineto{\pgfqpoint{4.175172in}{2.527436in}}%
\pgfpathclose%
\pgfusepath{fill}%
\end{pgfscope}%
\begin{pgfscope}%
\pgfpathrectangle{\pgfqpoint{1.150000in}{0.150000in}}{\pgfqpoint{5.700000in}{5.700000in}}%
\pgfusepath{clip}%
\pgfsetbuttcap%
\pgfsetroundjoin%
\definecolor{currentfill}{rgb}{0.218130,0.347432,0.550038}%
\pgfsetfillcolor{currentfill}%
\pgfsetfillopacity{0.800000}%
\pgfsetlinewidth{0.000000pt}%
\definecolor{currentstroke}{rgb}{0.000000,0.000000,0.000000}%
\pgfsetstrokecolor{currentstroke}%
\pgfsetdash{}{0pt}%
\pgfpathmoveto{\pgfqpoint{4.857097in}{2.888539in}}%
\pgfpathlineto{\pgfqpoint{4.870879in}{2.891152in}}%
\pgfpathlineto{\pgfqpoint{4.884673in}{2.893946in}}%
\pgfpathlineto{\pgfqpoint{4.898480in}{2.896920in}}%
\pgfpathlineto{\pgfqpoint{4.912299in}{2.900073in}}%
\pgfpathlineto{\pgfqpoint{4.919828in}{2.909381in}}%
\pgfpathlineto{\pgfqpoint{4.927355in}{2.918802in}}%
\pgfpathlineto{\pgfqpoint{4.934879in}{2.928342in}}%
\pgfpathlineto{\pgfqpoint{4.942400in}{2.938006in}}%
\pgfpathlineto{\pgfqpoint{4.928597in}{2.935365in}}%
\pgfpathlineto{\pgfqpoint{4.914806in}{2.932903in}}%
\pgfpathlineto{\pgfqpoint{4.901027in}{2.930621in}}%
\pgfpathlineto{\pgfqpoint{4.887260in}{2.928519in}}%
\pgfpathlineto{\pgfqpoint{4.879724in}{2.918332in}}%
\pgfpathlineto{\pgfqpoint{4.872184in}{2.908277in}}%
\pgfpathlineto{\pgfqpoint{4.864642in}{2.898348in}}%
\pgfpathlineto{\pgfqpoint{4.857097in}{2.888539in}}%
\pgfpathclose%
\pgfusepath{fill}%
\end{pgfscope}%
\begin{pgfscope}%
\pgfpathrectangle{\pgfqpoint{1.150000in}{0.150000in}}{\pgfqpoint{5.700000in}{5.700000in}}%
\pgfusepath{clip}%
\pgfsetbuttcap%
\pgfsetroundjoin%
\definecolor{currentfill}{rgb}{0.208623,0.367752,0.552675}%
\pgfsetfillcolor{currentfill}%
\pgfsetfillopacity{0.800000}%
\pgfsetlinewidth{0.000000pt}%
\definecolor{currentstroke}{rgb}{0.000000,0.000000,0.000000}%
\pgfsetstrokecolor{currentstroke}%
\pgfsetdash{}{0pt}%
\pgfpathmoveto{\pgfqpoint{4.942400in}{2.938006in}}%
\pgfpathlineto{\pgfqpoint{4.956215in}{2.940825in}}%
\pgfpathlineto{\pgfqpoint{4.970043in}{2.943824in}}%
\pgfpathlineto{\pgfqpoint{4.983884in}{2.947001in}}%
\pgfpathlineto{\pgfqpoint{4.997737in}{2.950356in}}%
\pgfpathlineto{\pgfqpoint{5.005239in}{2.959618in}}%
\pgfpathlineto{\pgfqpoint{5.012738in}{2.969009in}}%
\pgfpathlineto{\pgfqpoint{5.020234in}{2.978535in}}%
\pgfpathlineto{\pgfqpoint{5.027728in}{2.988202in}}%
\pgfpathlineto{\pgfqpoint{5.013892in}{2.985391in}}%
\pgfpathlineto{\pgfqpoint{5.000068in}{2.982758in}}%
\pgfpathlineto{\pgfqpoint{4.986257in}{2.980303in}}%
\pgfpathlineto{\pgfqpoint{4.972458in}{2.978026in}}%
\pgfpathlineto{\pgfqpoint{4.964947in}{2.967804in}}%
\pgfpathlineto{\pgfqpoint{4.957434in}{2.957731in}}%
\pgfpathlineto{\pgfqpoint{4.949918in}{2.947800in}}%
\pgfpathlineto{\pgfqpoint{4.942400in}{2.938006in}}%
\pgfpathclose%
\pgfusepath{fill}%
\end{pgfscope}%
\begin{pgfscope}%
\pgfpathrectangle{\pgfqpoint{1.150000in}{0.150000in}}{\pgfqpoint{5.700000in}{5.700000in}}%
\pgfusepath{clip}%
\pgfsetbuttcap%
\pgfsetroundjoin%
\definecolor{currentfill}{rgb}{0.283187,0.125848,0.444960}%
\pgfsetfillcolor{currentfill}%
\pgfsetfillopacity{0.800000}%
\pgfsetlinewidth{0.000000pt}%
\definecolor{currentstroke}{rgb}{0.000000,0.000000,0.000000}%
\pgfsetstrokecolor{currentstroke}%
\pgfsetdash{}{0pt}%
\pgfpathmoveto{\pgfqpoint{3.694791in}{2.374143in}}%
\pgfpathlineto{\pgfqpoint{3.708229in}{2.369535in}}%
\pgfpathlineto{\pgfqpoint{3.721671in}{2.365143in}}%
\pgfpathlineto{\pgfqpoint{3.735117in}{2.360968in}}%
\pgfpathlineto{\pgfqpoint{3.748567in}{2.357007in}}%
\pgfpathlineto{\pgfqpoint{3.756476in}{2.367680in}}%
\pgfpathlineto{\pgfqpoint{3.764380in}{2.378388in}}%
\pgfpathlineto{\pgfqpoint{3.772279in}{2.389131in}}%
\pgfpathlineto{\pgfqpoint{3.780173in}{2.399912in}}%
\pgfpathlineto{\pgfqpoint{3.766731in}{2.403971in}}%
\pgfpathlineto{\pgfqpoint{3.753294in}{2.408245in}}%
\pgfpathlineto{\pgfqpoint{3.739862in}{2.412734in}}%
\pgfpathlineto{\pgfqpoint{3.726433in}{2.417441in}}%
\pgfpathlineto{\pgfqpoint{3.718530in}{2.406551in}}%
\pgfpathlineto{\pgfqpoint{3.710623in}{2.395705in}}%
\pgfpathlineto{\pgfqpoint{3.702710in}{2.384903in}}%
\pgfpathlineto{\pgfqpoint{3.694791in}{2.374143in}}%
\pgfpathclose%
\pgfusepath{fill}%
\end{pgfscope}%
\begin{pgfscope}%
\pgfpathrectangle{\pgfqpoint{1.150000in}{0.150000in}}{\pgfqpoint{5.700000in}{5.700000in}}%
\pgfusepath{clip}%
\pgfsetbuttcap%
\pgfsetroundjoin%
\definecolor{currentfill}{rgb}{0.277134,0.185228,0.489898}%
\pgfsetfillcolor{currentfill}%
\pgfsetfillopacity{0.800000}%
\pgfsetlinewidth{0.000000pt}%
\definecolor{currentstroke}{rgb}{0.000000,0.000000,0.000000}%
\pgfsetstrokecolor{currentstroke}%
\pgfsetdash{}{0pt}%
\pgfpathmoveto{\pgfqpoint{4.089925in}{2.488643in}}%
\pgfpathlineto{\pgfqpoint{4.103445in}{2.487615in}}%
\pgfpathlineto{\pgfqpoint{4.116974in}{2.486785in}}%
\pgfpathlineto{\pgfqpoint{4.130509in}{2.486155in}}%
\pgfpathlineto{\pgfqpoint{4.144053in}{2.485723in}}%
\pgfpathlineto{\pgfqpoint{4.151840in}{2.496105in}}%
\pgfpathlineto{\pgfqpoint{4.159622in}{2.506516in}}%
\pgfpathlineto{\pgfqpoint{4.167400in}{2.516959in}}%
\pgfpathlineto{\pgfqpoint{4.175172in}{2.527436in}}%
\pgfpathlineto{\pgfqpoint{4.161637in}{2.528093in}}%
\pgfpathlineto{\pgfqpoint{4.148109in}{2.528948in}}%
\pgfpathlineto{\pgfqpoint{4.134590in}{2.530001in}}%
\pgfpathlineto{\pgfqpoint{4.121078in}{2.531254in}}%
\pgfpathlineto{\pgfqpoint{4.113297in}{2.520541in}}%
\pgfpathlineto{\pgfqpoint{4.105511in}{2.509870in}}%
\pgfpathlineto{\pgfqpoint{4.097720in}{2.499238in}}%
\pgfpathlineto{\pgfqpoint{4.089925in}{2.488643in}}%
\pgfpathclose%
\pgfusepath{fill}%
\end{pgfscope}%
\begin{pgfscope}%
\pgfpathrectangle{\pgfqpoint{1.150000in}{0.150000in}}{\pgfqpoint{5.700000in}{5.700000in}}%
\pgfusepath{clip}%
\pgfsetbuttcap%
\pgfsetroundjoin%
\definecolor{currentfill}{rgb}{0.276194,0.190074,0.493001}%
\pgfsetfillcolor{currentfill}%
\pgfsetfillopacity{0.800000}%
\pgfsetlinewidth{0.000000pt}%
\definecolor{currentstroke}{rgb}{0.000000,0.000000,0.000000}%
\pgfsetstrokecolor{currentstroke}%
\pgfsetdash{}{0pt}%
\pgfpathmoveto{\pgfqpoint{3.083013in}{2.525181in}}%
\pgfpathlineto{\pgfqpoint{3.096490in}{2.511936in}}%
\pgfpathlineto{\pgfqpoint{3.109962in}{2.498962in}}%
\pgfpathlineto{\pgfqpoint{3.123431in}{2.486259in}}%
\pgfpathlineto{\pgfqpoint{3.136896in}{2.473823in}}%
\pgfpathlineto{\pgfqpoint{3.145007in}{2.483726in}}%
\pgfpathlineto{\pgfqpoint{3.153110in}{2.493722in}}%
\pgfpathlineto{\pgfqpoint{3.161206in}{2.503810in}}%
\pgfpathlineto{\pgfqpoint{3.169296in}{2.513991in}}%
\pgfpathlineto{\pgfqpoint{3.155846in}{2.526397in}}%
\pgfpathlineto{\pgfqpoint{3.142393in}{2.539070in}}%
\pgfpathlineto{\pgfqpoint{3.128936in}{2.552013in}}%
\pgfpathlineto{\pgfqpoint{3.115476in}{2.565227in}}%
\pgfpathlineto{\pgfqpoint{3.107371in}{2.555065in}}%
\pgfpathlineto{\pgfqpoint{3.099259in}{2.545004in}}%
\pgfpathlineto{\pgfqpoint{3.091140in}{2.535042in}}%
\pgfpathlineto{\pgfqpoint{3.083013in}{2.525181in}}%
\pgfpathclose%
\pgfusepath{fill}%
\end{pgfscope}%
\begin{pgfscope}%
\pgfpathrectangle{\pgfqpoint{1.150000in}{0.150000in}}{\pgfqpoint{5.700000in}{5.700000in}}%
\pgfusepath{clip}%
\pgfsetbuttcap%
\pgfsetroundjoin%
\definecolor{currentfill}{rgb}{0.199430,0.387607,0.554642}%
\pgfsetfillcolor{currentfill}%
\pgfsetfillopacity{0.800000}%
\pgfsetlinewidth{0.000000pt}%
\definecolor{currentstroke}{rgb}{0.000000,0.000000,0.000000}%
\pgfsetstrokecolor{currentstroke}%
\pgfsetdash{}{0pt}%
\pgfpathmoveto{\pgfqpoint{5.027728in}{2.988202in}}%
\pgfpathlineto{\pgfqpoint{5.041577in}{2.991190in}}%
\pgfpathlineto{\pgfqpoint{5.055439in}{2.994355in}}%
\pgfpathlineto{\pgfqpoint{5.069313in}{2.997698in}}%
\pgfpathlineto{\pgfqpoint{5.083201in}{3.001218in}}%
\pgfpathlineto{\pgfqpoint{5.090675in}{3.010468in}}%
\pgfpathlineto{\pgfqpoint{5.098147in}{3.019864in}}%
\pgfpathlineto{\pgfqpoint{5.105616in}{3.029413in}}%
\pgfpathlineto{\pgfqpoint{5.113084in}{3.039122in}}%
\pgfpathlineto{\pgfqpoint{5.099215in}{3.036180in}}%
\pgfpathlineto{\pgfqpoint{5.085359in}{3.033413in}}%
\pgfpathlineto{\pgfqpoint{5.071515in}{3.030823in}}%
\pgfpathlineto{\pgfqpoint{5.057685in}{3.028410in}}%
\pgfpathlineto{\pgfqpoint{5.050198in}{3.018114in}}%
\pgfpathlineto{\pgfqpoint{5.042710in}{3.007985in}}%
\pgfpathlineto{\pgfqpoint{5.035220in}{2.998016in}}%
\pgfpathlineto{\pgfqpoint{5.027728in}{2.988202in}}%
\pgfpathclose%
\pgfusepath{fill}%
\end{pgfscope}%
\begin{pgfscope}%
\pgfpathrectangle{\pgfqpoint{1.150000in}{0.150000in}}{\pgfqpoint{5.700000in}{5.700000in}}%
\pgfusepath{clip}%
\pgfsetbuttcap%
\pgfsetroundjoin%
\definecolor{currentfill}{rgb}{0.231674,0.318106,0.544834}%
\pgfsetfillcolor{currentfill}%
\pgfsetfillopacity{0.800000}%
\pgfsetlinewidth{0.000000pt}%
\definecolor{currentstroke}{rgb}{0.000000,0.000000,0.000000}%
\pgfsetstrokecolor{currentstroke}%
\pgfsetdash{}{0pt}%
\pgfpathmoveto{\pgfqpoint{2.812294in}{2.850697in}}%
\pgfpathlineto{\pgfqpoint{2.825901in}{2.831518in}}%
\pgfpathlineto{\pgfqpoint{2.839499in}{2.812661in}}%
\pgfpathlineto{\pgfqpoint{2.853089in}{2.794123in}}%
\pgfpathlineto{\pgfqpoint{2.866669in}{2.775902in}}%
\pgfpathlineto{\pgfqpoint{2.874868in}{2.785588in}}%
\pgfpathlineto{\pgfqpoint{2.883059in}{2.795406in}}%
\pgfpathlineto{\pgfqpoint{2.891241in}{2.805356in}}%
\pgfpathlineto{\pgfqpoint{2.899415in}{2.815438in}}%
\pgfpathlineto{\pgfqpoint{2.885854in}{2.833626in}}%
\pgfpathlineto{\pgfqpoint{2.872284in}{2.852131in}}%
\pgfpathlineto{\pgfqpoint{2.858706in}{2.870954in}}%
\pgfpathlineto{\pgfqpoint{2.845118in}{2.890099in}}%
\pgfpathlineto{\pgfqpoint{2.836925in}{2.880038in}}%
\pgfpathlineto{\pgfqpoint{2.828723in}{2.870118in}}%
\pgfpathlineto{\pgfqpoint{2.820513in}{2.860337in}}%
\pgfpathlineto{\pgfqpoint{2.812294in}{2.850697in}}%
\pgfpathclose%
\pgfusepath{fill}%
\end{pgfscope}%
\begin{pgfscope}%
\pgfpathrectangle{\pgfqpoint{1.150000in}{0.150000in}}{\pgfqpoint{5.700000in}{5.700000in}}%
\pgfusepath{clip}%
\pgfsetbuttcap%
\pgfsetroundjoin%
\definecolor{currentfill}{rgb}{0.280255,0.165693,0.476498}%
\pgfsetfillcolor{currentfill}%
\pgfsetfillopacity{0.800000}%
\pgfsetlinewidth{0.000000pt}%
\definecolor{currentstroke}{rgb}{0.000000,0.000000,0.000000}%
\pgfsetstrokecolor{currentstroke}%
\pgfsetdash{}{0pt}%
\pgfpathmoveto{\pgfqpoint{4.004653in}{2.451926in}}%
\pgfpathlineto{\pgfqpoint{4.018153in}{2.450285in}}%
\pgfpathlineto{\pgfqpoint{4.031660in}{2.448847in}}%
\pgfpathlineto{\pgfqpoint{4.045174in}{2.447610in}}%
\pgfpathlineto{\pgfqpoint{4.058695in}{2.446574in}}%
\pgfpathlineto{\pgfqpoint{4.066510in}{2.457050in}}%
\pgfpathlineto{\pgfqpoint{4.074320in}{2.467551in}}%
\pgfpathlineto{\pgfqpoint{4.082125in}{2.478082in}}%
\pgfpathlineto{\pgfqpoint{4.089925in}{2.488643in}}%
\pgfpathlineto{\pgfqpoint{4.076412in}{2.489872in}}%
\pgfpathlineto{\pgfqpoint{4.062906in}{2.491302in}}%
\pgfpathlineto{\pgfqpoint{4.049408in}{2.492933in}}%
\pgfpathlineto{\pgfqpoint{4.035916in}{2.494767in}}%
\pgfpathlineto{\pgfqpoint{4.028108in}{2.484001in}}%
\pgfpathlineto{\pgfqpoint{4.020294in}{2.473274in}}%
\pgfpathlineto{\pgfqpoint{4.012476in}{2.462583in}}%
\pgfpathlineto{\pgfqpoint{4.004653in}{2.451926in}}%
\pgfpathclose%
\pgfusepath{fill}%
\end{pgfscope}%
\begin{pgfscope}%
\pgfpathrectangle{\pgfqpoint{1.150000in}{0.150000in}}{\pgfqpoint{5.700000in}{5.700000in}}%
\pgfusepath{clip}%
\pgfsetbuttcap%
\pgfsetroundjoin%
\definecolor{currentfill}{rgb}{0.192357,0.403199,0.555836}%
\pgfsetfillcolor{currentfill}%
\pgfsetfillopacity{0.800000}%
\pgfsetlinewidth{0.000000pt}%
\definecolor{currentstroke}{rgb}{0.000000,0.000000,0.000000}%
\pgfsetstrokecolor{currentstroke}%
\pgfsetdash{}{0pt}%
\pgfpathmoveto{\pgfqpoint{5.113084in}{3.039122in}}%
\pgfpathlineto{\pgfqpoint{5.126967in}{3.042241in}}%
\pgfpathlineto{\pgfqpoint{5.140862in}{3.045536in}}%
\pgfpathlineto{\pgfqpoint{5.154771in}{3.049007in}}%
\pgfpathlineto{\pgfqpoint{5.168693in}{3.052653in}}%
\pgfpathlineto{\pgfqpoint{5.176140in}{3.061931in}}%
\pgfpathlineto{\pgfqpoint{5.183586in}{3.071375in}}%
\pgfpathlineto{\pgfqpoint{5.191030in}{3.080991in}}%
\pgfpathlineto{\pgfqpoint{5.198473in}{3.090787in}}%
\pgfpathlineto{\pgfqpoint{5.184571in}{3.087749in}}%
\pgfpathlineto{\pgfqpoint{5.170682in}{3.084887in}}%
\pgfpathlineto{\pgfqpoint{5.156807in}{3.082200in}}%
\pgfpathlineto{\pgfqpoint{5.142944in}{3.079688in}}%
\pgfpathlineto{\pgfqpoint{5.135480in}{3.069274in}}%
\pgfpathlineto{\pgfqpoint{5.128016in}{3.059046in}}%
\pgfpathlineto{\pgfqpoint{5.120551in}{3.048998in}}%
\pgfpathlineto{\pgfqpoint{5.113084in}{3.039122in}}%
\pgfpathclose%
\pgfusepath{fill}%
\end{pgfscope}%
\begin{pgfscope}%
\pgfpathrectangle{\pgfqpoint{1.150000in}{0.150000in}}{\pgfqpoint{5.700000in}{5.700000in}}%
\pgfusepath{clip}%
\pgfsetbuttcap%
\pgfsetroundjoin%
\definecolor{currentfill}{rgb}{0.283229,0.120777,0.440584}%
\pgfsetfillcolor{currentfill}%
\pgfsetfillopacity{0.800000}%
\pgfsetlinewidth{0.000000pt}%
\definecolor{currentstroke}{rgb}{0.000000,0.000000,0.000000}%
\pgfsetstrokecolor{currentstroke}%
\pgfsetdash{}{0pt}%
\pgfpathmoveto{\pgfqpoint{3.470014in}{2.360834in}}%
\pgfpathlineto{\pgfqpoint{3.483438in}{2.353635in}}%
\pgfpathlineto{\pgfqpoint{3.496863in}{2.346666in}}%
\pgfpathlineto{\pgfqpoint{3.510291in}{2.339927in}}%
\pgfpathlineto{\pgfqpoint{3.523720in}{2.333417in}}%
\pgfpathlineto{\pgfqpoint{3.531702in}{2.343933in}}%
\pgfpathlineto{\pgfqpoint{3.539678in}{2.354497in}}%
\pgfpathlineto{\pgfqpoint{3.547648in}{2.365110in}}%
\pgfpathlineto{\pgfqpoint{3.555613in}{2.375774in}}%
\pgfpathlineto{\pgfqpoint{3.542194in}{2.382319in}}%
\pgfpathlineto{\pgfqpoint{3.528778in}{2.389092in}}%
\pgfpathlineto{\pgfqpoint{3.515363in}{2.396096in}}%
\pgfpathlineto{\pgfqpoint{3.501950in}{2.403330in}}%
\pgfpathlineto{\pgfqpoint{3.493975in}{2.392620in}}%
\pgfpathlineto{\pgfqpoint{3.485993in}{2.381968in}}%
\pgfpathlineto{\pgfqpoint{3.478006in}{2.371373in}}%
\pgfpathlineto{\pgfqpoint{3.470014in}{2.360834in}}%
\pgfpathclose%
\pgfusepath{fill}%
\end{pgfscope}%
\begin{pgfscope}%
\pgfpathrectangle{\pgfqpoint{1.150000in}{0.150000in}}{\pgfqpoint{5.700000in}{5.700000in}}%
\pgfusepath{clip}%
\pgfsetbuttcap%
\pgfsetroundjoin%
\definecolor{currentfill}{rgb}{0.283072,0.130895,0.449241}%
\pgfsetfillcolor{currentfill}%
\pgfsetfillopacity{0.800000}%
\pgfsetlinewidth{0.000000pt}%
\definecolor{currentstroke}{rgb}{0.000000,0.000000,0.000000}%
\pgfsetstrokecolor{currentstroke}%
\pgfsetdash{}{0pt}%
\pgfpathmoveto{\pgfqpoint{3.330535in}{2.385294in}}%
\pgfpathlineto{\pgfqpoint{3.343965in}{2.376197in}}%
\pgfpathlineto{\pgfqpoint{3.357395in}{2.367343in}}%
\pgfpathlineto{\pgfqpoint{3.370825in}{2.358730in}}%
\pgfpathlineto{\pgfqpoint{3.384256in}{2.350357in}}%
\pgfpathlineto{\pgfqpoint{3.392284in}{2.360667in}}%
\pgfpathlineto{\pgfqpoint{3.400306in}{2.371040in}}%
\pgfpathlineto{\pgfqpoint{3.408322in}{2.381475in}}%
\pgfpathlineto{\pgfqpoint{3.416332in}{2.391973in}}%
\pgfpathlineto{\pgfqpoint{3.402914in}{2.400349in}}%
\pgfpathlineto{\pgfqpoint{3.389496in}{2.408965in}}%
\pgfpathlineto{\pgfqpoint{3.376079in}{2.417823in}}%
\pgfpathlineto{\pgfqpoint{3.362661in}{2.426922in}}%
\pgfpathlineto{\pgfqpoint{3.354639in}{2.416410in}}%
\pgfpathlineto{\pgfqpoint{3.346610in}{2.405968in}}%
\pgfpathlineto{\pgfqpoint{3.338576in}{2.395596in}}%
\pgfpathlineto{\pgfqpoint{3.330535in}{2.385294in}}%
\pgfpathclose%
\pgfusepath{fill}%
\end{pgfscope}%
\begin{pgfscope}%
\pgfpathrectangle{\pgfqpoint{1.150000in}{0.150000in}}{\pgfqpoint{5.700000in}{5.700000in}}%
\pgfusepath{clip}%
\pgfsetbuttcap%
\pgfsetroundjoin%
\definecolor{currentfill}{rgb}{0.183898,0.422383,0.556944}%
\pgfsetfillcolor{currentfill}%
\pgfsetfillopacity{0.800000}%
\pgfsetlinewidth{0.000000pt}%
\definecolor{currentstroke}{rgb}{0.000000,0.000000,0.000000}%
\pgfsetstrokecolor{currentstroke}%
\pgfsetdash{}{0pt}%
\pgfpathmoveto{\pgfqpoint{5.198473in}{3.090787in}}%
\pgfpathlineto{\pgfqpoint{5.212388in}{3.093999in}}%
\pgfpathlineto{\pgfqpoint{5.226317in}{3.097386in}}%
\pgfpathlineto{\pgfqpoint{5.240260in}{3.100947in}}%
\pgfpathlineto{\pgfqpoint{5.254216in}{3.104682in}}%
\pgfpathlineto{\pgfqpoint{5.261638in}{3.114035in}}%
\pgfpathlineto{\pgfqpoint{5.269058in}{3.123574in}}%
\pgfpathlineto{\pgfqpoint{5.276478in}{3.133305in}}%
\pgfpathlineto{\pgfqpoint{5.283898in}{3.143237in}}%
\pgfpathlineto{\pgfqpoint{5.269964in}{3.140143in}}%
\pgfpathlineto{\pgfqpoint{5.256043in}{3.137222in}}%
\pgfpathlineto{\pgfqpoint{5.242135in}{3.134475in}}%
\pgfpathlineto{\pgfqpoint{5.228241in}{3.131902in}}%
\pgfpathlineto{\pgfqpoint{5.220799in}{3.121319in}}%
\pgfpathlineto{\pgfqpoint{5.213358in}{3.110943in}}%
\pgfpathlineto{\pgfqpoint{5.205916in}{3.100768in}}%
\pgfpathlineto{\pgfqpoint{5.198473in}{3.090787in}}%
\pgfpathclose%
\pgfusepath{fill}%
\end{pgfscope}%
\begin{pgfscope}%
\pgfpathrectangle{\pgfqpoint{1.150000in}{0.150000in}}{\pgfqpoint{5.700000in}{5.700000in}}%
\pgfusepath{clip}%
\pgfsetbuttcap%
\pgfsetroundjoin%
\definecolor{currentfill}{rgb}{0.279574,0.170599,0.479997}%
\pgfsetfillcolor{currentfill}%
\pgfsetfillopacity{0.800000}%
\pgfsetlinewidth{0.000000pt}%
\definecolor{currentstroke}{rgb}{0.000000,0.000000,0.000000}%
\pgfsetstrokecolor{currentstroke}%
\pgfsetdash{}{0pt}%
\pgfpathmoveto{\pgfqpoint{3.136896in}{2.473823in}}%
\pgfpathlineto{\pgfqpoint{3.150359in}{2.461652in}}%
\pgfpathlineto{\pgfqpoint{3.163819in}{2.449746in}}%
\pgfpathlineto{\pgfqpoint{3.177276in}{2.438101in}}%
\pgfpathlineto{\pgfqpoint{3.190730in}{2.426716in}}%
\pgfpathlineto{\pgfqpoint{3.198825in}{2.436661in}}%
\pgfpathlineto{\pgfqpoint{3.206913in}{2.446690in}}%
\pgfpathlineto{\pgfqpoint{3.214995in}{2.456804in}}%
\pgfpathlineto{\pgfqpoint{3.223069in}{2.467003in}}%
\pgfpathlineto{\pgfqpoint{3.209629in}{2.478358in}}%
\pgfpathlineto{\pgfqpoint{3.196187in}{2.489973in}}%
\pgfpathlineto{\pgfqpoint{3.182743in}{2.501850in}}%
\pgfpathlineto{\pgfqpoint{3.169296in}{2.513991in}}%
\pgfpathlineto{\pgfqpoint{3.161206in}{2.503810in}}%
\pgfpathlineto{\pgfqpoint{3.153110in}{2.493722in}}%
\pgfpathlineto{\pgfqpoint{3.145007in}{2.483726in}}%
\pgfpathlineto{\pgfqpoint{3.136896in}{2.473823in}}%
\pgfpathclose%
\pgfusepath{fill}%
\end{pgfscope}%
\begin{pgfscope}%
\pgfpathrectangle{\pgfqpoint{1.150000in}{0.150000in}}{\pgfqpoint{5.700000in}{5.700000in}}%
\pgfusepath{clip}%
\pgfsetbuttcap%
\pgfsetroundjoin%
\definecolor{currentfill}{rgb}{0.281887,0.150881,0.465405}%
\pgfsetfillcolor{currentfill}%
\pgfsetfillopacity{0.800000}%
\pgfsetlinewidth{0.000000pt}%
\definecolor{currentstroke}{rgb}{0.000000,0.000000,0.000000}%
\pgfsetstrokecolor{currentstroke}%
\pgfsetdash{}{0pt}%
\pgfpathmoveto{\pgfqpoint{3.919345in}{2.417550in}}%
\pgfpathlineto{\pgfqpoint{3.932827in}{2.415252in}}%
\pgfpathlineto{\pgfqpoint{3.946315in}{2.413161in}}%
\pgfpathlineto{\pgfqpoint{3.959810in}{2.411274in}}%
\pgfpathlineto{\pgfqpoint{3.973311in}{2.409591in}}%
\pgfpathlineto{\pgfqpoint{3.981154in}{2.420135in}}%
\pgfpathlineto{\pgfqpoint{3.988992in}{2.430705in}}%
\pgfpathlineto{\pgfqpoint{3.996825in}{2.441301in}}%
\pgfpathlineto{\pgfqpoint{4.004653in}{2.451926in}}%
\pgfpathlineto{\pgfqpoint{3.991160in}{2.453770in}}%
\pgfpathlineto{\pgfqpoint{3.977674in}{2.455819in}}%
\pgfpathlineto{\pgfqpoint{3.964194in}{2.458072in}}%
\pgfpathlineto{\pgfqpoint{3.950720in}{2.460531in}}%
\pgfpathlineto{\pgfqpoint{3.942884in}{2.449732in}}%
\pgfpathlineto{\pgfqpoint{3.935042in}{2.438971in}}%
\pgfpathlineto{\pgfqpoint{3.927196in}{2.428244in}}%
\pgfpathlineto{\pgfqpoint{3.919345in}{2.417550in}}%
\pgfpathclose%
\pgfusepath{fill}%
\end{pgfscope}%
\begin{pgfscope}%
\pgfpathrectangle{\pgfqpoint{1.150000in}{0.150000in}}{\pgfqpoint{5.700000in}{5.700000in}}%
\pgfusepath{clip}%
\pgfsetbuttcap%
\pgfsetroundjoin%
\definecolor{currentfill}{rgb}{0.216210,0.351535,0.550627}%
\pgfsetfillcolor{currentfill}%
\pgfsetfillopacity{0.800000}%
\pgfsetlinewidth{0.000000pt}%
\definecolor{currentstroke}{rgb}{0.000000,0.000000,0.000000}%
\pgfsetstrokecolor{currentstroke}%
\pgfsetdash{}{0pt}%
\pgfpathmoveto{\pgfqpoint{2.757765in}{2.930698in}}%
\pgfpathlineto{\pgfqpoint{2.771412in}{2.910199in}}%
\pgfpathlineto{\pgfqpoint{2.785049in}{2.890034in}}%
\pgfpathlineto{\pgfqpoint{2.798676in}{2.870201in}}%
\pgfpathlineto{\pgfqpoint{2.812294in}{2.850697in}}%
\pgfpathlineto{\pgfqpoint{2.820513in}{2.860337in}}%
\pgfpathlineto{\pgfqpoint{2.828723in}{2.870118in}}%
\pgfpathlineto{\pgfqpoint{2.836925in}{2.880038in}}%
\pgfpathlineto{\pgfqpoint{2.845118in}{2.890099in}}%
\pgfpathlineto{\pgfqpoint{2.831522in}{2.909570in}}%
\pgfpathlineto{\pgfqpoint{2.817915in}{2.929368in}}%
\pgfpathlineto{\pgfqpoint{2.804299in}{2.949498in}}%
\pgfpathlineto{\pgfqpoint{2.790672in}{2.969963in}}%
\pgfpathlineto{\pgfqpoint{2.782459in}{2.959924in}}%
\pgfpathlineto{\pgfqpoint{2.774236in}{2.950034in}}%
\pgfpathlineto{\pgfqpoint{2.766005in}{2.940292in}}%
\pgfpathlineto{\pgfqpoint{2.757765in}{2.930698in}}%
\pgfpathclose%
\pgfusepath{fill}%
\end{pgfscope}%
\begin{pgfscope}%
\pgfpathrectangle{\pgfqpoint{1.150000in}{0.150000in}}{\pgfqpoint{5.700000in}{5.700000in}}%
\pgfusepath{clip}%
\pgfsetbuttcap%
\pgfsetroundjoin%
\definecolor{currentfill}{rgb}{0.283229,0.120777,0.440584}%
\pgfsetfillcolor{currentfill}%
\pgfsetfillopacity{0.800000}%
\pgfsetlinewidth{0.000000pt}%
\definecolor{currentstroke}{rgb}{0.000000,0.000000,0.000000}%
\pgfsetstrokecolor{currentstroke}%
\pgfsetdash{}{0pt}%
\pgfpathmoveto{\pgfqpoint{3.609314in}{2.351854in}}%
\pgfpathlineto{\pgfqpoint{3.622747in}{2.346434in}}%
\pgfpathlineto{\pgfqpoint{3.636183in}{2.341235in}}%
\pgfpathlineto{\pgfqpoint{3.649622in}{2.336256in}}%
\pgfpathlineto{\pgfqpoint{3.663066in}{2.331496in}}%
\pgfpathlineto{\pgfqpoint{3.671005in}{2.342101in}}%
\pgfpathlineto{\pgfqpoint{3.678939in}{2.352744in}}%
\pgfpathlineto{\pgfqpoint{3.686868in}{2.363424in}}%
\pgfpathlineto{\pgfqpoint{3.694791in}{2.374143in}}%
\pgfpathlineto{\pgfqpoint{3.681358in}{2.378970in}}%
\pgfpathlineto{\pgfqpoint{3.667928in}{2.384016in}}%
\pgfpathlineto{\pgfqpoint{3.654501in}{2.389282in}}%
\pgfpathlineto{\pgfqpoint{3.641079in}{2.394769in}}%
\pgfpathlineto{\pgfqpoint{3.633145in}{2.383971in}}%
\pgfpathlineto{\pgfqpoint{3.625207in}{2.373220in}}%
\pgfpathlineto{\pgfqpoint{3.617263in}{2.362515in}}%
\pgfpathlineto{\pgfqpoint{3.609314in}{2.351854in}}%
\pgfpathclose%
\pgfusepath{fill}%
\end{pgfscope}%
\begin{pgfscope}%
\pgfpathrectangle{\pgfqpoint{1.150000in}{0.150000in}}{\pgfqpoint{5.700000in}{5.700000in}}%
\pgfusepath{clip}%
\pgfsetbuttcap%
\pgfsetroundjoin%
\definecolor{currentfill}{rgb}{0.175841,0.441290,0.557685}%
\pgfsetfillcolor{currentfill}%
\pgfsetfillopacity{0.800000}%
\pgfsetlinewidth{0.000000pt}%
\definecolor{currentstroke}{rgb}{0.000000,0.000000,0.000000}%
\pgfsetstrokecolor{currentstroke}%
\pgfsetdash{}{0pt}%
\pgfpathmoveto{\pgfqpoint{5.283898in}{3.143237in}}%
\pgfpathlineto{\pgfqpoint{5.297846in}{3.146505in}}%
\pgfpathlineto{\pgfqpoint{5.311808in}{3.149947in}}%
\pgfpathlineto{\pgfqpoint{5.325784in}{3.153561in}}%
\pgfpathlineto{\pgfqpoint{5.339774in}{3.157349in}}%
\pgfpathlineto{\pgfqpoint{5.347172in}{3.166828in}}%
\pgfpathlineto{\pgfqpoint{5.354569in}{3.176515in}}%
\pgfpathlineto{\pgfqpoint{5.361967in}{3.186416in}}%
\pgfpathlineto{\pgfqpoint{5.369366in}{3.196541in}}%
\pgfpathlineto{\pgfqpoint{5.355399in}{3.193426in}}%
\pgfpathlineto{\pgfqpoint{5.341447in}{3.190484in}}%
\pgfpathlineto{\pgfqpoint{5.327508in}{3.187714in}}%
\pgfpathlineto{\pgfqpoint{5.313582in}{3.185117in}}%
\pgfpathlineto{\pgfqpoint{5.306160in}{3.174310in}}%
\pgfpathlineto{\pgfqpoint{5.298739in}{3.163732in}}%
\pgfpathlineto{\pgfqpoint{5.291318in}{3.153377in}}%
\pgfpathlineto{\pgfqpoint{5.283898in}{3.143237in}}%
\pgfpathclose%
\pgfusepath{fill}%
\end{pgfscope}%
\begin{pgfscope}%
\pgfpathrectangle{\pgfqpoint{1.150000in}{0.150000in}}{\pgfqpoint{5.700000in}{5.700000in}}%
\pgfusepath{clip}%
\pgfsetbuttcap%
\pgfsetroundjoin%
\definecolor{currentfill}{rgb}{0.166617,0.463708,0.558119}%
\pgfsetfillcolor{currentfill}%
\pgfsetfillopacity{0.800000}%
\pgfsetlinewidth{0.000000pt}%
\definecolor{currentstroke}{rgb}{0.000000,0.000000,0.000000}%
\pgfsetstrokecolor{currentstroke}%
\pgfsetdash{}{0pt}%
\pgfpathmoveto{\pgfqpoint{5.369366in}{3.196541in}}%
\pgfpathlineto{\pgfqpoint{5.383346in}{3.199828in}}%
\pgfpathlineto{\pgfqpoint{5.397340in}{3.203287in}}%
\pgfpathlineto{\pgfqpoint{5.411349in}{3.206919in}}%
\pgfpathlineto{\pgfqpoint{5.425372in}{3.210722in}}%
\pgfpathlineto{\pgfqpoint{5.432747in}{3.220384in}}%
\pgfpathlineto{\pgfqpoint{5.440123in}{3.230277in}}%
\pgfpathlineto{\pgfqpoint{5.447501in}{3.240410in}}%
\pgfpathlineto{\pgfqpoint{5.454881in}{3.250789in}}%
\pgfpathlineto{\pgfqpoint{5.440883in}{3.247690in}}%
\pgfpathlineto{\pgfqpoint{5.426900in}{3.244763in}}%
\pgfpathlineto{\pgfqpoint{5.412930in}{3.242008in}}%
\pgfpathlineto{\pgfqpoint{5.398974in}{3.239424in}}%
\pgfpathlineto{\pgfqpoint{5.391569in}{3.228330in}}%
\pgfpathlineto{\pgfqpoint{5.384166in}{3.217490in}}%
\pgfpathlineto{\pgfqpoint{5.376765in}{3.206896in}}%
\pgfpathlineto{\pgfqpoint{5.369366in}{3.196541in}}%
\pgfpathclose%
\pgfusepath{fill}%
\end{pgfscope}%
\begin{pgfscope}%
\pgfpathrectangle{\pgfqpoint{1.150000in}{0.150000in}}{\pgfqpoint{5.700000in}{5.700000in}}%
\pgfusepath{clip}%
\pgfsetbuttcap%
\pgfsetroundjoin%
\definecolor{currentfill}{rgb}{0.282623,0.140926,0.457517}%
\pgfsetfillcolor{currentfill}%
\pgfsetfillopacity{0.800000}%
\pgfsetlinewidth{0.000000pt}%
\definecolor{currentstroke}{rgb}{0.000000,0.000000,0.000000}%
\pgfsetstrokecolor{currentstroke}%
\pgfsetdash{}{0pt}%
\pgfpathmoveto{\pgfqpoint{3.833988in}{2.385805in}}%
\pgfpathlineto{\pgfqpoint{3.847455in}{2.382806in}}%
\pgfpathlineto{\pgfqpoint{3.860927in}{2.380016in}}%
\pgfpathlineto{\pgfqpoint{3.874406in}{2.377435in}}%
\pgfpathlineto{\pgfqpoint{3.887890in}{2.375061in}}%
\pgfpathlineto{\pgfqpoint{3.895761in}{2.385644in}}%
\pgfpathlineto{\pgfqpoint{3.903627in}{2.396252in}}%
\pgfpathlineto{\pgfqpoint{3.911489in}{2.406887in}}%
\pgfpathlineto{\pgfqpoint{3.919345in}{2.417550in}}%
\pgfpathlineto{\pgfqpoint{3.905869in}{2.420054in}}%
\pgfpathlineto{\pgfqpoint{3.892399in}{2.422766in}}%
\pgfpathlineto{\pgfqpoint{3.878935in}{2.425685in}}%
\pgfpathlineto{\pgfqpoint{3.865477in}{2.428814in}}%
\pgfpathlineto{\pgfqpoint{3.857612in}{2.418009in}}%
\pgfpathlineto{\pgfqpoint{3.849743in}{2.407241in}}%
\pgfpathlineto{\pgfqpoint{3.841868in}{2.396506in}}%
\pgfpathlineto{\pgfqpoint{3.833988in}{2.385805in}}%
\pgfpathclose%
\pgfusepath{fill}%
\end{pgfscope}%
\begin{pgfscope}%
\pgfpathrectangle{\pgfqpoint{1.150000in}{0.150000in}}{\pgfqpoint{5.700000in}{5.700000in}}%
\pgfusepath{clip}%
\pgfsetbuttcap%
\pgfsetroundjoin%
\definecolor{currentfill}{rgb}{0.281887,0.150881,0.465405}%
\pgfsetfillcolor{currentfill}%
\pgfsetfillopacity{0.800000}%
\pgfsetlinewidth{0.000000pt}%
\definecolor{currentstroke}{rgb}{0.000000,0.000000,0.000000}%
\pgfsetstrokecolor{currentstroke}%
\pgfsetdash{}{0pt}%
\pgfpathmoveto{\pgfqpoint{3.190730in}{2.426716in}}%
\pgfpathlineto{\pgfqpoint{3.204183in}{2.415589in}}%
\pgfpathlineto{\pgfqpoint{3.217634in}{2.404718in}}%
\pgfpathlineto{\pgfqpoint{3.231083in}{2.394102in}}%
\pgfpathlineto{\pgfqpoint{3.244530in}{2.383738in}}%
\pgfpathlineto{\pgfqpoint{3.252610in}{2.393724in}}%
\pgfpathlineto{\pgfqpoint{3.260683in}{2.403786in}}%
\pgfpathlineto{\pgfqpoint{3.268750in}{2.413926in}}%
\pgfpathlineto{\pgfqpoint{3.276811in}{2.424141in}}%
\pgfpathlineto{\pgfqpoint{3.263378in}{2.434476in}}%
\pgfpathlineto{\pgfqpoint{3.249943in}{2.445063in}}%
\pgfpathlineto{\pgfqpoint{3.236507in}{2.455905in}}%
\pgfpathlineto{\pgfqpoint{3.223069in}{2.467003in}}%
\pgfpathlineto{\pgfqpoint{3.214995in}{2.456804in}}%
\pgfpathlineto{\pgfqpoint{3.206913in}{2.446690in}}%
\pgfpathlineto{\pgfqpoint{3.198825in}{2.436661in}}%
\pgfpathlineto{\pgfqpoint{3.190730in}{2.426716in}}%
\pgfpathclose%
\pgfusepath{fill}%
\end{pgfscope}%
\begin{pgfscope}%
\pgfpathrectangle{\pgfqpoint{1.150000in}{0.150000in}}{\pgfqpoint{5.700000in}{5.700000in}}%
\pgfusepath{clip}%
\pgfsetbuttcap%
\pgfsetroundjoin%
\definecolor{currentfill}{rgb}{0.201239,0.383670,0.554294}%
\pgfsetfillcolor{currentfill}%
\pgfsetfillopacity{0.800000}%
\pgfsetlinewidth{0.000000pt}%
\definecolor{currentstroke}{rgb}{0.000000,0.000000,0.000000}%
\pgfsetstrokecolor{currentstroke}%
\pgfsetdash{}{0pt}%
\pgfpathmoveto{\pgfqpoint{2.703061in}{3.016110in}}%
\pgfpathlineto{\pgfqpoint{2.716755in}{2.994238in}}%
\pgfpathlineto{\pgfqpoint{2.730436in}{2.972715in}}%
\pgfpathlineto{\pgfqpoint{2.744106in}{2.951536in}}%
\pgfpathlineto{\pgfqpoint{2.757765in}{2.930698in}}%
\pgfpathlineto{\pgfqpoint{2.766005in}{2.940292in}}%
\pgfpathlineto{\pgfqpoint{2.774236in}{2.950034in}}%
\pgfpathlineto{\pgfqpoint{2.782459in}{2.959924in}}%
\pgfpathlineto{\pgfqpoint{2.790672in}{2.969963in}}%
\pgfpathlineto{\pgfqpoint{2.777034in}{2.990766in}}%
\pgfpathlineto{\pgfqpoint{2.763386in}{3.011909in}}%
\pgfpathlineto{\pgfqpoint{2.749726in}{3.033398in}}%
\pgfpathlineto{\pgfqpoint{2.736055in}{3.055234in}}%
\pgfpathlineto{\pgfqpoint{2.727821in}{3.045219in}}%
\pgfpathlineto{\pgfqpoint{2.719577in}{3.035359in}}%
\pgfpathlineto{\pgfqpoint{2.711324in}{3.025657in}}%
\pgfpathlineto{\pgfqpoint{2.703061in}{3.016110in}}%
\pgfpathclose%
\pgfusepath{fill}%
\end{pgfscope}%
\begin{pgfscope}%
\pgfpathrectangle{\pgfqpoint{1.150000in}{0.150000in}}{\pgfqpoint{5.700000in}{5.700000in}}%
\pgfusepath{clip}%
\pgfsetbuttcap%
\pgfsetroundjoin%
\definecolor{currentfill}{rgb}{0.159194,0.482237,0.558073}%
\pgfsetfillcolor{currentfill}%
\pgfsetfillopacity{0.800000}%
\pgfsetlinewidth{0.000000pt}%
\definecolor{currentstroke}{rgb}{0.000000,0.000000,0.000000}%
\pgfsetstrokecolor{currentstroke}%
\pgfsetdash{}{0pt}%
\pgfpathmoveto{\pgfqpoint{5.454881in}{3.250789in}}%
\pgfpathlineto{\pgfqpoint{5.468893in}{3.254058in}}%
\pgfpathlineto{\pgfqpoint{5.482919in}{3.257498in}}%
\pgfpathlineto{\pgfqpoint{5.496959in}{3.261110in}}%
\pgfpathlineto{\pgfqpoint{5.511015in}{3.264892in}}%
\pgfpathlineto{\pgfqpoint{5.518370in}{3.274801in}}%
\pgfpathlineto{\pgfqpoint{5.525728in}{3.284965in}}%
\pgfpathlineto{\pgfqpoint{5.533088in}{3.295394in}}%
\pgfpathlineto{\pgfqpoint{5.540452in}{3.306095in}}%
\pgfpathlineto{\pgfqpoint{5.526424in}{3.303050in}}%
\pgfpathlineto{\pgfqpoint{5.512410in}{3.300175in}}%
\pgfpathlineto{\pgfqpoint{5.498410in}{3.297470in}}%
\pgfpathlineto{\pgfqpoint{5.484425in}{3.294936in}}%
\pgfpathlineto{\pgfqpoint{5.477034in}{3.283488in}}%
\pgfpathlineto{\pgfqpoint{5.469647in}{3.272319in}}%
\pgfpathlineto{\pgfqpoint{5.462263in}{3.261422in}}%
\pgfpathlineto{\pgfqpoint{5.454881in}{3.250789in}}%
\pgfpathclose%
\pgfusepath{fill}%
\end{pgfscope}%
\begin{pgfscope}%
\pgfpathrectangle{\pgfqpoint{1.150000in}{0.150000in}}{\pgfqpoint{5.700000in}{5.700000in}}%
\pgfusepath{clip}%
\pgfsetbuttcap%
\pgfsetroundjoin%
\definecolor{currentfill}{rgb}{0.283229,0.120777,0.440584}%
\pgfsetfillcolor{currentfill}%
\pgfsetfillopacity{0.800000}%
\pgfsetlinewidth{0.000000pt}%
\definecolor{currentstroke}{rgb}{0.000000,0.000000,0.000000}%
\pgfsetstrokecolor{currentstroke}%
\pgfsetdash{}{0pt}%
\pgfpathmoveto{\pgfqpoint{3.384256in}{2.350357in}}%
\pgfpathlineto{\pgfqpoint{3.397687in}{2.342221in}}%
\pgfpathlineto{\pgfqpoint{3.411119in}{2.334323in}}%
\pgfpathlineto{\pgfqpoint{3.424551in}{2.326660in}}%
\pgfpathlineto{\pgfqpoint{3.437985in}{2.319232in}}%
\pgfpathlineto{\pgfqpoint{3.446001in}{2.329551in}}%
\pgfpathlineto{\pgfqpoint{3.454011in}{2.339924in}}%
\pgfpathlineto{\pgfqpoint{3.462015in}{2.350352in}}%
\pgfpathlineto{\pgfqpoint{3.470014in}{2.360834in}}%
\pgfpathlineto{\pgfqpoint{3.456592in}{2.368267in}}%
\pgfpathlineto{\pgfqpoint{3.443171in}{2.375933in}}%
\pgfpathlineto{\pgfqpoint{3.429751in}{2.383835in}}%
\pgfpathlineto{\pgfqpoint{3.416332in}{2.391973in}}%
\pgfpathlineto{\pgfqpoint{3.408322in}{2.381475in}}%
\pgfpathlineto{\pgfqpoint{3.400306in}{2.371040in}}%
\pgfpathlineto{\pgfqpoint{3.392284in}{2.360667in}}%
\pgfpathlineto{\pgfqpoint{3.384256in}{2.350357in}}%
\pgfpathclose%
\pgfusepath{fill}%
\end{pgfscope}%
\begin{pgfscope}%
\pgfpathrectangle{\pgfqpoint{1.150000in}{0.150000in}}{\pgfqpoint{5.700000in}{5.700000in}}%
\pgfusepath{clip}%
\pgfsetbuttcap%
\pgfsetroundjoin%
\definecolor{currentfill}{rgb}{0.283072,0.130895,0.449241}%
\pgfsetfillcolor{currentfill}%
\pgfsetfillopacity{0.800000}%
\pgfsetlinewidth{0.000000pt}%
\definecolor{currentstroke}{rgb}{0.000000,0.000000,0.000000}%
\pgfsetstrokecolor{currentstroke}%
\pgfsetdash{}{0pt}%
\pgfpathmoveto{\pgfqpoint{3.748567in}{2.357007in}}%
\pgfpathlineto{\pgfqpoint{3.762022in}{2.353260in}}%
\pgfpathlineto{\pgfqpoint{3.775482in}{2.349726in}}%
\pgfpathlineto{\pgfqpoint{3.788948in}{2.346405in}}%
\pgfpathlineto{\pgfqpoint{3.802418in}{2.343294in}}%
\pgfpathlineto{\pgfqpoint{3.810318in}{2.353881in}}%
\pgfpathlineto{\pgfqpoint{3.818213in}{2.364494in}}%
\pgfpathlineto{\pgfqpoint{3.826103in}{2.375135in}}%
\pgfpathlineto{\pgfqpoint{3.833988in}{2.385805in}}%
\pgfpathlineto{\pgfqpoint{3.820526in}{2.389014in}}%
\pgfpathlineto{\pgfqpoint{3.807070in}{2.392435in}}%
\pgfpathlineto{\pgfqpoint{3.793619in}{2.396067in}}%
\pgfpathlineto{\pgfqpoint{3.780173in}{2.399912in}}%
\pgfpathlineto{\pgfqpoint{3.772279in}{2.389131in}}%
\pgfpathlineto{\pgfqpoint{3.764380in}{2.378388in}}%
\pgfpathlineto{\pgfqpoint{3.756476in}{2.367680in}}%
\pgfpathlineto{\pgfqpoint{3.748567in}{2.357007in}}%
\pgfpathclose%
\pgfusepath{fill}%
\end{pgfscope}%
\begin{pgfscope}%
\pgfpathrectangle{\pgfqpoint{1.150000in}{0.150000in}}{\pgfqpoint{5.700000in}{5.700000in}}%
\pgfusepath{clip}%
\pgfsetbuttcap%
\pgfsetroundjoin%
\definecolor{currentfill}{rgb}{0.283197,0.115680,0.436115}%
\pgfsetfillcolor{currentfill}%
\pgfsetfillopacity{0.800000}%
\pgfsetlinewidth{0.000000pt}%
\definecolor{currentstroke}{rgb}{0.000000,0.000000,0.000000}%
\pgfsetstrokecolor{currentstroke}%
\pgfsetdash{}{0pt}%
\pgfpathmoveto{\pgfqpoint{3.523720in}{2.333417in}}%
\pgfpathlineto{\pgfqpoint{3.537152in}{2.327135in}}%
\pgfpathlineto{\pgfqpoint{3.550586in}{2.321078in}}%
\pgfpathlineto{\pgfqpoint{3.564023in}{2.315247in}}%
\pgfpathlineto{\pgfqpoint{3.577463in}{2.309639in}}%
\pgfpathlineto{\pgfqpoint{3.585434in}{2.320131in}}%
\pgfpathlineto{\pgfqpoint{3.593399in}{2.330663in}}%
\pgfpathlineto{\pgfqpoint{3.601359in}{2.341238in}}%
\pgfpathlineto{\pgfqpoint{3.609314in}{2.351854in}}%
\pgfpathlineto{\pgfqpoint{3.595885in}{2.357498in}}%
\pgfpathlineto{\pgfqpoint{3.582458in}{2.363364in}}%
\pgfpathlineto{\pgfqpoint{3.569034in}{2.369456in}}%
\pgfpathlineto{\pgfqpoint{3.555613in}{2.375774in}}%
\pgfpathlineto{\pgfqpoint{3.547648in}{2.365110in}}%
\pgfpathlineto{\pgfqpoint{3.539678in}{2.354497in}}%
\pgfpathlineto{\pgfqpoint{3.531702in}{2.343933in}}%
\pgfpathlineto{\pgfqpoint{3.523720in}{2.333417in}}%
\pgfpathclose%
\pgfusepath{fill}%
\end{pgfscope}%
\begin{pgfscope}%
\pgfpathrectangle{\pgfqpoint{1.150000in}{0.150000in}}{\pgfqpoint{5.700000in}{5.700000in}}%
\pgfusepath{clip}%
\pgfsetbuttcap%
\pgfsetroundjoin%
\definecolor{currentfill}{rgb}{0.260571,0.246922,0.522828}%
\pgfsetfillcolor{currentfill}%
\pgfsetfillopacity{0.800000}%
\pgfsetlinewidth{0.000000pt}%
\definecolor{currentstroke}{rgb}{0.000000,0.000000,0.000000}%
\pgfsetstrokecolor{currentstroke}%
\pgfsetdash{}{0pt}%
\pgfpathmoveto{\pgfqpoint{4.400064in}{2.613662in}}%
\pgfpathlineto{\pgfqpoint{4.413698in}{2.614978in}}%
\pgfpathlineto{\pgfqpoint{4.427342in}{2.616484in}}%
\pgfpathlineto{\pgfqpoint{4.440996in}{2.618179in}}%
\pgfpathlineto{\pgfqpoint{4.454660in}{2.620064in}}%
\pgfpathlineto{\pgfqpoint{4.462353in}{2.629761in}}%
\pgfpathlineto{\pgfqpoint{4.470041in}{2.639495in}}%
\pgfpathlineto{\pgfqpoint{4.477724in}{2.649270in}}%
\pgfpathlineto{\pgfqpoint{4.485403in}{2.659090in}}%
\pgfpathlineto{\pgfqpoint{4.471749in}{2.657527in}}%
\pgfpathlineto{\pgfqpoint{4.458105in}{2.656153in}}%
\pgfpathlineto{\pgfqpoint{4.444471in}{2.654968in}}%
\pgfpathlineto{\pgfqpoint{4.430847in}{2.653972in}}%
\pgfpathlineto{\pgfqpoint{4.423159in}{2.643820in}}%
\pgfpathlineto{\pgfqpoint{4.415465in}{2.633720in}}%
\pgfpathlineto{\pgfqpoint{4.407767in}{2.623669in}}%
\pgfpathlineto{\pgfqpoint{4.400064in}{2.613662in}}%
\pgfpathclose%
\pgfusepath{fill}%
\end{pgfscope}%
\begin{pgfscope}%
\pgfpathrectangle{\pgfqpoint{1.150000in}{0.150000in}}{\pgfqpoint{5.700000in}{5.700000in}}%
\pgfusepath{clip}%
\pgfsetbuttcap%
\pgfsetroundjoin%
\definecolor{currentfill}{rgb}{0.253935,0.265254,0.529983}%
\pgfsetfillcolor{currentfill}%
\pgfsetfillopacity{0.800000}%
\pgfsetlinewidth{0.000000pt}%
\definecolor{currentstroke}{rgb}{0.000000,0.000000,0.000000}%
\pgfsetstrokecolor{currentstroke}%
\pgfsetdash{}{0pt}%
\pgfpathmoveto{\pgfqpoint{4.485403in}{2.659090in}}%
\pgfpathlineto{\pgfqpoint{4.499067in}{2.660841in}}%
\pgfpathlineto{\pgfqpoint{4.512742in}{2.662779in}}%
\pgfpathlineto{\pgfqpoint{4.526428in}{2.664905in}}%
\pgfpathlineto{\pgfqpoint{4.540124in}{2.667219in}}%
\pgfpathlineto{\pgfqpoint{4.547787in}{2.676745in}}%
\pgfpathlineto{\pgfqpoint{4.555446in}{2.686316in}}%
\pgfpathlineto{\pgfqpoint{4.563101in}{2.695936in}}%
\pgfpathlineto{\pgfqpoint{4.570751in}{2.705609in}}%
\pgfpathlineto{\pgfqpoint{4.557065in}{2.703650in}}%
\pgfpathlineto{\pgfqpoint{4.543391in}{2.701877in}}%
\pgfpathlineto{\pgfqpoint{4.529727in}{2.700291in}}%
\pgfpathlineto{\pgfqpoint{4.516073in}{2.698893in}}%
\pgfpathlineto{\pgfqpoint{4.508412in}{2.688855in}}%
\pgfpathlineto{\pgfqpoint{4.500747in}{2.678878in}}%
\pgfpathlineto{\pgfqpoint{4.493077in}{2.668957in}}%
\pgfpathlineto{\pgfqpoint{4.485403in}{2.659090in}}%
\pgfpathclose%
\pgfusepath{fill}%
\end{pgfscope}%
\begin{pgfscope}%
\pgfpathrectangle{\pgfqpoint{1.150000in}{0.150000in}}{\pgfqpoint{5.700000in}{5.700000in}}%
\pgfusepath{clip}%
\pgfsetbuttcap%
\pgfsetroundjoin%
\definecolor{currentfill}{rgb}{0.266580,0.228262,0.514349}%
\pgfsetfillcolor{currentfill}%
\pgfsetfillopacity{0.800000}%
\pgfsetlinewidth{0.000000pt}%
\definecolor{currentstroke}{rgb}{0.000000,0.000000,0.000000}%
\pgfsetstrokecolor{currentstroke}%
\pgfsetdash{}{0pt}%
\pgfpathmoveto{\pgfqpoint{4.314730in}{2.569494in}}%
\pgfpathlineto{\pgfqpoint{4.328335in}{2.570334in}}%
\pgfpathlineto{\pgfqpoint{4.341949in}{2.571367in}}%
\pgfpathlineto{\pgfqpoint{4.355573in}{2.572591in}}%
\pgfpathlineto{\pgfqpoint{4.369206in}{2.574007in}}%
\pgfpathlineto{\pgfqpoint{4.376928in}{2.583872in}}%
\pgfpathlineto{\pgfqpoint{4.384645in}{2.593767in}}%
\pgfpathlineto{\pgfqpoint{4.392357in}{2.603696in}}%
\pgfpathlineto{\pgfqpoint{4.400064in}{2.613662in}}%
\pgfpathlineto{\pgfqpoint{4.386440in}{2.612536in}}%
\pgfpathlineto{\pgfqpoint{4.372826in}{2.611601in}}%
\pgfpathlineto{\pgfqpoint{4.359221in}{2.610858in}}%
\pgfpathlineto{\pgfqpoint{4.345626in}{2.610306in}}%
\pgfpathlineto{\pgfqpoint{4.337909in}{2.600039in}}%
\pgfpathlineto{\pgfqpoint{4.330188in}{2.589817in}}%
\pgfpathlineto{\pgfqpoint{4.322461in}{2.579636in}}%
\pgfpathlineto{\pgfqpoint{4.314730in}{2.569494in}}%
\pgfpathclose%
\pgfusepath{fill}%
\end{pgfscope}%
\begin{pgfscope}%
\pgfpathrectangle{\pgfqpoint{1.150000in}{0.150000in}}{\pgfqpoint{5.700000in}{5.700000in}}%
\pgfusepath{clip}%
\pgfsetbuttcap%
\pgfsetroundjoin%
\definecolor{currentfill}{rgb}{0.282884,0.135920,0.453427}%
\pgfsetfillcolor{currentfill}%
\pgfsetfillopacity{0.800000}%
\pgfsetlinewidth{0.000000pt}%
\definecolor{currentstroke}{rgb}{0.000000,0.000000,0.000000}%
\pgfsetstrokecolor{currentstroke}%
\pgfsetdash{}{0pt}%
\pgfpathmoveto{\pgfqpoint{3.244530in}{2.383738in}}%
\pgfpathlineto{\pgfqpoint{3.257976in}{2.373625in}}%
\pgfpathlineto{\pgfqpoint{3.271422in}{2.363762in}}%
\pgfpathlineto{\pgfqpoint{3.284866in}{2.354146in}}%
\pgfpathlineto{\pgfqpoint{3.298310in}{2.344776in}}%
\pgfpathlineto{\pgfqpoint{3.306375in}{2.354802in}}%
\pgfpathlineto{\pgfqpoint{3.314435in}{2.364897in}}%
\pgfpathlineto{\pgfqpoint{3.322488in}{2.375061in}}%
\pgfpathlineto{\pgfqpoint{3.330535in}{2.385294in}}%
\pgfpathlineto{\pgfqpoint{3.317105in}{2.394636in}}%
\pgfpathlineto{\pgfqpoint{3.303674in}{2.404223in}}%
\pgfpathlineto{\pgfqpoint{3.290243in}{2.414057in}}%
\pgfpathlineto{\pgfqpoint{3.276811in}{2.424141in}}%
\pgfpathlineto{\pgfqpoint{3.268750in}{2.413926in}}%
\pgfpathlineto{\pgfqpoint{3.260683in}{2.403786in}}%
\pgfpathlineto{\pgfqpoint{3.252610in}{2.393724in}}%
\pgfpathlineto{\pgfqpoint{3.244530in}{2.383738in}}%
\pgfpathclose%
\pgfusepath{fill}%
\end{pgfscope}%
\begin{pgfscope}%
\pgfpathrectangle{\pgfqpoint{1.150000in}{0.150000in}}{\pgfqpoint{5.700000in}{5.700000in}}%
\pgfusepath{clip}%
\pgfsetbuttcap%
\pgfsetroundjoin%
\definecolor{currentfill}{rgb}{0.246811,0.283237,0.535941}%
\pgfsetfillcolor{currentfill}%
\pgfsetfillopacity{0.800000}%
\pgfsetlinewidth{0.000000pt}%
\definecolor{currentstroke}{rgb}{0.000000,0.000000,0.000000}%
\pgfsetstrokecolor{currentstroke}%
\pgfsetdash{}{0pt}%
\pgfpathmoveto{\pgfqpoint{4.570751in}{2.705609in}}%
\pgfpathlineto{\pgfqpoint{4.584447in}{2.707755in}}%
\pgfpathlineto{\pgfqpoint{4.598154in}{2.710087in}}%
\pgfpathlineto{\pgfqpoint{4.611872in}{2.712604in}}%
\pgfpathlineto{\pgfqpoint{4.625602in}{2.715306in}}%
\pgfpathlineto{\pgfqpoint{4.633236in}{2.724663in}}%
\pgfpathlineto{\pgfqpoint{4.640865in}{2.734074in}}%
\pgfpathlineto{\pgfqpoint{4.648491in}{2.743544in}}%
\pgfpathlineto{\pgfqpoint{4.656111in}{2.753076in}}%
\pgfpathlineto{\pgfqpoint{4.642394in}{2.750760in}}%
\pgfpathlineto{\pgfqpoint{4.628687in}{2.748628in}}%
\pgfpathlineto{\pgfqpoint{4.614991in}{2.746681in}}%
\pgfpathlineto{\pgfqpoint{4.601307in}{2.744920in}}%
\pgfpathlineto{\pgfqpoint{4.593674in}{2.734991in}}%
\pgfpathlineto{\pgfqpoint{4.586037in}{2.725132in}}%
\pgfpathlineto{\pgfqpoint{4.578396in}{2.715340in}}%
\pgfpathlineto{\pgfqpoint{4.570751in}{2.705609in}}%
\pgfpathclose%
\pgfusepath{fill}%
\end{pgfscope}%
\begin{pgfscope}%
\pgfpathrectangle{\pgfqpoint{1.150000in}{0.150000in}}{\pgfqpoint{5.700000in}{5.700000in}}%
\pgfusepath{clip}%
\pgfsetbuttcap%
\pgfsetroundjoin%
\definecolor{currentfill}{rgb}{0.271828,0.209303,0.504434}%
\pgfsetfillcolor{currentfill}%
\pgfsetfillopacity{0.800000}%
\pgfsetlinewidth{0.000000pt}%
\definecolor{currentstroke}{rgb}{0.000000,0.000000,0.000000}%
\pgfsetstrokecolor{currentstroke}%
\pgfsetdash{}{0pt}%
\pgfpathmoveto{\pgfqpoint{4.229396in}{2.526776in}}%
\pgfpathlineto{\pgfqpoint{4.242973in}{2.527100in}}%
\pgfpathlineto{\pgfqpoint{4.256559in}{2.527618in}}%
\pgfpathlineto{\pgfqpoint{4.270154in}{2.528331in}}%
\pgfpathlineto{\pgfqpoint{4.283758in}{2.529236in}}%
\pgfpathlineto{\pgfqpoint{4.291509in}{2.539260in}}%
\pgfpathlineto{\pgfqpoint{4.299254in}{2.549309in}}%
\pgfpathlineto{\pgfqpoint{4.306995in}{2.559386in}}%
\pgfpathlineto{\pgfqpoint{4.314730in}{2.569494in}}%
\pgfpathlineto{\pgfqpoint{4.301135in}{2.568846in}}%
\pgfpathlineto{\pgfqpoint{4.287549in}{2.568391in}}%
\pgfpathlineto{\pgfqpoint{4.273971in}{2.568129in}}%
\pgfpathlineto{\pgfqpoint{4.260403in}{2.568062in}}%
\pgfpathlineto{\pgfqpoint{4.252658in}{2.557685in}}%
\pgfpathlineto{\pgfqpoint{4.244909in}{2.547347in}}%
\pgfpathlineto{\pgfqpoint{4.237155in}{2.537045in}}%
\pgfpathlineto{\pgfqpoint{4.229396in}{2.526776in}}%
\pgfpathclose%
\pgfusepath{fill}%
\end{pgfscope}%
\begin{pgfscope}%
\pgfpathrectangle{\pgfqpoint{1.150000in}{0.150000in}}{\pgfqpoint{5.700000in}{5.700000in}}%
\pgfusepath{clip}%
\pgfsetbuttcap%
\pgfsetroundjoin%
\definecolor{currentfill}{rgb}{0.151918,0.500685,0.557587}%
\pgfsetfillcolor{currentfill}%
\pgfsetfillopacity{0.800000}%
\pgfsetlinewidth{0.000000pt}%
\definecolor{currentstroke}{rgb}{0.000000,0.000000,0.000000}%
\pgfsetstrokecolor{currentstroke}%
\pgfsetdash{}{0pt}%
\pgfpathmoveto{\pgfqpoint{5.540452in}{3.306095in}}%
\pgfpathlineto{\pgfqpoint{5.554494in}{3.309310in}}%
\pgfpathlineto{\pgfqpoint{5.568552in}{3.312695in}}%
\pgfpathlineto{\pgfqpoint{5.582623in}{3.316250in}}%
\pgfpathlineto{\pgfqpoint{5.596710in}{3.319975in}}%
\pgfpathlineto{\pgfqpoint{5.604048in}{3.330200in}}%
\pgfpathlineto{\pgfqpoint{5.611391in}{3.340706in}}%
\pgfpathlineto{\pgfqpoint{5.618737in}{3.351503in}}%
\pgfpathlineto{\pgfqpoint{5.604672in}{3.348352in}}%
\pgfpathlineto{\pgfqpoint{5.590622in}{3.345371in}}%
\pgfpathlineto{\pgfqpoint{5.576586in}{3.342558in}}%
\pgfpathlineto{\pgfqpoint{5.562564in}{3.339916in}}%
\pgfpathlineto{\pgfqpoint{5.555190in}{3.328347in}}%
\pgfpathlineto{\pgfqpoint{5.547819in}{3.317076in}}%
\pgfpathlineto{\pgfqpoint{5.540452in}{3.306095in}}%
\pgfpathclose%
\pgfusepath{fill}%
\end{pgfscope}%
\begin{pgfscope}%
\pgfpathrectangle{\pgfqpoint{1.150000in}{0.150000in}}{\pgfqpoint{5.700000in}{5.700000in}}%
\pgfusepath{clip}%
\pgfsetbuttcap%
\pgfsetroundjoin%
\definecolor{currentfill}{rgb}{0.237441,0.305202,0.541921}%
\pgfsetfillcolor{currentfill}%
\pgfsetfillopacity{0.800000}%
\pgfsetlinewidth{0.000000pt}%
\definecolor{currentstroke}{rgb}{0.000000,0.000000,0.000000}%
\pgfsetstrokecolor{currentstroke}%
\pgfsetdash{}{0pt}%
\pgfpathmoveto{\pgfqpoint{4.656111in}{2.753076in}}%
\pgfpathlineto{\pgfqpoint{4.669840in}{2.755577in}}%
\pgfpathlineto{\pgfqpoint{4.683581in}{2.758262in}}%
\pgfpathlineto{\pgfqpoint{4.697333in}{2.761131in}}%
\pgfpathlineto{\pgfqpoint{4.711097in}{2.764183in}}%
\pgfpathlineto{\pgfqpoint{4.718701in}{2.773378in}}%
\pgfpathlineto{\pgfqpoint{4.726301in}{2.782638in}}%
\pgfpathlineto{\pgfqpoint{4.733896in}{2.791967in}}%
\pgfpathlineto{\pgfqpoint{4.741488in}{2.801370in}}%
\pgfpathlineto{\pgfqpoint{4.727737in}{2.798736in}}%
\pgfpathlineto{\pgfqpoint{4.713998in}{2.796285in}}%
\pgfpathlineto{\pgfqpoint{4.700270in}{2.794017in}}%
\pgfpathlineto{\pgfqpoint{4.686553in}{2.791933in}}%
\pgfpathlineto{\pgfqpoint{4.678949in}{2.782100in}}%
\pgfpathlineto{\pgfqpoint{4.671340in}{2.772350in}}%
\pgfpathlineto{\pgfqpoint{4.663728in}{2.762677in}}%
\pgfpathlineto{\pgfqpoint{4.656111in}{2.753076in}}%
\pgfpathclose%
\pgfusepath{fill}%
\end{pgfscope}%
\begin{pgfscope}%
\pgfpathrectangle{\pgfqpoint{1.150000in}{0.150000in}}{\pgfqpoint{5.700000in}{5.700000in}}%
\pgfusepath{clip}%
\pgfsetbuttcap%
\pgfsetroundjoin%
\definecolor{currentfill}{rgb}{0.266580,0.228262,0.514349}%
\pgfsetfillcolor{currentfill}%
\pgfsetfillopacity{0.800000}%
\pgfsetlinewidth{0.000000pt}%
\definecolor{currentstroke}{rgb}{0.000000,0.000000,0.000000}%
\pgfsetstrokecolor{currentstroke}%
\pgfsetdash{}{0pt}%
\pgfpathmoveto{\pgfqpoint{2.942313in}{2.603242in}}%
\pgfpathlineto{\pgfqpoint{2.955847in}{2.587680in}}%
\pgfpathlineto{\pgfqpoint{2.969375in}{2.572409in}}%
\pgfpathlineto{\pgfqpoint{2.982898in}{2.557424in}}%
\pgfpathlineto{\pgfqpoint{2.996415in}{2.542725in}}%
\pgfpathlineto{\pgfqpoint{3.004589in}{2.552112in}}%
\pgfpathlineto{\pgfqpoint{3.012756in}{2.561607in}}%
\pgfpathlineto{\pgfqpoint{3.020914in}{2.571210in}}%
\pgfpathlineto{\pgfqpoint{3.029066in}{2.580921in}}%
\pgfpathlineto{\pgfqpoint{3.015567in}{2.595557in}}%
\pgfpathlineto{\pgfqpoint{3.002062in}{2.610478in}}%
\pgfpathlineto{\pgfqpoint{2.988553in}{2.625686in}}%
\pgfpathlineto{\pgfqpoint{2.975037in}{2.641184in}}%
\pgfpathlineto{\pgfqpoint{2.966868in}{2.631524in}}%
\pgfpathlineto{\pgfqpoint{2.958691in}{2.621981in}}%
\pgfpathlineto{\pgfqpoint{2.950506in}{2.612553in}}%
\pgfpathlineto{\pgfqpoint{2.942313in}{2.603242in}}%
\pgfpathclose%
\pgfusepath{fill}%
\end{pgfscope}%
\begin{pgfscope}%
\pgfpathrectangle{\pgfqpoint{1.150000in}{0.150000in}}{\pgfqpoint{5.700000in}{5.700000in}}%
\pgfusepath{clip}%
\pgfsetbuttcap%
\pgfsetroundjoin%
\definecolor{currentfill}{rgb}{0.257322,0.256130,0.526563}%
\pgfsetfillcolor{currentfill}%
\pgfsetfillopacity{0.800000}%
\pgfsetlinewidth{0.000000pt}%
\definecolor{currentstroke}{rgb}{0.000000,0.000000,0.000000}%
\pgfsetstrokecolor{currentstroke}%
\pgfsetdash{}{0pt}%
\pgfpathmoveto{\pgfqpoint{2.888109in}{2.668438in}}%
\pgfpathlineto{\pgfqpoint{2.901670in}{2.651691in}}%
\pgfpathlineto{\pgfqpoint{2.915225in}{2.635245in}}%
\pgfpathlineto{\pgfqpoint{2.928772in}{2.619096in}}%
\pgfpathlineto{\pgfqpoint{2.942313in}{2.603242in}}%
\pgfpathlineto{\pgfqpoint{2.950506in}{2.612553in}}%
\pgfpathlineto{\pgfqpoint{2.958691in}{2.621981in}}%
\pgfpathlineto{\pgfqpoint{2.966868in}{2.631524in}}%
\pgfpathlineto{\pgfqpoint{2.975037in}{2.641184in}}%
\pgfpathlineto{\pgfqpoint{2.961515in}{2.656974in}}%
\pgfpathlineto{\pgfqpoint{2.947987in}{2.673059in}}%
\pgfpathlineto{\pgfqpoint{2.934452in}{2.689441in}}%
\pgfpathlineto{\pgfqpoint{2.920911in}{2.706123in}}%
\pgfpathlineto{\pgfqpoint{2.912723in}{2.696516in}}%
\pgfpathlineto{\pgfqpoint{2.904527in}{2.687032in}}%
\pgfpathlineto{\pgfqpoint{2.896322in}{2.677673in}}%
\pgfpathlineto{\pgfqpoint{2.888109in}{2.668438in}}%
\pgfpathclose%
\pgfusepath{fill}%
\end{pgfscope}%
\begin{pgfscope}%
\pgfpathrectangle{\pgfqpoint{1.150000in}{0.150000in}}{\pgfqpoint{5.700000in}{5.700000in}}%
\pgfusepath{clip}%
\pgfsetbuttcap%
\pgfsetroundjoin%
\definecolor{currentfill}{rgb}{0.229739,0.322361,0.545706}%
\pgfsetfillcolor{currentfill}%
\pgfsetfillopacity{0.800000}%
\pgfsetlinewidth{0.000000pt}%
\definecolor{currentstroke}{rgb}{0.000000,0.000000,0.000000}%
\pgfsetstrokecolor{currentstroke}%
\pgfsetdash{}{0pt}%
\pgfpathmoveto{\pgfqpoint{4.741488in}{2.801370in}}%
\pgfpathlineto{\pgfqpoint{4.755251in}{2.804187in}}%
\pgfpathlineto{\pgfqpoint{4.769025in}{2.807187in}}%
\pgfpathlineto{\pgfqpoint{4.782812in}{2.810368in}}%
\pgfpathlineto{\pgfqpoint{4.796611in}{2.813731in}}%
\pgfpathlineto{\pgfqpoint{4.804185in}{2.822776in}}%
\pgfpathlineto{\pgfqpoint{4.811755in}{2.831898in}}%
\pgfpathlineto{\pgfqpoint{4.819321in}{2.841103in}}%
\pgfpathlineto{\pgfqpoint{4.826883in}{2.850394in}}%
\pgfpathlineto{\pgfqpoint{4.813098in}{2.847481in}}%
\pgfpathlineto{\pgfqpoint{4.799325in}{2.844749in}}%
\pgfpathlineto{\pgfqpoint{4.785564in}{2.842199in}}%
\pgfpathlineto{\pgfqpoint{4.771815in}{2.839831in}}%
\pgfpathlineto{\pgfqpoint{4.764239in}{2.830079in}}%
\pgfpathlineto{\pgfqpoint{4.756659in}{2.820421in}}%
\pgfpathlineto{\pgfqpoint{4.749075in}{2.810854in}}%
\pgfpathlineto{\pgfqpoint{4.741488in}{2.801370in}}%
\pgfpathclose%
\pgfusepath{fill}%
\end{pgfscope}%
\begin{pgfscope}%
\pgfpathrectangle{\pgfqpoint{1.150000in}{0.150000in}}{\pgfqpoint{5.700000in}{5.700000in}}%
\pgfusepath{clip}%
\pgfsetbuttcap%
\pgfsetroundjoin%
\definecolor{currentfill}{rgb}{0.276194,0.190074,0.493001}%
\pgfsetfillcolor{currentfill}%
\pgfsetfillopacity{0.800000}%
\pgfsetlinewidth{0.000000pt}%
\definecolor{currentstroke}{rgb}{0.000000,0.000000,0.000000}%
\pgfsetstrokecolor{currentstroke}%
\pgfsetdash{}{0pt}%
\pgfpathmoveto{\pgfqpoint{4.144053in}{2.485723in}}%
\pgfpathlineto{\pgfqpoint{4.157605in}{2.485489in}}%
\pgfpathlineto{\pgfqpoint{4.171165in}{2.485452in}}%
\pgfpathlineto{\pgfqpoint{4.184734in}{2.485611in}}%
\pgfpathlineto{\pgfqpoint{4.198311in}{2.485966in}}%
\pgfpathlineto{\pgfqpoint{4.206089in}{2.496134in}}%
\pgfpathlineto{\pgfqpoint{4.213863in}{2.506323in}}%
\pgfpathlineto{\pgfqpoint{4.221632in}{2.516536in}}%
\pgfpathlineto{\pgfqpoint{4.229396in}{2.526776in}}%
\pgfpathlineto{\pgfqpoint{4.215827in}{2.526647in}}%
\pgfpathlineto{\pgfqpoint{4.202267in}{2.526713in}}%
\pgfpathlineto{\pgfqpoint{4.188715in}{2.526976in}}%
\pgfpathlineto{\pgfqpoint{4.175172in}{2.527436in}}%
\pgfpathlineto{\pgfqpoint{4.167400in}{2.516959in}}%
\pgfpathlineto{\pgfqpoint{4.159622in}{2.506516in}}%
\pgfpathlineto{\pgfqpoint{4.151840in}{2.496105in}}%
\pgfpathlineto{\pgfqpoint{4.144053in}{2.485723in}}%
\pgfpathclose%
\pgfusepath{fill}%
\end{pgfscope}%
\begin{pgfscope}%
\pgfpathrectangle{\pgfqpoint{1.150000in}{0.150000in}}{\pgfqpoint{5.700000in}{5.700000in}}%
\pgfusepath{clip}%
\pgfsetbuttcap%
\pgfsetroundjoin%
\definecolor{currentfill}{rgb}{0.273006,0.204520,0.501721}%
\pgfsetfillcolor{currentfill}%
\pgfsetfillopacity{0.800000}%
\pgfsetlinewidth{0.000000pt}%
\definecolor{currentstroke}{rgb}{0.000000,0.000000,0.000000}%
\pgfsetstrokecolor{currentstroke}%
\pgfsetdash{}{0pt}%
\pgfpathmoveto{\pgfqpoint{2.996415in}{2.542725in}}%
\pgfpathlineto{\pgfqpoint{3.009927in}{2.528309in}}%
\pgfpathlineto{\pgfqpoint{3.023434in}{2.514174in}}%
\pgfpathlineto{\pgfqpoint{3.036936in}{2.500316in}}%
\pgfpathlineto{\pgfqpoint{3.050434in}{2.486735in}}%
\pgfpathlineto{\pgfqpoint{3.058590in}{2.496196in}}%
\pgfpathlineto{\pgfqpoint{3.066738in}{2.505758in}}%
\pgfpathlineto{\pgfqpoint{3.074879in}{2.515419in}}%
\pgfpathlineto{\pgfqpoint{3.083013in}{2.525181in}}%
\pgfpathlineto{\pgfqpoint{3.069533in}{2.538699in}}%
\pgfpathlineto{\pgfqpoint{3.056048in}{2.552494in}}%
\pgfpathlineto{\pgfqpoint{3.042559in}{2.566567in}}%
\pgfpathlineto{\pgfqpoint{3.029066in}{2.580921in}}%
\pgfpathlineto{\pgfqpoint{3.020914in}{2.571210in}}%
\pgfpathlineto{\pgfqpoint{3.012756in}{2.561607in}}%
\pgfpathlineto{\pgfqpoint{3.004589in}{2.552112in}}%
\pgfpathlineto{\pgfqpoint{2.996415in}{2.542725in}}%
\pgfpathclose%
\pgfusepath{fill}%
\end{pgfscope}%
\begin{pgfscope}%
\pgfpathrectangle{\pgfqpoint{1.150000in}{0.150000in}}{\pgfqpoint{5.700000in}{5.700000in}}%
\pgfusepath{clip}%
\pgfsetbuttcap%
\pgfsetroundjoin%
\definecolor{currentfill}{rgb}{0.246811,0.283237,0.535941}%
\pgfsetfillcolor{currentfill}%
\pgfsetfillopacity{0.800000}%
\pgfsetlinewidth{0.000000pt}%
\definecolor{currentstroke}{rgb}{0.000000,0.000000,0.000000}%
\pgfsetstrokecolor{currentstroke}%
\pgfsetdash{}{0pt}%
\pgfpathmoveto{\pgfqpoint{2.833787in}{2.738477in}}%
\pgfpathlineto{\pgfqpoint{2.847379in}{2.720503in}}%
\pgfpathlineto{\pgfqpoint{2.860964in}{2.702841in}}%
\pgfpathlineto{\pgfqpoint{2.874540in}{2.685487in}}%
\pgfpathlineto{\pgfqpoint{2.888109in}{2.668438in}}%
\pgfpathlineto{\pgfqpoint{2.896322in}{2.677673in}}%
\pgfpathlineto{\pgfqpoint{2.904527in}{2.687032in}}%
\pgfpathlineto{\pgfqpoint{2.912723in}{2.696516in}}%
\pgfpathlineto{\pgfqpoint{2.920911in}{2.706123in}}%
\pgfpathlineto{\pgfqpoint{2.907362in}{2.723107in}}%
\pgfpathlineto{\pgfqpoint{2.893805in}{2.740397in}}%
\pgfpathlineto{\pgfqpoint{2.880241in}{2.757994in}}%
\pgfpathlineto{\pgfqpoint{2.866669in}{2.775902in}}%
\pgfpathlineto{\pgfqpoint{2.858461in}{2.766347in}}%
\pgfpathlineto{\pgfqpoint{2.850245in}{2.756925in}}%
\pgfpathlineto{\pgfqpoint{2.842020in}{2.747635in}}%
\pgfpathlineto{\pgfqpoint{2.833787in}{2.738477in}}%
\pgfpathclose%
\pgfusepath{fill}%
\end{pgfscope}%
\begin{pgfscope}%
\pgfpathrectangle{\pgfqpoint{1.150000in}{0.150000in}}{\pgfqpoint{5.700000in}{5.700000in}}%
\pgfusepath{clip}%
\pgfsetbuttcap%
\pgfsetroundjoin%
\definecolor{currentfill}{rgb}{0.220057,0.343307,0.549413}%
\pgfsetfillcolor{currentfill}%
\pgfsetfillopacity{0.800000}%
\pgfsetlinewidth{0.000000pt}%
\definecolor{currentstroke}{rgb}{0.000000,0.000000,0.000000}%
\pgfsetstrokecolor{currentstroke}%
\pgfsetdash{}{0pt}%
\pgfpathmoveto{\pgfqpoint{4.826883in}{2.850394in}}%
\pgfpathlineto{\pgfqpoint{4.840680in}{2.853488in}}%
\pgfpathlineto{\pgfqpoint{4.854489in}{2.856763in}}%
\pgfpathlineto{\pgfqpoint{4.868311in}{2.860218in}}%
\pgfpathlineto{\pgfqpoint{4.882145in}{2.863854in}}%
\pgfpathlineto{\pgfqpoint{4.889689in}{2.872768in}}%
\pgfpathlineto{\pgfqpoint{4.897229in}{2.881772in}}%
\pgfpathlineto{\pgfqpoint{4.904766in}{2.890872in}}%
\pgfpathlineto{\pgfqpoint{4.912299in}{2.900073in}}%
\pgfpathlineto{\pgfqpoint{4.898480in}{2.896920in}}%
\pgfpathlineto{\pgfqpoint{4.884673in}{2.893946in}}%
\pgfpathlineto{\pgfqpoint{4.870879in}{2.891152in}}%
\pgfpathlineto{\pgfqpoint{4.857097in}{2.888539in}}%
\pgfpathlineto{\pgfqpoint{4.849548in}{2.878845in}}%
\pgfpathlineto{\pgfqpoint{4.841996in}{2.869260in}}%
\pgfpathlineto{\pgfqpoint{4.834441in}{2.859778in}}%
\pgfpathlineto{\pgfqpoint{4.826883in}{2.850394in}}%
\pgfpathclose%
\pgfusepath{fill}%
\end{pgfscope}%
\begin{pgfscope}%
\pgfpathrectangle{\pgfqpoint{1.150000in}{0.150000in}}{\pgfqpoint{5.700000in}{5.700000in}}%
\pgfusepath{clip}%
\pgfsetbuttcap%
\pgfsetroundjoin%
\definecolor{currentfill}{rgb}{0.283229,0.120777,0.440584}%
\pgfsetfillcolor{currentfill}%
\pgfsetfillopacity{0.800000}%
\pgfsetlinewidth{0.000000pt}%
\definecolor{currentstroke}{rgb}{0.000000,0.000000,0.000000}%
\pgfsetstrokecolor{currentstroke}%
\pgfsetdash{}{0pt}%
\pgfpathmoveto{\pgfqpoint{3.663066in}{2.331496in}}%
\pgfpathlineto{\pgfqpoint{3.676513in}{2.326954in}}%
\pgfpathlineto{\pgfqpoint{3.689964in}{2.322630in}}%
\pgfpathlineto{\pgfqpoint{3.703419in}{2.318521in}}%
\pgfpathlineto{\pgfqpoint{3.716879in}{2.314628in}}%
\pgfpathlineto{\pgfqpoint{3.724809in}{2.325178in}}%
\pgfpathlineto{\pgfqpoint{3.732733in}{2.335757in}}%
\pgfpathlineto{\pgfqpoint{3.740653in}{2.346366in}}%
\pgfpathlineto{\pgfqpoint{3.748567in}{2.357007in}}%
\pgfpathlineto{\pgfqpoint{3.735117in}{2.360968in}}%
\pgfpathlineto{\pgfqpoint{3.721671in}{2.365143in}}%
\pgfpathlineto{\pgfqpoint{3.708229in}{2.369535in}}%
\pgfpathlineto{\pgfqpoint{3.694791in}{2.374143in}}%
\pgfpathlineto{\pgfqpoint{3.686868in}{2.363424in}}%
\pgfpathlineto{\pgfqpoint{3.678939in}{2.352744in}}%
\pgfpathlineto{\pgfqpoint{3.671005in}{2.342101in}}%
\pgfpathlineto{\pgfqpoint{3.663066in}{2.331496in}}%
\pgfpathclose%
\pgfusepath{fill}%
\end{pgfscope}%
\begin{pgfscope}%
\pgfpathrectangle{\pgfqpoint{1.150000in}{0.150000in}}{\pgfqpoint{5.700000in}{5.700000in}}%
\pgfusepath{clip}%
\pgfsetbuttcap%
\pgfsetroundjoin%
\definecolor{currentfill}{rgb}{0.278826,0.175490,0.483397}%
\pgfsetfillcolor{currentfill}%
\pgfsetfillopacity{0.800000}%
\pgfsetlinewidth{0.000000pt}%
\definecolor{currentstroke}{rgb}{0.000000,0.000000,0.000000}%
\pgfsetstrokecolor{currentstroke}%
\pgfsetdash{}{0pt}%
\pgfpathmoveto{\pgfqpoint{4.058695in}{2.446574in}}%
\pgfpathlineto{\pgfqpoint{4.072224in}{2.445739in}}%
\pgfpathlineto{\pgfqpoint{4.085760in}{2.445104in}}%
\pgfpathlineto{\pgfqpoint{4.099304in}{2.444667in}}%
\pgfpathlineto{\pgfqpoint{4.112856in}{2.444430in}}%
\pgfpathlineto{\pgfqpoint{4.120663in}{2.454723in}}%
\pgfpathlineto{\pgfqpoint{4.128464in}{2.465035in}}%
\pgfpathlineto{\pgfqpoint{4.136261in}{2.475367in}}%
\pgfpathlineto{\pgfqpoint{4.144053in}{2.485723in}}%
\pgfpathlineto{\pgfqpoint{4.130509in}{2.486155in}}%
\pgfpathlineto{\pgfqpoint{4.116974in}{2.486785in}}%
\pgfpathlineto{\pgfqpoint{4.103445in}{2.487615in}}%
\pgfpathlineto{\pgfqpoint{4.089925in}{2.488643in}}%
\pgfpathlineto{\pgfqpoint{4.082125in}{2.478082in}}%
\pgfpathlineto{\pgfqpoint{4.074320in}{2.467551in}}%
\pgfpathlineto{\pgfqpoint{4.066510in}{2.457050in}}%
\pgfpathlineto{\pgfqpoint{4.058695in}{2.446574in}}%
\pgfpathclose%
\pgfusepath{fill}%
\end{pgfscope}%
\begin{pgfscope}%
\pgfpathrectangle{\pgfqpoint{1.150000in}{0.150000in}}{\pgfqpoint{5.700000in}{5.700000in}}%
\pgfusepath{clip}%
\pgfsetbuttcap%
\pgfsetroundjoin%
\definecolor{currentfill}{rgb}{0.210503,0.363727,0.552206}%
\pgfsetfillcolor{currentfill}%
\pgfsetfillopacity{0.800000}%
\pgfsetlinewidth{0.000000pt}%
\definecolor{currentstroke}{rgb}{0.000000,0.000000,0.000000}%
\pgfsetstrokecolor{currentstroke}%
\pgfsetdash{}{0pt}%
\pgfpathmoveto{\pgfqpoint{4.912299in}{2.900073in}}%
\pgfpathlineto{\pgfqpoint{4.926130in}{2.903406in}}%
\pgfpathlineto{\pgfqpoint{4.939975in}{2.906918in}}%
\pgfpathlineto{\pgfqpoint{4.953832in}{2.910609in}}%
\pgfpathlineto{\pgfqpoint{4.967702in}{2.914479in}}%
\pgfpathlineto{\pgfqpoint{4.975216in}{2.923285in}}%
\pgfpathlineto{\pgfqpoint{4.982726in}{2.932196in}}%
\pgfpathlineto{\pgfqpoint{4.990233in}{2.941218in}}%
\pgfpathlineto{\pgfqpoint{4.997737in}{2.950356in}}%
\pgfpathlineto{\pgfqpoint{4.983884in}{2.947001in}}%
\pgfpathlineto{\pgfqpoint{4.970043in}{2.943824in}}%
\pgfpathlineto{\pgfqpoint{4.956215in}{2.940825in}}%
\pgfpathlineto{\pgfqpoint{4.942400in}{2.938006in}}%
\pgfpathlineto{\pgfqpoint{4.934879in}{2.928342in}}%
\pgfpathlineto{\pgfqpoint{4.927355in}{2.918802in}}%
\pgfpathlineto{\pgfqpoint{4.919828in}{2.909381in}}%
\pgfpathlineto{\pgfqpoint{4.912299in}{2.900073in}}%
\pgfpathclose%
\pgfusepath{fill}%
\end{pgfscope}%
\begin{pgfscope}%
\pgfpathrectangle{\pgfqpoint{1.150000in}{0.150000in}}{\pgfqpoint{5.700000in}{5.700000in}}%
\pgfusepath{clip}%
\pgfsetbuttcap%
\pgfsetroundjoin%
\definecolor{currentfill}{rgb}{0.278012,0.180367,0.486697}%
\pgfsetfillcolor{currentfill}%
\pgfsetfillopacity{0.800000}%
\pgfsetlinewidth{0.000000pt}%
\definecolor{currentstroke}{rgb}{0.000000,0.000000,0.000000}%
\pgfsetstrokecolor{currentstroke}%
\pgfsetdash{}{0pt}%
\pgfpathmoveto{\pgfqpoint{3.050434in}{2.486735in}}%
\pgfpathlineto{\pgfqpoint{3.063927in}{2.473428in}}%
\pgfpathlineto{\pgfqpoint{3.077417in}{2.460393in}}%
\pgfpathlineto{\pgfqpoint{3.090902in}{2.447627in}}%
\pgfpathlineto{\pgfqpoint{3.104384in}{2.435130in}}%
\pgfpathlineto{\pgfqpoint{3.112523in}{2.444665in}}%
\pgfpathlineto{\pgfqpoint{3.120655in}{2.454292in}}%
\pgfpathlineto{\pgfqpoint{3.128779in}{2.464011in}}%
\pgfpathlineto{\pgfqpoint{3.136896in}{2.473823in}}%
\pgfpathlineto{\pgfqpoint{3.123431in}{2.486259in}}%
\pgfpathlineto{\pgfqpoint{3.109962in}{2.498962in}}%
\pgfpathlineto{\pgfqpoint{3.096490in}{2.511936in}}%
\pgfpathlineto{\pgfqpoint{3.083013in}{2.525181in}}%
\pgfpathlineto{\pgfqpoint{3.074879in}{2.515419in}}%
\pgfpathlineto{\pgfqpoint{3.066738in}{2.505758in}}%
\pgfpathlineto{\pgfqpoint{3.058590in}{2.496196in}}%
\pgfpathlineto{\pgfqpoint{3.050434in}{2.486735in}}%
\pgfpathclose%
\pgfusepath{fill}%
\end{pgfscope}%
\begin{pgfscope}%
\pgfpathrectangle{\pgfqpoint{1.150000in}{0.150000in}}{\pgfqpoint{5.700000in}{5.700000in}}%
\pgfusepath{clip}%
\pgfsetbuttcap%
\pgfsetroundjoin%
\definecolor{currentfill}{rgb}{0.203063,0.379716,0.553925}%
\pgfsetfillcolor{currentfill}%
\pgfsetfillopacity{0.800000}%
\pgfsetlinewidth{0.000000pt}%
\definecolor{currentstroke}{rgb}{0.000000,0.000000,0.000000}%
\pgfsetstrokecolor{currentstroke}%
\pgfsetdash{}{0pt}%
\pgfpathmoveto{\pgfqpoint{4.997737in}{2.950356in}}%
\pgfpathlineto{\pgfqpoint{5.011604in}{2.953890in}}%
\pgfpathlineto{\pgfqpoint{5.025483in}{2.957601in}}%
\pgfpathlineto{\pgfqpoint{5.039376in}{2.961490in}}%
\pgfpathlineto{\pgfqpoint{5.053282in}{2.965557in}}%
\pgfpathlineto{\pgfqpoint{5.060766in}{2.974284in}}%
\pgfpathlineto{\pgfqpoint{5.068247in}{2.983132in}}%
\pgfpathlineto{\pgfqpoint{5.075725in}{2.992108in}}%
\pgfpathlineto{\pgfqpoint{5.083201in}{3.001218in}}%
\pgfpathlineto{\pgfqpoint{5.069313in}{2.997698in}}%
\pgfpathlineto{\pgfqpoint{5.055439in}{2.994355in}}%
\pgfpathlineto{\pgfqpoint{5.041577in}{2.991190in}}%
\pgfpathlineto{\pgfqpoint{5.027728in}{2.988202in}}%
\pgfpathlineto{\pgfqpoint{5.020234in}{2.978535in}}%
\pgfpathlineto{\pgfqpoint{5.012738in}{2.969009in}}%
\pgfpathlineto{\pgfqpoint{5.005239in}{2.959618in}}%
\pgfpathlineto{\pgfqpoint{4.997737in}{2.950356in}}%
\pgfpathclose%
\pgfusepath{fill}%
\end{pgfscope}%
\begin{pgfscope}%
\pgfpathrectangle{\pgfqpoint{1.150000in}{0.150000in}}{\pgfqpoint{5.700000in}{5.700000in}}%
\pgfusepath{clip}%
\pgfsetbuttcap%
\pgfsetroundjoin%
\definecolor{currentfill}{rgb}{0.233603,0.313828,0.543914}%
\pgfsetfillcolor{currentfill}%
\pgfsetfillopacity{0.800000}%
\pgfsetlinewidth{0.000000pt}%
\definecolor{currentstroke}{rgb}{0.000000,0.000000,0.000000}%
\pgfsetstrokecolor{currentstroke}%
\pgfsetdash{}{0pt}%
\pgfpathmoveto{\pgfqpoint{2.779326in}{2.813536in}}%
\pgfpathlineto{\pgfqpoint{2.792955in}{2.794290in}}%
\pgfpathlineto{\pgfqpoint{2.806575in}{2.775367in}}%
\pgfpathlineto{\pgfqpoint{2.820185in}{2.756764in}}%
\pgfpathlineto{\pgfqpoint{2.833787in}{2.738477in}}%
\pgfpathlineto{\pgfqpoint{2.842020in}{2.747635in}}%
\pgfpathlineto{\pgfqpoint{2.850245in}{2.756925in}}%
\pgfpathlineto{\pgfqpoint{2.858461in}{2.766347in}}%
\pgfpathlineto{\pgfqpoint{2.866669in}{2.775902in}}%
\pgfpathlineto{\pgfqpoint{2.853089in}{2.794123in}}%
\pgfpathlineto{\pgfqpoint{2.839499in}{2.812661in}}%
\pgfpathlineto{\pgfqpoint{2.825901in}{2.831518in}}%
\pgfpathlineto{\pgfqpoint{2.812294in}{2.850697in}}%
\pgfpathlineto{\pgfqpoint{2.804065in}{2.841196in}}%
\pgfpathlineto{\pgfqpoint{2.795828in}{2.831836in}}%
\pgfpathlineto{\pgfqpoint{2.787582in}{2.822616in}}%
\pgfpathlineto{\pgfqpoint{2.779326in}{2.813536in}}%
\pgfpathclose%
\pgfusepath{fill}%
\end{pgfscope}%
\begin{pgfscope}%
\pgfpathrectangle{\pgfqpoint{1.150000in}{0.150000in}}{\pgfqpoint{5.700000in}{5.700000in}}%
\pgfusepath{clip}%
\pgfsetbuttcap%
\pgfsetroundjoin%
\definecolor{currentfill}{rgb}{0.280868,0.160771,0.472899}%
\pgfsetfillcolor{currentfill}%
\pgfsetfillopacity{0.800000}%
\pgfsetlinewidth{0.000000pt}%
\definecolor{currentstroke}{rgb}{0.000000,0.000000,0.000000}%
\pgfsetstrokecolor{currentstroke}%
\pgfsetdash{}{0pt}%
\pgfpathmoveto{\pgfqpoint{3.973311in}{2.409591in}}%
\pgfpathlineto{\pgfqpoint{3.986819in}{2.408112in}}%
\pgfpathlineto{\pgfqpoint{4.000334in}{2.406836in}}%
\pgfpathlineto{\pgfqpoint{4.013856in}{2.405761in}}%
\pgfpathlineto{\pgfqpoint{4.027386in}{2.404888in}}%
\pgfpathlineto{\pgfqpoint{4.035221in}{2.415282in}}%
\pgfpathlineto{\pgfqpoint{4.043050in}{2.425693in}}%
\pgfpathlineto{\pgfqpoint{4.050875in}{2.436123in}}%
\pgfpathlineto{\pgfqpoint{4.058695in}{2.446574in}}%
\pgfpathlineto{\pgfqpoint{4.045174in}{2.447610in}}%
\pgfpathlineto{\pgfqpoint{4.031660in}{2.448847in}}%
\pgfpathlineto{\pgfqpoint{4.018153in}{2.450285in}}%
\pgfpathlineto{\pgfqpoint{4.004653in}{2.451926in}}%
\pgfpathlineto{\pgfqpoint{3.996825in}{2.441301in}}%
\pgfpathlineto{\pgfqpoint{3.988992in}{2.430705in}}%
\pgfpathlineto{\pgfqpoint{3.981154in}{2.420135in}}%
\pgfpathlineto{\pgfqpoint{3.973311in}{2.409591in}}%
\pgfpathclose%
\pgfusepath{fill}%
\end{pgfscope}%
\begin{pgfscope}%
\pgfpathrectangle{\pgfqpoint{1.150000in}{0.150000in}}{\pgfqpoint{5.700000in}{5.700000in}}%
\pgfusepath{clip}%
\pgfsetbuttcap%
\pgfsetroundjoin%
\definecolor{currentfill}{rgb}{0.283091,0.110553,0.431554}%
\pgfsetfillcolor{currentfill}%
\pgfsetfillopacity{0.800000}%
\pgfsetlinewidth{0.000000pt}%
\definecolor{currentstroke}{rgb}{0.000000,0.000000,0.000000}%
\pgfsetstrokecolor{currentstroke}%
\pgfsetdash{}{0pt}%
\pgfpathmoveto{\pgfqpoint{3.437985in}{2.319232in}}%
\pgfpathlineto{\pgfqpoint{3.451421in}{2.312035in}}%
\pgfpathlineto{\pgfqpoint{3.464858in}{2.305071in}}%
\pgfpathlineto{\pgfqpoint{3.478297in}{2.298336in}}%
\pgfpathlineto{\pgfqpoint{3.491738in}{2.291830in}}%
\pgfpathlineto{\pgfqpoint{3.499742in}{2.302157in}}%
\pgfpathlineto{\pgfqpoint{3.507740in}{2.312530in}}%
\pgfpathlineto{\pgfqpoint{3.515733in}{2.322950in}}%
\pgfpathlineto{\pgfqpoint{3.523720in}{2.333417in}}%
\pgfpathlineto{\pgfqpoint{3.510291in}{2.339927in}}%
\pgfpathlineto{\pgfqpoint{3.496863in}{2.346666in}}%
\pgfpathlineto{\pgfqpoint{3.483438in}{2.353635in}}%
\pgfpathlineto{\pgfqpoint{3.470014in}{2.360834in}}%
\pgfpathlineto{\pgfqpoint{3.462015in}{2.350352in}}%
\pgfpathlineto{\pgfqpoint{3.454011in}{2.339924in}}%
\pgfpathlineto{\pgfqpoint{3.446001in}{2.329551in}}%
\pgfpathlineto{\pgfqpoint{3.437985in}{2.319232in}}%
\pgfpathclose%
\pgfusepath{fill}%
\end{pgfscope}%
\begin{pgfscope}%
\pgfpathrectangle{\pgfqpoint{1.150000in}{0.150000in}}{\pgfqpoint{5.700000in}{5.700000in}}%
\pgfusepath{clip}%
\pgfsetbuttcap%
\pgfsetroundjoin%
\definecolor{currentfill}{rgb}{0.194100,0.399323,0.555565}%
\pgfsetfillcolor{currentfill}%
\pgfsetfillopacity{0.800000}%
\pgfsetlinewidth{0.000000pt}%
\definecolor{currentstroke}{rgb}{0.000000,0.000000,0.000000}%
\pgfsetstrokecolor{currentstroke}%
\pgfsetdash{}{0pt}%
\pgfpathmoveto{\pgfqpoint{5.083201in}{3.001218in}}%
\pgfpathlineto{\pgfqpoint{5.097103in}{3.004914in}}%
\pgfpathlineto{\pgfqpoint{5.111017in}{3.008787in}}%
\pgfpathlineto{\pgfqpoint{5.124946in}{3.012836in}}%
\pgfpathlineto{\pgfqpoint{5.138888in}{3.017061in}}%
\pgfpathlineto{\pgfqpoint{5.146342in}{3.025744in}}%
\pgfpathlineto{\pgfqpoint{5.153795in}{3.034566in}}%
\pgfpathlineto{\pgfqpoint{5.161245in}{3.043533in}}%
\pgfpathlineto{\pgfqpoint{5.168693in}{3.052653in}}%
\pgfpathlineto{\pgfqpoint{5.154771in}{3.049007in}}%
\pgfpathlineto{\pgfqpoint{5.140862in}{3.045536in}}%
\pgfpathlineto{\pgfqpoint{5.126967in}{3.042241in}}%
\pgfpathlineto{\pgfqpoint{5.113084in}{3.039122in}}%
\pgfpathlineto{\pgfqpoint{5.105616in}{3.029413in}}%
\pgfpathlineto{\pgfqpoint{5.098147in}{3.019864in}}%
\pgfpathlineto{\pgfqpoint{5.090675in}{3.010468in}}%
\pgfpathlineto{\pgfqpoint{5.083201in}{3.001218in}}%
\pgfpathclose%
\pgfusepath{fill}%
\end{pgfscope}%
\begin{pgfscope}%
\pgfpathrectangle{\pgfqpoint{1.150000in}{0.150000in}}{\pgfqpoint{5.700000in}{5.700000in}}%
\pgfusepath{clip}%
\pgfsetbuttcap%
\pgfsetroundjoin%
\definecolor{currentfill}{rgb}{0.283229,0.120777,0.440584}%
\pgfsetfillcolor{currentfill}%
\pgfsetfillopacity{0.800000}%
\pgfsetlinewidth{0.000000pt}%
\definecolor{currentstroke}{rgb}{0.000000,0.000000,0.000000}%
\pgfsetstrokecolor{currentstroke}%
\pgfsetdash{}{0pt}%
\pgfpathmoveto{\pgfqpoint{3.298310in}{2.344776in}}%
\pgfpathlineto{\pgfqpoint{3.311753in}{2.335650in}}%
\pgfpathlineto{\pgfqpoint{3.325196in}{2.326768in}}%
\pgfpathlineto{\pgfqpoint{3.338640in}{2.318127in}}%
\pgfpathlineto{\pgfqpoint{3.352083in}{2.309725in}}%
\pgfpathlineto{\pgfqpoint{3.360135in}{2.319792in}}%
\pgfpathlineto{\pgfqpoint{3.368181in}{2.329919in}}%
\pgfpathlineto{\pgfqpoint{3.376222in}{2.340107in}}%
\pgfpathlineto{\pgfqpoint{3.384256in}{2.350357in}}%
\pgfpathlineto{\pgfqpoint{3.370825in}{2.358730in}}%
\pgfpathlineto{\pgfqpoint{3.357395in}{2.367343in}}%
\pgfpathlineto{\pgfqpoint{3.343965in}{2.376197in}}%
\pgfpathlineto{\pgfqpoint{3.330535in}{2.385294in}}%
\pgfpathlineto{\pgfqpoint{3.322488in}{2.375061in}}%
\pgfpathlineto{\pgfqpoint{3.314435in}{2.364897in}}%
\pgfpathlineto{\pgfqpoint{3.306375in}{2.354802in}}%
\pgfpathlineto{\pgfqpoint{3.298310in}{2.344776in}}%
\pgfpathclose%
\pgfusepath{fill}%
\end{pgfscope}%
\begin{pgfscope}%
\pgfpathrectangle{\pgfqpoint{1.150000in}{0.150000in}}{\pgfqpoint{5.700000in}{5.700000in}}%
\pgfusepath{clip}%
\pgfsetbuttcap%
\pgfsetroundjoin%
\definecolor{currentfill}{rgb}{0.282290,0.145912,0.461510}%
\pgfsetfillcolor{currentfill}%
\pgfsetfillopacity{0.800000}%
\pgfsetlinewidth{0.000000pt}%
\definecolor{currentstroke}{rgb}{0.000000,0.000000,0.000000}%
\pgfsetstrokecolor{currentstroke}%
\pgfsetdash{}{0pt}%
\pgfpathmoveto{\pgfqpoint{3.887890in}{2.375061in}}%
\pgfpathlineto{\pgfqpoint{3.901380in}{2.372894in}}%
\pgfpathlineto{\pgfqpoint{3.914877in}{2.370932in}}%
\pgfpathlineto{\pgfqpoint{3.928380in}{2.369176in}}%
\pgfpathlineto{\pgfqpoint{3.941889in}{2.367624in}}%
\pgfpathlineto{\pgfqpoint{3.949752in}{2.378088in}}%
\pgfpathlineto{\pgfqpoint{3.957610in}{2.388570in}}%
\pgfpathlineto{\pgfqpoint{3.965463in}{2.399070in}}%
\pgfpathlineto{\pgfqpoint{3.973311in}{2.409591in}}%
\pgfpathlineto{\pgfqpoint{3.959810in}{2.411274in}}%
\pgfpathlineto{\pgfqpoint{3.946315in}{2.413161in}}%
\pgfpathlineto{\pgfqpoint{3.932827in}{2.415252in}}%
\pgfpathlineto{\pgfqpoint{3.919345in}{2.417550in}}%
\pgfpathlineto{\pgfqpoint{3.911489in}{2.406887in}}%
\pgfpathlineto{\pgfqpoint{3.903627in}{2.396252in}}%
\pgfpathlineto{\pgfqpoint{3.895761in}{2.385644in}}%
\pgfpathlineto{\pgfqpoint{3.887890in}{2.375061in}}%
\pgfpathclose%
\pgfusepath{fill}%
\end{pgfscope}%
\begin{pgfscope}%
\pgfpathrectangle{\pgfqpoint{1.150000in}{0.150000in}}{\pgfqpoint{5.700000in}{5.700000in}}%
\pgfusepath{clip}%
\pgfsetbuttcap%
\pgfsetroundjoin%
\definecolor{currentfill}{rgb}{0.280868,0.160771,0.472899}%
\pgfsetfillcolor{currentfill}%
\pgfsetfillopacity{0.800000}%
\pgfsetlinewidth{0.000000pt}%
\definecolor{currentstroke}{rgb}{0.000000,0.000000,0.000000}%
\pgfsetstrokecolor{currentstroke}%
\pgfsetdash{}{0pt}%
\pgfpathmoveto{\pgfqpoint{3.104384in}{2.435130in}}%
\pgfpathlineto{\pgfqpoint{3.117863in}{2.422898in}}%
\pgfpathlineto{\pgfqpoint{3.131339in}{2.410930in}}%
\pgfpathlineto{\pgfqpoint{3.144812in}{2.399224in}}%
\pgfpathlineto{\pgfqpoint{3.158283in}{2.387778in}}%
\pgfpathlineto{\pgfqpoint{3.166405in}{2.397386in}}%
\pgfpathlineto{\pgfqpoint{3.174520in}{2.407078in}}%
\pgfpathlineto{\pgfqpoint{3.182629in}{2.416855in}}%
\pgfpathlineto{\pgfqpoint{3.190730in}{2.426716in}}%
\pgfpathlineto{\pgfqpoint{3.177276in}{2.438101in}}%
\pgfpathlineto{\pgfqpoint{3.163819in}{2.449746in}}%
\pgfpathlineto{\pgfqpoint{3.150359in}{2.461652in}}%
\pgfpathlineto{\pgfqpoint{3.136896in}{2.473823in}}%
\pgfpathlineto{\pgfqpoint{3.128779in}{2.464011in}}%
\pgfpathlineto{\pgfqpoint{3.120655in}{2.454292in}}%
\pgfpathlineto{\pgfqpoint{3.112523in}{2.444665in}}%
\pgfpathlineto{\pgfqpoint{3.104384in}{2.435130in}}%
\pgfpathclose%
\pgfusepath{fill}%
\end{pgfscope}%
\begin{pgfscope}%
\pgfpathrectangle{\pgfqpoint{1.150000in}{0.150000in}}{\pgfqpoint{5.700000in}{5.700000in}}%
\pgfusepath{clip}%
\pgfsetbuttcap%
\pgfsetroundjoin%
\definecolor{currentfill}{rgb}{0.185556,0.418570,0.556753}%
\pgfsetfillcolor{currentfill}%
\pgfsetfillopacity{0.800000}%
\pgfsetlinewidth{0.000000pt}%
\definecolor{currentstroke}{rgb}{0.000000,0.000000,0.000000}%
\pgfsetstrokecolor{currentstroke}%
\pgfsetdash{}{0pt}%
\pgfpathmoveto{\pgfqpoint{5.168693in}{3.052653in}}%
\pgfpathlineto{\pgfqpoint{5.182629in}{3.056474in}}%
\pgfpathlineto{\pgfqpoint{5.196579in}{3.060471in}}%
\pgfpathlineto{\pgfqpoint{5.210543in}{3.064643in}}%
\pgfpathlineto{\pgfqpoint{5.224521in}{3.068990in}}%
\pgfpathlineto{\pgfqpoint{5.231947in}{3.077669in}}%
\pgfpathlineto{\pgfqpoint{5.239371in}{3.086506in}}%
\pgfpathlineto{\pgfqpoint{5.246794in}{3.095508in}}%
\pgfpathlineto{\pgfqpoint{5.254216in}{3.104682in}}%
\pgfpathlineto{\pgfqpoint{5.240260in}{3.100947in}}%
\pgfpathlineto{\pgfqpoint{5.226317in}{3.097386in}}%
\pgfpathlineto{\pgfqpoint{5.212388in}{3.093999in}}%
\pgfpathlineto{\pgfqpoint{5.198473in}{3.090787in}}%
\pgfpathlineto{\pgfqpoint{5.191030in}{3.080991in}}%
\pgfpathlineto{\pgfqpoint{5.183586in}{3.071375in}}%
\pgfpathlineto{\pgfqpoint{5.176140in}{3.061931in}}%
\pgfpathlineto{\pgfqpoint{5.168693in}{3.052653in}}%
\pgfpathclose%
\pgfusepath{fill}%
\end{pgfscope}%
\begin{pgfscope}%
\pgfpathrectangle{\pgfqpoint{1.150000in}{0.150000in}}{\pgfqpoint{5.700000in}{5.700000in}}%
\pgfusepath{clip}%
\pgfsetbuttcap%
\pgfsetroundjoin%
\definecolor{currentfill}{rgb}{0.283091,0.110553,0.431554}%
\pgfsetfillcolor{currentfill}%
\pgfsetfillopacity{0.800000}%
\pgfsetlinewidth{0.000000pt}%
\definecolor{currentstroke}{rgb}{0.000000,0.000000,0.000000}%
\pgfsetstrokecolor{currentstroke}%
\pgfsetdash{}{0pt}%
\pgfpathmoveto{\pgfqpoint{3.577463in}{2.309639in}}%
\pgfpathlineto{\pgfqpoint{3.590906in}{2.304254in}}%
\pgfpathlineto{\pgfqpoint{3.604352in}{2.299090in}}%
\pgfpathlineto{\pgfqpoint{3.617802in}{2.294147in}}%
\pgfpathlineto{\pgfqpoint{3.631255in}{2.289423in}}%
\pgfpathlineto{\pgfqpoint{3.639216in}{2.299891in}}%
\pgfpathlineto{\pgfqpoint{3.647171in}{2.310392in}}%
\pgfpathlineto{\pgfqpoint{3.655121in}{2.320926in}}%
\pgfpathlineto{\pgfqpoint{3.663066in}{2.331496in}}%
\pgfpathlineto{\pgfqpoint{3.649622in}{2.336256in}}%
\pgfpathlineto{\pgfqpoint{3.636183in}{2.341235in}}%
\pgfpathlineto{\pgfqpoint{3.622747in}{2.346434in}}%
\pgfpathlineto{\pgfqpoint{3.609314in}{2.351854in}}%
\pgfpathlineto{\pgfqpoint{3.601359in}{2.341238in}}%
\pgfpathlineto{\pgfqpoint{3.593399in}{2.330663in}}%
\pgfpathlineto{\pgfqpoint{3.585434in}{2.320131in}}%
\pgfpathlineto{\pgfqpoint{3.577463in}{2.309639in}}%
\pgfpathclose%
\pgfusepath{fill}%
\end{pgfscope}%
\begin{pgfscope}%
\pgfpathrectangle{\pgfqpoint{1.150000in}{0.150000in}}{\pgfqpoint{5.700000in}{5.700000in}}%
\pgfusepath{clip}%
\pgfsetbuttcap%
\pgfsetroundjoin%
\definecolor{currentfill}{rgb}{0.220057,0.343307,0.549413}%
\pgfsetfillcolor{currentfill}%
\pgfsetfillopacity{0.800000}%
\pgfsetlinewidth{0.000000pt}%
\definecolor{currentstroke}{rgb}{0.000000,0.000000,0.000000}%
\pgfsetstrokecolor{currentstroke}%
\pgfsetdash{}{0pt}%
\pgfpathmoveto{\pgfqpoint{2.724709in}{2.893805in}}%
\pgfpathlineto{\pgfqpoint{2.738379in}{2.873238in}}%
\pgfpathlineto{\pgfqpoint{2.752039in}{2.853007in}}%
\pgfpathlineto{\pgfqpoint{2.765688in}{2.833107in}}%
\pgfpathlineto{\pgfqpoint{2.779326in}{2.813536in}}%
\pgfpathlineto{\pgfqpoint{2.787582in}{2.822616in}}%
\pgfpathlineto{\pgfqpoint{2.795828in}{2.831836in}}%
\pgfpathlineto{\pgfqpoint{2.804065in}{2.841196in}}%
\pgfpathlineto{\pgfqpoint{2.812294in}{2.850697in}}%
\pgfpathlineto{\pgfqpoint{2.798676in}{2.870201in}}%
\pgfpathlineto{\pgfqpoint{2.785049in}{2.890034in}}%
\pgfpathlineto{\pgfqpoint{2.771412in}{2.910199in}}%
\pgfpathlineto{\pgfqpoint{2.757765in}{2.930698in}}%
\pgfpathlineto{\pgfqpoint{2.749515in}{2.921252in}}%
\pgfpathlineto{\pgfqpoint{2.741256in}{2.911955in}}%
\pgfpathlineto{\pgfqpoint{2.732987in}{2.902806in}}%
\pgfpathlineto{\pgfqpoint{2.724709in}{2.893805in}}%
\pgfpathclose%
\pgfusepath{fill}%
\end{pgfscope}%
\begin{pgfscope}%
\pgfpathrectangle{\pgfqpoint{1.150000in}{0.150000in}}{\pgfqpoint{5.700000in}{5.700000in}}%
\pgfusepath{clip}%
\pgfsetbuttcap%
\pgfsetroundjoin%
\definecolor{currentfill}{rgb}{0.177423,0.437527,0.557565}%
\pgfsetfillcolor{currentfill}%
\pgfsetfillopacity{0.800000}%
\pgfsetlinewidth{0.000000pt}%
\definecolor{currentstroke}{rgb}{0.000000,0.000000,0.000000}%
\pgfsetstrokecolor{currentstroke}%
\pgfsetdash{}{0pt}%
\pgfpathmoveto{\pgfqpoint{5.254216in}{3.104682in}}%
\pgfpathlineto{\pgfqpoint{5.268187in}{3.108592in}}%
\pgfpathlineto{\pgfqpoint{5.282171in}{3.112675in}}%
\pgfpathlineto{\pgfqpoint{5.296170in}{3.116933in}}%
\pgfpathlineto{\pgfqpoint{5.310183in}{3.121364in}}%
\pgfpathlineto{\pgfqpoint{5.317582in}{3.130085in}}%
\pgfpathlineto{\pgfqpoint{5.324980in}{3.138985in}}%
\pgfpathlineto{\pgfqpoint{5.332377in}{3.148071in}}%
\pgfpathlineto{\pgfqpoint{5.339774in}{3.157349in}}%
\pgfpathlineto{\pgfqpoint{5.325784in}{3.153561in}}%
\pgfpathlineto{\pgfqpoint{5.311808in}{3.149947in}}%
\pgfpathlineto{\pgfqpoint{5.297846in}{3.146505in}}%
\pgfpathlineto{\pgfqpoint{5.283898in}{3.143237in}}%
\pgfpathlineto{\pgfqpoint{5.276478in}{3.133305in}}%
\pgfpathlineto{\pgfqpoint{5.269058in}{3.123574in}}%
\pgfpathlineto{\pgfqpoint{5.261638in}{3.114035in}}%
\pgfpathlineto{\pgfqpoint{5.254216in}{3.104682in}}%
\pgfpathclose%
\pgfusepath{fill}%
\end{pgfscope}%
\begin{pgfscope}%
\pgfpathrectangle{\pgfqpoint{1.150000in}{0.150000in}}{\pgfqpoint{5.700000in}{5.700000in}}%
\pgfusepath{clip}%
\pgfsetbuttcap%
\pgfsetroundjoin%
\definecolor{currentfill}{rgb}{0.283072,0.130895,0.449241}%
\pgfsetfillcolor{currentfill}%
\pgfsetfillopacity{0.800000}%
\pgfsetlinewidth{0.000000pt}%
\definecolor{currentstroke}{rgb}{0.000000,0.000000,0.000000}%
\pgfsetstrokecolor{currentstroke}%
\pgfsetdash{}{0pt}%
\pgfpathmoveto{\pgfqpoint{3.802418in}{2.343294in}}%
\pgfpathlineto{\pgfqpoint{3.815893in}{2.340394in}}%
\pgfpathlineto{\pgfqpoint{3.829375in}{2.337703in}}%
\pgfpathlineto{\pgfqpoint{3.842861in}{2.335221in}}%
\pgfpathlineto{\pgfqpoint{3.856354in}{2.332946in}}%
\pgfpathlineto{\pgfqpoint{3.864246in}{2.343445in}}%
\pgfpathlineto{\pgfqpoint{3.872132in}{2.353963in}}%
\pgfpathlineto{\pgfqpoint{3.880014in}{2.364501in}}%
\pgfpathlineto{\pgfqpoint{3.887890in}{2.375061in}}%
\pgfpathlineto{\pgfqpoint{3.874406in}{2.377435in}}%
\pgfpathlineto{\pgfqpoint{3.860927in}{2.380016in}}%
\pgfpathlineto{\pgfqpoint{3.847455in}{2.382806in}}%
\pgfpathlineto{\pgfqpoint{3.833988in}{2.385805in}}%
\pgfpathlineto{\pgfqpoint{3.826103in}{2.375135in}}%
\pgfpathlineto{\pgfqpoint{3.818213in}{2.364494in}}%
\pgfpathlineto{\pgfqpoint{3.810318in}{2.353881in}}%
\pgfpathlineto{\pgfqpoint{3.802418in}{2.343294in}}%
\pgfpathclose%
\pgfusepath{fill}%
\end{pgfscope}%
\begin{pgfscope}%
\pgfpathrectangle{\pgfqpoint{1.150000in}{0.150000in}}{\pgfqpoint{5.700000in}{5.700000in}}%
\pgfusepath{clip}%
\pgfsetbuttcap%
\pgfsetroundjoin%
\definecolor{currentfill}{rgb}{0.169646,0.456262,0.558030}%
\pgfsetfillcolor{currentfill}%
\pgfsetfillopacity{0.800000}%
\pgfsetlinewidth{0.000000pt}%
\definecolor{currentstroke}{rgb}{0.000000,0.000000,0.000000}%
\pgfsetstrokecolor{currentstroke}%
\pgfsetdash{}{0pt}%
\pgfpathmoveto{\pgfqpoint{5.339774in}{3.157349in}}%
\pgfpathlineto{\pgfqpoint{5.353779in}{3.161310in}}%
\pgfpathlineto{\pgfqpoint{5.367797in}{3.165443in}}%
\pgfpathlineto{\pgfqpoint{5.381830in}{3.169750in}}%
\pgfpathlineto{\pgfqpoint{5.395878in}{3.174229in}}%
\pgfpathlineto{\pgfqpoint{5.403251in}{3.183044in}}%
\pgfpathlineto{\pgfqpoint{5.410624in}{3.192059in}}%
\pgfpathlineto{\pgfqpoint{5.417998in}{3.201283in}}%
\pgfpathlineto{\pgfqpoint{5.425372in}{3.210722in}}%
\pgfpathlineto{\pgfqpoint{5.411349in}{3.206919in}}%
\pgfpathlineto{\pgfqpoint{5.397340in}{3.203287in}}%
\pgfpathlineto{\pgfqpoint{5.383346in}{3.199828in}}%
\pgfpathlineto{\pgfqpoint{5.369366in}{3.196541in}}%
\pgfpathlineto{\pgfqpoint{5.361967in}{3.186416in}}%
\pgfpathlineto{\pgfqpoint{5.354569in}{3.176515in}}%
\pgfpathlineto{\pgfqpoint{5.347172in}{3.166828in}}%
\pgfpathlineto{\pgfqpoint{5.339774in}{3.157349in}}%
\pgfpathclose%
\pgfusepath{fill}%
\end{pgfscope}%
\begin{pgfscope}%
\pgfpathrectangle{\pgfqpoint{1.150000in}{0.150000in}}{\pgfqpoint{5.700000in}{5.700000in}}%
\pgfusepath{clip}%
\pgfsetbuttcap%
\pgfsetroundjoin%
\definecolor{currentfill}{rgb}{0.282290,0.145912,0.461510}%
\pgfsetfillcolor{currentfill}%
\pgfsetfillopacity{0.800000}%
\pgfsetlinewidth{0.000000pt}%
\definecolor{currentstroke}{rgb}{0.000000,0.000000,0.000000}%
\pgfsetstrokecolor{currentstroke}%
\pgfsetdash{}{0pt}%
\pgfpathmoveto{\pgfqpoint{3.158283in}{2.387778in}}%
\pgfpathlineto{\pgfqpoint{3.171751in}{2.376590in}}%
\pgfpathlineto{\pgfqpoint{3.185217in}{2.365659in}}%
\pgfpathlineto{\pgfqpoint{3.198682in}{2.354982in}}%
\pgfpathlineto{\pgfqpoint{3.212144in}{2.344558in}}%
\pgfpathlineto{\pgfqpoint{3.220251in}{2.354238in}}%
\pgfpathlineto{\pgfqpoint{3.228350in}{2.363995in}}%
\pgfpathlineto{\pgfqpoint{3.236444in}{2.373828in}}%
\pgfpathlineto{\pgfqpoint{3.244530in}{2.383738in}}%
\pgfpathlineto{\pgfqpoint{3.231083in}{2.394102in}}%
\pgfpathlineto{\pgfqpoint{3.217634in}{2.404718in}}%
\pgfpathlineto{\pgfqpoint{3.204183in}{2.415589in}}%
\pgfpathlineto{\pgfqpoint{3.190730in}{2.426716in}}%
\pgfpathlineto{\pgfqpoint{3.182629in}{2.416855in}}%
\pgfpathlineto{\pgfqpoint{3.174520in}{2.407078in}}%
\pgfpathlineto{\pgfqpoint{3.166405in}{2.397386in}}%
\pgfpathlineto{\pgfqpoint{3.158283in}{2.387778in}}%
\pgfpathclose%
\pgfusepath{fill}%
\end{pgfscope}%
\begin{pgfscope}%
\pgfpathrectangle{\pgfqpoint{1.150000in}{0.150000in}}{\pgfqpoint{5.700000in}{5.700000in}}%
\pgfusepath{clip}%
\pgfsetbuttcap%
\pgfsetroundjoin%
\definecolor{currentfill}{rgb}{0.162142,0.474838,0.558140}%
\pgfsetfillcolor{currentfill}%
\pgfsetfillopacity{0.800000}%
\pgfsetlinewidth{0.000000pt}%
\definecolor{currentstroke}{rgb}{0.000000,0.000000,0.000000}%
\pgfsetstrokecolor{currentstroke}%
\pgfsetdash{}{0pt}%
\pgfpathmoveto{\pgfqpoint{5.425372in}{3.210722in}}%
\pgfpathlineto{\pgfqpoint{5.439409in}{3.214697in}}%
\pgfpathlineto{\pgfqpoint{5.453462in}{3.218844in}}%
\pgfpathlineto{\pgfqpoint{5.467528in}{3.223162in}}%
\pgfpathlineto{\pgfqpoint{5.481610in}{3.227652in}}%
\pgfpathlineto{\pgfqpoint{5.488959in}{3.236619in}}%
\pgfpathlineto{\pgfqpoint{5.496310in}{3.245809in}}%
\pgfpathlineto{\pgfqpoint{5.503661in}{3.255231in}}%
\pgfpathlineto{\pgfqpoint{5.511015in}{3.264892in}}%
\pgfpathlineto{\pgfqpoint{5.496959in}{3.261110in}}%
\pgfpathlineto{\pgfqpoint{5.482919in}{3.257498in}}%
\pgfpathlineto{\pgfqpoint{5.468893in}{3.254058in}}%
\pgfpathlineto{\pgfqpoint{5.454881in}{3.250789in}}%
\pgfpathlineto{\pgfqpoint{5.447501in}{3.240410in}}%
\pgfpathlineto{\pgfqpoint{5.440123in}{3.230277in}}%
\pgfpathlineto{\pgfqpoint{5.432747in}{3.220384in}}%
\pgfpathlineto{\pgfqpoint{5.425372in}{3.210722in}}%
\pgfpathclose%
\pgfusepath{fill}%
\end{pgfscope}%
\begin{pgfscope}%
\pgfpathrectangle{\pgfqpoint{1.150000in}{0.150000in}}{\pgfqpoint{5.700000in}{5.700000in}}%
\pgfusepath{clip}%
\pgfsetbuttcap%
\pgfsetroundjoin%
\definecolor{currentfill}{rgb}{0.204903,0.375746,0.553533}%
\pgfsetfillcolor{currentfill}%
\pgfsetfillopacity{0.800000}%
\pgfsetlinewidth{0.000000pt}%
\definecolor{currentstroke}{rgb}{0.000000,0.000000,0.000000}%
\pgfsetstrokecolor{currentstroke}%
\pgfsetdash{}{0pt}%
\pgfpathmoveto{\pgfqpoint{2.669914in}{2.979489in}}%
\pgfpathlineto{\pgfqpoint{2.683630in}{2.957549in}}%
\pgfpathlineto{\pgfqpoint{2.697335in}{2.935957in}}%
\pgfpathlineto{\pgfqpoint{2.711028in}{2.914710in}}%
\pgfpathlineto{\pgfqpoint{2.724709in}{2.893805in}}%
\pgfpathlineto{\pgfqpoint{2.732987in}{2.902806in}}%
\pgfpathlineto{\pgfqpoint{2.741256in}{2.911955in}}%
\pgfpathlineto{\pgfqpoint{2.749515in}{2.921252in}}%
\pgfpathlineto{\pgfqpoint{2.757765in}{2.930698in}}%
\pgfpathlineto{\pgfqpoint{2.744106in}{2.951536in}}%
\pgfpathlineto{\pgfqpoint{2.730436in}{2.972715in}}%
\pgfpathlineto{\pgfqpoint{2.716755in}{2.994238in}}%
\pgfpathlineto{\pgfqpoint{2.703061in}{3.016110in}}%
\pgfpathlineto{\pgfqpoint{2.694789in}{3.006720in}}%
\pgfpathlineto{\pgfqpoint{2.686507in}{2.997487in}}%
\pgfpathlineto{\pgfqpoint{2.678215in}{2.988410in}}%
\pgfpathlineto{\pgfqpoint{2.669914in}{2.979489in}}%
\pgfpathclose%
\pgfusepath{fill}%
\end{pgfscope}%
\begin{pgfscope}%
\pgfpathrectangle{\pgfqpoint{1.150000in}{0.150000in}}{\pgfqpoint{5.700000in}{5.700000in}}%
\pgfusepath{clip}%
\pgfsetbuttcap%
\pgfsetroundjoin%
\definecolor{currentfill}{rgb}{0.283091,0.110553,0.431554}%
\pgfsetfillcolor{currentfill}%
\pgfsetfillopacity{0.800000}%
\pgfsetlinewidth{0.000000pt}%
\definecolor{currentstroke}{rgb}{0.000000,0.000000,0.000000}%
\pgfsetstrokecolor{currentstroke}%
\pgfsetdash{}{0pt}%
\pgfpathmoveto{\pgfqpoint{3.352083in}{2.309725in}}%
\pgfpathlineto{\pgfqpoint{3.365527in}{2.301562in}}%
\pgfpathlineto{\pgfqpoint{3.378972in}{2.293636in}}%
\pgfpathlineto{\pgfqpoint{3.392417in}{2.285946in}}%
\pgfpathlineto{\pgfqpoint{3.405864in}{2.278489in}}%
\pgfpathlineto{\pgfqpoint{3.413903in}{2.288595in}}%
\pgfpathlineto{\pgfqpoint{3.421936in}{2.298754in}}%
\pgfpathlineto{\pgfqpoint{3.429964in}{2.308966in}}%
\pgfpathlineto{\pgfqpoint{3.437985in}{2.319232in}}%
\pgfpathlineto{\pgfqpoint{3.424551in}{2.326660in}}%
\pgfpathlineto{\pgfqpoint{3.411119in}{2.334323in}}%
\pgfpathlineto{\pgfqpoint{3.397687in}{2.342221in}}%
\pgfpathlineto{\pgfqpoint{3.384256in}{2.350357in}}%
\pgfpathlineto{\pgfqpoint{3.376222in}{2.340107in}}%
\pgfpathlineto{\pgfqpoint{3.368181in}{2.329919in}}%
\pgfpathlineto{\pgfqpoint{3.360135in}{2.319792in}}%
\pgfpathlineto{\pgfqpoint{3.352083in}{2.309725in}}%
\pgfpathclose%
\pgfusepath{fill}%
\end{pgfscope}%
\begin{pgfscope}%
\pgfpathrectangle{\pgfqpoint{1.150000in}{0.150000in}}{\pgfqpoint{5.700000in}{5.700000in}}%
\pgfusepath{clip}%
\pgfsetbuttcap%
\pgfsetroundjoin%
\definecolor{currentfill}{rgb}{0.283229,0.120777,0.440584}%
\pgfsetfillcolor{currentfill}%
\pgfsetfillopacity{0.800000}%
\pgfsetlinewidth{0.000000pt}%
\definecolor{currentstroke}{rgb}{0.000000,0.000000,0.000000}%
\pgfsetstrokecolor{currentstroke}%
\pgfsetdash{}{0pt}%
\pgfpathmoveto{\pgfqpoint{3.716879in}{2.314628in}}%
\pgfpathlineto{\pgfqpoint{3.730343in}{2.310948in}}%
\pgfpathlineto{\pgfqpoint{3.743812in}{2.307482in}}%
\pgfpathlineto{\pgfqpoint{3.757287in}{2.304228in}}%
\pgfpathlineto{\pgfqpoint{3.770766in}{2.301185in}}%
\pgfpathlineto{\pgfqpoint{3.778687in}{2.311679in}}%
\pgfpathlineto{\pgfqpoint{3.786602in}{2.322195in}}%
\pgfpathlineto{\pgfqpoint{3.794513in}{2.332733in}}%
\pgfpathlineto{\pgfqpoint{3.802418in}{2.343294in}}%
\pgfpathlineto{\pgfqpoint{3.788948in}{2.346405in}}%
\pgfpathlineto{\pgfqpoint{3.775482in}{2.349726in}}%
\pgfpathlineto{\pgfqpoint{3.762022in}{2.353260in}}%
\pgfpathlineto{\pgfqpoint{3.748567in}{2.357007in}}%
\pgfpathlineto{\pgfqpoint{3.740653in}{2.346366in}}%
\pgfpathlineto{\pgfqpoint{3.732733in}{2.335757in}}%
\pgfpathlineto{\pgfqpoint{3.724809in}{2.325178in}}%
\pgfpathlineto{\pgfqpoint{3.716879in}{2.314628in}}%
\pgfpathclose%
\pgfusepath{fill}%
\end{pgfscope}%
\begin{pgfscope}%
\pgfpathrectangle{\pgfqpoint{1.150000in}{0.150000in}}{\pgfqpoint{5.700000in}{5.700000in}}%
\pgfusepath{clip}%
\pgfsetbuttcap%
\pgfsetroundjoin%
\definecolor{currentfill}{rgb}{0.282910,0.105393,0.426902}%
\pgfsetfillcolor{currentfill}%
\pgfsetfillopacity{0.800000}%
\pgfsetlinewidth{0.000000pt}%
\definecolor{currentstroke}{rgb}{0.000000,0.000000,0.000000}%
\pgfsetstrokecolor{currentstroke}%
\pgfsetdash{}{0pt}%
\pgfpathmoveto{\pgfqpoint{3.491738in}{2.291830in}}%
\pgfpathlineto{\pgfqpoint{3.505181in}{2.285551in}}%
\pgfpathlineto{\pgfqpoint{3.518626in}{2.279499in}}%
\pgfpathlineto{\pgfqpoint{3.532074in}{2.273672in}}%
\pgfpathlineto{\pgfqpoint{3.545525in}{2.268068in}}%
\pgfpathlineto{\pgfqpoint{3.553517in}{2.278403in}}%
\pgfpathlineto{\pgfqpoint{3.561505in}{2.288776in}}%
\pgfpathlineto{\pgfqpoint{3.569487in}{2.299188in}}%
\pgfpathlineto{\pgfqpoint{3.577463in}{2.309639in}}%
\pgfpathlineto{\pgfqpoint{3.564023in}{2.315247in}}%
\pgfpathlineto{\pgfqpoint{3.550586in}{2.321078in}}%
\pgfpathlineto{\pgfqpoint{3.537152in}{2.327135in}}%
\pgfpathlineto{\pgfqpoint{3.523720in}{2.333417in}}%
\pgfpathlineto{\pgfqpoint{3.515733in}{2.322950in}}%
\pgfpathlineto{\pgfqpoint{3.507740in}{2.312530in}}%
\pgfpathlineto{\pgfqpoint{3.499742in}{2.302157in}}%
\pgfpathlineto{\pgfqpoint{3.491738in}{2.291830in}}%
\pgfpathclose%
\pgfusepath{fill}%
\end{pgfscope}%
\begin{pgfscope}%
\pgfpathrectangle{\pgfqpoint{1.150000in}{0.150000in}}{\pgfqpoint{5.700000in}{5.700000in}}%
\pgfusepath{clip}%
\pgfsetbuttcap%
\pgfsetroundjoin%
\definecolor{currentfill}{rgb}{0.154815,0.493313,0.557840}%
\pgfsetfillcolor{currentfill}%
\pgfsetfillopacity{0.800000}%
\pgfsetlinewidth{0.000000pt}%
\definecolor{currentstroke}{rgb}{0.000000,0.000000,0.000000}%
\pgfsetstrokecolor{currentstroke}%
\pgfsetdash{}{0pt}%
\pgfpathmoveto{\pgfqpoint{5.511015in}{3.264892in}}%
\pgfpathlineto{\pgfqpoint{5.525085in}{3.268845in}}%
\pgfpathlineto{\pgfqpoint{5.539169in}{3.272968in}}%
\pgfpathlineto{\pgfqpoint{5.553269in}{3.277263in}}%
\pgfpathlineto{\pgfqpoint{5.567384in}{3.281727in}}%
\pgfpathlineto{\pgfqpoint{5.574712in}{3.290908in}}%
\pgfpathlineto{\pgfqpoint{5.582042in}{3.300338in}}%
\pgfpathlineto{\pgfqpoint{5.589374in}{3.310024in}}%
\pgfpathlineto{\pgfqpoint{5.596710in}{3.319975in}}%
\pgfpathlineto{\pgfqpoint{5.582623in}{3.316250in}}%
\pgfpathlineto{\pgfqpoint{5.568552in}{3.312695in}}%
\pgfpathlineto{\pgfqpoint{5.554494in}{3.309310in}}%
\pgfpathlineto{\pgfqpoint{5.540452in}{3.306095in}}%
\pgfpathlineto{\pgfqpoint{5.533088in}{3.295394in}}%
\pgfpathlineto{\pgfqpoint{5.525728in}{3.284965in}}%
\pgfpathlineto{\pgfqpoint{5.518370in}{3.274801in}}%
\pgfpathlineto{\pgfqpoint{5.511015in}{3.264892in}}%
\pgfpathclose%
\pgfusepath{fill}%
\end{pgfscope}%
\begin{pgfscope}%
\pgfpathrectangle{\pgfqpoint{1.150000in}{0.150000in}}{\pgfqpoint{5.700000in}{5.700000in}}%
\pgfusepath{clip}%
\pgfsetbuttcap%
\pgfsetroundjoin%
\definecolor{currentfill}{rgb}{0.263663,0.237631,0.518762}%
\pgfsetfillcolor{currentfill}%
\pgfsetfillopacity{0.800000}%
\pgfsetlinewidth{0.000000pt}%
\definecolor{currentstroke}{rgb}{0.000000,0.000000,0.000000}%
\pgfsetstrokecolor{currentstroke}%
\pgfsetdash{}{0pt}%
\pgfpathmoveto{\pgfqpoint{4.369206in}{2.574007in}}%
\pgfpathlineto{\pgfqpoint{4.382850in}{2.575613in}}%
\pgfpathlineto{\pgfqpoint{4.396503in}{2.577409in}}%
\pgfpathlineto{\pgfqpoint{4.410166in}{2.579395in}}%
\pgfpathlineto{\pgfqpoint{4.423840in}{2.581570in}}%
\pgfpathlineto{\pgfqpoint{4.431553in}{2.591156in}}%
\pgfpathlineto{\pgfqpoint{4.439260in}{2.600765in}}%
\pgfpathlineto{\pgfqpoint{4.446962in}{2.610399in}}%
\pgfpathlineto{\pgfqpoint{4.454660in}{2.620064in}}%
\pgfpathlineto{\pgfqpoint{4.440996in}{2.618179in}}%
\pgfpathlineto{\pgfqpoint{4.427342in}{2.616484in}}%
\pgfpathlineto{\pgfqpoint{4.413698in}{2.614978in}}%
\pgfpathlineto{\pgfqpoint{4.400064in}{2.613662in}}%
\pgfpathlineto{\pgfqpoint{4.392357in}{2.603696in}}%
\pgfpathlineto{\pgfqpoint{4.384645in}{2.593767in}}%
\pgfpathlineto{\pgfqpoint{4.376928in}{2.583872in}}%
\pgfpathlineto{\pgfqpoint{4.369206in}{2.574007in}}%
\pgfpathclose%
\pgfusepath{fill}%
\end{pgfscope}%
\begin{pgfscope}%
\pgfpathrectangle{\pgfqpoint{1.150000in}{0.150000in}}{\pgfqpoint{5.700000in}{5.700000in}}%
\pgfusepath{clip}%
\pgfsetbuttcap%
\pgfsetroundjoin%
\definecolor{currentfill}{rgb}{0.269308,0.218818,0.509577}%
\pgfsetfillcolor{currentfill}%
\pgfsetfillopacity{0.800000}%
\pgfsetlinewidth{0.000000pt}%
\definecolor{currentstroke}{rgb}{0.000000,0.000000,0.000000}%
\pgfsetstrokecolor{currentstroke}%
\pgfsetdash{}{0pt}%
\pgfpathmoveto{\pgfqpoint{4.283758in}{2.529236in}}%
\pgfpathlineto{\pgfqpoint{4.297372in}{2.530335in}}%
\pgfpathlineto{\pgfqpoint{4.310995in}{2.531626in}}%
\pgfpathlineto{\pgfqpoint{4.324628in}{2.533109in}}%
\pgfpathlineto{\pgfqpoint{4.338271in}{2.534783in}}%
\pgfpathlineto{\pgfqpoint{4.346012in}{2.544560in}}%
\pgfpathlineto{\pgfqpoint{4.353748in}{2.554354in}}%
\pgfpathlineto{\pgfqpoint{4.361480in}{2.564168in}}%
\pgfpathlineto{\pgfqpoint{4.369206in}{2.574007in}}%
\pgfpathlineto{\pgfqpoint{4.355573in}{2.572591in}}%
\pgfpathlineto{\pgfqpoint{4.341949in}{2.571367in}}%
\pgfpathlineto{\pgfqpoint{4.328335in}{2.570334in}}%
\pgfpathlineto{\pgfqpoint{4.314730in}{2.569494in}}%
\pgfpathlineto{\pgfqpoint{4.306995in}{2.559386in}}%
\pgfpathlineto{\pgfqpoint{4.299254in}{2.549309in}}%
\pgfpathlineto{\pgfqpoint{4.291509in}{2.539260in}}%
\pgfpathlineto{\pgfqpoint{4.283758in}{2.529236in}}%
\pgfpathclose%
\pgfusepath{fill}%
\end{pgfscope}%
\begin{pgfscope}%
\pgfpathrectangle{\pgfqpoint{1.150000in}{0.150000in}}{\pgfqpoint{5.700000in}{5.700000in}}%
\pgfusepath{clip}%
\pgfsetbuttcap%
\pgfsetroundjoin%
\definecolor{currentfill}{rgb}{0.283072,0.130895,0.449241}%
\pgfsetfillcolor{currentfill}%
\pgfsetfillopacity{0.800000}%
\pgfsetlinewidth{0.000000pt}%
\definecolor{currentstroke}{rgb}{0.000000,0.000000,0.000000}%
\pgfsetstrokecolor{currentstroke}%
\pgfsetdash{}{0pt}%
\pgfpathmoveto{\pgfqpoint{3.212144in}{2.344558in}}%
\pgfpathlineto{\pgfqpoint{3.225605in}{2.334385in}}%
\pgfpathlineto{\pgfqpoint{3.239065in}{2.324461in}}%
\pgfpathlineto{\pgfqpoint{3.252524in}{2.314785in}}%
\pgfpathlineto{\pgfqpoint{3.265983in}{2.305356in}}%
\pgfpathlineto{\pgfqpoint{3.274074in}{2.315108in}}%
\pgfpathlineto{\pgfqpoint{3.282159in}{2.324928in}}%
\pgfpathlineto{\pgfqpoint{3.290238in}{2.334818in}}%
\pgfpathlineto{\pgfqpoint{3.298310in}{2.344776in}}%
\pgfpathlineto{\pgfqpoint{3.284866in}{2.354146in}}%
\pgfpathlineto{\pgfqpoint{3.271422in}{2.363762in}}%
\pgfpathlineto{\pgfqpoint{3.257976in}{2.373625in}}%
\pgfpathlineto{\pgfqpoint{3.244530in}{2.383738in}}%
\pgfpathlineto{\pgfqpoint{3.236444in}{2.373828in}}%
\pgfpathlineto{\pgfqpoint{3.228350in}{2.363995in}}%
\pgfpathlineto{\pgfqpoint{3.220251in}{2.354238in}}%
\pgfpathlineto{\pgfqpoint{3.212144in}{2.344558in}}%
\pgfpathclose%
\pgfusepath{fill}%
\end{pgfscope}%
\begin{pgfscope}%
\pgfpathrectangle{\pgfqpoint{1.150000in}{0.150000in}}{\pgfqpoint{5.700000in}{5.700000in}}%
\pgfusepath{clip}%
\pgfsetbuttcap%
\pgfsetroundjoin%
\definecolor{currentfill}{rgb}{0.255645,0.260703,0.528312}%
\pgfsetfillcolor{currentfill}%
\pgfsetfillopacity{0.800000}%
\pgfsetlinewidth{0.000000pt}%
\definecolor{currentstroke}{rgb}{0.000000,0.000000,0.000000}%
\pgfsetstrokecolor{currentstroke}%
\pgfsetdash{}{0pt}%
\pgfpathmoveto{\pgfqpoint{4.454660in}{2.620064in}}%
\pgfpathlineto{\pgfqpoint{4.468334in}{2.622137in}}%
\pgfpathlineto{\pgfqpoint{4.482020in}{2.624398in}}%
\pgfpathlineto{\pgfqpoint{4.495716in}{2.626847in}}%
\pgfpathlineto{\pgfqpoint{4.509422in}{2.629483in}}%
\pgfpathlineto{\pgfqpoint{4.517105in}{2.638869in}}%
\pgfpathlineto{\pgfqpoint{4.524783in}{2.648285in}}%
\pgfpathlineto{\pgfqpoint{4.532456in}{2.657733in}}%
\pgfpathlineto{\pgfqpoint{4.540124in}{2.667219in}}%
\pgfpathlineto{\pgfqpoint{4.526428in}{2.664905in}}%
\pgfpathlineto{\pgfqpoint{4.512742in}{2.662779in}}%
\pgfpathlineto{\pgfqpoint{4.499067in}{2.660841in}}%
\pgfpathlineto{\pgfqpoint{4.485403in}{2.659090in}}%
\pgfpathlineto{\pgfqpoint{4.477724in}{2.649270in}}%
\pgfpathlineto{\pgfqpoint{4.470041in}{2.639495in}}%
\pgfpathlineto{\pgfqpoint{4.462353in}{2.629761in}}%
\pgfpathlineto{\pgfqpoint{4.454660in}{2.620064in}}%
\pgfpathclose%
\pgfusepath{fill}%
\end{pgfscope}%
\begin{pgfscope}%
\pgfpathrectangle{\pgfqpoint{1.150000in}{0.150000in}}{\pgfqpoint{5.700000in}{5.700000in}}%
\pgfusepath{clip}%
\pgfsetbuttcap%
\pgfsetroundjoin%
\definecolor{currentfill}{rgb}{0.274128,0.199721,0.498911}%
\pgfsetfillcolor{currentfill}%
\pgfsetfillopacity{0.800000}%
\pgfsetlinewidth{0.000000pt}%
\definecolor{currentstroke}{rgb}{0.000000,0.000000,0.000000}%
\pgfsetstrokecolor{currentstroke}%
\pgfsetdash{}{0pt}%
\pgfpathmoveto{\pgfqpoint{4.198311in}{2.485966in}}%
\pgfpathlineto{\pgfqpoint{4.211897in}{2.486516in}}%
\pgfpathlineto{\pgfqpoint{4.225491in}{2.487260in}}%
\pgfpathlineto{\pgfqpoint{4.239095in}{2.488199in}}%
\pgfpathlineto{\pgfqpoint{4.252708in}{2.489332in}}%
\pgfpathlineto{\pgfqpoint{4.260478in}{2.499285in}}%
\pgfpathlineto{\pgfqpoint{4.268243in}{2.509252in}}%
\pgfpathlineto{\pgfqpoint{4.276003in}{2.519235in}}%
\pgfpathlineto{\pgfqpoint{4.283758in}{2.529236in}}%
\pgfpathlineto{\pgfqpoint{4.270154in}{2.528331in}}%
\pgfpathlineto{\pgfqpoint{4.256559in}{2.527618in}}%
\pgfpathlineto{\pgfqpoint{4.242973in}{2.527100in}}%
\pgfpathlineto{\pgfqpoint{4.229396in}{2.526776in}}%
\pgfpathlineto{\pgfqpoint{4.221632in}{2.516536in}}%
\pgfpathlineto{\pgfqpoint{4.213863in}{2.506323in}}%
\pgfpathlineto{\pgfqpoint{4.206089in}{2.496134in}}%
\pgfpathlineto{\pgfqpoint{4.198311in}{2.485966in}}%
\pgfpathclose%
\pgfusepath{fill}%
\end{pgfscope}%
\begin{pgfscope}%
\pgfpathrectangle{\pgfqpoint{1.150000in}{0.150000in}}{\pgfqpoint{5.700000in}{5.700000in}}%
\pgfusepath{clip}%
\pgfsetbuttcap%
\pgfsetroundjoin%
\definecolor{currentfill}{rgb}{0.248629,0.278775,0.534556}%
\pgfsetfillcolor{currentfill}%
\pgfsetfillopacity{0.800000}%
\pgfsetlinewidth{0.000000pt}%
\definecolor{currentstroke}{rgb}{0.000000,0.000000,0.000000}%
\pgfsetstrokecolor{currentstroke}%
\pgfsetdash{}{0pt}%
\pgfpathmoveto{\pgfqpoint{4.540124in}{2.667219in}}%
\pgfpathlineto{\pgfqpoint{4.553831in}{2.669719in}}%
\pgfpathlineto{\pgfqpoint{4.567550in}{2.672405in}}%
\pgfpathlineto{\pgfqpoint{4.581279in}{2.675277in}}%
\pgfpathlineto{\pgfqpoint{4.595020in}{2.678334in}}%
\pgfpathlineto{\pgfqpoint{4.602673in}{2.687517in}}%
\pgfpathlineto{\pgfqpoint{4.610321in}{2.696738in}}%
\pgfpathlineto{\pgfqpoint{4.617964in}{2.705999in}}%
\pgfpathlineto{\pgfqpoint{4.625602in}{2.715306in}}%
\pgfpathlineto{\pgfqpoint{4.611872in}{2.712604in}}%
\pgfpathlineto{\pgfqpoint{4.598154in}{2.710087in}}%
\pgfpathlineto{\pgfqpoint{4.584447in}{2.707755in}}%
\pgfpathlineto{\pgfqpoint{4.570751in}{2.705609in}}%
\pgfpathlineto{\pgfqpoint{4.563101in}{2.695936in}}%
\pgfpathlineto{\pgfqpoint{4.555446in}{2.686316in}}%
\pgfpathlineto{\pgfqpoint{4.547787in}{2.676745in}}%
\pgfpathlineto{\pgfqpoint{4.540124in}{2.667219in}}%
\pgfpathclose%
\pgfusepath{fill}%
\end{pgfscope}%
\begin{pgfscope}%
\pgfpathrectangle{\pgfqpoint{1.150000in}{0.150000in}}{\pgfqpoint{5.700000in}{5.700000in}}%
\pgfusepath{clip}%
\pgfsetbuttcap%
\pgfsetroundjoin%
\definecolor{currentfill}{rgb}{0.239346,0.300855,0.540844}%
\pgfsetfillcolor{currentfill}%
\pgfsetfillopacity{0.800000}%
\pgfsetlinewidth{0.000000pt}%
\definecolor{currentstroke}{rgb}{0.000000,0.000000,0.000000}%
\pgfsetstrokecolor{currentstroke}%
\pgfsetdash{}{0pt}%
\pgfpathmoveto{\pgfqpoint{4.625602in}{2.715306in}}%
\pgfpathlineto{\pgfqpoint{4.639343in}{2.718193in}}%
\pgfpathlineto{\pgfqpoint{4.653096in}{2.721265in}}%
\pgfpathlineto{\pgfqpoint{4.666860in}{2.724521in}}%
\pgfpathlineto{\pgfqpoint{4.680637in}{2.727961in}}%
\pgfpathlineto{\pgfqpoint{4.688259in}{2.736942in}}%
\pgfpathlineto{\pgfqpoint{4.695876in}{2.745970in}}%
\pgfpathlineto{\pgfqpoint{4.703489in}{2.755049in}}%
\pgfpathlineto{\pgfqpoint{4.711097in}{2.764183in}}%
\pgfpathlineto{\pgfqpoint{4.697333in}{2.761131in}}%
\pgfpathlineto{\pgfqpoint{4.683581in}{2.758262in}}%
\pgfpathlineto{\pgfqpoint{4.669840in}{2.755577in}}%
\pgfpathlineto{\pgfqpoint{4.656111in}{2.753076in}}%
\pgfpathlineto{\pgfqpoint{4.648491in}{2.743544in}}%
\pgfpathlineto{\pgfqpoint{4.640865in}{2.734074in}}%
\pgfpathlineto{\pgfqpoint{4.633236in}{2.724663in}}%
\pgfpathlineto{\pgfqpoint{4.625602in}{2.715306in}}%
\pgfpathclose%
\pgfusepath{fill}%
\end{pgfscope}%
\begin{pgfscope}%
\pgfpathrectangle{\pgfqpoint{1.150000in}{0.150000in}}{\pgfqpoint{5.700000in}{5.700000in}}%
\pgfusepath{clip}%
\pgfsetbuttcap%
\pgfsetroundjoin%
\definecolor{currentfill}{rgb}{0.277134,0.185228,0.489898}%
\pgfsetfillcolor{currentfill}%
\pgfsetfillopacity{0.800000}%
\pgfsetlinewidth{0.000000pt}%
\definecolor{currentstroke}{rgb}{0.000000,0.000000,0.000000}%
\pgfsetstrokecolor{currentstroke}%
\pgfsetdash{}{0pt}%
\pgfpathmoveto{\pgfqpoint{4.112856in}{2.444430in}}%
\pgfpathlineto{\pgfqpoint{4.126416in}{2.444390in}}%
\pgfpathlineto{\pgfqpoint{4.139985in}{2.444547in}}%
\pgfpathlineto{\pgfqpoint{4.153562in}{2.444900in}}%
\pgfpathlineto{\pgfqpoint{4.167147in}{2.445450in}}%
\pgfpathlineto{\pgfqpoint{4.174945in}{2.455561in}}%
\pgfpathlineto{\pgfqpoint{4.182739in}{2.465681in}}%
\pgfpathlineto{\pgfqpoint{4.190527in}{2.475816in}}%
\pgfpathlineto{\pgfqpoint{4.198311in}{2.485966in}}%
\pgfpathlineto{\pgfqpoint{4.184734in}{2.485611in}}%
\pgfpathlineto{\pgfqpoint{4.171165in}{2.485452in}}%
\pgfpathlineto{\pgfqpoint{4.157605in}{2.485489in}}%
\pgfpathlineto{\pgfqpoint{4.144053in}{2.485723in}}%
\pgfpathlineto{\pgfqpoint{4.136261in}{2.475367in}}%
\pgfpathlineto{\pgfqpoint{4.128464in}{2.465035in}}%
\pgfpathlineto{\pgfqpoint{4.120663in}{2.454723in}}%
\pgfpathlineto{\pgfqpoint{4.112856in}{2.444430in}}%
\pgfpathclose%
\pgfusepath{fill}%
\end{pgfscope}%
\begin{pgfscope}%
\pgfpathrectangle{\pgfqpoint{1.150000in}{0.150000in}}{\pgfqpoint{5.700000in}{5.700000in}}%
\pgfusepath{clip}%
\pgfsetbuttcap%
\pgfsetroundjoin%
\definecolor{currentfill}{rgb}{0.267968,0.223549,0.512008}%
\pgfsetfillcolor{currentfill}%
\pgfsetfillopacity{0.800000}%
\pgfsetlinewidth{0.000000pt}%
\definecolor{currentstroke}{rgb}{0.000000,0.000000,0.000000}%
\pgfsetstrokecolor{currentstroke}%
\pgfsetdash{}{0pt}%
\pgfpathmoveto{\pgfqpoint{2.909458in}{2.567166in}}%
\pgfpathlineto{\pgfqpoint{2.923012in}{2.551509in}}%
\pgfpathlineto{\pgfqpoint{2.936560in}{2.536142in}}%
\pgfpathlineto{\pgfqpoint{2.950103in}{2.521063in}}%
\pgfpathlineto{\pgfqpoint{2.963639in}{2.506269in}}%
\pgfpathlineto{\pgfqpoint{2.971845in}{2.515219in}}%
\pgfpathlineto{\pgfqpoint{2.980043in}{2.524278in}}%
\pgfpathlineto{\pgfqpoint{2.988233in}{2.533447in}}%
\pgfpathlineto{\pgfqpoint{2.996415in}{2.542725in}}%
\pgfpathlineto{\pgfqpoint{2.982898in}{2.557424in}}%
\pgfpathlineto{\pgfqpoint{2.969375in}{2.572409in}}%
\pgfpathlineto{\pgfqpoint{2.955847in}{2.587680in}}%
\pgfpathlineto{\pgfqpoint{2.942313in}{2.603242in}}%
\pgfpathlineto{\pgfqpoint{2.934111in}{2.594047in}}%
\pgfpathlineto{\pgfqpoint{2.925902in}{2.584969in}}%
\pgfpathlineto{\pgfqpoint{2.917684in}{2.576009in}}%
\pgfpathlineto{\pgfqpoint{2.909458in}{2.567166in}}%
\pgfpathclose%
\pgfusepath{fill}%
\end{pgfscope}%
\begin{pgfscope}%
\pgfpathrectangle{\pgfqpoint{1.150000in}{0.150000in}}{\pgfqpoint{5.700000in}{5.700000in}}%
\pgfusepath{clip}%
\pgfsetbuttcap%
\pgfsetroundjoin%
\definecolor{currentfill}{rgb}{0.258965,0.251537,0.524736}%
\pgfsetfillcolor{currentfill}%
\pgfsetfillopacity{0.800000}%
\pgfsetlinewidth{0.000000pt}%
\definecolor{currentstroke}{rgb}{0.000000,0.000000,0.000000}%
\pgfsetstrokecolor{currentstroke}%
\pgfsetdash{}{0pt}%
\pgfpathmoveto{\pgfqpoint{2.855171in}{2.632747in}}%
\pgfpathlineto{\pgfqpoint{2.868754in}{2.615904in}}%
\pgfpathlineto{\pgfqpoint{2.882329in}{2.599361in}}%
\pgfpathlineto{\pgfqpoint{2.895897in}{2.583116in}}%
\pgfpathlineto{\pgfqpoint{2.909458in}{2.567166in}}%
\pgfpathlineto{\pgfqpoint{2.917684in}{2.576009in}}%
\pgfpathlineto{\pgfqpoint{2.925902in}{2.584969in}}%
\pgfpathlineto{\pgfqpoint{2.934111in}{2.594047in}}%
\pgfpathlineto{\pgfqpoint{2.942313in}{2.603242in}}%
\pgfpathlineto{\pgfqpoint{2.928772in}{2.619096in}}%
\pgfpathlineto{\pgfqpoint{2.915225in}{2.635245in}}%
\pgfpathlineto{\pgfqpoint{2.901670in}{2.651691in}}%
\pgfpathlineto{\pgfqpoint{2.888109in}{2.668438in}}%
\pgfpathlineto{\pgfqpoint{2.879888in}{2.659327in}}%
\pgfpathlineto{\pgfqpoint{2.871658in}{2.650342in}}%
\pgfpathlineto{\pgfqpoint{2.863419in}{2.641481in}}%
\pgfpathlineto{\pgfqpoint{2.855171in}{2.632747in}}%
\pgfpathclose%
\pgfusepath{fill}%
\end{pgfscope}%
\begin{pgfscope}%
\pgfpathrectangle{\pgfqpoint{1.150000in}{0.150000in}}{\pgfqpoint{5.700000in}{5.700000in}}%
\pgfusepath{clip}%
\pgfsetbuttcap%
\pgfsetroundjoin%
\definecolor{currentfill}{rgb}{0.283091,0.110553,0.431554}%
\pgfsetfillcolor{currentfill}%
\pgfsetfillopacity{0.800000}%
\pgfsetlinewidth{0.000000pt}%
\definecolor{currentstroke}{rgb}{0.000000,0.000000,0.000000}%
\pgfsetstrokecolor{currentstroke}%
\pgfsetdash{}{0pt}%
\pgfpathmoveto{\pgfqpoint{3.631255in}{2.289423in}}%
\pgfpathlineto{\pgfqpoint{3.644712in}{2.284917in}}%
\pgfpathlineto{\pgfqpoint{3.658173in}{2.280629in}}%
\pgfpathlineto{\pgfqpoint{3.671638in}{2.276557in}}%
\pgfpathlineto{\pgfqpoint{3.685107in}{2.272699in}}%
\pgfpathlineto{\pgfqpoint{3.693058in}{2.283143in}}%
\pgfpathlineto{\pgfqpoint{3.701003in}{2.293611in}}%
\pgfpathlineto{\pgfqpoint{3.708944in}{2.304106in}}%
\pgfpathlineto{\pgfqpoint{3.716879in}{2.314628in}}%
\pgfpathlineto{\pgfqpoint{3.703419in}{2.318521in}}%
\pgfpathlineto{\pgfqpoint{3.689964in}{2.322630in}}%
\pgfpathlineto{\pgfqpoint{3.676513in}{2.326954in}}%
\pgfpathlineto{\pgfqpoint{3.663066in}{2.331496in}}%
\pgfpathlineto{\pgfqpoint{3.655121in}{2.320926in}}%
\pgfpathlineto{\pgfqpoint{3.647171in}{2.310392in}}%
\pgfpathlineto{\pgfqpoint{3.639216in}{2.299891in}}%
\pgfpathlineto{\pgfqpoint{3.631255in}{2.289423in}}%
\pgfpathclose%
\pgfusepath{fill}%
\end{pgfscope}%
\begin{pgfscope}%
\pgfpathrectangle{\pgfqpoint{1.150000in}{0.150000in}}{\pgfqpoint{5.700000in}{5.700000in}}%
\pgfusepath{clip}%
\pgfsetbuttcap%
\pgfsetroundjoin%
\definecolor{currentfill}{rgb}{0.231674,0.318106,0.544834}%
\pgfsetfillcolor{currentfill}%
\pgfsetfillopacity{0.800000}%
\pgfsetlinewidth{0.000000pt}%
\definecolor{currentstroke}{rgb}{0.000000,0.000000,0.000000}%
\pgfsetstrokecolor{currentstroke}%
\pgfsetdash{}{0pt}%
\pgfpathmoveto{\pgfqpoint{4.711097in}{2.764183in}}%
\pgfpathlineto{\pgfqpoint{4.724873in}{2.767419in}}%
\pgfpathlineto{\pgfqpoint{4.738661in}{2.770837in}}%
\pgfpathlineto{\pgfqpoint{4.752461in}{2.774438in}}%
\pgfpathlineto{\pgfqpoint{4.766273in}{2.778221in}}%
\pgfpathlineto{\pgfqpoint{4.773864in}{2.787008in}}%
\pgfpathlineto{\pgfqpoint{4.781451in}{2.795852in}}%
\pgfpathlineto{\pgfqpoint{4.789033in}{2.804758in}}%
\pgfpathlineto{\pgfqpoint{4.796611in}{2.813731in}}%
\pgfpathlineto{\pgfqpoint{4.782812in}{2.810368in}}%
\pgfpathlineto{\pgfqpoint{4.769025in}{2.807187in}}%
\pgfpathlineto{\pgfqpoint{4.755251in}{2.804187in}}%
\pgfpathlineto{\pgfqpoint{4.741488in}{2.801370in}}%
\pgfpathlineto{\pgfqpoint{4.733896in}{2.791967in}}%
\pgfpathlineto{\pgfqpoint{4.726301in}{2.782638in}}%
\pgfpathlineto{\pgfqpoint{4.718701in}{2.773378in}}%
\pgfpathlineto{\pgfqpoint{4.711097in}{2.764183in}}%
\pgfpathclose%
\pgfusepath{fill}%
\end{pgfscope}%
\begin{pgfscope}%
\pgfpathrectangle{\pgfqpoint{1.150000in}{0.150000in}}{\pgfqpoint{5.700000in}{5.700000in}}%
\pgfusepath{clip}%
\pgfsetbuttcap%
\pgfsetroundjoin%
\definecolor{currentfill}{rgb}{0.147607,0.511733,0.557049}%
\pgfsetfillcolor{currentfill}%
\pgfsetfillopacity{0.800000}%
\pgfsetlinewidth{0.000000pt}%
\definecolor{currentstroke}{rgb}{0.000000,0.000000,0.000000}%
\pgfsetstrokecolor{currentstroke}%
\pgfsetdash{}{0pt}%
\pgfpathmoveto{\pgfqpoint{5.596710in}{3.319975in}}%
\pgfpathlineto{\pgfqpoint{5.610811in}{3.323869in}}%
\pgfpathlineto{\pgfqpoint{5.624928in}{3.327934in}}%
\pgfpathlineto{\pgfqpoint{5.639060in}{3.332167in}}%
\pgfpathlineto{\pgfqpoint{5.653207in}{3.336571in}}%
\pgfpathlineto{\pgfqpoint{5.660516in}{3.346035in}}%
\pgfpathlineto{\pgfqpoint{5.667828in}{3.355775in}}%
\pgfpathlineto{\pgfqpoint{5.675145in}{3.365797in}}%
\pgfpathlineto{\pgfqpoint{5.661020in}{3.361970in}}%
\pgfpathlineto{\pgfqpoint{5.646911in}{3.358312in}}%
\pgfpathlineto{\pgfqpoint{5.632817in}{3.354823in}}%
\pgfpathlineto{\pgfqpoint{5.618737in}{3.351503in}}%
\pgfpathlineto{\pgfqpoint{5.611391in}{3.340706in}}%
\pgfpathlineto{\pgfqpoint{5.604048in}{3.330200in}}%
\pgfpathlineto{\pgfqpoint{5.596710in}{3.319975in}}%
\pgfpathclose%
\pgfusepath{fill}%
\end{pgfscope}%
\begin{pgfscope}%
\pgfpathrectangle{\pgfqpoint{1.150000in}{0.150000in}}{\pgfqpoint{5.700000in}{5.700000in}}%
\pgfusepath{clip}%
\pgfsetbuttcap%
\pgfsetroundjoin%
\definecolor{currentfill}{rgb}{0.274128,0.199721,0.498911}%
\pgfsetfillcolor{currentfill}%
\pgfsetfillopacity{0.800000}%
\pgfsetlinewidth{0.000000pt}%
\definecolor{currentstroke}{rgb}{0.000000,0.000000,0.000000}%
\pgfsetstrokecolor{currentstroke}%
\pgfsetdash{}{0pt}%
\pgfpathmoveto{\pgfqpoint{2.963639in}{2.506269in}}%
\pgfpathlineto{\pgfqpoint{2.977170in}{2.491758in}}%
\pgfpathlineto{\pgfqpoint{2.990696in}{2.477528in}}%
\pgfpathlineto{\pgfqpoint{3.004217in}{2.463576in}}%
\pgfpathlineto{\pgfqpoint{3.017733in}{2.449901in}}%
\pgfpathlineto{\pgfqpoint{3.025920in}{2.458957in}}%
\pgfpathlineto{\pgfqpoint{3.034099in}{2.468115in}}%
\pgfpathlineto{\pgfqpoint{3.042270in}{2.477375in}}%
\pgfpathlineto{\pgfqpoint{3.050434in}{2.486735in}}%
\pgfpathlineto{\pgfqpoint{3.036936in}{2.500316in}}%
\pgfpathlineto{\pgfqpoint{3.023434in}{2.514174in}}%
\pgfpathlineto{\pgfqpoint{3.009927in}{2.528309in}}%
\pgfpathlineto{\pgfqpoint{2.996415in}{2.542725in}}%
\pgfpathlineto{\pgfqpoint{2.988233in}{2.533447in}}%
\pgfpathlineto{\pgfqpoint{2.980043in}{2.524278in}}%
\pgfpathlineto{\pgfqpoint{2.971845in}{2.515219in}}%
\pgfpathlineto{\pgfqpoint{2.963639in}{2.506269in}}%
\pgfpathclose%
\pgfusepath{fill}%
\end{pgfscope}%
\begin{pgfscope}%
\pgfpathrectangle{\pgfqpoint{1.150000in}{0.150000in}}{\pgfqpoint{5.700000in}{5.700000in}}%
\pgfusepath{clip}%
\pgfsetbuttcap%
\pgfsetroundjoin%
\definecolor{currentfill}{rgb}{0.280255,0.165693,0.476498}%
\pgfsetfillcolor{currentfill}%
\pgfsetfillopacity{0.800000}%
\pgfsetlinewidth{0.000000pt}%
\definecolor{currentstroke}{rgb}{0.000000,0.000000,0.000000}%
\pgfsetstrokecolor{currentstroke}%
\pgfsetdash{}{0pt}%
\pgfpathmoveto{\pgfqpoint{4.027386in}{2.404888in}}%
\pgfpathlineto{\pgfqpoint{4.040923in}{2.404216in}}%
\pgfpathlineto{\pgfqpoint{4.054467in}{2.403743in}}%
\pgfpathlineto{\pgfqpoint{4.068020in}{2.403470in}}%
\pgfpathlineto{\pgfqpoint{4.081580in}{2.403395in}}%
\pgfpathlineto{\pgfqpoint{4.089406in}{2.413637in}}%
\pgfpathlineto{\pgfqpoint{4.097228in}{2.423889in}}%
\pgfpathlineto{\pgfqpoint{4.105045in}{2.434152in}}%
\pgfpathlineto{\pgfqpoint{4.112856in}{2.444430in}}%
\pgfpathlineto{\pgfqpoint{4.099304in}{2.444667in}}%
\pgfpathlineto{\pgfqpoint{4.085760in}{2.445104in}}%
\pgfpathlineto{\pgfqpoint{4.072224in}{2.445739in}}%
\pgfpathlineto{\pgfqpoint{4.058695in}{2.446574in}}%
\pgfpathlineto{\pgfqpoint{4.050875in}{2.436123in}}%
\pgfpathlineto{\pgfqpoint{4.043050in}{2.425693in}}%
\pgfpathlineto{\pgfqpoint{4.035221in}{2.415282in}}%
\pgfpathlineto{\pgfqpoint{4.027386in}{2.404888in}}%
\pgfpathclose%
\pgfusepath{fill}%
\end{pgfscope}%
\begin{pgfscope}%
\pgfpathrectangle{\pgfqpoint{1.150000in}{0.150000in}}{\pgfqpoint{5.700000in}{5.700000in}}%
\pgfusepath{clip}%
\pgfsetbuttcap%
\pgfsetroundjoin%
\definecolor{currentfill}{rgb}{0.248629,0.278775,0.534556}%
\pgfsetfillcolor{currentfill}%
\pgfsetfillopacity{0.800000}%
\pgfsetlinewidth{0.000000pt}%
\definecolor{currentstroke}{rgb}{0.000000,0.000000,0.000000}%
\pgfsetstrokecolor{currentstroke}%
\pgfsetdash{}{0pt}%
\pgfpathmoveto{\pgfqpoint{2.800762in}{2.703175in}}%
\pgfpathlineto{\pgfqpoint{2.814377in}{2.685104in}}%
\pgfpathlineto{\pgfqpoint{2.827983in}{2.667344in}}%
\pgfpathlineto{\pgfqpoint{2.841581in}{2.649893in}}%
\pgfpathlineto{\pgfqpoint{2.855171in}{2.632747in}}%
\pgfpathlineto{\pgfqpoint{2.863419in}{2.641481in}}%
\pgfpathlineto{\pgfqpoint{2.871658in}{2.650342in}}%
\pgfpathlineto{\pgfqpoint{2.879888in}{2.659327in}}%
\pgfpathlineto{\pgfqpoint{2.888109in}{2.668438in}}%
\pgfpathlineto{\pgfqpoint{2.874540in}{2.685487in}}%
\pgfpathlineto{\pgfqpoint{2.860964in}{2.702841in}}%
\pgfpathlineto{\pgfqpoint{2.847379in}{2.720503in}}%
\pgfpathlineto{\pgfqpoint{2.833787in}{2.738477in}}%
\pgfpathlineto{\pgfqpoint{2.825544in}{2.729451in}}%
\pgfpathlineto{\pgfqpoint{2.817293in}{2.720559in}}%
\pgfpathlineto{\pgfqpoint{2.809032in}{2.711800in}}%
\pgfpathlineto{\pgfqpoint{2.800762in}{2.703175in}}%
\pgfpathclose%
\pgfusepath{fill}%
\end{pgfscope}%
\begin{pgfscope}%
\pgfpathrectangle{\pgfqpoint{1.150000in}{0.150000in}}{\pgfqpoint{5.700000in}{5.700000in}}%
\pgfusepath{clip}%
\pgfsetbuttcap%
\pgfsetroundjoin%
\definecolor{currentfill}{rgb}{0.221989,0.339161,0.548752}%
\pgfsetfillcolor{currentfill}%
\pgfsetfillopacity{0.800000}%
\pgfsetlinewidth{0.000000pt}%
\definecolor{currentstroke}{rgb}{0.000000,0.000000,0.000000}%
\pgfsetstrokecolor{currentstroke}%
\pgfsetdash{}{0pt}%
\pgfpathmoveto{\pgfqpoint{4.796611in}{2.813731in}}%
\pgfpathlineto{\pgfqpoint{4.810422in}{2.817276in}}%
\pgfpathlineto{\pgfqpoint{4.824246in}{2.821002in}}%
\pgfpathlineto{\pgfqpoint{4.838082in}{2.824910in}}%
\pgfpathlineto{\pgfqpoint{4.851932in}{2.828998in}}%
\pgfpathlineto{\pgfqpoint{4.859491in}{2.837603in}}%
\pgfpathlineto{\pgfqpoint{4.867047in}{2.846277in}}%
\pgfpathlineto{\pgfqpoint{4.874598in}{2.855026in}}%
\pgfpathlineto{\pgfqpoint{4.882145in}{2.863854in}}%
\pgfpathlineto{\pgfqpoint{4.868311in}{2.860218in}}%
\pgfpathlineto{\pgfqpoint{4.854489in}{2.856763in}}%
\pgfpathlineto{\pgfqpoint{4.840680in}{2.853488in}}%
\pgfpathlineto{\pgfqpoint{4.826883in}{2.850394in}}%
\pgfpathlineto{\pgfqpoint{4.819321in}{2.841103in}}%
\pgfpathlineto{\pgfqpoint{4.811755in}{2.831898in}}%
\pgfpathlineto{\pgfqpoint{4.804185in}{2.822776in}}%
\pgfpathlineto{\pgfqpoint{4.796611in}{2.813731in}}%
\pgfpathclose%
\pgfusepath{fill}%
\end{pgfscope}%
\begin{pgfscope}%
\pgfpathrectangle{\pgfqpoint{1.150000in}{0.150000in}}{\pgfqpoint{5.700000in}{5.700000in}}%
\pgfusepath{clip}%
\pgfsetbuttcap%
\pgfsetroundjoin%
\definecolor{currentfill}{rgb}{0.214298,0.355619,0.551184}%
\pgfsetfillcolor{currentfill}%
\pgfsetfillopacity{0.800000}%
\pgfsetlinewidth{0.000000pt}%
\definecolor{currentstroke}{rgb}{0.000000,0.000000,0.000000}%
\pgfsetstrokecolor{currentstroke}%
\pgfsetdash{}{0pt}%
\pgfpathmoveto{\pgfqpoint{4.882145in}{2.863854in}}%
\pgfpathlineto{\pgfqpoint{4.895993in}{2.867670in}}%
\pgfpathlineto{\pgfqpoint{4.909853in}{2.871666in}}%
\pgfpathlineto{\pgfqpoint{4.923726in}{2.875841in}}%
\pgfpathlineto{\pgfqpoint{4.937612in}{2.880196in}}%
\pgfpathlineto{\pgfqpoint{4.945140in}{2.888637in}}%
\pgfpathlineto{\pgfqpoint{4.952665in}{2.897161in}}%
\pgfpathlineto{\pgfqpoint{4.960185in}{2.905773in}}%
\pgfpathlineto{\pgfqpoint{4.967702in}{2.914479in}}%
\pgfpathlineto{\pgfqpoint{4.953832in}{2.910609in}}%
\pgfpathlineto{\pgfqpoint{4.939975in}{2.906918in}}%
\pgfpathlineto{\pgfqpoint{4.926130in}{2.903406in}}%
\pgfpathlineto{\pgfqpoint{4.912299in}{2.900073in}}%
\pgfpathlineto{\pgfqpoint{4.904766in}{2.890872in}}%
\pgfpathlineto{\pgfqpoint{4.897229in}{2.881772in}}%
\pgfpathlineto{\pgfqpoint{4.889689in}{2.872768in}}%
\pgfpathlineto{\pgfqpoint{4.882145in}{2.863854in}}%
\pgfpathclose%
\pgfusepath{fill}%
\end{pgfscope}%
\begin{pgfscope}%
\pgfpathrectangle{\pgfqpoint{1.150000in}{0.150000in}}{\pgfqpoint{5.700000in}{5.700000in}}%
\pgfusepath{clip}%
\pgfsetbuttcap%
\pgfsetroundjoin%
\definecolor{currentfill}{rgb}{0.281887,0.150881,0.465405}%
\pgfsetfillcolor{currentfill}%
\pgfsetfillopacity{0.800000}%
\pgfsetlinewidth{0.000000pt}%
\definecolor{currentstroke}{rgb}{0.000000,0.000000,0.000000}%
\pgfsetstrokecolor{currentstroke}%
\pgfsetdash{}{0pt}%
\pgfpathmoveto{\pgfqpoint{3.941889in}{2.367624in}}%
\pgfpathlineto{\pgfqpoint{3.955406in}{2.366276in}}%
\pgfpathlineto{\pgfqpoint{3.968929in}{2.365131in}}%
\pgfpathlineto{\pgfqpoint{3.982459in}{2.364188in}}%
\pgfpathlineto{\pgfqpoint{3.995997in}{2.363446in}}%
\pgfpathlineto{\pgfqpoint{4.003852in}{2.373790in}}%
\pgfpathlineto{\pgfqpoint{4.011701in}{2.384144in}}%
\pgfpathlineto{\pgfqpoint{4.019546in}{2.394510in}}%
\pgfpathlineto{\pgfqpoint{4.027386in}{2.404888in}}%
\pgfpathlineto{\pgfqpoint{4.013856in}{2.405761in}}%
\pgfpathlineto{\pgfqpoint{4.000334in}{2.406836in}}%
\pgfpathlineto{\pgfqpoint{3.986819in}{2.408112in}}%
\pgfpathlineto{\pgfqpoint{3.973311in}{2.409591in}}%
\pgfpathlineto{\pgfqpoint{3.965463in}{2.399070in}}%
\pgfpathlineto{\pgfqpoint{3.957610in}{2.388570in}}%
\pgfpathlineto{\pgfqpoint{3.949752in}{2.378088in}}%
\pgfpathlineto{\pgfqpoint{3.941889in}{2.367624in}}%
\pgfpathclose%
\pgfusepath{fill}%
\end{pgfscope}%
\begin{pgfscope}%
\pgfpathrectangle{\pgfqpoint{1.150000in}{0.150000in}}{\pgfqpoint{5.700000in}{5.700000in}}%
\pgfusepath{clip}%
\pgfsetbuttcap%
\pgfsetroundjoin%
\definecolor{currentfill}{rgb}{0.278826,0.175490,0.483397}%
\pgfsetfillcolor{currentfill}%
\pgfsetfillopacity{0.800000}%
\pgfsetlinewidth{0.000000pt}%
\definecolor{currentstroke}{rgb}{0.000000,0.000000,0.000000}%
\pgfsetstrokecolor{currentstroke}%
\pgfsetdash{}{0pt}%
\pgfpathmoveto{\pgfqpoint{3.017733in}{2.449901in}}%
\pgfpathlineto{\pgfqpoint{3.031245in}{2.436500in}}%
\pgfpathlineto{\pgfqpoint{3.044753in}{2.423372in}}%
\pgfpathlineto{\pgfqpoint{3.058256in}{2.410513in}}%
\pgfpathlineto{\pgfqpoint{3.071756in}{2.397922in}}%
\pgfpathlineto{\pgfqpoint{3.079924in}{2.407083in}}%
\pgfpathlineto{\pgfqpoint{3.088085in}{2.416339in}}%
\pgfpathlineto{\pgfqpoint{3.096238in}{2.425688in}}%
\pgfpathlineto{\pgfqpoint{3.104384in}{2.435130in}}%
\pgfpathlineto{\pgfqpoint{3.090902in}{2.447627in}}%
\pgfpathlineto{\pgfqpoint{3.077417in}{2.460393in}}%
\pgfpathlineto{\pgfqpoint{3.063927in}{2.473428in}}%
\pgfpathlineto{\pgfqpoint{3.050434in}{2.486735in}}%
\pgfpathlineto{\pgfqpoint{3.042270in}{2.477375in}}%
\pgfpathlineto{\pgfqpoint{3.034099in}{2.468115in}}%
\pgfpathlineto{\pgfqpoint{3.025920in}{2.458957in}}%
\pgfpathlineto{\pgfqpoint{3.017733in}{2.449901in}}%
\pgfpathclose%
\pgfusepath{fill}%
\end{pgfscope}%
\begin{pgfscope}%
\pgfpathrectangle{\pgfqpoint{1.150000in}{0.150000in}}{\pgfqpoint{5.700000in}{5.700000in}}%
\pgfusepath{clip}%
\pgfsetbuttcap%
\pgfsetroundjoin%
\definecolor{currentfill}{rgb}{0.282910,0.105393,0.426902}%
\pgfsetfillcolor{currentfill}%
\pgfsetfillopacity{0.800000}%
\pgfsetlinewidth{0.000000pt}%
\definecolor{currentstroke}{rgb}{0.000000,0.000000,0.000000}%
\pgfsetstrokecolor{currentstroke}%
\pgfsetdash{}{0pt}%
\pgfpathmoveto{\pgfqpoint{3.405864in}{2.278489in}}%
\pgfpathlineto{\pgfqpoint{3.419311in}{2.271266in}}%
\pgfpathlineto{\pgfqpoint{3.432760in}{2.264274in}}%
\pgfpathlineto{\pgfqpoint{3.446211in}{2.257511in}}%
\pgfpathlineto{\pgfqpoint{3.459664in}{2.250978in}}%
\pgfpathlineto{\pgfqpoint{3.467691in}{2.261123in}}%
\pgfpathlineto{\pgfqpoint{3.475712in}{2.271313in}}%
\pgfpathlineto{\pgfqpoint{3.483728in}{2.281549in}}%
\pgfpathlineto{\pgfqpoint{3.491738in}{2.291830in}}%
\pgfpathlineto{\pgfqpoint{3.478297in}{2.298336in}}%
\pgfpathlineto{\pgfqpoint{3.464858in}{2.305071in}}%
\pgfpathlineto{\pgfqpoint{3.451421in}{2.312035in}}%
\pgfpathlineto{\pgfqpoint{3.437985in}{2.319232in}}%
\pgfpathlineto{\pgfqpoint{3.429964in}{2.308966in}}%
\pgfpathlineto{\pgfqpoint{3.421936in}{2.298754in}}%
\pgfpathlineto{\pgfqpoint{3.413903in}{2.288595in}}%
\pgfpathlineto{\pgfqpoint{3.405864in}{2.278489in}}%
\pgfpathclose%
\pgfusepath{fill}%
\end{pgfscope}%
\begin{pgfscope}%
\pgfpathrectangle{\pgfqpoint{1.150000in}{0.150000in}}{\pgfqpoint{5.700000in}{5.700000in}}%
\pgfusepath{clip}%
\pgfsetbuttcap%
\pgfsetroundjoin%
\definecolor{currentfill}{rgb}{0.235526,0.309527,0.542944}%
\pgfsetfillcolor{currentfill}%
\pgfsetfillopacity{0.800000}%
\pgfsetlinewidth{0.000000pt}%
\definecolor{currentstroke}{rgb}{0.000000,0.000000,0.000000}%
\pgfsetstrokecolor{currentstroke}%
\pgfsetdash{}{0pt}%
\pgfpathmoveto{\pgfqpoint{2.746212in}{2.778628in}}%
\pgfpathlineto{\pgfqpoint{2.759864in}{2.759283in}}%
\pgfpathlineto{\pgfqpoint{2.773506in}{2.740262in}}%
\pgfpathlineto{\pgfqpoint{2.787139in}{2.721560in}}%
\pgfpathlineto{\pgfqpoint{2.800762in}{2.703175in}}%
\pgfpathlineto{\pgfqpoint{2.809032in}{2.711800in}}%
\pgfpathlineto{\pgfqpoint{2.817293in}{2.720559in}}%
\pgfpathlineto{\pgfqpoint{2.825544in}{2.729451in}}%
\pgfpathlineto{\pgfqpoint{2.833787in}{2.738477in}}%
\pgfpathlineto{\pgfqpoint{2.820185in}{2.756764in}}%
\pgfpathlineto{\pgfqpoint{2.806575in}{2.775367in}}%
\pgfpathlineto{\pgfqpoint{2.792955in}{2.794290in}}%
\pgfpathlineto{\pgfqpoint{2.779326in}{2.813536in}}%
\pgfpathlineto{\pgfqpoint{2.771062in}{2.804597in}}%
\pgfpathlineto{\pgfqpoint{2.762788in}{2.795798in}}%
\pgfpathlineto{\pgfqpoint{2.754505in}{2.787142in}}%
\pgfpathlineto{\pgfqpoint{2.746212in}{2.778628in}}%
\pgfpathclose%
\pgfusepath{fill}%
\end{pgfscope}%
\begin{pgfscope}%
\pgfpathrectangle{\pgfqpoint{1.150000in}{0.150000in}}{\pgfqpoint{5.700000in}{5.700000in}}%
\pgfusepath{clip}%
\pgfsetbuttcap%
\pgfsetroundjoin%
\definecolor{currentfill}{rgb}{0.204903,0.375746,0.553533}%
\pgfsetfillcolor{currentfill}%
\pgfsetfillopacity{0.800000}%
\pgfsetlinewidth{0.000000pt}%
\definecolor{currentstroke}{rgb}{0.000000,0.000000,0.000000}%
\pgfsetstrokecolor{currentstroke}%
\pgfsetdash{}{0pt}%
\pgfpathmoveto{\pgfqpoint{4.967702in}{2.914479in}}%
\pgfpathlineto{\pgfqpoint{4.981586in}{2.918528in}}%
\pgfpathlineto{\pgfqpoint{4.995482in}{2.922755in}}%
\pgfpathlineto{\pgfqpoint{5.009392in}{2.927161in}}%
\pgfpathlineto{\pgfqpoint{5.023316in}{2.931744in}}%
\pgfpathlineto{\pgfqpoint{5.030813in}{2.940045in}}%
\pgfpathlineto{\pgfqpoint{5.038306in}{2.948443in}}%
\pgfpathlineto{\pgfqpoint{5.045796in}{2.956945in}}%
\pgfpathlineto{\pgfqpoint{5.053282in}{2.965557in}}%
\pgfpathlineto{\pgfqpoint{5.039376in}{2.961490in}}%
\pgfpathlineto{\pgfqpoint{5.025483in}{2.957601in}}%
\pgfpathlineto{\pgfqpoint{5.011604in}{2.953890in}}%
\pgfpathlineto{\pgfqpoint{4.997737in}{2.950356in}}%
\pgfpathlineto{\pgfqpoint{4.990233in}{2.941218in}}%
\pgfpathlineto{\pgfqpoint{4.982726in}{2.932196in}}%
\pgfpathlineto{\pgfqpoint{4.975216in}{2.923285in}}%
\pgfpathlineto{\pgfqpoint{4.967702in}{2.914479in}}%
\pgfpathclose%
\pgfusepath{fill}%
\end{pgfscope}%
\begin{pgfscope}%
\pgfpathrectangle{\pgfqpoint{1.150000in}{0.150000in}}{\pgfqpoint{5.700000in}{5.700000in}}%
\pgfusepath{clip}%
\pgfsetbuttcap%
\pgfsetroundjoin%
\definecolor{currentfill}{rgb}{0.283197,0.115680,0.436115}%
\pgfsetfillcolor{currentfill}%
\pgfsetfillopacity{0.800000}%
\pgfsetlinewidth{0.000000pt}%
\definecolor{currentstroke}{rgb}{0.000000,0.000000,0.000000}%
\pgfsetstrokecolor{currentstroke}%
\pgfsetdash{}{0pt}%
\pgfpathmoveto{\pgfqpoint{3.265983in}{2.305356in}}%
\pgfpathlineto{\pgfqpoint{3.279440in}{2.296171in}}%
\pgfpathlineto{\pgfqpoint{3.292898in}{2.287229in}}%
\pgfpathlineto{\pgfqpoint{3.306355in}{2.278528in}}%
\pgfpathlineto{\pgfqpoint{3.319812in}{2.270068in}}%
\pgfpathlineto{\pgfqpoint{3.327889in}{2.279891in}}%
\pgfpathlineto{\pgfqpoint{3.335960in}{2.289775in}}%
\pgfpathlineto{\pgfqpoint{3.344025in}{2.299720in}}%
\pgfpathlineto{\pgfqpoint{3.352083in}{2.309725in}}%
\pgfpathlineto{\pgfqpoint{3.338640in}{2.318127in}}%
\pgfpathlineto{\pgfqpoint{3.325196in}{2.326768in}}%
\pgfpathlineto{\pgfqpoint{3.311753in}{2.335650in}}%
\pgfpathlineto{\pgfqpoint{3.298310in}{2.344776in}}%
\pgfpathlineto{\pgfqpoint{3.290238in}{2.334818in}}%
\pgfpathlineto{\pgfqpoint{3.282159in}{2.324928in}}%
\pgfpathlineto{\pgfqpoint{3.274074in}{2.315108in}}%
\pgfpathlineto{\pgfqpoint{3.265983in}{2.305356in}}%
\pgfpathclose%
\pgfusepath{fill}%
\end{pgfscope}%
\begin{pgfscope}%
\pgfpathrectangle{\pgfqpoint{1.150000in}{0.150000in}}{\pgfqpoint{5.700000in}{5.700000in}}%
\pgfusepath{clip}%
\pgfsetbuttcap%
\pgfsetroundjoin%
\definecolor{currentfill}{rgb}{0.195860,0.395433,0.555276}%
\pgfsetfillcolor{currentfill}%
\pgfsetfillopacity{0.800000}%
\pgfsetlinewidth{0.000000pt}%
\definecolor{currentstroke}{rgb}{0.000000,0.000000,0.000000}%
\pgfsetstrokecolor{currentstroke}%
\pgfsetdash{}{0pt}%
\pgfpathmoveto{\pgfqpoint{5.053282in}{2.965557in}}%
\pgfpathlineto{\pgfqpoint{5.067202in}{2.969801in}}%
\pgfpathlineto{\pgfqpoint{5.081136in}{2.974222in}}%
\pgfpathlineto{\pgfqpoint{5.095083in}{2.978820in}}%
\pgfpathlineto{\pgfqpoint{5.109044in}{2.983594in}}%
\pgfpathlineto{\pgfqpoint{5.116509in}{2.991784in}}%
\pgfpathlineto{\pgfqpoint{5.123972in}{3.000087in}}%
\pgfpathlineto{\pgfqpoint{5.131431in}{3.008511in}}%
\pgfpathlineto{\pgfqpoint{5.138888in}{3.017061in}}%
\pgfpathlineto{\pgfqpoint{5.124946in}{3.012836in}}%
\pgfpathlineto{\pgfqpoint{5.111017in}{3.008787in}}%
\pgfpathlineto{\pgfqpoint{5.097103in}{3.004914in}}%
\pgfpathlineto{\pgfqpoint{5.083201in}{3.001218in}}%
\pgfpathlineto{\pgfqpoint{5.075725in}{2.992108in}}%
\pgfpathlineto{\pgfqpoint{5.068247in}{2.983132in}}%
\pgfpathlineto{\pgfqpoint{5.060766in}{2.974284in}}%
\pgfpathlineto{\pgfqpoint{5.053282in}{2.965557in}}%
\pgfpathclose%
\pgfusepath{fill}%
\end{pgfscope}%
\begin{pgfscope}%
\pgfpathrectangle{\pgfqpoint{1.150000in}{0.150000in}}{\pgfqpoint{5.700000in}{5.700000in}}%
\pgfusepath{clip}%
\pgfsetbuttcap%
\pgfsetroundjoin%
\definecolor{currentfill}{rgb}{0.282884,0.135920,0.453427}%
\pgfsetfillcolor{currentfill}%
\pgfsetfillopacity{0.800000}%
\pgfsetlinewidth{0.000000pt}%
\definecolor{currentstroke}{rgb}{0.000000,0.000000,0.000000}%
\pgfsetstrokecolor{currentstroke}%
\pgfsetdash{}{0pt}%
\pgfpathmoveto{\pgfqpoint{3.856354in}{2.332946in}}%
\pgfpathlineto{\pgfqpoint{3.869853in}{2.330878in}}%
\pgfpathlineto{\pgfqpoint{3.883358in}{2.329017in}}%
\pgfpathlineto{\pgfqpoint{3.896869in}{2.327360in}}%
\pgfpathlineto{\pgfqpoint{3.910387in}{2.325908in}}%
\pgfpathlineto{\pgfqpoint{3.918270in}{2.336319in}}%
\pgfpathlineto{\pgfqpoint{3.926148in}{2.346741in}}%
\pgfpathlineto{\pgfqpoint{3.934021in}{2.357176in}}%
\pgfpathlineto{\pgfqpoint{3.941889in}{2.367624in}}%
\pgfpathlineto{\pgfqpoint{3.928380in}{2.369176in}}%
\pgfpathlineto{\pgfqpoint{3.914877in}{2.370932in}}%
\pgfpathlineto{\pgfqpoint{3.901380in}{2.372894in}}%
\pgfpathlineto{\pgfqpoint{3.887890in}{2.375061in}}%
\pgfpathlineto{\pgfqpoint{3.880014in}{2.364501in}}%
\pgfpathlineto{\pgfqpoint{3.872132in}{2.353963in}}%
\pgfpathlineto{\pgfqpoint{3.864246in}{2.343445in}}%
\pgfpathlineto{\pgfqpoint{3.856354in}{2.332946in}}%
\pgfpathclose%
\pgfusepath{fill}%
\end{pgfscope}%
\begin{pgfscope}%
\pgfpathrectangle{\pgfqpoint{1.150000in}{0.150000in}}{\pgfqpoint{5.700000in}{5.700000in}}%
\pgfusepath{clip}%
\pgfsetbuttcap%
\pgfsetroundjoin%
\definecolor{currentfill}{rgb}{0.282910,0.105393,0.426902}%
\pgfsetfillcolor{currentfill}%
\pgfsetfillopacity{0.800000}%
\pgfsetlinewidth{0.000000pt}%
\definecolor{currentstroke}{rgb}{0.000000,0.000000,0.000000}%
\pgfsetstrokecolor{currentstroke}%
\pgfsetdash{}{0pt}%
\pgfpathmoveto{\pgfqpoint{3.545525in}{2.268068in}}%
\pgfpathlineto{\pgfqpoint{3.558978in}{2.262688in}}%
\pgfpathlineto{\pgfqpoint{3.572435in}{2.257529in}}%
\pgfpathlineto{\pgfqpoint{3.585895in}{2.252590in}}%
\pgfpathlineto{\pgfqpoint{3.599358in}{2.247871in}}%
\pgfpathlineto{\pgfqpoint{3.607340in}{2.258212in}}%
\pgfpathlineto{\pgfqpoint{3.615317in}{2.268585in}}%
\pgfpathlineto{\pgfqpoint{3.623289in}{2.278988in}}%
\pgfpathlineto{\pgfqpoint{3.631255in}{2.289423in}}%
\pgfpathlineto{\pgfqpoint{3.617802in}{2.294147in}}%
\pgfpathlineto{\pgfqpoint{3.604352in}{2.299090in}}%
\pgfpathlineto{\pgfqpoint{3.590906in}{2.304254in}}%
\pgfpathlineto{\pgfqpoint{3.577463in}{2.309639in}}%
\pgfpathlineto{\pgfqpoint{3.569487in}{2.299188in}}%
\pgfpathlineto{\pgfqpoint{3.561505in}{2.288776in}}%
\pgfpathlineto{\pgfqpoint{3.553517in}{2.278403in}}%
\pgfpathlineto{\pgfqpoint{3.545525in}{2.268068in}}%
\pgfpathclose%
\pgfusepath{fill}%
\end{pgfscope}%
\begin{pgfscope}%
\pgfpathrectangle{\pgfqpoint{1.150000in}{0.150000in}}{\pgfqpoint{5.700000in}{5.700000in}}%
\pgfusepath{clip}%
\pgfsetbuttcap%
\pgfsetroundjoin%
\definecolor{currentfill}{rgb}{0.281412,0.155834,0.469201}%
\pgfsetfillcolor{currentfill}%
\pgfsetfillopacity{0.800000}%
\pgfsetlinewidth{0.000000pt}%
\definecolor{currentstroke}{rgb}{0.000000,0.000000,0.000000}%
\pgfsetstrokecolor{currentstroke}%
\pgfsetdash{}{0pt}%
\pgfpathmoveto{\pgfqpoint{3.071756in}{2.397922in}}%
\pgfpathlineto{\pgfqpoint{3.085253in}{2.385597in}}%
\pgfpathlineto{\pgfqpoint{3.098746in}{2.373536in}}%
\pgfpathlineto{\pgfqpoint{3.112236in}{2.361738in}}%
\pgfpathlineto{\pgfqpoint{3.125723in}{2.350199in}}%
\pgfpathlineto{\pgfqpoint{3.133874in}{2.359465in}}%
\pgfpathlineto{\pgfqpoint{3.142017in}{2.368817in}}%
\pgfpathlineto{\pgfqpoint{3.150154in}{2.378255in}}%
\pgfpathlineto{\pgfqpoint{3.158283in}{2.387778in}}%
\pgfpathlineto{\pgfqpoint{3.144812in}{2.399224in}}%
\pgfpathlineto{\pgfqpoint{3.131339in}{2.410930in}}%
\pgfpathlineto{\pgfqpoint{3.117863in}{2.422898in}}%
\pgfpathlineto{\pgfqpoint{3.104384in}{2.435130in}}%
\pgfpathlineto{\pgfqpoint{3.096238in}{2.425688in}}%
\pgfpathlineto{\pgfqpoint{3.088085in}{2.416339in}}%
\pgfpathlineto{\pgfqpoint{3.079924in}{2.407083in}}%
\pgfpathlineto{\pgfqpoint{3.071756in}{2.397922in}}%
\pgfpathclose%
\pgfusepath{fill}%
\end{pgfscope}%
\begin{pgfscope}%
\pgfpathrectangle{\pgfqpoint{1.150000in}{0.150000in}}{\pgfqpoint{5.700000in}{5.700000in}}%
\pgfusepath{clip}%
\pgfsetbuttcap%
\pgfsetroundjoin%
\definecolor{currentfill}{rgb}{0.187231,0.414746,0.556547}%
\pgfsetfillcolor{currentfill}%
\pgfsetfillopacity{0.800000}%
\pgfsetlinewidth{0.000000pt}%
\definecolor{currentstroke}{rgb}{0.000000,0.000000,0.000000}%
\pgfsetstrokecolor{currentstroke}%
\pgfsetdash{}{0pt}%
\pgfpathmoveto{\pgfqpoint{5.138888in}{3.017061in}}%
\pgfpathlineto{\pgfqpoint{5.152844in}{3.021463in}}%
\pgfpathlineto{\pgfqpoint{5.166814in}{3.026040in}}%
\pgfpathlineto{\pgfqpoint{5.180798in}{3.030793in}}%
\pgfpathlineto{\pgfqpoint{5.194797in}{3.035722in}}%
\pgfpathlineto{\pgfqpoint{5.202231in}{3.043835in}}%
\pgfpathlineto{\pgfqpoint{5.209663in}{3.052079in}}%
\pgfpathlineto{\pgfqpoint{5.217093in}{3.060462in}}%
\pgfpathlineto{\pgfqpoint{5.224521in}{3.068990in}}%
\pgfpathlineto{\pgfqpoint{5.210543in}{3.064643in}}%
\pgfpathlineto{\pgfqpoint{5.196579in}{3.060471in}}%
\pgfpathlineto{\pgfqpoint{5.182629in}{3.056474in}}%
\pgfpathlineto{\pgfqpoint{5.168693in}{3.052653in}}%
\pgfpathlineto{\pgfqpoint{5.161245in}{3.043533in}}%
\pgfpathlineto{\pgfqpoint{5.153795in}{3.034566in}}%
\pgfpathlineto{\pgfqpoint{5.146342in}{3.025744in}}%
\pgfpathlineto{\pgfqpoint{5.138888in}{3.017061in}}%
\pgfpathclose%
\pgfusepath{fill}%
\end{pgfscope}%
\begin{pgfscope}%
\pgfpathrectangle{\pgfqpoint{1.150000in}{0.150000in}}{\pgfqpoint{5.700000in}{5.700000in}}%
\pgfusepath{clip}%
\pgfsetbuttcap%
\pgfsetroundjoin%
\definecolor{currentfill}{rgb}{0.221989,0.339161,0.548752}%
\pgfsetfillcolor{currentfill}%
\pgfsetfillopacity{0.800000}%
\pgfsetlinewidth{0.000000pt}%
\definecolor{currentstroke}{rgb}{0.000000,0.000000,0.000000}%
\pgfsetstrokecolor{currentstroke}%
\pgfsetdash{}{0pt}%
\pgfpathmoveto{\pgfqpoint{2.691500in}{2.859296in}}%
\pgfpathlineto{\pgfqpoint{2.705194in}{2.838629in}}%
\pgfpathlineto{\pgfqpoint{2.718877in}{2.818297in}}%
\pgfpathlineto{\pgfqpoint{2.732550in}{2.798298in}}%
\pgfpathlineto{\pgfqpoint{2.746212in}{2.778628in}}%
\pgfpathlineto{\pgfqpoint{2.754505in}{2.787142in}}%
\pgfpathlineto{\pgfqpoint{2.762788in}{2.795798in}}%
\pgfpathlineto{\pgfqpoint{2.771062in}{2.804597in}}%
\pgfpathlineto{\pgfqpoint{2.779326in}{2.813536in}}%
\pgfpathlineto{\pgfqpoint{2.765688in}{2.833107in}}%
\pgfpathlineto{\pgfqpoint{2.752039in}{2.853007in}}%
\pgfpathlineto{\pgfqpoint{2.738379in}{2.873238in}}%
\pgfpathlineto{\pgfqpoint{2.724709in}{2.893805in}}%
\pgfpathlineto{\pgfqpoint{2.716421in}{2.884953in}}%
\pgfpathlineto{\pgfqpoint{2.708124in}{2.876251in}}%
\pgfpathlineto{\pgfqpoint{2.699817in}{2.867698in}}%
\pgfpathlineto{\pgfqpoint{2.691500in}{2.859296in}}%
\pgfpathclose%
\pgfusepath{fill}%
\end{pgfscope}%
\begin{pgfscope}%
\pgfpathrectangle{\pgfqpoint{1.150000in}{0.150000in}}{\pgfqpoint{5.700000in}{5.700000in}}%
\pgfusepath{clip}%
\pgfsetbuttcap%
\pgfsetroundjoin%
\definecolor{currentfill}{rgb}{0.179019,0.433756,0.557430}%
\pgfsetfillcolor{currentfill}%
\pgfsetfillopacity{0.800000}%
\pgfsetlinewidth{0.000000pt}%
\definecolor{currentstroke}{rgb}{0.000000,0.000000,0.000000}%
\pgfsetstrokecolor{currentstroke}%
\pgfsetdash{}{0pt}%
\pgfpathmoveto{\pgfqpoint{5.224521in}{3.068990in}}%
\pgfpathlineto{\pgfqpoint{5.238513in}{3.073512in}}%
\pgfpathlineto{\pgfqpoint{5.252519in}{3.078208in}}%
\pgfpathlineto{\pgfqpoint{5.266540in}{3.083079in}}%
\pgfpathlineto{\pgfqpoint{5.280576in}{3.088125in}}%
\pgfpathlineto{\pgfqpoint{5.287980in}{3.096202in}}%
\pgfpathlineto{\pgfqpoint{5.295382in}{3.104429in}}%
\pgfpathlineto{\pgfqpoint{5.302783in}{3.112815in}}%
\pgfpathlineto{\pgfqpoint{5.310183in}{3.121364in}}%
\pgfpathlineto{\pgfqpoint{5.296170in}{3.116933in}}%
\pgfpathlineto{\pgfqpoint{5.282171in}{3.112675in}}%
\pgfpathlineto{\pgfqpoint{5.268187in}{3.108592in}}%
\pgfpathlineto{\pgfqpoint{5.254216in}{3.104682in}}%
\pgfpathlineto{\pgfqpoint{5.246794in}{3.095508in}}%
\pgfpathlineto{\pgfqpoint{5.239371in}{3.086506in}}%
\pgfpathlineto{\pgfqpoint{5.231947in}{3.077669in}}%
\pgfpathlineto{\pgfqpoint{5.224521in}{3.068990in}}%
\pgfpathclose%
\pgfusepath{fill}%
\end{pgfscope}%
\begin{pgfscope}%
\pgfpathrectangle{\pgfqpoint{1.150000in}{0.150000in}}{\pgfqpoint{5.700000in}{5.700000in}}%
\pgfusepath{clip}%
\pgfsetbuttcap%
\pgfsetroundjoin%
\definecolor{currentfill}{rgb}{0.283229,0.120777,0.440584}%
\pgfsetfillcolor{currentfill}%
\pgfsetfillopacity{0.800000}%
\pgfsetlinewidth{0.000000pt}%
\definecolor{currentstroke}{rgb}{0.000000,0.000000,0.000000}%
\pgfsetstrokecolor{currentstroke}%
\pgfsetdash{}{0pt}%
\pgfpathmoveto{\pgfqpoint{3.770766in}{2.301185in}}%
\pgfpathlineto{\pgfqpoint{3.784250in}{2.298353in}}%
\pgfpathlineto{\pgfqpoint{3.797740in}{2.295730in}}%
\pgfpathlineto{\pgfqpoint{3.811236in}{2.293316in}}%
\pgfpathlineto{\pgfqpoint{3.824737in}{2.291109in}}%
\pgfpathlineto{\pgfqpoint{3.832649in}{2.301547in}}%
\pgfpathlineto{\pgfqpoint{3.840556in}{2.311998in}}%
\pgfpathlineto{\pgfqpoint{3.848458in}{2.322464in}}%
\pgfpathlineto{\pgfqpoint{3.856354in}{2.332946in}}%
\pgfpathlineto{\pgfqpoint{3.842861in}{2.335221in}}%
\pgfpathlineto{\pgfqpoint{3.829375in}{2.337703in}}%
\pgfpathlineto{\pgfqpoint{3.815893in}{2.340394in}}%
\pgfpathlineto{\pgfqpoint{3.802418in}{2.343294in}}%
\pgfpathlineto{\pgfqpoint{3.794513in}{2.332733in}}%
\pgfpathlineto{\pgfqpoint{3.786602in}{2.322195in}}%
\pgfpathlineto{\pgfqpoint{3.778687in}{2.311679in}}%
\pgfpathlineto{\pgfqpoint{3.770766in}{2.301185in}}%
\pgfpathclose%
\pgfusepath{fill}%
\end{pgfscope}%
\begin{pgfscope}%
\pgfpathrectangle{\pgfqpoint{1.150000in}{0.150000in}}{\pgfqpoint{5.700000in}{5.700000in}}%
\pgfusepath{clip}%
\pgfsetbuttcap%
\pgfsetroundjoin%
\definecolor{currentfill}{rgb}{0.171176,0.452530,0.557965}%
\pgfsetfillcolor{currentfill}%
\pgfsetfillopacity{0.800000}%
\pgfsetlinewidth{0.000000pt}%
\definecolor{currentstroke}{rgb}{0.000000,0.000000,0.000000}%
\pgfsetstrokecolor{currentstroke}%
\pgfsetdash{}{0pt}%
\pgfpathmoveto{\pgfqpoint{5.310183in}{3.121364in}}%
\pgfpathlineto{\pgfqpoint{5.324211in}{3.125969in}}%
\pgfpathlineto{\pgfqpoint{5.338253in}{3.130748in}}%
\pgfpathlineto{\pgfqpoint{5.352310in}{3.135700in}}%
\pgfpathlineto{\pgfqpoint{5.366382in}{3.140825in}}%
\pgfpathlineto{\pgfqpoint{5.373757in}{3.148912in}}%
\pgfpathlineto{\pgfqpoint{5.381132in}{3.157170in}}%
\pgfpathlineto{\pgfqpoint{5.388505in}{3.165607in}}%
\pgfpathlineto{\pgfqpoint{5.395878in}{3.174229in}}%
\pgfpathlineto{\pgfqpoint{5.381830in}{3.169750in}}%
\pgfpathlineto{\pgfqpoint{5.367797in}{3.165443in}}%
\pgfpathlineto{\pgfqpoint{5.353779in}{3.161310in}}%
\pgfpathlineto{\pgfqpoint{5.339774in}{3.157349in}}%
\pgfpathlineto{\pgfqpoint{5.332377in}{3.148071in}}%
\pgfpathlineto{\pgfqpoint{5.324980in}{3.138985in}}%
\pgfpathlineto{\pgfqpoint{5.317582in}{3.130085in}}%
\pgfpathlineto{\pgfqpoint{5.310183in}{3.121364in}}%
\pgfpathclose%
\pgfusepath{fill}%
\end{pgfscope}%
\begin{pgfscope}%
\pgfpathrectangle{\pgfqpoint{1.150000in}{0.150000in}}{\pgfqpoint{5.700000in}{5.700000in}}%
\pgfusepath{clip}%
\pgfsetbuttcap%
\pgfsetroundjoin%
\definecolor{currentfill}{rgb}{0.282884,0.135920,0.453427}%
\pgfsetfillcolor{currentfill}%
\pgfsetfillopacity{0.800000}%
\pgfsetlinewidth{0.000000pt}%
\definecolor{currentstroke}{rgb}{0.000000,0.000000,0.000000}%
\pgfsetstrokecolor{currentstroke}%
\pgfsetdash{}{0pt}%
\pgfpathmoveto{\pgfqpoint{3.125723in}{2.350199in}}%
\pgfpathlineto{\pgfqpoint{3.139208in}{2.338919in}}%
\pgfpathlineto{\pgfqpoint{3.152691in}{2.327896in}}%
\pgfpathlineto{\pgfqpoint{3.166172in}{2.317127in}}%
\pgfpathlineto{\pgfqpoint{3.179650in}{2.306611in}}%
\pgfpathlineto{\pgfqpoint{3.187784in}{2.315981in}}%
\pgfpathlineto{\pgfqpoint{3.195911in}{2.325429in}}%
\pgfpathlineto{\pgfqpoint{3.204031in}{2.334955in}}%
\pgfpathlineto{\pgfqpoint{3.212144in}{2.344558in}}%
\pgfpathlineto{\pgfqpoint{3.198682in}{2.354982in}}%
\pgfpathlineto{\pgfqpoint{3.185217in}{2.365659in}}%
\pgfpathlineto{\pgfqpoint{3.171751in}{2.376590in}}%
\pgfpathlineto{\pgfqpoint{3.158283in}{2.387778in}}%
\pgfpathlineto{\pgfqpoint{3.150154in}{2.378255in}}%
\pgfpathlineto{\pgfqpoint{3.142017in}{2.368817in}}%
\pgfpathlineto{\pgfqpoint{3.133874in}{2.359465in}}%
\pgfpathlineto{\pgfqpoint{3.125723in}{2.350199in}}%
\pgfpathclose%
\pgfusepath{fill}%
\end{pgfscope}%
\begin{pgfscope}%
\pgfpathrectangle{\pgfqpoint{1.150000in}{0.150000in}}{\pgfqpoint{5.700000in}{5.700000in}}%
\pgfusepath{clip}%
\pgfsetbuttcap%
\pgfsetroundjoin%
\definecolor{currentfill}{rgb}{0.163625,0.471133,0.558148}%
\pgfsetfillcolor{currentfill}%
\pgfsetfillopacity{0.800000}%
\pgfsetlinewidth{0.000000pt}%
\definecolor{currentstroke}{rgb}{0.000000,0.000000,0.000000}%
\pgfsetstrokecolor{currentstroke}%
\pgfsetdash{}{0pt}%
\pgfpathmoveto{\pgfqpoint{5.395878in}{3.174229in}}%
\pgfpathlineto{\pgfqpoint{5.409941in}{3.178880in}}%
\pgfpathlineto{\pgfqpoint{5.424018in}{3.183704in}}%
\pgfpathlineto{\pgfqpoint{5.438111in}{3.188701in}}%
\pgfpathlineto{\pgfqpoint{5.452219in}{3.193870in}}%
\pgfpathlineto{\pgfqpoint{5.459567in}{3.202018in}}%
\pgfpathlineto{\pgfqpoint{5.466914in}{3.210360in}}%
\pgfpathlineto{\pgfqpoint{5.474262in}{3.218902in}}%
\pgfpathlineto{\pgfqpoint{5.481610in}{3.227652in}}%
\pgfpathlineto{\pgfqpoint{5.467528in}{3.223162in}}%
\pgfpathlineto{\pgfqpoint{5.453462in}{3.218844in}}%
\pgfpathlineto{\pgfqpoint{5.439409in}{3.214697in}}%
\pgfpathlineto{\pgfqpoint{5.425372in}{3.210722in}}%
\pgfpathlineto{\pgfqpoint{5.417998in}{3.201283in}}%
\pgfpathlineto{\pgfqpoint{5.410624in}{3.192059in}}%
\pgfpathlineto{\pgfqpoint{5.403251in}{3.183044in}}%
\pgfpathlineto{\pgfqpoint{5.395878in}{3.174229in}}%
\pgfpathclose%
\pgfusepath{fill}%
\end{pgfscope}%
\begin{pgfscope}%
\pgfpathrectangle{\pgfqpoint{1.150000in}{0.150000in}}{\pgfqpoint{5.700000in}{5.700000in}}%
\pgfusepath{clip}%
\pgfsetbuttcap%
\pgfsetroundjoin%
\definecolor{currentfill}{rgb}{0.282910,0.105393,0.426902}%
\pgfsetfillcolor{currentfill}%
\pgfsetfillopacity{0.800000}%
\pgfsetlinewidth{0.000000pt}%
\definecolor{currentstroke}{rgb}{0.000000,0.000000,0.000000}%
\pgfsetstrokecolor{currentstroke}%
\pgfsetdash{}{0pt}%
\pgfpathmoveto{\pgfqpoint{3.319812in}{2.270068in}}%
\pgfpathlineto{\pgfqpoint{3.333270in}{2.261845in}}%
\pgfpathlineto{\pgfqpoint{3.346728in}{2.253860in}}%
\pgfpathlineto{\pgfqpoint{3.360187in}{2.246111in}}%
\pgfpathlineto{\pgfqpoint{3.373646in}{2.238596in}}%
\pgfpathlineto{\pgfqpoint{3.381710in}{2.248489in}}%
\pgfpathlineto{\pgfqpoint{3.389767in}{2.258436in}}%
\pgfpathlineto{\pgfqpoint{3.397818in}{2.268436in}}%
\pgfpathlineto{\pgfqpoint{3.405864in}{2.278489in}}%
\pgfpathlineto{\pgfqpoint{3.392417in}{2.285946in}}%
\pgfpathlineto{\pgfqpoint{3.378972in}{2.293636in}}%
\pgfpathlineto{\pgfqpoint{3.365527in}{2.301562in}}%
\pgfpathlineto{\pgfqpoint{3.352083in}{2.309725in}}%
\pgfpathlineto{\pgfqpoint{3.344025in}{2.299720in}}%
\pgfpathlineto{\pgfqpoint{3.335960in}{2.289775in}}%
\pgfpathlineto{\pgfqpoint{3.327889in}{2.279891in}}%
\pgfpathlineto{\pgfqpoint{3.319812in}{2.270068in}}%
\pgfpathclose%
\pgfusepath{fill}%
\end{pgfscope}%
\begin{pgfscope}%
\pgfpathrectangle{\pgfqpoint{1.150000in}{0.150000in}}{\pgfqpoint{5.700000in}{5.700000in}}%
\pgfusepath{clip}%
\pgfsetbuttcap%
\pgfsetroundjoin%
\definecolor{currentfill}{rgb}{0.283091,0.110553,0.431554}%
\pgfsetfillcolor{currentfill}%
\pgfsetfillopacity{0.800000}%
\pgfsetlinewidth{0.000000pt}%
\definecolor{currentstroke}{rgb}{0.000000,0.000000,0.000000}%
\pgfsetstrokecolor{currentstroke}%
\pgfsetdash{}{0pt}%
\pgfpathmoveto{\pgfqpoint{3.685107in}{2.272699in}}%
\pgfpathlineto{\pgfqpoint{3.698581in}{2.269056in}}%
\pgfpathlineto{\pgfqpoint{3.712059in}{2.265627in}}%
\pgfpathlineto{\pgfqpoint{3.725543in}{2.262409in}}%
\pgfpathlineto{\pgfqpoint{3.739031in}{2.259403in}}%
\pgfpathlineto{\pgfqpoint{3.746973in}{2.269821in}}%
\pgfpathlineto{\pgfqpoint{3.754909in}{2.280257in}}%
\pgfpathlineto{\pgfqpoint{3.762840in}{2.290712in}}%
\pgfpathlineto{\pgfqpoint{3.770766in}{2.301185in}}%
\pgfpathlineto{\pgfqpoint{3.757287in}{2.304228in}}%
\pgfpathlineto{\pgfqpoint{3.743812in}{2.307482in}}%
\pgfpathlineto{\pgfqpoint{3.730343in}{2.310948in}}%
\pgfpathlineto{\pgfqpoint{3.716879in}{2.314628in}}%
\pgfpathlineto{\pgfqpoint{3.708944in}{2.304106in}}%
\pgfpathlineto{\pgfqpoint{3.701003in}{2.293611in}}%
\pgfpathlineto{\pgfqpoint{3.693058in}{2.283143in}}%
\pgfpathlineto{\pgfqpoint{3.685107in}{2.272699in}}%
\pgfpathclose%
\pgfusepath{fill}%
\end{pgfscope}%
\begin{pgfscope}%
\pgfpathrectangle{\pgfqpoint{1.150000in}{0.150000in}}{\pgfqpoint{5.700000in}{5.700000in}}%
\pgfusepath{clip}%
\pgfsetbuttcap%
\pgfsetroundjoin%
\definecolor{currentfill}{rgb}{0.282656,0.100196,0.422160}%
\pgfsetfillcolor{currentfill}%
\pgfsetfillopacity{0.800000}%
\pgfsetlinewidth{0.000000pt}%
\definecolor{currentstroke}{rgb}{0.000000,0.000000,0.000000}%
\pgfsetstrokecolor{currentstroke}%
\pgfsetdash{}{0pt}%
\pgfpathmoveto{\pgfqpoint{3.459664in}{2.250978in}}%
\pgfpathlineto{\pgfqpoint{3.473119in}{2.244673in}}%
\pgfpathlineto{\pgfqpoint{3.486576in}{2.238594in}}%
\pgfpathlineto{\pgfqpoint{3.500035in}{2.232739in}}%
\pgfpathlineto{\pgfqpoint{3.513497in}{2.227109in}}%
\pgfpathlineto{\pgfqpoint{3.521512in}{2.237293in}}%
\pgfpathlineto{\pgfqpoint{3.529522in}{2.247514in}}%
\pgfpathlineto{\pgfqpoint{3.537526in}{2.257772in}}%
\pgfpathlineto{\pgfqpoint{3.545525in}{2.268068in}}%
\pgfpathlineto{\pgfqpoint{3.532074in}{2.273672in}}%
\pgfpathlineto{\pgfqpoint{3.518626in}{2.279499in}}%
\pgfpathlineto{\pgfqpoint{3.505181in}{2.285551in}}%
\pgfpathlineto{\pgfqpoint{3.491738in}{2.291830in}}%
\pgfpathlineto{\pgfqpoint{3.483728in}{2.281549in}}%
\pgfpathlineto{\pgfqpoint{3.475712in}{2.271313in}}%
\pgfpathlineto{\pgfqpoint{3.467691in}{2.261123in}}%
\pgfpathlineto{\pgfqpoint{3.459664in}{2.250978in}}%
\pgfpathclose%
\pgfusepath{fill}%
\end{pgfscope}%
\begin{pgfscope}%
\pgfpathrectangle{\pgfqpoint{1.150000in}{0.150000in}}{\pgfqpoint{5.700000in}{5.700000in}}%
\pgfusepath{clip}%
\pgfsetbuttcap%
\pgfsetroundjoin%
\definecolor{currentfill}{rgb}{0.206756,0.371758,0.553117}%
\pgfsetfillcolor{currentfill}%
\pgfsetfillopacity{0.800000}%
\pgfsetlinewidth{0.000000pt}%
\definecolor{currentstroke}{rgb}{0.000000,0.000000,0.000000}%
\pgfsetstrokecolor{currentstroke}%
\pgfsetdash{}{0pt}%
\pgfpathmoveto{\pgfqpoint{2.636606in}{2.945384in}}%
\pgfpathlineto{\pgfqpoint{2.650348in}{2.923342in}}%
\pgfpathlineto{\pgfqpoint{2.664077in}{2.901649in}}%
\pgfpathlineto{\pgfqpoint{2.677794in}{2.880301in}}%
\pgfpathlineto{\pgfqpoint{2.691500in}{2.859296in}}%
\pgfpathlineto{\pgfqpoint{2.699817in}{2.867698in}}%
\pgfpathlineto{\pgfqpoint{2.708124in}{2.876251in}}%
\pgfpathlineto{\pgfqpoint{2.716421in}{2.884953in}}%
\pgfpathlineto{\pgfqpoint{2.724709in}{2.893805in}}%
\pgfpathlineto{\pgfqpoint{2.711028in}{2.914710in}}%
\pgfpathlineto{\pgfqpoint{2.697335in}{2.935957in}}%
\pgfpathlineto{\pgfqpoint{2.683630in}{2.957549in}}%
\pgfpathlineto{\pgfqpoint{2.669914in}{2.979489in}}%
\pgfpathlineto{\pgfqpoint{2.661602in}{2.970726in}}%
\pgfpathlineto{\pgfqpoint{2.653280in}{2.962120in}}%
\pgfpathlineto{\pgfqpoint{2.644948in}{2.953673in}}%
\pgfpathlineto{\pgfqpoint{2.636606in}{2.945384in}}%
\pgfpathclose%
\pgfusepath{fill}%
\end{pgfscope}%
\begin{pgfscope}%
\pgfpathrectangle{\pgfqpoint{1.150000in}{0.150000in}}{\pgfqpoint{5.700000in}{5.700000in}}%
\pgfusepath{clip}%
\pgfsetbuttcap%
\pgfsetroundjoin%
\definecolor{currentfill}{rgb}{0.156270,0.489624,0.557936}%
\pgfsetfillcolor{currentfill}%
\pgfsetfillopacity{0.800000}%
\pgfsetlinewidth{0.000000pt}%
\definecolor{currentstroke}{rgb}{0.000000,0.000000,0.000000}%
\pgfsetstrokecolor{currentstroke}%
\pgfsetdash{}{0pt}%
\pgfpathmoveto{\pgfqpoint{5.481610in}{3.227652in}}%
\pgfpathlineto{\pgfqpoint{5.495707in}{3.232314in}}%
\pgfpathlineto{\pgfqpoint{5.509819in}{3.237147in}}%
\pgfpathlineto{\pgfqpoint{5.523947in}{3.242151in}}%
\pgfpathlineto{\pgfqpoint{5.538090in}{3.247327in}}%
\pgfpathlineto{\pgfqpoint{5.545412in}{3.255594in}}%
\pgfpathlineto{\pgfqpoint{5.552734in}{3.264078in}}%
\pgfpathlineto{\pgfqpoint{5.560059in}{3.272787in}}%
\pgfpathlineto{\pgfqpoint{5.567384in}{3.281727in}}%
\pgfpathlineto{\pgfqpoint{5.553269in}{3.277263in}}%
\pgfpathlineto{\pgfqpoint{5.539169in}{3.272968in}}%
\pgfpathlineto{\pgfqpoint{5.525085in}{3.268845in}}%
\pgfpathlineto{\pgfqpoint{5.511015in}{3.264892in}}%
\pgfpathlineto{\pgfqpoint{5.503661in}{3.255231in}}%
\pgfpathlineto{\pgfqpoint{5.496310in}{3.245809in}}%
\pgfpathlineto{\pgfqpoint{5.488959in}{3.236619in}}%
\pgfpathlineto{\pgfqpoint{5.481610in}{3.227652in}}%
\pgfpathclose%
\pgfusepath{fill}%
\end{pgfscope}%
\begin{pgfscope}%
\pgfpathrectangle{\pgfqpoint{1.150000in}{0.150000in}}{\pgfqpoint{5.700000in}{5.700000in}}%
\pgfusepath{clip}%
\pgfsetbuttcap%
\pgfsetroundjoin%
\definecolor{currentfill}{rgb}{0.270595,0.214069,0.507052}%
\pgfsetfillcolor{currentfill}%
\pgfsetfillopacity{0.800000}%
\pgfsetlinewidth{0.000000pt}%
\definecolor{currentstroke}{rgb}{0.000000,0.000000,0.000000}%
\pgfsetstrokecolor{currentstroke}%
\pgfsetdash{}{0pt}%
\pgfpathmoveto{\pgfqpoint{4.252708in}{2.489332in}}%
\pgfpathlineto{\pgfqpoint{4.266330in}{2.490658in}}%
\pgfpathlineto{\pgfqpoint{4.279962in}{2.492176in}}%
\pgfpathlineto{\pgfqpoint{4.293604in}{2.493886in}}%
\pgfpathlineto{\pgfqpoint{4.307255in}{2.495788in}}%
\pgfpathlineto{\pgfqpoint{4.315017in}{2.505526in}}%
\pgfpathlineto{\pgfqpoint{4.322773in}{2.515269in}}%
\pgfpathlineto{\pgfqpoint{4.330524in}{2.525020in}}%
\pgfpathlineto{\pgfqpoint{4.338271in}{2.534783in}}%
\pgfpathlineto{\pgfqpoint{4.324628in}{2.533109in}}%
\pgfpathlineto{\pgfqpoint{4.310995in}{2.531626in}}%
\pgfpathlineto{\pgfqpoint{4.297372in}{2.530335in}}%
\pgfpathlineto{\pgfqpoint{4.283758in}{2.529236in}}%
\pgfpathlineto{\pgfqpoint{4.276003in}{2.519235in}}%
\pgfpathlineto{\pgfqpoint{4.268243in}{2.509252in}}%
\pgfpathlineto{\pgfqpoint{4.260478in}{2.499285in}}%
\pgfpathlineto{\pgfqpoint{4.252708in}{2.489332in}}%
\pgfpathclose%
\pgfusepath{fill}%
\end{pgfscope}%
\begin{pgfscope}%
\pgfpathrectangle{\pgfqpoint{1.150000in}{0.150000in}}{\pgfqpoint{5.700000in}{5.700000in}}%
\pgfusepath{clip}%
\pgfsetbuttcap%
\pgfsetroundjoin%
\definecolor{currentfill}{rgb}{0.265145,0.232956,0.516599}%
\pgfsetfillcolor{currentfill}%
\pgfsetfillopacity{0.800000}%
\pgfsetlinewidth{0.000000pt}%
\definecolor{currentstroke}{rgb}{0.000000,0.000000,0.000000}%
\pgfsetstrokecolor{currentstroke}%
\pgfsetdash{}{0pt}%
\pgfpathmoveto{\pgfqpoint{4.338271in}{2.534783in}}%
\pgfpathlineto{\pgfqpoint{4.351923in}{2.536648in}}%
\pgfpathlineto{\pgfqpoint{4.365586in}{2.538704in}}%
\pgfpathlineto{\pgfqpoint{4.379258in}{2.540949in}}%
\pgfpathlineto{\pgfqpoint{4.392942in}{2.543385in}}%
\pgfpathlineto{\pgfqpoint{4.400674in}{2.552914in}}%
\pgfpathlineto{\pgfqpoint{4.408401in}{2.562452in}}%
\pgfpathlineto{\pgfqpoint{4.416123in}{2.572003in}}%
\pgfpathlineto{\pgfqpoint{4.423840in}{2.581570in}}%
\pgfpathlineto{\pgfqpoint{4.410166in}{2.579395in}}%
\pgfpathlineto{\pgfqpoint{4.396503in}{2.577409in}}%
\pgfpathlineto{\pgfqpoint{4.382850in}{2.575613in}}%
\pgfpathlineto{\pgfqpoint{4.369206in}{2.574007in}}%
\pgfpathlineto{\pgfqpoint{4.361480in}{2.564168in}}%
\pgfpathlineto{\pgfqpoint{4.353748in}{2.554354in}}%
\pgfpathlineto{\pgfqpoint{4.346012in}{2.544560in}}%
\pgfpathlineto{\pgfqpoint{4.338271in}{2.534783in}}%
\pgfpathclose%
\pgfusepath{fill}%
\end{pgfscope}%
\begin{pgfscope}%
\pgfpathrectangle{\pgfqpoint{1.150000in}{0.150000in}}{\pgfqpoint{5.700000in}{5.700000in}}%
\pgfusepath{clip}%
\pgfsetbuttcap%
\pgfsetroundjoin%
\definecolor{currentfill}{rgb}{0.275191,0.194905,0.496005}%
\pgfsetfillcolor{currentfill}%
\pgfsetfillopacity{0.800000}%
\pgfsetlinewidth{0.000000pt}%
\definecolor{currentstroke}{rgb}{0.000000,0.000000,0.000000}%
\pgfsetstrokecolor{currentstroke}%
\pgfsetdash{}{0pt}%
\pgfpathmoveto{\pgfqpoint{4.167147in}{2.445450in}}%
\pgfpathlineto{\pgfqpoint{4.180741in}{2.446196in}}%
\pgfpathlineto{\pgfqpoint{4.194344in}{2.447136in}}%
\pgfpathlineto{\pgfqpoint{4.207957in}{2.448270in}}%
\pgfpathlineto{\pgfqpoint{4.221578in}{2.449599in}}%
\pgfpathlineto{\pgfqpoint{4.229368in}{2.459525in}}%
\pgfpathlineto{\pgfqpoint{4.237153in}{2.469454in}}%
\pgfpathlineto{\pgfqpoint{4.244933in}{2.479389in}}%
\pgfpathlineto{\pgfqpoint{4.252708in}{2.489332in}}%
\pgfpathlineto{\pgfqpoint{4.239095in}{2.488199in}}%
\pgfpathlineto{\pgfqpoint{4.225491in}{2.487260in}}%
\pgfpathlineto{\pgfqpoint{4.211897in}{2.486516in}}%
\pgfpathlineto{\pgfqpoint{4.198311in}{2.485966in}}%
\pgfpathlineto{\pgfqpoint{4.190527in}{2.475816in}}%
\pgfpathlineto{\pgfqpoint{4.182739in}{2.465681in}}%
\pgfpathlineto{\pgfqpoint{4.174945in}{2.455561in}}%
\pgfpathlineto{\pgfqpoint{4.167147in}{2.445450in}}%
\pgfpathclose%
\pgfusepath{fill}%
\end{pgfscope}%
\begin{pgfscope}%
\pgfpathrectangle{\pgfqpoint{1.150000in}{0.150000in}}{\pgfqpoint{5.700000in}{5.700000in}}%
\pgfusepath{clip}%
\pgfsetbuttcap%
\pgfsetroundjoin%
\definecolor{currentfill}{rgb}{0.149039,0.508051,0.557250}%
\pgfsetfillcolor{currentfill}%
\pgfsetfillopacity{0.800000}%
\pgfsetlinewidth{0.000000pt}%
\definecolor{currentstroke}{rgb}{0.000000,0.000000,0.000000}%
\pgfsetstrokecolor{currentstroke}%
\pgfsetdash{}{0pt}%
\pgfpathmoveto{\pgfqpoint{5.567384in}{3.281727in}}%
\pgfpathlineto{\pgfqpoint{5.581515in}{3.286363in}}%
\pgfpathlineto{\pgfqpoint{5.595660in}{3.291168in}}%
\pgfpathlineto{\pgfqpoint{5.609822in}{3.296144in}}%
\pgfpathlineto{\pgfqpoint{5.623999in}{3.301291in}}%
\pgfpathlineto{\pgfqpoint{5.631297in}{3.309741in}}%
\pgfpathlineto{\pgfqpoint{5.638598in}{3.318432in}}%
\pgfpathlineto{\pgfqpoint{5.645901in}{3.327372in}}%
\pgfpathlineto{\pgfqpoint{5.653207in}{3.336571in}}%
\pgfpathlineto{\pgfqpoint{5.639060in}{3.332167in}}%
\pgfpathlineto{\pgfqpoint{5.624928in}{3.327934in}}%
\pgfpathlineto{\pgfqpoint{5.610811in}{3.323869in}}%
\pgfpathlineto{\pgfqpoint{5.596710in}{3.319975in}}%
\pgfpathlineto{\pgfqpoint{5.589374in}{3.310024in}}%
\pgfpathlineto{\pgfqpoint{5.582042in}{3.300338in}}%
\pgfpathlineto{\pgfqpoint{5.574712in}{3.290908in}}%
\pgfpathlineto{\pgfqpoint{5.567384in}{3.281727in}}%
\pgfpathclose%
\pgfusepath{fill}%
\end{pgfscope}%
\begin{pgfscope}%
\pgfpathrectangle{\pgfqpoint{1.150000in}{0.150000in}}{\pgfqpoint{5.700000in}{5.700000in}}%
\pgfusepath{clip}%
\pgfsetbuttcap%
\pgfsetroundjoin%
\definecolor{currentfill}{rgb}{0.283229,0.120777,0.440584}%
\pgfsetfillcolor{currentfill}%
\pgfsetfillopacity{0.800000}%
\pgfsetlinewidth{0.000000pt}%
\definecolor{currentstroke}{rgb}{0.000000,0.000000,0.000000}%
\pgfsetstrokecolor{currentstroke}%
\pgfsetdash{}{0pt}%
\pgfpathmoveto{\pgfqpoint{3.179650in}{2.306611in}}%
\pgfpathlineto{\pgfqpoint{3.193128in}{2.296346in}}%
\pgfpathlineto{\pgfqpoint{3.206603in}{2.286331in}}%
\pgfpathlineto{\pgfqpoint{3.220078in}{2.276564in}}%
\pgfpathlineto{\pgfqpoint{3.233551in}{2.267044in}}%
\pgfpathlineto{\pgfqpoint{3.241669in}{2.276517in}}%
\pgfpathlineto{\pgfqpoint{3.249780in}{2.286060in}}%
\pgfpathlineto{\pgfqpoint{3.257885in}{2.295673in}}%
\pgfpathlineto{\pgfqpoint{3.265983in}{2.305356in}}%
\pgfpathlineto{\pgfqpoint{3.252524in}{2.314785in}}%
\pgfpathlineto{\pgfqpoint{3.239065in}{2.324461in}}%
\pgfpathlineto{\pgfqpoint{3.225605in}{2.334385in}}%
\pgfpathlineto{\pgfqpoint{3.212144in}{2.344558in}}%
\pgfpathlineto{\pgfqpoint{3.204031in}{2.334955in}}%
\pgfpathlineto{\pgfqpoint{3.195911in}{2.325429in}}%
\pgfpathlineto{\pgfqpoint{3.187784in}{2.315981in}}%
\pgfpathlineto{\pgfqpoint{3.179650in}{2.306611in}}%
\pgfpathclose%
\pgfusepath{fill}%
\end{pgfscope}%
\begin{pgfscope}%
\pgfpathrectangle{\pgfqpoint{1.150000in}{0.150000in}}{\pgfqpoint{5.700000in}{5.700000in}}%
\pgfusepath{clip}%
\pgfsetbuttcap%
\pgfsetroundjoin%
\definecolor{currentfill}{rgb}{0.258965,0.251537,0.524736}%
\pgfsetfillcolor{currentfill}%
\pgfsetfillopacity{0.800000}%
\pgfsetlinewidth{0.000000pt}%
\definecolor{currentstroke}{rgb}{0.000000,0.000000,0.000000}%
\pgfsetstrokecolor{currentstroke}%
\pgfsetdash{}{0pt}%
\pgfpathmoveto{\pgfqpoint{4.423840in}{2.581570in}}%
\pgfpathlineto{\pgfqpoint{4.437525in}{2.583934in}}%
\pgfpathlineto{\pgfqpoint{4.451220in}{2.586487in}}%
\pgfpathlineto{\pgfqpoint{4.464926in}{2.589228in}}%
\pgfpathlineto{\pgfqpoint{4.478643in}{2.592156in}}%
\pgfpathlineto{\pgfqpoint{4.486345in}{2.601462in}}%
\pgfpathlineto{\pgfqpoint{4.494043in}{2.610783in}}%
\pgfpathlineto{\pgfqpoint{4.501735in}{2.620122in}}%
\pgfpathlineto{\pgfqpoint{4.509422in}{2.629483in}}%
\pgfpathlineto{\pgfqpoint{4.495716in}{2.626847in}}%
\pgfpathlineto{\pgfqpoint{4.482020in}{2.624398in}}%
\pgfpathlineto{\pgfqpoint{4.468334in}{2.622137in}}%
\pgfpathlineto{\pgfqpoint{4.454660in}{2.620064in}}%
\pgfpathlineto{\pgfqpoint{4.446962in}{2.610399in}}%
\pgfpathlineto{\pgfqpoint{4.439260in}{2.600765in}}%
\pgfpathlineto{\pgfqpoint{4.431553in}{2.591156in}}%
\pgfpathlineto{\pgfqpoint{4.423840in}{2.581570in}}%
\pgfpathclose%
\pgfusepath{fill}%
\end{pgfscope}%
\begin{pgfscope}%
\pgfpathrectangle{\pgfqpoint{1.150000in}{0.150000in}}{\pgfqpoint{5.700000in}{5.700000in}}%
\pgfusepath{clip}%
\pgfsetbuttcap%
\pgfsetroundjoin%
\definecolor{currentfill}{rgb}{0.278826,0.175490,0.483397}%
\pgfsetfillcolor{currentfill}%
\pgfsetfillopacity{0.800000}%
\pgfsetlinewidth{0.000000pt}%
\definecolor{currentstroke}{rgb}{0.000000,0.000000,0.000000}%
\pgfsetstrokecolor{currentstroke}%
\pgfsetdash{}{0pt}%
\pgfpathmoveto{\pgfqpoint{4.081580in}{2.403395in}}%
\pgfpathlineto{\pgfqpoint{4.095148in}{2.403518in}}%
\pgfpathlineto{\pgfqpoint{4.108725in}{2.403839in}}%
\pgfpathlineto{\pgfqpoint{4.122310in}{2.404356in}}%
\pgfpathlineto{\pgfqpoint{4.135903in}{2.405070in}}%
\pgfpathlineto{\pgfqpoint{4.143722in}{2.415160in}}%
\pgfpathlineto{\pgfqpoint{4.151535in}{2.425252in}}%
\pgfpathlineto{\pgfqpoint{4.159344in}{2.435348in}}%
\pgfpathlineto{\pgfqpoint{4.167147in}{2.445450in}}%
\pgfpathlineto{\pgfqpoint{4.153562in}{2.444900in}}%
\pgfpathlineto{\pgfqpoint{4.139985in}{2.444547in}}%
\pgfpathlineto{\pgfqpoint{4.126416in}{2.444390in}}%
\pgfpathlineto{\pgfqpoint{4.112856in}{2.444430in}}%
\pgfpathlineto{\pgfqpoint{4.105045in}{2.434152in}}%
\pgfpathlineto{\pgfqpoint{4.097228in}{2.423889in}}%
\pgfpathlineto{\pgfqpoint{4.089406in}{2.413637in}}%
\pgfpathlineto{\pgfqpoint{4.081580in}{2.403395in}}%
\pgfpathclose%
\pgfusepath{fill}%
\end{pgfscope}%
\begin{pgfscope}%
\pgfpathrectangle{\pgfqpoint{1.150000in}{0.150000in}}{\pgfqpoint{5.700000in}{5.700000in}}%
\pgfusepath{clip}%
\pgfsetbuttcap%
\pgfsetroundjoin%
\definecolor{currentfill}{rgb}{0.250425,0.274290,0.533103}%
\pgfsetfillcolor{currentfill}%
\pgfsetfillopacity{0.800000}%
\pgfsetlinewidth{0.000000pt}%
\definecolor{currentstroke}{rgb}{0.000000,0.000000,0.000000}%
\pgfsetstrokecolor{currentstroke}%
\pgfsetdash{}{0pt}%
\pgfpathmoveto{\pgfqpoint{4.509422in}{2.629483in}}%
\pgfpathlineto{\pgfqpoint{4.523140in}{2.632306in}}%
\pgfpathlineto{\pgfqpoint{4.536869in}{2.635316in}}%
\pgfpathlineto{\pgfqpoint{4.550610in}{2.638512in}}%
\pgfpathlineto{\pgfqpoint{4.564362in}{2.641894in}}%
\pgfpathlineto{\pgfqpoint{4.572034in}{2.650968in}}%
\pgfpathlineto{\pgfqpoint{4.579701in}{2.660064in}}%
\pgfpathlineto{\pgfqpoint{4.587363in}{2.669184in}}%
\pgfpathlineto{\pgfqpoint{4.595020in}{2.678334in}}%
\pgfpathlineto{\pgfqpoint{4.581279in}{2.675277in}}%
\pgfpathlineto{\pgfqpoint{4.567550in}{2.672405in}}%
\pgfpathlineto{\pgfqpoint{4.553831in}{2.669719in}}%
\pgfpathlineto{\pgfqpoint{4.540124in}{2.667219in}}%
\pgfpathlineto{\pgfqpoint{4.532456in}{2.657733in}}%
\pgfpathlineto{\pgfqpoint{4.524783in}{2.648285in}}%
\pgfpathlineto{\pgfqpoint{4.517105in}{2.638869in}}%
\pgfpathlineto{\pgfqpoint{4.509422in}{2.629483in}}%
\pgfpathclose%
\pgfusepath{fill}%
\end{pgfscope}%
\begin{pgfscope}%
\pgfpathrectangle{\pgfqpoint{1.150000in}{0.150000in}}{\pgfqpoint{5.700000in}{5.700000in}}%
\pgfusepath{clip}%
\pgfsetbuttcap%
\pgfsetroundjoin%
\definecolor{currentfill}{rgb}{0.282656,0.100196,0.422160}%
\pgfsetfillcolor{currentfill}%
\pgfsetfillopacity{0.800000}%
\pgfsetlinewidth{0.000000pt}%
\definecolor{currentstroke}{rgb}{0.000000,0.000000,0.000000}%
\pgfsetstrokecolor{currentstroke}%
\pgfsetdash{}{0pt}%
\pgfpathmoveto{\pgfqpoint{3.599358in}{2.247871in}}%
\pgfpathlineto{\pgfqpoint{3.612825in}{2.243370in}}%
\pgfpathlineto{\pgfqpoint{3.626296in}{2.239087in}}%
\pgfpathlineto{\pgfqpoint{3.639771in}{2.235019in}}%
\pgfpathlineto{\pgfqpoint{3.653251in}{2.231167in}}%
\pgfpathlineto{\pgfqpoint{3.661223in}{2.241515in}}%
\pgfpathlineto{\pgfqpoint{3.669189in}{2.251886in}}%
\pgfpathlineto{\pgfqpoint{3.677151in}{2.262281in}}%
\pgfpathlineto{\pgfqpoint{3.685107in}{2.272699in}}%
\pgfpathlineto{\pgfqpoint{3.671638in}{2.276557in}}%
\pgfpathlineto{\pgfqpoint{3.658173in}{2.280629in}}%
\pgfpathlineto{\pgfqpoint{3.644712in}{2.284917in}}%
\pgfpathlineto{\pgfqpoint{3.631255in}{2.289423in}}%
\pgfpathlineto{\pgfqpoint{3.623289in}{2.278988in}}%
\pgfpathlineto{\pgfqpoint{3.615317in}{2.268585in}}%
\pgfpathlineto{\pgfqpoint{3.607340in}{2.258212in}}%
\pgfpathlineto{\pgfqpoint{3.599358in}{2.247871in}}%
\pgfpathclose%
\pgfusepath{fill}%
\end{pgfscope}%
\begin{pgfscope}%
\pgfpathrectangle{\pgfqpoint{1.150000in}{0.150000in}}{\pgfqpoint{5.700000in}{5.700000in}}%
\pgfusepath{clip}%
\pgfsetbuttcap%
\pgfsetroundjoin%
\definecolor{currentfill}{rgb}{0.243113,0.292092,0.538516}%
\pgfsetfillcolor{currentfill}%
\pgfsetfillopacity{0.800000}%
\pgfsetlinewidth{0.000000pt}%
\definecolor{currentstroke}{rgb}{0.000000,0.000000,0.000000}%
\pgfsetstrokecolor{currentstroke}%
\pgfsetdash{}{0pt}%
\pgfpathmoveto{\pgfqpoint{4.595020in}{2.678334in}}%
\pgfpathlineto{\pgfqpoint{4.608773in}{2.681577in}}%
\pgfpathlineto{\pgfqpoint{4.622537in}{2.685005in}}%
\pgfpathlineto{\pgfqpoint{4.636313in}{2.688617in}}%
\pgfpathlineto{\pgfqpoint{4.650102in}{2.692414in}}%
\pgfpathlineto{\pgfqpoint{4.657743in}{2.701252in}}%
\pgfpathlineto{\pgfqpoint{4.665379in}{2.710120in}}%
\pgfpathlineto{\pgfqpoint{4.673010in}{2.719021in}}%
\pgfpathlineto{\pgfqpoint{4.680637in}{2.727961in}}%
\pgfpathlineto{\pgfqpoint{4.666860in}{2.724521in}}%
\pgfpathlineto{\pgfqpoint{4.653096in}{2.721265in}}%
\pgfpathlineto{\pgfqpoint{4.639343in}{2.718193in}}%
\pgfpathlineto{\pgfqpoint{4.625602in}{2.715306in}}%
\pgfpathlineto{\pgfqpoint{4.617964in}{2.705999in}}%
\pgfpathlineto{\pgfqpoint{4.610321in}{2.696738in}}%
\pgfpathlineto{\pgfqpoint{4.602673in}{2.687517in}}%
\pgfpathlineto{\pgfqpoint{4.595020in}{2.678334in}}%
\pgfpathclose%
\pgfusepath{fill}%
\end{pgfscope}%
\begin{pgfscope}%
\pgfpathrectangle{\pgfqpoint{1.150000in}{0.150000in}}{\pgfqpoint{5.700000in}{5.700000in}}%
\pgfusepath{clip}%
\pgfsetbuttcap%
\pgfsetroundjoin%
\definecolor{currentfill}{rgb}{0.280868,0.160771,0.472899}%
\pgfsetfillcolor{currentfill}%
\pgfsetfillopacity{0.800000}%
\pgfsetlinewidth{0.000000pt}%
\definecolor{currentstroke}{rgb}{0.000000,0.000000,0.000000}%
\pgfsetstrokecolor{currentstroke}%
\pgfsetdash{}{0pt}%
\pgfpathmoveto{\pgfqpoint{3.995997in}{2.363446in}}%
\pgfpathlineto{\pgfqpoint{4.009542in}{2.362905in}}%
\pgfpathlineto{\pgfqpoint{4.023095in}{2.362564in}}%
\pgfpathlineto{\pgfqpoint{4.036655in}{2.362423in}}%
\pgfpathlineto{\pgfqpoint{4.050223in}{2.362480in}}%
\pgfpathlineto{\pgfqpoint{4.058070in}{2.372704in}}%
\pgfpathlineto{\pgfqpoint{4.065912in}{2.382930in}}%
\pgfpathlineto{\pgfqpoint{4.073748in}{2.393160in}}%
\pgfpathlineto{\pgfqpoint{4.081580in}{2.403395in}}%
\pgfpathlineto{\pgfqpoint{4.068020in}{2.403470in}}%
\pgfpathlineto{\pgfqpoint{4.054467in}{2.403743in}}%
\pgfpathlineto{\pgfqpoint{4.040923in}{2.404216in}}%
\pgfpathlineto{\pgfqpoint{4.027386in}{2.404888in}}%
\pgfpathlineto{\pgfqpoint{4.019546in}{2.394510in}}%
\pgfpathlineto{\pgfqpoint{4.011701in}{2.384144in}}%
\pgfpathlineto{\pgfqpoint{4.003852in}{2.373790in}}%
\pgfpathlineto{\pgfqpoint{3.995997in}{2.363446in}}%
\pgfpathclose%
\pgfusepath{fill}%
\end{pgfscope}%
\begin{pgfscope}%
\pgfpathrectangle{\pgfqpoint{1.150000in}{0.150000in}}{\pgfqpoint{5.700000in}{5.700000in}}%
\pgfusepath{clip}%
\pgfsetbuttcap%
\pgfsetroundjoin%
\definecolor{currentfill}{rgb}{0.269308,0.218818,0.509577}%
\pgfsetfillcolor{currentfill}%
\pgfsetfillopacity{0.800000}%
\pgfsetlinewidth{0.000000pt}%
\definecolor{currentstroke}{rgb}{0.000000,0.000000,0.000000}%
\pgfsetstrokecolor{currentstroke}%
\pgfsetdash{}{0pt}%
\pgfpathmoveto{\pgfqpoint{2.876467in}{2.532990in}}%
\pgfpathlineto{\pgfqpoint{2.890043in}{2.517205in}}%
\pgfpathlineto{\pgfqpoint{2.903613in}{2.501710in}}%
\pgfpathlineto{\pgfqpoint{2.917176in}{2.486504in}}%
\pgfpathlineto{\pgfqpoint{2.930733in}{2.471583in}}%
\pgfpathlineto{\pgfqpoint{2.938972in}{2.480086in}}%
\pgfpathlineto{\pgfqpoint{2.947203in}{2.488702in}}%
\pgfpathlineto{\pgfqpoint{2.955425in}{2.497430in}}%
\pgfpathlineto{\pgfqpoint{2.963639in}{2.506269in}}%
\pgfpathlineto{\pgfqpoint{2.950103in}{2.521063in}}%
\pgfpathlineto{\pgfqpoint{2.936560in}{2.536142in}}%
\pgfpathlineto{\pgfqpoint{2.923012in}{2.551509in}}%
\pgfpathlineto{\pgfqpoint{2.909458in}{2.567166in}}%
\pgfpathlineto{\pgfqpoint{2.901223in}{2.558442in}}%
\pgfpathlineto{\pgfqpoint{2.892980in}{2.549838in}}%
\pgfpathlineto{\pgfqpoint{2.884728in}{2.541353in}}%
\pgfpathlineto{\pgfqpoint{2.876467in}{2.532990in}}%
\pgfpathclose%
\pgfusepath{fill}%
\end{pgfscope}%
\begin{pgfscope}%
\pgfpathrectangle{\pgfqpoint{1.150000in}{0.150000in}}{\pgfqpoint{5.700000in}{5.700000in}}%
\pgfusepath{clip}%
\pgfsetbuttcap%
\pgfsetroundjoin%
\definecolor{currentfill}{rgb}{0.260571,0.246922,0.522828}%
\pgfsetfillcolor{currentfill}%
\pgfsetfillopacity{0.800000}%
\pgfsetlinewidth{0.000000pt}%
\definecolor{currentstroke}{rgb}{0.000000,0.000000,0.000000}%
\pgfsetstrokecolor{currentstroke}%
\pgfsetdash{}{0pt}%
\pgfpathmoveto{\pgfqpoint{2.822093in}{2.599084in}}%
\pgfpathlineto{\pgfqpoint{2.835698in}{2.582112in}}%
\pgfpathlineto{\pgfqpoint{2.849295in}{2.565441in}}%
\pgfpathlineto{\pgfqpoint{2.862885in}{2.549067in}}%
\pgfpathlineto{\pgfqpoint{2.876467in}{2.532990in}}%
\pgfpathlineto{\pgfqpoint{2.884728in}{2.541353in}}%
\pgfpathlineto{\pgfqpoint{2.892980in}{2.549838in}}%
\pgfpathlineto{\pgfqpoint{2.901223in}{2.558442in}}%
\pgfpathlineto{\pgfqpoint{2.909458in}{2.567166in}}%
\pgfpathlineto{\pgfqpoint{2.895897in}{2.583116in}}%
\pgfpathlineto{\pgfqpoint{2.882329in}{2.599361in}}%
\pgfpathlineto{\pgfqpoint{2.868754in}{2.615904in}}%
\pgfpathlineto{\pgfqpoint{2.855171in}{2.632747in}}%
\pgfpathlineto{\pgfqpoint{2.846915in}{2.624139in}}%
\pgfpathlineto{\pgfqpoint{2.838650in}{2.615659in}}%
\pgfpathlineto{\pgfqpoint{2.830376in}{2.607307in}}%
\pgfpathlineto{\pgfqpoint{2.822093in}{2.599084in}}%
\pgfpathclose%
\pgfusepath{fill}%
\end{pgfscope}%
\begin{pgfscope}%
\pgfpathrectangle{\pgfqpoint{1.150000in}{0.150000in}}{\pgfqpoint{5.700000in}{5.700000in}}%
\pgfusepath{clip}%
\pgfsetbuttcap%
\pgfsetroundjoin%
\definecolor{currentfill}{rgb}{0.233603,0.313828,0.543914}%
\pgfsetfillcolor{currentfill}%
\pgfsetfillopacity{0.800000}%
\pgfsetlinewidth{0.000000pt}%
\definecolor{currentstroke}{rgb}{0.000000,0.000000,0.000000}%
\pgfsetstrokecolor{currentstroke}%
\pgfsetdash{}{0pt}%
\pgfpathmoveto{\pgfqpoint{4.680637in}{2.727961in}}%
\pgfpathlineto{\pgfqpoint{4.694425in}{2.731584in}}%
\pgfpathlineto{\pgfqpoint{4.708226in}{2.735391in}}%
\pgfpathlineto{\pgfqpoint{4.722039in}{2.739381in}}%
\pgfpathlineto{\pgfqpoint{4.735864in}{2.743553in}}%
\pgfpathlineto{\pgfqpoint{4.743473in}{2.752157in}}%
\pgfpathlineto{\pgfqpoint{4.751078in}{2.760800in}}%
\pgfpathlineto{\pgfqpoint{4.758678in}{2.769487in}}%
\pgfpathlineto{\pgfqpoint{4.766273in}{2.778221in}}%
\pgfpathlineto{\pgfqpoint{4.752461in}{2.774438in}}%
\pgfpathlineto{\pgfqpoint{4.738661in}{2.770837in}}%
\pgfpathlineto{\pgfqpoint{4.724873in}{2.767419in}}%
\pgfpathlineto{\pgfqpoint{4.711097in}{2.764183in}}%
\pgfpathlineto{\pgfqpoint{4.703489in}{2.755049in}}%
\pgfpathlineto{\pgfqpoint{4.695876in}{2.745970in}}%
\pgfpathlineto{\pgfqpoint{4.688259in}{2.736942in}}%
\pgfpathlineto{\pgfqpoint{4.680637in}{2.727961in}}%
\pgfpathclose%
\pgfusepath{fill}%
\end{pgfscope}%
\begin{pgfscope}%
\pgfpathrectangle{\pgfqpoint{1.150000in}{0.150000in}}{\pgfqpoint{5.700000in}{5.700000in}}%
\pgfusepath{clip}%
\pgfsetbuttcap%
\pgfsetroundjoin%
\definecolor{currentfill}{rgb}{0.275191,0.194905,0.496005}%
\pgfsetfillcolor{currentfill}%
\pgfsetfillopacity{0.800000}%
\pgfsetlinewidth{0.000000pt}%
\definecolor{currentstroke}{rgb}{0.000000,0.000000,0.000000}%
\pgfsetstrokecolor{currentstroke}%
\pgfsetdash{}{0pt}%
\pgfpathmoveto{\pgfqpoint{2.930733in}{2.471583in}}%
\pgfpathlineto{\pgfqpoint{2.944285in}{2.456946in}}%
\pgfpathlineto{\pgfqpoint{2.957831in}{2.442589in}}%
\pgfpathlineto{\pgfqpoint{2.971372in}{2.428512in}}%
\pgfpathlineto{\pgfqpoint{2.984908in}{2.414711in}}%
\pgfpathlineto{\pgfqpoint{2.993126in}{2.423352in}}%
\pgfpathlineto{\pgfqpoint{3.001337in}{2.432098in}}%
\pgfpathlineto{\pgfqpoint{3.009539in}{2.440948in}}%
\pgfpathlineto{\pgfqpoint{3.017733in}{2.449901in}}%
\pgfpathlineto{\pgfqpoint{3.004217in}{2.463576in}}%
\pgfpathlineto{\pgfqpoint{2.990696in}{2.477528in}}%
\pgfpathlineto{\pgfqpoint{2.977170in}{2.491758in}}%
\pgfpathlineto{\pgfqpoint{2.963639in}{2.506269in}}%
\pgfpathlineto{\pgfqpoint{2.955425in}{2.497430in}}%
\pgfpathlineto{\pgfqpoint{2.947203in}{2.488702in}}%
\pgfpathlineto{\pgfqpoint{2.938972in}{2.480086in}}%
\pgfpathlineto{\pgfqpoint{2.930733in}{2.471583in}}%
\pgfpathclose%
\pgfusepath{fill}%
\end{pgfscope}%
\begin{pgfscope}%
\pgfpathrectangle{\pgfqpoint{1.150000in}{0.150000in}}{\pgfqpoint{5.700000in}{5.700000in}}%
\pgfusepath{clip}%
\pgfsetbuttcap%
\pgfsetroundjoin%
\definecolor{currentfill}{rgb}{0.282623,0.140926,0.457517}%
\pgfsetfillcolor{currentfill}%
\pgfsetfillopacity{0.800000}%
\pgfsetlinewidth{0.000000pt}%
\definecolor{currentstroke}{rgb}{0.000000,0.000000,0.000000}%
\pgfsetstrokecolor{currentstroke}%
\pgfsetdash{}{0pt}%
\pgfpathmoveto{\pgfqpoint{3.910387in}{2.325908in}}%
\pgfpathlineto{\pgfqpoint{3.923912in}{2.324660in}}%
\pgfpathlineto{\pgfqpoint{3.937443in}{2.323615in}}%
\pgfpathlineto{\pgfqpoint{3.950982in}{2.322772in}}%
\pgfpathlineto{\pgfqpoint{3.964528in}{2.322130in}}%
\pgfpathlineto{\pgfqpoint{3.972402in}{2.332453in}}%
\pgfpathlineto{\pgfqpoint{3.980272in}{2.342779in}}%
\pgfpathlineto{\pgfqpoint{3.988137in}{2.353109in}}%
\pgfpathlineto{\pgfqpoint{3.995997in}{2.363446in}}%
\pgfpathlineto{\pgfqpoint{3.982459in}{2.364188in}}%
\pgfpathlineto{\pgfqpoint{3.968929in}{2.365131in}}%
\pgfpathlineto{\pgfqpoint{3.955406in}{2.366276in}}%
\pgfpathlineto{\pgfqpoint{3.941889in}{2.367624in}}%
\pgfpathlineto{\pgfqpoint{3.934021in}{2.357176in}}%
\pgfpathlineto{\pgfqpoint{3.926148in}{2.346741in}}%
\pgfpathlineto{\pgfqpoint{3.918270in}{2.336319in}}%
\pgfpathlineto{\pgfqpoint{3.910387in}{2.325908in}}%
\pgfpathclose%
\pgfusepath{fill}%
\end{pgfscope}%
\begin{pgfscope}%
\pgfpathrectangle{\pgfqpoint{1.150000in}{0.150000in}}{\pgfqpoint{5.700000in}{5.700000in}}%
\pgfusepath{clip}%
\pgfsetbuttcap%
\pgfsetroundjoin%
\definecolor{currentfill}{rgb}{0.223925,0.334994,0.548053}%
\pgfsetfillcolor{currentfill}%
\pgfsetfillopacity{0.800000}%
\pgfsetlinewidth{0.000000pt}%
\definecolor{currentstroke}{rgb}{0.000000,0.000000,0.000000}%
\pgfsetstrokecolor{currentstroke}%
\pgfsetdash{}{0pt}%
\pgfpathmoveto{\pgfqpoint{4.766273in}{2.778221in}}%
\pgfpathlineto{\pgfqpoint{4.780098in}{2.782186in}}%
\pgfpathlineto{\pgfqpoint{4.793936in}{2.786333in}}%
\pgfpathlineto{\pgfqpoint{4.807786in}{2.790662in}}%
\pgfpathlineto{\pgfqpoint{4.821650in}{2.795172in}}%
\pgfpathlineto{\pgfqpoint{4.829227in}{2.803549in}}%
\pgfpathlineto{\pgfqpoint{4.836800in}{2.811976in}}%
\pgfpathlineto{\pgfqpoint{4.844368in}{2.820457in}}%
\pgfpathlineto{\pgfqpoint{4.851932in}{2.828998in}}%
\pgfpathlineto{\pgfqpoint{4.838082in}{2.824910in}}%
\pgfpathlineto{\pgfqpoint{4.824246in}{2.821002in}}%
\pgfpathlineto{\pgfqpoint{4.810422in}{2.817276in}}%
\pgfpathlineto{\pgfqpoint{4.796611in}{2.813731in}}%
\pgfpathlineto{\pgfqpoint{4.789033in}{2.804758in}}%
\pgfpathlineto{\pgfqpoint{4.781451in}{2.795852in}}%
\pgfpathlineto{\pgfqpoint{4.773864in}{2.787008in}}%
\pgfpathlineto{\pgfqpoint{4.766273in}{2.778221in}}%
\pgfpathclose%
\pgfusepath{fill}%
\end{pgfscope}%
\begin{pgfscope}%
\pgfpathrectangle{\pgfqpoint{1.150000in}{0.150000in}}{\pgfqpoint{5.700000in}{5.700000in}}%
\pgfusepath{clip}%
\pgfsetbuttcap%
\pgfsetroundjoin%
\definecolor{currentfill}{rgb}{0.250425,0.274290,0.533103}%
\pgfsetfillcolor{currentfill}%
\pgfsetfillopacity{0.800000}%
\pgfsetlinewidth{0.000000pt}%
\definecolor{currentstroke}{rgb}{0.000000,0.000000,0.000000}%
\pgfsetstrokecolor{currentstroke}%
\pgfsetdash{}{0pt}%
\pgfpathmoveto{\pgfqpoint{2.767591in}{2.670030in}}%
\pgfpathlineto{\pgfqpoint{2.781230in}{2.651829in}}%
\pgfpathlineto{\pgfqpoint{2.794859in}{2.633940in}}%
\pgfpathlineto{\pgfqpoint{2.808480in}{2.616359in}}%
\pgfpathlineto{\pgfqpoint{2.822093in}{2.599084in}}%
\pgfpathlineto{\pgfqpoint{2.830376in}{2.607307in}}%
\pgfpathlineto{\pgfqpoint{2.838650in}{2.615659in}}%
\pgfpathlineto{\pgfqpoint{2.846915in}{2.624139in}}%
\pgfpathlineto{\pgfqpoint{2.855171in}{2.632747in}}%
\pgfpathlineto{\pgfqpoint{2.841581in}{2.649893in}}%
\pgfpathlineto{\pgfqpoint{2.827983in}{2.667344in}}%
\pgfpathlineto{\pgfqpoint{2.814377in}{2.685104in}}%
\pgfpathlineto{\pgfqpoint{2.800762in}{2.703175in}}%
\pgfpathlineto{\pgfqpoint{2.792484in}{2.694685in}}%
\pgfpathlineto{\pgfqpoint{2.784196in}{2.686330in}}%
\pgfpathlineto{\pgfqpoint{2.775898in}{2.678112in}}%
\pgfpathlineto{\pgfqpoint{2.767591in}{2.670030in}}%
\pgfpathclose%
\pgfusepath{fill}%
\end{pgfscope}%
\begin{pgfscope}%
\pgfpathrectangle{\pgfqpoint{1.150000in}{0.150000in}}{\pgfqpoint{5.700000in}{5.700000in}}%
\pgfusepath{clip}%
\pgfsetbuttcap%
\pgfsetroundjoin%
\definecolor{currentfill}{rgb}{0.141935,0.526453,0.555991}%
\pgfsetfillcolor{currentfill}%
\pgfsetfillopacity{0.800000}%
\pgfsetlinewidth{0.000000pt}%
\definecolor{currentstroke}{rgb}{0.000000,0.000000,0.000000}%
\pgfsetstrokecolor{currentstroke}%
\pgfsetdash{}{0pt}%
\pgfpathmoveto{\pgfqpoint{5.653207in}{3.336571in}}%
\pgfpathlineto{\pgfqpoint{5.667369in}{3.341144in}}%
\pgfpathlineto{\pgfqpoint{5.681548in}{3.345886in}}%
\pgfpathlineto{\pgfqpoint{5.695742in}{3.350798in}}%
\pgfpathlineto{\pgfqpoint{5.709952in}{3.355879in}}%
\pgfpathlineto{\pgfqpoint{5.717230in}{3.364580in}}%
\pgfpathlineto{\pgfqpoint{5.724511in}{3.373549in}}%
\pgfpathlineto{\pgfqpoint{5.731796in}{3.382794in}}%
\pgfpathlineto{\pgfqpoint{5.717610in}{3.378292in}}%
\pgfpathlineto{\pgfqpoint{5.703440in}{3.373958in}}%
\pgfpathlineto{\pgfqpoint{5.689284in}{3.369793in}}%
\pgfpathlineto{\pgfqpoint{5.675145in}{3.365797in}}%
\pgfpathlineto{\pgfqpoint{5.667828in}{3.355775in}}%
\pgfpathlineto{\pgfqpoint{5.660516in}{3.346035in}}%
\pgfpathlineto{\pgfqpoint{5.653207in}{3.336571in}}%
\pgfpathclose%
\pgfusepath{fill}%
\end{pgfscope}%
\begin{pgfscope}%
\pgfpathrectangle{\pgfqpoint{1.150000in}{0.150000in}}{\pgfqpoint{5.700000in}{5.700000in}}%
\pgfusepath{clip}%
\pgfsetbuttcap%
\pgfsetroundjoin%
\definecolor{currentfill}{rgb}{0.282327,0.094955,0.417331}%
\pgfsetfillcolor{currentfill}%
\pgfsetfillopacity{0.800000}%
\pgfsetlinewidth{0.000000pt}%
\definecolor{currentstroke}{rgb}{0.000000,0.000000,0.000000}%
\pgfsetstrokecolor{currentstroke}%
\pgfsetdash{}{0pt}%
\pgfpathmoveto{\pgfqpoint{3.373646in}{2.238596in}}%
\pgfpathlineto{\pgfqpoint{3.387107in}{2.231314in}}%
\pgfpathlineto{\pgfqpoint{3.400569in}{2.224263in}}%
\pgfpathlineto{\pgfqpoint{3.414032in}{2.217442in}}%
\pgfpathlineto{\pgfqpoint{3.427497in}{2.210851in}}%
\pgfpathlineto{\pgfqpoint{3.435548in}{2.220815in}}%
\pgfpathlineto{\pgfqpoint{3.443592in}{2.230824in}}%
\pgfpathlineto{\pgfqpoint{3.451631in}{2.240879in}}%
\pgfpathlineto{\pgfqpoint{3.459664in}{2.250978in}}%
\pgfpathlineto{\pgfqpoint{3.446211in}{2.257511in}}%
\pgfpathlineto{\pgfqpoint{3.432760in}{2.264274in}}%
\pgfpathlineto{\pgfqpoint{3.419311in}{2.271266in}}%
\pgfpathlineto{\pgfqpoint{3.405864in}{2.278489in}}%
\pgfpathlineto{\pgfqpoint{3.397818in}{2.268436in}}%
\pgfpathlineto{\pgfqpoint{3.389767in}{2.258436in}}%
\pgfpathlineto{\pgfqpoint{3.381710in}{2.248489in}}%
\pgfpathlineto{\pgfqpoint{3.373646in}{2.238596in}}%
\pgfpathclose%
\pgfusepath{fill}%
\end{pgfscope}%
\begin{pgfscope}%
\pgfpathrectangle{\pgfqpoint{1.150000in}{0.150000in}}{\pgfqpoint{5.700000in}{5.700000in}}%
\pgfusepath{clip}%
\pgfsetbuttcap%
\pgfsetroundjoin%
\definecolor{currentfill}{rgb}{0.214298,0.355619,0.551184}%
\pgfsetfillcolor{currentfill}%
\pgfsetfillopacity{0.800000}%
\pgfsetlinewidth{0.000000pt}%
\definecolor{currentstroke}{rgb}{0.000000,0.000000,0.000000}%
\pgfsetstrokecolor{currentstroke}%
\pgfsetdash{}{0pt}%
\pgfpathmoveto{\pgfqpoint{4.851932in}{2.828998in}}%
\pgfpathlineto{\pgfqpoint{4.865794in}{2.833266in}}%
\pgfpathlineto{\pgfqpoint{4.879669in}{2.837715in}}%
\pgfpathlineto{\pgfqpoint{4.893557in}{2.842344in}}%
\pgfpathlineto{\pgfqpoint{4.907459in}{2.847154in}}%
\pgfpathlineto{\pgfqpoint{4.915004in}{2.855316in}}%
\pgfpathlineto{\pgfqpoint{4.922544in}{2.863541in}}%
\pgfpathlineto{\pgfqpoint{4.930080in}{2.871832in}}%
\pgfpathlineto{\pgfqpoint{4.937612in}{2.880196in}}%
\pgfpathlineto{\pgfqpoint{4.923726in}{2.875841in}}%
\pgfpathlineto{\pgfqpoint{4.909853in}{2.871666in}}%
\pgfpathlineto{\pgfqpoint{4.895993in}{2.867670in}}%
\pgfpathlineto{\pgfqpoint{4.882145in}{2.863854in}}%
\pgfpathlineto{\pgfqpoint{4.874598in}{2.855026in}}%
\pgfpathlineto{\pgfqpoint{4.867047in}{2.846277in}}%
\pgfpathlineto{\pgfqpoint{4.859491in}{2.837603in}}%
\pgfpathlineto{\pgfqpoint{4.851932in}{2.828998in}}%
\pgfpathclose%
\pgfusepath{fill}%
\end{pgfscope}%
\begin{pgfscope}%
\pgfpathrectangle{\pgfqpoint{1.150000in}{0.150000in}}{\pgfqpoint{5.700000in}{5.700000in}}%
\pgfusepath{clip}%
\pgfsetbuttcap%
\pgfsetroundjoin%
\definecolor{currentfill}{rgb}{0.279574,0.170599,0.479997}%
\pgfsetfillcolor{currentfill}%
\pgfsetfillopacity{0.800000}%
\pgfsetlinewidth{0.000000pt}%
\definecolor{currentstroke}{rgb}{0.000000,0.000000,0.000000}%
\pgfsetstrokecolor{currentstroke}%
\pgfsetdash{}{0pt}%
\pgfpathmoveto{\pgfqpoint{2.984908in}{2.414711in}}%
\pgfpathlineto{\pgfqpoint{2.998439in}{2.401184in}}%
\pgfpathlineto{\pgfqpoint{3.011966in}{2.387930in}}%
\pgfpathlineto{\pgfqpoint{3.025489in}{2.374946in}}%
\pgfpathlineto{\pgfqpoint{3.039008in}{2.362231in}}%
\pgfpathlineto{\pgfqpoint{3.047206in}{2.371009in}}%
\pgfpathlineto{\pgfqpoint{3.055397in}{2.379884in}}%
\pgfpathlineto{\pgfqpoint{3.063580in}{2.388855in}}%
\pgfpathlineto{\pgfqpoint{3.071756in}{2.397922in}}%
\pgfpathlineto{\pgfqpoint{3.058256in}{2.410513in}}%
\pgfpathlineto{\pgfqpoint{3.044753in}{2.423372in}}%
\pgfpathlineto{\pgfqpoint{3.031245in}{2.436500in}}%
\pgfpathlineto{\pgfqpoint{3.017733in}{2.449901in}}%
\pgfpathlineto{\pgfqpoint{3.009539in}{2.440948in}}%
\pgfpathlineto{\pgfqpoint{3.001337in}{2.432098in}}%
\pgfpathlineto{\pgfqpoint{2.993126in}{2.423352in}}%
\pgfpathlineto{\pgfqpoint{2.984908in}{2.414711in}}%
\pgfpathclose%
\pgfusepath{fill}%
\end{pgfscope}%
\begin{pgfscope}%
\pgfpathrectangle{\pgfqpoint{1.150000in}{0.150000in}}{\pgfqpoint{5.700000in}{5.700000in}}%
\pgfusepath{clip}%
\pgfsetbuttcap%
\pgfsetroundjoin%
\definecolor{currentfill}{rgb}{0.283187,0.125848,0.444960}%
\pgfsetfillcolor{currentfill}%
\pgfsetfillopacity{0.800000}%
\pgfsetlinewidth{0.000000pt}%
\definecolor{currentstroke}{rgb}{0.000000,0.000000,0.000000}%
\pgfsetstrokecolor{currentstroke}%
\pgfsetdash{}{0pt}%
\pgfpathmoveto{\pgfqpoint{3.824737in}{2.291109in}}%
\pgfpathlineto{\pgfqpoint{3.838245in}{2.289110in}}%
\pgfpathlineto{\pgfqpoint{3.851758in}{2.287317in}}%
\pgfpathlineto{\pgfqpoint{3.865278in}{2.285729in}}%
\pgfpathlineto{\pgfqpoint{3.878804in}{2.284345in}}%
\pgfpathlineto{\pgfqpoint{3.886707in}{2.294726in}}%
\pgfpathlineto{\pgfqpoint{3.894606in}{2.305113in}}%
\pgfpathlineto{\pgfqpoint{3.902499in}{2.315506in}}%
\pgfpathlineto{\pgfqpoint{3.910387in}{2.325908in}}%
\pgfpathlineto{\pgfqpoint{3.896869in}{2.327360in}}%
\pgfpathlineto{\pgfqpoint{3.883358in}{2.329017in}}%
\pgfpathlineto{\pgfqpoint{3.869853in}{2.330878in}}%
\pgfpathlineto{\pgfqpoint{3.856354in}{2.332946in}}%
\pgfpathlineto{\pgfqpoint{3.848458in}{2.322464in}}%
\pgfpathlineto{\pgfqpoint{3.840556in}{2.311998in}}%
\pgfpathlineto{\pgfqpoint{3.832649in}{2.301547in}}%
\pgfpathlineto{\pgfqpoint{3.824737in}{2.291109in}}%
\pgfpathclose%
\pgfusepath{fill}%
\end{pgfscope}%
\begin{pgfscope}%
\pgfpathrectangle{\pgfqpoint{1.150000in}{0.150000in}}{\pgfqpoint{5.700000in}{5.700000in}}%
\pgfusepath{clip}%
\pgfsetbuttcap%
\pgfsetroundjoin%
\definecolor{currentfill}{rgb}{0.283091,0.110553,0.431554}%
\pgfsetfillcolor{currentfill}%
\pgfsetfillopacity{0.800000}%
\pgfsetlinewidth{0.000000pt}%
\definecolor{currentstroke}{rgb}{0.000000,0.000000,0.000000}%
\pgfsetstrokecolor{currentstroke}%
\pgfsetdash{}{0pt}%
\pgfpathmoveto{\pgfqpoint{3.233551in}{2.267044in}}%
\pgfpathlineto{\pgfqpoint{3.247024in}{2.257768in}}%
\pgfpathlineto{\pgfqpoint{3.260497in}{2.248735in}}%
\pgfpathlineto{\pgfqpoint{3.273969in}{2.239944in}}%
\pgfpathlineto{\pgfqpoint{3.287440in}{2.231392in}}%
\pgfpathlineto{\pgfqpoint{3.295543in}{2.240968in}}%
\pgfpathlineto{\pgfqpoint{3.303639in}{2.250606in}}%
\pgfpathlineto{\pgfqpoint{3.311729in}{2.260306in}}%
\pgfpathlineto{\pgfqpoint{3.319812in}{2.270068in}}%
\pgfpathlineto{\pgfqpoint{3.306355in}{2.278528in}}%
\pgfpathlineto{\pgfqpoint{3.292898in}{2.287229in}}%
\pgfpathlineto{\pgfqpoint{3.279440in}{2.296171in}}%
\pgfpathlineto{\pgfqpoint{3.265983in}{2.305356in}}%
\pgfpathlineto{\pgfqpoint{3.257885in}{2.295673in}}%
\pgfpathlineto{\pgfqpoint{3.249780in}{2.286060in}}%
\pgfpathlineto{\pgfqpoint{3.241669in}{2.276517in}}%
\pgfpathlineto{\pgfqpoint{3.233551in}{2.267044in}}%
\pgfpathclose%
\pgfusepath{fill}%
\end{pgfscope}%
\begin{pgfscope}%
\pgfpathrectangle{\pgfqpoint{1.150000in}{0.150000in}}{\pgfqpoint{5.700000in}{5.700000in}}%
\pgfusepath{clip}%
\pgfsetbuttcap%
\pgfsetroundjoin%
\definecolor{currentfill}{rgb}{0.206756,0.371758,0.553117}%
\pgfsetfillcolor{currentfill}%
\pgfsetfillopacity{0.800000}%
\pgfsetlinewidth{0.000000pt}%
\definecolor{currentstroke}{rgb}{0.000000,0.000000,0.000000}%
\pgfsetstrokecolor{currentstroke}%
\pgfsetdash{}{0pt}%
\pgfpathmoveto{\pgfqpoint{4.937612in}{2.880196in}}%
\pgfpathlineto{\pgfqpoint{4.951512in}{2.884730in}}%
\pgfpathlineto{\pgfqpoint{4.965425in}{2.889443in}}%
\pgfpathlineto{\pgfqpoint{4.979352in}{2.894335in}}%
\pgfpathlineto{\pgfqpoint{4.993293in}{2.899405in}}%
\pgfpathlineto{\pgfqpoint{5.000805in}{2.907371in}}%
\pgfpathlineto{\pgfqpoint{5.008312in}{2.915413in}}%
\pgfpathlineto{\pgfqpoint{5.015816in}{2.923535in}}%
\pgfpathlineto{\pgfqpoint{5.023316in}{2.931744in}}%
\pgfpathlineto{\pgfqpoint{5.009392in}{2.927161in}}%
\pgfpathlineto{\pgfqpoint{4.995482in}{2.922755in}}%
\pgfpathlineto{\pgfqpoint{4.981586in}{2.918528in}}%
\pgfpathlineto{\pgfqpoint{4.967702in}{2.914479in}}%
\pgfpathlineto{\pgfqpoint{4.960185in}{2.905773in}}%
\pgfpathlineto{\pgfqpoint{4.952665in}{2.897161in}}%
\pgfpathlineto{\pgfqpoint{4.945140in}{2.888637in}}%
\pgfpathlineto{\pgfqpoint{4.937612in}{2.880196in}}%
\pgfpathclose%
\pgfusepath{fill}%
\end{pgfscope}%
\begin{pgfscope}%
\pgfpathrectangle{\pgfqpoint{1.150000in}{0.150000in}}{\pgfqpoint{5.700000in}{5.700000in}}%
\pgfusepath{clip}%
\pgfsetbuttcap%
\pgfsetroundjoin%
\definecolor{currentfill}{rgb}{0.237441,0.305202,0.541921}%
\pgfsetfillcolor{currentfill}%
\pgfsetfillopacity{0.800000}%
\pgfsetlinewidth{0.000000pt}%
\definecolor{currentstroke}{rgb}{0.000000,0.000000,0.000000}%
\pgfsetstrokecolor{currentstroke}%
\pgfsetdash{}{0pt}%
\pgfpathmoveto{\pgfqpoint{2.712944in}{2.746007in}}%
\pgfpathlineto{\pgfqpoint{2.726621in}{2.726531in}}%
\pgfpathlineto{\pgfqpoint{2.740287in}{2.707378in}}%
\pgfpathlineto{\pgfqpoint{2.753944in}{2.688546in}}%
\pgfpathlineto{\pgfqpoint{2.767591in}{2.670030in}}%
\pgfpathlineto{\pgfqpoint{2.775898in}{2.678112in}}%
\pgfpathlineto{\pgfqpoint{2.784196in}{2.686330in}}%
\pgfpathlineto{\pgfqpoint{2.792484in}{2.694685in}}%
\pgfpathlineto{\pgfqpoint{2.800762in}{2.703175in}}%
\pgfpathlineto{\pgfqpoint{2.787139in}{2.721560in}}%
\pgfpathlineto{\pgfqpoint{2.773506in}{2.740262in}}%
\pgfpathlineto{\pgfqpoint{2.759864in}{2.759283in}}%
\pgfpathlineto{\pgfqpoint{2.746212in}{2.778628in}}%
\pgfpathlineto{\pgfqpoint{2.737910in}{2.770256in}}%
\pgfpathlineto{\pgfqpoint{2.729598in}{2.762028in}}%
\pgfpathlineto{\pgfqpoint{2.721276in}{2.753945in}}%
\pgfpathlineto{\pgfqpoint{2.712944in}{2.746007in}}%
\pgfpathclose%
\pgfusepath{fill}%
\end{pgfscope}%
\begin{pgfscope}%
\pgfpathrectangle{\pgfqpoint{1.150000in}{0.150000in}}{\pgfqpoint{5.700000in}{5.700000in}}%
\pgfusepath{clip}%
\pgfsetbuttcap%
\pgfsetroundjoin%
\definecolor{currentfill}{rgb}{0.282327,0.094955,0.417331}%
\pgfsetfillcolor{currentfill}%
\pgfsetfillopacity{0.800000}%
\pgfsetlinewidth{0.000000pt}%
\definecolor{currentstroke}{rgb}{0.000000,0.000000,0.000000}%
\pgfsetstrokecolor{currentstroke}%
\pgfsetdash{}{0pt}%
\pgfpathmoveto{\pgfqpoint{3.513497in}{2.227109in}}%
\pgfpathlineto{\pgfqpoint{3.526962in}{2.221702in}}%
\pgfpathlineto{\pgfqpoint{3.540429in}{2.216517in}}%
\pgfpathlineto{\pgfqpoint{3.553900in}{2.211552in}}%
\pgfpathlineto{\pgfqpoint{3.567374in}{2.206806in}}%
\pgfpathlineto{\pgfqpoint{3.575379in}{2.217028in}}%
\pgfpathlineto{\pgfqpoint{3.583377in}{2.227279in}}%
\pgfpathlineto{\pgfqpoint{3.591370in}{2.237560in}}%
\pgfpathlineto{\pgfqpoint{3.599358in}{2.247871in}}%
\pgfpathlineto{\pgfqpoint{3.585895in}{2.252590in}}%
\pgfpathlineto{\pgfqpoint{3.572435in}{2.257529in}}%
\pgfpathlineto{\pgfqpoint{3.558978in}{2.262688in}}%
\pgfpathlineto{\pgfqpoint{3.545525in}{2.268068in}}%
\pgfpathlineto{\pgfqpoint{3.537526in}{2.257772in}}%
\pgfpathlineto{\pgfqpoint{3.529522in}{2.247514in}}%
\pgfpathlineto{\pgfqpoint{3.521512in}{2.237293in}}%
\pgfpathlineto{\pgfqpoint{3.513497in}{2.227109in}}%
\pgfpathclose%
\pgfusepath{fill}%
\end{pgfscope}%
\begin{pgfscope}%
\pgfpathrectangle{\pgfqpoint{1.150000in}{0.150000in}}{\pgfqpoint{5.700000in}{5.700000in}}%
\pgfusepath{clip}%
\pgfsetbuttcap%
\pgfsetroundjoin%
\definecolor{currentfill}{rgb}{0.197636,0.391528,0.554969}%
\pgfsetfillcolor{currentfill}%
\pgfsetfillopacity{0.800000}%
\pgfsetlinewidth{0.000000pt}%
\definecolor{currentstroke}{rgb}{0.000000,0.000000,0.000000}%
\pgfsetstrokecolor{currentstroke}%
\pgfsetdash{}{0pt}%
\pgfpathmoveto{\pgfqpoint{5.023316in}{2.931744in}}%
\pgfpathlineto{\pgfqpoint{5.037254in}{2.936506in}}%
\pgfpathlineto{\pgfqpoint{5.051205in}{2.941445in}}%
\pgfpathlineto{\pgfqpoint{5.065171in}{2.946562in}}%
\pgfpathlineto{\pgfqpoint{5.079150in}{2.951856in}}%
\pgfpathlineto{\pgfqpoint{5.086629in}{2.959650in}}%
\pgfpathlineto{\pgfqpoint{5.094104in}{2.967533in}}%
\pgfpathlineto{\pgfqpoint{5.101576in}{2.975513in}}%
\pgfpathlineto{\pgfqpoint{5.109044in}{2.983594in}}%
\pgfpathlineto{\pgfqpoint{5.095083in}{2.978820in}}%
\pgfpathlineto{\pgfqpoint{5.081136in}{2.974222in}}%
\pgfpathlineto{\pgfqpoint{5.067202in}{2.969801in}}%
\pgfpathlineto{\pgfqpoint{5.053282in}{2.965557in}}%
\pgfpathlineto{\pgfqpoint{5.045796in}{2.956945in}}%
\pgfpathlineto{\pgfqpoint{5.038306in}{2.948443in}}%
\pgfpathlineto{\pgfqpoint{5.030813in}{2.940045in}}%
\pgfpathlineto{\pgfqpoint{5.023316in}{2.931744in}}%
\pgfpathclose%
\pgfusepath{fill}%
\end{pgfscope}%
\begin{pgfscope}%
\pgfpathrectangle{\pgfqpoint{1.150000in}{0.150000in}}{\pgfqpoint{5.700000in}{5.700000in}}%
\pgfusepath{clip}%
\pgfsetbuttcap%
\pgfsetroundjoin%
\definecolor{currentfill}{rgb}{0.281887,0.150881,0.465405}%
\pgfsetfillcolor{currentfill}%
\pgfsetfillopacity{0.800000}%
\pgfsetlinewidth{0.000000pt}%
\definecolor{currentstroke}{rgb}{0.000000,0.000000,0.000000}%
\pgfsetstrokecolor{currentstroke}%
\pgfsetdash{}{0pt}%
\pgfpathmoveto{\pgfqpoint{3.039008in}{2.362231in}}%
\pgfpathlineto{\pgfqpoint{3.052523in}{2.349781in}}%
\pgfpathlineto{\pgfqpoint{3.066034in}{2.337596in}}%
\pgfpathlineto{\pgfqpoint{3.079543in}{2.325673in}}%
\pgfpathlineto{\pgfqpoint{3.093048in}{2.314011in}}%
\pgfpathlineto{\pgfqpoint{3.101228in}{2.322925in}}%
\pgfpathlineto{\pgfqpoint{3.109401in}{2.331928in}}%
\pgfpathlineto{\pgfqpoint{3.117566in}{2.341020in}}%
\pgfpathlineto{\pgfqpoint{3.125723in}{2.350199in}}%
\pgfpathlineto{\pgfqpoint{3.112236in}{2.361738in}}%
\pgfpathlineto{\pgfqpoint{3.098746in}{2.373536in}}%
\pgfpathlineto{\pgfqpoint{3.085253in}{2.385597in}}%
\pgfpathlineto{\pgfqpoint{3.071756in}{2.397922in}}%
\pgfpathlineto{\pgfqpoint{3.063580in}{2.388855in}}%
\pgfpathlineto{\pgfqpoint{3.055397in}{2.379884in}}%
\pgfpathlineto{\pgfqpoint{3.047206in}{2.371009in}}%
\pgfpathlineto{\pgfqpoint{3.039008in}{2.362231in}}%
\pgfpathclose%
\pgfusepath{fill}%
\end{pgfscope}%
\begin{pgfscope}%
\pgfpathrectangle{\pgfqpoint{1.150000in}{0.150000in}}{\pgfqpoint{5.700000in}{5.700000in}}%
\pgfusepath{clip}%
\pgfsetbuttcap%
\pgfsetroundjoin%
\definecolor{currentfill}{rgb}{0.283197,0.115680,0.436115}%
\pgfsetfillcolor{currentfill}%
\pgfsetfillopacity{0.800000}%
\pgfsetlinewidth{0.000000pt}%
\definecolor{currentstroke}{rgb}{0.000000,0.000000,0.000000}%
\pgfsetstrokecolor{currentstroke}%
\pgfsetdash{}{0pt}%
\pgfpathmoveto{\pgfqpoint{3.739031in}{2.259403in}}%
\pgfpathlineto{\pgfqpoint{3.752525in}{2.256608in}}%
\pgfpathlineto{\pgfqpoint{3.766024in}{2.254022in}}%
\pgfpathlineto{\pgfqpoint{3.779528in}{2.251644in}}%
\pgfpathlineto{\pgfqpoint{3.793038in}{2.249475in}}%
\pgfpathlineto{\pgfqpoint{3.800971in}{2.259868in}}%
\pgfpathlineto{\pgfqpoint{3.808898in}{2.270271in}}%
\pgfpathlineto{\pgfqpoint{3.816820in}{2.280684in}}%
\pgfpathlineto{\pgfqpoint{3.824737in}{2.291109in}}%
\pgfpathlineto{\pgfqpoint{3.811236in}{2.293316in}}%
\pgfpathlineto{\pgfqpoint{3.797740in}{2.295730in}}%
\pgfpathlineto{\pgfqpoint{3.784250in}{2.298353in}}%
\pgfpathlineto{\pgfqpoint{3.770766in}{2.301185in}}%
\pgfpathlineto{\pgfqpoint{3.762840in}{2.290712in}}%
\pgfpathlineto{\pgfqpoint{3.754909in}{2.280257in}}%
\pgfpathlineto{\pgfqpoint{3.746973in}{2.269821in}}%
\pgfpathlineto{\pgfqpoint{3.739031in}{2.259403in}}%
\pgfpathclose%
\pgfusepath{fill}%
\end{pgfscope}%
\begin{pgfscope}%
\pgfpathrectangle{\pgfqpoint{1.150000in}{0.150000in}}{\pgfqpoint{5.700000in}{5.700000in}}%
\pgfusepath{clip}%
\pgfsetbuttcap%
\pgfsetroundjoin%
\definecolor{currentfill}{rgb}{0.188923,0.410910,0.556326}%
\pgfsetfillcolor{currentfill}%
\pgfsetfillopacity{0.800000}%
\pgfsetlinewidth{0.000000pt}%
\definecolor{currentstroke}{rgb}{0.000000,0.000000,0.000000}%
\pgfsetstrokecolor{currentstroke}%
\pgfsetdash{}{0pt}%
\pgfpathmoveto{\pgfqpoint{5.109044in}{2.983594in}}%
\pgfpathlineto{\pgfqpoint{5.123020in}{2.988546in}}%
\pgfpathlineto{\pgfqpoint{5.137009in}{2.993674in}}%
\pgfpathlineto{\pgfqpoint{5.151013in}{2.998978in}}%
\pgfpathlineto{\pgfqpoint{5.165032in}{3.004459in}}%
\pgfpathlineto{\pgfqpoint{5.172478in}{3.012109in}}%
\pgfpathlineto{\pgfqpoint{5.179920in}{3.019865in}}%
\pgfpathlineto{\pgfqpoint{5.187360in}{3.027734in}}%
\pgfpathlineto{\pgfqpoint{5.194797in}{3.035722in}}%
\pgfpathlineto{\pgfqpoint{5.180798in}{3.030793in}}%
\pgfpathlineto{\pgfqpoint{5.166814in}{3.026040in}}%
\pgfpathlineto{\pgfqpoint{5.152844in}{3.021463in}}%
\pgfpathlineto{\pgfqpoint{5.138888in}{3.017061in}}%
\pgfpathlineto{\pgfqpoint{5.131431in}{3.008511in}}%
\pgfpathlineto{\pgfqpoint{5.123972in}{3.000087in}}%
\pgfpathlineto{\pgfqpoint{5.116509in}{2.991784in}}%
\pgfpathlineto{\pgfqpoint{5.109044in}{2.983594in}}%
\pgfpathclose%
\pgfusepath{fill}%
\end{pgfscope}%
\begin{pgfscope}%
\pgfpathrectangle{\pgfqpoint{1.150000in}{0.150000in}}{\pgfqpoint{5.700000in}{5.700000in}}%
\pgfusepath{clip}%
\pgfsetbuttcap%
\pgfsetroundjoin%
\definecolor{currentfill}{rgb}{0.221989,0.339161,0.548752}%
\pgfsetfillcolor{currentfill}%
\pgfsetfillopacity{0.800000}%
\pgfsetlinewidth{0.000000pt}%
\definecolor{currentstroke}{rgb}{0.000000,0.000000,0.000000}%
\pgfsetstrokecolor{currentstroke}%
\pgfsetdash{}{0pt}%
\pgfpathmoveto{\pgfqpoint{2.658131in}{2.827204in}}%
\pgfpathlineto{\pgfqpoint{2.671851in}{2.806404in}}%
\pgfpathlineto{\pgfqpoint{2.685560in}{2.785940in}}%
\pgfpathlineto{\pgfqpoint{2.699257in}{2.765809in}}%
\pgfpathlineto{\pgfqpoint{2.712944in}{2.746007in}}%
\pgfpathlineto{\pgfqpoint{2.721276in}{2.753945in}}%
\pgfpathlineto{\pgfqpoint{2.729598in}{2.762028in}}%
\pgfpathlineto{\pgfqpoint{2.737910in}{2.770256in}}%
\pgfpathlineto{\pgfqpoint{2.746212in}{2.778628in}}%
\pgfpathlineto{\pgfqpoint{2.732550in}{2.798298in}}%
\pgfpathlineto{\pgfqpoint{2.718877in}{2.818297in}}%
\pgfpathlineto{\pgfqpoint{2.705194in}{2.838629in}}%
\pgfpathlineto{\pgfqpoint{2.691500in}{2.859296in}}%
\pgfpathlineto{\pgfqpoint{2.683173in}{2.851044in}}%
\pgfpathlineto{\pgfqpoint{2.674836in}{2.842945in}}%
\pgfpathlineto{\pgfqpoint{2.666489in}{2.834998in}}%
\pgfpathlineto{\pgfqpoint{2.658131in}{2.827204in}}%
\pgfpathclose%
\pgfusepath{fill}%
\end{pgfscope}%
\begin{pgfscope}%
\pgfpathrectangle{\pgfqpoint{1.150000in}{0.150000in}}{\pgfqpoint{5.700000in}{5.700000in}}%
\pgfusepath{clip}%
\pgfsetbuttcap%
\pgfsetroundjoin%
\definecolor{currentfill}{rgb}{0.180629,0.429975,0.557282}%
\pgfsetfillcolor{currentfill}%
\pgfsetfillopacity{0.800000}%
\pgfsetlinewidth{0.000000pt}%
\definecolor{currentstroke}{rgb}{0.000000,0.000000,0.000000}%
\pgfsetstrokecolor{currentstroke}%
\pgfsetdash{}{0pt}%
\pgfpathmoveto{\pgfqpoint{5.194797in}{3.035722in}}%
\pgfpathlineto{\pgfqpoint{5.208810in}{3.040826in}}%
\pgfpathlineto{\pgfqpoint{5.222838in}{3.046106in}}%
\pgfpathlineto{\pgfqpoint{5.236880in}{3.051560in}}%
\pgfpathlineto{\pgfqpoint{5.250937in}{3.057191in}}%
\pgfpathlineto{\pgfqpoint{5.258350in}{3.064731in}}%
\pgfpathlineto{\pgfqpoint{5.265761in}{3.072396in}}%
\pgfpathlineto{\pgfqpoint{5.273169in}{3.080192in}}%
\pgfpathlineto{\pgfqpoint{5.280576in}{3.088125in}}%
\pgfpathlineto{\pgfqpoint{5.266540in}{3.083079in}}%
\pgfpathlineto{\pgfqpoint{5.252519in}{3.078208in}}%
\pgfpathlineto{\pgfqpoint{5.238513in}{3.073512in}}%
\pgfpathlineto{\pgfqpoint{5.224521in}{3.068990in}}%
\pgfpathlineto{\pgfqpoint{5.217093in}{3.060462in}}%
\pgfpathlineto{\pgfqpoint{5.209663in}{3.052079in}}%
\pgfpathlineto{\pgfqpoint{5.202231in}{3.043835in}}%
\pgfpathlineto{\pgfqpoint{5.194797in}{3.035722in}}%
\pgfpathclose%
\pgfusepath{fill}%
\end{pgfscope}%
\begin{pgfscope}%
\pgfpathrectangle{\pgfqpoint{1.150000in}{0.150000in}}{\pgfqpoint{5.700000in}{5.700000in}}%
\pgfusepath{clip}%
\pgfsetbuttcap%
\pgfsetroundjoin%
\definecolor{currentfill}{rgb}{0.283072,0.130895,0.449241}%
\pgfsetfillcolor{currentfill}%
\pgfsetfillopacity{0.800000}%
\pgfsetlinewidth{0.000000pt}%
\definecolor{currentstroke}{rgb}{0.000000,0.000000,0.000000}%
\pgfsetstrokecolor{currentstroke}%
\pgfsetdash{}{0pt}%
\pgfpathmoveto{\pgfqpoint{3.093048in}{2.314011in}}%
\pgfpathlineto{\pgfqpoint{3.106551in}{2.302607in}}%
\pgfpathlineto{\pgfqpoint{3.120051in}{2.291460in}}%
\pgfpathlineto{\pgfqpoint{3.133549in}{2.280567in}}%
\pgfpathlineto{\pgfqpoint{3.147045in}{2.269928in}}%
\pgfpathlineto{\pgfqpoint{3.155207in}{2.278978in}}%
\pgfpathlineto{\pgfqpoint{3.163362in}{2.288109in}}%
\pgfpathlineto{\pgfqpoint{3.171510in}{2.297320in}}%
\pgfpathlineto{\pgfqpoint{3.179650in}{2.306611in}}%
\pgfpathlineto{\pgfqpoint{3.166172in}{2.317127in}}%
\pgfpathlineto{\pgfqpoint{3.152691in}{2.327896in}}%
\pgfpathlineto{\pgfqpoint{3.139208in}{2.338919in}}%
\pgfpathlineto{\pgfqpoint{3.125723in}{2.350199in}}%
\pgfpathlineto{\pgfqpoint{3.117566in}{2.341020in}}%
\pgfpathlineto{\pgfqpoint{3.109401in}{2.331928in}}%
\pgfpathlineto{\pgfqpoint{3.101228in}{2.322925in}}%
\pgfpathlineto{\pgfqpoint{3.093048in}{2.314011in}}%
\pgfpathclose%
\pgfusepath{fill}%
\end{pgfscope}%
\begin{pgfscope}%
\pgfpathrectangle{\pgfqpoint{1.150000in}{0.150000in}}{\pgfqpoint{5.700000in}{5.700000in}}%
\pgfusepath{clip}%
\pgfsetbuttcap%
\pgfsetroundjoin%
\definecolor{currentfill}{rgb}{0.172719,0.448791,0.557885}%
\pgfsetfillcolor{currentfill}%
\pgfsetfillopacity{0.800000}%
\pgfsetlinewidth{0.000000pt}%
\definecolor{currentstroke}{rgb}{0.000000,0.000000,0.000000}%
\pgfsetstrokecolor{currentstroke}%
\pgfsetdash{}{0pt}%
\pgfpathmoveto{\pgfqpoint{5.280576in}{3.088125in}}%
\pgfpathlineto{\pgfqpoint{5.294626in}{3.093345in}}%
\pgfpathlineto{\pgfqpoint{5.308691in}{3.098739in}}%
\pgfpathlineto{\pgfqpoint{5.322771in}{3.104307in}}%
\pgfpathlineto{\pgfqpoint{5.336867in}{3.110050in}}%
\pgfpathlineto{\pgfqpoint{5.344248in}{3.117522in}}%
\pgfpathlineto{\pgfqpoint{5.351628in}{3.125137in}}%
\pgfpathlineto{\pgfqpoint{5.359006in}{3.132903in}}%
\pgfpathlineto{\pgfqpoint{5.366382in}{3.140825in}}%
\pgfpathlineto{\pgfqpoint{5.352310in}{3.135700in}}%
\pgfpathlineto{\pgfqpoint{5.338253in}{3.130748in}}%
\pgfpathlineto{\pgfqpoint{5.324211in}{3.125969in}}%
\pgfpathlineto{\pgfqpoint{5.310183in}{3.121364in}}%
\pgfpathlineto{\pgfqpoint{5.302783in}{3.112815in}}%
\pgfpathlineto{\pgfqpoint{5.295382in}{3.104429in}}%
\pgfpathlineto{\pgfqpoint{5.287980in}{3.096202in}}%
\pgfpathlineto{\pgfqpoint{5.280576in}{3.088125in}}%
\pgfpathclose%
\pgfusepath{fill}%
\end{pgfscope}%
\begin{pgfscope}%
\pgfpathrectangle{\pgfqpoint{1.150000in}{0.150000in}}{\pgfqpoint{5.700000in}{5.700000in}}%
\pgfusepath{clip}%
\pgfsetbuttcap%
\pgfsetroundjoin%
\definecolor{currentfill}{rgb}{0.282656,0.100196,0.422160}%
\pgfsetfillcolor{currentfill}%
\pgfsetfillopacity{0.800000}%
\pgfsetlinewidth{0.000000pt}%
\definecolor{currentstroke}{rgb}{0.000000,0.000000,0.000000}%
\pgfsetstrokecolor{currentstroke}%
\pgfsetdash{}{0pt}%
\pgfpathmoveto{\pgfqpoint{3.653251in}{2.231167in}}%
\pgfpathlineto{\pgfqpoint{3.666734in}{2.227530in}}%
\pgfpathlineto{\pgfqpoint{3.680222in}{2.224105in}}%
\pgfpathlineto{\pgfqpoint{3.693715in}{2.220894in}}%
\pgfpathlineto{\pgfqpoint{3.707213in}{2.217893in}}%
\pgfpathlineto{\pgfqpoint{3.715175in}{2.228247in}}%
\pgfpathlineto{\pgfqpoint{3.723132in}{2.238617in}}%
\pgfpathlineto{\pgfqpoint{3.731084in}{2.249002in}}%
\pgfpathlineto{\pgfqpoint{3.739031in}{2.259403in}}%
\pgfpathlineto{\pgfqpoint{3.725543in}{2.262409in}}%
\pgfpathlineto{\pgfqpoint{3.712059in}{2.265627in}}%
\pgfpathlineto{\pgfqpoint{3.698581in}{2.269056in}}%
\pgfpathlineto{\pgfqpoint{3.685107in}{2.272699in}}%
\pgfpathlineto{\pgfqpoint{3.677151in}{2.262281in}}%
\pgfpathlineto{\pgfqpoint{3.669189in}{2.251886in}}%
\pgfpathlineto{\pgfqpoint{3.661223in}{2.241515in}}%
\pgfpathlineto{\pgfqpoint{3.653251in}{2.231167in}}%
\pgfpathclose%
\pgfusepath{fill}%
\end{pgfscope}%
\begin{pgfscope}%
\pgfpathrectangle{\pgfqpoint{1.150000in}{0.150000in}}{\pgfqpoint{5.700000in}{5.700000in}}%
\pgfusepath{clip}%
\pgfsetbuttcap%
\pgfsetroundjoin%
\definecolor{currentfill}{rgb}{0.282656,0.100196,0.422160}%
\pgfsetfillcolor{currentfill}%
\pgfsetfillopacity{0.800000}%
\pgfsetlinewidth{0.000000pt}%
\definecolor{currentstroke}{rgb}{0.000000,0.000000,0.000000}%
\pgfsetstrokecolor{currentstroke}%
\pgfsetdash{}{0pt}%
\pgfpathmoveto{\pgfqpoint{3.287440in}{2.231392in}}%
\pgfpathlineto{\pgfqpoint{3.300912in}{2.223080in}}%
\pgfpathlineto{\pgfqpoint{3.314385in}{2.215005in}}%
\pgfpathlineto{\pgfqpoint{3.327857in}{2.207165in}}%
\pgfpathlineto{\pgfqpoint{3.341331in}{2.199560in}}%
\pgfpathlineto{\pgfqpoint{3.349419in}{2.209237in}}%
\pgfpathlineto{\pgfqpoint{3.357501in}{2.218969in}}%
\pgfpathlineto{\pgfqpoint{3.365577in}{2.228756in}}%
\pgfpathlineto{\pgfqpoint{3.373646in}{2.238596in}}%
\pgfpathlineto{\pgfqpoint{3.360187in}{2.246111in}}%
\pgfpathlineto{\pgfqpoint{3.346728in}{2.253860in}}%
\pgfpathlineto{\pgfqpoint{3.333270in}{2.261845in}}%
\pgfpathlineto{\pgfqpoint{3.319812in}{2.270068in}}%
\pgfpathlineto{\pgfqpoint{3.311729in}{2.260306in}}%
\pgfpathlineto{\pgfqpoint{3.303639in}{2.250606in}}%
\pgfpathlineto{\pgfqpoint{3.295543in}{2.240968in}}%
\pgfpathlineto{\pgfqpoint{3.287440in}{2.231392in}}%
\pgfpathclose%
\pgfusepath{fill}%
\end{pgfscope}%
\begin{pgfscope}%
\pgfpathrectangle{\pgfqpoint{1.150000in}{0.150000in}}{\pgfqpoint{5.700000in}{5.700000in}}%
\pgfusepath{clip}%
\pgfsetbuttcap%
\pgfsetroundjoin%
\definecolor{currentfill}{rgb}{0.281924,0.089666,0.412415}%
\pgfsetfillcolor{currentfill}%
\pgfsetfillopacity{0.800000}%
\pgfsetlinewidth{0.000000pt}%
\definecolor{currentstroke}{rgb}{0.000000,0.000000,0.000000}%
\pgfsetstrokecolor{currentstroke}%
\pgfsetdash{}{0pt}%
\pgfpathmoveto{\pgfqpoint{3.427497in}{2.210851in}}%
\pgfpathlineto{\pgfqpoint{3.440964in}{2.204487in}}%
\pgfpathlineto{\pgfqpoint{3.454434in}{2.198350in}}%
\pgfpathlineto{\pgfqpoint{3.467905in}{2.192438in}}%
\pgfpathlineto{\pgfqpoint{3.481379in}{2.186750in}}%
\pgfpathlineto{\pgfqpoint{3.489417in}{2.196784in}}%
\pgfpathlineto{\pgfqpoint{3.497449in}{2.206855in}}%
\pgfpathlineto{\pgfqpoint{3.505476in}{2.216963in}}%
\pgfpathlineto{\pgfqpoint{3.513497in}{2.227109in}}%
\pgfpathlineto{\pgfqpoint{3.500035in}{2.232739in}}%
\pgfpathlineto{\pgfqpoint{3.486576in}{2.238594in}}%
\pgfpathlineto{\pgfqpoint{3.473119in}{2.244673in}}%
\pgfpathlineto{\pgfqpoint{3.459664in}{2.250978in}}%
\pgfpathlineto{\pgfqpoint{3.451631in}{2.240879in}}%
\pgfpathlineto{\pgfqpoint{3.443592in}{2.230824in}}%
\pgfpathlineto{\pgfqpoint{3.435548in}{2.220815in}}%
\pgfpathlineto{\pgfqpoint{3.427497in}{2.210851in}}%
\pgfpathclose%
\pgfusepath{fill}%
\end{pgfscope}%
\begin{pgfscope}%
\pgfpathrectangle{\pgfqpoint{1.150000in}{0.150000in}}{\pgfqpoint{5.700000in}{5.700000in}}%
\pgfusepath{clip}%
\pgfsetbuttcap%
\pgfsetroundjoin%
\definecolor{currentfill}{rgb}{0.165117,0.467423,0.558141}%
\pgfsetfillcolor{currentfill}%
\pgfsetfillopacity{0.800000}%
\pgfsetlinewidth{0.000000pt}%
\definecolor{currentstroke}{rgb}{0.000000,0.000000,0.000000}%
\pgfsetstrokecolor{currentstroke}%
\pgfsetdash{}{0pt}%
\pgfpathmoveto{\pgfqpoint{5.366382in}{3.140825in}}%
\pgfpathlineto{\pgfqpoint{5.380469in}{3.146124in}}%
\pgfpathlineto{\pgfqpoint{5.394572in}{3.151596in}}%
\pgfpathlineto{\pgfqpoint{5.408689in}{3.157241in}}%
\pgfpathlineto{\pgfqpoint{5.422823in}{3.163060in}}%
\pgfpathlineto{\pgfqpoint{5.430173in}{3.170509in}}%
\pgfpathlineto{\pgfqpoint{5.437522in}{3.178122in}}%
\pgfpathlineto{\pgfqpoint{5.444871in}{3.185907in}}%
\pgfpathlineto{\pgfqpoint{5.452219in}{3.193870in}}%
\pgfpathlineto{\pgfqpoint{5.438111in}{3.188701in}}%
\pgfpathlineto{\pgfqpoint{5.424018in}{3.183704in}}%
\pgfpathlineto{\pgfqpoint{5.409941in}{3.178880in}}%
\pgfpathlineto{\pgfqpoint{5.395878in}{3.174229in}}%
\pgfpathlineto{\pgfqpoint{5.388505in}{3.165607in}}%
\pgfpathlineto{\pgfqpoint{5.381132in}{3.157170in}}%
\pgfpathlineto{\pgfqpoint{5.373757in}{3.148912in}}%
\pgfpathlineto{\pgfqpoint{5.366382in}{3.140825in}}%
\pgfpathclose%
\pgfusepath{fill}%
\end{pgfscope}%
\begin{pgfscope}%
\pgfpathrectangle{\pgfqpoint{1.150000in}{0.150000in}}{\pgfqpoint{5.700000in}{5.700000in}}%
\pgfusepath{clip}%
\pgfsetbuttcap%
\pgfsetroundjoin%
\definecolor{currentfill}{rgb}{0.206756,0.371758,0.553117}%
\pgfsetfillcolor{currentfill}%
\pgfsetfillopacity{0.800000}%
\pgfsetlinewidth{0.000000pt}%
\definecolor{currentstroke}{rgb}{0.000000,0.000000,0.000000}%
\pgfsetstrokecolor{currentstroke}%
\pgfsetdash{}{0pt}%
\pgfpathmoveto{\pgfqpoint{2.603131in}{2.913829in}}%
\pgfpathlineto{\pgfqpoint{2.616900in}{2.891652in}}%
\pgfpathlineto{\pgfqpoint{2.630656in}{2.869825in}}%
\pgfpathlineto{\pgfqpoint{2.644400in}{2.848343in}}%
\pgfpathlineto{\pgfqpoint{2.658131in}{2.827204in}}%
\pgfpathlineto{\pgfqpoint{2.666489in}{2.834998in}}%
\pgfpathlineto{\pgfqpoint{2.674836in}{2.842945in}}%
\pgfpathlineto{\pgfqpoint{2.683173in}{2.851044in}}%
\pgfpathlineto{\pgfqpoint{2.691500in}{2.859296in}}%
\pgfpathlineto{\pgfqpoint{2.677794in}{2.880301in}}%
\pgfpathlineto{\pgfqpoint{2.664077in}{2.901649in}}%
\pgfpathlineto{\pgfqpoint{2.650348in}{2.923342in}}%
\pgfpathlineto{\pgfqpoint{2.636606in}{2.945384in}}%
\pgfpathlineto{\pgfqpoint{2.628253in}{2.937254in}}%
\pgfpathlineto{\pgfqpoint{2.619890in}{2.929284in}}%
\pgfpathlineto{\pgfqpoint{2.611516in}{2.921476in}}%
\pgfpathlineto{\pgfqpoint{2.603131in}{2.913829in}}%
\pgfpathclose%
\pgfusepath{fill}%
\end{pgfscope}%
\begin{pgfscope}%
\pgfpathrectangle{\pgfqpoint{1.150000in}{0.150000in}}{\pgfqpoint{5.700000in}{5.700000in}}%
\pgfusepath{clip}%
\pgfsetbuttcap%
\pgfsetroundjoin%
\definecolor{currentfill}{rgb}{0.157729,0.485932,0.558013}%
\pgfsetfillcolor{currentfill}%
\pgfsetfillopacity{0.800000}%
\pgfsetlinewidth{0.000000pt}%
\definecolor{currentstroke}{rgb}{0.000000,0.000000,0.000000}%
\pgfsetstrokecolor{currentstroke}%
\pgfsetdash{}{0pt}%
\pgfpathmoveto{\pgfqpoint{5.452219in}{3.193870in}}%
\pgfpathlineto{\pgfqpoint{5.466342in}{3.199211in}}%
\pgfpathlineto{\pgfqpoint{5.480481in}{3.204724in}}%
\pgfpathlineto{\pgfqpoint{5.494635in}{3.210409in}}%
\pgfpathlineto{\pgfqpoint{5.508806in}{3.216267in}}%
\pgfpathlineto{\pgfqpoint{5.516127in}{3.223746in}}%
\pgfpathlineto{\pgfqpoint{5.523447in}{3.231410in}}%
\pgfpathlineto{\pgfqpoint{5.530768in}{3.239268in}}%
\pgfpathlineto{\pgfqpoint{5.538090in}{3.247327in}}%
\pgfpathlineto{\pgfqpoint{5.523947in}{3.242151in}}%
\pgfpathlineto{\pgfqpoint{5.509819in}{3.237147in}}%
\pgfpathlineto{\pgfqpoint{5.495707in}{3.232314in}}%
\pgfpathlineto{\pgfqpoint{5.481610in}{3.227652in}}%
\pgfpathlineto{\pgfqpoint{5.474262in}{3.218902in}}%
\pgfpathlineto{\pgfqpoint{5.466914in}{3.210360in}}%
\pgfpathlineto{\pgfqpoint{5.459567in}{3.202018in}}%
\pgfpathlineto{\pgfqpoint{5.452219in}{3.193870in}}%
\pgfpathclose%
\pgfusepath{fill}%
\end{pgfscope}%
\begin{pgfscope}%
\pgfpathrectangle{\pgfqpoint{1.150000in}{0.150000in}}{\pgfqpoint{5.700000in}{5.700000in}}%
\pgfusepath{clip}%
\pgfsetbuttcap%
\pgfsetroundjoin%
\definecolor{currentfill}{rgb}{0.271828,0.209303,0.504434}%
\pgfsetfillcolor{currentfill}%
\pgfsetfillopacity{0.800000}%
\pgfsetlinewidth{0.000000pt}%
\definecolor{currentstroke}{rgb}{0.000000,0.000000,0.000000}%
\pgfsetstrokecolor{currentstroke}%
\pgfsetdash{}{0pt}%
\pgfpathmoveto{\pgfqpoint{4.221578in}{2.449599in}}%
\pgfpathlineto{\pgfqpoint{4.235209in}{2.451121in}}%
\pgfpathlineto{\pgfqpoint{4.248849in}{2.452835in}}%
\pgfpathlineto{\pgfqpoint{4.262499in}{2.454742in}}%
\pgfpathlineto{\pgfqpoint{4.276158in}{2.456841in}}%
\pgfpathlineto{\pgfqpoint{4.283940in}{2.466583in}}%
\pgfpathlineto{\pgfqpoint{4.291717in}{2.476320in}}%
\pgfpathlineto{\pgfqpoint{4.299489in}{2.486054in}}%
\pgfpathlineto{\pgfqpoint{4.307255in}{2.495788in}}%
\pgfpathlineto{\pgfqpoint{4.293604in}{2.493886in}}%
\pgfpathlineto{\pgfqpoint{4.279962in}{2.492176in}}%
\pgfpathlineto{\pgfqpoint{4.266330in}{2.490658in}}%
\pgfpathlineto{\pgfqpoint{4.252708in}{2.489332in}}%
\pgfpathlineto{\pgfqpoint{4.244933in}{2.479389in}}%
\pgfpathlineto{\pgfqpoint{4.237153in}{2.469454in}}%
\pgfpathlineto{\pgfqpoint{4.229368in}{2.459525in}}%
\pgfpathlineto{\pgfqpoint{4.221578in}{2.449599in}}%
\pgfpathclose%
\pgfusepath{fill}%
\end{pgfscope}%
\begin{pgfscope}%
\pgfpathrectangle{\pgfqpoint{1.150000in}{0.150000in}}{\pgfqpoint{5.700000in}{5.700000in}}%
\pgfusepath{clip}%
\pgfsetbuttcap%
\pgfsetroundjoin%
\definecolor{currentfill}{rgb}{0.277134,0.185228,0.489898}%
\pgfsetfillcolor{currentfill}%
\pgfsetfillopacity{0.800000}%
\pgfsetlinewidth{0.000000pt}%
\definecolor{currentstroke}{rgb}{0.000000,0.000000,0.000000}%
\pgfsetstrokecolor{currentstroke}%
\pgfsetdash{}{0pt}%
\pgfpathmoveto{\pgfqpoint{4.135903in}{2.405070in}}%
\pgfpathlineto{\pgfqpoint{4.149506in}{2.405980in}}%
\pgfpathlineto{\pgfqpoint{4.163117in}{2.407084in}}%
\pgfpathlineto{\pgfqpoint{4.176737in}{2.408383in}}%
\pgfpathlineto{\pgfqpoint{4.190366in}{2.409877in}}%
\pgfpathlineto{\pgfqpoint{4.198177in}{2.419814in}}%
\pgfpathlineto{\pgfqpoint{4.205982in}{2.429745in}}%
\pgfpathlineto{\pgfqpoint{4.213783in}{2.439673in}}%
\pgfpathlineto{\pgfqpoint{4.221578in}{2.449599in}}%
\pgfpathlineto{\pgfqpoint{4.207957in}{2.448270in}}%
\pgfpathlineto{\pgfqpoint{4.194344in}{2.447136in}}%
\pgfpathlineto{\pgfqpoint{4.180741in}{2.446196in}}%
\pgfpathlineto{\pgfqpoint{4.167147in}{2.445450in}}%
\pgfpathlineto{\pgfqpoint{4.159344in}{2.435348in}}%
\pgfpathlineto{\pgfqpoint{4.151535in}{2.425252in}}%
\pgfpathlineto{\pgfqpoint{4.143722in}{2.415160in}}%
\pgfpathlineto{\pgfqpoint{4.135903in}{2.405070in}}%
\pgfpathclose%
\pgfusepath{fill}%
\end{pgfscope}%
\begin{pgfscope}%
\pgfpathrectangle{\pgfqpoint{1.150000in}{0.150000in}}{\pgfqpoint{5.700000in}{5.700000in}}%
\pgfusepath{clip}%
\pgfsetbuttcap%
\pgfsetroundjoin%
\definecolor{currentfill}{rgb}{0.266580,0.228262,0.514349}%
\pgfsetfillcolor{currentfill}%
\pgfsetfillopacity{0.800000}%
\pgfsetlinewidth{0.000000pt}%
\definecolor{currentstroke}{rgb}{0.000000,0.000000,0.000000}%
\pgfsetstrokecolor{currentstroke}%
\pgfsetdash{}{0pt}%
\pgfpathmoveto{\pgfqpoint{4.307255in}{2.495788in}}%
\pgfpathlineto{\pgfqpoint{4.320916in}{2.497882in}}%
\pgfpathlineto{\pgfqpoint{4.334588in}{2.500166in}}%
\pgfpathlineto{\pgfqpoint{4.348270in}{2.502640in}}%
\pgfpathlineto{\pgfqpoint{4.361962in}{2.505305in}}%
\pgfpathlineto{\pgfqpoint{4.369714in}{2.514825in}}%
\pgfpathlineto{\pgfqpoint{4.377462in}{2.524343in}}%
\pgfpathlineto{\pgfqpoint{4.385204in}{2.533862in}}%
\pgfpathlineto{\pgfqpoint{4.392942in}{2.543385in}}%
\pgfpathlineto{\pgfqpoint{4.379258in}{2.540949in}}%
\pgfpathlineto{\pgfqpoint{4.365586in}{2.538704in}}%
\pgfpathlineto{\pgfqpoint{4.351923in}{2.536648in}}%
\pgfpathlineto{\pgfqpoint{4.338271in}{2.534783in}}%
\pgfpathlineto{\pgfqpoint{4.330524in}{2.525020in}}%
\pgfpathlineto{\pgfqpoint{4.322773in}{2.515269in}}%
\pgfpathlineto{\pgfqpoint{4.315017in}{2.505526in}}%
\pgfpathlineto{\pgfqpoint{4.307255in}{2.495788in}}%
\pgfpathclose%
\pgfusepath{fill}%
\end{pgfscope}%
\begin{pgfscope}%
\pgfpathrectangle{\pgfqpoint{1.150000in}{0.150000in}}{\pgfqpoint{5.700000in}{5.700000in}}%
\pgfusepath{clip}%
\pgfsetbuttcap%
\pgfsetroundjoin%
\definecolor{currentfill}{rgb}{0.279574,0.170599,0.479997}%
\pgfsetfillcolor{currentfill}%
\pgfsetfillopacity{0.800000}%
\pgfsetlinewidth{0.000000pt}%
\definecolor{currentstroke}{rgb}{0.000000,0.000000,0.000000}%
\pgfsetstrokecolor{currentstroke}%
\pgfsetdash{}{0pt}%
\pgfpathmoveto{\pgfqpoint{4.050223in}{2.362480in}}%
\pgfpathlineto{\pgfqpoint{4.063800in}{2.362736in}}%
\pgfpathlineto{\pgfqpoint{4.077384in}{2.363189in}}%
\pgfpathlineto{\pgfqpoint{4.090977in}{2.363839in}}%
\pgfpathlineto{\pgfqpoint{4.104579in}{2.364686in}}%
\pgfpathlineto{\pgfqpoint{4.112417in}{2.374789in}}%
\pgfpathlineto{\pgfqpoint{4.120251in}{2.384886in}}%
\pgfpathlineto{\pgfqpoint{4.128080in}{2.394979in}}%
\pgfpathlineto{\pgfqpoint{4.135903in}{2.405070in}}%
\pgfpathlineto{\pgfqpoint{4.122310in}{2.404356in}}%
\pgfpathlineto{\pgfqpoint{4.108725in}{2.403839in}}%
\pgfpathlineto{\pgfqpoint{4.095148in}{2.403518in}}%
\pgfpathlineto{\pgfqpoint{4.081580in}{2.403395in}}%
\pgfpathlineto{\pgfqpoint{4.073748in}{2.393160in}}%
\pgfpathlineto{\pgfqpoint{4.065912in}{2.382930in}}%
\pgfpathlineto{\pgfqpoint{4.058070in}{2.372704in}}%
\pgfpathlineto{\pgfqpoint{4.050223in}{2.362480in}}%
\pgfpathclose%
\pgfusepath{fill}%
\end{pgfscope}%
\begin{pgfscope}%
\pgfpathrectangle{\pgfqpoint{1.150000in}{0.150000in}}{\pgfqpoint{5.700000in}{5.700000in}}%
\pgfusepath{clip}%
\pgfsetbuttcap%
\pgfsetroundjoin%
\definecolor{currentfill}{rgb}{0.260571,0.246922,0.522828}%
\pgfsetfillcolor{currentfill}%
\pgfsetfillopacity{0.800000}%
\pgfsetlinewidth{0.000000pt}%
\definecolor{currentstroke}{rgb}{0.000000,0.000000,0.000000}%
\pgfsetstrokecolor{currentstroke}%
\pgfsetdash{}{0pt}%
\pgfpathmoveto{\pgfqpoint{4.392942in}{2.543385in}}%
\pgfpathlineto{\pgfqpoint{4.406636in}{2.546009in}}%
\pgfpathlineto{\pgfqpoint{4.420340in}{2.548823in}}%
\pgfpathlineto{\pgfqpoint{4.434055in}{2.551824in}}%
\pgfpathlineto{\pgfqpoint{4.447782in}{2.555014in}}%
\pgfpathlineto{\pgfqpoint{4.455505in}{2.564294in}}%
\pgfpathlineto{\pgfqpoint{4.463223in}{2.573575in}}%
\pgfpathlineto{\pgfqpoint{4.470935in}{2.582862in}}%
\pgfpathlineto{\pgfqpoint{4.478643in}{2.592156in}}%
\pgfpathlineto{\pgfqpoint{4.464926in}{2.589228in}}%
\pgfpathlineto{\pgfqpoint{4.451220in}{2.586487in}}%
\pgfpathlineto{\pgfqpoint{4.437525in}{2.583934in}}%
\pgfpathlineto{\pgfqpoint{4.423840in}{2.581570in}}%
\pgfpathlineto{\pgfqpoint{4.416123in}{2.572003in}}%
\pgfpathlineto{\pgfqpoint{4.408401in}{2.562452in}}%
\pgfpathlineto{\pgfqpoint{4.400674in}{2.552914in}}%
\pgfpathlineto{\pgfqpoint{4.392942in}{2.543385in}}%
\pgfpathclose%
\pgfusepath{fill}%
\end{pgfscope}%
\begin{pgfscope}%
\pgfpathrectangle{\pgfqpoint{1.150000in}{0.150000in}}{\pgfqpoint{5.700000in}{5.700000in}}%
\pgfusepath{clip}%
\pgfsetbuttcap%
\pgfsetroundjoin%
\definecolor{currentfill}{rgb}{0.283197,0.115680,0.436115}%
\pgfsetfillcolor{currentfill}%
\pgfsetfillopacity{0.800000}%
\pgfsetlinewidth{0.000000pt}%
\definecolor{currentstroke}{rgb}{0.000000,0.000000,0.000000}%
\pgfsetstrokecolor{currentstroke}%
\pgfsetdash{}{0pt}%
\pgfpathmoveto{\pgfqpoint{3.147045in}{2.269928in}}%
\pgfpathlineto{\pgfqpoint{3.160539in}{2.259541in}}%
\pgfpathlineto{\pgfqpoint{3.174032in}{2.249403in}}%
\pgfpathlineto{\pgfqpoint{3.187523in}{2.239513in}}%
\pgfpathlineto{\pgfqpoint{3.201013in}{2.229870in}}%
\pgfpathlineto{\pgfqpoint{3.209158in}{2.239054in}}%
\pgfpathlineto{\pgfqpoint{3.217296in}{2.248312in}}%
\pgfpathlineto{\pgfqpoint{3.225427in}{2.257642in}}%
\pgfpathlineto{\pgfqpoint{3.233551in}{2.267044in}}%
\pgfpathlineto{\pgfqpoint{3.220078in}{2.276564in}}%
\pgfpathlineto{\pgfqpoint{3.206603in}{2.286331in}}%
\pgfpathlineto{\pgfqpoint{3.193128in}{2.296346in}}%
\pgfpathlineto{\pgfqpoint{3.179650in}{2.306611in}}%
\pgfpathlineto{\pgfqpoint{3.171510in}{2.297320in}}%
\pgfpathlineto{\pgfqpoint{3.163362in}{2.288109in}}%
\pgfpathlineto{\pgfqpoint{3.155207in}{2.278978in}}%
\pgfpathlineto{\pgfqpoint{3.147045in}{2.269928in}}%
\pgfpathclose%
\pgfusepath{fill}%
\end{pgfscope}%
\begin{pgfscope}%
\pgfpathrectangle{\pgfqpoint{1.150000in}{0.150000in}}{\pgfqpoint{5.700000in}{5.700000in}}%
\pgfusepath{clip}%
\pgfsetbuttcap%
\pgfsetroundjoin%
\definecolor{currentfill}{rgb}{0.150476,0.504369,0.557430}%
\pgfsetfillcolor{currentfill}%
\pgfsetfillopacity{0.800000}%
\pgfsetlinewidth{0.000000pt}%
\definecolor{currentstroke}{rgb}{0.000000,0.000000,0.000000}%
\pgfsetstrokecolor{currentstroke}%
\pgfsetdash{}{0pt}%
\pgfpathmoveto{\pgfqpoint{5.538090in}{3.247327in}}%
\pgfpathlineto{\pgfqpoint{5.552248in}{3.252674in}}%
\pgfpathlineto{\pgfqpoint{5.566423in}{3.258192in}}%
\pgfpathlineto{\pgfqpoint{5.580613in}{3.263882in}}%
\pgfpathlineto{\pgfqpoint{5.594819in}{3.269742in}}%
\pgfpathlineto{\pgfqpoint{5.602113in}{3.277308in}}%
\pgfpathlineto{\pgfqpoint{5.609407in}{3.285082in}}%
\pgfpathlineto{\pgfqpoint{5.616702in}{3.293074in}}%
\pgfpathlineto{\pgfqpoint{5.623999in}{3.301291in}}%
\pgfpathlineto{\pgfqpoint{5.609822in}{3.296144in}}%
\pgfpathlineto{\pgfqpoint{5.595660in}{3.291168in}}%
\pgfpathlineto{\pgfqpoint{5.581515in}{3.286363in}}%
\pgfpathlineto{\pgfqpoint{5.567384in}{3.281727in}}%
\pgfpathlineto{\pgfqpoint{5.560059in}{3.272787in}}%
\pgfpathlineto{\pgfqpoint{5.552734in}{3.264078in}}%
\pgfpathlineto{\pgfqpoint{5.545412in}{3.255594in}}%
\pgfpathlineto{\pgfqpoint{5.538090in}{3.247327in}}%
\pgfpathclose%
\pgfusepath{fill}%
\end{pgfscope}%
\begin{pgfscope}%
\pgfpathrectangle{\pgfqpoint{1.150000in}{0.150000in}}{\pgfqpoint{5.700000in}{5.700000in}}%
\pgfusepath{clip}%
\pgfsetbuttcap%
\pgfsetroundjoin%
\definecolor{currentfill}{rgb}{0.282327,0.094955,0.417331}%
\pgfsetfillcolor{currentfill}%
\pgfsetfillopacity{0.800000}%
\pgfsetlinewidth{0.000000pt}%
\definecolor{currentstroke}{rgb}{0.000000,0.000000,0.000000}%
\pgfsetstrokecolor{currentstroke}%
\pgfsetdash{}{0pt}%
\pgfpathmoveto{\pgfqpoint{3.567374in}{2.206806in}}%
\pgfpathlineto{\pgfqpoint{3.580852in}{2.202279in}}%
\pgfpathlineto{\pgfqpoint{3.594334in}{2.197970in}}%
\pgfpathlineto{\pgfqpoint{3.607819in}{2.193876in}}%
\pgfpathlineto{\pgfqpoint{3.621309in}{2.189998in}}%
\pgfpathlineto{\pgfqpoint{3.629302in}{2.200258in}}%
\pgfpathlineto{\pgfqpoint{3.637290in}{2.210539in}}%
\pgfpathlineto{\pgfqpoint{3.645273in}{2.220842in}}%
\pgfpathlineto{\pgfqpoint{3.653251in}{2.231167in}}%
\pgfpathlineto{\pgfqpoint{3.639771in}{2.235019in}}%
\pgfpathlineto{\pgfqpoint{3.626296in}{2.239087in}}%
\pgfpathlineto{\pgfqpoint{3.612825in}{2.243370in}}%
\pgfpathlineto{\pgfqpoint{3.599358in}{2.247871in}}%
\pgfpathlineto{\pgfqpoint{3.591370in}{2.237560in}}%
\pgfpathlineto{\pgfqpoint{3.583377in}{2.227279in}}%
\pgfpathlineto{\pgfqpoint{3.575379in}{2.217028in}}%
\pgfpathlineto{\pgfqpoint{3.567374in}{2.206806in}}%
\pgfpathclose%
\pgfusepath{fill}%
\end{pgfscope}%
\begin{pgfscope}%
\pgfpathrectangle{\pgfqpoint{1.150000in}{0.150000in}}{\pgfqpoint{5.700000in}{5.700000in}}%
\pgfusepath{clip}%
\pgfsetbuttcap%
\pgfsetroundjoin%
\definecolor{currentfill}{rgb}{0.252194,0.269783,0.531579}%
\pgfsetfillcolor{currentfill}%
\pgfsetfillopacity{0.800000}%
\pgfsetlinewidth{0.000000pt}%
\definecolor{currentstroke}{rgb}{0.000000,0.000000,0.000000}%
\pgfsetstrokecolor{currentstroke}%
\pgfsetdash{}{0pt}%
\pgfpathmoveto{\pgfqpoint{4.478643in}{2.592156in}}%
\pgfpathlineto{\pgfqpoint{4.492371in}{2.595272in}}%
\pgfpathlineto{\pgfqpoint{4.506110in}{2.598575in}}%
\pgfpathlineto{\pgfqpoint{4.519861in}{2.602065in}}%
\pgfpathlineto{\pgfqpoint{4.533623in}{2.605740in}}%
\pgfpathlineto{\pgfqpoint{4.541315in}{2.614765in}}%
\pgfpathlineto{\pgfqpoint{4.549003in}{2.623796in}}%
\pgfpathlineto{\pgfqpoint{4.556685in}{2.632838in}}%
\pgfpathlineto{\pgfqpoint{4.564362in}{2.641894in}}%
\pgfpathlineto{\pgfqpoint{4.550610in}{2.638512in}}%
\pgfpathlineto{\pgfqpoint{4.536869in}{2.635316in}}%
\pgfpathlineto{\pgfqpoint{4.523140in}{2.632306in}}%
\pgfpathlineto{\pgfqpoint{4.509422in}{2.629483in}}%
\pgfpathlineto{\pgfqpoint{4.501735in}{2.620122in}}%
\pgfpathlineto{\pgfqpoint{4.494043in}{2.610783in}}%
\pgfpathlineto{\pgfqpoint{4.486345in}{2.601462in}}%
\pgfpathlineto{\pgfqpoint{4.478643in}{2.592156in}}%
\pgfpathclose%
\pgfusepath{fill}%
\end{pgfscope}%
\begin{pgfscope}%
\pgfpathrectangle{\pgfqpoint{1.150000in}{0.150000in}}{\pgfqpoint{5.700000in}{5.700000in}}%
\pgfusepath{clip}%
\pgfsetbuttcap%
\pgfsetroundjoin%
\definecolor{currentfill}{rgb}{0.281887,0.150881,0.465405}%
\pgfsetfillcolor{currentfill}%
\pgfsetfillopacity{0.800000}%
\pgfsetlinewidth{0.000000pt}%
\definecolor{currentstroke}{rgb}{0.000000,0.000000,0.000000}%
\pgfsetstrokecolor{currentstroke}%
\pgfsetdash{}{0pt}%
\pgfpathmoveto{\pgfqpoint{3.964528in}{2.322130in}}%
\pgfpathlineto{\pgfqpoint{3.978081in}{2.321690in}}%
\pgfpathlineto{\pgfqpoint{3.991641in}{2.321450in}}%
\pgfpathlineto{\pgfqpoint{4.005210in}{2.321409in}}%
\pgfpathlineto{\pgfqpoint{4.018786in}{2.321567in}}%
\pgfpathlineto{\pgfqpoint{4.026653in}{2.331801in}}%
\pgfpathlineto{\pgfqpoint{4.034515in}{2.342030in}}%
\pgfpathlineto{\pgfqpoint{4.042371in}{2.352256in}}%
\pgfpathlineto{\pgfqpoint{4.050223in}{2.362480in}}%
\pgfpathlineto{\pgfqpoint{4.036655in}{2.362423in}}%
\pgfpathlineto{\pgfqpoint{4.023095in}{2.362564in}}%
\pgfpathlineto{\pgfqpoint{4.009542in}{2.362905in}}%
\pgfpathlineto{\pgfqpoint{3.995997in}{2.363446in}}%
\pgfpathlineto{\pgfqpoint{3.988137in}{2.353109in}}%
\pgfpathlineto{\pgfqpoint{3.980272in}{2.342779in}}%
\pgfpathlineto{\pgfqpoint{3.972402in}{2.332453in}}%
\pgfpathlineto{\pgfqpoint{3.964528in}{2.322130in}}%
\pgfpathclose%
\pgfusepath{fill}%
\end{pgfscope}%
\begin{pgfscope}%
\pgfpathrectangle{\pgfqpoint{1.150000in}{0.150000in}}{\pgfqpoint{5.700000in}{5.700000in}}%
\pgfusepath{clip}%
\pgfsetbuttcap%
\pgfsetroundjoin%
\definecolor{currentfill}{rgb}{0.244972,0.287675,0.537260}%
\pgfsetfillcolor{currentfill}%
\pgfsetfillopacity{0.800000}%
\pgfsetlinewidth{0.000000pt}%
\definecolor{currentstroke}{rgb}{0.000000,0.000000,0.000000}%
\pgfsetstrokecolor{currentstroke}%
\pgfsetdash{}{0pt}%
\pgfpathmoveto{\pgfqpoint{4.564362in}{2.641894in}}%
\pgfpathlineto{\pgfqpoint{4.578125in}{2.645462in}}%
\pgfpathlineto{\pgfqpoint{4.591901in}{2.649215in}}%
\pgfpathlineto{\pgfqpoint{4.605688in}{2.653154in}}%
\pgfpathlineto{\pgfqpoint{4.619487in}{2.657277in}}%
\pgfpathlineto{\pgfqpoint{4.627149in}{2.666037in}}%
\pgfpathlineto{\pgfqpoint{4.634805in}{2.674810in}}%
\pgfpathlineto{\pgfqpoint{4.642456in}{2.683602in}}%
\pgfpathlineto{\pgfqpoint{4.650102in}{2.692414in}}%
\pgfpathlineto{\pgfqpoint{4.636313in}{2.688617in}}%
\pgfpathlineto{\pgfqpoint{4.622537in}{2.685005in}}%
\pgfpathlineto{\pgfqpoint{4.608773in}{2.681577in}}%
\pgfpathlineto{\pgfqpoint{4.595020in}{2.678334in}}%
\pgfpathlineto{\pgfqpoint{4.587363in}{2.669184in}}%
\pgfpathlineto{\pgfqpoint{4.579701in}{2.660064in}}%
\pgfpathlineto{\pgfqpoint{4.572034in}{2.650968in}}%
\pgfpathlineto{\pgfqpoint{4.564362in}{2.641894in}}%
\pgfpathclose%
\pgfusepath{fill}%
\end{pgfscope}%
\begin{pgfscope}%
\pgfpathrectangle{\pgfqpoint{1.150000in}{0.150000in}}{\pgfqpoint{5.700000in}{5.700000in}}%
\pgfusepath{clip}%
\pgfsetbuttcap%
\pgfsetroundjoin%
\definecolor{currentfill}{rgb}{0.143343,0.522773,0.556295}%
\pgfsetfillcolor{currentfill}%
\pgfsetfillopacity{0.800000}%
\pgfsetlinewidth{0.000000pt}%
\definecolor{currentstroke}{rgb}{0.000000,0.000000,0.000000}%
\pgfsetstrokecolor{currentstroke}%
\pgfsetdash{}{0pt}%
\pgfpathmoveto{\pgfqpoint{5.623999in}{3.301291in}}%
\pgfpathlineto{\pgfqpoint{5.638192in}{3.306607in}}%
\pgfpathlineto{\pgfqpoint{5.652401in}{3.312095in}}%
\pgfpathlineto{\pgfqpoint{5.666626in}{3.317752in}}%
\pgfpathlineto{\pgfqpoint{5.680867in}{3.323580in}}%
\pgfpathlineto{\pgfqpoint{5.688135in}{3.331295in}}%
\pgfpathlineto{\pgfqpoint{5.695405in}{3.339245in}}%
\pgfpathlineto{\pgfqpoint{5.702677in}{3.347436in}}%
\pgfpathlineto{\pgfqpoint{5.709952in}{3.355879in}}%
\pgfpathlineto{\pgfqpoint{5.695742in}{3.350798in}}%
\pgfpathlineto{\pgfqpoint{5.681548in}{3.345886in}}%
\pgfpathlineto{\pgfqpoint{5.667369in}{3.341144in}}%
\pgfpathlineto{\pgfqpoint{5.653207in}{3.336571in}}%
\pgfpathlineto{\pgfqpoint{5.645901in}{3.327372in}}%
\pgfpathlineto{\pgfqpoint{5.638598in}{3.318432in}}%
\pgfpathlineto{\pgfqpoint{5.631297in}{3.309741in}}%
\pgfpathlineto{\pgfqpoint{5.623999in}{3.301291in}}%
\pgfpathclose%
\pgfusepath{fill}%
\end{pgfscope}%
\begin{pgfscope}%
\pgfpathrectangle{\pgfqpoint{1.150000in}{0.150000in}}{\pgfqpoint{5.700000in}{5.700000in}}%
\pgfusepath{clip}%
\pgfsetbuttcap%
\pgfsetroundjoin%
\definecolor{currentfill}{rgb}{0.282884,0.135920,0.453427}%
\pgfsetfillcolor{currentfill}%
\pgfsetfillopacity{0.800000}%
\pgfsetlinewidth{0.000000pt}%
\definecolor{currentstroke}{rgb}{0.000000,0.000000,0.000000}%
\pgfsetstrokecolor{currentstroke}%
\pgfsetdash{}{0pt}%
\pgfpathmoveto{\pgfqpoint{3.878804in}{2.284345in}}%
\pgfpathlineto{\pgfqpoint{3.892337in}{2.283166in}}%
\pgfpathlineto{\pgfqpoint{3.905877in}{2.282190in}}%
\pgfpathlineto{\pgfqpoint{3.919423in}{2.281416in}}%
\pgfpathlineto{\pgfqpoint{3.932977in}{2.280844in}}%
\pgfpathlineto{\pgfqpoint{3.940872in}{2.291167in}}%
\pgfpathlineto{\pgfqpoint{3.948763in}{2.301489in}}%
\pgfpathlineto{\pgfqpoint{3.956648in}{2.311809in}}%
\pgfpathlineto{\pgfqpoint{3.964528in}{2.322130in}}%
\pgfpathlineto{\pgfqpoint{3.950982in}{2.322772in}}%
\pgfpathlineto{\pgfqpoint{3.937443in}{2.323615in}}%
\pgfpathlineto{\pgfqpoint{3.923912in}{2.324660in}}%
\pgfpathlineto{\pgfqpoint{3.910387in}{2.325908in}}%
\pgfpathlineto{\pgfqpoint{3.902499in}{2.315506in}}%
\pgfpathlineto{\pgfqpoint{3.894606in}{2.305113in}}%
\pgfpathlineto{\pgfqpoint{3.886707in}{2.294726in}}%
\pgfpathlineto{\pgfqpoint{3.878804in}{2.284345in}}%
\pgfpathclose%
\pgfusepath{fill}%
\end{pgfscope}%
\begin{pgfscope}%
\pgfpathrectangle{\pgfqpoint{1.150000in}{0.150000in}}{\pgfqpoint{5.700000in}{5.700000in}}%
\pgfusepath{clip}%
\pgfsetbuttcap%
\pgfsetroundjoin%
\definecolor{currentfill}{rgb}{0.235526,0.309527,0.542944}%
\pgfsetfillcolor{currentfill}%
\pgfsetfillopacity{0.800000}%
\pgfsetlinewidth{0.000000pt}%
\definecolor{currentstroke}{rgb}{0.000000,0.000000,0.000000}%
\pgfsetstrokecolor{currentstroke}%
\pgfsetdash{}{0pt}%
\pgfpathmoveto{\pgfqpoint{4.650102in}{2.692414in}}%
\pgfpathlineto{\pgfqpoint{4.663902in}{2.696395in}}%
\pgfpathlineto{\pgfqpoint{4.677715in}{2.700560in}}%
\pgfpathlineto{\pgfqpoint{4.691540in}{2.704908in}}%
\pgfpathlineto{\pgfqpoint{4.705377in}{2.709439in}}%
\pgfpathlineto{\pgfqpoint{4.713006in}{2.717931in}}%
\pgfpathlineto{\pgfqpoint{4.720631in}{2.726444in}}%
\pgfpathlineto{\pgfqpoint{4.728250in}{2.734984in}}%
\pgfpathlineto{\pgfqpoint{4.735864in}{2.743553in}}%
\pgfpathlineto{\pgfqpoint{4.722039in}{2.739381in}}%
\pgfpathlineto{\pgfqpoint{4.708226in}{2.735391in}}%
\pgfpathlineto{\pgfqpoint{4.694425in}{2.731584in}}%
\pgfpathlineto{\pgfqpoint{4.680637in}{2.727961in}}%
\pgfpathlineto{\pgfqpoint{4.673010in}{2.719021in}}%
\pgfpathlineto{\pgfqpoint{4.665379in}{2.710120in}}%
\pgfpathlineto{\pgfqpoint{4.657743in}{2.701252in}}%
\pgfpathlineto{\pgfqpoint{4.650102in}{2.692414in}}%
\pgfpathclose%
\pgfusepath{fill}%
\end{pgfscope}%
\begin{pgfscope}%
\pgfpathrectangle{\pgfqpoint{1.150000in}{0.150000in}}{\pgfqpoint{5.700000in}{5.700000in}}%
\pgfusepath{clip}%
\pgfsetbuttcap%
\pgfsetroundjoin%
\definecolor{currentfill}{rgb}{0.270595,0.214069,0.507052}%
\pgfsetfillcolor{currentfill}%
\pgfsetfillopacity{0.800000}%
\pgfsetlinewidth{0.000000pt}%
\definecolor{currentstroke}{rgb}{0.000000,0.000000,0.000000}%
\pgfsetstrokecolor{currentstroke}%
\pgfsetdash{}{0pt}%
\pgfpathmoveto{\pgfqpoint{2.843336in}{2.500765in}}%
\pgfpathlineto{\pgfqpoint{2.856935in}{2.484820in}}%
\pgfpathlineto{\pgfqpoint{2.870527in}{2.469166in}}%
\pgfpathlineto{\pgfqpoint{2.884113in}{2.453801in}}%
\pgfpathlineto{\pgfqpoint{2.897692in}{2.438721in}}%
\pgfpathlineto{\pgfqpoint{2.905965in}{2.446762in}}%
\pgfpathlineto{\pgfqpoint{2.914230in}{2.454920in}}%
\pgfpathlineto{\pgfqpoint{2.922486in}{2.463194in}}%
\pgfpathlineto{\pgfqpoint{2.930733in}{2.471583in}}%
\pgfpathlineto{\pgfqpoint{2.917176in}{2.486504in}}%
\pgfpathlineto{\pgfqpoint{2.903613in}{2.501710in}}%
\pgfpathlineto{\pgfqpoint{2.890043in}{2.517205in}}%
\pgfpathlineto{\pgfqpoint{2.876467in}{2.532990in}}%
\pgfpathlineto{\pgfqpoint{2.868198in}{2.524748in}}%
\pgfpathlineto{\pgfqpoint{2.859920in}{2.516629in}}%
\pgfpathlineto{\pgfqpoint{2.851633in}{2.508634in}}%
\pgfpathlineto{\pgfqpoint{2.843336in}{2.500765in}}%
\pgfpathclose%
\pgfusepath{fill}%
\end{pgfscope}%
\begin{pgfscope}%
\pgfpathrectangle{\pgfqpoint{1.150000in}{0.150000in}}{\pgfqpoint{5.700000in}{5.700000in}}%
\pgfusepath{clip}%
\pgfsetbuttcap%
\pgfsetroundjoin%
\definecolor{currentfill}{rgb}{0.262138,0.242286,0.520837}%
\pgfsetfillcolor{currentfill}%
\pgfsetfillopacity{0.800000}%
\pgfsetlinewidth{0.000000pt}%
\definecolor{currentstroke}{rgb}{0.000000,0.000000,0.000000}%
\pgfsetstrokecolor{currentstroke}%
\pgfsetdash{}{0pt}%
\pgfpathmoveto{\pgfqpoint{2.788867in}{2.567502in}}%
\pgfpathlineto{\pgfqpoint{2.802496in}{2.550369in}}%
\pgfpathlineto{\pgfqpoint{2.816117in}{2.533537in}}%
\pgfpathlineto{\pgfqpoint{2.829730in}{2.517003in}}%
\pgfpathlineto{\pgfqpoint{2.843336in}{2.500765in}}%
\pgfpathlineto{\pgfqpoint{2.851633in}{2.508634in}}%
\pgfpathlineto{\pgfqpoint{2.859920in}{2.516629in}}%
\pgfpathlineto{\pgfqpoint{2.868198in}{2.524748in}}%
\pgfpathlineto{\pgfqpoint{2.876467in}{2.532990in}}%
\pgfpathlineto{\pgfqpoint{2.862885in}{2.549067in}}%
\pgfpathlineto{\pgfqpoint{2.849295in}{2.565441in}}%
\pgfpathlineto{\pgfqpoint{2.835698in}{2.582112in}}%
\pgfpathlineto{\pgfqpoint{2.822093in}{2.599084in}}%
\pgfpathlineto{\pgfqpoint{2.813800in}{2.590990in}}%
\pgfpathlineto{\pgfqpoint{2.805499in}{2.583028in}}%
\pgfpathlineto{\pgfqpoint{2.797188in}{2.575198in}}%
\pgfpathlineto{\pgfqpoint{2.788867in}{2.567502in}}%
\pgfpathclose%
\pgfusepath{fill}%
\end{pgfscope}%
\begin{pgfscope}%
\pgfpathrectangle{\pgfqpoint{1.150000in}{0.150000in}}{\pgfqpoint{5.700000in}{5.700000in}}%
\pgfusepath{clip}%
\pgfsetbuttcap%
\pgfsetroundjoin%
\definecolor{currentfill}{rgb}{0.276194,0.190074,0.493001}%
\pgfsetfillcolor{currentfill}%
\pgfsetfillopacity{0.800000}%
\pgfsetlinewidth{0.000000pt}%
\definecolor{currentstroke}{rgb}{0.000000,0.000000,0.000000}%
\pgfsetstrokecolor{currentstroke}%
\pgfsetdash{}{0pt}%
\pgfpathmoveto{\pgfqpoint{2.897692in}{2.438721in}}%
\pgfpathlineto{\pgfqpoint{2.911266in}{2.423925in}}%
\pgfpathlineto{\pgfqpoint{2.924833in}{2.409411in}}%
\pgfpathlineto{\pgfqpoint{2.938396in}{2.395175in}}%
\pgfpathlineto{\pgfqpoint{2.951953in}{2.381217in}}%
\pgfpathlineto{\pgfqpoint{2.960204in}{2.389427in}}%
\pgfpathlineto{\pgfqpoint{2.968447in}{2.397747in}}%
\pgfpathlineto{\pgfqpoint{2.976682in}{2.406176in}}%
\pgfpathlineto{\pgfqpoint{2.984908in}{2.414711in}}%
\pgfpathlineto{\pgfqpoint{2.971372in}{2.428512in}}%
\pgfpathlineto{\pgfqpoint{2.957831in}{2.442589in}}%
\pgfpathlineto{\pgfqpoint{2.944285in}{2.456946in}}%
\pgfpathlineto{\pgfqpoint{2.930733in}{2.471583in}}%
\pgfpathlineto{\pgfqpoint{2.922486in}{2.463194in}}%
\pgfpathlineto{\pgfqpoint{2.914230in}{2.454920in}}%
\pgfpathlineto{\pgfqpoint{2.905965in}{2.446762in}}%
\pgfpathlineto{\pgfqpoint{2.897692in}{2.438721in}}%
\pgfpathclose%
\pgfusepath{fill}%
\end{pgfscope}%
\begin{pgfscope}%
\pgfpathrectangle{\pgfqpoint{1.150000in}{0.150000in}}{\pgfqpoint{5.700000in}{5.700000in}}%
\pgfusepath{clip}%
\pgfsetbuttcap%
\pgfsetroundjoin%
\definecolor{currentfill}{rgb}{0.281924,0.089666,0.412415}%
\pgfsetfillcolor{currentfill}%
\pgfsetfillopacity{0.800000}%
\pgfsetlinewidth{0.000000pt}%
\definecolor{currentstroke}{rgb}{0.000000,0.000000,0.000000}%
\pgfsetstrokecolor{currentstroke}%
\pgfsetdash{}{0pt}%
\pgfpathmoveto{\pgfqpoint{3.341331in}{2.199560in}}%
\pgfpathlineto{\pgfqpoint{3.354805in}{2.192188in}}%
\pgfpathlineto{\pgfqpoint{3.368281in}{2.185047in}}%
\pgfpathlineto{\pgfqpoint{3.381758in}{2.178137in}}%
\pgfpathlineto{\pgfqpoint{3.395236in}{2.171456in}}%
\pgfpathlineto{\pgfqpoint{3.403310in}{2.181235in}}%
\pgfpathlineto{\pgfqpoint{3.411379in}{2.191061in}}%
\pgfpathlineto{\pgfqpoint{3.419441in}{2.200933in}}%
\pgfpathlineto{\pgfqpoint{3.427497in}{2.210851in}}%
\pgfpathlineto{\pgfqpoint{3.414032in}{2.217442in}}%
\pgfpathlineto{\pgfqpoint{3.400569in}{2.224263in}}%
\pgfpathlineto{\pgfqpoint{3.387107in}{2.231314in}}%
\pgfpathlineto{\pgfqpoint{3.373646in}{2.238596in}}%
\pgfpathlineto{\pgfqpoint{3.365577in}{2.228756in}}%
\pgfpathlineto{\pgfqpoint{3.357501in}{2.218969in}}%
\pgfpathlineto{\pgfqpoint{3.349419in}{2.209237in}}%
\pgfpathlineto{\pgfqpoint{3.341331in}{2.199560in}}%
\pgfpathclose%
\pgfusepath{fill}%
\end{pgfscope}%
\begin{pgfscope}%
\pgfpathrectangle{\pgfqpoint{1.150000in}{0.150000in}}{\pgfqpoint{5.700000in}{5.700000in}}%
\pgfusepath{clip}%
\pgfsetbuttcap%
\pgfsetroundjoin%
\definecolor{currentfill}{rgb}{0.225863,0.330805,0.547314}%
\pgfsetfillcolor{currentfill}%
\pgfsetfillopacity{0.800000}%
\pgfsetlinewidth{0.000000pt}%
\definecolor{currentstroke}{rgb}{0.000000,0.000000,0.000000}%
\pgfsetstrokecolor{currentstroke}%
\pgfsetdash{}{0pt}%
\pgfpathmoveto{\pgfqpoint{4.735864in}{2.743553in}}%
\pgfpathlineto{\pgfqpoint{4.749702in}{2.747908in}}%
\pgfpathlineto{\pgfqpoint{4.763553in}{2.752446in}}%
\pgfpathlineto{\pgfqpoint{4.777417in}{2.757165in}}%
\pgfpathlineto{\pgfqpoint{4.791293in}{2.762067in}}%
\pgfpathlineto{\pgfqpoint{4.798890in}{2.770292in}}%
\pgfpathlineto{\pgfqpoint{4.806481in}{2.778548in}}%
\pgfpathlineto{\pgfqpoint{4.814068in}{2.786840in}}%
\pgfpathlineto{\pgfqpoint{4.821650in}{2.795172in}}%
\pgfpathlineto{\pgfqpoint{4.807786in}{2.790662in}}%
\pgfpathlineto{\pgfqpoint{4.793936in}{2.786333in}}%
\pgfpathlineto{\pgfqpoint{4.780098in}{2.782186in}}%
\pgfpathlineto{\pgfqpoint{4.766273in}{2.778221in}}%
\pgfpathlineto{\pgfqpoint{4.758678in}{2.769487in}}%
\pgfpathlineto{\pgfqpoint{4.751078in}{2.760800in}}%
\pgfpathlineto{\pgfqpoint{4.743473in}{2.752157in}}%
\pgfpathlineto{\pgfqpoint{4.735864in}{2.743553in}}%
\pgfpathclose%
\pgfusepath{fill}%
\end{pgfscope}%
\begin{pgfscope}%
\pgfpathrectangle{\pgfqpoint{1.150000in}{0.150000in}}{\pgfqpoint{5.700000in}{5.700000in}}%
\pgfusepath{clip}%
\pgfsetbuttcap%
\pgfsetroundjoin%
\definecolor{currentfill}{rgb}{0.252194,0.269783,0.531579}%
\pgfsetfillcolor{currentfill}%
\pgfsetfillopacity{0.800000}%
\pgfsetlinewidth{0.000000pt}%
\definecolor{currentstroke}{rgb}{0.000000,0.000000,0.000000}%
\pgfsetstrokecolor{currentstroke}%
\pgfsetdash{}{0pt}%
\pgfpathmoveto{\pgfqpoint{2.734267in}{2.639097in}}%
\pgfpathlineto{\pgfqpoint{2.747931in}{2.620733in}}%
\pgfpathlineto{\pgfqpoint{2.761585in}{2.602681in}}%
\pgfpathlineto{\pgfqpoint{2.775230in}{2.584938in}}%
\pgfpathlineto{\pgfqpoint{2.788867in}{2.567502in}}%
\pgfpathlineto{\pgfqpoint{2.797188in}{2.575198in}}%
\pgfpathlineto{\pgfqpoint{2.805499in}{2.583028in}}%
\pgfpathlineto{\pgfqpoint{2.813800in}{2.590990in}}%
\pgfpathlineto{\pgfqpoint{2.822093in}{2.599084in}}%
\pgfpathlineto{\pgfqpoint{2.808480in}{2.616359in}}%
\pgfpathlineto{\pgfqpoint{2.794859in}{2.633940in}}%
\pgfpathlineto{\pgfqpoint{2.781230in}{2.651829in}}%
\pgfpathlineto{\pgfqpoint{2.767591in}{2.670030in}}%
\pgfpathlineto{\pgfqpoint{2.759275in}{2.662087in}}%
\pgfpathlineto{\pgfqpoint{2.750949in}{2.654283in}}%
\pgfpathlineto{\pgfqpoint{2.742613in}{2.646619in}}%
\pgfpathlineto{\pgfqpoint{2.734267in}{2.639097in}}%
\pgfpathclose%
\pgfusepath{fill}%
\end{pgfscope}%
\begin{pgfscope}%
\pgfpathrectangle{\pgfqpoint{1.150000in}{0.150000in}}{\pgfqpoint{5.700000in}{5.700000in}}%
\pgfusepath{clip}%
\pgfsetbuttcap%
\pgfsetroundjoin%
\definecolor{currentfill}{rgb}{0.283229,0.120777,0.440584}%
\pgfsetfillcolor{currentfill}%
\pgfsetfillopacity{0.800000}%
\pgfsetlinewidth{0.000000pt}%
\definecolor{currentstroke}{rgb}{0.000000,0.000000,0.000000}%
\pgfsetstrokecolor{currentstroke}%
\pgfsetdash{}{0pt}%
\pgfpathmoveto{\pgfqpoint{3.793038in}{2.249475in}}%
\pgfpathlineto{\pgfqpoint{3.806554in}{2.247513in}}%
\pgfpathlineto{\pgfqpoint{3.820076in}{2.245757in}}%
\pgfpathlineto{\pgfqpoint{3.833605in}{2.244207in}}%
\pgfpathlineto{\pgfqpoint{3.847139in}{2.242861in}}%
\pgfpathlineto{\pgfqpoint{3.855063in}{2.253228in}}%
\pgfpathlineto{\pgfqpoint{3.862982in}{2.263597in}}%
\pgfpathlineto{\pgfqpoint{3.870896in}{2.273969in}}%
\pgfpathlineto{\pgfqpoint{3.878804in}{2.284345in}}%
\pgfpathlineto{\pgfqpoint{3.865278in}{2.285729in}}%
\pgfpathlineto{\pgfqpoint{3.851758in}{2.287317in}}%
\pgfpathlineto{\pgfqpoint{3.838245in}{2.289110in}}%
\pgfpathlineto{\pgfqpoint{3.824737in}{2.291109in}}%
\pgfpathlineto{\pgfqpoint{3.816820in}{2.280684in}}%
\pgfpathlineto{\pgfqpoint{3.808898in}{2.270271in}}%
\pgfpathlineto{\pgfqpoint{3.800971in}{2.259868in}}%
\pgfpathlineto{\pgfqpoint{3.793038in}{2.249475in}}%
\pgfpathclose%
\pgfusepath{fill}%
\end{pgfscope}%
\begin{pgfscope}%
\pgfpathrectangle{\pgfqpoint{1.150000in}{0.150000in}}{\pgfqpoint{5.700000in}{5.700000in}}%
\pgfusepath{clip}%
\pgfsetbuttcap%
\pgfsetroundjoin%
\definecolor{currentfill}{rgb}{0.216210,0.351535,0.550627}%
\pgfsetfillcolor{currentfill}%
\pgfsetfillopacity{0.800000}%
\pgfsetlinewidth{0.000000pt}%
\definecolor{currentstroke}{rgb}{0.000000,0.000000,0.000000}%
\pgfsetstrokecolor{currentstroke}%
\pgfsetdash{}{0pt}%
\pgfpathmoveto{\pgfqpoint{4.821650in}{2.795172in}}%
\pgfpathlineto{\pgfqpoint{4.835526in}{2.799863in}}%
\pgfpathlineto{\pgfqpoint{4.849416in}{2.804735in}}%
\pgfpathlineto{\pgfqpoint{4.863319in}{2.809788in}}%
\pgfpathlineto{\pgfqpoint{4.877236in}{2.815021in}}%
\pgfpathlineto{\pgfqpoint{4.884799in}{2.822986in}}%
\pgfpathlineto{\pgfqpoint{4.892357in}{2.830994in}}%
\pgfpathlineto{\pgfqpoint{4.899910in}{2.839048in}}%
\pgfpathlineto{\pgfqpoint{4.907459in}{2.847154in}}%
\pgfpathlineto{\pgfqpoint{4.893557in}{2.842344in}}%
\pgfpathlineto{\pgfqpoint{4.879669in}{2.837715in}}%
\pgfpathlineto{\pgfqpoint{4.865794in}{2.833266in}}%
\pgfpathlineto{\pgfqpoint{4.851932in}{2.828998in}}%
\pgfpathlineto{\pgfqpoint{4.844368in}{2.820457in}}%
\pgfpathlineto{\pgfqpoint{4.836800in}{2.811976in}}%
\pgfpathlineto{\pgfqpoint{4.829227in}{2.803549in}}%
\pgfpathlineto{\pgfqpoint{4.821650in}{2.795172in}}%
\pgfpathclose%
\pgfusepath{fill}%
\end{pgfscope}%
\begin{pgfscope}%
\pgfpathrectangle{\pgfqpoint{1.150000in}{0.150000in}}{\pgfqpoint{5.700000in}{5.700000in}}%
\pgfusepath{clip}%
\pgfsetbuttcap%
\pgfsetroundjoin%
\definecolor{currentfill}{rgb}{0.280255,0.165693,0.476498}%
\pgfsetfillcolor{currentfill}%
\pgfsetfillopacity{0.800000}%
\pgfsetlinewidth{0.000000pt}%
\definecolor{currentstroke}{rgb}{0.000000,0.000000,0.000000}%
\pgfsetstrokecolor{currentstroke}%
\pgfsetdash{}{0pt}%
\pgfpathmoveto{\pgfqpoint{2.951953in}{2.381217in}}%
\pgfpathlineto{\pgfqpoint{2.965505in}{2.367533in}}%
\pgfpathlineto{\pgfqpoint{2.979052in}{2.354121in}}%
\pgfpathlineto{\pgfqpoint{2.992596in}{2.340981in}}%
\pgfpathlineto{\pgfqpoint{3.006134in}{2.328108in}}%
\pgfpathlineto{\pgfqpoint{3.014365in}{2.336488in}}%
\pgfpathlineto{\pgfqpoint{3.022587in}{2.344969in}}%
\pgfpathlineto{\pgfqpoint{3.030801in}{2.353550in}}%
\pgfpathlineto{\pgfqpoint{3.039008in}{2.362231in}}%
\pgfpathlineto{\pgfqpoint{3.025489in}{2.374946in}}%
\pgfpathlineto{\pgfqpoint{3.011966in}{2.387930in}}%
\pgfpathlineto{\pgfqpoint{2.998439in}{2.401184in}}%
\pgfpathlineto{\pgfqpoint{2.984908in}{2.414711in}}%
\pgfpathlineto{\pgfqpoint{2.976682in}{2.406176in}}%
\pgfpathlineto{\pgfqpoint{2.968447in}{2.397747in}}%
\pgfpathlineto{\pgfqpoint{2.960204in}{2.389427in}}%
\pgfpathlineto{\pgfqpoint{2.951953in}{2.381217in}}%
\pgfpathclose%
\pgfusepath{fill}%
\end{pgfscope}%
\begin{pgfscope}%
\pgfpathrectangle{\pgfqpoint{1.150000in}{0.150000in}}{\pgfqpoint{5.700000in}{5.700000in}}%
\pgfusepath{clip}%
\pgfsetbuttcap%
\pgfsetroundjoin%
\definecolor{currentfill}{rgb}{0.281446,0.084320,0.407414}%
\pgfsetfillcolor{currentfill}%
\pgfsetfillopacity{0.800000}%
\pgfsetlinewidth{0.000000pt}%
\definecolor{currentstroke}{rgb}{0.000000,0.000000,0.000000}%
\pgfsetstrokecolor{currentstroke}%
\pgfsetdash{}{0pt}%
\pgfpathmoveto{\pgfqpoint{3.481379in}{2.186750in}}%
\pgfpathlineto{\pgfqpoint{3.494855in}{2.181285in}}%
\pgfpathlineto{\pgfqpoint{3.508334in}{2.176042in}}%
\pgfpathlineto{\pgfqpoint{3.521817in}{2.171019in}}%
\pgfpathlineto{\pgfqpoint{3.535302in}{2.166216in}}%
\pgfpathlineto{\pgfqpoint{3.543329in}{2.176319in}}%
\pgfpathlineto{\pgfqpoint{3.551350in}{2.186452in}}%
\pgfpathlineto{\pgfqpoint{3.559365in}{2.196614in}}%
\pgfpathlineto{\pgfqpoint{3.567374in}{2.206806in}}%
\pgfpathlineto{\pgfqpoint{3.553900in}{2.211552in}}%
\pgfpathlineto{\pgfqpoint{3.540429in}{2.216517in}}%
\pgfpathlineto{\pgfqpoint{3.526962in}{2.221702in}}%
\pgfpathlineto{\pgfqpoint{3.513497in}{2.227109in}}%
\pgfpathlineto{\pgfqpoint{3.505476in}{2.216963in}}%
\pgfpathlineto{\pgfqpoint{3.497449in}{2.206855in}}%
\pgfpathlineto{\pgfqpoint{3.489417in}{2.196784in}}%
\pgfpathlineto{\pgfqpoint{3.481379in}{2.186750in}}%
\pgfpathclose%
\pgfusepath{fill}%
\end{pgfscope}%
\begin{pgfscope}%
\pgfpathrectangle{\pgfqpoint{1.150000in}{0.150000in}}{\pgfqpoint{5.700000in}{5.700000in}}%
\pgfusepath{clip}%
\pgfsetbuttcap%
\pgfsetroundjoin%
\definecolor{currentfill}{rgb}{0.282910,0.105393,0.426902}%
\pgfsetfillcolor{currentfill}%
\pgfsetfillopacity{0.800000}%
\pgfsetlinewidth{0.000000pt}%
\definecolor{currentstroke}{rgb}{0.000000,0.000000,0.000000}%
\pgfsetstrokecolor{currentstroke}%
\pgfsetdash{}{0pt}%
\pgfpathmoveto{\pgfqpoint{3.201013in}{2.229870in}}%
\pgfpathlineto{\pgfqpoint{3.214502in}{2.220472in}}%
\pgfpathlineto{\pgfqpoint{3.227990in}{2.211317in}}%
\pgfpathlineto{\pgfqpoint{3.241477in}{2.202403in}}%
\pgfpathlineto{\pgfqpoint{3.254965in}{2.193730in}}%
\pgfpathlineto{\pgfqpoint{3.263094in}{2.203048in}}%
\pgfpathlineto{\pgfqpoint{3.271216in}{2.212432in}}%
\pgfpathlineto{\pgfqpoint{3.279331in}{2.221880in}}%
\pgfpathlineto{\pgfqpoint{3.287440in}{2.231392in}}%
\pgfpathlineto{\pgfqpoint{3.273969in}{2.239944in}}%
\pgfpathlineto{\pgfqpoint{3.260497in}{2.248735in}}%
\pgfpathlineto{\pgfqpoint{3.247024in}{2.257768in}}%
\pgfpathlineto{\pgfqpoint{3.233551in}{2.267044in}}%
\pgfpathlineto{\pgfqpoint{3.225427in}{2.257642in}}%
\pgfpathlineto{\pgfqpoint{3.217296in}{2.248312in}}%
\pgfpathlineto{\pgfqpoint{3.209158in}{2.239054in}}%
\pgfpathlineto{\pgfqpoint{3.201013in}{2.229870in}}%
\pgfpathclose%
\pgfusepath{fill}%
\end{pgfscope}%
\begin{pgfscope}%
\pgfpathrectangle{\pgfqpoint{1.150000in}{0.150000in}}{\pgfqpoint{5.700000in}{5.700000in}}%
\pgfusepath{clip}%
\pgfsetbuttcap%
\pgfsetroundjoin%
\definecolor{currentfill}{rgb}{0.136408,0.541173,0.554483}%
\pgfsetfillcolor{currentfill}%
\pgfsetfillopacity{0.800000}%
\pgfsetlinewidth{0.000000pt}%
\definecolor{currentstroke}{rgb}{0.000000,0.000000,0.000000}%
\pgfsetstrokecolor{currentstroke}%
\pgfsetdash{}{0pt}%
\pgfpathmoveto{\pgfqpoint{5.709952in}{3.355879in}}%
\pgfpathlineto{\pgfqpoint{5.724178in}{3.361129in}}%
\pgfpathlineto{\pgfqpoint{5.738421in}{3.366549in}}%
\pgfpathlineto{\pgfqpoint{5.752680in}{3.372139in}}%
\pgfpathlineto{\pgfqpoint{5.766955in}{3.377897in}}%
\pgfpathlineto{\pgfqpoint{5.774200in}{3.385832in}}%
\pgfpathlineto{\pgfqpoint{5.781449in}{3.394027in}}%
\pgfpathlineto{\pgfqpoint{5.788701in}{3.402491in}}%
\pgfpathlineto{\pgfqpoint{5.774451in}{3.397314in}}%
\pgfpathlineto{\pgfqpoint{5.760217in}{3.392305in}}%
\pgfpathlineto{\pgfqpoint{5.745999in}{3.387465in}}%
\pgfpathlineto{\pgfqpoint{5.731796in}{3.382794in}}%
\pgfpathlineto{\pgfqpoint{5.724511in}{3.373549in}}%
\pgfpathlineto{\pgfqpoint{5.717230in}{3.364580in}}%
\pgfpathlineto{\pgfqpoint{5.709952in}{3.355879in}}%
\pgfpathclose%
\pgfusepath{fill}%
\end{pgfscope}%
\begin{pgfscope}%
\pgfpathrectangle{\pgfqpoint{1.150000in}{0.150000in}}{\pgfqpoint{5.700000in}{5.700000in}}%
\pgfusepath{clip}%
\pgfsetbuttcap%
\pgfsetroundjoin%
\definecolor{currentfill}{rgb}{0.206756,0.371758,0.553117}%
\pgfsetfillcolor{currentfill}%
\pgfsetfillopacity{0.800000}%
\pgfsetlinewidth{0.000000pt}%
\definecolor{currentstroke}{rgb}{0.000000,0.000000,0.000000}%
\pgfsetstrokecolor{currentstroke}%
\pgfsetdash{}{0pt}%
\pgfpathmoveto{\pgfqpoint{4.907459in}{2.847154in}}%
\pgfpathlineto{\pgfqpoint{4.921375in}{2.852143in}}%
\pgfpathlineto{\pgfqpoint{4.935304in}{2.857311in}}%
\pgfpathlineto{\pgfqpoint{4.949247in}{2.862659in}}%
\pgfpathlineto{\pgfqpoint{4.963204in}{2.868187in}}%
\pgfpathlineto{\pgfqpoint{4.970733in}{2.875905in}}%
\pgfpathlineto{\pgfqpoint{4.978257in}{2.883677in}}%
\pgfpathlineto{\pgfqpoint{4.985777in}{2.891509in}}%
\pgfpathlineto{\pgfqpoint{4.993293in}{2.899405in}}%
\pgfpathlineto{\pgfqpoint{4.979352in}{2.894335in}}%
\pgfpathlineto{\pgfqpoint{4.965425in}{2.889443in}}%
\pgfpathlineto{\pgfqpoint{4.951512in}{2.884730in}}%
\pgfpathlineto{\pgfqpoint{4.937612in}{2.880196in}}%
\pgfpathlineto{\pgfqpoint{4.930080in}{2.871832in}}%
\pgfpathlineto{\pgfqpoint{4.922544in}{2.863541in}}%
\pgfpathlineto{\pgfqpoint{4.915004in}{2.855316in}}%
\pgfpathlineto{\pgfqpoint{4.907459in}{2.847154in}}%
\pgfpathclose%
\pgfusepath{fill}%
\end{pgfscope}%
\begin{pgfscope}%
\pgfpathrectangle{\pgfqpoint{1.150000in}{0.150000in}}{\pgfqpoint{5.700000in}{5.700000in}}%
\pgfusepath{clip}%
\pgfsetbuttcap%
\pgfsetroundjoin%
\definecolor{currentfill}{rgb}{0.239346,0.300855,0.540844}%
\pgfsetfillcolor{currentfill}%
\pgfsetfillopacity{0.800000}%
\pgfsetlinewidth{0.000000pt}%
\definecolor{currentstroke}{rgb}{0.000000,0.000000,0.000000}%
\pgfsetstrokecolor{currentstroke}%
\pgfsetdash{}{0pt}%
\pgfpathmoveto{\pgfqpoint{2.679517in}{2.715728in}}%
\pgfpathlineto{\pgfqpoint{2.693220in}{2.696088in}}%
\pgfpathlineto{\pgfqpoint{2.706912in}{2.676771in}}%
\pgfpathlineto{\pgfqpoint{2.720595in}{2.657775in}}%
\pgfpathlineto{\pgfqpoint{2.734267in}{2.639097in}}%
\pgfpathlineto{\pgfqpoint{2.742613in}{2.646619in}}%
\pgfpathlineto{\pgfqpoint{2.750949in}{2.654283in}}%
\pgfpathlineto{\pgfqpoint{2.759275in}{2.662087in}}%
\pgfpathlineto{\pgfqpoint{2.767591in}{2.670030in}}%
\pgfpathlineto{\pgfqpoint{2.753944in}{2.688546in}}%
\pgfpathlineto{\pgfqpoint{2.740287in}{2.707378in}}%
\pgfpathlineto{\pgfqpoint{2.726621in}{2.726531in}}%
\pgfpathlineto{\pgfqpoint{2.712944in}{2.746007in}}%
\pgfpathlineto{\pgfqpoint{2.704603in}{2.738215in}}%
\pgfpathlineto{\pgfqpoint{2.696251in}{2.730570in}}%
\pgfpathlineto{\pgfqpoint{2.687889in}{2.723074in}}%
\pgfpathlineto{\pgfqpoint{2.679517in}{2.715728in}}%
\pgfpathclose%
\pgfusepath{fill}%
\end{pgfscope}%
\begin{pgfscope}%
\pgfpathrectangle{\pgfqpoint{1.150000in}{0.150000in}}{\pgfqpoint{5.700000in}{5.700000in}}%
\pgfusepath{clip}%
\pgfsetbuttcap%
\pgfsetroundjoin%
\definecolor{currentfill}{rgb}{0.282910,0.105393,0.426902}%
\pgfsetfillcolor{currentfill}%
\pgfsetfillopacity{0.800000}%
\pgfsetlinewidth{0.000000pt}%
\definecolor{currentstroke}{rgb}{0.000000,0.000000,0.000000}%
\pgfsetstrokecolor{currentstroke}%
\pgfsetdash{}{0pt}%
\pgfpathmoveto{\pgfqpoint{3.707213in}{2.217893in}}%
\pgfpathlineto{\pgfqpoint{3.720716in}{2.215103in}}%
\pgfpathlineto{\pgfqpoint{3.734224in}{2.212523in}}%
\pgfpathlineto{\pgfqpoint{3.747738in}{2.210152in}}%
\pgfpathlineto{\pgfqpoint{3.761257in}{2.207988in}}%
\pgfpathlineto{\pgfqpoint{3.769210in}{2.218348in}}%
\pgfpathlineto{\pgfqpoint{3.777158in}{2.228716in}}%
\pgfpathlineto{\pgfqpoint{3.785101in}{2.239091in}}%
\pgfpathlineto{\pgfqpoint{3.793038in}{2.249475in}}%
\pgfpathlineto{\pgfqpoint{3.779528in}{2.251644in}}%
\pgfpathlineto{\pgfqpoint{3.766024in}{2.254022in}}%
\pgfpathlineto{\pgfqpoint{3.752525in}{2.256608in}}%
\pgfpathlineto{\pgfqpoint{3.739031in}{2.259403in}}%
\pgfpathlineto{\pgfqpoint{3.731084in}{2.249002in}}%
\pgfpathlineto{\pgfqpoint{3.723132in}{2.238617in}}%
\pgfpathlineto{\pgfqpoint{3.715175in}{2.228247in}}%
\pgfpathlineto{\pgfqpoint{3.707213in}{2.217893in}}%
\pgfpathclose%
\pgfusepath{fill}%
\end{pgfscope}%
\begin{pgfscope}%
\pgfpathrectangle{\pgfqpoint{1.150000in}{0.150000in}}{\pgfqpoint{5.700000in}{5.700000in}}%
\pgfusepath{clip}%
\pgfsetbuttcap%
\pgfsetroundjoin%
\definecolor{currentfill}{rgb}{0.282290,0.145912,0.461510}%
\pgfsetfillcolor{currentfill}%
\pgfsetfillopacity{0.800000}%
\pgfsetlinewidth{0.000000pt}%
\definecolor{currentstroke}{rgb}{0.000000,0.000000,0.000000}%
\pgfsetstrokecolor{currentstroke}%
\pgfsetdash{}{0pt}%
\pgfpathmoveto{\pgfqpoint{3.006134in}{2.328108in}}%
\pgfpathlineto{\pgfqpoint{3.019669in}{2.315503in}}%
\pgfpathlineto{\pgfqpoint{3.033201in}{2.303161in}}%
\pgfpathlineto{\pgfqpoint{3.046729in}{2.291083in}}%
\pgfpathlineto{\pgfqpoint{3.060253in}{2.279264in}}%
\pgfpathlineto{\pgfqpoint{3.068463in}{2.287812in}}%
\pgfpathlineto{\pgfqpoint{3.076666in}{2.296453in}}%
\pgfpathlineto{\pgfqpoint{3.084861in}{2.305186in}}%
\pgfpathlineto{\pgfqpoint{3.093048in}{2.314011in}}%
\pgfpathlineto{\pgfqpoint{3.079543in}{2.325673in}}%
\pgfpathlineto{\pgfqpoint{3.066034in}{2.337596in}}%
\pgfpathlineto{\pgfqpoint{3.052523in}{2.349781in}}%
\pgfpathlineto{\pgfqpoint{3.039008in}{2.362231in}}%
\pgfpathlineto{\pgfqpoint{3.030801in}{2.353550in}}%
\pgfpathlineto{\pgfqpoint{3.022587in}{2.344969in}}%
\pgfpathlineto{\pgfqpoint{3.014365in}{2.336488in}}%
\pgfpathlineto{\pgfqpoint{3.006134in}{2.328108in}}%
\pgfpathclose%
\pgfusepath{fill}%
\end{pgfscope}%
\begin{pgfscope}%
\pgfpathrectangle{\pgfqpoint{1.150000in}{0.150000in}}{\pgfqpoint{5.700000in}{5.700000in}}%
\pgfusepath{clip}%
\pgfsetbuttcap%
\pgfsetroundjoin%
\definecolor{currentfill}{rgb}{0.197636,0.391528,0.554969}%
\pgfsetfillcolor{currentfill}%
\pgfsetfillopacity{0.800000}%
\pgfsetlinewidth{0.000000pt}%
\definecolor{currentstroke}{rgb}{0.000000,0.000000,0.000000}%
\pgfsetstrokecolor{currentstroke}%
\pgfsetdash{}{0pt}%
\pgfpathmoveto{\pgfqpoint{4.993293in}{2.899405in}}%
\pgfpathlineto{\pgfqpoint{5.007248in}{2.904654in}}%
\pgfpathlineto{\pgfqpoint{5.021216in}{2.910082in}}%
\pgfpathlineto{\pgfqpoint{5.035199in}{2.915688in}}%
\pgfpathlineto{\pgfqpoint{5.049197in}{2.921472in}}%
\pgfpathlineto{\pgfqpoint{5.056691in}{2.928961in}}%
\pgfpathlineto{\pgfqpoint{5.064182in}{2.936517in}}%
\pgfpathlineto{\pgfqpoint{5.071668in}{2.944147in}}%
\pgfpathlineto{\pgfqpoint{5.079150in}{2.951856in}}%
\pgfpathlineto{\pgfqpoint{5.065171in}{2.946562in}}%
\pgfpathlineto{\pgfqpoint{5.051205in}{2.941445in}}%
\pgfpathlineto{\pgfqpoint{5.037254in}{2.936506in}}%
\pgfpathlineto{\pgfqpoint{5.023316in}{2.931744in}}%
\pgfpathlineto{\pgfqpoint{5.015816in}{2.923535in}}%
\pgfpathlineto{\pgfqpoint{5.008312in}{2.915413in}}%
\pgfpathlineto{\pgfqpoint{5.000805in}{2.907371in}}%
\pgfpathlineto{\pgfqpoint{4.993293in}{2.899405in}}%
\pgfpathclose%
\pgfusepath{fill}%
\end{pgfscope}%
\begin{pgfscope}%
\pgfpathrectangle{\pgfqpoint{1.150000in}{0.150000in}}{\pgfqpoint{5.700000in}{5.700000in}}%
\pgfusepath{clip}%
\pgfsetbuttcap%
\pgfsetroundjoin%
\definecolor{currentfill}{rgb}{0.188923,0.410910,0.556326}%
\pgfsetfillcolor{currentfill}%
\pgfsetfillopacity{0.800000}%
\pgfsetlinewidth{0.000000pt}%
\definecolor{currentstroke}{rgb}{0.000000,0.000000,0.000000}%
\pgfsetstrokecolor{currentstroke}%
\pgfsetdash{}{0pt}%
\pgfpathmoveto{\pgfqpoint{5.079150in}{2.951856in}}%
\pgfpathlineto{\pgfqpoint{5.093144in}{2.957328in}}%
\pgfpathlineto{\pgfqpoint{5.107153in}{2.962977in}}%
\pgfpathlineto{\pgfqpoint{5.121176in}{2.968803in}}%
\pgfpathlineto{\pgfqpoint{5.135214in}{2.974806in}}%
\pgfpathlineto{\pgfqpoint{5.142674in}{2.982089in}}%
\pgfpathlineto{\pgfqpoint{5.150130in}{2.989455in}}%
\pgfpathlineto{\pgfqpoint{5.157583in}{2.996910in}}%
\pgfpathlineto{\pgfqpoint{5.165032in}{3.004459in}}%
\pgfpathlineto{\pgfqpoint{5.151013in}{2.998978in}}%
\pgfpathlineto{\pgfqpoint{5.137009in}{2.993674in}}%
\pgfpathlineto{\pgfqpoint{5.123020in}{2.988546in}}%
\pgfpathlineto{\pgfqpoint{5.109044in}{2.983594in}}%
\pgfpathlineto{\pgfqpoint{5.101576in}{2.975513in}}%
\pgfpathlineto{\pgfqpoint{5.094104in}{2.967533in}}%
\pgfpathlineto{\pgfqpoint{5.086629in}{2.959650in}}%
\pgfpathlineto{\pgfqpoint{5.079150in}{2.951856in}}%
\pgfpathclose%
\pgfusepath{fill}%
\end{pgfscope}%
\begin{pgfscope}%
\pgfpathrectangle{\pgfqpoint{1.150000in}{0.150000in}}{\pgfqpoint{5.700000in}{5.700000in}}%
\pgfusepath{clip}%
\pgfsetbuttcap%
\pgfsetroundjoin%
\definecolor{currentfill}{rgb}{0.223925,0.334994,0.548053}%
\pgfsetfillcolor{currentfill}%
\pgfsetfillopacity{0.800000}%
\pgfsetlinewidth{0.000000pt}%
\definecolor{currentstroke}{rgb}{0.000000,0.000000,0.000000}%
\pgfsetstrokecolor{currentstroke}%
\pgfsetdash{}{0pt}%
\pgfpathmoveto{\pgfqpoint{2.624596in}{2.797587in}}%
\pgfpathlineto{\pgfqpoint{2.638343in}{2.776621in}}%
\pgfpathlineto{\pgfqpoint{2.652079in}{2.755992in}}%
\pgfpathlineto{\pgfqpoint{2.665803in}{2.735695in}}%
\pgfpathlineto{\pgfqpoint{2.679517in}{2.715728in}}%
\pgfpathlineto{\pgfqpoint{2.687889in}{2.723074in}}%
\pgfpathlineto{\pgfqpoint{2.696251in}{2.730570in}}%
\pgfpathlineto{\pgfqpoint{2.704603in}{2.738215in}}%
\pgfpathlineto{\pgfqpoint{2.712944in}{2.746007in}}%
\pgfpathlineto{\pgfqpoint{2.699257in}{2.765809in}}%
\pgfpathlineto{\pgfqpoint{2.685560in}{2.785940in}}%
\pgfpathlineto{\pgfqpoint{2.671851in}{2.806404in}}%
\pgfpathlineto{\pgfqpoint{2.658131in}{2.827204in}}%
\pgfpathlineto{\pgfqpoint{2.649763in}{2.819565in}}%
\pgfpathlineto{\pgfqpoint{2.641385in}{2.812082in}}%
\pgfpathlineto{\pgfqpoint{2.632996in}{2.804756in}}%
\pgfpathlineto{\pgfqpoint{2.624596in}{2.797587in}}%
\pgfpathclose%
\pgfusepath{fill}%
\end{pgfscope}%
\begin{pgfscope}%
\pgfpathrectangle{\pgfqpoint{1.150000in}{0.150000in}}{\pgfqpoint{5.700000in}{5.700000in}}%
\pgfusepath{clip}%
\pgfsetbuttcap%
\pgfsetroundjoin%
\definecolor{currentfill}{rgb}{0.282327,0.094955,0.417331}%
\pgfsetfillcolor{currentfill}%
\pgfsetfillopacity{0.800000}%
\pgfsetlinewidth{0.000000pt}%
\definecolor{currentstroke}{rgb}{0.000000,0.000000,0.000000}%
\pgfsetstrokecolor{currentstroke}%
\pgfsetdash{}{0pt}%
\pgfpathmoveto{\pgfqpoint{3.621309in}{2.189998in}}%
\pgfpathlineto{\pgfqpoint{3.634802in}{2.186335in}}%
\pgfpathlineto{\pgfqpoint{3.648300in}{2.182885in}}%
\pgfpathlineto{\pgfqpoint{3.661803in}{2.179647in}}%
\pgfpathlineto{\pgfqpoint{3.675311in}{2.176621in}}%
\pgfpathlineto{\pgfqpoint{3.683294in}{2.186918in}}%
\pgfpathlineto{\pgfqpoint{3.691272in}{2.197229in}}%
\pgfpathlineto{\pgfqpoint{3.699245in}{2.207554in}}%
\pgfpathlineto{\pgfqpoint{3.707213in}{2.217893in}}%
\pgfpathlineto{\pgfqpoint{3.693715in}{2.220894in}}%
\pgfpathlineto{\pgfqpoint{3.680222in}{2.224105in}}%
\pgfpathlineto{\pgfqpoint{3.666734in}{2.227530in}}%
\pgfpathlineto{\pgfqpoint{3.653251in}{2.231167in}}%
\pgfpathlineto{\pgfqpoint{3.645273in}{2.220842in}}%
\pgfpathlineto{\pgfqpoint{3.637290in}{2.210539in}}%
\pgfpathlineto{\pgfqpoint{3.629302in}{2.200258in}}%
\pgfpathlineto{\pgfqpoint{3.621309in}{2.189998in}}%
\pgfpathclose%
\pgfusepath{fill}%
\end{pgfscope}%
\begin{pgfscope}%
\pgfpathrectangle{\pgfqpoint{1.150000in}{0.150000in}}{\pgfqpoint{5.700000in}{5.700000in}}%
\pgfusepath{clip}%
\pgfsetbuttcap%
\pgfsetroundjoin%
\definecolor{currentfill}{rgb}{0.180629,0.429975,0.557282}%
\pgfsetfillcolor{currentfill}%
\pgfsetfillopacity{0.800000}%
\pgfsetlinewidth{0.000000pt}%
\definecolor{currentstroke}{rgb}{0.000000,0.000000,0.000000}%
\pgfsetstrokecolor{currentstroke}%
\pgfsetdash{}{0pt}%
\pgfpathmoveto{\pgfqpoint{5.165032in}{3.004459in}}%
\pgfpathlineto{\pgfqpoint{5.179065in}{3.010116in}}%
\pgfpathlineto{\pgfqpoint{5.193113in}{3.015950in}}%
\pgfpathlineto{\pgfqpoint{5.207176in}{3.021959in}}%
\pgfpathlineto{\pgfqpoint{5.221255in}{3.028144in}}%
\pgfpathlineto{\pgfqpoint{5.228680in}{3.035251in}}%
\pgfpathlineto{\pgfqpoint{5.236102in}{3.042457in}}%
\pgfpathlineto{\pgfqpoint{5.243521in}{3.049768in}}%
\pgfpathlineto{\pgfqpoint{5.250937in}{3.057191in}}%
\pgfpathlineto{\pgfqpoint{5.236880in}{3.051560in}}%
\pgfpathlineto{\pgfqpoint{5.222838in}{3.046106in}}%
\pgfpathlineto{\pgfqpoint{5.208810in}{3.040826in}}%
\pgfpathlineto{\pgfqpoint{5.194797in}{3.035722in}}%
\pgfpathlineto{\pgfqpoint{5.187360in}{3.027734in}}%
\pgfpathlineto{\pgfqpoint{5.179920in}{3.019865in}}%
\pgfpathlineto{\pgfqpoint{5.172478in}{3.012109in}}%
\pgfpathlineto{\pgfqpoint{5.165032in}{3.004459in}}%
\pgfpathclose%
\pgfusepath{fill}%
\end{pgfscope}%
\begin{pgfscope}%
\pgfpathrectangle{\pgfqpoint{1.150000in}{0.150000in}}{\pgfqpoint{5.700000in}{5.700000in}}%
\pgfusepath{clip}%
\pgfsetbuttcap%
\pgfsetroundjoin%
\definecolor{currentfill}{rgb}{0.283187,0.125848,0.444960}%
\pgfsetfillcolor{currentfill}%
\pgfsetfillopacity{0.800000}%
\pgfsetlinewidth{0.000000pt}%
\definecolor{currentstroke}{rgb}{0.000000,0.000000,0.000000}%
\pgfsetstrokecolor{currentstroke}%
\pgfsetdash{}{0pt}%
\pgfpathmoveto{\pgfqpoint{3.060253in}{2.279264in}}%
\pgfpathlineto{\pgfqpoint{3.073775in}{2.267705in}}%
\pgfpathlineto{\pgfqpoint{3.087294in}{2.256403in}}%
\pgfpathlineto{\pgfqpoint{3.100810in}{2.245356in}}%
\pgfpathlineto{\pgfqpoint{3.114324in}{2.234562in}}%
\pgfpathlineto{\pgfqpoint{3.122516in}{2.243276in}}%
\pgfpathlineto{\pgfqpoint{3.130699in}{2.252076in}}%
\pgfpathlineto{\pgfqpoint{3.138876in}{2.260961in}}%
\pgfpathlineto{\pgfqpoint{3.147045in}{2.269928in}}%
\pgfpathlineto{\pgfqpoint{3.133549in}{2.280567in}}%
\pgfpathlineto{\pgfqpoint{3.120051in}{2.291460in}}%
\pgfpathlineto{\pgfqpoint{3.106551in}{2.302607in}}%
\pgfpathlineto{\pgfqpoint{3.093048in}{2.314011in}}%
\pgfpathlineto{\pgfqpoint{3.084861in}{2.305186in}}%
\pgfpathlineto{\pgfqpoint{3.076666in}{2.296453in}}%
\pgfpathlineto{\pgfqpoint{3.068463in}{2.287812in}}%
\pgfpathlineto{\pgfqpoint{3.060253in}{2.279264in}}%
\pgfpathclose%
\pgfusepath{fill}%
\end{pgfscope}%
\begin{pgfscope}%
\pgfpathrectangle{\pgfqpoint{1.150000in}{0.150000in}}{\pgfqpoint{5.700000in}{5.700000in}}%
\pgfusepath{clip}%
\pgfsetbuttcap%
\pgfsetroundjoin%
\definecolor{currentfill}{rgb}{0.281446,0.084320,0.407414}%
\pgfsetfillcolor{currentfill}%
\pgfsetfillopacity{0.800000}%
\pgfsetlinewidth{0.000000pt}%
\definecolor{currentstroke}{rgb}{0.000000,0.000000,0.000000}%
\pgfsetstrokecolor{currentstroke}%
\pgfsetdash{}{0pt}%
\pgfpathmoveto{\pgfqpoint{3.395236in}{2.171456in}}%
\pgfpathlineto{\pgfqpoint{3.408716in}{2.165003in}}%
\pgfpathlineto{\pgfqpoint{3.422198in}{2.158777in}}%
\pgfpathlineto{\pgfqpoint{3.435682in}{2.152776in}}%
\pgfpathlineto{\pgfqpoint{3.449168in}{2.146999in}}%
\pgfpathlineto{\pgfqpoint{3.457230in}{2.156878in}}%
\pgfpathlineto{\pgfqpoint{3.465285in}{2.166797in}}%
\pgfpathlineto{\pgfqpoint{3.473335in}{2.176754in}}%
\pgfpathlineto{\pgfqpoint{3.481379in}{2.186750in}}%
\pgfpathlineto{\pgfqpoint{3.467905in}{2.192438in}}%
\pgfpathlineto{\pgfqpoint{3.454434in}{2.198350in}}%
\pgfpathlineto{\pgfqpoint{3.440964in}{2.204487in}}%
\pgfpathlineto{\pgfqpoint{3.427497in}{2.210851in}}%
\pgfpathlineto{\pgfqpoint{3.419441in}{2.200933in}}%
\pgfpathlineto{\pgfqpoint{3.411379in}{2.191061in}}%
\pgfpathlineto{\pgfqpoint{3.403310in}{2.181235in}}%
\pgfpathlineto{\pgfqpoint{3.395236in}{2.171456in}}%
\pgfpathclose%
\pgfusepath{fill}%
\end{pgfscope}%
\begin{pgfscope}%
\pgfpathrectangle{\pgfqpoint{1.150000in}{0.150000in}}{\pgfqpoint{5.700000in}{5.700000in}}%
\pgfusepath{clip}%
\pgfsetbuttcap%
\pgfsetroundjoin%
\definecolor{currentfill}{rgb}{0.281924,0.089666,0.412415}%
\pgfsetfillcolor{currentfill}%
\pgfsetfillopacity{0.800000}%
\pgfsetlinewidth{0.000000pt}%
\definecolor{currentstroke}{rgb}{0.000000,0.000000,0.000000}%
\pgfsetstrokecolor{currentstroke}%
\pgfsetdash{}{0pt}%
\pgfpathmoveto{\pgfqpoint{3.254965in}{2.193730in}}%
\pgfpathlineto{\pgfqpoint{3.268452in}{2.185296in}}%
\pgfpathlineto{\pgfqpoint{3.281939in}{2.177099in}}%
\pgfpathlineto{\pgfqpoint{3.295427in}{2.169138in}}%
\pgfpathlineto{\pgfqpoint{3.308915in}{2.161412in}}%
\pgfpathlineto{\pgfqpoint{3.317029in}{2.170863in}}%
\pgfpathlineto{\pgfqpoint{3.325136in}{2.180372in}}%
\pgfpathlineto{\pgfqpoint{3.333237in}{2.189938in}}%
\pgfpathlineto{\pgfqpoint{3.341331in}{2.199560in}}%
\pgfpathlineto{\pgfqpoint{3.327857in}{2.207165in}}%
\pgfpathlineto{\pgfqpoint{3.314385in}{2.215005in}}%
\pgfpathlineto{\pgfqpoint{3.300912in}{2.223080in}}%
\pgfpathlineto{\pgfqpoint{3.287440in}{2.231392in}}%
\pgfpathlineto{\pgfqpoint{3.279331in}{2.221880in}}%
\pgfpathlineto{\pgfqpoint{3.271216in}{2.212432in}}%
\pgfpathlineto{\pgfqpoint{3.263094in}{2.203048in}}%
\pgfpathlineto{\pgfqpoint{3.254965in}{2.193730in}}%
\pgfpathclose%
\pgfusepath{fill}%
\end{pgfscope}%
\begin{pgfscope}%
\pgfpathrectangle{\pgfqpoint{1.150000in}{0.150000in}}{\pgfqpoint{5.700000in}{5.700000in}}%
\pgfusepath{clip}%
\pgfsetbuttcap%
\pgfsetroundjoin%
\definecolor{currentfill}{rgb}{0.172719,0.448791,0.557885}%
\pgfsetfillcolor{currentfill}%
\pgfsetfillopacity{0.800000}%
\pgfsetlinewidth{0.000000pt}%
\definecolor{currentstroke}{rgb}{0.000000,0.000000,0.000000}%
\pgfsetstrokecolor{currentstroke}%
\pgfsetdash{}{0pt}%
\pgfpathmoveto{\pgfqpoint{5.250937in}{3.057191in}}%
\pgfpathlineto{\pgfqpoint{5.265009in}{3.062996in}}%
\pgfpathlineto{\pgfqpoint{5.279097in}{3.068976in}}%
\pgfpathlineto{\pgfqpoint{5.293200in}{3.075132in}}%
\pgfpathlineto{\pgfqpoint{5.307318in}{3.081462in}}%
\pgfpathlineto{\pgfqpoint{5.314709in}{3.088427in}}%
\pgfpathlineto{\pgfqpoint{5.322098in}{3.095509in}}%
\pgfpathlineto{\pgfqpoint{5.329484in}{3.102715in}}%
\pgfpathlineto{\pgfqpoint{5.336867in}{3.110050in}}%
\pgfpathlineto{\pgfqpoint{5.322771in}{3.104307in}}%
\pgfpathlineto{\pgfqpoint{5.308691in}{3.098739in}}%
\pgfpathlineto{\pgfqpoint{5.294626in}{3.093345in}}%
\pgfpathlineto{\pgfqpoint{5.280576in}{3.088125in}}%
\pgfpathlineto{\pgfqpoint{5.273169in}{3.080192in}}%
\pgfpathlineto{\pgfqpoint{5.265761in}{3.072396in}}%
\pgfpathlineto{\pgfqpoint{5.258350in}{3.064731in}}%
\pgfpathlineto{\pgfqpoint{5.250937in}{3.057191in}}%
\pgfpathclose%
\pgfusepath{fill}%
\end{pgfscope}%
\begin{pgfscope}%
\pgfpathrectangle{\pgfqpoint{1.150000in}{0.150000in}}{\pgfqpoint{5.700000in}{5.700000in}}%
\pgfusepath{clip}%
\pgfsetbuttcap%
\pgfsetroundjoin%
\definecolor{currentfill}{rgb}{0.165117,0.467423,0.558141}%
\pgfsetfillcolor{currentfill}%
\pgfsetfillopacity{0.800000}%
\pgfsetlinewidth{0.000000pt}%
\definecolor{currentstroke}{rgb}{0.000000,0.000000,0.000000}%
\pgfsetstrokecolor{currentstroke}%
\pgfsetdash{}{0pt}%
\pgfpathmoveto{\pgfqpoint{5.336867in}{3.110050in}}%
\pgfpathlineto{\pgfqpoint{5.350978in}{3.115967in}}%
\pgfpathlineto{\pgfqpoint{5.365104in}{3.122057in}}%
\pgfpathlineto{\pgfqpoint{5.379246in}{3.128322in}}%
\pgfpathlineto{\pgfqpoint{5.393404in}{3.134761in}}%
\pgfpathlineto{\pgfqpoint{5.400762in}{3.141625in}}%
\pgfpathlineto{\pgfqpoint{5.408117in}{3.148624in}}%
\pgfpathlineto{\pgfqpoint{5.415471in}{3.155767in}}%
\pgfpathlineto{\pgfqpoint{5.422823in}{3.163060in}}%
\pgfpathlineto{\pgfqpoint{5.408689in}{3.157241in}}%
\pgfpathlineto{\pgfqpoint{5.394572in}{3.151596in}}%
\pgfpathlineto{\pgfqpoint{5.380469in}{3.146124in}}%
\pgfpathlineto{\pgfqpoint{5.366382in}{3.140825in}}%
\pgfpathlineto{\pgfqpoint{5.359006in}{3.132903in}}%
\pgfpathlineto{\pgfqpoint{5.351628in}{3.125137in}}%
\pgfpathlineto{\pgfqpoint{5.344248in}{3.117522in}}%
\pgfpathlineto{\pgfqpoint{5.336867in}{3.110050in}}%
\pgfpathclose%
\pgfusepath{fill}%
\end{pgfscope}%
\begin{pgfscope}%
\pgfpathrectangle{\pgfqpoint{1.150000in}{0.150000in}}{\pgfqpoint{5.700000in}{5.700000in}}%
\pgfusepath{clip}%
\pgfsetbuttcap%
\pgfsetroundjoin%
\definecolor{currentfill}{rgb}{0.278012,0.180367,0.486697}%
\pgfsetfillcolor{currentfill}%
\pgfsetfillopacity{0.800000}%
\pgfsetlinewidth{0.000000pt}%
\definecolor{currentstroke}{rgb}{0.000000,0.000000,0.000000}%
\pgfsetstrokecolor{currentstroke}%
\pgfsetdash{}{0pt}%
\pgfpathmoveto{\pgfqpoint{4.104579in}{2.364686in}}%
\pgfpathlineto{\pgfqpoint{4.118189in}{2.365728in}}%
\pgfpathlineto{\pgfqpoint{4.131808in}{2.366966in}}%
\pgfpathlineto{\pgfqpoint{4.145436in}{2.368399in}}%
\pgfpathlineto{\pgfqpoint{4.159073in}{2.370025in}}%
\pgfpathlineto{\pgfqpoint{4.166904in}{2.380007in}}%
\pgfpathlineto{\pgfqpoint{4.174730in}{2.389975in}}%
\pgfpathlineto{\pgfqpoint{4.182551in}{2.399931in}}%
\pgfpathlineto{\pgfqpoint{4.190366in}{2.409877in}}%
\pgfpathlineto{\pgfqpoint{4.176737in}{2.408383in}}%
\pgfpathlineto{\pgfqpoint{4.163117in}{2.407084in}}%
\pgfpathlineto{\pgfqpoint{4.149506in}{2.405980in}}%
\pgfpathlineto{\pgfqpoint{4.135903in}{2.405070in}}%
\pgfpathlineto{\pgfqpoint{4.128080in}{2.394979in}}%
\pgfpathlineto{\pgfqpoint{4.120251in}{2.384886in}}%
\pgfpathlineto{\pgfqpoint{4.112417in}{2.374789in}}%
\pgfpathlineto{\pgfqpoint{4.104579in}{2.364686in}}%
\pgfpathclose%
\pgfusepath{fill}%
\end{pgfscope}%
\begin{pgfscope}%
\pgfpathrectangle{\pgfqpoint{1.150000in}{0.150000in}}{\pgfqpoint{5.700000in}{5.700000in}}%
\pgfusepath{clip}%
\pgfsetbuttcap%
\pgfsetroundjoin%
\definecolor{currentfill}{rgb}{0.274128,0.199721,0.498911}%
\pgfsetfillcolor{currentfill}%
\pgfsetfillopacity{0.800000}%
\pgfsetlinewidth{0.000000pt}%
\definecolor{currentstroke}{rgb}{0.000000,0.000000,0.000000}%
\pgfsetstrokecolor{currentstroke}%
\pgfsetdash{}{0pt}%
\pgfpathmoveto{\pgfqpoint{4.190366in}{2.409877in}}%
\pgfpathlineto{\pgfqpoint{4.204005in}{2.411563in}}%
\pgfpathlineto{\pgfqpoint{4.217654in}{2.413443in}}%
\pgfpathlineto{\pgfqpoint{4.231312in}{2.415516in}}%
\pgfpathlineto{\pgfqpoint{4.244979in}{2.417781in}}%
\pgfpathlineto{\pgfqpoint{4.252782in}{2.427564in}}%
\pgfpathlineto{\pgfqpoint{4.260579in}{2.437334in}}%
\pgfpathlineto{\pgfqpoint{4.268371in}{2.447093in}}%
\pgfpathlineto{\pgfqpoint{4.276158in}{2.456841in}}%
\pgfpathlineto{\pgfqpoint{4.262499in}{2.454742in}}%
\pgfpathlineto{\pgfqpoint{4.248849in}{2.452835in}}%
\pgfpathlineto{\pgfqpoint{4.235209in}{2.451121in}}%
\pgfpathlineto{\pgfqpoint{4.221578in}{2.449599in}}%
\pgfpathlineto{\pgfqpoint{4.213783in}{2.439673in}}%
\pgfpathlineto{\pgfqpoint{4.205982in}{2.429745in}}%
\pgfpathlineto{\pgfqpoint{4.198177in}{2.419814in}}%
\pgfpathlineto{\pgfqpoint{4.190366in}{2.409877in}}%
\pgfpathclose%
\pgfusepath{fill}%
\end{pgfscope}%
\begin{pgfscope}%
\pgfpathrectangle{\pgfqpoint{1.150000in}{0.150000in}}{\pgfqpoint{5.700000in}{5.700000in}}%
\pgfusepath{clip}%
\pgfsetbuttcap%
\pgfsetroundjoin%
\definecolor{currentfill}{rgb}{0.280868,0.160771,0.472899}%
\pgfsetfillcolor{currentfill}%
\pgfsetfillopacity{0.800000}%
\pgfsetlinewidth{0.000000pt}%
\definecolor{currentstroke}{rgb}{0.000000,0.000000,0.000000}%
\pgfsetstrokecolor{currentstroke}%
\pgfsetdash{}{0pt}%
\pgfpathmoveto{\pgfqpoint{4.018786in}{2.321567in}}%
\pgfpathlineto{\pgfqpoint{4.032370in}{2.321924in}}%
\pgfpathlineto{\pgfqpoint{4.045962in}{2.322479in}}%
\pgfpathlineto{\pgfqpoint{4.059563in}{2.323230in}}%
\pgfpathlineto{\pgfqpoint{4.073172in}{2.324179in}}%
\pgfpathlineto{\pgfqpoint{4.081032in}{2.334323in}}%
\pgfpathlineto{\pgfqpoint{4.088886in}{2.344454in}}%
\pgfpathlineto{\pgfqpoint{4.096735in}{2.354575in}}%
\pgfpathlineto{\pgfqpoint{4.104579in}{2.364686in}}%
\pgfpathlineto{\pgfqpoint{4.090977in}{2.363839in}}%
\pgfpathlineto{\pgfqpoint{4.077384in}{2.363189in}}%
\pgfpathlineto{\pgfqpoint{4.063800in}{2.362736in}}%
\pgfpathlineto{\pgfqpoint{4.050223in}{2.362480in}}%
\pgfpathlineto{\pgfqpoint{4.042371in}{2.352256in}}%
\pgfpathlineto{\pgfqpoint{4.034515in}{2.342030in}}%
\pgfpathlineto{\pgfqpoint{4.026653in}{2.331801in}}%
\pgfpathlineto{\pgfqpoint{4.018786in}{2.321567in}}%
\pgfpathclose%
\pgfusepath{fill}%
\end{pgfscope}%
\begin{pgfscope}%
\pgfpathrectangle{\pgfqpoint{1.150000in}{0.150000in}}{\pgfqpoint{5.700000in}{5.700000in}}%
\pgfusepath{clip}%
\pgfsetbuttcap%
\pgfsetroundjoin%
\definecolor{currentfill}{rgb}{0.267968,0.223549,0.512008}%
\pgfsetfillcolor{currentfill}%
\pgfsetfillopacity{0.800000}%
\pgfsetlinewidth{0.000000pt}%
\definecolor{currentstroke}{rgb}{0.000000,0.000000,0.000000}%
\pgfsetstrokecolor{currentstroke}%
\pgfsetdash{}{0pt}%
\pgfpathmoveto{\pgfqpoint{4.276158in}{2.456841in}}%
\pgfpathlineto{\pgfqpoint{4.289828in}{2.459132in}}%
\pgfpathlineto{\pgfqpoint{4.303508in}{2.461613in}}%
\pgfpathlineto{\pgfqpoint{4.317198in}{2.464285in}}%
\pgfpathlineto{\pgfqpoint{4.330899in}{2.467148in}}%
\pgfpathlineto{\pgfqpoint{4.338672in}{2.476703in}}%
\pgfpathlineto{\pgfqpoint{4.346441in}{2.486246in}}%
\pgfpathlineto{\pgfqpoint{4.354204in}{2.495779in}}%
\pgfpathlineto{\pgfqpoint{4.361962in}{2.505305in}}%
\pgfpathlineto{\pgfqpoint{4.348270in}{2.502640in}}%
\pgfpathlineto{\pgfqpoint{4.334588in}{2.500166in}}%
\pgfpathlineto{\pgfqpoint{4.320916in}{2.497882in}}%
\pgfpathlineto{\pgfqpoint{4.307255in}{2.495788in}}%
\pgfpathlineto{\pgfqpoint{4.299489in}{2.486054in}}%
\pgfpathlineto{\pgfqpoint{4.291717in}{2.476320in}}%
\pgfpathlineto{\pgfqpoint{4.283940in}{2.466583in}}%
\pgfpathlineto{\pgfqpoint{4.276158in}{2.456841in}}%
\pgfpathclose%
\pgfusepath{fill}%
\end{pgfscope}%
\begin{pgfscope}%
\pgfpathrectangle{\pgfqpoint{1.150000in}{0.150000in}}{\pgfqpoint{5.700000in}{5.700000in}}%
\pgfusepath{clip}%
\pgfsetbuttcap%
\pgfsetroundjoin%
\definecolor{currentfill}{rgb}{0.208623,0.367752,0.552675}%
\pgfsetfillcolor{currentfill}%
\pgfsetfillopacity{0.800000}%
\pgfsetlinewidth{0.000000pt}%
\definecolor{currentstroke}{rgb}{0.000000,0.000000,0.000000}%
\pgfsetstrokecolor{currentstroke}%
\pgfsetdash{}{0pt}%
\pgfpathmoveto{\pgfqpoint{2.569483in}{2.884880in}}%
\pgfpathlineto{\pgfqpoint{2.583281in}{2.862536in}}%
\pgfpathlineto{\pgfqpoint{2.597065in}{2.840541in}}%
\pgfpathlineto{\pgfqpoint{2.610837in}{2.818893in}}%
\pgfpathlineto{\pgfqpoint{2.624596in}{2.797587in}}%
\pgfpathlineto{\pgfqpoint{2.632996in}{2.804756in}}%
\pgfpathlineto{\pgfqpoint{2.641385in}{2.812082in}}%
\pgfpathlineto{\pgfqpoint{2.649763in}{2.819565in}}%
\pgfpathlineto{\pgfqpoint{2.658131in}{2.827204in}}%
\pgfpathlineto{\pgfqpoint{2.644400in}{2.848343in}}%
\pgfpathlineto{\pgfqpoint{2.630656in}{2.869825in}}%
\pgfpathlineto{\pgfqpoint{2.616900in}{2.891652in}}%
\pgfpathlineto{\pgfqpoint{2.603131in}{2.913829in}}%
\pgfpathlineto{\pgfqpoint{2.594736in}{2.906345in}}%
\pgfpathlineto{\pgfqpoint{2.586329in}{2.899024in}}%
\pgfpathlineto{\pgfqpoint{2.577912in}{2.891869in}}%
\pgfpathlineto{\pgfqpoint{2.569483in}{2.884880in}}%
\pgfpathclose%
\pgfusepath{fill}%
\end{pgfscope}%
\begin{pgfscope}%
\pgfpathrectangle{\pgfqpoint{1.150000in}{0.150000in}}{\pgfqpoint{5.700000in}{5.700000in}}%
\pgfusepath{clip}%
\pgfsetbuttcap%
\pgfsetroundjoin%
\definecolor{currentfill}{rgb}{0.281446,0.084320,0.407414}%
\pgfsetfillcolor{currentfill}%
\pgfsetfillopacity{0.800000}%
\pgfsetlinewidth{0.000000pt}%
\definecolor{currentstroke}{rgb}{0.000000,0.000000,0.000000}%
\pgfsetstrokecolor{currentstroke}%
\pgfsetdash{}{0pt}%
\pgfpathmoveto{\pgfqpoint{3.535302in}{2.166216in}}%
\pgfpathlineto{\pgfqpoint{3.548791in}{2.161632in}}%
\pgfpathlineto{\pgfqpoint{3.562284in}{2.157265in}}%
\pgfpathlineto{\pgfqpoint{3.575780in}{2.153115in}}%
\pgfpathlineto{\pgfqpoint{3.589280in}{2.149180in}}%
\pgfpathlineto{\pgfqpoint{3.597295in}{2.159352in}}%
\pgfpathlineto{\pgfqpoint{3.605305in}{2.169545in}}%
\pgfpathlineto{\pgfqpoint{3.613310in}{2.179761in}}%
\pgfpathlineto{\pgfqpoint{3.621309in}{2.189998in}}%
\pgfpathlineto{\pgfqpoint{3.607819in}{2.193876in}}%
\pgfpathlineto{\pgfqpoint{3.594334in}{2.197970in}}%
\pgfpathlineto{\pgfqpoint{3.580852in}{2.202279in}}%
\pgfpathlineto{\pgfqpoint{3.567374in}{2.206806in}}%
\pgfpathlineto{\pgfqpoint{3.559365in}{2.196614in}}%
\pgfpathlineto{\pgfqpoint{3.551350in}{2.186452in}}%
\pgfpathlineto{\pgfqpoint{3.543329in}{2.176319in}}%
\pgfpathlineto{\pgfqpoint{3.535302in}{2.166216in}}%
\pgfpathclose%
\pgfusepath{fill}%
\end{pgfscope}%
\begin{pgfscope}%
\pgfpathrectangle{\pgfqpoint{1.150000in}{0.150000in}}{\pgfqpoint{5.700000in}{5.700000in}}%
\pgfusepath{clip}%
\pgfsetbuttcap%
\pgfsetroundjoin%
\definecolor{currentfill}{rgb}{0.157729,0.485932,0.558013}%
\pgfsetfillcolor{currentfill}%
\pgfsetfillopacity{0.800000}%
\pgfsetlinewidth{0.000000pt}%
\definecolor{currentstroke}{rgb}{0.000000,0.000000,0.000000}%
\pgfsetstrokecolor{currentstroke}%
\pgfsetdash{}{0pt}%
\pgfpathmoveto{\pgfqpoint{5.422823in}{3.163060in}}%
\pgfpathlineto{\pgfqpoint{5.436971in}{3.169051in}}%
\pgfpathlineto{\pgfqpoint{5.451136in}{3.175216in}}%
\pgfpathlineto{\pgfqpoint{5.465317in}{3.181553in}}%
\pgfpathlineto{\pgfqpoint{5.479514in}{3.188064in}}%
\pgfpathlineto{\pgfqpoint{5.486838in}{3.194873in}}%
\pgfpathlineto{\pgfqpoint{5.494162in}{3.201838in}}%
\pgfpathlineto{\pgfqpoint{5.501484in}{3.208967in}}%
\pgfpathlineto{\pgfqpoint{5.508806in}{3.216267in}}%
\pgfpathlineto{\pgfqpoint{5.494635in}{3.210409in}}%
\pgfpathlineto{\pgfqpoint{5.480481in}{3.204724in}}%
\pgfpathlineto{\pgfqpoint{5.466342in}{3.199211in}}%
\pgfpathlineto{\pgfqpoint{5.452219in}{3.193870in}}%
\pgfpathlineto{\pgfqpoint{5.444871in}{3.185907in}}%
\pgfpathlineto{\pgfqpoint{5.437522in}{3.178122in}}%
\pgfpathlineto{\pgfqpoint{5.430173in}{3.170509in}}%
\pgfpathlineto{\pgfqpoint{5.422823in}{3.163060in}}%
\pgfpathclose%
\pgfusepath{fill}%
\end{pgfscope}%
\begin{pgfscope}%
\pgfpathrectangle{\pgfqpoint{1.150000in}{0.150000in}}{\pgfqpoint{5.700000in}{5.700000in}}%
\pgfusepath{clip}%
\pgfsetbuttcap%
\pgfsetroundjoin%
\definecolor{currentfill}{rgb}{0.262138,0.242286,0.520837}%
\pgfsetfillcolor{currentfill}%
\pgfsetfillopacity{0.800000}%
\pgfsetlinewidth{0.000000pt}%
\definecolor{currentstroke}{rgb}{0.000000,0.000000,0.000000}%
\pgfsetstrokecolor{currentstroke}%
\pgfsetdash{}{0pt}%
\pgfpathmoveto{\pgfqpoint{4.361962in}{2.505305in}}%
\pgfpathlineto{\pgfqpoint{4.375665in}{2.508159in}}%
\pgfpathlineto{\pgfqpoint{4.389378in}{2.511202in}}%
\pgfpathlineto{\pgfqpoint{4.403102in}{2.514433in}}%
\pgfpathlineto{\pgfqpoint{4.416838in}{2.517854in}}%
\pgfpathlineto{\pgfqpoint{4.424582in}{2.527156in}}%
\pgfpathlineto{\pgfqpoint{4.432320in}{2.536448in}}%
\pgfpathlineto{\pgfqpoint{4.440054in}{2.545733in}}%
\pgfpathlineto{\pgfqpoint{4.447782in}{2.555014in}}%
\pgfpathlineto{\pgfqpoint{4.434055in}{2.551824in}}%
\pgfpathlineto{\pgfqpoint{4.420340in}{2.548823in}}%
\pgfpathlineto{\pgfqpoint{4.406636in}{2.546009in}}%
\pgfpathlineto{\pgfqpoint{4.392942in}{2.543385in}}%
\pgfpathlineto{\pgfqpoint{4.385204in}{2.533862in}}%
\pgfpathlineto{\pgfqpoint{4.377462in}{2.524343in}}%
\pgfpathlineto{\pgfqpoint{4.369714in}{2.514825in}}%
\pgfpathlineto{\pgfqpoint{4.361962in}{2.505305in}}%
\pgfpathclose%
\pgfusepath{fill}%
\end{pgfscope}%
\begin{pgfscope}%
\pgfpathrectangle{\pgfqpoint{1.150000in}{0.150000in}}{\pgfqpoint{5.700000in}{5.700000in}}%
\pgfusepath{clip}%
\pgfsetbuttcap%
\pgfsetroundjoin%
\definecolor{currentfill}{rgb}{0.282623,0.140926,0.457517}%
\pgfsetfillcolor{currentfill}%
\pgfsetfillopacity{0.800000}%
\pgfsetlinewidth{0.000000pt}%
\definecolor{currentstroke}{rgb}{0.000000,0.000000,0.000000}%
\pgfsetstrokecolor{currentstroke}%
\pgfsetdash{}{0pt}%
\pgfpathmoveto{\pgfqpoint{3.932977in}{2.280844in}}%
\pgfpathlineto{\pgfqpoint{3.946538in}{2.280473in}}%
\pgfpathlineto{\pgfqpoint{3.960107in}{2.280302in}}%
\pgfpathlineto{\pgfqpoint{3.973683in}{2.280331in}}%
\pgfpathlineto{\pgfqpoint{3.987267in}{2.280560in}}%
\pgfpathlineto{\pgfqpoint{3.995155in}{2.290825in}}%
\pgfpathlineto{\pgfqpoint{4.003037in}{2.301081in}}%
\pgfpathlineto{\pgfqpoint{4.010914in}{2.311328in}}%
\pgfpathlineto{\pgfqpoint{4.018786in}{2.321567in}}%
\pgfpathlineto{\pgfqpoint{4.005210in}{2.321409in}}%
\pgfpathlineto{\pgfqpoint{3.991641in}{2.321450in}}%
\pgfpathlineto{\pgfqpoint{3.978081in}{2.321690in}}%
\pgfpathlineto{\pgfqpoint{3.964528in}{2.322130in}}%
\pgfpathlineto{\pgfqpoint{3.956648in}{2.311809in}}%
\pgfpathlineto{\pgfqpoint{3.948763in}{2.301489in}}%
\pgfpathlineto{\pgfqpoint{3.940872in}{2.291167in}}%
\pgfpathlineto{\pgfqpoint{3.932977in}{2.280844in}}%
\pgfpathclose%
\pgfusepath{fill}%
\end{pgfscope}%
\begin{pgfscope}%
\pgfpathrectangle{\pgfqpoint{1.150000in}{0.150000in}}{\pgfqpoint{5.700000in}{5.700000in}}%
\pgfusepath{clip}%
\pgfsetbuttcap%
\pgfsetroundjoin%
\definecolor{currentfill}{rgb}{0.283091,0.110553,0.431554}%
\pgfsetfillcolor{currentfill}%
\pgfsetfillopacity{0.800000}%
\pgfsetlinewidth{0.000000pt}%
\definecolor{currentstroke}{rgb}{0.000000,0.000000,0.000000}%
\pgfsetstrokecolor{currentstroke}%
\pgfsetdash{}{0pt}%
\pgfpathmoveto{\pgfqpoint{3.114324in}{2.234562in}}%
\pgfpathlineto{\pgfqpoint{3.127836in}{2.224020in}}%
\pgfpathlineto{\pgfqpoint{3.141347in}{2.213727in}}%
\pgfpathlineto{\pgfqpoint{3.154855in}{2.203684in}}%
\pgfpathlineto{\pgfqpoint{3.168363in}{2.193886in}}%
\pgfpathlineto{\pgfqpoint{3.176536in}{2.202767in}}%
\pgfpathlineto{\pgfqpoint{3.184702in}{2.211725in}}%
\pgfpathlineto{\pgfqpoint{3.192861in}{2.220760in}}%
\pgfpathlineto{\pgfqpoint{3.201013in}{2.229870in}}%
\pgfpathlineto{\pgfqpoint{3.187523in}{2.239513in}}%
\pgfpathlineto{\pgfqpoint{3.174032in}{2.249403in}}%
\pgfpathlineto{\pgfqpoint{3.160539in}{2.259541in}}%
\pgfpathlineto{\pgfqpoint{3.147045in}{2.269928in}}%
\pgfpathlineto{\pgfqpoint{3.138876in}{2.260961in}}%
\pgfpathlineto{\pgfqpoint{3.130699in}{2.252076in}}%
\pgfpathlineto{\pgfqpoint{3.122516in}{2.243276in}}%
\pgfpathlineto{\pgfqpoint{3.114324in}{2.234562in}}%
\pgfpathclose%
\pgfusepath{fill}%
\end{pgfscope}%
\begin{pgfscope}%
\pgfpathrectangle{\pgfqpoint{1.150000in}{0.150000in}}{\pgfqpoint{5.700000in}{5.700000in}}%
\pgfusepath{clip}%
\pgfsetbuttcap%
\pgfsetroundjoin%
\definecolor{currentfill}{rgb}{0.253935,0.265254,0.529983}%
\pgfsetfillcolor{currentfill}%
\pgfsetfillopacity{0.800000}%
\pgfsetlinewidth{0.000000pt}%
\definecolor{currentstroke}{rgb}{0.000000,0.000000,0.000000}%
\pgfsetstrokecolor{currentstroke}%
\pgfsetdash{}{0pt}%
\pgfpathmoveto{\pgfqpoint{4.447782in}{2.555014in}}%
\pgfpathlineto{\pgfqpoint{4.461519in}{2.558392in}}%
\pgfpathlineto{\pgfqpoint{4.475268in}{2.561957in}}%
\pgfpathlineto{\pgfqpoint{4.489029in}{2.565709in}}%
\pgfpathlineto{\pgfqpoint{4.502801in}{2.569647in}}%
\pgfpathlineto{\pgfqpoint{4.510514in}{2.578677in}}%
\pgfpathlineto{\pgfqpoint{4.518222in}{2.587700in}}%
\pgfpathlineto{\pgfqpoint{4.525925in}{2.596720in}}%
\pgfpathlineto{\pgfqpoint{4.533623in}{2.605740in}}%
\pgfpathlineto{\pgfqpoint{4.519861in}{2.602065in}}%
\pgfpathlineto{\pgfqpoint{4.506110in}{2.598575in}}%
\pgfpathlineto{\pgfqpoint{4.492371in}{2.595272in}}%
\pgfpathlineto{\pgfqpoint{4.478643in}{2.592156in}}%
\pgfpathlineto{\pgfqpoint{4.470935in}{2.582862in}}%
\pgfpathlineto{\pgfqpoint{4.463223in}{2.573575in}}%
\pgfpathlineto{\pgfqpoint{4.455505in}{2.564294in}}%
\pgfpathlineto{\pgfqpoint{4.447782in}{2.555014in}}%
\pgfpathclose%
\pgfusepath{fill}%
\end{pgfscope}%
\begin{pgfscope}%
\pgfpathrectangle{\pgfqpoint{1.150000in}{0.150000in}}{\pgfqpoint{5.700000in}{5.700000in}}%
\pgfusepath{clip}%
\pgfsetbuttcap%
\pgfsetroundjoin%
\definecolor{currentfill}{rgb}{0.150476,0.504369,0.557430}%
\pgfsetfillcolor{currentfill}%
\pgfsetfillopacity{0.800000}%
\pgfsetlinewidth{0.000000pt}%
\definecolor{currentstroke}{rgb}{0.000000,0.000000,0.000000}%
\pgfsetstrokecolor{currentstroke}%
\pgfsetdash{}{0pt}%
\pgfpathmoveto{\pgfqpoint{5.508806in}{3.216267in}}%
\pgfpathlineto{\pgfqpoint{5.522992in}{3.222297in}}%
\pgfpathlineto{\pgfqpoint{5.537194in}{3.228499in}}%
\pgfpathlineto{\pgfqpoint{5.551413in}{3.234873in}}%
\pgfpathlineto{\pgfqpoint{5.565647in}{3.241419in}}%
\pgfpathlineto{\pgfqpoint{5.572941in}{3.248224in}}%
\pgfpathlineto{\pgfqpoint{5.580234in}{3.255208in}}%
\pgfpathlineto{\pgfqpoint{5.587526in}{3.262378in}}%
\pgfpathlineto{\pgfqpoint{5.594819in}{3.269742in}}%
\pgfpathlineto{\pgfqpoint{5.580613in}{3.263882in}}%
\pgfpathlineto{\pgfqpoint{5.566423in}{3.258192in}}%
\pgfpathlineto{\pgfqpoint{5.552248in}{3.252674in}}%
\pgfpathlineto{\pgfqpoint{5.538090in}{3.247327in}}%
\pgfpathlineto{\pgfqpoint{5.530768in}{3.239268in}}%
\pgfpathlineto{\pgfqpoint{5.523447in}{3.231410in}}%
\pgfpathlineto{\pgfqpoint{5.516127in}{3.223746in}}%
\pgfpathlineto{\pgfqpoint{5.508806in}{3.216267in}}%
\pgfpathclose%
\pgfusepath{fill}%
\end{pgfscope}%
\begin{pgfscope}%
\pgfpathrectangle{\pgfqpoint{1.150000in}{0.150000in}}{\pgfqpoint{5.700000in}{5.700000in}}%
\pgfusepath{clip}%
\pgfsetbuttcap%
\pgfsetroundjoin%
\definecolor{currentfill}{rgb}{0.283187,0.125848,0.444960}%
\pgfsetfillcolor{currentfill}%
\pgfsetfillopacity{0.800000}%
\pgfsetlinewidth{0.000000pt}%
\definecolor{currentstroke}{rgb}{0.000000,0.000000,0.000000}%
\pgfsetstrokecolor{currentstroke}%
\pgfsetdash{}{0pt}%
\pgfpathmoveto{\pgfqpoint{3.847139in}{2.242861in}}%
\pgfpathlineto{\pgfqpoint{3.860681in}{2.241719in}}%
\pgfpathlineto{\pgfqpoint{3.874229in}{2.240781in}}%
\pgfpathlineto{\pgfqpoint{3.887784in}{2.240045in}}%
\pgfpathlineto{\pgfqpoint{3.901346in}{2.239511in}}%
\pgfpathlineto{\pgfqpoint{3.909261in}{2.249852in}}%
\pgfpathlineto{\pgfqpoint{3.917172in}{2.260187in}}%
\pgfpathlineto{\pgfqpoint{3.925077in}{2.270517in}}%
\pgfpathlineto{\pgfqpoint{3.932977in}{2.280844in}}%
\pgfpathlineto{\pgfqpoint{3.919423in}{2.281416in}}%
\pgfpathlineto{\pgfqpoint{3.905877in}{2.282190in}}%
\pgfpathlineto{\pgfqpoint{3.892337in}{2.283166in}}%
\pgfpathlineto{\pgfqpoint{3.878804in}{2.284345in}}%
\pgfpathlineto{\pgfqpoint{3.870896in}{2.273969in}}%
\pgfpathlineto{\pgfqpoint{3.862982in}{2.263597in}}%
\pgfpathlineto{\pgfqpoint{3.855063in}{2.253228in}}%
\pgfpathlineto{\pgfqpoint{3.847139in}{2.242861in}}%
\pgfpathclose%
\pgfusepath{fill}%
\end{pgfscope}%
\begin{pgfscope}%
\pgfpathrectangle{\pgfqpoint{1.150000in}{0.150000in}}{\pgfqpoint{5.700000in}{5.700000in}}%
\pgfusepath{clip}%
\pgfsetbuttcap%
\pgfsetroundjoin%
\definecolor{currentfill}{rgb}{0.244972,0.287675,0.537260}%
\pgfsetfillcolor{currentfill}%
\pgfsetfillopacity{0.800000}%
\pgfsetlinewidth{0.000000pt}%
\definecolor{currentstroke}{rgb}{0.000000,0.000000,0.000000}%
\pgfsetstrokecolor{currentstroke}%
\pgfsetdash{}{0pt}%
\pgfpathmoveto{\pgfqpoint{4.533623in}{2.605740in}}%
\pgfpathlineto{\pgfqpoint{4.547397in}{2.609602in}}%
\pgfpathlineto{\pgfqpoint{4.561183in}{2.613650in}}%
\pgfpathlineto{\pgfqpoint{4.574980in}{2.617883in}}%
\pgfpathlineto{\pgfqpoint{4.588791in}{2.622302in}}%
\pgfpathlineto{\pgfqpoint{4.596473in}{2.631043in}}%
\pgfpathlineto{\pgfqpoint{4.604150in}{2.639784in}}%
\pgfpathlineto{\pgfqpoint{4.611821in}{2.648527in}}%
\pgfpathlineto{\pgfqpoint{4.619487in}{2.657277in}}%
\pgfpathlineto{\pgfqpoint{4.605688in}{2.653154in}}%
\pgfpathlineto{\pgfqpoint{4.591901in}{2.649215in}}%
\pgfpathlineto{\pgfqpoint{4.578125in}{2.645462in}}%
\pgfpathlineto{\pgfqpoint{4.564362in}{2.641894in}}%
\pgfpathlineto{\pgfqpoint{4.556685in}{2.632838in}}%
\pgfpathlineto{\pgfqpoint{4.549003in}{2.623796in}}%
\pgfpathlineto{\pgfqpoint{4.541315in}{2.614765in}}%
\pgfpathlineto{\pgfqpoint{4.533623in}{2.605740in}}%
\pgfpathclose%
\pgfusepath{fill}%
\end{pgfscope}%
\begin{pgfscope}%
\pgfpathrectangle{\pgfqpoint{1.150000in}{0.150000in}}{\pgfqpoint{5.700000in}{5.700000in}}%
\pgfusepath{clip}%
\pgfsetbuttcap%
\pgfsetroundjoin%
\definecolor{currentfill}{rgb}{0.143343,0.522773,0.556295}%
\pgfsetfillcolor{currentfill}%
\pgfsetfillopacity{0.800000}%
\pgfsetlinewidth{0.000000pt}%
\definecolor{currentstroke}{rgb}{0.000000,0.000000,0.000000}%
\pgfsetstrokecolor{currentstroke}%
\pgfsetdash{}{0pt}%
\pgfpathmoveto{\pgfqpoint{5.594819in}{3.269742in}}%
\pgfpathlineto{\pgfqpoint{5.609042in}{3.275774in}}%
\pgfpathlineto{\pgfqpoint{5.623281in}{3.281977in}}%
\pgfpathlineto{\pgfqpoint{5.637536in}{3.288352in}}%
\pgfpathlineto{\pgfqpoint{5.651808in}{3.294897in}}%
\pgfpathlineto{\pgfqpoint{5.659071in}{3.301757in}}%
\pgfpathlineto{\pgfqpoint{5.666336in}{3.308818in}}%
\pgfpathlineto{\pgfqpoint{5.673601in}{3.316090in}}%
\pgfpathlineto{\pgfqpoint{5.680867in}{3.323580in}}%
\pgfpathlineto{\pgfqpoint{5.666626in}{3.317752in}}%
\pgfpathlineto{\pgfqpoint{5.652401in}{3.312095in}}%
\pgfpathlineto{\pgfqpoint{5.638192in}{3.306607in}}%
\pgfpathlineto{\pgfqpoint{5.623999in}{3.301291in}}%
\pgfpathlineto{\pgfqpoint{5.616702in}{3.293074in}}%
\pgfpathlineto{\pgfqpoint{5.609407in}{3.285082in}}%
\pgfpathlineto{\pgfqpoint{5.602113in}{3.277308in}}%
\pgfpathlineto{\pgfqpoint{5.594819in}{3.269742in}}%
\pgfpathclose%
\pgfusepath{fill}%
\end{pgfscope}%
\begin{pgfscope}%
\pgfpathrectangle{\pgfqpoint{1.150000in}{0.150000in}}{\pgfqpoint{5.700000in}{5.700000in}}%
\pgfusepath{clip}%
\pgfsetbuttcap%
\pgfsetroundjoin%
\definecolor{currentfill}{rgb}{0.237441,0.305202,0.541921}%
\pgfsetfillcolor{currentfill}%
\pgfsetfillopacity{0.800000}%
\pgfsetlinewidth{0.000000pt}%
\definecolor{currentstroke}{rgb}{0.000000,0.000000,0.000000}%
\pgfsetstrokecolor{currentstroke}%
\pgfsetdash{}{0pt}%
\pgfpathmoveto{\pgfqpoint{4.619487in}{2.657277in}}%
\pgfpathlineto{\pgfqpoint{4.633299in}{2.661584in}}%
\pgfpathlineto{\pgfqpoint{4.647123in}{2.666076in}}%
\pgfpathlineto{\pgfqpoint{4.660960in}{2.670752in}}%
\pgfpathlineto{\pgfqpoint{4.674809in}{2.675611in}}%
\pgfpathlineto{\pgfqpoint{4.682459in}{2.684055in}}%
\pgfpathlineto{\pgfqpoint{4.690104in}{2.692505in}}%
\pgfpathlineto{\pgfqpoint{4.697743in}{2.700965in}}%
\pgfpathlineto{\pgfqpoint{4.705377in}{2.709439in}}%
\pgfpathlineto{\pgfqpoint{4.691540in}{2.704908in}}%
\pgfpathlineto{\pgfqpoint{4.677715in}{2.700560in}}%
\pgfpathlineto{\pgfqpoint{4.663902in}{2.696395in}}%
\pgfpathlineto{\pgfqpoint{4.650102in}{2.692414in}}%
\pgfpathlineto{\pgfqpoint{4.642456in}{2.683602in}}%
\pgfpathlineto{\pgfqpoint{4.634805in}{2.674810in}}%
\pgfpathlineto{\pgfqpoint{4.627149in}{2.666037in}}%
\pgfpathlineto{\pgfqpoint{4.619487in}{2.657277in}}%
\pgfpathclose%
\pgfusepath{fill}%
\end{pgfscope}%
\begin{pgfscope}%
\pgfpathrectangle{\pgfqpoint{1.150000in}{0.150000in}}{\pgfqpoint{5.700000in}{5.700000in}}%
\pgfusepath{clip}%
\pgfsetbuttcap%
\pgfsetroundjoin%
\definecolor{currentfill}{rgb}{0.281446,0.084320,0.407414}%
\pgfsetfillcolor{currentfill}%
\pgfsetfillopacity{0.800000}%
\pgfsetlinewidth{0.000000pt}%
\definecolor{currentstroke}{rgb}{0.000000,0.000000,0.000000}%
\pgfsetstrokecolor{currentstroke}%
\pgfsetdash{}{0pt}%
\pgfpathmoveto{\pgfqpoint{3.308915in}{2.161412in}}%
\pgfpathlineto{\pgfqpoint{3.322404in}{2.153919in}}%
\pgfpathlineto{\pgfqpoint{3.335894in}{2.146657in}}%
\pgfpathlineto{\pgfqpoint{3.349385in}{2.139626in}}%
\pgfpathlineto{\pgfqpoint{3.362877in}{2.132825in}}%
\pgfpathlineto{\pgfqpoint{3.370976in}{2.142409in}}%
\pgfpathlineto{\pgfqpoint{3.379069in}{2.152042in}}%
\pgfpathlineto{\pgfqpoint{3.387156in}{2.161725in}}%
\pgfpathlineto{\pgfqpoint{3.395236in}{2.171456in}}%
\pgfpathlineto{\pgfqpoint{3.381758in}{2.178137in}}%
\pgfpathlineto{\pgfqpoint{3.368281in}{2.185047in}}%
\pgfpathlineto{\pgfqpoint{3.354805in}{2.192188in}}%
\pgfpathlineto{\pgfqpoint{3.341331in}{2.199560in}}%
\pgfpathlineto{\pgfqpoint{3.333237in}{2.189938in}}%
\pgfpathlineto{\pgfqpoint{3.325136in}{2.180372in}}%
\pgfpathlineto{\pgfqpoint{3.317029in}{2.170863in}}%
\pgfpathlineto{\pgfqpoint{3.308915in}{2.161412in}}%
\pgfpathclose%
\pgfusepath{fill}%
\end{pgfscope}%
\begin{pgfscope}%
\pgfpathrectangle{\pgfqpoint{1.150000in}{0.150000in}}{\pgfqpoint{5.700000in}{5.700000in}}%
\pgfusepath{clip}%
\pgfsetbuttcap%
\pgfsetroundjoin%
\definecolor{currentfill}{rgb}{0.283091,0.110553,0.431554}%
\pgfsetfillcolor{currentfill}%
\pgfsetfillopacity{0.800000}%
\pgfsetlinewidth{0.000000pt}%
\definecolor{currentstroke}{rgb}{0.000000,0.000000,0.000000}%
\pgfsetstrokecolor{currentstroke}%
\pgfsetdash{}{0pt}%
\pgfpathmoveto{\pgfqpoint{3.761257in}{2.207988in}}%
\pgfpathlineto{\pgfqpoint{3.774782in}{2.206032in}}%
\pgfpathlineto{\pgfqpoint{3.788313in}{2.204283in}}%
\pgfpathlineto{\pgfqpoint{3.801850in}{2.202739in}}%
\pgfpathlineto{\pgfqpoint{3.815393in}{2.201399in}}%
\pgfpathlineto{\pgfqpoint{3.823337in}{2.211765in}}%
\pgfpathlineto{\pgfqpoint{3.831276in}{2.222130in}}%
\pgfpathlineto{\pgfqpoint{3.839211in}{2.232495in}}%
\pgfpathlineto{\pgfqpoint{3.847139in}{2.242861in}}%
\pgfpathlineto{\pgfqpoint{3.833605in}{2.244207in}}%
\pgfpathlineto{\pgfqpoint{3.820076in}{2.245757in}}%
\pgfpathlineto{\pgfqpoint{3.806554in}{2.247513in}}%
\pgfpathlineto{\pgfqpoint{3.793038in}{2.249475in}}%
\pgfpathlineto{\pgfqpoint{3.785101in}{2.239091in}}%
\pgfpathlineto{\pgfqpoint{3.777158in}{2.228716in}}%
\pgfpathlineto{\pgfqpoint{3.769210in}{2.218348in}}%
\pgfpathlineto{\pgfqpoint{3.761257in}{2.207988in}}%
\pgfpathclose%
\pgfusepath{fill}%
\end{pgfscope}%
\begin{pgfscope}%
\pgfpathrectangle{\pgfqpoint{1.150000in}{0.150000in}}{\pgfqpoint{5.700000in}{5.700000in}}%
\pgfusepath{clip}%
\pgfsetbuttcap%
\pgfsetroundjoin%
\definecolor{currentfill}{rgb}{0.270595,0.214069,0.507052}%
\pgfsetfillcolor{currentfill}%
\pgfsetfillopacity{0.800000}%
\pgfsetlinewidth{0.000000pt}%
\definecolor{currentstroke}{rgb}{0.000000,0.000000,0.000000}%
\pgfsetstrokecolor{currentstroke}%
\pgfsetdash{}{0pt}%
\pgfpathmoveto{\pgfqpoint{2.810058in}{2.470569in}}%
\pgfpathlineto{\pgfqpoint{2.823681in}{2.454432in}}%
\pgfpathlineto{\pgfqpoint{2.837297in}{2.438586in}}%
\pgfpathlineto{\pgfqpoint{2.850907in}{2.423029in}}%
\pgfpathlineto{\pgfqpoint{2.864510in}{2.407759in}}%
\pgfpathlineto{\pgfqpoint{2.872819in}{2.415316in}}%
\pgfpathlineto{\pgfqpoint{2.881119in}{2.422997in}}%
\pgfpathlineto{\pgfqpoint{2.889410in}{2.430799in}}%
\pgfpathlineto{\pgfqpoint{2.897692in}{2.438721in}}%
\pgfpathlineto{\pgfqpoint{2.884113in}{2.453801in}}%
\pgfpathlineto{\pgfqpoint{2.870527in}{2.469166in}}%
\pgfpathlineto{\pgfqpoint{2.856935in}{2.484820in}}%
\pgfpathlineto{\pgfqpoint{2.843336in}{2.500765in}}%
\pgfpathlineto{\pgfqpoint{2.835031in}{2.493023in}}%
\pgfpathlineto{\pgfqpoint{2.826716in}{2.485408in}}%
\pgfpathlineto{\pgfqpoint{2.818392in}{2.477923in}}%
\pgfpathlineto{\pgfqpoint{2.810058in}{2.470569in}}%
\pgfpathclose%
\pgfusepath{fill}%
\end{pgfscope}%
\begin{pgfscope}%
\pgfpathrectangle{\pgfqpoint{1.150000in}{0.150000in}}{\pgfqpoint{5.700000in}{5.700000in}}%
\pgfusepath{clip}%
\pgfsetbuttcap%
\pgfsetroundjoin%
\definecolor{currentfill}{rgb}{0.262138,0.242286,0.520837}%
\pgfsetfillcolor{currentfill}%
\pgfsetfillopacity{0.800000}%
\pgfsetlinewidth{0.000000pt}%
\definecolor{currentstroke}{rgb}{0.000000,0.000000,0.000000}%
\pgfsetstrokecolor{currentstroke}%
\pgfsetdash{}{0pt}%
\pgfpathmoveto{\pgfqpoint{2.755488in}{2.538078in}}%
\pgfpathlineto{\pgfqpoint{2.769143in}{2.520751in}}%
\pgfpathlineto{\pgfqpoint{2.782789in}{2.503726in}}%
\pgfpathlineto{\pgfqpoint{2.796427in}{2.486999in}}%
\pgfpathlineto{\pgfqpoint{2.810058in}{2.470569in}}%
\pgfpathlineto{\pgfqpoint{2.818392in}{2.477923in}}%
\pgfpathlineto{\pgfqpoint{2.826716in}{2.485408in}}%
\pgfpathlineto{\pgfqpoint{2.835031in}{2.493023in}}%
\pgfpathlineto{\pgfqpoint{2.843336in}{2.500765in}}%
\pgfpathlineto{\pgfqpoint{2.829730in}{2.517003in}}%
\pgfpathlineto{\pgfqpoint{2.816117in}{2.533537in}}%
\pgfpathlineto{\pgfqpoint{2.802496in}{2.550369in}}%
\pgfpathlineto{\pgfqpoint{2.788867in}{2.567502in}}%
\pgfpathlineto{\pgfqpoint{2.780537in}{2.559940in}}%
\pgfpathlineto{\pgfqpoint{2.772197in}{2.552515in}}%
\pgfpathlineto{\pgfqpoint{2.763848in}{2.545227in}}%
\pgfpathlineto{\pgfqpoint{2.755488in}{2.538078in}}%
\pgfpathclose%
\pgfusepath{fill}%
\end{pgfscope}%
\begin{pgfscope}%
\pgfpathrectangle{\pgfqpoint{1.150000in}{0.150000in}}{\pgfqpoint{5.700000in}{5.700000in}}%
\pgfusepath{clip}%
\pgfsetbuttcap%
\pgfsetroundjoin%
\definecolor{currentfill}{rgb}{0.227802,0.326594,0.546532}%
\pgfsetfillcolor{currentfill}%
\pgfsetfillopacity{0.800000}%
\pgfsetlinewidth{0.000000pt}%
\definecolor{currentstroke}{rgb}{0.000000,0.000000,0.000000}%
\pgfsetstrokecolor{currentstroke}%
\pgfsetdash{}{0pt}%
\pgfpathmoveto{\pgfqpoint{4.705377in}{2.709439in}}%
\pgfpathlineto{\pgfqpoint{4.719228in}{2.714154in}}%
\pgfpathlineto{\pgfqpoint{4.733091in}{2.719051in}}%
\pgfpathlineto{\pgfqpoint{4.746967in}{2.724131in}}%
\pgfpathlineto{\pgfqpoint{4.760857in}{2.729393in}}%
\pgfpathlineto{\pgfqpoint{4.768474in}{2.737536in}}%
\pgfpathlineto{\pgfqpoint{4.776085in}{2.745693in}}%
\pgfpathlineto{\pgfqpoint{4.783692in}{2.753869in}}%
\pgfpathlineto{\pgfqpoint{4.791293in}{2.762067in}}%
\pgfpathlineto{\pgfqpoint{4.777417in}{2.757165in}}%
\pgfpathlineto{\pgfqpoint{4.763553in}{2.752446in}}%
\pgfpathlineto{\pgfqpoint{4.749702in}{2.747908in}}%
\pgfpathlineto{\pgfqpoint{4.735864in}{2.743553in}}%
\pgfpathlineto{\pgfqpoint{4.728250in}{2.734984in}}%
\pgfpathlineto{\pgfqpoint{4.720631in}{2.726444in}}%
\pgfpathlineto{\pgfqpoint{4.713006in}{2.717931in}}%
\pgfpathlineto{\pgfqpoint{4.705377in}{2.709439in}}%
\pgfpathclose%
\pgfusepath{fill}%
\end{pgfscope}%
\begin{pgfscope}%
\pgfpathrectangle{\pgfqpoint{1.150000in}{0.150000in}}{\pgfqpoint{5.700000in}{5.700000in}}%
\pgfusepath{clip}%
\pgfsetbuttcap%
\pgfsetroundjoin%
\definecolor{currentfill}{rgb}{0.277134,0.185228,0.489898}%
\pgfsetfillcolor{currentfill}%
\pgfsetfillopacity{0.800000}%
\pgfsetlinewidth{0.000000pt}%
\definecolor{currentstroke}{rgb}{0.000000,0.000000,0.000000}%
\pgfsetstrokecolor{currentstroke}%
\pgfsetdash{}{0pt}%
\pgfpathmoveto{\pgfqpoint{2.864510in}{2.407759in}}%
\pgfpathlineto{\pgfqpoint{2.878107in}{2.392772in}}%
\pgfpathlineto{\pgfqpoint{2.891697in}{2.378067in}}%
\pgfpathlineto{\pgfqpoint{2.905283in}{2.363642in}}%
\pgfpathlineto{\pgfqpoint{2.918862in}{2.349493in}}%
\pgfpathlineto{\pgfqpoint{2.927148in}{2.357253in}}%
\pgfpathlineto{\pgfqpoint{2.935425in}{2.365128in}}%
\pgfpathlineto{\pgfqpoint{2.943693in}{2.373116in}}%
\pgfpathlineto{\pgfqpoint{2.951953in}{2.381217in}}%
\pgfpathlineto{\pgfqpoint{2.938396in}{2.395175in}}%
\pgfpathlineto{\pgfqpoint{2.924833in}{2.409411in}}%
\pgfpathlineto{\pgfqpoint{2.911266in}{2.423925in}}%
\pgfpathlineto{\pgfqpoint{2.897692in}{2.438721in}}%
\pgfpathlineto{\pgfqpoint{2.889410in}{2.430799in}}%
\pgfpathlineto{\pgfqpoint{2.881119in}{2.422997in}}%
\pgfpathlineto{\pgfqpoint{2.872819in}{2.415316in}}%
\pgfpathlineto{\pgfqpoint{2.864510in}{2.407759in}}%
\pgfpathclose%
\pgfusepath{fill}%
\end{pgfscope}%
\begin{pgfscope}%
\pgfpathrectangle{\pgfqpoint{1.150000in}{0.150000in}}{\pgfqpoint{5.700000in}{5.700000in}}%
\pgfusepath{clip}%
\pgfsetbuttcap%
\pgfsetroundjoin%
\definecolor{currentfill}{rgb}{0.136408,0.541173,0.554483}%
\pgfsetfillcolor{currentfill}%
\pgfsetfillopacity{0.800000}%
\pgfsetlinewidth{0.000000pt}%
\definecolor{currentstroke}{rgb}{0.000000,0.000000,0.000000}%
\pgfsetstrokecolor{currentstroke}%
\pgfsetdash{}{0pt}%
\pgfpathmoveto{\pgfqpoint{5.680867in}{3.323580in}}%
\pgfpathlineto{\pgfqpoint{5.695125in}{3.329577in}}%
\pgfpathlineto{\pgfqpoint{5.709399in}{3.335746in}}%
\pgfpathlineto{\pgfqpoint{5.723691in}{3.342084in}}%
\pgfpathlineto{\pgfqpoint{5.737999in}{3.348593in}}%
\pgfpathlineto{\pgfqpoint{5.745235in}{3.355571in}}%
\pgfpathlineto{\pgfqpoint{5.752472in}{3.362775in}}%
\pgfpathlineto{\pgfqpoint{5.759712in}{3.370214in}}%
\pgfpathlineto{\pgfqpoint{5.766955in}{3.377897in}}%
\pgfpathlineto{\pgfqpoint{5.752680in}{3.372139in}}%
\pgfpathlineto{\pgfqpoint{5.738421in}{3.366549in}}%
\pgfpathlineto{\pgfqpoint{5.724178in}{3.361129in}}%
\pgfpathlineto{\pgfqpoint{5.709952in}{3.355879in}}%
\pgfpathlineto{\pgfqpoint{5.702677in}{3.347436in}}%
\pgfpathlineto{\pgfqpoint{5.695405in}{3.339245in}}%
\pgfpathlineto{\pgfqpoint{5.688135in}{3.331295in}}%
\pgfpathlineto{\pgfqpoint{5.680867in}{3.323580in}}%
\pgfpathclose%
\pgfusepath{fill}%
\end{pgfscope}%
\begin{pgfscope}%
\pgfpathrectangle{\pgfqpoint{1.150000in}{0.150000in}}{\pgfqpoint{5.700000in}{5.700000in}}%
\pgfusepath{clip}%
\pgfsetbuttcap%
\pgfsetroundjoin%
\definecolor{currentfill}{rgb}{0.280894,0.078907,0.402329}%
\pgfsetfillcolor{currentfill}%
\pgfsetfillopacity{0.800000}%
\pgfsetlinewidth{0.000000pt}%
\definecolor{currentstroke}{rgb}{0.000000,0.000000,0.000000}%
\pgfsetstrokecolor{currentstroke}%
\pgfsetdash{}{0pt}%
\pgfpathmoveto{\pgfqpoint{3.449168in}{2.146999in}}%
\pgfpathlineto{\pgfqpoint{3.462657in}{2.141445in}}%
\pgfpathlineto{\pgfqpoint{3.476149in}{2.136113in}}%
\pgfpathlineto{\pgfqpoint{3.489643in}{2.131001in}}%
\pgfpathlineto{\pgfqpoint{3.503140in}{2.126110in}}%
\pgfpathlineto{\pgfqpoint{3.511189in}{2.136090in}}%
\pgfpathlineto{\pgfqpoint{3.519233in}{2.146101in}}%
\pgfpathlineto{\pgfqpoint{3.527270in}{2.156144in}}%
\pgfpathlineto{\pgfqpoint{3.535302in}{2.166216in}}%
\pgfpathlineto{\pgfqpoint{3.521817in}{2.171019in}}%
\pgfpathlineto{\pgfqpoint{3.508334in}{2.176042in}}%
\pgfpathlineto{\pgfqpoint{3.494855in}{2.181285in}}%
\pgfpathlineto{\pgfqpoint{3.481379in}{2.186750in}}%
\pgfpathlineto{\pgfqpoint{3.473335in}{2.176754in}}%
\pgfpathlineto{\pgfqpoint{3.465285in}{2.166797in}}%
\pgfpathlineto{\pgfqpoint{3.457230in}{2.156878in}}%
\pgfpathlineto{\pgfqpoint{3.449168in}{2.146999in}}%
\pgfpathclose%
\pgfusepath{fill}%
\end{pgfscope}%
\begin{pgfscope}%
\pgfpathrectangle{\pgfqpoint{1.150000in}{0.150000in}}{\pgfqpoint{5.700000in}{5.700000in}}%
\pgfusepath{clip}%
\pgfsetbuttcap%
\pgfsetroundjoin%
\definecolor{currentfill}{rgb}{0.282656,0.100196,0.422160}%
\pgfsetfillcolor{currentfill}%
\pgfsetfillopacity{0.800000}%
\pgfsetlinewidth{0.000000pt}%
\definecolor{currentstroke}{rgb}{0.000000,0.000000,0.000000}%
\pgfsetstrokecolor{currentstroke}%
\pgfsetdash{}{0pt}%
\pgfpathmoveto{\pgfqpoint{3.168363in}{2.193886in}}%
\pgfpathlineto{\pgfqpoint{3.181869in}{2.184334in}}%
\pgfpathlineto{\pgfqpoint{3.195374in}{2.175025in}}%
\pgfpathlineto{\pgfqpoint{3.208878in}{2.165959in}}%
\pgfpathlineto{\pgfqpoint{3.222382in}{2.157133in}}%
\pgfpathlineto{\pgfqpoint{3.230538in}{2.166179in}}%
\pgfpathlineto{\pgfqpoint{3.238687in}{2.175294in}}%
\pgfpathlineto{\pgfqpoint{3.246829in}{2.184479in}}%
\pgfpathlineto{\pgfqpoint{3.254965in}{2.193730in}}%
\pgfpathlineto{\pgfqpoint{3.241477in}{2.202403in}}%
\pgfpathlineto{\pgfqpoint{3.227990in}{2.211317in}}%
\pgfpathlineto{\pgfqpoint{3.214502in}{2.220472in}}%
\pgfpathlineto{\pgfqpoint{3.201013in}{2.229870in}}%
\pgfpathlineto{\pgfqpoint{3.192861in}{2.220760in}}%
\pgfpathlineto{\pgfqpoint{3.184702in}{2.211725in}}%
\pgfpathlineto{\pgfqpoint{3.176536in}{2.202767in}}%
\pgfpathlineto{\pgfqpoint{3.168363in}{2.193886in}}%
\pgfpathclose%
\pgfusepath{fill}%
\end{pgfscope}%
\begin{pgfscope}%
\pgfpathrectangle{\pgfqpoint{1.150000in}{0.150000in}}{\pgfqpoint{5.700000in}{5.700000in}}%
\pgfusepath{clip}%
\pgfsetbuttcap%
\pgfsetroundjoin%
\definecolor{currentfill}{rgb}{0.252194,0.269783,0.531579}%
\pgfsetfillcolor{currentfill}%
\pgfsetfillopacity{0.800000}%
\pgfsetlinewidth{0.000000pt}%
\definecolor{currentstroke}{rgb}{0.000000,0.000000,0.000000}%
\pgfsetstrokecolor{currentstroke}%
\pgfsetdash{}{0pt}%
\pgfpathmoveto{\pgfqpoint{2.700783in}{2.610452in}}%
\pgfpathlineto{\pgfqpoint{2.714473in}{2.591893in}}%
\pgfpathlineto{\pgfqpoint{2.728154in}{2.573646in}}%
\pgfpathlineto{\pgfqpoint{2.741826in}{2.555709in}}%
\pgfpathlineto{\pgfqpoint{2.755488in}{2.538078in}}%
\pgfpathlineto{\pgfqpoint{2.763848in}{2.545227in}}%
\pgfpathlineto{\pgfqpoint{2.772197in}{2.552515in}}%
\pgfpathlineto{\pgfqpoint{2.780537in}{2.559940in}}%
\pgfpathlineto{\pgfqpoint{2.788867in}{2.567502in}}%
\pgfpathlineto{\pgfqpoint{2.775230in}{2.584938in}}%
\pgfpathlineto{\pgfqpoint{2.761585in}{2.602681in}}%
\pgfpathlineto{\pgfqpoint{2.747931in}{2.620733in}}%
\pgfpathlineto{\pgfqpoint{2.734267in}{2.639097in}}%
\pgfpathlineto{\pgfqpoint{2.725911in}{2.631718in}}%
\pgfpathlineto{\pgfqpoint{2.717546in}{2.624483in}}%
\pgfpathlineto{\pgfqpoint{2.709170in}{2.617393in}}%
\pgfpathlineto{\pgfqpoint{2.700783in}{2.610452in}}%
\pgfpathclose%
\pgfusepath{fill}%
\end{pgfscope}%
\begin{pgfscope}%
\pgfpathrectangle{\pgfqpoint{1.150000in}{0.150000in}}{\pgfqpoint{5.700000in}{5.700000in}}%
\pgfusepath{clip}%
\pgfsetbuttcap%
\pgfsetroundjoin%
\definecolor{currentfill}{rgb}{0.218130,0.347432,0.550038}%
\pgfsetfillcolor{currentfill}%
\pgfsetfillopacity{0.800000}%
\pgfsetlinewidth{0.000000pt}%
\definecolor{currentstroke}{rgb}{0.000000,0.000000,0.000000}%
\pgfsetstrokecolor{currentstroke}%
\pgfsetdash{}{0pt}%
\pgfpathmoveto{\pgfqpoint{4.791293in}{2.762067in}}%
\pgfpathlineto{\pgfqpoint{4.805183in}{2.767150in}}%
\pgfpathlineto{\pgfqpoint{4.819087in}{2.772414in}}%
\pgfpathlineto{\pgfqpoint{4.833004in}{2.777860in}}%
\pgfpathlineto{\pgfqpoint{4.846934in}{2.783487in}}%
\pgfpathlineto{\pgfqpoint{4.854517in}{2.791331in}}%
\pgfpathlineto{\pgfqpoint{4.862095in}{2.799198in}}%
\pgfpathlineto{\pgfqpoint{4.869668in}{2.807093in}}%
\pgfpathlineto{\pgfqpoint{4.877236in}{2.815021in}}%
\pgfpathlineto{\pgfqpoint{4.863319in}{2.809788in}}%
\pgfpathlineto{\pgfqpoint{4.849416in}{2.804735in}}%
\pgfpathlineto{\pgfqpoint{4.835526in}{2.799863in}}%
\pgfpathlineto{\pgfqpoint{4.821650in}{2.795172in}}%
\pgfpathlineto{\pgfqpoint{4.814068in}{2.786840in}}%
\pgfpathlineto{\pgfqpoint{4.806481in}{2.778548in}}%
\pgfpathlineto{\pgfqpoint{4.798890in}{2.770292in}}%
\pgfpathlineto{\pgfqpoint{4.791293in}{2.762067in}}%
\pgfpathclose%
\pgfusepath{fill}%
\end{pgfscope}%
\begin{pgfscope}%
\pgfpathrectangle{\pgfqpoint{1.150000in}{0.150000in}}{\pgfqpoint{5.700000in}{5.700000in}}%
\pgfusepath{clip}%
\pgfsetbuttcap%
\pgfsetroundjoin%
\definecolor{currentfill}{rgb}{0.280255,0.165693,0.476498}%
\pgfsetfillcolor{currentfill}%
\pgfsetfillopacity{0.800000}%
\pgfsetlinewidth{0.000000pt}%
\definecolor{currentstroke}{rgb}{0.000000,0.000000,0.000000}%
\pgfsetstrokecolor{currentstroke}%
\pgfsetdash{}{0pt}%
\pgfpathmoveto{\pgfqpoint{2.918862in}{2.349493in}}%
\pgfpathlineto{\pgfqpoint{2.932437in}{2.335620in}}%
\pgfpathlineto{\pgfqpoint{2.946006in}{2.322020in}}%
\pgfpathlineto{\pgfqpoint{2.959571in}{2.308690in}}%
\pgfpathlineto{\pgfqpoint{2.973131in}{2.295629in}}%
\pgfpathlineto{\pgfqpoint{2.981395in}{2.303590in}}%
\pgfpathlineto{\pgfqpoint{2.989649in}{2.311658in}}%
\pgfpathlineto{\pgfqpoint{2.997896in}{2.319831in}}%
\pgfpathlineto{\pgfqpoint{3.006134in}{2.328108in}}%
\pgfpathlineto{\pgfqpoint{2.992596in}{2.340981in}}%
\pgfpathlineto{\pgfqpoint{2.979052in}{2.354121in}}%
\pgfpathlineto{\pgfqpoint{2.965505in}{2.367533in}}%
\pgfpathlineto{\pgfqpoint{2.951953in}{2.381217in}}%
\pgfpathlineto{\pgfqpoint{2.943693in}{2.373116in}}%
\pgfpathlineto{\pgfqpoint{2.935425in}{2.365128in}}%
\pgfpathlineto{\pgfqpoint{2.927148in}{2.357253in}}%
\pgfpathlineto{\pgfqpoint{2.918862in}{2.349493in}}%
\pgfpathclose%
\pgfusepath{fill}%
\end{pgfscope}%
\begin{pgfscope}%
\pgfpathrectangle{\pgfqpoint{1.150000in}{0.150000in}}{\pgfqpoint{5.700000in}{5.700000in}}%
\pgfusepath{clip}%
\pgfsetbuttcap%
\pgfsetroundjoin%
\definecolor{currentfill}{rgb}{0.282327,0.094955,0.417331}%
\pgfsetfillcolor{currentfill}%
\pgfsetfillopacity{0.800000}%
\pgfsetlinewidth{0.000000pt}%
\definecolor{currentstroke}{rgb}{0.000000,0.000000,0.000000}%
\pgfsetstrokecolor{currentstroke}%
\pgfsetdash{}{0pt}%
\pgfpathmoveto{\pgfqpoint{3.675311in}{2.176621in}}%
\pgfpathlineto{\pgfqpoint{3.688823in}{2.173806in}}%
\pgfpathlineto{\pgfqpoint{3.702341in}{2.171201in}}%
\pgfpathlineto{\pgfqpoint{3.715864in}{2.168804in}}%
\pgfpathlineto{\pgfqpoint{3.729392in}{2.166616in}}%
\pgfpathlineto{\pgfqpoint{3.737366in}{2.176949in}}%
\pgfpathlineto{\pgfqpoint{3.745335in}{2.187289in}}%
\pgfpathlineto{\pgfqpoint{3.753298in}{2.197635in}}%
\pgfpathlineto{\pgfqpoint{3.761257in}{2.207988in}}%
\pgfpathlineto{\pgfqpoint{3.747738in}{2.210152in}}%
\pgfpathlineto{\pgfqpoint{3.734224in}{2.212523in}}%
\pgfpathlineto{\pgfqpoint{3.720716in}{2.215103in}}%
\pgfpathlineto{\pgfqpoint{3.707213in}{2.217893in}}%
\pgfpathlineto{\pgfqpoint{3.699245in}{2.207554in}}%
\pgfpathlineto{\pgfqpoint{3.691272in}{2.197229in}}%
\pgfpathlineto{\pgfqpoint{3.683294in}{2.186918in}}%
\pgfpathlineto{\pgfqpoint{3.675311in}{2.176621in}}%
\pgfpathclose%
\pgfusepath{fill}%
\end{pgfscope}%
\begin{pgfscope}%
\pgfpathrectangle{\pgfqpoint{1.150000in}{0.150000in}}{\pgfqpoint{5.700000in}{5.700000in}}%
\pgfusepath{clip}%
\pgfsetbuttcap%
\pgfsetroundjoin%
\definecolor{currentfill}{rgb}{0.208623,0.367752,0.552675}%
\pgfsetfillcolor{currentfill}%
\pgfsetfillopacity{0.800000}%
\pgfsetlinewidth{0.000000pt}%
\definecolor{currentstroke}{rgb}{0.000000,0.000000,0.000000}%
\pgfsetstrokecolor{currentstroke}%
\pgfsetdash{}{0pt}%
\pgfpathmoveto{\pgfqpoint{4.877236in}{2.815021in}}%
\pgfpathlineto{\pgfqpoint{4.891166in}{2.820435in}}%
\pgfpathlineto{\pgfqpoint{4.905110in}{2.826029in}}%
\pgfpathlineto{\pgfqpoint{4.919068in}{2.831803in}}%
\pgfpathlineto{\pgfqpoint{4.933040in}{2.837756in}}%
\pgfpathlineto{\pgfqpoint{4.940588in}{2.845308in}}%
\pgfpathlineto{\pgfqpoint{4.948132in}{2.852893in}}%
\pgfpathlineto{\pgfqpoint{4.955670in}{2.860518in}}%
\pgfpathlineto{\pgfqpoint{4.963204in}{2.868187in}}%
\pgfpathlineto{\pgfqpoint{4.949247in}{2.862659in}}%
\pgfpathlineto{\pgfqpoint{4.935304in}{2.857311in}}%
\pgfpathlineto{\pgfqpoint{4.921375in}{2.852143in}}%
\pgfpathlineto{\pgfqpoint{4.907459in}{2.847154in}}%
\pgfpathlineto{\pgfqpoint{4.899910in}{2.839048in}}%
\pgfpathlineto{\pgfqpoint{4.892357in}{2.830994in}}%
\pgfpathlineto{\pgfqpoint{4.884799in}{2.822986in}}%
\pgfpathlineto{\pgfqpoint{4.877236in}{2.815021in}}%
\pgfpathclose%
\pgfusepath{fill}%
\end{pgfscope}%
\begin{pgfscope}%
\pgfpathrectangle{\pgfqpoint{1.150000in}{0.150000in}}{\pgfqpoint{5.700000in}{5.700000in}}%
\pgfusepath{clip}%
\pgfsetbuttcap%
\pgfsetroundjoin%
\definecolor{currentfill}{rgb}{0.239346,0.300855,0.540844}%
\pgfsetfillcolor{currentfill}%
\pgfsetfillopacity{0.800000}%
\pgfsetlinewidth{0.000000pt}%
\definecolor{currentstroke}{rgb}{0.000000,0.000000,0.000000}%
\pgfsetstrokecolor{currentstroke}%
\pgfsetdash{}{0pt}%
\pgfpathmoveto{\pgfqpoint{2.645923in}{2.687869in}}%
\pgfpathlineto{\pgfqpoint{2.659653in}{2.668032in}}%
\pgfpathlineto{\pgfqpoint{2.673373in}{2.648518in}}%
\pgfpathlineto{\pgfqpoint{2.687083in}{2.629326in}}%
\pgfpathlineto{\pgfqpoint{2.700783in}{2.610452in}}%
\pgfpathlineto{\pgfqpoint{2.709170in}{2.617393in}}%
\pgfpathlineto{\pgfqpoint{2.717546in}{2.624483in}}%
\pgfpathlineto{\pgfqpoint{2.725911in}{2.631718in}}%
\pgfpathlineto{\pgfqpoint{2.734267in}{2.639097in}}%
\pgfpathlineto{\pgfqpoint{2.720595in}{2.657775in}}%
\pgfpathlineto{\pgfqpoint{2.706912in}{2.676771in}}%
\pgfpathlineto{\pgfqpoint{2.693220in}{2.696088in}}%
\pgfpathlineto{\pgfqpoint{2.679517in}{2.715728in}}%
\pgfpathlineto{\pgfqpoint{2.671134in}{2.708533in}}%
\pgfpathlineto{\pgfqpoint{2.662741in}{2.701490in}}%
\pgfpathlineto{\pgfqpoint{2.654337in}{2.694602in}}%
\pgfpathlineto{\pgfqpoint{2.645923in}{2.687869in}}%
\pgfpathclose%
\pgfusepath{fill}%
\end{pgfscope}%
\begin{pgfscope}%
\pgfpathrectangle{\pgfqpoint{1.150000in}{0.150000in}}{\pgfqpoint{5.700000in}{5.700000in}}%
\pgfusepath{clip}%
\pgfsetbuttcap%
\pgfsetroundjoin%
\definecolor{currentfill}{rgb}{0.129933,0.559582,0.551864}%
\pgfsetfillcolor{currentfill}%
\pgfsetfillopacity{0.800000}%
\pgfsetlinewidth{0.000000pt}%
\definecolor{currentstroke}{rgb}{0.000000,0.000000,0.000000}%
\pgfsetstrokecolor{currentstroke}%
\pgfsetdash{}{0pt}%
\pgfpathmoveto{\pgfqpoint{5.766955in}{3.377897in}}%
\pgfpathlineto{\pgfqpoint{5.781247in}{3.383825in}}%
\pgfpathlineto{\pgfqpoint{5.795556in}{3.389923in}}%
\pgfpathlineto{\pgfqpoint{5.809881in}{3.396190in}}%
\pgfpathlineto{\pgfqpoint{5.824224in}{3.402627in}}%
\pgfpathlineto{\pgfqpoint{5.831436in}{3.409791in}}%
\pgfpathlineto{\pgfqpoint{5.838650in}{3.417209in}}%
\pgfpathlineto{\pgfqpoint{5.845868in}{3.424888in}}%
\pgfpathlineto{\pgfqpoint{5.831551in}{3.419036in}}%
\pgfpathlineto{\pgfqpoint{5.817251in}{3.413352in}}%
\pgfpathlineto{\pgfqpoint{5.802968in}{3.407837in}}%
\pgfpathlineto{\pgfqpoint{5.788701in}{3.402491in}}%
\pgfpathlineto{\pgfqpoint{5.781449in}{3.394027in}}%
\pgfpathlineto{\pgfqpoint{5.774200in}{3.385832in}}%
\pgfpathlineto{\pgfqpoint{5.766955in}{3.377897in}}%
\pgfpathclose%
\pgfusepath{fill}%
\end{pgfscope}%
\begin{pgfscope}%
\pgfpathrectangle{\pgfqpoint{1.150000in}{0.150000in}}{\pgfqpoint{5.700000in}{5.700000in}}%
\pgfusepath{clip}%
\pgfsetbuttcap%
\pgfsetroundjoin%
\definecolor{currentfill}{rgb}{0.282623,0.140926,0.457517}%
\pgfsetfillcolor{currentfill}%
\pgfsetfillopacity{0.800000}%
\pgfsetlinewidth{0.000000pt}%
\definecolor{currentstroke}{rgb}{0.000000,0.000000,0.000000}%
\pgfsetstrokecolor{currentstroke}%
\pgfsetdash{}{0pt}%
\pgfpathmoveto{\pgfqpoint{2.973131in}{2.295629in}}%
\pgfpathlineto{\pgfqpoint{2.986688in}{2.282835in}}%
\pgfpathlineto{\pgfqpoint{3.000240in}{2.270306in}}%
\pgfpathlineto{\pgfqpoint{3.013788in}{2.258040in}}%
\pgfpathlineto{\pgfqpoint{3.027333in}{2.246034in}}%
\pgfpathlineto{\pgfqpoint{3.035575in}{2.254195in}}%
\pgfpathlineto{\pgfqpoint{3.043809in}{2.262454in}}%
\pgfpathlineto{\pgfqpoint{3.052035in}{2.270811in}}%
\pgfpathlineto{\pgfqpoint{3.060253in}{2.279264in}}%
\pgfpathlineto{\pgfqpoint{3.046729in}{2.291083in}}%
\pgfpathlineto{\pgfqpoint{3.033201in}{2.303161in}}%
\pgfpathlineto{\pgfqpoint{3.019669in}{2.315503in}}%
\pgfpathlineto{\pgfqpoint{3.006134in}{2.328108in}}%
\pgfpathlineto{\pgfqpoint{2.997896in}{2.319831in}}%
\pgfpathlineto{\pgfqpoint{2.989649in}{2.311658in}}%
\pgfpathlineto{\pgfqpoint{2.981395in}{2.303590in}}%
\pgfpathlineto{\pgfqpoint{2.973131in}{2.295629in}}%
\pgfpathclose%
\pgfusepath{fill}%
\end{pgfscope}%
\begin{pgfscope}%
\pgfpathrectangle{\pgfqpoint{1.150000in}{0.150000in}}{\pgfqpoint{5.700000in}{5.700000in}}%
\pgfusepath{clip}%
\pgfsetbuttcap%
\pgfsetroundjoin%
\definecolor{currentfill}{rgb}{0.199430,0.387607,0.554642}%
\pgfsetfillcolor{currentfill}%
\pgfsetfillopacity{0.800000}%
\pgfsetlinewidth{0.000000pt}%
\definecolor{currentstroke}{rgb}{0.000000,0.000000,0.000000}%
\pgfsetstrokecolor{currentstroke}%
\pgfsetdash{}{0pt}%
\pgfpathmoveto{\pgfqpoint{4.963204in}{2.868187in}}%
\pgfpathlineto{\pgfqpoint{4.977175in}{2.873893in}}%
\pgfpathlineto{\pgfqpoint{4.991160in}{2.879779in}}%
\pgfpathlineto{\pgfqpoint{5.005159in}{2.885844in}}%
\pgfpathlineto{\pgfqpoint{5.019174in}{2.892087in}}%
\pgfpathlineto{\pgfqpoint{5.026686in}{2.899358in}}%
\pgfpathlineto{\pgfqpoint{5.034194in}{2.906676in}}%
\pgfpathlineto{\pgfqpoint{5.041698in}{2.914045in}}%
\pgfpathlineto{\pgfqpoint{5.049197in}{2.921472in}}%
\pgfpathlineto{\pgfqpoint{5.035199in}{2.915688in}}%
\pgfpathlineto{\pgfqpoint{5.021216in}{2.910082in}}%
\pgfpathlineto{\pgfqpoint{5.007248in}{2.904654in}}%
\pgfpathlineto{\pgfqpoint{4.993293in}{2.899405in}}%
\pgfpathlineto{\pgfqpoint{4.985777in}{2.891509in}}%
\pgfpathlineto{\pgfqpoint{4.978257in}{2.883677in}}%
\pgfpathlineto{\pgfqpoint{4.970733in}{2.875905in}}%
\pgfpathlineto{\pgfqpoint{4.963204in}{2.868187in}}%
\pgfpathclose%
\pgfusepath{fill}%
\end{pgfscope}%
\begin{pgfscope}%
\pgfpathrectangle{\pgfqpoint{1.150000in}{0.150000in}}{\pgfqpoint{5.700000in}{5.700000in}}%
\pgfusepath{clip}%
\pgfsetbuttcap%
\pgfsetroundjoin%
\definecolor{currentfill}{rgb}{0.281446,0.084320,0.407414}%
\pgfsetfillcolor{currentfill}%
\pgfsetfillopacity{0.800000}%
\pgfsetlinewidth{0.000000pt}%
\definecolor{currentstroke}{rgb}{0.000000,0.000000,0.000000}%
\pgfsetstrokecolor{currentstroke}%
\pgfsetdash{}{0pt}%
\pgfpathmoveto{\pgfqpoint{3.589280in}{2.149180in}}%
\pgfpathlineto{\pgfqpoint{3.602784in}{2.145459in}}%
\pgfpathlineto{\pgfqpoint{3.616293in}{2.141952in}}%
\pgfpathlineto{\pgfqpoint{3.629806in}{2.138658in}}%
\pgfpathlineto{\pgfqpoint{3.643323in}{2.135575in}}%
\pgfpathlineto{\pgfqpoint{3.651328in}{2.145816in}}%
\pgfpathlineto{\pgfqpoint{3.659328in}{2.156070in}}%
\pgfpathlineto{\pgfqpoint{3.667322in}{2.166339in}}%
\pgfpathlineto{\pgfqpoint{3.675311in}{2.176621in}}%
\pgfpathlineto{\pgfqpoint{3.661803in}{2.179647in}}%
\pgfpathlineto{\pgfqpoint{3.648300in}{2.182885in}}%
\pgfpathlineto{\pgfqpoint{3.634802in}{2.186335in}}%
\pgfpathlineto{\pgfqpoint{3.621309in}{2.189998in}}%
\pgfpathlineto{\pgfqpoint{3.613310in}{2.179761in}}%
\pgfpathlineto{\pgfqpoint{3.605305in}{2.169545in}}%
\pgfpathlineto{\pgfqpoint{3.597295in}{2.159352in}}%
\pgfpathlineto{\pgfqpoint{3.589280in}{2.149180in}}%
\pgfpathclose%
\pgfusepath{fill}%
\end{pgfscope}%
\begin{pgfscope}%
\pgfpathrectangle{\pgfqpoint{1.150000in}{0.150000in}}{\pgfqpoint{5.700000in}{5.700000in}}%
\pgfusepath{clip}%
\pgfsetbuttcap%
\pgfsetroundjoin%
\definecolor{currentfill}{rgb}{0.190631,0.407061,0.556089}%
\pgfsetfillcolor{currentfill}%
\pgfsetfillopacity{0.800000}%
\pgfsetlinewidth{0.000000pt}%
\definecolor{currentstroke}{rgb}{0.000000,0.000000,0.000000}%
\pgfsetstrokecolor{currentstroke}%
\pgfsetdash{}{0pt}%
\pgfpathmoveto{\pgfqpoint{5.049197in}{2.921472in}}%
\pgfpathlineto{\pgfqpoint{5.063209in}{2.927434in}}%
\pgfpathlineto{\pgfqpoint{5.077235in}{2.933574in}}%
\pgfpathlineto{\pgfqpoint{5.091276in}{2.939891in}}%
\pgfpathlineto{\pgfqpoint{5.105333in}{2.946387in}}%
\pgfpathlineto{\pgfqpoint{5.112810in}{2.953396in}}%
\pgfpathlineto{\pgfqpoint{5.120282in}{2.960465in}}%
\pgfpathlineto{\pgfqpoint{5.127750in}{2.967600in}}%
\pgfpathlineto{\pgfqpoint{5.135214in}{2.974806in}}%
\pgfpathlineto{\pgfqpoint{5.121176in}{2.968803in}}%
\pgfpathlineto{\pgfqpoint{5.107153in}{2.962977in}}%
\pgfpathlineto{\pgfqpoint{5.093144in}{2.957328in}}%
\pgfpathlineto{\pgfqpoint{5.079150in}{2.951856in}}%
\pgfpathlineto{\pgfqpoint{5.071668in}{2.944147in}}%
\pgfpathlineto{\pgfqpoint{5.064182in}{2.936517in}}%
\pgfpathlineto{\pgfqpoint{5.056691in}{2.928961in}}%
\pgfpathlineto{\pgfqpoint{5.049197in}{2.921472in}}%
\pgfpathclose%
\pgfusepath{fill}%
\end{pgfscope}%
\begin{pgfscope}%
\pgfpathrectangle{\pgfqpoint{1.150000in}{0.150000in}}{\pgfqpoint{5.700000in}{5.700000in}}%
\pgfusepath{clip}%
\pgfsetbuttcap%
\pgfsetroundjoin%
\definecolor{currentfill}{rgb}{0.280267,0.073417,0.397163}%
\pgfsetfillcolor{currentfill}%
\pgfsetfillopacity{0.800000}%
\pgfsetlinewidth{0.000000pt}%
\definecolor{currentstroke}{rgb}{0.000000,0.000000,0.000000}%
\pgfsetstrokecolor{currentstroke}%
\pgfsetdash{}{0pt}%
\pgfpathmoveto{\pgfqpoint{3.362877in}{2.132825in}}%
\pgfpathlineto{\pgfqpoint{3.376371in}{2.126251in}}%
\pgfpathlineto{\pgfqpoint{3.389867in}{2.119904in}}%
\pgfpathlineto{\pgfqpoint{3.403364in}{2.113782in}}%
\pgfpathlineto{\pgfqpoint{3.416864in}{2.107885in}}%
\pgfpathlineto{\pgfqpoint{3.424949in}{2.117601in}}%
\pgfpathlineto{\pgfqpoint{3.433028in}{2.127359in}}%
\pgfpathlineto{\pgfqpoint{3.441101in}{2.137159in}}%
\pgfpathlineto{\pgfqpoint{3.449168in}{2.146999in}}%
\pgfpathlineto{\pgfqpoint{3.435682in}{2.152776in}}%
\pgfpathlineto{\pgfqpoint{3.422198in}{2.158777in}}%
\pgfpathlineto{\pgfqpoint{3.408716in}{2.165003in}}%
\pgfpathlineto{\pgfqpoint{3.395236in}{2.171456in}}%
\pgfpathlineto{\pgfqpoint{3.387156in}{2.161725in}}%
\pgfpathlineto{\pgfqpoint{3.379069in}{2.152042in}}%
\pgfpathlineto{\pgfqpoint{3.370976in}{2.142409in}}%
\pgfpathlineto{\pgfqpoint{3.362877in}{2.132825in}}%
\pgfpathclose%
\pgfusepath{fill}%
\end{pgfscope}%
\begin{pgfscope}%
\pgfpathrectangle{\pgfqpoint{1.150000in}{0.150000in}}{\pgfqpoint{5.700000in}{5.700000in}}%
\pgfusepath{clip}%
\pgfsetbuttcap%
\pgfsetroundjoin%
\definecolor{currentfill}{rgb}{0.281446,0.084320,0.407414}%
\pgfsetfillcolor{currentfill}%
\pgfsetfillopacity{0.800000}%
\pgfsetlinewidth{0.000000pt}%
\definecolor{currentstroke}{rgb}{0.000000,0.000000,0.000000}%
\pgfsetstrokecolor{currentstroke}%
\pgfsetdash{}{0pt}%
\pgfpathmoveto{\pgfqpoint{3.222382in}{2.157133in}}%
\pgfpathlineto{\pgfqpoint{3.235886in}{2.148545in}}%
\pgfpathlineto{\pgfqpoint{3.249389in}{2.140195in}}%
\pgfpathlineto{\pgfqpoint{3.262893in}{2.132082in}}%
\pgfpathlineto{\pgfqpoint{3.276396in}{2.124203in}}%
\pgfpathlineto{\pgfqpoint{3.284536in}{2.133413in}}%
\pgfpathlineto{\pgfqpoint{3.292669in}{2.142686in}}%
\pgfpathlineto{\pgfqpoint{3.300795in}{2.152019in}}%
\pgfpathlineto{\pgfqpoint{3.308915in}{2.161412in}}%
\pgfpathlineto{\pgfqpoint{3.295427in}{2.169138in}}%
\pgfpathlineto{\pgfqpoint{3.281939in}{2.177099in}}%
\pgfpathlineto{\pgfqpoint{3.268452in}{2.185296in}}%
\pgfpathlineto{\pgfqpoint{3.254965in}{2.193730in}}%
\pgfpathlineto{\pgfqpoint{3.246829in}{2.184479in}}%
\pgfpathlineto{\pgfqpoint{3.238687in}{2.175294in}}%
\pgfpathlineto{\pgfqpoint{3.230538in}{2.166179in}}%
\pgfpathlineto{\pgfqpoint{3.222382in}{2.157133in}}%
\pgfpathclose%
\pgfusepath{fill}%
\end{pgfscope}%
\begin{pgfscope}%
\pgfpathrectangle{\pgfqpoint{1.150000in}{0.150000in}}{\pgfqpoint{5.700000in}{5.700000in}}%
\pgfusepath{clip}%
\pgfsetbuttcap%
\pgfsetroundjoin%
\definecolor{currentfill}{rgb}{0.223925,0.334994,0.548053}%
\pgfsetfillcolor{currentfill}%
\pgfsetfillopacity{0.800000}%
\pgfsetlinewidth{0.000000pt}%
\definecolor{currentstroke}{rgb}{0.000000,0.000000,0.000000}%
\pgfsetstrokecolor{currentstroke}%
\pgfsetdash{}{0pt}%
\pgfpathmoveto{\pgfqpoint{2.590887in}{2.770522in}}%
\pgfpathlineto{\pgfqpoint{2.604663in}{2.749357in}}%
\pgfpathlineto{\pgfqpoint{2.618428in}{2.728528in}}%
\pgfpathlineto{\pgfqpoint{2.632181in}{2.708034in}}%
\pgfpathlineto{\pgfqpoint{2.645923in}{2.687869in}}%
\pgfpathlineto{\pgfqpoint{2.654337in}{2.694602in}}%
\pgfpathlineto{\pgfqpoint{2.662741in}{2.701490in}}%
\pgfpathlineto{\pgfqpoint{2.671134in}{2.708533in}}%
\pgfpathlineto{\pgfqpoint{2.679517in}{2.715728in}}%
\pgfpathlineto{\pgfqpoint{2.665803in}{2.735695in}}%
\pgfpathlineto{\pgfqpoint{2.652079in}{2.755992in}}%
\pgfpathlineto{\pgfqpoint{2.638343in}{2.776621in}}%
\pgfpathlineto{\pgfqpoint{2.624596in}{2.797587in}}%
\pgfpathlineto{\pgfqpoint{2.616185in}{2.790578in}}%
\pgfpathlineto{\pgfqpoint{2.607764in}{2.783730in}}%
\pgfpathlineto{\pgfqpoint{2.599331in}{2.777044in}}%
\pgfpathlineto{\pgfqpoint{2.590887in}{2.770522in}}%
\pgfpathclose%
\pgfusepath{fill}%
\end{pgfscope}%
\begin{pgfscope}%
\pgfpathrectangle{\pgfqpoint{1.150000in}{0.150000in}}{\pgfqpoint{5.700000in}{5.700000in}}%
\pgfusepath{clip}%
\pgfsetbuttcap%
\pgfsetroundjoin%
\definecolor{currentfill}{rgb}{0.283187,0.125848,0.444960}%
\pgfsetfillcolor{currentfill}%
\pgfsetfillopacity{0.800000}%
\pgfsetlinewidth{0.000000pt}%
\definecolor{currentstroke}{rgb}{0.000000,0.000000,0.000000}%
\pgfsetstrokecolor{currentstroke}%
\pgfsetdash{}{0pt}%
\pgfpathmoveto{\pgfqpoint{3.027333in}{2.246034in}}%
\pgfpathlineto{\pgfqpoint{3.040875in}{2.234288in}}%
\pgfpathlineto{\pgfqpoint{3.054414in}{2.222798in}}%
\pgfpathlineto{\pgfqpoint{3.067950in}{2.211564in}}%
\pgfpathlineto{\pgfqpoint{3.081484in}{2.200584in}}%
\pgfpathlineto{\pgfqpoint{3.089706in}{2.208944in}}%
\pgfpathlineto{\pgfqpoint{3.097919in}{2.217394in}}%
\pgfpathlineto{\pgfqpoint{3.106126in}{2.225934in}}%
\pgfpathlineto{\pgfqpoint{3.114324in}{2.234562in}}%
\pgfpathlineto{\pgfqpoint{3.100810in}{2.245356in}}%
\pgfpathlineto{\pgfqpoint{3.087294in}{2.256403in}}%
\pgfpathlineto{\pgfqpoint{3.073775in}{2.267705in}}%
\pgfpathlineto{\pgfqpoint{3.060253in}{2.279264in}}%
\pgfpathlineto{\pgfqpoint{3.052035in}{2.270811in}}%
\pgfpathlineto{\pgfqpoint{3.043809in}{2.262454in}}%
\pgfpathlineto{\pgfqpoint{3.035575in}{2.254195in}}%
\pgfpathlineto{\pgfqpoint{3.027333in}{2.246034in}}%
\pgfpathclose%
\pgfusepath{fill}%
\end{pgfscope}%
\begin{pgfscope}%
\pgfpathrectangle{\pgfqpoint{1.150000in}{0.150000in}}{\pgfqpoint{5.700000in}{5.700000in}}%
\pgfusepath{clip}%
\pgfsetbuttcap%
\pgfsetroundjoin%
\definecolor{currentfill}{rgb}{0.182256,0.426184,0.557120}%
\pgfsetfillcolor{currentfill}%
\pgfsetfillopacity{0.800000}%
\pgfsetlinewidth{0.000000pt}%
\definecolor{currentstroke}{rgb}{0.000000,0.000000,0.000000}%
\pgfsetstrokecolor{currentstroke}%
\pgfsetdash{}{0pt}%
\pgfpathmoveto{\pgfqpoint{5.135214in}{2.974806in}}%
\pgfpathlineto{\pgfqpoint{5.149267in}{2.980986in}}%
\pgfpathlineto{\pgfqpoint{5.163335in}{2.987343in}}%
\pgfpathlineto{\pgfqpoint{5.177418in}{2.993877in}}%
\pgfpathlineto{\pgfqpoint{5.191516in}{3.000587in}}%
\pgfpathlineto{\pgfqpoint{5.198957in}{3.007358in}}%
\pgfpathlineto{\pgfqpoint{5.206393in}{3.014203in}}%
\pgfpathlineto{\pgfqpoint{5.213826in}{3.021130in}}%
\pgfpathlineto{\pgfqpoint{5.221255in}{3.028144in}}%
\pgfpathlineto{\pgfqpoint{5.207176in}{3.021959in}}%
\pgfpathlineto{\pgfqpoint{5.193113in}{3.015950in}}%
\pgfpathlineto{\pgfqpoint{5.179065in}{3.010116in}}%
\pgfpathlineto{\pgfqpoint{5.165032in}{3.004459in}}%
\pgfpathlineto{\pgfqpoint{5.157583in}{2.996910in}}%
\pgfpathlineto{\pgfqpoint{5.150130in}{2.989455in}}%
\pgfpathlineto{\pgfqpoint{5.142674in}{2.982089in}}%
\pgfpathlineto{\pgfqpoint{5.135214in}{2.974806in}}%
\pgfpathclose%
\pgfusepath{fill}%
\end{pgfscope}%
\begin{pgfscope}%
\pgfpathrectangle{\pgfqpoint{1.150000in}{0.150000in}}{\pgfqpoint{5.700000in}{5.700000in}}%
\pgfusepath{clip}%
\pgfsetbuttcap%
\pgfsetroundjoin%
\definecolor{currentfill}{rgb}{0.278826,0.175490,0.483397}%
\pgfsetfillcolor{currentfill}%
\pgfsetfillopacity{0.800000}%
\pgfsetlinewidth{0.000000pt}%
\definecolor{currentstroke}{rgb}{0.000000,0.000000,0.000000}%
\pgfsetstrokecolor{currentstroke}%
\pgfsetdash{}{0pt}%
\pgfpathmoveto{\pgfqpoint{4.073172in}{2.324179in}}%
\pgfpathlineto{\pgfqpoint{4.086790in}{2.325323in}}%
\pgfpathlineto{\pgfqpoint{4.100417in}{2.326663in}}%
\pgfpathlineto{\pgfqpoint{4.114053in}{2.328198in}}%
\pgfpathlineto{\pgfqpoint{4.127698in}{2.329927in}}%
\pgfpathlineto{\pgfqpoint{4.135549in}{2.339980in}}%
\pgfpathlineto{\pgfqpoint{4.143396in}{2.350014in}}%
\pgfpathlineto{\pgfqpoint{4.151237in}{2.360028in}}%
\pgfpathlineto{\pgfqpoint{4.159073in}{2.370025in}}%
\pgfpathlineto{\pgfqpoint{4.145436in}{2.368399in}}%
\pgfpathlineto{\pgfqpoint{4.131808in}{2.366966in}}%
\pgfpathlineto{\pgfqpoint{4.118189in}{2.365728in}}%
\pgfpathlineto{\pgfqpoint{4.104579in}{2.364686in}}%
\pgfpathlineto{\pgfqpoint{4.096735in}{2.354575in}}%
\pgfpathlineto{\pgfqpoint{4.088886in}{2.344454in}}%
\pgfpathlineto{\pgfqpoint{4.081032in}{2.334323in}}%
\pgfpathlineto{\pgfqpoint{4.073172in}{2.324179in}}%
\pgfpathclose%
\pgfusepath{fill}%
\end{pgfscope}%
\begin{pgfscope}%
\pgfpathrectangle{\pgfqpoint{1.150000in}{0.150000in}}{\pgfqpoint{5.700000in}{5.700000in}}%
\pgfusepath{clip}%
\pgfsetbuttcap%
\pgfsetroundjoin%
\definecolor{currentfill}{rgb}{0.174274,0.445044,0.557792}%
\pgfsetfillcolor{currentfill}%
\pgfsetfillopacity{0.800000}%
\pgfsetlinewidth{0.000000pt}%
\definecolor{currentstroke}{rgb}{0.000000,0.000000,0.000000}%
\pgfsetstrokecolor{currentstroke}%
\pgfsetdash{}{0pt}%
\pgfpathmoveto{\pgfqpoint{5.221255in}{3.028144in}}%
\pgfpathlineto{\pgfqpoint{5.235348in}{3.034505in}}%
\pgfpathlineto{\pgfqpoint{5.249457in}{3.041042in}}%
\pgfpathlineto{\pgfqpoint{5.263582in}{3.047755in}}%
\pgfpathlineto{\pgfqpoint{5.277722in}{3.054643in}}%
\pgfpathlineto{\pgfqpoint{5.285126in}{3.061204in}}%
\pgfpathlineto{\pgfqpoint{5.292527in}{3.067857in}}%
\pgfpathlineto{\pgfqpoint{5.299924in}{3.074607in}}%
\pgfpathlineto{\pgfqpoint{5.307318in}{3.081462in}}%
\pgfpathlineto{\pgfqpoint{5.293200in}{3.075132in}}%
\pgfpathlineto{\pgfqpoint{5.279097in}{3.068976in}}%
\pgfpathlineto{\pgfqpoint{5.265009in}{3.062996in}}%
\pgfpathlineto{\pgfqpoint{5.250937in}{3.057191in}}%
\pgfpathlineto{\pgfqpoint{5.243521in}{3.049768in}}%
\pgfpathlineto{\pgfqpoint{5.236102in}{3.042457in}}%
\pgfpathlineto{\pgfqpoint{5.228680in}{3.035251in}}%
\pgfpathlineto{\pgfqpoint{5.221255in}{3.028144in}}%
\pgfpathclose%
\pgfusepath{fill}%
\end{pgfscope}%
\begin{pgfscope}%
\pgfpathrectangle{\pgfqpoint{1.150000in}{0.150000in}}{\pgfqpoint{5.700000in}{5.700000in}}%
\pgfusepath{clip}%
\pgfsetbuttcap%
\pgfsetroundjoin%
\definecolor{currentfill}{rgb}{0.275191,0.194905,0.496005}%
\pgfsetfillcolor{currentfill}%
\pgfsetfillopacity{0.800000}%
\pgfsetlinewidth{0.000000pt}%
\definecolor{currentstroke}{rgb}{0.000000,0.000000,0.000000}%
\pgfsetstrokecolor{currentstroke}%
\pgfsetdash{}{0pt}%
\pgfpathmoveto{\pgfqpoint{4.159073in}{2.370025in}}%
\pgfpathlineto{\pgfqpoint{4.172720in}{2.371846in}}%
\pgfpathlineto{\pgfqpoint{4.186376in}{2.373860in}}%
\pgfpathlineto{\pgfqpoint{4.200042in}{2.376067in}}%
\pgfpathlineto{\pgfqpoint{4.213717in}{2.378466in}}%
\pgfpathlineto{\pgfqpoint{4.221540in}{2.388325in}}%
\pgfpathlineto{\pgfqpoint{4.229359in}{2.398163in}}%
\pgfpathlineto{\pgfqpoint{4.237172in}{2.407981in}}%
\pgfpathlineto{\pgfqpoint{4.244979in}{2.417781in}}%
\pgfpathlineto{\pgfqpoint{4.231312in}{2.415516in}}%
\pgfpathlineto{\pgfqpoint{4.217654in}{2.413443in}}%
\pgfpathlineto{\pgfqpoint{4.204005in}{2.411563in}}%
\pgfpathlineto{\pgfqpoint{4.190366in}{2.409877in}}%
\pgfpathlineto{\pgfqpoint{4.182551in}{2.399931in}}%
\pgfpathlineto{\pgfqpoint{4.174730in}{2.389975in}}%
\pgfpathlineto{\pgfqpoint{4.166904in}{2.380007in}}%
\pgfpathlineto{\pgfqpoint{4.159073in}{2.370025in}}%
\pgfpathclose%
\pgfusepath{fill}%
\end{pgfscope}%
\begin{pgfscope}%
\pgfpathrectangle{\pgfqpoint{1.150000in}{0.150000in}}{\pgfqpoint{5.700000in}{5.700000in}}%
\pgfusepath{clip}%
\pgfsetbuttcap%
\pgfsetroundjoin%
\definecolor{currentfill}{rgb}{0.281412,0.155834,0.469201}%
\pgfsetfillcolor{currentfill}%
\pgfsetfillopacity{0.800000}%
\pgfsetlinewidth{0.000000pt}%
\definecolor{currentstroke}{rgb}{0.000000,0.000000,0.000000}%
\pgfsetstrokecolor{currentstroke}%
\pgfsetdash{}{0pt}%
\pgfpathmoveto{\pgfqpoint{3.987267in}{2.280560in}}%
\pgfpathlineto{\pgfqpoint{4.000859in}{2.280986in}}%
\pgfpathlineto{\pgfqpoint{4.014459in}{2.281611in}}%
\pgfpathlineto{\pgfqpoint{4.028068in}{2.282433in}}%
\pgfpathlineto{\pgfqpoint{4.041685in}{2.283452in}}%
\pgfpathlineto{\pgfqpoint{4.049564in}{2.293659in}}%
\pgfpathlineto{\pgfqpoint{4.057439in}{2.303848in}}%
\pgfpathlineto{\pgfqpoint{4.065308in}{2.314021in}}%
\pgfpathlineto{\pgfqpoint{4.073172in}{2.324179in}}%
\pgfpathlineto{\pgfqpoint{4.059563in}{2.323230in}}%
\pgfpathlineto{\pgfqpoint{4.045962in}{2.322479in}}%
\pgfpathlineto{\pgfqpoint{4.032370in}{2.321924in}}%
\pgfpathlineto{\pgfqpoint{4.018786in}{2.321567in}}%
\pgfpathlineto{\pgfqpoint{4.010914in}{2.311328in}}%
\pgfpathlineto{\pgfqpoint{4.003037in}{2.301081in}}%
\pgfpathlineto{\pgfqpoint{3.995155in}{2.290825in}}%
\pgfpathlineto{\pgfqpoint{3.987267in}{2.280560in}}%
\pgfpathclose%
\pgfusepath{fill}%
\end{pgfscope}%
\begin{pgfscope}%
\pgfpathrectangle{\pgfqpoint{1.150000in}{0.150000in}}{\pgfqpoint{5.700000in}{5.700000in}}%
\pgfusepath{clip}%
\pgfsetbuttcap%
\pgfsetroundjoin%
\definecolor{currentfill}{rgb}{0.270595,0.214069,0.507052}%
\pgfsetfillcolor{currentfill}%
\pgfsetfillopacity{0.800000}%
\pgfsetlinewidth{0.000000pt}%
\definecolor{currentstroke}{rgb}{0.000000,0.000000,0.000000}%
\pgfsetstrokecolor{currentstroke}%
\pgfsetdash{}{0pt}%
\pgfpathmoveto{\pgfqpoint{4.244979in}{2.417781in}}%
\pgfpathlineto{\pgfqpoint{4.258657in}{2.420237in}}%
\pgfpathlineto{\pgfqpoint{4.272345in}{2.422885in}}%
\pgfpathlineto{\pgfqpoint{4.286043in}{2.425723in}}%
\pgfpathlineto{\pgfqpoint{4.299752in}{2.428752in}}%
\pgfpathlineto{\pgfqpoint{4.307547in}{2.438382in}}%
\pgfpathlineto{\pgfqpoint{4.315336in}{2.447989in}}%
\pgfpathlineto{\pgfqpoint{4.323120in}{2.457577in}}%
\pgfpathlineto{\pgfqpoint{4.330899in}{2.467148in}}%
\pgfpathlineto{\pgfqpoint{4.317198in}{2.464285in}}%
\pgfpathlineto{\pgfqpoint{4.303508in}{2.461613in}}%
\pgfpathlineto{\pgfqpoint{4.289828in}{2.459132in}}%
\pgfpathlineto{\pgfqpoint{4.276158in}{2.456841in}}%
\pgfpathlineto{\pgfqpoint{4.268371in}{2.447093in}}%
\pgfpathlineto{\pgfqpoint{4.260579in}{2.437334in}}%
\pgfpathlineto{\pgfqpoint{4.252782in}{2.427564in}}%
\pgfpathlineto{\pgfqpoint{4.244979in}{2.417781in}}%
\pgfpathclose%
\pgfusepath{fill}%
\end{pgfscope}%
\begin{pgfscope}%
\pgfpathrectangle{\pgfqpoint{1.150000in}{0.150000in}}{\pgfqpoint{5.700000in}{5.700000in}}%
\pgfusepath{clip}%
\pgfsetbuttcap%
\pgfsetroundjoin%
\definecolor{currentfill}{rgb}{0.280894,0.078907,0.402329}%
\pgfsetfillcolor{currentfill}%
\pgfsetfillopacity{0.800000}%
\pgfsetlinewidth{0.000000pt}%
\definecolor{currentstroke}{rgb}{0.000000,0.000000,0.000000}%
\pgfsetstrokecolor{currentstroke}%
\pgfsetdash{}{0pt}%
\pgfpathmoveto{\pgfqpoint{3.503140in}{2.126110in}}%
\pgfpathlineto{\pgfqpoint{3.516641in}{2.121437in}}%
\pgfpathlineto{\pgfqpoint{3.530145in}{2.116982in}}%
\pgfpathlineto{\pgfqpoint{3.543652in}{2.112743in}}%
\pgfpathlineto{\pgfqpoint{3.557164in}{2.108720in}}%
\pgfpathlineto{\pgfqpoint{3.565201in}{2.118800in}}%
\pgfpathlineto{\pgfqpoint{3.573233in}{2.128904in}}%
\pgfpathlineto{\pgfqpoint{3.581259in}{2.139030in}}%
\pgfpathlineto{\pgfqpoint{3.589280in}{2.149180in}}%
\pgfpathlineto{\pgfqpoint{3.575780in}{2.153115in}}%
\pgfpathlineto{\pgfqpoint{3.562284in}{2.157265in}}%
\pgfpathlineto{\pgfqpoint{3.548791in}{2.161632in}}%
\pgfpathlineto{\pgfqpoint{3.535302in}{2.166216in}}%
\pgfpathlineto{\pgfqpoint{3.527270in}{2.156144in}}%
\pgfpathlineto{\pgfqpoint{3.519233in}{2.146101in}}%
\pgfpathlineto{\pgfqpoint{3.511189in}{2.136090in}}%
\pgfpathlineto{\pgfqpoint{3.503140in}{2.126110in}}%
\pgfpathclose%
\pgfusepath{fill}%
\end{pgfscope}%
\begin{pgfscope}%
\pgfpathrectangle{\pgfqpoint{1.150000in}{0.150000in}}{\pgfqpoint{5.700000in}{5.700000in}}%
\pgfusepath{clip}%
\pgfsetbuttcap%
\pgfsetroundjoin%
\definecolor{currentfill}{rgb}{0.282884,0.135920,0.453427}%
\pgfsetfillcolor{currentfill}%
\pgfsetfillopacity{0.800000}%
\pgfsetlinewidth{0.000000pt}%
\definecolor{currentstroke}{rgb}{0.000000,0.000000,0.000000}%
\pgfsetstrokecolor{currentstroke}%
\pgfsetdash{}{0pt}%
\pgfpathmoveto{\pgfqpoint{3.901346in}{2.239511in}}%
\pgfpathlineto{\pgfqpoint{3.914915in}{2.239178in}}%
\pgfpathlineto{\pgfqpoint{3.928491in}{2.239046in}}%
\pgfpathlineto{\pgfqpoint{3.942075in}{2.239113in}}%
\pgfpathlineto{\pgfqpoint{3.955667in}{2.239380in}}%
\pgfpathlineto{\pgfqpoint{3.963575in}{2.249694in}}%
\pgfpathlineto{\pgfqpoint{3.971477in}{2.259995in}}%
\pgfpathlineto{\pgfqpoint{3.979375in}{2.270283in}}%
\pgfpathlineto{\pgfqpoint{3.987267in}{2.280560in}}%
\pgfpathlineto{\pgfqpoint{3.973683in}{2.280331in}}%
\pgfpathlineto{\pgfqpoint{3.960107in}{2.280302in}}%
\pgfpathlineto{\pgfqpoint{3.946538in}{2.280473in}}%
\pgfpathlineto{\pgfqpoint{3.932977in}{2.280844in}}%
\pgfpathlineto{\pgfqpoint{3.925077in}{2.270517in}}%
\pgfpathlineto{\pgfqpoint{3.917172in}{2.260187in}}%
\pgfpathlineto{\pgfqpoint{3.909261in}{2.249852in}}%
\pgfpathlineto{\pgfqpoint{3.901346in}{2.239511in}}%
\pgfpathclose%
\pgfusepath{fill}%
\end{pgfscope}%
\begin{pgfscope}%
\pgfpathrectangle{\pgfqpoint{1.150000in}{0.150000in}}{\pgfqpoint{5.700000in}{5.700000in}}%
\pgfusepath{clip}%
\pgfsetbuttcap%
\pgfsetroundjoin%
\definecolor{currentfill}{rgb}{0.263663,0.237631,0.518762}%
\pgfsetfillcolor{currentfill}%
\pgfsetfillopacity{0.800000}%
\pgfsetlinewidth{0.000000pt}%
\definecolor{currentstroke}{rgb}{0.000000,0.000000,0.000000}%
\pgfsetstrokecolor{currentstroke}%
\pgfsetdash{}{0pt}%
\pgfpathmoveto{\pgfqpoint{4.330899in}{2.467148in}}%
\pgfpathlineto{\pgfqpoint{4.344610in}{2.470200in}}%
\pgfpathlineto{\pgfqpoint{4.358332in}{2.473441in}}%
\pgfpathlineto{\pgfqpoint{4.372065in}{2.476872in}}%
\pgfpathlineto{\pgfqpoint{4.385809in}{2.480491in}}%
\pgfpathlineto{\pgfqpoint{4.393574in}{2.489860in}}%
\pgfpathlineto{\pgfqpoint{4.401334in}{2.499208in}}%
\pgfpathlineto{\pgfqpoint{4.409089in}{2.508539in}}%
\pgfpathlineto{\pgfqpoint{4.416838in}{2.517854in}}%
\pgfpathlineto{\pgfqpoint{4.403102in}{2.514433in}}%
\pgfpathlineto{\pgfqpoint{4.389378in}{2.511202in}}%
\pgfpathlineto{\pgfqpoint{4.375665in}{2.508159in}}%
\pgfpathlineto{\pgfqpoint{4.361962in}{2.505305in}}%
\pgfpathlineto{\pgfqpoint{4.354204in}{2.495779in}}%
\pgfpathlineto{\pgfqpoint{4.346441in}{2.486246in}}%
\pgfpathlineto{\pgfqpoint{4.338672in}{2.476703in}}%
\pgfpathlineto{\pgfqpoint{4.330899in}{2.467148in}}%
\pgfpathclose%
\pgfusepath{fill}%
\end{pgfscope}%
\begin{pgfscope}%
\pgfpathrectangle{\pgfqpoint{1.150000in}{0.150000in}}{\pgfqpoint{5.700000in}{5.700000in}}%
\pgfusepath{clip}%
\pgfsetbuttcap%
\pgfsetroundjoin%
\definecolor{currentfill}{rgb}{0.166617,0.463708,0.558119}%
\pgfsetfillcolor{currentfill}%
\pgfsetfillopacity{0.800000}%
\pgfsetlinewidth{0.000000pt}%
\definecolor{currentstroke}{rgb}{0.000000,0.000000,0.000000}%
\pgfsetstrokecolor{currentstroke}%
\pgfsetdash{}{0pt}%
\pgfpathmoveto{\pgfqpoint{5.307318in}{3.081462in}}%
\pgfpathlineto{\pgfqpoint{5.321452in}{3.087967in}}%
\pgfpathlineto{\pgfqpoint{5.335602in}{3.094647in}}%
\pgfpathlineto{\pgfqpoint{5.349768in}{3.101502in}}%
\pgfpathlineto{\pgfqpoint{5.363949in}{3.108532in}}%
\pgfpathlineto{\pgfqpoint{5.371317in}{3.114918in}}%
\pgfpathlineto{\pgfqpoint{5.378682in}{3.121414in}}%
\pgfpathlineto{\pgfqpoint{5.386045in}{3.128026in}}%
\pgfpathlineto{\pgfqpoint{5.393404in}{3.134761in}}%
\pgfpathlineto{\pgfqpoint{5.379246in}{3.128322in}}%
\pgfpathlineto{\pgfqpoint{5.365104in}{3.122057in}}%
\pgfpathlineto{\pgfqpoint{5.350978in}{3.115967in}}%
\pgfpathlineto{\pgfqpoint{5.336867in}{3.110050in}}%
\pgfpathlineto{\pgfqpoint{5.329484in}{3.102715in}}%
\pgfpathlineto{\pgfqpoint{5.322098in}{3.095509in}}%
\pgfpathlineto{\pgfqpoint{5.314709in}{3.088427in}}%
\pgfpathlineto{\pgfqpoint{5.307318in}{3.081462in}}%
\pgfpathclose%
\pgfusepath{fill}%
\end{pgfscope}%
\begin{pgfscope}%
\pgfpathrectangle{\pgfqpoint{1.150000in}{0.150000in}}{\pgfqpoint{5.700000in}{5.700000in}}%
\pgfusepath{clip}%
\pgfsetbuttcap%
\pgfsetroundjoin%
\definecolor{currentfill}{rgb}{0.283197,0.115680,0.436115}%
\pgfsetfillcolor{currentfill}%
\pgfsetfillopacity{0.800000}%
\pgfsetlinewidth{0.000000pt}%
\definecolor{currentstroke}{rgb}{0.000000,0.000000,0.000000}%
\pgfsetstrokecolor{currentstroke}%
\pgfsetdash{}{0pt}%
\pgfpathmoveto{\pgfqpoint{3.815393in}{2.201399in}}%
\pgfpathlineto{\pgfqpoint{3.828943in}{2.200264in}}%
\pgfpathlineto{\pgfqpoint{3.842499in}{2.199332in}}%
\pgfpathlineto{\pgfqpoint{3.856062in}{2.198603in}}%
\pgfpathlineto{\pgfqpoint{3.869633in}{2.198076in}}%
\pgfpathlineto{\pgfqpoint{3.877568in}{2.208447in}}%
\pgfpathlineto{\pgfqpoint{3.885499in}{2.218809in}}%
\pgfpathlineto{\pgfqpoint{3.893425in}{2.229163in}}%
\pgfpathlineto{\pgfqpoint{3.901346in}{2.239511in}}%
\pgfpathlineto{\pgfqpoint{3.887784in}{2.240045in}}%
\pgfpathlineto{\pgfqpoint{3.874229in}{2.240781in}}%
\pgfpathlineto{\pgfqpoint{3.860681in}{2.241719in}}%
\pgfpathlineto{\pgfqpoint{3.847139in}{2.242861in}}%
\pgfpathlineto{\pgfqpoint{3.839211in}{2.232495in}}%
\pgfpathlineto{\pgfqpoint{3.831276in}{2.222130in}}%
\pgfpathlineto{\pgfqpoint{3.823337in}{2.211765in}}%
\pgfpathlineto{\pgfqpoint{3.815393in}{2.201399in}}%
\pgfpathclose%
\pgfusepath{fill}%
\end{pgfscope}%
\begin{pgfscope}%
\pgfpathrectangle{\pgfqpoint{1.150000in}{0.150000in}}{\pgfqpoint{5.700000in}{5.700000in}}%
\pgfusepath{clip}%
\pgfsetbuttcap%
\pgfsetroundjoin%
\definecolor{currentfill}{rgb}{0.282910,0.105393,0.426902}%
\pgfsetfillcolor{currentfill}%
\pgfsetfillopacity{0.800000}%
\pgfsetlinewidth{0.000000pt}%
\definecolor{currentstroke}{rgb}{0.000000,0.000000,0.000000}%
\pgfsetstrokecolor{currentstroke}%
\pgfsetdash{}{0pt}%
\pgfpathmoveto{\pgfqpoint{3.081484in}{2.200584in}}%
\pgfpathlineto{\pgfqpoint{3.095015in}{2.189856in}}%
\pgfpathlineto{\pgfqpoint{3.108544in}{2.179378in}}%
\pgfpathlineto{\pgfqpoint{3.122072in}{2.169148in}}%
\pgfpathlineto{\pgfqpoint{3.135598in}{2.159165in}}%
\pgfpathlineto{\pgfqpoint{3.143800in}{2.167722in}}%
\pgfpathlineto{\pgfqpoint{3.151995in}{2.176363in}}%
\pgfpathlineto{\pgfqpoint{3.160182in}{2.185084in}}%
\pgfpathlineto{\pgfqpoint{3.168363in}{2.193886in}}%
\pgfpathlineto{\pgfqpoint{3.154855in}{2.203684in}}%
\pgfpathlineto{\pgfqpoint{3.141347in}{2.213727in}}%
\pgfpathlineto{\pgfqpoint{3.127836in}{2.224020in}}%
\pgfpathlineto{\pgfqpoint{3.114324in}{2.234562in}}%
\pgfpathlineto{\pgfqpoint{3.106126in}{2.225934in}}%
\pgfpathlineto{\pgfqpoint{3.097919in}{2.217394in}}%
\pgfpathlineto{\pgfqpoint{3.089706in}{2.208944in}}%
\pgfpathlineto{\pgfqpoint{3.081484in}{2.200584in}}%
\pgfpathclose%
\pgfusepath{fill}%
\end{pgfscope}%
\begin{pgfscope}%
\pgfpathrectangle{\pgfqpoint{1.150000in}{0.150000in}}{\pgfqpoint{5.700000in}{5.700000in}}%
\pgfusepath{clip}%
\pgfsetbuttcap%
\pgfsetroundjoin%
\definecolor{currentfill}{rgb}{0.255645,0.260703,0.528312}%
\pgfsetfillcolor{currentfill}%
\pgfsetfillopacity{0.800000}%
\pgfsetlinewidth{0.000000pt}%
\definecolor{currentstroke}{rgb}{0.000000,0.000000,0.000000}%
\pgfsetstrokecolor{currentstroke}%
\pgfsetdash{}{0pt}%
\pgfpathmoveto{\pgfqpoint{4.416838in}{2.517854in}}%
\pgfpathlineto{\pgfqpoint{4.430584in}{2.521462in}}%
\pgfpathlineto{\pgfqpoint{4.444343in}{2.525258in}}%
\pgfpathlineto{\pgfqpoint{4.458112in}{2.529241in}}%
\pgfpathlineto{\pgfqpoint{4.471893in}{2.533411in}}%
\pgfpathlineto{\pgfqpoint{4.479628in}{2.542494in}}%
\pgfpathlineto{\pgfqpoint{4.487358in}{2.551559in}}%
\pgfpathlineto{\pgfqpoint{4.495082in}{2.560609in}}%
\pgfpathlineto{\pgfqpoint{4.502801in}{2.569647in}}%
\pgfpathlineto{\pgfqpoint{4.489029in}{2.565709in}}%
\pgfpathlineto{\pgfqpoint{4.475268in}{2.561957in}}%
\pgfpathlineto{\pgfqpoint{4.461519in}{2.558392in}}%
\pgfpathlineto{\pgfqpoint{4.447782in}{2.555014in}}%
\pgfpathlineto{\pgfqpoint{4.440054in}{2.545733in}}%
\pgfpathlineto{\pgfqpoint{4.432320in}{2.536448in}}%
\pgfpathlineto{\pgfqpoint{4.424582in}{2.527156in}}%
\pgfpathlineto{\pgfqpoint{4.416838in}{2.517854in}}%
\pgfpathclose%
\pgfusepath{fill}%
\end{pgfscope}%
\begin{pgfscope}%
\pgfpathrectangle{\pgfqpoint{1.150000in}{0.150000in}}{\pgfqpoint{5.700000in}{5.700000in}}%
\pgfusepath{clip}%
\pgfsetbuttcap%
\pgfsetroundjoin%
\definecolor{currentfill}{rgb}{0.208623,0.367752,0.552675}%
\pgfsetfillcolor{currentfill}%
\pgfsetfillopacity{0.800000}%
\pgfsetlinewidth{0.000000pt}%
\definecolor{currentstroke}{rgb}{0.000000,0.000000,0.000000}%
\pgfsetstrokecolor{currentstroke}%
\pgfsetdash{}{0pt}%
\pgfpathmoveto{\pgfqpoint{2.535653in}{2.858618in}}%
\pgfpathlineto{\pgfqpoint{2.549481in}{2.836072in}}%
\pgfpathlineto{\pgfqpoint{2.563296in}{2.813876in}}%
\pgfpathlineto{\pgfqpoint{2.577098in}{2.792027in}}%
\pgfpathlineto{\pgfqpoint{2.590887in}{2.770522in}}%
\pgfpathlineto{\pgfqpoint{2.599331in}{2.777044in}}%
\pgfpathlineto{\pgfqpoint{2.607764in}{2.783730in}}%
\pgfpathlineto{\pgfqpoint{2.616185in}{2.790578in}}%
\pgfpathlineto{\pgfqpoint{2.624596in}{2.797587in}}%
\pgfpathlineto{\pgfqpoint{2.610837in}{2.818893in}}%
\pgfpathlineto{\pgfqpoint{2.597065in}{2.840541in}}%
\pgfpathlineto{\pgfqpoint{2.583281in}{2.862536in}}%
\pgfpathlineto{\pgfqpoint{2.569483in}{2.884880in}}%
\pgfpathlineto{\pgfqpoint{2.561043in}{2.878059in}}%
\pgfpathlineto{\pgfqpoint{2.552592in}{2.871407in}}%
\pgfpathlineto{\pgfqpoint{2.544128in}{2.864926in}}%
\pgfpathlineto{\pgfqpoint{2.535653in}{2.858618in}}%
\pgfpathclose%
\pgfusepath{fill}%
\end{pgfscope}%
\begin{pgfscope}%
\pgfpathrectangle{\pgfqpoint{1.150000in}{0.150000in}}{\pgfqpoint{5.700000in}{5.700000in}}%
\pgfusepath{clip}%
\pgfsetbuttcap%
\pgfsetroundjoin%
\definecolor{currentfill}{rgb}{0.159194,0.482237,0.558073}%
\pgfsetfillcolor{currentfill}%
\pgfsetfillopacity{0.800000}%
\pgfsetlinewidth{0.000000pt}%
\definecolor{currentstroke}{rgb}{0.000000,0.000000,0.000000}%
\pgfsetstrokecolor{currentstroke}%
\pgfsetdash{}{0pt}%
\pgfpathmoveto{\pgfqpoint{5.393404in}{3.134761in}}%
\pgfpathlineto{\pgfqpoint{5.407578in}{3.141373in}}%
\pgfpathlineto{\pgfqpoint{5.421768in}{3.148160in}}%
\pgfpathlineto{\pgfqpoint{5.435974in}{3.155120in}}%
\pgfpathlineto{\pgfqpoint{5.450197in}{3.162255in}}%
\pgfpathlineto{\pgfqpoint{5.457529in}{3.168507in}}%
\pgfpathlineto{\pgfqpoint{5.464859in}{3.174888in}}%
\pgfpathlineto{\pgfqpoint{5.472188in}{3.181405in}}%
\pgfpathlineto{\pgfqpoint{5.479514in}{3.188064in}}%
\pgfpathlineto{\pgfqpoint{5.465317in}{3.181553in}}%
\pgfpathlineto{\pgfqpoint{5.451136in}{3.175216in}}%
\pgfpathlineto{\pgfqpoint{5.436971in}{3.169051in}}%
\pgfpathlineto{\pgfqpoint{5.422823in}{3.163060in}}%
\pgfpathlineto{\pgfqpoint{5.415471in}{3.155767in}}%
\pgfpathlineto{\pgfqpoint{5.408117in}{3.148624in}}%
\pgfpathlineto{\pgfqpoint{5.400762in}{3.141625in}}%
\pgfpathlineto{\pgfqpoint{5.393404in}{3.134761in}}%
\pgfpathclose%
\pgfusepath{fill}%
\end{pgfscope}%
\begin{pgfscope}%
\pgfpathrectangle{\pgfqpoint{1.150000in}{0.150000in}}{\pgfqpoint{5.700000in}{5.700000in}}%
\pgfusepath{clip}%
\pgfsetbuttcap%
\pgfsetroundjoin%
\definecolor{currentfill}{rgb}{0.246811,0.283237,0.535941}%
\pgfsetfillcolor{currentfill}%
\pgfsetfillopacity{0.800000}%
\pgfsetlinewidth{0.000000pt}%
\definecolor{currentstroke}{rgb}{0.000000,0.000000,0.000000}%
\pgfsetstrokecolor{currentstroke}%
\pgfsetdash{}{0pt}%
\pgfpathmoveto{\pgfqpoint{4.502801in}{2.569647in}}%
\pgfpathlineto{\pgfqpoint{4.516584in}{2.573773in}}%
\pgfpathlineto{\pgfqpoint{4.530380in}{2.578084in}}%
\pgfpathlineto{\pgfqpoint{4.544188in}{2.582581in}}%
\pgfpathlineto{\pgfqpoint{4.558008in}{2.587263in}}%
\pgfpathlineto{\pgfqpoint{4.565712in}{2.596040in}}%
\pgfpathlineto{\pgfqpoint{4.573410in}{2.604804in}}%
\pgfpathlineto{\pgfqpoint{4.581103in}{2.613556in}}%
\pgfpathlineto{\pgfqpoint{4.588791in}{2.622302in}}%
\pgfpathlineto{\pgfqpoint{4.574980in}{2.617883in}}%
\pgfpathlineto{\pgfqpoint{4.561183in}{2.613650in}}%
\pgfpathlineto{\pgfqpoint{4.547397in}{2.609602in}}%
\pgfpathlineto{\pgfqpoint{4.533623in}{2.605740in}}%
\pgfpathlineto{\pgfqpoint{4.525925in}{2.596720in}}%
\pgfpathlineto{\pgfqpoint{4.518222in}{2.587700in}}%
\pgfpathlineto{\pgfqpoint{4.510514in}{2.578677in}}%
\pgfpathlineto{\pgfqpoint{4.502801in}{2.569647in}}%
\pgfpathclose%
\pgfusepath{fill}%
\end{pgfscope}%
\begin{pgfscope}%
\pgfpathrectangle{\pgfqpoint{1.150000in}{0.150000in}}{\pgfqpoint{5.700000in}{5.700000in}}%
\pgfusepath{clip}%
\pgfsetbuttcap%
\pgfsetroundjoin%
\definecolor{currentfill}{rgb}{0.282656,0.100196,0.422160}%
\pgfsetfillcolor{currentfill}%
\pgfsetfillopacity{0.800000}%
\pgfsetlinewidth{0.000000pt}%
\definecolor{currentstroke}{rgb}{0.000000,0.000000,0.000000}%
\pgfsetstrokecolor{currentstroke}%
\pgfsetdash{}{0pt}%
\pgfpathmoveto{\pgfqpoint{3.729392in}{2.166616in}}%
\pgfpathlineto{\pgfqpoint{3.742926in}{2.164634in}}%
\pgfpathlineto{\pgfqpoint{3.756466in}{2.162860in}}%
\pgfpathlineto{\pgfqpoint{3.770012in}{2.161291in}}%
\pgfpathlineto{\pgfqpoint{3.783564in}{2.159927in}}%
\pgfpathlineto{\pgfqpoint{3.791529in}{2.170297in}}%
\pgfpathlineto{\pgfqpoint{3.799489in}{2.180666in}}%
\pgfpathlineto{\pgfqpoint{3.807444in}{2.191033in}}%
\pgfpathlineto{\pgfqpoint{3.815393in}{2.201399in}}%
\pgfpathlineto{\pgfqpoint{3.801850in}{2.202739in}}%
\pgfpathlineto{\pgfqpoint{3.788313in}{2.204283in}}%
\pgfpathlineto{\pgfqpoint{3.774782in}{2.206032in}}%
\pgfpathlineto{\pgfqpoint{3.761257in}{2.207988in}}%
\pgfpathlineto{\pgfqpoint{3.753298in}{2.197635in}}%
\pgfpathlineto{\pgfqpoint{3.745335in}{2.187289in}}%
\pgfpathlineto{\pgfqpoint{3.737366in}{2.176949in}}%
\pgfpathlineto{\pgfqpoint{3.729392in}{2.166616in}}%
\pgfpathclose%
\pgfusepath{fill}%
\end{pgfscope}%
\begin{pgfscope}%
\pgfpathrectangle{\pgfqpoint{1.150000in}{0.150000in}}{\pgfqpoint{5.700000in}{5.700000in}}%
\pgfusepath{clip}%
\pgfsetbuttcap%
\pgfsetroundjoin%
\definecolor{currentfill}{rgb}{0.150476,0.504369,0.557430}%
\pgfsetfillcolor{currentfill}%
\pgfsetfillopacity{0.800000}%
\pgfsetlinewidth{0.000000pt}%
\definecolor{currentstroke}{rgb}{0.000000,0.000000,0.000000}%
\pgfsetstrokecolor{currentstroke}%
\pgfsetdash{}{0pt}%
\pgfpathmoveto{\pgfqpoint{5.479514in}{3.188064in}}%
\pgfpathlineto{\pgfqpoint{5.493727in}{3.194748in}}%
\pgfpathlineto{\pgfqpoint{5.507956in}{3.201604in}}%
\pgfpathlineto{\pgfqpoint{5.522202in}{3.208634in}}%
\pgfpathlineto{\pgfqpoint{5.536465in}{3.215836in}}%
\pgfpathlineto{\pgfqpoint{5.543763in}{3.222001in}}%
\pgfpathlineto{\pgfqpoint{5.551059in}{3.228314in}}%
\pgfpathlineto{\pgfqpoint{5.558354in}{3.234785in}}%
\pgfpathlineto{\pgfqpoint{5.565647in}{3.241419in}}%
\pgfpathlineto{\pgfqpoint{5.551413in}{3.234873in}}%
\pgfpathlineto{\pgfqpoint{5.537194in}{3.228499in}}%
\pgfpathlineto{\pgfqpoint{5.522992in}{3.222297in}}%
\pgfpathlineto{\pgfqpoint{5.508806in}{3.216267in}}%
\pgfpathlineto{\pgfqpoint{5.501484in}{3.208967in}}%
\pgfpathlineto{\pgfqpoint{5.494162in}{3.201838in}}%
\pgfpathlineto{\pgfqpoint{5.486838in}{3.194873in}}%
\pgfpathlineto{\pgfqpoint{5.479514in}{3.188064in}}%
\pgfpathclose%
\pgfusepath{fill}%
\end{pgfscope}%
\begin{pgfscope}%
\pgfpathrectangle{\pgfqpoint{1.150000in}{0.150000in}}{\pgfqpoint{5.700000in}{5.700000in}}%
\pgfusepath{clip}%
\pgfsetbuttcap%
\pgfsetroundjoin%
\definecolor{currentfill}{rgb}{0.280894,0.078907,0.402329}%
\pgfsetfillcolor{currentfill}%
\pgfsetfillopacity{0.800000}%
\pgfsetlinewidth{0.000000pt}%
\definecolor{currentstroke}{rgb}{0.000000,0.000000,0.000000}%
\pgfsetstrokecolor{currentstroke}%
\pgfsetdash{}{0pt}%
\pgfpathmoveto{\pgfqpoint{3.276396in}{2.124203in}}%
\pgfpathlineto{\pgfqpoint{3.289901in}{2.116557in}}%
\pgfpathlineto{\pgfqpoint{3.303406in}{2.109143in}}%
\pgfpathlineto{\pgfqpoint{3.316912in}{2.101960in}}%
\pgfpathlineto{\pgfqpoint{3.330419in}{2.095006in}}%
\pgfpathlineto{\pgfqpoint{3.338543in}{2.104381in}}%
\pgfpathlineto{\pgfqpoint{3.346661in}{2.113810in}}%
\pgfpathlineto{\pgfqpoint{3.354772in}{2.123291in}}%
\pgfpathlineto{\pgfqpoint{3.362877in}{2.132825in}}%
\pgfpathlineto{\pgfqpoint{3.349385in}{2.139626in}}%
\pgfpathlineto{\pgfqpoint{3.335894in}{2.146657in}}%
\pgfpathlineto{\pgfqpoint{3.322404in}{2.153919in}}%
\pgfpathlineto{\pgfqpoint{3.308915in}{2.161412in}}%
\pgfpathlineto{\pgfqpoint{3.300795in}{2.152019in}}%
\pgfpathlineto{\pgfqpoint{3.292669in}{2.142686in}}%
\pgfpathlineto{\pgfqpoint{3.284536in}{2.133413in}}%
\pgfpathlineto{\pgfqpoint{3.276396in}{2.124203in}}%
\pgfpathclose%
\pgfusepath{fill}%
\end{pgfscope}%
\begin{pgfscope}%
\pgfpathrectangle{\pgfqpoint{1.150000in}{0.150000in}}{\pgfqpoint{5.700000in}{5.700000in}}%
\pgfusepath{clip}%
\pgfsetbuttcap%
\pgfsetroundjoin%
\definecolor{currentfill}{rgb}{0.237441,0.305202,0.541921}%
\pgfsetfillcolor{currentfill}%
\pgfsetfillopacity{0.800000}%
\pgfsetlinewidth{0.000000pt}%
\definecolor{currentstroke}{rgb}{0.000000,0.000000,0.000000}%
\pgfsetstrokecolor{currentstroke}%
\pgfsetdash{}{0pt}%
\pgfpathmoveto{\pgfqpoint{4.588791in}{2.622302in}}%
\pgfpathlineto{\pgfqpoint{4.602613in}{2.626905in}}%
\pgfpathlineto{\pgfqpoint{4.616448in}{2.631693in}}%
\pgfpathlineto{\pgfqpoint{4.630295in}{2.636665in}}%
\pgfpathlineto{\pgfqpoint{4.644155in}{2.641821in}}%
\pgfpathlineto{\pgfqpoint{4.651827in}{2.650278in}}%
\pgfpathlineto{\pgfqpoint{4.659493in}{2.658726in}}%
\pgfpathlineto{\pgfqpoint{4.667154in}{2.667169in}}%
\pgfpathlineto{\pgfqpoint{4.674809in}{2.675611in}}%
\pgfpathlineto{\pgfqpoint{4.660960in}{2.670752in}}%
\pgfpathlineto{\pgfqpoint{4.647123in}{2.666076in}}%
\pgfpathlineto{\pgfqpoint{4.633299in}{2.661584in}}%
\pgfpathlineto{\pgfqpoint{4.619487in}{2.657277in}}%
\pgfpathlineto{\pgfqpoint{4.611821in}{2.648527in}}%
\pgfpathlineto{\pgfqpoint{4.604150in}{2.639784in}}%
\pgfpathlineto{\pgfqpoint{4.596473in}{2.631043in}}%
\pgfpathlineto{\pgfqpoint{4.588791in}{2.622302in}}%
\pgfpathclose%
\pgfusepath{fill}%
\end{pgfscope}%
\begin{pgfscope}%
\pgfpathrectangle{\pgfqpoint{1.150000in}{0.150000in}}{\pgfqpoint{5.700000in}{5.700000in}}%
\pgfusepath{clip}%
\pgfsetbuttcap%
\pgfsetroundjoin%
\definecolor{currentfill}{rgb}{0.280267,0.073417,0.397163}%
\pgfsetfillcolor{currentfill}%
\pgfsetfillopacity{0.800000}%
\pgfsetlinewidth{0.000000pt}%
\definecolor{currentstroke}{rgb}{0.000000,0.000000,0.000000}%
\pgfsetstrokecolor{currentstroke}%
\pgfsetdash{}{0pt}%
\pgfpathmoveto{\pgfqpoint{3.416864in}{2.107885in}}%
\pgfpathlineto{\pgfqpoint{3.430365in}{2.102211in}}%
\pgfpathlineto{\pgfqpoint{3.443870in}{2.096759in}}%
\pgfpathlineto{\pgfqpoint{3.457377in}{2.091527in}}%
\pgfpathlineto{\pgfqpoint{3.470886in}{2.086516in}}%
\pgfpathlineto{\pgfqpoint{3.478959in}{2.096364in}}%
\pgfpathlineto{\pgfqpoint{3.487025in}{2.106246in}}%
\pgfpathlineto{\pgfqpoint{3.495085in}{2.116162in}}%
\pgfpathlineto{\pgfqpoint{3.503140in}{2.126110in}}%
\pgfpathlineto{\pgfqpoint{3.489643in}{2.131001in}}%
\pgfpathlineto{\pgfqpoint{3.476149in}{2.136113in}}%
\pgfpathlineto{\pgfqpoint{3.462657in}{2.141445in}}%
\pgfpathlineto{\pgfqpoint{3.449168in}{2.146999in}}%
\pgfpathlineto{\pgfqpoint{3.441101in}{2.137159in}}%
\pgfpathlineto{\pgfqpoint{3.433028in}{2.127359in}}%
\pgfpathlineto{\pgfqpoint{3.424949in}{2.117601in}}%
\pgfpathlineto{\pgfqpoint{3.416864in}{2.107885in}}%
\pgfpathclose%
\pgfusepath{fill}%
\end{pgfscope}%
\begin{pgfscope}%
\pgfpathrectangle{\pgfqpoint{1.150000in}{0.150000in}}{\pgfqpoint{5.700000in}{5.700000in}}%
\pgfusepath{clip}%
\pgfsetbuttcap%
\pgfsetroundjoin%
\definecolor{currentfill}{rgb}{0.143343,0.522773,0.556295}%
\pgfsetfillcolor{currentfill}%
\pgfsetfillopacity{0.800000}%
\pgfsetlinewidth{0.000000pt}%
\definecolor{currentstroke}{rgb}{0.000000,0.000000,0.000000}%
\pgfsetstrokecolor{currentstroke}%
\pgfsetdash{}{0pt}%
\pgfpathmoveto{\pgfqpoint{5.565647in}{3.241419in}}%
\pgfpathlineto{\pgfqpoint{5.579899in}{3.248137in}}%
\pgfpathlineto{\pgfqpoint{5.594167in}{3.255027in}}%
\pgfpathlineto{\pgfqpoint{5.608452in}{3.262090in}}%
\pgfpathlineto{\pgfqpoint{5.622754in}{3.269324in}}%
\pgfpathlineto{\pgfqpoint{5.630018in}{3.275453in}}%
\pgfpathlineto{\pgfqpoint{5.637282in}{3.281753in}}%
\pgfpathlineto{\pgfqpoint{5.644545in}{3.288232in}}%
\pgfpathlineto{\pgfqpoint{5.651808in}{3.294897in}}%
\pgfpathlineto{\pgfqpoint{5.637536in}{3.288352in}}%
\pgfpathlineto{\pgfqpoint{5.623281in}{3.281977in}}%
\pgfpathlineto{\pgfqpoint{5.609042in}{3.275774in}}%
\pgfpathlineto{\pgfqpoint{5.594819in}{3.269742in}}%
\pgfpathlineto{\pgfqpoint{5.587526in}{3.262378in}}%
\pgfpathlineto{\pgfqpoint{5.580234in}{3.255208in}}%
\pgfpathlineto{\pgfqpoint{5.572941in}{3.248224in}}%
\pgfpathlineto{\pgfqpoint{5.565647in}{3.241419in}}%
\pgfpathclose%
\pgfusepath{fill}%
\end{pgfscope}%
\begin{pgfscope}%
\pgfpathrectangle{\pgfqpoint{1.150000in}{0.150000in}}{\pgfqpoint{5.700000in}{5.700000in}}%
\pgfusepath{clip}%
\pgfsetbuttcap%
\pgfsetroundjoin%
\definecolor{currentfill}{rgb}{0.227802,0.326594,0.546532}%
\pgfsetfillcolor{currentfill}%
\pgfsetfillopacity{0.800000}%
\pgfsetlinewidth{0.000000pt}%
\definecolor{currentstroke}{rgb}{0.000000,0.000000,0.000000}%
\pgfsetstrokecolor{currentstroke}%
\pgfsetdash{}{0pt}%
\pgfpathmoveto{\pgfqpoint{4.674809in}{2.675611in}}%
\pgfpathlineto{\pgfqpoint{4.688671in}{2.680654in}}%
\pgfpathlineto{\pgfqpoint{4.702546in}{2.685880in}}%
\pgfpathlineto{\pgfqpoint{4.716435in}{2.691289in}}%
\pgfpathlineto{\pgfqpoint{4.730336in}{2.696881in}}%
\pgfpathlineto{\pgfqpoint{4.737975in}{2.705008in}}%
\pgfpathlineto{\pgfqpoint{4.745607in}{2.713133in}}%
\pgfpathlineto{\pgfqpoint{4.753235in}{2.721260in}}%
\pgfpathlineto{\pgfqpoint{4.760857in}{2.729393in}}%
\pgfpathlineto{\pgfqpoint{4.746967in}{2.724131in}}%
\pgfpathlineto{\pgfqpoint{4.733091in}{2.719051in}}%
\pgfpathlineto{\pgfqpoint{4.719228in}{2.714154in}}%
\pgfpathlineto{\pgfqpoint{4.705377in}{2.709439in}}%
\pgfpathlineto{\pgfqpoint{4.697743in}{2.700965in}}%
\pgfpathlineto{\pgfqpoint{4.690104in}{2.692505in}}%
\pgfpathlineto{\pgfqpoint{4.682459in}{2.684055in}}%
\pgfpathlineto{\pgfqpoint{4.674809in}{2.675611in}}%
\pgfpathclose%
\pgfusepath{fill}%
\end{pgfscope}%
\begin{pgfscope}%
\pgfpathrectangle{\pgfqpoint{1.150000in}{0.150000in}}{\pgfqpoint{5.700000in}{5.700000in}}%
\pgfusepath{clip}%
\pgfsetbuttcap%
\pgfsetroundjoin%
\definecolor{currentfill}{rgb}{0.281924,0.089666,0.412415}%
\pgfsetfillcolor{currentfill}%
\pgfsetfillopacity{0.800000}%
\pgfsetlinewidth{0.000000pt}%
\definecolor{currentstroke}{rgb}{0.000000,0.000000,0.000000}%
\pgfsetstrokecolor{currentstroke}%
\pgfsetdash{}{0pt}%
\pgfpathmoveto{\pgfqpoint{3.643323in}{2.135575in}}%
\pgfpathlineto{\pgfqpoint{3.656846in}{2.132703in}}%
\pgfpathlineto{\pgfqpoint{3.670373in}{2.130041in}}%
\pgfpathlineto{\pgfqpoint{3.683906in}{2.127588in}}%
\pgfpathlineto{\pgfqpoint{3.697444in}{2.125343in}}%
\pgfpathlineto{\pgfqpoint{3.705439in}{2.135652in}}%
\pgfpathlineto{\pgfqpoint{3.713429in}{2.145967in}}%
\pgfpathlineto{\pgfqpoint{3.721413in}{2.156288in}}%
\pgfpathlineto{\pgfqpoint{3.729392in}{2.166616in}}%
\pgfpathlineto{\pgfqpoint{3.715864in}{2.168804in}}%
\pgfpathlineto{\pgfqpoint{3.702341in}{2.171201in}}%
\pgfpathlineto{\pgfqpoint{3.688823in}{2.173806in}}%
\pgfpathlineto{\pgfqpoint{3.675311in}{2.176621in}}%
\pgfpathlineto{\pgfqpoint{3.667322in}{2.166339in}}%
\pgfpathlineto{\pgfqpoint{3.659328in}{2.156070in}}%
\pgfpathlineto{\pgfqpoint{3.651328in}{2.145816in}}%
\pgfpathlineto{\pgfqpoint{3.643323in}{2.135575in}}%
\pgfpathclose%
\pgfusepath{fill}%
\end{pgfscope}%
\begin{pgfscope}%
\pgfpathrectangle{\pgfqpoint{1.150000in}{0.150000in}}{\pgfqpoint{5.700000in}{5.700000in}}%
\pgfusepath{clip}%
\pgfsetbuttcap%
\pgfsetroundjoin%
\definecolor{currentfill}{rgb}{0.270595,0.214069,0.507052}%
\pgfsetfillcolor{currentfill}%
\pgfsetfillopacity{0.800000}%
\pgfsetlinewidth{0.000000pt}%
\definecolor{currentstroke}{rgb}{0.000000,0.000000,0.000000}%
\pgfsetstrokecolor{currentstroke}%
\pgfsetdash{}{0pt}%
\pgfpathmoveto{\pgfqpoint{2.776626in}{2.442498in}}%
\pgfpathlineto{\pgfqpoint{2.790275in}{2.426136in}}%
\pgfpathlineto{\pgfqpoint{2.803917in}{2.410066in}}%
\pgfpathlineto{\pgfqpoint{2.817552in}{2.394286in}}%
\pgfpathlineto{\pgfqpoint{2.831180in}{2.378792in}}%
\pgfpathlineto{\pgfqpoint{2.839527in}{2.385840in}}%
\pgfpathlineto{\pgfqpoint{2.847864in}{2.393019in}}%
\pgfpathlineto{\pgfqpoint{2.856192in}{2.400326in}}%
\pgfpathlineto{\pgfqpoint{2.864510in}{2.407759in}}%
\pgfpathlineto{\pgfqpoint{2.850907in}{2.423029in}}%
\pgfpathlineto{\pgfqpoint{2.837297in}{2.438586in}}%
\pgfpathlineto{\pgfqpoint{2.823681in}{2.454432in}}%
\pgfpathlineto{\pgfqpoint{2.810058in}{2.470569in}}%
\pgfpathlineto{\pgfqpoint{2.801715in}{2.463348in}}%
\pgfpathlineto{\pgfqpoint{2.793361in}{2.456261in}}%
\pgfpathlineto{\pgfqpoint{2.784998in}{2.449310in}}%
\pgfpathlineto{\pgfqpoint{2.776626in}{2.442498in}}%
\pgfpathclose%
\pgfusepath{fill}%
\end{pgfscope}%
\begin{pgfscope}%
\pgfpathrectangle{\pgfqpoint{1.150000in}{0.150000in}}{\pgfqpoint{5.700000in}{5.700000in}}%
\pgfusepath{clip}%
\pgfsetbuttcap%
\pgfsetroundjoin%
\definecolor{currentfill}{rgb}{0.282327,0.094955,0.417331}%
\pgfsetfillcolor{currentfill}%
\pgfsetfillopacity{0.800000}%
\pgfsetlinewidth{0.000000pt}%
\definecolor{currentstroke}{rgb}{0.000000,0.000000,0.000000}%
\pgfsetstrokecolor{currentstroke}%
\pgfsetdash{}{0pt}%
\pgfpathmoveto{\pgfqpoint{3.135598in}{2.159165in}}%
\pgfpathlineto{\pgfqpoint{3.149122in}{2.149428in}}%
\pgfpathlineto{\pgfqpoint{3.162645in}{2.139934in}}%
\pgfpathlineto{\pgfqpoint{3.176167in}{2.130682in}}%
\pgfpathlineto{\pgfqpoint{3.189689in}{2.121671in}}%
\pgfpathlineto{\pgfqpoint{3.197873in}{2.130425in}}%
\pgfpathlineto{\pgfqpoint{3.206049in}{2.139254in}}%
\pgfpathlineto{\pgfqpoint{3.214219in}{2.148157in}}%
\pgfpathlineto{\pgfqpoint{3.222382in}{2.157133in}}%
\pgfpathlineto{\pgfqpoint{3.208878in}{2.165959in}}%
\pgfpathlineto{\pgfqpoint{3.195374in}{2.175025in}}%
\pgfpathlineto{\pgfqpoint{3.181869in}{2.184334in}}%
\pgfpathlineto{\pgfqpoint{3.168363in}{2.193886in}}%
\pgfpathlineto{\pgfqpoint{3.160182in}{2.185084in}}%
\pgfpathlineto{\pgfqpoint{3.151995in}{2.176363in}}%
\pgfpathlineto{\pgfqpoint{3.143800in}{2.167722in}}%
\pgfpathlineto{\pgfqpoint{3.135598in}{2.159165in}}%
\pgfpathclose%
\pgfusepath{fill}%
\end{pgfscope}%
\begin{pgfscope}%
\pgfpathrectangle{\pgfqpoint{1.150000in}{0.150000in}}{\pgfqpoint{5.700000in}{5.700000in}}%
\pgfusepath{clip}%
\pgfsetbuttcap%
\pgfsetroundjoin%
\definecolor{currentfill}{rgb}{0.277134,0.185228,0.489898}%
\pgfsetfillcolor{currentfill}%
\pgfsetfillopacity{0.800000}%
\pgfsetlinewidth{0.000000pt}%
\definecolor{currentstroke}{rgb}{0.000000,0.000000,0.000000}%
\pgfsetstrokecolor{currentstroke}%
\pgfsetdash{}{0pt}%
\pgfpathmoveto{\pgfqpoint{2.831180in}{2.378792in}}%
\pgfpathlineto{\pgfqpoint{2.844802in}{2.363582in}}%
\pgfpathlineto{\pgfqpoint{2.858417in}{2.348654in}}%
\pgfpathlineto{\pgfqpoint{2.872027in}{2.334007in}}%
\pgfpathlineto{\pgfqpoint{2.885630in}{2.319636in}}%
\pgfpathlineto{\pgfqpoint{2.893952in}{2.326920in}}%
\pgfpathlineto{\pgfqpoint{2.902264in}{2.334325in}}%
\pgfpathlineto{\pgfqpoint{2.910568in}{2.341850in}}%
\pgfpathlineto{\pgfqpoint{2.918862in}{2.349493in}}%
\pgfpathlineto{\pgfqpoint{2.905283in}{2.363642in}}%
\pgfpathlineto{\pgfqpoint{2.891697in}{2.378067in}}%
\pgfpathlineto{\pgfqpoint{2.878107in}{2.392772in}}%
\pgfpathlineto{\pgfqpoint{2.864510in}{2.407759in}}%
\pgfpathlineto{\pgfqpoint{2.856192in}{2.400326in}}%
\pgfpathlineto{\pgfqpoint{2.847864in}{2.393019in}}%
\pgfpathlineto{\pgfqpoint{2.839527in}{2.385840in}}%
\pgfpathlineto{\pgfqpoint{2.831180in}{2.378792in}}%
\pgfpathclose%
\pgfusepath{fill}%
\end{pgfscope}%
\begin{pgfscope}%
\pgfpathrectangle{\pgfqpoint{1.150000in}{0.150000in}}{\pgfqpoint{5.700000in}{5.700000in}}%
\pgfusepath{clip}%
\pgfsetbuttcap%
\pgfsetroundjoin%
\definecolor{currentfill}{rgb}{0.136408,0.541173,0.554483}%
\pgfsetfillcolor{currentfill}%
\pgfsetfillopacity{0.800000}%
\pgfsetlinewidth{0.000000pt}%
\definecolor{currentstroke}{rgb}{0.000000,0.000000,0.000000}%
\pgfsetstrokecolor{currentstroke}%
\pgfsetdash{}{0pt}%
\pgfpathmoveto{\pgfqpoint{5.651808in}{3.294897in}}%
\pgfpathlineto{\pgfqpoint{5.666097in}{3.301614in}}%
\pgfpathlineto{\pgfqpoint{5.680403in}{3.308502in}}%
\pgfpathlineto{\pgfqpoint{5.694725in}{3.315561in}}%
\pgfpathlineto{\pgfqpoint{5.709065in}{3.322791in}}%
\pgfpathlineto{\pgfqpoint{5.716298in}{3.328942in}}%
\pgfpathlineto{\pgfqpoint{5.723530in}{3.335287in}}%
\pgfpathlineto{\pgfqpoint{5.730764in}{3.341835in}}%
\pgfpathlineto{\pgfqpoint{5.737999in}{3.348593in}}%
\pgfpathlineto{\pgfqpoint{5.723691in}{3.342084in}}%
\pgfpathlineto{\pgfqpoint{5.709399in}{3.335746in}}%
\pgfpathlineto{\pgfqpoint{5.695125in}{3.329577in}}%
\pgfpathlineto{\pgfqpoint{5.680867in}{3.323580in}}%
\pgfpathlineto{\pgfqpoint{5.673601in}{3.316090in}}%
\pgfpathlineto{\pgfqpoint{5.666336in}{3.308818in}}%
\pgfpathlineto{\pgfqpoint{5.659071in}{3.301757in}}%
\pgfpathlineto{\pgfqpoint{5.651808in}{3.294897in}}%
\pgfpathclose%
\pgfusepath{fill}%
\end{pgfscope}%
\begin{pgfscope}%
\pgfpathrectangle{\pgfqpoint{1.150000in}{0.150000in}}{\pgfqpoint{5.700000in}{5.700000in}}%
\pgfusepath{clip}%
\pgfsetbuttcap%
\pgfsetroundjoin%
\definecolor{currentfill}{rgb}{0.262138,0.242286,0.520837}%
\pgfsetfillcolor{currentfill}%
\pgfsetfillopacity{0.800000}%
\pgfsetlinewidth{0.000000pt}%
\definecolor{currentstroke}{rgb}{0.000000,0.000000,0.000000}%
\pgfsetstrokecolor{currentstroke}%
\pgfsetdash{}{0pt}%
\pgfpathmoveto{\pgfqpoint{2.721949in}{2.510909in}}%
\pgfpathlineto{\pgfqpoint{2.735630in}{2.493356in}}%
\pgfpathlineto{\pgfqpoint{2.749304in}{2.476105in}}%
\pgfpathlineto{\pgfqpoint{2.762969in}{2.459153in}}%
\pgfpathlineto{\pgfqpoint{2.776626in}{2.442498in}}%
\pgfpathlineto{\pgfqpoint{2.784998in}{2.449310in}}%
\pgfpathlineto{\pgfqpoint{2.793361in}{2.456261in}}%
\pgfpathlineto{\pgfqpoint{2.801715in}{2.463348in}}%
\pgfpathlineto{\pgfqpoint{2.810058in}{2.470569in}}%
\pgfpathlineto{\pgfqpoint{2.796427in}{2.486999in}}%
\pgfpathlineto{\pgfqpoint{2.782789in}{2.503726in}}%
\pgfpathlineto{\pgfqpoint{2.769143in}{2.520751in}}%
\pgfpathlineto{\pgfqpoint{2.755488in}{2.538078in}}%
\pgfpathlineto{\pgfqpoint{2.747119in}{2.531070in}}%
\pgfpathlineto{\pgfqpoint{2.738739in}{2.524205in}}%
\pgfpathlineto{\pgfqpoint{2.730349in}{2.517484in}}%
\pgfpathlineto{\pgfqpoint{2.721949in}{2.510909in}}%
\pgfpathclose%
\pgfusepath{fill}%
\end{pgfscope}%
\begin{pgfscope}%
\pgfpathrectangle{\pgfqpoint{1.150000in}{0.150000in}}{\pgfqpoint{5.700000in}{5.700000in}}%
\pgfusepath{clip}%
\pgfsetbuttcap%
\pgfsetroundjoin%
\definecolor{currentfill}{rgb}{0.218130,0.347432,0.550038}%
\pgfsetfillcolor{currentfill}%
\pgfsetfillopacity{0.800000}%
\pgfsetlinewidth{0.000000pt}%
\definecolor{currentstroke}{rgb}{0.000000,0.000000,0.000000}%
\pgfsetstrokecolor{currentstroke}%
\pgfsetdash{}{0pt}%
\pgfpathmoveto{\pgfqpoint{4.760857in}{2.729393in}}%
\pgfpathlineto{\pgfqpoint{4.774760in}{2.734837in}}%
\pgfpathlineto{\pgfqpoint{4.788676in}{2.740463in}}%
\pgfpathlineto{\pgfqpoint{4.802606in}{2.746271in}}%
\pgfpathlineto{\pgfqpoint{4.816550in}{2.752261in}}%
\pgfpathlineto{\pgfqpoint{4.824154in}{2.760054in}}%
\pgfpathlineto{\pgfqpoint{4.831753in}{2.767853in}}%
\pgfpathlineto{\pgfqpoint{4.839346in}{2.775662in}}%
\pgfpathlineto{\pgfqpoint{4.846934in}{2.783487in}}%
\pgfpathlineto{\pgfqpoint{4.833004in}{2.777860in}}%
\pgfpathlineto{\pgfqpoint{4.819087in}{2.772414in}}%
\pgfpathlineto{\pgfqpoint{4.805183in}{2.767150in}}%
\pgfpathlineto{\pgfqpoint{4.791293in}{2.762067in}}%
\pgfpathlineto{\pgfqpoint{4.783692in}{2.753869in}}%
\pgfpathlineto{\pgfqpoint{4.776085in}{2.745693in}}%
\pgfpathlineto{\pgfqpoint{4.768474in}{2.737536in}}%
\pgfpathlineto{\pgfqpoint{4.760857in}{2.729393in}}%
\pgfpathclose%
\pgfusepath{fill}%
\end{pgfscope}%
\begin{pgfscope}%
\pgfpathrectangle{\pgfqpoint{1.150000in}{0.150000in}}{\pgfqpoint{5.700000in}{5.700000in}}%
\pgfusepath{clip}%
\pgfsetbuttcap%
\pgfsetroundjoin%
\definecolor{currentfill}{rgb}{0.280868,0.160771,0.472899}%
\pgfsetfillcolor{currentfill}%
\pgfsetfillopacity{0.800000}%
\pgfsetlinewidth{0.000000pt}%
\definecolor{currentstroke}{rgb}{0.000000,0.000000,0.000000}%
\pgfsetstrokecolor{currentstroke}%
\pgfsetdash{}{0pt}%
\pgfpathmoveto{\pgfqpoint{2.885630in}{2.319636in}}%
\pgfpathlineto{\pgfqpoint{2.899228in}{2.305542in}}%
\pgfpathlineto{\pgfqpoint{2.912821in}{2.291720in}}%
\pgfpathlineto{\pgfqpoint{2.926409in}{2.278170in}}%
\pgfpathlineto{\pgfqpoint{2.939993in}{2.264889in}}%
\pgfpathlineto{\pgfqpoint{2.948290in}{2.272405in}}%
\pgfpathlineto{\pgfqpoint{2.956579in}{2.280035in}}%
\pgfpathlineto{\pgfqpoint{2.964860in}{2.287777in}}%
\pgfpathlineto{\pgfqpoint{2.973131in}{2.295629in}}%
\pgfpathlineto{\pgfqpoint{2.959571in}{2.308690in}}%
\pgfpathlineto{\pgfqpoint{2.946006in}{2.322020in}}%
\pgfpathlineto{\pgfqpoint{2.932437in}{2.335620in}}%
\pgfpathlineto{\pgfqpoint{2.918862in}{2.349493in}}%
\pgfpathlineto{\pgfqpoint{2.910568in}{2.341850in}}%
\pgfpathlineto{\pgfqpoint{2.902264in}{2.334325in}}%
\pgfpathlineto{\pgfqpoint{2.893952in}{2.326920in}}%
\pgfpathlineto{\pgfqpoint{2.885630in}{2.319636in}}%
\pgfpathclose%
\pgfusepath{fill}%
\end{pgfscope}%
\begin{pgfscope}%
\pgfpathrectangle{\pgfqpoint{1.150000in}{0.150000in}}{\pgfqpoint{5.700000in}{5.700000in}}%
\pgfusepath{clip}%
\pgfsetbuttcap%
\pgfsetroundjoin%
\definecolor{currentfill}{rgb}{0.129933,0.559582,0.551864}%
\pgfsetfillcolor{currentfill}%
\pgfsetfillopacity{0.800000}%
\pgfsetlinewidth{0.000000pt}%
\definecolor{currentstroke}{rgb}{0.000000,0.000000,0.000000}%
\pgfsetstrokecolor{currentstroke}%
\pgfsetdash{}{0pt}%
\pgfpathmoveto{\pgfqpoint{5.737999in}{3.348593in}}%
\pgfpathlineto{\pgfqpoint{5.752324in}{3.355272in}}%
\pgfpathlineto{\pgfqpoint{5.766666in}{3.362122in}}%
\pgfpathlineto{\pgfqpoint{5.781025in}{3.369142in}}%
\pgfpathlineto{\pgfqpoint{5.795402in}{3.376333in}}%
\pgfpathlineto{\pgfqpoint{5.802605in}{3.382568in}}%
\pgfpathlineto{\pgfqpoint{5.809809in}{3.389024in}}%
\pgfpathlineto{\pgfqpoint{5.817016in}{3.395707in}}%
\pgfpathlineto{\pgfqpoint{5.824224in}{3.402627in}}%
\pgfpathlineto{\pgfqpoint{5.809881in}{3.396190in}}%
\pgfpathlineto{\pgfqpoint{5.795556in}{3.389923in}}%
\pgfpathlineto{\pgfqpoint{5.781247in}{3.383825in}}%
\pgfpathlineto{\pgfqpoint{5.766955in}{3.377897in}}%
\pgfpathlineto{\pgfqpoint{5.759712in}{3.370214in}}%
\pgfpathlineto{\pgfqpoint{5.752472in}{3.362775in}}%
\pgfpathlineto{\pgfqpoint{5.745235in}{3.355571in}}%
\pgfpathlineto{\pgfqpoint{5.737999in}{3.348593in}}%
\pgfpathclose%
\pgfusepath{fill}%
\end{pgfscope}%
\begin{pgfscope}%
\pgfpathrectangle{\pgfqpoint{1.150000in}{0.150000in}}{\pgfqpoint{5.700000in}{5.700000in}}%
\pgfusepath{clip}%
\pgfsetbuttcap%
\pgfsetroundjoin%
\definecolor{currentfill}{rgb}{0.252194,0.269783,0.531579}%
\pgfsetfillcolor{currentfill}%
\pgfsetfillopacity{0.800000}%
\pgfsetlinewidth{0.000000pt}%
\definecolor{currentstroke}{rgb}{0.000000,0.000000,0.000000}%
\pgfsetstrokecolor{currentstroke}%
\pgfsetdash{}{0pt}%
\pgfpathmoveto{\pgfqpoint{2.667131in}{2.584193in}}%
\pgfpathlineto{\pgfqpoint{2.680850in}{2.565406in}}%
\pgfpathlineto{\pgfqpoint{2.694559in}{2.546931in}}%
\pgfpathlineto{\pgfqpoint{2.708258in}{2.528767in}}%
\pgfpathlineto{\pgfqpoint{2.721949in}{2.510909in}}%
\pgfpathlineto{\pgfqpoint{2.730349in}{2.517484in}}%
\pgfpathlineto{\pgfqpoint{2.738739in}{2.524205in}}%
\pgfpathlineto{\pgfqpoint{2.747119in}{2.531070in}}%
\pgfpathlineto{\pgfqpoint{2.755488in}{2.538078in}}%
\pgfpathlineto{\pgfqpoint{2.741826in}{2.555709in}}%
\pgfpathlineto{\pgfqpoint{2.728154in}{2.573646in}}%
\pgfpathlineto{\pgfqpoint{2.714473in}{2.591893in}}%
\pgfpathlineto{\pgfqpoint{2.700783in}{2.610452in}}%
\pgfpathlineto{\pgfqpoint{2.692386in}{2.603659in}}%
\pgfpathlineto{\pgfqpoint{2.683979in}{2.597017in}}%
\pgfpathlineto{\pgfqpoint{2.675560in}{2.590528in}}%
\pgfpathlineto{\pgfqpoint{2.667131in}{2.584193in}}%
\pgfpathclose%
\pgfusepath{fill}%
\end{pgfscope}%
\begin{pgfscope}%
\pgfpathrectangle{\pgfqpoint{1.150000in}{0.150000in}}{\pgfqpoint{5.700000in}{5.700000in}}%
\pgfusepath{clip}%
\pgfsetbuttcap%
\pgfsetroundjoin%
\definecolor{currentfill}{rgb}{0.208623,0.367752,0.552675}%
\pgfsetfillcolor{currentfill}%
\pgfsetfillopacity{0.800000}%
\pgfsetlinewidth{0.000000pt}%
\definecolor{currentstroke}{rgb}{0.000000,0.000000,0.000000}%
\pgfsetstrokecolor{currentstroke}%
\pgfsetdash{}{0pt}%
\pgfpathmoveto{\pgfqpoint{4.846934in}{2.783487in}}%
\pgfpathlineto{\pgfqpoint{4.860879in}{2.789295in}}%
\pgfpathlineto{\pgfqpoint{4.874837in}{2.795283in}}%
\pgfpathlineto{\pgfqpoint{4.888810in}{2.801452in}}%
\pgfpathlineto{\pgfqpoint{4.902797in}{2.807802in}}%
\pgfpathlineto{\pgfqpoint{4.910365in}{2.815262in}}%
\pgfpathlineto{\pgfqpoint{4.917929in}{2.822738in}}%
\pgfpathlineto{\pgfqpoint{4.925487in}{2.830235in}}%
\pgfpathlineto{\pgfqpoint{4.933040in}{2.837756in}}%
\pgfpathlineto{\pgfqpoint{4.919068in}{2.831803in}}%
\pgfpathlineto{\pgfqpoint{4.905110in}{2.826029in}}%
\pgfpathlineto{\pgfqpoint{4.891166in}{2.820435in}}%
\pgfpathlineto{\pgfqpoint{4.877236in}{2.815021in}}%
\pgfpathlineto{\pgfqpoint{4.869668in}{2.807093in}}%
\pgfpathlineto{\pgfqpoint{4.862095in}{2.799198in}}%
\pgfpathlineto{\pgfqpoint{4.854517in}{2.791331in}}%
\pgfpathlineto{\pgfqpoint{4.846934in}{2.783487in}}%
\pgfpathclose%
\pgfusepath{fill}%
\end{pgfscope}%
\begin{pgfscope}%
\pgfpathrectangle{\pgfqpoint{1.150000in}{0.150000in}}{\pgfqpoint{5.700000in}{5.700000in}}%
\pgfusepath{clip}%
\pgfsetbuttcap%
\pgfsetroundjoin%
\definecolor{currentfill}{rgb}{0.280894,0.078907,0.402329}%
\pgfsetfillcolor{currentfill}%
\pgfsetfillopacity{0.800000}%
\pgfsetlinewidth{0.000000pt}%
\definecolor{currentstroke}{rgb}{0.000000,0.000000,0.000000}%
\pgfsetstrokecolor{currentstroke}%
\pgfsetdash{}{0pt}%
\pgfpathmoveto{\pgfqpoint{3.557164in}{2.108720in}}%
\pgfpathlineto{\pgfqpoint{3.570679in}{2.104911in}}%
\pgfpathlineto{\pgfqpoint{3.584198in}{2.101316in}}%
\pgfpathlineto{\pgfqpoint{3.597722in}{2.097933in}}%
\pgfpathlineto{\pgfqpoint{3.611250in}{2.094763in}}%
\pgfpathlineto{\pgfqpoint{3.619277in}{2.104943in}}%
\pgfpathlineto{\pgfqpoint{3.627298in}{2.115138in}}%
\pgfpathlineto{\pgfqpoint{3.635313in}{2.125349in}}%
\pgfpathlineto{\pgfqpoint{3.643323in}{2.135575in}}%
\pgfpathlineto{\pgfqpoint{3.629806in}{2.138658in}}%
\pgfpathlineto{\pgfqpoint{3.616293in}{2.141952in}}%
\pgfpathlineto{\pgfqpoint{3.602784in}{2.145459in}}%
\pgfpathlineto{\pgfqpoint{3.589280in}{2.149180in}}%
\pgfpathlineto{\pgfqpoint{3.581259in}{2.139030in}}%
\pgfpathlineto{\pgfqpoint{3.573233in}{2.128904in}}%
\pgfpathlineto{\pgfqpoint{3.565201in}{2.118800in}}%
\pgfpathlineto{\pgfqpoint{3.557164in}{2.108720in}}%
\pgfpathclose%
\pgfusepath{fill}%
\end{pgfscope}%
\begin{pgfscope}%
\pgfpathrectangle{\pgfqpoint{1.150000in}{0.150000in}}{\pgfqpoint{5.700000in}{5.700000in}}%
\pgfusepath{clip}%
\pgfsetbuttcap%
\pgfsetroundjoin%
\definecolor{currentfill}{rgb}{0.282623,0.140926,0.457517}%
\pgfsetfillcolor{currentfill}%
\pgfsetfillopacity{0.800000}%
\pgfsetlinewidth{0.000000pt}%
\definecolor{currentstroke}{rgb}{0.000000,0.000000,0.000000}%
\pgfsetstrokecolor{currentstroke}%
\pgfsetdash{}{0pt}%
\pgfpathmoveto{\pgfqpoint{2.939993in}{2.264889in}}%
\pgfpathlineto{\pgfqpoint{2.953571in}{2.251874in}}%
\pgfpathlineto{\pgfqpoint{2.967146in}{2.239125in}}%
\pgfpathlineto{\pgfqpoint{2.980717in}{2.226639in}}%
\pgfpathlineto{\pgfqpoint{2.994283in}{2.214414in}}%
\pgfpathlineto{\pgfqpoint{3.002558in}{2.222162in}}%
\pgfpathlineto{\pgfqpoint{3.010825in}{2.230016in}}%
\pgfpathlineto{\pgfqpoint{3.019083in}{2.237974in}}%
\pgfpathlineto{\pgfqpoint{3.027333in}{2.246034in}}%
\pgfpathlineto{\pgfqpoint{3.013788in}{2.258040in}}%
\pgfpathlineto{\pgfqpoint{3.000240in}{2.270306in}}%
\pgfpathlineto{\pgfqpoint{2.986688in}{2.282835in}}%
\pgfpathlineto{\pgfqpoint{2.973131in}{2.295629in}}%
\pgfpathlineto{\pgfqpoint{2.964860in}{2.287777in}}%
\pgfpathlineto{\pgfqpoint{2.956579in}{2.280035in}}%
\pgfpathlineto{\pgfqpoint{2.948290in}{2.272405in}}%
\pgfpathlineto{\pgfqpoint{2.939993in}{2.264889in}}%
\pgfpathclose%
\pgfusepath{fill}%
\end{pgfscope}%
\begin{pgfscope}%
\pgfpathrectangle{\pgfqpoint{1.150000in}{0.150000in}}{\pgfqpoint{5.700000in}{5.700000in}}%
\pgfusepath{clip}%
\pgfsetbuttcap%
\pgfsetroundjoin%
\definecolor{currentfill}{rgb}{0.239346,0.300855,0.540844}%
\pgfsetfillcolor{currentfill}%
\pgfsetfillopacity{0.800000}%
\pgfsetlinewidth{0.000000pt}%
\definecolor{currentstroke}{rgb}{0.000000,0.000000,0.000000}%
\pgfsetstrokecolor{currentstroke}%
\pgfsetdash{}{0pt}%
\pgfpathmoveto{\pgfqpoint{2.612153in}{2.662529in}}%
\pgfpathlineto{\pgfqpoint{2.625914in}{2.642461in}}%
\pgfpathlineto{\pgfqpoint{2.639664in}{2.622718in}}%
\pgfpathlineto{\pgfqpoint{2.653403in}{2.603296in}}%
\pgfpathlineto{\pgfqpoint{2.667131in}{2.584193in}}%
\pgfpathlineto{\pgfqpoint{2.675560in}{2.590528in}}%
\pgfpathlineto{\pgfqpoint{2.683979in}{2.597017in}}%
\pgfpathlineto{\pgfqpoint{2.692386in}{2.603659in}}%
\pgfpathlineto{\pgfqpoint{2.700783in}{2.610452in}}%
\pgfpathlineto{\pgfqpoint{2.687083in}{2.629326in}}%
\pgfpathlineto{\pgfqpoint{2.673373in}{2.648518in}}%
\pgfpathlineto{\pgfqpoint{2.659653in}{2.668032in}}%
\pgfpathlineto{\pgfqpoint{2.645923in}{2.687869in}}%
\pgfpathlineto{\pgfqpoint{2.637497in}{2.681293in}}%
\pgfpathlineto{\pgfqpoint{2.629060in}{2.674877in}}%
\pgfpathlineto{\pgfqpoint{2.620612in}{2.668621in}}%
\pgfpathlineto{\pgfqpoint{2.612153in}{2.662529in}}%
\pgfpathclose%
\pgfusepath{fill}%
\end{pgfscope}%
\begin{pgfscope}%
\pgfpathrectangle{\pgfqpoint{1.150000in}{0.150000in}}{\pgfqpoint{5.700000in}{5.700000in}}%
\pgfusepath{clip}%
\pgfsetbuttcap%
\pgfsetroundjoin%
\definecolor{currentfill}{rgb}{0.199430,0.387607,0.554642}%
\pgfsetfillcolor{currentfill}%
\pgfsetfillopacity{0.800000}%
\pgfsetlinewidth{0.000000pt}%
\definecolor{currentstroke}{rgb}{0.000000,0.000000,0.000000}%
\pgfsetstrokecolor{currentstroke}%
\pgfsetdash{}{0pt}%
\pgfpathmoveto{\pgfqpoint{4.933040in}{2.837756in}}%
\pgfpathlineto{\pgfqpoint{4.947027in}{2.843890in}}%
\pgfpathlineto{\pgfqpoint{4.961028in}{2.850203in}}%
\pgfpathlineto{\pgfqpoint{4.975043in}{2.856696in}}%
\pgfpathlineto{\pgfqpoint{4.989074in}{2.863368in}}%
\pgfpathlineto{\pgfqpoint{4.996606in}{2.870503in}}%
\pgfpathlineto{\pgfqpoint{5.004134in}{2.877664in}}%
\pgfpathlineto{\pgfqpoint{5.011656in}{2.884857in}}%
\pgfpathlineto{\pgfqpoint{5.019174in}{2.892087in}}%
\pgfpathlineto{\pgfqpoint{5.005159in}{2.885844in}}%
\pgfpathlineto{\pgfqpoint{4.991160in}{2.879779in}}%
\pgfpathlineto{\pgfqpoint{4.977175in}{2.873893in}}%
\pgfpathlineto{\pgfqpoint{4.963204in}{2.868187in}}%
\pgfpathlineto{\pgfqpoint{4.955670in}{2.860518in}}%
\pgfpathlineto{\pgfqpoint{4.948132in}{2.852893in}}%
\pgfpathlineto{\pgfqpoint{4.940588in}{2.845308in}}%
\pgfpathlineto{\pgfqpoint{4.933040in}{2.837756in}}%
\pgfpathclose%
\pgfusepath{fill}%
\end{pgfscope}%
\begin{pgfscope}%
\pgfpathrectangle{\pgfqpoint{1.150000in}{0.150000in}}{\pgfqpoint{5.700000in}{5.700000in}}%
\pgfusepath{clip}%
\pgfsetbuttcap%
\pgfsetroundjoin%
\definecolor{currentfill}{rgb}{0.279566,0.067836,0.391917}%
\pgfsetfillcolor{currentfill}%
\pgfsetfillopacity{0.800000}%
\pgfsetlinewidth{0.000000pt}%
\definecolor{currentstroke}{rgb}{0.000000,0.000000,0.000000}%
\pgfsetstrokecolor{currentstroke}%
\pgfsetdash{}{0pt}%
\pgfpathmoveto{\pgfqpoint{3.330419in}{2.095006in}}%
\pgfpathlineto{\pgfqpoint{3.343927in}{2.088280in}}%
\pgfpathlineto{\pgfqpoint{3.357437in}{2.081781in}}%
\pgfpathlineto{\pgfqpoint{3.370949in}{2.075508in}}%
\pgfpathlineto{\pgfqpoint{3.384462in}{2.069459in}}%
\pgfpathlineto{\pgfqpoint{3.392572in}{2.078998in}}%
\pgfpathlineto{\pgfqpoint{3.400675in}{2.088582in}}%
\pgfpathlineto{\pgfqpoint{3.408772in}{2.098211in}}%
\pgfpathlineto{\pgfqpoint{3.416864in}{2.107885in}}%
\pgfpathlineto{\pgfqpoint{3.403364in}{2.113782in}}%
\pgfpathlineto{\pgfqpoint{3.389867in}{2.119904in}}%
\pgfpathlineto{\pgfqpoint{3.376371in}{2.126251in}}%
\pgfpathlineto{\pgfqpoint{3.362877in}{2.132825in}}%
\pgfpathlineto{\pgfqpoint{3.354772in}{2.123291in}}%
\pgfpathlineto{\pgfqpoint{3.346661in}{2.113810in}}%
\pgfpathlineto{\pgfqpoint{3.338543in}{2.104381in}}%
\pgfpathlineto{\pgfqpoint{3.330419in}{2.095006in}}%
\pgfpathclose%
\pgfusepath{fill}%
\end{pgfscope}%
\begin{pgfscope}%
\pgfpathrectangle{\pgfqpoint{1.150000in}{0.150000in}}{\pgfqpoint{5.700000in}{5.700000in}}%
\pgfusepath{clip}%
\pgfsetbuttcap%
\pgfsetroundjoin%
\definecolor{currentfill}{rgb}{0.124395,0.578002,0.548287}%
\pgfsetfillcolor{currentfill}%
\pgfsetfillopacity{0.800000}%
\pgfsetlinewidth{0.000000pt}%
\definecolor{currentstroke}{rgb}{0.000000,0.000000,0.000000}%
\pgfsetstrokecolor{currentstroke}%
\pgfsetdash{}{0pt}%
\pgfpathmoveto{\pgfqpoint{5.824224in}{3.402627in}}%
\pgfpathlineto{\pgfqpoint{5.838584in}{3.409233in}}%
\pgfpathlineto{\pgfqpoint{5.852962in}{3.416008in}}%
\pgfpathlineto{\pgfqpoint{5.867357in}{3.422954in}}%
\pgfpathlineto{\pgfqpoint{5.881769in}{3.430068in}}%
\pgfpathlineto{\pgfqpoint{5.888945in}{3.436459in}}%
\pgfpathlineto{\pgfqpoint{5.896124in}{3.443095in}}%
\pgfpathlineto{\pgfqpoint{5.903306in}{3.449986in}}%
\pgfpathlineto{\pgfqpoint{5.888920in}{3.443458in}}%
\pgfpathlineto{\pgfqpoint{5.874552in}{3.437099in}}%
\pgfpathlineto{\pgfqpoint{5.860201in}{3.430909in}}%
\pgfpathlineto{\pgfqpoint{5.845868in}{3.424888in}}%
\pgfpathlineto{\pgfqpoint{5.838650in}{3.417209in}}%
\pgfpathlineto{\pgfqpoint{5.831436in}{3.409791in}}%
\pgfpathlineto{\pgfqpoint{5.824224in}{3.402627in}}%
\pgfpathclose%
\pgfusepath{fill}%
\end{pgfscope}%
\begin{pgfscope}%
\pgfpathrectangle{\pgfqpoint{1.150000in}{0.150000in}}{\pgfqpoint{5.700000in}{5.700000in}}%
\pgfusepath{clip}%
\pgfsetbuttcap%
\pgfsetroundjoin%
\definecolor{currentfill}{rgb}{0.280894,0.078907,0.402329}%
\pgfsetfillcolor{currentfill}%
\pgfsetfillopacity{0.800000}%
\pgfsetlinewidth{0.000000pt}%
\definecolor{currentstroke}{rgb}{0.000000,0.000000,0.000000}%
\pgfsetstrokecolor{currentstroke}%
\pgfsetdash{}{0pt}%
\pgfpathmoveto{\pgfqpoint{3.189689in}{2.121671in}}%
\pgfpathlineto{\pgfqpoint{3.203209in}{2.112899in}}%
\pgfpathlineto{\pgfqpoint{3.216730in}{2.104365in}}%
\pgfpathlineto{\pgfqpoint{3.230250in}{2.096067in}}%
\pgfpathlineto{\pgfqpoint{3.243771in}{2.088004in}}%
\pgfpathlineto{\pgfqpoint{3.251937in}{2.096954in}}%
\pgfpathlineto{\pgfqpoint{3.260097in}{2.105972in}}%
\pgfpathlineto{\pgfqpoint{3.268250in}{2.115055in}}%
\pgfpathlineto{\pgfqpoint{3.276396in}{2.124203in}}%
\pgfpathlineto{\pgfqpoint{3.262893in}{2.132082in}}%
\pgfpathlineto{\pgfqpoint{3.249389in}{2.140195in}}%
\pgfpathlineto{\pgfqpoint{3.235886in}{2.148545in}}%
\pgfpathlineto{\pgfqpoint{3.222382in}{2.157133in}}%
\pgfpathlineto{\pgfqpoint{3.214219in}{2.148157in}}%
\pgfpathlineto{\pgfqpoint{3.206049in}{2.139254in}}%
\pgfpathlineto{\pgfqpoint{3.197873in}{2.130425in}}%
\pgfpathlineto{\pgfqpoint{3.189689in}{2.121671in}}%
\pgfpathclose%
\pgfusepath{fill}%
\end{pgfscope}%
\begin{pgfscope}%
\pgfpathrectangle{\pgfqpoint{1.150000in}{0.150000in}}{\pgfqpoint{5.700000in}{5.700000in}}%
\pgfusepath{clip}%
\pgfsetbuttcap%
\pgfsetroundjoin%
\definecolor{currentfill}{rgb}{0.190631,0.407061,0.556089}%
\pgfsetfillcolor{currentfill}%
\pgfsetfillopacity{0.800000}%
\pgfsetlinewidth{0.000000pt}%
\definecolor{currentstroke}{rgb}{0.000000,0.000000,0.000000}%
\pgfsetstrokecolor{currentstroke}%
\pgfsetdash{}{0pt}%
\pgfpathmoveto{\pgfqpoint{5.019174in}{2.892087in}}%
\pgfpathlineto{\pgfqpoint{5.033203in}{2.898509in}}%
\pgfpathlineto{\pgfqpoint{5.047246in}{2.905109in}}%
\pgfpathlineto{\pgfqpoint{5.061305in}{2.911888in}}%
\pgfpathlineto{\pgfqpoint{5.075379in}{2.918846in}}%
\pgfpathlineto{\pgfqpoint{5.082875in}{2.925667in}}%
\pgfpathlineto{\pgfqpoint{5.090366in}{2.932528in}}%
\pgfpathlineto{\pgfqpoint{5.097852in}{2.939432in}}%
\pgfpathlineto{\pgfqpoint{5.105333in}{2.946387in}}%
\pgfpathlineto{\pgfqpoint{5.091276in}{2.939891in}}%
\pgfpathlineto{\pgfqpoint{5.077235in}{2.933574in}}%
\pgfpathlineto{\pgfqpoint{5.063209in}{2.927434in}}%
\pgfpathlineto{\pgfqpoint{5.049197in}{2.921472in}}%
\pgfpathlineto{\pgfqpoint{5.041698in}{2.914045in}}%
\pgfpathlineto{\pgfqpoint{5.034194in}{2.906676in}}%
\pgfpathlineto{\pgfqpoint{5.026686in}{2.899358in}}%
\pgfpathlineto{\pgfqpoint{5.019174in}{2.892087in}}%
\pgfpathclose%
\pgfusepath{fill}%
\end{pgfscope}%
\begin{pgfscope}%
\pgfpathrectangle{\pgfqpoint{1.150000in}{0.150000in}}{\pgfqpoint{5.700000in}{5.700000in}}%
\pgfusepath{clip}%
\pgfsetbuttcap%
\pgfsetroundjoin%
\definecolor{currentfill}{rgb}{0.280255,0.165693,0.476498}%
\pgfsetfillcolor{currentfill}%
\pgfsetfillopacity{0.800000}%
\pgfsetlinewidth{0.000000pt}%
\definecolor{currentstroke}{rgb}{0.000000,0.000000,0.000000}%
\pgfsetstrokecolor{currentstroke}%
\pgfsetdash{}{0pt}%
\pgfpathmoveto{\pgfqpoint{4.041685in}{2.283452in}}%
\pgfpathlineto{\pgfqpoint{4.055310in}{2.284666in}}%
\pgfpathlineto{\pgfqpoint{4.068944in}{2.286077in}}%
\pgfpathlineto{\pgfqpoint{4.082587in}{2.287683in}}%
\pgfpathlineto{\pgfqpoint{4.096240in}{2.289483in}}%
\pgfpathlineto{\pgfqpoint{4.104112in}{2.299631in}}%
\pgfpathlineto{\pgfqpoint{4.111979in}{2.309753in}}%
\pgfpathlineto{\pgfqpoint{4.119841in}{2.319852in}}%
\pgfpathlineto{\pgfqpoint{4.127698in}{2.329927in}}%
\pgfpathlineto{\pgfqpoint{4.114053in}{2.328198in}}%
\pgfpathlineto{\pgfqpoint{4.100417in}{2.326663in}}%
\pgfpathlineto{\pgfqpoint{4.086790in}{2.325323in}}%
\pgfpathlineto{\pgfqpoint{4.073172in}{2.324179in}}%
\pgfpathlineto{\pgfqpoint{4.065308in}{2.314021in}}%
\pgfpathlineto{\pgfqpoint{4.057439in}{2.303848in}}%
\pgfpathlineto{\pgfqpoint{4.049564in}{2.293659in}}%
\pgfpathlineto{\pgfqpoint{4.041685in}{2.283452in}}%
\pgfpathclose%
\pgfusepath{fill}%
\end{pgfscope}%
\begin{pgfscope}%
\pgfpathrectangle{\pgfqpoint{1.150000in}{0.150000in}}{\pgfqpoint{5.700000in}{5.700000in}}%
\pgfusepath{clip}%
\pgfsetbuttcap%
\pgfsetroundjoin%
\definecolor{currentfill}{rgb}{0.283229,0.120777,0.440584}%
\pgfsetfillcolor{currentfill}%
\pgfsetfillopacity{0.800000}%
\pgfsetlinewidth{0.000000pt}%
\definecolor{currentstroke}{rgb}{0.000000,0.000000,0.000000}%
\pgfsetstrokecolor{currentstroke}%
\pgfsetdash{}{0pt}%
\pgfpathmoveto{\pgfqpoint{2.994283in}{2.214414in}}%
\pgfpathlineto{\pgfqpoint{3.007847in}{2.202449in}}%
\pgfpathlineto{\pgfqpoint{3.021407in}{2.190741in}}%
\pgfpathlineto{\pgfqpoint{3.034964in}{2.179288in}}%
\pgfpathlineto{\pgfqpoint{3.048518in}{2.168090in}}%
\pgfpathlineto{\pgfqpoint{3.056772in}{2.176068in}}%
\pgfpathlineto{\pgfqpoint{3.065017in}{2.184145in}}%
\pgfpathlineto{\pgfqpoint{3.073254in}{2.192317in}}%
\pgfpathlineto{\pgfqpoint{3.081484in}{2.200584in}}%
\pgfpathlineto{\pgfqpoint{3.067950in}{2.211564in}}%
\pgfpathlineto{\pgfqpoint{3.054414in}{2.222798in}}%
\pgfpathlineto{\pgfqpoint{3.040875in}{2.234288in}}%
\pgfpathlineto{\pgfqpoint{3.027333in}{2.246034in}}%
\pgfpathlineto{\pgfqpoint{3.019083in}{2.237974in}}%
\pgfpathlineto{\pgfqpoint{3.010825in}{2.230016in}}%
\pgfpathlineto{\pgfqpoint{3.002558in}{2.222162in}}%
\pgfpathlineto{\pgfqpoint{2.994283in}{2.214414in}}%
\pgfpathclose%
\pgfusepath{fill}%
\end{pgfscope}%
\begin{pgfscope}%
\pgfpathrectangle{\pgfqpoint{1.150000in}{0.150000in}}{\pgfqpoint{5.700000in}{5.700000in}}%
\pgfusepath{clip}%
\pgfsetbuttcap%
\pgfsetroundjoin%
\definecolor{currentfill}{rgb}{0.282290,0.145912,0.461510}%
\pgfsetfillcolor{currentfill}%
\pgfsetfillopacity{0.800000}%
\pgfsetlinewidth{0.000000pt}%
\definecolor{currentstroke}{rgb}{0.000000,0.000000,0.000000}%
\pgfsetstrokecolor{currentstroke}%
\pgfsetdash{}{0pt}%
\pgfpathmoveto{\pgfqpoint{3.955667in}{2.239380in}}%
\pgfpathlineto{\pgfqpoint{3.969267in}{2.239846in}}%
\pgfpathlineto{\pgfqpoint{3.982875in}{2.240509in}}%
\pgfpathlineto{\pgfqpoint{3.996491in}{2.241370in}}%
\pgfpathlineto{\pgfqpoint{4.010115in}{2.242428in}}%
\pgfpathlineto{\pgfqpoint{4.018015in}{2.252715in}}%
\pgfpathlineto{\pgfqpoint{4.025910in}{2.262981in}}%
\pgfpathlineto{\pgfqpoint{4.033800in}{2.273226in}}%
\pgfpathlineto{\pgfqpoint{4.041685in}{2.283452in}}%
\pgfpathlineto{\pgfqpoint{4.028068in}{2.282433in}}%
\pgfpathlineto{\pgfqpoint{4.014459in}{2.281611in}}%
\pgfpathlineto{\pgfqpoint{4.000859in}{2.280986in}}%
\pgfpathlineto{\pgfqpoint{3.987267in}{2.280560in}}%
\pgfpathlineto{\pgfqpoint{3.979375in}{2.270283in}}%
\pgfpathlineto{\pgfqpoint{3.971477in}{2.259995in}}%
\pgfpathlineto{\pgfqpoint{3.963575in}{2.249694in}}%
\pgfpathlineto{\pgfqpoint{3.955667in}{2.239380in}}%
\pgfpathclose%
\pgfusepath{fill}%
\end{pgfscope}%
\begin{pgfscope}%
\pgfpathrectangle{\pgfqpoint{1.150000in}{0.150000in}}{\pgfqpoint{5.700000in}{5.700000in}}%
\pgfusepath{clip}%
\pgfsetbuttcap%
\pgfsetroundjoin%
\definecolor{currentfill}{rgb}{0.277134,0.185228,0.489898}%
\pgfsetfillcolor{currentfill}%
\pgfsetfillopacity{0.800000}%
\pgfsetlinewidth{0.000000pt}%
\definecolor{currentstroke}{rgb}{0.000000,0.000000,0.000000}%
\pgfsetstrokecolor{currentstroke}%
\pgfsetdash{}{0pt}%
\pgfpathmoveto{\pgfqpoint{4.127698in}{2.329927in}}%
\pgfpathlineto{\pgfqpoint{4.141352in}{2.331850in}}%
\pgfpathlineto{\pgfqpoint{4.155015in}{2.333967in}}%
\pgfpathlineto{\pgfqpoint{4.168689in}{2.336277in}}%
\pgfpathlineto{\pgfqpoint{4.182371in}{2.338779in}}%
\pgfpathlineto{\pgfqpoint{4.190216in}{2.348742in}}%
\pgfpathlineto{\pgfqpoint{4.198055in}{2.358676in}}%
\pgfpathlineto{\pgfqpoint{4.205889in}{2.368584in}}%
\pgfpathlineto{\pgfqpoint{4.213717in}{2.378466in}}%
\pgfpathlineto{\pgfqpoint{4.200042in}{2.376067in}}%
\pgfpathlineto{\pgfqpoint{4.186376in}{2.373860in}}%
\pgfpathlineto{\pgfqpoint{4.172720in}{2.371846in}}%
\pgfpathlineto{\pgfqpoint{4.159073in}{2.370025in}}%
\pgfpathlineto{\pgfqpoint{4.151237in}{2.360028in}}%
\pgfpathlineto{\pgfqpoint{4.143396in}{2.350014in}}%
\pgfpathlineto{\pgfqpoint{4.135549in}{2.339980in}}%
\pgfpathlineto{\pgfqpoint{4.127698in}{2.329927in}}%
\pgfpathclose%
\pgfusepath{fill}%
\end{pgfscope}%
\begin{pgfscope}%
\pgfpathrectangle{\pgfqpoint{1.150000in}{0.150000in}}{\pgfqpoint{5.700000in}{5.700000in}}%
\pgfusepath{clip}%
\pgfsetbuttcap%
\pgfsetroundjoin%
\definecolor{currentfill}{rgb}{0.283187,0.125848,0.444960}%
\pgfsetfillcolor{currentfill}%
\pgfsetfillopacity{0.800000}%
\pgfsetlinewidth{0.000000pt}%
\definecolor{currentstroke}{rgb}{0.000000,0.000000,0.000000}%
\pgfsetstrokecolor{currentstroke}%
\pgfsetdash{}{0pt}%
\pgfpathmoveto{\pgfqpoint{3.869633in}{2.198076in}}%
\pgfpathlineto{\pgfqpoint{3.883210in}{2.197750in}}%
\pgfpathlineto{\pgfqpoint{3.896794in}{2.197625in}}%
\pgfpathlineto{\pgfqpoint{3.910386in}{2.197700in}}%
\pgfpathlineto{\pgfqpoint{3.923986in}{2.197974in}}%
\pgfpathlineto{\pgfqpoint{3.931914in}{2.208349in}}%
\pgfpathlineto{\pgfqpoint{3.939837in}{2.218708in}}%
\pgfpathlineto{\pgfqpoint{3.947755in}{2.229051in}}%
\pgfpathlineto{\pgfqpoint{3.955667in}{2.239380in}}%
\pgfpathlineto{\pgfqpoint{3.942075in}{2.239113in}}%
\pgfpathlineto{\pgfqpoint{3.928491in}{2.239046in}}%
\pgfpathlineto{\pgfqpoint{3.914915in}{2.239178in}}%
\pgfpathlineto{\pgfqpoint{3.901346in}{2.239511in}}%
\pgfpathlineto{\pgfqpoint{3.893425in}{2.229163in}}%
\pgfpathlineto{\pgfqpoint{3.885499in}{2.218809in}}%
\pgfpathlineto{\pgfqpoint{3.877568in}{2.208447in}}%
\pgfpathlineto{\pgfqpoint{3.869633in}{2.198076in}}%
\pgfpathclose%
\pgfusepath{fill}%
\end{pgfscope}%
\begin{pgfscope}%
\pgfpathrectangle{\pgfqpoint{1.150000in}{0.150000in}}{\pgfqpoint{5.700000in}{5.700000in}}%
\pgfusepath{clip}%
\pgfsetbuttcap%
\pgfsetroundjoin%
\definecolor{currentfill}{rgb}{0.279566,0.067836,0.391917}%
\pgfsetfillcolor{currentfill}%
\pgfsetfillopacity{0.800000}%
\pgfsetlinewidth{0.000000pt}%
\definecolor{currentstroke}{rgb}{0.000000,0.000000,0.000000}%
\pgfsetstrokecolor{currentstroke}%
\pgfsetdash{}{0pt}%
\pgfpathmoveto{\pgfqpoint{3.470886in}{2.086516in}}%
\pgfpathlineto{\pgfqpoint{3.484399in}{2.081723in}}%
\pgfpathlineto{\pgfqpoint{3.497915in}{2.077148in}}%
\pgfpathlineto{\pgfqpoint{3.511435in}{2.072790in}}%
\pgfpathlineto{\pgfqpoint{3.524958in}{2.068647in}}%
\pgfpathlineto{\pgfqpoint{3.533018in}{2.078626in}}%
\pgfpathlineto{\pgfqpoint{3.541072in}{2.088632in}}%
\pgfpathlineto{\pgfqpoint{3.549121in}{2.098663in}}%
\pgfpathlineto{\pgfqpoint{3.557164in}{2.108720in}}%
\pgfpathlineto{\pgfqpoint{3.543652in}{2.112743in}}%
\pgfpathlineto{\pgfqpoint{3.530145in}{2.116982in}}%
\pgfpathlineto{\pgfqpoint{3.516641in}{2.121437in}}%
\pgfpathlineto{\pgfqpoint{3.503140in}{2.126110in}}%
\pgfpathlineto{\pgfqpoint{3.495085in}{2.116162in}}%
\pgfpathlineto{\pgfqpoint{3.487025in}{2.106246in}}%
\pgfpathlineto{\pgfqpoint{3.478959in}{2.096364in}}%
\pgfpathlineto{\pgfqpoint{3.470886in}{2.086516in}}%
\pgfpathclose%
\pgfusepath{fill}%
\end{pgfscope}%
\begin{pgfscope}%
\pgfpathrectangle{\pgfqpoint{1.150000in}{0.150000in}}{\pgfqpoint{5.700000in}{5.700000in}}%
\pgfusepath{clip}%
\pgfsetbuttcap%
\pgfsetroundjoin%
\definecolor{currentfill}{rgb}{0.271828,0.209303,0.504434}%
\pgfsetfillcolor{currentfill}%
\pgfsetfillopacity{0.800000}%
\pgfsetlinewidth{0.000000pt}%
\definecolor{currentstroke}{rgb}{0.000000,0.000000,0.000000}%
\pgfsetstrokecolor{currentstroke}%
\pgfsetdash{}{0pt}%
\pgfpathmoveto{\pgfqpoint{4.213717in}{2.378466in}}%
\pgfpathlineto{\pgfqpoint{4.227403in}{2.381057in}}%
\pgfpathlineto{\pgfqpoint{4.241098in}{2.383840in}}%
\pgfpathlineto{\pgfqpoint{4.254804in}{2.386814in}}%
\pgfpathlineto{\pgfqpoint{4.268521in}{2.389978in}}%
\pgfpathlineto{\pgfqpoint{4.276336in}{2.399714in}}%
\pgfpathlineto{\pgfqpoint{4.284147in}{2.409421in}}%
\pgfpathlineto{\pgfqpoint{4.291952in}{2.419099in}}%
\pgfpathlineto{\pgfqpoint{4.299752in}{2.428752in}}%
\pgfpathlineto{\pgfqpoint{4.286043in}{2.425723in}}%
\pgfpathlineto{\pgfqpoint{4.272345in}{2.422885in}}%
\pgfpathlineto{\pgfqpoint{4.258657in}{2.420237in}}%
\pgfpathlineto{\pgfqpoint{4.244979in}{2.417781in}}%
\pgfpathlineto{\pgfqpoint{4.237172in}{2.407981in}}%
\pgfpathlineto{\pgfqpoint{4.229359in}{2.398163in}}%
\pgfpathlineto{\pgfqpoint{4.221540in}{2.388325in}}%
\pgfpathlineto{\pgfqpoint{4.213717in}{2.378466in}}%
\pgfpathclose%
\pgfusepath{fill}%
\end{pgfscope}%
\begin{pgfscope}%
\pgfpathrectangle{\pgfqpoint{1.150000in}{0.150000in}}{\pgfqpoint{5.700000in}{5.700000in}}%
\pgfusepath{clip}%
\pgfsetbuttcap%
\pgfsetroundjoin%
\definecolor{currentfill}{rgb}{0.182256,0.426184,0.557120}%
\pgfsetfillcolor{currentfill}%
\pgfsetfillopacity{0.800000}%
\pgfsetlinewidth{0.000000pt}%
\definecolor{currentstroke}{rgb}{0.000000,0.000000,0.000000}%
\pgfsetstrokecolor{currentstroke}%
\pgfsetdash{}{0pt}%
\pgfpathmoveto{\pgfqpoint{5.105333in}{2.946387in}}%
\pgfpathlineto{\pgfqpoint{5.119404in}{2.953060in}}%
\pgfpathlineto{\pgfqpoint{5.133491in}{2.959910in}}%
\pgfpathlineto{\pgfqpoint{5.147593in}{2.966938in}}%
\pgfpathlineto{\pgfqpoint{5.161711in}{2.974144in}}%
\pgfpathlineto{\pgfqpoint{5.169169in}{2.980670in}}%
\pgfpathlineto{\pgfqpoint{5.176623in}{2.987249in}}%
\pgfpathlineto{\pgfqpoint{5.184072in}{2.993886in}}%
\pgfpathlineto{\pgfqpoint{5.191516in}{3.000587in}}%
\pgfpathlineto{\pgfqpoint{5.177418in}{2.993877in}}%
\pgfpathlineto{\pgfqpoint{5.163335in}{2.987343in}}%
\pgfpathlineto{\pgfqpoint{5.149267in}{2.980986in}}%
\pgfpathlineto{\pgfqpoint{5.135214in}{2.974806in}}%
\pgfpathlineto{\pgfqpoint{5.127750in}{2.967600in}}%
\pgfpathlineto{\pgfqpoint{5.120282in}{2.960465in}}%
\pgfpathlineto{\pgfqpoint{5.112810in}{2.953396in}}%
\pgfpathlineto{\pgfqpoint{5.105333in}{2.946387in}}%
\pgfpathclose%
\pgfusepath{fill}%
\end{pgfscope}%
\begin{pgfscope}%
\pgfpathrectangle{\pgfqpoint{1.150000in}{0.150000in}}{\pgfqpoint{5.700000in}{5.700000in}}%
\pgfusepath{clip}%
\pgfsetbuttcap%
\pgfsetroundjoin%
\definecolor{currentfill}{rgb}{0.223925,0.334994,0.548053}%
\pgfsetfillcolor{currentfill}%
\pgfsetfillopacity{0.800000}%
\pgfsetlinewidth{0.000000pt}%
\definecolor{currentstroke}{rgb}{0.000000,0.000000,0.000000}%
\pgfsetstrokecolor{currentstroke}%
\pgfsetdash{}{0pt}%
\pgfpathmoveto{\pgfqpoint{2.556994in}{2.746109in}}%
\pgfpathlineto{\pgfqpoint{2.570802in}{2.724711in}}%
\pgfpathlineto{\pgfqpoint{2.584598in}{2.703651in}}%
\pgfpathlineto{\pgfqpoint{2.598381in}{2.682924in}}%
\pgfpathlineto{\pgfqpoint{2.612153in}{2.662529in}}%
\pgfpathlineto{\pgfqpoint{2.620612in}{2.668621in}}%
\pgfpathlineto{\pgfqpoint{2.629060in}{2.674877in}}%
\pgfpathlineto{\pgfqpoint{2.637497in}{2.681293in}}%
\pgfpathlineto{\pgfqpoint{2.645923in}{2.687869in}}%
\pgfpathlineto{\pgfqpoint{2.632181in}{2.708034in}}%
\pgfpathlineto{\pgfqpoint{2.618428in}{2.728528in}}%
\pgfpathlineto{\pgfqpoint{2.604663in}{2.749357in}}%
\pgfpathlineto{\pgfqpoint{2.590887in}{2.770522in}}%
\pgfpathlineto{\pgfqpoint{2.582431in}{2.764166in}}%
\pgfpathlineto{\pgfqpoint{2.573964in}{2.757977in}}%
\pgfpathlineto{\pgfqpoint{2.565485in}{2.751957in}}%
\pgfpathlineto{\pgfqpoint{2.556994in}{2.746109in}}%
\pgfpathclose%
\pgfusepath{fill}%
\end{pgfscope}%
\begin{pgfscope}%
\pgfpathrectangle{\pgfqpoint{1.150000in}{0.150000in}}{\pgfqpoint{5.700000in}{5.700000in}}%
\pgfusepath{clip}%
\pgfsetbuttcap%
\pgfsetroundjoin%
\definecolor{currentfill}{rgb}{0.265145,0.232956,0.516599}%
\pgfsetfillcolor{currentfill}%
\pgfsetfillopacity{0.800000}%
\pgfsetlinewidth{0.000000pt}%
\definecolor{currentstroke}{rgb}{0.000000,0.000000,0.000000}%
\pgfsetstrokecolor{currentstroke}%
\pgfsetdash{}{0pt}%
\pgfpathmoveto{\pgfqpoint{4.299752in}{2.428752in}}%
\pgfpathlineto{\pgfqpoint{4.313471in}{2.431971in}}%
\pgfpathlineto{\pgfqpoint{4.327201in}{2.435380in}}%
\pgfpathlineto{\pgfqpoint{4.340942in}{2.438978in}}%
\pgfpathlineto{\pgfqpoint{4.354694in}{2.442765in}}%
\pgfpathlineto{\pgfqpoint{4.362481in}{2.452239in}}%
\pgfpathlineto{\pgfqpoint{4.370263in}{2.461683in}}%
\pgfpathlineto{\pgfqpoint{4.378038in}{2.471100in}}%
\pgfpathlineto{\pgfqpoint{4.385809in}{2.480491in}}%
\pgfpathlineto{\pgfqpoint{4.372065in}{2.476872in}}%
\pgfpathlineto{\pgfqpoint{4.358332in}{2.473441in}}%
\pgfpathlineto{\pgfqpoint{4.344610in}{2.470200in}}%
\pgfpathlineto{\pgfqpoint{4.330899in}{2.467148in}}%
\pgfpathlineto{\pgfqpoint{4.323120in}{2.457577in}}%
\pgfpathlineto{\pgfqpoint{4.315336in}{2.447989in}}%
\pgfpathlineto{\pgfqpoint{4.307547in}{2.438382in}}%
\pgfpathlineto{\pgfqpoint{4.299752in}{2.428752in}}%
\pgfpathclose%
\pgfusepath{fill}%
\end{pgfscope}%
\begin{pgfscope}%
\pgfpathrectangle{\pgfqpoint{1.150000in}{0.150000in}}{\pgfqpoint{5.700000in}{5.700000in}}%
\pgfusepath{clip}%
\pgfsetbuttcap%
\pgfsetroundjoin%
\definecolor{currentfill}{rgb}{0.283091,0.110553,0.431554}%
\pgfsetfillcolor{currentfill}%
\pgfsetfillopacity{0.800000}%
\pgfsetlinewidth{0.000000pt}%
\definecolor{currentstroke}{rgb}{0.000000,0.000000,0.000000}%
\pgfsetstrokecolor{currentstroke}%
\pgfsetdash{}{0pt}%
\pgfpathmoveto{\pgfqpoint{3.783564in}{2.159927in}}%
\pgfpathlineto{\pgfqpoint{3.797123in}{2.158767in}}%
\pgfpathlineto{\pgfqpoint{3.810688in}{2.157810in}}%
\pgfpathlineto{\pgfqpoint{3.824259in}{2.157057in}}%
\pgfpathlineto{\pgfqpoint{3.837838in}{2.156505in}}%
\pgfpathlineto{\pgfqpoint{3.845794in}{2.166912in}}%
\pgfpathlineto{\pgfqpoint{3.853745in}{2.177309in}}%
\pgfpathlineto{\pgfqpoint{3.861691in}{2.187697in}}%
\pgfpathlineto{\pgfqpoint{3.869633in}{2.198076in}}%
\pgfpathlineto{\pgfqpoint{3.856062in}{2.198603in}}%
\pgfpathlineto{\pgfqpoint{3.842499in}{2.199332in}}%
\pgfpathlineto{\pgfqpoint{3.828943in}{2.200264in}}%
\pgfpathlineto{\pgfqpoint{3.815393in}{2.201399in}}%
\pgfpathlineto{\pgfqpoint{3.807444in}{2.191033in}}%
\pgfpathlineto{\pgfqpoint{3.799489in}{2.180666in}}%
\pgfpathlineto{\pgfqpoint{3.791529in}{2.170297in}}%
\pgfpathlineto{\pgfqpoint{3.783564in}{2.159927in}}%
\pgfpathclose%
\pgfusepath{fill}%
\end{pgfscope}%
\begin{pgfscope}%
\pgfpathrectangle{\pgfqpoint{1.150000in}{0.150000in}}{\pgfqpoint{5.700000in}{5.700000in}}%
\pgfusepath{clip}%
\pgfsetbuttcap%
\pgfsetroundjoin%
\definecolor{currentfill}{rgb}{0.174274,0.445044,0.557792}%
\pgfsetfillcolor{currentfill}%
\pgfsetfillopacity{0.800000}%
\pgfsetlinewidth{0.000000pt}%
\definecolor{currentstroke}{rgb}{0.000000,0.000000,0.000000}%
\pgfsetstrokecolor{currentstroke}%
\pgfsetdash{}{0pt}%
\pgfpathmoveto{\pgfqpoint{5.191516in}{3.000587in}}%
\pgfpathlineto{\pgfqpoint{5.205630in}{3.007474in}}%
\pgfpathlineto{\pgfqpoint{5.219760in}{3.014538in}}%
\pgfpathlineto{\pgfqpoint{5.233905in}{3.021778in}}%
\pgfpathlineto{\pgfqpoint{5.248066in}{3.029194in}}%
\pgfpathlineto{\pgfqpoint{5.255487in}{3.035449in}}%
\pgfpathlineto{\pgfqpoint{5.262902in}{3.041771in}}%
\pgfpathlineto{\pgfqpoint{5.270314in}{3.048167in}}%
\pgfpathlineto{\pgfqpoint{5.277722in}{3.054643in}}%
\pgfpathlineto{\pgfqpoint{5.263582in}{3.047755in}}%
\pgfpathlineto{\pgfqpoint{5.249457in}{3.041042in}}%
\pgfpathlineto{\pgfqpoint{5.235348in}{3.034505in}}%
\pgfpathlineto{\pgfqpoint{5.221255in}{3.028144in}}%
\pgfpathlineto{\pgfqpoint{5.213826in}{3.021130in}}%
\pgfpathlineto{\pgfqpoint{5.206393in}{3.014203in}}%
\pgfpathlineto{\pgfqpoint{5.198957in}{3.007358in}}%
\pgfpathlineto{\pgfqpoint{5.191516in}{3.000587in}}%
\pgfpathclose%
\pgfusepath{fill}%
\end{pgfscope}%
\begin{pgfscope}%
\pgfpathrectangle{\pgfqpoint{1.150000in}{0.150000in}}{\pgfqpoint{5.700000in}{5.700000in}}%
\pgfusepath{clip}%
\pgfsetbuttcap%
\pgfsetroundjoin%
\definecolor{currentfill}{rgb}{0.257322,0.256130,0.526563}%
\pgfsetfillcolor{currentfill}%
\pgfsetfillopacity{0.800000}%
\pgfsetlinewidth{0.000000pt}%
\definecolor{currentstroke}{rgb}{0.000000,0.000000,0.000000}%
\pgfsetstrokecolor{currentstroke}%
\pgfsetdash{}{0pt}%
\pgfpathmoveto{\pgfqpoint{4.385809in}{2.480491in}}%
\pgfpathlineto{\pgfqpoint{4.399564in}{2.484299in}}%
\pgfpathlineto{\pgfqpoint{4.413331in}{2.488294in}}%
\pgfpathlineto{\pgfqpoint{4.427109in}{2.492478in}}%
\pgfpathlineto{\pgfqpoint{4.440899in}{2.496848in}}%
\pgfpathlineto{\pgfqpoint{4.448656in}{2.506029in}}%
\pgfpathlineto{\pgfqpoint{4.456407in}{2.515181in}}%
\pgfpathlineto{\pgfqpoint{4.464153in}{2.524308in}}%
\pgfpathlineto{\pgfqpoint{4.471893in}{2.533411in}}%
\pgfpathlineto{\pgfqpoint{4.458112in}{2.529241in}}%
\pgfpathlineto{\pgfqpoint{4.444343in}{2.525258in}}%
\pgfpathlineto{\pgfqpoint{4.430584in}{2.521462in}}%
\pgfpathlineto{\pgfqpoint{4.416838in}{2.517854in}}%
\pgfpathlineto{\pgfqpoint{4.409089in}{2.508539in}}%
\pgfpathlineto{\pgfqpoint{4.401334in}{2.499208in}}%
\pgfpathlineto{\pgfqpoint{4.393574in}{2.489860in}}%
\pgfpathlineto{\pgfqpoint{4.385809in}{2.480491in}}%
\pgfpathclose%
\pgfusepath{fill}%
\end{pgfscope}%
\begin{pgfscope}%
\pgfpathrectangle{\pgfqpoint{1.150000in}{0.150000in}}{\pgfqpoint{5.700000in}{5.700000in}}%
\pgfusepath{clip}%
\pgfsetbuttcap%
\pgfsetroundjoin%
\definecolor{currentfill}{rgb}{0.282910,0.105393,0.426902}%
\pgfsetfillcolor{currentfill}%
\pgfsetfillopacity{0.800000}%
\pgfsetlinewidth{0.000000pt}%
\definecolor{currentstroke}{rgb}{0.000000,0.000000,0.000000}%
\pgfsetstrokecolor{currentstroke}%
\pgfsetdash{}{0pt}%
\pgfpathmoveto{\pgfqpoint{3.048518in}{2.168090in}}%
\pgfpathlineto{\pgfqpoint{3.062070in}{2.157144in}}%
\pgfpathlineto{\pgfqpoint{3.075620in}{2.146448in}}%
\pgfpathlineto{\pgfqpoint{3.089167in}{2.136001in}}%
\pgfpathlineto{\pgfqpoint{3.102712in}{2.125801in}}%
\pgfpathlineto{\pgfqpoint{3.110945in}{2.134009in}}%
\pgfpathlineto{\pgfqpoint{3.119170in}{2.142307in}}%
\pgfpathlineto{\pgfqpoint{3.127388in}{2.150693in}}%
\pgfpathlineto{\pgfqpoint{3.135598in}{2.159165in}}%
\pgfpathlineto{\pgfqpoint{3.122072in}{2.169148in}}%
\pgfpathlineto{\pgfqpoint{3.108544in}{2.179378in}}%
\pgfpathlineto{\pgfqpoint{3.095015in}{2.189856in}}%
\pgfpathlineto{\pgfqpoint{3.081484in}{2.200584in}}%
\pgfpathlineto{\pgfqpoint{3.073254in}{2.192317in}}%
\pgfpathlineto{\pgfqpoint{3.065017in}{2.184145in}}%
\pgfpathlineto{\pgfqpoint{3.056772in}{2.176068in}}%
\pgfpathlineto{\pgfqpoint{3.048518in}{2.168090in}}%
\pgfpathclose%
\pgfusepath{fill}%
\end{pgfscope}%
\begin{pgfscope}%
\pgfpathrectangle{\pgfqpoint{1.150000in}{0.150000in}}{\pgfqpoint{5.700000in}{5.700000in}}%
\pgfusepath{clip}%
\pgfsetbuttcap%
\pgfsetroundjoin%
\definecolor{currentfill}{rgb}{0.282327,0.094955,0.417331}%
\pgfsetfillcolor{currentfill}%
\pgfsetfillopacity{0.800000}%
\pgfsetlinewidth{0.000000pt}%
\definecolor{currentstroke}{rgb}{0.000000,0.000000,0.000000}%
\pgfsetstrokecolor{currentstroke}%
\pgfsetdash{}{0pt}%
\pgfpathmoveto{\pgfqpoint{3.697444in}{2.125343in}}%
\pgfpathlineto{\pgfqpoint{3.710988in}{2.123306in}}%
\pgfpathlineto{\pgfqpoint{3.724537in}{2.121475in}}%
\pgfpathlineto{\pgfqpoint{3.738092in}{2.119850in}}%
\pgfpathlineto{\pgfqpoint{3.751653in}{2.118430in}}%
\pgfpathlineto{\pgfqpoint{3.759639in}{2.128806in}}%
\pgfpathlineto{\pgfqpoint{3.767619in}{2.139181in}}%
\pgfpathlineto{\pgfqpoint{3.775594in}{2.149555in}}%
\pgfpathlineto{\pgfqpoint{3.783564in}{2.159927in}}%
\pgfpathlineto{\pgfqpoint{3.770012in}{2.161291in}}%
\pgfpathlineto{\pgfqpoint{3.756466in}{2.162860in}}%
\pgfpathlineto{\pgfqpoint{3.742926in}{2.164634in}}%
\pgfpathlineto{\pgfqpoint{3.729392in}{2.166616in}}%
\pgfpathlineto{\pgfqpoint{3.721413in}{2.156288in}}%
\pgfpathlineto{\pgfqpoint{3.713429in}{2.145967in}}%
\pgfpathlineto{\pgfqpoint{3.705439in}{2.135652in}}%
\pgfpathlineto{\pgfqpoint{3.697444in}{2.125343in}}%
\pgfpathclose%
\pgfusepath{fill}%
\end{pgfscope}%
\begin{pgfscope}%
\pgfpathrectangle{\pgfqpoint{1.150000in}{0.150000in}}{\pgfqpoint{5.700000in}{5.700000in}}%
\pgfusepath{clip}%
\pgfsetbuttcap%
\pgfsetroundjoin%
\definecolor{currentfill}{rgb}{0.165117,0.467423,0.558141}%
\pgfsetfillcolor{currentfill}%
\pgfsetfillopacity{0.800000}%
\pgfsetlinewidth{0.000000pt}%
\definecolor{currentstroke}{rgb}{0.000000,0.000000,0.000000}%
\pgfsetstrokecolor{currentstroke}%
\pgfsetdash{}{0pt}%
\pgfpathmoveto{\pgfqpoint{5.277722in}{3.054643in}}%
\pgfpathlineto{\pgfqpoint{5.291878in}{3.061707in}}%
\pgfpathlineto{\pgfqpoint{5.306050in}{3.068947in}}%
\pgfpathlineto{\pgfqpoint{5.320239in}{3.076362in}}%
\pgfpathlineto{\pgfqpoint{5.334443in}{3.083953in}}%
\pgfpathlineto{\pgfqpoint{5.341825in}{3.089965in}}%
\pgfpathlineto{\pgfqpoint{5.349204in}{3.096061in}}%
\pgfpathlineto{\pgfqpoint{5.356578in}{3.102248in}}%
\pgfpathlineto{\pgfqpoint{5.363949in}{3.108532in}}%
\pgfpathlineto{\pgfqpoint{5.349768in}{3.101502in}}%
\pgfpathlineto{\pgfqpoint{5.335602in}{3.094647in}}%
\pgfpathlineto{\pgfqpoint{5.321452in}{3.087967in}}%
\pgfpathlineto{\pgfqpoint{5.307318in}{3.081462in}}%
\pgfpathlineto{\pgfqpoint{5.299924in}{3.074607in}}%
\pgfpathlineto{\pgfqpoint{5.292527in}{3.067857in}}%
\pgfpathlineto{\pgfqpoint{5.285126in}{3.061204in}}%
\pgfpathlineto{\pgfqpoint{5.277722in}{3.054643in}}%
\pgfpathclose%
\pgfusepath{fill}%
\end{pgfscope}%
\begin{pgfscope}%
\pgfpathrectangle{\pgfqpoint{1.150000in}{0.150000in}}{\pgfqpoint{5.700000in}{5.700000in}}%
\pgfusepath{clip}%
\pgfsetbuttcap%
\pgfsetroundjoin%
\definecolor{currentfill}{rgb}{0.248629,0.278775,0.534556}%
\pgfsetfillcolor{currentfill}%
\pgfsetfillopacity{0.800000}%
\pgfsetlinewidth{0.000000pt}%
\definecolor{currentstroke}{rgb}{0.000000,0.000000,0.000000}%
\pgfsetstrokecolor{currentstroke}%
\pgfsetdash{}{0pt}%
\pgfpathmoveto{\pgfqpoint{4.471893in}{2.533411in}}%
\pgfpathlineto{\pgfqpoint{4.485686in}{2.537768in}}%
\pgfpathlineto{\pgfqpoint{4.499491in}{2.542312in}}%
\pgfpathlineto{\pgfqpoint{4.513308in}{2.547042in}}%
\pgfpathlineto{\pgfqpoint{4.527138in}{2.551957in}}%
\pgfpathlineto{\pgfqpoint{4.534864in}{2.560819in}}%
\pgfpathlineto{\pgfqpoint{4.542584in}{2.569656in}}%
\pgfpathlineto{\pgfqpoint{4.550299in}{2.578470in}}%
\pgfpathlineto{\pgfqpoint{4.558008in}{2.587263in}}%
\pgfpathlineto{\pgfqpoint{4.544188in}{2.582581in}}%
\pgfpathlineto{\pgfqpoint{4.530380in}{2.578084in}}%
\pgfpathlineto{\pgfqpoint{4.516584in}{2.573773in}}%
\pgfpathlineto{\pgfqpoint{4.502801in}{2.569647in}}%
\pgfpathlineto{\pgfqpoint{4.495082in}{2.560609in}}%
\pgfpathlineto{\pgfqpoint{4.487358in}{2.551559in}}%
\pgfpathlineto{\pgfqpoint{4.479628in}{2.542494in}}%
\pgfpathlineto{\pgfqpoint{4.471893in}{2.533411in}}%
\pgfpathclose%
\pgfusepath{fill}%
\end{pgfscope}%
\begin{pgfscope}%
\pgfpathrectangle{\pgfqpoint{1.150000in}{0.150000in}}{\pgfqpoint{5.700000in}{5.700000in}}%
\pgfusepath{clip}%
\pgfsetbuttcap%
\pgfsetroundjoin%
\definecolor{currentfill}{rgb}{0.280267,0.073417,0.397163}%
\pgfsetfillcolor{currentfill}%
\pgfsetfillopacity{0.800000}%
\pgfsetlinewidth{0.000000pt}%
\definecolor{currentstroke}{rgb}{0.000000,0.000000,0.000000}%
\pgfsetstrokecolor{currentstroke}%
\pgfsetdash{}{0pt}%
\pgfpathmoveto{\pgfqpoint{3.243771in}{2.088004in}}%
\pgfpathlineto{\pgfqpoint{3.257291in}{2.080174in}}%
\pgfpathlineto{\pgfqpoint{3.270813in}{2.072577in}}%
\pgfpathlineto{\pgfqpoint{3.284334in}{2.065210in}}%
\pgfpathlineto{\pgfqpoint{3.297857in}{2.058072in}}%
\pgfpathlineto{\pgfqpoint{3.306008in}{2.067218in}}%
\pgfpathlineto{\pgfqpoint{3.314151in}{2.076423in}}%
\pgfpathlineto{\pgfqpoint{3.322288in}{2.085686in}}%
\pgfpathlineto{\pgfqpoint{3.330419in}{2.095006in}}%
\pgfpathlineto{\pgfqpoint{3.316912in}{2.101960in}}%
\pgfpathlineto{\pgfqpoint{3.303406in}{2.109143in}}%
\pgfpathlineto{\pgfqpoint{3.289901in}{2.116557in}}%
\pgfpathlineto{\pgfqpoint{3.276396in}{2.124203in}}%
\pgfpathlineto{\pgfqpoint{3.268250in}{2.115055in}}%
\pgfpathlineto{\pgfqpoint{3.260097in}{2.105972in}}%
\pgfpathlineto{\pgfqpoint{3.251937in}{2.096954in}}%
\pgfpathlineto{\pgfqpoint{3.243771in}{2.088004in}}%
\pgfpathclose%
\pgfusepath{fill}%
\end{pgfscope}%
\begin{pgfscope}%
\pgfpathrectangle{\pgfqpoint{1.150000in}{0.150000in}}{\pgfqpoint{5.700000in}{5.700000in}}%
\pgfusepath{clip}%
\pgfsetbuttcap%
\pgfsetroundjoin%
\definecolor{currentfill}{rgb}{0.157729,0.485932,0.558013}%
\pgfsetfillcolor{currentfill}%
\pgfsetfillopacity{0.800000}%
\pgfsetlinewidth{0.000000pt}%
\definecolor{currentstroke}{rgb}{0.000000,0.000000,0.000000}%
\pgfsetstrokecolor{currentstroke}%
\pgfsetdash{}{0pt}%
\pgfpathmoveto{\pgfqpoint{5.363949in}{3.108532in}}%
\pgfpathlineto{\pgfqpoint{5.378147in}{3.115736in}}%
\pgfpathlineto{\pgfqpoint{5.392362in}{3.123115in}}%
\pgfpathlineto{\pgfqpoint{5.406592in}{3.130669in}}%
\pgfpathlineto{\pgfqpoint{5.420840in}{3.138398in}}%
\pgfpathlineto{\pgfqpoint{5.428184in}{3.144202in}}%
\pgfpathlineto{\pgfqpoint{5.435524in}{3.150109in}}%
\pgfpathlineto{\pgfqpoint{5.442862in}{3.156124in}}%
\pgfpathlineto{\pgfqpoint{5.450197in}{3.162255in}}%
\pgfpathlineto{\pgfqpoint{5.435974in}{3.155120in}}%
\pgfpathlineto{\pgfqpoint{5.421768in}{3.148160in}}%
\pgfpathlineto{\pgfqpoint{5.407578in}{3.141373in}}%
\pgfpathlineto{\pgfqpoint{5.393404in}{3.134761in}}%
\pgfpathlineto{\pgfqpoint{5.386045in}{3.128026in}}%
\pgfpathlineto{\pgfqpoint{5.378682in}{3.121414in}}%
\pgfpathlineto{\pgfqpoint{5.371317in}{3.114918in}}%
\pgfpathlineto{\pgfqpoint{5.363949in}{3.108532in}}%
\pgfpathclose%
\pgfusepath{fill}%
\end{pgfscope}%
\begin{pgfscope}%
\pgfpathrectangle{\pgfqpoint{1.150000in}{0.150000in}}{\pgfqpoint{5.700000in}{5.700000in}}%
\pgfusepath{clip}%
\pgfsetbuttcap%
\pgfsetroundjoin%
\definecolor{currentfill}{rgb}{0.206756,0.371758,0.553117}%
\pgfsetfillcolor{currentfill}%
\pgfsetfillopacity{0.800000}%
\pgfsetlinewidth{0.000000pt}%
\definecolor{currentstroke}{rgb}{0.000000,0.000000,0.000000}%
\pgfsetstrokecolor{currentstroke}%
\pgfsetdash{}{0pt}%
\pgfpathmoveto{\pgfqpoint{2.501633in}{2.835141in}}%
\pgfpathlineto{\pgfqpoint{2.515494in}{2.812360in}}%
\pgfpathlineto{\pgfqpoint{2.529340in}{2.789930in}}%
\pgfpathlineto{\pgfqpoint{2.543174in}{2.767847in}}%
\pgfpathlineto{\pgfqpoint{2.556994in}{2.746109in}}%
\pgfpathlineto{\pgfqpoint{2.565485in}{2.751957in}}%
\pgfpathlineto{\pgfqpoint{2.573964in}{2.757977in}}%
\pgfpathlineto{\pgfqpoint{2.582431in}{2.764166in}}%
\pgfpathlineto{\pgfqpoint{2.590887in}{2.770522in}}%
\pgfpathlineto{\pgfqpoint{2.577098in}{2.792027in}}%
\pgfpathlineto{\pgfqpoint{2.563296in}{2.813876in}}%
\pgfpathlineto{\pgfqpoint{2.549481in}{2.836072in}}%
\pgfpathlineto{\pgfqpoint{2.535653in}{2.858618in}}%
\pgfpathlineto{\pgfqpoint{2.527167in}{2.852483in}}%
\pgfpathlineto{\pgfqpoint{2.518668in}{2.846524in}}%
\pgfpathlineto{\pgfqpoint{2.510156in}{2.840743in}}%
\pgfpathlineto{\pgfqpoint{2.501633in}{2.835141in}}%
\pgfpathclose%
\pgfusepath{fill}%
\end{pgfscope}%
\begin{pgfscope}%
\pgfpathrectangle{\pgfqpoint{1.150000in}{0.150000in}}{\pgfqpoint{5.700000in}{5.700000in}}%
\pgfusepath{clip}%
\pgfsetbuttcap%
\pgfsetroundjoin%
\definecolor{currentfill}{rgb}{0.278791,0.062145,0.386592}%
\pgfsetfillcolor{currentfill}%
\pgfsetfillopacity{0.800000}%
\pgfsetlinewidth{0.000000pt}%
\definecolor{currentstroke}{rgb}{0.000000,0.000000,0.000000}%
\pgfsetstrokecolor{currentstroke}%
\pgfsetdash{}{0pt}%
\pgfpathmoveto{\pgfqpoint{3.384462in}{2.069459in}}%
\pgfpathlineto{\pgfqpoint{3.397978in}{2.063633in}}%
\pgfpathlineto{\pgfqpoint{3.411496in}{2.058030in}}%
\pgfpathlineto{\pgfqpoint{3.425016in}{2.052647in}}%
\pgfpathlineto{\pgfqpoint{3.438539in}{2.047485in}}%
\pgfpathlineto{\pgfqpoint{3.446635in}{2.057186in}}%
\pgfpathlineto{\pgfqpoint{3.454725in}{2.066926in}}%
\pgfpathlineto{\pgfqpoint{3.462808in}{2.076703in}}%
\pgfpathlineto{\pgfqpoint{3.470886in}{2.086516in}}%
\pgfpathlineto{\pgfqpoint{3.457377in}{2.091527in}}%
\pgfpathlineto{\pgfqpoint{3.443870in}{2.096759in}}%
\pgfpathlineto{\pgfqpoint{3.430365in}{2.102211in}}%
\pgfpathlineto{\pgfqpoint{3.416864in}{2.107885in}}%
\pgfpathlineto{\pgfqpoint{3.408772in}{2.098211in}}%
\pgfpathlineto{\pgfqpoint{3.400675in}{2.088582in}}%
\pgfpathlineto{\pgfqpoint{3.392572in}{2.078998in}}%
\pgfpathlineto{\pgfqpoint{3.384462in}{2.069459in}}%
\pgfpathclose%
\pgfusepath{fill}%
\end{pgfscope}%
\begin{pgfscope}%
\pgfpathrectangle{\pgfqpoint{1.150000in}{0.150000in}}{\pgfqpoint{5.700000in}{5.700000in}}%
\pgfusepath{clip}%
\pgfsetbuttcap%
\pgfsetroundjoin%
\definecolor{currentfill}{rgb}{0.239346,0.300855,0.540844}%
\pgfsetfillcolor{currentfill}%
\pgfsetfillopacity{0.800000}%
\pgfsetlinewidth{0.000000pt}%
\definecolor{currentstroke}{rgb}{0.000000,0.000000,0.000000}%
\pgfsetstrokecolor{currentstroke}%
\pgfsetdash{}{0pt}%
\pgfpathmoveto{\pgfqpoint{4.558008in}{2.587263in}}%
\pgfpathlineto{\pgfqpoint{4.571840in}{2.592131in}}%
\pgfpathlineto{\pgfqpoint{4.585685in}{2.597184in}}%
\pgfpathlineto{\pgfqpoint{4.599543in}{2.602422in}}%
\pgfpathlineto{\pgfqpoint{4.613413in}{2.607844in}}%
\pgfpathlineto{\pgfqpoint{4.621107in}{2.616368in}}%
\pgfpathlineto{\pgfqpoint{4.628796in}{2.624870in}}%
\pgfpathlineto{\pgfqpoint{4.636478in}{2.633353in}}%
\pgfpathlineto{\pgfqpoint{4.644155in}{2.641821in}}%
\pgfpathlineto{\pgfqpoint{4.630295in}{2.636665in}}%
\pgfpathlineto{\pgfqpoint{4.616448in}{2.631693in}}%
\pgfpathlineto{\pgfqpoint{4.602613in}{2.626905in}}%
\pgfpathlineto{\pgfqpoint{4.588791in}{2.622302in}}%
\pgfpathlineto{\pgfqpoint{4.581103in}{2.613556in}}%
\pgfpathlineto{\pgfqpoint{4.573410in}{2.604804in}}%
\pgfpathlineto{\pgfqpoint{4.565712in}{2.596040in}}%
\pgfpathlineto{\pgfqpoint{4.558008in}{2.587263in}}%
\pgfpathclose%
\pgfusepath{fill}%
\end{pgfscope}%
\begin{pgfscope}%
\pgfpathrectangle{\pgfqpoint{1.150000in}{0.150000in}}{\pgfqpoint{5.700000in}{5.700000in}}%
\pgfusepath{clip}%
\pgfsetbuttcap%
\pgfsetroundjoin%
\definecolor{currentfill}{rgb}{0.280894,0.078907,0.402329}%
\pgfsetfillcolor{currentfill}%
\pgfsetfillopacity{0.800000}%
\pgfsetlinewidth{0.000000pt}%
\definecolor{currentstroke}{rgb}{0.000000,0.000000,0.000000}%
\pgfsetstrokecolor{currentstroke}%
\pgfsetdash{}{0pt}%
\pgfpathmoveto{\pgfqpoint{3.611250in}{2.094763in}}%
\pgfpathlineto{\pgfqpoint{3.624783in}{2.091803in}}%
\pgfpathlineto{\pgfqpoint{3.638321in}{2.089053in}}%
\pgfpathlineto{\pgfqpoint{3.651863in}{2.086512in}}%
\pgfpathlineto{\pgfqpoint{3.665411in}{2.084180in}}%
\pgfpathlineto{\pgfqpoint{3.673428in}{2.094459in}}%
\pgfpathlineto{\pgfqpoint{3.681438in}{2.104747in}}%
\pgfpathlineto{\pgfqpoint{3.689444in}{2.115041in}}%
\pgfpathlineto{\pgfqpoint{3.697444in}{2.125343in}}%
\pgfpathlineto{\pgfqpoint{3.683906in}{2.127588in}}%
\pgfpathlineto{\pgfqpoint{3.670373in}{2.130041in}}%
\pgfpathlineto{\pgfqpoint{3.656846in}{2.132703in}}%
\pgfpathlineto{\pgfqpoint{3.643323in}{2.135575in}}%
\pgfpathlineto{\pgfqpoint{3.635313in}{2.125349in}}%
\pgfpathlineto{\pgfqpoint{3.627298in}{2.115138in}}%
\pgfpathlineto{\pgfqpoint{3.619277in}{2.104943in}}%
\pgfpathlineto{\pgfqpoint{3.611250in}{2.094763in}}%
\pgfpathclose%
\pgfusepath{fill}%
\end{pgfscope}%
\begin{pgfscope}%
\pgfpathrectangle{\pgfqpoint{1.150000in}{0.150000in}}{\pgfqpoint{5.700000in}{5.700000in}}%
\pgfusepath{clip}%
\pgfsetbuttcap%
\pgfsetroundjoin%
\definecolor{currentfill}{rgb}{0.150476,0.504369,0.557430}%
\pgfsetfillcolor{currentfill}%
\pgfsetfillopacity{0.800000}%
\pgfsetlinewidth{0.000000pt}%
\definecolor{currentstroke}{rgb}{0.000000,0.000000,0.000000}%
\pgfsetstrokecolor{currentstroke}%
\pgfsetdash{}{0pt}%
\pgfpathmoveto{\pgfqpoint{5.450197in}{3.162255in}}%
\pgfpathlineto{\pgfqpoint{5.464436in}{3.169563in}}%
\pgfpathlineto{\pgfqpoint{5.478692in}{3.177045in}}%
\pgfpathlineto{\pgfqpoint{5.492965in}{3.184701in}}%
\pgfpathlineto{\pgfqpoint{5.507254in}{3.192531in}}%
\pgfpathlineto{\pgfqpoint{5.514560in}{3.198168in}}%
\pgfpathlineto{\pgfqpoint{5.521864in}{3.203927in}}%
\pgfpathlineto{\pgfqpoint{5.529166in}{3.209814in}}%
\pgfpathlineto{\pgfqpoint{5.536465in}{3.215836in}}%
\pgfpathlineto{\pgfqpoint{5.522202in}{3.208634in}}%
\pgfpathlineto{\pgfqpoint{5.507956in}{3.201604in}}%
\pgfpathlineto{\pgfqpoint{5.493727in}{3.194748in}}%
\pgfpathlineto{\pgfqpoint{5.479514in}{3.188064in}}%
\pgfpathlineto{\pgfqpoint{5.472188in}{3.181405in}}%
\pgfpathlineto{\pgfqpoint{5.464859in}{3.174888in}}%
\pgfpathlineto{\pgfqpoint{5.457529in}{3.168507in}}%
\pgfpathlineto{\pgfqpoint{5.450197in}{3.162255in}}%
\pgfpathclose%
\pgfusepath{fill}%
\end{pgfscope}%
\begin{pgfscope}%
\pgfpathrectangle{\pgfqpoint{1.150000in}{0.150000in}}{\pgfqpoint{5.700000in}{5.700000in}}%
\pgfusepath{clip}%
\pgfsetbuttcap%
\pgfsetroundjoin%
\definecolor{currentfill}{rgb}{0.229739,0.322361,0.545706}%
\pgfsetfillcolor{currentfill}%
\pgfsetfillopacity{0.800000}%
\pgfsetlinewidth{0.000000pt}%
\definecolor{currentstroke}{rgb}{0.000000,0.000000,0.000000}%
\pgfsetstrokecolor{currentstroke}%
\pgfsetdash{}{0pt}%
\pgfpathmoveto{\pgfqpoint{4.644155in}{2.641821in}}%
\pgfpathlineto{\pgfqpoint{4.658028in}{2.647162in}}%
\pgfpathlineto{\pgfqpoint{4.671915in}{2.652686in}}%
\pgfpathlineto{\pgfqpoint{4.685814in}{2.658393in}}%
\pgfpathlineto{\pgfqpoint{4.699727in}{2.664284in}}%
\pgfpathlineto{\pgfqpoint{4.707388in}{2.672454in}}%
\pgfpathlineto{\pgfqpoint{4.715043in}{2.680608in}}%
\pgfpathlineto{\pgfqpoint{4.722692in}{2.688749in}}%
\pgfpathlineto{\pgfqpoint{4.730336in}{2.696881in}}%
\pgfpathlineto{\pgfqpoint{4.716435in}{2.691289in}}%
\pgfpathlineto{\pgfqpoint{4.702546in}{2.685880in}}%
\pgfpathlineto{\pgfqpoint{4.688671in}{2.680654in}}%
\pgfpathlineto{\pgfqpoint{4.674809in}{2.675611in}}%
\pgfpathlineto{\pgfqpoint{4.667154in}{2.667169in}}%
\pgfpathlineto{\pgfqpoint{4.659493in}{2.658726in}}%
\pgfpathlineto{\pgfqpoint{4.651827in}{2.650278in}}%
\pgfpathlineto{\pgfqpoint{4.644155in}{2.641821in}}%
\pgfpathclose%
\pgfusepath{fill}%
\end{pgfscope}%
\begin{pgfscope}%
\pgfpathrectangle{\pgfqpoint{1.150000in}{0.150000in}}{\pgfqpoint{5.700000in}{5.700000in}}%
\pgfusepath{clip}%
\pgfsetbuttcap%
\pgfsetroundjoin%
\definecolor{currentfill}{rgb}{0.281924,0.089666,0.412415}%
\pgfsetfillcolor{currentfill}%
\pgfsetfillopacity{0.800000}%
\pgfsetlinewidth{0.000000pt}%
\definecolor{currentstroke}{rgb}{0.000000,0.000000,0.000000}%
\pgfsetstrokecolor{currentstroke}%
\pgfsetdash{}{0pt}%
\pgfpathmoveto{\pgfqpoint{3.102712in}{2.125801in}}%
\pgfpathlineto{\pgfqpoint{3.116256in}{2.115846in}}%
\pgfpathlineto{\pgfqpoint{3.129798in}{2.106136in}}%
\pgfpathlineto{\pgfqpoint{3.143340in}{2.096667in}}%
\pgfpathlineto{\pgfqpoint{3.156880in}{2.087440in}}%
\pgfpathlineto{\pgfqpoint{3.165093in}{2.095876in}}%
\pgfpathlineto{\pgfqpoint{3.173299in}{2.104395in}}%
\pgfpathlineto{\pgfqpoint{3.181497in}{2.112994in}}%
\pgfpathlineto{\pgfqpoint{3.189689in}{2.121671in}}%
\pgfpathlineto{\pgfqpoint{3.176167in}{2.130682in}}%
\pgfpathlineto{\pgfqpoint{3.162645in}{2.139934in}}%
\pgfpathlineto{\pgfqpoint{3.149122in}{2.149428in}}%
\pgfpathlineto{\pgfqpoint{3.135598in}{2.159165in}}%
\pgfpathlineto{\pgfqpoint{3.127388in}{2.150693in}}%
\pgfpathlineto{\pgfqpoint{3.119170in}{2.142307in}}%
\pgfpathlineto{\pgfqpoint{3.110945in}{2.134009in}}%
\pgfpathlineto{\pgfqpoint{3.102712in}{2.125801in}}%
\pgfpathclose%
\pgfusepath{fill}%
\end{pgfscope}%
\begin{pgfscope}%
\pgfpathrectangle{\pgfqpoint{1.150000in}{0.150000in}}{\pgfqpoint{5.700000in}{5.700000in}}%
\pgfusepath{clip}%
\pgfsetbuttcap%
\pgfsetroundjoin%
\definecolor{currentfill}{rgb}{0.143343,0.522773,0.556295}%
\pgfsetfillcolor{currentfill}%
\pgfsetfillopacity{0.800000}%
\pgfsetlinewidth{0.000000pt}%
\definecolor{currentstroke}{rgb}{0.000000,0.000000,0.000000}%
\pgfsetstrokecolor{currentstroke}%
\pgfsetdash{}{0pt}%
\pgfpathmoveto{\pgfqpoint{5.536465in}{3.215836in}}%
\pgfpathlineto{\pgfqpoint{5.550745in}{3.223212in}}%
\pgfpathlineto{\pgfqpoint{5.565041in}{3.230760in}}%
\pgfpathlineto{\pgfqpoint{5.579355in}{3.238482in}}%
\pgfpathlineto{\pgfqpoint{5.593686in}{3.246377in}}%
\pgfpathlineto{\pgfqpoint{5.600955in}{3.251893in}}%
\pgfpathlineto{\pgfqpoint{5.608223in}{3.257552in}}%
\pgfpathlineto{\pgfqpoint{5.615489in}{3.263360in}}%
\pgfpathlineto{\pgfqpoint{5.622754in}{3.269324in}}%
\pgfpathlineto{\pgfqpoint{5.608452in}{3.262090in}}%
\pgfpathlineto{\pgfqpoint{5.594167in}{3.255027in}}%
\pgfpathlineto{\pgfqpoint{5.579899in}{3.248137in}}%
\pgfpathlineto{\pgfqpoint{5.565647in}{3.241419in}}%
\pgfpathlineto{\pgfqpoint{5.558354in}{3.234785in}}%
\pgfpathlineto{\pgfqpoint{5.551059in}{3.228314in}}%
\pgfpathlineto{\pgfqpoint{5.543763in}{3.222001in}}%
\pgfpathlineto{\pgfqpoint{5.536465in}{3.215836in}}%
\pgfpathclose%
\pgfusepath{fill}%
\end{pgfscope}%
\begin{pgfscope}%
\pgfpathrectangle{\pgfqpoint{1.150000in}{0.150000in}}{\pgfqpoint{5.700000in}{5.700000in}}%
\pgfusepath{clip}%
\pgfsetbuttcap%
\pgfsetroundjoin%
\definecolor{currentfill}{rgb}{0.270595,0.214069,0.507052}%
\pgfsetfillcolor{currentfill}%
\pgfsetfillopacity{0.800000}%
\pgfsetlinewidth{0.000000pt}%
\definecolor{currentstroke}{rgb}{0.000000,0.000000,0.000000}%
\pgfsetstrokecolor{currentstroke}%
\pgfsetdash{}{0pt}%
\pgfpathmoveto{\pgfqpoint{2.743031in}{2.416671in}}%
\pgfpathlineto{\pgfqpoint{2.756709in}{2.400052in}}%
\pgfpathlineto{\pgfqpoint{2.770378in}{2.383725in}}%
\pgfpathlineto{\pgfqpoint{2.784040in}{2.367688in}}%
\pgfpathlineto{\pgfqpoint{2.797695in}{2.351938in}}%
\pgfpathlineto{\pgfqpoint{2.806081in}{2.358446in}}%
\pgfpathlineto{\pgfqpoint{2.814458in}{2.365093in}}%
\pgfpathlineto{\pgfqpoint{2.822824in}{2.371875in}}%
\pgfpathlineto{\pgfqpoint{2.831180in}{2.378792in}}%
\pgfpathlineto{\pgfqpoint{2.817552in}{2.394286in}}%
\pgfpathlineto{\pgfqpoint{2.803917in}{2.410066in}}%
\pgfpathlineto{\pgfqpoint{2.790275in}{2.426136in}}%
\pgfpathlineto{\pgfqpoint{2.776626in}{2.442498in}}%
\pgfpathlineto{\pgfqpoint{2.768243in}{2.435825in}}%
\pgfpathlineto{\pgfqpoint{2.759849in}{2.429295in}}%
\pgfpathlineto{\pgfqpoint{2.751446in}{2.422910in}}%
\pgfpathlineto{\pgfqpoint{2.743031in}{2.416671in}}%
\pgfpathclose%
\pgfusepath{fill}%
\end{pgfscope}%
\begin{pgfscope}%
\pgfpathrectangle{\pgfqpoint{1.150000in}{0.150000in}}{\pgfqpoint{5.700000in}{5.700000in}}%
\pgfusepath{clip}%
\pgfsetbuttcap%
\pgfsetroundjoin%
\definecolor{currentfill}{rgb}{0.277134,0.185228,0.489898}%
\pgfsetfillcolor{currentfill}%
\pgfsetfillopacity{0.800000}%
\pgfsetlinewidth{0.000000pt}%
\definecolor{currentstroke}{rgb}{0.000000,0.000000,0.000000}%
\pgfsetstrokecolor{currentstroke}%
\pgfsetdash{}{0pt}%
\pgfpathmoveto{\pgfqpoint{2.797695in}{2.351938in}}%
\pgfpathlineto{\pgfqpoint{2.811343in}{2.336473in}}%
\pgfpathlineto{\pgfqpoint{2.824985in}{2.321291in}}%
\pgfpathlineto{\pgfqpoint{2.838620in}{2.306388in}}%
\pgfpathlineto{\pgfqpoint{2.852250in}{2.291764in}}%
\pgfpathlineto{\pgfqpoint{2.860609in}{2.298539in}}%
\pgfpathlineto{\pgfqpoint{2.868959in}{2.305444in}}%
\pgfpathlineto{\pgfqpoint{2.877299in}{2.312477in}}%
\pgfpathlineto{\pgfqpoint{2.885630in}{2.319636in}}%
\pgfpathlineto{\pgfqpoint{2.872027in}{2.334007in}}%
\pgfpathlineto{\pgfqpoint{2.858417in}{2.348654in}}%
\pgfpathlineto{\pgfqpoint{2.844802in}{2.363582in}}%
\pgfpathlineto{\pgfqpoint{2.831180in}{2.378792in}}%
\pgfpathlineto{\pgfqpoint{2.822824in}{2.371875in}}%
\pgfpathlineto{\pgfqpoint{2.814458in}{2.365093in}}%
\pgfpathlineto{\pgfqpoint{2.806081in}{2.358446in}}%
\pgfpathlineto{\pgfqpoint{2.797695in}{2.351938in}}%
\pgfpathclose%
\pgfusepath{fill}%
\end{pgfscope}%
\begin{pgfscope}%
\pgfpathrectangle{\pgfqpoint{1.150000in}{0.150000in}}{\pgfqpoint{5.700000in}{5.700000in}}%
\pgfusepath{clip}%
\pgfsetbuttcap%
\pgfsetroundjoin%
\definecolor{currentfill}{rgb}{0.220057,0.343307,0.549413}%
\pgfsetfillcolor{currentfill}%
\pgfsetfillopacity{0.800000}%
\pgfsetlinewidth{0.000000pt}%
\definecolor{currentstroke}{rgb}{0.000000,0.000000,0.000000}%
\pgfsetstrokecolor{currentstroke}%
\pgfsetdash{}{0pt}%
\pgfpathmoveto{\pgfqpoint{4.730336in}{2.696881in}}%
\pgfpathlineto{\pgfqpoint{4.744251in}{2.702656in}}%
\pgfpathlineto{\pgfqpoint{4.758180in}{2.708613in}}%
\pgfpathlineto{\pgfqpoint{4.772122in}{2.714752in}}%
\pgfpathlineto{\pgfqpoint{4.786078in}{2.721074in}}%
\pgfpathlineto{\pgfqpoint{4.793705in}{2.728881in}}%
\pgfpathlineto{\pgfqpoint{4.801326in}{2.736679in}}%
\pgfpathlineto{\pgfqpoint{4.808941in}{2.744471in}}%
\pgfpathlineto{\pgfqpoint{4.816550in}{2.752261in}}%
\pgfpathlineto{\pgfqpoint{4.802606in}{2.746271in}}%
\pgfpathlineto{\pgfqpoint{4.788676in}{2.740463in}}%
\pgfpathlineto{\pgfqpoint{4.774760in}{2.734837in}}%
\pgfpathlineto{\pgfqpoint{4.760857in}{2.729393in}}%
\pgfpathlineto{\pgfqpoint{4.753235in}{2.721260in}}%
\pgfpathlineto{\pgfqpoint{4.745607in}{2.713133in}}%
\pgfpathlineto{\pgfqpoint{4.737975in}{2.705008in}}%
\pgfpathlineto{\pgfqpoint{4.730336in}{2.696881in}}%
\pgfpathclose%
\pgfusepath{fill}%
\end{pgfscope}%
\begin{pgfscope}%
\pgfpathrectangle{\pgfqpoint{1.150000in}{0.150000in}}{\pgfqpoint{5.700000in}{5.700000in}}%
\pgfusepath{clip}%
\pgfsetbuttcap%
\pgfsetroundjoin%
\definecolor{currentfill}{rgb}{0.136408,0.541173,0.554483}%
\pgfsetfillcolor{currentfill}%
\pgfsetfillopacity{0.800000}%
\pgfsetlinewidth{0.000000pt}%
\definecolor{currentstroke}{rgb}{0.000000,0.000000,0.000000}%
\pgfsetstrokecolor{currentstroke}%
\pgfsetdash{}{0pt}%
\pgfpathmoveto{\pgfqpoint{5.622754in}{3.269324in}}%
\pgfpathlineto{\pgfqpoint{5.637073in}{3.276731in}}%
\pgfpathlineto{\pgfqpoint{5.651409in}{3.284310in}}%
\pgfpathlineto{\pgfqpoint{5.665763in}{3.292061in}}%
\pgfpathlineto{\pgfqpoint{5.680134in}{3.299985in}}%
\pgfpathlineto{\pgfqpoint{5.687368in}{3.305433in}}%
\pgfpathlineto{\pgfqpoint{5.694601in}{3.311045in}}%
\pgfpathlineto{\pgfqpoint{5.701833in}{3.316828in}}%
\pgfpathlineto{\pgfqpoint{5.709065in}{3.322791in}}%
\pgfpathlineto{\pgfqpoint{5.694725in}{3.315561in}}%
\pgfpathlineto{\pgfqpoint{5.680403in}{3.308502in}}%
\pgfpathlineto{\pgfqpoint{5.666097in}{3.301614in}}%
\pgfpathlineto{\pgfqpoint{5.651808in}{3.294897in}}%
\pgfpathlineto{\pgfqpoint{5.644545in}{3.288232in}}%
\pgfpathlineto{\pgfqpoint{5.637282in}{3.281753in}}%
\pgfpathlineto{\pgfqpoint{5.630018in}{3.275453in}}%
\pgfpathlineto{\pgfqpoint{5.622754in}{3.269324in}}%
\pgfpathclose%
\pgfusepath{fill}%
\end{pgfscope}%
\begin{pgfscope}%
\pgfpathrectangle{\pgfqpoint{1.150000in}{0.150000in}}{\pgfqpoint{5.700000in}{5.700000in}}%
\pgfusepath{clip}%
\pgfsetbuttcap%
\pgfsetroundjoin%
\definecolor{currentfill}{rgb}{0.262138,0.242286,0.520837}%
\pgfsetfillcolor{currentfill}%
\pgfsetfillopacity{0.800000}%
\pgfsetlinewidth{0.000000pt}%
\definecolor{currentstroke}{rgb}{0.000000,0.000000,0.000000}%
\pgfsetstrokecolor{currentstroke}%
\pgfsetdash{}{0pt}%
\pgfpathmoveto{\pgfqpoint{2.688240in}{2.486116in}}%
\pgfpathlineto{\pgfqpoint{2.701951in}{2.468304in}}%
\pgfpathlineto{\pgfqpoint{2.715653in}{2.450794in}}%
\pgfpathlineto{\pgfqpoint{2.729346in}{2.433584in}}%
\pgfpathlineto{\pgfqpoint{2.743031in}{2.416671in}}%
\pgfpathlineto{\pgfqpoint{2.751446in}{2.422910in}}%
\pgfpathlineto{\pgfqpoint{2.759849in}{2.429295in}}%
\pgfpathlineto{\pgfqpoint{2.768243in}{2.435825in}}%
\pgfpathlineto{\pgfqpoint{2.776626in}{2.442498in}}%
\pgfpathlineto{\pgfqpoint{2.762969in}{2.459153in}}%
\pgfpathlineto{\pgfqpoint{2.749304in}{2.476105in}}%
\pgfpathlineto{\pgfqpoint{2.735630in}{2.493356in}}%
\pgfpathlineto{\pgfqpoint{2.721949in}{2.510909in}}%
\pgfpathlineto{\pgfqpoint{2.713538in}{2.504483in}}%
\pgfpathlineto{\pgfqpoint{2.705116in}{2.498207in}}%
\pgfpathlineto{\pgfqpoint{2.696684in}{2.492084in}}%
\pgfpathlineto{\pgfqpoint{2.688240in}{2.486116in}}%
\pgfpathclose%
\pgfusepath{fill}%
\end{pgfscope}%
\begin{pgfscope}%
\pgfpathrectangle{\pgfqpoint{1.150000in}{0.150000in}}{\pgfqpoint{5.700000in}{5.700000in}}%
\pgfusepath{clip}%
\pgfsetbuttcap%
\pgfsetroundjoin%
\definecolor{currentfill}{rgb}{0.279566,0.067836,0.391917}%
\pgfsetfillcolor{currentfill}%
\pgfsetfillopacity{0.800000}%
\pgfsetlinewidth{0.000000pt}%
\definecolor{currentstroke}{rgb}{0.000000,0.000000,0.000000}%
\pgfsetstrokecolor{currentstroke}%
\pgfsetdash{}{0pt}%
\pgfpathmoveto{\pgfqpoint{3.524958in}{2.068647in}}%
\pgfpathlineto{\pgfqpoint{3.538485in}{2.064719in}}%
\pgfpathlineto{\pgfqpoint{3.552016in}{2.061004in}}%
\pgfpathlineto{\pgfqpoint{3.565550in}{2.057503in}}%
\pgfpathlineto{\pgfqpoint{3.579090in}{2.054213in}}%
\pgfpathlineto{\pgfqpoint{3.587138in}{2.064323in}}%
\pgfpathlineto{\pgfqpoint{3.595181in}{2.074452in}}%
\pgfpathlineto{\pgfqpoint{3.603218in}{2.084599in}}%
\pgfpathlineto{\pgfqpoint{3.611250in}{2.094763in}}%
\pgfpathlineto{\pgfqpoint{3.597722in}{2.097933in}}%
\pgfpathlineto{\pgfqpoint{3.584198in}{2.101316in}}%
\pgfpathlineto{\pgfqpoint{3.570679in}{2.104911in}}%
\pgfpathlineto{\pgfqpoint{3.557164in}{2.108720in}}%
\pgfpathlineto{\pgfqpoint{3.549121in}{2.098663in}}%
\pgfpathlineto{\pgfqpoint{3.541072in}{2.088632in}}%
\pgfpathlineto{\pgfqpoint{3.533018in}{2.078626in}}%
\pgfpathlineto{\pgfqpoint{3.524958in}{2.068647in}}%
\pgfpathclose%
\pgfusepath{fill}%
\end{pgfscope}%
\begin{pgfscope}%
\pgfpathrectangle{\pgfqpoint{1.150000in}{0.150000in}}{\pgfqpoint{5.700000in}{5.700000in}}%
\pgfusepath{clip}%
\pgfsetbuttcap%
\pgfsetroundjoin%
\definecolor{currentfill}{rgb}{0.280868,0.160771,0.472899}%
\pgfsetfillcolor{currentfill}%
\pgfsetfillopacity{0.800000}%
\pgfsetlinewidth{0.000000pt}%
\definecolor{currentstroke}{rgb}{0.000000,0.000000,0.000000}%
\pgfsetstrokecolor{currentstroke}%
\pgfsetdash{}{0pt}%
\pgfpathmoveto{\pgfqpoint{2.852250in}{2.291764in}}%
\pgfpathlineto{\pgfqpoint{2.865873in}{2.277415in}}%
\pgfpathlineto{\pgfqpoint{2.879491in}{2.263340in}}%
\pgfpathlineto{\pgfqpoint{2.893104in}{2.249537in}}%
\pgfpathlineto{\pgfqpoint{2.906711in}{2.236003in}}%
\pgfpathlineto{\pgfqpoint{2.915045in}{2.243043in}}%
\pgfpathlineto{\pgfqpoint{2.923370in}{2.250206in}}%
\pgfpathlineto{\pgfqpoint{2.931686in}{2.257488in}}%
\pgfpathlineto{\pgfqpoint{2.939993in}{2.264889in}}%
\pgfpathlineto{\pgfqpoint{2.926409in}{2.278170in}}%
\pgfpathlineto{\pgfqpoint{2.912821in}{2.291720in}}%
\pgfpathlineto{\pgfqpoint{2.899228in}{2.305542in}}%
\pgfpathlineto{\pgfqpoint{2.885630in}{2.319636in}}%
\pgfpathlineto{\pgfqpoint{2.877299in}{2.312477in}}%
\pgfpathlineto{\pgfqpoint{2.868959in}{2.305444in}}%
\pgfpathlineto{\pgfqpoint{2.860609in}{2.298539in}}%
\pgfpathlineto{\pgfqpoint{2.852250in}{2.291764in}}%
\pgfpathclose%
\pgfusepath{fill}%
\end{pgfscope}%
\begin{pgfscope}%
\pgfpathrectangle{\pgfqpoint{1.150000in}{0.150000in}}{\pgfqpoint{5.700000in}{5.700000in}}%
\pgfusepath{clip}%
\pgfsetbuttcap%
\pgfsetroundjoin%
\definecolor{currentfill}{rgb}{0.129933,0.559582,0.551864}%
\pgfsetfillcolor{currentfill}%
\pgfsetfillopacity{0.800000}%
\pgfsetlinewidth{0.000000pt}%
\definecolor{currentstroke}{rgb}{0.000000,0.000000,0.000000}%
\pgfsetstrokecolor{currentstroke}%
\pgfsetdash{}{0pt}%
\pgfpathmoveto{\pgfqpoint{5.709065in}{3.322791in}}%
\pgfpathlineto{\pgfqpoint{5.723423in}{3.330193in}}%
\pgfpathlineto{\pgfqpoint{5.737798in}{3.337767in}}%
\pgfpathlineto{\pgfqpoint{5.752190in}{3.345511in}}%
\pgfpathlineto{\pgfqpoint{5.766601in}{3.353427in}}%
\pgfpathlineto{\pgfqpoint{5.773801in}{3.358864in}}%
\pgfpathlineto{\pgfqpoint{5.781001in}{3.364489in}}%
\pgfpathlineto{\pgfqpoint{5.788201in}{3.370309in}}%
\pgfpathlineto{\pgfqpoint{5.795402in}{3.376333in}}%
\pgfpathlineto{\pgfqpoint{5.781025in}{3.369142in}}%
\pgfpathlineto{\pgfqpoint{5.766666in}{3.362122in}}%
\pgfpathlineto{\pgfqpoint{5.752324in}{3.355272in}}%
\pgfpathlineto{\pgfqpoint{5.737999in}{3.348593in}}%
\pgfpathlineto{\pgfqpoint{5.730764in}{3.341835in}}%
\pgfpathlineto{\pgfqpoint{5.723530in}{3.335287in}}%
\pgfpathlineto{\pgfqpoint{5.716298in}{3.328942in}}%
\pgfpathlineto{\pgfqpoint{5.709065in}{3.322791in}}%
\pgfpathclose%
\pgfusepath{fill}%
\end{pgfscope}%
\begin{pgfscope}%
\pgfpathrectangle{\pgfqpoint{1.150000in}{0.150000in}}{\pgfqpoint{5.700000in}{5.700000in}}%
\pgfusepath{clip}%
\pgfsetbuttcap%
\pgfsetroundjoin%
\definecolor{currentfill}{rgb}{0.208623,0.367752,0.552675}%
\pgfsetfillcolor{currentfill}%
\pgfsetfillopacity{0.800000}%
\pgfsetlinewidth{0.000000pt}%
\definecolor{currentstroke}{rgb}{0.000000,0.000000,0.000000}%
\pgfsetstrokecolor{currentstroke}%
\pgfsetdash{}{0pt}%
\pgfpathmoveto{\pgfqpoint{4.816550in}{2.752261in}}%
\pgfpathlineto{\pgfqpoint{4.830508in}{2.758432in}}%
\pgfpathlineto{\pgfqpoint{4.844480in}{2.764785in}}%
\pgfpathlineto{\pgfqpoint{4.858466in}{2.771318in}}%
\pgfpathlineto{\pgfqpoint{4.872467in}{2.778033in}}%
\pgfpathlineto{\pgfqpoint{4.880058in}{2.785473in}}%
\pgfpathlineto{\pgfqpoint{4.887643in}{2.792912in}}%
\pgfpathlineto{\pgfqpoint{4.895222in}{2.800353in}}%
\pgfpathlineto{\pgfqpoint{4.902797in}{2.807802in}}%
\pgfpathlineto{\pgfqpoint{4.888810in}{2.801452in}}%
\pgfpathlineto{\pgfqpoint{4.874837in}{2.795283in}}%
\pgfpathlineto{\pgfqpoint{4.860879in}{2.789295in}}%
\pgfpathlineto{\pgfqpoint{4.846934in}{2.783487in}}%
\pgfpathlineto{\pgfqpoint{4.839346in}{2.775662in}}%
\pgfpathlineto{\pgfqpoint{4.831753in}{2.767853in}}%
\pgfpathlineto{\pgfqpoint{4.824154in}{2.760054in}}%
\pgfpathlineto{\pgfqpoint{4.816550in}{2.752261in}}%
\pgfpathclose%
\pgfusepath{fill}%
\end{pgfscope}%
\begin{pgfscope}%
\pgfpathrectangle{\pgfqpoint{1.150000in}{0.150000in}}{\pgfqpoint{5.700000in}{5.700000in}}%
\pgfusepath{clip}%
\pgfsetbuttcap%
\pgfsetroundjoin%
\definecolor{currentfill}{rgb}{0.250425,0.274290,0.533103}%
\pgfsetfillcolor{currentfill}%
\pgfsetfillopacity{0.800000}%
\pgfsetlinewidth{0.000000pt}%
\definecolor{currentstroke}{rgb}{0.000000,0.000000,0.000000}%
\pgfsetstrokecolor{currentstroke}%
\pgfsetdash{}{0pt}%
\pgfpathmoveto{\pgfqpoint{2.633303in}{2.560442in}}%
\pgfpathlineto{\pgfqpoint{2.647052in}{2.541393in}}%
\pgfpathlineto{\pgfqpoint{2.660791in}{2.522658in}}%
\pgfpathlineto{\pgfqpoint{2.674520in}{2.504233in}}%
\pgfpathlineto{\pgfqpoint{2.688240in}{2.486116in}}%
\pgfpathlineto{\pgfqpoint{2.696684in}{2.492084in}}%
\pgfpathlineto{\pgfqpoint{2.705116in}{2.498207in}}%
\pgfpathlineto{\pgfqpoint{2.713538in}{2.504483in}}%
\pgfpathlineto{\pgfqpoint{2.721949in}{2.510909in}}%
\pgfpathlineto{\pgfqpoint{2.708258in}{2.528767in}}%
\pgfpathlineto{\pgfqpoint{2.694559in}{2.546931in}}%
\pgfpathlineto{\pgfqpoint{2.680850in}{2.565406in}}%
\pgfpathlineto{\pgfqpoint{2.667131in}{2.584193in}}%
\pgfpathlineto{\pgfqpoint{2.658691in}{2.578015in}}%
\pgfpathlineto{\pgfqpoint{2.650240in}{2.571996in}}%
\pgfpathlineto{\pgfqpoint{2.641777in}{2.566137in}}%
\pgfpathlineto{\pgfqpoint{2.633303in}{2.560442in}}%
\pgfpathclose%
\pgfusepath{fill}%
\end{pgfscope}%
\begin{pgfscope}%
\pgfpathrectangle{\pgfqpoint{1.150000in}{0.150000in}}{\pgfqpoint{5.700000in}{5.700000in}}%
\pgfusepath{clip}%
\pgfsetbuttcap%
\pgfsetroundjoin%
\definecolor{currentfill}{rgb}{0.278791,0.062145,0.386592}%
\pgfsetfillcolor{currentfill}%
\pgfsetfillopacity{0.800000}%
\pgfsetlinewidth{0.000000pt}%
\definecolor{currentstroke}{rgb}{0.000000,0.000000,0.000000}%
\pgfsetstrokecolor{currentstroke}%
\pgfsetdash{}{0pt}%
\pgfpathmoveto{\pgfqpoint{3.297857in}{2.058072in}}%
\pgfpathlineto{\pgfqpoint{3.311381in}{2.051163in}}%
\pgfpathlineto{\pgfqpoint{3.324906in}{2.044481in}}%
\pgfpathlineto{\pgfqpoint{3.338433in}{2.038025in}}%
\pgfpathlineto{\pgfqpoint{3.351961in}{2.031793in}}%
\pgfpathlineto{\pgfqpoint{3.360096in}{2.041134in}}%
\pgfpathlineto{\pgfqpoint{3.368224in}{2.050526in}}%
\pgfpathlineto{\pgfqpoint{3.376346in}{2.059968in}}%
\pgfpathlineto{\pgfqpoint{3.384462in}{2.069459in}}%
\pgfpathlineto{\pgfqpoint{3.370949in}{2.075508in}}%
\pgfpathlineto{\pgfqpoint{3.357437in}{2.081781in}}%
\pgfpathlineto{\pgfqpoint{3.343927in}{2.088280in}}%
\pgfpathlineto{\pgfqpoint{3.330419in}{2.095006in}}%
\pgfpathlineto{\pgfqpoint{3.322288in}{2.085686in}}%
\pgfpathlineto{\pgfqpoint{3.314151in}{2.076423in}}%
\pgfpathlineto{\pgfqpoint{3.306008in}{2.067218in}}%
\pgfpathlineto{\pgfqpoint{3.297857in}{2.058072in}}%
\pgfpathclose%
\pgfusepath{fill}%
\end{pgfscope}%
\begin{pgfscope}%
\pgfpathrectangle{\pgfqpoint{1.150000in}{0.150000in}}{\pgfqpoint{5.700000in}{5.700000in}}%
\pgfusepath{clip}%
\pgfsetbuttcap%
\pgfsetroundjoin%
\definecolor{currentfill}{rgb}{0.282623,0.140926,0.457517}%
\pgfsetfillcolor{currentfill}%
\pgfsetfillopacity{0.800000}%
\pgfsetlinewidth{0.000000pt}%
\definecolor{currentstroke}{rgb}{0.000000,0.000000,0.000000}%
\pgfsetstrokecolor{currentstroke}%
\pgfsetdash{}{0pt}%
\pgfpathmoveto{\pgfqpoint{2.906711in}{2.236003in}}%
\pgfpathlineto{\pgfqpoint{2.920314in}{2.222736in}}%
\pgfpathlineto{\pgfqpoint{2.933913in}{2.209735in}}%
\pgfpathlineto{\pgfqpoint{2.947507in}{2.196998in}}%
\pgfpathlineto{\pgfqpoint{2.961097in}{2.184521in}}%
\pgfpathlineto{\pgfqpoint{2.969407in}{2.191826in}}%
\pgfpathlineto{\pgfqpoint{2.977708in}{2.199244in}}%
\pgfpathlineto{\pgfqpoint{2.986000in}{2.206774in}}%
\pgfpathlineto{\pgfqpoint{2.994283in}{2.214414in}}%
\pgfpathlineto{\pgfqpoint{2.980717in}{2.226639in}}%
\pgfpathlineto{\pgfqpoint{2.967146in}{2.239125in}}%
\pgfpathlineto{\pgfqpoint{2.953571in}{2.251874in}}%
\pgfpathlineto{\pgfqpoint{2.939993in}{2.264889in}}%
\pgfpathlineto{\pgfqpoint{2.931686in}{2.257488in}}%
\pgfpathlineto{\pgfqpoint{2.923370in}{2.250206in}}%
\pgfpathlineto{\pgfqpoint{2.915045in}{2.243043in}}%
\pgfpathlineto{\pgfqpoint{2.906711in}{2.236003in}}%
\pgfpathclose%
\pgfusepath{fill}%
\end{pgfscope}%
\begin{pgfscope}%
\pgfpathrectangle{\pgfqpoint{1.150000in}{0.150000in}}{\pgfqpoint{5.700000in}{5.700000in}}%
\pgfusepath{clip}%
\pgfsetbuttcap%
\pgfsetroundjoin%
\definecolor{currentfill}{rgb}{0.124395,0.578002,0.548287}%
\pgfsetfillcolor{currentfill}%
\pgfsetfillopacity{0.800000}%
\pgfsetlinewidth{0.000000pt}%
\definecolor{currentstroke}{rgb}{0.000000,0.000000,0.000000}%
\pgfsetstrokecolor{currentstroke}%
\pgfsetdash{}{0pt}%
\pgfpathmoveto{\pgfqpoint{5.795402in}{3.376333in}}%
\pgfpathlineto{\pgfqpoint{5.809797in}{3.383694in}}%
\pgfpathlineto{\pgfqpoint{5.824210in}{3.391226in}}%
\pgfpathlineto{\pgfqpoint{5.838640in}{3.398928in}}%
\pgfpathlineto{\pgfqpoint{5.853088in}{3.406801in}}%
\pgfpathlineto{\pgfqpoint{5.860256in}{3.412291in}}%
\pgfpathlineto{\pgfqpoint{5.867425in}{3.417993in}}%
\pgfpathlineto{\pgfqpoint{5.874596in}{3.423916in}}%
\pgfpathlineto{\pgfqpoint{5.881769in}{3.430068in}}%
\pgfpathlineto{\pgfqpoint{5.867357in}{3.422954in}}%
\pgfpathlineto{\pgfqpoint{5.852962in}{3.416008in}}%
\pgfpathlineto{\pgfqpoint{5.838584in}{3.409233in}}%
\pgfpathlineto{\pgfqpoint{5.824224in}{3.402627in}}%
\pgfpathlineto{\pgfqpoint{5.817016in}{3.395707in}}%
\pgfpathlineto{\pgfqpoint{5.809809in}{3.389024in}}%
\pgfpathlineto{\pgfqpoint{5.802605in}{3.382568in}}%
\pgfpathlineto{\pgfqpoint{5.795402in}{3.376333in}}%
\pgfpathclose%
\pgfusepath{fill}%
\end{pgfscope}%
\begin{pgfscope}%
\pgfpathrectangle{\pgfqpoint{1.150000in}{0.150000in}}{\pgfqpoint{5.700000in}{5.700000in}}%
\pgfusepath{clip}%
\pgfsetbuttcap%
\pgfsetroundjoin%
\definecolor{currentfill}{rgb}{0.199430,0.387607,0.554642}%
\pgfsetfillcolor{currentfill}%
\pgfsetfillopacity{0.800000}%
\pgfsetlinewidth{0.000000pt}%
\definecolor{currentstroke}{rgb}{0.000000,0.000000,0.000000}%
\pgfsetstrokecolor{currentstroke}%
\pgfsetdash{}{0pt}%
\pgfpathmoveto{\pgfqpoint{4.902797in}{2.807802in}}%
\pgfpathlineto{\pgfqpoint{4.916798in}{2.814332in}}%
\pgfpathlineto{\pgfqpoint{4.930814in}{2.821043in}}%
\pgfpathlineto{\pgfqpoint{4.944844in}{2.827933in}}%
\pgfpathlineto{\pgfqpoint{4.958890in}{2.835003in}}%
\pgfpathlineto{\pgfqpoint{4.966444in}{2.842078in}}%
\pgfpathlineto{\pgfqpoint{4.973993in}{2.849160in}}%
\pgfpathlineto{\pgfqpoint{4.981536in}{2.856256in}}%
\pgfpathlineto{\pgfqpoint{4.989074in}{2.863368in}}%
\pgfpathlineto{\pgfqpoint{4.975043in}{2.856696in}}%
\pgfpathlineto{\pgfqpoint{4.961028in}{2.850203in}}%
\pgfpathlineto{\pgfqpoint{4.947027in}{2.843890in}}%
\pgfpathlineto{\pgfqpoint{4.933040in}{2.837756in}}%
\pgfpathlineto{\pgfqpoint{4.925487in}{2.830235in}}%
\pgfpathlineto{\pgfqpoint{4.917929in}{2.822738in}}%
\pgfpathlineto{\pgfqpoint{4.910365in}{2.815262in}}%
\pgfpathlineto{\pgfqpoint{4.902797in}{2.807802in}}%
\pgfpathclose%
\pgfusepath{fill}%
\end{pgfscope}%
\begin{pgfscope}%
\pgfpathrectangle{\pgfqpoint{1.150000in}{0.150000in}}{\pgfqpoint{5.700000in}{5.700000in}}%
\pgfusepath{clip}%
\pgfsetbuttcap%
\pgfsetroundjoin%
\definecolor{currentfill}{rgb}{0.280868,0.160771,0.472899}%
\pgfsetfillcolor{currentfill}%
\pgfsetfillopacity{0.800000}%
\pgfsetlinewidth{0.000000pt}%
\definecolor{currentstroke}{rgb}{0.000000,0.000000,0.000000}%
\pgfsetstrokecolor{currentstroke}%
\pgfsetdash{}{0pt}%
\pgfpathmoveto{\pgfqpoint{4.010115in}{2.242428in}}%
\pgfpathlineto{\pgfqpoint{4.023748in}{2.243682in}}%
\pgfpathlineto{\pgfqpoint{4.037390in}{2.245132in}}%
\pgfpathlineto{\pgfqpoint{4.051040in}{2.246777in}}%
\pgfpathlineto{\pgfqpoint{4.064699in}{2.248617in}}%
\pgfpathlineto{\pgfqpoint{4.072592in}{2.258876in}}%
\pgfpathlineto{\pgfqpoint{4.080480in}{2.269107in}}%
\pgfpathlineto{\pgfqpoint{4.088362in}{2.279309in}}%
\pgfpathlineto{\pgfqpoint{4.096240in}{2.289483in}}%
\pgfpathlineto{\pgfqpoint{4.082587in}{2.287683in}}%
\pgfpathlineto{\pgfqpoint{4.068944in}{2.286077in}}%
\pgfpathlineto{\pgfqpoint{4.055310in}{2.284666in}}%
\pgfpathlineto{\pgfqpoint{4.041685in}{2.283452in}}%
\pgfpathlineto{\pgfqpoint{4.033800in}{2.273226in}}%
\pgfpathlineto{\pgfqpoint{4.025910in}{2.262981in}}%
\pgfpathlineto{\pgfqpoint{4.018015in}{2.252715in}}%
\pgfpathlineto{\pgfqpoint{4.010115in}{2.242428in}}%
\pgfpathclose%
\pgfusepath{fill}%
\end{pgfscope}%
\begin{pgfscope}%
\pgfpathrectangle{\pgfqpoint{1.150000in}{0.150000in}}{\pgfqpoint{5.700000in}{5.700000in}}%
\pgfusepath{clip}%
\pgfsetbuttcap%
\pgfsetroundjoin%
\definecolor{currentfill}{rgb}{0.282884,0.135920,0.453427}%
\pgfsetfillcolor{currentfill}%
\pgfsetfillopacity{0.800000}%
\pgfsetlinewidth{0.000000pt}%
\definecolor{currentstroke}{rgb}{0.000000,0.000000,0.000000}%
\pgfsetstrokecolor{currentstroke}%
\pgfsetdash{}{0pt}%
\pgfpathmoveto{\pgfqpoint{3.923986in}{2.197974in}}%
\pgfpathlineto{\pgfqpoint{3.937593in}{2.198446in}}%
\pgfpathlineto{\pgfqpoint{3.951209in}{2.199117in}}%
\pgfpathlineto{\pgfqpoint{3.964832in}{2.199986in}}%
\pgfpathlineto{\pgfqpoint{3.978464in}{2.201051in}}%
\pgfpathlineto{\pgfqpoint{3.986385in}{2.211431in}}%
\pgfpathlineto{\pgfqpoint{3.994300in}{2.221786in}}%
\pgfpathlineto{\pgfqpoint{4.002210in}{2.232118in}}%
\pgfpathlineto{\pgfqpoint{4.010115in}{2.242428in}}%
\pgfpathlineto{\pgfqpoint{3.996491in}{2.241370in}}%
\pgfpathlineto{\pgfqpoint{3.982875in}{2.240509in}}%
\pgfpathlineto{\pgfqpoint{3.969267in}{2.239846in}}%
\pgfpathlineto{\pgfqpoint{3.955667in}{2.239380in}}%
\pgfpathlineto{\pgfqpoint{3.947755in}{2.229051in}}%
\pgfpathlineto{\pgfqpoint{3.939837in}{2.218708in}}%
\pgfpathlineto{\pgfqpoint{3.931914in}{2.208349in}}%
\pgfpathlineto{\pgfqpoint{3.923986in}{2.197974in}}%
\pgfpathclose%
\pgfusepath{fill}%
\end{pgfscope}%
\begin{pgfscope}%
\pgfpathrectangle{\pgfqpoint{1.150000in}{0.150000in}}{\pgfqpoint{5.700000in}{5.700000in}}%
\pgfusepath{clip}%
\pgfsetbuttcap%
\pgfsetroundjoin%
\definecolor{currentfill}{rgb}{0.280894,0.078907,0.402329}%
\pgfsetfillcolor{currentfill}%
\pgfsetfillopacity{0.800000}%
\pgfsetlinewidth{0.000000pt}%
\definecolor{currentstroke}{rgb}{0.000000,0.000000,0.000000}%
\pgfsetstrokecolor{currentstroke}%
\pgfsetdash{}{0pt}%
\pgfpathmoveto{\pgfqpoint{3.156880in}{2.087440in}}%
\pgfpathlineto{\pgfqpoint{3.170419in}{2.078452in}}%
\pgfpathlineto{\pgfqpoint{3.183958in}{2.069702in}}%
\pgfpathlineto{\pgfqpoint{3.197496in}{2.061188in}}%
\pgfpathlineto{\pgfqpoint{3.211034in}{2.052909in}}%
\pgfpathlineto{\pgfqpoint{3.219229in}{2.061573in}}%
\pgfpathlineto{\pgfqpoint{3.227416in}{2.070312in}}%
\pgfpathlineto{\pgfqpoint{3.235597in}{2.079123in}}%
\pgfpathlineto{\pgfqpoint{3.243771in}{2.088004in}}%
\pgfpathlineto{\pgfqpoint{3.230250in}{2.096067in}}%
\pgfpathlineto{\pgfqpoint{3.216730in}{2.104365in}}%
\pgfpathlineto{\pgfqpoint{3.203209in}{2.112899in}}%
\pgfpathlineto{\pgfqpoint{3.189689in}{2.121671in}}%
\pgfpathlineto{\pgfqpoint{3.181497in}{2.112994in}}%
\pgfpathlineto{\pgfqpoint{3.173299in}{2.104395in}}%
\pgfpathlineto{\pgfqpoint{3.165093in}{2.095876in}}%
\pgfpathlineto{\pgfqpoint{3.156880in}{2.087440in}}%
\pgfpathclose%
\pgfusepath{fill}%
\end{pgfscope}%
\begin{pgfscope}%
\pgfpathrectangle{\pgfqpoint{1.150000in}{0.150000in}}{\pgfqpoint{5.700000in}{5.700000in}}%
\pgfusepath{clip}%
\pgfsetbuttcap%
\pgfsetroundjoin%
\definecolor{currentfill}{rgb}{0.278012,0.180367,0.486697}%
\pgfsetfillcolor{currentfill}%
\pgfsetfillopacity{0.800000}%
\pgfsetlinewidth{0.000000pt}%
\definecolor{currentstroke}{rgb}{0.000000,0.000000,0.000000}%
\pgfsetstrokecolor{currentstroke}%
\pgfsetdash{}{0pt}%
\pgfpathmoveto{\pgfqpoint{4.096240in}{2.289483in}}%
\pgfpathlineto{\pgfqpoint{4.109901in}{2.291477in}}%
\pgfpathlineto{\pgfqpoint{4.123572in}{2.293666in}}%
\pgfpathlineto{\pgfqpoint{4.137252in}{2.296047in}}%
\pgfpathlineto{\pgfqpoint{4.150943in}{2.298621in}}%
\pgfpathlineto{\pgfqpoint{4.158808in}{2.308709in}}%
\pgfpathlineto{\pgfqpoint{4.166667in}{2.318764in}}%
\pgfpathlineto{\pgfqpoint{4.174522in}{2.328787in}}%
\pgfpathlineto{\pgfqpoint{4.182371in}{2.338779in}}%
\pgfpathlineto{\pgfqpoint{4.168689in}{2.336277in}}%
\pgfpathlineto{\pgfqpoint{4.155015in}{2.333967in}}%
\pgfpathlineto{\pgfqpoint{4.141352in}{2.331850in}}%
\pgfpathlineto{\pgfqpoint{4.127698in}{2.329927in}}%
\pgfpathlineto{\pgfqpoint{4.119841in}{2.319852in}}%
\pgfpathlineto{\pgfqpoint{4.111979in}{2.309753in}}%
\pgfpathlineto{\pgfqpoint{4.104112in}{2.299631in}}%
\pgfpathlineto{\pgfqpoint{4.096240in}{2.289483in}}%
\pgfpathclose%
\pgfusepath{fill}%
\end{pgfscope}%
\begin{pgfscope}%
\pgfpathrectangle{\pgfqpoint{1.150000in}{0.150000in}}{\pgfqpoint{5.700000in}{5.700000in}}%
\pgfusepath{clip}%
\pgfsetbuttcap%
\pgfsetroundjoin%
\definecolor{currentfill}{rgb}{0.283229,0.120777,0.440584}%
\pgfsetfillcolor{currentfill}%
\pgfsetfillopacity{0.800000}%
\pgfsetlinewidth{0.000000pt}%
\definecolor{currentstroke}{rgb}{0.000000,0.000000,0.000000}%
\pgfsetstrokecolor{currentstroke}%
\pgfsetdash{}{0pt}%
\pgfpathmoveto{\pgfqpoint{3.837838in}{2.156505in}}%
\pgfpathlineto{\pgfqpoint{3.851423in}{2.156155in}}%
\pgfpathlineto{\pgfqpoint{3.865016in}{2.156005in}}%
\pgfpathlineto{\pgfqpoint{3.878616in}{2.156056in}}%
\pgfpathlineto{\pgfqpoint{3.892223in}{2.156306in}}%
\pgfpathlineto{\pgfqpoint{3.900172in}{2.166748in}}%
\pgfpathlineto{\pgfqpoint{3.908115in}{2.177174in}}%
\pgfpathlineto{\pgfqpoint{3.916053in}{2.187582in}}%
\pgfpathlineto{\pgfqpoint{3.923986in}{2.197974in}}%
\pgfpathlineto{\pgfqpoint{3.910386in}{2.197700in}}%
\pgfpathlineto{\pgfqpoint{3.896794in}{2.197625in}}%
\pgfpathlineto{\pgfqpoint{3.883210in}{2.197750in}}%
\pgfpathlineto{\pgfqpoint{3.869633in}{2.198076in}}%
\pgfpathlineto{\pgfqpoint{3.861691in}{2.187697in}}%
\pgfpathlineto{\pgfqpoint{3.853745in}{2.177309in}}%
\pgfpathlineto{\pgfqpoint{3.845794in}{2.166912in}}%
\pgfpathlineto{\pgfqpoint{3.837838in}{2.156505in}}%
\pgfpathclose%
\pgfusepath{fill}%
\end{pgfscope}%
\begin{pgfscope}%
\pgfpathrectangle{\pgfqpoint{1.150000in}{0.150000in}}{\pgfqpoint{5.700000in}{5.700000in}}%
\pgfusepath{clip}%
\pgfsetbuttcap%
\pgfsetroundjoin%
\definecolor{currentfill}{rgb}{0.237441,0.305202,0.541921}%
\pgfsetfillcolor{currentfill}%
\pgfsetfillopacity{0.800000}%
\pgfsetlinewidth{0.000000pt}%
\definecolor{currentstroke}{rgb}{0.000000,0.000000,0.000000}%
\pgfsetstrokecolor{currentstroke}%
\pgfsetdash{}{0pt}%
\pgfpathmoveto{\pgfqpoint{2.578200in}{2.639829in}}%
\pgfpathlineto{\pgfqpoint{2.591992in}{2.619497in}}%
\pgfpathlineto{\pgfqpoint{2.605773in}{2.599491in}}%
\pgfpathlineto{\pgfqpoint{2.619544in}{2.579807in}}%
\pgfpathlineto{\pgfqpoint{2.633303in}{2.560442in}}%
\pgfpathlineto{\pgfqpoint{2.641777in}{2.566137in}}%
\pgfpathlineto{\pgfqpoint{2.650240in}{2.571996in}}%
\pgfpathlineto{\pgfqpoint{2.658691in}{2.578015in}}%
\pgfpathlineto{\pgfqpoint{2.667131in}{2.584193in}}%
\pgfpathlineto{\pgfqpoint{2.653403in}{2.603296in}}%
\pgfpathlineto{\pgfqpoint{2.639664in}{2.622718in}}%
\pgfpathlineto{\pgfqpoint{2.625914in}{2.642461in}}%
\pgfpathlineto{\pgfqpoint{2.612153in}{2.662529in}}%
\pgfpathlineto{\pgfqpoint{2.603683in}{2.656601in}}%
\pgfpathlineto{\pgfqpoint{2.595200in}{2.650840in}}%
\pgfpathlineto{\pgfqpoint{2.586706in}{2.645249in}}%
\pgfpathlineto{\pgfqpoint{2.578200in}{2.639829in}}%
\pgfpathclose%
\pgfusepath{fill}%
\end{pgfscope}%
\begin{pgfscope}%
\pgfpathrectangle{\pgfqpoint{1.150000in}{0.150000in}}{\pgfqpoint{5.700000in}{5.700000in}}%
\pgfusepath{clip}%
\pgfsetbuttcap%
\pgfsetroundjoin%
\definecolor{currentfill}{rgb}{0.278791,0.062145,0.386592}%
\pgfsetfillcolor{currentfill}%
\pgfsetfillopacity{0.800000}%
\pgfsetlinewidth{0.000000pt}%
\definecolor{currentstroke}{rgb}{0.000000,0.000000,0.000000}%
\pgfsetstrokecolor{currentstroke}%
\pgfsetdash{}{0pt}%
\pgfpathmoveto{\pgfqpoint{3.438539in}{2.047485in}}%
\pgfpathlineto{\pgfqpoint{3.452065in}{2.042541in}}%
\pgfpathlineto{\pgfqpoint{3.465594in}{2.037815in}}%
\pgfpathlineto{\pgfqpoint{3.479126in}{2.033305in}}%
\pgfpathlineto{\pgfqpoint{3.492661in}{2.029012in}}%
\pgfpathlineto{\pgfqpoint{3.500744in}{2.038876in}}%
\pgfpathlineto{\pgfqpoint{3.508821in}{2.048771in}}%
\pgfpathlineto{\pgfqpoint{3.516892in}{2.058695in}}%
\pgfpathlineto{\pgfqpoint{3.524958in}{2.068647in}}%
\pgfpathlineto{\pgfqpoint{3.511435in}{2.072790in}}%
\pgfpathlineto{\pgfqpoint{3.497915in}{2.077148in}}%
\pgfpathlineto{\pgfqpoint{3.484399in}{2.081723in}}%
\pgfpathlineto{\pgfqpoint{3.470886in}{2.086516in}}%
\pgfpathlineto{\pgfqpoint{3.462808in}{2.076703in}}%
\pgfpathlineto{\pgfqpoint{3.454725in}{2.066926in}}%
\pgfpathlineto{\pgfqpoint{3.446635in}{2.057186in}}%
\pgfpathlineto{\pgfqpoint{3.438539in}{2.047485in}}%
\pgfpathclose%
\pgfusepath{fill}%
\end{pgfscope}%
\begin{pgfscope}%
\pgfpathrectangle{\pgfqpoint{1.150000in}{0.150000in}}{\pgfqpoint{5.700000in}{5.700000in}}%
\pgfusepath{clip}%
\pgfsetbuttcap%
\pgfsetroundjoin%
\definecolor{currentfill}{rgb}{0.273006,0.204520,0.501721}%
\pgfsetfillcolor{currentfill}%
\pgfsetfillopacity{0.800000}%
\pgfsetlinewidth{0.000000pt}%
\definecolor{currentstroke}{rgb}{0.000000,0.000000,0.000000}%
\pgfsetstrokecolor{currentstroke}%
\pgfsetdash{}{0pt}%
\pgfpathmoveto{\pgfqpoint{4.182371in}{2.338779in}}%
\pgfpathlineto{\pgfqpoint{4.196064in}{2.341474in}}%
\pgfpathlineto{\pgfqpoint{4.209767in}{2.344360in}}%
\pgfpathlineto{\pgfqpoint{4.223481in}{2.347437in}}%
\pgfpathlineto{\pgfqpoint{4.237204in}{2.350705in}}%
\pgfpathlineto{\pgfqpoint{4.245041in}{2.360576in}}%
\pgfpathlineto{\pgfqpoint{4.252873in}{2.370411in}}%
\pgfpathlineto{\pgfqpoint{4.260699in}{2.380211in}}%
\pgfpathlineto{\pgfqpoint{4.268521in}{2.389978in}}%
\pgfpathlineto{\pgfqpoint{4.254804in}{2.386814in}}%
\pgfpathlineto{\pgfqpoint{4.241098in}{2.383840in}}%
\pgfpathlineto{\pgfqpoint{4.227403in}{2.381057in}}%
\pgfpathlineto{\pgfqpoint{4.213717in}{2.378466in}}%
\pgfpathlineto{\pgfqpoint{4.205889in}{2.368584in}}%
\pgfpathlineto{\pgfqpoint{4.198055in}{2.358676in}}%
\pgfpathlineto{\pgfqpoint{4.190216in}{2.348742in}}%
\pgfpathlineto{\pgfqpoint{4.182371in}{2.338779in}}%
\pgfpathclose%
\pgfusepath{fill}%
\end{pgfscope}%
\begin{pgfscope}%
\pgfpathrectangle{\pgfqpoint{1.150000in}{0.150000in}}{\pgfqpoint{5.700000in}{5.700000in}}%
\pgfusepath{clip}%
\pgfsetbuttcap%
\pgfsetroundjoin%
\definecolor{currentfill}{rgb}{0.190631,0.407061,0.556089}%
\pgfsetfillcolor{currentfill}%
\pgfsetfillopacity{0.800000}%
\pgfsetlinewidth{0.000000pt}%
\definecolor{currentstroke}{rgb}{0.000000,0.000000,0.000000}%
\pgfsetstrokecolor{currentstroke}%
\pgfsetdash{}{0pt}%
\pgfpathmoveto{\pgfqpoint{4.989074in}{2.863368in}}%
\pgfpathlineto{\pgfqpoint{5.003119in}{2.870220in}}%
\pgfpathlineto{\pgfqpoint{5.017179in}{2.877251in}}%
\pgfpathlineto{\pgfqpoint{5.031254in}{2.884460in}}%
\pgfpathlineto{\pgfqpoint{5.045345in}{2.891849in}}%
\pgfpathlineto{\pgfqpoint{5.052861in}{2.898565in}}%
\pgfpathlineto{\pgfqpoint{5.060373in}{2.905300in}}%
\pgfpathlineto{\pgfqpoint{5.067878in}{2.912058in}}%
\pgfpathlineto{\pgfqpoint{5.075379in}{2.918846in}}%
\pgfpathlineto{\pgfqpoint{5.061305in}{2.911888in}}%
\pgfpathlineto{\pgfqpoint{5.047246in}{2.905109in}}%
\pgfpathlineto{\pgfqpoint{5.033203in}{2.898509in}}%
\pgfpathlineto{\pgfqpoint{5.019174in}{2.892087in}}%
\pgfpathlineto{\pgfqpoint{5.011656in}{2.884857in}}%
\pgfpathlineto{\pgfqpoint{5.004134in}{2.877664in}}%
\pgfpathlineto{\pgfqpoint{4.996606in}{2.870503in}}%
\pgfpathlineto{\pgfqpoint{4.989074in}{2.863368in}}%
\pgfpathclose%
\pgfusepath{fill}%
\end{pgfscope}%
\begin{pgfscope}%
\pgfpathrectangle{\pgfqpoint{1.150000in}{0.150000in}}{\pgfqpoint{5.700000in}{5.700000in}}%
\pgfusepath{clip}%
\pgfsetbuttcap%
\pgfsetroundjoin%
\definecolor{currentfill}{rgb}{0.282656,0.100196,0.422160}%
\pgfsetfillcolor{currentfill}%
\pgfsetfillopacity{0.800000}%
\pgfsetlinewidth{0.000000pt}%
\definecolor{currentstroke}{rgb}{0.000000,0.000000,0.000000}%
\pgfsetstrokecolor{currentstroke}%
\pgfsetdash{}{0pt}%
\pgfpathmoveto{\pgfqpoint{3.751653in}{2.118430in}}%
\pgfpathlineto{\pgfqpoint{3.765220in}{2.117214in}}%
\pgfpathlineto{\pgfqpoint{3.778794in}{2.116201in}}%
\pgfpathlineto{\pgfqpoint{3.792374in}{2.115392in}}%
\pgfpathlineto{\pgfqpoint{3.805961in}{2.114784in}}%
\pgfpathlineto{\pgfqpoint{3.813938in}{2.125228in}}%
\pgfpathlineto{\pgfqpoint{3.821910in}{2.135663in}}%
\pgfpathlineto{\pgfqpoint{3.829876in}{2.146089in}}%
\pgfpathlineto{\pgfqpoint{3.837838in}{2.156505in}}%
\pgfpathlineto{\pgfqpoint{3.824259in}{2.157057in}}%
\pgfpathlineto{\pgfqpoint{3.810688in}{2.157810in}}%
\pgfpathlineto{\pgfqpoint{3.797123in}{2.158767in}}%
\pgfpathlineto{\pgfqpoint{3.783564in}{2.159927in}}%
\pgfpathlineto{\pgfqpoint{3.775594in}{2.149555in}}%
\pgfpathlineto{\pgfqpoint{3.767619in}{2.139181in}}%
\pgfpathlineto{\pgfqpoint{3.759639in}{2.128806in}}%
\pgfpathlineto{\pgfqpoint{3.751653in}{2.118430in}}%
\pgfpathclose%
\pgfusepath{fill}%
\end{pgfscope}%
\begin{pgfscope}%
\pgfpathrectangle{\pgfqpoint{1.150000in}{0.150000in}}{\pgfqpoint{5.700000in}{5.700000in}}%
\pgfusepath{clip}%
\pgfsetbuttcap%
\pgfsetroundjoin%
\definecolor{currentfill}{rgb}{0.283229,0.120777,0.440584}%
\pgfsetfillcolor{currentfill}%
\pgfsetfillopacity{0.800000}%
\pgfsetlinewidth{0.000000pt}%
\definecolor{currentstroke}{rgb}{0.000000,0.000000,0.000000}%
\pgfsetstrokecolor{currentstroke}%
\pgfsetdash{}{0pt}%
\pgfpathmoveto{\pgfqpoint{2.961097in}{2.184521in}}%
\pgfpathlineto{\pgfqpoint{2.974683in}{2.172305in}}%
\pgfpathlineto{\pgfqpoint{2.988266in}{2.160346in}}%
\pgfpathlineto{\pgfqpoint{3.001846in}{2.148643in}}%
\pgfpathlineto{\pgfqpoint{3.015422in}{2.137195in}}%
\pgfpathlineto{\pgfqpoint{3.023709in}{2.144762in}}%
\pgfpathlineto{\pgfqpoint{3.031987in}{2.152434in}}%
\pgfpathlineto{\pgfqpoint{3.040257in}{2.160211in}}%
\pgfpathlineto{\pgfqpoint{3.048518in}{2.168090in}}%
\pgfpathlineto{\pgfqpoint{3.034964in}{2.179288in}}%
\pgfpathlineto{\pgfqpoint{3.021407in}{2.190741in}}%
\pgfpathlineto{\pgfqpoint{3.007847in}{2.202449in}}%
\pgfpathlineto{\pgfqpoint{2.994283in}{2.214414in}}%
\pgfpathlineto{\pgfqpoint{2.986000in}{2.206774in}}%
\pgfpathlineto{\pgfqpoint{2.977708in}{2.199244in}}%
\pgfpathlineto{\pgfqpoint{2.969407in}{2.191826in}}%
\pgfpathlineto{\pgfqpoint{2.961097in}{2.184521in}}%
\pgfpathclose%
\pgfusepath{fill}%
\end{pgfscope}%
\begin{pgfscope}%
\pgfpathrectangle{\pgfqpoint{1.150000in}{0.150000in}}{\pgfqpoint{5.700000in}{5.700000in}}%
\pgfusepath{clip}%
\pgfsetbuttcap%
\pgfsetroundjoin%
\definecolor{currentfill}{rgb}{0.266580,0.228262,0.514349}%
\pgfsetfillcolor{currentfill}%
\pgfsetfillopacity{0.800000}%
\pgfsetlinewidth{0.000000pt}%
\definecolor{currentstroke}{rgb}{0.000000,0.000000,0.000000}%
\pgfsetstrokecolor{currentstroke}%
\pgfsetdash{}{0pt}%
\pgfpathmoveto{\pgfqpoint{4.268521in}{2.389978in}}%
\pgfpathlineto{\pgfqpoint{4.282248in}{2.393333in}}%
\pgfpathlineto{\pgfqpoint{4.295985in}{2.396877in}}%
\pgfpathlineto{\pgfqpoint{4.309734in}{2.400612in}}%
\pgfpathlineto{\pgfqpoint{4.323494in}{2.404535in}}%
\pgfpathlineto{\pgfqpoint{4.331302in}{2.414147in}}%
\pgfpathlineto{\pgfqpoint{4.339105in}{2.423721in}}%
\pgfpathlineto{\pgfqpoint{4.346902in}{2.433260in}}%
\pgfpathlineto{\pgfqpoint{4.354694in}{2.442765in}}%
\pgfpathlineto{\pgfqpoint{4.340942in}{2.438978in}}%
\pgfpathlineto{\pgfqpoint{4.327201in}{2.435380in}}%
\pgfpathlineto{\pgfqpoint{4.313471in}{2.431971in}}%
\pgfpathlineto{\pgfqpoint{4.299752in}{2.428752in}}%
\pgfpathlineto{\pgfqpoint{4.291952in}{2.419099in}}%
\pgfpathlineto{\pgfqpoint{4.284147in}{2.409421in}}%
\pgfpathlineto{\pgfqpoint{4.276336in}{2.399714in}}%
\pgfpathlineto{\pgfqpoint{4.268521in}{2.389978in}}%
\pgfpathclose%
\pgfusepath{fill}%
\end{pgfscope}%
\begin{pgfscope}%
\pgfpathrectangle{\pgfqpoint{1.150000in}{0.150000in}}{\pgfqpoint{5.700000in}{5.700000in}}%
\pgfusepath{clip}%
\pgfsetbuttcap%
\pgfsetroundjoin%
\definecolor{currentfill}{rgb}{0.120565,0.596422,0.543611}%
\pgfsetfillcolor{currentfill}%
\pgfsetfillopacity{0.800000}%
\pgfsetlinewidth{0.000000pt}%
\definecolor{currentstroke}{rgb}{0.000000,0.000000,0.000000}%
\pgfsetstrokecolor{currentstroke}%
\pgfsetdash{}{0pt}%
\pgfpathmoveto{\pgfqpoint{5.881769in}{3.430068in}}%
\pgfpathlineto{\pgfqpoint{5.896200in}{3.437353in}}%
\pgfpathlineto{\pgfqpoint{5.910648in}{3.444808in}}%
\pgfpathlineto{\pgfqpoint{5.925115in}{3.452432in}}%
\pgfpathlineto{\pgfqpoint{5.939600in}{3.460226in}}%
\pgfpathlineto{\pgfqpoint{5.946738in}{3.465838in}}%
\pgfpathlineto{\pgfqpoint{5.953880in}{3.471689in}}%
\pgfpathlineto{\pgfqpoint{5.961024in}{3.477787in}}%
\pgfpathlineto{\pgfqpoint{5.946568in}{3.470583in}}%
\pgfpathlineto{\pgfqpoint{5.932129in}{3.463549in}}%
\pgfpathlineto{\pgfqpoint{5.917709in}{3.456683in}}%
\pgfpathlineto{\pgfqpoint{5.903306in}{3.449986in}}%
\pgfpathlineto{\pgfqpoint{5.896124in}{3.443095in}}%
\pgfpathlineto{\pgfqpoint{5.888945in}{3.436459in}}%
\pgfpathlineto{\pgfqpoint{5.881769in}{3.430068in}}%
\pgfpathclose%
\pgfusepath{fill}%
\end{pgfscope}%
\begin{pgfscope}%
\pgfpathrectangle{\pgfqpoint{1.150000in}{0.150000in}}{\pgfqpoint{5.700000in}{5.700000in}}%
\pgfusepath{clip}%
\pgfsetbuttcap%
\pgfsetroundjoin%
\definecolor{currentfill}{rgb}{0.182256,0.426184,0.557120}%
\pgfsetfillcolor{currentfill}%
\pgfsetfillopacity{0.800000}%
\pgfsetlinewidth{0.000000pt}%
\definecolor{currentstroke}{rgb}{0.000000,0.000000,0.000000}%
\pgfsetstrokecolor{currentstroke}%
\pgfsetdash{}{0pt}%
\pgfpathmoveto{\pgfqpoint{5.075379in}{2.918846in}}%
\pgfpathlineto{\pgfqpoint{5.089468in}{2.925981in}}%
\pgfpathlineto{\pgfqpoint{5.103573in}{2.933295in}}%
\pgfpathlineto{\pgfqpoint{5.117694in}{2.940788in}}%
\pgfpathlineto{\pgfqpoint{5.131830in}{2.948458in}}%
\pgfpathlineto{\pgfqpoint{5.139308in}{2.954827in}}%
\pgfpathlineto{\pgfqpoint{5.146780in}{2.961228in}}%
\pgfpathlineto{\pgfqpoint{5.154248in}{2.967665in}}%
\pgfpathlineto{\pgfqpoint{5.161711in}{2.974144in}}%
\pgfpathlineto{\pgfqpoint{5.147593in}{2.966938in}}%
\pgfpathlineto{\pgfqpoint{5.133491in}{2.959910in}}%
\pgfpathlineto{\pgfqpoint{5.119404in}{2.953060in}}%
\pgfpathlineto{\pgfqpoint{5.105333in}{2.946387in}}%
\pgfpathlineto{\pgfqpoint{5.097852in}{2.939432in}}%
\pgfpathlineto{\pgfqpoint{5.090366in}{2.932528in}}%
\pgfpathlineto{\pgfqpoint{5.082875in}{2.925667in}}%
\pgfpathlineto{\pgfqpoint{5.075379in}{2.918846in}}%
\pgfpathclose%
\pgfusepath{fill}%
\end{pgfscope}%
\begin{pgfscope}%
\pgfpathrectangle{\pgfqpoint{1.150000in}{0.150000in}}{\pgfqpoint{5.700000in}{5.700000in}}%
\pgfusepath{clip}%
\pgfsetbuttcap%
\pgfsetroundjoin%
\definecolor{currentfill}{rgb}{0.258965,0.251537,0.524736}%
\pgfsetfillcolor{currentfill}%
\pgfsetfillopacity{0.800000}%
\pgfsetlinewidth{0.000000pt}%
\definecolor{currentstroke}{rgb}{0.000000,0.000000,0.000000}%
\pgfsetstrokecolor{currentstroke}%
\pgfsetdash{}{0pt}%
\pgfpathmoveto{\pgfqpoint{4.354694in}{2.442765in}}%
\pgfpathlineto{\pgfqpoint{4.368458in}{2.446741in}}%
\pgfpathlineto{\pgfqpoint{4.382232in}{2.450905in}}%
\pgfpathlineto{\pgfqpoint{4.396019in}{2.455257in}}%
\pgfpathlineto{\pgfqpoint{4.409817in}{2.459796in}}%
\pgfpathlineto{\pgfqpoint{4.417595in}{2.469113in}}%
\pgfpathlineto{\pgfqpoint{4.425369in}{2.478393in}}%
\pgfpathlineto{\pgfqpoint{4.433137in}{2.487637in}}%
\pgfpathlineto{\pgfqpoint{4.440899in}{2.496848in}}%
\pgfpathlineto{\pgfqpoint{4.427109in}{2.492478in}}%
\pgfpathlineto{\pgfqpoint{4.413331in}{2.488294in}}%
\pgfpathlineto{\pgfqpoint{4.399564in}{2.484299in}}%
\pgfpathlineto{\pgfqpoint{4.385809in}{2.480491in}}%
\pgfpathlineto{\pgfqpoint{4.378038in}{2.471100in}}%
\pgfpathlineto{\pgfqpoint{4.370263in}{2.461683in}}%
\pgfpathlineto{\pgfqpoint{4.362481in}{2.452239in}}%
\pgfpathlineto{\pgfqpoint{4.354694in}{2.442765in}}%
\pgfpathclose%
\pgfusepath{fill}%
\end{pgfscope}%
\begin{pgfscope}%
\pgfpathrectangle{\pgfqpoint{1.150000in}{0.150000in}}{\pgfqpoint{5.700000in}{5.700000in}}%
\pgfusepath{clip}%
\pgfsetbuttcap%
\pgfsetroundjoin%
\definecolor{currentfill}{rgb}{0.281446,0.084320,0.407414}%
\pgfsetfillcolor{currentfill}%
\pgfsetfillopacity{0.800000}%
\pgfsetlinewidth{0.000000pt}%
\definecolor{currentstroke}{rgb}{0.000000,0.000000,0.000000}%
\pgfsetstrokecolor{currentstroke}%
\pgfsetdash{}{0pt}%
\pgfpathmoveto{\pgfqpoint{3.665411in}{2.084180in}}%
\pgfpathlineto{\pgfqpoint{3.678965in}{2.082055in}}%
\pgfpathlineto{\pgfqpoint{3.692524in}{2.080136in}}%
\pgfpathlineto{\pgfqpoint{3.706088in}{2.078423in}}%
\pgfpathlineto{\pgfqpoint{3.719659in}{2.076916in}}%
\pgfpathlineto{\pgfqpoint{3.727665in}{2.087295in}}%
\pgfpathlineto{\pgfqpoint{3.735666in}{2.097674in}}%
\pgfpathlineto{\pgfqpoint{3.743662in}{2.108052in}}%
\pgfpathlineto{\pgfqpoint{3.751653in}{2.118430in}}%
\pgfpathlineto{\pgfqpoint{3.738092in}{2.119850in}}%
\pgfpathlineto{\pgfqpoint{3.724537in}{2.121475in}}%
\pgfpathlineto{\pgfqpoint{3.710988in}{2.123306in}}%
\pgfpathlineto{\pgfqpoint{3.697444in}{2.125343in}}%
\pgfpathlineto{\pgfqpoint{3.689444in}{2.115041in}}%
\pgfpathlineto{\pgfqpoint{3.681438in}{2.104747in}}%
\pgfpathlineto{\pgfqpoint{3.673428in}{2.094459in}}%
\pgfpathlineto{\pgfqpoint{3.665411in}{2.084180in}}%
\pgfpathclose%
\pgfusepath{fill}%
\end{pgfscope}%
\begin{pgfscope}%
\pgfpathrectangle{\pgfqpoint{1.150000in}{0.150000in}}{\pgfqpoint{5.700000in}{5.700000in}}%
\pgfusepath{clip}%
\pgfsetbuttcap%
\pgfsetroundjoin%
\definecolor{currentfill}{rgb}{0.221989,0.339161,0.548752}%
\pgfsetfillcolor{currentfill}%
\pgfsetfillopacity{0.800000}%
\pgfsetlinewidth{0.000000pt}%
\definecolor{currentstroke}{rgb}{0.000000,0.000000,0.000000}%
\pgfsetstrokecolor{currentstroke}%
\pgfsetdash{}{0pt}%
\pgfpathmoveto{\pgfqpoint{2.522910in}{2.724470in}}%
\pgfpathlineto{\pgfqpoint{2.536751in}{2.702806in}}%
\pgfpathlineto{\pgfqpoint{2.550580in}{2.681480in}}%
\pgfpathlineto{\pgfqpoint{2.564396in}{2.660489in}}%
\pgfpathlineto{\pgfqpoint{2.578200in}{2.639829in}}%
\pgfpathlineto{\pgfqpoint{2.586706in}{2.645249in}}%
\pgfpathlineto{\pgfqpoint{2.595200in}{2.650840in}}%
\pgfpathlineto{\pgfqpoint{2.603683in}{2.656601in}}%
\pgfpathlineto{\pgfqpoint{2.612153in}{2.662529in}}%
\pgfpathlineto{\pgfqpoint{2.598381in}{2.682924in}}%
\pgfpathlineto{\pgfqpoint{2.584598in}{2.703651in}}%
\pgfpathlineto{\pgfqpoint{2.570802in}{2.724711in}}%
\pgfpathlineto{\pgfqpoint{2.556994in}{2.746109in}}%
\pgfpathlineto{\pgfqpoint{2.548492in}{2.740434in}}%
\pgfpathlineto{\pgfqpoint{2.539977in}{2.734934in}}%
\pgfpathlineto{\pgfqpoint{2.531450in}{2.729612in}}%
\pgfpathlineto{\pgfqpoint{2.522910in}{2.724470in}}%
\pgfpathclose%
\pgfusepath{fill}%
\end{pgfscope}%
\begin{pgfscope}%
\pgfpathrectangle{\pgfqpoint{1.150000in}{0.150000in}}{\pgfqpoint{5.700000in}{5.700000in}}%
\pgfusepath{clip}%
\pgfsetbuttcap%
\pgfsetroundjoin%
\definecolor{currentfill}{rgb}{0.172719,0.448791,0.557885}%
\pgfsetfillcolor{currentfill}%
\pgfsetfillopacity{0.800000}%
\pgfsetlinewidth{0.000000pt}%
\definecolor{currentstroke}{rgb}{0.000000,0.000000,0.000000}%
\pgfsetstrokecolor{currentstroke}%
\pgfsetdash{}{0pt}%
\pgfpathmoveto{\pgfqpoint{5.161711in}{2.974144in}}%
\pgfpathlineto{\pgfqpoint{5.175844in}{2.981527in}}%
\pgfpathlineto{\pgfqpoint{5.189994in}{2.989087in}}%
\pgfpathlineto{\pgfqpoint{5.204159in}{2.996824in}}%
\pgfpathlineto{\pgfqpoint{5.218341in}{3.004739in}}%
\pgfpathlineto{\pgfqpoint{5.225779in}{3.010780in}}%
\pgfpathlineto{\pgfqpoint{5.233213in}{3.016865in}}%
\pgfpathlineto{\pgfqpoint{5.240642in}{3.023002in}}%
\pgfpathlineto{\pgfqpoint{5.248066in}{3.029194in}}%
\pgfpathlineto{\pgfqpoint{5.233905in}{3.021778in}}%
\pgfpathlineto{\pgfqpoint{5.219760in}{3.014538in}}%
\pgfpathlineto{\pgfqpoint{5.205630in}{3.007474in}}%
\pgfpathlineto{\pgfqpoint{5.191516in}{3.000587in}}%
\pgfpathlineto{\pgfqpoint{5.184072in}{2.993886in}}%
\pgfpathlineto{\pgfqpoint{5.176623in}{2.987249in}}%
\pgfpathlineto{\pgfqpoint{5.169169in}{2.980670in}}%
\pgfpathlineto{\pgfqpoint{5.161711in}{2.974144in}}%
\pgfpathclose%
\pgfusepath{fill}%
\end{pgfscope}%
\begin{pgfscope}%
\pgfpathrectangle{\pgfqpoint{1.150000in}{0.150000in}}{\pgfqpoint{5.700000in}{5.700000in}}%
\pgfusepath{clip}%
\pgfsetbuttcap%
\pgfsetroundjoin%
\definecolor{currentfill}{rgb}{0.250425,0.274290,0.533103}%
\pgfsetfillcolor{currentfill}%
\pgfsetfillopacity{0.800000}%
\pgfsetlinewidth{0.000000pt}%
\definecolor{currentstroke}{rgb}{0.000000,0.000000,0.000000}%
\pgfsetstrokecolor{currentstroke}%
\pgfsetdash{}{0pt}%
\pgfpathmoveto{\pgfqpoint{4.440899in}{2.496848in}}%
\pgfpathlineto{\pgfqpoint{4.454700in}{2.501406in}}%
\pgfpathlineto{\pgfqpoint{4.468514in}{2.506151in}}%
\pgfpathlineto{\pgfqpoint{4.482340in}{2.511082in}}%
\pgfpathlineto{\pgfqpoint{4.496178in}{2.516199in}}%
\pgfpathlineto{\pgfqpoint{4.503926in}{2.525190in}}%
\pgfpathlineto{\pgfqpoint{4.511669in}{2.534145in}}%
\pgfpathlineto{\pgfqpoint{4.519406in}{2.543067in}}%
\pgfpathlineto{\pgfqpoint{4.527138in}{2.551957in}}%
\pgfpathlineto{\pgfqpoint{4.513308in}{2.547042in}}%
\pgfpathlineto{\pgfqpoint{4.499491in}{2.542312in}}%
\pgfpathlineto{\pgfqpoint{4.485686in}{2.537768in}}%
\pgfpathlineto{\pgfqpoint{4.471893in}{2.533411in}}%
\pgfpathlineto{\pgfqpoint{4.464153in}{2.524308in}}%
\pgfpathlineto{\pgfqpoint{4.456407in}{2.515181in}}%
\pgfpathlineto{\pgfqpoint{4.448656in}{2.506029in}}%
\pgfpathlineto{\pgfqpoint{4.440899in}{2.496848in}}%
\pgfpathclose%
\pgfusepath{fill}%
\end{pgfscope}%
\begin{pgfscope}%
\pgfpathrectangle{\pgfqpoint{1.150000in}{0.150000in}}{\pgfqpoint{5.700000in}{5.700000in}}%
\pgfusepath{clip}%
\pgfsetbuttcap%
\pgfsetroundjoin%
\definecolor{currentfill}{rgb}{0.282656,0.100196,0.422160}%
\pgfsetfillcolor{currentfill}%
\pgfsetfillopacity{0.800000}%
\pgfsetlinewidth{0.000000pt}%
\definecolor{currentstroke}{rgb}{0.000000,0.000000,0.000000}%
\pgfsetstrokecolor{currentstroke}%
\pgfsetdash{}{0pt}%
\pgfpathmoveto{\pgfqpoint{3.015422in}{2.137195in}}%
\pgfpathlineto{\pgfqpoint{3.028996in}{2.125999in}}%
\pgfpathlineto{\pgfqpoint{3.042567in}{2.115053in}}%
\pgfpathlineto{\pgfqpoint{3.056135in}{2.104357in}}%
\pgfpathlineto{\pgfqpoint{3.069702in}{2.093908in}}%
\pgfpathlineto{\pgfqpoint{3.077967in}{2.101736in}}%
\pgfpathlineto{\pgfqpoint{3.086223in}{2.109662in}}%
\pgfpathlineto{\pgfqpoint{3.094472in}{2.117684in}}%
\pgfpathlineto{\pgfqpoint{3.102712in}{2.125801in}}%
\pgfpathlineto{\pgfqpoint{3.089167in}{2.136001in}}%
\pgfpathlineto{\pgfqpoint{3.075620in}{2.146448in}}%
\pgfpathlineto{\pgfqpoint{3.062070in}{2.157144in}}%
\pgfpathlineto{\pgfqpoint{3.048518in}{2.168090in}}%
\pgfpathlineto{\pgfqpoint{3.040257in}{2.160211in}}%
\pgfpathlineto{\pgfqpoint{3.031987in}{2.152434in}}%
\pgfpathlineto{\pgfqpoint{3.023709in}{2.144762in}}%
\pgfpathlineto{\pgfqpoint{3.015422in}{2.137195in}}%
\pgfpathclose%
\pgfusepath{fill}%
\end{pgfscope}%
\begin{pgfscope}%
\pgfpathrectangle{\pgfqpoint{1.150000in}{0.150000in}}{\pgfqpoint{5.700000in}{5.700000in}}%
\pgfusepath{clip}%
\pgfsetbuttcap%
\pgfsetroundjoin%
\definecolor{currentfill}{rgb}{0.279566,0.067836,0.391917}%
\pgfsetfillcolor{currentfill}%
\pgfsetfillopacity{0.800000}%
\pgfsetlinewidth{0.000000pt}%
\definecolor{currentstroke}{rgb}{0.000000,0.000000,0.000000}%
\pgfsetstrokecolor{currentstroke}%
\pgfsetdash{}{0pt}%
\pgfpathmoveto{\pgfqpoint{3.211034in}{2.052909in}}%
\pgfpathlineto{\pgfqpoint{3.224572in}{2.044864in}}%
\pgfpathlineto{\pgfqpoint{3.238111in}{2.037051in}}%
\pgfpathlineto{\pgfqpoint{3.251649in}{2.029469in}}%
\pgfpathlineto{\pgfqpoint{3.265189in}{2.022117in}}%
\pgfpathlineto{\pgfqpoint{3.273366in}{2.031008in}}%
\pgfpathlineto{\pgfqpoint{3.281537in}{2.039966in}}%
\pgfpathlineto{\pgfqpoint{3.289700in}{2.048988in}}%
\pgfpathlineto{\pgfqpoint{3.297857in}{2.058072in}}%
\pgfpathlineto{\pgfqpoint{3.284334in}{2.065210in}}%
\pgfpathlineto{\pgfqpoint{3.270813in}{2.072577in}}%
\pgfpathlineto{\pgfqpoint{3.257291in}{2.080174in}}%
\pgfpathlineto{\pgfqpoint{3.243771in}{2.088004in}}%
\pgfpathlineto{\pgfqpoint{3.235597in}{2.079123in}}%
\pgfpathlineto{\pgfqpoint{3.227416in}{2.070312in}}%
\pgfpathlineto{\pgfqpoint{3.219229in}{2.061573in}}%
\pgfpathlineto{\pgfqpoint{3.211034in}{2.052909in}}%
\pgfpathclose%
\pgfusepath{fill}%
\end{pgfscope}%
\begin{pgfscope}%
\pgfpathrectangle{\pgfqpoint{1.150000in}{0.150000in}}{\pgfqpoint{5.700000in}{5.700000in}}%
\pgfusepath{clip}%
\pgfsetbuttcap%
\pgfsetroundjoin%
\definecolor{currentfill}{rgb}{0.277941,0.056324,0.381191}%
\pgfsetfillcolor{currentfill}%
\pgfsetfillopacity{0.800000}%
\pgfsetlinewidth{0.000000pt}%
\definecolor{currentstroke}{rgb}{0.000000,0.000000,0.000000}%
\pgfsetstrokecolor{currentstroke}%
\pgfsetdash{}{0pt}%
\pgfpathmoveto{\pgfqpoint{3.351961in}{2.031793in}}%
\pgfpathlineto{\pgfqpoint{3.365491in}{2.025784in}}%
\pgfpathlineto{\pgfqpoint{3.379024in}{2.019998in}}%
\pgfpathlineto{\pgfqpoint{3.392558in}{2.014433in}}%
\pgfpathlineto{\pgfqpoint{3.406095in}{2.009088in}}%
\pgfpathlineto{\pgfqpoint{3.414215in}{2.018623in}}%
\pgfpathlineto{\pgfqpoint{3.422329in}{2.028202in}}%
\pgfpathlineto{\pgfqpoint{3.430437in}{2.037823in}}%
\pgfpathlineto{\pgfqpoint{3.438539in}{2.047485in}}%
\pgfpathlineto{\pgfqpoint{3.425016in}{2.052647in}}%
\pgfpathlineto{\pgfqpoint{3.411496in}{2.058030in}}%
\pgfpathlineto{\pgfqpoint{3.397978in}{2.063633in}}%
\pgfpathlineto{\pgfqpoint{3.384462in}{2.069459in}}%
\pgfpathlineto{\pgfqpoint{3.376346in}{2.059968in}}%
\pgfpathlineto{\pgfqpoint{3.368224in}{2.050526in}}%
\pgfpathlineto{\pgfqpoint{3.360096in}{2.041134in}}%
\pgfpathlineto{\pgfqpoint{3.351961in}{2.031793in}}%
\pgfpathclose%
\pgfusepath{fill}%
\end{pgfscope}%
\begin{pgfscope}%
\pgfpathrectangle{\pgfqpoint{1.150000in}{0.150000in}}{\pgfqpoint{5.700000in}{5.700000in}}%
\pgfusepath{clip}%
\pgfsetbuttcap%
\pgfsetroundjoin%
\definecolor{currentfill}{rgb}{0.280267,0.073417,0.397163}%
\pgfsetfillcolor{currentfill}%
\pgfsetfillopacity{0.800000}%
\pgfsetlinewidth{0.000000pt}%
\definecolor{currentstroke}{rgb}{0.000000,0.000000,0.000000}%
\pgfsetstrokecolor{currentstroke}%
\pgfsetdash{}{0pt}%
\pgfpathmoveto{\pgfqpoint{3.579090in}{2.054213in}}%
\pgfpathlineto{\pgfqpoint{3.592633in}{2.051134in}}%
\pgfpathlineto{\pgfqpoint{3.606182in}{2.048265in}}%
\pgfpathlineto{\pgfqpoint{3.619735in}{2.045605in}}%
\pgfpathlineto{\pgfqpoint{3.633293in}{2.043153in}}%
\pgfpathlineto{\pgfqpoint{3.641331in}{2.053395in}}%
\pgfpathlineto{\pgfqpoint{3.649363in}{2.063647in}}%
\pgfpathlineto{\pgfqpoint{3.657390in}{2.073909in}}%
\pgfpathlineto{\pgfqpoint{3.665411in}{2.084180in}}%
\pgfpathlineto{\pgfqpoint{3.651863in}{2.086512in}}%
\pgfpathlineto{\pgfqpoint{3.638321in}{2.089053in}}%
\pgfpathlineto{\pgfqpoint{3.624783in}{2.091803in}}%
\pgfpathlineto{\pgfqpoint{3.611250in}{2.094763in}}%
\pgfpathlineto{\pgfqpoint{3.603218in}{2.084599in}}%
\pgfpathlineto{\pgfqpoint{3.595181in}{2.074452in}}%
\pgfpathlineto{\pgfqpoint{3.587138in}{2.064323in}}%
\pgfpathlineto{\pgfqpoint{3.579090in}{2.054213in}}%
\pgfpathclose%
\pgfusepath{fill}%
\end{pgfscope}%
\begin{pgfscope}%
\pgfpathrectangle{\pgfqpoint{1.150000in}{0.150000in}}{\pgfqpoint{5.700000in}{5.700000in}}%
\pgfusepath{clip}%
\pgfsetbuttcap%
\pgfsetroundjoin%
\definecolor{currentfill}{rgb}{0.165117,0.467423,0.558141}%
\pgfsetfillcolor{currentfill}%
\pgfsetfillopacity{0.800000}%
\pgfsetlinewidth{0.000000pt}%
\definecolor{currentstroke}{rgb}{0.000000,0.000000,0.000000}%
\pgfsetstrokecolor{currentstroke}%
\pgfsetdash{}{0pt}%
\pgfpathmoveto{\pgfqpoint{5.248066in}{3.029194in}}%
\pgfpathlineto{\pgfqpoint{5.262244in}{3.036787in}}%
\pgfpathlineto{\pgfqpoint{5.276438in}{3.044557in}}%
\pgfpathlineto{\pgfqpoint{5.290648in}{3.052503in}}%
\pgfpathlineto{\pgfqpoint{5.304874in}{3.060625in}}%
\pgfpathlineto{\pgfqpoint{5.312273in}{3.066361in}}%
\pgfpathlineto{\pgfqpoint{5.319668in}{3.072157in}}%
\pgfpathlineto{\pgfqpoint{5.327058in}{3.078019in}}%
\pgfpathlineto{\pgfqpoint{5.334443in}{3.083953in}}%
\pgfpathlineto{\pgfqpoint{5.320239in}{3.076362in}}%
\pgfpathlineto{\pgfqpoint{5.306050in}{3.068947in}}%
\pgfpathlineto{\pgfqpoint{5.291878in}{3.061707in}}%
\pgfpathlineto{\pgfqpoint{5.277722in}{3.054643in}}%
\pgfpathlineto{\pgfqpoint{5.270314in}{3.048167in}}%
\pgfpathlineto{\pgfqpoint{5.262902in}{3.041771in}}%
\pgfpathlineto{\pgfqpoint{5.255487in}{3.035449in}}%
\pgfpathlineto{\pgfqpoint{5.248066in}{3.029194in}}%
\pgfpathclose%
\pgfusepath{fill}%
\end{pgfscope}%
\begin{pgfscope}%
\pgfpathrectangle{\pgfqpoint{1.150000in}{0.150000in}}{\pgfqpoint{5.700000in}{5.700000in}}%
\pgfusepath{clip}%
\pgfsetbuttcap%
\pgfsetroundjoin%
\definecolor{currentfill}{rgb}{0.241237,0.296485,0.539709}%
\pgfsetfillcolor{currentfill}%
\pgfsetfillopacity{0.800000}%
\pgfsetlinewidth{0.000000pt}%
\definecolor{currentstroke}{rgb}{0.000000,0.000000,0.000000}%
\pgfsetstrokecolor{currentstroke}%
\pgfsetdash{}{0pt}%
\pgfpathmoveto{\pgfqpoint{4.527138in}{2.551957in}}%
\pgfpathlineto{\pgfqpoint{4.540979in}{2.557058in}}%
\pgfpathlineto{\pgfqpoint{4.554834in}{2.562345in}}%
\pgfpathlineto{\pgfqpoint{4.568701in}{2.567817in}}%
\pgfpathlineto{\pgfqpoint{4.582581in}{2.573474in}}%
\pgfpathlineto{\pgfqpoint{4.590298in}{2.582114in}}%
\pgfpathlineto{\pgfqpoint{4.598009in}{2.590720in}}%
\pgfpathlineto{\pgfqpoint{4.605714in}{2.599296in}}%
\pgfpathlineto{\pgfqpoint{4.613413in}{2.607844in}}%
\pgfpathlineto{\pgfqpoint{4.599543in}{2.602422in}}%
\pgfpathlineto{\pgfqpoint{4.585685in}{2.597184in}}%
\pgfpathlineto{\pgfqpoint{4.571840in}{2.592131in}}%
\pgfpathlineto{\pgfqpoint{4.558008in}{2.587263in}}%
\pgfpathlineto{\pgfqpoint{4.550299in}{2.578470in}}%
\pgfpathlineto{\pgfqpoint{4.542584in}{2.569656in}}%
\pgfpathlineto{\pgfqpoint{4.534864in}{2.560819in}}%
\pgfpathlineto{\pgfqpoint{4.527138in}{2.551957in}}%
\pgfpathclose%
\pgfusepath{fill}%
\end{pgfscope}%
\begin{pgfscope}%
\pgfpathrectangle{\pgfqpoint{1.150000in}{0.150000in}}{\pgfqpoint{5.700000in}{5.700000in}}%
\pgfusepath{clip}%
\pgfsetbuttcap%
\pgfsetroundjoin%
\definecolor{currentfill}{rgb}{0.157729,0.485932,0.558013}%
\pgfsetfillcolor{currentfill}%
\pgfsetfillopacity{0.800000}%
\pgfsetlinewidth{0.000000pt}%
\definecolor{currentstroke}{rgb}{0.000000,0.000000,0.000000}%
\pgfsetstrokecolor{currentstroke}%
\pgfsetdash{}{0pt}%
\pgfpathmoveto{\pgfqpoint{5.334443in}{3.083953in}}%
\pgfpathlineto{\pgfqpoint{5.348665in}{3.091719in}}%
\pgfpathlineto{\pgfqpoint{5.362902in}{3.099662in}}%
\pgfpathlineto{\pgfqpoint{5.377157in}{3.107779in}}%
\pgfpathlineto{\pgfqpoint{5.391428in}{3.116073in}}%
\pgfpathlineto{\pgfqpoint{5.398787in}{3.121533in}}%
\pgfpathlineto{\pgfqpoint{5.406142in}{3.127069in}}%
\pgfpathlineto{\pgfqpoint{5.413493in}{3.132689in}}%
\pgfpathlineto{\pgfqpoint{5.420840in}{3.138398in}}%
\pgfpathlineto{\pgfqpoint{5.406592in}{3.130669in}}%
\pgfpathlineto{\pgfqpoint{5.392362in}{3.123115in}}%
\pgfpathlineto{\pgfqpoint{5.378147in}{3.115736in}}%
\pgfpathlineto{\pgfqpoint{5.363949in}{3.108532in}}%
\pgfpathlineto{\pgfqpoint{5.356578in}{3.102248in}}%
\pgfpathlineto{\pgfqpoint{5.349204in}{3.096061in}}%
\pgfpathlineto{\pgfqpoint{5.341825in}{3.089965in}}%
\pgfpathlineto{\pgfqpoint{5.334443in}{3.083953in}}%
\pgfpathclose%
\pgfusepath{fill}%
\end{pgfscope}%
\begin{pgfscope}%
\pgfpathrectangle{\pgfqpoint{1.150000in}{0.150000in}}{\pgfqpoint{5.700000in}{5.700000in}}%
\pgfusepath{clip}%
\pgfsetbuttcap%
\pgfsetroundjoin%
\definecolor{currentfill}{rgb}{0.206756,0.371758,0.553117}%
\pgfsetfillcolor{currentfill}%
\pgfsetfillopacity{0.800000}%
\pgfsetlinewidth{0.000000pt}%
\definecolor{currentstroke}{rgb}{0.000000,0.000000,0.000000}%
\pgfsetstrokecolor{currentstroke}%
\pgfsetdash{}{0pt}%
\pgfpathmoveto{\pgfqpoint{2.467412in}{2.814575in}}%
\pgfpathlineto{\pgfqpoint{2.481307in}{2.791524in}}%
\pgfpathlineto{\pgfqpoint{2.495189in}{2.768826in}}%
\pgfpathlineto{\pgfqpoint{2.509056in}{2.746476in}}%
\pgfpathlineto{\pgfqpoint{2.522910in}{2.724470in}}%
\pgfpathlineto{\pgfqpoint{2.531450in}{2.729612in}}%
\pgfpathlineto{\pgfqpoint{2.539977in}{2.734934in}}%
\pgfpathlineto{\pgfqpoint{2.548492in}{2.740434in}}%
\pgfpathlineto{\pgfqpoint{2.556994in}{2.746109in}}%
\pgfpathlineto{\pgfqpoint{2.543174in}{2.767847in}}%
\pgfpathlineto{\pgfqpoint{2.529340in}{2.789930in}}%
\pgfpathlineto{\pgfqpoint{2.515494in}{2.812360in}}%
\pgfpathlineto{\pgfqpoint{2.501633in}{2.835141in}}%
\pgfpathlineto{\pgfqpoint{2.493097in}{2.829721in}}%
\pgfpathlineto{\pgfqpoint{2.484548in}{2.824486in}}%
\pgfpathlineto{\pgfqpoint{2.475987in}{2.819436in}}%
\pgfpathlineto{\pgfqpoint{2.467412in}{2.814575in}}%
\pgfpathclose%
\pgfusepath{fill}%
\end{pgfscope}%
\begin{pgfscope}%
\pgfpathrectangle{\pgfqpoint{1.150000in}{0.150000in}}{\pgfqpoint{5.700000in}{5.700000in}}%
\pgfusepath{clip}%
\pgfsetbuttcap%
\pgfsetroundjoin%
\definecolor{currentfill}{rgb}{0.231674,0.318106,0.544834}%
\pgfsetfillcolor{currentfill}%
\pgfsetfillopacity{0.800000}%
\pgfsetlinewidth{0.000000pt}%
\definecolor{currentstroke}{rgb}{0.000000,0.000000,0.000000}%
\pgfsetstrokecolor{currentstroke}%
\pgfsetdash{}{0pt}%
\pgfpathmoveto{\pgfqpoint{4.613413in}{2.607844in}}%
\pgfpathlineto{\pgfqpoint{4.627297in}{2.613451in}}%
\pgfpathlineto{\pgfqpoint{4.641193in}{2.619241in}}%
\pgfpathlineto{\pgfqpoint{4.655103in}{2.625216in}}%
\pgfpathlineto{\pgfqpoint{4.669027in}{2.631374in}}%
\pgfpathlineto{\pgfqpoint{4.676710in}{2.639643in}}%
\pgfpathlineto{\pgfqpoint{4.684388in}{2.647882in}}%
\pgfpathlineto{\pgfqpoint{4.692061in}{2.656094in}}%
\pgfpathlineto{\pgfqpoint{4.699727in}{2.664284in}}%
\pgfpathlineto{\pgfqpoint{4.685814in}{2.658393in}}%
\pgfpathlineto{\pgfqpoint{4.671915in}{2.652686in}}%
\pgfpathlineto{\pgfqpoint{4.658028in}{2.647162in}}%
\pgfpathlineto{\pgfqpoint{4.644155in}{2.641821in}}%
\pgfpathlineto{\pgfqpoint{4.636478in}{2.633353in}}%
\pgfpathlineto{\pgfqpoint{4.628796in}{2.624870in}}%
\pgfpathlineto{\pgfqpoint{4.621107in}{2.616368in}}%
\pgfpathlineto{\pgfqpoint{4.613413in}{2.607844in}}%
\pgfpathclose%
\pgfusepath{fill}%
\end{pgfscope}%
\begin{pgfscope}%
\pgfpathrectangle{\pgfqpoint{1.150000in}{0.150000in}}{\pgfqpoint{5.700000in}{5.700000in}}%
\pgfusepath{clip}%
\pgfsetbuttcap%
\pgfsetroundjoin%
\definecolor{currentfill}{rgb}{0.281446,0.084320,0.407414}%
\pgfsetfillcolor{currentfill}%
\pgfsetfillopacity{0.800000}%
\pgfsetlinewidth{0.000000pt}%
\definecolor{currentstroke}{rgb}{0.000000,0.000000,0.000000}%
\pgfsetstrokecolor{currentstroke}%
\pgfsetdash{}{0pt}%
\pgfpathmoveto{\pgfqpoint{3.069702in}{2.093908in}}%
\pgfpathlineto{\pgfqpoint{3.083266in}{2.083704in}}%
\pgfpathlineto{\pgfqpoint{3.096829in}{2.073745in}}%
\pgfpathlineto{\pgfqpoint{3.110390in}{2.064029in}}%
\pgfpathlineto{\pgfqpoint{3.123950in}{2.054553in}}%
\pgfpathlineto{\pgfqpoint{3.132194in}{2.062642in}}%
\pgfpathlineto{\pgfqpoint{3.140430in}{2.070820in}}%
\pgfpathlineto{\pgfqpoint{3.148659in}{2.079087in}}%
\pgfpathlineto{\pgfqpoint{3.156880in}{2.087440in}}%
\pgfpathlineto{\pgfqpoint{3.143340in}{2.096667in}}%
\pgfpathlineto{\pgfqpoint{3.129798in}{2.106136in}}%
\pgfpathlineto{\pgfqpoint{3.116256in}{2.115846in}}%
\pgfpathlineto{\pgfqpoint{3.102712in}{2.125801in}}%
\pgfpathlineto{\pgfqpoint{3.094472in}{2.117684in}}%
\pgfpathlineto{\pgfqpoint{3.086223in}{2.109662in}}%
\pgfpathlineto{\pgfqpoint{3.077967in}{2.101736in}}%
\pgfpathlineto{\pgfqpoint{3.069702in}{2.093908in}}%
\pgfpathclose%
\pgfusepath{fill}%
\end{pgfscope}%
\begin{pgfscope}%
\pgfpathrectangle{\pgfqpoint{1.150000in}{0.150000in}}{\pgfqpoint{5.700000in}{5.700000in}}%
\pgfusepath{clip}%
\pgfsetbuttcap%
\pgfsetroundjoin%
\definecolor{currentfill}{rgb}{0.150476,0.504369,0.557430}%
\pgfsetfillcolor{currentfill}%
\pgfsetfillopacity{0.800000}%
\pgfsetlinewidth{0.000000pt}%
\definecolor{currentstroke}{rgb}{0.000000,0.000000,0.000000}%
\pgfsetstrokecolor{currentstroke}%
\pgfsetdash{}{0pt}%
\pgfpathmoveto{\pgfqpoint{5.420840in}{3.138398in}}%
\pgfpathlineto{\pgfqpoint{5.435104in}{3.146301in}}%
\pgfpathlineto{\pgfqpoint{5.449385in}{3.154379in}}%
\pgfpathlineto{\pgfqpoint{5.463684in}{3.162632in}}%
\pgfpathlineto{\pgfqpoint{5.477999in}{3.171060in}}%
\pgfpathlineto{\pgfqpoint{5.485318in}{3.176279in}}%
\pgfpathlineto{\pgfqpoint{5.492633in}{3.181593in}}%
\pgfpathlineto{\pgfqpoint{5.499945in}{3.187008in}}%
\pgfpathlineto{\pgfqpoint{5.507254in}{3.192531in}}%
\pgfpathlineto{\pgfqpoint{5.492965in}{3.184701in}}%
\pgfpathlineto{\pgfqpoint{5.478692in}{3.177045in}}%
\pgfpathlineto{\pgfqpoint{5.464436in}{3.169563in}}%
\pgfpathlineto{\pgfqpoint{5.450197in}{3.162255in}}%
\pgfpathlineto{\pgfqpoint{5.442862in}{3.156124in}}%
\pgfpathlineto{\pgfqpoint{5.435524in}{3.150109in}}%
\pgfpathlineto{\pgfqpoint{5.428184in}{3.144202in}}%
\pgfpathlineto{\pgfqpoint{5.420840in}{3.138398in}}%
\pgfpathclose%
\pgfusepath{fill}%
\end{pgfscope}%
\begin{pgfscope}%
\pgfpathrectangle{\pgfqpoint{1.150000in}{0.150000in}}{\pgfqpoint{5.700000in}{5.700000in}}%
\pgfusepath{clip}%
\pgfsetbuttcap%
\pgfsetroundjoin%
\definecolor{currentfill}{rgb}{0.278791,0.062145,0.386592}%
\pgfsetfillcolor{currentfill}%
\pgfsetfillopacity{0.800000}%
\pgfsetlinewidth{0.000000pt}%
\definecolor{currentstroke}{rgb}{0.000000,0.000000,0.000000}%
\pgfsetstrokecolor{currentstroke}%
\pgfsetdash{}{0pt}%
\pgfpathmoveto{\pgfqpoint{3.492661in}{2.029012in}}%
\pgfpathlineto{\pgfqpoint{3.506200in}{2.024933in}}%
\pgfpathlineto{\pgfqpoint{3.519743in}{2.021068in}}%
\pgfpathlineto{\pgfqpoint{3.533290in}{2.017415in}}%
\pgfpathlineto{\pgfqpoint{3.546840in}{2.013975in}}%
\pgfpathlineto{\pgfqpoint{3.554911in}{2.024001in}}%
\pgfpathlineto{\pgfqpoint{3.562976in}{2.034051in}}%
\pgfpathlineto{\pgfqpoint{3.571036in}{2.044121in}}%
\pgfpathlineto{\pgfqpoint{3.579090in}{2.054213in}}%
\pgfpathlineto{\pgfqpoint{3.565550in}{2.057503in}}%
\pgfpathlineto{\pgfqpoint{3.552016in}{2.061004in}}%
\pgfpathlineto{\pgfqpoint{3.538485in}{2.064719in}}%
\pgfpathlineto{\pgfqpoint{3.524958in}{2.068647in}}%
\pgfpathlineto{\pgfqpoint{3.516892in}{2.058695in}}%
\pgfpathlineto{\pgfqpoint{3.508821in}{2.048771in}}%
\pgfpathlineto{\pgfqpoint{3.500744in}{2.038876in}}%
\pgfpathlineto{\pgfqpoint{3.492661in}{2.029012in}}%
\pgfpathclose%
\pgfusepath{fill}%
\end{pgfscope}%
\begin{pgfscope}%
\pgfpathrectangle{\pgfqpoint{1.150000in}{0.150000in}}{\pgfqpoint{5.700000in}{5.700000in}}%
\pgfusepath{clip}%
\pgfsetbuttcap%
\pgfsetroundjoin%
\definecolor{currentfill}{rgb}{0.220057,0.343307,0.549413}%
\pgfsetfillcolor{currentfill}%
\pgfsetfillopacity{0.800000}%
\pgfsetlinewidth{0.000000pt}%
\definecolor{currentstroke}{rgb}{0.000000,0.000000,0.000000}%
\pgfsetstrokecolor{currentstroke}%
\pgfsetdash{}{0pt}%
\pgfpathmoveto{\pgfqpoint{4.699727in}{2.664284in}}%
\pgfpathlineto{\pgfqpoint{4.713653in}{2.670358in}}%
\pgfpathlineto{\pgfqpoint{4.727593in}{2.676615in}}%
\pgfpathlineto{\pgfqpoint{4.741547in}{2.683055in}}%
\pgfpathlineto{\pgfqpoint{4.755515in}{2.689677in}}%
\pgfpathlineto{\pgfqpoint{4.763165in}{2.697559in}}%
\pgfpathlineto{\pgfqpoint{4.770809in}{2.705416in}}%
\pgfpathlineto{\pgfqpoint{4.778446in}{2.713254in}}%
\pgfpathlineto{\pgfqpoint{4.786078in}{2.721074in}}%
\pgfpathlineto{\pgfqpoint{4.772122in}{2.714752in}}%
\pgfpathlineto{\pgfqpoint{4.758180in}{2.708613in}}%
\pgfpathlineto{\pgfqpoint{4.744251in}{2.702656in}}%
\pgfpathlineto{\pgfqpoint{4.730336in}{2.696881in}}%
\pgfpathlineto{\pgfqpoint{4.722692in}{2.688749in}}%
\pgfpathlineto{\pgfqpoint{4.715043in}{2.680608in}}%
\pgfpathlineto{\pgfqpoint{4.707388in}{2.672454in}}%
\pgfpathlineto{\pgfqpoint{4.699727in}{2.664284in}}%
\pgfpathclose%
\pgfusepath{fill}%
\end{pgfscope}%
\begin{pgfscope}%
\pgfpathrectangle{\pgfqpoint{1.150000in}{0.150000in}}{\pgfqpoint{5.700000in}{5.700000in}}%
\pgfusepath{clip}%
\pgfsetbuttcap%
\pgfsetroundjoin%
\definecolor{currentfill}{rgb}{0.143343,0.522773,0.556295}%
\pgfsetfillcolor{currentfill}%
\pgfsetfillopacity{0.800000}%
\pgfsetlinewidth{0.000000pt}%
\definecolor{currentstroke}{rgb}{0.000000,0.000000,0.000000}%
\pgfsetstrokecolor{currentstroke}%
\pgfsetdash{}{0pt}%
\pgfpathmoveto{\pgfqpoint{5.507254in}{3.192531in}}%
\pgfpathlineto{\pgfqpoint{5.521561in}{3.200534in}}%
\pgfpathlineto{\pgfqpoint{5.535885in}{3.208712in}}%
\pgfpathlineto{\pgfqpoint{5.550227in}{3.217064in}}%
\pgfpathlineto{\pgfqpoint{5.564586in}{3.225590in}}%
\pgfpathlineto{\pgfqpoint{5.571865in}{3.230609in}}%
\pgfpathlineto{\pgfqpoint{5.579141in}{3.235741in}}%
\pgfpathlineto{\pgfqpoint{5.586414in}{3.240995in}}%
\pgfpathlineto{\pgfqpoint{5.593686in}{3.246377in}}%
\pgfpathlineto{\pgfqpoint{5.579355in}{3.238482in}}%
\pgfpathlineto{\pgfqpoint{5.565041in}{3.230760in}}%
\pgfpathlineto{\pgfqpoint{5.550745in}{3.223212in}}%
\pgfpathlineto{\pgfqpoint{5.536465in}{3.215836in}}%
\pgfpathlineto{\pgfqpoint{5.529166in}{3.209814in}}%
\pgfpathlineto{\pgfqpoint{5.521864in}{3.203927in}}%
\pgfpathlineto{\pgfqpoint{5.514560in}{3.198168in}}%
\pgfpathlineto{\pgfqpoint{5.507254in}{3.192531in}}%
\pgfpathclose%
\pgfusepath{fill}%
\end{pgfscope}%
\begin{pgfscope}%
\pgfpathrectangle{\pgfqpoint{1.150000in}{0.150000in}}{\pgfqpoint{5.700000in}{5.700000in}}%
\pgfusepath{clip}%
\pgfsetbuttcap%
\pgfsetroundjoin%
\definecolor{currentfill}{rgb}{0.270595,0.214069,0.507052}%
\pgfsetfillcolor{currentfill}%
\pgfsetfillopacity{0.800000}%
\pgfsetlinewidth{0.000000pt}%
\definecolor{currentstroke}{rgb}{0.000000,0.000000,0.000000}%
\pgfsetstrokecolor{currentstroke}%
\pgfsetdash{}{0pt}%
\pgfpathmoveto{\pgfqpoint{2.709267in}{2.393228in}}%
\pgfpathlineto{\pgfqpoint{2.722974in}{2.376319in}}%
\pgfpathlineto{\pgfqpoint{2.736672in}{2.359703in}}%
\pgfpathlineto{\pgfqpoint{2.750364in}{2.343377in}}%
\pgfpathlineto{\pgfqpoint{2.764047in}{2.327339in}}%
\pgfpathlineto{\pgfqpoint{2.772475in}{2.333269in}}%
\pgfpathlineto{\pgfqpoint{2.780892in}{2.339347in}}%
\pgfpathlineto{\pgfqpoint{2.789299in}{2.345571in}}%
\pgfpathlineto{\pgfqpoint{2.797695in}{2.351938in}}%
\pgfpathlineto{\pgfqpoint{2.784040in}{2.367688in}}%
\pgfpathlineto{\pgfqpoint{2.770378in}{2.383725in}}%
\pgfpathlineto{\pgfqpoint{2.756709in}{2.400052in}}%
\pgfpathlineto{\pgfqpoint{2.743031in}{2.416671in}}%
\pgfpathlineto{\pgfqpoint{2.734607in}{2.410580in}}%
\pgfpathlineto{\pgfqpoint{2.726171in}{2.404642in}}%
\pgfpathlineto{\pgfqpoint{2.717724in}{2.398857in}}%
\pgfpathlineto{\pgfqpoint{2.709267in}{2.393228in}}%
\pgfpathclose%
\pgfusepath{fill}%
\end{pgfscope}%
\begin{pgfscope}%
\pgfpathrectangle{\pgfqpoint{1.150000in}{0.150000in}}{\pgfqpoint{5.700000in}{5.700000in}}%
\pgfusepath{clip}%
\pgfsetbuttcap%
\pgfsetroundjoin%
\definecolor{currentfill}{rgb}{0.277134,0.185228,0.489898}%
\pgfsetfillcolor{currentfill}%
\pgfsetfillopacity{0.800000}%
\pgfsetlinewidth{0.000000pt}%
\definecolor{currentstroke}{rgb}{0.000000,0.000000,0.000000}%
\pgfsetstrokecolor{currentstroke}%
\pgfsetdash{}{0pt}%
\pgfpathmoveto{\pgfqpoint{2.764047in}{2.327339in}}%
\pgfpathlineto{\pgfqpoint{2.777724in}{2.311585in}}%
\pgfpathlineto{\pgfqpoint{2.791393in}{2.296115in}}%
\pgfpathlineto{\pgfqpoint{2.805056in}{2.280926in}}%
\pgfpathlineto{\pgfqpoint{2.818713in}{2.266015in}}%
\pgfpathlineto{\pgfqpoint{2.827112in}{2.272244in}}%
\pgfpathlineto{\pgfqpoint{2.835501in}{2.278614in}}%
\pgfpathlineto{\pgfqpoint{2.843880in}{2.285122in}}%
\pgfpathlineto{\pgfqpoint{2.852250in}{2.291764in}}%
\pgfpathlineto{\pgfqpoint{2.838620in}{2.306388in}}%
\pgfpathlineto{\pgfqpoint{2.824985in}{2.321291in}}%
\pgfpathlineto{\pgfqpoint{2.811343in}{2.336473in}}%
\pgfpathlineto{\pgfqpoint{2.797695in}{2.351938in}}%
\pgfpathlineto{\pgfqpoint{2.789299in}{2.345571in}}%
\pgfpathlineto{\pgfqpoint{2.780892in}{2.339347in}}%
\pgfpathlineto{\pgfqpoint{2.772475in}{2.333269in}}%
\pgfpathlineto{\pgfqpoint{2.764047in}{2.327339in}}%
\pgfpathclose%
\pgfusepath{fill}%
\end{pgfscope}%
\begin{pgfscope}%
\pgfpathrectangle{\pgfqpoint{1.150000in}{0.150000in}}{\pgfqpoint{5.700000in}{5.700000in}}%
\pgfusepath{clip}%
\pgfsetbuttcap%
\pgfsetroundjoin%
\definecolor{currentfill}{rgb}{0.136408,0.541173,0.554483}%
\pgfsetfillcolor{currentfill}%
\pgfsetfillopacity{0.800000}%
\pgfsetlinewidth{0.000000pt}%
\definecolor{currentstroke}{rgb}{0.000000,0.000000,0.000000}%
\pgfsetstrokecolor{currentstroke}%
\pgfsetdash{}{0pt}%
\pgfpathmoveto{\pgfqpoint{5.593686in}{3.246377in}}%
\pgfpathlineto{\pgfqpoint{5.608034in}{3.254445in}}%
\pgfpathlineto{\pgfqpoint{5.622400in}{3.262686in}}%
\pgfpathlineto{\pgfqpoint{5.636784in}{3.271100in}}%
\pgfpathlineto{\pgfqpoint{5.651185in}{3.279688in}}%
\pgfpathlineto{\pgfqpoint{5.658425in}{3.284552in}}%
\pgfpathlineto{\pgfqpoint{5.665663in}{3.289552in}}%
\pgfpathlineto{\pgfqpoint{5.672899in}{3.294694in}}%
\pgfpathlineto{\pgfqpoint{5.680134in}{3.299985in}}%
\pgfpathlineto{\pgfqpoint{5.665763in}{3.292061in}}%
\pgfpathlineto{\pgfqpoint{5.651409in}{3.284310in}}%
\pgfpathlineto{\pgfqpoint{5.637073in}{3.276731in}}%
\pgfpathlineto{\pgfqpoint{5.622754in}{3.269324in}}%
\pgfpathlineto{\pgfqpoint{5.615489in}{3.263360in}}%
\pgfpathlineto{\pgfqpoint{5.608223in}{3.257552in}}%
\pgfpathlineto{\pgfqpoint{5.600955in}{3.251893in}}%
\pgfpathlineto{\pgfqpoint{5.593686in}{3.246377in}}%
\pgfpathclose%
\pgfusepath{fill}%
\end{pgfscope}%
\begin{pgfscope}%
\pgfpathrectangle{\pgfqpoint{1.150000in}{0.150000in}}{\pgfqpoint{5.700000in}{5.700000in}}%
\pgfusepath{clip}%
\pgfsetbuttcap%
\pgfsetroundjoin%
\definecolor{currentfill}{rgb}{0.277941,0.056324,0.381191}%
\pgfsetfillcolor{currentfill}%
\pgfsetfillopacity{0.800000}%
\pgfsetlinewidth{0.000000pt}%
\definecolor{currentstroke}{rgb}{0.000000,0.000000,0.000000}%
\pgfsetstrokecolor{currentstroke}%
\pgfsetdash{}{0pt}%
\pgfpathmoveto{\pgfqpoint{3.265189in}{2.022117in}}%
\pgfpathlineto{\pgfqpoint{3.278729in}{2.014993in}}%
\pgfpathlineto{\pgfqpoint{3.292271in}{2.008096in}}%
\pgfpathlineto{\pgfqpoint{3.305813in}{2.001425in}}%
\pgfpathlineto{\pgfqpoint{3.319357in}{1.994978in}}%
\pgfpathlineto{\pgfqpoint{3.327518in}{2.004096in}}%
\pgfpathlineto{\pgfqpoint{3.335672in}{2.013273in}}%
\pgfpathlineto{\pgfqpoint{3.343820in}{2.022505in}}%
\pgfpathlineto{\pgfqpoint{3.351961in}{2.031793in}}%
\pgfpathlineto{\pgfqpoint{3.338433in}{2.038025in}}%
\pgfpathlineto{\pgfqpoint{3.324906in}{2.044481in}}%
\pgfpathlineto{\pgfqpoint{3.311381in}{2.051163in}}%
\pgfpathlineto{\pgfqpoint{3.297857in}{2.058072in}}%
\pgfpathlineto{\pgfqpoint{3.289700in}{2.048988in}}%
\pgfpathlineto{\pgfqpoint{3.281537in}{2.039966in}}%
\pgfpathlineto{\pgfqpoint{3.273366in}{2.031008in}}%
\pgfpathlineto{\pgfqpoint{3.265189in}{2.022117in}}%
\pgfpathclose%
\pgfusepath{fill}%
\end{pgfscope}%
\begin{pgfscope}%
\pgfpathrectangle{\pgfqpoint{1.150000in}{0.150000in}}{\pgfqpoint{5.700000in}{5.700000in}}%
\pgfusepath{clip}%
\pgfsetbuttcap%
\pgfsetroundjoin%
\definecolor{currentfill}{rgb}{0.262138,0.242286,0.520837}%
\pgfsetfillcolor{currentfill}%
\pgfsetfillopacity{0.800000}%
\pgfsetlinewidth{0.000000pt}%
\definecolor{currentstroke}{rgb}{0.000000,0.000000,0.000000}%
\pgfsetstrokecolor{currentstroke}%
\pgfsetdash{}{0pt}%
\pgfpathmoveto{\pgfqpoint{2.654354in}{2.463840in}}%
\pgfpathlineto{\pgfqpoint{2.668095in}{2.445735in}}%
\pgfpathlineto{\pgfqpoint{2.681828in}{2.427933in}}%
\pgfpathlineto{\pgfqpoint{2.695552in}{2.410432in}}%
\pgfpathlineto{\pgfqpoint{2.709267in}{2.393228in}}%
\pgfpathlineto{\pgfqpoint{2.717724in}{2.398857in}}%
\pgfpathlineto{\pgfqpoint{2.726171in}{2.404642in}}%
\pgfpathlineto{\pgfqpoint{2.734607in}{2.410580in}}%
\pgfpathlineto{\pgfqpoint{2.743031in}{2.416671in}}%
\pgfpathlineto{\pgfqpoint{2.729346in}{2.433584in}}%
\pgfpathlineto{\pgfqpoint{2.715653in}{2.450794in}}%
\pgfpathlineto{\pgfqpoint{2.701951in}{2.468304in}}%
\pgfpathlineto{\pgfqpoint{2.688240in}{2.486116in}}%
\pgfpathlineto{\pgfqpoint{2.679786in}{2.480305in}}%
\pgfpathlineto{\pgfqpoint{2.671320in}{2.474653in}}%
\pgfpathlineto{\pgfqpoint{2.662843in}{2.469164in}}%
\pgfpathlineto{\pgfqpoint{2.654354in}{2.463840in}}%
\pgfpathclose%
\pgfusepath{fill}%
\end{pgfscope}%
\begin{pgfscope}%
\pgfpathrectangle{\pgfqpoint{1.150000in}{0.150000in}}{\pgfqpoint{5.700000in}{5.700000in}}%
\pgfusepath{clip}%
\pgfsetbuttcap%
\pgfsetroundjoin%
\definecolor{currentfill}{rgb}{0.280868,0.160771,0.472899}%
\pgfsetfillcolor{currentfill}%
\pgfsetfillopacity{0.800000}%
\pgfsetlinewidth{0.000000pt}%
\definecolor{currentstroke}{rgb}{0.000000,0.000000,0.000000}%
\pgfsetstrokecolor{currentstroke}%
\pgfsetdash{}{0pt}%
\pgfpathmoveto{\pgfqpoint{2.818713in}{2.266015in}}%
\pgfpathlineto{\pgfqpoint{2.832363in}{2.251380in}}%
\pgfpathlineto{\pgfqpoint{2.846008in}{2.237019in}}%
\pgfpathlineto{\pgfqpoint{2.859647in}{2.222930in}}%
\pgfpathlineto{\pgfqpoint{2.873281in}{2.209111in}}%
\pgfpathlineto{\pgfqpoint{2.881653in}{2.215639in}}%
\pgfpathlineto{\pgfqpoint{2.890015in}{2.222298in}}%
\pgfpathlineto{\pgfqpoint{2.898368in}{2.229087in}}%
\pgfpathlineto{\pgfqpoint{2.906711in}{2.236003in}}%
\pgfpathlineto{\pgfqpoint{2.893104in}{2.249537in}}%
\pgfpathlineto{\pgfqpoint{2.879491in}{2.263340in}}%
\pgfpathlineto{\pgfqpoint{2.865873in}{2.277415in}}%
\pgfpathlineto{\pgfqpoint{2.852250in}{2.291764in}}%
\pgfpathlineto{\pgfqpoint{2.843880in}{2.285122in}}%
\pgfpathlineto{\pgfqpoint{2.835501in}{2.278614in}}%
\pgfpathlineto{\pgfqpoint{2.827112in}{2.272244in}}%
\pgfpathlineto{\pgfqpoint{2.818713in}{2.266015in}}%
\pgfpathclose%
\pgfusepath{fill}%
\end{pgfscope}%
\begin{pgfscope}%
\pgfpathrectangle{\pgfqpoint{1.150000in}{0.150000in}}{\pgfqpoint{5.700000in}{5.700000in}}%
\pgfusepath{clip}%
\pgfsetbuttcap%
\pgfsetroundjoin%
\definecolor{currentfill}{rgb}{0.210503,0.363727,0.552206}%
\pgfsetfillcolor{currentfill}%
\pgfsetfillopacity{0.800000}%
\pgfsetlinewidth{0.000000pt}%
\definecolor{currentstroke}{rgb}{0.000000,0.000000,0.000000}%
\pgfsetstrokecolor{currentstroke}%
\pgfsetdash{}{0pt}%
\pgfpathmoveto{\pgfqpoint{4.786078in}{2.721074in}}%
\pgfpathlineto{\pgfqpoint{4.800049in}{2.727577in}}%
\pgfpathlineto{\pgfqpoint{4.814033in}{2.734263in}}%
\pgfpathlineto{\pgfqpoint{4.828032in}{2.741130in}}%
\pgfpathlineto{\pgfqpoint{4.842046in}{2.748179in}}%
\pgfpathlineto{\pgfqpoint{4.849660in}{2.755664in}}%
\pgfpathlineto{\pgfqpoint{4.857268in}{2.763133in}}%
\pgfpathlineto{\pgfqpoint{4.864870in}{2.770588in}}%
\pgfpathlineto{\pgfqpoint{4.872467in}{2.778033in}}%
\pgfpathlineto{\pgfqpoint{4.858466in}{2.771318in}}%
\pgfpathlineto{\pgfqpoint{4.844480in}{2.764785in}}%
\pgfpathlineto{\pgfqpoint{4.830508in}{2.758432in}}%
\pgfpathlineto{\pgfqpoint{4.816550in}{2.752261in}}%
\pgfpathlineto{\pgfqpoint{4.808941in}{2.744471in}}%
\pgfpathlineto{\pgfqpoint{4.801326in}{2.736679in}}%
\pgfpathlineto{\pgfqpoint{4.793705in}{2.728881in}}%
\pgfpathlineto{\pgfqpoint{4.786078in}{2.721074in}}%
\pgfpathclose%
\pgfusepath{fill}%
\end{pgfscope}%
\begin{pgfscope}%
\pgfpathrectangle{\pgfqpoint{1.150000in}{0.150000in}}{\pgfqpoint{5.700000in}{5.700000in}}%
\pgfusepath{clip}%
\pgfsetbuttcap%
\pgfsetroundjoin%
\definecolor{currentfill}{rgb}{0.281887,0.150881,0.465405}%
\pgfsetfillcolor{currentfill}%
\pgfsetfillopacity{0.800000}%
\pgfsetlinewidth{0.000000pt}%
\definecolor{currentstroke}{rgb}{0.000000,0.000000,0.000000}%
\pgfsetstrokecolor{currentstroke}%
\pgfsetdash{}{0pt}%
\pgfpathmoveto{\pgfqpoint{3.978464in}{2.201051in}}%
\pgfpathlineto{\pgfqpoint{3.992104in}{2.202313in}}%
\pgfpathlineto{\pgfqpoint{4.005753in}{2.203771in}}%
\pgfpathlineto{\pgfqpoint{4.019411in}{2.205424in}}%
\pgfpathlineto{\pgfqpoint{4.033077in}{2.207272in}}%
\pgfpathlineto{\pgfqpoint{4.040990in}{2.217655in}}%
\pgfpathlineto{\pgfqpoint{4.048899in}{2.228007in}}%
\pgfpathlineto{\pgfqpoint{4.056802in}{2.238327in}}%
\pgfpathlineto{\pgfqpoint{4.064699in}{2.248617in}}%
\pgfpathlineto{\pgfqpoint{4.051040in}{2.246777in}}%
\pgfpathlineto{\pgfqpoint{4.037390in}{2.245132in}}%
\pgfpathlineto{\pgfqpoint{4.023748in}{2.243682in}}%
\pgfpathlineto{\pgfqpoint{4.010115in}{2.242428in}}%
\pgfpathlineto{\pgfqpoint{4.002210in}{2.232118in}}%
\pgfpathlineto{\pgfqpoint{3.994300in}{2.221786in}}%
\pgfpathlineto{\pgfqpoint{3.986385in}{2.211431in}}%
\pgfpathlineto{\pgfqpoint{3.978464in}{2.201051in}}%
\pgfpathclose%
\pgfusepath{fill}%
\end{pgfscope}%
\begin{pgfscope}%
\pgfpathrectangle{\pgfqpoint{1.150000in}{0.150000in}}{\pgfqpoint{5.700000in}{5.700000in}}%
\pgfusepath{clip}%
\pgfsetbuttcap%
\pgfsetroundjoin%
\definecolor{currentfill}{rgb}{0.283072,0.130895,0.449241}%
\pgfsetfillcolor{currentfill}%
\pgfsetfillopacity{0.800000}%
\pgfsetlinewidth{0.000000pt}%
\definecolor{currentstroke}{rgb}{0.000000,0.000000,0.000000}%
\pgfsetstrokecolor{currentstroke}%
\pgfsetdash{}{0pt}%
\pgfpathmoveto{\pgfqpoint{3.892223in}{2.156306in}}%
\pgfpathlineto{\pgfqpoint{3.905839in}{2.156754in}}%
\pgfpathlineto{\pgfqpoint{3.919462in}{2.157401in}}%
\pgfpathlineto{\pgfqpoint{3.933093in}{2.158246in}}%
\pgfpathlineto{\pgfqpoint{3.946732in}{2.159288in}}%
\pgfpathlineto{\pgfqpoint{3.954673in}{2.169766in}}%
\pgfpathlineto{\pgfqpoint{3.962608in}{2.180219in}}%
\pgfpathlineto{\pgfqpoint{3.970539in}{2.190648in}}%
\pgfpathlineto{\pgfqpoint{3.978464in}{2.201051in}}%
\pgfpathlineto{\pgfqpoint{3.964832in}{2.199986in}}%
\pgfpathlineto{\pgfqpoint{3.951209in}{2.199117in}}%
\pgfpathlineto{\pgfqpoint{3.937593in}{2.198446in}}%
\pgfpathlineto{\pgfqpoint{3.923986in}{2.197974in}}%
\pgfpathlineto{\pgfqpoint{3.916053in}{2.187582in}}%
\pgfpathlineto{\pgfqpoint{3.908115in}{2.177174in}}%
\pgfpathlineto{\pgfqpoint{3.900172in}{2.166748in}}%
\pgfpathlineto{\pgfqpoint{3.892223in}{2.156306in}}%
\pgfpathclose%
\pgfusepath{fill}%
\end{pgfscope}%
\begin{pgfscope}%
\pgfpathrectangle{\pgfqpoint{1.150000in}{0.150000in}}{\pgfqpoint{5.700000in}{5.700000in}}%
\pgfusepath{clip}%
\pgfsetbuttcap%
\pgfsetroundjoin%
\definecolor{currentfill}{rgb}{0.278826,0.175490,0.483397}%
\pgfsetfillcolor{currentfill}%
\pgfsetfillopacity{0.800000}%
\pgfsetlinewidth{0.000000pt}%
\definecolor{currentstroke}{rgb}{0.000000,0.000000,0.000000}%
\pgfsetstrokecolor{currentstroke}%
\pgfsetdash{}{0pt}%
\pgfpathmoveto{\pgfqpoint{4.064699in}{2.248617in}}%
\pgfpathlineto{\pgfqpoint{4.078368in}{2.250651in}}%
\pgfpathlineto{\pgfqpoint{4.092046in}{2.252879in}}%
\pgfpathlineto{\pgfqpoint{4.105733in}{2.255300in}}%
\pgfpathlineto{\pgfqpoint{4.119430in}{2.257914in}}%
\pgfpathlineto{\pgfqpoint{4.127316in}{2.268146in}}%
\pgfpathlineto{\pgfqpoint{4.135197in}{2.278340in}}%
\pgfpathlineto{\pgfqpoint{4.143072in}{2.288498in}}%
\pgfpathlineto{\pgfqpoint{4.150943in}{2.298621in}}%
\pgfpathlineto{\pgfqpoint{4.137252in}{2.296047in}}%
\pgfpathlineto{\pgfqpoint{4.123572in}{2.293666in}}%
\pgfpathlineto{\pgfqpoint{4.109901in}{2.291477in}}%
\pgfpathlineto{\pgfqpoint{4.096240in}{2.289483in}}%
\pgfpathlineto{\pgfqpoint{4.088362in}{2.279309in}}%
\pgfpathlineto{\pgfqpoint{4.080480in}{2.269107in}}%
\pgfpathlineto{\pgfqpoint{4.072592in}{2.258876in}}%
\pgfpathlineto{\pgfqpoint{4.064699in}{2.248617in}}%
\pgfpathclose%
\pgfusepath{fill}%
\end{pgfscope}%
\begin{pgfscope}%
\pgfpathrectangle{\pgfqpoint{1.150000in}{0.150000in}}{\pgfqpoint{5.700000in}{5.700000in}}%
\pgfusepath{clip}%
\pgfsetbuttcap%
\pgfsetroundjoin%
\definecolor{currentfill}{rgb}{0.283091,0.110553,0.431554}%
\pgfsetfillcolor{currentfill}%
\pgfsetfillopacity{0.800000}%
\pgfsetlinewidth{0.000000pt}%
\definecolor{currentstroke}{rgb}{0.000000,0.000000,0.000000}%
\pgfsetstrokecolor{currentstroke}%
\pgfsetdash{}{0pt}%
\pgfpathmoveto{\pgfqpoint{3.805961in}{2.114784in}}%
\pgfpathlineto{\pgfqpoint{3.819555in}{2.114378in}}%
\pgfpathlineto{\pgfqpoint{3.833156in}{2.114173in}}%
\pgfpathlineto{\pgfqpoint{3.846764in}{2.114168in}}%
\pgfpathlineto{\pgfqpoint{3.860379in}{2.114362in}}%
\pgfpathlineto{\pgfqpoint{3.868348in}{2.124874in}}%
\pgfpathlineto{\pgfqpoint{3.876312in}{2.135368in}}%
\pgfpathlineto{\pgfqpoint{3.884270in}{2.145846in}}%
\pgfpathlineto{\pgfqpoint{3.892223in}{2.156306in}}%
\pgfpathlineto{\pgfqpoint{3.878616in}{2.156056in}}%
\pgfpathlineto{\pgfqpoint{3.865016in}{2.156005in}}%
\pgfpathlineto{\pgfqpoint{3.851423in}{2.156155in}}%
\pgfpathlineto{\pgfqpoint{3.837838in}{2.156505in}}%
\pgfpathlineto{\pgfqpoint{3.829876in}{2.146089in}}%
\pgfpathlineto{\pgfqpoint{3.821910in}{2.135663in}}%
\pgfpathlineto{\pgfqpoint{3.813938in}{2.125228in}}%
\pgfpathlineto{\pgfqpoint{3.805961in}{2.114784in}}%
\pgfpathclose%
\pgfusepath{fill}%
\end{pgfscope}%
\begin{pgfscope}%
\pgfpathrectangle{\pgfqpoint{1.150000in}{0.150000in}}{\pgfqpoint{5.700000in}{5.700000in}}%
\pgfusepath{clip}%
\pgfsetbuttcap%
\pgfsetroundjoin%
\definecolor{currentfill}{rgb}{0.129933,0.559582,0.551864}%
\pgfsetfillcolor{currentfill}%
\pgfsetfillopacity{0.800000}%
\pgfsetlinewidth{0.000000pt}%
\definecolor{currentstroke}{rgb}{0.000000,0.000000,0.000000}%
\pgfsetstrokecolor{currentstroke}%
\pgfsetdash{}{0pt}%
\pgfpathmoveto{\pgfqpoint{5.680134in}{3.299985in}}%
\pgfpathlineto{\pgfqpoint{5.694523in}{3.308081in}}%
\pgfpathlineto{\pgfqpoint{5.708930in}{3.316349in}}%
\pgfpathlineto{\pgfqpoint{5.723355in}{3.324790in}}%
\pgfpathlineto{\pgfqpoint{5.737798in}{3.333403in}}%
\pgfpathlineto{\pgfqpoint{5.745000in}{3.338166in}}%
\pgfpathlineto{\pgfqpoint{5.752201in}{3.343086in}}%
\pgfpathlineto{\pgfqpoint{5.759401in}{3.348171in}}%
\pgfpathlineto{\pgfqpoint{5.766601in}{3.353427in}}%
\pgfpathlineto{\pgfqpoint{5.752190in}{3.345511in}}%
\pgfpathlineto{\pgfqpoint{5.737798in}{3.337767in}}%
\pgfpathlineto{\pgfqpoint{5.723423in}{3.330193in}}%
\pgfpathlineto{\pgfqpoint{5.709065in}{3.322791in}}%
\pgfpathlineto{\pgfqpoint{5.701833in}{3.316828in}}%
\pgfpathlineto{\pgfqpoint{5.694601in}{3.311045in}}%
\pgfpathlineto{\pgfqpoint{5.687368in}{3.305433in}}%
\pgfpathlineto{\pgfqpoint{5.680134in}{3.299985in}}%
\pgfpathclose%
\pgfusepath{fill}%
\end{pgfscope}%
\begin{pgfscope}%
\pgfpathrectangle{\pgfqpoint{1.150000in}{0.150000in}}{\pgfqpoint{5.700000in}{5.700000in}}%
\pgfusepath{clip}%
\pgfsetbuttcap%
\pgfsetroundjoin%
\definecolor{currentfill}{rgb}{0.275191,0.194905,0.496005}%
\pgfsetfillcolor{currentfill}%
\pgfsetfillopacity{0.800000}%
\pgfsetlinewidth{0.000000pt}%
\definecolor{currentstroke}{rgb}{0.000000,0.000000,0.000000}%
\pgfsetstrokecolor{currentstroke}%
\pgfsetdash{}{0pt}%
\pgfpathmoveto{\pgfqpoint{4.150943in}{2.298621in}}%
\pgfpathlineto{\pgfqpoint{4.164643in}{2.301387in}}%
\pgfpathlineto{\pgfqpoint{4.178353in}{2.304345in}}%
\pgfpathlineto{\pgfqpoint{4.192073in}{2.307495in}}%
\pgfpathlineto{\pgfqpoint{4.205803in}{2.310836in}}%
\pgfpathlineto{\pgfqpoint{4.213662in}{2.320864in}}%
\pgfpathlineto{\pgfqpoint{4.221514in}{2.330851in}}%
\pgfpathlineto{\pgfqpoint{4.229362in}{2.340797in}}%
\pgfpathlineto{\pgfqpoint{4.237204in}{2.350705in}}%
\pgfpathlineto{\pgfqpoint{4.223481in}{2.347437in}}%
\pgfpathlineto{\pgfqpoint{4.209767in}{2.344360in}}%
\pgfpathlineto{\pgfqpoint{4.196064in}{2.341474in}}%
\pgfpathlineto{\pgfqpoint{4.182371in}{2.338779in}}%
\pgfpathlineto{\pgfqpoint{4.174522in}{2.328787in}}%
\pgfpathlineto{\pgfqpoint{4.166667in}{2.318764in}}%
\pgfpathlineto{\pgfqpoint{4.158808in}{2.308709in}}%
\pgfpathlineto{\pgfqpoint{4.150943in}{2.298621in}}%
\pgfpathclose%
\pgfusepath{fill}%
\end{pgfscope}%
\begin{pgfscope}%
\pgfpathrectangle{\pgfqpoint{1.150000in}{0.150000in}}{\pgfqpoint{5.700000in}{5.700000in}}%
\pgfusepath{clip}%
\pgfsetbuttcap%
\pgfsetroundjoin%
\definecolor{currentfill}{rgb}{0.277941,0.056324,0.381191}%
\pgfsetfillcolor{currentfill}%
\pgfsetfillopacity{0.800000}%
\pgfsetlinewidth{0.000000pt}%
\definecolor{currentstroke}{rgb}{0.000000,0.000000,0.000000}%
\pgfsetstrokecolor{currentstroke}%
\pgfsetdash{}{0pt}%
\pgfpathmoveto{\pgfqpoint{3.406095in}{2.009088in}}%
\pgfpathlineto{\pgfqpoint{3.419635in}{2.003961in}}%
\pgfpathlineto{\pgfqpoint{3.433177in}{1.999053in}}%
\pgfpathlineto{\pgfqpoint{3.446722in}{1.994361in}}%
\pgfpathlineto{\pgfqpoint{3.460271in}{1.989885in}}%
\pgfpathlineto{\pgfqpoint{3.468377in}{1.999615in}}%
\pgfpathlineto{\pgfqpoint{3.476478in}{2.009380in}}%
\pgfpathlineto{\pgfqpoint{3.484572in}{2.019179in}}%
\pgfpathlineto{\pgfqpoint{3.492661in}{2.029012in}}%
\pgfpathlineto{\pgfqpoint{3.479126in}{2.033305in}}%
\pgfpathlineto{\pgfqpoint{3.465594in}{2.037815in}}%
\pgfpathlineto{\pgfqpoint{3.452065in}{2.042541in}}%
\pgfpathlineto{\pgfqpoint{3.438539in}{2.047485in}}%
\pgfpathlineto{\pgfqpoint{3.430437in}{2.037823in}}%
\pgfpathlineto{\pgfqpoint{3.422329in}{2.028202in}}%
\pgfpathlineto{\pgfqpoint{3.414215in}{2.018623in}}%
\pgfpathlineto{\pgfqpoint{3.406095in}{2.009088in}}%
\pgfpathclose%
\pgfusepath{fill}%
\end{pgfscope}%
\begin{pgfscope}%
\pgfpathrectangle{\pgfqpoint{1.150000in}{0.150000in}}{\pgfqpoint{5.700000in}{5.700000in}}%
\pgfusepath{clip}%
\pgfsetbuttcap%
\pgfsetroundjoin%
\definecolor{currentfill}{rgb}{0.282623,0.140926,0.457517}%
\pgfsetfillcolor{currentfill}%
\pgfsetfillopacity{0.800000}%
\pgfsetlinewidth{0.000000pt}%
\definecolor{currentstroke}{rgb}{0.000000,0.000000,0.000000}%
\pgfsetstrokecolor{currentstroke}%
\pgfsetdash{}{0pt}%
\pgfpathmoveto{\pgfqpoint{2.873281in}{2.209111in}}%
\pgfpathlineto{\pgfqpoint{2.886909in}{2.195560in}}%
\pgfpathlineto{\pgfqpoint{2.900533in}{2.182274in}}%
\pgfpathlineto{\pgfqpoint{2.914152in}{2.169253in}}%
\pgfpathlineto{\pgfqpoint{2.927767in}{2.156493in}}%
\pgfpathlineto{\pgfqpoint{2.936114in}{2.163317in}}%
\pgfpathlineto{\pgfqpoint{2.944451in}{2.170265in}}%
\pgfpathlineto{\pgfqpoint{2.952778in}{2.177334in}}%
\pgfpathlineto{\pgfqpoint{2.961097in}{2.184521in}}%
\pgfpathlineto{\pgfqpoint{2.947507in}{2.196998in}}%
\pgfpathlineto{\pgfqpoint{2.933913in}{2.209735in}}%
\pgfpathlineto{\pgfqpoint{2.920314in}{2.222736in}}%
\pgfpathlineto{\pgfqpoint{2.906711in}{2.236003in}}%
\pgfpathlineto{\pgfqpoint{2.898368in}{2.229087in}}%
\pgfpathlineto{\pgfqpoint{2.890015in}{2.222298in}}%
\pgfpathlineto{\pgfqpoint{2.881653in}{2.215639in}}%
\pgfpathlineto{\pgfqpoint{2.873281in}{2.209111in}}%
\pgfpathclose%
\pgfusepath{fill}%
\end{pgfscope}%
\begin{pgfscope}%
\pgfpathrectangle{\pgfqpoint{1.150000in}{0.150000in}}{\pgfqpoint{5.700000in}{5.700000in}}%
\pgfusepath{clip}%
\pgfsetbuttcap%
\pgfsetroundjoin%
\definecolor{currentfill}{rgb}{0.250425,0.274290,0.533103}%
\pgfsetfillcolor{currentfill}%
\pgfsetfillopacity{0.800000}%
\pgfsetlinewidth{0.000000pt}%
\definecolor{currentstroke}{rgb}{0.000000,0.000000,0.000000}%
\pgfsetstrokecolor{currentstroke}%
\pgfsetdash{}{0pt}%
\pgfpathmoveto{\pgfqpoint{2.599289in}{2.539341in}}%
\pgfpathlineto{\pgfqpoint{2.613070in}{2.519997in}}%
\pgfpathlineto{\pgfqpoint{2.626841in}{2.500968in}}%
\pgfpathlineto{\pgfqpoint{2.640602in}{2.482250in}}%
\pgfpathlineto{\pgfqpoint{2.654354in}{2.463840in}}%
\pgfpathlineto{\pgfqpoint{2.662843in}{2.469164in}}%
\pgfpathlineto{\pgfqpoint{2.671320in}{2.474653in}}%
\pgfpathlineto{\pgfqpoint{2.679786in}{2.480305in}}%
\pgfpathlineto{\pgfqpoint{2.688240in}{2.486116in}}%
\pgfpathlineto{\pgfqpoint{2.674520in}{2.504233in}}%
\pgfpathlineto{\pgfqpoint{2.660791in}{2.522658in}}%
\pgfpathlineto{\pgfqpoint{2.647052in}{2.541393in}}%
\pgfpathlineto{\pgfqpoint{2.633303in}{2.560442in}}%
\pgfpathlineto{\pgfqpoint{2.624817in}{2.554912in}}%
\pgfpathlineto{\pgfqpoint{2.616320in}{2.549550in}}%
\pgfpathlineto{\pgfqpoint{2.607810in}{2.544359in}}%
\pgfpathlineto{\pgfqpoint{2.599289in}{2.539341in}}%
\pgfpathclose%
\pgfusepath{fill}%
\end{pgfscope}%
\begin{pgfscope}%
\pgfpathrectangle{\pgfqpoint{1.150000in}{0.150000in}}{\pgfqpoint{5.700000in}{5.700000in}}%
\pgfusepath{clip}%
\pgfsetbuttcap%
\pgfsetroundjoin%
\definecolor{currentfill}{rgb}{0.280267,0.073417,0.397163}%
\pgfsetfillcolor{currentfill}%
\pgfsetfillopacity{0.800000}%
\pgfsetlinewidth{0.000000pt}%
\definecolor{currentstroke}{rgb}{0.000000,0.000000,0.000000}%
\pgfsetstrokecolor{currentstroke}%
\pgfsetdash{}{0pt}%
\pgfpathmoveto{\pgfqpoint{3.123950in}{2.054553in}}%
\pgfpathlineto{\pgfqpoint{3.137509in}{2.045317in}}%
\pgfpathlineto{\pgfqpoint{3.151067in}{2.036319in}}%
\pgfpathlineto{\pgfqpoint{3.164625in}{2.027558in}}%
\pgfpathlineto{\pgfqpoint{3.178182in}{2.019032in}}%
\pgfpathlineto{\pgfqpoint{3.186406in}{2.027380in}}%
\pgfpathlineto{\pgfqpoint{3.194623in}{2.035810in}}%
\pgfpathlineto{\pgfqpoint{3.202832in}{2.044321in}}%
\pgfpathlineto{\pgfqpoint{3.211034in}{2.052909in}}%
\pgfpathlineto{\pgfqpoint{3.197496in}{2.061188in}}%
\pgfpathlineto{\pgfqpoint{3.183958in}{2.069702in}}%
\pgfpathlineto{\pgfqpoint{3.170419in}{2.078452in}}%
\pgfpathlineto{\pgfqpoint{3.156880in}{2.087440in}}%
\pgfpathlineto{\pgfqpoint{3.148659in}{2.079087in}}%
\pgfpathlineto{\pgfqpoint{3.140430in}{2.070820in}}%
\pgfpathlineto{\pgfqpoint{3.132194in}{2.062642in}}%
\pgfpathlineto{\pgfqpoint{3.123950in}{2.054553in}}%
\pgfpathclose%
\pgfusepath{fill}%
\end{pgfscope}%
\begin{pgfscope}%
\pgfpathrectangle{\pgfqpoint{1.150000in}{0.150000in}}{\pgfqpoint{5.700000in}{5.700000in}}%
\pgfusepath{clip}%
\pgfsetbuttcap%
\pgfsetroundjoin%
\definecolor{currentfill}{rgb}{0.282327,0.094955,0.417331}%
\pgfsetfillcolor{currentfill}%
\pgfsetfillopacity{0.800000}%
\pgfsetlinewidth{0.000000pt}%
\definecolor{currentstroke}{rgb}{0.000000,0.000000,0.000000}%
\pgfsetstrokecolor{currentstroke}%
\pgfsetdash{}{0pt}%
\pgfpathmoveto{\pgfqpoint{3.719659in}{2.076916in}}%
\pgfpathlineto{\pgfqpoint{3.733235in}{2.075613in}}%
\pgfpathlineto{\pgfqpoint{3.746818in}{2.074513in}}%
\pgfpathlineto{\pgfqpoint{3.760407in}{2.073616in}}%
\pgfpathlineto{\pgfqpoint{3.774002in}{2.072921in}}%
\pgfpathlineto{\pgfqpoint{3.782000in}{2.083399in}}%
\pgfpathlineto{\pgfqpoint{3.789992in}{2.093869in}}%
\pgfpathlineto{\pgfqpoint{3.797979in}{2.104331in}}%
\pgfpathlineto{\pgfqpoint{3.805961in}{2.114784in}}%
\pgfpathlineto{\pgfqpoint{3.792374in}{2.115392in}}%
\pgfpathlineto{\pgfqpoint{3.778794in}{2.116201in}}%
\pgfpathlineto{\pgfqpoint{3.765220in}{2.117214in}}%
\pgfpathlineto{\pgfqpoint{3.751653in}{2.118430in}}%
\pgfpathlineto{\pgfqpoint{3.743662in}{2.108052in}}%
\pgfpathlineto{\pgfqpoint{3.735666in}{2.097674in}}%
\pgfpathlineto{\pgfqpoint{3.727665in}{2.087295in}}%
\pgfpathlineto{\pgfqpoint{3.719659in}{2.076916in}}%
\pgfpathclose%
\pgfusepath{fill}%
\end{pgfscope}%
\begin{pgfscope}%
\pgfpathrectangle{\pgfqpoint{1.150000in}{0.150000in}}{\pgfqpoint{5.700000in}{5.700000in}}%
\pgfusepath{clip}%
\pgfsetbuttcap%
\pgfsetroundjoin%
\definecolor{currentfill}{rgb}{0.199430,0.387607,0.554642}%
\pgfsetfillcolor{currentfill}%
\pgfsetfillopacity{0.800000}%
\pgfsetlinewidth{0.000000pt}%
\definecolor{currentstroke}{rgb}{0.000000,0.000000,0.000000}%
\pgfsetstrokecolor{currentstroke}%
\pgfsetdash{}{0pt}%
\pgfpathmoveto{\pgfqpoint{4.872467in}{2.778033in}}%
\pgfpathlineto{\pgfqpoint{4.886482in}{2.784929in}}%
\pgfpathlineto{\pgfqpoint{4.900512in}{2.792005in}}%
\pgfpathlineto{\pgfqpoint{4.914556in}{2.799262in}}%
\pgfpathlineto{\pgfqpoint{4.928616in}{2.806700in}}%
\pgfpathlineto{\pgfqpoint{4.936193in}{2.813785in}}%
\pgfpathlineto{\pgfqpoint{4.943764in}{2.820862in}}%
\pgfpathlineto{\pgfqpoint{4.951330in}{2.827933in}}%
\pgfpathlineto{\pgfqpoint{4.958890in}{2.835003in}}%
\pgfpathlineto{\pgfqpoint{4.944844in}{2.827933in}}%
\pgfpathlineto{\pgfqpoint{4.930814in}{2.821043in}}%
\pgfpathlineto{\pgfqpoint{4.916798in}{2.814332in}}%
\pgfpathlineto{\pgfqpoint{4.902797in}{2.807802in}}%
\pgfpathlineto{\pgfqpoint{4.895222in}{2.800353in}}%
\pgfpathlineto{\pgfqpoint{4.887643in}{2.792912in}}%
\pgfpathlineto{\pgfqpoint{4.880058in}{2.785473in}}%
\pgfpathlineto{\pgfqpoint{4.872467in}{2.778033in}}%
\pgfpathclose%
\pgfusepath{fill}%
\end{pgfscope}%
\begin{pgfscope}%
\pgfpathrectangle{\pgfqpoint{1.150000in}{0.150000in}}{\pgfqpoint{5.700000in}{5.700000in}}%
\pgfusepath{clip}%
\pgfsetbuttcap%
\pgfsetroundjoin%
\definecolor{currentfill}{rgb}{0.124395,0.578002,0.548287}%
\pgfsetfillcolor{currentfill}%
\pgfsetfillopacity{0.800000}%
\pgfsetlinewidth{0.000000pt}%
\definecolor{currentstroke}{rgb}{0.000000,0.000000,0.000000}%
\pgfsetstrokecolor{currentstroke}%
\pgfsetdash{}{0pt}%
\pgfpathmoveto{\pgfqpoint{5.766601in}{3.353427in}}%
\pgfpathlineto{\pgfqpoint{5.781029in}{3.361515in}}%
\pgfpathlineto{\pgfqpoint{5.795476in}{3.369775in}}%
\pgfpathlineto{\pgfqpoint{5.809941in}{3.378206in}}%
\pgfpathlineto{\pgfqpoint{5.824424in}{3.386809in}}%
\pgfpathlineto{\pgfqpoint{5.831590in}{3.391528in}}%
\pgfpathlineto{\pgfqpoint{5.838755in}{3.396428in}}%
\pgfpathlineto{\pgfqpoint{5.845921in}{3.401516in}}%
\pgfpathlineto{\pgfqpoint{5.853088in}{3.406801in}}%
\pgfpathlineto{\pgfqpoint{5.838640in}{3.398928in}}%
\pgfpathlineto{\pgfqpoint{5.824210in}{3.391226in}}%
\pgfpathlineto{\pgfqpoint{5.809797in}{3.383694in}}%
\pgfpathlineto{\pgfqpoint{5.795402in}{3.376333in}}%
\pgfpathlineto{\pgfqpoint{5.788201in}{3.370309in}}%
\pgfpathlineto{\pgfqpoint{5.781001in}{3.364489in}}%
\pgfpathlineto{\pgfqpoint{5.773801in}{3.358864in}}%
\pgfpathlineto{\pgfqpoint{5.766601in}{3.353427in}}%
\pgfpathclose%
\pgfusepath{fill}%
\end{pgfscope}%
\begin{pgfscope}%
\pgfpathrectangle{\pgfqpoint{1.150000in}{0.150000in}}{\pgfqpoint{5.700000in}{5.700000in}}%
\pgfusepath{clip}%
\pgfsetbuttcap%
\pgfsetroundjoin%
\definecolor{currentfill}{rgb}{0.269308,0.218818,0.509577}%
\pgfsetfillcolor{currentfill}%
\pgfsetfillopacity{0.800000}%
\pgfsetlinewidth{0.000000pt}%
\definecolor{currentstroke}{rgb}{0.000000,0.000000,0.000000}%
\pgfsetstrokecolor{currentstroke}%
\pgfsetdash{}{0pt}%
\pgfpathmoveto{\pgfqpoint{4.237204in}{2.350705in}}%
\pgfpathlineto{\pgfqpoint{4.250939in}{2.354164in}}%
\pgfpathlineto{\pgfqpoint{4.264684in}{2.357813in}}%
\pgfpathlineto{\pgfqpoint{4.278439in}{2.361652in}}%
\pgfpathlineto{\pgfqpoint{4.292206in}{2.365680in}}%
\pgfpathlineto{\pgfqpoint{4.300036in}{2.375458in}}%
\pgfpathlineto{\pgfqpoint{4.307861in}{2.385192in}}%
\pgfpathlineto{\pgfqpoint{4.315680in}{2.394884in}}%
\pgfpathlineto{\pgfqpoint{4.323494in}{2.404535in}}%
\pgfpathlineto{\pgfqpoint{4.309734in}{2.400612in}}%
\pgfpathlineto{\pgfqpoint{4.295985in}{2.396877in}}%
\pgfpathlineto{\pgfqpoint{4.282248in}{2.393333in}}%
\pgfpathlineto{\pgfqpoint{4.268521in}{2.389978in}}%
\pgfpathlineto{\pgfqpoint{4.260699in}{2.380211in}}%
\pgfpathlineto{\pgfqpoint{4.252873in}{2.370411in}}%
\pgfpathlineto{\pgfqpoint{4.245041in}{2.360576in}}%
\pgfpathlineto{\pgfqpoint{4.237204in}{2.350705in}}%
\pgfpathclose%
\pgfusepath{fill}%
\end{pgfscope}%
\begin{pgfscope}%
\pgfpathrectangle{\pgfqpoint{1.150000in}{0.150000in}}{\pgfqpoint{5.700000in}{5.700000in}}%
\pgfusepath{clip}%
\pgfsetbuttcap%
\pgfsetroundjoin%
\definecolor{currentfill}{rgb}{0.120565,0.596422,0.543611}%
\pgfsetfillcolor{currentfill}%
\pgfsetfillopacity{0.800000}%
\pgfsetlinewidth{0.000000pt}%
\definecolor{currentstroke}{rgb}{0.000000,0.000000,0.000000}%
\pgfsetstrokecolor{currentstroke}%
\pgfsetdash{}{0pt}%
\pgfpathmoveto{\pgfqpoint{5.853088in}{3.406801in}}%
\pgfpathlineto{\pgfqpoint{5.867555in}{3.414845in}}%
\pgfpathlineto{\pgfqpoint{5.882040in}{3.423060in}}%
\pgfpathlineto{\pgfqpoint{5.896543in}{3.431446in}}%
\pgfpathlineto{\pgfqpoint{5.911065in}{3.440003in}}%
\pgfpathlineto{\pgfqpoint{5.918197in}{3.444742in}}%
\pgfpathlineto{\pgfqpoint{5.925329in}{3.449687in}}%
\pgfpathlineto{\pgfqpoint{5.932464in}{3.454845in}}%
\pgfpathlineto{\pgfqpoint{5.939600in}{3.460226in}}%
\pgfpathlineto{\pgfqpoint{5.925115in}{3.452432in}}%
\pgfpathlineto{\pgfqpoint{5.910648in}{3.444808in}}%
\pgfpathlineto{\pgfqpoint{5.896200in}{3.437353in}}%
\pgfpathlineto{\pgfqpoint{5.881769in}{3.430068in}}%
\pgfpathlineto{\pgfqpoint{5.874596in}{3.423916in}}%
\pgfpathlineto{\pgfqpoint{5.867425in}{3.417993in}}%
\pgfpathlineto{\pgfqpoint{5.860256in}{3.412291in}}%
\pgfpathlineto{\pgfqpoint{5.853088in}{3.406801in}}%
\pgfpathclose%
\pgfusepath{fill}%
\end{pgfscope}%
\begin{pgfscope}%
\pgfpathrectangle{\pgfqpoint{1.150000in}{0.150000in}}{\pgfqpoint{5.700000in}{5.700000in}}%
\pgfusepath{clip}%
\pgfsetbuttcap%
\pgfsetroundjoin%
\definecolor{currentfill}{rgb}{0.280894,0.078907,0.402329}%
\pgfsetfillcolor{currentfill}%
\pgfsetfillopacity{0.800000}%
\pgfsetlinewidth{0.000000pt}%
\definecolor{currentstroke}{rgb}{0.000000,0.000000,0.000000}%
\pgfsetstrokecolor{currentstroke}%
\pgfsetdash{}{0pt}%
\pgfpathmoveto{\pgfqpoint{3.633293in}{2.043153in}}%
\pgfpathlineto{\pgfqpoint{3.646857in}{2.040909in}}%
\pgfpathlineto{\pgfqpoint{3.660426in}{2.038872in}}%
\pgfpathlineto{\pgfqpoint{3.674000in}{2.037040in}}%
\pgfpathlineto{\pgfqpoint{3.687580in}{2.035414in}}%
\pgfpathlineto{\pgfqpoint{3.695608in}{2.045786in}}%
\pgfpathlineto{\pgfqpoint{3.703630in}{2.056161in}}%
\pgfpathlineto{\pgfqpoint{3.711647in}{2.066538in}}%
\pgfpathlineto{\pgfqpoint{3.719659in}{2.076916in}}%
\pgfpathlineto{\pgfqpoint{3.706088in}{2.078423in}}%
\pgfpathlineto{\pgfqpoint{3.692524in}{2.080136in}}%
\pgfpathlineto{\pgfqpoint{3.678965in}{2.082055in}}%
\pgfpathlineto{\pgfqpoint{3.665411in}{2.084180in}}%
\pgfpathlineto{\pgfqpoint{3.657390in}{2.073909in}}%
\pgfpathlineto{\pgfqpoint{3.649363in}{2.063647in}}%
\pgfpathlineto{\pgfqpoint{3.641331in}{2.053395in}}%
\pgfpathlineto{\pgfqpoint{3.633293in}{2.043153in}}%
\pgfpathclose%
\pgfusepath{fill}%
\end{pgfscope}%
\begin{pgfscope}%
\pgfpathrectangle{\pgfqpoint{1.150000in}{0.150000in}}{\pgfqpoint{5.700000in}{5.700000in}}%
\pgfusepath{clip}%
\pgfsetbuttcap%
\pgfsetroundjoin%
\definecolor{currentfill}{rgb}{0.190631,0.407061,0.556089}%
\pgfsetfillcolor{currentfill}%
\pgfsetfillopacity{0.800000}%
\pgfsetlinewidth{0.000000pt}%
\definecolor{currentstroke}{rgb}{0.000000,0.000000,0.000000}%
\pgfsetstrokecolor{currentstroke}%
\pgfsetdash{}{0pt}%
\pgfpathmoveto{\pgfqpoint{4.958890in}{2.835003in}}%
\pgfpathlineto{\pgfqpoint{4.972950in}{2.842254in}}%
\pgfpathlineto{\pgfqpoint{4.987026in}{2.849684in}}%
\pgfpathlineto{\pgfqpoint{5.001117in}{2.857294in}}%
\pgfpathlineto{\pgfqpoint{5.015223in}{2.865084in}}%
\pgfpathlineto{\pgfqpoint{5.022762in}{2.871770in}}%
\pgfpathlineto{\pgfqpoint{5.030295in}{2.878457in}}%
\pgfpathlineto{\pgfqpoint{5.037823in}{2.885148in}}%
\pgfpathlineto{\pgfqpoint{5.045345in}{2.891849in}}%
\pgfpathlineto{\pgfqpoint{5.031254in}{2.884460in}}%
\pgfpathlineto{\pgfqpoint{5.017179in}{2.877251in}}%
\pgfpathlineto{\pgfqpoint{5.003119in}{2.870220in}}%
\pgfpathlineto{\pgfqpoint{4.989074in}{2.863368in}}%
\pgfpathlineto{\pgfqpoint{4.981536in}{2.856256in}}%
\pgfpathlineto{\pgfqpoint{4.973993in}{2.849160in}}%
\pgfpathlineto{\pgfqpoint{4.966444in}{2.842078in}}%
\pgfpathlineto{\pgfqpoint{4.958890in}{2.835003in}}%
\pgfpathclose%
\pgfusepath{fill}%
\end{pgfscope}%
\begin{pgfscope}%
\pgfpathrectangle{\pgfqpoint{1.150000in}{0.150000in}}{\pgfqpoint{5.700000in}{5.700000in}}%
\pgfusepath{clip}%
\pgfsetbuttcap%
\pgfsetroundjoin%
\definecolor{currentfill}{rgb}{0.283229,0.120777,0.440584}%
\pgfsetfillcolor{currentfill}%
\pgfsetfillopacity{0.800000}%
\pgfsetlinewidth{0.000000pt}%
\definecolor{currentstroke}{rgb}{0.000000,0.000000,0.000000}%
\pgfsetstrokecolor{currentstroke}%
\pgfsetdash{}{0pt}%
\pgfpathmoveto{\pgfqpoint{2.927767in}{2.156493in}}%
\pgfpathlineto{\pgfqpoint{2.941378in}{2.143993in}}%
\pgfpathlineto{\pgfqpoint{2.954985in}{2.131752in}}%
\pgfpathlineto{\pgfqpoint{2.968589in}{2.119767in}}%
\pgfpathlineto{\pgfqpoint{2.982189in}{2.108036in}}%
\pgfpathlineto{\pgfqpoint{2.990510in}{2.115155in}}%
\pgfpathlineto{\pgfqpoint{2.998823in}{2.122389in}}%
\pgfpathlineto{\pgfqpoint{3.007127in}{2.129736in}}%
\pgfpathlineto{\pgfqpoint{3.015422in}{2.137195in}}%
\pgfpathlineto{\pgfqpoint{3.001846in}{2.148643in}}%
\pgfpathlineto{\pgfqpoint{2.988266in}{2.160346in}}%
\pgfpathlineto{\pgfqpoint{2.974683in}{2.172305in}}%
\pgfpathlineto{\pgfqpoint{2.961097in}{2.184521in}}%
\pgfpathlineto{\pgfqpoint{2.952778in}{2.177334in}}%
\pgfpathlineto{\pgfqpoint{2.944451in}{2.170265in}}%
\pgfpathlineto{\pgfqpoint{2.936114in}{2.163317in}}%
\pgfpathlineto{\pgfqpoint{2.927767in}{2.156493in}}%
\pgfpathclose%
\pgfusepath{fill}%
\end{pgfscope}%
\begin{pgfscope}%
\pgfpathrectangle{\pgfqpoint{1.150000in}{0.150000in}}{\pgfqpoint{5.700000in}{5.700000in}}%
\pgfusepath{clip}%
\pgfsetbuttcap%
\pgfsetroundjoin%
\definecolor{currentfill}{rgb}{0.262138,0.242286,0.520837}%
\pgfsetfillcolor{currentfill}%
\pgfsetfillopacity{0.800000}%
\pgfsetlinewidth{0.000000pt}%
\definecolor{currentstroke}{rgb}{0.000000,0.000000,0.000000}%
\pgfsetstrokecolor{currentstroke}%
\pgfsetdash{}{0pt}%
\pgfpathmoveto{\pgfqpoint{4.323494in}{2.404535in}}%
\pgfpathlineto{\pgfqpoint{4.337264in}{2.408647in}}%
\pgfpathlineto{\pgfqpoint{4.351047in}{2.412948in}}%
\pgfpathlineto{\pgfqpoint{4.364840in}{2.417437in}}%
\pgfpathlineto{\pgfqpoint{4.378646in}{2.422114in}}%
\pgfpathlineto{\pgfqpoint{4.386447in}{2.431600in}}%
\pgfpathlineto{\pgfqpoint{4.394242in}{2.441042in}}%
\pgfpathlineto{\pgfqpoint{4.402032in}{2.450440in}}%
\pgfpathlineto{\pgfqpoint{4.409817in}{2.459796in}}%
\pgfpathlineto{\pgfqpoint{4.396019in}{2.455257in}}%
\pgfpathlineto{\pgfqpoint{4.382232in}{2.450905in}}%
\pgfpathlineto{\pgfqpoint{4.368458in}{2.446741in}}%
\pgfpathlineto{\pgfqpoint{4.354694in}{2.442765in}}%
\pgfpathlineto{\pgfqpoint{4.346902in}{2.433260in}}%
\pgfpathlineto{\pgfqpoint{4.339105in}{2.423721in}}%
\pgfpathlineto{\pgfqpoint{4.331302in}{2.414147in}}%
\pgfpathlineto{\pgfqpoint{4.323494in}{2.404535in}}%
\pgfpathclose%
\pgfusepath{fill}%
\end{pgfscope}%
\begin{pgfscope}%
\pgfpathrectangle{\pgfqpoint{1.150000in}{0.150000in}}{\pgfqpoint{5.700000in}{5.700000in}}%
\pgfusepath{clip}%
\pgfsetbuttcap%
\pgfsetroundjoin%
\definecolor{currentfill}{rgb}{0.235526,0.309527,0.542944}%
\pgfsetfillcolor{currentfill}%
\pgfsetfillopacity{0.800000}%
\pgfsetlinewidth{0.000000pt}%
\definecolor{currentstroke}{rgb}{0.000000,0.000000,0.000000}%
\pgfsetstrokecolor{currentstroke}%
\pgfsetdash{}{0pt}%
\pgfpathmoveto{\pgfqpoint{2.544052in}{2.619913in}}%
\pgfpathlineto{\pgfqpoint{2.557879in}{2.599284in}}%
\pgfpathlineto{\pgfqpoint{2.571693in}{2.578981in}}%
\pgfpathlineto{\pgfqpoint{2.585497in}{2.559001in}}%
\pgfpathlineto{\pgfqpoint{2.599289in}{2.539341in}}%
\pgfpathlineto{\pgfqpoint{2.607810in}{2.544359in}}%
\pgfpathlineto{\pgfqpoint{2.616320in}{2.549550in}}%
\pgfpathlineto{\pgfqpoint{2.624817in}{2.554912in}}%
\pgfpathlineto{\pgfqpoint{2.633303in}{2.560442in}}%
\pgfpathlineto{\pgfqpoint{2.619544in}{2.579807in}}%
\pgfpathlineto{\pgfqpoint{2.605773in}{2.599491in}}%
\pgfpathlineto{\pgfqpoint{2.591992in}{2.619497in}}%
\pgfpathlineto{\pgfqpoint{2.578200in}{2.639829in}}%
\pgfpathlineto{\pgfqpoint{2.569682in}{2.634583in}}%
\pgfpathlineto{\pgfqpoint{2.561151in}{2.629513in}}%
\pgfpathlineto{\pgfqpoint{2.552608in}{2.624622in}}%
\pgfpathlineto{\pgfqpoint{2.544052in}{2.619913in}}%
\pgfpathclose%
\pgfusepath{fill}%
\end{pgfscope}%
\begin{pgfscope}%
\pgfpathrectangle{\pgfqpoint{1.150000in}{0.150000in}}{\pgfqpoint{5.700000in}{5.700000in}}%
\pgfusepath{clip}%
\pgfsetbuttcap%
\pgfsetroundjoin%
\definecolor{currentfill}{rgb}{0.180629,0.429975,0.557282}%
\pgfsetfillcolor{currentfill}%
\pgfsetfillopacity{0.800000}%
\pgfsetlinewidth{0.000000pt}%
\definecolor{currentstroke}{rgb}{0.000000,0.000000,0.000000}%
\pgfsetstrokecolor{currentstroke}%
\pgfsetdash{}{0pt}%
\pgfpathmoveto{\pgfqpoint{5.045345in}{2.891849in}}%
\pgfpathlineto{\pgfqpoint{5.059451in}{2.899417in}}%
\pgfpathlineto{\pgfqpoint{5.073573in}{2.907164in}}%
\pgfpathlineto{\pgfqpoint{5.087710in}{2.915090in}}%
\pgfpathlineto{\pgfqpoint{5.101864in}{2.923195in}}%
\pgfpathlineto{\pgfqpoint{5.109364in}{2.929489in}}%
\pgfpathlineto{\pgfqpoint{5.116858in}{2.935794in}}%
\pgfpathlineto{\pgfqpoint{5.124346in}{2.942115in}}%
\pgfpathlineto{\pgfqpoint{5.131830in}{2.948458in}}%
\pgfpathlineto{\pgfqpoint{5.117694in}{2.940788in}}%
\pgfpathlineto{\pgfqpoint{5.103573in}{2.933295in}}%
\pgfpathlineto{\pgfqpoint{5.089468in}{2.925981in}}%
\pgfpathlineto{\pgfqpoint{5.075379in}{2.918846in}}%
\pgfpathlineto{\pgfqpoint{5.067878in}{2.912058in}}%
\pgfpathlineto{\pgfqpoint{5.060373in}{2.905300in}}%
\pgfpathlineto{\pgfqpoint{5.052861in}{2.898565in}}%
\pgfpathlineto{\pgfqpoint{5.045345in}{2.891849in}}%
\pgfpathclose%
\pgfusepath{fill}%
\end{pgfscope}%
\begin{pgfscope}%
\pgfpathrectangle{\pgfqpoint{1.150000in}{0.150000in}}{\pgfqpoint{5.700000in}{5.700000in}}%
\pgfusepath{clip}%
\pgfsetbuttcap%
\pgfsetroundjoin%
\definecolor{currentfill}{rgb}{0.252194,0.269783,0.531579}%
\pgfsetfillcolor{currentfill}%
\pgfsetfillopacity{0.800000}%
\pgfsetlinewidth{0.000000pt}%
\definecolor{currentstroke}{rgb}{0.000000,0.000000,0.000000}%
\pgfsetstrokecolor{currentstroke}%
\pgfsetdash{}{0pt}%
\pgfpathmoveto{\pgfqpoint{4.409817in}{2.459796in}}%
\pgfpathlineto{\pgfqpoint{4.423626in}{2.464523in}}%
\pgfpathlineto{\pgfqpoint{4.437448in}{2.469437in}}%
\pgfpathlineto{\pgfqpoint{4.451282in}{2.474538in}}%
\pgfpathlineto{\pgfqpoint{4.465128in}{2.479825in}}%
\pgfpathlineto{\pgfqpoint{4.472899in}{2.488984in}}%
\pgfpathlineto{\pgfqpoint{4.480664in}{2.498098in}}%
\pgfpathlineto{\pgfqpoint{4.488424in}{2.507169in}}%
\pgfpathlineto{\pgfqpoint{4.496178in}{2.516199in}}%
\pgfpathlineto{\pgfqpoint{4.482340in}{2.511082in}}%
\pgfpathlineto{\pgfqpoint{4.468514in}{2.506151in}}%
\pgfpathlineto{\pgfqpoint{4.454700in}{2.501406in}}%
\pgfpathlineto{\pgfqpoint{4.440899in}{2.496848in}}%
\pgfpathlineto{\pgfqpoint{4.433137in}{2.487637in}}%
\pgfpathlineto{\pgfqpoint{4.425369in}{2.478393in}}%
\pgfpathlineto{\pgfqpoint{4.417595in}{2.469113in}}%
\pgfpathlineto{\pgfqpoint{4.409817in}{2.459796in}}%
\pgfpathclose%
\pgfusepath{fill}%
\end{pgfscope}%
\begin{pgfscope}%
\pgfpathrectangle{\pgfqpoint{1.150000in}{0.150000in}}{\pgfqpoint{5.700000in}{5.700000in}}%
\pgfusepath{clip}%
\pgfsetbuttcap%
\pgfsetroundjoin%
\definecolor{currentfill}{rgb}{0.279566,0.067836,0.391917}%
\pgfsetfillcolor{currentfill}%
\pgfsetfillopacity{0.800000}%
\pgfsetlinewidth{0.000000pt}%
\definecolor{currentstroke}{rgb}{0.000000,0.000000,0.000000}%
\pgfsetstrokecolor{currentstroke}%
\pgfsetdash{}{0pt}%
\pgfpathmoveto{\pgfqpoint{3.546840in}{2.013975in}}%
\pgfpathlineto{\pgfqpoint{3.560396in}{2.010745in}}%
\pgfpathlineto{\pgfqpoint{3.573955in}{2.007726in}}%
\pgfpathlineto{\pgfqpoint{3.587520in}{2.004916in}}%
\pgfpathlineto{\pgfqpoint{3.601089in}{2.002314in}}%
\pgfpathlineto{\pgfqpoint{3.609148in}{2.012503in}}%
\pgfpathlineto{\pgfqpoint{3.617202in}{2.022706in}}%
\pgfpathlineto{\pgfqpoint{3.625250in}{2.032923in}}%
\pgfpathlineto{\pgfqpoint{3.633293in}{2.043153in}}%
\pgfpathlineto{\pgfqpoint{3.619735in}{2.045605in}}%
\pgfpathlineto{\pgfqpoint{3.606182in}{2.048265in}}%
\pgfpathlineto{\pgfqpoint{3.592633in}{2.051134in}}%
\pgfpathlineto{\pgfqpoint{3.579090in}{2.054213in}}%
\pgfpathlineto{\pgfqpoint{3.571036in}{2.044121in}}%
\pgfpathlineto{\pgfqpoint{3.562976in}{2.034051in}}%
\pgfpathlineto{\pgfqpoint{3.554911in}{2.024001in}}%
\pgfpathlineto{\pgfqpoint{3.546840in}{2.013975in}}%
\pgfpathclose%
\pgfusepath{fill}%
\end{pgfscope}%
\begin{pgfscope}%
\pgfpathrectangle{\pgfqpoint{1.150000in}{0.150000in}}{\pgfqpoint{5.700000in}{5.700000in}}%
\pgfusepath{clip}%
\pgfsetbuttcap%
\pgfsetroundjoin%
\definecolor{currentfill}{rgb}{0.119483,0.614817,0.537692}%
\pgfsetfillcolor{currentfill}%
\pgfsetfillopacity{0.800000}%
\pgfsetlinewidth{0.000000pt}%
\definecolor{currentstroke}{rgb}{0.000000,0.000000,0.000000}%
\pgfsetstrokecolor{currentstroke}%
\pgfsetdash{}{0pt}%
\pgfpathmoveto{\pgfqpoint{5.939600in}{3.460226in}}%
\pgfpathlineto{\pgfqpoint{5.954103in}{3.468191in}}%
\pgfpathlineto{\pgfqpoint{5.968625in}{3.476326in}}%
\pgfpathlineto{\pgfqpoint{5.983165in}{3.484630in}}%
\pgfpathlineto{\pgfqpoint{5.997725in}{3.493106in}}%
\pgfpathlineto{\pgfqpoint{6.004825in}{3.497934in}}%
\pgfpathlineto{\pgfqpoint{6.011927in}{3.502995in}}%
\pgfpathlineto{\pgfqpoint{6.019032in}{3.508297in}}%
\pgfpathlineto{\pgfqpoint{6.004503in}{3.500415in}}%
\pgfpathlineto{\pgfqpoint{5.989991in}{3.492703in}}%
\pgfpathlineto{\pgfqpoint{5.975499in}{3.485161in}}%
\pgfpathlineto{\pgfqpoint{5.961024in}{3.477787in}}%
\pgfpathlineto{\pgfqpoint{5.953880in}{3.471689in}}%
\pgfpathlineto{\pgfqpoint{5.946738in}{3.465838in}}%
\pgfpathlineto{\pgfqpoint{5.939600in}{3.460226in}}%
\pgfpathclose%
\pgfusepath{fill}%
\end{pgfscope}%
\begin{pgfscope}%
\pgfpathrectangle{\pgfqpoint{1.150000in}{0.150000in}}{\pgfqpoint{5.700000in}{5.700000in}}%
\pgfusepath{clip}%
\pgfsetbuttcap%
\pgfsetroundjoin%
\definecolor{currentfill}{rgb}{0.277018,0.050344,0.375715}%
\pgfsetfillcolor{currentfill}%
\pgfsetfillopacity{0.800000}%
\pgfsetlinewidth{0.000000pt}%
\definecolor{currentstroke}{rgb}{0.000000,0.000000,0.000000}%
\pgfsetstrokecolor{currentstroke}%
\pgfsetdash{}{0pt}%
\pgfpathmoveto{\pgfqpoint{3.319357in}{1.994978in}}%
\pgfpathlineto{\pgfqpoint{3.332903in}{1.988756in}}%
\pgfpathlineto{\pgfqpoint{3.346451in}{1.982755in}}%
\pgfpathlineto{\pgfqpoint{3.360000in}{1.976976in}}%
\pgfpathlineto{\pgfqpoint{3.373552in}{1.971417in}}%
\pgfpathlineto{\pgfqpoint{3.381697in}{1.980760in}}%
\pgfpathlineto{\pgfqpoint{3.389836in}{1.990155in}}%
\pgfpathlineto{\pgfqpoint{3.397969in}{1.999598in}}%
\pgfpathlineto{\pgfqpoint{3.406095in}{2.009088in}}%
\pgfpathlineto{\pgfqpoint{3.392558in}{2.014433in}}%
\pgfpathlineto{\pgfqpoint{3.379024in}{2.019998in}}%
\pgfpathlineto{\pgfqpoint{3.365491in}{2.025784in}}%
\pgfpathlineto{\pgfqpoint{3.351961in}{2.031793in}}%
\pgfpathlineto{\pgfqpoint{3.343820in}{2.022505in}}%
\pgfpathlineto{\pgfqpoint{3.335672in}{2.013273in}}%
\pgfpathlineto{\pgfqpoint{3.327518in}{2.004096in}}%
\pgfpathlineto{\pgfqpoint{3.319357in}{1.994978in}}%
\pgfpathclose%
\pgfusepath{fill}%
\end{pgfscope}%
\begin{pgfscope}%
\pgfpathrectangle{\pgfqpoint{1.150000in}{0.150000in}}{\pgfqpoint{5.700000in}{5.700000in}}%
\pgfusepath{clip}%
\pgfsetbuttcap%
\pgfsetroundjoin%
\definecolor{currentfill}{rgb}{0.278791,0.062145,0.386592}%
\pgfsetfillcolor{currentfill}%
\pgfsetfillopacity{0.800000}%
\pgfsetlinewidth{0.000000pt}%
\definecolor{currentstroke}{rgb}{0.000000,0.000000,0.000000}%
\pgfsetstrokecolor{currentstroke}%
\pgfsetdash{}{0pt}%
\pgfpathmoveto{\pgfqpoint{3.178182in}{2.019032in}}%
\pgfpathlineto{\pgfqpoint{3.191739in}{2.010740in}}%
\pgfpathlineto{\pgfqpoint{3.205295in}{2.002680in}}%
\pgfpathlineto{\pgfqpoint{3.218852in}{1.994851in}}%
\pgfpathlineto{\pgfqpoint{3.232410in}{1.987252in}}%
\pgfpathlineto{\pgfqpoint{3.240615in}{1.995859in}}%
\pgfpathlineto{\pgfqpoint{3.248813in}{2.004540in}}%
\pgfpathlineto{\pgfqpoint{3.257005in}{2.013293in}}%
\pgfpathlineto{\pgfqpoint{3.265189in}{2.022117in}}%
\pgfpathlineto{\pgfqpoint{3.251649in}{2.029469in}}%
\pgfpathlineto{\pgfqpoint{3.238111in}{2.037051in}}%
\pgfpathlineto{\pgfqpoint{3.224572in}{2.044864in}}%
\pgfpathlineto{\pgfqpoint{3.211034in}{2.052909in}}%
\pgfpathlineto{\pgfqpoint{3.202832in}{2.044321in}}%
\pgfpathlineto{\pgfqpoint{3.194623in}{2.035810in}}%
\pgfpathlineto{\pgfqpoint{3.186406in}{2.027380in}}%
\pgfpathlineto{\pgfqpoint{3.178182in}{2.019032in}}%
\pgfpathclose%
\pgfusepath{fill}%
\end{pgfscope}%
\begin{pgfscope}%
\pgfpathrectangle{\pgfqpoint{1.150000in}{0.150000in}}{\pgfqpoint{5.700000in}{5.700000in}}%
\pgfusepath{clip}%
\pgfsetbuttcap%
\pgfsetroundjoin%
\definecolor{currentfill}{rgb}{0.282656,0.100196,0.422160}%
\pgfsetfillcolor{currentfill}%
\pgfsetfillopacity{0.800000}%
\pgfsetlinewidth{0.000000pt}%
\definecolor{currentstroke}{rgb}{0.000000,0.000000,0.000000}%
\pgfsetstrokecolor{currentstroke}%
\pgfsetdash{}{0pt}%
\pgfpathmoveto{\pgfqpoint{2.982189in}{2.108036in}}%
\pgfpathlineto{\pgfqpoint{2.995786in}{2.096558in}}%
\pgfpathlineto{\pgfqpoint{3.009380in}{2.085331in}}%
\pgfpathlineto{\pgfqpoint{3.022971in}{2.074353in}}%
\pgfpathlineto{\pgfqpoint{3.036560in}{2.063623in}}%
\pgfpathlineto{\pgfqpoint{3.044858in}{2.071035in}}%
\pgfpathlineto{\pgfqpoint{3.053148in}{2.078555in}}%
\pgfpathlineto{\pgfqpoint{3.061429in}{2.086180in}}%
\pgfpathlineto{\pgfqpoint{3.069702in}{2.093908in}}%
\pgfpathlineto{\pgfqpoint{3.056135in}{2.104357in}}%
\pgfpathlineto{\pgfqpoint{3.042567in}{2.115053in}}%
\pgfpathlineto{\pgfqpoint{3.028996in}{2.125999in}}%
\pgfpathlineto{\pgfqpoint{3.015422in}{2.137195in}}%
\pgfpathlineto{\pgfqpoint{3.007127in}{2.129736in}}%
\pgfpathlineto{\pgfqpoint{2.998823in}{2.122389in}}%
\pgfpathlineto{\pgfqpoint{2.990510in}{2.115155in}}%
\pgfpathlineto{\pgfqpoint{2.982189in}{2.108036in}}%
\pgfpathclose%
\pgfusepath{fill}%
\end{pgfscope}%
\begin{pgfscope}%
\pgfpathrectangle{\pgfqpoint{1.150000in}{0.150000in}}{\pgfqpoint{5.700000in}{5.700000in}}%
\pgfusepath{clip}%
\pgfsetbuttcap%
\pgfsetroundjoin%
\definecolor{currentfill}{rgb}{0.172719,0.448791,0.557885}%
\pgfsetfillcolor{currentfill}%
\pgfsetfillopacity{0.800000}%
\pgfsetlinewidth{0.000000pt}%
\definecolor{currentstroke}{rgb}{0.000000,0.000000,0.000000}%
\pgfsetstrokecolor{currentstroke}%
\pgfsetdash{}{0pt}%
\pgfpathmoveto{\pgfqpoint{5.131830in}{2.948458in}}%
\pgfpathlineto{\pgfqpoint{5.145982in}{2.956306in}}%
\pgfpathlineto{\pgfqpoint{5.160150in}{2.964333in}}%
\pgfpathlineto{\pgfqpoint{5.174334in}{2.972538in}}%
\pgfpathlineto{\pgfqpoint{5.188535in}{2.980920in}}%
\pgfpathlineto{\pgfqpoint{5.195994in}{2.986834in}}%
\pgfpathlineto{\pgfqpoint{5.203448in}{2.992772in}}%
\pgfpathlineto{\pgfqpoint{5.210897in}{2.998738in}}%
\pgfpathlineto{\pgfqpoint{5.218341in}{3.004739in}}%
\pgfpathlineto{\pgfqpoint{5.204159in}{2.996824in}}%
\pgfpathlineto{\pgfqpoint{5.189994in}{2.989087in}}%
\pgfpathlineto{\pgfqpoint{5.175844in}{2.981527in}}%
\pgfpathlineto{\pgfqpoint{5.161711in}{2.974144in}}%
\pgfpathlineto{\pgfqpoint{5.154248in}{2.967665in}}%
\pgfpathlineto{\pgfqpoint{5.146780in}{2.961228in}}%
\pgfpathlineto{\pgfqpoint{5.139308in}{2.954827in}}%
\pgfpathlineto{\pgfqpoint{5.131830in}{2.948458in}}%
\pgfpathclose%
\pgfusepath{fill}%
\end{pgfscope}%
\begin{pgfscope}%
\pgfpathrectangle{\pgfqpoint{1.150000in}{0.150000in}}{\pgfqpoint{5.700000in}{5.700000in}}%
\pgfusepath{clip}%
\pgfsetbuttcap%
\pgfsetroundjoin%
\definecolor{currentfill}{rgb}{0.220057,0.343307,0.549413}%
\pgfsetfillcolor{currentfill}%
\pgfsetfillopacity{0.800000}%
\pgfsetlinewidth{0.000000pt}%
\definecolor{currentstroke}{rgb}{0.000000,0.000000,0.000000}%
\pgfsetstrokecolor{currentstroke}%
\pgfsetdash{}{0pt}%
\pgfpathmoveto{\pgfqpoint{2.488623in}{2.705751in}}%
\pgfpathlineto{\pgfqpoint{2.502500in}{2.683786in}}%
\pgfpathlineto{\pgfqpoint{2.516363in}{2.662161in}}%
\pgfpathlineto{\pgfqpoint{2.530214in}{2.640871in}}%
\pgfpathlineto{\pgfqpoint{2.544052in}{2.619913in}}%
\pgfpathlineto{\pgfqpoint{2.552608in}{2.624622in}}%
\pgfpathlineto{\pgfqpoint{2.561151in}{2.629513in}}%
\pgfpathlineto{\pgfqpoint{2.569682in}{2.634583in}}%
\pgfpathlineto{\pgfqpoint{2.578200in}{2.639829in}}%
\pgfpathlineto{\pgfqpoint{2.564396in}{2.660489in}}%
\pgfpathlineto{\pgfqpoint{2.550580in}{2.681480in}}%
\pgfpathlineto{\pgfqpoint{2.536751in}{2.702806in}}%
\pgfpathlineto{\pgfqpoint{2.522910in}{2.724470in}}%
\pgfpathlineto{\pgfqpoint{2.514358in}{2.719510in}}%
\pgfpathlineto{\pgfqpoint{2.505793in}{2.714736in}}%
\pgfpathlineto{\pgfqpoint{2.497215in}{2.710148in}}%
\pgfpathlineto{\pgfqpoint{2.488623in}{2.705751in}}%
\pgfpathclose%
\pgfusepath{fill}%
\end{pgfscope}%
\begin{pgfscope}%
\pgfpathrectangle{\pgfqpoint{1.150000in}{0.150000in}}{\pgfqpoint{5.700000in}{5.700000in}}%
\pgfusepath{clip}%
\pgfsetbuttcap%
\pgfsetroundjoin%
\definecolor{currentfill}{rgb}{0.243113,0.292092,0.538516}%
\pgfsetfillcolor{currentfill}%
\pgfsetfillopacity{0.800000}%
\pgfsetlinewidth{0.000000pt}%
\definecolor{currentstroke}{rgb}{0.000000,0.000000,0.000000}%
\pgfsetstrokecolor{currentstroke}%
\pgfsetdash{}{0pt}%
\pgfpathmoveto{\pgfqpoint{4.496178in}{2.516199in}}%
\pgfpathlineto{\pgfqpoint{4.510029in}{2.521502in}}%
\pgfpathlineto{\pgfqpoint{4.523892in}{2.526991in}}%
\pgfpathlineto{\pgfqpoint{4.537768in}{2.532665in}}%
\pgfpathlineto{\pgfqpoint{4.551657in}{2.538525in}}%
\pgfpathlineto{\pgfqpoint{4.559396in}{2.547326in}}%
\pgfpathlineto{\pgfqpoint{4.567130in}{2.556082in}}%
\pgfpathlineto{\pgfqpoint{4.574859in}{2.564797in}}%
\pgfpathlineto{\pgfqpoint{4.582581in}{2.573474in}}%
\pgfpathlineto{\pgfqpoint{4.568701in}{2.567817in}}%
\pgfpathlineto{\pgfqpoint{4.554834in}{2.562345in}}%
\pgfpathlineto{\pgfqpoint{4.540979in}{2.557058in}}%
\pgfpathlineto{\pgfqpoint{4.527138in}{2.551957in}}%
\pgfpathlineto{\pgfqpoint{4.519406in}{2.543067in}}%
\pgfpathlineto{\pgfqpoint{4.511669in}{2.534145in}}%
\pgfpathlineto{\pgfqpoint{4.503926in}{2.525190in}}%
\pgfpathlineto{\pgfqpoint{4.496178in}{2.516199in}}%
\pgfpathclose%
\pgfusepath{fill}%
\end{pgfscope}%
\begin{pgfscope}%
\pgfpathrectangle{\pgfqpoint{1.150000in}{0.150000in}}{\pgfqpoint{5.700000in}{5.700000in}}%
\pgfusepath{clip}%
\pgfsetbuttcap%
\pgfsetroundjoin%
\definecolor{currentfill}{rgb}{0.165117,0.467423,0.558141}%
\pgfsetfillcolor{currentfill}%
\pgfsetfillopacity{0.800000}%
\pgfsetlinewidth{0.000000pt}%
\definecolor{currentstroke}{rgb}{0.000000,0.000000,0.000000}%
\pgfsetstrokecolor{currentstroke}%
\pgfsetdash{}{0pt}%
\pgfpathmoveto{\pgfqpoint{5.218341in}{3.004739in}}%
\pgfpathlineto{\pgfqpoint{5.232538in}{3.012831in}}%
\pgfpathlineto{\pgfqpoint{5.246753in}{3.021100in}}%
\pgfpathlineto{\pgfqpoint{5.260984in}{3.029547in}}%
\pgfpathlineto{\pgfqpoint{5.275231in}{3.038171in}}%
\pgfpathlineto{\pgfqpoint{5.282650in}{3.043723in}}%
\pgfpathlineto{\pgfqpoint{5.290063in}{3.049312in}}%
\pgfpathlineto{\pgfqpoint{5.297471in}{3.054944in}}%
\pgfpathlineto{\pgfqpoint{5.304874in}{3.060625in}}%
\pgfpathlineto{\pgfqpoint{5.290648in}{3.052503in}}%
\pgfpathlineto{\pgfqpoint{5.276438in}{3.044557in}}%
\pgfpathlineto{\pgfqpoint{5.262244in}{3.036787in}}%
\pgfpathlineto{\pgfqpoint{5.248066in}{3.029194in}}%
\pgfpathlineto{\pgfqpoint{5.240642in}{3.023002in}}%
\pgfpathlineto{\pgfqpoint{5.233213in}{3.016865in}}%
\pgfpathlineto{\pgfqpoint{5.225779in}{3.010780in}}%
\pgfpathlineto{\pgfqpoint{5.218341in}{3.004739in}}%
\pgfpathclose%
\pgfusepath{fill}%
\end{pgfscope}%
\begin{pgfscope}%
\pgfpathrectangle{\pgfqpoint{1.150000in}{0.150000in}}{\pgfqpoint{5.700000in}{5.700000in}}%
\pgfusepath{clip}%
\pgfsetbuttcap%
\pgfsetroundjoin%
\definecolor{currentfill}{rgb}{0.231674,0.318106,0.544834}%
\pgfsetfillcolor{currentfill}%
\pgfsetfillopacity{0.800000}%
\pgfsetlinewidth{0.000000pt}%
\definecolor{currentstroke}{rgb}{0.000000,0.000000,0.000000}%
\pgfsetstrokecolor{currentstroke}%
\pgfsetdash{}{0pt}%
\pgfpathmoveto{\pgfqpoint{4.582581in}{2.573474in}}%
\pgfpathlineto{\pgfqpoint{4.596474in}{2.579315in}}%
\pgfpathlineto{\pgfqpoint{4.610380in}{2.585341in}}%
\pgfpathlineto{\pgfqpoint{4.624300in}{2.591551in}}%
\pgfpathlineto{\pgfqpoint{4.638233in}{2.597946in}}%
\pgfpathlineto{\pgfqpoint{4.645940in}{2.606362in}}%
\pgfpathlineto{\pgfqpoint{4.653642in}{2.614737in}}%
\pgfpathlineto{\pgfqpoint{4.661337in}{2.623073in}}%
\pgfpathlineto{\pgfqpoint{4.669027in}{2.631374in}}%
\pgfpathlineto{\pgfqpoint{4.655103in}{2.625216in}}%
\pgfpathlineto{\pgfqpoint{4.641193in}{2.619241in}}%
\pgfpathlineto{\pgfqpoint{4.627297in}{2.613451in}}%
\pgfpathlineto{\pgfqpoint{4.613413in}{2.607844in}}%
\pgfpathlineto{\pgfqpoint{4.605714in}{2.599296in}}%
\pgfpathlineto{\pgfqpoint{4.598009in}{2.590720in}}%
\pgfpathlineto{\pgfqpoint{4.590298in}{2.582114in}}%
\pgfpathlineto{\pgfqpoint{4.582581in}{2.573474in}}%
\pgfpathclose%
\pgfusepath{fill}%
\end{pgfscope}%
\begin{pgfscope}%
\pgfpathrectangle{\pgfqpoint{1.150000in}{0.150000in}}{\pgfqpoint{5.700000in}{5.700000in}}%
\pgfusepath{clip}%
\pgfsetbuttcap%
\pgfsetroundjoin%
\definecolor{currentfill}{rgb}{0.277941,0.056324,0.381191}%
\pgfsetfillcolor{currentfill}%
\pgfsetfillopacity{0.800000}%
\pgfsetlinewidth{0.000000pt}%
\definecolor{currentstroke}{rgb}{0.000000,0.000000,0.000000}%
\pgfsetstrokecolor{currentstroke}%
\pgfsetdash{}{0pt}%
\pgfpathmoveto{\pgfqpoint{3.460271in}{1.989885in}}%
\pgfpathlineto{\pgfqpoint{3.473823in}{1.985624in}}%
\pgfpathlineto{\pgfqpoint{3.487378in}{1.981577in}}%
\pgfpathlineto{\pgfqpoint{3.500937in}{1.977742in}}%
\pgfpathlineto{\pgfqpoint{3.514501in}{1.974120in}}%
\pgfpathlineto{\pgfqpoint{3.522594in}{1.984043in}}%
\pgfpathlineto{\pgfqpoint{3.530682in}{1.993994in}}%
\pgfpathlineto{\pgfqpoint{3.538764in}{2.003972in}}%
\pgfpathlineto{\pgfqpoint{3.546840in}{2.013975in}}%
\pgfpathlineto{\pgfqpoint{3.533290in}{2.017415in}}%
\pgfpathlineto{\pgfqpoint{3.519743in}{2.021068in}}%
\pgfpathlineto{\pgfqpoint{3.506200in}{2.024933in}}%
\pgfpathlineto{\pgfqpoint{3.492661in}{2.029012in}}%
\pgfpathlineto{\pgfqpoint{3.484572in}{2.019179in}}%
\pgfpathlineto{\pgfqpoint{3.476478in}{2.009380in}}%
\pgfpathlineto{\pgfqpoint{3.468377in}{1.999615in}}%
\pgfpathlineto{\pgfqpoint{3.460271in}{1.989885in}}%
\pgfpathclose%
\pgfusepath{fill}%
\end{pgfscope}%
\begin{pgfscope}%
\pgfpathrectangle{\pgfqpoint{1.150000in}{0.150000in}}{\pgfqpoint{5.700000in}{5.700000in}}%
\pgfusepath{clip}%
\pgfsetbuttcap%
\pgfsetroundjoin%
\definecolor{currentfill}{rgb}{0.156270,0.489624,0.557936}%
\pgfsetfillcolor{currentfill}%
\pgfsetfillopacity{0.800000}%
\pgfsetlinewidth{0.000000pt}%
\definecolor{currentstroke}{rgb}{0.000000,0.000000,0.000000}%
\pgfsetstrokecolor{currentstroke}%
\pgfsetdash{}{0pt}%
\pgfpathmoveto{\pgfqpoint{5.304874in}{3.060625in}}%
\pgfpathlineto{\pgfqpoint{5.319118in}{3.068924in}}%
\pgfpathlineto{\pgfqpoint{5.333378in}{3.077400in}}%
\pgfpathlineto{\pgfqpoint{5.347655in}{3.086051in}}%
\pgfpathlineto{\pgfqpoint{5.361950in}{3.094880in}}%
\pgfpathlineto{\pgfqpoint{5.369326in}{3.100093in}}%
\pgfpathlineto{\pgfqpoint{5.376698in}{3.105359in}}%
\pgfpathlineto{\pgfqpoint{5.384065in}{3.110684in}}%
\pgfpathlineto{\pgfqpoint{5.391428in}{3.116073in}}%
\pgfpathlineto{\pgfqpoint{5.377157in}{3.107779in}}%
\pgfpathlineto{\pgfqpoint{5.362902in}{3.099662in}}%
\pgfpathlineto{\pgfqpoint{5.348665in}{3.091719in}}%
\pgfpathlineto{\pgfqpoint{5.334443in}{3.083953in}}%
\pgfpathlineto{\pgfqpoint{5.327058in}{3.078019in}}%
\pgfpathlineto{\pgfqpoint{5.319668in}{3.072157in}}%
\pgfpathlineto{\pgfqpoint{5.312273in}{3.066361in}}%
\pgfpathlineto{\pgfqpoint{5.304874in}{3.060625in}}%
\pgfpathclose%
\pgfusepath{fill}%
\end{pgfscope}%
\begin{pgfscope}%
\pgfpathrectangle{\pgfqpoint{1.150000in}{0.150000in}}{\pgfqpoint{5.700000in}{5.700000in}}%
\pgfusepath{clip}%
\pgfsetbuttcap%
\pgfsetroundjoin%
\definecolor{currentfill}{rgb}{0.281446,0.084320,0.407414}%
\pgfsetfillcolor{currentfill}%
\pgfsetfillopacity{0.800000}%
\pgfsetlinewidth{0.000000pt}%
\definecolor{currentstroke}{rgb}{0.000000,0.000000,0.000000}%
\pgfsetstrokecolor{currentstroke}%
\pgfsetdash{}{0pt}%
\pgfpathmoveto{\pgfqpoint{3.036560in}{2.063623in}}%
\pgfpathlineto{\pgfqpoint{3.050146in}{2.053139in}}%
\pgfpathlineto{\pgfqpoint{3.063731in}{2.042900in}}%
\pgfpathlineto{\pgfqpoint{3.077314in}{2.032903in}}%
\pgfpathlineto{\pgfqpoint{3.090895in}{2.023147in}}%
\pgfpathlineto{\pgfqpoint{3.099171in}{2.030852in}}%
\pgfpathlineto{\pgfqpoint{3.107439in}{2.038656in}}%
\pgfpathlineto{\pgfqpoint{3.115699in}{2.046557in}}%
\pgfpathlineto{\pgfqpoint{3.123950in}{2.054553in}}%
\pgfpathlineto{\pgfqpoint{3.110390in}{2.064029in}}%
\pgfpathlineto{\pgfqpoint{3.096829in}{2.073745in}}%
\pgfpathlineto{\pgfqpoint{3.083266in}{2.083704in}}%
\pgfpathlineto{\pgfqpoint{3.069702in}{2.093908in}}%
\pgfpathlineto{\pgfqpoint{3.061429in}{2.086180in}}%
\pgfpathlineto{\pgfqpoint{3.053148in}{2.078555in}}%
\pgfpathlineto{\pgfqpoint{3.044858in}{2.071035in}}%
\pgfpathlineto{\pgfqpoint{3.036560in}{2.063623in}}%
\pgfpathclose%
\pgfusepath{fill}%
\end{pgfscope}%
\begin{pgfscope}%
\pgfpathrectangle{\pgfqpoint{1.150000in}{0.150000in}}{\pgfqpoint{5.700000in}{5.700000in}}%
\pgfusepath{clip}%
\pgfsetbuttcap%
\pgfsetroundjoin%
\definecolor{currentfill}{rgb}{0.221989,0.339161,0.548752}%
\pgfsetfillcolor{currentfill}%
\pgfsetfillopacity{0.800000}%
\pgfsetlinewidth{0.000000pt}%
\definecolor{currentstroke}{rgb}{0.000000,0.000000,0.000000}%
\pgfsetstrokecolor{currentstroke}%
\pgfsetdash{}{0pt}%
\pgfpathmoveto{\pgfqpoint{4.669027in}{2.631374in}}%
\pgfpathlineto{\pgfqpoint{4.682964in}{2.637716in}}%
\pgfpathlineto{\pgfqpoint{4.696914in}{2.644242in}}%
\pgfpathlineto{\pgfqpoint{4.710879in}{2.650950in}}%
\pgfpathlineto{\pgfqpoint{4.724858in}{2.657842in}}%
\pgfpathlineto{\pgfqpoint{4.732531in}{2.665853in}}%
\pgfpathlineto{\pgfqpoint{4.740198in}{2.673827in}}%
\pgfpathlineto{\pgfqpoint{4.747860in}{2.681768in}}%
\pgfpathlineto{\pgfqpoint{4.755515in}{2.689677in}}%
\pgfpathlineto{\pgfqpoint{4.741547in}{2.683055in}}%
\pgfpathlineto{\pgfqpoint{4.727593in}{2.676615in}}%
\pgfpathlineto{\pgfqpoint{4.713653in}{2.670358in}}%
\pgfpathlineto{\pgfqpoint{4.699727in}{2.664284in}}%
\pgfpathlineto{\pgfqpoint{4.692061in}{2.656094in}}%
\pgfpathlineto{\pgfqpoint{4.684388in}{2.647882in}}%
\pgfpathlineto{\pgfqpoint{4.676710in}{2.639643in}}%
\pgfpathlineto{\pgfqpoint{4.669027in}{2.631374in}}%
\pgfpathclose%
\pgfusepath{fill}%
\end{pgfscope}%
\begin{pgfscope}%
\pgfpathrectangle{\pgfqpoint{1.150000in}{0.150000in}}{\pgfqpoint{5.700000in}{5.700000in}}%
\pgfusepath{clip}%
\pgfsetbuttcap%
\pgfsetroundjoin%
\definecolor{currentfill}{rgb}{0.204903,0.375746,0.553533}%
\pgfsetfillcolor{currentfill}%
\pgfsetfillopacity{0.800000}%
\pgfsetlinewidth{0.000000pt}%
\definecolor{currentstroke}{rgb}{0.000000,0.000000,0.000000}%
\pgfsetstrokecolor{currentstroke}%
\pgfsetdash{}{0pt}%
\pgfpathmoveto{\pgfqpoint{2.432980in}{2.797064in}}%
\pgfpathlineto{\pgfqpoint{2.446912in}{2.773710in}}%
\pgfpathlineto{\pgfqpoint{2.460830in}{2.750709in}}%
\pgfpathlineto{\pgfqpoint{2.474734in}{2.728057in}}%
\pgfpathlineto{\pgfqpoint{2.488623in}{2.705751in}}%
\pgfpathlineto{\pgfqpoint{2.497215in}{2.710148in}}%
\pgfpathlineto{\pgfqpoint{2.505793in}{2.714736in}}%
\pgfpathlineto{\pgfqpoint{2.514358in}{2.719510in}}%
\pgfpathlineto{\pgfqpoint{2.522910in}{2.724470in}}%
\pgfpathlineto{\pgfqpoint{2.509056in}{2.746476in}}%
\pgfpathlineto{\pgfqpoint{2.495189in}{2.768826in}}%
\pgfpathlineto{\pgfqpoint{2.481307in}{2.791524in}}%
\pgfpathlineto{\pgfqpoint{2.467412in}{2.814575in}}%
\pgfpathlineto{\pgfqpoint{2.458824in}{2.809904in}}%
\pgfpathlineto{\pgfqpoint{2.450223in}{2.805427in}}%
\pgfpathlineto{\pgfqpoint{2.441608in}{2.801146in}}%
\pgfpathlineto{\pgfqpoint{2.432980in}{2.797064in}}%
\pgfpathclose%
\pgfusepath{fill}%
\end{pgfscope}%
\begin{pgfscope}%
\pgfpathrectangle{\pgfqpoint{1.150000in}{0.150000in}}{\pgfqpoint{5.700000in}{5.700000in}}%
\pgfusepath{clip}%
\pgfsetbuttcap%
\pgfsetroundjoin%
\definecolor{currentfill}{rgb}{0.283229,0.120777,0.440584}%
\pgfsetfillcolor{currentfill}%
\pgfsetfillopacity{0.800000}%
\pgfsetlinewidth{0.000000pt}%
\definecolor{currentstroke}{rgb}{0.000000,0.000000,0.000000}%
\pgfsetstrokecolor{currentstroke}%
\pgfsetdash{}{0pt}%
\pgfpathmoveto{\pgfqpoint{3.860379in}{2.114362in}}%
\pgfpathlineto{\pgfqpoint{3.874003in}{2.114756in}}%
\pgfpathlineto{\pgfqpoint{3.887633in}{2.115347in}}%
\pgfpathlineto{\pgfqpoint{3.901272in}{2.116136in}}%
\pgfpathlineto{\pgfqpoint{3.914919in}{2.117123in}}%
\pgfpathlineto{\pgfqpoint{3.922879in}{2.127702in}}%
\pgfpathlineto{\pgfqpoint{3.930835in}{2.138255in}}%
\pgfpathlineto{\pgfqpoint{3.938786in}{2.148784in}}%
\pgfpathlineto{\pgfqpoint{3.946732in}{2.159288in}}%
\pgfpathlineto{\pgfqpoint{3.933093in}{2.158246in}}%
\pgfpathlineto{\pgfqpoint{3.919462in}{2.157401in}}%
\pgfpathlineto{\pgfqpoint{3.905839in}{2.156754in}}%
\pgfpathlineto{\pgfqpoint{3.892223in}{2.156306in}}%
\pgfpathlineto{\pgfqpoint{3.884270in}{2.145846in}}%
\pgfpathlineto{\pgfqpoint{3.876312in}{2.135368in}}%
\pgfpathlineto{\pgfqpoint{3.868348in}{2.124874in}}%
\pgfpathlineto{\pgfqpoint{3.860379in}{2.114362in}}%
\pgfpathclose%
\pgfusepath{fill}%
\end{pgfscope}%
\begin{pgfscope}%
\pgfpathrectangle{\pgfqpoint{1.150000in}{0.150000in}}{\pgfqpoint{5.700000in}{5.700000in}}%
\pgfusepath{clip}%
\pgfsetbuttcap%
\pgfsetroundjoin%
\definecolor{currentfill}{rgb}{0.282623,0.140926,0.457517}%
\pgfsetfillcolor{currentfill}%
\pgfsetfillopacity{0.800000}%
\pgfsetlinewidth{0.000000pt}%
\definecolor{currentstroke}{rgb}{0.000000,0.000000,0.000000}%
\pgfsetstrokecolor{currentstroke}%
\pgfsetdash{}{0pt}%
\pgfpathmoveto{\pgfqpoint{3.946732in}{2.159288in}}%
\pgfpathlineto{\pgfqpoint{3.960379in}{2.160526in}}%
\pgfpathlineto{\pgfqpoint{3.974035in}{2.161960in}}%
\pgfpathlineto{\pgfqpoint{3.987700in}{2.163590in}}%
\pgfpathlineto{\pgfqpoint{4.001374in}{2.165414in}}%
\pgfpathlineto{\pgfqpoint{4.009307in}{2.175928in}}%
\pgfpathlineto{\pgfqpoint{4.017236in}{2.186409in}}%
\pgfpathlineto{\pgfqpoint{4.025159in}{2.196857in}}%
\pgfpathlineto{\pgfqpoint{4.033077in}{2.207272in}}%
\pgfpathlineto{\pgfqpoint{4.019411in}{2.205424in}}%
\pgfpathlineto{\pgfqpoint{4.005753in}{2.203771in}}%
\pgfpathlineto{\pgfqpoint{3.992104in}{2.202313in}}%
\pgfpathlineto{\pgfqpoint{3.978464in}{2.201051in}}%
\pgfpathlineto{\pgfqpoint{3.970539in}{2.190648in}}%
\pgfpathlineto{\pgfqpoint{3.962608in}{2.180219in}}%
\pgfpathlineto{\pgfqpoint{3.954673in}{2.169766in}}%
\pgfpathlineto{\pgfqpoint{3.946732in}{2.159288in}}%
\pgfpathclose%
\pgfusepath{fill}%
\end{pgfscope}%
\begin{pgfscope}%
\pgfpathrectangle{\pgfqpoint{1.150000in}{0.150000in}}{\pgfqpoint{5.700000in}{5.700000in}}%
\pgfusepath{clip}%
\pgfsetbuttcap%
\pgfsetroundjoin%
\definecolor{currentfill}{rgb}{0.149039,0.508051,0.557250}%
\pgfsetfillcolor{currentfill}%
\pgfsetfillopacity{0.800000}%
\pgfsetlinewidth{0.000000pt}%
\definecolor{currentstroke}{rgb}{0.000000,0.000000,0.000000}%
\pgfsetstrokecolor{currentstroke}%
\pgfsetdash{}{0pt}%
\pgfpathmoveto{\pgfqpoint{5.391428in}{3.116073in}}%
\pgfpathlineto{\pgfqpoint{5.405717in}{3.124542in}}%
\pgfpathlineto{\pgfqpoint{5.420023in}{3.133187in}}%
\pgfpathlineto{\pgfqpoint{5.434346in}{3.142007in}}%
\pgfpathlineto{\pgfqpoint{5.448686in}{3.151003in}}%
\pgfpathlineto{\pgfqpoint{5.456021in}{3.155907in}}%
\pgfpathlineto{\pgfqpoint{5.463351in}{3.160881in}}%
\pgfpathlineto{\pgfqpoint{5.470677in}{3.165929in}}%
\pgfpathlineto{\pgfqpoint{5.477999in}{3.171060in}}%
\pgfpathlineto{\pgfqpoint{5.463684in}{3.162632in}}%
\pgfpathlineto{\pgfqpoint{5.449385in}{3.154379in}}%
\pgfpathlineto{\pgfqpoint{5.435104in}{3.146301in}}%
\pgfpathlineto{\pgfqpoint{5.420840in}{3.138398in}}%
\pgfpathlineto{\pgfqpoint{5.413493in}{3.132689in}}%
\pgfpathlineto{\pgfqpoint{5.406142in}{3.127069in}}%
\pgfpathlineto{\pgfqpoint{5.398787in}{3.121533in}}%
\pgfpathlineto{\pgfqpoint{5.391428in}{3.116073in}}%
\pgfpathclose%
\pgfusepath{fill}%
\end{pgfscope}%
\begin{pgfscope}%
\pgfpathrectangle{\pgfqpoint{1.150000in}{0.150000in}}{\pgfqpoint{5.700000in}{5.700000in}}%
\pgfusepath{clip}%
\pgfsetbuttcap%
\pgfsetroundjoin%
\definecolor{currentfill}{rgb}{0.282910,0.105393,0.426902}%
\pgfsetfillcolor{currentfill}%
\pgfsetfillopacity{0.800000}%
\pgfsetlinewidth{0.000000pt}%
\definecolor{currentstroke}{rgb}{0.000000,0.000000,0.000000}%
\pgfsetstrokecolor{currentstroke}%
\pgfsetdash{}{0pt}%
\pgfpathmoveto{\pgfqpoint{3.774002in}{2.072921in}}%
\pgfpathlineto{\pgfqpoint{3.787605in}{2.072428in}}%
\pgfpathlineto{\pgfqpoint{3.801214in}{2.072136in}}%
\pgfpathlineto{\pgfqpoint{3.814831in}{2.072044in}}%
\pgfpathlineto{\pgfqpoint{3.828454in}{2.072151in}}%
\pgfpathlineto{\pgfqpoint{3.836443in}{2.082728in}}%
\pgfpathlineto{\pgfqpoint{3.844427in}{2.093289in}}%
\pgfpathlineto{\pgfqpoint{3.852406in}{2.103834in}}%
\pgfpathlineto{\pgfqpoint{3.860379in}{2.114362in}}%
\pgfpathlineto{\pgfqpoint{3.846764in}{2.114168in}}%
\pgfpathlineto{\pgfqpoint{3.833156in}{2.114173in}}%
\pgfpathlineto{\pgfqpoint{3.819555in}{2.114378in}}%
\pgfpathlineto{\pgfqpoint{3.805961in}{2.114784in}}%
\pgfpathlineto{\pgfqpoint{3.797979in}{2.104331in}}%
\pgfpathlineto{\pgfqpoint{3.789992in}{2.093869in}}%
\pgfpathlineto{\pgfqpoint{3.782000in}{2.083399in}}%
\pgfpathlineto{\pgfqpoint{3.774002in}{2.072921in}}%
\pgfpathclose%
\pgfusepath{fill}%
\end{pgfscope}%
\begin{pgfscope}%
\pgfpathrectangle{\pgfqpoint{1.150000in}{0.150000in}}{\pgfqpoint{5.700000in}{5.700000in}}%
\pgfusepath{clip}%
\pgfsetbuttcap%
\pgfsetroundjoin%
\definecolor{currentfill}{rgb}{0.280255,0.165693,0.476498}%
\pgfsetfillcolor{currentfill}%
\pgfsetfillopacity{0.800000}%
\pgfsetlinewidth{0.000000pt}%
\definecolor{currentstroke}{rgb}{0.000000,0.000000,0.000000}%
\pgfsetstrokecolor{currentstroke}%
\pgfsetdash{}{0pt}%
\pgfpathmoveto{\pgfqpoint{4.033077in}{2.207272in}}%
\pgfpathlineto{\pgfqpoint{4.046753in}{2.209315in}}%
\pgfpathlineto{\pgfqpoint{4.060438in}{2.211551in}}%
\pgfpathlineto{\pgfqpoint{4.074132in}{2.213981in}}%
\pgfpathlineto{\pgfqpoint{4.087836in}{2.216603in}}%
\pgfpathlineto{\pgfqpoint{4.095742in}{2.226990in}}%
\pgfpathlineto{\pgfqpoint{4.103643in}{2.237337in}}%
\pgfpathlineto{\pgfqpoint{4.111539in}{2.247645in}}%
\pgfpathlineto{\pgfqpoint{4.119430in}{2.257914in}}%
\pgfpathlineto{\pgfqpoint{4.105733in}{2.255300in}}%
\pgfpathlineto{\pgfqpoint{4.092046in}{2.252879in}}%
\pgfpathlineto{\pgfqpoint{4.078368in}{2.250651in}}%
\pgfpathlineto{\pgfqpoint{4.064699in}{2.248617in}}%
\pgfpathlineto{\pgfqpoint{4.056802in}{2.238327in}}%
\pgfpathlineto{\pgfqpoint{4.048899in}{2.228007in}}%
\pgfpathlineto{\pgfqpoint{4.040990in}{2.217655in}}%
\pgfpathlineto{\pgfqpoint{4.033077in}{2.207272in}}%
\pgfpathclose%
\pgfusepath{fill}%
\end{pgfscope}%
\begin{pgfscope}%
\pgfpathrectangle{\pgfqpoint{1.150000in}{0.150000in}}{\pgfqpoint{5.700000in}{5.700000in}}%
\pgfusepath{clip}%
\pgfsetbuttcap%
\pgfsetroundjoin%
\definecolor{currentfill}{rgb}{0.277018,0.050344,0.375715}%
\pgfsetfillcolor{currentfill}%
\pgfsetfillopacity{0.800000}%
\pgfsetlinewidth{0.000000pt}%
\definecolor{currentstroke}{rgb}{0.000000,0.000000,0.000000}%
\pgfsetstrokecolor{currentstroke}%
\pgfsetdash{}{0pt}%
\pgfpathmoveto{\pgfqpoint{3.232410in}{1.987252in}}%
\pgfpathlineto{\pgfqpoint{3.245967in}{1.979882in}}%
\pgfpathlineto{\pgfqpoint{3.259526in}{1.972739in}}%
\pgfpathlineto{\pgfqpoint{3.273086in}{1.965822in}}%
\pgfpathlineto{\pgfqpoint{3.286647in}{1.959129in}}%
\pgfpathlineto{\pgfqpoint{3.294835in}{1.967994in}}%
\pgfpathlineto{\pgfqpoint{3.303016in}{1.976925in}}%
\pgfpathlineto{\pgfqpoint{3.311190in}{1.985921in}}%
\pgfpathlineto{\pgfqpoint{3.319357in}{1.994978in}}%
\pgfpathlineto{\pgfqpoint{3.305813in}{2.001425in}}%
\pgfpathlineto{\pgfqpoint{3.292271in}{2.008096in}}%
\pgfpathlineto{\pgfqpoint{3.278729in}{2.014993in}}%
\pgfpathlineto{\pgfqpoint{3.265189in}{2.022117in}}%
\pgfpathlineto{\pgfqpoint{3.257005in}{2.013293in}}%
\pgfpathlineto{\pgfqpoint{3.248813in}{2.004540in}}%
\pgfpathlineto{\pgfqpoint{3.240615in}{1.995859in}}%
\pgfpathlineto{\pgfqpoint{3.232410in}{1.987252in}}%
\pgfpathclose%
\pgfusepath{fill}%
\end{pgfscope}%
\begin{pgfscope}%
\pgfpathrectangle{\pgfqpoint{1.150000in}{0.150000in}}{\pgfqpoint{5.700000in}{5.700000in}}%
\pgfusepath{clip}%
\pgfsetbuttcap%
\pgfsetroundjoin%
\definecolor{currentfill}{rgb}{0.276194,0.190074,0.493001}%
\pgfsetfillcolor{currentfill}%
\pgfsetfillopacity{0.800000}%
\pgfsetlinewidth{0.000000pt}%
\definecolor{currentstroke}{rgb}{0.000000,0.000000,0.000000}%
\pgfsetstrokecolor{currentstroke}%
\pgfsetdash{}{0pt}%
\pgfpathmoveto{\pgfqpoint{4.119430in}{2.257914in}}%
\pgfpathlineto{\pgfqpoint{4.133137in}{2.260721in}}%
\pgfpathlineto{\pgfqpoint{4.146854in}{2.263720in}}%
\pgfpathlineto{\pgfqpoint{4.160581in}{2.266910in}}%
\pgfpathlineto{\pgfqpoint{4.174318in}{2.270291in}}%
\pgfpathlineto{\pgfqpoint{4.182198in}{2.280494in}}%
\pgfpathlineto{\pgfqpoint{4.190071in}{2.290652in}}%
\pgfpathlineto{\pgfqpoint{4.197940in}{2.300765in}}%
\pgfpathlineto{\pgfqpoint{4.205803in}{2.310836in}}%
\pgfpathlineto{\pgfqpoint{4.192073in}{2.307495in}}%
\pgfpathlineto{\pgfqpoint{4.178353in}{2.304345in}}%
\pgfpathlineto{\pgfqpoint{4.164643in}{2.301387in}}%
\pgfpathlineto{\pgfqpoint{4.150943in}{2.298621in}}%
\pgfpathlineto{\pgfqpoint{4.143072in}{2.288498in}}%
\pgfpathlineto{\pgfqpoint{4.135197in}{2.278340in}}%
\pgfpathlineto{\pgfqpoint{4.127316in}{2.268146in}}%
\pgfpathlineto{\pgfqpoint{4.119430in}{2.257914in}}%
\pgfpathclose%
\pgfusepath{fill}%
\end{pgfscope}%
\begin{pgfscope}%
\pgfpathrectangle{\pgfqpoint{1.150000in}{0.150000in}}{\pgfqpoint{5.700000in}{5.700000in}}%
\pgfusepath{clip}%
\pgfsetbuttcap%
\pgfsetroundjoin%
\definecolor{currentfill}{rgb}{0.281446,0.084320,0.407414}%
\pgfsetfillcolor{currentfill}%
\pgfsetfillopacity{0.800000}%
\pgfsetlinewidth{0.000000pt}%
\definecolor{currentstroke}{rgb}{0.000000,0.000000,0.000000}%
\pgfsetstrokecolor{currentstroke}%
\pgfsetdash{}{0pt}%
\pgfpathmoveto{\pgfqpoint{3.687580in}{2.035414in}}%
\pgfpathlineto{\pgfqpoint{3.701166in}{2.033992in}}%
\pgfpathlineto{\pgfqpoint{3.714758in}{2.032774in}}%
\pgfpathlineto{\pgfqpoint{3.728357in}{2.031758in}}%
\pgfpathlineto{\pgfqpoint{3.741961in}{2.030945in}}%
\pgfpathlineto{\pgfqpoint{3.749979in}{2.041447in}}%
\pgfpathlineto{\pgfqpoint{3.757992in}{2.051945in}}%
\pgfpathlineto{\pgfqpoint{3.766000in}{2.062436in}}%
\pgfpathlineto{\pgfqpoint{3.774002in}{2.072921in}}%
\pgfpathlineto{\pgfqpoint{3.760407in}{2.073616in}}%
\pgfpathlineto{\pgfqpoint{3.746818in}{2.074513in}}%
\pgfpathlineto{\pgfqpoint{3.733235in}{2.075613in}}%
\pgfpathlineto{\pgfqpoint{3.719659in}{2.076916in}}%
\pgfpathlineto{\pgfqpoint{3.711647in}{2.066538in}}%
\pgfpathlineto{\pgfqpoint{3.703630in}{2.056161in}}%
\pgfpathlineto{\pgfqpoint{3.695608in}{2.045786in}}%
\pgfpathlineto{\pgfqpoint{3.687580in}{2.035414in}}%
\pgfpathclose%
\pgfusepath{fill}%
\end{pgfscope}%
\begin{pgfscope}%
\pgfpathrectangle{\pgfqpoint{1.150000in}{0.150000in}}{\pgfqpoint{5.700000in}{5.700000in}}%
\pgfusepath{clip}%
\pgfsetbuttcap%
\pgfsetroundjoin%
\definecolor{currentfill}{rgb}{0.141935,0.526453,0.555991}%
\pgfsetfillcolor{currentfill}%
\pgfsetfillopacity{0.800000}%
\pgfsetlinewidth{0.000000pt}%
\definecolor{currentstroke}{rgb}{0.000000,0.000000,0.000000}%
\pgfsetstrokecolor{currentstroke}%
\pgfsetdash{}{0pt}%
\pgfpathmoveto{\pgfqpoint{5.477999in}{3.171060in}}%
\pgfpathlineto{\pgfqpoint{5.492333in}{3.179663in}}%
\pgfpathlineto{\pgfqpoint{5.506683in}{3.188440in}}%
\pgfpathlineto{\pgfqpoint{5.521051in}{3.197393in}}%
\pgfpathlineto{\pgfqpoint{5.535438in}{3.206521in}}%
\pgfpathlineto{\pgfqpoint{5.542730in}{3.211150in}}%
\pgfpathlineto{\pgfqpoint{5.550019in}{3.215867in}}%
\pgfpathlineto{\pgfqpoint{5.557304in}{3.220678in}}%
\pgfpathlineto{\pgfqpoint{5.564586in}{3.225590in}}%
\pgfpathlineto{\pgfqpoint{5.550227in}{3.217064in}}%
\pgfpathlineto{\pgfqpoint{5.535885in}{3.208712in}}%
\pgfpathlineto{\pgfqpoint{5.521561in}{3.200534in}}%
\pgfpathlineto{\pgfqpoint{5.507254in}{3.192531in}}%
\pgfpathlineto{\pgfqpoint{5.499945in}{3.187008in}}%
\pgfpathlineto{\pgfqpoint{5.492633in}{3.181593in}}%
\pgfpathlineto{\pgfqpoint{5.485318in}{3.176279in}}%
\pgfpathlineto{\pgfqpoint{5.477999in}{3.171060in}}%
\pgfpathclose%
\pgfusepath{fill}%
\end{pgfscope}%
\begin{pgfscope}%
\pgfpathrectangle{\pgfqpoint{1.150000in}{0.150000in}}{\pgfqpoint{5.700000in}{5.700000in}}%
\pgfusepath{clip}%
\pgfsetbuttcap%
\pgfsetroundjoin%
\definecolor{currentfill}{rgb}{0.210503,0.363727,0.552206}%
\pgfsetfillcolor{currentfill}%
\pgfsetfillopacity{0.800000}%
\pgfsetlinewidth{0.000000pt}%
\definecolor{currentstroke}{rgb}{0.000000,0.000000,0.000000}%
\pgfsetstrokecolor{currentstroke}%
\pgfsetdash{}{0pt}%
\pgfpathmoveto{\pgfqpoint{4.755515in}{2.689677in}}%
\pgfpathlineto{\pgfqpoint{4.769497in}{2.696482in}}%
\pgfpathlineto{\pgfqpoint{4.783494in}{2.703469in}}%
\pgfpathlineto{\pgfqpoint{4.797504in}{2.710638in}}%
\pgfpathlineto{\pgfqpoint{4.811530in}{2.717989in}}%
\pgfpathlineto{\pgfqpoint{4.819168in}{2.725581in}}%
\pgfpathlineto{\pgfqpoint{4.826800in}{2.733141in}}%
\pgfpathlineto{\pgfqpoint{4.834426in}{2.740672in}}%
\pgfpathlineto{\pgfqpoint{4.842046in}{2.748179in}}%
\pgfpathlineto{\pgfqpoint{4.828032in}{2.741130in}}%
\pgfpathlineto{\pgfqpoint{4.814033in}{2.734263in}}%
\pgfpathlineto{\pgfqpoint{4.800049in}{2.727577in}}%
\pgfpathlineto{\pgfqpoint{4.786078in}{2.721074in}}%
\pgfpathlineto{\pgfqpoint{4.778446in}{2.713254in}}%
\pgfpathlineto{\pgfqpoint{4.770809in}{2.705416in}}%
\pgfpathlineto{\pgfqpoint{4.763165in}{2.697559in}}%
\pgfpathlineto{\pgfqpoint{4.755515in}{2.689677in}}%
\pgfpathclose%
\pgfusepath{fill}%
\end{pgfscope}%
\begin{pgfscope}%
\pgfpathrectangle{\pgfqpoint{1.150000in}{0.150000in}}{\pgfqpoint{5.700000in}{5.700000in}}%
\pgfusepath{clip}%
\pgfsetbuttcap%
\pgfsetroundjoin%
\definecolor{currentfill}{rgb}{0.277018,0.050344,0.375715}%
\pgfsetfillcolor{currentfill}%
\pgfsetfillopacity{0.800000}%
\pgfsetlinewidth{0.000000pt}%
\definecolor{currentstroke}{rgb}{0.000000,0.000000,0.000000}%
\pgfsetstrokecolor{currentstroke}%
\pgfsetdash{}{0pt}%
\pgfpathmoveto{\pgfqpoint{3.373552in}{1.971417in}}%
\pgfpathlineto{\pgfqpoint{3.387106in}{1.966076in}}%
\pgfpathlineto{\pgfqpoint{3.400663in}{1.960954in}}%
\pgfpathlineto{\pgfqpoint{3.414222in}{1.956049in}}%
\pgfpathlineto{\pgfqpoint{3.427785in}{1.951359in}}%
\pgfpathlineto{\pgfqpoint{3.435915in}{1.960928in}}%
\pgfpathlineto{\pgfqpoint{3.444040in}{1.970541in}}%
\pgfpathlineto{\pgfqpoint{3.452158in}{1.980193in}}%
\pgfpathlineto{\pgfqpoint{3.460271in}{1.989885in}}%
\pgfpathlineto{\pgfqpoint{3.446722in}{1.994361in}}%
\pgfpathlineto{\pgfqpoint{3.433177in}{1.999053in}}%
\pgfpathlineto{\pgfqpoint{3.419635in}{2.003961in}}%
\pgfpathlineto{\pgfqpoint{3.406095in}{2.009088in}}%
\pgfpathlineto{\pgfqpoint{3.397969in}{1.999598in}}%
\pgfpathlineto{\pgfqpoint{3.389836in}{1.990155in}}%
\pgfpathlineto{\pgfqpoint{3.381697in}{1.980760in}}%
\pgfpathlineto{\pgfqpoint{3.373552in}{1.971417in}}%
\pgfpathclose%
\pgfusepath{fill}%
\end{pgfscope}%
\begin{pgfscope}%
\pgfpathrectangle{\pgfqpoint{1.150000in}{0.150000in}}{\pgfqpoint{5.700000in}{5.700000in}}%
\pgfusepath{clip}%
\pgfsetbuttcap%
\pgfsetroundjoin%
\definecolor{currentfill}{rgb}{0.276194,0.190074,0.493001}%
\pgfsetfillcolor{currentfill}%
\pgfsetfillopacity{0.800000}%
\pgfsetlinewidth{0.000000pt}%
\definecolor{currentstroke}{rgb}{0.000000,0.000000,0.000000}%
\pgfsetstrokecolor{currentstroke}%
\pgfsetdash{}{0pt}%
\pgfpathmoveto{\pgfqpoint{2.730227in}{2.305155in}}%
\pgfpathlineto{\pgfqpoint{2.743933in}{2.289081in}}%
\pgfpathlineto{\pgfqpoint{2.757633in}{2.273290in}}%
\pgfpathlineto{\pgfqpoint{2.771325in}{2.257780in}}%
\pgfpathlineto{\pgfqpoint{2.785011in}{2.242550in}}%
\pgfpathlineto{\pgfqpoint{2.793452in}{2.248192in}}%
\pgfpathlineto{\pgfqpoint{2.801883in}{2.253986in}}%
\pgfpathlineto{\pgfqpoint{2.810303in}{2.259927in}}%
\pgfpathlineto{\pgfqpoint{2.818713in}{2.266015in}}%
\pgfpathlineto{\pgfqpoint{2.805056in}{2.280926in}}%
\pgfpathlineto{\pgfqpoint{2.791393in}{2.296115in}}%
\pgfpathlineto{\pgfqpoint{2.777724in}{2.311585in}}%
\pgfpathlineto{\pgfqpoint{2.764047in}{2.327339in}}%
\pgfpathlineto{\pgfqpoint{2.755609in}{2.321559in}}%
\pgfpathlineto{\pgfqpoint{2.747159in}{2.315934in}}%
\pgfpathlineto{\pgfqpoint{2.738698in}{2.310465in}}%
\pgfpathlineto{\pgfqpoint{2.730227in}{2.305155in}}%
\pgfpathclose%
\pgfusepath{fill}%
\end{pgfscope}%
\begin{pgfscope}%
\pgfpathrectangle{\pgfqpoint{1.150000in}{0.150000in}}{\pgfqpoint{5.700000in}{5.700000in}}%
\pgfusepath{clip}%
\pgfsetbuttcap%
\pgfsetroundjoin%
\definecolor{currentfill}{rgb}{0.269308,0.218818,0.509577}%
\pgfsetfillcolor{currentfill}%
\pgfsetfillopacity{0.800000}%
\pgfsetlinewidth{0.000000pt}%
\definecolor{currentstroke}{rgb}{0.000000,0.000000,0.000000}%
\pgfsetstrokecolor{currentstroke}%
\pgfsetdash{}{0pt}%
\pgfpathmoveto{\pgfqpoint{2.675322in}{2.372334in}}%
\pgfpathlineto{\pgfqpoint{2.689060in}{2.355102in}}%
\pgfpathlineto{\pgfqpoint{2.702790in}{2.338163in}}%
\pgfpathlineto{\pgfqpoint{2.716512in}{2.321515in}}%
\pgfpathlineto{\pgfqpoint{2.730227in}{2.305155in}}%
\pgfpathlineto{\pgfqpoint{2.738698in}{2.310465in}}%
\pgfpathlineto{\pgfqpoint{2.747159in}{2.315934in}}%
\pgfpathlineto{\pgfqpoint{2.755609in}{2.321559in}}%
\pgfpathlineto{\pgfqpoint{2.764047in}{2.327339in}}%
\pgfpathlineto{\pgfqpoint{2.750364in}{2.343377in}}%
\pgfpathlineto{\pgfqpoint{2.736672in}{2.359703in}}%
\pgfpathlineto{\pgfqpoint{2.722974in}{2.376319in}}%
\pgfpathlineto{\pgfqpoint{2.709267in}{2.393228in}}%
\pgfpathlineto{\pgfqpoint{2.700798in}{2.387759in}}%
\pgfpathlineto{\pgfqpoint{2.692318in}{2.382451in}}%
\pgfpathlineto{\pgfqpoint{2.683826in}{2.377309in}}%
\pgfpathlineto{\pgfqpoint{2.675322in}{2.372334in}}%
\pgfpathclose%
\pgfusepath{fill}%
\end{pgfscope}%
\begin{pgfscope}%
\pgfpathrectangle{\pgfqpoint{1.150000in}{0.150000in}}{\pgfqpoint{5.700000in}{5.700000in}}%
\pgfusepath{clip}%
\pgfsetbuttcap%
\pgfsetroundjoin%
\definecolor{currentfill}{rgb}{0.280255,0.165693,0.476498}%
\pgfsetfillcolor{currentfill}%
\pgfsetfillopacity{0.800000}%
\pgfsetlinewidth{0.000000pt}%
\definecolor{currentstroke}{rgb}{0.000000,0.000000,0.000000}%
\pgfsetstrokecolor{currentstroke}%
\pgfsetdash{}{0pt}%
\pgfpathmoveto{\pgfqpoint{2.785011in}{2.242550in}}%
\pgfpathlineto{\pgfqpoint{2.798690in}{2.227596in}}%
\pgfpathlineto{\pgfqpoint{2.812363in}{2.212917in}}%
\pgfpathlineto{\pgfqpoint{2.826030in}{2.198510in}}%
\pgfpathlineto{\pgfqpoint{2.839692in}{2.184373in}}%
\pgfpathlineto{\pgfqpoint{2.848104in}{2.190346in}}%
\pgfpathlineto{\pgfqpoint{2.856507in}{2.196462in}}%
\pgfpathlineto{\pgfqpoint{2.864899in}{2.202718in}}%
\pgfpathlineto{\pgfqpoint{2.873281in}{2.209111in}}%
\pgfpathlineto{\pgfqpoint{2.859647in}{2.222930in}}%
\pgfpathlineto{\pgfqpoint{2.846008in}{2.237019in}}%
\pgfpathlineto{\pgfqpoint{2.832363in}{2.251380in}}%
\pgfpathlineto{\pgfqpoint{2.818713in}{2.266015in}}%
\pgfpathlineto{\pgfqpoint{2.810303in}{2.259927in}}%
\pgfpathlineto{\pgfqpoint{2.801883in}{2.253986in}}%
\pgfpathlineto{\pgfqpoint{2.793452in}{2.248192in}}%
\pgfpathlineto{\pgfqpoint{2.785011in}{2.242550in}}%
\pgfpathclose%
\pgfusepath{fill}%
\end{pgfscope}%
\begin{pgfscope}%
\pgfpathrectangle{\pgfqpoint{1.150000in}{0.150000in}}{\pgfqpoint{5.700000in}{5.700000in}}%
\pgfusepath{clip}%
\pgfsetbuttcap%
\pgfsetroundjoin%
\definecolor{currentfill}{rgb}{0.270595,0.214069,0.507052}%
\pgfsetfillcolor{currentfill}%
\pgfsetfillopacity{0.800000}%
\pgfsetlinewidth{0.000000pt}%
\definecolor{currentstroke}{rgb}{0.000000,0.000000,0.000000}%
\pgfsetstrokecolor{currentstroke}%
\pgfsetdash{}{0pt}%
\pgfpathmoveto{\pgfqpoint{4.205803in}{2.310836in}}%
\pgfpathlineto{\pgfqpoint{4.219545in}{2.314367in}}%
\pgfpathlineto{\pgfqpoint{4.233296in}{2.318089in}}%
\pgfpathlineto{\pgfqpoint{4.247059in}{2.322000in}}%
\pgfpathlineto{\pgfqpoint{4.260833in}{2.326101in}}%
\pgfpathlineto{\pgfqpoint{4.268684in}{2.336069in}}%
\pgfpathlineto{\pgfqpoint{4.276530in}{2.345987in}}%
\pgfpathlineto{\pgfqpoint{4.284371in}{2.355857in}}%
\pgfpathlineto{\pgfqpoint{4.292206in}{2.365680in}}%
\pgfpathlineto{\pgfqpoint{4.278439in}{2.361652in}}%
\pgfpathlineto{\pgfqpoint{4.264684in}{2.357813in}}%
\pgfpathlineto{\pgfqpoint{4.250939in}{2.354164in}}%
\pgfpathlineto{\pgfqpoint{4.237204in}{2.350705in}}%
\pgfpathlineto{\pgfqpoint{4.229362in}{2.340797in}}%
\pgfpathlineto{\pgfqpoint{4.221514in}{2.330851in}}%
\pgfpathlineto{\pgfqpoint{4.213662in}{2.320864in}}%
\pgfpathlineto{\pgfqpoint{4.205803in}{2.310836in}}%
\pgfpathclose%
\pgfusepath{fill}%
\end{pgfscope}%
\begin{pgfscope}%
\pgfpathrectangle{\pgfqpoint{1.150000in}{0.150000in}}{\pgfqpoint{5.700000in}{5.700000in}}%
\pgfusepath{clip}%
\pgfsetbuttcap%
\pgfsetroundjoin%
\definecolor{currentfill}{rgb}{0.135066,0.544853,0.554029}%
\pgfsetfillcolor{currentfill}%
\pgfsetfillopacity{0.800000}%
\pgfsetlinewidth{0.000000pt}%
\definecolor{currentstroke}{rgb}{0.000000,0.000000,0.000000}%
\pgfsetstrokecolor{currentstroke}%
\pgfsetdash{}{0pt}%
\pgfpathmoveto{\pgfqpoint{5.564586in}{3.225590in}}%
\pgfpathlineto{\pgfqpoint{5.578962in}{3.234290in}}%
\pgfpathlineto{\pgfqpoint{5.593357in}{3.243164in}}%
\pgfpathlineto{\pgfqpoint{5.607770in}{3.252212in}}%
\pgfpathlineto{\pgfqpoint{5.622201in}{3.261435in}}%
\pgfpathlineto{\pgfqpoint{5.629451in}{3.265831in}}%
\pgfpathlineto{\pgfqpoint{5.636699in}{3.270334in}}%
\pgfpathlineto{\pgfqpoint{5.643943in}{3.274951in}}%
\pgfpathlineto{\pgfqpoint{5.651185in}{3.279688in}}%
\pgfpathlineto{\pgfqpoint{5.636784in}{3.271100in}}%
\pgfpathlineto{\pgfqpoint{5.622400in}{3.262686in}}%
\pgfpathlineto{\pgfqpoint{5.608034in}{3.254445in}}%
\pgfpathlineto{\pgfqpoint{5.593686in}{3.246377in}}%
\pgfpathlineto{\pgfqpoint{5.586414in}{3.240995in}}%
\pgfpathlineto{\pgfqpoint{5.579141in}{3.235741in}}%
\pgfpathlineto{\pgfqpoint{5.571865in}{3.230609in}}%
\pgfpathlineto{\pgfqpoint{5.564586in}{3.225590in}}%
\pgfpathclose%
\pgfusepath{fill}%
\end{pgfscope}%
\begin{pgfscope}%
\pgfpathrectangle{\pgfqpoint{1.150000in}{0.150000in}}{\pgfqpoint{5.700000in}{5.700000in}}%
\pgfusepath{clip}%
\pgfsetbuttcap%
\pgfsetroundjoin%
\definecolor{currentfill}{rgb}{0.279566,0.067836,0.391917}%
\pgfsetfillcolor{currentfill}%
\pgfsetfillopacity{0.800000}%
\pgfsetlinewidth{0.000000pt}%
\definecolor{currentstroke}{rgb}{0.000000,0.000000,0.000000}%
\pgfsetstrokecolor{currentstroke}%
\pgfsetdash{}{0pt}%
\pgfpathmoveto{\pgfqpoint{3.090895in}{2.023147in}}%
\pgfpathlineto{\pgfqpoint{3.104475in}{2.013632in}}%
\pgfpathlineto{\pgfqpoint{3.118054in}{2.004354in}}%
\pgfpathlineto{\pgfqpoint{3.131632in}{1.995314in}}%
\pgfpathlineto{\pgfqpoint{3.145209in}{1.986509in}}%
\pgfpathlineto{\pgfqpoint{3.153464in}{1.994504in}}%
\pgfpathlineto{\pgfqpoint{3.161711in}{2.002592in}}%
\pgfpathlineto{\pgfqpoint{3.169950in}{2.010768in}}%
\pgfpathlineto{\pgfqpoint{3.178182in}{2.019032in}}%
\pgfpathlineto{\pgfqpoint{3.164625in}{2.027558in}}%
\pgfpathlineto{\pgfqpoint{3.151067in}{2.036319in}}%
\pgfpathlineto{\pgfqpoint{3.137509in}{2.045317in}}%
\pgfpathlineto{\pgfqpoint{3.123950in}{2.054553in}}%
\pgfpathlineto{\pgfqpoint{3.115699in}{2.046557in}}%
\pgfpathlineto{\pgfqpoint{3.107439in}{2.038656in}}%
\pgfpathlineto{\pgfqpoint{3.099171in}{2.030852in}}%
\pgfpathlineto{\pgfqpoint{3.090895in}{2.023147in}}%
\pgfpathclose%
\pgfusepath{fill}%
\end{pgfscope}%
\begin{pgfscope}%
\pgfpathrectangle{\pgfqpoint{1.150000in}{0.150000in}}{\pgfqpoint{5.700000in}{5.700000in}}%
\pgfusepath{clip}%
\pgfsetbuttcap%
\pgfsetroundjoin%
\definecolor{currentfill}{rgb}{0.260571,0.246922,0.522828}%
\pgfsetfillcolor{currentfill}%
\pgfsetfillopacity{0.800000}%
\pgfsetlinewidth{0.000000pt}%
\definecolor{currentstroke}{rgb}{0.000000,0.000000,0.000000}%
\pgfsetstrokecolor{currentstroke}%
\pgfsetdash{}{0pt}%
\pgfpathmoveto{\pgfqpoint{2.620279in}{2.444244in}}%
\pgfpathlineto{\pgfqpoint{2.634054in}{2.425814in}}%
\pgfpathlineto{\pgfqpoint{2.647819in}{2.407687in}}%
\pgfpathlineto{\pgfqpoint{2.661575in}{2.389861in}}%
\pgfpathlineto{\pgfqpoint{2.675322in}{2.372334in}}%
\pgfpathlineto{\pgfqpoint{2.683826in}{2.377309in}}%
\pgfpathlineto{\pgfqpoint{2.692318in}{2.382451in}}%
\pgfpathlineto{\pgfqpoint{2.700798in}{2.387759in}}%
\pgfpathlineto{\pgfqpoint{2.709267in}{2.393228in}}%
\pgfpathlineto{\pgfqpoint{2.695552in}{2.410432in}}%
\pgfpathlineto{\pgfqpoint{2.681828in}{2.427933in}}%
\pgfpathlineto{\pgfqpoint{2.668095in}{2.445735in}}%
\pgfpathlineto{\pgfqpoint{2.654354in}{2.463840in}}%
\pgfpathlineto{\pgfqpoint{2.645853in}{2.458682in}}%
\pgfpathlineto{\pgfqpoint{2.637341in}{2.453696in}}%
\pgfpathlineto{\pgfqpoint{2.628816in}{2.448882in}}%
\pgfpathlineto{\pgfqpoint{2.620279in}{2.444244in}}%
\pgfpathclose%
\pgfusepath{fill}%
\end{pgfscope}%
\begin{pgfscope}%
\pgfpathrectangle{\pgfqpoint{1.150000in}{0.150000in}}{\pgfqpoint{5.700000in}{5.700000in}}%
\pgfusepath{clip}%
\pgfsetbuttcap%
\pgfsetroundjoin%
\definecolor{currentfill}{rgb}{0.279566,0.067836,0.391917}%
\pgfsetfillcolor{currentfill}%
\pgfsetfillopacity{0.800000}%
\pgfsetlinewidth{0.000000pt}%
\definecolor{currentstroke}{rgb}{0.000000,0.000000,0.000000}%
\pgfsetstrokecolor{currentstroke}%
\pgfsetdash{}{0pt}%
\pgfpathmoveto{\pgfqpoint{3.601089in}{2.002314in}}%
\pgfpathlineto{\pgfqpoint{3.614663in}{1.999919in}}%
\pgfpathlineto{\pgfqpoint{3.628242in}{1.997732in}}%
\pgfpathlineto{\pgfqpoint{3.641827in}{1.995750in}}%
\pgfpathlineto{\pgfqpoint{3.655417in}{1.993973in}}%
\pgfpathlineto{\pgfqpoint{3.663466in}{2.004324in}}%
\pgfpathlineto{\pgfqpoint{3.671509in}{2.014682in}}%
\pgfpathlineto{\pgfqpoint{3.679547in}{2.025046in}}%
\pgfpathlineto{\pgfqpoint{3.687580in}{2.035414in}}%
\pgfpathlineto{\pgfqpoint{3.674000in}{2.037040in}}%
\pgfpathlineto{\pgfqpoint{3.660426in}{2.038872in}}%
\pgfpathlineto{\pgfqpoint{3.646857in}{2.040909in}}%
\pgfpathlineto{\pgfqpoint{3.633293in}{2.043153in}}%
\pgfpathlineto{\pgfqpoint{3.625250in}{2.032923in}}%
\pgfpathlineto{\pgfqpoint{3.617202in}{2.022706in}}%
\pgfpathlineto{\pgfqpoint{3.609148in}{2.012503in}}%
\pgfpathlineto{\pgfqpoint{3.601089in}{2.002314in}}%
\pgfpathclose%
\pgfusepath{fill}%
\end{pgfscope}%
\begin{pgfscope}%
\pgfpathrectangle{\pgfqpoint{1.150000in}{0.150000in}}{\pgfqpoint{5.700000in}{5.700000in}}%
\pgfusepath{clip}%
\pgfsetbuttcap%
\pgfsetroundjoin%
\definecolor{currentfill}{rgb}{0.199430,0.387607,0.554642}%
\pgfsetfillcolor{currentfill}%
\pgfsetfillopacity{0.800000}%
\pgfsetlinewidth{0.000000pt}%
\definecolor{currentstroke}{rgb}{0.000000,0.000000,0.000000}%
\pgfsetstrokecolor{currentstroke}%
\pgfsetdash{}{0pt}%
\pgfpathmoveto{\pgfqpoint{4.842046in}{2.748179in}}%
\pgfpathlineto{\pgfqpoint{4.856074in}{2.755409in}}%
\pgfpathlineto{\pgfqpoint{4.870117in}{2.762821in}}%
\pgfpathlineto{\pgfqpoint{4.884175in}{2.770413in}}%
\pgfpathlineto{\pgfqpoint{4.898247in}{2.778187in}}%
\pgfpathlineto{\pgfqpoint{4.905849in}{2.785349in}}%
\pgfpathlineto{\pgfqpoint{4.913444in}{2.792486in}}%
\pgfpathlineto{\pgfqpoint{4.921033in}{2.799602in}}%
\pgfpathlineto{\pgfqpoint{4.928616in}{2.806700in}}%
\pgfpathlineto{\pgfqpoint{4.914556in}{2.799262in}}%
\pgfpathlineto{\pgfqpoint{4.900512in}{2.792005in}}%
\pgfpathlineto{\pgfqpoint{4.886482in}{2.784929in}}%
\pgfpathlineto{\pgfqpoint{4.872467in}{2.778033in}}%
\pgfpathlineto{\pgfqpoint{4.864870in}{2.770588in}}%
\pgfpathlineto{\pgfqpoint{4.857268in}{2.763133in}}%
\pgfpathlineto{\pgfqpoint{4.849660in}{2.755664in}}%
\pgfpathlineto{\pgfqpoint{4.842046in}{2.748179in}}%
\pgfpathclose%
\pgfusepath{fill}%
\end{pgfscope}%
\begin{pgfscope}%
\pgfpathrectangle{\pgfqpoint{1.150000in}{0.150000in}}{\pgfqpoint{5.700000in}{5.700000in}}%
\pgfusepath{clip}%
\pgfsetbuttcap%
\pgfsetroundjoin%
\definecolor{currentfill}{rgb}{0.282623,0.140926,0.457517}%
\pgfsetfillcolor{currentfill}%
\pgfsetfillopacity{0.800000}%
\pgfsetlinewidth{0.000000pt}%
\definecolor{currentstroke}{rgb}{0.000000,0.000000,0.000000}%
\pgfsetstrokecolor{currentstroke}%
\pgfsetdash{}{0pt}%
\pgfpathmoveto{\pgfqpoint{2.839692in}{2.184373in}}%
\pgfpathlineto{\pgfqpoint{2.853348in}{2.170505in}}%
\pgfpathlineto{\pgfqpoint{2.866999in}{2.156903in}}%
\pgfpathlineto{\pgfqpoint{2.880645in}{2.143565in}}%
\pgfpathlineto{\pgfqpoint{2.894286in}{2.130489in}}%
\pgfpathlineto{\pgfqpoint{2.902671in}{2.136791in}}%
\pgfpathlineto{\pgfqpoint{2.911046in}{2.143227in}}%
\pgfpathlineto{\pgfqpoint{2.919412in}{2.149796in}}%
\pgfpathlineto{\pgfqpoint{2.927767in}{2.156493in}}%
\pgfpathlineto{\pgfqpoint{2.914152in}{2.169253in}}%
\pgfpathlineto{\pgfqpoint{2.900533in}{2.182274in}}%
\pgfpathlineto{\pgfqpoint{2.886909in}{2.195560in}}%
\pgfpathlineto{\pgfqpoint{2.873281in}{2.209111in}}%
\pgfpathlineto{\pgfqpoint{2.864899in}{2.202718in}}%
\pgfpathlineto{\pgfqpoint{2.856507in}{2.196462in}}%
\pgfpathlineto{\pgfqpoint{2.848104in}{2.190346in}}%
\pgfpathlineto{\pgfqpoint{2.839692in}{2.184373in}}%
\pgfpathclose%
\pgfusepath{fill}%
\end{pgfscope}%
\begin{pgfscope}%
\pgfpathrectangle{\pgfqpoint{1.150000in}{0.150000in}}{\pgfqpoint{5.700000in}{5.700000in}}%
\pgfusepath{clip}%
\pgfsetbuttcap%
\pgfsetroundjoin%
\definecolor{currentfill}{rgb}{0.128729,0.563265,0.551229}%
\pgfsetfillcolor{currentfill}%
\pgfsetfillopacity{0.800000}%
\pgfsetlinewidth{0.000000pt}%
\definecolor{currentstroke}{rgb}{0.000000,0.000000,0.000000}%
\pgfsetstrokecolor{currentstroke}%
\pgfsetdash{}{0pt}%
\pgfpathmoveto{\pgfqpoint{5.651185in}{3.279688in}}%
\pgfpathlineto{\pgfqpoint{5.665605in}{3.288449in}}%
\pgfpathlineto{\pgfqpoint{5.680043in}{3.297383in}}%
\pgfpathlineto{\pgfqpoint{5.694499in}{3.306491in}}%
\pgfpathlineto{\pgfqpoint{5.708973in}{3.315773in}}%
\pgfpathlineto{\pgfqpoint{5.716183in}{3.319982in}}%
\pgfpathlineto{\pgfqpoint{5.723390in}{3.324318in}}%
\pgfpathlineto{\pgfqpoint{5.730595in}{3.328790in}}%
\pgfpathlineto{\pgfqpoint{5.737798in}{3.333403in}}%
\pgfpathlineto{\pgfqpoint{5.723355in}{3.324790in}}%
\pgfpathlineto{\pgfqpoint{5.708930in}{3.316349in}}%
\pgfpathlineto{\pgfqpoint{5.694523in}{3.308081in}}%
\pgfpathlineto{\pgfqpoint{5.680134in}{3.299985in}}%
\pgfpathlineto{\pgfqpoint{5.672899in}{3.294694in}}%
\pgfpathlineto{\pgfqpoint{5.665663in}{3.289552in}}%
\pgfpathlineto{\pgfqpoint{5.658425in}{3.284552in}}%
\pgfpathlineto{\pgfqpoint{5.651185in}{3.279688in}}%
\pgfpathclose%
\pgfusepath{fill}%
\end{pgfscope}%
\begin{pgfscope}%
\pgfpathrectangle{\pgfqpoint{1.150000in}{0.150000in}}{\pgfqpoint{5.700000in}{5.700000in}}%
\pgfusepath{clip}%
\pgfsetbuttcap%
\pgfsetroundjoin%
\definecolor{currentfill}{rgb}{0.263663,0.237631,0.518762}%
\pgfsetfillcolor{currentfill}%
\pgfsetfillopacity{0.800000}%
\pgfsetlinewidth{0.000000pt}%
\definecolor{currentstroke}{rgb}{0.000000,0.000000,0.000000}%
\pgfsetstrokecolor{currentstroke}%
\pgfsetdash{}{0pt}%
\pgfpathmoveto{\pgfqpoint{4.292206in}{2.365680in}}%
\pgfpathlineto{\pgfqpoint{4.305984in}{2.369897in}}%
\pgfpathlineto{\pgfqpoint{4.319774in}{2.374303in}}%
\pgfpathlineto{\pgfqpoint{4.333574in}{2.378897in}}%
\pgfpathlineto{\pgfqpoint{4.347387in}{2.383680in}}%
\pgfpathlineto{\pgfqpoint{4.355210in}{2.393365in}}%
\pgfpathlineto{\pgfqpoint{4.363028in}{2.402997in}}%
\pgfpathlineto{\pgfqpoint{4.370840in}{2.412580in}}%
\pgfpathlineto{\pgfqpoint{4.378646in}{2.422114in}}%
\pgfpathlineto{\pgfqpoint{4.364840in}{2.417437in}}%
\pgfpathlineto{\pgfqpoint{4.351047in}{2.412948in}}%
\pgfpathlineto{\pgfqpoint{4.337264in}{2.408647in}}%
\pgfpathlineto{\pgfqpoint{4.323494in}{2.404535in}}%
\pgfpathlineto{\pgfqpoint{4.315680in}{2.394884in}}%
\pgfpathlineto{\pgfqpoint{4.307861in}{2.385192in}}%
\pgfpathlineto{\pgfqpoint{4.300036in}{2.375458in}}%
\pgfpathlineto{\pgfqpoint{4.292206in}{2.365680in}}%
\pgfpathclose%
\pgfusepath{fill}%
\end{pgfscope}%
\begin{pgfscope}%
\pgfpathrectangle{\pgfqpoint{1.150000in}{0.150000in}}{\pgfqpoint{5.700000in}{5.700000in}}%
\pgfusepath{clip}%
\pgfsetbuttcap%
\pgfsetroundjoin%
\definecolor{currentfill}{rgb}{0.248629,0.278775,0.534556}%
\pgfsetfillcolor{currentfill}%
\pgfsetfillopacity{0.800000}%
\pgfsetlinewidth{0.000000pt}%
\definecolor{currentstroke}{rgb}{0.000000,0.000000,0.000000}%
\pgfsetstrokecolor{currentstroke}%
\pgfsetdash{}{0pt}%
\pgfpathmoveto{\pgfqpoint{2.565078in}{2.521055in}}%
\pgfpathlineto{\pgfqpoint{2.578894in}{2.501383in}}%
\pgfpathlineto{\pgfqpoint{2.592700in}{2.482026in}}%
\pgfpathlineto{\pgfqpoint{2.606494in}{2.462981in}}%
\pgfpathlineto{\pgfqpoint{2.620279in}{2.444244in}}%
\pgfpathlineto{\pgfqpoint{2.628816in}{2.448882in}}%
\pgfpathlineto{\pgfqpoint{2.637341in}{2.453696in}}%
\pgfpathlineto{\pgfqpoint{2.645853in}{2.458682in}}%
\pgfpathlineto{\pgfqpoint{2.654354in}{2.463840in}}%
\pgfpathlineto{\pgfqpoint{2.640602in}{2.482250in}}%
\pgfpathlineto{\pgfqpoint{2.626841in}{2.500968in}}%
\pgfpathlineto{\pgfqpoint{2.613070in}{2.519997in}}%
\pgfpathlineto{\pgfqpoint{2.599289in}{2.539341in}}%
\pgfpathlineto{\pgfqpoint{2.590755in}{2.534498in}}%
\pgfpathlineto{\pgfqpoint{2.582209in}{2.529835in}}%
\pgfpathlineto{\pgfqpoint{2.573650in}{2.525352in}}%
\pgfpathlineto{\pgfqpoint{2.565078in}{2.521055in}}%
\pgfpathclose%
\pgfusepath{fill}%
\end{pgfscope}%
\begin{pgfscope}%
\pgfpathrectangle{\pgfqpoint{1.150000in}{0.150000in}}{\pgfqpoint{5.700000in}{5.700000in}}%
\pgfusepath{clip}%
\pgfsetbuttcap%
\pgfsetroundjoin%
\definecolor{currentfill}{rgb}{0.123463,0.581687,0.547445}%
\pgfsetfillcolor{currentfill}%
\pgfsetfillopacity{0.800000}%
\pgfsetlinewidth{0.000000pt}%
\definecolor{currentstroke}{rgb}{0.000000,0.000000,0.000000}%
\pgfsetstrokecolor{currentstroke}%
\pgfsetdash{}{0pt}%
\pgfpathmoveto{\pgfqpoint{5.737798in}{3.333403in}}%
\pgfpathlineto{\pgfqpoint{5.752259in}{3.342189in}}%
\pgfpathlineto{\pgfqpoint{5.766739in}{3.351148in}}%
\pgfpathlineto{\pgfqpoint{5.781237in}{3.360279in}}%
\pgfpathlineto{\pgfqpoint{5.795755in}{3.369584in}}%
\pgfpathlineto{\pgfqpoint{5.802924in}{3.373658in}}%
\pgfpathlineto{\pgfqpoint{5.810091in}{3.377882in}}%
\pgfpathlineto{\pgfqpoint{5.817258in}{3.382263in}}%
\pgfpathlineto{\pgfqpoint{5.824424in}{3.386809in}}%
\pgfpathlineto{\pgfqpoint{5.809941in}{3.378206in}}%
\pgfpathlineto{\pgfqpoint{5.795476in}{3.369775in}}%
\pgfpathlineto{\pgfqpoint{5.781029in}{3.361515in}}%
\pgfpathlineto{\pgfqpoint{5.766601in}{3.353427in}}%
\pgfpathlineto{\pgfqpoint{5.759401in}{3.348171in}}%
\pgfpathlineto{\pgfqpoint{5.752201in}{3.343086in}}%
\pgfpathlineto{\pgfqpoint{5.745000in}{3.338166in}}%
\pgfpathlineto{\pgfqpoint{5.737798in}{3.333403in}}%
\pgfpathclose%
\pgfusepath{fill}%
\end{pgfscope}%
\begin{pgfscope}%
\pgfpathrectangle{\pgfqpoint{1.150000in}{0.150000in}}{\pgfqpoint{5.700000in}{5.700000in}}%
\pgfusepath{clip}%
\pgfsetbuttcap%
\pgfsetroundjoin%
\definecolor{currentfill}{rgb}{0.190631,0.407061,0.556089}%
\pgfsetfillcolor{currentfill}%
\pgfsetfillopacity{0.800000}%
\pgfsetlinewidth{0.000000pt}%
\definecolor{currentstroke}{rgb}{0.000000,0.000000,0.000000}%
\pgfsetstrokecolor{currentstroke}%
\pgfsetdash{}{0pt}%
\pgfpathmoveto{\pgfqpoint{4.928616in}{2.806700in}}%
\pgfpathlineto{\pgfqpoint{4.942691in}{2.814319in}}%
\pgfpathlineto{\pgfqpoint{4.956781in}{2.822118in}}%
\pgfpathlineto{\pgfqpoint{4.970886in}{2.830097in}}%
\pgfpathlineto{\pgfqpoint{4.985008in}{2.838256in}}%
\pgfpathlineto{\pgfqpoint{4.992571in}{2.844984in}}%
\pgfpathlineto{\pgfqpoint{5.000127in}{2.851695in}}%
\pgfpathlineto{\pgfqpoint{5.007678in}{2.858394in}}%
\pgfpathlineto{\pgfqpoint{5.015223in}{2.865084in}}%
\pgfpathlineto{\pgfqpoint{5.001117in}{2.857294in}}%
\pgfpathlineto{\pgfqpoint{4.987026in}{2.849684in}}%
\pgfpathlineto{\pgfqpoint{4.972950in}{2.842254in}}%
\pgfpathlineto{\pgfqpoint{4.958890in}{2.835003in}}%
\pgfpathlineto{\pgfqpoint{4.951330in}{2.827933in}}%
\pgfpathlineto{\pgfqpoint{4.943764in}{2.820862in}}%
\pgfpathlineto{\pgfqpoint{4.936193in}{2.813785in}}%
\pgfpathlineto{\pgfqpoint{4.928616in}{2.806700in}}%
\pgfpathclose%
\pgfusepath{fill}%
\end{pgfscope}%
\begin{pgfscope}%
\pgfpathrectangle{\pgfqpoint{1.150000in}{0.150000in}}{\pgfqpoint{5.700000in}{5.700000in}}%
\pgfusepath{clip}%
\pgfsetbuttcap%
\pgfsetroundjoin%
\definecolor{currentfill}{rgb}{0.283229,0.120777,0.440584}%
\pgfsetfillcolor{currentfill}%
\pgfsetfillopacity{0.800000}%
\pgfsetlinewidth{0.000000pt}%
\definecolor{currentstroke}{rgb}{0.000000,0.000000,0.000000}%
\pgfsetstrokecolor{currentstroke}%
\pgfsetdash{}{0pt}%
\pgfpathmoveto{\pgfqpoint{2.894286in}{2.130489in}}%
\pgfpathlineto{\pgfqpoint{2.907923in}{2.117674in}}%
\pgfpathlineto{\pgfqpoint{2.921556in}{2.105117in}}%
\pgfpathlineto{\pgfqpoint{2.935185in}{2.092817in}}%
\pgfpathlineto{\pgfqpoint{2.948810in}{2.080772in}}%
\pgfpathlineto{\pgfqpoint{2.957169in}{2.087401in}}%
\pgfpathlineto{\pgfqpoint{2.965518in}{2.094156in}}%
\pgfpathlineto{\pgfqpoint{2.973858in}{2.101036in}}%
\pgfpathlineto{\pgfqpoint{2.982189in}{2.108036in}}%
\pgfpathlineto{\pgfqpoint{2.968589in}{2.119767in}}%
\pgfpathlineto{\pgfqpoint{2.954985in}{2.131752in}}%
\pgfpathlineto{\pgfqpoint{2.941378in}{2.143993in}}%
\pgfpathlineto{\pgfqpoint{2.927767in}{2.156493in}}%
\pgfpathlineto{\pgfqpoint{2.919412in}{2.149796in}}%
\pgfpathlineto{\pgfqpoint{2.911046in}{2.143227in}}%
\pgfpathlineto{\pgfqpoint{2.902671in}{2.136791in}}%
\pgfpathlineto{\pgfqpoint{2.894286in}{2.130489in}}%
\pgfpathclose%
\pgfusepath{fill}%
\end{pgfscope}%
\begin{pgfscope}%
\pgfpathrectangle{\pgfqpoint{1.150000in}{0.150000in}}{\pgfqpoint{5.700000in}{5.700000in}}%
\pgfusepath{clip}%
\pgfsetbuttcap%
\pgfsetroundjoin%
\definecolor{currentfill}{rgb}{0.277941,0.056324,0.381191}%
\pgfsetfillcolor{currentfill}%
\pgfsetfillopacity{0.800000}%
\pgfsetlinewidth{0.000000pt}%
\definecolor{currentstroke}{rgb}{0.000000,0.000000,0.000000}%
\pgfsetstrokecolor{currentstroke}%
\pgfsetdash{}{0pt}%
\pgfpathmoveto{\pgfqpoint{3.514501in}{1.974120in}}%
\pgfpathlineto{\pgfqpoint{3.528068in}{1.970709in}}%
\pgfpathlineto{\pgfqpoint{3.541639in}{1.967507in}}%
\pgfpathlineto{\pgfqpoint{3.555215in}{1.964515in}}%
\pgfpathlineto{\pgfqpoint{3.568796in}{1.961731in}}%
\pgfpathlineto{\pgfqpoint{3.576877in}{1.971848in}}%
\pgfpathlineto{\pgfqpoint{3.584953in}{1.981985in}}%
\pgfpathlineto{\pgfqpoint{3.593024in}{1.992141in}}%
\pgfpathlineto{\pgfqpoint{3.601089in}{2.002314in}}%
\pgfpathlineto{\pgfqpoint{3.587520in}{2.004916in}}%
\pgfpathlineto{\pgfqpoint{3.573955in}{2.007726in}}%
\pgfpathlineto{\pgfqpoint{3.560396in}{2.010745in}}%
\pgfpathlineto{\pgfqpoint{3.546840in}{2.013975in}}%
\pgfpathlineto{\pgfqpoint{3.538764in}{2.003972in}}%
\pgfpathlineto{\pgfqpoint{3.530682in}{1.993994in}}%
\pgfpathlineto{\pgfqpoint{3.522594in}{1.984043in}}%
\pgfpathlineto{\pgfqpoint{3.514501in}{1.974120in}}%
\pgfpathclose%
\pgfusepath{fill}%
\end{pgfscope}%
\begin{pgfscope}%
\pgfpathrectangle{\pgfqpoint{1.150000in}{0.150000in}}{\pgfqpoint{5.700000in}{5.700000in}}%
\pgfusepath{clip}%
\pgfsetbuttcap%
\pgfsetroundjoin%
\definecolor{currentfill}{rgb}{0.253935,0.265254,0.529983}%
\pgfsetfillcolor{currentfill}%
\pgfsetfillopacity{0.800000}%
\pgfsetlinewidth{0.000000pt}%
\definecolor{currentstroke}{rgb}{0.000000,0.000000,0.000000}%
\pgfsetstrokecolor{currentstroke}%
\pgfsetdash{}{0pt}%
\pgfpathmoveto{\pgfqpoint{4.378646in}{2.422114in}}%
\pgfpathlineto{\pgfqpoint{4.392463in}{2.426978in}}%
\pgfpathlineto{\pgfqpoint{4.406293in}{2.432030in}}%
\pgfpathlineto{\pgfqpoint{4.420134in}{2.437269in}}%
\pgfpathlineto{\pgfqpoint{4.433988in}{2.442694in}}%
\pgfpathlineto{\pgfqpoint{4.441781in}{2.452055in}}%
\pgfpathlineto{\pgfqpoint{4.449569in}{2.461363in}}%
\pgfpathlineto{\pgfqpoint{4.457352in}{2.470619in}}%
\pgfpathlineto{\pgfqpoint{4.465128in}{2.479825in}}%
\pgfpathlineto{\pgfqpoint{4.451282in}{2.474538in}}%
\pgfpathlineto{\pgfqpoint{4.437448in}{2.469437in}}%
\pgfpathlineto{\pgfqpoint{4.423626in}{2.464523in}}%
\pgfpathlineto{\pgfqpoint{4.409817in}{2.459796in}}%
\pgfpathlineto{\pgfqpoint{4.402032in}{2.450440in}}%
\pgfpathlineto{\pgfqpoint{4.394242in}{2.441042in}}%
\pgfpathlineto{\pgfqpoint{4.386447in}{2.431600in}}%
\pgfpathlineto{\pgfqpoint{4.378646in}{2.422114in}}%
\pgfpathclose%
\pgfusepath{fill}%
\end{pgfscope}%
\begin{pgfscope}%
\pgfpathrectangle{\pgfqpoint{1.150000in}{0.150000in}}{\pgfqpoint{5.700000in}{5.700000in}}%
\pgfusepath{clip}%
\pgfsetbuttcap%
\pgfsetroundjoin%
\definecolor{currentfill}{rgb}{0.120092,0.600104,0.542530}%
\pgfsetfillcolor{currentfill}%
\pgfsetfillopacity{0.800000}%
\pgfsetlinewidth{0.000000pt}%
\definecolor{currentstroke}{rgb}{0.000000,0.000000,0.000000}%
\pgfsetstrokecolor{currentstroke}%
\pgfsetdash{}{0pt}%
\pgfpathmoveto{\pgfqpoint{5.824424in}{3.386809in}}%
\pgfpathlineto{\pgfqpoint{5.838926in}{3.395584in}}%
\pgfpathlineto{\pgfqpoint{5.853446in}{3.404531in}}%
\pgfpathlineto{\pgfqpoint{5.867986in}{3.413650in}}%
\pgfpathlineto{\pgfqpoint{5.882544in}{3.422942in}}%
\pgfpathlineto{\pgfqpoint{5.889674in}{3.426939in}}%
\pgfpathlineto{\pgfqpoint{5.896804in}{3.431109in}}%
\pgfpathlineto{\pgfqpoint{5.903934in}{3.435461in}}%
\pgfpathlineto{\pgfqpoint{5.911065in}{3.440003in}}%
\pgfpathlineto{\pgfqpoint{5.896543in}{3.431446in}}%
\pgfpathlineto{\pgfqpoint{5.882040in}{3.423060in}}%
\pgfpathlineto{\pgfqpoint{5.867555in}{3.414845in}}%
\pgfpathlineto{\pgfqpoint{5.853088in}{3.406801in}}%
\pgfpathlineto{\pgfqpoint{5.845921in}{3.401516in}}%
\pgfpathlineto{\pgfqpoint{5.838755in}{3.396428in}}%
\pgfpathlineto{\pgfqpoint{5.831590in}{3.391528in}}%
\pgfpathlineto{\pgfqpoint{5.824424in}{3.386809in}}%
\pgfpathclose%
\pgfusepath{fill}%
\end{pgfscope}%
\begin{pgfscope}%
\pgfpathrectangle{\pgfqpoint{1.150000in}{0.150000in}}{\pgfqpoint{5.700000in}{5.700000in}}%
\pgfusepath{clip}%
\pgfsetbuttcap%
\pgfsetroundjoin%
\definecolor{currentfill}{rgb}{0.276022,0.044167,0.370164}%
\pgfsetfillcolor{currentfill}%
\pgfsetfillopacity{0.800000}%
\pgfsetlinewidth{0.000000pt}%
\definecolor{currentstroke}{rgb}{0.000000,0.000000,0.000000}%
\pgfsetstrokecolor{currentstroke}%
\pgfsetdash{}{0pt}%
\pgfpathmoveto{\pgfqpoint{3.286647in}{1.959129in}}%
\pgfpathlineto{\pgfqpoint{3.300209in}{1.952661in}}%
\pgfpathlineto{\pgfqpoint{3.313773in}{1.946414in}}%
\pgfpathlineto{\pgfqpoint{3.327338in}{1.940390in}}%
\pgfpathlineto{\pgfqpoint{3.340906in}{1.934585in}}%
\pgfpathlineto{\pgfqpoint{3.349077in}{1.943707in}}%
\pgfpathlineto{\pgfqpoint{3.357242in}{1.952888in}}%
\pgfpathlineto{\pgfqpoint{3.365400in}{1.962125in}}%
\pgfpathlineto{\pgfqpoint{3.373552in}{1.971417in}}%
\pgfpathlineto{\pgfqpoint{3.360000in}{1.976976in}}%
\pgfpathlineto{\pgfqpoint{3.346451in}{1.982755in}}%
\pgfpathlineto{\pgfqpoint{3.332903in}{1.988756in}}%
\pgfpathlineto{\pgfqpoint{3.319357in}{1.994978in}}%
\pgfpathlineto{\pgfqpoint{3.311190in}{1.985921in}}%
\pgfpathlineto{\pgfqpoint{3.303016in}{1.976925in}}%
\pgfpathlineto{\pgfqpoint{3.294835in}{1.967994in}}%
\pgfpathlineto{\pgfqpoint{3.286647in}{1.959129in}}%
\pgfpathclose%
\pgfusepath{fill}%
\end{pgfscope}%
\begin{pgfscope}%
\pgfpathrectangle{\pgfqpoint{1.150000in}{0.150000in}}{\pgfqpoint{5.700000in}{5.700000in}}%
\pgfusepath{clip}%
\pgfsetbuttcap%
\pgfsetroundjoin%
\definecolor{currentfill}{rgb}{0.233603,0.313828,0.543914}%
\pgfsetfillcolor{currentfill}%
\pgfsetfillopacity{0.800000}%
\pgfsetlinewidth{0.000000pt}%
\definecolor{currentstroke}{rgb}{0.000000,0.000000,0.000000}%
\pgfsetstrokecolor{currentstroke}%
\pgfsetdash{}{0pt}%
\pgfpathmoveto{\pgfqpoint{2.509700in}{2.602948in}}%
\pgfpathlineto{\pgfqpoint{2.523562in}{2.581987in}}%
\pgfpathlineto{\pgfqpoint{2.537412in}{2.561354in}}%
\pgfpathlineto{\pgfqpoint{2.551251in}{2.541044in}}%
\pgfpathlineto{\pgfqpoint{2.565078in}{2.521055in}}%
\pgfpathlineto{\pgfqpoint{2.573650in}{2.525352in}}%
\pgfpathlineto{\pgfqpoint{2.582209in}{2.529835in}}%
\pgfpathlineto{\pgfqpoint{2.590755in}{2.534498in}}%
\pgfpathlineto{\pgfqpoint{2.599289in}{2.539341in}}%
\pgfpathlineto{\pgfqpoint{2.585497in}{2.559001in}}%
\pgfpathlineto{\pgfqpoint{2.571693in}{2.578981in}}%
\pgfpathlineto{\pgfqpoint{2.557879in}{2.599284in}}%
\pgfpathlineto{\pgfqpoint{2.544052in}{2.619913in}}%
\pgfpathlineto{\pgfqpoint{2.535484in}{2.615388in}}%
\pgfpathlineto{\pgfqpoint{2.526902in}{2.611050in}}%
\pgfpathlineto{\pgfqpoint{2.518308in}{2.606902in}}%
\pgfpathlineto{\pgfqpoint{2.509700in}{2.602948in}}%
\pgfpathclose%
\pgfusepath{fill}%
\end{pgfscope}%
\begin{pgfscope}%
\pgfpathrectangle{\pgfqpoint{1.150000in}{0.150000in}}{\pgfqpoint{5.700000in}{5.700000in}}%
\pgfusepath{clip}%
\pgfsetbuttcap%
\pgfsetroundjoin%
\definecolor{currentfill}{rgb}{0.119699,0.618490,0.536347}%
\pgfsetfillcolor{currentfill}%
\pgfsetfillopacity{0.800000}%
\pgfsetlinewidth{0.000000pt}%
\definecolor{currentstroke}{rgb}{0.000000,0.000000,0.000000}%
\pgfsetstrokecolor{currentstroke}%
\pgfsetdash{}{0pt}%
\pgfpathmoveto{\pgfqpoint{5.911065in}{3.440003in}}%
\pgfpathlineto{\pgfqpoint{5.925606in}{3.448731in}}%
\pgfpathlineto{\pgfqpoint{5.940166in}{3.457631in}}%
\pgfpathlineto{\pgfqpoint{5.954745in}{3.466702in}}%
\pgfpathlineto{\pgfqpoint{5.969343in}{3.475944in}}%
\pgfpathlineto{\pgfqpoint{5.976437in}{3.479928in}}%
\pgfpathlineto{\pgfqpoint{5.983531in}{3.484111in}}%
\pgfpathlineto{\pgfqpoint{5.990627in}{3.488501in}}%
\pgfpathlineto{\pgfqpoint{5.997725in}{3.493106in}}%
\pgfpathlineto{\pgfqpoint{5.983165in}{3.484630in}}%
\pgfpathlineto{\pgfqpoint{5.968625in}{3.476326in}}%
\pgfpathlineto{\pgfqpoint{5.954103in}{3.468191in}}%
\pgfpathlineto{\pgfqpoint{5.939600in}{3.460226in}}%
\pgfpathlineto{\pgfqpoint{5.932464in}{3.454845in}}%
\pgfpathlineto{\pgfqpoint{5.925329in}{3.449687in}}%
\pgfpathlineto{\pgfqpoint{5.918197in}{3.444742in}}%
\pgfpathlineto{\pgfqpoint{5.911065in}{3.440003in}}%
\pgfpathclose%
\pgfusepath{fill}%
\end{pgfscope}%
\begin{pgfscope}%
\pgfpathrectangle{\pgfqpoint{1.150000in}{0.150000in}}{\pgfqpoint{5.700000in}{5.700000in}}%
\pgfusepath{clip}%
\pgfsetbuttcap%
\pgfsetroundjoin%
\definecolor{currentfill}{rgb}{0.180629,0.429975,0.557282}%
\pgfsetfillcolor{currentfill}%
\pgfsetfillopacity{0.800000}%
\pgfsetlinewidth{0.000000pt}%
\definecolor{currentstroke}{rgb}{0.000000,0.000000,0.000000}%
\pgfsetstrokecolor{currentstroke}%
\pgfsetdash{}{0pt}%
\pgfpathmoveto{\pgfqpoint{5.015223in}{2.865084in}}%
\pgfpathlineto{\pgfqpoint{5.029345in}{2.873053in}}%
\pgfpathlineto{\pgfqpoint{5.043483in}{2.881202in}}%
\pgfpathlineto{\pgfqpoint{5.057637in}{2.889531in}}%
\pgfpathlineto{\pgfqpoint{5.071807in}{2.898039in}}%
\pgfpathlineto{\pgfqpoint{5.079330in}{2.904335in}}%
\pgfpathlineto{\pgfqpoint{5.086847in}{2.910622in}}%
\pgfpathlineto{\pgfqpoint{5.094358in}{2.916908in}}%
\pgfpathlineto{\pgfqpoint{5.101864in}{2.923195in}}%
\pgfpathlineto{\pgfqpoint{5.087710in}{2.915090in}}%
\pgfpathlineto{\pgfqpoint{5.073573in}{2.907164in}}%
\pgfpathlineto{\pgfqpoint{5.059451in}{2.899417in}}%
\pgfpathlineto{\pgfqpoint{5.045345in}{2.891849in}}%
\pgfpathlineto{\pgfqpoint{5.037823in}{2.885148in}}%
\pgfpathlineto{\pgfqpoint{5.030295in}{2.878457in}}%
\pgfpathlineto{\pgfqpoint{5.022762in}{2.871770in}}%
\pgfpathlineto{\pgfqpoint{5.015223in}{2.865084in}}%
\pgfpathclose%
\pgfusepath{fill}%
\end{pgfscope}%
\begin{pgfscope}%
\pgfpathrectangle{\pgfqpoint{1.150000in}{0.150000in}}{\pgfqpoint{5.700000in}{5.700000in}}%
\pgfusepath{clip}%
\pgfsetbuttcap%
\pgfsetroundjoin%
\definecolor{currentfill}{rgb}{0.277941,0.056324,0.381191}%
\pgfsetfillcolor{currentfill}%
\pgfsetfillopacity{0.800000}%
\pgfsetlinewidth{0.000000pt}%
\definecolor{currentstroke}{rgb}{0.000000,0.000000,0.000000}%
\pgfsetstrokecolor{currentstroke}%
\pgfsetdash{}{0pt}%
\pgfpathmoveto{\pgfqpoint{3.145209in}{1.986509in}}%
\pgfpathlineto{\pgfqpoint{3.158785in}{1.977938in}}%
\pgfpathlineto{\pgfqpoint{3.172362in}{1.969599in}}%
\pgfpathlineto{\pgfqpoint{3.185938in}{1.961492in}}%
\pgfpathlineto{\pgfqpoint{3.199514in}{1.953615in}}%
\pgfpathlineto{\pgfqpoint{3.207749in}{1.961901in}}%
\pgfpathlineto{\pgfqpoint{3.215977in}{1.970271in}}%
\pgfpathlineto{\pgfqpoint{3.224197in}{1.978722in}}%
\pgfpathlineto{\pgfqpoint{3.232410in}{1.987252in}}%
\pgfpathlineto{\pgfqpoint{3.218852in}{1.994851in}}%
\pgfpathlineto{\pgfqpoint{3.205295in}{2.002680in}}%
\pgfpathlineto{\pgfqpoint{3.191739in}{2.010740in}}%
\pgfpathlineto{\pgfqpoint{3.178182in}{2.019032in}}%
\pgfpathlineto{\pgfqpoint{3.169950in}{2.010768in}}%
\pgfpathlineto{\pgfqpoint{3.161711in}{2.002592in}}%
\pgfpathlineto{\pgfqpoint{3.153464in}{1.994504in}}%
\pgfpathlineto{\pgfqpoint{3.145209in}{1.986509in}}%
\pgfpathclose%
\pgfusepath{fill}%
\end{pgfscope}%
\begin{pgfscope}%
\pgfpathrectangle{\pgfqpoint{1.150000in}{0.150000in}}{\pgfqpoint{5.700000in}{5.700000in}}%
\pgfusepath{clip}%
\pgfsetbuttcap%
\pgfsetroundjoin%
\definecolor{currentfill}{rgb}{0.244972,0.287675,0.537260}%
\pgfsetfillcolor{currentfill}%
\pgfsetfillopacity{0.800000}%
\pgfsetlinewidth{0.000000pt}%
\definecolor{currentstroke}{rgb}{0.000000,0.000000,0.000000}%
\pgfsetstrokecolor{currentstroke}%
\pgfsetdash{}{0pt}%
\pgfpathmoveto{\pgfqpoint{4.465128in}{2.479825in}}%
\pgfpathlineto{\pgfqpoint{4.478987in}{2.485299in}}%
\pgfpathlineto{\pgfqpoint{4.492858in}{2.490958in}}%
\pgfpathlineto{\pgfqpoint{4.506742in}{2.496804in}}%
\pgfpathlineto{\pgfqpoint{4.520639in}{2.502835in}}%
\pgfpathlineto{\pgfqpoint{4.528402in}{2.511835in}}%
\pgfpathlineto{\pgfqpoint{4.536160in}{2.520782in}}%
\pgfpathlineto{\pgfqpoint{4.543911in}{2.529678in}}%
\pgfpathlineto{\pgfqpoint{4.551657in}{2.538525in}}%
\pgfpathlineto{\pgfqpoint{4.537768in}{2.532665in}}%
\pgfpathlineto{\pgfqpoint{4.523892in}{2.526991in}}%
\pgfpathlineto{\pgfqpoint{4.510029in}{2.521502in}}%
\pgfpathlineto{\pgfqpoint{4.496178in}{2.516199in}}%
\pgfpathlineto{\pgfqpoint{4.488424in}{2.507169in}}%
\pgfpathlineto{\pgfqpoint{4.480664in}{2.498098in}}%
\pgfpathlineto{\pgfqpoint{4.472899in}{2.488984in}}%
\pgfpathlineto{\pgfqpoint{4.465128in}{2.479825in}}%
\pgfpathclose%
\pgfusepath{fill}%
\end{pgfscope}%
\begin{pgfscope}%
\pgfpathrectangle{\pgfqpoint{1.150000in}{0.150000in}}{\pgfqpoint{5.700000in}{5.700000in}}%
\pgfusepath{clip}%
\pgfsetbuttcap%
\pgfsetroundjoin%
\definecolor{currentfill}{rgb}{0.282656,0.100196,0.422160}%
\pgfsetfillcolor{currentfill}%
\pgfsetfillopacity{0.800000}%
\pgfsetlinewidth{0.000000pt}%
\definecolor{currentstroke}{rgb}{0.000000,0.000000,0.000000}%
\pgfsetstrokecolor{currentstroke}%
\pgfsetdash{}{0pt}%
\pgfpathmoveto{\pgfqpoint{2.948810in}{2.080772in}}%
\pgfpathlineto{\pgfqpoint{2.962432in}{2.068980in}}%
\pgfpathlineto{\pgfqpoint{2.976051in}{2.057439in}}%
\pgfpathlineto{\pgfqpoint{2.989666in}{2.046148in}}%
\pgfpathlineto{\pgfqpoint{3.003279in}{2.035105in}}%
\pgfpathlineto{\pgfqpoint{3.011613in}{2.042059in}}%
\pgfpathlineto{\pgfqpoint{3.019937in}{2.049132in}}%
\pgfpathlineto{\pgfqpoint{3.028253in}{2.056321in}}%
\pgfpathlineto{\pgfqpoint{3.036560in}{2.063623in}}%
\pgfpathlineto{\pgfqpoint{3.022971in}{2.074353in}}%
\pgfpathlineto{\pgfqpoint{3.009380in}{2.085331in}}%
\pgfpathlineto{\pgfqpoint{2.995786in}{2.096558in}}%
\pgfpathlineto{\pgfqpoint{2.982189in}{2.108036in}}%
\pgfpathlineto{\pgfqpoint{2.973858in}{2.101036in}}%
\pgfpathlineto{\pgfqpoint{2.965518in}{2.094156in}}%
\pgfpathlineto{\pgfqpoint{2.957169in}{2.087401in}}%
\pgfpathlineto{\pgfqpoint{2.948810in}{2.080772in}}%
\pgfpathclose%
\pgfusepath{fill}%
\end{pgfscope}%
\begin{pgfscope}%
\pgfpathrectangle{\pgfqpoint{1.150000in}{0.150000in}}{\pgfqpoint{5.700000in}{5.700000in}}%
\pgfusepath{clip}%
\pgfsetbuttcap%
\pgfsetroundjoin%
\definecolor{currentfill}{rgb}{0.277018,0.050344,0.375715}%
\pgfsetfillcolor{currentfill}%
\pgfsetfillopacity{0.800000}%
\pgfsetlinewidth{0.000000pt}%
\definecolor{currentstroke}{rgb}{0.000000,0.000000,0.000000}%
\pgfsetstrokecolor{currentstroke}%
\pgfsetdash{}{0pt}%
\pgfpathmoveto{\pgfqpoint{3.427785in}{1.951359in}}%
\pgfpathlineto{\pgfqpoint{3.441350in}{1.946884in}}%
\pgfpathlineto{\pgfqpoint{3.454919in}{1.942624in}}%
\pgfpathlineto{\pgfqpoint{3.468492in}{1.938576in}}%
\pgfpathlineto{\pgfqpoint{3.482068in}{1.934740in}}%
\pgfpathlineto{\pgfqpoint{3.490185in}{1.944535in}}%
\pgfpathlineto{\pgfqpoint{3.498296in}{1.954364in}}%
\pgfpathlineto{\pgfqpoint{3.506401in}{1.964226in}}%
\pgfpathlineto{\pgfqpoint{3.514501in}{1.974120in}}%
\pgfpathlineto{\pgfqpoint{3.500937in}{1.977742in}}%
\pgfpathlineto{\pgfqpoint{3.487378in}{1.981577in}}%
\pgfpathlineto{\pgfqpoint{3.473823in}{1.985624in}}%
\pgfpathlineto{\pgfqpoint{3.460271in}{1.989885in}}%
\pgfpathlineto{\pgfqpoint{3.452158in}{1.980193in}}%
\pgfpathlineto{\pgfqpoint{3.444040in}{1.970541in}}%
\pgfpathlineto{\pgfqpoint{3.435915in}{1.960928in}}%
\pgfpathlineto{\pgfqpoint{3.427785in}{1.951359in}}%
\pgfpathclose%
\pgfusepath{fill}%
\end{pgfscope}%
\begin{pgfscope}%
\pgfpathrectangle{\pgfqpoint{1.150000in}{0.150000in}}{\pgfqpoint{5.700000in}{5.700000in}}%
\pgfusepath{clip}%
\pgfsetbuttcap%
\pgfsetroundjoin%
\definecolor{currentfill}{rgb}{0.172719,0.448791,0.557885}%
\pgfsetfillcolor{currentfill}%
\pgfsetfillopacity{0.800000}%
\pgfsetlinewidth{0.000000pt}%
\definecolor{currentstroke}{rgb}{0.000000,0.000000,0.000000}%
\pgfsetstrokecolor{currentstroke}%
\pgfsetdash{}{0pt}%
\pgfpathmoveto{\pgfqpoint{5.101864in}{2.923195in}}%
\pgfpathlineto{\pgfqpoint{5.116033in}{2.931478in}}%
\pgfpathlineto{\pgfqpoint{5.130219in}{2.939941in}}%
\pgfpathlineto{\pgfqpoint{5.144421in}{2.948582in}}%
\pgfpathlineto{\pgfqpoint{5.158640in}{2.957402in}}%
\pgfpathlineto{\pgfqpoint{5.166123in}{2.963271in}}%
\pgfpathlineto{\pgfqpoint{5.173599in}{2.969144in}}%
\pgfpathlineto{\pgfqpoint{5.181070in}{2.975025in}}%
\pgfpathlineto{\pgfqpoint{5.188535in}{2.980920in}}%
\pgfpathlineto{\pgfqpoint{5.174334in}{2.972538in}}%
\pgfpathlineto{\pgfqpoint{5.160150in}{2.964333in}}%
\pgfpathlineto{\pgfqpoint{5.145982in}{2.956306in}}%
\pgfpathlineto{\pgfqpoint{5.131830in}{2.948458in}}%
\pgfpathlineto{\pgfqpoint{5.124346in}{2.942115in}}%
\pgfpathlineto{\pgfqpoint{5.116858in}{2.935794in}}%
\pgfpathlineto{\pgfqpoint{5.109364in}{2.929489in}}%
\pgfpathlineto{\pgfqpoint{5.101864in}{2.923195in}}%
\pgfpathclose%
\pgfusepath{fill}%
\end{pgfscope}%
\begin{pgfscope}%
\pgfpathrectangle{\pgfqpoint{1.150000in}{0.150000in}}{\pgfqpoint{5.700000in}{5.700000in}}%
\pgfusepath{clip}%
\pgfsetbuttcap%
\pgfsetroundjoin%
\definecolor{currentfill}{rgb}{0.233603,0.313828,0.543914}%
\pgfsetfillcolor{currentfill}%
\pgfsetfillopacity{0.800000}%
\pgfsetlinewidth{0.000000pt}%
\definecolor{currentstroke}{rgb}{0.000000,0.000000,0.000000}%
\pgfsetstrokecolor{currentstroke}%
\pgfsetdash{}{0pt}%
\pgfpathmoveto{\pgfqpoint{4.551657in}{2.538525in}}%
\pgfpathlineto{\pgfqpoint{4.565559in}{2.544570in}}%
\pgfpathlineto{\pgfqpoint{4.579474in}{2.550800in}}%
\pgfpathlineto{\pgfqpoint{4.593402in}{2.557214in}}%
\pgfpathlineto{\pgfqpoint{4.607344in}{2.563813in}}%
\pgfpathlineto{\pgfqpoint{4.615075in}{2.572422in}}%
\pgfpathlineto{\pgfqpoint{4.622801in}{2.580978in}}%
\pgfpathlineto{\pgfqpoint{4.630520in}{2.589485in}}%
\pgfpathlineto{\pgfqpoint{4.638233in}{2.597946in}}%
\pgfpathlineto{\pgfqpoint{4.624300in}{2.591551in}}%
\pgfpathlineto{\pgfqpoint{4.610380in}{2.585341in}}%
\pgfpathlineto{\pgfqpoint{4.596474in}{2.579315in}}%
\pgfpathlineto{\pgfqpoint{4.582581in}{2.573474in}}%
\pgfpathlineto{\pgfqpoint{4.574859in}{2.564797in}}%
\pgfpathlineto{\pgfqpoint{4.567130in}{2.556082in}}%
\pgfpathlineto{\pgfqpoint{4.559396in}{2.547326in}}%
\pgfpathlineto{\pgfqpoint{4.551657in}{2.538525in}}%
\pgfpathclose%
\pgfusepath{fill}%
\end{pgfscope}%
\begin{pgfscope}%
\pgfpathrectangle{\pgfqpoint{1.150000in}{0.150000in}}{\pgfqpoint{5.700000in}{5.700000in}}%
\pgfusepath{clip}%
\pgfsetbuttcap%
\pgfsetroundjoin%
\definecolor{currentfill}{rgb}{0.123444,0.636809,0.528763}%
\pgfsetfillcolor{currentfill}%
\pgfsetfillopacity{0.800000}%
\pgfsetlinewidth{0.000000pt}%
\definecolor{currentstroke}{rgb}{0.000000,0.000000,0.000000}%
\pgfsetstrokecolor{currentstroke}%
\pgfsetdash{}{0pt}%
\pgfpathmoveto{\pgfqpoint{5.997725in}{3.493106in}}%
\pgfpathlineto{\pgfqpoint{6.012303in}{3.501751in}}%
\pgfpathlineto{\pgfqpoint{6.026901in}{3.510568in}}%
\pgfpathlineto{\pgfqpoint{6.041518in}{3.519555in}}%
\pgfpathlineto{\pgfqpoint{6.056154in}{3.528713in}}%
\pgfpathlineto{\pgfqpoint{6.063214in}{3.532754in}}%
\pgfpathlineto{\pgfqpoint{6.070276in}{3.537020in}}%
\pgfpathlineto{\pgfqpoint{6.077340in}{3.541520in}}%
\pgfpathlineto{\pgfqpoint{6.062735in}{3.532960in}}%
\pgfpathlineto{\pgfqpoint{6.048148in}{3.524569in}}%
\pgfpathlineto{\pgfqpoint{6.033581in}{3.516348in}}%
\pgfpathlineto{\pgfqpoint{6.019032in}{3.508297in}}%
\pgfpathlineto{\pgfqpoint{6.011927in}{3.502995in}}%
\pgfpathlineto{\pgfqpoint{6.004825in}{3.497934in}}%
\pgfpathlineto{\pgfqpoint{5.997725in}{3.493106in}}%
\pgfpathclose%
\pgfusepath{fill}%
\end{pgfscope}%
\begin{pgfscope}%
\pgfpathrectangle{\pgfqpoint{1.150000in}{0.150000in}}{\pgfqpoint{5.700000in}{5.700000in}}%
\pgfusepath{clip}%
\pgfsetbuttcap%
\pgfsetroundjoin%
\definecolor{currentfill}{rgb}{0.218130,0.347432,0.550038}%
\pgfsetfillcolor{currentfill}%
\pgfsetfillopacity{0.800000}%
\pgfsetlinewidth{0.000000pt}%
\definecolor{currentstroke}{rgb}{0.000000,0.000000,0.000000}%
\pgfsetstrokecolor{currentstroke}%
\pgfsetdash{}{0pt}%
\pgfpathmoveto{\pgfqpoint{2.454122in}{2.690118in}}%
\pgfpathlineto{\pgfqpoint{2.468037in}{2.667820in}}%
\pgfpathlineto{\pgfqpoint{2.481937in}{2.645860in}}%
\pgfpathlineto{\pgfqpoint{2.495825in}{2.624238in}}%
\pgfpathlineto{\pgfqpoint{2.509700in}{2.602948in}}%
\pgfpathlineto{\pgfqpoint{2.518308in}{2.606902in}}%
\pgfpathlineto{\pgfqpoint{2.526902in}{2.611050in}}%
\pgfpathlineto{\pgfqpoint{2.535484in}{2.615388in}}%
\pgfpathlineto{\pgfqpoint{2.544052in}{2.619913in}}%
\pgfpathlineto{\pgfqpoint{2.530214in}{2.640871in}}%
\pgfpathlineto{\pgfqpoint{2.516363in}{2.662161in}}%
\pgfpathlineto{\pgfqpoint{2.502500in}{2.683786in}}%
\pgfpathlineto{\pgfqpoint{2.488623in}{2.705751in}}%
\pgfpathlineto{\pgfqpoint{2.480019in}{2.701546in}}%
\pgfpathlineto{\pgfqpoint{2.471401in}{2.697537in}}%
\pgfpathlineto{\pgfqpoint{2.462768in}{2.693727in}}%
\pgfpathlineto{\pgfqpoint{2.454122in}{2.690118in}}%
\pgfpathclose%
\pgfusepath{fill}%
\end{pgfscope}%
\begin{pgfscope}%
\pgfpathrectangle{\pgfqpoint{1.150000in}{0.150000in}}{\pgfqpoint{5.700000in}{5.700000in}}%
\pgfusepath{clip}%
\pgfsetbuttcap%
\pgfsetroundjoin%
\definecolor{currentfill}{rgb}{0.283197,0.115680,0.436115}%
\pgfsetfillcolor{currentfill}%
\pgfsetfillopacity{0.800000}%
\pgfsetlinewidth{0.000000pt}%
\definecolor{currentstroke}{rgb}{0.000000,0.000000,0.000000}%
\pgfsetstrokecolor{currentstroke}%
\pgfsetdash{}{0pt}%
\pgfpathmoveto{\pgfqpoint{3.828454in}{2.072151in}}%
\pgfpathlineto{\pgfqpoint{3.842085in}{2.072457in}}%
\pgfpathlineto{\pgfqpoint{3.855724in}{2.072962in}}%
\pgfpathlineto{\pgfqpoint{3.869370in}{2.073664in}}%
\pgfpathlineto{\pgfqpoint{3.883024in}{2.074564in}}%
\pgfpathlineto{\pgfqpoint{3.891005in}{2.085239in}}%
\pgfpathlineto{\pgfqpoint{3.898981in}{2.095891in}}%
\pgfpathlineto{\pgfqpoint{3.906953in}{2.106519in}}%
\pgfpathlineto{\pgfqpoint{3.914919in}{2.117123in}}%
\pgfpathlineto{\pgfqpoint{3.901272in}{2.116136in}}%
\pgfpathlineto{\pgfqpoint{3.887633in}{2.115347in}}%
\pgfpathlineto{\pgfqpoint{3.874003in}{2.114756in}}%
\pgfpathlineto{\pgfqpoint{3.860379in}{2.114362in}}%
\pgfpathlineto{\pgfqpoint{3.852406in}{2.103834in}}%
\pgfpathlineto{\pgfqpoint{3.844427in}{2.093289in}}%
\pgfpathlineto{\pgfqpoint{3.836443in}{2.082728in}}%
\pgfpathlineto{\pgfqpoint{3.828454in}{2.072151in}}%
\pgfpathclose%
\pgfusepath{fill}%
\end{pgfscope}%
\begin{pgfscope}%
\pgfpathrectangle{\pgfqpoint{1.150000in}{0.150000in}}{\pgfqpoint{5.700000in}{5.700000in}}%
\pgfusepath{clip}%
\pgfsetbuttcap%
\pgfsetroundjoin%
\definecolor{currentfill}{rgb}{0.282884,0.135920,0.453427}%
\pgfsetfillcolor{currentfill}%
\pgfsetfillopacity{0.800000}%
\pgfsetlinewidth{0.000000pt}%
\definecolor{currentstroke}{rgb}{0.000000,0.000000,0.000000}%
\pgfsetstrokecolor{currentstroke}%
\pgfsetdash{}{0pt}%
\pgfpathmoveto{\pgfqpoint{3.914919in}{2.117123in}}%
\pgfpathlineto{\pgfqpoint{3.928573in}{2.118306in}}%
\pgfpathlineto{\pgfqpoint{3.942237in}{2.119685in}}%
\pgfpathlineto{\pgfqpoint{3.955908in}{2.121259in}}%
\pgfpathlineto{\pgfqpoint{3.969589in}{2.123029in}}%
\pgfpathlineto{\pgfqpoint{3.977543in}{2.133674in}}%
\pgfpathlineto{\pgfqpoint{3.985491in}{2.144287in}}%
\pgfpathlineto{\pgfqpoint{3.993435in}{2.154867in}}%
\pgfpathlineto{\pgfqpoint{4.001374in}{2.165414in}}%
\pgfpathlineto{\pgfqpoint{3.987700in}{2.163590in}}%
\pgfpathlineto{\pgfqpoint{3.974035in}{2.161960in}}%
\pgfpathlineto{\pgfqpoint{3.960379in}{2.160526in}}%
\pgfpathlineto{\pgfqpoint{3.946732in}{2.159288in}}%
\pgfpathlineto{\pgfqpoint{3.938786in}{2.148784in}}%
\pgfpathlineto{\pgfqpoint{3.930835in}{2.138255in}}%
\pgfpathlineto{\pgfqpoint{3.922879in}{2.127702in}}%
\pgfpathlineto{\pgfqpoint{3.914919in}{2.117123in}}%
\pgfpathclose%
\pgfusepath{fill}%
\end{pgfscope}%
\begin{pgfscope}%
\pgfpathrectangle{\pgfqpoint{1.150000in}{0.150000in}}{\pgfqpoint{5.700000in}{5.700000in}}%
\pgfusepath{clip}%
\pgfsetbuttcap%
\pgfsetroundjoin%
\definecolor{currentfill}{rgb}{0.163625,0.471133,0.558148}%
\pgfsetfillcolor{currentfill}%
\pgfsetfillopacity{0.800000}%
\pgfsetlinewidth{0.000000pt}%
\definecolor{currentstroke}{rgb}{0.000000,0.000000,0.000000}%
\pgfsetstrokecolor{currentstroke}%
\pgfsetdash{}{0pt}%
\pgfpathmoveto{\pgfqpoint{5.188535in}{2.980920in}}%
\pgfpathlineto{\pgfqpoint{5.202752in}{2.989481in}}%
\pgfpathlineto{\pgfqpoint{5.216986in}{2.998220in}}%
\pgfpathlineto{\pgfqpoint{5.231236in}{3.007137in}}%
\pgfpathlineto{\pgfqpoint{5.245504in}{3.016231in}}%
\pgfpathlineto{\pgfqpoint{5.252944in}{3.021687in}}%
\pgfpathlineto{\pgfqpoint{5.260379in}{3.027158in}}%
\pgfpathlineto{\pgfqpoint{5.267808in}{3.032651in}}%
\pgfpathlineto{\pgfqpoint{5.275231in}{3.038171in}}%
\pgfpathlineto{\pgfqpoint{5.260984in}{3.029547in}}%
\pgfpathlineto{\pgfqpoint{5.246753in}{3.021100in}}%
\pgfpathlineto{\pgfqpoint{5.232538in}{3.012831in}}%
\pgfpathlineto{\pgfqpoint{5.218341in}{3.004739in}}%
\pgfpathlineto{\pgfqpoint{5.210897in}{2.998738in}}%
\pgfpathlineto{\pgfqpoint{5.203448in}{2.992772in}}%
\pgfpathlineto{\pgfqpoint{5.195994in}{2.986834in}}%
\pgfpathlineto{\pgfqpoint{5.188535in}{2.980920in}}%
\pgfpathclose%
\pgfusepath{fill}%
\end{pgfscope}%
\begin{pgfscope}%
\pgfpathrectangle{\pgfqpoint{1.150000in}{0.150000in}}{\pgfqpoint{5.700000in}{5.700000in}}%
\pgfusepath{clip}%
\pgfsetbuttcap%
\pgfsetroundjoin%
\definecolor{currentfill}{rgb}{0.282327,0.094955,0.417331}%
\pgfsetfillcolor{currentfill}%
\pgfsetfillopacity{0.800000}%
\pgfsetlinewidth{0.000000pt}%
\definecolor{currentstroke}{rgb}{0.000000,0.000000,0.000000}%
\pgfsetstrokecolor{currentstroke}%
\pgfsetdash{}{0pt}%
\pgfpathmoveto{\pgfqpoint{3.741961in}{2.030945in}}%
\pgfpathlineto{\pgfqpoint{3.755572in}{2.030333in}}%
\pgfpathlineto{\pgfqpoint{3.769190in}{2.029922in}}%
\pgfpathlineto{\pgfqpoint{3.782815in}{2.029711in}}%
\pgfpathlineto{\pgfqpoint{3.796447in}{2.029700in}}%
\pgfpathlineto{\pgfqpoint{3.804457in}{2.040333in}}%
\pgfpathlineto{\pgfqpoint{3.812461in}{2.050953in}}%
\pgfpathlineto{\pgfqpoint{3.820460in}{2.061559in}}%
\pgfpathlineto{\pgfqpoint{3.828454in}{2.072151in}}%
\pgfpathlineto{\pgfqpoint{3.814831in}{2.072044in}}%
\pgfpathlineto{\pgfqpoint{3.801214in}{2.072136in}}%
\pgfpathlineto{\pgfqpoint{3.787605in}{2.072428in}}%
\pgfpathlineto{\pgfqpoint{3.774002in}{2.072921in}}%
\pgfpathlineto{\pgfqpoint{3.766000in}{2.062436in}}%
\pgfpathlineto{\pgfqpoint{3.757992in}{2.051945in}}%
\pgfpathlineto{\pgfqpoint{3.749979in}{2.041447in}}%
\pgfpathlineto{\pgfqpoint{3.741961in}{2.030945in}}%
\pgfpathclose%
\pgfusepath{fill}%
\end{pgfscope}%
\begin{pgfscope}%
\pgfpathrectangle{\pgfqpoint{1.150000in}{0.150000in}}{\pgfqpoint{5.700000in}{5.700000in}}%
\pgfusepath{clip}%
\pgfsetbuttcap%
\pgfsetroundjoin%
\definecolor{currentfill}{rgb}{0.281412,0.155834,0.469201}%
\pgfsetfillcolor{currentfill}%
\pgfsetfillopacity{0.800000}%
\pgfsetlinewidth{0.000000pt}%
\definecolor{currentstroke}{rgb}{0.000000,0.000000,0.000000}%
\pgfsetstrokecolor{currentstroke}%
\pgfsetdash{}{0pt}%
\pgfpathmoveto{\pgfqpoint{4.001374in}{2.165414in}}%
\pgfpathlineto{\pgfqpoint{4.015056in}{2.167433in}}%
\pgfpathlineto{\pgfqpoint{4.028748in}{2.169646in}}%
\pgfpathlineto{\pgfqpoint{4.042449in}{2.172053in}}%
\pgfpathlineto{\pgfqpoint{4.056159in}{2.174652in}}%
\pgfpathlineto{\pgfqpoint{4.064086in}{2.185201in}}%
\pgfpathlineto{\pgfqpoint{4.072007in}{2.195709in}}%
\pgfpathlineto{\pgfqpoint{4.079924in}{2.206176in}}%
\pgfpathlineto{\pgfqpoint{4.087836in}{2.216603in}}%
\pgfpathlineto{\pgfqpoint{4.074132in}{2.213981in}}%
\pgfpathlineto{\pgfqpoint{4.060438in}{2.211551in}}%
\pgfpathlineto{\pgfqpoint{4.046753in}{2.209315in}}%
\pgfpathlineto{\pgfqpoint{4.033077in}{2.207272in}}%
\pgfpathlineto{\pgfqpoint{4.025159in}{2.196857in}}%
\pgfpathlineto{\pgfqpoint{4.017236in}{2.186409in}}%
\pgfpathlineto{\pgfqpoint{4.009307in}{2.175928in}}%
\pgfpathlineto{\pgfqpoint{4.001374in}{2.165414in}}%
\pgfpathclose%
\pgfusepath{fill}%
\end{pgfscope}%
\begin{pgfscope}%
\pgfpathrectangle{\pgfqpoint{1.150000in}{0.150000in}}{\pgfqpoint{5.700000in}{5.700000in}}%
\pgfusepath{clip}%
\pgfsetbuttcap%
\pgfsetroundjoin%
\definecolor{currentfill}{rgb}{0.281446,0.084320,0.407414}%
\pgfsetfillcolor{currentfill}%
\pgfsetfillopacity{0.800000}%
\pgfsetlinewidth{0.000000pt}%
\definecolor{currentstroke}{rgb}{0.000000,0.000000,0.000000}%
\pgfsetstrokecolor{currentstroke}%
\pgfsetdash{}{0pt}%
\pgfpathmoveto{\pgfqpoint{3.003279in}{2.035105in}}%
\pgfpathlineto{\pgfqpoint{3.016889in}{2.024308in}}%
\pgfpathlineto{\pgfqpoint{3.030497in}{2.013756in}}%
\pgfpathlineto{\pgfqpoint{3.044103in}{2.003447in}}%
\pgfpathlineto{\pgfqpoint{3.057707in}{1.993380in}}%
\pgfpathlineto{\pgfqpoint{3.066017in}{2.000659in}}%
\pgfpathlineto{\pgfqpoint{3.074318in}{2.008048in}}%
\pgfpathlineto{\pgfqpoint{3.082611in}{2.015545in}}%
\pgfpathlineto{\pgfqpoint{3.090895in}{2.023147in}}%
\pgfpathlineto{\pgfqpoint{3.077314in}{2.032903in}}%
\pgfpathlineto{\pgfqpoint{3.063731in}{2.042900in}}%
\pgfpathlineto{\pgfqpoint{3.050146in}{2.053139in}}%
\pgfpathlineto{\pgfqpoint{3.036560in}{2.063623in}}%
\pgfpathlineto{\pgfqpoint{3.028253in}{2.056321in}}%
\pgfpathlineto{\pgfqpoint{3.019937in}{2.049132in}}%
\pgfpathlineto{\pgfqpoint{3.011613in}{2.042059in}}%
\pgfpathlineto{\pgfqpoint{3.003279in}{2.035105in}}%
\pgfpathclose%
\pgfusepath{fill}%
\end{pgfscope}%
\begin{pgfscope}%
\pgfpathrectangle{\pgfqpoint{1.150000in}{0.150000in}}{\pgfqpoint{5.700000in}{5.700000in}}%
\pgfusepath{clip}%
\pgfsetbuttcap%
\pgfsetroundjoin%
\definecolor{currentfill}{rgb}{0.221989,0.339161,0.548752}%
\pgfsetfillcolor{currentfill}%
\pgfsetfillopacity{0.800000}%
\pgfsetlinewidth{0.000000pt}%
\definecolor{currentstroke}{rgb}{0.000000,0.000000,0.000000}%
\pgfsetstrokecolor{currentstroke}%
\pgfsetdash{}{0pt}%
\pgfpathmoveto{\pgfqpoint{4.638233in}{2.597946in}}%
\pgfpathlineto{\pgfqpoint{4.652180in}{2.604524in}}%
\pgfpathlineto{\pgfqpoint{4.666140in}{2.611286in}}%
\pgfpathlineto{\pgfqpoint{4.680115in}{2.618232in}}%
\pgfpathlineto{\pgfqpoint{4.694103in}{2.625362in}}%
\pgfpathlineto{\pgfqpoint{4.701801in}{2.633553in}}%
\pgfpathlineto{\pgfqpoint{4.709493in}{2.641694in}}%
\pgfpathlineto{\pgfqpoint{4.717178in}{2.649790in}}%
\pgfpathlineto{\pgfqpoint{4.724858in}{2.657842in}}%
\pgfpathlineto{\pgfqpoint{4.710879in}{2.650950in}}%
\pgfpathlineto{\pgfqpoint{4.696914in}{2.644242in}}%
\pgfpathlineto{\pgfqpoint{4.682964in}{2.637716in}}%
\pgfpathlineto{\pgfqpoint{4.669027in}{2.631374in}}%
\pgfpathlineto{\pgfqpoint{4.661337in}{2.623073in}}%
\pgfpathlineto{\pgfqpoint{4.653642in}{2.614737in}}%
\pgfpathlineto{\pgfqpoint{4.645940in}{2.606362in}}%
\pgfpathlineto{\pgfqpoint{4.638233in}{2.597946in}}%
\pgfpathclose%
\pgfusepath{fill}%
\end{pgfscope}%
\begin{pgfscope}%
\pgfpathrectangle{\pgfqpoint{1.150000in}{0.150000in}}{\pgfqpoint{5.700000in}{5.700000in}}%
\pgfusepath{clip}%
\pgfsetbuttcap%
\pgfsetroundjoin%
\definecolor{currentfill}{rgb}{0.278012,0.180367,0.486697}%
\pgfsetfillcolor{currentfill}%
\pgfsetfillopacity{0.800000}%
\pgfsetlinewidth{0.000000pt}%
\definecolor{currentstroke}{rgb}{0.000000,0.000000,0.000000}%
\pgfsetstrokecolor{currentstroke}%
\pgfsetdash{}{0pt}%
\pgfpathmoveto{\pgfqpoint{4.087836in}{2.216603in}}%
\pgfpathlineto{\pgfqpoint{4.101549in}{2.219418in}}%
\pgfpathlineto{\pgfqpoint{4.115273in}{2.222426in}}%
\pgfpathlineto{\pgfqpoint{4.129006in}{2.225625in}}%
\pgfpathlineto{\pgfqpoint{4.142750in}{2.229015in}}%
\pgfpathlineto{\pgfqpoint{4.150650in}{2.239405in}}%
\pgfpathlineto{\pgfqpoint{4.158544in}{2.249747in}}%
\pgfpathlineto{\pgfqpoint{4.166434in}{2.260043in}}%
\pgfpathlineto{\pgfqpoint{4.174318in}{2.270291in}}%
\pgfpathlineto{\pgfqpoint{4.160581in}{2.266910in}}%
\pgfpathlineto{\pgfqpoint{4.146854in}{2.263720in}}%
\pgfpathlineto{\pgfqpoint{4.133137in}{2.260721in}}%
\pgfpathlineto{\pgfqpoint{4.119430in}{2.257914in}}%
\pgfpathlineto{\pgfqpoint{4.111539in}{2.247645in}}%
\pgfpathlineto{\pgfqpoint{4.103643in}{2.237337in}}%
\pgfpathlineto{\pgfqpoint{4.095742in}{2.226990in}}%
\pgfpathlineto{\pgfqpoint{4.087836in}{2.216603in}}%
\pgfpathclose%
\pgfusepath{fill}%
\end{pgfscope}%
\begin{pgfscope}%
\pgfpathrectangle{\pgfqpoint{1.150000in}{0.150000in}}{\pgfqpoint{5.700000in}{5.700000in}}%
\pgfusepath{clip}%
\pgfsetbuttcap%
\pgfsetroundjoin%
\definecolor{currentfill}{rgb}{0.280894,0.078907,0.402329}%
\pgfsetfillcolor{currentfill}%
\pgfsetfillopacity{0.800000}%
\pgfsetlinewidth{0.000000pt}%
\definecolor{currentstroke}{rgb}{0.000000,0.000000,0.000000}%
\pgfsetstrokecolor{currentstroke}%
\pgfsetdash{}{0pt}%
\pgfpathmoveto{\pgfqpoint{3.655417in}{1.993973in}}%
\pgfpathlineto{\pgfqpoint{3.669013in}{1.992401in}}%
\pgfpathlineto{\pgfqpoint{3.682615in}{1.991033in}}%
\pgfpathlineto{\pgfqpoint{3.696223in}{1.989867in}}%
\pgfpathlineto{\pgfqpoint{3.709837in}{1.988904in}}%
\pgfpathlineto{\pgfqpoint{3.717876in}{1.999416in}}%
\pgfpathlineto{\pgfqpoint{3.725909in}{2.009928in}}%
\pgfpathlineto{\pgfqpoint{3.733938in}{2.020438in}}%
\pgfpathlineto{\pgfqpoint{3.741961in}{2.030945in}}%
\pgfpathlineto{\pgfqpoint{3.728357in}{2.031758in}}%
\pgfpathlineto{\pgfqpoint{3.714758in}{2.032774in}}%
\pgfpathlineto{\pgfqpoint{3.701166in}{2.033992in}}%
\pgfpathlineto{\pgfqpoint{3.687580in}{2.035414in}}%
\pgfpathlineto{\pgfqpoint{3.679547in}{2.025046in}}%
\pgfpathlineto{\pgfqpoint{3.671509in}{2.014682in}}%
\pgfpathlineto{\pgfqpoint{3.663466in}{2.004324in}}%
\pgfpathlineto{\pgfqpoint{3.655417in}{1.993973in}}%
\pgfpathclose%
\pgfusepath{fill}%
\end{pgfscope}%
\begin{pgfscope}%
\pgfpathrectangle{\pgfqpoint{1.150000in}{0.150000in}}{\pgfqpoint{5.700000in}{5.700000in}}%
\pgfusepath{clip}%
\pgfsetbuttcap%
\pgfsetroundjoin%
\definecolor{currentfill}{rgb}{0.156270,0.489624,0.557936}%
\pgfsetfillcolor{currentfill}%
\pgfsetfillopacity{0.800000}%
\pgfsetlinewidth{0.000000pt}%
\definecolor{currentstroke}{rgb}{0.000000,0.000000,0.000000}%
\pgfsetstrokecolor{currentstroke}%
\pgfsetdash{}{0pt}%
\pgfpathmoveto{\pgfqpoint{5.275231in}{3.038171in}}%
\pgfpathlineto{\pgfqpoint{5.289496in}{3.046972in}}%
\pgfpathlineto{\pgfqpoint{5.303777in}{3.055950in}}%
\pgfpathlineto{\pgfqpoint{5.318076in}{3.065106in}}%
\pgfpathlineto{\pgfqpoint{5.332393in}{3.074439in}}%
\pgfpathlineto{\pgfqpoint{5.339790in}{3.079499in}}%
\pgfpathlineto{\pgfqpoint{5.347182in}{3.084588in}}%
\pgfpathlineto{\pgfqpoint{5.354568in}{3.089713in}}%
\pgfpathlineto{\pgfqpoint{5.361950in}{3.094880in}}%
\pgfpathlineto{\pgfqpoint{5.347655in}{3.086051in}}%
\pgfpathlineto{\pgfqpoint{5.333378in}{3.077400in}}%
\pgfpathlineto{\pgfqpoint{5.319118in}{3.068924in}}%
\pgfpathlineto{\pgfqpoint{5.304874in}{3.060625in}}%
\pgfpathlineto{\pgfqpoint{5.297471in}{3.054944in}}%
\pgfpathlineto{\pgfqpoint{5.290063in}{3.049312in}}%
\pgfpathlineto{\pgfqpoint{5.282650in}{3.043723in}}%
\pgfpathlineto{\pgfqpoint{5.275231in}{3.038171in}}%
\pgfpathclose%
\pgfusepath{fill}%
\end{pgfscope}%
\begin{pgfscope}%
\pgfpathrectangle{\pgfqpoint{1.150000in}{0.150000in}}{\pgfqpoint{5.700000in}{5.700000in}}%
\pgfusepath{clip}%
\pgfsetbuttcap%
\pgfsetroundjoin%
\definecolor{currentfill}{rgb}{0.277018,0.050344,0.375715}%
\pgfsetfillcolor{currentfill}%
\pgfsetfillopacity{0.800000}%
\pgfsetlinewidth{0.000000pt}%
\definecolor{currentstroke}{rgb}{0.000000,0.000000,0.000000}%
\pgfsetstrokecolor{currentstroke}%
\pgfsetdash{}{0pt}%
\pgfpathmoveto{\pgfqpoint{3.199514in}{1.953615in}}%
\pgfpathlineto{\pgfqpoint{3.213091in}{1.945966in}}%
\pgfpathlineto{\pgfqpoint{3.226668in}{1.938545in}}%
\pgfpathlineto{\pgfqpoint{3.240246in}{1.931351in}}%
\pgfpathlineto{\pgfqpoint{3.253825in}{1.924381in}}%
\pgfpathlineto{\pgfqpoint{3.262041in}{1.932957in}}%
\pgfpathlineto{\pgfqpoint{3.270250in}{1.941608in}}%
\pgfpathlineto{\pgfqpoint{3.278452in}{1.950333in}}%
\pgfpathlineto{\pgfqpoint{3.286647in}{1.959129in}}%
\pgfpathlineto{\pgfqpoint{3.273086in}{1.965822in}}%
\pgfpathlineto{\pgfqpoint{3.259526in}{1.972739in}}%
\pgfpathlineto{\pgfqpoint{3.245967in}{1.979882in}}%
\pgfpathlineto{\pgfqpoint{3.232410in}{1.987252in}}%
\pgfpathlineto{\pgfqpoint{3.224197in}{1.978722in}}%
\pgfpathlineto{\pgfqpoint{3.215977in}{1.970271in}}%
\pgfpathlineto{\pgfqpoint{3.207749in}{1.961901in}}%
\pgfpathlineto{\pgfqpoint{3.199514in}{1.953615in}}%
\pgfpathclose%
\pgfusepath{fill}%
\end{pgfscope}%
\begin{pgfscope}%
\pgfpathrectangle{\pgfqpoint{1.150000in}{0.150000in}}{\pgfqpoint{5.700000in}{5.700000in}}%
\pgfusepath{clip}%
\pgfsetbuttcap%
\pgfsetroundjoin%
\definecolor{currentfill}{rgb}{0.276022,0.044167,0.370164}%
\pgfsetfillcolor{currentfill}%
\pgfsetfillopacity{0.800000}%
\pgfsetlinewidth{0.000000pt}%
\definecolor{currentstroke}{rgb}{0.000000,0.000000,0.000000}%
\pgfsetstrokecolor{currentstroke}%
\pgfsetdash{}{0pt}%
\pgfpathmoveto{\pgfqpoint{3.340906in}{1.934585in}}%
\pgfpathlineto{\pgfqpoint{3.354475in}{1.928999in}}%
\pgfpathlineto{\pgfqpoint{3.368047in}{1.923632in}}%
\pgfpathlineto{\pgfqpoint{3.381622in}{1.918481in}}%
\pgfpathlineto{\pgfqpoint{3.395199in}{1.913546in}}%
\pgfpathlineto{\pgfqpoint{3.403355in}{1.922926in}}%
\pgfpathlineto{\pgfqpoint{3.411505in}{1.932356in}}%
\pgfpathlineto{\pgfqpoint{3.419648in}{1.941834in}}%
\pgfpathlineto{\pgfqpoint{3.427785in}{1.951359in}}%
\pgfpathlineto{\pgfqpoint{3.414222in}{1.956049in}}%
\pgfpathlineto{\pgfqpoint{3.400663in}{1.960954in}}%
\pgfpathlineto{\pgfqpoint{3.387106in}{1.966076in}}%
\pgfpathlineto{\pgfqpoint{3.373552in}{1.971417in}}%
\pgfpathlineto{\pgfqpoint{3.365400in}{1.962125in}}%
\pgfpathlineto{\pgfqpoint{3.357242in}{1.952888in}}%
\pgfpathlineto{\pgfqpoint{3.349077in}{1.943707in}}%
\pgfpathlineto{\pgfqpoint{3.340906in}{1.934585in}}%
\pgfpathclose%
\pgfusepath{fill}%
\end{pgfscope}%
\begin{pgfscope}%
\pgfpathrectangle{\pgfqpoint{1.150000in}{0.150000in}}{\pgfqpoint{5.700000in}{5.700000in}}%
\pgfusepath{clip}%
\pgfsetbuttcap%
\pgfsetroundjoin%
\definecolor{currentfill}{rgb}{0.271828,0.209303,0.504434}%
\pgfsetfillcolor{currentfill}%
\pgfsetfillopacity{0.800000}%
\pgfsetlinewidth{0.000000pt}%
\definecolor{currentstroke}{rgb}{0.000000,0.000000,0.000000}%
\pgfsetstrokecolor{currentstroke}%
\pgfsetdash{}{0pt}%
\pgfpathmoveto{\pgfqpoint{4.174318in}{2.270291in}}%
\pgfpathlineto{\pgfqpoint{4.188066in}{2.273863in}}%
\pgfpathlineto{\pgfqpoint{4.201824in}{2.277626in}}%
\pgfpathlineto{\pgfqpoint{4.215593in}{2.281579in}}%
\pgfpathlineto{\pgfqpoint{4.229373in}{2.285721in}}%
\pgfpathlineto{\pgfqpoint{4.237246in}{2.295895in}}%
\pgfpathlineto{\pgfqpoint{4.245114in}{2.306016in}}%
\pgfpathlineto{\pgfqpoint{4.252976in}{2.316084in}}%
\pgfpathlineto{\pgfqpoint{4.260833in}{2.326101in}}%
\pgfpathlineto{\pgfqpoint{4.247059in}{2.322000in}}%
\pgfpathlineto{\pgfqpoint{4.233296in}{2.318089in}}%
\pgfpathlineto{\pgfqpoint{4.219545in}{2.314367in}}%
\pgfpathlineto{\pgfqpoint{4.205803in}{2.310836in}}%
\pgfpathlineto{\pgfqpoint{4.197940in}{2.300765in}}%
\pgfpathlineto{\pgfqpoint{4.190071in}{2.290652in}}%
\pgfpathlineto{\pgfqpoint{4.182198in}{2.280494in}}%
\pgfpathlineto{\pgfqpoint{4.174318in}{2.270291in}}%
\pgfpathclose%
\pgfusepath{fill}%
\end{pgfscope}%
\begin{pgfscope}%
\pgfpathrectangle{\pgfqpoint{1.150000in}{0.150000in}}{\pgfqpoint{5.700000in}{5.700000in}}%
\pgfusepath{clip}%
\pgfsetbuttcap%
\pgfsetroundjoin%
\definecolor{currentfill}{rgb}{0.210503,0.363727,0.552206}%
\pgfsetfillcolor{currentfill}%
\pgfsetfillopacity{0.800000}%
\pgfsetlinewidth{0.000000pt}%
\definecolor{currentstroke}{rgb}{0.000000,0.000000,0.000000}%
\pgfsetstrokecolor{currentstroke}%
\pgfsetdash{}{0pt}%
\pgfpathmoveto{\pgfqpoint{4.724858in}{2.657842in}}%
\pgfpathlineto{\pgfqpoint{4.738851in}{2.664916in}}%
\pgfpathlineto{\pgfqpoint{4.752858in}{2.672174in}}%
\pgfpathlineto{\pgfqpoint{4.766880in}{2.679614in}}%
\pgfpathlineto{\pgfqpoint{4.780916in}{2.687237in}}%
\pgfpathlineto{\pgfqpoint{4.788579in}{2.694989in}}%
\pgfpathlineto{\pgfqpoint{4.796235in}{2.702697in}}%
\pgfpathlineto{\pgfqpoint{4.803886in}{2.710363in}}%
\pgfpathlineto{\pgfqpoint{4.811530in}{2.717989in}}%
\pgfpathlineto{\pgfqpoint{4.797504in}{2.710638in}}%
\pgfpathlineto{\pgfqpoint{4.783494in}{2.703469in}}%
\pgfpathlineto{\pgfqpoint{4.769497in}{2.696482in}}%
\pgfpathlineto{\pgfqpoint{4.755515in}{2.689677in}}%
\pgfpathlineto{\pgfqpoint{4.747860in}{2.681768in}}%
\pgfpathlineto{\pgfqpoint{4.740198in}{2.673827in}}%
\pgfpathlineto{\pgfqpoint{4.732531in}{2.665853in}}%
\pgfpathlineto{\pgfqpoint{4.724858in}{2.657842in}}%
\pgfpathclose%
\pgfusepath{fill}%
\end{pgfscope}%
\begin{pgfscope}%
\pgfpathrectangle{\pgfqpoint{1.150000in}{0.150000in}}{\pgfqpoint{5.700000in}{5.700000in}}%
\pgfusepath{clip}%
\pgfsetbuttcap%
\pgfsetroundjoin%
\definecolor{currentfill}{rgb}{0.147607,0.511733,0.557049}%
\pgfsetfillcolor{currentfill}%
\pgfsetfillopacity{0.800000}%
\pgfsetlinewidth{0.000000pt}%
\definecolor{currentstroke}{rgb}{0.000000,0.000000,0.000000}%
\pgfsetstrokecolor{currentstroke}%
\pgfsetdash{}{0pt}%
\pgfpathmoveto{\pgfqpoint{5.361950in}{3.094880in}}%
\pgfpathlineto{\pgfqpoint{5.376261in}{3.103885in}}%
\pgfpathlineto{\pgfqpoint{5.390591in}{3.113066in}}%
\pgfpathlineto{\pgfqpoint{5.404937in}{3.122424in}}%
\pgfpathlineto{\pgfqpoint{5.419302in}{3.131959in}}%
\pgfpathlineto{\pgfqpoint{5.426655in}{3.136646in}}%
\pgfpathlineto{\pgfqpoint{5.434004in}{3.141379in}}%
\pgfpathlineto{\pgfqpoint{5.441347in}{3.146162in}}%
\pgfpathlineto{\pgfqpoint{5.448686in}{3.151003in}}%
\pgfpathlineto{\pgfqpoint{5.434346in}{3.142007in}}%
\pgfpathlineto{\pgfqpoint{5.420023in}{3.133187in}}%
\pgfpathlineto{\pgfqpoint{5.405717in}{3.124542in}}%
\pgfpathlineto{\pgfqpoint{5.391428in}{3.116073in}}%
\pgfpathlineto{\pgfqpoint{5.384065in}{3.110684in}}%
\pgfpathlineto{\pgfqpoint{5.376698in}{3.105359in}}%
\pgfpathlineto{\pgfqpoint{5.369326in}{3.100093in}}%
\pgfpathlineto{\pgfqpoint{5.361950in}{3.094880in}}%
\pgfpathclose%
\pgfusepath{fill}%
\end{pgfscope}%
\begin{pgfscope}%
\pgfpathrectangle{\pgfqpoint{1.150000in}{0.150000in}}{\pgfqpoint{5.700000in}{5.700000in}}%
\pgfusepath{clip}%
\pgfsetbuttcap%
\pgfsetroundjoin%
\definecolor{currentfill}{rgb}{0.278791,0.062145,0.386592}%
\pgfsetfillcolor{currentfill}%
\pgfsetfillopacity{0.800000}%
\pgfsetlinewidth{0.000000pt}%
\definecolor{currentstroke}{rgb}{0.000000,0.000000,0.000000}%
\pgfsetstrokecolor{currentstroke}%
\pgfsetdash{}{0pt}%
\pgfpathmoveto{\pgfqpoint{3.568796in}{1.961731in}}%
\pgfpathlineto{\pgfqpoint{3.582381in}{1.959155in}}%
\pgfpathlineto{\pgfqpoint{3.595972in}{1.956786in}}%
\pgfpathlineto{\pgfqpoint{3.609567in}{1.954623in}}%
\pgfpathlineto{\pgfqpoint{3.623168in}{1.952664in}}%
\pgfpathlineto{\pgfqpoint{3.631238in}{1.962975in}}%
\pgfpathlineto{\pgfqpoint{3.639303in}{1.973297in}}%
\pgfpathlineto{\pgfqpoint{3.647363in}{1.983631in}}%
\pgfpathlineto{\pgfqpoint{3.655417in}{1.993973in}}%
\pgfpathlineto{\pgfqpoint{3.641827in}{1.995750in}}%
\pgfpathlineto{\pgfqpoint{3.628242in}{1.997732in}}%
\pgfpathlineto{\pgfqpoint{3.614663in}{1.999919in}}%
\pgfpathlineto{\pgfqpoint{3.601089in}{2.002314in}}%
\pgfpathlineto{\pgfqpoint{3.593024in}{1.992141in}}%
\pgfpathlineto{\pgfqpoint{3.584953in}{1.981985in}}%
\pgfpathlineto{\pgfqpoint{3.576877in}{1.971848in}}%
\pgfpathlineto{\pgfqpoint{3.568796in}{1.961731in}}%
\pgfpathclose%
\pgfusepath{fill}%
\end{pgfscope}%
\begin{pgfscope}%
\pgfpathrectangle{\pgfqpoint{1.150000in}{0.150000in}}{\pgfqpoint{5.700000in}{5.700000in}}%
\pgfusepath{clip}%
\pgfsetbuttcap%
\pgfsetroundjoin%
\definecolor{currentfill}{rgb}{0.201239,0.383670,0.554294}%
\pgfsetfillcolor{currentfill}%
\pgfsetfillopacity{0.800000}%
\pgfsetlinewidth{0.000000pt}%
\definecolor{currentstroke}{rgb}{0.000000,0.000000,0.000000}%
\pgfsetstrokecolor{currentstroke}%
\pgfsetdash{}{0pt}%
\pgfpathmoveto{\pgfqpoint{2.398324in}{2.782778in}}%
\pgfpathlineto{\pgfqpoint{2.412296in}{2.759086in}}%
\pgfpathlineto{\pgfqpoint{2.426252in}{2.735748in}}%
\pgfpathlineto{\pgfqpoint{2.440194in}{2.712760in}}%
\pgfpathlineto{\pgfqpoint{2.454122in}{2.690118in}}%
\pgfpathlineto{\pgfqpoint{2.462768in}{2.693727in}}%
\pgfpathlineto{\pgfqpoint{2.471401in}{2.697537in}}%
\pgfpathlineto{\pgfqpoint{2.480019in}{2.701546in}}%
\pgfpathlineto{\pgfqpoint{2.488623in}{2.705751in}}%
\pgfpathlineto{\pgfqpoint{2.474734in}{2.728057in}}%
\pgfpathlineto{\pgfqpoint{2.460830in}{2.750709in}}%
\pgfpathlineto{\pgfqpoint{2.446912in}{2.773710in}}%
\pgfpathlineto{\pgfqpoint{2.432980in}{2.797064in}}%
\pgfpathlineto{\pgfqpoint{2.424337in}{2.793183in}}%
\pgfpathlineto{\pgfqpoint{2.415681in}{2.789506in}}%
\pgfpathlineto{\pgfqpoint{2.407010in}{2.786037in}}%
\pgfpathlineto{\pgfqpoint{2.398324in}{2.782778in}}%
\pgfpathclose%
\pgfusepath{fill}%
\end{pgfscope}%
\begin{pgfscope}%
\pgfpathrectangle{\pgfqpoint{1.150000in}{0.150000in}}{\pgfqpoint{5.700000in}{5.700000in}}%
\pgfusepath{clip}%
\pgfsetbuttcap%
\pgfsetroundjoin%
\definecolor{currentfill}{rgb}{0.265145,0.232956,0.516599}%
\pgfsetfillcolor{currentfill}%
\pgfsetfillopacity{0.800000}%
\pgfsetlinewidth{0.000000pt}%
\definecolor{currentstroke}{rgb}{0.000000,0.000000,0.000000}%
\pgfsetstrokecolor{currentstroke}%
\pgfsetdash{}{0pt}%
\pgfpathmoveto{\pgfqpoint{4.260833in}{2.326101in}}%
\pgfpathlineto{\pgfqpoint{4.274617in}{2.330392in}}%
\pgfpathlineto{\pgfqpoint{4.288413in}{2.334871in}}%
\pgfpathlineto{\pgfqpoint{4.302221in}{2.339539in}}%
\pgfpathlineto{\pgfqpoint{4.316040in}{2.344395in}}%
\pgfpathlineto{\pgfqpoint{4.323885in}{2.354301in}}%
\pgfpathlineto{\pgfqpoint{4.331724in}{2.364149in}}%
\pgfpathlineto{\pgfqpoint{4.339558in}{2.373942in}}%
\pgfpathlineto{\pgfqpoint{4.347387in}{2.383680in}}%
\pgfpathlineto{\pgfqpoint{4.333574in}{2.378897in}}%
\pgfpathlineto{\pgfqpoint{4.319774in}{2.374303in}}%
\pgfpathlineto{\pgfqpoint{4.305984in}{2.369897in}}%
\pgfpathlineto{\pgfqpoint{4.292206in}{2.365680in}}%
\pgfpathlineto{\pgfqpoint{4.284371in}{2.355857in}}%
\pgfpathlineto{\pgfqpoint{4.276530in}{2.345987in}}%
\pgfpathlineto{\pgfqpoint{4.268684in}{2.336069in}}%
\pgfpathlineto{\pgfqpoint{4.260833in}{2.326101in}}%
\pgfpathclose%
\pgfusepath{fill}%
\end{pgfscope}%
\begin{pgfscope}%
\pgfpathrectangle{\pgfqpoint{1.150000in}{0.150000in}}{\pgfqpoint{5.700000in}{5.700000in}}%
\pgfusepath{clip}%
\pgfsetbuttcap%
\pgfsetroundjoin%
\definecolor{currentfill}{rgb}{0.279566,0.067836,0.391917}%
\pgfsetfillcolor{currentfill}%
\pgfsetfillopacity{0.800000}%
\pgfsetlinewidth{0.000000pt}%
\definecolor{currentstroke}{rgb}{0.000000,0.000000,0.000000}%
\pgfsetstrokecolor{currentstroke}%
\pgfsetdash{}{0pt}%
\pgfpathmoveto{\pgfqpoint{3.057707in}{1.993380in}}%
\pgfpathlineto{\pgfqpoint{3.071309in}{1.983552in}}%
\pgfpathlineto{\pgfqpoint{3.084910in}{1.973964in}}%
\pgfpathlineto{\pgfqpoint{3.098510in}{1.964612in}}%
\pgfpathlineto{\pgfqpoint{3.112109in}{1.955496in}}%
\pgfpathlineto{\pgfqpoint{3.120396in}{1.963098in}}%
\pgfpathlineto{\pgfqpoint{3.128675in}{1.970803in}}%
\pgfpathlineto{\pgfqpoint{3.136946in}{1.978607in}}%
\pgfpathlineto{\pgfqpoint{3.145209in}{1.986509in}}%
\pgfpathlineto{\pgfqpoint{3.131632in}{1.995314in}}%
\pgfpathlineto{\pgfqpoint{3.118054in}{2.004354in}}%
\pgfpathlineto{\pgfqpoint{3.104475in}{2.013632in}}%
\pgfpathlineto{\pgfqpoint{3.090895in}{2.023147in}}%
\pgfpathlineto{\pgfqpoint{3.082611in}{2.015545in}}%
\pgfpathlineto{\pgfqpoint{3.074318in}{2.008048in}}%
\pgfpathlineto{\pgfqpoint{3.066017in}{2.000659in}}%
\pgfpathlineto{\pgfqpoint{3.057707in}{1.993380in}}%
\pgfpathclose%
\pgfusepath{fill}%
\end{pgfscope}%
\begin{pgfscope}%
\pgfpathrectangle{\pgfqpoint{1.150000in}{0.150000in}}{\pgfqpoint{5.700000in}{5.700000in}}%
\pgfusepath{clip}%
\pgfsetbuttcap%
\pgfsetroundjoin%
\definecolor{currentfill}{rgb}{0.140536,0.530132,0.555659}%
\pgfsetfillcolor{currentfill}%
\pgfsetfillopacity{0.800000}%
\pgfsetlinewidth{0.000000pt}%
\definecolor{currentstroke}{rgb}{0.000000,0.000000,0.000000}%
\pgfsetstrokecolor{currentstroke}%
\pgfsetdash{}{0pt}%
\pgfpathmoveto{\pgfqpoint{5.448686in}{3.151003in}}%
\pgfpathlineto{\pgfqpoint{5.463045in}{3.160175in}}%
\pgfpathlineto{\pgfqpoint{5.477421in}{3.169523in}}%
\pgfpathlineto{\pgfqpoint{5.491815in}{3.179047in}}%
\pgfpathlineto{\pgfqpoint{5.506227in}{3.188747in}}%
\pgfpathlineto{\pgfqpoint{5.513536in}{3.193091in}}%
\pgfpathlineto{\pgfqpoint{5.520841in}{3.197498in}}%
\pgfpathlineto{\pgfqpoint{5.528141in}{3.201972in}}%
\pgfpathlineto{\pgfqpoint{5.535438in}{3.206521in}}%
\pgfpathlineto{\pgfqpoint{5.521051in}{3.197393in}}%
\pgfpathlineto{\pgfqpoint{5.506683in}{3.188440in}}%
\pgfpathlineto{\pgfqpoint{5.492333in}{3.179663in}}%
\pgfpathlineto{\pgfqpoint{5.477999in}{3.171060in}}%
\pgfpathlineto{\pgfqpoint{5.470677in}{3.165929in}}%
\pgfpathlineto{\pgfqpoint{5.463351in}{3.160881in}}%
\pgfpathlineto{\pgfqpoint{5.456021in}{3.155907in}}%
\pgfpathlineto{\pgfqpoint{5.448686in}{3.151003in}}%
\pgfpathclose%
\pgfusepath{fill}%
\end{pgfscope}%
\begin{pgfscope}%
\pgfpathrectangle{\pgfqpoint{1.150000in}{0.150000in}}{\pgfqpoint{5.700000in}{5.700000in}}%
\pgfusepath{clip}%
\pgfsetbuttcap%
\pgfsetroundjoin%
\definecolor{currentfill}{rgb}{0.275191,0.194905,0.496005}%
\pgfsetfillcolor{currentfill}%
\pgfsetfillopacity{0.800000}%
\pgfsetlinewidth{0.000000pt}%
\definecolor{currentstroke}{rgb}{0.000000,0.000000,0.000000}%
\pgfsetstrokecolor{currentstroke}%
\pgfsetdash{}{0pt}%
\pgfpathmoveto{\pgfqpoint{2.696223in}{2.285571in}}%
\pgfpathlineto{\pgfqpoint{2.709962in}{2.269143in}}%
\pgfpathlineto{\pgfqpoint{2.723693in}{2.252999in}}%
\pgfpathlineto{\pgfqpoint{2.737417in}{2.237136in}}%
\pgfpathlineto{\pgfqpoint{2.751134in}{2.221553in}}%
\pgfpathlineto{\pgfqpoint{2.759620in}{2.226560in}}%
\pgfpathlineto{\pgfqpoint{2.768095in}{2.231731in}}%
\pgfpathlineto{\pgfqpoint{2.776558in}{2.237062in}}%
\pgfpathlineto{\pgfqpoint{2.785011in}{2.242550in}}%
\pgfpathlineto{\pgfqpoint{2.771325in}{2.257780in}}%
\pgfpathlineto{\pgfqpoint{2.757633in}{2.273290in}}%
\pgfpathlineto{\pgfqpoint{2.743933in}{2.289081in}}%
\pgfpathlineto{\pgfqpoint{2.730227in}{2.305155in}}%
\pgfpathlineto{\pgfqpoint{2.721743in}{2.300007in}}%
\pgfpathlineto{\pgfqpoint{2.713248in}{2.295026in}}%
\pgfpathlineto{\pgfqpoint{2.704742in}{2.290212in}}%
\pgfpathlineto{\pgfqpoint{2.696223in}{2.285571in}}%
\pgfpathclose%
\pgfusepath{fill}%
\end{pgfscope}%
\begin{pgfscope}%
\pgfpathrectangle{\pgfqpoint{1.150000in}{0.150000in}}{\pgfqpoint{5.700000in}{5.700000in}}%
\pgfusepath{clip}%
\pgfsetbuttcap%
\pgfsetroundjoin%
\definecolor{currentfill}{rgb}{0.201239,0.383670,0.554294}%
\pgfsetfillcolor{currentfill}%
\pgfsetfillopacity{0.800000}%
\pgfsetlinewidth{0.000000pt}%
\definecolor{currentstroke}{rgb}{0.000000,0.000000,0.000000}%
\pgfsetstrokecolor{currentstroke}%
\pgfsetdash{}{0pt}%
\pgfpathmoveto{\pgfqpoint{4.811530in}{2.717989in}}%
\pgfpathlineto{\pgfqpoint{4.825570in}{2.725523in}}%
\pgfpathlineto{\pgfqpoint{4.839625in}{2.733238in}}%
\pgfpathlineto{\pgfqpoint{4.853695in}{2.741135in}}%
\pgfpathlineto{\pgfqpoint{4.867781in}{2.749214in}}%
\pgfpathlineto{\pgfqpoint{4.875407in}{2.756514in}}%
\pgfpathlineto{\pgfqpoint{4.883027in}{2.763773in}}%
\pgfpathlineto{\pgfqpoint{4.890640in}{2.770997in}}%
\pgfpathlineto{\pgfqpoint{4.898247in}{2.778187in}}%
\pgfpathlineto{\pgfqpoint{4.884175in}{2.770413in}}%
\pgfpathlineto{\pgfqpoint{4.870117in}{2.762821in}}%
\pgfpathlineto{\pgfqpoint{4.856074in}{2.755409in}}%
\pgfpathlineto{\pgfqpoint{4.842046in}{2.748179in}}%
\pgfpathlineto{\pgfqpoint{4.834426in}{2.740672in}}%
\pgfpathlineto{\pgfqpoint{4.826800in}{2.733141in}}%
\pgfpathlineto{\pgfqpoint{4.819168in}{2.725581in}}%
\pgfpathlineto{\pgfqpoint{4.811530in}{2.717989in}}%
\pgfpathclose%
\pgfusepath{fill}%
\end{pgfscope}%
\begin{pgfscope}%
\pgfpathrectangle{\pgfqpoint{1.150000in}{0.150000in}}{\pgfqpoint{5.700000in}{5.700000in}}%
\pgfusepath{clip}%
\pgfsetbuttcap%
\pgfsetroundjoin%
\definecolor{currentfill}{rgb}{0.280255,0.165693,0.476498}%
\pgfsetfillcolor{currentfill}%
\pgfsetfillopacity{0.800000}%
\pgfsetlinewidth{0.000000pt}%
\definecolor{currentstroke}{rgb}{0.000000,0.000000,0.000000}%
\pgfsetstrokecolor{currentstroke}%
\pgfsetdash{}{0pt}%
\pgfpathmoveto{\pgfqpoint{2.751134in}{2.221553in}}%
\pgfpathlineto{\pgfqpoint{2.764844in}{2.206247in}}%
\pgfpathlineto{\pgfqpoint{2.778547in}{2.191216in}}%
\pgfpathlineto{\pgfqpoint{2.792244in}{2.176459in}}%
\pgfpathlineto{\pgfqpoint{2.805936in}{2.161972in}}%
\pgfpathlineto{\pgfqpoint{2.814391in}{2.167342in}}%
\pgfpathlineto{\pgfqpoint{2.822835in}{2.172868in}}%
\pgfpathlineto{\pgfqpoint{2.831269in}{2.178546in}}%
\pgfpathlineto{\pgfqpoint{2.839692in}{2.184373in}}%
\pgfpathlineto{\pgfqpoint{2.826030in}{2.198510in}}%
\pgfpathlineto{\pgfqpoint{2.812363in}{2.212917in}}%
\pgfpathlineto{\pgfqpoint{2.798690in}{2.227596in}}%
\pgfpathlineto{\pgfqpoint{2.785011in}{2.242550in}}%
\pgfpathlineto{\pgfqpoint{2.776558in}{2.237062in}}%
\pgfpathlineto{\pgfqpoint{2.768095in}{2.231731in}}%
\pgfpathlineto{\pgfqpoint{2.759620in}{2.226560in}}%
\pgfpathlineto{\pgfqpoint{2.751134in}{2.221553in}}%
\pgfpathclose%
\pgfusepath{fill}%
\end{pgfscope}%
\begin{pgfscope}%
\pgfpathrectangle{\pgfqpoint{1.150000in}{0.150000in}}{\pgfqpoint{5.700000in}{5.700000in}}%
\pgfusepath{clip}%
\pgfsetbuttcap%
\pgfsetroundjoin%
\definecolor{currentfill}{rgb}{0.267968,0.223549,0.512008}%
\pgfsetfillcolor{currentfill}%
\pgfsetfillopacity{0.800000}%
\pgfsetlinewidth{0.000000pt}%
\definecolor{currentstroke}{rgb}{0.000000,0.000000,0.000000}%
\pgfsetstrokecolor{currentstroke}%
\pgfsetdash{}{0pt}%
\pgfpathmoveto{\pgfqpoint{2.641186in}{2.354173in}}%
\pgfpathlineto{\pgfqpoint{2.654958in}{2.336584in}}%
\pgfpathlineto{\pgfqpoint{2.668722in}{2.319290in}}%
\pgfpathlineto{\pgfqpoint{2.682476in}{2.302286in}}%
\pgfpathlineto{\pgfqpoint{2.696223in}{2.285571in}}%
\pgfpathlineto{\pgfqpoint{2.704742in}{2.290212in}}%
\pgfpathlineto{\pgfqpoint{2.713248in}{2.295026in}}%
\pgfpathlineto{\pgfqpoint{2.721743in}{2.300007in}}%
\pgfpathlineto{\pgfqpoint{2.730227in}{2.305155in}}%
\pgfpathlineto{\pgfqpoint{2.716512in}{2.321515in}}%
\pgfpathlineto{\pgfqpoint{2.702790in}{2.338163in}}%
\pgfpathlineto{\pgfqpoint{2.689060in}{2.355102in}}%
\pgfpathlineto{\pgfqpoint{2.675322in}{2.372334in}}%
\pgfpathlineto{\pgfqpoint{2.666806in}{2.367530in}}%
\pgfpathlineto{\pgfqpoint{2.658279in}{2.362899in}}%
\pgfpathlineto{\pgfqpoint{2.649739in}{2.358446in}}%
\pgfpathlineto{\pgfqpoint{2.641186in}{2.354173in}}%
\pgfpathclose%
\pgfusepath{fill}%
\end{pgfscope}%
\begin{pgfscope}%
\pgfpathrectangle{\pgfqpoint{1.150000in}{0.150000in}}{\pgfqpoint{5.700000in}{5.700000in}}%
\pgfusepath{clip}%
\pgfsetbuttcap%
\pgfsetroundjoin%
\definecolor{currentfill}{rgb}{0.277018,0.050344,0.375715}%
\pgfsetfillcolor{currentfill}%
\pgfsetfillopacity{0.800000}%
\pgfsetlinewidth{0.000000pt}%
\definecolor{currentstroke}{rgb}{0.000000,0.000000,0.000000}%
\pgfsetstrokecolor{currentstroke}%
\pgfsetdash{}{0pt}%
\pgfpathmoveto{\pgfqpoint{3.482068in}{1.934740in}}%
\pgfpathlineto{\pgfqpoint{3.495648in}{1.931115in}}%
\pgfpathlineto{\pgfqpoint{3.509232in}{1.927700in}}%
\pgfpathlineto{\pgfqpoint{3.522820in}{1.924495in}}%
\pgfpathlineto{\pgfqpoint{3.536413in}{1.921498in}}%
\pgfpathlineto{\pgfqpoint{3.544517in}{1.931518in}}%
\pgfpathlineto{\pgfqpoint{3.552616in}{1.941565in}}%
\pgfpathlineto{\pgfqpoint{3.560709in}{1.951636in}}%
\pgfpathlineto{\pgfqpoint{3.568796in}{1.961731in}}%
\pgfpathlineto{\pgfqpoint{3.555215in}{1.964515in}}%
\pgfpathlineto{\pgfqpoint{3.541639in}{1.967507in}}%
\pgfpathlineto{\pgfqpoint{3.528068in}{1.970709in}}%
\pgfpathlineto{\pgfqpoint{3.514501in}{1.974120in}}%
\pgfpathlineto{\pgfqpoint{3.506401in}{1.964226in}}%
\pgfpathlineto{\pgfqpoint{3.498296in}{1.954364in}}%
\pgfpathlineto{\pgfqpoint{3.490185in}{1.944535in}}%
\pgfpathlineto{\pgfqpoint{3.482068in}{1.934740in}}%
\pgfpathclose%
\pgfusepath{fill}%
\end{pgfscope}%
\begin{pgfscope}%
\pgfpathrectangle{\pgfqpoint{1.150000in}{0.150000in}}{\pgfqpoint{5.700000in}{5.700000in}}%
\pgfusepath{clip}%
\pgfsetbuttcap%
\pgfsetroundjoin%
\definecolor{currentfill}{rgb}{0.133743,0.548535,0.553541}%
\pgfsetfillcolor{currentfill}%
\pgfsetfillopacity{0.800000}%
\pgfsetlinewidth{0.000000pt}%
\definecolor{currentstroke}{rgb}{0.000000,0.000000,0.000000}%
\pgfsetstrokecolor{currentstroke}%
\pgfsetdash{}{0pt}%
\pgfpathmoveto{\pgfqpoint{5.535438in}{3.206521in}}%
\pgfpathlineto{\pgfqpoint{5.549842in}{3.215824in}}%
\pgfpathlineto{\pgfqpoint{5.564264in}{3.225302in}}%
\pgfpathlineto{\pgfqpoint{5.578705in}{3.234955in}}%
\pgfpathlineto{\pgfqpoint{5.593164in}{3.244784in}}%
\pgfpathlineto{\pgfqpoint{5.600429in}{3.248820in}}%
\pgfpathlineto{\pgfqpoint{5.607690in}{3.252936in}}%
\pgfpathlineto{\pgfqpoint{5.614947in}{3.257139in}}%
\pgfpathlineto{\pgfqpoint{5.622201in}{3.261435in}}%
\pgfpathlineto{\pgfqpoint{5.607770in}{3.252212in}}%
\pgfpathlineto{\pgfqpoint{5.593357in}{3.243164in}}%
\pgfpathlineto{\pgfqpoint{5.578962in}{3.234290in}}%
\pgfpathlineto{\pgfqpoint{5.564586in}{3.225590in}}%
\pgfpathlineto{\pgfqpoint{5.557304in}{3.220678in}}%
\pgfpathlineto{\pgfqpoint{5.550019in}{3.215867in}}%
\pgfpathlineto{\pgfqpoint{5.542730in}{3.211150in}}%
\pgfpathlineto{\pgfqpoint{5.535438in}{3.206521in}}%
\pgfpathclose%
\pgfusepath{fill}%
\end{pgfscope}%
\begin{pgfscope}%
\pgfpathrectangle{\pgfqpoint{1.150000in}{0.150000in}}{\pgfqpoint{5.700000in}{5.700000in}}%
\pgfusepath{clip}%
\pgfsetbuttcap%
\pgfsetroundjoin%
\definecolor{currentfill}{rgb}{0.255645,0.260703,0.528312}%
\pgfsetfillcolor{currentfill}%
\pgfsetfillopacity{0.800000}%
\pgfsetlinewidth{0.000000pt}%
\definecolor{currentstroke}{rgb}{0.000000,0.000000,0.000000}%
\pgfsetstrokecolor{currentstroke}%
\pgfsetdash{}{0pt}%
\pgfpathmoveto{\pgfqpoint{4.347387in}{2.383680in}}%
\pgfpathlineto{\pgfqpoint{4.361211in}{2.388650in}}%
\pgfpathlineto{\pgfqpoint{4.375048in}{2.393808in}}%
\pgfpathlineto{\pgfqpoint{4.388896in}{2.399153in}}%
\pgfpathlineto{\pgfqpoint{4.402757in}{2.404685in}}%
\pgfpathlineto{\pgfqpoint{4.410573in}{2.414276in}}%
\pgfpathlineto{\pgfqpoint{4.418384in}{2.423806in}}%
\pgfpathlineto{\pgfqpoint{4.426189in}{2.433279in}}%
\pgfpathlineto{\pgfqpoint{4.433988in}{2.442694in}}%
\pgfpathlineto{\pgfqpoint{4.420134in}{2.437269in}}%
\pgfpathlineto{\pgfqpoint{4.406293in}{2.432030in}}%
\pgfpathlineto{\pgfqpoint{4.392463in}{2.426978in}}%
\pgfpathlineto{\pgfqpoint{4.378646in}{2.422114in}}%
\pgfpathlineto{\pgfqpoint{4.370840in}{2.412580in}}%
\pgfpathlineto{\pgfqpoint{4.363028in}{2.402997in}}%
\pgfpathlineto{\pgfqpoint{4.355210in}{2.393365in}}%
\pgfpathlineto{\pgfqpoint{4.347387in}{2.383680in}}%
\pgfpathclose%
\pgfusepath{fill}%
\end{pgfscope}%
\begin{pgfscope}%
\pgfpathrectangle{\pgfqpoint{1.150000in}{0.150000in}}{\pgfqpoint{5.700000in}{5.700000in}}%
\pgfusepath{clip}%
\pgfsetbuttcap%
\pgfsetroundjoin%
\definecolor{currentfill}{rgb}{0.282290,0.145912,0.461510}%
\pgfsetfillcolor{currentfill}%
\pgfsetfillopacity{0.800000}%
\pgfsetlinewidth{0.000000pt}%
\definecolor{currentstroke}{rgb}{0.000000,0.000000,0.000000}%
\pgfsetstrokecolor{currentstroke}%
\pgfsetdash{}{0pt}%
\pgfpathmoveto{\pgfqpoint{2.805936in}{2.161972in}}%
\pgfpathlineto{\pgfqpoint{2.819621in}{2.147753in}}%
\pgfpathlineto{\pgfqpoint{2.833301in}{2.133802in}}%
\pgfpathlineto{\pgfqpoint{2.846976in}{2.120115in}}%
\pgfpathlineto{\pgfqpoint{2.860645in}{2.106691in}}%
\pgfpathlineto{\pgfqpoint{2.869071in}{2.112423in}}%
\pgfpathlineto{\pgfqpoint{2.877486in}{2.118302in}}%
\pgfpathlineto{\pgfqpoint{2.885892in}{2.124325in}}%
\pgfpathlineto{\pgfqpoint{2.894286in}{2.130489in}}%
\pgfpathlineto{\pgfqpoint{2.880645in}{2.143565in}}%
\pgfpathlineto{\pgfqpoint{2.866999in}{2.156903in}}%
\pgfpathlineto{\pgfqpoint{2.853348in}{2.170505in}}%
\pgfpathlineto{\pgfqpoint{2.839692in}{2.184373in}}%
\pgfpathlineto{\pgfqpoint{2.831269in}{2.178546in}}%
\pgfpathlineto{\pgfqpoint{2.822835in}{2.172868in}}%
\pgfpathlineto{\pgfqpoint{2.814391in}{2.167342in}}%
\pgfpathlineto{\pgfqpoint{2.805936in}{2.161972in}}%
\pgfpathclose%
\pgfusepath{fill}%
\end{pgfscope}%
\begin{pgfscope}%
\pgfpathrectangle{\pgfqpoint{1.150000in}{0.150000in}}{\pgfqpoint{5.700000in}{5.700000in}}%
\pgfusepath{clip}%
\pgfsetbuttcap%
\pgfsetroundjoin%
\definecolor{currentfill}{rgb}{0.258965,0.251537,0.524736}%
\pgfsetfillcolor{currentfill}%
\pgfsetfillopacity{0.800000}%
\pgfsetlinewidth{0.000000pt}%
\definecolor{currentstroke}{rgb}{0.000000,0.000000,0.000000}%
\pgfsetstrokecolor{currentstroke}%
\pgfsetdash{}{0pt}%
\pgfpathmoveto{\pgfqpoint{2.586005in}{2.427517in}}%
\pgfpathlineto{\pgfqpoint{2.599815in}{2.408727in}}%
\pgfpathlineto{\pgfqpoint{2.613615in}{2.390242in}}%
\pgfpathlineto{\pgfqpoint{2.627405in}{2.372058in}}%
\pgfpathlineto{\pgfqpoint{2.641186in}{2.354173in}}%
\pgfpathlineto{\pgfqpoint{2.649739in}{2.358446in}}%
\pgfpathlineto{\pgfqpoint{2.658279in}{2.362899in}}%
\pgfpathlineto{\pgfqpoint{2.666806in}{2.367530in}}%
\pgfpathlineto{\pgfqpoint{2.675322in}{2.372334in}}%
\pgfpathlineto{\pgfqpoint{2.661575in}{2.389861in}}%
\pgfpathlineto{\pgfqpoint{2.647819in}{2.407687in}}%
\pgfpathlineto{\pgfqpoint{2.634054in}{2.425814in}}%
\pgfpathlineto{\pgfqpoint{2.620279in}{2.444244in}}%
\pgfpathlineto{\pgfqpoint{2.611730in}{2.439786in}}%
\pgfpathlineto{\pgfqpoint{2.603167in}{2.435509in}}%
\pgfpathlineto{\pgfqpoint{2.594593in}{2.431419in}}%
\pgfpathlineto{\pgfqpoint{2.586005in}{2.427517in}}%
\pgfpathclose%
\pgfusepath{fill}%
\end{pgfscope}%
\begin{pgfscope}%
\pgfpathrectangle{\pgfqpoint{1.150000in}{0.150000in}}{\pgfqpoint{5.700000in}{5.700000in}}%
\pgfusepath{clip}%
\pgfsetbuttcap%
\pgfsetroundjoin%
\definecolor{currentfill}{rgb}{0.127568,0.566949,0.550556}%
\pgfsetfillcolor{currentfill}%
\pgfsetfillopacity{0.800000}%
\pgfsetlinewidth{0.000000pt}%
\definecolor{currentstroke}{rgb}{0.000000,0.000000,0.000000}%
\pgfsetstrokecolor{currentstroke}%
\pgfsetdash{}{0pt}%
\pgfpathmoveto{\pgfqpoint{5.622201in}{3.261435in}}%
\pgfpathlineto{\pgfqpoint{5.636650in}{3.270832in}}%
\pgfpathlineto{\pgfqpoint{5.651118in}{3.280404in}}%
\pgfpathlineto{\pgfqpoint{5.665604in}{3.290151in}}%
\pgfpathlineto{\pgfqpoint{5.680109in}{3.300072in}}%
\pgfpathlineto{\pgfqpoint{5.687330in}{3.303841in}}%
\pgfpathlineto{\pgfqpoint{5.694547in}{3.307709in}}%
\pgfpathlineto{\pgfqpoint{5.701762in}{3.311685in}}%
\pgfpathlineto{\pgfqpoint{5.708973in}{3.315773in}}%
\pgfpathlineto{\pgfqpoint{5.694499in}{3.306491in}}%
\pgfpathlineto{\pgfqpoint{5.680043in}{3.297383in}}%
\pgfpathlineto{\pgfqpoint{5.665605in}{3.288449in}}%
\pgfpathlineto{\pgfqpoint{5.651185in}{3.279688in}}%
\pgfpathlineto{\pgfqpoint{5.643943in}{3.274951in}}%
\pgfpathlineto{\pgfqpoint{5.636699in}{3.270334in}}%
\pgfpathlineto{\pgfqpoint{5.629451in}{3.265831in}}%
\pgfpathlineto{\pgfqpoint{5.622201in}{3.261435in}}%
\pgfpathclose%
\pgfusepath{fill}%
\end{pgfscope}%
\begin{pgfscope}%
\pgfpathrectangle{\pgfqpoint{1.150000in}{0.150000in}}{\pgfqpoint{5.700000in}{5.700000in}}%
\pgfusepath{clip}%
\pgfsetbuttcap%
\pgfsetroundjoin%
\definecolor{currentfill}{rgb}{0.190631,0.407061,0.556089}%
\pgfsetfillcolor{currentfill}%
\pgfsetfillopacity{0.800000}%
\pgfsetlinewidth{0.000000pt}%
\definecolor{currentstroke}{rgb}{0.000000,0.000000,0.000000}%
\pgfsetstrokecolor{currentstroke}%
\pgfsetdash{}{0pt}%
\pgfpathmoveto{\pgfqpoint{4.898247in}{2.778187in}}%
\pgfpathlineto{\pgfqpoint{4.912336in}{2.786143in}}%
\pgfpathlineto{\pgfqpoint{4.926439in}{2.794279in}}%
\pgfpathlineto{\pgfqpoint{4.940559in}{2.802596in}}%
\pgfpathlineto{\pgfqpoint{4.954694in}{2.811094in}}%
\pgfpathlineto{\pgfqpoint{4.962282in}{2.817930in}}%
\pgfpathlineto{\pgfqpoint{4.969863in}{2.824733in}}%
\pgfpathlineto{\pgfqpoint{4.977439in}{2.831507in}}%
\pgfpathlineto{\pgfqpoint{4.985008in}{2.838256in}}%
\pgfpathlineto{\pgfqpoint{4.970886in}{2.830097in}}%
\pgfpathlineto{\pgfqpoint{4.956781in}{2.822118in}}%
\pgfpathlineto{\pgfqpoint{4.942691in}{2.814319in}}%
\pgfpathlineto{\pgfqpoint{4.928616in}{2.806700in}}%
\pgfpathlineto{\pgfqpoint{4.921033in}{2.799602in}}%
\pgfpathlineto{\pgfqpoint{4.913444in}{2.792486in}}%
\pgfpathlineto{\pgfqpoint{4.905849in}{2.785349in}}%
\pgfpathlineto{\pgfqpoint{4.898247in}{2.778187in}}%
\pgfpathclose%
\pgfusepath{fill}%
\end{pgfscope}%
\begin{pgfscope}%
\pgfpathrectangle{\pgfqpoint{1.150000in}{0.150000in}}{\pgfqpoint{5.700000in}{5.700000in}}%
\pgfusepath{clip}%
\pgfsetbuttcap%
\pgfsetroundjoin%
\definecolor{currentfill}{rgb}{0.274952,0.037752,0.364543}%
\pgfsetfillcolor{currentfill}%
\pgfsetfillopacity{0.800000}%
\pgfsetlinewidth{0.000000pt}%
\definecolor{currentstroke}{rgb}{0.000000,0.000000,0.000000}%
\pgfsetstrokecolor{currentstroke}%
\pgfsetdash{}{0pt}%
\pgfpathmoveto{\pgfqpoint{3.253825in}{1.924381in}}%
\pgfpathlineto{\pgfqpoint{3.267404in}{1.917634in}}%
\pgfpathlineto{\pgfqpoint{3.280986in}{1.911111in}}%
\pgfpathlineto{\pgfqpoint{3.294568in}{1.904809in}}%
\pgfpathlineto{\pgfqpoint{3.308153in}{1.898727in}}%
\pgfpathlineto{\pgfqpoint{3.316351in}{1.907592in}}%
\pgfpathlineto{\pgfqpoint{3.324543in}{1.916525in}}%
\pgfpathlineto{\pgfqpoint{3.332728in}{1.925523in}}%
\pgfpathlineto{\pgfqpoint{3.340906in}{1.934585in}}%
\pgfpathlineto{\pgfqpoint{3.327338in}{1.940390in}}%
\pgfpathlineto{\pgfqpoint{3.313773in}{1.946414in}}%
\pgfpathlineto{\pgfqpoint{3.300209in}{1.952661in}}%
\pgfpathlineto{\pgfqpoint{3.286647in}{1.959129in}}%
\pgfpathlineto{\pgfqpoint{3.278452in}{1.950333in}}%
\pgfpathlineto{\pgfqpoint{3.270250in}{1.941608in}}%
\pgfpathlineto{\pgfqpoint{3.262041in}{1.932957in}}%
\pgfpathlineto{\pgfqpoint{3.253825in}{1.924381in}}%
\pgfpathclose%
\pgfusepath{fill}%
\end{pgfscope}%
\begin{pgfscope}%
\pgfpathrectangle{\pgfqpoint{1.150000in}{0.150000in}}{\pgfqpoint{5.700000in}{5.700000in}}%
\pgfusepath{clip}%
\pgfsetbuttcap%
\pgfsetroundjoin%
\definecolor{currentfill}{rgb}{0.283229,0.120777,0.440584}%
\pgfsetfillcolor{currentfill}%
\pgfsetfillopacity{0.800000}%
\pgfsetlinewidth{0.000000pt}%
\definecolor{currentstroke}{rgb}{0.000000,0.000000,0.000000}%
\pgfsetstrokecolor{currentstroke}%
\pgfsetdash{}{0pt}%
\pgfpathmoveto{\pgfqpoint{2.860645in}{2.106691in}}%
\pgfpathlineto{\pgfqpoint{2.874310in}{2.093527in}}%
\pgfpathlineto{\pgfqpoint{2.887971in}{2.080623in}}%
\pgfpathlineto{\pgfqpoint{2.901627in}{2.067976in}}%
\pgfpathlineto{\pgfqpoint{2.915279in}{2.055584in}}%
\pgfpathlineto{\pgfqpoint{2.923677in}{2.061676in}}%
\pgfpathlineto{\pgfqpoint{2.932064in}{2.067906in}}%
\pgfpathlineto{\pgfqpoint{2.940442in}{2.074273in}}%
\pgfpathlineto{\pgfqpoint{2.948810in}{2.080772in}}%
\pgfpathlineto{\pgfqpoint{2.935185in}{2.092817in}}%
\pgfpathlineto{\pgfqpoint{2.921556in}{2.105117in}}%
\pgfpathlineto{\pgfqpoint{2.907923in}{2.117674in}}%
\pgfpathlineto{\pgfqpoint{2.894286in}{2.130489in}}%
\pgfpathlineto{\pgfqpoint{2.885892in}{2.124325in}}%
\pgfpathlineto{\pgfqpoint{2.877486in}{2.118302in}}%
\pgfpathlineto{\pgfqpoint{2.869071in}{2.112423in}}%
\pgfpathlineto{\pgfqpoint{2.860645in}{2.106691in}}%
\pgfpathclose%
\pgfusepath{fill}%
\end{pgfscope}%
\begin{pgfscope}%
\pgfpathrectangle{\pgfqpoint{1.150000in}{0.150000in}}{\pgfqpoint{5.700000in}{5.700000in}}%
\pgfusepath{clip}%
\pgfsetbuttcap%
\pgfsetroundjoin%
\definecolor{currentfill}{rgb}{0.246811,0.283237,0.535941}%
\pgfsetfillcolor{currentfill}%
\pgfsetfillopacity{0.800000}%
\pgfsetlinewidth{0.000000pt}%
\definecolor{currentstroke}{rgb}{0.000000,0.000000,0.000000}%
\pgfsetstrokecolor{currentstroke}%
\pgfsetdash{}{0pt}%
\pgfpathmoveto{\pgfqpoint{4.433988in}{2.442694in}}%
\pgfpathlineto{\pgfqpoint{4.447854in}{2.448307in}}%
\pgfpathlineto{\pgfqpoint{4.461733in}{2.454105in}}%
\pgfpathlineto{\pgfqpoint{4.475624in}{2.460090in}}%
\pgfpathlineto{\pgfqpoint{4.489529in}{2.466260in}}%
\pgfpathlineto{\pgfqpoint{4.497315in}{2.475494in}}%
\pgfpathlineto{\pgfqpoint{4.505096in}{2.484666in}}%
\pgfpathlineto{\pgfqpoint{4.512871in}{2.493779in}}%
\pgfpathlineto{\pgfqpoint{4.520639in}{2.502835in}}%
\pgfpathlineto{\pgfqpoint{4.506742in}{2.496804in}}%
\pgfpathlineto{\pgfqpoint{4.492858in}{2.490958in}}%
\pgfpathlineto{\pgfqpoint{4.478987in}{2.485299in}}%
\pgfpathlineto{\pgfqpoint{4.465128in}{2.479825in}}%
\pgfpathlineto{\pgfqpoint{4.457352in}{2.470619in}}%
\pgfpathlineto{\pgfqpoint{4.449569in}{2.461363in}}%
\pgfpathlineto{\pgfqpoint{4.441781in}{2.452055in}}%
\pgfpathlineto{\pgfqpoint{4.433988in}{2.442694in}}%
\pgfpathclose%
\pgfusepath{fill}%
\end{pgfscope}%
\begin{pgfscope}%
\pgfpathrectangle{\pgfqpoint{1.150000in}{0.150000in}}{\pgfqpoint{5.700000in}{5.700000in}}%
\pgfusepath{clip}%
\pgfsetbuttcap%
\pgfsetroundjoin%
\definecolor{currentfill}{rgb}{0.246811,0.283237,0.535941}%
\pgfsetfillcolor{currentfill}%
\pgfsetfillopacity{0.800000}%
\pgfsetlinewidth{0.000000pt}%
\definecolor{currentstroke}{rgb}{0.000000,0.000000,0.000000}%
\pgfsetstrokecolor{currentstroke}%
\pgfsetdash{}{0pt}%
\pgfpathmoveto{\pgfqpoint{2.530659in}{2.505772in}}%
\pgfpathlineto{\pgfqpoint{2.544512in}{2.485738in}}%
\pgfpathlineto{\pgfqpoint{2.558354in}{2.466019in}}%
\pgfpathlineto{\pgfqpoint{2.572184in}{2.446613in}}%
\pgfpathlineto{\pgfqpoint{2.586005in}{2.427517in}}%
\pgfpathlineto{\pgfqpoint{2.594593in}{2.431419in}}%
\pgfpathlineto{\pgfqpoint{2.603167in}{2.435509in}}%
\pgfpathlineto{\pgfqpoint{2.611730in}{2.439786in}}%
\pgfpathlineto{\pgfqpoint{2.620279in}{2.444244in}}%
\pgfpathlineto{\pgfqpoint{2.606494in}{2.462981in}}%
\pgfpathlineto{\pgfqpoint{2.592700in}{2.482026in}}%
\pgfpathlineto{\pgfqpoint{2.578894in}{2.501383in}}%
\pgfpathlineto{\pgfqpoint{2.565078in}{2.521055in}}%
\pgfpathlineto{\pgfqpoint{2.556493in}{2.516945in}}%
\pgfpathlineto{\pgfqpoint{2.547896in}{2.513025in}}%
\pgfpathlineto{\pgfqpoint{2.539284in}{2.509300in}}%
\pgfpathlineto{\pgfqpoint{2.530659in}{2.505772in}}%
\pgfpathclose%
\pgfusepath{fill}%
\end{pgfscope}%
\begin{pgfscope}%
\pgfpathrectangle{\pgfqpoint{1.150000in}{0.150000in}}{\pgfqpoint{5.700000in}{5.700000in}}%
\pgfusepath{clip}%
\pgfsetbuttcap%
\pgfsetroundjoin%
\definecolor{currentfill}{rgb}{0.122606,0.585371,0.546557}%
\pgfsetfillcolor{currentfill}%
\pgfsetfillopacity{0.800000}%
\pgfsetlinewidth{0.000000pt}%
\definecolor{currentstroke}{rgb}{0.000000,0.000000,0.000000}%
\pgfsetstrokecolor{currentstroke}%
\pgfsetdash{}{0pt}%
\pgfpathmoveto{\pgfqpoint{5.708973in}{3.315773in}}%
\pgfpathlineto{\pgfqpoint{5.723467in}{3.325228in}}%
\pgfpathlineto{\pgfqpoint{5.737979in}{3.334858in}}%
\pgfpathlineto{\pgfqpoint{5.752510in}{3.344661in}}%
\pgfpathlineto{\pgfqpoint{5.767060in}{3.354638in}}%
\pgfpathlineto{\pgfqpoint{5.774237in}{3.358187in}}%
\pgfpathlineto{\pgfqpoint{5.781412in}{3.361856in}}%
\pgfpathlineto{\pgfqpoint{5.788584in}{3.365652in}}%
\pgfpathlineto{\pgfqpoint{5.795755in}{3.369584in}}%
\pgfpathlineto{\pgfqpoint{5.781237in}{3.360279in}}%
\pgfpathlineto{\pgfqpoint{5.766739in}{3.351148in}}%
\pgfpathlineto{\pgfqpoint{5.752259in}{3.342189in}}%
\pgfpathlineto{\pgfqpoint{5.737798in}{3.333403in}}%
\pgfpathlineto{\pgfqpoint{5.730595in}{3.328790in}}%
\pgfpathlineto{\pgfqpoint{5.723390in}{3.324318in}}%
\pgfpathlineto{\pgfqpoint{5.716183in}{3.319982in}}%
\pgfpathlineto{\pgfqpoint{5.708973in}{3.315773in}}%
\pgfpathclose%
\pgfusepath{fill}%
\end{pgfscope}%
\begin{pgfscope}%
\pgfpathrectangle{\pgfqpoint{1.150000in}{0.150000in}}{\pgfqpoint{5.700000in}{5.700000in}}%
\pgfusepath{clip}%
\pgfsetbuttcap%
\pgfsetroundjoin%
\definecolor{currentfill}{rgb}{0.277941,0.056324,0.381191}%
\pgfsetfillcolor{currentfill}%
\pgfsetfillopacity{0.800000}%
\pgfsetlinewidth{0.000000pt}%
\definecolor{currentstroke}{rgb}{0.000000,0.000000,0.000000}%
\pgfsetstrokecolor{currentstroke}%
\pgfsetdash{}{0pt}%
\pgfpathmoveto{\pgfqpoint{3.112109in}{1.955496in}}%
\pgfpathlineto{\pgfqpoint{3.125706in}{1.946614in}}%
\pgfpathlineto{\pgfqpoint{3.139304in}{1.937965in}}%
\pgfpathlineto{\pgfqpoint{3.152900in}{1.929548in}}%
\pgfpathlineto{\pgfqpoint{3.166497in}{1.921361in}}%
\pgfpathlineto{\pgfqpoint{3.174763in}{1.929285in}}%
\pgfpathlineto{\pgfqpoint{3.183021in}{1.937304in}}%
\pgfpathlineto{\pgfqpoint{3.191271in}{1.945415in}}%
\pgfpathlineto{\pgfqpoint{3.199514in}{1.953615in}}%
\pgfpathlineto{\pgfqpoint{3.185938in}{1.961492in}}%
\pgfpathlineto{\pgfqpoint{3.172362in}{1.969599in}}%
\pgfpathlineto{\pgfqpoint{3.158785in}{1.977938in}}%
\pgfpathlineto{\pgfqpoint{3.145209in}{1.986509in}}%
\pgfpathlineto{\pgfqpoint{3.136946in}{1.978607in}}%
\pgfpathlineto{\pgfqpoint{3.128675in}{1.970803in}}%
\pgfpathlineto{\pgfqpoint{3.120396in}{1.963098in}}%
\pgfpathlineto{\pgfqpoint{3.112109in}{1.955496in}}%
\pgfpathclose%
\pgfusepath{fill}%
\end{pgfscope}%
\begin{pgfscope}%
\pgfpathrectangle{\pgfqpoint{1.150000in}{0.150000in}}{\pgfqpoint{5.700000in}{5.700000in}}%
\pgfusepath{clip}%
\pgfsetbuttcap%
\pgfsetroundjoin%
\definecolor{currentfill}{rgb}{0.180629,0.429975,0.557282}%
\pgfsetfillcolor{currentfill}%
\pgfsetfillopacity{0.800000}%
\pgfsetlinewidth{0.000000pt}%
\definecolor{currentstroke}{rgb}{0.000000,0.000000,0.000000}%
\pgfsetstrokecolor{currentstroke}%
\pgfsetdash{}{0pt}%
\pgfpathmoveto{\pgfqpoint{4.985008in}{2.838256in}}%
\pgfpathlineto{\pgfqpoint{4.999145in}{2.846596in}}%
\pgfpathlineto{\pgfqpoint{5.013297in}{2.855117in}}%
\pgfpathlineto{\pgfqpoint{5.027466in}{2.863817in}}%
\pgfpathlineto{\pgfqpoint{5.041652in}{2.872698in}}%
\pgfpathlineto{\pgfqpoint{5.049200in}{2.879066in}}%
\pgfpathlineto{\pgfqpoint{5.056742in}{2.885409in}}%
\pgfpathlineto{\pgfqpoint{5.064277in}{2.891732in}}%
\pgfpathlineto{\pgfqpoint{5.071807in}{2.898039in}}%
\pgfpathlineto{\pgfqpoint{5.057637in}{2.889531in}}%
\pgfpathlineto{\pgfqpoint{5.043483in}{2.881202in}}%
\pgfpathlineto{\pgfqpoint{5.029345in}{2.873053in}}%
\pgfpathlineto{\pgfqpoint{5.015223in}{2.865084in}}%
\pgfpathlineto{\pgfqpoint{5.007678in}{2.858394in}}%
\pgfpathlineto{\pgfqpoint{5.000127in}{2.851695in}}%
\pgfpathlineto{\pgfqpoint{4.992571in}{2.844984in}}%
\pgfpathlineto{\pgfqpoint{4.985008in}{2.838256in}}%
\pgfpathclose%
\pgfusepath{fill}%
\end{pgfscope}%
\begin{pgfscope}%
\pgfpathrectangle{\pgfqpoint{1.150000in}{0.150000in}}{\pgfqpoint{5.700000in}{5.700000in}}%
\pgfusepath{clip}%
\pgfsetbuttcap%
\pgfsetroundjoin%
\definecolor{currentfill}{rgb}{0.274952,0.037752,0.364543}%
\pgfsetfillcolor{currentfill}%
\pgfsetfillopacity{0.800000}%
\pgfsetlinewidth{0.000000pt}%
\definecolor{currentstroke}{rgb}{0.000000,0.000000,0.000000}%
\pgfsetstrokecolor{currentstroke}%
\pgfsetdash{}{0pt}%
\pgfpathmoveto{\pgfqpoint{3.395199in}{1.913546in}}%
\pgfpathlineto{\pgfqpoint{3.408779in}{1.908827in}}%
\pgfpathlineto{\pgfqpoint{3.422363in}{1.904321in}}%
\pgfpathlineto{\pgfqpoint{3.435949in}{1.900028in}}%
\pgfpathlineto{\pgfqpoint{3.449539in}{1.895947in}}%
\pgfpathlineto{\pgfqpoint{3.457680in}{1.905583in}}%
\pgfpathlineto{\pgfqpoint{3.465816in}{1.915262in}}%
\pgfpathlineto{\pgfqpoint{3.473945in}{1.924982in}}%
\pgfpathlineto{\pgfqpoint{3.482068in}{1.934740in}}%
\pgfpathlineto{\pgfqpoint{3.468492in}{1.938576in}}%
\pgfpathlineto{\pgfqpoint{3.454919in}{1.942624in}}%
\pgfpathlineto{\pgfqpoint{3.441350in}{1.946884in}}%
\pgfpathlineto{\pgfqpoint{3.427785in}{1.951359in}}%
\pgfpathlineto{\pgfqpoint{3.419648in}{1.941834in}}%
\pgfpathlineto{\pgfqpoint{3.411505in}{1.932356in}}%
\pgfpathlineto{\pgfqpoint{3.403355in}{1.922926in}}%
\pgfpathlineto{\pgfqpoint{3.395199in}{1.913546in}}%
\pgfpathclose%
\pgfusepath{fill}%
\end{pgfscope}%
\begin{pgfscope}%
\pgfpathrectangle{\pgfqpoint{1.150000in}{0.150000in}}{\pgfqpoint{5.700000in}{5.700000in}}%
\pgfusepath{clip}%
\pgfsetbuttcap%
\pgfsetroundjoin%
\definecolor{currentfill}{rgb}{0.119738,0.603785,0.541400}%
\pgfsetfillcolor{currentfill}%
\pgfsetfillopacity{0.800000}%
\pgfsetlinewidth{0.000000pt}%
\definecolor{currentstroke}{rgb}{0.000000,0.000000,0.000000}%
\pgfsetstrokecolor{currentstroke}%
\pgfsetdash{}{0pt}%
\pgfpathmoveto{\pgfqpoint{5.795755in}{3.369584in}}%
\pgfpathlineto{\pgfqpoint{5.810291in}{3.379061in}}%
\pgfpathlineto{\pgfqpoint{5.824846in}{3.388712in}}%
\pgfpathlineto{\pgfqpoint{5.839420in}{3.398536in}}%
\pgfpathlineto{\pgfqpoint{5.854014in}{3.408534in}}%
\pgfpathlineto{\pgfqpoint{5.861149in}{3.411914in}}%
\pgfpathlineto{\pgfqpoint{5.868281in}{3.415437in}}%
\pgfpathlineto{\pgfqpoint{5.875413in}{3.419110in}}%
\pgfpathlineto{\pgfqpoint{5.882544in}{3.422942in}}%
\pgfpathlineto{\pgfqpoint{5.867986in}{3.413650in}}%
\pgfpathlineto{\pgfqpoint{5.853446in}{3.404531in}}%
\pgfpathlineto{\pgfqpoint{5.838926in}{3.395584in}}%
\pgfpathlineto{\pgfqpoint{5.824424in}{3.386809in}}%
\pgfpathlineto{\pgfqpoint{5.817258in}{3.382263in}}%
\pgfpathlineto{\pgfqpoint{5.810091in}{3.377882in}}%
\pgfpathlineto{\pgfqpoint{5.802924in}{3.373658in}}%
\pgfpathlineto{\pgfqpoint{5.795755in}{3.369584in}}%
\pgfpathclose%
\pgfusepath{fill}%
\end{pgfscope}%
\begin{pgfscope}%
\pgfpathrectangle{\pgfqpoint{1.150000in}{0.150000in}}{\pgfqpoint{5.700000in}{5.700000in}}%
\pgfusepath{clip}%
\pgfsetbuttcap%
\pgfsetroundjoin%
\definecolor{currentfill}{rgb}{0.282910,0.105393,0.426902}%
\pgfsetfillcolor{currentfill}%
\pgfsetfillopacity{0.800000}%
\pgfsetlinewidth{0.000000pt}%
\definecolor{currentstroke}{rgb}{0.000000,0.000000,0.000000}%
\pgfsetstrokecolor{currentstroke}%
\pgfsetdash{}{0pt}%
\pgfpathmoveto{\pgfqpoint{3.796447in}{2.029700in}}%
\pgfpathlineto{\pgfqpoint{3.810087in}{2.029888in}}%
\pgfpathlineto{\pgfqpoint{3.823733in}{2.030274in}}%
\pgfpathlineto{\pgfqpoint{3.837387in}{2.030858in}}%
\pgfpathlineto{\pgfqpoint{3.851049in}{2.031639in}}%
\pgfpathlineto{\pgfqpoint{3.859051in}{2.042402in}}%
\pgfpathlineto{\pgfqpoint{3.867047in}{2.053145in}}%
\pgfpathlineto{\pgfqpoint{3.875038in}{2.063865in}}%
\pgfpathlineto{\pgfqpoint{3.883024in}{2.074564in}}%
\pgfpathlineto{\pgfqpoint{3.869370in}{2.073664in}}%
\pgfpathlineto{\pgfqpoint{3.855724in}{2.072962in}}%
\pgfpathlineto{\pgfqpoint{3.842085in}{2.072457in}}%
\pgfpathlineto{\pgfqpoint{3.828454in}{2.072151in}}%
\pgfpathlineto{\pgfqpoint{3.820460in}{2.061559in}}%
\pgfpathlineto{\pgfqpoint{3.812461in}{2.050953in}}%
\pgfpathlineto{\pgfqpoint{3.804457in}{2.040333in}}%
\pgfpathlineto{\pgfqpoint{3.796447in}{2.029700in}}%
\pgfpathclose%
\pgfusepath{fill}%
\end{pgfscope}%
\begin{pgfscope}%
\pgfpathrectangle{\pgfqpoint{1.150000in}{0.150000in}}{\pgfqpoint{5.700000in}{5.700000in}}%
\pgfusepath{clip}%
\pgfsetbuttcap%
\pgfsetroundjoin%
\definecolor{currentfill}{rgb}{0.283187,0.125848,0.444960}%
\pgfsetfillcolor{currentfill}%
\pgfsetfillopacity{0.800000}%
\pgfsetlinewidth{0.000000pt}%
\definecolor{currentstroke}{rgb}{0.000000,0.000000,0.000000}%
\pgfsetstrokecolor{currentstroke}%
\pgfsetdash{}{0pt}%
\pgfpathmoveto{\pgfqpoint{3.883024in}{2.074564in}}%
\pgfpathlineto{\pgfqpoint{3.896687in}{2.075660in}}%
\pgfpathlineto{\pgfqpoint{3.910357in}{2.076952in}}%
\pgfpathlineto{\pgfqpoint{3.924036in}{2.078440in}}%
\pgfpathlineto{\pgfqpoint{3.937723in}{2.080123in}}%
\pgfpathlineto{\pgfqpoint{3.945697in}{2.090897in}}%
\pgfpathlineto{\pgfqpoint{3.953666in}{2.101639in}}%
\pgfpathlineto{\pgfqpoint{3.961630in}{2.112350in}}%
\pgfpathlineto{\pgfqpoint{3.969589in}{2.123029in}}%
\pgfpathlineto{\pgfqpoint{3.955908in}{2.121259in}}%
\pgfpathlineto{\pgfqpoint{3.942237in}{2.119685in}}%
\pgfpathlineto{\pgfqpoint{3.928573in}{2.118306in}}%
\pgfpathlineto{\pgfqpoint{3.914919in}{2.117123in}}%
\pgfpathlineto{\pgfqpoint{3.906953in}{2.106519in}}%
\pgfpathlineto{\pgfqpoint{3.898981in}{2.095891in}}%
\pgfpathlineto{\pgfqpoint{3.891005in}{2.085239in}}%
\pgfpathlineto{\pgfqpoint{3.883024in}{2.074564in}}%
\pgfpathclose%
\pgfusepath{fill}%
\end{pgfscope}%
\begin{pgfscope}%
\pgfpathrectangle{\pgfqpoint{1.150000in}{0.150000in}}{\pgfqpoint{5.700000in}{5.700000in}}%
\pgfusepath{clip}%
\pgfsetbuttcap%
\pgfsetroundjoin%
\definecolor{currentfill}{rgb}{0.235526,0.309527,0.542944}%
\pgfsetfillcolor{currentfill}%
\pgfsetfillopacity{0.800000}%
\pgfsetlinewidth{0.000000pt}%
\definecolor{currentstroke}{rgb}{0.000000,0.000000,0.000000}%
\pgfsetstrokecolor{currentstroke}%
\pgfsetdash{}{0pt}%
\pgfpathmoveto{\pgfqpoint{4.520639in}{2.502835in}}%
\pgfpathlineto{\pgfqpoint{4.534550in}{2.509051in}}%
\pgfpathlineto{\pgfqpoint{4.548473in}{2.515453in}}%
\pgfpathlineto{\pgfqpoint{4.562410in}{2.522040in}}%
\pgfpathlineto{\pgfqpoint{4.576360in}{2.528811in}}%
\pgfpathlineto{\pgfqpoint{4.584115in}{2.537651in}}%
\pgfpathlineto{\pgfqpoint{4.591864in}{2.546430in}}%
\pgfpathlineto{\pgfqpoint{4.599607in}{2.555150in}}%
\pgfpathlineto{\pgfqpoint{4.607344in}{2.563813in}}%
\pgfpathlineto{\pgfqpoint{4.593402in}{2.557214in}}%
\pgfpathlineto{\pgfqpoint{4.579474in}{2.550800in}}%
\pgfpathlineto{\pgfqpoint{4.565559in}{2.544570in}}%
\pgfpathlineto{\pgfqpoint{4.551657in}{2.538525in}}%
\pgfpathlineto{\pgfqpoint{4.543911in}{2.529678in}}%
\pgfpathlineto{\pgfqpoint{4.536160in}{2.520782in}}%
\pgfpathlineto{\pgfqpoint{4.528402in}{2.511835in}}%
\pgfpathlineto{\pgfqpoint{4.520639in}{2.502835in}}%
\pgfpathclose%
\pgfusepath{fill}%
\end{pgfscope}%
\begin{pgfscope}%
\pgfpathrectangle{\pgfqpoint{1.150000in}{0.150000in}}{\pgfqpoint{5.700000in}{5.700000in}}%
\pgfusepath{clip}%
\pgfsetbuttcap%
\pgfsetroundjoin%
\definecolor{currentfill}{rgb}{0.282656,0.100196,0.422160}%
\pgfsetfillcolor{currentfill}%
\pgfsetfillopacity{0.800000}%
\pgfsetlinewidth{0.000000pt}%
\definecolor{currentstroke}{rgb}{0.000000,0.000000,0.000000}%
\pgfsetstrokecolor{currentstroke}%
\pgfsetdash{}{0pt}%
\pgfpathmoveto{\pgfqpoint{2.915279in}{2.055584in}}%
\pgfpathlineto{\pgfqpoint{2.928927in}{2.043445in}}%
\pgfpathlineto{\pgfqpoint{2.942572in}{2.031558in}}%
\pgfpathlineto{\pgfqpoint{2.956214in}{2.019922in}}%
\pgfpathlineto{\pgfqpoint{2.969852in}{2.008533in}}%
\pgfpathlineto{\pgfqpoint{2.978223in}{2.014983in}}%
\pgfpathlineto{\pgfqpoint{2.986584in}{2.021564in}}%
\pgfpathlineto{\pgfqpoint{2.994936in}{2.028272in}}%
\pgfpathlineto{\pgfqpoint{3.003279in}{2.035105in}}%
\pgfpathlineto{\pgfqpoint{2.989666in}{2.046148in}}%
\pgfpathlineto{\pgfqpoint{2.976051in}{2.057439in}}%
\pgfpathlineto{\pgfqpoint{2.962432in}{2.068980in}}%
\pgfpathlineto{\pgfqpoint{2.948810in}{2.080772in}}%
\pgfpathlineto{\pgfqpoint{2.940442in}{2.074273in}}%
\pgfpathlineto{\pgfqpoint{2.932064in}{2.067906in}}%
\pgfpathlineto{\pgfqpoint{2.923677in}{2.061676in}}%
\pgfpathlineto{\pgfqpoint{2.915279in}{2.055584in}}%
\pgfpathclose%
\pgfusepath{fill}%
\end{pgfscope}%
\begin{pgfscope}%
\pgfpathrectangle{\pgfqpoint{1.150000in}{0.150000in}}{\pgfqpoint{5.700000in}{5.700000in}}%
\pgfusepath{clip}%
\pgfsetbuttcap%
\pgfsetroundjoin%
\definecolor{currentfill}{rgb}{0.120081,0.622161,0.534946}%
\pgfsetfillcolor{currentfill}%
\pgfsetfillopacity{0.800000}%
\pgfsetlinewidth{0.000000pt}%
\definecolor{currentstroke}{rgb}{0.000000,0.000000,0.000000}%
\pgfsetstrokecolor{currentstroke}%
\pgfsetdash{}{0pt}%
\pgfpathmoveto{\pgfqpoint{5.882544in}{3.422942in}}%
\pgfpathlineto{\pgfqpoint{5.897122in}{3.432405in}}%
\pgfpathlineto{\pgfqpoint{5.911719in}{3.442042in}}%
\pgfpathlineto{\pgfqpoint{5.926335in}{3.451851in}}%
\pgfpathlineto{\pgfqpoint{5.940971in}{3.461832in}}%
\pgfpathlineto{\pgfqpoint{5.948064in}{3.465103in}}%
\pgfpathlineto{\pgfqpoint{5.955157in}{3.468539in}}%
\pgfpathlineto{\pgfqpoint{5.962250in}{3.472151in}}%
\pgfpathlineto{\pgfqpoint{5.969343in}{3.475944in}}%
\pgfpathlineto{\pgfqpoint{5.954745in}{3.466702in}}%
\pgfpathlineto{\pgfqpoint{5.940166in}{3.457631in}}%
\pgfpathlineto{\pgfqpoint{5.925606in}{3.448731in}}%
\pgfpathlineto{\pgfqpoint{5.911065in}{3.440003in}}%
\pgfpathlineto{\pgfqpoint{5.903934in}{3.435461in}}%
\pgfpathlineto{\pgfqpoint{5.896804in}{3.431109in}}%
\pgfpathlineto{\pgfqpoint{5.889674in}{3.426939in}}%
\pgfpathlineto{\pgfqpoint{5.882544in}{3.422942in}}%
\pgfpathclose%
\pgfusepath{fill}%
\end{pgfscope}%
\begin{pgfscope}%
\pgfpathrectangle{\pgfqpoint{1.150000in}{0.150000in}}{\pgfqpoint{5.700000in}{5.700000in}}%
\pgfusepath{clip}%
\pgfsetbuttcap%
\pgfsetroundjoin%
\definecolor{currentfill}{rgb}{0.231674,0.318106,0.544834}%
\pgfsetfillcolor{currentfill}%
\pgfsetfillopacity{0.800000}%
\pgfsetlinewidth{0.000000pt}%
\definecolor{currentstroke}{rgb}{0.000000,0.000000,0.000000}%
\pgfsetstrokecolor{currentstroke}%
\pgfsetdash{}{0pt}%
\pgfpathmoveto{\pgfqpoint{2.475130in}{2.589122in}}%
\pgfpathlineto{\pgfqpoint{2.489030in}{2.567796in}}%
\pgfpathlineto{\pgfqpoint{2.502919in}{2.546798in}}%
\pgfpathlineto{\pgfqpoint{2.516795in}{2.526124in}}%
\pgfpathlineto{\pgfqpoint{2.530659in}{2.505772in}}%
\pgfpathlineto{\pgfqpoint{2.539284in}{2.509300in}}%
\pgfpathlineto{\pgfqpoint{2.547896in}{2.513025in}}%
\pgfpathlineto{\pgfqpoint{2.556493in}{2.516945in}}%
\pgfpathlineto{\pgfqpoint{2.565078in}{2.521055in}}%
\pgfpathlineto{\pgfqpoint{2.551251in}{2.541044in}}%
\pgfpathlineto{\pgfqpoint{2.537412in}{2.561354in}}%
\pgfpathlineto{\pgfqpoint{2.523562in}{2.581987in}}%
\pgfpathlineto{\pgfqpoint{2.509700in}{2.602948in}}%
\pgfpathlineto{\pgfqpoint{2.501078in}{2.599189in}}%
\pgfpathlineto{\pgfqpoint{2.492443in}{2.595630in}}%
\pgfpathlineto{\pgfqpoint{2.483793in}{2.592273in}}%
\pgfpathlineto{\pgfqpoint{2.475130in}{2.589122in}}%
\pgfpathclose%
\pgfusepath{fill}%
\end{pgfscope}%
\begin{pgfscope}%
\pgfpathrectangle{\pgfqpoint{1.150000in}{0.150000in}}{\pgfqpoint{5.700000in}{5.700000in}}%
\pgfusepath{clip}%
\pgfsetbuttcap%
\pgfsetroundjoin%
\definecolor{currentfill}{rgb}{0.281446,0.084320,0.407414}%
\pgfsetfillcolor{currentfill}%
\pgfsetfillopacity{0.800000}%
\pgfsetlinewidth{0.000000pt}%
\definecolor{currentstroke}{rgb}{0.000000,0.000000,0.000000}%
\pgfsetstrokecolor{currentstroke}%
\pgfsetdash{}{0pt}%
\pgfpathmoveto{\pgfqpoint{3.709837in}{1.988904in}}%
\pgfpathlineto{\pgfqpoint{3.723457in}{1.988142in}}%
\pgfpathlineto{\pgfqpoint{3.737084in}{1.987581in}}%
\pgfpathlineto{\pgfqpoint{3.750718in}{1.987220in}}%
\pgfpathlineto{\pgfqpoint{3.764359in}{1.987059in}}%
\pgfpathlineto{\pgfqpoint{3.772388in}{1.997733in}}%
\pgfpathlineto{\pgfqpoint{3.780413in}{2.008399in}}%
\pgfpathlineto{\pgfqpoint{3.788433in}{2.019055in}}%
\pgfpathlineto{\pgfqpoint{3.796447in}{2.029700in}}%
\pgfpathlineto{\pgfqpoint{3.782815in}{2.029711in}}%
\pgfpathlineto{\pgfqpoint{3.769190in}{2.029922in}}%
\pgfpathlineto{\pgfqpoint{3.755572in}{2.030333in}}%
\pgfpathlineto{\pgfqpoint{3.741961in}{2.030945in}}%
\pgfpathlineto{\pgfqpoint{3.733938in}{2.020438in}}%
\pgfpathlineto{\pgfqpoint{3.725909in}{2.009928in}}%
\pgfpathlineto{\pgfqpoint{3.717876in}{1.999416in}}%
\pgfpathlineto{\pgfqpoint{3.709837in}{1.988904in}}%
\pgfpathclose%
\pgfusepath{fill}%
\end{pgfscope}%
\begin{pgfscope}%
\pgfpathrectangle{\pgfqpoint{1.150000in}{0.150000in}}{\pgfqpoint{5.700000in}{5.700000in}}%
\pgfusepath{clip}%
\pgfsetbuttcap%
\pgfsetroundjoin%
\definecolor{currentfill}{rgb}{0.281887,0.150881,0.465405}%
\pgfsetfillcolor{currentfill}%
\pgfsetfillopacity{0.800000}%
\pgfsetlinewidth{0.000000pt}%
\definecolor{currentstroke}{rgb}{0.000000,0.000000,0.000000}%
\pgfsetstrokecolor{currentstroke}%
\pgfsetdash{}{0pt}%
\pgfpathmoveto{\pgfqpoint{3.969589in}{2.123029in}}%
\pgfpathlineto{\pgfqpoint{3.983278in}{2.124993in}}%
\pgfpathlineto{\pgfqpoint{3.996976in}{2.127151in}}%
\pgfpathlineto{\pgfqpoint{4.010684in}{2.129502in}}%
\pgfpathlineto{\pgfqpoint{4.024401in}{2.132047in}}%
\pgfpathlineto{\pgfqpoint{4.032348in}{2.142759in}}%
\pgfpathlineto{\pgfqpoint{4.040290in}{2.153431in}}%
\pgfpathlineto{\pgfqpoint{4.048227in}{2.164062in}}%
\pgfpathlineto{\pgfqpoint{4.056159in}{2.174652in}}%
\pgfpathlineto{\pgfqpoint{4.042449in}{2.172053in}}%
\pgfpathlineto{\pgfqpoint{4.028748in}{2.169646in}}%
\pgfpathlineto{\pgfqpoint{4.015056in}{2.167433in}}%
\pgfpathlineto{\pgfqpoint{4.001374in}{2.165414in}}%
\pgfpathlineto{\pgfqpoint{3.993435in}{2.154867in}}%
\pgfpathlineto{\pgfqpoint{3.985491in}{2.144287in}}%
\pgfpathlineto{\pgfqpoint{3.977543in}{2.133674in}}%
\pgfpathlineto{\pgfqpoint{3.969589in}{2.123029in}}%
\pgfpathclose%
\pgfusepath{fill}%
\end{pgfscope}%
\begin{pgfscope}%
\pgfpathrectangle{\pgfqpoint{1.150000in}{0.150000in}}{\pgfqpoint{5.700000in}{5.700000in}}%
\pgfusepath{clip}%
\pgfsetbuttcap%
\pgfsetroundjoin%
\definecolor{currentfill}{rgb}{0.171176,0.452530,0.557965}%
\pgfsetfillcolor{currentfill}%
\pgfsetfillopacity{0.800000}%
\pgfsetlinewidth{0.000000pt}%
\definecolor{currentstroke}{rgb}{0.000000,0.000000,0.000000}%
\pgfsetstrokecolor{currentstroke}%
\pgfsetdash{}{0pt}%
\pgfpathmoveto{\pgfqpoint{5.071807in}{2.898039in}}%
\pgfpathlineto{\pgfqpoint{5.085993in}{2.906727in}}%
\pgfpathlineto{\pgfqpoint{5.100195in}{2.915594in}}%
\pgfpathlineto{\pgfqpoint{5.114414in}{2.924641in}}%
\pgfpathlineto{\pgfqpoint{5.128650in}{2.933867in}}%
\pgfpathlineto{\pgfqpoint{5.136157in}{2.939769in}}%
\pgfpathlineto{\pgfqpoint{5.143657in}{2.945655in}}%
\pgfpathlineto{\pgfqpoint{5.151152in}{2.951531in}}%
\pgfpathlineto{\pgfqpoint{5.158640in}{2.957402in}}%
\pgfpathlineto{\pgfqpoint{5.144421in}{2.948582in}}%
\pgfpathlineto{\pgfqpoint{5.130219in}{2.939941in}}%
\pgfpathlineto{\pgfqpoint{5.116033in}{2.931478in}}%
\pgfpathlineto{\pgfqpoint{5.101864in}{2.923195in}}%
\pgfpathlineto{\pgfqpoint{5.094358in}{2.916908in}}%
\pgfpathlineto{\pgfqpoint{5.086847in}{2.910622in}}%
\pgfpathlineto{\pgfqpoint{5.079330in}{2.904335in}}%
\pgfpathlineto{\pgfqpoint{5.071807in}{2.898039in}}%
\pgfpathclose%
\pgfusepath{fill}%
\end{pgfscope}%
\begin{pgfscope}%
\pgfpathrectangle{\pgfqpoint{1.150000in}{0.150000in}}{\pgfqpoint{5.700000in}{5.700000in}}%
\pgfusepath{clip}%
\pgfsetbuttcap%
\pgfsetroundjoin%
\definecolor{currentfill}{rgb}{0.124780,0.640461,0.527068}%
\pgfsetfillcolor{currentfill}%
\pgfsetfillopacity{0.800000}%
\pgfsetlinewidth{0.000000pt}%
\definecolor{currentstroke}{rgb}{0.000000,0.000000,0.000000}%
\pgfsetstrokecolor{currentstroke}%
\pgfsetdash{}{0pt}%
\pgfpathmoveto{\pgfqpoint{5.969343in}{3.475944in}}%
\pgfpathlineto{\pgfqpoint{5.983961in}{3.485358in}}%
\pgfpathlineto{\pgfqpoint{5.998598in}{3.494944in}}%
\pgfpathlineto{\pgfqpoint{6.013255in}{3.504702in}}%
\pgfpathlineto{\pgfqpoint{6.027931in}{3.514633in}}%
\pgfpathlineto{\pgfqpoint{6.034986in}{3.517857in}}%
\pgfpathlineto{\pgfqpoint{6.042041in}{3.521273in}}%
\pgfpathlineto{\pgfqpoint{6.049097in}{3.524889in}}%
\pgfpathlineto{\pgfqpoint{6.056154in}{3.528713in}}%
\pgfpathlineto{\pgfqpoint{6.041518in}{3.519555in}}%
\pgfpathlineto{\pgfqpoint{6.026901in}{3.510568in}}%
\pgfpathlineto{\pgfqpoint{6.012303in}{3.501751in}}%
\pgfpathlineto{\pgfqpoint{5.997725in}{3.493106in}}%
\pgfpathlineto{\pgfqpoint{5.990627in}{3.488501in}}%
\pgfpathlineto{\pgfqpoint{5.983531in}{3.484111in}}%
\pgfpathlineto{\pgfqpoint{5.976437in}{3.479928in}}%
\pgfpathlineto{\pgfqpoint{5.969343in}{3.475944in}}%
\pgfpathclose%
\pgfusepath{fill}%
\end{pgfscope}%
\begin{pgfscope}%
\pgfpathrectangle{\pgfqpoint{1.150000in}{0.150000in}}{\pgfqpoint{5.700000in}{5.700000in}}%
\pgfusepath{clip}%
\pgfsetbuttcap%
\pgfsetroundjoin%
\definecolor{currentfill}{rgb}{0.278826,0.175490,0.483397}%
\pgfsetfillcolor{currentfill}%
\pgfsetfillopacity{0.800000}%
\pgfsetlinewidth{0.000000pt}%
\definecolor{currentstroke}{rgb}{0.000000,0.000000,0.000000}%
\pgfsetstrokecolor{currentstroke}%
\pgfsetdash{}{0pt}%
\pgfpathmoveto{\pgfqpoint{4.056159in}{2.174652in}}%
\pgfpathlineto{\pgfqpoint{4.069879in}{2.177444in}}%
\pgfpathlineto{\pgfqpoint{4.083609in}{2.180429in}}%
\pgfpathlineto{\pgfqpoint{4.097348in}{2.183605in}}%
\pgfpathlineto{\pgfqpoint{4.111098in}{2.186972in}}%
\pgfpathlineto{\pgfqpoint{4.119019in}{2.197556in}}%
\pgfpathlineto{\pgfqpoint{4.126934in}{2.208091in}}%
\pgfpathlineto{\pgfqpoint{4.134844in}{2.218577in}}%
\pgfpathlineto{\pgfqpoint{4.142750in}{2.229015in}}%
\pgfpathlineto{\pgfqpoint{4.129006in}{2.225625in}}%
\pgfpathlineto{\pgfqpoint{4.115273in}{2.222426in}}%
\pgfpathlineto{\pgfqpoint{4.101549in}{2.219418in}}%
\pgfpathlineto{\pgfqpoint{4.087836in}{2.216603in}}%
\pgfpathlineto{\pgfqpoint{4.079924in}{2.206176in}}%
\pgfpathlineto{\pgfqpoint{4.072007in}{2.195709in}}%
\pgfpathlineto{\pgfqpoint{4.064086in}{2.185201in}}%
\pgfpathlineto{\pgfqpoint{4.056159in}{2.174652in}}%
\pgfpathclose%
\pgfusepath{fill}%
\end{pgfscope}%
\begin{pgfscope}%
\pgfpathrectangle{\pgfqpoint{1.150000in}{0.150000in}}{\pgfqpoint{5.700000in}{5.700000in}}%
\pgfusepath{clip}%
\pgfsetbuttcap%
\pgfsetroundjoin%
\definecolor{currentfill}{rgb}{0.279566,0.067836,0.391917}%
\pgfsetfillcolor{currentfill}%
\pgfsetfillopacity{0.800000}%
\pgfsetlinewidth{0.000000pt}%
\definecolor{currentstroke}{rgb}{0.000000,0.000000,0.000000}%
\pgfsetstrokecolor{currentstroke}%
\pgfsetdash{}{0pt}%
\pgfpathmoveto{\pgfqpoint{3.623168in}{1.952664in}}%
\pgfpathlineto{\pgfqpoint{3.636774in}{1.950911in}}%
\pgfpathlineto{\pgfqpoint{3.650386in}{1.949360in}}%
\pgfpathlineto{\pgfqpoint{3.664004in}{1.948013in}}%
\pgfpathlineto{\pgfqpoint{3.677628in}{1.946868in}}%
\pgfpathlineto{\pgfqpoint{3.685688in}{1.957372in}}%
\pgfpathlineto{\pgfqpoint{3.693743in}{1.967880in}}%
\pgfpathlineto{\pgfqpoint{3.701792in}{1.978391in}}%
\pgfpathlineto{\pgfqpoint{3.709837in}{1.988904in}}%
\pgfpathlineto{\pgfqpoint{3.696223in}{1.989867in}}%
\pgfpathlineto{\pgfqpoint{3.682615in}{1.991033in}}%
\pgfpathlineto{\pgfqpoint{3.669013in}{1.992401in}}%
\pgfpathlineto{\pgfqpoint{3.655417in}{1.993973in}}%
\pgfpathlineto{\pgfqpoint{3.647363in}{1.983631in}}%
\pgfpathlineto{\pgfqpoint{3.639303in}{1.973297in}}%
\pgfpathlineto{\pgfqpoint{3.631238in}{1.962975in}}%
\pgfpathlineto{\pgfqpoint{3.623168in}{1.952664in}}%
\pgfpathclose%
\pgfusepath{fill}%
\end{pgfscope}%
\begin{pgfscope}%
\pgfpathrectangle{\pgfqpoint{1.150000in}{0.150000in}}{\pgfqpoint{5.700000in}{5.700000in}}%
\pgfusepath{clip}%
\pgfsetbuttcap%
\pgfsetroundjoin%
\definecolor{currentfill}{rgb}{0.223925,0.334994,0.548053}%
\pgfsetfillcolor{currentfill}%
\pgfsetfillopacity{0.800000}%
\pgfsetlinewidth{0.000000pt}%
\definecolor{currentstroke}{rgb}{0.000000,0.000000,0.000000}%
\pgfsetstrokecolor{currentstroke}%
\pgfsetdash{}{0pt}%
\pgfpathmoveto{\pgfqpoint{4.607344in}{2.563813in}}%
\pgfpathlineto{\pgfqpoint{4.621300in}{2.570596in}}%
\pgfpathlineto{\pgfqpoint{4.635270in}{2.577564in}}%
\pgfpathlineto{\pgfqpoint{4.649253in}{2.584715in}}%
\pgfpathlineto{\pgfqpoint{4.663251in}{2.592050in}}%
\pgfpathlineto{\pgfqpoint{4.670973in}{2.600465in}}%
\pgfpathlineto{\pgfqpoint{4.678690in}{2.608821in}}%
\pgfpathlineto{\pgfqpoint{4.686400in}{2.617119in}}%
\pgfpathlineto{\pgfqpoint{4.694103in}{2.625362in}}%
\pgfpathlineto{\pgfqpoint{4.680115in}{2.618232in}}%
\pgfpathlineto{\pgfqpoint{4.666140in}{2.611286in}}%
\pgfpathlineto{\pgfqpoint{4.652180in}{2.604524in}}%
\pgfpathlineto{\pgfqpoint{4.638233in}{2.597946in}}%
\pgfpathlineto{\pgfqpoint{4.630520in}{2.589485in}}%
\pgfpathlineto{\pgfqpoint{4.622801in}{2.580978in}}%
\pgfpathlineto{\pgfqpoint{4.615075in}{2.572422in}}%
\pgfpathlineto{\pgfqpoint{4.607344in}{2.563813in}}%
\pgfpathclose%
\pgfusepath{fill}%
\end{pgfscope}%
\begin{pgfscope}%
\pgfpathrectangle{\pgfqpoint{1.150000in}{0.150000in}}{\pgfqpoint{5.700000in}{5.700000in}}%
\pgfusepath{clip}%
\pgfsetbuttcap%
\pgfsetroundjoin%
\definecolor{currentfill}{rgb}{0.163625,0.471133,0.558148}%
\pgfsetfillcolor{currentfill}%
\pgfsetfillopacity{0.800000}%
\pgfsetlinewidth{0.000000pt}%
\definecolor{currentstroke}{rgb}{0.000000,0.000000,0.000000}%
\pgfsetstrokecolor{currentstroke}%
\pgfsetdash{}{0pt}%
\pgfpathmoveto{\pgfqpoint{5.158640in}{2.957402in}}%
\pgfpathlineto{\pgfqpoint{5.172876in}{2.966400in}}%
\pgfpathlineto{\pgfqpoint{5.187128in}{2.975578in}}%
\pgfpathlineto{\pgfqpoint{5.201397in}{2.984934in}}%
\pgfpathlineto{\pgfqpoint{5.215684in}{2.994469in}}%
\pgfpathlineto{\pgfqpoint{5.223148in}{2.999911in}}%
\pgfpathlineto{\pgfqpoint{5.230606in}{3.005349in}}%
\pgfpathlineto{\pgfqpoint{5.238058in}{3.010787in}}%
\pgfpathlineto{\pgfqpoint{5.245504in}{3.016231in}}%
\pgfpathlineto{\pgfqpoint{5.231236in}{3.007137in}}%
\pgfpathlineto{\pgfqpoint{5.216986in}{2.998220in}}%
\pgfpathlineto{\pgfqpoint{5.202752in}{2.989481in}}%
\pgfpathlineto{\pgfqpoint{5.188535in}{2.980920in}}%
\pgfpathlineto{\pgfqpoint{5.181070in}{2.975025in}}%
\pgfpathlineto{\pgfqpoint{5.173599in}{2.969144in}}%
\pgfpathlineto{\pgfqpoint{5.166123in}{2.963271in}}%
\pgfpathlineto{\pgfqpoint{5.158640in}{2.957402in}}%
\pgfpathclose%
\pgfusepath{fill}%
\end{pgfscope}%
\begin{pgfscope}%
\pgfpathrectangle{\pgfqpoint{1.150000in}{0.150000in}}{\pgfqpoint{5.700000in}{5.700000in}}%
\pgfusepath{clip}%
\pgfsetbuttcap%
\pgfsetroundjoin%
\definecolor{currentfill}{rgb}{0.274128,0.199721,0.498911}%
\pgfsetfillcolor{currentfill}%
\pgfsetfillopacity{0.800000}%
\pgfsetlinewidth{0.000000pt}%
\definecolor{currentstroke}{rgb}{0.000000,0.000000,0.000000}%
\pgfsetstrokecolor{currentstroke}%
\pgfsetdash{}{0pt}%
\pgfpathmoveto{\pgfqpoint{4.142750in}{2.229015in}}%
\pgfpathlineto{\pgfqpoint{4.156504in}{2.232596in}}%
\pgfpathlineto{\pgfqpoint{4.170268in}{2.236368in}}%
\pgfpathlineto{\pgfqpoint{4.184043in}{2.240330in}}%
\pgfpathlineto{\pgfqpoint{4.197829in}{2.244482in}}%
\pgfpathlineto{\pgfqpoint{4.205723in}{2.254875in}}%
\pgfpathlineto{\pgfqpoint{4.213612in}{2.265212in}}%
\pgfpathlineto{\pgfqpoint{4.221495in}{2.275494in}}%
\pgfpathlineto{\pgfqpoint{4.229373in}{2.285721in}}%
\pgfpathlineto{\pgfqpoint{4.215593in}{2.281579in}}%
\pgfpathlineto{\pgfqpoint{4.201824in}{2.277626in}}%
\pgfpathlineto{\pgfqpoint{4.188066in}{2.273863in}}%
\pgfpathlineto{\pgfqpoint{4.174318in}{2.270291in}}%
\pgfpathlineto{\pgfqpoint{4.166434in}{2.260043in}}%
\pgfpathlineto{\pgfqpoint{4.158544in}{2.249747in}}%
\pgfpathlineto{\pgfqpoint{4.150650in}{2.239405in}}%
\pgfpathlineto{\pgfqpoint{4.142750in}{2.229015in}}%
\pgfpathclose%
\pgfusepath{fill}%
\end{pgfscope}%
\begin{pgfscope}%
\pgfpathrectangle{\pgfqpoint{1.150000in}{0.150000in}}{\pgfqpoint{5.700000in}{5.700000in}}%
\pgfusepath{clip}%
\pgfsetbuttcap%
\pgfsetroundjoin%
\definecolor{currentfill}{rgb}{0.281446,0.084320,0.407414}%
\pgfsetfillcolor{currentfill}%
\pgfsetfillopacity{0.800000}%
\pgfsetlinewidth{0.000000pt}%
\definecolor{currentstroke}{rgb}{0.000000,0.000000,0.000000}%
\pgfsetstrokecolor{currentstroke}%
\pgfsetdash{}{0pt}%
\pgfpathmoveto{\pgfqpoint{2.969852in}{2.008533in}}%
\pgfpathlineto{\pgfqpoint{2.983487in}{1.997391in}}%
\pgfpathlineto{\pgfqpoint{2.997120in}{1.986494in}}%
\pgfpathlineto{\pgfqpoint{3.010751in}{1.975841in}}%
\pgfpathlineto{\pgfqpoint{3.024379in}{1.965430in}}%
\pgfpathlineto{\pgfqpoint{3.032725in}{1.972236in}}%
\pgfpathlineto{\pgfqpoint{3.041061in}{1.979165in}}%
\pgfpathlineto{\pgfqpoint{3.049389in}{1.986214in}}%
\pgfpathlineto{\pgfqpoint{3.057707in}{1.993380in}}%
\pgfpathlineto{\pgfqpoint{3.044103in}{2.003447in}}%
\pgfpathlineto{\pgfqpoint{3.030497in}{2.013756in}}%
\pgfpathlineto{\pgfqpoint{3.016889in}{2.024308in}}%
\pgfpathlineto{\pgfqpoint{3.003279in}{2.035105in}}%
\pgfpathlineto{\pgfqpoint{2.994936in}{2.028272in}}%
\pgfpathlineto{\pgfqpoint{2.986584in}{2.021564in}}%
\pgfpathlineto{\pgfqpoint{2.978223in}{2.014983in}}%
\pgfpathlineto{\pgfqpoint{2.969852in}{2.008533in}}%
\pgfpathclose%
\pgfusepath{fill}%
\end{pgfscope}%
\begin{pgfscope}%
\pgfpathrectangle{\pgfqpoint{1.150000in}{0.150000in}}{\pgfqpoint{5.700000in}{5.700000in}}%
\pgfusepath{clip}%
\pgfsetbuttcap%
\pgfsetroundjoin%
\definecolor{currentfill}{rgb}{0.276022,0.044167,0.370164}%
\pgfsetfillcolor{currentfill}%
\pgfsetfillopacity{0.800000}%
\pgfsetlinewidth{0.000000pt}%
\definecolor{currentstroke}{rgb}{0.000000,0.000000,0.000000}%
\pgfsetstrokecolor{currentstroke}%
\pgfsetdash{}{0pt}%
\pgfpathmoveto{\pgfqpoint{3.166497in}{1.921361in}}%
\pgfpathlineto{\pgfqpoint{3.180094in}{1.913402in}}%
\pgfpathlineto{\pgfqpoint{3.193691in}{1.905672in}}%
\pgfpathlineto{\pgfqpoint{3.207288in}{1.898167in}}%
\pgfpathlineto{\pgfqpoint{3.220886in}{1.890888in}}%
\pgfpathlineto{\pgfqpoint{3.229132in}{1.899134in}}%
\pgfpathlineto{\pgfqpoint{3.237370in}{1.907467in}}%
\pgfpathlineto{\pgfqpoint{3.245601in}{1.915883in}}%
\pgfpathlineto{\pgfqpoint{3.253825in}{1.924381in}}%
\pgfpathlineto{\pgfqpoint{3.240246in}{1.931351in}}%
\pgfpathlineto{\pgfqpoint{3.226668in}{1.938545in}}%
\pgfpathlineto{\pgfqpoint{3.213091in}{1.945966in}}%
\pgfpathlineto{\pgfqpoint{3.199514in}{1.953615in}}%
\pgfpathlineto{\pgfqpoint{3.191271in}{1.945415in}}%
\pgfpathlineto{\pgfqpoint{3.183021in}{1.937304in}}%
\pgfpathlineto{\pgfqpoint{3.174763in}{1.929285in}}%
\pgfpathlineto{\pgfqpoint{3.166497in}{1.921361in}}%
\pgfpathclose%
\pgfusepath{fill}%
\end{pgfscope}%
\begin{pgfscope}%
\pgfpathrectangle{\pgfqpoint{1.150000in}{0.150000in}}{\pgfqpoint{5.700000in}{5.700000in}}%
\pgfusepath{clip}%
\pgfsetbuttcap%
\pgfsetroundjoin%
\definecolor{currentfill}{rgb}{0.274952,0.037752,0.364543}%
\pgfsetfillcolor{currentfill}%
\pgfsetfillopacity{0.800000}%
\pgfsetlinewidth{0.000000pt}%
\definecolor{currentstroke}{rgb}{0.000000,0.000000,0.000000}%
\pgfsetstrokecolor{currentstroke}%
\pgfsetdash{}{0pt}%
\pgfpathmoveto{\pgfqpoint{3.308153in}{1.898727in}}%
\pgfpathlineto{\pgfqpoint{3.321739in}{1.892864in}}%
\pgfpathlineto{\pgfqpoint{3.335327in}{1.887220in}}%
\pgfpathlineto{\pgfqpoint{3.348918in}{1.881792in}}%
\pgfpathlineto{\pgfqpoint{3.362511in}{1.876581in}}%
\pgfpathlineto{\pgfqpoint{3.370693in}{1.885735in}}%
\pgfpathlineto{\pgfqpoint{3.378868in}{1.894949in}}%
\pgfpathlineto{\pgfqpoint{3.387037in}{1.904220in}}%
\pgfpathlineto{\pgfqpoint{3.395199in}{1.913546in}}%
\pgfpathlineto{\pgfqpoint{3.381622in}{1.918481in}}%
\pgfpathlineto{\pgfqpoint{3.368047in}{1.923632in}}%
\pgfpathlineto{\pgfqpoint{3.354475in}{1.928999in}}%
\pgfpathlineto{\pgfqpoint{3.340906in}{1.934585in}}%
\pgfpathlineto{\pgfqpoint{3.332728in}{1.925523in}}%
\pgfpathlineto{\pgfqpoint{3.324543in}{1.916525in}}%
\pgfpathlineto{\pgfqpoint{3.316351in}{1.907592in}}%
\pgfpathlineto{\pgfqpoint{3.308153in}{1.898727in}}%
\pgfpathclose%
\pgfusepath{fill}%
\end{pgfscope}%
\begin{pgfscope}%
\pgfpathrectangle{\pgfqpoint{1.150000in}{0.150000in}}{\pgfqpoint{5.700000in}{5.700000in}}%
\pgfusepath{clip}%
\pgfsetbuttcap%
\pgfsetroundjoin%
\definecolor{currentfill}{rgb}{0.277941,0.056324,0.381191}%
\pgfsetfillcolor{currentfill}%
\pgfsetfillopacity{0.800000}%
\pgfsetlinewidth{0.000000pt}%
\definecolor{currentstroke}{rgb}{0.000000,0.000000,0.000000}%
\pgfsetstrokecolor{currentstroke}%
\pgfsetdash{}{0pt}%
\pgfpathmoveto{\pgfqpoint{3.536413in}{1.921498in}}%
\pgfpathlineto{\pgfqpoint{3.550010in}{1.918709in}}%
\pgfpathlineto{\pgfqpoint{3.563612in}{1.916126in}}%
\pgfpathlineto{\pgfqpoint{3.577219in}{1.913750in}}%
\pgfpathlineto{\pgfqpoint{3.590831in}{1.911579in}}%
\pgfpathlineto{\pgfqpoint{3.598924in}{1.921823in}}%
\pgfpathlineto{\pgfqpoint{3.607011in}{1.932087in}}%
\pgfpathlineto{\pgfqpoint{3.615092in}{1.942368in}}%
\pgfpathlineto{\pgfqpoint{3.623168in}{1.952664in}}%
\pgfpathlineto{\pgfqpoint{3.609567in}{1.954623in}}%
\pgfpathlineto{\pgfqpoint{3.595972in}{1.956786in}}%
\pgfpathlineto{\pgfqpoint{3.582381in}{1.959155in}}%
\pgfpathlineto{\pgfqpoint{3.568796in}{1.961731in}}%
\pgfpathlineto{\pgfqpoint{3.560709in}{1.951636in}}%
\pgfpathlineto{\pgfqpoint{3.552616in}{1.941565in}}%
\pgfpathlineto{\pgfqpoint{3.544517in}{1.931518in}}%
\pgfpathlineto{\pgfqpoint{3.536413in}{1.921498in}}%
\pgfpathclose%
\pgfusepath{fill}%
\end{pgfscope}%
\begin{pgfscope}%
\pgfpathrectangle{\pgfqpoint{1.150000in}{0.150000in}}{\pgfqpoint{5.700000in}{5.700000in}}%
\pgfusepath{clip}%
\pgfsetbuttcap%
\pgfsetroundjoin%
\definecolor{currentfill}{rgb}{0.216210,0.351535,0.550627}%
\pgfsetfillcolor{currentfill}%
\pgfsetfillopacity{0.800000}%
\pgfsetlinewidth{0.000000pt}%
\definecolor{currentstroke}{rgb}{0.000000,0.000000,0.000000}%
\pgfsetstrokecolor{currentstroke}%
\pgfsetdash{}{0pt}%
\pgfpathmoveto{\pgfqpoint{2.419394in}{2.677764in}}%
\pgfpathlineto{\pgfqpoint{2.433349in}{2.655096in}}%
\pgfpathlineto{\pgfqpoint{2.447289in}{2.632768in}}%
\pgfpathlineto{\pgfqpoint{2.461216in}{2.610778in}}%
\pgfpathlineto{\pgfqpoint{2.475130in}{2.589122in}}%
\pgfpathlineto{\pgfqpoint{2.483793in}{2.592273in}}%
\pgfpathlineto{\pgfqpoint{2.492443in}{2.595630in}}%
\pgfpathlineto{\pgfqpoint{2.501078in}{2.599189in}}%
\pgfpathlineto{\pgfqpoint{2.509700in}{2.602948in}}%
\pgfpathlineto{\pgfqpoint{2.495825in}{2.624238in}}%
\pgfpathlineto{\pgfqpoint{2.481937in}{2.645860in}}%
\pgfpathlineto{\pgfqpoint{2.468037in}{2.667820in}}%
\pgfpathlineto{\pgfqpoint{2.454122in}{2.690118in}}%
\pgfpathlineto{\pgfqpoint{2.445462in}{2.686714in}}%
\pgfpathlineto{\pgfqpoint{2.436788in}{2.683518in}}%
\pgfpathlineto{\pgfqpoint{2.428098in}{2.680534in}}%
\pgfpathlineto{\pgfqpoint{2.419394in}{2.677764in}}%
\pgfpathclose%
\pgfusepath{fill}%
\end{pgfscope}%
\begin{pgfscope}%
\pgfpathrectangle{\pgfqpoint{1.150000in}{0.150000in}}{\pgfqpoint{5.700000in}{5.700000in}}%
\pgfusepath{clip}%
\pgfsetbuttcap%
\pgfsetroundjoin%
\definecolor{currentfill}{rgb}{0.134692,0.658636,0.517649}%
\pgfsetfillcolor{currentfill}%
\pgfsetfillopacity{0.800000}%
\pgfsetlinewidth{0.000000pt}%
\definecolor{currentstroke}{rgb}{0.000000,0.000000,0.000000}%
\pgfsetstrokecolor{currentstroke}%
\pgfsetdash{}{0pt}%
\pgfpathmoveto{\pgfqpoint{6.056154in}{3.528713in}}%
\pgfpathlineto{\pgfqpoint{6.070810in}{3.538042in}}%
\pgfpathlineto{\pgfqpoint{6.085486in}{3.547542in}}%
\pgfpathlineto{\pgfqpoint{6.100182in}{3.557214in}}%
\pgfpathlineto{\pgfqpoint{6.114898in}{3.567057in}}%
\pgfpathlineto{\pgfqpoint{6.121915in}{3.570305in}}%
\pgfpathlineto{\pgfqpoint{6.128935in}{3.573772in}}%
\pgfpathlineto{\pgfqpoint{6.135957in}{3.577465in}}%
\pgfpathlineto{\pgfqpoint{6.121273in}{3.568223in}}%
\pgfpathlineto{\pgfqpoint{6.106609in}{3.559152in}}%
\pgfpathlineto{\pgfqpoint{6.091965in}{3.550251in}}%
\pgfpathlineto{\pgfqpoint{6.077340in}{3.541520in}}%
\pgfpathlineto{\pgfqpoint{6.070276in}{3.537020in}}%
\pgfpathlineto{\pgfqpoint{6.063214in}{3.532754in}}%
\pgfpathlineto{\pgfqpoint{6.056154in}{3.528713in}}%
\pgfpathclose%
\pgfusepath{fill}%
\end{pgfscope}%
\begin{pgfscope}%
\pgfpathrectangle{\pgfqpoint{1.150000in}{0.150000in}}{\pgfqpoint{5.700000in}{5.700000in}}%
\pgfusepath{clip}%
\pgfsetbuttcap%
\pgfsetroundjoin%
\definecolor{currentfill}{rgb}{0.212395,0.359683,0.551710}%
\pgfsetfillcolor{currentfill}%
\pgfsetfillopacity{0.800000}%
\pgfsetlinewidth{0.000000pt}%
\definecolor{currentstroke}{rgb}{0.000000,0.000000,0.000000}%
\pgfsetstrokecolor{currentstroke}%
\pgfsetdash{}{0pt}%
\pgfpathmoveto{\pgfqpoint{4.694103in}{2.625362in}}%
\pgfpathlineto{\pgfqpoint{4.708106in}{2.632674in}}%
\pgfpathlineto{\pgfqpoint{4.722124in}{2.640170in}}%
\pgfpathlineto{\pgfqpoint{4.736156in}{2.647850in}}%
\pgfpathlineto{\pgfqpoint{4.750202in}{2.655712in}}%
\pgfpathlineto{\pgfqpoint{4.757890in}{2.663676in}}%
\pgfpathlineto{\pgfqpoint{4.765572in}{2.671583in}}%
\pgfpathlineto{\pgfqpoint{4.773247in}{2.679435in}}%
\pgfpathlineto{\pgfqpoint{4.780916in}{2.687237in}}%
\pgfpathlineto{\pgfqpoint{4.766880in}{2.679614in}}%
\pgfpathlineto{\pgfqpoint{4.752858in}{2.672174in}}%
\pgfpathlineto{\pgfqpoint{4.738851in}{2.664916in}}%
\pgfpathlineto{\pgfqpoint{4.724858in}{2.657842in}}%
\pgfpathlineto{\pgfqpoint{4.717178in}{2.649790in}}%
\pgfpathlineto{\pgfqpoint{4.709493in}{2.641694in}}%
\pgfpathlineto{\pgfqpoint{4.701801in}{2.633553in}}%
\pgfpathlineto{\pgfqpoint{4.694103in}{2.625362in}}%
\pgfpathclose%
\pgfusepath{fill}%
\end{pgfscope}%
\begin{pgfscope}%
\pgfpathrectangle{\pgfqpoint{1.150000in}{0.150000in}}{\pgfqpoint{5.700000in}{5.700000in}}%
\pgfusepath{clip}%
\pgfsetbuttcap%
\pgfsetroundjoin%
\definecolor{currentfill}{rgb}{0.266580,0.228262,0.514349}%
\pgfsetfillcolor{currentfill}%
\pgfsetfillopacity{0.800000}%
\pgfsetlinewidth{0.000000pt}%
\definecolor{currentstroke}{rgb}{0.000000,0.000000,0.000000}%
\pgfsetstrokecolor{currentstroke}%
\pgfsetdash{}{0pt}%
\pgfpathmoveto{\pgfqpoint{4.229373in}{2.285721in}}%
\pgfpathlineto{\pgfqpoint{4.243164in}{2.290053in}}%
\pgfpathlineto{\pgfqpoint{4.256967in}{2.294574in}}%
\pgfpathlineto{\pgfqpoint{4.270780in}{2.299283in}}%
\pgfpathlineto{\pgfqpoint{4.284606in}{2.304182in}}%
\pgfpathlineto{\pgfqpoint{4.292472in}{2.314326in}}%
\pgfpathlineto{\pgfqpoint{4.300334in}{2.324409in}}%
\pgfpathlineto{\pgfqpoint{4.308190in}{2.334432in}}%
\pgfpathlineto{\pgfqpoint{4.316040in}{2.344395in}}%
\pgfpathlineto{\pgfqpoint{4.302221in}{2.339539in}}%
\pgfpathlineto{\pgfqpoint{4.288413in}{2.334871in}}%
\pgfpathlineto{\pgfqpoint{4.274617in}{2.330392in}}%
\pgfpathlineto{\pgfqpoint{4.260833in}{2.326101in}}%
\pgfpathlineto{\pgfqpoint{4.252976in}{2.316084in}}%
\pgfpathlineto{\pgfqpoint{4.245114in}{2.306016in}}%
\pgfpathlineto{\pgfqpoint{4.237246in}{2.295895in}}%
\pgfpathlineto{\pgfqpoint{4.229373in}{2.285721in}}%
\pgfpathclose%
\pgfusepath{fill}%
\end{pgfscope}%
\begin{pgfscope}%
\pgfpathrectangle{\pgfqpoint{1.150000in}{0.150000in}}{\pgfqpoint{5.700000in}{5.700000in}}%
\pgfusepath{clip}%
\pgfsetbuttcap%
\pgfsetroundjoin%
\definecolor{currentfill}{rgb}{0.154815,0.493313,0.557840}%
\pgfsetfillcolor{currentfill}%
\pgfsetfillopacity{0.800000}%
\pgfsetlinewidth{0.000000pt}%
\definecolor{currentstroke}{rgb}{0.000000,0.000000,0.000000}%
\pgfsetstrokecolor{currentstroke}%
\pgfsetdash{}{0pt}%
\pgfpathmoveto{\pgfqpoint{5.245504in}{3.016231in}}%
\pgfpathlineto{\pgfqpoint{5.259789in}{3.025504in}}%
\pgfpathlineto{\pgfqpoint{5.274091in}{3.034955in}}%
\pgfpathlineto{\pgfqpoint{5.288410in}{3.044585in}}%
\pgfpathlineto{\pgfqpoint{5.302747in}{3.054392in}}%
\pgfpathlineto{\pgfqpoint{5.310167in}{3.059386in}}%
\pgfpathlineto{\pgfqpoint{5.317581in}{3.064388in}}%
\pgfpathlineto{\pgfqpoint{5.324990in}{3.069404in}}%
\pgfpathlineto{\pgfqpoint{5.332393in}{3.074439in}}%
\pgfpathlineto{\pgfqpoint{5.318076in}{3.065106in}}%
\pgfpathlineto{\pgfqpoint{5.303777in}{3.055950in}}%
\pgfpathlineto{\pgfqpoint{5.289496in}{3.046972in}}%
\pgfpathlineto{\pgfqpoint{5.275231in}{3.038171in}}%
\pgfpathlineto{\pgfqpoint{5.267808in}{3.032651in}}%
\pgfpathlineto{\pgfqpoint{5.260379in}{3.027158in}}%
\pgfpathlineto{\pgfqpoint{5.252944in}{3.021687in}}%
\pgfpathlineto{\pgfqpoint{5.245504in}{3.016231in}}%
\pgfpathclose%
\pgfusepath{fill}%
\end{pgfscope}%
\begin{pgfscope}%
\pgfpathrectangle{\pgfqpoint{1.150000in}{0.150000in}}{\pgfqpoint{5.700000in}{5.700000in}}%
\pgfusepath{clip}%
\pgfsetbuttcap%
\pgfsetroundjoin%
\definecolor{currentfill}{rgb}{0.276022,0.044167,0.370164}%
\pgfsetfillcolor{currentfill}%
\pgfsetfillopacity{0.800000}%
\pgfsetlinewidth{0.000000pt}%
\definecolor{currentstroke}{rgb}{0.000000,0.000000,0.000000}%
\pgfsetstrokecolor{currentstroke}%
\pgfsetdash{}{0pt}%
\pgfpathmoveto{\pgfqpoint{3.449539in}{1.895947in}}%
\pgfpathlineto{\pgfqpoint{3.463133in}{1.892078in}}%
\pgfpathlineto{\pgfqpoint{3.476730in}{1.888418in}}%
\pgfpathlineto{\pgfqpoint{3.490331in}{1.884968in}}%
\pgfpathlineto{\pgfqpoint{3.503937in}{1.881727in}}%
\pgfpathlineto{\pgfqpoint{3.512065in}{1.891619in}}%
\pgfpathlineto{\pgfqpoint{3.520187in}{1.901547in}}%
\pgfpathlineto{\pgfqpoint{3.528303in}{1.911507in}}%
\pgfpathlineto{\pgfqpoint{3.536413in}{1.921498in}}%
\pgfpathlineto{\pgfqpoint{3.522820in}{1.924495in}}%
\pgfpathlineto{\pgfqpoint{3.509232in}{1.927700in}}%
\pgfpathlineto{\pgfqpoint{3.495648in}{1.931115in}}%
\pgfpathlineto{\pgfqpoint{3.482068in}{1.934740in}}%
\pgfpathlineto{\pgfqpoint{3.473945in}{1.924982in}}%
\pgfpathlineto{\pgfqpoint{3.465816in}{1.915262in}}%
\pgfpathlineto{\pgfqpoint{3.457680in}{1.905583in}}%
\pgfpathlineto{\pgfqpoint{3.449539in}{1.895947in}}%
\pgfpathclose%
\pgfusepath{fill}%
\end{pgfscope}%
\begin{pgfscope}%
\pgfpathrectangle{\pgfqpoint{1.150000in}{0.150000in}}{\pgfqpoint{5.700000in}{5.700000in}}%
\pgfusepath{clip}%
\pgfsetbuttcap%
\pgfsetroundjoin%
\definecolor{currentfill}{rgb}{0.147607,0.511733,0.557049}%
\pgfsetfillcolor{currentfill}%
\pgfsetfillopacity{0.800000}%
\pgfsetlinewidth{0.000000pt}%
\definecolor{currentstroke}{rgb}{0.000000,0.000000,0.000000}%
\pgfsetstrokecolor{currentstroke}%
\pgfsetdash{}{0pt}%
\pgfpathmoveto{\pgfqpoint{5.332393in}{3.074439in}}%
\pgfpathlineto{\pgfqpoint{5.346726in}{3.083950in}}%
\pgfpathlineto{\pgfqpoint{5.361078in}{3.093638in}}%
\pgfpathlineto{\pgfqpoint{5.375447in}{3.103503in}}%
\pgfpathlineto{\pgfqpoint{5.389835in}{3.113546in}}%
\pgfpathlineto{\pgfqpoint{5.397210in}{3.118110in}}%
\pgfpathlineto{\pgfqpoint{5.404579in}{3.122697in}}%
\pgfpathlineto{\pgfqpoint{5.411943in}{3.127311in}}%
\pgfpathlineto{\pgfqpoint{5.419302in}{3.131959in}}%
\pgfpathlineto{\pgfqpoint{5.404937in}{3.122424in}}%
\pgfpathlineto{\pgfqpoint{5.390591in}{3.113066in}}%
\pgfpathlineto{\pgfqpoint{5.376261in}{3.103885in}}%
\pgfpathlineto{\pgfqpoint{5.361950in}{3.094880in}}%
\pgfpathlineto{\pgfqpoint{5.354568in}{3.089713in}}%
\pgfpathlineto{\pgfqpoint{5.347182in}{3.084588in}}%
\pgfpathlineto{\pgfqpoint{5.339790in}{3.079499in}}%
\pgfpathlineto{\pgfqpoint{5.332393in}{3.074439in}}%
\pgfpathclose%
\pgfusepath{fill}%
\end{pgfscope}%
\begin{pgfscope}%
\pgfpathrectangle{\pgfqpoint{1.150000in}{0.150000in}}{\pgfqpoint{5.700000in}{5.700000in}}%
\pgfusepath{clip}%
\pgfsetbuttcap%
\pgfsetroundjoin%
\definecolor{currentfill}{rgb}{0.279566,0.067836,0.391917}%
\pgfsetfillcolor{currentfill}%
\pgfsetfillopacity{0.800000}%
\pgfsetlinewidth{0.000000pt}%
\definecolor{currentstroke}{rgb}{0.000000,0.000000,0.000000}%
\pgfsetstrokecolor{currentstroke}%
\pgfsetdash{}{0pt}%
\pgfpathmoveto{\pgfqpoint{3.024379in}{1.965430in}}%
\pgfpathlineto{\pgfqpoint{3.038005in}{1.955259in}}%
\pgfpathlineto{\pgfqpoint{3.051630in}{1.945326in}}%
\pgfpathlineto{\pgfqpoint{3.065253in}{1.935631in}}%
\pgfpathlineto{\pgfqpoint{3.078874in}{1.926172in}}%
\pgfpathlineto{\pgfqpoint{3.087196in}{1.933334in}}%
\pgfpathlineto{\pgfqpoint{3.095509in}{1.940611in}}%
\pgfpathlineto{\pgfqpoint{3.103813in}{1.947999in}}%
\pgfpathlineto{\pgfqpoint{3.112109in}{1.955496in}}%
\pgfpathlineto{\pgfqpoint{3.098510in}{1.964612in}}%
\pgfpathlineto{\pgfqpoint{3.084910in}{1.973964in}}%
\pgfpathlineto{\pgfqpoint{3.071309in}{1.983552in}}%
\pgfpathlineto{\pgfqpoint{3.057707in}{1.993380in}}%
\pgfpathlineto{\pgfqpoint{3.049389in}{1.986214in}}%
\pgfpathlineto{\pgfqpoint{3.041061in}{1.979165in}}%
\pgfpathlineto{\pgfqpoint{3.032725in}{1.972236in}}%
\pgfpathlineto{\pgfqpoint{3.024379in}{1.965430in}}%
\pgfpathclose%
\pgfusepath{fill}%
\end{pgfscope}%
\begin{pgfscope}%
\pgfpathrectangle{\pgfqpoint{1.150000in}{0.150000in}}{\pgfqpoint{5.700000in}{5.700000in}}%
\pgfusepath{clip}%
\pgfsetbuttcap%
\pgfsetroundjoin%
\definecolor{currentfill}{rgb}{0.258965,0.251537,0.524736}%
\pgfsetfillcolor{currentfill}%
\pgfsetfillopacity{0.800000}%
\pgfsetlinewidth{0.000000pt}%
\definecolor{currentstroke}{rgb}{0.000000,0.000000,0.000000}%
\pgfsetstrokecolor{currentstroke}%
\pgfsetdash{}{0pt}%
\pgfpathmoveto{\pgfqpoint{4.316040in}{2.344395in}}%
\pgfpathlineto{\pgfqpoint{4.329871in}{2.349440in}}%
\pgfpathlineto{\pgfqpoint{4.343714in}{2.354672in}}%
\pgfpathlineto{\pgfqpoint{4.357568in}{2.360091in}}%
\pgfpathlineto{\pgfqpoint{4.371436in}{2.365698in}}%
\pgfpathlineto{\pgfqpoint{4.379274in}{2.375541in}}%
\pgfpathlineto{\pgfqpoint{4.387107in}{2.385319in}}%
\pgfpathlineto{\pgfqpoint{4.394935in}{2.395034in}}%
\pgfpathlineto{\pgfqpoint{4.402757in}{2.404685in}}%
\pgfpathlineto{\pgfqpoint{4.388896in}{2.399153in}}%
\pgfpathlineto{\pgfqpoint{4.375048in}{2.393808in}}%
\pgfpathlineto{\pgfqpoint{4.361211in}{2.388650in}}%
\pgfpathlineto{\pgfqpoint{4.347387in}{2.383680in}}%
\pgfpathlineto{\pgfqpoint{4.339558in}{2.373942in}}%
\pgfpathlineto{\pgfqpoint{4.331724in}{2.364149in}}%
\pgfpathlineto{\pgfqpoint{4.323885in}{2.354301in}}%
\pgfpathlineto{\pgfqpoint{4.316040in}{2.344395in}}%
\pgfpathclose%
\pgfusepath{fill}%
\end{pgfscope}%
\begin{pgfscope}%
\pgfpathrectangle{\pgfqpoint{1.150000in}{0.150000in}}{\pgfqpoint{5.700000in}{5.700000in}}%
\pgfusepath{clip}%
\pgfsetbuttcap%
\pgfsetroundjoin%
\definecolor{currentfill}{rgb}{0.201239,0.383670,0.554294}%
\pgfsetfillcolor{currentfill}%
\pgfsetfillopacity{0.800000}%
\pgfsetlinewidth{0.000000pt}%
\definecolor{currentstroke}{rgb}{0.000000,0.000000,0.000000}%
\pgfsetstrokecolor{currentstroke}%
\pgfsetdash{}{0pt}%
\pgfpathmoveto{\pgfqpoint{4.780916in}{2.687237in}}%
\pgfpathlineto{\pgfqpoint{4.794967in}{2.695042in}}%
\pgfpathlineto{\pgfqpoint{4.809034in}{2.703029in}}%
\pgfpathlineto{\pgfqpoint{4.823115in}{2.711199in}}%
\pgfpathlineto{\pgfqpoint{4.837212in}{2.719551in}}%
\pgfpathlineto{\pgfqpoint{4.844864in}{2.727044in}}%
\pgfpathlineto{\pgfqpoint{4.852509in}{2.734483in}}%
\pgfpathlineto{\pgfqpoint{4.860148in}{2.741872in}}%
\pgfpathlineto{\pgfqpoint{4.867781in}{2.749214in}}%
\pgfpathlineto{\pgfqpoint{4.853695in}{2.741135in}}%
\pgfpathlineto{\pgfqpoint{4.839625in}{2.733238in}}%
\pgfpathlineto{\pgfqpoint{4.825570in}{2.725523in}}%
\pgfpathlineto{\pgfqpoint{4.811530in}{2.717989in}}%
\pgfpathlineto{\pgfqpoint{4.803886in}{2.710363in}}%
\pgfpathlineto{\pgfqpoint{4.796235in}{2.702697in}}%
\pgfpathlineto{\pgfqpoint{4.788579in}{2.694989in}}%
\pgfpathlineto{\pgfqpoint{4.780916in}{2.687237in}}%
\pgfpathclose%
\pgfusepath{fill}%
\end{pgfscope}%
\begin{pgfscope}%
\pgfpathrectangle{\pgfqpoint{1.150000in}{0.150000in}}{\pgfqpoint{5.700000in}{5.700000in}}%
\pgfusepath{clip}%
\pgfsetbuttcap%
\pgfsetroundjoin%
\definecolor{currentfill}{rgb}{0.199430,0.387607,0.554642}%
\pgfsetfillcolor{currentfill}%
\pgfsetfillopacity{0.800000}%
\pgfsetlinewidth{0.000000pt}%
\definecolor{currentstroke}{rgb}{0.000000,0.000000,0.000000}%
\pgfsetstrokecolor{currentstroke}%
\pgfsetdash{}{0pt}%
\pgfpathmoveto{\pgfqpoint{2.363431in}{2.771909in}}%
\pgfpathlineto{\pgfqpoint{2.377445in}{2.747844in}}%
\pgfpathlineto{\pgfqpoint{2.391443in}{2.724134in}}%
\pgfpathlineto{\pgfqpoint{2.405426in}{2.700775in}}%
\pgfpathlineto{\pgfqpoint{2.419394in}{2.677764in}}%
\pgfpathlineto{\pgfqpoint{2.428098in}{2.680534in}}%
\pgfpathlineto{\pgfqpoint{2.436788in}{2.683518in}}%
\pgfpathlineto{\pgfqpoint{2.445462in}{2.686714in}}%
\pgfpathlineto{\pgfqpoint{2.454122in}{2.690118in}}%
\pgfpathlineto{\pgfqpoint{2.440194in}{2.712760in}}%
\pgfpathlineto{\pgfqpoint{2.426252in}{2.735748in}}%
\pgfpathlineto{\pgfqpoint{2.412296in}{2.759086in}}%
\pgfpathlineto{\pgfqpoint{2.398324in}{2.782778in}}%
\pgfpathlineto{\pgfqpoint{2.389624in}{2.779732in}}%
\pgfpathlineto{\pgfqpoint{2.380908in}{2.776903in}}%
\pgfpathlineto{\pgfqpoint{2.372177in}{2.774294in}}%
\pgfpathlineto{\pgfqpoint{2.363431in}{2.771909in}}%
\pgfpathclose%
\pgfusepath{fill}%
\end{pgfscope}%
\begin{pgfscope}%
\pgfpathrectangle{\pgfqpoint{1.150000in}{0.150000in}}{\pgfqpoint{5.700000in}{5.700000in}}%
\pgfusepath{clip}%
\pgfsetbuttcap%
\pgfsetroundjoin%
\definecolor{currentfill}{rgb}{0.139147,0.533812,0.555298}%
\pgfsetfillcolor{currentfill}%
\pgfsetfillopacity{0.800000}%
\pgfsetlinewidth{0.000000pt}%
\definecolor{currentstroke}{rgb}{0.000000,0.000000,0.000000}%
\pgfsetstrokecolor{currentstroke}%
\pgfsetdash{}{0pt}%
\pgfpathmoveto{\pgfqpoint{5.419302in}{3.131959in}}%
\pgfpathlineto{\pgfqpoint{5.433684in}{3.141670in}}%
\pgfpathlineto{\pgfqpoint{5.448085in}{3.151559in}}%
\pgfpathlineto{\pgfqpoint{5.462504in}{3.161624in}}%
\pgfpathlineto{\pgfqpoint{5.476941in}{3.171866in}}%
\pgfpathlineto{\pgfqpoint{5.484270in}{3.176024in}}%
\pgfpathlineto{\pgfqpoint{5.491594in}{3.180219in}}%
\pgfpathlineto{\pgfqpoint{5.498913in}{3.184458in}}%
\pgfpathlineto{\pgfqpoint{5.506227in}{3.188747in}}%
\pgfpathlineto{\pgfqpoint{5.491815in}{3.179047in}}%
\pgfpathlineto{\pgfqpoint{5.477421in}{3.169523in}}%
\pgfpathlineto{\pgfqpoint{5.463045in}{3.160175in}}%
\pgfpathlineto{\pgfqpoint{5.448686in}{3.151003in}}%
\pgfpathlineto{\pgfqpoint{5.441347in}{3.146162in}}%
\pgfpathlineto{\pgfqpoint{5.434004in}{3.141379in}}%
\pgfpathlineto{\pgfqpoint{5.426655in}{3.136646in}}%
\pgfpathlineto{\pgfqpoint{5.419302in}{3.131959in}}%
\pgfpathclose%
\pgfusepath{fill}%
\end{pgfscope}%
\begin{pgfscope}%
\pgfpathrectangle{\pgfqpoint{1.150000in}{0.150000in}}{\pgfqpoint{5.700000in}{5.700000in}}%
\pgfusepath{clip}%
\pgfsetbuttcap%
\pgfsetroundjoin%
\definecolor{currentfill}{rgb}{0.274128,0.199721,0.498911}%
\pgfsetfillcolor{currentfill}%
\pgfsetfillopacity{0.800000}%
\pgfsetlinewidth{0.000000pt}%
\definecolor{currentstroke}{rgb}{0.000000,0.000000,0.000000}%
\pgfsetstrokecolor{currentstroke}%
\pgfsetdash{}{0pt}%
\pgfpathmoveto{\pgfqpoint{2.662026in}{2.268795in}}%
\pgfpathlineto{\pgfqpoint{2.675799in}{2.251979in}}%
\pgfpathlineto{\pgfqpoint{2.689564in}{2.235448in}}%
\pgfpathlineto{\pgfqpoint{2.703321in}{2.219199in}}%
\pgfpathlineto{\pgfqpoint{2.717071in}{2.203230in}}%
\pgfpathlineto{\pgfqpoint{2.725605in}{2.207548in}}%
\pgfpathlineto{\pgfqpoint{2.734126in}{2.212043in}}%
\pgfpathlineto{\pgfqpoint{2.742636in}{2.216713in}}%
\pgfpathlineto{\pgfqpoint{2.751134in}{2.221553in}}%
\pgfpathlineto{\pgfqpoint{2.737417in}{2.237136in}}%
\pgfpathlineto{\pgfqpoint{2.723693in}{2.252999in}}%
\pgfpathlineto{\pgfqpoint{2.709962in}{2.269143in}}%
\pgfpathlineto{\pgfqpoint{2.696223in}{2.285571in}}%
\pgfpathlineto{\pgfqpoint{2.687692in}{2.281105in}}%
\pgfpathlineto{\pgfqpoint{2.679149in}{2.276818in}}%
\pgfpathlineto{\pgfqpoint{2.670594in}{2.272714in}}%
\pgfpathlineto{\pgfqpoint{2.662026in}{2.268795in}}%
\pgfpathclose%
\pgfusepath{fill}%
\end{pgfscope}%
\begin{pgfscope}%
\pgfpathrectangle{\pgfqpoint{1.150000in}{0.150000in}}{\pgfqpoint{5.700000in}{5.700000in}}%
\pgfusepath{clip}%
\pgfsetbuttcap%
\pgfsetroundjoin%
\definecolor{currentfill}{rgb}{0.278826,0.175490,0.483397}%
\pgfsetfillcolor{currentfill}%
\pgfsetfillopacity{0.800000}%
\pgfsetlinewidth{0.000000pt}%
\definecolor{currentstroke}{rgb}{0.000000,0.000000,0.000000}%
\pgfsetstrokecolor{currentstroke}%
\pgfsetdash{}{0pt}%
\pgfpathmoveto{\pgfqpoint{2.717071in}{2.203230in}}%
\pgfpathlineto{\pgfqpoint{2.730814in}{2.187539in}}%
\pgfpathlineto{\pgfqpoint{2.744550in}{2.172124in}}%
\pgfpathlineto{\pgfqpoint{2.758279in}{2.156982in}}%
\pgfpathlineto{\pgfqpoint{2.772002in}{2.142111in}}%
\pgfpathlineto{\pgfqpoint{2.780502in}{2.146826in}}%
\pgfpathlineto{\pgfqpoint{2.788992in}{2.151710in}}%
\pgfpathlineto{\pgfqpoint{2.797469in}{2.156760in}}%
\pgfpathlineto{\pgfqpoint{2.805936in}{2.161972in}}%
\pgfpathlineto{\pgfqpoint{2.792244in}{2.176459in}}%
\pgfpathlineto{\pgfqpoint{2.778547in}{2.191216in}}%
\pgfpathlineto{\pgfqpoint{2.764844in}{2.206247in}}%
\pgfpathlineto{\pgfqpoint{2.751134in}{2.221553in}}%
\pgfpathlineto{\pgfqpoint{2.742636in}{2.216713in}}%
\pgfpathlineto{\pgfqpoint{2.734126in}{2.212043in}}%
\pgfpathlineto{\pgfqpoint{2.725605in}{2.207548in}}%
\pgfpathlineto{\pgfqpoint{2.717071in}{2.203230in}}%
\pgfpathclose%
\pgfusepath{fill}%
\end{pgfscope}%
\begin{pgfscope}%
\pgfpathrectangle{\pgfqpoint{1.150000in}{0.150000in}}{\pgfqpoint{5.700000in}{5.700000in}}%
\pgfusepath{clip}%
\pgfsetbuttcap%
\pgfsetroundjoin%
\definecolor{currentfill}{rgb}{0.274952,0.037752,0.364543}%
\pgfsetfillcolor{currentfill}%
\pgfsetfillopacity{0.800000}%
\pgfsetlinewidth{0.000000pt}%
\definecolor{currentstroke}{rgb}{0.000000,0.000000,0.000000}%
\pgfsetstrokecolor{currentstroke}%
\pgfsetdash{}{0pt}%
\pgfpathmoveto{\pgfqpoint{3.220886in}{1.890888in}}%
\pgfpathlineto{\pgfqpoint{3.234485in}{1.883833in}}%
\pgfpathlineto{\pgfqpoint{3.248084in}{1.877000in}}%
\pgfpathlineto{\pgfqpoint{3.261685in}{1.870389in}}%
\pgfpathlineto{\pgfqpoint{3.275288in}{1.863998in}}%
\pgfpathlineto{\pgfqpoint{3.283515in}{1.872565in}}%
\pgfpathlineto{\pgfqpoint{3.291735in}{1.881211in}}%
\pgfpathlineto{\pgfqpoint{3.299947in}{1.889932in}}%
\pgfpathlineto{\pgfqpoint{3.308153in}{1.898727in}}%
\pgfpathlineto{\pgfqpoint{3.294568in}{1.904809in}}%
\pgfpathlineto{\pgfqpoint{3.280986in}{1.911111in}}%
\pgfpathlineto{\pgfqpoint{3.267404in}{1.917634in}}%
\pgfpathlineto{\pgfqpoint{3.253825in}{1.924381in}}%
\pgfpathlineto{\pgfqpoint{3.245601in}{1.915883in}}%
\pgfpathlineto{\pgfqpoint{3.237370in}{1.907467in}}%
\pgfpathlineto{\pgfqpoint{3.229132in}{1.899134in}}%
\pgfpathlineto{\pgfqpoint{3.220886in}{1.890888in}}%
\pgfpathclose%
\pgfusepath{fill}%
\end{pgfscope}%
\begin{pgfscope}%
\pgfpathrectangle{\pgfqpoint{1.150000in}{0.150000in}}{\pgfqpoint{5.700000in}{5.700000in}}%
\pgfusepath{clip}%
\pgfsetbuttcap%
\pgfsetroundjoin%
\definecolor{currentfill}{rgb}{0.248629,0.278775,0.534556}%
\pgfsetfillcolor{currentfill}%
\pgfsetfillopacity{0.800000}%
\pgfsetlinewidth{0.000000pt}%
\definecolor{currentstroke}{rgb}{0.000000,0.000000,0.000000}%
\pgfsetstrokecolor{currentstroke}%
\pgfsetdash{}{0pt}%
\pgfpathmoveto{\pgfqpoint{4.402757in}{2.404685in}}%
\pgfpathlineto{\pgfqpoint{4.416630in}{2.410404in}}%
\pgfpathlineto{\pgfqpoint{4.430516in}{2.416310in}}%
\pgfpathlineto{\pgfqpoint{4.444414in}{2.422402in}}%
\pgfpathlineto{\pgfqpoint{4.458325in}{2.428680in}}%
\pgfpathlineto{\pgfqpoint{4.466135in}{2.438175in}}%
\pgfpathlineto{\pgfqpoint{4.473939in}{2.447603in}}%
\pgfpathlineto{\pgfqpoint{4.481737in}{2.456964in}}%
\pgfpathlineto{\pgfqpoint{4.489529in}{2.466260in}}%
\pgfpathlineto{\pgfqpoint{4.475624in}{2.460090in}}%
\pgfpathlineto{\pgfqpoint{4.461733in}{2.454105in}}%
\pgfpathlineto{\pgfqpoint{4.447854in}{2.448307in}}%
\pgfpathlineto{\pgfqpoint{4.433988in}{2.442694in}}%
\pgfpathlineto{\pgfqpoint{4.426189in}{2.433279in}}%
\pgfpathlineto{\pgfqpoint{4.418384in}{2.423806in}}%
\pgfpathlineto{\pgfqpoint{4.410573in}{2.414276in}}%
\pgfpathlineto{\pgfqpoint{4.402757in}{2.404685in}}%
\pgfpathclose%
\pgfusepath{fill}%
\end{pgfscope}%
\begin{pgfscope}%
\pgfpathrectangle{\pgfqpoint{1.150000in}{0.150000in}}{\pgfqpoint{5.700000in}{5.700000in}}%
\pgfusepath{clip}%
\pgfsetbuttcap%
\pgfsetroundjoin%
\definecolor{currentfill}{rgb}{0.266580,0.228262,0.514349}%
\pgfsetfillcolor{currentfill}%
\pgfsetfillopacity{0.800000}%
\pgfsetlinewidth{0.000000pt}%
\definecolor{currentstroke}{rgb}{0.000000,0.000000,0.000000}%
\pgfsetstrokecolor{currentstroke}%
\pgfsetdash{}{0pt}%
\pgfpathmoveto{\pgfqpoint{2.606847in}{2.338954in}}%
\pgfpathlineto{\pgfqpoint{2.620655in}{2.320975in}}%
\pgfpathlineto{\pgfqpoint{2.634454in}{2.303290in}}%
\pgfpathlineto{\pgfqpoint{2.648244in}{2.285898in}}%
\pgfpathlineto{\pgfqpoint{2.662026in}{2.268795in}}%
\pgfpathlineto{\pgfqpoint{2.670594in}{2.272714in}}%
\pgfpathlineto{\pgfqpoint{2.679149in}{2.276818in}}%
\pgfpathlineto{\pgfqpoint{2.687692in}{2.281105in}}%
\pgfpathlineto{\pgfqpoint{2.696223in}{2.285571in}}%
\pgfpathlineto{\pgfqpoint{2.682476in}{2.302286in}}%
\pgfpathlineto{\pgfqpoint{2.668722in}{2.319290in}}%
\pgfpathlineto{\pgfqpoint{2.654958in}{2.336584in}}%
\pgfpathlineto{\pgfqpoint{2.641186in}{2.354173in}}%
\pgfpathlineto{\pgfqpoint{2.632621in}{2.350084in}}%
\pgfpathlineto{\pgfqpoint{2.624043in}{2.346182in}}%
\pgfpathlineto{\pgfqpoint{2.615452in}{2.342471in}}%
\pgfpathlineto{\pgfqpoint{2.606847in}{2.338954in}}%
\pgfpathclose%
\pgfusepath{fill}%
\end{pgfscope}%
\begin{pgfscope}%
\pgfpathrectangle{\pgfqpoint{1.150000in}{0.150000in}}{\pgfqpoint{5.700000in}{5.700000in}}%
\pgfusepath{clip}%
\pgfsetbuttcap%
\pgfsetroundjoin%
\definecolor{currentfill}{rgb}{0.190631,0.407061,0.556089}%
\pgfsetfillcolor{currentfill}%
\pgfsetfillopacity{0.800000}%
\pgfsetlinewidth{0.000000pt}%
\definecolor{currentstroke}{rgb}{0.000000,0.000000,0.000000}%
\pgfsetstrokecolor{currentstroke}%
\pgfsetdash{}{0pt}%
\pgfpathmoveto{\pgfqpoint{4.867781in}{2.749214in}}%
\pgfpathlineto{\pgfqpoint{4.881881in}{2.757475in}}%
\pgfpathlineto{\pgfqpoint{4.895997in}{2.765917in}}%
\pgfpathlineto{\pgfqpoint{4.910129in}{2.774541in}}%
\pgfpathlineto{\pgfqpoint{4.924277in}{2.783346in}}%
\pgfpathlineto{\pgfqpoint{4.931891in}{2.790351in}}%
\pgfpathlineto{\pgfqpoint{4.939498in}{2.797309in}}%
\pgfpathlineto{\pgfqpoint{4.947099in}{2.804222in}}%
\pgfpathlineto{\pgfqpoint{4.954694in}{2.811094in}}%
\pgfpathlineto{\pgfqpoint{4.940559in}{2.802596in}}%
\pgfpathlineto{\pgfqpoint{4.926439in}{2.794279in}}%
\pgfpathlineto{\pgfqpoint{4.912336in}{2.786143in}}%
\pgfpathlineto{\pgfqpoint{4.898247in}{2.778187in}}%
\pgfpathlineto{\pgfqpoint{4.890640in}{2.770997in}}%
\pgfpathlineto{\pgfqpoint{4.883027in}{2.763773in}}%
\pgfpathlineto{\pgfqpoint{4.875407in}{2.756514in}}%
\pgfpathlineto{\pgfqpoint{4.867781in}{2.749214in}}%
\pgfpathclose%
\pgfusepath{fill}%
\end{pgfscope}%
\begin{pgfscope}%
\pgfpathrectangle{\pgfqpoint{1.150000in}{0.150000in}}{\pgfqpoint{5.700000in}{5.700000in}}%
\pgfusepath{clip}%
\pgfsetbuttcap%
\pgfsetroundjoin%
\definecolor{currentfill}{rgb}{0.282290,0.145912,0.461510}%
\pgfsetfillcolor{currentfill}%
\pgfsetfillopacity{0.800000}%
\pgfsetlinewidth{0.000000pt}%
\definecolor{currentstroke}{rgb}{0.000000,0.000000,0.000000}%
\pgfsetstrokecolor{currentstroke}%
\pgfsetdash{}{0pt}%
\pgfpathmoveto{\pgfqpoint{2.772002in}{2.142111in}}%
\pgfpathlineto{\pgfqpoint{2.785718in}{2.127510in}}%
\pgfpathlineto{\pgfqpoint{2.799429in}{2.113176in}}%
\pgfpathlineto{\pgfqpoint{2.813134in}{2.099108in}}%
\pgfpathlineto{\pgfqpoint{2.826834in}{2.085302in}}%
\pgfpathlineto{\pgfqpoint{2.835304in}{2.090411in}}%
\pgfpathlineto{\pgfqpoint{2.843762in}{2.095682in}}%
\pgfpathlineto{\pgfqpoint{2.852209in}{2.101109in}}%
\pgfpathlineto{\pgfqpoint{2.860645in}{2.106691in}}%
\pgfpathlineto{\pgfqpoint{2.846976in}{2.120115in}}%
\pgfpathlineto{\pgfqpoint{2.833301in}{2.133802in}}%
\pgfpathlineto{\pgfqpoint{2.819621in}{2.147753in}}%
\pgfpathlineto{\pgfqpoint{2.805936in}{2.161972in}}%
\pgfpathlineto{\pgfqpoint{2.797469in}{2.156760in}}%
\pgfpathlineto{\pgfqpoint{2.788992in}{2.151710in}}%
\pgfpathlineto{\pgfqpoint{2.780502in}{2.146826in}}%
\pgfpathlineto{\pgfqpoint{2.772002in}{2.142111in}}%
\pgfpathclose%
\pgfusepath{fill}%
\end{pgfscope}%
\begin{pgfscope}%
\pgfpathrectangle{\pgfqpoint{1.150000in}{0.150000in}}{\pgfqpoint{5.700000in}{5.700000in}}%
\pgfusepath{clip}%
\pgfsetbuttcap%
\pgfsetroundjoin%
\definecolor{currentfill}{rgb}{0.132444,0.552216,0.553018}%
\pgfsetfillcolor{currentfill}%
\pgfsetfillopacity{0.800000}%
\pgfsetlinewidth{0.000000pt}%
\definecolor{currentstroke}{rgb}{0.000000,0.000000,0.000000}%
\pgfsetstrokecolor{currentstroke}%
\pgfsetdash{}{0pt}%
\pgfpathmoveto{\pgfqpoint{5.506227in}{3.188747in}}%
\pgfpathlineto{\pgfqpoint{5.520658in}{3.198623in}}%
\pgfpathlineto{\pgfqpoint{5.535107in}{3.208675in}}%
\pgfpathlineto{\pgfqpoint{5.549574in}{3.218904in}}%
\pgfpathlineto{\pgfqpoint{5.564061in}{3.229309in}}%
\pgfpathlineto{\pgfqpoint{5.571344in}{3.233089in}}%
\pgfpathlineto{\pgfqpoint{5.578622in}{3.236925in}}%
\pgfpathlineto{\pgfqpoint{5.585895in}{3.240821in}}%
\pgfpathlineto{\pgfqpoint{5.593164in}{3.244784in}}%
\pgfpathlineto{\pgfqpoint{5.578705in}{3.234955in}}%
\pgfpathlineto{\pgfqpoint{5.564264in}{3.225302in}}%
\pgfpathlineto{\pgfqpoint{5.549842in}{3.215824in}}%
\pgfpathlineto{\pgfqpoint{5.535438in}{3.206521in}}%
\pgfpathlineto{\pgfqpoint{5.528141in}{3.201972in}}%
\pgfpathlineto{\pgfqpoint{5.520841in}{3.197498in}}%
\pgfpathlineto{\pgfqpoint{5.513536in}{3.193091in}}%
\pgfpathlineto{\pgfqpoint{5.506227in}{3.188747in}}%
\pgfpathclose%
\pgfusepath{fill}%
\end{pgfscope}%
\begin{pgfscope}%
\pgfpathrectangle{\pgfqpoint{1.150000in}{0.150000in}}{\pgfqpoint{5.700000in}{5.700000in}}%
\pgfusepath{clip}%
\pgfsetbuttcap%
\pgfsetroundjoin%
\definecolor{currentfill}{rgb}{0.273809,0.031497,0.358853}%
\pgfsetfillcolor{currentfill}%
\pgfsetfillopacity{0.800000}%
\pgfsetlinewidth{0.000000pt}%
\definecolor{currentstroke}{rgb}{0.000000,0.000000,0.000000}%
\pgfsetstrokecolor{currentstroke}%
\pgfsetdash{}{0pt}%
\pgfpathmoveto{\pgfqpoint{3.362511in}{1.876581in}}%
\pgfpathlineto{\pgfqpoint{3.376107in}{1.871585in}}%
\pgfpathlineto{\pgfqpoint{3.389705in}{1.866802in}}%
\pgfpathlineto{\pgfqpoint{3.403307in}{1.862233in}}%
\pgfpathlineto{\pgfqpoint{3.416911in}{1.857876in}}%
\pgfpathlineto{\pgfqpoint{3.425078in}{1.867318in}}%
\pgfpathlineto{\pgfqpoint{3.433238in}{1.876812in}}%
\pgfpathlineto{\pgfqpoint{3.441392in}{1.886356in}}%
\pgfpathlineto{\pgfqpoint{3.449539in}{1.895947in}}%
\pgfpathlineto{\pgfqpoint{3.435949in}{1.900028in}}%
\pgfpathlineto{\pgfqpoint{3.422363in}{1.904321in}}%
\pgfpathlineto{\pgfqpoint{3.408779in}{1.908827in}}%
\pgfpathlineto{\pgfqpoint{3.395199in}{1.913546in}}%
\pgfpathlineto{\pgfqpoint{3.387037in}{1.904220in}}%
\pgfpathlineto{\pgfqpoint{3.378868in}{1.894949in}}%
\pgfpathlineto{\pgfqpoint{3.370693in}{1.885735in}}%
\pgfpathlineto{\pgfqpoint{3.362511in}{1.876581in}}%
\pgfpathclose%
\pgfusepath{fill}%
\end{pgfscope}%
\begin{pgfscope}%
\pgfpathrectangle{\pgfqpoint{1.150000in}{0.150000in}}{\pgfqpoint{5.700000in}{5.700000in}}%
\pgfusepath{clip}%
\pgfsetbuttcap%
\pgfsetroundjoin%
\definecolor{currentfill}{rgb}{0.282327,0.094955,0.417331}%
\pgfsetfillcolor{currentfill}%
\pgfsetfillopacity{0.800000}%
\pgfsetlinewidth{0.000000pt}%
\definecolor{currentstroke}{rgb}{0.000000,0.000000,0.000000}%
\pgfsetstrokecolor{currentstroke}%
\pgfsetdash{}{0pt}%
\pgfpathmoveto{\pgfqpoint{3.764359in}{1.987059in}}%
\pgfpathlineto{\pgfqpoint{3.778006in}{1.987096in}}%
\pgfpathlineto{\pgfqpoint{3.791661in}{1.987332in}}%
\pgfpathlineto{\pgfqpoint{3.805323in}{1.987766in}}%
\pgfpathlineto{\pgfqpoint{3.818993in}{1.988397in}}%
\pgfpathlineto{\pgfqpoint{3.827015in}{1.999234in}}%
\pgfpathlineto{\pgfqpoint{3.835031in}{2.010054in}}%
\pgfpathlineto{\pgfqpoint{3.843043in}{2.020856in}}%
\pgfpathlineto{\pgfqpoint{3.851049in}{2.031639in}}%
\pgfpathlineto{\pgfqpoint{3.837387in}{2.030858in}}%
\pgfpathlineto{\pgfqpoint{3.823733in}{2.030274in}}%
\pgfpathlineto{\pgfqpoint{3.810087in}{2.029888in}}%
\pgfpathlineto{\pgfqpoint{3.796447in}{2.029700in}}%
\pgfpathlineto{\pgfqpoint{3.788433in}{2.019055in}}%
\pgfpathlineto{\pgfqpoint{3.780413in}{2.008399in}}%
\pgfpathlineto{\pgfqpoint{3.772388in}{1.997733in}}%
\pgfpathlineto{\pgfqpoint{3.764359in}{1.987059in}}%
\pgfpathclose%
\pgfusepath{fill}%
\end{pgfscope}%
\begin{pgfscope}%
\pgfpathrectangle{\pgfqpoint{1.150000in}{0.150000in}}{\pgfqpoint{5.700000in}{5.700000in}}%
\pgfusepath{clip}%
\pgfsetbuttcap%
\pgfsetroundjoin%
\definecolor{currentfill}{rgb}{0.283229,0.120777,0.440584}%
\pgfsetfillcolor{currentfill}%
\pgfsetfillopacity{0.800000}%
\pgfsetlinewidth{0.000000pt}%
\definecolor{currentstroke}{rgb}{0.000000,0.000000,0.000000}%
\pgfsetstrokecolor{currentstroke}%
\pgfsetdash{}{0pt}%
\pgfpathmoveto{\pgfqpoint{3.851049in}{2.031639in}}%
\pgfpathlineto{\pgfqpoint{3.864719in}{2.032617in}}%
\pgfpathlineto{\pgfqpoint{3.878397in}{2.033790in}}%
\pgfpathlineto{\pgfqpoint{3.892083in}{2.035160in}}%
\pgfpathlineto{\pgfqpoint{3.905777in}{2.036724in}}%
\pgfpathlineto{\pgfqpoint{3.913771in}{2.047617in}}%
\pgfpathlineto{\pgfqpoint{3.921760in}{2.058482in}}%
\pgfpathlineto{\pgfqpoint{3.929744in}{2.069318in}}%
\pgfpathlineto{\pgfqpoint{3.937723in}{2.080123in}}%
\pgfpathlineto{\pgfqpoint{3.924036in}{2.078440in}}%
\pgfpathlineto{\pgfqpoint{3.910357in}{2.076952in}}%
\pgfpathlineto{\pgfqpoint{3.896687in}{2.075660in}}%
\pgfpathlineto{\pgfqpoint{3.883024in}{2.074564in}}%
\pgfpathlineto{\pgfqpoint{3.875038in}{2.063865in}}%
\pgfpathlineto{\pgfqpoint{3.867047in}{2.053145in}}%
\pgfpathlineto{\pgfqpoint{3.859051in}{2.042402in}}%
\pgfpathlineto{\pgfqpoint{3.851049in}{2.031639in}}%
\pgfpathclose%
\pgfusepath{fill}%
\end{pgfscope}%
\begin{pgfscope}%
\pgfpathrectangle{\pgfqpoint{1.150000in}{0.150000in}}{\pgfqpoint{5.700000in}{5.700000in}}%
\pgfusepath{clip}%
\pgfsetbuttcap%
\pgfsetroundjoin%
\definecolor{currentfill}{rgb}{0.255645,0.260703,0.528312}%
\pgfsetfillcolor{currentfill}%
\pgfsetfillopacity{0.800000}%
\pgfsetlinewidth{0.000000pt}%
\definecolor{currentstroke}{rgb}{0.000000,0.000000,0.000000}%
\pgfsetstrokecolor{currentstroke}%
\pgfsetdash{}{0pt}%
\pgfpathmoveto{\pgfqpoint{2.551518in}{2.413866in}}%
\pgfpathlineto{\pgfqpoint{2.565366in}{2.394684in}}%
\pgfpathlineto{\pgfqpoint{2.579203in}{2.375806in}}%
\pgfpathlineto{\pgfqpoint{2.593030in}{2.357230in}}%
\pgfpathlineto{\pgfqpoint{2.606847in}{2.338954in}}%
\pgfpathlineto{\pgfqpoint{2.615452in}{2.342471in}}%
\pgfpathlineto{\pgfqpoint{2.624043in}{2.346182in}}%
\pgfpathlineto{\pgfqpoint{2.632621in}{2.350084in}}%
\pgfpathlineto{\pgfqpoint{2.641186in}{2.354173in}}%
\pgfpathlineto{\pgfqpoint{2.627405in}{2.372058in}}%
\pgfpathlineto{\pgfqpoint{2.613615in}{2.390242in}}%
\pgfpathlineto{\pgfqpoint{2.599815in}{2.408727in}}%
\pgfpathlineto{\pgfqpoint{2.586005in}{2.427517in}}%
\pgfpathlineto{\pgfqpoint{2.577403in}{2.423807in}}%
\pgfpathlineto{\pgfqpoint{2.568789in}{2.420293in}}%
\pgfpathlineto{\pgfqpoint{2.560161in}{2.416978in}}%
\pgfpathlineto{\pgfqpoint{2.551518in}{2.413866in}}%
\pgfpathclose%
\pgfusepath{fill}%
\end{pgfscope}%
\begin{pgfscope}%
\pgfpathrectangle{\pgfqpoint{1.150000in}{0.150000in}}{\pgfqpoint{5.700000in}{5.700000in}}%
\pgfusepath{clip}%
\pgfsetbuttcap%
\pgfsetroundjoin%
\definecolor{currentfill}{rgb}{0.277941,0.056324,0.381191}%
\pgfsetfillcolor{currentfill}%
\pgfsetfillopacity{0.800000}%
\pgfsetlinewidth{0.000000pt}%
\definecolor{currentstroke}{rgb}{0.000000,0.000000,0.000000}%
\pgfsetstrokecolor{currentstroke}%
\pgfsetdash{}{0pt}%
\pgfpathmoveto{\pgfqpoint{3.078874in}{1.926172in}}%
\pgfpathlineto{\pgfqpoint{3.092495in}{1.916948in}}%
\pgfpathlineto{\pgfqpoint{3.106114in}{1.907957in}}%
\pgfpathlineto{\pgfqpoint{3.119733in}{1.899197in}}%
\pgfpathlineto{\pgfqpoint{3.133352in}{1.890668in}}%
\pgfpathlineto{\pgfqpoint{3.141650in}{1.898184in}}%
\pgfpathlineto{\pgfqpoint{3.149941in}{1.905807in}}%
\pgfpathlineto{\pgfqpoint{3.158223in}{1.913534in}}%
\pgfpathlineto{\pgfqpoint{3.166497in}{1.921361in}}%
\pgfpathlineto{\pgfqpoint{3.152900in}{1.929548in}}%
\pgfpathlineto{\pgfqpoint{3.139304in}{1.937965in}}%
\pgfpathlineto{\pgfqpoint{3.125706in}{1.946614in}}%
\pgfpathlineto{\pgfqpoint{3.112109in}{1.955496in}}%
\pgfpathlineto{\pgfqpoint{3.103813in}{1.947999in}}%
\pgfpathlineto{\pgfqpoint{3.095509in}{1.940611in}}%
\pgfpathlineto{\pgfqpoint{3.087196in}{1.933334in}}%
\pgfpathlineto{\pgfqpoint{3.078874in}{1.926172in}}%
\pgfpathclose%
\pgfusepath{fill}%
\end{pgfscope}%
\begin{pgfscope}%
\pgfpathrectangle{\pgfqpoint{1.150000in}{0.150000in}}{\pgfqpoint{5.700000in}{5.700000in}}%
\pgfusepath{clip}%
\pgfsetbuttcap%
\pgfsetroundjoin%
\definecolor{currentfill}{rgb}{0.282623,0.140926,0.457517}%
\pgfsetfillcolor{currentfill}%
\pgfsetfillopacity{0.800000}%
\pgfsetlinewidth{0.000000pt}%
\definecolor{currentstroke}{rgb}{0.000000,0.000000,0.000000}%
\pgfsetstrokecolor{currentstroke}%
\pgfsetdash{}{0pt}%
\pgfpathmoveto{\pgfqpoint{3.937723in}{2.080123in}}%
\pgfpathlineto{\pgfqpoint{3.951419in}{2.082000in}}%
\pgfpathlineto{\pgfqpoint{3.965124in}{2.084071in}}%
\pgfpathlineto{\pgfqpoint{3.978838in}{2.086336in}}%
\pgfpathlineto{\pgfqpoint{3.992562in}{2.088794in}}%
\pgfpathlineto{\pgfqpoint{4.000529in}{2.099667in}}%
\pgfpathlineto{\pgfqpoint{4.008491in}{2.110500in}}%
\pgfpathlineto{\pgfqpoint{4.016448in}{2.121294in}}%
\pgfpathlineto{\pgfqpoint{4.024401in}{2.132047in}}%
\pgfpathlineto{\pgfqpoint{4.010684in}{2.129502in}}%
\pgfpathlineto{\pgfqpoint{3.996976in}{2.127151in}}%
\pgfpathlineto{\pgfqpoint{3.983278in}{2.124993in}}%
\pgfpathlineto{\pgfqpoint{3.969589in}{2.123029in}}%
\pgfpathlineto{\pgfqpoint{3.961630in}{2.112350in}}%
\pgfpathlineto{\pgfqpoint{3.953666in}{2.101639in}}%
\pgfpathlineto{\pgfqpoint{3.945697in}{2.090897in}}%
\pgfpathlineto{\pgfqpoint{3.937723in}{2.080123in}}%
\pgfpathclose%
\pgfusepath{fill}%
\end{pgfscope}%
\begin{pgfscope}%
\pgfpathrectangle{\pgfqpoint{1.150000in}{0.150000in}}{\pgfqpoint{5.700000in}{5.700000in}}%
\pgfusepath{clip}%
\pgfsetbuttcap%
\pgfsetroundjoin%
\definecolor{currentfill}{rgb}{0.280894,0.078907,0.402329}%
\pgfsetfillcolor{currentfill}%
\pgfsetfillopacity{0.800000}%
\pgfsetlinewidth{0.000000pt}%
\definecolor{currentstroke}{rgb}{0.000000,0.000000,0.000000}%
\pgfsetstrokecolor{currentstroke}%
\pgfsetdash{}{0pt}%
\pgfpathmoveto{\pgfqpoint{3.677628in}{1.946868in}}%
\pgfpathlineto{\pgfqpoint{3.691258in}{1.945925in}}%
\pgfpathlineto{\pgfqpoint{3.704895in}{1.945182in}}%
\pgfpathlineto{\pgfqpoint{3.718538in}{1.944640in}}%
\pgfpathlineto{\pgfqpoint{3.732187in}{1.944297in}}%
\pgfpathlineto{\pgfqpoint{3.740238in}{1.954994in}}%
\pgfpathlineto{\pgfqpoint{3.748283in}{1.965688in}}%
\pgfpathlineto{\pgfqpoint{3.756323in}{1.976376in}}%
\pgfpathlineto{\pgfqpoint{3.764359in}{1.987059in}}%
\pgfpathlineto{\pgfqpoint{3.750718in}{1.987220in}}%
\pgfpathlineto{\pgfqpoint{3.737084in}{1.987581in}}%
\pgfpathlineto{\pgfqpoint{3.723457in}{1.988142in}}%
\pgfpathlineto{\pgfqpoint{3.709837in}{1.988904in}}%
\pgfpathlineto{\pgfqpoint{3.701792in}{1.978391in}}%
\pgfpathlineto{\pgfqpoint{3.693743in}{1.967880in}}%
\pgfpathlineto{\pgfqpoint{3.685688in}{1.957372in}}%
\pgfpathlineto{\pgfqpoint{3.677628in}{1.946868in}}%
\pgfpathclose%
\pgfusepath{fill}%
\end{pgfscope}%
\begin{pgfscope}%
\pgfpathrectangle{\pgfqpoint{1.150000in}{0.150000in}}{\pgfqpoint{5.700000in}{5.700000in}}%
\pgfusepath{clip}%
\pgfsetbuttcap%
\pgfsetroundjoin%
\definecolor{currentfill}{rgb}{0.126453,0.570633,0.549841}%
\pgfsetfillcolor{currentfill}%
\pgfsetfillopacity{0.800000}%
\pgfsetlinewidth{0.000000pt}%
\definecolor{currentstroke}{rgb}{0.000000,0.000000,0.000000}%
\pgfsetstrokecolor{currentstroke}%
\pgfsetdash{}{0pt}%
\pgfpathmoveto{\pgfqpoint{5.593164in}{3.244784in}}%
\pgfpathlineto{\pgfqpoint{5.607642in}{3.254788in}}%
\pgfpathlineto{\pgfqpoint{5.622139in}{3.264967in}}%
\pgfpathlineto{\pgfqpoint{5.636654in}{3.275323in}}%
\pgfpathlineto{\pgfqpoint{5.651189in}{3.285854in}}%
\pgfpathlineto{\pgfqpoint{5.658425in}{3.289293in}}%
\pgfpathlineto{\pgfqpoint{5.665657in}{3.292804in}}%
\pgfpathlineto{\pgfqpoint{5.672885in}{3.296395in}}%
\pgfpathlineto{\pgfqpoint{5.680109in}{3.300072in}}%
\pgfpathlineto{\pgfqpoint{5.665604in}{3.290151in}}%
\pgfpathlineto{\pgfqpoint{5.651118in}{3.280404in}}%
\pgfpathlineto{\pgfqpoint{5.636650in}{3.270832in}}%
\pgfpathlineto{\pgfqpoint{5.622201in}{3.261435in}}%
\pgfpathlineto{\pgfqpoint{5.614947in}{3.257139in}}%
\pgfpathlineto{\pgfqpoint{5.607690in}{3.252936in}}%
\pgfpathlineto{\pgfqpoint{5.600429in}{3.248820in}}%
\pgfpathlineto{\pgfqpoint{5.593164in}{3.244784in}}%
\pgfpathclose%
\pgfusepath{fill}%
\end{pgfscope}%
\begin{pgfscope}%
\pgfpathrectangle{\pgfqpoint{1.150000in}{0.150000in}}{\pgfqpoint{5.700000in}{5.700000in}}%
\pgfusepath{clip}%
\pgfsetbuttcap%
\pgfsetroundjoin%
\definecolor{currentfill}{rgb}{0.283187,0.125848,0.444960}%
\pgfsetfillcolor{currentfill}%
\pgfsetfillopacity{0.800000}%
\pgfsetlinewidth{0.000000pt}%
\definecolor{currentstroke}{rgb}{0.000000,0.000000,0.000000}%
\pgfsetstrokecolor{currentstroke}%
\pgfsetdash{}{0pt}%
\pgfpathmoveto{\pgfqpoint{2.826834in}{2.085302in}}%
\pgfpathlineto{\pgfqpoint{2.840529in}{2.071758in}}%
\pgfpathlineto{\pgfqpoint{2.854219in}{2.058473in}}%
\pgfpathlineto{\pgfqpoint{2.867904in}{2.045446in}}%
\pgfpathlineto{\pgfqpoint{2.881585in}{2.032674in}}%
\pgfpathlineto{\pgfqpoint{2.890024in}{2.038176in}}%
\pgfpathlineto{\pgfqpoint{2.898453in}{2.043831in}}%
\pgfpathlineto{\pgfqpoint{2.906871in}{2.049634in}}%
\pgfpathlineto{\pgfqpoint{2.915279in}{2.055584in}}%
\pgfpathlineto{\pgfqpoint{2.901627in}{2.067976in}}%
\pgfpathlineto{\pgfqpoint{2.887971in}{2.080623in}}%
\pgfpathlineto{\pgfqpoint{2.874310in}{2.093527in}}%
\pgfpathlineto{\pgfqpoint{2.860645in}{2.106691in}}%
\pgfpathlineto{\pgfqpoint{2.852209in}{2.101109in}}%
\pgfpathlineto{\pgfqpoint{2.843762in}{2.095682in}}%
\pgfpathlineto{\pgfqpoint{2.835304in}{2.090411in}}%
\pgfpathlineto{\pgfqpoint{2.826834in}{2.085302in}}%
\pgfpathclose%
\pgfusepath{fill}%
\end{pgfscope}%
\begin{pgfscope}%
\pgfpathrectangle{\pgfqpoint{1.150000in}{0.150000in}}{\pgfqpoint{5.700000in}{5.700000in}}%
\pgfusepath{clip}%
\pgfsetbuttcap%
\pgfsetroundjoin%
\definecolor{currentfill}{rgb}{0.237441,0.305202,0.541921}%
\pgfsetfillcolor{currentfill}%
\pgfsetfillopacity{0.800000}%
\pgfsetlinewidth{0.000000pt}%
\definecolor{currentstroke}{rgb}{0.000000,0.000000,0.000000}%
\pgfsetstrokecolor{currentstroke}%
\pgfsetdash{}{0pt}%
\pgfpathmoveto{\pgfqpoint{4.489529in}{2.466260in}}%
\pgfpathlineto{\pgfqpoint{4.503447in}{2.472617in}}%
\pgfpathlineto{\pgfqpoint{4.517377in}{2.479158in}}%
\pgfpathlineto{\pgfqpoint{4.531322in}{2.485885in}}%
\pgfpathlineto{\pgfqpoint{4.545279in}{2.492797in}}%
\pgfpathlineto{\pgfqpoint{4.553059in}{2.501903in}}%
\pgfpathlineto{\pgfqpoint{4.560832in}{2.510939in}}%
\pgfpathlineto{\pgfqpoint{4.568599in}{2.519908in}}%
\pgfpathlineto{\pgfqpoint{4.576360in}{2.528811in}}%
\pgfpathlineto{\pgfqpoint{4.562410in}{2.522040in}}%
\pgfpathlineto{\pgfqpoint{4.548473in}{2.515453in}}%
\pgfpathlineto{\pgfqpoint{4.534550in}{2.509051in}}%
\pgfpathlineto{\pgfqpoint{4.520639in}{2.502835in}}%
\pgfpathlineto{\pgfqpoint{4.512871in}{2.493779in}}%
\pgfpathlineto{\pgfqpoint{4.505096in}{2.484666in}}%
\pgfpathlineto{\pgfqpoint{4.497315in}{2.475494in}}%
\pgfpathlineto{\pgfqpoint{4.489529in}{2.466260in}}%
\pgfpathclose%
\pgfusepath{fill}%
\end{pgfscope}%
\begin{pgfscope}%
\pgfpathrectangle{\pgfqpoint{1.150000in}{0.150000in}}{\pgfqpoint{5.700000in}{5.700000in}}%
\pgfusepath{clip}%
\pgfsetbuttcap%
\pgfsetroundjoin%
\definecolor{currentfill}{rgb}{0.180629,0.429975,0.557282}%
\pgfsetfillcolor{currentfill}%
\pgfsetfillopacity{0.800000}%
\pgfsetlinewidth{0.000000pt}%
\definecolor{currentstroke}{rgb}{0.000000,0.000000,0.000000}%
\pgfsetstrokecolor{currentstroke}%
\pgfsetdash{}{0pt}%
\pgfpathmoveto{\pgfqpoint{4.954694in}{2.811094in}}%
\pgfpathlineto{\pgfqpoint{4.968844in}{2.819774in}}%
\pgfpathlineto{\pgfqpoint{4.983011in}{2.828634in}}%
\pgfpathlineto{\pgfqpoint{4.997194in}{2.837674in}}%
\pgfpathlineto{\pgfqpoint{5.011394in}{2.846896in}}%
\pgfpathlineto{\pgfqpoint{5.018968in}{2.853404in}}%
\pgfpathlineto{\pgfqpoint{5.026536in}{2.859871in}}%
\pgfpathlineto{\pgfqpoint{5.034097in}{2.866301in}}%
\pgfpathlineto{\pgfqpoint{5.041652in}{2.872698in}}%
\pgfpathlineto{\pgfqpoint{5.027466in}{2.863817in}}%
\pgfpathlineto{\pgfqpoint{5.013297in}{2.855117in}}%
\pgfpathlineto{\pgfqpoint{4.999145in}{2.846596in}}%
\pgfpathlineto{\pgfqpoint{4.985008in}{2.838256in}}%
\pgfpathlineto{\pgfqpoint{4.977439in}{2.831507in}}%
\pgfpathlineto{\pgfqpoint{4.969863in}{2.824733in}}%
\pgfpathlineto{\pgfqpoint{4.962282in}{2.817930in}}%
\pgfpathlineto{\pgfqpoint{4.954694in}{2.811094in}}%
\pgfpathclose%
\pgfusepath{fill}%
\end{pgfscope}%
\begin{pgfscope}%
\pgfpathrectangle{\pgfqpoint{1.150000in}{0.150000in}}{\pgfqpoint{5.700000in}{5.700000in}}%
\pgfusepath{clip}%
\pgfsetbuttcap%
\pgfsetroundjoin%
\definecolor{currentfill}{rgb}{0.280255,0.165693,0.476498}%
\pgfsetfillcolor{currentfill}%
\pgfsetfillopacity{0.800000}%
\pgfsetlinewidth{0.000000pt}%
\definecolor{currentstroke}{rgb}{0.000000,0.000000,0.000000}%
\pgfsetstrokecolor{currentstroke}%
\pgfsetdash{}{0pt}%
\pgfpathmoveto{\pgfqpoint{4.024401in}{2.132047in}}%
\pgfpathlineto{\pgfqpoint{4.038127in}{2.134784in}}%
\pgfpathlineto{\pgfqpoint{4.051863in}{2.137714in}}%
\pgfpathlineto{\pgfqpoint{4.065609in}{2.140835in}}%
\pgfpathlineto{\pgfqpoint{4.079364in}{2.144148in}}%
\pgfpathlineto{\pgfqpoint{4.087305in}{2.154927in}}%
\pgfpathlineto{\pgfqpoint{4.095241in}{2.165658in}}%
\pgfpathlineto{\pgfqpoint{4.103172in}{2.176340in}}%
\pgfpathlineto{\pgfqpoint{4.111098in}{2.186972in}}%
\pgfpathlineto{\pgfqpoint{4.097348in}{2.183605in}}%
\pgfpathlineto{\pgfqpoint{4.083609in}{2.180429in}}%
\pgfpathlineto{\pgfqpoint{4.069879in}{2.177444in}}%
\pgfpathlineto{\pgfqpoint{4.056159in}{2.174652in}}%
\pgfpathlineto{\pgfqpoint{4.048227in}{2.164062in}}%
\pgfpathlineto{\pgfqpoint{4.040290in}{2.153431in}}%
\pgfpathlineto{\pgfqpoint{4.032348in}{2.142759in}}%
\pgfpathlineto{\pgfqpoint{4.024401in}{2.132047in}}%
\pgfpathclose%
\pgfusepath{fill}%
\end{pgfscope}%
\begin{pgfscope}%
\pgfpathrectangle{\pgfqpoint{1.150000in}{0.150000in}}{\pgfqpoint{5.700000in}{5.700000in}}%
\pgfusepath{clip}%
\pgfsetbuttcap%
\pgfsetroundjoin%
\definecolor{currentfill}{rgb}{0.278791,0.062145,0.386592}%
\pgfsetfillcolor{currentfill}%
\pgfsetfillopacity{0.800000}%
\pgfsetlinewidth{0.000000pt}%
\definecolor{currentstroke}{rgb}{0.000000,0.000000,0.000000}%
\pgfsetstrokecolor{currentstroke}%
\pgfsetdash{}{0pt}%
\pgfpathmoveto{\pgfqpoint{3.590831in}{1.911579in}}%
\pgfpathlineto{\pgfqpoint{3.604448in}{1.909612in}}%
\pgfpathlineto{\pgfqpoint{3.618071in}{1.907848in}}%
\pgfpathlineto{\pgfqpoint{3.631700in}{1.906288in}}%
\pgfpathlineto{\pgfqpoint{3.645334in}{1.904930in}}%
\pgfpathlineto{\pgfqpoint{3.653416in}{1.915400in}}%
\pgfpathlineto{\pgfqpoint{3.661492in}{1.925880in}}%
\pgfpathlineto{\pgfqpoint{3.669563in}{1.936370in}}%
\pgfpathlineto{\pgfqpoint{3.677628in}{1.946868in}}%
\pgfpathlineto{\pgfqpoint{3.664004in}{1.948013in}}%
\pgfpathlineto{\pgfqpoint{3.650386in}{1.949360in}}%
\pgfpathlineto{\pgfqpoint{3.636774in}{1.950911in}}%
\pgfpathlineto{\pgfqpoint{3.623168in}{1.952664in}}%
\pgfpathlineto{\pgfqpoint{3.615092in}{1.942368in}}%
\pgfpathlineto{\pgfqpoint{3.607011in}{1.932087in}}%
\pgfpathlineto{\pgfqpoint{3.598924in}{1.921823in}}%
\pgfpathlineto{\pgfqpoint{3.590831in}{1.911579in}}%
\pgfpathclose%
\pgfusepath{fill}%
\end{pgfscope}%
\begin{pgfscope}%
\pgfpathrectangle{\pgfqpoint{1.150000in}{0.150000in}}{\pgfqpoint{5.700000in}{5.700000in}}%
\pgfusepath{clip}%
\pgfsetbuttcap%
\pgfsetroundjoin%
\definecolor{currentfill}{rgb}{0.121148,0.592739,0.544641}%
\pgfsetfillcolor{currentfill}%
\pgfsetfillopacity{0.800000}%
\pgfsetlinewidth{0.000000pt}%
\definecolor{currentstroke}{rgb}{0.000000,0.000000,0.000000}%
\pgfsetstrokecolor{currentstroke}%
\pgfsetdash{}{0pt}%
\pgfpathmoveto{\pgfqpoint{5.680109in}{3.300072in}}%
\pgfpathlineto{\pgfqpoint{5.694634in}{3.310168in}}%
\pgfpathlineto{\pgfqpoint{5.709177in}{3.320438in}}%
\pgfpathlineto{\pgfqpoint{5.723739in}{3.330884in}}%
\pgfpathlineto{\pgfqpoint{5.738321in}{3.341506in}}%
\pgfpathlineto{\pgfqpoint{5.745511in}{3.344644in}}%
\pgfpathlineto{\pgfqpoint{5.752697in}{3.347874in}}%
\pgfpathlineto{\pgfqpoint{5.759880in}{3.351203in}}%
\pgfpathlineto{\pgfqpoint{5.767060in}{3.354638in}}%
\pgfpathlineto{\pgfqpoint{5.752510in}{3.344661in}}%
\pgfpathlineto{\pgfqpoint{5.737979in}{3.334858in}}%
\pgfpathlineto{\pgfqpoint{5.723467in}{3.325228in}}%
\pgfpathlineto{\pgfqpoint{5.708973in}{3.315773in}}%
\pgfpathlineto{\pgfqpoint{5.701762in}{3.311685in}}%
\pgfpathlineto{\pgfqpoint{5.694547in}{3.307709in}}%
\pgfpathlineto{\pgfqpoint{5.687330in}{3.303841in}}%
\pgfpathlineto{\pgfqpoint{5.680109in}{3.300072in}}%
\pgfpathclose%
\pgfusepath{fill}%
\end{pgfscope}%
\begin{pgfscope}%
\pgfpathrectangle{\pgfqpoint{1.150000in}{0.150000in}}{\pgfqpoint{5.700000in}{5.700000in}}%
\pgfusepath{clip}%
\pgfsetbuttcap%
\pgfsetroundjoin%
\definecolor{currentfill}{rgb}{0.243113,0.292092,0.538516}%
\pgfsetfillcolor{currentfill}%
\pgfsetfillopacity{0.800000}%
\pgfsetlinewidth{0.000000pt}%
\definecolor{currentstroke}{rgb}{0.000000,0.000000,0.000000}%
\pgfsetstrokecolor{currentstroke}%
\pgfsetdash{}{0pt}%
\pgfpathmoveto{\pgfqpoint{2.496019in}{2.493703in}}%
\pgfpathlineto{\pgfqpoint{2.509911in}{2.473273in}}%
\pgfpathlineto{\pgfqpoint{2.523791in}{2.453158in}}%
\pgfpathlineto{\pgfqpoint{2.537660in}{2.433357in}}%
\pgfpathlineto{\pgfqpoint{2.551518in}{2.413866in}}%
\pgfpathlineto{\pgfqpoint{2.560161in}{2.416978in}}%
\pgfpathlineto{\pgfqpoint{2.568789in}{2.420293in}}%
\pgfpathlineto{\pgfqpoint{2.577403in}{2.423807in}}%
\pgfpathlineto{\pgfqpoint{2.586005in}{2.427517in}}%
\pgfpathlineto{\pgfqpoint{2.572184in}{2.446613in}}%
\pgfpathlineto{\pgfqpoint{2.558354in}{2.466019in}}%
\pgfpathlineto{\pgfqpoint{2.544512in}{2.485738in}}%
\pgfpathlineto{\pgfqpoint{2.530659in}{2.505772in}}%
\pgfpathlineto{\pgfqpoint{2.522020in}{2.502445in}}%
\pgfpathlineto{\pgfqpoint{2.513368in}{2.499322in}}%
\pgfpathlineto{\pgfqpoint{2.504700in}{2.496407in}}%
\pgfpathlineto{\pgfqpoint{2.496019in}{2.493703in}}%
\pgfpathclose%
\pgfusepath{fill}%
\end{pgfscope}%
\begin{pgfscope}%
\pgfpathrectangle{\pgfqpoint{1.150000in}{0.150000in}}{\pgfqpoint{5.700000in}{5.700000in}}%
\pgfusepath{clip}%
\pgfsetbuttcap%
\pgfsetroundjoin%
\definecolor{currentfill}{rgb}{0.282910,0.105393,0.426902}%
\pgfsetfillcolor{currentfill}%
\pgfsetfillopacity{0.800000}%
\pgfsetlinewidth{0.000000pt}%
\definecolor{currentstroke}{rgb}{0.000000,0.000000,0.000000}%
\pgfsetstrokecolor{currentstroke}%
\pgfsetdash{}{0pt}%
\pgfpathmoveto{\pgfqpoint{2.881585in}{2.032674in}}%
\pgfpathlineto{\pgfqpoint{2.895262in}{2.020157in}}%
\pgfpathlineto{\pgfqpoint{2.908935in}{2.007891in}}%
\pgfpathlineto{\pgfqpoint{2.922604in}{1.995876in}}%
\pgfpathlineto{\pgfqpoint{2.936270in}{1.984110in}}%
\pgfpathlineto{\pgfqpoint{2.944680in}{1.990002in}}%
\pgfpathlineto{\pgfqpoint{2.953081in}{1.996039in}}%
\pgfpathlineto{\pgfqpoint{2.961471in}{2.002217in}}%
\pgfpathlineto{\pgfqpoint{2.969852in}{2.008533in}}%
\pgfpathlineto{\pgfqpoint{2.956214in}{2.019922in}}%
\pgfpathlineto{\pgfqpoint{2.942572in}{2.031558in}}%
\pgfpathlineto{\pgfqpoint{2.928927in}{2.043445in}}%
\pgfpathlineto{\pgfqpoint{2.915279in}{2.055584in}}%
\pgfpathlineto{\pgfqpoint{2.906871in}{2.049634in}}%
\pgfpathlineto{\pgfqpoint{2.898453in}{2.043831in}}%
\pgfpathlineto{\pgfqpoint{2.890024in}{2.038176in}}%
\pgfpathlineto{\pgfqpoint{2.881585in}{2.032674in}}%
\pgfpathclose%
\pgfusepath{fill}%
\end{pgfscope}%
\begin{pgfscope}%
\pgfpathrectangle{\pgfqpoint{1.150000in}{0.150000in}}{\pgfqpoint{5.700000in}{5.700000in}}%
\pgfusepath{clip}%
\pgfsetbuttcap%
\pgfsetroundjoin%
\definecolor{currentfill}{rgb}{0.276194,0.190074,0.493001}%
\pgfsetfillcolor{currentfill}%
\pgfsetfillopacity{0.800000}%
\pgfsetlinewidth{0.000000pt}%
\definecolor{currentstroke}{rgb}{0.000000,0.000000,0.000000}%
\pgfsetstrokecolor{currentstroke}%
\pgfsetdash{}{0pt}%
\pgfpathmoveto{\pgfqpoint{4.111098in}{2.186972in}}%
\pgfpathlineto{\pgfqpoint{4.124858in}{2.190531in}}%
\pgfpathlineto{\pgfqpoint{4.138629in}{2.194280in}}%
\pgfpathlineto{\pgfqpoint{4.152410in}{2.198219in}}%
\pgfpathlineto{\pgfqpoint{4.166201in}{2.202349in}}%
\pgfpathlineto{\pgfqpoint{4.174116in}{2.212967in}}%
\pgfpathlineto{\pgfqpoint{4.182026in}{2.223529in}}%
\pgfpathlineto{\pgfqpoint{4.189930in}{2.234034in}}%
\pgfpathlineto{\pgfqpoint{4.197829in}{2.244482in}}%
\pgfpathlineto{\pgfqpoint{4.184043in}{2.240330in}}%
\pgfpathlineto{\pgfqpoint{4.170268in}{2.236368in}}%
\pgfpathlineto{\pgfqpoint{4.156504in}{2.232596in}}%
\pgfpathlineto{\pgfqpoint{4.142750in}{2.229015in}}%
\pgfpathlineto{\pgfqpoint{4.134844in}{2.218577in}}%
\pgfpathlineto{\pgfqpoint{4.126934in}{2.208091in}}%
\pgfpathlineto{\pgfqpoint{4.119019in}{2.197556in}}%
\pgfpathlineto{\pgfqpoint{4.111098in}{2.186972in}}%
\pgfpathclose%
\pgfusepath{fill}%
\end{pgfscope}%
\begin{pgfscope}%
\pgfpathrectangle{\pgfqpoint{1.150000in}{0.150000in}}{\pgfqpoint{5.700000in}{5.700000in}}%
\pgfusepath{clip}%
\pgfsetbuttcap%
\pgfsetroundjoin%
\definecolor{currentfill}{rgb}{0.119423,0.611141,0.538982}%
\pgfsetfillcolor{currentfill}%
\pgfsetfillopacity{0.800000}%
\pgfsetlinewidth{0.000000pt}%
\definecolor{currentstroke}{rgb}{0.000000,0.000000,0.000000}%
\pgfsetstrokecolor{currentstroke}%
\pgfsetdash{}{0pt}%
\pgfpathmoveto{\pgfqpoint{5.767060in}{3.354638in}}%
\pgfpathlineto{\pgfqpoint{5.781629in}{3.364790in}}%
\pgfpathlineto{\pgfqpoint{5.796218in}{3.375116in}}%
\pgfpathlineto{\pgfqpoint{5.810827in}{3.385616in}}%
\pgfpathlineto{\pgfqpoint{5.825455in}{3.396291in}}%
\pgfpathlineto{\pgfqpoint{5.832598in}{3.399175in}}%
\pgfpathlineto{\pgfqpoint{5.839739in}{3.402171in}}%
\pgfpathlineto{\pgfqpoint{5.846878in}{3.405289in}}%
\pgfpathlineto{\pgfqpoint{5.854014in}{3.408534in}}%
\pgfpathlineto{\pgfqpoint{5.839420in}{3.398536in}}%
\pgfpathlineto{\pgfqpoint{5.824846in}{3.388712in}}%
\pgfpathlineto{\pgfqpoint{5.810291in}{3.379061in}}%
\pgfpathlineto{\pgfqpoint{5.795755in}{3.369584in}}%
\pgfpathlineto{\pgfqpoint{5.788584in}{3.365652in}}%
\pgfpathlineto{\pgfqpoint{5.781412in}{3.361856in}}%
\pgfpathlineto{\pgfqpoint{5.774237in}{3.358187in}}%
\pgfpathlineto{\pgfqpoint{5.767060in}{3.354638in}}%
\pgfpathclose%
\pgfusepath{fill}%
\end{pgfscope}%
\begin{pgfscope}%
\pgfpathrectangle{\pgfqpoint{1.150000in}{0.150000in}}{\pgfqpoint{5.700000in}{5.700000in}}%
\pgfusepath{clip}%
\pgfsetbuttcap%
\pgfsetroundjoin%
\definecolor{currentfill}{rgb}{0.225863,0.330805,0.547314}%
\pgfsetfillcolor{currentfill}%
\pgfsetfillopacity{0.800000}%
\pgfsetlinewidth{0.000000pt}%
\definecolor{currentstroke}{rgb}{0.000000,0.000000,0.000000}%
\pgfsetstrokecolor{currentstroke}%
\pgfsetdash{}{0pt}%
\pgfpathmoveto{\pgfqpoint{4.576360in}{2.528811in}}%
\pgfpathlineto{\pgfqpoint{4.590324in}{2.535767in}}%
\pgfpathlineto{\pgfqpoint{4.604302in}{2.542908in}}%
\pgfpathlineto{\pgfqpoint{4.618294in}{2.550233in}}%
\pgfpathlineto{\pgfqpoint{4.632300in}{2.557742in}}%
\pgfpathlineto{\pgfqpoint{4.640047in}{2.566421in}}%
\pgfpathlineto{\pgfqpoint{4.647788in}{2.575030in}}%
\pgfpathlineto{\pgfqpoint{4.655523in}{2.583573in}}%
\pgfpathlineto{\pgfqpoint{4.663251in}{2.592050in}}%
\pgfpathlineto{\pgfqpoint{4.649253in}{2.584715in}}%
\pgfpathlineto{\pgfqpoint{4.635270in}{2.577564in}}%
\pgfpathlineto{\pgfqpoint{4.621300in}{2.570596in}}%
\pgfpathlineto{\pgfqpoint{4.607344in}{2.563813in}}%
\pgfpathlineto{\pgfqpoint{4.599607in}{2.555150in}}%
\pgfpathlineto{\pgfqpoint{4.591864in}{2.546430in}}%
\pgfpathlineto{\pgfqpoint{4.584115in}{2.537651in}}%
\pgfpathlineto{\pgfqpoint{4.576360in}{2.528811in}}%
\pgfpathclose%
\pgfusepath{fill}%
\end{pgfscope}%
\begin{pgfscope}%
\pgfpathrectangle{\pgfqpoint{1.150000in}{0.150000in}}{\pgfqpoint{5.700000in}{5.700000in}}%
\pgfusepath{clip}%
\pgfsetbuttcap%
\pgfsetroundjoin%
\definecolor{currentfill}{rgb}{0.171176,0.452530,0.557965}%
\pgfsetfillcolor{currentfill}%
\pgfsetfillopacity{0.800000}%
\pgfsetlinewidth{0.000000pt}%
\definecolor{currentstroke}{rgb}{0.000000,0.000000,0.000000}%
\pgfsetstrokecolor{currentstroke}%
\pgfsetdash{}{0pt}%
\pgfpathmoveto{\pgfqpoint{5.041652in}{2.872698in}}%
\pgfpathlineto{\pgfqpoint{5.055853in}{2.881758in}}%
\pgfpathlineto{\pgfqpoint{5.070071in}{2.891000in}}%
\pgfpathlineto{\pgfqpoint{5.084306in}{2.900421in}}%
\pgfpathlineto{\pgfqpoint{5.098558in}{2.910022in}}%
\pgfpathlineto{\pgfqpoint{5.106091in}{2.916028in}}%
\pgfpathlineto{\pgfqpoint{5.113617in}{2.922001in}}%
\pgfpathlineto{\pgfqpoint{5.121137in}{2.927946in}}%
\pgfpathlineto{\pgfqpoint{5.128650in}{2.933867in}}%
\pgfpathlineto{\pgfqpoint{5.114414in}{2.924641in}}%
\pgfpathlineto{\pgfqpoint{5.100195in}{2.915594in}}%
\pgfpathlineto{\pgfqpoint{5.085993in}{2.906727in}}%
\pgfpathlineto{\pgfqpoint{5.071807in}{2.898039in}}%
\pgfpathlineto{\pgfqpoint{5.064277in}{2.891732in}}%
\pgfpathlineto{\pgfqpoint{5.056742in}{2.885409in}}%
\pgfpathlineto{\pgfqpoint{5.049200in}{2.879066in}}%
\pgfpathlineto{\pgfqpoint{5.041652in}{2.872698in}}%
\pgfpathclose%
\pgfusepath{fill}%
\end{pgfscope}%
\begin{pgfscope}%
\pgfpathrectangle{\pgfqpoint{1.150000in}{0.150000in}}{\pgfqpoint{5.700000in}{5.700000in}}%
\pgfusepath{clip}%
\pgfsetbuttcap%
\pgfsetroundjoin%
\definecolor{currentfill}{rgb}{0.276022,0.044167,0.370164}%
\pgfsetfillcolor{currentfill}%
\pgfsetfillopacity{0.800000}%
\pgfsetlinewidth{0.000000pt}%
\definecolor{currentstroke}{rgb}{0.000000,0.000000,0.000000}%
\pgfsetstrokecolor{currentstroke}%
\pgfsetdash{}{0pt}%
\pgfpathmoveto{\pgfqpoint{3.503937in}{1.881727in}}%
\pgfpathlineto{\pgfqpoint{3.517547in}{1.878693in}}%
\pgfpathlineto{\pgfqpoint{3.531161in}{1.875865in}}%
\pgfpathlineto{\pgfqpoint{3.544780in}{1.873244in}}%
\pgfpathlineto{\pgfqpoint{3.558404in}{1.870828in}}%
\pgfpathlineto{\pgfqpoint{3.566520in}{1.880977in}}%
\pgfpathlineto{\pgfqpoint{3.574629in}{1.891154in}}%
\pgfpathlineto{\pgfqpoint{3.582733in}{1.901355in}}%
\pgfpathlineto{\pgfqpoint{3.590831in}{1.911579in}}%
\pgfpathlineto{\pgfqpoint{3.577219in}{1.913750in}}%
\pgfpathlineto{\pgfqpoint{3.563612in}{1.916126in}}%
\pgfpathlineto{\pgfqpoint{3.550010in}{1.918709in}}%
\pgfpathlineto{\pgfqpoint{3.536413in}{1.921498in}}%
\pgfpathlineto{\pgfqpoint{3.528303in}{1.911507in}}%
\pgfpathlineto{\pgfqpoint{3.520187in}{1.901547in}}%
\pgfpathlineto{\pgfqpoint{3.512065in}{1.891619in}}%
\pgfpathlineto{\pgfqpoint{3.503937in}{1.881727in}}%
\pgfpathclose%
\pgfusepath{fill}%
\end{pgfscope}%
\begin{pgfscope}%
\pgfpathrectangle{\pgfqpoint{1.150000in}{0.150000in}}{\pgfqpoint{5.700000in}{5.700000in}}%
\pgfusepath{clip}%
\pgfsetbuttcap%
\pgfsetroundjoin%
\definecolor{currentfill}{rgb}{0.121380,0.629492,0.531973}%
\pgfsetfillcolor{currentfill}%
\pgfsetfillopacity{0.800000}%
\pgfsetlinewidth{0.000000pt}%
\definecolor{currentstroke}{rgb}{0.000000,0.000000,0.000000}%
\pgfsetstrokecolor{currentstroke}%
\pgfsetdash{}{0pt}%
\pgfpathmoveto{\pgfqpoint{5.854014in}{3.408534in}}%
\pgfpathlineto{\pgfqpoint{5.868627in}{3.418705in}}%
\pgfpathlineto{\pgfqpoint{5.883260in}{3.429050in}}%
\pgfpathlineto{\pgfqpoint{5.897913in}{3.439568in}}%
\pgfpathlineto{\pgfqpoint{5.912586in}{3.450261in}}%
\pgfpathlineto{\pgfqpoint{5.919685in}{3.452943in}}%
\pgfpathlineto{\pgfqpoint{5.926782in}{3.455760in}}%
\pgfpathlineto{\pgfqpoint{5.933877in}{3.458721in}}%
\pgfpathlineto{\pgfqpoint{5.940971in}{3.461832in}}%
\pgfpathlineto{\pgfqpoint{5.926335in}{3.451851in}}%
\pgfpathlineto{\pgfqpoint{5.911719in}{3.442042in}}%
\pgfpathlineto{\pgfqpoint{5.897122in}{3.432405in}}%
\pgfpathlineto{\pgfqpoint{5.882544in}{3.422942in}}%
\pgfpathlineto{\pgfqpoint{5.875413in}{3.419110in}}%
\pgfpathlineto{\pgfqpoint{5.868281in}{3.415437in}}%
\pgfpathlineto{\pgfqpoint{5.861149in}{3.411914in}}%
\pgfpathlineto{\pgfqpoint{5.854014in}{3.408534in}}%
\pgfpathclose%
\pgfusepath{fill}%
\end{pgfscope}%
\begin{pgfscope}%
\pgfpathrectangle{\pgfqpoint{1.150000in}{0.150000in}}{\pgfqpoint{5.700000in}{5.700000in}}%
\pgfusepath{clip}%
\pgfsetbuttcap%
\pgfsetroundjoin%
\definecolor{currentfill}{rgb}{0.273809,0.031497,0.358853}%
\pgfsetfillcolor{currentfill}%
\pgfsetfillopacity{0.800000}%
\pgfsetlinewidth{0.000000pt}%
\definecolor{currentstroke}{rgb}{0.000000,0.000000,0.000000}%
\pgfsetstrokecolor{currentstroke}%
\pgfsetdash{}{0pt}%
\pgfpathmoveto{\pgfqpoint{3.275288in}{1.863998in}}%
\pgfpathlineto{\pgfqpoint{3.288892in}{1.857827in}}%
\pgfpathlineto{\pgfqpoint{3.302498in}{1.851874in}}%
\pgfpathlineto{\pgfqpoint{3.316105in}{1.846138in}}%
\pgfpathlineto{\pgfqpoint{3.329715in}{1.840619in}}%
\pgfpathlineto{\pgfqpoint{3.337925in}{1.849506in}}%
\pgfpathlineto{\pgfqpoint{3.346127in}{1.858464in}}%
\pgfpathlineto{\pgfqpoint{3.354322in}{1.867490in}}%
\pgfpathlineto{\pgfqpoint{3.362511in}{1.876581in}}%
\pgfpathlineto{\pgfqpoint{3.348918in}{1.881792in}}%
\pgfpathlineto{\pgfqpoint{3.335327in}{1.887220in}}%
\pgfpathlineto{\pgfqpoint{3.321739in}{1.892864in}}%
\pgfpathlineto{\pgfqpoint{3.308153in}{1.898727in}}%
\pgfpathlineto{\pgfqpoint{3.299947in}{1.889932in}}%
\pgfpathlineto{\pgfqpoint{3.291735in}{1.881211in}}%
\pgfpathlineto{\pgfqpoint{3.283515in}{1.872565in}}%
\pgfpathlineto{\pgfqpoint{3.275288in}{1.863998in}}%
\pgfpathclose%
\pgfusepath{fill}%
\end{pgfscope}%
\begin{pgfscope}%
\pgfpathrectangle{\pgfqpoint{1.150000in}{0.150000in}}{\pgfqpoint{5.700000in}{5.700000in}}%
\pgfusepath{clip}%
\pgfsetbuttcap%
\pgfsetroundjoin%
\definecolor{currentfill}{rgb}{0.227802,0.326594,0.546532}%
\pgfsetfillcolor{currentfill}%
\pgfsetfillopacity{0.800000}%
\pgfsetlinewidth{0.000000pt}%
\definecolor{currentstroke}{rgb}{0.000000,0.000000,0.000000}%
\pgfsetstrokecolor{currentstroke}%
\pgfsetdash{}{0pt}%
\pgfpathmoveto{\pgfqpoint{2.440328in}{2.578649in}}%
\pgfpathlineto{\pgfqpoint{2.454270in}{2.556923in}}%
\pgfpathlineto{\pgfqpoint{2.468199in}{2.535525in}}%
\pgfpathlineto{\pgfqpoint{2.482115in}{2.514453in}}%
\pgfpathlineto{\pgfqpoint{2.496019in}{2.493703in}}%
\pgfpathlineto{\pgfqpoint{2.504700in}{2.496407in}}%
\pgfpathlineto{\pgfqpoint{2.513368in}{2.499322in}}%
\pgfpathlineto{\pgfqpoint{2.522020in}{2.502445in}}%
\pgfpathlineto{\pgfqpoint{2.530659in}{2.505772in}}%
\pgfpathlineto{\pgfqpoint{2.516795in}{2.526124in}}%
\pgfpathlineto{\pgfqpoint{2.502919in}{2.546798in}}%
\pgfpathlineto{\pgfqpoint{2.489030in}{2.567796in}}%
\pgfpathlineto{\pgfqpoint{2.475130in}{2.589122in}}%
\pgfpathlineto{\pgfqpoint{2.466451in}{2.586181in}}%
\pgfpathlineto{\pgfqpoint{2.457759in}{2.583452in}}%
\pgfpathlineto{\pgfqpoint{2.449051in}{2.580940in}}%
\pgfpathlineto{\pgfqpoint{2.440328in}{2.578649in}}%
\pgfpathclose%
\pgfusepath{fill}%
\end{pgfscope}%
\begin{pgfscope}%
\pgfpathrectangle{\pgfqpoint{1.150000in}{0.150000in}}{\pgfqpoint{5.700000in}{5.700000in}}%
\pgfusepath{clip}%
\pgfsetbuttcap%
\pgfsetroundjoin%
\definecolor{currentfill}{rgb}{0.269308,0.218818,0.509577}%
\pgfsetfillcolor{currentfill}%
\pgfsetfillopacity{0.800000}%
\pgfsetlinewidth{0.000000pt}%
\definecolor{currentstroke}{rgb}{0.000000,0.000000,0.000000}%
\pgfsetstrokecolor{currentstroke}%
\pgfsetdash{}{0pt}%
\pgfpathmoveto{\pgfqpoint{4.197829in}{2.244482in}}%
\pgfpathlineto{\pgfqpoint{4.211626in}{2.248823in}}%
\pgfpathlineto{\pgfqpoint{4.225434in}{2.253354in}}%
\pgfpathlineto{\pgfqpoint{4.239254in}{2.258073in}}%
\pgfpathlineto{\pgfqpoint{4.253085in}{2.262981in}}%
\pgfpathlineto{\pgfqpoint{4.260973in}{2.273376in}}%
\pgfpathlineto{\pgfqpoint{4.268856in}{2.283708in}}%
\pgfpathlineto{\pgfqpoint{4.276733in}{2.293976in}}%
\pgfpathlineto{\pgfqpoint{4.284606in}{2.304182in}}%
\pgfpathlineto{\pgfqpoint{4.270780in}{2.299283in}}%
\pgfpathlineto{\pgfqpoint{4.256967in}{2.294574in}}%
\pgfpathlineto{\pgfqpoint{4.243164in}{2.290053in}}%
\pgfpathlineto{\pgfqpoint{4.229373in}{2.285721in}}%
\pgfpathlineto{\pgfqpoint{4.221495in}{2.275494in}}%
\pgfpathlineto{\pgfqpoint{4.213612in}{2.265212in}}%
\pgfpathlineto{\pgfqpoint{4.205723in}{2.254875in}}%
\pgfpathlineto{\pgfqpoint{4.197829in}{2.244482in}}%
\pgfpathclose%
\pgfusepath{fill}%
\end{pgfscope}%
\begin{pgfscope}%
\pgfpathrectangle{\pgfqpoint{1.150000in}{0.150000in}}{\pgfqpoint{5.700000in}{5.700000in}}%
\pgfusepath{clip}%
\pgfsetbuttcap%
\pgfsetroundjoin%
\definecolor{currentfill}{rgb}{0.276022,0.044167,0.370164}%
\pgfsetfillcolor{currentfill}%
\pgfsetfillopacity{0.800000}%
\pgfsetlinewidth{0.000000pt}%
\definecolor{currentstroke}{rgb}{0.000000,0.000000,0.000000}%
\pgfsetstrokecolor{currentstroke}%
\pgfsetdash{}{0pt}%
\pgfpathmoveto{\pgfqpoint{3.133352in}{1.890668in}}%
\pgfpathlineto{\pgfqpoint{3.146970in}{1.882368in}}%
\pgfpathlineto{\pgfqpoint{3.160588in}{1.874295in}}%
\pgfpathlineto{\pgfqpoint{3.174206in}{1.866450in}}%
\pgfpathlineto{\pgfqpoint{3.187824in}{1.858829in}}%
\pgfpathlineto{\pgfqpoint{3.196102in}{1.866699in}}%
\pgfpathlineto{\pgfqpoint{3.204371in}{1.874668in}}%
\pgfpathlineto{\pgfqpoint{3.212632in}{1.882732in}}%
\pgfpathlineto{\pgfqpoint{3.220886in}{1.890888in}}%
\pgfpathlineto{\pgfqpoint{3.207288in}{1.898167in}}%
\pgfpathlineto{\pgfqpoint{3.193691in}{1.905672in}}%
\pgfpathlineto{\pgfqpoint{3.180094in}{1.913402in}}%
\pgfpathlineto{\pgfqpoint{3.166497in}{1.921361in}}%
\pgfpathlineto{\pgfqpoint{3.158223in}{1.913534in}}%
\pgfpathlineto{\pgfqpoint{3.149941in}{1.905807in}}%
\pgfpathlineto{\pgfqpoint{3.141650in}{1.898184in}}%
\pgfpathlineto{\pgfqpoint{3.133352in}{1.890668in}}%
\pgfpathclose%
\pgfusepath{fill}%
\end{pgfscope}%
\begin{pgfscope}%
\pgfpathrectangle{\pgfqpoint{1.150000in}{0.150000in}}{\pgfqpoint{5.700000in}{5.700000in}}%
\pgfusepath{clip}%
\pgfsetbuttcap%
\pgfsetroundjoin%
\definecolor{currentfill}{rgb}{0.128087,0.647749,0.523491}%
\pgfsetfillcolor{currentfill}%
\pgfsetfillopacity{0.800000}%
\pgfsetlinewidth{0.000000pt}%
\definecolor{currentstroke}{rgb}{0.000000,0.000000,0.000000}%
\pgfsetstrokecolor{currentstroke}%
\pgfsetdash{}{0pt}%
\pgfpathmoveto{\pgfqpoint{5.940971in}{3.461832in}}%
\pgfpathlineto{\pgfqpoint{5.955627in}{3.471987in}}%
\pgfpathlineto{\pgfqpoint{5.970303in}{3.482315in}}%
\pgfpathlineto{\pgfqpoint{5.984999in}{3.492816in}}%
\pgfpathlineto{\pgfqpoint{5.999715in}{3.503490in}}%
\pgfpathlineto{\pgfqpoint{6.006770in}{3.506029in}}%
\pgfpathlineto{\pgfqpoint{6.013824in}{3.508727in}}%
\pgfpathlineto{\pgfqpoint{6.020878in}{3.511592in}}%
\pgfpathlineto{\pgfqpoint{6.027931in}{3.514633in}}%
\pgfpathlineto{\pgfqpoint{6.013255in}{3.504702in}}%
\pgfpathlineto{\pgfqpoint{5.998598in}{3.494944in}}%
\pgfpathlineto{\pgfqpoint{5.983961in}{3.485358in}}%
\pgfpathlineto{\pgfqpoint{5.969343in}{3.475944in}}%
\pgfpathlineto{\pgfqpoint{5.962250in}{3.472151in}}%
\pgfpathlineto{\pgfqpoint{5.955157in}{3.468539in}}%
\pgfpathlineto{\pgfqpoint{5.948064in}{3.465103in}}%
\pgfpathlineto{\pgfqpoint{5.940971in}{3.461832in}}%
\pgfpathclose%
\pgfusepath{fill}%
\end{pgfscope}%
\begin{pgfscope}%
\pgfpathrectangle{\pgfqpoint{1.150000in}{0.150000in}}{\pgfqpoint{5.700000in}{5.700000in}}%
\pgfusepath{clip}%
\pgfsetbuttcap%
\pgfsetroundjoin%
\definecolor{currentfill}{rgb}{0.281446,0.084320,0.407414}%
\pgfsetfillcolor{currentfill}%
\pgfsetfillopacity{0.800000}%
\pgfsetlinewidth{0.000000pt}%
\definecolor{currentstroke}{rgb}{0.000000,0.000000,0.000000}%
\pgfsetstrokecolor{currentstroke}%
\pgfsetdash{}{0pt}%
\pgfpathmoveto{\pgfqpoint{2.936270in}{1.984110in}}%
\pgfpathlineto{\pgfqpoint{2.949932in}{1.972590in}}%
\pgfpathlineto{\pgfqpoint{2.963592in}{1.961316in}}%
\pgfpathlineto{\pgfqpoint{2.977248in}{1.950286in}}%
\pgfpathlineto{\pgfqpoint{2.990902in}{1.939498in}}%
\pgfpathlineto{\pgfqpoint{2.999286in}{1.945780in}}%
\pgfpathlineto{\pgfqpoint{3.007660in}{1.952198in}}%
\pgfpathlineto{\pgfqpoint{3.016024in}{1.958749in}}%
\pgfpathlineto{\pgfqpoint{3.024379in}{1.965430in}}%
\pgfpathlineto{\pgfqpoint{3.010751in}{1.975841in}}%
\pgfpathlineto{\pgfqpoint{2.997120in}{1.986494in}}%
\pgfpathlineto{\pgfqpoint{2.983487in}{1.997391in}}%
\pgfpathlineto{\pgfqpoint{2.969852in}{2.008533in}}%
\pgfpathlineto{\pgfqpoint{2.961471in}{2.002217in}}%
\pgfpathlineto{\pgfqpoint{2.953081in}{1.996039in}}%
\pgfpathlineto{\pgfqpoint{2.944680in}{1.990002in}}%
\pgfpathlineto{\pgfqpoint{2.936270in}{1.984110in}}%
\pgfpathclose%
\pgfusepath{fill}%
\end{pgfscope}%
\begin{pgfscope}%
\pgfpathrectangle{\pgfqpoint{1.150000in}{0.150000in}}{\pgfqpoint{5.700000in}{5.700000in}}%
\pgfusepath{clip}%
\pgfsetbuttcap%
\pgfsetroundjoin%
\definecolor{currentfill}{rgb}{0.162142,0.474838,0.558140}%
\pgfsetfillcolor{currentfill}%
\pgfsetfillopacity{0.800000}%
\pgfsetlinewidth{0.000000pt}%
\definecolor{currentstroke}{rgb}{0.000000,0.000000,0.000000}%
\pgfsetstrokecolor{currentstroke}%
\pgfsetdash{}{0pt}%
\pgfpathmoveto{\pgfqpoint{5.128650in}{2.933867in}}%
\pgfpathlineto{\pgfqpoint{5.142903in}{2.943273in}}%
\pgfpathlineto{\pgfqpoint{5.157172in}{2.952859in}}%
\pgfpathlineto{\pgfqpoint{5.171459in}{2.962624in}}%
\pgfpathlineto{\pgfqpoint{5.185763in}{2.972568in}}%
\pgfpathlineto{\pgfqpoint{5.193253in}{2.978073in}}%
\pgfpathlineto{\pgfqpoint{5.200736in}{2.983555in}}%
\pgfpathlineto{\pgfqpoint{5.208213in}{2.989019in}}%
\pgfpathlineto{\pgfqpoint{5.215684in}{2.994469in}}%
\pgfpathlineto{\pgfqpoint{5.201397in}{2.984934in}}%
\pgfpathlineto{\pgfqpoint{5.187128in}{2.975578in}}%
\pgfpathlineto{\pgfqpoint{5.172876in}{2.966400in}}%
\pgfpathlineto{\pgfqpoint{5.158640in}{2.957402in}}%
\pgfpathlineto{\pgfqpoint{5.151152in}{2.951531in}}%
\pgfpathlineto{\pgfqpoint{5.143657in}{2.945655in}}%
\pgfpathlineto{\pgfqpoint{5.136157in}{2.939769in}}%
\pgfpathlineto{\pgfqpoint{5.128650in}{2.933867in}}%
\pgfpathclose%
\pgfusepath{fill}%
\end{pgfscope}%
\begin{pgfscope}%
\pgfpathrectangle{\pgfqpoint{1.150000in}{0.150000in}}{\pgfqpoint{5.700000in}{5.700000in}}%
\pgfusepath{clip}%
\pgfsetbuttcap%
\pgfsetroundjoin%
\definecolor{currentfill}{rgb}{0.212395,0.359683,0.551710}%
\pgfsetfillcolor{currentfill}%
\pgfsetfillopacity{0.800000}%
\pgfsetlinewidth{0.000000pt}%
\definecolor{currentstroke}{rgb}{0.000000,0.000000,0.000000}%
\pgfsetstrokecolor{currentstroke}%
\pgfsetdash{}{0pt}%
\pgfpathmoveto{\pgfqpoint{4.663251in}{2.592050in}}%
\pgfpathlineto{\pgfqpoint{4.677263in}{2.599569in}}%
\pgfpathlineto{\pgfqpoint{4.691290in}{2.607272in}}%
\pgfpathlineto{\pgfqpoint{4.705331in}{2.615158in}}%
\pgfpathlineto{\pgfqpoint{4.719387in}{2.623228in}}%
\pgfpathlineto{\pgfqpoint{4.727100in}{2.631448in}}%
\pgfpathlineto{\pgfqpoint{4.734807in}{2.639601in}}%
\pgfpathlineto{\pgfqpoint{4.742508in}{2.647688in}}%
\pgfpathlineto{\pgfqpoint{4.750202in}{2.655712in}}%
\pgfpathlineto{\pgfqpoint{4.736156in}{2.647850in}}%
\pgfpathlineto{\pgfqpoint{4.722124in}{2.640170in}}%
\pgfpathlineto{\pgfqpoint{4.708106in}{2.632674in}}%
\pgfpathlineto{\pgfqpoint{4.694103in}{2.625362in}}%
\pgfpathlineto{\pgfqpoint{4.686400in}{2.617119in}}%
\pgfpathlineto{\pgfqpoint{4.678690in}{2.608821in}}%
\pgfpathlineto{\pgfqpoint{4.670973in}{2.600465in}}%
\pgfpathlineto{\pgfqpoint{4.663251in}{2.592050in}}%
\pgfpathclose%
\pgfusepath{fill}%
\end{pgfscope}%
\begin{pgfscope}%
\pgfpathrectangle{\pgfqpoint{1.150000in}{0.150000in}}{\pgfqpoint{5.700000in}{5.700000in}}%
\pgfusepath{clip}%
\pgfsetbuttcap%
\pgfsetroundjoin%
\definecolor{currentfill}{rgb}{0.140210,0.665859,0.513427}%
\pgfsetfillcolor{currentfill}%
\pgfsetfillopacity{0.800000}%
\pgfsetlinewidth{0.000000pt}%
\definecolor{currentstroke}{rgb}{0.000000,0.000000,0.000000}%
\pgfsetstrokecolor{currentstroke}%
\pgfsetdash{}{0pt}%
\pgfpathmoveto{\pgfqpoint{6.027931in}{3.514633in}}%
\pgfpathlineto{\pgfqpoint{6.042628in}{3.524735in}}%
\pgfpathlineto{\pgfqpoint{6.057345in}{3.535010in}}%
\pgfpathlineto{\pgfqpoint{6.072083in}{3.545457in}}%
\pgfpathlineto{\pgfqpoint{6.086840in}{3.556078in}}%
\pgfpathlineto{\pgfqpoint{6.093854in}{3.558537in}}%
\pgfpathlineto{\pgfqpoint{6.100867in}{3.561181in}}%
\pgfpathlineto{\pgfqpoint{6.107882in}{3.564018in}}%
\pgfpathlineto{\pgfqpoint{6.114898in}{3.567057in}}%
\pgfpathlineto{\pgfqpoint{6.100182in}{3.557214in}}%
\pgfpathlineto{\pgfqpoint{6.085486in}{3.547542in}}%
\pgfpathlineto{\pgfqpoint{6.070810in}{3.538042in}}%
\pgfpathlineto{\pgfqpoint{6.056154in}{3.528713in}}%
\pgfpathlineto{\pgfqpoint{6.049097in}{3.524889in}}%
\pgfpathlineto{\pgfqpoint{6.042041in}{3.521273in}}%
\pgfpathlineto{\pgfqpoint{6.034986in}{3.517857in}}%
\pgfpathlineto{\pgfqpoint{6.027931in}{3.514633in}}%
\pgfpathclose%
\pgfusepath{fill}%
\end{pgfscope}%
\begin{pgfscope}%
\pgfpathrectangle{\pgfqpoint{1.150000in}{0.150000in}}{\pgfqpoint{5.700000in}{5.700000in}}%
\pgfusepath{clip}%
\pgfsetbuttcap%
\pgfsetroundjoin%
\definecolor{currentfill}{rgb}{0.273809,0.031497,0.358853}%
\pgfsetfillcolor{currentfill}%
\pgfsetfillopacity{0.800000}%
\pgfsetlinewidth{0.000000pt}%
\definecolor{currentstroke}{rgb}{0.000000,0.000000,0.000000}%
\pgfsetstrokecolor{currentstroke}%
\pgfsetdash{}{0pt}%
\pgfpathmoveto{\pgfqpoint{3.416911in}{1.857876in}}%
\pgfpathlineto{\pgfqpoint{3.430519in}{1.853730in}}%
\pgfpathlineto{\pgfqpoint{3.444131in}{1.849794in}}%
\pgfpathlineto{\pgfqpoint{3.457746in}{1.846068in}}%
\pgfpathlineto{\pgfqpoint{3.471366in}{1.842550in}}%
\pgfpathlineto{\pgfqpoint{3.479518in}{1.852280in}}%
\pgfpathlineto{\pgfqpoint{3.487663in}{1.862055in}}%
\pgfpathlineto{\pgfqpoint{3.495803in}{1.871871in}}%
\pgfpathlineto{\pgfqpoint{3.503937in}{1.881727in}}%
\pgfpathlineto{\pgfqpoint{3.490331in}{1.884968in}}%
\pgfpathlineto{\pgfqpoint{3.476730in}{1.888418in}}%
\pgfpathlineto{\pgfqpoint{3.463133in}{1.892078in}}%
\pgfpathlineto{\pgfqpoint{3.449539in}{1.895947in}}%
\pgfpathlineto{\pgfqpoint{3.441392in}{1.886356in}}%
\pgfpathlineto{\pgfqpoint{3.433238in}{1.876812in}}%
\pgfpathlineto{\pgfqpoint{3.425078in}{1.867318in}}%
\pgfpathlineto{\pgfqpoint{3.416911in}{1.857876in}}%
\pgfpathclose%
\pgfusepath{fill}%
\end{pgfscope}%
\begin{pgfscope}%
\pgfpathrectangle{\pgfqpoint{1.150000in}{0.150000in}}{\pgfqpoint{5.700000in}{5.700000in}}%
\pgfusepath{clip}%
\pgfsetbuttcap%
\pgfsetroundjoin%
\definecolor{currentfill}{rgb}{0.260571,0.246922,0.522828}%
\pgfsetfillcolor{currentfill}%
\pgfsetfillopacity{0.800000}%
\pgfsetlinewidth{0.000000pt}%
\definecolor{currentstroke}{rgb}{0.000000,0.000000,0.000000}%
\pgfsetstrokecolor{currentstroke}%
\pgfsetdash{}{0pt}%
\pgfpathmoveto{\pgfqpoint{4.284606in}{2.304182in}}%
\pgfpathlineto{\pgfqpoint{4.298443in}{2.309268in}}%
\pgfpathlineto{\pgfqpoint{4.312291in}{2.314542in}}%
\pgfpathlineto{\pgfqpoint{4.326152in}{2.320004in}}%
\pgfpathlineto{\pgfqpoint{4.340025in}{2.325653in}}%
\pgfpathlineto{\pgfqpoint{4.347886in}{2.335767in}}%
\pgfpathlineto{\pgfqpoint{4.355741in}{2.345812in}}%
\pgfpathlineto{\pgfqpoint{4.363591in}{2.355788in}}%
\pgfpathlineto{\pgfqpoint{4.371436in}{2.365698in}}%
\pgfpathlineto{\pgfqpoint{4.357568in}{2.360091in}}%
\pgfpathlineto{\pgfqpoint{4.343714in}{2.354672in}}%
\pgfpathlineto{\pgfqpoint{4.329871in}{2.349440in}}%
\pgfpathlineto{\pgfqpoint{4.316040in}{2.344395in}}%
\pgfpathlineto{\pgfqpoint{4.308190in}{2.334432in}}%
\pgfpathlineto{\pgfqpoint{4.300334in}{2.324409in}}%
\pgfpathlineto{\pgfqpoint{4.292472in}{2.314326in}}%
\pgfpathlineto{\pgfqpoint{4.284606in}{2.304182in}}%
\pgfpathclose%
\pgfusepath{fill}%
\end{pgfscope}%
\begin{pgfscope}%
\pgfpathrectangle{\pgfqpoint{1.150000in}{0.150000in}}{\pgfqpoint{5.700000in}{5.700000in}}%
\pgfusepath{clip}%
\pgfsetbuttcap%
\pgfsetroundjoin%
\definecolor{currentfill}{rgb}{0.153364,0.497000,0.557724}%
\pgfsetfillcolor{currentfill}%
\pgfsetfillopacity{0.800000}%
\pgfsetlinewidth{0.000000pt}%
\definecolor{currentstroke}{rgb}{0.000000,0.000000,0.000000}%
\pgfsetstrokecolor{currentstroke}%
\pgfsetdash{}{0pt}%
\pgfpathmoveto{\pgfqpoint{5.215684in}{2.994469in}}%
\pgfpathlineto{\pgfqpoint{5.229987in}{3.004184in}}%
\pgfpathlineto{\pgfqpoint{5.244309in}{3.014077in}}%
\pgfpathlineto{\pgfqpoint{5.258647in}{3.024149in}}%
\pgfpathlineto{\pgfqpoint{5.273004in}{3.034400in}}%
\pgfpathlineto{\pgfqpoint{5.280449in}{3.039410in}}%
\pgfpathlineto{\pgfqpoint{5.287888in}{3.044409in}}%
\pgfpathlineto{\pgfqpoint{5.295321in}{3.049402in}}%
\pgfpathlineto{\pgfqpoint{5.302747in}{3.054392in}}%
\pgfpathlineto{\pgfqpoint{5.288410in}{3.044585in}}%
\pgfpathlineto{\pgfqpoint{5.274091in}{3.034955in}}%
\pgfpathlineto{\pgfqpoint{5.259789in}{3.025504in}}%
\pgfpathlineto{\pgfqpoint{5.245504in}{3.016231in}}%
\pgfpathlineto{\pgfqpoint{5.238058in}{3.010787in}}%
\pgfpathlineto{\pgfqpoint{5.230606in}{3.005349in}}%
\pgfpathlineto{\pgfqpoint{5.223148in}{2.999911in}}%
\pgfpathlineto{\pgfqpoint{5.215684in}{2.994469in}}%
\pgfpathclose%
\pgfusepath{fill}%
\end{pgfscope}%
\begin{pgfscope}%
\pgfpathrectangle{\pgfqpoint{1.150000in}{0.150000in}}{\pgfqpoint{5.700000in}{5.700000in}}%
\pgfusepath{clip}%
\pgfsetbuttcap%
\pgfsetroundjoin%
\definecolor{currentfill}{rgb}{0.212395,0.359683,0.551710}%
\pgfsetfillcolor{currentfill}%
\pgfsetfillopacity{0.800000}%
\pgfsetlinewidth{0.000000pt}%
\definecolor{currentstroke}{rgb}{0.000000,0.000000,0.000000}%
\pgfsetstrokecolor{currentstroke}%
\pgfsetdash{}{0pt}%
\pgfpathmoveto{\pgfqpoint{2.384425in}{2.668901in}}%
\pgfpathlineto{\pgfqpoint{2.398422in}{2.645829in}}%
\pgfpathlineto{\pgfqpoint{2.412405in}{2.623099in}}%
\pgfpathlineto{\pgfqpoint{2.426373in}{2.600706in}}%
\pgfpathlineto{\pgfqpoint{2.440328in}{2.578649in}}%
\pgfpathlineto{\pgfqpoint{2.449051in}{2.580940in}}%
\pgfpathlineto{\pgfqpoint{2.457759in}{2.583452in}}%
\pgfpathlineto{\pgfqpoint{2.466451in}{2.586181in}}%
\pgfpathlineto{\pgfqpoint{2.475130in}{2.589122in}}%
\pgfpathlineto{\pgfqpoint{2.461216in}{2.610778in}}%
\pgfpathlineto{\pgfqpoint{2.447289in}{2.632768in}}%
\pgfpathlineto{\pgfqpoint{2.433349in}{2.655096in}}%
\pgfpathlineto{\pgfqpoint{2.419394in}{2.677764in}}%
\pgfpathlineto{\pgfqpoint{2.410675in}{2.675212in}}%
\pgfpathlineto{\pgfqpoint{2.401941in}{2.672881in}}%
\pgfpathlineto{\pgfqpoint{2.393191in}{2.670777in}}%
\pgfpathlineto{\pgfqpoint{2.384425in}{2.668901in}}%
\pgfpathclose%
\pgfusepath{fill}%
\end{pgfscope}%
\begin{pgfscope}%
\pgfpathrectangle{\pgfqpoint{1.150000in}{0.150000in}}{\pgfqpoint{5.700000in}{5.700000in}}%
\pgfusepath{clip}%
\pgfsetbuttcap%
\pgfsetroundjoin%
\definecolor{currentfill}{rgb}{0.201239,0.383670,0.554294}%
\pgfsetfillcolor{currentfill}%
\pgfsetfillopacity{0.800000}%
\pgfsetlinewidth{0.000000pt}%
\definecolor{currentstroke}{rgb}{0.000000,0.000000,0.000000}%
\pgfsetstrokecolor{currentstroke}%
\pgfsetdash{}{0pt}%
\pgfpathmoveto{\pgfqpoint{4.750202in}{2.655712in}}%
\pgfpathlineto{\pgfqpoint{4.764264in}{2.663757in}}%
\pgfpathlineto{\pgfqpoint{4.778340in}{2.671985in}}%
\pgfpathlineto{\pgfqpoint{4.792432in}{2.680396in}}%
\pgfpathlineto{\pgfqpoint{4.806539in}{2.688989in}}%
\pgfpathlineto{\pgfqpoint{4.814217in}{2.696725in}}%
\pgfpathlineto{\pgfqpoint{4.821889in}{2.704395in}}%
\pgfpathlineto{\pgfqpoint{4.829554in}{2.712003in}}%
\pgfpathlineto{\pgfqpoint{4.837212in}{2.719551in}}%
\pgfpathlineto{\pgfqpoint{4.823115in}{2.711199in}}%
\pgfpathlineto{\pgfqpoint{4.809034in}{2.703029in}}%
\pgfpathlineto{\pgfqpoint{4.794967in}{2.695042in}}%
\pgfpathlineto{\pgfqpoint{4.780916in}{2.687237in}}%
\pgfpathlineto{\pgfqpoint{4.773247in}{2.679435in}}%
\pgfpathlineto{\pgfqpoint{4.765572in}{2.671583in}}%
\pgfpathlineto{\pgfqpoint{4.757890in}{2.663676in}}%
\pgfpathlineto{\pgfqpoint{4.750202in}{2.655712in}}%
\pgfpathclose%
\pgfusepath{fill}%
\end{pgfscope}%
\begin{pgfscope}%
\pgfpathrectangle{\pgfqpoint{1.150000in}{0.150000in}}{\pgfqpoint{5.700000in}{5.700000in}}%
\pgfusepath{clip}%
\pgfsetbuttcap%
\pgfsetroundjoin%
\definecolor{currentfill}{rgb}{0.157851,0.683765,0.501686}%
\pgfsetfillcolor{currentfill}%
\pgfsetfillopacity{0.800000}%
\pgfsetlinewidth{0.000000pt}%
\definecolor{currentstroke}{rgb}{0.000000,0.000000,0.000000}%
\pgfsetstrokecolor{currentstroke}%
\pgfsetdash{}{0pt}%
\pgfpathmoveto{\pgfqpoint{6.114898in}{3.567057in}}%
\pgfpathlineto{\pgfqpoint{6.129634in}{3.577071in}}%
\pgfpathlineto{\pgfqpoint{6.144390in}{3.587258in}}%
\pgfpathlineto{\pgfqpoint{6.159167in}{3.597616in}}%
\pgfpathlineto{\pgfqpoint{6.173965in}{3.608146in}}%
\pgfpathlineto{\pgfqpoint{6.180939in}{3.610597in}}%
\pgfpathlineto{\pgfqpoint{6.187915in}{3.613259in}}%
\pgfpathlineto{\pgfqpoint{6.194893in}{3.616141in}}%
\pgfpathlineto{\pgfqpoint{6.180128in}{3.606215in}}%
\pgfpathlineto{\pgfqpoint{6.165384in}{3.596461in}}%
\pgfpathlineto{\pgfqpoint{6.150661in}{3.586878in}}%
\pgfpathlineto{\pgfqpoint{6.135957in}{3.577465in}}%
\pgfpathlineto{\pgfqpoint{6.128935in}{3.573772in}}%
\pgfpathlineto{\pgfqpoint{6.121915in}{3.570305in}}%
\pgfpathlineto{\pgfqpoint{6.114898in}{3.567057in}}%
\pgfpathclose%
\pgfusepath{fill}%
\end{pgfscope}%
\begin{pgfscope}%
\pgfpathrectangle{\pgfqpoint{1.150000in}{0.150000in}}{\pgfqpoint{5.700000in}{5.700000in}}%
\pgfusepath{clip}%
\pgfsetbuttcap%
\pgfsetroundjoin%
\definecolor{currentfill}{rgb}{0.279566,0.067836,0.391917}%
\pgfsetfillcolor{currentfill}%
\pgfsetfillopacity{0.800000}%
\pgfsetlinewidth{0.000000pt}%
\definecolor{currentstroke}{rgb}{0.000000,0.000000,0.000000}%
\pgfsetstrokecolor{currentstroke}%
\pgfsetdash{}{0pt}%
\pgfpathmoveto{\pgfqpoint{2.990902in}{1.939498in}}%
\pgfpathlineto{\pgfqpoint{3.004554in}{1.928951in}}%
\pgfpathlineto{\pgfqpoint{3.018204in}{1.918643in}}%
\pgfpathlineto{\pgfqpoint{3.031852in}{1.908573in}}%
\pgfpathlineto{\pgfqpoint{3.045498in}{1.898739in}}%
\pgfpathlineto{\pgfqpoint{3.053856in}{1.905408in}}%
\pgfpathlineto{\pgfqpoint{3.062205in}{1.912206in}}%
\pgfpathlineto{\pgfqpoint{3.070544in}{1.919129in}}%
\pgfpathlineto{\pgfqpoint{3.078874in}{1.926172in}}%
\pgfpathlineto{\pgfqpoint{3.065253in}{1.935631in}}%
\pgfpathlineto{\pgfqpoint{3.051630in}{1.945326in}}%
\pgfpathlineto{\pgfqpoint{3.038005in}{1.955259in}}%
\pgfpathlineto{\pgfqpoint{3.024379in}{1.965430in}}%
\pgfpathlineto{\pgfqpoint{3.016024in}{1.958749in}}%
\pgfpathlineto{\pgfqpoint{3.007660in}{1.952198in}}%
\pgfpathlineto{\pgfqpoint{2.999286in}{1.945780in}}%
\pgfpathlineto{\pgfqpoint{2.990902in}{1.939498in}}%
\pgfpathclose%
\pgfusepath{fill}%
\end{pgfscope}%
\begin{pgfscope}%
\pgfpathrectangle{\pgfqpoint{1.150000in}{0.150000in}}{\pgfqpoint{5.700000in}{5.700000in}}%
\pgfusepath{clip}%
\pgfsetbuttcap%
\pgfsetroundjoin%
\definecolor{currentfill}{rgb}{0.250425,0.274290,0.533103}%
\pgfsetfillcolor{currentfill}%
\pgfsetfillopacity{0.800000}%
\pgfsetlinewidth{0.000000pt}%
\definecolor{currentstroke}{rgb}{0.000000,0.000000,0.000000}%
\pgfsetstrokecolor{currentstroke}%
\pgfsetdash{}{0pt}%
\pgfpathmoveto{\pgfqpoint{4.371436in}{2.365698in}}%
\pgfpathlineto{\pgfqpoint{4.385315in}{2.371492in}}%
\pgfpathlineto{\pgfqpoint{4.399207in}{2.377472in}}%
\pgfpathlineto{\pgfqpoint{4.413112in}{2.383639in}}%
\pgfpathlineto{\pgfqpoint{4.427029in}{2.389992in}}%
\pgfpathlineto{\pgfqpoint{4.434862in}{2.399773in}}%
\pgfpathlineto{\pgfqpoint{4.442689in}{2.409480in}}%
\pgfpathlineto{\pgfqpoint{4.450510in}{2.419115in}}%
\pgfpathlineto{\pgfqpoint{4.458325in}{2.428680in}}%
\pgfpathlineto{\pgfqpoint{4.444414in}{2.422402in}}%
\pgfpathlineto{\pgfqpoint{4.430516in}{2.416310in}}%
\pgfpathlineto{\pgfqpoint{4.416630in}{2.410404in}}%
\pgfpathlineto{\pgfqpoint{4.402757in}{2.404685in}}%
\pgfpathlineto{\pgfqpoint{4.394935in}{2.395034in}}%
\pgfpathlineto{\pgfqpoint{4.387107in}{2.385319in}}%
\pgfpathlineto{\pgfqpoint{4.379274in}{2.375541in}}%
\pgfpathlineto{\pgfqpoint{4.371436in}{2.365698in}}%
\pgfpathclose%
\pgfusepath{fill}%
\end{pgfscope}%
\begin{pgfscope}%
\pgfpathrectangle{\pgfqpoint{1.150000in}{0.150000in}}{\pgfqpoint{5.700000in}{5.700000in}}%
\pgfusepath{clip}%
\pgfsetbuttcap%
\pgfsetroundjoin%
\definecolor{currentfill}{rgb}{0.146180,0.515413,0.556823}%
\pgfsetfillcolor{currentfill}%
\pgfsetfillopacity{0.800000}%
\pgfsetlinewidth{0.000000pt}%
\definecolor{currentstroke}{rgb}{0.000000,0.000000,0.000000}%
\pgfsetstrokecolor{currentstroke}%
\pgfsetdash{}{0pt}%
\pgfpathmoveto{\pgfqpoint{5.302747in}{3.054392in}}%
\pgfpathlineto{\pgfqpoint{5.317102in}{3.064378in}}%
\pgfpathlineto{\pgfqpoint{5.331474in}{3.074542in}}%
\pgfpathlineto{\pgfqpoint{5.345865in}{3.084885in}}%
\pgfpathlineto{\pgfqpoint{5.360274in}{3.095406in}}%
\pgfpathlineto{\pgfqpoint{5.367673in}{3.099934in}}%
\pgfpathlineto{\pgfqpoint{5.375067in}{3.104464in}}%
\pgfpathlineto{\pgfqpoint{5.382454in}{3.108999in}}%
\pgfpathlineto{\pgfqpoint{5.389835in}{3.113546in}}%
\pgfpathlineto{\pgfqpoint{5.375447in}{3.103503in}}%
\pgfpathlineto{\pgfqpoint{5.361078in}{3.093638in}}%
\pgfpathlineto{\pgfqpoint{5.346726in}{3.083950in}}%
\pgfpathlineto{\pgfqpoint{5.332393in}{3.074439in}}%
\pgfpathlineto{\pgfqpoint{5.324990in}{3.069404in}}%
\pgfpathlineto{\pgfqpoint{5.317581in}{3.064388in}}%
\pgfpathlineto{\pgfqpoint{5.310167in}{3.059386in}}%
\pgfpathlineto{\pgfqpoint{5.302747in}{3.054392in}}%
\pgfpathclose%
\pgfusepath{fill}%
\end{pgfscope}%
\begin{pgfscope}%
\pgfpathrectangle{\pgfqpoint{1.150000in}{0.150000in}}{\pgfqpoint{5.700000in}{5.700000in}}%
\pgfusepath{clip}%
\pgfsetbuttcap%
\pgfsetroundjoin%
\definecolor{currentfill}{rgb}{0.283091,0.110553,0.431554}%
\pgfsetfillcolor{currentfill}%
\pgfsetfillopacity{0.800000}%
\pgfsetlinewidth{0.000000pt}%
\definecolor{currentstroke}{rgb}{0.000000,0.000000,0.000000}%
\pgfsetstrokecolor{currentstroke}%
\pgfsetdash{}{0pt}%
\pgfpathmoveto{\pgfqpoint{3.818993in}{1.988397in}}%
\pgfpathlineto{\pgfqpoint{3.832671in}{1.989225in}}%
\pgfpathlineto{\pgfqpoint{3.846356in}{1.990248in}}%
\pgfpathlineto{\pgfqpoint{3.860050in}{1.991468in}}%
\pgfpathlineto{\pgfqpoint{3.873752in}{1.992882in}}%
\pgfpathlineto{\pgfqpoint{3.881766in}{2.003880in}}%
\pgfpathlineto{\pgfqpoint{3.889775in}{2.014854in}}%
\pgfpathlineto{\pgfqpoint{3.897779in}{2.025803in}}%
\pgfpathlineto{\pgfqpoint{3.905777in}{2.036724in}}%
\pgfpathlineto{\pgfqpoint{3.892083in}{2.035160in}}%
\pgfpathlineto{\pgfqpoint{3.878397in}{2.033790in}}%
\pgfpathlineto{\pgfqpoint{3.864719in}{2.032617in}}%
\pgfpathlineto{\pgfqpoint{3.851049in}{2.031639in}}%
\pgfpathlineto{\pgfqpoint{3.843043in}{2.020856in}}%
\pgfpathlineto{\pgfqpoint{3.835031in}{2.010054in}}%
\pgfpathlineto{\pgfqpoint{3.827015in}{1.999234in}}%
\pgfpathlineto{\pgfqpoint{3.818993in}{1.988397in}}%
\pgfpathclose%
\pgfusepath{fill}%
\end{pgfscope}%
\begin{pgfscope}%
\pgfpathrectangle{\pgfqpoint{1.150000in}{0.150000in}}{\pgfqpoint{5.700000in}{5.700000in}}%
\pgfusepath{clip}%
\pgfsetbuttcap%
\pgfsetroundjoin%
\definecolor{currentfill}{rgb}{0.281924,0.089666,0.412415}%
\pgfsetfillcolor{currentfill}%
\pgfsetfillopacity{0.800000}%
\pgfsetlinewidth{0.000000pt}%
\definecolor{currentstroke}{rgb}{0.000000,0.000000,0.000000}%
\pgfsetstrokecolor{currentstroke}%
\pgfsetdash{}{0pt}%
\pgfpathmoveto{\pgfqpoint{3.732187in}{1.944297in}}%
\pgfpathlineto{\pgfqpoint{3.745844in}{1.944153in}}%
\pgfpathlineto{\pgfqpoint{3.759507in}{1.944207in}}%
\pgfpathlineto{\pgfqpoint{3.773178in}{1.944460in}}%
\pgfpathlineto{\pgfqpoint{3.786856in}{1.944909in}}%
\pgfpathlineto{\pgfqpoint{3.794898in}{1.955799in}}%
\pgfpathlineto{\pgfqpoint{3.802935in}{1.966679in}}%
\pgfpathlineto{\pgfqpoint{3.810967in}{1.977545in}}%
\pgfpathlineto{\pgfqpoint{3.818993in}{1.988397in}}%
\pgfpathlineto{\pgfqpoint{3.805323in}{1.987766in}}%
\pgfpathlineto{\pgfqpoint{3.791661in}{1.987332in}}%
\pgfpathlineto{\pgfqpoint{3.778006in}{1.987096in}}%
\pgfpathlineto{\pgfqpoint{3.764359in}{1.987059in}}%
\pgfpathlineto{\pgfqpoint{3.756323in}{1.976376in}}%
\pgfpathlineto{\pgfqpoint{3.748283in}{1.965688in}}%
\pgfpathlineto{\pgfqpoint{3.740238in}{1.954994in}}%
\pgfpathlineto{\pgfqpoint{3.732187in}{1.944297in}}%
\pgfpathclose%
\pgfusepath{fill}%
\end{pgfscope}%
\begin{pgfscope}%
\pgfpathrectangle{\pgfqpoint{1.150000in}{0.150000in}}{\pgfqpoint{5.700000in}{5.700000in}}%
\pgfusepath{clip}%
\pgfsetbuttcap%
\pgfsetroundjoin%
\definecolor{currentfill}{rgb}{0.273809,0.031497,0.358853}%
\pgfsetfillcolor{currentfill}%
\pgfsetfillopacity{0.800000}%
\pgfsetlinewidth{0.000000pt}%
\definecolor{currentstroke}{rgb}{0.000000,0.000000,0.000000}%
\pgfsetstrokecolor{currentstroke}%
\pgfsetdash{}{0pt}%
\pgfpathmoveto{\pgfqpoint{3.187824in}{1.858829in}}%
\pgfpathlineto{\pgfqpoint{3.201443in}{1.851433in}}%
\pgfpathlineto{\pgfqpoint{3.215063in}{1.844259in}}%
\pgfpathlineto{\pgfqpoint{3.228684in}{1.837308in}}%
\pgfpathlineto{\pgfqpoint{3.242305in}{1.830577in}}%
\pgfpathlineto{\pgfqpoint{3.250562in}{1.838799in}}%
\pgfpathlineto{\pgfqpoint{3.258812in}{1.847112in}}%
\pgfpathlineto{\pgfqpoint{3.267053in}{1.855513in}}%
\pgfpathlineto{\pgfqpoint{3.275288in}{1.863998in}}%
\pgfpathlineto{\pgfqpoint{3.261685in}{1.870389in}}%
\pgfpathlineto{\pgfqpoint{3.248084in}{1.877000in}}%
\pgfpathlineto{\pgfqpoint{3.234485in}{1.883833in}}%
\pgfpathlineto{\pgfqpoint{3.220886in}{1.890888in}}%
\pgfpathlineto{\pgfqpoint{3.212632in}{1.882732in}}%
\pgfpathlineto{\pgfqpoint{3.204371in}{1.874668in}}%
\pgfpathlineto{\pgfqpoint{3.196102in}{1.866699in}}%
\pgfpathlineto{\pgfqpoint{3.187824in}{1.858829in}}%
\pgfpathclose%
\pgfusepath{fill}%
\end{pgfscope}%
\begin{pgfscope}%
\pgfpathrectangle{\pgfqpoint{1.150000in}{0.150000in}}{\pgfqpoint{5.700000in}{5.700000in}}%
\pgfusepath{clip}%
\pgfsetbuttcap%
\pgfsetroundjoin%
\definecolor{currentfill}{rgb}{0.283072,0.130895,0.449241}%
\pgfsetfillcolor{currentfill}%
\pgfsetfillopacity{0.800000}%
\pgfsetlinewidth{0.000000pt}%
\definecolor{currentstroke}{rgb}{0.000000,0.000000,0.000000}%
\pgfsetstrokecolor{currentstroke}%
\pgfsetdash{}{0pt}%
\pgfpathmoveto{\pgfqpoint{3.905777in}{2.036724in}}%
\pgfpathlineto{\pgfqpoint{3.919481in}{2.038483in}}%
\pgfpathlineto{\pgfqpoint{3.933192in}{2.040436in}}%
\pgfpathlineto{\pgfqpoint{3.946913in}{2.042582in}}%
\pgfpathlineto{\pgfqpoint{3.960643in}{2.044922in}}%
\pgfpathlineto{\pgfqpoint{3.968630in}{2.055946in}}%
\pgfpathlineto{\pgfqpoint{3.976612in}{2.066933in}}%
\pgfpathlineto{\pgfqpoint{3.984589in}{2.077883in}}%
\pgfpathlineto{\pgfqpoint{3.992562in}{2.088794in}}%
\pgfpathlineto{\pgfqpoint{3.978838in}{2.086336in}}%
\pgfpathlineto{\pgfqpoint{3.965124in}{2.084071in}}%
\pgfpathlineto{\pgfqpoint{3.951419in}{2.082000in}}%
\pgfpathlineto{\pgfqpoint{3.937723in}{2.080123in}}%
\pgfpathlineto{\pgfqpoint{3.929744in}{2.069318in}}%
\pgfpathlineto{\pgfqpoint{3.921760in}{2.058482in}}%
\pgfpathlineto{\pgfqpoint{3.913771in}{2.047617in}}%
\pgfpathlineto{\pgfqpoint{3.905777in}{2.036724in}}%
\pgfpathclose%
\pgfusepath{fill}%
\end{pgfscope}%
\begin{pgfscope}%
\pgfpathrectangle{\pgfqpoint{1.150000in}{0.150000in}}{\pgfqpoint{5.700000in}{5.700000in}}%
\pgfusepath{clip}%
\pgfsetbuttcap%
\pgfsetroundjoin%
\definecolor{currentfill}{rgb}{0.279566,0.067836,0.391917}%
\pgfsetfillcolor{currentfill}%
\pgfsetfillopacity{0.800000}%
\pgfsetlinewidth{0.000000pt}%
\definecolor{currentstroke}{rgb}{0.000000,0.000000,0.000000}%
\pgfsetstrokecolor{currentstroke}%
\pgfsetdash{}{0pt}%
\pgfpathmoveto{\pgfqpoint{3.645334in}{1.904930in}}%
\pgfpathlineto{\pgfqpoint{3.658974in}{1.903773in}}%
\pgfpathlineto{\pgfqpoint{3.672621in}{1.902818in}}%
\pgfpathlineto{\pgfqpoint{3.686273in}{1.902062in}}%
\pgfpathlineto{\pgfqpoint{3.699932in}{1.901506in}}%
\pgfpathlineto{\pgfqpoint{3.708004in}{1.912201in}}%
\pgfpathlineto{\pgfqpoint{3.716070in}{1.922899in}}%
\pgfpathlineto{\pgfqpoint{3.724131in}{1.933598in}}%
\pgfpathlineto{\pgfqpoint{3.732187in}{1.944297in}}%
\pgfpathlineto{\pgfqpoint{3.718538in}{1.944640in}}%
\pgfpathlineto{\pgfqpoint{3.704895in}{1.945182in}}%
\pgfpathlineto{\pgfqpoint{3.691258in}{1.945925in}}%
\pgfpathlineto{\pgfqpoint{3.677628in}{1.946868in}}%
\pgfpathlineto{\pgfqpoint{3.669563in}{1.936370in}}%
\pgfpathlineto{\pgfqpoint{3.661492in}{1.925880in}}%
\pgfpathlineto{\pgfqpoint{3.653416in}{1.915400in}}%
\pgfpathlineto{\pgfqpoint{3.645334in}{1.904930in}}%
\pgfpathclose%
\pgfusepath{fill}%
\end{pgfscope}%
\begin{pgfscope}%
\pgfpathrectangle{\pgfqpoint{1.150000in}{0.150000in}}{\pgfqpoint{5.700000in}{5.700000in}}%
\pgfusepath{clip}%
\pgfsetbuttcap%
\pgfsetroundjoin%
\definecolor{currentfill}{rgb}{0.272594,0.025563,0.353093}%
\pgfsetfillcolor{currentfill}%
\pgfsetfillopacity{0.800000}%
\pgfsetlinewidth{0.000000pt}%
\definecolor{currentstroke}{rgb}{0.000000,0.000000,0.000000}%
\pgfsetstrokecolor{currentstroke}%
\pgfsetdash{}{0pt}%
\pgfpathmoveto{\pgfqpoint{3.329715in}{1.840619in}}%
\pgfpathlineto{\pgfqpoint{3.343328in}{1.835314in}}%
\pgfpathlineto{\pgfqpoint{3.356942in}{1.830224in}}%
\pgfpathlineto{\pgfqpoint{3.370560in}{1.825346in}}%
\pgfpathlineto{\pgfqpoint{3.384180in}{1.820681in}}%
\pgfpathlineto{\pgfqpoint{3.392373in}{1.829888in}}%
\pgfpathlineto{\pgfqpoint{3.400559in}{1.839158in}}%
\pgfpathlineto{\pgfqpoint{3.408738in}{1.848489in}}%
\pgfpathlineto{\pgfqpoint{3.416911in}{1.857876in}}%
\pgfpathlineto{\pgfqpoint{3.403307in}{1.862233in}}%
\pgfpathlineto{\pgfqpoint{3.389705in}{1.866802in}}%
\pgfpathlineto{\pgfqpoint{3.376107in}{1.871585in}}%
\pgfpathlineto{\pgfqpoint{3.362511in}{1.876581in}}%
\pgfpathlineto{\pgfqpoint{3.354322in}{1.867490in}}%
\pgfpathlineto{\pgfqpoint{3.346127in}{1.858464in}}%
\pgfpathlineto{\pgfqpoint{3.337925in}{1.849506in}}%
\pgfpathlineto{\pgfqpoint{3.329715in}{1.840619in}}%
\pgfpathclose%
\pgfusepath{fill}%
\end{pgfscope}%
\begin{pgfscope}%
\pgfpathrectangle{\pgfqpoint{1.150000in}{0.150000in}}{\pgfqpoint{5.700000in}{5.700000in}}%
\pgfusepath{clip}%
\pgfsetbuttcap%
\pgfsetroundjoin%
\definecolor{currentfill}{rgb}{0.190631,0.407061,0.556089}%
\pgfsetfillcolor{currentfill}%
\pgfsetfillopacity{0.800000}%
\pgfsetlinewidth{0.000000pt}%
\definecolor{currentstroke}{rgb}{0.000000,0.000000,0.000000}%
\pgfsetstrokecolor{currentstroke}%
\pgfsetdash{}{0pt}%
\pgfpathmoveto{\pgfqpoint{4.837212in}{2.719551in}}%
\pgfpathlineto{\pgfqpoint{4.851324in}{2.728086in}}%
\pgfpathlineto{\pgfqpoint{4.865452in}{2.736802in}}%
\pgfpathlineto{\pgfqpoint{4.879595in}{2.745701in}}%
\pgfpathlineto{\pgfqpoint{4.893755in}{2.754781in}}%
\pgfpathlineto{\pgfqpoint{4.901395in}{2.762011in}}%
\pgfpathlineto{\pgfqpoint{4.909029in}{2.769180in}}%
\pgfpathlineto{\pgfqpoint{4.916656in}{2.776290in}}%
\pgfpathlineto{\pgfqpoint{4.924277in}{2.783346in}}%
\pgfpathlineto{\pgfqpoint{4.910129in}{2.774541in}}%
\pgfpathlineto{\pgfqpoint{4.895997in}{2.765917in}}%
\pgfpathlineto{\pgfqpoint{4.881881in}{2.757475in}}%
\pgfpathlineto{\pgfqpoint{4.867781in}{2.749214in}}%
\pgfpathlineto{\pgfqpoint{4.860148in}{2.741872in}}%
\pgfpathlineto{\pgfqpoint{4.852509in}{2.734483in}}%
\pgfpathlineto{\pgfqpoint{4.844864in}{2.727044in}}%
\pgfpathlineto{\pgfqpoint{4.837212in}{2.719551in}}%
\pgfpathclose%
\pgfusepath{fill}%
\end{pgfscope}%
\begin{pgfscope}%
\pgfpathrectangle{\pgfqpoint{1.150000in}{0.150000in}}{\pgfqpoint{5.700000in}{5.700000in}}%
\pgfusepath{clip}%
\pgfsetbuttcap%
\pgfsetroundjoin%
\definecolor{currentfill}{rgb}{0.137770,0.537492,0.554906}%
\pgfsetfillcolor{currentfill}%
\pgfsetfillopacity{0.800000}%
\pgfsetlinewidth{0.000000pt}%
\definecolor{currentstroke}{rgb}{0.000000,0.000000,0.000000}%
\pgfsetstrokecolor{currentstroke}%
\pgfsetdash{}{0pt}%
\pgfpathmoveto{\pgfqpoint{5.389835in}{3.113546in}}%
\pgfpathlineto{\pgfqpoint{5.404240in}{3.123767in}}%
\pgfpathlineto{\pgfqpoint{5.418664in}{3.134166in}}%
\pgfpathlineto{\pgfqpoint{5.433106in}{3.144742in}}%
\pgfpathlineto{\pgfqpoint{5.447567in}{3.155497in}}%
\pgfpathlineto{\pgfqpoint{5.454919in}{3.159561in}}%
\pgfpathlineto{\pgfqpoint{5.462266in}{3.163641in}}%
\pgfpathlineto{\pgfqpoint{5.469606in}{3.167740in}}%
\pgfpathlineto{\pgfqpoint{5.476941in}{3.171866in}}%
\pgfpathlineto{\pgfqpoint{5.462504in}{3.161624in}}%
\pgfpathlineto{\pgfqpoint{5.448085in}{3.151559in}}%
\pgfpathlineto{\pgfqpoint{5.433684in}{3.141670in}}%
\pgfpathlineto{\pgfqpoint{5.419302in}{3.131959in}}%
\pgfpathlineto{\pgfqpoint{5.411943in}{3.127311in}}%
\pgfpathlineto{\pgfqpoint{5.404579in}{3.122697in}}%
\pgfpathlineto{\pgfqpoint{5.397210in}{3.118110in}}%
\pgfpathlineto{\pgfqpoint{5.389835in}{3.113546in}}%
\pgfpathclose%
\pgfusepath{fill}%
\end{pgfscope}%
\begin{pgfscope}%
\pgfpathrectangle{\pgfqpoint{1.150000in}{0.150000in}}{\pgfqpoint{5.700000in}{5.700000in}}%
\pgfusepath{clip}%
\pgfsetbuttcap%
\pgfsetroundjoin%
\definecolor{currentfill}{rgb}{0.281412,0.155834,0.469201}%
\pgfsetfillcolor{currentfill}%
\pgfsetfillopacity{0.800000}%
\pgfsetlinewidth{0.000000pt}%
\definecolor{currentstroke}{rgb}{0.000000,0.000000,0.000000}%
\pgfsetstrokecolor{currentstroke}%
\pgfsetdash{}{0pt}%
\pgfpathmoveto{\pgfqpoint{3.992562in}{2.088794in}}%
\pgfpathlineto{\pgfqpoint{4.006294in}{2.091445in}}%
\pgfpathlineto{\pgfqpoint{4.020036in}{2.094288in}}%
\pgfpathlineto{\pgfqpoint{4.033788in}{2.097323in}}%
\pgfpathlineto{\pgfqpoint{4.047550in}{2.100549in}}%
\pgfpathlineto{\pgfqpoint{4.055511in}{2.111520in}}%
\pgfpathlineto{\pgfqpoint{4.063467in}{2.122444in}}%
\pgfpathlineto{\pgfqpoint{4.071418in}{2.133320in}}%
\pgfpathlineto{\pgfqpoint{4.079364in}{2.144148in}}%
\pgfpathlineto{\pgfqpoint{4.065609in}{2.140835in}}%
\pgfpathlineto{\pgfqpoint{4.051863in}{2.137714in}}%
\pgfpathlineto{\pgfqpoint{4.038127in}{2.134784in}}%
\pgfpathlineto{\pgfqpoint{4.024401in}{2.132047in}}%
\pgfpathlineto{\pgfqpoint{4.016448in}{2.121294in}}%
\pgfpathlineto{\pgfqpoint{4.008491in}{2.110500in}}%
\pgfpathlineto{\pgfqpoint{4.000529in}{2.099667in}}%
\pgfpathlineto{\pgfqpoint{3.992562in}{2.088794in}}%
\pgfpathclose%
\pgfusepath{fill}%
\end{pgfscope}%
\begin{pgfscope}%
\pgfpathrectangle{\pgfqpoint{1.150000in}{0.150000in}}{\pgfqpoint{5.700000in}{5.700000in}}%
\pgfusepath{clip}%
\pgfsetbuttcap%
\pgfsetroundjoin%
\definecolor{currentfill}{rgb}{0.239346,0.300855,0.540844}%
\pgfsetfillcolor{currentfill}%
\pgfsetfillopacity{0.800000}%
\pgfsetlinewidth{0.000000pt}%
\definecolor{currentstroke}{rgb}{0.000000,0.000000,0.000000}%
\pgfsetstrokecolor{currentstroke}%
\pgfsetdash{}{0pt}%
\pgfpathmoveto{\pgfqpoint{4.458325in}{2.428680in}}%
\pgfpathlineto{\pgfqpoint{4.472250in}{2.435144in}}%
\pgfpathlineto{\pgfqpoint{4.486188in}{2.441793in}}%
\pgfpathlineto{\pgfqpoint{4.500139in}{2.448628in}}%
\pgfpathlineto{\pgfqpoint{4.514103in}{2.455649in}}%
\pgfpathlineto{\pgfqpoint{4.521906in}{2.465048in}}%
\pgfpathlineto{\pgfqpoint{4.529703in}{2.474372in}}%
\pgfpathlineto{\pgfqpoint{4.537494in}{2.483621in}}%
\pgfpathlineto{\pgfqpoint{4.545279in}{2.492797in}}%
\pgfpathlineto{\pgfqpoint{4.531322in}{2.485885in}}%
\pgfpathlineto{\pgfqpoint{4.517377in}{2.479158in}}%
\pgfpathlineto{\pgfqpoint{4.503447in}{2.472617in}}%
\pgfpathlineto{\pgfqpoint{4.489529in}{2.466260in}}%
\pgfpathlineto{\pgfqpoint{4.481737in}{2.456964in}}%
\pgfpathlineto{\pgfqpoint{4.473939in}{2.447603in}}%
\pgfpathlineto{\pgfqpoint{4.466135in}{2.438175in}}%
\pgfpathlineto{\pgfqpoint{4.458325in}{2.428680in}}%
\pgfpathclose%
\pgfusepath{fill}%
\end{pgfscope}%
\begin{pgfscope}%
\pgfpathrectangle{\pgfqpoint{1.150000in}{0.150000in}}{\pgfqpoint{5.700000in}{5.700000in}}%
\pgfusepath{clip}%
\pgfsetbuttcap%
\pgfsetroundjoin%
\definecolor{currentfill}{rgb}{0.277018,0.050344,0.375715}%
\pgfsetfillcolor{currentfill}%
\pgfsetfillopacity{0.800000}%
\pgfsetlinewidth{0.000000pt}%
\definecolor{currentstroke}{rgb}{0.000000,0.000000,0.000000}%
\pgfsetstrokecolor{currentstroke}%
\pgfsetdash{}{0pt}%
\pgfpathmoveto{\pgfqpoint{3.558404in}{1.870828in}}%
\pgfpathlineto{\pgfqpoint{3.572033in}{1.868617in}}%
\pgfpathlineto{\pgfqpoint{3.585668in}{1.866609in}}%
\pgfpathlineto{\pgfqpoint{3.599307in}{1.864804in}}%
\pgfpathlineto{\pgfqpoint{3.612953in}{1.863201in}}%
\pgfpathlineto{\pgfqpoint{3.621056in}{1.873607in}}%
\pgfpathlineto{\pgfqpoint{3.629154in}{1.884031in}}%
\pgfpathlineto{\pgfqpoint{3.637247in}{1.894473in}}%
\pgfpathlineto{\pgfqpoint{3.645334in}{1.904930in}}%
\pgfpathlineto{\pgfqpoint{3.631700in}{1.906288in}}%
\pgfpathlineto{\pgfqpoint{3.618071in}{1.907848in}}%
\pgfpathlineto{\pgfqpoint{3.604448in}{1.909612in}}%
\pgfpathlineto{\pgfqpoint{3.590831in}{1.911579in}}%
\pgfpathlineto{\pgfqpoint{3.582733in}{1.901355in}}%
\pgfpathlineto{\pgfqpoint{3.574629in}{1.891154in}}%
\pgfpathlineto{\pgfqpoint{3.566520in}{1.880977in}}%
\pgfpathlineto{\pgfqpoint{3.558404in}{1.870828in}}%
\pgfpathclose%
\pgfusepath{fill}%
\end{pgfscope}%
\begin{pgfscope}%
\pgfpathrectangle{\pgfqpoint{1.150000in}{0.150000in}}{\pgfqpoint{5.700000in}{5.700000in}}%
\pgfusepath{clip}%
\pgfsetbuttcap%
\pgfsetroundjoin%
\definecolor{currentfill}{rgb}{0.278012,0.180367,0.486697}%
\pgfsetfillcolor{currentfill}%
\pgfsetfillopacity{0.800000}%
\pgfsetlinewidth{0.000000pt}%
\definecolor{currentstroke}{rgb}{0.000000,0.000000,0.000000}%
\pgfsetstrokecolor{currentstroke}%
\pgfsetdash{}{0pt}%
\pgfpathmoveto{\pgfqpoint{2.682810in}{2.187809in}}%
\pgfpathlineto{\pgfqpoint{2.696588in}{2.171700in}}%
\pgfpathlineto{\pgfqpoint{2.710358in}{2.155866in}}%
\pgfpathlineto{\pgfqpoint{2.724122in}{2.140307in}}%
\pgfpathlineto{\pgfqpoint{2.737878in}{2.125020in}}%
\pgfpathlineto{\pgfqpoint{2.746428in}{2.129020in}}%
\pgfpathlineto{\pgfqpoint{2.754965in}{2.133204in}}%
\pgfpathlineto{\pgfqpoint{2.763489in}{2.137570in}}%
\pgfpathlineto{\pgfqpoint{2.772002in}{2.142111in}}%
\pgfpathlineto{\pgfqpoint{2.758279in}{2.156982in}}%
\pgfpathlineto{\pgfqpoint{2.744550in}{2.172124in}}%
\pgfpathlineto{\pgfqpoint{2.730814in}{2.187539in}}%
\pgfpathlineto{\pgfqpoint{2.717071in}{2.203230in}}%
\pgfpathlineto{\pgfqpoint{2.708525in}{2.199093in}}%
\pgfpathlineto{\pgfqpoint{2.699966in}{2.195142in}}%
\pgfpathlineto{\pgfqpoint{2.691395in}{2.191379in}}%
\pgfpathlineto{\pgfqpoint{2.682810in}{2.187809in}}%
\pgfpathclose%
\pgfusepath{fill}%
\end{pgfscope}%
\begin{pgfscope}%
\pgfpathrectangle{\pgfqpoint{1.150000in}{0.150000in}}{\pgfqpoint{5.700000in}{5.700000in}}%
\pgfusepath{clip}%
\pgfsetbuttcap%
\pgfsetroundjoin%
\definecolor{currentfill}{rgb}{0.271828,0.209303,0.504434}%
\pgfsetfillcolor{currentfill}%
\pgfsetfillopacity{0.800000}%
\pgfsetlinewidth{0.000000pt}%
\definecolor{currentstroke}{rgb}{0.000000,0.000000,0.000000}%
\pgfsetstrokecolor{currentstroke}%
\pgfsetdash{}{0pt}%
\pgfpathmoveto{\pgfqpoint{2.627621in}{2.255056in}}%
\pgfpathlineto{\pgfqpoint{2.641431in}{2.237818in}}%
\pgfpathlineto{\pgfqpoint{2.655232in}{2.220866in}}%
\pgfpathlineto{\pgfqpoint{2.669025in}{2.204197in}}%
\pgfpathlineto{\pgfqpoint{2.682810in}{2.187809in}}%
\pgfpathlineto{\pgfqpoint{2.691395in}{2.191379in}}%
\pgfpathlineto{\pgfqpoint{2.699966in}{2.195142in}}%
\pgfpathlineto{\pgfqpoint{2.708525in}{2.199093in}}%
\pgfpathlineto{\pgfqpoint{2.717071in}{2.203230in}}%
\pgfpathlineto{\pgfqpoint{2.703321in}{2.219199in}}%
\pgfpathlineto{\pgfqpoint{2.689564in}{2.235448in}}%
\pgfpathlineto{\pgfqpoint{2.675799in}{2.251979in}}%
\pgfpathlineto{\pgfqpoint{2.662026in}{2.268795in}}%
\pgfpathlineto{\pgfqpoint{2.653444in}{2.265066in}}%
\pgfpathlineto{\pgfqpoint{2.644850in}{2.261530in}}%
\pgfpathlineto{\pgfqpoint{2.636242in}{2.258192in}}%
\pgfpathlineto{\pgfqpoint{2.627621in}{2.255056in}}%
\pgfpathclose%
\pgfusepath{fill}%
\end{pgfscope}%
\begin{pgfscope}%
\pgfpathrectangle{\pgfqpoint{1.150000in}{0.150000in}}{\pgfqpoint{5.700000in}{5.700000in}}%
\pgfusepath{clip}%
\pgfsetbuttcap%
\pgfsetroundjoin%
\definecolor{currentfill}{rgb}{0.277941,0.056324,0.381191}%
\pgfsetfillcolor{currentfill}%
\pgfsetfillopacity{0.800000}%
\pgfsetlinewidth{0.000000pt}%
\definecolor{currentstroke}{rgb}{0.000000,0.000000,0.000000}%
\pgfsetstrokecolor{currentstroke}%
\pgfsetdash{}{0pt}%
\pgfpathmoveto{\pgfqpoint{3.045498in}{1.898739in}}%
\pgfpathlineto{\pgfqpoint{3.059143in}{1.889139in}}%
\pgfpathlineto{\pgfqpoint{3.072787in}{1.879773in}}%
\pgfpathlineto{\pgfqpoint{3.086429in}{1.870639in}}%
\pgfpathlineto{\pgfqpoint{3.100071in}{1.861736in}}%
\pgfpathlineto{\pgfqpoint{3.108404in}{1.868792in}}%
\pgfpathlineto{\pgfqpoint{3.116729in}{1.875968in}}%
\pgfpathlineto{\pgfqpoint{3.125045in}{1.883262in}}%
\pgfpathlineto{\pgfqpoint{3.133352in}{1.890668in}}%
\pgfpathlineto{\pgfqpoint{3.119733in}{1.899197in}}%
\pgfpathlineto{\pgfqpoint{3.106114in}{1.907957in}}%
\pgfpathlineto{\pgfqpoint{3.092495in}{1.916948in}}%
\pgfpathlineto{\pgfqpoint{3.078874in}{1.926172in}}%
\pgfpathlineto{\pgfqpoint{3.070544in}{1.919129in}}%
\pgfpathlineto{\pgfqpoint{3.062205in}{1.912206in}}%
\pgfpathlineto{\pgfqpoint{3.053856in}{1.905408in}}%
\pgfpathlineto{\pgfqpoint{3.045498in}{1.898739in}}%
\pgfpathclose%
\pgfusepath{fill}%
\end{pgfscope}%
\begin{pgfscope}%
\pgfpathrectangle{\pgfqpoint{1.150000in}{0.150000in}}{\pgfqpoint{5.700000in}{5.700000in}}%
\pgfusepath{clip}%
\pgfsetbuttcap%
\pgfsetroundjoin%
\definecolor{currentfill}{rgb}{0.194100,0.399323,0.555565}%
\pgfsetfillcolor{currentfill}%
\pgfsetfillopacity{0.800000}%
\pgfsetlinewidth{0.000000pt}%
\definecolor{currentstroke}{rgb}{0.000000,0.000000,0.000000}%
\pgfsetstrokecolor{currentstroke}%
\pgfsetdash{}{0pt}%
\pgfpathmoveto{\pgfqpoint{2.328287in}{2.764673in}}%
\pgfpathlineto{\pgfqpoint{2.342345in}{2.740200in}}%
\pgfpathlineto{\pgfqpoint{2.356387in}{2.716083in}}%
\pgfpathlineto{\pgfqpoint{2.370413in}{2.692318in}}%
\pgfpathlineto{\pgfqpoint{2.384425in}{2.668901in}}%
\pgfpathlineto{\pgfqpoint{2.393191in}{2.670777in}}%
\pgfpathlineto{\pgfqpoint{2.401941in}{2.672881in}}%
\pgfpathlineto{\pgfqpoint{2.410675in}{2.675212in}}%
\pgfpathlineto{\pgfqpoint{2.419394in}{2.677764in}}%
\pgfpathlineto{\pgfqpoint{2.405426in}{2.700775in}}%
\pgfpathlineto{\pgfqpoint{2.391443in}{2.724134in}}%
\pgfpathlineto{\pgfqpoint{2.377445in}{2.747844in}}%
\pgfpathlineto{\pgfqpoint{2.363431in}{2.771909in}}%
\pgfpathlineto{\pgfqpoint{2.354669in}{2.769750in}}%
\pgfpathlineto{\pgfqpoint{2.345891in}{2.767822in}}%
\pgfpathlineto{\pgfqpoint{2.337097in}{2.766129in}}%
\pgfpathlineto{\pgfqpoint{2.328287in}{2.764673in}}%
\pgfpathclose%
\pgfusepath{fill}%
\end{pgfscope}%
\begin{pgfscope}%
\pgfpathrectangle{\pgfqpoint{1.150000in}{0.150000in}}{\pgfqpoint{5.700000in}{5.700000in}}%
\pgfusepath{clip}%
\pgfsetbuttcap%
\pgfsetroundjoin%
\definecolor{currentfill}{rgb}{0.281412,0.155834,0.469201}%
\pgfsetfillcolor{currentfill}%
\pgfsetfillopacity{0.800000}%
\pgfsetlinewidth{0.000000pt}%
\definecolor{currentstroke}{rgb}{0.000000,0.000000,0.000000}%
\pgfsetstrokecolor{currentstroke}%
\pgfsetdash{}{0pt}%
\pgfpathmoveto{\pgfqpoint{2.737878in}{2.125020in}}%
\pgfpathlineto{\pgfqpoint{2.751629in}{2.110002in}}%
\pgfpathlineto{\pgfqpoint{2.765372in}{2.095253in}}%
\pgfpathlineto{\pgfqpoint{2.779110in}{2.080769in}}%
\pgfpathlineto{\pgfqpoint{2.792842in}{2.066549in}}%
\pgfpathlineto{\pgfqpoint{2.801358in}{2.070977in}}%
\pgfpathlineto{\pgfqpoint{2.809862in}{2.075581in}}%
\pgfpathlineto{\pgfqpoint{2.818354in}{2.080357in}}%
\pgfpathlineto{\pgfqpoint{2.826834in}{2.085302in}}%
\pgfpathlineto{\pgfqpoint{2.813134in}{2.099108in}}%
\pgfpathlineto{\pgfqpoint{2.799429in}{2.113176in}}%
\pgfpathlineto{\pgfqpoint{2.785718in}{2.127510in}}%
\pgfpathlineto{\pgfqpoint{2.772002in}{2.142111in}}%
\pgfpathlineto{\pgfqpoint{2.763489in}{2.137570in}}%
\pgfpathlineto{\pgfqpoint{2.754965in}{2.133204in}}%
\pgfpathlineto{\pgfqpoint{2.746428in}{2.129020in}}%
\pgfpathlineto{\pgfqpoint{2.737878in}{2.125020in}}%
\pgfpathclose%
\pgfusepath{fill}%
\end{pgfscope}%
\begin{pgfscope}%
\pgfpathrectangle{\pgfqpoint{1.150000in}{0.150000in}}{\pgfqpoint{5.700000in}{5.700000in}}%
\pgfusepath{clip}%
\pgfsetbuttcap%
\pgfsetroundjoin%
\definecolor{currentfill}{rgb}{0.129933,0.559582,0.551864}%
\pgfsetfillcolor{currentfill}%
\pgfsetfillopacity{0.800000}%
\pgfsetlinewidth{0.000000pt}%
\definecolor{currentstroke}{rgb}{0.000000,0.000000,0.000000}%
\pgfsetstrokecolor{currentstroke}%
\pgfsetdash{}{0pt}%
\pgfpathmoveto{\pgfqpoint{5.476941in}{3.171866in}}%
\pgfpathlineto{\pgfqpoint{5.491397in}{3.182286in}}%
\pgfpathlineto{\pgfqpoint{5.505871in}{3.192882in}}%
\pgfpathlineto{\pgfqpoint{5.520364in}{3.203656in}}%
\pgfpathlineto{\pgfqpoint{5.534877in}{3.214607in}}%
\pgfpathlineto{\pgfqpoint{5.542181in}{3.218231in}}%
\pgfpathlineto{\pgfqpoint{5.549479in}{3.221886in}}%
\pgfpathlineto{\pgfqpoint{5.556773in}{3.225576in}}%
\pgfpathlineto{\pgfqpoint{5.564061in}{3.229309in}}%
\pgfpathlineto{\pgfqpoint{5.549574in}{3.218904in}}%
\pgfpathlineto{\pgfqpoint{5.535107in}{3.208675in}}%
\pgfpathlineto{\pgfqpoint{5.520658in}{3.198623in}}%
\pgfpathlineto{\pgfqpoint{5.506227in}{3.188747in}}%
\pgfpathlineto{\pgfqpoint{5.498913in}{3.184458in}}%
\pgfpathlineto{\pgfqpoint{5.491594in}{3.180219in}}%
\pgfpathlineto{\pgfqpoint{5.484270in}{3.176024in}}%
\pgfpathlineto{\pgfqpoint{5.476941in}{3.171866in}}%
\pgfpathclose%
\pgfusepath{fill}%
\end{pgfscope}%
\begin{pgfscope}%
\pgfpathrectangle{\pgfqpoint{1.150000in}{0.150000in}}{\pgfqpoint{5.700000in}{5.700000in}}%
\pgfusepath{clip}%
\pgfsetbuttcap%
\pgfsetroundjoin%
\definecolor{currentfill}{rgb}{0.277134,0.185228,0.489898}%
\pgfsetfillcolor{currentfill}%
\pgfsetfillopacity{0.800000}%
\pgfsetlinewidth{0.000000pt}%
\definecolor{currentstroke}{rgb}{0.000000,0.000000,0.000000}%
\pgfsetstrokecolor{currentstroke}%
\pgfsetdash{}{0pt}%
\pgfpathmoveto{\pgfqpoint{4.079364in}{2.144148in}}%
\pgfpathlineto{\pgfqpoint{4.093130in}{2.147652in}}%
\pgfpathlineto{\pgfqpoint{4.106907in}{2.151346in}}%
\pgfpathlineto{\pgfqpoint{4.120693in}{2.155231in}}%
\pgfpathlineto{\pgfqpoint{4.134490in}{2.159306in}}%
\pgfpathlineto{\pgfqpoint{4.142426in}{2.170152in}}%
\pgfpathlineto{\pgfqpoint{4.150356in}{2.180941in}}%
\pgfpathlineto{\pgfqpoint{4.158281in}{2.191673in}}%
\pgfpathlineto{\pgfqpoint{4.166201in}{2.202349in}}%
\pgfpathlineto{\pgfqpoint{4.152410in}{2.198219in}}%
\pgfpathlineto{\pgfqpoint{4.138629in}{2.194280in}}%
\pgfpathlineto{\pgfqpoint{4.124858in}{2.190531in}}%
\pgfpathlineto{\pgfqpoint{4.111098in}{2.186972in}}%
\pgfpathlineto{\pgfqpoint{4.103172in}{2.176340in}}%
\pgfpathlineto{\pgfqpoint{4.095241in}{2.165658in}}%
\pgfpathlineto{\pgfqpoint{4.087305in}{2.154927in}}%
\pgfpathlineto{\pgfqpoint{4.079364in}{2.144148in}}%
\pgfpathclose%
\pgfusepath{fill}%
\end{pgfscope}%
\begin{pgfscope}%
\pgfpathrectangle{\pgfqpoint{1.150000in}{0.150000in}}{\pgfqpoint{5.700000in}{5.700000in}}%
\pgfusepath{clip}%
\pgfsetbuttcap%
\pgfsetroundjoin%
\definecolor{currentfill}{rgb}{0.263663,0.237631,0.518762}%
\pgfsetfillcolor{currentfill}%
\pgfsetfillopacity{0.800000}%
\pgfsetlinewidth{0.000000pt}%
\definecolor{currentstroke}{rgb}{0.000000,0.000000,0.000000}%
\pgfsetstrokecolor{currentstroke}%
\pgfsetdash{}{0pt}%
\pgfpathmoveto{\pgfqpoint{2.572292in}{2.326908in}}%
\pgfpathlineto{\pgfqpoint{2.586139in}{2.308505in}}%
\pgfpathlineto{\pgfqpoint{2.599975in}{2.290397in}}%
\pgfpathlineto{\pgfqpoint{2.613803in}{2.272581in}}%
\pgfpathlineto{\pgfqpoint{2.627621in}{2.255056in}}%
\pgfpathlineto{\pgfqpoint{2.636242in}{2.258192in}}%
\pgfpathlineto{\pgfqpoint{2.644850in}{2.261530in}}%
\pgfpathlineto{\pgfqpoint{2.653444in}{2.265066in}}%
\pgfpathlineto{\pgfqpoint{2.662026in}{2.268795in}}%
\pgfpathlineto{\pgfqpoint{2.648244in}{2.285898in}}%
\pgfpathlineto{\pgfqpoint{2.634454in}{2.303290in}}%
\pgfpathlineto{\pgfqpoint{2.620655in}{2.320975in}}%
\pgfpathlineto{\pgfqpoint{2.606847in}{2.338954in}}%
\pgfpathlineto{\pgfqpoint{2.598230in}{2.335635in}}%
\pgfpathlineto{\pgfqpoint{2.589598in}{2.332519in}}%
\pgfpathlineto{\pgfqpoint{2.580952in}{2.329609in}}%
\pgfpathlineto{\pgfqpoint{2.572292in}{2.326908in}}%
\pgfpathclose%
\pgfusepath{fill}%
\end{pgfscope}%
\begin{pgfscope}%
\pgfpathrectangle{\pgfqpoint{1.150000in}{0.150000in}}{\pgfqpoint{5.700000in}{5.700000in}}%
\pgfusepath{clip}%
\pgfsetbuttcap%
\pgfsetroundjoin%
\definecolor{currentfill}{rgb}{0.180629,0.429975,0.557282}%
\pgfsetfillcolor{currentfill}%
\pgfsetfillopacity{0.800000}%
\pgfsetlinewidth{0.000000pt}%
\definecolor{currentstroke}{rgb}{0.000000,0.000000,0.000000}%
\pgfsetstrokecolor{currentstroke}%
\pgfsetdash{}{0pt}%
\pgfpathmoveto{\pgfqpoint{4.924277in}{2.783346in}}%
\pgfpathlineto{\pgfqpoint{4.938441in}{2.792333in}}%
\pgfpathlineto{\pgfqpoint{4.952621in}{2.801501in}}%
\pgfpathlineto{\pgfqpoint{4.966817in}{2.810851in}}%
\pgfpathlineto{\pgfqpoint{4.981029in}{2.820382in}}%
\pgfpathlineto{\pgfqpoint{4.988631in}{2.827091in}}%
\pgfpathlineto{\pgfqpoint{4.996225in}{2.833743in}}%
\pgfpathlineto{\pgfqpoint{5.003813in}{2.840344in}}%
\pgfpathlineto{\pgfqpoint{5.011394in}{2.846896in}}%
\pgfpathlineto{\pgfqpoint{4.997194in}{2.837674in}}%
\pgfpathlineto{\pgfqpoint{4.983011in}{2.828634in}}%
\pgfpathlineto{\pgfqpoint{4.968844in}{2.819774in}}%
\pgfpathlineto{\pgfqpoint{4.954694in}{2.811094in}}%
\pgfpathlineto{\pgfqpoint{4.947099in}{2.804222in}}%
\pgfpathlineto{\pgfqpoint{4.939498in}{2.797309in}}%
\pgfpathlineto{\pgfqpoint{4.931891in}{2.790351in}}%
\pgfpathlineto{\pgfqpoint{4.924277in}{2.783346in}}%
\pgfpathclose%
\pgfusepath{fill}%
\end{pgfscope}%
\begin{pgfscope}%
\pgfpathrectangle{\pgfqpoint{1.150000in}{0.150000in}}{\pgfqpoint{5.700000in}{5.700000in}}%
\pgfusepath{clip}%
\pgfsetbuttcap%
\pgfsetroundjoin%
\definecolor{currentfill}{rgb}{0.283072,0.130895,0.449241}%
\pgfsetfillcolor{currentfill}%
\pgfsetfillopacity{0.800000}%
\pgfsetlinewidth{0.000000pt}%
\definecolor{currentstroke}{rgb}{0.000000,0.000000,0.000000}%
\pgfsetstrokecolor{currentstroke}%
\pgfsetdash{}{0pt}%
\pgfpathmoveto{\pgfqpoint{2.792842in}{2.066549in}}%
\pgfpathlineto{\pgfqpoint{2.806569in}{2.052591in}}%
\pgfpathlineto{\pgfqpoint{2.820290in}{2.038892in}}%
\pgfpathlineto{\pgfqpoint{2.834007in}{2.025452in}}%
\pgfpathlineto{\pgfqpoint{2.847718in}{2.012268in}}%
\pgfpathlineto{\pgfqpoint{2.856202in}{2.017122in}}%
\pgfpathlineto{\pgfqpoint{2.864674in}{2.022143in}}%
\pgfpathlineto{\pgfqpoint{2.873135in}{2.027329in}}%
\pgfpathlineto{\pgfqpoint{2.881585in}{2.032674in}}%
\pgfpathlineto{\pgfqpoint{2.867904in}{2.045446in}}%
\pgfpathlineto{\pgfqpoint{2.854219in}{2.058473in}}%
\pgfpathlineto{\pgfqpoint{2.840529in}{2.071758in}}%
\pgfpathlineto{\pgfqpoint{2.826834in}{2.085302in}}%
\pgfpathlineto{\pgfqpoint{2.818354in}{2.080357in}}%
\pgfpathlineto{\pgfqpoint{2.809862in}{2.075581in}}%
\pgfpathlineto{\pgfqpoint{2.801358in}{2.070977in}}%
\pgfpathlineto{\pgfqpoint{2.792842in}{2.066549in}}%
\pgfpathclose%
\pgfusepath{fill}%
\end{pgfscope}%
\begin{pgfscope}%
\pgfpathrectangle{\pgfqpoint{1.150000in}{0.150000in}}{\pgfqpoint{5.700000in}{5.700000in}}%
\pgfusepath{clip}%
\pgfsetbuttcap%
\pgfsetroundjoin%
\definecolor{currentfill}{rgb}{0.274952,0.037752,0.364543}%
\pgfsetfillcolor{currentfill}%
\pgfsetfillopacity{0.800000}%
\pgfsetlinewidth{0.000000pt}%
\definecolor{currentstroke}{rgb}{0.000000,0.000000,0.000000}%
\pgfsetstrokecolor{currentstroke}%
\pgfsetdash{}{0pt}%
\pgfpathmoveto{\pgfqpoint{3.471366in}{1.842550in}}%
\pgfpathlineto{\pgfqpoint{3.484989in}{1.839240in}}%
\pgfpathlineto{\pgfqpoint{3.498617in}{1.836137in}}%
\pgfpathlineto{\pgfqpoint{3.512249in}{1.833239in}}%
\pgfpathlineto{\pgfqpoint{3.525885in}{1.830547in}}%
\pgfpathlineto{\pgfqpoint{3.534024in}{1.840565in}}%
\pgfpathlineto{\pgfqpoint{3.542157in}{1.850620in}}%
\pgfpathlineto{\pgfqpoint{3.550283in}{1.860708in}}%
\pgfpathlineto{\pgfqpoint{3.558404in}{1.870828in}}%
\pgfpathlineto{\pgfqpoint{3.544780in}{1.873244in}}%
\pgfpathlineto{\pgfqpoint{3.531161in}{1.875865in}}%
\pgfpathlineto{\pgfqpoint{3.517547in}{1.878693in}}%
\pgfpathlineto{\pgfqpoint{3.503937in}{1.881727in}}%
\pgfpathlineto{\pgfqpoint{3.495803in}{1.871871in}}%
\pgfpathlineto{\pgfqpoint{3.487663in}{1.862055in}}%
\pgfpathlineto{\pgfqpoint{3.479518in}{1.852280in}}%
\pgfpathlineto{\pgfqpoint{3.471366in}{1.842550in}}%
\pgfpathclose%
\pgfusepath{fill}%
\end{pgfscope}%
\begin{pgfscope}%
\pgfpathrectangle{\pgfqpoint{1.150000in}{0.150000in}}{\pgfqpoint{5.700000in}{5.700000in}}%
\pgfusepath{clip}%
\pgfsetbuttcap%
\pgfsetroundjoin%
\definecolor{currentfill}{rgb}{0.225863,0.330805,0.547314}%
\pgfsetfillcolor{currentfill}%
\pgfsetfillopacity{0.800000}%
\pgfsetlinewidth{0.000000pt}%
\definecolor{currentstroke}{rgb}{0.000000,0.000000,0.000000}%
\pgfsetstrokecolor{currentstroke}%
\pgfsetdash{}{0pt}%
\pgfpathmoveto{\pgfqpoint{4.545279in}{2.492797in}}%
\pgfpathlineto{\pgfqpoint{4.559251in}{2.499894in}}%
\pgfpathlineto{\pgfqpoint{4.573236in}{2.507176in}}%
\pgfpathlineto{\pgfqpoint{4.587236in}{2.514643in}}%
\pgfpathlineto{\pgfqpoint{4.601249in}{2.522294in}}%
\pgfpathlineto{\pgfqpoint{4.609021in}{2.531270in}}%
\pgfpathlineto{\pgfqpoint{4.616787in}{2.540169in}}%
\pgfpathlineto{\pgfqpoint{4.624546in}{2.548992in}}%
\pgfpathlineto{\pgfqpoint{4.632300in}{2.557742in}}%
\pgfpathlineto{\pgfqpoint{4.618294in}{2.550233in}}%
\pgfpathlineto{\pgfqpoint{4.604302in}{2.542908in}}%
\pgfpathlineto{\pgfqpoint{4.590324in}{2.535767in}}%
\pgfpathlineto{\pgfqpoint{4.576360in}{2.528811in}}%
\pgfpathlineto{\pgfqpoint{4.568599in}{2.519908in}}%
\pgfpathlineto{\pgfqpoint{4.560832in}{2.510939in}}%
\pgfpathlineto{\pgfqpoint{4.553059in}{2.501903in}}%
\pgfpathlineto{\pgfqpoint{4.545279in}{2.492797in}}%
\pgfpathclose%
\pgfusepath{fill}%
\end{pgfscope}%
\begin{pgfscope}%
\pgfpathrectangle{\pgfqpoint{1.150000in}{0.150000in}}{\pgfqpoint{5.700000in}{5.700000in}}%
\pgfusepath{clip}%
\pgfsetbuttcap%
\pgfsetroundjoin%
\definecolor{currentfill}{rgb}{0.124395,0.578002,0.548287}%
\pgfsetfillcolor{currentfill}%
\pgfsetfillopacity{0.800000}%
\pgfsetlinewidth{0.000000pt}%
\definecolor{currentstroke}{rgb}{0.000000,0.000000,0.000000}%
\pgfsetstrokecolor{currentstroke}%
\pgfsetdash{}{0pt}%
\pgfpathmoveto{\pgfqpoint{5.564061in}{3.229309in}}%
\pgfpathlineto{\pgfqpoint{5.578566in}{3.239890in}}%
\pgfpathlineto{\pgfqpoint{5.593090in}{3.250648in}}%
\pgfpathlineto{\pgfqpoint{5.607634in}{3.261583in}}%
\pgfpathlineto{\pgfqpoint{5.622197in}{3.272695in}}%
\pgfpathlineto{\pgfqpoint{5.629452in}{3.275908in}}%
\pgfpathlineto{\pgfqpoint{5.636703in}{3.279168in}}%
\pgfpathlineto{\pgfqpoint{5.643948in}{3.282481in}}%
\pgfpathlineto{\pgfqpoint{5.651189in}{3.285854in}}%
\pgfpathlineto{\pgfqpoint{5.636654in}{3.275323in}}%
\pgfpathlineto{\pgfqpoint{5.622139in}{3.264967in}}%
\pgfpathlineto{\pgfqpoint{5.607642in}{3.254788in}}%
\pgfpathlineto{\pgfqpoint{5.593164in}{3.244784in}}%
\pgfpathlineto{\pgfqpoint{5.585895in}{3.240821in}}%
\pgfpathlineto{\pgfqpoint{5.578622in}{3.236925in}}%
\pgfpathlineto{\pgfqpoint{5.571344in}{3.233089in}}%
\pgfpathlineto{\pgfqpoint{5.564061in}{3.229309in}}%
\pgfpathclose%
\pgfusepath{fill}%
\end{pgfscope}%
\begin{pgfscope}%
\pgfpathrectangle{\pgfqpoint{1.150000in}{0.150000in}}{\pgfqpoint{5.700000in}{5.700000in}}%
\pgfusepath{clip}%
\pgfsetbuttcap%
\pgfsetroundjoin%
\definecolor{currentfill}{rgb}{0.252194,0.269783,0.531579}%
\pgfsetfillcolor{currentfill}%
\pgfsetfillopacity{0.800000}%
\pgfsetlinewidth{0.000000pt}%
\definecolor{currentstroke}{rgb}{0.000000,0.000000,0.000000}%
\pgfsetstrokecolor{currentstroke}%
\pgfsetdash{}{0pt}%
\pgfpathmoveto{\pgfqpoint{2.516806in}{2.403527in}}%
\pgfpathlineto{\pgfqpoint{2.530694in}{2.383916in}}%
\pgfpathlineto{\pgfqpoint{2.544570in}{2.364611in}}%
\pgfpathlineto{\pgfqpoint{2.558436in}{2.345609in}}%
\pgfpathlineto{\pgfqpoint{2.572292in}{2.326908in}}%
\pgfpathlineto{\pgfqpoint{2.580952in}{2.329609in}}%
\pgfpathlineto{\pgfqpoint{2.589598in}{2.332519in}}%
\pgfpathlineto{\pgfqpoint{2.598230in}{2.335635in}}%
\pgfpathlineto{\pgfqpoint{2.606847in}{2.338954in}}%
\pgfpathlineto{\pgfqpoint{2.593030in}{2.357230in}}%
\pgfpathlineto{\pgfqpoint{2.579203in}{2.375806in}}%
\pgfpathlineto{\pgfqpoint{2.565366in}{2.394684in}}%
\pgfpathlineto{\pgfqpoint{2.551518in}{2.413866in}}%
\pgfpathlineto{\pgfqpoint{2.542862in}{2.410961in}}%
\pgfpathlineto{\pgfqpoint{2.534191in}{2.408267in}}%
\pgfpathlineto{\pgfqpoint{2.525506in}{2.405788in}}%
\pgfpathlineto{\pgfqpoint{2.516806in}{2.403527in}}%
\pgfpathclose%
\pgfusepath{fill}%
\end{pgfscope}%
\begin{pgfscope}%
\pgfpathrectangle{\pgfqpoint{1.150000in}{0.150000in}}{\pgfqpoint{5.700000in}{5.700000in}}%
\pgfusepath{clip}%
\pgfsetbuttcap%
\pgfsetroundjoin%
\definecolor{currentfill}{rgb}{0.270595,0.214069,0.507052}%
\pgfsetfillcolor{currentfill}%
\pgfsetfillopacity{0.800000}%
\pgfsetlinewidth{0.000000pt}%
\definecolor{currentstroke}{rgb}{0.000000,0.000000,0.000000}%
\pgfsetstrokecolor{currentstroke}%
\pgfsetdash{}{0pt}%
\pgfpathmoveto{\pgfqpoint{4.166201in}{2.202349in}}%
\pgfpathlineto{\pgfqpoint{4.180004in}{2.206668in}}%
\pgfpathlineto{\pgfqpoint{4.193818in}{2.211176in}}%
\pgfpathlineto{\pgfqpoint{4.207643in}{2.215873in}}%
\pgfpathlineto{\pgfqpoint{4.221479in}{2.220759in}}%
\pgfpathlineto{\pgfqpoint{4.229388in}{2.231411in}}%
\pgfpathlineto{\pgfqpoint{4.237292in}{2.241999in}}%
\pgfpathlineto{\pgfqpoint{4.245191in}{2.252522in}}%
\pgfpathlineto{\pgfqpoint{4.253085in}{2.262981in}}%
\pgfpathlineto{\pgfqpoint{4.239254in}{2.258073in}}%
\pgfpathlineto{\pgfqpoint{4.225434in}{2.253354in}}%
\pgfpathlineto{\pgfqpoint{4.211626in}{2.248823in}}%
\pgfpathlineto{\pgfqpoint{4.197829in}{2.244482in}}%
\pgfpathlineto{\pgfqpoint{4.189930in}{2.234034in}}%
\pgfpathlineto{\pgfqpoint{4.182026in}{2.223529in}}%
\pgfpathlineto{\pgfqpoint{4.174116in}{2.212967in}}%
\pgfpathlineto{\pgfqpoint{4.166201in}{2.202349in}}%
\pgfpathclose%
\pgfusepath{fill}%
\end{pgfscope}%
\begin{pgfscope}%
\pgfpathrectangle{\pgfqpoint{1.150000in}{0.150000in}}{\pgfqpoint{5.700000in}{5.700000in}}%
\pgfusepath{clip}%
\pgfsetbuttcap%
\pgfsetroundjoin%
\definecolor{currentfill}{rgb}{0.272594,0.025563,0.353093}%
\pgfsetfillcolor{currentfill}%
\pgfsetfillopacity{0.800000}%
\pgfsetlinewidth{0.000000pt}%
\definecolor{currentstroke}{rgb}{0.000000,0.000000,0.000000}%
\pgfsetstrokecolor{currentstroke}%
\pgfsetdash{}{0pt}%
\pgfpathmoveto{\pgfqpoint{3.242305in}{1.830577in}}%
\pgfpathlineto{\pgfqpoint{3.255928in}{1.824065in}}%
\pgfpathlineto{\pgfqpoint{3.269553in}{1.817772in}}%
\pgfpathlineto{\pgfqpoint{3.283179in}{1.811696in}}%
\pgfpathlineto{\pgfqpoint{3.296807in}{1.805836in}}%
\pgfpathlineto{\pgfqpoint{3.305045in}{1.814410in}}%
\pgfpathlineto{\pgfqpoint{3.313276in}{1.823068in}}%
\pgfpathlineto{\pgfqpoint{3.321499in}{1.831805in}}%
\pgfpathlineto{\pgfqpoint{3.329715in}{1.840619in}}%
\pgfpathlineto{\pgfqpoint{3.316105in}{1.846138in}}%
\pgfpathlineto{\pgfqpoint{3.302498in}{1.851874in}}%
\pgfpathlineto{\pgfqpoint{3.288892in}{1.857827in}}%
\pgfpathlineto{\pgfqpoint{3.275288in}{1.863998in}}%
\pgfpathlineto{\pgfqpoint{3.267053in}{1.855513in}}%
\pgfpathlineto{\pgfqpoint{3.258812in}{1.847112in}}%
\pgfpathlineto{\pgfqpoint{3.250562in}{1.838799in}}%
\pgfpathlineto{\pgfqpoint{3.242305in}{1.830577in}}%
\pgfpathclose%
\pgfusepath{fill}%
\end{pgfscope}%
\begin{pgfscope}%
\pgfpathrectangle{\pgfqpoint{1.150000in}{0.150000in}}{\pgfqpoint{5.700000in}{5.700000in}}%
\pgfusepath{clip}%
\pgfsetbuttcap%
\pgfsetroundjoin%
\definecolor{currentfill}{rgb}{0.283091,0.110553,0.431554}%
\pgfsetfillcolor{currentfill}%
\pgfsetfillopacity{0.800000}%
\pgfsetlinewidth{0.000000pt}%
\definecolor{currentstroke}{rgb}{0.000000,0.000000,0.000000}%
\pgfsetstrokecolor{currentstroke}%
\pgfsetdash{}{0pt}%
\pgfpathmoveto{\pgfqpoint{2.847718in}{2.012268in}}%
\pgfpathlineto{\pgfqpoint{2.861425in}{1.999339in}}%
\pgfpathlineto{\pgfqpoint{2.875128in}{1.986662in}}%
\pgfpathlineto{\pgfqpoint{2.888827in}{1.974235in}}%
\pgfpathlineto{\pgfqpoint{2.902522in}{1.962058in}}%
\pgfpathlineto{\pgfqpoint{2.910975in}{1.967336in}}%
\pgfpathlineto{\pgfqpoint{2.919417in}{1.972773in}}%
\pgfpathlineto{\pgfqpoint{2.927849in}{1.978365in}}%
\pgfpathlineto{\pgfqpoint{2.936270in}{1.984110in}}%
\pgfpathlineto{\pgfqpoint{2.922604in}{1.995876in}}%
\pgfpathlineto{\pgfqpoint{2.908935in}{2.007891in}}%
\pgfpathlineto{\pgfqpoint{2.895262in}{2.020157in}}%
\pgfpathlineto{\pgfqpoint{2.881585in}{2.032674in}}%
\pgfpathlineto{\pgfqpoint{2.873135in}{2.027329in}}%
\pgfpathlineto{\pgfqpoint{2.864674in}{2.022143in}}%
\pgfpathlineto{\pgfqpoint{2.856202in}{2.017122in}}%
\pgfpathlineto{\pgfqpoint{2.847718in}{2.012268in}}%
\pgfpathclose%
\pgfusepath{fill}%
\end{pgfscope}%
\begin{pgfscope}%
\pgfpathrectangle{\pgfqpoint{1.150000in}{0.150000in}}{\pgfqpoint{5.700000in}{5.700000in}}%
\pgfusepath{clip}%
\pgfsetbuttcap%
\pgfsetroundjoin%
\definecolor{currentfill}{rgb}{0.120565,0.596422,0.543611}%
\pgfsetfillcolor{currentfill}%
\pgfsetfillopacity{0.800000}%
\pgfsetlinewidth{0.000000pt}%
\definecolor{currentstroke}{rgb}{0.000000,0.000000,0.000000}%
\pgfsetstrokecolor{currentstroke}%
\pgfsetdash{}{0pt}%
\pgfpathmoveto{\pgfqpoint{5.651189in}{3.285854in}}%
\pgfpathlineto{\pgfqpoint{5.665743in}{3.296561in}}%
\pgfpathlineto{\pgfqpoint{5.680316in}{3.307444in}}%
\pgfpathlineto{\pgfqpoint{5.694909in}{3.318504in}}%
\pgfpathlineto{\pgfqpoint{5.709522in}{3.329739in}}%
\pgfpathlineto{\pgfqpoint{5.716728in}{3.332577in}}%
\pgfpathlineto{\pgfqpoint{5.723930in}{3.335479in}}%
\pgfpathlineto{\pgfqpoint{5.731128in}{3.338453in}}%
\pgfpathlineto{\pgfqpoint{5.738321in}{3.341506in}}%
\pgfpathlineto{\pgfqpoint{5.723739in}{3.330884in}}%
\pgfpathlineto{\pgfqpoint{5.709177in}{3.320438in}}%
\pgfpathlineto{\pgfqpoint{5.694634in}{3.310168in}}%
\pgfpathlineto{\pgfqpoint{5.680109in}{3.300072in}}%
\pgfpathlineto{\pgfqpoint{5.672885in}{3.296395in}}%
\pgfpathlineto{\pgfqpoint{5.665657in}{3.292804in}}%
\pgfpathlineto{\pgfqpoint{5.658425in}{3.289293in}}%
\pgfpathlineto{\pgfqpoint{5.651189in}{3.285854in}}%
\pgfpathclose%
\pgfusepath{fill}%
\end{pgfscope}%
\begin{pgfscope}%
\pgfpathrectangle{\pgfqpoint{1.150000in}{0.150000in}}{\pgfqpoint{5.700000in}{5.700000in}}%
\pgfusepath{clip}%
\pgfsetbuttcap%
\pgfsetroundjoin%
\definecolor{currentfill}{rgb}{0.171176,0.452530,0.557965}%
\pgfsetfillcolor{currentfill}%
\pgfsetfillopacity{0.800000}%
\pgfsetlinewidth{0.000000pt}%
\definecolor{currentstroke}{rgb}{0.000000,0.000000,0.000000}%
\pgfsetstrokecolor{currentstroke}%
\pgfsetdash{}{0pt}%
\pgfpathmoveto{\pgfqpoint{5.011394in}{2.846896in}}%
\pgfpathlineto{\pgfqpoint{5.025610in}{2.856299in}}%
\pgfpathlineto{\pgfqpoint{5.039842in}{2.865882in}}%
\pgfpathlineto{\pgfqpoint{5.054092in}{2.875646in}}%
\pgfpathlineto{\pgfqpoint{5.068358in}{2.885592in}}%
\pgfpathlineto{\pgfqpoint{5.075918in}{2.891768in}}%
\pgfpathlineto{\pgfqpoint{5.083472in}{2.897897in}}%
\pgfpathlineto{\pgfqpoint{5.091018in}{2.903980in}}%
\pgfpathlineto{\pgfqpoint{5.098558in}{2.910022in}}%
\pgfpathlineto{\pgfqpoint{5.084306in}{2.900421in}}%
\pgfpathlineto{\pgfqpoint{5.070071in}{2.891000in}}%
\pgfpathlineto{\pgfqpoint{5.055853in}{2.881758in}}%
\pgfpathlineto{\pgfqpoint{5.041652in}{2.872698in}}%
\pgfpathlineto{\pgfqpoint{5.034097in}{2.866301in}}%
\pgfpathlineto{\pgfqpoint{5.026536in}{2.859871in}}%
\pgfpathlineto{\pgfqpoint{5.018968in}{2.853404in}}%
\pgfpathlineto{\pgfqpoint{5.011394in}{2.846896in}}%
\pgfpathclose%
\pgfusepath{fill}%
\end{pgfscope}%
\begin{pgfscope}%
\pgfpathrectangle{\pgfqpoint{1.150000in}{0.150000in}}{\pgfqpoint{5.700000in}{5.700000in}}%
\pgfusepath{clip}%
\pgfsetbuttcap%
\pgfsetroundjoin%
\definecolor{currentfill}{rgb}{0.239346,0.300855,0.540844}%
\pgfsetfillcolor{currentfill}%
\pgfsetfillopacity{0.800000}%
\pgfsetlinewidth{0.000000pt}%
\definecolor{currentstroke}{rgb}{0.000000,0.000000,0.000000}%
\pgfsetstrokecolor{currentstroke}%
\pgfsetdash{}{0pt}%
\pgfpathmoveto{\pgfqpoint{2.461143in}{2.485084in}}%
\pgfpathlineto{\pgfqpoint{2.475076in}{2.464221in}}%
\pgfpathlineto{\pgfqpoint{2.488998in}{2.443677in}}%
\pgfpathlineto{\pgfqpoint{2.502908in}{2.423446in}}%
\pgfpathlineto{\pgfqpoint{2.516806in}{2.403527in}}%
\pgfpathlineto{\pgfqpoint{2.525506in}{2.405788in}}%
\pgfpathlineto{\pgfqpoint{2.534191in}{2.408267in}}%
\pgfpathlineto{\pgfqpoint{2.542862in}{2.410961in}}%
\pgfpathlineto{\pgfqpoint{2.551518in}{2.413866in}}%
\pgfpathlineto{\pgfqpoint{2.537660in}{2.433357in}}%
\pgfpathlineto{\pgfqpoint{2.523791in}{2.453158in}}%
\pgfpathlineto{\pgfqpoint{2.509911in}{2.473273in}}%
\pgfpathlineto{\pgfqpoint{2.496019in}{2.493703in}}%
\pgfpathlineto{\pgfqpoint{2.487322in}{2.491215in}}%
\pgfpathlineto{\pgfqpoint{2.478611in}{2.488947in}}%
\pgfpathlineto{\pgfqpoint{2.469885in}{2.486901in}}%
\pgfpathlineto{\pgfqpoint{2.461143in}{2.485084in}}%
\pgfpathclose%
\pgfusepath{fill}%
\end{pgfscope}%
\begin{pgfscope}%
\pgfpathrectangle{\pgfqpoint{1.150000in}{0.150000in}}{\pgfqpoint{5.700000in}{5.700000in}}%
\pgfusepath{clip}%
\pgfsetbuttcap%
\pgfsetroundjoin%
\definecolor{currentfill}{rgb}{0.276022,0.044167,0.370164}%
\pgfsetfillcolor{currentfill}%
\pgfsetfillopacity{0.800000}%
\pgfsetlinewidth{0.000000pt}%
\definecolor{currentstroke}{rgb}{0.000000,0.000000,0.000000}%
\pgfsetstrokecolor{currentstroke}%
\pgfsetdash{}{0pt}%
\pgfpathmoveto{\pgfqpoint{3.100071in}{1.861736in}}%
\pgfpathlineto{\pgfqpoint{3.113712in}{1.853062in}}%
\pgfpathlineto{\pgfqpoint{3.127353in}{1.844616in}}%
\pgfpathlineto{\pgfqpoint{3.140993in}{1.836397in}}%
\pgfpathlineto{\pgfqpoint{3.154633in}{1.828403in}}%
\pgfpathlineto{\pgfqpoint{3.162944in}{1.835845in}}%
\pgfpathlineto{\pgfqpoint{3.171246in}{1.843399in}}%
\pgfpathlineto{\pgfqpoint{3.179539in}{1.851061in}}%
\pgfpathlineto{\pgfqpoint{3.187824in}{1.858829in}}%
\pgfpathlineto{\pgfqpoint{3.174206in}{1.866450in}}%
\pgfpathlineto{\pgfqpoint{3.160588in}{1.874295in}}%
\pgfpathlineto{\pgfqpoint{3.146970in}{1.882368in}}%
\pgfpathlineto{\pgfqpoint{3.133352in}{1.890668in}}%
\pgfpathlineto{\pgfqpoint{3.125045in}{1.883262in}}%
\pgfpathlineto{\pgfqpoint{3.116729in}{1.875968in}}%
\pgfpathlineto{\pgfqpoint{3.108404in}{1.868792in}}%
\pgfpathlineto{\pgfqpoint{3.100071in}{1.861736in}}%
\pgfpathclose%
\pgfusepath{fill}%
\end{pgfscope}%
\begin{pgfscope}%
\pgfpathrectangle{\pgfqpoint{1.150000in}{0.150000in}}{\pgfqpoint{5.700000in}{5.700000in}}%
\pgfusepath{clip}%
\pgfsetbuttcap%
\pgfsetroundjoin%
\definecolor{currentfill}{rgb}{0.119483,0.614817,0.537692}%
\pgfsetfillcolor{currentfill}%
\pgfsetfillopacity{0.800000}%
\pgfsetlinewidth{0.000000pt}%
\definecolor{currentstroke}{rgb}{0.000000,0.000000,0.000000}%
\pgfsetstrokecolor{currentstroke}%
\pgfsetdash{}{0pt}%
\pgfpathmoveto{\pgfqpoint{5.738321in}{3.341506in}}%
\pgfpathlineto{\pgfqpoint{5.752923in}{3.352302in}}%
\pgfpathlineto{\pgfqpoint{5.767544in}{3.363274in}}%
\pgfpathlineto{\pgfqpoint{5.782185in}{3.374422in}}%
\pgfpathlineto{\pgfqpoint{5.796847in}{3.385745in}}%
\pgfpathlineto{\pgfqpoint{5.804004in}{3.388247in}}%
\pgfpathlineto{\pgfqpoint{5.811158in}{3.390834in}}%
\pgfpathlineto{\pgfqpoint{5.818308in}{3.393513in}}%
\pgfpathlineto{\pgfqpoint{5.825455in}{3.396291in}}%
\pgfpathlineto{\pgfqpoint{5.810827in}{3.385616in}}%
\pgfpathlineto{\pgfqpoint{5.796218in}{3.375116in}}%
\pgfpathlineto{\pgfqpoint{5.781629in}{3.364790in}}%
\pgfpathlineto{\pgfqpoint{5.767060in}{3.354638in}}%
\pgfpathlineto{\pgfqpoint{5.759880in}{3.351203in}}%
\pgfpathlineto{\pgfqpoint{5.752697in}{3.347874in}}%
\pgfpathlineto{\pgfqpoint{5.745511in}{3.344644in}}%
\pgfpathlineto{\pgfqpoint{5.738321in}{3.341506in}}%
\pgfpathclose%
\pgfusepath{fill}%
\end{pgfscope}%
\begin{pgfscope}%
\pgfpathrectangle{\pgfqpoint{1.150000in}{0.150000in}}{\pgfqpoint{5.700000in}{5.700000in}}%
\pgfusepath{clip}%
\pgfsetbuttcap%
\pgfsetroundjoin%
\definecolor{currentfill}{rgb}{0.214298,0.355619,0.551184}%
\pgfsetfillcolor{currentfill}%
\pgfsetfillopacity{0.800000}%
\pgfsetlinewidth{0.000000pt}%
\definecolor{currentstroke}{rgb}{0.000000,0.000000,0.000000}%
\pgfsetstrokecolor{currentstroke}%
\pgfsetdash{}{0pt}%
\pgfpathmoveto{\pgfqpoint{4.632300in}{2.557742in}}%
\pgfpathlineto{\pgfqpoint{4.646320in}{2.565436in}}%
\pgfpathlineto{\pgfqpoint{4.660355in}{2.573313in}}%
\pgfpathlineto{\pgfqpoint{4.674404in}{2.581375in}}%
\pgfpathlineto{\pgfqpoint{4.688469in}{2.589620in}}%
\pgfpathlineto{\pgfqpoint{4.696208in}{2.598136in}}%
\pgfpathlineto{\pgfqpoint{4.703941in}{2.606574in}}%
\pgfpathlineto{\pgfqpoint{4.711667in}{2.614938in}}%
\pgfpathlineto{\pgfqpoint{4.719387in}{2.623228in}}%
\pgfpathlineto{\pgfqpoint{4.705331in}{2.615158in}}%
\pgfpathlineto{\pgfqpoint{4.691290in}{2.607272in}}%
\pgfpathlineto{\pgfqpoint{4.677263in}{2.599569in}}%
\pgfpathlineto{\pgfqpoint{4.663251in}{2.592050in}}%
\pgfpathlineto{\pgfqpoint{4.655523in}{2.583573in}}%
\pgfpathlineto{\pgfqpoint{4.647788in}{2.575030in}}%
\pgfpathlineto{\pgfqpoint{4.640047in}{2.566421in}}%
\pgfpathlineto{\pgfqpoint{4.632300in}{2.557742in}}%
\pgfpathclose%
\pgfusepath{fill}%
\end{pgfscope}%
\begin{pgfscope}%
\pgfpathrectangle{\pgfqpoint{1.150000in}{0.150000in}}{\pgfqpoint{5.700000in}{5.700000in}}%
\pgfusepath{clip}%
\pgfsetbuttcap%
\pgfsetroundjoin%
\definecolor{currentfill}{rgb}{0.262138,0.242286,0.520837}%
\pgfsetfillcolor{currentfill}%
\pgfsetfillopacity{0.800000}%
\pgfsetlinewidth{0.000000pt}%
\definecolor{currentstroke}{rgb}{0.000000,0.000000,0.000000}%
\pgfsetstrokecolor{currentstroke}%
\pgfsetdash{}{0pt}%
\pgfpathmoveto{\pgfqpoint{4.253085in}{2.262981in}}%
\pgfpathlineto{\pgfqpoint{4.266927in}{2.268078in}}%
\pgfpathlineto{\pgfqpoint{4.280782in}{2.273362in}}%
\pgfpathlineto{\pgfqpoint{4.294648in}{2.278834in}}%
\pgfpathlineto{\pgfqpoint{4.308526in}{2.284493in}}%
\pgfpathlineto{\pgfqpoint{4.316409in}{2.294890in}}%
\pgfpathlineto{\pgfqpoint{4.324287in}{2.305215in}}%
\pgfpathlineto{\pgfqpoint{4.332159in}{2.315469in}}%
\pgfpathlineto{\pgfqpoint{4.340025in}{2.325653in}}%
\pgfpathlineto{\pgfqpoint{4.326152in}{2.320004in}}%
\pgfpathlineto{\pgfqpoint{4.312291in}{2.314542in}}%
\pgfpathlineto{\pgfqpoint{4.298443in}{2.309268in}}%
\pgfpathlineto{\pgfqpoint{4.284606in}{2.304182in}}%
\pgfpathlineto{\pgfqpoint{4.276733in}{2.293976in}}%
\pgfpathlineto{\pgfqpoint{4.268856in}{2.283708in}}%
\pgfpathlineto{\pgfqpoint{4.260973in}{2.273376in}}%
\pgfpathlineto{\pgfqpoint{4.253085in}{2.262981in}}%
\pgfpathclose%
\pgfusepath{fill}%
\end{pgfscope}%
\begin{pgfscope}%
\pgfpathrectangle{\pgfqpoint{1.150000in}{0.150000in}}{\pgfqpoint{5.700000in}{5.700000in}}%
\pgfusepath{clip}%
\pgfsetbuttcap%
\pgfsetroundjoin%
\definecolor{currentfill}{rgb}{0.272594,0.025563,0.353093}%
\pgfsetfillcolor{currentfill}%
\pgfsetfillopacity{0.800000}%
\pgfsetlinewidth{0.000000pt}%
\definecolor{currentstroke}{rgb}{0.000000,0.000000,0.000000}%
\pgfsetstrokecolor{currentstroke}%
\pgfsetdash{}{0pt}%
\pgfpathmoveto{\pgfqpoint{3.384180in}{1.820681in}}%
\pgfpathlineto{\pgfqpoint{3.397804in}{1.816227in}}%
\pgfpathlineto{\pgfqpoint{3.411431in}{1.811984in}}%
\pgfpathlineto{\pgfqpoint{3.425061in}{1.807949in}}%
\pgfpathlineto{\pgfqpoint{3.438695in}{1.804124in}}%
\pgfpathlineto{\pgfqpoint{3.446872in}{1.813651in}}%
\pgfpathlineto{\pgfqpoint{3.455043in}{1.823233in}}%
\pgfpathlineto{\pgfqpoint{3.463207in}{1.832867in}}%
\pgfpathlineto{\pgfqpoint{3.471366in}{1.842550in}}%
\pgfpathlineto{\pgfqpoint{3.457746in}{1.846068in}}%
\pgfpathlineto{\pgfqpoint{3.444131in}{1.849794in}}%
\pgfpathlineto{\pgfqpoint{3.430519in}{1.853730in}}%
\pgfpathlineto{\pgfqpoint{3.416911in}{1.857876in}}%
\pgfpathlineto{\pgfqpoint{3.408738in}{1.848489in}}%
\pgfpathlineto{\pgfqpoint{3.400559in}{1.839158in}}%
\pgfpathlineto{\pgfqpoint{3.392373in}{1.829888in}}%
\pgfpathlineto{\pgfqpoint{3.384180in}{1.820681in}}%
\pgfpathclose%
\pgfusepath{fill}%
\end{pgfscope}%
\begin{pgfscope}%
\pgfpathrectangle{\pgfqpoint{1.150000in}{0.150000in}}{\pgfqpoint{5.700000in}{5.700000in}}%
\pgfusepath{clip}%
\pgfsetbuttcap%
\pgfsetroundjoin%
\definecolor{currentfill}{rgb}{0.281924,0.089666,0.412415}%
\pgfsetfillcolor{currentfill}%
\pgfsetfillopacity{0.800000}%
\pgfsetlinewidth{0.000000pt}%
\definecolor{currentstroke}{rgb}{0.000000,0.000000,0.000000}%
\pgfsetstrokecolor{currentstroke}%
\pgfsetdash{}{0pt}%
\pgfpathmoveto{\pgfqpoint{2.902522in}{1.962058in}}%
\pgfpathlineto{\pgfqpoint{2.916213in}{1.950129in}}%
\pgfpathlineto{\pgfqpoint{2.929901in}{1.938445in}}%
\pgfpathlineto{\pgfqpoint{2.943586in}{1.927006in}}%
\pgfpathlineto{\pgfqpoint{2.957268in}{1.915809in}}%
\pgfpathlineto{\pgfqpoint{2.965692in}{1.921508in}}%
\pgfpathlineto{\pgfqpoint{2.974106in}{1.927358in}}%
\pgfpathlineto{\pgfqpoint{2.982509in}{1.933357in}}%
\pgfpathlineto{\pgfqpoint{2.990902in}{1.939498in}}%
\pgfpathlineto{\pgfqpoint{2.977248in}{1.950286in}}%
\pgfpathlineto{\pgfqpoint{2.963592in}{1.961316in}}%
\pgfpathlineto{\pgfqpoint{2.949932in}{1.972590in}}%
\pgfpathlineto{\pgfqpoint{2.936270in}{1.984110in}}%
\pgfpathlineto{\pgfqpoint{2.927849in}{1.978365in}}%
\pgfpathlineto{\pgfqpoint{2.919417in}{1.972773in}}%
\pgfpathlineto{\pgfqpoint{2.910975in}{1.967336in}}%
\pgfpathlineto{\pgfqpoint{2.902522in}{1.962058in}}%
\pgfpathclose%
\pgfusepath{fill}%
\end{pgfscope}%
\begin{pgfscope}%
\pgfpathrectangle{\pgfqpoint{1.150000in}{0.150000in}}{\pgfqpoint{5.700000in}{5.700000in}}%
\pgfusepath{clip}%
\pgfsetbuttcap%
\pgfsetroundjoin%
\definecolor{currentfill}{rgb}{0.122312,0.633153,0.530398}%
\pgfsetfillcolor{currentfill}%
\pgfsetfillopacity{0.800000}%
\pgfsetlinewidth{0.000000pt}%
\definecolor{currentstroke}{rgb}{0.000000,0.000000,0.000000}%
\pgfsetstrokecolor{currentstroke}%
\pgfsetdash{}{0pt}%
\pgfpathmoveto{\pgfqpoint{5.825455in}{3.396291in}}%
\pgfpathlineto{\pgfqpoint{5.840103in}{3.407141in}}%
\pgfpathlineto{\pgfqpoint{5.854771in}{3.418165in}}%
\pgfpathlineto{\pgfqpoint{5.869459in}{3.429365in}}%
\pgfpathlineto{\pgfqpoint{5.884167in}{3.440740in}}%
\pgfpathlineto{\pgfqpoint{5.891276in}{3.442954in}}%
\pgfpathlineto{\pgfqpoint{5.898382in}{3.445274in}}%
\pgfpathlineto{\pgfqpoint{5.905485in}{3.447707in}}%
\pgfpathlineto{\pgfqpoint{5.912586in}{3.450261in}}%
\pgfpathlineto{\pgfqpoint{5.897913in}{3.439568in}}%
\pgfpathlineto{\pgfqpoint{5.883260in}{3.429050in}}%
\pgfpathlineto{\pgfqpoint{5.868627in}{3.418705in}}%
\pgfpathlineto{\pgfqpoint{5.854014in}{3.408534in}}%
\pgfpathlineto{\pgfqpoint{5.846878in}{3.405289in}}%
\pgfpathlineto{\pgfqpoint{5.839739in}{3.402171in}}%
\pgfpathlineto{\pgfqpoint{5.832598in}{3.399175in}}%
\pgfpathlineto{\pgfqpoint{5.825455in}{3.396291in}}%
\pgfpathclose%
\pgfusepath{fill}%
\end{pgfscope}%
\begin{pgfscope}%
\pgfpathrectangle{\pgfqpoint{1.150000in}{0.150000in}}{\pgfqpoint{5.700000in}{5.700000in}}%
\pgfusepath{clip}%
\pgfsetbuttcap%
\pgfsetroundjoin%
\definecolor{currentfill}{rgb}{0.160665,0.478540,0.558115}%
\pgfsetfillcolor{currentfill}%
\pgfsetfillopacity{0.800000}%
\pgfsetlinewidth{0.000000pt}%
\definecolor{currentstroke}{rgb}{0.000000,0.000000,0.000000}%
\pgfsetstrokecolor{currentstroke}%
\pgfsetdash{}{0pt}%
\pgfpathmoveto{\pgfqpoint{5.098558in}{2.910022in}}%
\pgfpathlineto{\pgfqpoint{5.112826in}{2.919804in}}%
\pgfpathlineto{\pgfqpoint{5.127112in}{2.929766in}}%
\pgfpathlineto{\pgfqpoint{5.141415in}{2.939908in}}%
\pgfpathlineto{\pgfqpoint{5.155736in}{2.950231in}}%
\pgfpathlineto{\pgfqpoint{5.163253in}{2.955872in}}%
\pgfpathlineto{\pgfqpoint{5.170763in}{2.961472in}}%
\pgfpathlineto{\pgfqpoint{5.178267in}{2.967036in}}%
\pgfpathlineto{\pgfqpoint{5.185763in}{2.972568in}}%
\pgfpathlineto{\pgfqpoint{5.171459in}{2.962624in}}%
\pgfpathlineto{\pgfqpoint{5.157172in}{2.952859in}}%
\pgfpathlineto{\pgfqpoint{5.142903in}{2.943273in}}%
\pgfpathlineto{\pgfqpoint{5.128650in}{2.933867in}}%
\pgfpathlineto{\pgfqpoint{5.121137in}{2.927946in}}%
\pgfpathlineto{\pgfqpoint{5.113617in}{2.922001in}}%
\pgfpathlineto{\pgfqpoint{5.106091in}{2.916028in}}%
\pgfpathlineto{\pgfqpoint{5.098558in}{2.910022in}}%
\pgfpathclose%
\pgfusepath{fill}%
\end{pgfscope}%
\begin{pgfscope}%
\pgfpathrectangle{\pgfqpoint{1.150000in}{0.150000in}}{\pgfqpoint{5.700000in}{5.700000in}}%
\pgfusepath{clip}%
\pgfsetbuttcap%
\pgfsetroundjoin%
\definecolor{currentfill}{rgb}{0.130067,0.651384,0.521608}%
\pgfsetfillcolor{currentfill}%
\pgfsetfillopacity{0.800000}%
\pgfsetlinewidth{0.000000pt}%
\definecolor{currentstroke}{rgb}{0.000000,0.000000,0.000000}%
\pgfsetstrokecolor{currentstroke}%
\pgfsetdash{}{0pt}%
\pgfpathmoveto{\pgfqpoint{5.912586in}{3.450261in}}%
\pgfpathlineto{\pgfqpoint{5.927279in}{3.461128in}}%
\pgfpathlineto{\pgfqpoint{5.941993in}{3.472169in}}%
\pgfpathlineto{\pgfqpoint{5.956726in}{3.483385in}}%
\pgfpathlineto{\pgfqpoint{5.971481in}{3.494775in}}%
\pgfpathlineto{\pgfqpoint{5.978542in}{3.496753in}}%
\pgfpathlineto{\pgfqpoint{5.985601in}{3.498860in}}%
\pgfpathlineto{\pgfqpoint{5.992659in}{3.501103in}}%
\pgfpathlineto{\pgfqpoint{5.999715in}{3.503490in}}%
\pgfpathlineto{\pgfqpoint{5.984999in}{3.492816in}}%
\pgfpathlineto{\pgfqpoint{5.970303in}{3.482315in}}%
\pgfpathlineto{\pgfqpoint{5.955627in}{3.471987in}}%
\pgfpathlineto{\pgfqpoint{5.940971in}{3.461832in}}%
\pgfpathlineto{\pgfqpoint{5.933877in}{3.458721in}}%
\pgfpathlineto{\pgfqpoint{5.926782in}{3.455760in}}%
\pgfpathlineto{\pgfqpoint{5.919685in}{3.452943in}}%
\pgfpathlineto{\pgfqpoint{5.912586in}{3.450261in}}%
\pgfpathclose%
\pgfusepath{fill}%
\end{pgfscope}%
\begin{pgfscope}%
\pgfpathrectangle{\pgfqpoint{1.150000in}{0.150000in}}{\pgfqpoint{5.700000in}{5.700000in}}%
\pgfusepath{clip}%
\pgfsetbuttcap%
\pgfsetroundjoin%
\definecolor{currentfill}{rgb}{0.223925,0.334994,0.548053}%
\pgfsetfillcolor{currentfill}%
\pgfsetfillopacity{0.800000}%
\pgfsetlinewidth{0.000000pt}%
\definecolor{currentstroke}{rgb}{0.000000,0.000000,0.000000}%
\pgfsetstrokecolor{currentstroke}%
\pgfsetdash{}{0pt}%
\pgfpathmoveto{\pgfqpoint{2.405281in}{2.571764in}}%
\pgfpathlineto{\pgfqpoint{2.419266in}{2.549603in}}%
\pgfpathlineto{\pgfqpoint{2.433238in}{2.527771in}}%
\pgfpathlineto{\pgfqpoint{2.447196in}{2.506266in}}%
\pgfpathlineto{\pgfqpoint{2.461143in}{2.485084in}}%
\pgfpathlineto{\pgfqpoint{2.469885in}{2.486901in}}%
\pgfpathlineto{\pgfqpoint{2.478611in}{2.488947in}}%
\pgfpathlineto{\pgfqpoint{2.487322in}{2.491215in}}%
\pgfpathlineto{\pgfqpoint{2.496019in}{2.493703in}}%
\pgfpathlineto{\pgfqpoint{2.482115in}{2.514453in}}%
\pgfpathlineto{\pgfqpoint{2.468199in}{2.535525in}}%
\pgfpathlineto{\pgfqpoint{2.454270in}{2.556923in}}%
\pgfpathlineto{\pgfqpoint{2.440328in}{2.578649in}}%
\pgfpathlineto{\pgfqpoint{2.431590in}{2.576581in}}%
\pgfpathlineto{\pgfqpoint{2.422836in}{2.574742in}}%
\pgfpathlineto{\pgfqpoint{2.414067in}{2.573135in}}%
\pgfpathlineto{\pgfqpoint{2.405281in}{2.571764in}}%
\pgfpathclose%
\pgfusepath{fill}%
\end{pgfscope}%
\begin{pgfscope}%
\pgfpathrectangle{\pgfqpoint{1.150000in}{0.150000in}}{\pgfqpoint{5.700000in}{5.700000in}}%
\pgfusepath{clip}%
\pgfsetbuttcap%
\pgfsetroundjoin%
\definecolor{currentfill}{rgb}{0.252194,0.269783,0.531579}%
\pgfsetfillcolor{currentfill}%
\pgfsetfillopacity{0.800000}%
\pgfsetlinewidth{0.000000pt}%
\definecolor{currentstroke}{rgb}{0.000000,0.000000,0.000000}%
\pgfsetstrokecolor{currentstroke}%
\pgfsetdash{}{0pt}%
\pgfpathmoveto{\pgfqpoint{4.340025in}{2.325653in}}%
\pgfpathlineto{\pgfqpoint{4.353910in}{2.331489in}}%
\pgfpathlineto{\pgfqpoint{4.367808in}{2.337513in}}%
\pgfpathlineto{\pgfqpoint{4.381719in}{2.343723in}}%
\pgfpathlineto{\pgfqpoint{4.395642in}{2.350119in}}%
\pgfpathlineto{\pgfqpoint{4.403497in}{2.360202in}}%
\pgfpathlineto{\pgfqpoint{4.411347in}{2.370208in}}%
\pgfpathlineto{\pgfqpoint{4.419191in}{2.380138in}}%
\pgfpathlineto{\pgfqpoint{4.427029in}{2.389992in}}%
\pgfpathlineto{\pgfqpoint{4.413112in}{2.383639in}}%
\pgfpathlineto{\pgfqpoint{4.399207in}{2.377472in}}%
\pgfpathlineto{\pgfqpoint{4.385315in}{2.371492in}}%
\pgfpathlineto{\pgfqpoint{4.371436in}{2.365698in}}%
\pgfpathlineto{\pgfqpoint{4.363591in}{2.355788in}}%
\pgfpathlineto{\pgfqpoint{4.355741in}{2.345812in}}%
\pgfpathlineto{\pgfqpoint{4.347886in}{2.335767in}}%
\pgfpathlineto{\pgfqpoint{4.340025in}{2.325653in}}%
\pgfpathclose%
\pgfusepath{fill}%
\end{pgfscope}%
\begin{pgfscope}%
\pgfpathrectangle{\pgfqpoint{1.150000in}{0.150000in}}{\pgfqpoint{5.700000in}{5.700000in}}%
\pgfusepath{clip}%
\pgfsetbuttcap%
\pgfsetroundjoin%
\definecolor{currentfill}{rgb}{0.282656,0.100196,0.422160}%
\pgfsetfillcolor{currentfill}%
\pgfsetfillopacity{0.800000}%
\pgfsetlinewidth{0.000000pt}%
\definecolor{currentstroke}{rgb}{0.000000,0.000000,0.000000}%
\pgfsetstrokecolor{currentstroke}%
\pgfsetdash{}{0pt}%
\pgfpathmoveto{\pgfqpoint{3.786856in}{1.944909in}}%
\pgfpathlineto{\pgfqpoint{3.800542in}{1.945555in}}%
\pgfpathlineto{\pgfqpoint{3.814235in}{1.946396in}}%
\pgfpathlineto{\pgfqpoint{3.827936in}{1.947434in}}%
\pgfpathlineto{\pgfqpoint{3.841646in}{1.948666in}}%
\pgfpathlineto{\pgfqpoint{3.849680in}{1.959750in}}%
\pgfpathlineto{\pgfqpoint{3.857709in}{1.970815in}}%
\pgfpathlineto{\pgfqpoint{3.865733in}{1.981860in}}%
\pgfpathlineto{\pgfqpoint{3.873752in}{1.992882in}}%
\pgfpathlineto{\pgfqpoint{3.860050in}{1.991468in}}%
\pgfpathlineto{\pgfqpoint{3.846356in}{1.990248in}}%
\pgfpathlineto{\pgfqpoint{3.832671in}{1.989225in}}%
\pgfpathlineto{\pgfqpoint{3.818993in}{1.988397in}}%
\pgfpathlineto{\pgfqpoint{3.810967in}{1.977545in}}%
\pgfpathlineto{\pgfqpoint{3.802935in}{1.966679in}}%
\pgfpathlineto{\pgfqpoint{3.794898in}{1.955799in}}%
\pgfpathlineto{\pgfqpoint{3.786856in}{1.944909in}}%
\pgfpathclose%
\pgfusepath{fill}%
\end{pgfscope}%
\begin{pgfscope}%
\pgfpathrectangle{\pgfqpoint{1.150000in}{0.150000in}}{\pgfqpoint{5.700000in}{5.700000in}}%
\pgfusepath{clip}%
\pgfsetbuttcap%
\pgfsetroundjoin%
\definecolor{currentfill}{rgb}{0.280894,0.078907,0.402329}%
\pgfsetfillcolor{currentfill}%
\pgfsetfillopacity{0.800000}%
\pgfsetlinewidth{0.000000pt}%
\definecolor{currentstroke}{rgb}{0.000000,0.000000,0.000000}%
\pgfsetstrokecolor{currentstroke}%
\pgfsetdash{}{0pt}%
\pgfpathmoveto{\pgfqpoint{3.699932in}{1.901506in}}%
\pgfpathlineto{\pgfqpoint{3.713598in}{1.901149in}}%
\pgfpathlineto{\pgfqpoint{3.727271in}{1.900990in}}%
\pgfpathlineto{\pgfqpoint{3.740950in}{1.901029in}}%
\pgfpathlineto{\pgfqpoint{3.754637in}{1.901265in}}%
\pgfpathlineto{\pgfqpoint{3.762700in}{1.912185in}}%
\pgfpathlineto{\pgfqpoint{3.770757in}{1.923100in}}%
\pgfpathlineto{\pgfqpoint{3.778809in}{1.934008in}}%
\pgfpathlineto{\pgfqpoint{3.786856in}{1.944909in}}%
\pgfpathlineto{\pgfqpoint{3.773178in}{1.944460in}}%
\pgfpathlineto{\pgfqpoint{3.759507in}{1.944207in}}%
\pgfpathlineto{\pgfqpoint{3.745844in}{1.944153in}}%
\pgfpathlineto{\pgfqpoint{3.732187in}{1.944297in}}%
\pgfpathlineto{\pgfqpoint{3.724131in}{1.933598in}}%
\pgfpathlineto{\pgfqpoint{3.716070in}{1.922899in}}%
\pgfpathlineto{\pgfqpoint{3.708004in}{1.912201in}}%
\pgfpathlineto{\pgfqpoint{3.699932in}{1.901506in}}%
\pgfpathclose%
\pgfusepath{fill}%
\end{pgfscope}%
\begin{pgfscope}%
\pgfpathrectangle{\pgfqpoint{1.150000in}{0.150000in}}{\pgfqpoint{5.700000in}{5.700000in}}%
\pgfusepath{clip}%
\pgfsetbuttcap%
\pgfsetroundjoin%
\definecolor{currentfill}{rgb}{0.203063,0.379716,0.553925}%
\pgfsetfillcolor{currentfill}%
\pgfsetfillopacity{0.800000}%
\pgfsetlinewidth{0.000000pt}%
\definecolor{currentstroke}{rgb}{0.000000,0.000000,0.000000}%
\pgfsetstrokecolor{currentstroke}%
\pgfsetdash{}{0pt}%
\pgfpathmoveto{\pgfqpoint{4.719387in}{2.623228in}}%
\pgfpathlineto{\pgfqpoint{4.733458in}{2.631481in}}%
\pgfpathlineto{\pgfqpoint{4.747544in}{2.639918in}}%
\pgfpathlineto{\pgfqpoint{4.761645in}{2.648537in}}%
\pgfpathlineto{\pgfqpoint{4.775761in}{2.657340in}}%
\pgfpathlineto{\pgfqpoint{4.783466in}{2.665364in}}%
\pgfpathlineto{\pgfqpoint{4.791163in}{2.673311in}}%
\pgfpathlineto{\pgfqpoint{4.798855in}{2.681186in}}%
\pgfpathlineto{\pgfqpoint{4.806539in}{2.688989in}}%
\pgfpathlineto{\pgfqpoint{4.792432in}{2.680396in}}%
\pgfpathlineto{\pgfqpoint{4.778340in}{2.671985in}}%
\pgfpathlineto{\pgfqpoint{4.764264in}{2.663757in}}%
\pgfpathlineto{\pgfqpoint{4.750202in}{2.655712in}}%
\pgfpathlineto{\pgfqpoint{4.742508in}{2.647688in}}%
\pgfpathlineto{\pgfqpoint{4.734807in}{2.639601in}}%
\pgfpathlineto{\pgfqpoint{4.727100in}{2.631448in}}%
\pgfpathlineto{\pgfqpoint{4.719387in}{2.623228in}}%
\pgfpathclose%
\pgfusepath{fill}%
\end{pgfscope}%
\begin{pgfscope}%
\pgfpathrectangle{\pgfqpoint{1.150000in}{0.150000in}}{\pgfqpoint{5.700000in}{5.700000in}}%
\pgfusepath{clip}%
\pgfsetbuttcap%
\pgfsetroundjoin%
\definecolor{currentfill}{rgb}{0.143303,0.669459,0.511215}%
\pgfsetfillcolor{currentfill}%
\pgfsetfillopacity{0.800000}%
\pgfsetlinewidth{0.000000pt}%
\definecolor{currentstroke}{rgb}{0.000000,0.000000,0.000000}%
\pgfsetstrokecolor{currentstroke}%
\pgfsetdash{}{0pt}%
\pgfpathmoveto{\pgfqpoint{5.999715in}{3.503490in}}%
\pgfpathlineto{\pgfqpoint{6.014451in}{3.514338in}}%
\pgfpathlineto{\pgfqpoint{6.029209in}{3.525360in}}%
\pgfpathlineto{\pgfqpoint{6.043986in}{3.536555in}}%
\pgfpathlineto{\pgfqpoint{6.058785in}{3.547925in}}%
\pgfpathlineto{\pgfqpoint{6.065800in}{3.549727in}}%
\pgfpathlineto{\pgfqpoint{6.072814in}{3.551681in}}%
\pgfpathlineto{\pgfqpoint{6.079827in}{3.553795in}}%
\pgfpathlineto{\pgfqpoint{6.086840in}{3.556078in}}%
\pgfpathlineto{\pgfqpoint{6.072083in}{3.545457in}}%
\pgfpathlineto{\pgfqpoint{6.057345in}{3.535010in}}%
\pgfpathlineto{\pgfqpoint{6.042628in}{3.524735in}}%
\pgfpathlineto{\pgfqpoint{6.027931in}{3.514633in}}%
\pgfpathlineto{\pgfqpoint{6.020878in}{3.511592in}}%
\pgfpathlineto{\pgfqpoint{6.013824in}{3.508727in}}%
\pgfpathlineto{\pgfqpoint{6.006770in}{3.506029in}}%
\pgfpathlineto{\pgfqpoint{5.999715in}{3.503490in}}%
\pgfpathclose%
\pgfusepath{fill}%
\end{pgfscope}%
\begin{pgfscope}%
\pgfpathrectangle{\pgfqpoint{1.150000in}{0.150000in}}{\pgfqpoint{5.700000in}{5.700000in}}%
\pgfusepath{clip}%
\pgfsetbuttcap%
\pgfsetroundjoin%
\definecolor{currentfill}{rgb}{0.283187,0.125848,0.444960}%
\pgfsetfillcolor{currentfill}%
\pgfsetfillopacity{0.800000}%
\pgfsetlinewidth{0.000000pt}%
\definecolor{currentstroke}{rgb}{0.000000,0.000000,0.000000}%
\pgfsetstrokecolor{currentstroke}%
\pgfsetdash{}{0pt}%
\pgfpathmoveto{\pgfqpoint{3.873752in}{1.992882in}}%
\pgfpathlineto{\pgfqpoint{3.887462in}{1.994490in}}%
\pgfpathlineto{\pgfqpoint{3.901180in}{1.996293in}}%
\pgfpathlineto{\pgfqpoint{3.914908in}{1.998289in}}%
\pgfpathlineto{\pgfqpoint{3.928644in}{2.000479in}}%
\pgfpathlineto{\pgfqpoint{3.936651in}{2.011639in}}%
\pgfpathlineto{\pgfqpoint{3.944653in}{2.022768in}}%
\pgfpathlineto{\pgfqpoint{3.952650in}{2.033862in}}%
\pgfpathlineto{\pgfqpoint{3.960643in}{2.044922in}}%
\pgfpathlineto{\pgfqpoint{3.946913in}{2.042582in}}%
\pgfpathlineto{\pgfqpoint{3.933192in}{2.040436in}}%
\pgfpathlineto{\pgfqpoint{3.919481in}{2.038483in}}%
\pgfpathlineto{\pgfqpoint{3.905777in}{2.036724in}}%
\pgfpathlineto{\pgfqpoint{3.897779in}{2.025803in}}%
\pgfpathlineto{\pgfqpoint{3.889775in}{2.014854in}}%
\pgfpathlineto{\pgfqpoint{3.881766in}{2.003880in}}%
\pgfpathlineto{\pgfqpoint{3.873752in}{1.992882in}}%
\pgfpathclose%
\pgfusepath{fill}%
\end{pgfscope}%
\begin{pgfscope}%
\pgfpathrectangle{\pgfqpoint{1.150000in}{0.150000in}}{\pgfqpoint{5.700000in}{5.700000in}}%
\pgfusepath{clip}%
\pgfsetbuttcap%
\pgfsetroundjoin%
\definecolor{currentfill}{rgb}{0.151918,0.500685,0.557587}%
\pgfsetfillcolor{currentfill}%
\pgfsetfillopacity{0.800000}%
\pgfsetlinewidth{0.000000pt}%
\definecolor{currentstroke}{rgb}{0.000000,0.000000,0.000000}%
\pgfsetstrokecolor{currentstroke}%
\pgfsetdash{}{0pt}%
\pgfpathmoveto{\pgfqpoint{5.185763in}{2.972568in}}%
\pgfpathlineto{\pgfqpoint{5.200085in}{2.982692in}}%
\pgfpathlineto{\pgfqpoint{5.214424in}{2.992996in}}%
\pgfpathlineto{\pgfqpoint{5.228781in}{3.003480in}}%
\pgfpathlineto{\pgfqpoint{5.243156in}{3.014144in}}%
\pgfpathlineto{\pgfqpoint{5.250628in}{3.019249in}}%
\pgfpathlineto{\pgfqpoint{5.258093in}{3.024324in}}%
\pgfpathlineto{\pgfqpoint{5.265552in}{3.029372in}}%
\pgfpathlineto{\pgfqpoint{5.273004in}{3.034400in}}%
\pgfpathlineto{\pgfqpoint{5.258647in}{3.024149in}}%
\pgfpathlineto{\pgfqpoint{5.244309in}{3.014077in}}%
\pgfpathlineto{\pgfqpoint{5.229987in}{3.004184in}}%
\pgfpathlineto{\pgfqpoint{5.215684in}{2.994469in}}%
\pgfpathlineto{\pgfqpoint{5.208213in}{2.989019in}}%
\pgfpathlineto{\pgfqpoint{5.200736in}{2.983555in}}%
\pgfpathlineto{\pgfqpoint{5.193253in}{2.978073in}}%
\pgfpathlineto{\pgfqpoint{5.185763in}{2.972568in}}%
\pgfpathclose%
\pgfusepath{fill}%
\end{pgfscope}%
\begin{pgfscope}%
\pgfpathrectangle{\pgfqpoint{1.150000in}{0.150000in}}{\pgfqpoint{5.700000in}{5.700000in}}%
\pgfusepath{clip}%
\pgfsetbuttcap%
\pgfsetroundjoin%
\definecolor{currentfill}{rgb}{0.278791,0.062145,0.386592}%
\pgfsetfillcolor{currentfill}%
\pgfsetfillopacity{0.800000}%
\pgfsetlinewidth{0.000000pt}%
\definecolor{currentstroke}{rgb}{0.000000,0.000000,0.000000}%
\pgfsetstrokecolor{currentstroke}%
\pgfsetdash{}{0pt}%
\pgfpathmoveto{\pgfqpoint{3.612953in}{1.863201in}}%
\pgfpathlineto{\pgfqpoint{3.626604in}{1.861800in}}%
\pgfpathlineto{\pgfqpoint{3.640260in}{1.860599in}}%
\pgfpathlineto{\pgfqpoint{3.653923in}{1.859599in}}%
\pgfpathlineto{\pgfqpoint{3.667593in}{1.858798in}}%
\pgfpathlineto{\pgfqpoint{3.675686in}{1.869460in}}%
\pgfpathlineto{\pgfqpoint{3.683773in}{1.880134in}}%
\pgfpathlineto{\pgfqpoint{3.691855in}{1.890816in}}%
\pgfpathlineto{\pgfqpoint{3.699932in}{1.901506in}}%
\pgfpathlineto{\pgfqpoint{3.686273in}{1.902062in}}%
\pgfpathlineto{\pgfqpoint{3.672621in}{1.902818in}}%
\pgfpathlineto{\pgfqpoint{3.658974in}{1.903773in}}%
\pgfpathlineto{\pgfqpoint{3.645334in}{1.904930in}}%
\pgfpathlineto{\pgfqpoint{3.637247in}{1.894473in}}%
\pgfpathlineto{\pgfqpoint{3.629154in}{1.884031in}}%
\pgfpathlineto{\pgfqpoint{3.621056in}{1.873607in}}%
\pgfpathlineto{\pgfqpoint{3.612953in}{1.863201in}}%
\pgfpathclose%
\pgfusepath{fill}%
\end{pgfscope}%
\begin{pgfscope}%
\pgfpathrectangle{\pgfqpoint{1.150000in}{0.150000in}}{\pgfqpoint{5.700000in}{5.700000in}}%
\pgfusepath{clip}%
\pgfsetbuttcap%
\pgfsetroundjoin%
\definecolor{currentfill}{rgb}{0.162016,0.687316,0.499129}%
\pgfsetfillcolor{currentfill}%
\pgfsetfillopacity{0.800000}%
\pgfsetlinewidth{0.000000pt}%
\definecolor{currentstroke}{rgb}{0.000000,0.000000,0.000000}%
\pgfsetstrokecolor{currentstroke}%
\pgfsetdash{}{0pt}%
\pgfpathmoveto{\pgfqpoint{6.086840in}{3.556078in}}%
\pgfpathlineto{\pgfqpoint{6.101619in}{3.566871in}}%
\pgfpathlineto{\pgfqpoint{6.116418in}{3.577837in}}%
\pgfpathlineto{\pgfqpoint{6.131239in}{3.588977in}}%
\pgfpathlineto{\pgfqpoint{6.146080in}{3.600290in}}%
\pgfpathlineto{\pgfqpoint{6.153051in}{3.601979in}}%
\pgfpathlineto{\pgfqpoint{6.160021in}{3.603846in}}%
\pgfpathlineto{\pgfqpoint{6.166993in}{3.605899in}}%
\pgfpathlineto{\pgfqpoint{6.173965in}{3.608146in}}%
\pgfpathlineto{\pgfqpoint{6.159167in}{3.597616in}}%
\pgfpathlineto{\pgfqpoint{6.144390in}{3.587258in}}%
\pgfpathlineto{\pgfqpoint{6.129634in}{3.577071in}}%
\pgfpathlineto{\pgfqpoint{6.114898in}{3.567057in}}%
\pgfpathlineto{\pgfqpoint{6.107882in}{3.564018in}}%
\pgfpathlineto{\pgfqpoint{6.100867in}{3.561181in}}%
\pgfpathlineto{\pgfqpoint{6.093854in}{3.558537in}}%
\pgfpathlineto{\pgfqpoint{6.086840in}{3.556078in}}%
\pgfpathclose%
\pgfusepath{fill}%
\end{pgfscope}%
\begin{pgfscope}%
\pgfpathrectangle{\pgfqpoint{1.150000in}{0.150000in}}{\pgfqpoint{5.700000in}{5.700000in}}%
\pgfusepath{clip}%
\pgfsetbuttcap%
\pgfsetroundjoin%
\definecolor{currentfill}{rgb}{0.280267,0.073417,0.397163}%
\pgfsetfillcolor{currentfill}%
\pgfsetfillopacity{0.800000}%
\pgfsetlinewidth{0.000000pt}%
\definecolor{currentstroke}{rgb}{0.000000,0.000000,0.000000}%
\pgfsetstrokecolor{currentstroke}%
\pgfsetdash{}{0pt}%
\pgfpathmoveto{\pgfqpoint{2.957268in}{1.915809in}}%
\pgfpathlineto{\pgfqpoint{2.970947in}{1.904853in}}%
\pgfpathlineto{\pgfqpoint{2.984624in}{1.894137in}}%
\pgfpathlineto{\pgfqpoint{2.998299in}{1.883658in}}%
\pgfpathlineto{\pgfqpoint{3.011971in}{1.873417in}}%
\pgfpathlineto{\pgfqpoint{3.020368in}{1.879536in}}%
\pgfpathlineto{\pgfqpoint{3.028754in}{1.885799in}}%
\pgfpathlineto{\pgfqpoint{3.037131in}{1.892201in}}%
\pgfpathlineto{\pgfqpoint{3.045498in}{1.898739in}}%
\pgfpathlineto{\pgfqpoint{3.031852in}{1.908573in}}%
\pgfpathlineto{\pgfqpoint{3.018204in}{1.918643in}}%
\pgfpathlineto{\pgfqpoint{3.004554in}{1.928951in}}%
\pgfpathlineto{\pgfqpoint{2.990902in}{1.939498in}}%
\pgfpathlineto{\pgfqpoint{2.982509in}{1.933357in}}%
\pgfpathlineto{\pgfqpoint{2.974106in}{1.927358in}}%
\pgfpathlineto{\pgfqpoint{2.965692in}{1.921508in}}%
\pgfpathlineto{\pgfqpoint{2.957268in}{1.915809in}}%
\pgfpathclose%
\pgfusepath{fill}%
\end{pgfscope}%
\begin{pgfscope}%
\pgfpathrectangle{\pgfqpoint{1.150000in}{0.150000in}}{\pgfqpoint{5.700000in}{5.700000in}}%
\pgfusepath{clip}%
\pgfsetbuttcap%
\pgfsetroundjoin%
\definecolor{currentfill}{rgb}{0.281887,0.150881,0.465405}%
\pgfsetfillcolor{currentfill}%
\pgfsetfillopacity{0.800000}%
\pgfsetlinewidth{0.000000pt}%
\definecolor{currentstroke}{rgb}{0.000000,0.000000,0.000000}%
\pgfsetstrokecolor{currentstroke}%
\pgfsetdash{}{0pt}%
\pgfpathmoveto{\pgfqpoint{3.960643in}{2.044922in}}%
\pgfpathlineto{\pgfqpoint{3.974382in}{2.047454in}}%
\pgfpathlineto{\pgfqpoint{3.988130in}{2.050178in}}%
\pgfpathlineto{\pgfqpoint{4.001888in}{2.053095in}}%
\pgfpathlineto{\pgfqpoint{4.015655in}{2.056203in}}%
\pgfpathlineto{\pgfqpoint{4.023636in}{2.067357in}}%
\pgfpathlineto{\pgfqpoint{4.031612in}{2.078466in}}%
\pgfpathlineto{\pgfqpoint{4.039584in}{2.089531in}}%
\pgfpathlineto{\pgfqpoint{4.047550in}{2.100549in}}%
\pgfpathlineto{\pgfqpoint{4.033788in}{2.097323in}}%
\pgfpathlineto{\pgfqpoint{4.020036in}{2.094288in}}%
\pgfpathlineto{\pgfqpoint{4.006294in}{2.091445in}}%
\pgfpathlineto{\pgfqpoint{3.992562in}{2.088794in}}%
\pgfpathlineto{\pgfqpoint{3.984589in}{2.077883in}}%
\pgfpathlineto{\pgfqpoint{3.976612in}{2.066933in}}%
\pgfpathlineto{\pgfqpoint{3.968630in}{2.055946in}}%
\pgfpathlineto{\pgfqpoint{3.960643in}{2.044922in}}%
\pgfpathclose%
\pgfusepath{fill}%
\end{pgfscope}%
\begin{pgfscope}%
\pgfpathrectangle{\pgfqpoint{1.150000in}{0.150000in}}{\pgfqpoint{5.700000in}{5.700000in}}%
\pgfusepath{clip}%
\pgfsetbuttcap%
\pgfsetroundjoin%
\definecolor{currentfill}{rgb}{0.273809,0.031497,0.358853}%
\pgfsetfillcolor{currentfill}%
\pgfsetfillopacity{0.800000}%
\pgfsetlinewidth{0.000000pt}%
\definecolor{currentstroke}{rgb}{0.000000,0.000000,0.000000}%
\pgfsetstrokecolor{currentstroke}%
\pgfsetdash{}{0pt}%
\pgfpathmoveto{\pgfqpoint{3.154633in}{1.828403in}}%
\pgfpathlineto{\pgfqpoint{3.168274in}{1.820634in}}%
\pgfpathlineto{\pgfqpoint{3.181915in}{1.813088in}}%
\pgfpathlineto{\pgfqpoint{3.195557in}{1.805763in}}%
\pgfpathlineto{\pgfqpoint{3.209199in}{1.798660in}}%
\pgfpathlineto{\pgfqpoint{3.217488in}{1.806486in}}%
\pgfpathlineto{\pgfqpoint{3.225768in}{1.814417in}}%
\pgfpathlineto{\pgfqpoint{3.234041in}{1.822448in}}%
\pgfpathlineto{\pgfqpoint{3.242305in}{1.830577in}}%
\pgfpathlineto{\pgfqpoint{3.228684in}{1.837308in}}%
\pgfpathlineto{\pgfqpoint{3.215063in}{1.844259in}}%
\pgfpathlineto{\pgfqpoint{3.201443in}{1.851433in}}%
\pgfpathlineto{\pgfqpoint{3.187824in}{1.858829in}}%
\pgfpathlineto{\pgfqpoint{3.179539in}{1.851061in}}%
\pgfpathlineto{\pgfqpoint{3.171246in}{1.843399in}}%
\pgfpathlineto{\pgfqpoint{3.162944in}{1.835845in}}%
\pgfpathlineto{\pgfqpoint{3.154633in}{1.828403in}}%
\pgfpathclose%
\pgfusepath{fill}%
\end{pgfscope}%
\begin{pgfscope}%
\pgfpathrectangle{\pgfqpoint{1.150000in}{0.150000in}}{\pgfqpoint{5.700000in}{5.700000in}}%
\pgfusepath{clip}%
\pgfsetbuttcap%
\pgfsetroundjoin%
\definecolor{currentfill}{rgb}{0.180653,0.701402,0.488189}%
\pgfsetfillcolor{currentfill}%
\pgfsetfillopacity{0.800000}%
\pgfsetlinewidth{0.000000pt}%
\definecolor{currentstroke}{rgb}{0.000000,0.000000,0.000000}%
\pgfsetstrokecolor{currentstroke}%
\pgfsetdash{}{0pt}%
\pgfpathmoveto{\pgfqpoint{6.173965in}{3.608146in}}%
\pgfpathlineto{\pgfqpoint{6.188784in}{3.618849in}}%
\pgfpathlineto{\pgfqpoint{6.203623in}{3.629725in}}%
\pgfpathlineto{\pgfqpoint{6.218484in}{3.640773in}}%
\pgfpathlineto{\pgfqpoint{6.225424in}{3.642621in}}%
\pgfpathlineto{\pgfqpoint{6.232366in}{3.644676in}}%
\pgfpathlineto{\pgfqpoint{6.239310in}{3.646945in}}%
\pgfpathlineto{\pgfqpoint{6.224483in}{3.636505in}}%
\pgfpathlineto{\pgfqpoint{6.209678in}{3.626237in}}%
\pgfpathlineto{\pgfqpoint{6.194893in}{3.616141in}}%
\pgfpathlineto{\pgfqpoint{6.187915in}{3.613259in}}%
\pgfpathlineto{\pgfqpoint{6.180939in}{3.610597in}}%
\pgfpathlineto{\pgfqpoint{6.173965in}{3.608146in}}%
\pgfpathclose%
\pgfusepath{fill}%
\end{pgfscope}%
\begin{pgfscope}%
\pgfpathrectangle{\pgfqpoint{1.150000in}{0.150000in}}{\pgfqpoint{5.700000in}{5.700000in}}%
\pgfusepath{clip}%
\pgfsetbuttcap%
\pgfsetroundjoin%
\definecolor{currentfill}{rgb}{0.272594,0.025563,0.353093}%
\pgfsetfillcolor{currentfill}%
\pgfsetfillopacity{0.800000}%
\pgfsetlinewidth{0.000000pt}%
\definecolor{currentstroke}{rgb}{0.000000,0.000000,0.000000}%
\pgfsetstrokecolor{currentstroke}%
\pgfsetdash{}{0pt}%
\pgfpathmoveto{\pgfqpoint{3.296807in}{1.805836in}}%
\pgfpathlineto{\pgfqpoint{3.310437in}{1.800191in}}%
\pgfpathlineto{\pgfqpoint{3.324069in}{1.794761in}}%
\pgfpathlineto{\pgfqpoint{3.337704in}{1.789544in}}%
\pgfpathlineto{\pgfqpoint{3.351342in}{1.784539in}}%
\pgfpathlineto{\pgfqpoint{3.359562in}{1.793465in}}%
\pgfpathlineto{\pgfqpoint{3.367775in}{1.802467in}}%
\pgfpathlineto{\pgfqpoint{3.375981in}{1.811539in}}%
\pgfpathlineto{\pgfqpoint{3.384180in}{1.820681in}}%
\pgfpathlineto{\pgfqpoint{3.370560in}{1.825346in}}%
\pgfpathlineto{\pgfqpoint{3.356942in}{1.830224in}}%
\pgfpathlineto{\pgfqpoint{3.343328in}{1.835314in}}%
\pgfpathlineto{\pgfqpoint{3.329715in}{1.840619in}}%
\pgfpathlineto{\pgfqpoint{3.321499in}{1.831805in}}%
\pgfpathlineto{\pgfqpoint{3.313276in}{1.823068in}}%
\pgfpathlineto{\pgfqpoint{3.305045in}{1.814410in}}%
\pgfpathlineto{\pgfqpoint{3.296807in}{1.805836in}}%
\pgfpathclose%
\pgfusepath{fill}%
\end{pgfscope}%
\begin{pgfscope}%
\pgfpathrectangle{\pgfqpoint{1.150000in}{0.150000in}}{\pgfqpoint{5.700000in}{5.700000in}}%
\pgfusepath{clip}%
\pgfsetbuttcap%
\pgfsetroundjoin%
\definecolor{currentfill}{rgb}{0.276022,0.044167,0.370164}%
\pgfsetfillcolor{currentfill}%
\pgfsetfillopacity{0.800000}%
\pgfsetlinewidth{0.000000pt}%
\definecolor{currentstroke}{rgb}{0.000000,0.000000,0.000000}%
\pgfsetstrokecolor{currentstroke}%
\pgfsetdash{}{0pt}%
\pgfpathmoveto{\pgfqpoint{3.525885in}{1.830547in}}%
\pgfpathlineto{\pgfqpoint{3.539527in}{1.828059in}}%
\pgfpathlineto{\pgfqpoint{3.553173in}{1.825775in}}%
\pgfpathlineto{\pgfqpoint{3.566825in}{1.823694in}}%
\pgfpathlineto{\pgfqpoint{3.580482in}{1.821815in}}%
\pgfpathlineto{\pgfqpoint{3.588608in}{1.832121in}}%
\pgfpathlineto{\pgfqpoint{3.596729in}{1.842456in}}%
\pgfpathlineto{\pgfqpoint{3.604843in}{1.852817in}}%
\pgfpathlineto{\pgfqpoint{3.612953in}{1.863201in}}%
\pgfpathlineto{\pgfqpoint{3.599307in}{1.864804in}}%
\pgfpathlineto{\pgfqpoint{3.585668in}{1.866609in}}%
\pgfpathlineto{\pgfqpoint{3.572033in}{1.868617in}}%
\pgfpathlineto{\pgfqpoint{3.558404in}{1.870828in}}%
\pgfpathlineto{\pgfqpoint{3.550283in}{1.860708in}}%
\pgfpathlineto{\pgfqpoint{3.542157in}{1.850620in}}%
\pgfpathlineto{\pgfqpoint{3.534024in}{1.840565in}}%
\pgfpathlineto{\pgfqpoint{3.525885in}{1.830547in}}%
\pgfpathclose%
\pgfusepath{fill}%
\end{pgfscope}%
\begin{pgfscope}%
\pgfpathrectangle{\pgfqpoint{1.150000in}{0.150000in}}{\pgfqpoint{5.700000in}{5.700000in}}%
\pgfusepath{clip}%
\pgfsetbuttcap%
\pgfsetroundjoin%
\definecolor{currentfill}{rgb}{0.241237,0.296485,0.539709}%
\pgfsetfillcolor{currentfill}%
\pgfsetfillopacity{0.800000}%
\pgfsetlinewidth{0.000000pt}%
\definecolor{currentstroke}{rgb}{0.000000,0.000000,0.000000}%
\pgfsetstrokecolor{currentstroke}%
\pgfsetdash{}{0pt}%
\pgfpathmoveto{\pgfqpoint{4.427029in}{2.389992in}}%
\pgfpathlineto{\pgfqpoint{4.440960in}{2.396532in}}%
\pgfpathlineto{\pgfqpoint{4.454904in}{2.403257in}}%
\pgfpathlineto{\pgfqpoint{4.468861in}{2.410168in}}%
\pgfpathlineto{\pgfqpoint{4.482832in}{2.417265in}}%
\pgfpathlineto{\pgfqpoint{4.490658in}{2.426982in}}%
\pgfpathlineto{\pgfqpoint{4.498479in}{2.436617in}}%
\pgfpathlineto{\pgfqpoint{4.506294in}{2.446172in}}%
\pgfpathlineto{\pgfqpoint{4.514103in}{2.455649in}}%
\pgfpathlineto{\pgfqpoint{4.500139in}{2.448628in}}%
\pgfpathlineto{\pgfqpoint{4.486188in}{2.441793in}}%
\pgfpathlineto{\pgfqpoint{4.472250in}{2.435144in}}%
\pgfpathlineto{\pgfqpoint{4.458325in}{2.428680in}}%
\pgfpathlineto{\pgfqpoint{4.450510in}{2.419115in}}%
\pgfpathlineto{\pgfqpoint{4.442689in}{2.409480in}}%
\pgfpathlineto{\pgfqpoint{4.434862in}{2.399773in}}%
\pgfpathlineto{\pgfqpoint{4.427029in}{2.389992in}}%
\pgfpathclose%
\pgfusepath{fill}%
\end{pgfscope}%
\begin{pgfscope}%
\pgfpathrectangle{\pgfqpoint{1.150000in}{0.150000in}}{\pgfqpoint{5.700000in}{5.700000in}}%
\pgfusepath{clip}%
\pgfsetbuttcap%
\pgfsetroundjoin%
\definecolor{currentfill}{rgb}{0.190631,0.407061,0.556089}%
\pgfsetfillcolor{currentfill}%
\pgfsetfillopacity{0.800000}%
\pgfsetlinewidth{0.000000pt}%
\definecolor{currentstroke}{rgb}{0.000000,0.000000,0.000000}%
\pgfsetstrokecolor{currentstroke}%
\pgfsetdash{}{0pt}%
\pgfpathmoveto{\pgfqpoint{4.806539in}{2.688989in}}%
\pgfpathlineto{\pgfqpoint{4.820662in}{2.697765in}}%
\pgfpathlineto{\pgfqpoint{4.834800in}{2.706724in}}%
\pgfpathlineto{\pgfqpoint{4.848954in}{2.715865in}}%
\pgfpathlineto{\pgfqpoint{4.863124in}{2.725189in}}%
\pgfpathlineto{\pgfqpoint{4.870792in}{2.732694in}}%
\pgfpathlineto{\pgfqpoint{4.878453in}{2.740126in}}%
\pgfpathlineto{\pgfqpoint{4.886107in}{2.747487in}}%
\pgfpathlineto{\pgfqpoint{4.893755in}{2.754781in}}%
\pgfpathlineto{\pgfqpoint{4.879595in}{2.745701in}}%
\pgfpathlineto{\pgfqpoint{4.865452in}{2.736802in}}%
\pgfpathlineto{\pgfqpoint{4.851324in}{2.728086in}}%
\pgfpathlineto{\pgfqpoint{4.837212in}{2.719551in}}%
\pgfpathlineto{\pgfqpoint{4.829554in}{2.712003in}}%
\pgfpathlineto{\pgfqpoint{4.821889in}{2.704395in}}%
\pgfpathlineto{\pgfqpoint{4.814217in}{2.696725in}}%
\pgfpathlineto{\pgfqpoint{4.806539in}{2.688989in}}%
\pgfpathclose%
\pgfusepath{fill}%
\end{pgfscope}%
\begin{pgfscope}%
\pgfpathrectangle{\pgfqpoint{1.150000in}{0.150000in}}{\pgfqpoint{5.700000in}{5.700000in}}%
\pgfusepath{clip}%
\pgfsetbuttcap%
\pgfsetroundjoin%
\definecolor{currentfill}{rgb}{0.144759,0.519093,0.556572}%
\pgfsetfillcolor{currentfill}%
\pgfsetfillopacity{0.800000}%
\pgfsetlinewidth{0.000000pt}%
\definecolor{currentstroke}{rgb}{0.000000,0.000000,0.000000}%
\pgfsetstrokecolor{currentstroke}%
\pgfsetdash{}{0pt}%
\pgfpathmoveto{\pgfqpoint{5.273004in}{3.034400in}}%
\pgfpathlineto{\pgfqpoint{5.287378in}{3.044830in}}%
\pgfpathlineto{\pgfqpoint{5.301771in}{3.055439in}}%
\pgfpathlineto{\pgfqpoint{5.316182in}{3.066228in}}%
\pgfpathlineto{\pgfqpoint{5.330611in}{3.077196in}}%
\pgfpathlineto{\pgfqpoint{5.338037in}{3.081773in}}%
\pgfpathlineto{\pgfqpoint{5.345456in}{3.086330in}}%
\pgfpathlineto{\pgfqpoint{5.352868in}{3.090872in}}%
\pgfpathlineto{\pgfqpoint{5.360274in}{3.095406in}}%
\pgfpathlineto{\pgfqpoint{5.345865in}{3.084885in}}%
\pgfpathlineto{\pgfqpoint{5.331474in}{3.074542in}}%
\pgfpathlineto{\pgfqpoint{5.317102in}{3.064378in}}%
\pgfpathlineto{\pgfqpoint{5.302747in}{3.054392in}}%
\pgfpathlineto{\pgfqpoint{5.295321in}{3.049402in}}%
\pgfpathlineto{\pgfqpoint{5.287888in}{3.044409in}}%
\pgfpathlineto{\pgfqpoint{5.280449in}{3.039410in}}%
\pgfpathlineto{\pgfqpoint{5.273004in}{3.034400in}}%
\pgfpathclose%
\pgfusepath{fill}%
\end{pgfscope}%
\begin{pgfscope}%
\pgfpathrectangle{\pgfqpoint{1.150000in}{0.150000in}}{\pgfqpoint{5.700000in}{5.700000in}}%
\pgfusepath{clip}%
\pgfsetbuttcap%
\pgfsetroundjoin%
\definecolor{currentfill}{rgb}{0.278826,0.175490,0.483397}%
\pgfsetfillcolor{currentfill}%
\pgfsetfillopacity{0.800000}%
\pgfsetlinewidth{0.000000pt}%
\definecolor{currentstroke}{rgb}{0.000000,0.000000,0.000000}%
\pgfsetstrokecolor{currentstroke}%
\pgfsetdash{}{0pt}%
\pgfpathmoveto{\pgfqpoint{4.047550in}{2.100549in}}%
\pgfpathlineto{\pgfqpoint{4.061321in}{2.103966in}}%
\pgfpathlineto{\pgfqpoint{4.075103in}{2.107574in}}%
\pgfpathlineto{\pgfqpoint{4.088895in}{2.111372in}}%
\pgfpathlineto{\pgfqpoint{4.102698in}{2.115361in}}%
\pgfpathlineto{\pgfqpoint{4.110654in}{2.126430in}}%
\pgfpathlineto{\pgfqpoint{4.118604in}{2.137445in}}%
\pgfpathlineto{\pgfqpoint{4.126550in}{2.148404in}}%
\pgfpathlineto{\pgfqpoint{4.134490in}{2.159306in}}%
\pgfpathlineto{\pgfqpoint{4.120693in}{2.155231in}}%
\pgfpathlineto{\pgfqpoint{4.106907in}{2.151346in}}%
\pgfpathlineto{\pgfqpoint{4.093130in}{2.147652in}}%
\pgfpathlineto{\pgfqpoint{4.079364in}{2.144148in}}%
\pgfpathlineto{\pgfqpoint{4.071418in}{2.133320in}}%
\pgfpathlineto{\pgfqpoint{4.063467in}{2.122444in}}%
\pgfpathlineto{\pgfqpoint{4.055511in}{2.111520in}}%
\pgfpathlineto{\pgfqpoint{4.047550in}{2.100549in}}%
\pgfpathclose%
\pgfusepath{fill}%
\end{pgfscope}%
\begin{pgfscope}%
\pgfpathrectangle{\pgfqpoint{1.150000in}{0.150000in}}{\pgfqpoint{5.700000in}{5.700000in}}%
\pgfusepath{clip}%
\pgfsetbuttcap%
\pgfsetroundjoin%
\definecolor{currentfill}{rgb}{0.206756,0.371758,0.553117}%
\pgfsetfillcolor{currentfill}%
\pgfsetfillopacity{0.800000}%
\pgfsetlinewidth{0.000000pt}%
\definecolor{currentstroke}{rgb}{0.000000,0.000000,0.000000}%
\pgfsetstrokecolor{currentstroke}%
\pgfsetdash{}{0pt}%
\pgfpathmoveto{\pgfqpoint{2.349199in}{2.663768in}}%
\pgfpathlineto{\pgfqpoint{2.363241in}{2.640257in}}%
\pgfpathlineto{\pgfqpoint{2.377269in}{2.617088in}}%
\pgfpathlineto{\pgfqpoint{2.391282in}{2.594258in}}%
\pgfpathlineto{\pgfqpoint{2.405281in}{2.571764in}}%
\pgfpathlineto{\pgfqpoint{2.414067in}{2.573135in}}%
\pgfpathlineto{\pgfqpoint{2.422836in}{2.574742in}}%
\pgfpathlineto{\pgfqpoint{2.431590in}{2.576581in}}%
\pgfpathlineto{\pgfqpoint{2.440328in}{2.578649in}}%
\pgfpathlineto{\pgfqpoint{2.426373in}{2.600706in}}%
\pgfpathlineto{\pgfqpoint{2.412405in}{2.623099in}}%
\pgfpathlineto{\pgfqpoint{2.398422in}{2.645829in}}%
\pgfpathlineto{\pgfqpoint{2.384425in}{2.668901in}}%
\pgfpathlineto{\pgfqpoint{2.375643in}{2.667258in}}%
\pgfpathlineto{\pgfqpoint{2.366845in}{2.665853in}}%
\pgfpathlineto{\pgfqpoint{2.358030in}{2.664688in}}%
\pgfpathlineto{\pgfqpoint{2.349199in}{2.663768in}}%
\pgfpathclose%
\pgfusepath{fill}%
\end{pgfscope}%
\begin{pgfscope}%
\pgfpathrectangle{\pgfqpoint{1.150000in}{0.150000in}}{\pgfqpoint{5.700000in}{5.700000in}}%
\pgfusepath{clip}%
\pgfsetbuttcap%
\pgfsetroundjoin%
\definecolor{currentfill}{rgb}{0.277941,0.056324,0.381191}%
\pgfsetfillcolor{currentfill}%
\pgfsetfillopacity{0.800000}%
\pgfsetlinewidth{0.000000pt}%
\definecolor{currentstroke}{rgb}{0.000000,0.000000,0.000000}%
\pgfsetstrokecolor{currentstroke}%
\pgfsetdash{}{0pt}%
\pgfpathmoveto{\pgfqpoint{3.011971in}{1.873417in}}%
\pgfpathlineto{\pgfqpoint{3.025642in}{1.863410in}}%
\pgfpathlineto{\pgfqpoint{3.039311in}{1.853637in}}%
\pgfpathlineto{\pgfqpoint{3.052979in}{1.844096in}}%
\pgfpathlineto{\pgfqpoint{3.066646in}{1.834787in}}%
\pgfpathlineto{\pgfqpoint{3.075016in}{1.841325in}}%
\pgfpathlineto{\pgfqpoint{3.083377in}{1.847999in}}%
\pgfpathlineto{\pgfqpoint{3.091728in}{1.854804in}}%
\pgfpathlineto{\pgfqpoint{3.100071in}{1.861736in}}%
\pgfpathlineto{\pgfqpoint{3.086429in}{1.870639in}}%
\pgfpathlineto{\pgfqpoint{3.072787in}{1.879773in}}%
\pgfpathlineto{\pgfqpoint{3.059143in}{1.889139in}}%
\pgfpathlineto{\pgfqpoint{3.045498in}{1.898739in}}%
\pgfpathlineto{\pgfqpoint{3.037131in}{1.892201in}}%
\pgfpathlineto{\pgfqpoint{3.028754in}{1.885799in}}%
\pgfpathlineto{\pgfqpoint{3.020368in}{1.879536in}}%
\pgfpathlineto{\pgfqpoint{3.011971in}{1.873417in}}%
\pgfpathclose%
\pgfusepath{fill}%
\end{pgfscope}%
\begin{pgfscope}%
\pgfpathrectangle{\pgfqpoint{1.150000in}{0.150000in}}{\pgfqpoint{5.700000in}{5.700000in}}%
\pgfusepath{clip}%
\pgfsetbuttcap%
\pgfsetroundjoin%
\definecolor{currentfill}{rgb}{0.136408,0.541173,0.554483}%
\pgfsetfillcolor{currentfill}%
\pgfsetfillopacity{0.800000}%
\pgfsetlinewidth{0.000000pt}%
\definecolor{currentstroke}{rgb}{0.000000,0.000000,0.000000}%
\pgfsetstrokecolor{currentstroke}%
\pgfsetdash{}{0pt}%
\pgfpathmoveto{\pgfqpoint{5.360274in}{3.095406in}}%
\pgfpathlineto{\pgfqpoint{5.374701in}{3.106105in}}%
\pgfpathlineto{\pgfqpoint{5.389147in}{3.116983in}}%
\pgfpathlineto{\pgfqpoint{5.403612in}{3.128040in}}%
\pgfpathlineto{\pgfqpoint{5.418095in}{3.139276in}}%
\pgfpathlineto{\pgfqpoint{5.425473in}{3.143336in}}%
\pgfpathlineto{\pgfqpoint{5.432844in}{3.147390in}}%
\pgfpathlineto{\pgfqpoint{5.440209in}{3.151441in}}%
\pgfpathlineto{\pgfqpoint{5.447567in}{3.155497in}}%
\pgfpathlineto{\pgfqpoint{5.433106in}{3.144742in}}%
\pgfpathlineto{\pgfqpoint{5.418664in}{3.134166in}}%
\pgfpathlineto{\pgfqpoint{5.404240in}{3.123767in}}%
\pgfpathlineto{\pgfqpoint{5.389835in}{3.113546in}}%
\pgfpathlineto{\pgfqpoint{5.382454in}{3.108999in}}%
\pgfpathlineto{\pgfqpoint{5.375067in}{3.104464in}}%
\pgfpathlineto{\pgfqpoint{5.367673in}{3.099934in}}%
\pgfpathlineto{\pgfqpoint{5.360274in}{3.095406in}}%
\pgfpathclose%
\pgfusepath{fill}%
\end{pgfscope}%
\begin{pgfscope}%
\pgfpathrectangle{\pgfqpoint{1.150000in}{0.150000in}}{\pgfqpoint{5.700000in}{5.700000in}}%
\pgfusepath{clip}%
\pgfsetbuttcap%
\pgfsetroundjoin%
\definecolor{currentfill}{rgb}{0.273809,0.031497,0.358853}%
\pgfsetfillcolor{currentfill}%
\pgfsetfillopacity{0.800000}%
\pgfsetlinewidth{0.000000pt}%
\definecolor{currentstroke}{rgb}{0.000000,0.000000,0.000000}%
\pgfsetstrokecolor{currentstroke}%
\pgfsetdash{}{0pt}%
\pgfpathmoveto{\pgfqpoint{3.438695in}{1.804124in}}%
\pgfpathlineto{\pgfqpoint{3.452333in}{1.800506in}}%
\pgfpathlineto{\pgfqpoint{3.465975in}{1.797094in}}%
\pgfpathlineto{\pgfqpoint{3.479621in}{1.793889in}}%
\pgfpathlineto{\pgfqpoint{3.493271in}{1.790889in}}%
\pgfpathlineto{\pgfqpoint{3.501434in}{1.800736in}}%
\pgfpathlineto{\pgfqpoint{3.509590in}{1.810630in}}%
\pgfpathlineto{\pgfqpoint{3.517741in}{1.820568in}}%
\pgfpathlineto{\pgfqpoint{3.525885in}{1.830547in}}%
\pgfpathlineto{\pgfqpoint{3.512249in}{1.833239in}}%
\pgfpathlineto{\pgfqpoint{3.498617in}{1.836137in}}%
\pgfpathlineto{\pgfqpoint{3.484989in}{1.839240in}}%
\pgfpathlineto{\pgfqpoint{3.471366in}{1.842550in}}%
\pgfpathlineto{\pgfqpoint{3.463207in}{1.832867in}}%
\pgfpathlineto{\pgfqpoint{3.455043in}{1.823233in}}%
\pgfpathlineto{\pgfqpoint{3.446872in}{1.813651in}}%
\pgfpathlineto{\pgfqpoint{3.438695in}{1.804124in}}%
\pgfpathclose%
\pgfusepath{fill}%
\end{pgfscope}%
\begin{pgfscope}%
\pgfpathrectangle{\pgfqpoint{1.150000in}{0.150000in}}{\pgfqpoint{5.700000in}{5.700000in}}%
\pgfusepath{clip}%
\pgfsetbuttcap%
\pgfsetroundjoin%
\definecolor{currentfill}{rgb}{0.273006,0.204520,0.501721}%
\pgfsetfillcolor{currentfill}%
\pgfsetfillopacity{0.800000}%
\pgfsetlinewidth{0.000000pt}%
\definecolor{currentstroke}{rgb}{0.000000,0.000000,0.000000}%
\pgfsetstrokecolor{currentstroke}%
\pgfsetdash{}{0pt}%
\pgfpathmoveto{\pgfqpoint{4.134490in}{2.159306in}}%
\pgfpathlineto{\pgfqpoint{4.148299in}{2.163570in}}%
\pgfpathlineto{\pgfqpoint{4.162118in}{2.168024in}}%
\pgfpathlineto{\pgfqpoint{4.175948in}{2.172667in}}%
\pgfpathlineto{\pgfqpoint{4.189789in}{2.177498in}}%
\pgfpathlineto{\pgfqpoint{4.197719in}{2.188411in}}%
\pgfpathlineto{\pgfqpoint{4.205644in}{2.199258in}}%
\pgfpathlineto{\pgfqpoint{4.213564in}{2.210041in}}%
\pgfpathlineto{\pgfqpoint{4.221479in}{2.220759in}}%
\pgfpathlineto{\pgfqpoint{4.207643in}{2.215873in}}%
\pgfpathlineto{\pgfqpoint{4.193818in}{2.211176in}}%
\pgfpathlineto{\pgfqpoint{4.180004in}{2.206668in}}%
\pgfpathlineto{\pgfqpoint{4.166201in}{2.202349in}}%
\pgfpathlineto{\pgfqpoint{4.158281in}{2.191673in}}%
\pgfpathlineto{\pgfqpoint{4.150356in}{2.180941in}}%
\pgfpathlineto{\pgfqpoint{4.142426in}{2.170152in}}%
\pgfpathlineto{\pgfqpoint{4.134490in}{2.159306in}}%
\pgfpathclose%
\pgfusepath{fill}%
\end{pgfscope}%
\begin{pgfscope}%
\pgfpathrectangle{\pgfqpoint{1.150000in}{0.150000in}}{\pgfqpoint{5.700000in}{5.700000in}}%
\pgfusepath{clip}%
\pgfsetbuttcap%
\pgfsetroundjoin%
\definecolor{currentfill}{rgb}{0.227802,0.326594,0.546532}%
\pgfsetfillcolor{currentfill}%
\pgfsetfillopacity{0.800000}%
\pgfsetlinewidth{0.000000pt}%
\definecolor{currentstroke}{rgb}{0.000000,0.000000,0.000000}%
\pgfsetstrokecolor{currentstroke}%
\pgfsetdash{}{0pt}%
\pgfpathmoveto{\pgfqpoint{4.514103in}{2.455649in}}%
\pgfpathlineto{\pgfqpoint{4.528081in}{2.462855in}}%
\pgfpathlineto{\pgfqpoint{4.542073in}{2.470246in}}%
\pgfpathlineto{\pgfqpoint{4.556079in}{2.477821in}}%
\pgfpathlineto{\pgfqpoint{4.570100in}{2.485582in}}%
\pgfpathlineto{\pgfqpoint{4.577896in}{2.494884in}}%
\pgfpathlineto{\pgfqpoint{4.585687in}{2.504103in}}%
\pgfpathlineto{\pgfqpoint{4.593471in}{2.513239in}}%
\pgfpathlineto{\pgfqpoint{4.601249in}{2.522294in}}%
\pgfpathlineto{\pgfqpoint{4.587236in}{2.514643in}}%
\pgfpathlineto{\pgfqpoint{4.573236in}{2.507176in}}%
\pgfpathlineto{\pgfqpoint{4.559251in}{2.499894in}}%
\pgfpathlineto{\pgfqpoint{4.545279in}{2.492797in}}%
\pgfpathlineto{\pgfqpoint{4.537494in}{2.483621in}}%
\pgfpathlineto{\pgfqpoint{4.529703in}{2.474372in}}%
\pgfpathlineto{\pgfqpoint{4.521906in}{2.465048in}}%
\pgfpathlineto{\pgfqpoint{4.514103in}{2.455649in}}%
\pgfpathclose%
\pgfusepath{fill}%
\end{pgfscope}%
\begin{pgfscope}%
\pgfpathrectangle{\pgfqpoint{1.150000in}{0.150000in}}{\pgfqpoint{5.700000in}{5.700000in}}%
\pgfusepath{clip}%
\pgfsetbuttcap%
\pgfsetroundjoin%
\definecolor{currentfill}{rgb}{0.180629,0.429975,0.557282}%
\pgfsetfillcolor{currentfill}%
\pgfsetfillopacity{0.800000}%
\pgfsetlinewidth{0.000000pt}%
\definecolor{currentstroke}{rgb}{0.000000,0.000000,0.000000}%
\pgfsetstrokecolor{currentstroke}%
\pgfsetdash{}{0pt}%
\pgfpathmoveto{\pgfqpoint{4.893755in}{2.754781in}}%
\pgfpathlineto{\pgfqpoint{4.907930in}{2.764044in}}%
\pgfpathlineto{\pgfqpoint{4.922122in}{2.773489in}}%
\pgfpathlineto{\pgfqpoint{4.936330in}{2.783115in}}%
\pgfpathlineto{\pgfqpoint{4.950555in}{2.792924in}}%
\pgfpathlineto{\pgfqpoint{4.958184in}{2.799889in}}%
\pgfpathlineto{\pgfqpoint{4.965806in}{2.806785in}}%
\pgfpathlineto{\pgfqpoint{4.973421in}{2.813615in}}%
\pgfpathlineto{\pgfqpoint{4.981029in}{2.820382in}}%
\pgfpathlineto{\pgfqpoint{4.966817in}{2.810851in}}%
\pgfpathlineto{\pgfqpoint{4.952621in}{2.801501in}}%
\pgfpathlineto{\pgfqpoint{4.938441in}{2.792333in}}%
\pgfpathlineto{\pgfqpoint{4.924277in}{2.783346in}}%
\pgfpathlineto{\pgfqpoint{4.916656in}{2.776290in}}%
\pgfpathlineto{\pgfqpoint{4.909029in}{2.769180in}}%
\pgfpathlineto{\pgfqpoint{4.901395in}{2.762011in}}%
\pgfpathlineto{\pgfqpoint{4.893755in}{2.754781in}}%
\pgfpathclose%
\pgfusepath{fill}%
\end{pgfscope}%
\begin{pgfscope}%
\pgfpathrectangle{\pgfqpoint{1.150000in}{0.150000in}}{\pgfqpoint{5.700000in}{5.700000in}}%
\pgfusepath{clip}%
\pgfsetbuttcap%
\pgfsetroundjoin%
\definecolor{currentfill}{rgb}{0.276194,0.190074,0.493001}%
\pgfsetfillcolor{currentfill}%
\pgfsetfillopacity{0.800000}%
\pgfsetlinewidth{0.000000pt}%
\definecolor{currentstroke}{rgb}{0.000000,0.000000,0.000000}%
\pgfsetstrokecolor{currentstroke}%
\pgfsetdash{}{0pt}%
\pgfpathmoveto{\pgfqpoint{2.648338in}{2.175543in}}%
\pgfpathlineto{\pgfqpoint{2.662153in}{2.158981in}}%
\pgfpathlineto{\pgfqpoint{2.675961in}{2.142695in}}%
\pgfpathlineto{\pgfqpoint{2.689760in}{2.126685in}}%
\pgfpathlineto{\pgfqpoint{2.703553in}{2.110947in}}%
\pgfpathlineto{\pgfqpoint{2.712154in}{2.114168in}}%
\pgfpathlineto{\pgfqpoint{2.720742in}{2.117590in}}%
\pgfpathlineto{\pgfqpoint{2.729317in}{2.121208in}}%
\pgfpathlineto{\pgfqpoint{2.737878in}{2.125020in}}%
\pgfpathlineto{\pgfqpoint{2.724122in}{2.140307in}}%
\pgfpathlineto{\pgfqpoint{2.710358in}{2.155866in}}%
\pgfpathlineto{\pgfqpoint{2.696588in}{2.171700in}}%
\pgfpathlineto{\pgfqpoint{2.682810in}{2.187809in}}%
\pgfpathlineto{\pgfqpoint{2.674213in}{2.184437in}}%
\pgfpathlineto{\pgfqpoint{2.665602in}{2.181265in}}%
\pgfpathlineto{\pgfqpoint{2.656977in}{2.178299in}}%
\pgfpathlineto{\pgfqpoint{2.648338in}{2.175543in}}%
\pgfpathclose%
\pgfusepath{fill}%
\end{pgfscope}%
\begin{pgfscope}%
\pgfpathrectangle{\pgfqpoint{1.150000in}{0.150000in}}{\pgfqpoint{5.700000in}{5.700000in}}%
\pgfusepath{clip}%
\pgfsetbuttcap%
\pgfsetroundjoin%
\definecolor{currentfill}{rgb}{0.280868,0.160771,0.472899}%
\pgfsetfillcolor{currentfill}%
\pgfsetfillopacity{0.800000}%
\pgfsetlinewidth{0.000000pt}%
\definecolor{currentstroke}{rgb}{0.000000,0.000000,0.000000}%
\pgfsetstrokecolor{currentstroke}%
\pgfsetdash{}{0pt}%
\pgfpathmoveto{\pgfqpoint{2.703553in}{2.110947in}}%
\pgfpathlineto{\pgfqpoint{2.717339in}{2.095480in}}%
\pgfpathlineto{\pgfqpoint{2.731118in}{2.080281in}}%
\pgfpathlineto{\pgfqpoint{2.744891in}{2.065349in}}%
\pgfpathlineto{\pgfqpoint{2.758657in}{2.050681in}}%
\pgfpathlineto{\pgfqpoint{2.767222in}{2.054363in}}%
\pgfpathlineto{\pgfqpoint{2.775775in}{2.058238in}}%
\pgfpathlineto{\pgfqpoint{2.784315in}{2.062301in}}%
\pgfpathlineto{\pgfqpoint{2.792842in}{2.066549in}}%
\pgfpathlineto{\pgfqpoint{2.779110in}{2.080769in}}%
\pgfpathlineto{\pgfqpoint{2.765372in}{2.095253in}}%
\pgfpathlineto{\pgfqpoint{2.751629in}{2.110002in}}%
\pgfpathlineto{\pgfqpoint{2.737878in}{2.125020in}}%
\pgfpathlineto{\pgfqpoint{2.729317in}{2.121208in}}%
\pgfpathlineto{\pgfqpoint{2.720742in}{2.117590in}}%
\pgfpathlineto{\pgfqpoint{2.712154in}{2.114168in}}%
\pgfpathlineto{\pgfqpoint{2.703553in}{2.110947in}}%
\pgfpathclose%
\pgfusepath{fill}%
\end{pgfscope}%
\begin{pgfscope}%
\pgfpathrectangle{\pgfqpoint{1.150000in}{0.150000in}}{\pgfqpoint{5.700000in}{5.700000in}}%
\pgfusepath{clip}%
\pgfsetbuttcap%
\pgfsetroundjoin%
\definecolor{currentfill}{rgb}{0.269308,0.218818,0.509577}%
\pgfsetfillcolor{currentfill}%
\pgfsetfillopacity{0.800000}%
\pgfsetlinewidth{0.000000pt}%
\definecolor{currentstroke}{rgb}{0.000000,0.000000,0.000000}%
\pgfsetstrokecolor{currentstroke}%
\pgfsetdash{}{0pt}%
\pgfpathmoveto{\pgfqpoint{2.592996in}{2.244608in}}%
\pgfpathlineto{\pgfqpoint{2.606844in}{2.226914in}}%
\pgfpathlineto{\pgfqpoint{2.620684in}{2.209508in}}%
\pgfpathlineto{\pgfqpoint{2.634515in}{2.192385in}}%
\pgfpathlineto{\pgfqpoint{2.648338in}{2.175543in}}%
\pgfpathlineto{\pgfqpoint{2.656977in}{2.178299in}}%
\pgfpathlineto{\pgfqpoint{2.665602in}{2.181265in}}%
\pgfpathlineto{\pgfqpoint{2.674213in}{2.184437in}}%
\pgfpathlineto{\pgfqpoint{2.682810in}{2.187809in}}%
\pgfpathlineto{\pgfqpoint{2.669025in}{2.204197in}}%
\pgfpathlineto{\pgfqpoint{2.655232in}{2.220866in}}%
\pgfpathlineto{\pgfqpoint{2.641431in}{2.237818in}}%
\pgfpathlineto{\pgfqpoint{2.627621in}{2.255056in}}%
\pgfpathlineto{\pgfqpoint{2.618986in}{2.252125in}}%
\pgfpathlineto{\pgfqpoint{2.610337in}{2.249404in}}%
\pgfpathlineto{\pgfqpoint{2.601673in}{2.246896in}}%
\pgfpathlineto{\pgfqpoint{2.592996in}{2.244608in}}%
\pgfpathclose%
\pgfusepath{fill}%
\end{pgfscope}%
\begin{pgfscope}%
\pgfpathrectangle{\pgfqpoint{1.150000in}{0.150000in}}{\pgfqpoint{5.700000in}{5.700000in}}%
\pgfusepath{clip}%
\pgfsetbuttcap%
\pgfsetroundjoin%
\definecolor{currentfill}{rgb}{0.128729,0.563265,0.551229}%
\pgfsetfillcolor{currentfill}%
\pgfsetfillopacity{0.800000}%
\pgfsetlinewidth{0.000000pt}%
\definecolor{currentstroke}{rgb}{0.000000,0.000000,0.000000}%
\pgfsetstrokecolor{currentstroke}%
\pgfsetdash{}{0pt}%
\pgfpathmoveto{\pgfqpoint{5.447567in}{3.155497in}}%
\pgfpathlineto{\pgfqpoint{5.462047in}{3.166429in}}%
\pgfpathlineto{\pgfqpoint{5.476545in}{3.177540in}}%
\pgfpathlineto{\pgfqpoint{5.491063in}{3.188829in}}%
\pgfpathlineto{\pgfqpoint{5.505600in}{3.200296in}}%
\pgfpathlineto{\pgfqpoint{5.512928in}{3.203857in}}%
\pgfpathlineto{\pgfqpoint{5.520251in}{3.207426in}}%
\pgfpathlineto{\pgfqpoint{5.527567in}{3.211007in}}%
\pgfpathlineto{\pgfqpoint{5.534877in}{3.214607in}}%
\pgfpathlineto{\pgfqpoint{5.520364in}{3.203656in}}%
\pgfpathlineto{\pgfqpoint{5.505871in}{3.192882in}}%
\pgfpathlineto{\pgfqpoint{5.491397in}{3.182286in}}%
\pgfpathlineto{\pgfqpoint{5.476941in}{3.171866in}}%
\pgfpathlineto{\pgfqpoint{5.469606in}{3.167740in}}%
\pgfpathlineto{\pgfqpoint{5.462266in}{3.163641in}}%
\pgfpathlineto{\pgfqpoint{5.454919in}{3.159561in}}%
\pgfpathlineto{\pgfqpoint{5.447567in}{3.155497in}}%
\pgfpathclose%
\pgfusepath{fill}%
\end{pgfscope}%
\begin{pgfscope}%
\pgfpathrectangle{\pgfqpoint{1.150000in}{0.150000in}}{\pgfqpoint{5.700000in}{5.700000in}}%
\pgfusepath{clip}%
\pgfsetbuttcap%
\pgfsetroundjoin%
\definecolor{currentfill}{rgb}{0.272594,0.025563,0.353093}%
\pgfsetfillcolor{currentfill}%
\pgfsetfillopacity{0.800000}%
\pgfsetlinewidth{0.000000pt}%
\definecolor{currentstroke}{rgb}{0.000000,0.000000,0.000000}%
\pgfsetstrokecolor{currentstroke}%
\pgfsetdash{}{0pt}%
\pgfpathmoveto{\pgfqpoint{3.209199in}{1.798660in}}%
\pgfpathlineto{\pgfqpoint{3.222842in}{1.791776in}}%
\pgfpathlineto{\pgfqpoint{3.236487in}{1.785110in}}%
\pgfpathlineto{\pgfqpoint{3.250133in}{1.778662in}}%
\pgfpathlineto{\pgfqpoint{3.263780in}{1.772430in}}%
\pgfpathlineto{\pgfqpoint{3.272048in}{1.780641in}}%
\pgfpathlineto{\pgfqpoint{3.280309in}{1.788948in}}%
\pgfpathlineto{\pgfqpoint{3.288562in}{1.797347in}}%
\pgfpathlineto{\pgfqpoint{3.296807in}{1.805836in}}%
\pgfpathlineto{\pgfqpoint{3.283179in}{1.811696in}}%
\pgfpathlineto{\pgfqpoint{3.269553in}{1.817772in}}%
\pgfpathlineto{\pgfqpoint{3.255928in}{1.824065in}}%
\pgfpathlineto{\pgfqpoint{3.242305in}{1.830577in}}%
\pgfpathlineto{\pgfqpoint{3.234041in}{1.822448in}}%
\pgfpathlineto{\pgfqpoint{3.225768in}{1.814417in}}%
\pgfpathlineto{\pgfqpoint{3.217488in}{1.806486in}}%
\pgfpathlineto{\pgfqpoint{3.209199in}{1.798660in}}%
\pgfpathclose%
\pgfusepath{fill}%
\end{pgfscope}%
\begin{pgfscope}%
\pgfpathrectangle{\pgfqpoint{1.150000in}{0.150000in}}{\pgfqpoint{5.700000in}{5.700000in}}%
\pgfusepath{clip}%
\pgfsetbuttcap%
\pgfsetroundjoin%
\definecolor{currentfill}{rgb}{0.265145,0.232956,0.516599}%
\pgfsetfillcolor{currentfill}%
\pgfsetfillopacity{0.800000}%
\pgfsetlinewidth{0.000000pt}%
\definecolor{currentstroke}{rgb}{0.000000,0.000000,0.000000}%
\pgfsetstrokecolor{currentstroke}%
\pgfsetdash{}{0pt}%
\pgfpathmoveto{\pgfqpoint{4.221479in}{2.220759in}}%
\pgfpathlineto{\pgfqpoint{4.235327in}{2.225833in}}%
\pgfpathlineto{\pgfqpoint{4.249186in}{2.231095in}}%
\pgfpathlineto{\pgfqpoint{4.263057in}{2.236545in}}%
\pgfpathlineto{\pgfqpoint{4.276941in}{2.242182in}}%
\pgfpathlineto{\pgfqpoint{4.284845in}{2.252869in}}%
\pgfpathlineto{\pgfqpoint{4.292744in}{2.263483in}}%
\pgfpathlineto{\pgfqpoint{4.300638in}{2.274025in}}%
\pgfpathlineto{\pgfqpoint{4.308526in}{2.284493in}}%
\pgfpathlineto{\pgfqpoint{4.294648in}{2.278834in}}%
\pgfpathlineto{\pgfqpoint{4.280782in}{2.273362in}}%
\pgfpathlineto{\pgfqpoint{4.266927in}{2.268078in}}%
\pgfpathlineto{\pgfqpoint{4.253085in}{2.262981in}}%
\pgfpathlineto{\pgfqpoint{4.245191in}{2.252522in}}%
\pgfpathlineto{\pgfqpoint{4.237292in}{2.241999in}}%
\pgfpathlineto{\pgfqpoint{4.229388in}{2.231411in}}%
\pgfpathlineto{\pgfqpoint{4.221479in}{2.220759in}}%
\pgfpathclose%
\pgfusepath{fill}%
\end{pgfscope}%
\begin{pgfscope}%
\pgfpathrectangle{\pgfqpoint{1.150000in}{0.150000in}}{\pgfqpoint{5.700000in}{5.700000in}}%
\pgfusepath{clip}%
\pgfsetbuttcap%
\pgfsetroundjoin%
\definecolor{currentfill}{rgb}{0.282884,0.135920,0.453427}%
\pgfsetfillcolor{currentfill}%
\pgfsetfillopacity{0.800000}%
\pgfsetlinewidth{0.000000pt}%
\definecolor{currentstroke}{rgb}{0.000000,0.000000,0.000000}%
\pgfsetstrokecolor{currentstroke}%
\pgfsetdash{}{0pt}%
\pgfpathmoveto{\pgfqpoint{2.758657in}{2.050681in}}%
\pgfpathlineto{\pgfqpoint{2.772418in}{2.036275in}}%
\pgfpathlineto{\pgfqpoint{2.786173in}{2.022130in}}%
\pgfpathlineto{\pgfqpoint{2.799922in}{2.008244in}}%
\pgfpathlineto{\pgfqpoint{2.813667in}{1.994614in}}%
\pgfpathlineto{\pgfqpoint{2.822198in}{1.998755in}}%
\pgfpathlineto{\pgfqpoint{2.830716in}{2.003081in}}%
\pgfpathlineto{\pgfqpoint{2.839223in}{2.007586in}}%
\pgfpathlineto{\pgfqpoint{2.847718in}{2.012268in}}%
\pgfpathlineto{\pgfqpoint{2.834007in}{2.025452in}}%
\pgfpathlineto{\pgfqpoint{2.820290in}{2.038892in}}%
\pgfpathlineto{\pgfqpoint{2.806569in}{2.052591in}}%
\pgfpathlineto{\pgfqpoint{2.792842in}{2.066549in}}%
\pgfpathlineto{\pgfqpoint{2.784315in}{2.062301in}}%
\pgfpathlineto{\pgfqpoint{2.775775in}{2.058238in}}%
\pgfpathlineto{\pgfqpoint{2.767222in}{2.054363in}}%
\pgfpathlineto{\pgfqpoint{2.758657in}{2.050681in}}%
\pgfpathclose%
\pgfusepath{fill}%
\end{pgfscope}%
\begin{pgfscope}%
\pgfpathrectangle{\pgfqpoint{1.150000in}{0.150000in}}{\pgfqpoint{5.700000in}{5.700000in}}%
\pgfusepath{clip}%
\pgfsetbuttcap%
\pgfsetroundjoin%
\definecolor{currentfill}{rgb}{0.190631,0.407061,0.556089}%
\pgfsetfillcolor{currentfill}%
\pgfsetfillopacity{0.800000}%
\pgfsetlinewidth{0.000000pt}%
\definecolor{currentstroke}{rgb}{0.000000,0.000000,0.000000}%
\pgfsetstrokecolor{currentstroke}%
\pgfsetdash{}{0pt}%
\pgfpathmoveto{\pgfqpoint{2.292874in}{2.761311in}}%
\pgfpathlineto{\pgfqpoint{2.306980in}{2.736393in}}%
\pgfpathlineto{\pgfqpoint{2.321068in}{2.711833in}}%
\pgfpathlineto{\pgfqpoint{2.335141in}{2.687626in}}%
\pgfpathlineto{\pgfqpoint{2.349199in}{2.663768in}}%
\pgfpathlineto{\pgfqpoint{2.358030in}{2.664688in}}%
\pgfpathlineto{\pgfqpoint{2.366845in}{2.665853in}}%
\pgfpathlineto{\pgfqpoint{2.375643in}{2.667258in}}%
\pgfpathlineto{\pgfqpoint{2.384425in}{2.668901in}}%
\pgfpathlineto{\pgfqpoint{2.370413in}{2.692318in}}%
\pgfpathlineto{\pgfqpoint{2.356387in}{2.716083in}}%
\pgfpathlineto{\pgfqpoint{2.342345in}{2.740200in}}%
\pgfpathlineto{\pgfqpoint{2.328287in}{2.764673in}}%
\pgfpathlineto{\pgfqpoint{2.319459in}{2.763459in}}%
\pgfpathlineto{\pgfqpoint{2.310615in}{2.762492in}}%
\pgfpathlineto{\pgfqpoint{2.301753in}{2.761774in}}%
\pgfpathlineto{\pgfqpoint{2.292874in}{2.761311in}}%
\pgfpathclose%
\pgfusepath{fill}%
\end{pgfscope}%
\begin{pgfscope}%
\pgfpathrectangle{\pgfqpoint{1.150000in}{0.150000in}}{\pgfqpoint{5.700000in}{5.700000in}}%
\pgfusepath{clip}%
\pgfsetbuttcap%
\pgfsetroundjoin%
\definecolor{currentfill}{rgb}{0.260571,0.246922,0.522828}%
\pgfsetfillcolor{currentfill}%
\pgfsetfillopacity{0.800000}%
\pgfsetlinewidth{0.000000pt}%
\definecolor{currentstroke}{rgb}{0.000000,0.000000,0.000000}%
\pgfsetstrokecolor{currentstroke}%
\pgfsetdash{}{0pt}%
\pgfpathmoveto{\pgfqpoint{2.537507in}{2.318291in}}%
\pgfpathlineto{\pgfqpoint{2.551394in}{2.299428in}}%
\pgfpathlineto{\pgfqpoint{2.565271in}{2.280862in}}%
\pgfpathlineto{\pgfqpoint{2.579138in}{2.262589in}}%
\pgfpathlineto{\pgfqpoint{2.592996in}{2.244608in}}%
\pgfpathlineto{\pgfqpoint{2.601673in}{2.246896in}}%
\pgfpathlineto{\pgfqpoint{2.610337in}{2.249404in}}%
\pgfpathlineto{\pgfqpoint{2.618986in}{2.252125in}}%
\pgfpathlineto{\pgfqpoint{2.627621in}{2.255056in}}%
\pgfpathlineto{\pgfqpoint{2.613803in}{2.272581in}}%
\pgfpathlineto{\pgfqpoint{2.599975in}{2.290397in}}%
\pgfpathlineto{\pgfqpoint{2.586139in}{2.308505in}}%
\pgfpathlineto{\pgfqpoint{2.572292in}{2.326908in}}%
\pgfpathlineto{\pgfqpoint{2.563618in}{2.324422in}}%
\pgfpathlineto{\pgfqpoint{2.554929in}{2.322154in}}%
\pgfpathlineto{\pgfqpoint{2.546226in}{2.320109in}}%
\pgfpathlineto{\pgfqpoint{2.537507in}{2.318291in}}%
\pgfpathclose%
\pgfusepath{fill}%
\end{pgfscope}%
\begin{pgfscope}%
\pgfpathrectangle{\pgfqpoint{1.150000in}{0.150000in}}{\pgfqpoint{5.700000in}{5.700000in}}%
\pgfusepath{clip}%
\pgfsetbuttcap%
\pgfsetroundjoin%
\definecolor{currentfill}{rgb}{0.216210,0.351535,0.550627}%
\pgfsetfillcolor{currentfill}%
\pgfsetfillopacity{0.800000}%
\pgfsetlinewidth{0.000000pt}%
\definecolor{currentstroke}{rgb}{0.000000,0.000000,0.000000}%
\pgfsetstrokecolor{currentstroke}%
\pgfsetdash{}{0pt}%
\pgfpathmoveto{\pgfqpoint{4.601249in}{2.522294in}}%
\pgfpathlineto{\pgfqpoint{4.615277in}{2.530130in}}%
\pgfpathlineto{\pgfqpoint{4.629319in}{2.538150in}}%
\pgfpathlineto{\pgfqpoint{4.643376in}{2.546354in}}%
\pgfpathlineto{\pgfqpoint{4.657448in}{2.554742in}}%
\pgfpathlineto{\pgfqpoint{4.665213in}{2.563588in}}%
\pgfpathlineto{\pgfqpoint{4.672971in}{2.572348in}}%
\pgfpathlineto{\pgfqpoint{4.680723in}{2.581025in}}%
\pgfpathlineto{\pgfqpoint{4.688469in}{2.589620in}}%
\pgfpathlineto{\pgfqpoint{4.674404in}{2.581375in}}%
\pgfpathlineto{\pgfqpoint{4.660355in}{2.573313in}}%
\pgfpathlineto{\pgfqpoint{4.646320in}{2.565436in}}%
\pgfpathlineto{\pgfqpoint{4.632300in}{2.557742in}}%
\pgfpathlineto{\pgfqpoint{4.624546in}{2.548992in}}%
\pgfpathlineto{\pgfqpoint{4.616787in}{2.540169in}}%
\pgfpathlineto{\pgfqpoint{4.609021in}{2.531270in}}%
\pgfpathlineto{\pgfqpoint{4.601249in}{2.522294in}}%
\pgfpathclose%
\pgfusepath{fill}%
\end{pgfscope}%
\begin{pgfscope}%
\pgfpathrectangle{\pgfqpoint{1.150000in}{0.150000in}}{\pgfqpoint{5.700000in}{5.700000in}}%
\pgfusepath{clip}%
\pgfsetbuttcap%
\pgfsetroundjoin%
\definecolor{currentfill}{rgb}{0.169646,0.456262,0.558030}%
\pgfsetfillcolor{currentfill}%
\pgfsetfillopacity{0.800000}%
\pgfsetlinewidth{0.000000pt}%
\definecolor{currentstroke}{rgb}{0.000000,0.000000,0.000000}%
\pgfsetstrokecolor{currentstroke}%
\pgfsetdash{}{0pt}%
\pgfpathmoveto{\pgfqpoint{4.981029in}{2.820382in}}%
\pgfpathlineto{\pgfqpoint{4.995259in}{2.830095in}}%
\pgfpathlineto{\pgfqpoint{5.009505in}{2.839989in}}%
\pgfpathlineto{\pgfqpoint{5.023767in}{2.850065in}}%
\pgfpathlineto{\pgfqpoint{5.038047in}{2.860322in}}%
\pgfpathlineto{\pgfqpoint{5.045636in}{2.866731in}}%
\pgfpathlineto{\pgfqpoint{5.053217in}{2.873077in}}%
\pgfpathlineto{\pgfqpoint{5.060791in}{2.879362in}}%
\pgfpathlineto{\pgfqpoint{5.068358in}{2.885592in}}%
\pgfpathlineto{\pgfqpoint{5.054092in}{2.875646in}}%
\pgfpathlineto{\pgfqpoint{5.039842in}{2.865882in}}%
\pgfpathlineto{\pgfqpoint{5.025610in}{2.856299in}}%
\pgfpathlineto{\pgfqpoint{5.011394in}{2.846896in}}%
\pgfpathlineto{\pgfqpoint{5.003813in}{2.840344in}}%
\pgfpathlineto{\pgfqpoint{4.996225in}{2.833743in}}%
\pgfpathlineto{\pgfqpoint{4.988631in}{2.827091in}}%
\pgfpathlineto{\pgfqpoint{4.981029in}{2.820382in}}%
\pgfpathclose%
\pgfusepath{fill}%
\end{pgfscope}%
\begin{pgfscope}%
\pgfpathrectangle{\pgfqpoint{1.150000in}{0.150000in}}{\pgfqpoint{5.700000in}{5.700000in}}%
\pgfusepath{clip}%
\pgfsetbuttcap%
\pgfsetroundjoin%
\definecolor{currentfill}{rgb}{0.271305,0.019942,0.347269}%
\pgfsetfillcolor{currentfill}%
\pgfsetfillopacity{0.800000}%
\pgfsetlinewidth{0.000000pt}%
\definecolor{currentstroke}{rgb}{0.000000,0.000000,0.000000}%
\pgfsetstrokecolor{currentstroke}%
\pgfsetdash{}{0pt}%
\pgfpathmoveto{\pgfqpoint{3.351342in}{1.784539in}}%
\pgfpathlineto{\pgfqpoint{3.364982in}{1.779745in}}%
\pgfpathlineto{\pgfqpoint{3.378625in}{1.775162in}}%
\pgfpathlineto{\pgfqpoint{3.392271in}{1.770788in}}%
\pgfpathlineto{\pgfqpoint{3.405921in}{1.766623in}}%
\pgfpathlineto{\pgfqpoint{3.414125in}{1.775901in}}%
\pgfpathlineto{\pgfqpoint{3.422321in}{1.785246in}}%
\pgfpathlineto{\pgfqpoint{3.430511in}{1.794654in}}%
\pgfpathlineto{\pgfqpoint{3.438695in}{1.804124in}}%
\pgfpathlineto{\pgfqpoint{3.425061in}{1.807949in}}%
\pgfpathlineto{\pgfqpoint{3.411431in}{1.811984in}}%
\pgfpathlineto{\pgfqpoint{3.397804in}{1.816227in}}%
\pgfpathlineto{\pgfqpoint{3.384180in}{1.820681in}}%
\pgfpathlineto{\pgfqpoint{3.375981in}{1.811539in}}%
\pgfpathlineto{\pgfqpoint{3.367775in}{1.802467in}}%
\pgfpathlineto{\pgfqpoint{3.359562in}{1.793465in}}%
\pgfpathlineto{\pgfqpoint{3.351342in}{1.784539in}}%
\pgfpathclose%
\pgfusepath{fill}%
\end{pgfscope}%
\begin{pgfscope}%
\pgfpathrectangle{\pgfqpoint{1.150000in}{0.150000in}}{\pgfqpoint{5.700000in}{5.700000in}}%
\pgfusepath{clip}%
\pgfsetbuttcap%
\pgfsetroundjoin%
\definecolor{currentfill}{rgb}{0.122606,0.585371,0.546557}%
\pgfsetfillcolor{currentfill}%
\pgfsetfillopacity{0.800000}%
\pgfsetlinewidth{0.000000pt}%
\definecolor{currentstroke}{rgb}{0.000000,0.000000,0.000000}%
\pgfsetstrokecolor{currentstroke}%
\pgfsetdash{}{0pt}%
\pgfpathmoveto{\pgfqpoint{5.534877in}{3.214607in}}%
\pgfpathlineto{\pgfqpoint{5.549408in}{3.225736in}}%
\pgfpathlineto{\pgfqpoint{5.563959in}{3.237043in}}%
\pgfpathlineto{\pgfqpoint{5.578529in}{3.248527in}}%
\pgfpathlineto{\pgfqpoint{5.593119in}{3.260189in}}%
\pgfpathlineto{\pgfqpoint{5.600397in}{3.263276in}}%
\pgfpathlineto{\pgfqpoint{5.607669in}{3.266385in}}%
\pgfpathlineto{\pgfqpoint{5.614936in}{3.269522in}}%
\pgfpathlineto{\pgfqpoint{5.622197in}{3.272695in}}%
\pgfpathlineto{\pgfqpoint{5.607634in}{3.261583in}}%
\pgfpathlineto{\pgfqpoint{5.593090in}{3.250648in}}%
\pgfpathlineto{\pgfqpoint{5.578566in}{3.239890in}}%
\pgfpathlineto{\pgfqpoint{5.564061in}{3.229309in}}%
\pgfpathlineto{\pgfqpoint{5.556773in}{3.225576in}}%
\pgfpathlineto{\pgfqpoint{5.549479in}{3.221886in}}%
\pgfpathlineto{\pgfqpoint{5.542181in}{3.218231in}}%
\pgfpathlineto{\pgfqpoint{5.534877in}{3.214607in}}%
\pgfpathclose%
\pgfusepath{fill}%
\end{pgfscope}%
\begin{pgfscope}%
\pgfpathrectangle{\pgfqpoint{1.150000in}{0.150000in}}{\pgfqpoint{5.700000in}{5.700000in}}%
\pgfusepath{clip}%
\pgfsetbuttcap%
\pgfsetroundjoin%
\definecolor{currentfill}{rgb}{0.276022,0.044167,0.370164}%
\pgfsetfillcolor{currentfill}%
\pgfsetfillopacity{0.800000}%
\pgfsetlinewidth{0.000000pt}%
\definecolor{currentstroke}{rgb}{0.000000,0.000000,0.000000}%
\pgfsetstrokecolor{currentstroke}%
\pgfsetdash{}{0pt}%
\pgfpathmoveto{\pgfqpoint{3.066646in}{1.834787in}}%
\pgfpathlineto{\pgfqpoint{3.080312in}{1.825707in}}%
\pgfpathlineto{\pgfqpoint{3.093976in}{1.816855in}}%
\pgfpathlineto{\pgfqpoint{3.107641in}{1.808230in}}%
\pgfpathlineto{\pgfqpoint{3.121305in}{1.799831in}}%
\pgfpathlineto{\pgfqpoint{3.129650in}{1.806787in}}%
\pgfpathlineto{\pgfqpoint{3.137987in}{1.813871in}}%
\pgfpathlineto{\pgfqpoint{3.146314in}{1.821077in}}%
\pgfpathlineto{\pgfqpoint{3.154633in}{1.828403in}}%
\pgfpathlineto{\pgfqpoint{3.140993in}{1.836397in}}%
\pgfpathlineto{\pgfqpoint{3.127353in}{1.844616in}}%
\pgfpathlineto{\pgfqpoint{3.113712in}{1.853062in}}%
\pgfpathlineto{\pgfqpoint{3.100071in}{1.861736in}}%
\pgfpathlineto{\pgfqpoint{3.091728in}{1.854804in}}%
\pgfpathlineto{\pgfqpoint{3.083377in}{1.847999in}}%
\pgfpathlineto{\pgfqpoint{3.075016in}{1.841325in}}%
\pgfpathlineto{\pgfqpoint{3.066646in}{1.834787in}}%
\pgfpathclose%
\pgfusepath{fill}%
\end{pgfscope}%
\begin{pgfscope}%
\pgfpathrectangle{\pgfqpoint{1.150000in}{0.150000in}}{\pgfqpoint{5.700000in}{5.700000in}}%
\pgfusepath{clip}%
\pgfsetbuttcap%
\pgfsetroundjoin%
\definecolor{currentfill}{rgb}{0.283197,0.115680,0.436115}%
\pgfsetfillcolor{currentfill}%
\pgfsetfillopacity{0.800000}%
\pgfsetlinewidth{0.000000pt}%
\definecolor{currentstroke}{rgb}{0.000000,0.000000,0.000000}%
\pgfsetstrokecolor{currentstroke}%
\pgfsetdash{}{0pt}%
\pgfpathmoveto{\pgfqpoint{2.813667in}{1.994614in}}%
\pgfpathlineto{\pgfqpoint{2.827406in}{1.981239in}}%
\pgfpathlineto{\pgfqpoint{2.841141in}{1.968117in}}%
\pgfpathlineto{\pgfqpoint{2.854872in}{1.955247in}}%
\pgfpathlineto{\pgfqpoint{2.868598in}{1.942626in}}%
\pgfpathlineto{\pgfqpoint{2.877096in}{1.947224in}}%
\pgfpathlineto{\pgfqpoint{2.885583in}{1.951999in}}%
\pgfpathlineto{\pgfqpoint{2.894058in}{1.956945in}}%
\pgfpathlineto{\pgfqpoint{2.902522in}{1.962058in}}%
\pgfpathlineto{\pgfqpoint{2.888827in}{1.974235in}}%
\pgfpathlineto{\pgfqpoint{2.875128in}{1.986662in}}%
\pgfpathlineto{\pgfqpoint{2.861425in}{1.999339in}}%
\pgfpathlineto{\pgfqpoint{2.847718in}{2.012268in}}%
\pgfpathlineto{\pgfqpoint{2.839223in}{2.007586in}}%
\pgfpathlineto{\pgfqpoint{2.830716in}{2.003081in}}%
\pgfpathlineto{\pgfqpoint{2.822198in}{1.998755in}}%
\pgfpathlineto{\pgfqpoint{2.813667in}{1.994614in}}%
\pgfpathclose%
\pgfusepath{fill}%
\end{pgfscope}%
\begin{pgfscope}%
\pgfpathrectangle{\pgfqpoint{1.150000in}{0.150000in}}{\pgfqpoint{5.700000in}{5.700000in}}%
\pgfusepath{clip}%
\pgfsetbuttcap%
\pgfsetroundjoin%
\definecolor{currentfill}{rgb}{0.248629,0.278775,0.534556}%
\pgfsetfillcolor{currentfill}%
\pgfsetfillopacity{0.800000}%
\pgfsetlinewidth{0.000000pt}%
\definecolor{currentstroke}{rgb}{0.000000,0.000000,0.000000}%
\pgfsetstrokecolor{currentstroke}%
\pgfsetdash{}{0pt}%
\pgfpathmoveto{\pgfqpoint{2.481853in}{2.396754in}}%
\pgfpathlineto{\pgfqpoint{2.495783in}{2.376681in}}%
\pgfpathlineto{\pgfqpoint{2.509702in}{2.356914in}}%
\pgfpathlineto{\pgfqpoint{2.523610in}{2.337452in}}%
\pgfpathlineto{\pgfqpoint{2.537507in}{2.318291in}}%
\pgfpathlineto{\pgfqpoint{2.546226in}{2.320109in}}%
\pgfpathlineto{\pgfqpoint{2.554929in}{2.322154in}}%
\pgfpathlineto{\pgfqpoint{2.563618in}{2.324422in}}%
\pgfpathlineto{\pgfqpoint{2.572292in}{2.326908in}}%
\pgfpathlineto{\pgfqpoint{2.558436in}{2.345609in}}%
\pgfpathlineto{\pgfqpoint{2.544570in}{2.364611in}}%
\pgfpathlineto{\pgfqpoint{2.530694in}{2.383916in}}%
\pgfpathlineto{\pgfqpoint{2.516806in}{2.403527in}}%
\pgfpathlineto{\pgfqpoint{2.508091in}{2.401489in}}%
\pgfpathlineto{\pgfqpoint{2.499361in}{2.399678in}}%
\pgfpathlineto{\pgfqpoint{2.490615in}{2.398098in}}%
\pgfpathlineto{\pgfqpoint{2.481853in}{2.396754in}}%
\pgfpathclose%
\pgfusepath{fill}%
\end{pgfscope}%
\begin{pgfscope}%
\pgfpathrectangle{\pgfqpoint{1.150000in}{0.150000in}}{\pgfqpoint{5.700000in}{5.700000in}}%
\pgfusepath{clip}%
\pgfsetbuttcap%
\pgfsetroundjoin%
\definecolor{currentfill}{rgb}{0.281924,0.089666,0.412415}%
\pgfsetfillcolor{currentfill}%
\pgfsetfillopacity{0.800000}%
\pgfsetlinewidth{0.000000pt}%
\definecolor{currentstroke}{rgb}{0.000000,0.000000,0.000000}%
\pgfsetstrokecolor{currentstroke}%
\pgfsetdash{}{0pt}%
\pgfpathmoveto{\pgfqpoint{3.754637in}{1.901265in}}%
\pgfpathlineto{\pgfqpoint{3.768331in}{1.901697in}}%
\pgfpathlineto{\pgfqpoint{3.782033in}{1.902326in}}%
\pgfpathlineto{\pgfqpoint{3.795742in}{1.903149in}}%
\pgfpathlineto{\pgfqpoint{3.809459in}{1.904168in}}%
\pgfpathlineto{\pgfqpoint{3.817513in}{1.915313in}}%
\pgfpathlineto{\pgfqpoint{3.825562in}{1.926446in}}%
\pgfpathlineto{\pgfqpoint{3.833606in}{1.937564in}}%
\pgfpathlineto{\pgfqpoint{3.841646in}{1.948666in}}%
\pgfpathlineto{\pgfqpoint{3.827936in}{1.947434in}}%
\pgfpathlineto{\pgfqpoint{3.814235in}{1.946396in}}%
\pgfpathlineto{\pgfqpoint{3.800542in}{1.945555in}}%
\pgfpathlineto{\pgfqpoint{3.786856in}{1.944909in}}%
\pgfpathlineto{\pgfqpoint{3.778809in}{1.934008in}}%
\pgfpathlineto{\pgfqpoint{3.770757in}{1.923100in}}%
\pgfpathlineto{\pgfqpoint{3.762700in}{1.912185in}}%
\pgfpathlineto{\pgfqpoint{3.754637in}{1.901265in}}%
\pgfpathclose%
\pgfusepath{fill}%
\end{pgfscope}%
\begin{pgfscope}%
\pgfpathrectangle{\pgfqpoint{1.150000in}{0.150000in}}{\pgfqpoint{5.700000in}{5.700000in}}%
\pgfusepath{clip}%
\pgfsetbuttcap%
\pgfsetroundjoin%
\definecolor{currentfill}{rgb}{0.279566,0.067836,0.391917}%
\pgfsetfillcolor{currentfill}%
\pgfsetfillopacity{0.800000}%
\pgfsetlinewidth{0.000000pt}%
\definecolor{currentstroke}{rgb}{0.000000,0.000000,0.000000}%
\pgfsetstrokecolor{currentstroke}%
\pgfsetdash{}{0pt}%
\pgfpathmoveto{\pgfqpoint{3.667593in}{1.858798in}}%
\pgfpathlineto{\pgfqpoint{3.681268in}{1.858196in}}%
\pgfpathlineto{\pgfqpoint{3.694950in}{1.857792in}}%
\pgfpathlineto{\pgfqpoint{3.708639in}{1.857586in}}%
\pgfpathlineto{\pgfqpoint{3.722335in}{1.857577in}}%
\pgfpathlineto{\pgfqpoint{3.730419in}{1.868496in}}%
\pgfpathlineto{\pgfqpoint{3.738497in}{1.879419in}}%
\pgfpathlineto{\pgfqpoint{3.746570in}{1.890342in}}%
\pgfpathlineto{\pgfqpoint{3.754637in}{1.901265in}}%
\pgfpathlineto{\pgfqpoint{3.740950in}{1.901029in}}%
\pgfpathlineto{\pgfqpoint{3.727271in}{1.900990in}}%
\pgfpathlineto{\pgfqpoint{3.713598in}{1.901149in}}%
\pgfpathlineto{\pgfqpoint{3.699932in}{1.901506in}}%
\pgfpathlineto{\pgfqpoint{3.691855in}{1.890816in}}%
\pgfpathlineto{\pgfqpoint{3.683773in}{1.880134in}}%
\pgfpathlineto{\pgfqpoint{3.675686in}{1.869460in}}%
\pgfpathlineto{\pgfqpoint{3.667593in}{1.858798in}}%
\pgfpathclose%
\pgfusepath{fill}%
\end{pgfscope}%
\begin{pgfscope}%
\pgfpathrectangle{\pgfqpoint{1.150000in}{0.150000in}}{\pgfqpoint{5.700000in}{5.700000in}}%
\pgfusepath{clip}%
\pgfsetbuttcap%
\pgfsetroundjoin%
\definecolor{currentfill}{rgb}{0.255645,0.260703,0.528312}%
\pgfsetfillcolor{currentfill}%
\pgfsetfillopacity{0.800000}%
\pgfsetlinewidth{0.000000pt}%
\definecolor{currentstroke}{rgb}{0.000000,0.000000,0.000000}%
\pgfsetstrokecolor{currentstroke}%
\pgfsetdash{}{0pt}%
\pgfpathmoveto{\pgfqpoint{4.308526in}{2.284493in}}%
\pgfpathlineto{\pgfqpoint{4.322417in}{2.290340in}}%
\pgfpathlineto{\pgfqpoint{4.336320in}{2.296374in}}%
\pgfpathlineto{\pgfqpoint{4.350236in}{2.302595in}}%
\pgfpathlineto{\pgfqpoint{4.364164in}{2.309002in}}%
\pgfpathlineto{\pgfqpoint{4.372042in}{2.319400in}}%
\pgfpathlineto{\pgfqpoint{4.379914in}{2.329719in}}%
\pgfpathlineto{\pgfqpoint{4.387781in}{2.339958in}}%
\pgfpathlineto{\pgfqpoint{4.395642in}{2.350119in}}%
\pgfpathlineto{\pgfqpoint{4.381719in}{2.343723in}}%
\pgfpathlineto{\pgfqpoint{4.367808in}{2.337513in}}%
\pgfpathlineto{\pgfqpoint{4.353910in}{2.331489in}}%
\pgfpathlineto{\pgfqpoint{4.340025in}{2.325653in}}%
\pgfpathlineto{\pgfqpoint{4.332159in}{2.315469in}}%
\pgfpathlineto{\pgfqpoint{4.324287in}{2.305215in}}%
\pgfpathlineto{\pgfqpoint{4.316409in}{2.294890in}}%
\pgfpathlineto{\pgfqpoint{4.308526in}{2.284493in}}%
\pgfpathclose%
\pgfusepath{fill}%
\end{pgfscope}%
\begin{pgfscope}%
\pgfpathrectangle{\pgfqpoint{1.150000in}{0.150000in}}{\pgfqpoint{5.700000in}{5.700000in}}%
\pgfusepath{clip}%
\pgfsetbuttcap%
\pgfsetroundjoin%
\definecolor{currentfill}{rgb}{0.283197,0.115680,0.436115}%
\pgfsetfillcolor{currentfill}%
\pgfsetfillopacity{0.800000}%
\pgfsetlinewidth{0.000000pt}%
\definecolor{currentstroke}{rgb}{0.000000,0.000000,0.000000}%
\pgfsetstrokecolor{currentstroke}%
\pgfsetdash{}{0pt}%
\pgfpathmoveto{\pgfqpoint{3.841646in}{1.948666in}}%
\pgfpathlineto{\pgfqpoint{3.855363in}{1.950093in}}%
\pgfpathlineto{\pgfqpoint{3.869089in}{1.951714in}}%
\pgfpathlineto{\pgfqpoint{3.882823in}{1.953528in}}%
\pgfpathlineto{\pgfqpoint{3.896566in}{1.955535in}}%
\pgfpathlineto{\pgfqpoint{3.904593in}{1.966813in}}%
\pgfpathlineto{\pgfqpoint{3.912615in}{1.978064in}}%
\pgfpathlineto{\pgfqpoint{3.920632in}{1.989287in}}%
\pgfpathlineto{\pgfqpoint{3.928644in}{2.000479in}}%
\pgfpathlineto{\pgfqpoint{3.914908in}{1.998289in}}%
\pgfpathlineto{\pgfqpoint{3.901180in}{1.996293in}}%
\pgfpathlineto{\pgfqpoint{3.887462in}{1.994490in}}%
\pgfpathlineto{\pgfqpoint{3.873752in}{1.992882in}}%
\pgfpathlineto{\pgfqpoint{3.865733in}{1.981860in}}%
\pgfpathlineto{\pgfqpoint{3.857709in}{1.970815in}}%
\pgfpathlineto{\pgfqpoint{3.849680in}{1.959750in}}%
\pgfpathlineto{\pgfqpoint{3.841646in}{1.948666in}}%
\pgfpathclose%
\pgfusepath{fill}%
\end{pgfscope}%
\begin{pgfscope}%
\pgfpathrectangle{\pgfqpoint{1.150000in}{0.150000in}}{\pgfqpoint{5.700000in}{5.700000in}}%
\pgfusepath{clip}%
\pgfsetbuttcap%
\pgfsetroundjoin%
\definecolor{currentfill}{rgb}{0.119738,0.603785,0.541400}%
\pgfsetfillcolor{currentfill}%
\pgfsetfillopacity{0.800000}%
\pgfsetlinewidth{0.000000pt}%
\definecolor{currentstroke}{rgb}{0.000000,0.000000,0.000000}%
\pgfsetstrokecolor{currentstroke}%
\pgfsetdash{}{0pt}%
\pgfpathmoveto{\pgfqpoint{5.622197in}{3.272695in}}%
\pgfpathlineto{\pgfqpoint{5.636779in}{3.283984in}}%
\pgfpathlineto{\pgfqpoint{5.651381in}{3.295449in}}%
\pgfpathlineto{\pgfqpoint{5.666003in}{3.307093in}}%
\pgfpathlineto{\pgfqpoint{5.680645in}{3.318913in}}%
\pgfpathlineto{\pgfqpoint{5.687872in}{3.321554in}}%
\pgfpathlineto{\pgfqpoint{5.695094in}{3.324235in}}%
\pgfpathlineto{\pgfqpoint{5.702310in}{3.326961in}}%
\pgfpathlineto{\pgfqpoint{5.709522in}{3.329739in}}%
\pgfpathlineto{\pgfqpoint{5.694909in}{3.318504in}}%
\pgfpathlineto{\pgfqpoint{5.680316in}{3.307444in}}%
\pgfpathlineto{\pgfqpoint{5.665743in}{3.296561in}}%
\pgfpathlineto{\pgfqpoint{5.651189in}{3.285854in}}%
\pgfpathlineto{\pgfqpoint{5.643948in}{3.282481in}}%
\pgfpathlineto{\pgfqpoint{5.636703in}{3.279168in}}%
\pgfpathlineto{\pgfqpoint{5.629452in}{3.275908in}}%
\pgfpathlineto{\pgfqpoint{5.622197in}{3.272695in}}%
\pgfpathclose%
\pgfusepath{fill}%
\end{pgfscope}%
\begin{pgfscope}%
\pgfpathrectangle{\pgfqpoint{1.150000in}{0.150000in}}{\pgfqpoint{5.700000in}{5.700000in}}%
\pgfusepath{clip}%
\pgfsetbuttcap%
\pgfsetroundjoin%
\definecolor{currentfill}{rgb}{0.277018,0.050344,0.375715}%
\pgfsetfillcolor{currentfill}%
\pgfsetfillopacity{0.800000}%
\pgfsetlinewidth{0.000000pt}%
\definecolor{currentstroke}{rgb}{0.000000,0.000000,0.000000}%
\pgfsetstrokecolor{currentstroke}%
\pgfsetdash{}{0pt}%
\pgfpathmoveto{\pgfqpoint{3.580482in}{1.821815in}}%
\pgfpathlineto{\pgfqpoint{3.594144in}{1.820137in}}%
\pgfpathlineto{\pgfqpoint{3.607812in}{1.818661in}}%
\pgfpathlineto{\pgfqpoint{3.621486in}{1.817384in}}%
\pgfpathlineto{\pgfqpoint{3.635166in}{1.816307in}}%
\pgfpathlineto{\pgfqpoint{3.643281in}{1.826901in}}%
\pgfpathlineto{\pgfqpoint{3.651390in}{1.837516in}}%
\pgfpathlineto{\pgfqpoint{3.659494in}{1.848149in}}%
\pgfpathlineto{\pgfqpoint{3.667593in}{1.858798in}}%
\pgfpathlineto{\pgfqpoint{3.653923in}{1.859599in}}%
\pgfpathlineto{\pgfqpoint{3.640260in}{1.860599in}}%
\pgfpathlineto{\pgfqpoint{3.626604in}{1.861800in}}%
\pgfpathlineto{\pgfqpoint{3.612953in}{1.863201in}}%
\pgfpathlineto{\pgfqpoint{3.604843in}{1.852817in}}%
\pgfpathlineto{\pgfqpoint{3.596729in}{1.842456in}}%
\pgfpathlineto{\pgfqpoint{3.588608in}{1.832121in}}%
\pgfpathlineto{\pgfqpoint{3.580482in}{1.821815in}}%
\pgfpathclose%
\pgfusepath{fill}%
\end{pgfscope}%
\begin{pgfscope}%
\pgfpathrectangle{\pgfqpoint{1.150000in}{0.150000in}}{\pgfqpoint{5.700000in}{5.700000in}}%
\pgfusepath{clip}%
\pgfsetbuttcap%
\pgfsetroundjoin%
\definecolor{currentfill}{rgb}{0.160665,0.478540,0.558115}%
\pgfsetfillcolor{currentfill}%
\pgfsetfillopacity{0.800000}%
\pgfsetlinewidth{0.000000pt}%
\definecolor{currentstroke}{rgb}{0.000000,0.000000,0.000000}%
\pgfsetstrokecolor{currentstroke}%
\pgfsetdash{}{0pt}%
\pgfpathmoveto{\pgfqpoint{5.068358in}{2.885592in}}%
\pgfpathlineto{\pgfqpoint{5.082642in}{2.895718in}}%
\pgfpathlineto{\pgfqpoint{5.096943in}{2.906025in}}%
\pgfpathlineto{\pgfqpoint{5.111261in}{2.916513in}}%
\pgfpathlineto{\pgfqpoint{5.125597in}{2.927182in}}%
\pgfpathlineto{\pgfqpoint{5.133142in}{2.933025in}}%
\pgfpathlineto{\pgfqpoint{5.140681in}{2.938812in}}%
\pgfpathlineto{\pgfqpoint{5.148212in}{2.944546in}}%
\pgfpathlineto{\pgfqpoint{5.155736in}{2.950231in}}%
\pgfpathlineto{\pgfqpoint{5.141415in}{2.939908in}}%
\pgfpathlineto{\pgfqpoint{5.127112in}{2.929766in}}%
\pgfpathlineto{\pgfqpoint{5.112826in}{2.919804in}}%
\pgfpathlineto{\pgfqpoint{5.098558in}{2.910022in}}%
\pgfpathlineto{\pgfqpoint{5.091018in}{2.903980in}}%
\pgfpathlineto{\pgfqpoint{5.083472in}{2.897897in}}%
\pgfpathlineto{\pgfqpoint{5.075918in}{2.891768in}}%
\pgfpathlineto{\pgfqpoint{5.068358in}{2.885592in}}%
\pgfpathclose%
\pgfusepath{fill}%
\end{pgfscope}%
\begin{pgfscope}%
\pgfpathrectangle{\pgfqpoint{1.150000in}{0.150000in}}{\pgfqpoint{5.700000in}{5.700000in}}%
\pgfusepath{clip}%
\pgfsetbuttcap%
\pgfsetroundjoin%
\definecolor{currentfill}{rgb}{0.282623,0.140926,0.457517}%
\pgfsetfillcolor{currentfill}%
\pgfsetfillopacity{0.800000}%
\pgfsetlinewidth{0.000000pt}%
\definecolor{currentstroke}{rgb}{0.000000,0.000000,0.000000}%
\pgfsetstrokecolor{currentstroke}%
\pgfsetdash{}{0pt}%
\pgfpathmoveto{\pgfqpoint{3.928644in}{2.000479in}}%
\pgfpathlineto{\pgfqpoint{3.942389in}{2.002861in}}%
\pgfpathlineto{\pgfqpoint{3.956144in}{2.005435in}}%
\pgfpathlineto{\pgfqpoint{3.969907in}{2.008201in}}%
\pgfpathlineto{\pgfqpoint{3.983681in}{2.011158in}}%
\pgfpathlineto{\pgfqpoint{3.991682in}{2.022481in}}%
\pgfpathlineto{\pgfqpoint{3.999678in}{2.033764in}}%
\pgfpathlineto{\pgfqpoint{4.007669in}{2.045004in}}%
\pgfpathlineto{\pgfqpoint{4.015655in}{2.056203in}}%
\pgfpathlineto{\pgfqpoint{4.001888in}{2.053095in}}%
\pgfpathlineto{\pgfqpoint{3.988130in}{2.050178in}}%
\pgfpathlineto{\pgfqpoint{3.974382in}{2.047454in}}%
\pgfpathlineto{\pgfqpoint{3.960643in}{2.044922in}}%
\pgfpathlineto{\pgfqpoint{3.952650in}{2.033862in}}%
\pgfpathlineto{\pgfqpoint{3.944653in}{2.022768in}}%
\pgfpathlineto{\pgfqpoint{3.936651in}{2.011639in}}%
\pgfpathlineto{\pgfqpoint{3.928644in}{2.000479in}}%
\pgfpathclose%
\pgfusepath{fill}%
\end{pgfscope}%
\begin{pgfscope}%
\pgfpathrectangle{\pgfqpoint{1.150000in}{0.150000in}}{\pgfqpoint{5.700000in}{5.700000in}}%
\pgfusepath{clip}%
\pgfsetbuttcap%
\pgfsetroundjoin%
\definecolor{currentfill}{rgb}{0.282327,0.094955,0.417331}%
\pgfsetfillcolor{currentfill}%
\pgfsetfillopacity{0.800000}%
\pgfsetlinewidth{0.000000pt}%
\definecolor{currentstroke}{rgb}{0.000000,0.000000,0.000000}%
\pgfsetstrokecolor{currentstroke}%
\pgfsetdash{}{0pt}%
\pgfpathmoveto{\pgfqpoint{2.868598in}{1.942626in}}%
\pgfpathlineto{\pgfqpoint{2.882320in}{1.930254in}}%
\pgfpathlineto{\pgfqpoint{2.896038in}{1.918127in}}%
\pgfpathlineto{\pgfqpoint{2.909753in}{1.906245in}}%
\pgfpathlineto{\pgfqpoint{2.923465in}{1.894607in}}%
\pgfpathlineto{\pgfqpoint{2.931932in}{1.899660in}}%
\pgfpathlineto{\pgfqpoint{2.940388in}{1.904881in}}%
\pgfpathlineto{\pgfqpoint{2.948834in}{1.910265in}}%
\pgfpathlineto{\pgfqpoint{2.957268in}{1.915809in}}%
\pgfpathlineto{\pgfqpoint{2.943586in}{1.927006in}}%
\pgfpathlineto{\pgfqpoint{2.929901in}{1.938445in}}%
\pgfpathlineto{\pgfqpoint{2.916213in}{1.950129in}}%
\pgfpathlineto{\pgfqpoint{2.902522in}{1.962058in}}%
\pgfpathlineto{\pgfqpoint{2.894058in}{1.956945in}}%
\pgfpathlineto{\pgfqpoint{2.885583in}{1.951999in}}%
\pgfpathlineto{\pgfqpoint{2.877096in}{1.947224in}}%
\pgfpathlineto{\pgfqpoint{2.868598in}{1.942626in}}%
\pgfpathclose%
\pgfusepath{fill}%
\end{pgfscope}%
\begin{pgfscope}%
\pgfpathrectangle{\pgfqpoint{1.150000in}{0.150000in}}{\pgfqpoint{5.700000in}{5.700000in}}%
\pgfusepath{clip}%
\pgfsetbuttcap%
\pgfsetroundjoin%
\definecolor{currentfill}{rgb}{0.203063,0.379716,0.553925}%
\pgfsetfillcolor{currentfill}%
\pgfsetfillopacity{0.800000}%
\pgfsetlinewidth{0.000000pt}%
\definecolor{currentstroke}{rgb}{0.000000,0.000000,0.000000}%
\pgfsetstrokecolor{currentstroke}%
\pgfsetdash{}{0pt}%
\pgfpathmoveto{\pgfqpoint{4.688469in}{2.589620in}}%
\pgfpathlineto{\pgfqpoint{4.702548in}{2.598049in}}%
\pgfpathlineto{\pgfqpoint{4.716642in}{2.606661in}}%
\pgfpathlineto{\pgfqpoint{4.730752in}{2.615458in}}%
\pgfpathlineto{\pgfqpoint{4.744877in}{2.624437in}}%
\pgfpathlineto{\pgfqpoint{4.752608in}{2.632789in}}%
\pgfpathlineto{\pgfqpoint{4.760332in}{2.641055in}}%
\pgfpathlineto{\pgfqpoint{4.768050in}{2.649238in}}%
\pgfpathlineto{\pgfqpoint{4.775761in}{2.657340in}}%
\pgfpathlineto{\pgfqpoint{4.761645in}{2.648537in}}%
\pgfpathlineto{\pgfqpoint{4.747544in}{2.639918in}}%
\pgfpathlineto{\pgfqpoint{4.733458in}{2.631481in}}%
\pgfpathlineto{\pgfqpoint{4.719387in}{2.623228in}}%
\pgfpathlineto{\pgfqpoint{4.711667in}{2.614938in}}%
\pgfpathlineto{\pgfqpoint{4.703941in}{2.606574in}}%
\pgfpathlineto{\pgfqpoint{4.696208in}{2.598136in}}%
\pgfpathlineto{\pgfqpoint{4.688469in}{2.589620in}}%
\pgfpathclose%
\pgfusepath{fill}%
\end{pgfscope}%
\begin{pgfscope}%
\pgfpathrectangle{\pgfqpoint{1.150000in}{0.150000in}}{\pgfqpoint{5.700000in}{5.700000in}}%
\pgfusepath{clip}%
\pgfsetbuttcap%
\pgfsetroundjoin%
\definecolor{currentfill}{rgb}{0.120081,0.622161,0.534946}%
\pgfsetfillcolor{currentfill}%
\pgfsetfillopacity{0.800000}%
\pgfsetlinewidth{0.000000pt}%
\definecolor{currentstroke}{rgb}{0.000000,0.000000,0.000000}%
\pgfsetstrokecolor{currentstroke}%
\pgfsetdash{}{0pt}%
\pgfpathmoveto{\pgfqpoint{5.709522in}{3.329739in}}%
\pgfpathlineto{\pgfqpoint{5.724154in}{3.341152in}}%
\pgfpathlineto{\pgfqpoint{5.738807in}{3.352741in}}%
\pgfpathlineto{\pgfqpoint{5.753480in}{3.364506in}}%
\pgfpathlineto{\pgfqpoint{5.768173in}{3.376449in}}%
\pgfpathlineto{\pgfqpoint{5.775348in}{3.378680in}}%
\pgfpathlineto{\pgfqpoint{5.782519in}{3.380968in}}%
\pgfpathlineto{\pgfqpoint{5.789685in}{3.383321in}}%
\pgfpathlineto{\pgfqpoint{5.796847in}{3.385745in}}%
\pgfpathlineto{\pgfqpoint{5.782185in}{3.374422in}}%
\pgfpathlineto{\pgfqpoint{5.767544in}{3.363274in}}%
\pgfpathlineto{\pgfqpoint{5.752923in}{3.352302in}}%
\pgfpathlineto{\pgfqpoint{5.738321in}{3.341506in}}%
\pgfpathlineto{\pgfqpoint{5.731128in}{3.338453in}}%
\pgfpathlineto{\pgfqpoint{5.723930in}{3.335479in}}%
\pgfpathlineto{\pgfqpoint{5.716728in}{3.332577in}}%
\pgfpathlineto{\pgfqpoint{5.709522in}{3.329739in}}%
\pgfpathclose%
\pgfusepath{fill}%
\end{pgfscope}%
\begin{pgfscope}%
\pgfpathrectangle{\pgfqpoint{1.150000in}{0.150000in}}{\pgfqpoint{5.700000in}{5.700000in}}%
\pgfusepath{clip}%
\pgfsetbuttcap%
\pgfsetroundjoin%
\definecolor{currentfill}{rgb}{0.233603,0.313828,0.543914}%
\pgfsetfillcolor{currentfill}%
\pgfsetfillopacity{0.800000}%
\pgfsetlinewidth{0.000000pt}%
\definecolor{currentstroke}{rgb}{0.000000,0.000000,0.000000}%
\pgfsetstrokecolor{currentstroke}%
\pgfsetdash{}{0pt}%
\pgfpathmoveto{\pgfqpoint{2.426015in}{2.480171in}}%
\pgfpathlineto{\pgfqpoint{2.439993in}{2.458843in}}%
\pgfpathlineto{\pgfqpoint{2.453958in}{2.437832in}}%
\pgfpathlineto{\pgfqpoint{2.467912in}{2.417137in}}%
\pgfpathlineto{\pgfqpoint{2.481853in}{2.396754in}}%
\pgfpathlineto{\pgfqpoint{2.490615in}{2.398098in}}%
\pgfpathlineto{\pgfqpoint{2.499361in}{2.399678in}}%
\pgfpathlineto{\pgfqpoint{2.508091in}{2.401489in}}%
\pgfpathlineto{\pgfqpoint{2.516806in}{2.403527in}}%
\pgfpathlineto{\pgfqpoint{2.502908in}{2.423446in}}%
\pgfpathlineto{\pgfqpoint{2.488998in}{2.443677in}}%
\pgfpathlineto{\pgfqpoint{2.475076in}{2.464221in}}%
\pgfpathlineto{\pgfqpoint{2.461143in}{2.485084in}}%
\pgfpathlineto{\pgfqpoint{2.452385in}{2.483497in}}%
\pgfpathlineto{\pgfqpoint{2.443611in}{2.482147in}}%
\pgfpathlineto{\pgfqpoint{2.434821in}{2.481037in}}%
\pgfpathlineto{\pgfqpoint{2.426015in}{2.480171in}}%
\pgfpathclose%
\pgfusepath{fill}%
\end{pgfscope}%
\begin{pgfscope}%
\pgfpathrectangle{\pgfqpoint{1.150000in}{0.150000in}}{\pgfqpoint{5.700000in}{5.700000in}}%
\pgfusepath{clip}%
\pgfsetbuttcap%
\pgfsetroundjoin%
\definecolor{currentfill}{rgb}{0.280255,0.165693,0.476498}%
\pgfsetfillcolor{currentfill}%
\pgfsetfillopacity{0.800000}%
\pgfsetlinewidth{0.000000pt}%
\definecolor{currentstroke}{rgb}{0.000000,0.000000,0.000000}%
\pgfsetstrokecolor{currentstroke}%
\pgfsetdash{}{0pt}%
\pgfpathmoveto{\pgfqpoint{4.015655in}{2.056203in}}%
\pgfpathlineto{\pgfqpoint{4.029432in}{2.059501in}}%
\pgfpathlineto{\pgfqpoint{4.043220in}{2.062991in}}%
\pgfpathlineto{\pgfqpoint{4.057017in}{2.066670in}}%
\pgfpathlineto{\pgfqpoint{4.070825in}{2.070540in}}%
\pgfpathlineto{\pgfqpoint{4.078801in}{2.081825in}}%
\pgfpathlineto{\pgfqpoint{4.086771in}{2.093057in}}%
\pgfpathlineto{\pgfqpoint{4.094737in}{2.104236in}}%
\pgfpathlineto{\pgfqpoint{4.102698in}{2.115361in}}%
\pgfpathlineto{\pgfqpoint{4.088895in}{2.111372in}}%
\pgfpathlineto{\pgfqpoint{4.075103in}{2.107574in}}%
\pgfpathlineto{\pgfqpoint{4.061321in}{2.103966in}}%
\pgfpathlineto{\pgfqpoint{4.047550in}{2.100549in}}%
\pgfpathlineto{\pgfqpoint{4.039584in}{2.089531in}}%
\pgfpathlineto{\pgfqpoint{4.031612in}{2.078466in}}%
\pgfpathlineto{\pgfqpoint{4.023636in}{2.067357in}}%
\pgfpathlineto{\pgfqpoint{4.015655in}{2.056203in}}%
\pgfpathclose%
\pgfusepath{fill}%
\end{pgfscope}%
\begin{pgfscope}%
\pgfpathrectangle{\pgfqpoint{1.150000in}{0.150000in}}{\pgfqpoint{5.700000in}{5.700000in}}%
\pgfusepath{clip}%
\pgfsetbuttcap%
\pgfsetroundjoin%
\definecolor{currentfill}{rgb}{0.273809,0.031497,0.358853}%
\pgfsetfillcolor{currentfill}%
\pgfsetfillopacity{0.800000}%
\pgfsetlinewidth{0.000000pt}%
\definecolor{currentstroke}{rgb}{0.000000,0.000000,0.000000}%
\pgfsetstrokecolor{currentstroke}%
\pgfsetdash{}{0pt}%
\pgfpathmoveto{\pgfqpoint{3.493271in}{1.790889in}}%
\pgfpathlineto{\pgfqpoint{3.506926in}{1.788094in}}%
\pgfpathlineto{\pgfqpoint{3.520585in}{1.785502in}}%
\pgfpathlineto{\pgfqpoint{3.534250in}{1.783113in}}%
\pgfpathlineto{\pgfqpoint{3.547919in}{1.780926in}}%
\pgfpathlineto{\pgfqpoint{3.556069in}{1.791092in}}%
\pgfpathlineto{\pgfqpoint{3.564212in}{1.801298in}}%
\pgfpathlineto{\pgfqpoint{3.572350in}{1.811540in}}%
\pgfpathlineto{\pgfqpoint{3.580482in}{1.821815in}}%
\pgfpathlineto{\pgfqpoint{3.566825in}{1.823694in}}%
\pgfpathlineto{\pgfqpoint{3.553173in}{1.825775in}}%
\pgfpathlineto{\pgfqpoint{3.539527in}{1.828059in}}%
\pgfpathlineto{\pgfqpoint{3.525885in}{1.830547in}}%
\pgfpathlineto{\pgfqpoint{3.517741in}{1.820568in}}%
\pgfpathlineto{\pgfqpoint{3.509590in}{1.810630in}}%
\pgfpathlineto{\pgfqpoint{3.501434in}{1.800736in}}%
\pgfpathlineto{\pgfqpoint{3.493271in}{1.790889in}}%
\pgfpathclose%
\pgfusepath{fill}%
\end{pgfscope}%
\begin{pgfscope}%
\pgfpathrectangle{\pgfqpoint{1.150000in}{0.150000in}}{\pgfqpoint{5.700000in}{5.700000in}}%
\pgfusepath{clip}%
\pgfsetbuttcap%
\pgfsetroundjoin%
\definecolor{currentfill}{rgb}{0.124780,0.640461,0.527068}%
\pgfsetfillcolor{currentfill}%
\pgfsetfillopacity{0.800000}%
\pgfsetlinewidth{0.000000pt}%
\definecolor{currentstroke}{rgb}{0.000000,0.000000,0.000000}%
\pgfsetstrokecolor{currentstroke}%
\pgfsetdash{}{0pt}%
\pgfpathmoveto{\pgfqpoint{5.796847in}{3.385745in}}%
\pgfpathlineto{\pgfqpoint{5.811528in}{3.397245in}}%
\pgfpathlineto{\pgfqpoint{5.826230in}{3.408920in}}%
\pgfpathlineto{\pgfqpoint{5.840952in}{3.420772in}}%
\pgfpathlineto{\pgfqpoint{5.855695in}{3.432801in}}%
\pgfpathlineto{\pgfqpoint{5.862819in}{3.434662in}}%
\pgfpathlineto{\pgfqpoint{5.869939in}{3.436601in}}%
\pgfpathlineto{\pgfqpoint{5.877055in}{3.438625in}}%
\pgfpathlineto{\pgfqpoint{5.884167in}{3.440740in}}%
\pgfpathlineto{\pgfqpoint{5.869459in}{3.429365in}}%
\pgfpathlineto{\pgfqpoint{5.854771in}{3.418165in}}%
\pgfpathlineto{\pgfqpoint{5.840103in}{3.407141in}}%
\pgfpathlineto{\pgfqpoint{5.825455in}{3.396291in}}%
\pgfpathlineto{\pgfqpoint{5.818308in}{3.393513in}}%
\pgfpathlineto{\pgfqpoint{5.811158in}{3.390834in}}%
\pgfpathlineto{\pgfqpoint{5.804004in}{3.388247in}}%
\pgfpathlineto{\pgfqpoint{5.796847in}{3.385745in}}%
\pgfpathclose%
\pgfusepath{fill}%
\end{pgfscope}%
\begin{pgfscope}%
\pgfpathrectangle{\pgfqpoint{1.150000in}{0.150000in}}{\pgfqpoint{5.700000in}{5.700000in}}%
\pgfusepath{clip}%
\pgfsetbuttcap%
\pgfsetroundjoin%
\definecolor{currentfill}{rgb}{0.243113,0.292092,0.538516}%
\pgfsetfillcolor{currentfill}%
\pgfsetfillopacity{0.800000}%
\pgfsetlinewidth{0.000000pt}%
\definecolor{currentstroke}{rgb}{0.000000,0.000000,0.000000}%
\pgfsetstrokecolor{currentstroke}%
\pgfsetdash{}{0pt}%
\pgfpathmoveto{\pgfqpoint{4.395642in}{2.350119in}}%
\pgfpathlineto{\pgfqpoint{4.409578in}{2.356702in}}%
\pgfpathlineto{\pgfqpoint{4.423527in}{2.363471in}}%
\pgfpathlineto{\pgfqpoint{4.437490in}{2.370425in}}%
\pgfpathlineto{\pgfqpoint{4.451466in}{2.377566in}}%
\pgfpathlineto{\pgfqpoint{4.459316in}{2.387618in}}%
\pgfpathlineto{\pgfqpoint{4.467160in}{2.397584in}}%
\pgfpathlineto{\pgfqpoint{4.474999in}{2.407466in}}%
\pgfpathlineto{\pgfqpoint{4.482832in}{2.417265in}}%
\pgfpathlineto{\pgfqpoint{4.468861in}{2.410168in}}%
\pgfpathlineto{\pgfqpoint{4.454904in}{2.403257in}}%
\pgfpathlineto{\pgfqpoint{4.440960in}{2.396532in}}%
\pgfpathlineto{\pgfqpoint{4.427029in}{2.389992in}}%
\pgfpathlineto{\pgfqpoint{4.419191in}{2.380138in}}%
\pgfpathlineto{\pgfqpoint{4.411347in}{2.370208in}}%
\pgfpathlineto{\pgfqpoint{4.403497in}{2.360202in}}%
\pgfpathlineto{\pgfqpoint{4.395642in}{2.350119in}}%
\pgfpathclose%
\pgfusepath{fill}%
\end{pgfscope}%
\begin{pgfscope}%
\pgfpathrectangle{\pgfqpoint{1.150000in}{0.150000in}}{\pgfqpoint{5.700000in}{5.700000in}}%
\pgfusepath{clip}%
\pgfsetbuttcap%
\pgfsetroundjoin%
\definecolor{currentfill}{rgb}{0.271305,0.019942,0.347269}%
\pgfsetfillcolor{currentfill}%
\pgfsetfillopacity{0.800000}%
\pgfsetlinewidth{0.000000pt}%
\definecolor{currentstroke}{rgb}{0.000000,0.000000,0.000000}%
\pgfsetstrokecolor{currentstroke}%
\pgfsetdash{}{0pt}%
\pgfpathmoveto{\pgfqpoint{3.263780in}{1.772430in}}%
\pgfpathlineto{\pgfqpoint{3.277429in}{1.766414in}}%
\pgfpathlineto{\pgfqpoint{3.291081in}{1.760612in}}%
\pgfpathlineto{\pgfqpoint{3.304734in}{1.755023in}}%
\pgfpathlineto{\pgfqpoint{3.318389in}{1.749647in}}%
\pgfpathlineto{\pgfqpoint{3.326638in}{1.758241in}}%
\pgfpathlineto{\pgfqpoint{3.334880in}{1.766923in}}%
\pgfpathlineto{\pgfqpoint{3.343114in}{1.775690in}}%
\pgfpathlineto{\pgfqpoint{3.351342in}{1.784539in}}%
\pgfpathlineto{\pgfqpoint{3.337704in}{1.789544in}}%
\pgfpathlineto{\pgfqpoint{3.324069in}{1.794761in}}%
\pgfpathlineto{\pgfqpoint{3.310437in}{1.800191in}}%
\pgfpathlineto{\pgfqpoint{3.296807in}{1.805836in}}%
\pgfpathlineto{\pgfqpoint{3.288562in}{1.797347in}}%
\pgfpathlineto{\pgfqpoint{3.280309in}{1.788948in}}%
\pgfpathlineto{\pgfqpoint{3.272048in}{1.780641in}}%
\pgfpathlineto{\pgfqpoint{3.263780in}{1.772430in}}%
\pgfpathclose%
\pgfusepath{fill}%
\end{pgfscope}%
\begin{pgfscope}%
\pgfpathrectangle{\pgfqpoint{1.150000in}{0.150000in}}{\pgfqpoint{5.700000in}{5.700000in}}%
\pgfusepath{clip}%
\pgfsetbuttcap%
\pgfsetroundjoin%
\definecolor{currentfill}{rgb}{0.273809,0.031497,0.358853}%
\pgfsetfillcolor{currentfill}%
\pgfsetfillopacity{0.800000}%
\pgfsetlinewidth{0.000000pt}%
\definecolor{currentstroke}{rgb}{0.000000,0.000000,0.000000}%
\pgfsetstrokecolor{currentstroke}%
\pgfsetdash{}{0pt}%
\pgfpathmoveto{\pgfqpoint{3.121305in}{1.799831in}}%
\pgfpathlineto{\pgfqpoint{3.134969in}{1.791657in}}%
\pgfpathlineto{\pgfqpoint{3.148633in}{1.783705in}}%
\pgfpathlineto{\pgfqpoint{3.162297in}{1.775976in}}%
\pgfpathlineto{\pgfqpoint{3.175961in}{1.768468in}}%
\pgfpathlineto{\pgfqpoint{3.184284in}{1.775841in}}%
\pgfpathlineto{\pgfqpoint{3.192597in}{1.783334in}}%
\pgfpathlineto{\pgfqpoint{3.200902in}{1.790941in}}%
\pgfpathlineto{\pgfqpoint{3.209199in}{1.798660in}}%
\pgfpathlineto{\pgfqpoint{3.195557in}{1.805763in}}%
\pgfpathlineto{\pgfqpoint{3.181915in}{1.813088in}}%
\pgfpathlineto{\pgfqpoint{3.168274in}{1.820634in}}%
\pgfpathlineto{\pgfqpoint{3.154633in}{1.828403in}}%
\pgfpathlineto{\pgfqpoint{3.146314in}{1.821077in}}%
\pgfpathlineto{\pgfqpoint{3.137987in}{1.813871in}}%
\pgfpathlineto{\pgfqpoint{3.129650in}{1.806787in}}%
\pgfpathlineto{\pgfqpoint{3.121305in}{1.799831in}}%
\pgfpathclose%
\pgfusepath{fill}%
\end{pgfscope}%
\begin{pgfscope}%
\pgfpathrectangle{\pgfqpoint{1.150000in}{0.150000in}}{\pgfqpoint{5.700000in}{5.700000in}}%
\pgfusepath{clip}%
\pgfsetbuttcap%
\pgfsetroundjoin%
\definecolor{currentfill}{rgb}{0.151918,0.500685,0.557587}%
\pgfsetfillcolor{currentfill}%
\pgfsetfillopacity{0.800000}%
\pgfsetlinewidth{0.000000pt}%
\definecolor{currentstroke}{rgb}{0.000000,0.000000,0.000000}%
\pgfsetstrokecolor{currentstroke}%
\pgfsetdash{}{0pt}%
\pgfpathmoveto{\pgfqpoint{5.155736in}{2.950231in}}%
\pgfpathlineto{\pgfqpoint{5.170074in}{2.960734in}}%
\pgfpathlineto{\pgfqpoint{5.184430in}{2.971417in}}%
\pgfpathlineto{\pgfqpoint{5.198804in}{2.982282in}}%
\pgfpathlineto{\pgfqpoint{5.213196in}{2.993326in}}%
\pgfpathlineto{\pgfqpoint{5.220697in}{2.998599in}}%
\pgfpathlineto{\pgfqpoint{5.228190in}{3.003823in}}%
\pgfpathlineto{\pgfqpoint{5.235677in}{3.009003in}}%
\pgfpathlineto{\pgfqpoint{5.243156in}{3.014144in}}%
\pgfpathlineto{\pgfqpoint{5.228781in}{3.003480in}}%
\pgfpathlineto{\pgfqpoint{5.214424in}{2.992996in}}%
\pgfpathlineto{\pgfqpoint{5.200085in}{2.982692in}}%
\pgfpathlineto{\pgfqpoint{5.185763in}{2.972568in}}%
\pgfpathlineto{\pgfqpoint{5.178267in}{2.967036in}}%
\pgfpathlineto{\pgfqpoint{5.170763in}{2.961472in}}%
\pgfpathlineto{\pgfqpoint{5.163253in}{2.955872in}}%
\pgfpathlineto{\pgfqpoint{5.155736in}{2.950231in}}%
\pgfpathclose%
\pgfusepath{fill}%
\end{pgfscope}%
\begin{pgfscope}%
\pgfpathrectangle{\pgfqpoint{1.150000in}{0.150000in}}{\pgfqpoint{5.700000in}{5.700000in}}%
\pgfusepath{clip}%
\pgfsetbuttcap%
\pgfsetroundjoin%
\definecolor{currentfill}{rgb}{0.137339,0.662252,0.515571}%
\pgfsetfillcolor{currentfill}%
\pgfsetfillopacity{0.800000}%
\pgfsetlinewidth{0.000000pt}%
\definecolor{currentstroke}{rgb}{0.000000,0.000000,0.000000}%
\pgfsetstrokecolor{currentstroke}%
\pgfsetdash{}{0pt}%
\pgfpathmoveto{\pgfqpoint{5.884167in}{3.440740in}}%
\pgfpathlineto{\pgfqpoint{5.898896in}{3.452291in}}%
\pgfpathlineto{\pgfqpoint{5.913646in}{3.464017in}}%
\pgfpathlineto{\pgfqpoint{5.928417in}{3.475919in}}%
\pgfpathlineto{\pgfqpoint{5.943208in}{3.487996in}}%
\pgfpathlineto{\pgfqpoint{5.950281in}{3.489536in}}%
\pgfpathlineto{\pgfqpoint{5.957351in}{3.491173in}}%
\pgfpathlineto{\pgfqpoint{5.964417in}{3.492917in}}%
\pgfpathlineto{\pgfqpoint{5.971481in}{3.494775in}}%
\pgfpathlineto{\pgfqpoint{5.956726in}{3.483385in}}%
\pgfpathlineto{\pgfqpoint{5.941993in}{3.472169in}}%
\pgfpathlineto{\pgfqpoint{5.927279in}{3.461128in}}%
\pgfpathlineto{\pgfqpoint{5.912586in}{3.450261in}}%
\pgfpathlineto{\pgfqpoint{5.905485in}{3.447707in}}%
\pgfpathlineto{\pgfqpoint{5.898382in}{3.445274in}}%
\pgfpathlineto{\pgfqpoint{5.891276in}{3.442954in}}%
\pgfpathlineto{\pgfqpoint{5.884167in}{3.440740in}}%
\pgfpathclose%
\pgfusepath{fill}%
\end{pgfscope}%
\begin{pgfscope}%
\pgfpathrectangle{\pgfqpoint{1.150000in}{0.150000in}}{\pgfqpoint{5.700000in}{5.700000in}}%
\pgfusepath{clip}%
\pgfsetbuttcap%
\pgfsetroundjoin%
\definecolor{currentfill}{rgb}{0.280894,0.078907,0.402329}%
\pgfsetfillcolor{currentfill}%
\pgfsetfillopacity{0.800000}%
\pgfsetlinewidth{0.000000pt}%
\definecolor{currentstroke}{rgb}{0.000000,0.000000,0.000000}%
\pgfsetstrokecolor{currentstroke}%
\pgfsetdash{}{0pt}%
\pgfpathmoveto{\pgfqpoint{2.923465in}{1.894607in}}%
\pgfpathlineto{\pgfqpoint{2.937174in}{1.883209in}}%
\pgfpathlineto{\pgfqpoint{2.950880in}{1.872052in}}%
\pgfpathlineto{\pgfqpoint{2.964583in}{1.861133in}}%
\pgfpathlineto{\pgfqpoint{2.978284in}{1.850451in}}%
\pgfpathlineto{\pgfqpoint{2.986721in}{1.855957in}}%
\pgfpathlineto{\pgfqpoint{2.995148in}{1.861623in}}%
\pgfpathlineto{\pgfqpoint{3.003565in}{1.867444in}}%
\pgfpathlineto{\pgfqpoint{3.011971in}{1.873417in}}%
\pgfpathlineto{\pgfqpoint{2.998299in}{1.883658in}}%
\pgfpathlineto{\pgfqpoint{2.984624in}{1.894137in}}%
\pgfpathlineto{\pgfqpoint{2.970947in}{1.904853in}}%
\pgfpathlineto{\pgfqpoint{2.957268in}{1.915809in}}%
\pgfpathlineto{\pgfqpoint{2.948834in}{1.910265in}}%
\pgfpathlineto{\pgfqpoint{2.940388in}{1.904881in}}%
\pgfpathlineto{\pgfqpoint{2.931932in}{1.899660in}}%
\pgfpathlineto{\pgfqpoint{2.923465in}{1.894607in}}%
\pgfpathclose%
\pgfusepath{fill}%
\end{pgfscope}%
\begin{pgfscope}%
\pgfpathrectangle{\pgfqpoint{1.150000in}{0.150000in}}{\pgfqpoint{5.700000in}{5.700000in}}%
\pgfusepath{clip}%
\pgfsetbuttcap%
\pgfsetroundjoin%
\definecolor{currentfill}{rgb}{0.190631,0.407061,0.556089}%
\pgfsetfillcolor{currentfill}%
\pgfsetfillopacity{0.800000}%
\pgfsetlinewidth{0.000000pt}%
\definecolor{currentstroke}{rgb}{0.000000,0.000000,0.000000}%
\pgfsetstrokecolor{currentstroke}%
\pgfsetdash{}{0pt}%
\pgfpathmoveto{\pgfqpoint{4.775761in}{2.657340in}}%
\pgfpathlineto{\pgfqpoint{4.789893in}{2.666326in}}%
\pgfpathlineto{\pgfqpoint{4.804041in}{2.675495in}}%
\pgfpathlineto{\pgfqpoint{4.818205in}{2.684847in}}%
\pgfpathlineto{\pgfqpoint{4.832385in}{2.694382in}}%
\pgfpathlineto{\pgfqpoint{4.840080in}{2.702207in}}%
\pgfpathlineto{\pgfqpoint{4.847768in}{2.709948in}}%
\pgfpathlineto{\pgfqpoint{4.855450in}{2.717608in}}%
\pgfpathlineto{\pgfqpoint{4.863124in}{2.725189in}}%
\pgfpathlineto{\pgfqpoint{4.848954in}{2.715865in}}%
\pgfpathlineto{\pgfqpoint{4.834800in}{2.706724in}}%
\pgfpathlineto{\pgfqpoint{4.820662in}{2.697765in}}%
\pgfpathlineto{\pgfqpoint{4.806539in}{2.688989in}}%
\pgfpathlineto{\pgfqpoint{4.798855in}{2.681186in}}%
\pgfpathlineto{\pgfqpoint{4.791163in}{2.673311in}}%
\pgfpathlineto{\pgfqpoint{4.783466in}{2.665364in}}%
\pgfpathlineto{\pgfqpoint{4.775761in}{2.657340in}}%
\pgfpathclose%
\pgfusepath{fill}%
\end{pgfscope}%
\begin{pgfscope}%
\pgfpathrectangle{\pgfqpoint{1.150000in}{0.150000in}}{\pgfqpoint{5.700000in}{5.700000in}}%
\pgfusepath{clip}%
\pgfsetbuttcap%
\pgfsetroundjoin%
\definecolor{currentfill}{rgb}{0.191090,0.708366,0.482284}%
\pgfsetfillcolor{currentfill}%
\pgfsetfillopacity{0.800000}%
\pgfsetlinewidth{0.000000pt}%
\definecolor{currentstroke}{rgb}{0.000000,0.000000,0.000000}%
\pgfsetstrokecolor{currentstroke}%
\pgfsetdash{}{0pt}%
\pgfpathmoveto{\pgfqpoint{6.146080in}{3.600290in}}%
\pgfpathlineto{\pgfqpoint{6.160943in}{3.611777in}}%
\pgfpathlineto{\pgfqpoint{6.175827in}{3.623438in}}%
\pgfpathlineto{\pgfqpoint{6.190733in}{3.635273in}}%
\pgfpathlineto{\pgfqpoint{6.197670in}{3.636381in}}%
\pgfpathlineto{\pgfqpoint{6.204608in}{3.637661in}}%
\pgfpathlineto{\pgfqpoint{6.211546in}{3.639122in}}%
\pgfpathlineto{\pgfqpoint{6.218484in}{3.640773in}}%
\pgfpathlineto{\pgfqpoint{6.203623in}{3.629725in}}%
\pgfpathlineto{\pgfqpoint{6.188784in}{3.618849in}}%
\pgfpathlineto{\pgfqpoint{6.173965in}{3.608146in}}%
\pgfpathlineto{\pgfqpoint{6.166993in}{3.605899in}}%
\pgfpathlineto{\pgfqpoint{6.160021in}{3.603846in}}%
\pgfpathlineto{\pgfqpoint{6.153051in}{3.601979in}}%
\pgfpathlineto{\pgfqpoint{6.146080in}{3.600290in}}%
\pgfpathclose%
\pgfusepath{fill}%
\end{pgfscope}%
\begin{pgfscope}%
\pgfpathrectangle{\pgfqpoint{1.150000in}{0.150000in}}{\pgfqpoint{5.700000in}{5.700000in}}%
\pgfusepath{clip}%
\pgfsetbuttcap%
\pgfsetroundjoin%
\definecolor{currentfill}{rgb}{0.275191,0.194905,0.496005}%
\pgfsetfillcolor{currentfill}%
\pgfsetfillopacity{0.800000}%
\pgfsetlinewidth{0.000000pt}%
\definecolor{currentstroke}{rgb}{0.000000,0.000000,0.000000}%
\pgfsetstrokecolor{currentstroke}%
\pgfsetdash{}{0pt}%
\pgfpathmoveto{\pgfqpoint{4.102698in}{2.115361in}}%
\pgfpathlineto{\pgfqpoint{4.116511in}{2.119538in}}%
\pgfpathlineto{\pgfqpoint{4.130335in}{2.123906in}}%
\pgfpathlineto{\pgfqpoint{4.144170in}{2.128462in}}%
\pgfpathlineto{\pgfqpoint{4.158016in}{2.133207in}}%
\pgfpathlineto{\pgfqpoint{4.165967in}{2.144375in}}%
\pgfpathlineto{\pgfqpoint{4.173913in}{2.155480in}}%
\pgfpathlineto{\pgfqpoint{4.181853in}{2.166521in}}%
\pgfpathlineto{\pgfqpoint{4.189789in}{2.177498in}}%
\pgfpathlineto{\pgfqpoint{4.175948in}{2.172667in}}%
\pgfpathlineto{\pgfqpoint{4.162118in}{2.168024in}}%
\pgfpathlineto{\pgfqpoint{4.148299in}{2.163570in}}%
\pgfpathlineto{\pgfqpoint{4.134490in}{2.159306in}}%
\pgfpathlineto{\pgfqpoint{4.126550in}{2.148404in}}%
\pgfpathlineto{\pgfqpoint{4.118604in}{2.137445in}}%
\pgfpathlineto{\pgfqpoint{4.110654in}{2.126430in}}%
\pgfpathlineto{\pgfqpoint{4.102698in}{2.115361in}}%
\pgfpathclose%
\pgfusepath{fill}%
\end{pgfscope}%
\begin{pgfscope}%
\pgfpathrectangle{\pgfqpoint{1.150000in}{0.150000in}}{\pgfqpoint{5.700000in}{5.700000in}}%
\pgfusepath{clip}%
\pgfsetbuttcap%
\pgfsetroundjoin%
\definecolor{currentfill}{rgb}{0.218130,0.347432,0.550038}%
\pgfsetfillcolor{currentfill}%
\pgfsetfillopacity{0.800000}%
\pgfsetlinewidth{0.000000pt}%
\definecolor{currentstroke}{rgb}{0.000000,0.000000,0.000000}%
\pgfsetstrokecolor{currentstroke}%
\pgfsetdash{}{0pt}%
\pgfpathmoveto{\pgfqpoint{2.369971in}{2.568729in}}%
\pgfpathlineto{\pgfqpoint{2.384002in}{2.546097in}}%
\pgfpathlineto{\pgfqpoint{2.398020in}{2.523795in}}%
\pgfpathlineto{\pgfqpoint{2.412024in}{2.501821in}}%
\pgfpathlineto{\pgfqpoint{2.426015in}{2.480171in}}%
\pgfpathlineto{\pgfqpoint{2.434821in}{2.481037in}}%
\pgfpathlineto{\pgfqpoint{2.443611in}{2.482147in}}%
\pgfpathlineto{\pgfqpoint{2.452385in}{2.483497in}}%
\pgfpathlineto{\pgfqpoint{2.461143in}{2.485084in}}%
\pgfpathlineto{\pgfqpoint{2.447196in}{2.506266in}}%
\pgfpathlineto{\pgfqpoint{2.433238in}{2.527771in}}%
\pgfpathlineto{\pgfqpoint{2.419266in}{2.549603in}}%
\pgfpathlineto{\pgfqpoint{2.405281in}{2.571764in}}%
\pgfpathlineto{\pgfqpoint{2.396478in}{2.570634in}}%
\pgfpathlineto{\pgfqpoint{2.387660in}{2.569748in}}%
\pgfpathlineto{\pgfqpoint{2.378824in}{2.569112in}}%
\pgfpathlineto{\pgfqpoint{2.369971in}{2.568729in}}%
\pgfpathclose%
\pgfusepath{fill}%
\end{pgfscope}%
\begin{pgfscope}%
\pgfpathrectangle{\pgfqpoint{1.150000in}{0.150000in}}{\pgfqpoint{5.700000in}{5.700000in}}%
\pgfusepath{clip}%
\pgfsetbuttcap%
\pgfsetroundjoin%
\definecolor{currentfill}{rgb}{0.153894,0.680203,0.504172}%
\pgfsetfillcolor{currentfill}%
\pgfsetfillopacity{0.800000}%
\pgfsetlinewidth{0.000000pt}%
\definecolor{currentstroke}{rgb}{0.000000,0.000000,0.000000}%
\pgfsetstrokecolor{currentstroke}%
\pgfsetdash{}{0pt}%
\pgfpathmoveto{\pgfqpoint{5.971481in}{3.494775in}}%
\pgfpathlineto{\pgfqpoint{5.986256in}{3.506340in}}%
\pgfpathlineto{\pgfqpoint{6.001052in}{3.518081in}}%
\pgfpathlineto{\pgfqpoint{6.015870in}{3.529996in}}%
\pgfpathlineto{\pgfqpoint{6.030708in}{3.542087in}}%
\pgfpathlineto{\pgfqpoint{6.037731in}{3.543357in}}%
\pgfpathlineto{\pgfqpoint{6.044751in}{3.544748in}}%
\pgfpathlineto{\pgfqpoint{6.051769in}{3.546268in}}%
\pgfpathlineto{\pgfqpoint{6.058785in}{3.547925in}}%
\pgfpathlineto{\pgfqpoint{6.043986in}{3.536555in}}%
\pgfpathlineto{\pgfqpoint{6.029209in}{3.525360in}}%
\pgfpathlineto{\pgfqpoint{6.014451in}{3.514338in}}%
\pgfpathlineto{\pgfqpoint{5.999715in}{3.503490in}}%
\pgfpathlineto{\pgfqpoint{5.992659in}{3.501103in}}%
\pgfpathlineto{\pgfqpoint{5.985601in}{3.498860in}}%
\pgfpathlineto{\pgfqpoint{5.978542in}{3.496753in}}%
\pgfpathlineto{\pgfqpoint{5.971481in}{3.494775in}}%
\pgfpathclose%
\pgfusepath{fill}%
\end{pgfscope}%
\begin{pgfscope}%
\pgfpathrectangle{\pgfqpoint{1.150000in}{0.150000in}}{\pgfqpoint{5.700000in}{5.700000in}}%
\pgfusepath{clip}%
\pgfsetbuttcap%
\pgfsetroundjoin%
\definecolor{currentfill}{rgb}{0.272594,0.025563,0.353093}%
\pgfsetfillcolor{currentfill}%
\pgfsetfillopacity{0.800000}%
\pgfsetlinewidth{0.000000pt}%
\definecolor{currentstroke}{rgb}{0.000000,0.000000,0.000000}%
\pgfsetstrokecolor{currentstroke}%
\pgfsetdash{}{0pt}%
\pgfpathmoveto{\pgfqpoint{3.405921in}{1.766623in}}%
\pgfpathlineto{\pgfqpoint{3.419574in}{1.762666in}}%
\pgfpathlineto{\pgfqpoint{3.433231in}{1.758915in}}%
\pgfpathlineto{\pgfqpoint{3.446892in}{1.755370in}}%
\pgfpathlineto{\pgfqpoint{3.460557in}{1.752031in}}%
\pgfpathlineto{\pgfqpoint{3.468745in}{1.761660in}}%
\pgfpathlineto{\pgfqpoint{3.476927in}{1.771348in}}%
\pgfpathlineto{\pgfqpoint{3.485102in}{1.781092in}}%
\pgfpathlineto{\pgfqpoint{3.493271in}{1.790889in}}%
\pgfpathlineto{\pgfqpoint{3.479621in}{1.793889in}}%
\pgfpathlineto{\pgfqpoint{3.465975in}{1.797094in}}%
\pgfpathlineto{\pgfqpoint{3.452333in}{1.800506in}}%
\pgfpathlineto{\pgfqpoint{3.438695in}{1.804124in}}%
\pgfpathlineto{\pgfqpoint{3.430511in}{1.794654in}}%
\pgfpathlineto{\pgfqpoint{3.422321in}{1.785246in}}%
\pgfpathlineto{\pgfqpoint{3.414125in}{1.775901in}}%
\pgfpathlineto{\pgfqpoint{3.405921in}{1.766623in}}%
\pgfpathclose%
\pgfusepath{fill}%
\end{pgfscope}%
\begin{pgfscope}%
\pgfpathrectangle{\pgfqpoint{1.150000in}{0.150000in}}{\pgfqpoint{5.700000in}{5.700000in}}%
\pgfusepath{clip}%
\pgfsetbuttcap%
\pgfsetroundjoin%
\definecolor{currentfill}{rgb}{0.170948,0.694384,0.493803}%
\pgfsetfillcolor{currentfill}%
\pgfsetfillopacity{0.800000}%
\pgfsetlinewidth{0.000000pt}%
\definecolor{currentstroke}{rgb}{0.000000,0.000000,0.000000}%
\pgfsetstrokecolor{currentstroke}%
\pgfsetdash{}{0pt}%
\pgfpathmoveto{\pgfqpoint{6.058785in}{3.547925in}}%
\pgfpathlineto{\pgfqpoint{6.073605in}{3.559469in}}%
\pgfpathlineto{\pgfqpoint{6.088446in}{3.571188in}}%
\pgfpathlineto{\pgfqpoint{6.103309in}{3.583081in}}%
\pgfpathlineto{\pgfqpoint{6.118193in}{3.595150in}}%
\pgfpathlineto{\pgfqpoint{6.125166in}{3.596209in}}%
\pgfpathlineto{\pgfqpoint{6.132138in}{3.597413in}}%
\pgfpathlineto{\pgfqpoint{6.139109in}{3.598771in}}%
\pgfpathlineto{\pgfqpoint{6.146080in}{3.600290in}}%
\pgfpathlineto{\pgfqpoint{6.131239in}{3.588977in}}%
\pgfpathlineto{\pgfqpoint{6.116418in}{3.577837in}}%
\pgfpathlineto{\pgfqpoint{6.101619in}{3.566871in}}%
\pgfpathlineto{\pgfqpoint{6.086840in}{3.556078in}}%
\pgfpathlineto{\pgfqpoint{6.079827in}{3.553795in}}%
\pgfpathlineto{\pgfqpoint{6.072814in}{3.551681in}}%
\pgfpathlineto{\pgfqpoint{6.065800in}{3.549727in}}%
\pgfpathlineto{\pgfqpoint{6.058785in}{3.547925in}}%
\pgfpathclose%
\pgfusepath{fill}%
\end{pgfscope}%
\begin{pgfscope}%
\pgfpathrectangle{\pgfqpoint{1.150000in}{0.150000in}}{\pgfqpoint{5.700000in}{5.700000in}}%
\pgfusepath{clip}%
\pgfsetbuttcap%
\pgfsetroundjoin%
\definecolor{currentfill}{rgb}{0.229739,0.322361,0.545706}%
\pgfsetfillcolor{currentfill}%
\pgfsetfillopacity{0.800000}%
\pgfsetlinewidth{0.000000pt}%
\definecolor{currentstroke}{rgb}{0.000000,0.000000,0.000000}%
\pgfsetstrokecolor{currentstroke}%
\pgfsetdash{}{0pt}%
\pgfpathmoveto{\pgfqpoint{4.482832in}{2.417265in}}%
\pgfpathlineto{\pgfqpoint{4.496816in}{2.424547in}}%
\pgfpathlineto{\pgfqpoint{4.510814in}{2.432015in}}%
\pgfpathlineto{\pgfqpoint{4.524826in}{2.439667in}}%
\pgfpathlineto{\pgfqpoint{4.538852in}{2.447505in}}%
\pgfpathlineto{\pgfqpoint{4.546673in}{2.457157in}}%
\pgfpathlineto{\pgfqpoint{4.554488in}{2.466720in}}%
\pgfpathlineto{\pgfqpoint{4.562297in}{2.476194in}}%
\pgfpathlineto{\pgfqpoint{4.570100in}{2.485582in}}%
\pgfpathlineto{\pgfqpoint{4.556079in}{2.477821in}}%
\pgfpathlineto{\pgfqpoint{4.542073in}{2.470246in}}%
\pgfpathlineto{\pgfqpoint{4.528081in}{2.462855in}}%
\pgfpathlineto{\pgfqpoint{4.514103in}{2.455649in}}%
\pgfpathlineto{\pgfqpoint{4.506294in}{2.446172in}}%
\pgfpathlineto{\pgfqpoint{4.498479in}{2.436617in}}%
\pgfpathlineto{\pgfqpoint{4.490658in}{2.426982in}}%
\pgfpathlineto{\pgfqpoint{4.482832in}{2.417265in}}%
\pgfpathclose%
\pgfusepath{fill}%
\end{pgfscope}%
\begin{pgfscope}%
\pgfpathrectangle{\pgfqpoint{1.150000in}{0.150000in}}{\pgfqpoint{5.700000in}{5.700000in}}%
\pgfusepath{clip}%
\pgfsetbuttcap%
\pgfsetroundjoin%
\definecolor{currentfill}{rgb}{0.143343,0.522773,0.556295}%
\pgfsetfillcolor{currentfill}%
\pgfsetfillopacity{0.800000}%
\pgfsetlinewidth{0.000000pt}%
\definecolor{currentstroke}{rgb}{0.000000,0.000000,0.000000}%
\pgfsetstrokecolor{currentstroke}%
\pgfsetdash{}{0pt}%
\pgfpathmoveto{\pgfqpoint{5.243156in}{3.014144in}}%
\pgfpathlineto{\pgfqpoint{5.257549in}{3.024987in}}%
\pgfpathlineto{\pgfqpoint{5.271960in}{3.036011in}}%
\pgfpathlineto{\pgfqpoint{5.286390in}{3.047214in}}%
\pgfpathlineto{\pgfqpoint{5.300838in}{3.058598in}}%
\pgfpathlineto{\pgfqpoint{5.308292in}{3.063301in}}%
\pgfpathlineto{\pgfqpoint{5.315739in}{3.067965in}}%
\pgfpathlineto{\pgfqpoint{5.323179in}{3.072595in}}%
\pgfpathlineto{\pgfqpoint{5.330611in}{3.077196in}}%
\pgfpathlineto{\pgfqpoint{5.316182in}{3.066228in}}%
\pgfpathlineto{\pgfqpoint{5.301771in}{3.055439in}}%
\pgfpathlineto{\pgfqpoint{5.287378in}{3.044830in}}%
\pgfpathlineto{\pgfqpoint{5.273004in}{3.034400in}}%
\pgfpathlineto{\pgfqpoint{5.265552in}{3.029372in}}%
\pgfpathlineto{\pgfqpoint{5.258093in}{3.024324in}}%
\pgfpathlineto{\pgfqpoint{5.250628in}{3.019249in}}%
\pgfpathlineto{\pgfqpoint{5.243156in}{3.014144in}}%
\pgfpathclose%
\pgfusepath{fill}%
\end{pgfscope}%
\begin{pgfscope}%
\pgfpathrectangle{\pgfqpoint{1.150000in}{0.150000in}}{\pgfqpoint{5.700000in}{5.700000in}}%
\pgfusepath{clip}%
\pgfsetbuttcap%
\pgfsetroundjoin%
\definecolor{currentfill}{rgb}{0.267968,0.223549,0.512008}%
\pgfsetfillcolor{currentfill}%
\pgfsetfillopacity{0.800000}%
\pgfsetlinewidth{0.000000pt}%
\definecolor{currentstroke}{rgb}{0.000000,0.000000,0.000000}%
\pgfsetstrokecolor{currentstroke}%
\pgfsetdash{}{0pt}%
\pgfpathmoveto{\pgfqpoint{4.189789in}{2.177498in}}%
\pgfpathlineto{\pgfqpoint{4.203642in}{2.182518in}}%
\pgfpathlineto{\pgfqpoint{4.217506in}{2.187726in}}%
\pgfpathlineto{\pgfqpoint{4.231382in}{2.193122in}}%
\pgfpathlineto{\pgfqpoint{4.245270in}{2.198705in}}%
\pgfpathlineto{\pgfqpoint{4.253195in}{2.209684in}}%
\pgfpathlineto{\pgfqpoint{4.261116in}{2.220590in}}%
\pgfpathlineto{\pgfqpoint{4.269031in}{2.231423in}}%
\pgfpathlineto{\pgfqpoint{4.276941in}{2.242182in}}%
\pgfpathlineto{\pgfqpoint{4.263057in}{2.236545in}}%
\pgfpathlineto{\pgfqpoint{4.249186in}{2.231095in}}%
\pgfpathlineto{\pgfqpoint{4.235327in}{2.225833in}}%
\pgfpathlineto{\pgfqpoint{4.221479in}{2.220759in}}%
\pgfpathlineto{\pgfqpoint{4.213564in}{2.210041in}}%
\pgfpathlineto{\pgfqpoint{4.205644in}{2.199258in}}%
\pgfpathlineto{\pgfqpoint{4.197719in}{2.188411in}}%
\pgfpathlineto{\pgfqpoint{4.189789in}{2.177498in}}%
\pgfpathclose%
\pgfusepath{fill}%
\end{pgfscope}%
\begin{pgfscope}%
\pgfpathrectangle{\pgfqpoint{1.150000in}{0.150000in}}{\pgfqpoint{5.700000in}{5.700000in}}%
\pgfusepath{clip}%
\pgfsetbuttcap%
\pgfsetroundjoin%
\definecolor{currentfill}{rgb}{0.278791,0.062145,0.386592}%
\pgfsetfillcolor{currentfill}%
\pgfsetfillopacity{0.800000}%
\pgfsetlinewidth{0.000000pt}%
\definecolor{currentstroke}{rgb}{0.000000,0.000000,0.000000}%
\pgfsetstrokecolor{currentstroke}%
\pgfsetdash{}{0pt}%
\pgfpathmoveto{\pgfqpoint{2.978284in}{1.850451in}}%
\pgfpathlineto{\pgfqpoint{2.991982in}{1.840005in}}%
\pgfpathlineto{\pgfqpoint{3.005679in}{1.829792in}}%
\pgfpathlineto{\pgfqpoint{3.019374in}{1.819812in}}%
\pgfpathlineto{\pgfqpoint{3.033067in}{1.810064in}}%
\pgfpathlineto{\pgfqpoint{3.041477in}{1.816022in}}%
\pgfpathlineto{\pgfqpoint{3.049876in}{1.822131in}}%
\pgfpathlineto{\pgfqpoint{3.058266in}{1.828387in}}%
\pgfpathlineto{\pgfqpoint{3.066646in}{1.834787in}}%
\pgfpathlineto{\pgfqpoint{3.052979in}{1.844096in}}%
\pgfpathlineto{\pgfqpoint{3.039311in}{1.853637in}}%
\pgfpathlineto{\pgfqpoint{3.025642in}{1.863410in}}%
\pgfpathlineto{\pgfqpoint{3.011971in}{1.873417in}}%
\pgfpathlineto{\pgfqpoint{3.003565in}{1.867444in}}%
\pgfpathlineto{\pgfqpoint{2.995148in}{1.861623in}}%
\pgfpathlineto{\pgfqpoint{2.986721in}{1.855957in}}%
\pgfpathlineto{\pgfqpoint{2.978284in}{1.850451in}}%
\pgfpathclose%
\pgfusepath{fill}%
\end{pgfscope}%
\begin{pgfscope}%
\pgfpathrectangle{\pgfqpoint{1.150000in}{0.150000in}}{\pgfqpoint{5.700000in}{5.700000in}}%
\pgfusepath{clip}%
\pgfsetbuttcap%
\pgfsetroundjoin%
\definecolor{currentfill}{rgb}{0.180629,0.429975,0.557282}%
\pgfsetfillcolor{currentfill}%
\pgfsetfillopacity{0.800000}%
\pgfsetlinewidth{0.000000pt}%
\definecolor{currentstroke}{rgb}{0.000000,0.000000,0.000000}%
\pgfsetstrokecolor{currentstroke}%
\pgfsetdash{}{0pt}%
\pgfpathmoveto{\pgfqpoint{4.863124in}{2.725189in}}%
\pgfpathlineto{\pgfqpoint{4.877311in}{2.734696in}}%
\pgfpathlineto{\pgfqpoint{4.891513in}{2.744385in}}%
\pgfpathlineto{\pgfqpoint{4.905732in}{2.754256in}}%
\pgfpathlineto{\pgfqpoint{4.919968in}{2.764310in}}%
\pgfpathlineto{\pgfqpoint{4.927625in}{2.771583in}}%
\pgfpathlineto{\pgfqpoint{4.935275in}{2.778774in}}%
\pgfpathlineto{\pgfqpoint{4.942919in}{2.785887in}}%
\pgfpathlineto{\pgfqpoint{4.950555in}{2.792924in}}%
\pgfpathlineto{\pgfqpoint{4.936330in}{2.783115in}}%
\pgfpathlineto{\pgfqpoint{4.922122in}{2.773489in}}%
\pgfpathlineto{\pgfqpoint{4.907930in}{2.764044in}}%
\pgfpathlineto{\pgfqpoint{4.893755in}{2.754781in}}%
\pgfpathlineto{\pgfqpoint{4.886107in}{2.747487in}}%
\pgfpathlineto{\pgfqpoint{4.878453in}{2.740126in}}%
\pgfpathlineto{\pgfqpoint{4.870792in}{2.732694in}}%
\pgfpathlineto{\pgfqpoint{4.863124in}{2.725189in}}%
\pgfpathclose%
\pgfusepath{fill}%
\end{pgfscope}%
\begin{pgfscope}%
\pgfpathrectangle{\pgfqpoint{1.150000in}{0.150000in}}{\pgfqpoint{5.700000in}{5.700000in}}%
\pgfusepath{clip}%
\pgfsetbuttcap%
\pgfsetroundjoin%
\definecolor{currentfill}{rgb}{0.135066,0.544853,0.554029}%
\pgfsetfillcolor{currentfill}%
\pgfsetfillopacity{0.800000}%
\pgfsetlinewidth{0.000000pt}%
\definecolor{currentstroke}{rgb}{0.000000,0.000000,0.000000}%
\pgfsetstrokecolor{currentstroke}%
\pgfsetdash{}{0pt}%
\pgfpathmoveto{\pgfqpoint{5.330611in}{3.077196in}}%
\pgfpathlineto{\pgfqpoint{5.345059in}{3.088344in}}%
\pgfpathlineto{\pgfqpoint{5.359526in}{3.099671in}}%
\pgfpathlineto{\pgfqpoint{5.374011in}{3.111177in}}%
\pgfpathlineto{\pgfqpoint{5.388515in}{3.122864in}}%
\pgfpathlineto{\pgfqpoint{5.395921in}{3.127003in}}%
\pgfpathlineto{\pgfqpoint{5.403319in}{3.131115in}}%
\pgfpathlineto{\pgfqpoint{5.410710in}{3.135204in}}%
\pgfpathlineto{\pgfqpoint{5.418095in}{3.139276in}}%
\pgfpathlineto{\pgfqpoint{5.403612in}{3.128040in}}%
\pgfpathlineto{\pgfqpoint{5.389147in}{3.116983in}}%
\pgfpathlineto{\pgfqpoint{5.374701in}{3.106105in}}%
\pgfpathlineto{\pgfqpoint{5.360274in}{3.095406in}}%
\pgfpathlineto{\pgfqpoint{5.352868in}{3.090872in}}%
\pgfpathlineto{\pgfqpoint{5.345456in}{3.086330in}}%
\pgfpathlineto{\pgfqpoint{5.338037in}{3.081773in}}%
\pgfpathlineto{\pgfqpoint{5.330611in}{3.077196in}}%
\pgfpathclose%
\pgfusepath{fill}%
\end{pgfscope}%
\begin{pgfscope}%
\pgfpathrectangle{\pgfqpoint{1.150000in}{0.150000in}}{\pgfqpoint{5.700000in}{5.700000in}}%
\pgfusepath{clip}%
\pgfsetbuttcap%
\pgfsetroundjoin%
\definecolor{currentfill}{rgb}{0.271305,0.019942,0.347269}%
\pgfsetfillcolor{currentfill}%
\pgfsetfillopacity{0.800000}%
\pgfsetlinewidth{0.000000pt}%
\definecolor{currentstroke}{rgb}{0.000000,0.000000,0.000000}%
\pgfsetstrokecolor{currentstroke}%
\pgfsetdash{}{0pt}%
\pgfpathmoveto{\pgfqpoint{3.175961in}{1.768468in}}%
\pgfpathlineto{\pgfqpoint{3.189627in}{1.761180in}}%
\pgfpathlineto{\pgfqpoint{3.203293in}{1.754110in}}%
\pgfpathlineto{\pgfqpoint{3.216960in}{1.747258in}}%
\pgfpathlineto{\pgfqpoint{3.230628in}{1.740623in}}%
\pgfpathlineto{\pgfqpoint{3.238928in}{1.748412in}}%
\pgfpathlineto{\pgfqpoint{3.247220in}{1.756312in}}%
\pgfpathlineto{\pgfqpoint{3.255504in}{1.764320in}}%
\pgfpathlineto{\pgfqpoint{3.263780in}{1.772430in}}%
\pgfpathlineto{\pgfqpoint{3.250133in}{1.778662in}}%
\pgfpathlineto{\pgfqpoint{3.236487in}{1.785110in}}%
\pgfpathlineto{\pgfqpoint{3.222842in}{1.791776in}}%
\pgfpathlineto{\pgfqpoint{3.209199in}{1.798660in}}%
\pgfpathlineto{\pgfqpoint{3.200902in}{1.790941in}}%
\pgfpathlineto{\pgfqpoint{3.192597in}{1.783334in}}%
\pgfpathlineto{\pgfqpoint{3.184284in}{1.775841in}}%
\pgfpathlineto{\pgfqpoint{3.175961in}{1.768468in}}%
\pgfpathclose%
\pgfusepath{fill}%
\end{pgfscope}%
\begin{pgfscope}%
\pgfpathrectangle{\pgfqpoint{1.150000in}{0.150000in}}{\pgfqpoint{5.700000in}{5.700000in}}%
\pgfusepath{clip}%
\pgfsetbuttcap%
\pgfsetroundjoin%
\definecolor{currentfill}{rgb}{0.201239,0.383670,0.554294}%
\pgfsetfillcolor{currentfill}%
\pgfsetfillopacity{0.800000}%
\pgfsetlinewidth{0.000000pt}%
\definecolor{currentstroke}{rgb}{0.000000,0.000000,0.000000}%
\pgfsetstrokecolor{currentstroke}%
\pgfsetdash{}{0pt}%
\pgfpathmoveto{\pgfqpoint{2.313699in}{2.662629in}}%
\pgfpathlineto{\pgfqpoint{2.327789in}{2.638641in}}%
\pgfpathlineto{\pgfqpoint{2.341865in}{2.614998in}}%
\pgfpathlineto{\pgfqpoint{2.355925in}{2.591695in}}%
\pgfpathlineto{\pgfqpoint{2.369971in}{2.568729in}}%
\pgfpathlineto{\pgfqpoint{2.378824in}{2.569112in}}%
\pgfpathlineto{\pgfqpoint{2.387660in}{2.569748in}}%
\pgfpathlineto{\pgfqpoint{2.396478in}{2.570634in}}%
\pgfpathlineto{\pgfqpoint{2.405281in}{2.571764in}}%
\pgfpathlineto{\pgfqpoint{2.391282in}{2.594258in}}%
\pgfpathlineto{\pgfqpoint{2.377269in}{2.617088in}}%
\pgfpathlineto{\pgfqpoint{2.363241in}{2.640257in}}%
\pgfpathlineto{\pgfqpoint{2.349199in}{2.663768in}}%
\pgfpathlineto{\pgfqpoint{2.340350in}{2.663098in}}%
\pgfpathlineto{\pgfqpoint{2.331484in}{2.662682in}}%
\pgfpathlineto{\pgfqpoint{2.322601in}{2.662524in}}%
\pgfpathlineto{\pgfqpoint{2.313699in}{2.662629in}}%
\pgfpathclose%
\pgfusepath{fill}%
\end{pgfscope}%
\begin{pgfscope}%
\pgfpathrectangle{\pgfqpoint{1.150000in}{0.150000in}}{\pgfqpoint{5.700000in}{5.700000in}}%
\pgfusepath{clip}%
\pgfsetbuttcap%
\pgfsetroundjoin%
\definecolor{currentfill}{rgb}{0.218130,0.347432,0.550038}%
\pgfsetfillcolor{currentfill}%
\pgfsetfillopacity{0.800000}%
\pgfsetlinewidth{0.000000pt}%
\definecolor{currentstroke}{rgb}{0.000000,0.000000,0.000000}%
\pgfsetstrokecolor{currentstroke}%
\pgfsetdash{}{0pt}%
\pgfpathmoveto{\pgfqpoint{4.570100in}{2.485582in}}%
\pgfpathlineto{\pgfqpoint{4.584134in}{2.493527in}}%
\pgfpathlineto{\pgfqpoint{4.598183in}{2.501657in}}%
\pgfpathlineto{\pgfqpoint{4.612247in}{2.509972in}}%
\pgfpathlineto{\pgfqpoint{4.626325in}{2.518471in}}%
\pgfpathlineto{\pgfqpoint{4.634115in}{2.527675in}}%
\pgfpathlineto{\pgfqpoint{4.641899in}{2.536787in}}%
\pgfpathlineto{\pgfqpoint{4.649677in}{2.545809in}}%
\pgfpathlineto{\pgfqpoint{4.657448in}{2.554742in}}%
\pgfpathlineto{\pgfqpoint{4.643376in}{2.546354in}}%
\pgfpathlineto{\pgfqpoint{4.629319in}{2.538150in}}%
\pgfpathlineto{\pgfqpoint{4.615277in}{2.530130in}}%
\pgfpathlineto{\pgfqpoint{4.601249in}{2.522294in}}%
\pgfpathlineto{\pgfqpoint{4.593471in}{2.513239in}}%
\pgfpathlineto{\pgfqpoint{4.585687in}{2.504103in}}%
\pgfpathlineto{\pgfqpoint{4.577896in}{2.494884in}}%
\pgfpathlineto{\pgfqpoint{4.570100in}{2.485582in}}%
\pgfpathclose%
\pgfusepath{fill}%
\end{pgfscope}%
\begin{pgfscope}%
\pgfpathrectangle{\pgfqpoint{1.150000in}{0.150000in}}{\pgfqpoint{5.700000in}{5.700000in}}%
\pgfusepath{clip}%
\pgfsetbuttcap%
\pgfsetroundjoin%
\definecolor{currentfill}{rgb}{0.280894,0.078907,0.402329}%
\pgfsetfillcolor{currentfill}%
\pgfsetfillopacity{0.800000}%
\pgfsetlinewidth{0.000000pt}%
\definecolor{currentstroke}{rgb}{0.000000,0.000000,0.000000}%
\pgfsetstrokecolor{currentstroke}%
\pgfsetdash{}{0pt}%
\pgfpathmoveto{\pgfqpoint{3.722335in}{1.857577in}}%
\pgfpathlineto{\pgfqpoint{3.736038in}{1.857765in}}%
\pgfpathlineto{\pgfqpoint{3.749749in}{1.858148in}}%
\pgfpathlineto{\pgfqpoint{3.763466in}{1.858726in}}%
\pgfpathlineto{\pgfqpoint{3.777191in}{1.859500in}}%
\pgfpathlineto{\pgfqpoint{3.785266in}{1.870676in}}%
\pgfpathlineto{\pgfqpoint{3.793335in}{1.881847in}}%
\pgfpathlineto{\pgfqpoint{3.801400in}{1.893012in}}%
\pgfpathlineto{\pgfqpoint{3.809459in}{1.904168in}}%
\pgfpathlineto{\pgfqpoint{3.795742in}{1.903149in}}%
\pgfpathlineto{\pgfqpoint{3.782033in}{1.902326in}}%
\pgfpathlineto{\pgfqpoint{3.768331in}{1.901697in}}%
\pgfpathlineto{\pgfqpoint{3.754637in}{1.901265in}}%
\pgfpathlineto{\pgfqpoint{3.746570in}{1.890342in}}%
\pgfpathlineto{\pgfqpoint{3.738497in}{1.879419in}}%
\pgfpathlineto{\pgfqpoint{3.730419in}{1.868496in}}%
\pgfpathlineto{\pgfqpoint{3.722335in}{1.857577in}}%
\pgfpathclose%
\pgfusepath{fill}%
\end{pgfscope}%
\begin{pgfscope}%
\pgfpathrectangle{\pgfqpoint{1.150000in}{0.150000in}}{\pgfqpoint{5.700000in}{5.700000in}}%
\pgfusepath{clip}%
\pgfsetbuttcap%
\pgfsetroundjoin%
\definecolor{currentfill}{rgb}{0.271305,0.019942,0.347269}%
\pgfsetfillcolor{currentfill}%
\pgfsetfillopacity{0.800000}%
\pgfsetlinewidth{0.000000pt}%
\definecolor{currentstroke}{rgb}{0.000000,0.000000,0.000000}%
\pgfsetstrokecolor{currentstroke}%
\pgfsetdash{}{0pt}%
\pgfpathmoveto{\pgfqpoint{3.318389in}{1.749647in}}%
\pgfpathlineto{\pgfqpoint{3.332048in}{1.744482in}}%
\pgfpathlineto{\pgfqpoint{3.345708in}{1.739528in}}%
\pgfpathlineto{\pgfqpoint{3.359372in}{1.734782in}}%
\pgfpathlineto{\pgfqpoint{3.373038in}{1.730246in}}%
\pgfpathlineto{\pgfqpoint{3.381270in}{1.739223in}}%
\pgfpathlineto{\pgfqpoint{3.389494in}{1.748281in}}%
\pgfpathlineto{\pgfqpoint{3.397711in}{1.757415in}}%
\pgfpathlineto{\pgfqpoint{3.405921in}{1.766623in}}%
\pgfpathlineto{\pgfqpoint{3.392271in}{1.770788in}}%
\pgfpathlineto{\pgfqpoint{3.378625in}{1.775162in}}%
\pgfpathlineto{\pgfqpoint{3.364982in}{1.779745in}}%
\pgfpathlineto{\pgfqpoint{3.351342in}{1.784539in}}%
\pgfpathlineto{\pgfqpoint{3.343114in}{1.775690in}}%
\pgfpathlineto{\pgfqpoint{3.334880in}{1.766923in}}%
\pgfpathlineto{\pgfqpoint{3.326638in}{1.758241in}}%
\pgfpathlineto{\pgfqpoint{3.318389in}{1.749647in}}%
\pgfpathclose%
\pgfusepath{fill}%
\end{pgfscope}%
\begin{pgfscope}%
\pgfpathrectangle{\pgfqpoint{1.150000in}{0.150000in}}{\pgfqpoint{5.700000in}{5.700000in}}%
\pgfusepath{clip}%
\pgfsetbuttcap%
\pgfsetroundjoin%
\definecolor{currentfill}{rgb}{0.282910,0.105393,0.426902}%
\pgfsetfillcolor{currentfill}%
\pgfsetfillopacity{0.800000}%
\pgfsetlinewidth{0.000000pt}%
\definecolor{currentstroke}{rgb}{0.000000,0.000000,0.000000}%
\pgfsetstrokecolor{currentstroke}%
\pgfsetdash{}{0pt}%
\pgfpathmoveto{\pgfqpoint{3.809459in}{1.904168in}}%
\pgfpathlineto{\pgfqpoint{3.823184in}{1.905381in}}%
\pgfpathlineto{\pgfqpoint{3.836917in}{1.906788in}}%
\pgfpathlineto{\pgfqpoint{3.850659in}{1.908389in}}%
\pgfpathlineto{\pgfqpoint{3.864409in}{1.910182in}}%
\pgfpathlineto{\pgfqpoint{3.872455in}{1.921553in}}%
\pgfpathlineto{\pgfqpoint{3.880497in}{1.932903in}}%
\pgfpathlineto{\pgfqpoint{3.888534in}{1.944231in}}%
\pgfpathlineto{\pgfqpoint{3.896566in}{1.955535in}}%
\pgfpathlineto{\pgfqpoint{3.882823in}{1.953528in}}%
\pgfpathlineto{\pgfqpoint{3.869089in}{1.951714in}}%
\pgfpathlineto{\pgfqpoint{3.855363in}{1.950093in}}%
\pgfpathlineto{\pgfqpoint{3.841646in}{1.948666in}}%
\pgfpathlineto{\pgfqpoint{3.833606in}{1.937564in}}%
\pgfpathlineto{\pgfqpoint{3.825562in}{1.926446in}}%
\pgfpathlineto{\pgfqpoint{3.817513in}{1.915313in}}%
\pgfpathlineto{\pgfqpoint{3.809459in}{1.904168in}}%
\pgfpathclose%
\pgfusepath{fill}%
\end{pgfscope}%
\begin{pgfscope}%
\pgfpathrectangle{\pgfqpoint{1.150000in}{0.150000in}}{\pgfqpoint{5.700000in}{5.700000in}}%
\pgfusepath{clip}%
\pgfsetbuttcap%
\pgfsetroundjoin%
\definecolor{currentfill}{rgb}{0.278791,0.062145,0.386592}%
\pgfsetfillcolor{currentfill}%
\pgfsetfillopacity{0.800000}%
\pgfsetlinewidth{0.000000pt}%
\definecolor{currentstroke}{rgb}{0.000000,0.000000,0.000000}%
\pgfsetstrokecolor{currentstroke}%
\pgfsetdash{}{0pt}%
\pgfpathmoveto{\pgfqpoint{3.635166in}{1.816307in}}%
\pgfpathlineto{\pgfqpoint{3.648852in}{1.815428in}}%
\pgfpathlineto{\pgfqpoint{3.662545in}{1.814748in}}%
\pgfpathlineto{\pgfqpoint{3.676244in}{1.814265in}}%
\pgfpathlineto{\pgfqpoint{3.689949in}{1.813980in}}%
\pgfpathlineto{\pgfqpoint{3.698054in}{1.824862in}}%
\pgfpathlineto{\pgfqpoint{3.706153in}{1.835758in}}%
\pgfpathlineto{\pgfqpoint{3.714247in}{1.846664in}}%
\pgfpathlineto{\pgfqpoint{3.722335in}{1.857577in}}%
\pgfpathlineto{\pgfqpoint{3.708639in}{1.857586in}}%
\pgfpathlineto{\pgfqpoint{3.694950in}{1.857792in}}%
\pgfpathlineto{\pgfqpoint{3.681268in}{1.858196in}}%
\pgfpathlineto{\pgfqpoint{3.667593in}{1.858798in}}%
\pgfpathlineto{\pgfqpoint{3.659494in}{1.848149in}}%
\pgfpathlineto{\pgfqpoint{3.651390in}{1.837516in}}%
\pgfpathlineto{\pgfqpoint{3.643281in}{1.826901in}}%
\pgfpathlineto{\pgfqpoint{3.635166in}{1.816307in}}%
\pgfpathclose%
\pgfusepath{fill}%
\end{pgfscope}%
\begin{pgfscope}%
\pgfpathrectangle{\pgfqpoint{1.150000in}{0.150000in}}{\pgfqpoint{5.700000in}{5.700000in}}%
\pgfusepath{clip}%
\pgfsetbuttcap%
\pgfsetroundjoin%
\definecolor{currentfill}{rgb}{0.257322,0.256130,0.526563}%
\pgfsetfillcolor{currentfill}%
\pgfsetfillopacity{0.800000}%
\pgfsetlinewidth{0.000000pt}%
\definecolor{currentstroke}{rgb}{0.000000,0.000000,0.000000}%
\pgfsetstrokecolor{currentstroke}%
\pgfsetdash{}{0pt}%
\pgfpathmoveto{\pgfqpoint{4.276941in}{2.242182in}}%
\pgfpathlineto{\pgfqpoint{4.290836in}{2.248007in}}%
\pgfpathlineto{\pgfqpoint{4.304744in}{2.254019in}}%
\pgfpathlineto{\pgfqpoint{4.318664in}{2.260218in}}%
\pgfpathlineto{\pgfqpoint{4.332597in}{2.266604in}}%
\pgfpathlineto{\pgfqpoint{4.340497in}{2.277325in}}%
\pgfpathlineto{\pgfqpoint{4.348392in}{2.287964in}}%
\pgfpathlineto{\pgfqpoint{4.356281in}{2.298523in}}%
\pgfpathlineto{\pgfqpoint{4.364164in}{2.309002in}}%
\pgfpathlineto{\pgfqpoint{4.350236in}{2.302595in}}%
\pgfpathlineto{\pgfqpoint{4.336320in}{2.296374in}}%
\pgfpathlineto{\pgfqpoint{4.322417in}{2.290340in}}%
\pgfpathlineto{\pgfqpoint{4.308526in}{2.284493in}}%
\pgfpathlineto{\pgfqpoint{4.300638in}{2.274025in}}%
\pgfpathlineto{\pgfqpoint{4.292744in}{2.263483in}}%
\pgfpathlineto{\pgfqpoint{4.284845in}{2.252869in}}%
\pgfpathlineto{\pgfqpoint{4.276941in}{2.242182in}}%
\pgfpathclose%
\pgfusepath{fill}%
\end{pgfscope}%
\begin{pgfscope}%
\pgfpathrectangle{\pgfqpoint{1.150000in}{0.150000in}}{\pgfqpoint{5.700000in}{5.700000in}}%
\pgfusepath{clip}%
\pgfsetbuttcap%
\pgfsetroundjoin%
\definecolor{currentfill}{rgb}{0.127568,0.566949,0.550556}%
\pgfsetfillcolor{currentfill}%
\pgfsetfillopacity{0.800000}%
\pgfsetlinewidth{0.000000pt}%
\definecolor{currentstroke}{rgb}{0.000000,0.000000,0.000000}%
\pgfsetstrokecolor{currentstroke}%
\pgfsetdash{}{0pt}%
\pgfpathmoveto{\pgfqpoint{5.418095in}{3.139276in}}%
\pgfpathlineto{\pgfqpoint{5.432597in}{3.150691in}}%
\pgfpathlineto{\pgfqpoint{5.447119in}{3.162285in}}%
\pgfpathlineto{\pgfqpoint{5.461660in}{3.174059in}}%
\pgfpathlineto{\pgfqpoint{5.476220in}{3.186011in}}%
\pgfpathlineto{\pgfqpoint{5.483575in}{3.189599in}}%
\pgfpathlineto{\pgfqpoint{5.490923in}{3.193172in}}%
\pgfpathlineto{\pgfqpoint{5.498265in}{3.196736in}}%
\pgfpathlineto{\pgfqpoint{5.505600in}{3.200296in}}%
\pgfpathlineto{\pgfqpoint{5.491063in}{3.188829in}}%
\pgfpathlineto{\pgfqpoint{5.476545in}{3.177540in}}%
\pgfpathlineto{\pgfqpoint{5.462047in}{3.166429in}}%
\pgfpathlineto{\pgfqpoint{5.447567in}{3.155497in}}%
\pgfpathlineto{\pgfqpoint{5.440209in}{3.151441in}}%
\pgfpathlineto{\pgfqpoint{5.432844in}{3.147390in}}%
\pgfpathlineto{\pgfqpoint{5.425473in}{3.143336in}}%
\pgfpathlineto{\pgfqpoint{5.418095in}{3.139276in}}%
\pgfpathclose%
\pgfusepath{fill}%
\end{pgfscope}%
\begin{pgfscope}%
\pgfpathrectangle{\pgfqpoint{1.150000in}{0.150000in}}{\pgfqpoint{5.700000in}{5.700000in}}%
\pgfusepath{clip}%
\pgfsetbuttcap%
\pgfsetroundjoin%
\definecolor{currentfill}{rgb}{0.169646,0.456262,0.558030}%
\pgfsetfillcolor{currentfill}%
\pgfsetfillopacity{0.800000}%
\pgfsetlinewidth{0.000000pt}%
\definecolor{currentstroke}{rgb}{0.000000,0.000000,0.000000}%
\pgfsetstrokecolor{currentstroke}%
\pgfsetdash{}{0pt}%
\pgfpathmoveto{\pgfqpoint{4.950555in}{2.792924in}}%
\pgfpathlineto{\pgfqpoint{4.964796in}{2.802915in}}%
\pgfpathlineto{\pgfqpoint{4.979054in}{2.813087in}}%
\pgfpathlineto{\pgfqpoint{4.993329in}{2.823442in}}%
\pgfpathlineto{\pgfqpoint{5.007622in}{2.833979in}}%
\pgfpathlineto{\pgfqpoint{5.015239in}{2.840677in}}%
\pgfpathlineto{\pgfqpoint{5.022849in}{2.847298in}}%
\pgfpathlineto{\pgfqpoint{5.030452in}{2.853845in}}%
\pgfpathlineto{\pgfqpoint{5.038047in}{2.860322in}}%
\pgfpathlineto{\pgfqpoint{5.023767in}{2.850065in}}%
\pgfpathlineto{\pgfqpoint{5.009505in}{2.839989in}}%
\pgfpathlineto{\pgfqpoint{4.995259in}{2.830095in}}%
\pgfpathlineto{\pgfqpoint{4.981029in}{2.820382in}}%
\pgfpathlineto{\pgfqpoint{4.973421in}{2.813615in}}%
\pgfpathlineto{\pgfqpoint{4.965806in}{2.806785in}}%
\pgfpathlineto{\pgfqpoint{4.958184in}{2.799889in}}%
\pgfpathlineto{\pgfqpoint{4.950555in}{2.792924in}}%
\pgfpathclose%
\pgfusepath{fill}%
\end{pgfscope}%
\begin{pgfscope}%
\pgfpathrectangle{\pgfqpoint{1.150000in}{0.150000in}}{\pgfqpoint{5.700000in}{5.700000in}}%
\pgfusepath{clip}%
\pgfsetbuttcap%
\pgfsetroundjoin%
\definecolor{currentfill}{rgb}{0.283072,0.130895,0.449241}%
\pgfsetfillcolor{currentfill}%
\pgfsetfillopacity{0.800000}%
\pgfsetlinewidth{0.000000pt}%
\definecolor{currentstroke}{rgb}{0.000000,0.000000,0.000000}%
\pgfsetstrokecolor{currentstroke}%
\pgfsetdash{}{0pt}%
\pgfpathmoveto{\pgfqpoint{3.896566in}{1.955535in}}%
\pgfpathlineto{\pgfqpoint{3.910318in}{1.957735in}}%
\pgfpathlineto{\pgfqpoint{3.924078in}{1.960127in}}%
\pgfpathlineto{\pgfqpoint{3.937848in}{1.962711in}}%
\pgfpathlineto{\pgfqpoint{3.951628in}{1.965486in}}%
\pgfpathlineto{\pgfqpoint{3.959648in}{1.976958in}}%
\pgfpathlineto{\pgfqpoint{3.967664in}{1.988395in}}%
\pgfpathlineto{\pgfqpoint{3.975675in}{1.999796in}}%
\pgfpathlineto{\pgfqpoint{3.983681in}{2.011158in}}%
\pgfpathlineto{\pgfqpoint{3.969907in}{2.008201in}}%
\pgfpathlineto{\pgfqpoint{3.956144in}{2.005435in}}%
\pgfpathlineto{\pgfqpoint{3.942389in}{2.002861in}}%
\pgfpathlineto{\pgfqpoint{3.928644in}{2.000479in}}%
\pgfpathlineto{\pgfqpoint{3.920632in}{1.989287in}}%
\pgfpathlineto{\pgfqpoint{3.912615in}{1.978064in}}%
\pgfpathlineto{\pgfqpoint{3.904593in}{1.966813in}}%
\pgfpathlineto{\pgfqpoint{3.896566in}{1.955535in}}%
\pgfpathclose%
\pgfusepath{fill}%
\end{pgfscope}%
\begin{pgfscope}%
\pgfpathrectangle{\pgfqpoint{1.150000in}{0.150000in}}{\pgfqpoint{5.700000in}{5.700000in}}%
\pgfusepath{clip}%
\pgfsetbuttcap%
\pgfsetroundjoin%
\definecolor{currentfill}{rgb}{0.274128,0.199721,0.498911}%
\pgfsetfillcolor{currentfill}%
\pgfsetfillopacity{0.800000}%
\pgfsetlinewidth{0.000000pt}%
\definecolor{currentstroke}{rgb}{0.000000,0.000000,0.000000}%
\pgfsetstrokecolor{currentstroke}%
\pgfsetdash{}{0pt}%
\pgfpathmoveto{\pgfqpoint{2.613640in}{2.166707in}}%
\pgfpathlineto{\pgfqpoint{2.627495in}{2.149657in}}%
\pgfpathlineto{\pgfqpoint{2.641342in}{2.132886in}}%
\pgfpathlineto{\pgfqpoint{2.655180in}{2.116390in}}%
\pgfpathlineto{\pgfqpoint{2.669012in}{2.100168in}}%
\pgfpathlineto{\pgfqpoint{2.677668in}{2.102538in}}%
\pgfpathlineto{\pgfqpoint{2.686310in}{2.105128in}}%
\pgfpathlineto{\pgfqpoint{2.694939in}{2.107932in}}%
\pgfpathlineto{\pgfqpoint{2.703553in}{2.110947in}}%
\pgfpathlineto{\pgfqpoint{2.689760in}{2.126685in}}%
\pgfpathlineto{\pgfqpoint{2.675961in}{2.142695in}}%
\pgfpathlineto{\pgfqpoint{2.662153in}{2.158981in}}%
\pgfpathlineto{\pgfqpoint{2.648338in}{2.175543in}}%
\pgfpathlineto{\pgfqpoint{2.639686in}{2.173001in}}%
\pgfpathlineto{\pgfqpoint{2.631019in}{2.170678in}}%
\pgfpathlineto{\pgfqpoint{2.622337in}{2.168578in}}%
\pgfpathlineto{\pgfqpoint{2.613640in}{2.166707in}}%
\pgfpathclose%
\pgfusepath{fill}%
\end{pgfscope}%
\begin{pgfscope}%
\pgfpathrectangle{\pgfqpoint{1.150000in}{0.150000in}}{\pgfqpoint{5.700000in}{5.700000in}}%
\pgfusepath{clip}%
\pgfsetbuttcap%
\pgfsetroundjoin%
\definecolor{currentfill}{rgb}{0.279574,0.170599,0.479997}%
\pgfsetfillcolor{currentfill}%
\pgfsetfillopacity{0.800000}%
\pgfsetlinewidth{0.000000pt}%
\definecolor{currentstroke}{rgb}{0.000000,0.000000,0.000000}%
\pgfsetstrokecolor{currentstroke}%
\pgfsetdash{}{0pt}%
\pgfpathmoveto{\pgfqpoint{2.669012in}{2.100168in}}%
\pgfpathlineto{\pgfqpoint{2.682835in}{2.084216in}}%
\pgfpathlineto{\pgfqpoint{2.696652in}{2.068534in}}%
\pgfpathlineto{\pgfqpoint{2.710462in}{2.053119in}}%
\pgfpathlineto{\pgfqpoint{2.724265in}{2.037970in}}%
\pgfpathlineto{\pgfqpoint{2.732883in}{2.040836in}}%
\pgfpathlineto{\pgfqpoint{2.741488in}{2.043913in}}%
\pgfpathlineto{\pgfqpoint{2.750079in}{2.047196in}}%
\pgfpathlineto{\pgfqpoint{2.758657in}{2.050681in}}%
\pgfpathlineto{\pgfqpoint{2.744891in}{2.065349in}}%
\pgfpathlineto{\pgfqpoint{2.731118in}{2.080281in}}%
\pgfpathlineto{\pgfqpoint{2.717339in}{2.095480in}}%
\pgfpathlineto{\pgfqpoint{2.703553in}{2.110947in}}%
\pgfpathlineto{\pgfqpoint{2.694939in}{2.107932in}}%
\pgfpathlineto{\pgfqpoint{2.686310in}{2.105128in}}%
\pgfpathlineto{\pgfqpoint{2.677668in}{2.102538in}}%
\pgfpathlineto{\pgfqpoint{2.669012in}{2.100168in}}%
\pgfpathclose%
\pgfusepath{fill}%
\end{pgfscope}%
\begin{pgfscope}%
\pgfpathrectangle{\pgfqpoint{1.150000in}{0.150000in}}{\pgfqpoint{5.700000in}{5.700000in}}%
\pgfusepath{clip}%
\pgfsetbuttcap%
\pgfsetroundjoin%
\definecolor{currentfill}{rgb}{0.276022,0.044167,0.370164}%
\pgfsetfillcolor{currentfill}%
\pgfsetfillopacity{0.800000}%
\pgfsetlinewidth{0.000000pt}%
\definecolor{currentstroke}{rgb}{0.000000,0.000000,0.000000}%
\pgfsetstrokecolor{currentstroke}%
\pgfsetdash{}{0pt}%
\pgfpathmoveto{\pgfqpoint{3.547919in}{1.780926in}}%
\pgfpathlineto{\pgfqpoint{3.561594in}{1.778941in}}%
\pgfpathlineto{\pgfqpoint{3.575274in}{1.777156in}}%
\pgfpathlineto{\pgfqpoint{3.588959in}{1.775571in}}%
\pgfpathlineto{\pgfqpoint{3.602651in}{1.774186in}}%
\pgfpathlineto{\pgfqpoint{3.610788in}{1.784672in}}%
\pgfpathlineto{\pgfqpoint{3.618920in}{1.795189in}}%
\pgfpathlineto{\pgfqpoint{3.627046in}{1.805735in}}%
\pgfpathlineto{\pgfqpoint{3.635166in}{1.816307in}}%
\pgfpathlineto{\pgfqpoint{3.621486in}{1.817384in}}%
\pgfpathlineto{\pgfqpoint{3.607812in}{1.818661in}}%
\pgfpathlineto{\pgfqpoint{3.594144in}{1.820137in}}%
\pgfpathlineto{\pgfqpoint{3.580482in}{1.821815in}}%
\pgfpathlineto{\pgfqpoint{3.572350in}{1.811540in}}%
\pgfpathlineto{\pgfqpoint{3.564212in}{1.801298in}}%
\pgfpathlineto{\pgfqpoint{3.556069in}{1.791092in}}%
\pgfpathlineto{\pgfqpoint{3.547919in}{1.780926in}}%
\pgfpathclose%
\pgfusepath{fill}%
\end{pgfscope}%
\begin{pgfscope}%
\pgfpathrectangle{\pgfqpoint{1.150000in}{0.150000in}}{\pgfqpoint{5.700000in}{5.700000in}}%
\pgfusepath{clip}%
\pgfsetbuttcap%
\pgfsetroundjoin%
\definecolor{currentfill}{rgb}{0.276022,0.044167,0.370164}%
\pgfsetfillcolor{currentfill}%
\pgfsetfillopacity{0.800000}%
\pgfsetlinewidth{0.000000pt}%
\definecolor{currentstroke}{rgb}{0.000000,0.000000,0.000000}%
\pgfsetstrokecolor{currentstroke}%
\pgfsetdash{}{0pt}%
\pgfpathmoveto{\pgfqpoint{3.033067in}{1.810064in}}%
\pgfpathlineto{\pgfqpoint{3.046760in}{1.800545in}}%
\pgfpathlineto{\pgfqpoint{3.060450in}{1.791255in}}%
\pgfpathlineto{\pgfqpoint{3.074141in}{1.782192in}}%
\pgfpathlineto{\pgfqpoint{3.087830in}{1.773355in}}%
\pgfpathlineto{\pgfqpoint{3.096213in}{1.779764in}}%
\pgfpathlineto{\pgfqpoint{3.104586in}{1.786315in}}%
\pgfpathlineto{\pgfqpoint{3.112950in}{1.793006in}}%
\pgfpathlineto{\pgfqpoint{3.121305in}{1.799831in}}%
\pgfpathlineto{\pgfqpoint{3.107641in}{1.808230in}}%
\pgfpathlineto{\pgfqpoint{3.093976in}{1.816855in}}%
\pgfpathlineto{\pgfqpoint{3.080312in}{1.825707in}}%
\pgfpathlineto{\pgfqpoint{3.066646in}{1.834787in}}%
\pgfpathlineto{\pgfqpoint{3.058266in}{1.828387in}}%
\pgfpathlineto{\pgfqpoint{3.049876in}{1.822131in}}%
\pgfpathlineto{\pgfqpoint{3.041477in}{1.816022in}}%
\pgfpathlineto{\pgfqpoint{3.033067in}{1.810064in}}%
\pgfpathclose%
\pgfusepath{fill}%
\end{pgfscope}%
\begin{pgfscope}%
\pgfpathrectangle{\pgfqpoint{1.150000in}{0.150000in}}{\pgfqpoint{5.700000in}{5.700000in}}%
\pgfusepath{clip}%
\pgfsetbuttcap%
\pgfsetroundjoin%
\definecolor{currentfill}{rgb}{0.266580,0.228262,0.514349}%
\pgfsetfillcolor{currentfill}%
\pgfsetfillopacity{0.800000}%
\pgfsetlinewidth{0.000000pt}%
\definecolor{currentstroke}{rgb}{0.000000,0.000000,0.000000}%
\pgfsetstrokecolor{currentstroke}%
\pgfsetdash{}{0pt}%
\pgfpathmoveto{\pgfqpoint{2.558134in}{2.237727in}}%
\pgfpathlineto{\pgfqpoint{2.572024in}{2.219544in}}%
\pgfpathlineto{\pgfqpoint{2.585905in}{2.201647in}}%
\pgfpathlineto{\pgfqpoint{2.599777in}{2.184036in}}%
\pgfpathlineto{\pgfqpoint{2.613640in}{2.166707in}}%
\pgfpathlineto{\pgfqpoint{2.622337in}{2.168578in}}%
\pgfpathlineto{\pgfqpoint{2.631019in}{2.170678in}}%
\pgfpathlineto{\pgfqpoint{2.639686in}{2.173001in}}%
\pgfpathlineto{\pgfqpoint{2.648338in}{2.175543in}}%
\pgfpathlineto{\pgfqpoint{2.634515in}{2.192385in}}%
\pgfpathlineto{\pgfqpoint{2.620684in}{2.209508in}}%
\pgfpathlineto{\pgfqpoint{2.606844in}{2.226914in}}%
\pgfpathlineto{\pgfqpoint{2.592996in}{2.244608in}}%
\pgfpathlineto{\pgfqpoint{2.584303in}{2.242542in}}%
\pgfpathlineto{\pgfqpoint{2.575595in}{2.240703in}}%
\pgfpathlineto{\pgfqpoint{2.566872in}{2.239097in}}%
\pgfpathlineto{\pgfqpoint{2.558134in}{2.237727in}}%
\pgfpathclose%
\pgfusepath{fill}%
\end{pgfscope}%
\begin{pgfscope}%
\pgfpathrectangle{\pgfqpoint{1.150000in}{0.150000in}}{\pgfqpoint{5.700000in}{5.700000in}}%
\pgfusepath{clip}%
\pgfsetbuttcap%
\pgfsetroundjoin%
\definecolor{currentfill}{rgb}{0.282290,0.145912,0.461510}%
\pgfsetfillcolor{currentfill}%
\pgfsetfillopacity{0.800000}%
\pgfsetlinewidth{0.000000pt}%
\definecolor{currentstroke}{rgb}{0.000000,0.000000,0.000000}%
\pgfsetstrokecolor{currentstroke}%
\pgfsetdash{}{0pt}%
\pgfpathmoveto{\pgfqpoint{2.724265in}{2.037970in}}%
\pgfpathlineto{\pgfqpoint{2.738062in}{2.023083in}}%
\pgfpathlineto{\pgfqpoint{2.751853in}{2.008457in}}%
\pgfpathlineto{\pgfqpoint{2.765638in}{1.994091in}}%
\pgfpathlineto{\pgfqpoint{2.779418in}{1.979982in}}%
\pgfpathlineto{\pgfqpoint{2.787999in}{1.983341in}}%
\pgfpathlineto{\pgfqpoint{2.796568in}{1.986902in}}%
\pgfpathlineto{\pgfqpoint{2.805124in}{1.990661in}}%
\pgfpathlineto{\pgfqpoint{2.813667in}{1.994614in}}%
\pgfpathlineto{\pgfqpoint{2.799922in}{2.008244in}}%
\pgfpathlineto{\pgfqpoint{2.786173in}{2.022130in}}%
\pgfpathlineto{\pgfqpoint{2.772418in}{2.036275in}}%
\pgfpathlineto{\pgfqpoint{2.758657in}{2.050681in}}%
\pgfpathlineto{\pgfqpoint{2.750079in}{2.047196in}}%
\pgfpathlineto{\pgfqpoint{2.741488in}{2.043913in}}%
\pgfpathlineto{\pgfqpoint{2.732883in}{2.040836in}}%
\pgfpathlineto{\pgfqpoint{2.724265in}{2.037970in}}%
\pgfpathclose%
\pgfusepath{fill}%
\end{pgfscope}%
\begin{pgfscope}%
\pgfpathrectangle{\pgfqpoint{1.150000in}{0.150000in}}{\pgfqpoint{5.700000in}{5.700000in}}%
\pgfusepath{clip}%
\pgfsetbuttcap%
\pgfsetroundjoin%
\definecolor{currentfill}{rgb}{0.281412,0.155834,0.469201}%
\pgfsetfillcolor{currentfill}%
\pgfsetfillopacity{0.800000}%
\pgfsetlinewidth{0.000000pt}%
\definecolor{currentstroke}{rgb}{0.000000,0.000000,0.000000}%
\pgfsetstrokecolor{currentstroke}%
\pgfsetdash{}{0pt}%
\pgfpathmoveto{\pgfqpoint{3.983681in}{2.011158in}}%
\pgfpathlineto{\pgfqpoint{3.997464in}{2.014306in}}%
\pgfpathlineto{\pgfqpoint{4.011257in}{2.017645in}}%
\pgfpathlineto{\pgfqpoint{4.025059in}{2.021174in}}%
\pgfpathlineto{\pgfqpoint{4.038872in}{2.024893in}}%
\pgfpathlineto{\pgfqpoint{4.046868in}{2.036379in}}%
\pgfpathlineto{\pgfqpoint{4.054858in}{2.047816in}}%
\pgfpathlineto{\pgfqpoint{4.062844in}{2.059204in}}%
\pgfpathlineto{\pgfqpoint{4.070825in}{2.070540in}}%
\pgfpathlineto{\pgfqpoint{4.057017in}{2.066670in}}%
\pgfpathlineto{\pgfqpoint{4.043220in}{2.062991in}}%
\pgfpathlineto{\pgfqpoint{4.029432in}{2.059501in}}%
\pgfpathlineto{\pgfqpoint{4.015655in}{2.056203in}}%
\pgfpathlineto{\pgfqpoint{4.007669in}{2.045004in}}%
\pgfpathlineto{\pgfqpoint{3.999678in}{2.033764in}}%
\pgfpathlineto{\pgfqpoint{3.991682in}{2.022481in}}%
\pgfpathlineto{\pgfqpoint{3.983681in}{2.011158in}}%
\pgfpathclose%
\pgfusepath{fill}%
\end{pgfscope}%
\begin{pgfscope}%
\pgfpathrectangle{\pgfqpoint{1.150000in}{0.150000in}}{\pgfqpoint{5.700000in}{5.700000in}}%
\pgfusepath{clip}%
\pgfsetbuttcap%
\pgfsetroundjoin%
\definecolor{currentfill}{rgb}{0.121831,0.589055,0.545623}%
\pgfsetfillcolor{currentfill}%
\pgfsetfillopacity{0.800000}%
\pgfsetlinewidth{0.000000pt}%
\definecolor{currentstroke}{rgb}{0.000000,0.000000,0.000000}%
\pgfsetstrokecolor{currentstroke}%
\pgfsetdash{}{0pt}%
\pgfpathmoveto{\pgfqpoint{5.505600in}{3.200296in}}%
\pgfpathlineto{\pgfqpoint{5.520156in}{3.211942in}}%
\pgfpathlineto{\pgfqpoint{5.534732in}{3.223767in}}%
\pgfpathlineto{\pgfqpoint{5.549328in}{3.235770in}}%
\pgfpathlineto{\pgfqpoint{5.563943in}{3.247952in}}%
\pgfpathlineto{\pgfqpoint{5.571247in}{3.251007in}}%
\pgfpathlineto{\pgfqpoint{5.578544in}{3.254061in}}%
\pgfpathlineto{\pgfqpoint{5.585835in}{3.257119in}}%
\pgfpathlineto{\pgfqpoint{5.593119in}{3.260189in}}%
\pgfpathlineto{\pgfqpoint{5.578529in}{3.248527in}}%
\pgfpathlineto{\pgfqpoint{5.563959in}{3.237043in}}%
\pgfpathlineto{\pgfqpoint{5.549408in}{3.225736in}}%
\pgfpathlineto{\pgfqpoint{5.534877in}{3.214607in}}%
\pgfpathlineto{\pgfqpoint{5.527567in}{3.211007in}}%
\pgfpathlineto{\pgfqpoint{5.520251in}{3.207426in}}%
\pgfpathlineto{\pgfqpoint{5.512928in}{3.203857in}}%
\pgfpathlineto{\pgfqpoint{5.505600in}{3.200296in}}%
\pgfpathclose%
\pgfusepath{fill}%
\end{pgfscope}%
\begin{pgfscope}%
\pgfpathrectangle{\pgfqpoint{1.150000in}{0.150000in}}{\pgfqpoint{5.700000in}{5.700000in}}%
\pgfusepath{clip}%
\pgfsetbuttcap%
\pgfsetroundjoin%
\definecolor{currentfill}{rgb}{0.204903,0.375746,0.553533}%
\pgfsetfillcolor{currentfill}%
\pgfsetfillopacity{0.800000}%
\pgfsetlinewidth{0.000000pt}%
\definecolor{currentstroke}{rgb}{0.000000,0.000000,0.000000}%
\pgfsetstrokecolor{currentstroke}%
\pgfsetdash{}{0pt}%
\pgfpathmoveto{\pgfqpoint{4.657448in}{2.554742in}}%
\pgfpathlineto{\pgfqpoint{4.671535in}{2.563314in}}%
\pgfpathlineto{\pgfqpoint{4.685637in}{2.572071in}}%
\pgfpathlineto{\pgfqpoint{4.699754in}{2.581011in}}%
\pgfpathlineto{\pgfqpoint{4.713886in}{2.590135in}}%
\pgfpathlineto{\pgfqpoint{4.721644in}{2.598849in}}%
\pgfpathlineto{\pgfqpoint{4.729395in}{2.607469in}}%
\pgfpathlineto{\pgfqpoint{4.737139in}{2.615998in}}%
\pgfpathlineto{\pgfqpoint{4.744877in}{2.624437in}}%
\pgfpathlineto{\pgfqpoint{4.730752in}{2.615458in}}%
\pgfpathlineto{\pgfqpoint{4.716642in}{2.606661in}}%
\pgfpathlineto{\pgfqpoint{4.702548in}{2.598049in}}%
\pgfpathlineto{\pgfqpoint{4.688469in}{2.589620in}}%
\pgfpathlineto{\pgfqpoint{4.680723in}{2.581025in}}%
\pgfpathlineto{\pgfqpoint{4.672971in}{2.572348in}}%
\pgfpathlineto{\pgfqpoint{4.665213in}{2.563588in}}%
\pgfpathlineto{\pgfqpoint{4.657448in}{2.554742in}}%
\pgfpathclose%
\pgfusepath{fill}%
\end{pgfscope}%
\begin{pgfscope}%
\pgfpathrectangle{\pgfqpoint{1.150000in}{0.150000in}}{\pgfqpoint{5.700000in}{5.700000in}}%
\pgfusepath{clip}%
\pgfsetbuttcap%
\pgfsetroundjoin%
\definecolor{currentfill}{rgb}{0.255645,0.260703,0.528312}%
\pgfsetfillcolor{currentfill}%
\pgfsetfillopacity{0.800000}%
\pgfsetlinewidth{0.000000pt}%
\definecolor{currentstroke}{rgb}{0.000000,0.000000,0.000000}%
\pgfsetstrokecolor{currentstroke}%
\pgfsetdash{}{0pt}%
\pgfpathmoveto{\pgfqpoint{2.502474in}{2.313381in}}%
\pgfpathlineto{\pgfqpoint{2.516404in}{2.294024in}}%
\pgfpathlineto{\pgfqpoint{2.530324in}{2.274965in}}%
\pgfpathlineto{\pgfqpoint{2.544234in}{2.256200in}}%
\pgfpathlineto{\pgfqpoint{2.558134in}{2.237727in}}%
\pgfpathlineto{\pgfqpoint{2.566872in}{2.239097in}}%
\pgfpathlineto{\pgfqpoint{2.575595in}{2.240703in}}%
\pgfpathlineto{\pgfqpoint{2.584303in}{2.242542in}}%
\pgfpathlineto{\pgfqpoint{2.592996in}{2.244608in}}%
\pgfpathlineto{\pgfqpoint{2.579138in}{2.262589in}}%
\pgfpathlineto{\pgfqpoint{2.565271in}{2.280862in}}%
\pgfpathlineto{\pgfqpoint{2.551394in}{2.299428in}}%
\pgfpathlineto{\pgfqpoint{2.537507in}{2.318291in}}%
\pgfpathlineto{\pgfqpoint{2.528772in}{2.316704in}}%
\pgfpathlineto{\pgfqpoint{2.520022in}{2.315354in}}%
\pgfpathlineto{\pgfqpoint{2.511256in}{2.314244in}}%
\pgfpathlineto{\pgfqpoint{2.502474in}{2.313381in}}%
\pgfpathclose%
\pgfusepath{fill}%
\end{pgfscope}%
\begin{pgfscope}%
\pgfpathrectangle{\pgfqpoint{1.150000in}{0.150000in}}{\pgfqpoint{5.700000in}{5.700000in}}%
\pgfusepath{clip}%
\pgfsetbuttcap%
\pgfsetroundjoin%
\definecolor{currentfill}{rgb}{0.283229,0.120777,0.440584}%
\pgfsetfillcolor{currentfill}%
\pgfsetfillopacity{0.800000}%
\pgfsetlinewidth{0.000000pt}%
\definecolor{currentstroke}{rgb}{0.000000,0.000000,0.000000}%
\pgfsetstrokecolor{currentstroke}%
\pgfsetdash{}{0pt}%
\pgfpathmoveto{\pgfqpoint{2.779418in}{1.979982in}}%
\pgfpathlineto{\pgfqpoint{2.793192in}{1.966128in}}%
\pgfpathlineto{\pgfqpoint{2.806961in}{1.952529in}}%
\pgfpathlineto{\pgfqpoint{2.820725in}{1.939181in}}%
\pgfpathlineto{\pgfqpoint{2.834485in}{1.926083in}}%
\pgfpathlineto{\pgfqpoint{2.843032in}{1.929932in}}%
\pgfpathlineto{\pgfqpoint{2.851566in}{1.933976in}}%
\pgfpathlineto{\pgfqpoint{2.860088in}{1.938209in}}%
\pgfpathlineto{\pgfqpoint{2.868598in}{1.942626in}}%
\pgfpathlineto{\pgfqpoint{2.854872in}{1.955247in}}%
\pgfpathlineto{\pgfqpoint{2.841141in}{1.968117in}}%
\pgfpathlineto{\pgfqpoint{2.827406in}{1.981239in}}%
\pgfpathlineto{\pgfqpoint{2.813667in}{1.994614in}}%
\pgfpathlineto{\pgfqpoint{2.805124in}{1.990661in}}%
\pgfpathlineto{\pgfqpoint{2.796568in}{1.986902in}}%
\pgfpathlineto{\pgfqpoint{2.787999in}{1.983341in}}%
\pgfpathlineto{\pgfqpoint{2.779418in}{1.979982in}}%
\pgfpathclose%
\pgfusepath{fill}%
\end{pgfscope}%
\begin{pgfscope}%
\pgfpathrectangle{\pgfqpoint{1.150000in}{0.150000in}}{\pgfqpoint{5.700000in}{5.700000in}}%
\pgfusepath{clip}%
\pgfsetbuttcap%
\pgfsetroundjoin%
\definecolor{currentfill}{rgb}{0.246811,0.283237,0.535941}%
\pgfsetfillcolor{currentfill}%
\pgfsetfillopacity{0.800000}%
\pgfsetlinewidth{0.000000pt}%
\definecolor{currentstroke}{rgb}{0.000000,0.000000,0.000000}%
\pgfsetstrokecolor{currentstroke}%
\pgfsetdash{}{0pt}%
\pgfpathmoveto{\pgfqpoint{4.364164in}{2.309002in}}%
\pgfpathlineto{\pgfqpoint{4.378105in}{2.315595in}}%
\pgfpathlineto{\pgfqpoint{4.392059in}{2.322375in}}%
\pgfpathlineto{\pgfqpoint{4.406027in}{2.329341in}}%
\pgfpathlineto{\pgfqpoint{4.420008in}{2.336493in}}%
\pgfpathlineto{\pgfqpoint{4.427881in}{2.346892in}}%
\pgfpathlineto{\pgfqpoint{4.435748in}{2.357204in}}%
\pgfpathlineto{\pgfqpoint{4.443610in}{2.367428in}}%
\pgfpathlineto{\pgfqpoint{4.451466in}{2.377566in}}%
\pgfpathlineto{\pgfqpoint{4.437490in}{2.370425in}}%
\pgfpathlineto{\pgfqpoint{4.423527in}{2.363471in}}%
\pgfpathlineto{\pgfqpoint{4.409578in}{2.356702in}}%
\pgfpathlineto{\pgfqpoint{4.395642in}{2.350119in}}%
\pgfpathlineto{\pgfqpoint{4.387781in}{2.339958in}}%
\pgfpathlineto{\pgfqpoint{4.379914in}{2.329719in}}%
\pgfpathlineto{\pgfqpoint{4.372042in}{2.319400in}}%
\pgfpathlineto{\pgfqpoint{4.364164in}{2.309002in}}%
\pgfpathclose%
\pgfusepath{fill}%
\end{pgfscope}%
\begin{pgfscope}%
\pgfpathrectangle{\pgfqpoint{1.150000in}{0.150000in}}{\pgfqpoint{5.700000in}{5.700000in}}%
\pgfusepath{clip}%
\pgfsetbuttcap%
\pgfsetroundjoin%
\definecolor{currentfill}{rgb}{0.272594,0.025563,0.353093}%
\pgfsetfillcolor{currentfill}%
\pgfsetfillopacity{0.800000}%
\pgfsetlinewidth{0.000000pt}%
\definecolor{currentstroke}{rgb}{0.000000,0.000000,0.000000}%
\pgfsetstrokecolor{currentstroke}%
\pgfsetdash{}{0pt}%
\pgfpathmoveto{\pgfqpoint{3.460557in}{1.752031in}}%
\pgfpathlineto{\pgfqpoint{3.474226in}{1.748896in}}%
\pgfpathlineto{\pgfqpoint{3.487900in}{1.745965in}}%
\pgfpathlineto{\pgfqpoint{3.501578in}{1.743236in}}%
\pgfpathlineto{\pgfqpoint{3.515261in}{1.740710in}}%
\pgfpathlineto{\pgfqpoint{3.523434in}{1.750690in}}%
\pgfpathlineto{\pgfqpoint{3.531602in}{1.760722in}}%
\pgfpathlineto{\pgfqpoint{3.539764in}{1.770802in}}%
\pgfpathlineto{\pgfqpoint{3.547919in}{1.780926in}}%
\pgfpathlineto{\pgfqpoint{3.534250in}{1.783113in}}%
\pgfpathlineto{\pgfqpoint{3.520585in}{1.785502in}}%
\pgfpathlineto{\pgfqpoint{3.506926in}{1.788094in}}%
\pgfpathlineto{\pgfqpoint{3.493271in}{1.790889in}}%
\pgfpathlineto{\pgfqpoint{3.485102in}{1.781092in}}%
\pgfpathlineto{\pgfqpoint{3.476927in}{1.771348in}}%
\pgfpathlineto{\pgfqpoint{3.468745in}{1.761660in}}%
\pgfpathlineto{\pgfqpoint{3.460557in}{1.752031in}}%
\pgfpathclose%
\pgfusepath{fill}%
\end{pgfscope}%
\begin{pgfscope}%
\pgfpathrectangle{\pgfqpoint{1.150000in}{0.150000in}}{\pgfqpoint{5.700000in}{5.700000in}}%
\pgfusepath{clip}%
\pgfsetbuttcap%
\pgfsetroundjoin%
\definecolor{currentfill}{rgb}{0.185556,0.418570,0.556753}%
\pgfsetfillcolor{currentfill}%
\pgfsetfillopacity{0.800000}%
\pgfsetlinewidth{0.000000pt}%
\definecolor{currentstroke}{rgb}{0.000000,0.000000,0.000000}%
\pgfsetstrokecolor{currentstroke}%
\pgfsetdash{}{0pt}%
\pgfpathmoveto{\pgfqpoint{2.257176in}{2.762086in}}%
\pgfpathlineto{\pgfqpoint{2.271332in}{2.736688in}}%
\pgfpathlineto{\pgfqpoint{2.285470in}{2.711648in}}%
\pgfpathlineto{\pgfqpoint{2.299593in}{2.686963in}}%
\pgfpathlineto{\pgfqpoint{2.313699in}{2.662629in}}%
\pgfpathlineto{\pgfqpoint{2.322601in}{2.662524in}}%
\pgfpathlineto{\pgfqpoint{2.331484in}{2.662682in}}%
\pgfpathlineto{\pgfqpoint{2.340350in}{2.663098in}}%
\pgfpathlineto{\pgfqpoint{2.349199in}{2.663768in}}%
\pgfpathlineto{\pgfqpoint{2.335141in}{2.687626in}}%
\pgfpathlineto{\pgfqpoint{2.321068in}{2.711833in}}%
\pgfpathlineto{\pgfqpoint{2.306980in}{2.736393in}}%
\pgfpathlineto{\pgfqpoint{2.292874in}{2.761311in}}%
\pgfpathlineto{\pgfqpoint{2.283977in}{2.761106in}}%
\pgfpathlineto{\pgfqpoint{2.275062in}{2.761163in}}%
\pgfpathlineto{\pgfqpoint{2.266129in}{2.761489in}}%
\pgfpathlineto{\pgfqpoint{2.257176in}{2.762086in}}%
\pgfpathclose%
\pgfusepath{fill}%
\end{pgfscope}%
\begin{pgfscope}%
\pgfpathrectangle{\pgfqpoint{1.150000in}{0.150000in}}{\pgfqpoint{5.700000in}{5.700000in}}%
\pgfusepath{clip}%
\pgfsetbuttcap%
\pgfsetroundjoin%
\definecolor{currentfill}{rgb}{0.160665,0.478540,0.558115}%
\pgfsetfillcolor{currentfill}%
\pgfsetfillopacity{0.800000}%
\pgfsetlinewidth{0.000000pt}%
\definecolor{currentstroke}{rgb}{0.000000,0.000000,0.000000}%
\pgfsetstrokecolor{currentstroke}%
\pgfsetdash{}{0pt}%
\pgfpathmoveto{\pgfqpoint{5.038047in}{2.860322in}}%
\pgfpathlineto{\pgfqpoint{5.052345in}{2.870760in}}%
\pgfpathlineto{\pgfqpoint{5.066659in}{2.881380in}}%
\pgfpathlineto{\pgfqpoint{5.080991in}{2.892182in}}%
\pgfpathlineto{\pgfqpoint{5.095341in}{2.903166in}}%
\pgfpathlineto{\pgfqpoint{5.102916in}{2.909274in}}%
\pgfpathlineto{\pgfqpoint{5.110484in}{2.915310in}}%
\pgfpathlineto{\pgfqpoint{5.118044in}{2.921278in}}%
\pgfpathlineto{\pgfqpoint{5.125597in}{2.927182in}}%
\pgfpathlineto{\pgfqpoint{5.111261in}{2.916513in}}%
\pgfpathlineto{\pgfqpoint{5.096943in}{2.906025in}}%
\pgfpathlineto{\pgfqpoint{5.082642in}{2.895718in}}%
\pgfpathlineto{\pgfqpoint{5.068358in}{2.885592in}}%
\pgfpathlineto{\pgfqpoint{5.060791in}{2.879362in}}%
\pgfpathlineto{\pgfqpoint{5.053217in}{2.873077in}}%
\pgfpathlineto{\pgfqpoint{5.045636in}{2.866731in}}%
\pgfpathlineto{\pgfqpoint{5.038047in}{2.860322in}}%
\pgfpathclose%
\pgfusepath{fill}%
\end{pgfscope}%
\begin{pgfscope}%
\pgfpathrectangle{\pgfqpoint{1.150000in}{0.150000in}}{\pgfqpoint{5.700000in}{5.700000in}}%
\pgfusepath{clip}%
\pgfsetbuttcap%
\pgfsetroundjoin%
\definecolor{currentfill}{rgb}{0.119423,0.611141,0.538982}%
\pgfsetfillcolor{currentfill}%
\pgfsetfillopacity{0.800000}%
\pgfsetlinewidth{0.000000pt}%
\definecolor{currentstroke}{rgb}{0.000000,0.000000,0.000000}%
\pgfsetstrokecolor{currentstroke}%
\pgfsetdash{}{0pt}%
\pgfpathmoveto{\pgfqpoint{5.593119in}{3.260189in}}%
\pgfpathlineto{\pgfqpoint{5.607729in}{3.272030in}}%
\pgfpathlineto{\pgfqpoint{5.622358in}{3.284048in}}%
\pgfpathlineto{\pgfqpoint{5.637008in}{3.296245in}}%
\pgfpathlineto{\pgfqpoint{5.651678in}{3.308621in}}%
\pgfpathlineto{\pgfqpoint{5.658929in}{3.311166in}}%
\pgfpathlineto{\pgfqpoint{5.666174in}{3.313725in}}%
\pgfpathlineto{\pgfqpoint{5.673413in}{3.316306in}}%
\pgfpathlineto{\pgfqpoint{5.680645in}{3.318913in}}%
\pgfpathlineto{\pgfqpoint{5.666003in}{3.307093in}}%
\pgfpathlineto{\pgfqpoint{5.651381in}{3.295449in}}%
\pgfpathlineto{\pgfqpoint{5.636779in}{3.283984in}}%
\pgfpathlineto{\pgfqpoint{5.622197in}{3.272695in}}%
\pgfpathlineto{\pgfqpoint{5.614936in}{3.269522in}}%
\pgfpathlineto{\pgfqpoint{5.607669in}{3.266385in}}%
\pgfpathlineto{\pgfqpoint{5.600397in}{3.263276in}}%
\pgfpathlineto{\pgfqpoint{5.593119in}{3.260189in}}%
\pgfpathclose%
\pgfusepath{fill}%
\end{pgfscope}%
\begin{pgfscope}%
\pgfpathrectangle{\pgfqpoint{1.150000in}{0.150000in}}{\pgfqpoint{5.700000in}{5.700000in}}%
\pgfusepath{clip}%
\pgfsetbuttcap%
\pgfsetroundjoin%
\definecolor{currentfill}{rgb}{0.269944,0.014625,0.341379}%
\pgfsetfillcolor{currentfill}%
\pgfsetfillopacity{0.800000}%
\pgfsetlinewidth{0.000000pt}%
\definecolor{currentstroke}{rgb}{0.000000,0.000000,0.000000}%
\pgfsetstrokecolor{currentstroke}%
\pgfsetdash{}{0pt}%
\pgfpathmoveto{\pgfqpoint{3.230628in}{1.740623in}}%
\pgfpathlineto{\pgfqpoint{3.244298in}{1.734202in}}%
\pgfpathlineto{\pgfqpoint{3.257969in}{1.727997in}}%
\pgfpathlineto{\pgfqpoint{3.271642in}{1.722005in}}%
\pgfpathlineto{\pgfqpoint{3.285318in}{1.716225in}}%
\pgfpathlineto{\pgfqpoint{3.293597in}{1.724430in}}%
\pgfpathlineto{\pgfqpoint{3.301869in}{1.732738in}}%
\pgfpathlineto{\pgfqpoint{3.310133in}{1.741145in}}%
\pgfpathlineto{\pgfqpoint{3.318389in}{1.749647in}}%
\pgfpathlineto{\pgfqpoint{3.304734in}{1.755023in}}%
\pgfpathlineto{\pgfqpoint{3.291081in}{1.760612in}}%
\pgfpathlineto{\pgfqpoint{3.277429in}{1.766414in}}%
\pgfpathlineto{\pgfqpoint{3.263780in}{1.772430in}}%
\pgfpathlineto{\pgfqpoint{3.255504in}{1.764320in}}%
\pgfpathlineto{\pgfqpoint{3.247220in}{1.756312in}}%
\pgfpathlineto{\pgfqpoint{3.238928in}{1.748412in}}%
\pgfpathlineto{\pgfqpoint{3.230628in}{1.740623in}}%
\pgfpathclose%
\pgfusepath{fill}%
\end{pgfscope}%
\begin{pgfscope}%
\pgfpathrectangle{\pgfqpoint{1.150000in}{0.150000in}}{\pgfqpoint{5.700000in}{5.700000in}}%
\pgfusepath{clip}%
\pgfsetbuttcap%
\pgfsetroundjoin%
\definecolor{currentfill}{rgb}{0.277134,0.185228,0.489898}%
\pgfsetfillcolor{currentfill}%
\pgfsetfillopacity{0.800000}%
\pgfsetlinewidth{0.000000pt}%
\definecolor{currentstroke}{rgb}{0.000000,0.000000,0.000000}%
\pgfsetstrokecolor{currentstroke}%
\pgfsetdash{}{0pt}%
\pgfpathmoveto{\pgfqpoint{4.070825in}{2.070540in}}%
\pgfpathlineto{\pgfqpoint{4.084643in}{2.074599in}}%
\pgfpathlineto{\pgfqpoint{4.098472in}{2.078848in}}%
\pgfpathlineto{\pgfqpoint{4.112312in}{2.083285in}}%
\pgfpathlineto{\pgfqpoint{4.126163in}{2.087912in}}%
\pgfpathlineto{\pgfqpoint{4.134134in}{2.099327in}}%
\pgfpathlineto{\pgfqpoint{4.142100in}{2.110682in}}%
\pgfpathlineto{\pgfqpoint{4.150060in}{2.121975in}}%
\pgfpathlineto{\pgfqpoint{4.158016in}{2.133207in}}%
\pgfpathlineto{\pgfqpoint{4.144170in}{2.128462in}}%
\pgfpathlineto{\pgfqpoint{4.130335in}{2.123906in}}%
\pgfpathlineto{\pgfqpoint{4.116511in}{2.119538in}}%
\pgfpathlineto{\pgfqpoint{4.102698in}{2.115361in}}%
\pgfpathlineto{\pgfqpoint{4.094737in}{2.104236in}}%
\pgfpathlineto{\pgfqpoint{4.086771in}{2.093057in}}%
\pgfpathlineto{\pgfqpoint{4.078801in}{2.081825in}}%
\pgfpathlineto{\pgfqpoint{4.070825in}{2.070540in}}%
\pgfpathclose%
\pgfusepath{fill}%
\end{pgfscope}%
\begin{pgfscope}%
\pgfpathrectangle{\pgfqpoint{1.150000in}{0.150000in}}{\pgfqpoint{5.700000in}{5.700000in}}%
\pgfusepath{clip}%
\pgfsetbuttcap%
\pgfsetroundjoin%
\definecolor{currentfill}{rgb}{0.243113,0.292092,0.538516}%
\pgfsetfillcolor{currentfill}%
\pgfsetfillopacity{0.800000}%
\pgfsetlinewidth{0.000000pt}%
\definecolor{currentstroke}{rgb}{0.000000,0.000000,0.000000}%
\pgfsetstrokecolor{currentstroke}%
\pgfsetdash{}{0pt}%
\pgfpathmoveto{\pgfqpoint{2.446642in}{2.393830in}}%
\pgfpathlineto{\pgfqpoint{2.460617in}{2.373259in}}%
\pgfpathlineto{\pgfqpoint{2.474581in}{2.352995in}}%
\pgfpathlineto{\pgfqpoint{2.488533in}{2.333037in}}%
\pgfpathlineto{\pgfqpoint{2.502474in}{2.313381in}}%
\pgfpathlineto{\pgfqpoint{2.511256in}{2.314244in}}%
\pgfpathlineto{\pgfqpoint{2.520022in}{2.315354in}}%
\pgfpathlineto{\pgfqpoint{2.528772in}{2.316704in}}%
\pgfpathlineto{\pgfqpoint{2.537507in}{2.318291in}}%
\pgfpathlineto{\pgfqpoint{2.523610in}{2.337452in}}%
\pgfpathlineto{\pgfqpoint{2.509702in}{2.356914in}}%
\pgfpathlineto{\pgfqpoint{2.495783in}{2.376681in}}%
\pgfpathlineto{\pgfqpoint{2.481853in}{2.396754in}}%
\pgfpathlineto{\pgfqpoint{2.473075in}{2.395651in}}%
\pgfpathlineto{\pgfqpoint{2.464281in}{2.394792in}}%
\pgfpathlineto{\pgfqpoint{2.455470in}{2.394184in}}%
\pgfpathlineto{\pgfqpoint{2.446642in}{2.393830in}}%
\pgfpathclose%
\pgfusepath{fill}%
\end{pgfscope}%
\begin{pgfscope}%
\pgfpathrectangle{\pgfqpoint{1.150000in}{0.150000in}}{\pgfqpoint{5.700000in}{5.700000in}}%
\pgfusepath{clip}%
\pgfsetbuttcap%
\pgfsetroundjoin%
\definecolor{currentfill}{rgb}{0.282656,0.100196,0.422160}%
\pgfsetfillcolor{currentfill}%
\pgfsetfillopacity{0.800000}%
\pgfsetlinewidth{0.000000pt}%
\definecolor{currentstroke}{rgb}{0.000000,0.000000,0.000000}%
\pgfsetstrokecolor{currentstroke}%
\pgfsetdash{}{0pt}%
\pgfpathmoveto{\pgfqpoint{2.834485in}{1.926083in}}%
\pgfpathlineto{\pgfqpoint{2.848240in}{1.913234in}}%
\pgfpathlineto{\pgfqpoint{2.861991in}{1.900631in}}%
\pgfpathlineto{\pgfqpoint{2.875739in}{1.888274in}}%
\pgfpathlineto{\pgfqpoint{2.889482in}{1.876160in}}%
\pgfpathlineto{\pgfqpoint{2.897996in}{1.880498in}}%
\pgfpathlineto{\pgfqpoint{2.906497in}{1.885021in}}%
\pgfpathlineto{\pgfqpoint{2.914987in}{1.889726in}}%
\pgfpathlineto{\pgfqpoint{2.923465in}{1.894607in}}%
\pgfpathlineto{\pgfqpoint{2.909753in}{1.906245in}}%
\pgfpathlineto{\pgfqpoint{2.896038in}{1.918127in}}%
\pgfpathlineto{\pgfqpoint{2.882320in}{1.930254in}}%
\pgfpathlineto{\pgfqpoint{2.868598in}{1.942626in}}%
\pgfpathlineto{\pgfqpoint{2.860088in}{1.938209in}}%
\pgfpathlineto{\pgfqpoint{2.851566in}{1.933976in}}%
\pgfpathlineto{\pgfqpoint{2.843032in}{1.929932in}}%
\pgfpathlineto{\pgfqpoint{2.834485in}{1.926083in}}%
\pgfpathclose%
\pgfusepath{fill}%
\end{pgfscope}%
\begin{pgfscope}%
\pgfpathrectangle{\pgfqpoint{1.150000in}{0.150000in}}{\pgfqpoint{5.700000in}{5.700000in}}%
\pgfusepath{clip}%
\pgfsetbuttcap%
\pgfsetroundjoin%
\definecolor{currentfill}{rgb}{0.273809,0.031497,0.358853}%
\pgfsetfillcolor{currentfill}%
\pgfsetfillopacity{0.800000}%
\pgfsetlinewidth{0.000000pt}%
\definecolor{currentstroke}{rgb}{0.000000,0.000000,0.000000}%
\pgfsetstrokecolor{currentstroke}%
\pgfsetdash{}{0pt}%
\pgfpathmoveto{\pgfqpoint{3.087830in}{1.773355in}}%
\pgfpathlineto{\pgfqpoint{3.101519in}{1.764743in}}%
\pgfpathlineto{\pgfqpoint{3.115207in}{1.756355in}}%
\pgfpathlineto{\pgfqpoint{3.128896in}{1.748189in}}%
\pgfpathlineto{\pgfqpoint{3.142584in}{1.740244in}}%
\pgfpathlineto{\pgfqpoint{3.150942in}{1.747102in}}%
\pgfpathlineto{\pgfqpoint{3.159291in}{1.754094in}}%
\pgfpathlineto{\pgfqpoint{3.167630in}{1.761218in}}%
\pgfpathlineto{\pgfqpoint{3.175961in}{1.768468in}}%
\pgfpathlineto{\pgfqpoint{3.162297in}{1.775976in}}%
\pgfpathlineto{\pgfqpoint{3.148633in}{1.783705in}}%
\pgfpathlineto{\pgfqpoint{3.134969in}{1.791657in}}%
\pgfpathlineto{\pgfqpoint{3.121305in}{1.799831in}}%
\pgfpathlineto{\pgfqpoint{3.112950in}{1.793006in}}%
\pgfpathlineto{\pgfqpoint{3.104586in}{1.786315in}}%
\pgfpathlineto{\pgfqpoint{3.096213in}{1.779764in}}%
\pgfpathlineto{\pgfqpoint{3.087830in}{1.773355in}}%
\pgfpathclose%
\pgfusepath{fill}%
\end{pgfscope}%
\begin{pgfscope}%
\pgfpathrectangle{\pgfqpoint{1.150000in}{0.150000in}}{\pgfqpoint{5.700000in}{5.700000in}}%
\pgfusepath{clip}%
\pgfsetbuttcap%
\pgfsetroundjoin%
\definecolor{currentfill}{rgb}{0.192357,0.403199,0.555836}%
\pgfsetfillcolor{currentfill}%
\pgfsetfillopacity{0.800000}%
\pgfsetlinewidth{0.000000pt}%
\definecolor{currentstroke}{rgb}{0.000000,0.000000,0.000000}%
\pgfsetstrokecolor{currentstroke}%
\pgfsetdash{}{0pt}%
\pgfpathmoveto{\pgfqpoint{4.744877in}{2.624437in}}%
\pgfpathlineto{\pgfqpoint{4.759018in}{2.633600in}}%
\pgfpathlineto{\pgfqpoint{4.773174in}{2.642947in}}%
\pgfpathlineto{\pgfqpoint{4.787346in}{2.652477in}}%
\pgfpathlineto{\pgfqpoint{4.801534in}{2.662190in}}%
\pgfpathlineto{\pgfqpoint{4.809257in}{2.670376in}}%
\pgfpathlineto{\pgfqpoint{4.816973in}{2.678469in}}%
\pgfpathlineto{\pgfqpoint{4.824682in}{2.686470in}}%
\pgfpathlineto{\pgfqpoint{4.832385in}{2.694382in}}%
\pgfpathlineto{\pgfqpoint{4.818205in}{2.684847in}}%
\pgfpathlineto{\pgfqpoint{4.804041in}{2.675495in}}%
\pgfpathlineto{\pgfqpoint{4.789893in}{2.666326in}}%
\pgfpathlineto{\pgfqpoint{4.775761in}{2.657340in}}%
\pgfpathlineto{\pgfqpoint{4.768050in}{2.649238in}}%
\pgfpathlineto{\pgfqpoint{4.760332in}{2.641055in}}%
\pgfpathlineto{\pgfqpoint{4.752608in}{2.632789in}}%
\pgfpathlineto{\pgfqpoint{4.744877in}{2.624437in}}%
\pgfpathclose%
\pgfusepath{fill}%
\end{pgfscope}%
\begin{pgfscope}%
\pgfpathrectangle{\pgfqpoint{1.150000in}{0.150000in}}{\pgfqpoint{5.700000in}{5.700000in}}%
\pgfusepath{clip}%
\pgfsetbuttcap%
\pgfsetroundjoin%
\definecolor{currentfill}{rgb}{0.121380,0.629492,0.531973}%
\pgfsetfillcolor{currentfill}%
\pgfsetfillopacity{0.800000}%
\pgfsetlinewidth{0.000000pt}%
\definecolor{currentstroke}{rgb}{0.000000,0.000000,0.000000}%
\pgfsetstrokecolor{currentstroke}%
\pgfsetdash{}{0pt}%
\pgfpathmoveto{\pgfqpoint{5.680645in}{3.318913in}}%
\pgfpathlineto{\pgfqpoint{5.695308in}{3.330912in}}%
\pgfpathlineto{\pgfqpoint{5.709990in}{3.343088in}}%
\pgfpathlineto{\pgfqpoint{5.724693in}{3.355442in}}%
\pgfpathlineto{\pgfqpoint{5.739417in}{3.367975in}}%
\pgfpathlineto{\pgfqpoint{5.746614in}{3.370039in}}%
\pgfpathlineto{\pgfqpoint{5.753806in}{3.372135in}}%
\pgfpathlineto{\pgfqpoint{5.760992in}{3.374270in}}%
\pgfpathlineto{\pgfqpoint{5.768173in}{3.376449in}}%
\pgfpathlineto{\pgfqpoint{5.753480in}{3.364506in}}%
\pgfpathlineto{\pgfqpoint{5.738807in}{3.352741in}}%
\pgfpathlineto{\pgfqpoint{5.724154in}{3.341152in}}%
\pgfpathlineto{\pgfqpoint{5.709522in}{3.329739in}}%
\pgfpathlineto{\pgfqpoint{5.702310in}{3.326961in}}%
\pgfpathlineto{\pgfqpoint{5.695094in}{3.324235in}}%
\pgfpathlineto{\pgfqpoint{5.687872in}{3.321554in}}%
\pgfpathlineto{\pgfqpoint{5.680645in}{3.318913in}}%
\pgfpathclose%
\pgfusepath{fill}%
\end{pgfscope}%
\begin{pgfscope}%
\pgfpathrectangle{\pgfqpoint{1.150000in}{0.150000in}}{\pgfqpoint{5.700000in}{5.700000in}}%
\pgfusepath{clip}%
\pgfsetbuttcap%
\pgfsetroundjoin%
\definecolor{currentfill}{rgb}{0.271305,0.019942,0.347269}%
\pgfsetfillcolor{currentfill}%
\pgfsetfillopacity{0.800000}%
\pgfsetlinewidth{0.000000pt}%
\definecolor{currentstroke}{rgb}{0.000000,0.000000,0.000000}%
\pgfsetstrokecolor{currentstroke}%
\pgfsetdash{}{0pt}%
\pgfpathmoveto{\pgfqpoint{3.373038in}{1.730246in}}%
\pgfpathlineto{\pgfqpoint{3.386708in}{1.725917in}}%
\pgfpathlineto{\pgfqpoint{3.400382in}{1.721796in}}%
\pgfpathlineto{\pgfqpoint{3.414058in}{1.717880in}}%
\pgfpathlineto{\pgfqpoint{3.427739in}{1.714169in}}%
\pgfpathlineto{\pgfqpoint{3.435954in}{1.723530in}}%
\pgfpathlineto{\pgfqpoint{3.444161in}{1.732963in}}%
\pgfpathlineto{\pgfqpoint{3.452362in}{1.742464in}}%
\pgfpathlineto{\pgfqpoint{3.460557in}{1.752031in}}%
\pgfpathlineto{\pgfqpoint{3.446892in}{1.755370in}}%
\pgfpathlineto{\pgfqpoint{3.433231in}{1.758915in}}%
\pgfpathlineto{\pgfqpoint{3.419574in}{1.762666in}}%
\pgfpathlineto{\pgfqpoint{3.405921in}{1.766623in}}%
\pgfpathlineto{\pgfqpoint{3.397711in}{1.757415in}}%
\pgfpathlineto{\pgfqpoint{3.389494in}{1.748281in}}%
\pgfpathlineto{\pgfqpoint{3.381270in}{1.739223in}}%
\pgfpathlineto{\pgfqpoint{3.373038in}{1.730246in}}%
\pgfpathclose%
\pgfusepath{fill}%
\end{pgfscope}%
\begin{pgfscope}%
\pgfpathrectangle{\pgfqpoint{1.150000in}{0.150000in}}{\pgfqpoint{5.700000in}{5.700000in}}%
\pgfusepath{clip}%
\pgfsetbuttcap%
\pgfsetroundjoin%
\definecolor{currentfill}{rgb}{0.150476,0.504369,0.557430}%
\pgfsetfillcolor{currentfill}%
\pgfsetfillopacity{0.800000}%
\pgfsetlinewidth{0.000000pt}%
\definecolor{currentstroke}{rgb}{0.000000,0.000000,0.000000}%
\pgfsetstrokecolor{currentstroke}%
\pgfsetdash{}{0pt}%
\pgfpathmoveto{\pgfqpoint{5.125597in}{2.927182in}}%
\pgfpathlineto{\pgfqpoint{5.139950in}{2.938032in}}%
\pgfpathlineto{\pgfqpoint{5.154322in}{2.949063in}}%
\pgfpathlineto{\pgfqpoint{5.168711in}{2.960276in}}%
\pgfpathlineto{\pgfqpoint{5.183119in}{2.971670in}}%
\pgfpathlineto{\pgfqpoint{5.190650in}{2.977177in}}%
\pgfpathlineto{\pgfqpoint{5.198173in}{2.982619in}}%
\pgfpathlineto{\pgfqpoint{5.205688in}{2.988001in}}%
\pgfpathlineto{\pgfqpoint{5.213196in}{2.993326in}}%
\pgfpathlineto{\pgfqpoint{5.198804in}{2.982282in}}%
\pgfpathlineto{\pgfqpoint{5.184430in}{2.971417in}}%
\pgfpathlineto{\pgfqpoint{5.170074in}{2.960734in}}%
\pgfpathlineto{\pgfqpoint{5.155736in}{2.950231in}}%
\pgfpathlineto{\pgfqpoint{5.148212in}{2.944546in}}%
\pgfpathlineto{\pgfqpoint{5.140681in}{2.938812in}}%
\pgfpathlineto{\pgfqpoint{5.133142in}{2.933025in}}%
\pgfpathlineto{\pgfqpoint{5.125597in}{2.927182in}}%
\pgfpathclose%
\pgfusepath{fill}%
\end{pgfscope}%
\begin{pgfscope}%
\pgfpathrectangle{\pgfqpoint{1.150000in}{0.150000in}}{\pgfqpoint{5.700000in}{5.700000in}}%
\pgfusepath{clip}%
\pgfsetbuttcap%
\pgfsetroundjoin%
\definecolor{currentfill}{rgb}{0.233603,0.313828,0.543914}%
\pgfsetfillcolor{currentfill}%
\pgfsetfillopacity{0.800000}%
\pgfsetlinewidth{0.000000pt}%
\definecolor{currentstroke}{rgb}{0.000000,0.000000,0.000000}%
\pgfsetstrokecolor{currentstroke}%
\pgfsetdash{}{0pt}%
\pgfpathmoveto{\pgfqpoint{4.451466in}{2.377566in}}%
\pgfpathlineto{\pgfqpoint{4.465456in}{2.384892in}}%
\pgfpathlineto{\pgfqpoint{4.479459in}{2.392404in}}%
\pgfpathlineto{\pgfqpoint{4.493476in}{2.400101in}}%
\pgfpathlineto{\pgfqpoint{4.507507in}{2.407983in}}%
\pgfpathlineto{\pgfqpoint{4.515353in}{2.418002in}}%
\pgfpathlineto{\pgfqpoint{4.523192in}{2.427929in}}%
\pgfpathlineto{\pgfqpoint{4.531025in}{2.437763in}}%
\pgfpathlineto{\pgfqpoint{4.538852in}{2.447505in}}%
\pgfpathlineto{\pgfqpoint{4.524826in}{2.439667in}}%
\pgfpathlineto{\pgfqpoint{4.510814in}{2.432015in}}%
\pgfpathlineto{\pgfqpoint{4.496816in}{2.424547in}}%
\pgfpathlineto{\pgfqpoint{4.482832in}{2.417265in}}%
\pgfpathlineto{\pgfqpoint{4.474999in}{2.407466in}}%
\pgfpathlineto{\pgfqpoint{4.467160in}{2.397584in}}%
\pgfpathlineto{\pgfqpoint{4.459316in}{2.387618in}}%
\pgfpathlineto{\pgfqpoint{4.451466in}{2.377566in}}%
\pgfpathclose%
\pgfusepath{fill}%
\end{pgfscope}%
\begin{pgfscope}%
\pgfpathrectangle{\pgfqpoint{1.150000in}{0.150000in}}{\pgfqpoint{5.700000in}{5.700000in}}%
\pgfusepath{clip}%
\pgfsetbuttcap%
\pgfsetroundjoin%
\definecolor{currentfill}{rgb}{0.270595,0.214069,0.507052}%
\pgfsetfillcolor{currentfill}%
\pgfsetfillopacity{0.800000}%
\pgfsetlinewidth{0.000000pt}%
\definecolor{currentstroke}{rgb}{0.000000,0.000000,0.000000}%
\pgfsetstrokecolor{currentstroke}%
\pgfsetdash{}{0pt}%
\pgfpathmoveto{\pgfqpoint{4.158016in}{2.133207in}}%
\pgfpathlineto{\pgfqpoint{4.171874in}{2.138140in}}%
\pgfpathlineto{\pgfqpoint{4.185743in}{2.143261in}}%
\pgfpathlineto{\pgfqpoint{4.199623in}{2.148571in}}%
\pgfpathlineto{\pgfqpoint{4.213515in}{2.154067in}}%
\pgfpathlineto{\pgfqpoint{4.221462in}{2.165334in}}%
\pgfpathlineto{\pgfqpoint{4.229403in}{2.176530in}}%
\pgfpathlineto{\pgfqpoint{4.237339in}{2.187654in}}%
\pgfpathlineto{\pgfqpoint{4.245270in}{2.198705in}}%
\pgfpathlineto{\pgfqpoint{4.231382in}{2.193122in}}%
\pgfpathlineto{\pgfqpoint{4.217506in}{2.187726in}}%
\pgfpathlineto{\pgfqpoint{4.203642in}{2.182518in}}%
\pgfpathlineto{\pgfqpoint{4.189789in}{2.177498in}}%
\pgfpathlineto{\pgfqpoint{4.181853in}{2.166521in}}%
\pgfpathlineto{\pgfqpoint{4.173913in}{2.155480in}}%
\pgfpathlineto{\pgfqpoint{4.165967in}{2.144375in}}%
\pgfpathlineto{\pgfqpoint{4.158016in}{2.133207in}}%
\pgfpathclose%
\pgfusepath{fill}%
\end{pgfscope}%
\begin{pgfscope}%
\pgfpathrectangle{\pgfqpoint{1.150000in}{0.150000in}}{\pgfqpoint{5.700000in}{5.700000in}}%
\pgfusepath{clip}%
\pgfsetbuttcap%
\pgfsetroundjoin%
\definecolor{currentfill}{rgb}{0.227802,0.326594,0.546532}%
\pgfsetfillcolor{currentfill}%
\pgfsetfillopacity{0.800000}%
\pgfsetlinewidth{0.000000pt}%
\definecolor{currentstroke}{rgb}{0.000000,0.000000,0.000000}%
\pgfsetstrokecolor{currentstroke}%
\pgfsetdash{}{0pt}%
\pgfpathmoveto{\pgfqpoint{2.390618in}{2.479250in}}%
\pgfpathlineto{\pgfqpoint{2.404643in}{2.457419in}}%
\pgfpathlineto{\pgfqpoint{2.418655in}{2.435907in}}%
\pgfpathlineto{\pgfqpoint{2.432655in}{2.414712in}}%
\pgfpathlineto{\pgfqpoint{2.446642in}{2.393830in}}%
\pgfpathlineto{\pgfqpoint{2.455470in}{2.394184in}}%
\pgfpathlineto{\pgfqpoint{2.464281in}{2.394792in}}%
\pgfpathlineto{\pgfqpoint{2.473075in}{2.395651in}}%
\pgfpathlineto{\pgfqpoint{2.481853in}{2.396754in}}%
\pgfpathlineto{\pgfqpoint{2.467912in}{2.417137in}}%
\pgfpathlineto{\pgfqpoint{2.453958in}{2.437832in}}%
\pgfpathlineto{\pgfqpoint{2.439993in}{2.458843in}}%
\pgfpathlineto{\pgfqpoint{2.426015in}{2.480171in}}%
\pgfpathlineto{\pgfqpoint{2.417191in}{2.479555in}}%
\pgfpathlineto{\pgfqpoint{2.408351in}{2.479193in}}%
\pgfpathlineto{\pgfqpoint{2.399493in}{2.479090in}}%
\pgfpathlineto{\pgfqpoint{2.390618in}{2.479250in}}%
\pgfpathclose%
\pgfusepath{fill}%
\end{pgfscope}%
\begin{pgfscope}%
\pgfpathrectangle{\pgfqpoint{1.150000in}{0.150000in}}{\pgfqpoint{5.700000in}{5.700000in}}%
\pgfusepath{clip}%
\pgfsetbuttcap%
\pgfsetroundjoin%
\definecolor{currentfill}{rgb}{0.281446,0.084320,0.407414}%
\pgfsetfillcolor{currentfill}%
\pgfsetfillopacity{0.800000}%
\pgfsetlinewidth{0.000000pt}%
\definecolor{currentstroke}{rgb}{0.000000,0.000000,0.000000}%
\pgfsetstrokecolor{currentstroke}%
\pgfsetdash{}{0pt}%
\pgfpathmoveto{\pgfqpoint{2.889482in}{1.876160in}}%
\pgfpathlineto{\pgfqpoint{2.903222in}{1.864289in}}%
\pgfpathlineto{\pgfqpoint{2.916959in}{1.852657in}}%
\pgfpathlineto{\pgfqpoint{2.930693in}{1.841265in}}%
\pgfpathlineto{\pgfqpoint{2.944424in}{1.830110in}}%
\pgfpathlineto{\pgfqpoint{2.952906in}{1.834934in}}%
\pgfpathlineto{\pgfqpoint{2.961376in}{1.839935in}}%
\pgfpathlineto{\pgfqpoint{2.969835in}{1.845109in}}%
\pgfpathlineto{\pgfqpoint{2.978284in}{1.850451in}}%
\pgfpathlineto{\pgfqpoint{2.964583in}{1.861133in}}%
\pgfpathlineto{\pgfqpoint{2.950880in}{1.872052in}}%
\pgfpathlineto{\pgfqpoint{2.937174in}{1.883209in}}%
\pgfpathlineto{\pgfqpoint{2.923465in}{1.894607in}}%
\pgfpathlineto{\pgfqpoint{2.914987in}{1.889726in}}%
\pgfpathlineto{\pgfqpoint{2.906497in}{1.885021in}}%
\pgfpathlineto{\pgfqpoint{2.897996in}{1.880498in}}%
\pgfpathlineto{\pgfqpoint{2.889482in}{1.876160in}}%
\pgfpathclose%
\pgfusepath{fill}%
\end{pgfscope}%
\begin{pgfscope}%
\pgfpathrectangle{\pgfqpoint{1.150000in}{0.150000in}}{\pgfqpoint{5.700000in}{5.700000in}}%
\pgfusepath{clip}%
\pgfsetbuttcap%
\pgfsetroundjoin%
\definecolor{currentfill}{rgb}{0.130067,0.651384,0.521608}%
\pgfsetfillcolor{currentfill}%
\pgfsetfillopacity{0.800000}%
\pgfsetlinewidth{0.000000pt}%
\definecolor{currentstroke}{rgb}{0.000000,0.000000,0.000000}%
\pgfsetstrokecolor{currentstroke}%
\pgfsetdash{}{0pt}%
\pgfpathmoveto{\pgfqpoint{5.768173in}{3.376449in}}%
\pgfpathlineto{\pgfqpoint{5.782886in}{3.388569in}}%
\pgfpathlineto{\pgfqpoint{5.797621in}{3.400867in}}%
\pgfpathlineto{\pgfqpoint{5.812376in}{3.413341in}}%
\pgfpathlineto{\pgfqpoint{5.827152in}{3.425994in}}%
\pgfpathlineto{\pgfqpoint{5.834295in}{3.427613in}}%
\pgfpathlineto{\pgfqpoint{5.841434in}{3.429283in}}%
\pgfpathlineto{\pgfqpoint{5.848567in}{3.431010in}}%
\pgfpathlineto{\pgfqpoint{5.855695in}{3.432801in}}%
\pgfpathlineto{\pgfqpoint{5.840952in}{3.420772in}}%
\pgfpathlineto{\pgfqpoint{5.826230in}{3.408920in}}%
\pgfpathlineto{\pgfqpoint{5.811528in}{3.397245in}}%
\pgfpathlineto{\pgfqpoint{5.796847in}{3.385745in}}%
\pgfpathlineto{\pgfqpoint{5.789685in}{3.383321in}}%
\pgfpathlineto{\pgfqpoint{5.782519in}{3.380968in}}%
\pgfpathlineto{\pgfqpoint{5.775348in}{3.378680in}}%
\pgfpathlineto{\pgfqpoint{5.768173in}{3.376449in}}%
\pgfpathclose%
\pgfusepath{fill}%
\end{pgfscope}%
\begin{pgfscope}%
\pgfpathrectangle{\pgfqpoint{1.150000in}{0.150000in}}{\pgfqpoint{5.700000in}{5.700000in}}%
\pgfusepath{clip}%
\pgfsetbuttcap%
\pgfsetroundjoin%
\definecolor{currentfill}{rgb}{0.143303,0.669459,0.511215}%
\pgfsetfillcolor{currentfill}%
\pgfsetfillopacity{0.800000}%
\pgfsetlinewidth{0.000000pt}%
\definecolor{currentstroke}{rgb}{0.000000,0.000000,0.000000}%
\pgfsetstrokecolor{currentstroke}%
\pgfsetdash{}{0pt}%
\pgfpathmoveto{\pgfqpoint{5.855695in}{3.432801in}}%
\pgfpathlineto{\pgfqpoint{5.870459in}{3.445006in}}%
\pgfpathlineto{\pgfqpoint{5.885244in}{3.457388in}}%
\pgfpathlineto{\pgfqpoint{5.900050in}{3.469947in}}%
\pgfpathlineto{\pgfqpoint{5.914878in}{3.482684in}}%
\pgfpathlineto{\pgfqpoint{5.921967in}{3.483900in}}%
\pgfpathlineto{\pgfqpoint{5.929052in}{3.485186in}}%
\pgfpathlineto{\pgfqpoint{5.936132in}{3.486549in}}%
\pgfpathlineto{\pgfqpoint{5.943208in}{3.487996in}}%
\pgfpathlineto{\pgfqpoint{5.928417in}{3.475919in}}%
\pgfpathlineto{\pgfqpoint{5.913646in}{3.464017in}}%
\pgfpathlineto{\pgfqpoint{5.898896in}{3.452291in}}%
\pgfpathlineto{\pgfqpoint{5.884167in}{3.440740in}}%
\pgfpathlineto{\pgfqpoint{5.877055in}{3.438625in}}%
\pgfpathlineto{\pgfqpoint{5.869939in}{3.436601in}}%
\pgfpathlineto{\pgfqpoint{5.862819in}{3.434662in}}%
\pgfpathlineto{\pgfqpoint{5.855695in}{3.432801in}}%
\pgfpathclose%
\pgfusepath{fill}%
\end{pgfscope}%
\begin{pgfscope}%
\pgfpathrectangle{\pgfqpoint{1.150000in}{0.150000in}}{\pgfqpoint{5.700000in}{5.700000in}}%
\pgfusepath{clip}%
\pgfsetbuttcap%
\pgfsetroundjoin%
\definecolor{currentfill}{rgb}{0.141935,0.526453,0.555991}%
\pgfsetfillcolor{currentfill}%
\pgfsetfillopacity{0.800000}%
\pgfsetlinewidth{0.000000pt}%
\definecolor{currentstroke}{rgb}{0.000000,0.000000,0.000000}%
\pgfsetstrokecolor{currentstroke}%
\pgfsetdash{}{0pt}%
\pgfpathmoveto{\pgfqpoint{5.213196in}{2.993326in}}%
\pgfpathlineto{\pgfqpoint{5.227606in}{3.004552in}}%
\pgfpathlineto{\pgfqpoint{5.242035in}{3.015958in}}%
\pgfpathlineto{\pgfqpoint{5.256482in}{3.027545in}}%
\pgfpathlineto{\pgfqpoint{5.270948in}{3.039313in}}%
\pgfpathlineto{\pgfqpoint{5.278432in}{3.044214in}}%
\pgfpathlineto{\pgfqpoint{5.285908in}{3.049059in}}%
\pgfpathlineto{\pgfqpoint{5.293377in}{3.053853in}}%
\pgfpathlineto{\pgfqpoint{5.300838in}{3.058598in}}%
\pgfpathlineto{\pgfqpoint{5.286390in}{3.047214in}}%
\pgfpathlineto{\pgfqpoint{5.271960in}{3.036011in}}%
\pgfpathlineto{\pgfqpoint{5.257549in}{3.024987in}}%
\pgfpathlineto{\pgfqpoint{5.243156in}{3.014144in}}%
\pgfpathlineto{\pgfqpoint{5.235677in}{3.009003in}}%
\pgfpathlineto{\pgfqpoint{5.228190in}{3.003823in}}%
\pgfpathlineto{\pgfqpoint{5.220697in}{2.998599in}}%
\pgfpathlineto{\pgfqpoint{5.213196in}{2.993326in}}%
\pgfpathclose%
\pgfusepath{fill}%
\end{pgfscope}%
\begin{pgfscope}%
\pgfpathrectangle{\pgfqpoint{1.150000in}{0.150000in}}{\pgfqpoint{5.700000in}{5.700000in}}%
\pgfusepath{clip}%
\pgfsetbuttcap%
\pgfsetroundjoin%
\definecolor{currentfill}{rgb}{0.280267,0.073417,0.397163}%
\pgfsetfillcolor{currentfill}%
\pgfsetfillopacity{0.800000}%
\pgfsetlinewidth{0.000000pt}%
\definecolor{currentstroke}{rgb}{0.000000,0.000000,0.000000}%
\pgfsetstrokecolor{currentstroke}%
\pgfsetdash{}{0pt}%
\pgfpathmoveto{\pgfqpoint{3.689949in}{1.813980in}}%
\pgfpathlineto{\pgfqpoint{3.703662in}{1.813890in}}%
\pgfpathlineto{\pgfqpoint{3.717381in}{1.813997in}}%
\pgfpathlineto{\pgfqpoint{3.731108in}{1.814298in}}%
\pgfpathlineto{\pgfqpoint{3.744842in}{1.814795in}}%
\pgfpathlineto{\pgfqpoint{3.752937in}{1.825966in}}%
\pgfpathlineto{\pgfqpoint{3.761027in}{1.837143in}}%
\pgfpathlineto{\pgfqpoint{3.769112in}{1.848321in}}%
\pgfpathlineto{\pgfqpoint{3.777191in}{1.859500in}}%
\pgfpathlineto{\pgfqpoint{3.763466in}{1.858726in}}%
\pgfpathlineto{\pgfqpoint{3.749749in}{1.858148in}}%
\pgfpathlineto{\pgfqpoint{3.736038in}{1.857765in}}%
\pgfpathlineto{\pgfqpoint{3.722335in}{1.857577in}}%
\pgfpathlineto{\pgfqpoint{3.714247in}{1.846664in}}%
\pgfpathlineto{\pgfqpoint{3.706153in}{1.835758in}}%
\pgfpathlineto{\pgfqpoint{3.698054in}{1.824862in}}%
\pgfpathlineto{\pgfqpoint{3.689949in}{1.813980in}}%
\pgfpathclose%
\pgfusepath{fill}%
\end{pgfscope}%
\begin{pgfscope}%
\pgfpathrectangle{\pgfqpoint{1.150000in}{0.150000in}}{\pgfqpoint{5.700000in}{5.700000in}}%
\pgfusepath{clip}%
\pgfsetbuttcap%
\pgfsetroundjoin%
\definecolor{currentfill}{rgb}{0.180629,0.429975,0.557282}%
\pgfsetfillcolor{currentfill}%
\pgfsetfillopacity{0.800000}%
\pgfsetlinewidth{0.000000pt}%
\definecolor{currentstroke}{rgb}{0.000000,0.000000,0.000000}%
\pgfsetstrokecolor{currentstroke}%
\pgfsetdash{}{0pt}%
\pgfpathmoveto{\pgfqpoint{4.832385in}{2.694382in}}%
\pgfpathlineto{\pgfqpoint{4.846580in}{2.704099in}}%
\pgfpathlineto{\pgfqpoint{4.860793in}{2.714000in}}%
\pgfpathlineto{\pgfqpoint{4.875021in}{2.724084in}}%
\pgfpathlineto{\pgfqpoint{4.889267in}{2.734351in}}%
\pgfpathlineto{\pgfqpoint{4.896953in}{2.741977in}}%
\pgfpathlineto{\pgfqpoint{4.904632in}{2.749510in}}%
\pgfpathlineto{\pgfqpoint{4.912303in}{2.756953in}}%
\pgfpathlineto{\pgfqpoint{4.919968in}{2.764310in}}%
\pgfpathlineto{\pgfqpoint{4.905732in}{2.754256in}}%
\pgfpathlineto{\pgfqpoint{4.891513in}{2.744385in}}%
\pgfpathlineto{\pgfqpoint{4.877311in}{2.734696in}}%
\pgfpathlineto{\pgfqpoint{4.863124in}{2.725189in}}%
\pgfpathlineto{\pgfqpoint{4.855450in}{2.717608in}}%
\pgfpathlineto{\pgfqpoint{4.847768in}{2.709948in}}%
\pgfpathlineto{\pgfqpoint{4.840080in}{2.702207in}}%
\pgfpathlineto{\pgfqpoint{4.832385in}{2.694382in}}%
\pgfpathclose%
\pgfusepath{fill}%
\end{pgfscope}%
\begin{pgfscope}%
\pgfpathrectangle{\pgfqpoint{1.150000in}{0.150000in}}{\pgfqpoint{5.700000in}{5.700000in}}%
\pgfusepath{clip}%
\pgfsetbuttcap%
\pgfsetroundjoin%
\definecolor{currentfill}{rgb}{0.282327,0.094955,0.417331}%
\pgfsetfillcolor{currentfill}%
\pgfsetfillopacity{0.800000}%
\pgfsetlinewidth{0.000000pt}%
\definecolor{currentstroke}{rgb}{0.000000,0.000000,0.000000}%
\pgfsetstrokecolor{currentstroke}%
\pgfsetdash{}{0pt}%
\pgfpathmoveto{\pgfqpoint{3.777191in}{1.859500in}}%
\pgfpathlineto{\pgfqpoint{3.790924in}{1.860468in}}%
\pgfpathlineto{\pgfqpoint{3.804665in}{1.861629in}}%
\pgfpathlineto{\pgfqpoint{3.818414in}{1.862984in}}%
\pgfpathlineto{\pgfqpoint{3.832172in}{1.864532in}}%
\pgfpathlineto{\pgfqpoint{3.840238in}{1.875965in}}%
\pgfpathlineto{\pgfqpoint{3.848300in}{1.887386in}}%
\pgfpathlineto{\pgfqpoint{3.856357in}{1.898792in}}%
\pgfpathlineto{\pgfqpoint{3.864409in}{1.910182in}}%
\pgfpathlineto{\pgfqpoint{3.850659in}{1.908389in}}%
\pgfpathlineto{\pgfqpoint{3.836917in}{1.906788in}}%
\pgfpathlineto{\pgfqpoint{3.823184in}{1.905381in}}%
\pgfpathlineto{\pgfqpoint{3.809459in}{1.904168in}}%
\pgfpathlineto{\pgfqpoint{3.801400in}{1.893012in}}%
\pgfpathlineto{\pgfqpoint{3.793335in}{1.881847in}}%
\pgfpathlineto{\pgfqpoint{3.785266in}{1.870676in}}%
\pgfpathlineto{\pgfqpoint{3.777191in}{1.859500in}}%
\pgfpathclose%
\pgfusepath{fill}%
\end{pgfscope}%
\begin{pgfscope}%
\pgfpathrectangle{\pgfqpoint{1.150000in}{0.150000in}}{\pgfqpoint{5.700000in}{5.700000in}}%
\pgfusepath{clip}%
\pgfsetbuttcap%
\pgfsetroundjoin%
\definecolor{currentfill}{rgb}{0.208030,0.718701,0.472873}%
\pgfsetfillcolor{currentfill}%
\pgfsetfillopacity{0.800000}%
\pgfsetlinewidth{0.000000pt}%
\definecolor{currentstroke}{rgb}{0.000000,0.000000,0.000000}%
\pgfsetstrokecolor{currentstroke}%
\pgfsetdash{}{0pt}%
\pgfpathmoveto{\pgfqpoint{6.118193in}{3.595150in}}%
\pgfpathlineto{\pgfqpoint{6.133098in}{3.607393in}}%
\pgfpathlineto{\pgfqpoint{6.148026in}{3.619812in}}%
\pgfpathlineto{\pgfqpoint{6.162976in}{3.632406in}}%
\pgfpathlineto{\pgfqpoint{6.169917in}{3.632905in}}%
\pgfpathlineto{\pgfqpoint{6.176857in}{3.633543in}}%
\pgfpathlineto{\pgfqpoint{6.183795in}{3.634330in}}%
\pgfpathlineto{\pgfqpoint{6.190733in}{3.635273in}}%
\pgfpathlineto{\pgfqpoint{6.175827in}{3.623438in}}%
\pgfpathlineto{\pgfqpoint{6.160943in}{3.611777in}}%
\pgfpathlineto{\pgfqpoint{6.146080in}{3.600290in}}%
\pgfpathlineto{\pgfqpoint{6.139109in}{3.598771in}}%
\pgfpathlineto{\pgfqpoint{6.132138in}{3.597413in}}%
\pgfpathlineto{\pgfqpoint{6.125166in}{3.596209in}}%
\pgfpathlineto{\pgfqpoint{6.118193in}{3.595150in}}%
\pgfpathclose%
\pgfusepath{fill}%
\end{pgfscope}%
\begin{pgfscope}%
\pgfpathrectangle{\pgfqpoint{1.150000in}{0.150000in}}{\pgfqpoint{5.700000in}{5.700000in}}%
\pgfusepath{clip}%
\pgfsetbuttcap%
\pgfsetroundjoin%
\definecolor{currentfill}{rgb}{0.277018,0.050344,0.375715}%
\pgfsetfillcolor{currentfill}%
\pgfsetfillopacity{0.800000}%
\pgfsetlinewidth{0.000000pt}%
\definecolor{currentstroke}{rgb}{0.000000,0.000000,0.000000}%
\pgfsetstrokecolor{currentstroke}%
\pgfsetdash{}{0pt}%
\pgfpathmoveto{\pgfqpoint{3.602651in}{1.774186in}}%
\pgfpathlineto{\pgfqpoint{3.616348in}{1.772999in}}%
\pgfpathlineto{\pgfqpoint{3.630051in}{1.772011in}}%
\pgfpathlineto{\pgfqpoint{3.643761in}{1.771220in}}%
\pgfpathlineto{\pgfqpoint{3.657477in}{1.770626in}}%
\pgfpathlineto{\pgfqpoint{3.665603in}{1.781432in}}%
\pgfpathlineto{\pgfqpoint{3.673724in}{1.792262in}}%
\pgfpathlineto{\pgfqpoint{3.681839in}{1.803112in}}%
\pgfpathlineto{\pgfqpoint{3.689949in}{1.813980in}}%
\pgfpathlineto{\pgfqpoint{3.676244in}{1.814265in}}%
\pgfpathlineto{\pgfqpoint{3.662545in}{1.814748in}}%
\pgfpathlineto{\pgfqpoint{3.648852in}{1.815428in}}%
\pgfpathlineto{\pgfqpoint{3.635166in}{1.816307in}}%
\pgfpathlineto{\pgfqpoint{3.627046in}{1.805735in}}%
\pgfpathlineto{\pgfqpoint{3.618920in}{1.795189in}}%
\pgfpathlineto{\pgfqpoint{3.610788in}{1.784672in}}%
\pgfpathlineto{\pgfqpoint{3.602651in}{1.774186in}}%
\pgfpathclose%
\pgfusepath{fill}%
\end{pgfscope}%
\begin{pgfscope}%
\pgfpathrectangle{\pgfqpoint{1.150000in}{0.150000in}}{\pgfqpoint{5.700000in}{5.700000in}}%
\pgfusepath{clip}%
\pgfsetbuttcap%
\pgfsetroundjoin%
\definecolor{currentfill}{rgb}{0.162016,0.687316,0.499129}%
\pgfsetfillcolor{currentfill}%
\pgfsetfillopacity{0.800000}%
\pgfsetlinewidth{0.000000pt}%
\definecolor{currentstroke}{rgb}{0.000000,0.000000,0.000000}%
\pgfsetstrokecolor{currentstroke}%
\pgfsetdash{}{0pt}%
\pgfpathmoveto{\pgfqpoint{5.943208in}{3.487996in}}%
\pgfpathlineto{\pgfqpoint{5.958021in}{3.500251in}}%
\pgfpathlineto{\pgfqpoint{5.972855in}{3.512681in}}%
\pgfpathlineto{\pgfqpoint{5.987711in}{3.525288in}}%
\pgfpathlineto{\pgfqpoint{6.002588in}{3.538072in}}%
\pgfpathlineto{\pgfqpoint{6.009623in}{3.538931in}}%
\pgfpathlineto{\pgfqpoint{6.016655in}{3.539882in}}%
\pgfpathlineto{\pgfqpoint{6.023683in}{3.540931in}}%
\pgfpathlineto{\pgfqpoint{6.030708in}{3.542087in}}%
\pgfpathlineto{\pgfqpoint{6.015870in}{3.529996in}}%
\pgfpathlineto{\pgfqpoint{6.001052in}{3.518081in}}%
\pgfpathlineto{\pgfqpoint{5.986256in}{3.506340in}}%
\pgfpathlineto{\pgfqpoint{5.971481in}{3.494775in}}%
\pgfpathlineto{\pgfqpoint{5.964417in}{3.492917in}}%
\pgfpathlineto{\pgfqpoint{5.957351in}{3.491173in}}%
\pgfpathlineto{\pgfqpoint{5.950281in}{3.489536in}}%
\pgfpathlineto{\pgfqpoint{5.943208in}{3.487996in}}%
\pgfpathclose%
\pgfusepath{fill}%
\end{pgfscope}%
\begin{pgfscope}%
\pgfpathrectangle{\pgfqpoint{1.150000in}{0.150000in}}{\pgfqpoint{5.700000in}{5.700000in}}%
\pgfusepath{clip}%
\pgfsetbuttcap%
\pgfsetroundjoin%
\definecolor{currentfill}{rgb}{0.220057,0.343307,0.549413}%
\pgfsetfillcolor{currentfill}%
\pgfsetfillopacity{0.800000}%
\pgfsetlinewidth{0.000000pt}%
\definecolor{currentstroke}{rgb}{0.000000,0.000000,0.000000}%
\pgfsetstrokecolor{currentstroke}%
\pgfsetdash{}{0pt}%
\pgfpathmoveto{\pgfqpoint{4.538852in}{2.447505in}}%
\pgfpathlineto{\pgfqpoint{4.552892in}{2.455528in}}%
\pgfpathlineto{\pgfqpoint{4.566947in}{2.463735in}}%
\pgfpathlineto{\pgfqpoint{4.581017in}{2.472127in}}%
\pgfpathlineto{\pgfqpoint{4.595101in}{2.480704in}}%
\pgfpathlineto{\pgfqpoint{4.602916in}{2.490291in}}%
\pgfpathlineto{\pgfqpoint{4.610726in}{2.499780in}}%
\pgfpathlineto{\pgfqpoint{4.618529in}{2.509173in}}%
\pgfpathlineto{\pgfqpoint{4.626325in}{2.518471in}}%
\pgfpathlineto{\pgfqpoint{4.612247in}{2.509972in}}%
\pgfpathlineto{\pgfqpoint{4.598183in}{2.501657in}}%
\pgfpathlineto{\pgfqpoint{4.584134in}{2.493527in}}%
\pgfpathlineto{\pgfqpoint{4.570100in}{2.485582in}}%
\pgfpathlineto{\pgfqpoint{4.562297in}{2.476194in}}%
\pgfpathlineto{\pgfqpoint{4.554488in}{2.466720in}}%
\pgfpathlineto{\pgfqpoint{4.546673in}{2.457157in}}%
\pgfpathlineto{\pgfqpoint{4.538852in}{2.447505in}}%
\pgfpathclose%
\pgfusepath{fill}%
\end{pgfscope}%
\begin{pgfscope}%
\pgfpathrectangle{\pgfqpoint{1.150000in}{0.150000in}}{\pgfqpoint{5.700000in}{5.700000in}}%
\pgfusepath{clip}%
\pgfsetbuttcap%
\pgfsetroundjoin%
\definecolor{currentfill}{rgb}{0.260571,0.246922,0.522828}%
\pgfsetfillcolor{currentfill}%
\pgfsetfillopacity{0.800000}%
\pgfsetlinewidth{0.000000pt}%
\definecolor{currentstroke}{rgb}{0.000000,0.000000,0.000000}%
\pgfsetstrokecolor{currentstroke}%
\pgfsetdash{}{0pt}%
\pgfpathmoveto{\pgfqpoint{4.245270in}{2.198705in}}%
\pgfpathlineto{\pgfqpoint{4.259170in}{2.204476in}}%
\pgfpathlineto{\pgfqpoint{4.273082in}{2.210434in}}%
\pgfpathlineto{\pgfqpoint{4.287006in}{2.216579in}}%
\pgfpathlineto{\pgfqpoint{4.300943in}{2.222910in}}%
\pgfpathlineto{\pgfqpoint{4.308865in}{2.233955in}}%
\pgfpathlineto{\pgfqpoint{4.316781in}{2.244919in}}%
\pgfpathlineto{\pgfqpoint{4.324692in}{2.255802in}}%
\pgfpathlineto{\pgfqpoint{4.332597in}{2.266604in}}%
\pgfpathlineto{\pgfqpoint{4.318664in}{2.260218in}}%
\pgfpathlineto{\pgfqpoint{4.304744in}{2.254019in}}%
\pgfpathlineto{\pgfqpoint{4.290836in}{2.248007in}}%
\pgfpathlineto{\pgfqpoint{4.276941in}{2.242182in}}%
\pgfpathlineto{\pgfqpoint{4.269031in}{2.231423in}}%
\pgfpathlineto{\pgfqpoint{4.261116in}{2.220590in}}%
\pgfpathlineto{\pgfqpoint{4.253195in}{2.209684in}}%
\pgfpathlineto{\pgfqpoint{4.245270in}{2.198705in}}%
\pgfpathclose%
\pgfusepath{fill}%
\end{pgfscope}%
\begin{pgfscope}%
\pgfpathrectangle{\pgfqpoint{1.150000in}{0.150000in}}{\pgfqpoint{5.700000in}{5.700000in}}%
\pgfusepath{clip}%
\pgfsetbuttcap%
\pgfsetroundjoin%
\definecolor{currentfill}{rgb}{0.283229,0.120777,0.440584}%
\pgfsetfillcolor{currentfill}%
\pgfsetfillopacity{0.800000}%
\pgfsetlinewidth{0.000000pt}%
\definecolor{currentstroke}{rgb}{0.000000,0.000000,0.000000}%
\pgfsetstrokecolor{currentstroke}%
\pgfsetdash{}{0pt}%
\pgfpathmoveto{\pgfqpoint{3.864409in}{1.910182in}}%
\pgfpathlineto{\pgfqpoint{3.878167in}{1.912168in}}%
\pgfpathlineto{\pgfqpoint{3.891934in}{1.914346in}}%
\pgfpathlineto{\pgfqpoint{3.905711in}{1.916715in}}%
\pgfpathlineto{\pgfqpoint{3.919496in}{1.919276in}}%
\pgfpathlineto{\pgfqpoint{3.927536in}{1.930873in}}%
\pgfpathlineto{\pgfqpoint{3.935572in}{1.942441in}}%
\pgfpathlineto{\pgfqpoint{3.943602in}{1.953979in}}%
\pgfpathlineto{\pgfqpoint{3.951628in}{1.965486in}}%
\pgfpathlineto{\pgfqpoint{3.937848in}{1.962711in}}%
\pgfpathlineto{\pgfqpoint{3.924078in}{1.960127in}}%
\pgfpathlineto{\pgfqpoint{3.910318in}{1.957735in}}%
\pgfpathlineto{\pgfqpoint{3.896566in}{1.955535in}}%
\pgfpathlineto{\pgfqpoint{3.888534in}{1.944231in}}%
\pgfpathlineto{\pgfqpoint{3.880497in}{1.932903in}}%
\pgfpathlineto{\pgfqpoint{3.872455in}{1.921553in}}%
\pgfpathlineto{\pgfqpoint{3.864409in}{1.910182in}}%
\pgfpathclose%
\pgfusepath{fill}%
\end{pgfscope}%
\begin{pgfscope}%
\pgfpathrectangle{\pgfqpoint{1.150000in}{0.150000in}}{\pgfqpoint{5.700000in}{5.700000in}}%
\pgfusepath{clip}%
\pgfsetbuttcap%
\pgfsetroundjoin%
\definecolor{currentfill}{rgb}{0.278791,0.062145,0.386592}%
\pgfsetfillcolor{currentfill}%
\pgfsetfillopacity{0.800000}%
\pgfsetlinewidth{0.000000pt}%
\definecolor{currentstroke}{rgb}{0.000000,0.000000,0.000000}%
\pgfsetstrokecolor{currentstroke}%
\pgfsetdash{}{0pt}%
\pgfpathmoveto{\pgfqpoint{2.944424in}{1.830110in}}%
\pgfpathlineto{\pgfqpoint{2.958153in}{1.819190in}}%
\pgfpathlineto{\pgfqpoint{2.971879in}{1.808505in}}%
\pgfpathlineto{\pgfqpoint{2.985603in}{1.798054in}}%
\pgfpathlineto{\pgfqpoint{2.999325in}{1.787833in}}%
\pgfpathlineto{\pgfqpoint{3.007777in}{1.793142in}}%
\pgfpathlineto{\pgfqpoint{3.016217in}{1.798619in}}%
\pgfpathlineto{\pgfqpoint{3.024648in}{1.804262in}}%
\pgfpathlineto{\pgfqpoint{3.033067in}{1.810064in}}%
\pgfpathlineto{\pgfqpoint{3.019374in}{1.819812in}}%
\pgfpathlineto{\pgfqpoint{3.005679in}{1.829792in}}%
\pgfpathlineto{\pgfqpoint{2.991982in}{1.840005in}}%
\pgfpathlineto{\pgfqpoint{2.978284in}{1.850451in}}%
\pgfpathlineto{\pgfqpoint{2.969835in}{1.845109in}}%
\pgfpathlineto{\pgfqpoint{2.961376in}{1.839935in}}%
\pgfpathlineto{\pgfqpoint{2.952906in}{1.834934in}}%
\pgfpathlineto{\pgfqpoint{2.944424in}{1.830110in}}%
\pgfpathclose%
\pgfusepath{fill}%
\end{pgfscope}%
\begin{pgfscope}%
\pgfpathrectangle{\pgfqpoint{1.150000in}{0.150000in}}{\pgfqpoint{5.700000in}{5.700000in}}%
\pgfusepath{clip}%
\pgfsetbuttcap%
\pgfsetroundjoin%
\definecolor{currentfill}{rgb}{0.269944,0.014625,0.341379}%
\pgfsetfillcolor{currentfill}%
\pgfsetfillopacity{0.800000}%
\pgfsetlinewidth{0.000000pt}%
\definecolor{currentstroke}{rgb}{0.000000,0.000000,0.000000}%
\pgfsetstrokecolor{currentstroke}%
\pgfsetdash{}{0pt}%
\pgfpathmoveto{\pgfqpoint{3.285318in}{1.716225in}}%
\pgfpathlineto{\pgfqpoint{3.298995in}{1.710657in}}%
\pgfpathlineto{\pgfqpoint{3.312674in}{1.705299in}}%
\pgfpathlineto{\pgfqpoint{3.326357in}{1.700151in}}%
\pgfpathlineto{\pgfqpoint{3.340042in}{1.695211in}}%
\pgfpathlineto{\pgfqpoint{3.348302in}{1.703831in}}%
\pgfpathlineto{\pgfqpoint{3.356555in}{1.712546in}}%
\pgfpathlineto{\pgfqpoint{3.364800in}{1.721352in}}%
\pgfpathlineto{\pgfqpoint{3.373038in}{1.730246in}}%
\pgfpathlineto{\pgfqpoint{3.359372in}{1.734782in}}%
\pgfpathlineto{\pgfqpoint{3.345708in}{1.739528in}}%
\pgfpathlineto{\pgfqpoint{3.332048in}{1.744482in}}%
\pgfpathlineto{\pgfqpoint{3.318389in}{1.749647in}}%
\pgfpathlineto{\pgfqpoint{3.310133in}{1.741145in}}%
\pgfpathlineto{\pgfqpoint{3.301869in}{1.732738in}}%
\pgfpathlineto{\pgfqpoint{3.293597in}{1.724430in}}%
\pgfpathlineto{\pgfqpoint{3.285318in}{1.716225in}}%
\pgfpathclose%
\pgfusepath{fill}%
\end{pgfscope}%
\begin{pgfscope}%
\pgfpathrectangle{\pgfqpoint{1.150000in}{0.150000in}}{\pgfqpoint{5.700000in}{5.700000in}}%
\pgfusepath{clip}%
\pgfsetbuttcap%
\pgfsetroundjoin%
\definecolor{currentfill}{rgb}{0.271305,0.019942,0.347269}%
\pgfsetfillcolor{currentfill}%
\pgfsetfillopacity{0.800000}%
\pgfsetlinewidth{0.000000pt}%
\definecolor{currentstroke}{rgb}{0.000000,0.000000,0.000000}%
\pgfsetstrokecolor{currentstroke}%
\pgfsetdash{}{0pt}%
\pgfpathmoveto{\pgfqpoint{3.142584in}{1.740244in}}%
\pgfpathlineto{\pgfqpoint{3.156273in}{1.732519in}}%
\pgfpathlineto{\pgfqpoint{3.169962in}{1.725013in}}%
\pgfpathlineto{\pgfqpoint{3.183652in}{1.717725in}}%
\pgfpathlineto{\pgfqpoint{3.197343in}{1.710653in}}%
\pgfpathlineto{\pgfqpoint{3.205677in}{1.717959in}}%
\pgfpathlineto{\pgfqpoint{3.214003in}{1.725392in}}%
\pgfpathlineto{\pgfqpoint{3.222320in}{1.732948in}}%
\pgfpathlineto{\pgfqpoint{3.230628in}{1.740623in}}%
\pgfpathlineto{\pgfqpoint{3.216960in}{1.747258in}}%
\pgfpathlineto{\pgfqpoint{3.203293in}{1.754110in}}%
\pgfpathlineto{\pgfqpoint{3.189627in}{1.761180in}}%
\pgfpathlineto{\pgfqpoint{3.175961in}{1.768468in}}%
\pgfpathlineto{\pgfqpoint{3.167630in}{1.761218in}}%
\pgfpathlineto{\pgfqpoint{3.159291in}{1.754094in}}%
\pgfpathlineto{\pgfqpoint{3.150942in}{1.747102in}}%
\pgfpathlineto{\pgfqpoint{3.142584in}{1.740244in}}%
\pgfpathclose%
\pgfusepath{fill}%
\end{pgfscope}%
\begin{pgfscope}%
\pgfpathrectangle{\pgfqpoint{1.150000in}{0.150000in}}{\pgfqpoint{5.700000in}{5.700000in}}%
\pgfusepath{clip}%
\pgfsetbuttcap%
\pgfsetroundjoin%
\definecolor{currentfill}{rgb}{0.212395,0.359683,0.551710}%
\pgfsetfillcolor{currentfill}%
\pgfsetfillopacity{0.800000}%
\pgfsetlinewidth{0.000000pt}%
\definecolor{currentstroke}{rgb}{0.000000,0.000000,0.000000}%
\pgfsetstrokecolor{currentstroke}%
\pgfsetdash{}{0pt}%
\pgfpathmoveto{\pgfqpoint{2.334380in}{2.569829in}}%
\pgfpathlineto{\pgfqpoint{2.348461in}{2.546690in}}%
\pgfpathlineto{\pgfqpoint{2.362527in}{2.523882in}}%
\pgfpathlineto{\pgfqpoint{2.376579in}{2.501403in}}%
\pgfpathlineto{\pgfqpoint{2.390618in}{2.479250in}}%
\pgfpathlineto{\pgfqpoint{2.399493in}{2.479090in}}%
\pgfpathlineto{\pgfqpoint{2.408351in}{2.479193in}}%
\pgfpathlineto{\pgfqpoint{2.417191in}{2.479555in}}%
\pgfpathlineto{\pgfqpoint{2.426015in}{2.480171in}}%
\pgfpathlineto{\pgfqpoint{2.412024in}{2.501821in}}%
\pgfpathlineto{\pgfqpoint{2.398020in}{2.523795in}}%
\pgfpathlineto{\pgfqpoint{2.384002in}{2.546097in}}%
\pgfpathlineto{\pgfqpoint{2.369971in}{2.568729in}}%
\pgfpathlineto{\pgfqpoint{2.361100in}{2.568605in}}%
\pgfpathlineto{\pgfqpoint{2.352212in}{2.568743in}}%
\pgfpathlineto{\pgfqpoint{2.343305in}{2.569150in}}%
\pgfpathlineto{\pgfqpoint{2.334380in}{2.569829in}}%
\pgfpathclose%
\pgfusepath{fill}%
\end{pgfscope}%
\begin{pgfscope}%
\pgfpathrectangle{\pgfqpoint{1.150000in}{0.150000in}}{\pgfqpoint{5.700000in}{5.700000in}}%
\pgfusepath{clip}%
\pgfsetbuttcap%
\pgfsetroundjoin%
\definecolor{currentfill}{rgb}{0.185783,0.704891,0.485273}%
\pgfsetfillcolor{currentfill}%
\pgfsetfillopacity{0.800000}%
\pgfsetlinewidth{0.000000pt}%
\definecolor{currentstroke}{rgb}{0.000000,0.000000,0.000000}%
\pgfsetstrokecolor{currentstroke}%
\pgfsetdash{}{0pt}%
\pgfpathmoveto{\pgfqpoint{6.030708in}{3.542087in}}%
\pgfpathlineto{\pgfqpoint{6.045569in}{3.554354in}}%
\pgfpathlineto{\pgfqpoint{6.060450in}{3.566797in}}%
\pgfpathlineto{\pgfqpoint{6.075354in}{3.579416in}}%
\pgfpathlineto{\pgfqpoint{6.090279in}{3.592211in}}%
\pgfpathlineto{\pgfqpoint{6.097261in}{3.592767in}}%
\pgfpathlineto{\pgfqpoint{6.104241in}{3.593436in}}%
\pgfpathlineto{\pgfqpoint{6.111218in}{3.594228in}}%
\pgfpathlineto{\pgfqpoint{6.118193in}{3.595150in}}%
\pgfpathlineto{\pgfqpoint{6.103309in}{3.583081in}}%
\pgfpathlineto{\pgfqpoint{6.088446in}{3.571188in}}%
\pgfpathlineto{\pgfqpoint{6.073605in}{3.559469in}}%
\pgfpathlineto{\pgfqpoint{6.058785in}{3.547925in}}%
\pgfpathlineto{\pgfqpoint{6.051769in}{3.546268in}}%
\pgfpathlineto{\pgfqpoint{6.044751in}{3.544748in}}%
\pgfpathlineto{\pgfqpoint{6.037731in}{3.543357in}}%
\pgfpathlineto{\pgfqpoint{6.030708in}{3.542087in}}%
\pgfpathclose%
\pgfusepath{fill}%
\end{pgfscope}%
\begin{pgfscope}%
\pgfpathrectangle{\pgfqpoint{1.150000in}{0.150000in}}{\pgfqpoint{5.700000in}{5.700000in}}%
\pgfusepath{clip}%
\pgfsetbuttcap%
\pgfsetroundjoin%
\definecolor{currentfill}{rgb}{0.273809,0.031497,0.358853}%
\pgfsetfillcolor{currentfill}%
\pgfsetfillopacity{0.800000}%
\pgfsetlinewidth{0.000000pt}%
\definecolor{currentstroke}{rgb}{0.000000,0.000000,0.000000}%
\pgfsetstrokecolor{currentstroke}%
\pgfsetdash{}{0pt}%
\pgfpathmoveto{\pgfqpoint{3.515261in}{1.740710in}}%
\pgfpathlineto{\pgfqpoint{3.528948in}{1.738385in}}%
\pgfpathlineto{\pgfqpoint{3.542641in}{1.736261in}}%
\pgfpathlineto{\pgfqpoint{3.556340in}{1.734337in}}%
\pgfpathlineto{\pgfqpoint{3.570043in}{1.732612in}}%
\pgfpathlineto{\pgfqpoint{3.578204in}{1.742944in}}%
\pgfpathlineto{\pgfqpoint{3.586359in}{1.753319in}}%
\pgfpathlineto{\pgfqpoint{3.594508in}{1.763734in}}%
\pgfpathlineto{\pgfqpoint{3.602651in}{1.774186in}}%
\pgfpathlineto{\pgfqpoint{3.588959in}{1.775571in}}%
\pgfpathlineto{\pgfqpoint{3.575274in}{1.777156in}}%
\pgfpathlineto{\pgfqpoint{3.561594in}{1.778941in}}%
\pgfpathlineto{\pgfqpoint{3.547919in}{1.780926in}}%
\pgfpathlineto{\pgfqpoint{3.539764in}{1.770802in}}%
\pgfpathlineto{\pgfqpoint{3.531602in}{1.760722in}}%
\pgfpathlineto{\pgfqpoint{3.523434in}{1.750690in}}%
\pgfpathlineto{\pgfqpoint{3.515261in}{1.740710in}}%
\pgfpathclose%
\pgfusepath{fill}%
\end{pgfscope}%
\begin{pgfscope}%
\pgfpathrectangle{\pgfqpoint{1.150000in}{0.150000in}}{\pgfqpoint{5.700000in}{5.700000in}}%
\pgfusepath{clip}%
\pgfsetbuttcap%
\pgfsetroundjoin%
\definecolor{currentfill}{rgb}{0.132444,0.552216,0.553018}%
\pgfsetfillcolor{currentfill}%
\pgfsetfillopacity{0.800000}%
\pgfsetlinewidth{0.000000pt}%
\definecolor{currentstroke}{rgb}{0.000000,0.000000,0.000000}%
\pgfsetstrokecolor{currentstroke}%
\pgfsetdash{}{0pt}%
\pgfpathmoveto{\pgfqpoint{5.300838in}{3.058598in}}%
\pgfpathlineto{\pgfqpoint{5.315305in}{3.070162in}}%
\pgfpathlineto{\pgfqpoint{5.329791in}{3.081907in}}%
\pgfpathlineto{\pgfqpoint{5.344296in}{3.093832in}}%
\pgfpathlineto{\pgfqpoint{5.358820in}{3.105938in}}%
\pgfpathlineto{\pgfqpoint{5.366255in}{3.110234in}}%
\pgfpathlineto{\pgfqpoint{5.373683in}{3.114484in}}%
\pgfpathlineto{\pgfqpoint{5.381103in}{3.118693in}}%
\pgfpathlineto{\pgfqpoint{5.388515in}{3.122864in}}%
\pgfpathlineto{\pgfqpoint{5.374011in}{3.111177in}}%
\pgfpathlineto{\pgfqpoint{5.359526in}{3.099671in}}%
\pgfpathlineto{\pgfqpoint{5.345059in}{3.088344in}}%
\pgfpathlineto{\pgfqpoint{5.330611in}{3.077196in}}%
\pgfpathlineto{\pgfqpoint{5.323179in}{3.072595in}}%
\pgfpathlineto{\pgfqpoint{5.315739in}{3.067965in}}%
\pgfpathlineto{\pgfqpoint{5.308292in}{3.063301in}}%
\pgfpathlineto{\pgfqpoint{5.300838in}{3.058598in}}%
\pgfpathclose%
\pgfusepath{fill}%
\end{pgfscope}%
\begin{pgfscope}%
\pgfpathrectangle{\pgfqpoint{1.150000in}{0.150000in}}{\pgfqpoint{5.700000in}{5.700000in}}%
\pgfusepath{clip}%
\pgfsetbuttcap%
\pgfsetroundjoin%
\definecolor{currentfill}{rgb}{0.282290,0.145912,0.461510}%
\pgfsetfillcolor{currentfill}%
\pgfsetfillopacity{0.800000}%
\pgfsetlinewidth{0.000000pt}%
\definecolor{currentstroke}{rgb}{0.000000,0.000000,0.000000}%
\pgfsetstrokecolor{currentstroke}%
\pgfsetdash{}{0pt}%
\pgfpathmoveto{\pgfqpoint{3.951628in}{1.965486in}}%
\pgfpathlineto{\pgfqpoint{3.965416in}{1.968451in}}%
\pgfpathlineto{\pgfqpoint{3.979215in}{1.971608in}}%
\pgfpathlineto{\pgfqpoint{3.993023in}{1.974954in}}%
\pgfpathlineto{\pgfqpoint{4.006841in}{1.978491in}}%
\pgfpathlineto{\pgfqpoint{4.014856in}{1.990158in}}%
\pgfpathlineto{\pgfqpoint{4.022867in}{2.001781in}}%
\pgfpathlineto{\pgfqpoint{4.030872in}{2.013360in}}%
\pgfpathlineto{\pgfqpoint{4.038872in}{2.024893in}}%
\pgfpathlineto{\pgfqpoint{4.025059in}{2.021174in}}%
\pgfpathlineto{\pgfqpoint{4.011257in}{2.017645in}}%
\pgfpathlineto{\pgfqpoint{3.997464in}{2.014306in}}%
\pgfpathlineto{\pgfqpoint{3.983681in}{2.011158in}}%
\pgfpathlineto{\pgfqpoint{3.975675in}{1.999796in}}%
\pgfpathlineto{\pgfqpoint{3.967664in}{1.988395in}}%
\pgfpathlineto{\pgfqpoint{3.959648in}{1.976958in}}%
\pgfpathlineto{\pgfqpoint{3.951628in}{1.965486in}}%
\pgfpathclose%
\pgfusepath{fill}%
\end{pgfscope}%
\begin{pgfscope}%
\pgfpathrectangle{\pgfqpoint{1.150000in}{0.150000in}}{\pgfqpoint{5.700000in}{5.700000in}}%
\pgfusepath{clip}%
\pgfsetbuttcap%
\pgfsetroundjoin%
\definecolor{currentfill}{rgb}{0.169646,0.456262,0.558030}%
\pgfsetfillcolor{currentfill}%
\pgfsetfillopacity{0.800000}%
\pgfsetlinewidth{0.000000pt}%
\definecolor{currentstroke}{rgb}{0.000000,0.000000,0.000000}%
\pgfsetstrokecolor{currentstroke}%
\pgfsetdash{}{0pt}%
\pgfpathmoveto{\pgfqpoint{4.919968in}{2.764310in}}%
\pgfpathlineto{\pgfqpoint{4.934220in}{2.774547in}}%
\pgfpathlineto{\pgfqpoint{4.948489in}{2.784966in}}%
\pgfpathlineto{\pgfqpoint{4.962776in}{2.795567in}}%
\pgfpathlineto{\pgfqpoint{4.977079in}{2.806352in}}%
\pgfpathlineto{\pgfqpoint{4.984726in}{2.813390in}}%
\pgfpathlineto{\pgfqpoint{4.992365in}{2.820338in}}%
\pgfpathlineto{\pgfqpoint{4.999997in}{2.827200in}}%
\pgfpathlineto{\pgfqpoint{5.007622in}{2.833979in}}%
\pgfpathlineto{\pgfqpoint{4.993329in}{2.823442in}}%
\pgfpathlineto{\pgfqpoint{4.979054in}{2.813087in}}%
\pgfpathlineto{\pgfqpoint{4.964796in}{2.802915in}}%
\pgfpathlineto{\pgfqpoint{4.950555in}{2.792924in}}%
\pgfpathlineto{\pgfqpoint{4.942919in}{2.785887in}}%
\pgfpathlineto{\pgfqpoint{4.935275in}{2.778774in}}%
\pgfpathlineto{\pgfqpoint{4.927625in}{2.771583in}}%
\pgfpathlineto{\pgfqpoint{4.919968in}{2.764310in}}%
\pgfpathclose%
\pgfusepath{fill}%
\end{pgfscope}%
\begin{pgfscope}%
\pgfpathrectangle{\pgfqpoint{1.150000in}{0.150000in}}{\pgfqpoint{5.700000in}{5.700000in}}%
\pgfusepath{clip}%
\pgfsetbuttcap%
\pgfsetroundjoin%
\definecolor{currentfill}{rgb}{0.248629,0.278775,0.534556}%
\pgfsetfillcolor{currentfill}%
\pgfsetfillopacity{0.800000}%
\pgfsetlinewidth{0.000000pt}%
\definecolor{currentstroke}{rgb}{0.000000,0.000000,0.000000}%
\pgfsetstrokecolor{currentstroke}%
\pgfsetdash{}{0pt}%
\pgfpathmoveto{\pgfqpoint{4.332597in}{2.266604in}}%
\pgfpathlineto{\pgfqpoint{4.346543in}{2.273176in}}%
\pgfpathlineto{\pgfqpoint{4.360502in}{2.279934in}}%
\pgfpathlineto{\pgfqpoint{4.374473in}{2.286879in}}%
\pgfpathlineto{\pgfqpoint{4.388458in}{2.294009in}}%
\pgfpathlineto{\pgfqpoint{4.396354in}{2.304764in}}%
\pgfpathlineto{\pgfqpoint{4.404244in}{2.315429in}}%
\pgfpathlineto{\pgfqpoint{4.412129in}{2.326005in}}%
\pgfpathlineto{\pgfqpoint{4.420008in}{2.336493in}}%
\pgfpathlineto{\pgfqpoint{4.406027in}{2.329341in}}%
\pgfpathlineto{\pgfqpoint{4.392059in}{2.322375in}}%
\pgfpathlineto{\pgfqpoint{4.378105in}{2.315595in}}%
\pgfpathlineto{\pgfqpoint{4.364164in}{2.309002in}}%
\pgfpathlineto{\pgfqpoint{4.356281in}{2.298523in}}%
\pgfpathlineto{\pgfqpoint{4.348392in}{2.287964in}}%
\pgfpathlineto{\pgfqpoint{4.340497in}{2.277325in}}%
\pgfpathlineto{\pgfqpoint{4.332597in}{2.266604in}}%
\pgfpathclose%
\pgfusepath{fill}%
\end{pgfscope}%
\begin{pgfscope}%
\pgfpathrectangle{\pgfqpoint{1.150000in}{0.150000in}}{\pgfqpoint{5.700000in}{5.700000in}}%
\pgfusepath{clip}%
\pgfsetbuttcap%
\pgfsetroundjoin%
\definecolor{currentfill}{rgb}{0.206756,0.371758,0.553117}%
\pgfsetfillcolor{currentfill}%
\pgfsetfillopacity{0.800000}%
\pgfsetlinewidth{0.000000pt}%
\definecolor{currentstroke}{rgb}{0.000000,0.000000,0.000000}%
\pgfsetstrokecolor{currentstroke}%
\pgfsetdash{}{0pt}%
\pgfpathmoveto{\pgfqpoint{4.626325in}{2.518471in}}%
\pgfpathlineto{\pgfqpoint{4.640419in}{2.527154in}}%
\pgfpathlineto{\pgfqpoint{4.654527in}{2.536021in}}%
\pgfpathlineto{\pgfqpoint{4.668651in}{2.545073in}}%
\pgfpathlineto{\pgfqpoint{4.682790in}{2.554309in}}%
\pgfpathlineto{\pgfqpoint{4.690574in}{2.563414in}}%
\pgfpathlineto{\pgfqpoint{4.698351in}{2.572419in}}%
\pgfpathlineto{\pgfqpoint{4.706122in}{2.581326in}}%
\pgfpathlineto{\pgfqpoint{4.713886in}{2.590135in}}%
\pgfpathlineto{\pgfqpoint{4.699754in}{2.581011in}}%
\pgfpathlineto{\pgfqpoint{4.685637in}{2.572071in}}%
\pgfpathlineto{\pgfqpoint{4.671535in}{2.563314in}}%
\pgfpathlineto{\pgfqpoint{4.657448in}{2.554742in}}%
\pgfpathlineto{\pgfqpoint{4.649677in}{2.545809in}}%
\pgfpathlineto{\pgfqpoint{4.641899in}{2.536787in}}%
\pgfpathlineto{\pgfqpoint{4.634115in}{2.527675in}}%
\pgfpathlineto{\pgfqpoint{4.626325in}{2.518471in}}%
\pgfpathclose%
\pgfusepath{fill}%
\end{pgfscope}%
\begin{pgfscope}%
\pgfpathrectangle{\pgfqpoint{1.150000in}{0.150000in}}{\pgfqpoint{5.700000in}{5.700000in}}%
\pgfusepath{clip}%
\pgfsetbuttcap%
\pgfsetroundjoin%
\definecolor{currentfill}{rgb}{0.271305,0.019942,0.347269}%
\pgfsetfillcolor{currentfill}%
\pgfsetfillopacity{0.800000}%
\pgfsetlinewidth{0.000000pt}%
\definecolor{currentstroke}{rgb}{0.000000,0.000000,0.000000}%
\pgfsetstrokecolor{currentstroke}%
\pgfsetdash{}{0pt}%
\pgfpathmoveto{\pgfqpoint{3.427739in}{1.714169in}}%
\pgfpathlineto{\pgfqpoint{3.441424in}{1.710663in}}%
\pgfpathlineto{\pgfqpoint{3.455112in}{1.707361in}}%
\pgfpathlineto{\pgfqpoint{3.468805in}{1.704261in}}%
\pgfpathlineto{\pgfqpoint{3.482503in}{1.701364in}}%
\pgfpathlineto{\pgfqpoint{3.490702in}{1.711107in}}%
\pgfpathlineto{\pgfqpoint{3.498894in}{1.720915in}}%
\pgfpathlineto{\pgfqpoint{3.507081in}{1.730784in}}%
\pgfpathlineto{\pgfqpoint{3.515261in}{1.740710in}}%
\pgfpathlineto{\pgfqpoint{3.501578in}{1.743236in}}%
\pgfpathlineto{\pgfqpoint{3.487900in}{1.745965in}}%
\pgfpathlineto{\pgfqpoint{3.474226in}{1.748896in}}%
\pgfpathlineto{\pgfqpoint{3.460557in}{1.752031in}}%
\pgfpathlineto{\pgfqpoint{3.452362in}{1.742464in}}%
\pgfpathlineto{\pgfqpoint{3.444161in}{1.732963in}}%
\pgfpathlineto{\pgfqpoint{3.435954in}{1.723530in}}%
\pgfpathlineto{\pgfqpoint{3.427739in}{1.714169in}}%
\pgfpathclose%
\pgfusepath{fill}%
\end{pgfscope}%
\begin{pgfscope}%
\pgfpathrectangle{\pgfqpoint{1.150000in}{0.150000in}}{\pgfqpoint{5.700000in}{5.700000in}}%
\pgfusepath{clip}%
\pgfsetbuttcap%
\pgfsetroundjoin%
\definecolor{currentfill}{rgb}{0.277018,0.050344,0.375715}%
\pgfsetfillcolor{currentfill}%
\pgfsetfillopacity{0.800000}%
\pgfsetlinewidth{0.000000pt}%
\definecolor{currentstroke}{rgb}{0.000000,0.000000,0.000000}%
\pgfsetstrokecolor{currentstroke}%
\pgfsetdash{}{0pt}%
\pgfpathmoveto{\pgfqpoint{2.999325in}{1.787833in}}%
\pgfpathlineto{\pgfqpoint{3.013045in}{1.777843in}}%
\pgfpathlineto{\pgfqpoint{3.026764in}{1.768082in}}%
\pgfpathlineto{\pgfqpoint{3.040482in}{1.758549in}}%
\pgfpathlineto{\pgfqpoint{3.054198in}{1.749242in}}%
\pgfpathlineto{\pgfqpoint{3.062622in}{1.755033in}}%
\pgfpathlineto{\pgfqpoint{3.071034in}{1.760986in}}%
\pgfpathlineto{\pgfqpoint{3.079437in}{1.767095in}}%
\pgfpathlineto{\pgfqpoint{3.087830in}{1.773355in}}%
\pgfpathlineto{\pgfqpoint{3.074141in}{1.782192in}}%
\pgfpathlineto{\pgfqpoint{3.060450in}{1.791255in}}%
\pgfpathlineto{\pgfqpoint{3.046760in}{1.800545in}}%
\pgfpathlineto{\pgfqpoint{3.033067in}{1.810064in}}%
\pgfpathlineto{\pgfqpoint{3.024648in}{1.804262in}}%
\pgfpathlineto{\pgfqpoint{3.016217in}{1.798619in}}%
\pgfpathlineto{\pgfqpoint{3.007777in}{1.793142in}}%
\pgfpathlineto{\pgfqpoint{2.999325in}{1.787833in}}%
\pgfpathclose%
\pgfusepath{fill}%
\end{pgfscope}%
\begin{pgfscope}%
\pgfpathrectangle{\pgfqpoint{1.150000in}{0.150000in}}{\pgfqpoint{5.700000in}{5.700000in}}%
\pgfusepath{clip}%
\pgfsetbuttcap%
\pgfsetroundjoin%
\definecolor{currentfill}{rgb}{0.278826,0.175490,0.483397}%
\pgfsetfillcolor{currentfill}%
\pgfsetfillopacity{0.800000}%
\pgfsetlinewidth{0.000000pt}%
\definecolor{currentstroke}{rgb}{0.000000,0.000000,0.000000}%
\pgfsetstrokecolor{currentstroke}%
\pgfsetdash{}{0pt}%
\pgfpathmoveto{\pgfqpoint{4.038872in}{2.024893in}}%
\pgfpathlineto{\pgfqpoint{4.052696in}{2.028802in}}%
\pgfpathlineto{\pgfqpoint{4.066529in}{2.032900in}}%
\pgfpathlineto{\pgfqpoint{4.080374in}{2.037186in}}%
\pgfpathlineto{\pgfqpoint{4.094229in}{2.041662in}}%
\pgfpathlineto{\pgfqpoint{4.102220in}{2.053310in}}%
\pgfpathlineto{\pgfqpoint{4.110206in}{2.064902in}}%
\pgfpathlineto{\pgfqpoint{4.118187in}{2.076436in}}%
\pgfpathlineto{\pgfqpoint{4.126163in}{2.087912in}}%
\pgfpathlineto{\pgfqpoint{4.112312in}{2.083285in}}%
\pgfpathlineto{\pgfqpoint{4.098472in}{2.078848in}}%
\pgfpathlineto{\pgfqpoint{4.084643in}{2.074599in}}%
\pgfpathlineto{\pgfqpoint{4.070825in}{2.070540in}}%
\pgfpathlineto{\pgfqpoint{4.062844in}{2.059204in}}%
\pgfpathlineto{\pgfqpoint{4.054858in}{2.047816in}}%
\pgfpathlineto{\pgfqpoint{4.046868in}{2.036379in}}%
\pgfpathlineto{\pgfqpoint{4.038872in}{2.024893in}}%
\pgfpathclose%
\pgfusepath{fill}%
\end{pgfscope}%
\begin{pgfscope}%
\pgfpathrectangle{\pgfqpoint{1.150000in}{0.150000in}}{\pgfqpoint{5.700000in}{5.700000in}}%
\pgfusepath{clip}%
\pgfsetbuttcap%
\pgfsetroundjoin%
\definecolor{currentfill}{rgb}{0.125394,0.574318,0.549086}%
\pgfsetfillcolor{currentfill}%
\pgfsetfillopacity{0.800000}%
\pgfsetlinewidth{0.000000pt}%
\definecolor{currentstroke}{rgb}{0.000000,0.000000,0.000000}%
\pgfsetstrokecolor{currentstroke}%
\pgfsetdash{}{0pt}%
\pgfpathmoveto{\pgfqpoint{5.388515in}{3.122864in}}%
\pgfpathlineto{\pgfqpoint{5.403039in}{3.134730in}}%
\pgfpathlineto{\pgfqpoint{5.417582in}{3.146777in}}%
\pgfpathlineto{\pgfqpoint{5.432145in}{3.159004in}}%
\pgfpathlineto{\pgfqpoint{5.446727in}{3.171411in}}%
\pgfpathlineto{\pgfqpoint{5.454111in}{3.175108in}}%
\pgfpathlineto{\pgfqpoint{5.461488in}{3.178771in}}%
\pgfpathlineto{\pgfqpoint{5.468858in}{3.182404in}}%
\pgfpathlineto{\pgfqpoint{5.476220in}{3.186011in}}%
\pgfpathlineto{\pgfqpoint{5.461660in}{3.174059in}}%
\pgfpathlineto{\pgfqpoint{5.447119in}{3.162285in}}%
\pgfpathlineto{\pgfqpoint{5.432597in}{3.150691in}}%
\pgfpathlineto{\pgfqpoint{5.418095in}{3.139276in}}%
\pgfpathlineto{\pgfqpoint{5.410710in}{3.135204in}}%
\pgfpathlineto{\pgfqpoint{5.403319in}{3.131115in}}%
\pgfpathlineto{\pgfqpoint{5.395921in}{3.127003in}}%
\pgfpathlineto{\pgfqpoint{5.388515in}{3.122864in}}%
\pgfpathclose%
\pgfusepath{fill}%
\end{pgfscope}%
\begin{pgfscope}%
\pgfpathrectangle{\pgfqpoint{1.150000in}{0.150000in}}{\pgfqpoint{5.700000in}{5.700000in}}%
\pgfusepath{clip}%
\pgfsetbuttcap%
\pgfsetroundjoin%
\definecolor{currentfill}{rgb}{0.195860,0.395433,0.555276}%
\pgfsetfillcolor{currentfill}%
\pgfsetfillopacity{0.800000}%
\pgfsetlinewidth{0.000000pt}%
\definecolor{currentstroke}{rgb}{0.000000,0.000000,0.000000}%
\pgfsetstrokecolor{currentstroke}%
\pgfsetdash{}{0pt}%
\pgfpathmoveto{\pgfqpoint{2.277906in}{2.665769in}}%
\pgfpathlineto{\pgfqpoint{2.292048in}{2.641270in}}%
\pgfpathlineto{\pgfqpoint{2.306174in}{2.617116in}}%
\pgfpathlineto{\pgfqpoint{2.320284in}{2.593303in}}%
\pgfpathlineto{\pgfqpoint{2.334380in}{2.569829in}}%
\pgfpathlineto{\pgfqpoint{2.343305in}{2.569150in}}%
\pgfpathlineto{\pgfqpoint{2.352212in}{2.568743in}}%
\pgfpathlineto{\pgfqpoint{2.361100in}{2.568605in}}%
\pgfpathlineto{\pgfqpoint{2.369971in}{2.568729in}}%
\pgfpathlineto{\pgfqpoint{2.355925in}{2.591695in}}%
\pgfpathlineto{\pgfqpoint{2.341865in}{2.614998in}}%
\pgfpathlineto{\pgfqpoint{2.327789in}{2.638641in}}%
\pgfpathlineto{\pgfqpoint{2.313699in}{2.662629in}}%
\pgfpathlineto{\pgfqpoint{2.304779in}{2.663001in}}%
\pgfpathlineto{\pgfqpoint{2.295840in}{2.663645in}}%
\pgfpathlineto{\pgfqpoint{2.286883in}{2.664566in}}%
\pgfpathlineto{\pgfqpoint{2.277906in}{2.665769in}}%
\pgfpathclose%
\pgfusepath{fill}%
\end{pgfscope}%
\begin{pgfscope}%
\pgfpathrectangle{\pgfqpoint{1.150000in}{0.150000in}}{\pgfqpoint{5.700000in}{5.700000in}}%
\pgfusepath{clip}%
\pgfsetbuttcap%
\pgfsetroundjoin%
\definecolor{currentfill}{rgb}{0.278012,0.180367,0.486697}%
\pgfsetfillcolor{currentfill}%
\pgfsetfillopacity{0.800000}%
\pgfsetlinewidth{0.000000pt}%
\definecolor{currentstroke}{rgb}{0.000000,0.000000,0.000000}%
\pgfsetstrokecolor{currentstroke}%
\pgfsetdash{}{0pt}%
\pgfpathmoveto{\pgfqpoint{2.634238in}{2.092979in}}%
\pgfpathlineto{\pgfqpoint{2.648102in}{2.076509in}}%
\pgfpathlineto{\pgfqpoint{2.661959in}{2.060309in}}%
\pgfpathlineto{\pgfqpoint{2.675809in}{2.044378in}}%
\pgfpathlineto{\pgfqpoint{2.689651in}{2.028712in}}%
\pgfpathlineto{\pgfqpoint{2.698327in}{2.030685in}}%
\pgfpathlineto{\pgfqpoint{2.706987in}{2.032890in}}%
\pgfpathlineto{\pgfqpoint{2.715633in}{2.035319in}}%
\pgfpathlineto{\pgfqpoint{2.724265in}{2.037970in}}%
\pgfpathlineto{\pgfqpoint{2.710462in}{2.053119in}}%
\pgfpathlineto{\pgfqpoint{2.696652in}{2.068534in}}%
\pgfpathlineto{\pgfqpoint{2.682835in}{2.084216in}}%
\pgfpathlineto{\pgfqpoint{2.669012in}{2.100168in}}%
\pgfpathlineto{\pgfqpoint{2.660341in}{2.098022in}}%
\pgfpathlineto{\pgfqpoint{2.651655in}{2.096105in}}%
\pgfpathlineto{\pgfqpoint{2.642954in}{2.094422in}}%
\pgfpathlineto{\pgfqpoint{2.634238in}{2.092979in}}%
\pgfpathclose%
\pgfusepath{fill}%
\end{pgfscope}%
\begin{pgfscope}%
\pgfpathrectangle{\pgfqpoint{1.150000in}{0.150000in}}{\pgfqpoint{5.700000in}{5.700000in}}%
\pgfusepath{clip}%
\pgfsetbuttcap%
\pgfsetroundjoin%
\definecolor{currentfill}{rgb}{0.159194,0.482237,0.558073}%
\pgfsetfillcolor{currentfill}%
\pgfsetfillopacity{0.800000}%
\pgfsetlinewidth{0.000000pt}%
\definecolor{currentstroke}{rgb}{0.000000,0.000000,0.000000}%
\pgfsetstrokecolor{currentstroke}%
\pgfsetdash{}{0pt}%
\pgfpathmoveto{\pgfqpoint{5.007622in}{2.833979in}}%
\pgfpathlineto{\pgfqpoint{5.021932in}{2.844698in}}%
\pgfpathlineto{\pgfqpoint{5.036259in}{2.855599in}}%
\pgfpathlineto{\pgfqpoint{5.050604in}{2.866683in}}%
\pgfpathlineto{\pgfqpoint{5.064966in}{2.877948in}}%
\pgfpathlineto{\pgfqpoint{5.072571in}{2.884377in}}%
\pgfpathlineto{\pgfqpoint{5.080169in}{2.890721in}}%
\pgfpathlineto{\pgfqpoint{5.087759in}{2.896983in}}%
\pgfpathlineto{\pgfqpoint{5.095341in}{2.903166in}}%
\pgfpathlineto{\pgfqpoint{5.080991in}{2.892182in}}%
\pgfpathlineto{\pgfqpoint{5.066659in}{2.881380in}}%
\pgfpathlineto{\pgfqpoint{5.052345in}{2.870760in}}%
\pgfpathlineto{\pgfqpoint{5.038047in}{2.860322in}}%
\pgfpathlineto{\pgfqpoint{5.030452in}{2.853845in}}%
\pgfpathlineto{\pgfqpoint{5.022849in}{2.847298in}}%
\pgfpathlineto{\pgfqpoint{5.015239in}{2.840677in}}%
\pgfpathlineto{\pgfqpoint{5.007622in}{2.833979in}}%
\pgfpathclose%
\pgfusepath{fill}%
\end{pgfscope}%
\begin{pgfscope}%
\pgfpathrectangle{\pgfqpoint{1.150000in}{0.150000in}}{\pgfqpoint{5.700000in}{5.700000in}}%
\pgfusepath{clip}%
\pgfsetbuttcap%
\pgfsetroundjoin%
\definecolor{currentfill}{rgb}{0.271828,0.209303,0.504434}%
\pgfsetfillcolor{currentfill}%
\pgfsetfillopacity{0.800000}%
\pgfsetlinewidth{0.000000pt}%
\definecolor{currentstroke}{rgb}{0.000000,0.000000,0.000000}%
\pgfsetstrokecolor{currentstroke}%
\pgfsetdash{}{0pt}%
\pgfpathmoveto{\pgfqpoint{2.578700in}{2.161600in}}%
\pgfpathlineto{\pgfqpoint{2.592597in}{2.144029in}}%
\pgfpathlineto{\pgfqpoint{2.606486in}{2.126736in}}%
\pgfpathlineto{\pgfqpoint{2.620366in}{2.109720in}}%
\pgfpathlineto{\pgfqpoint{2.634238in}{2.092979in}}%
\pgfpathlineto{\pgfqpoint{2.642954in}{2.094422in}}%
\pgfpathlineto{\pgfqpoint{2.651655in}{2.096105in}}%
\pgfpathlineto{\pgfqpoint{2.660341in}{2.098022in}}%
\pgfpathlineto{\pgfqpoint{2.669012in}{2.100168in}}%
\pgfpathlineto{\pgfqpoint{2.655180in}{2.116390in}}%
\pgfpathlineto{\pgfqpoint{2.641342in}{2.132886in}}%
\pgfpathlineto{\pgfqpoint{2.627495in}{2.149657in}}%
\pgfpathlineto{\pgfqpoint{2.613640in}{2.166707in}}%
\pgfpathlineto{\pgfqpoint{2.604928in}{2.165068in}}%
\pgfpathlineto{\pgfqpoint{2.596201in}{2.163667in}}%
\pgfpathlineto{\pgfqpoint{2.587458in}{2.162509in}}%
\pgfpathlineto{\pgfqpoint{2.578700in}{2.161600in}}%
\pgfpathclose%
\pgfusepath{fill}%
\end{pgfscope}%
\begin{pgfscope}%
\pgfpathrectangle{\pgfqpoint{1.150000in}{0.150000in}}{\pgfqpoint{5.700000in}{5.700000in}}%
\pgfusepath{clip}%
\pgfsetbuttcap%
\pgfsetroundjoin%
\definecolor{currentfill}{rgb}{0.269944,0.014625,0.341379}%
\pgfsetfillcolor{currentfill}%
\pgfsetfillopacity{0.800000}%
\pgfsetlinewidth{0.000000pt}%
\definecolor{currentstroke}{rgb}{0.000000,0.000000,0.000000}%
\pgfsetstrokecolor{currentstroke}%
\pgfsetdash{}{0pt}%
\pgfpathmoveto{\pgfqpoint{3.197343in}{1.710653in}}%
\pgfpathlineto{\pgfqpoint{3.211035in}{1.703798in}}%
\pgfpathlineto{\pgfqpoint{3.224728in}{1.697156in}}%
\pgfpathlineto{\pgfqpoint{3.238422in}{1.690728in}}%
\pgfpathlineto{\pgfqpoint{3.252119in}{1.684513in}}%
\pgfpathlineto{\pgfqpoint{3.260431in}{1.692267in}}%
\pgfpathlineto{\pgfqpoint{3.268734in}{1.700139in}}%
\pgfpathlineto{\pgfqpoint{3.277030in}{1.708127in}}%
\pgfpathlineto{\pgfqpoint{3.285318in}{1.716225in}}%
\pgfpathlineto{\pgfqpoint{3.271642in}{1.722005in}}%
\pgfpathlineto{\pgfqpoint{3.257969in}{1.727997in}}%
\pgfpathlineto{\pgfqpoint{3.244298in}{1.734202in}}%
\pgfpathlineto{\pgfqpoint{3.230628in}{1.740623in}}%
\pgfpathlineto{\pgfqpoint{3.222320in}{1.732948in}}%
\pgfpathlineto{\pgfqpoint{3.214003in}{1.725392in}}%
\pgfpathlineto{\pgfqpoint{3.205677in}{1.717959in}}%
\pgfpathlineto{\pgfqpoint{3.197343in}{1.710653in}}%
\pgfpathclose%
\pgfusepath{fill}%
\end{pgfscope}%
\begin{pgfscope}%
\pgfpathrectangle{\pgfqpoint{1.150000in}{0.150000in}}{\pgfqpoint{5.700000in}{5.700000in}}%
\pgfusepath{clip}%
\pgfsetbuttcap%
\pgfsetroundjoin%
\definecolor{currentfill}{rgb}{0.281412,0.155834,0.469201}%
\pgfsetfillcolor{currentfill}%
\pgfsetfillopacity{0.800000}%
\pgfsetlinewidth{0.000000pt}%
\definecolor{currentstroke}{rgb}{0.000000,0.000000,0.000000}%
\pgfsetstrokecolor{currentstroke}%
\pgfsetdash{}{0pt}%
\pgfpathmoveto{\pgfqpoint{2.689651in}{2.028712in}}%
\pgfpathlineto{\pgfqpoint{2.703487in}{2.013310in}}%
\pgfpathlineto{\pgfqpoint{2.717317in}{1.998169in}}%
\pgfpathlineto{\pgfqpoint{2.731140in}{1.983289in}}%
\pgfpathlineto{\pgfqpoint{2.744957in}{1.968667in}}%
\pgfpathlineto{\pgfqpoint{2.753593in}{1.971168in}}%
\pgfpathlineto{\pgfqpoint{2.762215in}{1.973890in}}%
\pgfpathlineto{\pgfqpoint{2.770823in}{1.976830in}}%
\pgfpathlineto{\pgfqpoint{2.779418in}{1.979982in}}%
\pgfpathlineto{\pgfqpoint{2.765638in}{1.994091in}}%
\pgfpathlineto{\pgfqpoint{2.751853in}{2.008457in}}%
\pgfpathlineto{\pgfqpoint{2.738062in}{2.023083in}}%
\pgfpathlineto{\pgfqpoint{2.724265in}{2.037970in}}%
\pgfpathlineto{\pgfqpoint{2.715633in}{2.035319in}}%
\pgfpathlineto{\pgfqpoint{2.706987in}{2.032890in}}%
\pgfpathlineto{\pgfqpoint{2.698327in}{2.030685in}}%
\pgfpathlineto{\pgfqpoint{2.689651in}{2.028712in}}%
\pgfpathclose%
\pgfusepath{fill}%
\end{pgfscope}%
\begin{pgfscope}%
\pgfpathrectangle{\pgfqpoint{1.150000in}{0.150000in}}{\pgfqpoint{5.700000in}{5.700000in}}%
\pgfusepath{clip}%
\pgfsetbuttcap%
\pgfsetroundjoin%
\definecolor{currentfill}{rgb}{0.120565,0.596422,0.543611}%
\pgfsetfillcolor{currentfill}%
\pgfsetfillopacity{0.800000}%
\pgfsetlinewidth{0.000000pt}%
\definecolor{currentstroke}{rgb}{0.000000,0.000000,0.000000}%
\pgfsetstrokecolor{currentstroke}%
\pgfsetdash{}{0pt}%
\pgfpathmoveto{\pgfqpoint{5.476220in}{3.186011in}}%
\pgfpathlineto{\pgfqpoint{5.490800in}{3.198144in}}%
\pgfpathlineto{\pgfqpoint{5.505399in}{3.210456in}}%
\pgfpathlineto{\pgfqpoint{5.520019in}{3.222948in}}%
\pgfpathlineto{\pgfqpoint{5.534659in}{3.235619in}}%
\pgfpathlineto{\pgfqpoint{5.541991in}{3.238731in}}%
\pgfpathlineto{\pgfqpoint{5.549315in}{3.241820in}}%
\pgfpathlineto{\pgfqpoint{5.556633in}{3.244892in}}%
\pgfpathlineto{\pgfqpoint{5.563943in}{3.247952in}}%
\pgfpathlineto{\pgfqpoint{5.549328in}{3.235770in}}%
\pgfpathlineto{\pgfqpoint{5.534732in}{3.223767in}}%
\pgfpathlineto{\pgfqpoint{5.520156in}{3.211942in}}%
\pgfpathlineto{\pgfqpoint{5.505600in}{3.200296in}}%
\pgfpathlineto{\pgfqpoint{5.498265in}{3.196736in}}%
\pgfpathlineto{\pgfqpoint{5.490923in}{3.193172in}}%
\pgfpathlineto{\pgfqpoint{5.483575in}{3.189599in}}%
\pgfpathlineto{\pgfqpoint{5.476220in}{3.186011in}}%
\pgfpathclose%
\pgfusepath{fill}%
\end{pgfscope}%
\begin{pgfscope}%
\pgfpathrectangle{\pgfqpoint{1.150000in}{0.150000in}}{\pgfqpoint{5.700000in}{5.700000in}}%
\pgfusepath{clip}%
\pgfsetbuttcap%
\pgfsetroundjoin%
\definecolor{currentfill}{rgb}{0.271828,0.209303,0.504434}%
\pgfsetfillcolor{currentfill}%
\pgfsetfillopacity{0.800000}%
\pgfsetlinewidth{0.000000pt}%
\definecolor{currentstroke}{rgb}{0.000000,0.000000,0.000000}%
\pgfsetstrokecolor{currentstroke}%
\pgfsetdash{}{0pt}%
\pgfpathmoveto{\pgfqpoint{4.126163in}{2.087912in}}%
\pgfpathlineto{\pgfqpoint{4.140024in}{2.092726in}}%
\pgfpathlineto{\pgfqpoint{4.153898in}{2.097729in}}%
\pgfpathlineto{\pgfqpoint{4.167782in}{2.102919in}}%
\pgfpathlineto{\pgfqpoint{4.181678in}{2.108297in}}%
\pgfpathlineto{\pgfqpoint{4.189645in}{2.119844in}}%
\pgfpathlineto{\pgfqpoint{4.197607in}{2.131321in}}%
\pgfpathlineto{\pgfqpoint{4.205564in}{2.142729in}}%
\pgfpathlineto{\pgfqpoint{4.213515in}{2.154067in}}%
\pgfpathlineto{\pgfqpoint{4.199623in}{2.148571in}}%
\pgfpathlineto{\pgfqpoint{4.185743in}{2.143261in}}%
\pgfpathlineto{\pgfqpoint{4.171874in}{2.138140in}}%
\pgfpathlineto{\pgfqpoint{4.158016in}{2.133207in}}%
\pgfpathlineto{\pgfqpoint{4.150060in}{2.121975in}}%
\pgfpathlineto{\pgfqpoint{4.142100in}{2.110682in}}%
\pgfpathlineto{\pgfqpoint{4.134134in}{2.099327in}}%
\pgfpathlineto{\pgfqpoint{4.126163in}{2.087912in}}%
\pgfpathclose%
\pgfusepath{fill}%
\end{pgfscope}%
\begin{pgfscope}%
\pgfpathrectangle{\pgfqpoint{1.150000in}{0.150000in}}{\pgfqpoint{5.700000in}{5.700000in}}%
\pgfusepath{clip}%
\pgfsetbuttcap%
\pgfsetroundjoin%
\definecolor{currentfill}{rgb}{0.235526,0.309527,0.542944}%
\pgfsetfillcolor{currentfill}%
\pgfsetfillopacity{0.800000}%
\pgfsetlinewidth{0.000000pt}%
\definecolor{currentstroke}{rgb}{0.000000,0.000000,0.000000}%
\pgfsetstrokecolor{currentstroke}%
\pgfsetdash{}{0pt}%
\pgfpathmoveto{\pgfqpoint{4.420008in}{2.336493in}}%
\pgfpathlineto{\pgfqpoint{4.434002in}{2.343830in}}%
\pgfpathlineto{\pgfqpoint{4.448010in}{2.351354in}}%
\pgfpathlineto{\pgfqpoint{4.462032in}{2.359062in}}%
\pgfpathlineto{\pgfqpoint{4.476067in}{2.366956in}}%
\pgfpathlineto{\pgfqpoint{4.483936in}{2.377356in}}%
\pgfpathlineto{\pgfqpoint{4.491799in}{2.387660in}}%
\pgfpathlineto{\pgfqpoint{4.499656in}{2.397869in}}%
\pgfpathlineto{\pgfqpoint{4.507507in}{2.407983in}}%
\pgfpathlineto{\pgfqpoint{4.493476in}{2.400101in}}%
\pgfpathlineto{\pgfqpoint{4.479459in}{2.392404in}}%
\pgfpathlineto{\pgfqpoint{4.465456in}{2.384892in}}%
\pgfpathlineto{\pgfqpoint{4.451466in}{2.377566in}}%
\pgfpathlineto{\pgfqpoint{4.443610in}{2.367428in}}%
\pgfpathlineto{\pgfqpoint{4.435748in}{2.357204in}}%
\pgfpathlineto{\pgfqpoint{4.427881in}{2.346892in}}%
\pgfpathlineto{\pgfqpoint{4.420008in}{2.336493in}}%
\pgfpathclose%
\pgfusepath{fill}%
\end{pgfscope}%
\begin{pgfscope}%
\pgfpathrectangle{\pgfqpoint{1.150000in}{0.150000in}}{\pgfqpoint{5.700000in}{5.700000in}}%
\pgfusepath{clip}%
\pgfsetbuttcap%
\pgfsetroundjoin%
\definecolor{currentfill}{rgb}{0.262138,0.242286,0.520837}%
\pgfsetfillcolor{currentfill}%
\pgfsetfillopacity{0.800000}%
\pgfsetlinewidth{0.000000pt}%
\definecolor{currentstroke}{rgb}{0.000000,0.000000,0.000000}%
\pgfsetstrokecolor{currentstroke}%
\pgfsetdash{}{0pt}%
\pgfpathmoveto{\pgfqpoint{2.523019in}{2.234716in}}%
\pgfpathlineto{\pgfqpoint{2.536953in}{2.216008in}}%
\pgfpathlineto{\pgfqpoint{2.550878in}{2.197587in}}%
\pgfpathlineto{\pgfqpoint{2.564794in}{2.179452in}}%
\pgfpathlineto{\pgfqpoint{2.578700in}{2.161600in}}%
\pgfpathlineto{\pgfqpoint{2.587458in}{2.162509in}}%
\pgfpathlineto{\pgfqpoint{2.596201in}{2.163667in}}%
\pgfpathlineto{\pgfqpoint{2.604928in}{2.165068in}}%
\pgfpathlineto{\pgfqpoint{2.613640in}{2.166707in}}%
\pgfpathlineto{\pgfqpoint{2.599777in}{2.184036in}}%
\pgfpathlineto{\pgfqpoint{2.585905in}{2.201647in}}%
\pgfpathlineto{\pgfqpoint{2.572024in}{2.219544in}}%
\pgfpathlineto{\pgfqpoint{2.558134in}{2.237727in}}%
\pgfpathlineto{\pgfqpoint{2.549379in}{2.236599in}}%
\pgfpathlineto{\pgfqpoint{2.540609in}{2.235718in}}%
\pgfpathlineto{\pgfqpoint{2.531822in}{2.235089in}}%
\pgfpathlineto{\pgfqpoint{2.523019in}{2.234716in}}%
\pgfpathclose%
\pgfusepath{fill}%
\end{pgfscope}%
\begin{pgfscope}%
\pgfpathrectangle{\pgfqpoint{1.150000in}{0.150000in}}{\pgfqpoint{5.700000in}{5.700000in}}%
\pgfusepath{clip}%
\pgfsetbuttcap%
\pgfsetroundjoin%
\definecolor{currentfill}{rgb}{0.192357,0.403199,0.555836}%
\pgfsetfillcolor{currentfill}%
\pgfsetfillopacity{0.800000}%
\pgfsetlinewidth{0.000000pt}%
\definecolor{currentstroke}{rgb}{0.000000,0.000000,0.000000}%
\pgfsetstrokecolor{currentstroke}%
\pgfsetdash{}{0pt}%
\pgfpathmoveto{\pgfqpoint{4.713886in}{2.590135in}}%
\pgfpathlineto{\pgfqpoint{4.728034in}{2.599443in}}%
\pgfpathlineto{\pgfqpoint{4.742198in}{2.608934in}}%
\pgfpathlineto{\pgfqpoint{4.756378in}{2.618610in}}%
\pgfpathlineto{\pgfqpoint{4.770574in}{2.628469in}}%
\pgfpathlineto{\pgfqpoint{4.778324in}{2.637050in}}%
\pgfpathlineto{\pgfqpoint{4.786068in}{2.645529in}}%
\pgfpathlineto{\pgfqpoint{4.793805in}{2.653909in}}%
\pgfpathlineto{\pgfqpoint{4.801534in}{2.662190in}}%
\pgfpathlineto{\pgfqpoint{4.787346in}{2.652477in}}%
\pgfpathlineto{\pgfqpoint{4.773174in}{2.642947in}}%
\pgfpathlineto{\pgfqpoint{4.759018in}{2.633600in}}%
\pgfpathlineto{\pgfqpoint{4.744877in}{2.624437in}}%
\pgfpathlineto{\pgfqpoint{4.737139in}{2.615998in}}%
\pgfpathlineto{\pgfqpoint{4.729395in}{2.607469in}}%
\pgfpathlineto{\pgfqpoint{4.721644in}{2.598849in}}%
\pgfpathlineto{\pgfqpoint{4.713886in}{2.590135in}}%
\pgfpathclose%
\pgfusepath{fill}%
\end{pgfscope}%
\begin{pgfscope}%
\pgfpathrectangle{\pgfqpoint{1.150000in}{0.150000in}}{\pgfqpoint{5.700000in}{5.700000in}}%
\pgfusepath{clip}%
\pgfsetbuttcap%
\pgfsetroundjoin%
\definecolor{currentfill}{rgb}{0.283072,0.130895,0.449241}%
\pgfsetfillcolor{currentfill}%
\pgfsetfillopacity{0.800000}%
\pgfsetlinewidth{0.000000pt}%
\definecolor{currentstroke}{rgb}{0.000000,0.000000,0.000000}%
\pgfsetstrokecolor{currentstroke}%
\pgfsetdash{}{0pt}%
\pgfpathmoveto{\pgfqpoint{2.744957in}{1.968667in}}%
\pgfpathlineto{\pgfqpoint{2.758768in}{1.954301in}}%
\pgfpathlineto{\pgfqpoint{2.772574in}{1.940189in}}%
\pgfpathlineto{\pgfqpoint{2.786374in}{1.926330in}}%
\pgfpathlineto{\pgfqpoint{2.800170in}{1.912721in}}%
\pgfpathlineto{\pgfqpoint{2.808768in}{1.915747in}}%
\pgfpathlineto{\pgfqpoint{2.817354in}{1.918985in}}%
\pgfpathlineto{\pgfqpoint{2.825926in}{1.922432in}}%
\pgfpathlineto{\pgfqpoint{2.834485in}{1.926083in}}%
\pgfpathlineto{\pgfqpoint{2.820725in}{1.939181in}}%
\pgfpathlineto{\pgfqpoint{2.806961in}{1.952529in}}%
\pgfpathlineto{\pgfqpoint{2.793192in}{1.966128in}}%
\pgfpathlineto{\pgfqpoint{2.779418in}{1.979982in}}%
\pgfpathlineto{\pgfqpoint{2.770823in}{1.976830in}}%
\pgfpathlineto{\pgfqpoint{2.762215in}{1.973890in}}%
\pgfpathlineto{\pgfqpoint{2.753593in}{1.971168in}}%
\pgfpathlineto{\pgfqpoint{2.744957in}{1.968667in}}%
\pgfpathclose%
\pgfusepath{fill}%
\end{pgfscope}%
\begin{pgfscope}%
\pgfpathrectangle{\pgfqpoint{1.150000in}{0.150000in}}{\pgfqpoint{5.700000in}{5.700000in}}%
\pgfusepath{clip}%
\pgfsetbuttcap%
\pgfsetroundjoin%
\definecolor{currentfill}{rgb}{0.269944,0.014625,0.341379}%
\pgfsetfillcolor{currentfill}%
\pgfsetfillopacity{0.800000}%
\pgfsetlinewidth{0.000000pt}%
\definecolor{currentstroke}{rgb}{0.000000,0.000000,0.000000}%
\pgfsetstrokecolor{currentstroke}%
\pgfsetdash{}{0pt}%
\pgfpathmoveto{\pgfqpoint{3.340042in}{1.695211in}}%
\pgfpathlineto{\pgfqpoint{3.353729in}{1.690480in}}%
\pgfpathlineto{\pgfqpoint{3.367420in}{1.685955in}}%
\pgfpathlineto{\pgfqpoint{3.381114in}{1.681636in}}%
\pgfpathlineto{\pgfqpoint{3.394812in}{1.677523in}}%
\pgfpathlineto{\pgfqpoint{3.403054in}{1.686558in}}%
\pgfpathlineto{\pgfqpoint{3.411289in}{1.695680in}}%
\pgfpathlineto{\pgfqpoint{3.419518in}{1.704885in}}%
\pgfpathlineto{\pgfqpoint{3.427739in}{1.714169in}}%
\pgfpathlineto{\pgfqpoint{3.414058in}{1.717880in}}%
\pgfpathlineto{\pgfqpoint{3.400382in}{1.721796in}}%
\pgfpathlineto{\pgfqpoint{3.386708in}{1.725917in}}%
\pgfpathlineto{\pgfqpoint{3.373038in}{1.730246in}}%
\pgfpathlineto{\pgfqpoint{3.364800in}{1.721352in}}%
\pgfpathlineto{\pgfqpoint{3.356555in}{1.712546in}}%
\pgfpathlineto{\pgfqpoint{3.348302in}{1.703831in}}%
\pgfpathlineto{\pgfqpoint{3.340042in}{1.695211in}}%
\pgfpathclose%
\pgfusepath{fill}%
\end{pgfscope}%
\begin{pgfscope}%
\pgfpathrectangle{\pgfqpoint{1.150000in}{0.150000in}}{\pgfqpoint{5.700000in}{5.700000in}}%
\pgfusepath{clip}%
\pgfsetbuttcap%
\pgfsetroundjoin%
\definecolor{currentfill}{rgb}{0.273809,0.031497,0.358853}%
\pgfsetfillcolor{currentfill}%
\pgfsetfillopacity{0.800000}%
\pgfsetlinewidth{0.000000pt}%
\definecolor{currentstroke}{rgb}{0.000000,0.000000,0.000000}%
\pgfsetstrokecolor{currentstroke}%
\pgfsetdash{}{0pt}%
\pgfpathmoveto{\pgfqpoint{3.054198in}{1.749242in}}%
\pgfpathlineto{\pgfqpoint{3.067914in}{1.740160in}}%
\pgfpathlineto{\pgfqpoint{3.081629in}{1.731302in}}%
\pgfpathlineto{\pgfqpoint{3.095343in}{1.722666in}}%
\pgfpathlineto{\pgfqpoint{3.109058in}{1.714252in}}%
\pgfpathlineto{\pgfqpoint{3.117454in}{1.720525in}}%
\pgfpathlineto{\pgfqpoint{3.125840in}{1.726952in}}%
\pgfpathlineto{\pgfqpoint{3.134217in}{1.733526in}}%
\pgfpathlineto{\pgfqpoint{3.142584in}{1.740244in}}%
\pgfpathlineto{\pgfqpoint{3.128896in}{1.748189in}}%
\pgfpathlineto{\pgfqpoint{3.115207in}{1.756355in}}%
\pgfpathlineto{\pgfqpoint{3.101519in}{1.764743in}}%
\pgfpathlineto{\pgfqpoint{3.087830in}{1.773355in}}%
\pgfpathlineto{\pgfqpoint{3.079437in}{1.767095in}}%
\pgfpathlineto{\pgfqpoint{3.071034in}{1.760986in}}%
\pgfpathlineto{\pgfqpoint{3.062622in}{1.755033in}}%
\pgfpathlineto{\pgfqpoint{3.054198in}{1.749242in}}%
\pgfpathclose%
\pgfusepath{fill}%
\end{pgfscope}%
\begin{pgfscope}%
\pgfpathrectangle{\pgfqpoint{1.150000in}{0.150000in}}{\pgfqpoint{5.700000in}{5.700000in}}%
\pgfusepath{clip}%
\pgfsetbuttcap%
\pgfsetroundjoin%
\definecolor{currentfill}{rgb}{0.250425,0.274290,0.533103}%
\pgfsetfillcolor{currentfill}%
\pgfsetfillopacity{0.800000}%
\pgfsetlinewidth{0.000000pt}%
\definecolor{currentstroke}{rgb}{0.000000,0.000000,0.000000}%
\pgfsetstrokecolor{currentstroke}%
\pgfsetdash{}{0pt}%
\pgfpathmoveto{\pgfqpoint{2.467177in}{2.312482in}}%
\pgfpathlineto{\pgfqpoint{2.481154in}{2.292596in}}%
\pgfpathlineto{\pgfqpoint{2.495119in}{2.273008in}}%
\pgfpathlineto{\pgfqpoint{2.509074in}{2.253716in}}%
\pgfpathlineto{\pgfqpoint{2.523019in}{2.234716in}}%
\pgfpathlineto{\pgfqpoint{2.531822in}{2.235089in}}%
\pgfpathlineto{\pgfqpoint{2.540609in}{2.235718in}}%
\pgfpathlineto{\pgfqpoint{2.549379in}{2.236599in}}%
\pgfpathlineto{\pgfqpoint{2.558134in}{2.237727in}}%
\pgfpathlineto{\pgfqpoint{2.544234in}{2.256200in}}%
\pgfpathlineto{\pgfqpoint{2.530324in}{2.274965in}}%
\pgfpathlineto{\pgfqpoint{2.516404in}{2.294024in}}%
\pgfpathlineto{\pgfqpoint{2.502474in}{2.313381in}}%
\pgfpathlineto{\pgfqpoint{2.493675in}{2.312767in}}%
\pgfpathlineto{\pgfqpoint{2.484860in}{2.312410in}}%
\pgfpathlineto{\pgfqpoint{2.476027in}{2.312313in}}%
\pgfpathlineto{\pgfqpoint{2.467177in}{2.312482in}}%
\pgfpathclose%
\pgfusepath{fill}%
\end{pgfscope}%
\begin{pgfscope}%
\pgfpathrectangle{\pgfqpoint{1.150000in}{0.150000in}}{\pgfqpoint{5.700000in}{5.700000in}}%
\pgfusepath{clip}%
\pgfsetbuttcap%
\pgfsetroundjoin%
\definecolor{currentfill}{rgb}{0.119699,0.618490,0.536347}%
\pgfsetfillcolor{currentfill}%
\pgfsetfillopacity{0.800000}%
\pgfsetlinewidth{0.000000pt}%
\definecolor{currentstroke}{rgb}{0.000000,0.000000,0.000000}%
\pgfsetstrokecolor{currentstroke}%
\pgfsetdash{}{0pt}%
\pgfpathmoveto{\pgfqpoint{5.563943in}{3.247952in}}%
\pgfpathlineto{\pgfqpoint{5.578579in}{3.260314in}}%
\pgfpathlineto{\pgfqpoint{5.593235in}{3.272855in}}%
\pgfpathlineto{\pgfqpoint{5.607911in}{3.285575in}}%
\pgfpathlineto{\pgfqpoint{5.622608in}{3.298476in}}%
\pgfpathlineto{\pgfqpoint{5.629886in}{3.301019in}}%
\pgfpathlineto{\pgfqpoint{5.637157in}{3.303553in}}%
\pgfpathlineto{\pgfqpoint{5.644421in}{3.306086in}}%
\pgfpathlineto{\pgfqpoint{5.651678in}{3.308621in}}%
\pgfpathlineto{\pgfqpoint{5.637008in}{3.296245in}}%
\pgfpathlineto{\pgfqpoint{5.622358in}{3.284048in}}%
\pgfpathlineto{\pgfqpoint{5.607729in}{3.272030in}}%
\pgfpathlineto{\pgfqpoint{5.593119in}{3.260189in}}%
\pgfpathlineto{\pgfqpoint{5.585835in}{3.257119in}}%
\pgfpathlineto{\pgfqpoint{5.578544in}{3.254061in}}%
\pgfpathlineto{\pgfqpoint{5.571247in}{3.251007in}}%
\pgfpathlineto{\pgfqpoint{5.563943in}{3.247952in}}%
\pgfpathclose%
\pgfusepath{fill}%
\end{pgfscope}%
\begin{pgfscope}%
\pgfpathrectangle{\pgfqpoint{1.150000in}{0.150000in}}{\pgfqpoint{5.700000in}{5.700000in}}%
\pgfusepath{clip}%
\pgfsetbuttcap%
\pgfsetroundjoin%
\definecolor{currentfill}{rgb}{0.283091,0.110553,0.431554}%
\pgfsetfillcolor{currentfill}%
\pgfsetfillopacity{0.800000}%
\pgfsetlinewidth{0.000000pt}%
\definecolor{currentstroke}{rgb}{0.000000,0.000000,0.000000}%
\pgfsetstrokecolor{currentstroke}%
\pgfsetdash{}{0pt}%
\pgfpathmoveto{\pgfqpoint{2.800170in}{1.912721in}}%
\pgfpathlineto{\pgfqpoint{2.813960in}{1.899362in}}%
\pgfpathlineto{\pgfqpoint{2.827746in}{1.886250in}}%
\pgfpathlineto{\pgfqpoint{2.841528in}{1.873384in}}%
\pgfpathlineto{\pgfqpoint{2.855306in}{1.860762in}}%
\pgfpathlineto{\pgfqpoint{2.863869in}{1.864309in}}%
\pgfpathlineto{\pgfqpoint{2.872419in}{1.868061in}}%
\pgfpathlineto{\pgfqpoint{2.880957in}{1.872013in}}%
\pgfpathlineto{\pgfqpoint{2.889482in}{1.876160in}}%
\pgfpathlineto{\pgfqpoint{2.875739in}{1.888274in}}%
\pgfpathlineto{\pgfqpoint{2.861991in}{1.900631in}}%
\pgfpathlineto{\pgfqpoint{2.848240in}{1.913234in}}%
\pgfpathlineto{\pgfqpoint{2.834485in}{1.926083in}}%
\pgfpathlineto{\pgfqpoint{2.825926in}{1.922432in}}%
\pgfpathlineto{\pgfqpoint{2.817354in}{1.918985in}}%
\pgfpathlineto{\pgfqpoint{2.808768in}{1.915747in}}%
\pgfpathlineto{\pgfqpoint{2.800170in}{1.912721in}}%
\pgfpathclose%
\pgfusepath{fill}%
\end{pgfscope}%
\begin{pgfscope}%
\pgfpathrectangle{\pgfqpoint{1.150000in}{0.150000in}}{\pgfqpoint{5.700000in}{5.700000in}}%
\pgfusepath{clip}%
\pgfsetbuttcap%
\pgfsetroundjoin%
\definecolor{currentfill}{rgb}{0.278791,0.062145,0.386592}%
\pgfsetfillcolor{currentfill}%
\pgfsetfillopacity{0.800000}%
\pgfsetlinewidth{0.000000pt}%
\definecolor{currentstroke}{rgb}{0.000000,0.000000,0.000000}%
\pgfsetstrokecolor{currentstroke}%
\pgfsetdash{}{0pt}%
\pgfpathmoveto{\pgfqpoint{3.657477in}{1.770626in}}%
\pgfpathlineto{\pgfqpoint{3.671200in}{1.770228in}}%
\pgfpathlineto{\pgfqpoint{3.684929in}{1.770026in}}%
\pgfpathlineto{\pgfqpoint{3.698665in}{1.770019in}}%
\pgfpathlineto{\pgfqpoint{3.712408in}{1.770207in}}%
\pgfpathlineto{\pgfqpoint{3.720525in}{1.781334in}}%
\pgfpathlineto{\pgfqpoint{3.728636in}{1.792476in}}%
\pgfpathlineto{\pgfqpoint{3.736741in}{1.803630in}}%
\pgfpathlineto{\pgfqpoint{3.744842in}{1.814795in}}%
\pgfpathlineto{\pgfqpoint{3.731108in}{1.814298in}}%
\pgfpathlineto{\pgfqpoint{3.717381in}{1.813997in}}%
\pgfpathlineto{\pgfqpoint{3.703662in}{1.813890in}}%
\pgfpathlineto{\pgfqpoint{3.689949in}{1.813980in}}%
\pgfpathlineto{\pgfqpoint{3.681839in}{1.803112in}}%
\pgfpathlineto{\pgfqpoint{3.673724in}{1.792262in}}%
\pgfpathlineto{\pgfqpoint{3.665603in}{1.781432in}}%
\pgfpathlineto{\pgfqpoint{3.657477in}{1.770626in}}%
\pgfpathclose%
\pgfusepath{fill}%
\end{pgfscope}%
\begin{pgfscope}%
\pgfpathrectangle{\pgfqpoint{1.150000in}{0.150000in}}{\pgfqpoint{5.700000in}{5.700000in}}%
\pgfusepath{clip}%
\pgfsetbuttcap%
\pgfsetroundjoin%
\definecolor{currentfill}{rgb}{0.281446,0.084320,0.407414}%
\pgfsetfillcolor{currentfill}%
\pgfsetfillopacity{0.800000}%
\pgfsetlinewidth{0.000000pt}%
\definecolor{currentstroke}{rgb}{0.000000,0.000000,0.000000}%
\pgfsetstrokecolor{currentstroke}%
\pgfsetdash{}{0pt}%
\pgfpathmoveto{\pgfqpoint{3.744842in}{1.814795in}}%
\pgfpathlineto{\pgfqpoint{3.758583in}{1.815485in}}%
\pgfpathlineto{\pgfqpoint{3.772332in}{1.816370in}}%
\pgfpathlineto{\pgfqpoint{3.786089in}{1.817447in}}%
\pgfpathlineto{\pgfqpoint{3.799854in}{1.818717in}}%
\pgfpathlineto{\pgfqpoint{3.807941in}{1.830178in}}%
\pgfpathlineto{\pgfqpoint{3.816023in}{1.841636in}}%
\pgfpathlineto{\pgfqpoint{3.824100in}{1.853088in}}%
\pgfpathlineto{\pgfqpoint{3.832172in}{1.864532in}}%
\pgfpathlineto{\pgfqpoint{3.818414in}{1.862984in}}%
\pgfpathlineto{\pgfqpoint{3.804665in}{1.861629in}}%
\pgfpathlineto{\pgfqpoint{3.790924in}{1.860468in}}%
\pgfpathlineto{\pgfqpoint{3.777191in}{1.859500in}}%
\pgfpathlineto{\pgfqpoint{3.769112in}{1.848321in}}%
\pgfpathlineto{\pgfqpoint{3.761027in}{1.837143in}}%
\pgfpathlineto{\pgfqpoint{3.752937in}{1.825966in}}%
\pgfpathlineto{\pgfqpoint{3.744842in}{1.814795in}}%
\pgfpathclose%
\pgfusepath{fill}%
\end{pgfscope}%
\begin{pgfscope}%
\pgfpathrectangle{\pgfqpoint{1.150000in}{0.150000in}}{\pgfqpoint{5.700000in}{5.700000in}}%
\pgfusepath{clip}%
\pgfsetbuttcap%
\pgfsetroundjoin%
\definecolor{currentfill}{rgb}{0.149039,0.508051,0.557250}%
\pgfsetfillcolor{currentfill}%
\pgfsetfillopacity{0.800000}%
\pgfsetlinewidth{0.000000pt}%
\definecolor{currentstroke}{rgb}{0.000000,0.000000,0.000000}%
\pgfsetstrokecolor{currentstroke}%
\pgfsetdash{}{0pt}%
\pgfpathmoveto{\pgfqpoint{5.095341in}{2.903166in}}%
\pgfpathlineto{\pgfqpoint{5.109709in}{2.914331in}}%
\pgfpathlineto{\pgfqpoint{5.124095in}{2.925678in}}%
\pgfpathlineto{\pgfqpoint{5.138499in}{2.937207in}}%
\pgfpathlineto{\pgfqpoint{5.152921in}{2.948919in}}%
\pgfpathlineto{\pgfqpoint{5.160482in}{2.954722in}}%
\pgfpathlineto{\pgfqpoint{5.168036in}{2.960446in}}%
\pgfpathlineto{\pgfqpoint{5.175581in}{2.966094in}}%
\pgfpathlineto{\pgfqpoint{5.183119in}{2.971670in}}%
\pgfpathlineto{\pgfqpoint{5.168711in}{2.960276in}}%
\pgfpathlineto{\pgfqpoint{5.154322in}{2.949063in}}%
\pgfpathlineto{\pgfqpoint{5.139950in}{2.938032in}}%
\pgfpathlineto{\pgfqpoint{5.125597in}{2.927182in}}%
\pgfpathlineto{\pgfqpoint{5.118044in}{2.921278in}}%
\pgfpathlineto{\pgfqpoint{5.110484in}{2.915310in}}%
\pgfpathlineto{\pgfqpoint{5.102916in}{2.909274in}}%
\pgfpathlineto{\pgfqpoint{5.095341in}{2.903166in}}%
\pgfpathclose%
\pgfusepath{fill}%
\end{pgfscope}%
\begin{pgfscope}%
\pgfpathrectangle{\pgfqpoint{1.150000in}{0.150000in}}{\pgfqpoint{5.700000in}{5.700000in}}%
\pgfusepath{clip}%
\pgfsetbuttcap%
\pgfsetroundjoin%
\definecolor{currentfill}{rgb}{0.263663,0.237631,0.518762}%
\pgfsetfillcolor{currentfill}%
\pgfsetfillopacity{0.800000}%
\pgfsetlinewidth{0.000000pt}%
\definecolor{currentstroke}{rgb}{0.000000,0.000000,0.000000}%
\pgfsetstrokecolor{currentstroke}%
\pgfsetdash{}{0pt}%
\pgfpathmoveto{\pgfqpoint{4.213515in}{2.154067in}}%
\pgfpathlineto{\pgfqpoint{4.227419in}{2.159752in}}%
\pgfpathlineto{\pgfqpoint{4.241335in}{2.165623in}}%
\pgfpathlineto{\pgfqpoint{4.255264in}{2.171681in}}%
\pgfpathlineto{\pgfqpoint{4.269204in}{2.177926in}}%
\pgfpathlineto{\pgfqpoint{4.277147in}{2.189292in}}%
\pgfpathlineto{\pgfqpoint{4.285084in}{2.200578in}}%
\pgfpathlineto{\pgfqpoint{4.293017in}{2.211784in}}%
\pgfpathlineto{\pgfqpoint{4.300943in}{2.222910in}}%
\pgfpathlineto{\pgfqpoint{4.287006in}{2.216579in}}%
\pgfpathlineto{\pgfqpoint{4.273082in}{2.210434in}}%
\pgfpathlineto{\pgfqpoint{4.259170in}{2.204476in}}%
\pgfpathlineto{\pgfqpoint{4.245270in}{2.198705in}}%
\pgfpathlineto{\pgfqpoint{4.237339in}{2.187654in}}%
\pgfpathlineto{\pgfqpoint{4.229403in}{2.176530in}}%
\pgfpathlineto{\pgfqpoint{4.221462in}{2.165334in}}%
\pgfpathlineto{\pgfqpoint{4.213515in}{2.154067in}}%
\pgfpathclose%
\pgfusepath{fill}%
\end{pgfscope}%
\begin{pgfscope}%
\pgfpathrectangle{\pgfqpoint{1.150000in}{0.150000in}}{\pgfqpoint{5.700000in}{5.700000in}}%
\pgfusepath{clip}%
\pgfsetbuttcap%
\pgfsetroundjoin%
\definecolor{currentfill}{rgb}{0.179019,0.433756,0.557430}%
\pgfsetfillcolor{currentfill}%
\pgfsetfillopacity{0.800000}%
\pgfsetlinewidth{0.000000pt}%
\definecolor{currentstroke}{rgb}{0.000000,0.000000,0.000000}%
\pgfsetstrokecolor{currentstroke}%
\pgfsetdash{}{0pt}%
\pgfpathmoveto{\pgfqpoint{2.221173in}{2.767290in}}%
\pgfpathlineto{\pgfqpoint{2.235382in}{2.741374in}}%
\pgfpathlineto{\pgfqpoint{2.249574in}{2.715818in}}%
\pgfpathlineto{\pgfqpoint{2.263748in}{2.690617in}}%
\pgfpathlineto{\pgfqpoint{2.277906in}{2.665769in}}%
\pgfpathlineto{\pgfqpoint{2.286883in}{2.664566in}}%
\pgfpathlineto{\pgfqpoint{2.295840in}{2.663645in}}%
\pgfpathlineto{\pgfqpoint{2.304779in}{2.663001in}}%
\pgfpathlineto{\pgfqpoint{2.313699in}{2.662629in}}%
\pgfpathlineto{\pgfqpoint{2.299593in}{2.686963in}}%
\pgfpathlineto{\pgfqpoint{2.285470in}{2.711648in}}%
\pgfpathlineto{\pgfqpoint{2.271332in}{2.736688in}}%
\pgfpathlineto{\pgfqpoint{2.257176in}{2.762086in}}%
\pgfpathlineto{\pgfqpoint{2.248205in}{2.762960in}}%
\pgfpathlineto{\pgfqpoint{2.239214in}{2.764115in}}%
\pgfpathlineto{\pgfqpoint{2.230204in}{2.765557in}}%
\pgfpathlineto{\pgfqpoint{2.221173in}{2.767290in}}%
\pgfpathclose%
\pgfusepath{fill}%
\end{pgfscope}%
\begin{pgfscope}%
\pgfpathrectangle{\pgfqpoint{1.150000in}{0.150000in}}{\pgfqpoint{5.700000in}{5.700000in}}%
\pgfusepath{clip}%
\pgfsetbuttcap%
\pgfsetroundjoin%
\definecolor{currentfill}{rgb}{0.283091,0.110553,0.431554}%
\pgfsetfillcolor{currentfill}%
\pgfsetfillopacity{0.800000}%
\pgfsetlinewidth{0.000000pt}%
\definecolor{currentstroke}{rgb}{0.000000,0.000000,0.000000}%
\pgfsetstrokecolor{currentstroke}%
\pgfsetdash{}{0pt}%
\pgfpathmoveto{\pgfqpoint{3.832172in}{1.864532in}}%
\pgfpathlineto{\pgfqpoint{3.845937in}{1.866272in}}%
\pgfpathlineto{\pgfqpoint{3.859711in}{1.868204in}}%
\pgfpathlineto{\pgfqpoint{3.873494in}{1.870327in}}%
\pgfpathlineto{\pgfqpoint{3.887286in}{1.872642in}}%
\pgfpathlineto{\pgfqpoint{3.895346in}{1.884333in}}%
\pgfpathlineto{\pgfqpoint{3.903401in}{1.896004in}}%
\pgfpathlineto{\pgfqpoint{3.911451in}{1.907652in}}%
\pgfpathlineto{\pgfqpoint{3.919496in}{1.919276in}}%
\pgfpathlineto{\pgfqpoint{3.905711in}{1.916715in}}%
\pgfpathlineto{\pgfqpoint{3.891934in}{1.914346in}}%
\pgfpathlineto{\pgfqpoint{3.878167in}{1.912168in}}%
\pgfpathlineto{\pgfqpoint{3.864409in}{1.910182in}}%
\pgfpathlineto{\pgfqpoint{3.856357in}{1.898792in}}%
\pgfpathlineto{\pgfqpoint{3.848300in}{1.887386in}}%
\pgfpathlineto{\pgfqpoint{3.840238in}{1.875965in}}%
\pgfpathlineto{\pgfqpoint{3.832172in}{1.864532in}}%
\pgfpathclose%
\pgfusepath{fill}%
\end{pgfscope}%
\begin{pgfscope}%
\pgfpathrectangle{\pgfqpoint{1.150000in}{0.150000in}}{\pgfqpoint{5.700000in}{5.700000in}}%
\pgfusepath{clip}%
\pgfsetbuttcap%
\pgfsetroundjoin%
\definecolor{currentfill}{rgb}{0.274952,0.037752,0.364543}%
\pgfsetfillcolor{currentfill}%
\pgfsetfillopacity{0.800000}%
\pgfsetlinewidth{0.000000pt}%
\definecolor{currentstroke}{rgb}{0.000000,0.000000,0.000000}%
\pgfsetstrokecolor{currentstroke}%
\pgfsetdash{}{0pt}%
\pgfpathmoveto{\pgfqpoint{3.570043in}{1.732612in}}%
\pgfpathlineto{\pgfqpoint{3.583753in}{1.731085in}}%
\pgfpathlineto{\pgfqpoint{3.597468in}{1.729757in}}%
\pgfpathlineto{\pgfqpoint{3.611189in}{1.728626in}}%
\pgfpathlineto{\pgfqpoint{3.624916in}{1.727692in}}%
\pgfpathlineto{\pgfqpoint{3.633065in}{1.738376in}}%
\pgfpathlineto{\pgfqpoint{3.641208in}{1.749095in}}%
\pgfpathlineto{\pgfqpoint{3.649345in}{1.759846in}}%
\pgfpathlineto{\pgfqpoint{3.657477in}{1.770626in}}%
\pgfpathlineto{\pgfqpoint{3.643761in}{1.771220in}}%
\pgfpathlineto{\pgfqpoint{3.630051in}{1.772011in}}%
\pgfpathlineto{\pgfqpoint{3.616348in}{1.772999in}}%
\pgfpathlineto{\pgfqpoint{3.602651in}{1.774186in}}%
\pgfpathlineto{\pgfqpoint{3.594508in}{1.763734in}}%
\pgfpathlineto{\pgfqpoint{3.586359in}{1.753319in}}%
\pgfpathlineto{\pgfqpoint{3.578204in}{1.742944in}}%
\pgfpathlineto{\pgfqpoint{3.570043in}{1.732612in}}%
\pgfpathclose%
\pgfusepath{fill}%
\end{pgfscope}%
\begin{pgfscope}%
\pgfpathrectangle{\pgfqpoint{1.150000in}{0.150000in}}{\pgfqpoint{5.700000in}{5.700000in}}%
\pgfusepath{clip}%
\pgfsetbuttcap%
\pgfsetroundjoin%
\definecolor{currentfill}{rgb}{0.221989,0.339161,0.548752}%
\pgfsetfillcolor{currentfill}%
\pgfsetfillopacity{0.800000}%
\pgfsetlinewidth{0.000000pt}%
\definecolor{currentstroke}{rgb}{0.000000,0.000000,0.000000}%
\pgfsetstrokecolor{currentstroke}%
\pgfsetdash{}{0pt}%
\pgfpathmoveto{\pgfqpoint{4.507507in}{2.407983in}}%
\pgfpathlineto{\pgfqpoint{4.521553in}{2.416050in}}%
\pgfpathlineto{\pgfqpoint{4.535613in}{2.424302in}}%
\pgfpathlineto{\pgfqpoint{4.549687in}{2.432739in}}%
\pgfpathlineto{\pgfqpoint{4.563776in}{2.441361in}}%
\pgfpathlineto{\pgfqpoint{4.571617in}{2.451348in}}%
\pgfpathlineto{\pgfqpoint{4.579451in}{2.461234in}}%
\pgfpathlineto{\pgfqpoint{4.587279in}{2.471019in}}%
\pgfpathlineto{\pgfqpoint{4.595101in}{2.480704in}}%
\pgfpathlineto{\pgfqpoint{4.581017in}{2.472127in}}%
\pgfpathlineto{\pgfqpoint{4.566947in}{2.463735in}}%
\pgfpathlineto{\pgfqpoint{4.552892in}{2.455528in}}%
\pgfpathlineto{\pgfqpoint{4.538852in}{2.447505in}}%
\pgfpathlineto{\pgfqpoint{4.531025in}{2.437763in}}%
\pgfpathlineto{\pgfqpoint{4.523192in}{2.427929in}}%
\pgfpathlineto{\pgfqpoint{4.515353in}{2.418002in}}%
\pgfpathlineto{\pgfqpoint{4.507507in}{2.407983in}}%
\pgfpathclose%
\pgfusepath{fill}%
\end{pgfscope}%
\begin{pgfscope}%
\pgfpathrectangle{\pgfqpoint{1.150000in}{0.150000in}}{\pgfqpoint{5.700000in}{5.700000in}}%
\pgfusepath{clip}%
\pgfsetbuttcap%
\pgfsetroundjoin%
\definecolor{currentfill}{rgb}{0.123444,0.636809,0.528763}%
\pgfsetfillcolor{currentfill}%
\pgfsetfillopacity{0.800000}%
\pgfsetlinewidth{0.000000pt}%
\definecolor{currentstroke}{rgb}{0.000000,0.000000,0.000000}%
\pgfsetstrokecolor{currentstroke}%
\pgfsetdash{}{0pt}%
\pgfpathmoveto{\pgfqpoint{5.651678in}{3.308621in}}%
\pgfpathlineto{\pgfqpoint{5.666369in}{3.321175in}}%
\pgfpathlineto{\pgfqpoint{5.681080in}{3.333909in}}%
\pgfpathlineto{\pgfqpoint{5.695812in}{3.346821in}}%
\pgfpathlineto{\pgfqpoint{5.710565in}{3.359913in}}%
\pgfpathlineto{\pgfqpoint{5.717788in}{3.361911in}}%
\pgfpathlineto{\pgfqpoint{5.725004in}{3.363917in}}%
\pgfpathlineto{\pgfqpoint{5.732213in}{3.365936in}}%
\pgfpathlineto{\pgfqpoint{5.739417in}{3.367975in}}%
\pgfpathlineto{\pgfqpoint{5.724693in}{3.355442in}}%
\pgfpathlineto{\pgfqpoint{5.709990in}{3.343088in}}%
\pgfpathlineto{\pgfqpoint{5.695308in}{3.330912in}}%
\pgfpathlineto{\pgfqpoint{5.680645in}{3.318913in}}%
\pgfpathlineto{\pgfqpoint{5.673413in}{3.316306in}}%
\pgfpathlineto{\pgfqpoint{5.666174in}{3.313725in}}%
\pgfpathlineto{\pgfqpoint{5.658929in}{3.311166in}}%
\pgfpathlineto{\pgfqpoint{5.651678in}{3.308621in}}%
\pgfpathclose%
\pgfusepath{fill}%
\end{pgfscope}%
\begin{pgfscope}%
\pgfpathrectangle{\pgfqpoint{1.150000in}{0.150000in}}{\pgfqpoint{5.700000in}{5.700000in}}%
\pgfusepath{clip}%
\pgfsetbuttcap%
\pgfsetroundjoin%
\definecolor{currentfill}{rgb}{0.237441,0.305202,0.541921}%
\pgfsetfillcolor{currentfill}%
\pgfsetfillopacity{0.800000}%
\pgfsetlinewidth{0.000000pt}%
\definecolor{currentstroke}{rgb}{0.000000,0.000000,0.000000}%
\pgfsetstrokecolor{currentstroke}%
\pgfsetdash{}{0pt}%
\pgfpathmoveto{\pgfqpoint{2.411155in}{2.395060in}}%
\pgfpathlineto{\pgfqpoint{2.425179in}{2.373955in}}%
\pgfpathlineto{\pgfqpoint{2.439190in}{2.353158in}}%
\pgfpathlineto{\pgfqpoint{2.453189in}{2.332668in}}%
\pgfpathlineto{\pgfqpoint{2.467177in}{2.312482in}}%
\pgfpathlineto{\pgfqpoint{2.476027in}{2.312313in}}%
\pgfpathlineto{\pgfqpoint{2.484860in}{2.312410in}}%
\pgfpathlineto{\pgfqpoint{2.493675in}{2.312767in}}%
\pgfpathlineto{\pgfqpoint{2.502474in}{2.313381in}}%
\pgfpathlineto{\pgfqpoint{2.488533in}{2.333037in}}%
\pgfpathlineto{\pgfqpoint{2.474581in}{2.352995in}}%
\pgfpathlineto{\pgfqpoint{2.460617in}{2.373259in}}%
\pgfpathlineto{\pgfqpoint{2.446642in}{2.393830in}}%
\pgfpathlineto{\pgfqpoint{2.437797in}{2.393735in}}%
\pgfpathlineto{\pgfqpoint{2.428934in}{2.393905in}}%
\pgfpathlineto{\pgfqpoint{2.420054in}{2.394345in}}%
\pgfpathlineto{\pgfqpoint{2.411155in}{2.395060in}}%
\pgfpathclose%
\pgfusepath{fill}%
\end{pgfscope}%
\begin{pgfscope}%
\pgfpathrectangle{\pgfqpoint{1.150000in}{0.150000in}}{\pgfqpoint{5.700000in}{5.700000in}}%
\pgfusepath{clip}%
\pgfsetbuttcap%
\pgfsetroundjoin%
\definecolor{currentfill}{rgb}{0.180629,0.429975,0.557282}%
\pgfsetfillcolor{currentfill}%
\pgfsetfillopacity{0.800000}%
\pgfsetlinewidth{0.000000pt}%
\definecolor{currentstroke}{rgb}{0.000000,0.000000,0.000000}%
\pgfsetstrokecolor{currentstroke}%
\pgfsetdash{}{0pt}%
\pgfpathmoveto{\pgfqpoint{4.801534in}{2.662190in}}%
\pgfpathlineto{\pgfqpoint{4.815739in}{2.672087in}}%
\pgfpathlineto{\pgfqpoint{4.829960in}{2.682167in}}%
\pgfpathlineto{\pgfqpoint{4.844197in}{2.692431in}}%
\pgfpathlineto{\pgfqpoint{4.858451in}{2.702878in}}%
\pgfpathlineto{\pgfqpoint{4.866166in}{2.710897in}}%
\pgfpathlineto{\pgfqpoint{4.873873in}{2.718813in}}%
\pgfpathlineto{\pgfqpoint{4.881574in}{2.726631in}}%
\pgfpathlineto{\pgfqpoint{4.889267in}{2.734351in}}%
\pgfpathlineto{\pgfqpoint{4.875021in}{2.724084in}}%
\pgfpathlineto{\pgfqpoint{4.860793in}{2.714000in}}%
\pgfpathlineto{\pgfqpoint{4.846580in}{2.704099in}}%
\pgfpathlineto{\pgfqpoint{4.832385in}{2.694382in}}%
\pgfpathlineto{\pgfqpoint{4.824682in}{2.686470in}}%
\pgfpathlineto{\pgfqpoint{4.816973in}{2.678469in}}%
\pgfpathlineto{\pgfqpoint{4.809257in}{2.670376in}}%
\pgfpathlineto{\pgfqpoint{4.801534in}{2.662190in}}%
\pgfpathclose%
\pgfusepath{fill}%
\end{pgfscope}%
\begin{pgfscope}%
\pgfpathrectangle{\pgfqpoint{1.150000in}{0.150000in}}{\pgfqpoint{5.700000in}{5.700000in}}%
\pgfusepath{clip}%
\pgfsetbuttcap%
\pgfsetroundjoin%
\definecolor{currentfill}{rgb}{0.281924,0.089666,0.412415}%
\pgfsetfillcolor{currentfill}%
\pgfsetfillopacity{0.800000}%
\pgfsetlinewidth{0.000000pt}%
\definecolor{currentstroke}{rgb}{0.000000,0.000000,0.000000}%
\pgfsetstrokecolor{currentstroke}%
\pgfsetdash{}{0pt}%
\pgfpathmoveto{\pgfqpoint{2.855306in}{1.860762in}}%
\pgfpathlineto{\pgfqpoint{2.869080in}{1.848382in}}%
\pgfpathlineto{\pgfqpoint{2.882850in}{1.836243in}}%
\pgfpathlineto{\pgfqpoint{2.896617in}{1.824344in}}%
\pgfpathlineto{\pgfqpoint{2.910380in}{1.812682in}}%
\pgfpathlineto{\pgfqpoint{2.918909in}{1.816749in}}%
\pgfpathlineto{\pgfqpoint{2.927426in}{1.821012in}}%
\pgfpathlineto{\pgfqpoint{2.935931in}{1.825468in}}%
\pgfpathlineto{\pgfqpoint{2.944424in}{1.830110in}}%
\pgfpathlineto{\pgfqpoint{2.930693in}{1.841265in}}%
\pgfpathlineto{\pgfqpoint{2.916959in}{1.852657in}}%
\pgfpathlineto{\pgfqpoint{2.903222in}{1.864289in}}%
\pgfpathlineto{\pgfqpoint{2.889482in}{1.876160in}}%
\pgfpathlineto{\pgfqpoint{2.880957in}{1.872013in}}%
\pgfpathlineto{\pgfqpoint{2.872419in}{1.868061in}}%
\pgfpathlineto{\pgfqpoint{2.863869in}{1.864309in}}%
\pgfpathlineto{\pgfqpoint{2.855306in}{1.860762in}}%
\pgfpathclose%
\pgfusepath{fill}%
\end{pgfscope}%
\begin{pgfscope}%
\pgfpathrectangle{\pgfqpoint{1.150000in}{0.150000in}}{\pgfqpoint{5.700000in}{5.700000in}}%
\pgfusepath{clip}%
\pgfsetbuttcap%
\pgfsetroundjoin%
\definecolor{currentfill}{rgb}{0.282884,0.135920,0.453427}%
\pgfsetfillcolor{currentfill}%
\pgfsetfillopacity{0.800000}%
\pgfsetlinewidth{0.000000pt}%
\definecolor{currentstroke}{rgb}{0.000000,0.000000,0.000000}%
\pgfsetstrokecolor{currentstroke}%
\pgfsetdash{}{0pt}%
\pgfpathmoveto{\pgfqpoint{3.919496in}{1.919276in}}%
\pgfpathlineto{\pgfqpoint{3.933291in}{1.922027in}}%
\pgfpathlineto{\pgfqpoint{3.947095in}{1.924969in}}%
\pgfpathlineto{\pgfqpoint{3.960909in}{1.928101in}}%
\pgfpathlineto{\pgfqpoint{3.974732in}{1.931423in}}%
\pgfpathlineto{\pgfqpoint{3.982767in}{1.943246in}}%
\pgfpathlineto{\pgfqpoint{3.990796in}{1.955033in}}%
\pgfpathlineto{\pgfqpoint{3.998821in}{1.966782in}}%
\pgfpathlineto{\pgfqpoint{4.006841in}{1.978491in}}%
\pgfpathlineto{\pgfqpoint{3.993023in}{1.974954in}}%
\pgfpathlineto{\pgfqpoint{3.979215in}{1.971608in}}%
\pgfpathlineto{\pgfqpoint{3.965416in}{1.968451in}}%
\pgfpathlineto{\pgfqpoint{3.951628in}{1.965486in}}%
\pgfpathlineto{\pgfqpoint{3.943602in}{1.953979in}}%
\pgfpathlineto{\pgfqpoint{3.935572in}{1.942441in}}%
\pgfpathlineto{\pgfqpoint{3.927536in}{1.930873in}}%
\pgfpathlineto{\pgfqpoint{3.919496in}{1.919276in}}%
\pgfpathclose%
\pgfusepath{fill}%
\end{pgfscope}%
\begin{pgfscope}%
\pgfpathrectangle{\pgfqpoint{1.150000in}{0.150000in}}{\pgfqpoint{5.700000in}{5.700000in}}%
\pgfusepath{clip}%
\pgfsetbuttcap%
\pgfsetroundjoin%
\definecolor{currentfill}{rgb}{0.272594,0.025563,0.353093}%
\pgfsetfillcolor{currentfill}%
\pgfsetfillopacity{0.800000}%
\pgfsetlinewidth{0.000000pt}%
\definecolor{currentstroke}{rgb}{0.000000,0.000000,0.000000}%
\pgfsetstrokecolor{currentstroke}%
\pgfsetdash{}{0pt}%
\pgfpathmoveto{\pgfqpoint{3.482503in}{1.701364in}}%
\pgfpathlineto{\pgfqpoint{3.496205in}{1.698668in}}%
\pgfpathlineto{\pgfqpoint{3.509911in}{1.696172in}}%
\pgfpathlineto{\pgfqpoint{3.523623in}{1.693877in}}%
\pgfpathlineto{\pgfqpoint{3.537340in}{1.691780in}}%
\pgfpathlineto{\pgfqpoint{3.545525in}{1.701907in}}%
\pgfpathlineto{\pgfqpoint{3.553704in}{1.712090in}}%
\pgfpathlineto{\pgfqpoint{3.561877in}{1.722326in}}%
\pgfpathlineto{\pgfqpoint{3.570043in}{1.732612in}}%
\pgfpathlineto{\pgfqpoint{3.556340in}{1.734337in}}%
\pgfpathlineto{\pgfqpoint{3.542641in}{1.736261in}}%
\pgfpathlineto{\pgfqpoint{3.528948in}{1.738385in}}%
\pgfpathlineto{\pgfqpoint{3.515261in}{1.740710in}}%
\pgfpathlineto{\pgfqpoint{3.507081in}{1.730784in}}%
\pgfpathlineto{\pgfqpoint{3.498894in}{1.720915in}}%
\pgfpathlineto{\pgfqpoint{3.490702in}{1.711107in}}%
\pgfpathlineto{\pgfqpoint{3.482503in}{1.701364in}}%
\pgfpathclose%
\pgfusepath{fill}%
\end{pgfscope}%
\begin{pgfscope}%
\pgfpathrectangle{\pgfqpoint{1.150000in}{0.150000in}}{\pgfqpoint{5.700000in}{5.700000in}}%
\pgfusepath{clip}%
\pgfsetbuttcap%
\pgfsetroundjoin%
\definecolor{currentfill}{rgb}{0.268510,0.009605,0.335427}%
\pgfsetfillcolor{currentfill}%
\pgfsetfillopacity{0.800000}%
\pgfsetlinewidth{0.000000pt}%
\definecolor{currentstroke}{rgb}{0.000000,0.000000,0.000000}%
\pgfsetstrokecolor{currentstroke}%
\pgfsetdash{}{0pt}%
\pgfpathmoveto{\pgfqpoint{3.252119in}{1.684513in}}%
\pgfpathlineto{\pgfqpoint{3.265817in}{1.678510in}}%
\pgfpathlineto{\pgfqpoint{3.279517in}{1.672717in}}%
\pgfpathlineto{\pgfqpoint{3.293219in}{1.667134in}}%
\pgfpathlineto{\pgfqpoint{3.306923in}{1.661759in}}%
\pgfpathlineto{\pgfqpoint{3.315215in}{1.669960in}}%
\pgfpathlineto{\pgfqpoint{3.323498in}{1.678271in}}%
\pgfpathlineto{\pgfqpoint{3.331774in}{1.686690in}}%
\pgfpathlineto{\pgfqpoint{3.340042in}{1.695211in}}%
\pgfpathlineto{\pgfqpoint{3.326357in}{1.700151in}}%
\pgfpathlineto{\pgfqpoint{3.312674in}{1.705299in}}%
\pgfpathlineto{\pgfqpoint{3.298995in}{1.710657in}}%
\pgfpathlineto{\pgfqpoint{3.285318in}{1.716225in}}%
\pgfpathlineto{\pgfqpoint{3.277030in}{1.708127in}}%
\pgfpathlineto{\pgfqpoint{3.268734in}{1.700139in}}%
\pgfpathlineto{\pgfqpoint{3.260431in}{1.692267in}}%
\pgfpathlineto{\pgfqpoint{3.252119in}{1.684513in}}%
\pgfpathclose%
\pgfusepath{fill}%
\end{pgfscope}%
\begin{pgfscope}%
\pgfpathrectangle{\pgfqpoint{1.150000in}{0.150000in}}{\pgfqpoint{5.700000in}{5.700000in}}%
\pgfusepath{clip}%
\pgfsetbuttcap%
\pgfsetroundjoin%
\definecolor{currentfill}{rgb}{0.134692,0.658636,0.517649}%
\pgfsetfillcolor{currentfill}%
\pgfsetfillopacity{0.800000}%
\pgfsetlinewidth{0.000000pt}%
\definecolor{currentstroke}{rgb}{0.000000,0.000000,0.000000}%
\pgfsetstrokecolor{currentstroke}%
\pgfsetdash{}{0pt}%
\pgfpathmoveto{\pgfqpoint{5.739417in}{3.367975in}}%
\pgfpathlineto{\pgfqpoint{5.754161in}{3.380685in}}%
\pgfpathlineto{\pgfqpoint{5.768927in}{3.393575in}}%
\pgfpathlineto{\pgfqpoint{5.783714in}{3.406643in}}%
\pgfpathlineto{\pgfqpoint{5.798522in}{3.419890in}}%
\pgfpathlineto{\pgfqpoint{5.805688in}{3.421373in}}%
\pgfpathlineto{\pgfqpoint{5.812849in}{3.422880in}}%
\pgfpathlineto{\pgfqpoint{5.820003in}{3.424419in}}%
\pgfpathlineto{\pgfqpoint{5.827152in}{3.425994in}}%
\pgfpathlineto{\pgfqpoint{5.812376in}{3.413341in}}%
\pgfpathlineto{\pgfqpoint{5.797621in}{3.400867in}}%
\pgfpathlineto{\pgfqpoint{5.782886in}{3.388569in}}%
\pgfpathlineto{\pgfqpoint{5.768173in}{3.376449in}}%
\pgfpathlineto{\pgfqpoint{5.760992in}{3.374270in}}%
\pgfpathlineto{\pgfqpoint{5.753806in}{3.372135in}}%
\pgfpathlineto{\pgfqpoint{5.746614in}{3.370039in}}%
\pgfpathlineto{\pgfqpoint{5.739417in}{3.367975in}}%
\pgfpathclose%
\pgfusepath{fill}%
\end{pgfscope}%
\begin{pgfscope}%
\pgfpathrectangle{\pgfqpoint{1.150000in}{0.150000in}}{\pgfqpoint{5.700000in}{5.700000in}}%
\pgfusepath{clip}%
\pgfsetbuttcap%
\pgfsetroundjoin%
\definecolor{currentfill}{rgb}{0.140536,0.530132,0.555659}%
\pgfsetfillcolor{currentfill}%
\pgfsetfillopacity{0.800000}%
\pgfsetlinewidth{0.000000pt}%
\definecolor{currentstroke}{rgb}{0.000000,0.000000,0.000000}%
\pgfsetstrokecolor{currentstroke}%
\pgfsetdash{}{0pt}%
\pgfpathmoveto{\pgfqpoint{5.183119in}{2.971670in}}%
\pgfpathlineto{\pgfqpoint{5.197546in}{2.983245in}}%
\pgfpathlineto{\pgfqpoint{5.211990in}{2.995002in}}%
\pgfpathlineto{\pgfqpoint{5.226454in}{3.006941in}}%
\pgfpathlineto{\pgfqpoint{5.240936in}{3.019061in}}%
\pgfpathlineto{\pgfqpoint{5.248451in}{3.024229in}}%
\pgfpathlineto{\pgfqpoint{5.255958in}{3.029324in}}%
\pgfpathlineto{\pgfqpoint{5.263457in}{3.034351in}}%
\pgfpathlineto{\pgfqpoint{5.270948in}{3.039313in}}%
\pgfpathlineto{\pgfqpoint{5.256482in}{3.027545in}}%
\pgfpathlineto{\pgfqpoint{5.242035in}{3.015958in}}%
\pgfpathlineto{\pgfqpoint{5.227606in}{3.004552in}}%
\pgfpathlineto{\pgfqpoint{5.213196in}{2.993326in}}%
\pgfpathlineto{\pgfqpoint{5.205688in}{2.988001in}}%
\pgfpathlineto{\pgfqpoint{5.198173in}{2.982619in}}%
\pgfpathlineto{\pgfqpoint{5.190650in}{2.977177in}}%
\pgfpathlineto{\pgfqpoint{5.183119in}{2.971670in}}%
\pgfpathclose%
\pgfusepath{fill}%
\end{pgfscope}%
\begin{pgfscope}%
\pgfpathrectangle{\pgfqpoint{1.150000in}{0.150000in}}{\pgfqpoint{5.700000in}{5.700000in}}%
\pgfusepath{clip}%
\pgfsetbuttcap%
\pgfsetroundjoin%
\definecolor{currentfill}{rgb}{0.272594,0.025563,0.353093}%
\pgfsetfillcolor{currentfill}%
\pgfsetfillopacity{0.800000}%
\pgfsetlinewidth{0.000000pt}%
\definecolor{currentstroke}{rgb}{0.000000,0.000000,0.000000}%
\pgfsetstrokecolor{currentstroke}%
\pgfsetdash{}{0pt}%
\pgfpathmoveto{\pgfqpoint{3.109058in}{1.714252in}}%
\pgfpathlineto{\pgfqpoint{3.122772in}{1.706058in}}%
\pgfpathlineto{\pgfqpoint{3.136486in}{1.698083in}}%
\pgfpathlineto{\pgfqpoint{3.150200in}{1.690327in}}%
\pgfpathlineto{\pgfqpoint{3.163915in}{1.682787in}}%
\pgfpathlineto{\pgfqpoint{3.172286in}{1.689541in}}%
\pgfpathlineto{\pgfqpoint{3.180647in}{1.696440in}}%
\pgfpathlineto{\pgfqpoint{3.189000in}{1.703479in}}%
\pgfpathlineto{\pgfqpoint{3.197343in}{1.710653in}}%
\pgfpathlineto{\pgfqpoint{3.183652in}{1.717725in}}%
\pgfpathlineto{\pgfqpoint{3.169962in}{1.725013in}}%
\pgfpathlineto{\pgfqpoint{3.156273in}{1.732519in}}%
\pgfpathlineto{\pgfqpoint{3.142584in}{1.740244in}}%
\pgfpathlineto{\pgfqpoint{3.134217in}{1.733526in}}%
\pgfpathlineto{\pgfqpoint{3.125840in}{1.726952in}}%
\pgfpathlineto{\pgfqpoint{3.117454in}{1.720525in}}%
\pgfpathlineto{\pgfqpoint{3.109058in}{1.714252in}}%
\pgfpathclose%
\pgfusepath{fill}%
\end{pgfscope}%
\begin{pgfscope}%
\pgfpathrectangle{\pgfqpoint{1.150000in}{0.150000in}}{\pgfqpoint{5.700000in}{5.700000in}}%
\pgfusepath{clip}%
\pgfsetbuttcap%
\pgfsetroundjoin%
\definecolor{currentfill}{rgb}{0.252194,0.269783,0.531579}%
\pgfsetfillcolor{currentfill}%
\pgfsetfillopacity{0.800000}%
\pgfsetlinewidth{0.000000pt}%
\definecolor{currentstroke}{rgb}{0.000000,0.000000,0.000000}%
\pgfsetstrokecolor{currentstroke}%
\pgfsetdash{}{0pt}%
\pgfpathmoveto{\pgfqpoint{4.300943in}{2.222910in}}%
\pgfpathlineto{\pgfqpoint{4.314893in}{2.229428in}}%
\pgfpathlineto{\pgfqpoint{4.328856in}{2.236132in}}%
\pgfpathlineto{\pgfqpoint{4.342831in}{2.243023in}}%
\pgfpathlineto{\pgfqpoint{4.356820in}{2.250099in}}%
\pgfpathlineto{\pgfqpoint{4.364738in}{2.261211in}}%
\pgfpathlineto{\pgfqpoint{4.372650in}{2.272233in}}%
\pgfpathlineto{\pgfqpoint{4.380557in}{2.283166in}}%
\pgfpathlineto{\pgfqpoint{4.388458in}{2.294009in}}%
\pgfpathlineto{\pgfqpoint{4.374473in}{2.286879in}}%
\pgfpathlineto{\pgfqpoint{4.360502in}{2.279934in}}%
\pgfpathlineto{\pgfqpoint{4.346543in}{2.273176in}}%
\pgfpathlineto{\pgfqpoint{4.332597in}{2.266604in}}%
\pgfpathlineto{\pgfqpoint{4.324692in}{2.255802in}}%
\pgfpathlineto{\pgfqpoint{4.316781in}{2.244919in}}%
\pgfpathlineto{\pgfqpoint{4.308865in}{2.233955in}}%
\pgfpathlineto{\pgfqpoint{4.300943in}{2.222910in}}%
\pgfpathclose%
\pgfusepath{fill}%
\end{pgfscope}%
\begin{pgfscope}%
\pgfpathrectangle{\pgfqpoint{1.150000in}{0.150000in}}{\pgfqpoint{5.700000in}{5.700000in}}%
\pgfusepath{clip}%
\pgfsetbuttcap%
\pgfsetroundjoin%
\definecolor{currentfill}{rgb}{0.280255,0.165693,0.476498}%
\pgfsetfillcolor{currentfill}%
\pgfsetfillopacity{0.800000}%
\pgfsetlinewidth{0.000000pt}%
\definecolor{currentstroke}{rgb}{0.000000,0.000000,0.000000}%
\pgfsetstrokecolor{currentstroke}%
\pgfsetdash{}{0pt}%
\pgfpathmoveto{\pgfqpoint{4.006841in}{1.978491in}}%
\pgfpathlineto{\pgfqpoint{4.020669in}{1.982216in}}%
\pgfpathlineto{\pgfqpoint{4.034508in}{1.986131in}}%
\pgfpathlineto{\pgfqpoint{4.048357in}{1.990235in}}%
\pgfpathlineto{\pgfqpoint{4.062217in}{1.994527in}}%
\pgfpathlineto{\pgfqpoint{4.070227in}{2.006389in}}%
\pgfpathlineto{\pgfqpoint{4.078233in}{2.018200in}}%
\pgfpathlineto{\pgfqpoint{4.086233in}{2.029958in}}%
\pgfpathlineto{\pgfqpoint{4.094229in}{2.041662in}}%
\pgfpathlineto{\pgfqpoint{4.080374in}{2.037186in}}%
\pgfpathlineto{\pgfqpoint{4.066529in}{2.032900in}}%
\pgfpathlineto{\pgfqpoint{4.052696in}{2.028802in}}%
\pgfpathlineto{\pgfqpoint{4.038872in}{2.024893in}}%
\pgfpathlineto{\pgfqpoint{4.030872in}{2.013360in}}%
\pgfpathlineto{\pgfqpoint{4.022867in}{2.001781in}}%
\pgfpathlineto{\pgfqpoint{4.014856in}{1.990158in}}%
\pgfpathlineto{\pgfqpoint{4.006841in}{1.978491in}}%
\pgfpathclose%
\pgfusepath{fill}%
\end{pgfscope}%
\begin{pgfscope}%
\pgfpathrectangle{\pgfqpoint{1.150000in}{0.150000in}}{\pgfqpoint{5.700000in}{5.700000in}}%
\pgfusepath{clip}%
\pgfsetbuttcap%
\pgfsetroundjoin%
\definecolor{currentfill}{rgb}{0.221989,0.339161,0.548752}%
\pgfsetfillcolor{currentfill}%
\pgfsetfillopacity{0.800000}%
\pgfsetlinewidth{0.000000pt}%
\definecolor{currentstroke}{rgb}{0.000000,0.000000,0.000000}%
\pgfsetstrokecolor{currentstroke}%
\pgfsetdash{}{0pt}%
\pgfpathmoveto{\pgfqpoint{2.354933in}{2.482627in}}%
\pgfpathlineto{\pgfqpoint{2.369008in}{2.460257in}}%
\pgfpathlineto{\pgfqpoint{2.383070in}{2.438208in}}%
\pgfpathlineto{\pgfqpoint{2.397119in}{2.416477in}}%
\pgfpathlineto{\pgfqpoint{2.411155in}{2.395060in}}%
\pgfpathlineto{\pgfqpoint{2.420054in}{2.394345in}}%
\pgfpathlineto{\pgfqpoint{2.428934in}{2.393905in}}%
\pgfpathlineto{\pgfqpoint{2.437797in}{2.393735in}}%
\pgfpathlineto{\pgfqpoint{2.446642in}{2.393830in}}%
\pgfpathlineto{\pgfqpoint{2.432655in}{2.414712in}}%
\pgfpathlineto{\pgfqpoint{2.418655in}{2.435907in}}%
\pgfpathlineto{\pgfqpoint{2.404643in}{2.457419in}}%
\pgfpathlineto{\pgfqpoint{2.390618in}{2.479250in}}%
\pgfpathlineto{\pgfqpoint{2.381724in}{2.479678in}}%
\pgfpathlineto{\pgfqpoint{2.372813in}{2.480381in}}%
\pgfpathlineto{\pgfqpoint{2.363882in}{2.481362in}}%
\pgfpathlineto{\pgfqpoint{2.354933in}{2.482627in}}%
\pgfpathclose%
\pgfusepath{fill}%
\end{pgfscope}%
\begin{pgfscope}%
\pgfpathrectangle{\pgfqpoint{1.150000in}{0.150000in}}{\pgfqpoint{5.700000in}{5.700000in}}%
\pgfusepath{clip}%
\pgfsetbuttcap%
\pgfsetroundjoin%
\definecolor{currentfill}{rgb}{0.150148,0.676631,0.506589}%
\pgfsetfillcolor{currentfill}%
\pgfsetfillopacity{0.800000}%
\pgfsetlinewidth{0.000000pt}%
\definecolor{currentstroke}{rgb}{0.000000,0.000000,0.000000}%
\pgfsetstrokecolor{currentstroke}%
\pgfsetdash{}{0pt}%
\pgfpathmoveto{\pgfqpoint{5.827152in}{3.425994in}}%
\pgfpathlineto{\pgfqpoint{5.841949in}{3.438825in}}%
\pgfpathlineto{\pgfqpoint{5.856768in}{3.451834in}}%
\pgfpathlineto{\pgfqpoint{5.871608in}{3.465021in}}%
\pgfpathlineto{\pgfqpoint{5.886470in}{3.478387in}}%
\pgfpathlineto{\pgfqpoint{5.893580in}{3.479390in}}%
\pgfpathlineto{\pgfqpoint{5.900685in}{3.480436in}}%
\pgfpathlineto{\pgfqpoint{5.907784in}{3.481532in}}%
\pgfpathlineto{\pgfqpoint{5.914878in}{3.482684in}}%
\pgfpathlineto{\pgfqpoint{5.900050in}{3.469947in}}%
\pgfpathlineto{\pgfqpoint{5.885244in}{3.457388in}}%
\pgfpathlineto{\pgfqpoint{5.870459in}{3.445006in}}%
\pgfpathlineto{\pgfqpoint{5.855695in}{3.432801in}}%
\pgfpathlineto{\pgfqpoint{5.848567in}{3.431010in}}%
\pgfpathlineto{\pgfqpoint{5.841434in}{3.429283in}}%
\pgfpathlineto{\pgfqpoint{5.834295in}{3.427613in}}%
\pgfpathlineto{\pgfqpoint{5.827152in}{3.425994in}}%
\pgfpathclose%
\pgfusepath{fill}%
\end{pgfscope}%
\begin{pgfscope}%
\pgfpathrectangle{\pgfqpoint{1.150000in}{0.150000in}}{\pgfqpoint{5.700000in}{5.700000in}}%
\pgfusepath{clip}%
\pgfsetbuttcap%
\pgfsetroundjoin%
\definecolor{currentfill}{rgb}{0.280267,0.073417,0.397163}%
\pgfsetfillcolor{currentfill}%
\pgfsetfillopacity{0.800000}%
\pgfsetlinewidth{0.000000pt}%
\definecolor{currentstroke}{rgb}{0.000000,0.000000,0.000000}%
\pgfsetstrokecolor{currentstroke}%
\pgfsetdash{}{0pt}%
\pgfpathmoveto{\pgfqpoint{2.910380in}{1.812682in}}%
\pgfpathlineto{\pgfqpoint{2.924141in}{1.801257in}}%
\pgfpathlineto{\pgfqpoint{2.937899in}{1.790066in}}%
\pgfpathlineto{\pgfqpoint{2.951654in}{1.779109in}}%
\pgfpathlineto{\pgfqpoint{2.965407in}{1.768384in}}%
\pgfpathlineto{\pgfqpoint{2.973904in}{1.772969in}}%
\pgfpathlineto{\pgfqpoint{2.982389in}{1.777742in}}%
\pgfpathlineto{\pgfqpoint{2.990862in}{1.782699in}}%
\pgfpathlineto{\pgfqpoint{2.999325in}{1.787833in}}%
\pgfpathlineto{\pgfqpoint{2.985603in}{1.798054in}}%
\pgfpathlineto{\pgfqpoint{2.971879in}{1.808505in}}%
\pgfpathlineto{\pgfqpoint{2.958153in}{1.819190in}}%
\pgfpathlineto{\pgfqpoint{2.944424in}{1.830110in}}%
\pgfpathlineto{\pgfqpoint{2.935931in}{1.825468in}}%
\pgfpathlineto{\pgfqpoint{2.927426in}{1.821012in}}%
\pgfpathlineto{\pgfqpoint{2.918909in}{1.816749in}}%
\pgfpathlineto{\pgfqpoint{2.910380in}{1.812682in}}%
\pgfpathclose%
\pgfusepath{fill}%
\end{pgfscope}%
\begin{pgfscope}%
\pgfpathrectangle{\pgfqpoint{1.150000in}{0.150000in}}{\pgfqpoint{5.700000in}{5.700000in}}%
\pgfusepath{clip}%
\pgfsetbuttcap%
\pgfsetroundjoin%
\definecolor{currentfill}{rgb}{0.208623,0.367752,0.552675}%
\pgfsetfillcolor{currentfill}%
\pgfsetfillopacity{0.800000}%
\pgfsetlinewidth{0.000000pt}%
\definecolor{currentstroke}{rgb}{0.000000,0.000000,0.000000}%
\pgfsetstrokecolor{currentstroke}%
\pgfsetdash{}{0pt}%
\pgfpathmoveto{\pgfqpoint{4.595101in}{2.480704in}}%
\pgfpathlineto{\pgfqpoint{4.609200in}{2.489465in}}%
\pgfpathlineto{\pgfqpoint{4.623314in}{2.498411in}}%
\pgfpathlineto{\pgfqpoint{4.637443in}{2.507541in}}%
\pgfpathlineto{\pgfqpoint{4.651588in}{2.516856in}}%
\pgfpathlineto{\pgfqpoint{4.659398in}{2.526377in}}%
\pgfpathlineto{\pgfqpoint{4.667202in}{2.535791in}}%
\pgfpathlineto{\pgfqpoint{4.674999in}{2.545102in}}%
\pgfpathlineto{\pgfqpoint{4.682790in}{2.554309in}}%
\pgfpathlineto{\pgfqpoint{4.668651in}{2.545073in}}%
\pgfpathlineto{\pgfqpoint{4.654527in}{2.536021in}}%
\pgfpathlineto{\pgfqpoint{4.640419in}{2.527154in}}%
\pgfpathlineto{\pgfqpoint{4.626325in}{2.518471in}}%
\pgfpathlineto{\pgfqpoint{4.618529in}{2.509173in}}%
\pgfpathlineto{\pgfqpoint{4.610726in}{2.499780in}}%
\pgfpathlineto{\pgfqpoint{4.602916in}{2.490291in}}%
\pgfpathlineto{\pgfqpoint{4.595101in}{2.480704in}}%
\pgfpathclose%
\pgfusepath{fill}%
\end{pgfscope}%
\begin{pgfscope}%
\pgfpathrectangle{\pgfqpoint{1.150000in}{0.150000in}}{\pgfqpoint{5.700000in}{5.700000in}}%
\pgfusepath{clip}%
\pgfsetbuttcap%
\pgfsetroundjoin%
\definecolor{currentfill}{rgb}{0.169646,0.456262,0.558030}%
\pgfsetfillcolor{currentfill}%
\pgfsetfillopacity{0.800000}%
\pgfsetlinewidth{0.000000pt}%
\definecolor{currentstroke}{rgb}{0.000000,0.000000,0.000000}%
\pgfsetstrokecolor{currentstroke}%
\pgfsetdash{}{0pt}%
\pgfpathmoveto{\pgfqpoint{4.889267in}{2.734351in}}%
\pgfpathlineto{\pgfqpoint{4.903529in}{2.744801in}}%
\pgfpathlineto{\pgfqpoint{4.917808in}{2.755434in}}%
\pgfpathlineto{\pgfqpoint{4.932104in}{2.766250in}}%
\pgfpathlineto{\pgfqpoint{4.946418in}{2.777249in}}%
\pgfpathlineto{\pgfqpoint{4.954094in}{2.784672in}}%
\pgfpathlineto{\pgfqpoint{4.961763in}{2.791996in}}%
\pgfpathlineto{\pgfqpoint{4.969425in}{2.799221in}}%
\pgfpathlineto{\pgfqpoint{4.977079in}{2.806352in}}%
\pgfpathlineto{\pgfqpoint{4.962776in}{2.795567in}}%
\pgfpathlineto{\pgfqpoint{4.948489in}{2.784966in}}%
\pgfpathlineto{\pgfqpoint{4.934220in}{2.774547in}}%
\pgfpathlineto{\pgfqpoint{4.919968in}{2.764310in}}%
\pgfpathlineto{\pgfqpoint{4.912303in}{2.756953in}}%
\pgfpathlineto{\pgfqpoint{4.904632in}{2.749510in}}%
\pgfpathlineto{\pgfqpoint{4.896953in}{2.741977in}}%
\pgfpathlineto{\pgfqpoint{4.889267in}{2.734351in}}%
\pgfpathclose%
\pgfusepath{fill}%
\end{pgfscope}%
\begin{pgfscope}%
\pgfpathrectangle{\pgfqpoint{1.150000in}{0.150000in}}{\pgfqpoint{5.700000in}{5.700000in}}%
\pgfusepath{clip}%
\pgfsetbuttcap%
\pgfsetroundjoin%
\definecolor{currentfill}{rgb}{0.269944,0.014625,0.341379}%
\pgfsetfillcolor{currentfill}%
\pgfsetfillopacity{0.800000}%
\pgfsetlinewidth{0.000000pt}%
\definecolor{currentstroke}{rgb}{0.000000,0.000000,0.000000}%
\pgfsetstrokecolor{currentstroke}%
\pgfsetdash{}{0pt}%
\pgfpathmoveto{\pgfqpoint{3.394812in}{1.677523in}}%
\pgfpathlineto{\pgfqpoint{3.408513in}{1.673614in}}%
\pgfpathlineto{\pgfqpoint{3.422218in}{1.669908in}}%
\pgfpathlineto{\pgfqpoint{3.435927in}{1.666406in}}%
\pgfpathlineto{\pgfqpoint{3.449640in}{1.663105in}}%
\pgfpathlineto{\pgfqpoint{3.457866in}{1.672555in}}%
\pgfpathlineto{\pgfqpoint{3.466085in}{1.682084in}}%
\pgfpathlineto{\pgfqpoint{3.474297in}{1.691688in}}%
\pgfpathlineto{\pgfqpoint{3.482503in}{1.701364in}}%
\pgfpathlineto{\pgfqpoint{3.468805in}{1.704261in}}%
\pgfpathlineto{\pgfqpoint{3.455112in}{1.707361in}}%
\pgfpathlineto{\pgfqpoint{3.441424in}{1.710663in}}%
\pgfpathlineto{\pgfqpoint{3.427739in}{1.714169in}}%
\pgfpathlineto{\pgfqpoint{3.419518in}{1.704885in}}%
\pgfpathlineto{\pgfqpoint{3.411289in}{1.695680in}}%
\pgfpathlineto{\pgfqpoint{3.403054in}{1.686558in}}%
\pgfpathlineto{\pgfqpoint{3.394812in}{1.677523in}}%
\pgfpathclose%
\pgfusepath{fill}%
\end{pgfscope}%
\begin{pgfscope}%
\pgfpathrectangle{\pgfqpoint{1.150000in}{0.150000in}}{\pgfqpoint{5.700000in}{5.700000in}}%
\pgfusepath{clip}%
\pgfsetbuttcap%
\pgfsetroundjoin%
\definecolor{currentfill}{rgb}{0.175707,0.697900,0.491033}%
\pgfsetfillcolor{currentfill}%
\pgfsetfillopacity{0.800000}%
\pgfsetlinewidth{0.000000pt}%
\definecolor{currentstroke}{rgb}{0.000000,0.000000,0.000000}%
\pgfsetstrokecolor{currentstroke}%
\pgfsetdash{}{0pt}%
\pgfpathmoveto{\pgfqpoint{5.914878in}{3.482684in}}%
\pgfpathlineto{\pgfqpoint{5.929727in}{3.495598in}}%
\pgfpathlineto{\pgfqpoint{5.944597in}{3.508690in}}%
\pgfpathlineto{\pgfqpoint{5.959490in}{3.521960in}}%
\pgfpathlineto{\pgfqpoint{5.974404in}{3.535408in}}%
\pgfpathlineto{\pgfqpoint{5.981457in}{3.535972in}}%
\pgfpathlineto{\pgfqpoint{5.988505in}{3.536600in}}%
\pgfpathlineto{\pgfqpoint{5.995549in}{3.537297in}}%
\pgfpathlineto{\pgfqpoint{6.002588in}{3.538072in}}%
\pgfpathlineto{\pgfqpoint{5.987711in}{3.525288in}}%
\pgfpathlineto{\pgfqpoint{5.972855in}{3.512681in}}%
\pgfpathlineto{\pgfqpoint{5.958021in}{3.500251in}}%
\pgfpathlineto{\pgfqpoint{5.943208in}{3.487996in}}%
\pgfpathlineto{\pgfqpoint{5.936132in}{3.486549in}}%
\pgfpathlineto{\pgfqpoint{5.929052in}{3.485186in}}%
\pgfpathlineto{\pgfqpoint{5.921967in}{3.483900in}}%
\pgfpathlineto{\pgfqpoint{5.914878in}{3.482684in}}%
\pgfpathclose%
\pgfusepath{fill}%
\end{pgfscope}%
\begin{pgfscope}%
\pgfpathrectangle{\pgfqpoint{1.150000in}{0.150000in}}{\pgfqpoint{5.700000in}{5.700000in}}%
\pgfusepath{clip}%
\pgfsetbuttcap%
\pgfsetroundjoin%
\definecolor{currentfill}{rgb}{0.226397,0.728888,0.462789}%
\pgfsetfillcolor{currentfill}%
\pgfsetfillopacity{0.800000}%
\pgfsetlinewidth{0.000000pt}%
\definecolor{currentstroke}{rgb}{0.000000,0.000000,0.000000}%
\pgfsetstrokecolor{currentstroke}%
\pgfsetdash{}{0pt}%
\pgfpathmoveto{\pgfqpoint{6.090279in}{3.592211in}}%
\pgfpathlineto{\pgfqpoint{6.105227in}{3.605183in}}%
\pgfpathlineto{\pgfqpoint{6.120196in}{3.618332in}}%
\pgfpathlineto{\pgfqpoint{6.135188in}{3.631657in}}%
\pgfpathlineto{\pgfqpoint{6.142139in}{3.631673in}}%
\pgfpathlineto{\pgfqpoint{6.149087in}{3.631798in}}%
\pgfpathlineto{\pgfqpoint{6.156033in}{3.632040in}}%
\pgfpathlineto{\pgfqpoint{6.162976in}{3.632406in}}%
\pgfpathlineto{\pgfqpoint{6.148026in}{3.619812in}}%
\pgfpathlineto{\pgfqpoint{6.133098in}{3.607393in}}%
\pgfpathlineto{\pgfqpoint{6.118193in}{3.595150in}}%
\pgfpathlineto{\pgfqpoint{6.111218in}{3.594228in}}%
\pgfpathlineto{\pgfqpoint{6.104241in}{3.593436in}}%
\pgfpathlineto{\pgfqpoint{6.097261in}{3.592767in}}%
\pgfpathlineto{\pgfqpoint{6.090279in}{3.592211in}}%
\pgfpathclose%
\pgfusepath{fill}%
\end{pgfscope}%
\begin{pgfscope}%
\pgfpathrectangle{\pgfqpoint{1.150000in}{0.150000in}}{\pgfqpoint{5.700000in}{5.700000in}}%
\pgfusepath{clip}%
\pgfsetbuttcap%
\pgfsetroundjoin%
\definecolor{currentfill}{rgb}{0.131172,0.555899,0.552459}%
\pgfsetfillcolor{currentfill}%
\pgfsetfillopacity{0.800000}%
\pgfsetlinewidth{0.000000pt}%
\definecolor{currentstroke}{rgb}{0.000000,0.000000,0.000000}%
\pgfsetstrokecolor{currentstroke}%
\pgfsetdash{}{0pt}%
\pgfpathmoveto{\pgfqpoint{5.270948in}{3.039313in}}%
\pgfpathlineto{\pgfqpoint{5.285433in}{3.051262in}}%
\pgfpathlineto{\pgfqpoint{5.299937in}{3.063393in}}%
\pgfpathlineto{\pgfqpoint{5.314461in}{3.075704in}}%
\pgfpathlineto{\pgfqpoint{5.329003in}{3.088198in}}%
\pgfpathlineto{\pgfqpoint{5.336469in}{3.092725in}}%
\pgfpathlineto{\pgfqpoint{5.343927in}{3.097188in}}%
\pgfpathlineto{\pgfqpoint{5.351378in}{3.101590in}}%
\pgfpathlineto{\pgfqpoint{5.358820in}{3.105938in}}%
\pgfpathlineto{\pgfqpoint{5.344296in}{3.093832in}}%
\pgfpathlineto{\pgfqpoint{5.329791in}{3.081907in}}%
\pgfpathlineto{\pgfqpoint{5.315305in}{3.070162in}}%
\pgfpathlineto{\pgfqpoint{5.300838in}{3.058598in}}%
\pgfpathlineto{\pgfqpoint{5.293377in}{3.053853in}}%
\pgfpathlineto{\pgfqpoint{5.285908in}{3.049059in}}%
\pgfpathlineto{\pgfqpoint{5.278432in}{3.044214in}}%
\pgfpathlineto{\pgfqpoint{5.270948in}{3.039313in}}%
\pgfpathclose%
\pgfusepath{fill}%
\end{pgfscope}%
\begin{pgfscope}%
\pgfpathrectangle{\pgfqpoint{1.150000in}{0.150000in}}{\pgfqpoint{5.700000in}{5.700000in}}%
\pgfusepath{clip}%
\pgfsetbuttcap%
\pgfsetroundjoin%
\definecolor{currentfill}{rgb}{0.202219,0.715272,0.476084}%
\pgfsetfillcolor{currentfill}%
\pgfsetfillopacity{0.800000}%
\pgfsetlinewidth{0.000000pt}%
\definecolor{currentstroke}{rgb}{0.000000,0.000000,0.000000}%
\pgfsetstrokecolor{currentstroke}%
\pgfsetdash{}{0pt}%
\pgfpathmoveto{\pgfqpoint{6.002588in}{3.538072in}}%
\pgfpathlineto{\pgfqpoint{6.017487in}{3.551033in}}%
\pgfpathlineto{\pgfqpoint{6.032408in}{3.564172in}}%
\pgfpathlineto{\pgfqpoint{6.047351in}{3.577488in}}%
\pgfpathlineto{\pgfqpoint{6.062317in}{3.590981in}}%
\pgfpathlineto{\pgfqpoint{6.069313in}{3.591155in}}%
\pgfpathlineto{\pgfqpoint{6.076305in}{3.591413in}}%
\pgfpathlineto{\pgfqpoint{6.083294in}{3.591762in}}%
\pgfpathlineto{\pgfqpoint{6.090279in}{3.592211in}}%
\pgfpathlineto{\pgfqpoint{6.075354in}{3.579416in}}%
\pgfpathlineto{\pgfqpoint{6.060450in}{3.566797in}}%
\pgfpathlineto{\pgfqpoint{6.045569in}{3.554354in}}%
\pgfpathlineto{\pgfqpoint{6.030708in}{3.542087in}}%
\pgfpathlineto{\pgfqpoint{6.023683in}{3.540931in}}%
\pgfpathlineto{\pgfqpoint{6.016655in}{3.539882in}}%
\pgfpathlineto{\pgfqpoint{6.009623in}{3.538931in}}%
\pgfpathlineto{\pgfqpoint{6.002588in}{3.538072in}}%
\pgfpathclose%
\pgfusepath{fill}%
\end{pgfscope}%
\begin{pgfscope}%
\pgfpathrectangle{\pgfqpoint{1.150000in}{0.150000in}}{\pgfqpoint{5.700000in}{5.700000in}}%
\pgfusepath{clip}%
\pgfsetbuttcap%
\pgfsetroundjoin%
\definecolor{currentfill}{rgb}{0.274128,0.199721,0.498911}%
\pgfsetfillcolor{currentfill}%
\pgfsetfillopacity{0.800000}%
\pgfsetlinewidth{0.000000pt}%
\definecolor{currentstroke}{rgb}{0.000000,0.000000,0.000000}%
\pgfsetstrokecolor{currentstroke}%
\pgfsetdash{}{0pt}%
\pgfpathmoveto{\pgfqpoint{4.094229in}{2.041662in}}%
\pgfpathlineto{\pgfqpoint{4.108095in}{2.046325in}}%
\pgfpathlineto{\pgfqpoint{4.121972in}{2.051177in}}%
\pgfpathlineto{\pgfqpoint{4.135861in}{2.056216in}}%
\pgfpathlineto{\pgfqpoint{4.149761in}{2.061443in}}%
\pgfpathlineto{\pgfqpoint{4.157748in}{2.073254in}}%
\pgfpathlineto{\pgfqpoint{4.165730in}{2.085002in}}%
\pgfpathlineto{\pgfqpoint{4.173706in}{2.096683in}}%
\pgfpathlineto{\pgfqpoint{4.181678in}{2.108297in}}%
\pgfpathlineto{\pgfqpoint{4.167782in}{2.102919in}}%
\pgfpathlineto{\pgfqpoint{4.153898in}{2.097729in}}%
\pgfpathlineto{\pgfqpoint{4.140024in}{2.092726in}}%
\pgfpathlineto{\pgfqpoint{4.126163in}{2.087912in}}%
\pgfpathlineto{\pgfqpoint{4.118187in}{2.076436in}}%
\pgfpathlineto{\pgfqpoint{4.110206in}{2.064902in}}%
\pgfpathlineto{\pgfqpoint{4.102220in}{2.053310in}}%
\pgfpathlineto{\pgfqpoint{4.094229in}{2.041662in}}%
\pgfpathclose%
\pgfusepath{fill}%
\end{pgfscope}%
\begin{pgfscope}%
\pgfpathrectangle{\pgfqpoint{1.150000in}{0.150000in}}{\pgfqpoint{5.700000in}{5.700000in}}%
\pgfusepath{clip}%
\pgfsetbuttcap%
\pgfsetroundjoin%
\definecolor{currentfill}{rgb}{0.239346,0.300855,0.540844}%
\pgfsetfillcolor{currentfill}%
\pgfsetfillopacity{0.800000}%
\pgfsetlinewidth{0.000000pt}%
\definecolor{currentstroke}{rgb}{0.000000,0.000000,0.000000}%
\pgfsetstrokecolor{currentstroke}%
\pgfsetdash{}{0pt}%
\pgfpathmoveto{\pgfqpoint{4.388458in}{2.294009in}}%
\pgfpathlineto{\pgfqpoint{4.402457in}{2.301326in}}%
\pgfpathlineto{\pgfqpoint{4.416469in}{2.308828in}}%
\pgfpathlineto{\pgfqpoint{4.430494in}{2.316515in}}%
\pgfpathlineto{\pgfqpoint{4.444534in}{2.324388in}}%
\pgfpathlineto{\pgfqpoint{4.452426in}{2.335176in}}%
\pgfpathlineto{\pgfqpoint{4.460312in}{2.345866in}}%
\pgfpathlineto{\pgfqpoint{4.468193in}{2.356459in}}%
\pgfpathlineto{\pgfqpoint{4.476067in}{2.366956in}}%
\pgfpathlineto{\pgfqpoint{4.462032in}{2.359062in}}%
\pgfpathlineto{\pgfqpoint{4.448010in}{2.351354in}}%
\pgfpathlineto{\pgfqpoint{4.434002in}{2.343830in}}%
\pgfpathlineto{\pgfqpoint{4.420008in}{2.336493in}}%
\pgfpathlineto{\pgfqpoint{4.412129in}{2.326005in}}%
\pgfpathlineto{\pgfqpoint{4.404244in}{2.315429in}}%
\pgfpathlineto{\pgfqpoint{4.396354in}{2.304764in}}%
\pgfpathlineto{\pgfqpoint{4.388458in}{2.294009in}}%
\pgfpathclose%
\pgfusepath{fill}%
\end{pgfscope}%
\begin{pgfscope}%
\pgfpathrectangle{\pgfqpoint{1.150000in}{0.150000in}}{\pgfqpoint{5.700000in}{5.700000in}}%
\pgfusepath{clip}%
\pgfsetbuttcap%
\pgfsetroundjoin%
\definecolor{currentfill}{rgb}{0.277941,0.056324,0.381191}%
\pgfsetfillcolor{currentfill}%
\pgfsetfillopacity{0.800000}%
\pgfsetlinewidth{0.000000pt}%
\definecolor{currentstroke}{rgb}{0.000000,0.000000,0.000000}%
\pgfsetstrokecolor{currentstroke}%
\pgfsetdash{}{0pt}%
\pgfpathmoveto{\pgfqpoint{2.965407in}{1.768384in}}%
\pgfpathlineto{\pgfqpoint{2.979158in}{1.757890in}}%
\pgfpathlineto{\pgfqpoint{2.992906in}{1.747625in}}%
\pgfpathlineto{\pgfqpoint{3.006654in}{1.737588in}}%
\pgfpathlineto{\pgfqpoint{3.020399in}{1.727778in}}%
\pgfpathlineto{\pgfqpoint{3.028866in}{1.732879in}}%
\pgfpathlineto{\pgfqpoint{3.037321in}{1.738160in}}%
\pgfpathlineto{\pgfqpoint{3.045765in}{1.743616in}}%
\pgfpathlineto{\pgfqpoint{3.054198in}{1.749242in}}%
\pgfpathlineto{\pgfqpoint{3.040482in}{1.758549in}}%
\pgfpathlineto{\pgfqpoint{3.026764in}{1.768082in}}%
\pgfpathlineto{\pgfqpoint{3.013045in}{1.777843in}}%
\pgfpathlineto{\pgfqpoint{2.999325in}{1.787833in}}%
\pgfpathlineto{\pgfqpoint{2.990862in}{1.782699in}}%
\pgfpathlineto{\pgfqpoint{2.982389in}{1.777742in}}%
\pgfpathlineto{\pgfqpoint{2.973904in}{1.772969in}}%
\pgfpathlineto{\pgfqpoint{2.965407in}{1.768384in}}%
\pgfpathclose%
\pgfusepath{fill}%
\end{pgfscope}%
\begin{pgfscope}%
\pgfpathrectangle{\pgfqpoint{1.150000in}{0.150000in}}{\pgfqpoint{5.700000in}{5.700000in}}%
\pgfusepath{clip}%
\pgfsetbuttcap%
\pgfsetroundjoin%
\definecolor{currentfill}{rgb}{0.204903,0.375746,0.553533}%
\pgfsetfillcolor{currentfill}%
\pgfsetfillopacity{0.800000}%
\pgfsetlinewidth{0.000000pt}%
\definecolor{currentstroke}{rgb}{0.000000,0.000000,0.000000}%
\pgfsetstrokecolor{currentstroke}%
\pgfsetdash{}{0pt}%
\pgfpathmoveto{\pgfqpoint{2.298489in}{2.575374in}}%
\pgfpathlineto{\pgfqpoint{2.312622in}{2.551691in}}%
\pgfpathlineto{\pgfqpoint{2.326740in}{2.528341in}}%
\pgfpathlineto{\pgfqpoint{2.340844in}{2.505320in}}%
\pgfpathlineto{\pgfqpoint{2.354933in}{2.482627in}}%
\pgfpathlineto{\pgfqpoint{2.363882in}{2.481362in}}%
\pgfpathlineto{\pgfqpoint{2.372813in}{2.480381in}}%
\pgfpathlineto{\pgfqpoint{2.381724in}{2.479678in}}%
\pgfpathlineto{\pgfqpoint{2.390618in}{2.479250in}}%
\pgfpathlineto{\pgfqpoint{2.376579in}{2.501403in}}%
\pgfpathlineto{\pgfqpoint{2.362527in}{2.523882in}}%
\pgfpathlineto{\pgfqpoint{2.348461in}{2.546690in}}%
\pgfpathlineto{\pgfqpoint{2.334380in}{2.569829in}}%
\pgfpathlineto{\pgfqpoint{2.325436in}{2.570786in}}%
\pgfpathlineto{\pgfqpoint{2.316473in}{2.572025in}}%
\pgfpathlineto{\pgfqpoint{2.307491in}{2.573553in}}%
\pgfpathlineto{\pgfqpoint{2.298489in}{2.575374in}}%
\pgfpathclose%
\pgfusepath{fill}%
\end{pgfscope}%
\begin{pgfscope}%
\pgfpathrectangle{\pgfqpoint{1.150000in}{0.150000in}}{\pgfqpoint{5.700000in}{5.700000in}}%
\pgfusepath{clip}%
\pgfsetbuttcap%
\pgfsetroundjoin%
\definecolor{currentfill}{rgb}{0.124395,0.578002,0.548287}%
\pgfsetfillcolor{currentfill}%
\pgfsetfillopacity{0.800000}%
\pgfsetlinewidth{0.000000pt}%
\definecolor{currentstroke}{rgb}{0.000000,0.000000,0.000000}%
\pgfsetstrokecolor{currentstroke}%
\pgfsetdash{}{0pt}%
\pgfpathmoveto{\pgfqpoint{5.358820in}{3.105938in}}%
\pgfpathlineto{\pgfqpoint{5.373364in}{3.118224in}}%
\pgfpathlineto{\pgfqpoint{5.387927in}{3.130692in}}%
\pgfpathlineto{\pgfqpoint{5.402510in}{3.143341in}}%
\pgfpathlineto{\pgfqpoint{5.417113in}{3.156171in}}%
\pgfpathlineto{\pgfqpoint{5.424528in}{3.160057in}}%
\pgfpathlineto{\pgfqpoint{5.431935in}{3.163890in}}%
\pgfpathlineto{\pgfqpoint{5.439335in}{3.167673in}}%
\pgfpathlineto{\pgfqpoint{5.446727in}{3.171411in}}%
\pgfpathlineto{\pgfqpoint{5.432145in}{3.159004in}}%
\pgfpathlineto{\pgfqpoint{5.417582in}{3.146777in}}%
\pgfpathlineto{\pgfqpoint{5.403039in}{3.134730in}}%
\pgfpathlineto{\pgfqpoint{5.388515in}{3.122864in}}%
\pgfpathlineto{\pgfqpoint{5.381103in}{3.118693in}}%
\pgfpathlineto{\pgfqpoint{5.373683in}{3.114484in}}%
\pgfpathlineto{\pgfqpoint{5.366255in}{3.110234in}}%
\pgfpathlineto{\pgfqpoint{5.358820in}{3.105938in}}%
\pgfpathclose%
\pgfusepath{fill}%
\end{pgfscope}%
\begin{pgfscope}%
\pgfpathrectangle{\pgfqpoint{1.150000in}{0.150000in}}{\pgfqpoint{5.700000in}{5.700000in}}%
\pgfusepath{clip}%
\pgfsetbuttcap%
\pgfsetroundjoin%
\definecolor{currentfill}{rgb}{0.269944,0.014625,0.341379}%
\pgfsetfillcolor{currentfill}%
\pgfsetfillopacity{0.800000}%
\pgfsetlinewidth{0.000000pt}%
\definecolor{currentstroke}{rgb}{0.000000,0.000000,0.000000}%
\pgfsetstrokecolor{currentstroke}%
\pgfsetdash{}{0pt}%
\pgfpathmoveto{\pgfqpoint{3.163915in}{1.682787in}}%
\pgfpathlineto{\pgfqpoint{3.177631in}{1.675463in}}%
\pgfpathlineto{\pgfqpoint{3.191348in}{1.668354in}}%
\pgfpathlineto{\pgfqpoint{3.205065in}{1.661458in}}%
\pgfpathlineto{\pgfqpoint{3.218784in}{1.654775in}}%
\pgfpathlineto{\pgfqpoint{3.227131in}{1.662010in}}%
\pgfpathlineto{\pgfqpoint{3.235469in}{1.669380in}}%
\pgfpathlineto{\pgfqpoint{3.243798in}{1.676883in}}%
\pgfpathlineto{\pgfqpoint{3.252119in}{1.684513in}}%
\pgfpathlineto{\pgfqpoint{3.238422in}{1.690728in}}%
\pgfpathlineto{\pgfqpoint{3.224728in}{1.697156in}}%
\pgfpathlineto{\pgfqpoint{3.211035in}{1.703798in}}%
\pgfpathlineto{\pgfqpoint{3.197343in}{1.710653in}}%
\pgfpathlineto{\pgfqpoint{3.189000in}{1.703479in}}%
\pgfpathlineto{\pgfqpoint{3.180647in}{1.696440in}}%
\pgfpathlineto{\pgfqpoint{3.172286in}{1.689541in}}%
\pgfpathlineto{\pgfqpoint{3.163915in}{1.682787in}}%
\pgfpathclose%
\pgfusepath{fill}%
\end{pgfscope}%
\begin{pgfscope}%
\pgfpathrectangle{\pgfqpoint{1.150000in}{0.150000in}}{\pgfqpoint{5.700000in}{5.700000in}}%
\pgfusepath{clip}%
\pgfsetbuttcap%
\pgfsetroundjoin%
\definecolor{currentfill}{rgb}{0.159194,0.482237,0.558073}%
\pgfsetfillcolor{currentfill}%
\pgfsetfillopacity{0.800000}%
\pgfsetlinewidth{0.000000pt}%
\definecolor{currentstroke}{rgb}{0.000000,0.000000,0.000000}%
\pgfsetstrokecolor{currentstroke}%
\pgfsetdash{}{0pt}%
\pgfpathmoveto{\pgfqpoint{4.977079in}{2.806352in}}%
\pgfpathlineto{\pgfqpoint{4.991400in}{2.817319in}}%
\pgfpathlineto{\pgfqpoint{5.005739in}{2.828468in}}%
\pgfpathlineto{\pgfqpoint{5.020095in}{2.839801in}}%
\pgfpathlineto{\pgfqpoint{5.034469in}{2.851316in}}%
\pgfpathlineto{\pgfqpoint{5.042105in}{2.858118in}}%
\pgfpathlineto{\pgfqpoint{5.049733in}{2.864822in}}%
\pgfpathlineto{\pgfqpoint{5.057353in}{2.871431in}}%
\pgfpathlineto{\pgfqpoint{5.064966in}{2.877948in}}%
\pgfpathlineto{\pgfqpoint{5.050604in}{2.866683in}}%
\pgfpathlineto{\pgfqpoint{5.036259in}{2.855599in}}%
\pgfpathlineto{\pgfqpoint{5.021932in}{2.844698in}}%
\pgfpathlineto{\pgfqpoint{5.007622in}{2.833979in}}%
\pgfpathlineto{\pgfqpoint{4.999997in}{2.827200in}}%
\pgfpathlineto{\pgfqpoint{4.992365in}{2.820338in}}%
\pgfpathlineto{\pgfqpoint{4.984726in}{2.813390in}}%
\pgfpathlineto{\pgfqpoint{4.977079in}{2.806352in}}%
\pgfpathclose%
\pgfusepath{fill}%
\end{pgfscope}%
\begin{pgfscope}%
\pgfpathrectangle{\pgfqpoint{1.150000in}{0.150000in}}{\pgfqpoint{5.700000in}{5.700000in}}%
\pgfusepath{clip}%
\pgfsetbuttcap%
\pgfsetroundjoin%
\definecolor{currentfill}{rgb}{0.194100,0.399323,0.555565}%
\pgfsetfillcolor{currentfill}%
\pgfsetfillopacity{0.800000}%
\pgfsetlinewidth{0.000000pt}%
\definecolor{currentstroke}{rgb}{0.000000,0.000000,0.000000}%
\pgfsetstrokecolor{currentstroke}%
\pgfsetdash{}{0pt}%
\pgfpathmoveto{\pgfqpoint{4.682790in}{2.554309in}}%
\pgfpathlineto{\pgfqpoint{4.696944in}{2.563729in}}%
\pgfpathlineto{\pgfqpoint{4.711115in}{2.573333in}}%
\pgfpathlineto{\pgfqpoint{4.725301in}{2.583121in}}%
\pgfpathlineto{\pgfqpoint{4.739503in}{2.593093in}}%
\pgfpathlineto{\pgfqpoint{4.747281in}{2.602098in}}%
\pgfpathlineto{\pgfqpoint{4.755052in}{2.610995in}}%
\pgfpathlineto{\pgfqpoint{4.762816in}{2.619785in}}%
\pgfpathlineto{\pgfqpoint{4.770574in}{2.628469in}}%
\pgfpathlineto{\pgfqpoint{4.756378in}{2.618610in}}%
\pgfpathlineto{\pgfqpoint{4.742198in}{2.608934in}}%
\pgfpathlineto{\pgfqpoint{4.728034in}{2.599443in}}%
\pgfpathlineto{\pgfqpoint{4.713886in}{2.590135in}}%
\pgfpathlineto{\pgfqpoint{4.706122in}{2.581326in}}%
\pgfpathlineto{\pgfqpoint{4.698351in}{2.572419in}}%
\pgfpathlineto{\pgfqpoint{4.690574in}{2.563414in}}%
\pgfpathlineto{\pgfqpoint{4.682790in}{2.554309in}}%
\pgfpathclose%
\pgfusepath{fill}%
\end{pgfscope}%
\begin{pgfscope}%
\pgfpathrectangle{\pgfqpoint{1.150000in}{0.150000in}}{\pgfqpoint{5.700000in}{5.700000in}}%
\pgfusepath{clip}%
\pgfsetbuttcap%
\pgfsetroundjoin%
\definecolor{currentfill}{rgb}{0.268510,0.009605,0.335427}%
\pgfsetfillcolor{currentfill}%
\pgfsetfillopacity{0.800000}%
\pgfsetlinewidth{0.000000pt}%
\definecolor{currentstroke}{rgb}{0.000000,0.000000,0.000000}%
\pgfsetstrokecolor{currentstroke}%
\pgfsetdash{}{0pt}%
\pgfpathmoveto{\pgfqpoint{3.306923in}{1.661759in}}%
\pgfpathlineto{\pgfqpoint{3.320630in}{1.656592in}}%
\pgfpathlineto{\pgfqpoint{3.334340in}{1.651633in}}%
\pgfpathlineto{\pgfqpoint{3.348053in}{1.646879in}}%
\pgfpathlineto{\pgfqpoint{3.361769in}{1.642331in}}%
\pgfpathlineto{\pgfqpoint{3.370041in}{1.650979in}}%
\pgfpathlineto{\pgfqpoint{3.378305in}{1.659729in}}%
\pgfpathlineto{\pgfqpoint{3.386562in}{1.668579in}}%
\pgfpathlineto{\pgfqpoint{3.394812in}{1.677523in}}%
\pgfpathlineto{\pgfqpoint{3.381114in}{1.681636in}}%
\pgfpathlineto{\pgfqpoint{3.367420in}{1.685955in}}%
\pgfpathlineto{\pgfqpoint{3.353729in}{1.690480in}}%
\pgfpathlineto{\pgfqpoint{3.340042in}{1.695211in}}%
\pgfpathlineto{\pgfqpoint{3.331774in}{1.686690in}}%
\pgfpathlineto{\pgfqpoint{3.323498in}{1.678271in}}%
\pgfpathlineto{\pgfqpoint{3.315215in}{1.669960in}}%
\pgfpathlineto{\pgfqpoint{3.306923in}{1.661759in}}%
\pgfpathclose%
\pgfusepath{fill}%
\end{pgfscope}%
\begin{pgfscope}%
\pgfpathrectangle{\pgfqpoint{1.150000in}{0.150000in}}{\pgfqpoint{5.700000in}{5.700000in}}%
\pgfusepath{clip}%
\pgfsetbuttcap%
\pgfsetroundjoin%
\definecolor{currentfill}{rgb}{0.280267,0.073417,0.397163}%
\pgfsetfillcolor{currentfill}%
\pgfsetfillopacity{0.800000}%
\pgfsetlinewidth{0.000000pt}%
\definecolor{currentstroke}{rgb}{0.000000,0.000000,0.000000}%
\pgfsetstrokecolor{currentstroke}%
\pgfsetdash{}{0pt}%
\pgfpathmoveto{\pgfqpoint{3.712408in}{1.770207in}}%
\pgfpathlineto{\pgfqpoint{3.726159in}{1.770589in}}%
\pgfpathlineto{\pgfqpoint{3.739917in}{1.771164in}}%
\pgfpathlineto{\pgfqpoint{3.753683in}{1.771933in}}%
\pgfpathlineto{\pgfqpoint{3.767456in}{1.772893in}}%
\pgfpathlineto{\pgfqpoint{3.775563in}{1.784341in}}%
\pgfpathlineto{\pgfqpoint{3.783665in}{1.795796in}}%
\pgfpathlineto{\pgfqpoint{3.791762in}{1.807256in}}%
\pgfpathlineto{\pgfqpoint{3.799854in}{1.818717in}}%
\pgfpathlineto{\pgfqpoint{3.786089in}{1.817447in}}%
\pgfpathlineto{\pgfqpoint{3.772332in}{1.816370in}}%
\pgfpathlineto{\pgfqpoint{3.758583in}{1.815485in}}%
\pgfpathlineto{\pgfqpoint{3.744842in}{1.814795in}}%
\pgfpathlineto{\pgfqpoint{3.736741in}{1.803630in}}%
\pgfpathlineto{\pgfqpoint{3.728636in}{1.792476in}}%
\pgfpathlineto{\pgfqpoint{3.720525in}{1.781334in}}%
\pgfpathlineto{\pgfqpoint{3.712408in}{1.770207in}}%
\pgfpathclose%
\pgfusepath{fill}%
\end{pgfscope}%
\begin{pgfscope}%
\pgfpathrectangle{\pgfqpoint{1.150000in}{0.150000in}}{\pgfqpoint{5.700000in}{5.700000in}}%
\pgfusepath{clip}%
\pgfsetbuttcap%
\pgfsetroundjoin%
\definecolor{currentfill}{rgb}{0.277018,0.050344,0.375715}%
\pgfsetfillcolor{currentfill}%
\pgfsetfillopacity{0.800000}%
\pgfsetlinewidth{0.000000pt}%
\definecolor{currentstroke}{rgb}{0.000000,0.000000,0.000000}%
\pgfsetstrokecolor{currentstroke}%
\pgfsetdash{}{0pt}%
\pgfpathmoveto{\pgfqpoint{3.624916in}{1.727692in}}%
\pgfpathlineto{\pgfqpoint{3.638649in}{1.726954in}}%
\pgfpathlineto{\pgfqpoint{3.652389in}{1.726412in}}%
\pgfpathlineto{\pgfqpoint{3.666136in}{1.726065in}}%
\pgfpathlineto{\pgfqpoint{3.679889in}{1.725912in}}%
\pgfpathlineto{\pgfqpoint{3.688027in}{1.736948in}}%
\pgfpathlineto{\pgfqpoint{3.696160in}{1.748011in}}%
\pgfpathlineto{\pgfqpoint{3.704287in}{1.759099in}}%
\pgfpathlineto{\pgfqpoint{3.712408in}{1.770207in}}%
\pgfpathlineto{\pgfqpoint{3.698665in}{1.770019in}}%
\pgfpathlineto{\pgfqpoint{3.684929in}{1.770026in}}%
\pgfpathlineto{\pgfqpoint{3.671200in}{1.770228in}}%
\pgfpathlineto{\pgfqpoint{3.657477in}{1.770626in}}%
\pgfpathlineto{\pgfqpoint{3.649345in}{1.759846in}}%
\pgfpathlineto{\pgfqpoint{3.641208in}{1.749095in}}%
\pgfpathlineto{\pgfqpoint{3.633065in}{1.738376in}}%
\pgfpathlineto{\pgfqpoint{3.624916in}{1.727692in}}%
\pgfpathclose%
\pgfusepath{fill}%
\end{pgfscope}%
\begin{pgfscope}%
\pgfpathrectangle{\pgfqpoint{1.150000in}{0.150000in}}{\pgfqpoint{5.700000in}{5.700000in}}%
\pgfusepath{clip}%
\pgfsetbuttcap%
\pgfsetroundjoin%
\definecolor{currentfill}{rgb}{0.266580,0.228262,0.514349}%
\pgfsetfillcolor{currentfill}%
\pgfsetfillopacity{0.800000}%
\pgfsetlinewidth{0.000000pt}%
\definecolor{currentstroke}{rgb}{0.000000,0.000000,0.000000}%
\pgfsetstrokecolor{currentstroke}%
\pgfsetdash{}{0pt}%
\pgfpathmoveto{\pgfqpoint{4.181678in}{2.108297in}}%
\pgfpathlineto{\pgfqpoint{4.195586in}{2.113862in}}%
\pgfpathlineto{\pgfqpoint{4.209506in}{2.119615in}}%
\pgfpathlineto{\pgfqpoint{4.223438in}{2.125554in}}%
\pgfpathlineto{\pgfqpoint{4.237382in}{2.131680in}}%
\pgfpathlineto{\pgfqpoint{4.245345in}{2.143357in}}%
\pgfpathlineto{\pgfqpoint{4.253304in}{2.154958in}}%
\pgfpathlineto{\pgfqpoint{4.261257in}{2.166481in}}%
\pgfpathlineto{\pgfqpoint{4.269204in}{2.177926in}}%
\pgfpathlineto{\pgfqpoint{4.255264in}{2.171681in}}%
\pgfpathlineto{\pgfqpoint{4.241335in}{2.165623in}}%
\pgfpathlineto{\pgfqpoint{4.227419in}{2.159752in}}%
\pgfpathlineto{\pgfqpoint{4.213515in}{2.154067in}}%
\pgfpathlineto{\pgfqpoint{4.205564in}{2.142729in}}%
\pgfpathlineto{\pgfqpoint{4.197607in}{2.131321in}}%
\pgfpathlineto{\pgfqpoint{4.189645in}{2.119844in}}%
\pgfpathlineto{\pgfqpoint{4.181678in}{2.108297in}}%
\pgfpathclose%
\pgfusepath{fill}%
\end{pgfscope}%
\begin{pgfscope}%
\pgfpathrectangle{\pgfqpoint{1.150000in}{0.150000in}}{\pgfqpoint{5.700000in}{5.700000in}}%
\pgfusepath{clip}%
\pgfsetbuttcap%
\pgfsetroundjoin%
\definecolor{currentfill}{rgb}{0.282656,0.100196,0.422160}%
\pgfsetfillcolor{currentfill}%
\pgfsetfillopacity{0.800000}%
\pgfsetlinewidth{0.000000pt}%
\definecolor{currentstroke}{rgb}{0.000000,0.000000,0.000000}%
\pgfsetstrokecolor{currentstroke}%
\pgfsetdash{}{0pt}%
\pgfpathmoveto{\pgfqpoint{3.799854in}{1.818717in}}%
\pgfpathlineto{\pgfqpoint{3.813628in}{1.820180in}}%
\pgfpathlineto{\pgfqpoint{3.827409in}{1.821834in}}%
\pgfpathlineto{\pgfqpoint{3.841199in}{1.823680in}}%
\pgfpathlineto{\pgfqpoint{3.854998in}{1.825716in}}%
\pgfpathlineto{\pgfqpoint{3.863077in}{1.837467in}}%
\pgfpathlineto{\pgfqpoint{3.871152in}{1.849206in}}%
\pgfpathlineto{\pgfqpoint{3.879221in}{1.860932in}}%
\pgfpathlineto{\pgfqpoint{3.887286in}{1.872642in}}%
\pgfpathlineto{\pgfqpoint{3.873494in}{1.870327in}}%
\pgfpathlineto{\pgfqpoint{3.859711in}{1.868204in}}%
\pgfpathlineto{\pgfqpoint{3.845937in}{1.866272in}}%
\pgfpathlineto{\pgfqpoint{3.832172in}{1.864532in}}%
\pgfpathlineto{\pgfqpoint{3.824100in}{1.853088in}}%
\pgfpathlineto{\pgfqpoint{3.816023in}{1.841636in}}%
\pgfpathlineto{\pgfqpoint{3.807941in}{1.830178in}}%
\pgfpathlineto{\pgfqpoint{3.799854in}{1.818717in}}%
\pgfpathclose%
\pgfusepath{fill}%
\end{pgfscope}%
\begin{pgfscope}%
\pgfpathrectangle{\pgfqpoint{1.150000in}{0.150000in}}{\pgfqpoint{5.700000in}{5.700000in}}%
\pgfusepath{clip}%
\pgfsetbuttcap%
\pgfsetroundjoin%
\definecolor{currentfill}{rgb}{0.273809,0.031497,0.358853}%
\pgfsetfillcolor{currentfill}%
\pgfsetfillopacity{0.800000}%
\pgfsetlinewidth{0.000000pt}%
\definecolor{currentstroke}{rgb}{0.000000,0.000000,0.000000}%
\pgfsetstrokecolor{currentstroke}%
\pgfsetdash{}{0pt}%
\pgfpathmoveto{\pgfqpoint{3.537340in}{1.691780in}}%
\pgfpathlineto{\pgfqpoint{3.551062in}{1.689883in}}%
\pgfpathlineto{\pgfqpoint{3.564790in}{1.688183in}}%
\pgfpathlineto{\pgfqpoint{3.578524in}{1.686681in}}%
\pgfpathlineto{\pgfqpoint{3.592263in}{1.685375in}}%
\pgfpathlineto{\pgfqpoint{3.600435in}{1.695885in}}%
\pgfpathlineto{\pgfqpoint{3.608601in}{1.706443in}}%
\pgfpathlineto{\pgfqpoint{3.616761in}{1.717047in}}%
\pgfpathlineto{\pgfqpoint{3.624916in}{1.727692in}}%
\pgfpathlineto{\pgfqpoint{3.611189in}{1.728626in}}%
\pgfpathlineto{\pgfqpoint{3.597468in}{1.729757in}}%
\pgfpathlineto{\pgfqpoint{3.583753in}{1.731085in}}%
\pgfpathlineto{\pgfqpoint{3.570043in}{1.732612in}}%
\pgfpathlineto{\pgfqpoint{3.561877in}{1.722326in}}%
\pgfpathlineto{\pgfqpoint{3.553704in}{1.712090in}}%
\pgfpathlineto{\pgfqpoint{3.545525in}{1.701907in}}%
\pgfpathlineto{\pgfqpoint{3.537340in}{1.691780in}}%
\pgfpathclose%
\pgfusepath{fill}%
\end{pgfscope}%
\begin{pgfscope}%
\pgfpathrectangle{\pgfqpoint{1.150000in}{0.150000in}}{\pgfqpoint{5.700000in}{5.700000in}}%
\pgfusepath{clip}%
\pgfsetbuttcap%
\pgfsetroundjoin%
\definecolor{currentfill}{rgb}{0.120092,0.600104,0.542530}%
\pgfsetfillcolor{currentfill}%
\pgfsetfillopacity{0.800000}%
\pgfsetlinewidth{0.000000pt}%
\definecolor{currentstroke}{rgb}{0.000000,0.000000,0.000000}%
\pgfsetstrokecolor{currentstroke}%
\pgfsetdash{}{0pt}%
\pgfpathmoveto{\pgfqpoint{5.446727in}{3.171411in}}%
\pgfpathlineto{\pgfqpoint{5.461329in}{3.183998in}}%
\pgfpathlineto{\pgfqpoint{5.475951in}{3.196766in}}%
\pgfpathlineto{\pgfqpoint{5.490593in}{3.209716in}}%
\pgfpathlineto{\pgfqpoint{5.505256in}{3.222846in}}%
\pgfpathlineto{\pgfqpoint{5.512618in}{3.226099in}}%
\pgfpathlineto{\pgfqpoint{5.519973in}{3.229308in}}%
\pgfpathlineto{\pgfqpoint{5.527320in}{3.232480in}}%
\pgfpathlineto{\pgfqpoint{5.534659in}{3.235619in}}%
\pgfpathlineto{\pgfqpoint{5.520019in}{3.222948in}}%
\pgfpathlineto{\pgfqpoint{5.505399in}{3.210456in}}%
\pgfpathlineto{\pgfqpoint{5.490800in}{3.198144in}}%
\pgfpathlineto{\pgfqpoint{5.476220in}{3.186011in}}%
\pgfpathlineto{\pgfqpoint{5.468858in}{3.182404in}}%
\pgfpathlineto{\pgfqpoint{5.461488in}{3.178771in}}%
\pgfpathlineto{\pgfqpoint{5.454111in}{3.175108in}}%
\pgfpathlineto{\pgfqpoint{5.446727in}{3.171411in}}%
\pgfpathclose%
\pgfusepath{fill}%
\end{pgfscope}%
\begin{pgfscope}%
\pgfpathrectangle{\pgfqpoint{1.150000in}{0.150000in}}{\pgfqpoint{5.700000in}{5.700000in}}%
\pgfusepath{clip}%
\pgfsetbuttcap%
\pgfsetroundjoin%
\definecolor{currentfill}{rgb}{0.223925,0.334994,0.548053}%
\pgfsetfillcolor{currentfill}%
\pgfsetfillopacity{0.800000}%
\pgfsetlinewidth{0.000000pt}%
\definecolor{currentstroke}{rgb}{0.000000,0.000000,0.000000}%
\pgfsetstrokecolor{currentstroke}%
\pgfsetdash{}{0pt}%
\pgfpathmoveto{\pgfqpoint{4.476067in}{2.366956in}}%
\pgfpathlineto{\pgfqpoint{4.490117in}{2.375035in}}%
\pgfpathlineto{\pgfqpoint{4.504182in}{2.383299in}}%
\pgfpathlineto{\pgfqpoint{4.518260in}{2.391749in}}%
\pgfpathlineto{\pgfqpoint{4.532354in}{2.400383in}}%
\pgfpathlineto{\pgfqpoint{4.540218in}{2.410783in}}%
\pgfpathlineto{\pgfqpoint{4.548077in}{2.421079in}}%
\pgfpathlineto{\pgfqpoint{4.555930in}{2.431272in}}%
\pgfpathlineto{\pgfqpoint{4.563776in}{2.441361in}}%
\pgfpathlineto{\pgfqpoint{4.549687in}{2.432739in}}%
\pgfpathlineto{\pgfqpoint{4.535613in}{2.424302in}}%
\pgfpathlineto{\pgfqpoint{4.521553in}{2.416050in}}%
\pgfpathlineto{\pgfqpoint{4.507507in}{2.407983in}}%
\pgfpathlineto{\pgfqpoint{4.499656in}{2.397869in}}%
\pgfpathlineto{\pgfqpoint{4.491799in}{2.387660in}}%
\pgfpathlineto{\pgfqpoint{4.483936in}{2.377356in}}%
\pgfpathlineto{\pgfqpoint{4.476067in}{2.366956in}}%
\pgfpathclose%
\pgfusepath{fill}%
\end{pgfscope}%
\begin{pgfscope}%
\pgfpathrectangle{\pgfqpoint{1.150000in}{0.150000in}}{\pgfqpoint{5.700000in}{5.700000in}}%
\pgfusepath{clip}%
\pgfsetbuttcap%
\pgfsetroundjoin%
\definecolor{currentfill}{rgb}{0.283187,0.125848,0.444960}%
\pgfsetfillcolor{currentfill}%
\pgfsetfillopacity{0.800000}%
\pgfsetlinewidth{0.000000pt}%
\definecolor{currentstroke}{rgb}{0.000000,0.000000,0.000000}%
\pgfsetstrokecolor{currentstroke}%
\pgfsetdash{}{0pt}%
\pgfpathmoveto{\pgfqpoint{3.887286in}{1.872642in}}%
\pgfpathlineto{\pgfqpoint{3.901087in}{1.875147in}}%
\pgfpathlineto{\pgfqpoint{3.914897in}{1.877842in}}%
\pgfpathlineto{\pgfqpoint{3.928717in}{1.880728in}}%
\pgfpathlineto{\pgfqpoint{3.942546in}{1.883803in}}%
\pgfpathlineto{\pgfqpoint{3.950600in}{1.895753in}}%
\pgfpathlineto{\pgfqpoint{3.958649in}{1.907674in}}%
\pgfpathlineto{\pgfqpoint{3.966693in}{1.919565in}}%
\pgfpathlineto{\pgfqpoint{3.974732in}{1.931423in}}%
\pgfpathlineto{\pgfqpoint{3.960909in}{1.928101in}}%
\pgfpathlineto{\pgfqpoint{3.947095in}{1.924969in}}%
\pgfpathlineto{\pgfqpoint{3.933291in}{1.922027in}}%
\pgfpathlineto{\pgfqpoint{3.919496in}{1.919276in}}%
\pgfpathlineto{\pgfqpoint{3.911451in}{1.907652in}}%
\pgfpathlineto{\pgfqpoint{3.903401in}{1.896004in}}%
\pgfpathlineto{\pgfqpoint{3.895346in}{1.884333in}}%
\pgfpathlineto{\pgfqpoint{3.887286in}{1.872642in}}%
\pgfpathclose%
\pgfusepath{fill}%
\end{pgfscope}%
\begin{pgfscope}%
\pgfpathrectangle{\pgfqpoint{1.150000in}{0.150000in}}{\pgfqpoint{5.700000in}{5.700000in}}%
\pgfusepath{clip}%
\pgfsetbuttcap%
\pgfsetroundjoin%
\definecolor{currentfill}{rgb}{0.274952,0.037752,0.364543}%
\pgfsetfillcolor{currentfill}%
\pgfsetfillopacity{0.800000}%
\pgfsetlinewidth{0.000000pt}%
\definecolor{currentstroke}{rgb}{0.000000,0.000000,0.000000}%
\pgfsetstrokecolor{currentstroke}%
\pgfsetdash{}{0pt}%
\pgfpathmoveto{\pgfqpoint{3.020399in}{1.727778in}}%
\pgfpathlineto{\pgfqpoint{3.034144in}{1.718193in}}%
\pgfpathlineto{\pgfqpoint{3.047887in}{1.708833in}}%
\pgfpathlineto{\pgfqpoint{3.061630in}{1.699695in}}%
\pgfpathlineto{\pgfqpoint{3.075371in}{1.690779in}}%
\pgfpathlineto{\pgfqpoint{3.083809in}{1.696394in}}%
\pgfpathlineto{\pgfqpoint{3.092235in}{1.702182in}}%
\pgfpathlineto{\pgfqpoint{3.100652in}{1.708136in}}%
\pgfpathlineto{\pgfqpoint{3.109058in}{1.714252in}}%
\pgfpathlineto{\pgfqpoint{3.095343in}{1.722666in}}%
\pgfpathlineto{\pgfqpoint{3.081629in}{1.731302in}}%
\pgfpathlineto{\pgfqpoint{3.067914in}{1.740160in}}%
\pgfpathlineto{\pgfqpoint{3.054198in}{1.749242in}}%
\pgfpathlineto{\pgfqpoint{3.045765in}{1.743616in}}%
\pgfpathlineto{\pgfqpoint{3.037321in}{1.738160in}}%
\pgfpathlineto{\pgfqpoint{3.028866in}{1.732879in}}%
\pgfpathlineto{\pgfqpoint{3.020399in}{1.727778in}}%
\pgfpathclose%
\pgfusepath{fill}%
\end{pgfscope}%
\begin{pgfscope}%
\pgfpathrectangle{\pgfqpoint{1.150000in}{0.150000in}}{\pgfqpoint{5.700000in}{5.700000in}}%
\pgfusepath{clip}%
\pgfsetbuttcap%
\pgfsetroundjoin%
\definecolor{currentfill}{rgb}{0.275191,0.194905,0.496005}%
\pgfsetfillcolor{currentfill}%
\pgfsetfillopacity{0.800000}%
\pgfsetlinewidth{0.000000pt}%
\definecolor{currentstroke}{rgb}{0.000000,0.000000,0.000000}%
\pgfsetstrokecolor{currentstroke}%
\pgfsetdash{}{0pt}%
\pgfpathmoveto{\pgfqpoint{2.599216in}{2.089703in}}%
\pgfpathlineto{\pgfqpoint{2.613123in}{2.072680in}}%
\pgfpathlineto{\pgfqpoint{2.627023in}{2.055928in}}%
\pgfpathlineto{\pgfqpoint{2.640915in}{2.039444in}}%
\pgfpathlineto{\pgfqpoint{2.654800in}{2.023227in}}%
\pgfpathlineto{\pgfqpoint{2.663536in}{2.024226in}}%
\pgfpathlineto{\pgfqpoint{2.672256in}{2.025477in}}%
\pgfpathlineto{\pgfqpoint{2.680962in}{2.026974in}}%
\pgfpathlineto{\pgfqpoint{2.689651in}{2.028712in}}%
\pgfpathlineto{\pgfqpoint{2.675809in}{2.044378in}}%
\pgfpathlineto{\pgfqpoint{2.661959in}{2.060309in}}%
\pgfpathlineto{\pgfqpoint{2.648102in}{2.076509in}}%
\pgfpathlineto{\pgfqpoint{2.634238in}{2.092979in}}%
\pgfpathlineto{\pgfqpoint{2.625506in}{2.091780in}}%
\pgfpathlineto{\pgfqpoint{2.616759in}{2.090830in}}%
\pgfpathlineto{\pgfqpoint{2.607996in}{2.090136in}}%
\pgfpathlineto{\pgfqpoint{2.599216in}{2.089703in}}%
\pgfpathclose%
\pgfusepath{fill}%
\end{pgfscope}%
\begin{pgfscope}%
\pgfpathrectangle{\pgfqpoint{1.150000in}{0.150000in}}{\pgfqpoint{5.700000in}{5.700000in}}%
\pgfusepath{clip}%
\pgfsetbuttcap%
\pgfsetroundjoin%
\definecolor{currentfill}{rgb}{0.279574,0.170599,0.479997}%
\pgfsetfillcolor{currentfill}%
\pgfsetfillopacity{0.800000}%
\pgfsetlinewidth{0.000000pt}%
\definecolor{currentstroke}{rgb}{0.000000,0.000000,0.000000}%
\pgfsetstrokecolor{currentstroke}%
\pgfsetdash{}{0pt}%
\pgfpathmoveto{\pgfqpoint{2.654800in}{2.023227in}}%
\pgfpathlineto{\pgfqpoint{2.668677in}{2.007275in}}%
\pgfpathlineto{\pgfqpoint{2.682547in}{1.991586in}}%
\pgfpathlineto{\pgfqpoint{2.696411in}{1.976157in}}%
\pgfpathlineto{\pgfqpoint{2.710268in}{1.960987in}}%
\pgfpathlineto{\pgfqpoint{2.718962in}{1.962548in}}%
\pgfpathlineto{\pgfqpoint{2.727642in}{1.964352in}}%
\pgfpathlineto{\pgfqpoint{2.736306in}{1.966393in}}%
\pgfpathlineto{\pgfqpoint{2.744957in}{1.968667in}}%
\pgfpathlineto{\pgfqpoint{2.731140in}{1.983289in}}%
\pgfpathlineto{\pgfqpoint{2.717317in}{1.998169in}}%
\pgfpathlineto{\pgfqpoint{2.703487in}{2.013310in}}%
\pgfpathlineto{\pgfqpoint{2.689651in}{2.028712in}}%
\pgfpathlineto{\pgfqpoint{2.680962in}{2.026974in}}%
\pgfpathlineto{\pgfqpoint{2.672256in}{2.025477in}}%
\pgfpathlineto{\pgfqpoint{2.663536in}{2.024226in}}%
\pgfpathlineto{\pgfqpoint{2.654800in}{2.023227in}}%
\pgfpathclose%
\pgfusepath{fill}%
\end{pgfscope}%
\begin{pgfscope}%
\pgfpathrectangle{\pgfqpoint{1.150000in}{0.150000in}}{\pgfqpoint{5.700000in}{5.700000in}}%
\pgfusepath{clip}%
\pgfsetbuttcap%
\pgfsetroundjoin%
\definecolor{currentfill}{rgb}{0.188923,0.410910,0.556326}%
\pgfsetfillcolor{currentfill}%
\pgfsetfillopacity{0.800000}%
\pgfsetlinewidth{0.000000pt}%
\definecolor{currentstroke}{rgb}{0.000000,0.000000,0.000000}%
\pgfsetstrokecolor{currentstroke}%
\pgfsetdash{}{0pt}%
\pgfpathmoveto{\pgfqpoint{2.241800in}{2.673504in}}%
\pgfpathlineto{\pgfqpoint{2.255997in}{2.648455in}}%
\pgfpathlineto{\pgfqpoint{2.270177in}{2.623752in}}%
\pgfpathlineto{\pgfqpoint{2.284340in}{2.599393in}}%
\pgfpathlineto{\pgfqpoint{2.298489in}{2.575374in}}%
\pgfpathlineto{\pgfqpoint{2.307491in}{2.573553in}}%
\pgfpathlineto{\pgfqpoint{2.316473in}{2.572025in}}%
\pgfpathlineto{\pgfqpoint{2.325436in}{2.570786in}}%
\pgfpathlineto{\pgfqpoint{2.334380in}{2.569829in}}%
\pgfpathlineto{\pgfqpoint{2.320284in}{2.593303in}}%
\pgfpathlineto{\pgfqpoint{2.306174in}{2.617116in}}%
\pgfpathlineto{\pgfqpoint{2.292048in}{2.641270in}}%
\pgfpathlineto{\pgfqpoint{2.277906in}{2.665769in}}%
\pgfpathlineto{\pgfqpoint{2.268910in}{2.667260in}}%
\pgfpathlineto{\pgfqpoint{2.259894in}{2.669042in}}%
\pgfpathlineto{\pgfqpoint{2.250857in}{2.671121in}}%
\pgfpathlineto{\pgfqpoint{2.241800in}{2.673504in}}%
\pgfpathclose%
\pgfusepath{fill}%
\end{pgfscope}%
\begin{pgfscope}%
\pgfpathrectangle{\pgfqpoint{1.150000in}{0.150000in}}{\pgfqpoint{5.700000in}{5.700000in}}%
\pgfusepath{clip}%
\pgfsetbuttcap%
\pgfsetroundjoin%
\definecolor{currentfill}{rgb}{0.149039,0.508051,0.557250}%
\pgfsetfillcolor{currentfill}%
\pgfsetfillopacity{0.800000}%
\pgfsetlinewidth{0.000000pt}%
\definecolor{currentstroke}{rgb}{0.000000,0.000000,0.000000}%
\pgfsetstrokecolor{currentstroke}%
\pgfsetdash{}{0pt}%
\pgfpathmoveto{\pgfqpoint{5.064966in}{2.877948in}}%
\pgfpathlineto{\pgfqpoint{5.079347in}{2.889397in}}%
\pgfpathlineto{\pgfqpoint{5.093746in}{2.901027in}}%
\pgfpathlineto{\pgfqpoint{5.108163in}{2.912840in}}%
\pgfpathlineto{\pgfqpoint{5.122598in}{2.924836in}}%
\pgfpathlineto{\pgfqpoint{5.130191in}{2.930994in}}%
\pgfpathlineto{\pgfqpoint{5.137775in}{2.937058in}}%
\pgfpathlineto{\pgfqpoint{5.145352in}{2.943031in}}%
\pgfpathlineto{\pgfqpoint{5.152921in}{2.948919in}}%
\pgfpathlineto{\pgfqpoint{5.138499in}{2.937207in}}%
\pgfpathlineto{\pgfqpoint{5.124095in}{2.925678in}}%
\pgfpathlineto{\pgfqpoint{5.109709in}{2.914331in}}%
\pgfpathlineto{\pgfqpoint{5.095341in}{2.903166in}}%
\pgfpathlineto{\pgfqpoint{5.087759in}{2.896983in}}%
\pgfpathlineto{\pgfqpoint{5.080169in}{2.890721in}}%
\pgfpathlineto{\pgfqpoint{5.072571in}{2.884377in}}%
\pgfpathlineto{\pgfqpoint{5.064966in}{2.877948in}}%
\pgfpathclose%
\pgfusepath{fill}%
\end{pgfscope}%
\begin{pgfscope}%
\pgfpathrectangle{\pgfqpoint{1.150000in}{0.150000in}}{\pgfqpoint{5.700000in}{5.700000in}}%
\pgfusepath{clip}%
\pgfsetbuttcap%
\pgfsetroundjoin%
\definecolor{currentfill}{rgb}{0.267968,0.223549,0.512008}%
\pgfsetfillcolor{currentfill}%
\pgfsetfillopacity{0.800000}%
\pgfsetlinewidth{0.000000pt}%
\definecolor{currentstroke}{rgb}{0.000000,0.000000,0.000000}%
\pgfsetstrokecolor{currentstroke}%
\pgfsetdash{}{0pt}%
\pgfpathmoveto{\pgfqpoint{2.543499in}{2.160546in}}%
\pgfpathlineto{\pgfqpoint{2.557441in}{2.142418in}}%
\pgfpathlineto{\pgfqpoint{2.571375in}{2.124570in}}%
\pgfpathlineto{\pgfqpoint{2.585300in}{2.106999in}}%
\pgfpathlineto{\pgfqpoint{2.599216in}{2.089703in}}%
\pgfpathlineto{\pgfqpoint{2.607996in}{2.090136in}}%
\pgfpathlineto{\pgfqpoint{2.616759in}{2.090830in}}%
\pgfpathlineto{\pgfqpoint{2.625506in}{2.091780in}}%
\pgfpathlineto{\pgfqpoint{2.634238in}{2.092979in}}%
\pgfpathlineto{\pgfqpoint{2.620366in}{2.109720in}}%
\pgfpathlineto{\pgfqpoint{2.606486in}{2.126736in}}%
\pgfpathlineto{\pgfqpoint{2.592597in}{2.144029in}}%
\pgfpathlineto{\pgfqpoint{2.578700in}{2.161600in}}%
\pgfpathlineto{\pgfqpoint{2.569925in}{2.160943in}}%
\pgfpathlineto{\pgfqpoint{2.561133in}{2.160545in}}%
\pgfpathlineto{\pgfqpoint{2.552325in}{2.160411in}}%
\pgfpathlineto{\pgfqpoint{2.543499in}{2.160546in}}%
\pgfpathclose%
\pgfusepath{fill}%
\end{pgfscope}%
\begin{pgfscope}%
\pgfpathrectangle{\pgfqpoint{1.150000in}{0.150000in}}{\pgfqpoint{5.700000in}{5.700000in}}%
\pgfusepath{clip}%
\pgfsetbuttcap%
\pgfsetroundjoin%
\definecolor{currentfill}{rgb}{0.282290,0.145912,0.461510}%
\pgfsetfillcolor{currentfill}%
\pgfsetfillopacity{0.800000}%
\pgfsetlinewidth{0.000000pt}%
\definecolor{currentstroke}{rgb}{0.000000,0.000000,0.000000}%
\pgfsetstrokecolor{currentstroke}%
\pgfsetdash{}{0pt}%
\pgfpathmoveto{\pgfqpoint{2.710268in}{1.960987in}}%
\pgfpathlineto{\pgfqpoint{2.724119in}{1.946074in}}%
\pgfpathlineto{\pgfqpoint{2.737964in}{1.931416in}}%
\pgfpathlineto{\pgfqpoint{2.751803in}{1.917011in}}%
\pgfpathlineto{\pgfqpoint{2.765637in}{1.902858in}}%
\pgfpathlineto{\pgfqpoint{2.774291in}{1.904978in}}%
\pgfpathlineto{\pgfqpoint{2.782931in}{1.907332in}}%
\pgfpathlineto{\pgfqpoint{2.791558in}{1.909915in}}%
\pgfpathlineto{\pgfqpoint{2.800170in}{1.912721in}}%
\pgfpathlineto{\pgfqpoint{2.786374in}{1.926330in}}%
\pgfpathlineto{\pgfqpoint{2.772574in}{1.940189in}}%
\pgfpathlineto{\pgfqpoint{2.758768in}{1.954301in}}%
\pgfpathlineto{\pgfqpoint{2.744957in}{1.968667in}}%
\pgfpathlineto{\pgfqpoint{2.736306in}{1.966393in}}%
\pgfpathlineto{\pgfqpoint{2.727642in}{1.964352in}}%
\pgfpathlineto{\pgfqpoint{2.718962in}{1.962548in}}%
\pgfpathlineto{\pgfqpoint{2.710268in}{1.960987in}}%
\pgfpathclose%
\pgfusepath{fill}%
\end{pgfscope}%
\begin{pgfscope}%
\pgfpathrectangle{\pgfqpoint{1.150000in}{0.150000in}}{\pgfqpoint{5.700000in}{5.700000in}}%
\pgfusepath{clip}%
\pgfsetbuttcap%
\pgfsetroundjoin%
\definecolor{currentfill}{rgb}{0.182256,0.426184,0.557120}%
\pgfsetfillcolor{currentfill}%
\pgfsetfillopacity{0.800000}%
\pgfsetlinewidth{0.000000pt}%
\definecolor{currentstroke}{rgb}{0.000000,0.000000,0.000000}%
\pgfsetstrokecolor{currentstroke}%
\pgfsetdash{}{0pt}%
\pgfpathmoveto{\pgfqpoint{4.770574in}{2.628469in}}%
\pgfpathlineto{\pgfqpoint{4.784786in}{2.638512in}}%
\pgfpathlineto{\pgfqpoint{4.799014in}{2.648739in}}%
\pgfpathlineto{\pgfqpoint{4.813259in}{2.659149in}}%
\pgfpathlineto{\pgfqpoint{4.827521in}{2.669743in}}%
\pgfpathlineto{\pgfqpoint{4.835264in}{2.678190in}}%
\pgfpathlineto{\pgfqpoint{4.843000in}{2.686527in}}%
\pgfpathlineto{\pgfqpoint{4.850729in}{2.694755in}}%
\pgfpathlineto{\pgfqpoint{4.858451in}{2.702878in}}%
\pgfpathlineto{\pgfqpoint{4.844197in}{2.692431in}}%
\pgfpathlineto{\pgfqpoint{4.829960in}{2.682167in}}%
\pgfpathlineto{\pgfqpoint{4.815739in}{2.672087in}}%
\pgfpathlineto{\pgfqpoint{4.801534in}{2.662190in}}%
\pgfpathlineto{\pgfqpoint{4.793805in}{2.653909in}}%
\pgfpathlineto{\pgfqpoint{4.786068in}{2.645529in}}%
\pgfpathlineto{\pgfqpoint{4.778324in}{2.637050in}}%
\pgfpathlineto{\pgfqpoint{4.770574in}{2.628469in}}%
\pgfpathclose%
\pgfusepath{fill}%
\end{pgfscope}%
\begin{pgfscope}%
\pgfpathrectangle{\pgfqpoint{1.150000in}{0.150000in}}{\pgfqpoint{5.700000in}{5.700000in}}%
\pgfusepath{clip}%
\pgfsetbuttcap%
\pgfsetroundjoin%
\definecolor{currentfill}{rgb}{0.271305,0.019942,0.347269}%
\pgfsetfillcolor{currentfill}%
\pgfsetfillopacity{0.800000}%
\pgfsetlinewidth{0.000000pt}%
\definecolor{currentstroke}{rgb}{0.000000,0.000000,0.000000}%
\pgfsetstrokecolor{currentstroke}%
\pgfsetdash{}{0pt}%
\pgfpathmoveto{\pgfqpoint{3.449640in}{1.663105in}}%
\pgfpathlineto{\pgfqpoint{3.463357in}{1.660006in}}%
\pgfpathlineto{\pgfqpoint{3.477079in}{1.657108in}}%
\pgfpathlineto{\pgfqpoint{3.490806in}{1.654409in}}%
\pgfpathlineto{\pgfqpoint{3.504537in}{1.651910in}}%
\pgfpathlineto{\pgfqpoint{3.512747in}{1.661775in}}%
\pgfpathlineto{\pgfqpoint{3.520951in}{1.671711in}}%
\pgfpathlineto{\pgfqpoint{3.529149in}{1.681714in}}%
\pgfpathlineto{\pgfqpoint{3.537340in}{1.691780in}}%
\pgfpathlineto{\pgfqpoint{3.523623in}{1.693877in}}%
\pgfpathlineto{\pgfqpoint{3.509911in}{1.696172in}}%
\pgfpathlineto{\pgfqpoint{3.496205in}{1.698668in}}%
\pgfpathlineto{\pgfqpoint{3.482503in}{1.701364in}}%
\pgfpathlineto{\pgfqpoint{3.474297in}{1.691688in}}%
\pgfpathlineto{\pgfqpoint{3.466085in}{1.682084in}}%
\pgfpathlineto{\pgfqpoint{3.457866in}{1.672555in}}%
\pgfpathlineto{\pgfqpoint{3.449640in}{1.663105in}}%
\pgfpathclose%
\pgfusepath{fill}%
\end{pgfscope}%
\begin{pgfscope}%
\pgfpathrectangle{\pgfqpoint{1.150000in}{0.150000in}}{\pgfqpoint{5.700000in}{5.700000in}}%
\pgfusepath{clip}%
\pgfsetbuttcap%
\pgfsetroundjoin%
\definecolor{currentfill}{rgb}{0.257322,0.256130,0.526563}%
\pgfsetfillcolor{currentfill}%
\pgfsetfillopacity{0.800000}%
\pgfsetlinewidth{0.000000pt}%
\definecolor{currentstroke}{rgb}{0.000000,0.000000,0.000000}%
\pgfsetstrokecolor{currentstroke}%
\pgfsetdash{}{0pt}%
\pgfpathmoveto{\pgfqpoint{2.487632in}{2.235902in}}%
\pgfpathlineto{\pgfqpoint{2.501614in}{2.216632in}}%
\pgfpathlineto{\pgfqpoint{2.515585in}{2.197651in}}%
\pgfpathlineto{\pgfqpoint{2.529547in}{2.178956in}}%
\pgfpathlineto{\pgfqpoint{2.543499in}{2.160546in}}%
\pgfpathlineto{\pgfqpoint{2.552325in}{2.160411in}}%
\pgfpathlineto{\pgfqpoint{2.561133in}{2.160545in}}%
\pgfpathlineto{\pgfqpoint{2.569925in}{2.160943in}}%
\pgfpathlineto{\pgfqpoint{2.578700in}{2.161600in}}%
\pgfpathlineto{\pgfqpoint{2.564794in}{2.179452in}}%
\pgfpathlineto{\pgfqpoint{2.550878in}{2.197587in}}%
\pgfpathlineto{\pgfqpoint{2.536953in}{2.216008in}}%
\pgfpathlineto{\pgfqpoint{2.523019in}{2.234716in}}%
\pgfpathlineto{\pgfqpoint{2.514198in}{2.234606in}}%
\pgfpathlineto{\pgfqpoint{2.505361in}{2.234763in}}%
\pgfpathlineto{\pgfqpoint{2.496505in}{2.235193in}}%
\pgfpathlineto{\pgfqpoint{2.487632in}{2.235902in}}%
\pgfpathclose%
\pgfusepath{fill}%
\end{pgfscope}%
\begin{pgfscope}%
\pgfpathrectangle{\pgfqpoint{1.150000in}{0.150000in}}{\pgfqpoint{5.700000in}{5.700000in}}%
\pgfusepath{clip}%
\pgfsetbuttcap%
\pgfsetroundjoin%
\definecolor{currentfill}{rgb}{0.255645,0.260703,0.528312}%
\pgfsetfillcolor{currentfill}%
\pgfsetfillopacity{0.800000}%
\pgfsetlinewidth{0.000000pt}%
\definecolor{currentstroke}{rgb}{0.000000,0.000000,0.000000}%
\pgfsetstrokecolor{currentstroke}%
\pgfsetdash{}{0pt}%
\pgfpathmoveto{\pgfqpoint{4.269204in}{2.177926in}}%
\pgfpathlineto{\pgfqpoint{4.283158in}{2.184358in}}%
\pgfpathlineto{\pgfqpoint{4.297124in}{2.190975in}}%
\pgfpathlineto{\pgfqpoint{4.311103in}{2.197779in}}%
\pgfpathlineto{\pgfqpoint{4.325094in}{2.204769in}}%
\pgfpathlineto{\pgfqpoint{4.333034in}{2.216234in}}%
\pgfpathlineto{\pgfqpoint{4.340968in}{2.227611in}}%
\pgfpathlineto{\pgfqpoint{4.348897in}{2.238899in}}%
\pgfpathlineto{\pgfqpoint{4.356820in}{2.250099in}}%
\pgfpathlineto{\pgfqpoint{4.342831in}{2.243023in}}%
\pgfpathlineto{\pgfqpoint{4.328856in}{2.236132in}}%
\pgfpathlineto{\pgfqpoint{4.314893in}{2.229428in}}%
\pgfpathlineto{\pgfqpoint{4.300943in}{2.222910in}}%
\pgfpathlineto{\pgfqpoint{4.293017in}{2.211784in}}%
\pgfpathlineto{\pgfqpoint{4.285084in}{2.200578in}}%
\pgfpathlineto{\pgfqpoint{4.277147in}{2.189292in}}%
\pgfpathlineto{\pgfqpoint{4.269204in}{2.177926in}}%
\pgfpathclose%
\pgfusepath{fill}%
\end{pgfscope}%
\begin{pgfscope}%
\pgfpathrectangle{\pgfqpoint{1.150000in}{0.150000in}}{\pgfqpoint{5.700000in}{5.700000in}}%
\pgfusepath{clip}%
\pgfsetbuttcap%
\pgfsetroundjoin%
\definecolor{currentfill}{rgb}{0.120638,0.625828,0.533488}%
\pgfsetfillcolor{currentfill}%
\pgfsetfillopacity{0.800000}%
\pgfsetlinewidth{0.000000pt}%
\definecolor{currentstroke}{rgb}{0.000000,0.000000,0.000000}%
\pgfsetstrokecolor{currentstroke}%
\pgfsetdash{}{0pt}%
\pgfpathmoveto{\pgfqpoint{5.534659in}{3.235619in}}%
\pgfpathlineto{\pgfqpoint{5.549319in}{3.248472in}}%
\pgfpathlineto{\pgfqpoint{5.564000in}{3.261504in}}%
\pgfpathlineto{\pgfqpoint{5.578701in}{3.274717in}}%
\pgfpathlineto{\pgfqpoint{5.593423in}{3.288111in}}%
\pgfpathlineto{\pgfqpoint{5.600730in}{3.290742in}}%
\pgfpathlineto{\pgfqpoint{5.608030in}{3.293343in}}%
\pgfpathlineto{\pgfqpoint{5.615323in}{3.295919in}}%
\pgfpathlineto{\pgfqpoint{5.622608in}{3.298476in}}%
\pgfpathlineto{\pgfqpoint{5.607911in}{3.285575in}}%
\pgfpathlineto{\pgfqpoint{5.593235in}{3.272855in}}%
\pgfpathlineto{\pgfqpoint{5.578579in}{3.260314in}}%
\pgfpathlineto{\pgfqpoint{5.563943in}{3.247952in}}%
\pgfpathlineto{\pgfqpoint{5.556633in}{3.244892in}}%
\pgfpathlineto{\pgfqpoint{5.549315in}{3.241820in}}%
\pgfpathlineto{\pgfqpoint{5.541991in}{3.238731in}}%
\pgfpathlineto{\pgfqpoint{5.534659in}{3.235619in}}%
\pgfpathclose%
\pgfusepath{fill}%
\end{pgfscope}%
\begin{pgfscope}%
\pgfpathrectangle{\pgfqpoint{1.150000in}{0.150000in}}{\pgfqpoint{5.700000in}{5.700000in}}%
\pgfusepath{clip}%
\pgfsetbuttcap%
\pgfsetroundjoin%
\definecolor{currentfill}{rgb}{0.281412,0.155834,0.469201}%
\pgfsetfillcolor{currentfill}%
\pgfsetfillopacity{0.800000}%
\pgfsetlinewidth{0.000000pt}%
\definecolor{currentstroke}{rgb}{0.000000,0.000000,0.000000}%
\pgfsetstrokecolor{currentstroke}%
\pgfsetdash{}{0pt}%
\pgfpathmoveto{\pgfqpoint{3.974732in}{1.931423in}}%
\pgfpathlineto{\pgfqpoint{3.988566in}{1.934934in}}%
\pgfpathlineto{\pgfqpoint{4.002409in}{1.938634in}}%
\pgfpathlineto{\pgfqpoint{4.016263in}{1.942522in}}%
\pgfpathlineto{\pgfqpoint{4.030127in}{1.946599in}}%
\pgfpathlineto{\pgfqpoint{4.038157in}{1.958650in}}%
\pgfpathlineto{\pgfqpoint{4.046181in}{1.970656in}}%
\pgfpathlineto{\pgfqpoint{4.054202in}{1.982616in}}%
\pgfpathlineto{\pgfqpoint{4.062217in}{1.994527in}}%
\pgfpathlineto{\pgfqpoint{4.048357in}{1.990235in}}%
\pgfpathlineto{\pgfqpoint{4.034508in}{1.986131in}}%
\pgfpathlineto{\pgfqpoint{4.020669in}{1.982216in}}%
\pgfpathlineto{\pgfqpoint{4.006841in}{1.978491in}}%
\pgfpathlineto{\pgfqpoint{3.998821in}{1.966782in}}%
\pgfpathlineto{\pgfqpoint{3.990796in}{1.955033in}}%
\pgfpathlineto{\pgfqpoint{3.982767in}{1.943246in}}%
\pgfpathlineto{\pgfqpoint{3.974732in}{1.931423in}}%
\pgfpathclose%
\pgfusepath{fill}%
\end{pgfscope}%
\begin{pgfscope}%
\pgfpathrectangle{\pgfqpoint{1.150000in}{0.150000in}}{\pgfqpoint{5.700000in}{5.700000in}}%
\pgfusepath{clip}%
\pgfsetbuttcap%
\pgfsetroundjoin%
\definecolor{currentfill}{rgb}{0.283229,0.120777,0.440584}%
\pgfsetfillcolor{currentfill}%
\pgfsetfillopacity{0.800000}%
\pgfsetlinewidth{0.000000pt}%
\definecolor{currentstroke}{rgb}{0.000000,0.000000,0.000000}%
\pgfsetstrokecolor{currentstroke}%
\pgfsetdash{}{0pt}%
\pgfpathmoveto{\pgfqpoint{2.765637in}{1.902858in}}%
\pgfpathlineto{\pgfqpoint{2.779465in}{1.888955in}}%
\pgfpathlineto{\pgfqpoint{2.793289in}{1.875299in}}%
\pgfpathlineto{\pgfqpoint{2.807108in}{1.861890in}}%
\pgfpathlineto{\pgfqpoint{2.820922in}{1.848725in}}%
\pgfpathlineto{\pgfqpoint{2.829538in}{1.851401in}}%
\pgfpathlineto{\pgfqpoint{2.838141in}{1.854303in}}%
\pgfpathlineto{\pgfqpoint{2.846730in}{1.857425in}}%
\pgfpathlineto{\pgfqpoint{2.855306in}{1.860762in}}%
\pgfpathlineto{\pgfqpoint{2.841528in}{1.873384in}}%
\pgfpathlineto{\pgfqpoint{2.827746in}{1.886250in}}%
\pgfpathlineto{\pgfqpoint{2.813960in}{1.899362in}}%
\pgfpathlineto{\pgfqpoint{2.800170in}{1.912721in}}%
\pgfpathlineto{\pgfqpoint{2.791558in}{1.909915in}}%
\pgfpathlineto{\pgfqpoint{2.782931in}{1.907332in}}%
\pgfpathlineto{\pgfqpoint{2.774291in}{1.904978in}}%
\pgfpathlineto{\pgfqpoint{2.765637in}{1.902858in}}%
\pgfpathclose%
\pgfusepath{fill}%
\end{pgfscope}%
\begin{pgfscope}%
\pgfpathrectangle{\pgfqpoint{1.150000in}{0.150000in}}{\pgfqpoint{5.700000in}{5.700000in}}%
\pgfusepath{clip}%
\pgfsetbuttcap%
\pgfsetroundjoin%
\definecolor{currentfill}{rgb}{0.268510,0.009605,0.335427}%
\pgfsetfillcolor{currentfill}%
\pgfsetfillopacity{0.800000}%
\pgfsetlinewidth{0.000000pt}%
\definecolor{currentstroke}{rgb}{0.000000,0.000000,0.000000}%
\pgfsetstrokecolor{currentstroke}%
\pgfsetdash{}{0pt}%
\pgfpathmoveto{\pgfqpoint{3.218784in}{1.654775in}}%
\pgfpathlineto{\pgfqpoint{3.232505in}{1.648304in}}%
\pgfpathlineto{\pgfqpoint{3.246227in}{1.642044in}}%
\pgfpathlineto{\pgfqpoint{3.259950in}{1.635994in}}%
\pgfpathlineto{\pgfqpoint{3.273676in}{1.630152in}}%
\pgfpathlineto{\pgfqpoint{3.282001in}{1.637866in}}%
\pgfpathlineto{\pgfqpoint{3.290316in}{1.645708in}}%
\pgfpathlineto{\pgfqpoint{3.298624in}{1.653674in}}%
\pgfpathlineto{\pgfqpoint{3.306923in}{1.661759in}}%
\pgfpathlineto{\pgfqpoint{3.293219in}{1.667134in}}%
\pgfpathlineto{\pgfqpoint{3.279517in}{1.672717in}}%
\pgfpathlineto{\pgfqpoint{3.265817in}{1.678510in}}%
\pgfpathlineto{\pgfqpoint{3.252119in}{1.684513in}}%
\pgfpathlineto{\pgfqpoint{3.243798in}{1.676883in}}%
\pgfpathlineto{\pgfqpoint{3.235469in}{1.669380in}}%
\pgfpathlineto{\pgfqpoint{3.227131in}{1.662010in}}%
\pgfpathlineto{\pgfqpoint{3.218784in}{1.654775in}}%
\pgfpathclose%
\pgfusepath{fill}%
\end{pgfscope}%
\begin{pgfscope}%
\pgfpathrectangle{\pgfqpoint{1.150000in}{0.150000in}}{\pgfqpoint{5.700000in}{5.700000in}}%
\pgfusepath{clip}%
\pgfsetbuttcap%
\pgfsetroundjoin%
\definecolor{currentfill}{rgb}{0.244972,0.287675,0.537260}%
\pgfsetfillcolor{currentfill}%
\pgfsetfillopacity{0.800000}%
\pgfsetlinewidth{0.000000pt}%
\definecolor{currentstroke}{rgb}{0.000000,0.000000,0.000000}%
\pgfsetstrokecolor{currentstroke}%
\pgfsetdash{}{0pt}%
\pgfpathmoveto{\pgfqpoint{2.431596in}{2.315923in}}%
\pgfpathlineto{\pgfqpoint{2.445622in}{2.295471in}}%
\pgfpathlineto{\pgfqpoint{2.459636in}{2.275319in}}%
\pgfpathlineto{\pgfqpoint{2.473640in}{2.255463in}}%
\pgfpathlineto{\pgfqpoint{2.487632in}{2.235902in}}%
\pgfpathlineto{\pgfqpoint{2.496505in}{2.235193in}}%
\pgfpathlineto{\pgfqpoint{2.505361in}{2.234763in}}%
\pgfpathlineto{\pgfqpoint{2.514198in}{2.234606in}}%
\pgfpathlineto{\pgfqpoint{2.523019in}{2.234716in}}%
\pgfpathlineto{\pgfqpoint{2.509074in}{2.253716in}}%
\pgfpathlineto{\pgfqpoint{2.495119in}{2.273008in}}%
\pgfpathlineto{\pgfqpoint{2.481154in}{2.292596in}}%
\pgfpathlineto{\pgfqpoint{2.467177in}{2.312482in}}%
\pgfpathlineto{\pgfqpoint{2.458309in}{2.312922in}}%
\pgfpathlineto{\pgfqpoint{2.449423in}{2.313638in}}%
\pgfpathlineto{\pgfqpoint{2.440519in}{2.314637in}}%
\pgfpathlineto{\pgfqpoint{2.431596in}{2.315923in}}%
\pgfpathclose%
\pgfusepath{fill}%
\end{pgfscope}%
\begin{pgfscope}%
\pgfpathrectangle{\pgfqpoint{1.150000in}{0.150000in}}{\pgfqpoint{5.700000in}{5.700000in}}%
\pgfusepath{clip}%
\pgfsetbuttcap%
\pgfsetroundjoin%
\definecolor{currentfill}{rgb}{0.210503,0.363727,0.552206}%
\pgfsetfillcolor{currentfill}%
\pgfsetfillopacity{0.800000}%
\pgfsetlinewidth{0.000000pt}%
\definecolor{currentstroke}{rgb}{0.000000,0.000000,0.000000}%
\pgfsetstrokecolor{currentstroke}%
\pgfsetdash{}{0pt}%
\pgfpathmoveto{\pgfqpoint{4.563776in}{2.441361in}}%
\pgfpathlineto{\pgfqpoint{4.577880in}{2.450168in}}%
\pgfpathlineto{\pgfqpoint{4.591999in}{2.459159in}}%
\pgfpathlineto{\pgfqpoint{4.606134in}{2.468335in}}%
\pgfpathlineto{\pgfqpoint{4.620283in}{2.477696in}}%
\pgfpathlineto{\pgfqpoint{4.628119in}{2.487649in}}%
\pgfpathlineto{\pgfqpoint{4.635948in}{2.497494in}}%
\pgfpathlineto{\pgfqpoint{4.643771in}{2.507229in}}%
\pgfpathlineto{\pgfqpoint{4.651588in}{2.516856in}}%
\pgfpathlineto{\pgfqpoint{4.637443in}{2.507541in}}%
\pgfpathlineto{\pgfqpoint{4.623314in}{2.498411in}}%
\pgfpathlineto{\pgfqpoint{4.609200in}{2.489465in}}%
\pgfpathlineto{\pgfqpoint{4.595101in}{2.480704in}}%
\pgfpathlineto{\pgfqpoint{4.587279in}{2.471019in}}%
\pgfpathlineto{\pgfqpoint{4.579451in}{2.461234in}}%
\pgfpathlineto{\pgfqpoint{4.571617in}{2.451348in}}%
\pgfpathlineto{\pgfqpoint{4.563776in}{2.441361in}}%
\pgfpathclose%
\pgfusepath{fill}%
\end{pgfscope}%
\begin{pgfscope}%
\pgfpathrectangle{\pgfqpoint{1.150000in}{0.150000in}}{\pgfqpoint{5.700000in}{5.700000in}}%
\pgfusepath{clip}%
\pgfsetbuttcap%
\pgfsetroundjoin%
\definecolor{currentfill}{rgb}{0.128087,0.647749,0.523491}%
\pgfsetfillcolor{currentfill}%
\pgfsetfillopacity{0.800000}%
\pgfsetlinewidth{0.000000pt}%
\definecolor{currentstroke}{rgb}{0.000000,0.000000,0.000000}%
\pgfsetstrokecolor{currentstroke}%
\pgfsetdash{}{0pt}%
\pgfpathmoveto{\pgfqpoint{5.622608in}{3.298476in}}%
\pgfpathlineto{\pgfqpoint{5.637325in}{3.311556in}}%
\pgfpathlineto{\pgfqpoint{5.652064in}{3.324816in}}%
\pgfpathlineto{\pgfqpoint{5.666823in}{3.338256in}}%
\pgfpathlineto{\pgfqpoint{5.681603in}{3.351877in}}%
\pgfpathlineto{\pgfqpoint{5.688855in}{3.353904in}}%
\pgfpathlineto{\pgfqpoint{5.696099in}{3.355916in}}%
\pgfpathlineto{\pgfqpoint{5.703335in}{3.357917in}}%
\pgfpathlineto{\pgfqpoint{5.710565in}{3.359913in}}%
\pgfpathlineto{\pgfqpoint{5.695812in}{3.346821in}}%
\pgfpathlineto{\pgfqpoint{5.681080in}{3.333909in}}%
\pgfpathlineto{\pgfqpoint{5.666369in}{3.321175in}}%
\pgfpathlineto{\pgfqpoint{5.651678in}{3.308621in}}%
\pgfpathlineto{\pgfqpoint{5.644421in}{3.306086in}}%
\pgfpathlineto{\pgfqpoint{5.637157in}{3.303553in}}%
\pgfpathlineto{\pgfqpoint{5.629886in}{3.301019in}}%
\pgfpathlineto{\pgfqpoint{5.622608in}{3.298476in}}%
\pgfpathclose%
\pgfusepath{fill}%
\end{pgfscope}%
\begin{pgfscope}%
\pgfpathrectangle{\pgfqpoint{1.150000in}{0.150000in}}{\pgfqpoint{5.700000in}{5.700000in}}%
\pgfusepath{clip}%
\pgfsetbuttcap%
\pgfsetroundjoin%
\definecolor{currentfill}{rgb}{0.139147,0.533812,0.555298}%
\pgfsetfillcolor{currentfill}%
\pgfsetfillopacity{0.800000}%
\pgfsetlinewidth{0.000000pt}%
\definecolor{currentstroke}{rgb}{0.000000,0.000000,0.000000}%
\pgfsetstrokecolor{currentstroke}%
\pgfsetdash{}{0pt}%
\pgfpathmoveto{\pgfqpoint{5.152921in}{2.948919in}}%
\pgfpathlineto{\pgfqpoint{5.167362in}{2.960812in}}%
\pgfpathlineto{\pgfqpoint{5.181821in}{2.972887in}}%
\pgfpathlineto{\pgfqpoint{5.196300in}{2.985145in}}%
\pgfpathlineto{\pgfqpoint{5.210797in}{2.997586in}}%
\pgfpathlineto{\pgfqpoint{5.218344in}{3.003083in}}%
\pgfpathlineto{\pgfqpoint{5.225883in}{3.008492in}}%
\pgfpathlineto{\pgfqpoint{5.233414in}{3.013817in}}%
\pgfpathlineto{\pgfqpoint{5.240936in}{3.019061in}}%
\pgfpathlineto{\pgfqpoint{5.226454in}{3.006941in}}%
\pgfpathlineto{\pgfqpoint{5.211990in}{2.995002in}}%
\pgfpathlineto{\pgfqpoint{5.197546in}{2.983245in}}%
\pgfpathlineto{\pgfqpoint{5.183119in}{2.971670in}}%
\pgfpathlineto{\pgfqpoint{5.175581in}{2.966094in}}%
\pgfpathlineto{\pgfqpoint{5.168036in}{2.960446in}}%
\pgfpathlineto{\pgfqpoint{5.160482in}{2.954722in}}%
\pgfpathlineto{\pgfqpoint{5.152921in}{2.948919in}}%
\pgfpathclose%
\pgfusepath{fill}%
\end{pgfscope}%
\begin{pgfscope}%
\pgfpathrectangle{\pgfqpoint{1.150000in}{0.150000in}}{\pgfqpoint{5.700000in}{5.700000in}}%
\pgfusepath{clip}%
\pgfsetbuttcap%
\pgfsetroundjoin%
\definecolor{currentfill}{rgb}{0.282656,0.100196,0.422160}%
\pgfsetfillcolor{currentfill}%
\pgfsetfillopacity{0.800000}%
\pgfsetlinewidth{0.000000pt}%
\definecolor{currentstroke}{rgb}{0.000000,0.000000,0.000000}%
\pgfsetstrokecolor{currentstroke}%
\pgfsetdash{}{0pt}%
\pgfpathmoveto{\pgfqpoint{2.820922in}{1.848725in}}%
\pgfpathlineto{\pgfqpoint{2.834732in}{1.835804in}}%
\pgfpathlineto{\pgfqpoint{2.848538in}{1.823124in}}%
\pgfpathlineto{\pgfqpoint{2.862340in}{1.810684in}}%
\pgfpathlineto{\pgfqpoint{2.876138in}{1.798482in}}%
\pgfpathlineto{\pgfqpoint{2.884718in}{1.801712in}}%
\pgfpathlineto{\pgfqpoint{2.893285in}{1.805158in}}%
\pgfpathlineto{\pgfqpoint{2.901839in}{1.808817in}}%
\pgfpathlineto{\pgfqpoint{2.910380in}{1.812682in}}%
\pgfpathlineto{\pgfqpoint{2.896617in}{1.824344in}}%
\pgfpathlineto{\pgfqpoint{2.882850in}{1.836243in}}%
\pgfpathlineto{\pgfqpoint{2.869080in}{1.848382in}}%
\pgfpathlineto{\pgfqpoint{2.855306in}{1.860762in}}%
\pgfpathlineto{\pgfqpoint{2.846730in}{1.857425in}}%
\pgfpathlineto{\pgfqpoint{2.838141in}{1.854303in}}%
\pgfpathlineto{\pgfqpoint{2.829538in}{1.851401in}}%
\pgfpathlineto{\pgfqpoint{2.820922in}{1.848725in}}%
\pgfpathclose%
\pgfusepath{fill}%
\end{pgfscope}%
\begin{pgfscope}%
\pgfpathrectangle{\pgfqpoint{1.150000in}{0.150000in}}{\pgfqpoint{5.700000in}{5.700000in}}%
\pgfusepath{clip}%
\pgfsetbuttcap%
\pgfsetroundjoin%
\definecolor{currentfill}{rgb}{0.272594,0.025563,0.353093}%
\pgfsetfillcolor{currentfill}%
\pgfsetfillopacity{0.800000}%
\pgfsetlinewidth{0.000000pt}%
\definecolor{currentstroke}{rgb}{0.000000,0.000000,0.000000}%
\pgfsetstrokecolor{currentstroke}%
\pgfsetdash{}{0pt}%
\pgfpathmoveto{\pgfqpoint{3.075371in}{1.690779in}}%
\pgfpathlineto{\pgfqpoint{3.089113in}{1.682083in}}%
\pgfpathlineto{\pgfqpoint{3.102854in}{1.673607in}}%
\pgfpathlineto{\pgfqpoint{3.116595in}{1.665349in}}%
\pgfpathlineto{\pgfqpoint{3.130336in}{1.657309in}}%
\pgfpathlineto{\pgfqpoint{3.138746in}{1.663438in}}%
\pgfpathlineto{\pgfqpoint{3.147145in}{1.669730in}}%
\pgfpathlineto{\pgfqpoint{3.155535in}{1.676182in}}%
\pgfpathlineto{\pgfqpoint{3.163915in}{1.682787in}}%
\pgfpathlineto{\pgfqpoint{3.150200in}{1.690327in}}%
\pgfpathlineto{\pgfqpoint{3.136486in}{1.698083in}}%
\pgfpathlineto{\pgfqpoint{3.122772in}{1.706058in}}%
\pgfpathlineto{\pgfqpoint{3.109058in}{1.714252in}}%
\pgfpathlineto{\pgfqpoint{3.100652in}{1.708136in}}%
\pgfpathlineto{\pgfqpoint{3.092235in}{1.702182in}}%
\pgfpathlineto{\pgfqpoint{3.083809in}{1.696394in}}%
\pgfpathlineto{\pgfqpoint{3.075371in}{1.690779in}}%
\pgfpathclose%
\pgfusepath{fill}%
\end{pgfscope}%
\begin{pgfscope}%
\pgfpathrectangle{\pgfqpoint{1.150000in}{0.150000in}}{\pgfqpoint{5.700000in}{5.700000in}}%
\pgfusepath{clip}%
\pgfsetbuttcap%
\pgfsetroundjoin%
\definecolor{currentfill}{rgb}{0.268510,0.009605,0.335427}%
\pgfsetfillcolor{currentfill}%
\pgfsetfillopacity{0.800000}%
\pgfsetlinewidth{0.000000pt}%
\definecolor{currentstroke}{rgb}{0.000000,0.000000,0.000000}%
\pgfsetstrokecolor{currentstroke}%
\pgfsetdash{}{0pt}%
\pgfpathmoveto{\pgfqpoint{3.361769in}{1.642331in}}%
\pgfpathlineto{\pgfqpoint{3.375488in}{1.637987in}}%
\pgfpathlineto{\pgfqpoint{3.389211in}{1.633847in}}%
\pgfpathlineto{\pgfqpoint{3.402937in}{1.629910in}}%
\pgfpathlineto{\pgfqpoint{3.416667in}{1.626175in}}%
\pgfpathlineto{\pgfqpoint{3.424921in}{1.635269in}}%
\pgfpathlineto{\pgfqpoint{3.433168in}{1.644458in}}%
\pgfpathlineto{\pgfqpoint{3.441407in}{1.653739in}}%
\pgfpathlineto{\pgfqpoint{3.449640in}{1.663105in}}%
\pgfpathlineto{\pgfqpoint{3.435927in}{1.666406in}}%
\pgfpathlineto{\pgfqpoint{3.422218in}{1.669908in}}%
\pgfpathlineto{\pgfqpoint{3.408513in}{1.673614in}}%
\pgfpathlineto{\pgfqpoint{3.394812in}{1.677523in}}%
\pgfpathlineto{\pgfqpoint{3.386562in}{1.668579in}}%
\pgfpathlineto{\pgfqpoint{3.378305in}{1.659729in}}%
\pgfpathlineto{\pgfqpoint{3.370041in}{1.650979in}}%
\pgfpathlineto{\pgfqpoint{3.361769in}{1.642331in}}%
\pgfpathclose%
\pgfusepath{fill}%
\end{pgfscope}%
\begin{pgfscope}%
\pgfpathrectangle{\pgfqpoint{1.150000in}{0.150000in}}{\pgfqpoint{5.700000in}{5.700000in}}%
\pgfusepath{clip}%
\pgfsetbuttcap%
\pgfsetroundjoin%
\definecolor{currentfill}{rgb}{0.277134,0.185228,0.489898}%
\pgfsetfillcolor{currentfill}%
\pgfsetfillopacity{0.800000}%
\pgfsetlinewidth{0.000000pt}%
\definecolor{currentstroke}{rgb}{0.000000,0.000000,0.000000}%
\pgfsetstrokecolor{currentstroke}%
\pgfsetdash{}{0pt}%
\pgfpathmoveto{\pgfqpoint{4.062217in}{1.994527in}}%
\pgfpathlineto{\pgfqpoint{4.076087in}{1.999007in}}%
\pgfpathlineto{\pgfqpoint{4.089969in}{2.003676in}}%
\pgfpathlineto{\pgfqpoint{4.103861in}{2.008532in}}%
\pgfpathlineto{\pgfqpoint{4.117765in}{2.013575in}}%
\pgfpathlineto{\pgfqpoint{4.125771in}{2.025632in}}%
\pgfpathlineto{\pgfqpoint{4.133773in}{2.037630in}}%
\pgfpathlineto{\pgfqpoint{4.141769in}{2.049568in}}%
\pgfpathlineto{\pgfqpoint{4.149761in}{2.061443in}}%
\pgfpathlineto{\pgfqpoint{4.135861in}{2.056216in}}%
\pgfpathlineto{\pgfqpoint{4.121972in}{2.051177in}}%
\pgfpathlineto{\pgfqpoint{4.108095in}{2.046325in}}%
\pgfpathlineto{\pgfqpoint{4.094229in}{2.041662in}}%
\pgfpathlineto{\pgfqpoint{4.086233in}{2.029958in}}%
\pgfpathlineto{\pgfqpoint{4.078233in}{2.018200in}}%
\pgfpathlineto{\pgfqpoint{4.070227in}{2.006389in}}%
\pgfpathlineto{\pgfqpoint{4.062217in}{1.994527in}}%
\pgfpathclose%
\pgfusepath{fill}%
\end{pgfscope}%
\begin{pgfscope}%
\pgfpathrectangle{\pgfqpoint{1.150000in}{0.150000in}}{\pgfqpoint{5.700000in}{5.700000in}}%
\pgfusepath{clip}%
\pgfsetbuttcap%
\pgfsetroundjoin%
\definecolor{currentfill}{rgb}{0.169646,0.456262,0.558030}%
\pgfsetfillcolor{currentfill}%
\pgfsetfillopacity{0.800000}%
\pgfsetlinewidth{0.000000pt}%
\definecolor{currentstroke}{rgb}{0.000000,0.000000,0.000000}%
\pgfsetstrokecolor{currentstroke}%
\pgfsetdash{}{0pt}%
\pgfpathmoveto{\pgfqpoint{4.858451in}{2.702878in}}%
\pgfpathlineto{\pgfqpoint{4.872722in}{2.713508in}}%
\pgfpathlineto{\pgfqpoint{4.887010in}{2.724322in}}%
\pgfpathlineto{\pgfqpoint{4.901315in}{2.735320in}}%
\pgfpathlineto{\pgfqpoint{4.915637in}{2.746501in}}%
\pgfpathlineto{\pgfqpoint{4.923344in}{2.754350in}}%
\pgfpathlineto{\pgfqpoint{4.931043in}{2.762090in}}%
\pgfpathlineto{\pgfqpoint{4.938734in}{2.769722in}}%
\pgfpathlineto{\pgfqpoint{4.946418in}{2.777249in}}%
\pgfpathlineto{\pgfqpoint{4.932104in}{2.766250in}}%
\pgfpathlineto{\pgfqpoint{4.917808in}{2.755434in}}%
\pgfpathlineto{\pgfqpoint{4.903529in}{2.744801in}}%
\pgfpathlineto{\pgfqpoint{4.889267in}{2.734351in}}%
\pgfpathlineto{\pgfqpoint{4.881574in}{2.726631in}}%
\pgfpathlineto{\pgfqpoint{4.873873in}{2.718813in}}%
\pgfpathlineto{\pgfqpoint{4.866166in}{2.710897in}}%
\pgfpathlineto{\pgfqpoint{4.858451in}{2.702878in}}%
\pgfpathclose%
\pgfusepath{fill}%
\end{pgfscope}%
\begin{pgfscope}%
\pgfpathrectangle{\pgfqpoint{1.150000in}{0.150000in}}{\pgfqpoint{5.700000in}{5.700000in}}%
\pgfusepath{clip}%
\pgfsetbuttcap%
\pgfsetroundjoin%
\definecolor{currentfill}{rgb}{0.241237,0.296485,0.539709}%
\pgfsetfillcolor{currentfill}%
\pgfsetfillopacity{0.800000}%
\pgfsetlinewidth{0.000000pt}%
\definecolor{currentstroke}{rgb}{0.000000,0.000000,0.000000}%
\pgfsetstrokecolor{currentstroke}%
\pgfsetdash{}{0pt}%
\pgfpathmoveto{\pgfqpoint{4.356820in}{2.250099in}}%
\pgfpathlineto{\pgfqpoint{4.370822in}{2.257362in}}%
\pgfpathlineto{\pgfqpoint{4.384837in}{2.264810in}}%
\pgfpathlineto{\pgfqpoint{4.398866in}{2.272443in}}%
\pgfpathlineto{\pgfqpoint{4.412909in}{2.280262in}}%
\pgfpathlineto{\pgfqpoint{4.420824in}{2.291440in}}%
\pgfpathlineto{\pgfqpoint{4.428733in}{2.302520in}}%
\pgfpathlineto{\pgfqpoint{4.436636in}{2.313503in}}%
\pgfpathlineto{\pgfqpoint{4.444534in}{2.324388in}}%
\pgfpathlineto{\pgfqpoint{4.430494in}{2.316515in}}%
\pgfpathlineto{\pgfqpoint{4.416469in}{2.308828in}}%
\pgfpathlineto{\pgfqpoint{4.402457in}{2.301326in}}%
\pgfpathlineto{\pgfqpoint{4.388458in}{2.294009in}}%
\pgfpathlineto{\pgfqpoint{4.380557in}{2.283166in}}%
\pgfpathlineto{\pgfqpoint{4.372650in}{2.272233in}}%
\pgfpathlineto{\pgfqpoint{4.364738in}{2.261211in}}%
\pgfpathlineto{\pgfqpoint{4.356820in}{2.250099in}}%
\pgfpathclose%
\pgfusepath{fill}%
\end{pgfscope}%
\begin{pgfscope}%
\pgfpathrectangle{\pgfqpoint{1.150000in}{0.150000in}}{\pgfqpoint{5.700000in}{5.700000in}}%
\pgfusepath{clip}%
\pgfsetbuttcap%
\pgfsetroundjoin%
\definecolor{currentfill}{rgb}{0.172719,0.448791,0.557885}%
\pgfsetfillcolor{currentfill}%
\pgfsetfillopacity{0.800000}%
\pgfsetlinewidth{0.000000pt}%
\definecolor{currentstroke}{rgb}{0.000000,0.000000,0.000000}%
\pgfsetstrokecolor{currentstroke}%
\pgfsetdash{}{0pt}%
\pgfpathmoveto{\pgfqpoint{2.184844in}{2.777237in}}%
\pgfpathlineto{\pgfqpoint{2.199109in}{2.750766in}}%
\pgfpathlineto{\pgfqpoint{2.213357in}{2.724656in}}%
\pgfpathlineto{\pgfqpoint{2.227587in}{2.698903in}}%
\pgfpathlineto{\pgfqpoint{2.241800in}{2.673504in}}%
\pgfpathlineto{\pgfqpoint{2.250857in}{2.671121in}}%
\pgfpathlineto{\pgfqpoint{2.259894in}{2.669042in}}%
\pgfpathlineto{\pgfqpoint{2.268910in}{2.667260in}}%
\pgfpathlineto{\pgfqpoint{2.277906in}{2.665769in}}%
\pgfpathlineto{\pgfqpoint{2.263748in}{2.690617in}}%
\pgfpathlineto{\pgfqpoint{2.249574in}{2.715818in}}%
\pgfpathlineto{\pgfqpoint{2.235382in}{2.741374in}}%
\pgfpathlineto{\pgfqpoint{2.221173in}{2.767290in}}%
\pgfpathlineto{\pgfqpoint{2.212122in}{2.769319in}}%
\pgfpathlineto{\pgfqpoint{2.203050in}{2.771650in}}%
\pgfpathlineto{\pgfqpoint{2.193958in}{2.774287in}}%
\pgfpathlineto{\pgfqpoint{2.184844in}{2.777237in}}%
\pgfpathclose%
\pgfusepath{fill}%
\end{pgfscope}%
\begin{pgfscope}%
\pgfpathrectangle{\pgfqpoint{1.150000in}{0.150000in}}{\pgfqpoint{5.700000in}{5.700000in}}%
\pgfusepath{clip}%
\pgfsetbuttcap%
\pgfsetroundjoin%
\definecolor{currentfill}{rgb}{0.140210,0.665859,0.513427}%
\pgfsetfillcolor{currentfill}%
\pgfsetfillopacity{0.800000}%
\pgfsetlinewidth{0.000000pt}%
\definecolor{currentstroke}{rgb}{0.000000,0.000000,0.000000}%
\pgfsetstrokecolor{currentstroke}%
\pgfsetdash{}{0pt}%
\pgfpathmoveto{\pgfqpoint{5.710565in}{3.359913in}}%
\pgfpathlineto{\pgfqpoint{5.725339in}{3.373185in}}%
\pgfpathlineto{\pgfqpoint{5.740134in}{3.386636in}}%
\pgfpathlineto{\pgfqpoint{5.754951in}{3.400267in}}%
\pgfpathlineto{\pgfqpoint{5.769789in}{3.414079in}}%
\pgfpathlineto{\pgfqpoint{5.776982in}{3.415526in}}%
\pgfpathlineto{\pgfqpoint{5.784169in}{3.416973in}}%
\pgfpathlineto{\pgfqpoint{5.791349in}{3.418425in}}%
\pgfpathlineto{\pgfqpoint{5.798522in}{3.419890in}}%
\pgfpathlineto{\pgfqpoint{5.783714in}{3.406643in}}%
\pgfpathlineto{\pgfqpoint{5.768927in}{3.393575in}}%
\pgfpathlineto{\pgfqpoint{5.754161in}{3.380685in}}%
\pgfpathlineto{\pgfqpoint{5.739417in}{3.367975in}}%
\pgfpathlineto{\pgfqpoint{5.732213in}{3.365936in}}%
\pgfpathlineto{\pgfqpoint{5.725004in}{3.363917in}}%
\pgfpathlineto{\pgfqpoint{5.717788in}{3.361911in}}%
\pgfpathlineto{\pgfqpoint{5.710565in}{3.359913in}}%
\pgfpathclose%
\pgfusepath{fill}%
\end{pgfscope}%
\begin{pgfscope}%
\pgfpathrectangle{\pgfqpoint{1.150000in}{0.150000in}}{\pgfqpoint{5.700000in}{5.700000in}}%
\pgfusepath{clip}%
\pgfsetbuttcap%
\pgfsetroundjoin%
\definecolor{currentfill}{rgb}{0.229739,0.322361,0.545706}%
\pgfsetfillcolor{currentfill}%
\pgfsetfillopacity{0.800000}%
\pgfsetlinewidth{0.000000pt}%
\definecolor{currentstroke}{rgb}{0.000000,0.000000,0.000000}%
\pgfsetstrokecolor{currentstroke}%
\pgfsetdash{}{0pt}%
\pgfpathmoveto{\pgfqpoint{2.375372in}{2.400775in}}%
\pgfpathlineto{\pgfqpoint{2.389447in}{2.379100in}}%
\pgfpathlineto{\pgfqpoint{2.403509in}{2.357734in}}%
\pgfpathlineto{\pgfqpoint{2.417559in}{2.336676in}}%
\pgfpathlineto{\pgfqpoint{2.431596in}{2.315923in}}%
\pgfpathlineto{\pgfqpoint{2.440519in}{2.314637in}}%
\pgfpathlineto{\pgfqpoint{2.449423in}{2.313638in}}%
\pgfpathlineto{\pgfqpoint{2.458309in}{2.312922in}}%
\pgfpathlineto{\pgfqpoint{2.467177in}{2.312482in}}%
\pgfpathlineto{\pgfqpoint{2.453189in}{2.332668in}}%
\pgfpathlineto{\pgfqpoint{2.439190in}{2.353158in}}%
\pgfpathlineto{\pgfqpoint{2.425179in}{2.373955in}}%
\pgfpathlineto{\pgfqpoint{2.411155in}{2.395060in}}%
\pgfpathlineto{\pgfqpoint{2.402238in}{2.396055in}}%
\pgfpathlineto{\pgfqpoint{2.393302in}{2.397335in}}%
\pgfpathlineto{\pgfqpoint{2.384347in}{2.398907in}}%
\pgfpathlineto{\pgfqpoint{2.375372in}{2.400775in}}%
\pgfpathclose%
\pgfusepath{fill}%
\end{pgfscope}%
\begin{pgfscope}%
\pgfpathrectangle{\pgfqpoint{1.150000in}{0.150000in}}{\pgfqpoint{5.700000in}{5.700000in}}%
\pgfusepath{clip}%
\pgfsetbuttcap%
\pgfsetroundjoin%
\definecolor{currentfill}{rgb}{0.280894,0.078907,0.402329}%
\pgfsetfillcolor{currentfill}%
\pgfsetfillopacity{0.800000}%
\pgfsetlinewidth{0.000000pt}%
\definecolor{currentstroke}{rgb}{0.000000,0.000000,0.000000}%
\pgfsetstrokecolor{currentstroke}%
\pgfsetdash{}{0pt}%
\pgfpathmoveto{\pgfqpoint{2.876138in}{1.798482in}}%
\pgfpathlineto{\pgfqpoint{2.889933in}{1.786517in}}%
\pgfpathlineto{\pgfqpoint{2.903725in}{1.774788in}}%
\pgfpathlineto{\pgfqpoint{2.917513in}{1.763292in}}%
\pgfpathlineto{\pgfqpoint{2.931299in}{1.752029in}}%
\pgfpathlineto{\pgfqpoint{2.939845in}{1.755810in}}%
\pgfpathlineto{\pgfqpoint{2.948378in}{1.759800in}}%
\pgfpathlineto{\pgfqpoint{2.956898in}{1.763993in}}%
\pgfpathlineto{\pgfqpoint{2.965407in}{1.768384in}}%
\pgfpathlineto{\pgfqpoint{2.951654in}{1.779109in}}%
\pgfpathlineto{\pgfqpoint{2.937899in}{1.790066in}}%
\pgfpathlineto{\pgfqpoint{2.924141in}{1.801257in}}%
\pgfpathlineto{\pgfqpoint{2.910380in}{1.812682in}}%
\pgfpathlineto{\pgfqpoint{2.901839in}{1.808817in}}%
\pgfpathlineto{\pgfqpoint{2.893285in}{1.805158in}}%
\pgfpathlineto{\pgfqpoint{2.884718in}{1.801712in}}%
\pgfpathlineto{\pgfqpoint{2.876138in}{1.798482in}}%
\pgfpathclose%
\pgfusepath{fill}%
\end{pgfscope}%
\begin{pgfscope}%
\pgfpathrectangle{\pgfqpoint{1.150000in}{0.150000in}}{\pgfqpoint{5.700000in}{5.700000in}}%
\pgfusepath{clip}%
\pgfsetbuttcap%
\pgfsetroundjoin%
\definecolor{currentfill}{rgb}{0.129933,0.559582,0.551864}%
\pgfsetfillcolor{currentfill}%
\pgfsetfillopacity{0.800000}%
\pgfsetlinewidth{0.000000pt}%
\definecolor{currentstroke}{rgb}{0.000000,0.000000,0.000000}%
\pgfsetstrokecolor{currentstroke}%
\pgfsetdash{}{0pt}%
\pgfpathmoveto{\pgfqpoint{5.240936in}{3.019061in}}%
\pgfpathlineto{\pgfqpoint{5.255438in}{3.031364in}}%
\pgfpathlineto{\pgfqpoint{5.269958in}{3.043848in}}%
\pgfpathlineto{\pgfqpoint{5.284498in}{3.056514in}}%
\pgfpathlineto{\pgfqpoint{5.299058in}{3.069363in}}%
\pgfpathlineto{\pgfqpoint{5.306556in}{3.074189in}}%
\pgfpathlineto{\pgfqpoint{5.314047in}{3.078934in}}%
\pgfpathlineto{\pgfqpoint{5.321529in}{3.083602in}}%
\pgfpathlineto{\pgfqpoint{5.329003in}{3.088198in}}%
\pgfpathlineto{\pgfqpoint{5.314461in}{3.075704in}}%
\pgfpathlineto{\pgfqpoint{5.299937in}{3.063393in}}%
\pgfpathlineto{\pgfqpoint{5.285433in}{3.051262in}}%
\pgfpathlineto{\pgfqpoint{5.270948in}{3.039313in}}%
\pgfpathlineto{\pgfqpoint{5.263457in}{3.034351in}}%
\pgfpathlineto{\pgfqpoint{5.255958in}{3.029324in}}%
\pgfpathlineto{\pgfqpoint{5.248451in}{3.024229in}}%
\pgfpathlineto{\pgfqpoint{5.240936in}{3.019061in}}%
\pgfpathclose%
\pgfusepath{fill}%
\end{pgfscope}%
\begin{pgfscope}%
\pgfpathrectangle{\pgfqpoint{1.150000in}{0.150000in}}{\pgfqpoint{5.700000in}{5.700000in}}%
\pgfusepath{clip}%
\pgfsetbuttcap%
\pgfsetroundjoin%
\definecolor{currentfill}{rgb}{0.278791,0.062145,0.386592}%
\pgfsetfillcolor{currentfill}%
\pgfsetfillopacity{0.800000}%
\pgfsetlinewidth{0.000000pt}%
\definecolor{currentstroke}{rgb}{0.000000,0.000000,0.000000}%
\pgfsetstrokecolor{currentstroke}%
\pgfsetdash{}{0pt}%
\pgfpathmoveto{\pgfqpoint{3.679889in}{1.725912in}}%
\pgfpathlineto{\pgfqpoint{3.693650in}{1.725953in}}%
\pgfpathlineto{\pgfqpoint{3.707417in}{1.726188in}}%
\pgfpathlineto{\pgfqpoint{3.721192in}{1.726616in}}%
\pgfpathlineto{\pgfqpoint{3.734974in}{1.727236in}}%
\pgfpathlineto{\pgfqpoint{3.743103in}{1.738624in}}%
\pgfpathlineto{\pgfqpoint{3.751226in}{1.750032in}}%
\pgfpathlineto{\pgfqpoint{3.759343in}{1.761456in}}%
\pgfpathlineto{\pgfqpoint{3.767456in}{1.772893in}}%
\pgfpathlineto{\pgfqpoint{3.753683in}{1.771933in}}%
\pgfpathlineto{\pgfqpoint{3.739917in}{1.771164in}}%
\pgfpathlineto{\pgfqpoint{3.726159in}{1.770589in}}%
\pgfpathlineto{\pgfqpoint{3.712408in}{1.770207in}}%
\pgfpathlineto{\pgfqpoint{3.704287in}{1.759099in}}%
\pgfpathlineto{\pgfqpoint{3.696160in}{1.748011in}}%
\pgfpathlineto{\pgfqpoint{3.688027in}{1.736948in}}%
\pgfpathlineto{\pgfqpoint{3.679889in}{1.725912in}}%
\pgfpathclose%
\pgfusepath{fill}%
\end{pgfscope}%
\begin{pgfscope}%
\pgfpathrectangle{\pgfqpoint{1.150000in}{0.150000in}}{\pgfqpoint{5.700000in}{5.700000in}}%
\pgfusepath{clip}%
\pgfsetbuttcap%
\pgfsetroundjoin%
\definecolor{currentfill}{rgb}{0.269308,0.218818,0.509577}%
\pgfsetfillcolor{currentfill}%
\pgfsetfillopacity{0.800000}%
\pgfsetlinewidth{0.000000pt}%
\definecolor{currentstroke}{rgb}{0.000000,0.000000,0.000000}%
\pgfsetstrokecolor{currentstroke}%
\pgfsetdash{}{0pt}%
\pgfpathmoveto{\pgfqpoint{4.149761in}{2.061443in}}%
\pgfpathlineto{\pgfqpoint{4.163673in}{2.066857in}}%
\pgfpathlineto{\pgfqpoint{4.177596in}{2.072458in}}%
\pgfpathlineto{\pgfqpoint{4.191531in}{2.078245in}}%
\pgfpathlineto{\pgfqpoint{4.205478in}{2.084220in}}%
\pgfpathlineto{\pgfqpoint{4.213462in}{2.096195in}}%
\pgfpathlineto{\pgfqpoint{4.221440in}{2.108098in}}%
\pgfpathlineto{\pgfqpoint{4.229414in}{2.119926in}}%
\pgfpathlineto{\pgfqpoint{4.237382in}{2.131680in}}%
\pgfpathlineto{\pgfqpoint{4.223438in}{2.125554in}}%
\pgfpathlineto{\pgfqpoint{4.209506in}{2.119615in}}%
\pgfpathlineto{\pgfqpoint{4.195586in}{2.113862in}}%
\pgfpathlineto{\pgfqpoint{4.181678in}{2.108297in}}%
\pgfpathlineto{\pgfqpoint{4.173706in}{2.096683in}}%
\pgfpathlineto{\pgfqpoint{4.165730in}{2.085002in}}%
\pgfpathlineto{\pgfqpoint{4.157748in}{2.073254in}}%
\pgfpathlineto{\pgfqpoint{4.149761in}{2.061443in}}%
\pgfpathclose%
\pgfusepath{fill}%
\end{pgfscope}%
\begin{pgfscope}%
\pgfpathrectangle{\pgfqpoint{1.150000in}{0.150000in}}{\pgfqpoint{5.700000in}{5.700000in}}%
\pgfusepath{clip}%
\pgfsetbuttcap%
\pgfsetroundjoin%
\definecolor{currentfill}{rgb}{0.162016,0.687316,0.499129}%
\pgfsetfillcolor{currentfill}%
\pgfsetfillopacity{0.800000}%
\pgfsetlinewidth{0.000000pt}%
\definecolor{currentstroke}{rgb}{0.000000,0.000000,0.000000}%
\pgfsetstrokecolor{currentstroke}%
\pgfsetdash{}{0pt}%
\pgfpathmoveto{\pgfqpoint{5.798522in}{3.419890in}}%
\pgfpathlineto{\pgfqpoint{5.813351in}{3.433316in}}%
\pgfpathlineto{\pgfqpoint{5.828202in}{3.446922in}}%
\pgfpathlineto{\pgfqpoint{5.843075in}{3.460708in}}%
\pgfpathlineto{\pgfqpoint{5.857970in}{3.474673in}}%
\pgfpathlineto{\pgfqpoint{5.865105in}{3.475570in}}%
\pgfpathlineto{\pgfqpoint{5.872233in}{3.476483in}}%
\pgfpathlineto{\pgfqpoint{5.879355in}{3.477420in}}%
\pgfpathlineto{\pgfqpoint{5.886470in}{3.478387in}}%
\pgfpathlineto{\pgfqpoint{5.871608in}{3.465021in}}%
\pgfpathlineto{\pgfqpoint{5.856768in}{3.451834in}}%
\pgfpathlineto{\pgfqpoint{5.841949in}{3.438825in}}%
\pgfpathlineto{\pgfqpoint{5.827152in}{3.425994in}}%
\pgfpathlineto{\pgfqpoint{5.820003in}{3.424419in}}%
\pgfpathlineto{\pgfqpoint{5.812849in}{3.422880in}}%
\pgfpathlineto{\pgfqpoint{5.805688in}{3.421373in}}%
\pgfpathlineto{\pgfqpoint{5.798522in}{3.419890in}}%
\pgfpathclose%
\pgfusepath{fill}%
\end{pgfscope}%
\begin{pgfscope}%
\pgfpathrectangle{\pgfqpoint{1.150000in}{0.150000in}}{\pgfqpoint{5.700000in}{5.700000in}}%
\pgfusepath{clip}%
\pgfsetbuttcap%
\pgfsetroundjoin%
\definecolor{currentfill}{rgb}{0.281924,0.089666,0.412415}%
\pgfsetfillcolor{currentfill}%
\pgfsetfillopacity{0.800000}%
\pgfsetlinewidth{0.000000pt}%
\definecolor{currentstroke}{rgb}{0.000000,0.000000,0.000000}%
\pgfsetstrokecolor{currentstroke}%
\pgfsetdash{}{0pt}%
\pgfpathmoveto{\pgfqpoint{3.767456in}{1.772893in}}%
\pgfpathlineto{\pgfqpoint{3.781237in}{1.774047in}}%
\pgfpathlineto{\pgfqpoint{3.795026in}{1.775391in}}%
\pgfpathlineto{\pgfqpoint{3.808824in}{1.776927in}}%
\pgfpathlineto{\pgfqpoint{3.822630in}{1.778654in}}%
\pgfpathlineto{\pgfqpoint{3.830729in}{1.790423in}}%
\pgfpathlineto{\pgfqpoint{3.838824in}{1.802192in}}%
\pgfpathlineto{\pgfqpoint{3.846913in}{1.813957in}}%
\pgfpathlineto{\pgfqpoint{3.854998in}{1.825716in}}%
\pgfpathlineto{\pgfqpoint{3.841199in}{1.823680in}}%
\pgfpathlineto{\pgfqpoint{3.827409in}{1.821834in}}%
\pgfpathlineto{\pgfqpoint{3.813628in}{1.820180in}}%
\pgfpathlineto{\pgfqpoint{3.799854in}{1.818717in}}%
\pgfpathlineto{\pgfqpoint{3.791762in}{1.807256in}}%
\pgfpathlineto{\pgfqpoint{3.783665in}{1.795796in}}%
\pgfpathlineto{\pgfqpoint{3.775563in}{1.784341in}}%
\pgfpathlineto{\pgfqpoint{3.767456in}{1.772893in}}%
\pgfpathclose%
\pgfusepath{fill}%
\end{pgfscope}%
\begin{pgfscope}%
\pgfpathrectangle{\pgfqpoint{1.150000in}{0.150000in}}{\pgfqpoint{5.700000in}{5.700000in}}%
\pgfusepath{clip}%
\pgfsetbuttcap%
\pgfsetroundjoin%
\definecolor{currentfill}{rgb}{0.276022,0.044167,0.370164}%
\pgfsetfillcolor{currentfill}%
\pgfsetfillopacity{0.800000}%
\pgfsetlinewidth{0.000000pt}%
\definecolor{currentstroke}{rgb}{0.000000,0.000000,0.000000}%
\pgfsetstrokecolor{currentstroke}%
\pgfsetdash{}{0pt}%
\pgfpathmoveto{\pgfqpoint{3.592263in}{1.685375in}}%
\pgfpathlineto{\pgfqpoint{3.606008in}{1.684265in}}%
\pgfpathlineto{\pgfqpoint{3.619760in}{1.683351in}}%
\pgfpathlineto{\pgfqpoint{3.633518in}{1.682632in}}%
\pgfpathlineto{\pgfqpoint{3.647282in}{1.682107in}}%
\pgfpathlineto{\pgfqpoint{3.655442in}{1.693001in}}%
\pgfpathlineto{\pgfqpoint{3.663597in}{1.703935in}}%
\pgfpathlineto{\pgfqpoint{3.671746in}{1.714907in}}%
\pgfpathlineto{\pgfqpoint{3.679889in}{1.725912in}}%
\pgfpathlineto{\pgfqpoint{3.666136in}{1.726065in}}%
\pgfpathlineto{\pgfqpoint{3.652389in}{1.726412in}}%
\pgfpathlineto{\pgfqpoint{3.638649in}{1.726954in}}%
\pgfpathlineto{\pgfqpoint{3.624916in}{1.727692in}}%
\pgfpathlineto{\pgfqpoint{3.616761in}{1.717047in}}%
\pgfpathlineto{\pgfqpoint{3.608601in}{1.706443in}}%
\pgfpathlineto{\pgfqpoint{3.600435in}{1.695885in}}%
\pgfpathlineto{\pgfqpoint{3.592263in}{1.685375in}}%
\pgfpathclose%
\pgfusepath{fill}%
\end{pgfscope}%
\begin{pgfscope}%
\pgfpathrectangle{\pgfqpoint{1.150000in}{0.150000in}}{\pgfqpoint{5.700000in}{5.700000in}}%
\pgfusepath{clip}%
\pgfsetbuttcap%
\pgfsetroundjoin%
\definecolor{currentfill}{rgb}{0.195860,0.395433,0.555276}%
\pgfsetfillcolor{currentfill}%
\pgfsetfillopacity{0.800000}%
\pgfsetlinewidth{0.000000pt}%
\definecolor{currentstroke}{rgb}{0.000000,0.000000,0.000000}%
\pgfsetstrokecolor{currentstroke}%
\pgfsetdash{}{0pt}%
\pgfpathmoveto{\pgfqpoint{4.651588in}{2.516856in}}%
\pgfpathlineto{\pgfqpoint{4.665748in}{2.526355in}}%
\pgfpathlineto{\pgfqpoint{4.679924in}{2.536038in}}%
\pgfpathlineto{\pgfqpoint{4.694116in}{2.545906in}}%
\pgfpathlineto{\pgfqpoint{4.708324in}{2.555958in}}%
\pgfpathlineto{\pgfqpoint{4.716129in}{2.565411in}}%
\pgfpathlineto{\pgfqpoint{4.723927in}{2.574751in}}%
\pgfpathlineto{\pgfqpoint{4.731719in}{2.583978in}}%
\pgfpathlineto{\pgfqpoint{4.739503in}{2.593093in}}%
\pgfpathlineto{\pgfqpoint{4.725301in}{2.583121in}}%
\pgfpathlineto{\pgfqpoint{4.711115in}{2.573333in}}%
\pgfpathlineto{\pgfqpoint{4.696944in}{2.563729in}}%
\pgfpathlineto{\pgfqpoint{4.682790in}{2.554309in}}%
\pgfpathlineto{\pgfqpoint{4.674999in}{2.545102in}}%
\pgfpathlineto{\pgfqpoint{4.667202in}{2.535791in}}%
\pgfpathlineto{\pgfqpoint{4.659398in}{2.526377in}}%
\pgfpathlineto{\pgfqpoint{4.651588in}{2.516856in}}%
\pgfpathclose%
\pgfusepath{fill}%
\end{pgfscope}%
\begin{pgfscope}%
\pgfpathrectangle{\pgfqpoint{1.150000in}{0.150000in}}{\pgfqpoint{5.700000in}{5.700000in}}%
\pgfusepath{clip}%
\pgfsetbuttcap%
\pgfsetroundjoin%
\definecolor{currentfill}{rgb}{0.268510,0.009605,0.335427}%
\pgfsetfillcolor{currentfill}%
\pgfsetfillopacity{0.800000}%
\pgfsetlinewidth{0.000000pt}%
\definecolor{currentstroke}{rgb}{0.000000,0.000000,0.000000}%
\pgfsetstrokecolor{currentstroke}%
\pgfsetdash{}{0pt}%
\pgfpathmoveto{\pgfqpoint{3.273676in}{1.630152in}}%
\pgfpathlineto{\pgfqpoint{3.287404in}{1.624518in}}%
\pgfpathlineto{\pgfqpoint{3.301135in}{1.619092in}}%
\pgfpathlineto{\pgfqpoint{3.314868in}{1.613872in}}%
\pgfpathlineto{\pgfqpoint{3.328603in}{1.608857in}}%
\pgfpathlineto{\pgfqpoint{3.336907in}{1.617049in}}%
\pgfpathlineto{\pgfqpoint{3.345202in}{1.625362in}}%
\pgfpathlineto{\pgfqpoint{3.353489in}{1.633791in}}%
\pgfpathlineto{\pgfqpoint{3.361769in}{1.642331in}}%
\pgfpathlineto{\pgfqpoint{3.348053in}{1.646879in}}%
\pgfpathlineto{\pgfqpoint{3.334340in}{1.651633in}}%
\pgfpathlineto{\pgfqpoint{3.320630in}{1.656592in}}%
\pgfpathlineto{\pgfqpoint{3.306923in}{1.661759in}}%
\pgfpathlineto{\pgfqpoint{3.298624in}{1.653674in}}%
\pgfpathlineto{\pgfqpoint{3.290316in}{1.645708in}}%
\pgfpathlineto{\pgfqpoint{3.282001in}{1.637866in}}%
\pgfpathlineto{\pgfqpoint{3.273676in}{1.630152in}}%
\pgfpathclose%
\pgfusepath{fill}%
\end{pgfscope}%
\begin{pgfscope}%
\pgfpathrectangle{\pgfqpoint{1.150000in}{0.150000in}}{\pgfqpoint{5.700000in}{5.700000in}}%
\pgfusepath{clip}%
\pgfsetbuttcap%
\pgfsetroundjoin%
\definecolor{currentfill}{rgb}{0.159194,0.482237,0.558073}%
\pgfsetfillcolor{currentfill}%
\pgfsetfillopacity{0.800000}%
\pgfsetlinewidth{0.000000pt}%
\definecolor{currentstroke}{rgb}{0.000000,0.000000,0.000000}%
\pgfsetstrokecolor{currentstroke}%
\pgfsetdash{}{0pt}%
\pgfpathmoveto{\pgfqpoint{4.946418in}{2.777249in}}%
\pgfpathlineto{\pgfqpoint{4.960749in}{2.788431in}}%
\pgfpathlineto{\pgfqpoint{4.975098in}{2.799797in}}%
\pgfpathlineto{\pgfqpoint{4.989464in}{2.811345in}}%
\pgfpathlineto{\pgfqpoint{5.003848in}{2.823078in}}%
\pgfpathlineto{\pgfqpoint{5.011515in}{2.830298in}}%
\pgfpathlineto{\pgfqpoint{5.019174in}{2.837409in}}%
\pgfpathlineto{\pgfqpoint{5.026826in}{2.844414in}}%
\pgfpathlineto{\pgfqpoint{5.034469in}{2.851316in}}%
\pgfpathlineto{\pgfqpoint{5.020095in}{2.839801in}}%
\pgfpathlineto{\pgfqpoint{5.005739in}{2.828468in}}%
\pgfpathlineto{\pgfqpoint{4.991400in}{2.817319in}}%
\pgfpathlineto{\pgfqpoint{4.977079in}{2.806352in}}%
\pgfpathlineto{\pgfqpoint{4.969425in}{2.799221in}}%
\pgfpathlineto{\pgfqpoint{4.961763in}{2.791996in}}%
\pgfpathlineto{\pgfqpoint{4.954094in}{2.784672in}}%
\pgfpathlineto{\pgfqpoint{4.946418in}{2.777249in}}%
\pgfpathclose%
\pgfusepath{fill}%
\end{pgfscope}%
\begin{pgfscope}%
\pgfpathrectangle{\pgfqpoint{1.150000in}{0.150000in}}{\pgfqpoint{5.700000in}{5.700000in}}%
\pgfusepath{clip}%
\pgfsetbuttcap%
\pgfsetroundjoin%
\definecolor{currentfill}{rgb}{0.283197,0.115680,0.436115}%
\pgfsetfillcolor{currentfill}%
\pgfsetfillopacity{0.800000}%
\pgfsetlinewidth{0.000000pt}%
\definecolor{currentstroke}{rgb}{0.000000,0.000000,0.000000}%
\pgfsetstrokecolor{currentstroke}%
\pgfsetdash{}{0pt}%
\pgfpathmoveto{\pgfqpoint{3.854998in}{1.825716in}}%
\pgfpathlineto{\pgfqpoint{3.868805in}{1.827943in}}%
\pgfpathlineto{\pgfqpoint{3.882621in}{1.830360in}}%
\pgfpathlineto{\pgfqpoint{3.896447in}{1.832967in}}%
\pgfpathlineto{\pgfqpoint{3.910282in}{1.835763in}}%
\pgfpathlineto{\pgfqpoint{3.918355in}{1.847804in}}%
\pgfpathlineto{\pgfqpoint{3.926423in}{1.859826in}}%
\pgfpathlineto{\pgfqpoint{3.934487in}{1.871826in}}%
\pgfpathlineto{\pgfqpoint{3.942546in}{1.883803in}}%
\pgfpathlineto{\pgfqpoint{3.928717in}{1.880728in}}%
\pgfpathlineto{\pgfqpoint{3.914897in}{1.877842in}}%
\pgfpathlineto{\pgfqpoint{3.901087in}{1.875147in}}%
\pgfpathlineto{\pgfqpoint{3.887286in}{1.872642in}}%
\pgfpathlineto{\pgfqpoint{3.879221in}{1.860932in}}%
\pgfpathlineto{\pgfqpoint{3.871152in}{1.849206in}}%
\pgfpathlineto{\pgfqpoint{3.863077in}{1.837467in}}%
\pgfpathlineto{\pgfqpoint{3.854998in}{1.825716in}}%
\pgfpathclose%
\pgfusepath{fill}%
\end{pgfscope}%
\begin{pgfscope}%
\pgfpathrectangle{\pgfqpoint{1.150000in}{0.150000in}}{\pgfqpoint{5.700000in}{5.700000in}}%
\pgfusepath{clip}%
\pgfsetbuttcap%
\pgfsetroundjoin%
\definecolor{currentfill}{rgb}{0.185783,0.704891,0.485273}%
\pgfsetfillcolor{currentfill}%
\pgfsetfillopacity{0.800000}%
\pgfsetlinewidth{0.000000pt}%
\definecolor{currentstroke}{rgb}{0.000000,0.000000,0.000000}%
\pgfsetstrokecolor{currentstroke}%
\pgfsetdash{}{0pt}%
\pgfpathmoveto{\pgfqpoint{5.886470in}{3.478387in}}%
\pgfpathlineto{\pgfqpoint{5.901354in}{3.491931in}}%
\pgfpathlineto{\pgfqpoint{5.916260in}{3.505655in}}%
\pgfpathlineto{\pgfqpoint{5.931188in}{3.519558in}}%
\pgfpathlineto{\pgfqpoint{5.946138in}{3.533641in}}%
\pgfpathlineto{\pgfqpoint{5.953213in}{3.534022in}}%
\pgfpathlineto{\pgfqpoint{5.960282in}{3.534440in}}%
\pgfpathlineto{\pgfqpoint{5.967346in}{3.534899in}}%
\pgfpathlineto{\pgfqpoint{5.974404in}{3.535408in}}%
\pgfpathlineto{\pgfqpoint{5.959490in}{3.521960in}}%
\pgfpathlineto{\pgfqpoint{5.944597in}{3.508690in}}%
\pgfpathlineto{\pgfqpoint{5.929727in}{3.495598in}}%
\pgfpathlineto{\pgfqpoint{5.914878in}{3.482684in}}%
\pgfpathlineto{\pgfqpoint{5.907784in}{3.481532in}}%
\pgfpathlineto{\pgfqpoint{5.900685in}{3.480436in}}%
\pgfpathlineto{\pgfqpoint{5.893580in}{3.479390in}}%
\pgfpathlineto{\pgfqpoint{5.886470in}{3.478387in}}%
\pgfpathclose%
\pgfusepath{fill}%
\end{pgfscope}%
\begin{pgfscope}%
\pgfpathrectangle{\pgfqpoint{1.150000in}{0.150000in}}{\pgfqpoint{5.700000in}{5.700000in}}%
\pgfusepath{clip}%
\pgfsetbuttcap%
\pgfsetroundjoin%
\definecolor{currentfill}{rgb}{0.214298,0.355619,0.551184}%
\pgfsetfillcolor{currentfill}%
\pgfsetfillopacity{0.800000}%
\pgfsetlinewidth{0.000000pt}%
\definecolor{currentstroke}{rgb}{0.000000,0.000000,0.000000}%
\pgfsetstrokecolor{currentstroke}%
\pgfsetdash{}{0pt}%
\pgfpathmoveto{\pgfqpoint{2.318939in}{2.490637in}}%
\pgfpathlineto{\pgfqpoint{2.333068in}{2.467692in}}%
\pgfpathlineto{\pgfqpoint{2.347183in}{2.445068in}}%
\pgfpathlineto{\pgfqpoint{2.361284in}{2.422764in}}%
\pgfpathlineto{\pgfqpoint{2.375372in}{2.400775in}}%
\pgfpathlineto{\pgfqpoint{2.384347in}{2.398907in}}%
\pgfpathlineto{\pgfqpoint{2.393302in}{2.397335in}}%
\pgfpathlineto{\pgfqpoint{2.402238in}{2.396055in}}%
\pgfpathlineto{\pgfqpoint{2.411155in}{2.395060in}}%
\pgfpathlineto{\pgfqpoint{2.397119in}{2.416477in}}%
\pgfpathlineto{\pgfqpoint{2.383070in}{2.438208in}}%
\pgfpathlineto{\pgfqpoint{2.369008in}{2.460257in}}%
\pgfpathlineto{\pgfqpoint{2.354933in}{2.482627in}}%
\pgfpathlineto{\pgfqpoint{2.345964in}{2.484182in}}%
\pgfpathlineto{\pgfqpoint{2.336976in}{2.486031in}}%
\pgfpathlineto{\pgfqpoint{2.327968in}{2.488181in}}%
\pgfpathlineto{\pgfqpoint{2.318939in}{2.490637in}}%
\pgfpathclose%
\pgfusepath{fill}%
\end{pgfscope}%
\begin{pgfscope}%
\pgfpathrectangle{\pgfqpoint{1.150000in}{0.150000in}}{\pgfqpoint{5.700000in}{5.700000in}}%
\pgfusepath{clip}%
\pgfsetbuttcap%
\pgfsetroundjoin%
\definecolor{currentfill}{rgb}{0.271305,0.019942,0.347269}%
\pgfsetfillcolor{currentfill}%
\pgfsetfillopacity{0.800000}%
\pgfsetlinewidth{0.000000pt}%
\definecolor{currentstroke}{rgb}{0.000000,0.000000,0.000000}%
\pgfsetstrokecolor{currentstroke}%
\pgfsetdash{}{0pt}%
\pgfpathmoveto{\pgfqpoint{3.130336in}{1.657309in}}%
\pgfpathlineto{\pgfqpoint{3.144077in}{1.649484in}}%
\pgfpathlineto{\pgfqpoint{3.157819in}{1.641875in}}%
\pgfpathlineto{\pgfqpoint{3.171562in}{1.634479in}}%
\pgfpathlineto{\pgfqpoint{3.185305in}{1.627296in}}%
\pgfpathlineto{\pgfqpoint{3.193689in}{1.633938in}}%
\pgfpathlineto{\pgfqpoint{3.202063in}{1.640735in}}%
\pgfpathlineto{\pgfqpoint{3.210428in}{1.647682in}}%
\pgfpathlineto{\pgfqpoint{3.218784in}{1.654775in}}%
\pgfpathlineto{\pgfqpoint{3.205065in}{1.661458in}}%
\pgfpathlineto{\pgfqpoint{3.191348in}{1.668354in}}%
\pgfpathlineto{\pgfqpoint{3.177631in}{1.675463in}}%
\pgfpathlineto{\pgfqpoint{3.163915in}{1.682787in}}%
\pgfpathlineto{\pgfqpoint{3.155535in}{1.676182in}}%
\pgfpathlineto{\pgfqpoint{3.147145in}{1.669730in}}%
\pgfpathlineto{\pgfqpoint{3.138746in}{1.663438in}}%
\pgfpathlineto{\pgfqpoint{3.130336in}{1.657309in}}%
\pgfpathclose%
\pgfusepath{fill}%
\end{pgfscope}%
\begin{pgfscope}%
\pgfpathrectangle{\pgfqpoint{1.150000in}{0.150000in}}{\pgfqpoint{5.700000in}{5.700000in}}%
\pgfusepath{clip}%
\pgfsetbuttcap%
\pgfsetroundjoin%
\definecolor{currentfill}{rgb}{0.272594,0.025563,0.353093}%
\pgfsetfillcolor{currentfill}%
\pgfsetfillopacity{0.800000}%
\pgfsetlinewidth{0.000000pt}%
\definecolor{currentstroke}{rgb}{0.000000,0.000000,0.000000}%
\pgfsetstrokecolor{currentstroke}%
\pgfsetdash{}{0pt}%
\pgfpathmoveto{\pgfqpoint{3.504537in}{1.651910in}}%
\pgfpathlineto{\pgfqpoint{3.518273in}{1.649609in}}%
\pgfpathlineto{\pgfqpoint{3.532015in}{1.647506in}}%
\pgfpathlineto{\pgfqpoint{3.545762in}{1.645601in}}%
\pgfpathlineto{\pgfqpoint{3.559514in}{1.643892in}}%
\pgfpathlineto{\pgfqpoint{3.567711in}{1.654172in}}%
\pgfpathlineto{\pgfqpoint{3.575901in}{1.664514in}}%
\pgfpathlineto{\pgfqpoint{3.584085in}{1.674917in}}%
\pgfpathlineto{\pgfqpoint{3.592263in}{1.685375in}}%
\pgfpathlineto{\pgfqpoint{3.578524in}{1.686681in}}%
\pgfpathlineto{\pgfqpoint{3.564790in}{1.688183in}}%
\pgfpathlineto{\pgfqpoint{3.551062in}{1.689883in}}%
\pgfpathlineto{\pgfqpoint{3.537340in}{1.691780in}}%
\pgfpathlineto{\pgfqpoint{3.529149in}{1.681714in}}%
\pgfpathlineto{\pgfqpoint{3.520951in}{1.671711in}}%
\pgfpathlineto{\pgfqpoint{3.512747in}{1.661775in}}%
\pgfpathlineto{\pgfqpoint{3.504537in}{1.651910in}}%
\pgfpathclose%
\pgfusepath{fill}%
\end{pgfscope}%
\begin{pgfscope}%
\pgfpathrectangle{\pgfqpoint{1.150000in}{0.150000in}}{\pgfqpoint{5.700000in}{5.700000in}}%
\pgfusepath{clip}%
\pgfsetbuttcap%
\pgfsetroundjoin%
\definecolor{currentfill}{rgb}{0.246070,0.738910,0.452024}%
\pgfsetfillcolor{currentfill}%
\pgfsetfillopacity{0.800000}%
\pgfsetlinewidth{0.000000pt}%
\definecolor{currentstroke}{rgb}{0.000000,0.000000,0.000000}%
\pgfsetstrokecolor{currentstroke}%
\pgfsetdash{}{0pt}%
\pgfpathmoveto{\pgfqpoint{6.062317in}{3.590981in}}%
\pgfpathlineto{\pgfqpoint{6.077305in}{3.604653in}}%
\pgfpathlineto{\pgfqpoint{6.092315in}{3.618503in}}%
\pgfpathlineto{\pgfqpoint{6.107349in}{3.632532in}}%
\pgfpathlineto{\pgfqpoint{6.114315in}{3.632188in}}%
\pgfpathlineto{\pgfqpoint{6.121276in}{3.631922in}}%
\pgfpathlineto{\pgfqpoint{6.128234in}{3.631743in}}%
\pgfpathlineto{\pgfqpoint{6.135188in}{3.631657in}}%
\pgfpathlineto{\pgfqpoint{6.120196in}{3.618332in}}%
\pgfpathlineto{\pgfqpoint{6.105227in}{3.605183in}}%
\pgfpathlineto{\pgfqpoint{6.090279in}{3.592211in}}%
\pgfpathlineto{\pgfqpoint{6.083294in}{3.591762in}}%
\pgfpathlineto{\pgfqpoint{6.076305in}{3.591413in}}%
\pgfpathlineto{\pgfqpoint{6.069313in}{3.591155in}}%
\pgfpathlineto{\pgfqpoint{6.062317in}{3.590981in}}%
\pgfpathclose%
\pgfusepath{fill}%
\end{pgfscope}%
\begin{pgfscope}%
\pgfpathrectangle{\pgfqpoint{1.150000in}{0.150000in}}{\pgfqpoint{5.700000in}{5.700000in}}%
\pgfusepath{clip}%
\pgfsetbuttcap%
\pgfsetroundjoin%
\definecolor{currentfill}{rgb}{0.278791,0.062145,0.386592}%
\pgfsetfillcolor{currentfill}%
\pgfsetfillopacity{0.800000}%
\pgfsetlinewidth{0.000000pt}%
\definecolor{currentstroke}{rgb}{0.000000,0.000000,0.000000}%
\pgfsetstrokecolor{currentstroke}%
\pgfsetdash{}{0pt}%
\pgfpathmoveto{\pgfqpoint{2.931299in}{1.752029in}}%
\pgfpathlineto{\pgfqpoint{2.945083in}{1.740998in}}%
\pgfpathlineto{\pgfqpoint{2.958864in}{1.730195in}}%
\pgfpathlineto{\pgfqpoint{2.972643in}{1.719622in}}%
\pgfpathlineto{\pgfqpoint{2.986420in}{1.709275in}}%
\pgfpathlineto{\pgfqpoint{2.994933in}{1.713605in}}%
\pgfpathlineto{\pgfqpoint{3.003433in}{1.718136in}}%
\pgfpathlineto{\pgfqpoint{3.011922in}{1.722862in}}%
\pgfpathlineto{\pgfqpoint{3.020399in}{1.727778in}}%
\pgfpathlineto{\pgfqpoint{3.006654in}{1.737588in}}%
\pgfpathlineto{\pgfqpoint{2.992906in}{1.747625in}}%
\pgfpathlineto{\pgfqpoint{2.979158in}{1.757890in}}%
\pgfpathlineto{\pgfqpoint{2.965407in}{1.768384in}}%
\pgfpathlineto{\pgfqpoint{2.956898in}{1.763993in}}%
\pgfpathlineto{\pgfqpoint{2.948378in}{1.759800in}}%
\pgfpathlineto{\pgfqpoint{2.939845in}{1.755810in}}%
\pgfpathlineto{\pgfqpoint{2.931299in}{1.752029in}}%
\pgfpathclose%
\pgfusepath{fill}%
\end{pgfscope}%
\begin{pgfscope}%
\pgfpathrectangle{\pgfqpoint{1.150000in}{0.150000in}}{\pgfqpoint{5.700000in}{5.700000in}}%
\pgfusepath{clip}%
\pgfsetbuttcap%
\pgfsetroundjoin%
\definecolor{currentfill}{rgb}{0.227802,0.326594,0.546532}%
\pgfsetfillcolor{currentfill}%
\pgfsetfillopacity{0.800000}%
\pgfsetlinewidth{0.000000pt}%
\definecolor{currentstroke}{rgb}{0.000000,0.000000,0.000000}%
\pgfsetstrokecolor{currentstroke}%
\pgfsetdash{}{0pt}%
\pgfpathmoveto{\pgfqpoint{4.444534in}{2.324388in}}%
\pgfpathlineto{\pgfqpoint{4.458588in}{2.332446in}}%
\pgfpathlineto{\pgfqpoint{4.472656in}{2.340689in}}%
\pgfpathlineto{\pgfqpoint{4.486738in}{2.349118in}}%
\pgfpathlineto{\pgfqpoint{4.500835in}{2.357731in}}%
\pgfpathlineto{\pgfqpoint{4.508723in}{2.368552in}}%
\pgfpathlineto{\pgfqpoint{4.516606in}{2.379268in}}%
\pgfpathlineto{\pgfqpoint{4.524483in}{2.389878in}}%
\pgfpathlineto{\pgfqpoint{4.532354in}{2.400383in}}%
\pgfpathlineto{\pgfqpoint{4.518260in}{2.391749in}}%
\pgfpathlineto{\pgfqpoint{4.504182in}{2.383299in}}%
\pgfpathlineto{\pgfqpoint{4.490117in}{2.375035in}}%
\pgfpathlineto{\pgfqpoint{4.476067in}{2.366956in}}%
\pgfpathlineto{\pgfqpoint{4.468193in}{2.356459in}}%
\pgfpathlineto{\pgfqpoint{4.460312in}{2.345866in}}%
\pgfpathlineto{\pgfqpoint{4.452426in}{2.335176in}}%
\pgfpathlineto{\pgfqpoint{4.444534in}{2.324388in}}%
\pgfpathclose%
\pgfusepath{fill}%
\end{pgfscope}%
\begin{pgfscope}%
\pgfpathrectangle{\pgfqpoint{1.150000in}{0.150000in}}{\pgfqpoint{5.700000in}{5.700000in}}%
\pgfusepath{clip}%
\pgfsetbuttcap%
\pgfsetroundjoin%
\definecolor{currentfill}{rgb}{0.122606,0.585371,0.546557}%
\pgfsetfillcolor{currentfill}%
\pgfsetfillopacity{0.800000}%
\pgfsetlinewidth{0.000000pt}%
\definecolor{currentstroke}{rgb}{0.000000,0.000000,0.000000}%
\pgfsetstrokecolor{currentstroke}%
\pgfsetdash{}{0pt}%
\pgfpathmoveto{\pgfqpoint{5.329003in}{3.088198in}}%
\pgfpathlineto{\pgfqpoint{5.343565in}{3.100873in}}%
\pgfpathlineto{\pgfqpoint{5.358147in}{3.113729in}}%
\pgfpathlineto{\pgfqpoint{5.372749in}{3.126768in}}%
\pgfpathlineto{\pgfqpoint{5.387371in}{3.139989in}}%
\pgfpathlineto{\pgfqpoint{5.394819in}{3.144138in}}%
\pgfpathlineto{\pgfqpoint{5.402258in}{3.148216in}}%
\pgfpathlineto{\pgfqpoint{5.409690in}{3.152225in}}%
\pgfpathlineto{\pgfqpoint{5.417113in}{3.156171in}}%
\pgfpathlineto{\pgfqpoint{5.402510in}{3.143341in}}%
\pgfpathlineto{\pgfqpoint{5.387927in}{3.130692in}}%
\pgfpathlineto{\pgfqpoint{5.373364in}{3.118224in}}%
\pgfpathlineto{\pgfqpoint{5.358820in}{3.105938in}}%
\pgfpathlineto{\pgfqpoint{5.351378in}{3.101590in}}%
\pgfpathlineto{\pgfqpoint{5.343927in}{3.097188in}}%
\pgfpathlineto{\pgfqpoint{5.336469in}{3.092725in}}%
\pgfpathlineto{\pgfqpoint{5.329003in}{3.088198in}}%
\pgfpathclose%
\pgfusepath{fill}%
\end{pgfscope}%
\begin{pgfscope}%
\pgfpathrectangle{\pgfqpoint{1.150000in}{0.150000in}}{\pgfqpoint{5.700000in}{5.700000in}}%
\pgfusepath{clip}%
\pgfsetbuttcap%
\pgfsetroundjoin%
\definecolor{currentfill}{rgb}{0.220124,0.725509,0.466226}%
\pgfsetfillcolor{currentfill}%
\pgfsetfillopacity{0.800000}%
\pgfsetlinewidth{0.000000pt}%
\definecolor{currentstroke}{rgb}{0.000000,0.000000,0.000000}%
\pgfsetstrokecolor{currentstroke}%
\pgfsetdash{}{0pt}%
\pgfpathmoveto{\pgfqpoint{5.974404in}{3.535408in}}%
\pgfpathlineto{\pgfqpoint{5.989341in}{3.549034in}}%
\pgfpathlineto{\pgfqpoint{6.004300in}{3.562839in}}%
\pgfpathlineto{\pgfqpoint{6.019281in}{3.576823in}}%
\pgfpathlineto{\pgfqpoint{6.034285in}{3.590987in}}%
\pgfpathlineto{\pgfqpoint{6.041301in}{3.590895in}}%
\pgfpathlineto{\pgfqpoint{6.048311in}{3.590859in}}%
\pgfpathlineto{\pgfqpoint{6.055316in}{3.590885in}}%
\pgfpathlineto{\pgfqpoint{6.062317in}{3.590981in}}%
\pgfpathlineto{\pgfqpoint{6.047351in}{3.577488in}}%
\pgfpathlineto{\pgfqpoint{6.032408in}{3.564172in}}%
\pgfpathlineto{\pgfqpoint{6.017487in}{3.551033in}}%
\pgfpathlineto{\pgfqpoint{6.002588in}{3.538072in}}%
\pgfpathlineto{\pgfqpoint{5.995549in}{3.537297in}}%
\pgfpathlineto{\pgfqpoint{5.988505in}{3.536600in}}%
\pgfpathlineto{\pgfqpoint{5.981457in}{3.535972in}}%
\pgfpathlineto{\pgfqpoint{5.974404in}{3.535408in}}%
\pgfpathclose%
\pgfusepath{fill}%
\end{pgfscope}%
\begin{pgfscope}%
\pgfpathrectangle{\pgfqpoint{1.150000in}{0.150000in}}{\pgfqpoint{5.700000in}{5.700000in}}%
\pgfusepath{clip}%
\pgfsetbuttcap%
\pgfsetroundjoin%
\definecolor{currentfill}{rgb}{0.282290,0.145912,0.461510}%
\pgfsetfillcolor{currentfill}%
\pgfsetfillopacity{0.800000}%
\pgfsetlinewidth{0.000000pt}%
\definecolor{currentstroke}{rgb}{0.000000,0.000000,0.000000}%
\pgfsetstrokecolor{currentstroke}%
\pgfsetdash{}{0pt}%
\pgfpathmoveto{\pgfqpoint{3.942546in}{1.883803in}}%
\pgfpathlineto{\pgfqpoint{3.956384in}{1.887067in}}%
\pgfpathlineto{\pgfqpoint{3.970233in}{1.890520in}}%
\pgfpathlineto{\pgfqpoint{3.984092in}{1.894161in}}%
\pgfpathlineto{\pgfqpoint{3.997960in}{1.897990in}}%
\pgfpathlineto{\pgfqpoint{4.006009in}{1.910200in}}%
\pgfpathlineto{\pgfqpoint{4.014053in}{1.922372in}}%
\pgfpathlineto{\pgfqpoint{4.022092in}{1.934506in}}%
\pgfpathlineto{\pgfqpoint{4.030127in}{1.946599in}}%
\pgfpathlineto{\pgfqpoint{4.016263in}{1.942522in}}%
\pgfpathlineto{\pgfqpoint{4.002409in}{1.938634in}}%
\pgfpathlineto{\pgfqpoint{3.988566in}{1.934934in}}%
\pgfpathlineto{\pgfqpoint{3.974732in}{1.931423in}}%
\pgfpathlineto{\pgfqpoint{3.966693in}{1.919565in}}%
\pgfpathlineto{\pgfqpoint{3.958649in}{1.907674in}}%
\pgfpathlineto{\pgfqpoint{3.950600in}{1.895753in}}%
\pgfpathlineto{\pgfqpoint{3.942546in}{1.883803in}}%
\pgfpathclose%
\pgfusepath{fill}%
\end{pgfscope}%
\begin{pgfscope}%
\pgfpathrectangle{\pgfqpoint{1.150000in}{0.150000in}}{\pgfqpoint{5.700000in}{5.700000in}}%
\pgfusepath{clip}%
\pgfsetbuttcap%
\pgfsetroundjoin%
\definecolor{currentfill}{rgb}{0.258965,0.251537,0.524736}%
\pgfsetfillcolor{currentfill}%
\pgfsetfillopacity{0.800000}%
\pgfsetlinewidth{0.000000pt}%
\definecolor{currentstroke}{rgb}{0.000000,0.000000,0.000000}%
\pgfsetstrokecolor{currentstroke}%
\pgfsetdash{}{0pt}%
\pgfpathmoveto{\pgfqpoint{4.237382in}{2.131680in}}%
\pgfpathlineto{\pgfqpoint{4.251339in}{2.137992in}}%
\pgfpathlineto{\pgfqpoint{4.265308in}{2.144491in}}%
\pgfpathlineto{\pgfqpoint{4.279290in}{2.151175in}}%
\pgfpathlineto{\pgfqpoint{4.293284in}{2.158046in}}%
\pgfpathlineto{\pgfqpoint{4.301245in}{2.169855in}}%
\pgfpathlineto{\pgfqpoint{4.309200in}{2.181579in}}%
\pgfpathlineto{\pgfqpoint{4.317150in}{2.193217in}}%
\pgfpathlineto{\pgfqpoint{4.325094in}{2.204769in}}%
\pgfpathlineto{\pgfqpoint{4.311103in}{2.197779in}}%
\pgfpathlineto{\pgfqpoint{4.297124in}{2.190975in}}%
\pgfpathlineto{\pgfqpoint{4.283158in}{2.184358in}}%
\pgfpathlineto{\pgfqpoint{4.269204in}{2.177926in}}%
\pgfpathlineto{\pgfqpoint{4.261257in}{2.166481in}}%
\pgfpathlineto{\pgfqpoint{4.253304in}{2.154958in}}%
\pgfpathlineto{\pgfqpoint{4.245345in}{2.143357in}}%
\pgfpathlineto{\pgfqpoint{4.237382in}{2.131680in}}%
\pgfpathclose%
\pgfusepath{fill}%
\end{pgfscope}%
\begin{pgfscope}%
\pgfpathrectangle{\pgfqpoint{1.150000in}{0.150000in}}{\pgfqpoint{5.700000in}{5.700000in}}%
\pgfusepath{clip}%
\pgfsetbuttcap%
\pgfsetroundjoin%
\definecolor{currentfill}{rgb}{0.269944,0.014625,0.341379}%
\pgfsetfillcolor{currentfill}%
\pgfsetfillopacity{0.800000}%
\pgfsetlinewidth{0.000000pt}%
\definecolor{currentstroke}{rgb}{0.000000,0.000000,0.000000}%
\pgfsetstrokecolor{currentstroke}%
\pgfsetdash{}{0pt}%
\pgfpathmoveto{\pgfqpoint{3.416667in}{1.626175in}}%
\pgfpathlineto{\pgfqpoint{3.430401in}{1.622641in}}%
\pgfpathlineto{\pgfqpoint{3.444139in}{1.619308in}}%
\pgfpathlineto{\pgfqpoint{3.457882in}{1.616175in}}%
\pgfpathlineto{\pgfqpoint{3.471629in}{1.613240in}}%
\pgfpathlineto{\pgfqpoint{3.479866in}{1.622781in}}%
\pgfpathlineto{\pgfqpoint{3.488096in}{1.632409in}}%
\pgfpathlineto{\pgfqpoint{3.496320in}{1.642120in}}%
\pgfpathlineto{\pgfqpoint{3.504537in}{1.651910in}}%
\pgfpathlineto{\pgfqpoint{3.490806in}{1.654409in}}%
\pgfpathlineto{\pgfqpoint{3.477079in}{1.657108in}}%
\pgfpathlineto{\pgfqpoint{3.463357in}{1.660006in}}%
\pgfpathlineto{\pgfqpoint{3.449640in}{1.663105in}}%
\pgfpathlineto{\pgfqpoint{3.441407in}{1.653739in}}%
\pgfpathlineto{\pgfqpoint{3.433168in}{1.644458in}}%
\pgfpathlineto{\pgfqpoint{3.424921in}{1.635269in}}%
\pgfpathlineto{\pgfqpoint{3.416667in}{1.626175in}}%
\pgfpathclose%
\pgfusepath{fill}%
\end{pgfscope}%
\begin{pgfscope}%
\pgfpathrectangle{\pgfqpoint{1.150000in}{0.150000in}}{\pgfqpoint{5.700000in}{5.700000in}}%
\pgfusepath{clip}%
\pgfsetbuttcap%
\pgfsetroundjoin%
\definecolor{currentfill}{rgb}{0.183898,0.422383,0.556944}%
\pgfsetfillcolor{currentfill}%
\pgfsetfillopacity{0.800000}%
\pgfsetlinewidth{0.000000pt}%
\definecolor{currentstroke}{rgb}{0.000000,0.000000,0.000000}%
\pgfsetstrokecolor{currentstroke}%
\pgfsetdash{}{0pt}%
\pgfpathmoveto{\pgfqpoint{4.739503in}{2.593093in}}%
\pgfpathlineto{\pgfqpoint{4.753722in}{2.603249in}}%
\pgfpathlineto{\pgfqpoint{4.767957in}{2.613589in}}%
\pgfpathlineto{\pgfqpoint{4.782208in}{2.624113in}}%
\pgfpathlineto{\pgfqpoint{4.796476in}{2.634822in}}%
\pgfpathlineto{\pgfqpoint{4.804248in}{2.643726in}}%
\pgfpathlineto{\pgfqpoint{4.812013in}{2.652513in}}%
\pgfpathlineto{\pgfqpoint{4.819770in}{2.661185in}}%
\pgfpathlineto{\pgfqpoint{4.827521in}{2.669743in}}%
\pgfpathlineto{\pgfqpoint{4.813259in}{2.659149in}}%
\pgfpathlineto{\pgfqpoint{4.799014in}{2.648739in}}%
\pgfpathlineto{\pgfqpoint{4.784786in}{2.638512in}}%
\pgfpathlineto{\pgfqpoint{4.770574in}{2.628469in}}%
\pgfpathlineto{\pgfqpoint{4.762816in}{2.619785in}}%
\pgfpathlineto{\pgfqpoint{4.755052in}{2.610995in}}%
\pgfpathlineto{\pgfqpoint{4.747281in}{2.602098in}}%
\pgfpathlineto{\pgfqpoint{4.739503in}{2.593093in}}%
\pgfpathclose%
\pgfusepath{fill}%
\end{pgfscope}%
\begin{pgfscope}%
\pgfpathrectangle{\pgfqpoint{1.150000in}{0.150000in}}{\pgfqpoint{5.700000in}{5.700000in}}%
\pgfusepath{clip}%
\pgfsetbuttcap%
\pgfsetroundjoin%
\definecolor{currentfill}{rgb}{0.149039,0.508051,0.557250}%
\pgfsetfillcolor{currentfill}%
\pgfsetfillopacity{0.800000}%
\pgfsetlinewidth{0.000000pt}%
\definecolor{currentstroke}{rgb}{0.000000,0.000000,0.000000}%
\pgfsetstrokecolor{currentstroke}%
\pgfsetdash{}{0pt}%
\pgfpathmoveto{\pgfqpoint{5.034469in}{2.851316in}}%
\pgfpathlineto{\pgfqpoint{5.048861in}{2.863015in}}%
\pgfpathlineto{\pgfqpoint{5.063272in}{2.874896in}}%
\pgfpathlineto{\pgfqpoint{5.077700in}{2.886961in}}%
\pgfpathlineto{\pgfqpoint{5.092148in}{2.899208in}}%
\pgfpathlineto{\pgfqpoint{5.099772in}{2.905771in}}%
\pgfpathlineto{\pgfqpoint{5.107389in}{2.912228in}}%
\pgfpathlineto{\pgfqpoint{5.114997in}{2.918582in}}%
\pgfpathlineto{\pgfqpoint{5.122598in}{2.924836in}}%
\pgfpathlineto{\pgfqpoint{5.108163in}{2.912840in}}%
\pgfpathlineto{\pgfqpoint{5.093746in}{2.901027in}}%
\pgfpathlineto{\pgfqpoint{5.079347in}{2.889397in}}%
\pgfpathlineto{\pgfqpoint{5.064966in}{2.877948in}}%
\pgfpathlineto{\pgfqpoint{5.057353in}{2.871431in}}%
\pgfpathlineto{\pgfqpoint{5.049733in}{2.864822in}}%
\pgfpathlineto{\pgfqpoint{5.042105in}{2.858118in}}%
\pgfpathlineto{\pgfqpoint{5.034469in}{2.851316in}}%
\pgfpathclose%
\pgfusepath{fill}%
\end{pgfscope}%
\begin{pgfscope}%
\pgfpathrectangle{\pgfqpoint{1.150000in}{0.150000in}}{\pgfqpoint{5.700000in}{5.700000in}}%
\pgfusepath{clip}%
\pgfsetbuttcap%
\pgfsetroundjoin%
\definecolor{currentfill}{rgb}{0.197636,0.391528,0.554969}%
\pgfsetfillcolor{currentfill}%
\pgfsetfillopacity{0.800000}%
\pgfsetlinewidth{0.000000pt}%
\definecolor{currentstroke}{rgb}{0.000000,0.000000,0.000000}%
\pgfsetstrokecolor{currentstroke}%
\pgfsetdash{}{0pt}%
\pgfpathmoveto{\pgfqpoint{2.262276in}{2.585701in}}%
\pgfpathlineto{\pgfqpoint{2.276464in}{2.561436in}}%
\pgfpathlineto{\pgfqpoint{2.290638in}{2.537506in}}%
\pgfpathlineto{\pgfqpoint{2.304796in}{2.513908in}}%
\pgfpathlineto{\pgfqpoint{2.318939in}{2.490637in}}%
\pgfpathlineto{\pgfqpoint{2.327968in}{2.488181in}}%
\pgfpathlineto{\pgfqpoint{2.336976in}{2.486031in}}%
\pgfpathlineto{\pgfqpoint{2.345964in}{2.484182in}}%
\pgfpathlineto{\pgfqpoint{2.354933in}{2.482627in}}%
\pgfpathlineto{\pgfqpoint{2.340844in}{2.505320in}}%
\pgfpathlineto{\pgfqpoint{2.326740in}{2.528341in}}%
\pgfpathlineto{\pgfqpoint{2.312622in}{2.551691in}}%
\pgfpathlineto{\pgfqpoint{2.298489in}{2.575374in}}%
\pgfpathlineto{\pgfqpoint{2.289467in}{2.577493in}}%
\pgfpathlineto{\pgfqpoint{2.280424in}{2.579917in}}%
\pgfpathlineto{\pgfqpoint{2.271360in}{2.582651in}}%
\pgfpathlineto{\pgfqpoint{2.262276in}{2.585701in}}%
\pgfpathclose%
\pgfusepath{fill}%
\end{pgfscope}%
\begin{pgfscope}%
\pgfpathrectangle{\pgfqpoint{1.150000in}{0.150000in}}{\pgfqpoint{5.700000in}{5.700000in}}%
\pgfusepath{clip}%
\pgfsetbuttcap%
\pgfsetroundjoin%
\definecolor{currentfill}{rgb}{0.276022,0.044167,0.370164}%
\pgfsetfillcolor{currentfill}%
\pgfsetfillopacity{0.800000}%
\pgfsetlinewidth{0.000000pt}%
\definecolor{currentstroke}{rgb}{0.000000,0.000000,0.000000}%
\pgfsetstrokecolor{currentstroke}%
\pgfsetdash{}{0pt}%
\pgfpathmoveto{\pgfqpoint{2.986420in}{1.709275in}}%
\pgfpathlineto{\pgfqpoint{3.000196in}{1.699154in}}%
\pgfpathlineto{\pgfqpoint{3.013970in}{1.689258in}}%
\pgfpathlineto{\pgfqpoint{3.027742in}{1.679585in}}%
\pgfpathlineto{\pgfqpoint{3.041514in}{1.670134in}}%
\pgfpathlineto{\pgfqpoint{3.049995in}{1.675012in}}%
\pgfpathlineto{\pgfqpoint{3.058465in}{1.680083in}}%
\pgfpathlineto{\pgfqpoint{3.066924in}{1.685340in}}%
\pgfpathlineto{\pgfqpoint{3.075371in}{1.690779in}}%
\pgfpathlineto{\pgfqpoint{3.061630in}{1.699695in}}%
\pgfpathlineto{\pgfqpoint{3.047887in}{1.708833in}}%
\pgfpathlineto{\pgfqpoint{3.034144in}{1.718193in}}%
\pgfpathlineto{\pgfqpoint{3.020399in}{1.727778in}}%
\pgfpathlineto{\pgfqpoint{3.011922in}{1.722862in}}%
\pgfpathlineto{\pgfqpoint{3.003433in}{1.718136in}}%
\pgfpathlineto{\pgfqpoint{2.994933in}{1.713605in}}%
\pgfpathlineto{\pgfqpoint{2.986420in}{1.709275in}}%
\pgfpathclose%
\pgfusepath{fill}%
\end{pgfscope}%
\begin{pgfscope}%
\pgfpathrectangle{\pgfqpoint{1.150000in}{0.150000in}}{\pgfqpoint{5.700000in}{5.700000in}}%
\pgfusepath{clip}%
\pgfsetbuttcap%
\pgfsetroundjoin%
\definecolor{currentfill}{rgb}{0.119512,0.607464,0.540218}%
\pgfsetfillcolor{currentfill}%
\pgfsetfillopacity{0.800000}%
\pgfsetlinewidth{0.000000pt}%
\definecolor{currentstroke}{rgb}{0.000000,0.000000,0.000000}%
\pgfsetstrokecolor{currentstroke}%
\pgfsetdash{}{0pt}%
\pgfpathmoveto{\pgfqpoint{5.417113in}{3.156171in}}%
\pgfpathlineto{\pgfqpoint{5.431736in}{3.169182in}}%
\pgfpathlineto{\pgfqpoint{5.446379in}{3.182375in}}%
\pgfpathlineto{\pgfqpoint{5.461042in}{3.195750in}}%
\pgfpathlineto{\pgfqpoint{5.475726in}{3.209306in}}%
\pgfpathlineto{\pgfqpoint{5.483121in}{3.212780in}}%
\pgfpathlineto{\pgfqpoint{5.490507in}{3.216191in}}%
\pgfpathlineto{\pgfqpoint{5.497886in}{3.219545in}}%
\pgfpathlineto{\pgfqpoint{5.505256in}{3.222846in}}%
\pgfpathlineto{\pgfqpoint{5.490593in}{3.209716in}}%
\pgfpathlineto{\pgfqpoint{5.475951in}{3.196766in}}%
\pgfpathlineto{\pgfqpoint{5.461329in}{3.183998in}}%
\pgfpathlineto{\pgfqpoint{5.446727in}{3.171411in}}%
\pgfpathlineto{\pgfqpoint{5.439335in}{3.167673in}}%
\pgfpathlineto{\pgfqpoint{5.431935in}{3.163890in}}%
\pgfpathlineto{\pgfqpoint{5.424528in}{3.160057in}}%
\pgfpathlineto{\pgfqpoint{5.417113in}{3.156171in}}%
\pgfpathclose%
\pgfusepath{fill}%
\end{pgfscope}%
\begin{pgfscope}%
\pgfpathrectangle{\pgfqpoint{1.150000in}{0.150000in}}{\pgfqpoint{5.700000in}{5.700000in}}%
\pgfusepath{clip}%
\pgfsetbuttcap%
\pgfsetroundjoin%
\definecolor{currentfill}{rgb}{0.278826,0.175490,0.483397}%
\pgfsetfillcolor{currentfill}%
\pgfsetfillopacity{0.800000}%
\pgfsetlinewidth{0.000000pt}%
\definecolor{currentstroke}{rgb}{0.000000,0.000000,0.000000}%
\pgfsetstrokecolor{currentstroke}%
\pgfsetdash{}{0pt}%
\pgfpathmoveto{\pgfqpoint{4.030127in}{1.946599in}}%
\pgfpathlineto{\pgfqpoint{4.044002in}{1.950864in}}%
\pgfpathlineto{\pgfqpoint{4.057887in}{1.955317in}}%
\pgfpathlineto{\pgfqpoint{4.071783in}{1.959957in}}%
\pgfpathlineto{\pgfqpoint{4.085691in}{1.964784in}}%
\pgfpathlineto{\pgfqpoint{4.093716in}{1.977063in}}%
\pgfpathlineto{\pgfqpoint{4.101737in}{1.989288in}}%
\pgfpathlineto{\pgfqpoint{4.109753in}{2.001460in}}%
\pgfpathlineto{\pgfqpoint{4.117765in}{2.013575in}}%
\pgfpathlineto{\pgfqpoint{4.103861in}{2.008532in}}%
\pgfpathlineto{\pgfqpoint{4.089969in}{2.003676in}}%
\pgfpathlineto{\pgfqpoint{4.076087in}{1.999007in}}%
\pgfpathlineto{\pgfqpoint{4.062217in}{1.994527in}}%
\pgfpathlineto{\pgfqpoint{4.054202in}{1.982616in}}%
\pgfpathlineto{\pgfqpoint{4.046181in}{1.970656in}}%
\pgfpathlineto{\pgfqpoint{4.038157in}{1.958650in}}%
\pgfpathlineto{\pgfqpoint{4.030127in}{1.946599in}}%
\pgfpathclose%
\pgfusepath{fill}%
\end{pgfscope}%
\begin{pgfscope}%
\pgfpathrectangle{\pgfqpoint{1.150000in}{0.150000in}}{\pgfqpoint{5.700000in}{5.700000in}}%
\pgfusepath{clip}%
\pgfsetbuttcap%
\pgfsetroundjoin%
\definecolor{currentfill}{rgb}{0.212395,0.359683,0.551710}%
\pgfsetfillcolor{currentfill}%
\pgfsetfillopacity{0.800000}%
\pgfsetlinewidth{0.000000pt}%
\definecolor{currentstroke}{rgb}{0.000000,0.000000,0.000000}%
\pgfsetstrokecolor{currentstroke}%
\pgfsetdash{}{0pt}%
\pgfpathmoveto{\pgfqpoint{4.532354in}{2.400383in}}%
\pgfpathlineto{\pgfqpoint{4.546462in}{2.409202in}}%
\pgfpathlineto{\pgfqpoint{4.560585in}{2.418206in}}%
\pgfpathlineto{\pgfqpoint{4.574723in}{2.427394in}}%
\pgfpathlineto{\pgfqpoint{4.588876in}{2.436767in}}%
\pgfpathlineto{\pgfqpoint{4.596737in}{2.447168in}}%
\pgfpathlineto{\pgfqpoint{4.604592in}{2.457455in}}%
\pgfpathlineto{\pgfqpoint{4.612441in}{2.467631in}}%
\pgfpathlineto{\pgfqpoint{4.620283in}{2.477696in}}%
\pgfpathlineto{\pgfqpoint{4.606134in}{2.468335in}}%
\pgfpathlineto{\pgfqpoint{4.591999in}{2.459159in}}%
\pgfpathlineto{\pgfqpoint{4.577880in}{2.450168in}}%
\pgfpathlineto{\pgfqpoint{4.563776in}{2.441361in}}%
\pgfpathlineto{\pgfqpoint{4.555930in}{2.431272in}}%
\pgfpathlineto{\pgfqpoint{4.548077in}{2.421079in}}%
\pgfpathlineto{\pgfqpoint{4.540218in}{2.410783in}}%
\pgfpathlineto{\pgfqpoint{4.532354in}{2.400383in}}%
\pgfpathclose%
\pgfusepath{fill}%
\end{pgfscope}%
\begin{pgfscope}%
\pgfpathrectangle{\pgfqpoint{1.150000in}{0.150000in}}{\pgfqpoint{5.700000in}{5.700000in}}%
\pgfusepath{clip}%
\pgfsetbuttcap%
\pgfsetroundjoin%
\definecolor{currentfill}{rgb}{0.268510,0.009605,0.335427}%
\pgfsetfillcolor{currentfill}%
\pgfsetfillopacity{0.800000}%
\pgfsetlinewidth{0.000000pt}%
\definecolor{currentstroke}{rgb}{0.000000,0.000000,0.000000}%
\pgfsetstrokecolor{currentstroke}%
\pgfsetdash{}{0pt}%
\pgfpathmoveto{\pgfqpoint{3.185305in}{1.627296in}}%
\pgfpathlineto{\pgfqpoint{3.199050in}{1.620325in}}%
\pgfpathlineto{\pgfqpoint{3.212795in}{1.613565in}}%
\pgfpathlineto{\pgfqpoint{3.226543in}{1.607015in}}%
\pgfpathlineto{\pgfqpoint{3.240291in}{1.600674in}}%
\pgfpathlineto{\pgfqpoint{3.248651in}{1.607828in}}%
\pgfpathlineto{\pgfqpoint{3.257002in}{1.615128in}}%
\pgfpathlineto{\pgfqpoint{3.265343in}{1.622571in}}%
\pgfpathlineto{\pgfqpoint{3.273676in}{1.630152in}}%
\pgfpathlineto{\pgfqpoint{3.259950in}{1.635994in}}%
\pgfpathlineto{\pgfqpoint{3.246227in}{1.642044in}}%
\pgfpathlineto{\pgfqpoint{3.232505in}{1.648304in}}%
\pgfpathlineto{\pgfqpoint{3.218784in}{1.654775in}}%
\pgfpathlineto{\pgfqpoint{3.210428in}{1.647682in}}%
\pgfpathlineto{\pgfqpoint{3.202063in}{1.640735in}}%
\pgfpathlineto{\pgfqpoint{3.193689in}{1.633938in}}%
\pgfpathlineto{\pgfqpoint{3.185305in}{1.627296in}}%
\pgfpathclose%
\pgfusepath{fill}%
\end{pgfscope}%
\begin{pgfscope}%
\pgfpathrectangle{\pgfqpoint{1.150000in}{0.150000in}}{\pgfqpoint{5.700000in}{5.700000in}}%
\pgfusepath{clip}%
\pgfsetbuttcap%
\pgfsetroundjoin%
\definecolor{currentfill}{rgb}{0.244972,0.287675,0.537260}%
\pgfsetfillcolor{currentfill}%
\pgfsetfillopacity{0.800000}%
\pgfsetlinewidth{0.000000pt}%
\definecolor{currentstroke}{rgb}{0.000000,0.000000,0.000000}%
\pgfsetstrokecolor{currentstroke}%
\pgfsetdash{}{0pt}%
\pgfpathmoveto{\pgfqpoint{4.325094in}{2.204769in}}%
\pgfpathlineto{\pgfqpoint{4.339100in}{2.211945in}}%
\pgfpathlineto{\pgfqpoint{4.353118in}{2.219306in}}%
\pgfpathlineto{\pgfqpoint{4.367150in}{2.226853in}}%
\pgfpathlineto{\pgfqpoint{4.381195in}{2.234585in}}%
\pgfpathlineto{\pgfqpoint{4.389132in}{2.246149in}}%
\pgfpathlineto{\pgfqpoint{4.397063in}{2.257616in}}%
\pgfpathlineto{\pgfqpoint{4.404989in}{2.268988in}}%
\pgfpathlineto{\pgfqpoint{4.412909in}{2.280262in}}%
\pgfpathlineto{\pgfqpoint{4.398866in}{2.272443in}}%
\pgfpathlineto{\pgfqpoint{4.384837in}{2.264810in}}%
\pgfpathlineto{\pgfqpoint{4.370822in}{2.257362in}}%
\pgfpathlineto{\pgfqpoint{4.356820in}{2.250099in}}%
\pgfpathlineto{\pgfqpoint{4.348897in}{2.238899in}}%
\pgfpathlineto{\pgfqpoint{4.340968in}{2.227611in}}%
\pgfpathlineto{\pgfqpoint{4.333034in}{2.216234in}}%
\pgfpathlineto{\pgfqpoint{4.325094in}{2.204769in}}%
\pgfpathclose%
\pgfusepath{fill}%
\end{pgfscope}%
\begin{pgfscope}%
\pgfpathrectangle{\pgfqpoint{1.150000in}{0.150000in}}{\pgfqpoint{5.700000in}{5.700000in}}%
\pgfusepath{clip}%
\pgfsetbuttcap%
\pgfsetroundjoin%
\definecolor{currentfill}{rgb}{0.268510,0.009605,0.335427}%
\pgfsetfillcolor{currentfill}%
\pgfsetfillopacity{0.800000}%
\pgfsetlinewidth{0.000000pt}%
\definecolor{currentstroke}{rgb}{0.000000,0.000000,0.000000}%
\pgfsetstrokecolor{currentstroke}%
\pgfsetdash{}{0pt}%
\pgfpathmoveto{\pgfqpoint{3.328603in}{1.608857in}}%
\pgfpathlineto{\pgfqpoint{3.342342in}{1.604046in}}%
\pgfpathlineto{\pgfqpoint{3.356084in}{1.599439in}}%
\pgfpathlineto{\pgfqpoint{3.369829in}{1.595035in}}%
\pgfpathlineto{\pgfqpoint{3.383577in}{1.590834in}}%
\pgfpathlineto{\pgfqpoint{3.391861in}{1.599505in}}%
\pgfpathlineto{\pgfqpoint{3.400137in}{1.608288in}}%
\pgfpathlineto{\pgfqpoint{3.408406in}{1.617180in}}%
\pgfpathlineto{\pgfqpoint{3.416667in}{1.626175in}}%
\pgfpathlineto{\pgfqpoint{3.402937in}{1.629910in}}%
\pgfpathlineto{\pgfqpoint{3.389211in}{1.633847in}}%
\pgfpathlineto{\pgfqpoint{3.375488in}{1.637987in}}%
\pgfpathlineto{\pgfqpoint{3.361769in}{1.642331in}}%
\pgfpathlineto{\pgfqpoint{3.353489in}{1.633791in}}%
\pgfpathlineto{\pgfqpoint{3.345202in}{1.625362in}}%
\pgfpathlineto{\pgfqpoint{3.336907in}{1.617049in}}%
\pgfpathlineto{\pgfqpoint{3.328603in}{1.608857in}}%
\pgfpathclose%
\pgfusepath{fill}%
\end{pgfscope}%
\begin{pgfscope}%
\pgfpathrectangle{\pgfqpoint{1.150000in}{0.150000in}}{\pgfqpoint{5.700000in}{5.700000in}}%
\pgfusepath{clip}%
\pgfsetbuttcap%
\pgfsetroundjoin%
\definecolor{currentfill}{rgb}{0.277134,0.185228,0.489898}%
\pgfsetfillcolor{currentfill}%
\pgfsetfillopacity{0.800000}%
\pgfsetlinewidth{0.000000pt}%
\definecolor{currentstroke}{rgb}{0.000000,0.000000,0.000000}%
\pgfsetstrokecolor{currentstroke}%
\pgfsetdash{}{0pt}%
\pgfpathmoveto{\pgfqpoint{2.619691in}{2.021862in}}%
\pgfpathlineto{\pgfqpoint{2.633613in}{2.005325in}}%
\pgfpathlineto{\pgfqpoint{2.647527in}{1.989051in}}%
\pgfpathlineto{\pgfqpoint{2.661434in}{1.973039in}}%
\pgfpathlineto{\pgfqpoint{2.675334in}{1.957286in}}%
\pgfpathlineto{\pgfqpoint{2.684092in}{1.957819in}}%
\pgfpathlineto{\pgfqpoint{2.692833in}{1.958617in}}%
\pgfpathlineto{\pgfqpoint{2.701558in}{1.959675in}}%
\pgfpathlineto{\pgfqpoint{2.710268in}{1.960987in}}%
\pgfpathlineto{\pgfqpoint{2.696411in}{1.976157in}}%
\pgfpathlineto{\pgfqpoint{2.682547in}{1.991586in}}%
\pgfpathlineto{\pgfqpoint{2.668677in}{2.007275in}}%
\pgfpathlineto{\pgfqpoint{2.654800in}{2.023227in}}%
\pgfpathlineto{\pgfqpoint{2.646047in}{2.022486in}}%
\pgfpathlineto{\pgfqpoint{2.637279in}{2.022007in}}%
\pgfpathlineto{\pgfqpoint{2.628494in}{2.021797in}}%
\pgfpathlineto{\pgfqpoint{2.619691in}{2.021862in}}%
\pgfpathclose%
\pgfusepath{fill}%
\end{pgfscope}%
\begin{pgfscope}%
\pgfpathrectangle{\pgfqpoint{1.150000in}{0.150000in}}{\pgfqpoint{5.700000in}{5.700000in}}%
\pgfusepath{clip}%
\pgfsetbuttcap%
\pgfsetroundjoin%
\definecolor{currentfill}{rgb}{0.271828,0.209303,0.504434}%
\pgfsetfillcolor{currentfill}%
\pgfsetfillopacity{0.800000}%
\pgfsetlinewidth{0.000000pt}%
\definecolor{currentstroke}{rgb}{0.000000,0.000000,0.000000}%
\pgfsetstrokecolor{currentstroke}%
\pgfsetdash{}{0pt}%
\pgfpathmoveto{\pgfqpoint{2.563925in}{2.090688in}}%
\pgfpathlineto{\pgfqpoint{2.577879in}{2.073076in}}%
\pgfpathlineto{\pgfqpoint{2.591825in}{2.055735in}}%
\pgfpathlineto{\pgfqpoint{2.605762in}{2.038665in}}%
\pgfpathlineto{\pgfqpoint{2.619691in}{2.021862in}}%
\pgfpathlineto{\pgfqpoint{2.628494in}{2.021797in}}%
\pgfpathlineto{\pgfqpoint{2.637279in}{2.022007in}}%
\pgfpathlineto{\pgfqpoint{2.646047in}{2.022486in}}%
\pgfpathlineto{\pgfqpoint{2.654800in}{2.023227in}}%
\pgfpathlineto{\pgfqpoint{2.640915in}{2.039444in}}%
\pgfpathlineto{\pgfqpoint{2.627023in}{2.055928in}}%
\pgfpathlineto{\pgfqpoint{2.613123in}{2.072680in}}%
\pgfpathlineto{\pgfqpoint{2.599216in}{2.089703in}}%
\pgfpathlineto{\pgfqpoint{2.590419in}{2.089535in}}%
\pgfpathlineto{\pgfqpoint{2.581605in}{2.089640in}}%
\pgfpathlineto{\pgfqpoint{2.572774in}{2.090022in}}%
\pgfpathlineto{\pgfqpoint{2.563925in}{2.090688in}}%
\pgfpathclose%
\pgfusepath{fill}%
\end{pgfscope}%
\begin{pgfscope}%
\pgfpathrectangle{\pgfqpoint{1.150000in}{0.150000in}}{\pgfqpoint{5.700000in}{5.700000in}}%
\pgfusepath{clip}%
\pgfsetbuttcap%
\pgfsetroundjoin%
\definecolor{currentfill}{rgb}{0.121380,0.629492,0.531973}%
\pgfsetfillcolor{currentfill}%
\pgfsetfillopacity{0.800000}%
\pgfsetlinewidth{0.000000pt}%
\definecolor{currentstroke}{rgb}{0.000000,0.000000,0.000000}%
\pgfsetstrokecolor{currentstroke}%
\pgfsetdash{}{0pt}%
\pgfpathmoveto{\pgfqpoint{5.505256in}{3.222846in}}%
\pgfpathlineto{\pgfqpoint{5.519939in}{3.236157in}}%
\pgfpathlineto{\pgfqpoint{5.534643in}{3.249650in}}%
\pgfpathlineto{\pgfqpoint{5.549367in}{3.263324in}}%
\pgfpathlineto{\pgfqpoint{5.564113in}{3.277180in}}%
\pgfpathlineto{\pgfqpoint{5.571452in}{3.279984in}}%
\pgfpathlineto{\pgfqpoint{5.578784in}{3.282737in}}%
\pgfpathlineto{\pgfqpoint{5.586107in}{3.285444in}}%
\pgfpathlineto{\pgfqpoint{5.593423in}{3.288111in}}%
\pgfpathlineto{\pgfqpoint{5.578701in}{3.274717in}}%
\pgfpathlineto{\pgfqpoint{5.564000in}{3.261504in}}%
\pgfpathlineto{\pgfqpoint{5.549319in}{3.248472in}}%
\pgfpathlineto{\pgfqpoint{5.534659in}{3.235619in}}%
\pgfpathlineto{\pgfqpoint{5.527320in}{3.232480in}}%
\pgfpathlineto{\pgfqpoint{5.519973in}{3.229308in}}%
\pgfpathlineto{\pgfqpoint{5.512618in}{3.226099in}}%
\pgfpathlineto{\pgfqpoint{5.505256in}{3.222846in}}%
\pgfpathclose%
\pgfusepath{fill}%
\end{pgfscope}%
\begin{pgfscope}%
\pgfpathrectangle{\pgfqpoint{1.150000in}{0.150000in}}{\pgfqpoint{5.700000in}{5.700000in}}%
\pgfusepath{clip}%
\pgfsetbuttcap%
\pgfsetroundjoin%
\definecolor{currentfill}{rgb}{0.281412,0.155834,0.469201}%
\pgfsetfillcolor{currentfill}%
\pgfsetfillopacity{0.800000}%
\pgfsetlinewidth{0.000000pt}%
\definecolor{currentstroke}{rgb}{0.000000,0.000000,0.000000}%
\pgfsetstrokecolor{currentstroke}%
\pgfsetdash{}{0pt}%
\pgfpathmoveto{\pgfqpoint{2.675334in}{1.957286in}}%
\pgfpathlineto{\pgfqpoint{2.689228in}{1.941791in}}%
\pgfpathlineto{\pgfqpoint{2.703115in}{1.926552in}}%
\pgfpathlineto{\pgfqpoint{2.716995in}{1.911567in}}%
\pgfpathlineto{\pgfqpoint{2.730870in}{1.896835in}}%
\pgfpathlineto{\pgfqpoint{2.739585in}{1.897961in}}%
\pgfpathlineto{\pgfqpoint{2.748284in}{1.899344in}}%
\pgfpathlineto{\pgfqpoint{2.756968in}{1.900978in}}%
\pgfpathlineto{\pgfqpoint{2.765637in}{1.902858in}}%
\pgfpathlineto{\pgfqpoint{2.751803in}{1.917011in}}%
\pgfpathlineto{\pgfqpoint{2.737964in}{1.931416in}}%
\pgfpathlineto{\pgfqpoint{2.724119in}{1.946074in}}%
\pgfpathlineto{\pgfqpoint{2.710268in}{1.960987in}}%
\pgfpathlineto{\pgfqpoint{2.701558in}{1.959675in}}%
\pgfpathlineto{\pgfqpoint{2.692833in}{1.958617in}}%
\pgfpathlineto{\pgfqpoint{2.684092in}{1.957819in}}%
\pgfpathlineto{\pgfqpoint{2.675334in}{1.957286in}}%
\pgfpathclose%
\pgfusepath{fill}%
\end{pgfscope}%
\begin{pgfscope}%
\pgfpathrectangle{\pgfqpoint{1.150000in}{0.150000in}}{\pgfqpoint{5.700000in}{5.700000in}}%
\pgfusepath{clip}%
\pgfsetbuttcap%
\pgfsetroundjoin%
\definecolor{currentfill}{rgb}{0.171176,0.452530,0.557965}%
\pgfsetfillcolor{currentfill}%
\pgfsetfillopacity{0.800000}%
\pgfsetlinewidth{0.000000pt}%
\definecolor{currentstroke}{rgb}{0.000000,0.000000,0.000000}%
\pgfsetstrokecolor{currentstroke}%
\pgfsetdash{}{0pt}%
\pgfpathmoveto{\pgfqpoint{4.827521in}{2.669743in}}%
\pgfpathlineto{\pgfqpoint{4.841799in}{2.680521in}}%
\pgfpathlineto{\pgfqpoint{4.856095in}{2.691483in}}%
\pgfpathlineto{\pgfqpoint{4.870407in}{2.702629in}}%
\pgfpathlineto{\pgfqpoint{4.884737in}{2.713959in}}%
\pgfpathlineto{\pgfqpoint{4.892474in}{2.722270in}}%
\pgfpathlineto{\pgfqpoint{4.900202in}{2.730462in}}%
\pgfpathlineto{\pgfqpoint{4.907924in}{2.738539in}}%
\pgfpathlineto{\pgfqpoint{4.915637in}{2.746501in}}%
\pgfpathlineto{\pgfqpoint{4.901315in}{2.735320in}}%
\pgfpathlineto{\pgfqpoint{4.887010in}{2.724322in}}%
\pgfpathlineto{\pgfqpoint{4.872722in}{2.713508in}}%
\pgfpathlineto{\pgfqpoint{4.858451in}{2.702878in}}%
\pgfpathlineto{\pgfqpoint{4.850729in}{2.694755in}}%
\pgfpathlineto{\pgfqpoint{4.843000in}{2.686527in}}%
\pgfpathlineto{\pgfqpoint{4.835264in}{2.678190in}}%
\pgfpathlineto{\pgfqpoint{4.827521in}{2.669743in}}%
\pgfpathclose%
\pgfusepath{fill}%
\end{pgfscope}%
\begin{pgfscope}%
\pgfpathrectangle{\pgfqpoint{1.150000in}{0.150000in}}{\pgfqpoint{5.700000in}{5.700000in}}%
\pgfusepath{clip}%
\pgfsetbuttcap%
\pgfsetroundjoin%
\definecolor{currentfill}{rgb}{0.137770,0.537492,0.554906}%
\pgfsetfillcolor{currentfill}%
\pgfsetfillopacity{0.800000}%
\pgfsetlinewidth{0.000000pt}%
\definecolor{currentstroke}{rgb}{0.000000,0.000000,0.000000}%
\pgfsetstrokecolor{currentstroke}%
\pgfsetdash{}{0pt}%
\pgfpathmoveto{\pgfqpoint{5.122598in}{2.924836in}}%
\pgfpathlineto{\pgfqpoint{5.137052in}{2.937015in}}%
\pgfpathlineto{\pgfqpoint{5.151525in}{2.949376in}}%
\pgfpathlineto{\pgfqpoint{5.166017in}{2.961921in}}%
\pgfpathlineto{\pgfqpoint{5.180528in}{2.974649in}}%
\pgfpathlineto{\pgfqpoint{5.188107in}{2.980532in}}%
\pgfpathlineto{\pgfqpoint{5.195679in}{2.986314in}}%
\pgfpathlineto{\pgfqpoint{5.203242in}{2.991997in}}%
\pgfpathlineto{\pgfqpoint{5.210797in}{2.997586in}}%
\pgfpathlineto{\pgfqpoint{5.196300in}{2.985145in}}%
\pgfpathlineto{\pgfqpoint{5.181821in}{2.972887in}}%
\pgfpathlineto{\pgfqpoint{5.167362in}{2.960812in}}%
\pgfpathlineto{\pgfqpoint{5.152921in}{2.948919in}}%
\pgfpathlineto{\pgfqpoint{5.145352in}{2.943031in}}%
\pgfpathlineto{\pgfqpoint{5.137775in}{2.937058in}}%
\pgfpathlineto{\pgfqpoint{5.130191in}{2.930994in}}%
\pgfpathlineto{\pgfqpoint{5.122598in}{2.924836in}}%
\pgfpathclose%
\pgfusepath{fill}%
\end{pgfscope}%
\begin{pgfscope}%
\pgfpathrectangle{\pgfqpoint{1.150000in}{0.150000in}}{\pgfqpoint{5.700000in}{5.700000in}}%
\pgfusepath{clip}%
\pgfsetbuttcap%
\pgfsetroundjoin%
\definecolor{currentfill}{rgb}{0.262138,0.242286,0.520837}%
\pgfsetfillcolor{currentfill}%
\pgfsetfillopacity{0.800000}%
\pgfsetlinewidth{0.000000pt}%
\definecolor{currentstroke}{rgb}{0.000000,0.000000,0.000000}%
\pgfsetstrokecolor{currentstroke}%
\pgfsetdash{}{0pt}%
\pgfpathmoveto{\pgfqpoint{2.508018in}{2.163897in}}%
\pgfpathlineto{\pgfqpoint{2.522009in}{2.145176in}}%
\pgfpathlineto{\pgfqpoint{2.535990in}{2.126735in}}%
\pgfpathlineto{\pgfqpoint{2.549962in}{2.108573in}}%
\pgfpathlineto{\pgfqpoint{2.563925in}{2.090688in}}%
\pgfpathlineto{\pgfqpoint{2.572774in}{2.090022in}}%
\pgfpathlineto{\pgfqpoint{2.581605in}{2.089640in}}%
\pgfpathlineto{\pgfqpoint{2.590419in}{2.089535in}}%
\pgfpathlineto{\pgfqpoint{2.599216in}{2.089703in}}%
\pgfpathlineto{\pgfqpoint{2.585300in}{2.106999in}}%
\pgfpathlineto{\pgfqpoint{2.571375in}{2.124570in}}%
\pgfpathlineto{\pgfqpoint{2.557441in}{2.142418in}}%
\pgfpathlineto{\pgfqpoint{2.543499in}{2.160546in}}%
\pgfpathlineto{\pgfqpoint{2.534656in}{2.160957in}}%
\pgfpathlineto{\pgfqpoint{2.525795in}{2.161648in}}%
\pgfpathlineto{\pgfqpoint{2.516916in}{2.162626in}}%
\pgfpathlineto{\pgfqpoint{2.508018in}{2.163897in}}%
\pgfpathclose%
\pgfusepath{fill}%
\end{pgfscope}%
\begin{pgfscope}%
\pgfpathrectangle{\pgfqpoint{1.150000in}{0.150000in}}{\pgfqpoint{5.700000in}{5.700000in}}%
\pgfusepath{clip}%
\pgfsetbuttcap%
\pgfsetroundjoin%
\definecolor{currentfill}{rgb}{0.271828,0.209303,0.504434}%
\pgfsetfillcolor{currentfill}%
\pgfsetfillopacity{0.800000}%
\pgfsetlinewidth{0.000000pt}%
\definecolor{currentstroke}{rgb}{0.000000,0.000000,0.000000}%
\pgfsetstrokecolor{currentstroke}%
\pgfsetdash{}{0pt}%
\pgfpathmoveto{\pgfqpoint{4.117765in}{2.013575in}}%
\pgfpathlineto{\pgfqpoint{4.131680in}{2.018805in}}%
\pgfpathlineto{\pgfqpoint{4.145606in}{2.024222in}}%
\pgfpathlineto{\pgfqpoint{4.159545in}{2.029826in}}%
\pgfpathlineto{\pgfqpoint{4.173495in}{2.035616in}}%
\pgfpathlineto{\pgfqpoint{4.181498in}{2.047870in}}%
\pgfpathlineto{\pgfqpoint{4.189497in}{2.060056in}}%
\pgfpathlineto{\pgfqpoint{4.197490in}{2.072173in}}%
\pgfpathlineto{\pgfqpoint{4.205478in}{2.084220in}}%
\pgfpathlineto{\pgfqpoint{4.191531in}{2.078245in}}%
\pgfpathlineto{\pgfqpoint{4.177596in}{2.072458in}}%
\pgfpathlineto{\pgfqpoint{4.163673in}{2.066857in}}%
\pgfpathlineto{\pgfqpoint{4.149761in}{2.061443in}}%
\pgfpathlineto{\pgfqpoint{4.141769in}{2.049568in}}%
\pgfpathlineto{\pgfqpoint{4.133773in}{2.037630in}}%
\pgfpathlineto{\pgfqpoint{4.125771in}{2.025632in}}%
\pgfpathlineto{\pgfqpoint{4.117765in}{2.013575in}}%
\pgfpathclose%
\pgfusepath{fill}%
\end{pgfscope}%
\begin{pgfscope}%
\pgfpathrectangle{\pgfqpoint{1.150000in}{0.150000in}}{\pgfqpoint{5.700000in}{5.700000in}}%
\pgfusepath{clip}%
\pgfsetbuttcap%
\pgfsetroundjoin%
\definecolor{currentfill}{rgb}{0.277941,0.056324,0.381191}%
\pgfsetfillcolor{currentfill}%
\pgfsetfillopacity{0.800000}%
\pgfsetlinewidth{0.000000pt}%
\definecolor{currentstroke}{rgb}{0.000000,0.000000,0.000000}%
\pgfsetstrokecolor{currentstroke}%
\pgfsetdash{}{0pt}%
\pgfpathmoveto{\pgfqpoint{3.647282in}{1.682107in}}%
\pgfpathlineto{\pgfqpoint{3.661053in}{1.681776in}}%
\pgfpathlineto{\pgfqpoint{3.674831in}{1.681638in}}%
\pgfpathlineto{\pgfqpoint{3.688615in}{1.681693in}}%
\pgfpathlineto{\pgfqpoint{3.702407in}{1.681941in}}%
\pgfpathlineto{\pgfqpoint{3.710557in}{1.693219in}}%
\pgfpathlineto{\pgfqpoint{3.718702in}{1.704530in}}%
\pgfpathlineto{\pgfqpoint{3.726841in}{1.715870in}}%
\pgfpathlineto{\pgfqpoint{3.734974in}{1.727236in}}%
\pgfpathlineto{\pgfqpoint{3.721192in}{1.726616in}}%
\pgfpathlineto{\pgfqpoint{3.707417in}{1.726188in}}%
\pgfpathlineto{\pgfqpoint{3.693650in}{1.725953in}}%
\pgfpathlineto{\pgfqpoint{3.679889in}{1.725912in}}%
\pgfpathlineto{\pgfqpoint{3.671746in}{1.714907in}}%
\pgfpathlineto{\pgfqpoint{3.663597in}{1.703935in}}%
\pgfpathlineto{\pgfqpoint{3.655442in}{1.693001in}}%
\pgfpathlineto{\pgfqpoint{3.647282in}{1.682107in}}%
\pgfpathclose%
\pgfusepath{fill}%
\end{pgfscope}%
\begin{pgfscope}%
\pgfpathrectangle{\pgfqpoint{1.150000in}{0.150000in}}{\pgfqpoint{5.700000in}{5.700000in}}%
\pgfusepath{clip}%
\pgfsetbuttcap%
\pgfsetroundjoin%
\definecolor{currentfill}{rgb}{0.283072,0.130895,0.449241}%
\pgfsetfillcolor{currentfill}%
\pgfsetfillopacity{0.800000}%
\pgfsetlinewidth{0.000000pt}%
\definecolor{currentstroke}{rgb}{0.000000,0.000000,0.000000}%
\pgfsetstrokecolor{currentstroke}%
\pgfsetdash{}{0pt}%
\pgfpathmoveto{\pgfqpoint{2.730870in}{1.896835in}}%
\pgfpathlineto{\pgfqpoint{2.744739in}{1.882352in}}%
\pgfpathlineto{\pgfqpoint{2.758603in}{1.868119in}}%
\pgfpathlineto{\pgfqpoint{2.772461in}{1.854132in}}%
\pgfpathlineto{\pgfqpoint{2.786314in}{1.840391in}}%
\pgfpathlineto{\pgfqpoint{2.794988in}{1.842108in}}%
\pgfpathlineto{\pgfqpoint{2.803647in}{1.844073in}}%
\pgfpathlineto{\pgfqpoint{2.812292in}{1.846281in}}%
\pgfpathlineto{\pgfqpoint{2.820922in}{1.848725in}}%
\pgfpathlineto{\pgfqpoint{2.807108in}{1.861890in}}%
\pgfpathlineto{\pgfqpoint{2.793289in}{1.875299in}}%
\pgfpathlineto{\pgfqpoint{2.779465in}{1.888955in}}%
\pgfpathlineto{\pgfqpoint{2.765637in}{1.902858in}}%
\pgfpathlineto{\pgfqpoint{2.756968in}{1.900978in}}%
\pgfpathlineto{\pgfqpoint{2.748284in}{1.899344in}}%
\pgfpathlineto{\pgfqpoint{2.739585in}{1.897961in}}%
\pgfpathlineto{\pgfqpoint{2.730870in}{1.896835in}}%
\pgfpathclose%
\pgfusepath{fill}%
\end{pgfscope}%
\begin{pgfscope}%
\pgfpathrectangle{\pgfqpoint{1.150000in}{0.150000in}}{\pgfqpoint{5.700000in}{5.700000in}}%
\pgfusepath{clip}%
\pgfsetbuttcap%
\pgfsetroundjoin%
\definecolor{currentfill}{rgb}{0.273809,0.031497,0.358853}%
\pgfsetfillcolor{currentfill}%
\pgfsetfillopacity{0.800000}%
\pgfsetlinewidth{0.000000pt}%
\definecolor{currentstroke}{rgb}{0.000000,0.000000,0.000000}%
\pgfsetstrokecolor{currentstroke}%
\pgfsetdash{}{0pt}%
\pgfpathmoveto{\pgfqpoint{3.041514in}{1.670134in}}%
\pgfpathlineto{\pgfqpoint{3.055284in}{1.660904in}}%
\pgfpathlineto{\pgfqpoint{3.069054in}{1.651894in}}%
\pgfpathlineto{\pgfqpoint{3.082824in}{1.643102in}}%
\pgfpathlineto{\pgfqpoint{3.096593in}{1.634528in}}%
\pgfpathlineto{\pgfqpoint{3.105045in}{1.639953in}}%
\pgfpathlineto{\pgfqpoint{3.113486in}{1.645561in}}%
\pgfpathlineto{\pgfqpoint{3.121916in}{1.651348in}}%
\pgfpathlineto{\pgfqpoint{3.130336in}{1.657309in}}%
\pgfpathlineto{\pgfqpoint{3.116595in}{1.665349in}}%
\pgfpathlineto{\pgfqpoint{3.102854in}{1.673607in}}%
\pgfpathlineto{\pgfqpoint{3.089113in}{1.682083in}}%
\pgfpathlineto{\pgfqpoint{3.075371in}{1.690779in}}%
\pgfpathlineto{\pgfqpoint{3.066924in}{1.685340in}}%
\pgfpathlineto{\pgfqpoint{3.058465in}{1.680083in}}%
\pgfpathlineto{\pgfqpoint{3.049995in}{1.675012in}}%
\pgfpathlineto{\pgfqpoint{3.041514in}{1.670134in}}%
\pgfpathclose%
\pgfusepath{fill}%
\end{pgfscope}%
\begin{pgfscope}%
\pgfpathrectangle{\pgfqpoint{1.150000in}{0.150000in}}{\pgfqpoint{5.700000in}{5.700000in}}%
\pgfusepath{clip}%
\pgfsetbuttcap%
\pgfsetroundjoin%
\definecolor{currentfill}{rgb}{0.280894,0.078907,0.402329}%
\pgfsetfillcolor{currentfill}%
\pgfsetfillopacity{0.800000}%
\pgfsetlinewidth{0.000000pt}%
\definecolor{currentstroke}{rgb}{0.000000,0.000000,0.000000}%
\pgfsetstrokecolor{currentstroke}%
\pgfsetdash{}{0pt}%
\pgfpathmoveto{\pgfqpoint{3.734974in}{1.727236in}}%
\pgfpathlineto{\pgfqpoint{3.748764in}{1.728048in}}%
\pgfpathlineto{\pgfqpoint{3.762562in}{1.729051in}}%
\pgfpathlineto{\pgfqpoint{3.776367in}{1.730245in}}%
\pgfpathlineto{\pgfqpoint{3.790181in}{1.731630in}}%
\pgfpathlineto{\pgfqpoint{3.798301in}{1.743372in}}%
\pgfpathlineto{\pgfqpoint{3.806416in}{1.755125in}}%
\pgfpathlineto{\pgfqpoint{3.814525in}{1.766887in}}%
\pgfpathlineto{\pgfqpoint{3.822630in}{1.778654in}}%
\pgfpathlineto{\pgfqpoint{3.808824in}{1.776927in}}%
\pgfpathlineto{\pgfqpoint{3.795026in}{1.775391in}}%
\pgfpathlineto{\pgfqpoint{3.781237in}{1.774047in}}%
\pgfpathlineto{\pgfqpoint{3.767456in}{1.772893in}}%
\pgfpathlineto{\pgfqpoint{3.759343in}{1.761456in}}%
\pgfpathlineto{\pgfqpoint{3.751226in}{1.750032in}}%
\pgfpathlineto{\pgfqpoint{3.743103in}{1.738624in}}%
\pgfpathlineto{\pgfqpoint{3.734974in}{1.727236in}}%
\pgfpathclose%
\pgfusepath{fill}%
\end{pgfscope}%
\begin{pgfscope}%
\pgfpathrectangle{\pgfqpoint{1.150000in}{0.150000in}}{\pgfqpoint{5.700000in}{5.700000in}}%
\pgfusepath{clip}%
\pgfsetbuttcap%
\pgfsetroundjoin%
\definecolor{currentfill}{rgb}{0.180629,0.429975,0.557282}%
\pgfsetfillcolor{currentfill}%
\pgfsetfillopacity{0.800000}%
\pgfsetlinewidth{0.000000pt}%
\definecolor{currentstroke}{rgb}{0.000000,0.000000,0.000000}%
\pgfsetstrokecolor{currentstroke}%
\pgfsetdash{}{0pt}%
\pgfpathmoveto{\pgfqpoint{2.205359in}{2.686172in}}%
\pgfpathlineto{\pgfqpoint{2.219613in}{2.660535in}}%
\pgfpathlineto{\pgfqpoint{2.233850in}{2.635247in}}%
\pgfpathlineto{\pgfqpoint{2.248071in}{2.610303in}}%
\pgfpathlineto{\pgfqpoint{2.262276in}{2.585701in}}%
\pgfpathlineto{\pgfqpoint{2.271360in}{2.582651in}}%
\pgfpathlineto{\pgfqpoint{2.280424in}{2.579917in}}%
\pgfpathlineto{\pgfqpoint{2.289467in}{2.577493in}}%
\pgfpathlineto{\pgfqpoint{2.298489in}{2.575374in}}%
\pgfpathlineto{\pgfqpoint{2.284340in}{2.599393in}}%
\pgfpathlineto{\pgfqpoint{2.270177in}{2.623752in}}%
\pgfpathlineto{\pgfqpoint{2.255997in}{2.648455in}}%
\pgfpathlineto{\pgfqpoint{2.241800in}{2.673504in}}%
\pgfpathlineto{\pgfqpoint{2.232722in}{2.676194in}}%
\pgfpathlineto{\pgfqpoint{2.223623in}{2.679199in}}%
\pgfpathlineto{\pgfqpoint{2.214502in}{2.682523in}}%
\pgfpathlineto{\pgfqpoint{2.205359in}{2.686172in}}%
\pgfpathclose%
\pgfusepath{fill}%
\end{pgfscope}%
\begin{pgfscope}%
\pgfpathrectangle{\pgfqpoint{1.150000in}{0.150000in}}{\pgfqpoint{5.700000in}{5.700000in}}%
\pgfusepath{clip}%
\pgfsetbuttcap%
\pgfsetroundjoin%
\definecolor{currentfill}{rgb}{0.273809,0.031497,0.358853}%
\pgfsetfillcolor{currentfill}%
\pgfsetfillopacity{0.800000}%
\pgfsetlinewidth{0.000000pt}%
\definecolor{currentstroke}{rgb}{0.000000,0.000000,0.000000}%
\pgfsetstrokecolor{currentstroke}%
\pgfsetdash{}{0pt}%
\pgfpathmoveto{\pgfqpoint{3.559514in}{1.643892in}}%
\pgfpathlineto{\pgfqpoint{3.573272in}{1.642379in}}%
\pgfpathlineto{\pgfqpoint{3.587036in}{1.641061in}}%
\pgfpathlineto{\pgfqpoint{3.600806in}{1.639938in}}%
\pgfpathlineto{\pgfqpoint{3.614582in}{1.639009in}}%
\pgfpathlineto{\pgfqpoint{3.622766in}{1.649705in}}%
\pgfpathlineto{\pgfqpoint{3.630944in}{1.660455in}}%
\pgfpathlineto{\pgfqpoint{3.639116in}{1.671257in}}%
\pgfpathlineto{\pgfqpoint{3.647282in}{1.682107in}}%
\pgfpathlineto{\pgfqpoint{3.633518in}{1.682632in}}%
\pgfpathlineto{\pgfqpoint{3.619760in}{1.683351in}}%
\pgfpathlineto{\pgfqpoint{3.606008in}{1.684265in}}%
\pgfpathlineto{\pgfqpoint{3.592263in}{1.685375in}}%
\pgfpathlineto{\pgfqpoint{3.584085in}{1.674917in}}%
\pgfpathlineto{\pgfqpoint{3.575901in}{1.664514in}}%
\pgfpathlineto{\pgfqpoint{3.567711in}{1.654172in}}%
\pgfpathlineto{\pgfqpoint{3.559514in}{1.643892in}}%
\pgfpathclose%
\pgfusepath{fill}%
\end{pgfscope}%
\begin{pgfscope}%
\pgfpathrectangle{\pgfqpoint{1.150000in}{0.150000in}}{\pgfqpoint{5.700000in}{5.700000in}}%
\pgfusepath{clip}%
\pgfsetbuttcap%
\pgfsetroundjoin%
\definecolor{currentfill}{rgb}{0.250425,0.274290,0.533103}%
\pgfsetfillcolor{currentfill}%
\pgfsetfillopacity{0.800000}%
\pgfsetlinewidth{0.000000pt}%
\definecolor{currentstroke}{rgb}{0.000000,0.000000,0.000000}%
\pgfsetstrokecolor{currentstroke}%
\pgfsetdash{}{0pt}%
\pgfpathmoveto{\pgfqpoint{2.451953in}{2.241636in}}%
\pgfpathlineto{\pgfqpoint{2.465985in}{2.221768in}}%
\pgfpathlineto{\pgfqpoint{2.480007in}{2.202191in}}%
\pgfpathlineto{\pgfqpoint{2.494018in}{2.182901in}}%
\pgfpathlineto{\pgfqpoint{2.508018in}{2.163897in}}%
\pgfpathlineto{\pgfqpoint{2.516916in}{2.162626in}}%
\pgfpathlineto{\pgfqpoint{2.525795in}{2.161648in}}%
\pgfpathlineto{\pgfqpoint{2.534656in}{2.160957in}}%
\pgfpathlineto{\pgfqpoint{2.543499in}{2.160546in}}%
\pgfpathlineto{\pgfqpoint{2.529547in}{2.178956in}}%
\pgfpathlineto{\pgfqpoint{2.515585in}{2.197651in}}%
\pgfpathlineto{\pgfqpoint{2.501614in}{2.216632in}}%
\pgfpathlineto{\pgfqpoint{2.487632in}{2.235902in}}%
\pgfpathlineto{\pgfqpoint{2.478741in}{2.236894in}}%
\pgfpathlineto{\pgfqpoint{2.469830in}{2.238177in}}%
\pgfpathlineto{\pgfqpoint{2.460901in}{2.239755in}}%
\pgfpathlineto{\pgfqpoint{2.451953in}{2.241636in}}%
\pgfpathclose%
\pgfusepath{fill}%
\end{pgfscope}%
\begin{pgfscope}%
\pgfpathrectangle{\pgfqpoint{1.150000in}{0.150000in}}{\pgfqpoint{5.700000in}{5.700000in}}%
\pgfusepath{clip}%
\pgfsetbuttcap%
\pgfsetroundjoin%
\definecolor{currentfill}{rgb}{0.282910,0.105393,0.426902}%
\pgfsetfillcolor{currentfill}%
\pgfsetfillopacity{0.800000}%
\pgfsetlinewidth{0.000000pt}%
\definecolor{currentstroke}{rgb}{0.000000,0.000000,0.000000}%
\pgfsetstrokecolor{currentstroke}%
\pgfsetdash{}{0pt}%
\pgfpathmoveto{\pgfqpoint{3.822630in}{1.778654in}}%
\pgfpathlineto{\pgfqpoint{3.836444in}{1.780571in}}%
\pgfpathlineto{\pgfqpoint{3.850267in}{1.782678in}}%
\pgfpathlineto{\pgfqpoint{3.864099in}{1.784974in}}%
\pgfpathlineto{\pgfqpoint{3.877940in}{1.787460in}}%
\pgfpathlineto{\pgfqpoint{3.886033in}{1.799551in}}%
\pgfpathlineto{\pgfqpoint{3.894121in}{1.811634in}}%
\pgfpathlineto{\pgfqpoint{3.902204in}{1.823706in}}%
\pgfpathlineto{\pgfqpoint{3.910282in}{1.835763in}}%
\pgfpathlineto{\pgfqpoint{3.896447in}{1.832967in}}%
\pgfpathlineto{\pgfqpoint{3.882621in}{1.830360in}}%
\pgfpathlineto{\pgfqpoint{3.868805in}{1.827943in}}%
\pgfpathlineto{\pgfqpoint{3.854998in}{1.825716in}}%
\pgfpathlineto{\pgfqpoint{3.846913in}{1.813957in}}%
\pgfpathlineto{\pgfqpoint{3.838824in}{1.802192in}}%
\pgfpathlineto{\pgfqpoint{3.830729in}{1.790423in}}%
\pgfpathlineto{\pgfqpoint{3.822630in}{1.778654in}}%
\pgfpathclose%
\pgfusepath{fill}%
\end{pgfscope}%
\begin{pgfscope}%
\pgfpathrectangle{\pgfqpoint{1.150000in}{0.150000in}}{\pgfqpoint{5.700000in}{5.700000in}}%
\pgfusepath{clip}%
\pgfsetbuttcap%
\pgfsetroundjoin%
\definecolor{currentfill}{rgb}{0.197636,0.391528,0.554969}%
\pgfsetfillcolor{currentfill}%
\pgfsetfillopacity{0.800000}%
\pgfsetlinewidth{0.000000pt}%
\definecolor{currentstroke}{rgb}{0.000000,0.000000,0.000000}%
\pgfsetstrokecolor{currentstroke}%
\pgfsetdash{}{0pt}%
\pgfpathmoveto{\pgfqpoint{4.620283in}{2.477696in}}%
\pgfpathlineto{\pgfqpoint{4.634448in}{2.487241in}}%
\pgfpathlineto{\pgfqpoint{4.648629in}{2.496970in}}%
\pgfpathlineto{\pgfqpoint{4.662825in}{2.506884in}}%
\pgfpathlineto{\pgfqpoint{4.677037in}{2.516983in}}%
\pgfpathlineto{\pgfqpoint{4.684869in}{2.526903in}}%
\pgfpathlineto{\pgfqpoint{4.692694in}{2.536705in}}%
\pgfpathlineto{\pgfqpoint{4.700512in}{2.546389in}}%
\pgfpathlineto{\pgfqpoint{4.708324in}{2.555958in}}%
\pgfpathlineto{\pgfqpoint{4.694116in}{2.545906in}}%
\pgfpathlineto{\pgfqpoint{4.679924in}{2.536038in}}%
\pgfpathlineto{\pgfqpoint{4.665748in}{2.526355in}}%
\pgfpathlineto{\pgfqpoint{4.651588in}{2.516856in}}%
\pgfpathlineto{\pgfqpoint{4.643771in}{2.507229in}}%
\pgfpathlineto{\pgfqpoint{4.635948in}{2.497494in}}%
\pgfpathlineto{\pgfqpoint{4.628119in}{2.487649in}}%
\pgfpathlineto{\pgfqpoint{4.620283in}{2.477696in}}%
\pgfpathclose%
\pgfusepath{fill}%
\end{pgfscope}%
\begin{pgfscope}%
\pgfpathrectangle{\pgfqpoint{1.150000in}{0.150000in}}{\pgfqpoint{5.700000in}{5.700000in}}%
\pgfusepath{clip}%
\pgfsetbuttcap%
\pgfsetroundjoin%
\definecolor{currentfill}{rgb}{0.132268,0.655014,0.519661}%
\pgfsetfillcolor{currentfill}%
\pgfsetfillopacity{0.800000}%
\pgfsetlinewidth{0.000000pt}%
\definecolor{currentstroke}{rgb}{0.000000,0.000000,0.000000}%
\pgfsetstrokecolor{currentstroke}%
\pgfsetdash{}{0pt}%
\pgfpathmoveto{\pgfqpoint{5.593423in}{3.288111in}}%
\pgfpathlineto{\pgfqpoint{5.608165in}{3.301686in}}%
\pgfpathlineto{\pgfqpoint{5.622929in}{3.315442in}}%
\pgfpathlineto{\pgfqpoint{5.637715in}{3.329379in}}%
\pgfpathlineto{\pgfqpoint{5.652521in}{3.343498in}}%
\pgfpathlineto{\pgfqpoint{5.659804in}{3.345644in}}%
\pgfpathlineto{\pgfqpoint{5.667078in}{3.347753in}}%
\pgfpathlineto{\pgfqpoint{5.674345in}{3.349829in}}%
\pgfpathlineto{\pgfqpoint{5.681603in}{3.351877in}}%
\pgfpathlineto{\pgfqpoint{5.666823in}{3.338256in}}%
\pgfpathlineto{\pgfqpoint{5.652064in}{3.324816in}}%
\pgfpathlineto{\pgfqpoint{5.637325in}{3.311556in}}%
\pgfpathlineto{\pgfqpoint{5.622608in}{3.298476in}}%
\pgfpathlineto{\pgfqpoint{5.615323in}{3.295919in}}%
\pgfpathlineto{\pgfqpoint{5.608030in}{3.293343in}}%
\pgfpathlineto{\pgfqpoint{5.600730in}{3.290742in}}%
\pgfpathlineto{\pgfqpoint{5.593423in}{3.288111in}}%
\pgfpathclose%
\pgfusepath{fill}%
\end{pgfscope}%
\begin{pgfscope}%
\pgfpathrectangle{\pgfqpoint{1.150000in}{0.150000in}}{\pgfqpoint{5.700000in}{5.700000in}}%
\pgfusepath{clip}%
\pgfsetbuttcap%
\pgfsetroundjoin%
\definecolor{currentfill}{rgb}{0.283091,0.110553,0.431554}%
\pgfsetfillcolor{currentfill}%
\pgfsetfillopacity{0.800000}%
\pgfsetlinewidth{0.000000pt}%
\definecolor{currentstroke}{rgb}{0.000000,0.000000,0.000000}%
\pgfsetstrokecolor{currentstroke}%
\pgfsetdash{}{0pt}%
\pgfpathmoveto{\pgfqpoint{2.786314in}{1.840391in}}%
\pgfpathlineto{\pgfqpoint{2.800163in}{1.826894in}}%
\pgfpathlineto{\pgfqpoint{2.814007in}{1.813638in}}%
\pgfpathlineto{\pgfqpoint{2.827847in}{1.800624in}}%
\pgfpathlineto{\pgfqpoint{2.841682in}{1.787848in}}%
\pgfpathlineto{\pgfqpoint{2.850317in}{1.790153in}}%
\pgfpathlineto{\pgfqpoint{2.858938in}{1.792697in}}%
\pgfpathlineto{\pgfqpoint{2.867545in}{1.795476in}}%
\pgfpathlineto{\pgfqpoint{2.876138in}{1.798482in}}%
\pgfpathlineto{\pgfqpoint{2.862340in}{1.810684in}}%
\pgfpathlineto{\pgfqpoint{2.848538in}{1.823124in}}%
\pgfpathlineto{\pgfqpoint{2.834732in}{1.835804in}}%
\pgfpathlineto{\pgfqpoint{2.820922in}{1.848725in}}%
\pgfpathlineto{\pgfqpoint{2.812292in}{1.846281in}}%
\pgfpathlineto{\pgfqpoint{2.803647in}{1.844073in}}%
\pgfpathlineto{\pgfqpoint{2.794988in}{1.842108in}}%
\pgfpathlineto{\pgfqpoint{2.786314in}{1.840391in}}%
\pgfpathclose%
\pgfusepath{fill}%
\end{pgfscope}%
\begin{pgfscope}%
\pgfpathrectangle{\pgfqpoint{1.150000in}{0.150000in}}{\pgfqpoint{5.700000in}{5.700000in}}%
\pgfusepath{clip}%
\pgfsetbuttcap%
\pgfsetroundjoin%
\definecolor{currentfill}{rgb}{0.231674,0.318106,0.544834}%
\pgfsetfillcolor{currentfill}%
\pgfsetfillopacity{0.800000}%
\pgfsetlinewidth{0.000000pt}%
\definecolor{currentstroke}{rgb}{0.000000,0.000000,0.000000}%
\pgfsetstrokecolor{currentstroke}%
\pgfsetdash{}{0pt}%
\pgfpathmoveto{\pgfqpoint{4.412909in}{2.280262in}}%
\pgfpathlineto{\pgfqpoint{4.426966in}{2.288267in}}%
\pgfpathlineto{\pgfqpoint{4.441037in}{2.296456in}}%
\pgfpathlineto{\pgfqpoint{4.455122in}{2.304831in}}%
\pgfpathlineto{\pgfqpoint{4.469222in}{2.313390in}}%
\pgfpathlineto{\pgfqpoint{4.477134in}{2.324634in}}%
\pgfpathlineto{\pgfqpoint{4.485040in}{2.335772in}}%
\pgfpathlineto{\pgfqpoint{4.492940in}{2.346804in}}%
\pgfpathlineto{\pgfqpoint{4.500835in}{2.357731in}}%
\pgfpathlineto{\pgfqpoint{4.486738in}{2.349118in}}%
\pgfpathlineto{\pgfqpoint{4.472656in}{2.340689in}}%
\pgfpathlineto{\pgfqpoint{4.458588in}{2.332446in}}%
\pgfpathlineto{\pgfqpoint{4.444534in}{2.324388in}}%
\pgfpathlineto{\pgfqpoint{4.436636in}{2.313503in}}%
\pgfpathlineto{\pgfqpoint{4.428733in}{2.302520in}}%
\pgfpathlineto{\pgfqpoint{4.420824in}{2.291440in}}%
\pgfpathlineto{\pgfqpoint{4.412909in}{2.280262in}}%
\pgfpathclose%
\pgfusepath{fill}%
\end{pgfscope}%
\begin{pgfscope}%
\pgfpathrectangle{\pgfqpoint{1.150000in}{0.150000in}}{\pgfqpoint{5.700000in}{5.700000in}}%
\pgfusepath{clip}%
\pgfsetbuttcap%
\pgfsetroundjoin%
\definecolor{currentfill}{rgb}{0.269944,0.014625,0.341379}%
\pgfsetfillcolor{currentfill}%
\pgfsetfillopacity{0.800000}%
\pgfsetlinewidth{0.000000pt}%
\definecolor{currentstroke}{rgb}{0.000000,0.000000,0.000000}%
\pgfsetstrokecolor{currentstroke}%
\pgfsetdash{}{0pt}%
\pgfpathmoveto{\pgfqpoint{3.471629in}{1.613240in}}%
\pgfpathlineto{\pgfqpoint{3.485380in}{1.610505in}}%
\pgfpathlineto{\pgfqpoint{3.499137in}{1.607967in}}%
\pgfpathlineto{\pgfqpoint{3.512898in}{1.605626in}}%
\pgfpathlineto{\pgfqpoint{3.526665in}{1.603482in}}%
\pgfpathlineto{\pgfqpoint{3.534887in}{1.613470in}}%
\pgfpathlineto{\pgfqpoint{3.543102in}{1.623537in}}%
\pgfpathlineto{\pgfqpoint{3.551311in}{1.633679in}}%
\pgfpathlineto{\pgfqpoint{3.559514in}{1.643892in}}%
\pgfpathlineto{\pgfqpoint{3.545762in}{1.645601in}}%
\pgfpathlineto{\pgfqpoint{3.532015in}{1.647506in}}%
\pgfpathlineto{\pgfqpoint{3.518273in}{1.649609in}}%
\pgfpathlineto{\pgfqpoint{3.504537in}{1.651910in}}%
\pgfpathlineto{\pgfqpoint{3.496320in}{1.642120in}}%
\pgfpathlineto{\pgfqpoint{3.488096in}{1.632409in}}%
\pgfpathlineto{\pgfqpoint{3.479866in}{1.622781in}}%
\pgfpathlineto{\pgfqpoint{3.471629in}{1.613240in}}%
\pgfpathclose%
\pgfusepath{fill}%
\end{pgfscope}%
\begin{pgfscope}%
\pgfpathrectangle{\pgfqpoint{1.150000in}{0.150000in}}{\pgfqpoint{5.700000in}{5.700000in}}%
\pgfusepath{clip}%
\pgfsetbuttcap%
\pgfsetroundjoin%
\definecolor{currentfill}{rgb}{0.282884,0.135920,0.453427}%
\pgfsetfillcolor{currentfill}%
\pgfsetfillopacity{0.800000}%
\pgfsetlinewidth{0.000000pt}%
\definecolor{currentstroke}{rgb}{0.000000,0.000000,0.000000}%
\pgfsetstrokecolor{currentstroke}%
\pgfsetdash{}{0pt}%
\pgfpathmoveto{\pgfqpoint{3.910282in}{1.835763in}}%
\pgfpathlineto{\pgfqpoint{3.924126in}{1.838749in}}%
\pgfpathlineto{\pgfqpoint{3.937980in}{1.841922in}}%
\pgfpathlineto{\pgfqpoint{3.951844in}{1.845285in}}%
\pgfpathlineto{\pgfqpoint{3.965717in}{1.848835in}}%
\pgfpathlineto{\pgfqpoint{3.973785in}{1.861167in}}%
\pgfpathlineto{\pgfqpoint{3.981848in}{1.873472in}}%
\pgfpathlineto{\pgfqpoint{3.989907in}{1.885747in}}%
\pgfpathlineto{\pgfqpoint{3.997960in}{1.897990in}}%
\pgfpathlineto{\pgfqpoint{3.984092in}{1.894161in}}%
\pgfpathlineto{\pgfqpoint{3.970233in}{1.890520in}}%
\pgfpathlineto{\pgfqpoint{3.956384in}{1.887067in}}%
\pgfpathlineto{\pgfqpoint{3.942546in}{1.883803in}}%
\pgfpathlineto{\pgfqpoint{3.934487in}{1.871826in}}%
\pgfpathlineto{\pgfqpoint{3.926423in}{1.859826in}}%
\pgfpathlineto{\pgfqpoint{3.918355in}{1.847804in}}%
\pgfpathlineto{\pgfqpoint{3.910282in}{1.835763in}}%
\pgfpathclose%
\pgfusepath{fill}%
\end{pgfscope}%
\begin{pgfscope}%
\pgfpathrectangle{\pgfqpoint{1.150000in}{0.150000in}}{\pgfqpoint{5.700000in}{5.700000in}}%
\pgfusepath{clip}%
\pgfsetbuttcap%
\pgfsetroundjoin%
\definecolor{currentfill}{rgb}{0.128729,0.563265,0.551229}%
\pgfsetfillcolor{currentfill}%
\pgfsetfillopacity{0.800000}%
\pgfsetlinewidth{0.000000pt}%
\definecolor{currentstroke}{rgb}{0.000000,0.000000,0.000000}%
\pgfsetstrokecolor{currentstroke}%
\pgfsetdash{}{0pt}%
\pgfpathmoveto{\pgfqpoint{5.210797in}{2.997586in}}%
\pgfpathlineto{\pgfqpoint{5.225314in}{3.010209in}}%
\pgfpathlineto{\pgfqpoint{5.239849in}{3.023014in}}%
\pgfpathlineto{\pgfqpoint{5.254405in}{3.036003in}}%
\pgfpathlineto{\pgfqpoint{5.268980in}{3.049175in}}%
\pgfpathlineto{\pgfqpoint{5.276512in}{3.054362in}}%
\pgfpathlineto{\pgfqpoint{5.284036in}{3.059453in}}%
\pgfpathlineto{\pgfqpoint{5.291551in}{3.064452in}}%
\pgfpathlineto{\pgfqpoint{5.299058in}{3.069363in}}%
\pgfpathlineto{\pgfqpoint{5.284498in}{3.056514in}}%
\pgfpathlineto{\pgfqpoint{5.269958in}{3.043848in}}%
\pgfpathlineto{\pgfqpoint{5.255438in}{3.031364in}}%
\pgfpathlineto{\pgfqpoint{5.240936in}{3.019061in}}%
\pgfpathlineto{\pgfqpoint{5.233414in}{3.013817in}}%
\pgfpathlineto{\pgfqpoint{5.225883in}{3.008492in}}%
\pgfpathlineto{\pgfqpoint{5.218344in}{3.003083in}}%
\pgfpathlineto{\pgfqpoint{5.210797in}{2.997586in}}%
\pgfpathclose%
\pgfusepath{fill}%
\end{pgfscope}%
\begin{pgfscope}%
\pgfpathrectangle{\pgfqpoint{1.150000in}{0.150000in}}{\pgfqpoint{5.700000in}{5.700000in}}%
\pgfusepath{clip}%
\pgfsetbuttcap%
\pgfsetroundjoin%
\definecolor{currentfill}{rgb}{0.237441,0.305202,0.541921}%
\pgfsetfillcolor{currentfill}%
\pgfsetfillopacity{0.800000}%
\pgfsetlinewidth{0.000000pt}%
\definecolor{currentstroke}{rgb}{0.000000,0.000000,0.000000}%
\pgfsetstrokecolor{currentstroke}%
\pgfsetdash{}{0pt}%
\pgfpathmoveto{\pgfqpoint{2.395711in}{2.324059in}}%
\pgfpathlineto{\pgfqpoint{2.409789in}{2.303005in}}%
\pgfpathlineto{\pgfqpoint{2.423855in}{2.282251in}}%
\pgfpathlineto{\pgfqpoint{2.437910in}{2.261796in}}%
\pgfpathlineto{\pgfqpoint{2.451953in}{2.241636in}}%
\pgfpathlineto{\pgfqpoint{2.460901in}{2.239755in}}%
\pgfpathlineto{\pgfqpoint{2.469830in}{2.238177in}}%
\pgfpathlineto{\pgfqpoint{2.478741in}{2.236894in}}%
\pgfpathlineto{\pgfqpoint{2.487632in}{2.235902in}}%
\pgfpathlineto{\pgfqpoint{2.473640in}{2.255463in}}%
\pgfpathlineto{\pgfqpoint{2.459636in}{2.275319in}}%
\pgfpathlineto{\pgfqpoint{2.445622in}{2.295471in}}%
\pgfpathlineto{\pgfqpoint{2.431596in}{2.315923in}}%
\pgfpathlineto{\pgfqpoint{2.422654in}{2.317502in}}%
\pgfpathlineto{\pgfqpoint{2.413693in}{2.319381in}}%
\pgfpathlineto{\pgfqpoint{2.404712in}{2.321564in}}%
\pgfpathlineto{\pgfqpoint{2.395711in}{2.324059in}}%
\pgfpathclose%
\pgfusepath{fill}%
\end{pgfscope}%
\begin{pgfscope}%
\pgfpathrectangle{\pgfqpoint{1.150000in}{0.150000in}}{\pgfqpoint{5.700000in}{5.700000in}}%
\pgfusepath{clip}%
\pgfsetbuttcap%
\pgfsetroundjoin%
\definecolor{currentfill}{rgb}{0.262138,0.242286,0.520837}%
\pgfsetfillcolor{currentfill}%
\pgfsetfillopacity{0.800000}%
\pgfsetlinewidth{0.000000pt}%
\definecolor{currentstroke}{rgb}{0.000000,0.000000,0.000000}%
\pgfsetstrokecolor{currentstroke}%
\pgfsetdash{}{0pt}%
\pgfpathmoveto{\pgfqpoint{4.205478in}{2.084220in}}%
\pgfpathlineto{\pgfqpoint{4.219438in}{2.090380in}}%
\pgfpathlineto{\pgfqpoint{4.233410in}{2.096727in}}%
\pgfpathlineto{\pgfqpoint{4.247394in}{2.103260in}}%
\pgfpathlineto{\pgfqpoint{4.261391in}{2.109978in}}%
\pgfpathlineto{\pgfqpoint{4.269372in}{2.122118in}}%
\pgfpathlineto{\pgfqpoint{4.277348in}{2.134176in}}%
\pgfpathlineto{\pgfqpoint{4.285319in}{2.146152in}}%
\pgfpathlineto{\pgfqpoint{4.293284in}{2.158046in}}%
\pgfpathlineto{\pgfqpoint{4.279290in}{2.151175in}}%
\pgfpathlineto{\pgfqpoint{4.265308in}{2.144491in}}%
\pgfpathlineto{\pgfqpoint{4.251339in}{2.137992in}}%
\pgfpathlineto{\pgfqpoint{4.237382in}{2.131680in}}%
\pgfpathlineto{\pgfqpoint{4.229414in}{2.119926in}}%
\pgfpathlineto{\pgfqpoint{4.221440in}{2.108098in}}%
\pgfpathlineto{\pgfqpoint{4.213462in}{2.096195in}}%
\pgfpathlineto{\pgfqpoint{4.205478in}{2.084220in}}%
\pgfpathclose%
\pgfusepath{fill}%
\end{pgfscope}%
\begin{pgfscope}%
\pgfpathrectangle{\pgfqpoint{1.150000in}{0.150000in}}{\pgfqpoint{5.700000in}{5.700000in}}%
\pgfusepath{clip}%
\pgfsetbuttcap%
\pgfsetroundjoin%
\definecolor{currentfill}{rgb}{0.268510,0.009605,0.335427}%
\pgfsetfillcolor{currentfill}%
\pgfsetfillopacity{0.800000}%
\pgfsetlinewidth{0.000000pt}%
\definecolor{currentstroke}{rgb}{0.000000,0.000000,0.000000}%
\pgfsetstrokecolor{currentstroke}%
\pgfsetdash{}{0pt}%
\pgfpathmoveto{\pgfqpoint{3.240291in}{1.600674in}}%
\pgfpathlineto{\pgfqpoint{3.254042in}{1.594542in}}%
\pgfpathlineto{\pgfqpoint{3.267795in}{1.588616in}}%
\pgfpathlineto{\pgfqpoint{3.281549in}{1.582897in}}%
\pgfpathlineto{\pgfqpoint{3.295307in}{1.577383in}}%
\pgfpathlineto{\pgfqpoint{3.303644in}{1.585047in}}%
\pgfpathlineto{\pgfqpoint{3.311972in}{1.592851in}}%
\pgfpathlineto{\pgfqpoint{3.320292in}{1.600789in}}%
\pgfpathlineto{\pgfqpoint{3.328603in}{1.608857in}}%
\pgfpathlineto{\pgfqpoint{3.314868in}{1.613872in}}%
\pgfpathlineto{\pgfqpoint{3.301135in}{1.619092in}}%
\pgfpathlineto{\pgfqpoint{3.287404in}{1.624518in}}%
\pgfpathlineto{\pgfqpoint{3.273676in}{1.630152in}}%
\pgfpathlineto{\pgfqpoint{3.265343in}{1.622571in}}%
\pgfpathlineto{\pgfqpoint{3.257002in}{1.615128in}}%
\pgfpathlineto{\pgfqpoint{3.248651in}{1.607828in}}%
\pgfpathlineto{\pgfqpoint{3.240291in}{1.600674in}}%
\pgfpathclose%
\pgfusepath{fill}%
\end{pgfscope}%
\begin{pgfscope}%
\pgfpathrectangle{\pgfqpoint{1.150000in}{0.150000in}}{\pgfqpoint{5.700000in}{5.700000in}}%
\pgfusepath{clip}%
\pgfsetbuttcap%
\pgfsetroundjoin%
\definecolor{currentfill}{rgb}{0.150148,0.676631,0.506589}%
\pgfsetfillcolor{currentfill}%
\pgfsetfillopacity{0.800000}%
\pgfsetlinewidth{0.000000pt}%
\definecolor{currentstroke}{rgb}{0.000000,0.000000,0.000000}%
\pgfsetstrokecolor{currentstroke}%
\pgfsetdash{}{0pt}%
\pgfpathmoveto{\pgfqpoint{5.681603in}{3.351877in}}%
\pgfpathlineto{\pgfqpoint{5.696405in}{3.365679in}}%
\pgfpathlineto{\pgfqpoint{5.711229in}{3.379662in}}%
\pgfpathlineto{\pgfqpoint{5.726074in}{3.393825in}}%
\pgfpathlineto{\pgfqpoint{5.740941in}{3.408171in}}%
\pgfpathlineto{\pgfqpoint{5.748164in}{3.409677in}}%
\pgfpathlineto{\pgfqpoint{5.755380in}{3.411161in}}%
\pgfpathlineto{\pgfqpoint{5.762588in}{3.412626in}}%
\pgfpathlineto{\pgfqpoint{5.769789in}{3.414079in}}%
\pgfpathlineto{\pgfqpoint{5.754951in}{3.400267in}}%
\pgfpathlineto{\pgfqpoint{5.740134in}{3.386636in}}%
\pgfpathlineto{\pgfqpoint{5.725339in}{3.373185in}}%
\pgfpathlineto{\pgfqpoint{5.710565in}{3.359913in}}%
\pgfpathlineto{\pgfqpoint{5.703335in}{3.357917in}}%
\pgfpathlineto{\pgfqpoint{5.696099in}{3.355916in}}%
\pgfpathlineto{\pgfqpoint{5.688855in}{3.353904in}}%
\pgfpathlineto{\pgfqpoint{5.681603in}{3.351877in}}%
\pgfpathclose%
\pgfusepath{fill}%
\end{pgfscope}%
\begin{pgfscope}%
\pgfpathrectangle{\pgfqpoint{1.150000in}{0.150000in}}{\pgfqpoint{5.700000in}{5.700000in}}%
\pgfusepath{clip}%
\pgfsetbuttcap%
\pgfsetroundjoin%
\definecolor{currentfill}{rgb}{0.159194,0.482237,0.558073}%
\pgfsetfillcolor{currentfill}%
\pgfsetfillopacity{0.800000}%
\pgfsetlinewidth{0.000000pt}%
\definecolor{currentstroke}{rgb}{0.000000,0.000000,0.000000}%
\pgfsetstrokecolor{currentstroke}%
\pgfsetdash{}{0pt}%
\pgfpathmoveto{\pgfqpoint{4.915637in}{2.746501in}}%
\pgfpathlineto{\pgfqpoint{4.929977in}{2.757865in}}%
\pgfpathlineto{\pgfqpoint{4.944335in}{2.769413in}}%
\pgfpathlineto{\pgfqpoint{4.958710in}{2.781145in}}%
\pgfpathlineto{\pgfqpoint{4.973103in}{2.793061in}}%
\pgfpathlineto{\pgfqpoint{4.980801in}{2.800740in}}%
\pgfpathlineto{\pgfqpoint{4.988491in}{2.808301in}}%
\pgfpathlineto{\pgfqpoint{4.996174in}{2.815746in}}%
\pgfpathlineto{\pgfqpoint{5.003848in}{2.823078in}}%
\pgfpathlineto{\pgfqpoint{4.989464in}{2.811345in}}%
\pgfpathlineto{\pgfqpoint{4.975098in}{2.799797in}}%
\pgfpathlineto{\pgfqpoint{4.960749in}{2.788431in}}%
\pgfpathlineto{\pgfqpoint{4.946418in}{2.777249in}}%
\pgfpathlineto{\pgfqpoint{4.938734in}{2.769722in}}%
\pgfpathlineto{\pgfqpoint{4.931043in}{2.762090in}}%
\pgfpathlineto{\pgfqpoint{4.923344in}{2.754350in}}%
\pgfpathlineto{\pgfqpoint{4.915637in}{2.746501in}}%
\pgfpathclose%
\pgfusepath{fill}%
\end{pgfscope}%
\begin{pgfscope}%
\pgfpathrectangle{\pgfqpoint{1.150000in}{0.150000in}}{\pgfqpoint{5.700000in}{5.700000in}}%
\pgfusepath{clip}%
\pgfsetbuttcap%
\pgfsetroundjoin%
\definecolor{currentfill}{rgb}{0.281924,0.089666,0.412415}%
\pgfsetfillcolor{currentfill}%
\pgfsetfillopacity{0.800000}%
\pgfsetlinewidth{0.000000pt}%
\definecolor{currentstroke}{rgb}{0.000000,0.000000,0.000000}%
\pgfsetstrokecolor{currentstroke}%
\pgfsetdash{}{0pt}%
\pgfpathmoveto{\pgfqpoint{2.841682in}{1.787848in}}%
\pgfpathlineto{\pgfqpoint{2.855514in}{1.775309in}}%
\pgfpathlineto{\pgfqpoint{2.869342in}{1.763007in}}%
\pgfpathlineto{\pgfqpoint{2.883167in}{1.750939in}}%
\pgfpathlineto{\pgfqpoint{2.896989in}{1.739104in}}%
\pgfpathlineto{\pgfqpoint{2.905586in}{1.741995in}}%
\pgfpathlineto{\pgfqpoint{2.914170in}{1.745116in}}%
\pgfpathlineto{\pgfqpoint{2.922741in}{1.748463in}}%
\pgfpathlineto{\pgfqpoint{2.931299in}{1.752029in}}%
\pgfpathlineto{\pgfqpoint{2.917513in}{1.763292in}}%
\pgfpathlineto{\pgfqpoint{2.903725in}{1.774788in}}%
\pgfpathlineto{\pgfqpoint{2.889933in}{1.786517in}}%
\pgfpathlineto{\pgfqpoint{2.876138in}{1.798482in}}%
\pgfpathlineto{\pgfqpoint{2.867545in}{1.795476in}}%
\pgfpathlineto{\pgfqpoint{2.858938in}{1.792697in}}%
\pgfpathlineto{\pgfqpoint{2.850317in}{1.790153in}}%
\pgfpathlineto{\pgfqpoint{2.841682in}{1.787848in}}%
\pgfpathclose%
\pgfusepath{fill}%
\end{pgfscope}%
\begin{pgfscope}%
\pgfpathrectangle{\pgfqpoint{1.150000in}{0.150000in}}{\pgfqpoint{5.700000in}{5.700000in}}%
\pgfusepath{clip}%
\pgfsetbuttcap%
\pgfsetroundjoin%
\definecolor{currentfill}{rgb}{0.271305,0.019942,0.347269}%
\pgfsetfillcolor{currentfill}%
\pgfsetfillopacity{0.800000}%
\pgfsetlinewidth{0.000000pt}%
\definecolor{currentstroke}{rgb}{0.000000,0.000000,0.000000}%
\pgfsetstrokecolor{currentstroke}%
\pgfsetdash{}{0pt}%
\pgfpathmoveto{\pgfqpoint{3.096593in}{1.634528in}}%
\pgfpathlineto{\pgfqpoint{3.110362in}{1.626170in}}%
\pgfpathlineto{\pgfqpoint{3.124131in}{1.618027in}}%
\pgfpathlineto{\pgfqpoint{3.137901in}{1.610099in}}%
\pgfpathlineto{\pgfqpoint{3.151670in}{1.602383in}}%
\pgfpathlineto{\pgfqpoint{3.160094in}{1.608353in}}%
\pgfpathlineto{\pgfqpoint{3.168508in}{1.614499in}}%
\pgfpathlineto{\pgfqpoint{3.176911in}{1.620815in}}%
\pgfpathlineto{\pgfqpoint{3.185305in}{1.627296in}}%
\pgfpathlineto{\pgfqpoint{3.171562in}{1.634479in}}%
\pgfpathlineto{\pgfqpoint{3.157819in}{1.641875in}}%
\pgfpathlineto{\pgfqpoint{3.144077in}{1.649484in}}%
\pgfpathlineto{\pgfqpoint{3.130336in}{1.657309in}}%
\pgfpathlineto{\pgfqpoint{3.121916in}{1.651348in}}%
\pgfpathlineto{\pgfqpoint{3.113486in}{1.645561in}}%
\pgfpathlineto{\pgfqpoint{3.105045in}{1.639953in}}%
\pgfpathlineto{\pgfqpoint{3.096593in}{1.634528in}}%
\pgfpathclose%
\pgfusepath{fill}%
\end{pgfscope}%
\begin{pgfscope}%
\pgfpathrectangle{\pgfqpoint{1.150000in}{0.150000in}}{\pgfqpoint{5.700000in}{5.700000in}}%
\pgfusepath{clip}%
\pgfsetbuttcap%
\pgfsetroundjoin%
\definecolor{currentfill}{rgb}{0.268510,0.009605,0.335427}%
\pgfsetfillcolor{currentfill}%
\pgfsetfillopacity{0.800000}%
\pgfsetlinewidth{0.000000pt}%
\definecolor{currentstroke}{rgb}{0.000000,0.000000,0.000000}%
\pgfsetstrokecolor{currentstroke}%
\pgfsetdash{}{0pt}%
\pgfpathmoveto{\pgfqpoint{3.383577in}{1.590834in}}%
\pgfpathlineto{\pgfqpoint{3.397329in}{1.586833in}}%
\pgfpathlineto{\pgfqpoint{3.411085in}{1.583033in}}%
\pgfpathlineto{\pgfqpoint{3.424845in}{1.579433in}}%
\pgfpathlineto{\pgfqpoint{3.438609in}{1.576032in}}%
\pgfpathlineto{\pgfqpoint{3.446874in}{1.585182in}}%
\pgfpathlineto{\pgfqpoint{3.455133in}{1.594437in}}%
\pgfpathlineto{\pgfqpoint{3.463384in}{1.603791in}}%
\pgfpathlineto{\pgfqpoint{3.471629in}{1.613240in}}%
\pgfpathlineto{\pgfqpoint{3.457882in}{1.616175in}}%
\pgfpathlineto{\pgfqpoint{3.444139in}{1.619308in}}%
\pgfpathlineto{\pgfqpoint{3.430401in}{1.622641in}}%
\pgfpathlineto{\pgfqpoint{3.416667in}{1.626175in}}%
\pgfpathlineto{\pgfqpoint{3.408406in}{1.617180in}}%
\pgfpathlineto{\pgfqpoint{3.400137in}{1.608288in}}%
\pgfpathlineto{\pgfqpoint{3.391861in}{1.599505in}}%
\pgfpathlineto{\pgfqpoint{3.383577in}{1.590834in}}%
\pgfpathclose%
\pgfusepath{fill}%
\end{pgfscope}%
\begin{pgfscope}%
\pgfpathrectangle{\pgfqpoint{1.150000in}{0.150000in}}{\pgfqpoint{5.700000in}{5.700000in}}%
\pgfusepath{clip}%
\pgfsetbuttcap%
\pgfsetroundjoin%
\definecolor{currentfill}{rgb}{0.175707,0.697900,0.491033}%
\pgfsetfillcolor{currentfill}%
\pgfsetfillopacity{0.800000}%
\pgfsetlinewidth{0.000000pt}%
\definecolor{currentstroke}{rgb}{0.000000,0.000000,0.000000}%
\pgfsetstrokecolor{currentstroke}%
\pgfsetdash{}{0pt}%
\pgfpathmoveto{\pgfqpoint{5.769789in}{3.414079in}}%
\pgfpathlineto{\pgfqpoint{5.784649in}{3.428071in}}%
\pgfpathlineto{\pgfqpoint{5.799531in}{3.442244in}}%
\pgfpathlineto{\pgfqpoint{5.814435in}{3.456597in}}%
\pgfpathlineto{\pgfqpoint{5.829361in}{3.471132in}}%
\pgfpathlineto{\pgfqpoint{5.836524in}{3.472023in}}%
\pgfpathlineto{\pgfqpoint{5.843680in}{3.472906in}}%
\pgfpathlineto{\pgfqpoint{5.850828in}{3.473787in}}%
\pgfpathlineto{\pgfqpoint{5.857970in}{3.474673in}}%
\pgfpathlineto{\pgfqpoint{5.843075in}{3.460708in}}%
\pgfpathlineto{\pgfqpoint{5.828202in}{3.446922in}}%
\pgfpathlineto{\pgfqpoint{5.813351in}{3.433316in}}%
\pgfpathlineto{\pgfqpoint{5.798522in}{3.419890in}}%
\pgfpathlineto{\pgfqpoint{5.791349in}{3.418425in}}%
\pgfpathlineto{\pgfqpoint{5.784169in}{3.416973in}}%
\pgfpathlineto{\pgfqpoint{5.776982in}{3.415526in}}%
\pgfpathlineto{\pgfqpoint{5.769789in}{3.414079in}}%
\pgfpathclose%
\pgfusepath{fill}%
\end{pgfscope}%
\begin{pgfscope}%
\pgfpathrectangle{\pgfqpoint{1.150000in}{0.150000in}}{\pgfqpoint{5.700000in}{5.700000in}}%
\pgfusepath{clip}%
\pgfsetbuttcap%
\pgfsetroundjoin%
\definecolor{currentfill}{rgb}{0.280255,0.165693,0.476498}%
\pgfsetfillcolor{currentfill}%
\pgfsetfillopacity{0.800000}%
\pgfsetlinewidth{0.000000pt}%
\definecolor{currentstroke}{rgb}{0.000000,0.000000,0.000000}%
\pgfsetstrokecolor{currentstroke}%
\pgfsetdash{}{0pt}%
\pgfpathmoveto{\pgfqpoint{3.997960in}{1.897990in}}%
\pgfpathlineto{\pgfqpoint{4.011839in}{1.902008in}}%
\pgfpathlineto{\pgfqpoint{4.025729in}{1.906213in}}%
\pgfpathlineto{\pgfqpoint{4.039629in}{1.910605in}}%
\pgfpathlineto{\pgfqpoint{4.053541in}{1.915184in}}%
\pgfpathlineto{\pgfqpoint{4.061585in}{1.927653in}}%
\pgfpathlineto{\pgfqpoint{4.069625in}{1.940078in}}%
\pgfpathlineto{\pgfqpoint{4.077660in}{1.952455in}}%
\pgfpathlineto{\pgfqpoint{4.085691in}{1.964784in}}%
\pgfpathlineto{\pgfqpoint{4.071783in}{1.959957in}}%
\pgfpathlineto{\pgfqpoint{4.057887in}{1.955317in}}%
\pgfpathlineto{\pgfqpoint{4.044002in}{1.950864in}}%
\pgfpathlineto{\pgfqpoint{4.030127in}{1.946599in}}%
\pgfpathlineto{\pgfqpoint{4.022092in}{1.934506in}}%
\pgfpathlineto{\pgfqpoint{4.014053in}{1.922372in}}%
\pgfpathlineto{\pgfqpoint{4.006009in}{1.910200in}}%
\pgfpathlineto{\pgfqpoint{3.997960in}{1.897990in}}%
\pgfpathclose%
\pgfusepath{fill}%
\end{pgfscope}%
\begin{pgfscope}%
\pgfpathrectangle{\pgfqpoint{1.150000in}{0.150000in}}{\pgfqpoint{5.700000in}{5.700000in}}%
\pgfusepath{clip}%
\pgfsetbuttcap%
\pgfsetroundjoin%
\definecolor{currentfill}{rgb}{0.183898,0.422383,0.556944}%
\pgfsetfillcolor{currentfill}%
\pgfsetfillopacity{0.800000}%
\pgfsetlinewidth{0.000000pt}%
\definecolor{currentstroke}{rgb}{0.000000,0.000000,0.000000}%
\pgfsetstrokecolor{currentstroke}%
\pgfsetdash{}{0pt}%
\pgfpathmoveto{\pgfqpoint{4.708324in}{2.555958in}}%
\pgfpathlineto{\pgfqpoint{4.722548in}{2.566194in}}%
\pgfpathlineto{\pgfqpoint{4.736788in}{2.576615in}}%
\pgfpathlineto{\pgfqpoint{4.751045in}{2.587219in}}%
\pgfpathlineto{\pgfqpoint{4.765319in}{2.598009in}}%
\pgfpathlineto{\pgfqpoint{4.773119in}{2.607394in}}%
\pgfpathlineto{\pgfqpoint{4.780911in}{2.616657in}}%
\pgfpathlineto{\pgfqpoint{4.788697in}{2.625799in}}%
\pgfpathlineto{\pgfqpoint{4.796476in}{2.634822in}}%
\pgfpathlineto{\pgfqpoint{4.782208in}{2.624113in}}%
\pgfpathlineto{\pgfqpoint{4.767957in}{2.613589in}}%
\pgfpathlineto{\pgfqpoint{4.753722in}{2.603249in}}%
\pgfpathlineto{\pgfqpoint{4.739503in}{2.593093in}}%
\pgfpathlineto{\pgfqpoint{4.731719in}{2.583978in}}%
\pgfpathlineto{\pgfqpoint{4.723927in}{2.574751in}}%
\pgfpathlineto{\pgfqpoint{4.716129in}{2.565411in}}%
\pgfpathlineto{\pgfqpoint{4.708324in}{2.555958in}}%
\pgfpathclose%
\pgfusepath{fill}%
\end{pgfscope}%
\begin{pgfscope}%
\pgfpathrectangle{\pgfqpoint{1.150000in}{0.150000in}}{\pgfqpoint{5.700000in}{5.700000in}}%
\pgfusepath{clip}%
\pgfsetbuttcap%
\pgfsetroundjoin%
\definecolor{currentfill}{rgb}{0.221989,0.339161,0.548752}%
\pgfsetfillcolor{currentfill}%
\pgfsetfillopacity{0.800000}%
\pgfsetlinewidth{0.000000pt}%
\definecolor{currentstroke}{rgb}{0.000000,0.000000,0.000000}%
\pgfsetstrokecolor{currentstroke}%
\pgfsetdash{}{0pt}%
\pgfpathmoveto{\pgfqpoint{2.339271in}{2.411334in}}%
\pgfpathlineto{\pgfqpoint{2.353401in}{2.389051in}}%
\pgfpathlineto{\pgfqpoint{2.367517in}{2.367079in}}%
\pgfpathlineto{\pgfqpoint{2.381620in}{2.345416in}}%
\pgfpathlineto{\pgfqpoint{2.395711in}{2.324059in}}%
\pgfpathlineto{\pgfqpoint{2.404712in}{2.321564in}}%
\pgfpathlineto{\pgfqpoint{2.413693in}{2.319381in}}%
\pgfpathlineto{\pgfqpoint{2.422654in}{2.317502in}}%
\pgfpathlineto{\pgfqpoint{2.431596in}{2.315923in}}%
\pgfpathlineto{\pgfqpoint{2.417559in}{2.336676in}}%
\pgfpathlineto{\pgfqpoint{2.403509in}{2.357734in}}%
\pgfpathlineto{\pgfqpoint{2.389447in}{2.379100in}}%
\pgfpathlineto{\pgfqpoint{2.375372in}{2.400775in}}%
\pgfpathlineto{\pgfqpoint{2.366378in}{2.402946in}}%
\pgfpathlineto{\pgfqpoint{2.357363in}{2.405426in}}%
\pgfpathlineto{\pgfqpoint{2.348328in}{2.408220in}}%
\pgfpathlineto{\pgfqpoint{2.339271in}{2.411334in}}%
\pgfpathclose%
\pgfusepath{fill}%
\end{pgfscope}%
\begin{pgfscope}%
\pgfpathrectangle{\pgfqpoint{1.150000in}{0.150000in}}{\pgfqpoint{5.700000in}{5.700000in}}%
\pgfusepath{clip}%
\pgfsetbuttcap%
\pgfsetroundjoin%
\definecolor{currentfill}{rgb}{0.165117,0.467423,0.558141}%
\pgfsetfillcolor{currentfill}%
\pgfsetfillopacity{0.800000}%
\pgfsetlinewidth{0.000000pt}%
\definecolor{currentstroke}{rgb}{0.000000,0.000000,0.000000}%
\pgfsetstrokecolor{currentstroke}%
\pgfsetdash{}{0pt}%
\pgfpathmoveto{\pgfqpoint{2.148165in}{2.792272in}}%
\pgfpathlineto{\pgfqpoint{2.162491in}{2.765206in}}%
\pgfpathlineto{\pgfqpoint{2.176798in}{2.738504in}}%
\pgfpathlineto{\pgfqpoint{2.191087in}{2.712160in}}%
\pgfpathlineto{\pgfqpoint{2.205359in}{2.686172in}}%
\pgfpathlineto{\pgfqpoint{2.214502in}{2.682523in}}%
\pgfpathlineto{\pgfqpoint{2.223623in}{2.679199in}}%
\pgfpathlineto{\pgfqpoint{2.232722in}{2.676194in}}%
\pgfpathlineto{\pgfqpoint{2.241800in}{2.673504in}}%
\pgfpathlineto{\pgfqpoint{2.227587in}{2.698903in}}%
\pgfpathlineto{\pgfqpoint{2.213357in}{2.724656in}}%
\pgfpathlineto{\pgfqpoint{2.199109in}{2.750766in}}%
\pgfpathlineto{\pgfqpoint{2.184844in}{2.777237in}}%
\pgfpathlineto{\pgfqpoint{2.175708in}{2.780505in}}%
\pgfpathlineto{\pgfqpoint{2.166549in}{2.784096in}}%
\pgfpathlineto{\pgfqpoint{2.157368in}{2.788017in}}%
\pgfpathlineto{\pgfqpoint{2.148165in}{2.792272in}}%
\pgfpathclose%
\pgfusepath{fill}%
\end{pgfscope}%
\begin{pgfscope}%
\pgfpathrectangle{\pgfqpoint{1.150000in}{0.150000in}}{\pgfqpoint{5.700000in}{5.700000in}}%
\pgfusepath{clip}%
\pgfsetbuttcap%
\pgfsetroundjoin%
\definecolor{currentfill}{rgb}{0.280267,0.073417,0.397163}%
\pgfsetfillcolor{currentfill}%
\pgfsetfillopacity{0.800000}%
\pgfsetlinewidth{0.000000pt}%
\definecolor{currentstroke}{rgb}{0.000000,0.000000,0.000000}%
\pgfsetstrokecolor{currentstroke}%
\pgfsetdash{}{0pt}%
\pgfpathmoveto{\pgfqpoint{2.896989in}{1.739104in}}%
\pgfpathlineto{\pgfqpoint{2.910807in}{1.727501in}}%
\pgfpathlineto{\pgfqpoint{2.924623in}{1.716128in}}%
\pgfpathlineto{\pgfqpoint{2.938436in}{1.704984in}}%
\pgfpathlineto{\pgfqpoint{2.952247in}{1.694068in}}%
\pgfpathlineto{\pgfqpoint{2.960809in}{1.697541in}}%
\pgfpathlineto{\pgfqpoint{2.969359in}{1.701238in}}%
\pgfpathlineto{\pgfqpoint{2.977896in}{1.705151in}}%
\pgfpathlineto{\pgfqpoint{2.986420in}{1.709275in}}%
\pgfpathlineto{\pgfqpoint{2.972643in}{1.719622in}}%
\pgfpathlineto{\pgfqpoint{2.958864in}{1.730195in}}%
\pgfpathlineto{\pgfqpoint{2.945083in}{1.740998in}}%
\pgfpathlineto{\pgfqpoint{2.931299in}{1.752029in}}%
\pgfpathlineto{\pgfqpoint{2.922741in}{1.748463in}}%
\pgfpathlineto{\pgfqpoint{2.914170in}{1.745116in}}%
\pgfpathlineto{\pgfqpoint{2.905586in}{1.741995in}}%
\pgfpathlineto{\pgfqpoint{2.896989in}{1.739104in}}%
\pgfpathclose%
\pgfusepath{fill}%
\end{pgfscope}%
\begin{pgfscope}%
\pgfpathrectangle{\pgfqpoint{1.150000in}{0.150000in}}{\pgfqpoint{5.700000in}{5.700000in}}%
\pgfusepath{clip}%
\pgfsetbuttcap%
\pgfsetroundjoin%
\definecolor{currentfill}{rgb}{0.121831,0.589055,0.545623}%
\pgfsetfillcolor{currentfill}%
\pgfsetfillopacity{0.800000}%
\pgfsetlinewidth{0.000000pt}%
\definecolor{currentstroke}{rgb}{0.000000,0.000000,0.000000}%
\pgfsetstrokecolor{currentstroke}%
\pgfsetdash{}{0pt}%
\pgfpathmoveto{\pgfqpoint{5.299058in}{3.069363in}}%
\pgfpathlineto{\pgfqpoint{5.313637in}{3.082394in}}%
\pgfpathlineto{\pgfqpoint{5.328236in}{3.095608in}}%
\pgfpathlineto{\pgfqpoint{5.342855in}{3.109005in}}%
\pgfpathlineto{\pgfqpoint{5.357494in}{3.122584in}}%
\pgfpathlineto{\pgfqpoint{5.364976in}{3.127065in}}%
\pgfpathlineto{\pgfqpoint{5.372450in}{3.131456in}}%
\pgfpathlineto{\pgfqpoint{5.379914in}{3.135763in}}%
\pgfpathlineto{\pgfqpoint{5.387371in}{3.139989in}}%
\pgfpathlineto{\pgfqpoint{5.372749in}{3.126768in}}%
\pgfpathlineto{\pgfqpoint{5.358147in}{3.113729in}}%
\pgfpathlineto{\pgfqpoint{5.343565in}{3.100873in}}%
\pgfpathlineto{\pgfqpoint{5.329003in}{3.088198in}}%
\pgfpathlineto{\pgfqpoint{5.321529in}{3.083602in}}%
\pgfpathlineto{\pgfqpoint{5.314047in}{3.078934in}}%
\pgfpathlineto{\pgfqpoint{5.306556in}{3.074189in}}%
\pgfpathlineto{\pgfqpoint{5.299058in}{3.069363in}}%
\pgfpathclose%
\pgfusepath{fill}%
\end{pgfscope}%
\begin{pgfscope}%
\pgfpathrectangle{\pgfqpoint{1.150000in}{0.150000in}}{\pgfqpoint{5.700000in}{5.700000in}}%
\pgfusepath{clip}%
\pgfsetbuttcap%
\pgfsetroundjoin%
\definecolor{currentfill}{rgb}{0.216210,0.351535,0.550627}%
\pgfsetfillcolor{currentfill}%
\pgfsetfillopacity{0.800000}%
\pgfsetlinewidth{0.000000pt}%
\definecolor{currentstroke}{rgb}{0.000000,0.000000,0.000000}%
\pgfsetstrokecolor{currentstroke}%
\pgfsetdash{}{0pt}%
\pgfpathmoveto{\pgfqpoint{4.500835in}{2.357731in}}%
\pgfpathlineto{\pgfqpoint{4.514946in}{2.366529in}}%
\pgfpathlineto{\pgfqpoint{4.529073in}{2.375513in}}%
\pgfpathlineto{\pgfqpoint{4.543214in}{2.384681in}}%
\pgfpathlineto{\pgfqpoint{4.557370in}{2.394034in}}%
\pgfpathlineto{\pgfqpoint{4.565256in}{2.404888in}}%
\pgfpathlineto{\pgfqpoint{4.573136in}{2.415628in}}%
\pgfpathlineto{\pgfqpoint{4.581009in}{2.426254in}}%
\pgfpathlineto{\pgfqpoint{4.588876in}{2.436767in}}%
\pgfpathlineto{\pgfqpoint{4.574723in}{2.427394in}}%
\pgfpathlineto{\pgfqpoint{4.560585in}{2.418206in}}%
\pgfpathlineto{\pgfqpoint{4.546462in}{2.409202in}}%
\pgfpathlineto{\pgfqpoint{4.532354in}{2.400383in}}%
\pgfpathlineto{\pgfqpoint{4.524483in}{2.389878in}}%
\pgfpathlineto{\pgfqpoint{4.516606in}{2.379268in}}%
\pgfpathlineto{\pgfqpoint{4.508723in}{2.368552in}}%
\pgfpathlineto{\pgfqpoint{4.500835in}{2.357731in}}%
\pgfpathclose%
\pgfusepath{fill}%
\end{pgfscope}%
\begin{pgfscope}%
\pgfpathrectangle{\pgfqpoint{1.150000in}{0.150000in}}{\pgfqpoint{5.700000in}{5.700000in}}%
\pgfusepath{clip}%
\pgfsetbuttcap%
\pgfsetroundjoin%
\definecolor{currentfill}{rgb}{0.202219,0.715272,0.476084}%
\pgfsetfillcolor{currentfill}%
\pgfsetfillopacity{0.800000}%
\pgfsetlinewidth{0.000000pt}%
\definecolor{currentstroke}{rgb}{0.000000,0.000000,0.000000}%
\pgfsetstrokecolor{currentstroke}%
\pgfsetdash{}{0pt}%
\pgfpathmoveto{\pgfqpoint{5.857970in}{3.474673in}}%
\pgfpathlineto{\pgfqpoint{5.872887in}{3.488819in}}%
\pgfpathlineto{\pgfqpoint{5.887826in}{3.503145in}}%
\pgfpathlineto{\pgfqpoint{5.902788in}{3.517651in}}%
\pgfpathlineto{\pgfqpoint{5.917773in}{3.532339in}}%
\pgfpathlineto{\pgfqpoint{5.924874in}{3.532644in}}%
\pgfpathlineto{\pgfqpoint{5.931969in}{3.532958in}}%
\pgfpathlineto{\pgfqpoint{5.939057in}{3.533288in}}%
\pgfpathlineto{\pgfqpoint{5.946138in}{3.533641in}}%
\pgfpathlineto{\pgfqpoint{5.931188in}{3.519558in}}%
\pgfpathlineto{\pgfqpoint{5.916260in}{3.505655in}}%
\pgfpathlineto{\pgfqpoint{5.901354in}{3.491931in}}%
\pgfpathlineto{\pgfqpoint{5.886470in}{3.478387in}}%
\pgfpathlineto{\pgfqpoint{5.879355in}{3.477420in}}%
\pgfpathlineto{\pgfqpoint{5.872233in}{3.476483in}}%
\pgfpathlineto{\pgfqpoint{5.865105in}{3.475570in}}%
\pgfpathlineto{\pgfqpoint{5.857970in}{3.474673in}}%
\pgfpathclose%
\pgfusepath{fill}%
\end{pgfscope}%
\begin{pgfscope}%
\pgfpathrectangle{\pgfqpoint{1.150000in}{0.150000in}}{\pgfqpoint{5.700000in}{5.700000in}}%
\pgfusepath{clip}%
\pgfsetbuttcap%
\pgfsetroundjoin%
\definecolor{currentfill}{rgb}{0.248629,0.278775,0.534556}%
\pgfsetfillcolor{currentfill}%
\pgfsetfillopacity{0.800000}%
\pgfsetlinewidth{0.000000pt}%
\definecolor{currentstroke}{rgb}{0.000000,0.000000,0.000000}%
\pgfsetstrokecolor{currentstroke}%
\pgfsetdash{}{0pt}%
\pgfpathmoveto{\pgfqpoint{4.293284in}{2.158046in}}%
\pgfpathlineto{\pgfqpoint{4.307292in}{2.165102in}}%
\pgfpathlineto{\pgfqpoint{4.321313in}{2.172344in}}%
\pgfpathlineto{\pgfqpoint{4.335347in}{2.179771in}}%
\pgfpathlineto{\pgfqpoint{4.349395in}{2.187384in}}%
\pgfpathlineto{\pgfqpoint{4.357353in}{2.199325in}}%
\pgfpathlineto{\pgfqpoint{4.365306in}{2.211172in}}%
\pgfpathlineto{\pgfqpoint{4.373253in}{2.222926in}}%
\pgfpathlineto{\pgfqpoint{4.381195in}{2.234585in}}%
\pgfpathlineto{\pgfqpoint{4.367150in}{2.226853in}}%
\pgfpathlineto{\pgfqpoint{4.353118in}{2.219306in}}%
\pgfpathlineto{\pgfqpoint{4.339100in}{2.211945in}}%
\pgfpathlineto{\pgfqpoint{4.325094in}{2.204769in}}%
\pgfpathlineto{\pgfqpoint{4.317150in}{2.193217in}}%
\pgfpathlineto{\pgfqpoint{4.309200in}{2.181579in}}%
\pgfpathlineto{\pgfqpoint{4.301245in}{2.169855in}}%
\pgfpathlineto{\pgfqpoint{4.293284in}{2.158046in}}%
\pgfpathclose%
\pgfusepath{fill}%
\end{pgfscope}%
\begin{pgfscope}%
\pgfpathrectangle{\pgfqpoint{1.150000in}{0.150000in}}{\pgfqpoint{5.700000in}{5.700000in}}%
\pgfusepath{clip}%
\pgfsetbuttcap%
\pgfsetroundjoin%
\definecolor{currentfill}{rgb}{0.149039,0.508051,0.557250}%
\pgfsetfillcolor{currentfill}%
\pgfsetfillopacity{0.800000}%
\pgfsetlinewidth{0.000000pt}%
\definecolor{currentstroke}{rgb}{0.000000,0.000000,0.000000}%
\pgfsetstrokecolor{currentstroke}%
\pgfsetdash{}{0pt}%
\pgfpathmoveto{\pgfqpoint{5.003848in}{2.823078in}}%
\pgfpathlineto{\pgfqpoint{5.018251in}{2.834993in}}%
\pgfpathlineto{\pgfqpoint{5.032672in}{2.847092in}}%
\pgfpathlineto{\pgfqpoint{5.047111in}{2.859375in}}%
\pgfpathlineto{\pgfqpoint{5.061568in}{2.871842in}}%
\pgfpathlineto{\pgfqpoint{5.069225in}{2.878856in}}%
\pgfpathlineto{\pgfqpoint{5.076874in}{2.885754in}}%
\pgfpathlineto{\pgfqpoint{5.084515in}{2.892537in}}%
\pgfpathlineto{\pgfqpoint{5.092148in}{2.899208in}}%
\pgfpathlineto{\pgfqpoint{5.077700in}{2.886961in}}%
\pgfpathlineto{\pgfqpoint{5.063272in}{2.874896in}}%
\pgfpathlineto{\pgfqpoint{5.048861in}{2.863015in}}%
\pgfpathlineto{\pgfqpoint{5.034469in}{2.851316in}}%
\pgfpathlineto{\pgfqpoint{5.026826in}{2.844414in}}%
\pgfpathlineto{\pgfqpoint{5.019174in}{2.837409in}}%
\pgfpathlineto{\pgfqpoint{5.011515in}{2.830298in}}%
\pgfpathlineto{\pgfqpoint{5.003848in}{2.823078in}}%
\pgfpathclose%
\pgfusepath{fill}%
\end{pgfscope}%
\begin{pgfscope}%
\pgfpathrectangle{\pgfqpoint{1.150000in}{0.150000in}}{\pgfqpoint{5.700000in}{5.700000in}}%
\pgfusepath{clip}%
\pgfsetbuttcap%
\pgfsetroundjoin%
\definecolor{currentfill}{rgb}{0.266941,0.748751,0.440573}%
\pgfsetfillcolor{currentfill}%
\pgfsetfillopacity{0.800000}%
\pgfsetlinewidth{0.000000pt}%
\definecolor{currentstroke}{rgb}{0.000000,0.000000,0.000000}%
\pgfsetstrokecolor{currentstroke}%
\pgfsetdash{}{0pt}%
\pgfpathmoveto{\pgfqpoint{6.034285in}{3.590987in}}%
\pgfpathlineto{\pgfqpoint{6.049312in}{3.605330in}}%
\pgfpathlineto{\pgfqpoint{6.064362in}{3.619852in}}%
\pgfpathlineto{\pgfqpoint{6.079435in}{3.634555in}}%
\pgfpathlineto{\pgfqpoint{6.086422in}{3.633967in}}%
\pgfpathlineto{\pgfqpoint{6.093402in}{3.633429in}}%
\pgfpathlineto{\pgfqpoint{6.100378in}{3.632948in}}%
\pgfpathlineto{\pgfqpoint{6.107349in}{3.632532in}}%
\pgfpathlineto{\pgfqpoint{6.092315in}{3.618503in}}%
\pgfpathlineto{\pgfqpoint{6.077305in}{3.604653in}}%
\pgfpathlineto{\pgfqpoint{6.062317in}{3.590981in}}%
\pgfpathlineto{\pgfqpoint{6.055316in}{3.590885in}}%
\pgfpathlineto{\pgfqpoint{6.048311in}{3.590859in}}%
\pgfpathlineto{\pgfqpoint{6.041301in}{3.590895in}}%
\pgfpathlineto{\pgfqpoint{6.034285in}{3.590987in}}%
\pgfpathclose%
\pgfusepath{fill}%
\end{pgfscope}%
\begin{pgfscope}%
\pgfpathrectangle{\pgfqpoint{1.150000in}{0.150000in}}{\pgfqpoint{5.700000in}{5.700000in}}%
\pgfusepath{clip}%
\pgfsetbuttcap%
\pgfsetroundjoin%
\definecolor{currentfill}{rgb}{0.274128,0.199721,0.498911}%
\pgfsetfillcolor{currentfill}%
\pgfsetfillopacity{0.800000}%
\pgfsetlinewidth{0.000000pt}%
\definecolor{currentstroke}{rgb}{0.000000,0.000000,0.000000}%
\pgfsetstrokecolor{currentstroke}%
\pgfsetdash{}{0pt}%
\pgfpathmoveto{\pgfqpoint{4.085691in}{1.964784in}}%
\pgfpathlineto{\pgfqpoint{4.099609in}{1.969799in}}%
\pgfpathlineto{\pgfqpoint{4.113539in}{1.975000in}}%
\pgfpathlineto{\pgfqpoint{4.127481in}{1.980387in}}%
\pgfpathlineto{\pgfqpoint{4.141434in}{1.985961in}}%
\pgfpathlineto{\pgfqpoint{4.149456in}{1.998467in}}%
\pgfpathlineto{\pgfqpoint{4.157474in}{2.010913in}}%
\pgfpathlineto{\pgfqpoint{4.165487in}{2.023297in}}%
\pgfpathlineto{\pgfqpoint{4.173495in}{2.035616in}}%
\pgfpathlineto{\pgfqpoint{4.159545in}{2.029826in}}%
\pgfpathlineto{\pgfqpoint{4.145606in}{2.024222in}}%
\pgfpathlineto{\pgfqpoint{4.131680in}{2.018805in}}%
\pgfpathlineto{\pgfqpoint{4.117765in}{2.013575in}}%
\pgfpathlineto{\pgfqpoint{4.109753in}{2.001460in}}%
\pgfpathlineto{\pgfqpoint{4.101737in}{1.989288in}}%
\pgfpathlineto{\pgfqpoint{4.093716in}{1.977063in}}%
\pgfpathlineto{\pgfqpoint{4.085691in}{1.964784in}}%
\pgfpathclose%
\pgfusepath{fill}%
\end{pgfscope}%
\begin{pgfscope}%
\pgfpathrectangle{\pgfqpoint{1.150000in}{0.150000in}}{\pgfqpoint{5.700000in}{5.700000in}}%
\pgfusepath{clip}%
\pgfsetbuttcap%
\pgfsetroundjoin%
\definecolor{currentfill}{rgb}{0.239374,0.735588,0.455688}%
\pgfsetfillcolor{currentfill}%
\pgfsetfillopacity{0.800000}%
\pgfsetlinewidth{0.000000pt}%
\definecolor{currentstroke}{rgb}{0.000000,0.000000,0.000000}%
\pgfsetstrokecolor{currentstroke}%
\pgfsetdash{}{0pt}%
\pgfpathmoveto{\pgfqpoint{5.946138in}{3.533641in}}%
\pgfpathlineto{\pgfqpoint{5.961111in}{3.547903in}}%
\pgfpathlineto{\pgfqpoint{5.976106in}{3.562346in}}%
\pgfpathlineto{\pgfqpoint{5.991124in}{3.576969in}}%
\pgfpathlineto{\pgfqpoint{6.006166in}{3.591773in}}%
\pgfpathlineto{\pgfqpoint{6.013205in}{3.591527in}}%
\pgfpathlineto{\pgfqpoint{6.020238in}{3.591310in}}%
\pgfpathlineto{\pgfqpoint{6.027264in}{3.591128in}}%
\pgfpathlineto{\pgfqpoint{6.034285in}{3.590987in}}%
\pgfpathlineto{\pgfqpoint{6.019281in}{3.576823in}}%
\pgfpathlineto{\pgfqpoint{6.004300in}{3.562839in}}%
\pgfpathlineto{\pgfqpoint{5.989341in}{3.549034in}}%
\pgfpathlineto{\pgfqpoint{5.974404in}{3.535408in}}%
\pgfpathlineto{\pgfqpoint{5.967346in}{3.534899in}}%
\pgfpathlineto{\pgfqpoint{5.960282in}{3.534440in}}%
\pgfpathlineto{\pgfqpoint{5.953213in}{3.534022in}}%
\pgfpathlineto{\pgfqpoint{5.946138in}{3.533641in}}%
\pgfpathclose%
\pgfusepath{fill}%
\end{pgfscope}%
\begin{pgfscope}%
\pgfpathrectangle{\pgfqpoint{1.150000in}{0.150000in}}{\pgfqpoint{5.700000in}{5.700000in}}%
\pgfusepath{clip}%
\pgfsetbuttcap%
\pgfsetroundjoin%
\definecolor{currentfill}{rgb}{0.204903,0.375746,0.553533}%
\pgfsetfillcolor{currentfill}%
\pgfsetfillopacity{0.800000}%
\pgfsetlinewidth{0.000000pt}%
\definecolor{currentstroke}{rgb}{0.000000,0.000000,0.000000}%
\pgfsetstrokecolor{currentstroke}%
\pgfsetdash{}{0pt}%
\pgfpathmoveto{\pgfqpoint{2.282614in}{2.503641in}}%
\pgfpathlineto{\pgfqpoint{2.296800in}{2.480082in}}%
\pgfpathlineto{\pgfqpoint{2.310971in}{2.456847in}}%
\pgfpathlineto{\pgfqpoint{2.325128in}{2.433932in}}%
\pgfpathlineto{\pgfqpoint{2.339271in}{2.411334in}}%
\pgfpathlineto{\pgfqpoint{2.348328in}{2.408220in}}%
\pgfpathlineto{\pgfqpoint{2.357363in}{2.405426in}}%
\pgfpathlineto{\pgfqpoint{2.366378in}{2.402946in}}%
\pgfpathlineto{\pgfqpoint{2.375372in}{2.400775in}}%
\pgfpathlineto{\pgfqpoint{2.361284in}{2.422764in}}%
\pgfpathlineto{\pgfqpoint{2.347183in}{2.445068in}}%
\pgfpathlineto{\pgfqpoint{2.333068in}{2.467692in}}%
\pgfpathlineto{\pgfqpoint{2.318939in}{2.490637in}}%
\pgfpathlineto{\pgfqpoint{2.309890in}{2.493405in}}%
\pgfpathlineto{\pgfqpoint{2.300820in}{2.496491in}}%
\pgfpathlineto{\pgfqpoint{2.291728in}{2.499901in}}%
\pgfpathlineto{\pgfqpoint{2.282614in}{2.503641in}}%
\pgfpathclose%
\pgfusepath{fill}%
\end{pgfscope}%
\begin{pgfscope}%
\pgfpathrectangle{\pgfqpoint{1.150000in}{0.150000in}}{\pgfqpoint{5.700000in}{5.700000in}}%
\pgfusepath{clip}%
\pgfsetbuttcap%
\pgfsetroundjoin%
\definecolor{currentfill}{rgb}{0.279566,0.067836,0.391917}%
\pgfsetfillcolor{currentfill}%
\pgfsetfillopacity{0.800000}%
\pgfsetlinewidth{0.000000pt}%
\definecolor{currentstroke}{rgb}{0.000000,0.000000,0.000000}%
\pgfsetstrokecolor{currentstroke}%
\pgfsetdash{}{0pt}%
\pgfpathmoveto{\pgfqpoint{3.702407in}{1.681941in}}%
\pgfpathlineto{\pgfqpoint{3.716207in}{1.682380in}}%
\pgfpathlineto{\pgfqpoint{3.730013in}{1.683010in}}%
\pgfpathlineto{\pgfqpoint{3.743828in}{1.683831in}}%
\pgfpathlineto{\pgfqpoint{3.757650in}{1.684843in}}%
\pgfpathlineto{\pgfqpoint{3.765791in}{1.696506in}}%
\pgfpathlineto{\pgfqpoint{3.773926in}{1.708194in}}%
\pgfpathlineto{\pgfqpoint{3.782056in}{1.719903in}}%
\pgfpathlineto{\pgfqpoint{3.790181in}{1.731630in}}%
\pgfpathlineto{\pgfqpoint{3.776367in}{1.730245in}}%
\pgfpathlineto{\pgfqpoint{3.762562in}{1.729051in}}%
\pgfpathlineto{\pgfqpoint{3.748764in}{1.728048in}}%
\pgfpathlineto{\pgfqpoint{3.734974in}{1.727236in}}%
\pgfpathlineto{\pgfqpoint{3.726841in}{1.715870in}}%
\pgfpathlineto{\pgfqpoint{3.718702in}{1.704530in}}%
\pgfpathlineto{\pgfqpoint{3.710557in}{1.693219in}}%
\pgfpathlineto{\pgfqpoint{3.702407in}{1.681941in}}%
\pgfpathclose%
\pgfusepath{fill}%
\end{pgfscope}%
\begin{pgfscope}%
\pgfpathrectangle{\pgfqpoint{1.150000in}{0.150000in}}{\pgfqpoint{5.700000in}{5.700000in}}%
\pgfusepath{clip}%
\pgfsetbuttcap%
\pgfsetroundjoin%
\definecolor{currentfill}{rgb}{0.277941,0.056324,0.381191}%
\pgfsetfillcolor{currentfill}%
\pgfsetfillopacity{0.800000}%
\pgfsetlinewidth{0.000000pt}%
\definecolor{currentstroke}{rgb}{0.000000,0.000000,0.000000}%
\pgfsetstrokecolor{currentstroke}%
\pgfsetdash{}{0pt}%
\pgfpathmoveto{\pgfqpoint{2.952247in}{1.694068in}}%
\pgfpathlineto{\pgfqpoint{2.966056in}{1.683377in}}%
\pgfpathlineto{\pgfqpoint{2.979863in}{1.672912in}}%
\pgfpathlineto{\pgfqpoint{2.993668in}{1.662671in}}%
\pgfpathlineto{\pgfqpoint{3.007471in}{1.652651in}}%
\pgfpathlineto{\pgfqpoint{3.016000in}{1.656707in}}%
\pgfpathlineto{\pgfqpoint{3.024516in}{1.660976in}}%
\pgfpathlineto{\pgfqpoint{3.033021in}{1.665454in}}%
\pgfpathlineto{\pgfqpoint{3.041514in}{1.670134in}}%
\pgfpathlineto{\pgfqpoint{3.027742in}{1.679585in}}%
\pgfpathlineto{\pgfqpoint{3.013970in}{1.689258in}}%
\pgfpathlineto{\pgfqpoint{3.000196in}{1.699154in}}%
\pgfpathlineto{\pgfqpoint{2.986420in}{1.709275in}}%
\pgfpathlineto{\pgfqpoint{2.977896in}{1.705151in}}%
\pgfpathlineto{\pgfqpoint{2.969359in}{1.701238in}}%
\pgfpathlineto{\pgfqpoint{2.960809in}{1.697541in}}%
\pgfpathlineto{\pgfqpoint{2.952247in}{1.694068in}}%
\pgfpathclose%
\pgfusepath{fill}%
\end{pgfscope}%
\begin{pgfscope}%
\pgfpathrectangle{\pgfqpoint{1.150000in}{0.150000in}}{\pgfqpoint{5.700000in}{5.700000in}}%
\pgfusepath{clip}%
\pgfsetbuttcap%
\pgfsetroundjoin%
\definecolor{currentfill}{rgb}{0.276022,0.044167,0.370164}%
\pgfsetfillcolor{currentfill}%
\pgfsetfillopacity{0.800000}%
\pgfsetlinewidth{0.000000pt}%
\definecolor{currentstroke}{rgb}{0.000000,0.000000,0.000000}%
\pgfsetstrokecolor{currentstroke}%
\pgfsetdash{}{0pt}%
\pgfpathmoveto{\pgfqpoint{3.614582in}{1.639009in}}%
\pgfpathlineto{\pgfqpoint{3.628365in}{1.638274in}}%
\pgfpathlineto{\pgfqpoint{3.642154in}{1.637732in}}%
\pgfpathlineto{\pgfqpoint{3.655950in}{1.637383in}}%
\pgfpathlineto{\pgfqpoint{3.669752in}{1.637226in}}%
\pgfpathlineto{\pgfqpoint{3.677925in}{1.648337in}}%
\pgfpathlineto{\pgfqpoint{3.686091in}{1.659496in}}%
\pgfpathlineto{\pgfqpoint{3.694252in}{1.670698in}}%
\pgfpathlineto{\pgfqpoint{3.702407in}{1.681941in}}%
\pgfpathlineto{\pgfqpoint{3.688615in}{1.681693in}}%
\pgfpathlineto{\pgfqpoint{3.674831in}{1.681638in}}%
\pgfpathlineto{\pgfqpoint{3.661053in}{1.681776in}}%
\pgfpathlineto{\pgfqpoint{3.647282in}{1.682107in}}%
\pgfpathlineto{\pgfqpoint{3.639116in}{1.671257in}}%
\pgfpathlineto{\pgfqpoint{3.630944in}{1.660455in}}%
\pgfpathlineto{\pgfqpoint{3.622766in}{1.649705in}}%
\pgfpathlineto{\pgfqpoint{3.614582in}{1.639009in}}%
\pgfpathclose%
\pgfusepath{fill}%
\end{pgfscope}%
\begin{pgfscope}%
\pgfpathrectangle{\pgfqpoint{1.150000in}{0.150000in}}{\pgfqpoint{5.700000in}{5.700000in}}%
\pgfusepath{clip}%
\pgfsetbuttcap%
\pgfsetroundjoin%
\definecolor{currentfill}{rgb}{0.119483,0.614817,0.537692}%
\pgfsetfillcolor{currentfill}%
\pgfsetfillopacity{0.800000}%
\pgfsetlinewidth{0.000000pt}%
\definecolor{currentstroke}{rgb}{0.000000,0.000000,0.000000}%
\pgfsetstrokecolor{currentstroke}%
\pgfsetdash{}{0pt}%
\pgfpathmoveto{\pgfqpoint{5.387371in}{3.139989in}}%
\pgfpathlineto{\pgfqpoint{5.402013in}{3.153392in}}%
\pgfpathlineto{\pgfqpoint{5.416675in}{3.166978in}}%
\pgfpathlineto{\pgfqpoint{5.431358in}{3.180746in}}%
\pgfpathlineto{\pgfqpoint{5.446061in}{3.194698in}}%
\pgfpathlineto{\pgfqpoint{5.453490in}{3.198466in}}%
\pgfpathlineto{\pgfqpoint{5.460911in}{3.202154in}}%
\pgfpathlineto{\pgfqpoint{5.468322in}{3.205766in}}%
\pgfpathlineto{\pgfqpoint{5.475726in}{3.209306in}}%
\pgfpathlineto{\pgfqpoint{5.461042in}{3.195750in}}%
\pgfpathlineto{\pgfqpoint{5.446379in}{3.182375in}}%
\pgfpathlineto{\pgfqpoint{5.431736in}{3.169182in}}%
\pgfpathlineto{\pgfqpoint{5.417113in}{3.156171in}}%
\pgfpathlineto{\pgfqpoint{5.409690in}{3.152225in}}%
\pgfpathlineto{\pgfqpoint{5.402258in}{3.148216in}}%
\pgfpathlineto{\pgfqpoint{5.394819in}{3.144138in}}%
\pgfpathlineto{\pgfqpoint{5.387371in}{3.139989in}}%
\pgfpathclose%
\pgfusepath{fill}%
\end{pgfscope}%
\begin{pgfscope}%
\pgfpathrectangle{\pgfqpoint{1.150000in}{0.150000in}}{\pgfqpoint{5.700000in}{5.700000in}}%
\pgfusepath{clip}%
\pgfsetbuttcap%
\pgfsetroundjoin%
\definecolor{currentfill}{rgb}{0.267004,0.004874,0.329415}%
\pgfsetfillcolor{currentfill}%
\pgfsetfillopacity{0.800000}%
\pgfsetlinewidth{0.000000pt}%
\definecolor{currentstroke}{rgb}{0.000000,0.000000,0.000000}%
\pgfsetstrokecolor{currentstroke}%
\pgfsetdash{}{0pt}%
\pgfpathmoveto{\pgfqpoint{3.295307in}{1.577383in}}%
\pgfpathlineto{\pgfqpoint{3.309066in}{1.572073in}}%
\pgfpathlineto{\pgfqpoint{3.322829in}{1.566968in}}%
\pgfpathlineto{\pgfqpoint{3.336594in}{1.562065in}}%
\pgfpathlineto{\pgfqpoint{3.350362in}{1.557365in}}%
\pgfpathlineto{\pgfqpoint{3.358678in}{1.565540in}}%
\pgfpathlineto{\pgfqpoint{3.366986in}{1.573846in}}%
\pgfpathlineto{\pgfqpoint{3.375285in}{1.582279in}}%
\pgfpathlineto{\pgfqpoint{3.383577in}{1.590834in}}%
\pgfpathlineto{\pgfqpoint{3.369829in}{1.595035in}}%
\pgfpathlineto{\pgfqpoint{3.356084in}{1.599439in}}%
\pgfpathlineto{\pgfqpoint{3.342342in}{1.604046in}}%
\pgfpathlineto{\pgfqpoint{3.328603in}{1.608857in}}%
\pgfpathlineto{\pgfqpoint{3.320292in}{1.600789in}}%
\pgfpathlineto{\pgfqpoint{3.311972in}{1.592851in}}%
\pgfpathlineto{\pgfqpoint{3.303644in}{1.585047in}}%
\pgfpathlineto{\pgfqpoint{3.295307in}{1.577383in}}%
\pgfpathclose%
\pgfusepath{fill}%
\end{pgfscope}%
\begin{pgfscope}%
\pgfpathrectangle{\pgfqpoint{1.150000in}{0.150000in}}{\pgfqpoint{5.700000in}{5.700000in}}%
\pgfusepath{clip}%
\pgfsetbuttcap%
\pgfsetroundjoin%
\definecolor{currentfill}{rgb}{0.269944,0.014625,0.341379}%
\pgfsetfillcolor{currentfill}%
\pgfsetfillopacity{0.800000}%
\pgfsetlinewidth{0.000000pt}%
\definecolor{currentstroke}{rgb}{0.000000,0.000000,0.000000}%
\pgfsetstrokecolor{currentstroke}%
\pgfsetdash{}{0pt}%
\pgfpathmoveto{\pgfqpoint{3.151670in}{1.602383in}}%
\pgfpathlineto{\pgfqpoint{3.165441in}{1.594880in}}%
\pgfpathlineto{\pgfqpoint{3.179212in}{1.587588in}}%
\pgfpathlineto{\pgfqpoint{3.192985in}{1.580506in}}%
\pgfpathlineto{\pgfqpoint{3.206759in}{1.573633in}}%
\pgfpathlineto{\pgfqpoint{3.215156in}{1.580147in}}%
\pgfpathlineto{\pgfqpoint{3.223544in}{1.586829in}}%
\pgfpathlineto{\pgfqpoint{3.231922in}{1.593673in}}%
\pgfpathlineto{\pgfqpoint{3.240291in}{1.600674in}}%
\pgfpathlineto{\pgfqpoint{3.226543in}{1.607015in}}%
\pgfpathlineto{\pgfqpoint{3.212795in}{1.613565in}}%
\pgfpathlineto{\pgfqpoint{3.199050in}{1.620325in}}%
\pgfpathlineto{\pgfqpoint{3.185305in}{1.627296in}}%
\pgfpathlineto{\pgfqpoint{3.176911in}{1.620815in}}%
\pgfpathlineto{\pgfqpoint{3.168508in}{1.614499in}}%
\pgfpathlineto{\pgfqpoint{3.160094in}{1.608353in}}%
\pgfpathlineto{\pgfqpoint{3.151670in}{1.602383in}}%
\pgfpathclose%
\pgfusepath{fill}%
\end{pgfscope}%
\begin{pgfscope}%
\pgfpathrectangle{\pgfqpoint{1.150000in}{0.150000in}}{\pgfqpoint{5.700000in}{5.700000in}}%
\pgfusepath{clip}%
\pgfsetbuttcap%
\pgfsetroundjoin%
\definecolor{currentfill}{rgb}{0.171176,0.452530,0.557965}%
\pgfsetfillcolor{currentfill}%
\pgfsetfillopacity{0.800000}%
\pgfsetlinewidth{0.000000pt}%
\definecolor{currentstroke}{rgb}{0.000000,0.000000,0.000000}%
\pgfsetstrokecolor{currentstroke}%
\pgfsetdash{}{0pt}%
\pgfpathmoveto{\pgfqpoint{4.796476in}{2.634822in}}%
\pgfpathlineto{\pgfqpoint{4.810761in}{2.645714in}}%
\pgfpathlineto{\pgfqpoint{4.825063in}{2.656791in}}%
\pgfpathlineto{\pgfqpoint{4.839382in}{2.668052in}}%
\pgfpathlineto{\pgfqpoint{4.853719in}{2.679497in}}%
\pgfpathlineto{\pgfqpoint{4.861484in}{2.688298in}}%
\pgfpathlineto{\pgfqpoint{4.869243in}{2.696975in}}%
\pgfpathlineto{\pgfqpoint{4.876994in}{2.705528in}}%
\pgfpathlineto{\pgfqpoint{4.884737in}{2.713959in}}%
\pgfpathlineto{\pgfqpoint{4.870407in}{2.702629in}}%
\pgfpathlineto{\pgfqpoint{4.856095in}{2.691483in}}%
\pgfpathlineto{\pgfqpoint{4.841799in}{2.680521in}}%
\pgfpathlineto{\pgfqpoint{4.827521in}{2.669743in}}%
\pgfpathlineto{\pgfqpoint{4.819770in}{2.661185in}}%
\pgfpathlineto{\pgfqpoint{4.812013in}{2.652513in}}%
\pgfpathlineto{\pgfqpoint{4.804248in}{2.643726in}}%
\pgfpathlineto{\pgfqpoint{4.796476in}{2.634822in}}%
\pgfpathclose%
\pgfusepath{fill}%
\end{pgfscope}%
\begin{pgfscope}%
\pgfpathrectangle{\pgfqpoint{1.150000in}{0.150000in}}{\pgfqpoint{5.700000in}{5.700000in}}%
\pgfusepath{clip}%
\pgfsetbuttcap%
\pgfsetroundjoin%
\definecolor{currentfill}{rgb}{0.282327,0.094955,0.417331}%
\pgfsetfillcolor{currentfill}%
\pgfsetfillopacity{0.800000}%
\pgfsetlinewidth{0.000000pt}%
\definecolor{currentstroke}{rgb}{0.000000,0.000000,0.000000}%
\pgfsetstrokecolor{currentstroke}%
\pgfsetdash{}{0pt}%
\pgfpathmoveto{\pgfqpoint{3.790181in}{1.731630in}}%
\pgfpathlineto{\pgfqpoint{3.804003in}{1.733206in}}%
\pgfpathlineto{\pgfqpoint{3.817834in}{1.734970in}}%
\pgfpathlineto{\pgfqpoint{3.831673in}{1.736925in}}%
\pgfpathlineto{\pgfqpoint{3.845520in}{1.739068in}}%
\pgfpathlineto{\pgfqpoint{3.853633in}{1.751164in}}%
\pgfpathlineto{\pgfqpoint{3.861740in}{1.763263in}}%
\pgfpathlineto{\pgfqpoint{3.869843in}{1.775363in}}%
\pgfpathlineto{\pgfqpoint{3.877940in}{1.787460in}}%
\pgfpathlineto{\pgfqpoint{3.864099in}{1.784974in}}%
\pgfpathlineto{\pgfqpoint{3.850267in}{1.782678in}}%
\pgfpathlineto{\pgfqpoint{3.836444in}{1.780571in}}%
\pgfpathlineto{\pgfqpoint{3.822630in}{1.778654in}}%
\pgfpathlineto{\pgfqpoint{3.814525in}{1.766887in}}%
\pgfpathlineto{\pgfqpoint{3.806416in}{1.755125in}}%
\pgfpathlineto{\pgfqpoint{3.798301in}{1.743372in}}%
\pgfpathlineto{\pgfqpoint{3.790181in}{1.731630in}}%
\pgfpathclose%
\pgfusepath{fill}%
\end{pgfscope}%
\begin{pgfscope}%
\pgfpathrectangle{\pgfqpoint{1.150000in}{0.150000in}}{\pgfqpoint{5.700000in}{5.700000in}}%
\pgfusepath{clip}%
\pgfsetbuttcap%
\pgfsetroundjoin%
\definecolor{currentfill}{rgb}{0.272594,0.025563,0.353093}%
\pgfsetfillcolor{currentfill}%
\pgfsetfillopacity{0.800000}%
\pgfsetlinewidth{0.000000pt}%
\definecolor{currentstroke}{rgb}{0.000000,0.000000,0.000000}%
\pgfsetstrokecolor{currentstroke}%
\pgfsetdash{}{0pt}%
\pgfpathmoveto{\pgfqpoint{3.526665in}{1.603482in}}%
\pgfpathlineto{\pgfqpoint{3.540437in}{1.601534in}}%
\pgfpathlineto{\pgfqpoint{3.554214in}{1.599781in}}%
\pgfpathlineto{\pgfqpoint{3.567998in}{1.598222in}}%
\pgfpathlineto{\pgfqpoint{3.581787in}{1.596858in}}%
\pgfpathlineto{\pgfqpoint{3.589995in}{1.607293in}}%
\pgfpathlineto{\pgfqpoint{3.598197in}{1.617800in}}%
\pgfpathlineto{\pgfqpoint{3.606393in}{1.628373in}}%
\pgfpathlineto{\pgfqpoint{3.614582in}{1.639009in}}%
\pgfpathlineto{\pgfqpoint{3.600806in}{1.639938in}}%
\pgfpathlineto{\pgfqpoint{3.587036in}{1.641061in}}%
\pgfpathlineto{\pgfqpoint{3.573272in}{1.642379in}}%
\pgfpathlineto{\pgfqpoint{3.559514in}{1.643892in}}%
\pgfpathlineto{\pgfqpoint{3.551311in}{1.633679in}}%
\pgfpathlineto{\pgfqpoint{3.543102in}{1.623537in}}%
\pgfpathlineto{\pgfqpoint{3.534887in}{1.613470in}}%
\pgfpathlineto{\pgfqpoint{3.526665in}{1.603482in}}%
\pgfpathclose%
\pgfusepath{fill}%
\end{pgfscope}%
\begin{pgfscope}%
\pgfpathrectangle{\pgfqpoint{1.150000in}{0.150000in}}{\pgfqpoint{5.700000in}{5.700000in}}%
\pgfusepath{clip}%
\pgfsetbuttcap%
\pgfsetroundjoin%
\definecolor{currentfill}{rgb}{0.201239,0.383670,0.554294}%
\pgfsetfillcolor{currentfill}%
\pgfsetfillopacity{0.800000}%
\pgfsetlinewidth{0.000000pt}%
\definecolor{currentstroke}{rgb}{0.000000,0.000000,0.000000}%
\pgfsetstrokecolor{currentstroke}%
\pgfsetdash{}{0pt}%
\pgfpathmoveto{\pgfqpoint{4.588876in}{2.436767in}}%
\pgfpathlineto{\pgfqpoint{4.603045in}{2.446325in}}%
\pgfpathlineto{\pgfqpoint{4.617230in}{2.456068in}}%
\pgfpathlineto{\pgfqpoint{4.631430in}{2.465995in}}%
\pgfpathlineto{\pgfqpoint{4.645646in}{2.476107in}}%
\pgfpathlineto{\pgfqpoint{4.653503in}{2.486506in}}%
\pgfpathlineto{\pgfqpoint{4.661355in}{2.496785in}}%
\pgfpathlineto{\pgfqpoint{4.669199in}{2.506944in}}%
\pgfpathlineto{\pgfqpoint{4.677037in}{2.516983in}}%
\pgfpathlineto{\pgfqpoint{4.662825in}{2.506884in}}%
\pgfpathlineto{\pgfqpoint{4.648629in}{2.496970in}}%
\pgfpathlineto{\pgfqpoint{4.634448in}{2.487241in}}%
\pgfpathlineto{\pgfqpoint{4.620283in}{2.477696in}}%
\pgfpathlineto{\pgfqpoint{4.612441in}{2.467631in}}%
\pgfpathlineto{\pgfqpoint{4.604592in}{2.457455in}}%
\pgfpathlineto{\pgfqpoint{4.596737in}{2.447168in}}%
\pgfpathlineto{\pgfqpoint{4.588876in}{2.436767in}}%
\pgfpathclose%
\pgfusepath{fill}%
\end{pgfscope}%
\begin{pgfscope}%
\pgfpathrectangle{\pgfqpoint{1.150000in}{0.150000in}}{\pgfqpoint{5.700000in}{5.700000in}}%
\pgfusepath{clip}%
\pgfsetbuttcap%
\pgfsetroundjoin%
\definecolor{currentfill}{rgb}{0.137770,0.537492,0.554906}%
\pgfsetfillcolor{currentfill}%
\pgfsetfillopacity{0.800000}%
\pgfsetlinewidth{0.000000pt}%
\definecolor{currentstroke}{rgb}{0.000000,0.000000,0.000000}%
\pgfsetstrokecolor{currentstroke}%
\pgfsetdash{}{0pt}%
\pgfpathmoveto{\pgfqpoint{5.092148in}{2.899208in}}%
\pgfpathlineto{\pgfqpoint{5.106614in}{2.911640in}}%
\pgfpathlineto{\pgfqpoint{5.121098in}{2.924254in}}%
\pgfpathlineto{\pgfqpoint{5.135602in}{2.937053in}}%
\pgfpathlineto{\pgfqpoint{5.150125in}{2.950035in}}%
\pgfpathlineto{\pgfqpoint{5.157738in}{2.956356in}}%
\pgfpathlineto{\pgfqpoint{5.165343in}{2.962564in}}%
\pgfpathlineto{\pgfqpoint{5.172940in}{2.968660in}}%
\pgfpathlineto{\pgfqpoint{5.180528in}{2.974649in}}%
\pgfpathlineto{\pgfqpoint{5.166017in}{2.961921in}}%
\pgfpathlineto{\pgfqpoint{5.151525in}{2.949376in}}%
\pgfpathlineto{\pgfqpoint{5.137052in}{2.937015in}}%
\pgfpathlineto{\pgfqpoint{5.122598in}{2.924836in}}%
\pgfpathlineto{\pgfqpoint{5.114997in}{2.918582in}}%
\pgfpathlineto{\pgfqpoint{5.107389in}{2.912228in}}%
\pgfpathlineto{\pgfqpoint{5.099772in}{2.905771in}}%
\pgfpathlineto{\pgfqpoint{5.092148in}{2.899208in}}%
\pgfpathclose%
\pgfusepath{fill}%
\end{pgfscope}%
\begin{pgfscope}%
\pgfpathrectangle{\pgfqpoint{1.150000in}{0.150000in}}{\pgfqpoint{5.700000in}{5.700000in}}%
\pgfusepath{clip}%
\pgfsetbuttcap%
\pgfsetroundjoin%
\definecolor{currentfill}{rgb}{0.233603,0.313828,0.543914}%
\pgfsetfillcolor{currentfill}%
\pgfsetfillopacity{0.800000}%
\pgfsetlinewidth{0.000000pt}%
\definecolor{currentstroke}{rgb}{0.000000,0.000000,0.000000}%
\pgfsetstrokecolor{currentstroke}%
\pgfsetdash{}{0pt}%
\pgfpathmoveto{\pgfqpoint{4.381195in}{2.234585in}}%
\pgfpathlineto{\pgfqpoint{4.395255in}{2.242503in}}%
\pgfpathlineto{\pgfqpoint{4.409328in}{2.250605in}}%
\pgfpathlineto{\pgfqpoint{4.423416in}{2.258893in}}%
\pgfpathlineto{\pgfqpoint{4.437517in}{2.267366in}}%
\pgfpathlineto{\pgfqpoint{4.445452in}{2.279029in}}%
\pgfpathlineto{\pgfqpoint{4.453381in}{2.290587in}}%
\pgfpathlineto{\pgfqpoint{4.461304in}{2.302041in}}%
\pgfpathlineto{\pgfqpoint{4.469222in}{2.313390in}}%
\pgfpathlineto{\pgfqpoint{4.455122in}{2.304831in}}%
\pgfpathlineto{\pgfqpoint{4.441037in}{2.296456in}}%
\pgfpathlineto{\pgfqpoint{4.426966in}{2.288267in}}%
\pgfpathlineto{\pgfqpoint{4.412909in}{2.280262in}}%
\pgfpathlineto{\pgfqpoint{4.404989in}{2.268988in}}%
\pgfpathlineto{\pgfqpoint{4.397063in}{2.257616in}}%
\pgfpathlineto{\pgfqpoint{4.389132in}{2.246149in}}%
\pgfpathlineto{\pgfqpoint{4.381195in}{2.234585in}}%
\pgfpathclose%
\pgfusepath{fill}%
\end{pgfscope}%
\begin{pgfscope}%
\pgfpathrectangle{\pgfqpoint{1.150000in}{0.150000in}}{\pgfqpoint{5.700000in}{5.700000in}}%
\pgfusepath{clip}%
\pgfsetbuttcap%
\pgfsetroundjoin%
\definecolor{currentfill}{rgb}{0.283229,0.120777,0.440584}%
\pgfsetfillcolor{currentfill}%
\pgfsetfillopacity{0.800000}%
\pgfsetlinewidth{0.000000pt}%
\definecolor{currentstroke}{rgb}{0.000000,0.000000,0.000000}%
\pgfsetstrokecolor{currentstroke}%
\pgfsetdash{}{0pt}%
\pgfpathmoveto{\pgfqpoint{3.877940in}{1.787460in}}%
\pgfpathlineto{\pgfqpoint{3.891791in}{1.790134in}}%
\pgfpathlineto{\pgfqpoint{3.905650in}{1.792997in}}%
\pgfpathlineto{\pgfqpoint{3.919519in}{1.796048in}}%
\pgfpathlineto{\pgfqpoint{3.933398in}{1.799287in}}%
\pgfpathlineto{\pgfqpoint{3.941485in}{1.811701in}}%
\pgfpathlineto{\pgfqpoint{3.949567in}{1.824099in}}%
\pgfpathlineto{\pgfqpoint{3.957645in}{1.836478in}}%
\pgfpathlineto{\pgfqpoint{3.965717in}{1.848835in}}%
\pgfpathlineto{\pgfqpoint{3.951844in}{1.845285in}}%
\pgfpathlineto{\pgfqpoint{3.937980in}{1.841922in}}%
\pgfpathlineto{\pgfqpoint{3.924126in}{1.838749in}}%
\pgfpathlineto{\pgfqpoint{3.910282in}{1.835763in}}%
\pgfpathlineto{\pgfqpoint{3.902204in}{1.823706in}}%
\pgfpathlineto{\pgfqpoint{3.894121in}{1.811634in}}%
\pgfpathlineto{\pgfqpoint{3.886033in}{1.799551in}}%
\pgfpathlineto{\pgfqpoint{3.877940in}{1.787460in}}%
\pgfpathclose%
\pgfusepath{fill}%
\end{pgfscope}%
\begin{pgfscope}%
\pgfpathrectangle{\pgfqpoint{1.150000in}{0.150000in}}{\pgfqpoint{5.700000in}{5.700000in}}%
\pgfusepath{clip}%
\pgfsetbuttcap%
\pgfsetroundjoin%
\definecolor{currentfill}{rgb}{0.265145,0.232956,0.516599}%
\pgfsetfillcolor{currentfill}%
\pgfsetfillopacity{0.800000}%
\pgfsetlinewidth{0.000000pt}%
\definecolor{currentstroke}{rgb}{0.000000,0.000000,0.000000}%
\pgfsetstrokecolor{currentstroke}%
\pgfsetdash{}{0pt}%
\pgfpathmoveto{\pgfqpoint{4.173495in}{2.035616in}}%
\pgfpathlineto{\pgfqpoint{4.187457in}{2.041592in}}%
\pgfpathlineto{\pgfqpoint{4.201432in}{2.047755in}}%
\pgfpathlineto{\pgfqpoint{4.215419in}{2.054103in}}%
\pgfpathlineto{\pgfqpoint{4.229418in}{2.060637in}}%
\pgfpathlineto{\pgfqpoint{4.237419in}{2.073087in}}%
\pgfpathlineto{\pgfqpoint{4.245415in}{2.085462in}}%
\pgfpathlineto{\pgfqpoint{4.253406in}{2.097759in}}%
\pgfpathlineto{\pgfqpoint{4.261391in}{2.109978in}}%
\pgfpathlineto{\pgfqpoint{4.247394in}{2.103260in}}%
\pgfpathlineto{\pgfqpoint{4.233410in}{2.096727in}}%
\pgfpathlineto{\pgfqpoint{4.219438in}{2.090380in}}%
\pgfpathlineto{\pgfqpoint{4.205478in}{2.084220in}}%
\pgfpathlineto{\pgfqpoint{4.197490in}{2.072173in}}%
\pgfpathlineto{\pgfqpoint{4.189497in}{2.060056in}}%
\pgfpathlineto{\pgfqpoint{4.181498in}{2.047870in}}%
\pgfpathlineto{\pgfqpoint{4.173495in}{2.035616in}}%
\pgfpathclose%
\pgfusepath{fill}%
\end{pgfscope}%
\begin{pgfscope}%
\pgfpathrectangle{\pgfqpoint{1.150000in}{0.150000in}}{\pgfqpoint{5.700000in}{5.700000in}}%
\pgfusepath{clip}%
\pgfsetbuttcap%
\pgfsetroundjoin%
\definecolor{currentfill}{rgb}{0.123444,0.636809,0.528763}%
\pgfsetfillcolor{currentfill}%
\pgfsetfillopacity{0.800000}%
\pgfsetlinewidth{0.000000pt}%
\definecolor{currentstroke}{rgb}{0.000000,0.000000,0.000000}%
\pgfsetstrokecolor{currentstroke}%
\pgfsetdash{}{0pt}%
\pgfpathmoveto{\pgfqpoint{5.475726in}{3.209306in}}%
\pgfpathlineto{\pgfqpoint{5.490430in}{3.223045in}}%
\pgfpathlineto{\pgfqpoint{5.505156in}{3.236967in}}%
\pgfpathlineto{\pgfqpoint{5.519902in}{3.251071in}}%
\pgfpathlineto{\pgfqpoint{5.534670in}{3.265357in}}%
\pgfpathlineto{\pgfqpoint{5.542043in}{3.268414in}}%
\pgfpathlineto{\pgfqpoint{5.549408in}{3.271400in}}%
\pgfpathlineto{\pgfqpoint{5.556765in}{3.274320in}}%
\pgfpathlineto{\pgfqpoint{5.564113in}{3.277180in}}%
\pgfpathlineto{\pgfqpoint{5.549367in}{3.263324in}}%
\pgfpathlineto{\pgfqpoint{5.534643in}{3.249650in}}%
\pgfpathlineto{\pgfqpoint{5.519939in}{3.236157in}}%
\pgfpathlineto{\pgfqpoint{5.505256in}{3.222846in}}%
\pgfpathlineto{\pgfqpoint{5.497886in}{3.219545in}}%
\pgfpathlineto{\pgfqpoint{5.490507in}{3.216191in}}%
\pgfpathlineto{\pgfqpoint{5.483121in}{3.212780in}}%
\pgfpathlineto{\pgfqpoint{5.475726in}{3.209306in}}%
\pgfpathclose%
\pgfusepath{fill}%
\end{pgfscope}%
\begin{pgfscope}%
\pgfpathrectangle{\pgfqpoint{1.150000in}{0.150000in}}{\pgfqpoint{5.700000in}{5.700000in}}%
\pgfusepath{clip}%
\pgfsetbuttcap%
\pgfsetroundjoin%
\definecolor{currentfill}{rgb}{0.268510,0.009605,0.335427}%
\pgfsetfillcolor{currentfill}%
\pgfsetfillopacity{0.800000}%
\pgfsetlinewidth{0.000000pt}%
\definecolor{currentstroke}{rgb}{0.000000,0.000000,0.000000}%
\pgfsetstrokecolor{currentstroke}%
\pgfsetdash{}{0pt}%
\pgfpathmoveto{\pgfqpoint{3.438609in}{1.576032in}}%
\pgfpathlineto{\pgfqpoint{3.452377in}{1.572830in}}%
\pgfpathlineto{\pgfqpoint{3.466150in}{1.569825in}}%
\pgfpathlineto{\pgfqpoint{3.479927in}{1.567018in}}%
\pgfpathlineto{\pgfqpoint{3.493709in}{1.564407in}}%
\pgfpathlineto{\pgfqpoint{3.501959in}{1.574035in}}%
\pgfpathlineto{\pgfqpoint{3.510201in}{1.583760in}}%
\pgfpathlineto{\pgfqpoint{3.518436in}{1.593577in}}%
\pgfpathlineto{\pgfqpoint{3.526665in}{1.603482in}}%
\pgfpathlineto{\pgfqpoint{3.512898in}{1.605626in}}%
\pgfpathlineto{\pgfqpoint{3.499137in}{1.607967in}}%
\pgfpathlineto{\pgfqpoint{3.485380in}{1.610505in}}%
\pgfpathlineto{\pgfqpoint{3.471629in}{1.613240in}}%
\pgfpathlineto{\pgfqpoint{3.463384in}{1.603791in}}%
\pgfpathlineto{\pgfqpoint{3.455133in}{1.594437in}}%
\pgfpathlineto{\pgfqpoint{3.446874in}{1.585182in}}%
\pgfpathlineto{\pgfqpoint{3.438609in}{1.576032in}}%
\pgfpathclose%
\pgfusepath{fill}%
\end{pgfscope}%
\begin{pgfscope}%
\pgfpathrectangle{\pgfqpoint{1.150000in}{0.150000in}}{\pgfqpoint{5.700000in}{5.700000in}}%
\pgfusepath{clip}%
\pgfsetbuttcap%
\pgfsetroundjoin%
\definecolor{currentfill}{rgb}{0.274952,0.037752,0.364543}%
\pgfsetfillcolor{currentfill}%
\pgfsetfillopacity{0.800000}%
\pgfsetlinewidth{0.000000pt}%
\definecolor{currentstroke}{rgb}{0.000000,0.000000,0.000000}%
\pgfsetstrokecolor{currentstroke}%
\pgfsetdash{}{0pt}%
\pgfpathmoveto{\pgfqpoint{3.007471in}{1.652651in}}%
\pgfpathlineto{\pgfqpoint{3.021273in}{1.642854in}}%
\pgfpathlineto{\pgfqpoint{3.035074in}{1.633276in}}%
\pgfpathlineto{\pgfqpoint{3.048875in}{1.623917in}}%
\pgfpathlineto{\pgfqpoint{3.062674in}{1.614776in}}%
\pgfpathlineto{\pgfqpoint{3.071171in}{1.619411in}}%
\pgfpathlineto{\pgfqpoint{3.079656in}{1.624252in}}%
\pgfpathlineto{\pgfqpoint{3.088130in}{1.629293in}}%
\pgfpathlineto{\pgfqpoint{3.096593in}{1.634528in}}%
\pgfpathlineto{\pgfqpoint{3.082824in}{1.643102in}}%
\pgfpathlineto{\pgfqpoint{3.069054in}{1.651894in}}%
\pgfpathlineto{\pgfqpoint{3.055284in}{1.660904in}}%
\pgfpathlineto{\pgfqpoint{3.041514in}{1.670134in}}%
\pgfpathlineto{\pgfqpoint{3.033021in}{1.665454in}}%
\pgfpathlineto{\pgfqpoint{3.024516in}{1.660976in}}%
\pgfpathlineto{\pgfqpoint{3.016000in}{1.656707in}}%
\pgfpathlineto{\pgfqpoint{3.007471in}{1.652651in}}%
\pgfpathclose%
\pgfusepath{fill}%
\end{pgfscope}%
\begin{pgfscope}%
\pgfpathrectangle{\pgfqpoint{1.150000in}{0.150000in}}{\pgfqpoint{5.700000in}{5.700000in}}%
\pgfusepath{clip}%
\pgfsetbuttcap%
\pgfsetroundjoin%
\definecolor{currentfill}{rgb}{0.188923,0.410910,0.556326}%
\pgfsetfillcolor{currentfill}%
\pgfsetfillopacity{0.800000}%
\pgfsetlinewidth{0.000000pt}%
\definecolor{currentstroke}{rgb}{0.000000,0.000000,0.000000}%
\pgfsetstrokecolor{currentstroke}%
\pgfsetdash{}{0pt}%
\pgfpathmoveto{\pgfqpoint{2.225717in}{2.601174in}}%
\pgfpathlineto{\pgfqpoint{2.239965in}{2.576290in}}%
\pgfpathlineto{\pgfqpoint{2.254197in}{2.551742in}}%
\pgfpathlineto{\pgfqpoint{2.268413in}{2.527527in}}%
\pgfpathlineto{\pgfqpoint{2.282614in}{2.503641in}}%
\pgfpathlineto{\pgfqpoint{2.291728in}{2.499901in}}%
\pgfpathlineto{\pgfqpoint{2.300820in}{2.496491in}}%
\pgfpathlineto{\pgfqpoint{2.309890in}{2.493405in}}%
\pgfpathlineto{\pgfqpoint{2.318939in}{2.490637in}}%
\pgfpathlineto{\pgfqpoint{2.304796in}{2.513908in}}%
\pgfpathlineto{\pgfqpoint{2.290638in}{2.537506in}}%
\pgfpathlineto{\pgfqpoint{2.276464in}{2.561436in}}%
\pgfpathlineto{\pgfqpoint{2.262276in}{2.585701in}}%
\pgfpathlineto{\pgfqpoint{2.253169in}{2.589072in}}%
\pgfpathlineto{\pgfqpoint{2.244041in}{2.592770in}}%
\pgfpathlineto{\pgfqpoint{2.234891in}{2.596802in}}%
\pgfpathlineto{\pgfqpoint{2.225717in}{2.601174in}}%
\pgfpathclose%
\pgfusepath{fill}%
\end{pgfscope}%
\begin{pgfscope}%
\pgfpathrectangle{\pgfqpoint{1.150000in}{0.150000in}}{\pgfqpoint{5.700000in}{5.700000in}}%
\pgfusepath{clip}%
\pgfsetbuttcap%
\pgfsetroundjoin%
\definecolor{currentfill}{rgb}{0.281412,0.155834,0.469201}%
\pgfsetfillcolor{currentfill}%
\pgfsetfillopacity{0.800000}%
\pgfsetlinewidth{0.000000pt}%
\definecolor{currentstroke}{rgb}{0.000000,0.000000,0.000000}%
\pgfsetstrokecolor{currentstroke}%
\pgfsetdash{}{0pt}%
\pgfpathmoveto{\pgfqpoint{3.965717in}{1.848835in}}%
\pgfpathlineto{\pgfqpoint{3.979601in}{1.852572in}}%
\pgfpathlineto{\pgfqpoint{3.993495in}{1.856498in}}%
\pgfpathlineto{\pgfqpoint{4.007400in}{1.860610in}}%
\pgfpathlineto{\pgfqpoint{4.021315in}{1.864909in}}%
\pgfpathlineto{\pgfqpoint{4.029378in}{1.877533in}}%
\pgfpathlineto{\pgfqpoint{4.037437in}{1.890122in}}%
\pgfpathlineto{\pgfqpoint{4.045491in}{1.902673in}}%
\pgfpathlineto{\pgfqpoint{4.053541in}{1.915184in}}%
\pgfpathlineto{\pgfqpoint{4.039629in}{1.910605in}}%
\pgfpathlineto{\pgfqpoint{4.025729in}{1.906213in}}%
\pgfpathlineto{\pgfqpoint{4.011839in}{1.902008in}}%
\pgfpathlineto{\pgfqpoint{3.997960in}{1.897990in}}%
\pgfpathlineto{\pgfqpoint{3.989907in}{1.885747in}}%
\pgfpathlineto{\pgfqpoint{3.981848in}{1.873472in}}%
\pgfpathlineto{\pgfqpoint{3.973785in}{1.861167in}}%
\pgfpathlineto{\pgfqpoint{3.965717in}{1.848835in}}%
\pgfpathclose%
\pgfusepath{fill}%
\end{pgfscope}%
\begin{pgfscope}%
\pgfpathrectangle{\pgfqpoint{1.150000in}{0.150000in}}{\pgfqpoint{5.700000in}{5.700000in}}%
\pgfusepath{clip}%
\pgfsetbuttcap%
\pgfsetroundjoin%
\definecolor{currentfill}{rgb}{0.274128,0.199721,0.498911}%
\pgfsetfillcolor{currentfill}%
\pgfsetfillopacity{0.800000}%
\pgfsetlinewidth{0.000000pt}%
\definecolor{currentstroke}{rgb}{0.000000,0.000000,0.000000}%
\pgfsetstrokecolor{currentstroke}%
\pgfsetdash{}{0pt}%
\pgfpathmoveto{\pgfqpoint{2.584307in}{2.024986in}}%
\pgfpathlineto{\pgfqpoint{2.598276in}{2.007828in}}%
\pgfpathlineto{\pgfqpoint{2.612237in}{1.990934in}}%
\pgfpathlineto{\pgfqpoint{2.626191in}{1.974303in}}%
\pgfpathlineto{\pgfqpoint{2.640136in}{1.957932in}}%
\pgfpathlineto{\pgfqpoint{2.648962in}{1.957342in}}%
\pgfpathlineto{\pgfqpoint{2.657770in}{1.957042in}}%
\pgfpathlineto{\pgfqpoint{2.666560in}{1.957025in}}%
\pgfpathlineto{\pgfqpoint{2.675334in}{1.957286in}}%
\pgfpathlineto{\pgfqpoint{2.661434in}{1.973039in}}%
\pgfpathlineto{\pgfqpoint{2.647527in}{1.989051in}}%
\pgfpathlineto{\pgfqpoint{2.633613in}{2.005325in}}%
\pgfpathlineto{\pgfqpoint{2.619691in}{2.021862in}}%
\pgfpathlineto{\pgfqpoint{2.610872in}{2.022207in}}%
\pgfpathlineto{\pgfqpoint{2.602035in}{2.022839in}}%
\pgfpathlineto{\pgfqpoint{2.593180in}{2.023763in}}%
\pgfpathlineto{\pgfqpoint{2.584307in}{2.024986in}}%
\pgfpathclose%
\pgfusepath{fill}%
\end{pgfscope}%
\begin{pgfscope}%
\pgfpathrectangle{\pgfqpoint{1.150000in}{0.150000in}}{\pgfqpoint{5.700000in}{5.700000in}}%
\pgfusepath{clip}%
\pgfsetbuttcap%
\pgfsetroundjoin%
\definecolor{currentfill}{rgb}{0.279574,0.170599,0.479997}%
\pgfsetfillcolor{currentfill}%
\pgfsetfillopacity{0.800000}%
\pgfsetlinewidth{0.000000pt}%
\definecolor{currentstroke}{rgb}{0.000000,0.000000,0.000000}%
\pgfsetstrokecolor{currentstroke}%
\pgfsetdash{}{0pt}%
\pgfpathmoveto{\pgfqpoint{2.640136in}{1.957932in}}%
\pgfpathlineto{\pgfqpoint{2.654075in}{1.941820in}}%
\pgfpathlineto{\pgfqpoint{2.668007in}{1.925965in}}%
\pgfpathlineto{\pgfqpoint{2.681932in}{1.910364in}}%
\pgfpathlineto{\pgfqpoint{2.695851in}{1.895017in}}%
\pgfpathlineto{\pgfqpoint{2.704630in}{1.895056in}}%
\pgfpathlineto{\pgfqpoint{2.713393in}{1.895376in}}%
\pgfpathlineto{\pgfqpoint{2.722140in}{1.895971in}}%
\pgfpathlineto{\pgfqpoint{2.730870in}{1.896835in}}%
\pgfpathlineto{\pgfqpoint{2.716995in}{1.911567in}}%
\pgfpathlineto{\pgfqpoint{2.703115in}{1.926552in}}%
\pgfpathlineto{\pgfqpoint{2.689228in}{1.941791in}}%
\pgfpathlineto{\pgfqpoint{2.675334in}{1.957286in}}%
\pgfpathlineto{\pgfqpoint{2.666560in}{1.957025in}}%
\pgfpathlineto{\pgfqpoint{2.657770in}{1.957042in}}%
\pgfpathlineto{\pgfqpoint{2.648962in}{1.957342in}}%
\pgfpathlineto{\pgfqpoint{2.640136in}{1.957932in}}%
\pgfpathclose%
\pgfusepath{fill}%
\end{pgfscope}%
\begin{pgfscope}%
\pgfpathrectangle{\pgfqpoint{1.150000in}{0.150000in}}{\pgfqpoint{5.700000in}{5.700000in}}%
\pgfusepath{clip}%
\pgfsetbuttcap%
\pgfsetroundjoin%
\definecolor{currentfill}{rgb}{0.160665,0.478540,0.558115}%
\pgfsetfillcolor{currentfill}%
\pgfsetfillopacity{0.800000}%
\pgfsetlinewidth{0.000000pt}%
\definecolor{currentstroke}{rgb}{0.000000,0.000000,0.000000}%
\pgfsetstrokecolor{currentstroke}%
\pgfsetdash{}{0pt}%
\pgfpathmoveto{\pgfqpoint{4.884737in}{2.713959in}}%
\pgfpathlineto{\pgfqpoint{4.899085in}{2.725472in}}%
\pgfpathlineto{\pgfqpoint{4.913450in}{2.737170in}}%
\pgfpathlineto{\pgfqpoint{4.927833in}{2.749052in}}%
\pgfpathlineto{\pgfqpoint{4.942234in}{2.761118in}}%
\pgfpathlineto{\pgfqpoint{4.949963in}{2.769292in}}%
\pgfpathlineto{\pgfqpoint{4.957684in}{2.777339in}}%
\pgfpathlineto{\pgfqpoint{4.965398in}{2.785261in}}%
\pgfpathlineto{\pgfqpoint{4.973103in}{2.793061in}}%
\pgfpathlineto{\pgfqpoint{4.958710in}{2.781145in}}%
\pgfpathlineto{\pgfqpoint{4.944335in}{2.769413in}}%
\pgfpathlineto{\pgfqpoint{4.929977in}{2.757865in}}%
\pgfpathlineto{\pgfqpoint{4.915637in}{2.746501in}}%
\pgfpathlineto{\pgfqpoint{4.907924in}{2.738539in}}%
\pgfpathlineto{\pgfqpoint{4.900202in}{2.730462in}}%
\pgfpathlineto{\pgfqpoint{4.892474in}{2.722270in}}%
\pgfpathlineto{\pgfqpoint{4.884737in}{2.713959in}}%
\pgfpathclose%
\pgfusepath{fill}%
\end{pgfscope}%
\begin{pgfscope}%
\pgfpathrectangle{\pgfqpoint{1.150000in}{0.150000in}}{\pgfqpoint{5.700000in}{5.700000in}}%
\pgfusepath{clip}%
\pgfsetbuttcap%
\pgfsetroundjoin%
\definecolor{currentfill}{rgb}{0.266580,0.228262,0.514349}%
\pgfsetfillcolor{currentfill}%
\pgfsetfillopacity{0.800000}%
\pgfsetlinewidth{0.000000pt}%
\definecolor{currentstroke}{rgb}{0.000000,0.000000,0.000000}%
\pgfsetstrokecolor{currentstroke}%
\pgfsetdash{}{0pt}%
\pgfpathmoveto{\pgfqpoint{2.528346in}{2.096306in}}%
\pgfpathlineto{\pgfqpoint{2.542350in}{2.078069in}}%
\pgfpathlineto{\pgfqpoint{2.556344in}{2.060105in}}%
\pgfpathlineto{\pgfqpoint{2.570330in}{2.042411in}}%
\pgfpathlineto{\pgfqpoint{2.584307in}{2.024986in}}%
\pgfpathlineto{\pgfqpoint{2.593180in}{2.023763in}}%
\pgfpathlineto{\pgfqpoint{2.602035in}{2.022839in}}%
\pgfpathlineto{\pgfqpoint{2.610872in}{2.022207in}}%
\pgfpathlineto{\pgfqpoint{2.619691in}{2.021862in}}%
\pgfpathlineto{\pgfqpoint{2.605762in}{2.038665in}}%
\pgfpathlineto{\pgfqpoint{2.591825in}{2.055735in}}%
\pgfpathlineto{\pgfqpoint{2.577879in}{2.073076in}}%
\pgfpathlineto{\pgfqpoint{2.563925in}{2.090688in}}%
\pgfpathlineto{\pgfqpoint{2.555058in}{2.091643in}}%
\pgfpathlineto{\pgfqpoint{2.546173in}{2.092893in}}%
\pgfpathlineto{\pgfqpoint{2.537269in}{2.094446in}}%
\pgfpathlineto{\pgfqpoint{2.528346in}{2.096306in}}%
\pgfpathclose%
\pgfusepath{fill}%
\end{pgfscope}%
\begin{pgfscope}%
\pgfpathrectangle{\pgfqpoint{1.150000in}{0.150000in}}{\pgfqpoint{5.700000in}{5.700000in}}%
\pgfusepath{clip}%
\pgfsetbuttcap%
\pgfsetroundjoin%
\definecolor{currentfill}{rgb}{0.268510,0.009605,0.335427}%
\pgfsetfillcolor{currentfill}%
\pgfsetfillopacity{0.800000}%
\pgfsetlinewidth{0.000000pt}%
\definecolor{currentstroke}{rgb}{0.000000,0.000000,0.000000}%
\pgfsetstrokecolor{currentstroke}%
\pgfsetdash{}{0pt}%
\pgfpathmoveto{\pgfqpoint{3.206759in}{1.573633in}}%
\pgfpathlineto{\pgfqpoint{3.220534in}{1.566969in}}%
\pgfpathlineto{\pgfqpoint{3.234310in}{1.560512in}}%
\pgfpathlineto{\pgfqpoint{3.248089in}{1.554261in}}%
\pgfpathlineto{\pgfqpoint{3.261869in}{1.548216in}}%
\pgfpathlineto{\pgfqpoint{3.270242in}{1.555274in}}%
\pgfpathlineto{\pgfqpoint{3.278606in}{1.562491in}}%
\pgfpathlineto{\pgfqpoint{3.286961in}{1.569862in}}%
\pgfpathlineto{\pgfqpoint{3.295307in}{1.577383in}}%
\pgfpathlineto{\pgfqpoint{3.281549in}{1.582897in}}%
\pgfpathlineto{\pgfqpoint{3.267795in}{1.588616in}}%
\pgfpathlineto{\pgfqpoint{3.254042in}{1.594542in}}%
\pgfpathlineto{\pgfqpoint{3.240291in}{1.600674in}}%
\pgfpathlineto{\pgfqpoint{3.231922in}{1.593673in}}%
\pgfpathlineto{\pgfqpoint{3.223544in}{1.586829in}}%
\pgfpathlineto{\pgfqpoint{3.215156in}{1.580147in}}%
\pgfpathlineto{\pgfqpoint{3.206759in}{1.573633in}}%
\pgfpathclose%
\pgfusepath{fill}%
\end{pgfscope}%
\begin{pgfscope}%
\pgfpathrectangle{\pgfqpoint{1.150000in}{0.150000in}}{\pgfqpoint{5.700000in}{5.700000in}}%
\pgfusepath{clip}%
\pgfsetbuttcap%
\pgfsetroundjoin%
\definecolor{currentfill}{rgb}{0.282290,0.145912,0.461510}%
\pgfsetfillcolor{currentfill}%
\pgfsetfillopacity{0.800000}%
\pgfsetlinewidth{0.000000pt}%
\definecolor{currentstroke}{rgb}{0.000000,0.000000,0.000000}%
\pgfsetstrokecolor{currentstroke}%
\pgfsetdash{}{0pt}%
\pgfpathmoveto{\pgfqpoint{2.695851in}{1.895017in}}%
\pgfpathlineto{\pgfqpoint{2.709763in}{1.879921in}}%
\pgfpathlineto{\pgfqpoint{2.723670in}{1.865074in}}%
\pgfpathlineto{\pgfqpoint{2.737570in}{1.850475in}}%
\pgfpathlineto{\pgfqpoint{2.751466in}{1.836123in}}%
\pgfpathlineto{\pgfqpoint{2.760201in}{1.836788in}}%
\pgfpathlineto{\pgfqpoint{2.768921in}{1.837725in}}%
\pgfpathlineto{\pgfqpoint{2.777625in}{1.838928in}}%
\pgfpathlineto{\pgfqpoint{2.786314in}{1.840391in}}%
\pgfpathlineto{\pgfqpoint{2.772461in}{1.854132in}}%
\pgfpathlineto{\pgfqpoint{2.758603in}{1.868119in}}%
\pgfpathlineto{\pgfqpoint{2.744739in}{1.882352in}}%
\pgfpathlineto{\pgfqpoint{2.730870in}{1.896835in}}%
\pgfpathlineto{\pgfqpoint{2.722140in}{1.895971in}}%
\pgfpathlineto{\pgfqpoint{2.713393in}{1.895376in}}%
\pgfpathlineto{\pgfqpoint{2.704630in}{1.895056in}}%
\pgfpathlineto{\pgfqpoint{2.695851in}{1.895017in}}%
\pgfpathclose%
\pgfusepath{fill}%
\end{pgfscope}%
\begin{pgfscope}%
\pgfpathrectangle{\pgfqpoint{1.150000in}{0.150000in}}{\pgfqpoint{5.700000in}{5.700000in}}%
\pgfusepath{clip}%
\pgfsetbuttcap%
\pgfsetroundjoin%
\definecolor{currentfill}{rgb}{0.137339,0.662252,0.515571}%
\pgfsetfillcolor{currentfill}%
\pgfsetfillopacity{0.800000}%
\pgfsetlinewidth{0.000000pt}%
\definecolor{currentstroke}{rgb}{0.000000,0.000000,0.000000}%
\pgfsetstrokecolor{currentstroke}%
\pgfsetdash{}{0pt}%
\pgfpathmoveto{\pgfqpoint{5.564113in}{3.277180in}}%
\pgfpathlineto{\pgfqpoint{5.578880in}{3.291218in}}%
\pgfpathlineto{\pgfqpoint{5.593668in}{3.305439in}}%
\pgfpathlineto{\pgfqpoint{5.608477in}{3.319842in}}%
\pgfpathlineto{\pgfqpoint{5.623308in}{3.334427in}}%
\pgfpathlineto{\pgfqpoint{5.630624in}{3.336778in}}%
\pgfpathlineto{\pgfqpoint{5.637932in}{3.339070in}}%
\pgfpathlineto{\pgfqpoint{5.645231in}{3.341308in}}%
\pgfpathlineto{\pgfqpoint{5.652521in}{3.343498in}}%
\pgfpathlineto{\pgfqpoint{5.637715in}{3.329379in}}%
\pgfpathlineto{\pgfqpoint{5.622929in}{3.315442in}}%
\pgfpathlineto{\pgfqpoint{5.608165in}{3.301686in}}%
\pgfpathlineto{\pgfqpoint{5.593423in}{3.288111in}}%
\pgfpathlineto{\pgfqpoint{5.586107in}{3.285444in}}%
\pgfpathlineto{\pgfqpoint{5.578784in}{3.282737in}}%
\pgfpathlineto{\pgfqpoint{5.571452in}{3.279984in}}%
\pgfpathlineto{\pgfqpoint{5.564113in}{3.277180in}}%
\pgfpathclose%
\pgfusepath{fill}%
\end{pgfscope}%
\begin{pgfscope}%
\pgfpathrectangle{\pgfqpoint{1.150000in}{0.150000in}}{\pgfqpoint{5.700000in}{5.700000in}}%
\pgfusepath{clip}%
\pgfsetbuttcap%
\pgfsetroundjoin%
\definecolor{currentfill}{rgb}{0.128729,0.563265,0.551229}%
\pgfsetfillcolor{currentfill}%
\pgfsetfillopacity{0.800000}%
\pgfsetlinewidth{0.000000pt}%
\definecolor{currentstroke}{rgb}{0.000000,0.000000,0.000000}%
\pgfsetstrokecolor{currentstroke}%
\pgfsetdash{}{0pt}%
\pgfpathmoveto{\pgfqpoint{5.180528in}{2.974649in}}%
\pgfpathlineto{\pgfqpoint{5.195058in}{2.987560in}}%
\pgfpathlineto{\pgfqpoint{5.209607in}{3.000654in}}%
\pgfpathlineto{\pgfqpoint{5.224177in}{3.013932in}}%
\pgfpathlineto{\pgfqpoint{5.238766in}{3.027394in}}%
\pgfpathlineto{\pgfqpoint{5.246332in}{3.033001in}}%
\pgfpathlineto{\pgfqpoint{5.253890in}{3.038497in}}%
\pgfpathlineto{\pgfqpoint{5.261439in}{3.043888in}}%
\pgfpathlineto{\pgfqpoint{5.268980in}{3.049175in}}%
\pgfpathlineto{\pgfqpoint{5.254405in}{3.036003in}}%
\pgfpathlineto{\pgfqpoint{5.239849in}{3.023014in}}%
\pgfpathlineto{\pgfqpoint{5.225314in}{3.010209in}}%
\pgfpathlineto{\pgfqpoint{5.210797in}{2.997586in}}%
\pgfpathlineto{\pgfqpoint{5.203242in}{2.991997in}}%
\pgfpathlineto{\pgfqpoint{5.195679in}{2.986314in}}%
\pgfpathlineto{\pgfqpoint{5.188107in}{2.980532in}}%
\pgfpathlineto{\pgfqpoint{5.180528in}{2.974649in}}%
\pgfpathclose%
\pgfusepath{fill}%
\end{pgfscope}%
\begin{pgfscope}%
\pgfpathrectangle{\pgfqpoint{1.150000in}{0.150000in}}{\pgfqpoint{5.700000in}{5.700000in}}%
\pgfusepath{clip}%
\pgfsetbuttcap%
\pgfsetroundjoin%
\definecolor{currentfill}{rgb}{0.187231,0.414746,0.556547}%
\pgfsetfillcolor{currentfill}%
\pgfsetfillopacity{0.800000}%
\pgfsetlinewidth{0.000000pt}%
\definecolor{currentstroke}{rgb}{0.000000,0.000000,0.000000}%
\pgfsetstrokecolor{currentstroke}%
\pgfsetdash{}{0pt}%
\pgfpathmoveto{\pgfqpoint{4.677037in}{2.516983in}}%
\pgfpathlineto{\pgfqpoint{4.691266in}{2.527266in}}%
\pgfpathlineto{\pgfqpoint{4.705511in}{2.537733in}}%
\pgfpathlineto{\pgfqpoint{4.719772in}{2.548385in}}%
\pgfpathlineto{\pgfqpoint{4.734050in}{2.559222in}}%
\pgfpathlineto{\pgfqpoint{4.741877in}{2.569107in}}%
\pgfpathlineto{\pgfqpoint{4.749698in}{2.578866in}}%
\pgfpathlineto{\pgfqpoint{4.757512in}{2.588500in}}%
\pgfpathlineto{\pgfqpoint{4.765319in}{2.598009in}}%
\pgfpathlineto{\pgfqpoint{4.751045in}{2.587219in}}%
\pgfpathlineto{\pgfqpoint{4.736788in}{2.576615in}}%
\pgfpathlineto{\pgfqpoint{4.722548in}{2.566194in}}%
\pgfpathlineto{\pgfqpoint{4.708324in}{2.555958in}}%
\pgfpathlineto{\pgfqpoint{4.700512in}{2.546389in}}%
\pgfpathlineto{\pgfqpoint{4.692694in}{2.536705in}}%
\pgfpathlineto{\pgfqpoint{4.684869in}{2.526903in}}%
\pgfpathlineto{\pgfqpoint{4.677037in}{2.516983in}}%
\pgfpathclose%
\pgfusepath{fill}%
\end{pgfscope}%
\begin{pgfscope}%
\pgfpathrectangle{\pgfqpoint{1.150000in}{0.150000in}}{\pgfqpoint{5.700000in}{5.700000in}}%
\pgfusepath{clip}%
\pgfsetbuttcap%
\pgfsetroundjoin%
\definecolor{currentfill}{rgb}{0.255645,0.260703,0.528312}%
\pgfsetfillcolor{currentfill}%
\pgfsetfillopacity{0.800000}%
\pgfsetlinewidth{0.000000pt}%
\definecolor{currentstroke}{rgb}{0.000000,0.000000,0.000000}%
\pgfsetstrokecolor{currentstroke}%
\pgfsetdash{}{0pt}%
\pgfpathmoveto{\pgfqpoint{2.472237in}{2.172028in}}%
\pgfpathlineto{\pgfqpoint{2.486279in}{2.152677in}}%
\pgfpathlineto{\pgfqpoint{2.500311in}{2.133608in}}%
\pgfpathlineto{\pgfqpoint{2.514334in}{2.114818in}}%
\pgfpathlineto{\pgfqpoint{2.528346in}{2.096306in}}%
\pgfpathlineto{\pgfqpoint{2.537269in}{2.094446in}}%
\pgfpathlineto{\pgfqpoint{2.546173in}{2.092893in}}%
\pgfpathlineto{\pgfqpoint{2.555058in}{2.091643in}}%
\pgfpathlineto{\pgfqpoint{2.563925in}{2.090688in}}%
\pgfpathlineto{\pgfqpoint{2.549962in}{2.108573in}}%
\pgfpathlineto{\pgfqpoint{2.535990in}{2.126735in}}%
\pgfpathlineto{\pgfqpoint{2.522009in}{2.145176in}}%
\pgfpathlineto{\pgfqpoint{2.508018in}{2.163897in}}%
\pgfpathlineto{\pgfqpoint{2.499102in}{2.165466in}}%
\pgfpathlineto{\pgfqpoint{2.490167in}{2.167340in}}%
\pgfpathlineto{\pgfqpoint{2.481211in}{2.169525in}}%
\pgfpathlineto{\pgfqpoint{2.472237in}{2.172028in}}%
\pgfpathclose%
\pgfusepath{fill}%
\end{pgfscope}%
\begin{pgfscope}%
\pgfpathrectangle{\pgfqpoint{1.150000in}{0.150000in}}{\pgfqpoint{5.700000in}{5.700000in}}%
\pgfusepath{clip}%
\pgfsetbuttcap%
\pgfsetroundjoin%
\definecolor{currentfill}{rgb}{0.218130,0.347432,0.550038}%
\pgfsetfillcolor{currentfill}%
\pgfsetfillopacity{0.800000}%
\pgfsetlinewidth{0.000000pt}%
\definecolor{currentstroke}{rgb}{0.000000,0.000000,0.000000}%
\pgfsetstrokecolor{currentstroke}%
\pgfsetdash{}{0pt}%
\pgfpathmoveto{\pgfqpoint{4.469222in}{2.313390in}}%
\pgfpathlineto{\pgfqpoint{4.483336in}{2.322135in}}%
\pgfpathlineto{\pgfqpoint{4.497465in}{2.331065in}}%
\pgfpathlineto{\pgfqpoint{4.511609in}{2.340179in}}%
\pgfpathlineto{\pgfqpoint{4.525768in}{2.349478in}}%
\pgfpathlineto{\pgfqpoint{4.533677in}{2.360788in}}%
\pgfpathlineto{\pgfqpoint{4.541581in}{2.371984in}}%
\pgfpathlineto{\pgfqpoint{4.549479in}{2.383066in}}%
\pgfpathlineto{\pgfqpoint{4.557370in}{2.394034in}}%
\pgfpathlineto{\pgfqpoint{4.543214in}{2.384681in}}%
\pgfpathlineto{\pgfqpoint{4.529073in}{2.375513in}}%
\pgfpathlineto{\pgfqpoint{4.514946in}{2.366529in}}%
\pgfpathlineto{\pgfqpoint{4.500835in}{2.357731in}}%
\pgfpathlineto{\pgfqpoint{4.492940in}{2.346804in}}%
\pgfpathlineto{\pgfqpoint{4.485040in}{2.335772in}}%
\pgfpathlineto{\pgfqpoint{4.477134in}{2.324634in}}%
\pgfpathlineto{\pgfqpoint{4.469222in}{2.313390in}}%
\pgfpathclose%
\pgfusepath{fill}%
\end{pgfscope}%
\begin{pgfscope}%
\pgfpathrectangle{\pgfqpoint{1.150000in}{0.150000in}}{\pgfqpoint{5.700000in}{5.700000in}}%
\pgfusepath{clip}%
\pgfsetbuttcap%
\pgfsetroundjoin%
\definecolor{currentfill}{rgb}{0.267004,0.004874,0.329415}%
\pgfsetfillcolor{currentfill}%
\pgfsetfillopacity{0.800000}%
\pgfsetlinewidth{0.000000pt}%
\definecolor{currentstroke}{rgb}{0.000000,0.000000,0.000000}%
\pgfsetstrokecolor{currentstroke}%
\pgfsetdash{}{0pt}%
\pgfpathmoveto{\pgfqpoint{3.350362in}{1.557365in}}%
\pgfpathlineto{\pgfqpoint{3.364134in}{1.552865in}}%
\pgfpathlineto{\pgfqpoint{3.377909in}{1.548567in}}%
\pgfpathlineto{\pgfqpoint{3.391688in}{1.544468in}}%
\pgfpathlineto{\pgfqpoint{3.405470in}{1.540569in}}%
\pgfpathlineto{\pgfqpoint{3.413766in}{1.549255in}}%
\pgfpathlineto{\pgfqpoint{3.422055in}{1.558064in}}%
\pgfpathlineto{\pgfqpoint{3.430336in}{1.566991in}}%
\pgfpathlineto{\pgfqpoint{3.438609in}{1.576032in}}%
\pgfpathlineto{\pgfqpoint{3.424845in}{1.579433in}}%
\pgfpathlineto{\pgfqpoint{3.411085in}{1.583033in}}%
\pgfpathlineto{\pgfqpoint{3.397329in}{1.586833in}}%
\pgfpathlineto{\pgfqpoint{3.383577in}{1.590834in}}%
\pgfpathlineto{\pgfqpoint{3.375285in}{1.582279in}}%
\pgfpathlineto{\pgfqpoint{3.366986in}{1.573846in}}%
\pgfpathlineto{\pgfqpoint{3.358678in}{1.565540in}}%
\pgfpathlineto{\pgfqpoint{3.350362in}{1.557365in}}%
\pgfpathclose%
\pgfusepath{fill}%
\end{pgfscope}%
\begin{pgfscope}%
\pgfpathrectangle{\pgfqpoint{1.150000in}{0.150000in}}{\pgfqpoint{5.700000in}{5.700000in}}%
\pgfusepath{clip}%
\pgfsetbuttcap%
\pgfsetroundjoin%
\definecolor{currentfill}{rgb}{0.253935,0.265254,0.529983}%
\pgfsetfillcolor{currentfill}%
\pgfsetfillopacity{0.800000}%
\pgfsetlinewidth{0.000000pt}%
\definecolor{currentstroke}{rgb}{0.000000,0.000000,0.000000}%
\pgfsetstrokecolor{currentstroke}%
\pgfsetdash{}{0pt}%
\pgfpathmoveto{\pgfqpoint{4.261391in}{2.109978in}}%
\pgfpathlineto{\pgfqpoint{4.275402in}{2.116882in}}%
\pgfpathlineto{\pgfqpoint{4.289425in}{2.123972in}}%
\pgfpathlineto{\pgfqpoint{4.303461in}{2.131247in}}%
\pgfpathlineto{\pgfqpoint{4.317510in}{2.138707in}}%
\pgfpathlineto{\pgfqpoint{4.325489in}{2.151011in}}%
\pgfpathlineto{\pgfqpoint{4.333463in}{2.163226in}}%
\pgfpathlineto{\pgfqpoint{4.341432in}{2.175351in}}%
\pgfpathlineto{\pgfqpoint{4.349395in}{2.187384in}}%
\pgfpathlineto{\pgfqpoint{4.335347in}{2.179771in}}%
\pgfpathlineto{\pgfqpoint{4.321313in}{2.172344in}}%
\pgfpathlineto{\pgfqpoint{4.307292in}{2.165102in}}%
\pgfpathlineto{\pgfqpoint{4.293284in}{2.158046in}}%
\pgfpathlineto{\pgfqpoint{4.285319in}{2.146152in}}%
\pgfpathlineto{\pgfqpoint{4.277348in}{2.134176in}}%
\pgfpathlineto{\pgfqpoint{4.269372in}{2.122118in}}%
\pgfpathlineto{\pgfqpoint{4.261391in}{2.109978in}}%
\pgfpathclose%
\pgfusepath{fill}%
\end{pgfscope}%
\begin{pgfscope}%
\pgfpathrectangle{\pgfqpoint{1.150000in}{0.150000in}}{\pgfqpoint{5.700000in}{5.700000in}}%
\pgfusepath{clip}%
\pgfsetbuttcap%
\pgfsetroundjoin%
\definecolor{currentfill}{rgb}{0.283187,0.125848,0.444960}%
\pgfsetfillcolor{currentfill}%
\pgfsetfillopacity{0.800000}%
\pgfsetlinewidth{0.000000pt}%
\definecolor{currentstroke}{rgb}{0.000000,0.000000,0.000000}%
\pgfsetstrokecolor{currentstroke}%
\pgfsetdash{}{0pt}%
\pgfpathmoveto{\pgfqpoint{2.751466in}{1.836123in}}%
\pgfpathlineto{\pgfqpoint{2.765356in}{1.822015in}}%
\pgfpathlineto{\pgfqpoint{2.779241in}{1.808149in}}%
\pgfpathlineto{\pgfqpoint{2.793121in}{1.794525in}}%
\pgfpathlineto{\pgfqpoint{2.806997in}{1.781141in}}%
\pgfpathlineto{\pgfqpoint{2.815691in}{1.782429in}}%
\pgfpathlineto{\pgfqpoint{2.824369in}{1.783980in}}%
\pgfpathlineto{\pgfqpoint{2.833033in}{1.785788in}}%
\pgfpathlineto{\pgfqpoint{2.841682in}{1.787848in}}%
\pgfpathlineto{\pgfqpoint{2.827847in}{1.800624in}}%
\pgfpathlineto{\pgfqpoint{2.814007in}{1.813638in}}%
\pgfpathlineto{\pgfqpoint{2.800163in}{1.826894in}}%
\pgfpathlineto{\pgfqpoint{2.786314in}{1.840391in}}%
\pgfpathlineto{\pgfqpoint{2.777625in}{1.838928in}}%
\pgfpathlineto{\pgfqpoint{2.768921in}{1.837725in}}%
\pgfpathlineto{\pgfqpoint{2.760201in}{1.836788in}}%
\pgfpathlineto{\pgfqpoint{2.751466in}{1.836123in}}%
\pgfpathclose%
\pgfusepath{fill}%
\end{pgfscope}%
\begin{pgfscope}%
\pgfpathrectangle{\pgfqpoint{1.150000in}{0.150000in}}{\pgfqpoint{5.700000in}{5.700000in}}%
\pgfusepath{clip}%
\pgfsetbuttcap%
\pgfsetroundjoin%
\definecolor{currentfill}{rgb}{0.272594,0.025563,0.353093}%
\pgfsetfillcolor{currentfill}%
\pgfsetfillopacity{0.800000}%
\pgfsetlinewidth{0.000000pt}%
\definecolor{currentstroke}{rgb}{0.000000,0.000000,0.000000}%
\pgfsetstrokecolor{currentstroke}%
\pgfsetdash{}{0pt}%
\pgfpathmoveto{\pgfqpoint{3.062674in}{1.614776in}}%
\pgfpathlineto{\pgfqpoint{3.076473in}{1.605852in}}%
\pgfpathlineto{\pgfqpoint{3.090272in}{1.597143in}}%
\pgfpathlineto{\pgfqpoint{3.104070in}{1.588649in}}%
\pgfpathlineto{\pgfqpoint{3.117869in}{1.580368in}}%
\pgfpathlineto{\pgfqpoint{3.126335in}{1.585581in}}%
\pgfpathlineto{\pgfqpoint{3.134791in}{1.590992in}}%
\pgfpathlineto{\pgfqpoint{3.143236in}{1.596594in}}%
\pgfpathlineto{\pgfqpoint{3.151670in}{1.602383in}}%
\pgfpathlineto{\pgfqpoint{3.137901in}{1.610099in}}%
\pgfpathlineto{\pgfqpoint{3.124131in}{1.618027in}}%
\pgfpathlineto{\pgfqpoint{3.110362in}{1.626170in}}%
\pgfpathlineto{\pgfqpoint{3.096593in}{1.634528in}}%
\pgfpathlineto{\pgfqpoint{3.088130in}{1.629293in}}%
\pgfpathlineto{\pgfqpoint{3.079656in}{1.624252in}}%
\pgfpathlineto{\pgfqpoint{3.071171in}{1.619411in}}%
\pgfpathlineto{\pgfqpoint{3.062674in}{1.614776in}}%
\pgfpathclose%
\pgfusepath{fill}%
\end{pgfscope}%
\begin{pgfscope}%
\pgfpathrectangle{\pgfqpoint{1.150000in}{0.150000in}}{\pgfqpoint{5.700000in}{5.700000in}}%
\pgfusepath{clip}%
\pgfsetbuttcap%
\pgfsetroundjoin%
\definecolor{currentfill}{rgb}{0.277134,0.185228,0.489898}%
\pgfsetfillcolor{currentfill}%
\pgfsetfillopacity{0.800000}%
\pgfsetlinewidth{0.000000pt}%
\definecolor{currentstroke}{rgb}{0.000000,0.000000,0.000000}%
\pgfsetstrokecolor{currentstroke}%
\pgfsetdash{}{0pt}%
\pgfpathmoveto{\pgfqpoint{4.053541in}{1.915184in}}%
\pgfpathlineto{\pgfqpoint{4.067463in}{1.919950in}}%
\pgfpathlineto{\pgfqpoint{4.081396in}{1.924903in}}%
\pgfpathlineto{\pgfqpoint{4.095340in}{1.930042in}}%
\pgfpathlineto{\pgfqpoint{4.109296in}{1.935366in}}%
\pgfpathlineto{\pgfqpoint{4.117338in}{1.948096in}}%
\pgfpathlineto{\pgfqpoint{4.125375in}{1.960773in}}%
\pgfpathlineto{\pgfqpoint{4.133407in}{1.973395in}}%
\pgfpathlineto{\pgfqpoint{4.141434in}{1.985961in}}%
\pgfpathlineto{\pgfqpoint{4.127481in}{1.980387in}}%
\pgfpathlineto{\pgfqpoint{4.113539in}{1.975000in}}%
\pgfpathlineto{\pgfqpoint{4.099609in}{1.969799in}}%
\pgfpathlineto{\pgfqpoint{4.085691in}{1.964784in}}%
\pgfpathlineto{\pgfqpoint{4.077660in}{1.952455in}}%
\pgfpathlineto{\pgfqpoint{4.069625in}{1.940078in}}%
\pgfpathlineto{\pgfqpoint{4.061585in}{1.927653in}}%
\pgfpathlineto{\pgfqpoint{4.053541in}{1.915184in}}%
\pgfpathclose%
\pgfusepath{fill}%
\end{pgfscope}%
\begin{pgfscope}%
\pgfpathrectangle{\pgfqpoint{1.150000in}{0.150000in}}{\pgfqpoint{5.700000in}{5.700000in}}%
\pgfusepath{clip}%
\pgfsetbuttcap%
\pgfsetroundjoin%
\definecolor{currentfill}{rgb}{0.243113,0.292092,0.538516}%
\pgfsetfillcolor{currentfill}%
\pgfsetfillopacity{0.800000}%
\pgfsetlinewidth{0.000000pt}%
\definecolor{currentstroke}{rgb}{0.000000,0.000000,0.000000}%
\pgfsetstrokecolor{currentstroke}%
\pgfsetdash{}{0pt}%
\pgfpathmoveto{\pgfqpoint{2.415960in}{2.252297in}}%
\pgfpathlineto{\pgfqpoint{2.430046in}{2.231795in}}%
\pgfpathlineto{\pgfqpoint{2.444120in}{2.211584in}}%
\pgfpathlineto{\pgfqpoint{2.458184in}{2.191663in}}%
\pgfpathlineto{\pgfqpoint{2.472237in}{2.172028in}}%
\pgfpathlineto{\pgfqpoint{2.481211in}{2.169525in}}%
\pgfpathlineto{\pgfqpoint{2.490167in}{2.167340in}}%
\pgfpathlineto{\pgfqpoint{2.499102in}{2.165466in}}%
\pgfpathlineto{\pgfqpoint{2.508018in}{2.163897in}}%
\pgfpathlineto{\pgfqpoint{2.494018in}{2.182901in}}%
\pgfpathlineto{\pgfqpoint{2.480007in}{2.202191in}}%
\pgfpathlineto{\pgfqpoint{2.465985in}{2.221768in}}%
\pgfpathlineto{\pgfqpoint{2.451953in}{2.241636in}}%
\pgfpathlineto{\pgfqpoint{2.442985in}{2.243824in}}%
\pgfpathlineto{\pgfqpoint{2.433997in}{2.246326in}}%
\pgfpathlineto{\pgfqpoint{2.424989in}{2.249148in}}%
\pgfpathlineto{\pgfqpoint{2.415960in}{2.252297in}}%
\pgfpathclose%
\pgfusepath{fill}%
\end{pgfscope}%
\begin{pgfscope}%
\pgfpathrectangle{\pgfqpoint{1.150000in}{0.150000in}}{\pgfqpoint{5.700000in}{5.700000in}}%
\pgfusepath{clip}%
\pgfsetbuttcap%
\pgfsetroundjoin%
\definecolor{currentfill}{rgb}{0.157851,0.683765,0.501686}%
\pgfsetfillcolor{currentfill}%
\pgfsetfillopacity{0.800000}%
\pgfsetlinewidth{0.000000pt}%
\definecolor{currentstroke}{rgb}{0.000000,0.000000,0.000000}%
\pgfsetstrokecolor{currentstroke}%
\pgfsetdash{}{0pt}%
\pgfpathmoveto{\pgfqpoint{5.652521in}{3.343498in}}%
\pgfpathlineto{\pgfqpoint{5.667350in}{3.357799in}}%
\pgfpathlineto{\pgfqpoint{5.682200in}{3.372282in}}%
\pgfpathlineto{\pgfqpoint{5.697072in}{3.386947in}}%
\pgfpathlineto{\pgfqpoint{5.711966in}{3.401795in}}%
\pgfpathlineto{\pgfqpoint{5.719222in}{3.403452in}}%
\pgfpathlineto{\pgfqpoint{5.726470in}{3.405064in}}%
\pgfpathlineto{\pgfqpoint{5.733710in}{3.406635in}}%
\pgfpathlineto{\pgfqpoint{5.740941in}{3.408171in}}%
\pgfpathlineto{\pgfqpoint{5.726074in}{3.393825in}}%
\pgfpathlineto{\pgfqpoint{5.711229in}{3.379662in}}%
\pgfpathlineto{\pgfqpoint{5.696405in}{3.365679in}}%
\pgfpathlineto{\pgfqpoint{5.681603in}{3.351877in}}%
\pgfpathlineto{\pgfqpoint{5.674345in}{3.349829in}}%
\pgfpathlineto{\pgfqpoint{5.667078in}{3.347753in}}%
\pgfpathlineto{\pgfqpoint{5.659804in}{3.345644in}}%
\pgfpathlineto{\pgfqpoint{5.652521in}{3.343498in}}%
\pgfpathclose%
\pgfusepath{fill}%
\end{pgfscope}%
\begin{pgfscope}%
\pgfpathrectangle{\pgfqpoint{1.150000in}{0.150000in}}{\pgfqpoint{5.700000in}{5.700000in}}%
\pgfusepath{clip}%
\pgfsetbuttcap%
\pgfsetroundjoin%
\definecolor{currentfill}{rgb}{0.282910,0.105393,0.426902}%
\pgfsetfillcolor{currentfill}%
\pgfsetfillopacity{0.800000}%
\pgfsetlinewidth{0.000000pt}%
\definecolor{currentstroke}{rgb}{0.000000,0.000000,0.000000}%
\pgfsetstrokecolor{currentstroke}%
\pgfsetdash{}{0pt}%
\pgfpathmoveto{\pgfqpoint{2.806997in}{1.781141in}}%
\pgfpathlineto{\pgfqpoint{2.820868in}{1.767994in}}%
\pgfpathlineto{\pgfqpoint{2.834735in}{1.755085in}}%
\pgfpathlineto{\pgfqpoint{2.848598in}{1.742410in}}%
\pgfpathlineto{\pgfqpoint{2.862458in}{1.729969in}}%
\pgfpathlineto{\pgfqpoint{2.871112in}{1.731877in}}%
\pgfpathlineto{\pgfqpoint{2.879752in}{1.734039in}}%
\pgfpathlineto{\pgfqpoint{2.888377in}{1.736451in}}%
\pgfpathlineto{\pgfqpoint{2.896989in}{1.739104in}}%
\pgfpathlineto{\pgfqpoint{2.883167in}{1.750939in}}%
\pgfpathlineto{\pgfqpoint{2.869342in}{1.763007in}}%
\pgfpathlineto{\pgfqpoint{2.855514in}{1.775309in}}%
\pgfpathlineto{\pgfqpoint{2.841682in}{1.787848in}}%
\pgfpathlineto{\pgfqpoint{2.833033in}{1.785788in}}%
\pgfpathlineto{\pgfqpoint{2.824369in}{1.783980in}}%
\pgfpathlineto{\pgfqpoint{2.815691in}{1.782429in}}%
\pgfpathlineto{\pgfqpoint{2.806997in}{1.781141in}}%
\pgfpathclose%
\pgfusepath{fill}%
\end{pgfscope}%
\begin{pgfscope}%
\pgfpathrectangle{\pgfqpoint{1.150000in}{0.150000in}}{\pgfqpoint{5.700000in}{5.700000in}}%
\pgfusepath{clip}%
\pgfsetbuttcap%
\pgfsetroundjoin%
\definecolor{currentfill}{rgb}{0.172719,0.448791,0.557885}%
\pgfsetfillcolor{currentfill}%
\pgfsetfillopacity{0.800000}%
\pgfsetlinewidth{0.000000pt}%
\definecolor{currentstroke}{rgb}{0.000000,0.000000,0.000000}%
\pgfsetstrokecolor{currentstroke}%
\pgfsetdash{}{0pt}%
\pgfpathmoveto{\pgfqpoint{2.168558in}{2.704140in}}%
\pgfpathlineto{\pgfqpoint{2.182874in}{2.677877in}}%
\pgfpathlineto{\pgfqpoint{2.197172in}{2.651964in}}%
\pgfpathlineto{\pgfqpoint{2.211453in}{2.626398in}}%
\pgfpathlineto{\pgfqpoint{2.225717in}{2.601174in}}%
\pgfpathlineto{\pgfqpoint{2.234891in}{2.596802in}}%
\pgfpathlineto{\pgfqpoint{2.244041in}{2.592770in}}%
\pgfpathlineto{\pgfqpoint{2.253169in}{2.589072in}}%
\pgfpathlineto{\pgfqpoint{2.262276in}{2.585701in}}%
\pgfpathlineto{\pgfqpoint{2.248071in}{2.610303in}}%
\pgfpathlineto{\pgfqpoint{2.233850in}{2.635247in}}%
\pgfpathlineto{\pgfqpoint{2.219613in}{2.660535in}}%
\pgfpathlineto{\pgfqpoint{2.205359in}{2.686172in}}%
\pgfpathlineto{\pgfqpoint{2.196193in}{2.690152in}}%
\pgfpathlineto{\pgfqpoint{2.187005in}{2.694469in}}%
\pgfpathlineto{\pgfqpoint{2.177793in}{2.699130in}}%
\pgfpathlineto{\pgfqpoint{2.168558in}{2.704140in}}%
\pgfpathclose%
\pgfusepath{fill}%
\end{pgfscope}%
\begin{pgfscope}%
\pgfpathrectangle{\pgfqpoint{1.150000in}{0.150000in}}{\pgfqpoint{5.700000in}{5.700000in}}%
\pgfusepath{clip}%
\pgfsetbuttcap%
\pgfsetroundjoin%
\definecolor{currentfill}{rgb}{0.277941,0.056324,0.381191}%
\pgfsetfillcolor{currentfill}%
\pgfsetfillopacity{0.800000}%
\pgfsetlinewidth{0.000000pt}%
\definecolor{currentstroke}{rgb}{0.000000,0.000000,0.000000}%
\pgfsetstrokecolor{currentstroke}%
\pgfsetdash{}{0pt}%
\pgfpathmoveto{\pgfqpoint{3.669752in}{1.637226in}}%
\pgfpathlineto{\pgfqpoint{3.683562in}{1.637260in}}%
\pgfpathlineto{\pgfqpoint{3.697379in}{1.637486in}}%
\pgfpathlineto{\pgfqpoint{3.711203in}{1.637902in}}%
\pgfpathlineto{\pgfqpoint{3.725034in}{1.638509in}}%
\pgfpathlineto{\pgfqpoint{3.733197in}{1.650037in}}%
\pgfpathlineto{\pgfqpoint{3.741353in}{1.661604in}}%
\pgfpathlineto{\pgfqpoint{3.749504in}{1.673208in}}%
\pgfpathlineto{\pgfqpoint{3.757650in}{1.684843in}}%
\pgfpathlineto{\pgfqpoint{3.743828in}{1.683831in}}%
\pgfpathlineto{\pgfqpoint{3.730013in}{1.683010in}}%
\pgfpathlineto{\pgfqpoint{3.716207in}{1.682380in}}%
\pgfpathlineto{\pgfqpoint{3.702407in}{1.681941in}}%
\pgfpathlineto{\pgfqpoint{3.694252in}{1.670698in}}%
\pgfpathlineto{\pgfqpoint{3.686091in}{1.659496in}}%
\pgfpathlineto{\pgfqpoint{3.677925in}{1.648337in}}%
\pgfpathlineto{\pgfqpoint{3.669752in}{1.637226in}}%
\pgfpathclose%
\pgfusepath{fill}%
\end{pgfscope}%
\begin{pgfscope}%
\pgfpathrectangle{\pgfqpoint{1.150000in}{0.150000in}}{\pgfqpoint{5.700000in}{5.700000in}}%
\pgfusepath{clip}%
\pgfsetbuttcap%
\pgfsetroundjoin%
\definecolor{currentfill}{rgb}{0.149039,0.508051,0.557250}%
\pgfsetfillcolor{currentfill}%
\pgfsetfillopacity{0.800000}%
\pgfsetlinewidth{0.000000pt}%
\definecolor{currentstroke}{rgb}{0.000000,0.000000,0.000000}%
\pgfsetstrokecolor{currentstroke}%
\pgfsetdash{}{0pt}%
\pgfpathmoveto{\pgfqpoint{4.973103in}{2.793061in}}%
\pgfpathlineto{\pgfqpoint{4.987515in}{2.805161in}}%
\pgfpathlineto{\pgfqpoint{5.001944in}{2.817445in}}%
\pgfpathlineto{\pgfqpoint{5.016393in}{2.829913in}}%
\pgfpathlineto{\pgfqpoint{5.030859in}{2.842565in}}%
\pgfpathlineto{\pgfqpoint{5.038549in}{2.850072in}}%
\pgfpathlineto{\pgfqpoint{5.046230in}{2.857452in}}%
\pgfpathlineto{\pgfqpoint{5.053903in}{2.864708in}}%
\pgfpathlineto{\pgfqpoint{5.061568in}{2.871842in}}%
\pgfpathlineto{\pgfqpoint{5.047111in}{2.859375in}}%
\pgfpathlineto{\pgfqpoint{5.032672in}{2.847092in}}%
\pgfpathlineto{\pgfqpoint{5.018251in}{2.834993in}}%
\pgfpathlineto{\pgfqpoint{5.003848in}{2.823078in}}%
\pgfpathlineto{\pgfqpoint{4.996174in}{2.815746in}}%
\pgfpathlineto{\pgfqpoint{4.988491in}{2.808301in}}%
\pgfpathlineto{\pgfqpoint{4.980801in}{2.800740in}}%
\pgfpathlineto{\pgfqpoint{4.973103in}{2.793061in}}%
\pgfpathclose%
\pgfusepath{fill}%
\end{pgfscope}%
\begin{pgfscope}%
\pgfpathrectangle{\pgfqpoint{1.150000in}{0.150000in}}{\pgfqpoint{5.700000in}{5.700000in}}%
\pgfusepath{clip}%
\pgfsetbuttcap%
\pgfsetroundjoin%
\definecolor{currentfill}{rgb}{0.273809,0.031497,0.358853}%
\pgfsetfillcolor{currentfill}%
\pgfsetfillopacity{0.800000}%
\pgfsetlinewidth{0.000000pt}%
\definecolor{currentstroke}{rgb}{0.000000,0.000000,0.000000}%
\pgfsetstrokecolor{currentstroke}%
\pgfsetdash{}{0pt}%
\pgfpathmoveto{\pgfqpoint{3.581787in}{1.596858in}}%
\pgfpathlineto{\pgfqpoint{3.595582in}{1.595687in}}%
\pgfpathlineto{\pgfqpoint{3.609383in}{1.594710in}}%
\pgfpathlineto{\pgfqpoint{3.623191in}{1.593924in}}%
\pgfpathlineto{\pgfqpoint{3.637005in}{1.593331in}}%
\pgfpathlineto{\pgfqpoint{3.645201in}{1.604214in}}%
\pgfpathlineto{\pgfqpoint{3.653391in}{1.615160in}}%
\pgfpathlineto{\pgfqpoint{3.661574in}{1.626166in}}%
\pgfpathlineto{\pgfqpoint{3.669752in}{1.637226in}}%
\pgfpathlineto{\pgfqpoint{3.655950in}{1.637383in}}%
\pgfpathlineto{\pgfqpoint{3.642154in}{1.637732in}}%
\pgfpathlineto{\pgfqpoint{3.628365in}{1.638274in}}%
\pgfpathlineto{\pgfqpoint{3.614582in}{1.639009in}}%
\pgfpathlineto{\pgfqpoint{3.606393in}{1.628373in}}%
\pgfpathlineto{\pgfqpoint{3.598197in}{1.617800in}}%
\pgfpathlineto{\pgfqpoint{3.589995in}{1.607293in}}%
\pgfpathlineto{\pgfqpoint{3.581787in}{1.596858in}}%
\pgfpathclose%
\pgfusepath{fill}%
\end{pgfscope}%
\begin{pgfscope}%
\pgfpathrectangle{\pgfqpoint{1.150000in}{0.150000in}}{\pgfqpoint{5.700000in}{5.700000in}}%
\pgfusepath{clip}%
\pgfsetbuttcap%
\pgfsetroundjoin%
\definecolor{currentfill}{rgb}{0.121148,0.592739,0.544641}%
\pgfsetfillcolor{currentfill}%
\pgfsetfillopacity{0.800000}%
\pgfsetlinewidth{0.000000pt}%
\definecolor{currentstroke}{rgb}{0.000000,0.000000,0.000000}%
\pgfsetstrokecolor{currentstroke}%
\pgfsetdash{}{0pt}%
\pgfpathmoveto{\pgfqpoint{5.268980in}{3.049175in}}%
\pgfpathlineto{\pgfqpoint{5.283575in}{3.062529in}}%
\pgfpathlineto{\pgfqpoint{5.298189in}{3.076068in}}%
\pgfpathlineto{\pgfqpoint{5.312824in}{3.089790in}}%
\pgfpathlineto{\pgfqpoint{5.327479in}{3.103695in}}%
\pgfpathlineto{\pgfqpoint{5.334996in}{3.108570in}}%
\pgfpathlineto{\pgfqpoint{5.342505in}{3.113341in}}%
\pgfpathlineto{\pgfqpoint{5.350004in}{3.118011in}}%
\pgfpathlineto{\pgfqpoint{5.357494in}{3.122584in}}%
\pgfpathlineto{\pgfqpoint{5.342855in}{3.109005in}}%
\pgfpathlineto{\pgfqpoint{5.328236in}{3.095608in}}%
\pgfpathlineto{\pgfqpoint{5.313637in}{3.082394in}}%
\pgfpathlineto{\pgfqpoint{5.299058in}{3.069363in}}%
\pgfpathlineto{\pgfqpoint{5.291551in}{3.064452in}}%
\pgfpathlineto{\pgfqpoint{5.284036in}{3.059453in}}%
\pgfpathlineto{\pgfqpoint{5.276512in}{3.054362in}}%
\pgfpathlineto{\pgfqpoint{5.268980in}{3.049175in}}%
\pgfpathclose%
\pgfusepath{fill}%
\end{pgfscope}%
\begin{pgfscope}%
\pgfpathrectangle{\pgfqpoint{1.150000in}{0.150000in}}{\pgfqpoint{5.700000in}{5.700000in}}%
\pgfusepath{clip}%
\pgfsetbuttcap%
\pgfsetroundjoin%
\definecolor{currentfill}{rgb}{0.281446,0.084320,0.407414}%
\pgfsetfillcolor{currentfill}%
\pgfsetfillopacity{0.800000}%
\pgfsetlinewidth{0.000000pt}%
\definecolor{currentstroke}{rgb}{0.000000,0.000000,0.000000}%
\pgfsetstrokecolor{currentstroke}%
\pgfsetdash{}{0pt}%
\pgfpathmoveto{\pgfqpoint{3.757650in}{1.684843in}}%
\pgfpathlineto{\pgfqpoint{3.771480in}{1.686044in}}%
\pgfpathlineto{\pgfqpoint{3.785319in}{1.687435in}}%
\pgfpathlineto{\pgfqpoint{3.799166in}{1.689015in}}%
\pgfpathlineto{\pgfqpoint{3.813021in}{1.690784in}}%
\pgfpathlineto{\pgfqpoint{3.821153in}{1.702833in}}%
\pgfpathlineto{\pgfqpoint{3.829281in}{1.714899in}}%
\pgfpathlineto{\pgfqpoint{3.837403in}{1.726979in}}%
\pgfpathlineto{\pgfqpoint{3.845520in}{1.739068in}}%
\pgfpathlineto{\pgfqpoint{3.831673in}{1.736925in}}%
\pgfpathlineto{\pgfqpoint{3.817834in}{1.734970in}}%
\pgfpathlineto{\pgfqpoint{3.804003in}{1.733206in}}%
\pgfpathlineto{\pgfqpoint{3.790181in}{1.731630in}}%
\pgfpathlineto{\pgfqpoint{3.782056in}{1.719903in}}%
\pgfpathlineto{\pgfqpoint{3.773926in}{1.708194in}}%
\pgfpathlineto{\pgfqpoint{3.765791in}{1.696506in}}%
\pgfpathlineto{\pgfqpoint{3.757650in}{1.684843in}}%
\pgfpathclose%
\pgfusepath{fill}%
\end{pgfscope}%
\begin{pgfscope}%
\pgfpathrectangle{\pgfqpoint{1.150000in}{0.150000in}}{\pgfqpoint{5.700000in}{5.700000in}}%
\pgfusepath{clip}%
\pgfsetbuttcap%
\pgfsetroundjoin%
\definecolor{currentfill}{rgb}{0.227802,0.326594,0.546532}%
\pgfsetfillcolor{currentfill}%
\pgfsetfillopacity{0.800000}%
\pgfsetlinewidth{0.000000pt}%
\definecolor{currentstroke}{rgb}{0.000000,0.000000,0.000000}%
\pgfsetstrokecolor{currentstroke}%
\pgfsetdash{}{0pt}%
\pgfpathmoveto{\pgfqpoint{2.359497in}{2.337272in}}%
\pgfpathlineto{\pgfqpoint{2.373631in}{2.315578in}}%
\pgfpathlineto{\pgfqpoint{2.387753in}{2.294186in}}%
\pgfpathlineto{\pgfqpoint{2.401862in}{2.273093in}}%
\pgfpathlineto{\pgfqpoint{2.415960in}{2.252297in}}%
\pgfpathlineto{\pgfqpoint{2.424989in}{2.249148in}}%
\pgfpathlineto{\pgfqpoint{2.433997in}{2.246326in}}%
\pgfpathlineto{\pgfqpoint{2.442985in}{2.243824in}}%
\pgfpathlineto{\pgfqpoint{2.451953in}{2.241636in}}%
\pgfpathlineto{\pgfqpoint{2.437910in}{2.261796in}}%
\pgfpathlineto{\pgfqpoint{2.423855in}{2.282251in}}%
\pgfpathlineto{\pgfqpoint{2.409789in}{2.303005in}}%
\pgfpathlineto{\pgfqpoint{2.395711in}{2.324059in}}%
\pgfpathlineto{\pgfqpoint{2.386689in}{2.326871in}}%
\pgfpathlineto{\pgfqpoint{2.377646in}{2.330006in}}%
\pgfpathlineto{\pgfqpoint{2.368582in}{2.333471in}}%
\pgfpathlineto{\pgfqpoint{2.359497in}{2.337272in}}%
\pgfpathclose%
\pgfusepath{fill}%
\end{pgfscope}%
\begin{pgfscope}%
\pgfpathrectangle{\pgfqpoint{1.150000in}{0.150000in}}{\pgfqpoint{5.700000in}{5.700000in}}%
\pgfusepath{clip}%
\pgfsetbuttcap%
\pgfsetroundjoin%
\definecolor{currentfill}{rgb}{0.185783,0.704891,0.485273}%
\pgfsetfillcolor{currentfill}%
\pgfsetfillopacity{0.800000}%
\pgfsetlinewidth{0.000000pt}%
\definecolor{currentstroke}{rgb}{0.000000,0.000000,0.000000}%
\pgfsetstrokecolor{currentstroke}%
\pgfsetdash{}{0pt}%
\pgfpathmoveto{\pgfqpoint{5.740941in}{3.408171in}}%
\pgfpathlineto{\pgfqpoint{5.755830in}{3.422698in}}%
\pgfpathlineto{\pgfqpoint{5.770741in}{3.437406in}}%
\pgfpathlineto{\pgfqpoint{5.785675in}{3.452297in}}%
\pgfpathlineto{\pgfqpoint{5.800631in}{3.467371in}}%
\pgfpathlineto{\pgfqpoint{5.807826in}{3.468352in}}%
\pgfpathlineto{\pgfqpoint{5.815012in}{3.469303in}}%
\pgfpathlineto{\pgfqpoint{5.822191in}{3.470227in}}%
\pgfpathlineto{\pgfqpoint{5.829361in}{3.471132in}}%
\pgfpathlineto{\pgfqpoint{5.814435in}{3.456597in}}%
\pgfpathlineto{\pgfqpoint{5.799531in}{3.442244in}}%
\pgfpathlineto{\pgfqpoint{5.784649in}{3.428071in}}%
\pgfpathlineto{\pgfqpoint{5.769789in}{3.414079in}}%
\pgfpathlineto{\pgfqpoint{5.762588in}{3.412626in}}%
\pgfpathlineto{\pgfqpoint{5.755380in}{3.411161in}}%
\pgfpathlineto{\pgfqpoint{5.748164in}{3.409677in}}%
\pgfpathlineto{\pgfqpoint{5.740941in}{3.408171in}}%
\pgfpathclose%
\pgfusepath{fill}%
\end{pgfscope}%
\begin{pgfscope}%
\pgfpathrectangle{\pgfqpoint{1.150000in}{0.150000in}}{\pgfqpoint{5.700000in}{5.700000in}}%
\pgfusepath{clip}%
\pgfsetbuttcap%
\pgfsetroundjoin%
\definecolor{currentfill}{rgb}{0.281446,0.084320,0.407414}%
\pgfsetfillcolor{currentfill}%
\pgfsetfillopacity{0.800000}%
\pgfsetlinewidth{0.000000pt}%
\definecolor{currentstroke}{rgb}{0.000000,0.000000,0.000000}%
\pgfsetstrokecolor{currentstroke}%
\pgfsetdash{}{0pt}%
\pgfpathmoveto{\pgfqpoint{2.862458in}{1.729969in}}%
\pgfpathlineto{\pgfqpoint{2.876314in}{1.717761in}}%
\pgfpathlineto{\pgfqpoint{2.890167in}{1.705783in}}%
\pgfpathlineto{\pgfqpoint{2.904017in}{1.694034in}}%
\pgfpathlineto{\pgfqpoint{2.917864in}{1.682514in}}%
\pgfpathlineto{\pgfqpoint{2.926481in}{1.685039in}}%
\pgfpathlineto{\pgfqpoint{2.935083in}{1.687811in}}%
\pgfpathlineto{\pgfqpoint{2.943672in}{1.690822in}}%
\pgfpathlineto{\pgfqpoint{2.952247in}{1.694068in}}%
\pgfpathlineto{\pgfqpoint{2.938436in}{1.704984in}}%
\pgfpathlineto{\pgfqpoint{2.924623in}{1.716128in}}%
\pgfpathlineto{\pgfqpoint{2.910807in}{1.727501in}}%
\pgfpathlineto{\pgfqpoint{2.896989in}{1.739104in}}%
\pgfpathlineto{\pgfqpoint{2.888377in}{1.736451in}}%
\pgfpathlineto{\pgfqpoint{2.879752in}{1.734039in}}%
\pgfpathlineto{\pgfqpoint{2.871112in}{1.731877in}}%
\pgfpathlineto{\pgfqpoint{2.862458in}{1.729969in}}%
\pgfpathclose%
\pgfusepath{fill}%
\end{pgfscope}%
\begin{pgfscope}%
\pgfpathrectangle{\pgfqpoint{1.150000in}{0.150000in}}{\pgfqpoint{5.700000in}{5.700000in}}%
\pgfusepath{clip}%
\pgfsetbuttcap%
\pgfsetroundjoin%
\definecolor{currentfill}{rgb}{0.269944,0.014625,0.341379}%
\pgfsetfillcolor{currentfill}%
\pgfsetfillopacity{0.800000}%
\pgfsetlinewidth{0.000000pt}%
\definecolor{currentstroke}{rgb}{0.000000,0.000000,0.000000}%
\pgfsetstrokecolor{currentstroke}%
\pgfsetdash{}{0pt}%
\pgfpathmoveto{\pgfqpoint{3.493709in}{1.564407in}}%
\pgfpathlineto{\pgfqpoint{3.507497in}{1.561991in}}%
\pgfpathlineto{\pgfqpoint{3.521289in}{1.559771in}}%
\pgfpathlineto{\pgfqpoint{3.535087in}{1.557746in}}%
\pgfpathlineto{\pgfqpoint{3.548890in}{1.555914in}}%
\pgfpathlineto{\pgfqpoint{3.557124in}{1.566022in}}%
\pgfpathlineto{\pgfqpoint{3.565352in}{1.576218in}}%
\pgfpathlineto{\pgfqpoint{3.573572in}{1.586498in}}%
\pgfpathlineto{\pgfqpoint{3.581787in}{1.596858in}}%
\pgfpathlineto{\pgfqpoint{3.567998in}{1.598222in}}%
\pgfpathlineto{\pgfqpoint{3.554214in}{1.599781in}}%
\pgfpathlineto{\pgfqpoint{3.540437in}{1.601534in}}%
\pgfpathlineto{\pgfqpoint{3.526665in}{1.603482in}}%
\pgfpathlineto{\pgfqpoint{3.518436in}{1.593577in}}%
\pgfpathlineto{\pgfqpoint{3.510201in}{1.583760in}}%
\pgfpathlineto{\pgfqpoint{3.501959in}{1.574035in}}%
\pgfpathlineto{\pgfqpoint{3.493709in}{1.564407in}}%
\pgfpathclose%
\pgfusepath{fill}%
\end{pgfscope}%
\begin{pgfscope}%
\pgfpathrectangle{\pgfqpoint{1.150000in}{0.150000in}}{\pgfqpoint{5.700000in}{5.700000in}}%
\pgfusepath{clip}%
\pgfsetbuttcap%
\pgfsetroundjoin%
\definecolor{currentfill}{rgb}{0.239346,0.300855,0.540844}%
\pgfsetfillcolor{currentfill}%
\pgfsetfillopacity{0.800000}%
\pgfsetlinewidth{0.000000pt}%
\definecolor{currentstroke}{rgb}{0.000000,0.000000,0.000000}%
\pgfsetstrokecolor{currentstroke}%
\pgfsetdash{}{0pt}%
\pgfpathmoveto{\pgfqpoint{4.349395in}{2.187384in}}%
\pgfpathlineto{\pgfqpoint{4.363456in}{2.195182in}}%
\pgfpathlineto{\pgfqpoint{4.377532in}{2.203165in}}%
\pgfpathlineto{\pgfqpoint{4.391621in}{2.211333in}}%
\pgfpathlineto{\pgfqpoint{4.405724in}{2.219686in}}%
\pgfpathlineto{\pgfqpoint{4.413681in}{2.231759in}}%
\pgfpathlineto{\pgfqpoint{4.421632in}{2.243730in}}%
\pgfpathlineto{\pgfqpoint{4.429577in}{2.255600in}}%
\pgfpathlineto{\pgfqpoint{4.437517in}{2.267366in}}%
\pgfpathlineto{\pgfqpoint{4.423416in}{2.258893in}}%
\pgfpathlineto{\pgfqpoint{4.409328in}{2.250605in}}%
\pgfpathlineto{\pgfqpoint{4.395255in}{2.242503in}}%
\pgfpathlineto{\pgfqpoint{4.381195in}{2.234585in}}%
\pgfpathlineto{\pgfqpoint{4.373253in}{2.222926in}}%
\pgfpathlineto{\pgfqpoint{4.365306in}{2.211172in}}%
\pgfpathlineto{\pgfqpoint{4.357353in}{2.199325in}}%
\pgfpathlineto{\pgfqpoint{4.349395in}{2.187384in}}%
\pgfpathclose%
\pgfusepath{fill}%
\end{pgfscope}%
\begin{pgfscope}%
\pgfpathrectangle{\pgfqpoint{1.150000in}{0.150000in}}{\pgfqpoint{5.700000in}{5.700000in}}%
\pgfusepath{clip}%
\pgfsetbuttcap%
\pgfsetroundjoin%
\definecolor{currentfill}{rgb}{0.283091,0.110553,0.431554}%
\pgfsetfillcolor{currentfill}%
\pgfsetfillopacity{0.800000}%
\pgfsetlinewidth{0.000000pt}%
\definecolor{currentstroke}{rgb}{0.000000,0.000000,0.000000}%
\pgfsetstrokecolor{currentstroke}%
\pgfsetdash{}{0pt}%
\pgfpathmoveto{\pgfqpoint{3.845520in}{1.739068in}}%
\pgfpathlineto{\pgfqpoint{3.859377in}{1.741399in}}%
\pgfpathlineto{\pgfqpoint{3.873243in}{1.743919in}}%
\pgfpathlineto{\pgfqpoint{3.887118in}{1.746627in}}%
\pgfpathlineto{\pgfqpoint{3.901003in}{1.749522in}}%
\pgfpathlineto{\pgfqpoint{3.909109in}{1.761973in}}%
\pgfpathlineto{\pgfqpoint{3.917210in}{1.774420in}}%
\pgfpathlineto{\pgfqpoint{3.925307in}{1.786859in}}%
\pgfpathlineto{\pgfqpoint{3.933398in}{1.799287in}}%
\pgfpathlineto{\pgfqpoint{3.919519in}{1.796048in}}%
\pgfpathlineto{\pgfqpoint{3.905650in}{1.792997in}}%
\pgfpathlineto{\pgfqpoint{3.891791in}{1.790134in}}%
\pgfpathlineto{\pgfqpoint{3.877940in}{1.787460in}}%
\pgfpathlineto{\pgfqpoint{3.869843in}{1.775363in}}%
\pgfpathlineto{\pgfqpoint{3.861740in}{1.763263in}}%
\pgfpathlineto{\pgfqpoint{3.853633in}{1.751164in}}%
\pgfpathlineto{\pgfqpoint{3.845520in}{1.739068in}}%
\pgfpathclose%
\pgfusepath{fill}%
\end{pgfscope}%
\begin{pgfscope}%
\pgfpathrectangle{\pgfqpoint{1.150000in}{0.150000in}}{\pgfqpoint{5.700000in}{5.700000in}}%
\pgfusepath{clip}%
\pgfsetbuttcap%
\pgfsetroundjoin%
\definecolor{currentfill}{rgb}{0.172719,0.448791,0.557885}%
\pgfsetfillcolor{currentfill}%
\pgfsetfillopacity{0.800000}%
\pgfsetlinewidth{0.000000pt}%
\definecolor{currentstroke}{rgb}{0.000000,0.000000,0.000000}%
\pgfsetstrokecolor{currentstroke}%
\pgfsetdash{}{0pt}%
\pgfpathmoveto{\pgfqpoint{4.765319in}{2.598009in}}%
\pgfpathlineto{\pgfqpoint{4.779609in}{2.608982in}}%
\pgfpathlineto{\pgfqpoint{4.793916in}{2.620140in}}%
\pgfpathlineto{\pgfqpoint{4.808241in}{2.631483in}}%
\pgfpathlineto{\pgfqpoint{4.822583in}{2.643009in}}%
\pgfpathlineto{\pgfqpoint{4.830377in}{2.652326in}}%
\pgfpathlineto{\pgfqpoint{4.838165in}{2.661512in}}%
\pgfpathlineto{\pgfqpoint{4.845946in}{2.670568in}}%
\pgfpathlineto{\pgfqpoint{4.853719in}{2.679497in}}%
\pgfpathlineto{\pgfqpoint{4.839382in}{2.668052in}}%
\pgfpathlineto{\pgfqpoint{4.825063in}{2.656791in}}%
\pgfpathlineto{\pgfqpoint{4.810761in}{2.645714in}}%
\pgfpathlineto{\pgfqpoint{4.796476in}{2.634822in}}%
\pgfpathlineto{\pgfqpoint{4.788697in}{2.625799in}}%
\pgfpathlineto{\pgfqpoint{4.780911in}{2.616657in}}%
\pgfpathlineto{\pgfqpoint{4.773119in}{2.607394in}}%
\pgfpathlineto{\pgfqpoint{4.765319in}{2.598009in}}%
\pgfpathclose%
\pgfusepath{fill}%
\end{pgfscope}%
\begin{pgfscope}%
\pgfpathrectangle{\pgfqpoint{1.150000in}{0.150000in}}{\pgfqpoint{5.700000in}{5.700000in}}%
\pgfusepath{clip}%
\pgfsetbuttcap%
\pgfsetroundjoin%
\definecolor{currentfill}{rgb}{0.269308,0.218818,0.509577}%
\pgfsetfillcolor{currentfill}%
\pgfsetfillopacity{0.800000}%
\pgfsetlinewidth{0.000000pt}%
\definecolor{currentstroke}{rgb}{0.000000,0.000000,0.000000}%
\pgfsetstrokecolor{currentstroke}%
\pgfsetdash{}{0pt}%
\pgfpathmoveto{\pgfqpoint{4.141434in}{1.985961in}}%
\pgfpathlineto{\pgfqpoint{4.155399in}{1.991720in}}%
\pgfpathlineto{\pgfqpoint{4.169376in}{1.997666in}}%
\pgfpathlineto{\pgfqpoint{4.183365in}{2.003797in}}%
\pgfpathlineto{\pgfqpoint{4.197366in}{2.010114in}}%
\pgfpathlineto{\pgfqpoint{4.205387in}{2.022850in}}%
\pgfpathlineto{\pgfqpoint{4.213402in}{2.035517in}}%
\pgfpathlineto{\pgfqpoint{4.221413in}{2.048113in}}%
\pgfpathlineto{\pgfqpoint{4.229418in}{2.060637in}}%
\pgfpathlineto{\pgfqpoint{4.215419in}{2.054103in}}%
\pgfpathlineto{\pgfqpoint{4.201432in}{2.047755in}}%
\pgfpathlineto{\pgfqpoint{4.187457in}{2.041592in}}%
\pgfpathlineto{\pgfqpoint{4.173495in}{2.035616in}}%
\pgfpathlineto{\pgfqpoint{4.165487in}{2.023297in}}%
\pgfpathlineto{\pgfqpoint{4.157474in}{2.010913in}}%
\pgfpathlineto{\pgfqpoint{4.149456in}{1.998467in}}%
\pgfpathlineto{\pgfqpoint{4.141434in}{1.985961in}}%
\pgfpathclose%
\pgfusepath{fill}%
\end{pgfscope}%
\begin{pgfscope}%
\pgfpathrectangle{\pgfqpoint{1.150000in}{0.150000in}}{\pgfqpoint{5.700000in}{5.700000in}}%
\pgfusepath{clip}%
\pgfsetbuttcap%
\pgfsetroundjoin%
\definecolor{currentfill}{rgb}{0.267004,0.004874,0.329415}%
\pgfsetfillcolor{currentfill}%
\pgfsetfillopacity{0.800000}%
\pgfsetlinewidth{0.000000pt}%
\definecolor{currentstroke}{rgb}{0.000000,0.000000,0.000000}%
\pgfsetstrokecolor{currentstroke}%
\pgfsetdash{}{0pt}%
\pgfpathmoveto{\pgfqpoint{3.261869in}{1.548216in}}%
\pgfpathlineto{\pgfqpoint{3.275652in}{1.542375in}}%
\pgfpathlineto{\pgfqpoint{3.289437in}{1.536738in}}%
\pgfpathlineto{\pgfqpoint{3.303224in}{1.531305in}}%
\pgfpathlineto{\pgfqpoint{3.317015in}{1.526073in}}%
\pgfpathlineto{\pgfqpoint{3.325365in}{1.533675in}}%
\pgfpathlineto{\pgfqpoint{3.333706in}{1.541427in}}%
\pgfpathlineto{\pgfqpoint{3.342038in}{1.549325in}}%
\pgfpathlineto{\pgfqpoint{3.350362in}{1.557365in}}%
\pgfpathlineto{\pgfqpoint{3.336594in}{1.562065in}}%
\pgfpathlineto{\pgfqpoint{3.322829in}{1.566968in}}%
\pgfpathlineto{\pgfqpoint{3.309066in}{1.572073in}}%
\pgfpathlineto{\pgfqpoint{3.295307in}{1.577383in}}%
\pgfpathlineto{\pgfqpoint{3.286961in}{1.569862in}}%
\pgfpathlineto{\pgfqpoint{3.278606in}{1.562491in}}%
\pgfpathlineto{\pgfqpoint{3.270242in}{1.555274in}}%
\pgfpathlineto{\pgfqpoint{3.261869in}{1.548216in}}%
\pgfpathclose%
\pgfusepath{fill}%
\end{pgfscope}%
\begin{pgfscope}%
\pgfpathrectangle{\pgfqpoint{1.150000in}{0.150000in}}{\pgfqpoint{5.700000in}{5.700000in}}%
\pgfusepath{clip}%
\pgfsetbuttcap%
\pgfsetroundjoin%
\definecolor{currentfill}{rgb}{0.203063,0.379716,0.553925}%
\pgfsetfillcolor{currentfill}%
\pgfsetfillopacity{0.800000}%
\pgfsetlinewidth{0.000000pt}%
\definecolor{currentstroke}{rgb}{0.000000,0.000000,0.000000}%
\pgfsetstrokecolor{currentstroke}%
\pgfsetdash{}{0pt}%
\pgfpathmoveto{\pgfqpoint{4.557370in}{2.394034in}}%
\pgfpathlineto{\pgfqpoint{4.571542in}{2.403571in}}%
\pgfpathlineto{\pgfqpoint{4.585730in}{2.413293in}}%
\pgfpathlineto{\pgfqpoint{4.599933in}{2.423200in}}%
\pgfpathlineto{\pgfqpoint{4.614151in}{2.433292in}}%
\pgfpathlineto{\pgfqpoint{4.622034in}{2.444179in}}%
\pgfpathlineto{\pgfqpoint{4.629911in}{2.454943in}}%
\pgfpathlineto{\pgfqpoint{4.637782in}{2.465586in}}%
\pgfpathlineto{\pgfqpoint{4.645646in}{2.476107in}}%
\pgfpathlineto{\pgfqpoint{4.631430in}{2.465995in}}%
\pgfpathlineto{\pgfqpoint{4.617230in}{2.456068in}}%
\pgfpathlineto{\pgfqpoint{4.603045in}{2.446325in}}%
\pgfpathlineto{\pgfqpoint{4.588876in}{2.436767in}}%
\pgfpathlineto{\pgfqpoint{4.581009in}{2.426254in}}%
\pgfpathlineto{\pgfqpoint{4.573136in}{2.415628in}}%
\pgfpathlineto{\pgfqpoint{4.565256in}{2.404888in}}%
\pgfpathlineto{\pgfqpoint{4.557370in}{2.394034in}}%
\pgfpathclose%
\pgfusepath{fill}%
\end{pgfscope}%
\begin{pgfscope}%
\pgfpathrectangle{\pgfqpoint{1.150000in}{0.150000in}}{\pgfqpoint{5.700000in}{5.700000in}}%
\pgfusepath{clip}%
\pgfsetbuttcap%
\pgfsetroundjoin%
\definecolor{currentfill}{rgb}{0.269944,0.014625,0.341379}%
\pgfsetfillcolor{currentfill}%
\pgfsetfillopacity{0.800000}%
\pgfsetlinewidth{0.000000pt}%
\definecolor{currentstroke}{rgb}{0.000000,0.000000,0.000000}%
\pgfsetstrokecolor{currentstroke}%
\pgfsetdash{}{0pt}%
\pgfpathmoveto{\pgfqpoint{3.117869in}{1.580368in}}%
\pgfpathlineto{\pgfqpoint{3.131667in}{1.572299in}}%
\pgfpathlineto{\pgfqpoint{3.145467in}{1.564442in}}%
\pgfpathlineto{\pgfqpoint{3.159266in}{1.556795in}}%
\pgfpathlineto{\pgfqpoint{3.173067in}{1.549358in}}%
\pgfpathlineto{\pgfqpoint{3.181505in}{1.555149in}}%
\pgfpathlineto{\pgfqpoint{3.189933in}{1.561128in}}%
\pgfpathlineto{\pgfqpoint{3.198351in}{1.567292in}}%
\pgfpathlineto{\pgfqpoint{3.206759in}{1.573633in}}%
\pgfpathlineto{\pgfqpoint{3.192985in}{1.580506in}}%
\pgfpathlineto{\pgfqpoint{3.179212in}{1.587588in}}%
\pgfpathlineto{\pgfqpoint{3.165441in}{1.594880in}}%
\pgfpathlineto{\pgfqpoint{3.151670in}{1.602383in}}%
\pgfpathlineto{\pgfqpoint{3.143236in}{1.596594in}}%
\pgfpathlineto{\pgfqpoint{3.134791in}{1.590992in}}%
\pgfpathlineto{\pgfqpoint{3.126335in}{1.585581in}}%
\pgfpathlineto{\pgfqpoint{3.117869in}{1.580368in}}%
\pgfpathclose%
\pgfusepath{fill}%
\end{pgfscope}%
\begin{pgfscope}%
\pgfpathrectangle{\pgfqpoint{1.150000in}{0.150000in}}{\pgfqpoint{5.700000in}{5.700000in}}%
\pgfusepath{clip}%
\pgfsetbuttcap%
\pgfsetroundjoin%
\definecolor{currentfill}{rgb}{0.220124,0.725509,0.466226}%
\pgfsetfillcolor{currentfill}%
\pgfsetfillopacity{0.800000}%
\pgfsetlinewidth{0.000000pt}%
\definecolor{currentstroke}{rgb}{0.000000,0.000000,0.000000}%
\pgfsetstrokecolor{currentstroke}%
\pgfsetdash{}{0pt}%
\pgfpathmoveto{\pgfqpoint{5.829361in}{3.471132in}}%
\pgfpathlineto{\pgfqpoint{5.844310in}{3.485848in}}%
\pgfpathlineto{\pgfqpoint{5.859281in}{3.500746in}}%
\pgfpathlineto{\pgfqpoint{5.874276in}{3.515826in}}%
\pgfpathlineto{\pgfqpoint{5.889293in}{3.531088in}}%
\pgfpathlineto{\pgfqpoint{5.896424in}{3.531418in}}%
\pgfpathlineto{\pgfqpoint{5.903548in}{3.531732in}}%
\pgfpathlineto{\pgfqpoint{5.910664in}{3.532037in}}%
\pgfpathlineto{\pgfqpoint{5.917773in}{3.532339in}}%
\pgfpathlineto{\pgfqpoint{5.902788in}{3.517651in}}%
\pgfpathlineto{\pgfqpoint{5.887826in}{3.503145in}}%
\pgfpathlineto{\pgfqpoint{5.872887in}{3.488819in}}%
\pgfpathlineto{\pgfqpoint{5.857970in}{3.474673in}}%
\pgfpathlineto{\pgfqpoint{5.850828in}{3.473787in}}%
\pgfpathlineto{\pgfqpoint{5.843680in}{3.472906in}}%
\pgfpathlineto{\pgfqpoint{5.836524in}{3.472023in}}%
\pgfpathlineto{\pgfqpoint{5.829361in}{3.471132in}}%
\pgfpathclose%
\pgfusepath{fill}%
\end{pgfscope}%
\begin{pgfscope}%
\pgfpathrectangle{\pgfqpoint{1.150000in}{0.150000in}}{\pgfqpoint{5.700000in}{5.700000in}}%
\pgfusepath{clip}%
\pgfsetbuttcap%
\pgfsetroundjoin%
\definecolor{currentfill}{rgb}{0.119699,0.618490,0.536347}%
\pgfsetfillcolor{currentfill}%
\pgfsetfillopacity{0.800000}%
\pgfsetlinewidth{0.000000pt}%
\definecolor{currentstroke}{rgb}{0.000000,0.000000,0.000000}%
\pgfsetstrokecolor{currentstroke}%
\pgfsetdash{}{0pt}%
\pgfpathmoveto{\pgfqpoint{5.357494in}{3.122584in}}%
\pgfpathlineto{\pgfqpoint{5.372154in}{3.136347in}}%
\pgfpathlineto{\pgfqpoint{5.386834in}{3.150293in}}%
\pgfpathlineto{\pgfqpoint{5.401535in}{3.164423in}}%
\pgfpathlineto{\pgfqpoint{5.416257in}{3.178736in}}%
\pgfpathlineto{\pgfqpoint{5.423721in}{3.182868in}}%
\pgfpathlineto{\pgfqpoint{5.431177in}{3.186903in}}%
\pgfpathlineto{\pgfqpoint{5.438624in}{3.190845in}}%
\pgfpathlineto{\pgfqpoint{5.446061in}{3.194698in}}%
\pgfpathlineto{\pgfqpoint{5.431358in}{3.180746in}}%
\pgfpathlineto{\pgfqpoint{5.416675in}{3.166978in}}%
\pgfpathlineto{\pgfqpoint{5.402013in}{3.153392in}}%
\pgfpathlineto{\pgfqpoint{5.387371in}{3.139989in}}%
\pgfpathlineto{\pgfqpoint{5.379914in}{3.135763in}}%
\pgfpathlineto{\pgfqpoint{5.372450in}{3.131456in}}%
\pgfpathlineto{\pgfqpoint{5.364976in}{3.127065in}}%
\pgfpathlineto{\pgfqpoint{5.357494in}{3.122584in}}%
\pgfpathclose%
\pgfusepath{fill}%
\end{pgfscope}%
\begin{pgfscope}%
\pgfpathrectangle{\pgfqpoint{1.150000in}{0.150000in}}{\pgfqpoint{5.700000in}{5.700000in}}%
\pgfusepath{clip}%
\pgfsetbuttcap%
\pgfsetroundjoin%
\definecolor{currentfill}{rgb}{0.282623,0.140926,0.457517}%
\pgfsetfillcolor{currentfill}%
\pgfsetfillopacity{0.800000}%
\pgfsetlinewidth{0.000000pt}%
\definecolor{currentstroke}{rgb}{0.000000,0.000000,0.000000}%
\pgfsetstrokecolor{currentstroke}%
\pgfsetdash{}{0pt}%
\pgfpathmoveto{\pgfqpoint{3.933398in}{1.799287in}}%
\pgfpathlineto{\pgfqpoint{3.947287in}{1.802713in}}%
\pgfpathlineto{\pgfqpoint{3.961186in}{1.806326in}}%
\pgfpathlineto{\pgfqpoint{3.975095in}{1.810126in}}%
\pgfpathlineto{\pgfqpoint{3.989014in}{1.814113in}}%
\pgfpathlineto{\pgfqpoint{3.997096in}{1.826851in}}%
\pgfpathlineto{\pgfqpoint{4.005174in}{1.839565in}}%
\pgfpathlineto{\pgfqpoint{4.013246in}{1.852252in}}%
\pgfpathlineto{\pgfqpoint{4.021315in}{1.864909in}}%
\pgfpathlineto{\pgfqpoint{4.007400in}{1.860610in}}%
\pgfpathlineto{\pgfqpoint{3.993495in}{1.856498in}}%
\pgfpathlineto{\pgfqpoint{3.979601in}{1.852572in}}%
\pgfpathlineto{\pgfqpoint{3.965717in}{1.848835in}}%
\pgfpathlineto{\pgfqpoint{3.957645in}{1.836478in}}%
\pgfpathlineto{\pgfqpoint{3.949567in}{1.824099in}}%
\pgfpathlineto{\pgfqpoint{3.941485in}{1.811701in}}%
\pgfpathlineto{\pgfqpoint{3.933398in}{1.799287in}}%
\pgfpathclose%
\pgfusepath{fill}%
\end{pgfscope}%
\begin{pgfscope}%
\pgfpathrectangle{\pgfqpoint{1.150000in}{0.150000in}}{\pgfqpoint{5.700000in}{5.700000in}}%
\pgfusepath{clip}%
\pgfsetbuttcap%
\pgfsetroundjoin%
\definecolor{currentfill}{rgb}{0.212395,0.359683,0.551710}%
\pgfsetfillcolor{currentfill}%
\pgfsetfillopacity{0.800000}%
\pgfsetlinewidth{0.000000pt}%
\definecolor{currentstroke}{rgb}{0.000000,0.000000,0.000000}%
\pgfsetstrokecolor{currentstroke}%
\pgfsetdash{}{0pt}%
\pgfpathmoveto{\pgfqpoint{2.302828in}{2.427121in}}%
\pgfpathlineto{\pgfqpoint{2.317016in}{2.404192in}}%
\pgfpathlineto{\pgfqpoint{2.331190in}{2.381576in}}%
\pgfpathlineto{\pgfqpoint{2.345350in}{2.359270in}}%
\pgfpathlineto{\pgfqpoint{2.359497in}{2.337272in}}%
\pgfpathlineto{\pgfqpoint{2.368582in}{2.333471in}}%
\pgfpathlineto{\pgfqpoint{2.377646in}{2.330006in}}%
\pgfpathlineto{\pgfqpoint{2.386689in}{2.326871in}}%
\pgfpathlineto{\pgfqpoint{2.395711in}{2.324059in}}%
\pgfpathlineto{\pgfqpoint{2.381620in}{2.345416in}}%
\pgfpathlineto{\pgfqpoint{2.367517in}{2.367079in}}%
\pgfpathlineto{\pgfqpoint{2.353401in}{2.389051in}}%
\pgfpathlineto{\pgfqpoint{2.339271in}{2.411334in}}%
\pgfpathlineto{\pgfqpoint{2.330193in}{2.414775in}}%
\pgfpathlineto{\pgfqpoint{2.321094in}{2.418549in}}%
\pgfpathlineto{\pgfqpoint{2.311972in}{2.422662in}}%
\pgfpathlineto{\pgfqpoint{2.302828in}{2.427121in}}%
\pgfpathclose%
\pgfusepath{fill}%
\end{pgfscope}%
\begin{pgfscope}%
\pgfpathrectangle{\pgfqpoint{1.150000in}{0.150000in}}{\pgfqpoint{5.700000in}{5.700000in}}%
\pgfusepath{clip}%
\pgfsetbuttcap%
\pgfsetroundjoin%
\definecolor{currentfill}{rgb}{0.279566,0.067836,0.391917}%
\pgfsetfillcolor{currentfill}%
\pgfsetfillopacity{0.800000}%
\pgfsetlinewidth{0.000000pt}%
\definecolor{currentstroke}{rgb}{0.000000,0.000000,0.000000}%
\pgfsetstrokecolor{currentstroke}%
\pgfsetdash{}{0pt}%
\pgfpathmoveto{\pgfqpoint{2.917864in}{1.682514in}}%
\pgfpathlineto{\pgfqpoint{2.931709in}{1.671220in}}%
\pgfpathlineto{\pgfqpoint{2.945551in}{1.660152in}}%
\pgfpathlineto{\pgfqpoint{2.959391in}{1.649308in}}%
\pgfpathlineto{\pgfqpoint{2.973229in}{1.638687in}}%
\pgfpathlineto{\pgfqpoint{2.981809in}{1.641828in}}%
\pgfpathlineto{\pgfqpoint{2.990376in}{1.645206in}}%
\pgfpathlineto{\pgfqpoint{2.998930in}{1.648816in}}%
\pgfpathlineto{\pgfqpoint{3.007471in}{1.652651in}}%
\pgfpathlineto{\pgfqpoint{2.993668in}{1.662671in}}%
\pgfpathlineto{\pgfqpoint{2.979863in}{1.672912in}}%
\pgfpathlineto{\pgfqpoint{2.966056in}{1.683377in}}%
\pgfpathlineto{\pgfqpoint{2.952247in}{1.694068in}}%
\pgfpathlineto{\pgfqpoint{2.943672in}{1.690822in}}%
\pgfpathlineto{\pgfqpoint{2.935083in}{1.687811in}}%
\pgfpathlineto{\pgfqpoint{2.926481in}{1.685039in}}%
\pgfpathlineto{\pgfqpoint{2.917864in}{1.682514in}}%
\pgfpathclose%
\pgfusepath{fill}%
\end{pgfscope}%
\begin{pgfscope}%
\pgfpathrectangle{\pgfqpoint{1.150000in}{0.150000in}}{\pgfqpoint{5.700000in}{5.700000in}}%
\pgfusepath{clip}%
\pgfsetbuttcap%
\pgfsetroundjoin%
\definecolor{currentfill}{rgb}{0.268510,0.009605,0.335427}%
\pgfsetfillcolor{currentfill}%
\pgfsetfillopacity{0.800000}%
\pgfsetlinewidth{0.000000pt}%
\definecolor{currentstroke}{rgb}{0.000000,0.000000,0.000000}%
\pgfsetstrokecolor{currentstroke}%
\pgfsetdash{}{0pt}%
\pgfpathmoveto{\pgfqpoint{3.405470in}{1.540569in}}%
\pgfpathlineto{\pgfqpoint{3.419257in}{1.536868in}}%
\pgfpathlineto{\pgfqpoint{3.433047in}{1.533364in}}%
\pgfpathlineto{\pgfqpoint{3.446842in}{1.530058in}}%
\pgfpathlineto{\pgfqpoint{3.460641in}{1.526948in}}%
\pgfpathlineto{\pgfqpoint{3.468919in}{1.536145in}}%
\pgfpathlineto{\pgfqpoint{3.477190in}{1.545457in}}%
\pgfpathlineto{\pgfqpoint{3.485453in}{1.554879in}}%
\pgfpathlineto{\pgfqpoint{3.493709in}{1.564407in}}%
\pgfpathlineto{\pgfqpoint{3.479927in}{1.567018in}}%
\pgfpathlineto{\pgfqpoint{3.466150in}{1.569825in}}%
\pgfpathlineto{\pgfqpoint{3.452377in}{1.572830in}}%
\pgfpathlineto{\pgfqpoint{3.438609in}{1.576032in}}%
\pgfpathlineto{\pgfqpoint{3.430336in}{1.566991in}}%
\pgfpathlineto{\pgfqpoint{3.422055in}{1.558064in}}%
\pgfpathlineto{\pgfqpoint{3.413766in}{1.549255in}}%
\pgfpathlineto{\pgfqpoint{3.405470in}{1.540569in}}%
\pgfpathclose%
\pgfusepath{fill}%
\end{pgfscope}%
\begin{pgfscope}%
\pgfpathrectangle{\pgfqpoint{1.150000in}{0.150000in}}{\pgfqpoint{5.700000in}{5.700000in}}%
\pgfusepath{clip}%
\pgfsetbuttcap%
\pgfsetroundjoin%
\definecolor{currentfill}{rgb}{0.288921,0.758394,0.428426}%
\pgfsetfillcolor{currentfill}%
\pgfsetfillopacity{0.800000}%
\pgfsetlinewidth{0.000000pt}%
\definecolor{currentstroke}{rgb}{0.000000,0.000000,0.000000}%
\pgfsetstrokecolor{currentstroke}%
\pgfsetdash{}{0pt}%
\pgfpathmoveto{\pgfqpoint{6.006166in}{3.591773in}}%
\pgfpathlineto{\pgfqpoint{6.021230in}{3.606757in}}%
\pgfpathlineto{\pgfqpoint{6.036318in}{3.621923in}}%
\pgfpathlineto{\pgfqpoint{6.051429in}{3.637271in}}%
\pgfpathlineto{\pgfqpoint{6.058440in}{3.636551in}}%
\pgfpathlineto{\pgfqpoint{6.065445in}{3.635854in}}%
\pgfpathlineto{\pgfqpoint{6.072443in}{3.635187in}}%
\pgfpathlineto{\pgfqpoint{6.079435in}{3.634555in}}%
\pgfpathlineto{\pgfqpoint{6.064362in}{3.619852in}}%
\pgfpathlineto{\pgfqpoint{6.049312in}{3.605330in}}%
\pgfpathlineto{\pgfqpoint{6.034285in}{3.590987in}}%
\pgfpathlineto{\pgfqpoint{6.027264in}{3.591128in}}%
\pgfpathlineto{\pgfqpoint{6.020238in}{3.591310in}}%
\pgfpathlineto{\pgfqpoint{6.013205in}{3.591527in}}%
\pgfpathlineto{\pgfqpoint{6.006166in}{3.591773in}}%
\pgfpathclose%
\pgfusepath{fill}%
\end{pgfscope}%
\begin{pgfscope}%
\pgfpathrectangle{\pgfqpoint{1.150000in}{0.150000in}}{\pgfqpoint{5.700000in}{5.700000in}}%
\pgfusepath{clip}%
\pgfsetbuttcap%
\pgfsetroundjoin%
\definecolor{currentfill}{rgb}{0.137770,0.537492,0.554906}%
\pgfsetfillcolor{currentfill}%
\pgfsetfillopacity{0.800000}%
\pgfsetlinewidth{0.000000pt}%
\definecolor{currentstroke}{rgb}{0.000000,0.000000,0.000000}%
\pgfsetstrokecolor{currentstroke}%
\pgfsetdash{}{0pt}%
\pgfpathmoveto{\pgfqpoint{5.061568in}{2.871842in}}%
\pgfpathlineto{\pgfqpoint{5.076045in}{2.884493in}}%
\pgfpathlineto{\pgfqpoint{5.090540in}{2.897327in}}%
\pgfpathlineto{\pgfqpoint{5.105055in}{2.910346in}}%
\pgfpathlineto{\pgfqpoint{5.119588in}{2.923549in}}%
\pgfpathlineto{\pgfqpoint{5.127235in}{2.930356in}}%
\pgfpathlineto{\pgfqpoint{5.134874in}{2.937037in}}%
\pgfpathlineto{\pgfqpoint{5.142504in}{2.943596in}}%
\pgfpathlineto{\pgfqpoint{5.150125in}{2.950035in}}%
\pgfpathlineto{\pgfqpoint{5.135602in}{2.937053in}}%
\pgfpathlineto{\pgfqpoint{5.121098in}{2.924254in}}%
\pgfpathlineto{\pgfqpoint{5.106614in}{2.911640in}}%
\pgfpathlineto{\pgfqpoint{5.092148in}{2.899208in}}%
\pgfpathlineto{\pgfqpoint{5.084515in}{2.892537in}}%
\pgfpathlineto{\pgfqpoint{5.076874in}{2.885754in}}%
\pgfpathlineto{\pgfqpoint{5.069225in}{2.878856in}}%
\pgfpathlineto{\pgfqpoint{5.061568in}{2.871842in}}%
\pgfpathclose%
\pgfusepath{fill}%
\end{pgfscope}%
\begin{pgfscope}%
\pgfpathrectangle{\pgfqpoint{1.150000in}{0.150000in}}{\pgfqpoint{5.700000in}{5.700000in}}%
\pgfusepath{clip}%
\pgfsetbuttcap%
\pgfsetroundjoin%
\definecolor{currentfill}{rgb}{0.157729,0.485932,0.558013}%
\pgfsetfillcolor{currentfill}%
\pgfsetfillopacity{0.800000}%
\pgfsetlinewidth{0.000000pt}%
\definecolor{currentstroke}{rgb}{0.000000,0.000000,0.000000}%
\pgfsetstrokecolor{currentstroke}%
\pgfsetdash{}{0pt}%
\pgfpathmoveto{\pgfqpoint{2.111111in}{2.812764in}}%
\pgfpathlineto{\pgfqpoint{2.125501in}{2.785065in}}%
\pgfpathlineto{\pgfqpoint{2.139872in}{2.757731in}}%
\pgfpathlineto{\pgfqpoint{2.154224in}{2.730757in}}%
\pgfpathlineto{\pgfqpoint{2.168558in}{2.704140in}}%
\pgfpathlineto{\pgfqpoint{2.177793in}{2.699130in}}%
\pgfpathlineto{\pgfqpoint{2.187005in}{2.694469in}}%
\pgfpathlineto{\pgfqpoint{2.196193in}{2.690152in}}%
\pgfpathlineto{\pgfqpoint{2.205359in}{2.686172in}}%
\pgfpathlineto{\pgfqpoint{2.191087in}{2.712160in}}%
\pgfpathlineto{\pgfqpoint{2.176798in}{2.738504in}}%
\pgfpathlineto{\pgfqpoint{2.162491in}{2.765206in}}%
\pgfpathlineto{\pgfqpoint{2.148165in}{2.792272in}}%
\pgfpathlineto{\pgfqpoint{2.138937in}{2.796868in}}%
\pgfpathlineto{\pgfqpoint{2.129686in}{2.801812in}}%
\pgfpathlineto{\pgfqpoint{2.120411in}{2.807108in}}%
\pgfpathlineto{\pgfqpoint{2.111111in}{2.812764in}}%
\pgfpathclose%
\pgfusepath{fill}%
\end{pgfscope}%
\begin{pgfscope}%
\pgfpathrectangle{\pgfqpoint{1.150000in}{0.150000in}}{\pgfqpoint{5.700000in}{5.700000in}}%
\pgfusepath{clip}%
\pgfsetbuttcap%
\pgfsetroundjoin%
\definecolor{currentfill}{rgb}{0.259857,0.745492,0.444467}%
\pgfsetfillcolor{currentfill}%
\pgfsetfillopacity{0.800000}%
\pgfsetlinewidth{0.000000pt}%
\definecolor{currentstroke}{rgb}{0.000000,0.000000,0.000000}%
\pgfsetstrokecolor{currentstroke}%
\pgfsetdash{}{0pt}%
\pgfpathmoveto{\pgfqpoint{5.917773in}{3.532339in}}%
\pgfpathlineto{\pgfqpoint{5.932780in}{3.547208in}}%
\pgfpathlineto{\pgfqpoint{5.947810in}{3.562258in}}%
\pgfpathlineto{\pgfqpoint{5.962864in}{3.577490in}}%
\pgfpathlineto{\pgfqpoint{5.977940in}{3.592904in}}%
\pgfpathlineto{\pgfqpoint{5.985008in}{3.592612in}}%
\pgfpathlineto{\pgfqpoint{5.992067in}{3.592322in}}%
\pgfpathlineto{\pgfqpoint{5.999120in}{3.592040in}}%
\pgfpathlineto{\pgfqpoint{6.006166in}{3.591773in}}%
\pgfpathlineto{\pgfqpoint{5.991124in}{3.576969in}}%
\pgfpathlineto{\pgfqpoint{5.976106in}{3.562346in}}%
\pgfpathlineto{\pgfqpoint{5.961111in}{3.547903in}}%
\pgfpathlineto{\pgfqpoint{5.946138in}{3.533641in}}%
\pgfpathlineto{\pgfqpoint{5.939057in}{3.533288in}}%
\pgfpathlineto{\pgfqpoint{5.931969in}{3.532958in}}%
\pgfpathlineto{\pgfqpoint{5.924874in}{3.532644in}}%
\pgfpathlineto{\pgfqpoint{5.917773in}{3.532339in}}%
\pgfpathclose%
\pgfusepath{fill}%
\end{pgfscope}%
\begin{pgfscope}%
\pgfpathrectangle{\pgfqpoint{1.150000in}{0.150000in}}{\pgfqpoint{5.700000in}{5.700000in}}%
\pgfusepath{clip}%
\pgfsetbuttcap%
\pgfsetroundjoin%
\definecolor{currentfill}{rgb}{0.257322,0.256130,0.526563}%
\pgfsetfillcolor{currentfill}%
\pgfsetfillopacity{0.800000}%
\pgfsetlinewidth{0.000000pt}%
\definecolor{currentstroke}{rgb}{0.000000,0.000000,0.000000}%
\pgfsetstrokecolor{currentstroke}%
\pgfsetdash{}{0pt}%
\pgfpathmoveto{\pgfqpoint{4.229418in}{2.060637in}}%
\pgfpathlineto{\pgfqpoint{4.243430in}{2.067356in}}%
\pgfpathlineto{\pgfqpoint{4.257455in}{2.074261in}}%
\pgfpathlineto{\pgfqpoint{4.271493in}{2.081351in}}%
\pgfpathlineto{\pgfqpoint{4.285544in}{2.088626in}}%
\pgfpathlineto{\pgfqpoint{4.293543in}{2.101273in}}%
\pgfpathlineto{\pgfqpoint{4.301537in}{2.113837in}}%
\pgfpathlineto{\pgfqpoint{4.309526in}{2.126315in}}%
\pgfpathlineto{\pgfqpoint{4.317510in}{2.138707in}}%
\pgfpathlineto{\pgfqpoint{4.303461in}{2.131247in}}%
\pgfpathlineto{\pgfqpoint{4.289425in}{2.123972in}}%
\pgfpathlineto{\pgfqpoint{4.275402in}{2.116882in}}%
\pgfpathlineto{\pgfqpoint{4.261391in}{2.109978in}}%
\pgfpathlineto{\pgfqpoint{4.253406in}{2.097759in}}%
\pgfpathlineto{\pgfqpoint{4.245415in}{2.085462in}}%
\pgfpathlineto{\pgfqpoint{4.237419in}{2.073087in}}%
\pgfpathlineto{\pgfqpoint{4.229418in}{2.060637in}}%
\pgfpathclose%
\pgfusepath{fill}%
\end{pgfscope}%
\begin{pgfscope}%
\pgfpathrectangle{\pgfqpoint{1.150000in}{0.150000in}}{\pgfqpoint{5.700000in}{5.700000in}}%
\pgfusepath{clip}%
\pgfsetbuttcap%
\pgfsetroundjoin%
\definecolor{currentfill}{rgb}{0.221989,0.339161,0.548752}%
\pgfsetfillcolor{currentfill}%
\pgfsetfillopacity{0.800000}%
\pgfsetlinewidth{0.000000pt}%
\definecolor{currentstroke}{rgb}{0.000000,0.000000,0.000000}%
\pgfsetstrokecolor{currentstroke}%
\pgfsetdash{}{0pt}%
\pgfpathmoveto{\pgfqpoint{4.437517in}{2.267366in}}%
\pgfpathlineto{\pgfqpoint{4.451634in}{2.276024in}}%
\pgfpathlineto{\pgfqpoint{4.465764in}{2.284866in}}%
\pgfpathlineto{\pgfqpoint{4.479910in}{2.293894in}}%
\pgfpathlineto{\pgfqpoint{4.494071in}{2.303106in}}%
\pgfpathlineto{\pgfqpoint{4.502004in}{2.314868in}}%
\pgfpathlineto{\pgfqpoint{4.509931in}{2.326518in}}%
\pgfpathlineto{\pgfqpoint{4.517852in}{2.338055in}}%
\pgfpathlineto{\pgfqpoint{4.525768in}{2.349478in}}%
\pgfpathlineto{\pgfqpoint{4.511609in}{2.340179in}}%
\pgfpathlineto{\pgfqpoint{4.497465in}{2.331065in}}%
\pgfpathlineto{\pgfqpoint{4.483336in}{2.322135in}}%
\pgfpathlineto{\pgfqpoint{4.469222in}{2.313390in}}%
\pgfpathlineto{\pgfqpoint{4.461304in}{2.302041in}}%
\pgfpathlineto{\pgfqpoint{4.453381in}{2.290587in}}%
\pgfpathlineto{\pgfqpoint{4.445452in}{2.279029in}}%
\pgfpathlineto{\pgfqpoint{4.437517in}{2.267366in}}%
\pgfpathclose%
\pgfusepath{fill}%
\end{pgfscope}%
\begin{pgfscope}%
\pgfpathrectangle{\pgfqpoint{1.150000in}{0.150000in}}{\pgfqpoint{5.700000in}{5.700000in}}%
\pgfusepath{clip}%
\pgfsetbuttcap%
\pgfsetroundjoin%
\definecolor{currentfill}{rgb}{0.160665,0.478540,0.558115}%
\pgfsetfillcolor{currentfill}%
\pgfsetfillopacity{0.800000}%
\pgfsetlinewidth{0.000000pt}%
\definecolor{currentstroke}{rgb}{0.000000,0.000000,0.000000}%
\pgfsetstrokecolor{currentstroke}%
\pgfsetdash{}{0pt}%
\pgfpathmoveto{\pgfqpoint{4.853719in}{2.679497in}}%
\pgfpathlineto{\pgfqpoint{4.868073in}{2.691126in}}%
\pgfpathlineto{\pgfqpoint{4.882444in}{2.702940in}}%
\pgfpathlineto{\pgfqpoint{4.896834in}{2.714938in}}%
\pgfpathlineto{\pgfqpoint{4.911241in}{2.727121in}}%
\pgfpathlineto{\pgfqpoint{4.919001in}{2.735819in}}%
\pgfpathlineto{\pgfqpoint{4.926753in}{2.744384in}}%
\pgfpathlineto{\pgfqpoint{4.934497in}{2.752816in}}%
\pgfpathlineto{\pgfqpoint{4.942234in}{2.761118in}}%
\pgfpathlineto{\pgfqpoint{4.927833in}{2.749052in}}%
\pgfpathlineto{\pgfqpoint{4.913450in}{2.737170in}}%
\pgfpathlineto{\pgfqpoint{4.899085in}{2.725472in}}%
\pgfpathlineto{\pgfqpoint{4.884737in}{2.713959in}}%
\pgfpathlineto{\pgfqpoint{4.876994in}{2.705528in}}%
\pgfpathlineto{\pgfqpoint{4.869243in}{2.696975in}}%
\pgfpathlineto{\pgfqpoint{4.861484in}{2.688298in}}%
\pgfpathlineto{\pgfqpoint{4.853719in}{2.679497in}}%
\pgfpathclose%
\pgfusepath{fill}%
\end{pgfscope}%
\begin{pgfscope}%
\pgfpathrectangle{\pgfqpoint{1.150000in}{0.150000in}}{\pgfqpoint{5.700000in}{5.700000in}}%
\pgfusepath{clip}%
\pgfsetbuttcap%
\pgfsetroundjoin%
\definecolor{currentfill}{rgb}{0.278826,0.175490,0.483397}%
\pgfsetfillcolor{currentfill}%
\pgfsetfillopacity{0.800000}%
\pgfsetlinewidth{0.000000pt}%
\definecolor{currentstroke}{rgb}{0.000000,0.000000,0.000000}%
\pgfsetstrokecolor{currentstroke}%
\pgfsetdash{}{0pt}%
\pgfpathmoveto{\pgfqpoint{4.021315in}{1.864909in}}%
\pgfpathlineto{\pgfqpoint{4.035240in}{1.869394in}}%
\pgfpathlineto{\pgfqpoint{4.049177in}{1.874066in}}%
\pgfpathlineto{\pgfqpoint{4.063125in}{1.878924in}}%
\pgfpathlineto{\pgfqpoint{4.077084in}{1.883967in}}%
\pgfpathlineto{\pgfqpoint{4.085144in}{1.896884in}}%
\pgfpathlineto{\pgfqpoint{4.093199in}{1.909758in}}%
\pgfpathlineto{\pgfqpoint{4.101250in}{1.922586in}}%
\pgfpathlineto{\pgfqpoint{4.109296in}{1.935366in}}%
\pgfpathlineto{\pgfqpoint{4.095340in}{1.930042in}}%
\pgfpathlineto{\pgfqpoint{4.081396in}{1.924903in}}%
\pgfpathlineto{\pgfqpoint{4.067463in}{1.919950in}}%
\pgfpathlineto{\pgfqpoint{4.053541in}{1.915184in}}%
\pgfpathlineto{\pgfqpoint{4.045491in}{1.902673in}}%
\pgfpathlineto{\pgfqpoint{4.037437in}{1.890122in}}%
\pgfpathlineto{\pgfqpoint{4.029378in}{1.877533in}}%
\pgfpathlineto{\pgfqpoint{4.021315in}{1.864909in}}%
\pgfpathclose%
\pgfusepath{fill}%
\end{pgfscope}%
\begin{pgfscope}%
\pgfpathrectangle{\pgfqpoint{1.150000in}{0.150000in}}{\pgfqpoint{5.700000in}{5.700000in}}%
\pgfusepath{clip}%
\pgfsetbuttcap%
\pgfsetroundjoin%
\definecolor{currentfill}{rgb}{0.188923,0.410910,0.556326}%
\pgfsetfillcolor{currentfill}%
\pgfsetfillopacity{0.800000}%
\pgfsetlinewidth{0.000000pt}%
\definecolor{currentstroke}{rgb}{0.000000,0.000000,0.000000}%
\pgfsetstrokecolor{currentstroke}%
\pgfsetdash{}{0pt}%
\pgfpathmoveto{\pgfqpoint{4.645646in}{2.476107in}}%
\pgfpathlineto{\pgfqpoint{4.659878in}{2.486403in}}%
\pgfpathlineto{\pgfqpoint{4.674126in}{2.496884in}}%
\pgfpathlineto{\pgfqpoint{4.688391in}{2.507550in}}%
\pgfpathlineto{\pgfqpoint{4.702672in}{2.518400in}}%
\pgfpathlineto{\pgfqpoint{4.710527in}{2.528799in}}%
\pgfpathlineto{\pgfqpoint{4.718375in}{2.539068in}}%
\pgfpathlineto{\pgfqpoint{4.726216in}{2.549209in}}%
\pgfpathlineto{\pgfqpoint{4.734050in}{2.559222in}}%
\pgfpathlineto{\pgfqpoint{4.719772in}{2.548385in}}%
\pgfpathlineto{\pgfqpoint{4.705511in}{2.537733in}}%
\pgfpathlineto{\pgfqpoint{4.691266in}{2.527266in}}%
\pgfpathlineto{\pgfqpoint{4.677037in}{2.516983in}}%
\pgfpathlineto{\pgfqpoint{4.669199in}{2.506944in}}%
\pgfpathlineto{\pgfqpoint{4.661355in}{2.496785in}}%
\pgfpathlineto{\pgfqpoint{4.653503in}{2.486506in}}%
\pgfpathlineto{\pgfqpoint{4.645646in}{2.476107in}}%
\pgfpathclose%
\pgfusepath{fill}%
\end{pgfscope}%
\begin{pgfscope}%
\pgfpathrectangle{\pgfqpoint{1.150000in}{0.150000in}}{\pgfqpoint{5.700000in}{5.700000in}}%
\pgfusepath{clip}%
\pgfsetbuttcap%
\pgfsetroundjoin%
\definecolor{currentfill}{rgb}{0.126326,0.644107,0.525311}%
\pgfsetfillcolor{currentfill}%
\pgfsetfillopacity{0.800000}%
\pgfsetlinewidth{0.000000pt}%
\definecolor{currentstroke}{rgb}{0.000000,0.000000,0.000000}%
\pgfsetstrokecolor{currentstroke}%
\pgfsetdash{}{0pt}%
\pgfpathmoveto{\pgfqpoint{5.446061in}{3.194698in}}%
\pgfpathlineto{\pgfqpoint{5.460786in}{3.208832in}}%
\pgfpathlineto{\pgfqpoint{5.475531in}{3.223150in}}%
\pgfpathlineto{\pgfqpoint{5.490298in}{3.237651in}}%
\pgfpathlineto{\pgfqpoint{5.505086in}{3.252336in}}%
\pgfpathlineto{\pgfqpoint{5.512495in}{3.255720in}}%
\pgfpathlineto{\pgfqpoint{5.519896in}{3.259015in}}%
\pgfpathlineto{\pgfqpoint{5.527287in}{3.262226in}}%
\pgfpathlineto{\pgfqpoint{5.534670in}{3.265357in}}%
\pgfpathlineto{\pgfqpoint{5.519902in}{3.251071in}}%
\pgfpathlineto{\pgfqpoint{5.505156in}{3.236967in}}%
\pgfpathlineto{\pgfqpoint{5.490430in}{3.223045in}}%
\pgfpathlineto{\pgfqpoint{5.475726in}{3.209306in}}%
\pgfpathlineto{\pgfqpoint{5.468322in}{3.205766in}}%
\pgfpathlineto{\pgfqpoint{5.460911in}{3.202154in}}%
\pgfpathlineto{\pgfqpoint{5.453490in}{3.198466in}}%
\pgfpathlineto{\pgfqpoint{5.446061in}{3.194698in}}%
\pgfpathclose%
\pgfusepath{fill}%
\end{pgfscope}%
\begin{pgfscope}%
\pgfpathrectangle{\pgfqpoint{1.150000in}{0.150000in}}{\pgfqpoint{5.700000in}{5.700000in}}%
\pgfusepath{clip}%
\pgfsetbuttcap%
\pgfsetroundjoin%
\definecolor{currentfill}{rgb}{0.277018,0.050344,0.375715}%
\pgfsetfillcolor{currentfill}%
\pgfsetfillopacity{0.800000}%
\pgfsetlinewidth{0.000000pt}%
\definecolor{currentstroke}{rgb}{0.000000,0.000000,0.000000}%
\pgfsetstrokecolor{currentstroke}%
\pgfsetdash{}{0pt}%
\pgfpathmoveto{\pgfqpoint{2.973229in}{1.638687in}}%
\pgfpathlineto{\pgfqpoint{2.987065in}{1.628288in}}%
\pgfpathlineto{\pgfqpoint{3.000900in}{1.618110in}}%
\pgfpathlineto{\pgfqpoint{3.014733in}{1.608150in}}%
\pgfpathlineto{\pgfqpoint{3.028565in}{1.598409in}}%
\pgfpathlineto{\pgfqpoint{3.037111in}{1.602163in}}%
\pgfpathlineto{\pgfqpoint{3.045644in}{1.606146in}}%
\pgfpathlineto{\pgfqpoint{3.054165in}{1.610353in}}%
\pgfpathlineto{\pgfqpoint{3.062674in}{1.614776in}}%
\pgfpathlineto{\pgfqpoint{3.048875in}{1.623917in}}%
\pgfpathlineto{\pgfqpoint{3.035074in}{1.633276in}}%
\pgfpathlineto{\pgfqpoint{3.021273in}{1.642854in}}%
\pgfpathlineto{\pgfqpoint{3.007471in}{1.652651in}}%
\pgfpathlineto{\pgfqpoint{2.998930in}{1.648816in}}%
\pgfpathlineto{\pgfqpoint{2.990376in}{1.645206in}}%
\pgfpathlineto{\pgfqpoint{2.981809in}{1.641828in}}%
\pgfpathlineto{\pgfqpoint{2.973229in}{1.638687in}}%
\pgfpathclose%
\pgfusepath{fill}%
\end{pgfscope}%
\begin{pgfscope}%
\pgfpathrectangle{\pgfqpoint{1.150000in}{0.150000in}}{\pgfqpoint{5.700000in}{5.700000in}}%
\pgfusepath{clip}%
\pgfsetbuttcap%
\pgfsetroundjoin%
\definecolor{currentfill}{rgb}{0.195860,0.395433,0.555276}%
\pgfsetfillcolor{currentfill}%
\pgfsetfillopacity{0.800000}%
\pgfsetlinewidth{0.000000pt}%
\definecolor{currentstroke}{rgb}{0.000000,0.000000,0.000000}%
\pgfsetstrokecolor{currentstroke}%
\pgfsetdash{}{0pt}%
\pgfpathmoveto{\pgfqpoint{2.245933in}{2.522027in}}%
\pgfpathlineto{\pgfqpoint{2.260179in}{2.497816in}}%
\pgfpathlineto{\pgfqpoint{2.274410in}{2.473930in}}%
\pgfpathlineto{\pgfqpoint{2.288627in}{2.450366in}}%
\pgfpathlineto{\pgfqpoint{2.302828in}{2.427121in}}%
\pgfpathlineto{\pgfqpoint{2.311972in}{2.422662in}}%
\pgfpathlineto{\pgfqpoint{2.321094in}{2.418549in}}%
\pgfpathlineto{\pgfqpoint{2.330193in}{2.414775in}}%
\pgfpathlineto{\pgfqpoint{2.339271in}{2.411334in}}%
\pgfpathlineto{\pgfqpoint{2.325128in}{2.433932in}}%
\pgfpathlineto{\pgfqpoint{2.310971in}{2.456847in}}%
\pgfpathlineto{\pgfqpoint{2.296800in}{2.480082in}}%
\pgfpathlineto{\pgfqpoint{2.282614in}{2.503641in}}%
\pgfpathlineto{\pgfqpoint{2.273478in}{2.507717in}}%
\pgfpathlineto{\pgfqpoint{2.264319in}{2.512136in}}%
\pgfpathlineto{\pgfqpoint{2.255138in}{2.516904in}}%
\pgfpathlineto{\pgfqpoint{2.245933in}{2.522027in}}%
\pgfpathclose%
\pgfusepath{fill}%
\end{pgfscope}%
\begin{pgfscope}%
\pgfpathrectangle{\pgfqpoint{1.150000in}{0.150000in}}{\pgfqpoint{5.700000in}{5.700000in}}%
\pgfusepath{clip}%
\pgfsetbuttcap%
\pgfsetroundjoin%
\definecolor{currentfill}{rgb}{0.268510,0.009605,0.335427}%
\pgfsetfillcolor{currentfill}%
\pgfsetfillopacity{0.800000}%
\pgfsetlinewidth{0.000000pt}%
\definecolor{currentstroke}{rgb}{0.000000,0.000000,0.000000}%
\pgfsetstrokecolor{currentstroke}%
\pgfsetdash{}{0pt}%
\pgfpathmoveto{\pgfqpoint{3.173067in}{1.549358in}}%
\pgfpathlineto{\pgfqpoint{3.186869in}{1.542129in}}%
\pgfpathlineto{\pgfqpoint{3.200671in}{1.535108in}}%
\pgfpathlineto{\pgfqpoint{3.214476in}{1.528293in}}%
\pgfpathlineto{\pgfqpoint{3.228281in}{1.521684in}}%
\pgfpathlineto{\pgfqpoint{3.236693in}{1.528051in}}%
\pgfpathlineto{\pgfqpoint{3.245095in}{1.534599in}}%
\pgfpathlineto{\pgfqpoint{3.253487in}{1.541322in}}%
\pgfpathlineto{\pgfqpoint{3.261869in}{1.548216in}}%
\pgfpathlineto{\pgfqpoint{3.248089in}{1.554261in}}%
\pgfpathlineto{\pgfqpoint{3.234310in}{1.560512in}}%
\pgfpathlineto{\pgfqpoint{3.220534in}{1.566969in}}%
\pgfpathlineto{\pgfqpoint{3.206759in}{1.573633in}}%
\pgfpathlineto{\pgfqpoint{3.198351in}{1.567292in}}%
\pgfpathlineto{\pgfqpoint{3.189933in}{1.561128in}}%
\pgfpathlineto{\pgfqpoint{3.181505in}{1.555149in}}%
\pgfpathlineto{\pgfqpoint{3.173067in}{1.549358in}}%
\pgfpathclose%
\pgfusepath{fill}%
\end{pgfscope}%
\begin{pgfscope}%
\pgfpathrectangle{\pgfqpoint{1.150000in}{0.150000in}}{\pgfqpoint{5.700000in}{5.700000in}}%
\pgfusepath{clip}%
\pgfsetbuttcap%
\pgfsetroundjoin%
\definecolor{currentfill}{rgb}{0.267004,0.004874,0.329415}%
\pgfsetfillcolor{currentfill}%
\pgfsetfillopacity{0.800000}%
\pgfsetlinewidth{0.000000pt}%
\definecolor{currentstroke}{rgb}{0.000000,0.000000,0.000000}%
\pgfsetstrokecolor{currentstroke}%
\pgfsetdash{}{0pt}%
\pgfpathmoveto{\pgfqpoint{3.317015in}{1.526073in}}%
\pgfpathlineto{\pgfqpoint{3.330807in}{1.521043in}}%
\pgfpathlineto{\pgfqpoint{3.344603in}{1.516214in}}%
\pgfpathlineto{\pgfqpoint{3.358403in}{1.511585in}}%
\pgfpathlineto{\pgfqpoint{3.372205in}{1.507154in}}%
\pgfpathlineto{\pgfqpoint{3.380534in}{1.515298in}}%
\pgfpathlineto{\pgfqpoint{3.388854in}{1.523586in}}%
\pgfpathlineto{\pgfqpoint{3.397166in}{1.532011in}}%
\pgfpathlineto{\pgfqpoint{3.405470in}{1.540569in}}%
\pgfpathlineto{\pgfqpoint{3.391688in}{1.544468in}}%
\pgfpathlineto{\pgfqpoint{3.377909in}{1.548567in}}%
\pgfpathlineto{\pgfqpoint{3.364134in}{1.552865in}}%
\pgfpathlineto{\pgfqpoint{3.350362in}{1.557365in}}%
\pgfpathlineto{\pgfqpoint{3.342038in}{1.549325in}}%
\pgfpathlineto{\pgfqpoint{3.333706in}{1.541427in}}%
\pgfpathlineto{\pgfqpoint{3.325365in}{1.533675in}}%
\pgfpathlineto{\pgfqpoint{3.317015in}{1.526073in}}%
\pgfpathclose%
\pgfusepath{fill}%
\end{pgfscope}%
\begin{pgfscope}%
\pgfpathrectangle{\pgfqpoint{1.150000in}{0.150000in}}{\pgfqpoint{5.700000in}{5.700000in}}%
\pgfusepath{clip}%
\pgfsetbuttcap%
\pgfsetroundjoin%
\definecolor{currentfill}{rgb}{0.127568,0.566949,0.550556}%
\pgfsetfillcolor{currentfill}%
\pgfsetfillopacity{0.800000}%
\pgfsetlinewidth{0.000000pt}%
\definecolor{currentstroke}{rgb}{0.000000,0.000000,0.000000}%
\pgfsetstrokecolor{currentstroke}%
\pgfsetdash{}{0pt}%
\pgfpathmoveto{\pgfqpoint{5.150125in}{2.950035in}}%
\pgfpathlineto{\pgfqpoint{5.164668in}{2.963201in}}%
\pgfpathlineto{\pgfqpoint{5.179230in}{2.976551in}}%
\pgfpathlineto{\pgfqpoint{5.193811in}{2.990085in}}%
\pgfpathlineto{\pgfqpoint{5.208412in}{3.003804in}}%
\pgfpathlineto{\pgfqpoint{5.216014in}{3.009882in}}%
\pgfpathlineto{\pgfqpoint{5.223606in}{3.015838in}}%
\pgfpathlineto{\pgfqpoint{5.231190in}{3.021674in}}%
\pgfpathlineto{\pgfqpoint{5.238766in}{3.027394in}}%
\pgfpathlineto{\pgfqpoint{5.224177in}{3.013932in}}%
\pgfpathlineto{\pgfqpoint{5.209607in}{3.000654in}}%
\pgfpathlineto{\pgfqpoint{5.195058in}{2.987560in}}%
\pgfpathlineto{\pgfqpoint{5.180528in}{2.974649in}}%
\pgfpathlineto{\pgfqpoint{5.172940in}{2.968660in}}%
\pgfpathlineto{\pgfqpoint{5.165343in}{2.962564in}}%
\pgfpathlineto{\pgfqpoint{5.157738in}{2.956356in}}%
\pgfpathlineto{\pgfqpoint{5.150125in}{2.950035in}}%
\pgfpathclose%
\pgfusepath{fill}%
\end{pgfscope}%
\begin{pgfscope}%
\pgfpathrectangle{\pgfqpoint{1.150000in}{0.150000in}}{\pgfqpoint{5.700000in}{5.700000in}}%
\pgfusepath{clip}%
\pgfsetbuttcap%
\pgfsetroundjoin%
\definecolor{currentfill}{rgb}{0.276022,0.044167,0.370164}%
\pgfsetfillcolor{currentfill}%
\pgfsetfillopacity{0.800000}%
\pgfsetlinewidth{0.000000pt}%
\definecolor{currentstroke}{rgb}{0.000000,0.000000,0.000000}%
\pgfsetstrokecolor{currentstroke}%
\pgfsetdash{}{0pt}%
\pgfpathmoveto{\pgfqpoint{3.637005in}{1.593331in}}%
\pgfpathlineto{\pgfqpoint{3.650826in}{1.592929in}}%
\pgfpathlineto{\pgfqpoint{3.664654in}{1.592718in}}%
\pgfpathlineto{\pgfqpoint{3.678489in}{1.592698in}}%
\pgfpathlineto{\pgfqpoint{3.692331in}{1.592867in}}%
\pgfpathlineto{\pgfqpoint{3.700515in}{1.604199in}}%
\pgfpathlineto{\pgfqpoint{3.708694in}{1.615585in}}%
\pgfpathlineto{\pgfqpoint{3.716867in}{1.627023in}}%
\pgfpathlineto{\pgfqpoint{3.725034in}{1.638509in}}%
\pgfpathlineto{\pgfqpoint{3.711203in}{1.637902in}}%
\pgfpathlineto{\pgfqpoint{3.697379in}{1.637486in}}%
\pgfpathlineto{\pgfqpoint{3.683562in}{1.637260in}}%
\pgfpathlineto{\pgfqpoint{3.669752in}{1.637226in}}%
\pgfpathlineto{\pgfqpoint{3.661574in}{1.626166in}}%
\pgfpathlineto{\pgfqpoint{3.653391in}{1.615160in}}%
\pgfpathlineto{\pgfqpoint{3.645201in}{1.604214in}}%
\pgfpathlineto{\pgfqpoint{3.637005in}{1.593331in}}%
\pgfpathclose%
\pgfusepath{fill}%
\end{pgfscope}%
\begin{pgfscope}%
\pgfpathrectangle{\pgfqpoint{1.150000in}{0.150000in}}{\pgfqpoint{5.700000in}{5.700000in}}%
\pgfusepath{clip}%
\pgfsetbuttcap%
\pgfsetroundjoin%
\definecolor{currentfill}{rgb}{0.280267,0.073417,0.397163}%
\pgfsetfillcolor{currentfill}%
\pgfsetfillopacity{0.800000}%
\pgfsetlinewidth{0.000000pt}%
\definecolor{currentstroke}{rgb}{0.000000,0.000000,0.000000}%
\pgfsetstrokecolor{currentstroke}%
\pgfsetdash{}{0pt}%
\pgfpathmoveto{\pgfqpoint{3.725034in}{1.638509in}}%
\pgfpathlineto{\pgfqpoint{3.738874in}{1.639305in}}%
\pgfpathlineto{\pgfqpoint{3.752721in}{1.640290in}}%
\pgfpathlineto{\pgfqpoint{3.766576in}{1.641464in}}%
\pgfpathlineto{\pgfqpoint{3.780439in}{1.642827in}}%
\pgfpathlineto{\pgfqpoint{3.788593in}{1.654773in}}%
\pgfpathlineto{\pgfqpoint{3.796740in}{1.666750in}}%
\pgfpathlineto{\pgfqpoint{3.804883in}{1.678755in}}%
\pgfpathlineto{\pgfqpoint{3.813021in}{1.690784in}}%
\pgfpathlineto{\pgfqpoint{3.799166in}{1.689015in}}%
\pgfpathlineto{\pgfqpoint{3.785319in}{1.687435in}}%
\pgfpathlineto{\pgfqpoint{3.771480in}{1.686044in}}%
\pgfpathlineto{\pgfqpoint{3.757650in}{1.684843in}}%
\pgfpathlineto{\pgfqpoint{3.749504in}{1.673208in}}%
\pgfpathlineto{\pgfqpoint{3.741353in}{1.661604in}}%
\pgfpathlineto{\pgfqpoint{3.733197in}{1.650037in}}%
\pgfpathlineto{\pgfqpoint{3.725034in}{1.638509in}}%
\pgfpathclose%
\pgfusepath{fill}%
\end{pgfscope}%
\begin{pgfscope}%
\pgfpathrectangle{\pgfqpoint{1.150000in}{0.150000in}}{\pgfqpoint{5.700000in}{5.700000in}}%
\pgfusepath{clip}%
\pgfsetbuttcap%
\pgfsetroundjoin%
\definecolor{currentfill}{rgb}{0.272594,0.025563,0.353093}%
\pgfsetfillcolor{currentfill}%
\pgfsetfillopacity{0.800000}%
\pgfsetlinewidth{0.000000pt}%
\definecolor{currentstroke}{rgb}{0.000000,0.000000,0.000000}%
\pgfsetstrokecolor{currentstroke}%
\pgfsetdash{}{0pt}%
\pgfpathmoveto{\pgfqpoint{3.548890in}{1.555914in}}%
\pgfpathlineto{\pgfqpoint{3.562699in}{1.554276in}}%
\pgfpathlineto{\pgfqpoint{3.576514in}{1.552831in}}%
\pgfpathlineto{\pgfqpoint{3.590335in}{1.551578in}}%
\pgfpathlineto{\pgfqpoint{3.604162in}{1.550516in}}%
\pgfpathlineto{\pgfqpoint{3.612382in}{1.561104in}}%
\pgfpathlineto{\pgfqpoint{3.620596in}{1.571772in}}%
\pgfpathlineto{\pgfqpoint{3.628804in}{1.582516in}}%
\pgfpathlineto{\pgfqpoint{3.637005in}{1.593331in}}%
\pgfpathlineto{\pgfqpoint{3.623191in}{1.593924in}}%
\pgfpathlineto{\pgfqpoint{3.609383in}{1.594710in}}%
\pgfpathlineto{\pgfqpoint{3.595582in}{1.595687in}}%
\pgfpathlineto{\pgfqpoint{3.581787in}{1.596858in}}%
\pgfpathlineto{\pgfqpoint{3.573572in}{1.586498in}}%
\pgfpathlineto{\pgfqpoint{3.565352in}{1.576218in}}%
\pgfpathlineto{\pgfqpoint{3.557124in}{1.566022in}}%
\pgfpathlineto{\pgfqpoint{3.548890in}{1.555914in}}%
\pgfpathclose%
\pgfusepath{fill}%
\end{pgfscope}%
\begin{pgfscope}%
\pgfpathrectangle{\pgfqpoint{1.150000in}{0.150000in}}{\pgfqpoint{5.700000in}{5.700000in}}%
\pgfusepath{clip}%
\pgfsetbuttcap%
\pgfsetroundjoin%
\definecolor{currentfill}{rgb}{0.143303,0.669459,0.511215}%
\pgfsetfillcolor{currentfill}%
\pgfsetfillopacity{0.800000}%
\pgfsetlinewidth{0.000000pt}%
\definecolor{currentstroke}{rgb}{0.000000,0.000000,0.000000}%
\pgfsetstrokecolor{currentstroke}%
\pgfsetdash{}{0pt}%
\pgfpathmoveto{\pgfqpoint{5.534670in}{3.265357in}}%
\pgfpathlineto{\pgfqpoint{5.549459in}{3.279827in}}%
\pgfpathlineto{\pgfqpoint{5.564269in}{3.294480in}}%
\pgfpathlineto{\pgfqpoint{5.579101in}{3.309316in}}%
\pgfpathlineto{\pgfqpoint{5.593955in}{3.324337in}}%
\pgfpathlineto{\pgfqpoint{5.601307in}{3.326972in}}%
\pgfpathlineto{\pgfqpoint{5.608650in}{3.329529in}}%
\pgfpathlineto{\pgfqpoint{5.615984in}{3.332012in}}%
\pgfpathlineto{\pgfqpoint{5.623308in}{3.334427in}}%
\pgfpathlineto{\pgfqpoint{5.608477in}{3.319842in}}%
\pgfpathlineto{\pgfqpoint{5.593668in}{3.305439in}}%
\pgfpathlineto{\pgfqpoint{5.578880in}{3.291218in}}%
\pgfpathlineto{\pgfqpoint{5.564113in}{3.277180in}}%
\pgfpathlineto{\pgfqpoint{5.556765in}{3.274320in}}%
\pgfpathlineto{\pgfqpoint{5.549408in}{3.271400in}}%
\pgfpathlineto{\pgfqpoint{5.542043in}{3.268414in}}%
\pgfpathlineto{\pgfqpoint{5.534670in}{3.265357in}}%
\pgfpathclose%
\pgfusepath{fill}%
\end{pgfscope}%
\begin{pgfscope}%
\pgfpathrectangle{\pgfqpoint{1.150000in}{0.150000in}}{\pgfqpoint{5.700000in}{5.700000in}}%
\pgfusepath{clip}%
\pgfsetbuttcap%
\pgfsetroundjoin%
\definecolor{currentfill}{rgb}{0.271828,0.209303,0.504434}%
\pgfsetfillcolor{currentfill}%
\pgfsetfillopacity{0.800000}%
\pgfsetlinewidth{0.000000pt}%
\definecolor{currentstroke}{rgb}{0.000000,0.000000,0.000000}%
\pgfsetstrokecolor{currentstroke}%
\pgfsetdash{}{0pt}%
\pgfpathmoveto{\pgfqpoint{4.109296in}{1.935366in}}%
\pgfpathlineto{\pgfqpoint{4.123264in}{1.940877in}}%
\pgfpathlineto{\pgfqpoint{4.137243in}{1.946573in}}%
\pgfpathlineto{\pgfqpoint{4.151235in}{1.952455in}}%
\pgfpathlineto{\pgfqpoint{4.165238in}{1.958522in}}%
\pgfpathlineto{\pgfqpoint{4.173277in}{1.971513in}}%
\pgfpathlineto{\pgfqpoint{4.181312in}{1.984443in}}%
\pgfpathlineto{\pgfqpoint{4.189341in}{1.997311in}}%
\pgfpathlineto{\pgfqpoint{4.197366in}{2.010114in}}%
\pgfpathlineto{\pgfqpoint{4.183365in}{2.003797in}}%
\pgfpathlineto{\pgfqpoint{4.169376in}{1.997666in}}%
\pgfpathlineto{\pgfqpoint{4.155399in}{1.991720in}}%
\pgfpathlineto{\pgfqpoint{4.141434in}{1.985961in}}%
\pgfpathlineto{\pgfqpoint{4.133407in}{1.973395in}}%
\pgfpathlineto{\pgfqpoint{4.125375in}{1.960773in}}%
\pgfpathlineto{\pgfqpoint{4.117338in}{1.948096in}}%
\pgfpathlineto{\pgfqpoint{4.109296in}{1.935366in}}%
\pgfpathclose%
\pgfusepath{fill}%
\end{pgfscope}%
\begin{pgfscope}%
\pgfpathrectangle{\pgfqpoint{1.150000in}{0.150000in}}{\pgfqpoint{5.700000in}{5.700000in}}%
\pgfusepath{clip}%
\pgfsetbuttcap%
\pgfsetroundjoin%
\definecolor{currentfill}{rgb}{0.282656,0.100196,0.422160}%
\pgfsetfillcolor{currentfill}%
\pgfsetfillopacity{0.800000}%
\pgfsetlinewidth{0.000000pt}%
\definecolor{currentstroke}{rgb}{0.000000,0.000000,0.000000}%
\pgfsetstrokecolor{currentstroke}%
\pgfsetdash{}{0pt}%
\pgfpathmoveto{\pgfqpoint{3.813021in}{1.690784in}}%
\pgfpathlineto{\pgfqpoint{3.826885in}{1.692741in}}%
\pgfpathlineto{\pgfqpoint{3.840757in}{1.694887in}}%
\pgfpathlineto{\pgfqpoint{3.854639in}{1.697219in}}%
\pgfpathlineto{\pgfqpoint{3.868530in}{1.699739in}}%
\pgfpathlineto{\pgfqpoint{3.876655in}{1.712175in}}%
\pgfpathlineto{\pgfqpoint{3.884776in}{1.724620in}}%
\pgfpathlineto{\pgfqpoint{3.892892in}{1.737070in}}%
\pgfpathlineto{\pgfqpoint{3.901003in}{1.749522in}}%
\pgfpathlineto{\pgfqpoint{3.887118in}{1.746627in}}%
\pgfpathlineto{\pgfqpoint{3.873243in}{1.743919in}}%
\pgfpathlineto{\pgfqpoint{3.859377in}{1.741399in}}%
\pgfpathlineto{\pgfqpoint{3.845520in}{1.739068in}}%
\pgfpathlineto{\pgfqpoint{3.837403in}{1.726979in}}%
\pgfpathlineto{\pgfqpoint{3.829281in}{1.714899in}}%
\pgfpathlineto{\pgfqpoint{3.821153in}{1.702833in}}%
\pgfpathlineto{\pgfqpoint{3.813021in}{1.690784in}}%
\pgfpathclose%
\pgfusepath{fill}%
\end{pgfscope}%
\begin{pgfscope}%
\pgfpathrectangle{\pgfqpoint{1.150000in}{0.150000in}}{\pgfqpoint{5.700000in}{5.700000in}}%
\pgfusepath{clip}%
\pgfsetbuttcap%
\pgfsetroundjoin%
\definecolor{currentfill}{rgb}{0.243113,0.292092,0.538516}%
\pgfsetfillcolor{currentfill}%
\pgfsetfillopacity{0.800000}%
\pgfsetlinewidth{0.000000pt}%
\definecolor{currentstroke}{rgb}{0.000000,0.000000,0.000000}%
\pgfsetstrokecolor{currentstroke}%
\pgfsetdash{}{0pt}%
\pgfpathmoveto{\pgfqpoint{4.317510in}{2.138707in}}%
\pgfpathlineto{\pgfqpoint{4.331574in}{2.146353in}}%
\pgfpathlineto{\pgfqpoint{4.345650in}{2.154183in}}%
\pgfpathlineto{\pgfqpoint{4.359741in}{2.162198in}}%
\pgfpathlineto{\pgfqpoint{4.373845in}{2.170398in}}%
\pgfpathlineto{\pgfqpoint{4.381823in}{2.182867in}}%
\pgfpathlineto{\pgfqpoint{4.389795in}{2.195239in}}%
\pgfpathlineto{\pgfqpoint{4.397763in}{2.207512in}}%
\pgfpathlineto{\pgfqpoint{4.405724in}{2.219686in}}%
\pgfpathlineto{\pgfqpoint{4.391621in}{2.211333in}}%
\pgfpathlineto{\pgfqpoint{4.377532in}{2.203165in}}%
\pgfpathlineto{\pgfqpoint{4.363456in}{2.195182in}}%
\pgfpathlineto{\pgfqpoint{4.349395in}{2.187384in}}%
\pgfpathlineto{\pgfqpoint{4.341432in}{2.175351in}}%
\pgfpathlineto{\pgfqpoint{4.333463in}{2.163226in}}%
\pgfpathlineto{\pgfqpoint{4.325489in}{2.151011in}}%
\pgfpathlineto{\pgfqpoint{4.317510in}{2.138707in}}%
\pgfpathclose%
\pgfusepath{fill}%
\end{pgfscope}%
\begin{pgfscope}%
\pgfpathrectangle{\pgfqpoint{1.150000in}{0.150000in}}{\pgfqpoint{5.700000in}{5.700000in}}%
\pgfusepath{clip}%
\pgfsetbuttcap%
\pgfsetroundjoin%
\definecolor{currentfill}{rgb}{0.149039,0.508051,0.557250}%
\pgfsetfillcolor{currentfill}%
\pgfsetfillopacity{0.800000}%
\pgfsetlinewidth{0.000000pt}%
\definecolor{currentstroke}{rgb}{0.000000,0.000000,0.000000}%
\pgfsetstrokecolor{currentstroke}%
\pgfsetdash{}{0pt}%
\pgfpathmoveto{\pgfqpoint{4.942234in}{2.761118in}}%
\pgfpathlineto{\pgfqpoint{4.956653in}{2.773369in}}%
\pgfpathlineto{\pgfqpoint{4.971091in}{2.785804in}}%
\pgfpathlineto{\pgfqpoint{4.985546in}{2.798423in}}%
\pgfpathlineto{\pgfqpoint{5.000021in}{2.811227in}}%
\pgfpathlineto{\pgfqpoint{5.007743in}{2.819262in}}%
\pgfpathlineto{\pgfqpoint{5.015456in}{2.827162in}}%
\pgfpathlineto{\pgfqpoint{5.023162in}{2.834929in}}%
\pgfpathlineto{\pgfqpoint{5.030859in}{2.842565in}}%
\pgfpathlineto{\pgfqpoint{5.016393in}{2.829913in}}%
\pgfpathlineto{\pgfqpoint{5.001944in}{2.817445in}}%
\pgfpathlineto{\pgfqpoint{4.987515in}{2.805161in}}%
\pgfpathlineto{\pgfqpoint{4.973103in}{2.793061in}}%
\pgfpathlineto{\pgfqpoint{4.965398in}{2.785261in}}%
\pgfpathlineto{\pgfqpoint{4.957684in}{2.777339in}}%
\pgfpathlineto{\pgfqpoint{4.949963in}{2.769292in}}%
\pgfpathlineto{\pgfqpoint{4.942234in}{2.761118in}}%
\pgfpathclose%
\pgfusepath{fill}%
\end{pgfscope}%
\begin{pgfscope}%
\pgfpathrectangle{\pgfqpoint{1.150000in}{0.150000in}}{\pgfqpoint{5.700000in}{5.700000in}}%
\pgfusepath{clip}%
\pgfsetbuttcap%
\pgfsetroundjoin%
\definecolor{currentfill}{rgb}{0.273809,0.031497,0.358853}%
\pgfsetfillcolor{currentfill}%
\pgfsetfillopacity{0.800000}%
\pgfsetlinewidth{0.000000pt}%
\definecolor{currentstroke}{rgb}{0.000000,0.000000,0.000000}%
\pgfsetstrokecolor{currentstroke}%
\pgfsetdash{}{0pt}%
\pgfpathmoveto{\pgfqpoint{3.028565in}{1.598409in}}%
\pgfpathlineto{\pgfqpoint{3.042397in}{1.588885in}}%
\pgfpathlineto{\pgfqpoint{3.056227in}{1.579577in}}%
\pgfpathlineto{\pgfqpoint{3.070057in}{1.570483in}}%
\pgfpathlineto{\pgfqpoint{3.083886in}{1.561604in}}%
\pgfpathlineto{\pgfqpoint{3.092399in}{1.565970in}}%
\pgfpathlineto{\pgfqpoint{3.100901in}{1.570556in}}%
\pgfpathlineto{\pgfqpoint{3.109390in}{1.575358in}}%
\pgfpathlineto{\pgfqpoint{3.117869in}{1.580368in}}%
\pgfpathlineto{\pgfqpoint{3.104070in}{1.588649in}}%
\pgfpathlineto{\pgfqpoint{3.090272in}{1.597143in}}%
\pgfpathlineto{\pgfqpoint{3.076473in}{1.605852in}}%
\pgfpathlineto{\pgfqpoint{3.062674in}{1.614776in}}%
\pgfpathlineto{\pgfqpoint{3.054165in}{1.610353in}}%
\pgfpathlineto{\pgfqpoint{3.045644in}{1.606146in}}%
\pgfpathlineto{\pgfqpoint{3.037111in}{1.602163in}}%
\pgfpathlineto{\pgfqpoint{3.028565in}{1.598409in}}%
\pgfpathclose%
\pgfusepath{fill}%
\end{pgfscope}%
\begin{pgfscope}%
\pgfpathrectangle{\pgfqpoint{1.150000in}{0.150000in}}{\pgfqpoint{5.700000in}{5.700000in}}%
\pgfusepath{clip}%
\pgfsetbuttcap%
\pgfsetroundjoin%
\definecolor{currentfill}{rgb}{0.276194,0.190074,0.493001}%
\pgfsetfillcolor{currentfill}%
\pgfsetfillopacity{0.800000}%
\pgfsetlinewidth{0.000000pt}%
\definecolor{currentstroke}{rgb}{0.000000,0.000000,0.000000}%
\pgfsetstrokecolor{currentstroke}%
\pgfsetdash{}{0pt}%
\pgfpathmoveto{\pgfqpoint{2.604654in}{1.963320in}}%
\pgfpathlineto{\pgfqpoint{2.618641in}{1.946555in}}%
\pgfpathlineto{\pgfqpoint{2.632621in}{1.930047in}}%
\pgfpathlineto{\pgfqpoint{2.646593in}{1.913795in}}%
\pgfpathlineto{\pgfqpoint{2.660559in}{1.897797in}}%
\pgfpathlineto{\pgfqpoint{2.669408in}{1.896649in}}%
\pgfpathlineto{\pgfqpoint{2.678240in}{1.895807in}}%
\pgfpathlineto{\pgfqpoint{2.687054in}{1.895265in}}%
\pgfpathlineto{\pgfqpoint{2.695851in}{1.895017in}}%
\pgfpathlineto{\pgfqpoint{2.681932in}{1.910364in}}%
\pgfpathlineto{\pgfqpoint{2.668007in}{1.925965in}}%
\pgfpathlineto{\pgfqpoint{2.654075in}{1.941820in}}%
\pgfpathlineto{\pgfqpoint{2.640136in}{1.957932in}}%
\pgfpathlineto{\pgfqpoint{2.631293in}{1.958819in}}%
\pgfpathlineto{\pgfqpoint{2.622432in}{1.960008in}}%
\pgfpathlineto{\pgfqpoint{2.613552in}{1.961506in}}%
\pgfpathlineto{\pgfqpoint{2.604654in}{1.963320in}}%
\pgfpathclose%
\pgfusepath{fill}%
\end{pgfscope}%
\begin{pgfscope}%
\pgfpathrectangle{\pgfqpoint{1.150000in}{0.150000in}}{\pgfqpoint{5.700000in}{5.700000in}}%
\pgfusepath{clip}%
\pgfsetbuttcap%
\pgfsetroundjoin%
\definecolor{currentfill}{rgb}{0.206756,0.371758,0.553117}%
\pgfsetfillcolor{currentfill}%
\pgfsetfillopacity{0.800000}%
\pgfsetlinewidth{0.000000pt}%
\definecolor{currentstroke}{rgb}{0.000000,0.000000,0.000000}%
\pgfsetstrokecolor{currentstroke}%
\pgfsetdash{}{0pt}%
\pgfpathmoveto{\pgfqpoint{4.525768in}{2.349478in}}%
\pgfpathlineto{\pgfqpoint{4.539942in}{2.358962in}}%
\pgfpathlineto{\pgfqpoint{4.554131in}{2.368631in}}%
\pgfpathlineto{\pgfqpoint{4.568336in}{2.378484in}}%
\pgfpathlineto{\pgfqpoint{4.582557in}{2.388522in}}%
\pgfpathlineto{\pgfqpoint{4.590465in}{2.399898in}}%
\pgfpathlineto{\pgfqpoint{4.598366in}{2.411152in}}%
\pgfpathlineto{\pgfqpoint{4.606262in}{2.422283in}}%
\pgfpathlineto{\pgfqpoint{4.614151in}{2.433292in}}%
\pgfpathlineto{\pgfqpoint{4.599933in}{2.423200in}}%
\pgfpathlineto{\pgfqpoint{4.585730in}{2.413293in}}%
\pgfpathlineto{\pgfqpoint{4.571542in}{2.403571in}}%
\pgfpathlineto{\pgfqpoint{4.557370in}{2.394034in}}%
\pgfpathlineto{\pgfqpoint{4.549479in}{2.383066in}}%
\pgfpathlineto{\pgfqpoint{4.541581in}{2.371984in}}%
\pgfpathlineto{\pgfqpoint{4.533677in}{2.360788in}}%
\pgfpathlineto{\pgfqpoint{4.525768in}{2.349478in}}%
\pgfpathclose%
\pgfusepath{fill}%
\end{pgfscope}%
\begin{pgfscope}%
\pgfpathrectangle{\pgfqpoint{1.150000in}{0.150000in}}{\pgfqpoint{5.700000in}{5.700000in}}%
\pgfusepath{clip}%
\pgfsetbuttcap%
\pgfsetroundjoin%
\definecolor{currentfill}{rgb}{0.269308,0.218818,0.509577}%
\pgfsetfillcolor{currentfill}%
\pgfsetfillopacity{0.800000}%
\pgfsetlinewidth{0.000000pt}%
\definecolor{currentstroke}{rgb}{0.000000,0.000000,0.000000}%
\pgfsetstrokecolor{currentstroke}%
\pgfsetdash{}{0pt}%
\pgfpathmoveto{\pgfqpoint{2.548625in}{2.032998in}}%
\pgfpathlineto{\pgfqpoint{2.562645in}{2.015182in}}%
\pgfpathlineto{\pgfqpoint{2.576656in}{1.997632in}}%
\pgfpathlineto{\pgfqpoint{2.590659in}{1.980345in}}%
\pgfpathlineto{\pgfqpoint{2.604654in}{1.963320in}}%
\pgfpathlineto{\pgfqpoint{2.613552in}{1.961506in}}%
\pgfpathlineto{\pgfqpoint{2.622432in}{1.960008in}}%
\pgfpathlineto{\pgfqpoint{2.631293in}{1.958819in}}%
\pgfpathlineto{\pgfqpoint{2.640136in}{1.957932in}}%
\pgfpathlineto{\pgfqpoint{2.626191in}{1.974303in}}%
\pgfpathlineto{\pgfqpoint{2.612237in}{1.990934in}}%
\pgfpathlineto{\pgfqpoint{2.598276in}{2.007828in}}%
\pgfpathlineto{\pgfqpoint{2.584307in}{2.024986in}}%
\pgfpathlineto{\pgfqpoint{2.575415in}{2.026515in}}%
\pgfpathlineto{\pgfqpoint{2.566505in}{2.028355in}}%
\pgfpathlineto{\pgfqpoint{2.557575in}{2.030514in}}%
\pgfpathlineto{\pgfqpoint{2.548625in}{2.032998in}}%
\pgfpathclose%
\pgfusepath{fill}%
\end{pgfscope}%
\begin{pgfscope}%
\pgfpathrectangle{\pgfqpoint{1.150000in}{0.150000in}}{\pgfqpoint{5.700000in}{5.700000in}}%
\pgfusepath{clip}%
\pgfsetbuttcap%
\pgfsetroundjoin%
\definecolor{currentfill}{rgb}{0.280255,0.165693,0.476498}%
\pgfsetfillcolor{currentfill}%
\pgfsetfillopacity{0.800000}%
\pgfsetlinewidth{0.000000pt}%
\definecolor{currentstroke}{rgb}{0.000000,0.000000,0.000000}%
\pgfsetstrokecolor{currentstroke}%
\pgfsetdash{}{0pt}%
\pgfpathmoveto{\pgfqpoint{2.660559in}{1.897797in}}%
\pgfpathlineto{\pgfqpoint{2.674518in}{1.882052in}}%
\pgfpathlineto{\pgfqpoint{2.688470in}{1.866556in}}%
\pgfpathlineto{\pgfqpoint{2.702416in}{1.851310in}}%
\pgfpathlineto{\pgfqpoint{2.716356in}{1.836310in}}%
\pgfpathlineto{\pgfqpoint{2.725159in}{1.835823in}}%
\pgfpathlineto{\pgfqpoint{2.733945in}{1.835634in}}%
\pgfpathlineto{\pgfqpoint{2.742713in}{1.835736in}}%
\pgfpathlineto{\pgfqpoint{2.751466in}{1.836123in}}%
\pgfpathlineto{\pgfqpoint{2.737570in}{1.850475in}}%
\pgfpathlineto{\pgfqpoint{2.723670in}{1.865074in}}%
\pgfpathlineto{\pgfqpoint{2.709763in}{1.879921in}}%
\pgfpathlineto{\pgfqpoint{2.695851in}{1.895017in}}%
\pgfpathlineto{\pgfqpoint{2.687054in}{1.895265in}}%
\pgfpathlineto{\pgfqpoint{2.678240in}{1.895807in}}%
\pgfpathlineto{\pgfqpoint{2.669408in}{1.896649in}}%
\pgfpathlineto{\pgfqpoint{2.660559in}{1.897797in}}%
\pgfpathclose%
\pgfusepath{fill}%
\end{pgfscope}%
\begin{pgfscope}%
\pgfpathrectangle{\pgfqpoint{1.150000in}{0.150000in}}{\pgfqpoint{5.700000in}{5.700000in}}%
\pgfusepath{clip}%
\pgfsetbuttcap%
\pgfsetroundjoin%
\definecolor{currentfill}{rgb}{0.268510,0.009605,0.335427}%
\pgfsetfillcolor{currentfill}%
\pgfsetfillopacity{0.800000}%
\pgfsetlinewidth{0.000000pt}%
\definecolor{currentstroke}{rgb}{0.000000,0.000000,0.000000}%
\pgfsetstrokecolor{currentstroke}%
\pgfsetdash{}{0pt}%
\pgfpathmoveto{\pgfqpoint{3.460641in}{1.526948in}}%
\pgfpathlineto{\pgfqpoint{3.474445in}{1.524034in}}%
\pgfpathlineto{\pgfqpoint{3.488254in}{1.521315in}}%
\pgfpathlineto{\pgfqpoint{3.502068in}{1.518790in}}%
\pgfpathlineto{\pgfqpoint{3.515887in}{1.516459in}}%
\pgfpathlineto{\pgfqpoint{3.524148in}{1.526167in}}%
\pgfpathlineto{\pgfqpoint{3.532402in}{1.535982in}}%
\pgfpathlineto{\pgfqpoint{3.540650in}{1.545899in}}%
\pgfpathlineto{\pgfqpoint{3.548890in}{1.555914in}}%
\pgfpathlineto{\pgfqpoint{3.535087in}{1.557746in}}%
\pgfpathlineto{\pgfqpoint{3.521289in}{1.559771in}}%
\pgfpathlineto{\pgfqpoint{3.507497in}{1.561991in}}%
\pgfpathlineto{\pgfqpoint{3.493709in}{1.564407in}}%
\pgfpathlineto{\pgfqpoint{3.485453in}{1.554879in}}%
\pgfpathlineto{\pgfqpoint{3.477190in}{1.545457in}}%
\pgfpathlineto{\pgfqpoint{3.468919in}{1.536145in}}%
\pgfpathlineto{\pgfqpoint{3.460641in}{1.526948in}}%
\pgfpathclose%
\pgfusepath{fill}%
\end{pgfscope}%
\begin{pgfscope}%
\pgfpathrectangle{\pgfqpoint{1.150000in}{0.150000in}}{\pgfqpoint{5.700000in}{5.700000in}}%
\pgfusepath{clip}%
\pgfsetbuttcap%
\pgfsetroundjoin%
\definecolor{currentfill}{rgb}{0.174274,0.445044,0.557792}%
\pgfsetfillcolor{currentfill}%
\pgfsetfillopacity{0.800000}%
\pgfsetlinewidth{0.000000pt}%
\definecolor{currentstroke}{rgb}{0.000000,0.000000,0.000000}%
\pgfsetstrokecolor{currentstroke}%
\pgfsetdash{}{0pt}%
\pgfpathmoveto{\pgfqpoint{4.734050in}{2.559222in}}%
\pgfpathlineto{\pgfqpoint{4.748345in}{2.570243in}}%
\pgfpathlineto{\pgfqpoint{4.762656in}{2.581449in}}%
\pgfpathlineto{\pgfqpoint{4.776985in}{2.592839in}}%
\pgfpathlineto{\pgfqpoint{4.791331in}{2.604414in}}%
\pgfpathlineto{\pgfqpoint{4.799154in}{2.614264in}}%
\pgfpathlineto{\pgfqpoint{4.806971in}{2.623980in}}%
\pgfpathlineto{\pgfqpoint{4.814780in}{2.633561in}}%
\pgfpathlineto{\pgfqpoint{4.822583in}{2.643009in}}%
\pgfpathlineto{\pgfqpoint{4.808241in}{2.631483in}}%
\pgfpathlineto{\pgfqpoint{4.793916in}{2.620140in}}%
\pgfpathlineto{\pgfqpoint{4.779609in}{2.608982in}}%
\pgfpathlineto{\pgfqpoint{4.765319in}{2.598009in}}%
\pgfpathlineto{\pgfqpoint{4.757512in}{2.588500in}}%
\pgfpathlineto{\pgfqpoint{4.749698in}{2.578866in}}%
\pgfpathlineto{\pgfqpoint{4.741877in}{2.569107in}}%
\pgfpathlineto{\pgfqpoint{4.734050in}{2.559222in}}%
\pgfpathclose%
\pgfusepath{fill}%
\end{pgfscope}%
\begin{pgfscope}%
\pgfpathrectangle{\pgfqpoint{1.150000in}{0.150000in}}{\pgfqpoint{5.700000in}{5.700000in}}%
\pgfusepath{clip}%
\pgfsetbuttcap%
\pgfsetroundjoin%
\definecolor{currentfill}{rgb}{0.180629,0.429975,0.557282}%
\pgfsetfillcolor{currentfill}%
\pgfsetfillopacity{0.800000}%
\pgfsetlinewidth{0.000000pt}%
\definecolor{currentstroke}{rgb}{0.000000,0.000000,0.000000}%
\pgfsetstrokecolor{currentstroke}%
\pgfsetdash{}{0pt}%
\pgfpathmoveto{\pgfqpoint{2.188788in}{2.622185in}}%
\pgfpathlineto{\pgfqpoint{2.203099in}{2.596642in}}%
\pgfpathlineto{\pgfqpoint{2.217393in}{2.571437in}}%
\pgfpathlineto{\pgfqpoint{2.231671in}{2.546566in}}%
\pgfpathlineto{\pgfqpoint{2.245933in}{2.522027in}}%
\pgfpathlineto{\pgfqpoint{2.255138in}{2.516904in}}%
\pgfpathlineto{\pgfqpoint{2.264319in}{2.512136in}}%
\pgfpathlineto{\pgfqpoint{2.273478in}{2.507717in}}%
\pgfpathlineto{\pgfqpoint{2.282614in}{2.503641in}}%
\pgfpathlineto{\pgfqpoint{2.268413in}{2.527527in}}%
\pgfpathlineto{\pgfqpoint{2.254197in}{2.551742in}}%
\pgfpathlineto{\pgfqpoint{2.239965in}{2.576290in}}%
\pgfpathlineto{\pgfqpoint{2.225717in}{2.601174in}}%
\pgfpathlineto{\pgfqpoint{2.216521in}{2.605892in}}%
\pgfpathlineto{\pgfqpoint{2.207301in}{2.610962in}}%
\pgfpathlineto{\pgfqpoint{2.198056in}{2.616391in}}%
\pgfpathlineto{\pgfqpoint{2.188788in}{2.622185in}}%
\pgfpathclose%
\pgfusepath{fill}%
\end{pgfscope}%
\begin{pgfscope}%
\pgfpathrectangle{\pgfqpoint{1.150000in}{0.150000in}}{\pgfqpoint{5.700000in}{5.700000in}}%
\pgfusepath{clip}%
\pgfsetbuttcap%
\pgfsetroundjoin%
\definecolor{currentfill}{rgb}{0.258965,0.251537,0.524736}%
\pgfsetfillcolor{currentfill}%
\pgfsetfillopacity{0.800000}%
\pgfsetlinewidth{0.000000pt}%
\definecolor{currentstroke}{rgb}{0.000000,0.000000,0.000000}%
\pgfsetstrokecolor{currentstroke}%
\pgfsetdash{}{0pt}%
\pgfpathmoveto{\pgfqpoint{2.492456in}{2.106959in}}%
\pgfpathlineto{\pgfqpoint{2.506513in}{2.088060in}}%
\pgfpathlineto{\pgfqpoint{2.520559in}{2.069435in}}%
\pgfpathlineto{\pgfqpoint{2.534597in}{2.051081in}}%
\pgfpathlineto{\pgfqpoint{2.548625in}{2.032998in}}%
\pgfpathlineto{\pgfqpoint{2.557575in}{2.030514in}}%
\pgfpathlineto{\pgfqpoint{2.566505in}{2.028355in}}%
\pgfpathlineto{\pgfqpoint{2.575415in}{2.026515in}}%
\pgfpathlineto{\pgfqpoint{2.584307in}{2.024986in}}%
\pgfpathlineto{\pgfqpoint{2.570330in}{2.042411in}}%
\pgfpathlineto{\pgfqpoint{2.556344in}{2.060105in}}%
\pgfpathlineto{\pgfqpoint{2.542350in}{2.078069in}}%
\pgfpathlineto{\pgfqpoint{2.528346in}{2.096306in}}%
\pgfpathlineto{\pgfqpoint{2.519404in}{2.098481in}}%
\pgfpathlineto{\pgfqpoint{2.510442in}{2.100977in}}%
\pgfpathlineto{\pgfqpoint{2.501459in}{2.103801in}}%
\pgfpathlineto{\pgfqpoint{2.492456in}{2.106959in}}%
\pgfpathclose%
\pgfusepath{fill}%
\end{pgfscope}%
\begin{pgfscope}%
\pgfpathrectangle{\pgfqpoint{1.150000in}{0.150000in}}{\pgfqpoint{5.700000in}{5.700000in}}%
\pgfusepath{clip}%
\pgfsetbuttcap%
\pgfsetroundjoin%
\definecolor{currentfill}{rgb}{0.282623,0.140926,0.457517}%
\pgfsetfillcolor{currentfill}%
\pgfsetfillopacity{0.800000}%
\pgfsetlinewidth{0.000000pt}%
\definecolor{currentstroke}{rgb}{0.000000,0.000000,0.000000}%
\pgfsetstrokecolor{currentstroke}%
\pgfsetdash{}{0pt}%
\pgfpathmoveto{\pgfqpoint{2.716356in}{1.836310in}}%
\pgfpathlineto{\pgfqpoint{2.730291in}{1.821556in}}%
\pgfpathlineto{\pgfqpoint{2.744220in}{1.807046in}}%
\pgfpathlineto{\pgfqpoint{2.758143in}{1.792777in}}%
\pgfpathlineto{\pgfqpoint{2.772062in}{1.778749in}}%
\pgfpathlineto{\pgfqpoint{2.780820in}{1.778920in}}%
\pgfpathlineto{\pgfqpoint{2.789562in}{1.779380in}}%
\pgfpathlineto{\pgfqpoint{2.798287in}{1.780123in}}%
\pgfpathlineto{\pgfqpoint{2.806997in}{1.781141in}}%
\pgfpathlineto{\pgfqpoint{2.793121in}{1.794525in}}%
\pgfpathlineto{\pgfqpoint{2.779241in}{1.808149in}}%
\pgfpathlineto{\pgfqpoint{2.765356in}{1.822015in}}%
\pgfpathlineto{\pgfqpoint{2.751466in}{1.836123in}}%
\pgfpathlineto{\pgfqpoint{2.742713in}{1.835736in}}%
\pgfpathlineto{\pgfqpoint{2.733945in}{1.835634in}}%
\pgfpathlineto{\pgfqpoint{2.725159in}{1.835823in}}%
\pgfpathlineto{\pgfqpoint{2.716356in}{1.836310in}}%
\pgfpathclose%
\pgfusepath{fill}%
\end{pgfscope}%
\begin{pgfscope}%
\pgfpathrectangle{\pgfqpoint{1.150000in}{0.150000in}}{\pgfqpoint{5.700000in}{5.700000in}}%
\pgfusepath{clip}%
\pgfsetbuttcap%
\pgfsetroundjoin%
\definecolor{currentfill}{rgb}{0.283072,0.130895,0.449241}%
\pgfsetfillcolor{currentfill}%
\pgfsetfillopacity{0.800000}%
\pgfsetlinewidth{0.000000pt}%
\definecolor{currentstroke}{rgb}{0.000000,0.000000,0.000000}%
\pgfsetstrokecolor{currentstroke}%
\pgfsetdash{}{0pt}%
\pgfpathmoveto{\pgfqpoint{3.901003in}{1.749522in}}%
\pgfpathlineto{\pgfqpoint{3.914897in}{1.752605in}}%
\pgfpathlineto{\pgfqpoint{3.928801in}{1.755874in}}%
\pgfpathlineto{\pgfqpoint{3.942714in}{1.759330in}}%
\pgfpathlineto{\pgfqpoint{3.956638in}{1.762972in}}%
\pgfpathlineto{\pgfqpoint{3.964739in}{1.775779in}}%
\pgfpathlineto{\pgfqpoint{3.972835in}{1.788574in}}%
\pgfpathlineto{\pgfqpoint{3.980927in}{1.801352in}}%
\pgfpathlineto{\pgfqpoint{3.989014in}{1.814113in}}%
\pgfpathlineto{\pgfqpoint{3.975095in}{1.810126in}}%
\pgfpathlineto{\pgfqpoint{3.961186in}{1.806326in}}%
\pgfpathlineto{\pgfqpoint{3.947287in}{1.802713in}}%
\pgfpathlineto{\pgfqpoint{3.933398in}{1.799287in}}%
\pgfpathlineto{\pgfqpoint{3.925307in}{1.786859in}}%
\pgfpathlineto{\pgfqpoint{3.917210in}{1.774420in}}%
\pgfpathlineto{\pgfqpoint{3.909109in}{1.761973in}}%
\pgfpathlineto{\pgfqpoint{3.901003in}{1.749522in}}%
\pgfpathclose%
\pgfusepath{fill}%
\end{pgfscope}%
\begin{pgfscope}%
\pgfpathrectangle{\pgfqpoint{1.150000in}{0.150000in}}{\pgfqpoint{5.700000in}{5.700000in}}%
\pgfusepath{clip}%
\pgfsetbuttcap%
\pgfsetroundjoin%
\definecolor{currentfill}{rgb}{0.120565,0.596422,0.543611}%
\pgfsetfillcolor{currentfill}%
\pgfsetfillopacity{0.800000}%
\pgfsetlinewidth{0.000000pt}%
\definecolor{currentstroke}{rgb}{0.000000,0.000000,0.000000}%
\pgfsetstrokecolor{currentstroke}%
\pgfsetdash{}{0pt}%
\pgfpathmoveto{\pgfqpoint{5.238766in}{3.027394in}}%
\pgfpathlineto{\pgfqpoint{5.253374in}{3.041040in}}%
\pgfpathlineto{\pgfqpoint{5.268003in}{3.054869in}}%
\pgfpathlineto{\pgfqpoint{5.282653in}{3.068883in}}%
\pgfpathlineto{\pgfqpoint{5.297322in}{3.083082in}}%
\pgfpathlineto{\pgfqpoint{5.304875in}{3.088409in}}%
\pgfpathlineto{\pgfqpoint{5.312419in}{3.093618in}}%
\pgfpathlineto{\pgfqpoint{5.319954in}{3.098712in}}%
\pgfpathlineto{\pgfqpoint{5.327479in}{3.103695in}}%
\pgfpathlineto{\pgfqpoint{5.312824in}{3.089790in}}%
\pgfpathlineto{\pgfqpoint{5.298189in}{3.076068in}}%
\pgfpathlineto{\pgfqpoint{5.283575in}{3.062529in}}%
\pgfpathlineto{\pgfqpoint{5.268980in}{3.049175in}}%
\pgfpathlineto{\pgfqpoint{5.261439in}{3.043888in}}%
\pgfpathlineto{\pgfqpoint{5.253890in}{3.038497in}}%
\pgfpathlineto{\pgfqpoint{5.246332in}{3.033001in}}%
\pgfpathlineto{\pgfqpoint{5.238766in}{3.027394in}}%
\pgfpathclose%
\pgfusepath{fill}%
\end{pgfscope}%
\begin{pgfscope}%
\pgfpathrectangle{\pgfqpoint{1.150000in}{0.150000in}}{\pgfqpoint{5.700000in}{5.700000in}}%
\pgfusepath{clip}%
\pgfsetbuttcap%
\pgfsetroundjoin%
\definecolor{currentfill}{rgb}{0.166383,0.690856,0.496502}%
\pgfsetfillcolor{currentfill}%
\pgfsetfillopacity{0.800000}%
\pgfsetlinewidth{0.000000pt}%
\definecolor{currentstroke}{rgb}{0.000000,0.000000,0.000000}%
\pgfsetstrokecolor{currentstroke}%
\pgfsetdash{}{0pt}%
\pgfpathmoveto{\pgfqpoint{5.623308in}{3.334427in}}%
\pgfpathlineto{\pgfqpoint{5.638161in}{3.349196in}}%
\pgfpathlineto{\pgfqpoint{5.653036in}{3.364147in}}%
\pgfpathlineto{\pgfqpoint{5.667934in}{3.379283in}}%
\pgfpathlineto{\pgfqpoint{5.682853in}{3.394601in}}%
\pgfpathlineto{\pgfqpoint{5.690145in}{3.396495in}}%
\pgfpathlineto{\pgfqpoint{5.697427in}{3.398321in}}%
\pgfpathlineto{\pgfqpoint{5.704701in}{3.400086in}}%
\pgfpathlineto{\pgfqpoint{5.711966in}{3.401795in}}%
\pgfpathlineto{\pgfqpoint{5.697072in}{3.386947in}}%
\pgfpathlineto{\pgfqpoint{5.682200in}{3.372282in}}%
\pgfpathlineto{\pgfqpoint{5.667350in}{3.357799in}}%
\pgfpathlineto{\pgfqpoint{5.652521in}{3.343498in}}%
\pgfpathlineto{\pgfqpoint{5.645231in}{3.341308in}}%
\pgfpathlineto{\pgfqpoint{5.637932in}{3.339070in}}%
\pgfpathlineto{\pgfqpoint{5.630624in}{3.336778in}}%
\pgfpathlineto{\pgfqpoint{5.623308in}{3.334427in}}%
\pgfpathclose%
\pgfusepath{fill}%
\end{pgfscope}%
\begin{pgfscope}%
\pgfpathrectangle{\pgfqpoint{1.150000in}{0.150000in}}{\pgfqpoint{5.700000in}{5.700000in}}%
\pgfusepath{clip}%
\pgfsetbuttcap%
\pgfsetroundjoin%
\definecolor{currentfill}{rgb}{0.267004,0.004874,0.329415}%
\pgfsetfillcolor{currentfill}%
\pgfsetfillopacity{0.800000}%
\pgfsetlinewidth{0.000000pt}%
\definecolor{currentstroke}{rgb}{0.000000,0.000000,0.000000}%
\pgfsetstrokecolor{currentstroke}%
\pgfsetdash{}{0pt}%
\pgfpathmoveto{\pgfqpoint{3.228281in}{1.521684in}}%
\pgfpathlineto{\pgfqpoint{3.242089in}{1.515280in}}%
\pgfpathlineto{\pgfqpoint{3.255898in}{1.509080in}}%
\pgfpathlineto{\pgfqpoint{3.269709in}{1.503083in}}%
\pgfpathlineto{\pgfqpoint{3.283523in}{1.497288in}}%
\pgfpathlineto{\pgfqpoint{3.291910in}{1.504231in}}%
\pgfpathlineto{\pgfqpoint{3.300287in}{1.511346in}}%
\pgfpathlineto{\pgfqpoint{3.308656in}{1.518629in}}%
\pgfpathlineto{\pgfqpoint{3.317015in}{1.526073in}}%
\pgfpathlineto{\pgfqpoint{3.303224in}{1.531305in}}%
\pgfpathlineto{\pgfqpoint{3.289437in}{1.536738in}}%
\pgfpathlineto{\pgfqpoint{3.275652in}{1.542375in}}%
\pgfpathlineto{\pgfqpoint{3.261869in}{1.548216in}}%
\pgfpathlineto{\pgfqpoint{3.253487in}{1.541322in}}%
\pgfpathlineto{\pgfqpoint{3.245095in}{1.534599in}}%
\pgfpathlineto{\pgfqpoint{3.236693in}{1.528051in}}%
\pgfpathlineto{\pgfqpoint{3.228281in}{1.521684in}}%
\pgfpathclose%
\pgfusepath{fill}%
\end{pgfscope}%
\begin{pgfscope}%
\pgfpathrectangle{\pgfqpoint{1.150000in}{0.150000in}}{\pgfqpoint{5.700000in}{5.700000in}}%
\pgfusepath{clip}%
\pgfsetbuttcap%
\pgfsetroundjoin%
\definecolor{currentfill}{rgb}{0.248629,0.278775,0.534556}%
\pgfsetfillcolor{currentfill}%
\pgfsetfillopacity{0.800000}%
\pgfsetlinewidth{0.000000pt}%
\definecolor{currentstroke}{rgb}{0.000000,0.000000,0.000000}%
\pgfsetstrokecolor{currentstroke}%
\pgfsetdash{}{0pt}%
\pgfpathmoveto{\pgfqpoint{2.436130in}{2.185341in}}%
\pgfpathlineto{\pgfqpoint{2.450228in}{2.165323in}}%
\pgfpathlineto{\pgfqpoint{2.464314in}{2.145588in}}%
\pgfpathlineto{\pgfqpoint{2.478390in}{2.126135in}}%
\pgfpathlineto{\pgfqpoint{2.492456in}{2.106959in}}%
\pgfpathlineto{\pgfqpoint{2.501459in}{2.103801in}}%
\pgfpathlineto{\pgfqpoint{2.510442in}{2.100977in}}%
\pgfpathlineto{\pgfqpoint{2.519404in}{2.098481in}}%
\pgfpathlineto{\pgfqpoint{2.528346in}{2.096306in}}%
\pgfpathlineto{\pgfqpoint{2.514334in}{2.114818in}}%
\pgfpathlineto{\pgfqpoint{2.500311in}{2.133608in}}%
\pgfpathlineto{\pgfqpoint{2.486279in}{2.152677in}}%
\pgfpathlineto{\pgfqpoint{2.472237in}{2.172028in}}%
\pgfpathlineto{\pgfqpoint{2.463241in}{2.174854in}}%
\pgfpathlineto{\pgfqpoint{2.454225in}{2.178010in}}%
\pgfpathlineto{\pgfqpoint{2.445189in}{2.181504in}}%
\pgfpathlineto{\pgfqpoint{2.436130in}{2.185341in}}%
\pgfpathclose%
\pgfusepath{fill}%
\end{pgfscope}%
\begin{pgfscope}%
\pgfpathrectangle{\pgfqpoint{1.150000in}{0.150000in}}{\pgfqpoint{5.700000in}{5.700000in}}%
\pgfusepath{clip}%
\pgfsetbuttcap%
\pgfsetroundjoin%
\definecolor{currentfill}{rgb}{0.283197,0.115680,0.436115}%
\pgfsetfillcolor{currentfill}%
\pgfsetfillopacity{0.800000}%
\pgfsetlinewidth{0.000000pt}%
\definecolor{currentstroke}{rgb}{0.000000,0.000000,0.000000}%
\pgfsetstrokecolor{currentstroke}%
\pgfsetdash{}{0pt}%
\pgfpathmoveto{\pgfqpoint{2.772062in}{1.778749in}}%
\pgfpathlineto{\pgfqpoint{2.785976in}{1.764960in}}%
\pgfpathlineto{\pgfqpoint{2.799885in}{1.751408in}}%
\pgfpathlineto{\pgfqpoint{2.813790in}{1.738092in}}%
\pgfpathlineto{\pgfqpoint{2.827690in}{1.725010in}}%
\pgfpathlineto{\pgfqpoint{2.836405in}{1.725836in}}%
\pgfpathlineto{\pgfqpoint{2.845105in}{1.726942in}}%
\pgfpathlineto{\pgfqpoint{2.853789in}{1.728322in}}%
\pgfpathlineto{\pgfqpoint{2.862458in}{1.729969in}}%
\pgfpathlineto{\pgfqpoint{2.848598in}{1.742410in}}%
\pgfpathlineto{\pgfqpoint{2.834735in}{1.755085in}}%
\pgfpathlineto{\pgfqpoint{2.820868in}{1.767994in}}%
\pgfpathlineto{\pgfqpoint{2.806997in}{1.781141in}}%
\pgfpathlineto{\pgfqpoint{2.798287in}{1.780123in}}%
\pgfpathlineto{\pgfqpoint{2.789562in}{1.779380in}}%
\pgfpathlineto{\pgfqpoint{2.780820in}{1.778920in}}%
\pgfpathlineto{\pgfqpoint{2.772062in}{1.778749in}}%
\pgfpathclose%
\pgfusepath{fill}%
\end{pgfscope}%
\begin{pgfscope}%
\pgfpathrectangle{\pgfqpoint{1.150000in}{0.150000in}}{\pgfqpoint{5.700000in}{5.700000in}}%
\pgfusepath{clip}%
\pgfsetbuttcap%
\pgfsetroundjoin%
\definecolor{currentfill}{rgb}{0.260571,0.246922,0.522828}%
\pgfsetfillcolor{currentfill}%
\pgfsetfillopacity{0.800000}%
\pgfsetlinewidth{0.000000pt}%
\definecolor{currentstroke}{rgb}{0.000000,0.000000,0.000000}%
\pgfsetstrokecolor{currentstroke}%
\pgfsetdash{}{0pt}%
\pgfpathmoveto{\pgfqpoint{4.197366in}{2.010114in}}%
\pgfpathlineto{\pgfqpoint{4.211380in}{2.016616in}}%
\pgfpathlineto{\pgfqpoint{4.225407in}{2.023303in}}%
\pgfpathlineto{\pgfqpoint{4.239446in}{2.030175in}}%
\pgfpathlineto{\pgfqpoint{4.253499in}{2.037232in}}%
\pgfpathlineto{\pgfqpoint{4.261517in}{2.050197in}}%
\pgfpathlineto{\pgfqpoint{4.269531in}{2.063086in}}%
\pgfpathlineto{\pgfqpoint{4.277540in}{2.075896in}}%
\pgfpathlineto{\pgfqpoint{4.285544in}{2.088626in}}%
\pgfpathlineto{\pgfqpoint{4.271493in}{2.081351in}}%
\pgfpathlineto{\pgfqpoint{4.257455in}{2.074261in}}%
\pgfpathlineto{\pgfqpoint{4.243430in}{2.067356in}}%
\pgfpathlineto{\pgfqpoint{4.229418in}{2.060637in}}%
\pgfpathlineto{\pgfqpoint{4.221413in}{2.048113in}}%
\pgfpathlineto{\pgfqpoint{4.213402in}{2.035517in}}%
\pgfpathlineto{\pgfqpoint{4.205387in}{2.022850in}}%
\pgfpathlineto{\pgfqpoint{4.197366in}{2.010114in}}%
\pgfpathclose%
\pgfusepath{fill}%
\end{pgfscope}%
\begin{pgfscope}%
\pgfpathrectangle{\pgfqpoint{1.150000in}{0.150000in}}{\pgfqpoint{5.700000in}{5.700000in}}%
\pgfusepath{clip}%
\pgfsetbuttcap%
\pgfsetroundjoin%
\definecolor{currentfill}{rgb}{0.267004,0.004874,0.329415}%
\pgfsetfillcolor{currentfill}%
\pgfsetfillopacity{0.800000}%
\pgfsetlinewidth{0.000000pt}%
\definecolor{currentstroke}{rgb}{0.000000,0.000000,0.000000}%
\pgfsetstrokecolor{currentstroke}%
\pgfsetdash{}{0pt}%
\pgfpathmoveto{\pgfqpoint{3.372205in}{1.507154in}}%
\pgfpathlineto{\pgfqpoint{3.386011in}{1.502923in}}%
\pgfpathlineto{\pgfqpoint{3.399821in}{1.498888in}}%
\pgfpathlineto{\pgfqpoint{3.413635in}{1.495051in}}%
\pgfpathlineto{\pgfqpoint{3.427453in}{1.491410in}}%
\pgfpathlineto{\pgfqpoint{3.435762in}{1.500097in}}%
\pgfpathlineto{\pgfqpoint{3.444063in}{1.508919in}}%
\pgfpathlineto{\pgfqpoint{3.452356in}{1.517871in}}%
\pgfpathlineto{\pgfqpoint{3.460641in}{1.526948in}}%
\pgfpathlineto{\pgfqpoint{3.446842in}{1.530058in}}%
\pgfpathlineto{\pgfqpoint{3.433047in}{1.533364in}}%
\pgfpathlineto{\pgfqpoint{3.419257in}{1.536868in}}%
\pgfpathlineto{\pgfqpoint{3.405470in}{1.540569in}}%
\pgfpathlineto{\pgfqpoint{3.397166in}{1.532011in}}%
\pgfpathlineto{\pgfqpoint{3.388854in}{1.523586in}}%
\pgfpathlineto{\pgfqpoint{3.380534in}{1.515298in}}%
\pgfpathlineto{\pgfqpoint{3.372205in}{1.507154in}}%
\pgfpathclose%
\pgfusepath{fill}%
\end{pgfscope}%
\begin{pgfscope}%
\pgfpathrectangle{\pgfqpoint{1.150000in}{0.150000in}}{\pgfqpoint{5.700000in}{5.700000in}}%
\pgfusepath{clip}%
\pgfsetbuttcap%
\pgfsetroundjoin%
\definecolor{currentfill}{rgb}{0.280868,0.160771,0.472899}%
\pgfsetfillcolor{currentfill}%
\pgfsetfillopacity{0.800000}%
\pgfsetlinewidth{0.000000pt}%
\definecolor{currentstroke}{rgb}{0.000000,0.000000,0.000000}%
\pgfsetstrokecolor{currentstroke}%
\pgfsetdash{}{0pt}%
\pgfpathmoveto{\pgfqpoint{3.989014in}{1.814113in}}%
\pgfpathlineto{\pgfqpoint{4.002943in}{1.818285in}}%
\pgfpathlineto{\pgfqpoint{4.016884in}{1.822644in}}%
\pgfpathlineto{\pgfqpoint{4.030835in}{1.827189in}}%
\pgfpathlineto{\pgfqpoint{4.044797in}{1.831919in}}%
\pgfpathlineto{\pgfqpoint{4.052876in}{1.844983in}}%
\pgfpathlineto{\pgfqpoint{4.060950in}{1.858014in}}%
\pgfpathlineto{\pgfqpoint{4.069019in}{1.871010in}}%
\pgfpathlineto{\pgfqpoint{4.077084in}{1.883967in}}%
\pgfpathlineto{\pgfqpoint{4.063125in}{1.878924in}}%
\pgfpathlineto{\pgfqpoint{4.049177in}{1.874066in}}%
\pgfpathlineto{\pgfqpoint{4.035240in}{1.869394in}}%
\pgfpathlineto{\pgfqpoint{4.021315in}{1.864909in}}%
\pgfpathlineto{\pgfqpoint{4.013246in}{1.852252in}}%
\pgfpathlineto{\pgfqpoint{4.005174in}{1.839565in}}%
\pgfpathlineto{\pgfqpoint{3.997096in}{1.826851in}}%
\pgfpathlineto{\pgfqpoint{3.989014in}{1.814113in}}%
\pgfpathclose%
\pgfusepath{fill}%
\end{pgfscope}%
\begin{pgfscope}%
\pgfpathrectangle{\pgfqpoint{1.150000in}{0.150000in}}{\pgfqpoint{5.700000in}{5.700000in}}%
\pgfusepath{clip}%
\pgfsetbuttcap%
\pgfsetroundjoin%
\definecolor{currentfill}{rgb}{0.227802,0.326594,0.546532}%
\pgfsetfillcolor{currentfill}%
\pgfsetfillopacity{0.800000}%
\pgfsetlinewidth{0.000000pt}%
\definecolor{currentstroke}{rgb}{0.000000,0.000000,0.000000}%
\pgfsetstrokecolor{currentstroke}%
\pgfsetdash{}{0pt}%
\pgfpathmoveto{\pgfqpoint{4.405724in}{2.219686in}}%
\pgfpathlineto{\pgfqpoint{4.419842in}{2.228223in}}%
\pgfpathlineto{\pgfqpoint{4.433974in}{2.236946in}}%
\pgfpathlineto{\pgfqpoint{4.448121in}{2.245853in}}%
\pgfpathlineto{\pgfqpoint{4.462283in}{2.254945in}}%
\pgfpathlineto{\pgfqpoint{4.470238in}{2.267151in}}%
\pgfpathlineto{\pgfqpoint{4.478188in}{2.279247in}}%
\pgfpathlineto{\pgfqpoint{4.486132in}{2.291232in}}%
\pgfpathlineto{\pgfqpoint{4.494071in}{2.303106in}}%
\pgfpathlineto{\pgfqpoint{4.479910in}{2.293894in}}%
\pgfpathlineto{\pgfqpoint{4.465764in}{2.284866in}}%
\pgfpathlineto{\pgfqpoint{4.451634in}{2.276024in}}%
\pgfpathlineto{\pgfqpoint{4.437517in}{2.267366in}}%
\pgfpathlineto{\pgfqpoint{4.429577in}{2.255600in}}%
\pgfpathlineto{\pgfqpoint{4.421632in}{2.243730in}}%
\pgfpathlineto{\pgfqpoint{4.413681in}{2.231759in}}%
\pgfpathlineto{\pgfqpoint{4.405724in}{2.219686in}}%
\pgfpathclose%
\pgfusepath{fill}%
\end{pgfscope}%
\begin{pgfscope}%
\pgfpathrectangle{\pgfqpoint{1.150000in}{0.150000in}}{\pgfqpoint{5.700000in}{5.700000in}}%
\pgfusepath{clip}%
\pgfsetbuttcap%
\pgfsetroundjoin%
\definecolor{currentfill}{rgb}{0.271305,0.019942,0.347269}%
\pgfsetfillcolor{currentfill}%
\pgfsetfillopacity{0.800000}%
\pgfsetlinewidth{0.000000pt}%
\definecolor{currentstroke}{rgb}{0.000000,0.000000,0.000000}%
\pgfsetstrokecolor{currentstroke}%
\pgfsetdash{}{0pt}%
\pgfpathmoveto{\pgfqpoint{3.083886in}{1.561604in}}%
\pgfpathlineto{\pgfqpoint{3.097715in}{1.552937in}}%
\pgfpathlineto{\pgfqpoint{3.111544in}{1.544481in}}%
\pgfpathlineto{\pgfqpoint{3.125374in}{1.536237in}}%
\pgfpathlineto{\pgfqpoint{3.139204in}{1.528202in}}%
\pgfpathlineto{\pgfqpoint{3.147686in}{1.533178in}}%
\pgfpathlineto{\pgfqpoint{3.156158in}{1.538367in}}%
\pgfpathlineto{\pgfqpoint{3.164618in}{1.543762in}}%
\pgfpathlineto{\pgfqpoint{3.173067in}{1.549358in}}%
\pgfpathlineto{\pgfqpoint{3.159266in}{1.556795in}}%
\pgfpathlineto{\pgfqpoint{3.145467in}{1.564442in}}%
\pgfpathlineto{\pgfqpoint{3.131667in}{1.572299in}}%
\pgfpathlineto{\pgfqpoint{3.117869in}{1.580368in}}%
\pgfpathlineto{\pgfqpoint{3.109390in}{1.575358in}}%
\pgfpathlineto{\pgfqpoint{3.100901in}{1.570556in}}%
\pgfpathlineto{\pgfqpoint{3.092399in}{1.565970in}}%
\pgfpathlineto{\pgfqpoint{3.083886in}{1.561604in}}%
\pgfpathclose%
\pgfusepath{fill}%
\end{pgfscope}%
\begin{pgfscope}%
\pgfpathrectangle{\pgfqpoint{1.150000in}{0.150000in}}{\pgfqpoint{5.700000in}{5.700000in}}%
\pgfusepath{clip}%
\pgfsetbuttcap%
\pgfsetroundjoin%
\definecolor{currentfill}{rgb}{0.202219,0.715272,0.476084}%
\pgfsetfillcolor{currentfill}%
\pgfsetfillopacity{0.800000}%
\pgfsetlinewidth{0.000000pt}%
\definecolor{currentstroke}{rgb}{0.000000,0.000000,0.000000}%
\pgfsetstrokecolor{currentstroke}%
\pgfsetdash{}{0pt}%
\pgfpathmoveto{\pgfqpoint{5.711966in}{3.401795in}}%
\pgfpathlineto{\pgfqpoint{5.726882in}{3.416826in}}%
\pgfpathlineto{\pgfqpoint{5.741821in}{3.432039in}}%
\pgfpathlineto{\pgfqpoint{5.756783in}{3.447437in}}%
\pgfpathlineto{\pgfqpoint{5.771767in}{3.463017in}}%
\pgfpathlineto{\pgfqpoint{5.778996in}{3.464181in}}%
\pgfpathlineto{\pgfqpoint{5.786216in}{3.465291in}}%
\pgfpathlineto{\pgfqpoint{5.793428in}{3.466352in}}%
\pgfpathlineto{\pgfqpoint{5.800631in}{3.467371in}}%
\pgfpathlineto{\pgfqpoint{5.785675in}{3.452297in}}%
\pgfpathlineto{\pgfqpoint{5.770741in}{3.437406in}}%
\pgfpathlineto{\pgfqpoint{5.755830in}{3.422698in}}%
\pgfpathlineto{\pgfqpoint{5.740941in}{3.408171in}}%
\pgfpathlineto{\pgfqpoint{5.733710in}{3.406635in}}%
\pgfpathlineto{\pgfqpoint{5.726470in}{3.405064in}}%
\pgfpathlineto{\pgfqpoint{5.719222in}{3.403452in}}%
\pgfpathlineto{\pgfqpoint{5.711966in}{3.401795in}}%
\pgfpathclose%
\pgfusepath{fill}%
\end{pgfscope}%
\begin{pgfscope}%
\pgfpathrectangle{\pgfqpoint{1.150000in}{0.150000in}}{\pgfqpoint{5.700000in}{5.700000in}}%
\pgfusepath{clip}%
\pgfsetbuttcap%
\pgfsetroundjoin%
\definecolor{currentfill}{rgb}{0.282327,0.094955,0.417331}%
\pgfsetfillcolor{currentfill}%
\pgfsetfillopacity{0.800000}%
\pgfsetlinewidth{0.000000pt}%
\definecolor{currentstroke}{rgb}{0.000000,0.000000,0.000000}%
\pgfsetstrokecolor{currentstroke}%
\pgfsetdash{}{0pt}%
\pgfpathmoveto{\pgfqpoint{2.827690in}{1.725010in}}%
\pgfpathlineto{\pgfqpoint{2.841587in}{1.712161in}}%
\pgfpathlineto{\pgfqpoint{2.855480in}{1.699544in}}%
\pgfpathlineto{\pgfqpoint{2.869369in}{1.687156in}}%
\pgfpathlineto{\pgfqpoint{2.883255in}{1.674998in}}%
\pgfpathlineto{\pgfqpoint{2.891930in}{1.676476in}}%
\pgfpathlineto{\pgfqpoint{2.900589in}{1.678226in}}%
\pgfpathlineto{\pgfqpoint{2.909234in}{1.680241in}}%
\pgfpathlineto{\pgfqpoint{2.917864in}{1.682514in}}%
\pgfpathlineto{\pgfqpoint{2.904017in}{1.694034in}}%
\pgfpathlineto{\pgfqpoint{2.890167in}{1.705783in}}%
\pgfpathlineto{\pgfqpoint{2.876314in}{1.717761in}}%
\pgfpathlineto{\pgfqpoint{2.862458in}{1.729969in}}%
\pgfpathlineto{\pgfqpoint{2.853789in}{1.728322in}}%
\pgfpathlineto{\pgfqpoint{2.845105in}{1.726942in}}%
\pgfpathlineto{\pgfqpoint{2.836405in}{1.725836in}}%
\pgfpathlineto{\pgfqpoint{2.827690in}{1.725010in}}%
\pgfpathclose%
\pgfusepath{fill}%
\end{pgfscope}%
\begin{pgfscope}%
\pgfpathrectangle{\pgfqpoint{1.150000in}{0.150000in}}{\pgfqpoint{5.700000in}{5.700000in}}%
\pgfusepath{clip}%
\pgfsetbuttcap%
\pgfsetroundjoin%
\definecolor{currentfill}{rgb}{0.137770,0.537492,0.554906}%
\pgfsetfillcolor{currentfill}%
\pgfsetfillopacity{0.800000}%
\pgfsetlinewidth{0.000000pt}%
\definecolor{currentstroke}{rgb}{0.000000,0.000000,0.000000}%
\pgfsetstrokecolor{currentstroke}%
\pgfsetdash{}{0pt}%
\pgfpathmoveto{\pgfqpoint{5.030859in}{2.842565in}}%
\pgfpathlineto{\pgfqpoint{5.045345in}{2.855402in}}%
\pgfpathlineto{\pgfqpoint{5.059850in}{2.868423in}}%
\pgfpathlineto{\pgfqpoint{5.074373in}{2.881629in}}%
\pgfpathlineto{\pgfqpoint{5.088916in}{2.895019in}}%
\pgfpathlineto{\pgfqpoint{5.096597in}{2.902352in}}%
\pgfpathlineto{\pgfqpoint{5.104269in}{2.909550in}}%
\pgfpathlineto{\pgfqpoint{5.111933in}{2.916615in}}%
\pgfpathlineto{\pgfqpoint{5.119588in}{2.923549in}}%
\pgfpathlineto{\pgfqpoint{5.105055in}{2.910346in}}%
\pgfpathlineto{\pgfqpoint{5.090540in}{2.897327in}}%
\pgfpathlineto{\pgfqpoint{5.076045in}{2.884493in}}%
\pgfpathlineto{\pgfqpoint{5.061568in}{2.871842in}}%
\pgfpathlineto{\pgfqpoint{5.053903in}{2.864708in}}%
\pgfpathlineto{\pgfqpoint{5.046230in}{2.857452in}}%
\pgfpathlineto{\pgfqpoint{5.038549in}{2.850072in}}%
\pgfpathlineto{\pgfqpoint{5.030859in}{2.842565in}}%
\pgfpathclose%
\pgfusepath{fill}%
\end{pgfscope}%
\begin{pgfscope}%
\pgfpathrectangle{\pgfqpoint{1.150000in}{0.150000in}}{\pgfqpoint{5.700000in}{5.700000in}}%
\pgfusepath{clip}%
\pgfsetbuttcap%
\pgfsetroundjoin%
\definecolor{currentfill}{rgb}{0.233603,0.313828,0.543914}%
\pgfsetfillcolor{currentfill}%
\pgfsetfillopacity{0.800000}%
\pgfsetlinewidth{0.000000pt}%
\definecolor{currentstroke}{rgb}{0.000000,0.000000,0.000000}%
\pgfsetstrokecolor{currentstroke}%
\pgfsetdash{}{0pt}%
\pgfpathmoveto{\pgfqpoint{2.379628in}{2.268292in}}%
\pgfpathlineto{\pgfqpoint{2.393771in}{2.247118in}}%
\pgfpathlineto{\pgfqpoint{2.407902in}{2.226236in}}%
\pgfpathlineto{\pgfqpoint{2.422022in}{2.205645in}}%
\pgfpathlineto{\pgfqpoint{2.436130in}{2.185341in}}%
\pgfpathlineto{\pgfqpoint{2.445189in}{2.181504in}}%
\pgfpathlineto{\pgfqpoint{2.454225in}{2.178010in}}%
\pgfpathlineto{\pgfqpoint{2.463241in}{2.174854in}}%
\pgfpathlineto{\pgfqpoint{2.472237in}{2.172028in}}%
\pgfpathlineto{\pgfqpoint{2.458184in}{2.191663in}}%
\pgfpathlineto{\pgfqpoint{2.444120in}{2.211584in}}%
\pgfpathlineto{\pgfqpoint{2.430046in}{2.231795in}}%
\pgfpathlineto{\pgfqpoint{2.415960in}{2.252297in}}%
\pgfpathlineto{\pgfqpoint{2.406910in}{2.255779in}}%
\pgfpathlineto{\pgfqpoint{2.397838in}{2.259601in}}%
\pgfpathlineto{\pgfqpoint{2.388744in}{2.263770in}}%
\pgfpathlineto{\pgfqpoint{2.379628in}{2.268292in}}%
\pgfpathclose%
\pgfusepath{fill}%
\end{pgfscope}%
\begin{pgfscope}%
\pgfpathrectangle{\pgfqpoint{1.150000in}{0.150000in}}{\pgfqpoint{5.700000in}{5.700000in}}%
\pgfusepath{clip}%
\pgfsetbuttcap%
\pgfsetroundjoin%
\definecolor{currentfill}{rgb}{0.190631,0.407061,0.556089}%
\pgfsetfillcolor{currentfill}%
\pgfsetfillopacity{0.800000}%
\pgfsetlinewidth{0.000000pt}%
\definecolor{currentstroke}{rgb}{0.000000,0.000000,0.000000}%
\pgfsetstrokecolor{currentstroke}%
\pgfsetdash{}{0pt}%
\pgfpathmoveto{\pgfqpoint{4.614151in}{2.433292in}}%
\pgfpathlineto{\pgfqpoint{4.628386in}{2.443568in}}%
\pgfpathlineto{\pgfqpoint{4.642637in}{2.454030in}}%
\pgfpathlineto{\pgfqpoint{4.656904in}{2.464675in}}%
\pgfpathlineto{\pgfqpoint{4.671188in}{2.475506in}}%
\pgfpathlineto{\pgfqpoint{4.679069in}{2.486425in}}%
\pgfpathlineto{\pgfqpoint{4.686943in}{2.497214in}}%
\pgfpathlineto{\pgfqpoint{4.694811in}{2.507872in}}%
\pgfpathlineto{\pgfqpoint{4.702672in}{2.518400in}}%
\pgfpathlineto{\pgfqpoint{4.688391in}{2.507550in}}%
\pgfpathlineto{\pgfqpoint{4.674126in}{2.496884in}}%
\pgfpathlineto{\pgfqpoint{4.659878in}{2.486403in}}%
\pgfpathlineto{\pgfqpoint{4.645646in}{2.476107in}}%
\pgfpathlineto{\pgfqpoint{4.637782in}{2.465586in}}%
\pgfpathlineto{\pgfqpoint{4.629911in}{2.454943in}}%
\pgfpathlineto{\pgfqpoint{4.622034in}{2.444179in}}%
\pgfpathlineto{\pgfqpoint{4.614151in}{2.433292in}}%
\pgfpathclose%
\pgfusepath{fill}%
\end{pgfscope}%
\begin{pgfscope}%
\pgfpathrectangle{\pgfqpoint{1.150000in}{0.150000in}}{\pgfqpoint{5.700000in}{5.700000in}}%
\pgfusepath{clip}%
\pgfsetbuttcap%
\pgfsetroundjoin%
\definecolor{currentfill}{rgb}{0.120081,0.622161,0.534946}%
\pgfsetfillcolor{currentfill}%
\pgfsetfillopacity{0.800000}%
\pgfsetlinewidth{0.000000pt}%
\definecolor{currentstroke}{rgb}{0.000000,0.000000,0.000000}%
\pgfsetstrokecolor{currentstroke}%
\pgfsetdash{}{0pt}%
\pgfpathmoveto{\pgfqpoint{5.327479in}{3.103695in}}%
\pgfpathlineto{\pgfqpoint{5.342155in}{3.117785in}}%
\pgfpathlineto{\pgfqpoint{5.356852in}{3.132058in}}%
\pgfpathlineto{\pgfqpoint{5.371569in}{3.146517in}}%
\pgfpathlineto{\pgfqpoint{5.386307in}{3.161159in}}%
\pgfpathlineto{\pgfqpoint{5.393808in}{3.165718in}}%
\pgfpathlineto{\pgfqpoint{5.401300in}{3.170165in}}%
\pgfpathlineto{\pgfqpoint{5.408783in}{3.174503in}}%
\pgfpathlineto{\pgfqpoint{5.416257in}{3.178736in}}%
\pgfpathlineto{\pgfqpoint{5.401535in}{3.164423in}}%
\pgfpathlineto{\pgfqpoint{5.386834in}{3.150293in}}%
\pgfpathlineto{\pgfqpoint{5.372154in}{3.136347in}}%
\pgfpathlineto{\pgfqpoint{5.357494in}{3.122584in}}%
\pgfpathlineto{\pgfqpoint{5.350004in}{3.118011in}}%
\pgfpathlineto{\pgfqpoint{5.342505in}{3.113341in}}%
\pgfpathlineto{\pgfqpoint{5.334996in}{3.108570in}}%
\pgfpathlineto{\pgfqpoint{5.327479in}{3.103695in}}%
\pgfpathclose%
\pgfusepath{fill}%
\end{pgfscope}%
\begin{pgfscope}%
\pgfpathrectangle{\pgfqpoint{1.150000in}{0.150000in}}{\pgfqpoint{5.700000in}{5.700000in}}%
\pgfusepath{clip}%
\pgfsetbuttcap%
\pgfsetroundjoin%
\definecolor{currentfill}{rgb}{0.162142,0.474838,0.558140}%
\pgfsetfillcolor{currentfill}%
\pgfsetfillopacity{0.800000}%
\pgfsetlinewidth{0.000000pt}%
\definecolor{currentstroke}{rgb}{0.000000,0.000000,0.000000}%
\pgfsetstrokecolor{currentstroke}%
\pgfsetdash{}{0pt}%
\pgfpathmoveto{\pgfqpoint{4.822583in}{2.643009in}}%
\pgfpathlineto{\pgfqpoint{4.836942in}{2.654721in}}%
\pgfpathlineto{\pgfqpoint{4.851319in}{2.666617in}}%
\pgfpathlineto{\pgfqpoint{4.865713in}{2.678698in}}%
\pgfpathlineto{\pgfqpoint{4.880126in}{2.690964in}}%
\pgfpathlineto{\pgfqpoint{4.887916in}{2.700211in}}%
\pgfpathlineto{\pgfqpoint{4.895699in}{2.709318in}}%
\pgfpathlineto{\pgfqpoint{4.903474in}{2.718288in}}%
\pgfpathlineto{\pgfqpoint{4.911241in}{2.727121in}}%
\pgfpathlineto{\pgfqpoint{4.896834in}{2.714938in}}%
\pgfpathlineto{\pgfqpoint{4.882444in}{2.702940in}}%
\pgfpathlineto{\pgfqpoint{4.868073in}{2.691126in}}%
\pgfpathlineto{\pgfqpoint{4.853719in}{2.679497in}}%
\pgfpathlineto{\pgfqpoint{4.845946in}{2.670568in}}%
\pgfpathlineto{\pgfqpoint{4.838165in}{2.661512in}}%
\pgfpathlineto{\pgfqpoint{4.830377in}{2.652326in}}%
\pgfpathlineto{\pgfqpoint{4.822583in}{2.643009in}}%
\pgfpathclose%
\pgfusepath{fill}%
\end{pgfscope}%
\begin{pgfscope}%
\pgfpathrectangle{\pgfqpoint{1.150000in}{0.150000in}}{\pgfqpoint{5.700000in}{5.700000in}}%
\pgfusepath{clip}%
\pgfsetbuttcap%
\pgfsetroundjoin%
\definecolor{currentfill}{rgb}{0.165117,0.467423,0.558141}%
\pgfsetfillcolor{currentfill}%
\pgfsetfillopacity{0.800000}%
\pgfsetlinewidth{0.000000pt}%
\definecolor{currentstroke}{rgb}{0.000000,0.000000,0.000000}%
\pgfsetstrokecolor{currentstroke}%
\pgfsetdash{}{0pt}%
\pgfpathmoveto{\pgfqpoint{2.131370in}{2.727805in}}%
\pgfpathlineto{\pgfqpoint{2.145752in}{2.700876in}}%
\pgfpathlineto{\pgfqpoint{2.160115in}{2.674298in}}%
\pgfpathlineto{\pgfqpoint{2.174460in}{2.648069in}}%
\pgfpathlineto{\pgfqpoint{2.188788in}{2.622185in}}%
\pgfpathlineto{\pgfqpoint{2.198056in}{2.616391in}}%
\pgfpathlineto{\pgfqpoint{2.207301in}{2.610962in}}%
\pgfpathlineto{\pgfqpoint{2.216521in}{2.605892in}}%
\pgfpathlineto{\pgfqpoint{2.225717in}{2.601174in}}%
\pgfpathlineto{\pgfqpoint{2.211453in}{2.626398in}}%
\pgfpathlineto{\pgfqpoint{2.197172in}{2.651964in}}%
\pgfpathlineto{\pgfqpoint{2.182874in}{2.677877in}}%
\pgfpathlineto{\pgfqpoint{2.168558in}{2.704140in}}%
\pgfpathlineto{\pgfqpoint{2.159298in}{2.709506in}}%
\pgfpathlineto{\pgfqpoint{2.150014in}{2.715235in}}%
\pgfpathlineto{\pgfqpoint{2.140705in}{2.721332in}}%
\pgfpathlineto{\pgfqpoint{2.131370in}{2.727805in}}%
\pgfpathclose%
\pgfusepath{fill}%
\end{pgfscope}%
\begin{pgfscope}%
\pgfpathrectangle{\pgfqpoint{1.150000in}{0.150000in}}{\pgfqpoint{5.700000in}{5.700000in}}%
\pgfusepath{clip}%
\pgfsetbuttcap%
\pgfsetroundjoin%
\definecolor{currentfill}{rgb}{0.239374,0.735588,0.455688}%
\pgfsetfillcolor{currentfill}%
\pgfsetfillopacity{0.800000}%
\pgfsetlinewidth{0.000000pt}%
\definecolor{currentstroke}{rgb}{0.000000,0.000000,0.000000}%
\pgfsetstrokecolor{currentstroke}%
\pgfsetdash{}{0pt}%
\pgfpathmoveto{\pgfqpoint{5.800631in}{3.467371in}}%
\pgfpathlineto{\pgfqpoint{5.815610in}{3.482627in}}%
\pgfpathlineto{\pgfqpoint{5.830612in}{3.498066in}}%
\pgfpathlineto{\pgfqpoint{5.845637in}{3.513688in}}%
\pgfpathlineto{\pgfqpoint{5.860685in}{3.529494in}}%
\pgfpathlineto{\pgfqpoint{5.867849in}{3.529946in}}%
\pgfpathlineto{\pgfqpoint{5.875006in}{3.530358in}}%
\pgfpathlineto{\pgfqpoint{5.882153in}{3.530737in}}%
\pgfpathlineto{\pgfqpoint{5.889293in}{3.531088in}}%
\pgfpathlineto{\pgfqpoint{5.874276in}{3.515826in}}%
\pgfpathlineto{\pgfqpoint{5.859281in}{3.500746in}}%
\pgfpathlineto{\pgfqpoint{5.844310in}{3.485848in}}%
\pgfpathlineto{\pgfqpoint{5.829361in}{3.471132in}}%
\pgfpathlineto{\pgfqpoint{5.822191in}{3.470227in}}%
\pgfpathlineto{\pgfqpoint{5.815012in}{3.469303in}}%
\pgfpathlineto{\pgfqpoint{5.807826in}{3.468352in}}%
\pgfpathlineto{\pgfqpoint{5.800631in}{3.467371in}}%
\pgfpathclose%
\pgfusepath{fill}%
\end{pgfscope}%
\begin{pgfscope}%
\pgfpathrectangle{\pgfqpoint{1.150000in}{0.150000in}}{\pgfqpoint{5.700000in}{5.700000in}}%
\pgfusepath{clip}%
\pgfsetbuttcap%
\pgfsetroundjoin%
\definecolor{currentfill}{rgb}{0.280894,0.078907,0.402329}%
\pgfsetfillcolor{currentfill}%
\pgfsetfillopacity{0.800000}%
\pgfsetlinewidth{0.000000pt}%
\definecolor{currentstroke}{rgb}{0.000000,0.000000,0.000000}%
\pgfsetstrokecolor{currentstroke}%
\pgfsetdash{}{0pt}%
\pgfpathmoveto{\pgfqpoint{2.883255in}{1.674998in}}%
\pgfpathlineto{\pgfqpoint{2.897138in}{1.663066in}}%
\pgfpathlineto{\pgfqpoint{2.911019in}{1.651361in}}%
\pgfpathlineto{\pgfqpoint{2.924896in}{1.639881in}}%
\pgfpathlineto{\pgfqpoint{2.938771in}{1.628624in}}%
\pgfpathlineto{\pgfqpoint{2.947407in}{1.630752in}}%
\pgfpathlineto{\pgfqpoint{2.956028in}{1.633143in}}%
\pgfpathlineto{\pgfqpoint{2.964636in}{1.635790in}}%
\pgfpathlineto{\pgfqpoint{2.973229in}{1.638687in}}%
\pgfpathlineto{\pgfqpoint{2.959391in}{1.649308in}}%
\pgfpathlineto{\pgfqpoint{2.945551in}{1.660152in}}%
\pgfpathlineto{\pgfqpoint{2.931709in}{1.671220in}}%
\pgfpathlineto{\pgfqpoint{2.917864in}{1.682514in}}%
\pgfpathlineto{\pgfqpoint{2.909234in}{1.680241in}}%
\pgfpathlineto{\pgfqpoint{2.900589in}{1.678226in}}%
\pgfpathlineto{\pgfqpoint{2.891930in}{1.676476in}}%
\pgfpathlineto{\pgfqpoint{2.883255in}{1.674998in}}%
\pgfpathclose%
\pgfusepath{fill}%
\end{pgfscope}%
\begin{pgfscope}%
\pgfpathrectangle{\pgfqpoint{1.150000in}{0.150000in}}{\pgfqpoint{5.700000in}{5.700000in}}%
\pgfusepath{clip}%
\pgfsetbuttcap%
\pgfsetroundjoin%
\definecolor{currentfill}{rgb}{0.278791,0.062145,0.386592}%
\pgfsetfillcolor{currentfill}%
\pgfsetfillopacity{0.800000}%
\pgfsetlinewidth{0.000000pt}%
\definecolor{currentstroke}{rgb}{0.000000,0.000000,0.000000}%
\pgfsetstrokecolor{currentstroke}%
\pgfsetdash{}{0pt}%
\pgfpathmoveto{\pgfqpoint{3.692331in}{1.592867in}}%
\pgfpathlineto{\pgfqpoint{3.706180in}{1.593226in}}%
\pgfpathlineto{\pgfqpoint{3.720037in}{1.593774in}}%
\pgfpathlineto{\pgfqpoint{3.733901in}{1.594511in}}%
\pgfpathlineto{\pgfqpoint{3.747774in}{1.595436in}}%
\pgfpathlineto{\pgfqpoint{3.755948in}{1.607217in}}%
\pgfpathlineto{\pgfqpoint{3.764117in}{1.619045in}}%
\pgfpathlineto{\pgfqpoint{3.772281in}{1.630916in}}%
\pgfpathlineto{\pgfqpoint{3.780439in}{1.642827in}}%
\pgfpathlineto{\pgfqpoint{3.766576in}{1.641464in}}%
\pgfpathlineto{\pgfqpoint{3.752721in}{1.640290in}}%
\pgfpathlineto{\pgfqpoint{3.738874in}{1.639305in}}%
\pgfpathlineto{\pgfqpoint{3.725034in}{1.638509in}}%
\pgfpathlineto{\pgfqpoint{3.716867in}{1.627023in}}%
\pgfpathlineto{\pgfqpoint{3.708694in}{1.615585in}}%
\pgfpathlineto{\pgfqpoint{3.700515in}{1.604199in}}%
\pgfpathlineto{\pgfqpoint{3.692331in}{1.592867in}}%
\pgfpathclose%
\pgfusepath{fill}%
\end{pgfscope}%
\begin{pgfscope}%
\pgfpathrectangle{\pgfqpoint{1.150000in}{0.150000in}}{\pgfqpoint{5.700000in}{5.700000in}}%
\pgfusepath{clip}%
\pgfsetbuttcap%
\pgfsetroundjoin%
\definecolor{currentfill}{rgb}{0.274952,0.037752,0.364543}%
\pgfsetfillcolor{currentfill}%
\pgfsetfillopacity{0.800000}%
\pgfsetlinewidth{0.000000pt}%
\definecolor{currentstroke}{rgb}{0.000000,0.000000,0.000000}%
\pgfsetstrokecolor{currentstroke}%
\pgfsetdash{}{0pt}%
\pgfpathmoveto{\pgfqpoint{3.604162in}{1.550516in}}%
\pgfpathlineto{\pgfqpoint{3.617995in}{1.549646in}}%
\pgfpathlineto{\pgfqpoint{3.631835in}{1.548967in}}%
\pgfpathlineto{\pgfqpoint{3.645682in}{1.548478in}}%
\pgfpathlineto{\pgfqpoint{3.659535in}{1.548179in}}%
\pgfpathlineto{\pgfqpoint{3.667743in}{1.559247in}}%
\pgfpathlineto{\pgfqpoint{3.675945in}{1.570387in}}%
\pgfpathlineto{\pgfqpoint{3.684141in}{1.581595in}}%
\pgfpathlineto{\pgfqpoint{3.692331in}{1.592867in}}%
\pgfpathlineto{\pgfqpoint{3.678489in}{1.592698in}}%
\pgfpathlineto{\pgfqpoint{3.664654in}{1.592718in}}%
\pgfpathlineto{\pgfqpoint{3.650826in}{1.592929in}}%
\pgfpathlineto{\pgfqpoint{3.637005in}{1.593331in}}%
\pgfpathlineto{\pgfqpoint{3.628804in}{1.582516in}}%
\pgfpathlineto{\pgfqpoint{3.620596in}{1.571772in}}%
\pgfpathlineto{\pgfqpoint{3.612382in}{1.561104in}}%
\pgfpathlineto{\pgfqpoint{3.604162in}{1.550516in}}%
\pgfpathclose%
\pgfusepath{fill}%
\end{pgfscope}%
\begin{pgfscope}%
\pgfpathrectangle{\pgfqpoint{1.150000in}{0.150000in}}{\pgfqpoint{5.700000in}{5.700000in}}%
\pgfusepath{clip}%
\pgfsetbuttcap%
\pgfsetroundjoin%
\definecolor{currentfill}{rgb}{0.218130,0.347432,0.550038}%
\pgfsetfillcolor{currentfill}%
\pgfsetfillopacity{0.800000}%
\pgfsetlinewidth{0.000000pt}%
\definecolor{currentstroke}{rgb}{0.000000,0.000000,0.000000}%
\pgfsetstrokecolor{currentstroke}%
\pgfsetdash{}{0pt}%
\pgfpathmoveto{\pgfqpoint{2.322930in}{2.355972in}}%
\pgfpathlineto{\pgfqpoint{2.337124in}{2.333599in}}%
\pgfpathlineto{\pgfqpoint{2.351305in}{2.311530in}}%
\pgfpathlineto{\pgfqpoint{2.365473in}{2.289762in}}%
\pgfpathlineto{\pgfqpoint{2.379628in}{2.268292in}}%
\pgfpathlineto{\pgfqpoint{2.388744in}{2.263770in}}%
\pgfpathlineto{\pgfqpoint{2.397838in}{2.259601in}}%
\pgfpathlineto{\pgfqpoint{2.406910in}{2.255779in}}%
\pgfpathlineto{\pgfqpoint{2.415960in}{2.252297in}}%
\pgfpathlineto{\pgfqpoint{2.401862in}{2.273093in}}%
\pgfpathlineto{\pgfqpoint{2.387753in}{2.294186in}}%
\pgfpathlineto{\pgfqpoint{2.373631in}{2.315578in}}%
\pgfpathlineto{\pgfqpoint{2.359497in}{2.337272in}}%
\pgfpathlineto{\pgfqpoint{2.350390in}{2.341416in}}%
\pgfpathlineto{\pgfqpoint{2.341260in}{2.345909in}}%
\pgfpathlineto{\pgfqpoint{2.332107in}{2.350759in}}%
\pgfpathlineto{\pgfqpoint{2.322930in}{2.355972in}}%
\pgfpathclose%
\pgfusepath{fill}%
\end{pgfscope}%
\begin{pgfscope}%
\pgfpathrectangle{\pgfqpoint{1.150000in}{0.150000in}}{\pgfqpoint{5.700000in}{5.700000in}}%
\pgfusepath{clip}%
\pgfsetbuttcap%
\pgfsetroundjoin%
\definecolor{currentfill}{rgb}{0.275191,0.194905,0.496005}%
\pgfsetfillcolor{currentfill}%
\pgfsetfillopacity{0.800000}%
\pgfsetlinewidth{0.000000pt}%
\definecolor{currentstroke}{rgb}{0.000000,0.000000,0.000000}%
\pgfsetstrokecolor{currentstroke}%
\pgfsetdash{}{0pt}%
\pgfpathmoveto{\pgfqpoint{4.077084in}{1.883967in}}%
\pgfpathlineto{\pgfqpoint{4.091054in}{1.889197in}}%
\pgfpathlineto{\pgfqpoint{4.105036in}{1.894611in}}%
\pgfpathlineto{\pgfqpoint{4.119030in}{1.900211in}}%
\pgfpathlineto{\pgfqpoint{4.133035in}{1.905995in}}%
\pgfpathlineto{\pgfqpoint{4.141093in}{1.919206in}}%
\pgfpathlineto{\pgfqpoint{4.149146in}{1.932366in}}%
\pgfpathlineto{\pgfqpoint{4.157194in}{1.945472in}}%
\pgfpathlineto{\pgfqpoint{4.165238in}{1.958522in}}%
\pgfpathlineto{\pgfqpoint{4.151235in}{1.952455in}}%
\pgfpathlineto{\pgfqpoint{4.137243in}{1.946573in}}%
\pgfpathlineto{\pgfqpoint{4.123264in}{1.940877in}}%
\pgfpathlineto{\pgfqpoint{4.109296in}{1.935366in}}%
\pgfpathlineto{\pgfqpoint{4.101250in}{1.922586in}}%
\pgfpathlineto{\pgfqpoint{4.093199in}{1.909758in}}%
\pgfpathlineto{\pgfqpoint{4.085144in}{1.896884in}}%
\pgfpathlineto{\pgfqpoint{4.077084in}{1.883967in}}%
\pgfpathclose%
\pgfusepath{fill}%
\end{pgfscope}%
\begin{pgfscope}%
\pgfpathrectangle{\pgfqpoint{1.150000in}{0.150000in}}{\pgfqpoint{5.700000in}{5.700000in}}%
\pgfusepath{clip}%
\pgfsetbuttcap%
\pgfsetroundjoin%
\definecolor{currentfill}{rgb}{0.246811,0.283237,0.535941}%
\pgfsetfillcolor{currentfill}%
\pgfsetfillopacity{0.800000}%
\pgfsetlinewidth{0.000000pt}%
\definecolor{currentstroke}{rgb}{0.000000,0.000000,0.000000}%
\pgfsetstrokecolor{currentstroke}%
\pgfsetdash{}{0pt}%
\pgfpathmoveto{\pgfqpoint{4.285544in}{2.088626in}}%
\pgfpathlineto{\pgfqpoint{4.299609in}{2.096086in}}%
\pgfpathlineto{\pgfqpoint{4.313686in}{2.103730in}}%
\pgfpathlineto{\pgfqpoint{4.327778in}{2.111560in}}%
\pgfpathlineto{\pgfqpoint{4.341883in}{2.119574in}}%
\pgfpathlineto{\pgfqpoint{4.349882in}{2.132419in}}%
\pgfpathlineto{\pgfqpoint{4.357875in}{2.145173in}}%
\pgfpathlineto{\pgfqpoint{4.365863in}{2.157833in}}%
\pgfpathlineto{\pgfqpoint{4.373845in}{2.170398in}}%
\pgfpathlineto{\pgfqpoint{4.359741in}{2.162198in}}%
\pgfpathlineto{\pgfqpoint{4.345650in}{2.154183in}}%
\pgfpathlineto{\pgfqpoint{4.331574in}{2.146353in}}%
\pgfpathlineto{\pgfqpoint{4.317510in}{2.138707in}}%
\pgfpathlineto{\pgfqpoint{4.309526in}{2.126315in}}%
\pgfpathlineto{\pgfqpoint{4.301537in}{2.113837in}}%
\pgfpathlineto{\pgfqpoint{4.293543in}{2.101273in}}%
\pgfpathlineto{\pgfqpoint{4.285544in}{2.088626in}}%
\pgfpathclose%
\pgfusepath{fill}%
\end{pgfscope}%
\begin{pgfscope}%
\pgfpathrectangle{\pgfqpoint{1.150000in}{0.150000in}}{\pgfqpoint{5.700000in}{5.700000in}}%
\pgfusepath{clip}%
\pgfsetbuttcap%
\pgfsetroundjoin%
\definecolor{currentfill}{rgb}{0.281446,0.084320,0.407414}%
\pgfsetfillcolor{currentfill}%
\pgfsetfillopacity{0.800000}%
\pgfsetlinewidth{0.000000pt}%
\definecolor{currentstroke}{rgb}{0.000000,0.000000,0.000000}%
\pgfsetstrokecolor{currentstroke}%
\pgfsetdash{}{0pt}%
\pgfpathmoveto{\pgfqpoint{3.780439in}{1.642827in}}%
\pgfpathlineto{\pgfqpoint{3.794311in}{1.644378in}}%
\pgfpathlineto{\pgfqpoint{3.808191in}{1.646117in}}%
\pgfpathlineto{\pgfqpoint{3.822080in}{1.648043in}}%
\pgfpathlineto{\pgfqpoint{3.835978in}{1.650155in}}%
\pgfpathlineto{\pgfqpoint{3.844123in}{1.662520in}}%
\pgfpathlineto{\pgfqpoint{3.852264in}{1.674908in}}%
\pgfpathlineto{\pgfqpoint{3.860399in}{1.687316in}}%
\pgfpathlineto{\pgfqpoint{3.868530in}{1.699739in}}%
\pgfpathlineto{\pgfqpoint{3.854639in}{1.697219in}}%
\pgfpathlineto{\pgfqpoint{3.840757in}{1.694887in}}%
\pgfpathlineto{\pgfqpoint{3.826885in}{1.692741in}}%
\pgfpathlineto{\pgfqpoint{3.813021in}{1.690784in}}%
\pgfpathlineto{\pgfqpoint{3.804883in}{1.678755in}}%
\pgfpathlineto{\pgfqpoint{3.796740in}{1.666750in}}%
\pgfpathlineto{\pgfqpoint{3.788593in}{1.654773in}}%
\pgfpathlineto{\pgfqpoint{3.780439in}{1.642827in}}%
\pgfpathclose%
\pgfusepath{fill}%
\end{pgfscope}%
\begin{pgfscope}%
\pgfpathrectangle{\pgfqpoint{1.150000in}{0.150000in}}{\pgfqpoint{5.700000in}{5.700000in}}%
\pgfusepath{clip}%
\pgfsetbuttcap%
\pgfsetroundjoin%
\definecolor{currentfill}{rgb}{0.267004,0.004874,0.329415}%
\pgfsetfillcolor{currentfill}%
\pgfsetfillopacity{0.800000}%
\pgfsetlinewidth{0.000000pt}%
\definecolor{currentstroke}{rgb}{0.000000,0.000000,0.000000}%
\pgfsetstrokecolor{currentstroke}%
\pgfsetdash{}{0pt}%
\pgfpathmoveto{\pgfqpoint{3.283523in}{1.497288in}}%
\pgfpathlineto{\pgfqpoint{3.297339in}{1.491695in}}%
\pgfpathlineto{\pgfqpoint{3.311158in}{1.486302in}}%
\pgfpathlineto{\pgfqpoint{3.324980in}{1.481110in}}%
\pgfpathlineto{\pgfqpoint{3.338804in}{1.476116in}}%
\pgfpathlineto{\pgfqpoint{3.347168in}{1.483634in}}%
\pgfpathlineto{\pgfqpoint{3.355522in}{1.491317in}}%
\pgfpathlineto{\pgfqpoint{3.363868in}{1.499159in}}%
\pgfpathlineto{\pgfqpoint{3.372205in}{1.507154in}}%
\pgfpathlineto{\pgfqpoint{3.358403in}{1.511585in}}%
\pgfpathlineto{\pgfqpoint{3.344603in}{1.516214in}}%
\pgfpathlineto{\pgfqpoint{3.330807in}{1.521043in}}%
\pgfpathlineto{\pgfqpoint{3.317015in}{1.526073in}}%
\pgfpathlineto{\pgfqpoint{3.308656in}{1.518629in}}%
\pgfpathlineto{\pgfqpoint{3.300287in}{1.511346in}}%
\pgfpathlineto{\pgfqpoint{3.291910in}{1.504231in}}%
\pgfpathlineto{\pgfqpoint{3.283523in}{1.497288in}}%
\pgfpathclose%
\pgfusepath{fill}%
\end{pgfscope}%
\begin{pgfscope}%
\pgfpathrectangle{\pgfqpoint{1.150000in}{0.150000in}}{\pgfqpoint{5.700000in}{5.700000in}}%
\pgfusepath{clip}%
\pgfsetbuttcap%
\pgfsetroundjoin%
\definecolor{currentfill}{rgb}{0.319809,0.770914,0.411152}%
\pgfsetfillcolor{currentfill}%
\pgfsetfillopacity{0.800000}%
\pgfsetlinewidth{0.000000pt}%
\definecolor{currentstroke}{rgb}{0.000000,0.000000,0.000000}%
\pgfsetstrokecolor{currentstroke}%
\pgfsetdash{}{0pt}%
\pgfpathmoveto{\pgfqpoint{5.977940in}{3.592904in}}%
\pgfpathlineto{\pgfqpoint{5.993041in}{3.608501in}}%
\pgfpathlineto{\pgfqpoint{6.008165in}{3.624280in}}%
\pgfpathlineto{\pgfqpoint{6.023313in}{3.640242in}}%
\pgfpathlineto{\pgfqpoint{6.030353in}{3.639498in}}%
\pgfpathlineto{\pgfqpoint{6.037386in}{3.638751in}}%
\pgfpathlineto{\pgfqpoint{6.044411in}{3.638006in}}%
\pgfpathlineto{\pgfqpoint{6.051429in}{3.637271in}}%
\pgfpathlineto{\pgfqpoint{6.036318in}{3.621923in}}%
\pgfpathlineto{\pgfqpoint{6.021230in}{3.606757in}}%
\pgfpathlineto{\pgfqpoint{6.006166in}{3.591773in}}%
\pgfpathlineto{\pgfqpoint{5.999120in}{3.592040in}}%
\pgfpathlineto{\pgfqpoint{5.992067in}{3.592322in}}%
\pgfpathlineto{\pgfqpoint{5.985008in}{3.592612in}}%
\pgfpathlineto{\pgfqpoint{5.977940in}{3.592904in}}%
\pgfpathclose%
\pgfusepath{fill}%
\end{pgfscope}%
\begin{pgfscope}%
\pgfpathrectangle{\pgfqpoint{1.150000in}{0.150000in}}{\pgfqpoint{5.700000in}{5.700000in}}%
\pgfusepath{clip}%
\pgfsetbuttcap%
\pgfsetroundjoin%
\definecolor{currentfill}{rgb}{0.269944,0.014625,0.341379}%
\pgfsetfillcolor{currentfill}%
\pgfsetfillopacity{0.800000}%
\pgfsetlinewidth{0.000000pt}%
\definecolor{currentstroke}{rgb}{0.000000,0.000000,0.000000}%
\pgfsetstrokecolor{currentstroke}%
\pgfsetdash{}{0pt}%
\pgfpathmoveto{\pgfqpoint{3.515887in}{1.516459in}}%
\pgfpathlineto{\pgfqpoint{3.529711in}{1.514322in}}%
\pgfpathlineto{\pgfqpoint{3.543540in}{1.512377in}}%
\pgfpathlineto{\pgfqpoint{3.557376in}{1.510625in}}%
\pgfpathlineto{\pgfqpoint{3.571217in}{1.509064in}}%
\pgfpathlineto{\pgfqpoint{3.579463in}{1.519283in}}%
\pgfpathlineto{\pgfqpoint{3.587702in}{1.529601in}}%
\pgfpathlineto{\pgfqpoint{3.595935in}{1.540014in}}%
\pgfpathlineto{\pgfqpoint{3.604162in}{1.550516in}}%
\pgfpathlineto{\pgfqpoint{3.590335in}{1.551578in}}%
\pgfpathlineto{\pgfqpoint{3.576514in}{1.552831in}}%
\pgfpathlineto{\pgfqpoint{3.562699in}{1.554276in}}%
\pgfpathlineto{\pgfqpoint{3.548890in}{1.555914in}}%
\pgfpathlineto{\pgfqpoint{3.540650in}{1.545899in}}%
\pgfpathlineto{\pgfqpoint{3.532402in}{1.535982in}}%
\pgfpathlineto{\pgfqpoint{3.524148in}{1.526167in}}%
\pgfpathlineto{\pgfqpoint{3.515887in}{1.516459in}}%
\pgfpathclose%
\pgfusepath{fill}%
\end{pgfscope}%
\begin{pgfscope}%
\pgfpathrectangle{\pgfqpoint{1.150000in}{0.150000in}}{\pgfqpoint{5.700000in}{5.700000in}}%
\pgfusepath{clip}%
\pgfsetbuttcap%
\pgfsetroundjoin%
\definecolor{currentfill}{rgb}{0.281477,0.755203,0.432552}%
\pgfsetfillcolor{currentfill}%
\pgfsetfillopacity{0.800000}%
\pgfsetlinewidth{0.000000pt}%
\definecolor{currentstroke}{rgb}{0.000000,0.000000,0.000000}%
\pgfsetstrokecolor{currentstroke}%
\pgfsetdash{}{0pt}%
\pgfpathmoveto{\pgfqpoint{5.889293in}{3.531088in}}%
\pgfpathlineto{\pgfqpoint{5.904333in}{3.546533in}}%
\pgfpathlineto{\pgfqpoint{5.919397in}{3.562160in}}%
\pgfpathlineto{\pgfqpoint{5.934484in}{3.577971in}}%
\pgfpathlineto{\pgfqpoint{5.949594in}{3.593965in}}%
\pgfpathlineto{\pgfqpoint{5.956693in}{3.593729in}}%
\pgfpathlineto{\pgfqpoint{5.963783in}{3.593469in}}%
\pgfpathlineto{\pgfqpoint{5.970866in}{3.593192in}}%
\pgfpathlineto{\pgfqpoint{5.977940in}{3.592904in}}%
\pgfpathlineto{\pgfqpoint{5.962864in}{3.577490in}}%
\pgfpathlineto{\pgfqpoint{5.947810in}{3.562258in}}%
\pgfpathlineto{\pgfqpoint{5.932780in}{3.547208in}}%
\pgfpathlineto{\pgfqpoint{5.917773in}{3.532339in}}%
\pgfpathlineto{\pgfqpoint{5.910664in}{3.532037in}}%
\pgfpathlineto{\pgfqpoint{5.903548in}{3.531732in}}%
\pgfpathlineto{\pgfqpoint{5.896424in}{3.531418in}}%
\pgfpathlineto{\pgfqpoint{5.889293in}{3.531088in}}%
\pgfpathclose%
\pgfusepath{fill}%
\end{pgfscope}%
\begin{pgfscope}%
\pgfpathrectangle{\pgfqpoint{1.150000in}{0.150000in}}{\pgfqpoint{5.700000in}{5.700000in}}%
\pgfusepath{clip}%
\pgfsetbuttcap%
\pgfsetroundjoin%
\definecolor{currentfill}{rgb}{0.128087,0.647749,0.523491}%
\pgfsetfillcolor{currentfill}%
\pgfsetfillopacity{0.800000}%
\pgfsetlinewidth{0.000000pt}%
\definecolor{currentstroke}{rgb}{0.000000,0.000000,0.000000}%
\pgfsetstrokecolor{currentstroke}%
\pgfsetdash{}{0pt}%
\pgfpathmoveto{\pgfqpoint{5.416257in}{3.178736in}}%
\pgfpathlineto{\pgfqpoint{5.430999in}{3.193234in}}%
\pgfpathlineto{\pgfqpoint{5.445763in}{3.207915in}}%
\pgfpathlineto{\pgfqpoint{5.460549in}{3.222781in}}%
\pgfpathlineto{\pgfqpoint{5.475355in}{3.237832in}}%
\pgfpathlineto{\pgfqpoint{5.482802in}{3.241612in}}%
\pgfpathlineto{\pgfqpoint{5.490239in}{3.245286in}}%
\pgfpathlineto{\pgfqpoint{5.497667in}{3.248860in}}%
\pgfpathlineto{\pgfqpoint{5.505086in}{3.252336in}}%
\pgfpathlineto{\pgfqpoint{5.490298in}{3.237651in}}%
\pgfpathlineto{\pgfqpoint{5.475531in}{3.223150in}}%
\pgfpathlineto{\pgfqpoint{5.460786in}{3.208832in}}%
\pgfpathlineto{\pgfqpoint{5.446061in}{3.194698in}}%
\pgfpathlineto{\pgfqpoint{5.438624in}{3.190845in}}%
\pgfpathlineto{\pgfqpoint{5.431177in}{3.186903in}}%
\pgfpathlineto{\pgfqpoint{5.423721in}{3.182868in}}%
\pgfpathlineto{\pgfqpoint{5.416257in}{3.178736in}}%
\pgfpathclose%
\pgfusepath{fill}%
\end{pgfscope}%
\begin{pgfscope}%
\pgfpathrectangle{\pgfqpoint{1.150000in}{0.150000in}}{\pgfqpoint{5.700000in}{5.700000in}}%
\pgfusepath{clip}%
\pgfsetbuttcap%
\pgfsetroundjoin%
\definecolor{currentfill}{rgb}{0.269944,0.014625,0.341379}%
\pgfsetfillcolor{currentfill}%
\pgfsetfillopacity{0.800000}%
\pgfsetlinewidth{0.000000pt}%
\definecolor{currentstroke}{rgb}{0.000000,0.000000,0.000000}%
\pgfsetstrokecolor{currentstroke}%
\pgfsetdash{}{0pt}%
\pgfpathmoveto{\pgfqpoint{3.139204in}{1.528202in}}%
\pgfpathlineto{\pgfqpoint{3.153034in}{1.520376in}}%
\pgfpathlineto{\pgfqpoint{3.166865in}{1.512757in}}%
\pgfpathlineto{\pgfqpoint{3.180697in}{1.505346in}}%
\pgfpathlineto{\pgfqpoint{3.194530in}{1.498140in}}%
\pgfpathlineto{\pgfqpoint{3.202984in}{1.503726in}}%
\pgfpathlineto{\pgfqpoint{3.211427in}{1.509516in}}%
\pgfpathlineto{\pgfqpoint{3.219859in}{1.515504in}}%
\pgfpathlineto{\pgfqpoint{3.228281in}{1.521684in}}%
\pgfpathlineto{\pgfqpoint{3.214476in}{1.528293in}}%
\pgfpathlineto{\pgfqpoint{3.200671in}{1.535108in}}%
\pgfpathlineto{\pgfqpoint{3.186869in}{1.542129in}}%
\pgfpathlineto{\pgfqpoint{3.173067in}{1.549358in}}%
\pgfpathlineto{\pgfqpoint{3.164618in}{1.543762in}}%
\pgfpathlineto{\pgfqpoint{3.156158in}{1.538367in}}%
\pgfpathlineto{\pgfqpoint{3.147686in}{1.533178in}}%
\pgfpathlineto{\pgfqpoint{3.139204in}{1.528202in}}%
\pgfpathclose%
\pgfusepath{fill}%
\end{pgfscope}%
\begin{pgfscope}%
\pgfpathrectangle{\pgfqpoint{1.150000in}{0.150000in}}{\pgfqpoint{5.700000in}{5.700000in}}%
\pgfusepath{clip}%
\pgfsetbuttcap%
\pgfsetroundjoin%
\definecolor{currentfill}{rgb}{0.127568,0.566949,0.550556}%
\pgfsetfillcolor{currentfill}%
\pgfsetfillopacity{0.800000}%
\pgfsetlinewidth{0.000000pt}%
\definecolor{currentstroke}{rgb}{0.000000,0.000000,0.000000}%
\pgfsetstrokecolor{currentstroke}%
\pgfsetdash{}{0pt}%
\pgfpathmoveto{\pgfqpoint{5.119588in}{2.923549in}}%
\pgfpathlineto{\pgfqpoint{5.134142in}{2.936937in}}%
\pgfpathlineto{\pgfqpoint{5.148714in}{2.950509in}}%
\pgfpathlineto{\pgfqpoint{5.163307in}{2.964266in}}%
\pgfpathlineto{\pgfqpoint{5.177919in}{2.978208in}}%
\pgfpathlineto{\pgfqpoint{5.185556in}{2.984805in}}%
\pgfpathlineto{\pgfqpoint{5.193183in}{2.991268in}}%
\pgfpathlineto{\pgfqpoint{5.200802in}{2.997600in}}%
\pgfpathlineto{\pgfqpoint{5.208412in}{3.003804in}}%
\pgfpathlineto{\pgfqpoint{5.193811in}{2.990085in}}%
\pgfpathlineto{\pgfqpoint{5.179230in}{2.976551in}}%
\pgfpathlineto{\pgfqpoint{5.164668in}{2.963201in}}%
\pgfpathlineto{\pgfqpoint{5.150125in}{2.950035in}}%
\pgfpathlineto{\pgfqpoint{5.142504in}{2.943596in}}%
\pgfpathlineto{\pgfqpoint{5.134874in}{2.937037in}}%
\pgfpathlineto{\pgfqpoint{5.127235in}{2.930356in}}%
\pgfpathlineto{\pgfqpoint{5.119588in}{2.923549in}}%
\pgfpathclose%
\pgfusepath{fill}%
\end{pgfscope}%
\begin{pgfscope}%
\pgfpathrectangle{\pgfqpoint{1.150000in}{0.150000in}}{\pgfqpoint{5.700000in}{5.700000in}}%
\pgfusepath{clip}%
\pgfsetbuttcap%
\pgfsetroundjoin%
\definecolor{currentfill}{rgb}{0.210503,0.363727,0.552206}%
\pgfsetfillcolor{currentfill}%
\pgfsetfillopacity{0.800000}%
\pgfsetlinewidth{0.000000pt}%
\definecolor{currentstroke}{rgb}{0.000000,0.000000,0.000000}%
\pgfsetstrokecolor{currentstroke}%
\pgfsetdash{}{0pt}%
\pgfpathmoveto{\pgfqpoint{4.494071in}{2.303106in}}%
\pgfpathlineto{\pgfqpoint{4.508246in}{2.312503in}}%
\pgfpathlineto{\pgfqpoint{4.522437in}{2.322085in}}%
\pgfpathlineto{\pgfqpoint{4.536644in}{2.331851in}}%
\pgfpathlineto{\pgfqpoint{4.550865in}{2.341802in}}%
\pgfpathlineto{\pgfqpoint{4.558797in}{2.353663in}}%
\pgfpathlineto{\pgfqpoint{4.566723in}{2.365404in}}%
\pgfpathlineto{\pgfqpoint{4.574643in}{2.377024in}}%
\pgfpathlineto{\pgfqpoint{4.582557in}{2.388522in}}%
\pgfpathlineto{\pgfqpoint{4.568336in}{2.378484in}}%
\pgfpathlineto{\pgfqpoint{4.554131in}{2.368631in}}%
\pgfpathlineto{\pgfqpoint{4.539942in}{2.358962in}}%
\pgfpathlineto{\pgfqpoint{4.525768in}{2.349478in}}%
\pgfpathlineto{\pgfqpoint{4.517852in}{2.338055in}}%
\pgfpathlineto{\pgfqpoint{4.509931in}{2.326518in}}%
\pgfpathlineto{\pgfqpoint{4.502004in}{2.314868in}}%
\pgfpathlineto{\pgfqpoint{4.494071in}{2.303106in}}%
\pgfpathclose%
\pgfusepath{fill}%
\end{pgfscope}%
\begin{pgfscope}%
\pgfpathrectangle{\pgfqpoint{1.150000in}{0.150000in}}{\pgfqpoint{5.700000in}{5.700000in}}%
\pgfusepath{clip}%
\pgfsetbuttcap%
\pgfsetroundjoin%
\definecolor{currentfill}{rgb}{0.283197,0.115680,0.436115}%
\pgfsetfillcolor{currentfill}%
\pgfsetfillopacity{0.800000}%
\pgfsetlinewidth{0.000000pt}%
\definecolor{currentstroke}{rgb}{0.000000,0.000000,0.000000}%
\pgfsetstrokecolor{currentstroke}%
\pgfsetdash{}{0pt}%
\pgfpathmoveto{\pgfqpoint{3.868530in}{1.699739in}}%
\pgfpathlineto{\pgfqpoint{3.882430in}{1.702446in}}%
\pgfpathlineto{\pgfqpoint{3.896339in}{1.705340in}}%
\pgfpathlineto{\pgfqpoint{3.910258in}{1.708420in}}%
\pgfpathlineto{\pgfqpoint{3.924187in}{1.711685in}}%
\pgfpathlineto{\pgfqpoint{3.932307in}{1.724509in}}%
\pgfpathlineto{\pgfqpoint{3.940422in}{1.737334in}}%
\pgfpathlineto{\pgfqpoint{3.948532in}{1.750156in}}%
\pgfpathlineto{\pgfqpoint{3.956638in}{1.762972in}}%
\pgfpathlineto{\pgfqpoint{3.942714in}{1.759330in}}%
\pgfpathlineto{\pgfqpoint{3.928801in}{1.755874in}}%
\pgfpathlineto{\pgfqpoint{3.914897in}{1.752605in}}%
\pgfpathlineto{\pgfqpoint{3.901003in}{1.749522in}}%
\pgfpathlineto{\pgfqpoint{3.892892in}{1.737070in}}%
\pgfpathlineto{\pgfqpoint{3.884776in}{1.724620in}}%
\pgfpathlineto{\pgfqpoint{3.876655in}{1.712175in}}%
\pgfpathlineto{\pgfqpoint{3.868530in}{1.699739in}}%
\pgfpathclose%
\pgfusepath{fill}%
\end{pgfscope}%
\begin{pgfscope}%
\pgfpathrectangle{\pgfqpoint{1.150000in}{0.150000in}}{\pgfqpoint{5.700000in}{5.700000in}}%
\pgfusepath{clip}%
\pgfsetbuttcap%
\pgfsetroundjoin%
\definecolor{currentfill}{rgb}{0.278791,0.062145,0.386592}%
\pgfsetfillcolor{currentfill}%
\pgfsetfillopacity{0.800000}%
\pgfsetlinewidth{0.000000pt}%
\definecolor{currentstroke}{rgb}{0.000000,0.000000,0.000000}%
\pgfsetstrokecolor{currentstroke}%
\pgfsetdash{}{0pt}%
\pgfpathmoveto{\pgfqpoint{2.938771in}{1.628624in}}%
\pgfpathlineto{\pgfqpoint{2.952644in}{1.617589in}}%
\pgfpathlineto{\pgfqpoint{2.966515in}{1.606775in}}%
\pgfpathlineto{\pgfqpoint{2.980384in}{1.596181in}}%
\pgfpathlineto{\pgfqpoint{2.994251in}{1.585806in}}%
\pgfpathlineto{\pgfqpoint{3.002850in}{1.588582in}}%
\pgfpathlineto{\pgfqpoint{3.011435in}{1.591612in}}%
\pgfpathlineto{\pgfqpoint{3.020007in}{1.594890in}}%
\pgfpathlineto{\pgfqpoint{3.028565in}{1.598409in}}%
\pgfpathlineto{\pgfqpoint{3.014733in}{1.608150in}}%
\pgfpathlineto{\pgfqpoint{3.000900in}{1.618110in}}%
\pgfpathlineto{\pgfqpoint{2.987065in}{1.628288in}}%
\pgfpathlineto{\pgfqpoint{2.973229in}{1.638687in}}%
\pgfpathlineto{\pgfqpoint{2.964636in}{1.635790in}}%
\pgfpathlineto{\pgfqpoint{2.956028in}{1.633143in}}%
\pgfpathlineto{\pgfqpoint{2.947407in}{1.630752in}}%
\pgfpathlineto{\pgfqpoint{2.938771in}{1.628624in}}%
\pgfpathclose%
\pgfusepath{fill}%
\end{pgfscope}%
\begin{pgfscope}%
\pgfpathrectangle{\pgfqpoint{1.150000in}{0.150000in}}{\pgfqpoint{5.700000in}{5.700000in}}%
\pgfusepath{clip}%
\pgfsetbuttcap%
\pgfsetroundjoin%
\definecolor{currentfill}{rgb}{0.177423,0.437527,0.557565}%
\pgfsetfillcolor{currentfill}%
\pgfsetfillopacity{0.800000}%
\pgfsetlinewidth{0.000000pt}%
\definecolor{currentstroke}{rgb}{0.000000,0.000000,0.000000}%
\pgfsetstrokecolor{currentstroke}%
\pgfsetdash{}{0pt}%
\pgfpathmoveto{\pgfqpoint{4.702672in}{2.518400in}}%
\pgfpathlineto{\pgfqpoint{4.716970in}{2.529435in}}%
\pgfpathlineto{\pgfqpoint{4.731285in}{2.540655in}}%
\pgfpathlineto{\pgfqpoint{4.745617in}{2.552060in}}%
\pgfpathlineto{\pgfqpoint{4.759966in}{2.563649in}}%
\pgfpathlineto{\pgfqpoint{4.767818in}{2.574046in}}%
\pgfpathlineto{\pgfqpoint{4.775662in}{2.584306in}}%
\pgfpathlineto{\pgfqpoint{4.783500in}{2.594428in}}%
\pgfpathlineto{\pgfqpoint{4.791331in}{2.604414in}}%
\pgfpathlineto{\pgfqpoint{4.776985in}{2.592839in}}%
\pgfpathlineto{\pgfqpoint{4.762656in}{2.581449in}}%
\pgfpathlineto{\pgfqpoint{4.748345in}{2.570243in}}%
\pgfpathlineto{\pgfqpoint{4.734050in}{2.559222in}}%
\pgfpathlineto{\pgfqpoint{4.726216in}{2.549209in}}%
\pgfpathlineto{\pgfqpoint{4.718375in}{2.539068in}}%
\pgfpathlineto{\pgfqpoint{4.710527in}{2.528799in}}%
\pgfpathlineto{\pgfqpoint{4.702672in}{2.518400in}}%
\pgfpathclose%
\pgfusepath{fill}%
\end{pgfscope}%
\begin{pgfscope}%
\pgfpathrectangle{\pgfqpoint{1.150000in}{0.150000in}}{\pgfqpoint{5.700000in}{5.700000in}}%
\pgfusepath{clip}%
\pgfsetbuttcap%
\pgfsetroundjoin%
\definecolor{currentfill}{rgb}{0.203063,0.379716,0.553925}%
\pgfsetfillcolor{currentfill}%
\pgfsetfillopacity{0.800000}%
\pgfsetlinewidth{0.000000pt}%
\definecolor{currentstroke}{rgb}{0.000000,0.000000,0.000000}%
\pgfsetstrokecolor{currentstroke}%
\pgfsetdash{}{0pt}%
\pgfpathmoveto{\pgfqpoint{2.266018in}{2.448550in}}%
\pgfpathlineto{\pgfqpoint{2.280267in}{2.424936in}}%
\pgfpathlineto{\pgfqpoint{2.294502in}{2.401637in}}%
\pgfpathlineto{\pgfqpoint{2.308723in}{2.378650in}}%
\pgfpathlineto{\pgfqpoint{2.322930in}{2.355972in}}%
\pgfpathlineto{\pgfqpoint{2.332107in}{2.350759in}}%
\pgfpathlineto{\pgfqpoint{2.341260in}{2.345909in}}%
\pgfpathlineto{\pgfqpoint{2.350390in}{2.341416in}}%
\pgfpathlineto{\pgfqpoint{2.359497in}{2.337272in}}%
\pgfpathlineto{\pgfqpoint{2.345350in}{2.359270in}}%
\pgfpathlineto{\pgfqpoint{2.331190in}{2.381576in}}%
\pgfpathlineto{\pgfqpoint{2.317016in}{2.404192in}}%
\pgfpathlineto{\pgfqpoint{2.302828in}{2.427121in}}%
\pgfpathlineto{\pgfqpoint{2.293661in}{2.431933in}}%
\pgfpathlineto{\pgfqpoint{2.284471in}{2.437104in}}%
\pgfpathlineto{\pgfqpoint{2.275256in}{2.442640in}}%
\pgfpathlineto{\pgfqpoint{2.266018in}{2.448550in}}%
\pgfpathclose%
\pgfusepath{fill}%
\end{pgfscope}%
\begin{pgfscope}%
\pgfpathrectangle{\pgfqpoint{1.150000in}{0.150000in}}{\pgfqpoint{5.700000in}{5.700000in}}%
\pgfusepath{clip}%
\pgfsetbuttcap%
\pgfsetroundjoin%
\definecolor{currentfill}{rgb}{0.149039,0.508051,0.557250}%
\pgfsetfillcolor{currentfill}%
\pgfsetfillopacity{0.800000}%
\pgfsetlinewidth{0.000000pt}%
\definecolor{currentstroke}{rgb}{0.000000,0.000000,0.000000}%
\pgfsetstrokecolor{currentstroke}%
\pgfsetdash{}{0pt}%
\pgfpathmoveto{\pgfqpoint{4.911241in}{2.727121in}}%
\pgfpathlineto{\pgfqpoint{4.925667in}{2.739489in}}%
\pgfpathlineto{\pgfqpoint{4.940110in}{2.752041in}}%
\pgfpathlineto{\pgfqpoint{4.954573in}{2.764778in}}%
\pgfpathlineto{\pgfqpoint{4.969054in}{2.777700in}}%
\pgfpathlineto{\pgfqpoint{4.976808in}{2.786294in}}%
\pgfpathlineto{\pgfqpoint{4.984553in}{2.794745in}}%
\pgfpathlineto{\pgfqpoint{4.992291in}{2.803055in}}%
\pgfpathlineto{\pgfqpoint{5.000021in}{2.811227in}}%
\pgfpathlineto{\pgfqpoint{4.985546in}{2.798423in}}%
\pgfpathlineto{\pgfqpoint{4.971091in}{2.785804in}}%
\pgfpathlineto{\pgfqpoint{4.956653in}{2.773369in}}%
\pgfpathlineto{\pgfqpoint{4.942234in}{2.761118in}}%
\pgfpathlineto{\pgfqpoint{4.934497in}{2.752816in}}%
\pgfpathlineto{\pgfqpoint{4.926753in}{2.744384in}}%
\pgfpathlineto{\pgfqpoint{4.919001in}{2.735819in}}%
\pgfpathlineto{\pgfqpoint{4.911241in}{2.727121in}}%
\pgfpathclose%
\pgfusepath{fill}%
\end{pgfscope}%
\begin{pgfscope}%
\pgfpathrectangle{\pgfqpoint{1.150000in}{0.150000in}}{\pgfqpoint{5.700000in}{5.700000in}}%
\pgfusepath{clip}%
\pgfsetbuttcap%
\pgfsetroundjoin%
\definecolor{currentfill}{rgb}{0.267004,0.004874,0.329415}%
\pgfsetfillcolor{currentfill}%
\pgfsetfillopacity{0.800000}%
\pgfsetlinewidth{0.000000pt}%
\definecolor{currentstroke}{rgb}{0.000000,0.000000,0.000000}%
\pgfsetstrokecolor{currentstroke}%
\pgfsetdash{}{0pt}%
\pgfpathmoveto{\pgfqpoint{3.427453in}{1.491410in}}%
\pgfpathlineto{\pgfqpoint{3.441275in}{1.487965in}}%
\pgfpathlineto{\pgfqpoint{3.455102in}{1.484715in}}%
\pgfpathlineto{\pgfqpoint{3.468933in}{1.481660in}}%
\pgfpathlineto{\pgfqpoint{3.482769in}{1.478798in}}%
\pgfpathlineto{\pgfqpoint{3.491059in}{1.488028in}}%
\pgfpathlineto{\pgfqpoint{3.499342in}{1.497385in}}%
\pgfpathlineto{\pgfqpoint{3.507618in}{1.506864in}}%
\pgfpathlineto{\pgfqpoint{3.515887in}{1.516459in}}%
\pgfpathlineto{\pgfqpoint{3.502068in}{1.518790in}}%
\pgfpathlineto{\pgfqpoint{3.488254in}{1.521315in}}%
\pgfpathlineto{\pgfqpoint{3.474445in}{1.524034in}}%
\pgfpathlineto{\pgfqpoint{3.460641in}{1.526948in}}%
\pgfpathlineto{\pgfqpoint{3.452356in}{1.517871in}}%
\pgfpathlineto{\pgfqpoint{3.444063in}{1.508919in}}%
\pgfpathlineto{\pgfqpoint{3.435762in}{1.500097in}}%
\pgfpathlineto{\pgfqpoint{3.427453in}{1.491410in}}%
\pgfpathclose%
\pgfusepath{fill}%
\end{pgfscope}%
\begin{pgfscope}%
\pgfpathrectangle{\pgfqpoint{1.150000in}{0.150000in}}{\pgfqpoint{5.700000in}{5.700000in}}%
\pgfusepath{clip}%
\pgfsetbuttcap%
\pgfsetroundjoin%
\definecolor{currentfill}{rgb}{0.265145,0.232956,0.516599}%
\pgfsetfillcolor{currentfill}%
\pgfsetfillopacity{0.800000}%
\pgfsetlinewidth{0.000000pt}%
\definecolor{currentstroke}{rgb}{0.000000,0.000000,0.000000}%
\pgfsetstrokecolor{currentstroke}%
\pgfsetdash{}{0pt}%
\pgfpathmoveto{\pgfqpoint{4.165238in}{1.958522in}}%
\pgfpathlineto{\pgfqpoint{4.179254in}{1.964774in}}%
\pgfpathlineto{\pgfqpoint{4.193282in}{1.971210in}}%
\pgfpathlineto{\pgfqpoint{4.207323in}{1.977832in}}%
\pgfpathlineto{\pgfqpoint{4.221376in}{1.984638in}}%
\pgfpathlineto{\pgfqpoint{4.229414in}{1.997892in}}%
\pgfpathlineto{\pgfqpoint{4.237447in}{2.011077in}}%
\pgfpathlineto{\pgfqpoint{4.245475in}{2.024191in}}%
\pgfpathlineto{\pgfqpoint{4.253499in}{2.037232in}}%
\pgfpathlineto{\pgfqpoint{4.239446in}{2.030175in}}%
\pgfpathlineto{\pgfqpoint{4.225407in}{2.023303in}}%
\pgfpathlineto{\pgfqpoint{4.211380in}{2.016616in}}%
\pgfpathlineto{\pgfqpoint{4.197366in}{2.010114in}}%
\pgfpathlineto{\pgfqpoint{4.189341in}{1.997311in}}%
\pgfpathlineto{\pgfqpoint{4.181312in}{1.984443in}}%
\pgfpathlineto{\pgfqpoint{4.173277in}{1.971513in}}%
\pgfpathlineto{\pgfqpoint{4.165238in}{1.958522in}}%
\pgfpathclose%
\pgfusepath{fill}%
\end{pgfscope}%
\begin{pgfscope}%
\pgfpathrectangle{\pgfqpoint{1.150000in}{0.150000in}}{\pgfqpoint{5.700000in}{5.700000in}}%
\pgfusepath{clip}%
\pgfsetbuttcap%
\pgfsetroundjoin%
\definecolor{currentfill}{rgb}{0.149039,0.508051,0.557250}%
\pgfsetfillcolor{currentfill}%
\pgfsetfillopacity{0.800000}%
\pgfsetlinewidth{0.000000pt}%
\definecolor{currentstroke}{rgb}{0.000000,0.000000,0.000000}%
\pgfsetstrokecolor{currentstroke}%
\pgfsetdash{}{0pt}%
\pgfpathmoveto{\pgfqpoint{2.073655in}{2.839113in}}%
\pgfpathlineto{\pgfqpoint{2.088113in}{2.810740in}}%
\pgfpathlineto{\pgfqpoint{2.102552in}{2.782733in}}%
\pgfpathlineto{\pgfqpoint{2.116970in}{2.755090in}}%
\pgfpathlineto{\pgfqpoint{2.131370in}{2.727805in}}%
\pgfpathlineto{\pgfqpoint{2.140705in}{2.721332in}}%
\pgfpathlineto{\pgfqpoint{2.150014in}{2.715235in}}%
\pgfpathlineto{\pgfqpoint{2.159298in}{2.709506in}}%
\pgfpathlineto{\pgfqpoint{2.168558in}{2.704140in}}%
\pgfpathlineto{\pgfqpoint{2.154224in}{2.730757in}}%
\pgfpathlineto{\pgfqpoint{2.139872in}{2.757731in}}%
\pgfpathlineto{\pgfqpoint{2.125501in}{2.785065in}}%
\pgfpathlineto{\pgfqpoint{2.111111in}{2.812764in}}%
\pgfpathlineto{\pgfqpoint{2.101786in}{2.818786in}}%
\pgfpathlineto{\pgfqpoint{2.092435in}{2.825181in}}%
\pgfpathlineto{\pgfqpoint{2.083059in}{2.831954in}}%
\pgfpathlineto{\pgfqpoint{2.073655in}{2.839113in}}%
\pgfpathclose%
\pgfusepath{fill}%
\end{pgfscope}%
\begin{pgfscope}%
\pgfpathrectangle{\pgfqpoint{1.150000in}{0.150000in}}{\pgfqpoint{5.700000in}{5.700000in}}%
\pgfusepath{clip}%
\pgfsetbuttcap%
\pgfsetroundjoin%
\definecolor{currentfill}{rgb}{0.281887,0.150881,0.465405}%
\pgfsetfillcolor{currentfill}%
\pgfsetfillopacity{0.800000}%
\pgfsetlinewidth{0.000000pt}%
\definecolor{currentstroke}{rgb}{0.000000,0.000000,0.000000}%
\pgfsetstrokecolor{currentstroke}%
\pgfsetdash{}{0pt}%
\pgfpathmoveto{\pgfqpoint{3.956638in}{1.762972in}}%
\pgfpathlineto{\pgfqpoint{3.970572in}{1.766800in}}%
\pgfpathlineto{\pgfqpoint{3.984516in}{1.770814in}}%
\pgfpathlineto{\pgfqpoint{3.998471in}{1.775014in}}%
\pgfpathlineto{\pgfqpoint{4.012437in}{1.779398in}}%
\pgfpathlineto{\pgfqpoint{4.020534in}{1.792562in}}%
\pgfpathlineto{\pgfqpoint{4.028626in}{1.805706in}}%
\pgfpathlineto{\pgfqpoint{4.036714in}{1.818826in}}%
\pgfpathlineto{\pgfqpoint{4.044797in}{1.831919in}}%
\pgfpathlineto{\pgfqpoint{4.030835in}{1.827189in}}%
\pgfpathlineto{\pgfqpoint{4.016884in}{1.822644in}}%
\pgfpathlineto{\pgfqpoint{4.002943in}{1.818285in}}%
\pgfpathlineto{\pgfqpoint{3.989014in}{1.814113in}}%
\pgfpathlineto{\pgfqpoint{3.980927in}{1.801352in}}%
\pgfpathlineto{\pgfqpoint{3.972835in}{1.788574in}}%
\pgfpathlineto{\pgfqpoint{3.964739in}{1.775779in}}%
\pgfpathlineto{\pgfqpoint{3.956638in}{1.762972in}}%
\pgfpathclose%
\pgfusepath{fill}%
\end{pgfscope}%
\begin{pgfscope}%
\pgfpathrectangle{\pgfqpoint{1.150000in}{0.150000in}}{\pgfqpoint{5.700000in}{5.700000in}}%
\pgfusepath{clip}%
\pgfsetbuttcap%
\pgfsetroundjoin%
\definecolor{currentfill}{rgb}{0.150148,0.676631,0.506589}%
\pgfsetfillcolor{currentfill}%
\pgfsetfillopacity{0.800000}%
\pgfsetlinewidth{0.000000pt}%
\definecolor{currentstroke}{rgb}{0.000000,0.000000,0.000000}%
\pgfsetstrokecolor{currentstroke}%
\pgfsetdash{}{0pt}%
\pgfpathmoveto{\pgfqpoint{5.505086in}{3.252336in}}%
\pgfpathlineto{\pgfqpoint{5.519896in}{3.267205in}}%
\pgfpathlineto{\pgfqpoint{5.534727in}{3.282258in}}%
\pgfpathlineto{\pgfqpoint{5.549580in}{3.297496in}}%
\pgfpathlineto{\pgfqpoint{5.564455in}{3.312919in}}%
\pgfpathlineto{\pgfqpoint{5.571844in}{3.315914in}}%
\pgfpathlineto{\pgfqpoint{5.579224in}{3.318812in}}%
\pgfpathlineto{\pgfqpoint{5.586594in}{3.321618in}}%
\pgfpathlineto{\pgfqpoint{5.593955in}{3.324337in}}%
\pgfpathlineto{\pgfqpoint{5.579101in}{3.309316in}}%
\pgfpathlineto{\pgfqpoint{5.564269in}{3.294480in}}%
\pgfpathlineto{\pgfqpoint{5.549459in}{3.279827in}}%
\pgfpathlineto{\pgfqpoint{5.534670in}{3.265357in}}%
\pgfpathlineto{\pgfqpoint{5.527287in}{3.262226in}}%
\pgfpathlineto{\pgfqpoint{5.519896in}{3.259015in}}%
\pgfpathlineto{\pgfqpoint{5.512495in}{3.255720in}}%
\pgfpathlineto{\pgfqpoint{5.505086in}{3.252336in}}%
\pgfpathclose%
\pgfusepath{fill}%
\end{pgfscope}%
\begin{pgfscope}%
\pgfpathrectangle{\pgfqpoint{1.150000in}{0.150000in}}{\pgfqpoint{5.700000in}{5.700000in}}%
\pgfusepath{clip}%
\pgfsetbuttcap%
\pgfsetroundjoin%
\definecolor{currentfill}{rgb}{0.231674,0.318106,0.544834}%
\pgfsetfillcolor{currentfill}%
\pgfsetfillopacity{0.800000}%
\pgfsetlinewidth{0.000000pt}%
\definecolor{currentstroke}{rgb}{0.000000,0.000000,0.000000}%
\pgfsetstrokecolor{currentstroke}%
\pgfsetdash{}{0pt}%
\pgfpathmoveto{\pgfqpoint{4.373845in}{2.170398in}}%
\pgfpathlineto{\pgfqpoint{4.387964in}{2.178783in}}%
\pgfpathlineto{\pgfqpoint{4.402097in}{2.187352in}}%
\pgfpathlineto{\pgfqpoint{4.416245in}{2.196106in}}%
\pgfpathlineto{\pgfqpoint{4.430407in}{2.205044in}}%
\pgfpathlineto{\pgfqpoint{4.438384in}{2.217679in}}%
\pgfpathlineto{\pgfqpoint{4.446356in}{2.230208in}}%
\pgfpathlineto{\pgfqpoint{4.454322in}{2.242631in}}%
\pgfpathlineto{\pgfqpoint{4.462283in}{2.254945in}}%
\pgfpathlineto{\pgfqpoint{4.448121in}{2.245853in}}%
\pgfpathlineto{\pgfqpoint{4.433974in}{2.236946in}}%
\pgfpathlineto{\pgfqpoint{4.419842in}{2.228223in}}%
\pgfpathlineto{\pgfqpoint{4.405724in}{2.219686in}}%
\pgfpathlineto{\pgfqpoint{4.397763in}{2.207512in}}%
\pgfpathlineto{\pgfqpoint{4.389795in}{2.195239in}}%
\pgfpathlineto{\pgfqpoint{4.381823in}{2.182867in}}%
\pgfpathlineto{\pgfqpoint{4.373845in}{2.170398in}}%
\pgfpathclose%
\pgfusepath{fill}%
\end{pgfscope}%
\begin{pgfscope}%
\pgfpathrectangle{\pgfqpoint{1.150000in}{0.150000in}}{\pgfqpoint{5.700000in}{5.700000in}}%
\pgfusepath{clip}%
\pgfsetbuttcap%
\pgfsetroundjoin%
\definecolor{currentfill}{rgb}{0.276022,0.044167,0.370164}%
\pgfsetfillcolor{currentfill}%
\pgfsetfillopacity{0.800000}%
\pgfsetlinewidth{0.000000pt}%
\definecolor{currentstroke}{rgb}{0.000000,0.000000,0.000000}%
\pgfsetstrokecolor{currentstroke}%
\pgfsetdash{}{0pt}%
\pgfpathmoveto{\pgfqpoint{2.994251in}{1.585806in}}%
\pgfpathlineto{\pgfqpoint{3.008117in}{1.575648in}}%
\pgfpathlineto{\pgfqpoint{3.021982in}{1.565707in}}%
\pgfpathlineto{\pgfqpoint{3.035845in}{1.555981in}}%
\pgfpathlineto{\pgfqpoint{3.049708in}{1.546469in}}%
\pgfpathlineto{\pgfqpoint{3.058272in}{1.549891in}}%
\pgfpathlineto{\pgfqpoint{3.066823in}{1.553558in}}%
\pgfpathlineto{\pgfqpoint{3.075361in}{1.557464in}}%
\pgfpathlineto{\pgfqpoint{3.083886in}{1.561604in}}%
\pgfpathlineto{\pgfqpoint{3.070057in}{1.570483in}}%
\pgfpathlineto{\pgfqpoint{3.056227in}{1.579577in}}%
\pgfpathlineto{\pgfqpoint{3.042397in}{1.588885in}}%
\pgfpathlineto{\pgfqpoint{3.028565in}{1.598409in}}%
\pgfpathlineto{\pgfqpoint{3.020007in}{1.594890in}}%
\pgfpathlineto{\pgfqpoint{3.011435in}{1.591612in}}%
\pgfpathlineto{\pgfqpoint{3.002850in}{1.588582in}}%
\pgfpathlineto{\pgfqpoint{2.994251in}{1.585806in}}%
\pgfpathclose%
\pgfusepath{fill}%
\end{pgfscope}%
\begin{pgfscope}%
\pgfpathrectangle{\pgfqpoint{1.150000in}{0.150000in}}{\pgfqpoint{5.700000in}{5.700000in}}%
\pgfusepath{clip}%
\pgfsetbuttcap%
\pgfsetroundjoin%
\definecolor{currentfill}{rgb}{0.120565,0.596422,0.543611}%
\pgfsetfillcolor{currentfill}%
\pgfsetfillopacity{0.800000}%
\pgfsetlinewidth{0.000000pt}%
\definecolor{currentstroke}{rgb}{0.000000,0.000000,0.000000}%
\pgfsetstrokecolor{currentstroke}%
\pgfsetdash{}{0pt}%
\pgfpathmoveto{\pgfqpoint{5.208412in}{3.003804in}}%
\pgfpathlineto{\pgfqpoint{5.223034in}{3.017707in}}%
\pgfpathlineto{\pgfqpoint{5.237675in}{3.031795in}}%
\pgfpathlineto{\pgfqpoint{5.252337in}{3.046068in}}%
\pgfpathlineto{\pgfqpoint{5.267020in}{3.060526in}}%
\pgfpathlineto{\pgfqpoint{5.274609in}{3.066358in}}%
\pgfpathlineto{\pgfqpoint{5.282189in}{3.072060in}}%
\pgfpathlineto{\pgfqpoint{5.289760in}{3.077633in}}%
\pgfpathlineto{\pgfqpoint{5.297322in}{3.083082in}}%
\pgfpathlineto{\pgfqpoint{5.282653in}{3.068883in}}%
\pgfpathlineto{\pgfqpoint{5.268003in}{3.054869in}}%
\pgfpathlineto{\pgfqpoint{5.253374in}{3.041040in}}%
\pgfpathlineto{\pgfqpoint{5.238766in}{3.027394in}}%
\pgfpathlineto{\pgfqpoint{5.231190in}{3.021674in}}%
\pgfpathlineto{\pgfqpoint{5.223606in}{3.015838in}}%
\pgfpathlineto{\pgfqpoint{5.216014in}{3.009882in}}%
\pgfpathlineto{\pgfqpoint{5.208412in}{3.003804in}}%
\pgfpathclose%
\pgfusepath{fill}%
\end{pgfscope}%
\begin{pgfscope}%
\pgfpathrectangle{\pgfqpoint{1.150000in}{0.150000in}}{\pgfqpoint{5.700000in}{5.700000in}}%
\pgfusepath{clip}%
\pgfsetbuttcap%
\pgfsetroundjoin%
\definecolor{currentfill}{rgb}{0.268510,0.009605,0.335427}%
\pgfsetfillcolor{currentfill}%
\pgfsetfillopacity{0.800000}%
\pgfsetlinewidth{0.000000pt}%
\definecolor{currentstroke}{rgb}{0.000000,0.000000,0.000000}%
\pgfsetstrokecolor{currentstroke}%
\pgfsetdash{}{0pt}%
\pgfpathmoveto{\pgfqpoint{3.194530in}{1.498140in}}%
\pgfpathlineto{\pgfqpoint{3.208364in}{1.491139in}}%
\pgfpathlineto{\pgfqpoint{3.222200in}{1.484343in}}%
\pgfpathlineto{\pgfqpoint{3.236038in}{1.477750in}}%
\pgfpathlineto{\pgfqpoint{3.249877in}{1.471359in}}%
\pgfpathlineto{\pgfqpoint{3.258304in}{1.477553in}}%
\pgfpathlineto{\pgfqpoint{3.266720in}{1.483944in}}%
\pgfpathlineto{\pgfqpoint{3.275127in}{1.490524in}}%
\pgfpathlineto{\pgfqpoint{3.283523in}{1.497288in}}%
\pgfpathlineto{\pgfqpoint{3.269709in}{1.503083in}}%
\pgfpathlineto{\pgfqpoint{3.255898in}{1.509080in}}%
\pgfpathlineto{\pgfqpoint{3.242089in}{1.515280in}}%
\pgfpathlineto{\pgfqpoint{3.228281in}{1.521684in}}%
\pgfpathlineto{\pgfqpoint{3.219859in}{1.515504in}}%
\pgfpathlineto{\pgfqpoint{3.211427in}{1.509516in}}%
\pgfpathlineto{\pgfqpoint{3.202984in}{1.503726in}}%
\pgfpathlineto{\pgfqpoint{3.194530in}{1.498140in}}%
\pgfpathclose%
\pgfusepath{fill}%
\end{pgfscope}%
\begin{pgfscope}%
\pgfpathrectangle{\pgfqpoint{1.150000in}{0.150000in}}{\pgfqpoint{5.700000in}{5.700000in}}%
\pgfusepath{clip}%
\pgfsetbuttcap%
\pgfsetroundjoin%
\definecolor{currentfill}{rgb}{0.194100,0.399323,0.555565}%
\pgfsetfillcolor{currentfill}%
\pgfsetfillopacity{0.800000}%
\pgfsetlinewidth{0.000000pt}%
\definecolor{currentstroke}{rgb}{0.000000,0.000000,0.000000}%
\pgfsetstrokecolor{currentstroke}%
\pgfsetdash{}{0pt}%
\pgfpathmoveto{\pgfqpoint{4.582557in}{2.388522in}}%
\pgfpathlineto{\pgfqpoint{4.596794in}{2.398745in}}%
\pgfpathlineto{\pgfqpoint{4.611046in}{2.409152in}}%
\pgfpathlineto{\pgfqpoint{4.625315in}{2.419744in}}%
\pgfpathlineto{\pgfqpoint{4.639600in}{2.430521in}}%
\pgfpathlineto{\pgfqpoint{4.647507in}{2.441963in}}%
\pgfpathlineto{\pgfqpoint{4.655407in}{2.453275in}}%
\pgfpathlineto{\pgfqpoint{4.663301in}{2.464456in}}%
\pgfpathlineto{\pgfqpoint{4.671188in}{2.475506in}}%
\pgfpathlineto{\pgfqpoint{4.656904in}{2.464675in}}%
\pgfpathlineto{\pgfqpoint{4.642637in}{2.454030in}}%
\pgfpathlineto{\pgfqpoint{4.628386in}{2.443568in}}%
\pgfpathlineto{\pgfqpoint{4.614151in}{2.433292in}}%
\pgfpathlineto{\pgfqpoint{4.606262in}{2.422283in}}%
\pgfpathlineto{\pgfqpoint{4.598366in}{2.411152in}}%
\pgfpathlineto{\pgfqpoint{4.590465in}{2.399898in}}%
\pgfpathlineto{\pgfqpoint{4.582557in}{2.388522in}}%
\pgfpathclose%
\pgfusepath{fill}%
\end{pgfscope}%
\begin{pgfscope}%
\pgfpathrectangle{\pgfqpoint{1.150000in}{0.150000in}}{\pgfqpoint{5.700000in}{5.700000in}}%
\pgfusepath{clip}%
\pgfsetbuttcap%
\pgfsetroundjoin%
\definecolor{currentfill}{rgb}{0.185556,0.418570,0.556753}%
\pgfsetfillcolor{currentfill}%
\pgfsetfillopacity{0.800000}%
\pgfsetlinewidth{0.000000pt}%
\definecolor{currentstroke}{rgb}{0.000000,0.000000,0.000000}%
\pgfsetstrokecolor{currentstroke}%
\pgfsetdash{}{0pt}%
\pgfpathmoveto{\pgfqpoint{2.208868in}{2.546212in}}%
\pgfpathlineto{\pgfqpoint{2.223179in}{2.521309in}}%
\pgfpathlineto{\pgfqpoint{2.237474in}{2.496734in}}%
\pgfpathlineto{\pgfqpoint{2.251753in}{2.472482in}}%
\pgfpathlineto{\pgfqpoint{2.266018in}{2.448550in}}%
\pgfpathlineto{\pgfqpoint{2.275256in}{2.442640in}}%
\pgfpathlineto{\pgfqpoint{2.284471in}{2.437104in}}%
\pgfpathlineto{\pgfqpoint{2.293661in}{2.431933in}}%
\pgfpathlineto{\pgfqpoint{2.302828in}{2.427121in}}%
\pgfpathlineto{\pgfqpoint{2.288627in}{2.450366in}}%
\pgfpathlineto{\pgfqpoint{2.274410in}{2.473930in}}%
\pgfpathlineto{\pgfqpoint{2.260179in}{2.497816in}}%
\pgfpathlineto{\pgfqpoint{2.245933in}{2.522027in}}%
\pgfpathlineto{\pgfqpoint{2.236703in}{2.527513in}}%
\pgfpathlineto{\pgfqpoint{2.227450in}{2.533367in}}%
\pgfpathlineto{\pgfqpoint{2.218172in}{2.539598in}}%
\pgfpathlineto{\pgfqpoint{2.208868in}{2.546212in}}%
\pgfpathclose%
\pgfusepath{fill}%
\end{pgfscope}%
\begin{pgfscope}%
\pgfpathrectangle{\pgfqpoint{1.150000in}{0.150000in}}{\pgfqpoint{5.700000in}{5.700000in}}%
\pgfusepath{clip}%
\pgfsetbuttcap%
\pgfsetroundjoin%
\definecolor{currentfill}{rgb}{0.267004,0.004874,0.329415}%
\pgfsetfillcolor{currentfill}%
\pgfsetfillopacity{0.800000}%
\pgfsetlinewidth{0.000000pt}%
\definecolor{currentstroke}{rgb}{0.000000,0.000000,0.000000}%
\pgfsetstrokecolor{currentstroke}%
\pgfsetdash{}{0pt}%
\pgfpathmoveto{\pgfqpoint{3.338804in}{1.476116in}}%
\pgfpathlineto{\pgfqpoint{3.352632in}{1.471321in}}%
\pgfpathlineto{\pgfqpoint{3.366463in}{1.466724in}}%
\pgfpathlineto{\pgfqpoint{3.380298in}{1.462324in}}%
\pgfpathlineto{\pgfqpoint{3.394136in}{1.458121in}}%
\pgfpathlineto{\pgfqpoint{3.402478in}{1.466214in}}%
\pgfpathlineto{\pgfqpoint{3.410811in}{1.474463in}}%
\pgfpathlineto{\pgfqpoint{3.419136in}{1.482864in}}%
\pgfpathlineto{\pgfqpoint{3.427453in}{1.491410in}}%
\pgfpathlineto{\pgfqpoint{3.413635in}{1.495051in}}%
\pgfpathlineto{\pgfqpoint{3.399821in}{1.498888in}}%
\pgfpathlineto{\pgfqpoint{3.386011in}{1.502923in}}%
\pgfpathlineto{\pgfqpoint{3.372205in}{1.507154in}}%
\pgfpathlineto{\pgfqpoint{3.363868in}{1.499159in}}%
\pgfpathlineto{\pgfqpoint{3.355522in}{1.491317in}}%
\pgfpathlineto{\pgfqpoint{3.347168in}{1.483634in}}%
\pgfpathlineto{\pgfqpoint{3.338804in}{1.476116in}}%
\pgfpathclose%
\pgfusepath{fill}%
\end{pgfscope}%
\begin{pgfscope}%
\pgfpathrectangle{\pgfqpoint{1.150000in}{0.150000in}}{\pgfqpoint{5.700000in}{5.700000in}}%
\pgfusepath{clip}%
\pgfsetbuttcap%
\pgfsetroundjoin%
\definecolor{currentfill}{rgb}{0.180653,0.701402,0.488189}%
\pgfsetfillcolor{currentfill}%
\pgfsetfillopacity{0.800000}%
\pgfsetlinewidth{0.000000pt}%
\definecolor{currentstroke}{rgb}{0.000000,0.000000,0.000000}%
\pgfsetstrokecolor{currentstroke}%
\pgfsetdash{}{0pt}%
\pgfpathmoveto{\pgfqpoint{5.593955in}{3.324337in}}%
\pgfpathlineto{\pgfqpoint{5.608832in}{3.339541in}}%
\pgfpathlineto{\pgfqpoint{5.623730in}{3.354929in}}%
\pgfpathlineto{\pgfqpoint{5.638651in}{3.370502in}}%
\pgfpathlineto{\pgfqpoint{5.653594in}{3.386260in}}%
\pgfpathlineto{\pgfqpoint{5.660923in}{3.388470in}}%
\pgfpathlineto{\pgfqpoint{5.668242in}{3.390594in}}%
\pgfpathlineto{\pgfqpoint{5.675552in}{3.392636in}}%
\pgfpathlineto{\pgfqpoint{5.682853in}{3.394601in}}%
\pgfpathlineto{\pgfqpoint{5.667934in}{3.379283in}}%
\pgfpathlineto{\pgfqpoint{5.653036in}{3.364147in}}%
\pgfpathlineto{\pgfqpoint{5.638161in}{3.349196in}}%
\pgfpathlineto{\pgfqpoint{5.623308in}{3.334427in}}%
\pgfpathlineto{\pgfqpoint{5.615984in}{3.332012in}}%
\pgfpathlineto{\pgfqpoint{5.608650in}{3.329529in}}%
\pgfpathlineto{\pgfqpoint{5.601307in}{3.326972in}}%
\pgfpathlineto{\pgfqpoint{5.593955in}{3.324337in}}%
\pgfpathclose%
\pgfusepath{fill}%
\end{pgfscope}%
\begin{pgfscope}%
\pgfpathrectangle{\pgfqpoint{1.150000in}{0.150000in}}{\pgfqpoint{5.700000in}{5.700000in}}%
\pgfusepath{clip}%
\pgfsetbuttcap%
\pgfsetroundjoin%
\definecolor{currentfill}{rgb}{0.137770,0.537492,0.554906}%
\pgfsetfillcolor{currentfill}%
\pgfsetfillopacity{0.800000}%
\pgfsetlinewidth{0.000000pt}%
\definecolor{currentstroke}{rgb}{0.000000,0.000000,0.000000}%
\pgfsetstrokecolor{currentstroke}%
\pgfsetdash{}{0pt}%
\pgfpathmoveto{\pgfqpoint{5.000021in}{2.811227in}}%
\pgfpathlineto{\pgfqpoint{5.014514in}{2.824216in}}%
\pgfpathlineto{\pgfqpoint{5.029027in}{2.837390in}}%
\pgfpathlineto{\pgfqpoint{5.043558in}{2.850749in}}%
\pgfpathlineto{\pgfqpoint{5.058109in}{2.864293in}}%
\pgfpathlineto{\pgfqpoint{5.065823in}{2.872188in}}%
\pgfpathlineto{\pgfqpoint{5.073530in}{2.879939in}}%
\pgfpathlineto{\pgfqpoint{5.081227in}{2.887549in}}%
\pgfpathlineto{\pgfqpoint{5.088916in}{2.895019in}}%
\pgfpathlineto{\pgfqpoint{5.074373in}{2.881629in}}%
\pgfpathlineto{\pgfqpoint{5.059850in}{2.868423in}}%
\pgfpathlineto{\pgfqpoint{5.045345in}{2.855402in}}%
\pgfpathlineto{\pgfqpoint{5.030859in}{2.842565in}}%
\pgfpathlineto{\pgfqpoint{5.023162in}{2.834929in}}%
\pgfpathlineto{\pgfqpoint{5.015456in}{2.827162in}}%
\pgfpathlineto{\pgfqpoint{5.007743in}{2.819262in}}%
\pgfpathlineto{\pgfqpoint{5.000021in}{2.811227in}}%
\pgfpathclose%
\pgfusepath{fill}%
\end{pgfscope}%
\begin{pgfscope}%
\pgfpathrectangle{\pgfqpoint{1.150000in}{0.150000in}}{\pgfqpoint{5.700000in}{5.700000in}}%
\pgfusepath{clip}%
\pgfsetbuttcap%
\pgfsetroundjoin%
\definecolor{currentfill}{rgb}{0.277134,0.185228,0.489898}%
\pgfsetfillcolor{currentfill}%
\pgfsetfillopacity{0.800000}%
\pgfsetlinewidth{0.000000pt}%
\definecolor{currentstroke}{rgb}{0.000000,0.000000,0.000000}%
\pgfsetstrokecolor{currentstroke}%
\pgfsetdash{}{0pt}%
\pgfpathmoveto{\pgfqpoint{4.044797in}{1.831919in}}%
\pgfpathlineto{\pgfqpoint{4.058770in}{1.836835in}}%
\pgfpathlineto{\pgfqpoint{4.072755in}{1.841935in}}%
\pgfpathlineto{\pgfqpoint{4.086751in}{1.847220in}}%
\pgfpathlineto{\pgfqpoint{4.100758in}{1.852690in}}%
\pgfpathlineto{\pgfqpoint{4.108834in}{1.866080in}}%
\pgfpathlineto{\pgfqpoint{4.116906in}{1.879430in}}%
\pgfpathlineto{\pgfqpoint{4.124973in}{1.892735in}}%
\pgfpathlineto{\pgfqpoint{4.133035in}{1.905995in}}%
\pgfpathlineto{\pgfqpoint{4.119030in}{1.900211in}}%
\pgfpathlineto{\pgfqpoint{4.105036in}{1.894611in}}%
\pgfpathlineto{\pgfqpoint{4.091054in}{1.889197in}}%
\pgfpathlineto{\pgfqpoint{4.077084in}{1.883967in}}%
\pgfpathlineto{\pgfqpoint{4.069019in}{1.871010in}}%
\pgfpathlineto{\pgfqpoint{4.060950in}{1.858014in}}%
\pgfpathlineto{\pgfqpoint{4.052876in}{1.844983in}}%
\pgfpathlineto{\pgfqpoint{4.044797in}{1.831919in}}%
\pgfpathclose%
\pgfusepath{fill}%
\end{pgfscope}%
\begin{pgfscope}%
\pgfpathrectangle{\pgfqpoint{1.150000in}{0.150000in}}{\pgfqpoint{5.700000in}{5.700000in}}%
\pgfusepath{clip}%
\pgfsetbuttcap%
\pgfsetroundjoin%
\definecolor{currentfill}{rgb}{0.163625,0.471133,0.558148}%
\pgfsetfillcolor{currentfill}%
\pgfsetfillopacity{0.800000}%
\pgfsetlinewidth{0.000000pt}%
\definecolor{currentstroke}{rgb}{0.000000,0.000000,0.000000}%
\pgfsetstrokecolor{currentstroke}%
\pgfsetdash{}{0pt}%
\pgfpathmoveto{\pgfqpoint{4.791331in}{2.604414in}}%
\pgfpathlineto{\pgfqpoint{4.805694in}{2.616174in}}%
\pgfpathlineto{\pgfqpoint{4.820075in}{2.628119in}}%
\pgfpathlineto{\pgfqpoint{4.834474in}{2.640248in}}%
\pgfpathlineto{\pgfqpoint{4.848890in}{2.652563in}}%
\pgfpathlineto{\pgfqpoint{4.856710in}{2.662378in}}%
\pgfpathlineto{\pgfqpoint{4.864523in}{2.672048in}}%
\pgfpathlineto{\pgfqpoint{4.872328in}{2.681577in}}%
\pgfpathlineto{\pgfqpoint{4.880126in}{2.690964in}}%
\pgfpathlineto{\pgfqpoint{4.865713in}{2.678698in}}%
\pgfpathlineto{\pgfqpoint{4.851319in}{2.666617in}}%
\pgfpathlineto{\pgfqpoint{4.836942in}{2.654721in}}%
\pgfpathlineto{\pgfqpoint{4.822583in}{2.643009in}}%
\pgfpathlineto{\pgfqpoint{4.814780in}{2.633561in}}%
\pgfpathlineto{\pgfqpoint{4.806971in}{2.623980in}}%
\pgfpathlineto{\pgfqpoint{4.799154in}{2.614264in}}%
\pgfpathlineto{\pgfqpoint{4.791331in}{2.604414in}}%
\pgfpathclose%
\pgfusepath{fill}%
\end{pgfscope}%
\begin{pgfscope}%
\pgfpathrectangle{\pgfqpoint{1.150000in}{0.150000in}}{\pgfqpoint{5.700000in}{5.700000in}}%
\pgfusepath{clip}%
\pgfsetbuttcap%
\pgfsetroundjoin%
\definecolor{currentfill}{rgb}{0.252194,0.269783,0.531579}%
\pgfsetfillcolor{currentfill}%
\pgfsetfillopacity{0.800000}%
\pgfsetlinewidth{0.000000pt}%
\definecolor{currentstroke}{rgb}{0.000000,0.000000,0.000000}%
\pgfsetstrokecolor{currentstroke}%
\pgfsetdash{}{0pt}%
\pgfpathmoveto{\pgfqpoint{4.253499in}{2.037232in}}%
\pgfpathlineto{\pgfqpoint{4.267564in}{2.044473in}}%
\pgfpathlineto{\pgfqpoint{4.281643in}{2.051899in}}%
\pgfpathlineto{\pgfqpoint{4.295735in}{2.059510in}}%
\pgfpathlineto{\pgfqpoint{4.309841in}{2.067305in}}%
\pgfpathlineto{\pgfqpoint{4.317859in}{2.080502in}}%
\pgfpathlineto{\pgfqpoint{4.325872in}{2.093613in}}%
\pgfpathlineto{\pgfqpoint{4.333880in}{2.106638in}}%
\pgfpathlineto{\pgfqpoint{4.341883in}{2.119574in}}%
\pgfpathlineto{\pgfqpoint{4.327778in}{2.111560in}}%
\pgfpathlineto{\pgfqpoint{4.313686in}{2.103730in}}%
\pgfpathlineto{\pgfqpoint{4.299609in}{2.096086in}}%
\pgfpathlineto{\pgfqpoint{4.285544in}{2.088626in}}%
\pgfpathlineto{\pgfqpoint{4.277540in}{2.075896in}}%
\pgfpathlineto{\pgfqpoint{4.269531in}{2.063086in}}%
\pgfpathlineto{\pgfqpoint{4.261517in}{2.050197in}}%
\pgfpathlineto{\pgfqpoint{4.253499in}{2.037232in}}%
\pgfpathclose%
\pgfusepath{fill}%
\end{pgfscope}%
\begin{pgfscope}%
\pgfpathrectangle{\pgfqpoint{1.150000in}{0.150000in}}{\pgfqpoint{5.700000in}{5.700000in}}%
\pgfusepath{clip}%
\pgfsetbuttcap%
\pgfsetroundjoin%
\definecolor{currentfill}{rgb}{0.277018,0.050344,0.375715}%
\pgfsetfillcolor{currentfill}%
\pgfsetfillopacity{0.800000}%
\pgfsetlinewidth{0.000000pt}%
\definecolor{currentstroke}{rgb}{0.000000,0.000000,0.000000}%
\pgfsetstrokecolor{currentstroke}%
\pgfsetdash{}{0pt}%
\pgfpathmoveto{\pgfqpoint{3.659535in}{1.548179in}}%
\pgfpathlineto{\pgfqpoint{3.673395in}{1.548069in}}%
\pgfpathlineto{\pgfqpoint{3.687263in}{1.548148in}}%
\pgfpathlineto{\pgfqpoint{3.701138in}{1.548416in}}%
\pgfpathlineto{\pgfqpoint{3.715020in}{1.548871in}}%
\pgfpathlineto{\pgfqpoint{3.723217in}{1.560420in}}%
\pgfpathlineto{\pgfqpoint{3.731408in}{1.572033in}}%
\pgfpathlineto{\pgfqpoint{3.739594in}{1.583707in}}%
\pgfpathlineto{\pgfqpoint{3.747774in}{1.595436in}}%
\pgfpathlineto{\pgfqpoint{3.733901in}{1.594511in}}%
\pgfpathlineto{\pgfqpoint{3.720037in}{1.593774in}}%
\pgfpathlineto{\pgfqpoint{3.706180in}{1.593226in}}%
\pgfpathlineto{\pgfqpoint{3.692331in}{1.592867in}}%
\pgfpathlineto{\pgfqpoint{3.684141in}{1.581595in}}%
\pgfpathlineto{\pgfqpoint{3.675945in}{1.570387in}}%
\pgfpathlineto{\pgfqpoint{3.667743in}{1.559247in}}%
\pgfpathlineto{\pgfqpoint{3.659535in}{1.548179in}}%
\pgfpathclose%
\pgfusepath{fill}%
\end{pgfscope}%
\begin{pgfscope}%
\pgfpathrectangle{\pgfqpoint{1.150000in}{0.150000in}}{\pgfqpoint{5.700000in}{5.700000in}}%
\pgfusepath{clip}%
\pgfsetbuttcap%
\pgfsetroundjoin%
\definecolor{currentfill}{rgb}{0.277134,0.185228,0.489898}%
\pgfsetfillcolor{currentfill}%
\pgfsetfillopacity{0.800000}%
\pgfsetlinewidth{0.000000pt}%
\definecolor{currentstroke}{rgb}{0.000000,0.000000,0.000000}%
\pgfsetstrokecolor{currentstroke}%
\pgfsetdash{}{0pt}%
\pgfpathmoveto{\pgfqpoint{2.624972in}{1.905595in}}%
\pgfpathlineto{\pgfqpoint{2.638981in}{1.889163in}}%
\pgfpathlineto{\pgfqpoint{2.652982in}{1.872983in}}%
\pgfpathlineto{\pgfqpoint{2.666977in}{1.857052in}}%
\pgfpathlineto{\pgfqpoint{2.680965in}{1.841370in}}%
\pgfpathlineto{\pgfqpoint{2.689841in}{1.839625in}}%
\pgfpathlineto{\pgfqpoint{2.698697in}{1.838204in}}%
\pgfpathlineto{\pgfqpoint{2.707536in}{1.837102in}}%
\pgfpathlineto{\pgfqpoint{2.716356in}{1.836310in}}%
\pgfpathlineto{\pgfqpoint{2.702416in}{1.851310in}}%
\pgfpathlineto{\pgfqpoint{2.688470in}{1.866556in}}%
\pgfpathlineto{\pgfqpoint{2.674518in}{1.882052in}}%
\pgfpathlineto{\pgfqpoint{2.660559in}{1.897797in}}%
\pgfpathlineto{\pgfqpoint{2.651691in}{1.899259in}}%
\pgfpathlineto{\pgfqpoint{2.642804in}{1.901042in}}%
\pgfpathlineto{\pgfqpoint{2.633898in}{1.903151in}}%
\pgfpathlineto{\pgfqpoint{2.624972in}{1.905595in}}%
\pgfpathclose%
\pgfusepath{fill}%
\end{pgfscope}%
\begin{pgfscope}%
\pgfpathrectangle{\pgfqpoint{1.150000in}{0.150000in}}{\pgfqpoint{5.700000in}{5.700000in}}%
\pgfusepath{clip}%
\pgfsetbuttcap%
\pgfsetroundjoin%
\definecolor{currentfill}{rgb}{0.270595,0.214069,0.507052}%
\pgfsetfillcolor{currentfill}%
\pgfsetfillopacity{0.800000}%
\pgfsetlinewidth{0.000000pt}%
\definecolor{currentstroke}{rgb}{0.000000,0.000000,0.000000}%
\pgfsetstrokecolor{currentstroke}%
\pgfsetdash{}{0pt}%
\pgfpathmoveto{\pgfqpoint{2.568863in}{1.973872in}}%
\pgfpathlineto{\pgfqpoint{2.582902in}{1.956416in}}%
\pgfpathlineto{\pgfqpoint{2.596933in}{1.939219in}}%
\pgfpathlineto{\pgfqpoint{2.610956in}{1.922279in}}%
\pgfpathlineto{\pgfqpoint{2.624972in}{1.905595in}}%
\pgfpathlineto{\pgfqpoint{2.633898in}{1.903151in}}%
\pgfpathlineto{\pgfqpoint{2.642804in}{1.901042in}}%
\pgfpathlineto{\pgfqpoint{2.651691in}{1.899259in}}%
\pgfpathlineto{\pgfqpoint{2.660559in}{1.897797in}}%
\pgfpathlineto{\pgfqpoint{2.646593in}{1.913795in}}%
\pgfpathlineto{\pgfqpoint{2.632621in}{1.930047in}}%
\pgfpathlineto{\pgfqpoint{2.618641in}{1.946555in}}%
\pgfpathlineto{\pgfqpoint{2.604654in}{1.963320in}}%
\pgfpathlineto{\pgfqpoint{2.595736in}{1.965457in}}%
\pgfpathlineto{\pgfqpoint{2.586798in}{1.967923in}}%
\pgfpathlineto{\pgfqpoint{2.577841in}{1.970725in}}%
\pgfpathlineto{\pgfqpoint{2.568863in}{1.973872in}}%
\pgfpathclose%
\pgfusepath{fill}%
\end{pgfscope}%
\begin{pgfscope}%
\pgfpathrectangle{\pgfqpoint{1.150000in}{0.150000in}}{\pgfqpoint{5.700000in}{5.700000in}}%
\pgfusepath{clip}%
\pgfsetbuttcap%
\pgfsetroundjoin%
\definecolor{currentfill}{rgb}{0.280267,0.073417,0.397163}%
\pgfsetfillcolor{currentfill}%
\pgfsetfillopacity{0.800000}%
\pgfsetlinewidth{0.000000pt}%
\definecolor{currentstroke}{rgb}{0.000000,0.000000,0.000000}%
\pgfsetstrokecolor{currentstroke}%
\pgfsetdash{}{0pt}%
\pgfpathmoveto{\pgfqpoint{3.747774in}{1.595436in}}%
\pgfpathlineto{\pgfqpoint{3.761654in}{1.596549in}}%
\pgfpathlineto{\pgfqpoint{3.775542in}{1.597849in}}%
\pgfpathlineto{\pgfqpoint{3.789439in}{1.599336in}}%
\pgfpathlineto{\pgfqpoint{3.803344in}{1.601010in}}%
\pgfpathlineto{\pgfqpoint{3.811510in}{1.613242in}}%
\pgfpathlineto{\pgfqpoint{3.819671in}{1.625512in}}%
\pgfpathlineto{\pgfqpoint{3.827827in}{1.637818in}}%
\pgfpathlineto{\pgfqpoint{3.835978in}{1.650155in}}%
\pgfpathlineto{\pgfqpoint{3.822080in}{1.648043in}}%
\pgfpathlineto{\pgfqpoint{3.808191in}{1.646117in}}%
\pgfpathlineto{\pgfqpoint{3.794311in}{1.644378in}}%
\pgfpathlineto{\pgfqpoint{3.780439in}{1.642827in}}%
\pgfpathlineto{\pgfqpoint{3.772281in}{1.630916in}}%
\pgfpathlineto{\pgfqpoint{3.764117in}{1.619045in}}%
\pgfpathlineto{\pgfqpoint{3.755948in}{1.607217in}}%
\pgfpathlineto{\pgfqpoint{3.747774in}{1.595436in}}%
\pgfpathclose%
\pgfusepath{fill}%
\end{pgfscope}%
\begin{pgfscope}%
\pgfpathrectangle{\pgfqpoint{1.150000in}{0.150000in}}{\pgfqpoint{5.700000in}{5.700000in}}%
\pgfusepath{clip}%
\pgfsetbuttcap%
\pgfsetroundjoin%
\definecolor{currentfill}{rgb}{0.272594,0.025563,0.353093}%
\pgfsetfillcolor{currentfill}%
\pgfsetfillopacity{0.800000}%
\pgfsetlinewidth{0.000000pt}%
\definecolor{currentstroke}{rgb}{0.000000,0.000000,0.000000}%
\pgfsetstrokecolor{currentstroke}%
\pgfsetdash{}{0pt}%
\pgfpathmoveto{\pgfqpoint{3.571217in}{1.509064in}}%
\pgfpathlineto{\pgfqpoint{3.585064in}{1.507694in}}%
\pgfpathlineto{\pgfqpoint{3.598917in}{1.506515in}}%
\pgfpathlineto{\pgfqpoint{3.612776in}{1.505526in}}%
\pgfpathlineto{\pgfqpoint{3.626642in}{1.504726in}}%
\pgfpathlineto{\pgfqpoint{3.634875in}{1.515457in}}%
\pgfpathlineto{\pgfqpoint{3.643101in}{1.526280in}}%
\pgfpathlineto{\pgfqpoint{3.651321in}{1.537189in}}%
\pgfpathlineto{\pgfqpoint{3.659535in}{1.548179in}}%
\pgfpathlineto{\pgfqpoint{3.645682in}{1.548478in}}%
\pgfpathlineto{\pgfqpoint{3.631835in}{1.548967in}}%
\pgfpathlineto{\pgfqpoint{3.617995in}{1.549646in}}%
\pgfpathlineto{\pgfqpoint{3.604162in}{1.550516in}}%
\pgfpathlineto{\pgfqpoint{3.595935in}{1.540014in}}%
\pgfpathlineto{\pgfqpoint{3.587702in}{1.529601in}}%
\pgfpathlineto{\pgfqpoint{3.579463in}{1.519283in}}%
\pgfpathlineto{\pgfqpoint{3.571217in}{1.509064in}}%
\pgfpathclose%
\pgfusepath{fill}%
\end{pgfscope}%
\begin{pgfscope}%
\pgfpathrectangle{\pgfqpoint{1.150000in}{0.150000in}}{\pgfqpoint{5.700000in}{5.700000in}}%
\pgfusepath{clip}%
\pgfsetbuttcap%
\pgfsetroundjoin%
\definecolor{currentfill}{rgb}{0.280868,0.160771,0.472899}%
\pgfsetfillcolor{currentfill}%
\pgfsetfillopacity{0.800000}%
\pgfsetlinewidth{0.000000pt}%
\definecolor{currentstroke}{rgb}{0.000000,0.000000,0.000000}%
\pgfsetstrokecolor{currentstroke}%
\pgfsetdash{}{0pt}%
\pgfpathmoveto{\pgfqpoint{2.680965in}{1.841370in}}%
\pgfpathlineto{\pgfqpoint{2.694947in}{1.825934in}}%
\pgfpathlineto{\pgfqpoint{2.708923in}{1.810742in}}%
\pgfpathlineto{\pgfqpoint{2.722894in}{1.795793in}}%
\pgfpathlineto{\pgfqpoint{2.736858in}{1.781086in}}%
\pgfpathlineto{\pgfqpoint{2.745686in}{1.780034in}}%
\pgfpathlineto{\pgfqpoint{2.754495in}{1.779299in}}%
\pgfpathlineto{\pgfqpoint{2.763287in}{1.778873in}}%
\pgfpathlineto{\pgfqpoint{2.772062in}{1.778749in}}%
\pgfpathlineto{\pgfqpoint{2.758143in}{1.792777in}}%
\pgfpathlineto{\pgfqpoint{2.744220in}{1.807046in}}%
\pgfpathlineto{\pgfqpoint{2.730291in}{1.821556in}}%
\pgfpathlineto{\pgfqpoint{2.716356in}{1.836310in}}%
\pgfpathlineto{\pgfqpoint{2.707536in}{1.837102in}}%
\pgfpathlineto{\pgfqpoint{2.698697in}{1.838204in}}%
\pgfpathlineto{\pgfqpoint{2.689841in}{1.839625in}}%
\pgfpathlineto{\pgfqpoint{2.680965in}{1.841370in}}%
\pgfpathclose%
\pgfusepath{fill}%
\end{pgfscope}%
\begin{pgfscope}%
\pgfpathrectangle{\pgfqpoint{1.150000in}{0.150000in}}{\pgfqpoint{5.700000in}{5.700000in}}%
\pgfusepath{clip}%
\pgfsetbuttcap%
\pgfsetroundjoin%
\definecolor{currentfill}{rgb}{0.262138,0.242286,0.520837}%
\pgfsetfillcolor{currentfill}%
\pgfsetfillopacity{0.800000}%
\pgfsetlinewidth{0.000000pt}%
\definecolor{currentstroke}{rgb}{0.000000,0.000000,0.000000}%
\pgfsetstrokecolor{currentstroke}%
\pgfsetdash{}{0pt}%
\pgfpathmoveto{\pgfqpoint{2.512622in}{2.046321in}}%
\pgfpathlineto{\pgfqpoint{2.526695in}{2.027811in}}%
\pgfpathlineto{\pgfqpoint{2.540760in}{2.009567in}}%
\pgfpathlineto{\pgfqpoint{2.554816in}{1.991588in}}%
\pgfpathlineto{\pgfqpoint{2.568863in}{1.973872in}}%
\pgfpathlineto{\pgfqpoint{2.577841in}{1.970725in}}%
\pgfpathlineto{\pgfqpoint{2.586798in}{1.967923in}}%
\pgfpathlineto{\pgfqpoint{2.595736in}{1.965457in}}%
\pgfpathlineto{\pgfqpoint{2.604654in}{1.963320in}}%
\pgfpathlineto{\pgfqpoint{2.590659in}{1.980345in}}%
\pgfpathlineto{\pgfqpoint{2.576656in}{1.997632in}}%
\pgfpathlineto{\pgfqpoint{2.562645in}{2.015182in}}%
\pgfpathlineto{\pgfqpoint{2.548625in}{2.032998in}}%
\pgfpathlineto{\pgfqpoint{2.539655in}{2.035814in}}%
\pgfpathlineto{\pgfqpoint{2.530665in}{2.038968in}}%
\pgfpathlineto{\pgfqpoint{2.521654in}{2.042468in}}%
\pgfpathlineto{\pgfqpoint{2.512622in}{2.046321in}}%
\pgfpathclose%
\pgfusepath{fill}%
\end{pgfscope}%
\begin{pgfscope}%
\pgfpathrectangle{\pgfqpoint{1.150000in}{0.150000in}}{\pgfqpoint{5.700000in}{5.700000in}}%
\pgfusepath{clip}%
\pgfsetbuttcap%
\pgfsetroundjoin%
\definecolor{currentfill}{rgb}{0.273809,0.031497,0.358853}%
\pgfsetfillcolor{currentfill}%
\pgfsetfillopacity{0.800000}%
\pgfsetlinewidth{0.000000pt}%
\definecolor{currentstroke}{rgb}{0.000000,0.000000,0.000000}%
\pgfsetstrokecolor{currentstroke}%
\pgfsetdash{}{0pt}%
\pgfpathmoveto{\pgfqpoint{3.049708in}{1.546469in}}%
\pgfpathlineto{\pgfqpoint{3.063570in}{1.537170in}}%
\pgfpathlineto{\pgfqpoint{3.077432in}{1.528083in}}%
\pgfpathlineto{\pgfqpoint{3.091293in}{1.519207in}}%
\pgfpathlineto{\pgfqpoint{3.105155in}{1.510541in}}%
\pgfpathlineto{\pgfqpoint{3.113685in}{1.514607in}}%
\pgfpathlineto{\pgfqpoint{3.122203in}{1.518910in}}%
\pgfpathlineto{\pgfqpoint{3.130709in}{1.523444in}}%
\pgfpathlineto{\pgfqpoint{3.139204in}{1.528202in}}%
\pgfpathlineto{\pgfqpoint{3.125374in}{1.536237in}}%
\pgfpathlineto{\pgfqpoint{3.111544in}{1.544481in}}%
\pgfpathlineto{\pgfqpoint{3.097715in}{1.552937in}}%
\pgfpathlineto{\pgfqpoint{3.083886in}{1.561604in}}%
\pgfpathlineto{\pgfqpoint{3.075361in}{1.557464in}}%
\pgfpathlineto{\pgfqpoint{3.066823in}{1.553558in}}%
\pgfpathlineto{\pgfqpoint{3.058272in}{1.549891in}}%
\pgfpathlineto{\pgfqpoint{3.049708in}{1.546469in}}%
\pgfpathclose%
\pgfusepath{fill}%
\end{pgfscope}%
\begin{pgfscope}%
\pgfpathrectangle{\pgfqpoint{1.150000in}{0.150000in}}{\pgfqpoint{5.700000in}{5.700000in}}%
\pgfusepath{clip}%
\pgfsetbuttcap%
\pgfsetroundjoin%
\definecolor{currentfill}{rgb}{0.120638,0.625828,0.533488}%
\pgfsetfillcolor{currentfill}%
\pgfsetfillopacity{0.800000}%
\pgfsetlinewidth{0.000000pt}%
\definecolor{currentstroke}{rgb}{0.000000,0.000000,0.000000}%
\pgfsetstrokecolor{currentstroke}%
\pgfsetdash{}{0pt}%
\pgfpathmoveto{\pgfqpoint{5.297322in}{3.083082in}}%
\pgfpathlineto{\pgfqpoint{5.312012in}{3.097465in}}%
\pgfpathlineto{\pgfqpoint{5.326723in}{3.112033in}}%
\pgfpathlineto{\pgfqpoint{5.341455in}{3.126787in}}%
\pgfpathlineto{\pgfqpoint{5.356208in}{3.141725in}}%
\pgfpathlineto{\pgfqpoint{5.363747in}{3.146770in}}%
\pgfpathlineto{\pgfqpoint{5.371276in}{3.151689in}}%
\pgfpathlineto{\pgfqpoint{5.378796in}{3.156484in}}%
\pgfpathlineto{\pgfqpoint{5.386307in}{3.161159in}}%
\pgfpathlineto{\pgfqpoint{5.371569in}{3.146517in}}%
\pgfpathlineto{\pgfqpoint{5.356852in}{3.132058in}}%
\pgfpathlineto{\pgfqpoint{5.342155in}{3.117785in}}%
\pgfpathlineto{\pgfqpoint{5.327479in}{3.103695in}}%
\pgfpathlineto{\pgfqpoint{5.319954in}{3.098712in}}%
\pgfpathlineto{\pgfqpoint{5.312419in}{3.093618in}}%
\pgfpathlineto{\pgfqpoint{5.304875in}{3.088409in}}%
\pgfpathlineto{\pgfqpoint{5.297322in}{3.083082in}}%
\pgfpathclose%
\pgfusepath{fill}%
\end{pgfscope}%
\begin{pgfscope}%
\pgfpathrectangle{\pgfqpoint{1.150000in}{0.150000in}}{\pgfqpoint{5.700000in}{5.700000in}}%
\pgfusepath{clip}%
\pgfsetbuttcap%
\pgfsetroundjoin%
\definecolor{currentfill}{rgb}{0.282884,0.135920,0.453427}%
\pgfsetfillcolor{currentfill}%
\pgfsetfillopacity{0.800000}%
\pgfsetlinewidth{0.000000pt}%
\definecolor{currentstroke}{rgb}{0.000000,0.000000,0.000000}%
\pgfsetstrokecolor{currentstroke}%
\pgfsetdash{}{0pt}%
\pgfpathmoveto{\pgfqpoint{2.736858in}{1.781086in}}%
\pgfpathlineto{\pgfqpoint{2.750817in}{1.766618in}}%
\pgfpathlineto{\pgfqpoint{2.764772in}{1.752388in}}%
\pgfpathlineto{\pgfqpoint{2.778721in}{1.738395in}}%
\pgfpathlineto{\pgfqpoint{2.792665in}{1.724637in}}%
\pgfpathlineto{\pgfqpoint{2.801447in}{1.724277in}}%
\pgfpathlineto{\pgfqpoint{2.810211in}{1.724223in}}%
\pgfpathlineto{\pgfqpoint{2.818959in}{1.724470in}}%
\pgfpathlineto{\pgfqpoint{2.827690in}{1.725010in}}%
\pgfpathlineto{\pgfqpoint{2.813790in}{1.738092in}}%
\pgfpathlineto{\pgfqpoint{2.799885in}{1.751408in}}%
\pgfpathlineto{\pgfqpoint{2.785976in}{1.764960in}}%
\pgfpathlineto{\pgfqpoint{2.772062in}{1.778749in}}%
\pgfpathlineto{\pgfqpoint{2.763287in}{1.778873in}}%
\pgfpathlineto{\pgfqpoint{2.754495in}{1.779299in}}%
\pgfpathlineto{\pgfqpoint{2.745686in}{1.780034in}}%
\pgfpathlineto{\pgfqpoint{2.736858in}{1.781086in}}%
\pgfpathclose%
\pgfusepath{fill}%
\end{pgfscope}%
\begin{pgfscope}%
\pgfpathrectangle{\pgfqpoint{1.150000in}{0.150000in}}{\pgfqpoint{5.700000in}{5.700000in}}%
\pgfusepath{clip}%
\pgfsetbuttcap%
\pgfsetroundjoin%
\definecolor{currentfill}{rgb}{0.214298,0.355619,0.551184}%
\pgfsetfillcolor{currentfill}%
\pgfsetfillopacity{0.800000}%
\pgfsetlinewidth{0.000000pt}%
\definecolor{currentstroke}{rgb}{0.000000,0.000000,0.000000}%
\pgfsetstrokecolor{currentstroke}%
\pgfsetdash{}{0pt}%
\pgfpathmoveto{\pgfqpoint{4.462283in}{2.254945in}}%
\pgfpathlineto{\pgfqpoint{4.476459in}{2.264222in}}%
\pgfpathlineto{\pgfqpoint{4.490651in}{2.273683in}}%
\pgfpathlineto{\pgfqpoint{4.504858in}{2.283328in}}%
\pgfpathlineto{\pgfqpoint{4.519080in}{2.293159in}}%
\pgfpathlineto{\pgfqpoint{4.527035in}{2.305497in}}%
\pgfpathlineto{\pgfqpoint{4.534984in}{2.317718in}}%
\pgfpathlineto{\pgfqpoint{4.542928in}{2.329820in}}%
\pgfpathlineto{\pgfqpoint{4.550865in}{2.341802in}}%
\pgfpathlineto{\pgfqpoint{4.536644in}{2.331851in}}%
\pgfpathlineto{\pgfqpoint{4.522437in}{2.322085in}}%
\pgfpathlineto{\pgfqpoint{4.508246in}{2.312503in}}%
\pgfpathlineto{\pgfqpoint{4.494071in}{2.303106in}}%
\pgfpathlineto{\pgfqpoint{4.486132in}{2.291232in}}%
\pgfpathlineto{\pgfqpoint{4.478188in}{2.279247in}}%
\pgfpathlineto{\pgfqpoint{4.470238in}{2.267151in}}%
\pgfpathlineto{\pgfqpoint{4.462283in}{2.254945in}}%
\pgfpathclose%
\pgfusepath{fill}%
\end{pgfscope}%
\begin{pgfscope}%
\pgfpathrectangle{\pgfqpoint{1.150000in}{0.150000in}}{\pgfqpoint{5.700000in}{5.700000in}}%
\pgfusepath{clip}%
\pgfsetbuttcap%
\pgfsetroundjoin%
\definecolor{currentfill}{rgb}{0.250425,0.274290,0.533103}%
\pgfsetfillcolor{currentfill}%
\pgfsetfillopacity{0.800000}%
\pgfsetlinewidth{0.000000pt}%
\definecolor{currentstroke}{rgb}{0.000000,0.000000,0.000000}%
\pgfsetstrokecolor{currentstroke}%
\pgfsetdash{}{0pt}%
\pgfpathmoveto{\pgfqpoint{2.456231in}{2.123074in}}%
\pgfpathlineto{\pgfqpoint{2.470344in}{2.103475in}}%
\pgfpathlineto{\pgfqpoint{2.484446in}{2.084152in}}%
\pgfpathlineto{\pgfqpoint{2.498539in}{2.065101in}}%
\pgfpathlineto{\pgfqpoint{2.512622in}{2.046321in}}%
\pgfpathlineto{\pgfqpoint{2.521654in}{2.042468in}}%
\pgfpathlineto{\pgfqpoint{2.530665in}{2.038968in}}%
\pgfpathlineto{\pgfqpoint{2.539655in}{2.035814in}}%
\pgfpathlineto{\pgfqpoint{2.548625in}{2.032998in}}%
\pgfpathlineto{\pgfqpoint{2.534597in}{2.051081in}}%
\pgfpathlineto{\pgfqpoint{2.520559in}{2.069435in}}%
\pgfpathlineto{\pgfqpoint{2.506513in}{2.088060in}}%
\pgfpathlineto{\pgfqpoint{2.492456in}{2.106959in}}%
\pgfpathlineto{\pgfqpoint{2.483433in}{2.110458in}}%
\pgfpathlineto{\pgfqpoint{2.474387in}{2.114306in}}%
\pgfpathlineto{\pgfqpoint{2.465320in}{2.118509in}}%
\pgfpathlineto{\pgfqpoint{2.456231in}{2.123074in}}%
\pgfpathclose%
\pgfusepath{fill}%
\end{pgfscope}%
\begin{pgfscope}%
\pgfpathrectangle{\pgfqpoint{1.150000in}{0.150000in}}{\pgfqpoint{5.700000in}{5.700000in}}%
\pgfusepath{clip}%
\pgfsetbuttcap%
\pgfsetroundjoin%
\definecolor{currentfill}{rgb}{0.282910,0.105393,0.426902}%
\pgfsetfillcolor{currentfill}%
\pgfsetfillopacity{0.800000}%
\pgfsetlinewidth{0.000000pt}%
\definecolor{currentstroke}{rgb}{0.000000,0.000000,0.000000}%
\pgfsetstrokecolor{currentstroke}%
\pgfsetdash{}{0pt}%
\pgfpathmoveto{\pgfqpoint{3.835978in}{1.650155in}}%
\pgfpathlineto{\pgfqpoint{3.849884in}{1.652455in}}%
\pgfpathlineto{\pgfqpoint{3.863800in}{1.654941in}}%
\pgfpathlineto{\pgfqpoint{3.877725in}{1.657613in}}%
\pgfpathlineto{\pgfqpoint{3.891659in}{1.660470in}}%
\pgfpathlineto{\pgfqpoint{3.899798in}{1.673254in}}%
\pgfpathlineto{\pgfqpoint{3.907933in}{1.686054in}}%
\pgfpathlineto{\pgfqpoint{3.916062in}{1.698866in}}%
\pgfpathlineto{\pgfqpoint{3.924187in}{1.711685in}}%
\pgfpathlineto{\pgfqpoint{3.910258in}{1.708420in}}%
\pgfpathlineto{\pgfqpoint{3.896339in}{1.705340in}}%
\pgfpathlineto{\pgfqpoint{3.882430in}{1.702446in}}%
\pgfpathlineto{\pgfqpoint{3.868530in}{1.699739in}}%
\pgfpathlineto{\pgfqpoint{3.860399in}{1.687316in}}%
\pgfpathlineto{\pgfqpoint{3.852264in}{1.674908in}}%
\pgfpathlineto{\pgfqpoint{3.844123in}{1.662520in}}%
\pgfpathlineto{\pgfqpoint{3.835978in}{1.650155in}}%
\pgfpathclose%
\pgfusepath{fill}%
\end{pgfscope}%
\begin{pgfscope}%
\pgfpathrectangle{\pgfqpoint{1.150000in}{0.150000in}}{\pgfqpoint{5.700000in}{5.700000in}}%
\pgfusepath{clip}%
\pgfsetbuttcap%
\pgfsetroundjoin%
\definecolor{currentfill}{rgb}{0.268510,0.009605,0.335427}%
\pgfsetfillcolor{currentfill}%
\pgfsetfillopacity{0.800000}%
\pgfsetlinewidth{0.000000pt}%
\definecolor{currentstroke}{rgb}{0.000000,0.000000,0.000000}%
\pgfsetstrokecolor{currentstroke}%
\pgfsetdash{}{0pt}%
\pgfpathmoveto{\pgfqpoint{3.482769in}{1.478798in}}%
\pgfpathlineto{\pgfqpoint{3.496610in}{1.476129in}}%
\pgfpathlineto{\pgfqpoint{3.510455in}{1.473653in}}%
\pgfpathlineto{\pgfqpoint{3.524307in}{1.471369in}}%
\pgfpathlineto{\pgfqpoint{3.538163in}{1.469277in}}%
\pgfpathlineto{\pgfqpoint{3.546437in}{1.479050in}}%
\pgfpathlineto{\pgfqpoint{3.554704in}{1.488942in}}%
\pgfpathlineto{\pgfqpoint{3.562964in}{1.498949in}}%
\pgfpathlineto{\pgfqpoint{3.571217in}{1.509064in}}%
\pgfpathlineto{\pgfqpoint{3.557376in}{1.510625in}}%
\pgfpathlineto{\pgfqpoint{3.543540in}{1.512377in}}%
\pgfpathlineto{\pgfqpoint{3.529711in}{1.514322in}}%
\pgfpathlineto{\pgfqpoint{3.515887in}{1.516459in}}%
\pgfpathlineto{\pgfqpoint{3.507618in}{1.506864in}}%
\pgfpathlineto{\pgfqpoint{3.499342in}{1.497385in}}%
\pgfpathlineto{\pgfqpoint{3.491059in}{1.488028in}}%
\pgfpathlineto{\pgfqpoint{3.482769in}{1.478798in}}%
\pgfpathclose%
\pgfusepath{fill}%
\end{pgfscope}%
\begin{pgfscope}%
\pgfpathrectangle{\pgfqpoint{1.150000in}{0.150000in}}{\pgfqpoint{5.700000in}{5.700000in}}%
\pgfusepath{clip}%
\pgfsetbuttcap%
\pgfsetroundjoin%
\definecolor{currentfill}{rgb}{0.214000,0.722114,0.469588}%
\pgfsetfillcolor{currentfill}%
\pgfsetfillopacity{0.800000}%
\pgfsetlinewidth{0.000000pt}%
\definecolor{currentstroke}{rgb}{0.000000,0.000000,0.000000}%
\pgfsetstrokecolor{currentstroke}%
\pgfsetdash{}{0pt}%
\pgfpathmoveto{\pgfqpoint{5.682853in}{3.394601in}}%
\pgfpathlineto{\pgfqpoint{5.697795in}{3.410104in}}%
\pgfpathlineto{\pgfqpoint{5.712760in}{3.425791in}}%
\pgfpathlineto{\pgfqpoint{5.727748in}{3.441663in}}%
\pgfpathlineto{\pgfqpoint{5.742758in}{3.457719in}}%
\pgfpathlineto{\pgfqpoint{5.750025in}{3.459151in}}%
\pgfpathlineto{\pgfqpoint{5.757281in}{3.460508in}}%
\pgfpathlineto{\pgfqpoint{5.764529in}{3.461795in}}%
\pgfpathlineto{\pgfqpoint{5.771767in}{3.463017in}}%
\pgfpathlineto{\pgfqpoint{5.756783in}{3.447437in}}%
\pgfpathlineto{\pgfqpoint{5.741821in}{3.432039in}}%
\pgfpathlineto{\pgfqpoint{5.726882in}{3.416826in}}%
\pgfpathlineto{\pgfqpoint{5.711966in}{3.401795in}}%
\pgfpathlineto{\pgfqpoint{5.704701in}{3.400086in}}%
\pgfpathlineto{\pgfqpoint{5.697427in}{3.398321in}}%
\pgfpathlineto{\pgfqpoint{5.690145in}{3.396495in}}%
\pgfpathlineto{\pgfqpoint{5.682853in}{3.394601in}}%
\pgfpathclose%
\pgfusepath{fill}%
\end{pgfscope}%
\begin{pgfscope}%
\pgfpathrectangle{\pgfqpoint{1.150000in}{0.150000in}}{\pgfqpoint{5.700000in}{5.700000in}}%
\pgfusepath{clip}%
\pgfsetbuttcap%
\pgfsetroundjoin%
\definecolor{currentfill}{rgb}{0.171176,0.452530,0.557965}%
\pgfsetfillcolor{currentfill}%
\pgfsetfillopacity{0.800000}%
\pgfsetlinewidth{0.000000pt}%
\definecolor{currentstroke}{rgb}{0.000000,0.000000,0.000000}%
\pgfsetstrokecolor{currentstroke}%
\pgfsetdash{}{0pt}%
\pgfpathmoveto{\pgfqpoint{2.151459in}{2.649155in}}%
\pgfpathlineto{\pgfqpoint{2.165837in}{2.622913in}}%
\pgfpathlineto{\pgfqpoint{2.180198in}{2.597010in}}%
\pgfpathlineto{\pgfqpoint{2.194541in}{2.571444in}}%
\pgfpathlineto{\pgfqpoint{2.208868in}{2.546212in}}%
\pgfpathlineto{\pgfqpoint{2.218172in}{2.539598in}}%
\pgfpathlineto{\pgfqpoint{2.227450in}{2.533367in}}%
\pgfpathlineto{\pgfqpoint{2.236703in}{2.527513in}}%
\pgfpathlineto{\pgfqpoint{2.245933in}{2.522027in}}%
\pgfpathlineto{\pgfqpoint{2.231671in}{2.546566in}}%
\pgfpathlineto{\pgfqpoint{2.217393in}{2.571437in}}%
\pgfpathlineto{\pgfqpoint{2.203099in}{2.596642in}}%
\pgfpathlineto{\pgfqpoint{2.188788in}{2.622185in}}%
\pgfpathlineto{\pgfqpoint{2.179494in}{2.628352in}}%
\pgfpathlineto{\pgfqpoint{2.170175in}{2.634897in}}%
\pgfpathlineto{\pgfqpoint{2.160830in}{2.641829in}}%
\pgfpathlineto{\pgfqpoint{2.151459in}{2.649155in}}%
\pgfpathclose%
\pgfusepath{fill}%
\end{pgfscope}%
\begin{pgfscope}%
\pgfpathrectangle{\pgfqpoint{1.150000in}{0.150000in}}{\pgfqpoint{5.700000in}{5.700000in}}%
\pgfusepath{clip}%
\pgfsetbuttcap%
\pgfsetroundjoin%
\definecolor{currentfill}{rgb}{0.283197,0.115680,0.436115}%
\pgfsetfillcolor{currentfill}%
\pgfsetfillopacity{0.800000}%
\pgfsetlinewidth{0.000000pt}%
\definecolor{currentstroke}{rgb}{0.000000,0.000000,0.000000}%
\pgfsetstrokecolor{currentstroke}%
\pgfsetdash{}{0pt}%
\pgfpathmoveto{\pgfqpoint{2.792665in}{1.724637in}}%
\pgfpathlineto{\pgfqpoint{2.806605in}{1.711113in}}%
\pgfpathlineto{\pgfqpoint{2.820541in}{1.697821in}}%
\pgfpathlineto{\pgfqpoint{2.834473in}{1.684759in}}%
\pgfpathlineto{\pgfqpoint{2.848401in}{1.671927in}}%
\pgfpathlineto{\pgfqpoint{2.857139in}{1.672255in}}%
\pgfpathlineto{\pgfqpoint{2.865860in}{1.672880in}}%
\pgfpathlineto{\pgfqpoint{2.874566in}{1.673797in}}%
\pgfpathlineto{\pgfqpoint{2.883255in}{1.674998in}}%
\pgfpathlineto{\pgfqpoint{2.869369in}{1.687156in}}%
\pgfpathlineto{\pgfqpoint{2.855480in}{1.699544in}}%
\pgfpathlineto{\pgfqpoint{2.841587in}{1.712161in}}%
\pgfpathlineto{\pgfqpoint{2.827690in}{1.725010in}}%
\pgfpathlineto{\pgfqpoint{2.818959in}{1.724470in}}%
\pgfpathlineto{\pgfqpoint{2.810211in}{1.724223in}}%
\pgfpathlineto{\pgfqpoint{2.801447in}{1.724277in}}%
\pgfpathlineto{\pgfqpoint{2.792665in}{1.724637in}}%
\pgfpathclose%
\pgfusepath{fill}%
\end{pgfscope}%
\begin{pgfscope}%
\pgfpathrectangle{\pgfqpoint{1.150000in}{0.150000in}}{\pgfqpoint{5.700000in}{5.700000in}}%
\pgfusepath{clip}%
\pgfsetbuttcap%
\pgfsetroundjoin%
\definecolor{currentfill}{rgb}{0.269308,0.218818,0.509577}%
\pgfsetfillcolor{currentfill}%
\pgfsetfillopacity{0.800000}%
\pgfsetlinewidth{0.000000pt}%
\definecolor{currentstroke}{rgb}{0.000000,0.000000,0.000000}%
\pgfsetstrokecolor{currentstroke}%
\pgfsetdash{}{0pt}%
\pgfpathmoveto{\pgfqpoint{4.133035in}{1.905995in}}%
\pgfpathlineto{\pgfqpoint{4.147052in}{1.911965in}}%
\pgfpathlineto{\pgfqpoint{4.161082in}{1.918119in}}%
\pgfpathlineto{\pgfqpoint{4.175124in}{1.924457in}}%
\pgfpathlineto{\pgfqpoint{4.189179in}{1.930980in}}%
\pgfpathlineto{\pgfqpoint{4.197235in}{1.944486in}}%
\pgfpathlineto{\pgfqpoint{4.205287in}{1.957933in}}%
\pgfpathlineto{\pgfqpoint{4.213334in}{1.971317in}}%
\pgfpathlineto{\pgfqpoint{4.221376in}{1.984638in}}%
\pgfpathlineto{\pgfqpoint{4.207323in}{1.977832in}}%
\pgfpathlineto{\pgfqpoint{4.193282in}{1.971210in}}%
\pgfpathlineto{\pgfqpoint{4.179254in}{1.964774in}}%
\pgfpathlineto{\pgfqpoint{4.165238in}{1.958522in}}%
\pgfpathlineto{\pgfqpoint{4.157194in}{1.945472in}}%
\pgfpathlineto{\pgfqpoint{4.149146in}{1.932366in}}%
\pgfpathlineto{\pgfqpoint{4.141093in}{1.919206in}}%
\pgfpathlineto{\pgfqpoint{4.133035in}{1.905995in}}%
\pgfpathclose%
\pgfusepath{fill}%
\end{pgfscope}%
\begin{pgfscope}%
\pgfpathrectangle{\pgfqpoint{1.150000in}{0.150000in}}{\pgfqpoint{5.700000in}{5.700000in}}%
\pgfusepath{clip}%
\pgfsetbuttcap%
\pgfsetroundjoin%
\definecolor{currentfill}{rgb}{0.267004,0.004874,0.329415}%
\pgfsetfillcolor{currentfill}%
\pgfsetfillopacity{0.800000}%
\pgfsetlinewidth{0.000000pt}%
\definecolor{currentstroke}{rgb}{0.000000,0.000000,0.000000}%
\pgfsetstrokecolor{currentstroke}%
\pgfsetdash{}{0pt}%
\pgfpathmoveto{\pgfqpoint{3.249877in}{1.471359in}}%
\pgfpathlineto{\pgfqpoint{3.263719in}{1.465170in}}%
\pgfpathlineto{\pgfqpoint{3.277562in}{1.459182in}}%
\pgfpathlineto{\pgfqpoint{3.291408in}{1.453393in}}%
\pgfpathlineto{\pgfqpoint{3.305257in}{1.447805in}}%
\pgfpathlineto{\pgfqpoint{3.313658in}{1.454607in}}%
\pgfpathlineto{\pgfqpoint{3.322050in}{1.461597in}}%
\pgfpathlineto{\pgfqpoint{3.330432in}{1.468769in}}%
\pgfpathlineto{\pgfqpoint{3.338804in}{1.476116in}}%
\pgfpathlineto{\pgfqpoint{3.324980in}{1.481110in}}%
\pgfpathlineto{\pgfqpoint{3.311158in}{1.486302in}}%
\pgfpathlineto{\pgfqpoint{3.297339in}{1.491695in}}%
\pgfpathlineto{\pgfqpoint{3.283523in}{1.497288in}}%
\pgfpathlineto{\pgfqpoint{3.275127in}{1.490524in}}%
\pgfpathlineto{\pgfqpoint{3.266720in}{1.483944in}}%
\pgfpathlineto{\pgfqpoint{3.258304in}{1.477553in}}%
\pgfpathlineto{\pgfqpoint{3.249877in}{1.471359in}}%
\pgfpathclose%
\pgfusepath{fill}%
\end{pgfscope}%
\begin{pgfscope}%
\pgfpathrectangle{\pgfqpoint{1.150000in}{0.150000in}}{\pgfqpoint{5.700000in}{5.700000in}}%
\pgfusepath{clip}%
\pgfsetbuttcap%
\pgfsetroundjoin%
\definecolor{currentfill}{rgb}{0.237441,0.305202,0.541921}%
\pgfsetfillcolor{currentfill}%
\pgfsetfillopacity{0.800000}%
\pgfsetlinewidth{0.000000pt}%
\definecolor{currentstroke}{rgb}{0.000000,0.000000,0.000000}%
\pgfsetstrokecolor{currentstroke}%
\pgfsetdash{}{0pt}%
\pgfpathmoveto{\pgfqpoint{2.399674in}{2.204269in}}%
\pgfpathlineto{\pgfqpoint{2.413830in}{2.183546in}}%
\pgfpathlineto{\pgfqpoint{2.427974in}{2.163107in}}%
\pgfpathlineto{\pgfqpoint{2.442108in}{2.142951in}}%
\pgfpathlineto{\pgfqpoint{2.456231in}{2.123074in}}%
\pgfpathlineto{\pgfqpoint{2.465320in}{2.118509in}}%
\pgfpathlineto{\pgfqpoint{2.474387in}{2.114306in}}%
\pgfpathlineto{\pgfqpoint{2.483433in}{2.110458in}}%
\pgfpathlineto{\pgfqpoint{2.492456in}{2.106959in}}%
\pgfpathlineto{\pgfqpoint{2.478390in}{2.126135in}}%
\pgfpathlineto{\pgfqpoint{2.464314in}{2.145588in}}%
\pgfpathlineto{\pgfqpoint{2.450228in}{2.165323in}}%
\pgfpathlineto{\pgfqpoint{2.436130in}{2.185341in}}%
\pgfpathlineto{\pgfqpoint{2.427050in}{2.189529in}}%
\pgfpathlineto{\pgfqpoint{2.417947in}{2.194075in}}%
\pgfpathlineto{\pgfqpoint{2.408822in}{2.198986in}}%
\pgfpathlineto{\pgfqpoint{2.399674in}{2.204269in}}%
\pgfpathclose%
\pgfusepath{fill}%
\end{pgfscope}%
\begin{pgfscope}%
\pgfpathrectangle{\pgfqpoint{1.150000in}{0.150000in}}{\pgfqpoint{5.700000in}{5.700000in}}%
\pgfusepath{clip}%
\pgfsetbuttcap%
\pgfsetroundjoin%
\definecolor{currentfill}{rgb}{0.179019,0.433756,0.557430}%
\pgfsetfillcolor{currentfill}%
\pgfsetfillopacity{0.800000}%
\pgfsetlinewidth{0.000000pt}%
\definecolor{currentstroke}{rgb}{0.000000,0.000000,0.000000}%
\pgfsetstrokecolor{currentstroke}%
\pgfsetdash{}{0pt}%
\pgfpathmoveto{\pgfqpoint{4.671188in}{2.475506in}}%
\pgfpathlineto{\pgfqpoint{4.685488in}{2.486521in}}%
\pgfpathlineto{\pgfqpoint{4.699805in}{2.497721in}}%
\pgfpathlineto{\pgfqpoint{4.714139in}{2.509106in}}%
\pgfpathlineto{\pgfqpoint{4.728490in}{2.520676in}}%
\pgfpathlineto{\pgfqpoint{4.736369in}{2.531628in}}%
\pgfpathlineto{\pgfqpoint{4.744242in}{2.542440in}}%
\pgfpathlineto{\pgfqpoint{4.752107in}{2.553114in}}%
\pgfpathlineto{\pgfqpoint{4.759966in}{2.563649in}}%
\pgfpathlineto{\pgfqpoint{4.745617in}{2.552060in}}%
\pgfpathlineto{\pgfqpoint{4.731285in}{2.540655in}}%
\pgfpathlineto{\pgfqpoint{4.716970in}{2.529435in}}%
\pgfpathlineto{\pgfqpoint{4.702672in}{2.518400in}}%
\pgfpathlineto{\pgfqpoint{4.694811in}{2.507872in}}%
\pgfpathlineto{\pgfqpoint{4.686943in}{2.497214in}}%
\pgfpathlineto{\pgfqpoint{4.679069in}{2.486425in}}%
\pgfpathlineto{\pgfqpoint{4.671188in}{2.475506in}}%
\pgfpathclose%
\pgfusepath{fill}%
\end{pgfscope}%
\begin{pgfscope}%
\pgfpathrectangle{\pgfqpoint{1.150000in}{0.150000in}}{\pgfqpoint{5.700000in}{5.700000in}}%
\pgfusepath{clip}%
\pgfsetbuttcap%
\pgfsetroundjoin%
\definecolor{currentfill}{rgb}{0.126453,0.570633,0.549841}%
\pgfsetfillcolor{currentfill}%
\pgfsetfillopacity{0.800000}%
\pgfsetlinewidth{0.000000pt}%
\definecolor{currentstroke}{rgb}{0.000000,0.000000,0.000000}%
\pgfsetstrokecolor{currentstroke}%
\pgfsetdash{}{0pt}%
\pgfpathmoveto{\pgfqpoint{5.088916in}{2.895019in}}%
\pgfpathlineto{\pgfqpoint{5.103479in}{2.908595in}}%
\pgfpathlineto{\pgfqpoint{5.118061in}{2.922356in}}%
\pgfpathlineto{\pgfqpoint{5.132663in}{2.936302in}}%
\pgfpathlineto{\pgfqpoint{5.147284in}{2.950433in}}%
\pgfpathlineto{\pgfqpoint{5.154956in}{2.957590in}}%
\pgfpathlineto{\pgfqpoint{5.162619in}{2.964603in}}%
\pgfpathlineto{\pgfqpoint{5.170274in}{2.971475in}}%
\pgfpathlineto{\pgfqpoint{5.177919in}{2.978208in}}%
\pgfpathlineto{\pgfqpoint{5.163307in}{2.964266in}}%
\pgfpathlineto{\pgfqpoint{5.148714in}{2.950509in}}%
\pgfpathlineto{\pgfqpoint{5.134142in}{2.936937in}}%
\pgfpathlineto{\pgfqpoint{5.119588in}{2.923549in}}%
\pgfpathlineto{\pgfqpoint{5.111933in}{2.916615in}}%
\pgfpathlineto{\pgfqpoint{5.104269in}{2.909550in}}%
\pgfpathlineto{\pgfqpoint{5.096597in}{2.902352in}}%
\pgfpathlineto{\pgfqpoint{5.088916in}{2.895019in}}%
\pgfpathclose%
\pgfusepath{fill}%
\end{pgfscope}%
\begin{pgfscope}%
\pgfpathrectangle{\pgfqpoint{1.150000in}{0.150000in}}{\pgfqpoint{5.700000in}{5.700000in}}%
\pgfusepath{clip}%
\pgfsetbuttcap%
\pgfsetroundjoin%
\definecolor{currentfill}{rgb}{0.282884,0.135920,0.453427}%
\pgfsetfillcolor{currentfill}%
\pgfsetfillopacity{0.800000}%
\pgfsetlinewidth{0.000000pt}%
\definecolor{currentstroke}{rgb}{0.000000,0.000000,0.000000}%
\pgfsetstrokecolor{currentstroke}%
\pgfsetdash{}{0pt}%
\pgfpathmoveto{\pgfqpoint{3.924187in}{1.711685in}}%
\pgfpathlineto{\pgfqpoint{3.938125in}{1.715137in}}%
\pgfpathlineto{\pgfqpoint{3.952074in}{1.718774in}}%
\pgfpathlineto{\pgfqpoint{3.966033in}{1.722596in}}%
\pgfpathlineto{\pgfqpoint{3.980003in}{1.726603in}}%
\pgfpathlineto{\pgfqpoint{3.988118in}{1.739816in}}%
\pgfpathlineto{\pgfqpoint{3.996229in}{1.753022in}}%
\pgfpathlineto{\pgfqpoint{4.004335in}{1.766217in}}%
\pgfpathlineto{\pgfqpoint{4.012437in}{1.779398in}}%
\pgfpathlineto{\pgfqpoint{3.998471in}{1.775014in}}%
\pgfpathlineto{\pgfqpoint{3.984516in}{1.770814in}}%
\pgfpathlineto{\pgfqpoint{3.970572in}{1.766800in}}%
\pgfpathlineto{\pgfqpoint{3.956638in}{1.762972in}}%
\pgfpathlineto{\pgfqpoint{3.948532in}{1.750156in}}%
\pgfpathlineto{\pgfqpoint{3.940422in}{1.737334in}}%
\pgfpathlineto{\pgfqpoint{3.932307in}{1.724509in}}%
\pgfpathlineto{\pgfqpoint{3.924187in}{1.711685in}}%
\pgfpathclose%
\pgfusepath{fill}%
\end{pgfscope}%
\begin{pgfscope}%
\pgfpathrectangle{\pgfqpoint{1.150000in}{0.150000in}}{\pgfqpoint{5.700000in}{5.700000in}}%
\pgfusepath{clip}%
\pgfsetbuttcap%
\pgfsetroundjoin%
\definecolor{currentfill}{rgb}{0.150476,0.504369,0.557430}%
\pgfsetfillcolor{currentfill}%
\pgfsetfillopacity{0.800000}%
\pgfsetlinewidth{0.000000pt}%
\definecolor{currentstroke}{rgb}{0.000000,0.000000,0.000000}%
\pgfsetstrokecolor{currentstroke}%
\pgfsetdash{}{0pt}%
\pgfpathmoveto{\pgfqpoint{4.880126in}{2.690964in}}%
\pgfpathlineto{\pgfqpoint{4.894557in}{2.703415in}}%
\pgfpathlineto{\pgfqpoint{4.909006in}{2.716051in}}%
\pgfpathlineto{\pgfqpoint{4.923473in}{2.728872in}}%
\pgfpathlineto{\pgfqpoint{4.937959in}{2.741878in}}%
\pgfpathlineto{\pgfqpoint{4.945745in}{2.751054in}}%
\pgfpathlineto{\pgfqpoint{4.953522in}{2.760082in}}%
\pgfpathlineto{\pgfqpoint{4.961292in}{2.768964in}}%
\pgfpathlineto{\pgfqpoint{4.969054in}{2.777700in}}%
\pgfpathlineto{\pgfqpoint{4.954573in}{2.764778in}}%
\pgfpathlineto{\pgfqpoint{4.940110in}{2.752041in}}%
\pgfpathlineto{\pgfqpoint{4.925667in}{2.739489in}}%
\pgfpathlineto{\pgfqpoint{4.911241in}{2.727121in}}%
\pgfpathlineto{\pgfqpoint{4.903474in}{2.718288in}}%
\pgfpathlineto{\pgfqpoint{4.895699in}{2.709318in}}%
\pgfpathlineto{\pgfqpoint{4.887916in}{2.700211in}}%
\pgfpathlineto{\pgfqpoint{4.880126in}{2.690964in}}%
\pgfpathclose%
\pgfusepath{fill}%
\end{pgfscope}%
\begin{pgfscope}%
\pgfpathrectangle{\pgfqpoint{1.150000in}{0.150000in}}{\pgfqpoint{5.700000in}{5.700000in}}%
\pgfusepath{clip}%
\pgfsetbuttcap%
\pgfsetroundjoin%
\definecolor{currentfill}{rgb}{0.235526,0.309527,0.542944}%
\pgfsetfillcolor{currentfill}%
\pgfsetfillopacity{0.800000}%
\pgfsetlinewidth{0.000000pt}%
\definecolor{currentstroke}{rgb}{0.000000,0.000000,0.000000}%
\pgfsetstrokecolor{currentstroke}%
\pgfsetdash{}{0pt}%
\pgfpathmoveto{\pgfqpoint{4.341883in}{2.119574in}}%
\pgfpathlineto{\pgfqpoint{4.356003in}{2.127772in}}%
\pgfpathlineto{\pgfqpoint{4.370136in}{2.136155in}}%
\pgfpathlineto{\pgfqpoint{4.384284in}{2.144722in}}%
\pgfpathlineto{\pgfqpoint{4.398446in}{2.153474in}}%
\pgfpathlineto{\pgfqpoint{4.406444in}{2.166518in}}%
\pgfpathlineto{\pgfqpoint{4.414437in}{2.179462in}}%
\pgfpathlineto{\pgfqpoint{4.422424in}{2.192305in}}%
\pgfpathlineto{\pgfqpoint{4.430407in}{2.205044in}}%
\pgfpathlineto{\pgfqpoint{4.416245in}{2.196106in}}%
\pgfpathlineto{\pgfqpoint{4.402097in}{2.187352in}}%
\pgfpathlineto{\pgfqpoint{4.387964in}{2.178783in}}%
\pgfpathlineto{\pgfqpoint{4.373845in}{2.170398in}}%
\pgfpathlineto{\pgfqpoint{4.365863in}{2.157833in}}%
\pgfpathlineto{\pgfqpoint{4.357875in}{2.145173in}}%
\pgfpathlineto{\pgfqpoint{4.349882in}{2.132419in}}%
\pgfpathlineto{\pgfqpoint{4.341883in}{2.119574in}}%
\pgfpathclose%
\pgfusepath{fill}%
\end{pgfscope}%
\begin{pgfscope}%
\pgfpathrectangle{\pgfqpoint{1.150000in}{0.150000in}}{\pgfqpoint{5.700000in}{5.700000in}}%
\pgfusepath{clip}%
\pgfsetbuttcap%
\pgfsetroundjoin%
\definecolor{currentfill}{rgb}{0.259857,0.745492,0.444467}%
\pgfsetfillcolor{currentfill}%
\pgfsetfillopacity{0.800000}%
\pgfsetlinewidth{0.000000pt}%
\definecolor{currentstroke}{rgb}{0.000000,0.000000,0.000000}%
\pgfsetstrokecolor{currentstroke}%
\pgfsetdash{}{0pt}%
\pgfpathmoveto{\pgfqpoint{5.771767in}{3.463017in}}%
\pgfpathlineto{\pgfqpoint{5.786774in}{3.478782in}}%
\pgfpathlineto{\pgfqpoint{5.801805in}{3.494731in}}%
\pgfpathlineto{\pgfqpoint{5.816858in}{3.510864in}}%
\pgfpathlineto{\pgfqpoint{5.831936in}{3.527183in}}%
\pgfpathlineto{\pgfqpoint{5.839137in}{3.527848in}}%
\pgfpathlineto{\pgfqpoint{5.846328in}{3.528451in}}%
\pgfpathlineto{\pgfqpoint{5.853511in}{3.528998in}}%
\pgfpathlineto{\pgfqpoint{5.860685in}{3.529494in}}%
\pgfpathlineto{\pgfqpoint{5.845637in}{3.513688in}}%
\pgfpathlineto{\pgfqpoint{5.830612in}{3.498066in}}%
\pgfpathlineto{\pgfqpoint{5.815610in}{3.482627in}}%
\pgfpathlineto{\pgfqpoint{5.800631in}{3.467371in}}%
\pgfpathlineto{\pgfqpoint{5.793428in}{3.466352in}}%
\pgfpathlineto{\pgfqpoint{5.786216in}{3.465291in}}%
\pgfpathlineto{\pgfqpoint{5.778996in}{3.464181in}}%
\pgfpathlineto{\pgfqpoint{5.771767in}{3.463017in}}%
\pgfpathclose%
\pgfusepath{fill}%
\end{pgfscope}%
\begin{pgfscope}%
\pgfpathrectangle{\pgfqpoint{1.150000in}{0.150000in}}{\pgfqpoint{5.700000in}{5.700000in}}%
\pgfusepath{clip}%
\pgfsetbuttcap%
\pgfsetroundjoin%
\definecolor{currentfill}{rgb}{0.267004,0.004874,0.329415}%
\pgfsetfillcolor{currentfill}%
\pgfsetfillopacity{0.800000}%
\pgfsetlinewidth{0.000000pt}%
\definecolor{currentstroke}{rgb}{0.000000,0.000000,0.000000}%
\pgfsetstrokecolor{currentstroke}%
\pgfsetdash{}{0pt}%
\pgfpathmoveto{\pgfqpoint{3.394136in}{1.458121in}}%
\pgfpathlineto{\pgfqpoint{3.407978in}{1.454112in}}%
\pgfpathlineto{\pgfqpoint{3.421824in}{1.450299in}}%
\pgfpathlineto{\pgfqpoint{3.435674in}{1.446681in}}%
\pgfpathlineto{\pgfqpoint{3.449529in}{1.443256in}}%
\pgfpathlineto{\pgfqpoint{3.457851in}{1.451924in}}%
\pgfpathlineto{\pgfqpoint{3.466165in}{1.460741in}}%
\pgfpathlineto{\pgfqpoint{3.474471in}{1.469701in}}%
\pgfpathlineto{\pgfqpoint{3.482769in}{1.478798in}}%
\pgfpathlineto{\pgfqpoint{3.468933in}{1.481660in}}%
\pgfpathlineto{\pgfqpoint{3.455102in}{1.484715in}}%
\pgfpathlineto{\pgfqpoint{3.441275in}{1.487965in}}%
\pgfpathlineto{\pgfqpoint{3.427453in}{1.491410in}}%
\pgfpathlineto{\pgfqpoint{3.419136in}{1.482864in}}%
\pgfpathlineto{\pgfqpoint{3.410811in}{1.474463in}}%
\pgfpathlineto{\pgfqpoint{3.402478in}{1.466214in}}%
\pgfpathlineto{\pgfqpoint{3.394136in}{1.458121in}}%
\pgfpathclose%
\pgfusepath{fill}%
\end{pgfscope}%
\begin{pgfscope}%
\pgfpathrectangle{\pgfqpoint{1.150000in}{0.150000in}}{\pgfqpoint{5.700000in}{5.700000in}}%
\pgfusepath{clip}%
\pgfsetbuttcap%
\pgfsetroundjoin%
\definecolor{currentfill}{rgb}{0.282327,0.094955,0.417331}%
\pgfsetfillcolor{currentfill}%
\pgfsetfillopacity{0.800000}%
\pgfsetlinewidth{0.000000pt}%
\definecolor{currentstroke}{rgb}{0.000000,0.000000,0.000000}%
\pgfsetstrokecolor{currentstroke}%
\pgfsetdash{}{0pt}%
\pgfpathmoveto{\pgfqpoint{2.848401in}{1.671927in}}%
\pgfpathlineto{\pgfqpoint{2.862326in}{1.659324in}}%
\pgfpathlineto{\pgfqpoint{2.876247in}{1.646946in}}%
\pgfpathlineto{\pgfqpoint{2.890165in}{1.634795in}}%
\pgfpathlineto{\pgfqpoint{2.904080in}{1.622867in}}%
\pgfpathlineto{\pgfqpoint{2.912776in}{1.623879in}}%
\pgfpathlineto{\pgfqpoint{2.921456in}{1.625181in}}%
\pgfpathlineto{\pgfqpoint{2.930121in}{1.626764in}}%
\pgfpathlineto{\pgfqpoint{2.938771in}{1.628624in}}%
\pgfpathlineto{\pgfqpoint{2.924896in}{1.639881in}}%
\pgfpathlineto{\pgfqpoint{2.911019in}{1.651361in}}%
\pgfpathlineto{\pgfqpoint{2.897138in}{1.663066in}}%
\pgfpathlineto{\pgfqpoint{2.883255in}{1.674998in}}%
\pgfpathlineto{\pgfqpoint{2.874566in}{1.673797in}}%
\pgfpathlineto{\pgfqpoint{2.865860in}{1.672880in}}%
\pgfpathlineto{\pgfqpoint{2.857139in}{1.672255in}}%
\pgfpathlineto{\pgfqpoint{2.848401in}{1.671927in}}%
\pgfpathclose%
\pgfusepath{fill}%
\end{pgfscope}%
\begin{pgfscope}%
\pgfpathrectangle{\pgfqpoint{1.150000in}{0.150000in}}{\pgfqpoint{5.700000in}{5.700000in}}%
\pgfusepath{clip}%
\pgfsetbuttcap%
\pgfsetroundjoin%
\definecolor{currentfill}{rgb}{0.271305,0.019942,0.347269}%
\pgfsetfillcolor{currentfill}%
\pgfsetfillopacity{0.800000}%
\pgfsetlinewidth{0.000000pt}%
\definecolor{currentstroke}{rgb}{0.000000,0.000000,0.000000}%
\pgfsetstrokecolor{currentstroke}%
\pgfsetdash{}{0pt}%
\pgfpathmoveto{\pgfqpoint{3.105155in}{1.510541in}}%
\pgfpathlineto{\pgfqpoint{3.119016in}{1.502084in}}%
\pgfpathlineto{\pgfqpoint{3.132878in}{1.493835in}}%
\pgfpathlineto{\pgfqpoint{3.146740in}{1.485794in}}%
\pgfpathlineto{\pgfqpoint{3.160603in}{1.477958in}}%
\pgfpathlineto{\pgfqpoint{3.169102in}{1.482667in}}%
\pgfpathlineto{\pgfqpoint{3.177589in}{1.487604in}}%
\pgfpathlineto{\pgfqpoint{3.186065in}{1.492764in}}%
\pgfpathlineto{\pgfqpoint{3.194530in}{1.498140in}}%
\pgfpathlineto{\pgfqpoint{3.180697in}{1.505346in}}%
\pgfpathlineto{\pgfqpoint{3.166865in}{1.512757in}}%
\pgfpathlineto{\pgfqpoint{3.153034in}{1.520376in}}%
\pgfpathlineto{\pgfqpoint{3.139204in}{1.528202in}}%
\pgfpathlineto{\pgfqpoint{3.130709in}{1.523444in}}%
\pgfpathlineto{\pgfqpoint{3.122203in}{1.518910in}}%
\pgfpathlineto{\pgfqpoint{3.113685in}{1.514607in}}%
\pgfpathlineto{\pgfqpoint{3.105155in}{1.510541in}}%
\pgfpathclose%
\pgfusepath{fill}%
\end{pgfscope}%
\begin{pgfscope}%
\pgfpathrectangle{\pgfqpoint{1.150000in}{0.150000in}}{\pgfqpoint{5.700000in}{5.700000in}}%
\pgfusepath{clip}%
\pgfsetbuttcap%
\pgfsetroundjoin%
\definecolor{currentfill}{rgb}{0.132268,0.655014,0.519661}%
\pgfsetfillcolor{currentfill}%
\pgfsetfillopacity{0.800000}%
\pgfsetlinewidth{0.000000pt}%
\definecolor{currentstroke}{rgb}{0.000000,0.000000,0.000000}%
\pgfsetstrokecolor{currentstroke}%
\pgfsetdash{}{0pt}%
\pgfpathmoveto{\pgfqpoint{5.386307in}{3.161159in}}%
\pgfpathlineto{\pgfqpoint{5.401066in}{3.175987in}}%
\pgfpathlineto{\pgfqpoint{5.415847in}{3.190999in}}%
\pgfpathlineto{\pgfqpoint{5.430649in}{3.206197in}}%
\pgfpathlineto{\pgfqpoint{5.445473in}{3.221580in}}%
\pgfpathlineto{\pgfqpoint{5.452958in}{3.225820in}}%
\pgfpathlineto{\pgfqpoint{5.460433in}{3.229940in}}%
\pgfpathlineto{\pgfqpoint{5.467899in}{3.233943in}}%
\pgfpathlineto{\pgfqpoint{5.475355in}{3.237832in}}%
\pgfpathlineto{\pgfqpoint{5.460549in}{3.222781in}}%
\pgfpathlineto{\pgfqpoint{5.445763in}{3.207915in}}%
\pgfpathlineto{\pgfqpoint{5.430999in}{3.193234in}}%
\pgfpathlineto{\pgfqpoint{5.416257in}{3.178736in}}%
\pgfpathlineto{\pgfqpoint{5.408783in}{3.174503in}}%
\pgfpathlineto{\pgfqpoint{5.401300in}{3.170165in}}%
\pgfpathlineto{\pgfqpoint{5.393808in}{3.165718in}}%
\pgfpathlineto{\pgfqpoint{5.386307in}{3.161159in}}%
\pgfpathclose%
\pgfusepath{fill}%
\end{pgfscope}%
\begin{pgfscope}%
\pgfpathrectangle{\pgfqpoint{1.150000in}{0.150000in}}{\pgfqpoint{5.700000in}{5.700000in}}%
\pgfusepath{clip}%
\pgfsetbuttcap%
\pgfsetroundjoin%
\definecolor{currentfill}{rgb}{0.223925,0.334994,0.548053}%
\pgfsetfillcolor{currentfill}%
\pgfsetfillopacity{0.800000}%
\pgfsetlinewidth{0.000000pt}%
\definecolor{currentstroke}{rgb}{0.000000,0.000000,0.000000}%
\pgfsetstrokecolor{currentstroke}%
\pgfsetdash{}{0pt}%
\pgfpathmoveto{\pgfqpoint{2.342931in}{2.290056in}}%
\pgfpathlineto{\pgfqpoint{2.357135in}{2.268170in}}%
\pgfpathlineto{\pgfqpoint{2.371327in}{2.246579in}}%
\pgfpathlineto{\pgfqpoint{2.385506in}{2.225279in}}%
\pgfpathlineto{\pgfqpoint{2.399674in}{2.204269in}}%
\pgfpathlineto{\pgfqpoint{2.408822in}{2.198986in}}%
\pgfpathlineto{\pgfqpoint{2.417947in}{2.194075in}}%
\pgfpathlineto{\pgfqpoint{2.427050in}{2.189529in}}%
\pgfpathlineto{\pgfqpoint{2.436130in}{2.185341in}}%
\pgfpathlineto{\pgfqpoint{2.422022in}{2.205645in}}%
\pgfpathlineto{\pgfqpoint{2.407902in}{2.226236in}}%
\pgfpathlineto{\pgfqpoint{2.393771in}{2.247118in}}%
\pgfpathlineto{\pgfqpoint{2.379628in}{2.268292in}}%
\pgfpathlineto{\pgfqpoint{2.370489in}{2.273175in}}%
\pgfpathlineto{\pgfqpoint{2.361327in}{2.278425in}}%
\pgfpathlineto{\pgfqpoint{2.352141in}{2.284050in}}%
\pgfpathlineto{\pgfqpoint{2.342931in}{2.290056in}}%
\pgfpathclose%
\pgfusepath{fill}%
\end{pgfscope}%
\begin{pgfscope}%
\pgfpathrectangle{\pgfqpoint{1.150000in}{0.150000in}}{\pgfqpoint{5.700000in}{5.700000in}}%
\pgfusepath{clip}%
\pgfsetbuttcap%
\pgfsetroundjoin%
\definecolor{currentfill}{rgb}{0.197636,0.391528,0.554969}%
\pgfsetfillcolor{currentfill}%
\pgfsetfillopacity{0.800000}%
\pgfsetlinewidth{0.000000pt}%
\definecolor{currentstroke}{rgb}{0.000000,0.000000,0.000000}%
\pgfsetstrokecolor{currentstroke}%
\pgfsetdash{}{0pt}%
\pgfpathmoveto{\pgfqpoint{4.550865in}{2.341802in}}%
\pgfpathlineto{\pgfqpoint{4.565103in}{2.351937in}}%
\pgfpathlineto{\pgfqpoint{4.579357in}{2.362258in}}%
\pgfpathlineto{\pgfqpoint{4.593626in}{2.372763in}}%
\pgfpathlineto{\pgfqpoint{4.607912in}{2.383452in}}%
\pgfpathlineto{\pgfqpoint{4.615843in}{2.395414in}}%
\pgfpathlineto{\pgfqpoint{4.623768in}{2.407246in}}%
\pgfpathlineto{\pgfqpoint{4.631687in}{2.418949in}}%
\pgfpathlineto{\pgfqpoint{4.639600in}{2.430521in}}%
\pgfpathlineto{\pgfqpoint{4.625315in}{2.419744in}}%
\pgfpathlineto{\pgfqpoint{4.611046in}{2.409152in}}%
\pgfpathlineto{\pgfqpoint{4.596794in}{2.398745in}}%
\pgfpathlineto{\pgfqpoint{4.582557in}{2.388522in}}%
\pgfpathlineto{\pgfqpoint{4.574643in}{2.377024in}}%
\pgfpathlineto{\pgfqpoint{4.566723in}{2.365404in}}%
\pgfpathlineto{\pgfqpoint{4.558797in}{2.353663in}}%
\pgfpathlineto{\pgfqpoint{4.550865in}{2.341802in}}%
\pgfpathclose%
\pgfusepath{fill}%
\end{pgfscope}%
\begin{pgfscope}%
\pgfpathrectangle{\pgfqpoint{1.150000in}{0.150000in}}{\pgfqpoint{5.700000in}{5.700000in}}%
\pgfusepath{clip}%
\pgfsetbuttcap%
\pgfsetroundjoin%
\definecolor{currentfill}{rgb}{0.344074,0.780029,0.397381}%
\pgfsetfillcolor{currentfill}%
\pgfsetfillopacity{0.800000}%
\pgfsetlinewidth{0.000000pt}%
\definecolor{currentstroke}{rgb}{0.000000,0.000000,0.000000}%
\pgfsetstrokecolor{currentstroke}%
\pgfsetdash{}{0pt}%
\pgfpathmoveto{\pgfqpoint{5.949594in}{3.593965in}}%
\pgfpathlineto{\pgfqpoint{5.964729in}{3.610143in}}%
\pgfpathlineto{\pgfqpoint{5.979887in}{3.626505in}}%
\pgfpathlineto{\pgfqpoint{5.995070in}{3.643052in}}%
\pgfpathlineto{\pgfqpoint{6.002143in}{3.642386in}}%
\pgfpathlineto{\pgfqpoint{6.009208in}{3.641692in}}%
\pgfpathlineto{\pgfqpoint{6.016264in}{3.640975in}}%
\pgfpathlineto{\pgfqpoint{6.023313in}{3.640242in}}%
\pgfpathlineto{\pgfqpoint{6.008165in}{3.624280in}}%
\pgfpathlineto{\pgfqpoint{5.993041in}{3.608501in}}%
\pgfpathlineto{\pgfqpoint{5.977940in}{3.592904in}}%
\pgfpathlineto{\pgfqpoint{5.970866in}{3.593192in}}%
\pgfpathlineto{\pgfqpoint{5.963783in}{3.593469in}}%
\pgfpathlineto{\pgfqpoint{5.956693in}{3.593729in}}%
\pgfpathlineto{\pgfqpoint{5.949594in}{3.593965in}}%
\pgfpathclose%
\pgfusepath{fill}%
\end{pgfscope}%
\begin{pgfscope}%
\pgfpathrectangle{\pgfqpoint{1.150000in}{0.150000in}}{\pgfqpoint{5.700000in}{5.700000in}}%
\pgfusepath{clip}%
\pgfsetbuttcap%
\pgfsetroundjoin%
\definecolor{currentfill}{rgb}{0.304148,0.764704,0.419943}%
\pgfsetfillcolor{currentfill}%
\pgfsetfillopacity{0.800000}%
\pgfsetlinewidth{0.000000pt}%
\definecolor{currentstroke}{rgb}{0.000000,0.000000,0.000000}%
\pgfsetstrokecolor{currentstroke}%
\pgfsetdash{}{0pt}%
\pgfpathmoveto{\pgfqpoint{5.860685in}{3.529494in}}%
\pgfpathlineto{\pgfqpoint{5.875756in}{3.545484in}}%
\pgfpathlineto{\pgfqpoint{5.890851in}{3.561658in}}%
\pgfpathlineto{\pgfqpoint{5.905970in}{3.578016in}}%
\pgfpathlineto{\pgfqpoint{5.921113in}{3.594560in}}%
\pgfpathlineto{\pgfqpoint{5.928246in}{3.594476in}}%
\pgfpathlineto{\pgfqpoint{5.935371in}{3.594345in}}%
\pgfpathlineto{\pgfqpoint{5.942487in}{3.594173in}}%
\pgfpathlineto{\pgfqpoint{5.949594in}{3.593965in}}%
\pgfpathlineto{\pgfqpoint{5.934484in}{3.577971in}}%
\pgfpathlineto{\pgfqpoint{5.919397in}{3.562160in}}%
\pgfpathlineto{\pgfqpoint{5.904333in}{3.546533in}}%
\pgfpathlineto{\pgfqpoint{5.889293in}{3.531088in}}%
\pgfpathlineto{\pgfqpoint{5.882153in}{3.530737in}}%
\pgfpathlineto{\pgfqpoint{5.875006in}{3.530358in}}%
\pgfpathlineto{\pgfqpoint{5.867849in}{3.529946in}}%
\pgfpathlineto{\pgfqpoint{5.860685in}{3.529494in}}%
\pgfpathclose%
\pgfusepath{fill}%
\end{pgfscope}%
\begin{pgfscope}%
\pgfpathrectangle{\pgfqpoint{1.150000in}{0.150000in}}{\pgfqpoint{5.700000in}{5.700000in}}%
\pgfusepath{clip}%
\pgfsetbuttcap%
\pgfsetroundjoin%
\definecolor{currentfill}{rgb}{0.279574,0.170599,0.479997}%
\pgfsetfillcolor{currentfill}%
\pgfsetfillopacity{0.800000}%
\pgfsetlinewidth{0.000000pt}%
\definecolor{currentstroke}{rgb}{0.000000,0.000000,0.000000}%
\pgfsetstrokecolor{currentstroke}%
\pgfsetdash{}{0pt}%
\pgfpathmoveto{\pgfqpoint{4.012437in}{1.779398in}}%
\pgfpathlineto{\pgfqpoint{4.026413in}{1.783968in}}%
\pgfpathlineto{\pgfqpoint{4.040400in}{1.788722in}}%
\pgfpathlineto{\pgfqpoint{4.054399in}{1.793661in}}%
\pgfpathlineto{\pgfqpoint{4.068409in}{1.798784in}}%
\pgfpathlineto{\pgfqpoint{4.076503in}{1.812307in}}%
\pgfpathlineto{\pgfqpoint{4.084593in}{1.825800in}}%
\pgfpathlineto{\pgfqpoint{4.092678in}{1.839263in}}%
\pgfpathlineto{\pgfqpoint{4.100758in}{1.852690in}}%
\pgfpathlineto{\pgfqpoint{4.086751in}{1.847220in}}%
\pgfpathlineto{\pgfqpoint{4.072755in}{1.841935in}}%
\pgfpathlineto{\pgfqpoint{4.058770in}{1.836835in}}%
\pgfpathlineto{\pgfqpoint{4.044797in}{1.831919in}}%
\pgfpathlineto{\pgfqpoint{4.036714in}{1.818826in}}%
\pgfpathlineto{\pgfqpoint{4.028626in}{1.805706in}}%
\pgfpathlineto{\pgfqpoint{4.020534in}{1.792562in}}%
\pgfpathlineto{\pgfqpoint{4.012437in}{1.779398in}}%
\pgfpathclose%
\pgfusepath{fill}%
\end{pgfscope}%
\begin{pgfscope}%
\pgfpathrectangle{\pgfqpoint{1.150000in}{0.150000in}}{\pgfqpoint{5.700000in}{5.700000in}}%
\pgfusepath{clip}%
\pgfsetbuttcap%
\pgfsetroundjoin%
\definecolor{currentfill}{rgb}{0.280267,0.073417,0.397163}%
\pgfsetfillcolor{currentfill}%
\pgfsetfillopacity{0.800000}%
\pgfsetlinewidth{0.000000pt}%
\definecolor{currentstroke}{rgb}{0.000000,0.000000,0.000000}%
\pgfsetstrokecolor{currentstroke}%
\pgfsetdash{}{0pt}%
\pgfpathmoveto{\pgfqpoint{2.904080in}{1.622867in}}%
\pgfpathlineto{\pgfqpoint{2.917992in}{1.611162in}}%
\pgfpathlineto{\pgfqpoint{2.931902in}{1.599679in}}%
\pgfpathlineto{\pgfqpoint{2.945810in}{1.588416in}}%
\pgfpathlineto{\pgfqpoint{2.959715in}{1.577372in}}%
\pgfpathlineto{\pgfqpoint{2.968371in}{1.579067in}}%
\pgfpathlineto{\pgfqpoint{2.977012in}{1.581042in}}%
\pgfpathlineto{\pgfqpoint{2.985639in}{1.583291in}}%
\pgfpathlineto{\pgfqpoint{2.994251in}{1.585806in}}%
\pgfpathlineto{\pgfqpoint{2.980384in}{1.596181in}}%
\pgfpathlineto{\pgfqpoint{2.966515in}{1.606775in}}%
\pgfpathlineto{\pgfqpoint{2.952644in}{1.617589in}}%
\pgfpathlineto{\pgfqpoint{2.938771in}{1.628624in}}%
\pgfpathlineto{\pgfqpoint{2.930121in}{1.626764in}}%
\pgfpathlineto{\pgfqpoint{2.921456in}{1.625181in}}%
\pgfpathlineto{\pgfqpoint{2.912776in}{1.623879in}}%
\pgfpathlineto{\pgfqpoint{2.904080in}{1.622867in}}%
\pgfpathclose%
\pgfusepath{fill}%
\end{pgfscope}%
\begin{pgfscope}%
\pgfpathrectangle{\pgfqpoint{1.150000in}{0.150000in}}{\pgfqpoint{5.700000in}{5.700000in}}%
\pgfusepath{clip}%
\pgfsetbuttcap%
\pgfsetroundjoin%
\definecolor{currentfill}{rgb}{0.257322,0.256130,0.526563}%
\pgfsetfillcolor{currentfill}%
\pgfsetfillopacity{0.800000}%
\pgfsetlinewidth{0.000000pt}%
\definecolor{currentstroke}{rgb}{0.000000,0.000000,0.000000}%
\pgfsetstrokecolor{currentstroke}%
\pgfsetdash{}{0pt}%
\pgfpathmoveto{\pgfqpoint{4.221376in}{1.984638in}}%
\pgfpathlineto{\pgfqpoint{4.235442in}{1.991628in}}%
\pgfpathlineto{\pgfqpoint{4.249522in}{1.998803in}}%
\pgfpathlineto{\pgfqpoint{4.263614in}{2.006162in}}%
\pgfpathlineto{\pgfqpoint{4.277720in}{2.013706in}}%
\pgfpathlineto{\pgfqpoint{4.285758in}{2.027223in}}%
\pgfpathlineto{\pgfqpoint{4.293790in}{2.040664in}}%
\pgfpathlineto{\pgfqpoint{4.301818in}{2.054025in}}%
\pgfpathlineto{\pgfqpoint{4.309841in}{2.067305in}}%
\pgfpathlineto{\pgfqpoint{4.295735in}{2.059510in}}%
\pgfpathlineto{\pgfqpoint{4.281643in}{2.051899in}}%
\pgfpathlineto{\pgfqpoint{4.267564in}{2.044473in}}%
\pgfpathlineto{\pgfqpoint{4.253499in}{2.037232in}}%
\pgfpathlineto{\pgfqpoint{4.245475in}{2.024191in}}%
\pgfpathlineto{\pgfqpoint{4.237447in}{2.011077in}}%
\pgfpathlineto{\pgfqpoint{4.229414in}{1.997892in}}%
\pgfpathlineto{\pgfqpoint{4.221376in}{1.984638in}}%
\pgfpathclose%
\pgfusepath{fill}%
\end{pgfscope}%
\begin{pgfscope}%
\pgfpathrectangle{\pgfqpoint{1.150000in}{0.150000in}}{\pgfqpoint{5.700000in}{5.700000in}}%
\pgfusepath{clip}%
\pgfsetbuttcap%
\pgfsetroundjoin%
\definecolor{currentfill}{rgb}{0.154815,0.493313,0.557840}%
\pgfsetfillcolor{currentfill}%
\pgfsetfillopacity{0.800000}%
\pgfsetlinewidth{0.000000pt}%
\definecolor{currentstroke}{rgb}{0.000000,0.000000,0.000000}%
\pgfsetstrokecolor{currentstroke}%
\pgfsetdash{}{0pt}%
\pgfpathmoveto{\pgfqpoint{2.093768in}{2.757591in}}%
\pgfpathlineto{\pgfqpoint{2.108219in}{2.729954in}}%
\pgfpathlineto{\pgfqpoint{2.122650in}{2.702672in}}%
\pgfpathlineto{\pgfqpoint{2.137064in}{2.675740in}}%
\pgfpathlineto{\pgfqpoint{2.151459in}{2.649155in}}%
\pgfpathlineto{\pgfqpoint{2.160830in}{2.641829in}}%
\pgfpathlineto{\pgfqpoint{2.170175in}{2.634897in}}%
\pgfpathlineto{\pgfqpoint{2.179494in}{2.628352in}}%
\pgfpathlineto{\pgfqpoint{2.188788in}{2.622185in}}%
\pgfpathlineto{\pgfqpoint{2.174460in}{2.648069in}}%
\pgfpathlineto{\pgfqpoint{2.160115in}{2.674298in}}%
\pgfpathlineto{\pgfqpoint{2.145752in}{2.700876in}}%
\pgfpathlineto{\pgfqpoint{2.131370in}{2.727805in}}%
\pgfpathlineto{\pgfqpoint{2.122010in}{2.734660in}}%
\pgfpathlineto{\pgfqpoint{2.112623in}{2.741905in}}%
\pgfpathlineto{\pgfqpoint{2.103209in}{2.749546in}}%
\pgfpathlineto{\pgfqpoint{2.093768in}{2.757591in}}%
\pgfpathclose%
\pgfusepath{fill}%
\end{pgfscope}%
\begin{pgfscope}%
\pgfpathrectangle{\pgfqpoint{1.150000in}{0.150000in}}{\pgfqpoint{5.700000in}{5.700000in}}%
\pgfusepath{clip}%
\pgfsetbuttcap%
\pgfsetroundjoin%
\definecolor{currentfill}{rgb}{0.206756,0.371758,0.553117}%
\pgfsetfillcolor{currentfill}%
\pgfsetfillopacity{0.800000}%
\pgfsetlinewidth{0.000000pt}%
\definecolor{currentstroke}{rgb}{0.000000,0.000000,0.000000}%
\pgfsetstrokecolor{currentstroke}%
\pgfsetdash{}{0pt}%
\pgfpathmoveto{\pgfqpoint{2.285984in}{2.380597in}}%
\pgfpathlineto{\pgfqpoint{2.300241in}{2.357507in}}%
\pgfpathlineto{\pgfqpoint{2.314484in}{2.334722in}}%
\pgfpathlineto{\pgfqpoint{2.328714in}{2.312239in}}%
\pgfpathlineto{\pgfqpoint{2.342931in}{2.290056in}}%
\pgfpathlineto{\pgfqpoint{2.352141in}{2.284050in}}%
\pgfpathlineto{\pgfqpoint{2.361327in}{2.278425in}}%
\pgfpathlineto{\pgfqpoint{2.370489in}{2.273175in}}%
\pgfpathlineto{\pgfqpoint{2.379628in}{2.268292in}}%
\pgfpathlineto{\pgfqpoint{2.365473in}{2.289762in}}%
\pgfpathlineto{\pgfqpoint{2.351305in}{2.311530in}}%
\pgfpathlineto{\pgfqpoint{2.337124in}{2.333599in}}%
\pgfpathlineto{\pgfqpoint{2.322930in}{2.355972in}}%
\pgfpathlineto{\pgfqpoint{2.313731in}{2.361555in}}%
\pgfpathlineto{\pgfqpoint{2.304506in}{2.367515in}}%
\pgfpathlineto{\pgfqpoint{2.295258in}{2.373860in}}%
\pgfpathlineto{\pgfqpoint{2.285984in}{2.380597in}}%
\pgfpathclose%
\pgfusepath{fill}%
\end{pgfscope}%
\begin{pgfscope}%
\pgfpathrectangle{\pgfqpoint{1.150000in}{0.150000in}}{\pgfqpoint{5.700000in}{5.700000in}}%
\pgfusepath{clip}%
\pgfsetbuttcap%
\pgfsetroundjoin%
\definecolor{currentfill}{rgb}{0.120092,0.600104,0.542530}%
\pgfsetfillcolor{currentfill}%
\pgfsetfillopacity{0.800000}%
\pgfsetlinewidth{0.000000pt}%
\definecolor{currentstroke}{rgb}{0.000000,0.000000,0.000000}%
\pgfsetstrokecolor{currentstroke}%
\pgfsetdash{}{0pt}%
\pgfpathmoveto{\pgfqpoint{5.177919in}{2.978208in}}%
\pgfpathlineto{\pgfqpoint{5.192551in}{2.992335in}}%
\pgfpathlineto{\pgfqpoint{5.207204in}{3.006648in}}%
\pgfpathlineto{\pgfqpoint{5.221877in}{3.021146in}}%
\pgfpathlineto{\pgfqpoint{5.236571in}{3.035830in}}%
\pgfpathlineto{\pgfqpoint{5.244197in}{3.042214in}}%
\pgfpathlineto{\pgfqpoint{5.251814in}{3.048457in}}%
\pgfpathlineto{\pgfqpoint{5.259421in}{3.054560in}}%
\pgfpathlineto{\pgfqpoint{5.267020in}{3.060526in}}%
\pgfpathlineto{\pgfqpoint{5.252337in}{3.046068in}}%
\pgfpathlineto{\pgfqpoint{5.237675in}{3.031795in}}%
\pgfpathlineto{\pgfqpoint{5.223034in}{3.017707in}}%
\pgfpathlineto{\pgfqpoint{5.208412in}{3.003804in}}%
\pgfpathlineto{\pgfqpoint{5.200802in}{2.997600in}}%
\pgfpathlineto{\pgfqpoint{5.193183in}{2.991268in}}%
\pgfpathlineto{\pgfqpoint{5.185556in}{2.984805in}}%
\pgfpathlineto{\pgfqpoint{5.177919in}{2.978208in}}%
\pgfpathclose%
\pgfusepath{fill}%
\end{pgfscope}%
\begin{pgfscope}%
\pgfpathrectangle{\pgfqpoint{1.150000in}{0.150000in}}{\pgfqpoint{5.700000in}{5.700000in}}%
\pgfusepath{clip}%
\pgfsetbuttcap%
\pgfsetroundjoin%
\definecolor{currentfill}{rgb}{0.274952,0.037752,0.364543}%
\pgfsetfillcolor{currentfill}%
\pgfsetfillopacity{0.800000}%
\pgfsetlinewidth{0.000000pt}%
\definecolor{currentstroke}{rgb}{0.000000,0.000000,0.000000}%
\pgfsetstrokecolor{currentstroke}%
\pgfsetdash{}{0pt}%
\pgfpathmoveto{\pgfqpoint{3.626642in}{1.504726in}}%
\pgfpathlineto{\pgfqpoint{3.640515in}{1.504116in}}%
\pgfpathlineto{\pgfqpoint{3.654394in}{1.503694in}}%
\pgfpathlineto{\pgfqpoint{3.668281in}{1.503460in}}%
\pgfpathlineto{\pgfqpoint{3.682174in}{1.503414in}}%
\pgfpathlineto{\pgfqpoint{3.690395in}{1.514658in}}%
\pgfpathlineto{\pgfqpoint{3.698609in}{1.525986in}}%
\pgfpathlineto{\pgfqpoint{3.706817in}{1.537392in}}%
\pgfpathlineto{\pgfqpoint{3.715020in}{1.548871in}}%
\pgfpathlineto{\pgfqpoint{3.701138in}{1.548416in}}%
\pgfpathlineto{\pgfqpoint{3.687263in}{1.548148in}}%
\pgfpathlineto{\pgfqpoint{3.673395in}{1.548069in}}%
\pgfpathlineto{\pgfqpoint{3.659535in}{1.548179in}}%
\pgfpathlineto{\pgfqpoint{3.651321in}{1.537189in}}%
\pgfpathlineto{\pgfqpoint{3.643101in}{1.526280in}}%
\pgfpathlineto{\pgfqpoint{3.634875in}{1.515457in}}%
\pgfpathlineto{\pgfqpoint{3.626642in}{1.504726in}}%
\pgfpathclose%
\pgfusepath{fill}%
\end{pgfscope}%
\begin{pgfscope}%
\pgfpathrectangle{\pgfqpoint{1.150000in}{0.150000in}}{\pgfqpoint{5.700000in}{5.700000in}}%
\pgfusepath{clip}%
\pgfsetbuttcap%
\pgfsetroundjoin%
\definecolor{currentfill}{rgb}{0.278791,0.062145,0.386592}%
\pgfsetfillcolor{currentfill}%
\pgfsetfillopacity{0.800000}%
\pgfsetlinewidth{0.000000pt}%
\definecolor{currentstroke}{rgb}{0.000000,0.000000,0.000000}%
\pgfsetstrokecolor{currentstroke}%
\pgfsetdash{}{0pt}%
\pgfpathmoveto{\pgfqpoint{3.715020in}{1.548871in}}%
\pgfpathlineto{\pgfqpoint{3.728910in}{1.549514in}}%
\pgfpathlineto{\pgfqpoint{3.742808in}{1.550344in}}%
\pgfpathlineto{\pgfqpoint{3.756713in}{1.551361in}}%
\pgfpathlineto{\pgfqpoint{3.770627in}{1.552565in}}%
\pgfpathlineto{\pgfqpoint{3.778815in}{1.564595in}}%
\pgfpathlineto{\pgfqpoint{3.786997in}{1.576683in}}%
\pgfpathlineto{\pgfqpoint{3.795173in}{1.588823in}}%
\pgfpathlineto{\pgfqpoint{3.803344in}{1.601010in}}%
\pgfpathlineto{\pgfqpoint{3.789439in}{1.599336in}}%
\pgfpathlineto{\pgfqpoint{3.775542in}{1.597849in}}%
\pgfpathlineto{\pgfqpoint{3.761654in}{1.596549in}}%
\pgfpathlineto{\pgfqpoint{3.747774in}{1.595436in}}%
\pgfpathlineto{\pgfqpoint{3.739594in}{1.583707in}}%
\pgfpathlineto{\pgfqpoint{3.731408in}{1.572033in}}%
\pgfpathlineto{\pgfqpoint{3.723217in}{1.560420in}}%
\pgfpathlineto{\pgfqpoint{3.715020in}{1.548871in}}%
\pgfpathclose%
\pgfusepath{fill}%
\end{pgfscope}%
\begin{pgfscope}%
\pgfpathrectangle{\pgfqpoint{1.150000in}{0.150000in}}{\pgfqpoint{5.700000in}{5.700000in}}%
\pgfusepath{clip}%
\pgfsetbuttcap%
\pgfsetroundjoin%
\definecolor{currentfill}{rgb}{0.165117,0.467423,0.558141}%
\pgfsetfillcolor{currentfill}%
\pgfsetfillopacity{0.800000}%
\pgfsetlinewidth{0.000000pt}%
\definecolor{currentstroke}{rgb}{0.000000,0.000000,0.000000}%
\pgfsetstrokecolor{currentstroke}%
\pgfsetdash{}{0pt}%
\pgfpathmoveto{\pgfqpoint{4.759966in}{2.563649in}}%
\pgfpathlineto{\pgfqpoint{4.774332in}{2.575424in}}%
\pgfpathlineto{\pgfqpoint{4.788716in}{2.587383in}}%
\pgfpathlineto{\pgfqpoint{4.803118in}{2.599528in}}%
\pgfpathlineto{\pgfqpoint{4.817537in}{2.611857in}}%
\pgfpathlineto{\pgfqpoint{4.825386in}{2.622253in}}%
\pgfpathlineto{\pgfqpoint{4.833228in}{2.632501in}}%
\pgfpathlineto{\pgfqpoint{4.841063in}{2.642605in}}%
\pgfpathlineto{\pgfqpoint{4.848890in}{2.652563in}}%
\pgfpathlineto{\pgfqpoint{4.834474in}{2.640248in}}%
\pgfpathlineto{\pgfqpoint{4.820075in}{2.628119in}}%
\pgfpathlineto{\pgfqpoint{4.805694in}{2.616174in}}%
\pgfpathlineto{\pgfqpoint{4.791331in}{2.604414in}}%
\pgfpathlineto{\pgfqpoint{4.783500in}{2.594428in}}%
\pgfpathlineto{\pgfqpoint{4.775662in}{2.584306in}}%
\pgfpathlineto{\pgfqpoint{4.767818in}{2.574046in}}%
\pgfpathlineto{\pgfqpoint{4.759966in}{2.563649in}}%
\pgfpathclose%
\pgfusepath{fill}%
\end{pgfscope}%
\begin{pgfscope}%
\pgfpathrectangle{\pgfqpoint{1.150000in}{0.150000in}}{\pgfqpoint{5.700000in}{5.700000in}}%
\pgfusepath{clip}%
\pgfsetbuttcap%
\pgfsetroundjoin%
\definecolor{currentfill}{rgb}{0.153894,0.680203,0.504172}%
\pgfsetfillcolor{currentfill}%
\pgfsetfillopacity{0.800000}%
\pgfsetlinewidth{0.000000pt}%
\definecolor{currentstroke}{rgb}{0.000000,0.000000,0.000000}%
\pgfsetstrokecolor{currentstroke}%
\pgfsetdash{}{0pt}%
\pgfpathmoveto{\pgfqpoint{5.475355in}{3.237832in}}%
\pgfpathlineto{\pgfqpoint{5.490184in}{3.253067in}}%
\pgfpathlineto{\pgfqpoint{5.505034in}{3.268488in}}%
\pgfpathlineto{\pgfqpoint{5.519907in}{3.284094in}}%
\pgfpathlineto{\pgfqpoint{5.534801in}{3.299886in}}%
\pgfpathlineto{\pgfqpoint{5.542229in}{3.303310in}}%
\pgfpathlineto{\pgfqpoint{5.549648in}{3.306621in}}%
\pgfpathlineto{\pgfqpoint{5.557056in}{3.309823in}}%
\pgfpathlineto{\pgfqpoint{5.564455in}{3.312919in}}%
\pgfpathlineto{\pgfqpoint{5.549580in}{3.297496in}}%
\pgfpathlineto{\pgfqpoint{5.534727in}{3.282258in}}%
\pgfpathlineto{\pgfqpoint{5.519896in}{3.267205in}}%
\pgfpathlineto{\pgfqpoint{5.505086in}{3.252336in}}%
\pgfpathlineto{\pgfqpoint{5.497667in}{3.248860in}}%
\pgfpathlineto{\pgfqpoint{5.490239in}{3.245286in}}%
\pgfpathlineto{\pgfqpoint{5.482802in}{3.241612in}}%
\pgfpathlineto{\pgfqpoint{5.475355in}{3.237832in}}%
\pgfpathclose%
\pgfusepath{fill}%
\end{pgfscope}%
\begin{pgfscope}%
\pgfpathrectangle{\pgfqpoint{1.150000in}{0.150000in}}{\pgfqpoint{5.700000in}{5.700000in}}%
\pgfusepath{clip}%
\pgfsetbuttcap%
\pgfsetroundjoin%
\definecolor{currentfill}{rgb}{0.267004,0.004874,0.329415}%
\pgfsetfillcolor{currentfill}%
\pgfsetfillopacity{0.800000}%
\pgfsetlinewidth{0.000000pt}%
\definecolor{currentstroke}{rgb}{0.000000,0.000000,0.000000}%
\pgfsetstrokecolor{currentstroke}%
\pgfsetdash{}{0pt}%
\pgfpathmoveto{\pgfqpoint{3.305257in}{1.447805in}}%
\pgfpathlineto{\pgfqpoint{3.319108in}{1.442414in}}%
\pgfpathlineto{\pgfqpoint{3.332962in}{1.437222in}}%
\pgfpathlineto{\pgfqpoint{3.346819in}{1.432226in}}%
\pgfpathlineto{\pgfqpoint{3.360680in}{1.427427in}}%
\pgfpathlineto{\pgfqpoint{3.369057in}{1.434837in}}%
\pgfpathlineto{\pgfqpoint{3.377426in}{1.442427in}}%
\pgfpathlineto{\pgfqpoint{3.385785in}{1.450190in}}%
\pgfpathlineto{\pgfqpoint{3.394136in}{1.458121in}}%
\pgfpathlineto{\pgfqpoint{3.380298in}{1.462324in}}%
\pgfpathlineto{\pgfqpoint{3.366463in}{1.466724in}}%
\pgfpathlineto{\pgfqpoint{3.352632in}{1.471321in}}%
\pgfpathlineto{\pgfqpoint{3.338804in}{1.476116in}}%
\pgfpathlineto{\pgfqpoint{3.330432in}{1.468769in}}%
\pgfpathlineto{\pgfqpoint{3.322050in}{1.461597in}}%
\pgfpathlineto{\pgfqpoint{3.313658in}{1.454607in}}%
\pgfpathlineto{\pgfqpoint{3.305257in}{1.447805in}}%
\pgfpathclose%
\pgfusepath{fill}%
\end{pgfscope}%
\begin{pgfscope}%
\pgfpathrectangle{\pgfqpoint{1.150000in}{0.150000in}}{\pgfqpoint{5.700000in}{5.700000in}}%
\pgfusepath{clip}%
\pgfsetbuttcap%
\pgfsetroundjoin%
\definecolor{currentfill}{rgb}{0.137770,0.537492,0.554906}%
\pgfsetfillcolor{currentfill}%
\pgfsetfillopacity{0.800000}%
\pgfsetlinewidth{0.000000pt}%
\definecolor{currentstroke}{rgb}{0.000000,0.000000,0.000000}%
\pgfsetstrokecolor{currentstroke}%
\pgfsetdash{}{0pt}%
\pgfpathmoveto{\pgfqpoint{4.969054in}{2.777700in}}%
\pgfpathlineto{\pgfqpoint{4.983554in}{2.790808in}}%
\pgfpathlineto{\pgfqpoint{4.998072in}{2.804101in}}%
\pgfpathlineto{\pgfqpoint{5.012610in}{2.817579in}}%
\pgfpathlineto{\pgfqpoint{5.027167in}{2.831242in}}%
\pgfpathlineto{\pgfqpoint{5.034915in}{2.839729in}}%
\pgfpathlineto{\pgfqpoint{5.042655in}{2.848066in}}%
\pgfpathlineto{\pgfqpoint{5.050386in}{2.856253in}}%
\pgfpathlineto{\pgfqpoint{5.058109in}{2.864293in}}%
\pgfpathlineto{\pgfqpoint{5.043558in}{2.850749in}}%
\pgfpathlineto{\pgfqpoint{5.029027in}{2.837390in}}%
\pgfpathlineto{\pgfqpoint{5.014514in}{2.824216in}}%
\pgfpathlineto{\pgfqpoint{5.000021in}{2.811227in}}%
\pgfpathlineto{\pgfqpoint{4.992291in}{2.803055in}}%
\pgfpathlineto{\pgfqpoint{4.984553in}{2.794745in}}%
\pgfpathlineto{\pgfqpoint{4.976808in}{2.786294in}}%
\pgfpathlineto{\pgfqpoint{4.969054in}{2.777700in}}%
\pgfpathclose%
\pgfusepath{fill}%
\end{pgfscope}%
\begin{pgfscope}%
\pgfpathrectangle{\pgfqpoint{1.150000in}{0.150000in}}{\pgfqpoint{5.700000in}{5.700000in}}%
\pgfusepath{clip}%
\pgfsetbuttcap%
\pgfsetroundjoin%
\definecolor{currentfill}{rgb}{0.271305,0.019942,0.347269}%
\pgfsetfillcolor{currentfill}%
\pgfsetfillopacity{0.800000}%
\pgfsetlinewidth{0.000000pt}%
\definecolor{currentstroke}{rgb}{0.000000,0.000000,0.000000}%
\pgfsetstrokecolor{currentstroke}%
\pgfsetdash{}{0pt}%
\pgfpathmoveto{\pgfqpoint{3.538163in}{1.469277in}}%
\pgfpathlineto{\pgfqpoint{3.552025in}{1.467376in}}%
\pgfpathlineto{\pgfqpoint{3.565893in}{1.465665in}}%
\pgfpathlineto{\pgfqpoint{3.579767in}{1.464143in}}%
\pgfpathlineto{\pgfqpoint{3.593647in}{1.462812in}}%
\pgfpathlineto{\pgfqpoint{3.601906in}{1.473129in}}%
\pgfpathlineto{\pgfqpoint{3.610158in}{1.483557in}}%
\pgfpathlineto{\pgfqpoint{3.618403in}{1.494091in}}%
\pgfpathlineto{\pgfqpoint{3.626642in}{1.504726in}}%
\pgfpathlineto{\pgfqpoint{3.612776in}{1.505526in}}%
\pgfpathlineto{\pgfqpoint{3.598917in}{1.506515in}}%
\pgfpathlineto{\pgfqpoint{3.585064in}{1.507694in}}%
\pgfpathlineto{\pgfqpoint{3.571217in}{1.509064in}}%
\pgfpathlineto{\pgfqpoint{3.562964in}{1.498949in}}%
\pgfpathlineto{\pgfqpoint{3.554704in}{1.488942in}}%
\pgfpathlineto{\pgfqpoint{3.546437in}{1.479050in}}%
\pgfpathlineto{\pgfqpoint{3.538163in}{1.469277in}}%
\pgfpathclose%
\pgfusepath{fill}%
\end{pgfscope}%
\begin{pgfscope}%
\pgfpathrectangle{\pgfqpoint{1.150000in}{0.150000in}}{\pgfqpoint{5.700000in}{5.700000in}}%
\pgfusepath{clip}%
\pgfsetbuttcap%
\pgfsetroundjoin%
\definecolor{currentfill}{rgb}{0.269944,0.014625,0.341379}%
\pgfsetfillcolor{currentfill}%
\pgfsetfillopacity{0.800000}%
\pgfsetlinewidth{0.000000pt}%
\definecolor{currentstroke}{rgb}{0.000000,0.000000,0.000000}%
\pgfsetstrokecolor{currentstroke}%
\pgfsetdash{}{0pt}%
\pgfpathmoveto{\pgfqpoint{3.160603in}{1.477958in}}%
\pgfpathlineto{\pgfqpoint{3.174466in}{1.470328in}}%
\pgfpathlineto{\pgfqpoint{3.188331in}{1.462902in}}%
\pgfpathlineto{\pgfqpoint{3.202197in}{1.455680in}}%
\pgfpathlineto{\pgfqpoint{3.216064in}{1.448660in}}%
\pgfpathlineto{\pgfqpoint{3.224534in}{1.454011in}}%
\pgfpathlineto{\pgfqpoint{3.232992in}{1.459581in}}%
\pgfpathlineto{\pgfqpoint{3.241440in}{1.465366in}}%
\pgfpathlineto{\pgfqpoint{3.249877in}{1.471359in}}%
\pgfpathlineto{\pgfqpoint{3.236038in}{1.477750in}}%
\pgfpathlineto{\pgfqpoint{3.222200in}{1.484343in}}%
\pgfpathlineto{\pgfqpoint{3.208364in}{1.491139in}}%
\pgfpathlineto{\pgfqpoint{3.194530in}{1.498140in}}%
\pgfpathlineto{\pgfqpoint{3.186065in}{1.492764in}}%
\pgfpathlineto{\pgfqpoint{3.177589in}{1.487604in}}%
\pgfpathlineto{\pgfqpoint{3.169102in}{1.482667in}}%
\pgfpathlineto{\pgfqpoint{3.160603in}{1.477958in}}%
\pgfpathclose%
\pgfusepath{fill}%
\end{pgfscope}%
\begin{pgfscope}%
\pgfpathrectangle{\pgfqpoint{1.150000in}{0.150000in}}{\pgfqpoint{5.700000in}{5.700000in}}%
\pgfusepath{clip}%
\pgfsetbuttcap%
\pgfsetroundjoin%
\definecolor{currentfill}{rgb}{0.218130,0.347432,0.550038}%
\pgfsetfillcolor{currentfill}%
\pgfsetfillopacity{0.800000}%
\pgfsetlinewidth{0.000000pt}%
\definecolor{currentstroke}{rgb}{0.000000,0.000000,0.000000}%
\pgfsetstrokecolor{currentstroke}%
\pgfsetdash{}{0pt}%
\pgfpathmoveto{\pgfqpoint{4.430407in}{2.205044in}}%
\pgfpathlineto{\pgfqpoint{4.444584in}{2.214167in}}%
\pgfpathlineto{\pgfqpoint{4.458775in}{2.223474in}}%
\pgfpathlineto{\pgfqpoint{4.472982in}{2.232966in}}%
\pgfpathlineto{\pgfqpoint{4.487205in}{2.242642in}}%
\pgfpathlineto{\pgfqpoint{4.495182in}{2.255444in}}%
\pgfpathlineto{\pgfqpoint{4.503154in}{2.268131in}}%
\pgfpathlineto{\pgfqpoint{4.511120in}{2.280703in}}%
\pgfpathlineto{\pgfqpoint{4.519080in}{2.293159in}}%
\pgfpathlineto{\pgfqpoint{4.504858in}{2.283328in}}%
\pgfpathlineto{\pgfqpoint{4.490651in}{2.273683in}}%
\pgfpathlineto{\pgfqpoint{4.476459in}{2.264222in}}%
\pgfpathlineto{\pgfqpoint{4.462283in}{2.254945in}}%
\pgfpathlineto{\pgfqpoint{4.454322in}{2.242631in}}%
\pgfpathlineto{\pgfqpoint{4.446356in}{2.230208in}}%
\pgfpathlineto{\pgfqpoint{4.438384in}{2.217679in}}%
\pgfpathlineto{\pgfqpoint{4.430407in}{2.205044in}}%
\pgfpathclose%
\pgfusepath{fill}%
\end{pgfscope}%
\begin{pgfscope}%
\pgfpathrectangle{\pgfqpoint{1.150000in}{0.150000in}}{\pgfqpoint{5.700000in}{5.700000in}}%
\pgfusepath{clip}%
\pgfsetbuttcap%
\pgfsetroundjoin%
\definecolor{currentfill}{rgb}{0.281924,0.089666,0.412415}%
\pgfsetfillcolor{currentfill}%
\pgfsetfillopacity{0.800000}%
\pgfsetlinewidth{0.000000pt}%
\definecolor{currentstroke}{rgb}{0.000000,0.000000,0.000000}%
\pgfsetstrokecolor{currentstroke}%
\pgfsetdash{}{0pt}%
\pgfpathmoveto{\pgfqpoint{3.803344in}{1.601010in}}%
\pgfpathlineto{\pgfqpoint{3.817258in}{1.602871in}}%
\pgfpathlineto{\pgfqpoint{3.831181in}{1.604917in}}%
\pgfpathlineto{\pgfqpoint{3.845113in}{1.607149in}}%
\pgfpathlineto{\pgfqpoint{3.859053in}{1.609567in}}%
\pgfpathlineto{\pgfqpoint{3.867212in}{1.622249in}}%
\pgfpathlineto{\pgfqpoint{3.875366in}{1.634964in}}%
\pgfpathlineto{\pgfqpoint{3.883515in}{1.647705in}}%
\pgfpathlineto{\pgfqpoint{3.891659in}{1.660470in}}%
\pgfpathlineto{\pgfqpoint{3.877725in}{1.657613in}}%
\pgfpathlineto{\pgfqpoint{3.863800in}{1.654941in}}%
\pgfpathlineto{\pgfqpoint{3.849884in}{1.652455in}}%
\pgfpathlineto{\pgfqpoint{3.835978in}{1.650155in}}%
\pgfpathlineto{\pgfqpoint{3.827827in}{1.637818in}}%
\pgfpathlineto{\pgfqpoint{3.819671in}{1.625512in}}%
\pgfpathlineto{\pgfqpoint{3.811510in}{1.613242in}}%
\pgfpathlineto{\pgfqpoint{3.803344in}{1.601010in}}%
\pgfpathclose%
\pgfusepath{fill}%
\end{pgfscope}%
\begin{pgfscope}%
\pgfpathrectangle{\pgfqpoint{1.150000in}{0.150000in}}{\pgfqpoint{5.700000in}{5.700000in}}%
\pgfusepath{clip}%
\pgfsetbuttcap%
\pgfsetroundjoin%
\definecolor{currentfill}{rgb}{0.277941,0.056324,0.381191}%
\pgfsetfillcolor{currentfill}%
\pgfsetfillopacity{0.800000}%
\pgfsetlinewidth{0.000000pt}%
\definecolor{currentstroke}{rgb}{0.000000,0.000000,0.000000}%
\pgfsetstrokecolor{currentstroke}%
\pgfsetdash{}{0pt}%
\pgfpathmoveto{\pgfqpoint{2.959715in}{1.577372in}}%
\pgfpathlineto{\pgfqpoint{2.973618in}{1.566547in}}%
\pgfpathlineto{\pgfqpoint{2.987520in}{1.555938in}}%
\pgfpathlineto{\pgfqpoint{3.001420in}{1.545545in}}%
\pgfpathlineto{\pgfqpoint{3.015319in}{1.535366in}}%
\pgfpathlineto{\pgfqpoint{3.023937in}{1.537741in}}%
\pgfpathlineto{\pgfqpoint{3.032541in}{1.540387in}}%
\pgfpathlineto{\pgfqpoint{3.041131in}{1.543299in}}%
\pgfpathlineto{\pgfqpoint{3.049708in}{1.546469in}}%
\pgfpathlineto{\pgfqpoint{3.035845in}{1.555981in}}%
\pgfpathlineto{\pgfqpoint{3.021982in}{1.565707in}}%
\pgfpathlineto{\pgfqpoint{3.008117in}{1.575648in}}%
\pgfpathlineto{\pgfqpoint{2.994251in}{1.585806in}}%
\pgfpathlineto{\pgfqpoint{2.985639in}{1.583291in}}%
\pgfpathlineto{\pgfqpoint{2.977012in}{1.581042in}}%
\pgfpathlineto{\pgfqpoint{2.968371in}{1.579067in}}%
\pgfpathlineto{\pgfqpoint{2.959715in}{1.577372in}}%
\pgfpathclose%
\pgfusepath{fill}%
\end{pgfscope}%
\begin{pgfscope}%
\pgfpathrectangle{\pgfqpoint{1.150000in}{0.150000in}}{\pgfqpoint{5.700000in}{5.700000in}}%
\pgfusepath{clip}%
\pgfsetbuttcap%
\pgfsetroundjoin%
\definecolor{currentfill}{rgb}{0.273006,0.204520,0.501721}%
\pgfsetfillcolor{currentfill}%
\pgfsetfillopacity{0.800000}%
\pgfsetlinewidth{0.000000pt}%
\definecolor{currentstroke}{rgb}{0.000000,0.000000,0.000000}%
\pgfsetstrokecolor{currentstroke}%
\pgfsetdash{}{0pt}%
\pgfpathmoveto{\pgfqpoint{4.100758in}{1.852690in}}%
\pgfpathlineto{\pgfqpoint{4.114778in}{1.858345in}}%
\pgfpathlineto{\pgfqpoint{4.128809in}{1.864184in}}%
\pgfpathlineto{\pgfqpoint{4.142852in}{1.870207in}}%
\pgfpathlineto{\pgfqpoint{4.156908in}{1.876413in}}%
\pgfpathlineto{\pgfqpoint{4.164982in}{1.890131in}}%
\pgfpathlineto{\pgfqpoint{4.173052in}{1.903800in}}%
\pgfpathlineto{\pgfqpoint{4.181118in}{1.917417in}}%
\pgfpathlineto{\pgfqpoint{4.189179in}{1.930980in}}%
\pgfpathlineto{\pgfqpoint{4.175124in}{1.924457in}}%
\pgfpathlineto{\pgfqpoint{4.161082in}{1.918119in}}%
\pgfpathlineto{\pgfqpoint{4.147052in}{1.911965in}}%
\pgfpathlineto{\pgfqpoint{4.133035in}{1.905995in}}%
\pgfpathlineto{\pgfqpoint{4.124973in}{1.892735in}}%
\pgfpathlineto{\pgfqpoint{4.116906in}{1.879430in}}%
\pgfpathlineto{\pgfqpoint{4.108834in}{1.866080in}}%
\pgfpathlineto{\pgfqpoint{4.100758in}{1.852690in}}%
\pgfpathclose%
\pgfusepath{fill}%
\end{pgfscope}%
\begin{pgfscope}%
\pgfpathrectangle{\pgfqpoint{1.150000in}{0.150000in}}{\pgfqpoint{5.700000in}{5.700000in}}%
\pgfusepath{clip}%
\pgfsetbuttcap%
\pgfsetroundjoin%
\definecolor{currentfill}{rgb}{0.267004,0.004874,0.329415}%
\pgfsetfillcolor{currentfill}%
\pgfsetfillopacity{0.800000}%
\pgfsetlinewidth{0.000000pt}%
\definecolor{currentstroke}{rgb}{0.000000,0.000000,0.000000}%
\pgfsetstrokecolor{currentstroke}%
\pgfsetdash{}{0pt}%
\pgfpathmoveto{\pgfqpoint{3.449529in}{1.443256in}}%
\pgfpathlineto{\pgfqpoint{3.463388in}{1.440024in}}%
\pgfpathlineto{\pgfqpoint{3.477252in}{1.436985in}}%
\pgfpathlineto{\pgfqpoint{3.491120in}{1.434137in}}%
\pgfpathlineto{\pgfqpoint{3.504994in}{1.431482in}}%
\pgfpathlineto{\pgfqpoint{3.513298in}{1.440725in}}%
\pgfpathlineto{\pgfqpoint{3.521594in}{1.450109in}}%
\pgfpathlineto{\pgfqpoint{3.529882in}{1.459628in}}%
\pgfpathlineto{\pgfqpoint{3.538163in}{1.469277in}}%
\pgfpathlineto{\pgfqpoint{3.524307in}{1.471369in}}%
\pgfpathlineto{\pgfqpoint{3.510455in}{1.473653in}}%
\pgfpathlineto{\pgfqpoint{3.496610in}{1.476129in}}%
\pgfpathlineto{\pgfqpoint{3.482769in}{1.478798in}}%
\pgfpathlineto{\pgfqpoint{3.474471in}{1.469701in}}%
\pgfpathlineto{\pgfqpoint{3.466165in}{1.460741in}}%
\pgfpathlineto{\pgfqpoint{3.457851in}{1.451924in}}%
\pgfpathlineto{\pgfqpoint{3.449529in}{1.443256in}}%
\pgfpathclose%
\pgfusepath{fill}%
\end{pgfscope}%
\begin{pgfscope}%
\pgfpathrectangle{\pgfqpoint{1.150000in}{0.150000in}}{\pgfqpoint{5.700000in}{5.700000in}}%
\pgfusepath{clip}%
\pgfsetbuttcap%
\pgfsetroundjoin%
\definecolor{currentfill}{rgb}{0.283229,0.120777,0.440584}%
\pgfsetfillcolor{currentfill}%
\pgfsetfillopacity{0.800000}%
\pgfsetlinewidth{0.000000pt}%
\definecolor{currentstroke}{rgb}{0.000000,0.000000,0.000000}%
\pgfsetstrokecolor{currentstroke}%
\pgfsetdash{}{0pt}%
\pgfpathmoveto{\pgfqpoint{3.891659in}{1.660470in}}%
\pgfpathlineto{\pgfqpoint{3.905603in}{1.663513in}}%
\pgfpathlineto{\pgfqpoint{3.919557in}{1.666741in}}%
\pgfpathlineto{\pgfqpoint{3.933521in}{1.670154in}}%
\pgfpathlineto{\pgfqpoint{3.947494in}{1.673751in}}%
\pgfpathlineto{\pgfqpoint{3.955628in}{1.686956in}}%
\pgfpathlineto{\pgfqpoint{3.963758in}{1.700169in}}%
\pgfpathlineto{\pgfqpoint{3.971883in}{1.713386in}}%
\pgfpathlineto{\pgfqpoint{3.980003in}{1.726603in}}%
\pgfpathlineto{\pgfqpoint{3.966033in}{1.722596in}}%
\pgfpathlineto{\pgfqpoint{3.952074in}{1.718774in}}%
\pgfpathlineto{\pgfqpoint{3.938125in}{1.715137in}}%
\pgfpathlineto{\pgfqpoint{3.924187in}{1.711685in}}%
\pgfpathlineto{\pgfqpoint{3.916062in}{1.698866in}}%
\pgfpathlineto{\pgfqpoint{3.907933in}{1.686054in}}%
\pgfpathlineto{\pgfqpoint{3.899798in}{1.673254in}}%
\pgfpathlineto{\pgfqpoint{3.891659in}{1.660470in}}%
\pgfpathclose%
\pgfusepath{fill}%
\end{pgfscope}%
\begin{pgfscope}%
\pgfpathrectangle{\pgfqpoint{1.150000in}{0.150000in}}{\pgfqpoint{5.700000in}{5.700000in}}%
\pgfusepath{clip}%
\pgfsetbuttcap%
\pgfsetroundjoin%
\definecolor{currentfill}{rgb}{0.190631,0.407061,0.556089}%
\pgfsetfillcolor{currentfill}%
\pgfsetfillopacity{0.800000}%
\pgfsetlinewidth{0.000000pt}%
\definecolor{currentstroke}{rgb}{0.000000,0.000000,0.000000}%
\pgfsetstrokecolor{currentstroke}%
\pgfsetdash{}{0pt}%
\pgfpathmoveto{\pgfqpoint{2.228811in}{2.476063in}}%
\pgfpathlineto{\pgfqpoint{2.243126in}{2.451725in}}%
\pgfpathlineto{\pgfqpoint{2.257427in}{2.427703in}}%
\pgfpathlineto{\pgfqpoint{2.271712in}{2.403994in}}%
\pgfpathlineto{\pgfqpoint{2.285984in}{2.380597in}}%
\pgfpathlineto{\pgfqpoint{2.295258in}{2.373860in}}%
\pgfpathlineto{\pgfqpoint{2.304506in}{2.367515in}}%
\pgfpathlineto{\pgfqpoint{2.313731in}{2.361555in}}%
\pgfpathlineto{\pgfqpoint{2.322930in}{2.355972in}}%
\pgfpathlineto{\pgfqpoint{2.308723in}{2.378650in}}%
\pgfpathlineto{\pgfqpoint{2.294502in}{2.401637in}}%
\pgfpathlineto{\pgfqpoint{2.280267in}{2.424936in}}%
\pgfpathlineto{\pgfqpoint{2.266018in}{2.448550in}}%
\pgfpathlineto{\pgfqpoint{2.256754in}{2.454840in}}%
\pgfpathlineto{\pgfqpoint{2.247466in}{2.461517in}}%
\pgfpathlineto{\pgfqpoint{2.238151in}{2.468589in}}%
\pgfpathlineto{\pgfqpoint{2.228811in}{2.476063in}}%
\pgfpathclose%
\pgfusepath{fill}%
\end{pgfscope}%
\begin{pgfscope}%
\pgfpathrectangle{\pgfqpoint{1.150000in}{0.150000in}}{\pgfqpoint{5.700000in}{5.700000in}}%
\pgfusepath{clip}%
\pgfsetbuttcap%
\pgfsetroundjoin%
\definecolor{currentfill}{rgb}{0.191090,0.708366,0.482284}%
\pgfsetfillcolor{currentfill}%
\pgfsetfillopacity{0.800000}%
\pgfsetlinewidth{0.000000pt}%
\definecolor{currentstroke}{rgb}{0.000000,0.000000,0.000000}%
\pgfsetstrokecolor{currentstroke}%
\pgfsetdash{}{0pt}%
\pgfpathmoveto{\pgfqpoint{5.564455in}{3.312919in}}%
\pgfpathlineto{\pgfqpoint{5.579353in}{3.328526in}}%
\pgfpathlineto{\pgfqpoint{5.594272in}{3.344319in}}%
\pgfpathlineto{\pgfqpoint{5.609215in}{3.360297in}}%
\pgfpathlineto{\pgfqpoint{5.624180in}{3.376461in}}%
\pgfpathlineto{\pgfqpoint{5.631548in}{3.379064in}}%
\pgfpathlineto{\pgfqpoint{5.638907in}{3.381561in}}%
\pgfpathlineto{\pgfqpoint{5.646255in}{3.383959in}}%
\pgfpathlineto{\pgfqpoint{5.653594in}{3.386260in}}%
\pgfpathlineto{\pgfqpoint{5.638651in}{3.370502in}}%
\pgfpathlineto{\pgfqpoint{5.623730in}{3.354929in}}%
\pgfpathlineto{\pgfqpoint{5.608832in}{3.339541in}}%
\pgfpathlineto{\pgfqpoint{5.593955in}{3.324337in}}%
\pgfpathlineto{\pgfqpoint{5.586594in}{3.321618in}}%
\pgfpathlineto{\pgfqpoint{5.579224in}{3.318812in}}%
\pgfpathlineto{\pgfqpoint{5.571844in}{3.315914in}}%
\pgfpathlineto{\pgfqpoint{5.564455in}{3.312919in}}%
\pgfpathclose%
\pgfusepath{fill}%
\end{pgfscope}%
\begin{pgfscope}%
\pgfpathrectangle{\pgfqpoint{1.150000in}{0.150000in}}{\pgfqpoint{5.700000in}{5.700000in}}%
\pgfusepath{clip}%
\pgfsetbuttcap%
\pgfsetroundjoin%
\definecolor{currentfill}{rgb}{0.182256,0.426184,0.557120}%
\pgfsetfillcolor{currentfill}%
\pgfsetfillopacity{0.800000}%
\pgfsetlinewidth{0.000000pt}%
\definecolor{currentstroke}{rgb}{0.000000,0.000000,0.000000}%
\pgfsetstrokecolor{currentstroke}%
\pgfsetdash{}{0pt}%
\pgfpathmoveto{\pgfqpoint{4.639600in}{2.430521in}}%
\pgfpathlineto{\pgfqpoint{4.653902in}{2.441483in}}%
\pgfpathlineto{\pgfqpoint{4.668220in}{2.452630in}}%
\pgfpathlineto{\pgfqpoint{4.682555in}{2.463961in}}%
\pgfpathlineto{\pgfqpoint{4.696908in}{2.475478in}}%
\pgfpathlineto{\pgfqpoint{4.704813in}{2.486986in}}%
\pgfpathlineto{\pgfqpoint{4.712712in}{2.498355in}}%
\pgfpathlineto{\pgfqpoint{4.720605in}{2.509585in}}%
\pgfpathlineto{\pgfqpoint{4.728490in}{2.520676in}}%
\pgfpathlineto{\pgfqpoint{4.714139in}{2.509106in}}%
\pgfpathlineto{\pgfqpoint{4.699805in}{2.497721in}}%
\pgfpathlineto{\pgfqpoint{4.685488in}{2.486521in}}%
\pgfpathlineto{\pgfqpoint{4.671188in}{2.475506in}}%
\pgfpathlineto{\pgfqpoint{4.663301in}{2.464456in}}%
\pgfpathlineto{\pgfqpoint{4.655407in}{2.453275in}}%
\pgfpathlineto{\pgfqpoint{4.647507in}{2.441963in}}%
\pgfpathlineto{\pgfqpoint{4.639600in}{2.430521in}}%
\pgfpathclose%
\pgfusepath{fill}%
\end{pgfscope}%
\begin{pgfscope}%
\pgfpathrectangle{\pgfqpoint{1.150000in}{0.150000in}}{\pgfqpoint{5.700000in}{5.700000in}}%
\pgfusepath{clip}%
\pgfsetbuttcap%
\pgfsetroundjoin%
\definecolor{currentfill}{rgb}{0.241237,0.296485,0.539709}%
\pgfsetfillcolor{currentfill}%
\pgfsetfillopacity{0.800000}%
\pgfsetlinewidth{0.000000pt}%
\definecolor{currentstroke}{rgb}{0.000000,0.000000,0.000000}%
\pgfsetstrokecolor{currentstroke}%
\pgfsetdash{}{0pt}%
\pgfpathmoveto{\pgfqpoint{4.309841in}{2.067305in}}%
\pgfpathlineto{\pgfqpoint{4.323960in}{2.075284in}}%
\pgfpathlineto{\pgfqpoint{4.338094in}{2.083448in}}%
\pgfpathlineto{\pgfqpoint{4.352241in}{2.091795in}}%
\pgfpathlineto{\pgfqpoint{4.366403in}{2.100327in}}%
\pgfpathlineto{\pgfqpoint{4.374421in}{2.113756in}}%
\pgfpathlineto{\pgfqpoint{4.382434in}{2.127091in}}%
\pgfpathlineto{\pgfqpoint{4.390443in}{2.140331in}}%
\pgfpathlineto{\pgfqpoint{4.398446in}{2.153474in}}%
\pgfpathlineto{\pgfqpoint{4.384284in}{2.144722in}}%
\pgfpathlineto{\pgfqpoint{4.370136in}{2.136155in}}%
\pgfpathlineto{\pgfqpoint{4.356003in}{2.127772in}}%
\pgfpathlineto{\pgfqpoint{4.341883in}{2.119574in}}%
\pgfpathlineto{\pgfqpoint{4.333880in}{2.106638in}}%
\pgfpathlineto{\pgfqpoint{4.325872in}{2.093613in}}%
\pgfpathlineto{\pgfqpoint{4.317859in}{2.080502in}}%
\pgfpathlineto{\pgfqpoint{4.309841in}{2.067305in}}%
\pgfpathclose%
\pgfusepath{fill}%
\end{pgfscope}%
\begin{pgfscope}%
\pgfpathrectangle{\pgfqpoint{1.150000in}{0.150000in}}{\pgfqpoint{5.700000in}{5.700000in}}%
\pgfusepath{clip}%
\pgfsetbuttcap%
\pgfsetroundjoin%
\definecolor{currentfill}{rgb}{0.121380,0.629492,0.531973}%
\pgfsetfillcolor{currentfill}%
\pgfsetfillopacity{0.800000}%
\pgfsetlinewidth{0.000000pt}%
\definecolor{currentstroke}{rgb}{0.000000,0.000000,0.000000}%
\pgfsetstrokecolor{currentstroke}%
\pgfsetdash{}{0pt}%
\pgfpathmoveto{\pgfqpoint{5.267020in}{3.060526in}}%
\pgfpathlineto{\pgfqpoint{5.281723in}{3.075169in}}%
\pgfpathlineto{\pgfqpoint{5.296447in}{3.089998in}}%
\pgfpathlineto{\pgfqpoint{5.311192in}{3.105013in}}%
\pgfpathlineto{\pgfqpoint{5.325958in}{3.120214in}}%
\pgfpathlineto{\pgfqpoint{5.333535in}{3.125797in}}%
\pgfpathlineto{\pgfqpoint{5.341102in}{3.131242in}}%
\pgfpathlineto{\pgfqpoint{5.348660in}{3.136550in}}%
\pgfpathlineto{\pgfqpoint{5.356208in}{3.141725in}}%
\pgfpathlineto{\pgfqpoint{5.341455in}{3.126787in}}%
\pgfpathlineto{\pgfqpoint{5.326723in}{3.112033in}}%
\pgfpathlineto{\pgfqpoint{5.312012in}{3.097465in}}%
\pgfpathlineto{\pgfqpoint{5.297322in}{3.083082in}}%
\pgfpathlineto{\pgfqpoint{5.289760in}{3.077633in}}%
\pgfpathlineto{\pgfqpoint{5.282189in}{3.072060in}}%
\pgfpathlineto{\pgfqpoint{5.274609in}{3.066358in}}%
\pgfpathlineto{\pgfqpoint{5.267020in}{3.060526in}}%
\pgfpathclose%
\pgfusepath{fill}%
\end{pgfscope}%
\begin{pgfscope}%
\pgfpathrectangle{\pgfqpoint{1.150000in}{0.150000in}}{\pgfqpoint{5.700000in}{5.700000in}}%
\pgfusepath{clip}%
\pgfsetbuttcap%
\pgfsetroundjoin%
\definecolor{currentfill}{rgb}{0.139147,0.533812,0.555298}%
\pgfsetfillcolor{currentfill}%
\pgfsetfillopacity{0.800000}%
\pgfsetlinewidth{0.000000pt}%
\definecolor{currentstroke}{rgb}{0.000000,0.000000,0.000000}%
\pgfsetstrokecolor{currentstroke}%
\pgfsetdash{}{0pt}%
\pgfpathmoveto{\pgfqpoint{2.035768in}{2.871749in}}%
\pgfpathlineto{\pgfqpoint{2.050298in}{2.842660in}}%
\pgfpathlineto{\pgfqpoint{2.064808in}{2.813940in}}%
\pgfpathlineto{\pgfqpoint{2.079298in}{2.785584in}}%
\pgfpathlineto{\pgfqpoint{2.093768in}{2.757591in}}%
\pgfpathlineto{\pgfqpoint{2.103209in}{2.749546in}}%
\pgfpathlineto{\pgfqpoint{2.112623in}{2.741905in}}%
\pgfpathlineto{\pgfqpoint{2.122010in}{2.734660in}}%
\pgfpathlineto{\pgfqpoint{2.131370in}{2.727805in}}%
\pgfpathlineto{\pgfqpoint{2.116970in}{2.755090in}}%
\pgfpathlineto{\pgfqpoint{2.102552in}{2.782733in}}%
\pgfpathlineto{\pgfqpoint{2.088113in}{2.810740in}}%
\pgfpathlineto{\pgfqpoint{2.073655in}{2.839113in}}%
\pgfpathlineto{\pgfqpoint{2.064225in}{2.846666in}}%
\pgfpathlineto{\pgfqpoint{2.054768in}{2.854617in}}%
\pgfpathlineto{\pgfqpoint{2.045282in}{2.862976in}}%
\pgfpathlineto{\pgfqpoint{2.035768in}{2.871749in}}%
\pgfpathclose%
\pgfusepath{fill}%
\end{pgfscope}%
\begin{pgfscope}%
\pgfpathrectangle{\pgfqpoint{1.150000in}{0.150000in}}{\pgfqpoint{5.700000in}{5.700000in}}%
\pgfusepath{clip}%
\pgfsetbuttcap%
\pgfsetroundjoin%
\definecolor{currentfill}{rgb}{0.276022,0.044167,0.370164}%
\pgfsetfillcolor{currentfill}%
\pgfsetfillopacity{0.800000}%
\pgfsetlinewidth{0.000000pt}%
\definecolor{currentstroke}{rgb}{0.000000,0.000000,0.000000}%
\pgfsetstrokecolor{currentstroke}%
\pgfsetdash{}{0pt}%
\pgfpathmoveto{\pgfqpoint{3.015319in}{1.535366in}}%
\pgfpathlineto{\pgfqpoint{3.029216in}{1.525401in}}%
\pgfpathlineto{\pgfqpoint{3.043113in}{1.515649in}}%
\pgfpathlineto{\pgfqpoint{3.057009in}{1.506108in}}%
\pgfpathlineto{\pgfqpoint{3.070904in}{1.496777in}}%
\pgfpathlineto{\pgfqpoint{3.079487in}{1.499830in}}%
\pgfpathlineto{\pgfqpoint{3.088056in}{1.503146in}}%
\pgfpathlineto{\pgfqpoint{3.096612in}{1.506719in}}%
\pgfpathlineto{\pgfqpoint{3.105155in}{1.510541in}}%
\pgfpathlineto{\pgfqpoint{3.091293in}{1.519207in}}%
\pgfpathlineto{\pgfqpoint{3.077432in}{1.528083in}}%
\pgfpathlineto{\pgfqpoint{3.063570in}{1.537170in}}%
\pgfpathlineto{\pgfqpoint{3.049708in}{1.546469in}}%
\pgfpathlineto{\pgfqpoint{3.041131in}{1.543299in}}%
\pgfpathlineto{\pgfqpoint{3.032541in}{1.540387in}}%
\pgfpathlineto{\pgfqpoint{3.023937in}{1.537741in}}%
\pgfpathlineto{\pgfqpoint{3.015319in}{1.535366in}}%
\pgfpathclose%
\pgfusepath{fill}%
\end{pgfscope}%
\begin{pgfscope}%
\pgfpathrectangle{\pgfqpoint{1.150000in}{0.150000in}}{\pgfqpoint{5.700000in}{5.700000in}}%
\pgfusepath{clip}%
\pgfsetbuttcap%
\pgfsetroundjoin%
\definecolor{currentfill}{rgb}{0.151918,0.500685,0.557587}%
\pgfsetfillcolor{currentfill}%
\pgfsetfillopacity{0.800000}%
\pgfsetlinewidth{0.000000pt}%
\definecolor{currentstroke}{rgb}{0.000000,0.000000,0.000000}%
\pgfsetstrokecolor{currentstroke}%
\pgfsetdash{}{0pt}%
\pgfpathmoveto{\pgfqpoint{4.848890in}{2.652563in}}%
\pgfpathlineto{\pgfqpoint{4.863325in}{2.665063in}}%
\pgfpathlineto{\pgfqpoint{4.877778in}{2.677749in}}%
\pgfpathlineto{\pgfqpoint{4.892249in}{2.690619in}}%
\pgfpathlineto{\pgfqpoint{4.906739in}{2.703675in}}%
\pgfpathlineto{\pgfqpoint{4.914556in}{2.713453in}}%
\pgfpathlineto{\pgfqpoint{4.922365in}{2.723079in}}%
\pgfpathlineto{\pgfqpoint{4.930166in}{2.732553in}}%
\pgfpathlineto{\pgfqpoint{4.937959in}{2.741878in}}%
\pgfpathlineto{\pgfqpoint{4.923473in}{2.728872in}}%
\pgfpathlineto{\pgfqpoint{4.909006in}{2.716051in}}%
\pgfpathlineto{\pgfqpoint{4.894557in}{2.703415in}}%
\pgfpathlineto{\pgfqpoint{4.880126in}{2.690964in}}%
\pgfpathlineto{\pgfqpoint{4.872328in}{2.681577in}}%
\pgfpathlineto{\pgfqpoint{4.864523in}{2.672048in}}%
\pgfpathlineto{\pgfqpoint{4.856710in}{2.662378in}}%
\pgfpathlineto{\pgfqpoint{4.848890in}{2.652563in}}%
\pgfpathclose%
\pgfusepath{fill}%
\end{pgfscope}%
\begin{pgfscope}%
\pgfpathrectangle{\pgfqpoint{1.150000in}{0.150000in}}{\pgfqpoint{5.700000in}{5.700000in}}%
\pgfusepath{clip}%
\pgfsetbuttcap%
\pgfsetroundjoin%
\definecolor{currentfill}{rgb}{0.127568,0.566949,0.550556}%
\pgfsetfillcolor{currentfill}%
\pgfsetfillopacity{0.800000}%
\pgfsetlinewidth{0.000000pt}%
\definecolor{currentstroke}{rgb}{0.000000,0.000000,0.000000}%
\pgfsetstrokecolor{currentstroke}%
\pgfsetdash{}{0pt}%
\pgfpathmoveto{\pgfqpoint{5.058109in}{2.864293in}}%
\pgfpathlineto{\pgfqpoint{5.072679in}{2.878023in}}%
\pgfpathlineto{\pgfqpoint{5.087269in}{2.891938in}}%
\pgfpathlineto{\pgfqpoint{5.101879in}{2.906039in}}%
\pgfpathlineto{\pgfqpoint{5.116509in}{2.920326in}}%
\pgfpathlineto{\pgfqpoint{5.124216in}{2.928079in}}%
\pgfpathlineto{\pgfqpoint{5.131914in}{2.935680in}}%
\pgfpathlineto{\pgfqpoint{5.139604in}{2.943130in}}%
\pgfpathlineto{\pgfqpoint{5.147284in}{2.950433in}}%
\pgfpathlineto{\pgfqpoint{5.132663in}{2.936302in}}%
\pgfpathlineto{\pgfqpoint{5.118061in}{2.922356in}}%
\pgfpathlineto{\pgfqpoint{5.103479in}{2.908595in}}%
\pgfpathlineto{\pgfqpoint{5.088916in}{2.895019in}}%
\pgfpathlineto{\pgfqpoint{5.081227in}{2.887549in}}%
\pgfpathlineto{\pgfqpoint{5.073530in}{2.879939in}}%
\pgfpathlineto{\pgfqpoint{5.065823in}{2.872188in}}%
\pgfpathlineto{\pgfqpoint{5.058109in}{2.864293in}}%
\pgfpathclose%
\pgfusepath{fill}%
\end{pgfscope}%
\begin{pgfscope}%
\pgfpathrectangle{\pgfqpoint{1.150000in}{0.150000in}}{\pgfqpoint{5.700000in}{5.700000in}}%
\pgfusepath{clip}%
\pgfsetbuttcap%
\pgfsetroundjoin%
\definecolor{currentfill}{rgb}{0.267004,0.004874,0.329415}%
\pgfsetfillcolor{currentfill}%
\pgfsetfillopacity{0.800000}%
\pgfsetlinewidth{0.000000pt}%
\definecolor{currentstroke}{rgb}{0.000000,0.000000,0.000000}%
\pgfsetstrokecolor{currentstroke}%
\pgfsetdash{}{0pt}%
\pgfpathmoveto{\pgfqpoint{3.216064in}{1.448660in}}%
\pgfpathlineto{\pgfqpoint{3.229933in}{1.441842in}}%
\pgfpathlineto{\pgfqpoint{3.243804in}{1.435226in}}%
\pgfpathlineto{\pgfqpoint{3.257676in}{1.428809in}}%
\pgfpathlineto{\pgfqpoint{3.271551in}{1.422592in}}%
\pgfpathlineto{\pgfqpoint{3.279993in}{1.428583in}}%
\pgfpathlineto{\pgfqpoint{3.288424in}{1.434787in}}%
\pgfpathlineto{\pgfqpoint{3.296846in}{1.441196in}}%
\pgfpathlineto{\pgfqpoint{3.305257in}{1.447805in}}%
\pgfpathlineto{\pgfqpoint{3.291408in}{1.453393in}}%
\pgfpathlineto{\pgfqpoint{3.277562in}{1.459182in}}%
\pgfpathlineto{\pgfqpoint{3.263719in}{1.465170in}}%
\pgfpathlineto{\pgfqpoint{3.249877in}{1.471359in}}%
\pgfpathlineto{\pgfqpoint{3.241440in}{1.465366in}}%
\pgfpathlineto{\pgfqpoint{3.232992in}{1.459581in}}%
\pgfpathlineto{\pgfqpoint{3.224534in}{1.454011in}}%
\pgfpathlineto{\pgfqpoint{3.216064in}{1.448660in}}%
\pgfpathclose%
\pgfusepath{fill}%
\end{pgfscope}%
\begin{pgfscope}%
\pgfpathrectangle{\pgfqpoint{1.150000in}{0.150000in}}{\pgfqpoint{5.700000in}{5.700000in}}%
\pgfusepath{clip}%
\pgfsetbuttcap%
\pgfsetroundjoin%
\definecolor{currentfill}{rgb}{0.281412,0.155834,0.469201}%
\pgfsetfillcolor{currentfill}%
\pgfsetfillopacity{0.800000}%
\pgfsetlinewidth{0.000000pt}%
\definecolor{currentstroke}{rgb}{0.000000,0.000000,0.000000}%
\pgfsetstrokecolor{currentstroke}%
\pgfsetdash{}{0pt}%
\pgfpathmoveto{\pgfqpoint{3.980003in}{1.726603in}}%
\pgfpathlineto{\pgfqpoint{3.993983in}{1.730794in}}%
\pgfpathlineto{\pgfqpoint{4.007973in}{1.735170in}}%
\pgfpathlineto{\pgfqpoint{4.021975in}{1.739730in}}%
\pgfpathlineto{\pgfqpoint{4.035987in}{1.744474in}}%
\pgfpathlineto{\pgfqpoint{4.044100in}{1.758078in}}%
\pgfpathlineto{\pgfqpoint{4.052207in}{1.771666in}}%
\pgfpathlineto{\pgfqpoint{4.060310in}{1.785236in}}%
\pgfpathlineto{\pgfqpoint{4.068409in}{1.798784in}}%
\pgfpathlineto{\pgfqpoint{4.054399in}{1.793661in}}%
\pgfpathlineto{\pgfqpoint{4.040400in}{1.788722in}}%
\pgfpathlineto{\pgfqpoint{4.026413in}{1.783968in}}%
\pgfpathlineto{\pgfqpoint{4.012437in}{1.779398in}}%
\pgfpathlineto{\pgfqpoint{4.004335in}{1.766217in}}%
\pgfpathlineto{\pgfqpoint{3.996229in}{1.753022in}}%
\pgfpathlineto{\pgfqpoint{3.988118in}{1.739816in}}%
\pgfpathlineto{\pgfqpoint{3.980003in}{1.726603in}}%
\pgfpathclose%
\pgfusepath{fill}%
\end{pgfscope}%
\begin{pgfscope}%
\pgfpathrectangle{\pgfqpoint{1.150000in}{0.150000in}}{\pgfqpoint{5.700000in}{5.700000in}}%
\pgfusepath{clip}%
\pgfsetbuttcap%
\pgfsetroundjoin%
\definecolor{currentfill}{rgb}{0.201239,0.383670,0.554294}%
\pgfsetfillcolor{currentfill}%
\pgfsetfillopacity{0.800000}%
\pgfsetlinewidth{0.000000pt}%
\definecolor{currentstroke}{rgb}{0.000000,0.000000,0.000000}%
\pgfsetstrokecolor{currentstroke}%
\pgfsetdash{}{0pt}%
\pgfpathmoveto{\pgfqpoint{4.519080in}{2.293159in}}%
\pgfpathlineto{\pgfqpoint{4.533318in}{2.303174in}}%
\pgfpathlineto{\pgfqpoint{4.547572in}{2.313373in}}%
\pgfpathlineto{\pgfqpoint{4.561842in}{2.323757in}}%
\pgfpathlineto{\pgfqpoint{4.576127in}{2.334326in}}%
\pgfpathlineto{\pgfqpoint{4.584082in}{2.346798in}}%
\pgfpathlineto{\pgfqpoint{4.592032in}{2.359144in}}%
\pgfpathlineto{\pgfqpoint{4.599975in}{2.371362in}}%
\pgfpathlineto{\pgfqpoint{4.607912in}{2.383452in}}%
\pgfpathlineto{\pgfqpoint{4.593626in}{2.372763in}}%
\pgfpathlineto{\pgfqpoint{4.579357in}{2.362258in}}%
\pgfpathlineto{\pgfqpoint{4.565103in}{2.351937in}}%
\pgfpathlineto{\pgfqpoint{4.550865in}{2.341802in}}%
\pgfpathlineto{\pgfqpoint{4.542928in}{2.329820in}}%
\pgfpathlineto{\pgfqpoint{4.534984in}{2.317718in}}%
\pgfpathlineto{\pgfqpoint{4.527035in}{2.305497in}}%
\pgfpathlineto{\pgfqpoint{4.519080in}{2.293159in}}%
\pgfpathclose%
\pgfusepath{fill}%
\end{pgfscope}%
\begin{pgfscope}%
\pgfpathrectangle{\pgfqpoint{1.150000in}{0.150000in}}{\pgfqpoint{5.700000in}{5.700000in}}%
\pgfusepath{clip}%
\pgfsetbuttcap%
\pgfsetroundjoin%
\definecolor{currentfill}{rgb}{0.260571,0.246922,0.522828}%
\pgfsetfillcolor{currentfill}%
\pgfsetfillopacity{0.800000}%
\pgfsetlinewidth{0.000000pt}%
\definecolor{currentstroke}{rgb}{0.000000,0.000000,0.000000}%
\pgfsetstrokecolor{currentstroke}%
\pgfsetdash{}{0pt}%
\pgfpathmoveto{\pgfqpoint{4.189179in}{1.930980in}}%
\pgfpathlineto{\pgfqpoint{4.203246in}{1.937687in}}%
\pgfpathlineto{\pgfqpoint{4.217326in}{1.944578in}}%
\pgfpathlineto{\pgfqpoint{4.231419in}{1.951652in}}%
\pgfpathlineto{\pgfqpoint{4.245525in}{1.958911in}}%
\pgfpathlineto{\pgfqpoint{4.253581in}{1.972714in}}%
\pgfpathlineto{\pgfqpoint{4.261632in}{1.986448in}}%
\pgfpathlineto{\pgfqpoint{4.269678in}{2.000113in}}%
\pgfpathlineto{\pgfqpoint{4.277720in}{2.013706in}}%
\pgfpathlineto{\pgfqpoint{4.263614in}{2.006162in}}%
\pgfpathlineto{\pgfqpoint{4.249522in}{1.998803in}}%
\pgfpathlineto{\pgfqpoint{4.235442in}{1.991628in}}%
\pgfpathlineto{\pgfqpoint{4.221376in}{1.984638in}}%
\pgfpathlineto{\pgfqpoint{4.213334in}{1.971317in}}%
\pgfpathlineto{\pgfqpoint{4.205287in}{1.957933in}}%
\pgfpathlineto{\pgfqpoint{4.197235in}{1.944486in}}%
\pgfpathlineto{\pgfqpoint{4.189179in}{1.930980in}}%
\pgfpathclose%
\pgfusepath{fill}%
\end{pgfscope}%
\begin{pgfscope}%
\pgfpathrectangle{\pgfqpoint{1.150000in}{0.150000in}}{\pgfqpoint{5.700000in}{5.700000in}}%
\pgfusepath{clip}%
\pgfsetbuttcap%
\pgfsetroundjoin%
\definecolor{currentfill}{rgb}{0.267004,0.004874,0.329415}%
\pgfsetfillcolor{currentfill}%
\pgfsetfillopacity{0.800000}%
\pgfsetlinewidth{0.000000pt}%
\definecolor{currentstroke}{rgb}{0.000000,0.000000,0.000000}%
\pgfsetstrokecolor{currentstroke}%
\pgfsetdash{}{0pt}%
\pgfpathmoveto{\pgfqpoint{3.360680in}{1.427427in}}%
\pgfpathlineto{\pgfqpoint{3.374544in}{1.422824in}}%
\pgfpathlineto{\pgfqpoint{3.388411in}{1.418415in}}%
\pgfpathlineto{\pgfqpoint{3.402282in}{1.414201in}}%
\pgfpathlineto{\pgfqpoint{3.416158in}{1.410181in}}%
\pgfpathlineto{\pgfqpoint{3.424513in}{1.418199in}}%
\pgfpathlineto{\pgfqpoint{3.432860in}{1.426387in}}%
\pgfpathlineto{\pgfqpoint{3.441199in}{1.434742in}}%
\pgfpathlineto{\pgfqpoint{3.449529in}{1.443256in}}%
\pgfpathlineto{\pgfqpoint{3.435674in}{1.446681in}}%
\pgfpathlineto{\pgfqpoint{3.421824in}{1.450299in}}%
\pgfpathlineto{\pgfqpoint{3.407978in}{1.454112in}}%
\pgfpathlineto{\pgfqpoint{3.394136in}{1.458121in}}%
\pgfpathlineto{\pgfqpoint{3.385785in}{1.450190in}}%
\pgfpathlineto{\pgfqpoint{3.377426in}{1.442427in}}%
\pgfpathlineto{\pgfqpoint{3.369057in}{1.434837in}}%
\pgfpathlineto{\pgfqpoint{3.360680in}{1.427427in}}%
\pgfpathclose%
\pgfusepath{fill}%
\end{pgfscope}%
\begin{pgfscope}%
\pgfpathrectangle{\pgfqpoint{1.150000in}{0.150000in}}{\pgfqpoint{5.700000in}{5.700000in}}%
\pgfusepath{clip}%
\pgfsetbuttcap%
\pgfsetroundjoin%
\definecolor{currentfill}{rgb}{0.232815,0.732247,0.459277}%
\pgfsetfillcolor{currentfill}%
\pgfsetfillopacity{0.800000}%
\pgfsetlinewidth{0.000000pt}%
\definecolor{currentstroke}{rgb}{0.000000,0.000000,0.000000}%
\pgfsetstrokecolor{currentstroke}%
\pgfsetdash{}{0pt}%
\pgfpathmoveto{\pgfqpoint{5.653594in}{3.386260in}}%
\pgfpathlineto{\pgfqpoint{5.668560in}{3.402203in}}%
\pgfpathlineto{\pgfqpoint{5.683549in}{3.418331in}}%
\pgfpathlineto{\pgfqpoint{5.698561in}{3.434645in}}%
\pgfpathlineto{\pgfqpoint{5.713596in}{3.451145in}}%
\pgfpathlineto{\pgfqpoint{5.720901in}{3.452925in}}%
\pgfpathlineto{\pgfqpoint{5.728197in}{3.454612in}}%
\pgfpathlineto{\pgfqpoint{5.735482in}{3.456208in}}%
\pgfpathlineto{\pgfqpoint{5.742758in}{3.457719in}}%
\pgfpathlineto{\pgfqpoint{5.727748in}{3.441663in}}%
\pgfpathlineto{\pgfqpoint{5.712760in}{3.425791in}}%
\pgfpathlineto{\pgfqpoint{5.697795in}{3.410104in}}%
\pgfpathlineto{\pgfqpoint{5.682853in}{3.394601in}}%
\pgfpathlineto{\pgfqpoint{5.675552in}{3.392636in}}%
\pgfpathlineto{\pgfqpoint{5.668242in}{3.390594in}}%
\pgfpathlineto{\pgfqpoint{5.660923in}{3.388470in}}%
\pgfpathlineto{\pgfqpoint{5.653594in}{3.386260in}}%
\pgfpathclose%
\pgfusepath{fill}%
\end{pgfscope}%
\begin{pgfscope}%
\pgfpathrectangle{\pgfqpoint{1.150000in}{0.150000in}}{\pgfqpoint{5.700000in}{5.700000in}}%
\pgfusepath{clip}%
\pgfsetbuttcap%
\pgfsetroundjoin%
\definecolor{currentfill}{rgb}{0.271828,0.209303,0.504434}%
\pgfsetfillcolor{currentfill}%
\pgfsetfillopacity{0.800000}%
\pgfsetlinewidth{0.000000pt}%
\definecolor{currentstroke}{rgb}{0.000000,0.000000,0.000000}%
\pgfsetstrokecolor{currentstroke}%
\pgfsetdash{}{0pt}%
\pgfpathmoveto{\pgfqpoint{2.589067in}{1.918856in}}%
\pgfpathlineto{\pgfqpoint{2.603129in}{1.901701in}}%
\pgfpathlineto{\pgfqpoint{2.617183in}{1.884799in}}%
\pgfpathlineto{\pgfqpoint{2.631230in}{1.868147in}}%
\pgfpathlineto{\pgfqpoint{2.645270in}{1.851745in}}%
\pgfpathlineto{\pgfqpoint{2.654224in}{1.848628in}}%
\pgfpathlineto{\pgfqpoint{2.663157in}{1.845864in}}%
\pgfpathlineto{\pgfqpoint{2.672071in}{1.843447in}}%
\pgfpathlineto{\pgfqpoint{2.680965in}{1.841370in}}%
\pgfpathlineto{\pgfqpoint{2.666977in}{1.857052in}}%
\pgfpathlineto{\pgfqpoint{2.652982in}{1.872983in}}%
\pgfpathlineto{\pgfqpoint{2.638981in}{1.889163in}}%
\pgfpathlineto{\pgfqpoint{2.624972in}{1.905595in}}%
\pgfpathlineto{\pgfqpoint{2.616027in}{1.908380in}}%
\pgfpathlineto{\pgfqpoint{2.607061in}{1.911513in}}%
\pgfpathlineto{\pgfqpoint{2.598075in}{1.915003in}}%
\pgfpathlineto{\pgfqpoint{2.589067in}{1.918856in}}%
\pgfpathclose%
\pgfusepath{fill}%
\end{pgfscope}%
\begin{pgfscope}%
\pgfpathrectangle{\pgfqpoint{1.150000in}{0.150000in}}{\pgfqpoint{5.700000in}{5.700000in}}%
\pgfusepath{clip}%
\pgfsetbuttcap%
\pgfsetroundjoin%
\definecolor{currentfill}{rgb}{0.278012,0.180367,0.486697}%
\pgfsetfillcolor{currentfill}%
\pgfsetfillopacity{0.800000}%
\pgfsetlinewidth{0.000000pt}%
\definecolor{currentstroke}{rgb}{0.000000,0.000000,0.000000}%
\pgfsetstrokecolor{currentstroke}%
\pgfsetdash{}{0pt}%
\pgfpathmoveto{\pgfqpoint{2.645270in}{1.851745in}}%
\pgfpathlineto{\pgfqpoint{2.659303in}{1.835590in}}%
\pgfpathlineto{\pgfqpoint{2.673329in}{1.819680in}}%
\pgfpathlineto{\pgfqpoint{2.687349in}{1.804015in}}%
\pgfpathlineto{\pgfqpoint{2.701363in}{1.788591in}}%
\pgfpathlineto{\pgfqpoint{2.710266in}{1.786205in}}%
\pgfpathlineto{\pgfqpoint{2.719149in}{1.784164in}}%
\pgfpathlineto{\pgfqpoint{2.728013in}{1.782460in}}%
\pgfpathlineto{\pgfqpoint{2.736858in}{1.781086in}}%
\pgfpathlineto{\pgfqpoint{2.722894in}{1.795793in}}%
\pgfpathlineto{\pgfqpoint{2.708923in}{1.810742in}}%
\pgfpathlineto{\pgfqpoint{2.694947in}{1.825934in}}%
\pgfpathlineto{\pgfqpoint{2.680965in}{1.841370in}}%
\pgfpathlineto{\pgfqpoint{2.672071in}{1.843447in}}%
\pgfpathlineto{\pgfqpoint{2.663157in}{1.845864in}}%
\pgfpathlineto{\pgfqpoint{2.654224in}{1.848628in}}%
\pgfpathlineto{\pgfqpoint{2.645270in}{1.851745in}}%
\pgfpathclose%
\pgfusepath{fill}%
\end{pgfscope}%
\begin{pgfscope}%
\pgfpathrectangle{\pgfqpoint{1.150000in}{0.150000in}}{\pgfqpoint{5.700000in}{5.700000in}}%
\pgfusepath{clip}%
\pgfsetbuttcap%
\pgfsetroundjoin%
\definecolor{currentfill}{rgb}{0.175841,0.441290,0.557685}%
\pgfsetfillcolor{currentfill}%
\pgfsetfillopacity{0.800000}%
\pgfsetlinewidth{0.000000pt}%
\definecolor{currentstroke}{rgb}{0.000000,0.000000,0.000000}%
\pgfsetstrokecolor{currentstroke}%
\pgfsetdash{}{0pt}%
\pgfpathmoveto{\pgfqpoint{2.171391in}{2.576643in}}%
\pgfpathlineto{\pgfqpoint{2.185771in}{2.551008in}}%
\pgfpathlineto{\pgfqpoint{2.200133in}{2.525702in}}%
\pgfpathlineto{\pgfqpoint{2.214480in}{2.500721in}}%
\pgfpathlineto{\pgfqpoint{2.228811in}{2.476063in}}%
\pgfpathlineto{\pgfqpoint{2.238151in}{2.468589in}}%
\pgfpathlineto{\pgfqpoint{2.247466in}{2.461517in}}%
\pgfpathlineto{\pgfqpoint{2.256754in}{2.454840in}}%
\pgfpathlineto{\pgfqpoint{2.266018in}{2.448550in}}%
\pgfpathlineto{\pgfqpoint{2.251753in}{2.472482in}}%
\pgfpathlineto{\pgfqpoint{2.237474in}{2.496734in}}%
\pgfpathlineto{\pgfqpoint{2.223179in}{2.521309in}}%
\pgfpathlineto{\pgfqpoint{2.208868in}{2.546212in}}%
\pgfpathlineto{\pgfqpoint{2.199539in}{2.553216in}}%
\pgfpathlineto{\pgfqpoint{2.190183in}{2.560617in}}%
\pgfpathlineto{\pgfqpoint{2.180801in}{2.568424in}}%
\pgfpathlineto{\pgfqpoint{2.171391in}{2.576643in}}%
\pgfpathclose%
\pgfusepath{fill}%
\end{pgfscope}%
\begin{pgfscope}%
\pgfpathrectangle{\pgfqpoint{1.150000in}{0.150000in}}{\pgfqpoint{5.700000in}{5.700000in}}%
\pgfusepath{clip}%
\pgfsetbuttcap%
\pgfsetroundjoin%
\definecolor{currentfill}{rgb}{0.263663,0.237631,0.518762}%
\pgfsetfillcolor{currentfill}%
\pgfsetfillopacity{0.800000}%
\pgfsetlinewidth{0.000000pt}%
\definecolor{currentstroke}{rgb}{0.000000,0.000000,0.000000}%
\pgfsetstrokecolor{currentstroke}%
\pgfsetdash{}{0pt}%
\pgfpathmoveto{\pgfqpoint{2.532740in}{1.990037in}}%
\pgfpathlineto{\pgfqpoint{2.546834in}{1.971853in}}%
\pgfpathlineto{\pgfqpoint{2.560920in}{1.953930in}}%
\pgfpathlineto{\pgfqpoint{2.574998in}{1.936265in}}%
\pgfpathlineto{\pgfqpoint{2.589067in}{1.918856in}}%
\pgfpathlineto{\pgfqpoint{2.598075in}{1.915003in}}%
\pgfpathlineto{\pgfqpoint{2.607061in}{1.911513in}}%
\pgfpathlineto{\pgfqpoint{2.616027in}{1.908380in}}%
\pgfpathlineto{\pgfqpoint{2.624972in}{1.905595in}}%
\pgfpathlineto{\pgfqpoint{2.610956in}{1.922279in}}%
\pgfpathlineto{\pgfqpoint{2.596933in}{1.939219in}}%
\pgfpathlineto{\pgfqpoint{2.582902in}{1.956416in}}%
\pgfpathlineto{\pgfqpoint{2.568863in}{1.973872in}}%
\pgfpathlineto{\pgfqpoint{2.559864in}{1.977368in}}%
\pgfpathlineto{\pgfqpoint{2.550845in}{1.981223in}}%
\pgfpathlineto{\pgfqpoint{2.541803in}{1.985443in}}%
\pgfpathlineto{\pgfqpoint{2.532740in}{1.990037in}}%
\pgfpathclose%
\pgfusepath{fill}%
\end{pgfscope}%
\begin{pgfscope}%
\pgfpathrectangle{\pgfqpoint{1.150000in}{0.150000in}}{\pgfqpoint{5.700000in}{5.700000in}}%
\pgfusepath{clip}%
\pgfsetbuttcap%
\pgfsetroundjoin%
\definecolor{currentfill}{rgb}{0.281412,0.155834,0.469201}%
\pgfsetfillcolor{currentfill}%
\pgfsetfillopacity{0.800000}%
\pgfsetlinewidth{0.000000pt}%
\definecolor{currentstroke}{rgb}{0.000000,0.000000,0.000000}%
\pgfsetstrokecolor{currentstroke}%
\pgfsetdash{}{0pt}%
\pgfpathmoveto{\pgfqpoint{2.701363in}{1.788591in}}%
\pgfpathlineto{\pgfqpoint{2.715371in}{1.773409in}}%
\pgfpathlineto{\pgfqpoint{2.729374in}{1.758465in}}%
\pgfpathlineto{\pgfqpoint{2.743371in}{1.743759in}}%
\pgfpathlineto{\pgfqpoint{2.757362in}{1.729289in}}%
\pgfpathlineto{\pgfqpoint{2.766215in}{1.727630in}}%
\pgfpathlineto{\pgfqpoint{2.775050in}{1.726306in}}%
\pgfpathlineto{\pgfqpoint{2.783867in}{1.725311in}}%
\pgfpathlineto{\pgfqpoint{2.792665in}{1.724637in}}%
\pgfpathlineto{\pgfqpoint{2.778721in}{1.738395in}}%
\pgfpathlineto{\pgfqpoint{2.764772in}{1.752388in}}%
\pgfpathlineto{\pgfqpoint{2.750817in}{1.766618in}}%
\pgfpathlineto{\pgfqpoint{2.736858in}{1.781086in}}%
\pgfpathlineto{\pgfqpoint{2.728013in}{1.782460in}}%
\pgfpathlineto{\pgfqpoint{2.719149in}{1.784164in}}%
\pgfpathlineto{\pgfqpoint{2.710266in}{1.786205in}}%
\pgfpathlineto{\pgfqpoint{2.701363in}{1.788591in}}%
\pgfpathclose%
\pgfusepath{fill}%
\end{pgfscope}%
\begin{pgfscope}%
\pgfpathrectangle{\pgfqpoint{1.150000in}{0.150000in}}{\pgfqpoint{5.700000in}{5.700000in}}%
\pgfusepath{clip}%
\pgfsetbuttcap%
\pgfsetroundjoin%
\definecolor{currentfill}{rgb}{0.253935,0.265254,0.529983}%
\pgfsetfillcolor{currentfill}%
\pgfsetfillopacity{0.800000}%
\pgfsetlinewidth{0.000000pt}%
\definecolor{currentstroke}{rgb}{0.000000,0.000000,0.000000}%
\pgfsetstrokecolor{currentstroke}%
\pgfsetdash{}{0pt}%
\pgfpathmoveto{\pgfqpoint{2.476271in}{2.065410in}}%
\pgfpathlineto{\pgfqpoint{2.490402in}{2.046167in}}%
\pgfpathlineto{\pgfqpoint{2.504524in}{2.027191in}}%
\pgfpathlineto{\pgfqpoint{2.518636in}{2.008482in}}%
\pgfpathlineto{\pgfqpoint{2.532740in}{1.990037in}}%
\pgfpathlineto{\pgfqpoint{2.541803in}{1.985443in}}%
\pgfpathlineto{\pgfqpoint{2.550845in}{1.981223in}}%
\pgfpathlineto{\pgfqpoint{2.559864in}{1.977368in}}%
\pgfpathlineto{\pgfqpoint{2.568863in}{1.973872in}}%
\pgfpathlineto{\pgfqpoint{2.554816in}{1.991588in}}%
\pgfpathlineto{\pgfqpoint{2.540760in}{2.009567in}}%
\pgfpathlineto{\pgfqpoint{2.526695in}{2.027811in}}%
\pgfpathlineto{\pgfqpoint{2.512622in}{2.046321in}}%
\pgfpathlineto{\pgfqpoint{2.503568in}{2.050535in}}%
\pgfpathlineto{\pgfqpoint{2.494491in}{2.055116in}}%
\pgfpathlineto{\pgfqpoint{2.485393in}{2.060072in}}%
\pgfpathlineto{\pgfqpoint{2.476271in}{2.065410in}}%
\pgfpathclose%
\pgfusepath{fill}%
\end{pgfscope}%
\begin{pgfscope}%
\pgfpathrectangle{\pgfqpoint{1.150000in}{0.150000in}}{\pgfqpoint{5.700000in}{5.700000in}}%
\pgfusepath{clip}%
\pgfsetbuttcap%
\pgfsetroundjoin%
\definecolor{currentfill}{rgb}{0.134692,0.658636,0.517649}%
\pgfsetfillcolor{currentfill}%
\pgfsetfillopacity{0.800000}%
\pgfsetlinewidth{0.000000pt}%
\definecolor{currentstroke}{rgb}{0.000000,0.000000,0.000000}%
\pgfsetstrokecolor{currentstroke}%
\pgfsetdash{}{0pt}%
\pgfpathmoveto{\pgfqpoint{5.356208in}{3.141725in}}%
\pgfpathlineto{\pgfqpoint{5.370983in}{3.156849in}}%
\pgfpathlineto{\pgfqpoint{5.385778in}{3.172159in}}%
\pgfpathlineto{\pgfqpoint{5.400596in}{3.187655in}}%
\pgfpathlineto{\pgfqpoint{5.415435in}{3.203337in}}%
\pgfpathlineto{\pgfqpoint{5.422959in}{3.208097in}}%
\pgfpathlineto{\pgfqpoint{5.430474in}{3.212722in}}%
\pgfpathlineto{\pgfqpoint{5.437978in}{3.217215in}}%
\pgfpathlineto{\pgfqpoint{5.445473in}{3.221580in}}%
\pgfpathlineto{\pgfqpoint{5.430649in}{3.206197in}}%
\pgfpathlineto{\pgfqpoint{5.415847in}{3.190999in}}%
\pgfpathlineto{\pgfqpoint{5.401066in}{3.175987in}}%
\pgfpathlineto{\pgfqpoint{5.386307in}{3.161159in}}%
\pgfpathlineto{\pgfqpoint{5.378796in}{3.156484in}}%
\pgfpathlineto{\pgfqpoint{5.371276in}{3.151689in}}%
\pgfpathlineto{\pgfqpoint{5.363747in}{3.146770in}}%
\pgfpathlineto{\pgfqpoint{5.356208in}{3.141725in}}%
\pgfpathclose%
\pgfusepath{fill}%
\end{pgfscope}%
\begin{pgfscope}%
\pgfpathrectangle{\pgfqpoint{1.150000in}{0.150000in}}{\pgfqpoint{5.700000in}{5.700000in}}%
\pgfusepath{clip}%
\pgfsetbuttcap%
\pgfsetroundjoin%
\definecolor{currentfill}{rgb}{0.277018,0.050344,0.375715}%
\pgfsetfillcolor{currentfill}%
\pgfsetfillopacity{0.800000}%
\pgfsetlinewidth{0.000000pt}%
\definecolor{currentstroke}{rgb}{0.000000,0.000000,0.000000}%
\pgfsetstrokecolor{currentstroke}%
\pgfsetdash{}{0pt}%
\pgfpathmoveto{\pgfqpoint{3.682174in}{1.503414in}}%
\pgfpathlineto{\pgfqpoint{3.696075in}{1.503556in}}%
\pgfpathlineto{\pgfqpoint{3.709983in}{1.503884in}}%
\pgfpathlineto{\pgfqpoint{3.723899in}{1.504399in}}%
\pgfpathlineto{\pgfqpoint{3.737823in}{1.505100in}}%
\pgfpathlineto{\pgfqpoint{3.746032in}{1.516858in}}%
\pgfpathlineto{\pgfqpoint{3.754236in}{1.528691in}}%
\pgfpathlineto{\pgfqpoint{3.762435in}{1.540595in}}%
\pgfpathlineto{\pgfqpoint{3.770627in}{1.552565in}}%
\pgfpathlineto{\pgfqpoint{3.756713in}{1.551361in}}%
\pgfpathlineto{\pgfqpoint{3.742808in}{1.550344in}}%
\pgfpathlineto{\pgfqpoint{3.728910in}{1.549514in}}%
\pgfpathlineto{\pgfqpoint{3.715020in}{1.548871in}}%
\pgfpathlineto{\pgfqpoint{3.706817in}{1.537392in}}%
\pgfpathlineto{\pgfqpoint{3.698609in}{1.525986in}}%
\pgfpathlineto{\pgfqpoint{3.690395in}{1.514658in}}%
\pgfpathlineto{\pgfqpoint{3.682174in}{1.503414in}}%
\pgfpathclose%
\pgfusepath{fill}%
\end{pgfscope}%
\begin{pgfscope}%
\pgfpathrectangle{\pgfqpoint{1.150000in}{0.150000in}}{\pgfqpoint{5.700000in}{5.700000in}}%
\pgfusepath{clip}%
\pgfsetbuttcap%
\pgfsetroundjoin%
\definecolor{currentfill}{rgb}{0.273809,0.031497,0.358853}%
\pgfsetfillcolor{currentfill}%
\pgfsetfillopacity{0.800000}%
\pgfsetlinewidth{0.000000pt}%
\definecolor{currentstroke}{rgb}{0.000000,0.000000,0.000000}%
\pgfsetstrokecolor{currentstroke}%
\pgfsetdash{}{0pt}%
\pgfpathmoveto{\pgfqpoint{3.070904in}{1.496777in}}%
\pgfpathlineto{\pgfqpoint{3.084799in}{1.487656in}}%
\pgfpathlineto{\pgfqpoint{3.098694in}{1.478743in}}%
\pgfpathlineto{\pgfqpoint{3.112589in}{1.470038in}}%
\pgfpathlineto{\pgfqpoint{3.126484in}{1.461539in}}%
\pgfpathlineto{\pgfqpoint{3.135032in}{1.465268in}}%
\pgfpathlineto{\pgfqpoint{3.143568in}{1.469252in}}%
\pgfpathlineto{\pgfqpoint{3.152091in}{1.473484in}}%
\pgfpathlineto{\pgfqpoint{3.160603in}{1.477958in}}%
\pgfpathlineto{\pgfqpoint{3.146740in}{1.485794in}}%
\pgfpathlineto{\pgfqpoint{3.132878in}{1.493835in}}%
\pgfpathlineto{\pgfqpoint{3.119016in}{1.502084in}}%
\pgfpathlineto{\pgfqpoint{3.105155in}{1.510541in}}%
\pgfpathlineto{\pgfqpoint{3.096612in}{1.506719in}}%
\pgfpathlineto{\pgfqpoint{3.088056in}{1.503146in}}%
\pgfpathlineto{\pgfqpoint{3.079487in}{1.499830in}}%
\pgfpathlineto{\pgfqpoint{3.070904in}{1.496777in}}%
\pgfpathclose%
\pgfusepath{fill}%
\end{pgfscope}%
\begin{pgfscope}%
\pgfpathrectangle{\pgfqpoint{1.150000in}{0.150000in}}{\pgfqpoint{5.700000in}{5.700000in}}%
\pgfusepath{clip}%
\pgfsetbuttcap%
\pgfsetroundjoin%
\definecolor{currentfill}{rgb}{0.223925,0.334994,0.548053}%
\pgfsetfillcolor{currentfill}%
\pgfsetfillopacity{0.800000}%
\pgfsetlinewidth{0.000000pt}%
\definecolor{currentstroke}{rgb}{0.000000,0.000000,0.000000}%
\pgfsetstrokecolor{currentstroke}%
\pgfsetdash{}{0pt}%
\pgfpathmoveto{\pgfqpoint{4.398446in}{2.153474in}}%
\pgfpathlineto{\pgfqpoint{4.412623in}{2.162410in}}%
\pgfpathlineto{\pgfqpoint{4.426814in}{2.171530in}}%
\pgfpathlineto{\pgfqpoint{4.441020in}{2.180835in}}%
\pgfpathlineto{\pgfqpoint{4.455242in}{2.190323in}}%
\pgfpathlineto{\pgfqpoint{4.463241in}{2.203567in}}%
\pgfpathlineto{\pgfqpoint{4.471234in}{2.216703in}}%
\pgfpathlineto{\pgfqpoint{4.479222in}{2.229728in}}%
\pgfpathlineto{\pgfqpoint{4.487205in}{2.242642in}}%
\pgfpathlineto{\pgfqpoint{4.472982in}{2.232966in}}%
\pgfpathlineto{\pgfqpoint{4.458775in}{2.223474in}}%
\pgfpathlineto{\pgfqpoint{4.444584in}{2.214167in}}%
\pgfpathlineto{\pgfqpoint{4.430407in}{2.205044in}}%
\pgfpathlineto{\pgfqpoint{4.422424in}{2.192305in}}%
\pgfpathlineto{\pgfqpoint{4.414437in}{2.179462in}}%
\pgfpathlineto{\pgfqpoint{4.406444in}{2.166518in}}%
\pgfpathlineto{\pgfqpoint{4.398446in}{2.153474in}}%
\pgfpathclose%
\pgfusepath{fill}%
\end{pgfscope}%
\begin{pgfscope}%
\pgfpathrectangle{\pgfqpoint{1.150000in}{0.150000in}}{\pgfqpoint{5.700000in}{5.700000in}}%
\pgfusepath{clip}%
\pgfsetbuttcap%
\pgfsetroundjoin%
\definecolor{currentfill}{rgb}{0.272594,0.025563,0.353093}%
\pgfsetfillcolor{currentfill}%
\pgfsetfillopacity{0.800000}%
\pgfsetlinewidth{0.000000pt}%
\definecolor{currentstroke}{rgb}{0.000000,0.000000,0.000000}%
\pgfsetstrokecolor{currentstroke}%
\pgfsetdash{}{0pt}%
\pgfpathmoveto{\pgfqpoint{3.593647in}{1.462812in}}%
\pgfpathlineto{\pgfqpoint{3.607533in}{1.461669in}}%
\pgfpathlineto{\pgfqpoint{3.621426in}{1.460714in}}%
\pgfpathlineto{\pgfqpoint{3.635325in}{1.459948in}}%
\pgfpathlineto{\pgfqpoint{3.649231in}{1.459369in}}%
\pgfpathlineto{\pgfqpoint{3.657476in}{1.470231in}}%
\pgfpathlineto{\pgfqpoint{3.665715in}{1.481196in}}%
\pgfpathlineto{\pgfqpoint{3.673948in}{1.492258in}}%
\pgfpathlineto{\pgfqpoint{3.682174in}{1.503414in}}%
\pgfpathlineto{\pgfqpoint{3.668281in}{1.503460in}}%
\pgfpathlineto{\pgfqpoint{3.654394in}{1.503694in}}%
\pgfpathlineto{\pgfqpoint{3.640515in}{1.504116in}}%
\pgfpathlineto{\pgfqpoint{3.626642in}{1.504726in}}%
\pgfpathlineto{\pgfqpoint{3.618403in}{1.494091in}}%
\pgfpathlineto{\pgfqpoint{3.610158in}{1.483557in}}%
\pgfpathlineto{\pgfqpoint{3.601906in}{1.473129in}}%
\pgfpathlineto{\pgfqpoint{3.593647in}{1.462812in}}%
\pgfpathclose%
\pgfusepath{fill}%
\end{pgfscope}%
\begin{pgfscope}%
\pgfpathrectangle{\pgfqpoint{1.150000in}{0.150000in}}{\pgfqpoint{5.700000in}{5.700000in}}%
\pgfusepath{clip}%
\pgfsetbuttcap%
\pgfsetroundjoin%
\definecolor{currentfill}{rgb}{0.283072,0.130895,0.449241}%
\pgfsetfillcolor{currentfill}%
\pgfsetfillopacity{0.800000}%
\pgfsetlinewidth{0.000000pt}%
\definecolor{currentstroke}{rgb}{0.000000,0.000000,0.000000}%
\pgfsetstrokecolor{currentstroke}%
\pgfsetdash{}{0pt}%
\pgfpathmoveto{\pgfqpoint{2.757362in}{1.729289in}}%
\pgfpathlineto{\pgfqpoint{2.771349in}{1.715053in}}%
\pgfpathlineto{\pgfqpoint{2.785331in}{1.701050in}}%
\pgfpathlineto{\pgfqpoint{2.799309in}{1.687279in}}%
\pgfpathlineto{\pgfqpoint{2.813282in}{1.673739in}}%
\pgfpathlineto{\pgfqpoint{2.822088in}{1.672803in}}%
\pgfpathlineto{\pgfqpoint{2.830876in}{1.672195in}}%
\pgfpathlineto{\pgfqpoint{2.839647in}{1.671905in}}%
\pgfpathlineto{\pgfqpoint{2.848401in}{1.671927in}}%
\pgfpathlineto{\pgfqpoint{2.834473in}{1.684759in}}%
\pgfpathlineto{\pgfqpoint{2.820541in}{1.697821in}}%
\pgfpathlineto{\pgfqpoint{2.806605in}{1.711113in}}%
\pgfpathlineto{\pgfqpoint{2.792665in}{1.724637in}}%
\pgfpathlineto{\pgfqpoint{2.783867in}{1.725311in}}%
\pgfpathlineto{\pgfqpoint{2.775050in}{1.726306in}}%
\pgfpathlineto{\pgfqpoint{2.766215in}{1.727630in}}%
\pgfpathlineto{\pgfqpoint{2.757362in}{1.729289in}}%
\pgfpathclose%
\pgfusepath{fill}%
\end{pgfscope}%
\begin{pgfscope}%
\pgfpathrectangle{\pgfqpoint{1.150000in}{0.150000in}}{\pgfqpoint{5.700000in}{5.700000in}}%
\pgfusepath{clip}%
\pgfsetbuttcap%
\pgfsetroundjoin%
\definecolor{currentfill}{rgb}{0.168126,0.459988,0.558082}%
\pgfsetfillcolor{currentfill}%
\pgfsetfillopacity{0.800000}%
\pgfsetlinewidth{0.000000pt}%
\definecolor{currentstroke}{rgb}{0.000000,0.000000,0.000000}%
\pgfsetstrokecolor{currentstroke}%
\pgfsetdash{}{0pt}%
\pgfpathmoveto{\pgfqpoint{4.728490in}{2.520676in}}%
\pgfpathlineto{\pgfqpoint{4.742859in}{2.532431in}}%
\pgfpathlineto{\pgfqpoint{4.757245in}{2.544371in}}%
\pgfpathlineto{\pgfqpoint{4.771648in}{2.556496in}}%
\pgfpathlineto{\pgfqpoint{4.786069in}{2.568807in}}%
\pgfpathlineto{\pgfqpoint{4.793947in}{2.579791in}}%
\pgfpathlineto{\pgfqpoint{4.801817in}{2.590627in}}%
\pgfpathlineto{\pgfqpoint{4.809681in}{2.601316in}}%
\pgfpathlineto{\pgfqpoint{4.817537in}{2.611857in}}%
\pgfpathlineto{\pgfqpoint{4.803118in}{2.599528in}}%
\pgfpathlineto{\pgfqpoint{4.788716in}{2.587383in}}%
\pgfpathlineto{\pgfqpoint{4.774332in}{2.575424in}}%
\pgfpathlineto{\pgfqpoint{4.759966in}{2.563649in}}%
\pgfpathlineto{\pgfqpoint{4.752107in}{2.553114in}}%
\pgfpathlineto{\pgfqpoint{4.744242in}{2.542440in}}%
\pgfpathlineto{\pgfqpoint{4.736369in}{2.531628in}}%
\pgfpathlineto{\pgfqpoint{4.728490in}{2.520676in}}%
\pgfpathclose%
\pgfusepath{fill}%
\end{pgfscope}%
\begin{pgfscope}%
\pgfpathrectangle{\pgfqpoint{1.150000in}{0.150000in}}{\pgfqpoint{5.700000in}{5.700000in}}%
\pgfusepath{clip}%
\pgfsetbuttcap%
\pgfsetroundjoin%
\definecolor{currentfill}{rgb}{0.280894,0.078907,0.402329}%
\pgfsetfillcolor{currentfill}%
\pgfsetfillopacity{0.800000}%
\pgfsetlinewidth{0.000000pt}%
\definecolor{currentstroke}{rgb}{0.000000,0.000000,0.000000}%
\pgfsetstrokecolor{currentstroke}%
\pgfsetdash{}{0pt}%
\pgfpathmoveto{\pgfqpoint{3.770627in}{1.552565in}}%
\pgfpathlineto{\pgfqpoint{3.784550in}{1.553954in}}%
\pgfpathlineto{\pgfqpoint{3.798480in}{1.555529in}}%
\pgfpathlineto{\pgfqpoint{3.812419in}{1.557290in}}%
\pgfpathlineto{\pgfqpoint{3.826367in}{1.559235in}}%
\pgfpathlineto{\pgfqpoint{3.834547in}{1.571749in}}%
\pgfpathlineto{\pgfqpoint{3.842721in}{1.584312in}}%
\pgfpathlineto{\pgfqpoint{3.850890in}{1.596919in}}%
\pgfpathlineto{\pgfqpoint{3.859053in}{1.609567in}}%
\pgfpathlineto{\pgfqpoint{3.845113in}{1.607149in}}%
\pgfpathlineto{\pgfqpoint{3.831181in}{1.604917in}}%
\pgfpathlineto{\pgfqpoint{3.817258in}{1.602871in}}%
\pgfpathlineto{\pgfqpoint{3.803344in}{1.601010in}}%
\pgfpathlineto{\pgfqpoint{3.795173in}{1.588823in}}%
\pgfpathlineto{\pgfqpoint{3.786997in}{1.576683in}}%
\pgfpathlineto{\pgfqpoint{3.778815in}{1.564595in}}%
\pgfpathlineto{\pgfqpoint{3.770627in}{1.552565in}}%
\pgfpathclose%
\pgfusepath{fill}%
\end{pgfscope}%
\begin{pgfscope}%
\pgfpathrectangle{\pgfqpoint{1.150000in}{0.150000in}}{\pgfqpoint{5.700000in}{5.700000in}}%
\pgfusepath{clip}%
\pgfsetbuttcap%
\pgfsetroundjoin%
\definecolor{currentfill}{rgb}{0.281477,0.755203,0.432552}%
\pgfsetfillcolor{currentfill}%
\pgfsetfillopacity{0.800000}%
\pgfsetlinewidth{0.000000pt}%
\definecolor{currentstroke}{rgb}{0.000000,0.000000,0.000000}%
\pgfsetstrokecolor{currentstroke}%
\pgfsetdash{}{0pt}%
\pgfpathmoveto{\pgfqpoint{5.742758in}{3.457719in}}%
\pgfpathlineto{\pgfqpoint{5.757792in}{3.473961in}}%
\pgfpathlineto{\pgfqpoint{5.772849in}{3.490388in}}%
\pgfpathlineto{\pgfqpoint{5.787930in}{3.507001in}}%
\pgfpathlineto{\pgfqpoint{5.803035in}{3.523799in}}%
\pgfpathlineto{\pgfqpoint{5.810275in}{3.524764in}}%
\pgfpathlineto{\pgfqpoint{5.817505in}{3.525646in}}%
\pgfpathlineto{\pgfqpoint{5.824725in}{3.526451in}}%
\pgfpathlineto{\pgfqpoint{5.831936in}{3.527183in}}%
\pgfpathlineto{\pgfqpoint{5.816858in}{3.510864in}}%
\pgfpathlineto{\pgfqpoint{5.801805in}{3.494731in}}%
\pgfpathlineto{\pgfqpoint{5.786774in}{3.478782in}}%
\pgfpathlineto{\pgfqpoint{5.771767in}{3.463017in}}%
\pgfpathlineto{\pgfqpoint{5.764529in}{3.461795in}}%
\pgfpathlineto{\pgfqpoint{5.757281in}{3.460508in}}%
\pgfpathlineto{\pgfqpoint{5.750025in}{3.459151in}}%
\pgfpathlineto{\pgfqpoint{5.742758in}{3.457719in}}%
\pgfpathclose%
\pgfusepath{fill}%
\end{pgfscope}%
\begin{pgfscope}%
\pgfpathrectangle{\pgfqpoint{1.150000in}{0.150000in}}{\pgfqpoint{5.700000in}{5.700000in}}%
\pgfusepath{clip}%
\pgfsetbuttcap%
\pgfsetroundjoin%
\definecolor{currentfill}{rgb}{0.275191,0.194905,0.496005}%
\pgfsetfillcolor{currentfill}%
\pgfsetfillopacity{0.800000}%
\pgfsetlinewidth{0.000000pt}%
\definecolor{currentstroke}{rgb}{0.000000,0.000000,0.000000}%
\pgfsetstrokecolor{currentstroke}%
\pgfsetdash{}{0pt}%
\pgfpathmoveto{\pgfqpoint{4.068409in}{1.798784in}}%
\pgfpathlineto{\pgfqpoint{4.082430in}{1.804091in}}%
\pgfpathlineto{\pgfqpoint{4.096463in}{1.809583in}}%
\pgfpathlineto{\pgfqpoint{4.110508in}{1.815258in}}%
\pgfpathlineto{\pgfqpoint{4.124565in}{1.821116in}}%
\pgfpathlineto{\pgfqpoint{4.132657in}{1.834999in}}%
\pgfpathlineto{\pgfqpoint{4.140745in}{1.848844in}}%
\pgfpathlineto{\pgfqpoint{4.148829in}{1.862650in}}%
\pgfpathlineto{\pgfqpoint{4.156908in}{1.876413in}}%
\pgfpathlineto{\pgfqpoint{4.142852in}{1.870207in}}%
\pgfpathlineto{\pgfqpoint{4.128809in}{1.864184in}}%
\pgfpathlineto{\pgfqpoint{4.114778in}{1.858345in}}%
\pgfpathlineto{\pgfqpoint{4.100758in}{1.852690in}}%
\pgfpathlineto{\pgfqpoint{4.092678in}{1.839263in}}%
\pgfpathlineto{\pgfqpoint{4.084593in}{1.825800in}}%
\pgfpathlineto{\pgfqpoint{4.076503in}{1.812307in}}%
\pgfpathlineto{\pgfqpoint{4.068409in}{1.798784in}}%
\pgfpathclose%
\pgfusepath{fill}%
\end{pgfscope}%
\begin{pgfscope}%
\pgfpathrectangle{\pgfqpoint{1.150000in}{0.150000in}}{\pgfqpoint{5.700000in}{5.700000in}}%
\pgfusepath{clip}%
\pgfsetbuttcap%
\pgfsetroundjoin%
\definecolor{currentfill}{rgb}{0.241237,0.296485,0.539709}%
\pgfsetfillcolor{currentfill}%
\pgfsetfillopacity{0.800000}%
\pgfsetlinewidth{0.000000pt}%
\definecolor{currentstroke}{rgb}{0.000000,0.000000,0.000000}%
\pgfsetstrokecolor{currentstroke}%
\pgfsetdash{}{0pt}%
\pgfpathmoveto{\pgfqpoint{2.419643in}{2.145109in}}%
\pgfpathlineto{\pgfqpoint{2.433816in}{2.124771in}}%
\pgfpathlineto{\pgfqpoint{2.447978in}{2.104710in}}%
\pgfpathlineto{\pgfqpoint{2.462130in}{2.084924in}}%
\pgfpathlineto{\pgfqpoint{2.476271in}{2.065410in}}%
\pgfpathlineto{\pgfqpoint{2.485393in}{2.060072in}}%
\pgfpathlineto{\pgfqpoint{2.494491in}{2.055116in}}%
\pgfpathlineto{\pgfqpoint{2.503568in}{2.050535in}}%
\pgfpathlineto{\pgfqpoint{2.512622in}{2.046321in}}%
\pgfpathlineto{\pgfqpoint{2.498539in}{2.065101in}}%
\pgfpathlineto{\pgfqpoint{2.484446in}{2.084152in}}%
\pgfpathlineto{\pgfqpoint{2.470344in}{2.103475in}}%
\pgfpathlineto{\pgfqpoint{2.456231in}{2.123074in}}%
\pgfpathlineto{\pgfqpoint{2.447119in}{2.128009in}}%
\pgfpathlineto{\pgfqpoint{2.437984in}{2.133322in}}%
\pgfpathlineto{\pgfqpoint{2.428826in}{2.139019in}}%
\pgfpathlineto{\pgfqpoint{2.419643in}{2.145109in}}%
\pgfpathclose%
\pgfusepath{fill}%
\end{pgfscope}%
\begin{pgfscope}%
\pgfpathrectangle{\pgfqpoint{1.150000in}{0.150000in}}{\pgfqpoint{5.700000in}{5.700000in}}%
\pgfusepath{clip}%
\pgfsetbuttcap%
\pgfsetroundjoin%
\definecolor{currentfill}{rgb}{0.269944,0.014625,0.341379}%
\pgfsetfillcolor{currentfill}%
\pgfsetfillopacity{0.800000}%
\pgfsetlinewidth{0.000000pt}%
\definecolor{currentstroke}{rgb}{0.000000,0.000000,0.000000}%
\pgfsetstrokecolor{currentstroke}%
\pgfsetdash{}{0pt}%
\pgfpathmoveto{\pgfqpoint{3.504994in}{1.431482in}}%
\pgfpathlineto{\pgfqpoint{3.518873in}{1.429016in}}%
\pgfpathlineto{\pgfqpoint{3.532757in}{1.426742in}}%
\pgfpathlineto{\pgfqpoint{3.546647in}{1.424657in}}%
\pgfpathlineto{\pgfqpoint{3.560543in}{1.422761in}}%
\pgfpathlineto{\pgfqpoint{3.568829in}{1.432580in}}%
\pgfpathlineto{\pgfqpoint{3.577109in}{1.442532in}}%
\pgfpathlineto{\pgfqpoint{3.585382in}{1.452611in}}%
\pgfpathlineto{\pgfqpoint{3.593647in}{1.462812in}}%
\pgfpathlineto{\pgfqpoint{3.579767in}{1.464143in}}%
\pgfpathlineto{\pgfqpoint{3.565893in}{1.465665in}}%
\pgfpathlineto{\pgfqpoint{3.552025in}{1.467376in}}%
\pgfpathlineto{\pgfqpoint{3.538163in}{1.469277in}}%
\pgfpathlineto{\pgfqpoint{3.529882in}{1.459628in}}%
\pgfpathlineto{\pgfqpoint{3.521594in}{1.450109in}}%
\pgfpathlineto{\pgfqpoint{3.513298in}{1.440725in}}%
\pgfpathlineto{\pgfqpoint{3.504994in}{1.431482in}}%
\pgfpathclose%
\pgfusepath{fill}%
\end{pgfscope}%
\begin{pgfscope}%
\pgfpathrectangle{\pgfqpoint{1.150000in}{0.150000in}}{\pgfqpoint{5.700000in}{5.700000in}}%
\pgfusepath{clip}%
\pgfsetbuttcap%
\pgfsetroundjoin%
\definecolor{currentfill}{rgb}{0.283091,0.110553,0.431554}%
\pgfsetfillcolor{currentfill}%
\pgfsetfillopacity{0.800000}%
\pgfsetlinewidth{0.000000pt}%
\definecolor{currentstroke}{rgb}{0.000000,0.000000,0.000000}%
\pgfsetstrokecolor{currentstroke}%
\pgfsetdash{}{0pt}%
\pgfpathmoveto{\pgfqpoint{2.813282in}{1.673739in}}%
\pgfpathlineto{\pgfqpoint{2.827251in}{1.660427in}}%
\pgfpathlineto{\pgfqpoint{2.841216in}{1.647342in}}%
\pgfpathlineto{\pgfqpoint{2.855177in}{1.634484in}}%
\pgfpathlineto{\pgfqpoint{2.869135in}{1.621850in}}%
\pgfpathlineto{\pgfqpoint{2.877896in}{1.621635in}}%
\pgfpathlineto{\pgfqpoint{2.886641in}{1.621738in}}%
\pgfpathlineto{\pgfqpoint{2.895368in}{1.622151in}}%
\pgfpathlineto{\pgfqpoint{2.904080in}{1.622867in}}%
\pgfpathlineto{\pgfqpoint{2.890165in}{1.634795in}}%
\pgfpathlineto{\pgfqpoint{2.876247in}{1.646946in}}%
\pgfpathlineto{\pgfqpoint{2.862326in}{1.659324in}}%
\pgfpathlineto{\pgfqpoint{2.848401in}{1.671927in}}%
\pgfpathlineto{\pgfqpoint{2.839647in}{1.671905in}}%
\pgfpathlineto{\pgfqpoint{2.830876in}{1.672195in}}%
\pgfpathlineto{\pgfqpoint{2.822088in}{1.672803in}}%
\pgfpathlineto{\pgfqpoint{2.813282in}{1.673739in}}%
\pgfpathclose%
\pgfusepath{fill}%
\end{pgfscope}%
\begin{pgfscope}%
\pgfpathrectangle{\pgfqpoint{1.150000in}{0.150000in}}{\pgfqpoint{5.700000in}{5.700000in}}%
\pgfusepath{clip}%
\pgfsetbuttcap%
\pgfsetroundjoin%
\definecolor{currentfill}{rgb}{0.120092,0.600104,0.542530}%
\pgfsetfillcolor{currentfill}%
\pgfsetfillopacity{0.800000}%
\pgfsetlinewidth{0.000000pt}%
\definecolor{currentstroke}{rgb}{0.000000,0.000000,0.000000}%
\pgfsetstrokecolor{currentstroke}%
\pgfsetdash{}{0pt}%
\pgfpathmoveto{\pgfqpoint{5.147284in}{2.950433in}}%
\pgfpathlineto{\pgfqpoint{5.161926in}{2.964750in}}%
\pgfpathlineto{\pgfqpoint{5.176588in}{2.979253in}}%
\pgfpathlineto{\pgfqpoint{5.191271in}{2.993942in}}%
\pgfpathlineto{\pgfqpoint{5.205974in}{3.008818in}}%
\pgfpathlineto{\pgfqpoint{5.213637in}{3.015797in}}%
\pgfpathlineto{\pgfqpoint{5.221291in}{3.022623in}}%
\pgfpathlineto{\pgfqpoint{5.228936in}{3.029300in}}%
\pgfpathlineto{\pgfqpoint{5.236571in}{3.035830in}}%
\pgfpathlineto{\pgfqpoint{5.221877in}{3.021146in}}%
\pgfpathlineto{\pgfqpoint{5.207204in}{3.006648in}}%
\pgfpathlineto{\pgfqpoint{5.192551in}{2.992335in}}%
\pgfpathlineto{\pgfqpoint{5.177919in}{2.978208in}}%
\pgfpathlineto{\pgfqpoint{5.170274in}{2.971475in}}%
\pgfpathlineto{\pgfqpoint{5.162619in}{2.964603in}}%
\pgfpathlineto{\pgfqpoint{5.154956in}{2.957590in}}%
\pgfpathlineto{\pgfqpoint{5.147284in}{2.950433in}}%
\pgfpathclose%
\pgfusepath{fill}%
\end{pgfscope}%
\begin{pgfscope}%
\pgfpathrectangle{\pgfqpoint{1.150000in}{0.150000in}}{\pgfqpoint{5.700000in}{5.700000in}}%
\pgfusepath{clip}%
\pgfsetbuttcap%
\pgfsetroundjoin%
\definecolor{currentfill}{rgb}{0.139147,0.533812,0.555298}%
\pgfsetfillcolor{currentfill}%
\pgfsetfillopacity{0.800000}%
\pgfsetlinewidth{0.000000pt}%
\definecolor{currentstroke}{rgb}{0.000000,0.000000,0.000000}%
\pgfsetstrokecolor{currentstroke}%
\pgfsetdash{}{0pt}%
\pgfpathmoveto{\pgfqpoint{4.937959in}{2.741878in}}%
\pgfpathlineto{\pgfqpoint{4.952464in}{2.755070in}}%
\pgfpathlineto{\pgfqpoint{4.966988in}{2.768447in}}%
\pgfpathlineto{\pgfqpoint{4.981531in}{2.782010in}}%
\pgfpathlineto{\pgfqpoint{4.996093in}{2.795759in}}%
\pgfpathlineto{\pgfqpoint{5.003874in}{2.804863in}}%
\pgfpathlineto{\pgfqpoint{5.011647in}{2.813811in}}%
\pgfpathlineto{\pgfqpoint{5.019411in}{2.822603in}}%
\pgfpathlineto{\pgfqpoint{5.027167in}{2.831242in}}%
\pgfpathlineto{\pgfqpoint{5.012610in}{2.817579in}}%
\pgfpathlineto{\pgfqpoint{4.998072in}{2.804101in}}%
\pgfpathlineto{\pgfqpoint{4.983554in}{2.790808in}}%
\pgfpathlineto{\pgfqpoint{4.969054in}{2.777700in}}%
\pgfpathlineto{\pgfqpoint{4.961292in}{2.768964in}}%
\pgfpathlineto{\pgfqpoint{4.953522in}{2.760082in}}%
\pgfpathlineto{\pgfqpoint{4.945745in}{2.751054in}}%
\pgfpathlineto{\pgfqpoint{4.937959in}{2.741878in}}%
\pgfpathclose%
\pgfusepath{fill}%
\end{pgfscope}%
\begin{pgfscope}%
\pgfpathrectangle{\pgfqpoint{1.150000in}{0.150000in}}{\pgfqpoint{5.700000in}{5.700000in}}%
\pgfusepath{clip}%
\pgfsetbuttcap%
\pgfsetroundjoin%
\definecolor{currentfill}{rgb}{0.283091,0.110553,0.431554}%
\pgfsetfillcolor{currentfill}%
\pgfsetfillopacity{0.800000}%
\pgfsetlinewidth{0.000000pt}%
\definecolor{currentstroke}{rgb}{0.000000,0.000000,0.000000}%
\pgfsetstrokecolor{currentstroke}%
\pgfsetdash{}{0pt}%
\pgfpathmoveto{\pgfqpoint{3.859053in}{1.609567in}}%
\pgfpathlineto{\pgfqpoint{3.873004in}{1.612169in}}%
\pgfpathlineto{\pgfqpoint{3.886963in}{1.614956in}}%
\pgfpathlineto{\pgfqpoint{3.900932in}{1.617928in}}%
\pgfpathlineto{\pgfqpoint{3.914911in}{1.621084in}}%
\pgfpathlineto{\pgfqpoint{3.923064in}{1.634219in}}%
\pgfpathlineto{\pgfqpoint{3.931212in}{1.647379in}}%
\pgfpathlineto{\pgfqpoint{3.939356in}{1.660557in}}%
\pgfpathlineto{\pgfqpoint{3.947494in}{1.673751in}}%
\pgfpathlineto{\pgfqpoint{3.933521in}{1.670154in}}%
\pgfpathlineto{\pgfqpoint{3.919557in}{1.666741in}}%
\pgfpathlineto{\pgfqpoint{3.905603in}{1.663513in}}%
\pgfpathlineto{\pgfqpoint{3.891659in}{1.660470in}}%
\pgfpathlineto{\pgfqpoint{3.883515in}{1.647705in}}%
\pgfpathlineto{\pgfqpoint{3.875366in}{1.634964in}}%
\pgfpathlineto{\pgfqpoint{3.867212in}{1.622249in}}%
\pgfpathlineto{\pgfqpoint{3.859053in}{1.609567in}}%
\pgfpathclose%
\pgfusepath{fill}%
\end{pgfscope}%
\begin{pgfscope}%
\pgfpathrectangle{\pgfqpoint{1.150000in}{0.150000in}}{\pgfqpoint{5.700000in}{5.700000in}}%
\pgfusepath{clip}%
\pgfsetbuttcap%
\pgfsetroundjoin%
\definecolor{currentfill}{rgb}{0.246811,0.283237,0.535941}%
\pgfsetfillcolor{currentfill}%
\pgfsetfillopacity{0.800000}%
\pgfsetlinewidth{0.000000pt}%
\definecolor{currentstroke}{rgb}{0.000000,0.000000,0.000000}%
\pgfsetstrokecolor{currentstroke}%
\pgfsetdash{}{0pt}%
\pgfpathmoveto{\pgfqpoint{4.277720in}{2.013706in}}%
\pgfpathlineto{\pgfqpoint{4.291840in}{2.021433in}}%
\pgfpathlineto{\pgfqpoint{4.305973in}{2.029344in}}%
\pgfpathlineto{\pgfqpoint{4.320120in}{2.037439in}}%
\pgfpathlineto{\pgfqpoint{4.334281in}{2.045718in}}%
\pgfpathlineto{\pgfqpoint{4.342319in}{2.059500in}}%
\pgfpathlineto{\pgfqpoint{4.350352in}{2.073197in}}%
\pgfpathlineto{\pgfqpoint{4.358380in}{2.086807in}}%
\pgfpathlineto{\pgfqpoint{4.366403in}{2.100327in}}%
\pgfpathlineto{\pgfqpoint{4.352241in}{2.091795in}}%
\pgfpathlineto{\pgfqpoint{4.338094in}{2.083448in}}%
\pgfpathlineto{\pgfqpoint{4.323960in}{2.075284in}}%
\pgfpathlineto{\pgfqpoint{4.309841in}{2.067305in}}%
\pgfpathlineto{\pgfqpoint{4.301818in}{2.054025in}}%
\pgfpathlineto{\pgfqpoint{4.293790in}{2.040664in}}%
\pgfpathlineto{\pgfqpoint{4.285758in}{2.027223in}}%
\pgfpathlineto{\pgfqpoint{4.277720in}{2.013706in}}%
\pgfpathclose%
\pgfusepath{fill}%
\end{pgfscope}%
\begin{pgfscope}%
\pgfpathrectangle{\pgfqpoint{1.150000in}{0.150000in}}{\pgfqpoint{5.700000in}{5.700000in}}%
\pgfusepath{clip}%
\pgfsetbuttcap%
\pgfsetroundjoin%
\definecolor{currentfill}{rgb}{0.267004,0.004874,0.329415}%
\pgfsetfillcolor{currentfill}%
\pgfsetfillopacity{0.800000}%
\pgfsetlinewidth{0.000000pt}%
\definecolor{currentstroke}{rgb}{0.000000,0.000000,0.000000}%
\pgfsetstrokecolor{currentstroke}%
\pgfsetdash{}{0pt}%
\pgfpathmoveto{\pgfqpoint{3.271551in}{1.422592in}}%
\pgfpathlineto{\pgfqpoint{3.285428in}{1.416573in}}%
\pgfpathlineto{\pgfqpoint{3.299307in}{1.410753in}}%
\pgfpathlineto{\pgfqpoint{3.313189in}{1.405129in}}%
\pgfpathlineto{\pgfqpoint{3.327074in}{1.399702in}}%
\pgfpathlineto{\pgfqpoint{3.335490in}{1.406334in}}%
\pgfpathlineto{\pgfqpoint{3.343896in}{1.413169in}}%
\pgfpathlineto{\pgfqpoint{3.352293in}{1.420203in}}%
\pgfpathlineto{\pgfqpoint{3.360680in}{1.427427in}}%
\pgfpathlineto{\pgfqpoint{3.346819in}{1.432226in}}%
\pgfpathlineto{\pgfqpoint{3.332962in}{1.437222in}}%
\pgfpathlineto{\pgfqpoint{3.319108in}{1.442414in}}%
\pgfpathlineto{\pgfqpoint{3.305257in}{1.447805in}}%
\pgfpathlineto{\pgfqpoint{3.296846in}{1.441196in}}%
\pgfpathlineto{\pgfqpoint{3.288424in}{1.434787in}}%
\pgfpathlineto{\pgfqpoint{3.279993in}{1.428583in}}%
\pgfpathlineto{\pgfqpoint{3.271551in}{1.422592in}}%
\pgfpathclose%
\pgfusepath{fill}%
\end{pgfscope}%
\begin{pgfscope}%
\pgfpathrectangle{\pgfqpoint{1.150000in}{0.150000in}}{\pgfqpoint{5.700000in}{5.700000in}}%
\pgfusepath{clip}%
\pgfsetbuttcap%
\pgfsetroundjoin%
\definecolor{currentfill}{rgb}{0.225863,0.330805,0.547314}%
\pgfsetfillcolor{currentfill}%
\pgfsetfillopacity{0.800000}%
\pgfsetlinewidth{0.000000pt}%
\definecolor{currentstroke}{rgb}{0.000000,0.000000,0.000000}%
\pgfsetstrokecolor{currentstroke}%
\pgfsetdash{}{0pt}%
\pgfpathmoveto{\pgfqpoint{2.362840in}{2.229273in}}%
\pgfpathlineto{\pgfqpoint{2.377058in}{2.207805in}}%
\pgfpathlineto{\pgfqpoint{2.391265in}{2.186623in}}%
\pgfpathlineto{\pgfqpoint{2.405460in}{2.165725in}}%
\pgfpathlineto{\pgfqpoint{2.419643in}{2.145109in}}%
\pgfpathlineto{\pgfqpoint{2.428826in}{2.139019in}}%
\pgfpathlineto{\pgfqpoint{2.437984in}{2.133322in}}%
\pgfpathlineto{\pgfqpoint{2.447119in}{2.128009in}}%
\pgfpathlineto{\pgfqpoint{2.456231in}{2.123074in}}%
\pgfpathlineto{\pgfqpoint{2.442108in}{2.142951in}}%
\pgfpathlineto{\pgfqpoint{2.427974in}{2.163107in}}%
\pgfpathlineto{\pgfqpoint{2.413830in}{2.183546in}}%
\pgfpathlineto{\pgfqpoint{2.399674in}{2.204269in}}%
\pgfpathlineto{\pgfqpoint{2.390502in}{2.209932in}}%
\pgfpathlineto{\pgfqpoint{2.381306in}{2.215981in}}%
\pgfpathlineto{\pgfqpoint{2.372085in}{2.222426in}}%
\pgfpathlineto{\pgfqpoint{2.362840in}{2.229273in}}%
\pgfpathclose%
\pgfusepath{fill}%
\end{pgfscope}%
\begin{pgfscope}%
\pgfpathrectangle{\pgfqpoint{1.150000in}{0.150000in}}{\pgfqpoint{5.700000in}{5.700000in}}%
\pgfusepath{clip}%
\pgfsetbuttcap%
\pgfsetroundjoin%
\definecolor{currentfill}{rgb}{0.185556,0.418570,0.556753}%
\pgfsetfillcolor{currentfill}%
\pgfsetfillopacity{0.800000}%
\pgfsetlinewidth{0.000000pt}%
\definecolor{currentstroke}{rgb}{0.000000,0.000000,0.000000}%
\pgfsetstrokecolor{currentstroke}%
\pgfsetdash{}{0pt}%
\pgfpathmoveto{\pgfqpoint{4.607912in}{2.383452in}}%
\pgfpathlineto{\pgfqpoint{4.622214in}{2.394327in}}%
\pgfpathlineto{\pgfqpoint{4.636533in}{2.405386in}}%
\pgfpathlineto{\pgfqpoint{4.650869in}{2.416630in}}%
\pgfpathlineto{\pgfqpoint{4.665221in}{2.428059in}}%
\pgfpathlineto{\pgfqpoint{4.673152in}{2.440120in}}%
\pgfpathlineto{\pgfqpoint{4.681077in}{2.452044in}}%
\pgfpathlineto{\pgfqpoint{4.688996in}{2.463830in}}%
\pgfpathlineto{\pgfqpoint{4.696908in}{2.475478in}}%
\pgfpathlineto{\pgfqpoint{4.682555in}{2.463961in}}%
\pgfpathlineto{\pgfqpoint{4.668220in}{2.452630in}}%
\pgfpathlineto{\pgfqpoint{4.653902in}{2.441483in}}%
\pgfpathlineto{\pgfqpoint{4.639600in}{2.430521in}}%
\pgfpathlineto{\pgfqpoint{4.631687in}{2.418949in}}%
\pgfpathlineto{\pgfqpoint{4.623768in}{2.407246in}}%
\pgfpathlineto{\pgfqpoint{4.615843in}{2.395414in}}%
\pgfpathlineto{\pgfqpoint{4.607912in}{2.383452in}}%
\pgfpathclose%
\pgfusepath{fill}%
\end{pgfscope}%
\begin{pgfscope}%
\pgfpathrectangle{\pgfqpoint{1.150000in}{0.150000in}}{\pgfqpoint{5.700000in}{5.700000in}}%
\pgfusepath{clip}%
\pgfsetbuttcap%
\pgfsetroundjoin%
\definecolor{currentfill}{rgb}{0.160665,0.478540,0.558115}%
\pgfsetfillcolor{currentfill}%
\pgfsetfillopacity{0.800000}%
\pgfsetlinewidth{0.000000pt}%
\definecolor{currentstroke}{rgb}{0.000000,0.000000,0.000000}%
\pgfsetstrokecolor{currentstroke}%
\pgfsetdash{}{0pt}%
\pgfpathmoveto{\pgfqpoint{2.113702in}{2.682534in}}%
\pgfpathlineto{\pgfqpoint{2.128151in}{2.655552in}}%
\pgfpathlineto{\pgfqpoint{2.142582in}{2.628911in}}%
\pgfpathlineto{\pgfqpoint{2.156995in}{2.602609in}}%
\pgfpathlineto{\pgfqpoint{2.171391in}{2.576643in}}%
\pgfpathlineto{\pgfqpoint{2.180801in}{2.568424in}}%
\pgfpathlineto{\pgfqpoint{2.190183in}{2.560617in}}%
\pgfpathlineto{\pgfqpoint{2.199539in}{2.553216in}}%
\pgfpathlineto{\pgfqpoint{2.208868in}{2.546212in}}%
\pgfpathlineto{\pgfqpoint{2.194541in}{2.571444in}}%
\pgfpathlineto{\pgfqpoint{2.180198in}{2.597010in}}%
\pgfpathlineto{\pgfqpoint{2.165837in}{2.622913in}}%
\pgfpathlineto{\pgfqpoint{2.151459in}{2.649155in}}%
\pgfpathlineto{\pgfqpoint{2.142061in}{2.656880in}}%
\pgfpathlineto{\pgfqpoint{2.132636in}{2.665014in}}%
\pgfpathlineto{\pgfqpoint{2.123183in}{2.673562in}}%
\pgfpathlineto{\pgfqpoint{2.113702in}{2.682534in}}%
\pgfpathclose%
\pgfusepath{fill}%
\end{pgfscope}%
\begin{pgfscope}%
\pgfpathrectangle{\pgfqpoint{1.150000in}{0.150000in}}{\pgfqpoint{5.700000in}{5.700000in}}%
\pgfusepath{clip}%
\pgfsetbuttcap%
\pgfsetroundjoin%
\definecolor{currentfill}{rgb}{0.327796,0.773980,0.406640}%
\pgfsetfillcolor{currentfill}%
\pgfsetfillopacity{0.800000}%
\pgfsetlinewidth{0.000000pt}%
\definecolor{currentstroke}{rgb}{0.000000,0.000000,0.000000}%
\pgfsetstrokecolor{currentstroke}%
\pgfsetdash{}{0pt}%
\pgfpathmoveto{\pgfqpoint{5.831936in}{3.527183in}}%
\pgfpathlineto{\pgfqpoint{5.847037in}{3.543686in}}%
\pgfpathlineto{\pgfqpoint{5.862161in}{3.560375in}}%
\pgfpathlineto{\pgfqpoint{5.877310in}{3.577250in}}%
\pgfpathlineto{\pgfqpoint{5.892483in}{3.594311in}}%
\pgfpathlineto{\pgfqpoint{5.899655in}{3.594472in}}%
\pgfpathlineto{\pgfqpoint{5.906817in}{3.594563in}}%
\pgfpathlineto{\pgfqpoint{5.913970in}{3.594591in}}%
\pgfpathlineto{\pgfqpoint{5.921113in}{3.594560in}}%
\pgfpathlineto{\pgfqpoint{5.905970in}{3.578016in}}%
\pgfpathlineto{\pgfqpoint{5.890851in}{3.561658in}}%
\pgfpathlineto{\pgfqpoint{5.875756in}{3.545484in}}%
\pgfpathlineto{\pgfqpoint{5.860685in}{3.529494in}}%
\pgfpathlineto{\pgfqpoint{5.853511in}{3.528998in}}%
\pgfpathlineto{\pgfqpoint{5.846328in}{3.528451in}}%
\pgfpathlineto{\pgfqpoint{5.839137in}{3.527848in}}%
\pgfpathlineto{\pgfqpoint{5.831936in}{3.527183in}}%
\pgfpathclose%
\pgfusepath{fill}%
\end{pgfscope}%
\begin{pgfscope}%
\pgfpathrectangle{\pgfqpoint{1.150000in}{0.150000in}}{\pgfqpoint{5.700000in}{5.700000in}}%
\pgfusepath{clip}%
\pgfsetbuttcap%
\pgfsetroundjoin%
\definecolor{currentfill}{rgb}{0.162016,0.687316,0.499129}%
\pgfsetfillcolor{currentfill}%
\pgfsetfillopacity{0.800000}%
\pgfsetlinewidth{0.000000pt}%
\definecolor{currentstroke}{rgb}{0.000000,0.000000,0.000000}%
\pgfsetstrokecolor{currentstroke}%
\pgfsetdash{}{0pt}%
\pgfpathmoveto{\pgfqpoint{5.445473in}{3.221580in}}%
\pgfpathlineto{\pgfqpoint{5.460319in}{3.237149in}}%
\pgfpathlineto{\pgfqpoint{5.475186in}{3.252904in}}%
\pgfpathlineto{\pgfqpoint{5.490076in}{3.268845in}}%
\pgfpathlineto{\pgfqpoint{5.504988in}{3.284973in}}%
\pgfpathlineto{\pgfqpoint{5.512457in}{3.288891in}}%
\pgfpathlineto{\pgfqpoint{5.519915in}{3.292680in}}%
\pgfpathlineto{\pgfqpoint{5.527363in}{3.296344in}}%
\pgfpathlineto{\pgfqpoint{5.534801in}{3.299886in}}%
\pgfpathlineto{\pgfqpoint{5.519907in}{3.284094in}}%
\pgfpathlineto{\pgfqpoint{5.505034in}{3.268488in}}%
\pgfpathlineto{\pgfqpoint{5.490184in}{3.253067in}}%
\pgfpathlineto{\pgfqpoint{5.475355in}{3.237832in}}%
\pgfpathlineto{\pgfqpoint{5.467899in}{3.233943in}}%
\pgfpathlineto{\pgfqpoint{5.460433in}{3.229940in}}%
\pgfpathlineto{\pgfqpoint{5.452958in}{3.225820in}}%
\pgfpathlineto{\pgfqpoint{5.445473in}{3.221580in}}%
\pgfpathclose%
\pgfusepath{fill}%
\end{pgfscope}%
\begin{pgfscope}%
\pgfpathrectangle{\pgfqpoint{1.150000in}{0.150000in}}{\pgfqpoint{5.700000in}{5.700000in}}%
\pgfusepath{clip}%
\pgfsetbuttcap%
\pgfsetroundjoin%
\definecolor{currentfill}{rgb}{0.369214,0.788888,0.382914}%
\pgfsetfillcolor{currentfill}%
\pgfsetfillopacity{0.800000}%
\pgfsetlinewidth{0.000000pt}%
\definecolor{currentstroke}{rgb}{0.000000,0.000000,0.000000}%
\pgfsetstrokecolor{currentstroke}%
\pgfsetdash{}{0pt}%
\pgfpathmoveto{\pgfqpoint{5.921113in}{3.594560in}}%
\pgfpathlineto{\pgfqpoint{5.936280in}{3.611288in}}%
\pgfpathlineto{\pgfqpoint{5.951471in}{3.628202in}}%
\pgfpathlineto{\pgfqpoint{5.966686in}{3.645301in}}%
\pgfpathlineto{\pgfqpoint{5.973796in}{3.644812in}}%
\pgfpathlineto{\pgfqpoint{5.980897in}{3.644270in}}%
\pgfpathlineto{\pgfqpoint{5.987988in}{3.643682in}}%
\pgfpathlineto{\pgfqpoint{5.995070in}{3.643052in}}%
\pgfpathlineto{\pgfqpoint{5.979887in}{3.626505in}}%
\pgfpathlineto{\pgfqpoint{5.964729in}{3.610143in}}%
\pgfpathlineto{\pgfqpoint{5.949594in}{3.593965in}}%
\pgfpathlineto{\pgfqpoint{5.942487in}{3.594173in}}%
\pgfpathlineto{\pgfqpoint{5.935371in}{3.594345in}}%
\pgfpathlineto{\pgfqpoint{5.928246in}{3.594476in}}%
\pgfpathlineto{\pgfqpoint{5.921113in}{3.594560in}}%
\pgfpathclose%
\pgfusepath{fill}%
\end{pgfscope}%
\begin{pgfscope}%
\pgfpathrectangle{\pgfqpoint{1.150000in}{0.150000in}}{\pgfqpoint{5.700000in}{5.700000in}}%
\pgfusepath{clip}%
\pgfsetbuttcap%
\pgfsetroundjoin%
\definecolor{currentfill}{rgb}{0.281924,0.089666,0.412415}%
\pgfsetfillcolor{currentfill}%
\pgfsetfillopacity{0.800000}%
\pgfsetlinewidth{0.000000pt}%
\definecolor{currentstroke}{rgb}{0.000000,0.000000,0.000000}%
\pgfsetstrokecolor{currentstroke}%
\pgfsetdash{}{0pt}%
\pgfpathmoveto{\pgfqpoint{2.869135in}{1.621850in}}%
\pgfpathlineto{\pgfqpoint{2.883090in}{1.609440in}}%
\pgfpathlineto{\pgfqpoint{2.897042in}{1.597252in}}%
\pgfpathlineto{\pgfqpoint{2.910991in}{1.585285in}}%
\pgfpathlineto{\pgfqpoint{2.924937in}{1.573538in}}%
\pgfpathlineto{\pgfqpoint{2.933655in}{1.574041in}}%
\pgfpathlineto{\pgfqpoint{2.942357in}{1.574852in}}%
\pgfpathlineto{\pgfqpoint{2.951044in}{1.575965in}}%
\pgfpathlineto{\pgfqpoint{2.959715in}{1.577372in}}%
\pgfpathlineto{\pgfqpoint{2.945810in}{1.588416in}}%
\pgfpathlineto{\pgfqpoint{2.931902in}{1.599679in}}%
\pgfpathlineto{\pgfqpoint{2.917992in}{1.611162in}}%
\pgfpathlineto{\pgfqpoint{2.904080in}{1.622867in}}%
\pgfpathlineto{\pgfqpoint{2.895368in}{1.622151in}}%
\pgfpathlineto{\pgfqpoint{2.886641in}{1.621738in}}%
\pgfpathlineto{\pgfqpoint{2.877896in}{1.621635in}}%
\pgfpathlineto{\pgfqpoint{2.869135in}{1.621850in}}%
\pgfpathclose%
\pgfusepath{fill}%
\end{pgfscope}%
\begin{pgfscope}%
\pgfpathrectangle{\pgfqpoint{1.150000in}{0.150000in}}{\pgfqpoint{5.700000in}{5.700000in}}%
\pgfusepath{clip}%
\pgfsetbuttcap%
\pgfsetroundjoin%
\definecolor{currentfill}{rgb}{0.271305,0.019942,0.347269}%
\pgfsetfillcolor{currentfill}%
\pgfsetfillopacity{0.800000}%
\pgfsetlinewidth{0.000000pt}%
\definecolor{currentstroke}{rgb}{0.000000,0.000000,0.000000}%
\pgfsetstrokecolor{currentstroke}%
\pgfsetdash{}{0pt}%
\pgfpathmoveto{\pgfqpoint{3.126484in}{1.461539in}}%
\pgfpathlineto{\pgfqpoint{3.140379in}{1.453245in}}%
\pgfpathlineto{\pgfqpoint{3.154275in}{1.445157in}}%
\pgfpathlineto{\pgfqpoint{3.168172in}{1.437272in}}%
\pgfpathlineto{\pgfqpoint{3.182070in}{1.429590in}}%
\pgfpathlineto{\pgfqpoint{3.190586in}{1.433995in}}%
\pgfpathlineto{\pgfqpoint{3.199090in}{1.438646in}}%
\pgfpathlineto{\pgfqpoint{3.207583in}{1.443536in}}%
\pgfpathlineto{\pgfqpoint{3.216064in}{1.448660in}}%
\pgfpathlineto{\pgfqpoint{3.202197in}{1.455680in}}%
\pgfpathlineto{\pgfqpoint{3.188331in}{1.462902in}}%
\pgfpathlineto{\pgfqpoint{3.174466in}{1.470328in}}%
\pgfpathlineto{\pgfqpoint{3.160603in}{1.477958in}}%
\pgfpathlineto{\pgfqpoint{3.152091in}{1.473484in}}%
\pgfpathlineto{\pgfqpoint{3.143568in}{1.469252in}}%
\pgfpathlineto{\pgfqpoint{3.135032in}{1.465268in}}%
\pgfpathlineto{\pgfqpoint{3.126484in}{1.461539in}}%
\pgfpathclose%
\pgfusepath{fill}%
\end{pgfscope}%
\begin{pgfscope}%
\pgfpathrectangle{\pgfqpoint{1.150000in}{0.150000in}}{\pgfqpoint{5.700000in}{5.700000in}}%
\pgfusepath{clip}%
\pgfsetbuttcap%
\pgfsetroundjoin%
\definecolor{currentfill}{rgb}{0.267004,0.004874,0.329415}%
\pgfsetfillcolor{currentfill}%
\pgfsetfillopacity{0.800000}%
\pgfsetlinewidth{0.000000pt}%
\definecolor{currentstroke}{rgb}{0.000000,0.000000,0.000000}%
\pgfsetstrokecolor{currentstroke}%
\pgfsetdash{}{0pt}%
\pgfpathmoveto{\pgfqpoint{3.416158in}{1.410181in}}%
\pgfpathlineto{\pgfqpoint{3.430037in}{1.406354in}}%
\pgfpathlineto{\pgfqpoint{3.443920in}{1.402719in}}%
\pgfpathlineto{\pgfqpoint{3.457808in}{1.399277in}}%
\pgfpathlineto{\pgfqpoint{3.471701in}{1.396025in}}%
\pgfpathlineto{\pgfqpoint{3.480036in}{1.404650in}}%
\pgfpathlineto{\pgfqpoint{3.488364in}{1.413438in}}%
\pgfpathlineto{\pgfqpoint{3.496683in}{1.422384in}}%
\pgfpathlineto{\pgfqpoint{3.504994in}{1.431482in}}%
\pgfpathlineto{\pgfqpoint{3.491120in}{1.434137in}}%
\pgfpathlineto{\pgfqpoint{3.477252in}{1.436985in}}%
\pgfpathlineto{\pgfqpoint{3.463388in}{1.440024in}}%
\pgfpathlineto{\pgfqpoint{3.449529in}{1.443256in}}%
\pgfpathlineto{\pgfqpoint{3.441199in}{1.434742in}}%
\pgfpathlineto{\pgfqpoint{3.432860in}{1.426387in}}%
\pgfpathlineto{\pgfqpoint{3.424513in}{1.418199in}}%
\pgfpathlineto{\pgfqpoint{3.416158in}{1.410181in}}%
\pgfpathclose%
\pgfusepath{fill}%
\end{pgfscope}%
\begin{pgfscope}%
\pgfpathrectangle{\pgfqpoint{1.150000in}{0.150000in}}{\pgfqpoint{5.700000in}{5.700000in}}%
\pgfusepath{clip}%
\pgfsetbuttcap%
\pgfsetroundjoin%
\definecolor{currentfill}{rgb}{0.282623,0.140926,0.457517}%
\pgfsetfillcolor{currentfill}%
\pgfsetfillopacity{0.800000}%
\pgfsetlinewidth{0.000000pt}%
\definecolor{currentstroke}{rgb}{0.000000,0.000000,0.000000}%
\pgfsetstrokecolor{currentstroke}%
\pgfsetdash{}{0pt}%
\pgfpathmoveto{\pgfqpoint{3.947494in}{1.673751in}}%
\pgfpathlineto{\pgfqpoint{3.961479in}{1.677533in}}%
\pgfpathlineto{\pgfqpoint{3.975473in}{1.681498in}}%
\pgfpathlineto{\pgfqpoint{3.989478in}{1.685648in}}%
\pgfpathlineto{\pgfqpoint{4.003494in}{1.689981in}}%
\pgfpathlineto{\pgfqpoint{4.011624in}{1.703609in}}%
\pgfpathlineto{\pgfqpoint{4.019750in}{1.717236in}}%
\pgfpathlineto{\pgfqpoint{4.027871in}{1.730859in}}%
\pgfpathlineto{\pgfqpoint{4.035987in}{1.744474in}}%
\pgfpathlineto{\pgfqpoint{4.021975in}{1.739730in}}%
\pgfpathlineto{\pgfqpoint{4.007973in}{1.735170in}}%
\pgfpathlineto{\pgfqpoint{3.993983in}{1.730794in}}%
\pgfpathlineto{\pgfqpoint{3.980003in}{1.726603in}}%
\pgfpathlineto{\pgfqpoint{3.971883in}{1.713386in}}%
\pgfpathlineto{\pgfqpoint{3.963758in}{1.700169in}}%
\pgfpathlineto{\pgfqpoint{3.955628in}{1.686956in}}%
\pgfpathlineto{\pgfqpoint{3.947494in}{1.673751in}}%
\pgfpathclose%
\pgfusepath{fill}%
\end{pgfscope}%
\begin{pgfscope}%
\pgfpathrectangle{\pgfqpoint{1.150000in}{0.150000in}}{\pgfqpoint{5.700000in}{5.700000in}}%
\pgfusepath{clip}%
\pgfsetbuttcap%
\pgfsetroundjoin%
\definecolor{currentfill}{rgb}{0.265145,0.232956,0.516599}%
\pgfsetfillcolor{currentfill}%
\pgfsetfillopacity{0.800000}%
\pgfsetlinewidth{0.000000pt}%
\definecolor{currentstroke}{rgb}{0.000000,0.000000,0.000000}%
\pgfsetstrokecolor{currentstroke}%
\pgfsetdash{}{0pt}%
\pgfpathmoveto{\pgfqpoint{4.156908in}{1.876413in}}%
\pgfpathlineto{\pgfqpoint{4.170976in}{1.882804in}}%
\pgfpathlineto{\pgfqpoint{4.185056in}{1.889378in}}%
\pgfpathlineto{\pgfqpoint{4.199150in}{1.896136in}}%
\pgfpathlineto{\pgfqpoint{4.213256in}{1.903078in}}%
\pgfpathlineto{\pgfqpoint{4.221330in}{1.917124in}}%
\pgfpathlineto{\pgfqpoint{4.229399in}{1.931114in}}%
\pgfpathlineto{\pgfqpoint{4.237464in}{1.945044in}}%
\pgfpathlineto{\pgfqpoint{4.245525in}{1.958911in}}%
\pgfpathlineto{\pgfqpoint{4.231419in}{1.951652in}}%
\pgfpathlineto{\pgfqpoint{4.217326in}{1.944578in}}%
\pgfpathlineto{\pgfqpoint{4.203246in}{1.937687in}}%
\pgfpathlineto{\pgfqpoint{4.189179in}{1.930980in}}%
\pgfpathlineto{\pgfqpoint{4.181118in}{1.917417in}}%
\pgfpathlineto{\pgfqpoint{4.173052in}{1.903800in}}%
\pgfpathlineto{\pgfqpoint{4.164982in}{1.890131in}}%
\pgfpathlineto{\pgfqpoint{4.156908in}{1.876413in}}%
\pgfpathclose%
\pgfusepath{fill}%
\end{pgfscope}%
\begin{pgfscope}%
\pgfpathrectangle{\pgfqpoint{1.150000in}{0.150000in}}{\pgfqpoint{5.700000in}{5.700000in}}%
\pgfusepath{clip}%
\pgfsetbuttcap%
\pgfsetroundjoin%
\definecolor{currentfill}{rgb}{0.206756,0.371758,0.553117}%
\pgfsetfillcolor{currentfill}%
\pgfsetfillopacity{0.800000}%
\pgfsetlinewidth{0.000000pt}%
\definecolor{currentstroke}{rgb}{0.000000,0.000000,0.000000}%
\pgfsetstrokecolor{currentstroke}%
\pgfsetdash{}{0pt}%
\pgfpathmoveto{\pgfqpoint{4.487205in}{2.242642in}}%
\pgfpathlineto{\pgfqpoint{4.501442in}{2.252503in}}%
\pgfpathlineto{\pgfqpoint{4.515696in}{2.262548in}}%
\pgfpathlineto{\pgfqpoint{4.529965in}{2.272777in}}%
\pgfpathlineto{\pgfqpoint{4.544249in}{2.283191in}}%
\pgfpathlineto{\pgfqpoint{4.552227in}{2.296159in}}%
\pgfpathlineto{\pgfqpoint{4.560200in}{2.309005in}}%
\pgfpathlineto{\pgfqpoint{4.568166in}{2.321728in}}%
\pgfpathlineto{\pgfqpoint{4.576127in}{2.334326in}}%
\pgfpathlineto{\pgfqpoint{4.561842in}{2.323757in}}%
\pgfpathlineto{\pgfqpoint{4.547572in}{2.313373in}}%
\pgfpathlineto{\pgfqpoint{4.533318in}{2.303174in}}%
\pgfpathlineto{\pgfqpoint{4.519080in}{2.293159in}}%
\pgfpathlineto{\pgfqpoint{4.511120in}{2.280703in}}%
\pgfpathlineto{\pgfqpoint{4.503154in}{2.268131in}}%
\pgfpathlineto{\pgfqpoint{4.495182in}{2.255444in}}%
\pgfpathlineto{\pgfqpoint{4.487205in}{2.242642in}}%
\pgfpathclose%
\pgfusepath{fill}%
\end{pgfscope}%
\begin{pgfscope}%
\pgfpathrectangle{\pgfqpoint{1.150000in}{0.150000in}}{\pgfqpoint{5.700000in}{5.700000in}}%
\pgfusepath{clip}%
\pgfsetbuttcap%
\pgfsetroundjoin%
\definecolor{currentfill}{rgb}{0.210503,0.363727,0.552206}%
\pgfsetfillcolor{currentfill}%
\pgfsetfillopacity{0.800000}%
\pgfsetlinewidth{0.000000pt}%
\definecolor{currentstroke}{rgb}{0.000000,0.000000,0.000000}%
\pgfsetstrokecolor{currentstroke}%
\pgfsetdash{}{0pt}%
\pgfpathmoveto{\pgfqpoint{2.305841in}{2.318054in}}%
\pgfpathlineto{\pgfqpoint{2.320110in}{2.295417in}}%
\pgfpathlineto{\pgfqpoint{2.334366in}{2.273076in}}%
\pgfpathlineto{\pgfqpoint{2.348609in}{2.251029in}}%
\pgfpathlineto{\pgfqpoint{2.362840in}{2.229273in}}%
\pgfpathlineto{\pgfqpoint{2.372085in}{2.222426in}}%
\pgfpathlineto{\pgfqpoint{2.381306in}{2.215981in}}%
\pgfpathlineto{\pgfqpoint{2.390502in}{2.209932in}}%
\pgfpathlineto{\pgfqpoint{2.399674in}{2.204269in}}%
\pgfpathlineto{\pgfqpoint{2.385506in}{2.225279in}}%
\pgfpathlineto{\pgfqpoint{2.371327in}{2.246579in}}%
\pgfpathlineto{\pgfqpoint{2.357135in}{2.268170in}}%
\pgfpathlineto{\pgfqpoint{2.342931in}{2.290056in}}%
\pgfpathlineto{\pgfqpoint{2.333697in}{2.296453in}}%
\pgfpathlineto{\pgfqpoint{2.324437in}{2.303246in}}%
\pgfpathlineto{\pgfqpoint{2.315152in}{2.310444in}}%
\pgfpathlineto{\pgfqpoint{2.305841in}{2.318054in}}%
\pgfpathclose%
\pgfusepath{fill}%
\end{pgfscope}%
\begin{pgfscope}%
\pgfpathrectangle{\pgfqpoint{1.150000in}{0.150000in}}{\pgfqpoint{5.700000in}{5.700000in}}%
\pgfusepath{clip}%
\pgfsetbuttcap%
\pgfsetroundjoin%
\definecolor{currentfill}{rgb}{0.153364,0.497000,0.557724}%
\pgfsetfillcolor{currentfill}%
\pgfsetfillopacity{0.800000}%
\pgfsetlinewidth{0.000000pt}%
\definecolor{currentstroke}{rgb}{0.000000,0.000000,0.000000}%
\pgfsetstrokecolor{currentstroke}%
\pgfsetdash{}{0pt}%
\pgfpathmoveto{\pgfqpoint{4.817537in}{2.611857in}}%
\pgfpathlineto{\pgfqpoint{4.831975in}{2.624372in}}%
\pgfpathlineto{\pgfqpoint{4.846430in}{2.637073in}}%
\pgfpathlineto{\pgfqpoint{4.860904in}{2.649959in}}%
\pgfpathlineto{\pgfqpoint{4.875397in}{2.663031in}}%
\pgfpathlineto{\pgfqpoint{4.883244in}{2.673423in}}%
\pgfpathlineto{\pgfqpoint{4.891083in}{2.683661in}}%
\pgfpathlineto{\pgfqpoint{4.898915in}{2.693745in}}%
\pgfpathlineto{\pgfqpoint{4.906739in}{2.703675in}}%
\pgfpathlineto{\pgfqpoint{4.892249in}{2.690619in}}%
\pgfpathlineto{\pgfqpoint{4.877778in}{2.677749in}}%
\pgfpathlineto{\pgfqpoint{4.863325in}{2.665063in}}%
\pgfpathlineto{\pgfqpoint{4.848890in}{2.652563in}}%
\pgfpathlineto{\pgfqpoint{4.841063in}{2.642605in}}%
\pgfpathlineto{\pgfqpoint{4.833228in}{2.632501in}}%
\pgfpathlineto{\pgfqpoint{4.825386in}{2.622253in}}%
\pgfpathlineto{\pgfqpoint{4.817537in}{2.611857in}}%
\pgfpathclose%
\pgfusepath{fill}%
\end{pgfscope}%
\begin{pgfscope}%
\pgfpathrectangle{\pgfqpoint{1.150000in}{0.150000in}}{\pgfqpoint{5.700000in}{5.700000in}}%
\pgfusepath{clip}%
\pgfsetbuttcap%
\pgfsetroundjoin%
\definecolor{currentfill}{rgb}{0.280267,0.073417,0.397163}%
\pgfsetfillcolor{currentfill}%
\pgfsetfillopacity{0.800000}%
\pgfsetlinewidth{0.000000pt}%
\definecolor{currentstroke}{rgb}{0.000000,0.000000,0.000000}%
\pgfsetstrokecolor{currentstroke}%
\pgfsetdash{}{0pt}%
\pgfpathmoveto{\pgfqpoint{2.924937in}{1.573538in}}%
\pgfpathlineto{\pgfqpoint{2.938881in}{1.562010in}}%
\pgfpathlineto{\pgfqpoint{2.952822in}{1.550699in}}%
\pgfpathlineto{\pgfqpoint{2.966762in}{1.539604in}}%
\pgfpathlineto{\pgfqpoint{2.980700in}{1.528724in}}%
\pgfpathlineto{\pgfqpoint{2.989377in}{1.529942in}}%
\pgfpathlineto{\pgfqpoint{2.998039in}{1.531460in}}%
\pgfpathlineto{\pgfqpoint{3.006686in}{1.533270in}}%
\pgfpathlineto{\pgfqpoint{3.015319in}{1.535366in}}%
\pgfpathlineto{\pgfqpoint{3.001420in}{1.545545in}}%
\pgfpathlineto{\pgfqpoint{2.987520in}{1.555938in}}%
\pgfpathlineto{\pgfqpoint{2.973618in}{1.566547in}}%
\pgfpathlineto{\pgfqpoint{2.959715in}{1.577372in}}%
\pgfpathlineto{\pgfqpoint{2.951044in}{1.575965in}}%
\pgfpathlineto{\pgfqpoint{2.942357in}{1.574852in}}%
\pgfpathlineto{\pgfqpoint{2.933655in}{1.574041in}}%
\pgfpathlineto{\pgfqpoint{2.924937in}{1.573538in}}%
\pgfpathclose%
\pgfusepath{fill}%
\end{pgfscope}%
\begin{pgfscope}%
\pgfpathrectangle{\pgfqpoint{1.150000in}{0.150000in}}{\pgfqpoint{5.700000in}{5.700000in}}%
\pgfusepath{clip}%
\pgfsetbuttcap%
\pgfsetroundjoin%
\definecolor{currentfill}{rgb}{0.122312,0.633153,0.530398}%
\pgfsetfillcolor{currentfill}%
\pgfsetfillopacity{0.800000}%
\pgfsetlinewidth{0.000000pt}%
\definecolor{currentstroke}{rgb}{0.000000,0.000000,0.000000}%
\pgfsetstrokecolor{currentstroke}%
\pgfsetdash{}{0pt}%
\pgfpathmoveto{\pgfqpoint{5.236571in}{3.035830in}}%
\pgfpathlineto{\pgfqpoint{5.251285in}{3.050699in}}%
\pgfpathlineto{\pgfqpoint{5.266021in}{3.065755in}}%
\pgfpathlineto{\pgfqpoint{5.280777in}{3.080997in}}%
\pgfpathlineto{\pgfqpoint{5.295555in}{3.096426in}}%
\pgfpathlineto{\pgfqpoint{5.303170in}{3.102597in}}%
\pgfpathlineto{\pgfqpoint{5.310776in}{3.108616in}}%
\pgfpathlineto{\pgfqpoint{5.318372in}{3.114487in}}%
\pgfpathlineto{\pgfqpoint{5.325958in}{3.120214in}}%
\pgfpathlineto{\pgfqpoint{5.311192in}{3.105013in}}%
\pgfpathlineto{\pgfqpoint{5.296447in}{3.089998in}}%
\pgfpathlineto{\pgfqpoint{5.281723in}{3.075169in}}%
\pgfpathlineto{\pgfqpoint{5.267020in}{3.060526in}}%
\pgfpathlineto{\pgfqpoint{5.259421in}{3.054560in}}%
\pgfpathlineto{\pgfqpoint{5.251814in}{3.048457in}}%
\pgfpathlineto{\pgfqpoint{5.244197in}{3.042214in}}%
\pgfpathlineto{\pgfqpoint{5.236571in}{3.035830in}}%
\pgfpathclose%
\pgfusepath{fill}%
\end{pgfscope}%
\begin{pgfscope}%
\pgfpathrectangle{\pgfqpoint{1.150000in}{0.150000in}}{\pgfqpoint{5.700000in}{5.700000in}}%
\pgfusepath{clip}%
\pgfsetbuttcap%
\pgfsetroundjoin%
\definecolor{currentfill}{rgb}{0.127568,0.566949,0.550556}%
\pgfsetfillcolor{currentfill}%
\pgfsetfillopacity{0.800000}%
\pgfsetlinewidth{0.000000pt}%
\definecolor{currentstroke}{rgb}{0.000000,0.000000,0.000000}%
\pgfsetstrokecolor{currentstroke}%
\pgfsetdash{}{0pt}%
\pgfpathmoveto{\pgfqpoint{5.027167in}{2.831242in}}%
\pgfpathlineto{\pgfqpoint{5.041744in}{2.845092in}}%
\pgfpathlineto{\pgfqpoint{5.056340in}{2.859127in}}%
\pgfpathlineto{\pgfqpoint{5.070957in}{2.873349in}}%
\pgfpathlineto{\pgfqpoint{5.085593in}{2.887757in}}%
\pgfpathlineto{\pgfqpoint{5.093335in}{2.896136in}}%
\pgfpathlineto{\pgfqpoint{5.101068in}{2.904357in}}%
\pgfpathlineto{\pgfqpoint{5.108793in}{2.912419in}}%
\pgfpathlineto{\pgfqpoint{5.116509in}{2.920326in}}%
\pgfpathlineto{\pgfqpoint{5.101879in}{2.906039in}}%
\pgfpathlineto{\pgfqpoint{5.087269in}{2.891938in}}%
\pgfpathlineto{\pgfqpoint{5.072679in}{2.878023in}}%
\pgfpathlineto{\pgfqpoint{5.058109in}{2.864293in}}%
\pgfpathlineto{\pgfqpoint{5.050386in}{2.856253in}}%
\pgfpathlineto{\pgfqpoint{5.042655in}{2.848066in}}%
\pgfpathlineto{\pgfqpoint{5.034915in}{2.839729in}}%
\pgfpathlineto{\pgfqpoint{5.027167in}{2.831242in}}%
\pgfpathclose%
\pgfusepath{fill}%
\end{pgfscope}%
\begin{pgfscope}%
\pgfpathrectangle{\pgfqpoint{1.150000in}{0.150000in}}{\pgfqpoint{5.700000in}{5.700000in}}%
\pgfusepath{clip}%
\pgfsetbuttcap%
\pgfsetroundjoin%
\definecolor{currentfill}{rgb}{0.202219,0.715272,0.476084}%
\pgfsetfillcolor{currentfill}%
\pgfsetfillopacity{0.800000}%
\pgfsetlinewidth{0.000000pt}%
\definecolor{currentstroke}{rgb}{0.000000,0.000000,0.000000}%
\pgfsetstrokecolor{currentstroke}%
\pgfsetdash{}{0pt}%
\pgfpathmoveto{\pgfqpoint{5.534801in}{3.299886in}}%
\pgfpathlineto{\pgfqpoint{5.549718in}{3.315864in}}%
\pgfpathlineto{\pgfqpoint{5.564658in}{3.332028in}}%
\pgfpathlineto{\pgfqpoint{5.579620in}{3.348378in}}%
\pgfpathlineto{\pgfqpoint{5.594605in}{3.364915in}}%
\pgfpathlineto{\pgfqpoint{5.602014in}{3.367981in}}%
\pgfpathlineto{\pgfqpoint{5.609413in}{3.370924in}}%
\pgfpathlineto{\pgfqpoint{5.616802in}{3.373749in}}%
\pgfpathlineto{\pgfqpoint{5.624180in}{3.376461in}}%
\pgfpathlineto{\pgfqpoint{5.609215in}{3.360297in}}%
\pgfpathlineto{\pgfqpoint{5.594272in}{3.344319in}}%
\pgfpathlineto{\pgfqpoint{5.579353in}{3.328526in}}%
\pgfpathlineto{\pgfqpoint{5.564455in}{3.312919in}}%
\pgfpathlineto{\pgfqpoint{5.557056in}{3.309823in}}%
\pgfpathlineto{\pgfqpoint{5.549648in}{3.306621in}}%
\pgfpathlineto{\pgfqpoint{5.542229in}{3.303310in}}%
\pgfpathlineto{\pgfqpoint{5.534801in}{3.299886in}}%
\pgfpathclose%
\pgfusepath{fill}%
\end{pgfscope}%
\begin{pgfscope}%
\pgfpathrectangle{\pgfqpoint{1.150000in}{0.150000in}}{\pgfqpoint{5.700000in}{5.700000in}}%
\pgfusepath{clip}%
\pgfsetbuttcap%
\pgfsetroundjoin%
\definecolor{currentfill}{rgb}{0.229739,0.322361,0.545706}%
\pgfsetfillcolor{currentfill}%
\pgfsetfillopacity{0.800000}%
\pgfsetlinewidth{0.000000pt}%
\definecolor{currentstroke}{rgb}{0.000000,0.000000,0.000000}%
\pgfsetstrokecolor{currentstroke}%
\pgfsetdash{}{0pt}%
\pgfpathmoveto{\pgfqpoint{4.366403in}{2.100327in}}%
\pgfpathlineto{\pgfqpoint{4.380579in}{2.109043in}}%
\pgfpathlineto{\pgfqpoint{4.394770in}{2.117943in}}%
\pgfpathlineto{\pgfqpoint{4.408975in}{2.127027in}}%
\pgfpathlineto{\pgfqpoint{4.423196in}{2.136294in}}%
\pgfpathlineto{\pgfqpoint{4.431215in}{2.149956in}}%
\pgfpathlineto{\pgfqpoint{4.439229in}{2.163516in}}%
\pgfpathlineto{\pgfqpoint{4.447238in}{2.176972in}}%
\pgfpathlineto{\pgfqpoint{4.455242in}{2.190323in}}%
\pgfpathlineto{\pgfqpoint{4.441020in}{2.180835in}}%
\pgfpathlineto{\pgfqpoint{4.426814in}{2.171530in}}%
\pgfpathlineto{\pgfqpoint{4.412623in}{2.162410in}}%
\pgfpathlineto{\pgfqpoint{4.398446in}{2.153474in}}%
\pgfpathlineto{\pgfqpoint{4.390443in}{2.140331in}}%
\pgfpathlineto{\pgfqpoint{4.382434in}{2.127091in}}%
\pgfpathlineto{\pgfqpoint{4.374421in}{2.113756in}}%
\pgfpathlineto{\pgfqpoint{4.366403in}{2.100327in}}%
\pgfpathclose%
\pgfusepath{fill}%
\end{pgfscope}%
\begin{pgfscope}%
\pgfpathrectangle{\pgfqpoint{1.150000in}{0.150000in}}{\pgfqpoint{5.700000in}{5.700000in}}%
\pgfusepath{clip}%
\pgfsetbuttcap%
\pgfsetroundjoin%
\definecolor{currentfill}{rgb}{0.278012,0.180367,0.486697}%
\pgfsetfillcolor{currentfill}%
\pgfsetfillopacity{0.800000}%
\pgfsetlinewidth{0.000000pt}%
\definecolor{currentstroke}{rgb}{0.000000,0.000000,0.000000}%
\pgfsetstrokecolor{currentstroke}%
\pgfsetdash{}{0pt}%
\pgfpathmoveto{\pgfqpoint{4.035987in}{1.744474in}}%
\pgfpathlineto{\pgfqpoint{4.050011in}{1.749402in}}%
\pgfpathlineto{\pgfqpoint{4.064046in}{1.754514in}}%
\pgfpathlineto{\pgfqpoint{4.078093in}{1.759809in}}%
\pgfpathlineto{\pgfqpoint{4.092151in}{1.765287in}}%
\pgfpathlineto{\pgfqpoint{4.100261in}{1.779282in}}%
\pgfpathlineto{\pgfqpoint{4.108367in}{1.793255in}}%
\pgfpathlineto{\pgfqpoint{4.116468in}{1.807201in}}%
\pgfpathlineto{\pgfqpoint{4.124565in}{1.821116in}}%
\pgfpathlineto{\pgfqpoint{4.110508in}{1.815258in}}%
\pgfpathlineto{\pgfqpoint{4.096463in}{1.809583in}}%
\pgfpathlineto{\pgfqpoint{4.082430in}{1.804091in}}%
\pgfpathlineto{\pgfqpoint{4.068409in}{1.798784in}}%
\pgfpathlineto{\pgfqpoint{4.060310in}{1.785236in}}%
\pgfpathlineto{\pgfqpoint{4.052207in}{1.771666in}}%
\pgfpathlineto{\pgfqpoint{4.044100in}{1.758078in}}%
\pgfpathlineto{\pgfqpoint{4.035987in}{1.744474in}}%
\pgfpathclose%
\pgfusepath{fill}%
\end{pgfscope}%
\begin{pgfscope}%
\pgfpathrectangle{\pgfqpoint{1.150000in}{0.150000in}}{\pgfqpoint{5.700000in}{5.700000in}}%
\pgfusepath{clip}%
\pgfsetbuttcap%
\pgfsetroundjoin%
\definecolor{currentfill}{rgb}{0.274952,0.037752,0.364543}%
\pgfsetfillcolor{currentfill}%
\pgfsetfillopacity{0.800000}%
\pgfsetlinewidth{0.000000pt}%
\definecolor{currentstroke}{rgb}{0.000000,0.000000,0.000000}%
\pgfsetstrokecolor{currentstroke}%
\pgfsetdash{}{0pt}%
\pgfpathmoveto{\pgfqpoint{3.649231in}{1.459369in}}%
\pgfpathlineto{\pgfqpoint{3.663144in}{1.458977in}}%
\pgfpathlineto{\pgfqpoint{3.677064in}{1.458772in}}%
\pgfpathlineto{\pgfqpoint{3.690991in}{1.458754in}}%
\pgfpathlineto{\pgfqpoint{3.704926in}{1.458921in}}%
\pgfpathlineto{\pgfqpoint{3.713159in}{1.470328in}}%
\pgfpathlineto{\pgfqpoint{3.721386in}{1.481830in}}%
\pgfpathlineto{\pgfqpoint{3.729607in}{1.493423in}}%
\pgfpathlineto{\pgfqpoint{3.737823in}{1.505100in}}%
\pgfpathlineto{\pgfqpoint{3.723899in}{1.504399in}}%
\pgfpathlineto{\pgfqpoint{3.709983in}{1.503884in}}%
\pgfpathlineto{\pgfqpoint{3.696075in}{1.503556in}}%
\pgfpathlineto{\pgfqpoint{3.682174in}{1.503414in}}%
\pgfpathlineto{\pgfqpoint{3.673948in}{1.492258in}}%
\pgfpathlineto{\pgfqpoint{3.665715in}{1.481196in}}%
\pgfpathlineto{\pgfqpoint{3.657476in}{1.470231in}}%
\pgfpathlineto{\pgfqpoint{3.649231in}{1.459369in}}%
\pgfpathclose%
\pgfusepath{fill}%
\end{pgfscope}%
\begin{pgfscope}%
\pgfpathrectangle{\pgfqpoint{1.150000in}{0.150000in}}{\pgfqpoint{5.700000in}{5.700000in}}%
\pgfusepath{clip}%
\pgfsetbuttcap%
\pgfsetroundjoin%
\definecolor{currentfill}{rgb}{0.267004,0.004874,0.329415}%
\pgfsetfillcolor{currentfill}%
\pgfsetfillopacity{0.800000}%
\pgfsetlinewidth{0.000000pt}%
\definecolor{currentstroke}{rgb}{0.000000,0.000000,0.000000}%
\pgfsetstrokecolor{currentstroke}%
\pgfsetdash{}{0pt}%
\pgfpathmoveto{\pgfqpoint{3.327074in}{1.399702in}}%
\pgfpathlineto{\pgfqpoint{3.340962in}{1.394471in}}%
\pgfpathlineto{\pgfqpoint{3.354853in}{1.389435in}}%
\pgfpathlineto{\pgfqpoint{3.368747in}{1.384593in}}%
\pgfpathlineto{\pgfqpoint{3.382645in}{1.379945in}}%
\pgfpathlineto{\pgfqpoint{3.391037in}{1.387217in}}%
\pgfpathlineto{\pgfqpoint{3.399419in}{1.394684in}}%
\pgfpathlineto{\pgfqpoint{3.407793in}{1.402341in}}%
\pgfpathlineto{\pgfqpoint{3.416158in}{1.410181in}}%
\pgfpathlineto{\pgfqpoint{3.402282in}{1.414201in}}%
\pgfpathlineto{\pgfqpoint{3.388411in}{1.418415in}}%
\pgfpathlineto{\pgfqpoint{3.374544in}{1.422824in}}%
\pgfpathlineto{\pgfqpoint{3.360680in}{1.427427in}}%
\pgfpathlineto{\pgfqpoint{3.352293in}{1.420203in}}%
\pgfpathlineto{\pgfqpoint{3.343896in}{1.413169in}}%
\pgfpathlineto{\pgfqpoint{3.335490in}{1.406334in}}%
\pgfpathlineto{\pgfqpoint{3.327074in}{1.399702in}}%
\pgfpathclose%
\pgfusepath{fill}%
\end{pgfscope}%
\begin{pgfscope}%
\pgfpathrectangle{\pgfqpoint{1.150000in}{0.150000in}}{\pgfqpoint{5.700000in}{5.700000in}}%
\pgfusepath{clip}%
\pgfsetbuttcap%
\pgfsetroundjoin%
\definecolor{currentfill}{rgb}{0.144759,0.519093,0.556572}%
\pgfsetfillcolor{currentfill}%
\pgfsetfillopacity{0.800000}%
\pgfsetlinewidth{0.000000pt}%
\definecolor{currentstroke}{rgb}{0.000000,0.000000,0.000000}%
\pgfsetstrokecolor{currentstroke}%
\pgfsetdash{}{0pt}%
\pgfpathmoveto{\pgfqpoint{2.055719in}{2.793953in}}%
\pgfpathlineto{\pgfqpoint{2.070244in}{2.765568in}}%
\pgfpathlineto{\pgfqpoint{2.084749in}{2.737539in}}%
\pgfpathlineto{\pgfqpoint{2.099235in}{2.709862in}}%
\pgfpathlineto{\pgfqpoint{2.113702in}{2.682534in}}%
\pgfpathlineto{\pgfqpoint{2.123183in}{2.673562in}}%
\pgfpathlineto{\pgfqpoint{2.132636in}{2.665014in}}%
\pgfpathlineto{\pgfqpoint{2.142061in}{2.656880in}}%
\pgfpathlineto{\pgfqpoint{2.151459in}{2.649155in}}%
\pgfpathlineto{\pgfqpoint{2.137064in}{2.675740in}}%
\pgfpathlineto{\pgfqpoint{2.122650in}{2.702672in}}%
\pgfpathlineto{\pgfqpoint{2.108219in}{2.729954in}}%
\pgfpathlineto{\pgfqpoint{2.093768in}{2.757591in}}%
\pgfpathlineto{\pgfqpoint{2.084299in}{2.766046in}}%
\pgfpathlineto{\pgfqpoint{2.074801in}{2.774920in}}%
\pgfpathlineto{\pgfqpoint{2.065275in}{2.784220in}}%
\pgfpathlineto{\pgfqpoint{2.055719in}{2.793953in}}%
\pgfpathclose%
\pgfusepath{fill}%
\end{pgfscope}%
\begin{pgfscope}%
\pgfpathrectangle{\pgfqpoint{1.150000in}{0.150000in}}{\pgfqpoint{5.700000in}{5.700000in}}%
\pgfusepath{clip}%
\pgfsetbuttcap%
\pgfsetroundjoin%
\definecolor{currentfill}{rgb}{0.279566,0.067836,0.391917}%
\pgfsetfillcolor{currentfill}%
\pgfsetfillopacity{0.800000}%
\pgfsetlinewidth{0.000000pt}%
\definecolor{currentstroke}{rgb}{0.000000,0.000000,0.000000}%
\pgfsetstrokecolor{currentstroke}%
\pgfsetdash{}{0pt}%
\pgfpathmoveto{\pgfqpoint{3.737823in}{1.505100in}}%
\pgfpathlineto{\pgfqpoint{3.751754in}{1.505987in}}%
\pgfpathlineto{\pgfqpoint{3.765694in}{1.507059in}}%
\pgfpathlineto{\pgfqpoint{3.779642in}{1.508317in}}%
\pgfpathlineto{\pgfqpoint{3.793598in}{1.509759in}}%
\pgfpathlineto{\pgfqpoint{3.801798in}{1.522032in}}%
\pgfpathlineto{\pgfqpoint{3.809993in}{1.534372in}}%
\pgfpathlineto{\pgfqpoint{3.818183in}{1.546775in}}%
\pgfpathlineto{\pgfqpoint{3.826367in}{1.559235in}}%
\pgfpathlineto{\pgfqpoint{3.812419in}{1.557290in}}%
\pgfpathlineto{\pgfqpoint{3.798480in}{1.555529in}}%
\pgfpathlineto{\pgfqpoint{3.784550in}{1.553954in}}%
\pgfpathlineto{\pgfqpoint{3.770627in}{1.552565in}}%
\pgfpathlineto{\pgfqpoint{3.762435in}{1.540595in}}%
\pgfpathlineto{\pgfqpoint{3.754236in}{1.528691in}}%
\pgfpathlineto{\pgfqpoint{3.746032in}{1.516858in}}%
\pgfpathlineto{\pgfqpoint{3.737823in}{1.505100in}}%
\pgfpathclose%
\pgfusepath{fill}%
\end{pgfscope}%
\begin{pgfscope}%
\pgfpathrectangle{\pgfqpoint{1.150000in}{0.150000in}}{\pgfqpoint{5.700000in}{5.700000in}}%
\pgfusepath{clip}%
\pgfsetbuttcap%
\pgfsetroundjoin%
\definecolor{currentfill}{rgb}{0.195860,0.395433,0.555276}%
\pgfsetfillcolor{currentfill}%
\pgfsetfillopacity{0.800000}%
\pgfsetlinewidth{0.000000pt}%
\definecolor{currentstroke}{rgb}{0.000000,0.000000,0.000000}%
\pgfsetstrokecolor{currentstroke}%
\pgfsetdash{}{0pt}%
\pgfpathmoveto{\pgfqpoint{2.248627in}{2.411616in}}%
\pgfpathlineto{\pgfqpoint{2.262952in}{2.387768in}}%
\pgfpathlineto{\pgfqpoint{2.277262in}{2.364227in}}%
\pgfpathlineto{\pgfqpoint{2.291558in}{2.340990in}}%
\pgfpathlineto{\pgfqpoint{2.305841in}{2.318054in}}%
\pgfpathlineto{\pgfqpoint{2.315152in}{2.310444in}}%
\pgfpathlineto{\pgfqpoint{2.324437in}{2.303246in}}%
\pgfpathlineto{\pgfqpoint{2.333697in}{2.296453in}}%
\pgfpathlineto{\pgfqpoint{2.342931in}{2.290056in}}%
\pgfpathlineto{\pgfqpoint{2.328714in}{2.312239in}}%
\pgfpathlineto{\pgfqpoint{2.314484in}{2.334722in}}%
\pgfpathlineto{\pgfqpoint{2.300241in}{2.357507in}}%
\pgfpathlineto{\pgfqpoint{2.285984in}{2.380597in}}%
\pgfpathlineto{\pgfqpoint{2.276684in}{2.387733in}}%
\pgfpathlineto{\pgfqpoint{2.267358in}{2.395277in}}%
\pgfpathlineto{\pgfqpoint{2.258006in}{2.403235in}}%
\pgfpathlineto{\pgfqpoint{2.248627in}{2.411616in}}%
\pgfpathclose%
\pgfusepath{fill}%
\end{pgfscope}%
\begin{pgfscope}%
\pgfpathrectangle{\pgfqpoint{1.150000in}{0.150000in}}{\pgfqpoint{5.700000in}{5.700000in}}%
\pgfusepath{clip}%
\pgfsetbuttcap%
\pgfsetroundjoin%
\definecolor{currentfill}{rgb}{0.268510,0.009605,0.335427}%
\pgfsetfillcolor{currentfill}%
\pgfsetfillopacity{0.800000}%
\pgfsetlinewidth{0.000000pt}%
\definecolor{currentstroke}{rgb}{0.000000,0.000000,0.000000}%
\pgfsetstrokecolor{currentstroke}%
\pgfsetdash{}{0pt}%
\pgfpathmoveto{\pgfqpoint{3.182070in}{1.429590in}}%
\pgfpathlineto{\pgfqpoint{3.195968in}{1.422110in}}%
\pgfpathlineto{\pgfqpoint{3.209868in}{1.414831in}}%
\pgfpathlineto{\pgfqpoint{3.223770in}{1.407753in}}%
\pgfpathlineto{\pgfqpoint{3.237673in}{1.400875in}}%
\pgfpathlineto{\pgfqpoint{3.246159in}{1.405954in}}%
\pgfpathlineto{\pgfqpoint{3.254634in}{1.411271in}}%
\pgfpathlineto{\pgfqpoint{3.263098in}{1.416819in}}%
\pgfpathlineto{\pgfqpoint{3.271551in}{1.422592in}}%
\pgfpathlineto{\pgfqpoint{3.257676in}{1.428809in}}%
\pgfpathlineto{\pgfqpoint{3.243804in}{1.435226in}}%
\pgfpathlineto{\pgfqpoint{3.229933in}{1.441842in}}%
\pgfpathlineto{\pgfqpoint{3.216064in}{1.448660in}}%
\pgfpathlineto{\pgfqpoint{3.207583in}{1.443536in}}%
\pgfpathlineto{\pgfqpoint{3.199090in}{1.438646in}}%
\pgfpathlineto{\pgfqpoint{3.190586in}{1.433995in}}%
\pgfpathlineto{\pgfqpoint{3.182070in}{1.429590in}}%
\pgfpathclose%
\pgfusepath{fill}%
\end{pgfscope}%
\begin{pgfscope}%
\pgfpathrectangle{\pgfqpoint{1.150000in}{0.150000in}}{\pgfqpoint{5.700000in}{5.700000in}}%
\pgfusepath{clip}%
\pgfsetbuttcap%
\pgfsetroundjoin%
\definecolor{currentfill}{rgb}{0.171176,0.452530,0.557965}%
\pgfsetfillcolor{currentfill}%
\pgfsetfillopacity{0.800000}%
\pgfsetlinewidth{0.000000pt}%
\definecolor{currentstroke}{rgb}{0.000000,0.000000,0.000000}%
\pgfsetstrokecolor{currentstroke}%
\pgfsetdash{}{0pt}%
\pgfpathmoveto{\pgfqpoint{4.696908in}{2.475478in}}%
\pgfpathlineto{\pgfqpoint{4.711277in}{2.487179in}}%
\pgfpathlineto{\pgfqpoint{4.725664in}{2.499066in}}%
\pgfpathlineto{\pgfqpoint{4.740068in}{2.511137in}}%
\pgfpathlineto{\pgfqpoint{4.754490in}{2.523394in}}%
\pgfpathlineto{\pgfqpoint{4.762395in}{2.534969in}}%
\pgfpathlineto{\pgfqpoint{4.770293in}{2.546396in}}%
\pgfpathlineto{\pgfqpoint{4.778185in}{2.557675in}}%
\pgfpathlineto{\pgfqpoint{4.786069in}{2.568807in}}%
\pgfpathlineto{\pgfqpoint{4.771648in}{2.556496in}}%
\pgfpathlineto{\pgfqpoint{4.757245in}{2.544371in}}%
\pgfpathlineto{\pgfqpoint{4.742859in}{2.532431in}}%
\pgfpathlineto{\pgfqpoint{4.728490in}{2.520676in}}%
\pgfpathlineto{\pgfqpoint{4.720605in}{2.509585in}}%
\pgfpathlineto{\pgfqpoint{4.712712in}{2.498355in}}%
\pgfpathlineto{\pgfqpoint{4.704813in}{2.486986in}}%
\pgfpathlineto{\pgfqpoint{4.696908in}{2.475478in}}%
\pgfpathclose%
\pgfusepath{fill}%
\end{pgfscope}%
\begin{pgfscope}%
\pgfpathrectangle{\pgfqpoint{1.150000in}{0.150000in}}{\pgfqpoint{5.700000in}{5.700000in}}%
\pgfusepath{clip}%
\pgfsetbuttcap%
\pgfsetroundjoin%
\definecolor{currentfill}{rgb}{0.271305,0.019942,0.347269}%
\pgfsetfillcolor{currentfill}%
\pgfsetfillopacity{0.800000}%
\pgfsetlinewidth{0.000000pt}%
\definecolor{currentstroke}{rgb}{0.000000,0.000000,0.000000}%
\pgfsetstrokecolor{currentstroke}%
\pgfsetdash{}{0pt}%
\pgfpathmoveto{\pgfqpoint{3.560543in}{1.422761in}}%
\pgfpathlineto{\pgfqpoint{3.574444in}{1.421054in}}%
\pgfpathlineto{\pgfqpoint{3.588351in}{1.419535in}}%
\pgfpathlineto{\pgfqpoint{3.602265in}{1.418204in}}%
\pgfpathlineto{\pgfqpoint{3.616185in}{1.417061in}}%
\pgfpathlineto{\pgfqpoint{3.624456in}{1.427456in}}%
\pgfpathlineto{\pgfqpoint{3.632721in}{1.437977in}}%
\pgfpathlineto{\pgfqpoint{3.640979in}{1.448616in}}%
\pgfpathlineto{\pgfqpoint{3.649231in}{1.459369in}}%
\pgfpathlineto{\pgfqpoint{3.635325in}{1.459948in}}%
\pgfpathlineto{\pgfqpoint{3.621426in}{1.460714in}}%
\pgfpathlineto{\pgfqpoint{3.607533in}{1.461669in}}%
\pgfpathlineto{\pgfqpoint{3.593647in}{1.462812in}}%
\pgfpathlineto{\pgfqpoint{3.585382in}{1.452611in}}%
\pgfpathlineto{\pgfqpoint{3.577109in}{1.442532in}}%
\pgfpathlineto{\pgfqpoint{3.568829in}{1.432580in}}%
\pgfpathlineto{\pgfqpoint{3.560543in}{1.422761in}}%
\pgfpathclose%
\pgfusepath{fill}%
\end{pgfscope}%
\begin{pgfscope}%
\pgfpathrectangle{\pgfqpoint{1.150000in}{0.150000in}}{\pgfqpoint{5.700000in}{5.700000in}}%
\pgfusepath{clip}%
\pgfsetbuttcap%
\pgfsetroundjoin%
\definecolor{currentfill}{rgb}{0.277941,0.056324,0.381191}%
\pgfsetfillcolor{currentfill}%
\pgfsetfillopacity{0.800000}%
\pgfsetlinewidth{0.000000pt}%
\definecolor{currentstroke}{rgb}{0.000000,0.000000,0.000000}%
\pgfsetstrokecolor{currentstroke}%
\pgfsetdash{}{0pt}%
\pgfpathmoveto{\pgfqpoint{2.980700in}{1.528724in}}%
\pgfpathlineto{\pgfqpoint{2.994636in}{1.518059in}}%
\pgfpathlineto{\pgfqpoint{3.008570in}{1.507606in}}%
\pgfpathlineto{\pgfqpoint{3.022504in}{1.497366in}}%
\pgfpathlineto{\pgfqpoint{3.036436in}{1.487336in}}%
\pgfpathlineto{\pgfqpoint{3.045074in}{1.489267in}}%
\pgfpathlineto{\pgfqpoint{3.053699in}{1.491488in}}%
\pgfpathlineto{\pgfqpoint{3.062308in}{1.493994in}}%
\pgfpathlineto{\pgfqpoint{3.070904in}{1.496777in}}%
\pgfpathlineto{\pgfqpoint{3.057009in}{1.506108in}}%
\pgfpathlineto{\pgfqpoint{3.043113in}{1.515649in}}%
\pgfpathlineto{\pgfqpoint{3.029216in}{1.525401in}}%
\pgfpathlineto{\pgfqpoint{3.015319in}{1.535366in}}%
\pgfpathlineto{\pgfqpoint{3.006686in}{1.533270in}}%
\pgfpathlineto{\pgfqpoint{2.998039in}{1.531460in}}%
\pgfpathlineto{\pgfqpoint{2.989377in}{1.529942in}}%
\pgfpathlineto{\pgfqpoint{2.980700in}{1.528724in}}%
\pgfpathclose%
\pgfusepath{fill}%
\end{pgfscope}%
\begin{pgfscope}%
\pgfpathrectangle{\pgfqpoint{1.150000in}{0.150000in}}{\pgfqpoint{5.700000in}{5.700000in}}%
\pgfusepath{clip}%
\pgfsetbuttcap%
\pgfsetroundjoin%
\definecolor{currentfill}{rgb}{0.252194,0.269783,0.531579}%
\pgfsetfillcolor{currentfill}%
\pgfsetfillopacity{0.800000}%
\pgfsetlinewidth{0.000000pt}%
\definecolor{currentstroke}{rgb}{0.000000,0.000000,0.000000}%
\pgfsetstrokecolor{currentstroke}%
\pgfsetdash{}{0pt}%
\pgfpathmoveto{\pgfqpoint{4.245525in}{1.958911in}}%
\pgfpathlineto{\pgfqpoint{4.259644in}{1.966353in}}%
\pgfpathlineto{\pgfqpoint{4.273777in}{1.973979in}}%
\pgfpathlineto{\pgfqpoint{4.287923in}{1.981789in}}%
\pgfpathlineto{\pgfqpoint{4.302084in}{1.989782in}}%
\pgfpathlineto{\pgfqpoint{4.310140in}{2.003882in}}%
\pgfpathlineto{\pgfqpoint{4.318192in}{2.017906in}}%
\pgfpathlineto{\pgfqpoint{4.326239in}{2.031852in}}%
\pgfpathlineto{\pgfqpoint{4.334281in}{2.045718in}}%
\pgfpathlineto{\pgfqpoint{4.320120in}{2.037439in}}%
\pgfpathlineto{\pgfqpoint{4.305973in}{2.029344in}}%
\pgfpathlineto{\pgfqpoint{4.291840in}{2.021433in}}%
\pgfpathlineto{\pgfqpoint{4.277720in}{2.013706in}}%
\pgfpathlineto{\pgfqpoint{4.269678in}{2.000113in}}%
\pgfpathlineto{\pgfqpoint{4.261632in}{1.986448in}}%
\pgfpathlineto{\pgfqpoint{4.253581in}{1.972714in}}%
\pgfpathlineto{\pgfqpoint{4.245525in}{1.958911in}}%
\pgfpathclose%
\pgfusepath{fill}%
\end{pgfscope}%
\begin{pgfscope}%
\pgfpathrectangle{\pgfqpoint{1.150000in}{0.150000in}}{\pgfqpoint{5.700000in}{5.700000in}}%
\pgfusepath{clip}%
\pgfsetbuttcap%
\pgfsetroundjoin%
\definecolor{currentfill}{rgb}{0.282327,0.094955,0.417331}%
\pgfsetfillcolor{currentfill}%
\pgfsetfillopacity{0.800000}%
\pgfsetlinewidth{0.000000pt}%
\definecolor{currentstroke}{rgb}{0.000000,0.000000,0.000000}%
\pgfsetstrokecolor{currentstroke}%
\pgfsetdash{}{0pt}%
\pgfpathmoveto{\pgfqpoint{3.826367in}{1.559235in}}%
\pgfpathlineto{\pgfqpoint{3.840324in}{1.561366in}}%
\pgfpathlineto{\pgfqpoint{3.854291in}{1.563680in}}%
\pgfpathlineto{\pgfqpoint{3.868266in}{1.566179in}}%
\pgfpathlineto{\pgfqpoint{3.882251in}{1.568862in}}%
\pgfpathlineto{\pgfqpoint{3.890423in}{1.581860in}}%
\pgfpathlineto{\pgfqpoint{3.898591in}{1.594900in}}%
\pgfpathlineto{\pgfqpoint{3.906753in}{1.607976in}}%
\pgfpathlineto{\pgfqpoint{3.914911in}{1.621084in}}%
\pgfpathlineto{\pgfqpoint{3.900932in}{1.617928in}}%
\pgfpathlineto{\pgfqpoint{3.886963in}{1.614956in}}%
\pgfpathlineto{\pgfqpoint{3.873004in}{1.612169in}}%
\pgfpathlineto{\pgfqpoint{3.859053in}{1.609567in}}%
\pgfpathlineto{\pgfqpoint{3.850890in}{1.596919in}}%
\pgfpathlineto{\pgfqpoint{3.842721in}{1.584312in}}%
\pgfpathlineto{\pgfqpoint{3.834547in}{1.571749in}}%
\pgfpathlineto{\pgfqpoint{3.826367in}{1.559235in}}%
\pgfpathclose%
\pgfusepath{fill}%
\end{pgfscope}%
\begin{pgfscope}%
\pgfpathrectangle{\pgfqpoint{1.150000in}{0.150000in}}{\pgfqpoint{5.700000in}{5.700000in}}%
\pgfusepath{clip}%
\pgfsetbuttcap%
\pgfsetroundjoin%
\definecolor{currentfill}{rgb}{0.246070,0.738910,0.452024}%
\pgfsetfillcolor{currentfill}%
\pgfsetfillopacity{0.800000}%
\pgfsetlinewidth{0.000000pt}%
\definecolor{currentstroke}{rgb}{0.000000,0.000000,0.000000}%
\pgfsetstrokecolor{currentstroke}%
\pgfsetdash{}{0pt}%
\pgfpathmoveto{\pgfqpoint{5.624180in}{3.376461in}}%
\pgfpathlineto{\pgfqpoint{5.639168in}{3.392811in}}%
\pgfpathlineto{\pgfqpoint{5.654179in}{3.409348in}}%
\pgfpathlineto{\pgfqpoint{5.669214in}{3.426071in}}%
\pgfpathlineto{\pgfqpoint{5.684272in}{3.442981in}}%
\pgfpathlineto{\pgfqpoint{5.691618in}{3.445187in}}%
\pgfpathlineto{\pgfqpoint{5.698954in}{3.447280in}}%
\pgfpathlineto{\pgfqpoint{5.706280in}{3.449264in}}%
\pgfpathlineto{\pgfqpoint{5.713596in}{3.451145in}}%
\pgfpathlineto{\pgfqpoint{5.698561in}{3.434645in}}%
\pgfpathlineto{\pgfqpoint{5.683549in}{3.418331in}}%
\pgfpathlineto{\pgfqpoint{5.668560in}{3.402203in}}%
\pgfpathlineto{\pgfqpoint{5.653594in}{3.386260in}}%
\pgfpathlineto{\pgfqpoint{5.646255in}{3.383959in}}%
\pgfpathlineto{\pgfqpoint{5.638907in}{3.381561in}}%
\pgfpathlineto{\pgfqpoint{5.631548in}{3.379064in}}%
\pgfpathlineto{\pgfqpoint{5.624180in}{3.376461in}}%
\pgfpathclose%
\pgfusepath{fill}%
\end{pgfscope}%
\begin{pgfscope}%
\pgfpathrectangle{\pgfqpoint{1.150000in}{0.150000in}}{\pgfqpoint{5.700000in}{5.700000in}}%
\pgfusepath{clip}%
\pgfsetbuttcap%
\pgfsetroundjoin%
\definecolor{currentfill}{rgb}{0.137339,0.662252,0.515571}%
\pgfsetfillcolor{currentfill}%
\pgfsetfillopacity{0.800000}%
\pgfsetlinewidth{0.000000pt}%
\definecolor{currentstroke}{rgb}{0.000000,0.000000,0.000000}%
\pgfsetstrokecolor{currentstroke}%
\pgfsetdash{}{0pt}%
\pgfpathmoveto{\pgfqpoint{5.325958in}{3.120214in}}%
\pgfpathlineto{\pgfqpoint{5.340746in}{3.135601in}}%
\pgfpathlineto{\pgfqpoint{5.355555in}{3.151174in}}%
\pgfpathlineto{\pgfqpoint{5.370386in}{3.166934in}}%
\pgfpathlineto{\pgfqpoint{5.385238in}{3.182882in}}%
\pgfpathlineto{\pgfqpoint{5.392802in}{3.188214in}}%
\pgfpathlineto{\pgfqpoint{5.400357in}{3.193399in}}%
\pgfpathlineto{\pgfqpoint{5.407901in}{3.198439in}}%
\pgfpathlineto{\pgfqpoint{5.415435in}{3.203337in}}%
\pgfpathlineto{\pgfqpoint{5.400596in}{3.187655in}}%
\pgfpathlineto{\pgfqpoint{5.385778in}{3.172159in}}%
\pgfpathlineto{\pgfqpoint{5.370983in}{3.156849in}}%
\pgfpathlineto{\pgfqpoint{5.356208in}{3.141725in}}%
\pgfpathlineto{\pgfqpoint{5.348660in}{3.136550in}}%
\pgfpathlineto{\pgfqpoint{5.341102in}{3.131242in}}%
\pgfpathlineto{\pgfqpoint{5.333535in}{3.125797in}}%
\pgfpathlineto{\pgfqpoint{5.325958in}{3.120214in}}%
\pgfpathclose%
\pgfusepath{fill}%
\end{pgfscope}%
\begin{pgfscope}%
\pgfpathrectangle{\pgfqpoint{1.150000in}{0.150000in}}{\pgfqpoint{5.700000in}{5.700000in}}%
\pgfusepath{clip}%
\pgfsetbuttcap%
\pgfsetroundjoin%
\definecolor{currentfill}{rgb}{0.268510,0.009605,0.335427}%
\pgfsetfillcolor{currentfill}%
\pgfsetfillopacity{0.800000}%
\pgfsetlinewidth{0.000000pt}%
\definecolor{currentstroke}{rgb}{0.000000,0.000000,0.000000}%
\pgfsetstrokecolor{currentstroke}%
\pgfsetdash{}{0pt}%
\pgfpathmoveto{\pgfqpoint{3.471701in}{1.396025in}}%
\pgfpathlineto{\pgfqpoint{3.485598in}{1.392964in}}%
\pgfpathlineto{\pgfqpoint{3.499501in}{1.390094in}}%
\pgfpathlineto{\pgfqpoint{3.513408in}{1.387413in}}%
\pgfpathlineto{\pgfqpoint{3.527321in}{1.384921in}}%
\pgfpathlineto{\pgfqpoint{3.535638in}{1.394154in}}%
\pgfpathlineto{\pgfqpoint{3.543947in}{1.403542in}}%
\pgfpathlineto{\pgfqpoint{3.552248in}{1.413080in}}%
\pgfpathlineto{\pgfqpoint{3.560543in}{1.422761in}}%
\pgfpathlineto{\pgfqpoint{3.546647in}{1.424657in}}%
\pgfpathlineto{\pgfqpoint{3.532757in}{1.426742in}}%
\pgfpathlineto{\pgfqpoint{3.518873in}{1.429016in}}%
\pgfpathlineto{\pgfqpoint{3.504994in}{1.431482in}}%
\pgfpathlineto{\pgfqpoint{3.496683in}{1.422384in}}%
\pgfpathlineto{\pgfqpoint{3.488364in}{1.413438in}}%
\pgfpathlineto{\pgfqpoint{3.480036in}{1.404650in}}%
\pgfpathlineto{\pgfqpoint{3.471701in}{1.396025in}}%
\pgfpathclose%
\pgfusepath{fill}%
\end{pgfscope}%
\begin{pgfscope}%
\pgfpathrectangle{\pgfqpoint{1.150000in}{0.150000in}}{\pgfqpoint{5.700000in}{5.700000in}}%
\pgfusepath{clip}%
\pgfsetbuttcap%
\pgfsetroundjoin%
\definecolor{currentfill}{rgb}{0.188923,0.410910,0.556326}%
\pgfsetfillcolor{currentfill}%
\pgfsetfillopacity{0.800000}%
\pgfsetlinewidth{0.000000pt}%
\definecolor{currentstroke}{rgb}{0.000000,0.000000,0.000000}%
\pgfsetstrokecolor{currentstroke}%
\pgfsetdash{}{0pt}%
\pgfpathmoveto{\pgfqpoint{4.576127in}{2.334326in}}%
\pgfpathlineto{\pgfqpoint{4.590429in}{2.345079in}}%
\pgfpathlineto{\pgfqpoint{4.604748in}{2.356017in}}%
\pgfpathlineto{\pgfqpoint{4.619083in}{2.367139in}}%
\pgfpathlineto{\pgfqpoint{4.633434in}{2.378447in}}%
\pgfpathlineto{\pgfqpoint{4.641390in}{2.391053in}}%
\pgfpathlineto{\pgfqpoint{4.649340in}{2.403524in}}%
\pgfpathlineto{\pgfqpoint{4.657284in}{2.415860in}}%
\pgfpathlineto{\pgfqpoint{4.665221in}{2.428059in}}%
\pgfpathlineto{\pgfqpoint{4.650869in}{2.416630in}}%
\pgfpathlineto{\pgfqpoint{4.636533in}{2.405386in}}%
\pgfpathlineto{\pgfqpoint{4.622214in}{2.394327in}}%
\pgfpathlineto{\pgfqpoint{4.607912in}{2.383452in}}%
\pgfpathlineto{\pgfqpoint{4.599975in}{2.371362in}}%
\pgfpathlineto{\pgfqpoint{4.592032in}{2.359144in}}%
\pgfpathlineto{\pgfqpoint{4.584082in}{2.346798in}}%
\pgfpathlineto{\pgfqpoint{4.576127in}{2.334326in}}%
\pgfpathclose%
\pgfusepath{fill}%
\end{pgfscope}%
\begin{pgfscope}%
\pgfpathrectangle{\pgfqpoint{1.150000in}{0.150000in}}{\pgfqpoint{5.700000in}{5.700000in}}%
\pgfusepath{clip}%
\pgfsetbuttcap%
\pgfsetroundjoin%
\definecolor{currentfill}{rgb}{0.140536,0.530132,0.555659}%
\pgfsetfillcolor{currentfill}%
\pgfsetfillopacity{0.800000}%
\pgfsetlinewidth{0.000000pt}%
\definecolor{currentstroke}{rgb}{0.000000,0.000000,0.000000}%
\pgfsetstrokecolor{currentstroke}%
\pgfsetdash{}{0pt}%
\pgfpathmoveto{\pgfqpoint{4.906739in}{2.703675in}}%
\pgfpathlineto{\pgfqpoint{4.921248in}{2.716917in}}%
\pgfpathlineto{\pgfqpoint{4.935776in}{2.730345in}}%
\pgfpathlineto{\pgfqpoint{4.950322in}{2.743958in}}%
\pgfpathlineto{\pgfqpoint{4.964888in}{2.757758in}}%
\pgfpathlineto{\pgfqpoint{4.972702in}{2.767498in}}%
\pgfpathlineto{\pgfqpoint{4.980507in}{2.777078in}}%
\pgfpathlineto{\pgfqpoint{4.988304in}{2.786498in}}%
\pgfpathlineto{\pgfqpoint{4.996093in}{2.795759in}}%
\pgfpathlineto{\pgfqpoint{4.981531in}{2.782010in}}%
\pgfpathlineto{\pgfqpoint{4.966988in}{2.768447in}}%
\pgfpathlineto{\pgfqpoint{4.952464in}{2.755070in}}%
\pgfpathlineto{\pgfqpoint{4.937959in}{2.741878in}}%
\pgfpathlineto{\pgfqpoint{4.930166in}{2.732553in}}%
\pgfpathlineto{\pgfqpoint{4.922365in}{2.723079in}}%
\pgfpathlineto{\pgfqpoint{4.914556in}{2.713453in}}%
\pgfpathlineto{\pgfqpoint{4.906739in}{2.703675in}}%
\pgfpathclose%
\pgfusepath{fill}%
\end{pgfscope}%
\begin{pgfscope}%
\pgfpathrectangle{\pgfqpoint{1.150000in}{0.150000in}}{\pgfqpoint{5.700000in}{5.700000in}}%
\pgfusepath{clip}%
\pgfsetbuttcap%
\pgfsetroundjoin%
\definecolor{currentfill}{rgb}{0.269308,0.218818,0.509577}%
\pgfsetfillcolor{currentfill}%
\pgfsetfillopacity{0.800000}%
\pgfsetlinewidth{0.000000pt}%
\definecolor{currentstroke}{rgb}{0.000000,0.000000,0.000000}%
\pgfsetstrokecolor{currentstroke}%
\pgfsetdash{}{0pt}%
\pgfpathmoveto{\pgfqpoint{4.124565in}{1.821116in}}%
\pgfpathlineto{\pgfqpoint{4.138634in}{1.827158in}}%
\pgfpathlineto{\pgfqpoint{4.152715in}{1.833384in}}%
\pgfpathlineto{\pgfqpoint{4.166809in}{1.839792in}}%
\pgfpathlineto{\pgfqpoint{4.180915in}{1.846384in}}%
\pgfpathlineto{\pgfqpoint{4.189007in}{1.860628in}}%
\pgfpathlineto{\pgfqpoint{4.197094in}{1.874827in}}%
\pgfpathlineto{\pgfqpoint{4.205177in}{1.888978in}}%
\pgfpathlineto{\pgfqpoint{4.213256in}{1.903078in}}%
\pgfpathlineto{\pgfqpoint{4.199150in}{1.896136in}}%
\pgfpathlineto{\pgfqpoint{4.185056in}{1.889378in}}%
\pgfpathlineto{\pgfqpoint{4.170976in}{1.882804in}}%
\pgfpathlineto{\pgfqpoint{4.156908in}{1.876413in}}%
\pgfpathlineto{\pgfqpoint{4.148829in}{1.862650in}}%
\pgfpathlineto{\pgfqpoint{4.140745in}{1.848844in}}%
\pgfpathlineto{\pgfqpoint{4.132657in}{1.834999in}}%
\pgfpathlineto{\pgfqpoint{4.124565in}{1.821116in}}%
\pgfpathclose%
\pgfusepath{fill}%
\end{pgfscope}%
\begin{pgfscope}%
\pgfpathrectangle{\pgfqpoint{1.150000in}{0.150000in}}{\pgfqpoint{5.700000in}{5.700000in}}%
\pgfusepath{clip}%
\pgfsetbuttcap%
\pgfsetroundjoin%
\definecolor{currentfill}{rgb}{0.283187,0.125848,0.444960}%
\pgfsetfillcolor{currentfill}%
\pgfsetfillopacity{0.800000}%
\pgfsetlinewidth{0.000000pt}%
\definecolor{currentstroke}{rgb}{0.000000,0.000000,0.000000}%
\pgfsetstrokecolor{currentstroke}%
\pgfsetdash{}{0pt}%
\pgfpathmoveto{\pgfqpoint{3.914911in}{1.621084in}}%
\pgfpathlineto{\pgfqpoint{3.928900in}{1.624424in}}%
\pgfpathlineto{\pgfqpoint{3.942899in}{1.627947in}}%
\pgfpathlineto{\pgfqpoint{3.956908in}{1.631654in}}%
\pgfpathlineto{\pgfqpoint{3.970927in}{1.635544in}}%
\pgfpathlineto{\pgfqpoint{3.979076in}{1.649134in}}%
\pgfpathlineto{\pgfqpoint{3.987220in}{1.662740in}}%
\pgfpathlineto{\pgfqpoint{3.995359in}{1.676357in}}%
\pgfpathlineto{\pgfqpoint{4.003494in}{1.689981in}}%
\pgfpathlineto{\pgfqpoint{3.989478in}{1.685648in}}%
\pgfpathlineto{\pgfqpoint{3.975473in}{1.681498in}}%
\pgfpathlineto{\pgfqpoint{3.961479in}{1.677533in}}%
\pgfpathlineto{\pgfqpoint{3.947494in}{1.673751in}}%
\pgfpathlineto{\pgfqpoint{3.939356in}{1.660557in}}%
\pgfpathlineto{\pgfqpoint{3.931212in}{1.647379in}}%
\pgfpathlineto{\pgfqpoint{3.923064in}{1.634219in}}%
\pgfpathlineto{\pgfqpoint{3.914911in}{1.621084in}}%
\pgfpathclose%
\pgfusepath{fill}%
\end{pgfscope}%
\begin{pgfscope}%
\pgfpathrectangle{\pgfqpoint{1.150000in}{0.150000in}}{\pgfqpoint{5.700000in}{5.700000in}}%
\pgfusepath{clip}%
\pgfsetbuttcap%
\pgfsetroundjoin%
\definecolor{currentfill}{rgb}{0.120092,0.600104,0.542530}%
\pgfsetfillcolor{currentfill}%
\pgfsetfillopacity{0.800000}%
\pgfsetlinewidth{0.000000pt}%
\definecolor{currentstroke}{rgb}{0.000000,0.000000,0.000000}%
\pgfsetstrokecolor{currentstroke}%
\pgfsetdash{}{0pt}%
\pgfpathmoveto{\pgfqpoint{5.116509in}{2.920326in}}%
\pgfpathlineto{\pgfqpoint{5.131158in}{2.934799in}}%
\pgfpathlineto{\pgfqpoint{5.145829in}{2.949458in}}%
\pgfpathlineto{\pgfqpoint{5.160519in}{2.964304in}}%
\pgfpathlineto{\pgfqpoint{5.175231in}{2.979337in}}%
\pgfpathlineto{\pgfqpoint{5.182930in}{2.986946in}}%
\pgfpathlineto{\pgfqpoint{5.190621in}{2.994395in}}%
\pgfpathlineto{\pgfqpoint{5.198302in}{3.001685in}}%
\pgfpathlineto{\pgfqpoint{5.205974in}{3.008818in}}%
\pgfpathlineto{\pgfqpoint{5.191271in}{2.993942in}}%
\pgfpathlineto{\pgfqpoint{5.176588in}{2.979253in}}%
\pgfpathlineto{\pgfqpoint{5.161926in}{2.964750in}}%
\pgfpathlineto{\pgfqpoint{5.147284in}{2.950433in}}%
\pgfpathlineto{\pgfqpoint{5.139604in}{2.943130in}}%
\pgfpathlineto{\pgfqpoint{5.131914in}{2.935680in}}%
\pgfpathlineto{\pgfqpoint{5.124216in}{2.928079in}}%
\pgfpathlineto{\pgfqpoint{5.116509in}{2.920326in}}%
\pgfpathclose%
\pgfusepath{fill}%
\end{pgfscope}%
\begin{pgfscope}%
\pgfpathrectangle{\pgfqpoint{1.150000in}{0.150000in}}{\pgfqpoint{5.700000in}{5.700000in}}%
\pgfusepath{clip}%
\pgfsetbuttcap%
\pgfsetroundjoin%
\definecolor{currentfill}{rgb}{0.179019,0.433756,0.557430}%
\pgfsetfillcolor{currentfill}%
\pgfsetfillopacity{0.800000}%
\pgfsetlinewidth{0.000000pt}%
\definecolor{currentstroke}{rgb}{0.000000,0.000000,0.000000}%
\pgfsetstrokecolor{currentstroke}%
\pgfsetdash{}{0pt}%
\pgfpathmoveto{\pgfqpoint{2.191177in}{2.510134in}}%
\pgfpathlineto{\pgfqpoint{2.205563in}{2.485030in}}%
\pgfpathlineto{\pgfqpoint{2.219933in}{2.460244in}}%
\pgfpathlineto{\pgfqpoint{2.234287in}{2.435774in}}%
\pgfpathlineto{\pgfqpoint{2.248627in}{2.411616in}}%
\pgfpathlineto{\pgfqpoint{2.258006in}{2.403235in}}%
\pgfpathlineto{\pgfqpoint{2.267358in}{2.395277in}}%
\pgfpathlineto{\pgfqpoint{2.276684in}{2.387733in}}%
\pgfpathlineto{\pgfqpoint{2.285984in}{2.380597in}}%
\pgfpathlineto{\pgfqpoint{2.271712in}{2.403994in}}%
\pgfpathlineto{\pgfqpoint{2.257427in}{2.427703in}}%
\pgfpathlineto{\pgfqpoint{2.243126in}{2.451725in}}%
\pgfpathlineto{\pgfqpoint{2.228811in}{2.476063in}}%
\pgfpathlineto{\pgfqpoint{2.219444in}{2.483947in}}%
\pgfpathlineto{\pgfqpoint{2.210049in}{2.492248in}}%
\pgfpathlineto{\pgfqpoint{2.200628in}{2.500975in}}%
\pgfpathlineto{\pgfqpoint{2.191177in}{2.510134in}}%
\pgfpathclose%
\pgfusepath{fill}%
\end{pgfscope}%
\begin{pgfscope}%
\pgfpathrectangle{\pgfqpoint{1.150000in}{0.150000in}}{\pgfqpoint{5.700000in}{5.700000in}}%
\pgfusepath{clip}%
\pgfsetbuttcap%
\pgfsetroundjoin%
\definecolor{currentfill}{rgb}{0.276022,0.044167,0.370164}%
\pgfsetfillcolor{currentfill}%
\pgfsetfillopacity{0.800000}%
\pgfsetlinewidth{0.000000pt}%
\definecolor{currentstroke}{rgb}{0.000000,0.000000,0.000000}%
\pgfsetstrokecolor{currentstroke}%
\pgfsetdash{}{0pt}%
\pgfpathmoveto{\pgfqpoint{3.036436in}{1.487336in}}%
\pgfpathlineto{\pgfqpoint{3.050367in}{1.477516in}}%
\pgfpathlineto{\pgfqpoint{3.064298in}{1.467905in}}%
\pgfpathlineto{\pgfqpoint{3.078228in}{1.458502in}}%
\pgfpathlineto{\pgfqpoint{3.092158in}{1.449305in}}%
\pgfpathlineto{\pgfqpoint{3.100760in}{1.451947in}}%
\pgfpathlineto{\pgfqpoint{3.109348in}{1.454871in}}%
\pgfpathlineto{\pgfqpoint{3.117923in}{1.458071in}}%
\pgfpathlineto{\pgfqpoint{3.126484in}{1.461539in}}%
\pgfpathlineto{\pgfqpoint{3.112589in}{1.470038in}}%
\pgfpathlineto{\pgfqpoint{3.098694in}{1.478743in}}%
\pgfpathlineto{\pgfqpoint{3.084799in}{1.487656in}}%
\pgfpathlineto{\pgfqpoint{3.070904in}{1.496777in}}%
\pgfpathlineto{\pgfqpoint{3.062308in}{1.493994in}}%
\pgfpathlineto{\pgfqpoint{3.053699in}{1.491488in}}%
\pgfpathlineto{\pgfqpoint{3.045074in}{1.489267in}}%
\pgfpathlineto{\pgfqpoint{3.036436in}{1.487336in}}%
\pgfpathclose%
\pgfusepath{fill}%
\end{pgfscope}%
\begin{pgfscope}%
\pgfpathrectangle{\pgfqpoint{1.150000in}{0.150000in}}{\pgfqpoint{5.700000in}{5.700000in}}%
\pgfusepath{clip}%
\pgfsetbuttcap%
\pgfsetroundjoin%
\definecolor{currentfill}{rgb}{0.210503,0.363727,0.552206}%
\pgfsetfillcolor{currentfill}%
\pgfsetfillopacity{0.800000}%
\pgfsetlinewidth{0.000000pt}%
\definecolor{currentstroke}{rgb}{0.000000,0.000000,0.000000}%
\pgfsetstrokecolor{currentstroke}%
\pgfsetdash{}{0pt}%
\pgfpathmoveto{\pgfqpoint{4.455242in}{2.190323in}}%
\pgfpathlineto{\pgfqpoint{4.469479in}{2.199996in}}%
\pgfpathlineto{\pgfqpoint{4.483731in}{2.209853in}}%
\pgfpathlineto{\pgfqpoint{4.497999in}{2.219894in}}%
\pgfpathlineto{\pgfqpoint{4.512282in}{2.230120in}}%
\pgfpathlineto{\pgfqpoint{4.520282in}{2.243564in}}%
\pgfpathlineto{\pgfqpoint{4.528277in}{2.256892in}}%
\pgfpathlineto{\pgfqpoint{4.536266in}{2.270101in}}%
\pgfpathlineto{\pgfqpoint{4.544249in}{2.283191in}}%
\pgfpathlineto{\pgfqpoint{4.529965in}{2.272777in}}%
\pgfpathlineto{\pgfqpoint{4.515696in}{2.262548in}}%
\pgfpathlineto{\pgfqpoint{4.501442in}{2.252503in}}%
\pgfpathlineto{\pgfqpoint{4.487205in}{2.242642in}}%
\pgfpathlineto{\pgfqpoint{4.479222in}{2.229728in}}%
\pgfpathlineto{\pgfqpoint{4.471234in}{2.216703in}}%
\pgfpathlineto{\pgfqpoint{4.463241in}{2.203567in}}%
\pgfpathlineto{\pgfqpoint{4.455242in}{2.190323in}}%
\pgfpathclose%
\pgfusepath{fill}%
\end{pgfscope}%
\begin{pgfscope}%
\pgfpathrectangle{\pgfqpoint{1.150000in}{0.150000in}}{\pgfqpoint{5.700000in}{5.700000in}}%
\pgfusepath{clip}%
\pgfsetbuttcap%
\pgfsetroundjoin%
\definecolor{currentfill}{rgb}{0.268510,0.009605,0.335427}%
\pgfsetfillcolor{currentfill}%
\pgfsetfillopacity{0.800000}%
\pgfsetlinewidth{0.000000pt}%
\definecolor{currentstroke}{rgb}{0.000000,0.000000,0.000000}%
\pgfsetstrokecolor{currentstroke}%
\pgfsetdash{}{0pt}%
\pgfpathmoveto{\pgfqpoint{3.237673in}{1.400875in}}%
\pgfpathlineto{\pgfqpoint{3.251578in}{1.394195in}}%
\pgfpathlineto{\pgfqpoint{3.265485in}{1.387713in}}%
\pgfpathlineto{\pgfqpoint{3.279394in}{1.381429in}}%
\pgfpathlineto{\pgfqpoint{3.293305in}{1.375341in}}%
\pgfpathlineto{\pgfqpoint{3.301763in}{1.381094in}}%
\pgfpathlineto{\pgfqpoint{3.310211in}{1.387076in}}%
\pgfpathlineto{\pgfqpoint{3.318647in}{1.393281in}}%
\pgfpathlineto{\pgfqpoint{3.327074in}{1.399702in}}%
\pgfpathlineto{\pgfqpoint{3.313189in}{1.405129in}}%
\pgfpathlineto{\pgfqpoint{3.299307in}{1.410753in}}%
\pgfpathlineto{\pgfqpoint{3.285428in}{1.416573in}}%
\pgfpathlineto{\pgfqpoint{3.271551in}{1.422592in}}%
\pgfpathlineto{\pgfqpoint{3.263098in}{1.416819in}}%
\pgfpathlineto{\pgfqpoint{3.254634in}{1.411271in}}%
\pgfpathlineto{\pgfqpoint{3.246159in}{1.405954in}}%
\pgfpathlineto{\pgfqpoint{3.237673in}{1.400875in}}%
\pgfpathclose%
\pgfusepath{fill}%
\end{pgfscope}%
\begin{pgfscope}%
\pgfpathrectangle{\pgfqpoint{1.150000in}{0.150000in}}{\pgfqpoint{5.700000in}{5.700000in}}%
\pgfusepath{clip}%
\pgfsetbuttcap%
\pgfsetroundjoin%
\definecolor{currentfill}{rgb}{0.296479,0.761561,0.424223}%
\pgfsetfillcolor{currentfill}%
\pgfsetfillopacity{0.800000}%
\pgfsetlinewidth{0.000000pt}%
\definecolor{currentstroke}{rgb}{0.000000,0.000000,0.000000}%
\pgfsetstrokecolor{currentstroke}%
\pgfsetdash{}{0pt}%
\pgfpathmoveto{\pgfqpoint{5.713596in}{3.451145in}}%
\pgfpathlineto{\pgfqpoint{5.728655in}{3.467831in}}%
\pgfpathlineto{\pgfqpoint{5.743737in}{3.484703in}}%
\pgfpathlineto{\pgfqpoint{5.758843in}{3.501763in}}%
\pgfpathlineto{\pgfqpoint{5.773973in}{3.519009in}}%
\pgfpathlineto{\pgfqpoint{5.781254in}{3.520356in}}%
\pgfpathlineto{\pgfqpoint{5.788525in}{3.521600in}}%
\pgfpathlineto{\pgfqpoint{5.795785in}{3.522746in}}%
\pgfpathlineto{\pgfqpoint{5.803035in}{3.523799in}}%
\pgfpathlineto{\pgfqpoint{5.787930in}{3.507001in}}%
\pgfpathlineto{\pgfqpoint{5.772849in}{3.490388in}}%
\pgfpathlineto{\pgfqpoint{5.757792in}{3.473961in}}%
\pgfpathlineto{\pgfqpoint{5.742758in}{3.457719in}}%
\pgfpathlineto{\pgfqpoint{5.735482in}{3.456208in}}%
\pgfpathlineto{\pgfqpoint{5.728197in}{3.454612in}}%
\pgfpathlineto{\pgfqpoint{5.720901in}{3.452925in}}%
\pgfpathlineto{\pgfqpoint{5.713596in}{3.451145in}}%
\pgfpathclose%
\pgfusepath{fill}%
\end{pgfscope}%
\begin{pgfscope}%
\pgfpathrectangle{\pgfqpoint{1.150000in}{0.150000in}}{\pgfqpoint{5.700000in}{5.700000in}}%
\pgfusepath{clip}%
\pgfsetbuttcap%
\pgfsetroundjoin%
\definecolor{currentfill}{rgb}{0.273006,0.204520,0.501721}%
\pgfsetfillcolor{currentfill}%
\pgfsetfillopacity{0.800000}%
\pgfsetlinewidth{0.000000pt}%
\definecolor{currentstroke}{rgb}{0.000000,0.000000,0.000000}%
\pgfsetstrokecolor{currentstroke}%
\pgfsetdash{}{0pt}%
\pgfpathmoveto{\pgfqpoint{2.609245in}{1.867908in}}%
\pgfpathlineto{\pgfqpoint{2.623332in}{1.850996in}}%
\pgfpathlineto{\pgfqpoint{2.637413in}{1.834332in}}%
\pgfpathlineto{\pgfqpoint{2.651486in}{1.817912in}}%
\pgfpathlineto{\pgfqpoint{2.665553in}{1.801735in}}%
\pgfpathlineto{\pgfqpoint{2.674536in}{1.797894in}}%
\pgfpathlineto{\pgfqpoint{2.683499in}{1.794428in}}%
\pgfpathlineto{\pgfqpoint{2.692441in}{1.791330in}}%
\pgfpathlineto{\pgfqpoint{2.701363in}{1.788591in}}%
\pgfpathlineto{\pgfqpoint{2.687349in}{1.804015in}}%
\pgfpathlineto{\pgfqpoint{2.673329in}{1.819680in}}%
\pgfpathlineto{\pgfqpoint{2.659303in}{1.835590in}}%
\pgfpathlineto{\pgfqpoint{2.645270in}{1.851745in}}%
\pgfpathlineto{\pgfqpoint{2.636296in}{1.855224in}}%
\pgfpathlineto{\pgfqpoint{2.627300in}{1.859072in}}%
\pgfpathlineto{\pgfqpoint{2.618283in}{1.863298in}}%
\pgfpathlineto{\pgfqpoint{2.609245in}{1.867908in}}%
\pgfpathclose%
\pgfusepath{fill}%
\end{pgfscope}%
\begin{pgfscope}%
\pgfpathrectangle{\pgfqpoint{1.150000in}{0.150000in}}{\pgfqpoint{5.700000in}{5.700000in}}%
\pgfusepath{clip}%
\pgfsetbuttcap%
\pgfsetroundjoin%
\definecolor{currentfill}{rgb}{0.267004,0.004874,0.329415}%
\pgfsetfillcolor{currentfill}%
\pgfsetfillopacity{0.800000}%
\pgfsetlinewidth{0.000000pt}%
\definecolor{currentstroke}{rgb}{0.000000,0.000000,0.000000}%
\pgfsetstrokecolor{currentstroke}%
\pgfsetdash{}{0pt}%
\pgfpathmoveto{\pgfqpoint{3.382645in}{1.379945in}}%
\pgfpathlineto{\pgfqpoint{3.396546in}{1.375490in}}%
\pgfpathlineto{\pgfqpoint{3.410451in}{1.371228in}}%
\pgfpathlineto{\pgfqpoint{3.424360in}{1.367157in}}%
\pgfpathlineto{\pgfqpoint{3.438274in}{1.363278in}}%
\pgfpathlineto{\pgfqpoint{3.446644in}{1.371190in}}%
\pgfpathlineto{\pgfqpoint{3.455005in}{1.379289in}}%
\pgfpathlineto{\pgfqpoint{3.463357in}{1.387569in}}%
\pgfpathlineto{\pgfqpoint{3.471701in}{1.396025in}}%
\pgfpathlineto{\pgfqpoint{3.457808in}{1.399277in}}%
\pgfpathlineto{\pgfqpoint{3.443920in}{1.402719in}}%
\pgfpathlineto{\pgfqpoint{3.430037in}{1.406354in}}%
\pgfpathlineto{\pgfqpoint{3.416158in}{1.410181in}}%
\pgfpathlineto{\pgfqpoint{3.407793in}{1.402341in}}%
\pgfpathlineto{\pgfqpoint{3.399419in}{1.394684in}}%
\pgfpathlineto{\pgfqpoint{3.391037in}{1.387217in}}%
\pgfpathlineto{\pgfqpoint{3.382645in}{1.379945in}}%
\pgfpathclose%
\pgfusepath{fill}%
\end{pgfscope}%
\begin{pgfscope}%
\pgfpathrectangle{\pgfqpoint{1.150000in}{0.150000in}}{\pgfqpoint{5.700000in}{5.700000in}}%
\pgfusepath{clip}%
\pgfsetbuttcap%
\pgfsetroundjoin%
\definecolor{currentfill}{rgb}{0.265145,0.232956,0.516599}%
\pgfsetfillcolor{currentfill}%
\pgfsetfillopacity{0.800000}%
\pgfsetlinewidth{0.000000pt}%
\definecolor{currentstroke}{rgb}{0.000000,0.000000,0.000000}%
\pgfsetstrokecolor{currentstroke}%
\pgfsetdash{}{0pt}%
\pgfpathmoveto{\pgfqpoint{2.552818in}{1.938056in}}%
\pgfpathlineto{\pgfqpoint{2.566937in}{1.920140in}}%
\pgfpathlineto{\pgfqpoint{2.581047in}{1.902478in}}%
\pgfpathlineto{\pgfqpoint{2.595150in}{1.885068in}}%
\pgfpathlineto{\pgfqpoint{2.609245in}{1.867908in}}%
\pgfpathlineto{\pgfqpoint{2.618283in}{1.863298in}}%
\pgfpathlineto{\pgfqpoint{2.627300in}{1.859072in}}%
\pgfpathlineto{\pgfqpoint{2.636296in}{1.855224in}}%
\pgfpathlineto{\pgfqpoint{2.645270in}{1.851745in}}%
\pgfpathlineto{\pgfqpoint{2.631230in}{1.868147in}}%
\pgfpathlineto{\pgfqpoint{2.617183in}{1.884799in}}%
\pgfpathlineto{\pgfqpoint{2.603129in}{1.901701in}}%
\pgfpathlineto{\pgfqpoint{2.589067in}{1.918856in}}%
\pgfpathlineto{\pgfqpoint{2.580038in}{1.923080in}}%
\pgfpathlineto{\pgfqpoint{2.570987in}{1.927682in}}%
\pgfpathlineto{\pgfqpoint{2.561914in}{1.932672in}}%
\pgfpathlineto{\pgfqpoint{2.552818in}{1.938056in}}%
\pgfpathclose%
\pgfusepath{fill}%
\end{pgfscope}%
\begin{pgfscope}%
\pgfpathrectangle{\pgfqpoint{1.150000in}{0.150000in}}{\pgfqpoint{5.700000in}{5.700000in}}%
\pgfusepath{clip}%
\pgfsetbuttcap%
\pgfsetroundjoin%
\definecolor{currentfill}{rgb}{0.278012,0.180367,0.486697}%
\pgfsetfillcolor{currentfill}%
\pgfsetfillopacity{0.800000}%
\pgfsetlinewidth{0.000000pt}%
\definecolor{currentstroke}{rgb}{0.000000,0.000000,0.000000}%
\pgfsetstrokecolor{currentstroke}%
\pgfsetdash{}{0pt}%
\pgfpathmoveto{\pgfqpoint{2.665553in}{1.801735in}}%
\pgfpathlineto{\pgfqpoint{2.679613in}{1.785801in}}%
\pgfpathlineto{\pgfqpoint{2.693667in}{1.770106in}}%
\pgfpathlineto{\pgfqpoint{2.707715in}{1.754650in}}%
\pgfpathlineto{\pgfqpoint{2.721758in}{1.739431in}}%
\pgfpathlineto{\pgfqpoint{2.730689in}{1.736354in}}%
\pgfpathlineto{\pgfqpoint{2.739599in}{1.733643in}}%
\pgfpathlineto{\pgfqpoint{2.748490in}{1.731291in}}%
\pgfpathlineto{\pgfqpoint{2.757362in}{1.729289in}}%
\pgfpathlineto{\pgfqpoint{2.743371in}{1.743759in}}%
\pgfpathlineto{\pgfqpoint{2.729374in}{1.758465in}}%
\pgfpathlineto{\pgfqpoint{2.715371in}{1.773409in}}%
\pgfpathlineto{\pgfqpoint{2.701363in}{1.788591in}}%
\pgfpathlineto{\pgfqpoint{2.692441in}{1.791330in}}%
\pgfpathlineto{\pgfqpoint{2.683499in}{1.794428in}}%
\pgfpathlineto{\pgfqpoint{2.674536in}{1.797894in}}%
\pgfpathlineto{\pgfqpoint{2.665553in}{1.801735in}}%
\pgfpathclose%
\pgfusepath{fill}%
\end{pgfscope}%
\begin{pgfscope}%
\pgfpathrectangle{\pgfqpoint{1.150000in}{0.150000in}}{\pgfqpoint{5.700000in}{5.700000in}}%
\pgfusepath{clip}%
\pgfsetbuttcap%
\pgfsetroundjoin%
\definecolor{currentfill}{rgb}{0.129933,0.559582,0.551864}%
\pgfsetfillcolor{currentfill}%
\pgfsetfillopacity{0.800000}%
\pgfsetlinewidth{0.000000pt}%
\definecolor{currentstroke}{rgb}{0.000000,0.000000,0.000000}%
\pgfsetstrokecolor{currentstroke}%
\pgfsetdash{}{0pt}%
\pgfpathmoveto{\pgfqpoint{1.997417in}{2.911130in}}%
\pgfpathlineto{\pgfqpoint{2.012024in}{2.881283in}}%
\pgfpathlineto{\pgfqpoint{2.026609in}{2.851807in}}%
\pgfpathlineto{\pgfqpoint{2.041174in}{2.822698in}}%
\pgfpathlineto{\pgfqpoint{2.055719in}{2.793953in}}%
\pgfpathlineto{\pgfqpoint{2.065275in}{2.784220in}}%
\pgfpathlineto{\pgfqpoint{2.074801in}{2.774920in}}%
\pgfpathlineto{\pgfqpoint{2.084299in}{2.766046in}}%
\pgfpathlineto{\pgfqpoint{2.093768in}{2.757591in}}%
\pgfpathlineto{\pgfqpoint{2.079298in}{2.785584in}}%
\pgfpathlineto{\pgfqpoint{2.064808in}{2.813940in}}%
\pgfpathlineto{\pgfqpoint{2.050298in}{2.842660in}}%
\pgfpathlineto{\pgfqpoint{2.035768in}{2.871749in}}%
\pgfpathlineto{\pgfqpoint{2.026225in}{2.880943in}}%
\pgfpathlineto{\pgfqpoint{2.016653in}{2.890566in}}%
\pgfpathlineto{\pgfqpoint{2.007050in}{2.900626in}}%
\pgfpathlineto{\pgfqpoint{1.997417in}{2.911130in}}%
\pgfpathclose%
\pgfusepath{fill}%
\end{pgfscope}%
\begin{pgfscope}%
\pgfpathrectangle{\pgfqpoint{1.150000in}{0.150000in}}{\pgfqpoint{5.700000in}{5.700000in}}%
\pgfusepath{clip}%
\pgfsetbuttcap%
\pgfsetroundjoin%
\definecolor{currentfill}{rgb}{0.156270,0.489624,0.557936}%
\pgfsetfillcolor{currentfill}%
\pgfsetfillopacity{0.800000}%
\pgfsetlinewidth{0.000000pt}%
\definecolor{currentstroke}{rgb}{0.000000,0.000000,0.000000}%
\pgfsetstrokecolor{currentstroke}%
\pgfsetdash{}{0pt}%
\pgfpathmoveto{\pgfqpoint{4.786069in}{2.568807in}}%
\pgfpathlineto{\pgfqpoint{4.800508in}{2.581303in}}%
\pgfpathlineto{\pgfqpoint{4.814966in}{2.593984in}}%
\pgfpathlineto{\pgfqpoint{4.829441in}{2.606851in}}%
\pgfpathlineto{\pgfqpoint{4.843935in}{2.619904in}}%
\pgfpathlineto{\pgfqpoint{4.851812in}{2.630920in}}%
\pgfpathlineto{\pgfqpoint{4.859681in}{2.641779in}}%
\pgfpathlineto{\pgfqpoint{4.867543in}{2.652483in}}%
\pgfpathlineto{\pgfqpoint{4.875397in}{2.663031in}}%
\pgfpathlineto{\pgfqpoint{4.860904in}{2.649959in}}%
\pgfpathlineto{\pgfqpoint{4.846430in}{2.637073in}}%
\pgfpathlineto{\pgfqpoint{4.831975in}{2.624372in}}%
\pgfpathlineto{\pgfqpoint{4.817537in}{2.611857in}}%
\pgfpathlineto{\pgfqpoint{4.809681in}{2.601316in}}%
\pgfpathlineto{\pgfqpoint{4.801817in}{2.590627in}}%
\pgfpathlineto{\pgfqpoint{4.793947in}{2.579791in}}%
\pgfpathlineto{\pgfqpoint{4.786069in}{2.568807in}}%
\pgfpathclose%
\pgfusepath{fill}%
\end{pgfscope}%
\begin{pgfscope}%
\pgfpathrectangle{\pgfqpoint{1.150000in}{0.150000in}}{\pgfqpoint{5.700000in}{5.700000in}}%
\pgfusepath{clip}%
\pgfsetbuttcap%
\pgfsetroundjoin%
\definecolor{currentfill}{rgb}{0.235526,0.309527,0.542944}%
\pgfsetfillcolor{currentfill}%
\pgfsetfillopacity{0.800000}%
\pgfsetlinewidth{0.000000pt}%
\definecolor{currentstroke}{rgb}{0.000000,0.000000,0.000000}%
\pgfsetstrokecolor{currentstroke}%
\pgfsetdash{}{0pt}%
\pgfpathmoveto{\pgfqpoint{4.334281in}{2.045718in}}%
\pgfpathlineto{\pgfqpoint{4.348457in}{2.054180in}}%
\pgfpathlineto{\pgfqpoint{4.362646in}{2.062827in}}%
\pgfpathlineto{\pgfqpoint{4.376850in}{2.071657in}}%
\pgfpathlineto{\pgfqpoint{4.391069in}{2.080670in}}%
\pgfpathlineto{\pgfqpoint{4.399108in}{2.094719in}}%
\pgfpathlineto{\pgfqpoint{4.407142in}{2.108674in}}%
\pgfpathlineto{\pgfqpoint{4.415171in}{2.122533in}}%
\pgfpathlineto{\pgfqpoint{4.423196in}{2.136294in}}%
\pgfpathlineto{\pgfqpoint{4.408975in}{2.127027in}}%
\pgfpathlineto{\pgfqpoint{4.394770in}{2.117943in}}%
\pgfpathlineto{\pgfqpoint{4.380579in}{2.109043in}}%
\pgfpathlineto{\pgfqpoint{4.366403in}{2.100327in}}%
\pgfpathlineto{\pgfqpoint{4.358380in}{2.086807in}}%
\pgfpathlineto{\pgfqpoint{4.350352in}{2.073197in}}%
\pgfpathlineto{\pgfqpoint{4.342319in}{2.059500in}}%
\pgfpathlineto{\pgfqpoint{4.334281in}{2.045718in}}%
\pgfpathclose%
\pgfusepath{fill}%
\end{pgfscope}%
\begin{pgfscope}%
\pgfpathrectangle{\pgfqpoint{1.150000in}{0.150000in}}{\pgfqpoint{5.700000in}{5.700000in}}%
\pgfusepath{clip}%
\pgfsetbuttcap%
\pgfsetroundjoin%
\definecolor{currentfill}{rgb}{0.255645,0.260703,0.528312}%
\pgfsetfillcolor{currentfill}%
\pgfsetfillopacity{0.800000}%
\pgfsetlinewidth{0.000000pt}%
\definecolor{currentstroke}{rgb}{0.000000,0.000000,0.000000}%
\pgfsetstrokecolor{currentstroke}%
\pgfsetdash{}{0pt}%
\pgfpathmoveto{\pgfqpoint{2.496257in}{2.012294in}}%
\pgfpathlineto{\pgfqpoint{2.510411in}{1.993345in}}%
\pgfpathlineto{\pgfqpoint{2.524555in}{1.974656in}}%
\pgfpathlineto{\pgfqpoint{2.538691in}{1.956227in}}%
\pgfpathlineto{\pgfqpoint{2.552818in}{1.938056in}}%
\pgfpathlineto{\pgfqpoint{2.561914in}{1.932672in}}%
\pgfpathlineto{\pgfqpoint{2.570987in}{1.927682in}}%
\pgfpathlineto{\pgfqpoint{2.580038in}{1.923080in}}%
\pgfpathlineto{\pgfqpoint{2.589067in}{1.918856in}}%
\pgfpathlineto{\pgfqpoint{2.574998in}{1.936265in}}%
\pgfpathlineto{\pgfqpoint{2.560920in}{1.953930in}}%
\pgfpathlineto{\pgfqpoint{2.546834in}{1.971853in}}%
\pgfpathlineto{\pgfqpoint{2.532740in}{1.990037in}}%
\pgfpathlineto{\pgfqpoint{2.523654in}{1.995010in}}%
\pgfpathlineto{\pgfqpoint{2.514545in}{2.000373in}}%
\pgfpathlineto{\pgfqpoint{2.505413in}{2.006131in}}%
\pgfpathlineto{\pgfqpoint{2.496257in}{2.012294in}}%
\pgfpathclose%
\pgfusepath{fill}%
\end{pgfscope}%
\begin{pgfscope}%
\pgfpathrectangle{\pgfqpoint{1.150000in}{0.150000in}}{\pgfqpoint{5.700000in}{5.700000in}}%
\pgfusepath{clip}%
\pgfsetbuttcap%
\pgfsetroundjoin%
\definecolor{currentfill}{rgb}{0.281412,0.155834,0.469201}%
\pgfsetfillcolor{currentfill}%
\pgfsetfillopacity{0.800000}%
\pgfsetlinewidth{0.000000pt}%
\definecolor{currentstroke}{rgb}{0.000000,0.000000,0.000000}%
\pgfsetstrokecolor{currentstroke}%
\pgfsetdash{}{0pt}%
\pgfpathmoveto{\pgfqpoint{2.721758in}{1.739431in}}%
\pgfpathlineto{\pgfqpoint{2.735795in}{1.724447in}}%
\pgfpathlineto{\pgfqpoint{2.749826in}{1.709697in}}%
\pgfpathlineto{\pgfqpoint{2.763853in}{1.695180in}}%
\pgfpathlineto{\pgfqpoint{2.777874in}{1.680894in}}%
\pgfpathlineto{\pgfqpoint{2.786754in}{1.678578in}}%
\pgfpathlineto{\pgfqpoint{2.795615in}{1.676618in}}%
\pgfpathlineto{\pgfqpoint{2.804458in}{1.675008in}}%
\pgfpathlineto{\pgfqpoint{2.813282in}{1.673739in}}%
\pgfpathlineto{\pgfqpoint{2.799309in}{1.687279in}}%
\pgfpathlineto{\pgfqpoint{2.785331in}{1.701050in}}%
\pgfpathlineto{\pgfqpoint{2.771349in}{1.715053in}}%
\pgfpathlineto{\pgfqpoint{2.757362in}{1.729289in}}%
\pgfpathlineto{\pgfqpoint{2.748490in}{1.731291in}}%
\pgfpathlineto{\pgfqpoint{2.739599in}{1.733643in}}%
\pgfpathlineto{\pgfqpoint{2.730689in}{1.736354in}}%
\pgfpathlineto{\pgfqpoint{2.721758in}{1.739431in}}%
\pgfpathclose%
\pgfusepath{fill}%
\end{pgfscope}%
\begin{pgfscope}%
\pgfpathrectangle{\pgfqpoint{1.150000in}{0.150000in}}{\pgfqpoint{5.700000in}{5.700000in}}%
\pgfusepath{clip}%
\pgfsetbuttcap%
\pgfsetroundjoin%
\definecolor{currentfill}{rgb}{0.280255,0.165693,0.476498}%
\pgfsetfillcolor{currentfill}%
\pgfsetfillopacity{0.800000}%
\pgfsetlinewidth{0.000000pt}%
\definecolor{currentstroke}{rgb}{0.000000,0.000000,0.000000}%
\pgfsetstrokecolor{currentstroke}%
\pgfsetdash{}{0pt}%
\pgfpathmoveto{\pgfqpoint{4.003494in}{1.689981in}}%
\pgfpathlineto{\pgfqpoint{4.017520in}{1.694498in}}%
\pgfpathlineto{\pgfqpoint{4.031558in}{1.699197in}}%
\pgfpathlineto{\pgfqpoint{4.045607in}{1.704080in}}%
\pgfpathlineto{\pgfqpoint{4.059667in}{1.709145in}}%
\pgfpathlineto{\pgfqpoint{4.067795in}{1.723197in}}%
\pgfpathlineto{\pgfqpoint{4.075918in}{1.737241in}}%
\pgfpathlineto{\pgfqpoint{4.084037in}{1.751272in}}%
\pgfpathlineto{\pgfqpoint{4.092151in}{1.765287in}}%
\pgfpathlineto{\pgfqpoint{4.078093in}{1.759809in}}%
\pgfpathlineto{\pgfqpoint{4.064046in}{1.754514in}}%
\pgfpathlineto{\pgfqpoint{4.050011in}{1.749402in}}%
\pgfpathlineto{\pgfqpoint{4.035987in}{1.744474in}}%
\pgfpathlineto{\pgfqpoint{4.027871in}{1.730859in}}%
\pgfpathlineto{\pgfqpoint{4.019750in}{1.717236in}}%
\pgfpathlineto{\pgfqpoint{4.011624in}{1.703609in}}%
\pgfpathlineto{\pgfqpoint{4.003494in}{1.689981in}}%
\pgfpathclose%
\pgfusepath{fill}%
\end{pgfscope}%
\begin{pgfscope}%
\pgfpathrectangle{\pgfqpoint{1.150000in}{0.150000in}}{\pgfqpoint{5.700000in}{5.700000in}}%
\pgfusepath{clip}%
\pgfsetbuttcap%
\pgfsetroundjoin%
\definecolor{currentfill}{rgb}{0.166383,0.690856,0.496502}%
\pgfsetfillcolor{currentfill}%
\pgfsetfillopacity{0.800000}%
\pgfsetlinewidth{0.000000pt}%
\definecolor{currentstroke}{rgb}{0.000000,0.000000,0.000000}%
\pgfsetstrokecolor{currentstroke}%
\pgfsetdash{}{0pt}%
\pgfpathmoveto{\pgfqpoint{5.415435in}{3.203337in}}%
\pgfpathlineto{\pgfqpoint{5.430296in}{3.219206in}}%
\pgfpathlineto{\pgfqpoint{5.445179in}{3.235261in}}%
\pgfpathlineto{\pgfqpoint{5.460085in}{3.251504in}}%
\pgfpathlineto{\pgfqpoint{5.475013in}{3.267934in}}%
\pgfpathlineto{\pgfqpoint{5.482522in}{3.272406in}}%
\pgfpathlineto{\pgfqpoint{5.490021in}{3.276734in}}%
\pgfpathlineto{\pgfqpoint{5.497510in}{3.280922in}}%
\pgfpathlineto{\pgfqpoint{5.504988in}{3.284973in}}%
\pgfpathlineto{\pgfqpoint{5.490076in}{3.268845in}}%
\pgfpathlineto{\pgfqpoint{5.475186in}{3.252904in}}%
\pgfpathlineto{\pgfqpoint{5.460319in}{3.237149in}}%
\pgfpathlineto{\pgfqpoint{5.445473in}{3.221580in}}%
\pgfpathlineto{\pgfqpoint{5.437978in}{3.217215in}}%
\pgfpathlineto{\pgfqpoint{5.430474in}{3.212722in}}%
\pgfpathlineto{\pgfqpoint{5.422959in}{3.208097in}}%
\pgfpathlineto{\pgfqpoint{5.415435in}{3.203337in}}%
\pgfpathclose%
\pgfusepath{fill}%
\end{pgfscope}%
\begin{pgfscope}%
\pgfpathrectangle{\pgfqpoint{1.150000in}{0.150000in}}{\pgfqpoint{5.700000in}{5.700000in}}%
\pgfusepath{clip}%
\pgfsetbuttcap%
\pgfsetroundjoin%
\definecolor{currentfill}{rgb}{0.243113,0.292092,0.538516}%
\pgfsetfillcolor{currentfill}%
\pgfsetfillopacity{0.800000}%
\pgfsetlinewidth{0.000000pt}%
\definecolor{currentstroke}{rgb}{0.000000,0.000000,0.000000}%
\pgfsetstrokecolor{currentstroke}%
\pgfsetdash{}{0pt}%
\pgfpathmoveto{\pgfqpoint{2.439545in}{2.090748in}}%
\pgfpathlineto{\pgfqpoint{2.453738in}{2.070732in}}%
\pgfpathlineto{\pgfqpoint{2.467921in}{2.050986in}}%
\pgfpathlineto{\pgfqpoint{2.482094in}{2.031508in}}%
\pgfpathlineto{\pgfqpoint{2.496257in}{2.012294in}}%
\pgfpathlineto{\pgfqpoint{2.505413in}{2.006131in}}%
\pgfpathlineto{\pgfqpoint{2.514545in}{2.000373in}}%
\pgfpathlineto{\pgfqpoint{2.523654in}{1.995010in}}%
\pgfpathlineto{\pgfqpoint{2.532740in}{1.990037in}}%
\pgfpathlineto{\pgfqpoint{2.518636in}{2.008482in}}%
\pgfpathlineto{\pgfqpoint{2.504524in}{2.027191in}}%
\pgfpathlineto{\pgfqpoint{2.490402in}{2.046167in}}%
\pgfpathlineto{\pgfqpoint{2.476271in}{2.065410in}}%
\pgfpathlineto{\pgfqpoint{2.467126in}{2.071139in}}%
\pgfpathlineto{\pgfqpoint{2.457957in}{2.077267in}}%
\pgfpathlineto{\pgfqpoint{2.448763in}{2.083800in}}%
\pgfpathlineto{\pgfqpoint{2.439545in}{2.090748in}}%
\pgfpathclose%
\pgfusepath{fill}%
\end{pgfscope}%
\begin{pgfscope}%
\pgfpathrectangle{\pgfqpoint{1.150000in}{0.150000in}}{\pgfqpoint{5.700000in}{5.700000in}}%
\pgfusepath{clip}%
\pgfsetbuttcap%
\pgfsetroundjoin%
\definecolor{currentfill}{rgb}{0.283072,0.130895,0.449241}%
\pgfsetfillcolor{currentfill}%
\pgfsetfillopacity{0.800000}%
\pgfsetlinewidth{0.000000pt}%
\definecolor{currentstroke}{rgb}{0.000000,0.000000,0.000000}%
\pgfsetstrokecolor{currentstroke}%
\pgfsetdash{}{0pt}%
\pgfpathmoveto{\pgfqpoint{2.777874in}{1.680894in}}%
\pgfpathlineto{\pgfqpoint{2.791891in}{1.666838in}}%
\pgfpathlineto{\pgfqpoint{2.805904in}{1.653009in}}%
\pgfpathlineto{\pgfqpoint{2.819912in}{1.639408in}}%
\pgfpathlineto{\pgfqpoint{2.833916in}{1.626033in}}%
\pgfpathlineto{\pgfqpoint{2.842748in}{1.624473in}}%
\pgfpathlineto{\pgfqpoint{2.851562in}{1.623262in}}%
\pgfpathlineto{\pgfqpoint{2.860357in}{1.622390in}}%
\pgfpathlineto{\pgfqpoint{2.869135in}{1.621850in}}%
\pgfpathlineto{\pgfqpoint{2.855177in}{1.634484in}}%
\pgfpathlineto{\pgfqpoint{2.841216in}{1.647342in}}%
\pgfpathlineto{\pgfqpoint{2.827251in}{1.660427in}}%
\pgfpathlineto{\pgfqpoint{2.813282in}{1.673739in}}%
\pgfpathlineto{\pgfqpoint{2.804458in}{1.675008in}}%
\pgfpathlineto{\pgfqpoint{2.795615in}{1.676618in}}%
\pgfpathlineto{\pgfqpoint{2.786754in}{1.678578in}}%
\pgfpathlineto{\pgfqpoint{2.777874in}{1.680894in}}%
\pgfpathclose%
\pgfusepath{fill}%
\end{pgfscope}%
\begin{pgfscope}%
\pgfpathrectangle{\pgfqpoint{1.150000in}{0.150000in}}{\pgfqpoint{5.700000in}{5.700000in}}%
\pgfusepath{clip}%
\pgfsetbuttcap%
\pgfsetroundjoin%
\definecolor{currentfill}{rgb}{0.273809,0.031497,0.358853}%
\pgfsetfillcolor{currentfill}%
\pgfsetfillopacity{0.800000}%
\pgfsetlinewidth{0.000000pt}%
\definecolor{currentstroke}{rgb}{0.000000,0.000000,0.000000}%
\pgfsetstrokecolor{currentstroke}%
\pgfsetdash{}{0pt}%
\pgfpathmoveto{\pgfqpoint{3.092158in}{1.449305in}}%
\pgfpathlineto{\pgfqpoint{3.106088in}{1.440315in}}%
\pgfpathlineto{\pgfqpoint{3.120018in}{1.431530in}}%
\pgfpathlineto{\pgfqpoint{3.133948in}{1.422949in}}%
\pgfpathlineto{\pgfqpoint{3.147878in}{1.414571in}}%
\pgfpathlineto{\pgfqpoint{3.156446in}{1.417922in}}%
\pgfpathlineto{\pgfqpoint{3.165000in}{1.421547in}}%
\pgfpathlineto{\pgfqpoint{3.173541in}{1.425438in}}%
\pgfpathlineto{\pgfqpoint{3.182070in}{1.429590in}}%
\pgfpathlineto{\pgfqpoint{3.168172in}{1.437272in}}%
\pgfpathlineto{\pgfqpoint{3.154275in}{1.445157in}}%
\pgfpathlineto{\pgfqpoint{3.140379in}{1.453245in}}%
\pgfpathlineto{\pgfqpoint{3.126484in}{1.461539in}}%
\pgfpathlineto{\pgfqpoint{3.117923in}{1.458071in}}%
\pgfpathlineto{\pgfqpoint{3.109348in}{1.454871in}}%
\pgfpathlineto{\pgfqpoint{3.100760in}{1.451947in}}%
\pgfpathlineto{\pgfqpoint{3.092158in}{1.449305in}}%
\pgfpathclose%
\pgfusepath{fill}%
\end{pgfscope}%
\begin{pgfscope}%
\pgfpathrectangle{\pgfqpoint{1.150000in}{0.150000in}}{\pgfqpoint{5.700000in}{5.700000in}}%
\pgfusepath{clip}%
\pgfsetbuttcap%
\pgfsetroundjoin%
\definecolor{currentfill}{rgb}{0.165117,0.467423,0.558141}%
\pgfsetfillcolor{currentfill}%
\pgfsetfillopacity{0.800000}%
\pgfsetlinewidth{0.000000pt}%
\definecolor{currentstroke}{rgb}{0.000000,0.000000,0.000000}%
\pgfsetstrokecolor{currentstroke}%
\pgfsetdash{}{0pt}%
\pgfpathmoveto{\pgfqpoint{2.133470in}{2.613797in}}%
\pgfpathlineto{\pgfqpoint{2.147922in}{2.587388in}}%
\pgfpathlineto{\pgfqpoint{2.162357in}{2.561310in}}%
\pgfpathlineto{\pgfqpoint{2.176776in}{2.535560in}}%
\pgfpathlineto{\pgfqpoint{2.191177in}{2.510134in}}%
\pgfpathlineto{\pgfqpoint{2.200628in}{2.500975in}}%
\pgfpathlineto{\pgfqpoint{2.210049in}{2.492248in}}%
\pgfpathlineto{\pgfqpoint{2.219444in}{2.483947in}}%
\pgfpathlineto{\pgfqpoint{2.228811in}{2.476063in}}%
\pgfpathlineto{\pgfqpoint{2.214480in}{2.500721in}}%
\pgfpathlineto{\pgfqpoint{2.200133in}{2.525702in}}%
\pgfpathlineto{\pgfqpoint{2.185771in}{2.551008in}}%
\pgfpathlineto{\pgfqpoint{2.171391in}{2.576643in}}%
\pgfpathlineto{\pgfqpoint{2.161954in}{2.585281in}}%
\pgfpathlineto{\pgfqpoint{2.152488in}{2.594348in}}%
\pgfpathlineto{\pgfqpoint{2.142994in}{2.603851in}}%
\pgfpathlineto{\pgfqpoint{2.133470in}{2.613797in}}%
\pgfpathclose%
\pgfusepath{fill}%
\end{pgfscope}%
\begin{pgfscope}%
\pgfpathrectangle{\pgfqpoint{1.150000in}{0.150000in}}{\pgfqpoint{5.700000in}{5.700000in}}%
\pgfusepath{clip}%
\pgfsetbuttcap%
\pgfsetroundjoin%
\definecolor{currentfill}{rgb}{0.257322,0.256130,0.526563}%
\pgfsetfillcolor{currentfill}%
\pgfsetfillopacity{0.800000}%
\pgfsetlinewidth{0.000000pt}%
\definecolor{currentstroke}{rgb}{0.000000,0.000000,0.000000}%
\pgfsetstrokecolor{currentstroke}%
\pgfsetdash{}{0pt}%
\pgfpathmoveto{\pgfqpoint{4.213256in}{1.903078in}}%
\pgfpathlineto{\pgfqpoint{4.227375in}{1.910203in}}%
\pgfpathlineto{\pgfqpoint{4.241507in}{1.917511in}}%
\pgfpathlineto{\pgfqpoint{4.255653in}{1.925002in}}%
\pgfpathlineto{\pgfqpoint{4.269812in}{1.932676in}}%
\pgfpathlineto{\pgfqpoint{4.277887in}{1.947053in}}%
\pgfpathlineto{\pgfqpoint{4.285957in}{1.961364in}}%
\pgfpathlineto{\pgfqpoint{4.294022in}{1.975608in}}%
\pgfpathlineto{\pgfqpoint{4.302084in}{1.989782in}}%
\pgfpathlineto{\pgfqpoint{4.287923in}{1.981789in}}%
\pgfpathlineto{\pgfqpoint{4.273777in}{1.973979in}}%
\pgfpathlineto{\pgfqpoint{4.259644in}{1.966353in}}%
\pgfpathlineto{\pgfqpoint{4.245525in}{1.958911in}}%
\pgfpathlineto{\pgfqpoint{4.237464in}{1.945044in}}%
\pgfpathlineto{\pgfqpoint{4.229399in}{1.931114in}}%
\pgfpathlineto{\pgfqpoint{4.221330in}{1.917124in}}%
\pgfpathlineto{\pgfqpoint{4.213256in}{1.903078in}}%
\pgfpathclose%
\pgfusepath{fill}%
\end{pgfscope}%
\begin{pgfscope}%
\pgfpathrectangle{\pgfqpoint{1.150000in}{0.150000in}}{\pgfqpoint{5.700000in}{5.700000in}}%
\pgfusepath{clip}%
\pgfsetbuttcap%
\pgfsetroundjoin%
\definecolor{currentfill}{rgb}{0.277941,0.056324,0.381191}%
\pgfsetfillcolor{currentfill}%
\pgfsetfillopacity{0.800000}%
\pgfsetlinewidth{0.000000pt}%
\definecolor{currentstroke}{rgb}{0.000000,0.000000,0.000000}%
\pgfsetstrokecolor{currentstroke}%
\pgfsetdash{}{0pt}%
\pgfpathmoveto{\pgfqpoint{3.704926in}{1.458921in}}%
\pgfpathlineto{\pgfqpoint{3.718868in}{1.459273in}}%
\pgfpathlineto{\pgfqpoint{3.732818in}{1.459811in}}%
\pgfpathlineto{\pgfqpoint{3.746775in}{1.460533in}}%
\pgfpathlineto{\pgfqpoint{3.760741in}{1.461440in}}%
\pgfpathlineto{\pgfqpoint{3.768964in}{1.473394in}}%
\pgfpathlineto{\pgfqpoint{3.777181in}{1.485435in}}%
\pgfpathlineto{\pgfqpoint{3.785392in}{1.497558in}}%
\pgfpathlineto{\pgfqpoint{3.793598in}{1.509759in}}%
\pgfpathlineto{\pgfqpoint{3.779642in}{1.508317in}}%
\pgfpathlineto{\pgfqpoint{3.765694in}{1.507059in}}%
\pgfpathlineto{\pgfqpoint{3.751754in}{1.505987in}}%
\pgfpathlineto{\pgfqpoint{3.737823in}{1.505100in}}%
\pgfpathlineto{\pgfqpoint{3.729607in}{1.493423in}}%
\pgfpathlineto{\pgfqpoint{3.721386in}{1.481830in}}%
\pgfpathlineto{\pgfqpoint{3.713159in}{1.470328in}}%
\pgfpathlineto{\pgfqpoint{3.704926in}{1.458921in}}%
\pgfpathclose%
\pgfusepath{fill}%
\end{pgfscope}%
\begin{pgfscope}%
\pgfpathrectangle{\pgfqpoint{1.150000in}{0.150000in}}{\pgfqpoint{5.700000in}{5.700000in}}%
\pgfusepath{clip}%
\pgfsetbuttcap%
\pgfsetroundjoin%
\definecolor{currentfill}{rgb}{0.128729,0.563265,0.551229}%
\pgfsetfillcolor{currentfill}%
\pgfsetfillopacity{0.800000}%
\pgfsetlinewidth{0.000000pt}%
\definecolor{currentstroke}{rgb}{0.000000,0.000000,0.000000}%
\pgfsetstrokecolor{currentstroke}%
\pgfsetdash{}{0pt}%
\pgfpathmoveto{\pgfqpoint{4.996093in}{2.795759in}}%
\pgfpathlineto{\pgfqpoint{5.010675in}{2.809694in}}%
\pgfpathlineto{\pgfqpoint{5.025276in}{2.823815in}}%
\pgfpathlineto{\pgfqpoint{5.039897in}{2.838123in}}%
\pgfpathlineto{\pgfqpoint{5.054538in}{2.852617in}}%
\pgfpathlineto{\pgfqpoint{5.062315in}{2.861648in}}%
\pgfpathlineto{\pgfqpoint{5.070083in}{2.870514in}}%
\pgfpathlineto{\pgfqpoint{5.077842in}{2.879217in}}%
\pgfpathlineto{\pgfqpoint{5.085593in}{2.887757in}}%
\pgfpathlineto{\pgfqpoint{5.070957in}{2.873349in}}%
\pgfpathlineto{\pgfqpoint{5.056340in}{2.859127in}}%
\pgfpathlineto{\pgfqpoint{5.041744in}{2.845092in}}%
\pgfpathlineto{\pgfqpoint{5.027167in}{2.831242in}}%
\pgfpathlineto{\pgfqpoint{5.019411in}{2.822603in}}%
\pgfpathlineto{\pgfqpoint{5.011647in}{2.813811in}}%
\pgfpathlineto{\pgfqpoint{5.003874in}{2.804863in}}%
\pgfpathlineto{\pgfqpoint{4.996093in}{2.795759in}}%
\pgfpathclose%
\pgfusepath{fill}%
\end{pgfscope}%
\begin{pgfscope}%
\pgfpathrectangle{\pgfqpoint{1.150000in}{0.150000in}}{\pgfqpoint{5.700000in}{5.700000in}}%
\pgfusepath{clip}%
\pgfsetbuttcap%
\pgfsetroundjoin%
\definecolor{currentfill}{rgb}{0.352360,0.783011,0.392636}%
\pgfsetfillcolor{currentfill}%
\pgfsetfillopacity{0.800000}%
\pgfsetlinewidth{0.000000pt}%
\definecolor{currentstroke}{rgb}{0.000000,0.000000,0.000000}%
\pgfsetstrokecolor{currentstroke}%
\pgfsetdash{}{0pt}%
\pgfpathmoveto{\pgfqpoint{5.803035in}{3.523799in}}%
\pgfpathlineto{\pgfqpoint{5.818163in}{3.540784in}}%
\pgfpathlineto{\pgfqpoint{5.833316in}{3.557956in}}%
\pgfpathlineto{\pgfqpoint{5.848493in}{3.575315in}}%
\pgfpathlineto{\pgfqpoint{5.863694in}{3.592861in}}%
\pgfpathlineto{\pgfqpoint{5.870907in}{3.593355in}}%
\pgfpathlineto{\pgfqpoint{5.878109in}{3.593757in}}%
\pgfpathlineto{\pgfqpoint{5.885301in}{3.594074in}}%
\pgfpathlineto{\pgfqpoint{5.892483in}{3.594311in}}%
\pgfpathlineto{\pgfqpoint{5.877310in}{3.577250in}}%
\pgfpathlineto{\pgfqpoint{5.862161in}{3.560375in}}%
\pgfpathlineto{\pgfqpoint{5.847037in}{3.543686in}}%
\pgfpathlineto{\pgfqpoint{5.831936in}{3.527183in}}%
\pgfpathlineto{\pgfqpoint{5.824725in}{3.526451in}}%
\pgfpathlineto{\pgfqpoint{5.817505in}{3.525646in}}%
\pgfpathlineto{\pgfqpoint{5.810275in}{3.524764in}}%
\pgfpathlineto{\pgfqpoint{5.803035in}{3.523799in}}%
\pgfpathclose%
\pgfusepath{fill}%
\end{pgfscope}%
\begin{pgfscope}%
\pgfpathrectangle{\pgfqpoint{1.150000in}{0.150000in}}{\pgfqpoint{5.700000in}{5.700000in}}%
\pgfusepath{clip}%
\pgfsetbuttcap%
\pgfsetroundjoin%
\definecolor{currentfill}{rgb}{0.273809,0.031497,0.358853}%
\pgfsetfillcolor{currentfill}%
\pgfsetfillopacity{0.800000}%
\pgfsetlinewidth{0.000000pt}%
\definecolor{currentstroke}{rgb}{0.000000,0.000000,0.000000}%
\pgfsetstrokecolor{currentstroke}%
\pgfsetdash{}{0pt}%
\pgfpathmoveto{\pgfqpoint{3.616185in}{1.417061in}}%
\pgfpathlineto{\pgfqpoint{3.630111in}{1.416104in}}%
\pgfpathlineto{\pgfqpoint{3.644044in}{1.415333in}}%
\pgfpathlineto{\pgfqpoint{3.657984in}{1.414749in}}%
\pgfpathlineto{\pgfqpoint{3.671931in}{1.414350in}}%
\pgfpathlineto{\pgfqpoint{3.680189in}{1.425324in}}%
\pgfpathlineto{\pgfqpoint{3.688441in}{1.436413in}}%
\pgfpathlineto{\pgfqpoint{3.696686in}{1.447614in}}%
\pgfpathlineto{\pgfqpoint{3.704926in}{1.458921in}}%
\pgfpathlineto{\pgfqpoint{3.690991in}{1.458754in}}%
\pgfpathlineto{\pgfqpoint{3.677064in}{1.458772in}}%
\pgfpathlineto{\pgfqpoint{3.663144in}{1.458977in}}%
\pgfpathlineto{\pgfqpoint{3.649231in}{1.459369in}}%
\pgfpathlineto{\pgfqpoint{3.640979in}{1.448616in}}%
\pgfpathlineto{\pgfqpoint{3.632721in}{1.437977in}}%
\pgfpathlineto{\pgfqpoint{3.624456in}{1.427456in}}%
\pgfpathlineto{\pgfqpoint{3.616185in}{1.417061in}}%
\pgfpathclose%
\pgfusepath{fill}%
\end{pgfscope}%
\begin{pgfscope}%
\pgfpathrectangle{\pgfqpoint{1.150000in}{0.150000in}}{\pgfqpoint{5.700000in}{5.700000in}}%
\pgfusepath{clip}%
\pgfsetbuttcap%
\pgfsetroundjoin%
\definecolor{currentfill}{rgb}{0.174274,0.445044,0.557792}%
\pgfsetfillcolor{currentfill}%
\pgfsetfillopacity{0.800000}%
\pgfsetlinewidth{0.000000pt}%
\definecolor{currentstroke}{rgb}{0.000000,0.000000,0.000000}%
\pgfsetstrokecolor{currentstroke}%
\pgfsetdash{}{0pt}%
\pgfpathmoveto{\pgfqpoint{4.665221in}{2.428059in}}%
\pgfpathlineto{\pgfqpoint{4.679590in}{2.439673in}}%
\pgfpathlineto{\pgfqpoint{4.693977in}{2.451472in}}%
\pgfpathlineto{\pgfqpoint{4.708381in}{2.463456in}}%
\pgfpathlineto{\pgfqpoint{4.722803in}{2.475625in}}%
\pgfpathlineto{\pgfqpoint{4.730734in}{2.487787in}}%
\pgfpathlineto{\pgfqpoint{4.738659in}{2.499803in}}%
\pgfpathlineto{\pgfqpoint{4.746578in}{2.511672in}}%
\pgfpathlineto{\pgfqpoint{4.754490in}{2.523394in}}%
\pgfpathlineto{\pgfqpoint{4.740068in}{2.511137in}}%
\pgfpathlineto{\pgfqpoint{4.725664in}{2.499066in}}%
\pgfpathlineto{\pgfqpoint{4.711277in}{2.487179in}}%
\pgfpathlineto{\pgfqpoint{4.696908in}{2.475478in}}%
\pgfpathlineto{\pgfqpoint{4.688996in}{2.463830in}}%
\pgfpathlineto{\pgfqpoint{4.681077in}{2.452044in}}%
\pgfpathlineto{\pgfqpoint{4.673152in}{2.440120in}}%
\pgfpathlineto{\pgfqpoint{4.665221in}{2.428059in}}%
\pgfpathclose%
\pgfusepath{fill}%
\end{pgfscope}%
\begin{pgfscope}%
\pgfpathrectangle{\pgfqpoint{1.150000in}{0.150000in}}{\pgfqpoint{5.700000in}{5.700000in}}%
\pgfusepath{clip}%
\pgfsetbuttcap%
\pgfsetroundjoin%
\definecolor{currentfill}{rgb}{0.122312,0.633153,0.530398}%
\pgfsetfillcolor{currentfill}%
\pgfsetfillopacity{0.800000}%
\pgfsetlinewidth{0.000000pt}%
\definecolor{currentstroke}{rgb}{0.000000,0.000000,0.000000}%
\pgfsetstrokecolor{currentstroke}%
\pgfsetdash{}{0pt}%
\pgfpathmoveto{\pgfqpoint{5.205974in}{3.008818in}}%
\pgfpathlineto{\pgfqpoint{5.220698in}{3.023880in}}%
\pgfpathlineto{\pgfqpoint{5.235443in}{3.039128in}}%
\pgfpathlineto{\pgfqpoint{5.250210in}{3.054564in}}%
\pgfpathlineto{\pgfqpoint{5.264997in}{3.070186in}}%
\pgfpathlineto{\pgfqpoint{5.272651in}{3.076985in}}%
\pgfpathlineto{\pgfqpoint{5.280295in}{3.083623in}}%
\pgfpathlineto{\pgfqpoint{5.287930in}{3.090103in}}%
\pgfpathlineto{\pgfqpoint{5.295555in}{3.096426in}}%
\pgfpathlineto{\pgfqpoint{5.280777in}{3.080997in}}%
\pgfpathlineto{\pgfqpoint{5.266021in}{3.065755in}}%
\pgfpathlineto{\pgfqpoint{5.251285in}{3.050699in}}%
\pgfpathlineto{\pgfqpoint{5.236571in}{3.035830in}}%
\pgfpathlineto{\pgfqpoint{5.228936in}{3.029300in}}%
\pgfpathlineto{\pgfqpoint{5.221291in}{3.022623in}}%
\pgfpathlineto{\pgfqpoint{5.213637in}{3.015797in}}%
\pgfpathlineto{\pgfqpoint{5.205974in}{3.008818in}}%
\pgfpathclose%
\pgfusepath{fill}%
\end{pgfscope}%
\begin{pgfscope}%
\pgfpathrectangle{\pgfqpoint{1.150000in}{0.150000in}}{\pgfqpoint{5.700000in}{5.700000in}}%
\pgfusepath{clip}%
\pgfsetbuttcap%
\pgfsetroundjoin%
\definecolor{currentfill}{rgb}{0.395174,0.797475,0.367757}%
\pgfsetfillcolor{currentfill}%
\pgfsetfillopacity{0.800000}%
\pgfsetlinewidth{0.000000pt}%
\definecolor{currentstroke}{rgb}{0.000000,0.000000,0.000000}%
\pgfsetstrokecolor{currentstroke}%
\pgfsetdash{}{0pt}%
\pgfpathmoveto{\pgfqpoint{5.892483in}{3.594311in}}%
\pgfpathlineto{\pgfqpoint{5.907680in}{3.611558in}}%
\pgfpathlineto{\pgfqpoint{5.922902in}{3.628992in}}%
\pgfpathlineto{\pgfqpoint{5.938149in}{3.646613in}}%
\pgfpathlineto{\pgfqpoint{5.945298in}{3.646393in}}%
\pgfpathlineto{\pgfqpoint{5.952438in}{3.646097in}}%
\pgfpathlineto{\pgfqpoint{5.959567in}{3.645732in}}%
\pgfpathlineto{\pgfqpoint{5.966686in}{3.645301in}}%
\pgfpathlineto{\pgfqpoint{5.951471in}{3.628202in}}%
\pgfpathlineto{\pgfqpoint{5.936280in}{3.611288in}}%
\pgfpathlineto{\pgfqpoint{5.921113in}{3.594560in}}%
\pgfpathlineto{\pgfqpoint{5.913970in}{3.594591in}}%
\pgfpathlineto{\pgfqpoint{5.906817in}{3.594563in}}%
\pgfpathlineto{\pgfqpoint{5.899655in}{3.594472in}}%
\pgfpathlineto{\pgfqpoint{5.892483in}{3.594311in}}%
\pgfpathclose%
\pgfusepath{fill}%
\end{pgfscope}%
\begin{pgfscope}%
\pgfpathrectangle{\pgfqpoint{1.150000in}{0.150000in}}{\pgfqpoint{5.700000in}{5.700000in}}%
\pgfusepath{clip}%
\pgfsetbuttcap%
\pgfsetroundjoin%
\definecolor{currentfill}{rgb}{0.281446,0.084320,0.407414}%
\pgfsetfillcolor{currentfill}%
\pgfsetfillopacity{0.800000}%
\pgfsetlinewidth{0.000000pt}%
\definecolor{currentstroke}{rgb}{0.000000,0.000000,0.000000}%
\pgfsetstrokecolor{currentstroke}%
\pgfsetdash{}{0pt}%
\pgfpathmoveto{\pgfqpoint{3.793598in}{1.509759in}}%
\pgfpathlineto{\pgfqpoint{3.807563in}{1.511385in}}%
\pgfpathlineto{\pgfqpoint{3.821536in}{1.513196in}}%
\pgfpathlineto{\pgfqpoint{3.835519in}{1.515190in}}%
\pgfpathlineto{\pgfqpoint{3.849510in}{1.517368in}}%
\pgfpathlineto{\pgfqpoint{3.857703in}{1.530157in}}%
\pgfpathlineto{\pgfqpoint{3.865891in}{1.543005in}}%
\pgfpathlineto{\pgfqpoint{3.874073in}{1.555909in}}%
\pgfpathlineto{\pgfqpoint{3.882251in}{1.568862in}}%
\pgfpathlineto{\pgfqpoint{3.868266in}{1.566179in}}%
\pgfpathlineto{\pgfqpoint{3.854291in}{1.563680in}}%
\pgfpathlineto{\pgfqpoint{3.840324in}{1.561366in}}%
\pgfpathlineto{\pgfqpoint{3.826367in}{1.559235in}}%
\pgfpathlineto{\pgfqpoint{3.818183in}{1.546775in}}%
\pgfpathlineto{\pgfqpoint{3.809993in}{1.534372in}}%
\pgfpathlineto{\pgfqpoint{3.801798in}{1.522032in}}%
\pgfpathlineto{\pgfqpoint{3.793598in}{1.509759in}}%
\pgfpathclose%
\pgfusepath{fill}%
\end{pgfscope}%
\begin{pgfscope}%
\pgfpathrectangle{\pgfqpoint{1.150000in}{0.150000in}}{\pgfqpoint{5.700000in}{5.700000in}}%
\pgfusepath{clip}%
\pgfsetbuttcap%
\pgfsetroundjoin%
\definecolor{currentfill}{rgb}{0.283091,0.110553,0.431554}%
\pgfsetfillcolor{currentfill}%
\pgfsetfillopacity{0.800000}%
\pgfsetlinewidth{0.000000pt}%
\definecolor{currentstroke}{rgb}{0.000000,0.000000,0.000000}%
\pgfsetstrokecolor{currentstroke}%
\pgfsetdash{}{0pt}%
\pgfpathmoveto{\pgfqpoint{2.833916in}{1.626033in}}%
\pgfpathlineto{\pgfqpoint{2.847917in}{1.612881in}}%
\pgfpathlineto{\pgfqpoint{2.861913in}{1.599953in}}%
\pgfpathlineto{\pgfqpoint{2.875907in}{1.587247in}}%
\pgfpathlineto{\pgfqpoint{2.889897in}{1.574761in}}%
\pgfpathlineto{\pgfqpoint{2.898683in}{1.573955in}}%
\pgfpathlineto{\pgfqpoint{2.907451in}{1.573488in}}%
\pgfpathlineto{\pgfqpoint{2.916202in}{1.573351in}}%
\pgfpathlineto{\pgfqpoint{2.924937in}{1.573538in}}%
\pgfpathlineto{\pgfqpoint{2.910991in}{1.585285in}}%
\pgfpathlineto{\pgfqpoint{2.897042in}{1.597252in}}%
\pgfpathlineto{\pgfqpoint{2.883090in}{1.609440in}}%
\pgfpathlineto{\pgfqpoint{2.869135in}{1.621850in}}%
\pgfpathlineto{\pgfqpoint{2.860357in}{1.622390in}}%
\pgfpathlineto{\pgfqpoint{2.851562in}{1.623262in}}%
\pgfpathlineto{\pgfqpoint{2.842748in}{1.624473in}}%
\pgfpathlineto{\pgfqpoint{2.833916in}{1.626033in}}%
\pgfpathclose%
\pgfusepath{fill}%
\end{pgfscope}%
\begin{pgfscope}%
\pgfpathrectangle{\pgfqpoint{1.150000in}{0.150000in}}{\pgfqpoint{5.700000in}{5.700000in}}%
\pgfusepath{clip}%
\pgfsetbuttcap%
\pgfsetroundjoin%
\definecolor{currentfill}{rgb}{0.229739,0.322361,0.545706}%
\pgfsetfillcolor{currentfill}%
\pgfsetfillopacity{0.800000}%
\pgfsetlinewidth{0.000000pt}%
\definecolor{currentstroke}{rgb}{0.000000,0.000000,0.000000}%
\pgfsetstrokecolor{currentstroke}%
\pgfsetdash{}{0pt}%
\pgfpathmoveto{\pgfqpoint{2.382664in}{2.173550in}}%
\pgfpathlineto{\pgfqpoint{2.396901in}{2.152434in}}%
\pgfpathlineto{\pgfqpoint{2.411127in}{2.131597in}}%
\pgfpathlineto{\pgfqpoint{2.425341in}{2.111035in}}%
\pgfpathlineto{\pgfqpoint{2.439545in}{2.090748in}}%
\pgfpathlineto{\pgfqpoint{2.448763in}{2.083800in}}%
\pgfpathlineto{\pgfqpoint{2.457957in}{2.077267in}}%
\pgfpathlineto{\pgfqpoint{2.467126in}{2.071139in}}%
\pgfpathlineto{\pgfqpoint{2.476271in}{2.065410in}}%
\pgfpathlineto{\pgfqpoint{2.462130in}{2.084924in}}%
\pgfpathlineto{\pgfqpoint{2.447978in}{2.104710in}}%
\pgfpathlineto{\pgfqpoint{2.433816in}{2.124771in}}%
\pgfpathlineto{\pgfqpoint{2.419643in}{2.145109in}}%
\pgfpathlineto{\pgfqpoint{2.410437in}{2.151599in}}%
\pgfpathlineto{\pgfqpoint{2.401205in}{2.158497in}}%
\pgfpathlineto{\pgfqpoint{2.391948in}{2.165811in}}%
\pgfpathlineto{\pgfqpoint{2.382664in}{2.173550in}}%
\pgfpathclose%
\pgfusepath{fill}%
\end{pgfscope}%
\begin{pgfscope}%
\pgfpathrectangle{\pgfqpoint{1.150000in}{0.150000in}}{\pgfqpoint{5.700000in}{5.700000in}}%
\pgfusepath{clip}%
\pgfsetbuttcap%
\pgfsetroundjoin%
\definecolor{currentfill}{rgb}{0.269944,0.014625,0.341379}%
\pgfsetfillcolor{currentfill}%
\pgfsetfillopacity{0.800000}%
\pgfsetlinewidth{0.000000pt}%
\definecolor{currentstroke}{rgb}{0.000000,0.000000,0.000000}%
\pgfsetstrokecolor{currentstroke}%
\pgfsetdash{}{0pt}%
\pgfpathmoveto{\pgfqpoint{3.527321in}{1.384921in}}%
\pgfpathlineto{\pgfqpoint{3.541239in}{1.382618in}}%
\pgfpathlineto{\pgfqpoint{3.555163in}{1.380503in}}%
\pgfpathlineto{\pgfqpoint{3.569092in}{1.378575in}}%
\pgfpathlineto{\pgfqpoint{3.583027in}{1.376835in}}%
\pgfpathlineto{\pgfqpoint{3.591328in}{1.386676in}}%
\pgfpathlineto{\pgfqpoint{3.599620in}{1.396665in}}%
\pgfpathlineto{\pgfqpoint{3.607906in}{1.406795in}}%
\pgfpathlineto{\pgfqpoint{3.616185in}{1.417061in}}%
\pgfpathlineto{\pgfqpoint{3.602265in}{1.418204in}}%
\pgfpathlineto{\pgfqpoint{3.588351in}{1.419535in}}%
\pgfpathlineto{\pgfqpoint{3.574444in}{1.421054in}}%
\pgfpathlineto{\pgfqpoint{3.560543in}{1.422761in}}%
\pgfpathlineto{\pgfqpoint{3.552248in}{1.413080in}}%
\pgfpathlineto{\pgfqpoint{3.543947in}{1.403542in}}%
\pgfpathlineto{\pgfqpoint{3.535638in}{1.394154in}}%
\pgfpathlineto{\pgfqpoint{3.527321in}{1.384921in}}%
\pgfpathclose%
\pgfusepath{fill}%
\end{pgfscope}%
\begin{pgfscope}%
\pgfpathrectangle{\pgfqpoint{1.150000in}{0.150000in}}{\pgfqpoint{5.700000in}{5.700000in}}%
\pgfusepath{clip}%
\pgfsetbuttcap%
\pgfsetroundjoin%
\definecolor{currentfill}{rgb}{0.194100,0.399323,0.555565}%
\pgfsetfillcolor{currentfill}%
\pgfsetfillopacity{0.800000}%
\pgfsetlinewidth{0.000000pt}%
\definecolor{currentstroke}{rgb}{0.000000,0.000000,0.000000}%
\pgfsetstrokecolor{currentstroke}%
\pgfsetdash{}{0pt}%
\pgfpathmoveto{\pgfqpoint{4.544249in}{2.283191in}}%
\pgfpathlineto{\pgfqpoint{4.558550in}{2.293789in}}%
\pgfpathlineto{\pgfqpoint{4.572868in}{2.304572in}}%
\pgfpathlineto{\pgfqpoint{4.587201in}{2.315539in}}%
\pgfpathlineto{\pgfqpoint{4.601552in}{2.326691in}}%
\pgfpathlineto{\pgfqpoint{4.609531in}{2.339828in}}%
\pgfpathlineto{\pgfqpoint{4.617505in}{2.352833in}}%
\pgfpathlineto{\pgfqpoint{4.625473in}{2.365707in}}%
\pgfpathlineto{\pgfqpoint{4.633434in}{2.378447in}}%
\pgfpathlineto{\pgfqpoint{4.619083in}{2.367139in}}%
\pgfpathlineto{\pgfqpoint{4.604748in}{2.356017in}}%
\pgfpathlineto{\pgfqpoint{4.590429in}{2.345079in}}%
\pgfpathlineto{\pgfqpoint{4.576127in}{2.334326in}}%
\pgfpathlineto{\pgfqpoint{4.568166in}{2.321728in}}%
\pgfpathlineto{\pgfqpoint{4.560200in}{2.309005in}}%
\pgfpathlineto{\pgfqpoint{4.552227in}{2.296159in}}%
\pgfpathlineto{\pgfqpoint{4.544249in}{2.283191in}}%
\pgfpathclose%
\pgfusepath{fill}%
\end{pgfscope}%
\begin{pgfscope}%
\pgfpathrectangle{\pgfqpoint{1.150000in}{0.150000in}}{\pgfqpoint{5.700000in}{5.700000in}}%
\pgfusepath{clip}%
\pgfsetbuttcap%
\pgfsetroundjoin%
\definecolor{currentfill}{rgb}{0.267004,0.004874,0.329415}%
\pgfsetfillcolor{currentfill}%
\pgfsetfillopacity{0.800000}%
\pgfsetlinewidth{0.000000pt}%
\definecolor{currentstroke}{rgb}{0.000000,0.000000,0.000000}%
\pgfsetstrokecolor{currentstroke}%
\pgfsetdash{}{0pt}%
\pgfpathmoveto{\pgfqpoint{3.293305in}{1.375341in}}%
\pgfpathlineto{\pgfqpoint{3.307219in}{1.369449in}}%
\pgfpathlineto{\pgfqpoint{3.321136in}{1.363753in}}%
\pgfpathlineto{\pgfqpoint{3.335055in}{1.358250in}}%
\pgfpathlineto{\pgfqpoint{3.348978in}{1.352942in}}%
\pgfpathlineto{\pgfqpoint{3.357410in}{1.359367in}}%
\pgfpathlineto{\pgfqpoint{3.365831in}{1.366014in}}%
\pgfpathlineto{\pgfqpoint{3.374243in}{1.372875in}}%
\pgfpathlineto{\pgfqpoint{3.382645in}{1.379945in}}%
\pgfpathlineto{\pgfqpoint{3.368747in}{1.384593in}}%
\pgfpathlineto{\pgfqpoint{3.354853in}{1.389435in}}%
\pgfpathlineto{\pgfqpoint{3.340962in}{1.394471in}}%
\pgfpathlineto{\pgfqpoint{3.327074in}{1.399702in}}%
\pgfpathlineto{\pgfqpoint{3.318647in}{1.393281in}}%
\pgfpathlineto{\pgfqpoint{3.310211in}{1.387076in}}%
\pgfpathlineto{\pgfqpoint{3.301763in}{1.381094in}}%
\pgfpathlineto{\pgfqpoint{3.293305in}{1.375341in}}%
\pgfpathclose%
\pgfusepath{fill}%
\end{pgfscope}%
\begin{pgfscope}%
\pgfpathrectangle{\pgfqpoint{1.150000in}{0.150000in}}{\pgfqpoint{5.700000in}{5.700000in}}%
\pgfusepath{clip}%
\pgfsetbuttcap%
\pgfsetroundjoin%
\definecolor{currentfill}{rgb}{0.274128,0.199721,0.498911}%
\pgfsetfillcolor{currentfill}%
\pgfsetfillopacity{0.800000}%
\pgfsetlinewidth{0.000000pt}%
\definecolor{currentstroke}{rgb}{0.000000,0.000000,0.000000}%
\pgfsetstrokecolor{currentstroke}%
\pgfsetdash{}{0pt}%
\pgfpathmoveto{\pgfqpoint{4.092151in}{1.765287in}}%
\pgfpathlineto{\pgfqpoint{4.106222in}{1.770948in}}%
\pgfpathlineto{\pgfqpoint{4.120304in}{1.776792in}}%
\pgfpathlineto{\pgfqpoint{4.134398in}{1.782819in}}%
\pgfpathlineto{\pgfqpoint{4.148505in}{1.789029in}}%
\pgfpathlineto{\pgfqpoint{4.156614in}{1.803418in}}%
\pgfpathlineto{\pgfqpoint{4.164719in}{1.817776in}}%
\pgfpathlineto{\pgfqpoint{4.172819in}{1.832099in}}%
\pgfpathlineto{\pgfqpoint{4.180915in}{1.846384in}}%
\pgfpathlineto{\pgfqpoint{4.166809in}{1.839792in}}%
\pgfpathlineto{\pgfqpoint{4.152715in}{1.833384in}}%
\pgfpathlineto{\pgfqpoint{4.138634in}{1.827158in}}%
\pgfpathlineto{\pgfqpoint{4.124565in}{1.821116in}}%
\pgfpathlineto{\pgfqpoint{4.116468in}{1.807201in}}%
\pgfpathlineto{\pgfqpoint{4.108367in}{1.793255in}}%
\pgfpathlineto{\pgfqpoint{4.100261in}{1.779282in}}%
\pgfpathlineto{\pgfqpoint{4.092151in}{1.765287in}}%
\pgfpathclose%
\pgfusepath{fill}%
\end{pgfscope}%
\begin{pgfscope}%
\pgfpathrectangle{\pgfqpoint{1.150000in}{0.150000in}}{\pgfqpoint{5.700000in}{5.700000in}}%
\pgfusepath{clip}%
\pgfsetbuttcap%
\pgfsetroundjoin%
\definecolor{currentfill}{rgb}{0.208030,0.718701,0.472873}%
\pgfsetfillcolor{currentfill}%
\pgfsetfillopacity{0.800000}%
\pgfsetlinewidth{0.000000pt}%
\definecolor{currentstroke}{rgb}{0.000000,0.000000,0.000000}%
\pgfsetstrokecolor{currentstroke}%
\pgfsetdash{}{0pt}%
\pgfpathmoveto{\pgfqpoint{5.504988in}{3.284973in}}%
\pgfpathlineto{\pgfqpoint{5.519923in}{3.301287in}}%
\pgfpathlineto{\pgfqpoint{5.534881in}{3.317789in}}%
\pgfpathlineto{\pgfqpoint{5.549861in}{3.334478in}}%
\pgfpathlineto{\pgfqpoint{5.564865in}{3.351355in}}%
\pgfpathlineto{\pgfqpoint{5.572316in}{3.354948in}}%
\pgfpathlineto{\pgfqpoint{5.579756in}{3.358403in}}%
\pgfpathlineto{\pgfqpoint{5.587186in}{3.361724in}}%
\pgfpathlineto{\pgfqpoint{5.594605in}{3.364915in}}%
\pgfpathlineto{\pgfqpoint{5.579620in}{3.348378in}}%
\pgfpathlineto{\pgfqpoint{5.564658in}{3.332028in}}%
\pgfpathlineto{\pgfqpoint{5.549718in}{3.315864in}}%
\pgfpathlineto{\pgfqpoint{5.534801in}{3.299886in}}%
\pgfpathlineto{\pgfqpoint{5.527363in}{3.296344in}}%
\pgfpathlineto{\pgfqpoint{5.519915in}{3.292680in}}%
\pgfpathlineto{\pgfqpoint{5.512457in}{3.288891in}}%
\pgfpathlineto{\pgfqpoint{5.504988in}{3.284973in}}%
\pgfpathclose%
\pgfusepath{fill}%
\end{pgfscope}%
\begin{pgfscope}%
\pgfpathrectangle{\pgfqpoint{1.150000in}{0.150000in}}{\pgfqpoint{5.700000in}{5.700000in}}%
\pgfusepath{clip}%
\pgfsetbuttcap%
\pgfsetroundjoin%
\definecolor{currentfill}{rgb}{0.283197,0.115680,0.436115}%
\pgfsetfillcolor{currentfill}%
\pgfsetfillopacity{0.800000}%
\pgfsetlinewidth{0.000000pt}%
\definecolor{currentstroke}{rgb}{0.000000,0.000000,0.000000}%
\pgfsetstrokecolor{currentstroke}%
\pgfsetdash{}{0pt}%
\pgfpathmoveto{\pgfqpoint{3.882251in}{1.568862in}}%
\pgfpathlineto{\pgfqpoint{3.896245in}{1.571728in}}%
\pgfpathlineto{\pgfqpoint{3.910249in}{1.574778in}}%
\pgfpathlineto{\pgfqpoint{3.924263in}{1.578010in}}%
\pgfpathlineto{\pgfqpoint{3.938287in}{1.581425in}}%
\pgfpathlineto{\pgfqpoint{3.946454in}{1.594910in}}%
\pgfpathlineto{\pgfqpoint{3.954616in}{1.608428in}}%
\pgfpathlineto{\pgfqpoint{3.962774in}{1.621974in}}%
\pgfpathlineto{\pgfqpoint{3.970927in}{1.635544in}}%
\pgfpathlineto{\pgfqpoint{3.956908in}{1.631654in}}%
\pgfpathlineto{\pgfqpoint{3.942899in}{1.627947in}}%
\pgfpathlineto{\pgfqpoint{3.928900in}{1.624424in}}%
\pgfpathlineto{\pgfqpoint{3.914911in}{1.621084in}}%
\pgfpathlineto{\pgfqpoint{3.906753in}{1.607976in}}%
\pgfpathlineto{\pgfqpoint{3.898591in}{1.594900in}}%
\pgfpathlineto{\pgfqpoint{3.890423in}{1.581860in}}%
\pgfpathlineto{\pgfqpoint{3.882251in}{1.568862in}}%
\pgfpathclose%
\pgfusepath{fill}%
\end{pgfscope}%
\begin{pgfscope}%
\pgfpathrectangle{\pgfqpoint{1.150000in}{0.150000in}}{\pgfqpoint{5.700000in}{5.700000in}}%
\pgfusepath{clip}%
\pgfsetbuttcap%
\pgfsetroundjoin%
\definecolor{currentfill}{rgb}{0.281924,0.089666,0.412415}%
\pgfsetfillcolor{currentfill}%
\pgfsetfillopacity{0.800000}%
\pgfsetlinewidth{0.000000pt}%
\definecolor{currentstroke}{rgb}{0.000000,0.000000,0.000000}%
\pgfsetstrokecolor{currentstroke}%
\pgfsetdash{}{0pt}%
\pgfpathmoveto{\pgfqpoint{2.889897in}{1.574761in}}%
\pgfpathlineto{\pgfqpoint{2.903885in}{1.562494in}}%
\pgfpathlineto{\pgfqpoint{2.917869in}{1.550445in}}%
\pgfpathlineto{\pgfqpoint{2.931851in}{1.538613in}}%
\pgfpathlineto{\pgfqpoint{2.945831in}{1.526998in}}%
\pgfpathlineto{\pgfqpoint{2.954573in}{1.526943in}}%
\pgfpathlineto{\pgfqpoint{2.963298in}{1.527217in}}%
\pgfpathlineto{\pgfqpoint{2.972007in}{1.527813in}}%
\pgfpathlineto{\pgfqpoint{2.980700in}{1.528724in}}%
\pgfpathlineto{\pgfqpoint{2.966762in}{1.539604in}}%
\pgfpathlineto{\pgfqpoint{2.952822in}{1.550699in}}%
\pgfpathlineto{\pgfqpoint{2.938881in}{1.562010in}}%
\pgfpathlineto{\pgfqpoint{2.924937in}{1.573538in}}%
\pgfpathlineto{\pgfqpoint{2.916202in}{1.573351in}}%
\pgfpathlineto{\pgfqpoint{2.907451in}{1.573488in}}%
\pgfpathlineto{\pgfqpoint{2.898683in}{1.573955in}}%
\pgfpathlineto{\pgfqpoint{2.889897in}{1.574761in}}%
\pgfpathclose%
\pgfusepath{fill}%
\end{pgfscope}%
\begin{pgfscope}%
\pgfpathrectangle{\pgfqpoint{1.150000in}{0.150000in}}{\pgfqpoint{5.700000in}{5.700000in}}%
\pgfusepath{clip}%
\pgfsetbuttcap%
\pgfsetroundjoin%
\definecolor{currentfill}{rgb}{0.141935,0.526453,0.555991}%
\pgfsetfillcolor{currentfill}%
\pgfsetfillopacity{0.800000}%
\pgfsetlinewidth{0.000000pt}%
\definecolor{currentstroke}{rgb}{0.000000,0.000000,0.000000}%
\pgfsetstrokecolor{currentstroke}%
\pgfsetdash{}{0pt}%
\pgfpathmoveto{\pgfqpoint{4.875397in}{2.663031in}}%
\pgfpathlineto{\pgfqpoint{4.889908in}{2.676288in}}%
\pgfpathlineto{\pgfqpoint{4.904438in}{2.689732in}}%
\pgfpathlineto{\pgfqpoint{4.918987in}{2.703361in}}%
\pgfpathlineto{\pgfqpoint{4.933556in}{2.717177in}}%
\pgfpathlineto{\pgfqpoint{4.941401in}{2.727567in}}%
\pgfpathlineto{\pgfqpoint{4.949238in}{2.737793in}}%
\pgfpathlineto{\pgfqpoint{4.957067in}{2.747857in}}%
\pgfpathlineto{\pgfqpoint{4.964888in}{2.757758in}}%
\pgfpathlineto{\pgfqpoint{4.950322in}{2.743958in}}%
\pgfpathlineto{\pgfqpoint{4.935776in}{2.730345in}}%
\pgfpathlineto{\pgfqpoint{4.921248in}{2.716917in}}%
\pgfpathlineto{\pgfqpoint{4.906739in}{2.703675in}}%
\pgfpathlineto{\pgfqpoint{4.898915in}{2.693745in}}%
\pgfpathlineto{\pgfqpoint{4.891083in}{2.683661in}}%
\pgfpathlineto{\pgfqpoint{4.883244in}{2.673423in}}%
\pgfpathlineto{\pgfqpoint{4.875397in}{2.663031in}}%
\pgfpathclose%
\pgfusepath{fill}%
\end{pgfscope}%
\begin{pgfscope}%
\pgfpathrectangle{\pgfqpoint{1.150000in}{0.150000in}}{\pgfqpoint{5.700000in}{5.700000in}}%
\pgfusepath{clip}%
\pgfsetbuttcap%
\pgfsetroundjoin%
\definecolor{currentfill}{rgb}{0.216210,0.351535,0.550627}%
\pgfsetfillcolor{currentfill}%
\pgfsetfillopacity{0.800000}%
\pgfsetlinewidth{0.000000pt}%
\definecolor{currentstroke}{rgb}{0.000000,0.000000,0.000000}%
\pgfsetstrokecolor{currentstroke}%
\pgfsetdash{}{0pt}%
\pgfpathmoveto{\pgfqpoint{4.423196in}{2.136294in}}%
\pgfpathlineto{\pgfqpoint{4.437431in}{2.145746in}}%
\pgfpathlineto{\pgfqpoint{4.451682in}{2.155381in}}%
\pgfpathlineto{\pgfqpoint{4.465948in}{2.165201in}}%
\pgfpathlineto{\pgfqpoint{4.480229in}{2.175204in}}%
\pgfpathlineto{\pgfqpoint{4.488250in}{2.189100in}}%
\pgfpathlineto{\pgfqpoint{4.496266in}{2.202886in}}%
\pgfpathlineto{\pgfqpoint{4.504277in}{2.216559in}}%
\pgfpathlineto{\pgfqpoint{4.512282in}{2.230120in}}%
\pgfpathlineto{\pgfqpoint{4.497999in}{2.219894in}}%
\pgfpathlineto{\pgfqpoint{4.483731in}{2.209853in}}%
\pgfpathlineto{\pgfqpoint{4.469479in}{2.199996in}}%
\pgfpathlineto{\pgfqpoint{4.455242in}{2.190323in}}%
\pgfpathlineto{\pgfqpoint{4.447238in}{2.176972in}}%
\pgfpathlineto{\pgfqpoint{4.439229in}{2.163516in}}%
\pgfpathlineto{\pgfqpoint{4.431215in}{2.149956in}}%
\pgfpathlineto{\pgfqpoint{4.423196in}{2.136294in}}%
\pgfpathclose%
\pgfusepath{fill}%
\end{pgfscope}%
\begin{pgfscope}%
\pgfpathrectangle{\pgfqpoint{1.150000in}{0.150000in}}{\pgfqpoint{5.700000in}{5.700000in}}%
\pgfusepath{clip}%
\pgfsetbuttcap%
\pgfsetroundjoin%
\definecolor{currentfill}{rgb}{0.214298,0.355619,0.551184}%
\pgfsetfillcolor{currentfill}%
\pgfsetfillopacity{0.800000}%
\pgfsetlinewidth{0.000000pt}%
\definecolor{currentstroke}{rgb}{0.000000,0.000000,0.000000}%
\pgfsetstrokecolor{currentstroke}%
\pgfsetdash{}{0pt}%
\pgfpathmoveto{\pgfqpoint{2.325598in}{2.260843in}}%
\pgfpathlineto{\pgfqpoint{2.339883in}{2.238590in}}%
\pgfpathlineto{\pgfqpoint{2.354156in}{2.216626in}}%
\pgfpathlineto{\pgfqpoint{2.368416in}{2.194946in}}%
\pgfpathlineto{\pgfqpoint{2.382664in}{2.173550in}}%
\pgfpathlineto{\pgfqpoint{2.391948in}{2.165811in}}%
\pgfpathlineto{\pgfqpoint{2.401205in}{2.158497in}}%
\pgfpathlineto{\pgfqpoint{2.410437in}{2.151599in}}%
\pgfpathlineto{\pgfqpoint{2.419643in}{2.145109in}}%
\pgfpathlineto{\pgfqpoint{2.405460in}{2.165725in}}%
\pgfpathlineto{\pgfqpoint{2.391265in}{2.186623in}}%
\pgfpathlineto{\pgfqpoint{2.377058in}{2.207805in}}%
\pgfpathlineto{\pgfqpoint{2.362840in}{2.229273in}}%
\pgfpathlineto{\pgfqpoint{2.353569in}{2.236530in}}%
\pgfpathlineto{\pgfqpoint{2.344272in}{2.244205in}}%
\pgfpathlineto{\pgfqpoint{2.334948in}{2.252307in}}%
\pgfpathlineto{\pgfqpoint{2.325598in}{2.260843in}}%
\pgfpathclose%
\pgfusepath{fill}%
\end{pgfscope}%
\begin{pgfscope}%
\pgfpathrectangle{\pgfqpoint{1.150000in}{0.150000in}}{\pgfqpoint{5.700000in}{5.700000in}}%
\pgfusepath{clip}%
\pgfsetbuttcap%
\pgfsetroundjoin%
\definecolor{currentfill}{rgb}{0.271305,0.019942,0.347269}%
\pgfsetfillcolor{currentfill}%
\pgfsetfillopacity{0.800000}%
\pgfsetlinewidth{0.000000pt}%
\definecolor{currentstroke}{rgb}{0.000000,0.000000,0.000000}%
\pgfsetstrokecolor{currentstroke}%
\pgfsetdash{}{0pt}%
\pgfpathmoveto{\pgfqpoint{3.147878in}{1.414571in}}%
\pgfpathlineto{\pgfqpoint{3.161810in}{1.406395in}}%
\pgfpathlineto{\pgfqpoint{3.175741in}{1.398421in}}%
\pgfpathlineto{\pgfqpoint{3.189674in}{1.390648in}}%
\pgfpathlineto{\pgfqpoint{3.203609in}{1.383074in}}%
\pgfpathlineto{\pgfqpoint{3.212143in}{1.387133in}}%
\pgfpathlineto{\pgfqpoint{3.220665in}{1.391458in}}%
\pgfpathlineto{\pgfqpoint{3.229175in}{1.396040in}}%
\pgfpathlineto{\pgfqpoint{3.237673in}{1.400875in}}%
\pgfpathlineto{\pgfqpoint{3.223770in}{1.407753in}}%
\pgfpathlineto{\pgfqpoint{3.209868in}{1.414831in}}%
\pgfpathlineto{\pgfqpoint{3.195968in}{1.422110in}}%
\pgfpathlineto{\pgfqpoint{3.182070in}{1.429590in}}%
\pgfpathlineto{\pgfqpoint{3.173541in}{1.425438in}}%
\pgfpathlineto{\pgfqpoint{3.165000in}{1.421547in}}%
\pgfpathlineto{\pgfqpoint{3.156446in}{1.417922in}}%
\pgfpathlineto{\pgfqpoint{3.147878in}{1.414571in}}%
\pgfpathclose%
\pgfusepath{fill}%
\end{pgfscope}%
\begin{pgfscope}%
\pgfpathrectangle{\pgfqpoint{1.150000in}{0.150000in}}{\pgfqpoint{5.700000in}{5.700000in}}%
\pgfusepath{clip}%
\pgfsetbuttcap%
\pgfsetroundjoin%
\definecolor{currentfill}{rgb}{0.267004,0.004874,0.329415}%
\pgfsetfillcolor{currentfill}%
\pgfsetfillopacity{0.800000}%
\pgfsetlinewidth{0.000000pt}%
\definecolor{currentstroke}{rgb}{0.000000,0.000000,0.000000}%
\pgfsetstrokecolor{currentstroke}%
\pgfsetdash{}{0pt}%
\pgfpathmoveto{\pgfqpoint{3.438274in}{1.363278in}}%
\pgfpathlineto{\pgfqpoint{3.452192in}{1.359589in}}%
\pgfpathlineto{\pgfqpoint{3.466114in}{1.356091in}}%
\pgfpathlineto{\pgfqpoint{3.480041in}{1.352782in}}%
\pgfpathlineto{\pgfqpoint{3.493972in}{1.349662in}}%
\pgfpathlineto{\pgfqpoint{3.502322in}{1.358214in}}%
\pgfpathlineto{\pgfqpoint{3.510663in}{1.366945in}}%
\pgfpathlineto{\pgfqpoint{3.518996in}{1.375849in}}%
\pgfpathlineto{\pgfqpoint{3.527321in}{1.384921in}}%
\pgfpathlineto{\pgfqpoint{3.513408in}{1.387413in}}%
\pgfpathlineto{\pgfqpoint{3.499501in}{1.390094in}}%
\pgfpathlineto{\pgfqpoint{3.485598in}{1.392964in}}%
\pgfpathlineto{\pgfqpoint{3.471701in}{1.396025in}}%
\pgfpathlineto{\pgfqpoint{3.463357in}{1.387569in}}%
\pgfpathlineto{\pgfqpoint{3.455005in}{1.379289in}}%
\pgfpathlineto{\pgfqpoint{3.446644in}{1.371190in}}%
\pgfpathlineto{\pgfqpoint{3.438274in}{1.363278in}}%
\pgfpathclose%
\pgfusepath{fill}%
\end{pgfscope}%
\begin{pgfscope}%
\pgfpathrectangle{\pgfqpoint{1.150000in}{0.150000in}}{\pgfqpoint{5.700000in}{5.700000in}}%
\pgfusepath{clip}%
\pgfsetbuttcap%
\pgfsetroundjoin%
\definecolor{currentfill}{rgb}{0.149039,0.508051,0.557250}%
\pgfsetfillcolor{currentfill}%
\pgfsetfillopacity{0.800000}%
\pgfsetlinewidth{0.000000pt}%
\definecolor{currentstroke}{rgb}{0.000000,0.000000,0.000000}%
\pgfsetstrokecolor{currentstroke}%
\pgfsetdash{}{0pt}%
\pgfpathmoveto{\pgfqpoint{2.075483in}{2.722806in}}%
\pgfpathlineto{\pgfqpoint{2.090007in}{2.695041in}}%
\pgfpathlineto{\pgfqpoint{2.104513in}{2.667620in}}%
\pgfpathlineto{\pgfqpoint{2.119001in}{2.640540in}}%
\pgfpathlineto{\pgfqpoint{2.133470in}{2.613797in}}%
\pgfpathlineto{\pgfqpoint{2.142994in}{2.603851in}}%
\pgfpathlineto{\pgfqpoint{2.152488in}{2.594348in}}%
\pgfpathlineto{\pgfqpoint{2.161954in}{2.585281in}}%
\pgfpathlineto{\pgfqpoint{2.171391in}{2.576643in}}%
\pgfpathlineto{\pgfqpoint{2.156995in}{2.602609in}}%
\pgfpathlineto{\pgfqpoint{2.142582in}{2.628911in}}%
\pgfpathlineto{\pgfqpoint{2.128151in}{2.655552in}}%
\pgfpathlineto{\pgfqpoint{2.113702in}{2.682534in}}%
\pgfpathlineto{\pgfqpoint{2.104192in}{2.691936in}}%
\pgfpathlineto{\pgfqpoint{2.094652in}{2.701777in}}%
\pgfpathlineto{\pgfqpoint{2.085083in}{2.712064in}}%
\pgfpathlineto{\pgfqpoint{2.075483in}{2.722806in}}%
\pgfpathclose%
\pgfusepath{fill}%
\end{pgfscope}%
\begin{pgfscope}%
\pgfpathrectangle{\pgfqpoint{1.150000in}{0.150000in}}{\pgfqpoint{5.700000in}{5.700000in}}%
\pgfusepath{clip}%
\pgfsetbuttcap%
\pgfsetroundjoin%
\definecolor{currentfill}{rgb}{0.241237,0.296485,0.539709}%
\pgfsetfillcolor{currentfill}%
\pgfsetfillopacity{0.800000}%
\pgfsetlinewidth{0.000000pt}%
\definecolor{currentstroke}{rgb}{0.000000,0.000000,0.000000}%
\pgfsetstrokecolor{currentstroke}%
\pgfsetdash{}{0pt}%
\pgfpathmoveto{\pgfqpoint{4.302084in}{1.989782in}}%
\pgfpathlineto{\pgfqpoint{4.316258in}{1.997958in}}%
\pgfpathlineto{\pgfqpoint{4.330446in}{2.006318in}}%
\pgfpathlineto{\pgfqpoint{4.344648in}{2.014861in}}%
\pgfpathlineto{\pgfqpoint{4.358865in}{2.023587in}}%
\pgfpathlineto{\pgfqpoint{4.366923in}{2.037986in}}%
\pgfpathlineto{\pgfqpoint{4.374977in}{2.052302in}}%
\pgfpathlineto{\pgfqpoint{4.383025in}{2.066530in}}%
\pgfpathlineto{\pgfqpoint{4.391069in}{2.080670in}}%
\pgfpathlineto{\pgfqpoint{4.376850in}{2.071657in}}%
\pgfpathlineto{\pgfqpoint{4.362646in}{2.062827in}}%
\pgfpathlineto{\pgfqpoint{4.348457in}{2.054180in}}%
\pgfpathlineto{\pgfqpoint{4.334281in}{2.045718in}}%
\pgfpathlineto{\pgfqpoint{4.326239in}{2.031852in}}%
\pgfpathlineto{\pgfqpoint{4.318192in}{2.017906in}}%
\pgfpathlineto{\pgfqpoint{4.310140in}{2.003882in}}%
\pgfpathlineto{\pgfqpoint{4.302084in}{1.989782in}}%
\pgfpathclose%
\pgfusepath{fill}%
\end{pgfscope}%
\begin{pgfscope}%
\pgfpathrectangle{\pgfqpoint{1.150000in}{0.150000in}}{\pgfqpoint{5.700000in}{5.700000in}}%
\pgfusepath{clip}%
\pgfsetbuttcap%
\pgfsetroundjoin%
\definecolor{currentfill}{rgb}{0.140210,0.665859,0.513427}%
\pgfsetfillcolor{currentfill}%
\pgfsetfillopacity{0.800000}%
\pgfsetlinewidth{0.000000pt}%
\definecolor{currentstroke}{rgb}{0.000000,0.000000,0.000000}%
\pgfsetstrokecolor{currentstroke}%
\pgfsetdash{}{0pt}%
\pgfpathmoveto{\pgfqpoint{5.295555in}{3.096426in}}%
\pgfpathlineto{\pgfqpoint{5.310354in}{3.112042in}}%
\pgfpathlineto{\pgfqpoint{5.325175in}{3.127845in}}%
\pgfpathlineto{\pgfqpoint{5.340017in}{3.143835in}}%
\pgfpathlineto{\pgfqpoint{5.354882in}{3.160013in}}%
\pgfpathlineto{\pgfqpoint{5.362486in}{3.165967in}}%
\pgfpathlineto{\pgfqpoint{5.370080in}{3.171761in}}%
\pgfpathlineto{\pgfqpoint{5.377664in}{3.177398in}}%
\pgfpathlineto{\pgfqpoint{5.385238in}{3.182882in}}%
\pgfpathlineto{\pgfqpoint{5.370386in}{3.166934in}}%
\pgfpathlineto{\pgfqpoint{5.355555in}{3.151174in}}%
\pgfpathlineto{\pgfqpoint{5.340746in}{3.135601in}}%
\pgfpathlineto{\pgfqpoint{5.325958in}{3.120214in}}%
\pgfpathlineto{\pgfqpoint{5.318372in}{3.114487in}}%
\pgfpathlineto{\pgfqpoint{5.310776in}{3.108616in}}%
\pgfpathlineto{\pgfqpoint{5.303170in}{3.102597in}}%
\pgfpathlineto{\pgfqpoint{5.295555in}{3.096426in}}%
\pgfpathclose%
\pgfusepath{fill}%
\end{pgfscope}%
\begin{pgfscope}%
\pgfpathrectangle{\pgfqpoint{1.150000in}{0.150000in}}{\pgfqpoint{5.700000in}{5.700000in}}%
\pgfusepath{clip}%
\pgfsetbuttcap%
\pgfsetroundjoin%
\definecolor{currentfill}{rgb}{0.120092,0.600104,0.542530}%
\pgfsetfillcolor{currentfill}%
\pgfsetfillopacity{0.800000}%
\pgfsetlinewidth{0.000000pt}%
\definecolor{currentstroke}{rgb}{0.000000,0.000000,0.000000}%
\pgfsetstrokecolor{currentstroke}%
\pgfsetdash{}{0pt}%
\pgfpathmoveto{\pgfqpoint{5.085593in}{2.887757in}}%
\pgfpathlineto{\pgfqpoint{5.100249in}{2.902351in}}%
\pgfpathlineto{\pgfqpoint{5.114926in}{2.917132in}}%
\pgfpathlineto{\pgfqpoint{5.129623in}{2.932100in}}%
\pgfpathlineto{\pgfqpoint{5.144341in}{2.947255in}}%
\pgfpathlineto{\pgfqpoint{5.152077in}{2.955526in}}%
\pgfpathlineto{\pgfqpoint{5.159804in}{2.963629in}}%
\pgfpathlineto{\pgfqpoint{5.167522in}{2.971565in}}%
\pgfpathlineto{\pgfqpoint{5.175231in}{2.979337in}}%
\pgfpathlineto{\pgfqpoint{5.160519in}{2.964304in}}%
\pgfpathlineto{\pgfqpoint{5.145829in}{2.949458in}}%
\pgfpathlineto{\pgfqpoint{5.131158in}{2.934799in}}%
\pgfpathlineto{\pgfqpoint{5.116509in}{2.920326in}}%
\pgfpathlineto{\pgfqpoint{5.108793in}{2.912419in}}%
\pgfpathlineto{\pgfqpoint{5.101068in}{2.904357in}}%
\pgfpathlineto{\pgfqpoint{5.093335in}{2.896136in}}%
\pgfpathlineto{\pgfqpoint{5.085593in}{2.887757in}}%
\pgfpathclose%
\pgfusepath{fill}%
\end{pgfscope}%
\begin{pgfscope}%
\pgfpathrectangle{\pgfqpoint{1.150000in}{0.150000in}}{\pgfqpoint{5.700000in}{5.700000in}}%
\pgfusepath{clip}%
\pgfsetbuttcap%
\pgfsetroundjoin%
\definecolor{currentfill}{rgb}{0.281887,0.150881,0.465405}%
\pgfsetfillcolor{currentfill}%
\pgfsetfillopacity{0.800000}%
\pgfsetlinewidth{0.000000pt}%
\definecolor{currentstroke}{rgb}{0.000000,0.000000,0.000000}%
\pgfsetstrokecolor{currentstroke}%
\pgfsetdash{}{0pt}%
\pgfpathmoveto{\pgfqpoint{3.970927in}{1.635544in}}%
\pgfpathlineto{\pgfqpoint{3.984957in}{1.639617in}}%
\pgfpathlineto{\pgfqpoint{3.998998in}{1.643873in}}%
\pgfpathlineto{\pgfqpoint{4.013050in}{1.648312in}}%
\pgfpathlineto{\pgfqpoint{4.027113in}{1.652933in}}%
\pgfpathlineto{\pgfqpoint{4.035258in}{1.666978in}}%
\pgfpathlineto{\pgfqpoint{4.043399in}{1.681032in}}%
\pgfpathlineto{\pgfqpoint{4.051535in}{1.695089in}}%
\pgfpathlineto{\pgfqpoint{4.059667in}{1.709145in}}%
\pgfpathlineto{\pgfqpoint{4.045607in}{1.704080in}}%
\pgfpathlineto{\pgfqpoint{4.031558in}{1.699197in}}%
\pgfpathlineto{\pgfqpoint{4.017520in}{1.694498in}}%
\pgfpathlineto{\pgfqpoint{4.003494in}{1.689981in}}%
\pgfpathlineto{\pgfqpoint{3.995359in}{1.676357in}}%
\pgfpathlineto{\pgfqpoint{3.987220in}{1.662740in}}%
\pgfpathlineto{\pgfqpoint{3.979076in}{1.649134in}}%
\pgfpathlineto{\pgfqpoint{3.970927in}{1.635544in}}%
\pgfpathclose%
\pgfusepath{fill}%
\end{pgfscope}%
\begin{pgfscope}%
\pgfpathrectangle{\pgfqpoint{1.150000in}{0.150000in}}{\pgfqpoint{5.700000in}{5.700000in}}%
\pgfusepath{clip}%
\pgfsetbuttcap%
\pgfsetroundjoin%
\definecolor{currentfill}{rgb}{0.280267,0.073417,0.397163}%
\pgfsetfillcolor{currentfill}%
\pgfsetfillopacity{0.800000}%
\pgfsetlinewidth{0.000000pt}%
\definecolor{currentstroke}{rgb}{0.000000,0.000000,0.000000}%
\pgfsetstrokecolor{currentstroke}%
\pgfsetdash{}{0pt}%
\pgfpathmoveto{\pgfqpoint{2.945831in}{1.526998in}}%
\pgfpathlineto{\pgfqpoint{2.959808in}{1.515596in}}%
\pgfpathlineto{\pgfqpoint{2.973784in}{1.504409in}}%
\pgfpathlineto{\pgfqpoint{2.987758in}{1.493434in}}%
\pgfpathlineto{\pgfqpoint{3.001730in}{1.482670in}}%
\pgfpathlineto{\pgfqpoint{3.010430in}{1.483363in}}%
\pgfpathlineto{\pgfqpoint{3.019114in}{1.484377in}}%
\pgfpathlineto{\pgfqpoint{3.027782in}{1.485703in}}%
\pgfpathlineto{\pgfqpoint{3.036436in}{1.487336in}}%
\pgfpathlineto{\pgfqpoint{3.022504in}{1.497366in}}%
\pgfpathlineto{\pgfqpoint{3.008570in}{1.507606in}}%
\pgfpathlineto{\pgfqpoint{2.994636in}{1.518059in}}%
\pgfpathlineto{\pgfqpoint{2.980700in}{1.528724in}}%
\pgfpathlineto{\pgfqpoint{2.972007in}{1.527813in}}%
\pgfpathlineto{\pgfqpoint{2.963298in}{1.527217in}}%
\pgfpathlineto{\pgfqpoint{2.954573in}{1.526943in}}%
\pgfpathlineto{\pgfqpoint{2.945831in}{1.526998in}}%
\pgfpathclose%
\pgfusepath{fill}%
\end{pgfscope}%
\begin{pgfscope}%
\pgfpathrectangle{\pgfqpoint{1.150000in}{0.150000in}}{\pgfqpoint{5.700000in}{5.700000in}}%
\pgfusepath{clip}%
\pgfsetbuttcap%
\pgfsetroundjoin%
\definecolor{currentfill}{rgb}{0.159194,0.482237,0.558073}%
\pgfsetfillcolor{currentfill}%
\pgfsetfillopacity{0.800000}%
\pgfsetlinewidth{0.000000pt}%
\definecolor{currentstroke}{rgb}{0.000000,0.000000,0.000000}%
\pgfsetstrokecolor{currentstroke}%
\pgfsetdash{}{0pt}%
\pgfpathmoveto{\pgfqpoint{4.754490in}{2.523394in}}%
\pgfpathlineto{\pgfqpoint{4.768930in}{2.535837in}}%
\pgfpathlineto{\pgfqpoint{4.783387in}{2.548465in}}%
\pgfpathlineto{\pgfqpoint{4.797863in}{2.561278in}}%
\pgfpathlineto{\pgfqpoint{4.812357in}{2.574278in}}%
\pgfpathlineto{\pgfqpoint{4.820262in}{2.585919in}}%
\pgfpathlineto{\pgfqpoint{4.828160in}{2.597403in}}%
\pgfpathlineto{\pgfqpoint{4.836051in}{2.608732in}}%
\pgfpathlineto{\pgfqpoint{4.843935in}{2.619904in}}%
\pgfpathlineto{\pgfqpoint{4.829441in}{2.606851in}}%
\pgfpathlineto{\pgfqpoint{4.814966in}{2.593984in}}%
\pgfpathlineto{\pgfqpoint{4.800508in}{2.581303in}}%
\pgfpathlineto{\pgfqpoint{4.786069in}{2.568807in}}%
\pgfpathlineto{\pgfqpoint{4.778185in}{2.557675in}}%
\pgfpathlineto{\pgfqpoint{4.770293in}{2.546396in}}%
\pgfpathlineto{\pgfqpoint{4.762395in}{2.534969in}}%
\pgfpathlineto{\pgfqpoint{4.754490in}{2.523394in}}%
\pgfpathclose%
\pgfusepath{fill}%
\end{pgfscope}%
\begin{pgfscope}%
\pgfpathrectangle{\pgfqpoint{1.150000in}{0.150000in}}{\pgfqpoint{5.700000in}{5.700000in}}%
\pgfusepath{clip}%
\pgfsetbuttcap%
\pgfsetroundjoin%
\definecolor{currentfill}{rgb}{0.259857,0.745492,0.444467}%
\pgfsetfillcolor{currentfill}%
\pgfsetfillopacity{0.800000}%
\pgfsetlinewidth{0.000000pt}%
\definecolor{currentstroke}{rgb}{0.000000,0.000000,0.000000}%
\pgfsetstrokecolor{currentstroke}%
\pgfsetdash{}{0pt}%
\pgfpathmoveto{\pgfqpoint{5.594605in}{3.364915in}}%
\pgfpathlineto{\pgfqpoint{5.609614in}{3.381640in}}%
\pgfpathlineto{\pgfqpoint{5.624645in}{3.398551in}}%
\pgfpathlineto{\pgfqpoint{5.639700in}{3.415651in}}%
\pgfpathlineto{\pgfqpoint{5.654779in}{3.432938in}}%
\pgfpathlineto{\pgfqpoint{5.662168in}{3.435640in}}%
\pgfpathlineto{\pgfqpoint{5.669547in}{3.438212in}}%
\pgfpathlineto{\pgfqpoint{5.676915in}{3.440658in}}%
\pgfpathlineto{\pgfqpoint{5.684272in}{3.442981in}}%
\pgfpathlineto{\pgfqpoint{5.669214in}{3.426071in}}%
\pgfpathlineto{\pgfqpoint{5.654179in}{3.409348in}}%
\pgfpathlineto{\pgfqpoint{5.639168in}{3.392811in}}%
\pgfpathlineto{\pgfqpoint{5.624180in}{3.376461in}}%
\pgfpathlineto{\pgfqpoint{5.616802in}{3.373749in}}%
\pgfpathlineto{\pgfqpoint{5.609413in}{3.370924in}}%
\pgfpathlineto{\pgfqpoint{5.602014in}{3.367981in}}%
\pgfpathlineto{\pgfqpoint{5.594605in}{3.364915in}}%
\pgfpathclose%
\pgfusepath{fill}%
\end{pgfscope}%
\begin{pgfscope}%
\pgfpathrectangle{\pgfqpoint{1.150000in}{0.150000in}}{\pgfqpoint{5.700000in}{5.700000in}}%
\pgfusepath{clip}%
\pgfsetbuttcap%
\pgfsetroundjoin%
\definecolor{currentfill}{rgb}{0.197636,0.391528,0.554969}%
\pgfsetfillcolor{currentfill}%
\pgfsetfillopacity{0.800000}%
\pgfsetlinewidth{0.000000pt}%
\definecolor{currentstroke}{rgb}{0.000000,0.000000,0.000000}%
\pgfsetstrokecolor{currentstroke}%
\pgfsetdash{}{0pt}%
\pgfpathmoveto{\pgfqpoint{2.268326in}{2.352782in}}%
\pgfpathlineto{\pgfqpoint{2.282664in}{2.329353in}}%
\pgfpathlineto{\pgfqpoint{2.296989in}{2.306222in}}%
\pgfpathlineto{\pgfqpoint{2.311300in}{2.283386in}}%
\pgfpathlineto{\pgfqpoint{2.325598in}{2.260843in}}%
\pgfpathlineto{\pgfqpoint{2.334948in}{2.252307in}}%
\pgfpathlineto{\pgfqpoint{2.344272in}{2.244205in}}%
\pgfpathlineto{\pgfqpoint{2.353569in}{2.236530in}}%
\pgfpathlineto{\pgfqpoint{2.362840in}{2.229273in}}%
\pgfpathlineto{\pgfqpoint{2.348609in}{2.251029in}}%
\pgfpathlineto{\pgfqpoint{2.334366in}{2.273076in}}%
\pgfpathlineto{\pgfqpoint{2.320110in}{2.295417in}}%
\pgfpathlineto{\pgfqpoint{2.305841in}{2.318054in}}%
\pgfpathlineto{\pgfqpoint{2.296503in}{2.326085in}}%
\pgfpathlineto{\pgfqpoint{2.287138in}{2.334544in}}%
\pgfpathlineto{\pgfqpoint{2.277746in}{2.343441in}}%
\pgfpathlineto{\pgfqpoint{2.268326in}{2.352782in}}%
\pgfpathclose%
\pgfusepath{fill}%
\end{pgfscope}%
\begin{pgfscope}%
\pgfpathrectangle{\pgfqpoint{1.150000in}{0.150000in}}{\pgfqpoint{5.700000in}{5.700000in}}%
\pgfusepath{clip}%
\pgfsetbuttcap%
\pgfsetroundjoin%
\definecolor{currentfill}{rgb}{0.262138,0.242286,0.520837}%
\pgfsetfillcolor{currentfill}%
\pgfsetfillopacity{0.800000}%
\pgfsetlinewidth{0.000000pt}%
\definecolor{currentstroke}{rgb}{0.000000,0.000000,0.000000}%
\pgfsetstrokecolor{currentstroke}%
\pgfsetdash{}{0pt}%
\pgfpathmoveto{\pgfqpoint{4.180915in}{1.846384in}}%
\pgfpathlineto{\pgfqpoint{4.195034in}{1.853159in}}%
\pgfpathlineto{\pgfqpoint{4.209166in}{1.860116in}}%
\pgfpathlineto{\pgfqpoint{4.223311in}{1.867256in}}%
\pgfpathlineto{\pgfqpoint{4.237470in}{1.874579in}}%
\pgfpathlineto{\pgfqpoint{4.245562in}{1.889186in}}%
\pgfpathlineto{\pgfqpoint{4.253650in}{1.903740in}}%
\pgfpathlineto{\pgfqpoint{4.261733in}{1.918237in}}%
\pgfpathlineto{\pgfqpoint{4.269812in}{1.932676in}}%
\pgfpathlineto{\pgfqpoint{4.255653in}{1.925002in}}%
\pgfpathlineto{\pgfqpoint{4.241507in}{1.917511in}}%
\pgfpathlineto{\pgfqpoint{4.227375in}{1.910203in}}%
\pgfpathlineto{\pgfqpoint{4.213256in}{1.903078in}}%
\pgfpathlineto{\pgfqpoint{4.205177in}{1.888978in}}%
\pgfpathlineto{\pgfqpoint{4.197094in}{1.874827in}}%
\pgfpathlineto{\pgfqpoint{4.189007in}{1.860628in}}%
\pgfpathlineto{\pgfqpoint{4.180915in}{1.846384in}}%
\pgfpathclose%
\pgfusepath{fill}%
\end{pgfscope}%
\begin{pgfscope}%
\pgfpathrectangle{\pgfqpoint{1.150000in}{0.150000in}}{\pgfqpoint{5.700000in}{5.700000in}}%
\pgfusepath{clip}%
\pgfsetbuttcap%
\pgfsetroundjoin%
\definecolor{currentfill}{rgb}{0.267004,0.004874,0.329415}%
\pgfsetfillcolor{currentfill}%
\pgfsetfillopacity{0.800000}%
\pgfsetlinewidth{0.000000pt}%
\definecolor{currentstroke}{rgb}{0.000000,0.000000,0.000000}%
\pgfsetstrokecolor{currentstroke}%
\pgfsetdash{}{0pt}%
\pgfpathmoveto{\pgfqpoint{3.348978in}{1.352942in}}%
\pgfpathlineto{\pgfqpoint{3.362903in}{1.347827in}}%
\pgfpathlineto{\pgfqpoint{3.376833in}{1.342904in}}%
\pgfpathlineto{\pgfqpoint{3.390765in}{1.338173in}}%
\pgfpathlineto{\pgfqpoint{3.404701in}{1.333633in}}%
\pgfpathlineto{\pgfqpoint{3.413109in}{1.340731in}}%
\pgfpathlineto{\pgfqpoint{3.421507in}{1.348042in}}%
\pgfpathlineto{\pgfqpoint{3.429895in}{1.355560in}}%
\pgfpathlineto{\pgfqpoint{3.438274in}{1.363278in}}%
\pgfpathlineto{\pgfqpoint{3.424360in}{1.367157in}}%
\pgfpathlineto{\pgfqpoint{3.410451in}{1.371228in}}%
\pgfpathlineto{\pgfqpoint{3.396546in}{1.375490in}}%
\pgfpathlineto{\pgfqpoint{3.382645in}{1.379945in}}%
\pgfpathlineto{\pgfqpoint{3.374243in}{1.372875in}}%
\pgfpathlineto{\pgfqpoint{3.365831in}{1.366014in}}%
\pgfpathlineto{\pgfqpoint{3.357410in}{1.359367in}}%
\pgfpathlineto{\pgfqpoint{3.348978in}{1.352942in}}%
\pgfpathclose%
\pgfusepath{fill}%
\end{pgfscope}%
\begin{pgfscope}%
\pgfpathrectangle{\pgfqpoint{1.150000in}{0.150000in}}{\pgfqpoint{5.700000in}{5.700000in}}%
\pgfusepath{clip}%
\pgfsetbuttcap%
\pgfsetroundjoin%
\definecolor{currentfill}{rgb}{0.177423,0.437527,0.557565}%
\pgfsetfillcolor{currentfill}%
\pgfsetfillopacity{0.800000}%
\pgfsetlinewidth{0.000000pt}%
\definecolor{currentstroke}{rgb}{0.000000,0.000000,0.000000}%
\pgfsetstrokecolor{currentstroke}%
\pgfsetdash{}{0pt}%
\pgfpathmoveto{\pgfqpoint{4.633434in}{2.378447in}}%
\pgfpathlineto{\pgfqpoint{4.647803in}{2.389939in}}%
\pgfpathlineto{\pgfqpoint{4.662189in}{2.401616in}}%
\pgfpathlineto{\pgfqpoint{4.676592in}{2.413479in}}%
\pgfpathlineto{\pgfqpoint{4.691012in}{2.425526in}}%
\pgfpathlineto{\pgfqpoint{4.698969in}{2.438267in}}%
\pgfpathlineto{\pgfqpoint{4.706920in}{2.450864in}}%
\pgfpathlineto{\pgfqpoint{4.714865in}{2.463317in}}%
\pgfpathlineto{\pgfqpoint{4.722803in}{2.475625in}}%
\pgfpathlineto{\pgfqpoint{4.708381in}{2.463456in}}%
\pgfpathlineto{\pgfqpoint{4.693977in}{2.451472in}}%
\pgfpathlineto{\pgfqpoint{4.679590in}{2.439673in}}%
\pgfpathlineto{\pgfqpoint{4.665221in}{2.428059in}}%
\pgfpathlineto{\pgfqpoint{4.657284in}{2.415860in}}%
\pgfpathlineto{\pgfqpoint{4.649340in}{2.403524in}}%
\pgfpathlineto{\pgfqpoint{4.641390in}{2.391053in}}%
\pgfpathlineto{\pgfqpoint{4.633434in}{2.378447in}}%
\pgfpathclose%
\pgfusepath{fill}%
\end{pgfscope}%
\begin{pgfscope}%
\pgfpathrectangle{\pgfqpoint{1.150000in}{0.150000in}}{\pgfqpoint{5.700000in}{5.700000in}}%
\pgfusepath{clip}%
\pgfsetbuttcap%
\pgfsetroundjoin%
\definecolor{currentfill}{rgb}{0.276022,0.044167,0.370164}%
\pgfsetfillcolor{currentfill}%
\pgfsetfillopacity{0.800000}%
\pgfsetlinewidth{0.000000pt}%
\definecolor{currentstroke}{rgb}{0.000000,0.000000,0.000000}%
\pgfsetstrokecolor{currentstroke}%
\pgfsetdash{}{0pt}%
\pgfpathmoveto{\pgfqpoint{3.671931in}{1.414350in}}%
\pgfpathlineto{\pgfqpoint{3.685885in}{1.414137in}}%
\pgfpathlineto{\pgfqpoint{3.699846in}{1.414108in}}%
\pgfpathlineto{\pgfqpoint{3.713815in}{1.414264in}}%
\pgfpathlineto{\pgfqpoint{3.727791in}{1.414604in}}%
\pgfpathlineto{\pgfqpoint{3.736038in}{1.426155in}}%
\pgfpathlineto{\pgfqpoint{3.744278in}{1.437816in}}%
\pgfpathlineto{\pgfqpoint{3.752512in}{1.449579in}}%
\pgfpathlineto{\pgfqpoint{3.760741in}{1.461440in}}%
\pgfpathlineto{\pgfqpoint{3.746775in}{1.460533in}}%
\pgfpathlineto{\pgfqpoint{3.732818in}{1.459811in}}%
\pgfpathlineto{\pgfqpoint{3.718868in}{1.459273in}}%
\pgfpathlineto{\pgfqpoint{3.704926in}{1.458921in}}%
\pgfpathlineto{\pgfqpoint{3.696686in}{1.447614in}}%
\pgfpathlineto{\pgfqpoint{3.688441in}{1.436413in}}%
\pgfpathlineto{\pgfqpoint{3.680189in}{1.425324in}}%
\pgfpathlineto{\pgfqpoint{3.671931in}{1.414350in}}%
\pgfpathclose%
\pgfusepath{fill}%
\end{pgfscope}%
\begin{pgfscope}%
\pgfpathrectangle{\pgfqpoint{1.150000in}{0.150000in}}{\pgfqpoint{5.700000in}{5.700000in}}%
\pgfusepath{clip}%
\pgfsetbuttcap%
\pgfsetroundjoin%
\definecolor{currentfill}{rgb}{0.269944,0.014625,0.341379}%
\pgfsetfillcolor{currentfill}%
\pgfsetfillopacity{0.800000}%
\pgfsetlinewidth{0.000000pt}%
\definecolor{currentstroke}{rgb}{0.000000,0.000000,0.000000}%
\pgfsetstrokecolor{currentstroke}%
\pgfsetdash{}{0pt}%
\pgfpathmoveto{\pgfqpoint{3.203609in}{1.383074in}}%
\pgfpathlineto{\pgfqpoint{3.217544in}{1.375700in}}%
\pgfpathlineto{\pgfqpoint{3.231481in}{1.368524in}}%
\pgfpathlineto{\pgfqpoint{3.245419in}{1.361545in}}%
\pgfpathlineto{\pgfqpoint{3.259360in}{1.354763in}}%
\pgfpathlineto{\pgfqpoint{3.267864in}{1.359529in}}%
\pgfpathlineto{\pgfqpoint{3.276356in}{1.364552in}}%
\pgfpathlineto{\pgfqpoint{3.284836in}{1.369825in}}%
\pgfpathlineto{\pgfqpoint{3.293305in}{1.375341in}}%
\pgfpathlineto{\pgfqpoint{3.279394in}{1.381429in}}%
\pgfpathlineto{\pgfqpoint{3.265485in}{1.387713in}}%
\pgfpathlineto{\pgfqpoint{3.251578in}{1.394195in}}%
\pgfpathlineto{\pgfqpoint{3.237673in}{1.400875in}}%
\pgfpathlineto{\pgfqpoint{3.229175in}{1.396040in}}%
\pgfpathlineto{\pgfqpoint{3.220665in}{1.391458in}}%
\pgfpathlineto{\pgfqpoint{3.212143in}{1.387133in}}%
\pgfpathlineto{\pgfqpoint{3.203609in}{1.383074in}}%
\pgfpathclose%
\pgfusepath{fill}%
\end{pgfscope}%
\begin{pgfscope}%
\pgfpathrectangle{\pgfqpoint{1.150000in}{0.150000in}}{\pgfqpoint{5.700000in}{5.700000in}}%
\pgfusepath{clip}%
\pgfsetbuttcap%
\pgfsetroundjoin%
\definecolor{currentfill}{rgb}{0.279566,0.067836,0.391917}%
\pgfsetfillcolor{currentfill}%
\pgfsetfillopacity{0.800000}%
\pgfsetlinewidth{0.000000pt}%
\definecolor{currentstroke}{rgb}{0.000000,0.000000,0.000000}%
\pgfsetstrokecolor{currentstroke}%
\pgfsetdash{}{0pt}%
\pgfpathmoveto{\pgfqpoint{3.760741in}{1.461440in}}%
\pgfpathlineto{\pgfqpoint{3.774715in}{1.462531in}}%
\pgfpathlineto{\pgfqpoint{3.788697in}{1.463805in}}%
\pgfpathlineto{\pgfqpoint{3.802688in}{1.465263in}}%
\pgfpathlineto{\pgfqpoint{3.816687in}{1.466904in}}%
\pgfpathlineto{\pgfqpoint{3.824901in}{1.479406in}}%
\pgfpathlineto{\pgfqpoint{3.833109in}{1.491987in}}%
\pgfpathlineto{\pgfqpoint{3.841313in}{1.504643in}}%
\pgfpathlineto{\pgfqpoint{3.849510in}{1.517368in}}%
\pgfpathlineto{\pgfqpoint{3.835519in}{1.515190in}}%
\pgfpathlineto{\pgfqpoint{3.821536in}{1.513196in}}%
\pgfpathlineto{\pgfqpoint{3.807563in}{1.511385in}}%
\pgfpathlineto{\pgfqpoint{3.793598in}{1.509759in}}%
\pgfpathlineto{\pgfqpoint{3.785392in}{1.497558in}}%
\pgfpathlineto{\pgfqpoint{3.777181in}{1.485435in}}%
\pgfpathlineto{\pgfqpoint{3.768964in}{1.473394in}}%
\pgfpathlineto{\pgfqpoint{3.760741in}{1.461440in}}%
\pgfpathclose%
\pgfusepath{fill}%
\end{pgfscope}%
\begin{pgfscope}%
\pgfpathrectangle{\pgfqpoint{1.150000in}{0.150000in}}{\pgfqpoint{5.700000in}{5.700000in}}%
\pgfusepath{clip}%
\pgfsetbuttcap%
\pgfsetroundjoin%
\definecolor{currentfill}{rgb}{0.271305,0.019942,0.347269}%
\pgfsetfillcolor{currentfill}%
\pgfsetfillopacity{0.800000}%
\pgfsetlinewidth{0.000000pt}%
\definecolor{currentstroke}{rgb}{0.000000,0.000000,0.000000}%
\pgfsetstrokecolor{currentstroke}%
\pgfsetdash{}{0pt}%
\pgfpathmoveto{\pgfqpoint{3.583027in}{1.376835in}}%
\pgfpathlineto{\pgfqpoint{3.596969in}{1.375281in}}%
\pgfpathlineto{\pgfqpoint{3.610917in}{1.373914in}}%
\pgfpathlineto{\pgfqpoint{3.624871in}{1.372732in}}%
\pgfpathlineto{\pgfqpoint{3.638831in}{1.371736in}}%
\pgfpathlineto{\pgfqpoint{3.647117in}{1.382187in}}%
\pgfpathlineto{\pgfqpoint{3.655395in}{1.392776in}}%
\pgfpathlineto{\pgfqpoint{3.663666in}{1.403500in}}%
\pgfpathlineto{\pgfqpoint{3.671931in}{1.414350in}}%
\pgfpathlineto{\pgfqpoint{3.657984in}{1.414749in}}%
\pgfpathlineto{\pgfqpoint{3.644044in}{1.415333in}}%
\pgfpathlineto{\pgfqpoint{3.630111in}{1.416104in}}%
\pgfpathlineto{\pgfqpoint{3.616185in}{1.417061in}}%
\pgfpathlineto{\pgfqpoint{3.607906in}{1.406795in}}%
\pgfpathlineto{\pgfqpoint{3.599620in}{1.396665in}}%
\pgfpathlineto{\pgfqpoint{3.591328in}{1.386676in}}%
\pgfpathlineto{\pgfqpoint{3.583027in}{1.376835in}}%
\pgfpathclose%
\pgfusepath{fill}%
\end{pgfscope}%
\begin{pgfscope}%
\pgfpathrectangle{\pgfqpoint{1.150000in}{0.150000in}}{\pgfqpoint{5.700000in}{5.700000in}}%
\pgfusepath{clip}%
\pgfsetbuttcap%
\pgfsetroundjoin%
\definecolor{currentfill}{rgb}{0.277941,0.056324,0.381191}%
\pgfsetfillcolor{currentfill}%
\pgfsetfillopacity{0.800000}%
\pgfsetlinewidth{0.000000pt}%
\definecolor{currentstroke}{rgb}{0.000000,0.000000,0.000000}%
\pgfsetstrokecolor{currentstroke}%
\pgfsetdash{}{0pt}%
\pgfpathmoveto{\pgfqpoint{3.001730in}{1.482670in}}%
\pgfpathlineto{\pgfqpoint{3.015701in}{1.472116in}}%
\pgfpathlineto{\pgfqpoint{3.029670in}{1.461772in}}%
\pgfpathlineto{\pgfqpoint{3.043639in}{1.451637in}}%
\pgfpathlineto{\pgfqpoint{3.057606in}{1.441708in}}%
\pgfpathlineto{\pgfqpoint{3.066267in}{1.443147in}}%
\pgfpathlineto{\pgfqpoint{3.074912in}{1.444898in}}%
\pgfpathlineto{\pgfqpoint{3.083542in}{1.446953in}}%
\pgfpathlineto{\pgfqpoint{3.092158in}{1.449305in}}%
\pgfpathlineto{\pgfqpoint{3.078228in}{1.458502in}}%
\pgfpathlineto{\pgfqpoint{3.064298in}{1.467905in}}%
\pgfpathlineto{\pgfqpoint{3.050367in}{1.477516in}}%
\pgfpathlineto{\pgfqpoint{3.036436in}{1.487336in}}%
\pgfpathlineto{\pgfqpoint{3.027782in}{1.485703in}}%
\pgfpathlineto{\pgfqpoint{3.019114in}{1.484377in}}%
\pgfpathlineto{\pgfqpoint{3.010430in}{1.483363in}}%
\pgfpathlineto{\pgfqpoint{3.001730in}{1.482670in}}%
\pgfpathclose%
\pgfusepath{fill}%
\end{pgfscope}%
\begin{pgfscope}%
\pgfpathrectangle{\pgfqpoint{1.150000in}{0.150000in}}{\pgfqpoint{5.700000in}{5.700000in}}%
\pgfusepath{clip}%
\pgfsetbuttcap%
\pgfsetroundjoin%
\definecolor{currentfill}{rgb}{0.277134,0.185228,0.489898}%
\pgfsetfillcolor{currentfill}%
\pgfsetfillopacity{0.800000}%
\pgfsetlinewidth{0.000000pt}%
\definecolor{currentstroke}{rgb}{0.000000,0.000000,0.000000}%
\pgfsetstrokecolor{currentstroke}%
\pgfsetdash{}{0pt}%
\pgfpathmoveto{\pgfqpoint{4.059667in}{1.709145in}}%
\pgfpathlineto{\pgfqpoint{4.073739in}{1.714394in}}%
\pgfpathlineto{\pgfqpoint{4.087823in}{1.719824in}}%
\pgfpathlineto{\pgfqpoint{4.101918in}{1.725437in}}%
\pgfpathlineto{\pgfqpoint{4.116025in}{1.731232in}}%
\pgfpathlineto{\pgfqpoint{4.124151in}{1.745710in}}%
\pgfpathlineto{\pgfqpoint{4.132273in}{1.760171in}}%
\pgfpathlineto{\pgfqpoint{4.140391in}{1.774612in}}%
\pgfpathlineto{\pgfqpoint{4.148505in}{1.789029in}}%
\pgfpathlineto{\pgfqpoint{4.134398in}{1.782819in}}%
\pgfpathlineto{\pgfqpoint{4.120304in}{1.776792in}}%
\pgfpathlineto{\pgfqpoint{4.106222in}{1.770948in}}%
\pgfpathlineto{\pgfqpoint{4.092151in}{1.765287in}}%
\pgfpathlineto{\pgfqpoint{4.084037in}{1.751272in}}%
\pgfpathlineto{\pgfqpoint{4.075918in}{1.737241in}}%
\pgfpathlineto{\pgfqpoint{4.067795in}{1.723197in}}%
\pgfpathlineto{\pgfqpoint{4.059667in}{1.709145in}}%
\pgfpathclose%
\pgfusepath{fill}%
\end{pgfscope}%
\begin{pgfscope}%
\pgfpathrectangle{\pgfqpoint{1.150000in}{0.150000in}}{\pgfqpoint{5.700000in}{5.700000in}}%
\pgfusepath{clip}%
\pgfsetbuttcap%
\pgfsetroundjoin%
\definecolor{currentfill}{rgb}{0.129933,0.559582,0.551864}%
\pgfsetfillcolor{currentfill}%
\pgfsetfillopacity{0.800000}%
\pgfsetlinewidth{0.000000pt}%
\definecolor{currentstroke}{rgb}{0.000000,0.000000,0.000000}%
\pgfsetstrokecolor{currentstroke}%
\pgfsetdash{}{0pt}%
\pgfpathmoveto{\pgfqpoint{4.964888in}{2.757758in}}%
\pgfpathlineto{\pgfqpoint{4.979474in}{2.771744in}}%
\pgfpathlineto{\pgfqpoint{4.994079in}{2.785917in}}%
\pgfpathlineto{\pgfqpoint{5.008703in}{2.800276in}}%
\pgfpathlineto{\pgfqpoint{5.023348in}{2.814822in}}%
\pgfpathlineto{\pgfqpoint{5.031158in}{2.824523in}}%
\pgfpathlineto{\pgfqpoint{5.038960in}{2.834056in}}%
\pgfpathlineto{\pgfqpoint{5.046753in}{2.843420in}}%
\pgfpathlineto{\pgfqpoint{5.054538in}{2.852617in}}%
\pgfpathlineto{\pgfqpoint{5.039897in}{2.838123in}}%
\pgfpathlineto{\pgfqpoint{5.025276in}{2.823815in}}%
\pgfpathlineto{\pgfqpoint{5.010675in}{2.809694in}}%
\pgfpathlineto{\pgfqpoint{4.996093in}{2.795759in}}%
\pgfpathlineto{\pgfqpoint{4.988304in}{2.786498in}}%
\pgfpathlineto{\pgfqpoint{4.980507in}{2.777078in}}%
\pgfpathlineto{\pgfqpoint{4.972702in}{2.767498in}}%
\pgfpathlineto{\pgfqpoint{4.964888in}{2.757758in}}%
\pgfpathclose%
\pgfusepath{fill}%
\end{pgfscope}%
\begin{pgfscope}%
\pgfpathrectangle{\pgfqpoint{1.150000in}{0.150000in}}{\pgfqpoint{5.700000in}{5.700000in}}%
\pgfusepath{clip}%
\pgfsetbuttcap%
\pgfsetroundjoin%
\definecolor{currentfill}{rgb}{0.199430,0.387607,0.554642}%
\pgfsetfillcolor{currentfill}%
\pgfsetfillopacity{0.800000}%
\pgfsetlinewidth{0.000000pt}%
\definecolor{currentstroke}{rgb}{0.000000,0.000000,0.000000}%
\pgfsetstrokecolor{currentstroke}%
\pgfsetdash{}{0pt}%
\pgfpathmoveto{\pgfqpoint{4.512282in}{2.230120in}}%
\pgfpathlineto{\pgfqpoint{4.526582in}{2.240529in}}%
\pgfpathlineto{\pgfqpoint{4.540897in}{2.251123in}}%
\pgfpathlineto{\pgfqpoint{4.555229in}{2.261901in}}%
\pgfpathlineto{\pgfqpoint{4.569577in}{2.272864in}}%
\pgfpathlineto{\pgfqpoint{4.577579in}{2.286510in}}%
\pgfpathlineto{\pgfqpoint{4.585575in}{2.300031in}}%
\pgfpathlineto{\pgfqpoint{4.593566in}{2.313426in}}%
\pgfpathlineto{\pgfqpoint{4.601552in}{2.326691in}}%
\pgfpathlineto{\pgfqpoint{4.587201in}{2.315539in}}%
\pgfpathlineto{\pgfqpoint{4.572868in}{2.304572in}}%
\pgfpathlineto{\pgfqpoint{4.558550in}{2.293789in}}%
\pgfpathlineto{\pgfqpoint{4.544249in}{2.283191in}}%
\pgfpathlineto{\pgfqpoint{4.536266in}{2.270101in}}%
\pgfpathlineto{\pgfqpoint{4.528277in}{2.256892in}}%
\pgfpathlineto{\pgfqpoint{4.520282in}{2.243564in}}%
\pgfpathlineto{\pgfqpoint{4.512282in}{2.230120in}}%
\pgfpathclose%
\pgfusepath{fill}%
\end{pgfscope}%
\begin{pgfscope}%
\pgfpathrectangle{\pgfqpoint{1.150000in}{0.150000in}}{\pgfqpoint{5.700000in}{5.700000in}}%
\pgfusepath{clip}%
\pgfsetbuttcap%
\pgfsetroundjoin%
\definecolor{currentfill}{rgb}{0.182256,0.426184,0.557120}%
\pgfsetfillcolor{currentfill}%
\pgfsetfillopacity{0.800000}%
\pgfsetlinewidth{0.000000pt}%
\definecolor{currentstroke}{rgb}{0.000000,0.000000,0.000000}%
\pgfsetstrokecolor{currentstroke}%
\pgfsetdash{}{0pt}%
\pgfpathmoveto{\pgfqpoint{2.210828in}{2.449531in}}%
\pgfpathlineto{\pgfqpoint{2.225225in}{2.424883in}}%
\pgfpathlineto{\pgfqpoint{2.239606in}{2.400544in}}%
\pgfpathlineto{\pgfqpoint{2.253973in}{2.376511in}}%
\pgfpathlineto{\pgfqpoint{2.268326in}{2.352782in}}%
\pgfpathlineto{\pgfqpoint{2.277746in}{2.343441in}}%
\pgfpathlineto{\pgfqpoint{2.287138in}{2.334544in}}%
\pgfpathlineto{\pgfqpoint{2.296503in}{2.326085in}}%
\pgfpathlineto{\pgfqpoint{2.305841in}{2.318054in}}%
\pgfpathlineto{\pgfqpoint{2.291558in}{2.340990in}}%
\pgfpathlineto{\pgfqpoint{2.277262in}{2.364227in}}%
\pgfpathlineto{\pgfqpoint{2.262952in}{2.387768in}}%
\pgfpathlineto{\pgfqpoint{2.248627in}{2.411616in}}%
\pgfpathlineto{\pgfqpoint{2.239220in}{2.420428in}}%
\pgfpathlineto{\pgfqpoint{2.229785in}{2.429679in}}%
\pgfpathlineto{\pgfqpoint{2.220321in}{2.439377in}}%
\pgfpathlineto{\pgfqpoint{2.210828in}{2.449531in}}%
\pgfpathclose%
\pgfusepath{fill}%
\end{pgfscope}%
\begin{pgfscope}%
\pgfpathrectangle{\pgfqpoint{1.150000in}{0.150000in}}{\pgfqpoint{5.700000in}{5.700000in}}%
\pgfusepath{clip}%
\pgfsetbuttcap%
\pgfsetroundjoin%
\definecolor{currentfill}{rgb}{0.170948,0.694384,0.493803}%
\pgfsetfillcolor{currentfill}%
\pgfsetfillopacity{0.800000}%
\pgfsetlinewidth{0.000000pt}%
\definecolor{currentstroke}{rgb}{0.000000,0.000000,0.000000}%
\pgfsetstrokecolor{currentstroke}%
\pgfsetdash{}{0pt}%
\pgfpathmoveto{\pgfqpoint{5.385238in}{3.182882in}}%
\pgfpathlineto{\pgfqpoint{5.400113in}{3.199016in}}%
\pgfpathlineto{\pgfqpoint{5.415010in}{3.215338in}}%
\pgfpathlineto{\pgfqpoint{5.429930in}{3.231848in}}%
\pgfpathlineto{\pgfqpoint{5.444872in}{3.248546in}}%
\pgfpathlineto{\pgfqpoint{5.452423in}{3.253625in}}%
\pgfpathlineto{\pgfqpoint{5.459963in}{3.258547in}}%
\pgfpathlineto{\pgfqpoint{5.467493in}{3.263316in}}%
\pgfpathlineto{\pgfqpoint{5.475013in}{3.267934in}}%
\pgfpathlineto{\pgfqpoint{5.460085in}{3.251504in}}%
\pgfpathlineto{\pgfqpoint{5.445179in}{3.235261in}}%
\pgfpathlineto{\pgfqpoint{5.430296in}{3.219206in}}%
\pgfpathlineto{\pgfqpoint{5.415435in}{3.203337in}}%
\pgfpathlineto{\pgfqpoint{5.407901in}{3.198439in}}%
\pgfpathlineto{\pgfqpoint{5.400357in}{3.193399in}}%
\pgfpathlineto{\pgfqpoint{5.392802in}{3.188214in}}%
\pgfpathlineto{\pgfqpoint{5.385238in}{3.182882in}}%
\pgfpathclose%
\pgfusepath{fill}%
\end{pgfscope}%
\begin{pgfscope}%
\pgfpathrectangle{\pgfqpoint{1.150000in}{0.150000in}}{\pgfqpoint{5.700000in}{5.700000in}}%
\pgfusepath{clip}%
\pgfsetbuttcap%
\pgfsetroundjoin%
\definecolor{currentfill}{rgb}{0.319809,0.770914,0.411152}%
\pgfsetfillcolor{currentfill}%
\pgfsetfillopacity{0.800000}%
\pgfsetlinewidth{0.000000pt}%
\definecolor{currentstroke}{rgb}{0.000000,0.000000,0.000000}%
\pgfsetstrokecolor{currentstroke}%
\pgfsetdash{}{0pt}%
\pgfpathmoveto{\pgfqpoint{5.684272in}{3.442981in}}%
\pgfpathlineto{\pgfqpoint{5.699353in}{3.460079in}}%
\pgfpathlineto{\pgfqpoint{5.714459in}{3.477364in}}%
\pgfpathlineto{\pgfqpoint{5.729588in}{3.494837in}}%
\pgfpathlineto{\pgfqpoint{5.744741in}{3.512499in}}%
\pgfpathlineto{\pgfqpoint{5.752066in}{3.514304in}}%
\pgfpathlineto{\pgfqpoint{5.759379in}{3.515988in}}%
\pgfpathlineto{\pgfqpoint{5.766681in}{3.517555in}}%
\pgfpathlineto{\pgfqpoint{5.773973in}{3.519009in}}%
\pgfpathlineto{\pgfqpoint{5.758843in}{3.501763in}}%
\pgfpathlineto{\pgfqpoint{5.743737in}{3.484703in}}%
\pgfpathlineto{\pgfqpoint{5.728655in}{3.467831in}}%
\pgfpathlineto{\pgfqpoint{5.713596in}{3.451145in}}%
\pgfpathlineto{\pgfqpoint{5.706280in}{3.449264in}}%
\pgfpathlineto{\pgfqpoint{5.698954in}{3.447280in}}%
\pgfpathlineto{\pgfqpoint{5.691618in}{3.445187in}}%
\pgfpathlineto{\pgfqpoint{5.684272in}{3.442981in}}%
\pgfpathclose%
\pgfusepath{fill}%
\end{pgfscope}%
\begin{pgfscope}%
\pgfpathrectangle{\pgfqpoint{1.150000in}{0.150000in}}{\pgfqpoint{5.700000in}{5.700000in}}%
\pgfusepath{clip}%
\pgfsetbuttcap%
\pgfsetroundjoin%
\definecolor{currentfill}{rgb}{0.133743,0.548535,0.553541}%
\pgfsetfillcolor{currentfill}%
\pgfsetfillopacity{0.800000}%
\pgfsetlinewidth{0.000000pt}%
\definecolor{currentstroke}{rgb}{0.000000,0.000000,0.000000}%
\pgfsetstrokecolor{currentstroke}%
\pgfsetdash{}{0pt}%
\pgfpathmoveto{\pgfqpoint{2.017190in}{2.837380in}}%
\pgfpathlineto{\pgfqpoint{2.031793in}{2.808202in}}%
\pgfpathlineto{\pgfqpoint{2.046376in}{2.779383in}}%
\pgfpathlineto{\pgfqpoint{2.060939in}{2.750919in}}%
\pgfpathlineto{\pgfqpoint{2.075483in}{2.722806in}}%
\pgfpathlineto{\pgfqpoint{2.085083in}{2.712064in}}%
\pgfpathlineto{\pgfqpoint{2.094652in}{2.701777in}}%
\pgfpathlineto{\pgfqpoint{2.104192in}{2.691936in}}%
\pgfpathlineto{\pgfqpoint{2.113702in}{2.682534in}}%
\pgfpathlineto{\pgfqpoint{2.099235in}{2.709862in}}%
\pgfpathlineto{\pgfqpoint{2.084749in}{2.737539in}}%
\pgfpathlineto{\pgfqpoint{2.070244in}{2.765568in}}%
\pgfpathlineto{\pgfqpoint{2.055719in}{2.793953in}}%
\pgfpathlineto{\pgfqpoint{2.046133in}{2.804128in}}%
\pgfpathlineto{\pgfqpoint{2.036517in}{2.814752in}}%
\pgfpathlineto{\pgfqpoint{2.026869in}{2.825833in}}%
\pgfpathlineto{\pgfqpoint{2.017190in}{2.837380in}}%
\pgfpathclose%
\pgfusepath{fill}%
\end{pgfscope}%
\begin{pgfscope}%
\pgfpathrectangle{\pgfqpoint{1.150000in}{0.150000in}}{\pgfqpoint{5.700000in}{5.700000in}}%
\pgfusepath{clip}%
\pgfsetbuttcap%
\pgfsetroundjoin%
\definecolor{currentfill}{rgb}{0.282656,0.100196,0.422160}%
\pgfsetfillcolor{currentfill}%
\pgfsetfillopacity{0.800000}%
\pgfsetlinewidth{0.000000pt}%
\definecolor{currentstroke}{rgb}{0.000000,0.000000,0.000000}%
\pgfsetstrokecolor{currentstroke}%
\pgfsetdash{}{0pt}%
\pgfpathmoveto{\pgfqpoint{3.849510in}{1.517368in}}%
\pgfpathlineto{\pgfqpoint{3.863511in}{1.519728in}}%
\pgfpathlineto{\pgfqpoint{3.877521in}{1.522272in}}%
\pgfpathlineto{\pgfqpoint{3.891541in}{1.524999in}}%
\pgfpathlineto{\pgfqpoint{3.905570in}{1.527907in}}%
\pgfpathlineto{\pgfqpoint{3.913757in}{1.541214in}}%
\pgfpathlineto{\pgfqpoint{3.921938in}{1.554573in}}%
\pgfpathlineto{\pgfqpoint{3.930115in}{1.567978in}}%
\pgfpathlineto{\pgfqpoint{3.938287in}{1.581425in}}%
\pgfpathlineto{\pgfqpoint{3.924263in}{1.578010in}}%
\pgfpathlineto{\pgfqpoint{3.910249in}{1.574778in}}%
\pgfpathlineto{\pgfqpoint{3.896245in}{1.571728in}}%
\pgfpathlineto{\pgfqpoint{3.882251in}{1.568862in}}%
\pgfpathlineto{\pgfqpoint{3.874073in}{1.555909in}}%
\pgfpathlineto{\pgfqpoint{3.865891in}{1.543005in}}%
\pgfpathlineto{\pgfqpoint{3.857703in}{1.530157in}}%
\pgfpathlineto{\pgfqpoint{3.849510in}{1.517368in}}%
\pgfpathclose%
\pgfusepath{fill}%
\end{pgfscope}%
\begin{pgfscope}%
\pgfpathrectangle{\pgfqpoint{1.150000in}{0.150000in}}{\pgfqpoint{5.700000in}{5.700000in}}%
\pgfusepath{clip}%
\pgfsetbuttcap%
\pgfsetroundjoin%
\definecolor{currentfill}{rgb}{0.221989,0.339161,0.548752}%
\pgfsetfillcolor{currentfill}%
\pgfsetfillopacity{0.800000}%
\pgfsetlinewidth{0.000000pt}%
\definecolor{currentstroke}{rgb}{0.000000,0.000000,0.000000}%
\pgfsetstrokecolor{currentstroke}%
\pgfsetdash{}{0pt}%
\pgfpathmoveto{\pgfqpoint{4.391069in}{2.080670in}}%
\pgfpathlineto{\pgfqpoint{4.405303in}{2.089867in}}%
\pgfpathlineto{\pgfqpoint{4.419551in}{2.099248in}}%
\pgfpathlineto{\pgfqpoint{4.433815in}{2.108812in}}%
\pgfpathlineto{\pgfqpoint{4.448094in}{2.118560in}}%
\pgfpathlineto{\pgfqpoint{4.456135in}{2.132876in}}%
\pgfpathlineto{\pgfqpoint{4.464172in}{2.147090in}}%
\pgfpathlineto{\pgfqpoint{4.472203in}{2.161200in}}%
\pgfpathlineto{\pgfqpoint{4.480229in}{2.175204in}}%
\pgfpathlineto{\pgfqpoint{4.465948in}{2.165201in}}%
\pgfpathlineto{\pgfqpoint{4.451682in}{2.155381in}}%
\pgfpathlineto{\pgfqpoint{4.437431in}{2.145746in}}%
\pgfpathlineto{\pgfqpoint{4.423196in}{2.136294in}}%
\pgfpathlineto{\pgfqpoint{4.415171in}{2.122533in}}%
\pgfpathlineto{\pgfqpoint{4.407142in}{2.108674in}}%
\pgfpathlineto{\pgfqpoint{4.399108in}{2.094719in}}%
\pgfpathlineto{\pgfqpoint{4.391069in}{2.080670in}}%
\pgfpathclose%
\pgfusepath{fill}%
\end{pgfscope}%
\begin{pgfscope}%
\pgfpathrectangle{\pgfqpoint{1.150000in}{0.150000in}}{\pgfqpoint{5.700000in}{5.700000in}}%
\pgfusepath{clip}%
\pgfsetbuttcap%
\pgfsetroundjoin%
\definecolor{currentfill}{rgb}{0.122312,0.633153,0.530398}%
\pgfsetfillcolor{currentfill}%
\pgfsetfillopacity{0.800000}%
\pgfsetlinewidth{0.000000pt}%
\definecolor{currentstroke}{rgb}{0.000000,0.000000,0.000000}%
\pgfsetstrokecolor{currentstroke}%
\pgfsetdash{}{0pt}%
\pgfpathmoveto{\pgfqpoint{5.175231in}{2.979337in}}%
\pgfpathlineto{\pgfqpoint{5.189963in}{2.994556in}}%
\pgfpathlineto{\pgfqpoint{5.204716in}{3.009963in}}%
\pgfpathlineto{\pgfqpoint{5.219490in}{3.025557in}}%
\pgfpathlineto{\pgfqpoint{5.234286in}{3.041339in}}%
\pgfpathlineto{\pgfqpoint{5.241978in}{3.048803in}}%
\pgfpathlineto{\pgfqpoint{5.249661in}{3.056098in}}%
\pgfpathlineto{\pgfqpoint{5.257334in}{3.063225in}}%
\pgfpathlineto{\pgfqpoint{5.264997in}{3.070186in}}%
\pgfpathlineto{\pgfqpoint{5.250210in}{3.054564in}}%
\pgfpathlineto{\pgfqpoint{5.235443in}{3.039128in}}%
\pgfpathlineto{\pgfqpoint{5.220698in}{3.023880in}}%
\pgfpathlineto{\pgfqpoint{5.205974in}{3.008818in}}%
\pgfpathlineto{\pgfqpoint{5.198302in}{3.001685in}}%
\pgfpathlineto{\pgfqpoint{5.190621in}{2.994395in}}%
\pgfpathlineto{\pgfqpoint{5.182930in}{2.986946in}}%
\pgfpathlineto{\pgfqpoint{5.175231in}{2.979337in}}%
\pgfpathclose%
\pgfusepath{fill}%
\end{pgfscope}%
\begin{pgfscope}%
\pgfpathrectangle{\pgfqpoint{1.150000in}{0.150000in}}{\pgfqpoint{5.700000in}{5.700000in}}%
\pgfusepath{clip}%
\pgfsetbuttcap%
\pgfsetroundjoin%
\definecolor{currentfill}{rgb}{0.268510,0.009605,0.335427}%
\pgfsetfillcolor{currentfill}%
\pgfsetfillopacity{0.800000}%
\pgfsetlinewidth{0.000000pt}%
\definecolor{currentstroke}{rgb}{0.000000,0.000000,0.000000}%
\pgfsetstrokecolor{currentstroke}%
\pgfsetdash{}{0pt}%
\pgfpathmoveto{\pgfqpoint{3.493972in}{1.349662in}}%
\pgfpathlineto{\pgfqpoint{3.507909in}{1.346731in}}%
\pgfpathlineto{\pgfqpoint{3.521851in}{1.343987in}}%
\pgfpathlineto{\pgfqpoint{3.535798in}{1.341431in}}%
\pgfpathlineto{\pgfqpoint{3.549751in}{1.339062in}}%
\pgfpathlineto{\pgfqpoint{3.558082in}{1.348254in}}%
\pgfpathlineto{\pgfqpoint{3.566405in}{1.357618in}}%
\pgfpathlineto{\pgfqpoint{3.574720in}{1.367147in}}%
\pgfpathlineto{\pgfqpoint{3.583027in}{1.376835in}}%
\pgfpathlineto{\pgfqpoint{3.569092in}{1.378575in}}%
\pgfpathlineto{\pgfqpoint{3.555163in}{1.380503in}}%
\pgfpathlineto{\pgfqpoint{3.541239in}{1.382618in}}%
\pgfpathlineto{\pgfqpoint{3.527321in}{1.384921in}}%
\pgfpathlineto{\pgfqpoint{3.518996in}{1.375849in}}%
\pgfpathlineto{\pgfqpoint{3.510663in}{1.366945in}}%
\pgfpathlineto{\pgfqpoint{3.502322in}{1.358214in}}%
\pgfpathlineto{\pgfqpoint{3.493972in}{1.349662in}}%
\pgfpathclose%
\pgfusepath{fill}%
\end{pgfscope}%
\begin{pgfscope}%
\pgfpathrectangle{\pgfqpoint{1.150000in}{0.150000in}}{\pgfqpoint{5.700000in}{5.700000in}}%
\pgfusepath{clip}%
\pgfsetbuttcap%
\pgfsetroundjoin%
\definecolor{currentfill}{rgb}{0.246811,0.283237,0.535941}%
\pgfsetfillcolor{currentfill}%
\pgfsetfillopacity{0.800000}%
\pgfsetlinewidth{0.000000pt}%
\definecolor{currentstroke}{rgb}{0.000000,0.000000,0.000000}%
\pgfsetstrokecolor{currentstroke}%
\pgfsetdash{}{0pt}%
\pgfpathmoveto{\pgfqpoint{4.269812in}{1.932676in}}%
\pgfpathlineto{\pgfqpoint{4.283985in}{1.940533in}}%
\pgfpathlineto{\pgfqpoint{4.298172in}{1.948574in}}%
\pgfpathlineto{\pgfqpoint{4.312373in}{1.956797in}}%
\pgfpathlineto{\pgfqpoint{4.326588in}{1.965203in}}%
\pgfpathlineto{\pgfqpoint{4.334664in}{1.979911in}}%
\pgfpathlineto{\pgfqpoint{4.342736in}{1.994547in}}%
\pgfpathlineto{\pgfqpoint{4.350803in}{2.009106in}}%
\pgfpathlineto{\pgfqpoint{4.358865in}{2.023587in}}%
\pgfpathlineto{\pgfqpoint{4.344648in}{2.014861in}}%
\pgfpathlineto{\pgfqpoint{4.330446in}{2.006318in}}%
\pgfpathlineto{\pgfqpoint{4.316258in}{1.997958in}}%
\pgfpathlineto{\pgfqpoint{4.302084in}{1.989782in}}%
\pgfpathlineto{\pgfqpoint{4.294022in}{1.975608in}}%
\pgfpathlineto{\pgfqpoint{4.285957in}{1.961364in}}%
\pgfpathlineto{\pgfqpoint{4.277887in}{1.947053in}}%
\pgfpathlineto{\pgfqpoint{4.269812in}{1.932676in}}%
\pgfpathclose%
\pgfusepath{fill}%
\end{pgfscope}%
\begin{pgfscope}%
\pgfpathrectangle{\pgfqpoint{1.150000in}{0.150000in}}{\pgfqpoint{5.700000in}{5.700000in}}%
\pgfusepath{clip}%
\pgfsetbuttcap%
\pgfsetroundjoin%
\definecolor{currentfill}{rgb}{0.144759,0.519093,0.556572}%
\pgfsetfillcolor{currentfill}%
\pgfsetfillopacity{0.800000}%
\pgfsetlinewidth{0.000000pt}%
\definecolor{currentstroke}{rgb}{0.000000,0.000000,0.000000}%
\pgfsetstrokecolor{currentstroke}%
\pgfsetdash{}{0pt}%
\pgfpathmoveto{\pgfqpoint{4.843935in}{2.619904in}}%
\pgfpathlineto{\pgfqpoint{4.858448in}{2.633142in}}%
\pgfpathlineto{\pgfqpoint{4.872979in}{2.646567in}}%
\pgfpathlineto{\pgfqpoint{4.887529in}{2.660178in}}%
\pgfpathlineto{\pgfqpoint{4.902099in}{2.673975in}}%
\pgfpathlineto{\pgfqpoint{4.909974in}{2.685023in}}%
\pgfpathlineto{\pgfqpoint{4.917842in}{2.695905in}}%
\pgfpathlineto{\pgfqpoint{4.925703in}{2.706623in}}%
\pgfpathlineto{\pgfqpoint{4.933556in}{2.717177in}}%
\pgfpathlineto{\pgfqpoint{4.918987in}{2.703361in}}%
\pgfpathlineto{\pgfqpoint{4.904438in}{2.689732in}}%
\pgfpathlineto{\pgfqpoint{4.889908in}{2.676288in}}%
\pgfpathlineto{\pgfqpoint{4.875397in}{2.663031in}}%
\pgfpathlineto{\pgfqpoint{4.867543in}{2.652483in}}%
\pgfpathlineto{\pgfqpoint{4.859681in}{2.641779in}}%
\pgfpathlineto{\pgfqpoint{4.851812in}{2.630920in}}%
\pgfpathlineto{\pgfqpoint{4.843935in}{2.619904in}}%
\pgfpathclose%
\pgfusepath{fill}%
\end{pgfscope}%
\begin{pgfscope}%
\pgfpathrectangle{\pgfqpoint{1.150000in}{0.150000in}}{\pgfqpoint{5.700000in}{5.700000in}}%
\pgfusepath{clip}%
\pgfsetbuttcap%
\pgfsetroundjoin%
\definecolor{currentfill}{rgb}{0.276022,0.044167,0.370164}%
\pgfsetfillcolor{currentfill}%
\pgfsetfillopacity{0.800000}%
\pgfsetlinewidth{0.000000pt}%
\definecolor{currentstroke}{rgb}{0.000000,0.000000,0.000000}%
\pgfsetstrokecolor{currentstroke}%
\pgfsetdash{}{0pt}%
\pgfpathmoveto{\pgfqpoint{3.057606in}{1.441708in}}%
\pgfpathlineto{\pgfqpoint{3.071574in}{1.431987in}}%
\pgfpathlineto{\pgfqpoint{3.085540in}{1.422470in}}%
\pgfpathlineto{\pgfqpoint{3.099507in}{1.413158in}}%
\pgfpathlineto{\pgfqpoint{3.113473in}{1.404050in}}%
\pgfpathlineto{\pgfqpoint{3.122095in}{1.406233in}}%
\pgfpathlineto{\pgfqpoint{3.130704in}{1.408719in}}%
\pgfpathlineto{\pgfqpoint{3.139298in}{1.411501in}}%
\pgfpathlineto{\pgfqpoint{3.147878in}{1.414571in}}%
\pgfpathlineto{\pgfqpoint{3.133948in}{1.422949in}}%
\pgfpathlineto{\pgfqpoint{3.120018in}{1.431530in}}%
\pgfpathlineto{\pgfqpoint{3.106088in}{1.440315in}}%
\pgfpathlineto{\pgfqpoint{3.092158in}{1.449305in}}%
\pgfpathlineto{\pgfqpoint{3.083542in}{1.446953in}}%
\pgfpathlineto{\pgfqpoint{3.074912in}{1.444898in}}%
\pgfpathlineto{\pgfqpoint{3.066267in}{1.443147in}}%
\pgfpathlineto{\pgfqpoint{3.057606in}{1.441708in}}%
\pgfpathclose%
\pgfusepath{fill}%
\end{pgfscope}%
\begin{pgfscope}%
\pgfpathrectangle{\pgfqpoint{1.150000in}{0.150000in}}{\pgfqpoint{5.700000in}{5.700000in}}%
\pgfusepath{clip}%
\pgfsetbuttcap%
\pgfsetroundjoin%
\definecolor{currentfill}{rgb}{0.282884,0.135920,0.453427}%
\pgfsetfillcolor{currentfill}%
\pgfsetfillopacity{0.800000}%
\pgfsetlinewidth{0.000000pt}%
\definecolor{currentstroke}{rgb}{0.000000,0.000000,0.000000}%
\pgfsetstrokecolor{currentstroke}%
\pgfsetdash{}{0pt}%
\pgfpathmoveto{\pgfqpoint{3.938287in}{1.581425in}}%
\pgfpathlineto{\pgfqpoint{3.952321in}{1.585023in}}%
\pgfpathlineto{\pgfqpoint{3.966366in}{1.588804in}}%
\pgfpathlineto{\pgfqpoint{3.980421in}{1.592766in}}%
\pgfpathlineto{\pgfqpoint{3.994487in}{1.596910in}}%
\pgfpathlineto{\pgfqpoint{4.002650in}{1.610883in}}%
\pgfpathlineto{\pgfqpoint{4.010809in}{1.624880in}}%
\pgfpathlineto{\pgfqpoint{4.018963in}{1.638898in}}%
\pgfpathlineto{\pgfqpoint{4.027113in}{1.652933in}}%
\pgfpathlineto{\pgfqpoint{4.013050in}{1.648312in}}%
\pgfpathlineto{\pgfqpoint{3.998998in}{1.643873in}}%
\pgfpathlineto{\pgfqpoint{3.984957in}{1.639617in}}%
\pgfpathlineto{\pgfqpoint{3.970927in}{1.635544in}}%
\pgfpathlineto{\pgfqpoint{3.962774in}{1.621974in}}%
\pgfpathlineto{\pgfqpoint{3.954616in}{1.608428in}}%
\pgfpathlineto{\pgfqpoint{3.946454in}{1.594910in}}%
\pgfpathlineto{\pgfqpoint{3.938287in}{1.581425in}}%
\pgfpathclose%
\pgfusepath{fill}%
\end{pgfscope}%
\begin{pgfscope}%
\pgfpathrectangle{\pgfqpoint{1.150000in}{0.150000in}}{\pgfqpoint{5.700000in}{5.700000in}}%
\pgfusepath{clip}%
\pgfsetbuttcap%
\pgfsetroundjoin%
\definecolor{currentfill}{rgb}{0.377779,0.791781,0.377939}%
\pgfsetfillcolor{currentfill}%
\pgfsetfillopacity{0.800000}%
\pgfsetlinewidth{0.000000pt}%
\definecolor{currentstroke}{rgb}{0.000000,0.000000,0.000000}%
\pgfsetstrokecolor{currentstroke}%
\pgfsetdash{}{0pt}%
\pgfpathmoveto{\pgfqpoint{5.773973in}{3.519009in}}%
\pgfpathlineto{\pgfqpoint{5.789127in}{3.536444in}}%
\pgfpathlineto{\pgfqpoint{5.804306in}{3.554066in}}%
\pgfpathlineto{\pgfqpoint{5.819509in}{3.571876in}}%
\pgfpathlineto{\pgfqpoint{5.834736in}{3.589875in}}%
\pgfpathlineto{\pgfqpoint{5.841992in}{3.590783in}}%
\pgfpathlineto{\pgfqpoint{5.849237in}{3.591581in}}%
\pgfpathlineto{\pgfqpoint{5.856471in}{3.592271in}}%
\pgfpathlineto{\pgfqpoint{5.863694in}{3.592861in}}%
\pgfpathlineto{\pgfqpoint{5.848493in}{3.575315in}}%
\pgfpathlineto{\pgfqpoint{5.833316in}{3.557956in}}%
\pgfpathlineto{\pgfqpoint{5.818163in}{3.540784in}}%
\pgfpathlineto{\pgfqpoint{5.803035in}{3.523799in}}%
\pgfpathlineto{\pgfqpoint{5.795785in}{3.522746in}}%
\pgfpathlineto{\pgfqpoint{5.788525in}{3.521600in}}%
\pgfpathlineto{\pgfqpoint{5.781254in}{3.520356in}}%
\pgfpathlineto{\pgfqpoint{5.773973in}{3.519009in}}%
\pgfpathclose%
\pgfusepath{fill}%
\end{pgfscope}%
\begin{pgfscope}%
\pgfpathrectangle{\pgfqpoint{1.150000in}{0.150000in}}{\pgfqpoint{5.700000in}{5.700000in}}%
\pgfusepath{clip}%
\pgfsetbuttcap%
\pgfsetroundjoin%
\definecolor{currentfill}{rgb}{0.268510,0.009605,0.335427}%
\pgfsetfillcolor{currentfill}%
\pgfsetfillopacity{0.800000}%
\pgfsetlinewidth{0.000000pt}%
\definecolor{currentstroke}{rgb}{0.000000,0.000000,0.000000}%
\pgfsetstrokecolor{currentstroke}%
\pgfsetdash{}{0pt}%
\pgfpathmoveto{\pgfqpoint{3.259360in}{1.354763in}}%
\pgfpathlineto{\pgfqpoint{3.273302in}{1.348178in}}%
\pgfpathlineto{\pgfqpoint{3.287247in}{1.341787in}}%
\pgfpathlineto{\pgfqpoint{3.301194in}{1.335591in}}%
\pgfpathlineto{\pgfqpoint{3.315144in}{1.329590in}}%
\pgfpathlineto{\pgfqpoint{3.323619in}{1.335062in}}%
\pgfpathlineto{\pgfqpoint{3.332083in}{1.340782in}}%
\pgfpathlineto{\pgfqpoint{3.340535in}{1.346745in}}%
\pgfpathlineto{\pgfqpoint{3.348978in}{1.352942in}}%
\pgfpathlineto{\pgfqpoint{3.335055in}{1.358250in}}%
\pgfpathlineto{\pgfqpoint{3.321136in}{1.363753in}}%
\pgfpathlineto{\pgfqpoint{3.307219in}{1.369449in}}%
\pgfpathlineto{\pgfqpoint{3.293305in}{1.375341in}}%
\pgfpathlineto{\pgfqpoint{3.284836in}{1.369825in}}%
\pgfpathlineto{\pgfqpoint{3.276356in}{1.364552in}}%
\pgfpathlineto{\pgfqpoint{3.267864in}{1.359529in}}%
\pgfpathlineto{\pgfqpoint{3.259360in}{1.354763in}}%
\pgfpathclose%
\pgfusepath{fill}%
\end{pgfscope}%
\begin{pgfscope}%
\pgfpathrectangle{\pgfqpoint{1.150000in}{0.150000in}}{\pgfqpoint{5.700000in}{5.700000in}}%
\pgfusepath{clip}%
\pgfsetbuttcap%
\pgfsetroundjoin%
\definecolor{currentfill}{rgb}{0.273006,0.204520,0.501721}%
\pgfsetfillcolor{currentfill}%
\pgfsetfillopacity{0.800000}%
\pgfsetlinewidth{0.000000pt}%
\definecolor{currentstroke}{rgb}{0.000000,0.000000,0.000000}%
\pgfsetstrokecolor{currentstroke}%
\pgfsetdash{}{0pt}%
\pgfpathmoveto{\pgfqpoint{2.629400in}{1.821016in}}%
\pgfpathlineto{\pgfqpoint{2.643517in}{1.804291in}}%
\pgfpathlineto{\pgfqpoint{2.657626in}{1.787807in}}%
\pgfpathlineto{\pgfqpoint{2.671729in}{1.771563in}}%
\pgfpathlineto{\pgfqpoint{2.685826in}{1.755557in}}%
\pgfpathlineto{\pgfqpoint{2.694841in}{1.750936in}}%
\pgfpathlineto{\pgfqpoint{2.703834in}{1.746714in}}%
\pgfpathlineto{\pgfqpoint{2.712807in}{1.742881in}}%
\pgfpathlineto{\pgfqpoint{2.721758in}{1.739431in}}%
\pgfpathlineto{\pgfqpoint{2.707715in}{1.754650in}}%
\pgfpathlineto{\pgfqpoint{2.693667in}{1.770106in}}%
\pgfpathlineto{\pgfqpoint{2.679613in}{1.785801in}}%
\pgfpathlineto{\pgfqpoint{2.665553in}{1.801735in}}%
\pgfpathlineto{\pgfqpoint{2.656548in}{1.805960in}}%
\pgfpathlineto{\pgfqpoint{2.647521in}{1.810576in}}%
\pgfpathlineto{\pgfqpoint{2.638472in}{1.815592in}}%
\pgfpathlineto{\pgfqpoint{2.629400in}{1.821016in}}%
\pgfpathclose%
\pgfusepath{fill}%
\end{pgfscope}%
\begin{pgfscope}%
\pgfpathrectangle{\pgfqpoint{1.150000in}{0.150000in}}{\pgfqpoint{5.700000in}{5.700000in}}%
\pgfusepath{clip}%
\pgfsetbuttcap%
\pgfsetroundjoin%
\definecolor{currentfill}{rgb}{0.265145,0.232956,0.516599}%
\pgfsetfillcolor{currentfill}%
\pgfsetfillopacity{0.800000}%
\pgfsetlinewidth{0.000000pt}%
\definecolor{currentstroke}{rgb}{0.000000,0.000000,0.000000}%
\pgfsetstrokecolor{currentstroke}%
\pgfsetdash{}{0pt}%
\pgfpathmoveto{\pgfqpoint{2.572862in}{1.890361in}}%
\pgfpathlineto{\pgfqpoint{2.587008in}{1.872654in}}%
\pgfpathlineto{\pgfqpoint{2.601146in}{1.855196in}}%
\pgfpathlineto{\pgfqpoint{2.615277in}{1.837983in}}%
\pgfpathlineto{\pgfqpoint{2.629400in}{1.821016in}}%
\pgfpathlineto{\pgfqpoint{2.638472in}{1.815592in}}%
\pgfpathlineto{\pgfqpoint{2.647521in}{1.810576in}}%
\pgfpathlineto{\pgfqpoint{2.656548in}{1.805960in}}%
\pgfpathlineto{\pgfqpoint{2.665553in}{1.801735in}}%
\pgfpathlineto{\pgfqpoint{2.651486in}{1.817912in}}%
\pgfpathlineto{\pgfqpoint{2.637413in}{1.834332in}}%
\pgfpathlineto{\pgfqpoint{2.623332in}{1.850996in}}%
\pgfpathlineto{\pgfqpoint{2.609245in}{1.867908in}}%
\pgfpathlineto{\pgfqpoint{2.600184in}{1.872911in}}%
\pgfpathlineto{\pgfqpoint{2.591100in}{1.878315in}}%
\pgfpathlineto{\pgfqpoint{2.581993in}{1.884129in}}%
\pgfpathlineto{\pgfqpoint{2.572862in}{1.890361in}}%
\pgfpathclose%
\pgfusepath{fill}%
\end{pgfscope}%
\begin{pgfscope}%
\pgfpathrectangle{\pgfqpoint{1.150000in}{0.150000in}}{\pgfqpoint{5.700000in}{5.700000in}}%
\pgfusepath{clip}%
\pgfsetbuttcap%
\pgfsetroundjoin%
\definecolor{currentfill}{rgb}{0.266580,0.228262,0.514349}%
\pgfsetfillcolor{currentfill}%
\pgfsetfillopacity{0.800000}%
\pgfsetlinewidth{0.000000pt}%
\definecolor{currentstroke}{rgb}{0.000000,0.000000,0.000000}%
\pgfsetstrokecolor{currentstroke}%
\pgfsetdash{}{0pt}%
\pgfpathmoveto{\pgfqpoint{4.148505in}{1.789029in}}%
\pgfpathlineto{\pgfqpoint{4.162624in}{1.795421in}}%
\pgfpathlineto{\pgfqpoint{4.176755in}{1.801995in}}%
\pgfpathlineto{\pgfqpoint{4.190900in}{1.808752in}}%
\pgfpathlineto{\pgfqpoint{4.205057in}{1.815691in}}%
\pgfpathlineto{\pgfqpoint{4.213167in}{1.830476in}}%
\pgfpathlineto{\pgfqpoint{4.221272in}{1.845221in}}%
\pgfpathlineto{\pgfqpoint{4.229373in}{1.859923in}}%
\pgfpathlineto{\pgfqpoint{4.237470in}{1.874579in}}%
\pgfpathlineto{\pgfqpoint{4.223311in}{1.867256in}}%
\pgfpathlineto{\pgfqpoint{4.209166in}{1.860116in}}%
\pgfpathlineto{\pgfqpoint{4.195034in}{1.853159in}}%
\pgfpathlineto{\pgfqpoint{4.180915in}{1.846384in}}%
\pgfpathlineto{\pgfqpoint{4.172819in}{1.832099in}}%
\pgfpathlineto{\pgfqpoint{4.164719in}{1.817776in}}%
\pgfpathlineto{\pgfqpoint{4.156614in}{1.803418in}}%
\pgfpathlineto{\pgfqpoint{4.148505in}{1.789029in}}%
\pgfpathclose%
\pgfusepath{fill}%
\end{pgfscope}%
\begin{pgfscope}%
\pgfpathrectangle{\pgfqpoint{1.150000in}{0.150000in}}{\pgfqpoint{5.700000in}{5.700000in}}%
\pgfusepath{clip}%
\pgfsetbuttcap%
\pgfsetroundjoin%
\definecolor{currentfill}{rgb}{0.168126,0.459988,0.558082}%
\pgfsetfillcolor{currentfill}%
\pgfsetfillopacity{0.800000}%
\pgfsetlinewidth{0.000000pt}%
\definecolor{currentstroke}{rgb}{0.000000,0.000000,0.000000}%
\pgfsetstrokecolor{currentstroke}%
\pgfsetdash{}{0pt}%
\pgfpathmoveto{\pgfqpoint{2.153084in}{2.551269in}}%
\pgfpathlineto{\pgfqpoint{2.167544in}{2.525357in}}%
\pgfpathlineto{\pgfqpoint{2.181988in}{2.499765in}}%
\pgfpathlineto{\pgfqpoint{2.196416in}{2.474491in}}%
\pgfpathlineto{\pgfqpoint{2.210828in}{2.449531in}}%
\pgfpathlineto{\pgfqpoint{2.220321in}{2.439377in}}%
\pgfpathlineto{\pgfqpoint{2.229785in}{2.429679in}}%
\pgfpathlineto{\pgfqpoint{2.239220in}{2.420428in}}%
\pgfpathlineto{\pgfqpoint{2.248627in}{2.411616in}}%
\pgfpathlineto{\pgfqpoint{2.234287in}{2.435774in}}%
\pgfpathlineto{\pgfqpoint{2.219933in}{2.460244in}}%
\pgfpathlineto{\pgfqpoint{2.205563in}{2.485030in}}%
\pgfpathlineto{\pgfqpoint{2.191177in}{2.510134in}}%
\pgfpathlineto{\pgfqpoint{2.181698in}{2.519735in}}%
\pgfpathlineto{\pgfqpoint{2.172190in}{2.529786in}}%
\pgfpathlineto{\pgfqpoint{2.162652in}{2.540294in}}%
\pgfpathlineto{\pgfqpoint{2.153084in}{2.551269in}}%
\pgfpathclose%
\pgfusepath{fill}%
\end{pgfscope}%
\begin{pgfscope}%
\pgfpathrectangle{\pgfqpoint{1.150000in}{0.150000in}}{\pgfqpoint{5.700000in}{5.700000in}}%
\pgfusepath{clip}%
\pgfsetbuttcap%
\pgfsetroundjoin%
\definecolor{currentfill}{rgb}{0.278012,0.180367,0.486697}%
\pgfsetfillcolor{currentfill}%
\pgfsetfillopacity{0.800000}%
\pgfsetlinewidth{0.000000pt}%
\definecolor{currentstroke}{rgb}{0.000000,0.000000,0.000000}%
\pgfsetstrokecolor{currentstroke}%
\pgfsetdash{}{0pt}%
\pgfpathmoveto{\pgfqpoint{2.685826in}{1.755557in}}%
\pgfpathlineto{\pgfqpoint{2.699917in}{1.739788in}}%
\pgfpathlineto{\pgfqpoint{2.714001in}{1.724254in}}%
\pgfpathlineto{\pgfqpoint{2.728080in}{1.708953in}}%
\pgfpathlineto{\pgfqpoint{2.742154in}{1.693885in}}%
\pgfpathlineto{\pgfqpoint{2.751115in}{1.690062in}}%
\pgfpathlineto{\pgfqpoint{2.760055in}{1.686628in}}%
\pgfpathlineto{\pgfqpoint{2.768974in}{1.683575in}}%
\pgfpathlineto{\pgfqpoint{2.777874in}{1.680894in}}%
\pgfpathlineto{\pgfqpoint{2.763853in}{1.695180in}}%
\pgfpathlineto{\pgfqpoint{2.749826in}{1.709697in}}%
\pgfpathlineto{\pgfqpoint{2.735795in}{1.724447in}}%
\pgfpathlineto{\pgfqpoint{2.721758in}{1.739431in}}%
\pgfpathlineto{\pgfqpoint{2.712807in}{1.742881in}}%
\pgfpathlineto{\pgfqpoint{2.703834in}{1.746714in}}%
\pgfpathlineto{\pgfqpoint{2.694841in}{1.750936in}}%
\pgfpathlineto{\pgfqpoint{2.685826in}{1.755557in}}%
\pgfpathclose%
\pgfusepath{fill}%
\end{pgfscope}%
\begin{pgfscope}%
\pgfpathrectangle{\pgfqpoint{1.150000in}{0.150000in}}{\pgfqpoint{5.700000in}{5.700000in}}%
\pgfusepath{clip}%
\pgfsetbuttcap%
\pgfsetroundjoin%
\definecolor{currentfill}{rgb}{0.267004,0.004874,0.329415}%
\pgfsetfillcolor{currentfill}%
\pgfsetfillopacity{0.800000}%
\pgfsetlinewidth{0.000000pt}%
\definecolor{currentstroke}{rgb}{0.000000,0.000000,0.000000}%
\pgfsetstrokecolor{currentstroke}%
\pgfsetdash{}{0pt}%
\pgfpathmoveto{\pgfqpoint{3.404701in}{1.333633in}}%
\pgfpathlineto{\pgfqpoint{3.418642in}{1.329284in}}%
\pgfpathlineto{\pgfqpoint{3.432586in}{1.325125in}}%
\pgfpathlineto{\pgfqpoint{3.446534in}{1.321156in}}%
\pgfpathlineto{\pgfqpoint{3.460487in}{1.317376in}}%
\pgfpathlineto{\pgfqpoint{3.468872in}{1.325146in}}%
\pgfpathlineto{\pgfqpoint{3.477248in}{1.333122in}}%
\pgfpathlineto{\pgfqpoint{3.485614in}{1.341296in}}%
\pgfpathlineto{\pgfqpoint{3.493972in}{1.349662in}}%
\pgfpathlineto{\pgfqpoint{3.480041in}{1.352782in}}%
\pgfpathlineto{\pgfqpoint{3.466114in}{1.356091in}}%
\pgfpathlineto{\pgfqpoint{3.452192in}{1.359589in}}%
\pgfpathlineto{\pgfqpoint{3.438274in}{1.363278in}}%
\pgfpathlineto{\pgfqpoint{3.429895in}{1.355560in}}%
\pgfpathlineto{\pgfqpoint{3.421507in}{1.348042in}}%
\pgfpathlineto{\pgfqpoint{3.413109in}{1.340731in}}%
\pgfpathlineto{\pgfqpoint{3.404701in}{1.333633in}}%
\pgfpathclose%
\pgfusepath{fill}%
\end{pgfscope}%
\begin{pgfscope}%
\pgfpathrectangle{\pgfqpoint{1.150000in}{0.150000in}}{\pgfqpoint{5.700000in}{5.700000in}}%
\pgfusepath{clip}%
\pgfsetbuttcap%
\pgfsetroundjoin%
\definecolor{currentfill}{rgb}{0.430983,0.808473,0.346476}%
\pgfsetfillcolor{currentfill}%
\pgfsetfillopacity{0.800000}%
\pgfsetlinewidth{0.000000pt}%
\definecolor{currentstroke}{rgb}{0.000000,0.000000,0.000000}%
\pgfsetstrokecolor{currentstroke}%
\pgfsetdash{}{0pt}%
\pgfpathmoveto{\pgfqpoint{5.863694in}{3.592861in}}%
\pgfpathlineto{\pgfqpoint{5.878920in}{3.610595in}}%
\pgfpathlineto{\pgfqpoint{5.894171in}{3.628517in}}%
\pgfpathlineto{\pgfqpoint{5.909447in}{3.646628in}}%
\pgfpathlineto{\pgfqpoint{5.916638in}{3.646764in}}%
\pgfpathlineto{\pgfqpoint{5.923819in}{3.646804in}}%
\pgfpathlineto{\pgfqpoint{5.930989in}{3.646752in}}%
\pgfpathlineto{\pgfqpoint{5.938149in}{3.646613in}}%
\pgfpathlineto{\pgfqpoint{5.922902in}{3.628992in}}%
\pgfpathlineto{\pgfqpoint{5.907680in}{3.611558in}}%
\pgfpathlineto{\pgfqpoint{5.892483in}{3.594311in}}%
\pgfpathlineto{\pgfqpoint{5.885301in}{3.594074in}}%
\pgfpathlineto{\pgfqpoint{5.878109in}{3.593757in}}%
\pgfpathlineto{\pgfqpoint{5.870907in}{3.593355in}}%
\pgfpathlineto{\pgfqpoint{5.863694in}{3.592861in}}%
\pgfpathclose%
\pgfusepath{fill}%
\end{pgfscope}%
\begin{pgfscope}%
\pgfpathrectangle{\pgfqpoint{1.150000in}{0.150000in}}{\pgfqpoint{5.700000in}{5.700000in}}%
\pgfusepath{clip}%
\pgfsetbuttcap%
\pgfsetroundjoin%
\definecolor{currentfill}{rgb}{0.255645,0.260703,0.528312}%
\pgfsetfillcolor{currentfill}%
\pgfsetfillopacity{0.800000}%
\pgfsetlinewidth{0.000000pt}%
\definecolor{currentstroke}{rgb}{0.000000,0.000000,0.000000}%
\pgfsetstrokecolor{currentstroke}%
\pgfsetdash{}{0pt}%
\pgfpathmoveto{\pgfqpoint{2.516195in}{1.963701in}}%
\pgfpathlineto{\pgfqpoint{2.530375in}{1.944985in}}%
\pgfpathlineto{\pgfqpoint{2.544546in}{1.926524in}}%
\pgfpathlineto{\pgfqpoint{2.558708in}{1.908317in}}%
\pgfpathlineto{\pgfqpoint{2.572862in}{1.890361in}}%
\pgfpathlineto{\pgfqpoint{2.581993in}{1.884129in}}%
\pgfpathlineto{\pgfqpoint{2.591100in}{1.878315in}}%
\pgfpathlineto{\pgfqpoint{2.600184in}{1.872911in}}%
\pgfpathlineto{\pgfqpoint{2.609245in}{1.867908in}}%
\pgfpathlineto{\pgfqpoint{2.595150in}{1.885068in}}%
\pgfpathlineto{\pgfqpoint{2.581047in}{1.902478in}}%
\pgfpathlineto{\pgfqpoint{2.566937in}{1.920140in}}%
\pgfpathlineto{\pgfqpoint{2.552818in}{1.938056in}}%
\pgfpathlineto{\pgfqpoint{2.543698in}{1.943843in}}%
\pgfpathlineto{\pgfqpoint{2.534555in}{1.950040in}}%
\pgfpathlineto{\pgfqpoint{2.525388in}{1.956657in}}%
\pgfpathlineto{\pgfqpoint{2.516195in}{1.963701in}}%
\pgfpathclose%
\pgfusepath{fill}%
\end{pgfscope}%
\begin{pgfscope}%
\pgfpathrectangle{\pgfqpoint{1.150000in}{0.150000in}}{\pgfqpoint{5.700000in}{5.700000in}}%
\pgfusepath{clip}%
\pgfsetbuttcap%
\pgfsetroundjoin%
\definecolor{currentfill}{rgb}{0.162142,0.474838,0.558140}%
\pgfsetfillcolor{currentfill}%
\pgfsetfillopacity{0.800000}%
\pgfsetlinewidth{0.000000pt}%
\definecolor{currentstroke}{rgb}{0.000000,0.000000,0.000000}%
\pgfsetstrokecolor{currentstroke}%
\pgfsetdash{}{0pt}%
\pgfpathmoveto{\pgfqpoint{4.722803in}{2.475625in}}%
\pgfpathlineto{\pgfqpoint{4.737242in}{2.487980in}}%
\pgfpathlineto{\pgfqpoint{4.751699in}{2.500520in}}%
\pgfpathlineto{\pgfqpoint{4.766174in}{2.513245in}}%
\pgfpathlineto{\pgfqpoint{4.780668in}{2.526157in}}%
\pgfpathlineto{\pgfqpoint{4.788600in}{2.538420in}}%
\pgfpathlineto{\pgfqpoint{4.796526in}{2.550528in}}%
\pgfpathlineto{\pgfqpoint{4.804445in}{2.562481in}}%
\pgfpathlineto{\pgfqpoint{4.812357in}{2.574278in}}%
\pgfpathlineto{\pgfqpoint{4.797863in}{2.561278in}}%
\pgfpathlineto{\pgfqpoint{4.783387in}{2.548465in}}%
\pgfpathlineto{\pgfqpoint{4.768930in}{2.535837in}}%
\pgfpathlineto{\pgfqpoint{4.754490in}{2.523394in}}%
\pgfpathlineto{\pgfqpoint{4.746578in}{2.511672in}}%
\pgfpathlineto{\pgfqpoint{4.738659in}{2.499803in}}%
\pgfpathlineto{\pgfqpoint{4.730734in}{2.487787in}}%
\pgfpathlineto{\pgfqpoint{4.722803in}{2.475625in}}%
\pgfpathclose%
\pgfusepath{fill}%
\end{pgfscope}%
\begin{pgfscope}%
\pgfpathrectangle{\pgfqpoint{1.150000in}{0.150000in}}{\pgfqpoint{5.700000in}{5.700000in}}%
\pgfusepath{clip}%
\pgfsetbuttcap%
\pgfsetroundjoin%
\definecolor{currentfill}{rgb}{0.220124,0.725509,0.466226}%
\pgfsetfillcolor{currentfill}%
\pgfsetfillopacity{0.800000}%
\pgfsetlinewidth{0.000000pt}%
\definecolor{currentstroke}{rgb}{0.000000,0.000000,0.000000}%
\pgfsetstrokecolor{currentstroke}%
\pgfsetdash{}{0pt}%
\pgfpathmoveto{\pgfqpoint{5.475013in}{3.267934in}}%
\pgfpathlineto{\pgfqpoint{5.489964in}{3.284552in}}%
\pgfpathlineto{\pgfqpoint{5.504937in}{3.301357in}}%
\pgfpathlineto{\pgfqpoint{5.519934in}{3.318351in}}%
\pgfpathlineto{\pgfqpoint{5.534953in}{3.335533in}}%
\pgfpathlineto{\pgfqpoint{5.542447in}{3.339714in}}%
\pgfpathlineto{\pgfqpoint{5.549930in}{3.343742in}}%
\pgfpathlineto{\pgfqpoint{5.557403in}{3.347621in}}%
\pgfpathlineto{\pgfqpoint{5.564865in}{3.351355in}}%
\pgfpathlineto{\pgfqpoint{5.549861in}{3.334478in}}%
\pgfpathlineto{\pgfqpoint{5.534881in}{3.317789in}}%
\pgfpathlineto{\pgfqpoint{5.519923in}{3.301287in}}%
\pgfpathlineto{\pgfqpoint{5.504988in}{3.284973in}}%
\pgfpathlineto{\pgfqpoint{5.497510in}{3.280922in}}%
\pgfpathlineto{\pgfqpoint{5.490021in}{3.276734in}}%
\pgfpathlineto{\pgfqpoint{5.482522in}{3.272406in}}%
\pgfpathlineto{\pgfqpoint{5.475013in}{3.267934in}}%
\pgfpathclose%
\pgfusepath{fill}%
\end{pgfscope}%
\begin{pgfscope}%
\pgfpathrectangle{\pgfqpoint{1.150000in}{0.150000in}}{\pgfqpoint{5.700000in}{5.700000in}}%
\pgfusepath{clip}%
\pgfsetbuttcap%
\pgfsetroundjoin%
\definecolor{currentfill}{rgb}{0.281412,0.155834,0.469201}%
\pgfsetfillcolor{currentfill}%
\pgfsetfillopacity{0.800000}%
\pgfsetlinewidth{0.000000pt}%
\definecolor{currentstroke}{rgb}{0.000000,0.000000,0.000000}%
\pgfsetstrokecolor{currentstroke}%
\pgfsetdash{}{0pt}%
\pgfpathmoveto{\pgfqpoint{2.742154in}{1.693885in}}%
\pgfpathlineto{\pgfqpoint{2.756222in}{1.679047in}}%
\pgfpathlineto{\pgfqpoint{2.770286in}{1.664438in}}%
\pgfpathlineto{\pgfqpoint{2.784344in}{1.650057in}}%
\pgfpathlineto{\pgfqpoint{2.798398in}{1.635903in}}%
\pgfpathlineto{\pgfqpoint{2.807307in}{1.632874in}}%
\pgfpathlineto{\pgfqpoint{2.816196in}{1.630225in}}%
\pgfpathlineto{\pgfqpoint{2.825066in}{1.627947in}}%
\pgfpathlineto{\pgfqpoint{2.833916in}{1.626033in}}%
\pgfpathlineto{\pgfqpoint{2.819912in}{1.639408in}}%
\pgfpathlineto{\pgfqpoint{2.805904in}{1.653009in}}%
\pgfpathlineto{\pgfqpoint{2.791891in}{1.666838in}}%
\pgfpathlineto{\pgfqpoint{2.777874in}{1.680894in}}%
\pgfpathlineto{\pgfqpoint{2.768974in}{1.683575in}}%
\pgfpathlineto{\pgfqpoint{2.760055in}{1.686628in}}%
\pgfpathlineto{\pgfqpoint{2.751115in}{1.690062in}}%
\pgfpathlineto{\pgfqpoint{2.742154in}{1.693885in}}%
\pgfpathclose%
\pgfusepath{fill}%
\end{pgfscope}%
\begin{pgfscope}%
\pgfpathrectangle{\pgfqpoint{1.150000in}{0.150000in}}{\pgfqpoint{5.700000in}{5.700000in}}%
\pgfusepath{clip}%
\pgfsetbuttcap%
\pgfsetroundjoin%
\definecolor{currentfill}{rgb}{0.120565,0.596422,0.543611}%
\pgfsetfillcolor{currentfill}%
\pgfsetfillopacity{0.800000}%
\pgfsetlinewidth{0.000000pt}%
\definecolor{currentstroke}{rgb}{0.000000,0.000000,0.000000}%
\pgfsetstrokecolor{currentstroke}%
\pgfsetdash{}{0pt}%
\pgfpathmoveto{\pgfqpoint{5.054538in}{2.852617in}}%
\pgfpathlineto{\pgfqpoint{5.069200in}{2.867298in}}%
\pgfpathlineto{\pgfqpoint{5.083881in}{2.882167in}}%
\pgfpathlineto{\pgfqpoint{5.098583in}{2.897222in}}%
\pgfpathlineto{\pgfqpoint{5.113306in}{2.912465in}}%
\pgfpathlineto{\pgfqpoint{5.121078in}{2.921421in}}%
\pgfpathlineto{\pgfqpoint{5.128841in}{2.930205in}}%
\pgfpathlineto{\pgfqpoint{5.136595in}{2.938816in}}%
\pgfpathlineto{\pgfqpoint{5.144341in}{2.947255in}}%
\pgfpathlineto{\pgfqpoint{5.129623in}{2.932100in}}%
\pgfpathlineto{\pgfqpoint{5.114926in}{2.917132in}}%
\pgfpathlineto{\pgfqpoint{5.100249in}{2.902351in}}%
\pgfpathlineto{\pgfqpoint{5.085593in}{2.887757in}}%
\pgfpathlineto{\pgfqpoint{5.077842in}{2.879217in}}%
\pgfpathlineto{\pgfqpoint{5.070083in}{2.870514in}}%
\pgfpathlineto{\pgfqpoint{5.062315in}{2.861648in}}%
\pgfpathlineto{\pgfqpoint{5.054538in}{2.852617in}}%
\pgfpathclose%
\pgfusepath{fill}%
\end{pgfscope}%
\begin{pgfscope}%
\pgfpathrectangle{\pgfqpoint{1.150000in}{0.150000in}}{\pgfqpoint{5.700000in}{5.700000in}}%
\pgfusepath{clip}%
\pgfsetbuttcap%
\pgfsetroundjoin%
\definecolor{currentfill}{rgb}{0.243113,0.292092,0.538516}%
\pgfsetfillcolor{currentfill}%
\pgfsetfillopacity{0.800000}%
\pgfsetlinewidth{0.000000pt}%
\definecolor{currentstroke}{rgb}{0.000000,0.000000,0.000000}%
\pgfsetstrokecolor{currentstroke}%
\pgfsetdash{}{0pt}%
\pgfpathmoveto{\pgfqpoint{2.459385in}{2.041156in}}%
\pgfpathlineto{\pgfqpoint{2.473602in}{2.021400in}}%
\pgfpathlineto{\pgfqpoint{2.487809in}{2.001907in}}%
\pgfpathlineto{\pgfqpoint{2.502007in}{1.982675in}}%
\pgfpathlineto{\pgfqpoint{2.516195in}{1.963701in}}%
\pgfpathlineto{\pgfqpoint{2.525388in}{1.956657in}}%
\pgfpathlineto{\pgfqpoint{2.534555in}{1.950040in}}%
\pgfpathlineto{\pgfqpoint{2.543698in}{1.943843in}}%
\pgfpathlineto{\pgfqpoint{2.552818in}{1.938056in}}%
\pgfpathlineto{\pgfqpoint{2.538691in}{1.956227in}}%
\pgfpathlineto{\pgfqpoint{2.524555in}{1.974656in}}%
\pgfpathlineto{\pgfqpoint{2.510411in}{1.993345in}}%
\pgfpathlineto{\pgfqpoint{2.496257in}{2.012294in}}%
\pgfpathlineto{\pgfqpoint{2.487076in}{2.018870in}}%
\pgfpathlineto{\pgfqpoint{2.477871in}{2.025867in}}%
\pgfpathlineto{\pgfqpoint{2.468641in}{2.033292in}}%
\pgfpathlineto{\pgfqpoint{2.459385in}{2.041156in}}%
\pgfpathclose%
\pgfusepath{fill}%
\end{pgfscope}%
\begin{pgfscope}%
\pgfpathrectangle{\pgfqpoint{1.150000in}{0.150000in}}{\pgfqpoint{5.700000in}{5.700000in}}%
\pgfusepath{clip}%
\pgfsetbuttcap%
\pgfsetroundjoin%
\definecolor{currentfill}{rgb}{0.180629,0.429975,0.557282}%
\pgfsetfillcolor{currentfill}%
\pgfsetfillopacity{0.800000}%
\pgfsetlinewidth{0.000000pt}%
\definecolor{currentstroke}{rgb}{0.000000,0.000000,0.000000}%
\pgfsetstrokecolor{currentstroke}%
\pgfsetdash{}{0pt}%
\pgfpathmoveto{\pgfqpoint{4.601552in}{2.326691in}}%
\pgfpathlineto{\pgfqpoint{4.615919in}{2.338028in}}%
\pgfpathlineto{\pgfqpoint{4.630302in}{2.349549in}}%
\pgfpathlineto{\pgfqpoint{4.644703in}{2.361256in}}%
\pgfpathlineto{\pgfqpoint{4.659122in}{2.373147in}}%
\pgfpathlineto{\pgfqpoint{4.667103in}{2.386452in}}%
\pgfpathlineto{\pgfqpoint{4.675079in}{2.399617in}}%
\pgfpathlineto{\pgfqpoint{4.683048in}{2.412642in}}%
\pgfpathlineto{\pgfqpoint{4.691012in}{2.425526in}}%
\pgfpathlineto{\pgfqpoint{4.676592in}{2.413479in}}%
\pgfpathlineto{\pgfqpoint{4.662189in}{2.401616in}}%
\pgfpathlineto{\pgfqpoint{4.647803in}{2.389939in}}%
\pgfpathlineto{\pgfqpoint{4.633434in}{2.378447in}}%
\pgfpathlineto{\pgfqpoint{4.625473in}{2.365707in}}%
\pgfpathlineto{\pgfqpoint{4.617505in}{2.352833in}}%
\pgfpathlineto{\pgfqpoint{4.609531in}{2.339828in}}%
\pgfpathlineto{\pgfqpoint{4.601552in}{2.326691in}}%
\pgfpathclose%
\pgfusepath{fill}%
\end{pgfscope}%
\begin{pgfscope}%
\pgfpathrectangle{\pgfqpoint{1.150000in}{0.150000in}}{\pgfqpoint{5.700000in}{5.700000in}}%
\pgfusepath{clip}%
\pgfsetbuttcap%
\pgfsetroundjoin%
\definecolor{currentfill}{rgb}{0.140210,0.665859,0.513427}%
\pgfsetfillcolor{currentfill}%
\pgfsetfillopacity{0.800000}%
\pgfsetlinewidth{0.000000pt}%
\definecolor{currentstroke}{rgb}{0.000000,0.000000,0.000000}%
\pgfsetstrokecolor{currentstroke}%
\pgfsetdash{}{0pt}%
\pgfpathmoveto{\pgfqpoint{5.264997in}{3.070186in}}%
\pgfpathlineto{\pgfqpoint{5.279806in}{3.085997in}}%
\pgfpathlineto{\pgfqpoint{5.294637in}{3.101994in}}%
\pgfpathlineto{\pgfqpoint{5.309489in}{3.118180in}}%
\pgfpathlineto{\pgfqpoint{5.324364in}{3.134554in}}%
\pgfpathlineto{\pgfqpoint{5.332009in}{3.141171in}}%
\pgfpathlineto{\pgfqpoint{5.339643in}{3.147618in}}%
\pgfpathlineto{\pgfqpoint{5.347267in}{3.153898in}}%
\pgfpathlineto{\pgfqpoint{5.354882in}{3.160013in}}%
\pgfpathlineto{\pgfqpoint{5.340017in}{3.143835in}}%
\pgfpathlineto{\pgfqpoint{5.325175in}{3.127845in}}%
\pgfpathlineto{\pgfqpoint{5.310354in}{3.112042in}}%
\pgfpathlineto{\pgfqpoint{5.295555in}{3.096426in}}%
\pgfpathlineto{\pgfqpoint{5.287930in}{3.090103in}}%
\pgfpathlineto{\pgfqpoint{5.280295in}{3.083623in}}%
\pgfpathlineto{\pgfqpoint{5.272651in}{3.076985in}}%
\pgfpathlineto{\pgfqpoint{5.264997in}{3.070186in}}%
\pgfpathclose%
\pgfusepath{fill}%
\end{pgfscope}%
\begin{pgfscope}%
\pgfpathrectangle{\pgfqpoint{1.150000in}{0.150000in}}{\pgfqpoint{5.700000in}{5.700000in}}%
\pgfusepath{clip}%
\pgfsetbuttcap%
\pgfsetroundjoin%
\definecolor{currentfill}{rgb}{0.283072,0.130895,0.449241}%
\pgfsetfillcolor{currentfill}%
\pgfsetfillopacity{0.800000}%
\pgfsetlinewidth{0.000000pt}%
\definecolor{currentstroke}{rgb}{0.000000,0.000000,0.000000}%
\pgfsetstrokecolor{currentstroke}%
\pgfsetdash{}{0pt}%
\pgfpathmoveto{\pgfqpoint{2.798398in}{1.635903in}}%
\pgfpathlineto{\pgfqpoint{2.812448in}{1.621974in}}%
\pgfpathlineto{\pgfqpoint{2.826493in}{1.608268in}}%
\pgfpathlineto{\pgfqpoint{2.840535in}{1.594786in}}%
\pgfpathlineto{\pgfqpoint{2.854573in}{1.581524in}}%
\pgfpathlineto{\pgfqpoint{2.863432in}{1.579286in}}%
\pgfpathlineto{\pgfqpoint{2.872272in}{1.577418in}}%
\pgfpathlineto{\pgfqpoint{2.881094in}{1.575912in}}%
\pgfpathlineto{\pgfqpoint{2.889897in}{1.574761in}}%
\pgfpathlineto{\pgfqpoint{2.875907in}{1.587247in}}%
\pgfpathlineto{\pgfqpoint{2.861913in}{1.599953in}}%
\pgfpathlineto{\pgfqpoint{2.847917in}{1.612881in}}%
\pgfpathlineto{\pgfqpoint{2.833916in}{1.626033in}}%
\pgfpathlineto{\pgfqpoint{2.825066in}{1.627947in}}%
\pgfpathlineto{\pgfqpoint{2.816196in}{1.630225in}}%
\pgfpathlineto{\pgfqpoint{2.807307in}{1.632874in}}%
\pgfpathlineto{\pgfqpoint{2.798398in}{1.635903in}}%
\pgfpathclose%
\pgfusepath{fill}%
\end{pgfscope}%
\begin{pgfscope}%
\pgfpathrectangle{\pgfqpoint{1.150000in}{0.150000in}}{\pgfqpoint{5.700000in}{5.700000in}}%
\pgfusepath{clip}%
\pgfsetbuttcap%
\pgfsetroundjoin%
\definecolor{currentfill}{rgb}{0.273809,0.031497,0.358853}%
\pgfsetfillcolor{currentfill}%
\pgfsetfillopacity{0.800000}%
\pgfsetlinewidth{0.000000pt}%
\definecolor{currentstroke}{rgb}{0.000000,0.000000,0.000000}%
\pgfsetstrokecolor{currentstroke}%
\pgfsetdash{}{0pt}%
\pgfpathmoveto{\pgfqpoint{3.113473in}{1.404050in}}%
\pgfpathlineto{\pgfqpoint{3.127439in}{1.395145in}}%
\pgfpathlineto{\pgfqpoint{3.141406in}{1.386441in}}%
\pgfpathlineto{\pgfqpoint{3.155373in}{1.377939in}}%
\pgfpathlineto{\pgfqpoint{3.169341in}{1.369637in}}%
\pgfpathlineto{\pgfqpoint{3.177928in}{1.372562in}}%
\pgfpathlineto{\pgfqpoint{3.186501in}{1.375781in}}%
\pgfpathlineto{\pgfqpoint{3.195061in}{1.379288in}}%
\pgfpathlineto{\pgfqpoint{3.203609in}{1.383074in}}%
\pgfpathlineto{\pgfqpoint{3.189674in}{1.390648in}}%
\pgfpathlineto{\pgfqpoint{3.175741in}{1.398421in}}%
\pgfpathlineto{\pgfqpoint{3.161810in}{1.406395in}}%
\pgfpathlineto{\pgfqpoint{3.147878in}{1.414571in}}%
\pgfpathlineto{\pgfqpoint{3.139298in}{1.411501in}}%
\pgfpathlineto{\pgfqpoint{3.130704in}{1.408719in}}%
\pgfpathlineto{\pgfqpoint{3.122095in}{1.406233in}}%
\pgfpathlineto{\pgfqpoint{3.113473in}{1.404050in}}%
\pgfpathclose%
\pgfusepath{fill}%
\end{pgfscope}%
\begin{pgfscope}%
\pgfpathrectangle{\pgfqpoint{1.150000in}{0.150000in}}{\pgfqpoint{5.700000in}{5.700000in}}%
\pgfusepath{clip}%
\pgfsetbuttcap%
\pgfsetroundjoin%
\definecolor{currentfill}{rgb}{0.279574,0.170599,0.479997}%
\pgfsetfillcolor{currentfill}%
\pgfsetfillopacity{0.800000}%
\pgfsetlinewidth{0.000000pt}%
\definecolor{currentstroke}{rgb}{0.000000,0.000000,0.000000}%
\pgfsetstrokecolor{currentstroke}%
\pgfsetdash{}{0pt}%
\pgfpathmoveto{\pgfqpoint{4.027113in}{1.652933in}}%
\pgfpathlineto{\pgfqpoint{4.041186in}{1.657736in}}%
\pgfpathlineto{\pgfqpoint{4.055272in}{1.662721in}}%
\pgfpathlineto{\pgfqpoint{4.069368in}{1.667888in}}%
\pgfpathlineto{\pgfqpoint{4.083477in}{1.673237in}}%
\pgfpathlineto{\pgfqpoint{4.091620in}{1.687740in}}%
\pgfpathlineto{\pgfqpoint{4.099760in}{1.702244in}}%
\pgfpathlineto{\pgfqpoint{4.107894in}{1.716742in}}%
\pgfpathlineto{\pgfqpoint{4.116025in}{1.731232in}}%
\pgfpathlineto{\pgfqpoint{4.101918in}{1.725437in}}%
\pgfpathlineto{\pgfqpoint{4.087823in}{1.719824in}}%
\pgfpathlineto{\pgfqpoint{4.073739in}{1.714394in}}%
\pgfpathlineto{\pgfqpoint{4.059667in}{1.709145in}}%
\pgfpathlineto{\pgfqpoint{4.051535in}{1.695089in}}%
\pgfpathlineto{\pgfqpoint{4.043399in}{1.681032in}}%
\pgfpathlineto{\pgfqpoint{4.035258in}{1.666978in}}%
\pgfpathlineto{\pgfqpoint{4.027113in}{1.652933in}}%
\pgfpathclose%
\pgfusepath{fill}%
\end{pgfscope}%
\begin{pgfscope}%
\pgfpathrectangle{\pgfqpoint{1.150000in}{0.150000in}}{\pgfqpoint{5.700000in}{5.700000in}}%
\pgfusepath{clip}%
\pgfsetbuttcap%
\pgfsetroundjoin%
\definecolor{currentfill}{rgb}{0.121831,0.589055,0.545623}%
\pgfsetfillcolor{currentfill}%
\pgfsetfillopacity{0.800000}%
\pgfsetlinewidth{0.000000pt}%
\definecolor{currentstroke}{rgb}{0.000000,0.000000,0.000000}%
\pgfsetstrokecolor{currentstroke}%
\pgfsetdash{}{0pt}%
\pgfpathmoveto{\pgfqpoint{1.958566in}{2.957752in}}%
\pgfpathlineto{\pgfqpoint{1.973255in}{2.927102in}}%
\pgfpathlineto{\pgfqpoint{1.987921in}{2.896826in}}%
\pgfpathlineto{\pgfqpoint{2.002566in}{2.866920in}}%
\pgfpathlineto{\pgfqpoint{2.017190in}{2.837380in}}%
\pgfpathlineto{\pgfqpoint{2.026869in}{2.825833in}}%
\pgfpathlineto{\pgfqpoint{2.036517in}{2.814752in}}%
\pgfpathlineto{\pgfqpoint{2.046133in}{2.804128in}}%
\pgfpathlineto{\pgfqpoint{2.055719in}{2.793953in}}%
\pgfpathlineto{\pgfqpoint{2.041174in}{2.822698in}}%
\pgfpathlineto{\pgfqpoint{2.026609in}{2.851807in}}%
\pgfpathlineto{\pgfqpoint{2.012024in}{2.881283in}}%
\pgfpathlineto{\pgfqpoint{1.997417in}{2.911130in}}%
\pgfpathlineto{\pgfqpoint{1.987752in}{2.922087in}}%
\pgfpathlineto{\pgfqpoint{1.978056in}{2.933504in}}%
\pgfpathlineto{\pgfqpoint{1.968328in}{2.945390in}}%
\pgfpathlineto{\pgfqpoint{1.958566in}{2.957752in}}%
\pgfpathclose%
\pgfusepath{fill}%
\end{pgfscope}%
\begin{pgfscope}%
\pgfpathrectangle{\pgfqpoint{1.150000in}{0.150000in}}{\pgfqpoint{5.700000in}{5.700000in}}%
\pgfusepath{clip}%
\pgfsetbuttcap%
\pgfsetroundjoin%
\definecolor{currentfill}{rgb}{0.203063,0.379716,0.553925}%
\pgfsetfillcolor{currentfill}%
\pgfsetfillopacity{0.800000}%
\pgfsetlinewidth{0.000000pt}%
\definecolor{currentstroke}{rgb}{0.000000,0.000000,0.000000}%
\pgfsetstrokecolor{currentstroke}%
\pgfsetdash{}{0pt}%
\pgfpathmoveto{\pgfqpoint{4.480229in}{2.175204in}}%
\pgfpathlineto{\pgfqpoint{4.494526in}{2.185392in}}%
\pgfpathlineto{\pgfqpoint{4.508839in}{2.195763in}}%
\pgfpathlineto{\pgfqpoint{4.523168in}{2.206318in}}%
\pgfpathlineto{\pgfqpoint{4.537514in}{2.217058in}}%
\pgfpathlineto{\pgfqpoint{4.545537in}{2.231189in}}%
\pgfpathlineto{\pgfqpoint{4.553556in}{2.245202in}}%
\pgfpathlineto{\pgfqpoint{4.561569in}{2.259094in}}%
\pgfpathlineto{\pgfqpoint{4.569577in}{2.272864in}}%
\pgfpathlineto{\pgfqpoint{4.555229in}{2.261901in}}%
\pgfpathlineto{\pgfqpoint{4.540897in}{2.251123in}}%
\pgfpathlineto{\pgfqpoint{4.526582in}{2.240529in}}%
\pgfpathlineto{\pgfqpoint{4.512282in}{2.230120in}}%
\pgfpathlineto{\pgfqpoint{4.504277in}{2.216559in}}%
\pgfpathlineto{\pgfqpoint{4.496266in}{2.202886in}}%
\pgfpathlineto{\pgfqpoint{4.488250in}{2.189100in}}%
\pgfpathlineto{\pgfqpoint{4.480229in}{2.175204in}}%
\pgfpathclose%
\pgfusepath{fill}%
\end{pgfscope}%
\begin{pgfscope}%
\pgfpathrectangle{\pgfqpoint{1.150000in}{0.150000in}}{\pgfqpoint{5.700000in}{5.700000in}}%
\pgfusepath{clip}%
\pgfsetbuttcap%
\pgfsetroundjoin%
\definecolor{currentfill}{rgb}{0.277941,0.056324,0.381191}%
\pgfsetfillcolor{currentfill}%
\pgfsetfillopacity{0.800000}%
\pgfsetlinewidth{0.000000pt}%
\definecolor{currentstroke}{rgb}{0.000000,0.000000,0.000000}%
\pgfsetstrokecolor{currentstroke}%
\pgfsetdash{}{0pt}%
\pgfpathmoveto{\pgfqpoint{3.727791in}{1.414604in}}%
\pgfpathlineto{\pgfqpoint{3.741776in}{1.415128in}}%
\pgfpathlineto{\pgfqpoint{3.755768in}{1.415835in}}%
\pgfpathlineto{\pgfqpoint{3.769768in}{1.416725in}}%
\pgfpathlineto{\pgfqpoint{3.783777in}{1.417797in}}%
\pgfpathlineto{\pgfqpoint{3.792013in}{1.429928in}}%
\pgfpathlineto{\pgfqpoint{3.800243in}{1.442160in}}%
\pgfpathlineto{\pgfqpoint{3.808468in}{1.454487in}}%
\pgfpathlineto{\pgfqpoint{3.816687in}{1.466904in}}%
\pgfpathlineto{\pgfqpoint{3.802688in}{1.465263in}}%
\pgfpathlineto{\pgfqpoint{3.788697in}{1.463805in}}%
\pgfpathlineto{\pgfqpoint{3.774715in}{1.462531in}}%
\pgfpathlineto{\pgfqpoint{3.760741in}{1.461440in}}%
\pgfpathlineto{\pgfqpoint{3.752512in}{1.449579in}}%
\pgfpathlineto{\pgfqpoint{3.744278in}{1.437816in}}%
\pgfpathlineto{\pgfqpoint{3.736038in}{1.426155in}}%
\pgfpathlineto{\pgfqpoint{3.727791in}{1.414604in}}%
\pgfpathclose%
\pgfusepath{fill}%
\end{pgfscope}%
\begin{pgfscope}%
\pgfpathrectangle{\pgfqpoint{1.150000in}{0.150000in}}{\pgfqpoint{5.700000in}{5.700000in}}%
\pgfusepath{clip}%
\pgfsetbuttcap%
\pgfsetroundjoin%
\definecolor{currentfill}{rgb}{0.273809,0.031497,0.358853}%
\pgfsetfillcolor{currentfill}%
\pgfsetfillopacity{0.800000}%
\pgfsetlinewidth{0.000000pt}%
\definecolor{currentstroke}{rgb}{0.000000,0.000000,0.000000}%
\pgfsetstrokecolor{currentstroke}%
\pgfsetdash{}{0pt}%
\pgfpathmoveto{\pgfqpoint{3.638831in}{1.371736in}}%
\pgfpathlineto{\pgfqpoint{3.652799in}{1.370925in}}%
\pgfpathlineto{\pgfqpoint{3.666773in}{1.370298in}}%
\pgfpathlineto{\pgfqpoint{3.680755in}{1.369856in}}%
\pgfpathlineto{\pgfqpoint{3.694743in}{1.369597in}}%
\pgfpathlineto{\pgfqpoint{3.703015in}{1.380657in}}%
\pgfpathlineto{\pgfqpoint{3.711280in}{1.391849in}}%
\pgfpathlineto{\pgfqpoint{3.719539in}{1.403167in}}%
\pgfpathlineto{\pgfqpoint{3.727791in}{1.414604in}}%
\pgfpathlineto{\pgfqpoint{3.713815in}{1.414264in}}%
\pgfpathlineto{\pgfqpoint{3.699846in}{1.414108in}}%
\pgfpathlineto{\pgfqpoint{3.685885in}{1.414137in}}%
\pgfpathlineto{\pgfqpoint{3.671931in}{1.414350in}}%
\pgfpathlineto{\pgfqpoint{3.663666in}{1.403500in}}%
\pgfpathlineto{\pgfqpoint{3.655395in}{1.392776in}}%
\pgfpathlineto{\pgfqpoint{3.647117in}{1.382187in}}%
\pgfpathlineto{\pgfqpoint{3.638831in}{1.371736in}}%
\pgfpathclose%
\pgfusepath{fill}%
\end{pgfscope}%
\begin{pgfscope}%
\pgfpathrectangle{\pgfqpoint{1.150000in}{0.150000in}}{\pgfqpoint{5.700000in}{5.700000in}}%
\pgfusepath{clip}%
\pgfsetbuttcap%
\pgfsetroundjoin%
\definecolor{currentfill}{rgb}{0.229739,0.322361,0.545706}%
\pgfsetfillcolor{currentfill}%
\pgfsetfillopacity{0.800000}%
\pgfsetlinewidth{0.000000pt}%
\definecolor{currentstroke}{rgb}{0.000000,0.000000,0.000000}%
\pgfsetstrokecolor{currentstroke}%
\pgfsetdash{}{0pt}%
\pgfpathmoveto{\pgfqpoint{2.402413in}{2.122849in}}%
\pgfpathlineto{\pgfqpoint{2.416672in}{2.102021in}}%
\pgfpathlineto{\pgfqpoint{2.430920in}{2.081464in}}%
\pgfpathlineto{\pgfqpoint{2.445157in}{2.061176in}}%
\pgfpathlineto{\pgfqpoint{2.459385in}{2.041156in}}%
\pgfpathlineto{\pgfqpoint{2.468641in}{2.033292in}}%
\pgfpathlineto{\pgfqpoint{2.477871in}{2.025867in}}%
\pgfpathlineto{\pgfqpoint{2.487076in}{2.018870in}}%
\pgfpathlineto{\pgfqpoint{2.496257in}{2.012294in}}%
\pgfpathlineto{\pgfqpoint{2.482094in}{2.031508in}}%
\pgfpathlineto{\pgfqpoint{2.467921in}{2.050986in}}%
\pgfpathlineto{\pgfqpoint{2.453738in}{2.070732in}}%
\pgfpathlineto{\pgfqpoint{2.439545in}{2.090748in}}%
\pgfpathlineto{\pgfqpoint{2.430301in}{2.098118in}}%
\pgfpathlineto{\pgfqpoint{2.421032in}{2.105920in}}%
\pgfpathlineto{\pgfqpoint{2.411736in}{2.114160in}}%
\pgfpathlineto{\pgfqpoint{2.402413in}{2.122849in}}%
\pgfpathclose%
\pgfusepath{fill}%
\end{pgfscope}%
\begin{pgfscope}%
\pgfpathrectangle{\pgfqpoint{1.150000in}{0.150000in}}{\pgfqpoint{5.700000in}{5.700000in}}%
\pgfusepath{clip}%
\pgfsetbuttcap%
\pgfsetroundjoin%
\definecolor{currentfill}{rgb}{0.229739,0.322361,0.545706}%
\pgfsetfillcolor{currentfill}%
\pgfsetfillopacity{0.800000}%
\pgfsetlinewidth{0.000000pt}%
\definecolor{currentstroke}{rgb}{0.000000,0.000000,0.000000}%
\pgfsetstrokecolor{currentstroke}%
\pgfsetdash{}{0pt}%
\pgfpathmoveto{\pgfqpoint{4.358865in}{2.023587in}}%
\pgfpathlineto{\pgfqpoint{4.373097in}{2.032497in}}%
\pgfpathlineto{\pgfqpoint{4.387343in}{2.041589in}}%
\pgfpathlineto{\pgfqpoint{4.401604in}{2.050865in}}%
\pgfpathlineto{\pgfqpoint{4.415881in}{2.060324in}}%
\pgfpathlineto{\pgfqpoint{4.423941in}{2.075024in}}%
\pgfpathlineto{\pgfqpoint{4.431997in}{2.089632in}}%
\pgfpathlineto{\pgfqpoint{4.440048in}{2.104145in}}%
\pgfpathlineto{\pgfqpoint{4.448094in}{2.118560in}}%
\pgfpathlineto{\pgfqpoint{4.433815in}{2.108812in}}%
\pgfpathlineto{\pgfqpoint{4.419551in}{2.099248in}}%
\pgfpathlineto{\pgfqpoint{4.405303in}{2.089867in}}%
\pgfpathlineto{\pgfqpoint{4.391069in}{2.080670in}}%
\pgfpathlineto{\pgfqpoint{4.383025in}{2.066530in}}%
\pgfpathlineto{\pgfqpoint{4.374977in}{2.052302in}}%
\pgfpathlineto{\pgfqpoint{4.366923in}{2.037986in}}%
\pgfpathlineto{\pgfqpoint{4.358865in}{2.023587in}}%
\pgfpathclose%
\pgfusepath{fill}%
\end{pgfscope}%
\begin{pgfscope}%
\pgfpathrectangle{\pgfqpoint{1.150000in}{0.150000in}}{\pgfqpoint{5.700000in}{5.700000in}}%
\pgfusepath{clip}%
\pgfsetbuttcap%
\pgfsetroundjoin%
\definecolor{currentfill}{rgb}{0.283091,0.110553,0.431554}%
\pgfsetfillcolor{currentfill}%
\pgfsetfillopacity{0.800000}%
\pgfsetlinewidth{0.000000pt}%
\definecolor{currentstroke}{rgb}{0.000000,0.000000,0.000000}%
\pgfsetstrokecolor{currentstroke}%
\pgfsetdash{}{0pt}%
\pgfpathmoveto{\pgfqpoint{2.854573in}{1.581524in}}%
\pgfpathlineto{\pgfqpoint{2.868607in}{1.568483in}}%
\pgfpathlineto{\pgfqpoint{2.882638in}{1.555661in}}%
\pgfpathlineto{\pgfqpoint{2.896666in}{1.543056in}}%
\pgfpathlineto{\pgfqpoint{2.910691in}{1.530668in}}%
\pgfpathlineto{\pgfqpoint{2.919503in}{1.529217in}}%
\pgfpathlineto{\pgfqpoint{2.928296in}{1.528127in}}%
\pgfpathlineto{\pgfqpoint{2.937072in}{1.527390in}}%
\pgfpathlineto{\pgfqpoint{2.945831in}{1.526998in}}%
\pgfpathlineto{\pgfqpoint{2.931851in}{1.538613in}}%
\pgfpathlineto{\pgfqpoint{2.917869in}{1.550445in}}%
\pgfpathlineto{\pgfqpoint{2.903885in}{1.562494in}}%
\pgfpathlineto{\pgfqpoint{2.889897in}{1.574761in}}%
\pgfpathlineto{\pgfqpoint{2.881094in}{1.575912in}}%
\pgfpathlineto{\pgfqpoint{2.872272in}{1.577418in}}%
\pgfpathlineto{\pgfqpoint{2.863432in}{1.579286in}}%
\pgfpathlineto{\pgfqpoint{2.854573in}{1.581524in}}%
\pgfpathclose%
\pgfusepath{fill}%
\end{pgfscope}%
\begin{pgfscope}%
\pgfpathrectangle{\pgfqpoint{1.150000in}{0.150000in}}{\pgfqpoint{5.700000in}{5.700000in}}%
\pgfusepath{clip}%
\pgfsetbuttcap%
\pgfsetroundjoin%
\definecolor{currentfill}{rgb}{0.281446,0.084320,0.407414}%
\pgfsetfillcolor{currentfill}%
\pgfsetfillopacity{0.800000}%
\pgfsetlinewidth{0.000000pt}%
\definecolor{currentstroke}{rgb}{0.000000,0.000000,0.000000}%
\pgfsetstrokecolor{currentstroke}%
\pgfsetdash{}{0pt}%
\pgfpathmoveto{\pgfqpoint{3.816687in}{1.466904in}}%
\pgfpathlineto{\pgfqpoint{3.830696in}{1.468728in}}%
\pgfpathlineto{\pgfqpoint{3.844713in}{1.470735in}}%
\pgfpathlineto{\pgfqpoint{3.858739in}{1.472923in}}%
\pgfpathlineto{\pgfqpoint{3.872775in}{1.475294in}}%
\pgfpathlineto{\pgfqpoint{3.880981in}{1.488345in}}%
\pgfpathlineto{\pgfqpoint{3.889183in}{1.501468in}}%
\pgfpathlineto{\pgfqpoint{3.897379in}{1.514657in}}%
\pgfpathlineto{\pgfqpoint{3.905570in}{1.527907in}}%
\pgfpathlineto{\pgfqpoint{3.891541in}{1.524999in}}%
\pgfpathlineto{\pgfqpoint{3.877521in}{1.522272in}}%
\pgfpathlineto{\pgfqpoint{3.863511in}{1.519728in}}%
\pgfpathlineto{\pgfqpoint{3.849510in}{1.517368in}}%
\pgfpathlineto{\pgfqpoint{3.841313in}{1.504643in}}%
\pgfpathlineto{\pgfqpoint{3.833109in}{1.491987in}}%
\pgfpathlineto{\pgfqpoint{3.824901in}{1.479406in}}%
\pgfpathlineto{\pgfqpoint{3.816687in}{1.466904in}}%
\pgfpathclose%
\pgfusepath{fill}%
\end{pgfscope}%
\begin{pgfscope}%
\pgfpathrectangle{\pgfqpoint{1.150000in}{0.150000in}}{\pgfqpoint{5.700000in}{5.700000in}}%
\pgfusepath{clip}%
\pgfsetbuttcap%
\pgfsetroundjoin%
\definecolor{currentfill}{rgb}{0.269944,0.014625,0.341379}%
\pgfsetfillcolor{currentfill}%
\pgfsetfillopacity{0.800000}%
\pgfsetlinewidth{0.000000pt}%
\definecolor{currentstroke}{rgb}{0.000000,0.000000,0.000000}%
\pgfsetstrokecolor{currentstroke}%
\pgfsetdash{}{0pt}%
\pgfpathmoveto{\pgfqpoint{3.549751in}{1.339062in}}%
\pgfpathlineto{\pgfqpoint{3.563709in}{1.336880in}}%
\pgfpathlineto{\pgfqpoint{3.577674in}{1.334883in}}%
\pgfpathlineto{\pgfqpoint{3.591644in}{1.333072in}}%
\pgfpathlineto{\pgfqpoint{3.605620in}{1.331447in}}%
\pgfpathlineto{\pgfqpoint{3.613934in}{1.341280in}}%
\pgfpathlineto{\pgfqpoint{3.622240in}{1.351277in}}%
\pgfpathlineto{\pgfqpoint{3.630539in}{1.361431in}}%
\pgfpathlineto{\pgfqpoint{3.638831in}{1.371736in}}%
\pgfpathlineto{\pgfqpoint{3.624871in}{1.372732in}}%
\pgfpathlineto{\pgfqpoint{3.610917in}{1.373914in}}%
\pgfpathlineto{\pgfqpoint{3.596969in}{1.375281in}}%
\pgfpathlineto{\pgfqpoint{3.583027in}{1.376835in}}%
\pgfpathlineto{\pgfqpoint{3.574720in}{1.367147in}}%
\pgfpathlineto{\pgfqpoint{3.566405in}{1.357618in}}%
\pgfpathlineto{\pgfqpoint{3.558082in}{1.348254in}}%
\pgfpathlineto{\pgfqpoint{3.549751in}{1.339062in}}%
\pgfpathclose%
\pgfusepath{fill}%
\end{pgfscope}%
\begin{pgfscope}%
\pgfpathrectangle{\pgfqpoint{1.150000in}{0.150000in}}{\pgfqpoint{5.700000in}{5.700000in}}%
\pgfusepath{clip}%
\pgfsetbuttcap%
\pgfsetroundjoin%
\definecolor{currentfill}{rgb}{0.131172,0.555899,0.552459}%
\pgfsetfillcolor{currentfill}%
\pgfsetfillopacity{0.800000}%
\pgfsetlinewidth{0.000000pt}%
\definecolor{currentstroke}{rgb}{0.000000,0.000000,0.000000}%
\pgfsetstrokecolor{currentstroke}%
\pgfsetdash{}{0pt}%
\pgfpathmoveto{\pgfqpoint{4.933556in}{2.717177in}}%
\pgfpathlineto{\pgfqpoint{4.948143in}{2.731179in}}%
\pgfpathlineto{\pgfqpoint{4.962750in}{2.745368in}}%
\pgfpathlineto{\pgfqpoint{4.977377in}{2.759744in}}%
\pgfpathlineto{\pgfqpoint{4.992024in}{2.774306in}}%
\pgfpathlineto{\pgfqpoint{4.999867in}{2.784693in}}%
\pgfpathlineto{\pgfqpoint{5.007702in}{2.794907in}}%
\pgfpathlineto{\pgfqpoint{5.015529in}{2.804950in}}%
\pgfpathlineto{\pgfqpoint{5.023348in}{2.814822in}}%
\pgfpathlineto{\pgfqpoint{5.008703in}{2.800276in}}%
\pgfpathlineto{\pgfqpoint{4.994079in}{2.785917in}}%
\pgfpathlineto{\pgfqpoint{4.979474in}{2.771744in}}%
\pgfpathlineto{\pgfqpoint{4.964888in}{2.757758in}}%
\pgfpathlineto{\pgfqpoint{4.957067in}{2.747857in}}%
\pgfpathlineto{\pgfqpoint{4.949238in}{2.737793in}}%
\pgfpathlineto{\pgfqpoint{4.941401in}{2.727567in}}%
\pgfpathlineto{\pgfqpoint{4.933556in}{2.717177in}}%
\pgfpathclose%
\pgfusepath{fill}%
\end{pgfscope}%
\begin{pgfscope}%
\pgfpathrectangle{\pgfqpoint{1.150000in}{0.150000in}}{\pgfqpoint{5.700000in}{5.700000in}}%
\pgfusepath{clip}%
\pgfsetbuttcap%
\pgfsetroundjoin%
\definecolor{currentfill}{rgb}{0.153364,0.497000,0.557724}%
\pgfsetfillcolor{currentfill}%
\pgfsetfillopacity{0.800000}%
\pgfsetlinewidth{0.000000pt}%
\definecolor{currentstroke}{rgb}{0.000000,0.000000,0.000000}%
\pgfsetstrokecolor{currentstroke}%
\pgfsetdash{}{0pt}%
\pgfpathmoveto{\pgfqpoint{2.095071in}{2.658186in}}%
\pgfpathlineto{\pgfqpoint{2.109601in}{2.630960in}}%
\pgfpathlineto{\pgfqpoint{2.124112in}{2.604068in}}%
\pgfpathlineto{\pgfqpoint{2.138607in}{2.577505in}}%
\pgfpathlineto{\pgfqpoint{2.153084in}{2.551269in}}%
\pgfpathlineto{\pgfqpoint{2.162652in}{2.540294in}}%
\pgfpathlineto{\pgfqpoint{2.172190in}{2.529786in}}%
\pgfpathlineto{\pgfqpoint{2.181698in}{2.519735in}}%
\pgfpathlineto{\pgfqpoint{2.191177in}{2.510134in}}%
\pgfpathlineto{\pgfqpoint{2.176776in}{2.535560in}}%
\pgfpathlineto{\pgfqpoint{2.162357in}{2.561310in}}%
\pgfpathlineto{\pgfqpoint{2.147922in}{2.587388in}}%
\pgfpathlineto{\pgfqpoint{2.133470in}{2.613797in}}%
\pgfpathlineto{\pgfqpoint{2.123917in}{2.624195in}}%
\pgfpathlineto{\pgfqpoint{2.114333in}{2.635054in}}%
\pgfpathlineto{\pgfqpoint{2.104718in}{2.646381in}}%
\pgfpathlineto{\pgfqpoint{2.095071in}{2.658186in}}%
\pgfpathclose%
\pgfusepath{fill}%
\end{pgfscope}%
\begin{pgfscope}%
\pgfpathrectangle{\pgfqpoint{1.150000in}{0.150000in}}{\pgfqpoint{5.700000in}{5.700000in}}%
\pgfusepath{clip}%
\pgfsetbuttcap%
\pgfsetroundjoin%
\definecolor{currentfill}{rgb}{0.274149,0.751988,0.436601}%
\pgfsetfillcolor{currentfill}%
\pgfsetfillopacity{0.800000}%
\pgfsetlinewidth{0.000000pt}%
\definecolor{currentstroke}{rgb}{0.000000,0.000000,0.000000}%
\pgfsetstrokecolor{currentstroke}%
\pgfsetdash{}{0pt}%
\pgfpathmoveto{\pgfqpoint{5.564865in}{3.351355in}}%
\pgfpathlineto{\pgfqpoint{5.579891in}{3.368420in}}%
\pgfpathlineto{\pgfqpoint{5.594941in}{3.385673in}}%
\pgfpathlineto{\pgfqpoint{5.610015in}{3.403115in}}%
\pgfpathlineto{\pgfqpoint{5.625112in}{3.420746in}}%
\pgfpathlineto{\pgfqpoint{5.632546in}{3.424009in}}%
\pgfpathlineto{\pgfqpoint{5.639968in}{3.427127in}}%
\pgfpathlineto{\pgfqpoint{5.647379in}{3.430102in}}%
\pgfpathlineto{\pgfqpoint{5.654779in}{3.432938in}}%
\pgfpathlineto{\pgfqpoint{5.639700in}{3.415651in}}%
\pgfpathlineto{\pgfqpoint{5.624645in}{3.398551in}}%
\pgfpathlineto{\pgfqpoint{5.609614in}{3.381640in}}%
\pgfpathlineto{\pgfqpoint{5.594605in}{3.364915in}}%
\pgfpathlineto{\pgfqpoint{5.587186in}{3.361724in}}%
\pgfpathlineto{\pgfqpoint{5.579756in}{3.358403in}}%
\pgfpathlineto{\pgfqpoint{5.572316in}{3.354948in}}%
\pgfpathlineto{\pgfqpoint{5.564865in}{3.351355in}}%
\pgfpathclose%
\pgfusepath{fill}%
\end{pgfscope}%
\begin{pgfscope}%
\pgfpathrectangle{\pgfqpoint{1.150000in}{0.150000in}}{\pgfqpoint{5.700000in}{5.700000in}}%
\pgfusepath{clip}%
\pgfsetbuttcap%
\pgfsetroundjoin%
\definecolor{currentfill}{rgb}{0.252194,0.269783,0.531579}%
\pgfsetfillcolor{currentfill}%
\pgfsetfillopacity{0.800000}%
\pgfsetlinewidth{0.000000pt}%
\definecolor{currentstroke}{rgb}{0.000000,0.000000,0.000000}%
\pgfsetstrokecolor{currentstroke}%
\pgfsetdash{}{0pt}%
\pgfpathmoveto{\pgfqpoint{4.237470in}{1.874579in}}%
\pgfpathlineto{\pgfqpoint{4.251641in}{1.882085in}}%
\pgfpathlineto{\pgfqpoint{4.265827in}{1.889773in}}%
\pgfpathlineto{\pgfqpoint{4.280025in}{1.897643in}}%
\pgfpathlineto{\pgfqpoint{4.294238in}{1.905696in}}%
\pgfpathlineto{\pgfqpoint{4.302332in}{1.920667in}}%
\pgfpathlineto{\pgfqpoint{4.310422in}{1.935578in}}%
\pgfpathlineto{\pgfqpoint{4.318507in}{1.950424in}}%
\pgfpathlineto{\pgfqpoint{4.326588in}{1.965203in}}%
\pgfpathlineto{\pgfqpoint{4.312373in}{1.956797in}}%
\pgfpathlineto{\pgfqpoint{4.298172in}{1.948574in}}%
\pgfpathlineto{\pgfqpoint{4.283985in}{1.940533in}}%
\pgfpathlineto{\pgfqpoint{4.269812in}{1.932676in}}%
\pgfpathlineto{\pgfqpoint{4.261733in}{1.918237in}}%
\pgfpathlineto{\pgfqpoint{4.253650in}{1.903740in}}%
\pgfpathlineto{\pgfqpoint{4.245562in}{1.889186in}}%
\pgfpathlineto{\pgfqpoint{4.237470in}{1.874579in}}%
\pgfpathclose%
\pgfusepath{fill}%
\end{pgfscope}%
\begin{pgfscope}%
\pgfpathrectangle{\pgfqpoint{1.150000in}{0.150000in}}{\pgfqpoint{5.700000in}{5.700000in}}%
\pgfusepath{clip}%
\pgfsetbuttcap%
\pgfsetroundjoin%
\definecolor{currentfill}{rgb}{0.267004,0.004874,0.329415}%
\pgfsetfillcolor{currentfill}%
\pgfsetfillopacity{0.800000}%
\pgfsetlinewidth{0.000000pt}%
\definecolor{currentstroke}{rgb}{0.000000,0.000000,0.000000}%
\pgfsetstrokecolor{currentstroke}%
\pgfsetdash{}{0pt}%
\pgfpathmoveto{\pgfqpoint{3.315144in}{1.329590in}}%
\pgfpathlineto{\pgfqpoint{3.329096in}{1.323781in}}%
\pgfpathlineto{\pgfqpoint{3.343052in}{1.318165in}}%
\pgfpathlineto{\pgfqpoint{3.357010in}{1.312741in}}%
\pgfpathlineto{\pgfqpoint{3.370971in}{1.307508in}}%
\pgfpathlineto{\pgfqpoint{3.379419in}{1.313686in}}%
\pgfpathlineto{\pgfqpoint{3.387857in}{1.320104in}}%
\pgfpathlineto{\pgfqpoint{3.396284in}{1.326755in}}%
\pgfpathlineto{\pgfqpoint{3.404701in}{1.333633in}}%
\pgfpathlineto{\pgfqpoint{3.390765in}{1.338173in}}%
\pgfpathlineto{\pgfqpoint{3.376833in}{1.342904in}}%
\pgfpathlineto{\pgfqpoint{3.362903in}{1.347827in}}%
\pgfpathlineto{\pgfqpoint{3.348978in}{1.352942in}}%
\pgfpathlineto{\pgfqpoint{3.340535in}{1.346745in}}%
\pgfpathlineto{\pgfqpoint{3.332083in}{1.340782in}}%
\pgfpathlineto{\pgfqpoint{3.323619in}{1.335062in}}%
\pgfpathlineto{\pgfqpoint{3.315144in}{1.329590in}}%
\pgfpathclose%
\pgfusepath{fill}%
\end{pgfscope}%
\begin{pgfscope}%
\pgfpathrectangle{\pgfqpoint{1.150000in}{0.150000in}}{\pgfqpoint{5.700000in}{5.700000in}}%
\pgfusepath{clip}%
\pgfsetbuttcap%
\pgfsetroundjoin%
\definecolor{currentfill}{rgb}{0.214298,0.355619,0.551184}%
\pgfsetfillcolor{currentfill}%
\pgfsetfillopacity{0.800000}%
\pgfsetlinewidth{0.000000pt}%
\definecolor{currentstroke}{rgb}{0.000000,0.000000,0.000000}%
\pgfsetstrokecolor{currentstroke}%
\pgfsetdash{}{0pt}%
\pgfpathmoveto{\pgfqpoint{2.345263in}{2.208917in}}%
\pgfpathlineto{\pgfqpoint{2.359568in}{2.186982in}}%
\pgfpathlineto{\pgfqpoint{2.373861in}{2.165327in}}%
\pgfpathlineto{\pgfqpoint{2.388143in}{2.143950in}}%
\pgfpathlineto{\pgfqpoint{2.402413in}{2.122849in}}%
\pgfpathlineto{\pgfqpoint{2.411736in}{2.114160in}}%
\pgfpathlineto{\pgfqpoint{2.421032in}{2.105920in}}%
\pgfpathlineto{\pgfqpoint{2.430301in}{2.098118in}}%
\pgfpathlineto{\pgfqpoint{2.439545in}{2.090748in}}%
\pgfpathlineto{\pgfqpoint{2.425341in}{2.111035in}}%
\pgfpathlineto{\pgfqpoint{2.411127in}{2.131597in}}%
\pgfpathlineto{\pgfqpoint{2.396901in}{2.152434in}}%
\pgfpathlineto{\pgfqpoint{2.382664in}{2.173550in}}%
\pgfpathlineto{\pgfqpoint{2.373355in}{2.181721in}}%
\pgfpathlineto{\pgfqpoint{2.364018in}{2.190334in}}%
\pgfpathlineto{\pgfqpoint{2.354654in}{2.199396in}}%
\pgfpathlineto{\pgfqpoint{2.345263in}{2.208917in}}%
\pgfpathclose%
\pgfusepath{fill}%
\end{pgfscope}%
\begin{pgfscope}%
\pgfpathrectangle{\pgfqpoint{1.150000in}{0.150000in}}{\pgfqpoint{5.700000in}{5.700000in}}%
\pgfusepath{clip}%
\pgfsetbuttcap%
\pgfsetroundjoin%
\definecolor{currentfill}{rgb}{0.282327,0.094955,0.417331}%
\pgfsetfillcolor{currentfill}%
\pgfsetfillopacity{0.800000}%
\pgfsetlinewidth{0.000000pt}%
\definecolor{currentstroke}{rgb}{0.000000,0.000000,0.000000}%
\pgfsetstrokecolor{currentstroke}%
\pgfsetdash{}{0pt}%
\pgfpathmoveto{\pgfqpoint{2.910691in}{1.530668in}}%
\pgfpathlineto{\pgfqpoint{2.924713in}{1.518495in}}%
\pgfpathlineto{\pgfqpoint{2.938733in}{1.506537in}}%
\pgfpathlineto{\pgfqpoint{2.952750in}{1.494791in}}%
\pgfpathlineto{\pgfqpoint{2.966766in}{1.483258in}}%
\pgfpathlineto{\pgfqpoint{2.975532in}{1.482591in}}%
\pgfpathlineto{\pgfqpoint{2.984281in}{1.482276in}}%
\pgfpathlineto{\pgfqpoint{2.993014in}{1.482305in}}%
\pgfpathlineto{\pgfqpoint{3.001730in}{1.482670in}}%
\pgfpathlineto{\pgfqpoint{2.987758in}{1.493434in}}%
\pgfpathlineto{\pgfqpoint{2.973784in}{1.504409in}}%
\pgfpathlineto{\pgfqpoint{2.959808in}{1.515596in}}%
\pgfpathlineto{\pgfqpoint{2.945831in}{1.526998in}}%
\pgfpathlineto{\pgfqpoint{2.937072in}{1.527390in}}%
\pgfpathlineto{\pgfqpoint{2.928296in}{1.528127in}}%
\pgfpathlineto{\pgfqpoint{2.919503in}{1.529217in}}%
\pgfpathlineto{\pgfqpoint{2.910691in}{1.530668in}}%
\pgfpathclose%
\pgfusepath{fill}%
\end{pgfscope}%
\begin{pgfscope}%
\pgfpathrectangle{\pgfqpoint{1.150000in}{0.150000in}}{\pgfqpoint{5.700000in}{5.700000in}}%
\pgfusepath{clip}%
\pgfsetbuttcap%
\pgfsetroundjoin%
\definecolor{currentfill}{rgb}{0.283229,0.120777,0.440584}%
\pgfsetfillcolor{currentfill}%
\pgfsetfillopacity{0.800000}%
\pgfsetlinewidth{0.000000pt}%
\definecolor{currentstroke}{rgb}{0.000000,0.000000,0.000000}%
\pgfsetstrokecolor{currentstroke}%
\pgfsetdash{}{0pt}%
\pgfpathmoveto{\pgfqpoint{3.905570in}{1.527907in}}%
\pgfpathlineto{\pgfqpoint{3.919610in}{1.530998in}}%
\pgfpathlineto{\pgfqpoint{3.933659in}{1.534271in}}%
\pgfpathlineto{\pgfqpoint{3.947718in}{1.537726in}}%
\pgfpathlineto{\pgfqpoint{3.961788in}{1.541362in}}%
\pgfpathlineto{\pgfqpoint{3.969969in}{1.555188in}}%
\pgfpathlineto{\pgfqpoint{3.978146in}{1.569058in}}%
\pgfpathlineto{\pgfqpoint{3.986319in}{1.582967in}}%
\pgfpathlineto{\pgfqpoint{3.994487in}{1.596910in}}%
\pgfpathlineto{\pgfqpoint{3.980421in}{1.592766in}}%
\pgfpathlineto{\pgfqpoint{3.966366in}{1.588804in}}%
\pgfpathlineto{\pgfqpoint{3.952321in}{1.585023in}}%
\pgfpathlineto{\pgfqpoint{3.938287in}{1.581425in}}%
\pgfpathlineto{\pgfqpoint{3.930115in}{1.567978in}}%
\pgfpathlineto{\pgfqpoint{3.921938in}{1.554573in}}%
\pgfpathlineto{\pgfqpoint{3.913757in}{1.541214in}}%
\pgfpathlineto{\pgfqpoint{3.905570in}{1.527907in}}%
\pgfpathclose%
\pgfusepath{fill}%
\end{pgfscope}%
\begin{pgfscope}%
\pgfpathrectangle{\pgfqpoint{1.150000in}{0.150000in}}{\pgfqpoint{5.700000in}{5.700000in}}%
\pgfusepath{clip}%
\pgfsetbuttcap%
\pgfsetroundjoin%
\definecolor{currentfill}{rgb}{0.271305,0.019942,0.347269}%
\pgfsetfillcolor{currentfill}%
\pgfsetfillopacity{0.800000}%
\pgfsetlinewidth{0.000000pt}%
\definecolor{currentstroke}{rgb}{0.000000,0.000000,0.000000}%
\pgfsetstrokecolor{currentstroke}%
\pgfsetdash{}{0pt}%
\pgfpathmoveto{\pgfqpoint{3.169341in}{1.369637in}}%
\pgfpathlineto{\pgfqpoint{3.183310in}{1.361534in}}%
\pgfpathlineto{\pgfqpoint{3.197279in}{1.353629in}}%
\pgfpathlineto{\pgfqpoint{3.211250in}{1.345923in}}%
\pgfpathlineto{\pgfqpoint{3.225222in}{1.338413in}}%
\pgfpathlineto{\pgfqpoint{3.233776in}{1.342079in}}%
\pgfpathlineto{\pgfqpoint{3.242316in}{1.346031in}}%
\pgfpathlineto{\pgfqpoint{3.250844in}{1.350262in}}%
\pgfpathlineto{\pgfqpoint{3.259360in}{1.354763in}}%
\pgfpathlineto{\pgfqpoint{3.245419in}{1.361545in}}%
\pgfpathlineto{\pgfqpoint{3.231481in}{1.368524in}}%
\pgfpathlineto{\pgfqpoint{3.217544in}{1.375700in}}%
\pgfpathlineto{\pgfqpoint{3.203609in}{1.383074in}}%
\pgfpathlineto{\pgfqpoint{3.195061in}{1.379288in}}%
\pgfpathlineto{\pgfqpoint{3.186501in}{1.375781in}}%
\pgfpathlineto{\pgfqpoint{3.177928in}{1.372562in}}%
\pgfpathlineto{\pgfqpoint{3.169341in}{1.369637in}}%
\pgfpathclose%
\pgfusepath{fill}%
\end{pgfscope}%
\begin{pgfscope}%
\pgfpathrectangle{\pgfqpoint{1.150000in}{0.150000in}}{\pgfqpoint{5.700000in}{5.700000in}}%
\pgfusepath{clip}%
\pgfsetbuttcap%
\pgfsetroundjoin%
\definecolor{currentfill}{rgb}{0.271828,0.209303,0.504434}%
\pgfsetfillcolor{currentfill}%
\pgfsetfillopacity{0.800000}%
\pgfsetlinewidth{0.000000pt}%
\definecolor{currentstroke}{rgb}{0.000000,0.000000,0.000000}%
\pgfsetstrokecolor{currentstroke}%
\pgfsetdash{}{0pt}%
\pgfpathmoveto{\pgfqpoint{4.116025in}{1.731232in}}%
\pgfpathlineto{\pgfqpoint{4.130145in}{1.737210in}}%
\pgfpathlineto{\pgfqpoint{4.144276in}{1.743369in}}%
\pgfpathlineto{\pgfqpoint{4.158421in}{1.749710in}}%
\pgfpathlineto{\pgfqpoint{4.172577in}{1.756233in}}%
\pgfpathlineto{\pgfqpoint{4.180704in}{1.771138in}}%
\pgfpathlineto{\pgfqpoint{4.188826in}{1.786019in}}%
\pgfpathlineto{\pgfqpoint{4.196944in}{1.800871in}}%
\pgfpathlineto{\pgfqpoint{4.205057in}{1.815691in}}%
\pgfpathlineto{\pgfqpoint{4.190900in}{1.808752in}}%
\pgfpathlineto{\pgfqpoint{4.176755in}{1.801995in}}%
\pgfpathlineto{\pgfqpoint{4.162624in}{1.795421in}}%
\pgfpathlineto{\pgfqpoint{4.148505in}{1.789029in}}%
\pgfpathlineto{\pgfqpoint{4.140391in}{1.774612in}}%
\pgfpathlineto{\pgfqpoint{4.132273in}{1.760171in}}%
\pgfpathlineto{\pgfqpoint{4.124151in}{1.745710in}}%
\pgfpathlineto{\pgfqpoint{4.116025in}{1.731232in}}%
\pgfpathclose%
\pgfusepath{fill}%
\end{pgfscope}%
\begin{pgfscope}%
\pgfpathrectangle{\pgfqpoint{1.150000in}{0.150000in}}{\pgfqpoint{5.700000in}{5.700000in}}%
\pgfusepath{clip}%
\pgfsetbuttcap%
\pgfsetroundjoin%
\definecolor{currentfill}{rgb}{0.267004,0.004874,0.329415}%
\pgfsetfillcolor{currentfill}%
\pgfsetfillopacity{0.800000}%
\pgfsetlinewidth{0.000000pt}%
\definecolor{currentstroke}{rgb}{0.000000,0.000000,0.000000}%
\pgfsetstrokecolor{currentstroke}%
\pgfsetdash{}{0pt}%
\pgfpathmoveto{\pgfqpoint{3.460487in}{1.317376in}}%
\pgfpathlineto{\pgfqpoint{3.474445in}{1.313784in}}%
\pgfpathlineto{\pgfqpoint{3.488407in}{1.310380in}}%
\pgfpathlineto{\pgfqpoint{3.502374in}{1.307163in}}%
\pgfpathlineto{\pgfqpoint{3.516346in}{1.304133in}}%
\pgfpathlineto{\pgfqpoint{3.524710in}{1.312576in}}%
\pgfpathlineto{\pgfqpoint{3.533065in}{1.321217in}}%
\pgfpathlineto{\pgfqpoint{3.541412in}{1.330047in}}%
\pgfpathlineto{\pgfqpoint{3.549751in}{1.339062in}}%
\pgfpathlineto{\pgfqpoint{3.535798in}{1.341431in}}%
\pgfpathlineto{\pgfqpoint{3.521851in}{1.343987in}}%
\pgfpathlineto{\pgfqpoint{3.507909in}{1.346731in}}%
\pgfpathlineto{\pgfqpoint{3.493972in}{1.349662in}}%
\pgfpathlineto{\pgfqpoint{3.485614in}{1.341296in}}%
\pgfpathlineto{\pgfqpoint{3.477248in}{1.333122in}}%
\pgfpathlineto{\pgfqpoint{3.468872in}{1.325146in}}%
\pgfpathlineto{\pgfqpoint{3.460487in}{1.317376in}}%
\pgfpathclose%
\pgfusepath{fill}%
\end{pgfscope}%
\begin{pgfscope}%
\pgfpathrectangle{\pgfqpoint{1.150000in}{0.150000in}}{\pgfqpoint{5.700000in}{5.700000in}}%
\pgfusepath{clip}%
\pgfsetbuttcap%
\pgfsetroundjoin%
\definecolor{currentfill}{rgb}{0.147607,0.511733,0.557049}%
\pgfsetfillcolor{currentfill}%
\pgfsetfillopacity{0.800000}%
\pgfsetlinewidth{0.000000pt}%
\definecolor{currentstroke}{rgb}{0.000000,0.000000,0.000000}%
\pgfsetstrokecolor{currentstroke}%
\pgfsetdash{}{0pt}%
\pgfpathmoveto{\pgfqpoint{4.812357in}{2.574278in}}%
\pgfpathlineto{\pgfqpoint{4.826870in}{2.587463in}}%
\pgfpathlineto{\pgfqpoint{4.841401in}{2.600834in}}%
\pgfpathlineto{\pgfqpoint{4.855952in}{2.614391in}}%
\pgfpathlineto{\pgfqpoint{4.870521in}{2.628135in}}%
\pgfpathlineto{\pgfqpoint{4.878426in}{2.639843in}}%
\pgfpathlineto{\pgfqpoint{4.886325in}{2.651385in}}%
\pgfpathlineto{\pgfqpoint{4.894215in}{2.662763in}}%
\pgfpathlineto{\pgfqpoint{4.902099in}{2.673975in}}%
\pgfpathlineto{\pgfqpoint{4.887529in}{2.660178in}}%
\pgfpathlineto{\pgfqpoint{4.872979in}{2.646567in}}%
\pgfpathlineto{\pgfqpoint{4.858448in}{2.633142in}}%
\pgfpathlineto{\pgfqpoint{4.843935in}{2.619904in}}%
\pgfpathlineto{\pgfqpoint{4.836051in}{2.608732in}}%
\pgfpathlineto{\pgfqpoint{4.828160in}{2.597403in}}%
\pgfpathlineto{\pgfqpoint{4.820262in}{2.585919in}}%
\pgfpathlineto{\pgfqpoint{4.812357in}{2.574278in}}%
\pgfpathclose%
\pgfusepath{fill}%
\end{pgfscope}%
\begin{pgfscope}%
\pgfpathrectangle{\pgfqpoint{1.150000in}{0.150000in}}{\pgfqpoint{5.700000in}{5.700000in}}%
\pgfusepath{clip}%
\pgfsetbuttcap%
\pgfsetroundjoin%
\definecolor{currentfill}{rgb}{0.122312,0.633153,0.530398}%
\pgfsetfillcolor{currentfill}%
\pgfsetfillopacity{0.800000}%
\pgfsetlinewidth{0.000000pt}%
\definecolor{currentstroke}{rgb}{0.000000,0.000000,0.000000}%
\pgfsetstrokecolor{currentstroke}%
\pgfsetdash{}{0pt}%
\pgfpathmoveto{\pgfqpoint{5.144341in}{2.947255in}}%
\pgfpathlineto{\pgfqpoint{5.159079in}{2.962598in}}%
\pgfpathlineto{\pgfqpoint{5.173839in}{2.978128in}}%
\pgfpathlineto{\pgfqpoint{5.188620in}{2.993846in}}%
\pgfpathlineto{\pgfqpoint{5.203422in}{3.009752in}}%
\pgfpathlineto{\pgfqpoint{5.211152in}{3.017912in}}%
\pgfpathlineto{\pgfqpoint{5.218873in}{3.025895in}}%
\pgfpathlineto{\pgfqpoint{5.226584in}{3.033704in}}%
\pgfpathlineto{\pgfqpoint{5.234286in}{3.041339in}}%
\pgfpathlineto{\pgfqpoint{5.219490in}{3.025557in}}%
\pgfpathlineto{\pgfqpoint{5.204716in}{3.009963in}}%
\pgfpathlineto{\pgfqpoint{5.189963in}{2.994556in}}%
\pgfpathlineto{\pgfqpoint{5.175231in}{2.979337in}}%
\pgfpathlineto{\pgfqpoint{5.167522in}{2.971565in}}%
\pgfpathlineto{\pgfqpoint{5.159804in}{2.963629in}}%
\pgfpathlineto{\pgfqpoint{5.152077in}{2.955526in}}%
\pgfpathlineto{\pgfqpoint{5.144341in}{2.947255in}}%
\pgfpathclose%
\pgfusepath{fill}%
\end{pgfscope}%
\begin{pgfscope}%
\pgfpathrectangle{\pgfqpoint{1.150000in}{0.150000in}}{\pgfqpoint{5.700000in}{5.700000in}}%
\pgfusepath{clip}%
\pgfsetbuttcap%
\pgfsetroundjoin%
\definecolor{currentfill}{rgb}{0.175707,0.697900,0.491033}%
\pgfsetfillcolor{currentfill}%
\pgfsetfillopacity{0.800000}%
\pgfsetlinewidth{0.000000pt}%
\definecolor{currentstroke}{rgb}{0.000000,0.000000,0.000000}%
\pgfsetstrokecolor{currentstroke}%
\pgfsetdash{}{0pt}%
\pgfpathmoveto{\pgfqpoint{5.354882in}{3.160013in}}%
\pgfpathlineto{\pgfqpoint{5.369768in}{3.176379in}}%
\pgfpathlineto{\pgfqpoint{5.384677in}{3.192934in}}%
\pgfpathlineto{\pgfqpoint{5.399609in}{3.209676in}}%
\pgfpathlineto{\pgfqpoint{5.414563in}{3.226608in}}%
\pgfpathlineto{\pgfqpoint{5.422156in}{3.232341in}}%
\pgfpathlineto{\pgfqpoint{5.429738in}{3.237907in}}%
\pgfpathlineto{\pgfqpoint{5.437310in}{3.243308in}}%
\pgfpathlineto{\pgfqpoint{5.444872in}{3.248546in}}%
\pgfpathlineto{\pgfqpoint{5.429930in}{3.231848in}}%
\pgfpathlineto{\pgfqpoint{5.415010in}{3.215338in}}%
\pgfpathlineto{\pgfqpoint{5.400113in}{3.199016in}}%
\pgfpathlineto{\pgfqpoint{5.385238in}{3.182882in}}%
\pgfpathlineto{\pgfqpoint{5.377664in}{3.177398in}}%
\pgfpathlineto{\pgfqpoint{5.370080in}{3.171761in}}%
\pgfpathlineto{\pgfqpoint{5.362486in}{3.165967in}}%
\pgfpathlineto{\pgfqpoint{5.354882in}{3.160013in}}%
\pgfpathclose%
\pgfusepath{fill}%
\end{pgfscope}%
\begin{pgfscope}%
\pgfpathrectangle{\pgfqpoint{1.150000in}{0.150000in}}{\pgfqpoint{5.700000in}{5.700000in}}%
\pgfusepath{clip}%
\pgfsetbuttcap%
\pgfsetroundjoin%
\definecolor{currentfill}{rgb}{0.199430,0.387607,0.554642}%
\pgfsetfillcolor{currentfill}%
\pgfsetfillopacity{0.800000}%
\pgfsetlinewidth{0.000000pt}%
\definecolor{currentstroke}{rgb}{0.000000,0.000000,0.000000}%
\pgfsetstrokecolor{currentstroke}%
\pgfsetdash{}{0pt}%
\pgfpathmoveto{\pgfqpoint{2.287915in}{2.299504in}}%
\pgfpathlineto{\pgfqpoint{2.302272in}{2.276425in}}%
\pgfpathlineto{\pgfqpoint{2.316615in}{2.253636in}}%
\pgfpathlineto{\pgfqpoint{2.330945in}{2.231134in}}%
\pgfpathlineto{\pgfqpoint{2.345263in}{2.208917in}}%
\pgfpathlineto{\pgfqpoint{2.354654in}{2.199396in}}%
\pgfpathlineto{\pgfqpoint{2.364018in}{2.190334in}}%
\pgfpathlineto{\pgfqpoint{2.373355in}{2.181721in}}%
\pgfpathlineto{\pgfqpoint{2.382664in}{2.173550in}}%
\pgfpathlineto{\pgfqpoint{2.368416in}{2.194946in}}%
\pgfpathlineto{\pgfqpoint{2.354156in}{2.216626in}}%
\pgfpathlineto{\pgfqpoint{2.339883in}{2.238590in}}%
\pgfpathlineto{\pgfqpoint{2.325598in}{2.260843in}}%
\pgfpathlineto{\pgfqpoint{2.316220in}{2.269822in}}%
\pgfpathlineto{\pgfqpoint{2.306814in}{2.279253in}}%
\pgfpathlineto{\pgfqpoint{2.297379in}{2.289144in}}%
\pgfpathlineto{\pgfqpoint{2.287915in}{2.299504in}}%
\pgfpathclose%
\pgfusepath{fill}%
\end{pgfscope}%
\begin{pgfscope}%
\pgfpathrectangle{\pgfqpoint{1.150000in}{0.150000in}}{\pgfqpoint{5.700000in}{5.700000in}}%
\pgfusepath{clip}%
\pgfsetbuttcap%
\pgfsetroundjoin%
\definecolor{currentfill}{rgb}{0.165117,0.467423,0.558141}%
\pgfsetfillcolor{currentfill}%
\pgfsetfillopacity{0.800000}%
\pgfsetlinewidth{0.000000pt}%
\definecolor{currentstroke}{rgb}{0.000000,0.000000,0.000000}%
\pgfsetstrokecolor{currentstroke}%
\pgfsetdash{}{0pt}%
\pgfpathmoveto{\pgfqpoint{4.691012in}{2.425526in}}%
\pgfpathlineto{\pgfqpoint{4.705450in}{2.437758in}}%
\pgfpathlineto{\pgfqpoint{4.719905in}{2.450176in}}%
\pgfpathlineto{\pgfqpoint{4.734379in}{2.462780in}}%
\pgfpathlineto{\pgfqpoint{4.748870in}{2.475568in}}%
\pgfpathlineto{\pgfqpoint{4.756830in}{2.488444in}}%
\pgfpathlineto{\pgfqpoint{4.764782in}{2.501168in}}%
\pgfpathlineto{\pgfqpoint{4.772728in}{2.513739in}}%
\pgfpathlineto{\pgfqpoint{4.780668in}{2.526157in}}%
\pgfpathlineto{\pgfqpoint{4.766174in}{2.513245in}}%
\pgfpathlineto{\pgfqpoint{4.751699in}{2.500520in}}%
\pgfpathlineto{\pgfqpoint{4.737242in}{2.487980in}}%
\pgfpathlineto{\pgfqpoint{4.722803in}{2.475625in}}%
\pgfpathlineto{\pgfqpoint{4.714865in}{2.463317in}}%
\pgfpathlineto{\pgfqpoint{4.706920in}{2.450864in}}%
\pgfpathlineto{\pgfqpoint{4.698969in}{2.438267in}}%
\pgfpathlineto{\pgfqpoint{4.691012in}{2.425526in}}%
\pgfpathclose%
\pgfusepath{fill}%
\end{pgfscope}%
\begin{pgfscope}%
\pgfpathrectangle{\pgfqpoint{1.150000in}{0.150000in}}{\pgfqpoint{5.700000in}{5.700000in}}%
\pgfusepath{clip}%
\pgfsetbuttcap%
\pgfsetroundjoin%
\definecolor{currentfill}{rgb}{0.335885,0.777018,0.402049}%
\pgfsetfillcolor{currentfill}%
\pgfsetfillopacity{0.800000}%
\pgfsetlinewidth{0.000000pt}%
\definecolor{currentstroke}{rgb}{0.000000,0.000000,0.000000}%
\pgfsetstrokecolor{currentstroke}%
\pgfsetdash{}{0pt}%
\pgfpathmoveto{\pgfqpoint{5.654779in}{3.432938in}}%
\pgfpathlineto{\pgfqpoint{5.669882in}{3.450414in}}%
\pgfpathlineto{\pgfqpoint{5.685008in}{3.468079in}}%
\pgfpathlineto{\pgfqpoint{5.700159in}{3.485932in}}%
\pgfpathlineto{\pgfqpoint{5.715334in}{3.503975in}}%
\pgfpathlineto{\pgfqpoint{5.722702in}{3.506310in}}%
\pgfpathlineto{\pgfqpoint{5.730060in}{3.508506in}}%
\pgfpathlineto{\pgfqpoint{5.737406in}{3.510568in}}%
\pgfpathlineto{\pgfqpoint{5.744741in}{3.512499in}}%
\pgfpathlineto{\pgfqpoint{5.729588in}{3.494837in}}%
\pgfpathlineto{\pgfqpoint{5.714459in}{3.477364in}}%
\pgfpathlineto{\pgfqpoint{5.699353in}{3.460079in}}%
\pgfpathlineto{\pgfqpoint{5.684272in}{3.442981in}}%
\pgfpathlineto{\pgfqpoint{5.676915in}{3.440658in}}%
\pgfpathlineto{\pgfqpoint{5.669547in}{3.438212in}}%
\pgfpathlineto{\pgfqpoint{5.662168in}{3.435640in}}%
\pgfpathlineto{\pgfqpoint{5.654779in}{3.432938in}}%
\pgfpathclose%
\pgfusepath{fill}%
\end{pgfscope}%
\begin{pgfscope}%
\pgfpathrectangle{\pgfqpoint{1.150000in}{0.150000in}}{\pgfqpoint{5.700000in}{5.700000in}}%
\pgfusepath{clip}%
\pgfsetbuttcap%
\pgfsetroundjoin%
\definecolor{currentfill}{rgb}{0.280894,0.078907,0.402329}%
\pgfsetfillcolor{currentfill}%
\pgfsetfillopacity{0.800000}%
\pgfsetlinewidth{0.000000pt}%
\definecolor{currentstroke}{rgb}{0.000000,0.000000,0.000000}%
\pgfsetstrokecolor{currentstroke}%
\pgfsetdash{}{0pt}%
\pgfpathmoveto{\pgfqpoint{2.966766in}{1.483258in}}%
\pgfpathlineto{\pgfqpoint{2.980779in}{1.471936in}}%
\pgfpathlineto{\pgfqpoint{2.994790in}{1.460823in}}%
\pgfpathlineto{\pgfqpoint{3.008801in}{1.449920in}}%
\pgfpathlineto{\pgfqpoint{3.022809in}{1.439225in}}%
\pgfpathlineto{\pgfqpoint{3.031533in}{1.439339in}}%
\pgfpathlineto{\pgfqpoint{3.040240in}{1.439797in}}%
\pgfpathlineto{\pgfqpoint{3.048931in}{1.440589in}}%
\pgfpathlineto{\pgfqpoint{3.057606in}{1.441708in}}%
\pgfpathlineto{\pgfqpoint{3.043639in}{1.451637in}}%
\pgfpathlineto{\pgfqpoint{3.029670in}{1.461772in}}%
\pgfpathlineto{\pgfqpoint{3.015701in}{1.472116in}}%
\pgfpathlineto{\pgfqpoint{3.001730in}{1.482670in}}%
\pgfpathlineto{\pgfqpoint{2.993014in}{1.482305in}}%
\pgfpathlineto{\pgfqpoint{2.984281in}{1.482276in}}%
\pgfpathlineto{\pgfqpoint{2.975532in}{1.482591in}}%
\pgfpathlineto{\pgfqpoint{2.966766in}{1.483258in}}%
\pgfpathclose%
\pgfusepath{fill}%
\end{pgfscope}%
\begin{pgfscope}%
\pgfpathrectangle{\pgfqpoint{1.150000in}{0.150000in}}{\pgfqpoint{5.700000in}{5.700000in}}%
\pgfusepath{clip}%
\pgfsetbuttcap%
\pgfsetroundjoin%
\definecolor{currentfill}{rgb}{0.185556,0.418570,0.556753}%
\pgfsetfillcolor{currentfill}%
\pgfsetfillopacity{0.800000}%
\pgfsetlinewidth{0.000000pt}%
\definecolor{currentstroke}{rgb}{0.000000,0.000000,0.000000}%
\pgfsetstrokecolor{currentstroke}%
\pgfsetdash{}{0pt}%
\pgfpathmoveto{\pgfqpoint{4.569577in}{2.272864in}}%
\pgfpathlineto{\pgfqpoint{4.583941in}{2.284011in}}%
\pgfpathlineto{\pgfqpoint{4.598323in}{2.295343in}}%
\pgfpathlineto{\pgfqpoint{4.612721in}{2.306859in}}%
\pgfpathlineto{\pgfqpoint{4.627136in}{2.318559in}}%
\pgfpathlineto{\pgfqpoint{4.635141in}{2.332409in}}%
\pgfpathlineto{\pgfqpoint{4.643140in}{2.346124in}}%
\pgfpathlineto{\pgfqpoint{4.651134in}{2.359704in}}%
\pgfpathlineto{\pgfqpoint{4.659122in}{2.373147in}}%
\pgfpathlineto{\pgfqpoint{4.644703in}{2.361256in}}%
\pgfpathlineto{\pgfqpoint{4.630302in}{2.349549in}}%
\pgfpathlineto{\pgfqpoint{4.615919in}{2.338028in}}%
\pgfpathlineto{\pgfqpoint{4.601552in}{2.326691in}}%
\pgfpathlineto{\pgfqpoint{4.593566in}{2.313426in}}%
\pgfpathlineto{\pgfqpoint{4.585575in}{2.300031in}}%
\pgfpathlineto{\pgfqpoint{4.577579in}{2.286510in}}%
\pgfpathlineto{\pgfqpoint{4.569577in}{2.272864in}}%
\pgfpathclose%
\pgfusepath{fill}%
\end{pgfscope}%
\begin{pgfscope}%
\pgfpathrectangle{\pgfqpoint{1.150000in}{0.150000in}}{\pgfqpoint{5.700000in}{5.700000in}}%
\pgfusepath{clip}%
\pgfsetbuttcap%
\pgfsetroundjoin%
\definecolor{currentfill}{rgb}{0.281412,0.155834,0.469201}%
\pgfsetfillcolor{currentfill}%
\pgfsetfillopacity{0.800000}%
\pgfsetlinewidth{0.000000pt}%
\definecolor{currentstroke}{rgb}{0.000000,0.000000,0.000000}%
\pgfsetstrokecolor{currentstroke}%
\pgfsetdash{}{0pt}%
\pgfpathmoveto{\pgfqpoint{3.994487in}{1.596910in}}%
\pgfpathlineto{\pgfqpoint{4.008563in}{1.601237in}}%
\pgfpathlineto{\pgfqpoint{4.022651in}{1.605744in}}%
\pgfpathlineto{\pgfqpoint{4.036750in}{1.610433in}}%
\pgfpathlineto{\pgfqpoint{4.050860in}{1.615304in}}%
\pgfpathlineto{\pgfqpoint{4.059021in}{1.629766in}}%
\pgfpathlineto{\pgfqpoint{4.067177in}{1.644245in}}%
\pgfpathlineto{\pgfqpoint{4.075329in}{1.658737in}}%
\pgfpathlineto{\pgfqpoint{4.083477in}{1.673237in}}%
\pgfpathlineto{\pgfqpoint{4.069368in}{1.667888in}}%
\pgfpathlineto{\pgfqpoint{4.055272in}{1.662721in}}%
\pgfpathlineto{\pgfqpoint{4.041186in}{1.657736in}}%
\pgfpathlineto{\pgfqpoint{4.027113in}{1.652933in}}%
\pgfpathlineto{\pgfqpoint{4.018963in}{1.638898in}}%
\pgfpathlineto{\pgfqpoint{4.010809in}{1.624880in}}%
\pgfpathlineto{\pgfqpoint{4.002650in}{1.610883in}}%
\pgfpathlineto{\pgfqpoint{3.994487in}{1.596910in}}%
\pgfpathclose%
\pgfusepath{fill}%
\end{pgfscope}%
\begin{pgfscope}%
\pgfpathrectangle{\pgfqpoint{1.150000in}{0.150000in}}{\pgfqpoint{5.700000in}{5.700000in}}%
\pgfusepath{clip}%
\pgfsetbuttcap%
\pgfsetroundjoin%
\definecolor{currentfill}{rgb}{0.210503,0.363727,0.552206}%
\pgfsetfillcolor{currentfill}%
\pgfsetfillopacity{0.800000}%
\pgfsetlinewidth{0.000000pt}%
\definecolor{currentstroke}{rgb}{0.000000,0.000000,0.000000}%
\pgfsetstrokecolor{currentstroke}%
\pgfsetdash{}{0pt}%
\pgfpathmoveto{\pgfqpoint{4.448094in}{2.118560in}}%
\pgfpathlineto{\pgfqpoint{4.462389in}{2.128492in}}%
\pgfpathlineto{\pgfqpoint{4.476699in}{2.138607in}}%
\pgfpathlineto{\pgfqpoint{4.491025in}{2.148906in}}%
\pgfpathlineto{\pgfqpoint{4.505367in}{2.159389in}}%
\pgfpathlineto{\pgfqpoint{4.513411in}{2.173974in}}%
\pgfpathlineto{\pgfqpoint{4.521450in}{2.188448in}}%
\pgfpathlineto{\pgfqpoint{4.529485in}{2.202810in}}%
\pgfpathlineto{\pgfqpoint{4.537514in}{2.217058in}}%
\pgfpathlineto{\pgfqpoint{4.523168in}{2.206318in}}%
\pgfpathlineto{\pgfqpoint{4.508839in}{2.195763in}}%
\pgfpathlineto{\pgfqpoint{4.494526in}{2.185392in}}%
\pgfpathlineto{\pgfqpoint{4.480229in}{2.175204in}}%
\pgfpathlineto{\pgfqpoint{4.472203in}{2.161200in}}%
\pgfpathlineto{\pgfqpoint{4.464172in}{2.147090in}}%
\pgfpathlineto{\pgfqpoint{4.456135in}{2.132876in}}%
\pgfpathlineto{\pgfqpoint{4.448094in}{2.118560in}}%
\pgfpathclose%
\pgfusepath{fill}%
\end{pgfscope}%
\begin{pgfscope}%
\pgfpathrectangle{\pgfqpoint{1.150000in}{0.150000in}}{\pgfqpoint{5.700000in}{5.700000in}}%
\pgfusepath{clip}%
\pgfsetbuttcap%
\pgfsetroundjoin%
\definecolor{currentfill}{rgb}{0.137770,0.537492,0.554906}%
\pgfsetfillcolor{currentfill}%
\pgfsetfillopacity{0.800000}%
\pgfsetlinewidth{0.000000pt}%
\definecolor{currentstroke}{rgb}{0.000000,0.000000,0.000000}%
\pgfsetstrokecolor{currentstroke}%
\pgfsetdash{}{0pt}%
\pgfpathmoveto{\pgfqpoint{2.036766in}{2.770487in}}%
\pgfpathlineto{\pgfqpoint{2.051371in}{2.741895in}}%
\pgfpathlineto{\pgfqpoint{2.065957in}{2.713650in}}%
\pgfpathlineto{\pgfqpoint{2.080523in}{2.685748in}}%
\pgfpathlineto{\pgfqpoint{2.095071in}{2.658186in}}%
\pgfpathlineto{\pgfqpoint{2.104718in}{2.646381in}}%
\pgfpathlineto{\pgfqpoint{2.114333in}{2.635054in}}%
\pgfpathlineto{\pgfqpoint{2.123917in}{2.624195in}}%
\pgfpathlineto{\pgfqpoint{2.133470in}{2.613797in}}%
\pgfpathlineto{\pgfqpoint{2.119001in}{2.640540in}}%
\pgfpathlineto{\pgfqpoint{2.104513in}{2.667620in}}%
\pgfpathlineto{\pgfqpoint{2.090007in}{2.695041in}}%
\pgfpathlineto{\pgfqpoint{2.075483in}{2.722806in}}%
\pgfpathlineto{\pgfqpoint{2.065852in}{2.734011in}}%
\pgfpathlineto{\pgfqpoint{2.056189in}{2.745687in}}%
\pgfpathlineto{\pgfqpoint{2.046494in}{2.757843in}}%
\pgfpathlineto{\pgfqpoint{2.036766in}{2.770487in}}%
\pgfpathclose%
\pgfusepath{fill}%
\end{pgfscope}%
\begin{pgfscope}%
\pgfpathrectangle{\pgfqpoint{1.150000in}{0.150000in}}{\pgfqpoint{5.700000in}{5.700000in}}%
\pgfusepath{clip}%
\pgfsetbuttcap%
\pgfsetroundjoin%
\definecolor{currentfill}{rgb}{0.235526,0.309527,0.542944}%
\pgfsetfillcolor{currentfill}%
\pgfsetfillopacity{0.800000}%
\pgfsetlinewidth{0.000000pt}%
\definecolor{currentstroke}{rgb}{0.000000,0.000000,0.000000}%
\pgfsetstrokecolor{currentstroke}%
\pgfsetdash{}{0pt}%
\pgfpathmoveto{\pgfqpoint{4.326588in}{1.965203in}}%
\pgfpathlineto{\pgfqpoint{4.340817in}{1.973791in}}%
\pgfpathlineto{\pgfqpoint{4.355061in}{1.982563in}}%
\pgfpathlineto{\pgfqpoint{4.369319in}{1.991517in}}%
\pgfpathlineto{\pgfqpoint{4.383592in}{2.000655in}}%
\pgfpathlineto{\pgfqpoint{4.391671in}{2.015697in}}%
\pgfpathlineto{\pgfqpoint{4.399746in}{2.030658in}}%
\pgfpathlineto{\pgfqpoint{4.407815in}{2.045535in}}%
\pgfpathlineto{\pgfqpoint{4.415881in}{2.060324in}}%
\pgfpathlineto{\pgfqpoint{4.401604in}{2.050865in}}%
\pgfpathlineto{\pgfqpoint{4.387343in}{2.041589in}}%
\pgfpathlineto{\pgfqpoint{4.373097in}{2.032497in}}%
\pgfpathlineto{\pgfqpoint{4.358865in}{2.023587in}}%
\pgfpathlineto{\pgfqpoint{4.350803in}{2.009106in}}%
\pgfpathlineto{\pgfqpoint{4.342736in}{1.994547in}}%
\pgfpathlineto{\pgfqpoint{4.334664in}{1.979911in}}%
\pgfpathlineto{\pgfqpoint{4.326588in}{1.965203in}}%
\pgfpathclose%
\pgfusepath{fill}%
\end{pgfscope}%
\begin{pgfscope}%
\pgfpathrectangle{\pgfqpoint{1.150000in}{0.150000in}}{\pgfqpoint{5.700000in}{5.700000in}}%
\pgfusepath{clip}%
\pgfsetbuttcap%
\pgfsetroundjoin%
\definecolor{currentfill}{rgb}{0.121148,0.592739,0.544641}%
\pgfsetfillcolor{currentfill}%
\pgfsetfillopacity{0.800000}%
\pgfsetlinewidth{0.000000pt}%
\definecolor{currentstroke}{rgb}{0.000000,0.000000,0.000000}%
\pgfsetstrokecolor{currentstroke}%
\pgfsetdash{}{0pt}%
\pgfpathmoveto{\pgfqpoint{5.023348in}{2.814822in}}%
\pgfpathlineto{\pgfqpoint{5.038013in}{2.829555in}}%
\pgfpathlineto{\pgfqpoint{5.052698in}{2.844475in}}%
\pgfpathlineto{\pgfqpoint{5.067403in}{2.859582in}}%
\pgfpathlineto{\pgfqpoint{5.082129in}{2.874878in}}%
\pgfpathlineto{\pgfqpoint{5.089936in}{2.884540in}}%
\pgfpathlineto{\pgfqpoint{5.097735in}{2.894025in}}%
\pgfpathlineto{\pgfqpoint{5.105525in}{2.903333in}}%
\pgfpathlineto{\pgfqpoint{5.113306in}{2.912465in}}%
\pgfpathlineto{\pgfqpoint{5.098583in}{2.897222in}}%
\pgfpathlineto{\pgfqpoint{5.083881in}{2.882167in}}%
\pgfpathlineto{\pgfqpoint{5.069200in}{2.867298in}}%
\pgfpathlineto{\pgfqpoint{5.054538in}{2.852617in}}%
\pgfpathlineto{\pgfqpoint{5.046753in}{2.843420in}}%
\pgfpathlineto{\pgfqpoint{5.038960in}{2.834056in}}%
\pgfpathlineto{\pgfqpoint{5.031158in}{2.824523in}}%
\pgfpathlineto{\pgfqpoint{5.023348in}{2.814822in}}%
\pgfpathclose%
\pgfusepath{fill}%
\end{pgfscope}%
\begin{pgfscope}%
\pgfpathrectangle{\pgfqpoint{1.150000in}{0.150000in}}{\pgfqpoint{5.700000in}{5.700000in}}%
\pgfusepath{clip}%
\pgfsetbuttcap%
\pgfsetroundjoin%
\definecolor{currentfill}{rgb}{0.267004,0.004874,0.329415}%
\pgfsetfillcolor{currentfill}%
\pgfsetfillopacity{0.800000}%
\pgfsetlinewidth{0.000000pt}%
\definecolor{currentstroke}{rgb}{0.000000,0.000000,0.000000}%
\pgfsetstrokecolor{currentstroke}%
\pgfsetdash{}{0pt}%
\pgfpathmoveto{\pgfqpoint{3.370971in}{1.307508in}}%
\pgfpathlineto{\pgfqpoint{3.384936in}{1.302466in}}%
\pgfpathlineto{\pgfqpoint{3.398905in}{1.297614in}}%
\pgfpathlineto{\pgfqpoint{3.412877in}{1.292952in}}%
\pgfpathlineto{\pgfqpoint{3.426853in}{1.288478in}}%
\pgfpathlineto{\pgfqpoint{3.435277in}{1.295361in}}%
\pgfpathlineto{\pgfqpoint{3.443690in}{1.302476in}}%
\pgfpathlineto{\pgfqpoint{3.452093in}{1.309817in}}%
\pgfpathlineto{\pgfqpoint{3.460487in}{1.317376in}}%
\pgfpathlineto{\pgfqpoint{3.446534in}{1.321156in}}%
\pgfpathlineto{\pgfqpoint{3.432586in}{1.325125in}}%
\pgfpathlineto{\pgfqpoint{3.418642in}{1.329284in}}%
\pgfpathlineto{\pgfqpoint{3.404701in}{1.333633in}}%
\pgfpathlineto{\pgfqpoint{3.396284in}{1.326755in}}%
\pgfpathlineto{\pgfqpoint{3.387857in}{1.320104in}}%
\pgfpathlineto{\pgfqpoint{3.379419in}{1.313686in}}%
\pgfpathlineto{\pgfqpoint{3.370971in}{1.307508in}}%
\pgfpathclose%
\pgfusepath{fill}%
\end{pgfscope}%
\begin{pgfscope}%
\pgfpathrectangle{\pgfqpoint{1.150000in}{0.150000in}}{\pgfqpoint{5.700000in}{5.700000in}}%
\pgfusepath{clip}%
\pgfsetbuttcap%
\pgfsetroundjoin%
\definecolor{currentfill}{rgb}{0.183898,0.422383,0.556944}%
\pgfsetfillcolor{currentfill}%
\pgfsetfillopacity{0.800000}%
\pgfsetlinewidth{0.000000pt}%
\definecolor{currentstroke}{rgb}{0.000000,0.000000,0.000000}%
\pgfsetstrokecolor{currentstroke}%
\pgfsetdash{}{0pt}%
\pgfpathmoveto{\pgfqpoint{2.230352in}{2.394768in}}%
\pgfpathlineto{\pgfqpoint{2.244764in}{2.370505in}}%
\pgfpathlineto{\pgfqpoint{2.259162in}{2.346541in}}%
\pgfpathlineto{\pgfqpoint{2.273545in}{2.322875in}}%
\pgfpathlineto{\pgfqpoint{2.287915in}{2.299504in}}%
\pgfpathlineto{\pgfqpoint{2.297379in}{2.289144in}}%
\pgfpathlineto{\pgfqpoint{2.306814in}{2.279253in}}%
\pgfpathlineto{\pgfqpoint{2.316220in}{2.269822in}}%
\pgfpathlineto{\pgfqpoint{2.325598in}{2.260843in}}%
\pgfpathlineto{\pgfqpoint{2.311300in}{2.283386in}}%
\pgfpathlineto{\pgfqpoint{2.296989in}{2.306222in}}%
\pgfpathlineto{\pgfqpoint{2.282664in}{2.329353in}}%
\pgfpathlineto{\pgfqpoint{2.268326in}{2.352782in}}%
\pgfpathlineto{\pgfqpoint{2.258876in}{2.362576in}}%
\pgfpathlineto{\pgfqpoint{2.249398in}{2.372833in}}%
\pgfpathlineto{\pgfqpoint{2.239890in}{2.383561in}}%
\pgfpathlineto{\pgfqpoint{2.230352in}{2.394768in}}%
\pgfpathclose%
\pgfusepath{fill}%
\end{pgfscope}%
\begin{pgfscope}%
\pgfpathrectangle{\pgfqpoint{1.150000in}{0.150000in}}{\pgfqpoint{5.700000in}{5.700000in}}%
\pgfusepath{clip}%
\pgfsetbuttcap%
\pgfsetroundjoin%
\definecolor{currentfill}{rgb}{0.276022,0.044167,0.370164}%
\pgfsetfillcolor{currentfill}%
\pgfsetfillopacity{0.800000}%
\pgfsetlinewidth{0.000000pt}%
\definecolor{currentstroke}{rgb}{0.000000,0.000000,0.000000}%
\pgfsetstrokecolor{currentstroke}%
\pgfsetdash{}{0pt}%
\pgfpathmoveto{\pgfqpoint{3.694743in}{1.369597in}}%
\pgfpathlineto{\pgfqpoint{3.708739in}{1.369522in}}%
\pgfpathlineto{\pgfqpoint{3.722743in}{1.369629in}}%
\pgfpathlineto{\pgfqpoint{3.736754in}{1.369920in}}%
\pgfpathlineto{\pgfqpoint{3.750773in}{1.370393in}}%
\pgfpathlineto{\pgfqpoint{3.759033in}{1.382065in}}%
\pgfpathlineto{\pgfqpoint{3.767287in}{1.393860in}}%
\pgfpathlineto{\pgfqpoint{3.775535in}{1.405773in}}%
\pgfpathlineto{\pgfqpoint{3.783777in}{1.417797in}}%
\pgfpathlineto{\pgfqpoint{3.769768in}{1.416725in}}%
\pgfpathlineto{\pgfqpoint{3.755768in}{1.415835in}}%
\pgfpathlineto{\pgfqpoint{3.741776in}{1.415128in}}%
\pgfpathlineto{\pgfqpoint{3.727791in}{1.414604in}}%
\pgfpathlineto{\pgfqpoint{3.719539in}{1.403167in}}%
\pgfpathlineto{\pgfqpoint{3.711280in}{1.391849in}}%
\pgfpathlineto{\pgfqpoint{3.703015in}{1.380657in}}%
\pgfpathlineto{\pgfqpoint{3.694743in}{1.369597in}}%
\pgfpathclose%
\pgfusepath{fill}%
\end{pgfscope}%
\begin{pgfscope}%
\pgfpathrectangle{\pgfqpoint{1.150000in}{0.150000in}}{\pgfqpoint{5.700000in}{5.700000in}}%
\pgfusepath{clip}%
\pgfsetbuttcap%
\pgfsetroundjoin%
\definecolor{currentfill}{rgb}{0.269944,0.014625,0.341379}%
\pgfsetfillcolor{currentfill}%
\pgfsetfillopacity{0.800000}%
\pgfsetlinewidth{0.000000pt}%
\definecolor{currentstroke}{rgb}{0.000000,0.000000,0.000000}%
\pgfsetstrokecolor{currentstroke}%
\pgfsetdash{}{0pt}%
\pgfpathmoveto{\pgfqpoint{3.225222in}{1.338413in}}%
\pgfpathlineto{\pgfqpoint{3.239196in}{1.331100in}}%
\pgfpathlineto{\pgfqpoint{3.253172in}{1.323982in}}%
\pgfpathlineto{\pgfqpoint{3.267149in}{1.317059in}}%
\pgfpathlineto{\pgfqpoint{3.281128in}{1.310331in}}%
\pgfpathlineto{\pgfqpoint{3.289650in}{1.314736in}}%
\pgfpathlineto{\pgfqpoint{3.298160in}{1.319420in}}%
\pgfpathlineto{\pgfqpoint{3.306658in}{1.324373in}}%
\pgfpathlineto{\pgfqpoint{3.315144in}{1.329590in}}%
\pgfpathlineto{\pgfqpoint{3.301194in}{1.335591in}}%
\pgfpathlineto{\pgfqpoint{3.287247in}{1.341787in}}%
\pgfpathlineto{\pgfqpoint{3.273302in}{1.348178in}}%
\pgfpathlineto{\pgfqpoint{3.259360in}{1.354763in}}%
\pgfpathlineto{\pgfqpoint{3.250844in}{1.350262in}}%
\pgfpathlineto{\pgfqpoint{3.242316in}{1.346031in}}%
\pgfpathlineto{\pgfqpoint{3.233776in}{1.342079in}}%
\pgfpathlineto{\pgfqpoint{3.225222in}{1.338413in}}%
\pgfpathclose%
\pgfusepath{fill}%
\end{pgfscope}%
\begin{pgfscope}%
\pgfpathrectangle{\pgfqpoint{1.150000in}{0.150000in}}{\pgfqpoint{5.700000in}{5.700000in}}%
\pgfusepath{clip}%
\pgfsetbuttcap%
\pgfsetroundjoin%
\definecolor{currentfill}{rgb}{0.258965,0.251537,0.524736}%
\pgfsetfillcolor{currentfill}%
\pgfsetfillopacity{0.800000}%
\pgfsetlinewidth{0.000000pt}%
\definecolor{currentstroke}{rgb}{0.000000,0.000000,0.000000}%
\pgfsetstrokecolor{currentstroke}%
\pgfsetdash{}{0pt}%
\pgfpathmoveto{\pgfqpoint{4.205057in}{1.815691in}}%
\pgfpathlineto{\pgfqpoint{4.219228in}{1.822812in}}%
\pgfpathlineto{\pgfqpoint{4.233412in}{1.830115in}}%
\pgfpathlineto{\pgfqpoint{4.247609in}{1.837601in}}%
\pgfpathlineto{\pgfqpoint{4.261820in}{1.845268in}}%
\pgfpathlineto{\pgfqpoint{4.269931in}{1.860449in}}%
\pgfpathlineto{\pgfqpoint{4.278038in}{1.875584in}}%
\pgfpathlineto{\pgfqpoint{4.286140in}{1.890667in}}%
\pgfpathlineto{\pgfqpoint{4.294238in}{1.905696in}}%
\pgfpathlineto{\pgfqpoint{4.280025in}{1.897643in}}%
\pgfpathlineto{\pgfqpoint{4.265827in}{1.889773in}}%
\pgfpathlineto{\pgfqpoint{4.251641in}{1.882085in}}%
\pgfpathlineto{\pgfqpoint{4.237470in}{1.874579in}}%
\pgfpathlineto{\pgfqpoint{4.229373in}{1.859923in}}%
\pgfpathlineto{\pgfqpoint{4.221272in}{1.845221in}}%
\pgfpathlineto{\pgfqpoint{4.213167in}{1.830476in}}%
\pgfpathlineto{\pgfqpoint{4.205057in}{1.815691in}}%
\pgfpathclose%
\pgfusepath{fill}%
\end{pgfscope}%
\begin{pgfscope}%
\pgfpathrectangle{\pgfqpoint{1.150000in}{0.150000in}}{\pgfqpoint{5.700000in}{5.700000in}}%
\pgfusepath{clip}%
\pgfsetbuttcap%
\pgfsetroundjoin%
\definecolor{currentfill}{rgb}{0.272594,0.025563,0.353093}%
\pgfsetfillcolor{currentfill}%
\pgfsetfillopacity{0.800000}%
\pgfsetlinewidth{0.000000pt}%
\definecolor{currentstroke}{rgb}{0.000000,0.000000,0.000000}%
\pgfsetstrokecolor{currentstroke}%
\pgfsetdash{}{0pt}%
\pgfpathmoveto{\pgfqpoint{3.605620in}{1.331447in}}%
\pgfpathlineto{\pgfqpoint{3.619603in}{1.330006in}}%
\pgfpathlineto{\pgfqpoint{3.633592in}{1.328749in}}%
\pgfpathlineto{\pgfqpoint{3.647587in}{1.327677in}}%
\pgfpathlineto{\pgfqpoint{3.661590in}{1.326787in}}%
\pgfpathlineto{\pgfqpoint{3.669889in}{1.337263in}}%
\pgfpathlineto{\pgfqpoint{3.678180in}{1.347894in}}%
\pgfpathlineto{\pgfqpoint{3.686465in}{1.358674in}}%
\pgfpathlineto{\pgfqpoint{3.694743in}{1.369597in}}%
\pgfpathlineto{\pgfqpoint{3.680755in}{1.369856in}}%
\pgfpathlineto{\pgfqpoint{3.666773in}{1.370298in}}%
\pgfpathlineto{\pgfqpoint{3.652799in}{1.370925in}}%
\pgfpathlineto{\pgfqpoint{3.638831in}{1.371736in}}%
\pgfpathlineto{\pgfqpoint{3.630539in}{1.361431in}}%
\pgfpathlineto{\pgfqpoint{3.622240in}{1.351277in}}%
\pgfpathlineto{\pgfqpoint{3.613934in}{1.341280in}}%
\pgfpathlineto{\pgfqpoint{3.605620in}{1.331447in}}%
\pgfpathclose%
\pgfusepath{fill}%
\end{pgfscope}%
\begin{pgfscope}%
\pgfpathrectangle{\pgfqpoint{1.150000in}{0.150000in}}{\pgfqpoint{5.700000in}{5.700000in}}%
\pgfusepath{clip}%
\pgfsetbuttcap%
\pgfsetroundjoin%
\definecolor{currentfill}{rgb}{0.280267,0.073417,0.397163}%
\pgfsetfillcolor{currentfill}%
\pgfsetfillopacity{0.800000}%
\pgfsetlinewidth{0.000000pt}%
\definecolor{currentstroke}{rgb}{0.000000,0.000000,0.000000}%
\pgfsetstrokecolor{currentstroke}%
\pgfsetdash{}{0pt}%
\pgfpathmoveto{\pgfqpoint{3.783777in}{1.417797in}}%
\pgfpathlineto{\pgfqpoint{3.797794in}{1.419053in}}%
\pgfpathlineto{\pgfqpoint{3.811819in}{1.420490in}}%
\pgfpathlineto{\pgfqpoint{3.825854in}{1.422109in}}%
\pgfpathlineto{\pgfqpoint{3.839897in}{1.423910in}}%
\pgfpathlineto{\pgfqpoint{3.848124in}{1.436622in}}%
\pgfpathlineto{\pgfqpoint{3.856346in}{1.449427in}}%
\pgfpathlineto{\pgfqpoint{3.864563in}{1.462320in}}%
\pgfpathlineto{\pgfqpoint{3.872775in}{1.475294in}}%
\pgfpathlineto{\pgfqpoint{3.858739in}{1.472923in}}%
\pgfpathlineto{\pgfqpoint{3.844713in}{1.470735in}}%
\pgfpathlineto{\pgfqpoint{3.830696in}{1.468728in}}%
\pgfpathlineto{\pgfqpoint{3.816687in}{1.466904in}}%
\pgfpathlineto{\pgfqpoint{3.808468in}{1.454487in}}%
\pgfpathlineto{\pgfqpoint{3.800243in}{1.442160in}}%
\pgfpathlineto{\pgfqpoint{3.792013in}{1.429928in}}%
\pgfpathlineto{\pgfqpoint{3.783777in}{1.417797in}}%
\pgfpathclose%
\pgfusepath{fill}%
\end{pgfscope}%
\begin{pgfscope}%
\pgfpathrectangle{\pgfqpoint{1.150000in}{0.150000in}}{\pgfqpoint{5.700000in}{5.700000in}}%
\pgfusepath{clip}%
\pgfsetbuttcap%
\pgfsetroundjoin%
\definecolor{currentfill}{rgb}{0.278791,0.062145,0.386592}%
\pgfsetfillcolor{currentfill}%
\pgfsetfillopacity{0.800000}%
\pgfsetlinewidth{0.000000pt}%
\definecolor{currentstroke}{rgb}{0.000000,0.000000,0.000000}%
\pgfsetstrokecolor{currentstroke}%
\pgfsetdash{}{0pt}%
\pgfpathmoveto{\pgfqpoint{3.022809in}{1.439225in}}%
\pgfpathlineto{\pgfqpoint{3.036817in}{1.428736in}}%
\pgfpathlineto{\pgfqpoint{3.050823in}{1.418454in}}%
\pgfpathlineto{\pgfqpoint{3.064829in}{1.408376in}}%
\pgfpathlineto{\pgfqpoint{3.078834in}{1.398503in}}%
\pgfpathlineto{\pgfqpoint{3.087517in}{1.399397in}}%
\pgfpathlineto{\pgfqpoint{3.096184in}{1.400624in}}%
\pgfpathlineto{\pgfqpoint{3.104836in}{1.402178in}}%
\pgfpathlineto{\pgfqpoint{3.113473in}{1.404050in}}%
\pgfpathlineto{\pgfqpoint{3.099507in}{1.413158in}}%
\pgfpathlineto{\pgfqpoint{3.085540in}{1.422470in}}%
\pgfpathlineto{\pgfqpoint{3.071574in}{1.431987in}}%
\pgfpathlineto{\pgfqpoint{3.057606in}{1.441708in}}%
\pgfpathlineto{\pgfqpoint{3.048931in}{1.440589in}}%
\pgfpathlineto{\pgfqpoint{3.040240in}{1.439797in}}%
\pgfpathlineto{\pgfqpoint{3.031533in}{1.439339in}}%
\pgfpathlineto{\pgfqpoint{3.022809in}{1.439225in}}%
\pgfpathclose%
\pgfusepath{fill}%
\end{pgfscope}%
\begin{pgfscope}%
\pgfpathrectangle{\pgfqpoint{1.150000in}{0.150000in}}{\pgfqpoint{5.700000in}{5.700000in}}%
\pgfusepath{clip}%
\pgfsetbuttcap%
\pgfsetroundjoin%
\definecolor{currentfill}{rgb}{0.226397,0.728888,0.462789}%
\pgfsetfillcolor{currentfill}%
\pgfsetfillopacity{0.800000}%
\pgfsetlinewidth{0.000000pt}%
\definecolor{currentstroke}{rgb}{0.000000,0.000000,0.000000}%
\pgfsetstrokecolor{currentstroke}%
\pgfsetdash{}{0pt}%
\pgfpathmoveto{\pgfqpoint{5.444872in}{3.248546in}}%
\pgfpathlineto{\pgfqpoint{5.459836in}{3.265433in}}%
\pgfpathlineto{\pgfqpoint{5.474824in}{3.282508in}}%
\pgfpathlineto{\pgfqpoint{5.489835in}{3.299772in}}%
\pgfpathlineto{\pgfqpoint{5.504869in}{3.317225in}}%
\pgfpathlineto{\pgfqpoint{5.512406in}{3.322047in}}%
\pgfpathlineto{\pgfqpoint{5.519933in}{3.326703in}}%
\pgfpathlineto{\pgfqpoint{5.527449in}{3.331198in}}%
\pgfpathlineto{\pgfqpoint{5.534953in}{3.335533in}}%
\pgfpathlineto{\pgfqpoint{5.519934in}{3.318351in}}%
\pgfpathlineto{\pgfqpoint{5.504937in}{3.301357in}}%
\pgfpathlineto{\pgfqpoint{5.489964in}{3.284552in}}%
\pgfpathlineto{\pgfqpoint{5.475013in}{3.267934in}}%
\pgfpathlineto{\pgfqpoint{5.467493in}{3.263316in}}%
\pgfpathlineto{\pgfqpoint{5.459963in}{3.258547in}}%
\pgfpathlineto{\pgfqpoint{5.452423in}{3.253625in}}%
\pgfpathlineto{\pgfqpoint{5.444872in}{3.248546in}}%
\pgfpathclose%
\pgfusepath{fill}%
\end{pgfscope}%
\begin{pgfscope}%
\pgfpathrectangle{\pgfqpoint{1.150000in}{0.150000in}}{\pgfqpoint{5.700000in}{5.700000in}}%
\pgfusepath{clip}%
\pgfsetbuttcap%
\pgfsetroundjoin%
\definecolor{currentfill}{rgb}{0.395174,0.797475,0.367757}%
\pgfsetfillcolor{currentfill}%
\pgfsetfillopacity{0.800000}%
\pgfsetlinewidth{0.000000pt}%
\definecolor{currentstroke}{rgb}{0.000000,0.000000,0.000000}%
\pgfsetstrokecolor{currentstroke}%
\pgfsetdash{}{0pt}%
\pgfpathmoveto{\pgfqpoint{5.744741in}{3.512499in}}%
\pgfpathlineto{\pgfqpoint{5.759919in}{3.530349in}}%
\pgfpathlineto{\pgfqpoint{5.775122in}{3.548389in}}%
\pgfpathlineto{\pgfqpoint{5.790349in}{3.566617in}}%
\pgfpathlineto{\pgfqpoint{5.805601in}{3.585036in}}%
\pgfpathlineto{\pgfqpoint{5.812902in}{3.586436in}}%
\pgfpathlineto{\pgfqpoint{5.820191in}{3.587706in}}%
\pgfpathlineto{\pgfqpoint{5.827469in}{3.588851in}}%
\pgfpathlineto{\pgfqpoint{5.834736in}{3.589875in}}%
\pgfpathlineto{\pgfqpoint{5.819509in}{3.571876in}}%
\pgfpathlineto{\pgfqpoint{5.804306in}{3.554066in}}%
\pgfpathlineto{\pgfqpoint{5.789127in}{3.536444in}}%
\pgfpathlineto{\pgfqpoint{5.773973in}{3.519009in}}%
\pgfpathlineto{\pgfqpoint{5.766681in}{3.517555in}}%
\pgfpathlineto{\pgfqpoint{5.759379in}{3.515988in}}%
\pgfpathlineto{\pgfqpoint{5.752066in}{3.514304in}}%
\pgfpathlineto{\pgfqpoint{5.744741in}{3.512499in}}%
\pgfpathclose%
\pgfusepath{fill}%
\end{pgfscope}%
\begin{pgfscope}%
\pgfpathrectangle{\pgfqpoint{1.150000in}{0.150000in}}{\pgfqpoint{5.700000in}{5.700000in}}%
\pgfusepath{clip}%
\pgfsetbuttcap%
\pgfsetroundjoin%
\definecolor{currentfill}{rgb}{0.140210,0.665859,0.513427}%
\pgfsetfillcolor{currentfill}%
\pgfsetfillopacity{0.800000}%
\pgfsetlinewidth{0.000000pt}%
\definecolor{currentstroke}{rgb}{0.000000,0.000000,0.000000}%
\pgfsetstrokecolor{currentstroke}%
\pgfsetdash{}{0pt}%
\pgfpathmoveto{\pgfqpoint{5.234286in}{3.041339in}}%
\pgfpathlineto{\pgfqpoint{5.249103in}{3.057309in}}%
\pgfpathlineto{\pgfqpoint{5.263942in}{3.073467in}}%
\pgfpathlineto{\pgfqpoint{5.278803in}{3.089813in}}%
\pgfpathlineto{\pgfqpoint{5.293686in}{3.106348in}}%
\pgfpathlineto{\pgfqpoint{5.301370in}{3.113665in}}%
\pgfpathlineto{\pgfqpoint{5.309045in}{3.120804in}}%
\pgfpathlineto{\pgfqpoint{5.316709in}{3.127766in}}%
\pgfpathlineto{\pgfqpoint{5.324364in}{3.134554in}}%
\pgfpathlineto{\pgfqpoint{5.309489in}{3.118180in}}%
\pgfpathlineto{\pgfqpoint{5.294637in}{3.101994in}}%
\pgfpathlineto{\pgfqpoint{5.279806in}{3.085997in}}%
\pgfpathlineto{\pgfqpoint{5.264997in}{3.070186in}}%
\pgfpathlineto{\pgfqpoint{5.257334in}{3.063225in}}%
\pgfpathlineto{\pgfqpoint{5.249661in}{3.056098in}}%
\pgfpathlineto{\pgfqpoint{5.241978in}{3.048803in}}%
\pgfpathlineto{\pgfqpoint{5.234286in}{3.041339in}}%
\pgfpathclose%
\pgfusepath{fill}%
\end{pgfscope}%
\begin{pgfscope}%
\pgfpathrectangle{\pgfqpoint{1.150000in}{0.150000in}}{\pgfqpoint{5.700000in}{5.700000in}}%
\pgfusepath{clip}%
\pgfsetbuttcap%
\pgfsetroundjoin%
\definecolor{currentfill}{rgb}{0.132444,0.552216,0.553018}%
\pgfsetfillcolor{currentfill}%
\pgfsetfillopacity{0.800000}%
\pgfsetlinewidth{0.000000pt}%
\definecolor{currentstroke}{rgb}{0.000000,0.000000,0.000000}%
\pgfsetstrokecolor{currentstroke}%
\pgfsetdash{}{0pt}%
\pgfpathmoveto{\pgfqpoint{4.902099in}{2.673975in}}%
\pgfpathlineto{\pgfqpoint{4.916687in}{2.687959in}}%
\pgfpathlineto{\pgfqpoint{4.931295in}{2.702129in}}%
\pgfpathlineto{\pgfqpoint{4.945923in}{2.716487in}}%
\pgfpathlineto{\pgfqpoint{4.960570in}{2.731031in}}%
\pgfpathlineto{\pgfqpoint{4.968446in}{2.742110in}}%
\pgfpathlineto{\pgfqpoint{4.976313in}{2.753015in}}%
\pgfpathlineto{\pgfqpoint{4.984173in}{2.763747in}}%
\pgfpathlineto{\pgfqpoint{4.992024in}{2.774306in}}%
\pgfpathlineto{\pgfqpoint{4.977377in}{2.759744in}}%
\pgfpathlineto{\pgfqpoint{4.962750in}{2.745368in}}%
\pgfpathlineto{\pgfqpoint{4.948143in}{2.731179in}}%
\pgfpathlineto{\pgfqpoint{4.933556in}{2.717177in}}%
\pgfpathlineto{\pgfqpoint{4.925703in}{2.706623in}}%
\pgfpathlineto{\pgfqpoint{4.917842in}{2.695905in}}%
\pgfpathlineto{\pgfqpoint{4.909974in}{2.685023in}}%
\pgfpathlineto{\pgfqpoint{4.902099in}{2.673975in}}%
\pgfpathclose%
\pgfusepath{fill}%
\end{pgfscope}%
\begin{pgfscope}%
\pgfpathrectangle{\pgfqpoint{1.150000in}{0.150000in}}{\pgfqpoint{5.700000in}{5.700000in}}%
\pgfusepath{clip}%
\pgfsetbuttcap%
\pgfsetroundjoin%
\definecolor{currentfill}{rgb}{0.275191,0.194905,0.496005}%
\pgfsetfillcolor{currentfill}%
\pgfsetfillopacity{0.800000}%
\pgfsetlinewidth{0.000000pt}%
\definecolor{currentstroke}{rgb}{0.000000,0.000000,0.000000}%
\pgfsetstrokecolor{currentstroke}%
\pgfsetdash{}{0pt}%
\pgfpathmoveto{\pgfqpoint{4.083477in}{1.673237in}}%
\pgfpathlineto{\pgfqpoint{4.097597in}{1.678767in}}%
\pgfpathlineto{\pgfqpoint{4.111729in}{1.684479in}}%
\pgfpathlineto{\pgfqpoint{4.125874in}{1.690372in}}%
\pgfpathlineto{\pgfqpoint{4.140030in}{1.696446in}}%
\pgfpathlineto{\pgfqpoint{4.148173in}{1.711410in}}%
\pgfpathlineto{\pgfqpoint{4.156312in}{1.726365in}}%
\pgfpathlineto{\pgfqpoint{4.164447in}{1.741307in}}%
\pgfpathlineto{\pgfqpoint{4.172577in}{1.756233in}}%
\pgfpathlineto{\pgfqpoint{4.158421in}{1.749710in}}%
\pgfpathlineto{\pgfqpoint{4.144276in}{1.743369in}}%
\pgfpathlineto{\pgfqpoint{4.130145in}{1.737210in}}%
\pgfpathlineto{\pgfqpoint{4.116025in}{1.731232in}}%
\pgfpathlineto{\pgfqpoint{4.107894in}{1.716742in}}%
\pgfpathlineto{\pgfqpoint{4.099760in}{1.702244in}}%
\pgfpathlineto{\pgfqpoint{4.091620in}{1.687740in}}%
\pgfpathlineto{\pgfqpoint{4.083477in}{1.673237in}}%
\pgfpathclose%
\pgfusepath{fill}%
\end{pgfscope}%
\begin{pgfscope}%
\pgfpathrectangle{\pgfqpoint{1.150000in}{0.150000in}}{\pgfqpoint{5.700000in}{5.700000in}}%
\pgfusepath{clip}%
\pgfsetbuttcap%
\pgfsetroundjoin%
\definecolor{currentfill}{rgb}{0.282910,0.105393,0.426902}%
\pgfsetfillcolor{currentfill}%
\pgfsetfillopacity{0.800000}%
\pgfsetlinewidth{0.000000pt}%
\definecolor{currentstroke}{rgb}{0.000000,0.000000,0.000000}%
\pgfsetstrokecolor{currentstroke}%
\pgfsetdash{}{0pt}%
\pgfpathmoveto{\pgfqpoint{3.872775in}{1.475294in}}%
\pgfpathlineto{\pgfqpoint{3.886820in}{1.477846in}}%
\pgfpathlineto{\pgfqpoint{3.900875in}{1.480580in}}%
\pgfpathlineto{\pgfqpoint{3.914939in}{1.483495in}}%
\pgfpathlineto{\pgfqpoint{3.929014in}{1.486591in}}%
\pgfpathlineto{\pgfqpoint{3.937215in}{1.500194in}}%
\pgfpathlineto{\pgfqpoint{3.945410in}{1.513860in}}%
\pgfpathlineto{\pgfqpoint{3.953601in}{1.527584in}}%
\pgfpathlineto{\pgfqpoint{3.961788in}{1.541362in}}%
\pgfpathlineto{\pgfqpoint{3.947718in}{1.537726in}}%
\pgfpathlineto{\pgfqpoint{3.933659in}{1.534271in}}%
\pgfpathlineto{\pgfqpoint{3.919610in}{1.530998in}}%
\pgfpathlineto{\pgfqpoint{3.905570in}{1.527907in}}%
\pgfpathlineto{\pgfqpoint{3.897379in}{1.514657in}}%
\pgfpathlineto{\pgfqpoint{3.889183in}{1.501468in}}%
\pgfpathlineto{\pgfqpoint{3.880981in}{1.488345in}}%
\pgfpathlineto{\pgfqpoint{3.872775in}{1.475294in}}%
\pgfpathclose%
\pgfusepath{fill}%
\end{pgfscope}%
\begin{pgfscope}%
\pgfpathrectangle{\pgfqpoint{1.150000in}{0.150000in}}{\pgfqpoint{5.700000in}{5.700000in}}%
\pgfusepath{clip}%
\pgfsetbuttcap%
\pgfsetroundjoin%
\definecolor{currentfill}{rgb}{0.268510,0.009605,0.335427}%
\pgfsetfillcolor{currentfill}%
\pgfsetfillopacity{0.800000}%
\pgfsetlinewidth{0.000000pt}%
\definecolor{currentstroke}{rgb}{0.000000,0.000000,0.000000}%
\pgfsetstrokecolor{currentstroke}%
\pgfsetdash{}{0pt}%
\pgfpathmoveto{\pgfqpoint{3.516346in}{1.304133in}}%
\pgfpathlineto{\pgfqpoint{3.530323in}{1.301290in}}%
\pgfpathlineto{\pgfqpoint{3.544305in}{1.298632in}}%
\pgfpathlineto{\pgfqpoint{3.558294in}{1.296160in}}%
\pgfpathlineto{\pgfqpoint{3.572287in}{1.293873in}}%
\pgfpathlineto{\pgfqpoint{3.580632in}{1.302990in}}%
\pgfpathlineto{\pgfqpoint{3.588969in}{1.312295in}}%
\pgfpathlineto{\pgfqpoint{3.597299in}{1.321783in}}%
\pgfpathlineto{\pgfqpoint{3.605620in}{1.331447in}}%
\pgfpathlineto{\pgfqpoint{3.591644in}{1.333072in}}%
\pgfpathlineto{\pgfqpoint{3.577674in}{1.334883in}}%
\pgfpathlineto{\pgfqpoint{3.563709in}{1.336880in}}%
\pgfpathlineto{\pgfqpoint{3.549751in}{1.339062in}}%
\pgfpathlineto{\pgfqpoint{3.541412in}{1.330047in}}%
\pgfpathlineto{\pgfqpoint{3.533065in}{1.321217in}}%
\pgfpathlineto{\pgfqpoint{3.524710in}{1.312576in}}%
\pgfpathlineto{\pgfqpoint{3.516346in}{1.304133in}}%
\pgfpathclose%
\pgfusepath{fill}%
\end{pgfscope}%
\begin{pgfscope}%
\pgfpathrectangle{\pgfqpoint{1.150000in}{0.150000in}}{\pgfqpoint{5.700000in}{5.700000in}}%
\pgfusepath{clip}%
\pgfsetbuttcap%
\pgfsetroundjoin%
\definecolor{currentfill}{rgb}{0.458674,0.816363,0.329727}%
\pgfsetfillcolor{currentfill}%
\pgfsetfillopacity{0.800000}%
\pgfsetlinewidth{0.000000pt}%
\definecolor{currentstroke}{rgb}{0.000000,0.000000,0.000000}%
\pgfsetstrokecolor{currentstroke}%
\pgfsetdash{}{0pt}%
\pgfpathmoveto{\pgfqpoint{5.834736in}{3.589875in}}%
\pgfpathlineto{\pgfqpoint{5.849989in}{3.608063in}}%
\pgfpathlineto{\pgfqpoint{5.865266in}{3.626440in}}%
\pgfpathlineto{\pgfqpoint{5.880569in}{3.645007in}}%
\pgfpathlineto{\pgfqpoint{5.887805in}{3.645583in}}%
\pgfpathlineto{\pgfqpoint{5.895030in}{3.646042in}}%
\pgfpathlineto{\pgfqpoint{5.902244in}{3.646389in}}%
\pgfpathlineto{\pgfqpoint{5.909447in}{3.646628in}}%
\pgfpathlineto{\pgfqpoint{5.894171in}{3.628517in}}%
\pgfpathlineto{\pgfqpoint{5.878920in}{3.610595in}}%
\pgfpathlineto{\pgfqpoint{5.863694in}{3.592861in}}%
\pgfpathlineto{\pgfqpoint{5.856471in}{3.592271in}}%
\pgfpathlineto{\pgfqpoint{5.849237in}{3.591581in}}%
\pgfpathlineto{\pgfqpoint{5.841992in}{3.590783in}}%
\pgfpathlineto{\pgfqpoint{5.834736in}{3.589875in}}%
\pgfpathclose%
\pgfusepath{fill}%
\end{pgfscope}%
\begin{pgfscope}%
\pgfpathrectangle{\pgfqpoint{1.150000in}{0.150000in}}{\pgfqpoint{5.700000in}{5.700000in}}%
\pgfusepath{clip}%
\pgfsetbuttcap%
\pgfsetroundjoin%
\definecolor{currentfill}{rgb}{0.150476,0.504369,0.557430}%
\pgfsetfillcolor{currentfill}%
\pgfsetfillopacity{0.800000}%
\pgfsetlinewidth{0.000000pt}%
\definecolor{currentstroke}{rgb}{0.000000,0.000000,0.000000}%
\pgfsetstrokecolor{currentstroke}%
\pgfsetdash{}{0pt}%
\pgfpathmoveto{\pgfqpoint{4.780668in}{2.526157in}}%
\pgfpathlineto{\pgfqpoint{4.795179in}{2.539254in}}%
\pgfpathlineto{\pgfqpoint{4.809710in}{2.552537in}}%
\pgfpathlineto{\pgfqpoint{4.824259in}{2.566006in}}%
\pgfpathlineto{\pgfqpoint{4.838827in}{2.579662in}}%
\pgfpathlineto{\pgfqpoint{4.846761in}{2.592026in}}%
\pgfpathlineto{\pgfqpoint{4.854688in}{2.604226in}}%
\pgfpathlineto{\pgfqpoint{4.862608in}{2.616263in}}%
\pgfpathlineto{\pgfqpoint{4.870521in}{2.628135in}}%
\pgfpathlineto{\pgfqpoint{4.855952in}{2.614391in}}%
\pgfpathlineto{\pgfqpoint{4.841401in}{2.600834in}}%
\pgfpathlineto{\pgfqpoint{4.826870in}{2.587463in}}%
\pgfpathlineto{\pgfqpoint{4.812357in}{2.574278in}}%
\pgfpathlineto{\pgfqpoint{4.804445in}{2.562481in}}%
\pgfpathlineto{\pgfqpoint{4.796526in}{2.550528in}}%
\pgfpathlineto{\pgfqpoint{4.788600in}{2.538420in}}%
\pgfpathlineto{\pgfqpoint{4.780668in}{2.526157in}}%
\pgfpathclose%
\pgfusepath{fill}%
\end{pgfscope}%
\begin{pgfscope}%
\pgfpathrectangle{\pgfqpoint{1.150000in}{0.150000in}}{\pgfqpoint{5.700000in}{5.700000in}}%
\pgfusepath{clip}%
\pgfsetbuttcap%
\pgfsetroundjoin%
\definecolor{currentfill}{rgb}{0.169646,0.456262,0.558030}%
\pgfsetfillcolor{currentfill}%
\pgfsetfillopacity{0.800000}%
\pgfsetlinewidth{0.000000pt}%
\definecolor{currentstroke}{rgb}{0.000000,0.000000,0.000000}%
\pgfsetstrokecolor{currentstroke}%
\pgfsetdash{}{0pt}%
\pgfpathmoveto{\pgfqpoint{2.172552in}{2.494875in}}%
\pgfpathlineto{\pgfqpoint{2.187025in}{2.469385in}}%
\pgfpathlineto{\pgfqpoint{2.201483in}{2.444205in}}%
\pgfpathlineto{\pgfqpoint{2.215925in}{2.419334in}}%
\pgfpathlineto{\pgfqpoint{2.230352in}{2.394768in}}%
\pgfpathlineto{\pgfqpoint{2.239890in}{2.383561in}}%
\pgfpathlineto{\pgfqpoint{2.249398in}{2.372833in}}%
\pgfpathlineto{\pgfqpoint{2.258876in}{2.362576in}}%
\pgfpathlineto{\pgfqpoint{2.268326in}{2.352782in}}%
\pgfpathlineto{\pgfqpoint{2.253973in}{2.376511in}}%
\pgfpathlineto{\pgfqpoint{2.239606in}{2.400544in}}%
\pgfpathlineto{\pgfqpoint{2.225225in}{2.424883in}}%
\pgfpathlineto{\pgfqpoint{2.210828in}{2.449531in}}%
\pgfpathlineto{\pgfqpoint{2.201305in}{2.460149in}}%
\pgfpathlineto{\pgfqpoint{2.191752in}{2.471240in}}%
\pgfpathlineto{\pgfqpoint{2.182168in}{2.482812in}}%
\pgfpathlineto{\pgfqpoint{2.172552in}{2.494875in}}%
\pgfpathclose%
\pgfusepath{fill}%
\end{pgfscope}%
\begin{pgfscope}%
\pgfpathrectangle{\pgfqpoint{1.150000in}{0.150000in}}{\pgfqpoint{5.700000in}{5.700000in}}%
\pgfusepath{clip}%
\pgfsetbuttcap%
\pgfsetroundjoin%
\definecolor{currentfill}{rgb}{0.277018,0.050344,0.375715}%
\pgfsetfillcolor{currentfill}%
\pgfsetfillopacity{0.800000}%
\pgfsetlinewidth{0.000000pt}%
\definecolor{currentstroke}{rgb}{0.000000,0.000000,0.000000}%
\pgfsetstrokecolor{currentstroke}%
\pgfsetdash{}{0pt}%
\pgfpathmoveto{\pgfqpoint{3.078834in}{1.398503in}}%
\pgfpathlineto{\pgfqpoint{3.092839in}{1.388833in}}%
\pgfpathlineto{\pgfqpoint{3.106843in}{1.379365in}}%
\pgfpathlineto{\pgfqpoint{3.120848in}{1.370099in}}%
\pgfpathlineto{\pgfqpoint{3.134852in}{1.361033in}}%
\pgfpathlineto{\pgfqpoint{3.143496in}{1.362704in}}%
\pgfpathlineto{\pgfqpoint{3.152126in}{1.364700in}}%
\pgfpathlineto{\pgfqpoint{3.160740in}{1.367013in}}%
\pgfpathlineto{\pgfqpoint{3.169341in}{1.369637in}}%
\pgfpathlineto{\pgfqpoint{3.155373in}{1.377939in}}%
\pgfpathlineto{\pgfqpoint{3.141406in}{1.386441in}}%
\pgfpathlineto{\pgfqpoint{3.127439in}{1.395145in}}%
\pgfpathlineto{\pgfqpoint{3.113473in}{1.404050in}}%
\pgfpathlineto{\pgfqpoint{3.104836in}{1.402178in}}%
\pgfpathlineto{\pgfqpoint{3.096184in}{1.400624in}}%
\pgfpathlineto{\pgfqpoint{3.087517in}{1.399397in}}%
\pgfpathlineto{\pgfqpoint{3.078834in}{1.398503in}}%
\pgfpathclose%
\pgfusepath{fill}%
\end{pgfscope}%
\begin{pgfscope}%
\pgfpathrectangle{\pgfqpoint{1.150000in}{0.150000in}}{\pgfqpoint{5.700000in}{5.700000in}}%
\pgfusepath{clip}%
\pgfsetbuttcap%
\pgfsetroundjoin%
\definecolor{currentfill}{rgb}{0.169646,0.456262,0.558030}%
\pgfsetfillcolor{currentfill}%
\pgfsetfillopacity{0.800000}%
\pgfsetlinewidth{0.000000pt}%
\definecolor{currentstroke}{rgb}{0.000000,0.000000,0.000000}%
\pgfsetstrokecolor{currentstroke}%
\pgfsetdash{}{0pt}%
\pgfpathmoveto{\pgfqpoint{4.659122in}{2.373147in}}%
\pgfpathlineto{\pgfqpoint{4.673557in}{2.385223in}}%
\pgfpathlineto{\pgfqpoint{4.688010in}{2.397484in}}%
\pgfpathlineto{\pgfqpoint{4.702481in}{2.409931in}}%
\pgfpathlineto{\pgfqpoint{4.716970in}{2.422563in}}%
\pgfpathlineto{\pgfqpoint{4.724954in}{2.436037in}}%
\pgfpathlineto{\pgfqpoint{4.732933in}{2.449364in}}%
\pgfpathlineto{\pgfqpoint{4.740905in}{2.462541in}}%
\pgfpathlineto{\pgfqpoint{4.748870in}{2.475568in}}%
\pgfpathlineto{\pgfqpoint{4.734379in}{2.462780in}}%
\pgfpathlineto{\pgfqpoint{4.719905in}{2.450176in}}%
\pgfpathlineto{\pgfqpoint{4.705450in}{2.437758in}}%
\pgfpathlineto{\pgfqpoint{4.691012in}{2.425526in}}%
\pgfpathlineto{\pgfqpoint{4.683048in}{2.412642in}}%
\pgfpathlineto{\pgfqpoint{4.675079in}{2.399617in}}%
\pgfpathlineto{\pgfqpoint{4.667103in}{2.386452in}}%
\pgfpathlineto{\pgfqpoint{4.659122in}{2.373147in}}%
\pgfpathclose%
\pgfusepath{fill}%
\end{pgfscope}%
\begin{pgfscope}%
\pgfpathrectangle{\pgfqpoint{1.150000in}{0.150000in}}{\pgfqpoint{5.700000in}{5.700000in}}%
\pgfusepath{clip}%
\pgfsetbuttcap%
\pgfsetroundjoin%
\definecolor{currentfill}{rgb}{0.124395,0.578002,0.548287}%
\pgfsetfillcolor{currentfill}%
\pgfsetfillopacity{0.800000}%
\pgfsetlinewidth{0.000000pt}%
\definecolor{currentstroke}{rgb}{0.000000,0.000000,0.000000}%
\pgfsetstrokecolor{currentstroke}%
\pgfsetdash{}{0pt}%
\pgfpathmoveto{\pgfqpoint{1.978144in}{2.888393in}}%
\pgfpathlineto{\pgfqpoint{1.992831in}{2.858379in}}%
\pgfpathlineto{\pgfqpoint{2.007496in}{2.828725in}}%
\pgfpathlineto{\pgfqpoint{2.022141in}{2.799429in}}%
\pgfpathlineto{\pgfqpoint{2.036766in}{2.770487in}}%
\pgfpathlineto{\pgfqpoint{2.046494in}{2.757843in}}%
\pgfpathlineto{\pgfqpoint{2.056189in}{2.745687in}}%
\pgfpathlineto{\pgfqpoint{2.065852in}{2.734011in}}%
\pgfpathlineto{\pgfqpoint{2.075483in}{2.722806in}}%
\pgfpathlineto{\pgfqpoint{2.060939in}{2.750919in}}%
\pgfpathlineto{\pgfqpoint{2.046376in}{2.779383in}}%
\pgfpathlineto{\pgfqpoint{2.031793in}{2.808202in}}%
\pgfpathlineto{\pgfqpoint{2.017190in}{2.837380in}}%
\pgfpathlineto{\pgfqpoint{2.007478in}{2.849401in}}%
\pgfpathlineto{\pgfqpoint{1.997734in}{2.861904in}}%
\pgfpathlineto{\pgfqpoint{1.987956in}{2.874899in}}%
\pgfpathlineto{\pgfqpoint{1.978144in}{2.888393in}}%
\pgfpathclose%
\pgfusepath{fill}%
\end{pgfscope}%
\begin{pgfscope}%
\pgfpathrectangle{\pgfqpoint{1.150000in}{0.150000in}}{\pgfqpoint{5.700000in}{5.700000in}}%
\pgfusepath{clip}%
\pgfsetbuttcap%
\pgfsetroundjoin%
\definecolor{currentfill}{rgb}{0.190631,0.407061,0.556089}%
\pgfsetfillcolor{currentfill}%
\pgfsetfillopacity{0.800000}%
\pgfsetlinewidth{0.000000pt}%
\definecolor{currentstroke}{rgb}{0.000000,0.000000,0.000000}%
\pgfsetstrokecolor{currentstroke}%
\pgfsetdash{}{0pt}%
\pgfpathmoveto{\pgfqpoint{4.537514in}{2.217058in}}%
\pgfpathlineto{\pgfqpoint{4.551875in}{2.227981in}}%
\pgfpathlineto{\pgfqpoint{4.566253in}{2.239089in}}%
\pgfpathlineto{\pgfqpoint{4.580648in}{2.250381in}}%
\pgfpathlineto{\pgfqpoint{4.595060in}{2.261858in}}%
\pgfpathlineto{\pgfqpoint{4.603087in}{2.276225in}}%
\pgfpathlineto{\pgfqpoint{4.611109in}{2.290466in}}%
\pgfpathlineto{\pgfqpoint{4.619125in}{2.304578in}}%
\pgfpathlineto{\pgfqpoint{4.627136in}{2.318559in}}%
\pgfpathlineto{\pgfqpoint{4.612721in}{2.306859in}}%
\pgfpathlineto{\pgfqpoint{4.598323in}{2.295343in}}%
\pgfpathlineto{\pgfqpoint{4.583941in}{2.284011in}}%
\pgfpathlineto{\pgfqpoint{4.569577in}{2.272864in}}%
\pgfpathlineto{\pgfqpoint{4.561569in}{2.259094in}}%
\pgfpathlineto{\pgfqpoint{4.553556in}{2.245202in}}%
\pgfpathlineto{\pgfqpoint{4.545537in}{2.231189in}}%
\pgfpathlineto{\pgfqpoint{4.537514in}{2.217058in}}%
\pgfpathclose%
\pgfusepath{fill}%
\end{pgfscope}%
\begin{pgfscope}%
\pgfpathrectangle{\pgfqpoint{1.150000in}{0.150000in}}{\pgfqpoint{5.700000in}{5.700000in}}%
\pgfusepath{clip}%
\pgfsetbuttcap%
\pgfsetroundjoin%
\definecolor{currentfill}{rgb}{0.268510,0.009605,0.335427}%
\pgfsetfillcolor{currentfill}%
\pgfsetfillopacity{0.800000}%
\pgfsetlinewidth{0.000000pt}%
\definecolor{currentstroke}{rgb}{0.000000,0.000000,0.000000}%
\pgfsetstrokecolor{currentstroke}%
\pgfsetdash{}{0pt}%
\pgfpathmoveto{\pgfqpoint{3.281128in}{1.310331in}}%
\pgfpathlineto{\pgfqpoint{3.295110in}{1.303795in}}%
\pgfpathlineto{\pgfqpoint{3.309094in}{1.297453in}}%
\pgfpathlineto{\pgfqpoint{3.323080in}{1.291302in}}%
\pgfpathlineto{\pgfqpoint{3.337070in}{1.285343in}}%
\pgfpathlineto{\pgfqpoint{3.345562in}{1.290488in}}%
\pgfpathlineto{\pgfqpoint{3.354043in}{1.295902in}}%
\pgfpathlineto{\pgfqpoint{3.362513in}{1.301578in}}%
\pgfpathlineto{\pgfqpoint{3.370971in}{1.307508in}}%
\pgfpathlineto{\pgfqpoint{3.357010in}{1.312741in}}%
\pgfpathlineto{\pgfqpoint{3.343052in}{1.318165in}}%
\pgfpathlineto{\pgfqpoint{3.329096in}{1.323781in}}%
\pgfpathlineto{\pgfqpoint{3.315144in}{1.329590in}}%
\pgfpathlineto{\pgfqpoint{3.306658in}{1.324373in}}%
\pgfpathlineto{\pgfqpoint{3.298160in}{1.319420in}}%
\pgfpathlineto{\pgfqpoint{3.289650in}{1.314736in}}%
\pgfpathlineto{\pgfqpoint{3.281128in}{1.310331in}}%
\pgfpathclose%
\pgfusepath{fill}%
\end{pgfscope}%
\begin{pgfscope}%
\pgfpathrectangle{\pgfqpoint{1.150000in}{0.150000in}}{\pgfqpoint{5.700000in}{5.700000in}}%
\pgfusepath{clip}%
\pgfsetbuttcap%
\pgfsetroundjoin%
\definecolor{currentfill}{rgb}{0.265145,0.232956,0.516599}%
\pgfsetfillcolor{currentfill}%
\pgfsetfillopacity{0.800000}%
\pgfsetlinewidth{0.000000pt}%
\definecolor{currentstroke}{rgb}{0.000000,0.000000,0.000000}%
\pgfsetstrokecolor{currentstroke}%
\pgfsetdash{}{0pt}%
\pgfpathmoveto{\pgfqpoint{2.592877in}{1.846961in}}%
\pgfpathlineto{\pgfqpoint{2.607054in}{1.829407in}}%
\pgfpathlineto{\pgfqpoint{2.621223in}{1.812096in}}%
\pgfpathlineto{\pgfqpoint{2.635385in}{1.795025in}}%
\pgfpathlineto{\pgfqpoint{2.649540in}{1.778194in}}%
\pgfpathlineto{\pgfqpoint{2.658646in}{1.771895in}}%
\pgfpathlineto{\pgfqpoint{2.667729in}{1.766028in}}%
\pgfpathlineto{\pgfqpoint{2.676789in}{1.760585in}}%
\pgfpathlineto{\pgfqpoint{2.685826in}{1.755557in}}%
\pgfpathlineto{\pgfqpoint{2.671729in}{1.771563in}}%
\pgfpathlineto{\pgfqpoint{2.657626in}{1.787807in}}%
\pgfpathlineto{\pgfqpoint{2.643517in}{1.804291in}}%
\pgfpathlineto{\pgfqpoint{2.629400in}{1.821016in}}%
\pgfpathlineto{\pgfqpoint{2.620305in}{1.826856in}}%
\pgfpathlineto{\pgfqpoint{2.611187in}{1.833121in}}%
\pgfpathlineto{\pgfqpoint{2.602044in}{1.839820in}}%
\pgfpathlineto{\pgfqpoint{2.592877in}{1.846961in}}%
\pgfpathclose%
\pgfusepath{fill}%
\end{pgfscope}%
\begin{pgfscope}%
\pgfpathrectangle{\pgfqpoint{1.150000in}{0.150000in}}{\pgfqpoint{5.700000in}{5.700000in}}%
\pgfusepath{clip}%
\pgfsetbuttcap%
\pgfsetroundjoin%
\definecolor{currentfill}{rgb}{0.273006,0.204520,0.501721}%
\pgfsetfillcolor{currentfill}%
\pgfsetfillopacity{0.800000}%
\pgfsetlinewidth{0.000000pt}%
\definecolor{currentstroke}{rgb}{0.000000,0.000000,0.000000}%
\pgfsetstrokecolor{currentstroke}%
\pgfsetdash{}{0pt}%
\pgfpathmoveto{\pgfqpoint{2.649540in}{1.778194in}}%
\pgfpathlineto{\pgfqpoint{2.663688in}{1.761601in}}%
\pgfpathlineto{\pgfqpoint{2.677830in}{1.745243in}}%
\pgfpathlineto{\pgfqpoint{2.691965in}{1.729121in}}%
\pgfpathlineto{\pgfqpoint{2.706095in}{1.713232in}}%
\pgfpathlineto{\pgfqpoint{2.715143in}{1.707769in}}%
\pgfpathlineto{\pgfqpoint{2.724168in}{1.702730in}}%
\pgfpathlineto{\pgfqpoint{2.733172in}{1.698104in}}%
\pgfpathlineto{\pgfqpoint{2.742154in}{1.693885in}}%
\pgfpathlineto{\pgfqpoint{2.728080in}{1.708953in}}%
\pgfpathlineto{\pgfqpoint{2.714001in}{1.724254in}}%
\pgfpathlineto{\pgfqpoint{2.699917in}{1.739788in}}%
\pgfpathlineto{\pgfqpoint{2.685826in}{1.755557in}}%
\pgfpathlineto{\pgfqpoint{2.676789in}{1.760585in}}%
\pgfpathlineto{\pgfqpoint{2.667729in}{1.766028in}}%
\pgfpathlineto{\pgfqpoint{2.658646in}{1.771895in}}%
\pgfpathlineto{\pgfqpoint{2.649540in}{1.778194in}}%
\pgfpathclose%
\pgfusepath{fill}%
\end{pgfscope}%
\begin{pgfscope}%
\pgfpathrectangle{\pgfqpoint{1.150000in}{0.150000in}}{\pgfqpoint{5.700000in}{5.700000in}}%
\pgfusepath{clip}%
\pgfsetbuttcap%
\pgfsetroundjoin%
\definecolor{currentfill}{rgb}{0.282623,0.140926,0.457517}%
\pgfsetfillcolor{currentfill}%
\pgfsetfillopacity{0.800000}%
\pgfsetlinewidth{0.000000pt}%
\definecolor{currentstroke}{rgb}{0.000000,0.000000,0.000000}%
\pgfsetstrokecolor{currentstroke}%
\pgfsetdash{}{0pt}%
\pgfpathmoveto{\pgfqpoint{3.961788in}{1.541362in}}%
\pgfpathlineto{\pgfqpoint{3.975868in}{1.545179in}}%
\pgfpathlineto{\pgfqpoint{3.989959in}{1.549178in}}%
\pgfpathlineto{\pgfqpoint{4.004060in}{1.553357in}}%
\pgfpathlineto{\pgfqpoint{4.018173in}{1.557717in}}%
\pgfpathlineto{\pgfqpoint{4.026351in}{1.572065in}}%
\pgfpathlineto{\pgfqpoint{4.034525in}{1.586449in}}%
\pgfpathlineto{\pgfqpoint{4.042695in}{1.600863in}}%
\pgfpathlineto{\pgfqpoint{4.050860in}{1.615304in}}%
\pgfpathlineto{\pgfqpoint{4.036750in}{1.610433in}}%
\pgfpathlineto{\pgfqpoint{4.022651in}{1.605744in}}%
\pgfpathlineto{\pgfqpoint{4.008563in}{1.601237in}}%
\pgfpathlineto{\pgfqpoint{3.994487in}{1.596910in}}%
\pgfpathlineto{\pgfqpoint{3.986319in}{1.582967in}}%
\pgfpathlineto{\pgfqpoint{3.978146in}{1.569058in}}%
\pgfpathlineto{\pgfqpoint{3.969969in}{1.555188in}}%
\pgfpathlineto{\pgfqpoint{3.961788in}{1.541362in}}%
\pgfpathclose%
\pgfusepath{fill}%
\end{pgfscope}%
\begin{pgfscope}%
\pgfpathrectangle{\pgfqpoint{1.150000in}{0.150000in}}{\pgfqpoint{5.700000in}{5.700000in}}%
\pgfusepath{clip}%
\pgfsetbuttcap%
\pgfsetroundjoin%
\definecolor{currentfill}{rgb}{0.216210,0.351535,0.550627}%
\pgfsetfillcolor{currentfill}%
\pgfsetfillopacity{0.800000}%
\pgfsetlinewidth{0.000000pt}%
\definecolor{currentstroke}{rgb}{0.000000,0.000000,0.000000}%
\pgfsetstrokecolor{currentstroke}%
\pgfsetdash{}{0pt}%
\pgfpathmoveto{\pgfqpoint{4.415881in}{2.060324in}}%
\pgfpathlineto{\pgfqpoint{4.430172in}{2.069967in}}%
\pgfpathlineto{\pgfqpoint{4.444479in}{2.079792in}}%
\pgfpathlineto{\pgfqpoint{4.458801in}{2.089801in}}%
\pgfpathlineto{\pgfqpoint{4.473140in}{2.099994in}}%
\pgfpathlineto{\pgfqpoint{4.481204in}{2.114996in}}%
\pgfpathlineto{\pgfqpoint{4.489263in}{2.129897in}}%
\pgfpathlineto{\pgfqpoint{4.497317in}{2.144696in}}%
\pgfpathlineto{\pgfqpoint{4.505367in}{2.159389in}}%
\pgfpathlineto{\pgfqpoint{4.491025in}{2.148906in}}%
\pgfpathlineto{\pgfqpoint{4.476699in}{2.138607in}}%
\pgfpathlineto{\pgfqpoint{4.462389in}{2.128492in}}%
\pgfpathlineto{\pgfqpoint{4.448094in}{2.118560in}}%
\pgfpathlineto{\pgfqpoint{4.440048in}{2.104145in}}%
\pgfpathlineto{\pgfqpoint{4.431997in}{2.089632in}}%
\pgfpathlineto{\pgfqpoint{4.423941in}{2.075024in}}%
\pgfpathlineto{\pgfqpoint{4.415881in}{2.060324in}}%
\pgfpathclose%
\pgfusepath{fill}%
\end{pgfscope}%
\begin{pgfscope}%
\pgfpathrectangle{\pgfqpoint{1.150000in}{0.150000in}}{\pgfqpoint{5.700000in}{5.700000in}}%
\pgfusepath{clip}%
\pgfsetbuttcap%
\pgfsetroundjoin%
\definecolor{currentfill}{rgb}{0.121380,0.629492,0.531973}%
\pgfsetfillcolor{currentfill}%
\pgfsetfillopacity{0.800000}%
\pgfsetlinewidth{0.000000pt}%
\definecolor{currentstroke}{rgb}{0.000000,0.000000,0.000000}%
\pgfsetstrokecolor{currentstroke}%
\pgfsetdash{}{0pt}%
\pgfpathmoveto{\pgfqpoint{5.113306in}{2.912465in}}%
\pgfpathlineto{\pgfqpoint{5.128049in}{2.927895in}}%
\pgfpathlineto{\pgfqpoint{5.142814in}{2.943513in}}%
\pgfpathlineto{\pgfqpoint{5.157599in}{2.959320in}}%
\pgfpathlineto{\pgfqpoint{5.172406in}{2.975315in}}%
\pgfpathlineto{\pgfqpoint{5.180174in}{2.984196in}}%
\pgfpathlineto{\pgfqpoint{5.187933in}{2.992896in}}%
\pgfpathlineto{\pgfqpoint{5.195682in}{3.001414in}}%
\pgfpathlineto{\pgfqpoint{5.203422in}{3.009752in}}%
\pgfpathlineto{\pgfqpoint{5.188620in}{2.993846in}}%
\pgfpathlineto{\pgfqpoint{5.173839in}{2.978128in}}%
\pgfpathlineto{\pgfqpoint{5.159079in}{2.962598in}}%
\pgfpathlineto{\pgfqpoint{5.144341in}{2.947255in}}%
\pgfpathlineto{\pgfqpoint{5.136595in}{2.938816in}}%
\pgfpathlineto{\pgfqpoint{5.128841in}{2.930205in}}%
\pgfpathlineto{\pgfqpoint{5.121078in}{2.921421in}}%
\pgfpathlineto{\pgfqpoint{5.113306in}{2.912465in}}%
\pgfpathclose%
\pgfusepath{fill}%
\end{pgfscope}%
\begin{pgfscope}%
\pgfpathrectangle{\pgfqpoint{1.150000in}{0.150000in}}{\pgfqpoint{5.700000in}{5.700000in}}%
\pgfusepath{clip}%
\pgfsetbuttcap%
\pgfsetroundjoin%
\definecolor{currentfill}{rgb}{0.281477,0.755203,0.432552}%
\pgfsetfillcolor{currentfill}%
\pgfsetfillopacity{0.800000}%
\pgfsetlinewidth{0.000000pt}%
\definecolor{currentstroke}{rgb}{0.000000,0.000000,0.000000}%
\pgfsetstrokecolor{currentstroke}%
\pgfsetdash{}{0pt}%
\pgfpathmoveto{\pgfqpoint{5.534953in}{3.335533in}}%
\pgfpathlineto{\pgfqpoint{5.549996in}{3.352905in}}%
\pgfpathlineto{\pgfqpoint{5.565063in}{3.370465in}}%
\pgfpathlineto{\pgfqpoint{5.580154in}{3.388215in}}%
\pgfpathlineto{\pgfqpoint{5.595268in}{3.406155in}}%
\pgfpathlineto{\pgfqpoint{5.602746in}{3.410040in}}%
\pgfpathlineto{\pgfqpoint{5.610213in}{3.413765in}}%
\pgfpathlineto{\pgfqpoint{5.617668in}{3.417332in}}%
\pgfpathlineto{\pgfqpoint{5.625112in}{3.420746in}}%
\pgfpathlineto{\pgfqpoint{5.610015in}{3.403115in}}%
\pgfpathlineto{\pgfqpoint{5.594941in}{3.385673in}}%
\pgfpathlineto{\pgfqpoint{5.579891in}{3.368420in}}%
\pgfpathlineto{\pgfqpoint{5.564865in}{3.351355in}}%
\pgfpathlineto{\pgfqpoint{5.557403in}{3.347621in}}%
\pgfpathlineto{\pgfqpoint{5.549930in}{3.343742in}}%
\pgfpathlineto{\pgfqpoint{5.542447in}{3.339714in}}%
\pgfpathlineto{\pgfqpoint{5.534953in}{3.335533in}}%
\pgfpathclose%
\pgfusepath{fill}%
\end{pgfscope}%
\begin{pgfscope}%
\pgfpathrectangle{\pgfqpoint{1.150000in}{0.150000in}}{\pgfqpoint{5.700000in}{5.700000in}}%
\pgfusepath{clip}%
\pgfsetbuttcap%
\pgfsetroundjoin%
\definecolor{currentfill}{rgb}{0.278012,0.180367,0.486697}%
\pgfsetfillcolor{currentfill}%
\pgfsetfillopacity{0.800000}%
\pgfsetlinewidth{0.000000pt}%
\definecolor{currentstroke}{rgb}{0.000000,0.000000,0.000000}%
\pgfsetstrokecolor{currentstroke}%
\pgfsetdash{}{0pt}%
\pgfpathmoveto{\pgfqpoint{2.706095in}{1.713232in}}%
\pgfpathlineto{\pgfqpoint{2.720218in}{1.697574in}}%
\pgfpathlineto{\pgfqpoint{2.734336in}{1.682147in}}%
\pgfpathlineto{\pgfqpoint{2.748449in}{1.666949in}}%
\pgfpathlineto{\pgfqpoint{2.762556in}{1.651978in}}%
\pgfpathlineto{\pgfqpoint{2.771549in}{1.647348in}}%
\pgfpathlineto{\pgfqpoint{2.780520in}{1.643131in}}%
\pgfpathlineto{\pgfqpoint{2.789469in}{1.639319in}}%
\pgfpathlineto{\pgfqpoint{2.798398in}{1.635903in}}%
\pgfpathlineto{\pgfqpoint{2.784344in}{1.650057in}}%
\pgfpathlineto{\pgfqpoint{2.770286in}{1.664438in}}%
\pgfpathlineto{\pgfqpoint{2.756222in}{1.679047in}}%
\pgfpathlineto{\pgfqpoint{2.742154in}{1.693885in}}%
\pgfpathlineto{\pgfqpoint{2.733172in}{1.698104in}}%
\pgfpathlineto{\pgfqpoint{2.724168in}{1.702730in}}%
\pgfpathlineto{\pgfqpoint{2.715143in}{1.707769in}}%
\pgfpathlineto{\pgfqpoint{2.706095in}{1.713232in}}%
\pgfpathclose%
\pgfusepath{fill}%
\end{pgfscope}%
\begin{pgfscope}%
\pgfpathrectangle{\pgfqpoint{1.150000in}{0.150000in}}{\pgfqpoint{5.700000in}{5.700000in}}%
\pgfusepath{clip}%
\pgfsetbuttcap%
\pgfsetroundjoin%
\definecolor{currentfill}{rgb}{0.255645,0.260703,0.528312}%
\pgfsetfillcolor{currentfill}%
\pgfsetfillopacity{0.800000}%
\pgfsetlinewidth{0.000000pt}%
\definecolor{currentstroke}{rgb}{0.000000,0.000000,0.000000}%
\pgfsetstrokecolor{currentstroke}%
\pgfsetdash{}{0pt}%
\pgfpathmoveto{\pgfqpoint{2.536092in}{1.919636in}}%
\pgfpathlineto{\pgfqpoint{2.550301in}{1.901095in}}%
\pgfpathlineto{\pgfqpoint{2.564501in}{1.882804in}}%
\pgfpathlineto{\pgfqpoint{2.578693in}{1.864759in}}%
\pgfpathlineto{\pgfqpoint{2.592877in}{1.846961in}}%
\pgfpathlineto{\pgfqpoint{2.602044in}{1.839820in}}%
\pgfpathlineto{\pgfqpoint{2.611187in}{1.833121in}}%
\pgfpathlineto{\pgfqpoint{2.620305in}{1.826856in}}%
\pgfpathlineto{\pgfqpoint{2.629400in}{1.821016in}}%
\pgfpathlineto{\pgfqpoint{2.615277in}{1.837983in}}%
\pgfpathlineto{\pgfqpoint{2.601146in}{1.855196in}}%
\pgfpathlineto{\pgfqpoint{2.587008in}{1.872654in}}%
\pgfpathlineto{\pgfqpoint{2.572862in}{1.890361in}}%
\pgfpathlineto{\pgfqpoint{2.563707in}{1.897018in}}%
\pgfpathlineto{\pgfqpoint{2.554527in}{1.904111in}}%
\pgfpathlineto{\pgfqpoint{2.545323in}{1.911647in}}%
\pgfpathlineto{\pgfqpoint{2.536092in}{1.919636in}}%
\pgfpathclose%
\pgfusepath{fill}%
\end{pgfscope}%
\begin{pgfscope}%
\pgfpathrectangle{\pgfqpoint{1.150000in}{0.150000in}}{\pgfqpoint{5.700000in}{5.700000in}}%
\pgfusepath{clip}%
\pgfsetbuttcap%
\pgfsetroundjoin%
\definecolor{currentfill}{rgb}{0.267004,0.004874,0.329415}%
\pgfsetfillcolor{currentfill}%
\pgfsetfillopacity{0.800000}%
\pgfsetlinewidth{0.000000pt}%
\definecolor{currentstroke}{rgb}{0.000000,0.000000,0.000000}%
\pgfsetstrokecolor{currentstroke}%
\pgfsetdash{}{0pt}%
\pgfpathmoveto{\pgfqpoint{3.426853in}{1.288478in}}%
\pgfpathlineto{\pgfqpoint{3.440834in}{1.284193in}}%
\pgfpathlineto{\pgfqpoint{3.454818in}{1.280096in}}%
\pgfpathlineto{\pgfqpoint{3.468807in}{1.276186in}}%
\pgfpathlineto{\pgfqpoint{3.482800in}{1.272463in}}%
\pgfpathlineto{\pgfqpoint{3.491201in}{1.280051in}}%
\pgfpathlineto{\pgfqpoint{3.499591in}{1.287864in}}%
\pgfpathlineto{\pgfqpoint{3.507973in}{1.295893in}}%
\pgfpathlineto{\pgfqpoint{3.516346in}{1.304133in}}%
\pgfpathlineto{\pgfqpoint{3.502374in}{1.307163in}}%
\pgfpathlineto{\pgfqpoint{3.488407in}{1.310380in}}%
\pgfpathlineto{\pgfqpoint{3.474445in}{1.313784in}}%
\pgfpathlineto{\pgfqpoint{3.460487in}{1.317376in}}%
\pgfpathlineto{\pgfqpoint{3.452093in}{1.309817in}}%
\pgfpathlineto{\pgfqpoint{3.443690in}{1.302476in}}%
\pgfpathlineto{\pgfqpoint{3.435277in}{1.295361in}}%
\pgfpathlineto{\pgfqpoint{3.426853in}{1.288478in}}%
\pgfpathclose%
\pgfusepath{fill}%
\end{pgfscope}%
\begin{pgfscope}%
\pgfpathrectangle{\pgfqpoint{1.150000in}{0.150000in}}{\pgfqpoint{5.700000in}{5.700000in}}%
\pgfusepath{clip}%
\pgfsetbuttcap%
\pgfsetroundjoin%
\definecolor{currentfill}{rgb}{0.241237,0.296485,0.539709}%
\pgfsetfillcolor{currentfill}%
\pgfsetfillopacity{0.800000}%
\pgfsetlinewidth{0.000000pt}%
\definecolor{currentstroke}{rgb}{0.000000,0.000000,0.000000}%
\pgfsetstrokecolor{currentstroke}%
\pgfsetdash{}{0pt}%
\pgfpathmoveto{\pgfqpoint{4.294238in}{1.905696in}}%
\pgfpathlineto{\pgfqpoint{4.308465in}{1.913931in}}%
\pgfpathlineto{\pgfqpoint{4.322706in}{1.922349in}}%
\pgfpathlineto{\pgfqpoint{4.336962in}{1.930949in}}%
\pgfpathlineto{\pgfqpoint{4.351232in}{1.939731in}}%
\pgfpathlineto{\pgfqpoint{4.359329in}{1.955069in}}%
\pgfpathlineto{\pgfqpoint{4.367421in}{1.970338in}}%
\pgfpathlineto{\pgfqpoint{4.375509in}{1.985534in}}%
\pgfpathlineto{\pgfqpoint{4.383592in}{2.000655in}}%
\pgfpathlineto{\pgfqpoint{4.369319in}{1.991517in}}%
\pgfpathlineto{\pgfqpoint{4.355061in}{1.982563in}}%
\pgfpathlineto{\pgfqpoint{4.340817in}{1.973791in}}%
\pgfpathlineto{\pgfqpoint{4.326588in}{1.965203in}}%
\pgfpathlineto{\pgfqpoint{4.318507in}{1.950424in}}%
\pgfpathlineto{\pgfqpoint{4.310422in}{1.935578in}}%
\pgfpathlineto{\pgfqpoint{4.302332in}{1.920667in}}%
\pgfpathlineto{\pgfqpoint{4.294238in}{1.905696in}}%
\pgfpathclose%
\pgfusepath{fill}%
\end{pgfscope}%
\begin{pgfscope}%
\pgfpathrectangle{\pgfqpoint{1.150000in}{0.150000in}}{\pgfqpoint{5.700000in}{5.700000in}}%
\pgfusepath{clip}%
\pgfsetbuttcap%
\pgfsetroundjoin%
\definecolor{currentfill}{rgb}{0.281412,0.155834,0.469201}%
\pgfsetfillcolor{currentfill}%
\pgfsetfillopacity{0.800000}%
\pgfsetlinewidth{0.000000pt}%
\definecolor{currentstroke}{rgb}{0.000000,0.000000,0.000000}%
\pgfsetstrokecolor{currentstroke}%
\pgfsetdash{}{0pt}%
\pgfpathmoveto{\pgfqpoint{2.762556in}{1.651978in}}%
\pgfpathlineto{\pgfqpoint{2.776659in}{1.637233in}}%
\pgfpathlineto{\pgfqpoint{2.790757in}{1.622714in}}%
\pgfpathlineto{\pgfqpoint{2.804850in}{1.608418in}}%
\pgfpathlineto{\pgfqpoint{2.818939in}{1.594344in}}%
\pgfpathlineto{\pgfqpoint{2.827878in}{1.590542in}}%
\pgfpathlineto{\pgfqpoint{2.836796in}{1.587144in}}%
\pgfpathlineto{\pgfqpoint{2.845694in}{1.584141in}}%
\pgfpathlineto{\pgfqpoint{2.854573in}{1.581524in}}%
\pgfpathlineto{\pgfqpoint{2.840535in}{1.594786in}}%
\pgfpathlineto{\pgfqpoint{2.826493in}{1.608268in}}%
\pgfpathlineto{\pgfqpoint{2.812448in}{1.621974in}}%
\pgfpathlineto{\pgfqpoint{2.798398in}{1.635903in}}%
\pgfpathlineto{\pgfqpoint{2.789469in}{1.639319in}}%
\pgfpathlineto{\pgfqpoint{2.780520in}{1.643131in}}%
\pgfpathlineto{\pgfqpoint{2.771549in}{1.647348in}}%
\pgfpathlineto{\pgfqpoint{2.762556in}{1.651978in}}%
\pgfpathclose%
\pgfusepath{fill}%
\end{pgfscope}%
\begin{pgfscope}%
\pgfpathrectangle{\pgfqpoint{1.150000in}{0.150000in}}{\pgfqpoint{5.700000in}{5.700000in}}%
\pgfusepath{clip}%
\pgfsetbuttcap%
\pgfsetroundjoin%
\definecolor{currentfill}{rgb}{0.243113,0.292092,0.538516}%
\pgfsetfillcolor{currentfill}%
\pgfsetfillopacity{0.800000}%
\pgfsetlinewidth{0.000000pt}%
\definecolor{currentstroke}{rgb}{0.000000,0.000000,0.000000}%
\pgfsetstrokecolor{currentstroke}%
\pgfsetdash{}{0pt}%
\pgfpathmoveto{\pgfqpoint{2.479169in}{1.996330in}}%
\pgfpathlineto{\pgfqpoint{2.493413in}{1.976774in}}%
\pgfpathlineto{\pgfqpoint{2.507649in}{1.957474in}}%
\pgfpathlineto{\pgfqpoint{2.521875in}{1.938429in}}%
\pgfpathlineto{\pgfqpoint{2.536092in}{1.919636in}}%
\pgfpathlineto{\pgfqpoint{2.545323in}{1.911647in}}%
\pgfpathlineto{\pgfqpoint{2.554527in}{1.904111in}}%
\pgfpathlineto{\pgfqpoint{2.563707in}{1.897018in}}%
\pgfpathlineto{\pgfqpoint{2.572862in}{1.890361in}}%
\pgfpathlineto{\pgfqpoint{2.558708in}{1.908317in}}%
\pgfpathlineto{\pgfqpoint{2.544546in}{1.926524in}}%
\pgfpathlineto{\pgfqpoint{2.530375in}{1.944985in}}%
\pgfpathlineto{\pgfqpoint{2.516195in}{1.963701in}}%
\pgfpathlineto{\pgfqpoint{2.506978in}{1.971182in}}%
\pgfpathlineto{\pgfqpoint{2.497735in}{1.979108in}}%
\pgfpathlineto{\pgfqpoint{2.488465in}{1.987488in}}%
\pgfpathlineto{\pgfqpoint{2.479169in}{1.996330in}}%
\pgfpathclose%
\pgfusepath{fill}%
\end{pgfscope}%
\begin{pgfscope}%
\pgfpathrectangle{\pgfqpoint{1.150000in}{0.150000in}}{\pgfqpoint{5.700000in}{5.700000in}}%
\pgfusepath{clip}%
\pgfsetbuttcap%
\pgfsetroundjoin%
\definecolor{currentfill}{rgb}{0.180653,0.701402,0.488189}%
\pgfsetfillcolor{currentfill}%
\pgfsetfillopacity{0.800000}%
\pgfsetlinewidth{0.000000pt}%
\definecolor{currentstroke}{rgb}{0.000000,0.000000,0.000000}%
\pgfsetstrokecolor{currentstroke}%
\pgfsetdash{}{0pt}%
\pgfpathmoveto{\pgfqpoint{5.324364in}{3.134554in}}%
\pgfpathlineto{\pgfqpoint{5.339261in}{3.151117in}}%
\pgfpathlineto{\pgfqpoint{5.354180in}{3.167868in}}%
\pgfpathlineto{\pgfqpoint{5.369121in}{3.184809in}}%
\pgfpathlineto{\pgfqpoint{5.384086in}{3.201939in}}%
\pgfpathlineto{\pgfqpoint{5.391721in}{3.208371in}}%
\pgfpathlineto{\pgfqpoint{5.399345in}{3.214625in}}%
\pgfpathlineto{\pgfqpoint{5.406959in}{3.220703in}}%
\pgfpathlineto{\pgfqpoint{5.414563in}{3.226608in}}%
\pgfpathlineto{\pgfqpoint{5.399609in}{3.209676in}}%
\pgfpathlineto{\pgfqpoint{5.384677in}{3.192934in}}%
\pgfpathlineto{\pgfqpoint{5.369768in}{3.176379in}}%
\pgfpathlineto{\pgfqpoint{5.354882in}{3.160013in}}%
\pgfpathlineto{\pgfqpoint{5.347267in}{3.153898in}}%
\pgfpathlineto{\pgfqpoint{5.339643in}{3.147618in}}%
\pgfpathlineto{\pgfqpoint{5.332009in}{3.141171in}}%
\pgfpathlineto{\pgfqpoint{5.324364in}{3.134554in}}%
\pgfpathclose%
\pgfusepath{fill}%
\end{pgfscope}%
\begin{pgfscope}%
\pgfpathrectangle{\pgfqpoint{1.150000in}{0.150000in}}{\pgfqpoint{5.700000in}{5.700000in}}%
\pgfusepath{clip}%
\pgfsetbuttcap%
\pgfsetroundjoin%
\definecolor{currentfill}{rgb}{0.263663,0.237631,0.518762}%
\pgfsetfillcolor{currentfill}%
\pgfsetfillopacity{0.800000}%
\pgfsetlinewidth{0.000000pt}%
\definecolor{currentstroke}{rgb}{0.000000,0.000000,0.000000}%
\pgfsetstrokecolor{currentstroke}%
\pgfsetdash{}{0pt}%
\pgfpathmoveto{\pgfqpoint{4.172577in}{1.756233in}}%
\pgfpathlineto{\pgfqpoint{4.186747in}{1.762937in}}%
\pgfpathlineto{\pgfqpoint{4.200930in}{1.769823in}}%
\pgfpathlineto{\pgfqpoint{4.215126in}{1.776891in}}%
\pgfpathlineto{\pgfqpoint{4.229335in}{1.784140in}}%
\pgfpathlineto{\pgfqpoint{4.237462in}{1.799474in}}%
\pgfpathlineto{\pgfqpoint{4.245586in}{1.814776in}}%
\pgfpathlineto{\pgfqpoint{4.253705in}{1.830042in}}%
\pgfpathlineto{\pgfqpoint{4.261820in}{1.845268in}}%
\pgfpathlineto{\pgfqpoint{4.247609in}{1.837601in}}%
\pgfpathlineto{\pgfqpoint{4.233412in}{1.830115in}}%
\pgfpathlineto{\pgfqpoint{4.219228in}{1.822812in}}%
\pgfpathlineto{\pgfqpoint{4.205057in}{1.815691in}}%
\pgfpathlineto{\pgfqpoint{4.196944in}{1.800871in}}%
\pgfpathlineto{\pgfqpoint{4.188826in}{1.786019in}}%
\pgfpathlineto{\pgfqpoint{4.180704in}{1.771138in}}%
\pgfpathlineto{\pgfqpoint{4.172577in}{1.756233in}}%
\pgfpathclose%
\pgfusepath{fill}%
\end{pgfscope}%
\begin{pgfscope}%
\pgfpathrectangle{\pgfqpoint{1.150000in}{0.150000in}}{\pgfqpoint{5.700000in}{5.700000in}}%
\pgfusepath{clip}%
\pgfsetbuttcap%
\pgfsetroundjoin%
\definecolor{currentfill}{rgb}{0.282884,0.135920,0.453427}%
\pgfsetfillcolor{currentfill}%
\pgfsetfillopacity{0.800000}%
\pgfsetlinewidth{0.000000pt}%
\definecolor{currentstroke}{rgb}{0.000000,0.000000,0.000000}%
\pgfsetstrokecolor{currentstroke}%
\pgfsetdash{}{0pt}%
\pgfpathmoveto{\pgfqpoint{2.818939in}{1.594344in}}%
\pgfpathlineto{\pgfqpoint{2.833024in}{1.580491in}}%
\pgfpathlineto{\pgfqpoint{2.847105in}{1.566858in}}%
\pgfpathlineto{\pgfqpoint{2.861182in}{1.553443in}}%
\pgfpathlineto{\pgfqpoint{2.875256in}{1.540246in}}%
\pgfpathlineto{\pgfqpoint{2.884144in}{1.537269in}}%
\pgfpathlineto{\pgfqpoint{2.893012in}{1.534685in}}%
\pgfpathlineto{\pgfqpoint{2.901861in}{1.532488in}}%
\pgfpathlineto{\pgfqpoint{2.910691in}{1.530668in}}%
\pgfpathlineto{\pgfqpoint{2.896666in}{1.543056in}}%
\pgfpathlineto{\pgfqpoint{2.882638in}{1.555661in}}%
\pgfpathlineto{\pgfqpoint{2.868607in}{1.568483in}}%
\pgfpathlineto{\pgfqpoint{2.854573in}{1.581524in}}%
\pgfpathlineto{\pgfqpoint{2.845694in}{1.584141in}}%
\pgfpathlineto{\pgfqpoint{2.836796in}{1.587144in}}%
\pgfpathlineto{\pgfqpoint{2.827878in}{1.590542in}}%
\pgfpathlineto{\pgfqpoint{2.818939in}{1.594344in}}%
\pgfpathclose%
\pgfusepath{fill}%
\end{pgfscope}%
\begin{pgfscope}%
\pgfpathrectangle{\pgfqpoint{1.150000in}{0.150000in}}{\pgfqpoint{5.700000in}{5.700000in}}%
\pgfusepath{clip}%
\pgfsetbuttcap%
\pgfsetroundjoin%
\definecolor{currentfill}{rgb}{0.274952,0.037752,0.364543}%
\pgfsetfillcolor{currentfill}%
\pgfsetfillopacity{0.800000}%
\pgfsetlinewidth{0.000000pt}%
\definecolor{currentstroke}{rgb}{0.000000,0.000000,0.000000}%
\pgfsetstrokecolor{currentstroke}%
\pgfsetdash{}{0pt}%
\pgfpathmoveto{\pgfqpoint{3.134852in}{1.361033in}}%
\pgfpathlineto{\pgfqpoint{3.148857in}{1.352167in}}%
\pgfpathlineto{\pgfqpoint{3.162862in}{1.343501in}}%
\pgfpathlineto{\pgfqpoint{3.176868in}{1.335032in}}%
\pgfpathlineto{\pgfqpoint{3.190875in}{1.326760in}}%
\pgfpathlineto{\pgfqpoint{3.199483in}{1.329206in}}%
\pgfpathlineto{\pgfqpoint{3.208076in}{1.331969in}}%
\pgfpathlineto{\pgfqpoint{3.216656in}{1.335041in}}%
\pgfpathlineto{\pgfqpoint{3.225222in}{1.338413in}}%
\pgfpathlineto{\pgfqpoint{3.211250in}{1.345923in}}%
\pgfpathlineto{\pgfqpoint{3.197279in}{1.353629in}}%
\pgfpathlineto{\pgfqpoint{3.183310in}{1.361534in}}%
\pgfpathlineto{\pgfqpoint{3.169341in}{1.369637in}}%
\pgfpathlineto{\pgfqpoint{3.160740in}{1.367013in}}%
\pgfpathlineto{\pgfqpoint{3.152126in}{1.364700in}}%
\pgfpathlineto{\pgfqpoint{3.143496in}{1.362704in}}%
\pgfpathlineto{\pgfqpoint{3.134852in}{1.361033in}}%
\pgfpathclose%
\pgfusepath{fill}%
\end{pgfscope}%
\begin{pgfscope}%
\pgfpathrectangle{\pgfqpoint{1.150000in}{0.150000in}}{\pgfqpoint{5.700000in}{5.700000in}}%
\pgfusepath{clip}%
\pgfsetbuttcap%
\pgfsetroundjoin%
\definecolor{currentfill}{rgb}{0.154815,0.493313,0.557840}%
\pgfsetfillcolor{currentfill}%
\pgfsetfillopacity{0.800000}%
\pgfsetlinewidth{0.000000pt}%
\definecolor{currentstroke}{rgb}{0.000000,0.000000,0.000000}%
\pgfsetstrokecolor{currentstroke}%
\pgfsetdash{}{0pt}%
\pgfpathmoveto{\pgfqpoint{2.114494in}{2.600006in}}%
\pgfpathlineto{\pgfqpoint{2.129034in}{2.573242in}}%
\pgfpathlineto{\pgfqpoint{2.143557in}{2.546801in}}%
\pgfpathlineto{\pgfqpoint{2.158063in}{2.520680in}}%
\pgfpathlineto{\pgfqpoint{2.172552in}{2.494875in}}%
\pgfpathlineto{\pgfqpoint{2.182168in}{2.482812in}}%
\pgfpathlineto{\pgfqpoint{2.191752in}{2.471240in}}%
\pgfpathlineto{\pgfqpoint{2.201305in}{2.460149in}}%
\pgfpathlineto{\pgfqpoint{2.210828in}{2.449531in}}%
\pgfpathlineto{\pgfqpoint{2.196416in}{2.474491in}}%
\pgfpathlineto{\pgfqpoint{2.181988in}{2.499765in}}%
\pgfpathlineto{\pgfqpoint{2.167544in}{2.525357in}}%
\pgfpathlineto{\pgfqpoint{2.153084in}{2.551269in}}%
\pgfpathlineto{\pgfqpoint{2.143484in}{2.562719in}}%
\pgfpathlineto{\pgfqpoint{2.133853in}{2.574652in}}%
\pgfpathlineto{\pgfqpoint{2.124190in}{2.587078in}}%
\pgfpathlineto{\pgfqpoint{2.114494in}{2.600006in}}%
\pgfpathclose%
\pgfusepath{fill}%
\end{pgfscope}%
\begin{pgfscope}%
\pgfpathrectangle{\pgfqpoint{1.150000in}{0.150000in}}{\pgfqpoint{5.700000in}{5.700000in}}%
\pgfusepath{clip}%
\pgfsetbuttcap%
\pgfsetroundjoin%
\definecolor{currentfill}{rgb}{0.229739,0.322361,0.545706}%
\pgfsetfillcolor{currentfill}%
\pgfsetfillopacity{0.800000}%
\pgfsetlinewidth{0.000000pt}%
\definecolor{currentstroke}{rgb}{0.000000,0.000000,0.000000}%
\pgfsetstrokecolor{currentstroke}%
\pgfsetdash{}{0pt}%
\pgfpathmoveto{\pgfqpoint{2.422091in}{2.077162in}}%
\pgfpathlineto{\pgfqpoint{2.436376in}{2.056560in}}%
\pgfpathlineto{\pgfqpoint{2.450650in}{2.036221in}}%
\pgfpathlineto{\pgfqpoint{2.464914in}{2.016146in}}%
\pgfpathlineto{\pgfqpoint{2.479169in}{1.996330in}}%
\pgfpathlineto{\pgfqpoint{2.488465in}{1.987488in}}%
\pgfpathlineto{\pgfqpoint{2.497735in}{1.979108in}}%
\pgfpathlineto{\pgfqpoint{2.506978in}{1.971182in}}%
\pgfpathlineto{\pgfqpoint{2.516195in}{1.963701in}}%
\pgfpathlineto{\pgfqpoint{2.502007in}{1.982675in}}%
\pgfpathlineto{\pgfqpoint{2.487809in}{2.001907in}}%
\pgfpathlineto{\pgfqpoint{2.473602in}{2.021400in}}%
\pgfpathlineto{\pgfqpoint{2.459385in}{2.041156in}}%
\pgfpathlineto{\pgfqpoint{2.450102in}{2.049465in}}%
\pgfpathlineto{\pgfqpoint{2.440793in}{2.058231in}}%
\pgfpathlineto{\pgfqpoint{2.431456in}{2.067460in}}%
\pgfpathlineto{\pgfqpoint{2.422091in}{2.077162in}}%
\pgfpathclose%
\pgfusepath{fill}%
\end{pgfscope}%
\begin{pgfscope}%
\pgfpathrectangle{\pgfqpoint{1.150000in}{0.150000in}}{\pgfqpoint{5.700000in}{5.700000in}}%
\pgfusepath{clip}%
\pgfsetbuttcap%
\pgfsetroundjoin%
\definecolor{currentfill}{rgb}{0.121831,0.589055,0.545623}%
\pgfsetfillcolor{currentfill}%
\pgfsetfillopacity{0.800000}%
\pgfsetlinewidth{0.000000pt}%
\definecolor{currentstroke}{rgb}{0.000000,0.000000,0.000000}%
\pgfsetstrokecolor{currentstroke}%
\pgfsetdash{}{0pt}%
\pgfpathmoveto{\pgfqpoint{4.992024in}{2.774306in}}%
\pgfpathlineto{\pgfqpoint{5.006691in}{2.789056in}}%
\pgfpathlineto{\pgfqpoint{5.021378in}{2.803993in}}%
\pgfpathlineto{\pgfqpoint{5.036085in}{2.819118in}}%
\pgfpathlineto{\pgfqpoint{5.050813in}{2.834431in}}%
\pgfpathlineto{\pgfqpoint{5.058655in}{2.844813in}}%
\pgfpathlineto{\pgfqpoint{5.066488in}{2.855015in}}%
\pgfpathlineto{\pgfqpoint{5.074313in}{2.865036in}}%
\pgfpathlineto{\pgfqpoint{5.082129in}{2.874878in}}%
\pgfpathlineto{\pgfqpoint{5.067403in}{2.859582in}}%
\pgfpathlineto{\pgfqpoint{5.052698in}{2.844475in}}%
\pgfpathlineto{\pgfqpoint{5.038013in}{2.829555in}}%
\pgfpathlineto{\pgfqpoint{5.023348in}{2.814822in}}%
\pgfpathlineto{\pgfqpoint{5.015529in}{2.804950in}}%
\pgfpathlineto{\pgfqpoint{5.007702in}{2.794907in}}%
\pgfpathlineto{\pgfqpoint{4.999867in}{2.784693in}}%
\pgfpathlineto{\pgfqpoint{4.992024in}{2.774306in}}%
\pgfpathclose%
\pgfusepath{fill}%
\end{pgfscope}%
\begin{pgfscope}%
\pgfpathrectangle{\pgfqpoint{1.150000in}{0.150000in}}{\pgfqpoint{5.700000in}{5.700000in}}%
\pgfusepath{clip}%
\pgfsetbuttcap%
\pgfsetroundjoin%
\definecolor{currentfill}{rgb}{0.274952,0.037752,0.364543}%
\pgfsetfillcolor{currentfill}%
\pgfsetfillopacity{0.800000}%
\pgfsetlinewidth{0.000000pt}%
\definecolor{currentstroke}{rgb}{0.000000,0.000000,0.000000}%
\pgfsetstrokecolor{currentstroke}%
\pgfsetdash{}{0pt}%
\pgfpathmoveto{\pgfqpoint{3.661590in}{1.326787in}}%
\pgfpathlineto{\pgfqpoint{3.675599in}{1.326081in}}%
\pgfpathlineto{\pgfqpoint{3.689616in}{1.325558in}}%
\pgfpathlineto{\pgfqpoint{3.703640in}{1.325217in}}%
\pgfpathlineto{\pgfqpoint{3.717671in}{1.325059in}}%
\pgfpathlineto{\pgfqpoint{3.725956in}{1.336177in}}%
\pgfpathlineto{\pgfqpoint{3.734235in}{1.347443in}}%
\pgfpathlineto{\pgfqpoint{3.742507in}{1.358850in}}%
\pgfpathlineto{\pgfqpoint{3.750773in}{1.370393in}}%
\pgfpathlineto{\pgfqpoint{3.736754in}{1.369920in}}%
\pgfpathlineto{\pgfqpoint{3.722743in}{1.369629in}}%
\pgfpathlineto{\pgfqpoint{3.708739in}{1.369522in}}%
\pgfpathlineto{\pgfqpoint{3.694743in}{1.369597in}}%
\pgfpathlineto{\pgfqpoint{3.686465in}{1.358674in}}%
\pgfpathlineto{\pgfqpoint{3.678180in}{1.347894in}}%
\pgfpathlineto{\pgfqpoint{3.669889in}{1.337263in}}%
\pgfpathlineto{\pgfqpoint{3.661590in}{1.326787in}}%
\pgfpathclose%
\pgfusepath{fill}%
\end{pgfscope}%
\begin{pgfscope}%
\pgfpathrectangle{\pgfqpoint{1.150000in}{0.150000in}}{\pgfqpoint{5.700000in}{5.700000in}}%
\pgfusepath{clip}%
\pgfsetbuttcap%
\pgfsetroundjoin%
\definecolor{currentfill}{rgb}{0.278791,0.062145,0.386592}%
\pgfsetfillcolor{currentfill}%
\pgfsetfillopacity{0.800000}%
\pgfsetlinewidth{0.000000pt}%
\definecolor{currentstroke}{rgb}{0.000000,0.000000,0.000000}%
\pgfsetstrokecolor{currentstroke}%
\pgfsetdash{}{0pt}%
\pgfpathmoveto{\pgfqpoint{3.750773in}{1.370393in}}%
\pgfpathlineto{\pgfqpoint{3.764800in}{1.371048in}}%
\pgfpathlineto{\pgfqpoint{3.778836in}{1.371885in}}%
\pgfpathlineto{\pgfqpoint{3.792879in}{1.372903in}}%
\pgfpathlineto{\pgfqpoint{3.806931in}{1.374102in}}%
\pgfpathlineto{\pgfqpoint{3.815181in}{1.386387in}}%
\pgfpathlineto{\pgfqpoint{3.823425in}{1.398787in}}%
\pgfpathlineto{\pgfqpoint{3.831664in}{1.411296in}}%
\pgfpathlineto{\pgfqpoint{3.839897in}{1.423910in}}%
\pgfpathlineto{\pgfqpoint{3.825854in}{1.422109in}}%
\pgfpathlineto{\pgfqpoint{3.811819in}{1.420490in}}%
\pgfpathlineto{\pgfqpoint{3.797794in}{1.419053in}}%
\pgfpathlineto{\pgfqpoint{3.783777in}{1.417797in}}%
\pgfpathlineto{\pgfqpoint{3.775535in}{1.405773in}}%
\pgfpathlineto{\pgfqpoint{3.767287in}{1.393860in}}%
\pgfpathlineto{\pgfqpoint{3.759033in}{1.382065in}}%
\pgfpathlineto{\pgfqpoint{3.750773in}{1.370393in}}%
\pgfpathclose%
\pgfusepath{fill}%
\end{pgfscope}%
\begin{pgfscope}%
\pgfpathrectangle{\pgfqpoint{1.150000in}{0.150000in}}{\pgfqpoint{5.700000in}{5.700000in}}%
\pgfusepath{clip}%
\pgfsetbuttcap%
\pgfsetroundjoin%
\definecolor{currentfill}{rgb}{0.278826,0.175490,0.483397}%
\pgfsetfillcolor{currentfill}%
\pgfsetfillopacity{0.800000}%
\pgfsetlinewidth{0.000000pt}%
\definecolor{currentstroke}{rgb}{0.000000,0.000000,0.000000}%
\pgfsetstrokecolor{currentstroke}%
\pgfsetdash{}{0pt}%
\pgfpathmoveto{\pgfqpoint{4.050860in}{1.615304in}}%
\pgfpathlineto{\pgfqpoint{4.064981in}{1.620355in}}%
\pgfpathlineto{\pgfqpoint{4.079115in}{1.625588in}}%
\pgfpathlineto{\pgfqpoint{4.093260in}{1.631001in}}%
\pgfpathlineto{\pgfqpoint{4.107417in}{1.636595in}}%
\pgfpathlineto{\pgfqpoint{4.115576in}{1.651549in}}%
\pgfpathlineto{\pgfqpoint{4.123732in}{1.666511in}}%
\pgfpathlineto{\pgfqpoint{4.131883in}{1.681479in}}%
\pgfpathlineto{\pgfqpoint{4.140030in}{1.696446in}}%
\pgfpathlineto{\pgfqpoint{4.125874in}{1.690372in}}%
\pgfpathlineto{\pgfqpoint{4.111729in}{1.684479in}}%
\pgfpathlineto{\pgfqpoint{4.097597in}{1.678767in}}%
\pgfpathlineto{\pgfqpoint{4.083477in}{1.673237in}}%
\pgfpathlineto{\pgfqpoint{4.075329in}{1.658737in}}%
\pgfpathlineto{\pgfqpoint{4.067177in}{1.644245in}}%
\pgfpathlineto{\pgfqpoint{4.059021in}{1.629766in}}%
\pgfpathlineto{\pgfqpoint{4.050860in}{1.615304in}}%
\pgfpathclose%
\pgfusepath{fill}%
\end{pgfscope}%
\begin{pgfscope}%
\pgfpathrectangle{\pgfqpoint{1.150000in}{0.150000in}}{\pgfqpoint{5.700000in}{5.700000in}}%
\pgfusepath{clip}%
\pgfsetbuttcap%
\pgfsetroundjoin%
\definecolor{currentfill}{rgb}{0.271305,0.019942,0.347269}%
\pgfsetfillcolor{currentfill}%
\pgfsetfillopacity{0.800000}%
\pgfsetlinewidth{0.000000pt}%
\definecolor{currentstroke}{rgb}{0.000000,0.000000,0.000000}%
\pgfsetstrokecolor{currentstroke}%
\pgfsetdash{}{0pt}%
\pgfpathmoveto{\pgfqpoint{3.572287in}{1.293873in}}%
\pgfpathlineto{\pgfqpoint{3.586287in}{1.291770in}}%
\pgfpathlineto{\pgfqpoint{3.600293in}{1.289852in}}%
\pgfpathlineto{\pgfqpoint{3.614304in}{1.288117in}}%
\pgfpathlineto{\pgfqpoint{3.628323in}{1.286566in}}%
\pgfpathlineto{\pgfqpoint{3.636651in}{1.296356in}}%
\pgfpathlineto{\pgfqpoint{3.644971in}{1.306328in}}%
\pgfpathlineto{\pgfqpoint{3.653284in}{1.316474in}}%
\pgfpathlineto{\pgfqpoint{3.661590in}{1.326787in}}%
\pgfpathlineto{\pgfqpoint{3.647587in}{1.327677in}}%
\pgfpathlineto{\pgfqpoint{3.633592in}{1.328749in}}%
\pgfpathlineto{\pgfqpoint{3.619603in}{1.330006in}}%
\pgfpathlineto{\pgfqpoint{3.605620in}{1.331447in}}%
\pgfpathlineto{\pgfqpoint{3.597299in}{1.321783in}}%
\pgfpathlineto{\pgfqpoint{3.588969in}{1.312295in}}%
\pgfpathlineto{\pgfqpoint{3.580632in}{1.302990in}}%
\pgfpathlineto{\pgfqpoint{3.572287in}{1.293873in}}%
\pgfpathclose%
\pgfusepath{fill}%
\end{pgfscope}%
\begin{pgfscope}%
\pgfpathrectangle{\pgfqpoint{1.150000in}{0.150000in}}{\pgfqpoint{5.700000in}{5.700000in}}%
\pgfusepath{clip}%
\pgfsetbuttcap%
\pgfsetroundjoin%
\definecolor{currentfill}{rgb}{0.283197,0.115680,0.436115}%
\pgfsetfillcolor{currentfill}%
\pgfsetfillopacity{0.800000}%
\pgfsetlinewidth{0.000000pt}%
\definecolor{currentstroke}{rgb}{0.000000,0.000000,0.000000}%
\pgfsetstrokecolor{currentstroke}%
\pgfsetdash{}{0pt}%
\pgfpathmoveto{\pgfqpoint{2.875256in}{1.540246in}}%
\pgfpathlineto{\pgfqpoint{2.889326in}{1.527265in}}%
\pgfpathlineto{\pgfqpoint{2.903394in}{1.514499in}}%
\pgfpathlineto{\pgfqpoint{2.917458in}{1.501947in}}%
\pgfpathlineto{\pgfqpoint{2.931520in}{1.489608in}}%
\pgfpathlineto{\pgfqpoint{2.940359in}{1.487452in}}%
\pgfpathlineto{\pgfqpoint{2.949180in}{1.485680in}}%
\pgfpathlineto{\pgfqpoint{2.957981in}{1.484285in}}%
\pgfpathlineto{\pgfqpoint{2.966766in}{1.483258in}}%
\pgfpathlineto{\pgfqpoint{2.952750in}{1.494791in}}%
\pgfpathlineto{\pgfqpoint{2.938733in}{1.506537in}}%
\pgfpathlineto{\pgfqpoint{2.924713in}{1.518495in}}%
\pgfpathlineto{\pgfqpoint{2.910691in}{1.530668in}}%
\pgfpathlineto{\pgfqpoint{2.901861in}{1.532488in}}%
\pgfpathlineto{\pgfqpoint{2.893012in}{1.534685in}}%
\pgfpathlineto{\pgfqpoint{2.884144in}{1.537269in}}%
\pgfpathlineto{\pgfqpoint{2.875256in}{1.540246in}}%
\pgfpathclose%
\pgfusepath{fill}%
\end{pgfscope}%
\begin{pgfscope}%
\pgfpathrectangle{\pgfqpoint{1.150000in}{0.150000in}}{\pgfqpoint{5.700000in}{5.700000in}}%
\pgfusepath{clip}%
\pgfsetbuttcap%
\pgfsetroundjoin%
\definecolor{currentfill}{rgb}{0.352360,0.783011,0.392636}%
\pgfsetfillcolor{currentfill}%
\pgfsetfillopacity{0.800000}%
\pgfsetlinewidth{0.000000pt}%
\definecolor{currentstroke}{rgb}{0.000000,0.000000,0.000000}%
\pgfsetstrokecolor{currentstroke}%
\pgfsetdash{}{0pt}%
\pgfpathmoveto{\pgfqpoint{5.625112in}{3.420746in}}%
\pgfpathlineto{\pgfqpoint{5.640234in}{3.438566in}}%
\pgfpathlineto{\pgfqpoint{5.655380in}{3.456576in}}%
\pgfpathlineto{\pgfqpoint{5.670549in}{3.474776in}}%
\pgfpathlineto{\pgfqpoint{5.685744in}{3.493166in}}%
\pgfpathlineto{\pgfqpoint{5.693159in}{3.496096in}}%
\pgfpathlineto{\pgfqpoint{5.700562in}{3.498872in}}%
\pgfpathlineto{\pgfqpoint{5.707953in}{3.501497in}}%
\pgfpathlineto{\pgfqpoint{5.715334in}{3.503975in}}%
\pgfpathlineto{\pgfqpoint{5.700159in}{3.485932in}}%
\pgfpathlineto{\pgfqpoint{5.685008in}{3.468079in}}%
\pgfpathlineto{\pgfqpoint{5.669882in}{3.450414in}}%
\pgfpathlineto{\pgfqpoint{5.654779in}{3.432938in}}%
\pgfpathlineto{\pgfqpoint{5.647379in}{3.430102in}}%
\pgfpathlineto{\pgfqpoint{5.639968in}{3.427127in}}%
\pgfpathlineto{\pgfqpoint{5.632546in}{3.424009in}}%
\pgfpathlineto{\pgfqpoint{5.625112in}{3.420746in}}%
\pgfpathclose%
\pgfusepath{fill}%
\end{pgfscope}%
\begin{pgfscope}%
\pgfpathrectangle{\pgfqpoint{1.150000in}{0.150000in}}{\pgfqpoint{5.700000in}{5.700000in}}%
\pgfusepath{clip}%
\pgfsetbuttcap%
\pgfsetroundjoin%
\definecolor{currentfill}{rgb}{0.281924,0.089666,0.412415}%
\pgfsetfillcolor{currentfill}%
\pgfsetfillopacity{0.800000}%
\pgfsetlinewidth{0.000000pt}%
\definecolor{currentstroke}{rgb}{0.000000,0.000000,0.000000}%
\pgfsetstrokecolor{currentstroke}%
\pgfsetdash{}{0pt}%
\pgfpathmoveto{\pgfqpoint{3.839897in}{1.423910in}}%
\pgfpathlineto{\pgfqpoint{3.853949in}{1.425892in}}%
\pgfpathlineto{\pgfqpoint{3.868011in}{1.428055in}}%
\pgfpathlineto{\pgfqpoint{3.882082in}{1.430399in}}%
\pgfpathlineto{\pgfqpoint{3.896162in}{1.432924in}}%
\pgfpathlineto{\pgfqpoint{3.904383in}{1.446219in}}%
\pgfpathlineto{\pgfqpoint{3.912598in}{1.459599in}}%
\pgfpathlineto{\pgfqpoint{3.920808in}{1.473058in}}%
\pgfpathlineto{\pgfqpoint{3.929014in}{1.486591in}}%
\pgfpathlineto{\pgfqpoint{3.914939in}{1.483495in}}%
\pgfpathlineto{\pgfqpoint{3.900875in}{1.480580in}}%
\pgfpathlineto{\pgfqpoint{3.886820in}{1.477846in}}%
\pgfpathlineto{\pgfqpoint{3.872775in}{1.475294in}}%
\pgfpathlineto{\pgfqpoint{3.864563in}{1.462320in}}%
\pgfpathlineto{\pgfqpoint{3.856346in}{1.449427in}}%
\pgfpathlineto{\pgfqpoint{3.848124in}{1.436622in}}%
\pgfpathlineto{\pgfqpoint{3.839897in}{1.423910in}}%
\pgfpathclose%
\pgfusepath{fill}%
\end{pgfscope}%
\begin{pgfscope}%
\pgfpathrectangle{\pgfqpoint{1.150000in}{0.150000in}}{\pgfqpoint{5.700000in}{5.700000in}}%
\pgfusepath{clip}%
\pgfsetbuttcap%
\pgfsetroundjoin%
\definecolor{currentfill}{rgb}{0.216210,0.351535,0.550627}%
\pgfsetfillcolor{currentfill}%
\pgfsetfillopacity{0.800000}%
\pgfsetlinewidth{0.000000pt}%
\definecolor{currentstroke}{rgb}{0.000000,0.000000,0.000000}%
\pgfsetstrokecolor{currentstroke}%
\pgfsetdash{}{0pt}%
\pgfpathmoveto{\pgfqpoint{2.364842in}{2.162260in}}%
\pgfpathlineto{\pgfqpoint{2.379171in}{2.140579in}}%
\pgfpathlineto{\pgfqpoint{2.393489in}{2.119170in}}%
\pgfpathlineto{\pgfqpoint{2.407795in}{2.098032in}}%
\pgfpathlineto{\pgfqpoint{2.422091in}{2.077162in}}%
\pgfpathlineto{\pgfqpoint{2.431456in}{2.067460in}}%
\pgfpathlineto{\pgfqpoint{2.440793in}{2.058231in}}%
\pgfpathlineto{\pgfqpoint{2.450102in}{2.049465in}}%
\pgfpathlineto{\pgfqpoint{2.459385in}{2.041156in}}%
\pgfpathlineto{\pgfqpoint{2.445157in}{2.061176in}}%
\pgfpathlineto{\pgfqpoint{2.430920in}{2.081464in}}%
\pgfpathlineto{\pgfqpoint{2.416672in}{2.102021in}}%
\pgfpathlineto{\pgfqpoint{2.402413in}{2.122849in}}%
\pgfpathlineto{\pgfqpoint{2.393063in}{2.131994in}}%
\pgfpathlineto{\pgfqpoint{2.383684in}{2.141605in}}%
\pgfpathlineto{\pgfqpoint{2.374278in}{2.151691in}}%
\pgfpathlineto{\pgfqpoint{2.364842in}{2.162260in}}%
\pgfpathclose%
\pgfusepath{fill}%
\end{pgfscope}%
\begin{pgfscope}%
\pgfpathrectangle{\pgfqpoint{1.150000in}{0.150000in}}{\pgfqpoint{5.700000in}{5.700000in}}%
\pgfusepath{clip}%
\pgfsetbuttcap%
\pgfsetroundjoin%
\definecolor{currentfill}{rgb}{0.135066,0.544853,0.554029}%
\pgfsetfillcolor{currentfill}%
\pgfsetfillopacity{0.800000}%
\pgfsetlinewidth{0.000000pt}%
\definecolor{currentstroke}{rgb}{0.000000,0.000000,0.000000}%
\pgfsetstrokecolor{currentstroke}%
\pgfsetdash{}{0pt}%
\pgfpathmoveto{\pgfqpoint{4.870521in}{2.628135in}}%
\pgfpathlineto{\pgfqpoint{4.885109in}{2.642065in}}%
\pgfpathlineto{\pgfqpoint{4.899717in}{2.656182in}}%
\pgfpathlineto{\pgfqpoint{4.914344in}{2.670486in}}%
\pgfpathlineto{\pgfqpoint{4.928991in}{2.684977in}}%
\pgfpathlineto{\pgfqpoint{4.936898in}{2.696751in}}%
\pgfpathlineto{\pgfqpoint{4.944796in}{2.708352in}}%
\pgfpathlineto{\pgfqpoint{4.952687in}{2.719778in}}%
\pgfpathlineto{\pgfqpoint{4.960570in}{2.731031in}}%
\pgfpathlineto{\pgfqpoint{4.945923in}{2.716487in}}%
\pgfpathlineto{\pgfqpoint{4.931295in}{2.702129in}}%
\pgfpathlineto{\pgfqpoint{4.916687in}{2.687959in}}%
\pgfpathlineto{\pgfqpoint{4.902099in}{2.673975in}}%
\pgfpathlineto{\pgfqpoint{4.894215in}{2.662763in}}%
\pgfpathlineto{\pgfqpoint{4.886325in}{2.651385in}}%
\pgfpathlineto{\pgfqpoint{4.878426in}{2.639843in}}%
\pgfpathlineto{\pgfqpoint{4.870521in}{2.628135in}}%
\pgfpathclose%
\pgfusepath{fill}%
\end{pgfscope}%
\begin{pgfscope}%
\pgfpathrectangle{\pgfqpoint{1.150000in}{0.150000in}}{\pgfqpoint{5.700000in}{5.700000in}}%
\pgfusepath{clip}%
\pgfsetbuttcap%
\pgfsetroundjoin%
\definecolor{currentfill}{rgb}{0.268510,0.009605,0.335427}%
\pgfsetfillcolor{currentfill}%
\pgfsetfillopacity{0.800000}%
\pgfsetlinewidth{0.000000pt}%
\definecolor{currentstroke}{rgb}{0.000000,0.000000,0.000000}%
\pgfsetstrokecolor{currentstroke}%
\pgfsetdash{}{0pt}%
\pgfpathmoveto{\pgfqpoint{3.337070in}{1.285343in}}%
\pgfpathlineto{\pgfqpoint{3.351062in}{1.279575in}}%
\pgfpathlineto{\pgfqpoint{3.365057in}{1.273997in}}%
\pgfpathlineto{\pgfqpoint{3.379056in}{1.268608in}}%
\pgfpathlineto{\pgfqpoint{3.393058in}{1.263409in}}%
\pgfpathlineto{\pgfqpoint{3.401523in}{1.269292in}}%
\pgfpathlineto{\pgfqpoint{3.409977in}{1.275437in}}%
\pgfpathlineto{\pgfqpoint{3.418420in}{1.281834in}}%
\pgfpathlineto{\pgfqpoint{3.426853in}{1.288478in}}%
\pgfpathlineto{\pgfqpoint{3.412877in}{1.292952in}}%
\pgfpathlineto{\pgfqpoint{3.398905in}{1.297614in}}%
\pgfpathlineto{\pgfqpoint{3.384936in}{1.302466in}}%
\pgfpathlineto{\pgfqpoint{3.370971in}{1.307508in}}%
\pgfpathlineto{\pgfqpoint{3.362513in}{1.301578in}}%
\pgfpathlineto{\pgfqpoint{3.354043in}{1.295902in}}%
\pgfpathlineto{\pgfqpoint{3.345562in}{1.290488in}}%
\pgfpathlineto{\pgfqpoint{3.337070in}{1.285343in}}%
\pgfpathclose%
\pgfusepath{fill}%
\end{pgfscope}%
\begin{pgfscope}%
\pgfpathrectangle{\pgfqpoint{1.150000in}{0.150000in}}{\pgfqpoint{5.700000in}{5.700000in}}%
\pgfusepath{clip}%
\pgfsetbuttcap%
\pgfsetroundjoin%
\definecolor{currentfill}{rgb}{0.120081,0.622161,0.534946}%
\pgfsetfillcolor{currentfill}%
\pgfsetfillopacity{0.800000}%
\pgfsetlinewidth{0.000000pt}%
\definecolor{currentstroke}{rgb}{0.000000,0.000000,0.000000}%
\pgfsetstrokecolor{currentstroke}%
\pgfsetdash{}{0pt}%
\pgfpathmoveto{\pgfqpoint{1.919179in}{3.012141in}}%
\pgfpathlineto{\pgfqpoint{1.933954in}{2.980643in}}%
\pgfpathlineto{\pgfqpoint{1.948706in}{2.949521in}}%
\pgfpathlineto{\pgfqpoint{1.963436in}{2.918773in}}%
\pgfpathlineto{\pgfqpoint{1.978144in}{2.888393in}}%
\pgfpathlineto{\pgfqpoint{1.987956in}{2.874899in}}%
\pgfpathlineto{\pgfqpoint{1.997734in}{2.861904in}}%
\pgfpathlineto{\pgfqpoint{2.007478in}{2.849401in}}%
\pgfpathlineto{\pgfqpoint{2.017190in}{2.837380in}}%
\pgfpathlineto{\pgfqpoint{2.002566in}{2.866920in}}%
\pgfpathlineto{\pgfqpoint{1.987921in}{2.896826in}}%
\pgfpathlineto{\pgfqpoint{1.973255in}{2.927102in}}%
\pgfpathlineto{\pgfqpoint{1.958566in}{2.957752in}}%
\pgfpathlineto{\pgfqpoint{1.948771in}{2.970599in}}%
\pgfpathlineto{\pgfqpoint{1.938942in}{2.983941in}}%
\pgfpathlineto{\pgfqpoint{1.929078in}{2.997785in}}%
\pgfpathlineto{\pgfqpoint{1.919179in}{3.012141in}}%
\pgfpathclose%
\pgfusepath{fill}%
\end{pgfscope}%
\begin{pgfscope}%
\pgfpathrectangle{\pgfqpoint{1.150000in}{0.150000in}}{\pgfqpoint{5.700000in}{5.700000in}}%
\pgfusepath{clip}%
\pgfsetbuttcap%
\pgfsetroundjoin%
\definecolor{currentfill}{rgb}{0.153364,0.497000,0.557724}%
\pgfsetfillcolor{currentfill}%
\pgfsetfillopacity{0.800000}%
\pgfsetlinewidth{0.000000pt}%
\definecolor{currentstroke}{rgb}{0.000000,0.000000,0.000000}%
\pgfsetstrokecolor{currentstroke}%
\pgfsetdash{}{0pt}%
\pgfpathmoveto{\pgfqpoint{4.748870in}{2.475568in}}%
\pgfpathlineto{\pgfqpoint{4.763380in}{2.488543in}}%
\pgfpathlineto{\pgfqpoint{4.777908in}{2.501703in}}%
\pgfpathlineto{\pgfqpoint{4.792455in}{2.515050in}}%
\pgfpathlineto{\pgfqpoint{4.807021in}{2.528582in}}%
\pgfpathlineto{\pgfqpoint{4.814982in}{2.541594in}}%
\pgfpathlineto{\pgfqpoint{4.822937in}{2.554445in}}%
\pgfpathlineto{\pgfqpoint{4.830885in}{2.567135in}}%
\pgfpathlineto{\pgfqpoint{4.838827in}{2.579662in}}%
\pgfpathlineto{\pgfqpoint{4.824259in}{2.566006in}}%
\pgfpathlineto{\pgfqpoint{4.809710in}{2.552537in}}%
\pgfpathlineto{\pgfqpoint{4.795179in}{2.539254in}}%
\pgfpathlineto{\pgfqpoint{4.780668in}{2.526157in}}%
\pgfpathlineto{\pgfqpoint{4.772728in}{2.513739in}}%
\pgfpathlineto{\pgfqpoint{4.764782in}{2.501168in}}%
\pgfpathlineto{\pgfqpoint{4.756830in}{2.488444in}}%
\pgfpathlineto{\pgfqpoint{4.748870in}{2.475568in}}%
\pgfpathclose%
\pgfusepath{fill}%
\end{pgfscope}%
\begin{pgfscope}%
\pgfpathrectangle{\pgfqpoint{1.150000in}{0.150000in}}{\pgfqpoint{5.700000in}{5.700000in}}%
\pgfusepath{clip}%
\pgfsetbuttcap%
\pgfsetroundjoin%
\definecolor{currentfill}{rgb}{0.140210,0.665859,0.513427}%
\pgfsetfillcolor{currentfill}%
\pgfsetfillopacity{0.800000}%
\pgfsetlinewidth{0.000000pt}%
\definecolor{currentstroke}{rgb}{0.000000,0.000000,0.000000}%
\pgfsetstrokecolor{currentstroke}%
\pgfsetdash{}{0pt}%
\pgfpathmoveto{\pgfqpoint{5.203422in}{3.009752in}}%
\pgfpathlineto{\pgfqpoint{5.218245in}{3.025846in}}%
\pgfpathlineto{\pgfqpoint{5.233091in}{3.042129in}}%
\pgfpathlineto{\pgfqpoint{5.247958in}{3.058601in}}%
\pgfpathlineto{\pgfqpoint{5.262847in}{3.075262in}}%
\pgfpathlineto{\pgfqpoint{5.270572in}{3.083310in}}%
\pgfpathlineto{\pgfqpoint{5.278286in}{3.091173in}}%
\pgfpathlineto{\pgfqpoint{5.285991in}{3.098851in}}%
\pgfpathlineto{\pgfqpoint{5.293686in}{3.106348in}}%
\pgfpathlineto{\pgfqpoint{5.278803in}{3.089813in}}%
\pgfpathlineto{\pgfqpoint{5.263942in}{3.073467in}}%
\pgfpathlineto{\pgfqpoint{5.249103in}{3.057309in}}%
\pgfpathlineto{\pgfqpoint{5.234286in}{3.041339in}}%
\pgfpathlineto{\pgfqpoint{5.226584in}{3.033704in}}%
\pgfpathlineto{\pgfqpoint{5.218873in}{3.025895in}}%
\pgfpathlineto{\pgfqpoint{5.211152in}{3.017912in}}%
\pgfpathlineto{\pgfqpoint{5.203422in}{3.009752in}}%
\pgfpathclose%
\pgfusepath{fill}%
\end{pgfscope}%
\begin{pgfscope}%
\pgfpathrectangle{\pgfqpoint{1.150000in}{0.150000in}}{\pgfqpoint{5.700000in}{5.700000in}}%
\pgfusepath{clip}%
\pgfsetbuttcap%
\pgfsetroundjoin%
\definecolor{currentfill}{rgb}{0.174274,0.445044,0.557792}%
\pgfsetfillcolor{currentfill}%
\pgfsetfillopacity{0.800000}%
\pgfsetlinewidth{0.000000pt}%
\definecolor{currentstroke}{rgb}{0.000000,0.000000,0.000000}%
\pgfsetstrokecolor{currentstroke}%
\pgfsetdash{}{0pt}%
\pgfpathmoveto{\pgfqpoint{4.627136in}{2.318559in}}%
\pgfpathlineto{\pgfqpoint{4.641568in}{2.330445in}}%
\pgfpathlineto{\pgfqpoint{4.656018in}{2.342515in}}%
\pgfpathlineto{\pgfqpoint{4.670486in}{2.354771in}}%
\pgfpathlineto{\pgfqpoint{4.684971in}{2.367211in}}%
\pgfpathlineto{\pgfqpoint{4.692980in}{2.381264in}}%
\pgfpathlineto{\pgfqpoint{4.700982in}{2.395175in}}%
\pgfpathlineto{\pgfqpoint{4.708979in}{2.408941in}}%
\pgfpathlineto{\pgfqpoint{4.716970in}{2.422563in}}%
\pgfpathlineto{\pgfqpoint{4.702481in}{2.409931in}}%
\pgfpathlineto{\pgfqpoint{4.688010in}{2.397484in}}%
\pgfpathlineto{\pgfqpoint{4.673557in}{2.385223in}}%
\pgfpathlineto{\pgfqpoint{4.659122in}{2.373147in}}%
\pgfpathlineto{\pgfqpoint{4.651134in}{2.359704in}}%
\pgfpathlineto{\pgfqpoint{4.643140in}{2.346124in}}%
\pgfpathlineto{\pgfqpoint{4.635141in}{2.332409in}}%
\pgfpathlineto{\pgfqpoint{4.627136in}{2.318559in}}%
\pgfpathclose%
\pgfusepath{fill}%
\end{pgfscope}%
\begin{pgfscope}%
\pgfpathrectangle{\pgfqpoint{1.150000in}{0.150000in}}{\pgfqpoint{5.700000in}{5.700000in}}%
\pgfusepath{clip}%
\pgfsetbuttcap%
\pgfsetroundjoin%
\definecolor{currentfill}{rgb}{0.197636,0.391528,0.554969}%
\pgfsetfillcolor{currentfill}%
\pgfsetfillopacity{0.800000}%
\pgfsetlinewidth{0.000000pt}%
\definecolor{currentstroke}{rgb}{0.000000,0.000000,0.000000}%
\pgfsetstrokecolor{currentstroke}%
\pgfsetdash{}{0pt}%
\pgfpathmoveto{\pgfqpoint{4.505367in}{2.159389in}}%
\pgfpathlineto{\pgfqpoint{4.519725in}{2.170055in}}%
\pgfpathlineto{\pgfqpoint{4.534099in}{2.180905in}}%
\pgfpathlineto{\pgfqpoint{4.548490in}{2.191939in}}%
\pgfpathlineto{\pgfqpoint{4.562897in}{2.203158in}}%
\pgfpathlineto{\pgfqpoint{4.570946in}{2.218013in}}%
\pgfpathlineto{\pgfqpoint{4.578989in}{2.232749in}}%
\pgfpathlineto{\pgfqpoint{4.587027in}{2.247365in}}%
\pgfpathlineto{\pgfqpoint{4.595060in}{2.261858in}}%
\pgfpathlineto{\pgfqpoint{4.580648in}{2.250381in}}%
\pgfpathlineto{\pgfqpoint{4.566253in}{2.239089in}}%
\pgfpathlineto{\pgfqpoint{4.551875in}{2.227981in}}%
\pgfpathlineto{\pgfqpoint{4.537514in}{2.217058in}}%
\pgfpathlineto{\pgfqpoint{4.529485in}{2.202810in}}%
\pgfpathlineto{\pgfqpoint{4.521450in}{2.188448in}}%
\pgfpathlineto{\pgfqpoint{4.513411in}{2.173974in}}%
\pgfpathlineto{\pgfqpoint{4.505367in}{2.159389in}}%
\pgfpathclose%
\pgfusepath{fill}%
\end{pgfscope}%
\begin{pgfscope}%
\pgfpathrectangle{\pgfqpoint{1.150000in}{0.150000in}}{\pgfqpoint{5.700000in}{5.700000in}}%
\pgfusepath{clip}%
\pgfsetbuttcap%
\pgfsetroundjoin%
\definecolor{currentfill}{rgb}{0.282656,0.100196,0.422160}%
\pgfsetfillcolor{currentfill}%
\pgfsetfillopacity{0.800000}%
\pgfsetlinewidth{0.000000pt}%
\definecolor{currentstroke}{rgb}{0.000000,0.000000,0.000000}%
\pgfsetstrokecolor{currentstroke}%
\pgfsetdash{}{0pt}%
\pgfpathmoveto{\pgfqpoint{2.931520in}{1.489608in}}%
\pgfpathlineto{\pgfqpoint{2.945580in}{1.477481in}}%
\pgfpathlineto{\pgfqpoint{2.959637in}{1.465564in}}%
\pgfpathlineto{\pgfqpoint{2.973692in}{1.453857in}}%
\pgfpathlineto{\pgfqpoint{2.987745in}{1.442358in}}%
\pgfpathlineto{\pgfqpoint{2.996537in}{1.441020in}}%
\pgfpathlineto{\pgfqpoint{3.005312in}{1.440057in}}%
\pgfpathlineto{\pgfqpoint{3.014069in}{1.439461in}}%
\pgfpathlineto{\pgfqpoint{3.022809in}{1.439225in}}%
\pgfpathlineto{\pgfqpoint{3.008801in}{1.449920in}}%
\pgfpathlineto{\pgfqpoint{2.994790in}{1.460823in}}%
\pgfpathlineto{\pgfqpoint{2.980779in}{1.471936in}}%
\pgfpathlineto{\pgfqpoint{2.966766in}{1.483258in}}%
\pgfpathlineto{\pgfqpoint{2.957981in}{1.484285in}}%
\pgfpathlineto{\pgfqpoint{2.949180in}{1.485680in}}%
\pgfpathlineto{\pgfqpoint{2.940359in}{1.487452in}}%
\pgfpathlineto{\pgfqpoint{2.931520in}{1.489608in}}%
\pgfpathclose%
\pgfusepath{fill}%
\end{pgfscope}%
\begin{pgfscope}%
\pgfpathrectangle{\pgfqpoint{1.150000in}{0.150000in}}{\pgfqpoint{5.700000in}{5.700000in}}%
\pgfusepath{clip}%
\pgfsetbuttcap%
\pgfsetroundjoin%
\definecolor{currentfill}{rgb}{0.221989,0.339161,0.548752}%
\pgfsetfillcolor{currentfill}%
\pgfsetfillopacity{0.800000}%
\pgfsetlinewidth{0.000000pt}%
\definecolor{currentstroke}{rgb}{0.000000,0.000000,0.000000}%
\pgfsetstrokecolor{currentstroke}%
\pgfsetdash{}{0pt}%
\pgfpathmoveto{\pgfqpoint{4.383592in}{2.000655in}}%
\pgfpathlineto{\pgfqpoint{4.397880in}{2.009975in}}%
\pgfpathlineto{\pgfqpoint{4.412184in}{2.019478in}}%
\pgfpathlineto{\pgfqpoint{4.426502in}{2.029163in}}%
\pgfpathlineto{\pgfqpoint{4.440836in}{2.039032in}}%
\pgfpathlineto{\pgfqpoint{4.448919in}{2.054410in}}%
\pgfpathlineto{\pgfqpoint{4.456997in}{2.069698in}}%
\pgfpathlineto{\pgfqpoint{4.465071in}{2.084893in}}%
\pgfpathlineto{\pgfqpoint{4.473140in}{2.099994in}}%
\pgfpathlineto{\pgfqpoint{4.458801in}{2.089801in}}%
\pgfpathlineto{\pgfqpoint{4.444479in}{2.079792in}}%
\pgfpathlineto{\pgfqpoint{4.430172in}{2.069967in}}%
\pgfpathlineto{\pgfqpoint{4.415881in}{2.060324in}}%
\pgfpathlineto{\pgfqpoint{4.407815in}{2.045535in}}%
\pgfpathlineto{\pgfqpoint{4.399746in}{2.030658in}}%
\pgfpathlineto{\pgfqpoint{4.391671in}{2.015697in}}%
\pgfpathlineto{\pgfqpoint{4.383592in}{2.000655in}}%
\pgfpathclose%
\pgfusepath{fill}%
\end{pgfscope}%
\begin{pgfscope}%
\pgfpathrectangle{\pgfqpoint{1.150000in}{0.150000in}}{\pgfqpoint{5.700000in}{5.700000in}}%
\pgfusepath{clip}%
\pgfsetbuttcap%
\pgfsetroundjoin%
\definecolor{currentfill}{rgb}{0.232815,0.732247,0.459277}%
\pgfsetfillcolor{currentfill}%
\pgfsetfillopacity{0.800000}%
\pgfsetlinewidth{0.000000pt}%
\definecolor{currentstroke}{rgb}{0.000000,0.000000,0.000000}%
\pgfsetstrokecolor{currentstroke}%
\pgfsetdash{}{0pt}%
\pgfpathmoveto{\pgfqpoint{5.414563in}{3.226608in}}%
\pgfpathlineto{\pgfqpoint{5.429540in}{3.243728in}}%
\pgfpathlineto{\pgfqpoint{5.444540in}{3.261038in}}%
\pgfpathlineto{\pgfqpoint{5.459563in}{3.278537in}}%
\pgfpathlineto{\pgfqpoint{5.474610in}{3.296227in}}%
\pgfpathlineto{\pgfqpoint{5.482191in}{3.301739in}}%
\pgfpathlineto{\pgfqpoint{5.489761in}{3.307074in}}%
\pgfpathlineto{\pgfqpoint{5.497321in}{3.312235in}}%
\pgfpathlineto{\pgfqpoint{5.504869in}{3.317225in}}%
\pgfpathlineto{\pgfqpoint{5.489835in}{3.299772in}}%
\pgfpathlineto{\pgfqpoint{5.474824in}{3.282508in}}%
\pgfpathlineto{\pgfqpoint{5.459836in}{3.265433in}}%
\pgfpathlineto{\pgfqpoint{5.444872in}{3.248546in}}%
\pgfpathlineto{\pgfqpoint{5.437310in}{3.243308in}}%
\pgfpathlineto{\pgfqpoint{5.429738in}{3.237907in}}%
\pgfpathlineto{\pgfqpoint{5.422156in}{3.232341in}}%
\pgfpathlineto{\pgfqpoint{5.414563in}{3.226608in}}%
\pgfpathclose%
\pgfusepath{fill}%
\end{pgfscope}%
\begin{pgfscope}%
\pgfpathrectangle{\pgfqpoint{1.150000in}{0.150000in}}{\pgfqpoint{5.700000in}{5.700000in}}%
\pgfusepath{clip}%
\pgfsetbuttcap%
\pgfsetroundjoin%
\definecolor{currentfill}{rgb}{0.268510,0.009605,0.335427}%
\pgfsetfillcolor{currentfill}%
\pgfsetfillopacity{0.800000}%
\pgfsetlinewidth{0.000000pt}%
\definecolor{currentstroke}{rgb}{0.000000,0.000000,0.000000}%
\pgfsetstrokecolor{currentstroke}%
\pgfsetdash{}{0pt}%
\pgfpathmoveto{\pgfqpoint{3.482800in}{1.272463in}}%
\pgfpathlineto{\pgfqpoint{3.496799in}{1.268926in}}%
\pgfpathlineto{\pgfqpoint{3.510802in}{1.265575in}}%
\pgfpathlineto{\pgfqpoint{3.524810in}{1.262410in}}%
\pgfpathlineto{\pgfqpoint{3.538823in}{1.259429in}}%
\pgfpathlineto{\pgfqpoint{3.547202in}{1.267723in}}%
\pgfpathlineto{\pgfqpoint{3.555572in}{1.276233in}}%
\pgfpathlineto{\pgfqpoint{3.563934in}{1.284952in}}%
\pgfpathlineto{\pgfqpoint{3.572287in}{1.293873in}}%
\pgfpathlineto{\pgfqpoint{3.558294in}{1.296160in}}%
\pgfpathlineto{\pgfqpoint{3.544305in}{1.298632in}}%
\pgfpathlineto{\pgfqpoint{3.530323in}{1.301290in}}%
\pgfpathlineto{\pgfqpoint{3.516346in}{1.304133in}}%
\pgfpathlineto{\pgfqpoint{3.507973in}{1.295893in}}%
\pgfpathlineto{\pgfqpoint{3.499591in}{1.287864in}}%
\pgfpathlineto{\pgfqpoint{3.491201in}{1.280051in}}%
\pgfpathlineto{\pgfqpoint{3.482800in}{1.272463in}}%
\pgfpathclose%
\pgfusepath{fill}%
\end{pgfscope}%
\begin{pgfscope}%
\pgfpathrectangle{\pgfqpoint{1.150000in}{0.150000in}}{\pgfqpoint{5.700000in}{5.700000in}}%
\pgfusepath{clip}%
\pgfsetbuttcap%
\pgfsetroundjoin%
\definecolor{currentfill}{rgb}{0.283187,0.125848,0.444960}%
\pgfsetfillcolor{currentfill}%
\pgfsetfillopacity{0.800000}%
\pgfsetlinewidth{0.000000pt}%
\definecolor{currentstroke}{rgb}{0.000000,0.000000,0.000000}%
\pgfsetstrokecolor{currentstroke}%
\pgfsetdash{}{0pt}%
\pgfpathmoveto{\pgfqpoint{3.929014in}{1.486591in}}%
\pgfpathlineto{\pgfqpoint{3.943099in}{1.489868in}}%
\pgfpathlineto{\pgfqpoint{3.957194in}{1.493326in}}%
\pgfpathlineto{\pgfqpoint{3.971299in}{1.496964in}}%
\pgfpathlineto{\pgfqpoint{3.985415in}{1.500782in}}%
\pgfpathlineto{\pgfqpoint{3.993611in}{1.514938in}}%
\pgfpathlineto{\pgfqpoint{4.001803in}{1.529149in}}%
\pgfpathlineto{\pgfqpoint{4.009990in}{1.543410in}}%
\pgfpathlineto{\pgfqpoint{4.018173in}{1.557717in}}%
\pgfpathlineto{\pgfqpoint{4.004060in}{1.553357in}}%
\pgfpathlineto{\pgfqpoint{3.989959in}{1.549178in}}%
\pgfpathlineto{\pgfqpoint{3.975868in}{1.545179in}}%
\pgfpathlineto{\pgfqpoint{3.961788in}{1.541362in}}%
\pgfpathlineto{\pgfqpoint{3.953601in}{1.527584in}}%
\pgfpathlineto{\pgfqpoint{3.945410in}{1.513860in}}%
\pgfpathlineto{\pgfqpoint{3.937215in}{1.500194in}}%
\pgfpathlineto{\pgfqpoint{3.929014in}{1.486591in}}%
\pgfpathclose%
\pgfusepath{fill}%
\end{pgfscope}%
\begin{pgfscope}%
\pgfpathrectangle{\pgfqpoint{1.150000in}{0.150000in}}{\pgfqpoint{5.700000in}{5.700000in}}%
\pgfusepath{clip}%
\pgfsetbuttcap%
\pgfsetroundjoin%
\definecolor{currentfill}{rgb}{0.272594,0.025563,0.353093}%
\pgfsetfillcolor{currentfill}%
\pgfsetfillopacity{0.800000}%
\pgfsetlinewidth{0.000000pt}%
\definecolor{currentstroke}{rgb}{0.000000,0.000000,0.000000}%
\pgfsetstrokecolor{currentstroke}%
\pgfsetdash{}{0pt}%
\pgfpathmoveto{\pgfqpoint{3.190875in}{1.326760in}}%
\pgfpathlineto{\pgfqpoint{3.204883in}{1.318685in}}%
\pgfpathlineto{\pgfqpoint{3.218892in}{1.310806in}}%
\pgfpathlineto{\pgfqpoint{3.232902in}{1.303123in}}%
\pgfpathlineto{\pgfqpoint{3.246914in}{1.295633in}}%
\pgfpathlineto{\pgfqpoint{3.255487in}{1.298853in}}%
\pgfpathlineto{\pgfqpoint{3.264047in}{1.302381in}}%
\pgfpathlineto{\pgfqpoint{3.272594in}{1.306210in}}%
\pgfpathlineto{\pgfqpoint{3.281128in}{1.310331in}}%
\pgfpathlineto{\pgfqpoint{3.267149in}{1.317059in}}%
\pgfpathlineto{\pgfqpoint{3.253172in}{1.323982in}}%
\pgfpathlineto{\pgfqpoint{3.239196in}{1.331100in}}%
\pgfpathlineto{\pgfqpoint{3.225222in}{1.338413in}}%
\pgfpathlineto{\pgfqpoint{3.216656in}{1.335041in}}%
\pgfpathlineto{\pgfqpoint{3.208076in}{1.331969in}}%
\pgfpathlineto{\pgfqpoint{3.199483in}{1.329206in}}%
\pgfpathlineto{\pgfqpoint{3.190875in}{1.326760in}}%
\pgfpathclose%
\pgfusepath{fill}%
\end{pgfscope}%
\begin{pgfscope}%
\pgfpathrectangle{\pgfqpoint{1.150000in}{0.150000in}}{\pgfqpoint{5.700000in}{5.700000in}}%
\pgfusepath{clip}%
\pgfsetbuttcap%
\pgfsetroundjoin%
\definecolor{currentfill}{rgb}{0.248629,0.278775,0.534556}%
\pgfsetfillcolor{currentfill}%
\pgfsetfillopacity{0.800000}%
\pgfsetlinewidth{0.000000pt}%
\definecolor{currentstroke}{rgb}{0.000000,0.000000,0.000000}%
\pgfsetstrokecolor{currentstroke}%
\pgfsetdash{}{0pt}%
\pgfpathmoveto{\pgfqpoint{4.261820in}{1.845268in}}%
\pgfpathlineto{\pgfqpoint{4.276045in}{1.853117in}}%
\pgfpathlineto{\pgfqpoint{4.290283in}{1.861148in}}%
\pgfpathlineto{\pgfqpoint{4.304536in}{1.869360in}}%
\pgfpathlineto{\pgfqpoint{4.318803in}{1.877755in}}%
\pgfpathlineto{\pgfqpoint{4.326916in}{1.893336in}}%
\pgfpathlineto{\pgfqpoint{4.335026in}{1.908861in}}%
\pgfpathlineto{\pgfqpoint{4.343131in}{1.924327in}}%
\pgfpathlineto{\pgfqpoint{4.351232in}{1.939731in}}%
\pgfpathlineto{\pgfqpoint{4.336962in}{1.930949in}}%
\pgfpathlineto{\pgfqpoint{4.322706in}{1.922349in}}%
\pgfpathlineto{\pgfqpoint{4.308465in}{1.913931in}}%
\pgfpathlineto{\pgfqpoint{4.294238in}{1.905696in}}%
\pgfpathlineto{\pgfqpoint{4.286140in}{1.890667in}}%
\pgfpathlineto{\pgfqpoint{4.278038in}{1.875584in}}%
\pgfpathlineto{\pgfqpoint{4.269931in}{1.860449in}}%
\pgfpathlineto{\pgfqpoint{4.261820in}{1.845268in}}%
\pgfpathclose%
\pgfusepath{fill}%
\end{pgfscope}%
\begin{pgfscope}%
\pgfpathrectangle{\pgfqpoint{1.150000in}{0.150000in}}{\pgfqpoint{5.700000in}{5.700000in}}%
\pgfusepath{clip}%
\pgfsetbuttcap%
\pgfsetroundjoin%
\definecolor{currentfill}{rgb}{0.201239,0.383670,0.554294}%
\pgfsetfillcolor{currentfill}%
\pgfsetfillopacity{0.800000}%
\pgfsetlinewidth{0.000000pt}%
\definecolor{currentstroke}{rgb}{0.000000,0.000000,0.000000}%
\pgfsetstrokecolor{currentstroke}%
\pgfsetdash{}{0pt}%
\pgfpathmoveto{\pgfqpoint{2.307403in}{2.251762in}}%
\pgfpathlineto{\pgfqpoint{2.321782in}{2.228966in}}%
\pgfpathlineto{\pgfqpoint{2.336147in}{2.206452in}}%
\pgfpathlineto{\pgfqpoint{2.350501in}{2.184217in}}%
\pgfpathlineto{\pgfqpoint{2.364842in}{2.162260in}}%
\pgfpathlineto{\pgfqpoint{2.374278in}{2.151691in}}%
\pgfpathlineto{\pgfqpoint{2.383684in}{2.141605in}}%
\pgfpathlineto{\pgfqpoint{2.393063in}{2.131994in}}%
\pgfpathlineto{\pgfqpoint{2.402413in}{2.122849in}}%
\pgfpathlineto{\pgfqpoint{2.388143in}{2.143950in}}%
\pgfpathlineto{\pgfqpoint{2.373861in}{2.165327in}}%
\pgfpathlineto{\pgfqpoint{2.359568in}{2.186982in}}%
\pgfpathlineto{\pgfqpoint{2.345263in}{2.208917in}}%
\pgfpathlineto{\pgfqpoint{2.335842in}{2.218904in}}%
\pgfpathlineto{\pgfqpoint{2.326392in}{2.229369in}}%
\pgfpathlineto{\pgfqpoint{2.316913in}{2.240318in}}%
\pgfpathlineto{\pgfqpoint{2.307403in}{2.251762in}}%
\pgfpathclose%
\pgfusepath{fill}%
\end{pgfscope}%
\begin{pgfscope}%
\pgfpathrectangle{\pgfqpoint{1.150000in}{0.150000in}}{\pgfqpoint{5.700000in}{5.700000in}}%
\pgfusepath{clip}%
\pgfsetbuttcap%
\pgfsetroundjoin%
\definecolor{currentfill}{rgb}{0.140536,0.530132,0.555659}%
\pgfsetfillcolor{currentfill}%
\pgfsetfillopacity{0.800000}%
\pgfsetlinewidth{0.000000pt}%
\definecolor{currentstroke}{rgb}{0.000000,0.000000,0.000000}%
\pgfsetstrokecolor{currentstroke}%
\pgfsetdash{}{0pt}%
\pgfpathmoveto{\pgfqpoint{2.056156in}{2.710355in}}%
\pgfpathlineto{\pgfqpoint{2.070768in}{2.682268in}}%
\pgfpathlineto{\pgfqpoint{2.085362in}{2.654516in}}%
\pgfpathlineto{\pgfqpoint{2.099937in}{2.627097in}}%
\pgfpathlineto{\pgfqpoint{2.114494in}{2.600006in}}%
\pgfpathlineto{\pgfqpoint{2.124190in}{2.587078in}}%
\pgfpathlineto{\pgfqpoint{2.133853in}{2.574652in}}%
\pgfpathlineto{\pgfqpoint{2.143484in}{2.562719in}}%
\pgfpathlineto{\pgfqpoint{2.153084in}{2.551269in}}%
\pgfpathlineto{\pgfqpoint{2.138607in}{2.577505in}}%
\pgfpathlineto{\pgfqpoint{2.124112in}{2.604068in}}%
\pgfpathlineto{\pgfqpoint{2.109601in}{2.630960in}}%
\pgfpathlineto{\pgfqpoint{2.095071in}{2.658186in}}%
\pgfpathlineto{\pgfqpoint{2.085392in}{2.670477in}}%
\pgfpathlineto{\pgfqpoint{2.075680in}{2.683262in}}%
\pgfpathlineto{\pgfqpoint{2.065935in}{2.696552in}}%
\pgfpathlineto{\pgfqpoint{2.056156in}{2.710355in}}%
\pgfpathclose%
\pgfusepath{fill}%
\end{pgfscope}%
\begin{pgfscope}%
\pgfpathrectangle{\pgfqpoint{1.150000in}{0.150000in}}{\pgfqpoint{5.700000in}{5.700000in}}%
\pgfusepath{clip}%
\pgfsetbuttcap%
\pgfsetroundjoin%
\definecolor{currentfill}{rgb}{0.269308,0.218818,0.509577}%
\pgfsetfillcolor{currentfill}%
\pgfsetfillopacity{0.800000}%
\pgfsetlinewidth{0.000000pt}%
\definecolor{currentstroke}{rgb}{0.000000,0.000000,0.000000}%
\pgfsetstrokecolor{currentstroke}%
\pgfsetdash{}{0pt}%
\pgfpathmoveto{\pgfqpoint{4.140030in}{1.696446in}}%
\pgfpathlineto{\pgfqpoint{4.154200in}{1.702702in}}%
\pgfpathlineto{\pgfqpoint{4.168381in}{1.709138in}}%
\pgfpathlineto{\pgfqpoint{4.182576in}{1.715756in}}%
\pgfpathlineto{\pgfqpoint{4.196783in}{1.722554in}}%
\pgfpathlineto{\pgfqpoint{4.204927in}{1.737979in}}%
\pgfpathlineto{\pgfqpoint{4.213067in}{1.753388in}}%
\pgfpathlineto{\pgfqpoint{4.221203in}{1.768776in}}%
\pgfpathlineto{\pgfqpoint{4.229335in}{1.784140in}}%
\pgfpathlineto{\pgfqpoint{4.215126in}{1.776891in}}%
\pgfpathlineto{\pgfqpoint{4.200930in}{1.769823in}}%
\pgfpathlineto{\pgfqpoint{4.186747in}{1.762937in}}%
\pgfpathlineto{\pgfqpoint{4.172577in}{1.756233in}}%
\pgfpathlineto{\pgfqpoint{4.164447in}{1.741307in}}%
\pgfpathlineto{\pgfqpoint{4.156312in}{1.726365in}}%
\pgfpathlineto{\pgfqpoint{4.148173in}{1.711410in}}%
\pgfpathlineto{\pgfqpoint{4.140030in}{1.696446in}}%
\pgfpathclose%
\pgfusepath{fill}%
\end{pgfscope}%
\begin{pgfscope}%
\pgfpathrectangle{\pgfqpoint{1.150000in}{0.150000in}}{\pgfqpoint{5.700000in}{5.700000in}}%
\pgfusepath{clip}%
\pgfsetbuttcap%
\pgfsetroundjoin%
\definecolor{currentfill}{rgb}{0.421908,0.805774,0.351910}%
\pgfsetfillcolor{currentfill}%
\pgfsetfillopacity{0.800000}%
\pgfsetlinewidth{0.000000pt}%
\definecolor{currentstroke}{rgb}{0.000000,0.000000,0.000000}%
\pgfsetstrokecolor{currentstroke}%
\pgfsetdash{}{0pt}%
\pgfpathmoveto{\pgfqpoint{5.715334in}{3.503975in}}%
\pgfpathlineto{\pgfqpoint{5.730533in}{3.522208in}}%
\pgfpathlineto{\pgfqpoint{5.745757in}{3.540631in}}%
\pgfpathlineto{\pgfqpoint{5.761006in}{3.559244in}}%
\pgfpathlineto{\pgfqpoint{5.776281in}{3.578048in}}%
\pgfpathlineto{\pgfqpoint{5.783628in}{3.580012in}}%
\pgfpathlineto{\pgfqpoint{5.790964in}{3.581828in}}%
\pgfpathlineto{\pgfqpoint{5.798288in}{3.583501in}}%
\pgfpathlineto{\pgfqpoint{5.805601in}{3.585036in}}%
\pgfpathlineto{\pgfqpoint{5.790349in}{3.566617in}}%
\pgfpathlineto{\pgfqpoint{5.775122in}{3.548389in}}%
\pgfpathlineto{\pgfqpoint{5.759919in}{3.530349in}}%
\pgfpathlineto{\pgfqpoint{5.744741in}{3.512499in}}%
\pgfpathlineto{\pgfqpoint{5.737406in}{3.510568in}}%
\pgfpathlineto{\pgfqpoint{5.730060in}{3.508506in}}%
\pgfpathlineto{\pgfqpoint{5.722702in}{3.506310in}}%
\pgfpathlineto{\pgfqpoint{5.715334in}{3.503975in}}%
\pgfpathclose%
\pgfusepath{fill}%
\end{pgfscope}%
\begin{pgfscope}%
\pgfpathrectangle{\pgfqpoint{1.150000in}{0.150000in}}{\pgfqpoint{5.700000in}{5.700000in}}%
\pgfusepath{clip}%
\pgfsetbuttcap%
\pgfsetroundjoin%
\definecolor{currentfill}{rgb}{0.280894,0.078907,0.402329}%
\pgfsetfillcolor{currentfill}%
\pgfsetfillopacity{0.800000}%
\pgfsetlinewidth{0.000000pt}%
\definecolor{currentstroke}{rgb}{0.000000,0.000000,0.000000}%
\pgfsetstrokecolor{currentstroke}%
\pgfsetdash{}{0pt}%
\pgfpathmoveto{\pgfqpoint{2.987745in}{1.442358in}}%
\pgfpathlineto{\pgfqpoint{3.001796in}{1.431067in}}%
\pgfpathlineto{\pgfqpoint{3.015846in}{1.419983in}}%
\pgfpathlineto{\pgfqpoint{3.029894in}{1.409104in}}%
\pgfpathlineto{\pgfqpoint{3.043941in}{1.398431in}}%
\pgfpathlineto{\pgfqpoint{3.052690in}{1.397907in}}%
\pgfpathlineto{\pgfqpoint{3.061421in}{1.397750in}}%
\pgfpathlineto{\pgfqpoint{3.070136in}{1.397951in}}%
\pgfpathlineto{\pgfqpoint{3.078834in}{1.398503in}}%
\pgfpathlineto{\pgfqpoint{3.064829in}{1.408376in}}%
\pgfpathlineto{\pgfqpoint{3.050823in}{1.418454in}}%
\pgfpathlineto{\pgfqpoint{3.036817in}{1.428736in}}%
\pgfpathlineto{\pgfqpoint{3.022809in}{1.439225in}}%
\pgfpathlineto{\pgfqpoint{3.014069in}{1.439461in}}%
\pgfpathlineto{\pgfqpoint{3.005312in}{1.440057in}}%
\pgfpathlineto{\pgfqpoint{2.996537in}{1.441020in}}%
\pgfpathlineto{\pgfqpoint{2.987745in}{1.442358in}}%
\pgfpathclose%
\pgfusepath{fill}%
\end{pgfscope}%
\begin{pgfscope}%
\pgfpathrectangle{\pgfqpoint{1.150000in}{0.150000in}}{\pgfqpoint{5.700000in}{5.700000in}}%
\pgfusepath{clip}%
\pgfsetbuttcap%
\pgfsetroundjoin%
\definecolor{currentfill}{rgb}{0.120638,0.625828,0.533488}%
\pgfsetfillcolor{currentfill}%
\pgfsetfillopacity{0.800000}%
\pgfsetlinewidth{0.000000pt}%
\definecolor{currentstroke}{rgb}{0.000000,0.000000,0.000000}%
\pgfsetstrokecolor{currentstroke}%
\pgfsetdash{}{0pt}%
\pgfpathmoveto{\pgfqpoint{5.082129in}{2.874878in}}%
\pgfpathlineto{\pgfqpoint{5.096876in}{2.890361in}}%
\pgfpathlineto{\pgfqpoint{5.111644in}{2.906032in}}%
\pgfpathlineto{\pgfqpoint{5.126432in}{2.921892in}}%
\pgfpathlineto{\pgfqpoint{5.141243in}{2.937941in}}%
\pgfpathlineto{\pgfqpoint{5.149047in}{2.947564in}}%
\pgfpathlineto{\pgfqpoint{5.156843in}{2.957000in}}%
\pgfpathlineto{\pgfqpoint{5.164629in}{2.966250in}}%
\pgfpathlineto{\pgfqpoint{5.172406in}{2.975315in}}%
\pgfpathlineto{\pgfqpoint{5.157599in}{2.959320in}}%
\pgfpathlineto{\pgfqpoint{5.142814in}{2.943513in}}%
\pgfpathlineto{\pgfqpoint{5.128049in}{2.927895in}}%
\pgfpathlineto{\pgfqpoint{5.113306in}{2.912465in}}%
\pgfpathlineto{\pgfqpoint{5.105525in}{2.903333in}}%
\pgfpathlineto{\pgfqpoint{5.097735in}{2.894025in}}%
\pgfpathlineto{\pgfqpoint{5.089936in}{2.884540in}}%
\pgfpathlineto{\pgfqpoint{5.082129in}{2.874878in}}%
\pgfpathclose%
\pgfusepath{fill}%
\end{pgfscope}%
\begin{pgfscope}%
\pgfpathrectangle{\pgfqpoint{1.150000in}{0.150000in}}{\pgfqpoint{5.700000in}{5.700000in}}%
\pgfusepath{clip}%
\pgfsetbuttcap%
\pgfsetroundjoin%
\definecolor{currentfill}{rgb}{0.185556,0.418570,0.556753}%
\pgfsetfillcolor{currentfill}%
\pgfsetfillopacity{0.800000}%
\pgfsetlinewidth{0.000000pt}%
\definecolor{currentstroke}{rgb}{0.000000,0.000000,0.000000}%
\pgfsetstrokecolor{currentstroke}%
\pgfsetdash{}{0pt}%
\pgfpathmoveto{\pgfqpoint{2.249757in}{2.345814in}}%
\pgfpathlineto{\pgfqpoint{2.264189in}{2.321866in}}%
\pgfpathlineto{\pgfqpoint{2.278607in}{2.298210in}}%
\pgfpathlineto{\pgfqpoint{2.293012in}{2.274842in}}%
\pgfpathlineto{\pgfqpoint{2.307403in}{2.251762in}}%
\pgfpathlineto{\pgfqpoint{2.316913in}{2.240318in}}%
\pgfpathlineto{\pgfqpoint{2.326392in}{2.229369in}}%
\pgfpathlineto{\pgfqpoint{2.335842in}{2.218904in}}%
\pgfpathlineto{\pgfqpoint{2.345263in}{2.208917in}}%
\pgfpathlineto{\pgfqpoint{2.330945in}{2.231134in}}%
\pgfpathlineto{\pgfqpoint{2.316615in}{2.253636in}}%
\pgfpathlineto{\pgfqpoint{2.302272in}{2.276425in}}%
\pgfpathlineto{\pgfqpoint{2.287915in}{2.299504in}}%
\pgfpathlineto{\pgfqpoint{2.278422in}{2.310342in}}%
\pgfpathlineto{\pgfqpoint{2.268898in}{2.321667in}}%
\pgfpathlineto{\pgfqpoint{2.259343in}{2.333488in}}%
\pgfpathlineto{\pgfqpoint{2.249757in}{2.345814in}}%
\pgfpathclose%
\pgfusepath{fill}%
\end{pgfscope}%
\begin{pgfscope}%
\pgfpathrectangle{\pgfqpoint{1.150000in}{0.150000in}}{\pgfqpoint{5.700000in}{5.700000in}}%
\pgfusepath{clip}%
\pgfsetbuttcap%
\pgfsetroundjoin%
\definecolor{currentfill}{rgb}{0.280868,0.160771,0.472899}%
\pgfsetfillcolor{currentfill}%
\pgfsetfillopacity{0.800000}%
\pgfsetlinewidth{0.000000pt}%
\definecolor{currentstroke}{rgb}{0.000000,0.000000,0.000000}%
\pgfsetstrokecolor{currentstroke}%
\pgfsetdash{}{0pt}%
\pgfpathmoveto{\pgfqpoint{4.018173in}{1.557717in}}%
\pgfpathlineto{\pgfqpoint{4.032297in}{1.562258in}}%
\pgfpathlineto{\pgfqpoint{4.046432in}{1.566979in}}%
\pgfpathlineto{\pgfqpoint{4.060578in}{1.571881in}}%
\pgfpathlineto{\pgfqpoint{4.074736in}{1.576963in}}%
\pgfpathlineto{\pgfqpoint{4.082913in}{1.591834in}}%
\pgfpathlineto{\pgfqpoint{4.091085in}{1.606733in}}%
\pgfpathlineto{\pgfqpoint{4.099253in}{1.621655in}}%
\pgfpathlineto{\pgfqpoint{4.107417in}{1.636595in}}%
\pgfpathlineto{\pgfqpoint{4.093260in}{1.631001in}}%
\pgfpathlineto{\pgfqpoint{4.079115in}{1.625588in}}%
\pgfpathlineto{\pgfqpoint{4.064981in}{1.620355in}}%
\pgfpathlineto{\pgfqpoint{4.050860in}{1.615304in}}%
\pgfpathlineto{\pgfqpoint{4.042695in}{1.600863in}}%
\pgfpathlineto{\pgfqpoint{4.034525in}{1.586449in}}%
\pgfpathlineto{\pgfqpoint{4.026351in}{1.572065in}}%
\pgfpathlineto{\pgfqpoint{4.018173in}{1.557717in}}%
\pgfpathclose%
\pgfusepath{fill}%
\end{pgfscope}%
\begin{pgfscope}%
\pgfpathrectangle{\pgfqpoint{1.150000in}{0.150000in}}{\pgfqpoint{5.700000in}{5.700000in}}%
\pgfusepath{clip}%
\pgfsetbuttcap%
\pgfsetroundjoin%
\definecolor{currentfill}{rgb}{0.477504,0.821444,0.318195}%
\pgfsetfillcolor{currentfill}%
\pgfsetfillopacity{0.800000}%
\pgfsetlinewidth{0.000000pt}%
\definecolor{currentstroke}{rgb}{0.000000,0.000000,0.000000}%
\pgfsetstrokecolor{currentstroke}%
\pgfsetdash{}{0pt}%
\pgfpathmoveto{\pgfqpoint{5.805601in}{3.585036in}}%
\pgfpathlineto{\pgfqpoint{5.820878in}{3.603644in}}%
\pgfpathlineto{\pgfqpoint{5.836180in}{3.622443in}}%
\pgfpathlineto{\pgfqpoint{5.851508in}{3.641433in}}%
\pgfpathlineto{\pgfqpoint{5.858791in}{3.642526in}}%
\pgfpathlineto{\pgfqpoint{5.866062in}{3.643483in}}%
\pgfpathlineto{\pgfqpoint{5.873321in}{3.644309in}}%
\pgfpathlineto{\pgfqpoint{5.880569in}{3.645007in}}%
\pgfpathlineto{\pgfqpoint{5.865266in}{3.626440in}}%
\pgfpathlineto{\pgfqpoint{5.849989in}{3.608063in}}%
\pgfpathlineto{\pgfqpoint{5.834736in}{3.589875in}}%
\pgfpathlineto{\pgfqpoint{5.827469in}{3.588851in}}%
\pgfpathlineto{\pgfqpoint{5.820191in}{3.587706in}}%
\pgfpathlineto{\pgfqpoint{5.812902in}{3.586436in}}%
\pgfpathlineto{\pgfqpoint{5.805601in}{3.585036in}}%
\pgfpathclose%
\pgfusepath{fill}%
\end{pgfscope}%
\begin{pgfscope}%
\pgfpathrectangle{\pgfqpoint{1.150000in}{0.150000in}}{\pgfqpoint{5.700000in}{5.700000in}}%
\pgfusepath{clip}%
\pgfsetbuttcap%
\pgfsetroundjoin%
\definecolor{currentfill}{rgb}{0.268510,0.009605,0.335427}%
\pgfsetfillcolor{currentfill}%
\pgfsetfillopacity{0.800000}%
\pgfsetlinewidth{0.000000pt}%
\definecolor{currentstroke}{rgb}{0.000000,0.000000,0.000000}%
\pgfsetstrokecolor{currentstroke}%
\pgfsetdash{}{0pt}%
\pgfpathmoveto{\pgfqpoint{3.393058in}{1.263409in}}%
\pgfpathlineto{\pgfqpoint{3.407063in}{1.258398in}}%
\pgfpathlineto{\pgfqpoint{3.421072in}{1.253575in}}%
\pgfpathlineto{\pgfqpoint{3.435086in}{1.248939in}}%
\pgfpathlineto{\pgfqpoint{3.449103in}{1.244490in}}%
\pgfpathlineto{\pgfqpoint{3.457542in}{1.251112in}}%
\pgfpathlineto{\pgfqpoint{3.465972in}{1.257986in}}%
\pgfpathlineto{\pgfqpoint{3.474391in}{1.265105in}}%
\pgfpathlineto{\pgfqpoint{3.482800in}{1.272463in}}%
\pgfpathlineto{\pgfqpoint{3.468807in}{1.276186in}}%
\pgfpathlineto{\pgfqpoint{3.454818in}{1.280096in}}%
\pgfpathlineto{\pgfqpoint{3.440834in}{1.284193in}}%
\pgfpathlineto{\pgfqpoint{3.426853in}{1.288478in}}%
\pgfpathlineto{\pgfqpoint{3.418420in}{1.281834in}}%
\pgfpathlineto{\pgfqpoint{3.409977in}{1.275437in}}%
\pgfpathlineto{\pgfqpoint{3.401523in}{1.269292in}}%
\pgfpathlineto{\pgfqpoint{3.393058in}{1.263409in}}%
\pgfpathclose%
\pgfusepath{fill}%
\end{pgfscope}%
\begin{pgfscope}%
\pgfpathrectangle{\pgfqpoint{1.150000in}{0.150000in}}{\pgfqpoint{5.700000in}{5.700000in}}%
\pgfusepath{clip}%
\pgfsetbuttcap%
\pgfsetroundjoin%
\definecolor{currentfill}{rgb}{0.277018,0.050344,0.375715}%
\pgfsetfillcolor{currentfill}%
\pgfsetfillopacity{0.800000}%
\pgfsetlinewidth{0.000000pt}%
\definecolor{currentstroke}{rgb}{0.000000,0.000000,0.000000}%
\pgfsetstrokecolor{currentstroke}%
\pgfsetdash{}{0pt}%
\pgfpathmoveto{\pgfqpoint{3.717671in}{1.325059in}}%
\pgfpathlineto{\pgfqpoint{3.731710in}{1.325082in}}%
\pgfpathlineto{\pgfqpoint{3.745756in}{1.325286in}}%
\pgfpathlineto{\pgfqpoint{3.759811in}{1.325672in}}%
\pgfpathlineto{\pgfqpoint{3.773873in}{1.326238in}}%
\pgfpathlineto{\pgfqpoint{3.782147in}{1.338001in}}%
\pgfpathlineto{\pgfqpoint{3.790414in}{1.349903in}}%
\pgfpathlineto{\pgfqpoint{3.798676in}{1.361939in}}%
\pgfpathlineto{\pgfqpoint{3.806931in}{1.374102in}}%
\pgfpathlineto{\pgfqpoint{3.792879in}{1.372903in}}%
\pgfpathlineto{\pgfqpoint{3.778836in}{1.371885in}}%
\pgfpathlineto{\pgfqpoint{3.764800in}{1.371048in}}%
\pgfpathlineto{\pgfqpoint{3.750773in}{1.370393in}}%
\pgfpathlineto{\pgfqpoint{3.742507in}{1.358850in}}%
\pgfpathlineto{\pgfqpoint{3.734235in}{1.347443in}}%
\pgfpathlineto{\pgfqpoint{3.725956in}{1.336177in}}%
\pgfpathlineto{\pgfqpoint{3.717671in}{1.325059in}}%
\pgfpathclose%
\pgfusepath{fill}%
\end{pgfscope}%
\begin{pgfscope}%
\pgfpathrectangle{\pgfqpoint{1.150000in}{0.150000in}}{\pgfqpoint{5.700000in}{5.700000in}}%
\pgfusepath{clip}%
\pgfsetbuttcap%
\pgfsetroundjoin%
\definecolor{currentfill}{rgb}{0.271305,0.019942,0.347269}%
\pgfsetfillcolor{currentfill}%
\pgfsetfillopacity{0.800000}%
\pgfsetlinewidth{0.000000pt}%
\definecolor{currentstroke}{rgb}{0.000000,0.000000,0.000000}%
\pgfsetstrokecolor{currentstroke}%
\pgfsetdash{}{0pt}%
\pgfpathmoveto{\pgfqpoint{3.246914in}{1.295633in}}%
\pgfpathlineto{\pgfqpoint{3.260928in}{1.288337in}}%
\pgfpathlineto{\pgfqpoint{3.274943in}{1.281234in}}%
\pgfpathlineto{\pgfqpoint{3.288961in}{1.274324in}}%
\pgfpathlineto{\pgfqpoint{3.302981in}{1.267605in}}%
\pgfpathlineto{\pgfqpoint{3.311521in}{1.271598in}}%
\pgfpathlineto{\pgfqpoint{3.320050in}{1.275890in}}%
\pgfpathlineto{\pgfqpoint{3.328566in}{1.280475in}}%
\pgfpathlineto{\pgfqpoint{3.337070in}{1.285343in}}%
\pgfpathlineto{\pgfqpoint{3.323080in}{1.291302in}}%
\pgfpathlineto{\pgfqpoint{3.309094in}{1.297453in}}%
\pgfpathlineto{\pgfqpoint{3.295110in}{1.303795in}}%
\pgfpathlineto{\pgfqpoint{3.281128in}{1.310331in}}%
\pgfpathlineto{\pgfqpoint{3.272594in}{1.306210in}}%
\pgfpathlineto{\pgfqpoint{3.264047in}{1.302381in}}%
\pgfpathlineto{\pgfqpoint{3.255487in}{1.298853in}}%
\pgfpathlineto{\pgfqpoint{3.246914in}{1.295633in}}%
\pgfpathclose%
\pgfusepath{fill}%
\end{pgfscope}%
\begin{pgfscope}%
\pgfpathrectangle{\pgfqpoint{1.150000in}{0.150000in}}{\pgfqpoint{5.700000in}{5.700000in}}%
\pgfusepath{clip}%
\pgfsetbuttcap%
\pgfsetroundjoin%
\definecolor{currentfill}{rgb}{0.123463,0.581687,0.547445}%
\pgfsetfillcolor{currentfill}%
\pgfsetfillopacity{0.800000}%
\pgfsetlinewidth{0.000000pt}%
\definecolor{currentstroke}{rgb}{0.000000,0.000000,0.000000}%
\pgfsetstrokecolor{currentstroke}%
\pgfsetdash{}{0pt}%
\pgfpathmoveto{\pgfqpoint{4.960570in}{2.731031in}}%
\pgfpathlineto{\pgfqpoint{4.975238in}{2.745763in}}%
\pgfpathlineto{\pgfqpoint{4.989925in}{2.760682in}}%
\pgfpathlineto{\pgfqpoint{5.004633in}{2.775788in}}%
\pgfpathlineto{\pgfqpoint{5.019362in}{2.791083in}}%
\pgfpathlineto{\pgfqpoint{5.027237in}{2.802193in}}%
\pgfpathlineto{\pgfqpoint{5.035104in}{2.813121in}}%
\pgfpathlineto{\pgfqpoint{5.042963in}{2.823867in}}%
\pgfpathlineto{\pgfqpoint{5.050813in}{2.834431in}}%
\pgfpathlineto{\pgfqpoint{5.036085in}{2.819118in}}%
\pgfpathlineto{\pgfqpoint{5.021378in}{2.803993in}}%
\pgfpathlineto{\pgfqpoint{5.006691in}{2.789056in}}%
\pgfpathlineto{\pgfqpoint{4.992024in}{2.774306in}}%
\pgfpathlineto{\pgfqpoint{4.984173in}{2.763747in}}%
\pgfpathlineto{\pgfqpoint{4.976313in}{2.753015in}}%
\pgfpathlineto{\pgfqpoint{4.968446in}{2.742110in}}%
\pgfpathlineto{\pgfqpoint{4.960570in}{2.731031in}}%
\pgfpathclose%
\pgfusepath{fill}%
\end{pgfscope}%
\begin{pgfscope}%
\pgfpathrectangle{\pgfqpoint{1.150000in}{0.150000in}}{\pgfqpoint{5.700000in}{5.700000in}}%
\pgfusepath{clip}%
\pgfsetbuttcap%
\pgfsetroundjoin%
\definecolor{currentfill}{rgb}{0.296479,0.761561,0.424223}%
\pgfsetfillcolor{currentfill}%
\pgfsetfillopacity{0.800000}%
\pgfsetlinewidth{0.000000pt}%
\definecolor{currentstroke}{rgb}{0.000000,0.000000,0.000000}%
\pgfsetstrokecolor{currentstroke}%
\pgfsetdash{}{0pt}%
\pgfpathmoveto{\pgfqpoint{5.504869in}{3.317225in}}%
\pgfpathlineto{\pgfqpoint{5.519927in}{3.334868in}}%
\pgfpathlineto{\pgfqpoint{5.535008in}{3.352701in}}%
\pgfpathlineto{\pgfqpoint{5.550113in}{3.370725in}}%
\pgfpathlineto{\pgfqpoint{5.565242in}{3.388939in}}%
\pgfpathlineto{\pgfqpoint{5.572766in}{3.393501in}}%
\pgfpathlineto{\pgfqpoint{5.580278in}{3.397888in}}%
\pgfpathlineto{\pgfqpoint{5.587778in}{3.402105in}}%
\pgfpathlineto{\pgfqpoint{5.595268in}{3.406155in}}%
\pgfpathlineto{\pgfqpoint{5.580154in}{3.388215in}}%
\pgfpathlineto{\pgfqpoint{5.565063in}{3.370465in}}%
\pgfpathlineto{\pgfqpoint{5.549996in}{3.352905in}}%
\pgfpathlineto{\pgfqpoint{5.534953in}{3.335533in}}%
\pgfpathlineto{\pgfqpoint{5.527449in}{3.331198in}}%
\pgfpathlineto{\pgfqpoint{5.519933in}{3.326703in}}%
\pgfpathlineto{\pgfqpoint{5.512406in}{3.322047in}}%
\pgfpathlineto{\pgfqpoint{5.504869in}{3.317225in}}%
\pgfpathclose%
\pgfusepath{fill}%
\end{pgfscope}%
\begin{pgfscope}%
\pgfpathrectangle{\pgfqpoint{1.150000in}{0.150000in}}{\pgfqpoint{5.700000in}{5.700000in}}%
\pgfusepath{clip}%
\pgfsetbuttcap%
\pgfsetroundjoin%
\definecolor{currentfill}{rgb}{0.180653,0.701402,0.488189}%
\pgfsetfillcolor{currentfill}%
\pgfsetfillopacity{0.800000}%
\pgfsetlinewidth{0.000000pt}%
\definecolor{currentstroke}{rgb}{0.000000,0.000000,0.000000}%
\pgfsetstrokecolor{currentstroke}%
\pgfsetdash{}{0pt}%
\pgfpathmoveto{\pgfqpoint{5.293686in}{3.106348in}}%
\pgfpathlineto{\pgfqpoint{5.308591in}{3.123072in}}%
\pgfpathlineto{\pgfqpoint{5.323518in}{3.139986in}}%
\pgfpathlineto{\pgfqpoint{5.338468in}{3.157089in}}%
\pgfpathlineto{\pgfqpoint{5.353441in}{3.174382in}}%
\pgfpathlineto{\pgfqpoint{5.361118in}{3.181549in}}%
\pgfpathlineto{\pgfqpoint{5.368784in}{3.188530in}}%
\pgfpathlineto{\pgfqpoint{5.376440in}{3.195326in}}%
\pgfpathlineto{\pgfqpoint{5.384086in}{3.201939in}}%
\pgfpathlineto{\pgfqpoint{5.369121in}{3.184809in}}%
\pgfpathlineto{\pgfqpoint{5.354180in}{3.167868in}}%
\pgfpathlineto{\pgfqpoint{5.339261in}{3.151117in}}%
\pgfpathlineto{\pgfqpoint{5.324364in}{3.134554in}}%
\pgfpathlineto{\pgfqpoint{5.316709in}{3.127766in}}%
\pgfpathlineto{\pgfqpoint{5.309045in}{3.120804in}}%
\pgfpathlineto{\pgfqpoint{5.301370in}{3.113665in}}%
\pgfpathlineto{\pgfqpoint{5.293686in}{3.106348in}}%
\pgfpathclose%
\pgfusepath{fill}%
\end{pgfscope}%
\begin{pgfscope}%
\pgfpathrectangle{\pgfqpoint{1.150000in}{0.150000in}}{\pgfqpoint{5.700000in}{5.700000in}}%
\pgfusepath{clip}%
\pgfsetbuttcap%
\pgfsetroundjoin%
\definecolor{currentfill}{rgb}{0.272594,0.025563,0.353093}%
\pgfsetfillcolor{currentfill}%
\pgfsetfillopacity{0.800000}%
\pgfsetlinewidth{0.000000pt}%
\definecolor{currentstroke}{rgb}{0.000000,0.000000,0.000000}%
\pgfsetstrokecolor{currentstroke}%
\pgfsetdash{}{0pt}%
\pgfpathmoveto{\pgfqpoint{3.628323in}{1.286566in}}%
\pgfpathlineto{\pgfqpoint{3.642347in}{1.285197in}}%
\pgfpathlineto{\pgfqpoint{3.656378in}{1.284011in}}%
\pgfpathlineto{\pgfqpoint{3.670417in}{1.283007in}}%
\pgfpathlineto{\pgfqpoint{3.684462in}{1.282185in}}%
\pgfpathlineto{\pgfqpoint{3.692774in}{1.292651in}}%
\pgfpathlineto{\pgfqpoint{3.701080in}{1.303289in}}%
\pgfpathlineto{\pgfqpoint{3.709379in}{1.314094in}}%
\pgfpathlineto{\pgfqpoint{3.717671in}{1.325059in}}%
\pgfpathlineto{\pgfqpoint{3.703640in}{1.325217in}}%
\pgfpathlineto{\pgfqpoint{3.689616in}{1.325558in}}%
\pgfpathlineto{\pgfqpoint{3.675599in}{1.326081in}}%
\pgfpathlineto{\pgfqpoint{3.661590in}{1.326787in}}%
\pgfpathlineto{\pgfqpoint{3.653284in}{1.316474in}}%
\pgfpathlineto{\pgfqpoint{3.644971in}{1.306328in}}%
\pgfpathlineto{\pgfqpoint{3.636651in}{1.296356in}}%
\pgfpathlineto{\pgfqpoint{3.628323in}{1.286566in}}%
\pgfpathclose%
\pgfusepath{fill}%
\end{pgfscope}%
\begin{pgfscope}%
\pgfpathrectangle{\pgfqpoint{1.150000in}{0.150000in}}{\pgfqpoint{5.700000in}{5.700000in}}%
\pgfusepath{clip}%
\pgfsetbuttcap%
\pgfsetroundjoin%
\definecolor{currentfill}{rgb}{0.280894,0.078907,0.402329}%
\pgfsetfillcolor{currentfill}%
\pgfsetfillopacity{0.800000}%
\pgfsetlinewidth{0.000000pt}%
\definecolor{currentstroke}{rgb}{0.000000,0.000000,0.000000}%
\pgfsetstrokecolor{currentstroke}%
\pgfsetdash{}{0pt}%
\pgfpathmoveto{\pgfqpoint{3.806931in}{1.374102in}}%
\pgfpathlineto{\pgfqpoint{3.820992in}{1.375482in}}%
\pgfpathlineto{\pgfqpoint{3.835062in}{1.377043in}}%
\pgfpathlineto{\pgfqpoint{3.849140in}{1.378785in}}%
\pgfpathlineto{\pgfqpoint{3.863228in}{1.380706in}}%
\pgfpathlineto{\pgfqpoint{3.871470in}{1.393605in}}%
\pgfpathlineto{\pgfqpoint{3.879706in}{1.406611in}}%
\pgfpathlineto{\pgfqpoint{3.887936in}{1.419720in}}%
\pgfpathlineto{\pgfqpoint{3.896162in}{1.432924in}}%
\pgfpathlineto{\pgfqpoint{3.882082in}{1.430399in}}%
\pgfpathlineto{\pgfqpoint{3.868011in}{1.428055in}}%
\pgfpathlineto{\pgfqpoint{3.853949in}{1.425892in}}%
\pgfpathlineto{\pgfqpoint{3.839897in}{1.423910in}}%
\pgfpathlineto{\pgfqpoint{3.831664in}{1.411296in}}%
\pgfpathlineto{\pgfqpoint{3.823425in}{1.398787in}}%
\pgfpathlineto{\pgfqpoint{3.815181in}{1.386387in}}%
\pgfpathlineto{\pgfqpoint{3.806931in}{1.374102in}}%
\pgfpathclose%
\pgfusepath{fill}%
\end{pgfscope}%
\begin{pgfscope}%
\pgfpathrectangle{\pgfqpoint{1.150000in}{0.150000in}}{\pgfqpoint{5.700000in}{5.700000in}}%
\pgfusepath{clip}%
\pgfsetbuttcap%
\pgfsetroundjoin%
\definecolor{currentfill}{rgb}{0.279566,0.067836,0.391917}%
\pgfsetfillcolor{currentfill}%
\pgfsetfillopacity{0.800000}%
\pgfsetlinewidth{0.000000pt}%
\definecolor{currentstroke}{rgb}{0.000000,0.000000,0.000000}%
\pgfsetstrokecolor{currentstroke}%
\pgfsetdash{}{0pt}%
\pgfpathmoveto{\pgfqpoint{3.043941in}{1.398431in}}%
\pgfpathlineto{\pgfqpoint{3.057987in}{1.387960in}}%
\pgfpathlineto{\pgfqpoint{3.072033in}{1.377693in}}%
\pgfpathlineto{\pgfqpoint{3.086078in}{1.367628in}}%
\pgfpathlineto{\pgfqpoint{3.100122in}{1.357764in}}%
\pgfpathlineto{\pgfqpoint{3.108829in}{1.358053in}}%
\pgfpathlineto{\pgfqpoint{3.117519in}{1.358700in}}%
\pgfpathlineto{\pgfqpoint{3.126193in}{1.359696in}}%
\pgfpathlineto{\pgfqpoint{3.134852in}{1.361033in}}%
\pgfpathlineto{\pgfqpoint{3.120848in}{1.370099in}}%
\pgfpathlineto{\pgfqpoint{3.106843in}{1.379365in}}%
\pgfpathlineto{\pgfqpoint{3.092839in}{1.388833in}}%
\pgfpathlineto{\pgfqpoint{3.078834in}{1.398503in}}%
\pgfpathlineto{\pgfqpoint{3.070136in}{1.397951in}}%
\pgfpathlineto{\pgfqpoint{3.061421in}{1.397750in}}%
\pgfpathlineto{\pgfqpoint{3.052690in}{1.397907in}}%
\pgfpathlineto{\pgfqpoint{3.043941in}{1.398431in}}%
\pgfpathclose%
\pgfusepath{fill}%
\end{pgfscope}%
\begin{pgfscope}%
\pgfpathrectangle{\pgfqpoint{1.150000in}{0.150000in}}{\pgfqpoint{5.700000in}{5.700000in}}%
\pgfusepath{clip}%
\pgfsetbuttcap%
\pgfsetroundjoin%
\definecolor{currentfill}{rgb}{0.137770,0.537492,0.554906}%
\pgfsetfillcolor{currentfill}%
\pgfsetfillopacity{0.800000}%
\pgfsetlinewidth{0.000000pt}%
\definecolor{currentstroke}{rgb}{0.000000,0.000000,0.000000}%
\pgfsetstrokecolor{currentstroke}%
\pgfsetdash{}{0pt}%
\pgfpathmoveto{\pgfqpoint{4.838827in}{2.579662in}}%
\pgfpathlineto{\pgfqpoint{4.853414in}{2.593504in}}%
\pgfpathlineto{\pgfqpoint{4.868020in}{2.607532in}}%
\pgfpathlineto{\pgfqpoint{4.882645in}{2.621748in}}%
\pgfpathlineto{\pgfqpoint{4.897290in}{2.636150in}}%
\pgfpathlineto{\pgfqpoint{4.905227in}{2.648616in}}%
\pgfpathlineto{\pgfqpoint{4.913156in}{2.660909in}}%
\pgfpathlineto{\pgfqpoint{4.921077in}{2.673030in}}%
\pgfpathlineto{\pgfqpoint{4.928991in}{2.684977in}}%
\pgfpathlineto{\pgfqpoint{4.914344in}{2.670486in}}%
\pgfpathlineto{\pgfqpoint{4.899717in}{2.656182in}}%
\pgfpathlineto{\pgfqpoint{4.885109in}{2.642065in}}%
\pgfpathlineto{\pgfqpoint{4.870521in}{2.628135in}}%
\pgfpathlineto{\pgfqpoint{4.862608in}{2.616263in}}%
\pgfpathlineto{\pgfqpoint{4.854688in}{2.604226in}}%
\pgfpathlineto{\pgfqpoint{4.846761in}{2.592026in}}%
\pgfpathlineto{\pgfqpoint{4.838827in}{2.579662in}}%
\pgfpathclose%
\pgfusepath{fill}%
\end{pgfscope}%
\begin{pgfscope}%
\pgfpathrectangle{\pgfqpoint{1.150000in}{0.150000in}}{\pgfqpoint{5.700000in}{5.700000in}}%
\pgfusepath{clip}%
\pgfsetbuttcap%
\pgfsetroundjoin%
\definecolor{currentfill}{rgb}{0.229739,0.322361,0.545706}%
\pgfsetfillcolor{currentfill}%
\pgfsetfillopacity{0.800000}%
\pgfsetlinewidth{0.000000pt}%
\definecolor{currentstroke}{rgb}{0.000000,0.000000,0.000000}%
\pgfsetstrokecolor{currentstroke}%
\pgfsetdash{}{0pt}%
\pgfpathmoveto{\pgfqpoint{4.351232in}{1.939731in}}%
\pgfpathlineto{\pgfqpoint{4.365517in}{1.948696in}}%
\pgfpathlineto{\pgfqpoint{4.379816in}{1.957843in}}%
\pgfpathlineto{\pgfqpoint{4.394131in}{1.967172in}}%
\pgfpathlineto{\pgfqpoint{4.408461in}{1.976683in}}%
\pgfpathlineto{\pgfqpoint{4.416561in}{1.992390in}}%
\pgfpathlineto{\pgfqpoint{4.424657in}{2.008019in}}%
\pgfpathlineto{\pgfqpoint{4.432749in}{2.023567in}}%
\pgfpathlineto{\pgfqpoint{4.440836in}{2.039032in}}%
\pgfpathlineto{\pgfqpoint{4.426502in}{2.029163in}}%
\pgfpathlineto{\pgfqpoint{4.412184in}{2.019478in}}%
\pgfpathlineto{\pgfqpoint{4.397880in}{2.009975in}}%
\pgfpathlineto{\pgfqpoint{4.383592in}{2.000655in}}%
\pgfpathlineto{\pgfqpoint{4.375509in}{1.985534in}}%
\pgfpathlineto{\pgfqpoint{4.367421in}{1.970338in}}%
\pgfpathlineto{\pgfqpoint{4.359329in}{1.955069in}}%
\pgfpathlineto{\pgfqpoint{4.351232in}{1.939731in}}%
\pgfpathclose%
\pgfusepath{fill}%
\end{pgfscope}%
\begin{pgfscope}%
\pgfpathrectangle{\pgfqpoint{1.150000in}{0.150000in}}{\pgfqpoint{5.700000in}{5.700000in}}%
\pgfusepath{clip}%
\pgfsetbuttcap%
\pgfsetroundjoin%
\definecolor{currentfill}{rgb}{0.126453,0.570633,0.549841}%
\pgfsetfillcolor{currentfill}%
\pgfsetfillopacity{0.800000}%
\pgfsetlinewidth{0.000000pt}%
\definecolor{currentstroke}{rgb}{0.000000,0.000000,0.000000}%
\pgfsetstrokecolor{currentstroke}%
\pgfsetdash{}{0pt}%
\pgfpathmoveto{\pgfqpoint{1.997514in}{2.826128in}}%
\pgfpathlineto{\pgfqpoint{2.012204in}{2.796665in}}%
\pgfpathlineto{\pgfqpoint{2.026874in}{2.767550in}}%
\pgfpathlineto{\pgfqpoint{2.041525in}{2.738781in}}%
\pgfpathlineto{\pgfqpoint{2.056156in}{2.710355in}}%
\pgfpathlineto{\pgfqpoint{2.065935in}{2.696552in}}%
\pgfpathlineto{\pgfqpoint{2.075680in}{2.683262in}}%
\pgfpathlineto{\pgfqpoint{2.085392in}{2.670477in}}%
\pgfpathlineto{\pgfqpoint{2.095071in}{2.658186in}}%
\pgfpathlineto{\pgfqpoint{2.080523in}{2.685748in}}%
\pgfpathlineto{\pgfqpoint{2.065957in}{2.713650in}}%
\pgfpathlineto{\pgfqpoint{2.051371in}{2.741895in}}%
\pgfpathlineto{\pgfqpoint{2.036766in}{2.770487in}}%
\pgfpathlineto{\pgfqpoint{2.027005in}{2.783629in}}%
\pgfpathlineto{\pgfqpoint{2.017209in}{2.797277in}}%
\pgfpathlineto{\pgfqpoint{2.007379in}{2.811440in}}%
\pgfpathlineto{\pgfqpoint{1.997514in}{2.826128in}}%
\pgfpathclose%
\pgfusepath{fill}%
\end{pgfscope}%
\begin{pgfscope}%
\pgfpathrectangle{\pgfqpoint{1.150000in}{0.150000in}}{\pgfqpoint{5.700000in}{5.700000in}}%
\pgfusepath{clip}%
\pgfsetbuttcap%
\pgfsetroundjoin%
\definecolor{currentfill}{rgb}{0.203063,0.379716,0.553925}%
\pgfsetfillcolor{currentfill}%
\pgfsetfillopacity{0.800000}%
\pgfsetlinewidth{0.000000pt}%
\definecolor{currentstroke}{rgb}{0.000000,0.000000,0.000000}%
\pgfsetstrokecolor{currentstroke}%
\pgfsetdash{}{0pt}%
\pgfpathmoveto{\pgfqpoint{4.473140in}{2.099994in}}%
\pgfpathlineto{\pgfqpoint{4.487494in}{2.110369in}}%
\pgfpathlineto{\pgfqpoint{4.501864in}{2.120928in}}%
\pgfpathlineto{\pgfqpoint{4.516250in}{2.131671in}}%
\pgfpathlineto{\pgfqpoint{4.530653in}{2.142597in}}%
\pgfpathlineto{\pgfqpoint{4.538721in}{2.157903in}}%
\pgfpathlineto{\pgfqpoint{4.546785in}{2.173100in}}%
\pgfpathlineto{\pgfqpoint{4.554844in}{2.188186in}}%
\pgfpathlineto{\pgfqpoint{4.562897in}{2.203158in}}%
\pgfpathlineto{\pgfqpoint{4.548490in}{2.191939in}}%
\pgfpathlineto{\pgfqpoint{4.534099in}{2.180905in}}%
\pgfpathlineto{\pgfqpoint{4.519725in}{2.170055in}}%
\pgfpathlineto{\pgfqpoint{4.505367in}{2.159389in}}%
\pgfpathlineto{\pgfqpoint{4.497317in}{2.144696in}}%
\pgfpathlineto{\pgfqpoint{4.489263in}{2.129897in}}%
\pgfpathlineto{\pgfqpoint{4.481204in}{2.114996in}}%
\pgfpathlineto{\pgfqpoint{4.473140in}{2.099994in}}%
\pgfpathclose%
\pgfusepath{fill}%
\end{pgfscope}%
\begin{pgfscope}%
\pgfpathrectangle{\pgfqpoint{1.150000in}{0.150000in}}{\pgfqpoint{5.700000in}{5.700000in}}%
\pgfusepath{clip}%
\pgfsetbuttcap%
\pgfsetroundjoin%
\definecolor{currentfill}{rgb}{0.255645,0.260703,0.528312}%
\pgfsetfillcolor{currentfill}%
\pgfsetfillopacity{0.800000}%
\pgfsetlinewidth{0.000000pt}%
\definecolor{currentstroke}{rgb}{0.000000,0.000000,0.000000}%
\pgfsetstrokecolor{currentstroke}%
\pgfsetdash{}{0pt}%
\pgfpathmoveto{\pgfqpoint{4.229335in}{1.784140in}}%
\pgfpathlineto{\pgfqpoint{4.243557in}{1.791570in}}%
\pgfpathlineto{\pgfqpoint{4.257793in}{1.799181in}}%
\pgfpathlineto{\pgfqpoint{4.272043in}{1.806974in}}%
\pgfpathlineto{\pgfqpoint{4.286307in}{1.814949in}}%
\pgfpathlineto{\pgfqpoint{4.294437in}{1.830715in}}%
\pgfpathlineto{\pgfqpoint{4.302563in}{1.846441in}}%
\pgfpathlineto{\pgfqpoint{4.310685in}{1.862122in}}%
\pgfpathlineto{\pgfqpoint{4.318803in}{1.877755in}}%
\pgfpathlineto{\pgfqpoint{4.304536in}{1.869360in}}%
\pgfpathlineto{\pgfqpoint{4.290283in}{1.861148in}}%
\pgfpathlineto{\pgfqpoint{4.276045in}{1.853117in}}%
\pgfpathlineto{\pgfqpoint{4.261820in}{1.845268in}}%
\pgfpathlineto{\pgfqpoint{4.253705in}{1.830042in}}%
\pgfpathlineto{\pgfqpoint{4.245586in}{1.814776in}}%
\pgfpathlineto{\pgfqpoint{4.237462in}{1.799474in}}%
\pgfpathlineto{\pgfqpoint{4.229335in}{1.784140in}}%
\pgfpathclose%
\pgfusepath{fill}%
\end{pgfscope}%
\begin{pgfscope}%
\pgfpathrectangle{\pgfqpoint{1.150000in}{0.150000in}}{\pgfqpoint{5.700000in}{5.700000in}}%
\pgfusepath{clip}%
\pgfsetbuttcap%
\pgfsetroundjoin%
\definecolor{currentfill}{rgb}{0.179019,0.433756,0.557430}%
\pgfsetfillcolor{currentfill}%
\pgfsetfillopacity{0.800000}%
\pgfsetlinewidth{0.000000pt}%
\definecolor{currentstroke}{rgb}{0.000000,0.000000,0.000000}%
\pgfsetstrokecolor{currentstroke}%
\pgfsetdash{}{0pt}%
\pgfpathmoveto{\pgfqpoint{4.595060in}{2.261858in}}%
\pgfpathlineto{\pgfqpoint{4.609488in}{2.273519in}}%
\pgfpathlineto{\pgfqpoint{4.623934in}{2.285364in}}%
\pgfpathlineto{\pgfqpoint{4.638397in}{2.297394in}}%
\pgfpathlineto{\pgfqpoint{4.652878in}{2.309609in}}%
\pgfpathlineto{\pgfqpoint{4.660910in}{2.324214in}}%
\pgfpathlineto{\pgfqpoint{4.668936in}{2.338685in}}%
\pgfpathlineto{\pgfqpoint{4.676956in}{2.353018in}}%
\pgfpathlineto{\pgfqpoint{4.684971in}{2.367211in}}%
\pgfpathlineto{\pgfqpoint{4.670486in}{2.354771in}}%
\pgfpathlineto{\pgfqpoint{4.656018in}{2.342515in}}%
\pgfpathlineto{\pgfqpoint{4.641568in}{2.330445in}}%
\pgfpathlineto{\pgfqpoint{4.627136in}{2.318559in}}%
\pgfpathlineto{\pgfqpoint{4.619125in}{2.304578in}}%
\pgfpathlineto{\pgfqpoint{4.611109in}{2.290466in}}%
\pgfpathlineto{\pgfqpoint{4.603087in}{2.276225in}}%
\pgfpathlineto{\pgfqpoint{4.595060in}{2.261858in}}%
\pgfpathclose%
\pgfusepath{fill}%
\end{pgfscope}%
\begin{pgfscope}%
\pgfpathrectangle{\pgfqpoint{1.150000in}{0.150000in}}{\pgfqpoint{5.700000in}{5.700000in}}%
\pgfusepath{clip}%
\pgfsetbuttcap%
\pgfsetroundjoin%
\definecolor{currentfill}{rgb}{0.157729,0.485932,0.558013}%
\pgfsetfillcolor{currentfill}%
\pgfsetfillopacity{0.800000}%
\pgfsetlinewidth{0.000000pt}%
\definecolor{currentstroke}{rgb}{0.000000,0.000000,0.000000}%
\pgfsetstrokecolor{currentstroke}%
\pgfsetdash{}{0pt}%
\pgfpathmoveto{\pgfqpoint{4.716970in}{2.422563in}}%
\pgfpathlineto{\pgfqpoint{4.731477in}{2.435380in}}%
\pgfpathlineto{\pgfqpoint{4.746002in}{2.448383in}}%
\pgfpathlineto{\pgfqpoint{4.760545in}{2.461572in}}%
\pgfpathlineto{\pgfqpoint{4.775107in}{2.474946in}}%
\pgfpathlineto{\pgfqpoint{4.783095in}{2.488591in}}%
\pgfpathlineto{\pgfqpoint{4.791077in}{2.502080in}}%
\pgfpathlineto{\pgfqpoint{4.799052in}{2.515410in}}%
\pgfpathlineto{\pgfqpoint{4.807021in}{2.528582in}}%
\pgfpathlineto{\pgfqpoint{4.792455in}{2.515050in}}%
\pgfpathlineto{\pgfqpoint{4.777908in}{2.501703in}}%
\pgfpathlineto{\pgfqpoint{4.763380in}{2.488543in}}%
\pgfpathlineto{\pgfqpoint{4.748870in}{2.475568in}}%
\pgfpathlineto{\pgfqpoint{4.740905in}{2.462541in}}%
\pgfpathlineto{\pgfqpoint{4.732933in}{2.449364in}}%
\pgfpathlineto{\pgfqpoint{4.724954in}{2.436037in}}%
\pgfpathlineto{\pgfqpoint{4.716970in}{2.422563in}}%
\pgfpathclose%
\pgfusepath{fill}%
\end{pgfscope}%
\begin{pgfscope}%
\pgfpathrectangle{\pgfqpoint{1.150000in}{0.150000in}}{\pgfqpoint{5.700000in}{5.700000in}}%
\pgfusepath{clip}%
\pgfsetbuttcap%
\pgfsetroundjoin%
\definecolor{currentfill}{rgb}{0.269944,0.014625,0.341379}%
\pgfsetfillcolor{currentfill}%
\pgfsetfillopacity{0.800000}%
\pgfsetlinewidth{0.000000pt}%
\definecolor{currentstroke}{rgb}{0.000000,0.000000,0.000000}%
\pgfsetstrokecolor{currentstroke}%
\pgfsetdash{}{0pt}%
\pgfpathmoveto{\pgfqpoint{3.538823in}{1.259429in}}%
\pgfpathlineto{\pgfqpoint{3.552842in}{1.256632in}}%
\pgfpathlineto{\pgfqpoint{3.566866in}{1.254020in}}%
\pgfpathlineto{\pgfqpoint{3.580896in}{1.251591in}}%
\pgfpathlineto{\pgfqpoint{3.594932in}{1.249345in}}%
\pgfpathlineto{\pgfqpoint{3.603292in}{1.258345in}}%
\pgfpathlineto{\pgfqpoint{3.611643in}{1.267554in}}%
\pgfpathlineto{\pgfqpoint{3.619987in}{1.276963in}}%
\pgfpathlineto{\pgfqpoint{3.628323in}{1.286566in}}%
\pgfpathlineto{\pgfqpoint{3.614304in}{1.288117in}}%
\pgfpathlineto{\pgfqpoint{3.600293in}{1.289852in}}%
\pgfpathlineto{\pgfqpoint{3.586287in}{1.291770in}}%
\pgfpathlineto{\pgfqpoint{3.572287in}{1.293873in}}%
\pgfpathlineto{\pgfqpoint{3.563934in}{1.284952in}}%
\pgfpathlineto{\pgfqpoint{3.555572in}{1.276233in}}%
\pgfpathlineto{\pgfqpoint{3.547202in}{1.267723in}}%
\pgfpathlineto{\pgfqpoint{3.538823in}{1.259429in}}%
\pgfpathclose%
\pgfusepath{fill}%
\end{pgfscope}%
\begin{pgfscope}%
\pgfpathrectangle{\pgfqpoint{1.150000in}{0.150000in}}{\pgfqpoint{5.700000in}{5.700000in}}%
\pgfusepath{clip}%
\pgfsetbuttcap%
\pgfsetroundjoin%
\definecolor{currentfill}{rgb}{0.171176,0.452530,0.557965}%
\pgfsetfillcolor{currentfill}%
\pgfsetfillopacity{0.800000}%
\pgfsetlinewidth{0.000000pt}%
\definecolor{currentstroke}{rgb}{0.000000,0.000000,0.000000}%
\pgfsetstrokecolor{currentstroke}%
\pgfsetdash{}{0pt}%
\pgfpathmoveto{\pgfqpoint{2.191883in}{2.444577in}}%
\pgfpathlineto{\pgfqpoint{2.206374in}{2.419435in}}%
\pgfpathlineto{\pgfqpoint{2.220850in}{2.394596in}}%
\pgfpathlineto{\pgfqpoint{2.235311in}{2.370057in}}%
\pgfpathlineto{\pgfqpoint{2.249757in}{2.345814in}}%
\pgfpathlineto{\pgfqpoint{2.259343in}{2.333488in}}%
\pgfpathlineto{\pgfqpoint{2.268898in}{2.321667in}}%
\pgfpathlineto{\pgfqpoint{2.278422in}{2.310342in}}%
\pgfpathlineto{\pgfqpoint{2.287915in}{2.299504in}}%
\pgfpathlineto{\pgfqpoint{2.273545in}{2.322875in}}%
\pgfpathlineto{\pgfqpoint{2.259162in}{2.346541in}}%
\pgfpathlineto{\pgfqpoint{2.244764in}{2.370505in}}%
\pgfpathlineto{\pgfqpoint{2.230352in}{2.394768in}}%
\pgfpathlineto{\pgfqpoint{2.220783in}{2.406464in}}%
\pgfpathlineto{\pgfqpoint{2.211182in}{2.418658in}}%
\pgfpathlineto{\pgfqpoint{2.201549in}{2.431359in}}%
\pgfpathlineto{\pgfqpoint{2.191883in}{2.444577in}}%
\pgfpathclose%
\pgfusepath{fill}%
\end{pgfscope}%
\begin{pgfscope}%
\pgfpathrectangle{\pgfqpoint{1.150000in}{0.150000in}}{\pgfqpoint{5.700000in}{5.700000in}}%
\pgfusepath{clip}%
\pgfsetbuttcap%
\pgfsetroundjoin%
\definecolor{currentfill}{rgb}{0.283091,0.110553,0.431554}%
\pgfsetfillcolor{currentfill}%
\pgfsetfillopacity{0.800000}%
\pgfsetlinewidth{0.000000pt}%
\definecolor{currentstroke}{rgb}{0.000000,0.000000,0.000000}%
\pgfsetstrokecolor{currentstroke}%
\pgfsetdash{}{0pt}%
\pgfpathmoveto{\pgfqpoint{3.896162in}{1.432924in}}%
\pgfpathlineto{\pgfqpoint{3.910252in}{1.435629in}}%
\pgfpathlineto{\pgfqpoint{3.924352in}{1.438514in}}%
\pgfpathlineto{\pgfqpoint{3.938462in}{1.441579in}}%
\pgfpathlineto{\pgfqpoint{3.952583in}{1.444824in}}%
\pgfpathlineto{\pgfqpoint{3.960798in}{1.458703in}}%
\pgfpathlineto{\pgfqpoint{3.969008in}{1.472660in}}%
\pgfpathlineto{\pgfqpoint{3.977214in}{1.486688in}}%
\pgfpathlineto{\pgfqpoint{3.985415in}{1.500782in}}%
\pgfpathlineto{\pgfqpoint{3.971299in}{1.496964in}}%
\pgfpathlineto{\pgfqpoint{3.957194in}{1.493326in}}%
\pgfpathlineto{\pgfqpoint{3.943099in}{1.489868in}}%
\pgfpathlineto{\pgfqpoint{3.929014in}{1.486591in}}%
\pgfpathlineto{\pgfqpoint{3.920808in}{1.473058in}}%
\pgfpathlineto{\pgfqpoint{3.912598in}{1.459599in}}%
\pgfpathlineto{\pgfqpoint{3.904383in}{1.446219in}}%
\pgfpathlineto{\pgfqpoint{3.896162in}{1.432924in}}%
\pgfpathclose%
\pgfusepath{fill}%
\end{pgfscope}%
\begin{pgfscope}%
\pgfpathrectangle{\pgfqpoint{1.150000in}{0.150000in}}{\pgfqpoint{5.700000in}{5.700000in}}%
\pgfusepath{clip}%
\pgfsetbuttcap%
\pgfsetroundjoin%
\definecolor{currentfill}{rgb}{0.274128,0.199721,0.498911}%
\pgfsetfillcolor{currentfill}%
\pgfsetfillopacity{0.800000}%
\pgfsetlinewidth{0.000000pt}%
\definecolor{currentstroke}{rgb}{0.000000,0.000000,0.000000}%
\pgfsetstrokecolor{currentstroke}%
\pgfsetdash{}{0pt}%
\pgfpathmoveto{\pgfqpoint{4.107417in}{1.636595in}}%
\pgfpathlineto{\pgfqpoint{4.121586in}{1.642369in}}%
\pgfpathlineto{\pgfqpoint{4.135767in}{1.648324in}}%
\pgfpathlineto{\pgfqpoint{4.149961in}{1.654460in}}%
\pgfpathlineto{\pgfqpoint{4.164168in}{1.660776in}}%
\pgfpathlineto{\pgfqpoint{4.172328in}{1.676223in}}%
\pgfpathlineto{\pgfqpoint{4.180484in}{1.691672in}}%
\pgfpathlineto{\pgfqpoint{4.188636in}{1.707117in}}%
\pgfpathlineto{\pgfqpoint{4.196783in}{1.722554in}}%
\pgfpathlineto{\pgfqpoint{4.182576in}{1.715756in}}%
\pgfpathlineto{\pgfqpoint{4.168381in}{1.709138in}}%
\pgfpathlineto{\pgfqpoint{4.154200in}{1.702702in}}%
\pgfpathlineto{\pgfqpoint{4.140030in}{1.696446in}}%
\pgfpathlineto{\pgfqpoint{4.131883in}{1.681479in}}%
\pgfpathlineto{\pgfqpoint{4.123732in}{1.666511in}}%
\pgfpathlineto{\pgfqpoint{4.115576in}{1.651549in}}%
\pgfpathlineto{\pgfqpoint{4.107417in}{1.636595in}}%
\pgfpathclose%
\pgfusepath{fill}%
\end{pgfscope}%
\begin{pgfscope}%
\pgfpathrectangle{\pgfqpoint{1.150000in}{0.150000in}}{\pgfqpoint{5.700000in}{5.700000in}}%
\pgfusepath{clip}%
\pgfsetbuttcap%
\pgfsetroundjoin%
\definecolor{currentfill}{rgb}{0.137339,0.662252,0.515571}%
\pgfsetfillcolor{currentfill}%
\pgfsetfillopacity{0.800000}%
\pgfsetlinewidth{0.000000pt}%
\definecolor{currentstroke}{rgb}{0.000000,0.000000,0.000000}%
\pgfsetstrokecolor{currentstroke}%
\pgfsetdash{}{0pt}%
\pgfpathmoveto{\pgfqpoint{5.172406in}{2.975315in}}%
\pgfpathlineto{\pgfqpoint{5.187235in}{2.991498in}}%
\pgfpathlineto{\pgfqpoint{5.202085in}{3.007871in}}%
\pgfpathlineto{\pgfqpoint{5.216957in}{3.024433in}}%
\pgfpathlineto{\pgfqpoint{5.231851in}{3.041184in}}%
\pgfpathlineto{\pgfqpoint{5.239615in}{3.049989in}}%
\pgfpathlineto{\pgfqpoint{5.247369in}{3.058603in}}%
\pgfpathlineto{\pgfqpoint{5.255113in}{3.067027in}}%
\pgfpathlineto{\pgfqpoint{5.262847in}{3.075262in}}%
\pgfpathlineto{\pgfqpoint{5.247958in}{3.058601in}}%
\pgfpathlineto{\pgfqpoint{5.233091in}{3.042129in}}%
\pgfpathlineto{\pgfqpoint{5.218245in}{3.025846in}}%
\pgfpathlineto{\pgfqpoint{5.203422in}{3.009752in}}%
\pgfpathlineto{\pgfqpoint{5.195682in}{3.001414in}}%
\pgfpathlineto{\pgfqpoint{5.187933in}{2.992896in}}%
\pgfpathlineto{\pgfqpoint{5.180174in}{2.984196in}}%
\pgfpathlineto{\pgfqpoint{5.172406in}{2.975315in}}%
\pgfpathclose%
\pgfusepath{fill}%
\end{pgfscope}%
\begin{pgfscope}%
\pgfpathrectangle{\pgfqpoint{1.150000in}{0.150000in}}{\pgfqpoint{5.700000in}{5.700000in}}%
\pgfusepath{clip}%
\pgfsetbuttcap%
\pgfsetroundjoin%
\definecolor{currentfill}{rgb}{0.277018,0.050344,0.375715}%
\pgfsetfillcolor{currentfill}%
\pgfsetfillopacity{0.800000}%
\pgfsetlinewidth{0.000000pt}%
\definecolor{currentstroke}{rgb}{0.000000,0.000000,0.000000}%
\pgfsetstrokecolor{currentstroke}%
\pgfsetdash{}{0pt}%
\pgfpathmoveto{\pgfqpoint{3.100122in}{1.357764in}}%
\pgfpathlineto{\pgfqpoint{3.114166in}{1.348100in}}%
\pgfpathlineto{\pgfqpoint{3.128210in}{1.338635in}}%
\pgfpathlineto{\pgfqpoint{3.142255in}{1.329369in}}%
\pgfpathlineto{\pgfqpoint{3.156299in}{1.320301in}}%
\pgfpathlineto{\pgfqpoint{3.164966in}{1.321401in}}%
\pgfpathlineto{\pgfqpoint{3.173617in}{1.322849in}}%
\pgfpathlineto{\pgfqpoint{3.182253in}{1.324639in}}%
\pgfpathlineto{\pgfqpoint{3.190875in}{1.326760in}}%
\pgfpathlineto{\pgfqpoint{3.176868in}{1.335032in}}%
\pgfpathlineto{\pgfqpoint{3.162862in}{1.343501in}}%
\pgfpathlineto{\pgfqpoint{3.148857in}{1.352167in}}%
\pgfpathlineto{\pgfqpoint{3.134852in}{1.361033in}}%
\pgfpathlineto{\pgfqpoint{3.126193in}{1.359696in}}%
\pgfpathlineto{\pgfqpoint{3.117519in}{1.358700in}}%
\pgfpathlineto{\pgfqpoint{3.108829in}{1.358053in}}%
\pgfpathlineto{\pgfqpoint{3.100122in}{1.357764in}}%
\pgfpathclose%
\pgfusepath{fill}%
\end{pgfscope}%
\begin{pgfscope}%
\pgfpathrectangle{\pgfqpoint{1.150000in}{0.150000in}}{\pgfqpoint{5.700000in}{5.700000in}}%
\pgfusepath{clip}%
\pgfsetbuttcap%
\pgfsetroundjoin%
\definecolor{currentfill}{rgb}{0.360741,0.785964,0.387814}%
\pgfsetfillcolor{currentfill}%
\pgfsetfillopacity{0.800000}%
\pgfsetlinewidth{0.000000pt}%
\definecolor{currentstroke}{rgb}{0.000000,0.000000,0.000000}%
\pgfsetstrokecolor{currentstroke}%
\pgfsetdash{}{0pt}%
\pgfpathmoveto{\pgfqpoint{5.595268in}{3.406155in}}%
\pgfpathlineto{\pgfqpoint{5.610406in}{3.424285in}}%
\pgfpathlineto{\pgfqpoint{5.625569in}{3.442606in}}%
\pgfpathlineto{\pgfqpoint{5.640756in}{3.461117in}}%
\pgfpathlineto{\pgfqpoint{5.655968in}{3.479820in}}%
\pgfpathlineto{\pgfqpoint{5.663430in}{3.483407in}}%
\pgfpathlineto{\pgfqpoint{5.670879in}{3.486825in}}%
\pgfpathlineto{\pgfqpoint{5.678317in}{3.490076in}}%
\pgfpathlineto{\pgfqpoint{5.685744in}{3.493166in}}%
\pgfpathlineto{\pgfqpoint{5.670549in}{3.474776in}}%
\pgfpathlineto{\pgfqpoint{5.655380in}{3.456576in}}%
\pgfpathlineto{\pgfqpoint{5.640234in}{3.438566in}}%
\pgfpathlineto{\pgfqpoint{5.625112in}{3.420746in}}%
\pgfpathlineto{\pgfqpoint{5.617668in}{3.417332in}}%
\pgfpathlineto{\pgfqpoint{5.610213in}{3.413765in}}%
\pgfpathlineto{\pgfqpoint{5.602746in}{3.410040in}}%
\pgfpathlineto{\pgfqpoint{5.595268in}{3.406155in}}%
\pgfpathclose%
\pgfusepath{fill}%
\end{pgfscope}%
\begin{pgfscope}%
\pgfpathrectangle{\pgfqpoint{1.150000in}{0.150000in}}{\pgfqpoint{5.700000in}{5.700000in}}%
\pgfusepath{clip}%
\pgfsetbuttcap%
\pgfsetroundjoin%
\definecolor{currentfill}{rgb}{0.269944,0.014625,0.341379}%
\pgfsetfillcolor{currentfill}%
\pgfsetfillopacity{0.800000}%
\pgfsetlinewidth{0.000000pt}%
\definecolor{currentstroke}{rgb}{0.000000,0.000000,0.000000}%
\pgfsetstrokecolor{currentstroke}%
\pgfsetdash{}{0pt}%
\pgfpathmoveto{\pgfqpoint{3.302981in}{1.267605in}}%
\pgfpathlineto{\pgfqpoint{3.317003in}{1.261077in}}%
\pgfpathlineto{\pgfqpoint{3.331028in}{1.254739in}}%
\pgfpathlineto{\pgfqpoint{3.345055in}{1.248591in}}%
\pgfpathlineto{\pgfqpoint{3.359085in}{1.242632in}}%
\pgfpathlineto{\pgfqpoint{3.367596in}{1.247398in}}%
\pgfpathlineto{\pgfqpoint{3.376095in}{1.252454in}}%
\pgfpathlineto{\pgfqpoint{3.384582in}{1.257794in}}%
\pgfpathlineto{\pgfqpoint{3.393058in}{1.263409in}}%
\pgfpathlineto{\pgfqpoint{3.379056in}{1.268608in}}%
\pgfpathlineto{\pgfqpoint{3.365057in}{1.273997in}}%
\pgfpathlineto{\pgfqpoint{3.351062in}{1.279575in}}%
\pgfpathlineto{\pgfqpoint{3.337070in}{1.285343in}}%
\pgfpathlineto{\pgfqpoint{3.328566in}{1.280475in}}%
\pgfpathlineto{\pgfqpoint{3.320050in}{1.275890in}}%
\pgfpathlineto{\pgfqpoint{3.311521in}{1.271598in}}%
\pgfpathlineto{\pgfqpoint{3.302981in}{1.267605in}}%
\pgfpathclose%
\pgfusepath{fill}%
\end{pgfscope}%
\begin{pgfscope}%
\pgfpathrectangle{\pgfqpoint{1.150000in}{0.150000in}}{\pgfqpoint{5.700000in}{5.700000in}}%
\pgfusepath{clip}%
\pgfsetbuttcap%
\pgfsetroundjoin%
\definecolor{currentfill}{rgb}{0.282290,0.145912,0.461510}%
\pgfsetfillcolor{currentfill}%
\pgfsetfillopacity{0.800000}%
\pgfsetlinewidth{0.000000pt}%
\definecolor{currentstroke}{rgb}{0.000000,0.000000,0.000000}%
\pgfsetstrokecolor{currentstroke}%
\pgfsetdash{}{0pt}%
\pgfpathmoveto{\pgfqpoint{3.985415in}{1.500782in}}%
\pgfpathlineto{\pgfqpoint{3.999541in}{1.504781in}}%
\pgfpathlineto{\pgfqpoint{4.013679in}{1.508959in}}%
\pgfpathlineto{\pgfqpoint{4.027827in}{1.513317in}}%
\pgfpathlineto{\pgfqpoint{4.041987in}{1.517855in}}%
\pgfpathlineto{\pgfqpoint{4.050181in}{1.532565in}}%
\pgfpathlineto{\pgfqpoint{4.058370in}{1.547323in}}%
\pgfpathlineto{\pgfqpoint{4.066555in}{1.562124in}}%
\pgfpathlineto{\pgfqpoint{4.074736in}{1.576963in}}%
\pgfpathlineto{\pgfqpoint{4.060578in}{1.571881in}}%
\pgfpathlineto{\pgfqpoint{4.046432in}{1.566979in}}%
\pgfpathlineto{\pgfqpoint{4.032297in}{1.562258in}}%
\pgfpathlineto{\pgfqpoint{4.018173in}{1.557717in}}%
\pgfpathlineto{\pgfqpoint{4.009990in}{1.543410in}}%
\pgfpathlineto{\pgfqpoint{4.001803in}{1.529149in}}%
\pgfpathlineto{\pgfqpoint{3.993611in}{1.514938in}}%
\pgfpathlineto{\pgfqpoint{3.985415in}{1.500782in}}%
\pgfpathclose%
\pgfusepath{fill}%
\end{pgfscope}%
\begin{pgfscope}%
\pgfpathrectangle{\pgfqpoint{1.150000in}{0.150000in}}{\pgfqpoint{5.700000in}{5.700000in}}%
\pgfusepath{clip}%
\pgfsetbuttcap%
\pgfsetroundjoin%
\definecolor{currentfill}{rgb}{0.268510,0.009605,0.335427}%
\pgfsetfillcolor{currentfill}%
\pgfsetfillopacity{0.800000}%
\pgfsetlinewidth{0.000000pt}%
\definecolor{currentstroke}{rgb}{0.000000,0.000000,0.000000}%
\pgfsetstrokecolor{currentstroke}%
\pgfsetdash{}{0pt}%
\pgfpathmoveto{\pgfqpoint{3.449103in}{1.244490in}}%
\pgfpathlineto{\pgfqpoint{3.463124in}{1.240227in}}%
\pgfpathlineto{\pgfqpoint{3.477150in}{1.236150in}}%
\pgfpathlineto{\pgfqpoint{3.491181in}{1.232258in}}%
\pgfpathlineto{\pgfqpoint{3.505216in}{1.228551in}}%
\pgfpathlineto{\pgfqpoint{3.513632in}{1.235912in}}%
\pgfpathlineto{\pgfqpoint{3.522038in}{1.243516in}}%
\pgfpathlineto{\pgfqpoint{3.530435in}{1.251357in}}%
\pgfpathlineto{\pgfqpoint{3.538823in}{1.259429in}}%
\pgfpathlineto{\pgfqpoint{3.524810in}{1.262410in}}%
\pgfpathlineto{\pgfqpoint{3.510802in}{1.265575in}}%
\pgfpathlineto{\pgfqpoint{3.496799in}{1.268926in}}%
\pgfpathlineto{\pgfqpoint{3.482800in}{1.272463in}}%
\pgfpathlineto{\pgfqpoint{3.474391in}{1.265105in}}%
\pgfpathlineto{\pgfqpoint{3.465972in}{1.257986in}}%
\pgfpathlineto{\pgfqpoint{3.457542in}{1.251112in}}%
\pgfpathlineto{\pgfqpoint{3.449103in}{1.244490in}}%
\pgfpathclose%
\pgfusepath{fill}%
\end{pgfscope}%
\begin{pgfscope}%
\pgfpathrectangle{\pgfqpoint{1.150000in}{0.150000in}}{\pgfqpoint{5.700000in}{5.700000in}}%
\pgfusepath{clip}%
\pgfsetbuttcap%
\pgfsetroundjoin%
\definecolor{currentfill}{rgb}{0.232815,0.732247,0.459277}%
\pgfsetfillcolor{currentfill}%
\pgfsetfillopacity{0.800000}%
\pgfsetlinewidth{0.000000pt}%
\definecolor{currentstroke}{rgb}{0.000000,0.000000,0.000000}%
\pgfsetstrokecolor{currentstroke}%
\pgfsetdash{}{0pt}%
\pgfpathmoveto{\pgfqpoint{5.384086in}{3.201939in}}%
\pgfpathlineto{\pgfqpoint{5.399073in}{3.219258in}}%
\pgfpathlineto{\pgfqpoint{5.414084in}{3.236768in}}%
\pgfpathlineto{\pgfqpoint{5.429117in}{3.254467in}}%
\pgfpathlineto{\pgfqpoint{5.444174in}{3.272358in}}%
\pgfpathlineto{\pgfqpoint{5.451800in}{3.278603in}}%
\pgfpathlineto{\pgfqpoint{5.459414in}{3.284662in}}%
\pgfpathlineto{\pgfqpoint{5.467017in}{3.290535in}}%
\pgfpathlineto{\pgfqpoint{5.474610in}{3.296227in}}%
\pgfpathlineto{\pgfqpoint{5.459563in}{3.278537in}}%
\pgfpathlineto{\pgfqpoint{5.444540in}{3.261038in}}%
\pgfpathlineto{\pgfqpoint{5.429540in}{3.243728in}}%
\pgfpathlineto{\pgfqpoint{5.414563in}{3.226608in}}%
\pgfpathlineto{\pgfqpoint{5.406959in}{3.220703in}}%
\pgfpathlineto{\pgfqpoint{5.399345in}{3.214625in}}%
\pgfpathlineto{\pgfqpoint{5.391721in}{3.208371in}}%
\pgfpathlineto{\pgfqpoint{5.384086in}{3.201939in}}%
\pgfpathclose%
\pgfusepath{fill}%
\end{pgfscope}%
\begin{pgfscope}%
\pgfpathrectangle{\pgfqpoint{1.150000in}{0.150000in}}{\pgfqpoint{5.700000in}{5.700000in}}%
\pgfusepath{clip}%
\pgfsetbuttcap%
\pgfsetroundjoin%
\definecolor{currentfill}{rgb}{0.156270,0.489624,0.557936}%
\pgfsetfillcolor{currentfill}%
\pgfsetfillopacity{0.800000}%
\pgfsetlinewidth{0.000000pt}%
\definecolor{currentstroke}{rgb}{0.000000,0.000000,0.000000}%
\pgfsetstrokecolor{currentstroke}%
\pgfsetdash{}{0pt}%
\pgfpathmoveto{\pgfqpoint{2.133761in}{2.548218in}}%
\pgfpathlineto{\pgfqpoint{2.148316in}{2.521841in}}%
\pgfpathlineto{\pgfqpoint{2.162854in}{2.495776in}}%
\pgfpathlineto{\pgfqpoint{2.177377in}{2.470023in}}%
\pgfpathlineto{\pgfqpoint{2.191883in}{2.444577in}}%
\pgfpathlineto{\pgfqpoint{2.201549in}{2.431359in}}%
\pgfpathlineto{\pgfqpoint{2.211182in}{2.418658in}}%
\pgfpathlineto{\pgfqpoint{2.220783in}{2.406464in}}%
\pgfpathlineto{\pgfqpoint{2.230352in}{2.394768in}}%
\pgfpathlineto{\pgfqpoint{2.215925in}{2.419334in}}%
\pgfpathlineto{\pgfqpoint{2.201483in}{2.444205in}}%
\pgfpathlineto{\pgfqpoint{2.187025in}{2.469385in}}%
\pgfpathlineto{\pgfqpoint{2.172552in}{2.494875in}}%
\pgfpathlineto{\pgfqpoint{2.162904in}{2.507438in}}%
\pgfpathlineto{\pgfqpoint{2.153223in}{2.520510in}}%
\pgfpathlineto{\pgfqpoint{2.143509in}{2.534100in}}%
\pgfpathlineto{\pgfqpoint{2.133761in}{2.548218in}}%
\pgfpathclose%
\pgfusepath{fill}%
\end{pgfscope}%
\begin{pgfscope}%
\pgfpathrectangle{\pgfqpoint{1.150000in}{0.150000in}}{\pgfqpoint{5.700000in}{5.700000in}}%
\pgfusepath{clip}%
\pgfsetbuttcap%
\pgfsetroundjoin%
\definecolor{currentfill}{rgb}{0.120081,0.622161,0.534946}%
\pgfsetfillcolor{currentfill}%
\pgfsetfillopacity{0.800000}%
\pgfsetlinewidth{0.000000pt}%
\definecolor{currentstroke}{rgb}{0.000000,0.000000,0.000000}%
\pgfsetstrokecolor{currentstroke}%
\pgfsetdash{}{0pt}%
\pgfpathmoveto{\pgfqpoint{5.050813in}{2.834431in}}%
\pgfpathlineto{\pgfqpoint{5.065562in}{2.849932in}}%
\pgfpathlineto{\pgfqpoint{5.080331in}{2.865621in}}%
\pgfpathlineto{\pgfqpoint{5.095122in}{2.881499in}}%
\pgfpathlineto{\pgfqpoint{5.109934in}{2.897565in}}%
\pgfpathlineto{\pgfqpoint{5.117774in}{2.907943in}}%
\pgfpathlineto{\pgfqpoint{5.125606in}{2.918131in}}%
\pgfpathlineto{\pgfqpoint{5.133429in}{2.928130in}}%
\pgfpathlineto{\pgfqpoint{5.141243in}{2.937941in}}%
\pgfpathlineto{\pgfqpoint{5.126432in}{2.921892in}}%
\pgfpathlineto{\pgfqpoint{5.111644in}{2.906032in}}%
\pgfpathlineto{\pgfqpoint{5.096876in}{2.890361in}}%
\pgfpathlineto{\pgfqpoint{5.082129in}{2.874878in}}%
\pgfpathlineto{\pgfqpoint{5.074313in}{2.865036in}}%
\pgfpathlineto{\pgfqpoint{5.066488in}{2.855015in}}%
\pgfpathlineto{\pgfqpoint{5.058655in}{2.844813in}}%
\pgfpathlineto{\pgfqpoint{5.050813in}{2.834431in}}%
\pgfpathclose%
\pgfusepath{fill}%
\end{pgfscope}%
\begin{pgfscope}%
\pgfpathrectangle{\pgfqpoint{1.150000in}{0.150000in}}{\pgfqpoint{5.700000in}{5.700000in}}%
\pgfusepath{clip}%
\pgfsetbuttcap%
\pgfsetroundjoin%
\definecolor{currentfill}{rgb}{0.237441,0.305202,0.541921}%
\pgfsetfillcolor{currentfill}%
\pgfsetfillopacity{0.800000}%
\pgfsetlinewidth{0.000000pt}%
\definecolor{currentstroke}{rgb}{0.000000,0.000000,0.000000}%
\pgfsetstrokecolor{currentstroke}%
\pgfsetdash{}{0pt}%
\pgfpathmoveto{\pgfqpoint{4.318803in}{1.877755in}}%
\pgfpathlineto{\pgfqpoint{4.333084in}{1.886331in}}%
\pgfpathlineto{\pgfqpoint{4.347380in}{1.895089in}}%
\pgfpathlineto{\pgfqpoint{4.361690in}{1.904029in}}%
\pgfpathlineto{\pgfqpoint{4.376016in}{1.913151in}}%
\pgfpathlineto{\pgfqpoint{4.384133in}{1.929133in}}%
\pgfpathlineto{\pgfqpoint{4.392247in}{1.945052in}}%
\pgfpathlineto{\pgfqpoint{4.400356in}{1.960903in}}%
\pgfpathlineto{\pgfqpoint{4.408461in}{1.976683in}}%
\pgfpathlineto{\pgfqpoint{4.394131in}{1.967172in}}%
\pgfpathlineto{\pgfqpoint{4.379816in}{1.957843in}}%
\pgfpathlineto{\pgfqpoint{4.365517in}{1.948696in}}%
\pgfpathlineto{\pgfqpoint{4.351232in}{1.939731in}}%
\pgfpathlineto{\pgfqpoint{4.343131in}{1.924327in}}%
\pgfpathlineto{\pgfqpoint{4.335026in}{1.908861in}}%
\pgfpathlineto{\pgfqpoint{4.326916in}{1.893336in}}%
\pgfpathlineto{\pgfqpoint{4.318803in}{1.877755in}}%
\pgfpathclose%
\pgfusepath{fill}%
\end{pgfscope}%
\begin{pgfscope}%
\pgfpathrectangle{\pgfqpoint{1.150000in}{0.150000in}}{\pgfqpoint{5.700000in}{5.700000in}}%
\pgfusepath{clip}%
\pgfsetbuttcap%
\pgfsetroundjoin%
\definecolor{currentfill}{rgb}{0.119483,0.614817,0.537692}%
\pgfsetfillcolor{currentfill}%
\pgfsetfillopacity{0.800000}%
\pgfsetlinewidth{0.000000pt}%
\definecolor{currentstroke}{rgb}{0.000000,0.000000,0.000000}%
\pgfsetstrokecolor{currentstroke}%
\pgfsetdash{}{0pt}%
\pgfpathmoveto{\pgfqpoint{1.938542in}{2.947551in}}%
\pgfpathlineto{\pgfqpoint{1.953317in}{2.916653in}}%
\pgfpathlineto{\pgfqpoint{1.968070in}{2.886119in}}%
\pgfpathlineto{\pgfqpoint{1.982802in}{2.855945in}}%
\pgfpathlineto{\pgfqpoint{1.997514in}{2.826128in}}%
\pgfpathlineto{\pgfqpoint{2.007379in}{2.811440in}}%
\pgfpathlineto{\pgfqpoint{2.017209in}{2.797277in}}%
\pgfpathlineto{\pgfqpoint{2.027005in}{2.783629in}}%
\pgfpathlineto{\pgfqpoint{2.036766in}{2.770487in}}%
\pgfpathlineto{\pgfqpoint{2.022141in}{2.799429in}}%
\pgfpathlineto{\pgfqpoint{2.007496in}{2.828725in}}%
\pgfpathlineto{\pgfqpoint{1.992831in}{2.858379in}}%
\pgfpathlineto{\pgfqpoint{1.978144in}{2.888393in}}%
\pgfpathlineto{\pgfqpoint{1.968297in}{2.902396in}}%
\pgfpathlineto{\pgfqpoint{1.958415in}{2.916918in}}%
\pgfpathlineto{\pgfqpoint{1.948497in}{2.931966in}}%
\pgfpathlineto{\pgfqpoint{1.938542in}{2.947551in}}%
\pgfpathclose%
\pgfusepath{fill}%
\end{pgfscope}%
\begin{pgfscope}%
\pgfpathrectangle{\pgfqpoint{1.150000in}{0.150000in}}{\pgfqpoint{5.700000in}{5.700000in}}%
\pgfusepath{clip}%
\pgfsetbuttcap%
\pgfsetroundjoin%
\definecolor{currentfill}{rgb}{0.210503,0.363727,0.552206}%
\pgfsetfillcolor{currentfill}%
\pgfsetfillopacity{0.800000}%
\pgfsetlinewidth{0.000000pt}%
\definecolor{currentstroke}{rgb}{0.000000,0.000000,0.000000}%
\pgfsetstrokecolor{currentstroke}%
\pgfsetdash{}{0pt}%
\pgfpathmoveto{\pgfqpoint{4.440836in}{2.039032in}}%
\pgfpathlineto{\pgfqpoint{4.455186in}{2.049083in}}%
\pgfpathlineto{\pgfqpoint{4.469552in}{2.059317in}}%
\pgfpathlineto{\pgfqpoint{4.483933in}{2.069735in}}%
\pgfpathlineto{\pgfqpoint{4.498331in}{2.080335in}}%
\pgfpathlineto{\pgfqpoint{4.506418in}{2.096050in}}%
\pgfpathlineto{\pgfqpoint{4.514501in}{2.111668in}}%
\pgfpathlineto{\pgfqpoint{4.522579in}{2.127184in}}%
\pgfpathlineto{\pgfqpoint{4.530653in}{2.142597in}}%
\pgfpathlineto{\pgfqpoint{4.516250in}{2.131671in}}%
\pgfpathlineto{\pgfqpoint{4.501864in}{2.120928in}}%
\pgfpathlineto{\pgfqpoint{4.487494in}{2.110369in}}%
\pgfpathlineto{\pgfqpoint{4.473140in}{2.099994in}}%
\pgfpathlineto{\pgfqpoint{4.465071in}{2.084893in}}%
\pgfpathlineto{\pgfqpoint{4.456997in}{2.069698in}}%
\pgfpathlineto{\pgfqpoint{4.448919in}{2.054410in}}%
\pgfpathlineto{\pgfqpoint{4.440836in}{2.039032in}}%
\pgfpathclose%
\pgfusepath{fill}%
\end{pgfscope}%
\begin{pgfscope}%
\pgfpathrectangle{\pgfqpoint{1.150000in}{0.150000in}}{\pgfqpoint{5.700000in}{5.700000in}}%
\pgfusepath{clip}%
\pgfsetbuttcap%
\pgfsetroundjoin%
\definecolor{currentfill}{rgb}{0.262138,0.242286,0.520837}%
\pgfsetfillcolor{currentfill}%
\pgfsetfillopacity{0.800000}%
\pgfsetlinewidth{0.000000pt}%
\definecolor{currentstroke}{rgb}{0.000000,0.000000,0.000000}%
\pgfsetstrokecolor{currentstroke}%
\pgfsetdash{}{0pt}%
\pgfpathmoveto{\pgfqpoint{4.196783in}{1.722554in}}%
\pgfpathlineto{\pgfqpoint{4.211004in}{1.729533in}}%
\pgfpathlineto{\pgfqpoint{4.225238in}{1.736693in}}%
\pgfpathlineto{\pgfqpoint{4.239485in}{1.744034in}}%
\pgfpathlineto{\pgfqpoint{4.253746in}{1.751556in}}%
\pgfpathlineto{\pgfqpoint{4.261892in}{1.767445in}}%
\pgfpathlineto{\pgfqpoint{4.270034in}{1.783309in}}%
\pgfpathlineto{\pgfqpoint{4.278173in}{1.799145in}}%
\pgfpathlineto{\pgfqpoint{4.286307in}{1.814949in}}%
\pgfpathlineto{\pgfqpoint{4.272043in}{1.806974in}}%
\pgfpathlineto{\pgfqpoint{4.257793in}{1.799181in}}%
\pgfpathlineto{\pgfqpoint{4.243557in}{1.791570in}}%
\pgfpathlineto{\pgfqpoint{4.229335in}{1.784140in}}%
\pgfpathlineto{\pgfqpoint{4.221203in}{1.768776in}}%
\pgfpathlineto{\pgfqpoint{4.213067in}{1.753388in}}%
\pgfpathlineto{\pgfqpoint{4.204927in}{1.737979in}}%
\pgfpathlineto{\pgfqpoint{4.196783in}{1.722554in}}%
\pgfpathclose%
\pgfusepath{fill}%
\end{pgfscope}%
\begin{pgfscope}%
\pgfpathrectangle{\pgfqpoint{1.150000in}{0.150000in}}{\pgfqpoint{5.700000in}{5.700000in}}%
\pgfusepath{clip}%
\pgfsetbuttcap%
\pgfsetroundjoin%
\definecolor{currentfill}{rgb}{0.274952,0.037752,0.364543}%
\pgfsetfillcolor{currentfill}%
\pgfsetfillopacity{0.800000}%
\pgfsetlinewidth{0.000000pt}%
\definecolor{currentstroke}{rgb}{0.000000,0.000000,0.000000}%
\pgfsetstrokecolor{currentstroke}%
\pgfsetdash{}{0pt}%
\pgfpathmoveto{\pgfqpoint{3.156299in}{1.320301in}}%
\pgfpathlineto{\pgfqpoint{3.170344in}{1.311430in}}%
\pgfpathlineto{\pgfqpoint{3.184390in}{1.302755in}}%
\pgfpathlineto{\pgfqpoint{3.198436in}{1.294276in}}%
\pgfpathlineto{\pgfqpoint{3.212484in}{1.285991in}}%
\pgfpathlineto{\pgfqpoint{3.221113in}{1.287899in}}%
\pgfpathlineto{\pgfqpoint{3.229727in}{1.290148in}}%
\pgfpathlineto{\pgfqpoint{3.238328in}{1.292729in}}%
\pgfpathlineto{\pgfqpoint{3.246914in}{1.295633in}}%
\pgfpathlineto{\pgfqpoint{3.232902in}{1.303123in}}%
\pgfpathlineto{\pgfqpoint{3.218892in}{1.310806in}}%
\pgfpathlineto{\pgfqpoint{3.204883in}{1.318685in}}%
\pgfpathlineto{\pgfqpoint{3.190875in}{1.326760in}}%
\pgfpathlineto{\pgfqpoint{3.182253in}{1.324639in}}%
\pgfpathlineto{\pgfqpoint{3.173617in}{1.322849in}}%
\pgfpathlineto{\pgfqpoint{3.164966in}{1.321401in}}%
\pgfpathlineto{\pgfqpoint{3.156299in}{1.320301in}}%
\pgfpathclose%
\pgfusepath{fill}%
\end{pgfscope}%
\begin{pgfscope}%
\pgfpathrectangle{\pgfqpoint{1.150000in}{0.150000in}}{\pgfqpoint{5.700000in}{5.700000in}}%
\pgfusepath{clip}%
\pgfsetbuttcap%
\pgfsetroundjoin%
\definecolor{currentfill}{rgb}{0.274952,0.037752,0.364543}%
\pgfsetfillcolor{currentfill}%
\pgfsetfillopacity{0.800000}%
\pgfsetlinewidth{0.000000pt}%
\definecolor{currentstroke}{rgb}{0.000000,0.000000,0.000000}%
\pgfsetstrokecolor{currentstroke}%
\pgfsetdash{}{0pt}%
\pgfpathmoveto{\pgfqpoint{3.684462in}{1.282185in}}%
\pgfpathlineto{\pgfqpoint{3.698514in}{1.281545in}}%
\pgfpathlineto{\pgfqpoint{3.712573in}{1.281085in}}%
\pgfpathlineto{\pgfqpoint{3.726640in}{1.280806in}}%
\pgfpathlineto{\pgfqpoint{3.740715in}{1.280708in}}%
\pgfpathlineto{\pgfqpoint{3.749014in}{1.291850in}}%
\pgfpathlineto{\pgfqpoint{3.757307in}{1.303156in}}%
\pgfpathlineto{\pgfqpoint{3.765593in}{1.314621in}}%
\pgfpathlineto{\pgfqpoint{3.773873in}{1.326238in}}%
\pgfpathlineto{\pgfqpoint{3.759811in}{1.325672in}}%
\pgfpathlineto{\pgfqpoint{3.745756in}{1.325286in}}%
\pgfpathlineto{\pgfqpoint{3.731710in}{1.325082in}}%
\pgfpathlineto{\pgfqpoint{3.717671in}{1.325059in}}%
\pgfpathlineto{\pgfqpoint{3.709379in}{1.314094in}}%
\pgfpathlineto{\pgfqpoint{3.701080in}{1.303289in}}%
\pgfpathlineto{\pgfqpoint{3.692774in}{1.292651in}}%
\pgfpathlineto{\pgfqpoint{3.684462in}{1.282185in}}%
\pgfpathclose%
\pgfusepath{fill}%
\end{pgfscope}%
\begin{pgfscope}%
\pgfpathrectangle{\pgfqpoint{1.150000in}{0.150000in}}{\pgfqpoint{5.700000in}{5.700000in}}%
\pgfusepath{clip}%
\pgfsetbuttcap%
\pgfsetroundjoin%
\definecolor{currentfill}{rgb}{0.124395,0.578002,0.548287}%
\pgfsetfillcolor{currentfill}%
\pgfsetfillopacity{0.800000}%
\pgfsetlinewidth{0.000000pt}%
\definecolor{currentstroke}{rgb}{0.000000,0.000000,0.000000}%
\pgfsetstrokecolor{currentstroke}%
\pgfsetdash{}{0pt}%
\pgfpathmoveto{\pgfqpoint{4.928991in}{2.684977in}}%
\pgfpathlineto{\pgfqpoint{4.943658in}{2.699656in}}%
\pgfpathlineto{\pgfqpoint{4.958345in}{2.714521in}}%
\pgfpathlineto{\pgfqpoint{4.973052in}{2.729575in}}%
\pgfpathlineto{\pgfqpoint{4.987779in}{2.744816in}}%
\pgfpathlineto{\pgfqpoint{4.995687in}{2.756657in}}%
\pgfpathlineto{\pgfqpoint{5.003587in}{2.768315in}}%
\pgfpathlineto{\pgfqpoint{5.011478in}{2.779790in}}%
\pgfpathlineto{\pgfqpoint{5.019362in}{2.791083in}}%
\pgfpathlineto{\pgfqpoint{5.004633in}{2.775788in}}%
\pgfpathlineto{\pgfqpoint{4.989925in}{2.760682in}}%
\pgfpathlineto{\pgfqpoint{4.975238in}{2.745763in}}%
\pgfpathlineto{\pgfqpoint{4.960570in}{2.731031in}}%
\pgfpathlineto{\pgfqpoint{4.952687in}{2.719778in}}%
\pgfpathlineto{\pgfqpoint{4.944796in}{2.708352in}}%
\pgfpathlineto{\pgfqpoint{4.936898in}{2.696751in}}%
\pgfpathlineto{\pgfqpoint{4.928991in}{2.684977in}}%
\pgfpathclose%
\pgfusepath{fill}%
\end{pgfscope}%
\begin{pgfscope}%
\pgfpathrectangle{\pgfqpoint{1.150000in}{0.150000in}}{\pgfqpoint{5.700000in}{5.700000in}}%
\pgfusepath{clip}%
\pgfsetbuttcap%
\pgfsetroundjoin%
\definecolor{currentfill}{rgb}{0.278791,0.062145,0.386592}%
\pgfsetfillcolor{currentfill}%
\pgfsetfillopacity{0.800000}%
\pgfsetlinewidth{0.000000pt}%
\definecolor{currentstroke}{rgb}{0.000000,0.000000,0.000000}%
\pgfsetstrokecolor{currentstroke}%
\pgfsetdash{}{0pt}%
\pgfpathmoveto{\pgfqpoint{3.773873in}{1.326238in}}%
\pgfpathlineto{\pgfqpoint{3.787944in}{1.326985in}}%
\pgfpathlineto{\pgfqpoint{3.802023in}{1.327912in}}%
\pgfpathlineto{\pgfqpoint{3.816110in}{1.329019in}}%
\pgfpathlineto{\pgfqpoint{3.830207in}{1.330306in}}%
\pgfpathlineto{\pgfqpoint{3.838471in}{1.342715in}}%
\pgfpathlineto{\pgfqpoint{3.846729in}{1.355255in}}%
\pgfpathlineto{\pgfqpoint{3.854981in}{1.367921in}}%
\pgfpathlineto{\pgfqpoint{3.863228in}{1.380706in}}%
\pgfpathlineto{\pgfqpoint{3.849140in}{1.378785in}}%
\pgfpathlineto{\pgfqpoint{3.835062in}{1.377043in}}%
\pgfpathlineto{\pgfqpoint{3.820992in}{1.375482in}}%
\pgfpathlineto{\pgfqpoint{3.806931in}{1.374102in}}%
\pgfpathlineto{\pgfqpoint{3.798676in}{1.361939in}}%
\pgfpathlineto{\pgfqpoint{3.790414in}{1.349903in}}%
\pgfpathlineto{\pgfqpoint{3.782147in}{1.338001in}}%
\pgfpathlineto{\pgfqpoint{3.773873in}{1.326238in}}%
\pgfpathclose%
\pgfusepath{fill}%
\end{pgfscope}%
\begin{pgfscope}%
\pgfpathrectangle{\pgfqpoint{1.150000in}{0.150000in}}{\pgfqpoint{5.700000in}{5.700000in}}%
\pgfusepath{clip}%
\pgfsetbuttcap%
\pgfsetroundjoin%
\definecolor{currentfill}{rgb}{0.183898,0.422383,0.556944}%
\pgfsetfillcolor{currentfill}%
\pgfsetfillopacity{0.800000}%
\pgfsetlinewidth{0.000000pt}%
\definecolor{currentstroke}{rgb}{0.000000,0.000000,0.000000}%
\pgfsetstrokecolor{currentstroke}%
\pgfsetdash{}{0pt}%
\pgfpathmoveto{\pgfqpoint{4.562897in}{2.203158in}}%
\pgfpathlineto{\pgfqpoint{4.577321in}{2.214560in}}%
\pgfpathlineto{\pgfqpoint{4.591763in}{2.226146in}}%
\pgfpathlineto{\pgfqpoint{4.606221in}{2.237917in}}%
\pgfpathlineto{\pgfqpoint{4.620696in}{2.249871in}}%
\pgfpathlineto{\pgfqpoint{4.628750in}{2.264999in}}%
\pgfpathlineto{\pgfqpoint{4.636798in}{2.279999in}}%
\pgfpathlineto{\pgfqpoint{4.644841in}{2.294870in}}%
\pgfpathlineto{\pgfqpoint{4.652878in}{2.309609in}}%
\pgfpathlineto{\pgfqpoint{4.638397in}{2.297394in}}%
\pgfpathlineto{\pgfqpoint{4.623934in}{2.285364in}}%
\pgfpathlineto{\pgfqpoint{4.609488in}{2.273519in}}%
\pgfpathlineto{\pgfqpoint{4.595060in}{2.261858in}}%
\pgfpathlineto{\pgfqpoint{4.587027in}{2.247365in}}%
\pgfpathlineto{\pgfqpoint{4.578989in}{2.232749in}}%
\pgfpathlineto{\pgfqpoint{4.570946in}{2.218013in}}%
\pgfpathlineto{\pgfqpoint{4.562897in}{2.203158in}}%
\pgfpathclose%
\pgfusepath{fill}%
\end{pgfscope}%
\begin{pgfscope}%
\pgfpathrectangle{\pgfqpoint{1.150000in}{0.150000in}}{\pgfqpoint{5.700000in}{5.700000in}}%
\pgfusepath{clip}%
\pgfsetbuttcap%
\pgfsetroundjoin%
\definecolor{currentfill}{rgb}{0.162142,0.474838,0.558140}%
\pgfsetfillcolor{currentfill}%
\pgfsetfillopacity{0.800000}%
\pgfsetlinewidth{0.000000pt}%
\definecolor{currentstroke}{rgb}{0.000000,0.000000,0.000000}%
\pgfsetstrokecolor{currentstroke}%
\pgfsetdash{}{0pt}%
\pgfpathmoveto{\pgfqpoint{4.684971in}{2.367211in}}%
\pgfpathlineto{\pgfqpoint{4.699474in}{2.379837in}}%
\pgfpathlineto{\pgfqpoint{4.713995in}{2.392648in}}%
\pgfpathlineto{\pgfqpoint{4.728534in}{2.405645in}}%
\pgfpathlineto{\pgfqpoint{4.743092in}{2.418827in}}%
\pgfpathlineto{\pgfqpoint{4.751105in}{2.433085in}}%
\pgfpathlineto{\pgfqpoint{4.759112in}{2.447192in}}%
\pgfpathlineto{\pgfqpoint{4.767113in}{2.461146in}}%
\pgfpathlineto{\pgfqpoint{4.775107in}{2.474946in}}%
\pgfpathlineto{\pgfqpoint{4.760545in}{2.461572in}}%
\pgfpathlineto{\pgfqpoint{4.746002in}{2.448383in}}%
\pgfpathlineto{\pgfqpoint{4.731477in}{2.435380in}}%
\pgfpathlineto{\pgfqpoint{4.716970in}{2.422563in}}%
\pgfpathlineto{\pgfqpoint{4.708979in}{2.408941in}}%
\pgfpathlineto{\pgfqpoint{4.700982in}{2.395175in}}%
\pgfpathlineto{\pgfqpoint{4.692980in}{2.381264in}}%
\pgfpathlineto{\pgfqpoint{4.684971in}{2.367211in}}%
\pgfpathclose%
\pgfusepath{fill}%
\end{pgfscope}%
\begin{pgfscope}%
\pgfpathrectangle{\pgfqpoint{1.150000in}{0.150000in}}{\pgfqpoint{5.700000in}{5.700000in}}%
\pgfusepath{clip}%
\pgfsetbuttcap%
\pgfsetroundjoin%
\definecolor{currentfill}{rgb}{0.141935,0.526453,0.555991}%
\pgfsetfillcolor{currentfill}%
\pgfsetfillopacity{0.800000}%
\pgfsetlinewidth{0.000000pt}%
\definecolor{currentstroke}{rgb}{0.000000,0.000000,0.000000}%
\pgfsetstrokecolor{currentstroke}%
\pgfsetdash{}{0pt}%
\pgfpathmoveto{\pgfqpoint{4.807021in}{2.528582in}}%
\pgfpathlineto{\pgfqpoint{4.821605in}{2.542301in}}%
\pgfpathlineto{\pgfqpoint{4.836208in}{2.556206in}}%
\pgfpathlineto{\pgfqpoint{4.850831in}{2.570298in}}%
\pgfpathlineto{\pgfqpoint{4.865473in}{2.584577in}}%
\pgfpathlineto{\pgfqpoint{4.873438in}{2.597725in}}%
\pgfpathlineto{\pgfqpoint{4.881396in}{2.610704in}}%
\pgfpathlineto{\pgfqpoint{4.889347in}{2.623513in}}%
\pgfpathlineto{\pgfqpoint{4.897290in}{2.636150in}}%
\pgfpathlineto{\pgfqpoint{4.882645in}{2.621748in}}%
\pgfpathlineto{\pgfqpoint{4.868020in}{2.607532in}}%
\pgfpathlineto{\pgfqpoint{4.853414in}{2.593504in}}%
\pgfpathlineto{\pgfqpoint{4.838827in}{2.579662in}}%
\pgfpathlineto{\pgfqpoint{4.830885in}{2.567135in}}%
\pgfpathlineto{\pgfqpoint{4.822937in}{2.554445in}}%
\pgfpathlineto{\pgfqpoint{4.814982in}{2.541594in}}%
\pgfpathlineto{\pgfqpoint{4.807021in}{2.528582in}}%
\pgfpathclose%
\pgfusepath{fill}%
\end{pgfscope}%
\begin{pgfscope}%
\pgfpathrectangle{\pgfqpoint{1.150000in}{0.150000in}}{\pgfqpoint{5.700000in}{5.700000in}}%
\pgfusepath{clip}%
\pgfsetbuttcap%
\pgfsetroundjoin%
\definecolor{currentfill}{rgb}{0.271305,0.019942,0.347269}%
\pgfsetfillcolor{currentfill}%
\pgfsetfillopacity{0.800000}%
\pgfsetlinewidth{0.000000pt}%
\definecolor{currentstroke}{rgb}{0.000000,0.000000,0.000000}%
\pgfsetstrokecolor{currentstroke}%
\pgfsetdash{}{0pt}%
\pgfpathmoveto{\pgfqpoint{3.594932in}{1.249345in}}%
\pgfpathlineto{\pgfqpoint{3.608974in}{1.247282in}}%
\pgfpathlineto{\pgfqpoint{3.623022in}{1.245401in}}%
\pgfpathlineto{\pgfqpoint{3.637076in}{1.243702in}}%
\pgfpathlineto{\pgfqpoint{3.651137in}{1.242185in}}%
\pgfpathlineto{\pgfqpoint{3.659479in}{1.251892in}}%
\pgfpathlineto{\pgfqpoint{3.667814in}{1.261799in}}%
\pgfpathlineto{\pgfqpoint{3.676142in}{1.271899in}}%
\pgfpathlineto{\pgfqpoint{3.684462in}{1.282185in}}%
\pgfpathlineto{\pgfqpoint{3.670417in}{1.283007in}}%
\pgfpathlineto{\pgfqpoint{3.656378in}{1.284011in}}%
\pgfpathlineto{\pgfqpoint{3.642347in}{1.285197in}}%
\pgfpathlineto{\pgfqpoint{3.628323in}{1.286566in}}%
\pgfpathlineto{\pgfqpoint{3.619987in}{1.276963in}}%
\pgfpathlineto{\pgfqpoint{3.611643in}{1.267554in}}%
\pgfpathlineto{\pgfqpoint{3.603292in}{1.258345in}}%
\pgfpathlineto{\pgfqpoint{3.594932in}{1.249345in}}%
\pgfpathclose%
\pgfusepath{fill}%
\end{pgfscope}%
\begin{pgfscope}%
\pgfpathrectangle{\pgfqpoint{1.150000in}{0.150000in}}{\pgfqpoint{5.700000in}{5.700000in}}%
\pgfusepath{clip}%
\pgfsetbuttcap%
\pgfsetroundjoin%
\definecolor{currentfill}{rgb}{0.277134,0.185228,0.489898}%
\pgfsetfillcolor{currentfill}%
\pgfsetfillopacity{0.800000}%
\pgfsetlinewidth{0.000000pt}%
\definecolor{currentstroke}{rgb}{0.000000,0.000000,0.000000}%
\pgfsetstrokecolor{currentstroke}%
\pgfsetdash{}{0pt}%
\pgfpathmoveto{\pgfqpoint{4.074736in}{1.576963in}}%
\pgfpathlineto{\pgfqpoint{4.088906in}{1.582224in}}%
\pgfpathlineto{\pgfqpoint{4.103087in}{1.587666in}}%
\pgfpathlineto{\pgfqpoint{4.117281in}{1.593287in}}%
\pgfpathlineto{\pgfqpoint{4.131487in}{1.599089in}}%
\pgfpathlineto{\pgfqpoint{4.139663in}{1.614486in}}%
\pgfpathlineto{\pgfqpoint{4.147835in}{1.629902in}}%
\pgfpathlineto{\pgfqpoint{4.156003in}{1.645334in}}%
\pgfpathlineto{\pgfqpoint{4.164168in}{1.660776in}}%
\pgfpathlineto{\pgfqpoint{4.149961in}{1.654460in}}%
\pgfpathlineto{\pgfqpoint{4.135767in}{1.648324in}}%
\pgfpathlineto{\pgfqpoint{4.121586in}{1.642369in}}%
\pgfpathlineto{\pgfqpoint{4.107417in}{1.636595in}}%
\pgfpathlineto{\pgfqpoint{4.099253in}{1.621655in}}%
\pgfpathlineto{\pgfqpoint{4.091085in}{1.606733in}}%
\pgfpathlineto{\pgfqpoint{4.082913in}{1.591834in}}%
\pgfpathlineto{\pgfqpoint{4.074736in}{1.576963in}}%
\pgfpathclose%
\pgfusepath{fill}%
\end{pgfscope}%
\begin{pgfscope}%
\pgfpathrectangle{\pgfqpoint{1.150000in}{0.150000in}}{\pgfqpoint{5.700000in}{5.700000in}}%
\pgfusepath{clip}%
\pgfsetbuttcap%
\pgfsetroundjoin%
\definecolor{currentfill}{rgb}{0.440137,0.811138,0.340967}%
\pgfsetfillcolor{currentfill}%
\pgfsetfillopacity{0.800000}%
\pgfsetlinewidth{0.000000pt}%
\definecolor{currentstroke}{rgb}{0.000000,0.000000,0.000000}%
\pgfsetstrokecolor{currentstroke}%
\pgfsetdash{}{0pt}%
\pgfpathmoveto{\pgfqpoint{5.685744in}{3.493166in}}%
\pgfpathlineto{\pgfqpoint{5.700963in}{3.511747in}}%
\pgfpathlineto{\pgfqpoint{5.716207in}{3.530519in}}%
\pgfpathlineto{\pgfqpoint{5.731476in}{3.549482in}}%
\pgfpathlineto{\pgfqpoint{5.746770in}{3.568637in}}%
\pgfpathlineto{\pgfqpoint{5.754166in}{3.571231in}}%
\pgfpathlineto{\pgfqpoint{5.761550in}{3.573662in}}%
\pgfpathlineto{\pgfqpoint{5.768921in}{3.575933in}}%
\pgfpathlineto{\pgfqpoint{5.776281in}{3.578048in}}%
\pgfpathlineto{\pgfqpoint{5.761006in}{3.559244in}}%
\pgfpathlineto{\pgfqpoint{5.745757in}{3.540631in}}%
\pgfpathlineto{\pgfqpoint{5.730533in}{3.522208in}}%
\pgfpathlineto{\pgfqpoint{5.715334in}{3.503975in}}%
\pgfpathlineto{\pgfqpoint{5.707953in}{3.501497in}}%
\pgfpathlineto{\pgfqpoint{5.700562in}{3.498872in}}%
\pgfpathlineto{\pgfqpoint{5.693159in}{3.496096in}}%
\pgfpathlineto{\pgfqpoint{5.685744in}{3.493166in}}%
\pgfpathclose%
\pgfusepath{fill}%
\end{pgfscope}%
\begin{pgfscope}%
\pgfpathrectangle{\pgfqpoint{1.150000in}{0.150000in}}{\pgfqpoint{5.700000in}{5.700000in}}%
\pgfusepath{clip}%
\pgfsetbuttcap%
\pgfsetroundjoin%
\definecolor{currentfill}{rgb}{0.282327,0.094955,0.417331}%
\pgfsetfillcolor{currentfill}%
\pgfsetfillopacity{0.800000}%
\pgfsetlinewidth{0.000000pt}%
\definecolor{currentstroke}{rgb}{0.000000,0.000000,0.000000}%
\pgfsetstrokecolor{currentstroke}%
\pgfsetdash{}{0pt}%
\pgfpathmoveto{\pgfqpoint{3.863228in}{1.380706in}}%
\pgfpathlineto{\pgfqpoint{3.877325in}{1.382808in}}%
\pgfpathlineto{\pgfqpoint{3.891431in}{1.385089in}}%
\pgfpathlineto{\pgfqpoint{3.905547in}{1.387549in}}%
\pgfpathlineto{\pgfqpoint{3.919673in}{1.390189in}}%
\pgfpathlineto{\pgfqpoint{3.927908in}{1.403704in}}%
\pgfpathlineto{\pgfqpoint{3.936138in}{1.417319in}}%
\pgfpathlineto{\pgfqpoint{3.944363in}{1.431027in}}%
\pgfpathlineto{\pgfqpoint{3.952583in}{1.444824in}}%
\pgfpathlineto{\pgfqpoint{3.938462in}{1.441579in}}%
\pgfpathlineto{\pgfqpoint{3.924352in}{1.438514in}}%
\pgfpathlineto{\pgfqpoint{3.910252in}{1.435629in}}%
\pgfpathlineto{\pgfqpoint{3.896162in}{1.432924in}}%
\pgfpathlineto{\pgfqpoint{3.887936in}{1.419720in}}%
\pgfpathlineto{\pgfqpoint{3.879706in}{1.406611in}}%
\pgfpathlineto{\pgfqpoint{3.871470in}{1.393605in}}%
\pgfpathlineto{\pgfqpoint{3.863228in}{1.380706in}}%
\pgfpathclose%
\pgfusepath{fill}%
\end{pgfscope}%
\begin{pgfscope}%
\pgfpathrectangle{\pgfqpoint{1.150000in}{0.150000in}}{\pgfqpoint{5.700000in}{5.700000in}}%
\pgfusepath{clip}%
\pgfsetbuttcap%
\pgfsetroundjoin%
\definecolor{currentfill}{rgb}{0.180653,0.701402,0.488189}%
\pgfsetfillcolor{currentfill}%
\pgfsetfillopacity{0.800000}%
\pgfsetlinewidth{0.000000pt}%
\definecolor{currentstroke}{rgb}{0.000000,0.000000,0.000000}%
\pgfsetstrokecolor{currentstroke}%
\pgfsetdash{}{0pt}%
\pgfpathmoveto{\pgfqpoint{5.262847in}{3.075262in}}%
\pgfpathlineto{\pgfqpoint{5.277759in}{3.092112in}}%
\pgfpathlineto{\pgfqpoint{5.292693in}{3.109152in}}%
\pgfpathlineto{\pgfqpoint{5.307650in}{3.126382in}}%
\pgfpathlineto{\pgfqpoint{5.322629in}{3.143803in}}%
\pgfpathlineto{\pgfqpoint{5.330347in}{3.151737in}}%
\pgfpathlineto{\pgfqpoint{5.338056in}{3.159478in}}%
\pgfpathlineto{\pgfqpoint{5.345753in}{3.167025in}}%
\pgfpathlineto{\pgfqpoint{5.353441in}{3.174382in}}%
\pgfpathlineto{\pgfqpoint{5.338468in}{3.157089in}}%
\pgfpathlineto{\pgfqpoint{5.323518in}{3.139986in}}%
\pgfpathlineto{\pgfqpoint{5.308591in}{3.123072in}}%
\pgfpathlineto{\pgfqpoint{5.293686in}{3.106348in}}%
\pgfpathlineto{\pgfqpoint{5.285991in}{3.098851in}}%
\pgfpathlineto{\pgfqpoint{5.278286in}{3.091173in}}%
\pgfpathlineto{\pgfqpoint{5.270572in}{3.083310in}}%
\pgfpathlineto{\pgfqpoint{5.262847in}{3.075262in}}%
\pgfpathclose%
\pgfusepath{fill}%
\end{pgfscope}%
\begin{pgfscope}%
\pgfpathrectangle{\pgfqpoint{1.150000in}{0.150000in}}{\pgfqpoint{5.700000in}{5.700000in}}%
\pgfusepath{clip}%
\pgfsetbuttcap%
\pgfsetroundjoin%
\definecolor{currentfill}{rgb}{0.268510,0.009605,0.335427}%
\pgfsetfillcolor{currentfill}%
\pgfsetfillopacity{0.800000}%
\pgfsetlinewidth{0.000000pt}%
\definecolor{currentstroke}{rgb}{0.000000,0.000000,0.000000}%
\pgfsetstrokecolor{currentstroke}%
\pgfsetdash{}{0pt}%
\pgfpathmoveto{\pgfqpoint{3.359085in}{1.242632in}}%
\pgfpathlineto{\pgfqpoint{3.373119in}{1.236862in}}%
\pgfpathlineto{\pgfqpoint{3.387155in}{1.231280in}}%
\pgfpathlineto{\pgfqpoint{3.401195in}{1.225885in}}%
\pgfpathlineto{\pgfqpoint{3.415239in}{1.220677in}}%
\pgfpathlineto{\pgfqpoint{3.423721in}{1.226214in}}%
\pgfpathlineto{\pgfqpoint{3.432193in}{1.232033in}}%
\pgfpathlineto{\pgfqpoint{3.440653in}{1.238128in}}%
\pgfpathlineto{\pgfqpoint{3.449103in}{1.244490in}}%
\pgfpathlineto{\pgfqpoint{3.435086in}{1.248939in}}%
\pgfpathlineto{\pgfqpoint{3.421072in}{1.253575in}}%
\pgfpathlineto{\pgfqpoint{3.407063in}{1.258398in}}%
\pgfpathlineto{\pgfqpoint{3.393058in}{1.263409in}}%
\pgfpathlineto{\pgfqpoint{3.384582in}{1.257794in}}%
\pgfpathlineto{\pgfqpoint{3.376095in}{1.252454in}}%
\pgfpathlineto{\pgfqpoint{3.367596in}{1.247398in}}%
\pgfpathlineto{\pgfqpoint{3.359085in}{1.242632in}}%
\pgfpathclose%
\pgfusepath{fill}%
\end{pgfscope}%
\begin{pgfscope}%
\pgfpathrectangle{\pgfqpoint{1.150000in}{0.150000in}}{\pgfqpoint{5.700000in}{5.700000in}}%
\pgfusepath{clip}%
\pgfsetbuttcap%
\pgfsetroundjoin%
\definecolor{currentfill}{rgb}{0.141935,0.526453,0.555991}%
\pgfsetfillcolor{currentfill}%
\pgfsetfillopacity{0.800000}%
\pgfsetlinewidth{0.000000pt}%
\definecolor{currentstroke}{rgb}{0.000000,0.000000,0.000000}%
\pgfsetstrokecolor{currentstroke}%
\pgfsetdash{}{0pt}%
\pgfpathmoveto{\pgfqpoint{2.075369in}{2.656923in}}%
\pgfpathlineto{\pgfqpoint{2.089994in}{2.629262in}}%
\pgfpathlineto{\pgfqpoint{2.104600in}{2.601926in}}%
\pgfpathlineto{\pgfqpoint{2.119189in}{2.574912in}}%
\pgfpathlineto{\pgfqpoint{2.133761in}{2.548218in}}%
\pgfpathlineto{\pgfqpoint{2.143509in}{2.534100in}}%
\pgfpathlineto{\pgfqpoint{2.153223in}{2.520510in}}%
\pgfpathlineto{\pgfqpoint{2.162904in}{2.507438in}}%
\pgfpathlineto{\pgfqpoint{2.172552in}{2.494875in}}%
\pgfpathlineto{\pgfqpoint{2.158063in}{2.520680in}}%
\pgfpathlineto{\pgfqpoint{2.143557in}{2.546801in}}%
\pgfpathlineto{\pgfqpoint{2.129034in}{2.573242in}}%
\pgfpathlineto{\pgfqpoint{2.114494in}{2.600006in}}%
\pgfpathlineto{\pgfqpoint{2.104765in}{2.613445in}}%
\pgfpathlineto{\pgfqpoint{2.095001in}{2.627405in}}%
\pgfpathlineto{\pgfqpoint{2.085203in}{2.641894in}}%
\pgfpathlineto{\pgfqpoint{2.075369in}{2.656923in}}%
\pgfpathclose%
\pgfusepath{fill}%
\end{pgfscope}%
\begin{pgfscope}%
\pgfpathrectangle{\pgfqpoint{1.150000in}{0.150000in}}{\pgfqpoint{5.700000in}{5.700000in}}%
\pgfusepath{clip}%
\pgfsetbuttcap%
\pgfsetroundjoin%
\definecolor{currentfill}{rgb}{0.304148,0.764704,0.419943}%
\pgfsetfillcolor{currentfill}%
\pgfsetfillopacity{0.800000}%
\pgfsetlinewidth{0.000000pt}%
\definecolor{currentstroke}{rgb}{0.000000,0.000000,0.000000}%
\pgfsetstrokecolor{currentstroke}%
\pgfsetdash{}{0pt}%
\pgfpathmoveto{\pgfqpoint{5.474610in}{3.296227in}}%
\pgfpathlineto{\pgfqpoint{5.489680in}{3.314107in}}%
\pgfpathlineto{\pgfqpoint{5.504774in}{3.332178in}}%
\pgfpathlineto{\pgfqpoint{5.519891in}{3.350439in}}%
\pgfpathlineto{\pgfqpoint{5.535033in}{3.368892in}}%
\pgfpathlineto{\pgfqpoint{5.542603in}{3.374180in}}%
\pgfpathlineto{\pgfqpoint{5.550161in}{3.379281in}}%
\pgfpathlineto{\pgfqpoint{5.557707in}{3.384200in}}%
\pgfpathlineto{\pgfqpoint{5.565242in}{3.388939in}}%
\pgfpathlineto{\pgfqpoint{5.550113in}{3.370725in}}%
\pgfpathlineto{\pgfqpoint{5.535008in}{3.352701in}}%
\pgfpathlineto{\pgfqpoint{5.519927in}{3.334868in}}%
\pgfpathlineto{\pgfqpoint{5.504869in}{3.317225in}}%
\pgfpathlineto{\pgfqpoint{5.497321in}{3.312235in}}%
\pgfpathlineto{\pgfqpoint{5.489761in}{3.307074in}}%
\pgfpathlineto{\pgfqpoint{5.482191in}{3.301739in}}%
\pgfpathlineto{\pgfqpoint{5.474610in}{3.296227in}}%
\pgfpathclose%
\pgfusepath{fill}%
\end{pgfscope}%
\begin{pgfscope}%
\pgfpathrectangle{\pgfqpoint{1.150000in}{0.150000in}}{\pgfqpoint{5.700000in}{5.700000in}}%
\pgfusepath{clip}%
\pgfsetbuttcap%
\pgfsetroundjoin%
\definecolor{currentfill}{rgb}{0.506271,0.828786,0.300362}%
\pgfsetfillcolor{currentfill}%
\pgfsetfillopacity{0.800000}%
\pgfsetlinewidth{0.000000pt}%
\definecolor{currentstroke}{rgb}{0.000000,0.000000,0.000000}%
\pgfsetstrokecolor{currentstroke}%
\pgfsetdash{}{0pt}%
\pgfpathmoveto{\pgfqpoint{5.776281in}{3.578048in}}%
\pgfpathlineto{\pgfqpoint{5.791580in}{3.597043in}}%
\pgfpathlineto{\pgfqpoint{5.806905in}{3.616230in}}%
\pgfpathlineto{\pgfqpoint{5.822256in}{3.635609in}}%
\pgfpathlineto{\pgfqpoint{5.829587in}{3.637292in}}%
\pgfpathlineto{\pgfqpoint{5.836906in}{3.638820in}}%
\pgfpathlineto{\pgfqpoint{5.844213in}{3.640199in}}%
\pgfpathlineto{\pgfqpoint{5.851508in}{3.641433in}}%
\pgfpathlineto{\pgfqpoint{5.836180in}{3.622443in}}%
\pgfpathlineto{\pgfqpoint{5.820878in}{3.603644in}}%
\pgfpathlineto{\pgfqpoint{5.805601in}{3.585036in}}%
\pgfpathlineto{\pgfqpoint{5.798288in}{3.583501in}}%
\pgfpathlineto{\pgfqpoint{5.790964in}{3.581828in}}%
\pgfpathlineto{\pgfqpoint{5.783628in}{3.580012in}}%
\pgfpathlineto{\pgfqpoint{5.776281in}{3.578048in}}%
\pgfpathclose%
\pgfusepath{fill}%
\end{pgfscope}%
\begin{pgfscope}%
\pgfpathrectangle{\pgfqpoint{1.150000in}{0.150000in}}{\pgfqpoint{5.700000in}{5.700000in}}%
\pgfusepath{clip}%
\pgfsetbuttcap%
\pgfsetroundjoin%
\definecolor{currentfill}{rgb}{0.268510,0.009605,0.335427}%
\pgfsetfillcolor{currentfill}%
\pgfsetfillopacity{0.800000}%
\pgfsetlinewidth{0.000000pt}%
\definecolor{currentstroke}{rgb}{0.000000,0.000000,0.000000}%
\pgfsetstrokecolor{currentstroke}%
\pgfsetdash{}{0pt}%
\pgfpathmoveto{\pgfqpoint{3.505216in}{1.228551in}}%
\pgfpathlineto{\pgfqpoint{3.519256in}{1.225029in}}%
\pgfpathlineto{\pgfqpoint{3.533301in}{1.221690in}}%
\pgfpathlineto{\pgfqpoint{3.547351in}{1.218534in}}%
\pgfpathlineto{\pgfqpoint{3.561407in}{1.215562in}}%
\pgfpathlineto{\pgfqpoint{3.569801in}{1.223661in}}%
\pgfpathlineto{\pgfqpoint{3.578187in}{1.231996in}}%
\pgfpathlineto{\pgfqpoint{3.586564in}{1.240560in}}%
\pgfpathlineto{\pgfqpoint{3.594932in}{1.249345in}}%
\pgfpathlineto{\pgfqpoint{3.580896in}{1.251591in}}%
\pgfpathlineto{\pgfqpoint{3.566866in}{1.254020in}}%
\pgfpathlineto{\pgfqpoint{3.552842in}{1.256632in}}%
\pgfpathlineto{\pgfqpoint{3.538823in}{1.259429in}}%
\pgfpathlineto{\pgfqpoint{3.530435in}{1.251357in}}%
\pgfpathlineto{\pgfqpoint{3.522038in}{1.243516in}}%
\pgfpathlineto{\pgfqpoint{3.513632in}{1.235912in}}%
\pgfpathlineto{\pgfqpoint{3.505216in}{1.228551in}}%
\pgfpathclose%
\pgfusepath{fill}%
\end{pgfscope}%
\begin{pgfscope}%
\pgfpathrectangle{\pgfqpoint{1.150000in}{0.150000in}}{\pgfqpoint{5.700000in}{5.700000in}}%
\pgfusepath{clip}%
\pgfsetbuttcap%
\pgfsetroundjoin%
\definecolor{currentfill}{rgb}{0.273809,0.031497,0.358853}%
\pgfsetfillcolor{currentfill}%
\pgfsetfillopacity{0.800000}%
\pgfsetlinewidth{0.000000pt}%
\definecolor{currentstroke}{rgb}{0.000000,0.000000,0.000000}%
\pgfsetstrokecolor{currentstroke}%
\pgfsetdash{}{0pt}%
\pgfpathmoveto{\pgfqpoint{3.212484in}{1.285991in}}%
\pgfpathlineto{\pgfqpoint{3.226532in}{1.277900in}}%
\pgfpathlineto{\pgfqpoint{3.240582in}{1.270003in}}%
\pgfpathlineto{\pgfqpoint{3.254634in}{1.262298in}}%
\pgfpathlineto{\pgfqpoint{3.268687in}{1.254785in}}%
\pgfpathlineto{\pgfqpoint{3.277281in}{1.257501in}}%
\pgfpathlineto{\pgfqpoint{3.285861in}{1.260548in}}%
\pgfpathlineto{\pgfqpoint{3.294427in}{1.263919in}}%
\pgfpathlineto{\pgfqpoint{3.302981in}{1.267605in}}%
\pgfpathlineto{\pgfqpoint{3.288961in}{1.274324in}}%
\pgfpathlineto{\pgfqpoint{3.274943in}{1.281234in}}%
\pgfpathlineto{\pgfqpoint{3.260928in}{1.288337in}}%
\pgfpathlineto{\pgfqpoint{3.246914in}{1.295633in}}%
\pgfpathlineto{\pgfqpoint{3.238328in}{1.292729in}}%
\pgfpathlineto{\pgfqpoint{3.229727in}{1.290148in}}%
\pgfpathlineto{\pgfqpoint{3.221113in}{1.287899in}}%
\pgfpathlineto{\pgfqpoint{3.212484in}{1.285991in}}%
\pgfpathclose%
\pgfusepath{fill}%
\end{pgfscope}%
\begin{pgfscope}%
\pgfpathrectangle{\pgfqpoint{1.150000in}{0.150000in}}{\pgfqpoint{5.700000in}{5.700000in}}%
\pgfusepath{clip}%
\pgfsetbuttcap%
\pgfsetroundjoin%
\definecolor{currentfill}{rgb}{0.283072,0.130895,0.449241}%
\pgfsetfillcolor{currentfill}%
\pgfsetfillopacity{0.800000}%
\pgfsetlinewidth{0.000000pt}%
\definecolor{currentstroke}{rgb}{0.000000,0.000000,0.000000}%
\pgfsetstrokecolor{currentstroke}%
\pgfsetdash{}{0pt}%
\pgfpathmoveto{\pgfqpoint{3.952583in}{1.444824in}}%
\pgfpathlineto{\pgfqpoint{3.966713in}{1.448248in}}%
\pgfpathlineto{\pgfqpoint{3.980854in}{1.451852in}}%
\pgfpathlineto{\pgfqpoint{3.995006in}{1.455635in}}%
\pgfpathlineto{\pgfqpoint{4.009168in}{1.459597in}}%
\pgfpathlineto{\pgfqpoint{4.017380in}{1.474063in}}%
\pgfpathlineto{\pgfqpoint{4.025587in}{1.488598in}}%
\pgfpathlineto{\pgfqpoint{4.033789in}{1.503197in}}%
\pgfpathlineto{\pgfqpoint{4.041987in}{1.517855in}}%
\pgfpathlineto{\pgfqpoint{4.027827in}{1.513317in}}%
\pgfpathlineto{\pgfqpoint{4.013679in}{1.508959in}}%
\pgfpathlineto{\pgfqpoint{3.999541in}{1.504781in}}%
\pgfpathlineto{\pgfqpoint{3.985415in}{1.500782in}}%
\pgfpathlineto{\pgfqpoint{3.977214in}{1.486688in}}%
\pgfpathlineto{\pgfqpoint{3.969008in}{1.472660in}}%
\pgfpathlineto{\pgfqpoint{3.960798in}{1.458703in}}%
\pgfpathlineto{\pgfqpoint{3.952583in}{1.444824in}}%
\pgfpathclose%
\pgfusepath{fill}%
\end{pgfscope}%
\begin{pgfscope}%
\pgfpathrectangle{\pgfqpoint{1.150000in}{0.150000in}}{\pgfqpoint{5.700000in}{5.700000in}}%
\pgfusepath{clip}%
\pgfsetbuttcap%
\pgfsetroundjoin%
\definecolor{currentfill}{rgb}{0.137339,0.662252,0.515571}%
\pgfsetfillcolor{currentfill}%
\pgfsetfillopacity{0.800000}%
\pgfsetlinewidth{0.000000pt}%
\definecolor{currentstroke}{rgb}{0.000000,0.000000,0.000000}%
\pgfsetstrokecolor{currentstroke}%
\pgfsetdash{}{0pt}%
\pgfpathmoveto{\pgfqpoint{5.141243in}{2.937941in}}%
\pgfpathlineto{\pgfqpoint{5.156074in}{2.954178in}}%
\pgfpathlineto{\pgfqpoint{5.170928in}{2.970605in}}%
\pgfpathlineto{\pgfqpoint{5.185803in}{2.987221in}}%
\pgfpathlineto{\pgfqpoint{5.200700in}{3.004027in}}%
\pgfpathlineto{\pgfqpoint{5.208502in}{3.013609in}}%
\pgfpathlineto{\pgfqpoint{5.216295in}{3.022995in}}%
\pgfpathlineto{\pgfqpoint{5.224078in}{3.032186in}}%
\pgfpathlineto{\pgfqpoint{5.231851in}{3.041184in}}%
\pgfpathlineto{\pgfqpoint{5.216957in}{3.024433in}}%
\pgfpathlineto{\pgfqpoint{5.202085in}{3.007871in}}%
\pgfpathlineto{\pgfqpoint{5.187235in}{2.991498in}}%
\pgfpathlineto{\pgfqpoint{5.172406in}{2.975315in}}%
\pgfpathlineto{\pgfqpoint{5.164629in}{2.966250in}}%
\pgfpathlineto{\pgfqpoint{5.156843in}{2.957000in}}%
\pgfpathlineto{\pgfqpoint{5.149047in}{2.947564in}}%
\pgfpathlineto{\pgfqpoint{5.141243in}{2.937941in}}%
\pgfpathclose%
\pgfusepath{fill}%
\end{pgfscope}%
\begin{pgfscope}%
\pgfpathrectangle{\pgfqpoint{1.150000in}{0.150000in}}{\pgfqpoint{5.700000in}{5.700000in}}%
\pgfusepath{clip}%
\pgfsetbuttcap%
\pgfsetroundjoin%
\definecolor{currentfill}{rgb}{0.244972,0.287675,0.537260}%
\pgfsetfillcolor{currentfill}%
\pgfsetfillopacity{0.800000}%
\pgfsetlinewidth{0.000000pt}%
\definecolor{currentstroke}{rgb}{0.000000,0.000000,0.000000}%
\pgfsetstrokecolor{currentstroke}%
\pgfsetdash{}{0pt}%
\pgfpathmoveto{\pgfqpoint{4.286307in}{1.814949in}}%
\pgfpathlineto{\pgfqpoint{4.300585in}{1.823104in}}%
\pgfpathlineto{\pgfqpoint{4.314877in}{1.831440in}}%
\pgfpathlineto{\pgfqpoint{4.329183in}{1.839958in}}%
\pgfpathlineto{\pgfqpoint{4.343504in}{1.848657in}}%
\pgfpathlineto{\pgfqpoint{4.351638in}{1.864858in}}%
\pgfpathlineto{\pgfqpoint{4.359768in}{1.881009in}}%
\pgfpathlineto{\pgfqpoint{4.367894in}{1.897108in}}%
\pgfpathlineto{\pgfqpoint{4.376016in}{1.913151in}}%
\pgfpathlineto{\pgfqpoint{4.361690in}{1.904029in}}%
\pgfpathlineto{\pgfqpoint{4.347380in}{1.895089in}}%
\pgfpathlineto{\pgfqpoint{4.333084in}{1.886331in}}%
\pgfpathlineto{\pgfqpoint{4.318803in}{1.877755in}}%
\pgfpathlineto{\pgfqpoint{4.310685in}{1.862122in}}%
\pgfpathlineto{\pgfqpoint{4.302563in}{1.846441in}}%
\pgfpathlineto{\pgfqpoint{4.294437in}{1.830715in}}%
\pgfpathlineto{\pgfqpoint{4.286307in}{1.814949in}}%
\pgfpathclose%
\pgfusepath{fill}%
\end{pgfscope}%
\begin{pgfscope}%
\pgfpathrectangle{\pgfqpoint{1.150000in}{0.150000in}}{\pgfqpoint{5.700000in}{5.700000in}}%
\pgfusepath{clip}%
\pgfsetbuttcap%
\pgfsetroundjoin%
\definecolor{currentfill}{rgb}{0.216210,0.351535,0.550627}%
\pgfsetfillcolor{currentfill}%
\pgfsetfillopacity{0.800000}%
\pgfsetlinewidth{0.000000pt}%
\definecolor{currentstroke}{rgb}{0.000000,0.000000,0.000000}%
\pgfsetstrokecolor{currentstroke}%
\pgfsetdash{}{0pt}%
\pgfpathmoveto{\pgfqpoint{4.408461in}{1.976683in}}%
\pgfpathlineto{\pgfqpoint{4.422806in}{1.986377in}}%
\pgfpathlineto{\pgfqpoint{4.437166in}{1.996254in}}%
\pgfpathlineto{\pgfqpoint{4.451543in}{2.006313in}}%
\pgfpathlineto{\pgfqpoint{4.465935in}{2.016554in}}%
\pgfpathlineto{\pgfqpoint{4.474041in}{2.032631in}}%
\pgfpathlineto{\pgfqpoint{4.482142in}{2.048622in}}%
\pgfpathlineto{\pgfqpoint{4.490238in}{2.064525in}}%
\pgfpathlineto{\pgfqpoint{4.498331in}{2.080335in}}%
\pgfpathlineto{\pgfqpoint{4.483933in}{2.069735in}}%
\pgfpathlineto{\pgfqpoint{4.469552in}{2.059317in}}%
\pgfpathlineto{\pgfqpoint{4.455186in}{2.049083in}}%
\pgfpathlineto{\pgfqpoint{4.440836in}{2.039032in}}%
\pgfpathlineto{\pgfqpoint{4.432749in}{2.023567in}}%
\pgfpathlineto{\pgfqpoint{4.424657in}{2.008019in}}%
\pgfpathlineto{\pgfqpoint{4.416561in}{1.992390in}}%
\pgfpathlineto{\pgfqpoint{4.408461in}{1.976683in}}%
\pgfpathclose%
\pgfusepath{fill}%
\end{pgfscope}%
\begin{pgfscope}%
\pgfpathrectangle{\pgfqpoint{1.150000in}{0.150000in}}{\pgfqpoint{5.700000in}{5.700000in}}%
\pgfusepath{clip}%
\pgfsetbuttcap%
\pgfsetroundjoin%
\definecolor{currentfill}{rgb}{0.266580,0.228262,0.514349}%
\pgfsetfillcolor{currentfill}%
\pgfsetfillopacity{0.800000}%
\pgfsetlinewidth{0.000000pt}%
\definecolor{currentstroke}{rgb}{0.000000,0.000000,0.000000}%
\pgfsetstrokecolor{currentstroke}%
\pgfsetdash{}{0pt}%
\pgfpathmoveto{\pgfqpoint{4.164168in}{1.660776in}}%
\pgfpathlineto{\pgfqpoint{4.178387in}{1.667272in}}%
\pgfpathlineto{\pgfqpoint{4.192619in}{1.673948in}}%
\pgfpathlineto{\pgfqpoint{4.206864in}{1.680804in}}%
\pgfpathlineto{\pgfqpoint{4.221122in}{1.687841in}}%
\pgfpathlineto{\pgfqpoint{4.229284in}{1.703784in}}%
\pgfpathlineto{\pgfqpoint{4.237442in}{1.719721in}}%
\pgfpathlineto{\pgfqpoint{4.245596in}{1.735646in}}%
\pgfpathlineto{\pgfqpoint{4.253746in}{1.751556in}}%
\pgfpathlineto{\pgfqpoint{4.239485in}{1.744034in}}%
\pgfpathlineto{\pgfqpoint{4.225238in}{1.736693in}}%
\pgfpathlineto{\pgfqpoint{4.211004in}{1.729533in}}%
\pgfpathlineto{\pgfqpoint{4.196783in}{1.722554in}}%
\pgfpathlineto{\pgfqpoint{4.188636in}{1.707117in}}%
\pgfpathlineto{\pgfqpoint{4.180484in}{1.691672in}}%
\pgfpathlineto{\pgfqpoint{4.172328in}{1.676223in}}%
\pgfpathlineto{\pgfqpoint{4.164168in}{1.660776in}}%
\pgfpathclose%
\pgfusepath{fill}%
\end{pgfscope}%
\begin{pgfscope}%
\pgfpathrectangle{\pgfqpoint{1.150000in}{0.150000in}}{\pgfqpoint{5.700000in}{5.700000in}}%
\pgfusepath{clip}%
\pgfsetbuttcap%
\pgfsetroundjoin%
\definecolor{currentfill}{rgb}{0.134692,0.658636,0.517649}%
\pgfsetfillcolor{currentfill}%
\pgfsetfillopacity{0.800000}%
\pgfsetlinewidth{0.000000pt}%
\definecolor{currentstroke}{rgb}{0.000000,0.000000,0.000000}%
\pgfsetstrokecolor{currentstroke}%
\pgfsetdash{}{0pt}%
\pgfpathmoveto{\pgfqpoint{1.879214in}{3.074861in}}%
\pgfpathlineto{\pgfqpoint{1.894081in}{3.042468in}}%
\pgfpathlineto{\pgfqpoint{1.908924in}{3.010454in}}%
\pgfpathlineto{\pgfqpoint{1.923744in}{2.978816in}}%
\pgfpathlineto{\pgfqpoint{1.938542in}{2.947551in}}%
\pgfpathlineto{\pgfqpoint{1.948497in}{2.931966in}}%
\pgfpathlineto{\pgfqpoint{1.958415in}{2.916918in}}%
\pgfpathlineto{\pgfqpoint{1.968297in}{2.902396in}}%
\pgfpathlineto{\pgfqpoint{1.978144in}{2.888393in}}%
\pgfpathlineto{\pgfqpoint{1.963436in}{2.918773in}}%
\pgfpathlineto{\pgfqpoint{1.948706in}{2.949521in}}%
\pgfpathlineto{\pgfqpoint{1.933954in}{2.980643in}}%
\pgfpathlineto{\pgfqpoint{1.919179in}{3.012141in}}%
\pgfpathlineto{\pgfqpoint{1.909244in}{3.027017in}}%
\pgfpathlineto{\pgfqpoint{1.899271in}{3.042423in}}%
\pgfpathlineto{\pgfqpoint{1.889262in}{3.058368in}}%
\pgfpathlineto{\pgfqpoint{1.879214in}{3.074861in}}%
\pgfpathclose%
\pgfusepath{fill}%
\end{pgfscope}%
\begin{pgfscope}%
\pgfpathrectangle{\pgfqpoint{1.150000in}{0.150000in}}{\pgfqpoint{5.700000in}{5.700000in}}%
\pgfusepath{clip}%
\pgfsetbuttcap%
\pgfsetroundjoin%
\definecolor{currentfill}{rgb}{0.190631,0.407061,0.556089}%
\pgfsetfillcolor{currentfill}%
\pgfsetfillopacity{0.800000}%
\pgfsetlinewidth{0.000000pt}%
\definecolor{currentstroke}{rgb}{0.000000,0.000000,0.000000}%
\pgfsetstrokecolor{currentstroke}%
\pgfsetdash{}{0pt}%
\pgfpathmoveto{\pgfqpoint{4.530653in}{2.142597in}}%
\pgfpathlineto{\pgfqpoint{4.545072in}{2.153706in}}%
\pgfpathlineto{\pgfqpoint{4.559508in}{2.165000in}}%
\pgfpathlineto{\pgfqpoint{4.573961in}{2.176477in}}%
\pgfpathlineto{\pgfqpoint{4.588431in}{2.188138in}}%
\pgfpathlineto{\pgfqpoint{4.596505in}{2.203750in}}%
\pgfpathlineto{\pgfqpoint{4.604574in}{2.219245in}}%
\pgfpathlineto{\pgfqpoint{4.612638in}{2.234619in}}%
\pgfpathlineto{\pgfqpoint{4.620696in}{2.249871in}}%
\pgfpathlineto{\pgfqpoint{4.606221in}{2.237917in}}%
\pgfpathlineto{\pgfqpoint{4.591763in}{2.226146in}}%
\pgfpathlineto{\pgfqpoint{4.577321in}{2.214560in}}%
\pgfpathlineto{\pgfqpoint{4.562897in}{2.203158in}}%
\pgfpathlineto{\pgfqpoint{4.554844in}{2.188186in}}%
\pgfpathlineto{\pgfqpoint{4.546785in}{2.173100in}}%
\pgfpathlineto{\pgfqpoint{4.538721in}{2.157903in}}%
\pgfpathlineto{\pgfqpoint{4.530653in}{2.142597in}}%
\pgfpathclose%
\pgfusepath{fill}%
\end{pgfscope}%
\begin{pgfscope}%
\pgfpathrectangle{\pgfqpoint{1.150000in}{0.150000in}}{\pgfqpoint{5.700000in}{5.700000in}}%
\pgfusepath{clip}%
\pgfsetbuttcap%
\pgfsetroundjoin%
\definecolor{currentfill}{rgb}{0.119483,0.614817,0.537692}%
\pgfsetfillcolor{currentfill}%
\pgfsetfillopacity{0.800000}%
\pgfsetlinewidth{0.000000pt}%
\definecolor{currentstroke}{rgb}{0.000000,0.000000,0.000000}%
\pgfsetstrokecolor{currentstroke}%
\pgfsetdash{}{0pt}%
\pgfpathmoveto{\pgfqpoint{5.019362in}{2.791083in}}%
\pgfpathlineto{\pgfqpoint{5.034111in}{2.806566in}}%
\pgfpathlineto{\pgfqpoint{5.048880in}{2.822237in}}%
\pgfpathlineto{\pgfqpoint{5.063671in}{2.838097in}}%
\pgfpathlineto{\pgfqpoint{5.078483in}{2.854146in}}%
\pgfpathlineto{\pgfqpoint{5.086359in}{2.865287in}}%
\pgfpathlineto{\pgfqpoint{5.094226in}{2.876238in}}%
\pgfpathlineto{\pgfqpoint{5.102084in}{2.886997in}}%
\pgfpathlineto{\pgfqpoint{5.109934in}{2.897565in}}%
\pgfpathlineto{\pgfqpoint{5.095122in}{2.881499in}}%
\pgfpathlineto{\pgfqpoint{5.080331in}{2.865621in}}%
\pgfpathlineto{\pgfqpoint{5.065562in}{2.849932in}}%
\pgfpathlineto{\pgfqpoint{5.050813in}{2.834431in}}%
\pgfpathlineto{\pgfqpoint{5.042963in}{2.823867in}}%
\pgfpathlineto{\pgfqpoint{5.035104in}{2.813121in}}%
\pgfpathlineto{\pgfqpoint{5.027237in}{2.802193in}}%
\pgfpathlineto{\pgfqpoint{5.019362in}{2.791083in}}%
\pgfpathclose%
\pgfusepath{fill}%
\end{pgfscope}%
\begin{pgfscope}%
\pgfpathrectangle{\pgfqpoint{1.150000in}{0.150000in}}{\pgfqpoint{5.700000in}{5.700000in}}%
\pgfusepath{clip}%
\pgfsetbuttcap%
\pgfsetroundjoin%
\definecolor{currentfill}{rgb}{0.166617,0.463708,0.558119}%
\pgfsetfillcolor{currentfill}%
\pgfsetfillopacity{0.800000}%
\pgfsetlinewidth{0.000000pt}%
\definecolor{currentstroke}{rgb}{0.000000,0.000000,0.000000}%
\pgfsetstrokecolor{currentstroke}%
\pgfsetdash{}{0pt}%
\pgfpathmoveto{\pgfqpoint{4.652878in}{2.309609in}}%
\pgfpathlineto{\pgfqpoint{4.667376in}{2.322008in}}%
\pgfpathlineto{\pgfqpoint{4.681893in}{2.334593in}}%
\pgfpathlineto{\pgfqpoint{4.696427in}{2.347363in}}%
\pgfpathlineto{\pgfqpoint{4.710979in}{2.360318in}}%
\pgfpathlineto{\pgfqpoint{4.719016in}{2.375163in}}%
\pgfpathlineto{\pgfqpoint{4.727047in}{2.389864in}}%
\pgfpathlineto{\pgfqpoint{4.735072in}{2.404419in}}%
\pgfpathlineto{\pgfqpoint{4.743092in}{2.418827in}}%
\pgfpathlineto{\pgfqpoint{4.728534in}{2.405645in}}%
\pgfpathlineto{\pgfqpoint{4.713995in}{2.392648in}}%
\pgfpathlineto{\pgfqpoint{4.699474in}{2.379837in}}%
\pgfpathlineto{\pgfqpoint{4.684971in}{2.367211in}}%
\pgfpathlineto{\pgfqpoint{4.676956in}{2.353018in}}%
\pgfpathlineto{\pgfqpoint{4.668936in}{2.338685in}}%
\pgfpathlineto{\pgfqpoint{4.660910in}{2.324214in}}%
\pgfpathlineto{\pgfqpoint{4.652878in}{2.309609in}}%
\pgfpathclose%
\pgfusepath{fill}%
\end{pgfscope}%
\begin{pgfscope}%
\pgfpathrectangle{\pgfqpoint{1.150000in}{0.150000in}}{\pgfqpoint{5.700000in}{5.700000in}}%
\pgfusepath{clip}%
\pgfsetbuttcap%
\pgfsetroundjoin%
\definecolor{currentfill}{rgb}{0.280255,0.165693,0.476498}%
\pgfsetfillcolor{currentfill}%
\pgfsetfillopacity{0.800000}%
\pgfsetlinewidth{0.000000pt}%
\definecolor{currentstroke}{rgb}{0.000000,0.000000,0.000000}%
\pgfsetstrokecolor{currentstroke}%
\pgfsetdash{}{0pt}%
\pgfpathmoveto{\pgfqpoint{4.041987in}{1.517855in}}%
\pgfpathlineto{\pgfqpoint{4.056158in}{1.522572in}}%
\pgfpathlineto{\pgfqpoint{4.070341in}{1.527468in}}%
\pgfpathlineto{\pgfqpoint{4.084535in}{1.532544in}}%
\pgfpathlineto{\pgfqpoint{4.098741in}{1.537799in}}%
\pgfpathlineto{\pgfqpoint{4.106934in}{1.553067in}}%
\pgfpathlineto{\pgfqpoint{4.115122in}{1.568374in}}%
\pgfpathlineto{\pgfqpoint{4.123307in}{1.583717in}}%
\pgfpathlineto{\pgfqpoint{4.131487in}{1.599089in}}%
\pgfpathlineto{\pgfqpoint{4.117281in}{1.593287in}}%
\pgfpathlineto{\pgfqpoint{4.103087in}{1.587666in}}%
\pgfpathlineto{\pgfqpoint{4.088906in}{1.582224in}}%
\pgfpathlineto{\pgfqpoint{4.074736in}{1.576963in}}%
\pgfpathlineto{\pgfqpoint{4.066555in}{1.562124in}}%
\pgfpathlineto{\pgfqpoint{4.058370in}{1.547323in}}%
\pgfpathlineto{\pgfqpoint{4.050181in}{1.532565in}}%
\pgfpathlineto{\pgfqpoint{4.041987in}{1.517855in}}%
\pgfpathclose%
\pgfusepath{fill}%
\end{pgfscope}%
\begin{pgfscope}%
\pgfpathrectangle{\pgfqpoint{1.150000in}{0.150000in}}{\pgfqpoint{5.700000in}{5.700000in}}%
\pgfusepath{clip}%
\pgfsetbuttcap%
\pgfsetroundjoin%
\definecolor{currentfill}{rgb}{0.268510,0.009605,0.335427}%
\pgfsetfillcolor{currentfill}%
\pgfsetfillopacity{0.800000}%
\pgfsetlinewidth{0.000000pt}%
\definecolor{currentstroke}{rgb}{0.000000,0.000000,0.000000}%
\pgfsetstrokecolor{currentstroke}%
\pgfsetdash{}{0pt}%
\pgfpathmoveto{\pgfqpoint{3.415239in}{1.220677in}}%
\pgfpathlineto{\pgfqpoint{3.429286in}{1.215655in}}%
\pgfpathlineto{\pgfqpoint{3.443337in}{1.210819in}}%
\pgfpathlineto{\pgfqpoint{3.457393in}{1.206168in}}%
\pgfpathlineto{\pgfqpoint{3.471452in}{1.201702in}}%
\pgfpathlineto{\pgfqpoint{3.479908in}{1.208011in}}%
\pgfpathlineto{\pgfqpoint{3.488354in}{1.214594in}}%
\pgfpathlineto{\pgfqpoint{3.496790in}{1.221443in}}%
\pgfpathlineto{\pgfqpoint{3.505216in}{1.228551in}}%
\pgfpathlineto{\pgfqpoint{3.491181in}{1.232258in}}%
\pgfpathlineto{\pgfqpoint{3.477150in}{1.236150in}}%
\pgfpathlineto{\pgfqpoint{3.463124in}{1.240227in}}%
\pgfpathlineto{\pgfqpoint{3.449103in}{1.244490in}}%
\pgfpathlineto{\pgfqpoint{3.440653in}{1.238128in}}%
\pgfpathlineto{\pgfqpoint{3.432193in}{1.232033in}}%
\pgfpathlineto{\pgfqpoint{3.423721in}{1.226214in}}%
\pgfpathlineto{\pgfqpoint{3.415239in}{1.220677in}}%
\pgfpathclose%
\pgfusepath{fill}%
\end{pgfscope}%
\begin{pgfscope}%
\pgfpathrectangle{\pgfqpoint{1.150000in}{0.150000in}}{\pgfqpoint{5.700000in}{5.700000in}}%
\pgfusepath{clip}%
\pgfsetbuttcap%
\pgfsetroundjoin%
\definecolor{currentfill}{rgb}{0.127568,0.566949,0.550556}%
\pgfsetfillcolor{currentfill}%
\pgfsetfillopacity{0.800000}%
\pgfsetlinewidth{0.000000pt}%
\definecolor{currentstroke}{rgb}{0.000000,0.000000,0.000000}%
\pgfsetstrokecolor{currentstroke}%
\pgfsetdash{}{0pt}%
\pgfpathmoveto{\pgfqpoint{4.897290in}{2.636150in}}%
\pgfpathlineto{\pgfqpoint{4.911955in}{2.650740in}}%
\pgfpathlineto{\pgfqpoint{4.926640in}{2.665517in}}%
\pgfpathlineto{\pgfqpoint{4.941345in}{2.680482in}}%
\pgfpathlineto{\pgfqpoint{4.956070in}{2.695634in}}%
\pgfpathlineto{\pgfqpoint{4.964009in}{2.708202in}}%
\pgfpathlineto{\pgfqpoint{4.971940in}{2.720588in}}%
\pgfpathlineto{\pgfqpoint{4.979863in}{2.732793in}}%
\pgfpathlineto{\pgfqpoint{4.987779in}{2.744816in}}%
\pgfpathlineto{\pgfqpoint{4.973052in}{2.729575in}}%
\pgfpathlineto{\pgfqpoint{4.958345in}{2.714521in}}%
\pgfpathlineto{\pgfqpoint{4.943658in}{2.699656in}}%
\pgfpathlineto{\pgfqpoint{4.928991in}{2.684977in}}%
\pgfpathlineto{\pgfqpoint{4.921077in}{2.673030in}}%
\pgfpathlineto{\pgfqpoint{4.913156in}{2.660909in}}%
\pgfpathlineto{\pgfqpoint{4.905227in}{2.648616in}}%
\pgfpathlineto{\pgfqpoint{4.897290in}{2.636150in}}%
\pgfpathclose%
\pgfusepath{fill}%
\end{pgfscope}%
\begin{pgfscope}%
\pgfpathrectangle{\pgfqpoint{1.150000in}{0.150000in}}{\pgfqpoint{5.700000in}{5.700000in}}%
\pgfusepath{clip}%
\pgfsetbuttcap%
\pgfsetroundjoin%
\definecolor{currentfill}{rgb}{0.127568,0.566949,0.550556}%
\pgfsetfillcolor{currentfill}%
\pgfsetfillopacity{0.800000}%
\pgfsetlinewidth{0.000000pt}%
\definecolor{currentstroke}{rgb}{0.000000,0.000000,0.000000}%
\pgfsetstrokecolor{currentstroke}%
\pgfsetdash{}{0pt}%
\pgfpathmoveto{\pgfqpoint{2.016685in}{2.770886in}}%
\pgfpathlineto{\pgfqpoint{2.031385in}{2.741891in}}%
\pgfpathlineto{\pgfqpoint{2.046065in}{2.713234in}}%
\pgfpathlineto{\pgfqpoint{2.060727in}{2.684912in}}%
\pgfpathlineto{\pgfqpoint{2.075369in}{2.656923in}}%
\pgfpathlineto{\pgfqpoint{2.085203in}{2.641894in}}%
\pgfpathlineto{\pgfqpoint{2.095001in}{2.627405in}}%
\pgfpathlineto{\pgfqpoint{2.104765in}{2.613445in}}%
\pgfpathlineto{\pgfqpoint{2.114494in}{2.600006in}}%
\pgfpathlineto{\pgfqpoint{2.099937in}{2.627097in}}%
\pgfpathlineto{\pgfqpoint{2.085362in}{2.654516in}}%
\pgfpathlineto{\pgfqpoint{2.070768in}{2.682268in}}%
\pgfpathlineto{\pgfqpoint{2.056156in}{2.710355in}}%
\pgfpathlineto{\pgfqpoint{2.046342in}{2.724680in}}%
\pgfpathlineto{\pgfqpoint{2.036493in}{2.739537in}}%
\pgfpathlineto{\pgfqpoint{2.026607in}{2.754936in}}%
\pgfpathlineto{\pgfqpoint{2.016685in}{2.770886in}}%
\pgfpathclose%
\pgfusepath{fill}%
\end{pgfscope}%
\begin{pgfscope}%
\pgfpathrectangle{\pgfqpoint{1.150000in}{0.150000in}}{\pgfqpoint{5.700000in}{5.700000in}}%
\pgfusepath{clip}%
\pgfsetbuttcap%
\pgfsetroundjoin%
\definecolor{currentfill}{rgb}{0.146180,0.515413,0.556823}%
\pgfsetfillcolor{currentfill}%
\pgfsetfillopacity{0.800000}%
\pgfsetlinewidth{0.000000pt}%
\definecolor{currentstroke}{rgb}{0.000000,0.000000,0.000000}%
\pgfsetstrokecolor{currentstroke}%
\pgfsetdash{}{0pt}%
\pgfpathmoveto{\pgfqpoint{4.775107in}{2.474946in}}%
\pgfpathlineto{\pgfqpoint{4.789688in}{2.488507in}}%
\pgfpathlineto{\pgfqpoint{4.804288in}{2.502254in}}%
\pgfpathlineto{\pgfqpoint{4.818906in}{2.516188in}}%
\pgfpathlineto{\pgfqpoint{4.833544in}{2.530308in}}%
\pgfpathlineto{\pgfqpoint{4.841537in}{2.544124in}}%
\pgfpathlineto{\pgfqpoint{4.849522in}{2.557775in}}%
\pgfpathlineto{\pgfqpoint{4.857501in}{2.571260in}}%
\pgfpathlineto{\pgfqpoint{4.865473in}{2.584577in}}%
\pgfpathlineto{\pgfqpoint{4.850831in}{2.570298in}}%
\pgfpathlineto{\pgfqpoint{4.836208in}{2.556206in}}%
\pgfpathlineto{\pgfqpoint{4.821605in}{2.542301in}}%
\pgfpathlineto{\pgfqpoint{4.807021in}{2.528582in}}%
\pgfpathlineto{\pgfqpoint{4.799052in}{2.515410in}}%
\pgfpathlineto{\pgfqpoint{4.791077in}{2.502080in}}%
\pgfpathlineto{\pgfqpoint{4.783095in}{2.488591in}}%
\pgfpathlineto{\pgfqpoint{4.775107in}{2.474946in}}%
\pgfpathclose%
\pgfusepath{fill}%
\end{pgfscope}%
\begin{pgfscope}%
\pgfpathrectangle{\pgfqpoint{1.150000in}{0.150000in}}{\pgfqpoint{5.700000in}{5.700000in}}%
\pgfusepath{clip}%
\pgfsetbuttcap%
\pgfsetroundjoin%
\definecolor{currentfill}{rgb}{0.239374,0.735588,0.455688}%
\pgfsetfillcolor{currentfill}%
\pgfsetfillopacity{0.800000}%
\pgfsetlinewidth{0.000000pt}%
\definecolor{currentstroke}{rgb}{0.000000,0.000000,0.000000}%
\pgfsetstrokecolor{currentstroke}%
\pgfsetdash{}{0pt}%
\pgfpathmoveto{\pgfqpoint{5.353441in}{3.174382in}}%
\pgfpathlineto{\pgfqpoint{5.368437in}{3.191865in}}%
\pgfpathlineto{\pgfqpoint{5.383455in}{3.209539in}}%
\pgfpathlineto{\pgfqpoint{5.398498in}{3.227403in}}%
\pgfpathlineto{\pgfqpoint{5.413563in}{3.245459in}}%
\pgfpathlineto{\pgfqpoint{5.421232in}{3.252475in}}%
\pgfpathlineto{\pgfqpoint{5.428891in}{3.259296in}}%
\pgfpathlineto{\pgfqpoint{5.436538in}{3.265923in}}%
\pgfpathlineto{\pgfqpoint{5.444174in}{3.272358in}}%
\pgfpathlineto{\pgfqpoint{5.429117in}{3.254467in}}%
\pgfpathlineto{\pgfqpoint{5.414084in}{3.236768in}}%
\pgfpathlineto{\pgfqpoint{5.399073in}{3.219258in}}%
\pgfpathlineto{\pgfqpoint{5.384086in}{3.201939in}}%
\pgfpathlineto{\pgfqpoint{5.376440in}{3.195326in}}%
\pgfpathlineto{\pgfqpoint{5.368784in}{3.188530in}}%
\pgfpathlineto{\pgfqpoint{5.361118in}{3.181549in}}%
\pgfpathlineto{\pgfqpoint{5.353441in}{3.174382in}}%
\pgfpathclose%
\pgfusepath{fill}%
\end{pgfscope}%
\begin{pgfscope}%
\pgfpathrectangle{\pgfqpoint{1.150000in}{0.150000in}}{\pgfqpoint{5.700000in}{5.700000in}}%
\pgfusepath{clip}%
\pgfsetbuttcap%
\pgfsetroundjoin%
\definecolor{currentfill}{rgb}{0.277018,0.050344,0.375715}%
\pgfsetfillcolor{currentfill}%
\pgfsetfillopacity{0.800000}%
\pgfsetlinewidth{0.000000pt}%
\definecolor{currentstroke}{rgb}{0.000000,0.000000,0.000000}%
\pgfsetstrokecolor{currentstroke}%
\pgfsetdash{}{0pt}%
\pgfpathmoveto{\pgfqpoint{3.740715in}{1.280708in}}%
\pgfpathlineto{\pgfqpoint{3.754797in}{1.280790in}}%
\pgfpathlineto{\pgfqpoint{3.768887in}{1.281052in}}%
\pgfpathlineto{\pgfqpoint{3.782986in}{1.281493in}}%
\pgfpathlineto{\pgfqpoint{3.797092in}{1.282114in}}%
\pgfpathlineto{\pgfqpoint{3.805380in}{1.293933in}}%
\pgfpathlineto{\pgfqpoint{3.813662in}{1.305909in}}%
\pgfpathlineto{\pgfqpoint{3.821937in}{1.318036in}}%
\pgfpathlineto{\pgfqpoint{3.830207in}{1.330306in}}%
\pgfpathlineto{\pgfqpoint{3.816110in}{1.329019in}}%
\pgfpathlineto{\pgfqpoint{3.802023in}{1.327912in}}%
\pgfpathlineto{\pgfqpoint{3.787944in}{1.326985in}}%
\pgfpathlineto{\pgfqpoint{3.773873in}{1.326238in}}%
\pgfpathlineto{\pgfqpoint{3.765593in}{1.314621in}}%
\pgfpathlineto{\pgfqpoint{3.757307in}{1.303156in}}%
\pgfpathlineto{\pgfqpoint{3.749014in}{1.291850in}}%
\pgfpathlineto{\pgfqpoint{3.740715in}{1.280708in}}%
\pgfpathclose%
\pgfusepath{fill}%
\end{pgfscope}%
\begin{pgfscope}%
\pgfpathrectangle{\pgfqpoint{1.150000in}{0.150000in}}{\pgfqpoint{5.700000in}{5.700000in}}%
\pgfusepath{clip}%
\pgfsetbuttcap%
\pgfsetroundjoin%
\definecolor{currentfill}{rgb}{0.377779,0.791781,0.377939}%
\pgfsetfillcolor{currentfill}%
\pgfsetfillopacity{0.800000}%
\pgfsetlinewidth{0.000000pt}%
\definecolor{currentstroke}{rgb}{0.000000,0.000000,0.000000}%
\pgfsetstrokecolor{currentstroke}%
\pgfsetdash{}{0pt}%
\pgfpathmoveto{\pgfqpoint{5.565242in}{3.388939in}}%
\pgfpathlineto{\pgfqpoint{5.580395in}{3.407344in}}%
\pgfpathlineto{\pgfqpoint{5.595573in}{3.425940in}}%
\pgfpathlineto{\pgfqpoint{5.610775in}{3.444729in}}%
\pgfpathlineto{\pgfqpoint{5.626003in}{3.463709in}}%
\pgfpathlineto{\pgfqpoint{5.633512in}{3.468008in}}%
\pgfpathlineto{\pgfqpoint{5.641009in}{3.472124in}}%
\pgfpathlineto{\pgfqpoint{5.648494in}{3.476060in}}%
\pgfpathlineto{\pgfqpoint{5.655968in}{3.479820in}}%
\pgfpathlineto{\pgfqpoint{5.640756in}{3.461117in}}%
\pgfpathlineto{\pgfqpoint{5.625569in}{3.442606in}}%
\pgfpathlineto{\pgfqpoint{5.610406in}{3.424285in}}%
\pgfpathlineto{\pgfqpoint{5.595268in}{3.406155in}}%
\pgfpathlineto{\pgfqpoint{5.587778in}{3.402105in}}%
\pgfpathlineto{\pgfqpoint{5.580278in}{3.397888in}}%
\pgfpathlineto{\pgfqpoint{5.572766in}{3.393501in}}%
\pgfpathlineto{\pgfqpoint{5.565242in}{3.388939in}}%
\pgfpathclose%
\pgfusepath{fill}%
\end{pgfscope}%
\begin{pgfscope}%
\pgfpathrectangle{\pgfqpoint{1.150000in}{0.150000in}}{\pgfqpoint{5.700000in}{5.700000in}}%
\pgfusepath{clip}%
\pgfsetbuttcap%
\pgfsetroundjoin%
\definecolor{currentfill}{rgb}{0.272594,0.025563,0.353093}%
\pgfsetfillcolor{currentfill}%
\pgfsetfillopacity{0.800000}%
\pgfsetlinewidth{0.000000pt}%
\definecolor{currentstroke}{rgb}{0.000000,0.000000,0.000000}%
\pgfsetstrokecolor{currentstroke}%
\pgfsetdash{}{0pt}%
\pgfpathmoveto{\pgfqpoint{3.268687in}{1.254785in}}%
\pgfpathlineto{\pgfqpoint{3.282742in}{1.247463in}}%
\pgfpathlineto{\pgfqpoint{3.296799in}{1.240332in}}%
\pgfpathlineto{\pgfqpoint{3.310858in}{1.233390in}}%
\pgfpathlineto{\pgfqpoint{3.324920in}{1.226638in}}%
\pgfpathlineto{\pgfqpoint{3.333480in}{1.230161in}}%
\pgfpathlineto{\pgfqpoint{3.342028in}{1.234006in}}%
\pgfpathlineto{\pgfqpoint{3.350563in}{1.238166in}}%
\pgfpathlineto{\pgfqpoint{3.359085in}{1.242632in}}%
\pgfpathlineto{\pgfqpoint{3.345055in}{1.248591in}}%
\pgfpathlineto{\pgfqpoint{3.331028in}{1.254739in}}%
\pgfpathlineto{\pgfqpoint{3.317003in}{1.261077in}}%
\pgfpathlineto{\pgfqpoint{3.302981in}{1.267605in}}%
\pgfpathlineto{\pgfqpoint{3.294427in}{1.263919in}}%
\pgfpathlineto{\pgfqpoint{3.285861in}{1.260548in}}%
\pgfpathlineto{\pgfqpoint{3.277281in}{1.257501in}}%
\pgfpathlineto{\pgfqpoint{3.268687in}{1.254785in}}%
\pgfpathclose%
\pgfusepath{fill}%
\end{pgfscope}%
\begin{pgfscope}%
\pgfpathrectangle{\pgfqpoint{1.150000in}{0.150000in}}{\pgfqpoint{5.700000in}{5.700000in}}%
\pgfusepath{clip}%
\pgfsetbuttcap%
\pgfsetroundjoin%
\definecolor{currentfill}{rgb}{0.273809,0.031497,0.358853}%
\pgfsetfillcolor{currentfill}%
\pgfsetfillopacity{0.800000}%
\pgfsetlinewidth{0.000000pt}%
\definecolor{currentstroke}{rgb}{0.000000,0.000000,0.000000}%
\pgfsetstrokecolor{currentstroke}%
\pgfsetdash{}{0pt}%
\pgfpathmoveto{\pgfqpoint{3.651137in}{1.242185in}}%
\pgfpathlineto{\pgfqpoint{3.665204in}{1.240849in}}%
\pgfpathlineto{\pgfqpoint{3.679279in}{1.239693in}}%
\pgfpathlineto{\pgfqpoint{3.693360in}{1.238719in}}%
\pgfpathlineto{\pgfqpoint{3.707448in}{1.237924in}}%
\pgfpathlineto{\pgfqpoint{3.715776in}{1.248339in}}%
\pgfpathlineto{\pgfqpoint{3.724096in}{1.258946in}}%
\pgfpathlineto{\pgfqpoint{3.732409in}{1.269738in}}%
\pgfpathlineto{\pgfqpoint{3.740715in}{1.280708in}}%
\pgfpathlineto{\pgfqpoint{3.726640in}{1.280806in}}%
\pgfpathlineto{\pgfqpoint{3.712573in}{1.281085in}}%
\pgfpathlineto{\pgfqpoint{3.698514in}{1.281545in}}%
\pgfpathlineto{\pgfqpoint{3.684462in}{1.282185in}}%
\pgfpathlineto{\pgfqpoint{3.676142in}{1.271899in}}%
\pgfpathlineto{\pgfqpoint{3.667814in}{1.261799in}}%
\pgfpathlineto{\pgfqpoint{3.659479in}{1.251892in}}%
\pgfpathlineto{\pgfqpoint{3.651137in}{1.242185in}}%
\pgfpathclose%
\pgfusepath{fill}%
\end{pgfscope}%
\begin{pgfscope}%
\pgfpathrectangle{\pgfqpoint{1.150000in}{0.150000in}}{\pgfqpoint{5.700000in}{5.700000in}}%
\pgfusepath{clip}%
\pgfsetbuttcap%
\pgfsetroundjoin%
\definecolor{currentfill}{rgb}{0.280894,0.078907,0.402329}%
\pgfsetfillcolor{currentfill}%
\pgfsetfillopacity{0.800000}%
\pgfsetlinewidth{0.000000pt}%
\definecolor{currentstroke}{rgb}{0.000000,0.000000,0.000000}%
\pgfsetstrokecolor{currentstroke}%
\pgfsetdash{}{0pt}%
\pgfpathmoveto{\pgfqpoint{3.830207in}{1.330306in}}%
\pgfpathlineto{\pgfqpoint{3.844312in}{1.331773in}}%
\pgfpathlineto{\pgfqpoint{3.858426in}{1.333418in}}%
\pgfpathlineto{\pgfqpoint{3.872549in}{1.335243in}}%
\pgfpathlineto{\pgfqpoint{3.886682in}{1.337246in}}%
\pgfpathlineto{\pgfqpoint{3.894938in}{1.350302in}}%
\pgfpathlineto{\pgfqpoint{3.903188in}{1.363482in}}%
\pgfpathlineto{\pgfqpoint{3.911433in}{1.376780in}}%
\pgfpathlineto{\pgfqpoint{3.919673in}{1.390189in}}%
\pgfpathlineto{\pgfqpoint{3.905547in}{1.387549in}}%
\pgfpathlineto{\pgfqpoint{3.891431in}{1.385089in}}%
\pgfpathlineto{\pgfqpoint{3.877325in}{1.382808in}}%
\pgfpathlineto{\pgfqpoint{3.863228in}{1.380706in}}%
\pgfpathlineto{\pgfqpoint{3.854981in}{1.367921in}}%
\pgfpathlineto{\pgfqpoint{3.846729in}{1.355255in}}%
\pgfpathlineto{\pgfqpoint{3.838471in}{1.342715in}}%
\pgfpathlineto{\pgfqpoint{3.830207in}{1.330306in}}%
\pgfpathclose%
\pgfusepath{fill}%
\end{pgfscope}%
\begin{pgfscope}%
\pgfpathrectangle{\pgfqpoint{1.150000in}{0.150000in}}{\pgfqpoint{5.700000in}{5.700000in}}%
\pgfusepath{clip}%
\pgfsetbuttcap%
\pgfsetroundjoin%
\definecolor{currentfill}{rgb}{0.269944,0.014625,0.341379}%
\pgfsetfillcolor{currentfill}%
\pgfsetfillopacity{0.800000}%
\pgfsetlinewidth{0.000000pt}%
\definecolor{currentstroke}{rgb}{0.000000,0.000000,0.000000}%
\pgfsetstrokecolor{currentstroke}%
\pgfsetdash{}{0pt}%
\pgfpathmoveto{\pgfqpoint{3.561407in}{1.215562in}}%
\pgfpathlineto{\pgfqpoint{3.575468in}{1.212772in}}%
\pgfpathlineto{\pgfqpoint{3.589535in}{1.210164in}}%
\pgfpathlineto{\pgfqpoint{3.603608in}{1.207738in}}%
\pgfpathlineto{\pgfqpoint{3.617686in}{1.205493in}}%
\pgfpathlineto{\pgfqpoint{3.626061in}{1.214331in}}%
\pgfpathlineto{\pgfqpoint{3.634428in}{1.223397in}}%
\pgfpathlineto{\pgfqpoint{3.642786in}{1.232684in}}%
\pgfpathlineto{\pgfqpoint{3.651137in}{1.242185in}}%
\pgfpathlineto{\pgfqpoint{3.637076in}{1.243702in}}%
\pgfpathlineto{\pgfqpoint{3.623022in}{1.245401in}}%
\pgfpathlineto{\pgfqpoint{3.608974in}{1.247282in}}%
\pgfpathlineto{\pgfqpoint{3.594932in}{1.249345in}}%
\pgfpathlineto{\pgfqpoint{3.586564in}{1.240560in}}%
\pgfpathlineto{\pgfqpoint{3.578187in}{1.231996in}}%
\pgfpathlineto{\pgfqpoint{3.569801in}{1.223661in}}%
\pgfpathlineto{\pgfqpoint{3.561407in}{1.215562in}}%
\pgfpathclose%
\pgfusepath{fill}%
\end{pgfscope}%
\begin{pgfscope}%
\pgfpathrectangle{\pgfqpoint{1.150000in}{0.150000in}}{\pgfqpoint{5.700000in}{5.700000in}}%
\pgfusepath{clip}%
\pgfsetbuttcap%
\pgfsetroundjoin%
\definecolor{currentfill}{rgb}{0.283197,0.115680,0.436115}%
\pgfsetfillcolor{currentfill}%
\pgfsetfillopacity{0.800000}%
\pgfsetlinewidth{0.000000pt}%
\definecolor{currentstroke}{rgb}{0.000000,0.000000,0.000000}%
\pgfsetstrokecolor{currentstroke}%
\pgfsetdash{}{0pt}%
\pgfpathmoveto{\pgfqpoint{3.919673in}{1.390189in}}%
\pgfpathlineto{\pgfqpoint{3.933809in}{1.393008in}}%
\pgfpathlineto{\pgfqpoint{3.947955in}{1.396006in}}%
\pgfpathlineto{\pgfqpoint{3.962110in}{1.399182in}}%
\pgfpathlineto{\pgfqpoint{3.976277in}{1.402537in}}%
\pgfpathlineto{\pgfqpoint{3.984507in}{1.416670in}}%
\pgfpathlineto{\pgfqpoint{3.992732in}{1.430895in}}%
\pgfpathlineto{\pgfqpoint{4.000952in}{1.445206in}}%
\pgfpathlineto{\pgfqpoint{4.009168in}{1.459597in}}%
\pgfpathlineto{\pgfqpoint{3.995006in}{1.455635in}}%
\pgfpathlineto{\pgfqpoint{3.980854in}{1.451852in}}%
\pgfpathlineto{\pgfqpoint{3.966713in}{1.448248in}}%
\pgfpathlineto{\pgfqpoint{3.952583in}{1.444824in}}%
\pgfpathlineto{\pgfqpoint{3.944363in}{1.431027in}}%
\pgfpathlineto{\pgfqpoint{3.936138in}{1.417319in}}%
\pgfpathlineto{\pgfqpoint{3.927908in}{1.403704in}}%
\pgfpathlineto{\pgfqpoint{3.919673in}{1.390189in}}%
\pgfpathclose%
\pgfusepath{fill}%
\end{pgfscope}%
\begin{pgfscope}%
\pgfpathrectangle{\pgfqpoint{1.150000in}{0.150000in}}{\pgfqpoint{5.700000in}{5.700000in}}%
\pgfusepath{clip}%
\pgfsetbuttcap%
\pgfsetroundjoin%
\definecolor{currentfill}{rgb}{0.252194,0.269783,0.531579}%
\pgfsetfillcolor{currentfill}%
\pgfsetfillopacity{0.800000}%
\pgfsetlinewidth{0.000000pt}%
\definecolor{currentstroke}{rgb}{0.000000,0.000000,0.000000}%
\pgfsetstrokecolor{currentstroke}%
\pgfsetdash{}{0pt}%
\pgfpathmoveto{\pgfqpoint{4.253746in}{1.751556in}}%
\pgfpathlineto{\pgfqpoint{4.268021in}{1.759258in}}%
\pgfpathlineto{\pgfqpoint{4.282309in}{1.767140in}}%
\pgfpathlineto{\pgfqpoint{4.296612in}{1.775203in}}%
\pgfpathlineto{\pgfqpoint{4.310929in}{1.783447in}}%
\pgfpathlineto{\pgfqpoint{4.319079in}{1.799803in}}%
\pgfpathlineto{\pgfqpoint{4.327224in}{1.816126in}}%
\pgfpathlineto{\pgfqpoint{4.335366in}{1.832412in}}%
\pgfpathlineto{\pgfqpoint{4.343504in}{1.848657in}}%
\pgfpathlineto{\pgfqpoint{4.329183in}{1.839958in}}%
\pgfpathlineto{\pgfqpoint{4.314877in}{1.831440in}}%
\pgfpathlineto{\pgfqpoint{4.300585in}{1.823104in}}%
\pgfpathlineto{\pgfqpoint{4.286307in}{1.814949in}}%
\pgfpathlineto{\pgfqpoint{4.278173in}{1.799145in}}%
\pgfpathlineto{\pgfqpoint{4.270034in}{1.783309in}}%
\pgfpathlineto{\pgfqpoint{4.261892in}{1.767445in}}%
\pgfpathlineto{\pgfqpoint{4.253746in}{1.751556in}}%
\pgfpathclose%
\pgfusepath{fill}%
\end{pgfscope}%
\begin{pgfscope}%
\pgfpathrectangle{\pgfqpoint{1.150000in}{0.150000in}}{\pgfqpoint{5.700000in}{5.700000in}}%
\pgfusepath{clip}%
\pgfsetbuttcap%
\pgfsetroundjoin%
\definecolor{currentfill}{rgb}{0.175707,0.697900,0.491033}%
\pgfsetfillcolor{currentfill}%
\pgfsetfillopacity{0.800000}%
\pgfsetlinewidth{0.000000pt}%
\definecolor{currentstroke}{rgb}{0.000000,0.000000,0.000000}%
\pgfsetstrokecolor{currentstroke}%
\pgfsetdash{}{0pt}%
\pgfpathmoveto{\pgfqpoint{5.231851in}{3.041184in}}%
\pgfpathlineto{\pgfqpoint{5.246768in}{3.058125in}}%
\pgfpathlineto{\pgfqpoint{5.261707in}{3.075256in}}%
\pgfpathlineto{\pgfqpoint{5.276668in}{3.092577in}}%
\pgfpathlineto{\pgfqpoint{5.291652in}{3.110090in}}%
\pgfpathlineto{\pgfqpoint{5.299412in}{3.118817in}}%
\pgfpathlineto{\pgfqpoint{5.307161in}{3.127344in}}%
\pgfpathlineto{\pgfqpoint{5.314900in}{3.135672in}}%
\pgfpathlineto{\pgfqpoint{5.322629in}{3.143803in}}%
\pgfpathlineto{\pgfqpoint{5.307650in}{3.126382in}}%
\pgfpathlineto{\pgfqpoint{5.292693in}{3.109152in}}%
\pgfpathlineto{\pgfqpoint{5.277759in}{3.092112in}}%
\pgfpathlineto{\pgfqpoint{5.262847in}{3.075262in}}%
\pgfpathlineto{\pgfqpoint{5.255113in}{3.067027in}}%
\pgfpathlineto{\pgfqpoint{5.247369in}{3.058603in}}%
\pgfpathlineto{\pgfqpoint{5.239615in}{3.049989in}}%
\pgfpathlineto{\pgfqpoint{5.231851in}{3.041184in}}%
\pgfpathclose%
\pgfusepath{fill}%
\end{pgfscope}%
\begin{pgfscope}%
\pgfpathrectangle{\pgfqpoint{1.150000in}{0.150000in}}{\pgfqpoint{5.700000in}{5.700000in}}%
\pgfusepath{clip}%
\pgfsetbuttcap%
\pgfsetroundjoin%
\definecolor{currentfill}{rgb}{0.223925,0.334994,0.548053}%
\pgfsetfillcolor{currentfill}%
\pgfsetfillopacity{0.800000}%
\pgfsetlinewidth{0.000000pt}%
\definecolor{currentstroke}{rgb}{0.000000,0.000000,0.000000}%
\pgfsetstrokecolor{currentstroke}%
\pgfsetdash{}{0pt}%
\pgfpathmoveto{\pgfqpoint{4.376016in}{1.913151in}}%
\pgfpathlineto{\pgfqpoint{4.390356in}{1.922454in}}%
\pgfpathlineto{\pgfqpoint{4.404712in}{1.931940in}}%
\pgfpathlineto{\pgfqpoint{4.419083in}{1.941607in}}%
\pgfpathlineto{\pgfqpoint{4.433469in}{1.951456in}}%
\pgfpathlineto{\pgfqpoint{4.441592in}{1.967842in}}%
\pgfpathlineto{\pgfqpoint{4.449711in}{1.984156in}}%
\pgfpathlineto{\pgfqpoint{4.457825in}{2.000395in}}%
\pgfpathlineto{\pgfqpoint{4.465935in}{2.016554in}}%
\pgfpathlineto{\pgfqpoint{4.451543in}{2.006313in}}%
\pgfpathlineto{\pgfqpoint{4.437166in}{1.996254in}}%
\pgfpathlineto{\pgfqpoint{4.422806in}{1.986377in}}%
\pgfpathlineto{\pgfqpoint{4.408461in}{1.976683in}}%
\pgfpathlineto{\pgfqpoint{4.400356in}{1.960903in}}%
\pgfpathlineto{\pgfqpoint{4.392247in}{1.945052in}}%
\pgfpathlineto{\pgfqpoint{4.384133in}{1.929133in}}%
\pgfpathlineto{\pgfqpoint{4.376016in}{1.913151in}}%
\pgfpathclose%
\pgfusepath{fill}%
\end{pgfscope}%
\begin{pgfscope}%
\pgfpathrectangle{\pgfqpoint{1.150000in}{0.150000in}}{\pgfqpoint{5.700000in}{5.700000in}}%
\pgfusepath{clip}%
\pgfsetbuttcap%
\pgfsetroundjoin%
\definecolor{currentfill}{rgb}{0.271828,0.209303,0.504434}%
\pgfsetfillcolor{currentfill}%
\pgfsetfillopacity{0.800000}%
\pgfsetlinewidth{0.000000pt}%
\definecolor{currentstroke}{rgb}{0.000000,0.000000,0.000000}%
\pgfsetstrokecolor{currentstroke}%
\pgfsetdash{}{0pt}%
\pgfpathmoveto{\pgfqpoint{4.131487in}{1.599089in}}%
\pgfpathlineto{\pgfqpoint{4.145705in}{1.605070in}}%
\pgfpathlineto{\pgfqpoint{4.159936in}{1.611230in}}%
\pgfpathlineto{\pgfqpoint{4.174180in}{1.617570in}}%
\pgfpathlineto{\pgfqpoint{4.188436in}{1.624089in}}%
\pgfpathlineto{\pgfqpoint{4.196614in}{1.640014in}}%
\pgfpathlineto{\pgfqpoint{4.204787in}{1.655951in}}%
\pgfpathlineto{\pgfqpoint{4.212957in}{1.671895in}}%
\pgfpathlineto{\pgfqpoint{4.221122in}{1.687841in}}%
\pgfpathlineto{\pgfqpoint{4.206864in}{1.680804in}}%
\pgfpathlineto{\pgfqpoint{4.192619in}{1.673948in}}%
\pgfpathlineto{\pgfqpoint{4.178387in}{1.667272in}}%
\pgfpathlineto{\pgfqpoint{4.164168in}{1.660776in}}%
\pgfpathlineto{\pgfqpoint{4.156003in}{1.645334in}}%
\pgfpathlineto{\pgfqpoint{4.147835in}{1.629902in}}%
\pgfpathlineto{\pgfqpoint{4.139663in}{1.614486in}}%
\pgfpathlineto{\pgfqpoint{4.131487in}{1.599089in}}%
\pgfpathclose%
\pgfusepath{fill}%
\end{pgfscope}%
\begin{pgfscope}%
\pgfpathrectangle{\pgfqpoint{1.150000in}{0.150000in}}{\pgfqpoint{5.700000in}{5.700000in}}%
\pgfusepath{clip}%
\pgfsetbuttcap%
\pgfsetroundjoin%
\definecolor{currentfill}{rgb}{0.197636,0.391528,0.554969}%
\pgfsetfillcolor{currentfill}%
\pgfsetfillopacity{0.800000}%
\pgfsetlinewidth{0.000000pt}%
\definecolor{currentstroke}{rgb}{0.000000,0.000000,0.000000}%
\pgfsetstrokecolor{currentstroke}%
\pgfsetdash{}{0pt}%
\pgfpathmoveto{\pgfqpoint{4.498331in}{2.080335in}}%
\pgfpathlineto{\pgfqpoint{4.512745in}{2.091119in}}%
\pgfpathlineto{\pgfqpoint{4.527175in}{2.102085in}}%
\pgfpathlineto{\pgfqpoint{4.541622in}{2.113236in}}%
\pgfpathlineto{\pgfqpoint{4.556085in}{2.124569in}}%
\pgfpathlineto{\pgfqpoint{4.564179in}{2.140624in}}%
\pgfpathlineto{\pgfqpoint{4.572268in}{2.156572in}}%
\pgfpathlineto{\pgfqpoint{4.580352in}{2.172411in}}%
\pgfpathlineto{\pgfqpoint{4.588431in}{2.188138in}}%
\pgfpathlineto{\pgfqpoint{4.573961in}{2.176477in}}%
\pgfpathlineto{\pgfqpoint{4.559508in}{2.165000in}}%
\pgfpathlineto{\pgfqpoint{4.545072in}{2.153706in}}%
\pgfpathlineto{\pgfqpoint{4.530653in}{2.142597in}}%
\pgfpathlineto{\pgfqpoint{4.522579in}{2.127184in}}%
\pgfpathlineto{\pgfqpoint{4.514501in}{2.111668in}}%
\pgfpathlineto{\pgfqpoint{4.506418in}{2.096050in}}%
\pgfpathlineto{\pgfqpoint{4.498331in}{2.080335in}}%
\pgfpathclose%
\pgfusepath{fill}%
\end{pgfscope}%
\begin{pgfscope}%
\pgfpathrectangle{\pgfqpoint{1.150000in}{0.150000in}}{\pgfqpoint{5.700000in}{5.700000in}}%
\pgfusepath{clip}%
\pgfsetbuttcap%
\pgfsetroundjoin%
\definecolor{currentfill}{rgb}{0.458674,0.816363,0.329727}%
\pgfsetfillcolor{currentfill}%
\pgfsetfillopacity{0.800000}%
\pgfsetlinewidth{0.000000pt}%
\definecolor{currentstroke}{rgb}{0.000000,0.000000,0.000000}%
\pgfsetstrokecolor{currentstroke}%
\pgfsetdash{}{0pt}%
\pgfpathmoveto{\pgfqpoint{5.655968in}{3.479820in}}%
\pgfpathlineto{\pgfqpoint{5.671204in}{3.498714in}}%
\pgfpathlineto{\pgfqpoint{5.686466in}{3.517801in}}%
\pgfpathlineto{\pgfqpoint{5.701753in}{3.537080in}}%
\pgfpathlineto{\pgfqpoint{5.717065in}{3.556551in}}%
\pgfpathlineto{\pgfqpoint{5.724510in}{3.559837in}}%
\pgfpathlineto{\pgfqpoint{5.731942in}{3.562944in}}%
\pgfpathlineto{\pgfqpoint{5.739362in}{3.565876in}}%
\pgfpathlineto{\pgfqpoint{5.746770in}{3.568637in}}%
\pgfpathlineto{\pgfqpoint{5.731476in}{3.549482in}}%
\pgfpathlineto{\pgfqpoint{5.716207in}{3.530519in}}%
\pgfpathlineto{\pgfqpoint{5.700963in}{3.511747in}}%
\pgfpathlineto{\pgfqpoint{5.685744in}{3.493166in}}%
\pgfpathlineto{\pgfqpoint{5.678317in}{3.490076in}}%
\pgfpathlineto{\pgfqpoint{5.670879in}{3.486825in}}%
\pgfpathlineto{\pgfqpoint{5.663430in}{3.483407in}}%
\pgfpathlineto{\pgfqpoint{5.655968in}{3.479820in}}%
\pgfpathclose%
\pgfusepath{fill}%
\end{pgfscope}%
\begin{pgfscope}%
\pgfpathrectangle{\pgfqpoint{1.150000in}{0.150000in}}{\pgfqpoint{5.700000in}{5.700000in}}%
\pgfusepath{clip}%
\pgfsetbuttcap%
\pgfsetroundjoin%
\definecolor{currentfill}{rgb}{0.271305,0.019942,0.347269}%
\pgfsetfillcolor{currentfill}%
\pgfsetfillopacity{0.800000}%
\pgfsetlinewidth{0.000000pt}%
\definecolor{currentstroke}{rgb}{0.000000,0.000000,0.000000}%
\pgfsetstrokecolor{currentstroke}%
\pgfsetdash{}{0pt}%
\pgfpathmoveto{\pgfqpoint{3.324920in}{1.226638in}}%
\pgfpathlineto{\pgfqpoint{3.338984in}{1.220075in}}%
\pgfpathlineto{\pgfqpoint{3.353051in}{1.213700in}}%
\pgfpathlineto{\pgfqpoint{3.367120in}{1.207512in}}%
\pgfpathlineto{\pgfqpoint{3.381193in}{1.201512in}}%
\pgfpathlineto{\pgfqpoint{3.389722in}{1.205840in}}%
\pgfpathlineto{\pgfqpoint{3.398240in}{1.210482in}}%
\pgfpathlineto{\pgfqpoint{3.406745in}{1.215430in}}%
\pgfpathlineto{\pgfqpoint{3.415239in}{1.220677in}}%
\pgfpathlineto{\pgfqpoint{3.401195in}{1.225885in}}%
\pgfpathlineto{\pgfqpoint{3.387155in}{1.231280in}}%
\pgfpathlineto{\pgfqpoint{3.373119in}{1.236862in}}%
\pgfpathlineto{\pgfqpoint{3.359085in}{1.242632in}}%
\pgfpathlineto{\pgfqpoint{3.350563in}{1.238166in}}%
\pgfpathlineto{\pgfqpoint{3.342028in}{1.234006in}}%
\pgfpathlineto{\pgfqpoint{3.333480in}{1.230161in}}%
\pgfpathlineto{\pgfqpoint{3.324920in}{1.226638in}}%
\pgfpathclose%
\pgfusepath{fill}%
\end{pgfscope}%
\begin{pgfscope}%
\pgfpathrectangle{\pgfqpoint{1.150000in}{0.150000in}}{\pgfqpoint{5.700000in}{5.700000in}}%
\pgfusepath{clip}%
\pgfsetbuttcap%
\pgfsetroundjoin%
\definecolor{currentfill}{rgb}{0.304148,0.764704,0.419943}%
\pgfsetfillcolor{currentfill}%
\pgfsetfillopacity{0.800000}%
\pgfsetlinewidth{0.000000pt}%
\definecolor{currentstroke}{rgb}{0.000000,0.000000,0.000000}%
\pgfsetstrokecolor{currentstroke}%
\pgfsetdash{}{0pt}%
\pgfpathmoveto{\pgfqpoint{5.444174in}{3.272358in}}%
\pgfpathlineto{\pgfqpoint{5.459255in}{3.290439in}}%
\pgfpathlineto{\pgfqpoint{5.474359in}{3.308712in}}%
\pgfpathlineto{\pgfqpoint{5.489488in}{3.327176in}}%
\pgfpathlineto{\pgfqpoint{5.504641in}{3.345833in}}%
\pgfpathlineto{\pgfqpoint{5.512256in}{3.351889in}}%
\pgfpathlineto{\pgfqpoint{5.519860in}{3.357750in}}%
\pgfpathlineto{\pgfqpoint{5.527452in}{3.363417in}}%
\pgfpathlineto{\pgfqpoint{5.535033in}{3.368892in}}%
\pgfpathlineto{\pgfqpoint{5.519891in}{3.350439in}}%
\pgfpathlineto{\pgfqpoint{5.504774in}{3.332178in}}%
\pgfpathlineto{\pgfqpoint{5.489680in}{3.314107in}}%
\pgfpathlineto{\pgfqpoint{5.474610in}{3.296227in}}%
\pgfpathlineto{\pgfqpoint{5.467017in}{3.290535in}}%
\pgfpathlineto{\pgfqpoint{5.459414in}{3.284662in}}%
\pgfpathlineto{\pgfqpoint{5.451800in}{3.278603in}}%
\pgfpathlineto{\pgfqpoint{5.444174in}{3.272358in}}%
\pgfpathclose%
\pgfusepath{fill}%
\end{pgfscope}%
\begin{pgfscope}%
\pgfpathrectangle{\pgfqpoint{1.150000in}{0.150000in}}{\pgfqpoint{5.700000in}{5.700000in}}%
\pgfusepath{clip}%
\pgfsetbuttcap%
\pgfsetroundjoin%
\definecolor{currentfill}{rgb}{0.119512,0.607464,0.540218}%
\pgfsetfillcolor{currentfill}%
\pgfsetfillopacity{0.800000}%
\pgfsetlinewidth{0.000000pt}%
\definecolor{currentstroke}{rgb}{0.000000,0.000000,0.000000}%
\pgfsetstrokecolor{currentstroke}%
\pgfsetdash{}{0pt}%
\pgfpathmoveto{\pgfqpoint{1.957684in}{2.890319in}}%
\pgfpathlineto{\pgfqpoint{1.972465in}{2.859935in}}%
\pgfpathlineto{\pgfqpoint{1.987225in}{2.829905in}}%
\pgfpathlineto{\pgfqpoint{2.001965in}{2.800223in}}%
\pgfpathlineto{\pgfqpoint{2.016685in}{2.770886in}}%
\pgfpathlineto{\pgfqpoint{2.026607in}{2.754936in}}%
\pgfpathlineto{\pgfqpoint{2.036493in}{2.739537in}}%
\pgfpathlineto{\pgfqpoint{2.046342in}{2.724680in}}%
\pgfpathlineto{\pgfqpoint{2.056156in}{2.710355in}}%
\pgfpathlineto{\pgfqpoint{2.041525in}{2.738781in}}%
\pgfpathlineto{\pgfqpoint{2.026874in}{2.767550in}}%
\pgfpathlineto{\pgfqpoint{2.012204in}{2.796665in}}%
\pgfpathlineto{\pgfqpoint{1.997514in}{2.826128in}}%
\pgfpathlineto{\pgfqpoint{1.987612in}{2.841351in}}%
\pgfpathlineto{\pgfqpoint{1.977673in}{2.857117in}}%
\pgfpathlineto{\pgfqpoint{1.967698in}{2.873436in}}%
\pgfpathlineto{\pgfqpoint{1.957684in}{2.890319in}}%
\pgfpathclose%
\pgfusepath{fill}%
\end{pgfscope}%
\begin{pgfscope}%
\pgfpathrectangle{\pgfqpoint{1.150000in}{0.150000in}}{\pgfqpoint{5.700000in}{5.700000in}}%
\pgfusepath{clip}%
\pgfsetbuttcap%
\pgfsetroundjoin%
\definecolor{currentfill}{rgb}{0.269944,0.014625,0.341379}%
\pgfsetfillcolor{currentfill}%
\pgfsetfillopacity{0.800000}%
\pgfsetlinewidth{0.000000pt}%
\definecolor{currentstroke}{rgb}{0.000000,0.000000,0.000000}%
\pgfsetstrokecolor{currentstroke}%
\pgfsetdash{}{0pt}%
\pgfpathmoveto{\pgfqpoint{3.471452in}{1.201702in}}%
\pgfpathlineto{\pgfqpoint{3.485516in}{1.197421in}}%
\pgfpathlineto{\pgfqpoint{3.499584in}{1.193323in}}%
\pgfpathlineto{\pgfqpoint{3.513657in}{1.189408in}}%
\pgfpathlineto{\pgfqpoint{3.527735in}{1.185676in}}%
\pgfpathlineto{\pgfqpoint{3.536168in}{1.192756in}}%
\pgfpathlineto{\pgfqpoint{3.544590in}{1.200102in}}%
\pgfpathlineto{\pgfqpoint{3.553003in}{1.207707in}}%
\pgfpathlineto{\pgfqpoint{3.561407in}{1.215562in}}%
\pgfpathlineto{\pgfqpoint{3.547351in}{1.218534in}}%
\pgfpathlineto{\pgfqpoint{3.533301in}{1.221690in}}%
\pgfpathlineto{\pgfqpoint{3.519256in}{1.225029in}}%
\pgfpathlineto{\pgfqpoint{3.505216in}{1.228551in}}%
\pgfpathlineto{\pgfqpoint{3.496790in}{1.221443in}}%
\pgfpathlineto{\pgfqpoint{3.488354in}{1.214594in}}%
\pgfpathlineto{\pgfqpoint{3.479908in}{1.208011in}}%
\pgfpathlineto{\pgfqpoint{3.471452in}{1.201702in}}%
\pgfpathclose%
\pgfusepath{fill}%
\end{pgfscope}%
\begin{pgfscope}%
\pgfpathrectangle{\pgfqpoint{1.150000in}{0.150000in}}{\pgfqpoint{5.700000in}{5.700000in}}%
\pgfusepath{clip}%
\pgfsetbuttcap%
\pgfsetroundjoin%
\definecolor{currentfill}{rgb}{0.132268,0.655014,0.519661}%
\pgfsetfillcolor{currentfill}%
\pgfsetfillopacity{0.800000}%
\pgfsetlinewidth{0.000000pt}%
\definecolor{currentstroke}{rgb}{0.000000,0.000000,0.000000}%
\pgfsetstrokecolor{currentstroke}%
\pgfsetdash{}{0pt}%
\pgfpathmoveto{\pgfqpoint{5.109934in}{2.897565in}}%
\pgfpathlineto{\pgfqpoint{5.124767in}{2.913821in}}%
\pgfpathlineto{\pgfqpoint{5.139622in}{2.930265in}}%
\pgfpathlineto{\pgfqpoint{5.154499in}{2.946900in}}%
\pgfpathlineto{\pgfqpoint{5.169397in}{2.963724in}}%
\pgfpathlineto{\pgfqpoint{5.177237in}{2.974097in}}%
\pgfpathlineto{\pgfqpoint{5.185067in}{2.984272in}}%
\pgfpathlineto{\pgfqpoint{5.192889in}{2.994248in}}%
\pgfpathlineto{\pgfqpoint{5.200700in}{3.004027in}}%
\pgfpathlineto{\pgfqpoint{5.185803in}{2.987221in}}%
\pgfpathlineto{\pgfqpoint{5.170928in}{2.970605in}}%
\pgfpathlineto{\pgfqpoint{5.156074in}{2.954178in}}%
\pgfpathlineto{\pgfqpoint{5.141243in}{2.937941in}}%
\pgfpathlineto{\pgfqpoint{5.133429in}{2.928130in}}%
\pgfpathlineto{\pgfqpoint{5.125606in}{2.918131in}}%
\pgfpathlineto{\pgfqpoint{5.117774in}{2.907943in}}%
\pgfpathlineto{\pgfqpoint{5.109934in}{2.897565in}}%
\pgfpathclose%
\pgfusepath{fill}%
\end{pgfscope}%
\begin{pgfscope}%
\pgfpathrectangle{\pgfqpoint{1.150000in}{0.150000in}}{\pgfqpoint{5.700000in}{5.700000in}}%
\pgfusepath{clip}%
\pgfsetbuttcap%
\pgfsetroundjoin%
\definecolor{currentfill}{rgb}{0.171176,0.452530,0.557965}%
\pgfsetfillcolor{currentfill}%
\pgfsetfillopacity{0.800000}%
\pgfsetlinewidth{0.000000pt}%
\definecolor{currentstroke}{rgb}{0.000000,0.000000,0.000000}%
\pgfsetstrokecolor{currentstroke}%
\pgfsetdash{}{0pt}%
\pgfpathmoveto{\pgfqpoint{4.620696in}{2.249871in}}%
\pgfpathlineto{\pgfqpoint{4.635189in}{2.262011in}}%
\pgfpathlineto{\pgfqpoint{4.649700in}{2.274335in}}%
\pgfpathlineto{\pgfqpoint{4.664228in}{2.286844in}}%
\pgfpathlineto{\pgfqpoint{4.678775in}{2.299537in}}%
\pgfpathlineto{\pgfqpoint{4.686834in}{2.314938in}}%
\pgfpathlineto{\pgfqpoint{4.694888in}{2.330203in}}%
\pgfpathlineto{\pgfqpoint{4.702937in}{2.345331in}}%
\pgfpathlineto{\pgfqpoint{4.710979in}{2.360318in}}%
\pgfpathlineto{\pgfqpoint{4.696427in}{2.347363in}}%
\pgfpathlineto{\pgfqpoint{4.681893in}{2.334593in}}%
\pgfpathlineto{\pgfqpoint{4.667376in}{2.322008in}}%
\pgfpathlineto{\pgfqpoint{4.652878in}{2.309609in}}%
\pgfpathlineto{\pgfqpoint{4.644841in}{2.294870in}}%
\pgfpathlineto{\pgfqpoint{4.636798in}{2.279999in}}%
\pgfpathlineto{\pgfqpoint{4.628750in}{2.264999in}}%
\pgfpathlineto{\pgfqpoint{4.620696in}{2.249871in}}%
\pgfpathclose%
\pgfusepath{fill}%
\end{pgfscope}%
\begin{pgfscope}%
\pgfpathrectangle{\pgfqpoint{1.150000in}{0.150000in}}{\pgfqpoint{5.700000in}{5.700000in}}%
\pgfusepath{clip}%
\pgfsetbuttcap%
\pgfsetroundjoin%
\definecolor{currentfill}{rgb}{0.281887,0.150881,0.465405}%
\pgfsetfillcolor{currentfill}%
\pgfsetfillopacity{0.800000}%
\pgfsetlinewidth{0.000000pt}%
\definecolor{currentstroke}{rgb}{0.000000,0.000000,0.000000}%
\pgfsetstrokecolor{currentstroke}%
\pgfsetdash{}{0pt}%
\pgfpathmoveto{\pgfqpoint{4.009168in}{1.459597in}}%
\pgfpathlineto{\pgfqpoint{4.023342in}{1.463738in}}%
\pgfpathlineto{\pgfqpoint{4.037526in}{1.468058in}}%
\pgfpathlineto{\pgfqpoint{4.051722in}{1.472556in}}%
\pgfpathlineto{\pgfqpoint{4.065930in}{1.477233in}}%
\pgfpathlineto{\pgfqpoint{4.074139in}{1.492288in}}%
\pgfpathlineto{\pgfqpoint{4.082344in}{1.507405in}}%
\pgfpathlineto{\pgfqpoint{4.090545in}{1.522577in}}%
\pgfpathlineto{\pgfqpoint{4.098741in}{1.537799in}}%
\pgfpathlineto{\pgfqpoint{4.084535in}{1.532544in}}%
\pgfpathlineto{\pgfqpoint{4.070341in}{1.527468in}}%
\pgfpathlineto{\pgfqpoint{4.056158in}{1.522572in}}%
\pgfpathlineto{\pgfqpoint{4.041987in}{1.517855in}}%
\pgfpathlineto{\pgfqpoint{4.033789in}{1.503197in}}%
\pgfpathlineto{\pgfqpoint{4.025587in}{1.488598in}}%
\pgfpathlineto{\pgfqpoint{4.017380in}{1.474063in}}%
\pgfpathlineto{\pgfqpoint{4.009168in}{1.459597in}}%
\pgfpathclose%
\pgfusepath{fill}%
\end{pgfscope}%
\begin{pgfscope}%
\pgfpathrectangle{\pgfqpoint{1.150000in}{0.150000in}}{\pgfqpoint{5.700000in}{5.700000in}}%
\pgfusepath{clip}%
\pgfsetbuttcap%
\pgfsetroundjoin%
\definecolor{currentfill}{rgb}{0.150476,0.504369,0.557430}%
\pgfsetfillcolor{currentfill}%
\pgfsetfillopacity{0.800000}%
\pgfsetlinewidth{0.000000pt}%
\definecolor{currentstroke}{rgb}{0.000000,0.000000,0.000000}%
\pgfsetstrokecolor{currentstroke}%
\pgfsetdash{}{0pt}%
\pgfpathmoveto{\pgfqpoint{4.743092in}{2.418827in}}%
\pgfpathlineto{\pgfqpoint{4.757668in}{2.432195in}}%
\pgfpathlineto{\pgfqpoint{4.772263in}{2.445749in}}%
\pgfpathlineto{\pgfqpoint{4.786877in}{2.459489in}}%
\pgfpathlineto{\pgfqpoint{4.801510in}{2.473415in}}%
\pgfpathlineto{\pgfqpoint{4.809528in}{2.487879in}}%
\pgfpathlineto{\pgfqpoint{4.817540in}{2.502184in}}%
\pgfpathlineto{\pgfqpoint{4.825545in}{2.516327in}}%
\pgfpathlineto{\pgfqpoint{4.833544in}{2.530308in}}%
\pgfpathlineto{\pgfqpoint{4.818906in}{2.516188in}}%
\pgfpathlineto{\pgfqpoint{4.804288in}{2.502254in}}%
\pgfpathlineto{\pgfqpoint{4.789688in}{2.488507in}}%
\pgfpathlineto{\pgfqpoint{4.775107in}{2.474946in}}%
\pgfpathlineto{\pgfqpoint{4.767113in}{2.461146in}}%
\pgfpathlineto{\pgfqpoint{4.759112in}{2.447192in}}%
\pgfpathlineto{\pgfqpoint{4.751105in}{2.433085in}}%
\pgfpathlineto{\pgfqpoint{4.743092in}{2.418827in}}%
\pgfpathclose%
\pgfusepath{fill}%
\end{pgfscope}%
\begin{pgfscope}%
\pgfpathrectangle{\pgfqpoint{1.150000in}{0.150000in}}{\pgfqpoint{5.700000in}{5.700000in}}%
\pgfusepath{clip}%
\pgfsetbuttcap%
\pgfsetroundjoin%
\definecolor{currentfill}{rgb}{0.119423,0.611141,0.538982}%
\pgfsetfillcolor{currentfill}%
\pgfsetfillopacity{0.800000}%
\pgfsetlinewidth{0.000000pt}%
\definecolor{currentstroke}{rgb}{0.000000,0.000000,0.000000}%
\pgfsetstrokecolor{currentstroke}%
\pgfsetdash{}{0pt}%
\pgfpathmoveto{\pgfqpoint{4.987779in}{2.744816in}}%
\pgfpathlineto{\pgfqpoint{5.002527in}{2.760245in}}%
\pgfpathlineto{\pgfqpoint{5.017296in}{2.775863in}}%
\pgfpathlineto{\pgfqpoint{5.032085in}{2.791670in}}%
\pgfpathlineto{\pgfqpoint{5.046896in}{2.807665in}}%
\pgfpathlineto{\pgfqpoint{5.054805in}{2.819573in}}%
\pgfpathlineto{\pgfqpoint{5.062706in}{2.831289in}}%
\pgfpathlineto{\pgfqpoint{5.070599in}{2.842813in}}%
\pgfpathlineto{\pgfqpoint{5.078483in}{2.854146in}}%
\pgfpathlineto{\pgfqpoint{5.063671in}{2.838097in}}%
\pgfpathlineto{\pgfqpoint{5.048880in}{2.822237in}}%
\pgfpathlineto{\pgfqpoint{5.034111in}{2.806566in}}%
\pgfpathlineto{\pgfqpoint{5.019362in}{2.791083in}}%
\pgfpathlineto{\pgfqpoint{5.011478in}{2.779790in}}%
\pgfpathlineto{\pgfqpoint{5.003587in}{2.768315in}}%
\pgfpathlineto{\pgfqpoint{4.995687in}{2.756657in}}%
\pgfpathlineto{\pgfqpoint{4.987779in}{2.744816in}}%
\pgfpathclose%
\pgfusepath{fill}%
\end{pgfscope}%
\begin{pgfscope}%
\pgfpathrectangle{\pgfqpoint{1.150000in}{0.150000in}}{\pgfqpoint{5.700000in}{5.700000in}}%
\pgfusepath{clip}%
\pgfsetbuttcap%
\pgfsetroundjoin%
\definecolor{currentfill}{rgb}{0.129933,0.559582,0.551864}%
\pgfsetfillcolor{currentfill}%
\pgfsetfillopacity{0.800000}%
\pgfsetlinewidth{0.000000pt}%
\definecolor{currentstroke}{rgb}{0.000000,0.000000,0.000000}%
\pgfsetstrokecolor{currentstroke}%
\pgfsetdash{}{0pt}%
\pgfpathmoveto{\pgfqpoint{4.865473in}{2.584577in}}%
\pgfpathlineto{\pgfqpoint{4.880135in}{2.599043in}}%
\pgfpathlineto{\pgfqpoint{4.894816in}{2.613696in}}%
\pgfpathlineto{\pgfqpoint{4.909518in}{2.628536in}}%
\pgfpathlineto{\pgfqpoint{4.924239in}{2.643565in}}%
\pgfpathlineto{\pgfqpoint{4.932208in}{2.656850in}}%
\pgfpathlineto{\pgfqpoint{4.940169in}{2.669958in}}%
\pgfpathlineto{\pgfqpoint{4.948123in}{2.682886in}}%
\pgfpathlineto{\pgfqpoint{4.956070in}{2.695634in}}%
\pgfpathlineto{\pgfqpoint{4.941345in}{2.680482in}}%
\pgfpathlineto{\pgfqpoint{4.926640in}{2.665517in}}%
\pgfpathlineto{\pgfqpoint{4.911955in}{2.650740in}}%
\pgfpathlineto{\pgfqpoint{4.897290in}{2.636150in}}%
\pgfpathlineto{\pgfqpoint{4.889347in}{2.623513in}}%
\pgfpathlineto{\pgfqpoint{4.881396in}{2.610704in}}%
\pgfpathlineto{\pgfqpoint{4.873438in}{2.597725in}}%
\pgfpathlineto{\pgfqpoint{4.865473in}{2.584577in}}%
\pgfpathclose%
\pgfusepath{fill}%
\end{pgfscope}%
\begin{pgfscope}%
\pgfpathrectangle{\pgfqpoint{1.150000in}{0.150000in}}{\pgfqpoint{5.700000in}{5.700000in}}%
\pgfusepath{clip}%
\pgfsetbuttcap%
\pgfsetroundjoin%
\definecolor{currentfill}{rgb}{0.525776,0.833491,0.288127}%
\pgfsetfillcolor{currentfill}%
\pgfsetfillopacity{0.800000}%
\pgfsetlinewidth{0.000000pt}%
\definecolor{currentstroke}{rgb}{0.000000,0.000000,0.000000}%
\pgfsetstrokecolor{currentstroke}%
\pgfsetdash{}{0pt}%
\pgfpathmoveto{\pgfqpoint{5.746770in}{3.568637in}}%
\pgfpathlineto{\pgfqpoint{5.762090in}{3.587985in}}%
\pgfpathlineto{\pgfqpoint{5.777435in}{3.607525in}}%
\pgfpathlineto{\pgfqpoint{5.792807in}{3.627259in}}%
\pgfpathlineto{\pgfqpoint{5.800188in}{3.629597in}}%
\pgfpathlineto{\pgfqpoint{5.807556in}{3.631766in}}%
\pgfpathlineto{\pgfqpoint{5.814912in}{3.633769in}}%
\pgfpathlineto{\pgfqpoint{5.822256in}{3.635609in}}%
\pgfpathlineto{\pgfqpoint{5.806905in}{3.616230in}}%
\pgfpathlineto{\pgfqpoint{5.791580in}{3.597043in}}%
\pgfpathlineto{\pgfqpoint{5.776281in}{3.578048in}}%
\pgfpathlineto{\pgfqpoint{5.768921in}{3.575933in}}%
\pgfpathlineto{\pgfqpoint{5.761550in}{3.573662in}}%
\pgfpathlineto{\pgfqpoint{5.754166in}{3.571231in}}%
\pgfpathlineto{\pgfqpoint{5.746770in}{3.568637in}}%
\pgfpathclose%
\pgfusepath{fill}%
\end{pgfscope}%
\begin{pgfscope}%
\pgfpathrectangle{\pgfqpoint{1.150000in}{0.150000in}}{\pgfqpoint{5.700000in}{5.700000in}}%
\pgfusepath{clip}%
\pgfsetbuttcap%
\pgfsetroundjoin%
\definecolor{currentfill}{rgb}{0.276022,0.044167,0.370164}%
\pgfsetfillcolor{currentfill}%
\pgfsetfillopacity{0.800000}%
\pgfsetlinewidth{0.000000pt}%
\definecolor{currentstroke}{rgb}{0.000000,0.000000,0.000000}%
\pgfsetstrokecolor{currentstroke}%
\pgfsetdash{}{0pt}%
\pgfpathmoveto{\pgfqpoint{3.707448in}{1.237924in}}%
\pgfpathlineto{\pgfqpoint{3.721544in}{1.237309in}}%
\pgfpathlineto{\pgfqpoint{3.735647in}{1.236874in}}%
\pgfpathlineto{\pgfqpoint{3.749758in}{1.236618in}}%
\pgfpathlineto{\pgfqpoint{3.763877in}{1.236540in}}%
\pgfpathlineto{\pgfqpoint{3.772190in}{1.247665in}}%
\pgfpathlineto{\pgfqpoint{3.780497in}{1.258973in}}%
\pgfpathlineto{\pgfqpoint{3.788798in}{1.270458in}}%
\pgfpathlineto{\pgfqpoint{3.797092in}{1.282114in}}%
\pgfpathlineto{\pgfqpoint{3.782986in}{1.281493in}}%
\pgfpathlineto{\pgfqpoint{3.768887in}{1.281052in}}%
\pgfpathlineto{\pgfqpoint{3.754797in}{1.280790in}}%
\pgfpathlineto{\pgfqpoint{3.740715in}{1.280708in}}%
\pgfpathlineto{\pgfqpoint{3.732409in}{1.269738in}}%
\pgfpathlineto{\pgfqpoint{3.724096in}{1.258946in}}%
\pgfpathlineto{\pgfqpoint{3.715776in}{1.248339in}}%
\pgfpathlineto{\pgfqpoint{3.707448in}{1.237924in}}%
\pgfpathclose%
\pgfusepath{fill}%
\end{pgfscope}%
\begin{pgfscope}%
\pgfpathrectangle{\pgfqpoint{1.150000in}{0.150000in}}{\pgfqpoint{5.700000in}{5.700000in}}%
\pgfusepath{clip}%
\pgfsetbuttcap%
\pgfsetroundjoin%
\definecolor{currentfill}{rgb}{0.279566,0.067836,0.391917}%
\pgfsetfillcolor{currentfill}%
\pgfsetfillopacity{0.800000}%
\pgfsetlinewidth{0.000000pt}%
\definecolor{currentstroke}{rgb}{0.000000,0.000000,0.000000}%
\pgfsetstrokecolor{currentstroke}%
\pgfsetdash{}{0pt}%
\pgfpathmoveto{\pgfqpoint{3.797092in}{1.282114in}}%
\pgfpathlineto{\pgfqpoint{3.811207in}{1.282914in}}%
\pgfpathlineto{\pgfqpoint{3.825330in}{1.283892in}}%
\pgfpathlineto{\pgfqpoint{3.839463in}{1.285049in}}%
\pgfpathlineto{\pgfqpoint{3.853604in}{1.286385in}}%
\pgfpathlineto{\pgfqpoint{3.861882in}{1.298883in}}%
\pgfpathlineto{\pgfqpoint{3.870154in}{1.311530in}}%
\pgfpathlineto{\pgfqpoint{3.878421in}{1.324320in}}%
\pgfpathlineto{\pgfqpoint{3.886682in}{1.337246in}}%
\pgfpathlineto{\pgfqpoint{3.872549in}{1.335243in}}%
\pgfpathlineto{\pgfqpoint{3.858426in}{1.333418in}}%
\pgfpathlineto{\pgfqpoint{3.844312in}{1.331773in}}%
\pgfpathlineto{\pgfqpoint{3.830207in}{1.330306in}}%
\pgfpathlineto{\pgfqpoint{3.821937in}{1.318036in}}%
\pgfpathlineto{\pgfqpoint{3.813662in}{1.305909in}}%
\pgfpathlineto{\pgfqpoint{3.805380in}{1.293933in}}%
\pgfpathlineto{\pgfqpoint{3.797092in}{1.282114in}}%
\pgfpathclose%
\pgfusepath{fill}%
\end{pgfscope}%
\begin{pgfscope}%
\pgfpathrectangle{\pgfqpoint{1.150000in}{0.150000in}}{\pgfqpoint{5.700000in}{5.700000in}}%
\pgfusepath{clip}%
\pgfsetbuttcap%
\pgfsetroundjoin%
\definecolor{currentfill}{rgb}{0.258965,0.251537,0.524736}%
\pgfsetfillcolor{currentfill}%
\pgfsetfillopacity{0.800000}%
\pgfsetlinewidth{0.000000pt}%
\definecolor{currentstroke}{rgb}{0.000000,0.000000,0.000000}%
\pgfsetstrokecolor{currentstroke}%
\pgfsetdash{}{0pt}%
\pgfpathmoveto{\pgfqpoint{4.221122in}{1.687841in}}%
\pgfpathlineto{\pgfqpoint{4.235394in}{1.695057in}}%
\pgfpathlineto{\pgfqpoint{4.249680in}{1.702454in}}%
\pgfpathlineto{\pgfqpoint{4.263979in}{1.710030in}}%
\pgfpathlineto{\pgfqpoint{4.278292in}{1.717786in}}%
\pgfpathlineto{\pgfqpoint{4.286457in}{1.734229in}}%
\pgfpathlineto{\pgfqpoint{4.294618in}{1.750656in}}%
\pgfpathlineto{\pgfqpoint{4.302775in}{1.767064in}}%
\pgfpathlineto{\pgfqpoint{4.310929in}{1.783447in}}%
\pgfpathlineto{\pgfqpoint{4.296612in}{1.775203in}}%
\pgfpathlineto{\pgfqpoint{4.282309in}{1.767140in}}%
\pgfpathlineto{\pgfqpoint{4.268021in}{1.759258in}}%
\pgfpathlineto{\pgfqpoint{4.253746in}{1.751556in}}%
\pgfpathlineto{\pgfqpoint{4.245596in}{1.735646in}}%
\pgfpathlineto{\pgfqpoint{4.237442in}{1.719721in}}%
\pgfpathlineto{\pgfqpoint{4.229284in}{1.703784in}}%
\pgfpathlineto{\pgfqpoint{4.221122in}{1.687841in}}%
\pgfpathclose%
\pgfusepath{fill}%
\end{pgfscope}%
\begin{pgfscope}%
\pgfpathrectangle{\pgfqpoint{1.150000in}{0.150000in}}{\pgfqpoint{5.700000in}{5.700000in}}%
\pgfusepath{clip}%
\pgfsetbuttcap%
\pgfsetroundjoin%
\definecolor{currentfill}{rgb}{0.272594,0.025563,0.353093}%
\pgfsetfillcolor{currentfill}%
\pgfsetfillopacity{0.800000}%
\pgfsetlinewidth{0.000000pt}%
\definecolor{currentstroke}{rgb}{0.000000,0.000000,0.000000}%
\pgfsetstrokecolor{currentstroke}%
\pgfsetdash{}{0pt}%
\pgfpathmoveto{\pgfqpoint{3.617686in}{1.205493in}}%
\pgfpathlineto{\pgfqpoint{3.631771in}{1.203430in}}%
\pgfpathlineto{\pgfqpoint{3.645863in}{1.201546in}}%
\pgfpathlineto{\pgfqpoint{3.659960in}{1.199843in}}%
\pgfpathlineto{\pgfqpoint{3.674065in}{1.198320in}}%
\pgfpathlineto{\pgfqpoint{3.682422in}{1.207898in}}%
\pgfpathlineto{\pgfqpoint{3.690772in}{1.217696in}}%
\pgfpathlineto{\pgfqpoint{3.699114in}{1.227707in}}%
\pgfpathlineto{\pgfqpoint{3.707448in}{1.237924in}}%
\pgfpathlineto{\pgfqpoint{3.693360in}{1.238719in}}%
\pgfpathlineto{\pgfqpoint{3.679279in}{1.239693in}}%
\pgfpathlineto{\pgfqpoint{3.665204in}{1.240849in}}%
\pgfpathlineto{\pgfqpoint{3.651137in}{1.242185in}}%
\pgfpathlineto{\pgfqpoint{3.642786in}{1.232684in}}%
\pgfpathlineto{\pgfqpoint{3.634428in}{1.223397in}}%
\pgfpathlineto{\pgfqpoint{3.626061in}{1.214331in}}%
\pgfpathlineto{\pgfqpoint{3.617686in}{1.205493in}}%
\pgfpathclose%
\pgfusepath{fill}%
\end{pgfscope}%
\begin{pgfscope}%
\pgfpathrectangle{\pgfqpoint{1.150000in}{0.150000in}}{\pgfqpoint{5.700000in}{5.700000in}}%
\pgfusepath{clip}%
\pgfsetbuttcap%
\pgfsetroundjoin%
\definecolor{currentfill}{rgb}{0.233603,0.313828,0.543914}%
\pgfsetfillcolor{currentfill}%
\pgfsetfillopacity{0.800000}%
\pgfsetlinewidth{0.000000pt}%
\definecolor{currentstroke}{rgb}{0.000000,0.000000,0.000000}%
\pgfsetstrokecolor{currentstroke}%
\pgfsetdash{}{0pt}%
\pgfpathmoveto{\pgfqpoint{4.343504in}{1.848657in}}%
\pgfpathlineto{\pgfqpoint{4.357840in}{1.857538in}}%
\pgfpathlineto{\pgfqpoint{4.372191in}{1.866599in}}%
\pgfpathlineto{\pgfqpoint{4.386556in}{1.875842in}}%
\pgfpathlineto{\pgfqpoint{4.400937in}{1.885266in}}%
\pgfpathlineto{\pgfqpoint{4.409076in}{1.901903in}}%
\pgfpathlineto{\pgfqpoint{4.417211in}{1.918483in}}%
\pgfpathlineto{\pgfqpoint{4.425342in}{1.935002in}}%
\pgfpathlineto{\pgfqpoint{4.433469in}{1.951456in}}%
\pgfpathlineto{\pgfqpoint{4.419083in}{1.941607in}}%
\pgfpathlineto{\pgfqpoint{4.404712in}{1.931940in}}%
\pgfpathlineto{\pgfqpoint{4.390356in}{1.922454in}}%
\pgfpathlineto{\pgfqpoint{4.376016in}{1.913151in}}%
\pgfpathlineto{\pgfqpoint{4.367894in}{1.897108in}}%
\pgfpathlineto{\pgfqpoint{4.359768in}{1.881009in}}%
\pgfpathlineto{\pgfqpoint{4.351638in}{1.864858in}}%
\pgfpathlineto{\pgfqpoint{4.343504in}{1.848657in}}%
\pgfpathclose%
\pgfusepath{fill}%
\end{pgfscope}%
\begin{pgfscope}%
\pgfpathrectangle{\pgfqpoint{1.150000in}{0.150000in}}{\pgfqpoint{5.700000in}{5.700000in}}%
\pgfusepath{clip}%
\pgfsetbuttcap%
\pgfsetroundjoin%
\definecolor{currentfill}{rgb}{0.239374,0.735588,0.455688}%
\pgfsetfillcolor{currentfill}%
\pgfsetfillopacity{0.800000}%
\pgfsetlinewidth{0.000000pt}%
\definecolor{currentstroke}{rgb}{0.000000,0.000000,0.000000}%
\pgfsetstrokecolor{currentstroke}%
\pgfsetdash{}{0pt}%
\pgfpathmoveto{\pgfqpoint{5.322629in}{3.143803in}}%
\pgfpathlineto{\pgfqpoint{5.337631in}{3.161414in}}%
\pgfpathlineto{\pgfqpoint{5.352657in}{3.179216in}}%
\pgfpathlineto{\pgfqpoint{5.367705in}{3.197209in}}%
\pgfpathlineto{\pgfqpoint{5.382778in}{3.215394in}}%
\pgfpathlineto{\pgfqpoint{5.390490in}{3.223214in}}%
\pgfpathlineto{\pgfqpoint{5.398192in}{3.230830in}}%
\pgfpathlineto{\pgfqpoint{5.405883in}{3.238244in}}%
\pgfpathlineto{\pgfqpoint{5.413563in}{3.245459in}}%
\pgfpathlineto{\pgfqpoint{5.398498in}{3.227403in}}%
\pgfpathlineto{\pgfqpoint{5.383455in}{3.209539in}}%
\pgfpathlineto{\pgfqpoint{5.368437in}{3.191865in}}%
\pgfpathlineto{\pgfqpoint{5.353441in}{3.174382in}}%
\pgfpathlineto{\pgfqpoint{5.345753in}{3.167025in}}%
\pgfpathlineto{\pgfqpoint{5.338056in}{3.159478in}}%
\pgfpathlineto{\pgfqpoint{5.330347in}{3.151737in}}%
\pgfpathlineto{\pgfqpoint{5.322629in}{3.143803in}}%
\pgfpathclose%
\pgfusepath{fill}%
\end{pgfscope}%
\begin{pgfscope}%
\pgfpathrectangle{\pgfqpoint{1.150000in}{0.150000in}}{\pgfqpoint{5.700000in}{5.700000in}}%
\pgfusepath{clip}%
\pgfsetbuttcap%
\pgfsetroundjoin%
\definecolor{currentfill}{rgb}{0.282656,0.100196,0.422160}%
\pgfsetfillcolor{currentfill}%
\pgfsetfillopacity{0.800000}%
\pgfsetlinewidth{0.000000pt}%
\definecolor{currentstroke}{rgb}{0.000000,0.000000,0.000000}%
\pgfsetstrokecolor{currentstroke}%
\pgfsetdash{}{0pt}%
\pgfpathmoveto{\pgfqpoint{3.886682in}{1.337246in}}%
\pgfpathlineto{\pgfqpoint{3.900824in}{1.339428in}}%
\pgfpathlineto{\pgfqpoint{3.914976in}{1.341788in}}%
\pgfpathlineto{\pgfqpoint{3.929137in}{1.344327in}}%
\pgfpathlineto{\pgfqpoint{3.943309in}{1.347043in}}%
\pgfpathlineto{\pgfqpoint{3.951558in}{1.360749in}}%
\pgfpathlineto{\pgfqpoint{3.959803in}{1.374571in}}%
\pgfpathlineto{\pgfqpoint{3.968042in}{1.388502in}}%
\pgfpathlineto{\pgfqpoint{3.976277in}{1.402537in}}%
\pgfpathlineto{\pgfqpoint{3.962110in}{1.399182in}}%
\pgfpathlineto{\pgfqpoint{3.947955in}{1.396006in}}%
\pgfpathlineto{\pgfqpoint{3.933809in}{1.393008in}}%
\pgfpathlineto{\pgfqpoint{3.919673in}{1.390189in}}%
\pgfpathlineto{\pgfqpoint{3.911433in}{1.376780in}}%
\pgfpathlineto{\pgfqpoint{3.903188in}{1.363482in}}%
\pgfpathlineto{\pgfqpoint{3.894938in}{1.350302in}}%
\pgfpathlineto{\pgfqpoint{3.886682in}{1.337246in}}%
\pgfpathclose%
\pgfusepath{fill}%
\end{pgfscope}%
\begin{pgfscope}%
\pgfpathrectangle{\pgfqpoint{1.150000in}{0.150000in}}{\pgfqpoint{5.700000in}{5.700000in}}%
\pgfusepath{clip}%
\pgfsetbuttcap%
\pgfsetroundjoin%
\definecolor{currentfill}{rgb}{0.276194,0.190074,0.493001}%
\pgfsetfillcolor{currentfill}%
\pgfsetfillopacity{0.800000}%
\pgfsetlinewidth{0.000000pt}%
\definecolor{currentstroke}{rgb}{0.000000,0.000000,0.000000}%
\pgfsetstrokecolor{currentstroke}%
\pgfsetdash{}{0pt}%
\pgfpathmoveto{\pgfqpoint{4.098741in}{1.537799in}}%
\pgfpathlineto{\pgfqpoint{4.112960in}{1.543233in}}%
\pgfpathlineto{\pgfqpoint{4.127190in}{1.548846in}}%
\pgfpathlineto{\pgfqpoint{4.141433in}{1.554638in}}%
\pgfpathlineto{\pgfqpoint{4.155688in}{1.560608in}}%
\pgfpathlineto{\pgfqpoint{4.163881in}{1.576436in}}%
\pgfpathlineto{\pgfqpoint{4.172070in}{1.592295in}}%
\pgfpathlineto{\pgfqpoint{4.180255in}{1.608181in}}%
\pgfpathlineto{\pgfqpoint{4.188436in}{1.624089in}}%
\pgfpathlineto{\pgfqpoint{4.174180in}{1.617570in}}%
\pgfpathlineto{\pgfqpoint{4.159936in}{1.611230in}}%
\pgfpathlineto{\pgfqpoint{4.145705in}{1.605070in}}%
\pgfpathlineto{\pgfqpoint{4.131487in}{1.599089in}}%
\pgfpathlineto{\pgfqpoint{4.123307in}{1.583717in}}%
\pgfpathlineto{\pgfqpoint{4.115122in}{1.568374in}}%
\pgfpathlineto{\pgfqpoint{4.106934in}{1.553067in}}%
\pgfpathlineto{\pgfqpoint{4.098741in}{1.537799in}}%
\pgfpathclose%
\pgfusepath{fill}%
\end{pgfscope}%
\begin{pgfscope}%
\pgfpathrectangle{\pgfqpoint{1.150000in}{0.150000in}}{\pgfqpoint{5.700000in}{5.700000in}}%
\pgfusepath{clip}%
\pgfsetbuttcap%
\pgfsetroundjoin%
\definecolor{currentfill}{rgb}{0.269944,0.014625,0.341379}%
\pgfsetfillcolor{currentfill}%
\pgfsetfillopacity{0.800000}%
\pgfsetlinewidth{0.000000pt}%
\definecolor{currentstroke}{rgb}{0.000000,0.000000,0.000000}%
\pgfsetstrokecolor{currentstroke}%
\pgfsetdash{}{0pt}%
\pgfpathmoveto{\pgfqpoint{3.381193in}{1.201512in}}%
\pgfpathlineto{\pgfqpoint{3.395269in}{1.195697in}}%
\pgfpathlineto{\pgfqpoint{3.409348in}{1.190069in}}%
\pgfpathlineto{\pgfqpoint{3.423431in}{1.184626in}}%
\pgfpathlineto{\pgfqpoint{3.437517in}{1.179368in}}%
\pgfpathlineto{\pgfqpoint{3.446018in}{1.184501in}}%
\pgfpathlineto{\pgfqpoint{3.454507in}{1.189939in}}%
\pgfpathlineto{\pgfqpoint{3.462985in}{1.195676in}}%
\pgfpathlineto{\pgfqpoint{3.471452in}{1.201702in}}%
\pgfpathlineto{\pgfqpoint{3.457393in}{1.206168in}}%
\pgfpathlineto{\pgfqpoint{3.443337in}{1.210819in}}%
\pgfpathlineto{\pgfqpoint{3.429286in}{1.215655in}}%
\pgfpathlineto{\pgfqpoint{3.415239in}{1.220677in}}%
\pgfpathlineto{\pgfqpoint{3.406745in}{1.215430in}}%
\pgfpathlineto{\pgfqpoint{3.398240in}{1.210482in}}%
\pgfpathlineto{\pgfqpoint{3.389722in}{1.205840in}}%
\pgfpathlineto{\pgfqpoint{3.381193in}{1.201512in}}%
\pgfpathclose%
\pgfusepath{fill}%
\end{pgfscope}%
\begin{pgfscope}%
\pgfpathrectangle{\pgfqpoint{1.150000in}{0.150000in}}{\pgfqpoint{5.700000in}{5.700000in}}%
\pgfusepath{clip}%
\pgfsetbuttcap%
\pgfsetroundjoin%
\definecolor{currentfill}{rgb}{0.203063,0.379716,0.553925}%
\pgfsetfillcolor{currentfill}%
\pgfsetfillopacity{0.800000}%
\pgfsetlinewidth{0.000000pt}%
\definecolor{currentstroke}{rgb}{0.000000,0.000000,0.000000}%
\pgfsetstrokecolor{currentstroke}%
\pgfsetdash{}{0pt}%
\pgfpathmoveto{\pgfqpoint{4.465935in}{2.016554in}}%
\pgfpathlineto{\pgfqpoint{4.480343in}{2.026978in}}%
\pgfpathlineto{\pgfqpoint{4.494767in}{2.037585in}}%
\pgfpathlineto{\pgfqpoint{4.509208in}{2.048374in}}%
\pgfpathlineto{\pgfqpoint{4.523665in}{2.059346in}}%
\pgfpathlineto{\pgfqpoint{4.531777in}{2.075796in}}%
\pgfpathlineto{\pgfqpoint{4.539884in}{2.092152in}}%
\pgfpathlineto{\pgfqpoint{4.547987in}{2.108411in}}%
\pgfpathlineto{\pgfqpoint{4.556085in}{2.124569in}}%
\pgfpathlineto{\pgfqpoint{4.541622in}{2.113236in}}%
\pgfpathlineto{\pgfqpoint{4.527175in}{2.102085in}}%
\pgfpathlineto{\pgfqpoint{4.512745in}{2.091119in}}%
\pgfpathlineto{\pgfqpoint{4.498331in}{2.080335in}}%
\pgfpathlineto{\pgfqpoint{4.490238in}{2.064525in}}%
\pgfpathlineto{\pgfqpoint{4.482142in}{2.048622in}}%
\pgfpathlineto{\pgfqpoint{4.474041in}{2.032631in}}%
\pgfpathlineto{\pgfqpoint{4.465935in}{2.016554in}}%
\pgfpathclose%
\pgfusepath{fill}%
\end{pgfscope}%
\begin{pgfscope}%
\pgfpathrectangle{\pgfqpoint{1.150000in}{0.150000in}}{\pgfqpoint{5.700000in}{5.700000in}}%
\pgfusepath{clip}%
\pgfsetbuttcap%
\pgfsetroundjoin%
\definecolor{currentfill}{rgb}{0.386433,0.794644,0.372886}%
\pgfsetfillcolor{currentfill}%
\pgfsetfillopacity{0.800000}%
\pgfsetlinewidth{0.000000pt}%
\definecolor{currentstroke}{rgb}{0.000000,0.000000,0.000000}%
\pgfsetstrokecolor{currentstroke}%
\pgfsetdash{}{0pt}%
\pgfpathmoveto{\pgfqpoint{5.535033in}{3.368892in}}%
\pgfpathlineto{\pgfqpoint{5.550200in}{3.387537in}}%
\pgfpathlineto{\pgfqpoint{5.565390in}{3.406374in}}%
\pgfpathlineto{\pgfqpoint{5.580606in}{3.425404in}}%
\pgfpathlineto{\pgfqpoint{5.595846in}{3.444626in}}%
\pgfpathlineto{\pgfqpoint{5.603403in}{3.449686in}}%
\pgfpathlineto{\pgfqpoint{5.610948in}{3.454551in}}%
\pgfpathlineto{\pgfqpoint{5.618481in}{3.459224in}}%
\pgfpathlineto{\pgfqpoint{5.626003in}{3.463709in}}%
\pgfpathlineto{\pgfqpoint{5.610775in}{3.444729in}}%
\pgfpathlineto{\pgfqpoint{5.595573in}{3.425940in}}%
\pgfpathlineto{\pgfqpoint{5.580395in}{3.407344in}}%
\pgfpathlineto{\pgfqpoint{5.565242in}{3.388939in}}%
\pgfpathlineto{\pgfqpoint{5.557707in}{3.384200in}}%
\pgfpathlineto{\pgfqpoint{5.550161in}{3.379281in}}%
\pgfpathlineto{\pgfqpoint{5.542603in}{3.374180in}}%
\pgfpathlineto{\pgfqpoint{5.535033in}{3.368892in}}%
\pgfpathclose%
\pgfusepath{fill}%
\end{pgfscope}%
\begin{pgfscope}%
\pgfpathrectangle{\pgfqpoint{1.150000in}{0.150000in}}{\pgfqpoint{5.700000in}{5.700000in}}%
\pgfusepath{clip}%
\pgfsetbuttcap%
\pgfsetroundjoin%
\definecolor{currentfill}{rgb}{0.177423,0.437527,0.557565}%
\pgfsetfillcolor{currentfill}%
\pgfsetfillopacity{0.800000}%
\pgfsetlinewidth{0.000000pt}%
\definecolor{currentstroke}{rgb}{0.000000,0.000000,0.000000}%
\pgfsetstrokecolor{currentstroke}%
\pgfsetdash{}{0pt}%
\pgfpathmoveto{\pgfqpoint{4.588431in}{2.188138in}}%
\pgfpathlineto{\pgfqpoint{4.602918in}{2.199983in}}%
\pgfpathlineto{\pgfqpoint{4.617422in}{2.212012in}}%
\pgfpathlineto{\pgfqpoint{4.631944in}{2.224226in}}%
\pgfpathlineto{\pgfqpoint{4.646484in}{2.236624in}}%
\pgfpathlineto{\pgfqpoint{4.654564in}{2.252544in}}%
\pgfpathlineto{\pgfqpoint{4.662640in}{2.268338in}}%
\pgfpathlineto{\pgfqpoint{4.670710in}{2.284003in}}%
\pgfpathlineto{\pgfqpoint{4.678775in}{2.299537in}}%
\pgfpathlineto{\pgfqpoint{4.664228in}{2.286844in}}%
\pgfpathlineto{\pgfqpoint{4.649700in}{2.274335in}}%
\pgfpathlineto{\pgfqpoint{4.635189in}{2.262011in}}%
\pgfpathlineto{\pgfqpoint{4.620696in}{2.249871in}}%
\pgfpathlineto{\pgfqpoint{4.612638in}{2.234619in}}%
\pgfpathlineto{\pgfqpoint{4.604574in}{2.219245in}}%
\pgfpathlineto{\pgfqpoint{4.596505in}{2.203750in}}%
\pgfpathlineto{\pgfqpoint{4.588431in}{2.188138in}}%
\pgfpathclose%
\pgfusepath{fill}%
\end{pgfscope}%
\begin{pgfscope}%
\pgfpathrectangle{\pgfqpoint{1.150000in}{0.150000in}}{\pgfqpoint{5.700000in}{5.700000in}}%
\pgfusepath{clip}%
\pgfsetbuttcap%
\pgfsetroundjoin%
\definecolor{currentfill}{rgb}{0.269944,0.014625,0.341379}%
\pgfsetfillcolor{currentfill}%
\pgfsetfillopacity{0.800000}%
\pgfsetlinewidth{0.000000pt}%
\definecolor{currentstroke}{rgb}{0.000000,0.000000,0.000000}%
\pgfsetstrokecolor{currentstroke}%
\pgfsetdash{}{0pt}%
\pgfpathmoveto{\pgfqpoint{3.527735in}{1.185676in}}%
\pgfpathlineto{\pgfqpoint{3.541818in}{1.182127in}}%
\pgfpathlineto{\pgfqpoint{3.555906in}{1.178760in}}%
\pgfpathlineto{\pgfqpoint{3.570000in}{1.175574in}}%
\pgfpathlineto{\pgfqpoint{3.584099in}{1.172569in}}%
\pgfpathlineto{\pgfqpoint{3.592509in}{1.180421in}}%
\pgfpathlineto{\pgfqpoint{3.600911in}{1.188531in}}%
\pgfpathlineto{\pgfqpoint{3.609303in}{1.196891in}}%
\pgfpathlineto{\pgfqpoint{3.617686in}{1.205493in}}%
\pgfpathlineto{\pgfqpoint{3.603608in}{1.207738in}}%
\pgfpathlineto{\pgfqpoint{3.589535in}{1.210164in}}%
\pgfpathlineto{\pgfqpoint{3.575468in}{1.212772in}}%
\pgfpathlineto{\pgfqpoint{3.561407in}{1.215562in}}%
\pgfpathlineto{\pgfqpoint{3.553003in}{1.207707in}}%
\pgfpathlineto{\pgfqpoint{3.544590in}{1.200102in}}%
\pgfpathlineto{\pgfqpoint{3.536168in}{1.192756in}}%
\pgfpathlineto{\pgfqpoint{3.527735in}{1.185676in}}%
\pgfpathclose%
\pgfusepath{fill}%
\end{pgfscope}%
\begin{pgfscope}%
\pgfpathrectangle{\pgfqpoint{1.150000in}{0.150000in}}{\pgfqpoint{5.700000in}{5.700000in}}%
\pgfusepath{clip}%
\pgfsetbuttcap%
\pgfsetroundjoin%
\definecolor{currentfill}{rgb}{0.130067,0.651384,0.521608}%
\pgfsetfillcolor{currentfill}%
\pgfsetfillopacity{0.800000}%
\pgfsetlinewidth{0.000000pt}%
\definecolor{currentstroke}{rgb}{0.000000,0.000000,0.000000}%
\pgfsetstrokecolor{currentstroke}%
\pgfsetdash{}{0pt}%
\pgfpathmoveto{\pgfqpoint{1.898340in}{3.015447in}}%
\pgfpathlineto{\pgfqpoint{1.913209in}{2.983618in}}%
\pgfpathlineto{\pgfqpoint{1.928056in}{2.952156in}}%
\pgfpathlineto{\pgfqpoint{1.942881in}{2.921058in}}%
\pgfpathlineto{\pgfqpoint{1.957684in}{2.890319in}}%
\pgfpathlineto{\pgfqpoint{1.967698in}{2.873436in}}%
\pgfpathlineto{\pgfqpoint{1.977673in}{2.857117in}}%
\pgfpathlineto{\pgfqpoint{1.987612in}{2.841351in}}%
\pgfpathlineto{\pgfqpoint{1.997514in}{2.826128in}}%
\pgfpathlineto{\pgfqpoint{1.982802in}{2.855945in}}%
\pgfpathlineto{\pgfqpoint{1.968070in}{2.886119in}}%
\pgfpathlineto{\pgfqpoint{1.953317in}{2.916653in}}%
\pgfpathlineto{\pgfqpoint{1.938542in}{2.947551in}}%
\pgfpathlineto{\pgfqpoint{1.928549in}{2.963681in}}%
\pgfpathlineto{\pgfqpoint{1.918519in}{2.980368in}}%
\pgfpathlineto{\pgfqpoint{1.908449in}{2.997620in}}%
\pgfpathlineto{\pgfqpoint{1.898340in}{3.015447in}}%
\pgfpathclose%
\pgfusepath{fill}%
\end{pgfscope}%
\begin{pgfscope}%
\pgfpathrectangle{\pgfqpoint{1.150000in}{0.150000in}}{\pgfqpoint{5.700000in}{5.700000in}}%
\pgfusepath{clip}%
\pgfsetbuttcap%
\pgfsetroundjoin%
\definecolor{currentfill}{rgb}{0.283072,0.130895,0.449241}%
\pgfsetfillcolor{currentfill}%
\pgfsetfillopacity{0.800000}%
\pgfsetlinewidth{0.000000pt}%
\definecolor{currentstroke}{rgb}{0.000000,0.000000,0.000000}%
\pgfsetstrokecolor{currentstroke}%
\pgfsetdash{}{0pt}%
\pgfpathmoveto{\pgfqpoint{3.976277in}{1.402537in}}%
\pgfpathlineto{\pgfqpoint{3.990454in}{1.406071in}}%
\pgfpathlineto{\pgfqpoint{4.004641in}{1.409782in}}%
\pgfpathlineto{\pgfqpoint{4.018840in}{1.413672in}}%
\pgfpathlineto{\pgfqpoint{4.033049in}{1.417739in}}%
\pgfpathlineto{\pgfqpoint{4.041276in}{1.432493in}}%
\pgfpathlineto{\pgfqpoint{4.049498in}{1.447330in}}%
\pgfpathlineto{\pgfqpoint{4.057716in}{1.462245in}}%
\pgfpathlineto{\pgfqpoint{4.065930in}{1.477233in}}%
\pgfpathlineto{\pgfqpoint{4.051722in}{1.472556in}}%
\pgfpathlineto{\pgfqpoint{4.037526in}{1.468058in}}%
\pgfpathlineto{\pgfqpoint{4.023342in}{1.463738in}}%
\pgfpathlineto{\pgfqpoint{4.009168in}{1.459597in}}%
\pgfpathlineto{\pgfqpoint{4.000952in}{1.445206in}}%
\pgfpathlineto{\pgfqpoint{3.992732in}{1.430895in}}%
\pgfpathlineto{\pgfqpoint{3.984507in}{1.416670in}}%
\pgfpathlineto{\pgfqpoint{3.976277in}{1.402537in}}%
\pgfpathclose%
\pgfusepath{fill}%
\end{pgfscope}%
\begin{pgfscope}%
\pgfpathrectangle{\pgfqpoint{1.150000in}{0.150000in}}{\pgfqpoint{5.700000in}{5.700000in}}%
\pgfusepath{clip}%
\pgfsetbuttcap%
\pgfsetroundjoin%
\definecolor{currentfill}{rgb}{0.170948,0.694384,0.493803}%
\pgfsetfillcolor{currentfill}%
\pgfsetfillopacity{0.800000}%
\pgfsetlinewidth{0.000000pt}%
\definecolor{currentstroke}{rgb}{0.000000,0.000000,0.000000}%
\pgfsetstrokecolor{currentstroke}%
\pgfsetdash{}{0pt}%
\pgfpathmoveto{\pgfqpoint{5.200700in}{3.004027in}}%
\pgfpathlineto{\pgfqpoint{5.215620in}{3.021022in}}%
\pgfpathlineto{\pgfqpoint{5.230561in}{3.038208in}}%
\pgfpathlineto{\pgfqpoint{5.245526in}{3.055585in}}%
\pgfpathlineto{\pgfqpoint{5.260513in}{3.073153in}}%
\pgfpathlineto{\pgfqpoint{5.268313in}{3.082693in}}%
\pgfpathlineto{\pgfqpoint{5.276103in}{3.092029in}}%
\pgfpathlineto{\pgfqpoint{5.283882in}{3.101161in}}%
\pgfpathlineto{\pgfqpoint{5.291652in}{3.110090in}}%
\pgfpathlineto{\pgfqpoint{5.276668in}{3.092577in}}%
\pgfpathlineto{\pgfqpoint{5.261707in}{3.075256in}}%
\pgfpathlineto{\pgfqpoint{5.246768in}{3.058125in}}%
\pgfpathlineto{\pgfqpoint{5.231851in}{3.041184in}}%
\pgfpathlineto{\pgfqpoint{5.224078in}{3.032186in}}%
\pgfpathlineto{\pgfqpoint{5.216295in}{3.022995in}}%
\pgfpathlineto{\pgfqpoint{5.208502in}{3.013609in}}%
\pgfpathlineto{\pgfqpoint{5.200700in}{3.004027in}}%
\pgfpathclose%
\pgfusepath{fill}%
\end{pgfscope}%
\begin{pgfscope}%
\pgfpathrectangle{\pgfqpoint{1.150000in}{0.150000in}}{\pgfqpoint{5.700000in}{5.700000in}}%
\pgfusepath{clip}%
\pgfsetbuttcap%
\pgfsetroundjoin%
\definecolor{currentfill}{rgb}{0.154815,0.493313,0.557840}%
\pgfsetfillcolor{currentfill}%
\pgfsetfillopacity{0.800000}%
\pgfsetlinewidth{0.000000pt}%
\definecolor{currentstroke}{rgb}{0.000000,0.000000,0.000000}%
\pgfsetstrokecolor{currentstroke}%
\pgfsetdash{}{0pt}%
\pgfpathmoveto{\pgfqpoint{4.710979in}{2.360318in}}%
\pgfpathlineto{\pgfqpoint{4.725550in}{2.373459in}}%
\pgfpathlineto{\pgfqpoint{4.740139in}{2.386785in}}%
\pgfpathlineto{\pgfqpoint{4.754747in}{2.400297in}}%
\pgfpathlineto{\pgfqpoint{4.769374in}{2.413995in}}%
\pgfpathlineto{\pgfqpoint{4.777417in}{2.429081in}}%
\pgfpathlineto{\pgfqpoint{4.785454in}{2.444014in}}%
\pgfpathlineto{\pgfqpoint{4.793485in}{2.458793in}}%
\pgfpathlineto{\pgfqpoint{4.801510in}{2.473415in}}%
\pgfpathlineto{\pgfqpoint{4.786877in}{2.459489in}}%
\pgfpathlineto{\pgfqpoint{4.772263in}{2.445749in}}%
\pgfpathlineto{\pgfqpoint{4.757668in}{2.432195in}}%
\pgfpathlineto{\pgfqpoint{4.743092in}{2.418827in}}%
\pgfpathlineto{\pgfqpoint{4.735072in}{2.404419in}}%
\pgfpathlineto{\pgfqpoint{4.727047in}{2.389864in}}%
\pgfpathlineto{\pgfqpoint{4.719016in}{2.375163in}}%
\pgfpathlineto{\pgfqpoint{4.710979in}{2.360318in}}%
\pgfpathclose%
\pgfusepath{fill}%
\end{pgfscope}%
\begin{pgfscope}%
\pgfpathrectangle{\pgfqpoint{1.150000in}{0.150000in}}{\pgfqpoint{5.700000in}{5.700000in}}%
\pgfusepath{clip}%
\pgfsetbuttcap%
\pgfsetroundjoin%
\definecolor{currentfill}{rgb}{0.130067,0.651384,0.521608}%
\pgfsetfillcolor{currentfill}%
\pgfsetfillopacity{0.800000}%
\pgfsetlinewidth{0.000000pt}%
\definecolor{currentstroke}{rgb}{0.000000,0.000000,0.000000}%
\pgfsetstrokecolor{currentstroke}%
\pgfsetdash{}{0pt}%
\pgfpathmoveto{\pgfqpoint{5.078483in}{2.854146in}}%
\pgfpathlineto{\pgfqpoint{5.093317in}{2.870384in}}%
\pgfpathlineto{\pgfqpoint{5.108171in}{2.886811in}}%
\pgfpathlineto{\pgfqpoint{5.123048in}{2.903429in}}%
\pgfpathlineto{\pgfqpoint{5.137947in}{2.920236in}}%
\pgfpathlineto{\pgfqpoint{5.145823in}{2.931408in}}%
\pgfpathlineto{\pgfqpoint{5.153690in}{2.942380in}}%
\pgfpathlineto{\pgfqpoint{5.161548in}{2.953152in}}%
\pgfpathlineto{\pgfqpoint{5.169397in}{2.963724in}}%
\pgfpathlineto{\pgfqpoint{5.154499in}{2.946900in}}%
\pgfpathlineto{\pgfqpoint{5.139622in}{2.930265in}}%
\pgfpathlineto{\pgfqpoint{5.124767in}{2.913821in}}%
\pgfpathlineto{\pgfqpoint{5.109934in}{2.897565in}}%
\pgfpathlineto{\pgfqpoint{5.102084in}{2.886997in}}%
\pgfpathlineto{\pgfqpoint{5.094226in}{2.876238in}}%
\pgfpathlineto{\pgfqpoint{5.086359in}{2.865287in}}%
\pgfpathlineto{\pgfqpoint{5.078483in}{2.854146in}}%
\pgfpathclose%
\pgfusepath{fill}%
\end{pgfscope}%
\begin{pgfscope}%
\pgfpathrectangle{\pgfqpoint{1.150000in}{0.150000in}}{\pgfqpoint{5.700000in}{5.700000in}}%
\pgfusepath{clip}%
\pgfsetbuttcap%
\pgfsetroundjoin%
\definecolor{currentfill}{rgb}{0.133743,0.548535,0.553541}%
\pgfsetfillcolor{currentfill}%
\pgfsetfillopacity{0.800000}%
\pgfsetlinewidth{0.000000pt}%
\definecolor{currentstroke}{rgb}{0.000000,0.000000,0.000000}%
\pgfsetstrokecolor{currentstroke}%
\pgfsetdash{}{0pt}%
\pgfpathmoveto{\pgfqpoint{4.833544in}{2.530308in}}%
\pgfpathlineto{\pgfqpoint{4.848202in}{2.544615in}}%
\pgfpathlineto{\pgfqpoint{4.862879in}{2.559109in}}%
\pgfpathlineto{\pgfqpoint{4.877576in}{2.573790in}}%
\pgfpathlineto{\pgfqpoint{4.892292in}{2.588658in}}%
\pgfpathlineto{\pgfqpoint{4.900290in}{2.602647in}}%
\pgfpathlineto{\pgfqpoint{4.908280in}{2.616462in}}%
\pgfpathlineto{\pgfqpoint{4.916263in}{2.630102in}}%
\pgfpathlineto{\pgfqpoint{4.924239in}{2.643565in}}%
\pgfpathlineto{\pgfqpoint{4.909518in}{2.628536in}}%
\pgfpathlineto{\pgfqpoint{4.894816in}{2.613696in}}%
\pgfpathlineto{\pgfqpoint{4.880135in}{2.599043in}}%
\pgfpathlineto{\pgfqpoint{4.865473in}{2.584577in}}%
\pgfpathlineto{\pgfqpoint{4.857501in}{2.571260in}}%
\pgfpathlineto{\pgfqpoint{4.849522in}{2.557775in}}%
\pgfpathlineto{\pgfqpoint{4.841537in}{2.544124in}}%
\pgfpathlineto{\pgfqpoint{4.833544in}{2.530308in}}%
\pgfpathclose%
\pgfusepath{fill}%
\end{pgfscope}%
\begin{pgfscope}%
\pgfpathrectangle{\pgfqpoint{1.150000in}{0.150000in}}{\pgfqpoint{5.700000in}{5.700000in}}%
\pgfusepath{clip}%
\pgfsetbuttcap%
\pgfsetroundjoin%
\definecolor{currentfill}{rgb}{0.120092,0.600104,0.542530}%
\pgfsetfillcolor{currentfill}%
\pgfsetfillopacity{0.800000}%
\pgfsetlinewidth{0.000000pt}%
\definecolor{currentstroke}{rgb}{0.000000,0.000000,0.000000}%
\pgfsetstrokecolor{currentstroke}%
\pgfsetdash{}{0pt}%
\pgfpathmoveto{\pgfqpoint{4.956070in}{2.695634in}}%
\pgfpathlineto{\pgfqpoint{4.970815in}{2.710975in}}%
\pgfpathlineto{\pgfqpoint{4.985582in}{2.726503in}}%
\pgfpathlineto{\pgfqpoint{5.000368in}{2.742221in}}%
\pgfpathlineto{\pgfqpoint{5.015176in}{2.758127in}}%
\pgfpathlineto{\pgfqpoint{5.023118in}{2.770797in}}%
\pgfpathlineto{\pgfqpoint{5.031052in}{2.783277in}}%
\pgfpathlineto{\pgfqpoint{5.038978in}{2.795567in}}%
\pgfpathlineto{\pgfqpoint{5.046896in}{2.807665in}}%
\pgfpathlineto{\pgfqpoint{5.032085in}{2.791670in}}%
\pgfpathlineto{\pgfqpoint{5.017296in}{2.775863in}}%
\pgfpathlineto{\pgfqpoint{5.002527in}{2.760245in}}%
\pgfpathlineto{\pgfqpoint{4.987779in}{2.744816in}}%
\pgfpathlineto{\pgfqpoint{4.979863in}{2.732793in}}%
\pgfpathlineto{\pgfqpoint{4.971940in}{2.720588in}}%
\pgfpathlineto{\pgfqpoint{4.964009in}{2.708202in}}%
\pgfpathlineto{\pgfqpoint{4.956070in}{2.695634in}}%
\pgfpathclose%
\pgfusepath{fill}%
\end{pgfscope}%
\begin{pgfscope}%
\pgfpathrectangle{\pgfqpoint{1.150000in}{0.150000in}}{\pgfqpoint{5.700000in}{5.700000in}}%
\pgfusepath{clip}%
\pgfsetbuttcap%
\pgfsetroundjoin%
\definecolor{currentfill}{rgb}{0.265145,0.232956,0.516599}%
\pgfsetfillcolor{currentfill}%
\pgfsetfillopacity{0.800000}%
\pgfsetlinewidth{0.000000pt}%
\definecolor{currentstroke}{rgb}{0.000000,0.000000,0.000000}%
\pgfsetstrokecolor{currentstroke}%
\pgfsetdash{}{0pt}%
\pgfpathmoveto{\pgfqpoint{4.188436in}{1.624089in}}%
\pgfpathlineto{\pgfqpoint{4.202706in}{1.630788in}}%
\pgfpathlineto{\pgfqpoint{4.216988in}{1.637666in}}%
\pgfpathlineto{\pgfqpoint{4.231284in}{1.644724in}}%
\pgfpathlineto{\pgfqpoint{4.245594in}{1.651961in}}%
\pgfpathlineto{\pgfqpoint{4.253774in}{1.668416in}}%
\pgfpathlineto{\pgfqpoint{4.261950in}{1.684875in}}%
\pgfpathlineto{\pgfqpoint{4.270123in}{1.701334in}}%
\pgfpathlineto{\pgfqpoint{4.278292in}{1.717786in}}%
\pgfpathlineto{\pgfqpoint{4.263979in}{1.710030in}}%
\pgfpathlineto{\pgfqpoint{4.249680in}{1.702454in}}%
\pgfpathlineto{\pgfqpoint{4.235394in}{1.695057in}}%
\pgfpathlineto{\pgfqpoint{4.221122in}{1.687841in}}%
\pgfpathlineto{\pgfqpoint{4.212957in}{1.671895in}}%
\pgfpathlineto{\pgfqpoint{4.204787in}{1.655951in}}%
\pgfpathlineto{\pgfqpoint{4.196614in}{1.640014in}}%
\pgfpathlineto{\pgfqpoint{4.188436in}{1.624089in}}%
\pgfpathclose%
\pgfusepath{fill}%
\end{pgfscope}%
\begin{pgfscope}%
\pgfpathrectangle{\pgfqpoint{1.150000in}{0.150000in}}{\pgfqpoint{5.700000in}{5.700000in}}%
\pgfusepath{clip}%
\pgfsetbuttcap%
\pgfsetroundjoin%
\definecolor{currentfill}{rgb}{0.241237,0.296485,0.539709}%
\pgfsetfillcolor{currentfill}%
\pgfsetfillopacity{0.800000}%
\pgfsetlinewidth{0.000000pt}%
\definecolor{currentstroke}{rgb}{0.000000,0.000000,0.000000}%
\pgfsetstrokecolor{currentstroke}%
\pgfsetdash{}{0pt}%
\pgfpathmoveto{\pgfqpoint{4.310929in}{1.783447in}}%
\pgfpathlineto{\pgfqpoint{4.325260in}{1.791872in}}%
\pgfpathlineto{\pgfqpoint{4.339606in}{1.800477in}}%
\pgfpathlineto{\pgfqpoint{4.353966in}{1.809263in}}%
\pgfpathlineto{\pgfqpoint{4.368342in}{1.818229in}}%
\pgfpathlineto{\pgfqpoint{4.376496in}{1.835054in}}%
\pgfpathlineto{\pgfqpoint{4.384647in}{1.851837in}}%
\pgfpathlineto{\pgfqpoint{4.392794in}{1.868576in}}%
\pgfpathlineto{\pgfqpoint{4.400937in}{1.885266in}}%
\pgfpathlineto{\pgfqpoint{4.386556in}{1.875842in}}%
\pgfpathlineto{\pgfqpoint{4.372191in}{1.866599in}}%
\pgfpathlineto{\pgfqpoint{4.357840in}{1.857538in}}%
\pgfpathlineto{\pgfqpoint{4.343504in}{1.848657in}}%
\pgfpathlineto{\pgfqpoint{4.335366in}{1.832412in}}%
\pgfpathlineto{\pgfqpoint{4.327224in}{1.816126in}}%
\pgfpathlineto{\pgfqpoint{4.319079in}{1.799803in}}%
\pgfpathlineto{\pgfqpoint{4.310929in}{1.783447in}}%
\pgfpathclose%
\pgfusepath{fill}%
\end{pgfscope}%
\begin{pgfscope}%
\pgfpathrectangle{\pgfqpoint{1.150000in}{0.150000in}}{\pgfqpoint{5.700000in}{5.700000in}}%
\pgfusepath{clip}%
\pgfsetbuttcap%
\pgfsetroundjoin%
\definecolor{currentfill}{rgb}{0.311925,0.767822,0.415586}%
\pgfsetfillcolor{currentfill}%
\pgfsetfillopacity{0.800000}%
\pgfsetlinewidth{0.000000pt}%
\definecolor{currentstroke}{rgb}{0.000000,0.000000,0.000000}%
\pgfsetstrokecolor{currentstroke}%
\pgfsetdash{}{0pt}%
\pgfpathmoveto{\pgfqpoint{5.413563in}{3.245459in}}%
\pgfpathlineto{\pgfqpoint{5.428653in}{3.263706in}}%
\pgfpathlineto{\pgfqpoint{5.443766in}{3.282145in}}%
\pgfpathlineto{\pgfqpoint{5.458903in}{3.300777in}}%
\pgfpathlineto{\pgfqpoint{5.474064in}{3.319601in}}%
\pgfpathlineto{\pgfqpoint{5.481726in}{3.326464in}}%
\pgfpathlineto{\pgfqpoint{5.489375in}{3.333122in}}%
\pgfpathlineto{\pgfqpoint{5.497014in}{3.339578in}}%
\pgfpathlineto{\pgfqpoint{5.504641in}{3.345833in}}%
\pgfpathlineto{\pgfqpoint{5.489488in}{3.327176in}}%
\pgfpathlineto{\pgfqpoint{5.474359in}{3.308712in}}%
\pgfpathlineto{\pgfqpoint{5.459255in}{3.290439in}}%
\pgfpathlineto{\pgfqpoint{5.444174in}{3.272358in}}%
\pgfpathlineto{\pgfqpoint{5.436538in}{3.265923in}}%
\pgfpathlineto{\pgfqpoint{5.428891in}{3.259296in}}%
\pgfpathlineto{\pgfqpoint{5.421232in}{3.252475in}}%
\pgfpathlineto{\pgfqpoint{5.413563in}{3.245459in}}%
\pgfpathclose%
\pgfusepath{fill}%
\end{pgfscope}%
\begin{pgfscope}%
\pgfpathrectangle{\pgfqpoint{1.150000in}{0.150000in}}{\pgfqpoint{5.700000in}{5.700000in}}%
\pgfusepath{clip}%
\pgfsetbuttcap%
\pgfsetroundjoin%
\definecolor{currentfill}{rgb}{0.269944,0.014625,0.341379}%
\pgfsetfillcolor{currentfill}%
\pgfsetfillopacity{0.800000}%
\pgfsetlinewidth{0.000000pt}%
\definecolor{currentstroke}{rgb}{0.000000,0.000000,0.000000}%
\pgfsetstrokecolor{currentstroke}%
\pgfsetdash{}{0pt}%
\pgfpathmoveto{\pgfqpoint{3.437517in}{1.179368in}}%
\pgfpathlineto{\pgfqpoint{3.451608in}{1.174294in}}%
\pgfpathlineto{\pgfqpoint{3.465702in}{1.169404in}}%
\pgfpathlineto{\pgfqpoint{3.479800in}{1.164697in}}%
\pgfpathlineto{\pgfqpoint{3.493903in}{1.160173in}}%
\pgfpathlineto{\pgfqpoint{3.502377in}{1.166110in}}%
\pgfpathlineto{\pgfqpoint{3.510840in}{1.172345in}}%
\pgfpathlineto{\pgfqpoint{3.519293in}{1.178870in}}%
\pgfpathlineto{\pgfqpoint{3.527735in}{1.185676in}}%
\pgfpathlineto{\pgfqpoint{3.513657in}{1.189408in}}%
\pgfpathlineto{\pgfqpoint{3.499584in}{1.193323in}}%
\pgfpathlineto{\pgfqpoint{3.485516in}{1.197421in}}%
\pgfpathlineto{\pgfqpoint{3.471452in}{1.201702in}}%
\pgfpathlineto{\pgfqpoint{3.462985in}{1.195676in}}%
\pgfpathlineto{\pgfqpoint{3.454507in}{1.189939in}}%
\pgfpathlineto{\pgfqpoint{3.446018in}{1.184501in}}%
\pgfpathlineto{\pgfqpoint{3.437517in}{1.179368in}}%
\pgfpathclose%
\pgfusepath{fill}%
\end{pgfscope}%
\begin{pgfscope}%
\pgfpathrectangle{\pgfqpoint{1.150000in}{0.150000in}}{\pgfqpoint{5.700000in}{5.700000in}}%
\pgfusepath{clip}%
\pgfsetbuttcap%
\pgfsetroundjoin%
\definecolor{currentfill}{rgb}{0.279574,0.170599,0.479997}%
\pgfsetfillcolor{currentfill}%
\pgfsetfillopacity{0.800000}%
\pgfsetlinewidth{0.000000pt}%
\definecolor{currentstroke}{rgb}{0.000000,0.000000,0.000000}%
\pgfsetstrokecolor{currentstroke}%
\pgfsetdash{}{0pt}%
\pgfpathmoveto{\pgfqpoint{4.065930in}{1.477233in}}%
\pgfpathlineto{\pgfqpoint{4.080148in}{1.482089in}}%
\pgfpathlineto{\pgfqpoint{4.094379in}{1.487123in}}%
\pgfpathlineto{\pgfqpoint{4.108622in}{1.492335in}}%
\pgfpathlineto{\pgfqpoint{4.122876in}{1.497725in}}%
\pgfpathlineto{\pgfqpoint{4.131085in}{1.513371in}}%
\pgfpathlineto{\pgfqpoint{4.139290in}{1.529071in}}%
\pgfpathlineto{\pgfqpoint{4.147491in}{1.544818in}}%
\pgfpathlineto{\pgfqpoint{4.155688in}{1.560608in}}%
\pgfpathlineto{\pgfqpoint{4.141433in}{1.554638in}}%
\pgfpathlineto{\pgfqpoint{4.127190in}{1.548846in}}%
\pgfpathlineto{\pgfqpoint{4.112960in}{1.543233in}}%
\pgfpathlineto{\pgfqpoint{4.098741in}{1.537799in}}%
\pgfpathlineto{\pgfqpoint{4.090545in}{1.522577in}}%
\pgfpathlineto{\pgfqpoint{4.082344in}{1.507405in}}%
\pgfpathlineto{\pgfqpoint{4.074139in}{1.492288in}}%
\pgfpathlineto{\pgfqpoint{4.065930in}{1.477233in}}%
\pgfpathclose%
\pgfusepath{fill}%
\end{pgfscope}%
\begin{pgfscope}%
\pgfpathrectangle{\pgfqpoint{1.150000in}{0.150000in}}{\pgfqpoint{5.700000in}{5.700000in}}%
\pgfusepath{clip}%
\pgfsetbuttcap%
\pgfsetroundjoin%
\definecolor{currentfill}{rgb}{0.277941,0.056324,0.381191}%
\pgfsetfillcolor{currentfill}%
\pgfsetfillopacity{0.800000}%
\pgfsetlinewidth{0.000000pt}%
\definecolor{currentstroke}{rgb}{0.000000,0.000000,0.000000}%
\pgfsetstrokecolor{currentstroke}%
\pgfsetdash{}{0pt}%
\pgfpathmoveto{\pgfqpoint{3.763877in}{1.236540in}}%
\pgfpathlineto{\pgfqpoint{3.778003in}{1.236642in}}%
\pgfpathlineto{\pgfqpoint{3.792138in}{1.236922in}}%
\pgfpathlineto{\pgfqpoint{3.806280in}{1.237380in}}%
\pgfpathlineto{\pgfqpoint{3.820432in}{1.238016in}}%
\pgfpathlineto{\pgfqpoint{3.828734in}{1.249851in}}%
\pgfpathlineto{\pgfqpoint{3.837030in}{1.261862in}}%
\pgfpathlineto{\pgfqpoint{3.845320in}{1.274042in}}%
\pgfpathlineto{\pgfqpoint{3.853604in}{1.286385in}}%
\pgfpathlineto{\pgfqpoint{3.839463in}{1.285049in}}%
\pgfpathlineto{\pgfqpoint{3.825330in}{1.283892in}}%
\pgfpathlineto{\pgfqpoint{3.811207in}{1.282914in}}%
\pgfpathlineto{\pgfqpoint{3.797092in}{1.282114in}}%
\pgfpathlineto{\pgfqpoint{3.788798in}{1.270458in}}%
\pgfpathlineto{\pgfqpoint{3.780497in}{1.258973in}}%
\pgfpathlineto{\pgfqpoint{3.772190in}{1.247665in}}%
\pgfpathlineto{\pgfqpoint{3.763877in}{1.236540in}}%
\pgfpathclose%
\pgfusepath{fill}%
\end{pgfscope}%
\begin{pgfscope}%
\pgfpathrectangle{\pgfqpoint{1.150000in}{0.150000in}}{\pgfqpoint{5.700000in}{5.700000in}}%
\pgfusepath{clip}%
\pgfsetbuttcap%
\pgfsetroundjoin%
\definecolor{currentfill}{rgb}{0.212395,0.359683,0.551710}%
\pgfsetfillcolor{currentfill}%
\pgfsetfillopacity{0.800000}%
\pgfsetlinewidth{0.000000pt}%
\definecolor{currentstroke}{rgb}{0.000000,0.000000,0.000000}%
\pgfsetstrokecolor{currentstroke}%
\pgfsetdash{}{0pt}%
\pgfpathmoveto{\pgfqpoint{4.433469in}{1.951456in}}%
\pgfpathlineto{\pgfqpoint{4.447872in}{1.961488in}}%
\pgfpathlineto{\pgfqpoint{4.462290in}{1.971701in}}%
\pgfpathlineto{\pgfqpoint{4.476724in}{1.982096in}}%
\pgfpathlineto{\pgfqpoint{4.491174in}{1.992674in}}%
\pgfpathlineto{\pgfqpoint{4.499303in}{2.009466in}}%
\pgfpathlineto{\pgfqpoint{4.507428in}{2.026178in}}%
\pgfpathlineto{\pgfqpoint{4.515549in}{2.042806in}}%
\pgfpathlineto{\pgfqpoint{4.523665in}{2.059346in}}%
\pgfpathlineto{\pgfqpoint{4.509208in}{2.048374in}}%
\pgfpathlineto{\pgfqpoint{4.494767in}{2.037585in}}%
\pgfpathlineto{\pgfqpoint{4.480343in}{2.026978in}}%
\pgfpathlineto{\pgfqpoint{4.465935in}{2.016554in}}%
\pgfpathlineto{\pgfqpoint{4.457825in}{2.000395in}}%
\pgfpathlineto{\pgfqpoint{4.449711in}{1.984156in}}%
\pgfpathlineto{\pgfqpoint{4.441592in}{1.967842in}}%
\pgfpathlineto{\pgfqpoint{4.433469in}{1.951456in}}%
\pgfpathclose%
\pgfusepath{fill}%
\end{pgfscope}%
\begin{pgfscope}%
\pgfpathrectangle{\pgfqpoint{1.150000in}{0.150000in}}{\pgfqpoint{5.700000in}{5.700000in}}%
\pgfusepath{clip}%
\pgfsetbuttcap%
\pgfsetroundjoin%
\definecolor{currentfill}{rgb}{0.468053,0.818921,0.323998}%
\pgfsetfillcolor{currentfill}%
\pgfsetfillopacity{0.800000}%
\pgfsetlinewidth{0.000000pt}%
\definecolor{currentstroke}{rgb}{0.000000,0.000000,0.000000}%
\pgfsetstrokecolor{currentstroke}%
\pgfsetdash{}{0pt}%
\pgfpathmoveto{\pgfqpoint{5.626003in}{3.463709in}}%
\pgfpathlineto{\pgfqpoint{5.641255in}{3.482882in}}%
\pgfpathlineto{\pgfqpoint{5.656532in}{3.502247in}}%
\pgfpathlineto{\pgfqpoint{5.671834in}{3.521806in}}%
\pgfpathlineto{\pgfqpoint{5.687162in}{3.541559in}}%
\pgfpathlineto{\pgfqpoint{5.694657in}{3.545591in}}%
\pgfpathlineto{\pgfqpoint{5.702139in}{3.549432in}}%
\pgfpathlineto{\pgfqpoint{5.709608in}{3.553084in}}%
\pgfpathlineto{\pgfqpoint{5.717065in}{3.556551in}}%
\pgfpathlineto{\pgfqpoint{5.701753in}{3.537080in}}%
\pgfpathlineto{\pgfqpoint{5.686466in}{3.517801in}}%
\pgfpathlineto{\pgfqpoint{5.671204in}{3.498714in}}%
\pgfpathlineto{\pgfqpoint{5.655968in}{3.479820in}}%
\pgfpathlineto{\pgfqpoint{5.648494in}{3.476060in}}%
\pgfpathlineto{\pgfqpoint{5.641009in}{3.472124in}}%
\pgfpathlineto{\pgfqpoint{5.633512in}{3.468008in}}%
\pgfpathlineto{\pgfqpoint{5.626003in}{3.463709in}}%
\pgfpathclose%
\pgfusepath{fill}%
\end{pgfscope}%
\begin{pgfscope}%
\pgfpathrectangle{\pgfqpoint{1.150000in}{0.150000in}}{\pgfqpoint{5.700000in}{5.700000in}}%
\pgfusepath{clip}%
\pgfsetbuttcap%
\pgfsetroundjoin%
\definecolor{currentfill}{rgb}{0.273809,0.031497,0.358853}%
\pgfsetfillcolor{currentfill}%
\pgfsetfillopacity{0.800000}%
\pgfsetlinewidth{0.000000pt}%
\definecolor{currentstroke}{rgb}{0.000000,0.000000,0.000000}%
\pgfsetstrokecolor{currentstroke}%
\pgfsetdash{}{0pt}%
\pgfpathmoveto{\pgfqpoint{3.674065in}{1.198320in}}%
\pgfpathlineto{\pgfqpoint{3.688176in}{1.196977in}}%
\pgfpathlineto{\pgfqpoint{3.702294in}{1.195813in}}%
\pgfpathlineto{\pgfqpoint{3.716419in}{1.194827in}}%
\pgfpathlineto{\pgfqpoint{3.730552in}{1.194020in}}%
\pgfpathlineto{\pgfqpoint{3.738894in}{1.204340in}}%
\pgfpathlineto{\pgfqpoint{3.747229in}{1.214871in}}%
\pgfpathlineto{\pgfqpoint{3.755556in}{1.225607in}}%
\pgfpathlineto{\pgfqpoint{3.763877in}{1.236540in}}%
\pgfpathlineto{\pgfqpoint{3.749758in}{1.236618in}}%
\pgfpathlineto{\pgfqpoint{3.735647in}{1.236874in}}%
\pgfpathlineto{\pgfqpoint{3.721544in}{1.237309in}}%
\pgfpathlineto{\pgfqpoint{3.707448in}{1.237924in}}%
\pgfpathlineto{\pgfqpoint{3.699114in}{1.227707in}}%
\pgfpathlineto{\pgfqpoint{3.690772in}{1.217696in}}%
\pgfpathlineto{\pgfqpoint{3.682422in}{1.207898in}}%
\pgfpathlineto{\pgfqpoint{3.674065in}{1.198320in}}%
\pgfpathclose%
\pgfusepath{fill}%
\end{pgfscope}%
\begin{pgfscope}%
\pgfpathrectangle{\pgfqpoint{1.150000in}{0.150000in}}{\pgfqpoint{5.700000in}{5.700000in}}%
\pgfusepath{clip}%
\pgfsetbuttcap%
\pgfsetroundjoin%
\definecolor{currentfill}{rgb}{0.281446,0.084320,0.407414}%
\pgfsetfillcolor{currentfill}%
\pgfsetfillopacity{0.800000}%
\pgfsetlinewidth{0.000000pt}%
\definecolor{currentstroke}{rgb}{0.000000,0.000000,0.000000}%
\pgfsetstrokecolor{currentstroke}%
\pgfsetdash{}{0pt}%
\pgfpathmoveto{\pgfqpoint{3.853604in}{1.286385in}}%
\pgfpathlineto{\pgfqpoint{3.867754in}{1.287898in}}%
\pgfpathlineto{\pgfqpoint{3.881913in}{1.289589in}}%
\pgfpathlineto{\pgfqpoint{3.896082in}{1.291459in}}%
\pgfpathlineto{\pgfqpoint{3.910260in}{1.293505in}}%
\pgfpathlineto{\pgfqpoint{3.918530in}{1.306684in}}%
\pgfpathlineto{\pgfqpoint{3.926795in}{1.320005in}}%
\pgfpathlineto{\pgfqpoint{3.935054in}{1.333460in}}%
\pgfpathlineto{\pgfqpoint{3.943309in}{1.347043in}}%
\pgfpathlineto{\pgfqpoint{3.929137in}{1.344327in}}%
\pgfpathlineto{\pgfqpoint{3.914976in}{1.341788in}}%
\pgfpathlineto{\pgfqpoint{3.900824in}{1.339428in}}%
\pgfpathlineto{\pgfqpoint{3.886682in}{1.337246in}}%
\pgfpathlineto{\pgfqpoint{3.878421in}{1.324320in}}%
\pgfpathlineto{\pgfqpoint{3.870154in}{1.311530in}}%
\pgfpathlineto{\pgfqpoint{3.861882in}{1.298883in}}%
\pgfpathlineto{\pgfqpoint{3.853604in}{1.286385in}}%
\pgfpathclose%
\pgfusepath{fill}%
\end{pgfscope}%
\begin{pgfscope}%
\pgfpathrectangle{\pgfqpoint{1.150000in}{0.150000in}}{\pgfqpoint{5.700000in}{5.700000in}}%
\pgfusepath{clip}%
\pgfsetbuttcap%
\pgfsetroundjoin%
\definecolor{currentfill}{rgb}{0.183898,0.422383,0.556944}%
\pgfsetfillcolor{currentfill}%
\pgfsetfillopacity{0.800000}%
\pgfsetlinewidth{0.000000pt}%
\definecolor{currentstroke}{rgb}{0.000000,0.000000,0.000000}%
\pgfsetstrokecolor{currentstroke}%
\pgfsetdash{}{0pt}%
\pgfpathmoveto{\pgfqpoint{4.556085in}{2.124569in}}%
\pgfpathlineto{\pgfqpoint{4.570566in}{2.136086in}}%
\pgfpathlineto{\pgfqpoint{4.585064in}{2.147787in}}%
\pgfpathlineto{\pgfqpoint{4.599579in}{2.159671in}}%
\pgfpathlineto{\pgfqpoint{4.614111in}{2.171739in}}%
\pgfpathlineto{\pgfqpoint{4.622212in}{2.188135in}}%
\pgfpathlineto{\pgfqpoint{4.630307in}{2.204417in}}%
\pgfpathlineto{\pgfqpoint{4.638398in}{2.220581in}}%
\pgfpathlineto{\pgfqpoint{4.646484in}{2.236624in}}%
\pgfpathlineto{\pgfqpoint{4.631944in}{2.224226in}}%
\pgfpathlineto{\pgfqpoint{4.617422in}{2.212012in}}%
\pgfpathlineto{\pgfqpoint{4.602918in}{2.199983in}}%
\pgfpathlineto{\pgfqpoint{4.588431in}{2.188138in}}%
\pgfpathlineto{\pgfqpoint{4.580352in}{2.172411in}}%
\pgfpathlineto{\pgfqpoint{4.572268in}{2.156572in}}%
\pgfpathlineto{\pgfqpoint{4.564179in}{2.140624in}}%
\pgfpathlineto{\pgfqpoint{4.556085in}{2.124569in}}%
\pgfpathclose%
\pgfusepath{fill}%
\end{pgfscope}%
\begin{pgfscope}%
\pgfpathrectangle{\pgfqpoint{1.150000in}{0.150000in}}{\pgfqpoint{5.700000in}{5.700000in}}%
\pgfusepath{clip}%
\pgfsetbuttcap%
\pgfsetroundjoin%
\definecolor{currentfill}{rgb}{0.271305,0.019942,0.347269}%
\pgfsetfillcolor{currentfill}%
\pgfsetfillopacity{0.800000}%
\pgfsetlinewidth{0.000000pt}%
\definecolor{currentstroke}{rgb}{0.000000,0.000000,0.000000}%
\pgfsetstrokecolor{currentstroke}%
\pgfsetdash{}{0pt}%
\pgfpathmoveto{\pgfqpoint{3.584099in}{1.172569in}}%
\pgfpathlineto{\pgfqpoint{3.598203in}{1.169746in}}%
\pgfpathlineto{\pgfqpoint{3.612314in}{1.167102in}}%
\pgfpathlineto{\pgfqpoint{3.626430in}{1.164639in}}%
\pgfpathlineto{\pgfqpoint{3.640553in}{1.162355in}}%
\pgfpathlineto{\pgfqpoint{3.648944in}{1.170980in}}%
\pgfpathlineto{\pgfqpoint{3.657326in}{1.179854in}}%
\pgfpathlineto{\pgfqpoint{3.665699in}{1.188970in}}%
\pgfpathlineto{\pgfqpoint{3.674065in}{1.198320in}}%
\pgfpathlineto{\pgfqpoint{3.659960in}{1.199843in}}%
\pgfpathlineto{\pgfqpoint{3.645863in}{1.201546in}}%
\pgfpathlineto{\pgfqpoint{3.631771in}{1.203430in}}%
\pgfpathlineto{\pgfqpoint{3.617686in}{1.205493in}}%
\pgfpathlineto{\pgfqpoint{3.609303in}{1.196891in}}%
\pgfpathlineto{\pgfqpoint{3.600911in}{1.188531in}}%
\pgfpathlineto{\pgfqpoint{3.592509in}{1.180421in}}%
\pgfpathlineto{\pgfqpoint{3.584099in}{1.172569in}}%
\pgfpathclose%
\pgfusepath{fill}%
\end{pgfscope}%
\begin{pgfscope}%
\pgfpathrectangle{\pgfqpoint{1.150000in}{0.150000in}}{\pgfqpoint{5.700000in}{5.700000in}}%
\pgfusepath{clip}%
\pgfsetbuttcap%
\pgfsetroundjoin%
\definecolor{currentfill}{rgb}{0.175707,0.697900,0.491033}%
\pgfsetfillcolor{currentfill}%
\pgfsetfillopacity{0.800000}%
\pgfsetlinewidth{0.000000pt}%
\definecolor{currentstroke}{rgb}{0.000000,0.000000,0.000000}%
\pgfsetstrokecolor{currentstroke}%
\pgfsetdash{}{0pt}%
\pgfpathmoveto{\pgfqpoint{1.838626in}{3.146516in}}%
\pgfpathlineto{\pgfqpoint{1.853591in}{3.113178in}}%
\pgfpathlineto{\pgfqpoint{1.868531in}{3.080223in}}%
\pgfpathlineto{\pgfqpoint{1.883447in}{3.047648in}}%
\pgfpathlineto{\pgfqpoint{1.898340in}{3.015447in}}%
\pgfpathlineto{\pgfqpoint{1.908449in}{2.997620in}}%
\pgfpathlineto{\pgfqpoint{1.918519in}{2.980368in}}%
\pgfpathlineto{\pgfqpoint{1.928549in}{2.963681in}}%
\pgfpathlineto{\pgfqpoint{1.938542in}{2.947551in}}%
\pgfpathlineto{\pgfqpoint{1.923744in}{2.978816in}}%
\pgfpathlineto{\pgfqpoint{1.908924in}{3.010454in}}%
\pgfpathlineto{\pgfqpoint{1.894081in}{3.042468in}}%
\pgfpathlineto{\pgfqpoint{1.879214in}{3.074861in}}%
\pgfpathlineto{\pgfqpoint{1.869127in}{3.091913in}}%
\pgfpathlineto{\pgfqpoint{1.859001in}{3.109533in}}%
\pgfpathlineto{\pgfqpoint{1.848834in}{3.127730in}}%
\pgfpathlineto{\pgfqpoint{1.838626in}{3.146516in}}%
\pgfpathclose%
\pgfusepath{fill}%
\end{pgfscope}%
\begin{pgfscope}%
\pgfpathrectangle{\pgfqpoint{1.150000in}{0.150000in}}{\pgfqpoint{5.700000in}{5.700000in}}%
\pgfusepath{clip}%
\pgfsetbuttcap%
\pgfsetroundjoin%
\definecolor{currentfill}{rgb}{0.232815,0.732247,0.459277}%
\pgfsetfillcolor{currentfill}%
\pgfsetfillopacity{0.800000}%
\pgfsetlinewidth{0.000000pt}%
\definecolor{currentstroke}{rgb}{0.000000,0.000000,0.000000}%
\pgfsetstrokecolor{currentstroke}%
\pgfsetdash{}{0pt}%
\pgfpathmoveto{\pgfqpoint{5.291652in}{3.110090in}}%
\pgfpathlineto{\pgfqpoint{5.306659in}{3.127793in}}%
\pgfpathlineto{\pgfqpoint{5.321689in}{3.145687in}}%
\pgfpathlineto{\pgfqpoint{5.336743in}{3.163773in}}%
\pgfpathlineto{\pgfqpoint{5.351820in}{3.182051in}}%
\pgfpathlineto{\pgfqpoint{5.359575in}{3.190700in}}%
\pgfpathlineto{\pgfqpoint{5.367320in}{3.199139in}}%
\pgfpathlineto{\pgfqpoint{5.375054in}{3.207370in}}%
\pgfpathlineto{\pgfqpoint{5.382778in}{3.215394in}}%
\pgfpathlineto{\pgfqpoint{5.367705in}{3.197209in}}%
\pgfpathlineto{\pgfqpoint{5.352657in}{3.179216in}}%
\pgfpathlineto{\pgfqpoint{5.337631in}{3.161414in}}%
\pgfpathlineto{\pgfqpoint{5.322629in}{3.143803in}}%
\pgfpathlineto{\pgfqpoint{5.314900in}{3.135672in}}%
\pgfpathlineto{\pgfqpoint{5.307161in}{3.127344in}}%
\pgfpathlineto{\pgfqpoint{5.299412in}{3.118817in}}%
\pgfpathlineto{\pgfqpoint{5.291652in}{3.110090in}}%
\pgfpathclose%
\pgfusepath{fill}%
\end{pgfscope}%
\begin{pgfscope}%
\pgfpathrectangle{\pgfqpoint{1.150000in}{0.150000in}}{\pgfqpoint{5.700000in}{5.700000in}}%
\pgfusepath{clip}%
\pgfsetbuttcap%
\pgfsetroundjoin%
\definecolor{currentfill}{rgb}{0.283197,0.115680,0.436115}%
\pgfsetfillcolor{currentfill}%
\pgfsetfillopacity{0.800000}%
\pgfsetlinewidth{0.000000pt}%
\definecolor{currentstroke}{rgb}{0.000000,0.000000,0.000000}%
\pgfsetstrokecolor{currentstroke}%
\pgfsetdash{}{0pt}%
\pgfpathmoveto{\pgfqpoint{3.943309in}{1.347043in}}%
\pgfpathlineto{\pgfqpoint{3.957490in}{1.349938in}}%
\pgfpathlineto{\pgfqpoint{3.971682in}{1.353010in}}%
\pgfpathlineto{\pgfqpoint{3.985885in}{1.356259in}}%
\pgfpathlineto{\pgfqpoint{4.000097in}{1.359686in}}%
\pgfpathlineto{\pgfqpoint{4.008342in}{1.374043in}}%
\pgfpathlineto{\pgfqpoint{4.016583in}{1.388509in}}%
\pgfpathlineto{\pgfqpoint{4.024818in}{1.403076in}}%
\pgfpathlineto{\pgfqpoint{4.033049in}{1.417739in}}%
\pgfpathlineto{\pgfqpoint{4.018840in}{1.413672in}}%
\pgfpathlineto{\pgfqpoint{4.004641in}{1.409782in}}%
\pgfpathlineto{\pgfqpoint{3.990454in}{1.406071in}}%
\pgfpathlineto{\pgfqpoint{3.976277in}{1.402537in}}%
\pgfpathlineto{\pgfqpoint{3.968042in}{1.388502in}}%
\pgfpathlineto{\pgfqpoint{3.959803in}{1.374571in}}%
\pgfpathlineto{\pgfqpoint{3.951558in}{1.360749in}}%
\pgfpathlineto{\pgfqpoint{3.943309in}{1.347043in}}%
\pgfpathclose%
\pgfusepath{fill}%
\end{pgfscope}%
\begin{pgfscope}%
\pgfpathrectangle{\pgfqpoint{1.150000in}{0.150000in}}{\pgfqpoint{5.700000in}{5.700000in}}%
\pgfusepath{clip}%
\pgfsetbuttcap%
\pgfsetroundjoin%
\definecolor{currentfill}{rgb}{0.545524,0.838039,0.275626}%
\pgfsetfillcolor{currentfill}%
\pgfsetfillopacity{0.800000}%
\pgfsetlinewidth{0.000000pt}%
\definecolor{currentstroke}{rgb}{0.000000,0.000000,0.000000}%
\pgfsetstrokecolor{currentstroke}%
\pgfsetdash{}{0pt}%
\pgfpathmoveto{\pgfqpoint{5.717065in}{3.556551in}}%
\pgfpathlineto{\pgfqpoint{5.732403in}{3.576216in}}%
\pgfpathlineto{\pgfqpoint{5.747767in}{3.596075in}}%
\pgfpathlineto{\pgfqpoint{5.763156in}{3.616127in}}%
\pgfpathlineto{\pgfqpoint{5.770588in}{3.619184in}}%
\pgfpathlineto{\pgfqpoint{5.778007in}{3.622056in}}%
\pgfpathlineto{\pgfqpoint{5.785413in}{3.624746in}}%
\pgfpathlineto{\pgfqpoint{5.792807in}{3.627259in}}%
\pgfpathlineto{\pgfqpoint{5.777435in}{3.607525in}}%
\pgfpathlineto{\pgfqpoint{5.762090in}{3.587985in}}%
\pgfpathlineto{\pgfqpoint{5.746770in}{3.568637in}}%
\pgfpathlineto{\pgfqpoint{5.739362in}{3.565876in}}%
\pgfpathlineto{\pgfqpoint{5.731942in}{3.562944in}}%
\pgfpathlineto{\pgfqpoint{5.724510in}{3.559837in}}%
\pgfpathlineto{\pgfqpoint{5.717065in}{3.556551in}}%
\pgfpathclose%
\pgfusepath{fill}%
\end{pgfscope}%
\begin{pgfscope}%
\pgfpathrectangle{\pgfqpoint{1.150000in}{0.150000in}}{\pgfqpoint{5.700000in}{5.700000in}}%
\pgfusepath{clip}%
\pgfsetbuttcap%
\pgfsetroundjoin%
\definecolor{currentfill}{rgb}{0.159194,0.482237,0.558073}%
\pgfsetfillcolor{currentfill}%
\pgfsetfillopacity{0.800000}%
\pgfsetlinewidth{0.000000pt}%
\definecolor{currentstroke}{rgb}{0.000000,0.000000,0.000000}%
\pgfsetstrokecolor{currentstroke}%
\pgfsetdash{}{0pt}%
\pgfpathmoveto{\pgfqpoint{4.678775in}{2.299537in}}%
\pgfpathlineto{\pgfqpoint{4.693339in}{2.312416in}}%
\pgfpathlineto{\pgfqpoint{4.707922in}{2.325480in}}%
\pgfpathlineto{\pgfqpoint{4.722523in}{2.338729in}}%
\pgfpathlineto{\pgfqpoint{4.737143in}{2.352164in}}%
\pgfpathlineto{\pgfqpoint{4.745209in}{2.367840in}}%
\pgfpathlineto{\pgfqpoint{4.753270in}{2.383372in}}%
\pgfpathlineto{\pgfqpoint{4.761325in}{2.398758in}}%
\pgfpathlineto{\pgfqpoint{4.769374in}{2.413995in}}%
\pgfpathlineto{\pgfqpoint{4.754747in}{2.400297in}}%
\pgfpathlineto{\pgfqpoint{4.740139in}{2.386785in}}%
\pgfpathlineto{\pgfqpoint{4.725550in}{2.373459in}}%
\pgfpathlineto{\pgfqpoint{4.710979in}{2.360318in}}%
\pgfpathlineto{\pgfqpoint{4.702937in}{2.345331in}}%
\pgfpathlineto{\pgfqpoint{4.694888in}{2.330203in}}%
\pgfpathlineto{\pgfqpoint{4.686834in}{2.314938in}}%
\pgfpathlineto{\pgfqpoint{4.678775in}{2.299537in}}%
\pgfpathclose%
\pgfusepath{fill}%
\end{pgfscope}%
\begin{pgfscope}%
\pgfpathrectangle{\pgfqpoint{1.150000in}{0.150000in}}{\pgfqpoint{5.700000in}{5.700000in}}%
\pgfusepath{clip}%
\pgfsetbuttcap%
\pgfsetroundjoin%
\definecolor{currentfill}{rgb}{0.248629,0.278775,0.534556}%
\pgfsetfillcolor{currentfill}%
\pgfsetfillopacity{0.800000}%
\pgfsetlinewidth{0.000000pt}%
\definecolor{currentstroke}{rgb}{0.000000,0.000000,0.000000}%
\pgfsetstrokecolor{currentstroke}%
\pgfsetdash{}{0pt}%
\pgfpathmoveto{\pgfqpoint{4.278292in}{1.717786in}}%
\pgfpathlineto{\pgfqpoint{4.292619in}{1.725723in}}%
\pgfpathlineto{\pgfqpoint{4.306960in}{1.733839in}}%
\pgfpathlineto{\pgfqpoint{4.321315in}{1.742135in}}%
\pgfpathlineto{\pgfqpoint{4.335685in}{1.750611in}}%
\pgfpathlineto{\pgfqpoint{4.343855in}{1.767555in}}%
\pgfpathlineto{\pgfqpoint{4.352021in}{1.784475in}}%
\pgfpathlineto{\pgfqpoint{4.360183in}{1.801368in}}%
\pgfpathlineto{\pgfqpoint{4.368342in}{1.818229in}}%
\pgfpathlineto{\pgfqpoint{4.353966in}{1.809263in}}%
\pgfpathlineto{\pgfqpoint{4.339606in}{1.800477in}}%
\pgfpathlineto{\pgfqpoint{4.325260in}{1.791872in}}%
\pgfpathlineto{\pgfqpoint{4.310929in}{1.783447in}}%
\pgfpathlineto{\pgfqpoint{4.302775in}{1.767064in}}%
\pgfpathlineto{\pgfqpoint{4.294618in}{1.750656in}}%
\pgfpathlineto{\pgfqpoint{4.286457in}{1.734229in}}%
\pgfpathlineto{\pgfqpoint{4.278292in}{1.717786in}}%
\pgfpathclose%
\pgfusepath{fill}%
\end{pgfscope}%
\begin{pgfscope}%
\pgfpathrectangle{\pgfqpoint{1.150000in}{0.150000in}}{\pgfqpoint{5.700000in}{5.700000in}}%
\pgfusepath{clip}%
\pgfsetbuttcap%
\pgfsetroundjoin%
\definecolor{currentfill}{rgb}{0.270595,0.214069,0.507052}%
\pgfsetfillcolor{currentfill}%
\pgfsetfillopacity{0.800000}%
\pgfsetlinewidth{0.000000pt}%
\definecolor{currentstroke}{rgb}{0.000000,0.000000,0.000000}%
\pgfsetstrokecolor{currentstroke}%
\pgfsetdash{}{0pt}%
\pgfpathmoveto{\pgfqpoint{4.155688in}{1.560608in}}%
\pgfpathlineto{\pgfqpoint{4.169955in}{1.566758in}}%
\pgfpathlineto{\pgfqpoint{4.184236in}{1.573086in}}%
\pgfpathlineto{\pgfqpoint{4.198530in}{1.579593in}}%
\pgfpathlineto{\pgfqpoint{4.212836in}{1.586278in}}%
\pgfpathlineto{\pgfqpoint{4.221031in}{1.602668in}}%
\pgfpathlineto{\pgfqpoint{4.229222in}{1.619082in}}%
\pgfpathlineto{\pgfqpoint{4.237410in}{1.635514in}}%
\pgfpathlineto{\pgfqpoint{4.245594in}{1.651961in}}%
\pgfpathlineto{\pgfqpoint{4.231284in}{1.644724in}}%
\pgfpathlineto{\pgfqpoint{4.216988in}{1.637666in}}%
\pgfpathlineto{\pgfqpoint{4.202706in}{1.630788in}}%
\pgfpathlineto{\pgfqpoint{4.188436in}{1.624089in}}%
\pgfpathlineto{\pgfqpoint{4.180255in}{1.608181in}}%
\pgfpathlineto{\pgfqpoint{4.172070in}{1.592295in}}%
\pgfpathlineto{\pgfqpoint{4.163881in}{1.576436in}}%
\pgfpathlineto{\pgfqpoint{4.155688in}{1.560608in}}%
\pgfpathclose%
\pgfusepath{fill}%
\end{pgfscope}%
\begin{pgfscope}%
\pgfpathrectangle{\pgfqpoint{1.150000in}{0.150000in}}{\pgfqpoint{5.700000in}{5.700000in}}%
\pgfusepath{clip}%
\pgfsetbuttcap%
\pgfsetroundjoin%
\definecolor{currentfill}{rgb}{0.137770,0.537492,0.554906}%
\pgfsetfillcolor{currentfill}%
\pgfsetfillopacity{0.800000}%
\pgfsetlinewidth{0.000000pt}%
\definecolor{currentstroke}{rgb}{0.000000,0.000000,0.000000}%
\pgfsetstrokecolor{currentstroke}%
\pgfsetdash{}{0pt}%
\pgfpathmoveto{\pgfqpoint{4.801510in}{2.473415in}}%
\pgfpathlineto{\pgfqpoint{4.816162in}{2.487528in}}%
\pgfpathlineto{\pgfqpoint{4.830833in}{2.501828in}}%
\pgfpathlineto{\pgfqpoint{4.845524in}{2.516314in}}%
\pgfpathlineto{\pgfqpoint{4.860235in}{2.530988in}}%
\pgfpathlineto{\pgfqpoint{4.868260in}{2.545660in}}%
\pgfpathlineto{\pgfqpoint{4.876277in}{2.560163in}}%
\pgfpathlineto{\pgfqpoint{4.884288in}{2.574497in}}%
\pgfpathlineto{\pgfqpoint{4.892292in}{2.588658in}}%
\pgfpathlineto{\pgfqpoint{4.877576in}{2.573790in}}%
\pgfpathlineto{\pgfqpoint{4.862879in}{2.559109in}}%
\pgfpathlineto{\pgfqpoint{4.848202in}{2.544615in}}%
\pgfpathlineto{\pgfqpoint{4.833544in}{2.530308in}}%
\pgfpathlineto{\pgfqpoint{4.825545in}{2.516327in}}%
\pgfpathlineto{\pgfqpoint{4.817540in}{2.502184in}}%
\pgfpathlineto{\pgfqpoint{4.809528in}{2.487879in}}%
\pgfpathlineto{\pgfqpoint{4.801510in}{2.473415in}}%
\pgfpathclose%
\pgfusepath{fill}%
\end{pgfscope}%
\begin{pgfscope}%
\pgfpathrectangle{\pgfqpoint{1.150000in}{0.150000in}}{\pgfqpoint{5.700000in}{5.700000in}}%
\pgfusepath{clip}%
\pgfsetbuttcap%
\pgfsetroundjoin%
\definecolor{currentfill}{rgb}{0.166383,0.690856,0.496502}%
\pgfsetfillcolor{currentfill}%
\pgfsetfillopacity{0.800000}%
\pgfsetlinewidth{0.000000pt}%
\definecolor{currentstroke}{rgb}{0.000000,0.000000,0.000000}%
\pgfsetstrokecolor{currentstroke}%
\pgfsetdash{}{0pt}%
\pgfpathmoveto{\pgfqpoint{5.169397in}{2.963724in}}%
\pgfpathlineto{\pgfqpoint{5.184318in}{2.980739in}}%
\pgfpathlineto{\pgfqpoint{5.199261in}{2.997944in}}%
\pgfpathlineto{\pgfqpoint{5.214227in}{3.015340in}}%
\pgfpathlineto{\pgfqpoint{5.229215in}{3.032927in}}%
\pgfpathlineto{\pgfqpoint{5.237054in}{3.043295in}}%
\pgfpathlineto{\pgfqpoint{5.244884in}{3.053454in}}%
\pgfpathlineto{\pgfqpoint{5.252703in}{3.063407in}}%
\pgfpathlineto{\pgfqpoint{5.260513in}{3.073153in}}%
\pgfpathlineto{\pgfqpoint{5.245526in}{3.055585in}}%
\pgfpathlineto{\pgfqpoint{5.230561in}{3.038208in}}%
\pgfpathlineto{\pgfqpoint{5.215620in}{3.021022in}}%
\pgfpathlineto{\pgfqpoint{5.200700in}{3.004027in}}%
\pgfpathlineto{\pgfqpoint{5.192889in}{2.994248in}}%
\pgfpathlineto{\pgfqpoint{5.185067in}{2.984272in}}%
\pgfpathlineto{\pgfqpoint{5.177237in}{2.974097in}}%
\pgfpathlineto{\pgfqpoint{5.169397in}{2.963724in}}%
\pgfpathclose%
\pgfusepath{fill}%
\end{pgfscope}%
\begin{pgfscope}%
\pgfpathrectangle{\pgfqpoint{1.150000in}{0.150000in}}{\pgfqpoint{5.700000in}{5.700000in}}%
\pgfusepath{clip}%
\pgfsetbuttcap%
\pgfsetroundjoin%
\definecolor{currentfill}{rgb}{0.395174,0.797475,0.367757}%
\pgfsetfillcolor{currentfill}%
\pgfsetfillopacity{0.800000}%
\pgfsetlinewidth{0.000000pt}%
\definecolor{currentstroke}{rgb}{0.000000,0.000000,0.000000}%
\pgfsetstrokecolor{currentstroke}%
\pgfsetdash{}{0pt}%
\pgfpathmoveto{\pgfqpoint{5.504641in}{3.345833in}}%
\pgfpathlineto{\pgfqpoint{5.519818in}{3.364682in}}%
\pgfpathlineto{\pgfqpoint{5.535019in}{3.383724in}}%
\pgfpathlineto{\pgfqpoint{5.550246in}{3.402958in}}%
\pgfpathlineto{\pgfqpoint{5.565497in}{3.422387in}}%
\pgfpathlineto{\pgfqpoint{5.573102in}{3.428252in}}%
\pgfpathlineto{\pgfqpoint{5.580695in}{3.433911in}}%
\pgfpathlineto{\pgfqpoint{5.588277in}{3.439369in}}%
\pgfpathlineto{\pgfqpoint{5.595846in}{3.444626in}}%
\pgfpathlineto{\pgfqpoint{5.580606in}{3.425404in}}%
\pgfpathlineto{\pgfqpoint{5.565390in}{3.406374in}}%
\pgfpathlineto{\pgfqpoint{5.550200in}{3.387537in}}%
\pgfpathlineto{\pgfqpoint{5.535033in}{3.368892in}}%
\pgfpathlineto{\pgfqpoint{5.527452in}{3.363417in}}%
\pgfpathlineto{\pgfqpoint{5.519860in}{3.357750in}}%
\pgfpathlineto{\pgfqpoint{5.512256in}{3.351889in}}%
\pgfpathlineto{\pgfqpoint{5.504641in}{3.345833in}}%
\pgfpathclose%
\pgfusepath{fill}%
\end{pgfscope}%
\begin{pgfscope}%
\pgfpathrectangle{\pgfqpoint{1.150000in}{0.150000in}}{\pgfqpoint{5.700000in}{5.700000in}}%
\pgfusepath{clip}%
\pgfsetbuttcap%
\pgfsetroundjoin%
\definecolor{currentfill}{rgb}{0.220057,0.343307,0.549413}%
\pgfsetfillcolor{currentfill}%
\pgfsetfillopacity{0.800000}%
\pgfsetlinewidth{0.000000pt}%
\definecolor{currentstroke}{rgb}{0.000000,0.000000,0.000000}%
\pgfsetstrokecolor{currentstroke}%
\pgfsetdash{}{0pt}%
\pgfpathmoveto{\pgfqpoint{4.400937in}{1.885266in}}%
\pgfpathlineto{\pgfqpoint{4.415334in}{1.894872in}}%
\pgfpathlineto{\pgfqpoint{4.429745in}{1.904659in}}%
\pgfpathlineto{\pgfqpoint{4.444173in}{1.914627in}}%
\pgfpathlineto{\pgfqpoint{4.458616in}{1.924777in}}%
\pgfpathlineto{\pgfqpoint{4.466762in}{1.941853in}}%
\pgfpathlineto{\pgfqpoint{4.474903in}{1.958864in}}%
\pgfpathlineto{\pgfqpoint{4.483041in}{1.975805in}}%
\pgfpathlineto{\pgfqpoint{4.491174in}{1.992674in}}%
\pgfpathlineto{\pgfqpoint{4.476724in}{1.982096in}}%
\pgfpathlineto{\pgfqpoint{4.462290in}{1.971701in}}%
\pgfpathlineto{\pgfqpoint{4.447872in}{1.961488in}}%
\pgfpathlineto{\pgfqpoint{4.433469in}{1.951456in}}%
\pgfpathlineto{\pgfqpoint{4.425342in}{1.935002in}}%
\pgfpathlineto{\pgfqpoint{4.417211in}{1.918483in}}%
\pgfpathlineto{\pgfqpoint{4.409076in}{1.901903in}}%
\pgfpathlineto{\pgfqpoint{4.400937in}{1.885266in}}%
\pgfpathclose%
\pgfusepath{fill}%
\end{pgfscope}%
\begin{pgfscope}%
\pgfpathrectangle{\pgfqpoint{1.150000in}{0.150000in}}{\pgfqpoint{5.700000in}{5.700000in}}%
\pgfusepath{clip}%
\pgfsetbuttcap%
\pgfsetroundjoin%
\definecolor{currentfill}{rgb}{0.121148,0.592739,0.544641}%
\pgfsetfillcolor{currentfill}%
\pgfsetfillopacity{0.800000}%
\pgfsetlinewidth{0.000000pt}%
\definecolor{currentstroke}{rgb}{0.000000,0.000000,0.000000}%
\pgfsetstrokecolor{currentstroke}%
\pgfsetdash{}{0pt}%
\pgfpathmoveto{\pgfqpoint{4.924239in}{2.643565in}}%
\pgfpathlineto{\pgfqpoint{4.938981in}{2.658781in}}%
\pgfpathlineto{\pgfqpoint{4.953743in}{2.674185in}}%
\pgfpathlineto{\pgfqpoint{4.968526in}{2.689778in}}%
\pgfpathlineto{\pgfqpoint{4.983330in}{2.705559in}}%
\pgfpathlineto{\pgfqpoint{4.991303in}{2.718983in}}%
\pgfpathlineto{\pgfqpoint{4.999269in}{2.732219in}}%
\pgfpathlineto{\pgfqpoint{5.007226in}{2.745268in}}%
\pgfpathlineto{\pgfqpoint{5.015176in}{2.758127in}}%
\pgfpathlineto{\pgfqpoint{5.000368in}{2.742221in}}%
\pgfpathlineto{\pgfqpoint{4.985582in}{2.726503in}}%
\pgfpathlineto{\pgfqpoint{4.970815in}{2.710975in}}%
\pgfpathlineto{\pgfqpoint{4.956070in}{2.695634in}}%
\pgfpathlineto{\pgfqpoint{4.948123in}{2.682886in}}%
\pgfpathlineto{\pgfqpoint{4.940169in}{2.669958in}}%
\pgfpathlineto{\pgfqpoint{4.932208in}{2.656850in}}%
\pgfpathlineto{\pgfqpoint{4.924239in}{2.643565in}}%
\pgfpathclose%
\pgfusepath{fill}%
\end{pgfscope}%
\begin{pgfscope}%
\pgfpathrectangle{\pgfqpoint{1.150000in}{0.150000in}}{\pgfqpoint{5.700000in}{5.700000in}}%
\pgfusepath{clip}%
\pgfsetbuttcap%
\pgfsetroundjoin%
\definecolor{currentfill}{rgb}{0.126326,0.644107,0.525311}%
\pgfsetfillcolor{currentfill}%
\pgfsetfillopacity{0.800000}%
\pgfsetlinewidth{0.000000pt}%
\definecolor{currentstroke}{rgb}{0.000000,0.000000,0.000000}%
\pgfsetstrokecolor{currentstroke}%
\pgfsetdash{}{0pt}%
\pgfpathmoveto{\pgfqpoint{5.046896in}{2.807665in}}%
\pgfpathlineto{\pgfqpoint{5.061728in}{2.823850in}}%
\pgfpathlineto{\pgfqpoint{5.076581in}{2.840224in}}%
\pgfpathlineto{\pgfqpoint{5.091456in}{2.856788in}}%
\pgfpathlineto{\pgfqpoint{5.106353in}{2.873541in}}%
\pgfpathlineto{\pgfqpoint{5.114264in}{2.885516in}}%
\pgfpathlineto{\pgfqpoint{5.122167in}{2.897289in}}%
\pgfpathlineto{\pgfqpoint{5.130061in}{2.908863in}}%
\pgfpathlineto{\pgfqpoint{5.137947in}{2.920236in}}%
\pgfpathlineto{\pgfqpoint{5.123048in}{2.903429in}}%
\pgfpathlineto{\pgfqpoint{5.108171in}{2.886811in}}%
\pgfpathlineto{\pgfqpoint{5.093317in}{2.870384in}}%
\pgfpathlineto{\pgfqpoint{5.078483in}{2.854146in}}%
\pgfpathlineto{\pgfqpoint{5.070599in}{2.842813in}}%
\pgfpathlineto{\pgfqpoint{5.062706in}{2.831289in}}%
\pgfpathlineto{\pgfqpoint{5.054805in}{2.819573in}}%
\pgfpathlineto{\pgfqpoint{5.046896in}{2.807665in}}%
\pgfpathclose%
\pgfusepath{fill}%
\end{pgfscope}%
\begin{pgfscope}%
\pgfpathrectangle{\pgfqpoint{1.150000in}{0.150000in}}{\pgfqpoint{5.700000in}{5.700000in}}%
\pgfusepath{clip}%
\pgfsetbuttcap%
\pgfsetroundjoin%
\definecolor{currentfill}{rgb}{0.269944,0.014625,0.341379}%
\pgfsetfillcolor{currentfill}%
\pgfsetfillopacity{0.800000}%
\pgfsetlinewidth{0.000000pt}%
\definecolor{currentstroke}{rgb}{0.000000,0.000000,0.000000}%
\pgfsetstrokecolor{currentstroke}%
\pgfsetdash{}{0pt}%
\pgfpathmoveto{\pgfqpoint{3.493903in}{1.160173in}}%
\pgfpathlineto{\pgfqpoint{3.508010in}{1.155831in}}%
\pgfpathlineto{\pgfqpoint{3.522122in}{1.151672in}}%
\pgfpathlineto{\pgfqpoint{3.536239in}{1.147693in}}%
\pgfpathlineto{\pgfqpoint{3.550361in}{1.143896in}}%
\pgfpathlineto{\pgfqpoint{3.558810in}{1.150639in}}%
\pgfpathlineto{\pgfqpoint{3.567250in}{1.157670in}}%
\pgfpathlineto{\pgfqpoint{3.575679in}{1.164983in}}%
\pgfpathlineto{\pgfqpoint{3.584099in}{1.172569in}}%
\pgfpathlineto{\pgfqpoint{3.570000in}{1.175574in}}%
\pgfpathlineto{\pgfqpoint{3.555906in}{1.178760in}}%
\pgfpathlineto{\pgfqpoint{3.541818in}{1.182127in}}%
\pgfpathlineto{\pgfqpoint{3.527735in}{1.185676in}}%
\pgfpathlineto{\pgfqpoint{3.519293in}{1.178870in}}%
\pgfpathlineto{\pgfqpoint{3.510840in}{1.172345in}}%
\pgfpathlineto{\pgfqpoint{3.502377in}{1.166110in}}%
\pgfpathlineto{\pgfqpoint{3.493903in}{1.160173in}}%
\pgfpathclose%
\pgfusepath{fill}%
\end{pgfscope}%
\begin{pgfscope}%
\pgfpathrectangle{\pgfqpoint{1.150000in}{0.150000in}}{\pgfqpoint{5.700000in}{5.700000in}}%
\pgfusepath{clip}%
\pgfsetbuttcap%
\pgfsetroundjoin%
\definecolor{currentfill}{rgb}{0.281412,0.155834,0.469201}%
\pgfsetfillcolor{currentfill}%
\pgfsetfillopacity{0.800000}%
\pgfsetlinewidth{0.000000pt}%
\definecolor{currentstroke}{rgb}{0.000000,0.000000,0.000000}%
\pgfsetstrokecolor{currentstroke}%
\pgfsetdash{}{0pt}%
\pgfpathmoveto{\pgfqpoint{4.033049in}{1.417739in}}%
\pgfpathlineto{\pgfqpoint{4.047270in}{1.421985in}}%
\pgfpathlineto{\pgfqpoint{4.061502in}{1.426408in}}%
\pgfpathlineto{\pgfqpoint{4.075745in}{1.431009in}}%
\pgfpathlineto{\pgfqpoint{4.090000in}{1.435787in}}%
\pgfpathlineto{\pgfqpoint{4.098226in}{1.451163in}}%
\pgfpathlineto{\pgfqpoint{4.106447in}{1.466615in}}%
\pgfpathlineto{\pgfqpoint{4.114663in}{1.482138in}}%
\pgfpathlineto{\pgfqpoint{4.122876in}{1.497725in}}%
\pgfpathlineto{\pgfqpoint{4.108622in}{1.492335in}}%
\pgfpathlineto{\pgfqpoint{4.094379in}{1.487123in}}%
\pgfpathlineto{\pgfqpoint{4.080148in}{1.482089in}}%
\pgfpathlineto{\pgfqpoint{4.065930in}{1.477233in}}%
\pgfpathlineto{\pgfqpoint{4.057716in}{1.462245in}}%
\pgfpathlineto{\pgfqpoint{4.049498in}{1.447330in}}%
\pgfpathlineto{\pgfqpoint{4.041276in}{1.432493in}}%
\pgfpathlineto{\pgfqpoint{4.033049in}{1.417739in}}%
\pgfpathclose%
\pgfusepath{fill}%
\end{pgfscope}%
\begin{pgfscope}%
\pgfpathrectangle{\pgfqpoint{1.150000in}{0.150000in}}{\pgfqpoint{5.700000in}{5.700000in}}%
\pgfusepath{clip}%
\pgfsetbuttcap%
\pgfsetroundjoin%
\definecolor{currentfill}{rgb}{0.190631,0.407061,0.556089}%
\pgfsetfillcolor{currentfill}%
\pgfsetfillopacity{0.800000}%
\pgfsetlinewidth{0.000000pt}%
\definecolor{currentstroke}{rgb}{0.000000,0.000000,0.000000}%
\pgfsetstrokecolor{currentstroke}%
\pgfsetdash{}{0pt}%
\pgfpathmoveto{\pgfqpoint{4.523665in}{2.059346in}}%
\pgfpathlineto{\pgfqpoint{4.538139in}{2.070502in}}%
\pgfpathlineto{\pgfqpoint{4.552629in}{2.081840in}}%
\pgfpathlineto{\pgfqpoint{4.567137in}{2.093361in}}%
\pgfpathlineto{\pgfqpoint{4.581662in}{2.105066in}}%
\pgfpathlineto{\pgfqpoint{4.589781in}{2.121891in}}%
\pgfpathlineto{\pgfqpoint{4.597896in}{2.138614in}}%
\pgfpathlineto{\pgfqpoint{4.606006in}{2.155231in}}%
\pgfpathlineto{\pgfqpoint{4.614111in}{2.171739in}}%
\pgfpathlineto{\pgfqpoint{4.599579in}{2.159671in}}%
\pgfpathlineto{\pgfqpoint{4.585064in}{2.147787in}}%
\pgfpathlineto{\pgfqpoint{4.570566in}{2.136086in}}%
\pgfpathlineto{\pgfqpoint{4.556085in}{2.124569in}}%
\pgfpathlineto{\pgfqpoint{4.547987in}{2.108411in}}%
\pgfpathlineto{\pgfqpoint{4.539884in}{2.092152in}}%
\pgfpathlineto{\pgfqpoint{4.531777in}{2.075796in}}%
\pgfpathlineto{\pgfqpoint{4.523665in}{2.059346in}}%
\pgfpathclose%
\pgfusepath{fill}%
\end{pgfscope}%
\begin{pgfscope}%
\pgfpathrectangle{\pgfqpoint{1.150000in}{0.150000in}}{\pgfqpoint{5.700000in}{5.700000in}}%
\pgfusepath{clip}%
\pgfsetbuttcap%
\pgfsetroundjoin%
\definecolor{currentfill}{rgb}{0.276022,0.044167,0.370164}%
\pgfsetfillcolor{currentfill}%
\pgfsetfillopacity{0.800000}%
\pgfsetlinewidth{0.000000pt}%
\definecolor{currentstroke}{rgb}{0.000000,0.000000,0.000000}%
\pgfsetstrokecolor{currentstroke}%
\pgfsetdash{}{0pt}%
\pgfpathmoveto{\pgfqpoint{3.730552in}{1.194020in}}%
\pgfpathlineto{\pgfqpoint{3.744692in}{1.193392in}}%
\pgfpathlineto{\pgfqpoint{3.758839in}{1.192942in}}%
\pgfpathlineto{\pgfqpoint{3.772995in}{1.192669in}}%
\pgfpathlineto{\pgfqpoint{3.787158in}{1.192574in}}%
\pgfpathlineto{\pgfqpoint{3.795486in}{1.203635in}}%
\pgfpathlineto{\pgfqpoint{3.803808in}{1.214901in}}%
\pgfpathlineto{\pgfqpoint{3.812123in}{1.226364in}}%
\pgfpathlineto{\pgfqpoint{3.820432in}{1.238016in}}%
\pgfpathlineto{\pgfqpoint{3.806280in}{1.237380in}}%
\pgfpathlineto{\pgfqpoint{3.792138in}{1.236922in}}%
\pgfpathlineto{\pgfqpoint{3.778003in}{1.236642in}}%
\pgfpathlineto{\pgfqpoint{3.763877in}{1.236540in}}%
\pgfpathlineto{\pgfqpoint{3.755556in}{1.225607in}}%
\pgfpathlineto{\pgfqpoint{3.747229in}{1.214871in}}%
\pgfpathlineto{\pgfqpoint{3.738894in}{1.204340in}}%
\pgfpathlineto{\pgfqpoint{3.730552in}{1.194020in}}%
\pgfpathclose%
\pgfusepath{fill}%
\end{pgfscope}%
\begin{pgfscope}%
\pgfpathrectangle{\pgfqpoint{1.150000in}{0.150000in}}{\pgfqpoint{5.700000in}{5.700000in}}%
\pgfusepath{clip}%
\pgfsetbuttcap%
\pgfsetroundjoin%
\definecolor{currentfill}{rgb}{0.280267,0.073417,0.397163}%
\pgfsetfillcolor{currentfill}%
\pgfsetfillopacity{0.800000}%
\pgfsetlinewidth{0.000000pt}%
\definecolor{currentstroke}{rgb}{0.000000,0.000000,0.000000}%
\pgfsetstrokecolor{currentstroke}%
\pgfsetdash{}{0pt}%
\pgfpathmoveto{\pgfqpoint{3.820432in}{1.238016in}}%
\pgfpathlineto{\pgfqpoint{3.834591in}{1.238829in}}%
\pgfpathlineto{\pgfqpoint{3.848760in}{1.239821in}}%
\pgfpathlineto{\pgfqpoint{3.862937in}{1.240989in}}%
\pgfpathlineto{\pgfqpoint{3.877124in}{1.242334in}}%
\pgfpathlineto{\pgfqpoint{3.885416in}{1.254882in}}%
\pgfpathlineto{\pgfqpoint{3.893703in}{1.267597in}}%
\pgfpathlineto{\pgfqpoint{3.901984in}{1.280474in}}%
\pgfpathlineto{\pgfqpoint{3.910260in}{1.293505in}}%
\pgfpathlineto{\pgfqpoint{3.896082in}{1.291459in}}%
\pgfpathlineto{\pgfqpoint{3.881913in}{1.289589in}}%
\pgfpathlineto{\pgfqpoint{3.867754in}{1.287898in}}%
\pgfpathlineto{\pgfqpoint{3.853604in}{1.286385in}}%
\pgfpathlineto{\pgfqpoint{3.845320in}{1.274042in}}%
\pgfpathlineto{\pgfqpoint{3.837030in}{1.261862in}}%
\pgfpathlineto{\pgfqpoint{3.828734in}{1.249851in}}%
\pgfpathlineto{\pgfqpoint{3.820432in}{1.238016in}}%
\pgfpathclose%
\pgfusepath{fill}%
\end{pgfscope}%
\begin{pgfscope}%
\pgfpathrectangle{\pgfqpoint{1.150000in}{0.150000in}}{\pgfqpoint{5.700000in}{5.700000in}}%
\pgfusepath{clip}%
\pgfsetbuttcap%
\pgfsetroundjoin%
\definecolor{currentfill}{rgb}{0.165117,0.467423,0.558141}%
\pgfsetfillcolor{currentfill}%
\pgfsetfillopacity{0.800000}%
\pgfsetlinewidth{0.000000pt}%
\definecolor{currentstroke}{rgb}{0.000000,0.000000,0.000000}%
\pgfsetstrokecolor{currentstroke}%
\pgfsetdash{}{0pt}%
\pgfpathmoveto{\pgfqpoint{4.646484in}{2.236624in}}%
\pgfpathlineto{\pgfqpoint{4.661041in}{2.249206in}}%
\pgfpathlineto{\pgfqpoint{4.675617in}{2.261974in}}%
\pgfpathlineto{\pgfqpoint{4.690211in}{2.274926in}}%
\pgfpathlineto{\pgfqpoint{4.704823in}{2.288063in}}%
\pgfpathlineto{\pgfqpoint{4.712911in}{2.304292in}}%
\pgfpathlineto{\pgfqpoint{4.720994in}{2.320387in}}%
\pgfpathlineto{\pgfqpoint{4.729071in}{2.336345in}}%
\pgfpathlineto{\pgfqpoint{4.737143in}{2.352164in}}%
\pgfpathlineto{\pgfqpoint{4.722523in}{2.338729in}}%
\pgfpathlineto{\pgfqpoint{4.707922in}{2.325480in}}%
\pgfpathlineto{\pgfqpoint{4.693339in}{2.312416in}}%
\pgfpathlineto{\pgfqpoint{4.678775in}{2.299537in}}%
\pgfpathlineto{\pgfqpoint{4.670710in}{2.284003in}}%
\pgfpathlineto{\pgfqpoint{4.662640in}{2.268338in}}%
\pgfpathlineto{\pgfqpoint{4.654564in}{2.252544in}}%
\pgfpathlineto{\pgfqpoint{4.646484in}{2.236624in}}%
\pgfpathclose%
\pgfusepath{fill}%
\end{pgfscope}%
\begin{pgfscope}%
\pgfpathrectangle{\pgfqpoint{1.150000in}{0.150000in}}{\pgfqpoint{5.700000in}{5.700000in}}%
\pgfusepath{clip}%
\pgfsetbuttcap%
\pgfsetroundjoin%
\definecolor{currentfill}{rgb}{0.272594,0.025563,0.353093}%
\pgfsetfillcolor{currentfill}%
\pgfsetfillopacity{0.800000}%
\pgfsetlinewidth{0.000000pt}%
\definecolor{currentstroke}{rgb}{0.000000,0.000000,0.000000}%
\pgfsetstrokecolor{currentstroke}%
\pgfsetdash{}{0pt}%
\pgfpathmoveto{\pgfqpoint{3.640553in}{1.162355in}}%
\pgfpathlineto{\pgfqpoint{3.654682in}{1.160251in}}%
\pgfpathlineto{\pgfqpoint{3.668817in}{1.158326in}}%
\pgfpathlineto{\pgfqpoint{3.682959in}{1.156579in}}%
\pgfpathlineto{\pgfqpoint{3.697108in}{1.155011in}}%
\pgfpathlineto{\pgfqpoint{3.705481in}{1.164408in}}%
\pgfpathlineto{\pgfqpoint{3.713846in}{1.174047in}}%
\pgfpathlineto{\pgfqpoint{3.722203in}{1.183921in}}%
\pgfpathlineto{\pgfqpoint{3.730552in}{1.194020in}}%
\pgfpathlineto{\pgfqpoint{3.716419in}{1.194827in}}%
\pgfpathlineto{\pgfqpoint{3.702294in}{1.195813in}}%
\pgfpathlineto{\pgfqpoint{3.688176in}{1.196977in}}%
\pgfpathlineto{\pgfqpoint{3.674065in}{1.198320in}}%
\pgfpathlineto{\pgfqpoint{3.665699in}{1.188970in}}%
\pgfpathlineto{\pgfqpoint{3.657326in}{1.179854in}}%
\pgfpathlineto{\pgfqpoint{3.648944in}{1.170980in}}%
\pgfpathlineto{\pgfqpoint{3.640553in}{1.162355in}}%
\pgfpathclose%
\pgfusepath{fill}%
\end{pgfscope}%
\begin{pgfscope}%
\pgfpathrectangle{\pgfqpoint{1.150000in}{0.150000in}}{\pgfqpoint{5.700000in}{5.700000in}}%
\pgfusepath{clip}%
\pgfsetbuttcap%
\pgfsetroundjoin%
\definecolor{currentfill}{rgb}{0.311925,0.767822,0.415586}%
\pgfsetfillcolor{currentfill}%
\pgfsetfillopacity{0.800000}%
\pgfsetlinewidth{0.000000pt}%
\definecolor{currentstroke}{rgb}{0.000000,0.000000,0.000000}%
\pgfsetstrokecolor{currentstroke}%
\pgfsetdash{}{0pt}%
\pgfpathmoveto{\pgfqpoint{5.382778in}{3.215394in}}%
\pgfpathlineto{\pgfqpoint{5.397874in}{3.233771in}}%
\pgfpathlineto{\pgfqpoint{5.412994in}{3.252341in}}%
\pgfpathlineto{\pgfqpoint{5.428137in}{3.271103in}}%
\pgfpathlineto{\pgfqpoint{5.443306in}{3.290058in}}%
\pgfpathlineto{\pgfqpoint{5.451012in}{3.297760in}}%
\pgfpathlineto{\pgfqpoint{5.458708in}{3.305251in}}%
\pgfpathlineto{\pgfqpoint{5.466392in}{3.312530in}}%
\pgfpathlineto{\pgfqpoint{5.474064in}{3.319601in}}%
\pgfpathlineto{\pgfqpoint{5.458903in}{3.300777in}}%
\pgfpathlineto{\pgfqpoint{5.443766in}{3.282145in}}%
\pgfpathlineto{\pgfqpoint{5.428653in}{3.263706in}}%
\pgfpathlineto{\pgfqpoint{5.413563in}{3.245459in}}%
\pgfpathlineto{\pgfqpoint{5.405883in}{3.238244in}}%
\pgfpathlineto{\pgfqpoint{5.398192in}{3.230830in}}%
\pgfpathlineto{\pgfqpoint{5.390490in}{3.223214in}}%
\pgfpathlineto{\pgfqpoint{5.382778in}{3.215394in}}%
\pgfpathclose%
\pgfusepath{fill}%
\end{pgfscope}%
\begin{pgfscope}%
\pgfpathrectangle{\pgfqpoint{1.150000in}{0.150000in}}{\pgfqpoint{5.700000in}{5.700000in}}%
\pgfusepath{clip}%
\pgfsetbuttcap%
\pgfsetroundjoin%
\definecolor{currentfill}{rgb}{0.282656,0.100196,0.422160}%
\pgfsetfillcolor{currentfill}%
\pgfsetfillopacity{0.800000}%
\pgfsetlinewidth{0.000000pt}%
\definecolor{currentstroke}{rgb}{0.000000,0.000000,0.000000}%
\pgfsetstrokecolor{currentstroke}%
\pgfsetdash{}{0pt}%
\pgfpathmoveto{\pgfqpoint{3.910260in}{1.293505in}}%
\pgfpathlineto{\pgfqpoint{3.924447in}{1.295729in}}%
\pgfpathlineto{\pgfqpoint{3.938645in}{1.298130in}}%
\pgfpathlineto{\pgfqpoint{3.952853in}{1.300708in}}%
\pgfpathlineto{\pgfqpoint{3.967070in}{1.303463in}}%
\pgfpathlineto{\pgfqpoint{3.975334in}{1.317325in}}%
\pgfpathlineto{\pgfqpoint{3.983594in}{1.331320in}}%
\pgfpathlineto{\pgfqpoint{3.991848in}{1.345443in}}%
\pgfpathlineto{\pgfqpoint{4.000097in}{1.359686in}}%
\pgfpathlineto{\pgfqpoint{3.985885in}{1.356259in}}%
\pgfpathlineto{\pgfqpoint{3.971682in}{1.353010in}}%
\pgfpathlineto{\pgfqpoint{3.957490in}{1.349938in}}%
\pgfpathlineto{\pgfqpoint{3.943309in}{1.347043in}}%
\pgfpathlineto{\pgfqpoint{3.935054in}{1.333460in}}%
\pgfpathlineto{\pgfqpoint{3.926795in}{1.320005in}}%
\pgfpathlineto{\pgfqpoint{3.918530in}{1.306684in}}%
\pgfpathlineto{\pgfqpoint{3.910260in}{1.293505in}}%
\pgfpathclose%
\pgfusepath{fill}%
\end{pgfscope}%
\begin{pgfscope}%
\pgfpathrectangle{\pgfqpoint{1.150000in}{0.150000in}}{\pgfqpoint{5.700000in}{5.700000in}}%
\pgfusepath{clip}%
\pgfsetbuttcap%
\pgfsetroundjoin%
\definecolor{currentfill}{rgb}{0.255645,0.260703,0.528312}%
\pgfsetfillcolor{currentfill}%
\pgfsetfillopacity{0.800000}%
\pgfsetlinewidth{0.000000pt}%
\definecolor{currentstroke}{rgb}{0.000000,0.000000,0.000000}%
\pgfsetstrokecolor{currentstroke}%
\pgfsetdash{}{0pt}%
\pgfpathmoveto{\pgfqpoint{4.245594in}{1.651961in}}%
\pgfpathlineto{\pgfqpoint{4.259917in}{1.659377in}}%
\pgfpathlineto{\pgfqpoint{4.274254in}{1.666972in}}%
\pgfpathlineto{\pgfqpoint{4.288605in}{1.674747in}}%
\pgfpathlineto{\pgfqpoint{4.302970in}{1.682701in}}%
\pgfpathlineto{\pgfqpoint{4.311154in}{1.699689in}}%
\pgfpathlineto{\pgfqpoint{4.319335in}{1.716674in}}%
\pgfpathlineto{\pgfqpoint{4.327512in}{1.733650in}}%
\pgfpathlineto{\pgfqpoint{4.335685in}{1.750611in}}%
\pgfpathlineto{\pgfqpoint{4.321315in}{1.742135in}}%
\pgfpathlineto{\pgfqpoint{4.306960in}{1.733839in}}%
\pgfpathlineto{\pgfqpoint{4.292619in}{1.725723in}}%
\pgfpathlineto{\pgfqpoint{4.278292in}{1.717786in}}%
\pgfpathlineto{\pgfqpoint{4.270123in}{1.701334in}}%
\pgfpathlineto{\pgfqpoint{4.261950in}{1.684875in}}%
\pgfpathlineto{\pgfqpoint{4.253774in}{1.668416in}}%
\pgfpathlineto{\pgfqpoint{4.245594in}{1.651961in}}%
\pgfpathclose%
\pgfusepath{fill}%
\end{pgfscope}%
\begin{pgfscope}%
\pgfpathrectangle{\pgfqpoint{1.150000in}{0.150000in}}{\pgfqpoint{5.700000in}{5.700000in}}%
\pgfusepath{clip}%
\pgfsetbuttcap%
\pgfsetroundjoin%
\definecolor{currentfill}{rgb}{0.275191,0.194905,0.496005}%
\pgfsetfillcolor{currentfill}%
\pgfsetfillopacity{0.800000}%
\pgfsetlinewidth{0.000000pt}%
\definecolor{currentstroke}{rgb}{0.000000,0.000000,0.000000}%
\pgfsetstrokecolor{currentstroke}%
\pgfsetdash{}{0pt}%
\pgfpathmoveto{\pgfqpoint{4.122876in}{1.497725in}}%
\pgfpathlineto{\pgfqpoint{4.137143in}{1.503293in}}%
\pgfpathlineto{\pgfqpoint{4.151422in}{1.509040in}}%
\pgfpathlineto{\pgfqpoint{4.165714in}{1.514964in}}%
\pgfpathlineto{\pgfqpoint{4.180019in}{1.521066in}}%
\pgfpathlineto{\pgfqpoint{4.188229in}{1.537306in}}%
\pgfpathlineto{\pgfqpoint{4.196435in}{1.553592in}}%
\pgfpathlineto{\pgfqpoint{4.204637in}{1.569918in}}%
\pgfpathlineto{\pgfqpoint{4.212836in}{1.586278in}}%
\pgfpathlineto{\pgfqpoint{4.198530in}{1.579593in}}%
\pgfpathlineto{\pgfqpoint{4.184236in}{1.573086in}}%
\pgfpathlineto{\pgfqpoint{4.169955in}{1.566758in}}%
\pgfpathlineto{\pgfqpoint{4.155688in}{1.560608in}}%
\pgfpathlineto{\pgfqpoint{4.147491in}{1.544818in}}%
\pgfpathlineto{\pgfqpoint{4.139290in}{1.529071in}}%
\pgfpathlineto{\pgfqpoint{4.131085in}{1.513371in}}%
\pgfpathlineto{\pgfqpoint{4.122876in}{1.497725in}}%
\pgfpathclose%
\pgfusepath{fill}%
\end{pgfscope}%
\begin{pgfscope}%
\pgfpathrectangle{\pgfqpoint{1.150000in}{0.150000in}}{\pgfqpoint{5.700000in}{5.700000in}}%
\pgfusepath{clip}%
\pgfsetbuttcap%
\pgfsetroundjoin%
\definecolor{currentfill}{rgb}{0.227802,0.326594,0.546532}%
\pgfsetfillcolor{currentfill}%
\pgfsetfillopacity{0.800000}%
\pgfsetlinewidth{0.000000pt}%
\definecolor{currentstroke}{rgb}{0.000000,0.000000,0.000000}%
\pgfsetstrokecolor{currentstroke}%
\pgfsetdash{}{0pt}%
\pgfpathmoveto{\pgfqpoint{4.368342in}{1.818229in}}%
\pgfpathlineto{\pgfqpoint{4.382732in}{1.827376in}}%
\pgfpathlineto{\pgfqpoint{4.397138in}{1.836704in}}%
\pgfpathlineto{\pgfqpoint{4.411559in}{1.846213in}}%
\pgfpathlineto{\pgfqpoint{4.425995in}{1.855902in}}%
\pgfpathlineto{\pgfqpoint{4.434156in}{1.873199in}}%
\pgfpathlineto{\pgfqpoint{4.442313in}{1.890446in}}%
\pgfpathlineto{\pgfqpoint{4.450467in}{1.907640in}}%
\pgfpathlineto{\pgfqpoint{4.458616in}{1.924777in}}%
\pgfpathlineto{\pgfqpoint{4.444173in}{1.914627in}}%
\pgfpathlineto{\pgfqpoint{4.429745in}{1.904659in}}%
\pgfpathlineto{\pgfqpoint{4.415334in}{1.894872in}}%
\pgfpathlineto{\pgfqpoint{4.400937in}{1.885266in}}%
\pgfpathlineto{\pgfqpoint{4.392794in}{1.868576in}}%
\pgfpathlineto{\pgfqpoint{4.384647in}{1.851837in}}%
\pgfpathlineto{\pgfqpoint{4.376496in}{1.835054in}}%
\pgfpathlineto{\pgfqpoint{4.368342in}{1.818229in}}%
\pgfpathclose%
\pgfusepath{fill}%
\end{pgfscope}%
\begin{pgfscope}%
\pgfpathrectangle{\pgfqpoint{1.150000in}{0.150000in}}{\pgfqpoint{5.700000in}{5.700000in}}%
\pgfusepath{clip}%
\pgfsetbuttcap%
\pgfsetroundjoin%
\definecolor{currentfill}{rgb}{0.487026,0.823929,0.312321}%
\pgfsetfillcolor{currentfill}%
\pgfsetfillopacity{0.800000}%
\pgfsetlinewidth{0.000000pt}%
\definecolor{currentstroke}{rgb}{0.000000,0.000000,0.000000}%
\pgfsetstrokecolor{currentstroke}%
\pgfsetdash{}{0pt}%
\pgfpathmoveto{\pgfqpoint{5.595846in}{3.444626in}}%
\pgfpathlineto{\pgfqpoint{5.611111in}{3.464041in}}%
\pgfpathlineto{\pgfqpoint{5.626402in}{3.483651in}}%
\pgfpathlineto{\pgfqpoint{5.641717in}{3.503454in}}%
\pgfpathlineto{\pgfqpoint{5.657059in}{3.523452in}}%
\pgfpathlineto{\pgfqpoint{5.664604in}{3.528281in}}%
\pgfpathlineto{\pgfqpoint{5.672136in}{3.532907in}}%
\pgfpathlineto{\pgfqpoint{5.679655in}{3.537332in}}%
\pgfpathlineto{\pgfqpoint{5.687162in}{3.541559in}}%
\pgfpathlineto{\pgfqpoint{5.671834in}{3.521806in}}%
\pgfpathlineto{\pgfqpoint{5.656532in}{3.502247in}}%
\pgfpathlineto{\pgfqpoint{5.641255in}{3.482882in}}%
\pgfpathlineto{\pgfqpoint{5.626003in}{3.463709in}}%
\pgfpathlineto{\pgfqpoint{5.618481in}{3.459224in}}%
\pgfpathlineto{\pgfqpoint{5.610948in}{3.454551in}}%
\pgfpathlineto{\pgfqpoint{5.603403in}{3.449686in}}%
\pgfpathlineto{\pgfqpoint{5.595846in}{3.444626in}}%
\pgfpathclose%
\pgfusepath{fill}%
\end{pgfscope}%
\begin{pgfscope}%
\pgfpathrectangle{\pgfqpoint{1.150000in}{0.150000in}}{\pgfqpoint{5.700000in}{5.700000in}}%
\pgfusepath{clip}%
\pgfsetbuttcap%
\pgfsetroundjoin%
\definecolor{currentfill}{rgb}{0.141935,0.526453,0.555991}%
\pgfsetfillcolor{currentfill}%
\pgfsetfillopacity{0.800000}%
\pgfsetlinewidth{0.000000pt}%
\definecolor{currentstroke}{rgb}{0.000000,0.000000,0.000000}%
\pgfsetstrokecolor{currentstroke}%
\pgfsetdash{}{0pt}%
\pgfpathmoveto{\pgfqpoint{4.769374in}{2.413995in}}%
\pgfpathlineto{\pgfqpoint{4.784020in}{2.427879in}}%
\pgfpathlineto{\pgfqpoint{4.798685in}{2.441949in}}%
\pgfpathlineto{\pgfqpoint{4.813369in}{2.456206in}}%
\pgfpathlineto{\pgfqpoint{4.828073in}{2.470650in}}%
\pgfpathlineto{\pgfqpoint{4.836123in}{2.485978in}}%
\pgfpathlineto{\pgfqpoint{4.844167in}{2.501146in}}%
\pgfpathlineto{\pgfqpoint{4.852204in}{2.516150in}}%
\pgfpathlineto{\pgfqpoint{4.860235in}{2.530988in}}%
\pgfpathlineto{\pgfqpoint{4.845524in}{2.516314in}}%
\pgfpathlineto{\pgfqpoint{4.830833in}{2.501828in}}%
\pgfpathlineto{\pgfqpoint{4.816162in}{2.487528in}}%
\pgfpathlineto{\pgfqpoint{4.801510in}{2.473415in}}%
\pgfpathlineto{\pgfqpoint{4.793485in}{2.458793in}}%
\pgfpathlineto{\pgfqpoint{4.785454in}{2.444014in}}%
\pgfpathlineto{\pgfqpoint{4.777417in}{2.429081in}}%
\pgfpathlineto{\pgfqpoint{4.769374in}{2.413995in}}%
\pgfpathclose%
\pgfusepath{fill}%
\end{pgfscope}%
\begin{pgfscope}%
\pgfpathrectangle{\pgfqpoint{1.150000in}{0.150000in}}{\pgfqpoint{5.700000in}{5.700000in}}%
\pgfusepath{clip}%
\pgfsetbuttcap%
\pgfsetroundjoin%
\definecolor{currentfill}{rgb}{0.226397,0.728888,0.462789}%
\pgfsetfillcolor{currentfill}%
\pgfsetfillopacity{0.800000}%
\pgfsetlinewidth{0.000000pt}%
\definecolor{currentstroke}{rgb}{0.000000,0.000000,0.000000}%
\pgfsetstrokecolor{currentstroke}%
\pgfsetdash{}{0pt}%
\pgfpathmoveto{\pgfqpoint{5.260513in}{3.073153in}}%
\pgfpathlineto{\pgfqpoint{5.275523in}{3.090912in}}%
\pgfpathlineto{\pgfqpoint{5.290556in}{3.108862in}}%
\pgfpathlineto{\pgfqpoint{5.305612in}{3.127005in}}%
\pgfpathlineto{\pgfqpoint{5.320692in}{3.145340in}}%
\pgfpathlineto{\pgfqpoint{5.328490in}{3.154837in}}%
\pgfpathlineto{\pgfqpoint{5.336277in}{3.164121in}}%
\pgfpathlineto{\pgfqpoint{5.344054in}{3.173192in}}%
\pgfpathlineto{\pgfqpoint{5.351820in}{3.182051in}}%
\pgfpathlineto{\pgfqpoint{5.336743in}{3.163773in}}%
\pgfpathlineto{\pgfqpoint{5.321689in}{3.145687in}}%
\pgfpathlineto{\pgfqpoint{5.306659in}{3.127793in}}%
\pgfpathlineto{\pgfqpoint{5.291652in}{3.110090in}}%
\pgfpathlineto{\pgfqpoint{5.283882in}{3.101161in}}%
\pgfpathlineto{\pgfqpoint{5.276103in}{3.092029in}}%
\pgfpathlineto{\pgfqpoint{5.268313in}{3.082693in}}%
\pgfpathlineto{\pgfqpoint{5.260513in}{3.073153in}}%
\pgfpathclose%
\pgfusepath{fill}%
\end{pgfscope}%
\begin{pgfscope}%
\pgfpathrectangle{\pgfqpoint{1.150000in}{0.150000in}}{\pgfqpoint{5.700000in}{5.700000in}}%
\pgfusepath{clip}%
\pgfsetbuttcap%
\pgfsetroundjoin%
\definecolor{currentfill}{rgb}{0.271305,0.019942,0.347269}%
\pgfsetfillcolor{currentfill}%
\pgfsetfillopacity{0.800000}%
\pgfsetlinewidth{0.000000pt}%
\definecolor{currentstroke}{rgb}{0.000000,0.000000,0.000000}%
\pgfsetstrokecolor{currentstroke}%
\pgfsetdash{}{0pt}%
\pgfpathmoveto{\pgfqpoint{3.550361in}{1.143896in}}%
\pgfpathlineto{\pgfqpoint{3.564488in}{1.140280in}}%
\pgfpathlineto{\pgfqpoint{3.578620in}{1.136844in}}%
\pgfpathlineto{\pgfqpoint{3.592758in}{1.133588in}}%
\pgfpathlineto{\pgfqpoint{3.606901in}{1.130511in}}%
\pgfpathlineto{\pgfqpoint{3.615328in}{1.138059in}}%
\pgfpathlineto{\pgfqpoint{3.623745in}{1.145887in}}%
\pgfpathlineto{\pgfqpoint{3.632154in}{1.153989in}}%
\pgfpathlineto{\pgfqpoint{3.640553in}{1.162355in}}%
\pgfpathlineto{\pgfqpoint{3.626430in}{1.164639in}}%
\pgfpathlineto{\pgfqpoint{3.612314in}{1.167102in}}%
\pgfpathlineto{\pgfqpoint{3.598203in}{1.169746in}}%
\pgfpathlineto{\pgfqpoint{3.584099in}{1.172569in}}%
\pgfpathlineto{\pgfqpoint{3.575679in}{1.164983in}}%
\pgfpathlineto{\pgfqpoint{3.567250in}{1.157670in}}%
\pgfpathlineto{\pgfqpoint{3.558810in}{1.150639in}}%
\pgfpathlineto{\pgfqpoint{3.550361in}{1.143896in}}%
\pgfpathclose%
\pgfusepath{fill}%
\end{pgfscope}%
\begin{pgfscope}%
\pgfpathrectangle{\pgfqpoint{1.150000in}{0.150000in}}{\pgfqpoint{5.700000in}{5.700000in}}%
\pgfusepath{clip}%
\pgfsetbuttcap%
\pgfsetroundjoin%
\definecolor{currentfill}{rgb}{0.282884,0.135920,0.453427}%
\pgfsetfillcolor{currentfill}%
\pgfsetfillopacity{0.800000}%
\pgfsetlinewidth{0.000000pt}%
\definecolor{currentstroke}{rgb}{0.000000,0.000000,0.000000}%
\pgfsetstrokecolor{currentstroke}%
\pgfsetdash{}{0pt}%
\pgfpathmoveto{\pgfqpoint{4.000097in}{1.359686in}}%
\pgfpathlineto{\pgfqpoint{4.014321in}{1.363290in}}%
\pgfpathlineto{\pgfqpoint{4.028556in}{1.367071in}}%
\pgfpathlineto{\pgfqpoint{4.042801in}{1.371029in}}%
\pgfpathlineto{\pgfqpoint{4.057058in}{1.375164in}}%
\pgfpathlineto{\pgfqpoint{4.065300in}{1.390176in}}%
\pgfpathlineto{\pgfqpoint{4.073538in}{1.405288in}}%
\pgfpathlineto{\pgfqpoint{4.081771in}{1.420493in}}%
\pgfpathlineto{\pgfqpoint{4.090000in}{1.435787in}}%
\pgfpathlineto{\pgfqpoint{4.075745in}{1.431009in}}%
\pgfpathlineto{\pgfqpoint{4.061502in}{1.426408in}}%
\pgfpathlineto{\pgfqpoint{4.047270in}{1.421985in}}%
\pgfpathlineto{\pgfqpoint{4.033049in}{1.417739in}}%
\pgfpathlineto{\pgfqpoint{4.024818in}{1.403076in}}%
\pgfpathlineto{\pgfqpoint{4.016583in}{1.388509in}}%
\pgfpathlineto{\pgfqpoint{4.008342in}{1.374043in}}%
\pgfpathlineto{\pgfqpoint{4.000097in}{1.359686in}}%
\pgfpathclose%
\pgfusepath{fill}%
\end{pgfscope}%
\begin{pgfscope}%
\pgfpathrectangle{\pgfqpoint{1.150000in}{0.150000in}}{\pgfqpoint{5.700000in}{5.700000in}}%
\pgfusepath{clip}%
\pgfsetbuttcap%
\pgfsetroundjoin%
\definecolor{currentfill}{rgb}{0.123463,0.581687,0.547445}%
\pgfsetfillcolor{currentfill}%
\pgfsetfillopacity{0.800000}%
\pgfsetlinewidth{0.000000pt}%
\definecolor{currentstroke}{rgb}{0.000000,0.000000,0.000000}%
\pgfsetstrokecolor{currentstroke}%
\pgfsetdash{}{0pt}%
\pgfpathmoveto{\pgfqpoint{4.892292in}{2.588658in}}%
\pgfpathlineto{\pgfqpoint{4.907029in}{2.603715in}}%
\pgfpathlineto{\pgfqpoint{4.921786in}{2.618959in}}%
\pgfpathlineto{\pgfqpoint{4.936564in}{2.634391in}}%
\pgfpathlineto{\pgfqpoint{4.951362in}{2.650012in}}%
\pgfpathlineto{\pgfqpoint{4.959365in}{2.664175in}}%
\pgfpathlineto{\pgfqpoint{4.967361in}{2.678154in}}%
\pgfpathlineto{\pgfqpoint{4.975349in}{2.691949in}}%
\pgfpathlineto{\pgfqpoint{4.983330in}{2.705559in}}%
\pgfpathlineto{\pgfqpoint{4.968526in}{2.689778in}}%
\pgfpathlineto{\pgfqpoint{4.953743in}{2.674185in}}%
\pgfpathlineto{\pgfqpoint{4.938981in}{2.658781in}}%
\pgfpathlineto{\pgfqpoint{4.924239in}{2.643565in}}%
\pgfpathlineto{\pgfqpoint{4.916263in}{2.630102in}}%
\pgfpathlineto{\pgfqpoint{4.908280in}{2.616462in}}%
\pgfpathlineto{\pgfqpoint{4.900290in}{2.602647in}}%
\pgfpathlineto{\pgfqpoint{4.892292in}{2.588658in}}%
\pgfpathclose%
\pgfusepath{fill}%
\end{pgfscope}%
\begin{pgfscope}%
\pgfpathrectangle{\pgfqpoint{1.150000in}{0.150000in}}{\pgfqpoint{5.700000in}{5.700000in}}%
\pgfusepath{clip}%
\pgfsetbuttcap%
\pgfsetroundjoin%
\definecolor{currentfill}{rgb}{0.197636,0.391528,0.554969}%
\pgfsetfillcolor{currentfill}%
\pgfsetfillopacity{0.800000}%
\pgfsetlinewidth{0.000000pt}%
\definecolor{currentstroke}{rgb}{0.000000,0.000000,0.000000}%
\pgfsetstrokecolor{currentstroke}%
\pgfsetdash{}{0pt}%
\pgfpathmoveto{\pgfqpoint{4.491174in}{1.992674in}}%
\pgfpathlineto{\pgfqpoint{4.505641in}{2.003434in}}%
\pgfpathlineto{\pgfqpoint{4.520124in}{2.014377in}}%
\pgfpathlineto{\pgfqpoint{4.534624in}{2.025502in}}%
\pgfpathlineto{\pgfqpoint{4.549141in}{2.036809in}}%
\pgfpathlineto{\pgfqpoint{4.557277in}{2.054010in}}%
\pgfpathlineto{\pgfqpoint{4.565410in}{2.071122in}}%
\pgfpathlineto{\pgfqpoint{4.573538in}{2.088142in}}%
\pgfpathlineto{\pgfqpoint{4.581662in}{2.105066in}}%
\pgfpathlineto{\pgfqpoint{4.567137in}{2.093361in}}%
\pgfpathlineto{\pgfqpoint{4.552629in}{2.081840in}}%
\pgfpathlineto{\pgfqpoint{4.538139in}{2.070502in}}%
\pgfpathlineto{\pgfqpoint{4.523665in}{2.059346in}}%
\pgfpathlineto{\pgfqpoint{4.515549in}{2.042806in}}%
\pgfpathlineto{\pgfqpoint{4.507428in}{2.026178in}}%
\pgfpathlineto{\pgfqpoint{4.499303in}{2.009466in}}%
\pgfpathlineto{\pgfqpoint{4.491174in}{1.992674in}}%
\pgfpathclose%
\pgfusepath{fill}%
\end{pgfscope}%
\begin{pgfscope}%
\pgfpathrectangle{\pgfqpoint{1.150000in}{0.150000in}}{\pgfqpoint{5.700000in}{5.700000in}}%
\pgfusepath{clip}%
\pgfsetbuttcap%
\pgfsetroundjoin%
\definecolor{currentfill}{rgb}{0.162016,0.687316,0.499129}%
\pgfsetfillcolor{currentfill}%
\pgfsetfillopacity{0.800000}%
\pgfsetlinewidth{0.000000pt}%
\definecolor{currentstroke}{rgb}{0.000000,0.000000,0.000000}%
\pgfsetstrokecolor{currentstroke}%
\pgfsetdash{}{0pt}%
\pgfpathmoveto{\pgfqpoint{5.137947in}{2.920236in}}%
\pgfpathlineto{\pgfqpoint{5.152867in}{2.937233in}}%
\pgfpathlineto{\pgfqpoint{5.167810in}{2.954421in}}%
\pgfpathlineto{\pgfqpoint{5.182775in}{2.971800in}}%
\pgfpathlineto{\pgfqpoint{5.197763in}{2.989370in}}%
\pgfpathlineto{\pgfqpoint{5.205640in}{3.000573in}}%
\pgfpathlineto{\pgfqpoint{5.213508in}{3.011567in}}%
\pgfpathlineto{\pgfqpoint{5.221366in}{3.022351in}}%
\pgfpathlineto{\pgfqpoint{5.229215in}{3.032927in}}%
\pgfpathlineto{\pgfqpoint{5.214227in}{3.015340in}}%
\pgfpathlineto{\pgfqpoint{5.199261in}{2.997944in}}%
\pgfpathlineto{\pgfqpoint{5.184318in}{2.980739in}}%
\pgfpathlineto{\pgfqpoint{5.169397in}{2.963724in}}%
\pgfpathlineto{\pgfqpoint{5.161548in}{2.953152in}}%
\pgfpathlineto{\pgfqpoint{5.153690in}{2.942380in}}%
\pgfpathlineto{\pgfqpoint{5.145823in}{2.931408in}}%
\pgfpathlineto{\pgfqpoint{5.137947in}{2.920236in}}%
\pgfpathclose%
\pgfusepath{fill}%
\end{pgfscope}%
\begin{pgfscope}%
\pgfpathrectangle{\pgfqpoint{1.150000in}{0.150000in}}{\pgfqpoint{5.700000in}{5.700000in}}%
\pgfusepath{clip}%
\pgfsetbuttcap%
\pgfsetroundjoin%
\definecolor{currentfill}{rgb}{0.123444,0.636809,0.528763}%
\pgfsetfillcolor{currentfill}%
\pgfsetfillopacity{0.800000}%
\pgfsetlinewidth{0.000000pt}%
\definecolor{currentstroke}{rgb}{0.000000,0.000000,0.000000}%
\pgfsetstrokecolor{currentstroke}%
\pgfsetdash{}{0pt}%
\pgfpathmoveto{\pgfqpoint{5.015176in}{2.758127in}}%
\pgfpathlineto{\pgfqpoint{5.030005in}{2.774223in}}%
\pgfpathlineto{\pgfqpoint{5.044855in}{2.790507in}}%
\pgfpathlineto{\pgfqpoint{5.059727in}{2.806982in}}%
\pgfpathlineto{\pgfqpoint{5.074621in}{2.823646in}}%
\pgfpathlineto{\pgfqpoint{5.082566in}{2.836419in}}%
\pgfpathlineto{\pgfqpoint{5.090504in}{2.848993in}}%
\pgfpathlineto{\pgfqpoint{5.098432in}{2.861367in}}%
\pgfpathlineto{\pgfqpoint{5.106353in}{2.873541in}}%
\pgfpathlineto{\pgfqpoint{5.091456in}{2.856788in}}%
\pgfpathlineto{\pgfqpoint{5.076581in}{2.840224in}}%
\pgfpathlineto{\pgfqpoint{5.061728in}{2.823850in}}%
\pgfpathlineto{\pgfqpoint{5.046896in}{2.807665in}}%
\pgfpathlineto{\pgfqpoint{5.038978in}{2.795567in}}%
\pgfpathlineto{\pgfqpoint{5.031052in}{2.783277in}}%
\pgfpathlineto{\pgfqpoint{5.023118in}{2.770797in}}%
\pgfpathlineto{\pgfqpoint{5.015176in}{2.758127in}}%
\pgfpathclose%
\pgfusepath{fill}%
\end{pgfscope}%
\begin{pgfscope}%
\pgfpathrectangle{\pgfqpoint{1.150000in}{0.150000in}}{\pgfqpoint{5.700000in}{5.700000in}}%
\pgfusepath{clip}%
\pgfsetbuttcap%
\pgfsetroundjoin%
\definecolor{currentfill}{rgb}{0.565498,0.842430,0.262877}%
\pgfsetfillcolor{currentfill}%
\pgfsetfillopacity{0.800000}%
\pgfsetlinewidth{0.000000pt}%
\definecolor{currentstroke}{rgb}{0.000000,0.000000,0.000000}%
\pgfsetstrokecolor{currentstroke}%
\pgfsetdash{}{0pt}%
\pgfpathmoveto{\pgfqpoint{5.687162in}{3.541559in}}%
\pgfpathlineto{\pgfqpoint{5.702516in}{3.561506in}}%
\pgfpathlineto{\pgfqpoint{5.717895in}{3.581647in}}%
\pgfpathlineto{\pgfqpoint{5.733301in}{3.601983in}}%
\pgfpathlineto{\pgfqpoint{5.740784in}{3.605813in}}%
\pgfpathlineto{\pgfqpoint{5.748254in}{3.609445in}}%
\pgfpathlineto{\pgfqpoint{5.755712in}{3.612882in}}%
\pgfpathlineto{\pgfqpoint{5.763156in}{3.616127in}}%
\pgfpathlineto{\pgfqpoint{5.747767in}{3.596075in}}%
\pgfpathlineto{\pgfqpoint{5.732403in}{3.576216in}}%
\pgfpathlineto{\pgfqpoint{5.717065in}{3.556551in}}%
\pgfpathlineto{\pgfqpoint{5.709608in}{3.553084in}}%
\pgfpathlineto{\pgfqpoint{5.702139in}{3.549432in}}%
\pgfpathlineto{\pgfqpoint{5.694657in}{3.545591in}}%
\pgfpathlineto{\pgfqpoint{5.687162in}{3.541559in}}%
\pgfpathclose%
\pgfusepath{fill}%
\end{pgfscope}%
\begin{pgfscope}%
\pgfpathrectangle{\pgfqpoint{1.150000in}{0.150000in}}{\pgfqpoint{5.700000in}{5.700000in}}%
\pgfusepath{clip}%
\pgfsetbuttcap%
\pgfsetroundjoin%
\definecolor{currentfill}{rgb}{0.171176,0.452530,0.557965}%
\pgfsetfillcolor{currentfill}%
\pgfsetfillopacity{0.800000}%
\pgfsetlinewidth{0.000000pt}%
\definecolor{currentstroke}{rgb}{0.000000,0.000000,0.000000}%
\pgfsetstrokecolor{currentstroke}%
\pgfsetdash{}{0pt}%
\pgfpathmoveto{\pgfqpoint{4.614111in}{2.171739in}}%
\pgfpathlineto{\pgfqpoint{4.628661in}{2.183991in}}%
\pgfpathlineto{\pgfqpoint{4.643229in}{2.196428in}}%
\pgfpathlineto{\pgfqpoint{4.657815in}{2.209048in}}%
\pgfpathlineto{\pgfqpoint{4.672419in}{2.221853in}}%
\pgfpathlineto{\pgfqpoint{4.680527in}{2.238594in}}%
\pgfpathlineto{\pgfqpoint{4.688631in}{2.255210in}}%
\pgfpathlineto{\pgfqpoint{4.696729in}{2.271701in}}%
\pgfpathlineto{\pgfqpoint{4.704823in}{2.288063in}}%
\pgfpathlineto{\pgfqpoint{4.690211in}{2.274926in}}%
\pgfpathlineto{\pgfqpoint{4.675617in}{2.261974in}}%
\pgfpathlineto{\pgfqpoint{4.661041in}{2.249206in}}%
\pgfpathlineto{\pgfqpoint{4.646484in}{2.236624in}}%
\pgfpathlineto{\pgfqpoint{4.638398in}{2.220581in}}%
\pgfpathlineto{\pgfqpoint{4.630307in}{2.204417in}}%
\pgfpathlineto{\pgfqpoint{4.622212in}{2.188135in}}%
\pgfpathlineto{\pgfqpoint{4.614111in}{2.171739in}}%
\pgfpathclose%
\pgfusepath{fill}%
\end{pgfscope}%
\begin{pgfscope}%
\pgfpathrectangle{\pgfqpoint{1.150000in}{0.150000in}}{\pgfqpoint{5.700000in}{5.700000in}}%
\pgfusepath{clip}%
\pgfsetbuttcap%
\pgfsetroundjoin%
\definecolor{currentfill}{rgb}{0.262138,0.242286,0.520837}%
\pgfsetfillcolor{currentfill}%
\pgfsetfillopacity{0.800000}%
\pgfsetlinewidth{0.000000pt}%
\definecolor{currentstroke}{rgb}{0.000000,0.000000,0.000000}%
\pgfsetstrokecolor{currentstroke}%
\pgfsetdash{}{0pt}%
\pgfpathmoveto{\pgfqpoint{4.212836in}{1.586278in}}%
\pgfpathlineto{\pgfqpoint{4.227156in}{1.593142in}}%
\pgfpathlineto{\pgfqpoint{4.241489in}{1.600184in}}%
\pgfpathlineto{\pgfqpoint{4.255836in}{1.607405in}}%
\pgfpathlineto{\pgfqpoint{4.270197in}{1.614805in}}%
\pgfpathlineto{\pgfqpoint{4.278396in}{1.631760in}}%
\pgfpathlineto{\pgfqpoint{4.286591in}{1.648731in}}%
\pgfpathlineto{\pgfqpoint{4.294782in}{1.665713in}}%
\pgfpathlineto{\pgfqpoint{4.302970in}{1.682701in}}%
\pgfpathlineto{\pgfqpoint{4.288605in}{1.674747in}}%
\pgfpathlineto{\pgfqpoint{4.274254in}{1.666972in}}%
\pgfpathlineto{\pgfqpoint{4.259917in}{1.659377in}}%
\pgfpathlineto{\pgfqpoint{4.245594in}{1.651961in}}%
\pgfpathlineto{\pgfqpoint{4.237410in}{1.635514in}}%
\pgfpathlineto{\pgfqpoint{4.229222in}{1.619082in}}%
\pgfpathlineto{\pgfqpoint{4.221031in}{1.602668in}}%
\pgfpathlineto{\pgfqpoint{4.212836in}{1.586278in}}%
\pgfpathclose%
\pgfusepath{fill}%
\end{pgfscope}%
\begin{pgfscope}%
\pgfpathrectangle{\pgfqpoint{1.150000in}{0.150000in}}{\pgfqpoint{5.700000in}{5.700000in}}%
\pgfusepath{clip}%
\pgfsetbuttcap%
\pgfsetroundjoin%
\definecolor{currentfill}{rgb}{0.395174,0.797475,0.367757}%
\pgfsetfillcolor{currentfill}%
\pgfsetfillopacity{0.800000}%
\pgfsetlinewidth{0.000000pt}%
\definecolor{currentstroke}{rgb}{0.000000,0.000000,0.000000}%
\pgfsetstrokecolor{currentstroke}%
\pgfsetdash{}{0pt}%
\pgfpathmoveto{\pgfqpoint{5.474064in}{3.319601in}}%
\pgfpathlineto{\pgfqpoint{5.489250in}{3.338617in}}%
\pgfpathlineto{\pgfqpoint{5.504461in}{3.357828in}}%
\pgfpathlineto{\pgfqpoint{5.519696in}{3.377232in}}%
\pgfpathlineto{\pgfqpoint{5.534956in}{3.396830in}}%
\pgfpathlineto{\pgfqpoint{5.542609in}{3.403538in}}%
\pgfpathlineto{\pgfqpoint{5.550250in}{3.410032in}}%
\pgfpathlineto{\pgfqpoint{5.557880in}{3.416314in}}%
\pgfpathlineto{\pgfqpoint{5.565497in}{3.422387in}}%
\pgfpathlineto{\pgfqpoint{5.550246in}{3.402958in}}%
\pgfpathlineto{\pgfqpoint{5.535019in}{3.383724in}}%
\pgfpathlineto{\pgfqpoint{5.519818in}{3.364682in}}%
\pgfpathlineto{\pgfqpoint{5.504641in}{3.345833in}}%
\pgfpathlineto{\pgfqpoint{5.497014in}{3.339578in}}%
\pgfpathlineto{\pgfqpoint{5.489375in}{3.333122in}}%
\pgfpathlineto{\pgfqpoint{5.481726in}{3.326464in}}%
\pgfpathlineto{\pgfqpoint{5.474064in}{3.319601in}}%
\pgfpathclose%
\pgfusepath{fill}%
\end{pgfscope}%
\begin{pgfscope}%
\pgfpathrectangle{\pgfqpoint{1.150000in}{0.150000in}}{\pgfqpoint{5.700000in}{5.700000in}}%
\pgfusepath{clip}%
\pgfsetbuttcap%
\pgfsetroundjoin%
\definecolor{currentfill}{rgb}{0.278791,0.062145,0.386592}%
\pgfsetfillcolor{currentfill}%
\pgfsetfillopacity{0.800000}%
\pgfsetlinewidth{0.000000pt}%
\definecolor{currentstroke}{rgb}{0.000000,0.000000,0.000000}%
\pgfsetstrokecolor{currentstroke}%
\pgfsetdash{}{0pt}%
\pgfpathmoveto{\pgfqpoint{3.787158in}{1.192574in}}%
\pgfpathlineto{\pgfqpoint{3.801329in}{1.192656in}}%
\pgfpathlineto{\pgfqpoint{3.815509in}{1.192915in}}%
\pgfpathlineto{\pgfqpoint{3.829697in}{1.193351in}}%
\pgfpathlineto{\pgfqpoint{3.843893in}{1.193963in}}%
\pgfpathlineto{\pgfqpoint{3.852210in}{1.205769in}}%
\pgfpathlineto{\pgfqpoint{3.860521in}{1.217771in}}%
\pgfpathlineto{\pgfqpoint{3.868825in}{1.229961in}}%
\pgfpathlineto{\pgfqpoint{3.877124in}{1.242334in}}%
\pgfpathlineto{\pgfqpoint{3.862937in}{1.240989in}}%
\pgfpathlineto{\pgfqpoint{3.848760in}{1.239821in}}%
\pgfpathlineto{\pgfqpoint{3.834591in}{1.238829in}}%
\pgfpathlineto{\pgfqpoint{3.820432in}{1.238016in}}%
\pgfpathlineto{\pgfqpoint{3.812123in}{1.226364in}}%
\pgfpathlineto{\pgfqpoint{3.803808in}{1.214901in}}%
\pgfpathlineto{\pgfqpoint{3.795486in}{1.203635in}}%
\pgfpathlineto{\pgfqpoint{3.787158in}{1.192574in}}%
\pgfpathclose%
\pgfusepath{fill}%
\end{pgfscope}%
\begin{pgfscope}%
\pgfpathrectangle{\pgfqpoint{1.150000in}{0.150000in}}{\pgfqpoint{5.700000in}{5.700000in}}%
\pgfusepath{clip}%
\pgfsetbuttcap%
\pgfsetroundjoin%
\definecolor{currentfill}{rgb}{0.278826,0.175490,0.483397}%
\pgfsetfillcolor{currentfill}%
\pgfsetfillopacity{0.800000}%
\pgfsetlinewidth{0.000000pt}%
\definecolor{currentstroke}{rgb}{0.000000,0.000000,0.000000}%
\pgfsetstrokecolor{currentstroke}%
\pgfsetdash{}{0pt}%
\pgfpathmoveto{\pgfqpoint{4.090000in}{1.435787in}}%
\pgfpathlineto{\pgfqpoint{4.104267in}{1.440743in}}%
\pgfpathlineto{\pgfqpoint{4.118546in}{1.445876in}}%
\pgfpathlineto{\pgfqpoint{4.132837in}{1.451186in}}%
\pgfpathlineto{\pgfqpoint{4.147140in}{1.456674in}}%
\pgfpathlineto{\pgfqpoint{4.155366in}{1.472675in}}%
\pgfpathlineto{\pgfqpoint{4.163587in}{1.488745in}}%
\pgfpathlineto{\pgfqpoint{4.171805in}{1.504877in}}%
\pgfpathlineto{\pgfqpoint{4.180019in}{1.521066in}}%
\pgfpathlineto{\pgfqpoint{4.165714in}{1.514964in}}%
\pgfpathlineto{\pgfqpoint{4.151422in}{1.509040in}}%
\pgfpathlineto{\pgfqpoint{4.137143in}{1.503293in}}%
\pgfpathlineto{\pgfqpoint{4.122876in}{1.497725in}}%
\pgfpathlineto{\pgfqpoint{4.114663in}{1.482138in}}%
\pgfpathlineto{\pgfqpoint{4.106447in}{1.466615in}}%
\pgfpathlineto{\pgfqpoint{4.098226in}{1.451163in}}%
\pgfpathlineto{\pgfqpoint{4.090000in}{1.435787in}}%
\pgfpathclose%
\pgfusepath{fill}%
\end{pgfscope}%
\begin{pgfscope}%
\pgfpathrectangle{\pgfqpoint{1.150000in}{0.150000in}}{\pgfqpoint{5.700000in}{5.700000in}}%
\pgfusepath{clip}%
\pgfsetbuttcap%
\pgfsetroundjoin%
\definecolor{currentfill}{rgb}{0.237441,0.305202,0.541921}%
\pgfsetfillcolor{currentfill}%
\pgfsetfillopacity{0.800000}%
\pgfsetlinewidth{0.000000pt}%
\definecolor{currentstroke}{rgb}{0.000000,0.000000,0.000000}%
\pgfsetstrokecolor{currentstroke}%
\pgfsetdash{}{0pt}%
\pgfpathmoveto{\pgfqpoint{4.335685in}{1.750611in}}%
\pgfpathlineto{\pgfqpoint{4.350070in}{1.759268in}}%
\pgfpathlineto{\pgfqpoint{4.364470in}{1.768104in}}%
\pgfpathlineto{\pgfqpoint{4.378884in}{1.777120in}}%
\pgfpathlineto{\pgfqpoint{4.393314in}{1.786317in}}%
\pgfpathlineto{\pgfqpoint{4.401490in}{1.803765in}}%
\pgfpathlineto{\pgfqpoint{4.409662in}{1.821181in}}%
\pgfpathlineto{\pgfqpoint{4.417830in}{1.838562in}}%
\pgfpathlineto{\pgfqpoint{4.425995in}{1.855902in}}%
\pgfpathlineto{\pgfqpoint{4.411559in}{1.846213in}}%
\pgfpathlineto{\pgfqpoint{4.397138in}{1.836704in}}%
\pgfpathlineto{\pgfqpoint{4.382732in}{1.827376in}}%
\pgfpathlineto{\pgfqpoint{4.368342in}{1.818229in}}%
\pgfpathlineto{\pgfqpoint{4.360183in}{1.801368in}}%
\pgfpathlineto{\pgfqpoint{4.352021in}{1.784475in}}%
\pgfpathlineto{\pgfqpoint{4.343855in}{1.767555in}}%
\pgfpathlineto{\pgfqpoint{4.335685in}{1.750611in}}%
\pgfpathclose%
\pgfusepath{fill}%
\end{pgfscope}%
\begin{pgfscope}%
\pgfpathrectangle{\pgfqpoint{1.150000in}{0.150000in}}{\pgfqpoint{5.700000in}{5.700000in}}%
\pgfusepath{clip}%
\pgfsetbuttcap%
\pgfsetroundjoin%
\definecolor{currentfill}{rgb}{0.274952,0.037752,0.364543}%
\pgfsetfillcolor{currentfill}%
\pgfsetfillopacity{0.800000}%
\pgfsetlinewidth{0.000000pt}%
\definecolor{currentstroke}{rgb}{0.000000,0.000000,0.000000}%
\pgfsetstrokecolor{currentstroke}%
\pgfsetdash{}{0pt}%
\pgfpathmoveto{\pgfqpoint{3.697108in}{1.155011in}}%
\pgfpathlineto{\pgfqpoint{3.711264in}{1.153621in}}%
\pgfpathlineto{\pgfqpoint{3.725426in}{1.152408in}}%
\pgfpathlineto{\pgfqpoint{3.739596in}{1.151373in}}%
\pgfpathlineto{\pgfqpoint{3.753774in}{1.150515in}}%
\pgfpathlineto{\pgfqpoint{3.762131in}{1.160687in}}%
\pgfpathlineto{\pgfqpoint{3.770480in}{1.171092in}}%
\pgfpathlineto{\pgfqpoint{3.778823in}{1.181724in}}%
\pgfpathlineto{\pgfqpoint{3.787158in}{1.192574in}}%
\pgfpathlineto{\pgfqpoint{3.772995in}{1.192669in}}%
\pgfpathlineto{\pgfqpoint{3.758839in}{1.192942in}}%
\pgfpathlineto{\pgfqpoint{3.744692in}{1.193392in}}%
\pgfpathlineto{\pgfqpoint{3.730552in}{1.194020in}}%
\pgfpathlineto{\pgfqpoint{3.722203in}{1.183921in}}%
\pgfpathlineto{\pgfqpoint{3.713846in}{1.174047in}}%
\pgfpathlineto{\pgfqpoint{3.705481in}{1.164408in}}%
\pgfpathlineto{\pgfqpoint{3.697108in}{1.155011in}}%
\pgfpathclose%
\pgfusepath{fill}%
\end{pgfscope}%
\begin{pgfscope}%
\pgfpathrectangle{\pgfqpoint{1.150000in}{0.150000in}}{\pgfqpoint{5.700000in}{5.700000in}}%
\pgfusepath{clip}%
\pgfsetbuttcap%
\pgfsetroundjoin%
\definecolor{currentfill}{rgb}{0.281924,0.089666,0.412415}%
\pgfsetfillcolor{currentfill}%
\pgfsetfillopacity{0.800000}%
\pgfsetlinewidth{0.000000pt}%
\definecolor{currentstroke}{rgb}{0.000000,0.000000,0.000000}%
\pgfsetstrokecolor{currentstroke}%
\pgfsetdash{}{0pt}%
\pgfpathmoveto{\pgfqpoint{3.877124in}{1.242334in}}%
\pgfpathlineto{\pgfqpoint{3.891319in}{1.243856in}}%
\pgfpathlineto{\pgfqpoint{3.905524in}{1.245555in}}%
\pgfpathlineto{\pgfqpoint{3.919738in}{1.247430in}}%
\pgfpathlineto{\pgfqpoint{3.933962in}{1.249481in}}%
\pgfpathlineto{\pgfqpoint{3.942247in}{1.262743in}}%
\pgfpathlineto{\pgfqpoint{3.950527in}{1.276165in}}%
\pgfpathlineto{\pgfqpoint{3.958801in}{1.289740in}}%
\pgfpathlineto{\pgfqpoint{3.967070in}{1.303463in}}%
\pgfpathlineto{\pgfqpoint{3.952853in}{1.300708in}}%
\pgfpathlineto{\pgfqpoint{3.938645in}{1.298130in}}%
\pgfpathlineto{\pgfqpoint{3.924447in}{1.295729in}}%
\pgfpathlineto{\pgfqpoint{3.910260in}{1.293505in}}%
\pgfpathlineto{\pgfqpoint{3.901984in}{1.280474in}}%
\pgfpathlineto{\pgfqpoint{3.893703in}{1.267597in}}%
\pgfpathlineto{\pgfqpoint{3.885416in}{1.254882in}}%
\pgfpathlineto{\pgfqpoint{3.877124in}{1.242334in}}%
\pgfpathclose%
\pgfusepath{fill}%
\end{pgfscope}%
\begin{pgfscope}%
\pgfpathrectangle{\pgfqpoint{1.150000in}{0.150000in}}{\pgfqpoint{5.700000in}{5.700000in}}%
\pgfusepath{clip}%
\pgfsetbuttcap%
\pgfsetroundjoin%
\definecolor{currentfill}{rgb}{0.147607,0.511733,0.557049}%
\pgfsetfillcolor{currentfill}%
\pgfsetfillopacity{0.800000}%
\pgfsetlinewidth{0.000000pt}%
\definecolor{currentstroke}{rgb}{0.000000,0.000000,0.000000}%
\pgfsetstrokecolor{currentstroke}%
\pgfsetdash{}{0pt}%
\pgfpathmoveto{\pgfqpoint{4.737143in}{2.352164in}}%
\pgfpathlineto{\pgfqpoint{4.751782in}{2.365784in}}%
\pgfpathlineto{\pgfqpoint{4.766440in}{2.379591in}}%
\pgfpathlineto{\pgfqpoint{4.781116in}{2.393583in}}%
\pgfpathlineto{\pgfqpoint{4.795812in}{2.407762in}}%
\pgfpathlineto{\pgfqpoint{4.803887in}{2.423715in}}%
\pgfpathlineto{\pgfqpoint{4.811955in}{2.439516in}}%
\pgfpathlineto{\pgfqpoint{4.820017in}{2.455162in}}%
\pgfpathlineto{\pgfqpoint{4.828073in}{2.470650in}}%
\pgfpathlineto{\pgfqpoint{4.813369in}{2.456206in}}%
\pgfpathlineto{\pgfqpoint{4.798685in}{2.441949in}}%
\pgfpathlineto{\pgfqpoint{4.784020in}{2.427879in}}%
\pgfpathlineto{\pgfqpoint{4.769374in}{2.413995in}}%
\pgfpathlineto{\pgfqpoint{4.761325in}{2.398758in}}%
\pgfpathlineto{\pgfqpoint{4.753270in}{2.383372in}}%
\pgfpathlineto{\pgfqpoint{4.745209in}{2.367840in}}%
\pgfpathlineto{\pgfqpoint{4.737143in}{2.352164in}}%
\pgfpathclose%
\pgfusepath{fill}%
\end{pgfscope}%
\begin{pgfscope}%
\pgfpathrectangle{\pgfqpoint{1.150000in}{0.150000in}}{\pgfqpoint{5.700000in}{5.700000in}}%
\pgfusepath{clip}%
\pgfsetbuttcap%
\pgfsetroundjoin%
\definecolor{currentfill}{rgb}{0.206756,0.371758,0.553117}%
\pgfsetfillcolor{currentfill}%
\pgfsetfillopacity{0.800000}%
\pgfsetlinewidth{0.000000pt}%
\definecolor{currentstroke}{rgb}{0.000000,0.000000,0.000000}%
\pgfsetstrokecolor{currentstroke}%
\pgfsetdash{}{0pt}%
\pgfpathmoveto{\pgfqpoint{4.458616in}{1.924777in}}%
\pgfpathlineto{\pgfqpoint{4.473076in}{1.935109in}}%
\pgfpathlineto{\pgfqpoint{4.487551in}{1.945623in}}%
\pgfpathlineto{\pgfqpoint{4.502043in}{1.956318in}}%
\pgfpathlineto{\pgfqpoint{4.516552in}{1.967195in}}%
\pgfpathlineto{\pgfqpoint{4.524705in}{1.984713in}}%
\pgfpathlineto{\pgfqpoint{4.532854in}{2.002157in}}%
\pgfpathlineto{\pgfqpoint{4.541000in}{2.019524in}}%
\pgfpathlineto{\pgfqpoint{4.549141in}{2.036809in}}%
\pgfpathlineto{\pgfqpoint{4.534624in}{2.025502in}}%
\pgfpathlineto{\pgfqpoint{4.520124in}{2.014377in}}%
\pgfpathlineto{\pgfqpoint{4.505641in}{2.003434in}}%
\pgfpathlineto{\pgfqpoint{4.491174in}{1.992674in}}%
\pgfpathlineto{\pgfqpoint{4.483041in}{1.975805in}}%
\pgfpathlineto{\pgfqpoint{4.474903in}{1.958864in}}%
\pgfpathlineto{\pgfqpoint{4.466762in}{1.941853in}}%
\pgfpathlineto{\pgfqpoint{4.458616in}{1.924777in}}%
\pgfpathclose%
\pgfusepath{fill}%
\end{pgfscope}%
\begin{pgfscope}%
\pgfpathrectangle{\pgfqpoint{1.150000in}{0.150000in}}{\pgfqpoint{5.700000in}{5.700000in}}%
\pgfusepath{clip}%
\pgfsetbuttcap%
\pgfsetroundjoin%
\definecolor{currentfill}{rgb}{0.272594,0.025563,0.353093}%
\pgfsetfillcolor{currentfill}%
\pgfsetfillopacity{0.800000}%
\pgfsetlinewidth{0.000000pt}%
\definecolor{currentstroke}{rgb}{0.000000,0.000000,0.000000}%
\pgfsetstrokecolor{currentstroke}%
\pgfsetdash{}{0pt}%
\pgfpathmoveto{\pgfqpoint{3.606901in}{1.130511in}}%
\pgfpathlineto{\pgfqpoint{3.621050in}{1.127614in}}%
\pgfpathlineto{\pgfqpoint{3.635205in}{1.124895in}}%
\pgfpathlineto{\pgfqpoint{3.649366in}{1.122355in}}%
\pgfpathlineto{\pgfqpoint{3.663533in}{1.119993in}}%
\pgfpathlineto{\pgfqpoint{3.671940in}{1.128346in}}%
\pgfpathlineto{\pgfqpoint{3.680338in}{1.136972in}}%
\pgfpathlineto{\pgfqpoint{3.688727in}{1.145863in}}%
\pgfpathlineto{\pgfqpoint{3.697108in}{1.155011in}}%
\pgfpathlineto{\pgfqpoint{3.682959in}{1.156579in}}%
\pgfpathlineto{\pgfqpoint{3.668817in}{1.158326in}}%
\pgfpathlineto{\pgfqpoint{3.654682in}{1.160251in}}%
\pgfpathlineto{\pgfqpoint{3.640553in}{1.162355in}}%
\pgfpathlineto{\pgfqpoint{3.632154in}{1.153989in}}%
\pgfpathlineto{\pgfqpoint{3.623745in}{1.145887in}}%
\pgfpathlineto{\pgfqpoint{3.615328in}{1.138059in}}%
\pgfpathlineto{\pgfqpoint{3.606901in}{1.130511in}}%
\pgfpathclose%
\pgfusepath{fill}%
\end{pgfscope}%
\begin{pgfscope}%
\pgfpathrectangle{\pgfqpoint{1.150000in}{0.150000in}}{\pgfqpoint{5.700000in}{5.700000in}}%
\pgfusepath{clip}%
\pgfsetbuttcap%
\pgfsetroundjoin%
\definecolor{currentfill}{rgb}{0.304148,0.764704,0.419943}%
\pgfsetfillcolor{currentfill}%
\pgfsetfillopacity{0.800000}%
\pgfsetlinewidth{0.000000pt}%
\definecolor{currentstroke}{rgb}{0.000000,0.000000,0.000000}%
\pgfsetstrokecolor{currentstroke}%
\pgfsetdash{}{0pt}%
\pgfpathmoveto{\pgfqpoint{5.351820in}{3.182051in}}%
\pgfpathlineto{\pgfqpoint{5.366920in}{3.200522in}}%
\pgfpathlineto{\pgfqpoint{5.382045in}{3.219185in}}%
\pgfpathlineto{\pgfqpoint{5.397194in}{3.238041in}}%
\pgfpathlineto{\pgfqpoint{5.412366in}{3.257091in}}%
\pgfpathlineto{\pgfqpoint{5.420118in}{3.265659in}}%
\pgfpathlineto{\pgfqpoint{5.427858in}{3.274009in}}%
\pgfpathlineto{\pgfqpoint{5.435588in}{3.282141in}}%
\pgfpathlineto{\pgfqpoint{5.443306in}{3.290058in}}%
\pgfpathlineto{\pgfqpoint{5.428137in}{3.271103in}}%
\pgfpathlineto{\pgfqpoint{5.412994in}{3.252341in}}%
\pgfpathlineto{\pgfqpoint{5.397874in}{3.233771in}}%
\pgfpathlineto{\pgfqpoint{5.382778in}{3.215394in}}%
\pgfpathlineto{\pgfqpoint{5.375054in}{3.207370in}}%
\pgfpathlineto{\pgfqpoint{5.367320in}{3.199139in}}%
\pgfpathlineto{\pgfqpoint{5.359575in}{3.190700in}}%
\pgfpathlineto{\pgfqpoint{5.351820in}{3.182051in}}%
\pgfpathclose%
\pgfusepath{fill}%
\end{pgfscope}%
\begin{pgfscope}%
\pgfpathrectangle{\pgfqpoint{1.150000in}{0.150000in}}{\pgfqpoint{5.700000in}{5.700000in}}%
\pgfusepath{clip}%
\pgfsetbuttcap%
\pgfsetroundjoin%
\definecolor{currentfill}{rgb}{0.283229,0.120777,0.440584}%
\pgfsetfillcolor{currentfill}%
\pgfsetfillopacity{0.800000}%
\pgfsetlinewidth{0.000000pt}%
\definecolor{currentstroke}{rgb}{0.000000,0.000000,0.000000}%
\pgfsetstrokecolor{currentstroke}%
\pgfsetdash{}{0pt}%
\pgfpathmoveto{\pgfqpoint{3.967070in}{1.303463in}}%
\pgfpathlineto{\pgfqpoint{3.981298in}{1.306394in}}%
\pgfpathlineto{\pgfqpoint{3.995536in}{1.309502in}}%
\pgfpathlineto{\pgfqpoint{4.009785in}{1.312786in}}%
\pgfpathlineto{\pgfqpoint{4.024045in}{1.316247in}}%
\pgfpathlineto{\pgfqpoint{4.032305in}{1.330794in}}%
\pgfpathlineto{\pgfqpoint{4.040560in}{1.345467in}}%
\pgfpathlineto{\pgfqpoint{4.048811in}{1.360259in}}%
\pgfpathlineto{\pgfqpoint{4.057058in}{1.375164in}}%
\pgfpathlineto{\pgfqpoint{4.042801in}{1.371029in}}%
\pgfpathlineto{\pgfqpoint{4.028556in}{1.367071in}}%
\pgfpathlineto{\pgfqpoint{4.014321in}{1.363290in}}%
\pgfpathlineto{\pgfqpoint{4.000097in}{1.359686in}}%
\pgfpathlineto{\pgfqpoint{3.991848in}{1.345443in}}%
\pgfpathlineto{\pgfqpoint{3.983594in}{1.331320in}}%
\pgfpathlineto{\pgfqpoint{3.975334in}{1.317325in}}%
\pgfpathlineto{\pgfqpoint{3.967070in}{1.303463in}}%
\pgfpathclose%
\pgfusepath{fill}%
\end{pgfscope}%
\begin{pgfscope}%
\pgfpathrectangle{\pgfqpoint{1.150000in}{0.150000in}}{\pgfqpoint{5.700000in}{5.700000in}}%
\pgfusepath{clip}%
\pgfsetbuttcap%
\pgfsetroundjoin%
\definecolor{currentfill}{rgb}{0.126453,0.570633,0.549841}%
\pgfsetfillcolor{currentfill}%
\pgfsetfillopacity{0.800000}%
\pgfsetlinewidth{0.000000pt}%
\definecolor{currentstroke}{rgb}{0.000000,0.000000,0.000000}%
\pgfsetstrokecolor{currentstroke}%
\pgfsetdash{}{0pt}%
\pgfpathmoveto{\pgfqpoint{4.860235in}{2.530988in}}%
\pgfpathlineto{\pgfqpoint{4.874966in}{2.545849in}}%
\pgfpathlineto{\pgfqpoint{4.889717in}{2.560898in}}%
\pgfpathlineto{\pgfqpoint{4.904488in}{2.576135in}}%
\pgfpathlineto{\pgfqpoint{4.919280in}{2.591559in}}%
\pgfpathlineto{\pgfqpoint{4.927311in}{2.606440in}}%
\pgfpathlineto{\pgfqpoint{4.935335in}{2.621143in}}%
\pgfpathlineto{\pgfqpoint{4.943352in}{2.635668in}}%
\pgfpathlineto{\pgfqpoint{4.951362in}{2.650012in}}%
\pgfpathlineto{\pgfqpoint{4.936564in}{2.634391in}}%
\pgfpathlineto{\pgfqpoint{4.921786in}{2.618959in}}%
\pgfpathlineto{\pgfqpoint{4.907029in}{2.603715in}}%
\pgfpathlineto{\pgfqpoint{4.892292in}{2.588658in}}%
\pgfpathlineto{\pgfqpoint{4.884288in}{2.574497in}}%
\pgfpathlineto{\pgfqpoint{4.876277in}{2.560163in}}%
\pgfpathlineto{\pgfqpoint{4.868260in}{2.545660in}}%
\pgfpathlineto{\pgfqpoint{4.860235in}{2.530988in}}%
\pgfpathclose%
\pgfusepath{fill}%
\end{pgfscope}%
\begin{pgfscope}%
\pgfpathrectangle{\pgfqpoint{1.150000in}{0.150000in}}{\pgfqpoint{5.700000in}{5.700000in}}%
\pgfusepath{clip}%
\pgfsetbuttcap%
\pgfsetroundjoin%
\definecolor{currentfill}{rgb}{0.220124,0.725509,0.466226}%
\pgfsetfillcolor{currentfill}%
\pgfsetfillopacity{0.800000}%
\pgfsetlinewidth{0.000000pt}%
\definecolor{currentstroke}{rgb}{0.000000,0.000000,0.000000}%
\pgfsetstrokecolor{currentstroke}%
\pgfsetdash{}{0pt}%
\pgfpathmoveto{\pgfqpoint{5.229215in}{3.032927in}}%
\pgfpathlineto{\pgfqpoint{5.244226in}{3.050706in}}%
\pgfpathlineto{\pgfqpoint{5.259260in}{3.068676in}}%
\pgfpathlineto{\pgfqpoint{5.274318in}{3.086838in}}%
\pgfpathlineto{\pgfqpoint{5.289399in}{3.105193in}}%
\pgfpathlineto{\pgfqpoint{5.297237in}{3.115555in}}%
\pgfpathlineto{\pgfqpoint{5.305066in}{3.125699in}}%
\pgfpathlineto{\pgfqpoint{5.312884in}{3.135627in}}%
\pgfpathlineto{\pgfqpoint{5.320692in}{3.145340in}}%
\pgfpathlineto{\pgfqpoint{5.305612in}{3.127005in}}%
\pgfpathlineto{\pgfqpoint{5.290556in}{3.108862in}}%
\pgfpathlineto{\pgfqpoint{5.275523in}{3.090912in}}%
\pgfpathlineto{\pgfqpoint{5.260513in}{3.073153in}}%
\pgfpathlineto{\pgfqpoint{5.252703in}{3.063407in}}%
\pgfpathlineto{\pgfqpoint{5.244884in}{3.053454in}}%
\pgfpathlineto{\pgfqpoint{5.237054in}{3.043295in}}%
\pgfpathlineto{\pgfqpoint{5.229215in}{3.032927in}}%
\pgfpathclose%
\pgfusepath{fill}%
\end{pgfscope}%
\begin{pgfscope}%
\pgfpathrectangle{\pgfqpoint{1.150000in}{0.150000in}}{\pgfqpoint{5.700000in}{5.700000in}}%
\pgfusepath{clip}%
\pgfsetbuttcap%
\pgfsetroundjoin%
\definecolor{currentfill}{rgb}{0.179019,0.433756,0.557430}%
\pgfsetfillcolor{currentfill}%
\pgfsetfillopacity{0.800000}%
\pgfsetlinewidth{0.000000pt}%
\definecolor{currentstroke}{rgb}{0.000000,0.000000,0.000000}%
\pgfsetstrokecolor{currentstroke}%
\pgfsetdash{}{0pt}%
\pgfpathmoveto{\pgfqpoint{4.581662in}{2.105066in}}%
\pgfpathlineto{\pgfqpoint{4.596204in}{2.116954in}}%
\pgfpathlineto{\pgfqpoint{4.610763in}{2.129025in}}%
\pgfpathlineto{\pgfqpoint{4.625341in}{2.141281in}}%
\pgfpathlineto{\pgfqpoint{4.639936in}{2.153720in}}%
\pgfpathlineto{\pgfqpoint{4.648064in}{2.170923in}}%
\pgfpathlineto{\pgfqpoint{4.656187in}{2.188015in}}%
\pgfpathlineto{\pgfqpoint{4.664305in}{2.204993in}}%
\pgfpathlineto{\pgfqpoint{4.672419in}{2.221853in}}%
\pgfpathlineto{\pgfqpoint{4.657815in}{2.209048in}}%
\pgfpathlineto{\pgfqpoint{4.643229in}{2.196428in}}%
\pgfpathlineto{\pgfqpoint{4.628661in}{2.183991in}}%
\pgfpathlineto{\pgfqpoint{4.614111in}{2.171739in}}%
\pgfpathlineto{\pgfqpoint{4.606006in}{2.155231in}}%
\pgfpathlineto{\pgfqpoint{4.597896in}{2.138614in}}%
\pgfpathlineto{\pgfqpoint{4.589781in}{2.121891in}}%
\pgfpathlineto{\pgfqpoint{4.581662in}{2.105066in}}%
\pgfpathclose%
\pgfusepath{fill}%
\end{pgfscope}%
\begin{pgfscope}%
\pgfpathrectangle{\pgfqpoint{1.150000in}{0.150000in}}{\pgfqpoint{5.700000in}{5.700000in}}%
\pgfusepath{clip}%
\pgfsetbuttcap%
\pgfsetroundjoin%
\definecolor{currentfill}{rgb}{0.120638,0.625828,0.533488}%
\pgfsetfillcolor{currentfill}%
\pgfsetfillopacity{0.800000}%
\pgfsetlinewidth{0.000000pt}%
\definecolor{currentstroke}{rgb}{0.000000,0.000000,0.000000}%
\pgfsetstrokecolor{currentstroke}%
\pgfsetdash{}{0pt}%
\pgfpathmoveto{\pgfqpoint{4.983330in}{2.705559in}}%
\pgfpathlineto{\pgfqpoint{4.998155in}{2.721530in}}%
\pgfpathlineto{\pgfqpoint{5.013000in}{2.737689in}}%
\pgfpathlineto{\pgfqpoint{5.027867in}{2.754038in}}%
\pgfpathlineto{\pgfqpoint{5.042756in}{2.770577in}}%
\pgfpathlineto{\pgfqpoint{5.050734in}{2.784139in}}%
\pgfpathlineto{\pgfqpoint{5.058705in}{2.797505in}}%
\pgfpathlineto{\pgfqpoint{5.066667in}{2.810674in}}%
\pgfpathlineto{\pgfqpoint{5.074621in}{2.823646in}}%
\pgfpathlineto{\pgfqpoint{5.059727in}{2.806982in}}%
\pgfpathlineto{\pgfqpoint{5.044855in}{2.790507in}}%
\pgfpathlineto{\pgfqpoint{5.030005in}{2.774223in}}%
\pgfpathlineto{\pgfqpoint{5.015176in}{2.758127in}}%
\pgfpathlineto{\pgfqpoint{5.007226in}{2.745268in}}%
\pgfpathlineto{\pgfqpoint{4.999269in}{2.732219in}}%
\pgfpathlineto{\pgfqpoint{4.991303in}{2.718983in}}%
\pgfpathlineto{\pgfqpoint{4.983330in}{2.705559in}}%
\pgfpathclose%
\pgfusepath{fill}%
\end{pgfscope}%
\begin{pgfscope}%
\pgfpathrectangle{\pgfqpoint{1.150000in}{0.150000in}}{\pgfqpoint{5.700000in}{5.700000in}}%
\pgfusepath{clip}%
\pgfsetbuttcap%
\pgfsetroundjoin%
\definecolor{currentfill}{rgb}{0.153894,0.680203,0.504172}%
\pgfsetfillcolor{currentfill}%
\pgfsetfillopacity{0.800000}%
\pgfsetlinewidth{0.000000pt}%
\definecolor{currentstroke}{rgb}{0.000000,0.000000,0.000000}%
\pgfsetstrokecolor{currentstroke}%
\pgfsetdash{}{0pt}%
\pgfpathmoveto{\pgfqpoint{5.106353in}{2.873541in}}%
\pgfpathlineto{\pgfqpoint{5.121271in}{2.890485in}}%
\pgfpathlineto{\pgfqpoint{5.136212in}{2.907620in}}%
\pgfpathlineto{\pgfqpoint{5.151175in}{2.924945in}}%
\pgfpathlineto{\pgfqpoint{5.166161in}{2.942462in}}%
\pgfpathlineto{\pgfqpoint{5.174075in}{2.954503in}}%
\pgfpathlineto{\pgfqpoint{5.181980in}{2.966335in}}%
\pgfpathlineto{\pgfqpoint{5.189876in}{2.977957in}}%
\pgfpathlineto{\pgfqpoint{5.197763in}{2.989370in}}%
\pgfpathlineto{\pgfqpoint{5.182775in}{2.971800in}}%
\pgfpathlineto{\pgfqpoint{5.167810in}{2.954421in}}%
\pgfpathlineto{\pgfqpoint{5.152867in}{2.937233in}}%
\pgfpathlineto{\pgfqpoint{5.137947in}{2.920236in}}%
\pgfpathlineto{\pgfqpoint{5.130061in}{2.908863in}}%
\pgfpathlineto{\pgfqpoint{5.122167in}{2.897289in}}%
\pgfpathlineto{\pgfqpoint{5.114264in}{2.885516in}}%
\pgfpathlineto{\pgfqpoint{5.106353in}{2.873541in}}%
\pgfpathclose%
\pgfusepath{fill}%
\end{pgfscope}%
\begin{pgfscope}%
\pgfpathrectangle{\pgfqpoint{1.150000in}{0.150000in}}{\pgfqpoint{5.700000in}{5.700000in}}%
\pgfusepath{clip}%
\pgfsetbuttcap%
\pgfsetroundjoin%
\definecolor{currentfill}{rgb}{0.269308,0.218818,0.509577}%
\pgfsetfillcolor{currentfill}%
\pgfsetfillopacity{0.800000}%
\pgfsetlinewidth{0.000000pt}%
\definecolor{currentstroke}{rgb}{0.000000,0.000000,0.000000}%
\pgfsetstrokecolor{currentstroke}%
\pgfsetdash{}{0pt}%
\pgfpathmoveto{\pgfqpoint{4.180019in}{1.521066in}}%
\pgfpathlineto{\pgfqpoint{4.194336in}{1.527346in}}%
\pgfpathlineto{\pgfqpoint{4.208666in}{1.533804in}}%
\pgfpathlineto{\pgfqpoint{4.223010in}{1.540440in}}%
\pgfpathlineto{\pgfqpoint{4.237367in}{1.547253in}}%
\pgfpathlineto{\pgfqpoint{4.245580in}{1.564090in}}%
\pgfpathlineto{\pgfqpoint{4.253789in}{1.580965in}}%
\pgfpathlineto{\pgfqpoint{4.261995in}{1.597872in}}%
\pgfpathlineto{\pgfqpoint{4.270197in}{1.614805in}}%
\pgfpathlineto{\pgfqpoint{4.255836in}{1.607405in}}%
\pgfpathlineto{\pgfqpoint{4.241489in}{1.600184in}}%
\pgfpathlineto{\pgfqpoint{4.227156in}{1.593142in}}%
\pgfpathlineto{\pgfqpoint{4.212836in}{1.586278in}}%
\pgfpathlineto{\pgfqpoint{4.204637in}{1.569918in}}%
\pgfpathlineto{\pgfqpoint{4.196435in}{1.553592in}}%
\pgfpathlineto{\pgfqpoint{4.188229in}{1.537306in}}%
\pgfpathlineto{\pgfqpoint{4.180019in}{1.521066in}}%
\pgfpathclose%
\pgfusepath{fill}%
\end{pgfscope}%
\begin{pgfscope}%
\pgfpathrectangle{\pgfqpoint{1.150000in}{0.150000in}}{\pgfqpoint{5.700000in}{5.700000in}}%
\pgfusepath{clip}%
\pgfsetbuttcap%
\pgfsetroundjoin%
\definecolor{currentfill}{rgb}{0.244972,0.287675,0.537260}%
\pgfsetfillcolor{currentfill}%
\pgfsetfillopacity{0.800000}%
\pgfsetlinewidth{0.000000pt}%
\definecolor{currentstroke}{rgb}{0.000000,0.000000,0.000000}%
\pgfsetstrokecolor{currentstroke}%
\pgfsetdash{}{0pt}%
\pgfpathmoveto{\pgfqpoint{4.302970in}{1.682701in}}%
\pgfpathlineto{\pgfqpoint{4.317349in}{1.690834in}}%
\pgfpathlineto{\pgfqpoint{4.331743in}{1.699146in}}%
\pgfpathlineto{\pgfqpoint{4.346152in}{1.707638in}}%
\pgfpathlineto{\pgfqpoint{4.360575in}{1.716310in}}%
\pgfpathlineto{\pgfqpoint{4.368765in}{1.733834in}}%
\pgfpathlineto{\pgfqpoint{4.376952in}{1.751347in}}%
\pgfpathlineto{\pgfqpoint{4.385135in}{1.768843in}}%
\pgfpathlineto{\pgfqpoint{4.393314in}{1.786317in}}%
\pgfpathlineto{\pgfqpoint{4.378884in}{1.777120in}}%
\pgfpathlineto{\pgfqpoint{4.364470in}{1.768104in}}%
\pgfpathlineto{\pgfqpoint{4.350070in}{1.759268in}}%
\pgfpathlineto{\pgfqpoint{4.335685in}{1.750611in}}%
\pgfpathlineto{\pgfqpoint{4.327512in}{1.733650in}}%
\pgfpathlineto{\pgfqpoint{4.319335in}{1.716674in}}%
\pgfpathlineto{\pgfqpoint{4.311154in}{1.699689in}}%
\pgfpathlineto{\pgfqpoint{4.302970in}{1.682701in}}%
\pgfpathclose%
\pgfusepath{fill}%
\end{pgfscope}%
\begin{pgfscope}%
\pgfpathrectangle{\pgfqpoint{1.150000in}{0.150000in}}{\pgfqpoint{5.700000in}{5.700000in}}%
\pgfusepath{clip}%
\pgfsetbuttcap%
\pgfsetroundjoin%
\definecolor{currentfill}{rgb}{0.487026,0.823929,0.312321}%
\pgfsetfillcolor{currentfill}%
\pgfsetfillopacity{0.800000}%
\pgfsetlinewidth{0.000000pt}%
\definecolor{currentstroke}{rgb}{0.000000,0.000000,0.000000}%
\pgfsetstrokecolor{currentstroke}%
\pgfsetdash{}{0pt}%
\pgfpathmoveto{\pgfqpoint{5.565497in}{3.422387in}}%
\pgfpathlineto{\pgfqpoint{5.580773in}{3.442009in}}%
\pgfpathlineto{\pgfqpoint{5.596075in}{3.461826in}}%
\pgfpathlineto{\pgfqpoint{5.611402in}{3.481837in}}%
\pgfpathlineto{\pgfqpoint{5.626755in}{3.502043in}}%
\pgfpathlineto{\pgfqpoint{5.634350in}{3.507714in}}%
\pgfpathlineto{\pgfqpoint{5.641932in}{3.513171in}}%
\pgfpathlineto{\pgfqpoint{5.649502in}{3.518416in}}%
\pgfpathlineto{\pgfqpoint{5.657059in}{3.523452in}}%
\pgfpathlineto{\pgfqpoint{5.641717in}{3.503454in}}%
\pgfpathlineto{\pgfqpoint{5.626402in}{3.483651in}}%
\pgfpathlineto{\pgfqpoint{5.611111in}{3.464041in}}%
\pgfpathlineto{\pgfqpoint{5.595846in}{3.444626in}}%
\pgfpathlineto{\pgfqpoint{5.588277in}{3.439369in}}%
\pgfpathlineto{\pgfqpoint{5.580695in}{3.433911in}}%
\pgfpathlineto{\pgfqpoint{5.573102in}{3.428252in}}%
\pgfpathlineto{\pgfqpoint{5.565497in}{3.422387in}}%
\pgfpathclose%
\pgfusepath{fill}%
\end{pgfscope}%
\begin{pgfscope}%
\pgfpathrectangle{\pgfqpoint{1.150000in}{0.150000in}}{\pgfqpoint{5.700000in}{5.700000in}}%
\pgfusepath{clip}%
\pgfsetbuttcap%
\pgfsetroundjoin%
\definecolor{currentfill}{rgb}{0.280868,0.160771,0.472899}%
\pgfsetfillcolor{currentfill}%
\pgfsetfillopacity{0.800000}%
\pgfsetlinewidth{0.000000pt}%
\definecolor{currentstroke}{rgb}{0.000000,0.000000,0.000000}%
\pgfsetstrokecolor{currentstroke}%
\pgfsetdash{}{0pt}%
\pgfpathmoveto{\pgfqpoint{4.057058in}{1.375164in}}%
\pgfpathlineto{\pgfqpoint{4.071326in}{1.379476in}}%
\pgfpathlineto{\pgfqpoint{4.085606in}{1.383965in}}%
\pgfpathlineto{\pgfqpoint{4.099897in}{1.388630in}}%
\pgfpathlineto{\pgfqpoint{4.114200in}{1.393471in}}%
\pgfpathlineto{\pgfqpoint{4.122441in}{1.409140in}}%
\pgfpathlineto{\pgfqpoint{4.130678in}{1.424900in}}%
\pgfpathlineto{\pgfqpoint{4.138911in}{1.440747in}}%
\pgfpathlineto{\pgfqpoint{4.147140in}{1.456674in}}%
\pgfpathlineto{\pgfqpoint{4.132837in}{1.451186in}}%
\pgfpathlineto{\pgfqpoint{4.118546in}{1.445876in}}%
\pgfpathlineto{\pgfqpoint{4.104267in}{1.440743in}}%
\pgfpathlineto{\pgfqpoint{4.090000in}{1.435787in}}%
\pgfpathlineto{\pgfqpoint{4.081771in}{1.420493in}}%
\pgfpathlineto{\pgfqpoint{4.073538in}{1.405288in}}%
\pgfpathlineto{\pgfqpoint{4.065300in}{1.390176in}}%
\pgfpathlineto{\pgfqpoint{4.057058in}{1.375164in}}%
\pgfpathclose%
\pgfusepath{fill}%
\end{pgfscope}%
\begin{pgfscope}%
\pgfpathrectangle{\pgfqpoint{1.150000in}{0.150000in}}{\pgfqpoint{5.700000in}{5.700000in}}%
\pgfusepath{clip}%
\pgfsetbuttcap%
\pgfsetroundjoin%
\definecolor{currentfill}{rgb}{0.153364,0.497000,0.557724}%
\pgfsetfillcolor{currentfill}%
\pgfsetfillopacity{0.800000}%
\pgfsetlinewidth{0.000000pt}%
\definecolor{currentstroke}{rgb}{0.000000,0.000000,0.000000}%
\pgfsetstrokecolor{currentstroke}%
\pgfsetdash{}{0pt}%
\pgfpathmoveto{\pgfqpoint{4.704823in}{2.288063in}}%
\pgfpathlineto{\pgfqpoint{4.719454in}{2.301385in}}%
\pgfpathlineto{\pgfqpoint{4.734103in}{2.314892in}}%
\pgfpathlineto{\pgfqpoint{4.748772in}{2.328586in}}%
\pgfpathlineto{\pgfqpoint{4.763459in}{2.342465in}}%
\pgfpathlineto{\pgfqpoint{4.771556in}{2.359006in}}%
\pgfpathlineto{\pgfqpoint{4.779647in}{2.375405in}}%
\pgfpathlineto{\pgfqpoint{4.787733in}{2.391657in}}%
\pgfpathlineto{\pgfqpoint{4.795812in}{2.407762in}}%
\pgfpathlineto{\pgfqpoint{4.781116in}{2.393583in}}%
\pgfpathlineto{\pgfqpoint{4.766440in}{2.379591in}}%
\pgfpathlineto{\pgfqpoint{4.751782in}{2.365784in}}%
\pgfpathlineto{\pgfqpoint{4.737143in}{2.352164in}}%
\pgfpathlineto{\pgfqpoint{4.729071in}{2.336345in}}%
\pgfpathlineto{\pgfqpoint{4.720994in}{2.320387in}}%
\pgfpathlineto{\pgfqpoint{4.712911in}{2.304292in}}%
\pgfpathlineto{\pgfqpoint{4.704823in}{2.288063in}}%
\pgfpathclose%
\pgfusepath{fill}%
\end{pgfscope}%
\begin{pgfscope}%
\pgfpathrectangle{\pgfqpoint{1.150000in}{0.150000in}}{\pgfqpoint{5.700000in}{5.700000in}}%
\pgfusepath{clip}%
\pgfsetbuttcap%
\pgfsetroundjoin%
\definecolor{currentfill}{rgb}{0.214298,0.355619,0.551184}%
\pgfsetfillcolor{currentfill}%
\pgfsetfillopacity{0.800000}%
\pgfsetlinewidth{0.000000pt}%
\definecolor{currentstroke}{rgb}{0.000000,0.000000,0.000000}%
\pgfsetstrokecolor{currentstroke}%
\pgfsetdash{}{0pt}%
\pgfpathmoveto{\pgfqpoint{4.425995in}{1.855902in}}%
\pgfpathlineto{\pgfqpoint{4.440447in}{1.865773in}}%
\pgfpathlineto{\pgfqpoint{4.454915in}{1.875825in}}%
\pgfpathlineto{\pgfqpoint{4.469400in}{1.886057in}}%
\pgfpathlineto{\pgfqpoint{4.483900in}{1.896471in}}%
\pgfpathlineto{\pgfqpoint{4.492069in}{1.914242in}}%
\pgfpathlineto{\pgfqpoint{4.500234in}{1.931956in}}%
\pgfpathlineto{\pgfqpoint{4.508395in}{1.949608in}}%
\pgfpathlineto{\pgfqpoint{4.516552in}{1.967195in}}%
\pgfpathlineto{\pgfqpoint{4.502043in}{1.956318in}}%
\pgfpathlineto{\pgfqpoint{4.487551in}{1.945623in}}%
\pgfpathlineto{\pgfqpoint{4.473076in}{1.935109in}}%
\pgfpathlineto{\pgfqpoint{4.458616in}{1.924777in}}%
\pgfpathlineto{\pgfqpoint{4.450467in}{1.907640in}}%
\pgfpathlineto{\pgfqpoint{4.442313in}{1.890446in}}%
\pgfpathlineto{\pgfqpoint{4.434156in}{1.873199in}}%
\pgfpathlineto{\pgfqpoint{4.425995in}{1.855902in}}%
\pgfpathclose%
\pgfusepath{fill}%
\end{pgfscope}%
\begin{pgfscope}%
\pgfpathrectangle{\pgfqpoint{1.150000in}{0.150000in}}{\pgfqpoint{5.700000in}{5.700000in}}%
\pgfusepath{clip}%
\pgfsetbuttcap%
\pgfsetroundjoin%
\definecolor{currentfill}{rgb}{0.277018,0.050344,0.375715}%
\pgfsetfillcolor{currentfill}%
\pgfsetfillopacity{0.800000}%
\pgfsetlinewidth{0.000000pt}%
\definecolor{currentstroke}{rgb}{0.000000,0.000000,0.000000}%
\pgfsetstrokecolor{currentstroke}%
\pgfsetdash{}{0pt}%
\pgfpathmoveto{\pgfqpoint{3.753774in}{1.150515in}}%
\pgfpathlineto{\pgfqpoint{3.767959in}{1.149834in}}%
\pgfpathlineto{\pgfqpoint{3.782151in}{1.149329in}}%
\pgfpathlineto{\pgfqpoint{3.796352in}{1.149001in}}%
\pgfpathlineto{\pgfqpoint{3.810560in}{1.148849in}}%
\pgfpathlineto{\pgfqpoint{3.818904in}{1.159796in}}%
\pgfpathlineto{\pgfqpoint{3.827240in}{1.170970in}}%
\pgfpathlineto{\pgfqpoint{3.835570in}{1.182361in}}%
\pgfpathlineto{\pgfqpoint{3.843893in}{1.193963in}}%
\pgfpathlineto{\pgfqpoint{3.829697in}{1.193351in}}%
\pgfpathlineto{\pgfqpoint{3.815509in}{1.192915in}}%
\pgfpathlineto{\pgfqpoint{3.801329in}{1.192656in}}%
\pgfpathlineto{\pgfqpoint{3.787158in}{1.192574in}}%
\pgfpathlineto{\pgfqpoint{3.778823in}{1.181724in}}%
\pgfpathlineto{\pgfqpoint{3.770480in}{1.171092in}}%
\pgfpathlineto{\pgfqpoint{3.762131in}{1.160687in}}%
\pgfpathlineto{\pgfqpoint{3.753774in}{1.150515in}}%
\pgfpathclose%
\pgfusepath{fill}%
\end{pgfscope}%
\begin{pgfscope}%
\pgfpathrectangle{\pgfqpoint{1.150000in}{0.150000in}}{\pgfqpoint{5.700000in}{5.700000in}}%
\pgfusepath{clip}%
\pgfsetbuttcap%
\pgfsetroundjoin%
\definecolor{currentfill}{rgb}{0.280267,0.073417,0.397163}%
\pgfsetfillcolor{currentfill}%
\pgfsetfillopacity{0.800000}%
\pgfsetlinewidth{0.000000pt}%
\definecolor{currentstroke}{rgb}{0.000000,0.000000,0.000000}%
\pgfsetstrokecolor{currentstroke}%
\pgfsetdash{}{0pt}%
\pgfpathmoveto{\pgfqpoint{3.843893in}{1.193963in}}%
\pgfpathlineto{\pgfqpoint{3.858098in}{1.194752in}}%
\pgfpathlineto{\pgfqpoint{3.872312in}{1.195717in}}%
\pgfpathlineto{\pgfqpoint{3.886535in}{1.196858in}}%
\pgfpathlineto{\pgfqpoint{3.900767in}{1.198174in}}%
\pgfpathlineto{\pgfqpoint{3.909074in}{1.210726in}}%
\pgfpathlineto{\pgfqpoint{3.917376in}{1.223465in}}%
\pgfpathlineto{\pgfqpoint{3.925672in}{1.236386in}}%
\pgfpathlineto{\pgfqpoint{3.933962in}{1.249481in}}%
\pgfpathlineto{\pgfqpoint{3.919738in}{1.247430in}}%
\pgfpathlineto{\pgfqpoint{3.905524in}{1.245555in}}%
\pgfpathlineto{\pgfqpoint{3.891319in}{1.243856in}}%
\pgfpathlineto{\pgfqpoint{3.877124in}{1.242334in}}%
\pgfpathlineto{\pgfqpoint{3.868825in}{1.229961in}}%
\pgfpathlineto{\pgfqpoint{3.860521in}{1.217771in}}%
\pgfpathlineto{\pgfqpoint{3.852210in}{1.205769in}}%
\pgfpathlineto{\pgfqpoint{3.843893in}{1.193963in}}%
\pgfpathclose%
\pgfusepath{fill}%
\end{pgfscope}%
\begin{pgfscope}%
\pgfpathrectangle{\pgfqpoint{1.150000in}{0.150000in}}{\pgfqpoint{5.700000in}{5.700000in}}%
\pgfusepath{clip}%
\pgfsetbuttcap%
\pgfsetroundjoin%
\definecolor{currentfill}{rgb}{0.131172,0.555899,0.552459}%
\pgfsetfillcolor{currentfill}%
\pgfsetfillopacity{0.800000}%
\pgfsetlinewidth{0.000000pt}%
\definecolor{currentstroke}{rgb}{0.000000,0.000000,0.000000}%
\pgfsetstrokecolor{currentstroke}%
\pgfsetdash{}{0pt}%
\pgfpathmoveto{\pgfqpoint{4.828073in}{2.470650in}}%
\pgfpathlineto{\pgfqpoint{4.842797in}{2.485281in}}%
\pgfpathlineto{\pgfqpoint{4.857540in}{2.500098in}}%
\pgfpathlineto{\pgfqpoint{4.872304in}{2.515104in}}%
\pgfpathlineto{\pgfqpoint{4.887088in}{2.530297in}}%
\pgfpathlineto{\pgfqpoint{4.895146in}{2.545870in}}%
\pgfpathlineto{\pgfqpoint{4.903197in}{2.561272in}}%
\pgfpathlineto{\pgfqpoint{4.911242in}{2.576503in}}%
\pgfpathlineto{\pgfqpoint{4.919280in}{2.591559in}}%
\pgfpathlineto{\pgfqpoint{4.904488in}{2.576135in}}%
\pgfpathlineto{\pgfqpoint{4.889717in}{2.560898in}}%
\pgfpathlineto{\pgfqpoint{4.874966in}{2.545849in}}%
\pgfpathlineto{\pgfqpoint{4.860235in}{2.530988in}}%
\pgfpathlineto{\pgfqpoint{4.852204in}{2.516150in}}%
\pgfpathlineto{\pgfqpoint{4.844167in}{2.501146in}}%
\pgfpathlineto{\pgfqpoint{4.836123in}{2.485978in}}%
\pgfpathlineto{\pgfqpoint{4.828073in}{2.470650in}}%
\pgfpathclose%
\pgfusepath{fill}%
\end{pgfscope}%
\begin{pgfscope}%
\pgfpathrectangle{\pgfqpoint{1.150000in}{0.150000in}}{\pgfqpoint{5.700000in}{5.700000in}}%
\pgfusepath{clip}%
\pgfsetbuttcap%
\pgfsetroundjoin%
\definecolor{currentfill}{rgb}{0.395174,0.797475,0.367757}%
\pgfsetfillcolor{currentfill}%
\pgfsetfillopacity{0.800000}%
\pgfsetlinewidth{0.000000pt}%
\definecolor{currentstroke}{rgb}{0.000000,0.000000,0.000000}%
\pgfsetstrokecolor{currentstroke}%
\pgfsetdash{}{0pt}%
\pgfpathmoveto{\pgfqpoint{5.443306in}{3.290058in}}%
\pgfpathlineto{\pgfqpoint{5.458498in}{3.309206in}}%
\pgfpathlineto{\pgfqpoint{5.473715in}{3.328549in}}%
\pgfpathlineto{\pgfqpoint{5.488957in}{3.348085in}}%
\pgfpathlineto{\pgfqpoint{5.504224in}{3.367816in}}%
\pgfpathlineto{\pgfqpoint{5.511925in}{3.375400in}}%
\pgfpathlineto{\pgfqpoint{5.519614in}{3.382763in}}%
\pgfpathlineto{\pgfqpoint{5.527291in}{3.389905in}}%
\pgfpathlineto{\pgfqpoint{5.534956in}{3.396830in}}%
\pgfpathlineto{\pgfqpoint{5.519696in}{3.377232in}}%
\pgfpathlineto{\pgfqpoint{5.504461in}{3.357828in}}%
\pgfpathlineto{\pgfqpoint{5.489250in}{3.338617in}}%
\pgfpathlineto{\pgfqpoint{5.474064in}{3.319601in}}%
\pgfpathlineto{\pgfqpoint{5.466392in}{3.312530in}}%
\pgfpathlineto{\pgfqpoint{5.458708in}{3.305251in}}%
\pgfpathlineto{\pgfqpoint{5.451012in}{3.297760in}}%
\pgfpathlineto{\pgfqpoint{5.443306in}{3.290058in}}%
\pgfpathclose%
\pgfusepath{fill}%
\end{pgfscope}%
\begin{pgfscope}%
\pgfpathrectangle{\pgfqpoint{1.150000in}{0.150000in}}{\pgfqpoint{5.700000in}{5.700000in}}%
\pgfusepath{clip}%
\pgfsetbuttcap%
\pgfsetroundjoin%
\definecolor{currentfill}{rgb}{0.273809,0.031497,0.358853}%
\pgfsetfillcolor{currentfill}%
\pgfsetfillopacity{0.800000}%
\pgfsetlinewidth{0.000000pt}%
\definecolor{currentstroke}{rgb}{0.000000,0.000000,0.000000}%
\pgfsetstrokecolor{currentstroke}%
\pgfsetdash{}{0pt}%
\pgfpathmoveto{\pgfqpoint{3.663533in}{1.119993in}}%
\pgfpathlineto{\pgfqpoint{3.677707in}{1.117808in}}%
\pgfpathlineto{\pgfqpoint{3.691887in}{1.115801in}}%
\pgfpathlineto{\pgfqpoint{3.706074in}{1.113971in}}%
\pgfpathlineto{\pgfqpoint{3.720268in}{1.112319in}}%
\pgfpathlineto{\pgfqpoint{3.728656in}{1.121478in}}%
\pgfpathlineto{\pgfqpoint{3.737037in}{1.130903in}}%
\pgfpathlineto{\pgfqpoint{3.745409in}{1.140584in}}%
\pgfpathlineto{\pgfqpoint{3.753774in}{1.150515in}}%
\pgfpathlineto{\pgfqpoint{3.739596in}{1.151373in}}%
\pgfpathlineto{\pgfqpoint{3.725426in}{1.152408in}}%
\pgfpathlineto{\pgfqpoint{3.711264in}{1.153621in}}%
\pgfpathlineto{\pgfqpoint{3.697108in}{1.155011in}}%
\pgfpathlineto{\pgfqpoint{3.688727in}{1.145863in}}%
\pgfpathlineto{\pgfqpoint{3.680338in}{1.136972in}}%
\pgfpathlineto{\pgfqpoint{3.671940in}{1.128346in}}%
\pgfpathlineto{\pgfqpoint{3.663533in}{1.119993in}}%
\pgfpathclose%
\pgfusepath{fill}%
\end{pgfscope}%
\begin{pgfscope}%
\pgfpathrectangle{\pgfqpoint{1.150000in}{0.150000in}}{\pgfqpoint{5.700000in}{5.700000in}}%
\pgfusepath{clip}%
\pgfsetbuttcap%
\pgfsetroundjoin%
\definecolor{currentfill}{rgb}{0.575563,0.844566,0.256415}%
\pgfsetfillcolor{currentfill}%
\pgfsetfillopacity{0.800000}%
\pgfsetlinewidth{0.000000pt}%
\definecolor{currentstroke}{rgb}{0.000000,0.000000,0.000000}%
\pgfsetstrokecolor{currentstroke}%
\pgfsetdash{}{0pt}%
\pgfpathmoveto{\pgfqpoint{5.657059in}{3.523452in}}%
\pgfpathlineto{\pgfqpoint{5.672426in}{3.543644in}}%
\pgfpathlineto{\pgfqpoint{5.687819in}{3.564032in}}%
\pgfpathlineto{\pgfqpoint{5.703239in}{3.584616in}}%
\pgfpathlineto{\pgfqpoint{5.710774in}{3.589270in}}%
\pgfpathlineto{\pgfqpoint{5.718296in}{3.593714in}}%
\pgfpathlineto{\pgfqpoint{5.725805in}{3.597951in}}%
\pgfpathlineto{\pgfqpoint{5.733301in}{3.601983in}}%
\pgfpathlineto{\pgfqpoint{5.717895in}{3.581647in}}%
\pgfpathlineto{\pgfqpoint{5.702516in}{3.561506in}}%
\pgfpathlineto{\pgfqpoint{5.687162in}{3.541559in}}%
\pgfpathlineto{\pgfqpoint{5.679655in}{3.537332in}}%
\pgfpathlineto{\pgfqpoint{5.672136in}{3.532907in}}%
\pgfpathlineto{\pgfqpoint{5.664604in}{3.528281in}}%
\pgfpathlineto{\pgfqpoint{5.657059in}{3.523452in}}%
\pgfpathclose%
\pgfusepath{fill}%
\end{pgfscope}%
\begin{pgfscope}%
\pgfpathrectangle{\pgfqpoint{1.150000in}{0.150000in}}{\pgfqpoint{5.700000in}{5.700000in}}%
\pgfusepath{clip}%
\pgfsetbuttcap%
\pgfsetroundjoin%
\definecolor{currentfill}{rgb}{0.282910,0.105393,0.426902}%
\pgfsetfillcolor{currentfill}%
\pgfsetfillopacity{0.800000}%
\pgfsetlinewidth{0.000000pt}%
\definecolor{currentstroke}{rgb}{0.000000,0.000000,0.000000}%
\pgfsetstrokecolor{currentstroke}%
\pgfsetdash{}{0pt}%
\pgfpathmoveto{\pgfqpoint{3.933962in}{1.249481in}}%
\pgfpathlineto{\pgfqpoint{3.948196in}{1.251708in}}%
\pgfpathlineto{\pgfqpoint{3.962440in}{1.254112in}}%
\pgfpathlineto{\pgfqpoint{3.976694in}{1.256691in}}%
\pgfpathlineto{\pgfqpoint{3.990958in}{1.259446in}}%
\pgfpathlineto{\pgfqpoint{3.999237in}{1.273424in}}%
\pgfpathlineto{\pgfqpoint{4.007511in}{1.287555in}}%
\pgfpathlineto{\pgfqpoint{4.015780in}{1.301831in}}%
\pgfpathlineto{\pgfqpoint{4.024045in}{1.316247in}}%
\pgfpathlineto{\pgfqpoint{4.009785in}{1.312786in}}%
\pgfpathlineto{\pgfqpoint{3.995536in}{1.309502in}}%
\pgfpathlineto{\pgfqpoint{3.981298in}{1.306394in}}%
\pgfpathlineto{\pgfqpoint{3.967070in}{1.303463in}}%
\pgfpathlineto{\pgfqpoint{3.958801in}{1.289740in}}%
\pgfpathlineto{\pgfqpoint{3.950527in}{1.276165in}}%
\pgfpathlineto{\pgfqpoint{3.942247in}{1.262743in}}%
\pgfpathlineto{\pgfqpoint{3.933962in}{1.249481in}}%
\pgfpathclose%
\pgfusepath{fill}%
\end{pgfscope}%
\begin{pgfscope}%
\pgfpathrectangle{\pgfqpoint{1.150000in}{0.150000in}}{\pgfqpoint{5.700000in}{5.700000in}}%
\pgfusepath{clip}%
\pgfsetbuttcap%
\pgfsetroundjoin%
\definecolor{currentfill}{rgb}{0.185556,0.418570,0.556753}%
\pgfsetfillcolor{currentfill}%
\pgfsetfillopacity{0.800000}%
\pgfsetlinewidth{0.000000pt}%
\definecolor{currentstroke}{rgb}{0.000000,0.000000,0.000000}%
\pgfsetstrokecolor{currentstroke}%
\pgfsetdash{}{0pt}%
\pgfpathmoveto{\pgfqpoint{4.549141in}{2.036809in}}%
\pgfpathlineto{\pgfqpoint{4.563674in}{2.048299in}}%
\pgfpathlineto{\pgfqpoint{4.578225in}{2.059973in}}%
\pgfpathlineto{\pgfqpoint{4.592794in}{2.071829in}}%
\pgfpathlineto{\pgfqpoint{4.607379in}{2.083868in}}%
\pgfpathlineto{\pgfqpoint{4.615525in}{2.101480in}}%
\pgfpathlineto{\pgfqpoint{4.623666in}{2.118995in}}%
\pgfpathlineto{\pgfqpoint{4.631803in}{2.136410in}}%
\pgfpathlineto{\pgfqpoint{4.639936in}{2.153720in}}%
\pgfpathlineto{\pgfqpoint{4.625341in}{2.141281in}}%
\pgfpathlineto{\pgfqpoint{4.610763in}{2.129025in}}%
\pgfpathlineto{\pgfqpoint{4.596204in}{2.116954in}}%
\pgfpathlineto{\pgfqpoint{4.581662in}{2.105066in}}%
\pgfpathlineto{\pgfqpoint{4.573538in}{2.088142in}}%
\pgfpathlineto{\pgfqpoint{4.565410in}{2.071122in}}%
\pgfpathlineto{\pgfqpoint{4.557277in}{2.054010in}}%
\pgfpathlineto{\pgfqpoint{4.549141in}{2.036809in}}%
\pgfpathclose%
\pgfusepath{fill}%
\end{pgfscope}%
\begin{pgfscope}%
\pgfpathrectangle{\pgfqpoint{1.150000in}{0.150000in}}{\pgfqpoint{5.700000in}{5.700000in}}%
\pgfusepath{clip}%
\pgfsetbuttcap%
\pgfsetroundjoin%
\definecolor{currentfill}{rgb}{0.274128,0.199721,0.498911}%
\pgfsetfillcolor{currentfill}%
\pgfsetfillopacity{0.800000}%
\pgfsetlinewidth{0.000000pt}%
\definecolor{currentstroke}{rgb}{0.000000,0.000000,0.000000}%
\pgfsetstrokecolor{currentstroke}%
\pgfsetdash{}{0pt}%
\pgfpathmoveto{\pgfqpoint{4.147140in}{1.456674in}}%
\pgfpathlineto{\pgfqpoint{4.161456in}{1.462339in}}%
\pgfpathlineto{\pgfqpoint{4.175784in}{1.468181in}}%
\pgfpathlineto{\pgfqpoint{4.190126in}{1.474200in}}%
\pgfpathlineto{\pgfqpoint{4.204480in}{1.480396in}}%
\pgfpathlineto{\pgfqpoint{4.212707in}{1.497025in}}%
\pgfpathlineto{\pgfqpoint{4.220930in}{1.513715in}}%
\pgfpathlineto{\pgfqpoint{4.229150in}{1.530460in}}%
\pgfpathlineto{\pgfqpoint{4.237367in}{1.547253in}}%
\pgfpathlineto{\pgfqpoint{4.223010in}{1.540440in}}%
\pgfpathlineto{\pgfqpoint{4.208666in}{1.533804in}}%
\pgfpathlineto{\pgfqpoint{4.194336in}{1.527346in}}%
\pgfpathlineto{\pgfqpoint{4.180019in}{1.521066in}}%
\pgfpathlineto{\pgfqpoint{4.171805in}{1.504877in}}%
\pgfpathlineto{\pgfqpoint{4.163587in}{1.488745in}}%
\pgfpathlineto{\pgfqpoint{4.155366in}{1.472675in}}%
\pgfpathlineto{\pgfqpoint{4.147140in}{1.456674in}}%
\pgfpathclose%
\pgfusepath{fill}%
\end{pgfscope}%
\begin{pgfscope}%
\pgfpathrectangle{\pgfqpoint{1.150000in}{0.150000in}}{\pgfqpoint{5.700000in}{5.700000in}}%
\pgfusepath{clip}%
\pgfsetbuttcap%
\pgfsetroundjoin%
\definecolor{currentfill}{rgb}{0.253935,0.265254,0.529983}%
\pgfsetfillcolor{currentfill}%
\pgfsetfillopacity{0.800000}%
\pgfsetlinewidth{0.000000pt}%
\definecolor{currentstroke}{rgb}{0.000000,0.000000,0.000000}%
\pgfsetstrokecolor{currentstroke}%
\pgfsetdash{}{0pt}%
\pgfpathmoveto{\pgfqpoint{4.270197in}{1.614805in}}%
\pgfpathlineto{\pgfqpoint{4.284571in}{1.622383in}}%
\pgfpathlineto{\pgfqpoint{4.298960in}{1.630140in}}%
\pgfpathlineto{\pgfqpoint{4.313363in}{1.638075in}}%
\pgfpathlineto{\pgfqpoint{4.327780in}{1.646189in}}%
\pgfpathlineto{\pgfqpoint{4.335984in}{1.663712in}}%
\pgfpathlineto{\pgfqpoint{4.344185in}{1.681243in}}%
\pgfpathlineto{\pgfqpoint{4.352382in}{1.698778in}}%
\pgfpathlineto{\pgfqpoint{4.360575in}{1.716310in}}%
\pgfpathlineto{\pgfqpoint{4.346152in}{1.707638in}}%
\pgfpathlineto{\pgfqpoint{4.331743in}{1.699146in}}%
\pgfpathlineto{\pgfqpoint{4.317349in}{1.690834in}}%
\pgfpathlineto{\pgfqpoint{4.302970in}{1.682701in}}%
\pgfpathlineto{\pgfqpoint{4.294782in}{1.665713in}}%
\pgfpathlineto{\pgfqpoint{4.286591in}{1.648731in}}%
\pgfpathlineto{\pgfqpoint{4.278396in}{1.631760in}}%
\pgfpathlineto{\pgfqpoint{4.270197in}{1.614805in}}%
\pgfpathclose%
\pgfusepath{fill}%
\end{pgfscope}%
\begin{pgfscope}%
\pgfpathrectangle{\pgfqpoint{1.150000in}{0.150000in}}{\pgfqpoint{5.700000in}{5.700000in}}%
\pgfusepath{clip}%
\pgfsetbuttcap%
\pgfsetroundjoin%
\definecolor{currentfill}{rgb}{0.119483,0.614817,0.537692}%
\pgfsetfillcolor{currentfill}%
\pgfsetfillopacity{0.800000}%
\pgfsetlinewidth{0.000000pt}%
\definecolor{currentstroke}{rgb}{0.000000,0.000000,0.000000}%
\pgfsetstrokecolor{currentstroke}%
\pgfsetdash{}{0pt}%
\pgfpathmoveto{\pgfqpoint{4.951362in}{2.650012in}}%
\pgfpathlineto{\pgfqpoint{4.966181in}{2.665822in}}%
\pgfpathlineto{\pgfqpoint{4.981021in}{2.681820in}}%
\pgfpathlineto{\pgfqpoint{4.995883in}{2.698007in}}%
\pgfpathlineto{\pgfqpoint{5.010765in}{2.714384in}}%
\pgfpathlineto{\pgfqpoint{5.018774in}{2.728722in}}%
\pgfpathlineto{\pgfqpoint{5.026776in}{2.742867in}}%
\pgfpathlineto{\pgfqpoint{5.034770in}{2.756819in}}%
\pgfpathlineto{\pgfqpoint{5.042756in}{2.770577in}}%
\pgfpathlineto{\pgfqpoint{5.027867in}{2.754038in}}%
\pgfpathlineto{\pgfqpoint{5.013000in}{2.737689in}}%
\pgfpathlineto{\pgfqpoint{4.998155in}{2.721530in}}%
\pgfpathlineto{\pgfqpoint{4.983330in}{2.705559in}}%
\pgfpathlineto{\pgfqpoint{4.975349in}{2.691949in}}%
\pgfpathlineto{\pgfqpoint{4.967361in}{2.678154in}}%
\pgfpathlineto{\pgfqpoint{4.959365in}{2.664175in}}%
\pgfpathlineto{\pgfqpoint{4.951362in}{2.650012in}}%
\pgfpathclose%
\pgfusepath{fill}%
\end{pgfscope}%
\begin{pgfscope}%
\pgfpathrectangle{\pgfqpoint{1.150000in}{0.150000in}}{\pgfqpoint{5.700000in}{5.700000in}}%
\pgfusepath{clip}%
\pgfsetbuttcap%
\pgfsetroundjoin%
\definecolor{currentfill}{rgb}{0.304148,0.764704,0.419943}%
\pgfsetfillcolor{currentfill}%
\pgfsetfillopacity{0.800000}%
\pgfsetlinewidth{0.000000pt}%
\definecolor{currentstroke}{rgb}{0.000000,0.000000,0.000000}%
\pgfsetstrokecolor{currentstroke}%
\pgfsetdash{}{0pt}%
\pgfpathmoveto{\pgfqpoint{5.320692in}{3.145340in}}%
\pgfpathlineto{\pgfqpoint{5.335796in}{3.163867in}}%
\pgfpathlineto{\pgfqpoint{5.350923in}{3.182588in}}%
\pgfpathlineto{\pgfqpoint{5.366074in}{3.201501in}}%
\pgfpathlineto{\pgfqpoint{5.381250in}{3.220609in}}%
\pgfpathlineto{\pgfqpoint{5.389045in}{3.230063in}}%
\pgfpathlineto{\pgfqpoint{5.396830in}{3.239294in}}%
\pgfpathlineto{\pgfqpoint{5.404604in}{3.248303in}}%
\pgfpathlineto{\pgfqpoint{5.412366in}{3.257091in}}%
\pgfpathlineto{\pgfqpoint{5.397194in}{3.238041in}}%
\pgfpathlineto{\pgfqpoint{5.382045in}{3.219185in}}%
\pgfpathlineto{\pgfqpoint{5.366920in}{3.200522in}}%
\pgfpathlineto{\pgfqpoint{5.351820in}{3.182051in}}%
\pgfpathlineto{\pgfqpoint{5.344054in}{3.173192in}}%
\pgfpathlineto{\pgfqpoint{5.336277in}{3.164121in}}%
\pgfpathlineto{\pgfqpoint{5.328490in}{3.154837in}}%
\pgfpathlineto{\pgfqpoint{5.320692in}{3.145340in}}%
\pgfpathclose%
\pgfusepath{fill}%
\end{pgfscope}%
\begin{pgfscope}%
\pgfpathrectangle{\pgfqpoint{1.150000in}{0.150000in}}{\pgfqpoint{5.700000in}{5.700000in}}%
\pgfusepath{clip}%
\pgfsetbuttcap%
\pgfsetroundjoin%
\definecolor{currentfill}{rgb}{0.223925,0.334994,0.548053}%
\pgfsetfillcolor{currentfill}%
\pgfsetfillopacity{0.800000}%
\pgfsetlinewidth{0.000000pt}%
\definecolor{currentstroke}{rgb}{0.000000,0.000000,0.000000}%
\pgfsetstrokecolor{currentstroke}%
\pgfsetdash{}{0pt}%
\pgfpathmoveto{\pgfqpoint{4.393314in}{1.786317in}}%
\pgfpathlineto{\pgfqpoint{4.407759in}{1.795694in}}%
\pgfpathlineto{\pgfqpoint{4.422220in}{1.805251in}}%
\pgfpathlineto{\pgfqpoint{4.436696in}{1.814988in}}%
\pgfpathlineto{\pgfqpoint{4.451188in}{1.824906in}}%
\pgfpathlineto{\pgfqpoint{4.459371in}{1.842861in}}%
\pgfpathlineto{\pgfqpoint{4.467551in}{1.860776in}}%
\pgfpathlineto{\pgfqpoint{4.475727in}{1.878648in}}%
\pgfpathlineto{\pgfqpoint{4.483900in}{1.896471in}}%
\pgfpathlineto{\pgfqpoint{4.469400in}{1.886057in}}%
\pgfpathlineto{\pgfqpoint{4.454915in}{1.875825in}}%
\pgfpathlineto{\pgfqpoint{4.440447in}{1.865773in}}%
\pgfpathlineto{\pgfqpoint{4.425995in}{1.855902in}}%
\pgfpathlineto{\pgfqpoint{4.417830in}{1.838562in}}%
\pgfpathlineto{\pgfqpoint{4.409662in}{1.821181in}}%
\pgfpathlineto{\pgfqpoint{4.401490in}{1.803765in}}%
\pgfpathlineto{\pgfqpoint{4.393314in}{1.786317in}}%
\pgfpathclose%
\pgfusepath{fill}%
\end{pgfscope}%
\begin{pgfscope}%
\pgfpathrectangle{\pgfqpoint{1.150000in}{0.150000in}}{\pgfqpoint{5.700000in}{5.700000in}}%
\pgfusepath{clip}%
\pgfsetbuttcap%
\pgfsetroundjoin%
\definecolor{currentfill}{rgb}{0.143303,0.669459,0.511215}%
\pgfsetfillcolor{currentfill}%
\pgfsetfillopacity{0.800000}%
\pgfsetlinewidth{0.000000pt}%
\definecolor{currentstroke}{rgb}{0.000000,0.000000,0.000000}%
\pgfsetstrokecolor{currentstroke}%
\pgfsetdash{}{0pt}%
\pgfpathmoveto{\pgfqpoint{5.074621in}{2.823646in}}%
\pgfpathlineto{\pgfqpoint{5.089536in}{2.840500in}}%
\pgfpathlineto{\pgfqpoint{5.104473in}{2.857545in}}%
\pgfpathlineto{\pgfqpoint{5.119433in}{2.874781in}}%
\pgfpathlineto{\pgfqpoint{5.134414in}{2.892207in}}%
\pgfpathlineto{\pgfqpoint{5.142364in}{2.905084in}}%
\pgfpathlineto{\pgfqpoint{5.150305in}{2.917752in}}%
\pgfpathlineto{\pgfqpoint{5.158237in}{2.930211in}}%
\pgfpathlineto{\pgfqpoint{5.166161in}{2.942462in}}%
\pgfpathlineto{\pgfqpoint{5.151175in}{2.924945in}}%
\pgfpathlineto{\pgfqpoint{5.136212in}{2.907620in}}%
\pgfpathlineto{\pgfqpoint{5.121271in}{2.890485in}}%
\pgfpathlineto{\pgfqpoint{5.106353in}{2.873541in}}%
\pgfpathlineto{\pgfqpoint{5.098432in}{2.861367in}}%
\pgfpathlineto{\pgfqpoint{5.090504in}{2.848993in}}%
\pgfpathlineto{\pgfqpoint{5.082566in}{2.836419in}}%
\pgfpathlineto{\pgfqpoint{5.074621in}{2.823646in}}%
\pgfpathclose%
\pgfusepath{fill}%
\end{pgfscope}%
\begin{pgfscope}%
\pgfpathrectangle{\pgfqpoint{1.150000in}{0.150000in}}{\pgfqpoint{5.700000in}{5.700000in}}%
\pgfusepath{clip}%
\pgfsetbuttcap%
\pgfsetroundjoin%
\definecolor{currentfill}{rgb}{0.282623,0.140926,0.457517}%
\pgfsetfillcolor{currentfill}%
\pgfsetfillopacity{0.800000}%
\pgfsetlinewidth{0.000000pt}%
\definecolor{currentstroke}{rgb}{0.000000,0.000000,0.000000}%
\pgfsetstrokecolor{currentstroke}%
\pgfsetdash{}{0pt}%
\pgfpathmoveto{\pgfqpoint{4.024045in}{1.316247in}}%
\pgfpathlineto{\pgfqpoint{4.038316in}{1.319883in}}%
\pgfpathlineto{\pgfqpoint{4.052597in}{1.323696in}}%
\pgfpathlineto{\pgfqpoint{4.066890in}{1.327685in}}%
\pgfpathlineto{\pgfqpoint{4.081194in}{1.331849in}}%
\pgfpathlineto{\pgfqpoint{4.089452in}{1.347084in}}%
\pgfpathlineto{\pgfqpoint{4.097705in}{1.362437in}}%
\pgfpathlineto{\pgfqpoint{4.105955in}{1.377902in}}%
\pgfpathlineto{\pgfqpoint{4.114200in}{1.393471in}}%
\pgfpathlineto{\pgfqpoint{4.099897in}{1.388630in}}%
\pgfpathlineto{\pgfqpoint{4.085606in}{1.383965in}}%
\pgfpathlineto{\pgfqpoint{4.071326in}{1.379476in}}%
\pgfpathlineto{\pgfqpoint{4.057058in}{1.375164in}}%
\pgfpathlineto{\pgfqpoint{4.048811in}{1.360259in}}%
\pgfpathlineto{\pgfqpoint{4.040560in}{1.345467in}}%
\pgfpathlineto{\pgfqpoint{4.032305in}{1.330794in}}%
\pgfpathlineto{\pgfqpoint{4.024045in}{1.316247in}}%
\pgfpathclose%
\pgfusepath{fill}%
\end{pgfscope}%
\begin{pgfscope}%
\pgfpathrectangle{\pgfqpoint{1.150000in}{0.150000in}}{\pgfqpoint{5.700000in}{5.700000in}}%
\pgfusepath{clip}%
\pgfsetbuttcap%
\pgfsetroundjoin%
\definecolor{currentfill}{rgb}{0.159194,0.482237,0.558073}%
\pgfsetfillcolor{currentfill}%
\pgfsetfillopacity{0.800000}%
\pgfsetlinewidth{0.000000pt}%
\definecolor{currentstroke}{rgb}{0.000000,0.000000,0.000000}%
\pgfsetstrokecolor{currentstroke}%
\pgfsetdash{}{0pt}%
\pgfpathmoveto{\pgfqpoint{4.672419in}{2.221853in}}%
\pgfpathlineto{\pgfqpoint{4.687041in}{2.234843in}}%
\pgfpathlineto{\pgfqpoint{4.701681in}{2.248017in}}%
\pgfpathlineto{\pgfqpoint{4.716341in}{2.261377in}}%
\pgfpathlineto{\pgfqpoint{4.731019in}{2.274922in}}%
\pgfpathlineto{\pgfqpoint{4.739137in}{2.292008in}}%
\pgfpathlineto{\pgfqpoint{4.747249in}{2.308963in}}%
\pgfpathlineto{\pgfqpoint{4.755357in}{2.325783in}}%
\pgfpathlineto{\pgfqpoint{4.763459in}{2.342465in}}%
\pgfpathlineto{\pgfqpoint{4.748772in}{2.328586in}}%
\pgfpathlineto{\pgfqpoint{4.734103in}{2.314892in}}%
\pgfpathlineto{\pgfqpoint{4.719454in}{2.301385in}}%
\pgfpathlineto{\pgfqpoint{4.704823in}{2.288063in}}%
\pgfpathlineto{\pgfqpoint{4.696729in}{2.271701in}}%
\pgfpathlineto{\pgfqpoint{4.688631in}{2.255210in}}%
\pgfpathlineto{\pgfqpoint{4.680527in}{2.238594in}}%
\pgfpathlineto{\pgfqpoint{4.672419in}{2.221853in}}%
\pgfpathclose%
\pgfusepath{fill}%
\end{pgfscope}%
\begin{pgfscope}%
\pgfpathrectangle{\pgfqpoint{1.150000in}{0.150000in}}{\pgfqpoint{5.700000in}{5.700000in}}%
\pgfusepath{clip}%
\pgfsetbuttcap%
\pgfsetroundjoin%
\definecolor{currentfill}{rgb}{0.214000,0.722114,0.469588}%
\pgfsetfillcolor{currentfill}%
\pgfsetfillopacity{0.800000}%
\pgfsetlinewidth{0.000000pt}%
\definecolor{currentstroke}{rgb}{0.000000,0.000000,0.000000}%
\pgfsetstrokecolor{currentstroke}%
\pgfsetdash{}{0pt}%
\pgfpathmoveto{\pgfqpoint{5.197763in}{2.989370in}}%
\pgfpathlineto{\pgfqpoint{5.212773in}{3.007132in}}%
\pgfpathlineto{\pgfqpoint{5.227807in}{3.025085in}}%
\pgfpathlineto{\pgfqpoint{5.242864in}{3.043231in}}%
\pgfpathlineto{\pgfqpoint{5.257944in}{3.061569in}}%
\pgfpathlineto{\pgfqpoint{5.265822in}{3.072803in}}%
\pgfpathlineto{\pgfqpoint{5.273691in}{3.083818in}}%
\pgfpathlineto{\pgfqpoint{5.281550in}{3.094614in}}%
\pgfpathlineto{\pgfqpoint{5.289399in}{3.105193in}}%
\pgfpathlineto{\pgfqpoint{5.274318in}{3.086838in}}%
\pgfpathlineto{\pgfqpoint{5.259260in}{3.068676in}}%
\pgfpathlineto{\pgfqpoint{5.244226in}{3.050706in}}%
\pgfpathlineto{\pgfqpoint{5.229215in}{3.032927in}}%
\pgfpathlineto{\pgfqpoint{5.221366in}{3.022351in}}%
\pgfpathlineto{\pgfqpoint{5.213508in}{3.011567in}}%
\pgfpathlineto{\pgfqpoint{5.205640in}{3.000573in}}%
\pgfpathlineto{\pgfqpoint{5.197763in}{2.989370in}}%
\pgfpathclose%
\pgfusepath{fill}%
\end{pgfscope}%
\begin{pgfscope}%
\pgfpathrectangle{\pgfqpoint{1.150000in}{0.150000in}}{\pgfqpoint{5.700000in}{5.700000in}}%
\pgfusepath{clip}%
\pgfsetbuttcap%
\pgfsetroundjoin%
\definecolor{currentfill}{rgb}{0.194100,0.399323,0.555565}%
\pgfsetfillcolor{currentfill}%
\pgfsetfillopacity{0.800000}%
\pgfsetlinewidth{0.000000pt}%
\definecolor{currentstroke}{rgb}{0.000000,0.000000,0.000000}%
\pgfsetstrokecolor{currentstroke}%
\pgfsetdash{}{0pt}%
\pgfpathmoveto{\pgfqpoint{4.516552in}{1.967195in}}%
\pgfpathlineto{\pgfqpoint{4.531077in}{1.978254in}}%
\pgfpathlineto{\pgfqpoint{4.545619in}{1.989496in}}%
\pgfpathlineto{\pgfqpoint{4.560178in}{2.000919in}}%
\pgfpathlineto{\pgfqpoint{4.574755in}{2.012526in}}%
\pgfpathlineto{\pgfqpoint{4.582917in}{2.030488in}}%
\pgfpathlineto{\pgfqpoint{4.591075in}{2.048368in}}%
\pgfpathlineto{\pgfqpoint{4.599229in}{2.066163in}}%
\pgfpathlineto{\pgfqpoint{4.607379in}{2.083868in}}%
\pgfpathlineto{\pgfqpoint{4.592794in}{2.071829in}}%
\pgfpathlineto{\pgfqpoint{4.578225in}{2.059973in}}%
\pgfpathlineto{\pgfqpoint{4.563674in}{2.048299in}}%
\pgfpathlineto{\pgfqpoint{4.549141in}{2.036809in}}%
\pgfpathlineto{\pgfqpoint{4.541000in}{2.019524in}}%
\pgfpathlineto{\pgfqpoint{4.532854in}{2.002157in}}%
\pgfpathlineto{\pgfqpoint{4.524705in}{1.984713in}}%
\pgfpathlineto{\pgfqpoint{4.516552in}{1.967195in}}%
\pgfpathclose%
\pgfusepath{fill}%
\end{pgfscope}%
\begin{pgfscope}%
\pgfpathrectangle{\pgfqpoint{1.150000in}{0.150000in}}{\pgfqpoint{5.700000in}{5.700000in}}%
\pgfusepath{clip}%
\pgfsetbuttcap%
\pgfsetroundjoin%
\definecolor{currentfill}{rgb}{0.135066,0.544853,0.554029}%
\pgfsetfillcolor{currentfill}%
\pgfsetfillopacity{0.800000}%
\pgfsetlinewidth{0.000000pt}%
\definecolor{currentstroke}{rgb}{0.000000,0.000000,0.000000}%
\pgfsetstrokecolor{currentstroke}%
\pgfsetdash{}{0pt}%
\pgfpathmoveto{\pgfqpoint{4.795812in}{2.407762in}}%
\pgfpathlineto{\pgfqpoint{4.810528in}{2.422127in}}%
\pgfpathlineto{\pgfqpoint{4.825263in}{2.436679in}}%
\pgfpathlineto{\pgfqpoint{4.840018in}{2.451418in}}%
\pgfpathlineto{\pgfqpoint{4.854793in}{2.466344in}}%
\pgfpathlineto{\pgfqpoint{4.862876in}{2.482577in}}%
\pgfpathlineto{\pgfqpoint{4.870953in}{2.498648in}}%
\pgfpathlineto{\pgfqpoint{4.879024in}{2.514556in}}%
\pgfpathlineto{\pgfqpoint{4.887088in}{2.530297in}}%
\pgfpathlineto{\pgfqpoint{4.872304in}{2.515104in}}%
\pgfpathlineto{\pgfqpoint{4.857540in}{2.500098in}}%
\pgfpathlineto{\pgfqpoint{4.842797in}{2.485281in}}%
\pgfpathlineto{\pgfqpoint{4.828073in}{2.470650in}}%
\pgfpathlineto{\pgfqpoint{4.820017in}{2.455162in}}%
\pgfpathlineto{\pgfqpoint{4.811955in}{2.439516in}}%
\pgfpathlineto{\pgfqpoint{4.803887in}{2.423715in}}%
\pgfpathlineto{\pgfqpoint{4.795812in}{2.407762in}}%
\pgfpathclose%
\pgfusepath{fill}%
\end{pgfscope}%
\begin{pgfscope}%
\pgfpathrectangle{\pgfqpoint{1.150000in}{0.150000in}}{\pgfqpoint{5.700000in}{5.700000in}}%
\pgfusepath{clip}%
\pgfsetbuttcap%
\pgfsetroundjoin%
\definecolor{currentfill}{rgb}{0.496615,0.826376,0.306377}%
\pgfsetfillcolor{currentfill}%
\pgfsetfillopacity{0.800000}%
\pgfsetlinewidth{0.000000pt}%
\definecolor{currentstroke}{rgb}{0.000000,0.000000,0.000000}%
\pgfsetstrokecolor{currentstroke}%
\pgfsetdash{}{0pt}%
\pgfpathmoveto{\pgfqpoint{5.534956in}{3.396830in}}%
\pgfpathlineto{\pgfqpoint{5.550241in}{3.416622in}}%
\pgfpathlineto{\pgfqpoint{5.565552in}{3.436610in}}%
\pgfpathlineto{\pgfqpoint{5.580888in}{3.456792in}}%
\pgfpathlineto{\pgfqpoint{5.596250in}{3.477171in}}%
\pgfpathlineto{\pgfqpoint{5.603895in}{3.483722in}}%
\pgfpathlineto{\pgfqpoint{5.611527in}{3.490049in}}%
\pgfpathlineto{\pgfqpoint{5.619147in}{3.496156in}}%
\pgfpathlineto{\pgfqpoint{5.626755in}{3.502043in}}%
\pgfpathlineto{\pgfqpoint{5.611402in}{3.481837in}}%
\pgfpathlineto{\pgfqpoint{5.596075in}{3.461826in}}%
\pgfpathlineto{\pgfqpoint{5.580773in}{3.442009in}}%
\pgfpathlineto{\pgfqpoint{5.565497in}{3.422387in}}%
\pgfpathlineto{\pgfqpoint{5.557880in}{3.416314in}}%
\pgfpathlineto{\pgfqpoint{5.550250in}{3.410032in}}%
\pgfpathlineto{\pgfqpoint{5.542609in}{3.403538in}}%
\pgfpathlineto{\pgfqpoint{5.534956in}{3.396830in}}%
\pgfpathclose%
\pgfusepath{fill}%
\end{pgfscope}%
\begin{pgfscope}%
\pgfpathrectangle{\pgfqpoint{1.150000in}{0.150000in}}{\pgfqpoint{5.700000in}{5.700000in}}%
\pgfusepath{clip}%
\pgfsetbuttcap%
\pgfsetroundjoin%
\definecolor{currentfill}{rgb}{0.278791,0.062145,0.386592}%
\pgfsetfillcolor{currentfill}%
\pgfsetfillopacity{0.800000}%
\pgfsetlinewidth{0.000000pt}%
\definecolor{currentstroke}{rgb}{0.000000,0.000000,0.000000}%
\pgfsetstrokecolor{currentstroke}%
\pgfsetdash{}{0pt}%
\pgfpathmoveto{\pgfqpoint{3.810560in}{1.148849in}}%
\pgfpathlineto{\pgfqpoint{3.824777in}{1.148873in}}%
\pgfpathlineto{\pgfqpoint{3.839002in}{1.149072in}}%
\pgfpathlineto{\pgfqpoint{3.853235in}{1.149448in}}%
\pgfpathlineto{\pgfqpoint{3.867478in}{1.149998in}}%
\pgfpathlineto{\pgfqpoint{3.875809in}{1.161723in}}%
\pgfpathlineto{\pgfqpoint{3.884135in}{1.173665in}}%
\pgfpathlineto{\pgfqpoint{3.892454in}{1.185818in}}%
\pgfpathlineto{\pgfqpoint{3.900767in}{1.198174in}}%
\pgfpathlineto{\pgfqpoint{3.886535in}{1.196858in}}%
\pgfpathlineto{\pgfqpoint{3.872312in}{1.195717in}}%
\pgfpathlineto{\pgfqpoint{3.858098in}{1.194752in}}%
\pgfpathlineto{\pgfqpoint{3.843893in}{1.193963in}}%
\pgfpathlineto{\pgfqpoint{3.835570in}{1.182361in}}%
\pgfpathlineto{\pgfqpoint{3.827240in}{1.170970in}}%
\pgfpathlineto{\pgfqpoint{3.818904in}{1.159796in}}%
\pgfpathlineto{\pgfqpoint{3.810560in}{1.148849in}}%
\pgfpathclose%
\pgfusepath{fill}%
\end{pgfscope}%
\begin{pgfscope}%
\pgfpathrectangle{\pgfqpoint{1.150000in}{0.150000in}}{\pgfqpoint{5.700000in}{5.700000in}}%
\pgfusepath{clip}%
\pgfsetbuttcap%
\pgfsetroundjoin%
\definecolor{currentfill}{rgb}{0.260571,0.246922,0.522828}%
\pgfsetfillcolor{currentfill}%
\pgfsetfillopacity{0.800000}%
\pgfsetlinewidth{0.000000pt}%
\definecolor{currentstroke}{rgb}{0.000000,0.000000,0.000000}%
\pgfsetstrokecolor{currentstroke}%
\pgfsetdash{}{0pt}%
\pgfpathmoveto{\pgfqpoint{4.237367in}{1.547253in}}%
\pgfpathlineto{\pgfqpoint{4.251737in}{1.554245in}}%
\pgfpathlineto{\pgfqpoint{4.266121in}{1.561414in}}%
\pgfpathlineto{\pgfqpoint{4.280519in}{1.568761in}}%
\pgfpathlineto{\pgfqpoint{4.294931in}{1.576286in}}%
\pgfpathlineto{\pgfqpoint{4.303148in}{1.593722in}}%
\pgfpathlineto{\pgfqpoint{4.311362in}{1.611189in}}%
\pgfpathlineto{\pgfqpoint{4.319573in}{1.628680in}}%
\pgfpathlineto{\pgfqpoint{4.327780in}{1.646189in}}%
\pgfpathlineto{\pgfqpoint{4.313363in}{1.638075in}}%
\pgfpathlineto{\pgfqpoint{4.298960in}{1.630140in}}%
\pgfpathlineto{\pgfqpoint{4.284571in}{1.622383in}}%
\pgfpathlineto{\pgfqpoint{4.270197in}{1.614805in}}%
\pgfpathlineto{\pgfqpoint{4.261995in}{1.597872in}}%
\pgfpathlineto{\pgfqpoint{4.253789in}{1.580965in}}%
\pgfpathlineto{\pgfqpoint{4.245580in}{1.564090in}}%
\pgfpathlineto{\pgfqpoint{4.237367in}{1.547253in}}%
\pgfpathclose%
\pgfusepath{fill}%
\end{pgfscope}%
\begin{pgfscope}%
\pgfpathrectangle{\pgfqpoint{1.150000in}{0.150000in}}{\pgfqpoint{5.700000in}{5.700000in}}%
\pgfusepath{clip}%
\pgfsetbuttcap%
\pgfsetroundjoin%
\definecolor{currentfill}{rgb}{0.278012,0.180367,0.486697}%
\pgfsetfillcolor{currentfill}%
\pgfsetfillopacity{0.800000}%
\pgfsetlinewidth{0.000000pt}%
\definecolor{currentstroke}{rgb}{0.000000,0.000000,0.000000}%
\pgfsetstrokecolor{currentstroke}%
\pgfsetdash{}{0pt}%
\pgfpathmoveto{\pgfqpoint{4.114200in}{1.393471in}}%
\pgfpathlineto{\pgfqpoint{4.128515in}{1.398489in}}%
\pgfpathlineto{\pgfqpoint{4.142842in}{1.403684in}}%
\pgfpathlineto{\pgfqpoint{4.157182in}{1.409055in}}%
\pgfpathlineto{\pgfqpoint{4.171534in}{1.414602in}}%
\pgfpathlineto{\pgfqpoint{4.179776in}{1.430930in}}%
\pgfpathlineto{\pgfqpoint{4.188014in}{1.447342in}}%
\pgfpathlineto{\pgfqpoint{4.196249in}{1.463833in}}%
\pgfpathlineto{\pgfqpoint{4.204480in}{1.480396in}}%
\pgfpathlineto{\pgfqpoint{4.190126in}{1.474200in}}%
\pgfpathlineto{\pgfqpoint{4.175784in}{1.468181in}}%
\pgfpathlineto{\pgfqpoint{4.161456in}{1.462339in}}%
\pgfpathlineto{\pgfqpoint{4.147140in}{1.456674in}}%
\pgfpathlineto{\pgfqpoint{4.138911in}{1.440747in}}%
\pgfpathlineto{\pgfqpoint{4.130678in}{1.424900in}}%
\pgfpathlineto{\pgfqpoint{4.122441in}{1.409140in}}%
\pgfpathlineto{\pgfqpoint{4.114200in}{1.393471in}}%
\pgfpathclose%
\pgfusepath{fill}%
\end{pgfscope}%
\begin{pgfscope}%
\pgfpathrectangle{\pgfqpoint{1.150000in}{0.150000in}}{\pgfqpoint{5.700000in}{5.700000in}}%
\pgfusepath{clip}%
\pgfsetbuttcap%
\pgfsetroundjoin%
\definecolor{currentfill}{rgb}{0.276022,0.044167,0.370164}%
\pgfsetfillcolor{currentfill}%
\pgfsetfillopacity{0.800000}%
\pgfsetlinewidth{0.000000pt}%
\definecolor{currentstroke}{rgb}{0.000000,0.000000,0.000000}%
\pgfsetstrokecolor{currentstroke}%
\pgfsetdash{}{0pt}%
\pgfpathmoveto{\pgfqpoint{3.720268in}{1.112319in}}%
\pgfpathlineto{\pgfqpoint{3.734469in}{1.110842in}}%
\pgfpathlineto{\pgfqpoint{3.748677in}{1.109542in}}%
\pgfpathlineto{\pgfqpoint{3.762892in}{1.108418in}}%
\pgfpathlineto{\pgfqpoint{3.777115in}{1.107470in}}%
\pgfpathlineto{\pgfqpoint{3.785487in}{1.117437in}}%
\pgfpathlineto{\pgfqpoint{3.793852in}{1.127662in}}%
\pgfpathlineto{\pgfqpoint{3.802210in}{1.138135in}}%
\pgfpathlineto{\pgfqpoint{3.810560in}{1.148849in}}%
\pgfpathlineto{\pgfqpoint{3.796352in}{1.149001in}}%
\pgfpathlineto{\pgfqpoint{3.782151in}{1.149329in}}%
\pgfpathlineto{\pgfqpoint{3.767959in}{1.149834in}}%
\pgfpathlineto{\pgfqpoint{3.753774in}{1.150515in}}%
\pgfpathlineto{\pgfqpoint{3.745409in}{1.140584in}}%
\pgfpathlineto{\pgfqpoint{3.737037in}{1.130903in}}%
\pgfpathlineto{\pgfqpoint{3.728656in}{1.121478in}}%
\pgfpathlineto{\pgfqpoint{3.720268in}{1.112319in}}%
\pgfpathclose%
\pgfusepath{fill}%
\end{pgfscope}%
\begin{pgfscope}%
\pgfpathrectangle{\pgfqpoint{1.150000in}{0.150000in}}{\pgfqpoint{5.700000in}{5.700000in}}%
\pgfusepath{clip}%
\pgfsetbuttcap%
\pgfsetroundjoin%
\definecolor{currentfill}{rgb}{0.281924,0.089666,0.412415}%
\pgfsetfillcolor{currentfill}%
\pgfsetfillopacity{0.800000}%
\pgfsetlinewidth{0.000000pt}%
\definecolor{currentstroke}{rgb}{0.000000,0.000000,0.000000}%
\pgfsetstrokecolor{currentstroke}%
\pgfsetdash{}{0pt}%
\pgfpathmoveto{\pgfqpoint{3.900767in}{1.198174in}}%
\pgfpathlineto{\pgfqpoint{3.915009in}{1.199666in}}%
\pgfpathlineto{\pgfqpoint{3.929259in}{1.201334in}}%
\pgfpathlineto{\pgfqpoint{3.943520in}{1.203176in}}%
\pgfpathlineto{\pgfqpoint{3.957790in}{1.205194in}}%
\pgfpathlineto{\pgfqpoint{3.966090in}{1.218494in}}%
\pgfpathlineto{\pgfqpoint{3.974384in}{1.231973in}}%
\pgfpathlineto{\pgfqpoint{3.982674in}{1.245626in}}%
\pgfpathlineto{\pgfqpoint{3.990958in}{1.259446in}}%
\pgfpathlineto{\pgfqpoint{3.976694in}{1.256691in}}%
\pgfpathlineto{\pgfqpoint{3.962440in}{1.254112in}}%
\pgfpathlineto{\pgfqpoint{3.948196in}{1.251708in}}%
\pgfpathlineto{\pgfqpoint{3.933962in}{1.249481in}}%
\pgfpathlineto{\pgfqpoint{3.925672in}{1.236386in}}%
\pgfpathlineto{\pgfqpoint{3.917376in}{1.223465in}}%
\pgfpathlineto{\pgfqpoint{3.909074in}{1.210726in}}%
\pgfpathlineto{\pgfqpoint{3.900767in}{1.198174in}}%
\pgfpathclose%
\pgfusepath{fill}%
\end{pgfscope}%
\begin{pgfscope}%
\pgfpathrectangle{\pgfqpoint{1.150000in}{0.150000in}}{\pgfqpoint{5.700000in}{5.700000in}}%
\pgfusepath{clip}%
\pgfsetbuttcap%
\pgfsetroundjoin%
\definecolor{currentfill}{rgb}{0.233603,0.313828,0.543914}%
\pgfsetfillcolor{currentfill}%
\pgfsetfillopacity{0.800000}%
\pgfsetlinewidth{0.000000pt}%
\definecolor{currentstroke}{rgb}{0.000000,0.000000,0.000000}%
\pgfsetstrokecolor{currentstroke}%
\pgfsetdash{}{0pt}%
\pgfpathmoveto{\pgfqpoint{4.360575in}{1.716310in}}%
\pgfpathlineto{\pgfqpoint{4.375013in}{1.725160in}}%
\pgfpathlineto{\pgfqpoint{4.389467in}{1.734190in}}%
\pgfpathlineto{\pgfqpoint{4.403935in}{1.743400in}}%
\pgfpathlineto{\pgfqpoint{4.418419in}{1.752790in}}%
\pgfpathlineto{\pgfqpoint{4.426617in}{1.770854in}}%
\pgfpathlineto{\pgfqpoint{4.434811in}{1.788898in}}%
\pgfpathlineto{\pgfqpoint{4.443001in}{1.806917in}}%
\pgfpathlineto{\pgfqpoint{4.451188in}{1.824906in}}%
\pgfpathlineto{\pgfqpoint{4.436696in}{1.814988in}}%
\pgfpathlineto{\pgfqpoint{4.422220in}{1.805251in}}%
\pgfpathlineto{\pgfqpoint{4.407759in}{1.795694in}}%
\pgfpathlineto{\pgfqpoint{4.393314in}{1.786317in}}%
\pgfpathlineto{\pgfqpoint{4.385135in}{1.768843in}}%
\pgfpathlineto{\pgfqpoint{4.376952in}{1.751347in}}%
\pgfpathlineto{\pgfqpoint{4.368765in}{1.733834in}}%
\pgfpathlineto{\pgfqpoint{4.360575in}{1.716310in}}%
\pgfpathclose%
\pgfusepath{fill}%
\end{pgfscope}%
\begin{pgfscope}%
\pgfpathrectangle{\pgfqpoint{1.150000in}{0.150000in}}{\pgfqpoint{5.700000in}{5.700000in}}%
\pgfusepath{clip}%
\pgfsetbuttcap%
\pgfsetroundjoin%
\definecolor{currentfill}{rgb}{0.119738,0.603785,0.541400}%
\pgfsetfillcolor{currentfill}%
\pgfsetfillopacity{0.800000}%
\pgfsetlinewidth{0.000000pt}%
\definecolor{currentstroke}{rgb}{0.000000,0.000000,0.000000}%
\pgfsetstrokecolor{currentstroke}%
\pgfsetdash{}{0pt}%
\pgfpathmoveto{\pgfqpoint{4.919280in}{2.591559in}}%
\pgfpathlineto{\pgfqpoint{4.934092in}{2.607172in}}%
\pgfpathlineto{\pgfqpoint{4.948925in}{2.622974in}}%
\pgfpathlineto{\pgfqpoint{4.963779in}{2.638964in}}%
\pgfpathlineto{\pgfqpoint{4.978654in}{2.655143in}}%
\pgfpathlineto{\pgfqpoint{4.986693in}{2.670235in}}%
\pgfpathlineto{\pgfqpoint{4.994724in}{2.685139in}}%
\pgfpathlineto{\pgfqpoint{5.002748in}{2.699857in}}%
\pgfpathlineto{\pgfqpoint{5.010765in}{2.714384in}}%
\pgfpathlineto{\pgfqpoint{4.995883in}{2.698007in}}%
\pgfpathlineto{\pgfqpoint{4.981021in}{2.681820in}}%
\pgfpathlineto{\pgfqpoint{4.966181in}{2.665822in}}%
\pgfpathlineto{\pgfqpoint{4.951362in}{2.650012in}}%
\pgfpathlineto{\pgfqpoint{4.943352in}{2.635668in}}%
\pgfpathlineto{\pgfqpoint{4.935335in}{2.621143in}}%
\pgfpathlineto{\pgfqpoint{4.927311in}{2.606440in}}%
\pgfpathlineto{\pgfqpoint{4.919280in}{2.591559in}}%
\pgfpathclose%
\pgfusepath{fill}%
\end{pgfscope}%
\begin{pgfscope}%
\pgfpathrectangle{\pgfqpoint{1.150000in}{0.150000in}}{\pgfqpoint{5.700000in}{5.700000in}}%
\pgfusepath{clip}%
\pgfsetbuttcap%
\pgfsetroundjoin%
\definecolor{currentfill}{rgb}{0.166617,0.463708,0.558119}%
\pgfsetfillcolor{currentfill}%
\pgfsetfillopacity{0.800000}%
\pgfsetlinewidth{0.000000pt}%
\definecolor{currentstroke}{rgb}{0.000000,0.000000,0.000000}%
\pgfsetstrokecolor{currentstroke}%
\pgfsetdash{}{0pt}%
\pgfpathmoveto{\pgfqpoint{4.639936in}{2.153720in}}%
\pgfpathlineto{\pgfqpoint{4.654549in}{2.166343in}}%
\pgfpathlineto{\pgfqpoint{4.669180in}{2.179150in}}%
\pgfpathlineto{\pgfqpoint{4.683830in}{2.192142in}}%
\pgfpathlineto{\pgfqpoint{4.698498in}{2.205318in}}%
\pgfpathlineto{\pgfqpoint{4.706635in}{2.222901in}}%
\pgfpathlineto{\pgfqpoint{4.714768in}{2.240365in}}%
\pgfpathlineto{\pgfqpoint{4.722896in}{2.257706in}}%
\pgfpathlineto{\pgfqpoint{4.731019in}{2.274922in}}%
\pgfpathlineto{\pgfqpoint{4.716341in}{2.261377in}}%
\pgfpathlineto{\pgfqpoint{4.701681in}{2.248017in}}%
\pgfpathlineto{\pgfqpoint{4.687041in}{2.234843in}}%
\pgfpathlineto{\pgfqpoint{4.672419in}{2.221853in}}%
\pgfpathlineto{\pgfqpoint{4.664305in}{2.204993in}}%
\pgfpathlineto{\pgfqpoint{4.656187in}{2.188015in}}%
\pgfpathlineto{\pgfqpoint{4.648064in}{2.170923in}}%
\pgfpathlineto{\pgfqpoint{4.639936in}{2.153720in}}%
\pgfpathclose%
\pgfusepath{fill}%
\end{pgfscope}%
\begin{pgfscope}%
\pgfpathrectangle{\pgfqpoint{1.150000in}{0.150000in}}{\pgfqpoint{5.700000in}{5.700000in}}%
\pgfusepath{clip}%
\pgfsetbuttcap%
\pgfsetroundjoin%
\definecolor{currentfill}{rgb}{0.395174,0.797475,0.367757}%
\pgfsetfillcolor{currentfill}%
\pgfsetfillopacity{0.800000}%
\pgfsetlinewidth{0.000000pt}%
\definecolor{currentstroke}{rgb}{0.000000,0.000000,0.000000}%
\pgfsetstrokecolor{currentstroke}%
\pgfsetdash{}{0pt}%
\pgfpathmoveto{\pgfqpoint{5.412366in}{3.257091in}}%
\pgfpathlineto{\pgfqpoint{5.427564in}{3.276334in}}%
\pgfpathlineto{\pgfqpoint{5.442785in}{3.295772in}}%
\pgfpathlineto{\pgfqpoint{5.458032in}{3.315404in}}%
\pgfpathlineto{\pgfqpoint{5.473304in}{3.335231in}}%
\pgfpathlineto{\pgfqpoint{5.481051in}{3.343718in}}%
\pgfpathlineto{\pgfqpoint{5.488787in}{3.351977in}}%
\pgfpathlineto{\pgfqpoint{5.496512in}{3.360009in}}%
\pgfpathlineto{\pgfqpoint{5.504224in}{3.367816in}}%
\pgfpathlineto{\pgfqpoint{5.488957in}{3.348085in}}%
\pgfpathlineto{\pgfqpoint{5.473715in}{3.328549in}}%
\pgfpathlineto{\pgfqpoint{5.458498in}{3.309206in}}%
\pgfpathlineto{\pgfqpoint{5.443306in}{3.290058in}}%
\pgfpathlineto{\pgfqpoint{5.435588in}{3.282141in}}%
\pgfpathlineto{\pgfqpoint{5.427858in}{3.274009in}}%
\pgfpathlineto{\pgfqpoint{5.420118in}{3.265659in}}%
\pgfpathlineto{\pgfqpoint{5.412366in}{3.257091in}}%
\pgfpathclose%
\pgfusepath{fill}%
\end{pgfscope}%
\begin{pgfscope}%
\pgfpathrectangle{\pgfqpoint{1.150000in}{0.150000in}}{\pgfqpoint{5.700000in}{5.700000in}}%
\pgfusepath{clip}%
\pgfsetbuttcap%
\pgfsetroundjoin%
\definecolor{currentfill}{rgb}{0.283187,0.125848,0.444960}%
\pgfsetfillcolor{currentfill}%
\pgfsetfillopacity{0.800000}%
\pgfsetlinewidth{0.000000pt}%
\definecolor{currentstroke}{rgb}{0.000000,0.000000,0.000000}%
\pgfsetstrokecolor{currentstroke}%
\pgfsetdash{}{0pt}%
\pgfpathmoveto{\pgfqpoint{3.990958in}{1.259446in}}%
\pgfpathlineto{\pgfqpoint{4.005232in}{1.262376in}}%
\pgfpathlineto{\pgfqpoint{4.019517in}{1.265481in}}%
\pgfpathlineto{\pgfqpoint{4.033813in}{1.268762in}}%
\pgfpathlineto{\pgfqpoint{4.048120in}{1.272219in}}%
\pgfpathlineto{\pgfqpoint{4.056395in}{1.286916in}}%
\pgfpathlineto{\pgfqpoint{4.064666in}{1.301758in}}%
\pgfpathlineto{\pgfqpoint{4.072932in}{1.316738in}}%
\pgfpathlineto{\pgfqpoint{4.081194in}{1.331849in}}%
\pgfpathlineto{\pgfqpoint{4.066890in}{1.327685in}}%
\pgfpathlineto{\pgfqpoint{4.052597in}{1.323696in}}%
\pgfpathlineto{\pgfqpoint{4.038316in}{1.319883in}}%
\pgfpathlineto{\pgfqpoint{4.024045in}{1.316247in}}%
\pgfpathlineto{\pgfqpoint{4.015780in}{1.301831in}}%
\pgfpathlineto{\pgfqpoint{4.007511in}{1.287555in}}%
\pgfpathlineto{\pgfqpoint{3.999237in}{1.273424in}}%
\pgfpathlineto{\pgfqpoint{3.990958in}{1.259446in}}%
\pgfpathclose%
\pgfusepath{fill}%
\end{pgfscope}%
\begin{pgfscope}%
\pgfpathrectangle{\pgfqpoint{1.150000in}{0.150000in}}{\pgfqpoint{5.700000in}{5.700000in}}%
\pgfusepath{clip}%
\pgfsetbuttcap%
\pgfsetroundjoin%
\definecolor{currentfill}{rgb}{0.137339,0.662252,0.515571}%
\pgfsetfillcolor{currentfill}%
\pgfsetfillopacity{0.800000}%
\pgfsetlinewidth{0.000000pt}%
\definecolor{currentstroke}{rgb}{0.000000,0.000000,0.000000}%
\pgfsetstrokecolor{currentstroke}%
\pgfsetdash{}{0pt}%
\pgfpathmoveto{\pgfqpoint{5.042756in}{2.770577in}}%
\pgfpathlineto{\pgfqpoint{5.057667in}{2.787305in}}%
\pgfpathlineto{\pgfqpoint{5.072599in}{2.804224in}}%
\pgfpathlineto{\pgfqpoint{5.087553in}{2.821334in}}%
\pgfpathlineto{\pgfqpoint{5.102530in}{2.838634in}}%
\pgfpathlineto{\pgfqpoint{5.110513in}{2.852336in}}%
\pgfpathlineto{\pgfqpoint{5.118489in}{2.865833in}}%
\pgfpathlineto{\pgfqpoint{5.126456in}{2.879124in}}%
\pgfpathlineto{\pgfqpoint{5.134414in}{2.892207in}}%
\pgfpathlineto{\pgfqpoint{5.119433in}{2.874781in}}%
\pgfpathlineto{\pgfqpoint{5.104473in}{2.857545in}}%
\pgfpathlineto{\pgfqpoint{5.089536in}{2.840500in}}%
\pgfpathlineto{\pgfqpoint{5.074621in}{2.823646in}}%
\pgfpathlineto{\pgfqpoint{5.066667in}{2.810674in}}%
\pgfpathlineto{\pgfqpoint{5.058705in}{2.797505in}}%
\pgfpathlineto{\pgfqpoint{5.050734in}{2.784139in}}%
\pgfpathlineto{\pgfqpoint{5.042756in}{2.770577in}}%
\pgfpathclose%
\pgfusepath{fill}%
\end{pgfscope}%
\begin{pgfscope}%
\pgfpathrectangle{\pgfqpoint{1.150000in}{0.150000in}}{\pgfqpoint{5.700000in}{5.700000in}}%
\pgfusepath{clip}%
\pgfsetbuttcap%
\pgfsetroundjoin%
\definecolor{currentfill}{rgb}{0.201239,0.383670,0.554294}%
\pgfsetfillcolor{currentfill}%
\pgfsetfillopacity{0.800000}%
\pgfsetlinewidth{0.000000pt}%
\definecolor{currentstroke}{rgb}{0.000000,0.000000,0.000000}%
\pgfsetstrokecolor{currentstroke}%
\pgfsetdash{}{0pt}%
\pgfpathmoveto{\pgfqpoint{4.483900in}{1.896471in}}%
\pgfpathlineto{\pgfqpoint{4.498416in}{1.907066in}}%
\pgfpathlineto{\pgfqpoint{4.512950in}{1.917843in}}%
\pgfpathlineto{\pgfqpoint{4.527499in}{1.928801in}}%
\pgfpathlineto{\pgfqpoint{4.542066in}{1.939941in}}%
\pgfpathlineto{\pgfqpoint{4.550244in}{1.958189in}}%
\pgfpathlineto{\pgfqpoint{4.558418in}{1.976372in}}%
\pgfpathlineto{\pgfqpoint{4.566588in}{1.994486in}}%
\pgfpathlineto{\pgfqpoint{4.574755in}{2.012526in}}%
\pgfpathlineto{\pgfqpoint{4.560178in}{2.000919in}}%
\pgfpathlineto{\pgfqpoint{4.545619in}{1.989496in}}%
\pgfpathlineto{\pgfqpoint{4.531077in}{1.978254in}}%
\pgfpathlineto{\pgfqpoint{4.516552in}{1.967195in}}%
\pgfpathlineto{\pgfqpoint{4.508395in}{1.949608in}}%
\pgfpathlineto{\pgfqpoint{4.500234in}{1.931956in}}%
\pgfpathlineto{\pgfqpoint{4.492069in}{1.914242in}}%
\pgfpathlineto{\pgfqpoint{4.483900in}{1.896471in}}%
\pgfpathclose%
\pgfusepath{fill}%
\end{pgfscope}%
\begin{pgfscope}%
\pgfpathrectangle{\pgfqpoint{1.150000in}{0.150000in}}{\pgfqpoint{5.700000in}{5.700000in}}%
\pgfusepath{clip}%
\pgfsetbuttcap%
\pgfsetroundjoin%
\definecolor{currentfill}{rgb}{0.585678,0.846661,0.249897}%
\pgfsetfillcolor{currentfill}%
\pgfsetfillopacity{0.800000}%
\pgfsetlinewidth{0.000000pt}%
\definecolor{currentstroke}{rgb}{0.000000,0.000000,0.000000}%
\pgfsetstrokecolor{currentstroke}%
\pgfsetdash{}{0pt}%
\pgfpathmoveto{\pgfqpoint{5.626755in}{3.502043in}}%
\pgfpathlineto{\pgfqpoint{5.642133in}{3.522445in}}%
\pgfpathlineto{\pgfqpoint{5.657538in}{3.543043in}}%
\pgfpathlineto{\pgfqpoint{5.672969in}{3.563838in}}%
\pgfpathlineto{\pgfqpoint{5.680556in}{3.569361in}}%
\pgfpathlineto{\pgfqpoint{5.688130in}{3.574664in}}%
\pgfpathlineto{\pgfqpoint{5.695691in}{3.579748in}}%
\pgfpathlineto{\pgfqpoint{5.703239in}{3.584616in}}%
\pgfpathlineto{\pgfqpoint{5.687819in}{3.564032in}}%
\pgfpathlineto{\pgfqpoint{5.672426in}{3.543644in}}%
\pgfpathlineto{\pgfqpoint{5.657059in}{3.523452in}}%
\pgfpathlineto{\pgfqpoint{5.649502in}{3.518416in}}%
\pgfpathlineto{\pgfqpoint{5.641932in}{3.513171in}}%
\pgfpathlineto{\pgfqpoint{5.634350in}{3.507714in}}%
\pgfpathlineto{\pgfqpoint{5.626755in}{3.502043in}}%
\pgfpathclose%
\pgfusepath{fill}%
\end{pgfscope}%
\begin{pgfscope}%
\pgfpathrectangle{\pgfqpoint{1.150000in}{0.150000in}}{\pgfqpoint{5.700000in}{5.700000in}}%
\pgfusepath{clip}%
\pgfsetbuttcap%
\pgfsetroundjoin%
\definecolor{currentfill}{rgb}{0.288921,0.758394,0.428426}%
\pgfsetfillcolor{currentfill}%
\pgfsetfillopacity{0.800000}%
\pgfsetlinewidth{0.000000pt}%
\definecolor{currentstroke}{rgb}{0.000000,0.000000,0.000000}%
\pgfsetstrokecolor{currentstroke}%
\pgfsetdash{}{0pt}%
\pgfpathmoveto{\pgfqpoint{5.289399in}{3.105193in}}%
\pgfpathlineto{\pgfqpoint{5.304503in}{3.123741in}}%
\pgfpathlineto{\pgfqpoint{5.319631in}{3.142482in}}%
\pgfpathlineto{\pgfqpoint{5.334783in}{3.161416in}}%
\pgfpathlineto{\pgfqpoint{5.349960in}{3.180544in}}%
\pgfpathlineto{\pgfqpoint{5.357798in}{3.190899in}}%
\pgfpathlineto{\pgfqpoint{5.365626in}{3.201028in}}%
\pgfpathlineto{\pgfqpoint{5.373443in}{3.210931in}}%
\pgfpathlineto{\pgfqpoint{5.381250in}{3.220609in}}%
\pgfpathlineto{\pgfqpoint{5.366074in}{3.201501in}}%
\pgfpathlineto{\pgfqpoint{5.350923in}{3.182588in}}%
\pgfpathlineto{\pgfqpoint{5.335796in}{3.163867in}}%
\pgfpathlineto{\pgfqpoint{5.320692in}{3.145340in}}%
\pgfpathlineto{\pgfqpoint{5.312884in}{3.135627in}}%
\pgfpathlineto{\pgfqpoint{5.305066in}{3.125699in}}%
\pgfpathlineto{\pgfqpoint{5.297237in}{3.115555in}}%
\pgfpathlineto{\pgfqpoint{5.289399in}{3.105193in}}%
\pgfpathclose%
\pgfusepath{fill}%
\end{pgfscope}%
\begin{pgfscope}%
\pgfpathrectangle{\pgfqpoint{1.150000in}{0.150000in}}{\pgfqpoint{5.700000in}{5.700000in}}%
\pgfusepath{clip}%
\pgfsetbuttcap%
\pgfsetroundjoin%
\definecolor{currentfill}{rgb}{0.202219,0.715272,0.476084}%
\pgfsetfillcolor{currentfill}%
\pgfsetfillopacity{0.800000}%
\pgfsetlinewidth{0.000000pt}%
\definecolor{currentstroke}{rgb}{0.000000,0.000000,0.000000}%
\pgfsetstrokecolor{currentstroke}%
\pgfsetdash{}{0pt}%
\pgfpathmoveto{\pgfqpoint{5.166161in}{2.942462in}}%
\pgfpathlineto{\pgfqpoint{5.181169in}{2.960170in}}%
\pgfpathlineto{\pgfqpoint{5.196200in}{2.978070in}}%
\pgfpathlineto{\pgfqpoint{5.211254in}{2.996162in}}%
\pgfpathlineto{\pgfqpoint{5.226332in}{3.014447in}}%
\pgfpathlineto{\pgfqpoint{5.234249in}{3.026556in}}%
\pgfpathlineto{\pgfqpoint{5.242157in}{3.038446in}}%
\pgfpathlineto{\pgfqpoint{5.250055in}{3.050117in}}%
\pgfpathlineto{\pgfqpoint{5.257944in}{3.061569in}}%
\pgfpathlineto{\pgfqpoint{5.242864in}{3.043231in}}%
\pgfpathlineto{\pgfqpoint{5.227807in}{3.025085in}}%
\pgfpathlineto{\pgfqpoint{5.212773in}{3.007132in}}%
\pgfpathlineto{\pgfqpoint{5.197763in}{2.989370in}}%
\pgfpathlineto{\pgfqpoint{5.189876in}{2.977957in}}%
\pgfpathlineto{\pgfqpoint{5.181980in}{2.966335in}}%
\pgfpathlineto{\pgfqpoint{5.174075in}{2.954503in}}%
\pgfpathlineto{\pgfqpoint{5.166161in}{2.942462in}}%
\pgfpathclose%
\pgfusepath{fill}%
\end{pgfscope}%
\begin{pgfscope}%
\pgfpathrectangle{\pgfqpoint{1.150000in}{0.150000in}}{\pgfqpoint{5.700000in}{5.700000in}}%
\pgfusepath{clip}%
\pgfsetbuttcap%
\pgfsetroundjoin%
\definecolor{currentfill}{rgb}{0.140536,0.530132,0.555659}%
\pgfsetfillcolor{currentfill}%
\pgfsetfillopacity{0.800000}%
\pgfsetlinewidth{0.000000pt}%
\definecolor{currentstroke}{rgb}{0.000000,0.000000,0.000000}%
\pgfsetstrokecolor{currentstroke}%
\pgfsetdash{}{0pt}%
\pgfpathmoveto{\pgfqpoint{4.763459in}{2.342465in}}%
\pgfpathlineto{\pgfqpoint{4.778166in}{2.356530in}}%
\pgfpathlineto{\pgfqpoint{4.792892in}{2.370781in}}%
\pgfpathlineto{\pgfqpoint{4.807637in}{2.385218in}}%
\pgfpathlineto{\pgfqpoint{4.822403in}{2.399842in}}%
\pgfpathlineto{\pgfqpoint{4.830509in}{2.416698in}}%
\pgfpathlineto{\pgfqpoint{4.838610in}{2.433402in}}%
\pgfpathlineto{\pgfqpoint{4.846705in}{2.449951in}}%
\pgfpathlineto{\pgfqpoint{4.854793in}{2.466344in}}%
\pgfpathlineto{\pgfqpoint{4.840018in}{2.451418in}}%
\pgfpathlineto{\pgfqpoint{4.825263in}{2.436679in}}%
\pgfpathlineto{\pgfqpoint{4.810528in}{2.422127in}}%
\pgfpathlineto{\pgfqpoint{4.795812in}{2.407762in}}%
\pgfpathlineto{\pgfqpoint{4.787733in}{2.391657in}}%
\pgfpathlineto{\pgfqpoint{4.779647in}{2.375405in}}%
\pgfpathlineto{\pgfqpoint{4.771556in}{2.359006in}}%
\pgfpathlineto{\pgfqpoint{4.763459in}{2.342465in}}%
\pgfpathclose%
\pgfusepath{fill}%
\end{pgfscope}%
\begin{pgfscope}%
\pgfpathrectangle{\pgfqpoint{1.150000in}{0.150000in}}{\pgfqpoint{5.700000in}{5.700000in}}%
\pgfusepath{clip}%
\pgfsetbuttcap%
\pgfsetroundjoin%
\definecolor{currentfill}{rgb}{0.267968,0.223549,0.512008}%
\pgfsetfillcolor{currentfill}%
\pgfsetfillopacity{0.800000}%
\pgfsetlinewidth{0.000000pt}%
\definecolor{currentstroke}{rgb}{0.000000,0.000000,0.000000}%
\pgfsetstrokecolor{currentstroke}%
\pgfsetdash{}{0pt}%
\pgfpathmoveto{\pgfqpoint{4.204480in}{1.480396in}}%
\pgfpathlineto{\pgfqpoint{4.218847in}{1.486769in}}%
\pgfpathlineto{\pgfqpoint{4.233227in}{1.493319in}}%
\pgfpathlineto{\pgfqpoint{4.247621in}{1.500046in}}%
\pgfpathlineto{\pgfqpoint{4.262028in}{1.506950in}}%
\pgfpathlineto{\pgfqpoint{4.270259in}{1.524211in}}%
\pgfpathlineto{\pgfqpoint{4.278486in}{1.541524in}}%
\pgfpathlineto{\pgfqpoint{4.286710in}{1.558885in}}%
\pgfpathlineto{\pgfqpoint{4.294931in}{1.576286in}}%
\pgfpathlineto{\pgfqpoint{4.280519in}{1.568761in}}%
\pgfpathlineto{\pgfqpoint{4.266121in}{1.561414in}}%
\pgfpathlineto{\pgfqpoint{4.251737in}{1.554245in}}%
\pgfpathlineto{\pgfqpoint{4.237367in}{1.547253in}}%
\pgfpathlineto{\pgfqpoint{4.229150in}{1.530460in}}%
\pgfpathlineto{\pgfqpoint{4.220930in}{1.513715in}}%
\pgfpathlineto{\pgfqpoint{4.212707in}{1.497025in}}%
\pgfpathlineto{\pgfqpoint{4.204480in}{1.480396in}}%
\pgfpathclose%
\pgfusepath{fill}%
\end{pgfscope}%
\begin{pgfscope}%
\pgfpathrectangle{\pgfqpoint{1.150000in}{0.150000in}}{\pgfqpoint{5.700000in}{5.700000in}}%
\pgfusepath{clip}%
\pgfsetbuttcap%
\pgfsetroundjoin%
\definecolor{currentfill}{rgb}{0.280868,0.160771,0.472899}%
\pgfsetfillcolor{currentfill}%
\pgfsetfillopacity{0.800000}%
\pgfsetlinewidth{0.000000pt}%
\definecolor{currentstroke}{rgb}{0.000000,0.000000,0.000000}%
\pgfsetstrokecolor{currentstroke}%
\pgfsetdash{}{0pt}%
\pgfpathmoveto{\pgfqpoint{4.081194in}{1.331849in}}%
\pgfpathlineto{\pgfqpoint{4.095510in}{1.336189in}}%
\pgfpathlineto{\pgfqpoint{4.109838in}{1.340705in}}%
\pgfpathlineto{\pgfqpoint{4.124177in}{1.345397in}}%
\pgfpathlineto{\pgfqpoint{4.138528in}{1.350264in}}%
\pgfpathlineto{\pgfqpoint{4.146785in}{1.366190in}}%
\pgfpathlineto{\pgfqpoint{4.155039in}{1.382226in}}%
\pgfpathlineto{\pgfqpoint{4.163288in}{1.398365in}}%
\pgfpathlineto{\pgfqpoint{4.171534in}{1.414602in}}%
\pgfpathlineto{\pgfqpoint{4.157182in}{1.409055in}}%
\pgfpathlineto{\pgfqpoint{4.142842in}{1.403684in}}%
\pgfpathlineto{\pgfqpoint{4.128515in}{1.398489in}}%
\pgfpathlineto{\pgfqpoint{4.114200in}{1.393471in}}%
\pgfpathlineto{\pgfqpoint{4.105955in}{1.377902in}}%
\pgfpathlineto{\pgfqpoint{4.097705in}{1.362437in}}%
\pgfpathlineto{\pgfqpoint{4.089452in}{1.347084in}}%
\pgfpathlineto{\pgfqpoint{4.081194in}{1.331849in}}%
\pgfpathclose%
\pgfusepath{fill}%
\end{pgfscope}%
\begin{pgfscope}%
\pgfpathrectangle{\pgfqpoint{1.150000in}{0.150000in}}{\pgfqpoint{5.700000in}{5.700000in}}%
\pgfusepath{clip}%
\pgfsetbuttcap%
\pgfsetroundjoin%
\definecolor{currentfill}{rgb}{0.243113,0.292092,0.538516}%
\pgfsetfillcolor{currentfill}%
\pgfsetfillopacity{0.800000}%
\pgfsetlinewidth{0.000000pt}%
\definecolor{currentstroke}{rgb}{0.000000,0.000000,0.000000}%
\pgfsetstrokecolor{currentstroke}%
\pgfsetdash{}{0pt}%
\pgfpathmoveto{\pgfqpoint{4.327780in}{1.646189in}}%
\pgfpathlineto{\pgfqpoint{4.342212in}{1.654482in}}%
\pgfpathlineto{\pgfqpoint{4.356659in}{1.662953in}}%
\pgfpathlineto{\pgfqpoint{4.371120in}{1.671603in}}%
\pgfpathlineto{\pgfqpoint{4.385597in}{1.680432in}}%
\pgfpathlineto{\pgfqpoint{4.393807in}{1.698526in}}%
\pgfpathlineto{\pgfqpoint{4.402015in}{1.716621in}}%
\pgfpathlineto{\pgfqpoint{4.410219in}{1.734710in}}%
\pgfpathlineto{\pgfqpoint{4.418419in}{1.752790in}}%
\pgfpathlineto{\pgfqpoint{4.403935in}{1.743400in}}%
\pgfpathlineto{\pgfqpoint{4.389467in}{1.734190in}}%
\pgfpathlineto{\pgfqpoint{4.375013in}{1.725160in}}%
\pgfpathlineto{\pgfqpoint{4.360575in}{1.716310in}}%
\pgfpathlineto{\pgfqpoint{4.352382in}{1.698778in}}%
\pgfpathlineto{\pgfqpoint{4.344185in}{1.681243in}}%
\pgfpathlineto{\pgfqpoint{4.335984in}{1.663712in}}%
\pgfpathlineto{\pgfqpoint{4.327780in}{1.646189in}}%
\pgfpathclose%
\pgfusepath{fill}%
\end{pgfscope}%
\begin{pgfscope}%
\pgfpathrectangle{\pgfqpoint{1.150000in}{0.150000in}}{\pgfqpoint{5.700000in}{5.700000in}}%
\pgfusepath{clip}%
\pgfsetbuttcap%
\pgfsetroundjoin%
\definecolor{currentfill}{rgb}{0.172719,0.448791,0.557885}%
\pgfsetfillcolor{currentfill}%
\pgfsetfillopacity{0.800000}%
\pgfsetlinewidth{0.000000pt}%
\definecolor{currentstroke}{rgb}{0.000000,0.000000,0.000000}%
\pgfsetstrokecolor{currentstroke}%
\pgfsetdash{}{0pt}%
\pgfpathmoveto{\pgfqpoint{4.607379in}{2.083868in}}%
\pgfpathlineto{\pgfqpoint{4.621983in}{2.096091in}}%
\pgfpathlineto{\pgfqpoint{4.636604in}{2.108497in}}%
\pgfpathlineto{\pgfqpoint{4.651244in}{2.121087in}}%
\pgfpathlineto{\pgfqpoint{4.665901in}{2.133860in}}%
\pgfpathlineto{\pgfqpoint{4.674057in}{2.151886in}}%
\pgfpathlineto{\pgfqpoint{4.682208in}{2.169807in}}%
\pgfpathlineto{\pgfqpoint{4.690355in}{2.187619in}}%
\pgfpathlineto{\pgfqpoint{4.698498in}{2.205318in}}%
\pgfpathlineto{\pgfqpoint{4.683830in}{2.192142in}}%
\pgfpathlineto{\pgfqpoint{4.669180in}{2.179150in}}%
\pgfpathlineto{\pgfqpoint{4.654549in}{2.166343in}}%
\pgfpathlineto{\pgfqpoint{4.639936in}{2.153720in}}%
\pgfpathlineto{\pgfqpoint{4.631803in}{2.136410in}}%
\pgfpathlineto{\pgfqpoint{4.623666in}{2.118995in}}%
\pgfpathlineto{\pgfqpoint{4.615525in}{2.101480in}}%
\pgfpathlineto{\pgfqpoint{4.607379in}{2.083868in}}%
\pgfpathclose%
\pgfusepath{fill}%
\end{pgfscope}%
\begin{pgfscope}%
\pgfpathrectangle{\pgfqpoint{1.150000in}{0.150000in}}{\pgfqpoint{5.700000in}{5.700000in}}%
\pgfusepath{clip}%
\pgfsetbuttcap%
\pgfsetroundjoin%
\definecolor{currentfill}{rgb}{0.121831,0.589055,0.545623}%
\pgfsetfillcolor{currentfill}%
\pgfsetfillopacity{0.800000}%
\pgfsetlinewidth{0.000000pt}%
\definecolor{currentstroke}{rgb}{0.000000,0.000000,0.000000}%
\pgfsetstrokecolor{currentstroke}%
\pgfsetdash{}{0pt}%
\pgfpathmoveto{\pgfqpoint{4.887088in}{2.530297in}}%
\pgfpathlineto{\pgfqpoint{4.901892in}{2.545678in}}%
\pgfpathlineto{\pgfqpoint{4.916717in}{2.561247in}}%
\pgfpathlineto{\pgfqpoint{4.931563in}{2.577004in}}%
\pgfpathlineto{\pgfqpoint{4.946429in}{2.592950in}}%
\pgfpathlineto{\pgfqpoint{4.954496in}{2.608769in}}%
\pgfpathlineto{\pgfqpoint{4.962555in}{2.624409in}}%
\pgfpathlineto{\pgfqpoint{4.970608in}{2.639868in}}%
\pgfpathlineto{\pgfqpoint{4.978654in}{2.655143in}}%
\pgfpathlineto{\pgfqpoint{4.963779in}{2.638964in}}%
\pgfpathlineto{\pgfqpoint{4.948925in}{2.622974in}}%
\pgfpathlineto{\pgfqpoint{4.934092in}{2.607172in}}%
\pgfpathlineto{\pgfqpoint{4.919280in}{2.591559in}}%
\pgfpathlineto{\pgfqpoint{4.911242in}{2.576503in}}%
\pgfpathlineto{\pgfqpoint{4.903197in}{2.561272in}}%
\pgfpathlineto{\pgfqpoint{4.895146in}{2.545870in}}%
\pgfpathlineto{\pgfqpoint{4.887088in}{2.530297in}}%
\pgfpathclose%
\pgfusepath{fill}%
\end{pgfscope}%
\begin{pgfscope}%
\pgfpathrectangle{\pgfqpoint{1.150000in}{0.150000in}}{\pgfqpoint{5.700000in}{5.700000in}}%
\pgfusepath{clip}%
\pgfsetbuttcap%
\pgfsetroundjoin%
\definecolor{currentfill}{rgb}{0.277941,0.056324,0.381191}%
\pgfsetfillcolor{currentfill}%
\pgfsetfillopacity{0.800000}%
\pgfsetlinewidth{0.000000pt}%
\definecolor{currentstroke}{rgb}{0.000000,0.000000,0.000000}%
\pgfsetstrokecolor{currentstroke}%
\pgfsetdash{}{0pt}%
\pgfpathmoveto{\pgfqpoint{3.777115in}{1.107470in}}%
\pgfpathlineto{\pgfqpoint{3.791345in}{1.106697in}}%
\pgfpathlineto{\pgfqpoint{3.805584in}{1.106100in}}%
\pgfpathlineto{\pgfqpoint{3.819830in}{1.105677in}}%
\pgfpathlineto{\pgfqpoint{3.834084in}{1.105430in}}%
\pgfpathlineto{\pgfqpoint{3.842443in}{1.116207in}}%
\pgfpathlineto{\pgfqpoint{3.850795in}{1.127232in}}%
\pgfpathlineto{\pgfqpoint{3.859139in}{1.138498in}}%
\pgfpathlineto{\pgfqpoint{3.867478in}{1.149998in}}%
\pgfpathlineto{\pgfqpoint{3.853235in}{1.149448in}}%
\pgfpathlineto{\pgfqpoint{3.839002in}{1.149072in}}%
\pgfpathlineto{\pgfqpoint{3.824777in}{1.148873in}}%
\pgfpathlineto{\pgfqpoint{3.810560in}{1.148849in}}%
\pgfpathlineto{\pgfqpoint{3.802210in}{1.138135in}}%
\pgfpathlineto{\pgfqpoint{3.793852in}{1.127662in}}%
\pgfpathlineto{\pgfqpoint{3.785487in}{1.117437in}}%
\pgfpathlineto{\pgfqpoint{3.777115in}{1.107470in}}%
\pgfpathclose%
\pgfusepath{fill}%
\end{pgfscope}%
\begin{pgfscope}%
\pgfpathrectangle{\pgfqpoint{1.150000in}{0.150000in}}{\pgfqpoint{5.700000in}{5.700000in}}%
\pgfusepath{clip}%
\pgfsetbuttcap%
\pgfsetroundjoin%
\definecolor{currentfill}{rgb}{0.280894,0.078907,0.402329}%
\pgfsetfillcolor{currentfill}%
\pgfsetfillopacity{0.800000}%
\pgfsetlinewidth{0.000000pt}%
\definecolor{currentstroke}{rgb}{0.000000,0.000000,0.000000}%
\pgfsetstrokecolor{currentstroke}%
\pgfsetdash{}{0pt}%
\pgfpathmoveto{\pgfqpoint{3.867478in}{1.149998in}}%
\pgfpathlineto{\pgfqpoint{3.881728in}{1.150723in}}%
\pgfpathlineto{\pgfqpoint{3.895988in}{1.151623in}}%
\pgfpathlineto{\pgfqpoint{3.910257in}{1.152698in}}%
\pgfpathlineto{\pgfqpoint{3.924536in}{1.153948in}}%
\pgfpathlineto{\pgfqpoint{3.932858in}{1.166452in}}%
\pgfpathlineto{\pgfqpoint{3.941174in}{1.179166in}}%
\pgfpathlineto{\pgfqpoint{3.949485in}{1.192083in}}%
\pgfpathlineto{\pgfqpoint{3.957790in}{1.205194in}}%
\pgfpathlineto{\pgfqpoint{3.943520in}{1.203176in}}%
\pgfpathlineto{\pgfqpoint{3.929259in}{1.201334in}}%
\pgfpathlineto{\pgfqpoint{3.915009in}{1.199666in}}%
\pgfpathlineto{\pgfqpoint{3.900767in}{1.198174in}}%
\pgfpathlineto{\pgfqpoint{3.892454in}{1.185818in}}%
\pgfpathlineto{\pgfqpoint{3.884135in}{1.173665in}}%
\pgfpathlineto{\pgfqpoint{3.875809in}{1.161723in}}%
\pgfpathlineto{\pgfqpoint{3.867478in}{1.149998in}}%
\pgfpathclose%
\pgfusepath{fill}%
\end{pgfscope}%
\begin{pgfscope}%
\pgfpathrectangle{\pgfqpoint{1.150000in}{0.150000in}}{\pgfqpoint{5.700000in}{5.700000in}}%
\pgfusepath{clip}%
\pgfsetbuttcap%
\pgfsetroundjoin%
\definecolor{currentfill}{rgb}{0.210503,0.363727,0.552206}%
\pgfsetfillcolor{currentfill}%
\pgfsetfillopacity{0.800000}%
\pgfsetlinewidth{0.000000pt}%
\definecolor{currentstroke}{rgb}{0.000000,0.000000,0.000000}%
\pgfsetstrokecolor{currentstroke}%
\pgfsetdash{}{0pt}%
\pgfpathmoveto{\pgfqpoint{4.451188in}{1.824906in}}%
\pgfpathlineto{\pgfqpoint{4.465696in}{1.835005in}}%
\pgfpathlineto{\pgfqpoint{4.480220in}{1.845284in}}%
\pgfpathlineto{\pgfqpoint{4.494761in}{1.855743in}}%
\pgfpathlineto{\pgfqpoint{4.509318in}{1.866384in}}%
\pgfpathlineto{\pgfqpoint{4.517510in}{1.884849in}}%
\pgfpathlineto{\pgfqpoint{4.525699in}{1.903266in}}%
\pgfpathlineto{\pgfqpoint{4.533884in}{1.921632in}}%
\pgfpathlineto{\pgfqpoint{4.542066in}{1.939941in}}%
\pgfpathlineto{\pgfqpoint{4.527499in}{1.928801in}}%
\pgfpathlineto{\pgfqpoint{4.512950in}{1.917843in}}%
\pgfpathlineto{\pgfqpoint{4.498416in}{1.907066in}}%
\pgfpathlineto{\pgfqpoint{4.483900in}{1.896471in}}%
\pgfpathlineto{\pgfqpoint{4.475727in}{1.878648in}}%
\pgfpathlineto{\pgfqpoint{4.467551in}{1.860776in}}%
\pgfpathlineto{\pgfqpoint{4.459371in}{1.842861in}}%
\pgfpathlineto{\pgfqpoint{4.451188in}{1.824906in}}%
\pgfpathclose%
\pgfusepath{fill}%
\end{pgfscope}%
\begin{pgfscope}%
\pgfpathrectangle{\pgfqpoint{1.150000in}{0.150000in}}{\pgfqpoint{5.700000in}{5.700000in}}%
\pgfusepath{clip}%
\pgfsetbuttcap%
\pgfsetroundjoin%
\definecolor{currentfill}{rgb}{0.496615,0.826376,0.306377}%
\pgfsetfillcolor{currentfill}%
\pgfsetfillopacity{0.800000}%
\pgfsetlinewidth{0.000000pt}%
\definecolor{currentstroke}{rgb}{0.000000,0.000000,0.000000}%
\pgfsetstrokecolor{currentstroke}%
\pgfsetdash{}{0pt}%
\pgfpathmoveto{\pgfqpoint{5.504224in}{3.367816in}}%
\pgfpathlineto{\pgfqpoint{5.519516in}{3.387742in}}%
\pgfpathlineto{\pgfqpoint{5.534834in}{3.407864in}}%
\pgfpathlineto{\pgfqpoint{5.550177in}{3.428181in}}%
\pgfpathlineto{\pgfqpoint{5.565546in}{3.448695in}}%
\pgfpathlineto{\pgfqpoint{5.573240in}{3.456158in}}%
\pgfpathlineto{\pgfqpoint{5.580923in}{3.463391in}}%
\pgfpathlineto{\pgfqpoint{5.588592in}{3.470395in}}%
\pgfpathlineto{\pgfqpoint{5.596250in}{3.477171in}}%
\pgfpathlineto{\pgfqpoint{5.580888in}{3.456792in}}%
\pgfpathlineto{\pgfqpoint{5.565552in}{3.436610in}}%
\pgfpathlineto{\pgfqpoint{5.550241in}{3.416622in}}%
\pgfpathlineto{\pgfqpoint{5.534956in}{3.396830in}}%
\pgfpathlineto{\pgfqpoint{5.527291in}{3.389905in}}%
\pgfpathlineto{\pgfqpoint{5.519614in}{3.382763in}}%
\pgfpathlineto{\pgfqpoint{5.511925in}{3.375400in}}%
\pgfpathlineto{\pgfqpoint{5.504224in}{3.367816in}}%
\pgfpathclose%
\pgfusepath{fill}%
\end{pgfscope}%
\begin{pgfscope}%
\pgfpathrectangle{\pgfqpoint{1.150000in}{0.150000in}}{\pgfqpoint{5.700000in}{5.700000in}}%
\pgfusepath{clip}%
\pgfsetbuttcap%
\pgfsetroundjoin%
\definecolor{currentfill}{rgb}{0.283091,0.110553,0.431554}%
\pgfsetfillcolor{currentfill}%
\pgfsetfillopacity{0.800000}%
\pgfsetlinewidth{0.000000pt}%
\definecolor{currentstroke}{rgb}{0.000000,0.000000,0.000000}%
\pgfsetstrokecolor{currentstroke}%
\pgfsetdash{}{0pt}%
\pgfpathmoveto{\pgfqpoint{3.957790in}{1.205194in}}%
\pgfpathlineto{\pgfqpoint{3.972070in}{1.207387in}}%
\pgfpathlineto{\pgfqpoint{3.986361in}{1.209754in}}%
\pgfpathlineto{\pgfqpoint{4.000661in}{1.212297in}}%
\pgfpathlineto{\pgfqpoint{4.014972in}{1.215013in}}%
\pgfpathlineto{\pgfqpoint{4.023266in}{1.229063in}}%
\pgfpathlineto{\pgfqpoint{4.031556in}{1.243285in}}%
\pgfpathlineto{\pgfqpoint{4.039840in}{1.257672in}}%
\pgfpathlineto{\pgfqpoint{4.048120in}{1.272219in}}%
\pgfpathlineto{\pgfqpoint{4.033813in}{1.268762in}}%
\pgfpathlineto{\pgfqpoint{4.019517in}{1.265481in}}%
\pgfpathlineto{\pgfqpoint{4.005232in}{1.262376in}}%
\pgfpathlineto{\pgfqpoint{3.990958in}{1.259446in}}%
\pgfpathlineto{\pgfqpoint{3.982674in}{1.245626in}}%
\pgfpathlineto{\pgfqpoint{3.974384in}{1.231973in}}%
\pgfpathlineto{\pgfqpoint{3.966090in}{1.218494in}}%
\pgfpathlineto{\pgfqpoint{3.957790in}{1.205194in}}%
\pgfpathclose%
\pgfusepath{fill}%
\end{pgfscope}%
\begin{pgfscope}%
\pgfpathrectangle{\pgfqpoint{1.150000in}{0.150000in}}{\pgfqpoint{5.700000in}{5.700000in}}%
\pgfusepath{clip}%
\pgfsetbuttcap%
\pgfsetroundjoin%
\definecolor{currentfill}{rgb}{0.130067,0.651384,0.521608}%
\pgfsetfillcolor{currentfill}%
\pgfsetfillopacity{0.800000}%
\pgfsetlinewidth{0.000000pt}%
\definecolor{currentstroke}{rgb}{0.000000,0.000000,0.000000}%
\pgfsetstrokecolor{currentstroke}%
\pgfsetdash{}{0pt}%
\pgfpathmoveto{\pgfqpoint{5.010765in}{2.714384in}}%
\pgfpathlineto{\pgfqpoint{5.025669in}{2.730951in}}%
\pgfpathlineto{\pgfqpoint{5.040595in}{2.747708in}}%
\pgfpathlineto{\pgfqpoint{5.055543in}{2.764655in}}%
\pgfpathlineto{\pgfqpoint{5.070513in}{2.781793in}}%
\pgfpathlineto{\pgfqpoint{5.078529in}{2.796306in}}%
\pgfpathlineto{\pgfqpoint{5.086537in}{2.810618in}}%
\pgfpathlineto{\pgfqpoint{5.094537in}{2.824728in}}%
\pgfpathlineto{\pgfqpoint{5.102530in}{2.838634in}}%
\pgfpathlineto{\pgfqpoint{5.087553in}{2.821334in}}%
\pgfpathlineto{\pgfqpoint{5.072599in}{2.804224in}}%
\pgfpathlineto{\pgfqpoint{5.057667in}{2.787305in}}%
\pgfpathlineto{\pgfqpoint{5.042756in}{2.770577in}}%
\pgfpathlineto{\pgfqpoint{5.034770in}{2.756819in}}%
\pgfpathlineto{\pgfqpoint{5.026776in}{2.742867in}}%
\pgfpathlineto{\pgfqpoint{5.018774in}{2.728722in}}%
\pgfpathlineto{\pgfqpoint{5.010765in}{2.714384in}}%
\pgfpathclose%
\pgfusepath{fill}%
\end{pgfscope}%
\begin{pgfscope}%
\pgfpathrectangle{\pgfqpoint{1.150000in}{0.150000in}}{\pgfqpoint{5.700000in}{5.700000in}}%
\pgfusepath{clip}%
\pgfsetbuttcap%
\pgfsetroundjoin%
\definecolor{currentfill}{rgb}{0.147607,0.511733,0.557049}%
\pgfsetfillcolor{currentfill}%
\pgfsetfillopacity{0.800000}%
\pgfsetlinewidth{0.000000pt}%
\definecolor{currentstroke}{rgb}{0.000000,0.000000,0.000000}%
\pgfsetstrokecolor{currentstroke}%
\pgfsetdash{}{0pt}%
\pgfpathmoveto{\pgfqpoint{4.731019in}{2.274922in}}%
\pgfpathlineto{\pgfqpoint{4.745716in}{2.288652in}}%
\pgfpathlineto{\pgfqpoint{4.760432in}{2.302567in}}%
\pgfpathlineto{\pgfqpoint{4.775167in}{2.316669in}}%
\pgfpathlineto{\pgfqpoint{4.789922in}{2.330956in}}%
\pgfpathlineto{\pgfqpoint{4.798051in}{2.348391in}}%
\pgfpathlineto{\pgfqpoint{4.806173in}{2.365686in}}%
\pgfpathlineto{\pgfqpoint{4.814291in}{2.382837in}}%
\pgfpathlineto{\pgfqpoint{4.822403in}{2.399842in}}%
\pgfpathlineto{\pgfqpoint{4.807637in}{2.385218in}}%
\pgfpathlineto{\pgfqpoint{4.792892in}{2.370781in}}%
\pgfpathlineto{\pgfqpoint{4.778166in}{2.356530in}}%
\pgfpathlineto{\pgfqpoint{4.763459in}{2.342465in}}%
\pgfpathlineto{\pgfqpoint{4.755357in}{2.325783in}}%
\pgfpathlineto{\pgfqpoint{4.747249in}{2.308963in}}%
\pgfpathlineto{\pgfqpoint{4.739137in}{2.292008in}}%
\pgfpathlineto{\pgfqpoint{4.731019in}{2.274922in}}%
\pgfpathclose%
\pgfusepath{fill}%
\end{pgfscope}%
\begin{pgfscope}%
\pgfpathrectangle{\pgfqpoint{1.150000in}{0.150000in}}{\pgfqpoint{5.700000in}{5.700000in}}%
\pgfusepath{clip}%
\pgfsetbuttcap%
\pgfsetroundjoin%
\definecolor{currentfill}{rgb}{0.273006,0.204520,0.501721}%
\pgfsetfillcolor{currentfill}%
\pgfsetfillopacity{0.800000}%
\pgfsetlinewidth{0.000000pt}%
\definecolor{currentstroke}{rgb}{0.000000,0.000000,0.000000}%
\pgfsetstrokecolor{currentstroke}%
\pgfsetdash{}{0pt}%
\pgfpathmoveto{\pgfqpoint{4.171534in}{1.414602in}}%
\pgfpathlineto{\pgfqpoint{4.185899in}{1.420325in}}%
\pgfpathlineto{\pgfqpoint{4.200276in}{1.426225in}}%
\pgfpathlineto{\pgfqpoint{4.214666in}{1.432301in}}%
\pgfpathlineto{\pgfqpoint{4.229070in}{1.438553in}}%
\pgfpathlineto{\pgfqpoint{4.237315in}{1.455543in}}%
\pgfpathlineto{\pgfqpoint{4.245556in}{1.472610in}}%
\pgfpathlineto{\pgfqpoint{4.253793in}{1.489748in}}%
\pgfpathlineto{\pgfqpoint{4.262028in}{1.506950in}}%
\pgfpathlineto{\pgfqpoint{4.247621in}{1.500046in}}%
\pgfpathlineto{\pgfqpoint{4.233227in}{1.493319in}}%
\pgfpathlineto{\pgfqpoint{4.218847in}{1.486769in}}%
\pgfpathlineto{\pgfqpoint{4.204480in}{1.480396in}}%
\pgfpathlineto{\pgfqpoint{4.196249in}{1.463833in}}%
\pgfpathlineto{\pgfqpoint{4.188014in}{1.447342in}}%
\pgfpathlineto{\pgfqpoint{4.179776in}{1.430930in}}%
\pgfpathlineto{\pgfqpoint{4.171534in}{1.414602in}}%
\pgfpathclose%
\pgfusepath{fill}%
\end{pgfscope}%
\begin{pgfscope}%
\pgfpathrectangle{\pgfqpoint{1.150000in}{0.150000in}}{\pgfqpoint{5.700000in}{5.700000in}}%
\pgfusepath{clip}%
\pgfsetbuttcap%
\pgfsetroundjoin%
\definecolor{currentfill}{rgb}{0.250425,0.274290,0.533103}%
\pgfsetfillcolor{currentfill}%
\pgfsetfillopacity{0.800000}%
\pgfsetlinewidth{0.000000pt}%
\definecolor{currentstroke}{rgb}{0.000000,0.000000,0.000000}%
\pgfsetstrokecolor{currentstroke}%
\pgfsetdash{}{0pt}%
\pgfpathmoveto{\pgfqpoint{4.294931in}{1.576286in}}%
\pgfpathlineto{\pgfqpoint{4.309357in}{1.583989in}}%
\pgfpathlineto{\pgfqpoint{4.323797in}{1.591869in}}%
\pgfpathlineto{\pgfqpoint{4.338252in}{1.599928in}}%
\pgfpathlineto{\pgfqpoint{4.352722in}{1.608164in}}%
\pgfpathlineto{\pgfqpoint{4.360945in}{1.626204in}}%
\pgfpathlineto{\pgfqpoint{4.369166in}{1.644265in}}%
\pgfpathlineto{\pgfqpoint{4.377383in}{1.662343in}}%
\pgfpathlineto{\pgfqpoint{4.385597in}{1.680432in}}%
\pgfpathlineto{\pgfqpoint{4.371120in}{1.671603in}}%
\pgfpathlineto{\pgfqpoint{4.356659in}{1.662953in}}%
\pgfpathlineto{\pgfqpoint{4.342212in}{1.654482in}}%
\pgfpathlineto{\pgfqpoint{4.327780in}{1.646189in}}%
\pgfpathlineto{\pgfqpoint{4.319573in}{1.628680in}}%
\pgfpathlineto{\pgfqpoint{4.311362in}{1.611189in}}%
\pgfpathlineto{\pgfqpoint{4.303148in}{1.593722in}}%
\pgfpathlineto{\pgfqpoint{4.294931in}{1.576286in}}%
\pgfpathclose%
\pgfusepath{fill}%
\end{pgfscope}%
\begin{pgfscope}%
\pgfpathrectangle{\pgfqpoint{1.150000in}{0.150000in}}{\pgfqpoint{5.700000in}{5.700000in}}%
\pgfusepath{clip}%
\pgfsetbuttcap%
\pgfsetroundjoin%
\definecolor{currentfill}{rgb}{0.386433,0.794644,0.372886}%
\pgfsetfillcolor{currentfill}%
\pgfsetfillopacity{0.800000}%
\pgfsetlinewidth{0.000000pt}%
\definecolor{currentstroke}{rgb}{0.000000,0.000000,0.000000}%
\pgfsetstrokecolor{currentstroke}%
\pgfsetdash{}{0pt}%
\pgfpathmoveto{\pgfqpoint{5.381250in}{3.220609in}}%
\pgfpathlineto{\pgfqpoint{5.396450in}{3.239910in}}%
\pgfpathlineto{\pgfqpoint{5.411674in}{3.259406in}}%
\pgfpathlineto{\pgfqpoint{5.426924in}{3.279097in}}%
\pgfpathlineto{\pgfqpoint{5.442198in}{3.298983in}}%
\pgfpathlineto{\pgfqpoint{5.449992in}{3.308393in}}%
\pgfpathlineto{\pgfqpoint{5.457774in}{3.317570in}}%
\pgfpathlineto{\pgfqpoint{5.465545in}{3.326516in}}%
\pgfpathlineto{\pgfqpoint{5.473304in}{3.335231in}}%
\pgfpathlineto{\pgfqpoint{5.458032in}{3.315404in}}%
\pgfpathlineto{\pgfqpoint{5.442785in}{3.295772in}}%
\pgfpathlineto{\pgfqpoint{5.427564in}{3.276334in}}%
\pgfpathlineto{\pgfqpoint{5.412366in}{3.257091in}}%
\pgfpathlineto{\pgfqpoint{5.404604in}{3.248303in}}%
\pgfpathlineto{\pgfqpoint{5.396830in}{3.239294in}}%
\pgfpathlineto{\pgfqpoint{5.389045in}{3.230063in}}%
\pgfpathlineto{\pgfqpoint{5.381250in}{3.220609in}}%
\pgfpathclose%
\pgfusepath{fill}%
\end{pgfscope}%
\begin{pgfscope}%
\pgfpathrectangle{\pgfqpoint{1.150000in}{0.150000in}}{\pgfqpoint{5.700000in}{5.700000in}}%
\pgfusepath{clip}%
\pgfsetbuttcap%
\pgfsetroundjoin%
\definecolor{currentfill}{rgb}{0.185783,0.704891,0.485273}%
\pgfsetfillcolor{currentfill}%
\pgfsetfillopacity{0.800000}%
\pgfsetlinewidth{0.000000pt}%
\definecolor{currentstroke}{rgb}{0.000000,0.000000,0.000000}%
\pgfsetstrokecolor{currentstroke}%
\pgfsetdash{}{0pt}%
\pgfpathmoveto{\pgfqpoint{5.134414in}{2.892207in}}%
\pgfpathlineto{\pgfqpoint{5.149419in}{2.909826in}}%
\pgfpathlineto{\pgfqpoint{5.164446in}{2.927636in}}%
\pgfpathlineto{\pgfqpoint{5.179496in}{2.945638in}}%
\pgfpathlineto{\pgfqpoint{5.194569in}{2.963832in}}%
\pgfpathlineto{\pgfqpoint{5.202524in}{2.976812in}}%
\pgfpathlineto{\pgfqpoint{5.210469in}{2.989575in}}%
\pgfpathlineto{\pgfqpoint{5.218405in}{3.002120in}}%
\pgfpathlineto{\pgfqpoint{5.226332in}{3.014447in}}%
\pgfpathlineto{\pgfqpoint{5.211254in}{2.996162in}}%
\pgfpathlineto{\pgfqpoint{5.196200in}{2.978070in}}%
\pgfpathlineto{\pgfqpoint{5.181169in}{2.960170in}}%
\pgfpathlineto{\pgfqpoint{5.166161in}{2.942462in}}%
\pgfpathlineto{\pgfqpoint{5.158237in}{2.930211in}}%
\pgfpathlineto{\pgfqpoint{5.150305in}{2.917752in}}%
\pgfpathlineto{\pgfqpoint{5.142364in}{2.905084in}}%
\pgfpathlineto{\pgfqpoint{5.134414in}{2.892207in}}%
\pgfpathclose%
\pgfusepath{fill}%
\end{pgfscope}%
\begin{pgfscope}%
\pgfpathrectangle{\pgfqpoint{1.150000in}{0.150000in}}{\pgfqpoint{5.700000in}{5.700000in}}%
\pgfusepath{clip}%
\pgfsetbuttcap%
\pgfsetroundjoin%
\definecolor{currentfill}{rgb}{0.180629,0.429975,0.557282}%
\pgfsetfillcolor{currentfill}%
\pgfsetfillopacity{0.800000}%
\pgfsetlinewidth{0.000000pt}%
\definecolor{currentstroke}{rgb}{0.000000,0.000000,0.000000}%
\pgfsetstrokecolor{currentstroke}%
\pgfsetdash{}{0pt}%
\pgfpathmoveto{\pgfqpoint{4.574755in}{2.012526in}}%
\pgfpathlineto{\pgfqpoint{4.589348in}{2.024314in}}%
\pgfpathlineto{\pgfqpoint{4.603959in}{2.036286in}}%
\pgfpathlineto{\pgfqpoint{4.618588in}{2.048440in}}%
\pgfpathlineto{\pgfqpoint{4.633235in}{2.060778in}}%
\pgfpathlineto{\pgfqpoint{4.641408in}{2.079187in}}%
\pgfpathlineto{\pgfqpoint{4.649577in}{2.097507in}}%
\pgfpathlineto{\pgfqpoint{4.657741in}{2.115733in}}%
\pgfpathlineto{\pgfqpoint{4.665901in}{2.133860in}}%
\pgfpathlineto{\pgfqpoint{4.651244in}{2.121087in}}%
\pgfpathlineto{\pgfqpoint{4.636604in}{2.108497in}}%
\pgfpathlineto{\pgfqpoint{4.621983in}{2.096091in}}%
\pgfpathlineto{\pgfqpoint{4.607379in}{2.083868in}}%
\pgfpathlineto{\pgfqpoint{4.599229in}{2.066163in}}%
\pgfpathlineto{\pgfqpoint{4.591075in}{2.048368in}}%
\pgfpathlineto{\pgfqpoint{4.582917in}{2.030488in}}%
\pgfpathlineto{\pgfqpoint{4.574755in}{2.012526in}}%
\pgfpathclose%
\pgfusepath{fill}%
\end{pgfscope}%
\begin{pgfscope}%
\pgfpathrectangle{\pgfqpoint{1.150000in}{0.150000in}}{\pgfqpoint{5.700000in}{5.700000in}}%
\pgfusepath{clip}%
\pgfsetbuttcap%
\pgfsetroundjoin%
\definecolor{currentfill}{rgb}{0.281477,0.755203,0.432552}%
\pgfsetfillcolor{currentfill}%
\pgfsetfillopacity{0.800000}%
\pgfsetlinewidth{0.000000pt}%
\definecolor{currentstroke}{rgb}{0.000000,0.000000,0.000000}%
\pgfsetstrokecolor{currentstroke}%
\pgfsetdash{}{0pt}%
\pgfpathmoveto{\pgfqpoint{5.257944in}{3.061569in}}%
\pgfpathlineto{\pgfqpoint{5.273047in}{3.080100in}}%
\pgfpathlineto{\pgfqpoint{5.288174in}{3.098824in}}%
\pgfpathlineto{\pgfqpoint{5.303325in}{3.117742in}}%
\pgfpathlineto{\pgfqpoint{5.318501in}{3.136854in}}%
\pgfpathlineto{\pgfqpoint{5.326381in}{3.148118in}}%
\pgfpathlineto{\pgfqpoint{5.334251in}{3.159154in}}%
\pgfpathlineto{\pgfqpoint{5.342111in}{3.169963in}}%
\pgfpathlineto{\pgfqpoint{5.349960in}{3.180544in}}%
\pgfpathlineto{\pgfqpoint{5.334783in}{3.161416in}}%
\pgfpathlineto{\pgfqpoint{5.319631in}{3.142482in}}%
\pgfpathlineto{\pgfqpoint{5.304503in}{3.123741in}}%
\pgfpathlineto{\pgfqpoint{5.289399in}{3.105193in}}%
\pgfpathlineto{\pgfqpoint{5.281550in}{3.094614in}}%
\pgfpathlineto{\pgfqpoint{5.273691in}{3.083818in}}%
\pgfpathlineto{\pgfqpoint{5.265822in}{3.072803in}}%
\pgfpathlineto{\pgfqpoint{5.257944in}{3.061569in}}%
\pgfpathclose%
\pgfusepath{fill}%
\end{pgfscope}%
\begin{pgfscope}%
\pgfpathrectangle{\pgfqpoint{1.150000in}{0.150000in}}{\pgfqpoint{5.700000in}{5.700000in}}%
\pgfusepath{clip}%
\pgfsetbuttcap%
\pgfsetroundjoin%
\definecolor{currentfill}{rgb}{0.282290,0.145912,0.461510}%
\pgfsetfillcolor{currentfill}%
\pgfsetfillopacity{0.800000}%
\pgfsetlinewidth{0.000000pt}%
\definecolor{currentstroke}{rgb}{0.000000,0.000000,0.000000}%
\pgfsetstrokecolor{currentstroke}%
\pgfsetdash{}{0pt}%
\pgfpathmoveto{\pgfqpoint{4.048120in}{1.272219in}}%
\pgfpathlineto{\pgfqpoint{4.062438in}{1.275850in}}%
\pgfpathlineto{\pgfqpoint{4.076767in}{1.279656in}}%
\pgfpathlineto{\pgfqpoint{4.091107in}{1.283637in}}%
\pgfpathlineto{\pgfqpoint{4.105459in}{1.287793in}}%
\pgfpathlineto{\pgfqpoint{4.113733in}{1.303212in}}%
\pgfpathlineto{\pgfqpoint{4.122002in}{1.318768in}}%
\pgfpathlineto{\pgfqpoint{4.130267in}{1.334455in}}%
\pgfpathlineto{\pgfqpoint{4.138528in}{1.350264in}}%
\pgfpathlineto{\pgfqpoint{4.124177in}{1.345397in}}%
\pgfpathlineto{\pgfqpoint{4.109838in}{1.340705in}}%
\pgfpathlineto{\pgfqpoint{4.095510in}{1.336189in}}%
\pgfpathlineto{\pgfqpoint{4.081194in}{1.331849in}}%
\pgfpathlineto{\pgfqpoint{4.072932in}{1.316738in}}%
\pgfpathlineto{\pgfqpoint{4.064666in}{1.301758in}}%
\pgfpathlineto{\pgfqpoint{4.056395in}{1.286916in}}%
\pgfpathlineto{\pgfqpoint{4.048120in}{1.272219in}}%
\pgfpathclose%
\pgfusepath{fill}%
\end{pgfscope}%
\begin{pgfscope}%
\pgfpathrectangle{\pgfqpoint{1.150000in}{0.150000in}}{\pgfqpoint{5.700000in}{5.700000in}}%
\pgfusepath{clip}%
\pgfsetbuttcap%
\pgfsetroundjoin%
\definecolor{currentfill}{rgb}{0.220057,0.343307,0.549413}%
\pgfsetfillcolor{currentfill}%
\pgfsetfillopacity{0.800000}%
\pgfsetlinewidth{0.000000pt}%
\definecolor{currentstroke}{rgb}{0.000000,0.000000,0.000000}%
\pgfsetstrokecolor{currentstroke}%
\pgfsetdash{}{0pt}%
\pgfpathmoveto{\pgfqpoint{4.418419in}{1.752790in}}%
\pgfpathlineto{\pgfqpoint{4.432919in}{1.762359in}}%
\pgfpathlineto{\pgfqpoint{4.447434in}{1.772108in}}%
\pgfpathlineto{\pgfqpoint{4.461966in}{1.782037in}}%
\pgfpathlineto{\pgfqpoint{4.476513in}{1.792145in}}%
\pgfpathlineto{\pgfqpoint{4.484720in}{1.810752in}}%
\pgfpathlineto{\pgfqpoint{4.492922in}{1.829331in}}%
\pgfpathlineto{\pgfqpoint{4.501122in}{1.847876in}}%
\pgfpathlineto{\pgfqpoint{4.509318in}{1.866384in}}%
\pgfpathlineto{\pgfqpoint{4.494761in}{1.855743in}}%
\pgfpathlineto{\pgfqpoint{4.480220in}{1.845284in}}%
\pgfpathlineto{\pgfqpoint{4.465696in}{1.835005in}}%
\pgfpathlineto{\pgfqpoint{4.451188in}{1.824906in}}%
\pgfpathlineto{\pgfqpoint{4.443001in}{1.806917in}}%
\pgfpathlineto{\pgfqpoint{4.434811in}{1.788898in}}%
\pgfpathlineto{\pgfqpoint{4.426617in}{1.770854in}}%
\pgfpathlineto{\pgfqpoint{4.418419in}{1.752790in}}%
\pgfpathclose%
\pgfusepath{fill}%
\end{pgfscope}%
\begin{pgfscope}%
\pgfpathrectangle{\pgfqpoint{1.150000in}{0.150000in}}{\pgfqpoint{5.700000in}{5.700000in}}%
\pgfusepath{clip}%
\pgfsetbuttcap%
\pgfsetroundjoin%
\definecolor{currentfill}{rgb}{0.124395,0.578002,0.548287}%
\pgfsetfillcolor{currentfill}%
\pgfsetfillopacity{0.800000}%
\pgfsetlinewidth{0.000000pt}%
\definecolor{currentstroke}{rgb}{0.000000,0.000000,0.000000}%
\pgfsetstrokecolor{currentstroke}%
\pgfsetdash{}{0pt}%
\pgfpathmoveto{\pgfqpoint{4.854793in}{2.466344in}}%
\pgfpathlineto{\pgfqpoint{4.869589in}{2.481457in}}%
\pgfpathlineto{\pgfqpoint{4.884404in}{2.496759in}}%
\pgfpathlineto{\pgfqpoint{4.899241in}{2.512248in}}%
\pgfpathlineto{\pgfqpoint{4.914097in}{2.527925in}}%
\pgfpathlineto{\pgfqpoint{4.922190in}{2.544439in}}%
\pgfpathlineto{\pgfqpoint{4.930276in}{2.560783in}}%
\pgfpathlineto{\pgfqpoint{4.938356in}{2.576954in}}%
\pgfpathlineto{\pgfqpoint{4.946429in}{2.592950in}}%
\pgfpathlineto{\pgfqpoint{4.931563in}{2.577004in}}%
\pgfpathlineto{\pgfqpoint{4.916717in}{2.561247in}}%
\pgfpathlineto{\pgfqpoint{4.901892in}{2.545678in}}%
\pgfpathlineto{\pgfqpoint{4.887088in}{2.530297in}}%
\pgfpathlineto{\pgfqpoint{4.879024in}{2.514556in}}%
\pgfpathlineto{\pgfqpoint{4.870953in}{2.498648in}}%
\pgfpathlineto{\pgfqpoint{4.862876in}{2.482577in}}%
\pgfpathlineto{\pgfqpoint{4.854793in}{2.466344in}}%
\pgfpathclose%
\pgfusepath{fill}%
\end{pgfscope}%
\begin{pgfscope}%
\pgfpathrectangle{\pgfqpoint{1.150000in}{0.150000in}}{\pgfqpoint{5.700000in}{5.700000in}}%
\pgfusepath{clip}%
\pgfsetbuttcap%
\pgfsetroundjoin%
\definecolor{currentfill}{rgb}{0.595839,0.848717,0.243329}%
\pgfsetfillcolor{currentfill}%
\pgfsetfillopacity{0.800000}%
\pgfsetlinewidth{0.000000pt}%
\definecolor{currentstroke}{rgb}{0.000000,0.000000,0.000000}%
\pgfsetstrokecolor{currentstroke}%
\pgfsetdash{}{0pt}%
\pgfpathmoveto{\pgfqpoint{5.596250in}{3.477171in}}%
\pgfpathlineto{\pgfqpoint{5.611638in}{3.497746in}}%
\pgfpathlineto{\pgfqpoint{5.627052in}{3.518517in}}%
\pgfpathlineto{\pgfqpoint{5.642492in}{3.539485in}}%
\pgfpathlineto{\pgfqpoint{5.650131in}{3.545917in}}%
\pgfpathlineto{\pgfqpoint{5.657756in}{3.552118in}}%
\pgfpathlineto{\pgfqpoint{5.665369in}{3.558091in}}%
\pgfpathlineto{\pgfqpoint{5.672969in}{3.563838in}}%
\pgfpathlineto{\pgfqpoint{5.657538in}{3.543043in}}%
\pgfpathlineto{\pgfqpoint{5.642133in}{3.522445in}}%
\pgfpathlineto{\pgfqpoint{5.626755in}{3.502043in}}%
\pgfpathlineto{\pgfqpoint{5.619147in}{3.496156in}}%
\pgfpathlineto{\pgfqpoint{5.611527in}{3.490049in}}%
\pgfpathlineto{\pgfqpoint{5.603895in}{3.483722in}}%
\pgfpathlineto{\pgfqpoint{5.596250in}{3.477171in}}%
\pgfpathclose%
\pgfusepath{fill}%
\end{pgfscope}%
\begin{pgfscope}%
\pgfpathrectangle{\pgfqpoint{1.150000in}{0.150000in}}{\pgfqpoint{5.700000in}{5.700000in}}%
\pgfusepath{clip}%
\pgfsetbuttcap%
\pgfsetroundjoin%
\definecolor{currentfill}{rgb}{0.153364,0.497000,0.557724}%
\pgfsetfillcolor{currentfill}%
\pgfsetfillopacity{0.800000}%
\pgfsetlinewidth{0.000000pt}%
\definecolor{currentstroke}{rgb}{0.000000,0.000000,0.000000}%
\pgfsetstrokecolor{currentstroke}%
\pgfsetdash{}{0pt}%
\pgfpathmoveto{\pgfqpoint{4.698498in}{2.205318in}}%
\pgfpathlineto{\pgfqpoint{4.713184in}{2.218679in}}%
\pgfpathlineto{\pgfqpoint{4.727890in}{2.232224in}}%
\pgfpathlineto{\pgfqpoint{4.742614in}{2.245955in}}%
\pgfpathlineto{\pgfqpoint{4.757358in}{2.259871in}}%
\pgfpathlineto{\pgfqpoint{4.765507in}{2.277838in}}%
\pgfpathlineto{\pgfqpoint{4.773650in}{2.295676in}}%
\pgfpathlineto{\pgfqpoint{4.781789in}{2.313383in}}%
\pgfpathlineto{\pgfqpoint{4.789922in}{2.330956in}}%
\pgfpathlineto{\pgfqpoint{4.775167in}{2.316669in}}%
\pgfpathlineto{\pgfqpoint{4.760432in}{2.302567in}}%
\pgfpathlineto{\pgfqpoint{4.745716in}{2.288652in}}%
\pgfpathlineto{\pgfqpoint{4.731019in}{2.274922in}}%
\pgfpathlineto{\pgfqpoint{4.722896in}{2.257706in}}%
\pgfpathlineto{\pgfqpoint{4.714768in}{2.240365in}}%
\pgfpathlineto{\pgfqpoint{4.706635in}{2.222901in}}%
\pgfpathlineto{\pgfqpoint{4.698498in}{2.205318in}}%
\pgfpathclose%
\pgfusepath{fill}%
\end{pgfscope}%
\begin{pgfscope}%
\pgfpathrectangle{\pgfqpoint{1.150000in}{0.150000in}}{\pgfqpoint{5.700000in}{5.700000in}}%
\pgfusepath{clip}%
\pgfsetbuttcap%
\pgfsetroundjoin%
\definecolor{currentfill}{rgb}{0.279566,0.067836,0.391917}%
\pgfsetfillcolor{currentfill}%
\pgfsetfillopacity{0.800000}%
\pgfsetlinewidth{0.000000pt}%
\definecolor{currentstroke}{rgb}{0.000000,0.000000,0.000000}%
\pgfsetstrokecolor{currentstroke}%
\pgfsetdash{}{0pt}%
\pgfpathmoveto{\pgfqpoint{3.834084in}{1.105430in}}%
\pgfpathlineto{\pgfqpoint{3.848347in}{1.105357in}}%
\pgfpathlineto{\pgfqpoint{3.862618in}{1.105459in}}%
\pgfpathlineto{\pgfqpoint{3.876898in}{1.105734in}}%
\pgfpathlineto{\pgfqpoint{3.891186in}{1.106184in}}%
\pgfpathlineto{\pgfqpoint{3.899533in}{1.117772in}}%
\pgfpathlineto{\pgfqpoint{3.907873in}{1.129600in}}%
\pgfpathlineto{\pgfqpoint{3.916208in}{1.141661in}}%
\pgfpathlineto{\pgfqpoint{3.924536in}{1.153948in}}%
\pgfpathlineto{\pgfqpoint{3.910257in}{1.152698in}}%
\pgfpathlineto{\pgfqpoint{3.895988in}{1.151623in}}%
\pgfpathlineto{\pgfqpoint{3.881728in}{1.150723in}}%
\pgfpathlineto{\pgfqpoint{3.867478in}{1.149998in}}%
\pgfpathlineto{\pgfqpoint{3.859139in}{1.138498in}}%
\pgfpathlineto{\pgfqpoint{3.850795in}{1.127232in}}%
\pgfpathlineto{\pgfqpoint{3.842443in}{1.116207in}}%
\pgfpathlineto{\pgfqpoint{3.834084in}{1.105430in}}%
\pgfpathclose%
\pgfusepath{fill}%
\end{pgfscope}%
\begin{pgfscope}%
\pgfpathrectangle{\pgfqpoint{1.150000in}{0.150000in}}{\pgfqpoint{5.700000in}{5.700000in}}%
\pgfusepath{clip}%
\pgfsetbuttcap%
\pgfsetroundjoin%
\definecolor{currentfill}{rgb}{0.277134,0.185228,0.489898}%
\pgfsetfillcolor{currentfill}%
\pgfsetfillopacity{0.800000}%
\pgfsetlinewidth{0.000000pt}%
\definecolor{currentstroke}{rgb}{0.000000,0.000000,0.000000}%
\pgfsetstrokecolor{currentstroke}%
\pgfsetdash{}{0pt}%
\pgfpathmoveto{\pgfqpoint{4.138528in}{1.350264in}}%
\pgfpathlineto{\pgfqpoint{4.152892in}{1.355306in}}%
\pgfpathlineto{\pgfqpoint{4.167267in}{1.360525in}}%
\pgfpathlineto{\pgfqpoint{4.181656in}{1.365918in}}%
\pgfpathlineto{\pgfqpoint{4.196057in}{1.371487in}}%
\pgfpathlineto{\pgfqpoint{4.204315in}{1.388107in}}%
\pgfpathlineto{\pgfqpoint{4.212570in}{1.404829in}}%
\pgfpathlineto{\pgfqpoint{4.220822in}{1.421646in}}%
\pgfpathlineto{\pgfqpoint{4.229070in}{1.438553in}}%
\pgfpathlineto{\pgfqpoint{4.214666in}{1.432301in}}%
\pgfpathlineto{\pgfqpoint{4.200276in}{1.426225in}}%
\pgfpathlineto{\pgfqpoint{4.185899in}{1.420325in}}%
\pgfpathlineto{\pgfqpoint{4.171534in}{1.414602in}}%
\pgfpathlineto{\pgfqpoint{4.163288in}{1.398365in}}%
\pgfpathlineto{\pgfqpoint{4.155039in}{1.382226in}}%
\pgfpathlineto{\pgfqpoint{4.146785in}{1.366190in}}%
\pgfpathlineto{\pgfqpoint{4.138528in}{1.350264in}}%
\pgfpathclose%
\pgfusepath{fill}%
\end{pgfscope}%
\begin{pgfscope}%
\pgfpathrectangle{\pgfqpoint{1.150000in}{0.150000in}}{\pgfqpoint{5.700000in}{5.700000in}}%
\pgfusepath{clip}%
\pgfsetbuttcap%
\pgfsetroundjoin%
\definecolor{currentfill}{rgb}{0.258965,0.251537,0.524736}%
\pgfsetfillcolor{currentfill}%
\pgfsetfillopacity{0.800000}%
\pgfsetlinewidth{0.000000pt}%
\definecolor{currentstroke}{rgb}{0.000000,0.000000,0.000000}%
\pgfsetstrokecolor{currentstroke}%
\pgfsetdash{}{0pt}%
\pgfpathmoveto{\pgfqpoint{4.262028in}{1.506950in}}%
\pgfpathlineto{\pgfqpoint{4.276449in}{1.514031in}}%
\pgfpathlineto{\pgfqpoint{4.290883in}{1.521290in}}%
\pgfpathlineto{\pgfqpoint{4.305332in}{1.528725in}}%
\pgfpathlineto{\pgfqpoint{4.319795in}{1.536337in}}%
\pgfpathlineto{\pgfqpoint{4.328032in}{1.554233in}}%
\pgfpathlineto{\pgfqpoint{4.336265in}{1.572173in}}%
\pgfpathlineto{\pgfqpoint{4.344495in}{1.590152in}}%
\pgfpathlineto{\pgfqpoint{4.352722in}{1.608164in}}%
\pgfpathlineto{\pgfqpoint{4.338252in}{1.599928in}}%
\pgfpathlineto{\pgfqpoint{4.323797in}{1.591869in}}%
\pgfpathlineto{\pgfqpoint{4.309357in}{1.583989in}}%
\pgfpathlineto{\pgfqpoint{4.294931in}{1.576286in}}%
\pgfpathlineto{\pgfqpoint{4.286710in}{1.558885in}}%
\pgfpathlineto{\pgfqpoint{4.278486in}{1.541524in}}%
\pgfpathlineto{\pgfqpoint{4.270259in}{1.524211in}}%
\pgfpathlineto{\pgfqpoint{4.262028in}{1.506950in}}%
\pgfpathclose%
\pgfusepath{fill}%
\end{pgfscope}%
\begin{pgfscope}%
\pgfpathrectangle{\pgfqpoint{1.150000in}{0.150000in}}{\pgfqpoint{5.700000in}{5.700000in}}%
\pgfusepath{clip}%
\pgfsetbuttcap%
\pgfsetroundjoin%
\definecolor{currentfill}{rgb}{0.124780,0.640461,0.527068}%
\pgfsetfillcolor{currentfill}%
\pgfsetfillopacity{0.800000}%
\pgfsetlinewidth{0.000000pt}%
\definecolor{currentstroke}{rgb}{0.000000,0.000000,0.000000}%
\pgfsetstrokecolor{currentstroke}%
\pgfsetdash{}{0pt}%
\pgfpathmoveto{\pgfqpoint{4.978654in}{2.655143in}}%
\pgfpathlineto{\pgfqpoint{4.993551in}{2.671512in}}%
\pgfpathlineto{\pgfqpoint{5.008469in}{2.688071in}}%
\pgfpathlineto{\pgfqpoint{5.023408in}{2.704819in}}%
\pgfpathlineto{\pgfqpoint{5.038370in}{2.721758in}}%
\pgfpathlineto{\pgfqpoint{5.046417in}{2.737061in}}%
\pgfpathlineto{\pgfqpoint{5.054457in}{2.752169in}}%
\pgfpathlineto{\pgfqpoint{5.062488in}{2.767080in}}%
\pgfpathlineto{\pgfqpoint{5.070513in}{2.781793in}}%
\pgfpathlineto{\pgfqpoint{5.055543in}{2.764655in}}%
\pgfpathlineto{\pgfqpoint{5.040595in}{2.747708in}}%
\pgfpathlineto{\pgfqpoint{5.025669in}{2.730951in}}%
\pgfpathlineto{\pgfqpoint{5.010765in}{2.714384in}}%
\pgfpathlineto{\pgfqpoint{5.002748in}{2.699857in}}%
\pgfpathlineto{\pgfqpoint{4.994724in}{2.685139in}}%
\pgfpathlineto{\pgfqpoint{4.986693in}{2.670235in}}%
\pgfpathlineto{\pgfqpoint{4.978654in}{2.655143in}}%
\pgfpathclose%
\pgfusepath{fill}%
\end{pgfscope}%
\begin{pgfscope}%
\pgfpathrectangle{\pgfqpoint{1.150000in}{0.150000in}}{\pgfqpoint{5.700000in}{5.700000in}}%
\pgfusepath{clip}%
\pgfsetbuttcap%
\pgfsetroundjoin%
\definecolor{currentfill}{rgb}{0.282327,0.094955,0.417331}%
\pgfsetfillcolor{currentfill}%
\pgfsetfillopacity{0.800000}%
\pgfsetlinewidth{0.000000pt}%
\definecolor{currentstroke}{rgb}{0.000000,0.000000,0.000000}%
\pgfsetstrokecolor{currentstroke}%
\pgfsetdash{}{0pt}%
\pgfpathmoveto{\pgfqpoint{3.924536in}{1.153948in}}%
\pgfpathlineto{\pgfqpoint{3.938823in}{1.155372in}}%
\pgfpathlineto{\pgfqpoint{3.953121in}{1.156970in}}%
\pgfpathlineto{\pgfqpoint{3.967428in}{1.158742in}}%
\pgfpathlineto{\pgfqpoint{3.981745in}{1.160688in}}%
\pgfpathlineto{\pgfqpoint{3.990059in}{1.173974in}}%
\pgfpathlineto{\pgfqpoint{3.998369in}{1.187461in}}%
\pgfpathlineto{\pgfqpoint{4.006673in}{1.201144in}}%
\pgfpathlineto{\pgfqpoint{4.014972in}{1.215013in}}%
\pgfpathlineto{\pgfqpoint{4.000661in}{1.212297in}}%
\pgfpathlineto{\pgfqpoint{3.986361in}{1.209754in}}%
\pgfpathlineto{\pgfqpoint{3.972070in}{1.207387in}}%
\pgfpathlineto{\pgfqpoint{3.957790in}{1.205194in}}%
\pgfpathlineto{\pgfqpoint{3.949485in}{1.192083in}}%
\pgfpathlineto{\pgfqpoint{3.941174in}{1.179166in}}%
\pgfpathlineto{\pgfqpoint{3.932858in}{1.166452in}}%
\pgfpathlineto{\pgfqpoint{3.924536in}{1.153948in}}%
\pgfpathclose%
\pgfusepath{fill}%
\end{pgfscope}%
\begin{pgfscope}%
\pgfpathrectangle{\pgfqpoint{1.150000in}{0.150000in}}{\pgfqpoint{5.700000in}{5.700000in}}%
\pgfusepath{clip}%
\pgfsetbuttcap%
\pgfsetroundjoin%
\definecolor{currentfill}{rgb}{0.188923,0.410910,0.556326}%
\pgfsetfillcolor{currentfill}%
\pgfsetfillopacity{0.800000}%
\pgfsetlinewidth{0.000000pt}%
\definecolor{currentstroke}{rgb}{0.000000,0.000000,0.000000}%
\pgfsetstrokecolor{currentstroke}%
\pgfsetdash{}{0pt}%
\pgfpathmoveto{\pgfqpoint{4.542066in}{1.939941in}}%
\pgfpathlineto{\pgfqpoint{4.556650in}{1.951263in}}%
\pgfpathlineto{\pgfqpoint{4.571250in}{1.962766in}}%
\pgfpathlineto{\pgfqpoint{4.585869in}{1.974452in}}%
\pgfpathlineto{\pgfqpoint{4.600504in}{1.986319in}}%
\pgfpathlineto{\pgfqpoint{4.608693in}{2.005048in}}%
\pgfpathlineto{\pgfqpoint{4.616878in}{2.023704in}}%
\pgfpathlineto{\pgfqpoint{4.625058in}{2.042282in}}%
\pgfpathlineto{\pgfqpoint{4.633235in}{2.060778in}}%
\pgfpathlineto{\pgfqpoint{4.618588in}{2.048440in}}%
\pgfpathlineto{\pgfqpoint{4.603959in}{2.036286in}}%
\pgfpathlineto{\pgfqpoint{4.589348in}{2.024314in}}%
\pgfpathlineto{\pgfqpoint{4.574755in}{2.012526in}}%
\pgfpathlineto{\pgfqpoint{4.566588in}{1.994486in}}%
\pgfpathlineto{\pgfqpoint{4.558418in}{1.976372in}}%
\pgfpathlineto{\pgfqpoint{4.550244in}{1.958189in}}%
\pgfpathlineto{\pgfqpoint{4.542066in}{1.939941in}}%
\pgfpathclose%
\pgfusepath{fill}%
\end{pgfscope}%
\begin{pgfscope}%
\pgfpathrectangle{\pgfqpoint{1.150000in}{0.150000in}}{\pgfqpoint{5.700000in}{5.700000in}}%
\pgfusepath{clip}%
\pgfsetbuttcap%
\pgfsetroundjoin%
\definecolor{currentfill}{rgb}{0.229739,0.322361,0.545706}%
\pgfsetfillcolor{currentfill}%
\pgfsetfillopacity{0.800000}%
\pgfsetlinewidth{0.000000pt}%
\definecolor{currentstroke}{rgb}{0.000000,0.000000,0.000000}%
\pgfsetstrokecolor{currentstroke}%
\pgfsetdash{}{0pt}%
\pgfpathmoveto{\pgfqpoint{4.385597in}{1.680432in}}%
\pgfpathlineto{\pgfqpoint{4.400088in}{1.689440in}}%
\pgfpathlineto{\pgfqpoint{4.414595in}{1.698626in}}%
\pgfpathlineto{\pgfqpoint{4.429118in}{1.707992in}}%
\pgfpathlineto{\pgfqpoint{4.443656in}{1.717537in}}%
\pgfpathlineto{\pgfqpoint{4.451875in}{1.736206in}}%
\pgfpathlineto{\pgfqpoint{4.460091in}{1.754867in}}%
\pgfpathlineto{\pgfqpoint{4.468304in}{1.773515in}}%
\pgfpathlineto{\pgfqpoint{4.476513in}{1.792145in}}%
\pgfpathlineto{\pgfqpoint{4.461966in}{1.782037in}}%
\pgfpathlineto{\pgfqpoint{4.447434in}{1.772108in}}%
\pgfpathlineto{\pgfqpoint{4.432919in}{1.762359in}}%
\pgfpathlineto{\pgfqpoint{4.418419in}{1.752790in}}%
\pgfpathlineto{\pgfqpoint{4.410219in}{1.734710in}}%
\pgfpathlineto{\pgfqpoint{4.402015in}{1.716621in}}%
\pgfpathlineto{\pgfqpoint{4.393807in}{1.698526in}}%
\pgfpathlineto{\pgfqpoint{4.385597in}{1.680432in}}%
\pgfpathclose%
\pgfusepath{fill}%
\end{pgfscope}%
\begin{pgfscope}%
\pgfpathrectangle{\pgfqpoint{1.150000in}{0.150000in}}{\pgfqpoint{5.700000in}{5.700000in}}%
\pgfusepath{clip}%
\pgfsetbuttcap%
\pgfsetroundjoin%
\definecolor{currentfill}{rgb}{0.283187,0.125848,0.444960}%
\pgfsetfillcolor{currentfill}%
\pgfsetfillopacity{0.800000}%
\pgfsetlinewidth{0.000000pt}%
\definecolor{currentstroke}{rgb}{0.000000,0.000000,0.000000}%
\pgfsetstrokecolor{currentstroke}%
\pgfsetdash{}{0pt}%
\pgfpathmoveto{\pgfqpoint{4.014972in}{1.215013in}}%
\pgfpathlineto{\pgfqpoint{4.029294in}{1.217904in}}%
\pgfpathlineto{\pgfqpoint{4.043626in}{1.220970in}}%
\pgfpathlineto{\pgfqpoint{4.057969in}{1.224210in}}%
\pgfpathlineto{\pgfqpoint{4.072323in}{1.227624in}}%
\pgfpathlineto{\pgfqpoint{4.080614in}{1.242426in}}%
\pgfpathlineto{\pgfqpoint{4.088900in}{1.257393in}}%
\pgfpathlineto{\pgfqpoint{4.097182in}{1.272518in}}%
\pgfpathlineto{\pgfqpoint{4.105459in}{1.287793in}}%
\pgfpathlineto{\pgfqpoint{4.091107in}{1.283637in}}%
\pgfpathlineto{\pgfqpoint{4.076767in}{1.279656in}}%
\pgfpathlineto{\pgfqpoint{4.062438in}{1.275850in}}%
\pgfpathlineto{\pgfqpoint{4.048120in}{1.272219in}}%
\pgfpathlineto{\pgfqpoint{4.039840in}{1.257672in}}%
\pgfpathlineto{\pgfqpoint{4.031556in}{1.243285in}}%
\pgfpathlineto{\pgfqpoint{4.023266in}{1.229063in}}%
\pgfpathlineto{\pgfqpoint{4.014972in}{1.215013in}}%
\pgfpathclose%
\pgfusepath{fill}%
\end{pgfscope}%
\begin{pgfscope}%
\pgfpathrectangle{\pgfqpoint{1.150000in}{0.150000in}}{\pgfqpoint{5.700000in}{5.700000in}}%
\pgfusepath{clip}%
\pgfsetbuttcap%
\pgfsetroundjoin%
\definecolor{currentfill}{rgb}{0.175707,0.697900,0.491033}%
\pgfsetfillcolor{currentfill}%
\pgfsetfillopacity{0.800000}%
\pgfsetlinewidth{0.000000pt}%
\definecolor{currentstroke}{rgb}{0.000000,0.000000,0.000000}%
\pgfsetstrokecolor{currentstroke}%
\pgfsetdash{}{0pt}%
\pgfpathmoveto{\pgfqpoint{5.102530in}{2.838634in}}%
\pgfpathlineto{\pgfqpoint{5.117528in}{2.856126in}}%
\pgfpathlineto{\pgfqpoint{5.132550in}{2.873809in}}%
\pgfpathlineto{\pgfqpoint{5.147594in}{2.891684in}}%
\pgfpathlineto{\pgfqpoint{5.162662in}{2.909752in}}%
\pgfpathlineto{\pgfqpoint{5.170652in}{2.923594in}}%
\pgfpathlineto{\pgfqpoint{5.178633in}{2.937222in}}%
\pgfpathlineto{\pgfqpoint{5.186606in}{2.950635in}}%
\pgfpathlineto{\pgfqpoint{5.194569in}{2.963832in}}%
\pgfpathlineto{\pgfqpoint{5.179496in}{2.945638in}}%
\pgfpathlineto{\pgfqpoint{5.164446in}{2.927636in}}%
\pgfpathlineto{\pgfqpoint{5.149419in}{2.909826in}}%
\pgfpathlineto{\pgfqpoint{5.134414in}{2.892207in}}%
\pgfpathlineto{\pgfqpoint{5.126456in}{2.879124in}}%
\pgfpathlineto{\pgfqpoint{5.118489in}{2.865833in}}%
\pgfpathlineto{\pgfqpoint{5.110513in}{2.852336in}}%
\pgfpathlineto{\pgfqpoint{5.102530in}{2.838634in}}%
\pgfpathclose%
\pgfusepath{fill}%
\end{pgfscope}%
\begin{pgfscope}%
\pgfpathrectangle{\pgfqpoint{1.150000in}{0.150000in}}{\pgfqpoint{5.700000in}{5.700000in}}%
\pgfusepath{clip}%
\pgfsetbuttcap%
\pgfsetroundjoin%
\definecolor{currentfill}{rgb}{0.496615,0.826376,0.306377}%
\pgfsetfillcolor{currentfill}%
\pgfsetfillopacity{0.800000}%
\pgfsetlinewidth{0.000000pt}%
\definecolor{currentstroke}{rgb}{0.000000,0.000000,0.000000}%
\pgfsetstrokecolor{currentstroke}%
\pgfsetdash{}{0pt}%
\pgfpathmoveto{\pgfqpoint{5.473304in}{3.335231in}}%
\pgfpathlineto{\pgfqpoint{5.488601in}{3.355254in}}%
\pgfpathlineto{\pgfqpoint{5.503923in}{3.375472in}}%
\pgfpathlineto{\pgfqpoint{5.519270in}{3.395887in}}%
\pgfpathlineto{\pgfqpoint{5.534644in}{3.416498in}}%
\pgfpathlineto{\pgfqpoint{5.542388in}{3.424902in}}%
\pgfpathlineto{\pgfqpoint{5.550119in}{3.433068in}}%
\pgfpathlineto{\pgfqpoint{5.557839in}{3.440998in}}%
\pgfpathlineto{\pgfqpoint{5.565546in}{3.448695in}}%
\pgfpathlineto{\pgfqpoint{5.550177in}{3.428181in}}%
\pgfpathlineto{\pgfqpoint{5.534834in}{3.407864in}}%
\pgfpathlineto{\pgfqpoint{5.519516in}{3.387742in}}%
\pgfpathlineto{\pgfqpoint{5.504224in}{3.367816in}}%
\pgfpathlineto{\pgfqpoint{5.496512in}{3.360009in}}%
\pgfpathlineto{\pgfqpoint{5.488787in}{3.351977in}}%
\pgfpathlineto{\pgfqpoint{5.481051in}{3.343718in}}%
\pgfpathlineto{\pgfqpoint{5.473304in}{3.335231in}}%
\pgfpathclose%
\pgfusepath{fill}%
\end{pgfscope}%
\begin{pgfscope}%
\pgfpathrectangle{\pgfqpoint{1.150000in}{0.150000in}}{\pgfqpoint{5.700000in}{5.700000in}}%
\pgfusepath{clip}%
\pgfsetbuttcap%
\pgfsetroundjoin%
\definecolor{currentfill}{rgb}{0.129933,0.559582,0.551864}%
\pgfsetfillcolor{currentfill}%
\pgfsetfillopacity{0.800000}%
\pgfsetlinewidth{0.000000pt}%
\definecolor{currentstroke}{rgb}{0.000000,0.000000,0.000000}%
\pgfsetstrokecolor{currentstroke}%
\pgfsetdash{}{0pt}%
\pgfpathmoveto{\pgfqpoint{4.822403in}{2.399842in}}%
\pgfpathlineto{\pgfqpoint{4.837188in}{2.414653in}}%
\pgfpathlineto{\pgfqpoint{4.851994in}{2.429651in}}%
\pgfpathlineto{\pgfqpoint{4.866819in}{2.444837in}}%
\pgfpathlineto{\pgfqpoint{4.881666in}{2.460210in}}%
\pgfpathlineto{\pgfqpoint{4.889783in}{2.477382in}}%
\pgfpathlineto{\pgfqpoint{4.897894in}{2.494394in}}%
\pgfpathlineto{\pgfqpoint{4.905999in}{2.511242in}}%
\pgfpathlineto{\pgfqpoint{4.914097in}{2.527925in}}%
\pgfpathlineto{\pgfqpoint{4.899241in}{2.512248in}}%
\pgfpathlineto{\pgfqpoint{4.884404in}{2.496759in}}%
\pgfpathlineto{\pgfqpoint{4.869589in}{2.481457in}}%
\pgfpathlineto{\pgfqpoint{4.854793in}{2.466344in}}%
\pgfpathlineto{\pgfqpoint{4.846705in}{2.449951in}}%
\pgfpathlineto{\pgfqpoint{4.838610in}{2.433402in}}%
\pgfpathlineto{\pgfqpoint{4.830509in}{2.416698in}}%
\pgfpathlineto{\pgfqpoint{4.822403in}{2.399842in}}%
\pgfpathclose%
\pgfusepath{fill}%
\end{pgfscope}%
\begin{pgfscope}%
\pgfpathrectangle{\pgfqpoint{1.150000in}{0.150000in}}{\pgfqpoint{5.700000in}{5.700000in}}%
\pgfusepath{clip}%
\pgfsetbuttcap%
\pgfsetroundjoin%
\definecolor{currentfill}{rgb}{0.266941,0.748751,0.440573}%
\pgfsetfillcolor{currentfill}%
\pgfsetfillopacity{0.800000}%
\pgfsetlinewidth{0.000000pt}%
\definecolor{currentstroke}{rgb}{0.000000,0.000000,0.000000}%
\pgfsetstrokecolor{currentstroke}%
\pgfsetdash{}{0pt}%
\pgfpathmoveto{\pgfqpoint{5.226332in}{3.014447in}}%
\pgfpathlineto{\pgfqpoint{5.241433in}{3.032925in}}%
\pgfpathlineto{\pgfqpoint{5.256557in}{3.051596in}}%
\pgfpathlineto{\pgfqpoint{5.271705in}{3.070460in}}%
\pgfpathlineto{\pgfqpoint{5.286878in}{3.089519in}}%
\pgfpathlineto{\pgfqpoint{5.294798in}{3.101695in}}%
\pgfpathlineto{\pgfqpoint{5.302709in}{3.113643in}}%
\pgfpathlineto{\pgfqpoint{5.310610in}{3.125363in}}%
\pgfpathlineto{\pgfqpoint{5.318501in}{3.136854in}}%
\pgfpathlineto{\pgfqpoint{5.303325in}{3.117742in}}%
\pgfpathlineto{\pgfqpoint{5.288174in}{3.098824in}}%
\pgfpathlineto{\pgfqpoint{5.273047in}{3.080100in}}%
\pgfpathlineto{\pgfqpoint{5.257944in}{3.061569in}}%
\pgfpathlineto{\pgfqpoint{5.250055in}{3.050117in}}%
\pgfpathlineto{\pgfqpoint{5.242157in}{3.038446in}}%
\pgfpathlineto{\pgfqpoint{5.234249in}{3.026556in}}%
\pgfpathlineto{\pgfqpoint{5.226332in}{3.014447in}}%
\pgfpathclose%
\pgfusepath{fill}%
\end{pgfscope}%
\begin{pgfscope}%
\pgfpathrectangle{\pgfqpoint{1.150000in}{0.150000in}}{\pgfqpoint{5.700000in}{5.700000in}}%
\pgfusepath{clip}%
\pgfsetbuttcap%
\pgfsetroundjoin%
\definecolor{currentfill}{rgb}{0.377779,0.791781,0.377939}%
\pgfsetfillcolor{currentfill}%
\pgfsetfillopacity{0.800000}%
\pgfsetlinewidth{0.000000pt}%
\definecolor{currentstroke}{rgb}{0.000000,0.000000,0.000000}%
\pgfsetstrokecolor{currentstroke}%
\pgfsetdash{}{0pt}%
\pgfpathmoveto{\pgfqpoint{5.349960in}{3.180544in}}%
\pgfpathlineto{\pgfqpoint{5.365160in}{3.199867in}}%
\pgfpathlineto{\pgfqpoint{5.380386in}{3.219384in}}%
\pgfpathlineto{\pgfqpoint{5.395636in}{3.239096in}}%
\pgfpathlineto{\pgfqpoint{5.410910in}{3.259004in}}%
\pgfpathlineto{\pgfqpoint{5.418749in}{3.269352in}}%
\pgfpathlineto{\pgfqpoint{5.426577in}{3.279463in}}%
\pgfpathlineto{\pgfqpoint{5.434393in}{3.289340in}}%
\pgfpathlineto{\pgfqpoint{5.442198in}{3.298983in}}%
\pgfpathlineto{\pgfqpoint{5.426924in}{3.279097in}}%
\pgfpathlineto{\pgfqpoint{5.411674in}{3.259406in}}%
\pgfpathlineto{\pgfqpoint{5.396450in}{3.239910in}}%
\pgfpathlineto{\pgfqpoint{5.381250in}{3.220609in}}%
\pgfpathlineto{\pgfqpoint{5.373443in}{3.210931in}}%
\pgfpathlineto{\pgfqpoint{5.365626in}{3.201028in}}%
\pgfpathlineto{\pgfqpoint{5.357798in}{3.190899in}}%
\pgfpathlineto{\pgfqpoint{5.349960in}{3.180544in}}%
\pgfpathclose%
\pgfusepath{fill}%
\end{pgfscope}%
\begin{pgfscope}%
\pgfpathrectangle{\pgfqpoint{1.150000in}{0.150000in}}{\pgfqpoint{5.700000in}{5.700000in}}%
\pgfusepath{clip}%
\pgfsetbuttcap%
\pgfsetroundjoin%
\definecolor{currentfill}{rgb}{0.160665,0.478540,0.558115}%
\pgfsetfillcolor{currentfill}%
\pgfsetfillopacity{0.800000}%
\pgfsetlinewidth{0.000000pt}%
\definecolor{currentstroke}{rgb}{0.000000,0.000000,0.000000}%
\pgfsetstrokecolor{currentstroke}%
\pgfsetdash{}{0pt}%
\pgfpathmoveto{\pgfqpoint{4.665901in}{2.133860in}}%
\pgfpathlineto{\pgfqpoint{4.680577in}{2.146818in}}%
\pgfpathlineto{\pgfqpoint{4.695272in}{2.159960in}}%
\pgfpathlineto{\pgfqpoint{4.709985in}{2.173286in}}%
\pgfpathlineto{\pgfqpoint{4.724717in}{2.186796in}}%
\pgfpathlineto{\pgfqpoint{4.732884in}{2.205240in}}%
\pgfpathlineto{\pgfqpoint{4.741047in}{2.223569in}}%
\pgfpathlineto{\pgfqpoint{4.749205in}{2.241781in}}%
\pgfpathlineto{\pgfqpoint{4.757358in}{2.259871in}}%
\pgfpathlineto{\pgfqpoint{4.742614in}{2.245955in}}%
\pgfpathlineto{\pgfqpoint{4.727890in}{2.232224in}}%
\pgfpathlineto{\pgfqpoint{4.713184in}{2.218679in}}%
\pgfpathlineto{\pgfqpoint{4.698498in}{2.205318in}}%
\pgfpathlineto{\pgfqpoint{4.690355in}{2.187619in}}%
\pgfpathlineto{\pgfqpoint{4.682208in}{2.169807in}}%
\pgfpathlineto{\pgfqpoint{4.674057in}{2.151886in}}%
\pgfpathlineto{\pgfqpoint{4.665901in}{2.133860in}}%
\pgfpathclose%
\pgfusepath{fill}%
\end{pgfscope}%
\begin{pgfscope}%
\pgfpathrectangle{\pgfqpoint{1.150000in}{0.150000in}}{\pgfqpoint{5.700000in}{5.700000in}}%
\pgfusepath{clip}%
\pgfsetbuttcap%
\pgfsetroundjoin%
\definecolor{currentfill}{rgb}{0.197636,0.391528,0.554969}%
\pgfsetfillcolor{currentfill}%
\pgfsetfillopacity{0.800000}%
\pgfsetlinewidth{0.000000pt}%
\definecolor{currentstroke}{rgb}{0.000000,0.000000,0.000000}%
\pgfsetstrokecolor{currentstroke}%
\pgfsetdash{}{0pt}%
\pgfpathmoveto{\pgfqpoint{4.509318in}{1.866384in}}%
\pgfpathlineto{\pgfqpoint{4.523891in}{1.877205in}}%
\pgfpathlineto{\pgfqpoint{4.538482in}{1.888208in}}%
\pgfpathlineto{\pgfqpoint{4.553089in}{1.899392in}}%
\pgfpathlineto{\pgfqpoint{4.567714in}{1.910757in}}%
\pgfpathlineto{\pgfqpoint{4.575917in}{1.929736in}}%
\pgfpathlineto{\pgfqpoint{4.584116in}{1.948659in}}%
\pgfpathlineto{\pgfqpoint{4.592312in}{1.967521in}}%
\pgfpathlineto{\pgfqpoint{4.600504in}{1.986319in}}%
\pgfpathlineto{\pgfqpoint{4.585869in}{1.974452in}}%
\pgfpathlineto{\pgfqpoint{4.571250in}{1.962766in}}%
\pgfpathlineto{\pgfqpoint{4.556650in}{1.951263in}}%
\pgfpathlineto{\pgfqpoint{4.542066in}{1.939941in}}%
\pgfpathlineto{\pgfqpoint{4.533884in}{1.921632in}}%
\pgfpathlineto{\pgfqpoint{4.525699in}{1.903266in}}%
\pgfpathlineto{\pgfqpoint{4.517510in}{1.884849in}}%
\pgfpathlineto{\pgfqpoint{4.509318in}{1.866384in}}%
\pgfpathclose%
\pgfusepath{fill}%
\end{pgfscope}%
\begin{pgfscope}%
\pgfpathrectangle{\pgfqpoint{1.150000in}{0.150000in}}{\pgfqpoint{5.700000in}{5.700000in}}%
\pgfusepath{clip}%
\pgfsetbuttcap%
\pgfsetroundjoin%
\definecolor{currentfill}{rgb}{0.266580,0.228262,0.514349}%
\pgfsetfillcolor{currentfill}%
\pgfsetfillopacity{0.800000}%
\pgfsetlinewidth{0.000000pt}%
\definecolor{currentstroke}{rgb}{0.000000,0.000000,0.000000}%
\pgfsetstrokecolor{currentstroke}%
\pgfsetdash{}{0pt}%
\pgfpathmoveto{\pgfqpoint{4.229070in}{1.438553in}}%
\pgfpathlineto{\pgfqpoint{4.243487in}{1.444982in}}%
\pgfpathlineto{\pgfqpoint{4.257917in}{1.451586in}}%
\pgfpathlineto{\pgfqpoint{4.272361in}{1.458367in}}%
\pgfpathlineto{\pgfqpoint{4.286818in}{1.465324in}}%
\pgfpathlineto{\pgfqpoint{4.295067in}{1.482980in}}%
\pgfpathlineto{\pgfqpoint{4.303313in}{1.500705in}}%
\pgfpathlineto{\pgfqpoint{4.311556in}{1.518493in}}%
\pgfpathlineto{\pgfqpoint{4.319795in}{1.536337in}}%
\pgfpathlineto{\pgfqpoint{4.305332in}{1.528725in}}%
\pgfpathlineto{\pgfqpoint{4.290883in}{1.521290in}}%
\pgfpathlineto{\pgfqpoint{4.276449in}{1.514031in}}%
\pgfpathlineto{\pgfqpoint{4.262028in}{1.506950in}}%
\pgfpathlineto{\pgfqpoint{4.253793in}{1.489748in}}%
\pgfpathlineto{\pgfqpoint{4.245556in}{1.472610in}}%
\pgfpathlineto{\pgfqpoint{4.237315in}{1.455543in}}%
\pgfpathlineto{\pgfqpoint{4.229070in}{1.438553in}}%
\pgfpathclose%
\pgfusepath{fill}%
\end{pgfscope}%
\begin{pgfscope}%
\pgfpathrectangle{\pgfqpoint{1.150000in}{0.150000in}}{\pgfqpoint{5.700000in}{5.700000in}}%
\pgfusepath{clip}%
\pgfsetbuttcap%
\pgfsetroundjoin%
\definecolor{currentfill}{rgb}{0.280255,0.165693,0.476498}%
\pgfsetfillcolor{currentfill}%
\pgfsetfillopacity{0.800000}%
\pgfsetlinewidth{0.000000pt}%
\definecolor{currentstroke}{rgb}{0.000000,0.000000,0.000000}%
\pgfsetstrokecolor{currentstroke}%
\pgfsetdash{}{0pt}%
\pgfpathmoveto{\pgfqpoint{4.105459in}{1.287793in}}%
\pgfpathlineto{\pgfqpoint{4.119823in}{1.292124in}}%
\pgfpathlineto{\pgfqpoint{4.134198in}{1.296630in}}%
\pgfpathlineto{\pgfqpoint{4.148586in}{1.301310in}}%
\pgfpathlineto{\pgfqpoint{4.162985in}{1.306165in}}%
\pgfpathlineto{\pgfqpoint{4.171259in}{1.322308in}}%
\pgfpathlineto{\pgfqpoint{4.179528in}{1.338581in}}%
\pgfpathlineto{\pgfqpoint{4.187794in}{1.354977in}}%
\pgfpathlineto{\pgfqpoint{4.196057in}{1.371487in}}%
\pgfpathlineto{\pgfqpoint{4.181656in}{1.365918in}}%
\pgfpathlineto{\pgfqpoint{4.167267in}{1.360525in}}%
\pgfpathlineto{\pgfqpoint{4.152892in}{1.355306in}}%
\pgfpathlineto{\pgfqpoint{4.138528in}{1.350264in}}%
\pgfpathlineto{\pgfqpoint{4.130267in}{1.334455in}}%
\pgfpathlineto{\pgfqpoint{4.122002in}{1.318768in}}%
\pgfpathlineto{\pgfqpoint{4.113733in}{1.303212in}}%
\pgfpathlineto{\pgfqpoint{4.105459in}{1.287793in}}%
\pgfpathclose%
\pgfusepath{fill}%
\end{pgfscope}%
\begin{pgfscope}%
\pgfpathrectangle{\pgfqpoint{1.150000in}{0.150000in}}{\pgfqpoint{5.700000in}{5.700000in}}%
\pgfusepath{clip}%
\pgfsetbuttcap%
\pgfsetroundjoin%
\definecolor{currentfill}{rgb}{0.120638,0.625828,0.533488}%
\pgfsetfillcolor{currentfill}%
\pgfsetfillopacity{0.800000}%
\pgfsetlinewidth{0.000000pt}%
\definecolor{currentstroke}{rgb}{0.000000,0.000000,0.000000}%
\pgfsetstrokecolor{currentstroke}%
\pgfsetdash{}{0pt}%
\pgfpathmoveto{\pgfqpoint{4.946429in}{2.592950in}}%
\pgfpathlineto{\pgfqpoint{4.961317in}{2.609085in}}%
\pgfpathlineto{\pgfqpoint{4.976226in}{2.625410in}}%
\pgfpathlineto{\pgfqpoint{4.991156in}{2.641924in}}%
\pgfpathlineto{\pgfqpoint{5.006109in}{2.658628in}}%
\pgfpathlineto{\pgfqpoint{5.014185in}{2.674694in}}%
\pgfpathlineto{\pgfqpoint{5.022254in}{2.690573in}}%
\pgfpathlineto{\pgfqpoint{5.030315in}{2.706261in}}%
\pgfpathlineto{\pgfqpoint{5.038370in}{2.721758in}}%
\pgfpathlineto{\pgfqpoint{5.023408in}{2.704819in}}%
\pgfpathlineto{\pgfqpoint{5.008469in}{2.688071in}}%
\pgfpathlineto{\pgfqpoint{4.993551in}{2.671512in}}%
\pgfpathlineto{\pgfqpoint{4.978654in}{2.655143in}}%
\pgfpathlineto{\pgfqpoint{4.970608in}{2.639868in}}%
\pgfpathlineto{\pgfqpoint{4.962555in}{2.624409in}}%
\pgfpathlineto{\pgfqpoint{4.954496in}{2.608769in}}%
\pgfpathlineto{\pgfqpoint{4.946429in}{2.592950in}}%
\pgfpathclose%
\pgfusepath{fill}%
\end{pgfscope}%
\begin{pgfscope}%
\pgfpathrectangle{\pgfqpoint{1.150000in}{0.150000in}}{\pgfqpoint{5.700000in}{5.700000in}}%
\pgfusepath{clip}%
\pgfsetbuttcap%
\pgfsetroundjoin%
\definecolor{currentfill}{rgb}{0.239346,0.300855,0.540844}%
\pgfsetfillcolor{currentfill}%
\pgfsetfillopacity{0.800000}%
\pgfsetlinewidth{0.000000pt}%
\definecolor{currentstroke}{rgb}{0.000000,0.000000,0.000000}%
\pgfsetstrokecolor{currentstroke}%
\pgfsetdash{}{0pt}%
\pgfpathmoveto{\pgfqpoint{4.352722in}{1.608164in}}%
\pgfpathlineto{\pgfqpoint{4.367206in}{1.616578in}}%
\pgfpathlineto{\pgfqpoint{4.381705in}{1.625171in}}%
\pgfpathlineto{\pgfqpoint{4.396219in}{1.633942in}}%
\pgfpathlineto{\pgfqpoint{4.410749in}{1.642891in}}%
\pgfpathlineto{\pgfqpoint{4.418980in}{1.661537in}}%
\pgfpathlineto{\pgfqpoint{4.427209in}{1.680197in}}%
\pgfpathlineto{\pgfqpoint{4.435434in}{1.698866in}}%
\pgfpathlineto{\pgfqpoint{4.443656in}{1.717537in}}%
\pgfpathlineto{\pgfqpoint{4.429118in}{1.707992in}}%
\pgfpathlineto{\pgfqpoint{4.414595in}{1.698626in}}%
\pgfpathlineto{\pgfqpoint{4.400088in}{1.689440in}}%
\pgfpathlineto{\pgfqpoint{4.385597in}{1.680432in}}%
\pgfpathlineto{\pgfqpoint{4.377383in}{1.662343in}}%
\pgfpathlineto{\pgfqpoint{4.369166in}{1.644265in}}%
\pgfpathlineto{\pgfqpoint{4.360945in}{1.626204in}}%
\pgfpathlineto{\pgfqpoint{4.352722in}{1.608164in}}%
\pgfpathclose%
\pgfusepath{fill}%
\end{pgfscope}%
\begin{pgfscope}%
\pgfpathrectangle{\pgfqpoint{1.150000in}{0.150000in}}{\pgfqpoint{5.700000in}{5.700000in}}%
\pgfusepath{clip}%
\pgfsetbuttcap%
\pgfsetroundjoin%
\definecolor{currentfill}{rgb}{0.281446,0.084320,0.407414}%
\pgfsetfillcolor{currentfill}%
\pgfsetfillopacity{0.800000}%
\pgfsetlinewidth{0.000000pt}%
\definecolor{currentstroke}{rgb}{0.000000,0.000000,0.000000}%
\pgfsetstrokecolor{currentstroke}%
\pgfsetdash{}{0pt}%
\pgfpathmoveto{\pgfqpoint{3.891186in}{1.106184in}}%
\pgfpathlineto{\pgfqpoint{3.905483in}{1.106808in}}%
\pgfpathlineto{\pgfqpoint{3.919790in}{1.107605in}}%
\pgfpathlineto{\pgfqpoint{3.934105in}{1.108576in}}%
\pgfpathlineto{\pgfqpoint{3.948430in}{1.109720in}}%
\pgfpathlineto{\pgfqpoint{3.956767in}{1.122121in}}%
\pgfpathlineto{\pgfqpoint{3.965099in}{1.134754in}}%
\pgfpathlineto{\pgfqpoint{3.973424in}{1.147612in}}%
\pgfpathlineto{\pgfqpoint{3.981745in}{1.160688in}}%
\pgfpathlineto{\pgfqpoint{3.967428in}{1.158742in}}%
\pgfpathlineto{\pgfqpoint{3.953121in}{1.156970in}}%
\pgfpathlineto{\pgfqpoint{3.938823in}{1.155372in}}%
\pgfpathlineto{\pgfqpoint{3.924536in}{1.153948in}}%
\pgfpathlineto{\pgfqpoint{3.916208in}{1.141661in}}%
\pgfpathlineto{\pgfqpoint{3.907873in}{1.129600in}}%
\pgfpathlineto{\pgfqpoint{3.899533in}{1.117772in}}%
\pgfpathlineto{\pgfqpoint{3.891186in}{1.106184in}}%
\pgfpathclose%
\pgfusepath{fill}%
\end{pgfscope}%
\begin{pgfscope}%
\pgfpathrectangle{\pgfqpoint{1.150000in}{0.150000in}}{\pgfqpoint{5.700000in}{5.700000in}}%
\pgfusepath{clip}%
\pgfsetbuttcap%
\pgfsetroundjoin%
\definecolor{currentfill}{rgb}{0.595839,0.848717,0.243329}%
\pgfsetfillcolor{currentfill}%
\pgfsetfillopacity{0.800000}%
\pgfsetlinewidth{0.000000pt}%
\definecolor{currentstroke}{rgb}{0.000000,0.000000,0.000000}%
\pgfsetstrokecolor{currentstroke}%
\pgfsetdash{}{0pt}%
\pgfpathmoveto{\pgfqpoint{5.565546in}{3.448695in}}%
\pgfpathlineto{\pgfqpoint{5.580940in}{3.469405in}}%
\pgfpathlineto{\pgfqpoint{5.596361in}{3.490312in}}%
\pgfpathlineto{\pgfqpoint{5.611808in}{3.511417in}}%
\pgfpathlineto{\pgfqpoint{5.619498in}{3.518790in}}%
\pgfpathlineto{\pgfqpoint{5.627176in}{3.525924in}}%
\pgfpathlineto{\pgfqpoint{5.634840in}{3.532822in}}%
\pgfpathlineto{\pgfqpoint{5.642492in}{3.539485in}}%
\pgfpathlineto{\pgfqpoint{5.627052in}{3.518517in}}%
\pgfpathlineto{\pgfqpoint{5.611638in}{3.497746in}}%
\pgfpathlineto{\pgfqpoint{5.596250in}{3.477171in}}%
\pgfpathlineto{\pgfqpoint{5.588592in}{3.470395in}}%
\pgfpathlineto{\pgfqpoint{5.580923in}{3.463391in}}%
\pgfpathlineto{\pgfqpoint{5.573240in}{3.456158in}}%
\pgfpathlineto{\pgfqpoint{5.565546in}{3.448695in}}%
\pgfpathclose%
\pgfusepath{fill}%
\end{pgfscope}%
\begin{pgfscope}%
\pgfpathrectangle{\pgfqpoint{1.150000in}{0.150000in}}{\pgfqpoint{5.700000in}{5.700000in}}%
\pgfusepath{clip}%
\pgfsetbuttcap%
\pgfsetroundjoin%
\definecolor{currentfill}{rgb}{0.135066,0.544853,0.554029}%
\pgfsetfillcolor{currentfill}%
\pgfsetfillopacity{0.800000}%
\pgfsetlinewidth{0.000000pt}%
\definecolor{currentstroke}{rgb}{0.000000,0.000000,0.000000}%
\pgfsetstrokecolor{currentstroke}%
\pgfsetdash{}{0pt}%
\pgfpathmoveto{\pgfqpoint{4.789922in}{2.330956in}}%
\pgfpathlineto{\pgfqpoint{4.804697in}{2.345430in}}%
\pgfpathlineto{\pgfqpoint{4.819491in}{2.360090in}}%
\pgfpathlineto{\pgfqpoint{4.834306in}{2.374937in}}%
\pgfpathlineto{\pgfqpoint{4.849140in}{2.389971in}}%
\pgfpathlineto{\pgfqpoint{4.857280in}{2.407757in}}%
\pgfpathlineto{\pgfqpoint{4.865414in}{2.425395in}}%
\pgfpathlineto{\pgfqpoint{4.873543in}{2.442880in}}%
\pgfpathlineto{\pgfqpoint{4.881666in}{2.460210in}}%
\pgfpathlineto{\pgfqpoint{4.866819in}{2.444837in}}%
\pgfpathlineto{\pgfqpoint{4.851994in}{2.429651in}}%
\pgfpathlineto{\pgfqpoint{4.837188in}{2.414653in}}%
\pgfpathlineto{\pgfqpoint{4.822403in}{2.399842in}}%
\pgfpathlineto{\pgfqpoint{4.814291in}{2.382837in}}%
\pgfpathlineto{\pgfqpoint{4.806173in}{2.365686in}}%
\pgfpathlineto{\pgfqpoint{4.798051in}{2.348391in}}%
\pgfpathlineto{\pgfqpoint{4.789922in}{2.330956in}}%
\pgfpathclose%
\pgfusepath{fill}%
\end{pgfscope}%
\begin{pgfscope}%
\pgfpathrectangle{\pgfqpoint{1.150000in}{0.150000in}}{\pgfqpoint{5.700000in}{5.700000in}}%
\pgfusepath{clip}%
\pgfsetbuttcap%
\pgfsetroundjoin%
\definecolor{currentfill}{rgb}{0.168126,0.459988,0.558082}%
\pgfsetfillcolor{currentfill}%
\pgfsetfillopacity{0.800000}%
\pgfsetlinewidth{0.000000pt}%
\definecolor{currentstroke}{rgb}{0.000000,0.000000,0.000000}%
\pgfsetstrokecolor{currentstroke}%
\pgfsetdash{}{0pt}%
\pgfpathmoveto{\pgfqpoint{4.633235in}{2.060778in}}%
\pgfpathlineto{\pgfqpoint{4.647900in}{2.073298in}}%
\pgfpathlineto{\pgfqpoint{4.662583in}{2.086002in}}%
\pgfpathlineto{\pgfqpoint{4.677284in}{2.098890in}}%
\pgfpathlineto{\pgfqpoint{4.692004in}{2.111961in}}%
\pgfpathlineto{\pgfqpoint{4.700189in}{2.130822in}}%
\pgfpathlineto{\pgfqpoint{4.708369in}{2.149584in}}%
\pgfpathlineto{\pgfqpoint{4.716545in}{2.168243in}}%
\pgfpathlineto{\pgfqpoint{4.724717in}{2.186796in}}%
\pgfpathlineto{\pgfqpoint{4.709985in}{2.173286in}}%
\pgfpathlineto{\pgfqpoint{4.695272in}{2.159960in}}%
\pgfpathlineto{\pgfqpoint{4.680577in}{2.146818in}}%
\pgfpathlineto{\pgfqpoint{4.665901in}{2.133860in}}%
\pgfpathlineto{\pgfqpoint{4.657741in}{2.115733in}}%
\pgfpathlineto{\pgfqpoint{4.649577in}{2.097507in}}%
\pgfpathlineto{\pgfqpoint{4.641408in}{2.079187in}}%
\pgfpathlineto{\pgfqpoint{4.633235in}{2.060778in}}%
\pgfpathclose%
\pgfusepath{fill}%
\end{pgfscope}%
\begin{pgfscope}%
\pgfpathrectangle{\pgfqpoint{1.150000in}{0.150000in}}{\pgfqpoint{5.700000in}{5.700000in}}%
\pgfusepath{clip}%
\pgfsetbuttcap%
\pgfsetroundjoin%
\definecolor{currentfill}{rgb}{0.283091,0.110553,0.431554}%
\pgfsetfillcolor{currentfill}%
\pgfsetfillopacity{0.800000}%
\pgfsetlinewidth{0.000000pt}%
\definecolor{currentstroke}{rgb}{0.000000,0.000000,0.000000}%
\pgfsetstrokecolor{currentstroke}%
\pgfsetdash{}{0pt}%
\pgfpathmoveto{\pgfqpoint{3.981745in}{1.160688in}}%
\pgfpathlineto{\pgfqpoint{3.996072in}{1.162808in}}%
\pgfpathlineto{\pgfqpoint{4.010409in}{1.165102in}}%
\pgfpathlineto{\pgfqpoint{4.024756in}{1.167569in}}%
\pgfpathlineto{\pgfqpoint{4.039114in}{1.170210in}}%
\pgfpathlineto{\pgfqpoint{4.047423in}{1.184279in}}%
\pgfpathlineto{\pgfqpoint{4.055728in}{1.198543in}}%
\pgfpathlineto{\pgfqpoint{4.064028in}{1.212993in}}%
\pgfpathlineto{\pgfqpoint{4.072323in}{1.227624in}}%
\pgfpathlineto{\pgfqpoint{4.057969in}{1.224210in}}%
\pgfpathlineto{\pgfqpoint{4.043626in}{1.220970in}}%
\pgfpathlineto{\pgfqpoint{4.029294in}{1.217904in}}%
\pgfpathlineto{\pgfqpoint{4.014972in}{1.215013in}}%
\pgfpathlineto{\pgfqpoint{4.006673in}{1.201144in}}%
\pgfpathlineto{\pgfqpoint{3.998369in}{1.187461in}}%
\pgfpathlineto{\pgfqpoint{3.990059in}{1.173974in}}%
\pgfpathlineto{\pgfqpoint{3.981745in}{1.160688in}}%
\pgfpathclose%
\pgfusepath{fill}%
\end{pgfscope}%
\begin{pgfscope}%
\pgfpathrectangle{\pgfqpoint{1.150000in}{0.150000in}}{\pgfqpoint{5.700000in}{5.700000in}}%
\pgfusepath{clip}%
\pgfsetbuttcap%
\pgfsetroundjoin%
\definecolor{currentfill}{rgb}{0.162016,0.687316,0.499129}%
\pgfsetfillcolor{currentfill}%
\pgfsetfillopacity{0.800000}%
\pgfsetlinewidth{0.000000pt}%
\definecolor{currentstroke}{rgb}{0.000000,0.000000,0.000000}%
\pgfsetstrokecolor{currentstroke}%
\pgfsetdash{}{0pt}%
\pgfpathmoveto{\pgfqpoint{5.070513in}{2.781793in}}%
\pgfpathlineto{\pgfqpoint{5.085505in}{2.799122in}}%
\pgfpathlineto{\pgfqpoint{5.100519in}{2.816642in}}%
\pgfpathlineto{\pgfqpoint{5.115556in}{2.834353in}}%
\pgfpathlineto{\pgfqpoint{5.130616in}{2.852257in}}%
\pgfpathlineto{\pgfqpoint{5.138640in}{2.866947in}}%
\pgfpathlineto{\pgfqpoint{5.146656in}{2.881427in}}%
\pgfpathlineto{\pgfqpoint{5.154663in}{2.895696in}}%
\pgfpathlineto{\pgfqpoint{5.162662in}{2.909752in}}%
\pgfpathlineto{\pgfqpoint{5.147594in}{2.891684in}}%
\pgfpathlineto{\pgfqpoint{5.132550in}{2.873809in}}%
\pgfpathlineto{\pgfqpoint{5.117528in}{2.856126in}}%
\pgfpathlineto{\pgfqpoint{5.102530in}{2.838634in}}%
\pgfpathlineto{\pgfqpoint{5.094537in}{2.824728in}}%
\pgfpathlineto{\pgfqpoint{5.086537in}{2.810618in}}%
\pgfpathlineto{\pgfqpoint{5.078529in}{2.796306in}}%
\pgfpathlineto{\pgfqpoint{5.070513in}{2.781793in}}%
\pgfpathclose%
\pgfusepath{fill}%
\end{pgfscope}%
\begin{pgfscope}%
\pgfpathrectangle{\pgfqpoint{1.150000in}{0.150000in}}{\pgfqpoint{5.700000in}{5.700000in}}%
\pgfusepath{clip}%
\pgfsetbuttcap%
\pgfsetroundjoin%
\definecolor{currentfill}{rgb}{0.206756,0.371758,0.553117}%
\pgfsetfillcolor{currentfill}%
\pgfsetfillopacity{0.800000}%
\pgfsetlinewidth{0.000000pt}%
\definecolor{currentstroke}{rgb}{0.000000,0.000000,0.000000}%
\pgfsetstrokecolor{currentstroke}%
\pgfsetdash{}{0pt}%
\pgfpathmoveto{\pgfqpoint{4.476513in}{1.792145in}}%
\pgfpathlineto{\pgfqpoint{4.491077in}{1.802434in}}%
\pgfpathlineto{\pgfqpoint{4.505657in}{1.812904in}}%
\pgfpathlineto{\pgfqpoint{4.520254in}{1.823553in}}%
\pgfpathlineto{\pgfqpoint{4.534868in}{1.834383in}}%
\pgfpathlineto{\pgfqpoint{4.543084in}{1.853536in}}%
\pgfpathlineto{\pgfqpoint{4.551297in}{1.872653in}}%
\pgfpathlineto{\pgfqpoint{4.559507in}{1.891728in}}%
\pgfpathlineto{\pgfqpoint{4.567714in}{1.910757in}}%
\pgfpathlineto{\pgfqpoint{4.553089in}{1.899392in}}%
\pgfpathlineto{\pgfqpoint{4.538482in}{1.888208in}}%
\pgfpathlineto{\pgfqpoint{4.523891in}{1.877205in}}%
\pgfpathlineto{\pgfqpoint{4.509318in}{1.866384in}}%
\pgfpathlineto{\pgfqpoint{4.501122in}{1.847876in}}%
\pgfpathlineto{\pgfqpoint{4.492922in}{1.829331in}}%
\pgfpathlineto{\pgfqpoint{4.484720in}{1.810752in}}%
\pgfpathlineto{\pgfqpoint{4.476513in}{1.792145in}}%
\pgfpathclose%
\pgfusepath{fill}%
\end{pgfscope}%
\begin{pgfscope}%
\pgfpathrectangle{\pgfqpoint{1.150000in}{0.150000in}}{\pgfqpoint{5.700000in}{5.700000in}}%
\pgfusepath{clip}%
\pgfsetbuttcap%
\pgfsetroundjoin%
\definecolor{currentfill}{rgb}{0.271828,0.209303,0.504434}%
\pgfsetfillcolor{currentfill}%
\pgfsetfillopacity{0.800000}%
\pgfsetlinewidth{0.000000pt}%
\definecolor{currentstroke}{rgb}{0.000000,0.000000,0.000000}%
\pgfsetstrokecolor{currentstroke}%
\pgfsetdash{}{0pt}%
\pgfpathmoveto{\pgfqpoint{4.196057in}{1.371487in}}%
\pgfpathlineto{\pgfqpoint{4.210470in}{1.377232in}}%
\pgfpathlineto{\pgfqpoint{4.224897in}{1.383152in}}%
\pgfpathlineto{\pgfqpoint{4.239337in}{1.389247in}}%
\pgfpathlineto{\pgfqpoint{4.253790in}{1.395518in}}%
\pgfpathlineto{\pgfqpoint{4.262052in}{1.412834in}}%
\pgfpathlineto{\pgfqpoint{4.270310in}{1.430245in}}%
\pgfpathlineto{\pgfqpoint{4.278566in}{1.447744in}}%
\pgfpathlineto{\pgfqpoint{4.286818in}{1.465324in}}%
\pgfpathlineto{\pgfqpoint{4.272361in}{1.458367in}}%
\pgfpathlineto{\pgfqpoint{4.257917in}{1.451586in}}%
\pgfpathlineto{\pgfqpoint{4.243487in}{1.444982in}}%
\pgfpathlineto{\pgfqpoint{4.229070in}{1.438553in}}%
\pgfpathlineto{\pgfqpoint{4.220822in}{1.421646in}}%
\pgfpathlineto{\pgfqpoint{4.212570in}{1.404829in}}%
\pgfpathlineto{\pgfqpoint{4.204315in}{1.388107in}}%
\pgfpathlineto{\pgfqpoint{4.196057in}{1.371487in}}%
\pgfpathclose%
\pgfusepath{fill}%
\end{pgfscope}%
\begin{pgfscope}%
\pgfpathrectangle{\pgfqpoint{1.150000in}{0.150000in}}{\pgfqpoint{5.700000in}{5.700000in}}%
\pgfusepath{clip}%
\pgfsetbuttcap%
\pgfsetroundjoin%
\definecolor{currentfill}{rgb}{0.252899,0.742211,0.448284}%
\pgfsetfillcolor{currentfill}%
\pgfsetfillopacity{0.800000}%
\pgfsetlinewidth{0.000000pt}%
\definecolor{currentstroke}{rgb}{0.000000,0.000000,0.000000}%
\pgfsetstrokecolor{currentstroke}%
\pgfsetdash{}{0pt}%
\pgfpathmoveto{\pgfqpoint{5.194569in}{2.963832in}}%
\pgfpathlineto{\pgfqpoint{5.209666in}{2.982220in}}%
\pgfpathlineto{\pgfqpoint{5.224786in}{3.000800in}}%
\pgfpathlineto{\pgfqpoint{5.239929in}{3.019574in}}%
\pgfpathlineto{\pgfqpoint{5.255097in}{3.038542in}}%
\pgfpathlineto{\pgfqpoint{5.263056in}{3.051626in}}%
\pgfpathlineto{\pgfqpoint{5.271007in}{3.064484in}}%
\pgfpathlineto{\pgfqpoint{5.278947in}{3.077115in}}%
\pgfpathlineto{\pgfqpoint{5.286878in}{3.089519in}}%
\pgfpathlineto{\pgfqpoint{5.271705in}{3.070460in}}%
\pgfpathlineto{\pgfqpoint{5.256557in}{3.051596in}}%
\pgfpathlineto{\pgfqpoint{5.241433in}{3.032925in}}%
\pgfpathlineto{\pgfqpoint{5.226332in}{3.014447in}}%
\pgfpathlineto{\pgfqpoint{5.218405in}{3.002120in}}%
\pgfpathlineto{\pgfqpoint{5.210469in}{2.989575in}}%
\pgfpathlineto{\pgfqpoint{5.202524in}{2.976812in}}%
\pgfpathlineto{\pgfqpoint{5.194569in}{2.963832in}}%
\pgfpathclose%
\pgfusepath{fill}%
\end{pgfscope}%
\begin{pgfscope}%
\pgfpathrectangle{\pgfqpoint{1.150000in}{0.150000in}}{\pgfqpoint{5.700000in}{5.700000in}}%
\pgfusepath{clip}%
\pgfsetbuttcap%
\pgfsetroundjoin%
\definecolor{currentfill}{rgb}{0.248629,0.278775,0.534556}%
\pgfsetfillcolor{currentfill}%
\pgfsetfillopacity{0.800000}%
\pgfsetlinewidth{0.000000pt}%
\definecolor{currentstroke}{rgb}{0.000000,0.000000,0.000000}%
\pgfsetstrokecolor{currentstroke}%
\pgfsetdash{}{0pt}%
\pgfpathmoveto{\pgfqpoint{4.319795in}{1.536337in}}%
\pgfpathlineto{\pgfqpoint{4.334273in}{1.544127in}}%
\pgfpathlineto{\pgfqpoint{4.348765in}{1.552094in}}%
\pgfpathlineto{\pgfqpoint{4.363271in}{1.560238in}}%
\pgfpathlineto{\pgfqpoint{4.377793in}{1.568560in}}%
\pgfpathlineto{\pgfqpoint{4.386036in}{1.587093in}}%
\pgfpathlineto{\pgfqpoint{4.394277in}{1.605663in}}%
\pgfpathlineto{\pgfqpoint{4.402514in}{1.624264in}}%
\pgfpathlineto{\pgfqpoint{4.410749in}{1.642891in}}%
\pgfpathlineto{\pgfqpoint{4.396219in}{1.633942in}}%
\pgfpathlineto{\pgfqpoint{4.381705in}{1.625171in}}%
\pgfpathlineto{\pgfqpoint{4.367206in}{1.616578in}}%
\pgfpathlineto{\pgfqpoint{4.352722in}{1.608164in}}%
\pgfpathlineto{\pgfqpoint{4.344495in}{1.590152in}}%
\pgfpathlineto{\pgfqpoint{4.336265in}{1.572173in}}%
\pgfpathlineto{\pgfqpoint{4.328032in}{1.554233in}}%
\pgfpathlineto{\pgfqpoint{4.319795in}{1.536337in}}%
\pgfpathclose%
\pgfusepath{fill}%
\end{pgfscope}%
\begin{pgfscope}%
\pgfpathrectangle{\pgfqpoint{1.150000in}{0.150000in}}{\pgfqpoint{5.700000in}{5.700000in}}%
\pgfusepath{clip}%
\pgfsetbuttcap%
\pgfsetroundjoin%
\definecolor{currentfill}{rgb}{0.487026,0.823929,0.312321}%
\pgfsetfillcolor{currentfill}%
\pgfsetfillopacity{0.800000}%
\pgfsetlinewidth{0.000000pt}%
\definecolor{currentstroke}{rgb}{0.000000,0.000000,0.000000}%
\pgfsetstrokecolor{currentstroke}%
\pgfsetdash{}{0pt}%
\pgfpathmoveto{\pgfqpoint{5.442198in}{3.298983in}}%
\pgfpathlineto{\pgfqpoint{5.457497in}{3.319065in}}%
\pgfpathlineto{\pgfqpoint{5.472822in}{3.339343in}}%
\pgfpathlineto{\pgfqpoint{5.488172in}{3.359818in}}%
\pgfpathlineto{\pgfqpoint{5.503548in}{3.380489in}}%
\pgfpathlineto{\pgfqpoint{5.511340in}{3.389853in}}%
\pgfpathlineto{\pgfqpoint{5.519120in}{3.398975in}}%
\pgfpathlineto{\pgfqpoint{5.526888in}{3.407857in}}%
\pgfpathlineto{\pgfqpoint{5.534644in}{3.416498in}}%
\pgfpathlineto{\pgfqpoint{5.519270in}{3.395887in}}%
\pgfpathlineto{\pgfqpoint{5.503923in}{3.375472in}}%
\pgfpathlineto{\pgfqpoint{5.488601in}{3.355254in}}%
\pgfpathlineto{\pgfqpoint{5.473304in}{3.335231in}}%
\pgfpathlineto{\pgfqpoint{5.465545in}{3.326516in}}%
\pgfpathlineto{\pgfqpoint{5.457774in}{3.317570in}}%
\pgfpathlineto{\pgfqpoint{5.449992in}{3.308393in}}%
\pgfpathlineto{\pgfqpoint{5.442198in}{3.298983in}}%
\pgfpathclose%
\pgfusepath{fill}%
\end{pgfscope}%
\begin{pgfscope}%
\pgfpathrectangle{\pgfqpoint{1.150000in}{0.150000in}}{\pgfqpoint{5.700000in}{5.700000in}}%
\pgfusepath{clip}%
\pgfsetbuttcap%
\pgfsetroundjoin%
\definecolor{currentfill}{rgb}{0.282290,0.145912,0.461510}%
\pgfsetfillcolor{currentfill}%
\pgfsetfillopacity{0.800000}%
\pgfsetlinewidth{0.000000pt}%
\definecolor{currentstroke}{rgb}{0.000000,0.000000,0.000000}%
\pgfsetstrokecolor{currentstroke}%
\pgfsetdash{}{0pt}%
\pgfpathmoveto{\pgfqpoint{4.072323in}{1.227624in}}%
\pgfpathlineto{\pgfqpoint{4.086688in}{1.231211in}}%
\pgfpathlineto{\pgfqpoint{4.101065in}{1.234973in}}%
\pgfpathlineto{\pgfqpoint{4.115453in}{1.238909in}}%
\pgfpathlineto{\pgfqpoint{4.129853in}{1.243019in}}%
\pgfpathlineto{\pgfqpoint{4.138142in}{1.258577in}}%
\pgfpathlineto{\pgfqpoint{4.146427in}{1.274292in}}%
\pgfpathlineto{\pgfqpoint{4.154708in}{1.290157in}}%
\pgfpathlineto{\pgfqpoint{4.162985in}{1.306165in}}%
\pgfpathlineto{\pgfqpoint{4.148586in}{1.301310in}}%
\pgfpathlineto{\pgfqpoint{4.134198in}{1.296630in}}%
\pgfpathlineto{\pgfqpoint{4.119823in}{1.292124in}}%
\pgfpathlineto{\pgfqpoint{4.105459in}{1.287793in}}%
\pgfpathlineto{\pgfqpoint{4.097182in}{1.272518in}}%
\pgfpathlineto{\pgfqpoint{4.088900in}{1.257393in}}%
\pgfpathlineto{\pgfqpoint{4.080614in}{1.242426in}}%
\pgfpathlineto{\pgfqpoint{4.072323in}{1.227624in}}%
\pgfpathclose%
\pgfusepath{fill}%
\end{pgfscope}%
\begin{pgfscope}%
\pgfpathrectangle{\pgfqpoint{1.150000in}{0.150000in}}{\pgfqpoint{5.700000in}{5.700000in}}%
\pgfusepath{clip}%
\pgfsetbuttcap%
\pgfsetroundjoin%
\definecolor{currentfill}{rgb}{0.369214,0.788888,0.382914}%
\pgfsetfillcolor{currentfill}%
\pgfsetfillopacity{0.800000}%
\pgfsetlinewidth{0.000000pt}%
\definecolor{currentstroke}{rgb}{0.000000,0.000000,0.000000}%
\pgfsetstrokecolor{currentstroke}%
\pgfsetdash{}{0pt}%
\pgfpathmoveto{\pgfqpoint{5.318501in}{3.136854in}}%
\pgfpathlineto{\pgfqpoint{5.333700in}{3.156161in}}%
\pgfpathlineto{\pgfqpoint{5.348924in}{3.175662in}}%
\pgfpathlineto{\pgfqpoint{5.364173in}{3.195358in}}%
\pgfpathlineto{\pgfqpoint{5.379446in}{3.215250in}}%
\pgfpathlineto{\pgfqpoint{5.387328in}{3.226544in}}%
\pgfpathlineto{\pgfqpoint{5.395200in}{3.237601in}}%
\pgfpathlineto{\pgfqpoint{5.403061in}{3.248421in}}%
\pgfpathlineto{\pgfqpoint{5.410910in}{3.259004in}}%
\pgfpathlineto{\pgfqpoint{5.395636in}{3.239096in}}%
\pgfpathlineto{\pgfqpoint{5.380386in}{3.219384in}}%
\pgfpathlineto{\pgfqpoint{5.365160in}{3.199867in}}%
\pgfpathlineto{\pgfqpoint{5.349960in}{3.180544in}}%
\pgfpathlineto{\pgfqpoint{5.342111in}{3.169963in}}%
\pgfpathlineto{\pgfqpoint{5.334251in}{3.159154in}}%
\pgfpathlineto{\pgfqpoint{5.326381in}{3.148118in}}%
\pgfpathlineto{\pgfqpoint{5.318501in}{3.136854in}}%
\pgfpathclose%
\pgfusepath{fill}%
\end{pgfscope}%
\begin{pgfscope}%
\pgfpathrectangle{\pgfqpoint{1.150000in}{0.150000in}}{\pgfqpoint{5.700000in}{5.700000in}}%
\pgfusepath{clip}%
\pgfsetbuttcap%
\pgfsetroundjoin%
\definecolor{currentfill}{rgb}{0.119423,0.611141,0.538982}%
\pgfsetfillcolor{currentfill}%
\pgfsetfillopacity{0.800000}%
\pgfsetlinewidth{0.000000pt}%
\definecolor{currentstroke}{rgb}{0.000000,0.000000,0.000000}%
\pgfsetstrokecolor{currentstroke}%
\pgfsetdash{}{0pt}%
\pgfpathmoveto{\pgfqpoint{4.914097in}{2.527925in}}%
\pgfpathlineto{\pgfqpoint{4.928975in}{2.543791in}}%
\pgfpathlineto{\pgfqpoint{4.943874in}{2.559845in}}%
\pgfpathlineto{\pgfqpoint{4.958794in}{2.576089in}}%
\pgfpathlineto{\pgfqpoint{4.973736in}{2.592522in}}%
\pgfpathlineto{\pgfqpoint{4.981839in}{2.609320in}}%
\pgfpathlineto{\pgfqpoint{4.989936in}{2.625938in}}%
\pgfpathlineto{\pgfqpoint{4.998026in}{2.642375in}}%
\pgfpathlineto{\pgfqpoint{5.006109in}{2.658628in}}%
\pgfpathlineto{\pgfqpoint{4.991156in}{2.641924in}}%
\pgfpathlineto{\pgfqpoint{4.976226in}{2.625410in}}%
\pgfpathlineto{\pgfqpoint{4.961317in}{2.609085in}}%
\pgfpathlineto{\pgfqpoint{4.946429in}{2.592950in}}%
\pgfpathlineto{\pgfqpoint{4.938356in}{2.576954in}}%
\pgfpathlineto{\pgfqpoint{4.930276in}{2.560783in}}%
\pgfpathlineto{\pgfqpoint{4.922190in}{2.544439in}}%
\pgfpathlineto{\pgfqpoint{4.914097in}{2.527925in}}%
\pgfpathclose%
\pgfusepath{fill}%
\end{pgfscope}%
\begin{pgfscope}%
\pgfpathrectangle{\pgfqpoint{1.150000in}{0.150000in}}{\pgfqpoint{5.700000in}{5.700000in}}%
\pgfusepath{clip}%
\pgfsetbuttcap%
\pgfsetroundjoin%
\definecolor{currentfill}{rgb}{0.141935,0.526453,0.555991}%
\pgfsetfillcolor{currentfill}%
\pgfsetfillopacity{0.800000}%
\pgfsetlinewidth{0.000000pt}%
\definecolor{currentstroke}{rgb}{0.000000,0.000000,0.000000}%
\pgfsetstrokecolor{currentstroke}%
\pgfsetdash{}{0pt}%
\pgfpathmoveto{\pgfqpoint{4.757358in}{2.259871in}}%
\pgfpathlineto{\pgfqpoint{4.772121in}{2.273973in}}%
\pgfpathlineto{\pgfqpoint{4.786904in}{2.288261in}}%
\pgfpathlineto{\pgfqpoint{4.801707in}{2.302734in}}%
\pgfpathlineto{\pgfqpoint{4.816529in}{2.317394in}}%
\pgfpathlineto{\pgfqpoint{4.824689in}{2.335746in}}%
\pgfpathlineto{\pgfqpoint{4.832845in}{2.353962in}}%
\pgfpathlineto{\pgfqpoint{4.840995in}{2.372038in}}%
\pgfpathlineto{\pgfqpoint{4.849140in}{2.389971in}}%
\pgfpathlineto{\pgfqpoint{4.834306in}{2.374937in}}%
\pgfpathlineto{\pgfqpoint{4.819491in}{2.360090in}}%
\pgfpathlineto{\pgfqpoint{4.804697in}{2.345430in}}%
\pgfpathlineto{\pgfqpoint{4.789922in}{2.330956in}}%
\pgfpathlineto{\pgfqpoint{4.781789in}{2.313383in}}%
\pgfpathlineto{\pgfqpoint{4.773650in}{2.295676in}}%
\pgfpathlineto{\pgfqpoint{4.765507in}{2.277838in}}%
\pgfpathlineto{\pgfqpoint{4.757358in}{2.259871in}}%
\pgfpathclose%
\pgfusepath{fill}%
\end{pgfscope}%
\begin{pgfscope}%
\pgfpathrectangle{\pgfqpoint{1.150000in}{0.150000in}}{\pgfqpoint{5.700000in}{5.700000in}}%
\pgfusepath{clip}%
\pgfsetbuttcap%
\pgfsetroundjoin%
\definecolor{currentfill}{rgb}{0.175841,0.441290,0.557685}%
\pgfsetfillcolor{currentfill}%
\pgfsetfillopacity{0.800000}%
\pgfsetlinewidth{0.000000pt}%
\definecolor{currentstroke}{rgb}{0.000000,0.000000,0.000000}%
\pgfsetstrokecolor{currentstroke}%
\pgfsetdash{}{0pt}%
\pgfpathmoveto{\pgfqpoint{4.600504in}{1.986319in}}%
\pgfpathlineto{\pgfqpoint{4.615158in}{1.998370in}}%
\pgfpathlineto{\pgfqpoint{4.629829in}{2.010603in}}%
\pgfpathlineto{\pgfqpoint{4.644518in}{2.023018in}}%
\pgfpathlineto{\pgfqpoint{4.659226in}{2.035617in}}%
\pgfpathlineto{\pgfqpoint{4.667426in}{2.054830in}}%
\pgfpathlineto{\pgfqpoint{4.675623in}{2.073961in}}%
\pgfpathlineto{\pgfqpoint{4.683816in}{2.093006in}}%
\pgfpathlineto{\pgfqpoint{4.692004in}{2.111961in}}%
\pgfpathlineto{\pgfqpoint{4.677284in}{2.098890in}}%
\pgfpathlineto{\pgfqpoint{4.662583in}{2.086002in}}%
\pgfpathlineto{\pgfqpoint{4.647900in}{2.073298in}}%
\pgfpathlineto{\pgfqpoint{4.633235in}{2.060778in}}%
\pgfpathlineto{\pgfqpoint{4.625058in}{2.042282in}}%
\pgfpathlineto{\pgfqpoint{4.616878in}{2.023704in}}%
\pgfpathlineto{\pgfqpoint{4.608693in}{2.005048in}}%
\pgfpathlineto{\pgfqpoint{4.600504in}{1.986319in}}%
\pgfpathclose%
\pgfusepath{fill}%
\end{pgfscope}%
\begin{pgfscope}%
\pgfpathrectangle{\pgfqpoint{1.150000in}{0.150000in}}{\pgfqpoint{5.700000in}{5.700000in}}%
\pgfusepath{clip}%
\pgfsetbuttcap%
\pgfsetroundjoin%
\definecolor{currentfill}{rgb}{0.216210,0.351535,0.550627}%
\pgfsetfillcolor{currentfill}%
\pgfsetfillopacity{0.800000}%
\pgfsetlinewidth{0.000000pt}%
\definecolor{currentstroke}{rgb}{0.000000,0.000000,0.000000}%
\pgfsetstrokecolor{currentstroke}%
\pgfsetdash{}{0pt}%
\pgfpathmoveto{\pgfqpoint{4.443656in}{1.717537in}}%
\pgfpathlineto{\pgfqpoint{4.458210in}{1.727261in}}%
\pgfpathlineto{\pgfqpoint{4.472781in}{1.737165in}}%
\pgfpathlineto{\pgfqpoint{4.487367in}{1.747247in}}%
\pgfpathlineto{\pgfqpoint{4.501970in}{1.757510in}}%
\pgfpathlineto{\pgfqpoint{4.510199in}{1.776757in}}%
\pgfpathlineto{\pgfqpoint{4.518425in}{1.795989in}}%
\pgfpathlineto{\pgfqpoint{4.526648in}{1.815199in}}%
\pgfpathlineto{\pgfqpoint{4.534868in}{1.834383in}}%
\pgfpathlineto{\pgfqpoint{4.520254in}{1.823553in}}%
\pgfpathlineto{\pgfqpoint{4.505657in}{1.812904in}}%
\pgfpathlineto{\pgfqpoint{4.491077in}{1.802434in}}%
\pgfpathlineto{\pgfqpoint{4.476513in}{1.792145in}}%
\pgfpathlineto{\pgfqpoint{4.468304in}{1.773515in}}%
\pgfpathlineto{\pgfqpoint{4.460091in}{1.754867in}}%
\pgfpathlineto{\pgfqpoint{4.451875in}{1.736206in}}%
\pgfpathlineto{\pgfqpoint{4.443656in}{1.717537in}}%
\pgfpathclose%
\pgfusepath{fill}%
\end{pgfscope}%
\begin{pgfscope}%
\pgfpathrectangle{\pgfqpoint{1.150000in}{0.150000in}}{\pgfqpoint{5.700000in}{5.700000in}}%
\pgfusepath{clip}%
\pgfsetbuttcap%
\pgfsetroundjoin%
\definecolor{currentfill}{rgb}{0.146616,0.673050,0.508936}%
\pgfsetfillcolor{currentfill}%
\pgfsetfillopacity{0.800000}%
\pgfsetlinewidth{0.000000pt}%
\definecolor{currentstroke}{rgb}{0.000000,0.000000,0.000000}%
\pgfsetstrokecolor{currentstroke}%
\pgfsetdash{}{0pt}%
\pgfpathmoveto{\pgfqpoint{5.038370in}{2.721758in}}%
\pgfpathlineto{\pgfqpoint{5.053354in}{2.738888in}}%
\pgfpathlineto{\pgfqpoint{5.068359in}{2.756208in}}%
\pgfpathlineto{\pgfqpoint{5.083388in}{2.773720in}}%
\pgfpathlineto{\pgfqpoint{5.098439in}{2.791423in}}%
\pgfpathlineto{\pgfqpoint{5.106495in}{2.806939in}}%
\pgfpathlineto{\pgfqpoint{5.114543in}{2.822252in}}%
\pgfpathlineto{\pgfqpoint{5.122584in}{2.837358in}}%
\pgfpathlineto{\pgfqpoint{5.130616in}{2.852257in}}%
\pgfpathlineto{\pgfqpoint{5.115556in}{2.834353in}}%
\pgfpathlineto{\pgfqpoint{5.100519in}{2.816642in}}%
\pgfpathlineto{\pgfqpoint{5.085505in}{2.799122in}}%
\pgfpathlineto{\pgfqpoint{5.070513in}{2.781793in}}%
\pgfpathlineto{\pgfqpoint{5.062488in}{2.767080in}}%
\pgfpathlineto{\pgfqpoint{5.054457in}{2.752169in}}%
\pgfpathlineto{\pgfqpoint{5.046417in}{2.737061in}}%
\pgfpathlineto{\pgfqpoint{5.038370in}{2.721758in}}%
\pgfpathclose%
\pgfusepath{fill}%
\end{pgfscope}%
\begin{pgfscope}%
\pgfpathrectangle{\pgfqpoint{1.150000in}{0.150000in}}{\pgfqpoint{5.700000in}{5.700000in}}%
\pgfusepath{clip}%
\pgfsetbuttcap%
\pgfsetroundjoin%
\definecolor{currentfill}{rgb}{0.276194,0.190074,0.493001}%
\pgfsetfillcolor{currentfill}%
\pgfsetfillopacity{0.800000}%
\pgfsetlinewidth{0.000000pt}%
\definecolor{currentstroke}{rgb}{0.000000,0.000000,0.000000}%
\pgfsetstrokecolor{currentstroke}%
\pgfsetdash{}{0pt}%
\pgfpathmoveto{\pgfqpoint{4.162985in}{1.306165in}}%
\pgfpathlineto{\pgfqpoint{4.177397in}{1.311194in}}%
\pgfpathlineto{\pgfqpoint{4.191821in}{1.316398in}}%
\pgfpathlineto{\pgfqpoint{4.206258in}{1.321777in}}%
\pgfpathlineto{\pgfqpoint{4.220708in}{1.327330in}}%
\pgfpathlineto{\pgfqpoint{4.228984in}{1.344202in}}%
\pgfpathlineto{\pgfqpoint{4.237256in}{1.361195in}}%
\pgfpathlineto{\pgfqpoint{4.245524in}{1.378302in}}%
\pgfpathlineto{\pgfqpoint{4.253790in}{1.395518in}}%
\pgfpathlineto{\pgfqpoint{4.239337in}{1.389247in}}%
\pgfpathlineto{\pgfqpoint{4.224897in}{1.383152in}}%
\pgfpathlineto{\pgfqpoint{4.210470in}{1.377232in}}%
\pgfpathlineto{\pgfqpoint{4.196057in}{1.371487in}}%
\pgfpathlineto{\pgfqpoint{4.187794in}{1.354977in}}%
\pgfpathlineto{\pgfqpoint{4.179528in}{1.338581in}}%
\pgfpathlineto{\pgfqpoint{4.171259in}{1.322308in}}%
\pgfpathlineto{\pgfqpoint{4.162985in}{1.306165in}}%
\pgfpathclose%
\pgfusepath{fill}%
\end{pgfscope}%
\begin{pgfscope}%
\pgfpathrectangle{\pgfqpoint{1.150000in}{0.150000in}}{\pgfqpoint{5.700000in}{5.700000in}}%
\pgfusepath{clip}%
\pgfsetbuttcap%
\pgfsetroundjoin%
\definecolor{currentfill}{rgb}{0.282656,0.100196,0.422160}%
\pgfsetfillcolor{currentfill}%
\pgfsetfillopacity{0.800000}%
\pgfsetlinewidth{0.000000pt}%
\definecolor{currentstroke}{rgb}{0.000000,0.000000,0.000000}%
\pgfsetstrokecolor{currentstroke}%
\pgfsetdash{}{0pt}%
\pgfpathmoveto{\pgfqpoint{3.948430in}{1.109720in}}%
\pgfpathlineto{\pgfqpoint{3.962765in}{1.111038in}}%
\pgfpathlineto{\pgfqpoint{3.977109in}{1.112528in}}%
\pgfpathlineto{\pgfqpoint{3.991463in}{1.114192in}}%
\pgfpathlineto{\pgfqpoint{4.005826in}{1.116029in}}%
\pgfpathlineto{\pgfqpoint{4.014156in}{1.129244in}}%
\pgfpathlineto{\pgfqpoint{4.022481in}{1.142684in}}%
\pgfpathlineto{\pgfqpoint{4.030800in}{1.156342in}}%
\pgfpathlineto{\pgfqpoint{4.039114in}{1.170210in}}%
\pgfpathlineto{\pgfqpoint{4.024756in}{1.167569in}}%
\pgfpathlineto{\pgfqpoint{4.010409in}{1.165102in}}%
\pgfpathlineto{\pgfqpoint{3.996072in}{1.162808in}}%
\pgfpathlineto{\pgfqpoint{3.981745in}{1.160688in}}%
\pgfpathlineto{\pgfqpoint{3.973424in}{1.147612in}}%
\pgfpathlineto{\pgfqpoint{3.965099in}{1.134754in}}%
\pgfpathlineto{\pgfqpoint{3.956767in}{1.122121in}}%
\pgfpathlineto{\pgfqpoint{3.948430in}{1.109720in}}%
\pgfpathclose%
\pgfusepath{fill}%
\end{pgfscope}%
\begin{pgfscope}%
\pgfpathrectangle{\pgfqpoint{1.150000in}{0.150000in}}{\pgfqpoint{5.700000in}{5.700000in}}%
\pgfusepath{clip}%
\pgfsetbuttcap%
\pgfsetroundjoin%
\definecolor{currentfill}{rgb}{0.257322,0.256130,0.526563}%
\pgfsetfillcolor{currentfill}%
\pgfsetfillopacity{0.800000}%
\pgfsetlinewidth{0.000000pt}%
\definecolor{currentstroke}{rgb}{0.000000,0.000000,0.000000}%
\pgfsetstrokecolor{currentstroke}%
\pgfsetdash{}{0pt}%
\pgfpathmoveto{\pgfqpoint{4.286818in}{1.465324in}}%
\pgfpathlineto{\pgfqpoint{4.301290in}{1.472458in}}%
\pgfpathlineto{\pgfqpoint{4.315775in}{1.479768in}}%
\pgfpathlineto{\pgfqpoint{4.330275in}{1.487254in}}%
\pgfpathlineto{\pgfqpoint{4.344789in}{1.494916in}}%
\pgfpathlineto{\pgfqpoint{4.353044in}{1.513241in}}%
\pgfpathlineto{\pgfqpoint{4.361297in}{1.531628in}}%
\pgfpathlineto{\pgfqpoint{4.369546in}{1.550069in}}%
\pgfpathlineto{\pgfqpoint{4.377793in}{1.568560in}}%
\pgfpathlineto{\pgfqpoint{4.363271in}{1.560238in}}%
\pgfpathlineto{\pgfqpoint{4.348765in}{1.552094in}}%
\pgfpathlineto{\pgfqpoint{4.334273in}{1.544127in}}%
\pgfpathlineto{\pgfqpoint{4.319795in}{1.536337in}}%
\pgfpathlineto{\pgfqpoint{4.311556in}{1.518493in}}%
\pgfpathlineto{\pgfqpoint{4.303313in}{1.500705in}}%
\pgfpathlineto{\pgfqpoint{4.295067in}{1.482980in}}%
\pgfpathlineto{\pgfqpoint{4.286818in}{1.465324in}}%
\pgfpathclose%
\pgfusepath{fill}%
\end{pgfscope}%
\begin{pgfscope}%
\pgfpathrectangle{\pgfqpoint{1.150000in}{0.150000in}}{\pgfqpoint{5.700000in}{5.700000in}}%
\pgfusepath{clip}%
\pgfsetbuttcap%
\pgfsetroundjoin%
\definecolor{currentfill}{rgb}{0.595839,0.848717,0.243329}%
\pgfsetfillcolor{currentfill}%
\pgfsetfillopacity{0.800000}%
\pgfsetlinewidth{0.000000pt}%
\definecolor{currentstroke}{rgb}{0.000000,0.000000,0.000000}%
\pgfsetstrokecolor{currentstroke}%
\pgfsetdash{}{0pt}%
\pgfpathmoveto{\pgfqpoint{5.534644in}{3.416498in}}%
\pgfpathlineto{\pgfqpoint{5.550043in}{3.437307in}}%
\pgfpathlineto{\pgfqpoint{5.565469in}{3.458313in}}%
\pgfpathlineto{\pgfqpoint{5.580921in}{3.479516in}}%
\pgfpathlineto{\pgfqpoint{5.588662in}{3.487857in}}%
\pgfpathlineto{\pgfqpoint{5.596390in}{3.495952in}}%
\pgfpathlineto{\pgfqpoint{5.604105in}{3.503806in}}%
\pgfpathlineto{\pgfqpoint{5.611808in}{3.511417in}}%
\pgfpathlineto{\pgfqpoint{5.596361in}{3.490312in}}%
\pgfpathlineto{\pgfqpoint{5.580940in}{3.469405in}}%
\pgfpathlineto{\pgfqpoint{5.565546in}{3.448695in}}%
\pgfpathlineto{\pgfqpoint{5.557839in}{3.440998in}}%
\pgfpathlineto{\pgfqpoint{5.550119in}{3.433068in}}%
\pgfpathlineto{\pgfqpoint{5.542388in}{3.424902in}}%
\pgfpathlineto{\pgfqpoint{5.534644in}{3.416498in}}%
\pgfpathclose%
\pgfusepath{fill}%
\end{pgfscope}%
\begin{pgfscope}%
\pgfpathrectangle{\pgfqpoint{1.150000in}{0.150000in}}{\pgfqpoint{5.700000in}{5.700000in}}%
\pgfusepath{clip}%
\pgfsetbuttcap%
\pgfsetroundjoin%
\definecolor{currentfill}{rgb}{0.120565,0.596422,0.543611}%
\pgfsetfillcolor{currentfill}%
\pgfsetfillopacity{0.800000}%
\pgfsetlinewidth{0.000000pt}%
\definecolor{currentstroke}{rgb}{0.000000,0.000000,0.000000}%
\pgfsetstrokecolor{currentstroke}%
\pgfsetdash{}{0pt}%
\pgfpathmoveto{\pgfqpoint{4.881666in}{2.460210in}}%
\pgfpathlineto{\pgfqpoint{4.896532in}{2.475771in}}%
\pgfpathlineto{\pgfqpoint{4.911420in}{2.491520in}}%
\pgfpathlineto{\pgfqpoint{4.926329in}{2.507458in}}%
\pgfpathlineto{\pgfqpoint{4.941259in}{2.523584in}}%
\pgfpathlineto{\pgfqpoint{4.949387in}{2.541075in}}%
\pgfpathlineto{\pgfqpoint{4.957510in}{2.558397in}}%
\pgfpathlineto{\pgfqpoint{4.965626in}{2.575547in}}%
\pgfpathlineto{\pgfqpoint{4.973736in}{2.592522in}}%
\pgfpathlineto{\pgfqpoint{4.958794in}{2.576089in}}%
\pgfpathlineto{\pgfqpoint{4.943874in}{2.559845in}}%
\pgfpathlineto{\pgfqpoint{4.928975in}{2.543791in}}%
\pgfpathlineto{\pgfqpoint{4.914097in}{2.527925in}}%
\pgfpathlineto{\pgfqpoint{4.905999in}{2.511242in}}%
\pgfpathlineto{\pgfqpoint{4.897894in}{2.494394in}}%
\pgfpathlineto{\pgfqpoint{4.889783in}{2.477382in}}%
\pgfpathlineto{\pgfqpoint{4.881666in}{2.460210in}}%
\pgfpathclose%
\pgfusepath{fill}%
\end{pgfscope}%
\begin{pgfscope}%
\pgfpathrectangle{\pgfqpoint{1.150000in}{0.150000in}}{\pgfqpoint{5.700000in}{5.700000in}}%
\pgfusepath{clip}%
\pgfsetbuttcap%
\pgfsetroundjoin%
\definecolor{currentfill}{rgb}{0.185556,0.418570,0.556753}%
\pgfsetfillcolor{currentfill}%
\pgfsetfillopacity{0.800000}%
\pgfsetlinewidth{0.000000pt}%
\definecolor{currentstroke}{rgb}{0.000000,0.000000,0.000000}%
\pgfsetstrokecolor{currentstroke}%
\pgfsetdash{}{0pt}%
\pgfpathmoveto{\pgfqpoint{4.567714in}{1.910757in}}%
\pgfpathlineto{\pgfqpoint{4.582356in}{1.922304in}}%
\pgfpathlineto{\pgfqpoint{4.597015in}{1.934033in}}%
\pgfpathlineto{\pgfqpoint{4.611692in}{1.945943in}}%
\pgfpathlineto{\pgfqpoint{4.626387in}{1.958035in}}%
\pgfpathlineto{\pgfqpoint{4.634602in}{1.977531in}}%
\pgfpathlineto{\pgfqpoint{4.642814in}{1.996963in}}%
\pgfpathlineto{\pgfqpoint{4.651022in}{2.016326in}}%
\pgfpathlineto{\pgfqpoint{4.659226in}{2.035617in}}%
\pgfpathlineto{\pgfqpoint{4.644518in}{2.023018in}}%
\pgfpathlineto{\pgfqpoint{4.629829in}{2.010603in}}%
\pgfpathlineto{\pgfqpoint{4.615158in}{1.998370in}}%
\pgfpathlineto{\pgfqpoint{4.600504in}{1.986319in}}%
\pgfpathlineto{\pgfqpoint{4.592312in}{1.967521in}}%
\pgfpathlineto{\pgfqpoint{4.584116in}{1.948659in}}%
\pgfpathlineto{\pgfqpoint{4.575917in}{1.929736in}}%
\pgfpathlineto{\pgfqpoint{4.567714in}{1.910757in}}%
\pgfpathclose%
\pgfusepath{fill}%
\end{pgfscope}%
\begin{pgfscope}%
\pgfpathrectangle{\pgfqpoint{1.150000in}{0.150000in}}{\pgfqpoint{5.700000in}{5.700000in}}%
\pgfusepath{clip}%
\pgfsetbuttcap%
\pgfsetroundjoin%
\definecolor{currentfill}{rgb}{0.149039,0.508051,0.557250}%
\pgfsetfillcolor{currentfill}%
\pgfsetfillopacity{0.800000}%
\pgfsetlinewidth{0.000000pt}%
\definecolor{currentstroke}{rgb}{0.000000,0.000000,0.000000}%
\pgfsetstrokecolor{currentstroke}%
\pgfsetdash{}{0pt}%
\pgfpathmoveto{\pgfqpoint{4.724717in}{2.186796in}}%
\pgfpathlineto{\pgfqpoint{4.739468in}{2.200492in}}%
\pgfpathlineto{\pgfqpoint{4.754238in}{2.214373in}}%
\pgfpathlineto{\pgfqpoint{4.769028in}{2.228438in}}%
\pgfpathlineto{\pgfqpoint{4.783837in}{2.242689in}}%
\pgfpathlineto{\pgfqpoint{4.792017in}{2.261553in}}%
\pgfpathlineto{\pgfqpoint{4.800193in}{2.280294in}}%
\pgfpathlineto{\pgfqpoint{4.808363in}{2.298909in}}%
\pgfpathlineto{\pgfqpoint{4.816529in}{2.317394in}}%
\pgfpathlineto{\pgfqpoint{4.801707in}{2.302734in}}%
\pgfpathlineto{\pgfqpoint{4.786904in}{2.288261in}}%
\pgfpathlineto{\pgfqpoint{4.772121in}{2.273973in}}%
\pgfpathlineto{\pgfqpoint{4.757358in}{2.259871in}}%
\pgfpathlineto{\pgfqpoint{4.749205in}{2.241781in}}%
\pgfpathlineto{\pgfqpoint{4.741047in}{2.223569in}}%
\pgfpathlineto{\pgfqpoint{4.732884in}{2.205240in}}%
\pgfpathlineto{\pgfqpoint{4.724717in}{2.186796in}}%
\pgfpathclose%
\pgfusepath{fill}%
\end{pgfscope}%
\begin{pgfscope}%
\pgfpathrectangle{\pgfqpoint{1.150000in}{0.150000in}}{\pgfqpoint{5.700000in}{5.700000in}}%
\pgfusepath{clip}%
\pgfsetbuttcap%
\pgfsetroundjoin%
\definecolor{currentfill}{rgb}{0.232815,0.732247,0.459277}%
\pgfsetfillcolor{currentfill}%
\pgfsetfillopacity{0.800000}%
\pgfsetlinewidth{0.000000pt}%
\definecolor{currentstroke}{rgb}{0.000000,0.000000,0.000000}%
\pgfsetstrokecolor{currentstroke}%
\pgfsetdash{}{0pt}%
\pgfpathmoveto{\pgfqpoint{5.162662in}{2.909752in}}%
\pgfpathlineto{\pgfqpoint{5.177752in}{2.928012in}}%
\pgfpathlineto{\pgfqpoint{5.192866in}{2.946465in}}%
\pgfpathlineto{\pgfqpoint{5.208003in}{2.965111in}}%
\pgfpathlineto{\pgfqpoint{5.223164in}{2.983951in}}%
\pgfpathlineto{\pgfqpoint{5.231161in}{2.997935in}}%
\pgfpathlineto{\pgfqpoint{5.239149in}{3.011695in}}%
\pgfpathlineto{\pgfqpoint{5.247127in}{3.025231in}}%
\pgfpathlineto{\pgfqpoint{5.255097in}{3.038542in}}%
\pgfpathlineto{\pgfqpoint{5.239929in}{3.019574in}}%
\pgfpathlineto{\pgfqpoint{5.224786in}{3.000800in}}%
\pgfpathlineto{\pgfqpoint{5.209666in}{2.982220in}}%
\pgfpathlineto{\pgfqpoint{5.194569in}{2.963832in}}%
\pgfpathlineto{\pgfqpoint{5.186606in}{2.950635in}}%
\pgfpathlineto{\pgfqpoint{5.178633in}{2.937222in}}%
\pgfpathlineto{\pgfqpoint{5.170652in}{2.923594in}}%
\pgfpathlineto{\pgfqpoint{5.162662in}{2.909752in}}%
\pgfpathclose%
\pgfusepath{fill}%
\end{pgfscope}%
\begin{pgfscope}%
\pgfpathrectangle{\pgfqpoint{1.150000in}{0.150000in}}{\pgfqpoint{5.700000in}{5.700000in}}%
\pgfusepath{clip}%
\pgfsetbuttcap%
\pgfsetroundjoin%
\definecolor{currentfill}{rgb}{0.283072,0.130895,0.449241}%
\pgfsetfillcolor{currentfill}%
\pgfsetfillopacity{0.800000}%
\pgfsetlinewidth{0.000000pt}%
\definecolor{currentstroke}{rgb}{0.000000,0.000000,0.000000}%
\pgfsetstrokecolor{currentstroke}%
\pgfsetdash{}{0pt}%
\pgfpathmoveto{\pgfqpoint{4.039114in}{1.170210in}}%
\pgfpathlineto{\pgfqpoint{4.053483in}{1.173024in}}%
\pgfpathlineto{\pgfqpoint{4.067862in}{1.176011in}}%
\pgfpathlineto{\pgfqpoint{4.082253in}{1.179172in}}%
\pgfpathlineto{\pgfqpoint{4.096654in}{1.182506in}}%
\pgfpathlineto{\pgfqpoint{4.104960in}{1.197362in}}%
\pgfpathlineto{\pgfqpoint{4.113262in}{1.212404in}}%
\pgfpathlineto{\pgfqpoint{4.121559in}{1.227626in}}%
\pgfpathlineto{\pgfqpoint{4.129853in}{1.243019in}}%
\pgfpathlineto{\pgfqpoint{4.115453in}{1.238909in}}%
\pgfpathlineto{\pgfqpoint{4.101065in}{1.234973in}}%
\pgfpathlineto{\pgfqpoint{4.086688in}{1.231211in}}%
\pgfpathlineto{\pgfqpoint{4.072323in}{1.227624in}}%
\pgfpathlineto{\pgfqpoint{4.064028in}{1.212993in}}%
\pgfpathlineto{\pgfqpoint{4.055728in}{1.198543in}}%
\pgfpathlineto{\pgfqpoint{4.047423in}{1.184279in}}%
\pgfpathlineto{\pgfqpoint{4.039114in}{1.170210in}}%
\pgfpathclose%
\pgfusepath{fill}%
\end{pgfscope}%
\begin{pgfscope}%
\pgfpathrectangle{\pgfqpoint{1.150000in}{0.150000in}}{\pgfqpoint{5.700000in}{5.700000in}}%
\pgfusepath{clip}%
\pgfsetbuttcap%
\pgfsetroundjoin%
\definecolor{currentfill}{rgb}{0.227802,0.326594,0.546532}%
\pgfsetfillcolor{currentfill}%
\pgfsetfillopacity{0.800000}%
\pgfsetlinewidth{0.000000pt}%
\definecolor{currentstroke}{rgb}{0.000000,0.000000,0.000000}%
\pgfsetstrokecolor{currentstroke}%
\pgfsetdash{}{0pt}%
\pgfpathmoveto{\pgfqpoint{4.410749in}{1.642891in}}%
\pgfpathlineto{\pgfqpoint{4.425294in}{1.652018in}}%
\pgfpathlineto{\pgfqpoint{4.439854in}{1.661324in}}%
\pgfpathlineto{\pgfqpoint{4.454431in}{1.670808in}}%
\pgfpathlineto{\pgfqpoint{4.469023in}{1.680471in}}%
\pgfpathlineto{\pgfqpoint{4.477264in}{1.699727in}}%
\pgfpathlineto{\pgfqpoint{4.485502in}{1.718989in}}%
\pgfpathlineto{\pgfqpoint{4.493737in}{1.738252in}}%
\pgfpathlineto{\pgfqpoint{4.501970in}{1.757510in}}%
\pgfpathlineto{\pgfqpoint{4.487367in}{1.747247in}}%
\pgfpathlineto{\pgfqpoint{4.472781in}{1.737165in}}%
\pgfpathlineto{\pgfqpoint{4.458210in}{1.727261in}}%
\pgfpathlineto{\pgfqpoint{4.443656in}{1.717537in}}%
\pgfpathlineto{\pgfqpoint{4.435434in}{1.698866in}}%
\pgfpathlineto{\pgfqpoint{4.427209in}{1.680197in}}%
\pgfpathlineto{\pgfqpoint{4.418980in}{1.661537in}}%
\pgfpathlineto{\pgfqpoint{4.410749in}{1.642891in}}%
\pgfpathclose%
\pgfusepath{fill}%
\end{pgfscope}%
\begin{pgfscope}%
\pgfpathrectangle{\pgfqpoint{1.150000in}{0.150000in}}{\pgfqpoint{5.700000in}{5.700000in}}%
\pgfusepath{clip}%
\pgfsetbuttcap%
\pgfsetroundjoin%
\definecolor{currentfill}{rgb}{0.352360,0.783011,0.392636}%
\pgfsetfillcolor{currentfill}%
\pgfsetfillopacity{0.800000}%
\pgfsetlinewidth{0.000000pt}%
\definecolor{currentstroke}{rgb}{0.000000,0.000000,0.000000}%
\pgfsetstrokecolor{currentstroke}%
\pgfsetdash{}{0pt}%
\pgfpathmoveto{\pgfqpoint{5.286878in}{3.089519in}}%
\pgfpathlineto{\pgfqpoint{5.302074in}{3.108772in}}%
\pgfpathlineto{\pgfqpoint{5.317295in}{3.128219in}}%
\pgfpathlineto{\pgfqpoint{5.332540in}{3.147862in}}%
\pgfpathlineto{\pgfqpoint{5.347810in}{3.167700in}}%
\pgfpathlineto{\pgfqpoint{5.355735in}{3.179944in}}%
\pgfpathlineto{\pgfqpoint{5.363649in}{3.191950in}}%
\pgfpathlineto{\pgfqpoint{5.371553in}{3.203719in}}%
\pgfpathlineto{\pgfqpoint{5.379446in}{3.215250in}}%
\pgfpathlineto{\pgfqpoint{5.364173in}{3.195358in}}%
\pgfpathlineto{\pgfqpoint{5.348924in}{3.175662in}}%
\pgfpathlineto{\pgfqpoint{5.333700in}{3.156161in}}%
\pgfpathlineto{\pgfqpoint{5.318501in}{3.136854in}}%
\pgfpathlineto{\pgfqpoint{5.310610in}{3.125363in}}%
\pgfpathlineto{\pgfqpoint{5.302709in}{3.113643in}}%
\pgfpathlineto{\pgfqpoint{5.294798in}{3.101695in}}%
\pgfpathlineto{\pgfqpoint{5.286878in}{3.089519in}}%
\pgfpathclose%
\pgfusepath{fill}%
\end{pgfscope}%
\begin{pgfscope}%
\pgfpathrectangle{\pgfqpoint{1.150000in}{0.150000in}}{\pgfqpoint{5.700000in}{5.700000in}}%
\pgfusepath{clip}%
\pgfsetbuttcap%
\pgfsetroundjoin%
\definecolor{currentfill}{rgb}{0.477504,0.821444,0.318195}%
\pgfsetfillcolor{currentfill}%
\pgfsetfillopacity{0.800000}%
\pgfsetlinewidth{0.000000pt}%
\definecolor{currentstroke}{rgb}{0.000000,0.000000,0.000000}%
\pgfsetstrokecolor{currentstroke}%
\pgfsetdash{}{0pt}%
\pgfpathmoveto{\pgfqpoint{5.410910in}{3.259004in}}%
\pgfpathlineto{\pgfqpoint{5.426210in}{3.279108in}}%
\pgfpathlineto{\pgfqpoint{5.441535in}{3.299408in}}%
\pgfpathlineto{\pgfqpoint{5.456886in}{3.319904in}}%
\pgfpathlineto{\pgfqpoint{5.472263in}{3.340598in}}%
\pgfpathlineto{\pgfqpoint{5.480102in}{3.350938in}}%
\pgfpathlineto{\pgfqpoint{5.487929in}{3.361032in}}%
\pgfpathlineto{\pgfqpoint{5.495744in}{3.370882in}}%
\pgfpathlineto{\pgfqpoint{5.503548in}{3.380489in}}%
\pgfpathlineto{\pgfqpoint{5.488172in}{3.359818in}}%
\pgfpathlineto{\pgfqpoint{5.472822in}{3.339343in}}%
\pgfpathlineto{\pgfqpoint{5.457497in}{3.319065in}}%
\pgfpathlineto{\pgfqpoint{5.442198in}{3.298983in}}%
\pgfpathlineto{\pgfqpoint{5.434393in}{3.289340in}}%
\pgfpathlineto{\pgfqpoint{5.426577in}{3.279463in}}%
\pgfpathlineto{\pgfqpoint{5.418749in}{3.269352in}}%
\pgfpathlineto{\pgfqpoint{5.410910in}{3.259004in}}%
\pgfpathclose%
\pgfusepath{fill}%
\end{pgfscope}%
\begin{pgfscope}%
\pgfpathrectangle{\pgfqpoint{1.150000in}{0.150000in}}{\pgfqpoint{5.700000in}{5.700000in}}%
\pgfusepath{clip}%
\pgfsetbuttcap%
\pgfsetroundjoin%
\definecolor{currentfill}{rgb}{0.265145,0.232956,0.516599}%
\pgfsetfillcolor{currentfill}%
\pgfsetfillopacity{0.800000}%
\pgfsetlinewidth{0.000000pt}%
\definecolor{currentstroke}{rgb}{0.000000,0.000000,0.000000}%
\pgfsetstrokecolor{currentstroke}%
\pgfsetdash{}{0pt}%
\pgfpathmoveto{\pgfqpoint{4.253790in}{1.395518in}}%
\pgfpathlineto{\pgfqpoint{4.268256in}{1.401964in}}%
\pgfpathlineto{\pgfqpoint{4.282736in}{1.408585in}}%
\pgfpathlineto{\pgfqpoint{4.297230in}{1.415382in}}%
\pgfpathlineto{\pgfqpoint{4.311738in}{1.422355in}}%
\pgfpathlineto{\pgfqpoint{4.320005in}{1.440371in}}%
\pgfpathlineto{\pgfqpoint{4.328269in}{1.458475in}}%
\pgfpathlineto{\pgfqpoint{4.336531in}{1.476659in}}%
\pgfpathlineto{\pgfqpoint{4.344789in}{1.494916in}}%
\pgfpathlineto{\pgfqpoint{4.330275in}{1.487254in}}%
\pgfpathlineto{\pgfqpoint{4.315775in}{1.479768in}}%
\pgfpathlineto{\pgfqpoint{4.301290in}{1.472458in}}%
\pgfpathlineto{\pgfqpoint{4.286818in}{1.465324in}}%
\pgfpathlineto{\pgfqpoint{4.278566in}{1.447744in}}%
\pgfpathlineto{\pgfqpoint{4.270310in}{1.430245in}}%
\pgfpathlineto{\pgfqpoint{4.262052in}{1.412834in}}%
\pgfpathlineto{\pgfqpoint{4.253790in}{1.395518in}}%
\pgfpathclose%
\pgfusepath{fill}%
\end{pgfscope}%
\begin{pgfscope}%
\pgfpathrectangle{\pgfqpoint{1.150000in}{0.150000in}}{\pgfqpoint{5.700000in}{5.700000in}}%
\pgfusepath{clip}%
\pgfsetbuttcap%
\pgfsetroundjoin%
\definecolor{currentfill}{rgb}{0.137339,0.662252,0.515571}%
\pgfsetfillcolor{currentfill}%
\pgfsetfillopacity{0.800000}%
\pgfsetlinewidth{0.000000pt}%
\definecolor{currentstroke}{rgb}{0.000000,0.000000,0.000000}%
\pgfsetstrokecolor{currentstroke}%
\pgfsetdash{}{0pt}%
\pgfpathmoveto{\pgfqpoint{5.006109in}{2.658628in}}%
\pgfpathlineto{\pgfqpoint{5.021083in}{2.675522in}}%
\pgfpathlineto{\pgfqpoint{5.036079in}{2.692606in}}%
\pgfpathlineto{\pgfqpoint{5.051097in}{2.709881in}}%
\pgfpathlineto{\pgfqpoint{5.066138in}{2.727348in}}%
\pgfpathlineto{\pgfqpoint{5.074224in}{2.743664in}}%
\pgfpathlineto{\pgfqpoint{5.082303in}{2.759783in}}%
\pgfpathlineto{\pgfqpoint{5.090375in}{2.775704in}}%
\pgfpathlineto{\pgfqpoint{5.098439in}{2.791423in}}%
\pgfpathlineto{\pgfqpoint{5.083388in}{2.773720in}}%
\pgfpathlineto{\pgfqpoint{5.068359in}{2.756208in}}%
\pgfpathlineto{\pgfqpoint{5.053354in}{2.738888in}}%
\pgfpathlineto{\pgfqpoint{5.038370in}{2.721758in}}%
\pgfpathlineto{\pgfqpoint{5.030315in}{2.706261in}}%
\pgfpathlineto{\pgfqpoint{5.022254in}{2.690573in}}%
\pgfpathlineto{\pgfqpoint{5.014185in}{2.674694in}}%
\pgfpathlineto{\pgfqpoint{5.006109in}{2.658628in}}%
\pgfpathclose%
\pgfusepath{fill}%
\end{pgfscope}%
\begin{pgfscope}%
\pgfpathrectangle{\pgfqpoint{1.150000in}{0.150000in}}{\pgfqpoint{5.700000in}{5.700000in}}%
\pgfusepath{clip}%
\pgfsetbuttcap%
\pgfsetroundjoin%
\definecolor{currentfill}{rgb}{0.279574,0.170599,0.479997}%
\pgfsetfillcolor{currentfill}%
\pgfsetfillopacity{0.800000}%
\pgfsetlinewidth{0.000000pt}%
\definecolor{currentstroke}{rgb}{0.000000,0.000000,0.000000}%
\pgfsetstrokecolor{currentstroke}%
\pgfsetdash{}{0pt}%
\pgfpathmoveto{\pgfqpoint{4.129853in}{1.243019in}}%
\pgfpathlineto{\pgfqpoint{4.144264in}{1.247303in}}%
\pgfpathlineto{\pgfqpoint{4.158688in}{1.251760in}}%
\pgfpathlineto{\pgfqpoint{4.173123in}{1.256391in}}%
\pgfpathlineto{\pgfqpoint{4.187571in}{1.261196in}}%
\pgfpathlineto{\pgfqpoint{4.195861in}{1.277513in}}%
\pgfpathlineto{\pgfqpoint{4.204147in}{1.293979in}}%
\pgfpathlineto{\pgfqpoint{4.212429in}{1.310587in}}%
\pgfpathlineto{\pgfqpoint{4.220708in}{1.327330in}}%
\pgfpathlineto{\pgfqpoint{4.206258in}{1.321777in}}%
\pgfpathlineto{\pgfqpoint{4.191821in}{1.316398in}}%
\pgfpathlineto{\pgfqpoint{4.177397in}{1.311194in}}%
\pgfpathlineto{\pgfqpoint{4.162985in}{1.306165in}}%
\pgfpathlineto{\pgfqpoint{4.154708in}{1.290157in}}%
\pgfpathlineto{\pgfqpoint{4.146427in}{1.274292in}}%
\pgfpathlineto{\pgfqpoint{4.138142in}{1.258577in}}%
\pgfpathlineto{\pgfqpoint{4.129853in}{1.243019in}}%
\pgfpathclose%
\pgfusepath{fill}%
\end{pgfscope}%
\begin{pgfscope}%
\pgfpathrectangle{\pgfqpoint{1.150000in}{0.150000in}}{\pgfqpoint{5.700000in}{5.700000in}}%
\pgfusepath{clip}%
\pgfsetbuttcap%
\pgfsetroundjoin%
\definecolor{currentfill}{rgb}{0.194100,0.399323,0.555565}%
\pgfsetfillcolor{currentfill}%
\pgfsetfillopacity{0.800000}%
\pgfsetlinewidth{0.000000pt}%
\definecolor{currentstroke}{rgb}{0.000000,0.000000,0.000000}%
\pgfsetstrokecolor{currentstroke}%
\pgfsetdash{}{0pt}%
\pgfpathmoveto{\pgfqpoint{4.534868in}{1.834383in}}%
\pgfpathlineto{\pgfqpoint{4.549498in}{1.845394in}}%
\pgfpathlineto{\pgfqpoint{4.564146in}{1.856585in}}%
\pgfpathlineto{\pgfqpoint{4.578811in}{1.867958in}}%
\pgfpathlineto{\pgfqpoint{4.593493in}{1.879511in}}%
\pgfpathlineto{\pgfqpoint{4.601721in}{1.899214in}}%
\pgfpathlineto{\pgfqpoint{4.609947in}{1.918872in}}%
\pgfpathlineto{\pgfqpoint{4.618168in}{1.938481in}}%
\pgfpathlineto{\pgfqpoint{4.626387in}{1.958035in}}%
\pgfpathlineto{\pgfqpoint{4.611692in}{1.945943in}}%
\pgfpathlineto{\pgfqpoint{4.597015in}{1.934033in}}%
\pgfpathlineto{\pgfqpoint{4.582356in}{1.922304in}}%
\pgfpathlineto{\pgfqpoint{4.567714in}{1.910757in}}%
\pgfpathlineto{\pgfqpoint{4.559507in}{1.891728in}}%
\pgfpathlineto{\pgfqpoint{4.551297in}{1.872653in}}%
\pgfpathlineto{\pgfqpoint{4.543084in}{1.853536in}}%
\pgfpathlineto{\pgfqpoint{4.534868in}{1.834383in}}%
\pgfpathclose%
\pgfusepath{fill}%
\end{pgfscope}%
\begin{pgfscope}%
\pgfpathrectangle{\pgfqpoint{1.150000in}{0.150000in}}{\pgfqpoint{5.700000in}{5.700000in}}%
\pgfusepath{clip}%
\pgfsetbuttcap%
\pgfsetroundjoin%
\definecolor{currentfill}{rgb}{0.156270,0.489624,0.557936}%
\pgfsetfillcolor{currentfill}%
\pgfsetfillopacity{0.800000}%
\pgfsetlinewidth{0.000000pt}%
\definecolor{currentstroke}{rgb}{0.000000,0.000000,0.000000}%
\pgfsetstrokecolor{currentstroke}%
\pgfsetdash{}{0pt}%
\pgfpathmoveto{\pgfqpoint{4.692004in}{2.111961in}}%
\pgfpathlineto{\pgfqpoint{4.706743in}{2.125216in}}%
\pgfpathlineto{\pgfqpoint{4.721500in}{2.138656in}}%
\pgfpathlineto{\pgfqpoint{4.736277in}{2.152280in}}%
\pgfpathlineto{\pgfqpoint{4.751073in}{2.166088in}}%
\pgfpathlineto{\pgfqpoint{4.759270in}{2.185403in}}%
\pgfpathlineto{\pgfqpoint{4.767464in}{2.204611in}}%
\pgfpathlineto{\pgfqpoint{4.775653in}{2.223707in}}%
\pgfpathlineto{\pgfqpoint{4.783837in}{2.242689in}}%
\pgfpathlineto{\pgfqpoint{4.769028in}{2.228438in}}%
\pgfpathlineto{\pgfqpoint{4.754238in}{2.214373in}}%
\pgfpathlineto{\pgfqpoint{4.739468in}{2.200492in}}%
\pgfpathlineto{\pgfqpoint{4.724717in}{2.186796in}}%
\pgfpathlineto{\pgfqpoint{4.716545in}{2.168243in}}%
\pgfpathlineto{\pgfqpoint{4.708369in}{2.149584in}}%
\pgfpathlineto{\pgfqpoint{4.700189in}{2.130822in}}%
\pgfpathlineto{\pgfqpoint{4.692004in}{2.111961in}}%
\pgfpathclose%
\pgfusepath{fill}%
\end{pgfscope}%
\begin{pgfscope}%
\pgfpathrectangle{\pgfqpoint{1.150000in}{0.150000in}}{\pgfqpoint{5.700000in}{5.700000in}}%
\pgfusepath{clip}%
\pgfsetbuttcap%
\pgfsetroundjoin%
\definecolor{currentfill}{rgb}{0.124395,0.578002,0.548287}%
\pgfsetfillcolor{currentfill}%
\pgfsetfillopacity{0.800000}%
\pgfsetlinewidth{0.000000pt}%
\definecolor{currentstroke}{rgb}{0.000000,0.000000,0.000000}%
\pgfsetstrokecolor{currentstroke}%
\pgfsetdash{}{0pt}%
\pgfpathmoveto{\pgfqpoint{4.849140in}{2.389971in}}%
\pgfpathlineto{\pgfqpoint{4.863995in}{2.405192in}}%
\pgfpathlineto{\pgfqpoint{4.878871in}{2.420600in}}%
\pgfpathlineto{\pgfqpoint{4.893767in}{2.436196in}}%
\pgfpathlineto{\pgfqpoint{4.908684in}{2.451981in}}%
\pgfpathlineto{\pgfqpoint{4.916837in}{2.470122in}}%
\pgfpathlineto{\pgfqpoint{4.924983in}{2.488104in}}%
\pgfpathlineto{\pgfqpoint{4.933124in}{2.505926in}}%
\pgfpathlineto{\pgfqpoint{4.941259in}{2.523584in}}%
\pgfpathlineto{\pgfqpoint{4.926329in}{2.507458in}}%
\pgfpathlineto{\pgfqpoint{4.911420in}{2.491520in}}%
\pgfpathlineto{\pgfqpoint{4.896532in}{2.475771in}}%
\pgfpathlineto{\pgfqpoint{4.881666in}{2.460210in}}%
\pgfpathlineto{\pgfqpoint{4.873543in}{2.442880in}}%
\pgfpathlineto{\pgfqpoint{4.865414in}{2.425395in}}%
\pgfpathlineto{\pgfqpoint{4.857280in}{2.407757in}}%
\pgfpathlineto{\pgfqpoint{4.849140in}{2.389971in}}%
\pgfpathclose%
\pgfusepath{fill}%
\end{pgfscope}%
\begin{pgfscope}%
\pgfpathrectangle{\pgfqpoint{1.150000in}{0.150000in}}{\pgfqpoint{5.700000in}{5.700000in}}%
\pgfusepath{clip}%
\pgfsetbuttcap%
\pgfsetroundjoin%
\definecolor{currentfill}{rgb}{0.237441,0.305202,0.541921}%
\pgfsetfillcolor{currentfill}%
\pgfsetfillopacity{0.800000}%
\pgfsetlinewidth{0.000000pt}%
\definecolor{currentstroke}{rgb}{0.000000,0.000000,0.000000}%
\pgfsetstrokecolor{currentstroke}%
\pgfsetdash{}{0pt}%
\pgfpathmoveto{\pgfqpoint{4.377793in}{1.568560in}}%
\pgfpathlineto{\pgfqpoint{4.392329in}{1.577058in}}%
\pgfpathlineto{\pgfqpoint{4.406881in}{1.585735in}}%
\pgfpathlineto{\pgfqpoint{4.421447in}{1.594589in}}%
\pgfpathlineto{\pgfqpoint{4.436030in}{1.603620in}}%
\pgfpathlineto{\pgfqpoint{4.444282in}{1.622795in}}%
\pgfpathlineto{\pgfqpoint{4.452532in}{1.641999in}}%
\pgfpathlineto{\pgfqpoint{4.460779in}{1.661226in}}%
\pgfpathlineto{\pgfqpoint{4.469023in}{1.680471in}}%
\pgfpathlineto{\pgfqpoint{4.454431in}{1.670808in}}%
\pgfpathlineto{\pgfqpoint{4.439854in}{1.661324in}}%
\pgfpathlineto{\pgfqpoint{4.425294in}{1.652018in}}%
\pgfpathlineto{\pgfqpoint{4.410749in}{1.642891in}}%
\pgfpathlineto{\pgfqpoint{4.402514in}{1.624264in}}%
\pgfpathlineto{\pgfqpoint{4.394277in}{1.605663in}}%
\pgfpathlineto{\pgfqpoint{4.386036in}{1.587093in}}%
\pgfpathlineto{\pgfqpoint{4.377793in}{1.568560in}}%
\pgfpathclose%
\pgfusepath{fill}%
\end{pgfscope}%
\begin{pgfscope}%
\pgfpathrectangle{\pgfqpoint{1.150000in}{0.150000in}}{\pgfqpoint{5.700000in}{5.700000in}}%
\pgfusepath{clip}%
\pgfsetbuttcap%
\pgfsetroundjoin%
\definecolor{currentfill}{rgb}{0.214000,0.722114,0.469588}%
\pgfsetfillcolor{currentfill}%
\pgfsetfillopacity{0.800000}%
\pgfsetlinewidth{0.000000pt}%
\definecolor{currentstroke}{rgb}{0.000000,0.000000,0.000000}%
\pgfsetstrokecolor{currentstroke}%
\pgfsetdash{}{0pt}%
\pgfpathmoveto{\pgfqpoint{5.130616in}{2.852257in}}%
\pgfpathlineto{\pgfqpoint{5.145699in}{2.870353in}}%
\pgfpathlineto{\pgfqpoint{5.160805in}{2.888641in}}%
\pgfpathlineto{\pgfqpoint{5.175934in}{2.907123in}}%
\pgfpathlineto{\pgfqpoint{5.191087in}{2.925798in}}%
\pgfpathlineto{\pgfqpoint{5.199119in}{2.940666in}}%
\pgfpathlineto{\pgfqpoint{5.207143in}{2.955316in}}%
\pgfpathlineto{\pgfqpoint{5.215158in}{2.969744in}}%
\pgfpathlineto{\pgfqpoint{5.223164in}{2.983951in}}%
\pgfpathlineto{\pgfqpoint{5.208003in}{2.965111in}}%
\pgfpathlineto{\pgfqpoint{5.192866in}{2.946465in}}%
\pgfpathlineto{\pgfqpoint{5.177752in}{2.928012in}}%
\pgfpathlineto{\pgfqpoint{5.162662in}{2.909752in}}%
\pgfpathlineto{\pgfqpoint{5.154663in}{2.895696in}}%
\pgfpathlineto{\pgfqpoint{5.146656in}{2.881427in}}%
\pgfpathlineto{\pgfqpoint{5.138640in}{2.866947in}}%
\pgfpathlineto{\pgfqpoint{5.130616in}{2.852257in}}%
\pgfpathclose%
\pgfusepath{fill}%
\end{pgfscope}%
\begin{pgfscope}%
\pgfpathrectangle{\pgfqpoint{1.150000in}{0.150000in}}{\pgfqpoint{5.700000in}{5.700000in}}%
\pgfusepath{clip}%
\pgfsetbuttcap%
\pgfsetroundjoin%
\definecolor{currentfill}{rgb}{0.283197,0.115680,0.436115}%
\pgfsetfillcolor{currentfill}%
\pgfsetfillopacity{0.800000}%
\pgfsetlinewidth{0.000000pt}%
\definecolor{currentstroke}{rgb}{0.000000,0.000000,0.000000}%
\pgfsetstrokecolor{currentstroke}%
\pgfsetdash{}{0pt}%
\pgfpathmoveto{\pgfqpoint{4.005826in}{1.116029in}}%
\pgfpathlineto{\pgfqpoint{4.020200in}{1.118038in}}%
\pgfpathlineto{\pgfqpoint{4.034585in}{1.120220in}}%
\pgfpathlineto{\pgfqpoint{4.048979in}{1.122574in}}%
\pgfpathlineto{\pgfqpoint{4.063385in}{1.125101in}}%
\pgfpathlineto{\pgfqpoint{4.071709in}{1.139134in}}%
\pgfpathlineto{\pgfqpoint{4.080029in}{1.153384in}}%
\pgfpathlineto{\pgfqpoint{4.088344in}{1.167844in}}%
\pgfpathlineto{\pgfqpoint{4.096654in}{1.182506in}}%
\pgfpathlineto{\pgfqpoint{4.082253in}{1.179172in}}%
\pgfpathlineto{\pgfqpoint{4.067862in}{1.176011in}}%
\pgfpathlineto{\pgfqpoint{4.053483in}{1.173024in}}%
\pgfpathlineto{\pgfqpoint{4.039114in}{1.170210in}}%
\pgfpathlineto{\pgfqpoint{4.030800in}{1.156342in}}%
\pgfpathlineto{\pgfqpoint{4.022481in}{1.142684in}}%
\pgfpathlineto{\pgfqpoint{4.014156in}{1.129244in}}%
\pgfpathlineto{\pgfqpoint{4.005826in}{1.116029in}}%
\pgfpathclose%
\pgfusepath{fill}%
\end{pgfscope}%
\begin{pgfscope}%
\pgfpathrectangle{\pgfqpoint{1.150000in}{0.150000in}}{\pgfqpoint{5.700000in}{5.700000in}}%
\pgfusepath{clip}%
\pgfsetbuttcap%
\pgfsetroundjoin%
\definecolor{currentfill}{rgb}{0.585678,0.846661,0.249897}%
\pgfsetfillcolor{currentfill}%
\pgfsetfillopacity{0.800000}%
\pgfsetlinewidth{0.000000pt}%
\definecolor{currentstroke}{rgb}{0.000000,0.000000,0.000000}%
\pgfsetstrokecolor{currentstroke}%
\pgfsetdash{}{0pt}%
\pgfpathmoveto{\pgfqpoint{5.503548in}{3.380489in}}%
\pgfpathlineto{\pgfqpoint{5.518950in}{3.401358in}}%
\pgfpathlineto{\pgfqpoint{5.534378in}{3.422424in}}%
\pgfpathlineto{\pgfqpoint{5.549832in}{3.443689in}}%
\pgfpathlineto{\pgfqpoint{5.557623in}{3.453018in}}%
\pgfpathlineto{\pgfqpoint{5.565402in}{3.462099in}}%
\pgfpathlineto{\pgfqpoint{5.573167in}{3.470931in}}%
\pgfpathlineto{\pgfqpoint{5.580921in}{3.479516in}}%
\pgfpathlineto{\pgfqpoint{5.565469in}{3.458313in}}%
\pgfpathlineto{\pgfqpoint{5.550043in}{3.437307in}}%
\pgfpathlineto{\pgfqpoint{5.534644in}{3.416498in}}%
\pgfpathlineto{\pgfqpoint{5.526888in}{3.407857in}}%
\pgfpathlineto{\pgfqpoint{5.519120in}{3.398975in}}%
\pgfpathlineto{\pgfqpoint{5.511340in}{3.389853in}}%
\pgfpathlineto{\pgfqpoint{5.503548in}{3.380489in}}%
\pgfpathclose%
\pgfusepath{fill}%
\end{pgfscope}%
\begin{pgfscope}%
\pgfpathrectangle{\pgfqpoint{1.150000in}{0.150000in}}{\pgfqpoint{5.700000in}{5.700000in}}%
\pgfusepath{clip}%
\pgfsetbuttcap%
\pgfsetroundjoin%
\definecolor{currentfill}{rgb}{0.270595,0.214069,0.507052}%
\pgfsetfillcolor{currentfill}%
\pgfsetfillopacity{0.800000}%
\pgfsetlinewidth{0.000000pt}%
\definecolor{currentstroke}{rgb}{0.000000,0.000000,0.000000}%
\pgfsetstrokecolor{currentstroke}%
\pgfsetdash{}{0pt}%
\pgfpathmoveto{\pgfqpoint{4.220708in}{1.327330in}}%
\pgfpathlineto{\pgfqpoint{4.235171in}{1.333058in}}%
\pgfpathlineto{\pgfqpoint{4.249647in}{1.338960in}}%
\pgfpathlineto{\pgfqpoint{4.264136in}{1.345037in}}%
\pgfpathlineto{\pgfqpoint{4.278638in}{1.351289in}}%
\pgfpathlineto{\pgfqpoint{4.286918in}{1.368892in}}%
\pgfpathlineto{\pgfqpoint{4.295194in}{1.386608in}}%
\pgfpathlineto{\pgfqpoint{4.303467in}{1.404432in}}%
\pgfpathlineto{\pgfqpoint{4.311738in}{1.422355in}}%
\pgfpathlineto{\pgfqpoint{4.297230in}{1.415382in}}%
\pgfpathlineto{\pgfqpoint{4.282736in}{1.408585in}}%
\pgfpathlineto{\pgfqpoint{4.268256in}{1.401964in}}%
\pgfpathlineto{\pgfqpoint{4.253790in}{1.395518in}}%
\pgfpathlineto{\pgfqpoint{4.245524in}{1.378302in}}%
\pgfpathlineto{\pgfqpoint{4.237256in}{1.361195in}}%
\pgfpathlineto{\pgfqpoint{4.228984in}{1.344202in}}%
\pgfpathlineto{\pgfqpoint{4.220708in}{1.327330in}}%
\pgfpathclose%
\pgfusepath{fill}%
\end{pgfscope}%
\begin{pgfscope}%
\pgfpathrectangle{\pgfqpoint{1.150000in}{0.150000in}}{\pgfqpoint{5.700000in}{5.700000in}}%
\pgfusepath{clip}%
\pgfsetbuttcap%
\pgfsetroundjoin%
\definecolor{currentfill}{rgb}{0.203063,0.379716,0.553925}%
\pgfsetfillcolor{currentfill}%
\pgfsetfillopacity{0.800000}%
\pgfsetlinewidth{0.000000pt}%
\definecolor{currentstroke}{rgb}{0.000000,0.000000,0.000000}%
\pgfsetstrokecolor{currentstroke}%
\pgfsetdash{}{0pt}%
\pgfpathmoveto{\pgfqpoint{4.501970in}{1.757510in}}%
\pgfpathlineto{\pgfqpoint{4.516589in}{1.767952in}}%
\pgfpathlineto{\pgfqpoint{4.531225in}{1.778574in}}%
\pgfpathlineto{\pgfqpoint{4.545878in}{1.789376in}}%
\pgfpathlineto{\pgfqpoint{4.560547in}{1.800358in}}%
\pgfpathlineto{\pgfqpoint{4.568788in}{1.820187in}}%
\pgfpathlineto{\pgfqpoint{4.577026in}{1.839993in}}%
\pgfpathlineto{\pgfqpoint{4.585261in}{1.859769in}}%
\pgfpathlineto{\pgfqpoint{4.593493in}{1.879511in}}%
\pgfpathlineto{\pgfqpoint{4.578811in}{1.867958in}}%
\pgfpathlineto{\pgfqpoint{4.564146in}{1.856585in}}%
\pgfpathlineto{\pgfqpoint{4.549498in}{1.845394in}}%
\pgfpathlineto{\pgfqpoint{4.534868in}{1.834383in}}%
\pgfpathlineto{\pgfqpoint{4.526648in}{1.815199in}}%
\pgfpathlineto{\pgfqpoint{4.518425in}{1.795989in}}%
\pgfpathlineto{\pgfqpoint{4.510199in}{1.776757in}}%
\pgfpathlineto{\pgfqpoint{4.501970in}{1.757510in}}%
\pgfpathclose%
\pgfusepath{fill}%
\end{pgfscope}%
\begin{pgfscope}%
\pgfpathrectangle{\pgfqpoint{1.150000in}{0.150000in}}{\pgfqpoint{5.700000in}{5.700000in}}%
\pgfusepath{clip}%
\pgfsetbuttcap%
\pgfsetroundjoin%
\definecolor{currentfill}{rgb}{0.163625,0.471133,0.558148}%
\pgfsetfillcolor{currentfill}%
\pgfsetfillopacity{0.800000}%
\pgfsetlinewidth{0.000000pt}%
\definecolor{currentstroke}{rgb}{0.000000,0.000000,0.000000}%
\pgfsetstrokecolor{currentstroke}%
\pgfsetdash{}{0pt}%
\pgfpathmoveto{\pgfqpoint{4.659226in}{2.035617in}}%
\pgfpathlineto{\pgfqpoint{4.673952in}{2.048398in}}%
\pgfpathlineto{\pgfqpoint{4.688696in}{2.061363in}}%
\pgfpathlineto{\pgfqpoint{4.703459in}{2.074511in}}%
\pgfpathlineto{\pgfqpoint{4.718241in}{2.087844in}}%
\pgfpathlineto{\pgfqpoint{4.726455in}{2.107544in}}%
\pgfpathlineto{\pgfqpoint{4.734665in}{2.127155in}}%
\pgfpathlineto{\pgfqpoint{4.742871in}{2.146671in}}%
\pgfpathlineto{\pgfqpoint{4.751073in}{2.166088in}}%
\pgfpathlineto{\pgfqpoint{4.736277in}{2.152280in}}%
\pgfpathlineto{\pgfqpoint{4.721500in}{2.138656in}}%
\pgfpathlineto{\pgfqpoint{4.706743in}{2.125216in}}%
\pgfpathlineto{\pgfqpoint{4.692004in}{2.111961in}}%
\pgfpathlineto{\pgfqpoint{4.683816in}{2.093006in}}%
\pgfpathlineto{\pgfqpoint{4.675623in}{2.073961in}}%
\pgfpathlineto{\pgfqpoint{4.667426in}{2.054830in}}%
\pgfpathlineto{\pgfqpoint{4.659226in}{2.035617in}}%
\pgfpathclose%
\pgfusepath{fill}%
\end{pgfscope}%
\begin{pgfscope}%
\pgfpathrectangle{\pgfqpoint{1.150000in}{0.150000in}}{\pgfqpoint{5.700000in}{5.700000in}}%
\pgfusepath{clip}%
\pgfsetbuttcap%
\pgfsetroundjoin%
\definecolor{currentfill}{rgb}{0.128087,0.647749,0.523491}%
\pgfsetfillcolor{currentfill}%
\pgfsetfillopacity{0.800000}%
\pgfsetlinewidth{0.000000pt}%
\definecolor{currentstroke}{rgb}{0.000000,0.000000,0.000000}%
\pgfsetstrokecolor{currentstroke}%
\pgfsetdash{}{0pt}%
\pgfpathmoveto{\pgfqpoint{4.973736in}{2.592522in}}%
\pgfpathlineto{\pgfqpoint{4.988699in}{2.609144in}}%
\pgfpathlineto{\pgfqpoint{5.003684in}{2.625957in}}%
\pgfpathlineto{\pgfqpoint{5.018691in}{2.642959in}}%
\pgfpathlineto{\pgfqpoint{5.033720in}{2.660153in}}%
\pgfpathlineto{\pgfqpoint{5.041835in}{2.677236in}}%
\pgfpathlineto{\pgfqpoint{5.049943in}{2.694132in}}%
\pgfpathlineto{\pgfqpoint{5.058044in}{2.710836in}}%
\pgfpathlineto{\pgfqpoint{5.066138in}{2.727348in}}%
\pgfpathlineto{\pgfqpoint{5.051097in}{2.709881in}}%
\pgfpathlineto{\pgfqpoint{5.036079in}{2.692606in}}%
\pgfpathlineto{\pgfqpoint{5.021083in}{2.675522in}}%
\pgfpathlineto{\pgfqpoint{5.006109in}{2.658628in}}%
\pgfpathlineto{\pgfqpoint{4.998026in}{2.642375in}}%
\pgfpathlineto{\pgfqpoint{4.989936in}{2.625938in}}%
\pgfpathlineto{\pgfqpoint{4.981839in}{2.609320in}}%
\pgfpathlineto{\pgfqpoint{4.973736in}{2.592522in}}%
\pgfpathclose%
\pgfusepath{fill}%
\end{pgfscope}%
\begin{pgfscope}%
\pgfpathrectangle{\pgfqpoint{1.150000in}{0.150000in}}{\pgfqpoint{5.700000in}{5.700000in}}%
\pgfusepath{clip}%
\pgfsetbuttcap%
\pgfsetroundjoin%
\definecolor{currentfill}{rgb}{0.327796,0.773980,0.406640}%
\pgfsetfillcolor{currentfill}%
\pgfsetfillopacity{0.800000}%
\pgfsetlinewidth{0.000000pt}%
\definecolor{currentstroke}{rgb}{0.000000,0.000000,0.000000}%
\pgfsetstrokecolor{currentstroke}%
\pgfsetdash{}{0pt}%
\pgfpathmoveto{\pgfqpoint{5.255097in}{3.038542in}}%
\pgfpathlineto{\pgfqpoint{5.270288in}{3.057704in}}%
\pgfpathlineto{\pgfqpoint{5.285504in}{3.077060in}}%
\pgfpathlineto{\pgfqpoint{5.300744in}{3.096612in}}%
\pgfpathlineto{\pgfqpoint{5.316009in}{3.116359in}}%
\pgfpathlineto{\pgfqpoint{5.323974in}{3.129549in}}%
\pgfpathlineto{\pgfqpoint{5.331930in}{3.142503in}}%
\pgfpathlineto{\pgfqpoint{5.339875in}{3.155220in}}%
\pgfpathlineto{\pgfqpoint{5.347810in}{3.167700in}}%
\pgfpathlineto{\pgfqpoint{5.332540in}{3.147862in}}%
\pgfpathlineto{\pgfqpoint{5.317295in}{3.128219in}}%
\pgfpathlineto{\pgfqpoint{5.302074in}{3.108772in}}%
\pgfpathlineto{\pgfqpoint{5.286878in}{3.089519in}}%
\pgfpathlineto{\pgfqpoint{5.278947in}{3.077115in}}%
\pgfpathlineto{\pgfqpoint{5.271007in}{3.064484in}}%
\pgfpathlineto{\pgfqpoint{5.263056in}{3.051626in}}%
\pgfpathlineto{\pgfqpoint{5.255097in}{3.038542in}}%
\pgfpathclose%
\pgfusepath{fill}%
\end{pgfscope}%
\begin{pgfscope}%
\pgfpathrectangle{\pgfqpoint{1.150000in}{0.150000in}}{\pgfqpoint{5.700000in}{5.700000in}}%
\pgfusepath{clip}%
\pgfsetbuttcap%
\pgfsetroundjoin%
\definecolor{currentfill}{rgb}{0.129933,0.559582,0.551864}%
\pgfsetfillcolor{currentfill}%
\pgfsetfillopacity{0.800000}%
\pgfsetlinewidth{0.000000pt}%
\definecolor{currentstroke}{rgb}{0.000000,0.000000,0.000000}%
\pgfsetstrokecolor{currentstroke}%
\pgfsetdash{}{0pt}%
\pgfpathmoveto{\pgfqpoint{4.816529in}{2.317394in}}%
\pgfpathlineto{\pgfqpoint{4.831371in}{2.332240in}}%
\pgfpathlineto{\pgfqpoint{4.846234in}{2.347273in}}%
\pgfpathlineto{\pgfqpoint{4.861117in}{2.362493in}}%
\pgfpathlineto{\pgfqpoint{4.876020in}{2.377901in}}%
\pgfpathlineto{\pgfqpoint{4.884194in}{2.396642in}}%
\pgfpathlineto{\pgfqpoint{4.892363in}{2.415238in}}%
\pgfpathlineto{\pgfqpoint{4.900527in}{2.433685in}}%
\pgfpathlineto{\pgfqpoint{4.908684in}{2.451981in}}%
\pgfpathlineto{\pgfqpoint{4.893767in}{2.436196in}}%
\pgfpathlineto{\pgfqpoint{4.878871in}{2.420600in}}%
\pgfpathlineto{\pgfqpoint{4.863995in}{2.405192in}}%
\pgfpathlineto{\pgfqpoint{4.849140in}{2.389971in}}%
\pgfpathlineto{\pgfqpoint{4.840995in}{2.372038in}}%
\pgfpathlineto{\pgfqpoint{4.832845in}{2.353962in}}%
\pgfpathlineto{\pgfqpoint{4.824689in}{2.335746in}}%
\pgfpathlineto{\pgfqpoint{4.816529in}{2.317394in}}%
\pgfpathclose%
\pgfusepath{fill}%
\end{pgfscope}%
\begin{pgfscope}%
\pgfpathrectangle{\pgfqpoint{1.150000in}{0.150000in}}{\pgfqpoint{5.700000in}{5.700000in}}%
\pgfusepath{clip}%
\pgfsetbuttcap%
\pgfsetroundjoin%
\definecolor{currentfill}{rgb}{0.246811,0.283237,0.535941}%
\pgfsetfillcolor{currentfill}%
\pgfsetfillopacity{0.800000}%
\pgfsetlinewidth{0.000000pt}%
\definecolor{currentstroke}{rgb}{0.000000,0.000000,0.000000}%
\pgfsetstrokecolor{currentstroke}%
\pgfsetdash{}{0pt}%
\pgfpathmoveto{\pgfqpoint{4.344789in}{1.494916in}}%
\pgfpathlineto{\pgfqpoint{4.359317in}{1.502755in}}%
\pgfpathlineto{\pgfqpoint{4.373861in}{1.510771in}}%
\pgfpathlineto{\pgfqpoint{4.388419in}{1.518963in}}%
\pgfpathlineto{\pgfqpoint{4.402992in}{1.527332in}}%
\pgfpathlineto{\pgfqpoint{4.411256in}{1.546330in}}%
\pgfpathlineto{\pgfqpoint{4.419517in}{1.565382in}}%
\pgfpathlineto{\pgfqpoint{4.427775in}{1.584480in}}%
\pgfpathlineto{\pgfqpoint{4.436030in}{1.603620in}}%
\pgfpathlineto{\pgfqpoint{4.421447in}{1.594589in}}%
\pgfpathlineto{\pgfqpoint{4.406881in}{1.585735in}}%
\pgfpathlineto{\pgfqpoint{4.392329in}{1.577058in}}%
\pgfpathlineto{\pgfqpoint{4.377793in}{1.568560in}}%
\pgfpathlineto{\pgfqpoint{4.369546in}{1.550069in}}%
\pgfpathlineto{\pgfqpoint{4.361297in}{1.531628in}}%
\pgfpathlineto{\pgfqpoint{4.353044in}{1.513241in}}%
\pgfpathlineto{\pgfqpoint{4.344789in}{1.494916in}}%
\pgfpathclose%
\pgfusepath{fill}%
\end{pgfscope}%
\begin{pgfscope}%
\pgfpathrectangle{\pgfqpoint{1.150000in}{0.150000in}}{\pgfqpoint{5.700000in}{5.700000in}}%
\pgfusepath{clip}%
\pgfsetbuttcap%
\pgfsetroundjoin%
\definecolor{currentfill}{rgb}{0.281887,0.150881,0.465405}%
\pgfsetfillcolor{currentfill}%
\pgfsetfillopacity{0.800000}%
\pgfsetlinewidth{0.000000pt}%
\definecolor{currentstroke}{rgb}{0.000000,0.000000,0.000000}%
\pgfsetstrokecolor{currentstroke}%
\pgfsetdash{}{0pt}%
\pgfpathmoveto{\pgfqpoint{4.096654in}{1.182506in}}%
\pgfpathlineto{\pgfqpoint{4.111067in}{1.186013in}}%
\pgfpathlineto{\pgfqpoint{4.125492in}{1.189693in}}%
\pgfpathlineto{\pgfqpoint{4.139927in}{1.193546in}}%
\pgfpathlineto{\pgfqpoint{4.154375in}{1.197572in}}%
\pgfpathlineto{\pgfqpoint{4.162680in}{1.213217in}}%
\pgfpathlineto{\pgfqpoint{4.170981in}{1.229041in}}%
\pgfpathlineto{\pgfqpoint{4.179278in}{1.245037in}}%
\pgfpathlineto{\pgfqpoint{4.187571in}{1.261196in}}%
\pgfpathlineto{\pgfqpoint{4.173123in}{1.256391in}}%
\pgfpathlineto{\pgfqpoint{4.158688in}{1.251760in}}%
\pgfpathlineto{\pgfqpoint{4.144264in}{1.247303in}}%
\pgfpathlineto{\pgfqpoint{4.129853in}{1.243019in}}%
\pgfpathlineto{\pgfqpoint{4.121559in}{1.227626in}}%
\pgfpathlineto{\pgfqpoint{4.113262in}{1.212404in}}%
\pgfpathlineto{\pgfqpoint{4.104960in}{1.197362in}}%
\pgfpathlineto{\pgfqpoint{4.096654in}{1.182506in}}%
\pgfpathclose%
\pgfusepath{fill}%
\end{pgfscope}%
\begin{pgfscope}%
\pgfpathrectangle{\pgfqpoint{1.150000in}{0.150000in}}{\pgfqpoint{5.700000in}{5.700000in}}%
\pgfusepath{clip}%
\pgfsetbuttcap%
\pgfsetroundjoin%
\definecolor{currentfill}{rgb}{0.468053,0.818921,0.323998}%
\pgfsetfillcolor{currentfill}%
\pgfsetfillopacity{0.800000}%
\pgfsetlinewidth{0.000000pt}%
\definecolor{currentstroke}{rgb}{0.000000,0.000000,0.000000}%
\pgfsetstrokecolor{currentstroke}%
\pgfsetdash{}{0pt}%
\pgfpathmoveto{\pgfqpoint{5.379446in}{3.215250in}}%
\pgfpathlineto{\pgfqpoint{5.394744in}{3.235338in}}%
\pgfpathlineto{\pgfqpoint{5.410068in}{3.255622in}}%
\pgfpathlineto{\pgfqpoint{5.425417in}{3.276103in}}%
\pgfpathlineto{\pgfqpoint{5.440792in}{3.296781in}}%
\pgfpathlineto{\pgfqpoint{5.448676in}{3.308105in}}%
\pgfpathlineto{\pgfqpoint{5.456550in}{3.319182in}}%
\pgfpathlineto{\pgfqpoint{5.464412in}{3.330013in}}%
\pgfpathlineto{\pgfqpoint{5.472263in}{3.340598in}}%
\pgfpathlineto{\pgfqpoint{5.456886in}{3.319904in}}%
\pgfpathlineto{\pgfqpoint{5.441535in}{3.299408in}}%
\pgfpathlineto{\pgfqpoint{5.426210in}{3.279108in}}%
\pgfpathlineto{\pgfqpoint{5.410910in}{3.259004in}}%
\pgfpathlineto{\pgfqpoint{5.403061in}{3.248421in}}%
\pgfpathlineto{\pgfqpoint{5.395200in}{3.237601in}}%
\pgfpathlineto{\pgfqpoint{5.387328in}{3.226544in}}%
\pgfpathlineto{\pgfqpoint{5.379446in}{3.215250in}}%
\pgfpathclose%
\pgfusepath{fill}%
\end{pgfscope}%
\begin{pgfscope}%
\pgfpathrectangle{\pgfqpoint{1.150000in}{0.150000in}}{\pgfqpoint{5.700000in}{5.700000in}}%
\pgfusepath{clip}%
\pgfsetbuttcap%
\pgfsetroundjoin%
\definecolor{currentfill}{rgb}{0.214298,0.355619,0.551184}%
\pgfsetfillcolor{currentfill}%
\pgfsetfillopacity{0.800000}%
\pgfsetlinewidth{0.000000pt}%
\definecolor{currentstroke}{rgb}{0.000000,0.000000,0.000000}%
\pgfsetstrokecolor{currentstroke}%
\pgfsetdash{}{0pt}%
\pgfpathmoveto{\pgfqpoint{4.469023in}{1.680471in}}%
\pgfpathlineto{\pgfqpoint{4.483631in}{1.690313in}}%
\pgfpathlineto{\pgfqpoint{4.498256in}{1.700333in}}%
\pgfpathlineto{\pgfqpoint{4.512897in}{1.710533in}}%
\pgfpathlineto{\pgfqpoint{4.527555in}{1.720911in}}%
\pgfpathlineto{\pgfqpoint{4.535807in}{1.740781in}}%
\pgfpathlineto{\pgfqpoint{4.544057in}{1.760650in}}%
\pgfpathlineto{\pgfqpoint{4.552303in}{1.780510in}}%
\pgfpathlineto{\pgfqpoint{4.560547in}{1.800358in}}%
\pgfpathlineto{\pgfqpoint{4.545878in}{1.789376in}}%
\pgfpathlineto{\pgfqpoint{4.531225in}{1.778574in}}%
\pgfpathlineto{\pgfqpoint{4.516589in}{1.767952in}}%
\pgfpathlineto{\pgfqpoint{4.501970in}{1.757510in}}%
\pgfpathlineto{\pgfqpoint{4.493737in}{1.738252in}}%
\pgfpathlineto{\pgfqpoint{4.485502in}{1.718989in}}%
\pgfpathlineto{\pgfqpoint{4.477264in}{1.699727in}}%
\pgfpathlineto{\pgfqpoint{4.469023in}{1.680471in}}%
\pgfpathclose%
\pgfusepath{fill}%
\end{pgfscope}%
\begin{pgfscope}%
\pgfpathrectangle{\pgfqpoint{1.150000in}{0.150000in}}{\pgfqpoint{5.700000in}{5.700000in}}%
\pgfusepath{clip}%
\pgfsetbuttcap%
\pgfsetroundjoin%
\definecolor{currentfill}{rgb}{0.191090,0.708366,0.482284}%
\pgfsetfillcolor{currentfill}%
\pgfsetfillopacity{0.800000}%
\pgfsetlinewidth{0.000000pt}%
\definecolor{currentstroke}{rgb}{0.000000,0.000000,0.000000}%
\pgfsetstrokecolor{currentstroke}%
\pgfsetdash{}{0pt}%
\pgfpathmoveto{\pgfqpoint{5.098439in}{2.791423in}}%
\pgfpathlineto{\pgfqpoint{5.113512in}{2.809318in}}%
\pgfpathlineto{\pgfqpoint{5.128609in}{2.827405in}}%
\pgfpathlineto{\pgfqpoint{5.143729in}{2.845685in}}%
\pgfpathlineto{\pgfqpoint{5.158872in}{2.864158in}}%
\pgfpathlineto{\pgfqpoint{5.166938in}{2.879890in}}%
\pgfpathlineto{\pgfqpoint{5.174996in}{2.895408in}}%
\pgfpathlineto{\pgfqpoint{5.183046in}{2.910712in}}%
\pgfpathlineto{\pgfqpoint{5.191087in}{2.925798in}}%
\pgfpathlineto{\pgfqpoint{5.175934in}{2.907123in}}%
\pgfpathlineto{\pgfqpoint{5.160805in}{2.888641in}}%
\pgfpathlineto{\pgfqpoint{5.145699in}{2.870353in}}%
\pgfpathlineto{\pgfqpoint{5.130616in}{2.852257in}}%
\pgfpathlineto{\pgfqpoint{5.122584in}{2.837358in}}%
\pgfpathlineto{\pgfqpoint{5.114543in}{2.822252in}}%
\pgfpathlineto{\pgfqpoint{5.106495in}{2.806939in}}%
\pgfpathlineto{\pgfqpoint{5.098439in}{2.791423in}}%
\pgfpathclose%
\pgfusepath{fill}%
\end{pgfscope}%
\begin{pgfscope}%
\pgfpathrectangle{\pgfqpoint{1.150000in}{0.150000in}}{\pgfqpoint{5.700000in}{5.700000in}}%
\pgfusepath{clip}%
\pgfsetbuttcap%
\pgfsetroundjoin%
\definecolor{currentfill}{rgb}{0.172719,0.448791,0.557885}%
\pgfsetfillcolor{currentfill}%
\pgfsetfillopacity{0.800000}%
\pgfsetlinewidth{0.000000pt}%
\definecolor{currentstroke}{rgb}{0.000000,0.000000,0.000000}%
\pgfsetstrokecolor{currentstroke}%
\pgfsetdash{}{0pt}%
\pgfpathmoveto{\pgfqpoint{4.626387in}{1.958035in}}%
\pgfpathlineto{\pgfqpoint{4.641100in}{1.970310in}}%
\pgfpathlineto{\pgfqpoint{4.655831in}{1.982767in}}%
\pgfpathlineto{\pgfqpoint{4.670580in}{1.995407in}}%
\pgfpathlineto{\pgfqpoint{4.685348in}{2.008229in}}%
\pgfpathlineto{\pgfqpoint{4.693577in}{2.028245in}}%
\pgfpathlineto{\pgfqpoint{4.701802in}{2.048190in}}%
\pgfpathlineto{\pgfqpoint{4.710023in}{2.068057in}}%
\pgfpathlineto{\pgfqpoint{4.718241in}{2.087844in}}%
\pgfpathlineto{\pgfqpoint{4.703459in}{2.074511in}}%
\pgfpathlineto{\pgfqpoint{4.688696in}{2.061363in}}%
\pgfpathlineto{\pgfqpoint{4.673952in}{2.048398in}}%
\pgfpathlineto{\pgfqpoint{4.659226in}{2.035617in}}%
\pgfpathlineto{\pgfqpoint{4.651022in}{2.016326in}}%
\pgfpathlineto{\pgfqpoint{4.642814in}{1.996963in}}%
\pgfpathlineto{\pgfqpoint{4.634602in}{1.977531in}}%
\pgfpathlineto{\pgfqpoint{4.626387in}{1.958035in}}%
\pgfpathclose%
\pgfusepath{fill}%
\end{pgfscope}%
\begin{pgfscope}%
\pgfpathrectangle{\pgfqpoint{1.150000in}{0.150000in}}{\pgfqpoint{5.700000in}{5.700000in}}%
\pgfusepath{clip}%
\pgfsetbuttcap%
\pgfsetroundjoin%
\definecolor{currentfill}{rgb}{0.276194,0.190074,0.493001}%
\pgfsetfillcolor{currentfill}%
\pgfsetfillopacity{0.800000}%
\pgfsetlinewidth{0.000000pt}%
\definecolor{currentstroke}{rgb}{0.000000,0.000000,0.000000}%
\pgfsetstrokecolor{currentstroke}%
\pgfsetdash{}{0pt}%
\pgfpathmoveto{\pgfqpoint{4.187571in}{1.261196in}}%
\pgfpathlineto{\pgfqpoint{4.202032in}{1.266175in}}%
\pgfpathlineto{\pgfqpoint{4.216505in}{1.271328in}}%
\pgfpathlineto{\pgfqpoint{4.230990in}{1.276654in}}%
\pgfpathlineto{\pgfqpoint{4.245489in}{1.282154in}}%
\pgfpathlineto{\pgfqpoint{4.253781in}{1.299232in}}%
\pgfpathlineto{\pgfqpoint{4.262070in}{1.316452in}}%
\pgfpathlineto{\pgfqpoint{4.270356in}{1.333807in}}%
\pgfpathlineto{\pgfqpoint{4.278638in}{1.351289in}}%
\pgfpathlineto{\pgfqpoint{4.264136in}{1.345037in}}%
\pgfpathlineto{\pgfqpoint{4.249647in}{1.338960in}}%
\pgfpathlineto{\pgfqpoint{4.235171in}{1.333058in}}%
\pgfpathlineto{\pgfqpoint{4.220708in}{1.327330in}}%
\pgfpathlineto{\pgfqpoint{4.212429in}{1.310587in}}%
\pgfpathlineto{\pgfqpoint{4.204147in}{1.293979in}}%
\pgfpathlineto{\pgfqpoint{4.195861in}{1.277513in}}%
\pgfpathlineto{\pgfqpoint{4.187571in}{1.261196in}}%
\pgfpathclose%
\pgfusepath{fill}%
\end{pgfscope}%
\begin{pgfscope}%
\pgfpathrectangle{\pgfqpoint{1.150000in}{0.150000in}}{\pgfqpoint{5.700000in}{5.700000in}}%
\pgfusepath{clip}%
\pgfsetbuttcap%
\pgfsetroundjoin%
\definecolor{currentfill}{rgb}{0.136408,0.541173,0.554483}%
\pgfsetfillcolor{currentfill}%
\pgfsetfillopacity{0.800000}%
\pgfsetlinewidth{0.000000pt}%
\definecolor{currentstroke}{rgb}{0.000000,0.000000,0.000000}%
\pgfsetstrokecolor{currentstroke}%
\pgfsetdash{}{0pt}%
\pgfpathmoveto{\pgfqpoint{4.783837in}{2.242689in}}%
\pgfpathlineto{\pgfqpoint{4.798667in}{2.257126in}}%
\pgfpathlineto{\pgfqpoint{4.813516in}{2.271749in}}%
\pgfpathlineto{\pgfqpoint{4.828385in}{2.286558in}}%
\pgfpathlineto{\pgfqpoint{4.843274in}{2.301554in}}%
\pgfpathlineto{\pgfqpoint{4.851468in}{2.320841in}}%
\pgfpathlineto{\pgfqpoint{4.859657in}{2.339996in}}%
\pgfpathlineto{\pgfqpoint{4.867841in}{2.359018in}}%
\pgfpathlineto{\pgfqpoint{4.876020in}{2.377901in}}%
\pgfpathlineto{\pgfqpoint{4.861117in}{2.362493in}}%
\pgfpathlineto{\pgfqpoint{4.846234in}{2.347273in}}%
\pgfpathlineto{\pgfqpoint{4.831371in}{2.332240in}}%
\pgfpathlineto{\pgfqpoint{4.816529in}{2.317394in}}%
\pgfpathlineto{\pgfqpoint{4.808363in}{2.298909in}}%
\pgfpathlineto{\pgfqpoint{4.800193in}{2.280294in}}%
\pgfpathlineto{\pgfqpoint{4.792017in}{2.261553in}}%
\pgfpathlineto{\pgfqpoint{4.783837in}{2.242689in}}%
\pgfpathclose%
\pgfusepath{fill}%
\end{pgfscope}%
\begin{pgfscope}%
\pgfpathrectangle{\pgfqpoint{1.150000in}{0.150000in}}{\pgfqpoint{5.700000in}{5.700000in}}%
\pgfusepath{clip}%
\pgfsetbuttcap%
\pgfsetroundjoin%
\definecolor{currentfill}{rgb}{0.255645,0.260703,0.528312}%
\pgfsetfillcolor{currentfill}%
\pgfsetfillopacity{0.800000}%
\pgfsetlinewidth{0.000000pt}%
\definecolor{currentstroke}{rgb}{0.000000,0.000000,0.000000}%
\pgfsetstrokecolor{currentstroke}%
\pgfsetdash{}{0pt}%
\pgfpathmoveto{\pgfqpoint{4.311738in}{1.422355in}}%
\pgfpathlineto{\pgfqpoint{4.326260in}{1.429503in}}%
\pgfpathlineto{\pgfqpoint{4.340796in}{1.436827in}}%
\pgfpathlineto{\pgfqpoint{4.355346in}{1.444326in}}%
\pgfpathlineto{\pgfqpoint{4.369911in}{1.452001in}}%
\pgfpathlineto{\pgfqpoint{4.378185in}{1.470722in}}%
\pgfpathlineto{\pgfqpoint{4.386457in}{1.489522in}}%
\pgfpathlineto{\pgfqpoint{4.394726in}{1.508394in}}%
\pgfpathlineto{\pgfqpoint{4.402992in}{1.527332in}}%
\pgfpathlineto{\pgfqpoint{4.388419in}{1.518963in}}%
\pgfpathlineto{\pgfqpoint{4.373861in}{1.510771in}}%
\pgfpathlineto{\pgfqpoint{4.359317in}{1.502755in}}%
\pgfpathlineto{\pgfqpoint{4.344789in}{1.494916in}}%
\pgfpathlineto{\pgfqpoint{4.336531in}{1.476659in}}%
\pgfpathlineto{\pgfqpoint{4.328269in}{1.458475in}}%
\pgfpathlineto{\pgfqpoint{4.320005in}{1.440371in}}%
\pgfpathlineto{\pgfqpoint{4.311738in}{1.422355in}}%
\pgfpathclose%
\pgfusepath{fill}%
\end{pgfscope}%
\begin{pgfscope}%
\pgfpathrectangle{\pgfqpoint{1.150000in}{0.150000in}}{\pgfqpoint{5.700000in}{5.700000in}}%
\pgfusepath{clip}%
\pgfsetbuttcap%
\pgfsetroundjoin%
\definecolor{currentfill}{rgb}{0.121380,0.629492,0.531973}%
\pgfsetfillcolor{currentfill}%
\pgfsetfillopacity{0.800000}%
\pgfsetlinewidth{0.000000pt}%
\definecolor{currentstroke}{rgb}{0.000000,0.000000,0.000000}%
\pgfsetstrokecolor{currentstroke}%
\pgfsetdash{}{0pt}%
\pgfpathmoveto{\pgfqpoint{4.941259in}{2.523584in}}%
\pgfpathlineto{\pgfqpoint{4.956210in}{2.539899in}}%
\pgfpathlineto{\pgfqpoint{4.971183in}{2.556404in}}%
\pgfpathlineto{\pgfqpoint{4.986177in}{2.573098in}}%
\pgfpathlineto{\pgfqpoint{5.001193in}{2.589983in}}%
\pgfpathlineto{\pgfqpoint{5.009335in}{2.607795in}}%
\pgfpathlineto{\pgfqpoint{5.017470in}{2.625430in}}%
\pgfpathlineto{\pgfqpoint{5.025598in}{2.642883in}}%
\pgfpathlineto{\pgfqpoint{5.033720in}{2.660153in}}%
\pgfpathlineto{\pgfqpoint{5.018691in}{2.642959in}}%
\pgfpathlineto{\pgfqpoint{5.003684in}{2.625957in}}%
\pgfpathlineto{\pgfqpoint{4.988699in}{2.609144in}}%
\pgfpathlineto{\pgfqpoint{4.973736in}{2.592522in}}%
\pgfpathlineto{\pgfqpoint{4.965626in}{2.575547in}}%
\pgfpathlineto{\pgfqpoint{4.957510in}{2.558397in}}%
\pgfpathlineto{\pgfqpoint{4.949387in}{2.541075in}}%
\pgfpathlineto{\pgfqpoint{4.941259in}{2.523584in}}%
\pgfpathclose%
\pgfusepath{fill}%
\end{pgfscope}%
\begin{pgfscope}%
\pgfpathrectangle{\pgfqpoint{1.150000in}{0.150000in}}{\pgfqpoint{5.700000in}{5.700000in}}%
\pgfusepath{clip}%
\pgfsetbuttcap%
\pgfsetroundjoin%
\definecolor{currentfill}{rgb}{0.282884,0.135920,0.453427}%
\pgfsetfillcolor{currentfill}%
\pgfsetfillopacity{0.800000}%
\pgfsetlinewidth{0.000000pt}%
\definecolor{currentstroke}{rgb}{0.000000,0.000000,0.000000}%
\pgfsetstrokecolor{currentstroke}%
\pgfsetdash{}{0pt}%
\pgfpathmoveto{\pgfqpoint{4.063385in}{1.125101in}}%
\pgfpathlineto{\pgfqpoint{4.077801in}{1.127801in}}%
\pgfpathlineto{\pgfqpoint{4.092228in}{1.130672in}}%
\pgfpathlineto{\pgfqpoint{4.106666in}{1.133716in}}%
\pgfpathlineto{\pgfqpoint{4.121115in}{1.136932in}}%
\pgfpathlineto{\pgfqpoint{4.129436in}{1.151785in}}%
\pgfpathlineto{\pgfqpoint{4.137753in}{1.166848in}}%
\pgfpathlineto{\pgfqpoint{4.146066in}{1.182113in}}%
\pgfpathlineto{\pgfqpoint{4.154375in}{1.197572in}}%
\pgfpathlineto{\pgfqpoint{4.139927in}{1.193546in}}%
\pgfpathlineto{\pgfqpoint{4.125492in}{1.189693in}}%
\pgfpathlineto{\pgfqpoint{4.111067in}{1.186013in}}%
\pgfpathlineto{\pgfqpoint{4.096654in}{1.182506in}}%
\pgfpathlineto{\pgfqpoint{4.088344in}{1.167844in}}%
\pgfpathlineto{\pgfqpoint{4.080029in}{1.153384in}}%
\pgfpathlineto{\pgfqpoint{4.071709in}{1.139134in}}%
\pgfpathlineto{\pgfqpoint{4.063385in}{1.125101in}}%
\pgfpathclose%
\pgfusepath{fill}%
\end{pgfscope}%
\begin{pgfscope}%
\pgfpathrectangle{\pgfqpoint{1.150000in}{0.150000in}}{\pgfqpoint{5.700000in}{5.700000in}}%
\pgfusepath{clip}%
\pgfsetbuttcap%
\pgfsetroundjoin%
\definecolor{currentfill}{rgb}{0.311925,0.767822,0.415586}%
\pgfsetfillcolor{currentfill}%
\pgfsetfillopacity{0.800000}%
\pgfsetlinewidth{0.000000pt}%
\definecolor{currentstroke}{rgb}{0.000000,0.000000,0.000000}%
\pgfsetstrokecolor{currentstroke}%
\pgfsetdash{}{0pt}%
\pgfpathmoveto{\pgfqpoint{5.223164in}{2.983951in}}%
\pgfpathlineto{\pgfqpoint{5.238349in}{3.002985in}}%
\pgfpathlineto{\pgfqpoint{5.253558in}{3.022213in}}%
\pgfpathlineto{\pgfqpoint{5.268791in}{3.041637in}}%
\pgfpathlineto{\pgfqpoint{5.284049in}{3.061255in}}%
\pgfpathlineto{\pgfqpoint{5.292053in}{3.075382in}}%
\pgfpathlineto{\pgfqpoint{5.300048in}{3.089275in}}%
\pgfpathlineto{\pgfqpoint{5.308033in}{3.102935in}}%
\pgfpathlineto{\pgfqpoint{5.316009in}{3.116359in}}%
\pgfpathlineto{\pgfqpoint{5.300744in}{3.096612in}}%
\pgfpathlineto{\pgfqpoint{5.285504in}{3.077060in}}%
\pgfpathlineto{\pgfqpoint{5.270288in}{3.057704in}}%
\pgfpathlineto{\pgfqpoint{5.255097in}{3.038542in}}%
\pgfpathlineto{\pgfqpoint{5.247127in}{3.025231in}}%
\pgfpathlineto{\pgfqpoint{5.239149in}{3.011695in}}%
\pgfpathlineto{\pgfqpoint{5.231161in}{2.997935in}}%
\pgfpathlineto{\pgfqpoint{5.223164in}{2.983951in}}%
\pgfpathclose%
\pgfusepath{fill}%
\end{pgfscope}%
\begin{pgfscope}%
\pgfpathrectangle{\pgfqpoint{1.150000in}{0.150000in}}{\pgfqpoint{5.700000in}{5.700000in}}%
\pgfusepath{clip}%
\pgfsetbuttcap%
\pgfsetroundjoin%
\definecolor{currentfill}{rgb}{0.575563,0.844566,0.256415}%
\pgfsetfillcolor{currentfill}%
\pgfsetfillopacity{0.800000}%
\pgfsetlinewidth{0.000000pt}%
\definecolor{currentstroke}{rgb}{0.000000,0.000000,0.000000}%
\pgfsetstrokecolor{currentstroke}%
\pgfsetdash{}{0pt}%
\pgfpathmoveto{\pgfqpoint{5.472263in}{3.340598in}}%
\pgfpathlineto{\pgfqpoint{5.487665in}{3.361489in}}%
\pgfpathlineto{\pgfqpoint{5.503093in}{3.382579in}}%
\pgfpathlineto{\pgfqpoint{5.518548in}{3.403867in}}%
\pgfpathlineto{\pgfqpoint{5.526387in}{3.414200in}}%
\pgfpathlineto{\pgfqpoint{5.534214in}{3.424281in}}%
\pgfpathlineto{\pgfqpoint{5.542030in}{3.434110in}}%
\pgfpathlineto{\pgfqpoint{5.549832in}{3.443689in}}%
\pgfpathlineto{\pgfqpoint{5.534378in}{3.422424in}}%
\pgfpathlineto{\pgfqpoint{5.518950in}{3.401358in}}%
\pgfpathlineto{\pgfqpoint{5.503548in}{3.380489in}}%
\pgfpathlineto{\pgfqpoint{5.495744in}{3.370882in}}%
\pgfpathlineto{\pgfqpoint{5.487929in}{3.361032in}}%
\pgfpathlineto{\pgfqpoint{5.480102in}{3.350938in}}%
\pgfpathlineto{\pgfqpoint{5.472263in}{3.340598in}}%
\pgfpathclose%
\pgfusepath{fill}%
\end{pgfscope}%
\begin{pgfscope}%
\pgfpathrectangle{\pgfqpoint{1.150000in}{0.150000in}}{\pgfqpoint{5.700000in}{5.700000in}}%
\pgfusepath{clip}%
\pgfsetbuttcap%
\pgfsetroundjoin%
\definecolor{currentfill}{rgb}{0.223925,0.334994,0.548053}%
\pgfsetfillcolor{currentfill}%
\pgfsetfillopacity{0.800000}%
\pgfsetlinewidth{0.000000pt}%
\definecolor{currentstroke}{rgb}{0.000000,0.000000,0.000000}%
\pgfsetstrokecolor{currentstroke}%
\pgfsetdash{}{0pt}%
\pgfpathmoveto{\pgfqpoint{4.436030in}{1.603620in}}%
\pgfpathlineto{\pgfqpoint{4.450628in}{1.612830in}}%
\pgfpathlineto{\pgfqpoint{4.465242in}{1.622217in}}%
\pgfpathlineto{\pgfqpoint{4.479872in}{1.631782in}}%
\pgfpathlineto{\pgfqpoint{4.494518in}{1.641526in}}%
\pgfpathlineto{\pgfqpoint{4.502781in}{1.661346in}}%
\pgfpathlineto{\pgfqpoint{4.511041in}{1.681188in}}%
\pgfpathlineto{\pgfqpoint{4.519299in}{1.701045in}}%
\pgfpathlineto{\pgfqpoint{4.527555in}{1.720911in}}%
\pgfpathlineto{\pgfqpoint{4.512897in}{1.710533in}}%
\pgfpathlineto{\pgfqpoint{4.498256in}{1.700333in}}%
\pgfpathlineto{\pgfqpoint{4.483631in}{1.690313in}}%
\pgfpathlineto{\pgfqpoint{4.469023in}{1.680471in}}%
\pgfpathlineto{\pgfqpoint{4.460779in}{1.661226in}}%
\pgfpathlineto{\pgfqpoint{4.452532in}{1.641999in}}%
\pgfpathlineto{\pgfqpoint{4.444282in}{1.622795in}}%
\pgfpathlineto{\pgfqpoint{4.436030in}{1.603620in}}%
\pgfpathclose%
\pgfusepath{fill}%
\end{pgfscope}%
\begin{pgfscope}%
\pgfpathrectangle{\pgfqpoint{1.150000in}{0.150000in}}{\pgfqpoint{5.700000in}{5.700000in}}%
\pgfusepath{clip}%
\pgfsetbuttcap%
\pgfsetroundjoin%
\definecolor{currentfill}{rgb}{0.180629,0.429975,0.557282}%
\pgfsetfillcolor{currentfill}%
\pgfsetfillopacity{0.800000}%
\pgfsetlinewidth{0.000000pt}%
\definecolor{currentstroke}{rgb}{0.000000,0.000000,0.000000}%
\pgfsetstrokecolor{currentstroke}%
\pgfsetdash{}{0pt}%
\pgfpathmoveto{\pgfqpoint{4.593493in}{1.879511in}}%
\pgfpathlineto{\pgfqpoint{4.608193in}{1.891246in}}%
\pgfpathlineto{\pgfqpoint{4.622910in}{1.903162in}}%
\pgfpathlineto{\pgfqpoint{4.637646in}{1.915260in}}%
\pgfpathlineto{\pgfqpoint{4.652400in}{1.927540in}}%
\pgfpathlineto{\pgfqpoint{4.660642in}{1.947796in}}%
\pgfpathlineto{\pgfqpoint{4.668880in}{1.967999in}}%
\pgfpathlineto{\pgfqpoint{4.677116in}{1.988145in}}%
\pgfpathlineto{\pgfqpoint{4.685348in}{2.008229in}}%
\pgfpathlineto{\pgfqpoint{4.670580in}{1.995407in}}%
\pgfpathlineto{\pgfqpoint{4.655831in}{1.982767in}}%
\pgfpathlineto{\pgfqpoint{4.641100in}{1.970310in}}%
\pgfpathlineto{\pgfqpoint{4.626387in}{1.958035in}}%
\pgfpathlineto{\pgfqpoint{4.618168in}{1.938481in}}%
\pgfpathlineto{\pgfqpoint{4.609947in}{1.918872in}}%
\pgfpathlineto{\pgfqpoint{4.601721in}{1.899214in}}%
\pgfpathlineto{\pgfqpoint{4.593493in}{1.879511in}}%
\pgfpathclose%
\pgfusepath{fill}%
\end{pgfscope}%
\begin{pgfscope}%
\pgfpathrectangle{\pgfqpoint{1.150000in}{0.150000in}}{\pgfqpoint{5.700000in}{5.700000in}}%
\pgfusepath{clip}%
\pgfsetbuttcap%
\pgfsetroundjoin%
\definecolor{currentfill}{rgb}{0.449368,0.813768,0.335384}%
\pgfsetfillcolor{currentfill}%
\pgfsetfillopacity{0.800000}%
\pgfsetlinewidth{0.000000pt}%
\definecolor{currentstroke}{rgb}{0.000000,0.000000,0.000000}%
\pgfsetstrokecolor{currentstroke}%
\pgfsetdash{}{0pt}%
\pgfpathmoveto{\pgfqpoint{5.347810in}{3.167700in}}%
\pgfpathlineto{\pgfqpoint{5.363105in}{3.187735in}}%
\pgfpathlineto{\pgfqpoint{5.378425in}{3.207965in}}%
\pgfpathlineto{\pgfqpoint{5.393770in}{3.228393in}}%
\pgfpathlineto{\pgfqpoint{5.409141in}{3.249017in}}%
\pgfpathlineto{\pgfqpoint{5.417070in}{3.261328in}}%
\pgfpathlineto{\pgfqpoint{5.424989in}{3.273393in}}%
\pgfpathlineto{\pgfqpoint{5.432896in}{3.285210in}}%
\pgfpathlineto{\pgfqpoint{5.440792in}{3.296781in}}%
\pgfpathlineto{\pgfqpoint{5.425417in}{3.276103in}}%
\pgfpathlineto{\pgfqpoint{5.410068in}{3.255622in}}%
\pgfpathlineto{\pgfqpoint{5.394744in}{3.235338in}}%
\pgfpathlineto{\pgfqpoint{5.379446in}{3.215250in}}%
\pgfpathlineto{\pgfqpoint{5.371553in}{3.203719in}}%
\pgfpathlineto{\pgfqpoint{5.363649in}{3.191950in}}%
\pgfpathlineto{\pgfqpoint{5.355735in}{3.179944in}}%
\pgfpathlineto{\pgfqpoint{5.347810in}{3.167700in}}%
\pgfpathclose%
\pgfusepath{fill}%
\end{pgfscope}%
\begin{pgfscope}%
\pgfpathrectangle{\pgfqpoint{1.150000in}{0.150000in}}{\pgfqpoint{5.700000in}{5.700000in}}%
\pgfusepath{clip}%
\pgfsetbuttcap%
\pgfsetroundjoin%
\definecolor{currentfill}{rgb}{0.143343,0.522773,0.556295}%
\pgfsetfillcolor{currentfill}%
\pgfsetfillopacity{0.800000}%
\pgfsetlinewidth{0.000000pt}%
\definecolor{currentstroke}{rgb}{0.000000,0.000000,0.000000}%
\pgfsetstrokecolor{currentstroke}%
\pgfsetdash{}{0pt}%
\pgfpathmoveto{\pgfqpoint{4.751073in}{2.166088in}}%
\pgfpathlineto{\pgfqpoint{4.765888in}{2.180082in}}%
\pgfpathlineto{\pgfqpoint{4.780723in}{2.194260in}}%
\pgfpathlineto{\pgfqpoint{4.795577in}{2.208624in}}%
\pgfpathlineto{\pgfqpoint{4.810452in}{2.223174in}}%
\pgfpathlineto{\pgfqpoint{4.818664in}{2.242946in}}%
\pgfpathlineto{\pgfqpoint{4.826872in}{2.262603in}}%
\pgfpathlineto{\pgfqpoint{4.835075in}{2.282140in}}%
\pgfpathlineto{\pgfqpoint{4.843274in}{2.301554in}}%
\pgfpathlineto{\pgfqpoint{4.828385in}{2.286558in}}%
\pgfpathlineto{\pgfqpoint{4.813516in}{2.271749in}}%
\pgfpathlineto{\pgfqpoint{4.798667in}{2.257126in}}%
\pgfpathlineto{\pgfqpoint{4.783837in}{2.242689in}}%
\pgfpathlineto{\pgfqpoint{4.775653in}{2.223707in}}%
\pgfpathlineto{\pgfqpoint{4.767464in}{2.204611in}}%
\pgfpathlineto{\pgfqpoint{4.759270in}{2.185403in}}%
\pgfpathlineto{\pgfqpoint{4.751073in}{2.166088in}}%
\pgfpathclose%
\pgfusepath{fill}%
\end{pgfscope}%
\begin{pgfscope}%
\pgfpathrectangle{\pgfqpoint{1.150000in}{0.150000in}}{\pgfqpoint{5.700000in}{5.700000in}}%
\pgfusepath{clip}%
\pgfsetbuttcap%
\pgfsetroundjoin%
\definecolor{currentfill}{rgb}{0.175707,0.697900,0.491033}%
\pgfsetfillcolor{currentfill}%
\pgfsetfillopacity{0.800000}%
\pgfsetlinewidth{0.000000pt}%
\definecolor{currentstroke}{rgb}{0.000000,0.000000,0.000000}%
\pgfsetstrokecolor{currentstroke}%
\pgfsetdash{}{0pt}%
\pgfpathmoveto{\pgfqpoint{5.066138in}{2.727348in}}%
\pgfpathlineto{\pgfqpoint{5.081201in}{2.745005in}}%
\pgfpathlineto{\pgfqpoint{5.096287in}{2.762855in}}%
\pgfpathlineto{\pgfqpoint{5.111396in}{2.780897in}}%
\pgfpathlineto{\pgfqpoint{5.126528in}{2.799131in}}%
\pgfpathlineto{\pgfqpoint{5.134625in}{2.815699in}}%
\pgfpathlineto{\pgfqpoint{5.142716in}{2.832061in}}%
\pgfpathlineto{\pgfqpoint{5.150798in}{2.848215in}}%
\pgfpathlineto{\pgfqpoint{5.158872in}{2.864158in}}%
\pgfpathlineto{\pgfqpoint{5.143729in}{2.845685in}}%
\pgfpathlineto{\pgfqpoint{5.128609in}{2.827405in}}%
\pgfpathlineto{\pgfqpoint{5.113512in}{2.809318in}}%
\pgfpathlineto{\pgfqpoint{5.098439in}{2.791423in}}%
\pgfpathlineto{\pgfqpoint{5.090375in}{2.775704in}}%
\pgfpathlineto{\pgfqpoint{5.082303in}{2.759783in}}%
\pgfpathlineto{\pgfqpoint{5.074224in}{2.743664in}}%
\pgfpathlineto{\pgfqpoint{5.066138in}{2.727348in}}%
\pgfpathclose%
\pgfusepath{fill}%
\end{pgfscope}%
\begin{pgfscope}%
\pgfpathrectangle{\pgfqpoint{1.150000in}{0.150000in}}{\pgfqpoint{5.700000in}{5.700000in}}%
\pgfusepath{clip}%
\pgfsetbuttcap%
\pgfsetroundjoin%
\definecolor{currentfill}{rgb}{0.263663,0.237631,0.518762}%
\pgfsetfillcolor{currentfill}%
\pgfsetfillopacity{0.800000}%
\pgfsetlinewidth{0.000000pt}%
\definecolor{currentstroke}{rgb}{0.000000,0.000000,0.000000}%
\pgfsetstrokecolor{currentstroke}%
\pgfsetdash{}{0pt}%
\pgfpathmoveto{\pgfqpoint{4.278638in}{1.351289in}}%
\pgfpathlineto{\pgfqpoint{4.293154in}{1.357715in}}%
\pgfpathlineto{\pgfqpoint{4.307684in}{1.364316in}}%
\pgfpathlineto{\pgfqpoint{4.322228in}{1.371092in}}%
\pgfpathlineto{\pgfqpoint{4.336786in}{1.378042in}}%
\pgfpathlineto{\pgfqpoint{4.345071in}{1.396380in}}%
\pgfpathlineto{\pgfqpoint{4.353354in}{1.414824in}}%
\pgfpathlineto{\pgfqpoint{4.361634in}{1.433366in}}%
\pgfpathlineto{\pgfqpoint{4.369911in}{1.452001in}}%
\pgfpathlineto{\pgfqpoint{4.355346in}{1.444326in}}%
\pgfpathlineto{\pgfqpoint{4.340796in}{1.436827in}}%
\pgfpathlineto{\pgfqpoint{4.326260in}{1.429503in}}%
\pgfpathlineto{\pgfqpoint{4.311738in}{1.422355in}}%
\pgfpathlineto{\pgfqpoint{4.303467in}{1.404432in}}%
\pgfpathlineto{\pgfqpoint{4.295194in}{1.386608in}}%
\pgfpathlineto{\pgfqpoint{4.286918in}{1.368892in}}%
\pgfpathlineto{\pgfqpoint{4.278638in}{1.351289in}}%
\pgfpathclose%
\pgfusepath{fill}%
\end{pgfscope}%
\begin{pgfscope}%
\pgfpathrectangle{\pgfqpoint{1.150000in}{0.150000in}}{\pgfqpoint{5.700000in}{5.700000in}}%
\pgfusepath{clip}%
\pgfsetbuttcap%
\pgfsetroundjoin%
\definecolor{currentfill}{rgb}{0.119423,0.611141,0.538982}%
\pgfsetfillcolor{currentfill}%
\pgfsetfillopacity{0.800000}%
\pgfsetlinewidth{0.000000pt}%
\definecolor{currentstroke}{rgb}{0.000000,0.000000,0.000000}%
\pgfsetstrokecolor{currentstroke}%
\pgfsetdash{}{0pt}%
\pgfpathmoveto{\pgfqpoint{4.908684in}{2.451981in}}%
\pgfpathlineto{\pgfqpoint{4.923623in}{2.467954in}}%
\pgfpathlineto{\pgfqpoint{4.938582in}{2.484115in}}%
\pgfpathlineto{\pgfqpoint{4.953563in}{2.500465in}}%
\pgfpathlineto{\pgfqpoint{4.968566in}{2.517004in}}%
\pgfpathlineto{\pgfqpoint{4.976732in}{2.535502in}}%
\pgfpathlineto{\pgfqpoint{4.984892in}{2.553833in}}%
\pgfpathlineto{\pgfqpoint{4.993046in}{2.571994in}}%
\pgfpathlineto{\pgfqpoint{5.001193in}{2.589983in}}%
\pgfpathlineto{\pgfqpoint{4.986177in}{2.573098in}}%
\pgfpathlineto{\pgfqpoint{4.971183in}{2.556404in}}%
\pgfpathlineto{\pgfqpoint{4.956210in}{2.539899in}}%
\pgfpathlineto{\pgfqpoint{4.941259in}{2.523584in}}%
\pgfpathlineto{\pgfqpoint{4.933124in}{2.505926in}}%
\pgfpathlineto{\pgfqpoint{4.924983in}{2.488104in}}%
\pgfpathlineto{\pgfqpoint{4.916837in}{2.470122in}}%
\pgfpathlineto{\pgfqpoint{4.908684in}{2.451981in}}%
\pgfpathclose%
\pgfusepath{fill}%
\end{pgfscope}%
\begin{pgfscope}%
\pgfpathrectangle{\pgfqpoint{1.150000in}{0.150000in}}{\pgfqpoint{5.700000in}{5.700000in}}%
\pgfusepath{clip}%
\pgfsetbuttcap%
\pgfsetroundjoin%
\definecolor{currentfill}{rgb}{0.279574,0.170599,0.479997}%
\pgfsetfillcolor{currentfill}%
\pgfsetfillopacity{0.800000}%
\pgfsetlinewidth{0.000000pt}%
\definecolor{currentstroke}{rgb}{0.000000,0.000000,0.000000}%
\pgfsetstrokecolor{currentstroke}%
\pgfsetdash{}{0pt}%
\pgfpathmoveto{\pgfqpoint{4.154375in}{1.197572in}}%
\pgfpathlineto{\pgfqpoint{4.168835in}{1.201771in}}%
\pgfpathlineto{\pgfqpoint{4.183306in}{1.206142in}}%
\pgfpathlineto{\pgfqpoint{4.197790in}{1.210687in}}%
\pgfpathlineto{\pgfqpoint{4.212287in}{1.215405in}}%
\pgfpathlineto{\pgfqpoint{4.220592in}{1.231843in}}%
\pgfpathlineto{\pgfqpoint{4.228895in}{1.248452in}}%
\pgfpathlineto{\pgfqpoint{4.237193in}{1.265225in}}%
\pgfpathlineto{\pgfqpoint{4.245489in}{1.282154in}}%
\pgfpathlineto{\pgfqpoint{4.230990in}{1.276654in}}%
\pgfpathlineto{\pgfqpoint{4.216505in}{1.271328in}}%
\pgfpathlineto{\pgfqpoint{4.202032in}{1.266175in}}%
\pgfpathlineto{\pgfqpoint{4.187571in}{1.261196in}}%
\pgfpathlineto{\pgfqpoint{4.179278in}{1.245037in}}%
\pgfpathlineto{\pgfqpoint{4.170981in}{1.229041in}}%
\pgfpathlineto{\pgfqpoint{4.162680in}{1.213217in}}%
\pgfpathlineto{\pgfqpoint{4.154375in}{1.197572in}}%
\pgfpathclose%
\pgfusepath{fill}%
\end{pgfscope}%
\begin{pgfscope}%
\pgfpathrectangle{\pgfqpoint{1.150000in}{0.150000in}}{\pgfqpoint{5.700000in}{5.700000in}}%
\pgfusepath{clip}%
\pgfsetbuttcap%
\pgfsetroundjoin%
\definecolor{currentfill}{rgb}{0.190631,0.407061,0.556089}%
\pgfsetfillcolor{currentfill}%
\pgfsetfillopacity{0.800000}%
\pgfsetlinewidth{0.000000pt}%
\definecolor{currentstroke}{rgb}{0.000000,0.000000,0.000000}%
\pgfsetstrokecolor{currentstroke}%
\pgfsetdash{}{0pt}%
\pgfpathmoveto{\pgfqpoint{4.560547in}{1.800358in}}%
\pgfpathlineto{\pgfqpoint{4.575234in}{1.811521in}}%
\pgfpathlineto{\pgfqpoint{4.589939in}{1.822863in}}%
\pgfpathlineto{\pgfqpoint{4.604661in}{1.834387in}}%
\pgfpathlineto{\pgfqpoint{4.619400in}{1.846091in}}%
\pgfpathlineto{\pgfqpoint{4.627655in}{1.866506in}}%
\pgfpathlineto{\pgfqpoint{4.635906in}{1.886890in}}%
\pgfpathlineto{\pgfqpoint{4.644154in}{1.907236in}}%
\pgfpathlineto{\pgfqpoint{4.652400in}{1.927540in}}%
\pgfpathlineto{\pgfqpoint{4.637646in}{1.915260in}}%
\pgfpathlineto{\pgfqpoint{4.622910in}{1.903162in}}%
\pgfpathlineto{\pgfqpoint{4.608193in}{1.891246in}}%
\pgfpathlineto{\pgfqpoint{4.593493in}{1.879511in}}%
\pgfpathlineto{\pgfqpoint{4.585261in}{1.859769in}}%
\pgfpathlineto{\pgfqpoint{4.577026in}{1.839993in}}%
\pgfpathlineto{\pgfqpoint{4.568788in}{1.820187in}}%
\pgfpathlineto{\pgfqpoint{4.560547in}{1.800358in}}%
\pgfpathclose%
\pgfusepath{fill}%
\end{pgfscope}%
\begin{pgfscope}%
\pgfpathrectangle{\pgfqpoint{1.150000in}{0.150000in}}{\pgfqpoint{5.700000in}{5.700000in}}%
\pgfusepath{clip}%
\pgfsetbuttcap%
\pgfsetroundjoin%
\definecolor{currentfill}{rgb}{0.235526,0.309527,0.542944}%
\pgfsetfillcolor{currentfill}%
\pgfsetfillopacity{0.800000}%
\pgfsetlinewidth{0.000000pt}%
\definecolor{currentstroke}{rgb}{0.000000,0.000000,0.000000}%
\pgfsetstrokecolor{currentstroke}%
\pgfsetdash{}{0pt}%
\pgfpathmoveto{\pgfqpoint{4.402992in}{1.527332in}}%
\pgfpathlineto{\pgfqpoint{4.417581in}{1.535878in}}%
\pgfpathlineto{\pgfqpoint{4.432184in}{1.544601in}}%
\pgfpathlineto{\pgfqpoint{4.446804in}{1.553500in}}%
\pgfpathlineto{\pgfqpoint{4.461439in}{1.562577in}}%
\pgfpathlineto{\pgfqpoint{4.469712in}{1.582252in}}%
\pgfpathlineto{\pgfqpoint{4.477983in}{1.601972in}}%
\pgfpathlineto{\pgfqpoint{4.486252in}{1.621732in}}%
\pgfpathlineto{\pgfqpoint{4.494518in}{1.641526in}}%
\pgfpathlineto{\pgfqpoint{4.479872in}{1.631782in}}%
\pgfpathlineto{\pgfqpoint{4.465242in}{1.622217in}}%
\pgfpathlineto{\pgfqpoint{4.450628in}{1.612830in}}%
\pgfpathlineto{\pgfqpoint{4.436030in}{1.603620in}}%
\pgfpathlineto{\pgfqpoint{4.427775in}{1.584480in}}%
\pgfpathlineto{\pgfqpoint{4.419517in}{1.565382in}}%
\pgfpathlineto{\pgfqpoint{4.411256in}{1.546330in}}%
\pgfpathlineto{\pgfqpoint{4.402992in}{1.527332in}}%
\pgfpathclose%
\pgfusepath{fill}%
\end{pgfscope}%
\begin{pgfscope}%
\pgfpathrectangle{\pgfqpoint{1.150000in}{0.150000in}}{\pgfqpoint{5.700000in}{5.700000in}}%
\pgfusepath{clip}%
\pgfsetbuttcap%
\pgfsetroundjoin%
\definecolor{currentfill}{rgb}{0.281477,0.755203,0.432552}%
\pgfsetfillcolor{currentfill}%
\pgfsetfillopacity{0.800000}%
\pgfsetlinewidth{0.000000pt}%
\definecolor{currentstroke}{rgb}{0.000000,0.000000,0.000000}%
\pgfsetstrokecolor{currentstroke}%
\pgfsetdash{}{0pt}%
\pgfpathmoveto{\pgfqpoint{5.191087in}{2.925798in}}%
\pgfpathlineto{\pgfqpoint{5.206263in}{2.944667in}}%
\pgfpathlineto{\pgfqpoint{5.221464in}{2.963730in}}%
\pgfpathlineto{\pgfqpoint{5.236688in}{2.982987in}}%
\pgfpathlineto{\pgfqpoint{5.251937in}{3.002439in}}%
\pgfpathlineto{\pgfqpoint{5.259979in}{3.017487in}}%
\pgfpathlineto{\pgfqpoint{5.268012in}{3.032307in}}%
\pgfpathlineto{\pgfqpoint{5.276035in}{3.046897in}}%
\pgfpathlineto{\pgfqpoint{5.284049in}{3.061255in}}%
\pgfpathlineto{\pgfqpoint{5.268791in}{3.041637in}}%
\pgfpathlineto{\pgfqpoint{5.253558in}{3.022213in}}%
\pgfpathlineto{\pgfqpoint{5.238349in}{3.002985in}}%
\pgfpathlineto{\pgfqpoint{5.223164in}{2.983951in}}%
\pgfpathlineto{\pgfqpoint{5.215158in}{2.969744in}}%
\pgfpathlineto{\pgfqpoint{5.207143in}{2.955316in}}%
\pgfpathlineto{\pgfqpoint{5.199119in}{2.940666in}}%
\pgfpathlineto{\pgfqpoint{5.191087in}{2.925798in}}%
\pgfpathclose%
\pgfusepath{fill}%
\end{pgfscope}%
\begin{pgfscope}%
\pgfpathrectangle{\pgfqpoint{1.150000in}{0.150000in}}{\pgfqpoint{5.700000in}{5.700000in}}%
\pgfusepath{clip}%
\pgfsetbuttcap%
\pgfsetroundjoin%
\definecolor{currentfill}{rgb}{0.151918,0.500685,0.557587}%
\pgfsetfillcolor{currentfill}%
\pgfsetfillopacity{0.800000}%
\pgfsetlinewidth{0.000000pt}%
\definecolor{currentstroke}{rgb}{0.000000,0.000000,0.000000}%
\pgfsetstrokecolor{currentstroke}%
\pgfsetdash{}{0pt}%
\pgfpathmoveto{\pgfqpoint{4.718241in}{2.087844in}}%
\pgfpathlineto{\pgfqpoint{4.733042in}{2.101360in}}%
\pgfpathlineto{\pgfqpoint{4.747862in}{2.115060in}}%
\pgfpathlineto{\pgfqpoint{4.762702in}{2.128945in}}%
\pgfpathlineto{\pgfqpoint{4.777561in}{2.143014in}}%
\pgfpathlineto{\pgfqpoint{4.785790in}{2.163206in}}%
\pgfpathlineto{\pgfqpoint{4.794015in}{2.183300in}}%
\pgfpathlineto{\pgfqpoint{4.802235in}{2.203290in}}%
\pgfpathlineto{\pgfqpoint{4.810452in}{2.223174in}}%
\pgfpathlineto{\pgfqpoint{4.795577in}{2.208624in}}%
\pgfpathlineto{\pgfqpoint{4.780723in}{2.194260in}}%
\pgfpathlineto{\pgfqpoint{4.765888in}{2.180082in}}%
\pgfpathlineto{\pgfqpoint{4.751073in}{2.166088in}}%
\pgfpathlineto{\pgfqpoint{4.742871in}{2.146671in}}%
\pgfpathlineto{\pgfqpoint{4.734665in}{2.127155in}}%
\pgfpathlineto{\pgfqpoint{4.726455in}{2.107544in}}%
\pgfpathlineto{\pgfqpoint{4.718241in}{2.087844in}}%
\pgfpathclose%
\pgfusepath{fill}%
\end{pgfscope}%
\begin{pgfscope}%
\pgfpathrectangle{\pgfqpoint{1.150000in}{0.150000in}}{\pgfqpoint{5.700000in}{5.700000in}}%
\pgfusepath{clip}%
\pgfsetbuttcap%
\pgfsetroundjoin%
\definecolor{currentfill}{rgb}{0.121148,0.592739,0.544641}%
\pgfsetfillcolor{currentfill}%
\pgfsetfillopacity{0.800000}%
\pgfsetlinewidth{0.000000pt}%
\definecolor{currentstroke}{rgb}{0.000000,0.000000,0.000000}%
\pgfsetstrokecolor{currentstroke}%
\pgfsetdash{}{0pt}%
\pgfpathmoveto{\pgfqpoint{4.876020in}{2.377901in}}%
\pgfpathlineto{\pgfqpoint{4.890945in}{2.393495in}}%
\pgfpathlineto{\pgfqpoint{4.905890in}{2.409278in}}%
\pgfpathlineto{\pgfqpoint{4.920857in}{2.425249in}}%
\pgfpathlineto{\pgfqpoint{4.935844in}{2.441408in}}%
\pgfpathlineto{\pgfqpoint{4.944033in}{2.460542in}}%
\pgfpathlineto{\pgfqpoint{4.952216in}{2.479521in}}%
\pgfpathlineto{\pgfqpoint{4.960394in}{2.498343in}}%
\pgfpathlineto{\pgfqpoint{4.968566in}{2.517004in}}%
\pgfpathlineto{\pgfqpoint{4.953563in}{2.500465in}}%
\pgfpathlineto{\pgfqpoint{4.938582in}{2.484115in}}%
\pgfpathlineto{\pgfqpoint{4.923623in}{2.467954in}}%
\pgfpathlineto{\pgfqpoint{4.908684in}{2.451981in}}%
\pgfpathlineto{\pgfqpoint{4.900527in}{2.433685in}}%
\pgfpathlineto{\pgfqpoint{4.892363in}{2.415238in}}%
\pgfpathlineto{\pgfqpoint{4.884194in}{2.396642in}}%
\pgfpathlineto{\pgfqpoint{4.876020in}{2.377901in}}%
\pgfpathclose%
\pgfusepath{fill}%
\end{pgfscope}%
\begin{pgfscope}%
\pgfpathrectangle{\pgfqpoint{1.150000in}{0.150000in}}{\pgfqpoint{5.700000in}{5.700000in}}%
\pgfusepath{clip}%
\pgfsetbuttcap%
\pgfsetroundjoin%
\definecolor{currentfill}{rgb}{0.270595,0.214069,0.507052}%
\pgfsetfillcolor{currentfill}%
\pgfsetfillopacity{0.800000}%
\pgfsetlinewidth{0.000000pt}%
\definecolor{currentstroke}{rgb}{0.000000,0.000000,0.000000}%
\pgfsetstrokecolor{currentstroke}%
\pgfsetdash{}{0pt}%
\pgfpathmoveto{\pgfqpoint{4.245489in}{1.282154in}}%
\pgfpathlineto{\pgfqpoint{4.260001in}{1.287827in}}%
\pgfpathlineto{\pgfqpoint{4.274526in}{1.293675in}}%
\pgfpathlineto{\pgfqpoint{4.289064in}{1.299696in}}%
\pgfpathlineto{\pgfqpoint{4.303616in}{1.305891in}}%
\pgfpathlineto{\pgfqpoint{4.311912in}{1.323735in}}%
\pgfpathlineto{\pgfqpoint{4.320206in}{1.341713in}}%
\pgfpathlineto{\pgfqpoint{4.328497in}{1.359818in}}%
\pgfpathlineto{\pgfqpoint{4.336786in}{1.378042in}}%
\pgfpathlineto{\pgfqpoint{4.322228in}{1.371092in}}%
\pgfpathlineto{\pgfqpoint{4.307684in}{1.364316in}}%
\pgfpathlineto{\pgfqpoint{4.293154in}{1.357715in}}%
\pgfpathlineto{\pgfqpoint{4.278638in}{1.351289in}}%
\pgfpathlineto{\pgfqpoint{4.270356in}{1.333807in}}%
\pgfpathlineto{\pgfqpoint{4.262070in}{1.316452in}}%
\pgfpathlineto{\pgfqpoint{4.253781in}{1.299232in}}%
\pgfpathlineto{\pgfqpoint{4.245489in}{1.282154in}}%
\pgfpathclose%
\pgfusepath{fill}%
\end{pgfscope}%
\begin{pgfscope}%
\pgfpathrectangle{\pgfqpoint{1.150000in}{0.150000in}}{\pgfqpoint{5.700000in}{5.700000in}}%
\pgfusepath{clip}%
\pgfsetbuttcap%
\pgfsetroundjoin%
\definecolor{currentfill}{rgb}{0.153894,0.680203,0.504172}%
\pgfsetfillcolor{currentfill}%
\pgfsetfillopacity{0.800000}%
\pgfsetlinewidth{0.000000pt}%
\definecolor{currentstroke}{rgb}{0.000000,0.000000,0.000000}%
\pgfsetstrokecolor{currentstroke}%
\pgfsetdash{}{0pt}%
\pgfpathmoveto{\pgfqpoint{5.033720in}{2.660153in}}%
\pgfpathlineto{\pgfqpoint{5.048771in}{2.677537in}}%
\pgfpathlineto{\pgfqpoint{5.063845in}{2.695112in}}%
\pgfpathlineto{\pgfqpoint{5.078942in}{2.712879in}}%
\pgfpathlineto{\pgfqpoint{5.094061in}{2.730838in}}%
\pgfpathlineto{\pgfqpoint{5.102189in}{2.748210in}}%
\pgfpathlineto{\pgfqpoint{5.110309in}{2.765384in}}%
\pgfpathlineto{\pgfqpoint{5.118422in}{2.782359in}}%
\pgfpathlineto{\pgfqpoint{5.126528in}{2.799131in}}%
\pgfpathlineto{\pgfqpoint{5.111396in}{2.780897in}}%
\pgfpathlineto{\pgfqpoint{5.096287in}{2.762855in}}%
\pgfpathlineto{\pgfqpoint{5.081201in}{2.745005in}}%
\pgfpathlineto{\pgfqpoint{5.066138in}{2.727348in}}%
\pgfpathlineto{\pgfqpoint{5.058044in}{2.710836in}}%
\pgfpathlineto{\pgfqpoint{5.049943in}{2.694132in}}%
\pgfpathlineto{\pgfqpoint{5.041835in}{2.677236in}}%
\pgfpathlineto{\pgfqpoint{5.033720in}{2.660153in}}%
\pgfpathclose%
\pgfusepath{fill}%
\end{pgfscope}%
\begin{pgfscope}%
\pgfpathrectangle{\pgfqpoint{1.150000in}{0.150000in}}{\pgfqpoint{5.700000in}{5.700000in}}%
\pgfusepath{clip}%
\pgfsetbuttcap%
\pgfsetroundjoin%
\definecolor{currentfill}{rgb}{0.555484,0.840254,0.269281}%
\pgfsetfillcolor{currentfill}%
\pgfsetfillopacity{0.800000}%
\pgfsetlinewidth{0.000000pt}%
\definecolor{currentstroke}{rgb}{0.000000,0.000000,0.000000}%
\pgfsetstrokecolor{currentstroke}%
\pgfsetdash{}{0pt}%
\pgfpathmoveto{\pgfqpoint{5.440792in}{3.296781in}}%
\pgfpathlineto{\pgfqpoint{5.456192in}{3.317657in}}%
\pgfpathlineto{\pgfqpoint{5.471619in}{3.338731in}}%
\pgfpathlineto{\pgfqpoint{5.487071in}{3.360004in}}%
\pgfpathlineto{\pgfqpoint{5.494958in}{3.371350in}}%
\pgfpathlineto{\pgfqpoint{5.502833in}{3.382442in}}%
\pgfpathlineto{\pgfqpoint{5.510696in}{3.393281in}}%
\pgfpathlineto{\pgfqpoint{5.518548in}{3.403867in}}%
\pgfpathlineto{\pgfqpoint{5.503093in}{3.382579in}}%
\pgfpathlineto{\pgfqpoint{5.487665in}{3.361489in}}%
\pgfpathlineto{\pgfqpoint{5.472263in}{3.340598in}}%
\pgfpathlineto{\pgfqpoint{5.464412in}{3.330013in}}%
\pgfpathlineto{\pgfqpoint{5.456550in}{3.319182in}}%
\pgfpathlineto{\pgfqpoint{5.448676in}{3.308105in}}%
\pgfpathlineto{\pgfqpoint{5.440792in}{3.296781in}}%
\pgfpathclose%
\pgfusepath{fill}%
\end{pgfscope}%
\begin{pgfscope}%
\pgfpathrectangle{\pgfqpoint{1.150000in}{0.150000in}}{\pgfqpoint{5.700000in}{5.700000in}}%
\pgfusepath{clip}%
\pgfsetbuttcap%
\pgfsetroundjoin%
\definecolor{currentfill}{rgb}{0.201239,0.383670,0.554294}%
\pgfsetfillcolor{currentfill}%
\pgfsetfillopacity{0.800000}%
\pgfsetlinewidth{0.000000pt}%
\definecolor{currentstroke}{rgb}{0.000000,0.000000,0.000000}%
\pgfsetstrokecolor{currentstroke}%
\pgfsetdash{}{0pt}%
\pgfpathmoveto{\pgfqpoint{4.527555in}{1.720911in}}%
\pgfpathlineto{\pgfqpoint{4.542229in}{1.731469in}}%
\pgfpathlineto{\pgfqpoint{4.556920in}{1.742207in}}%
\pgfpathlineto{\pgfqpoint{4.571629in}{1.753123in}}%
\pgfpathlineto{\pgfqpoint{4.586355in}{1.764220in}}%
\pgfpathlineto{\pgfqpoint{4.594620in}{1.784708in}}%
\pgfpathlineto{\pgfqpoint{4.602883in}{1.805187in}}%
\pgfpathlineto{\pgfqpoint{4.611143in}{1.825649in}}%
\pgfpathlineto{\pgfqpoint{4.619400in}{1.846091in}}%
\pgfpathlineto{\pgfqpoint{4.604661in}{1.834387in}}%
\pgfpathlineto{\pgfqpoint{4.589939in}{1.822863in}}%
\pgfpathlineto{\pgfqpoint{4.575234in}{1.811521in}}%
\pgfpathlineto{\pgfqpoint{4.560547in}{1.800358in}}%
\pgfpathlineto{\pgfqpoint{4.552303in}{1.780510in}}%
\pgfpathlineto{\pgfqpoint{4.544057in}{1.760650in}}%
\pgfpathlineto{\pgfqpoint{4.535807in}{1.740781in}}%
\pgfpathlineto{\pgfqpoint{4.527555in}{1.720911in}}%
\pgfpathclose%
\pgfusepath{fill}%
\end{pgfscope}%
\begin{pgfscope}%
\pgfpathrectangle{\pgfqpoint{1.150000in}{0.150000in}}{\pgfqpoint{5.700000in}{5.700000in}}%
\pgfusepath{clip}%
\pgfsetbuttcap%
\pgfsetroundjoin%
\definecolor{currentfill}{rgb}{0.421908,0.805774,0.351910}%
\pgfsetfillcolor{currentfill}%
\pgfsetfillopacity{0.800000}%
\pgfsetlinewidth{0.000000pt}%
\definecolor{currentstroke}{rgb}{0.000000,0.000000,0.000000}%
\pgfsetstrokecolor{currentstroke}%
\pgfsetdash{}{0pt}%
\pgfpathmoveto{\pgfqpoint{5.316009in}{3.116359in}}%
\pgfpathlineto{\pgfqpoint{5.331298in}{3.136302in}}%
\pgfpathlineto{\pgfqpoint{5.346613in}{3.156442in}}%
\pgfpathlineto{\pgfqpoint{5.361953in}{3.176778in}}%
\pgfpathlineto{\pgfqpoint{5.377318in}{3.197311in}}%
\pgfpathlineto{\pgfqpoint{5.385290in}{3.210606in}}%
\pgfpathlineto{\pgfqpoint{5.393251in}{3.223656in}}%
\pgfpathlineto{\pgfqpoint{5.401201in}{3.236460in}}%
\pgfpathlineto{\pgfqpoint{5.409141in}{3.249017in}}%
\pgfpathlineto{\pgfqpoint{5.393770in}{3.228393in}}%
\pgfpathlineto{\pgfqpoint{5.378425in}{3.207965in}}%
\pgfpathlineto{\pgfqpoint{5.363105in}{3.187735in}}%
\pgfpathlineto{\pgfqpoint{5.347810in}{3.167700in}}%
\pgfpathlineto{\pgfqpoint{5.339875in}{3.155220in}}%
\pgfpathlineto{\pgfqpoint{5.331930in}{3.142503in}}%
\pgfpathlineto{\pgfqpoint{5.323974in}{3.129549in}}%
\pgfpathlineto{\pgfqpoint{5.316009in}{3.116359in}}%
\pgfpathclose%
\pgfusepath{fill}%
\end{pgfscope}%
\begin{pgfscope}%
\pgfpathrectangle{\pgfqpoint{1.150000in}{0.150000in}}{\pgfqpoint{5.700000in}{5.700000in}}%
\pgfusepath{clip}%
\pgfsetbuttcap%
\pgfsetroundjoin%
\definecolor{currentfill}{rgb}{0.281412,0.155834,0.469201}%
\pgfsetfillcolor{currentfill}%
\pgfsetfillopacity{0.800000}%
\pgfsetlinewidth{0.000000pt}%
\definecolor{currentstroke}{rgb}{0.000000,0.000000,0.000000}%
\pgfsetstrokecolor{currentstroke}%
\pgfsetdash{}{0pt}%
\pgfpathmoveto{\pgfqpoint{4.121115in}{1.136932in}}%
\pgfpathlineto{\pgfqpoint{4.135576in}{1.140321in}}%
\pgfpathlineto{\pgfqpoint{4.150048in}{1.143881in}}%
\pgfpathlineto{\pgfqpoint{4.164532in}{1.147614in}}%
\pgfpathlineto{\pgfqpoint{4.179027in}{1.151518in}}%
\pgfpathlineto{\pgfqpoint{4.187348in}{1.167195in}}%
\pgfpathlineto{\pgfqpoint{4.195664in}{1.183073in}}%
\pgfpathlineto{\pgfqpoint{4.203977in}{1.199146in}}%
\pgfpathlineto{\pgfqpoint{4.212287in}{1.215405in}}%
\pgfpathlineto{\pgfqpoint{4.197790in}{1.210687in}}%
\pgfpathlineto{\pgfqpoint{4.183306in}{1.206142in}}%
\pgfpathlineto{\pgfqpoint{4.168835in}{1.201771in}}%
\pgfpathlineto{\pgfqpoint{4.154375in}{1.197572in}}%
\pgfpathlineto{\pgfqpoint{4.146066in}{1.182113in}}%
\pgfpathlineto{\pgfqpoint{4.137753in}{1.166848in}}%
\pgfpathlineto{\pgfqpoint{4.129436in}{1.151785in}}%
\pgfpathlineto{\pgfqpoint{4.121115in}{1.136932in}}%
\pgfpathclose%
\pgfusepath{fill}%
\end{pgfscope}%
\begin{pgfscope}%
\pgfpathrectangle{\pgfqpoint{1.150000in}{0.150000in}}{\pgfqpoint{5.700000in}{5.700000in}}%
\pgfusepath{clip}%
\pgfsetbuttcap%
\pgfsetroundjoin%
\definecolor{currentfill}{rgb}{0.244972,0.287675,0.537260}%
\pgfsetfillcolor{currentfill}%
\pgfsetfillopacity{0.800000}%
\pgfsetlinewidth{0.000000pt}%
\definecolor{currentstroke}{rgb}{0.000000,0.000000,0.000000}%
\pgfsetstrokecolor{currentstroke}%
\pgfsetdash{}{0pt}%
\pgfpathmoveto{\pgfqpoint{4.369911in}{1.452001in}}%
\pgfpathlineto{\pgfqpoint{4.384491in}{1.459852in}}%
\pgfpathlineto{\pgfqpoint{4.399085in}{1.467879in}}%
\pgfpathlineto{\pgfqpoint{4.413695in}{1.476082in}}%
\pgfpathlineto{\pgfqpoint{4.428319in}{1.484461in}}%
\pgfpathlineto{\pgfqpoint{4.436603in}{1.503889in}}%
\pgfpathlineto{\pgfqpoint{4.444884in}{1.523389in}}%
\pgfpathlineto{\pgfqpoint{4.453163in}{1.542954in}}%
\pgfpathlineto{\pgfqpoint{4.461439in}{1.562577in}}%
\pgfpathlineto{\pgfqpoint{4.446804in}{1.553500in}}%
\pgfpathlineto{\pgfqpoint{4.432184in}{1.544601in}}%
\pgfpathlineto{\pgfqpoint{4.417581in}{1.535878in}}%
\pgfpathlineto{\pgfqpoint{4.402992in}{1.527332in}}%
\pgfpathlineto{\pgfqpoint{4.394726in}{1.508394in}}%
\pgfpathlineto{\pgfqpoint{4.386457in}{1.489522in}}%
\pgfpathlineto{\pgfqpoint{4.378185in}{1.470722in}}%
\pgfpathlineto{\pgfqpoint{4.369911in}{1.452001in}}%
\pgfpathclose%
\pgfusepath{fill}%
\end{pgfscope}%
\begin{pgfscope}%
\pgfpathrectangle{\pgfqpoint{1.150000in}{0.150000in}}{\pgfqpoint{5.700000in}{5.700000in}}%
\pgfusepath{clip}%
\pgfsetbuttcap%
\pgfsetroundjoin%
\definecolor{currentfill}{rgb}{0.160665,0.478540,0.558115}%
\pgfsetfillcolor{currentfill}%
\pgfsetfillopacity{0.800000}%
\pgfsetlinewidth{0.000000pt}%
\definecolor{currentstroke}{rgb}{0.000000,0.000000,0.000000}%
\pgfsetstrokecolor{currentstroke}%
\pgfsetdash{}{0pt}%
\pgfpathmoveto{\pgfqpoint{4.685348in}{2.008229in}}%
\pgfpathlineto{\pgfqpoint{4.700135in}{2.021234in}}%
\pgfpathlineto{\pgfqpoint{4.714940in}{2.034423in}}%
\pgfpathlineto{\pgfqpoint{4.729764in}{2.047795in}}%
\pgfpathlineto{\pgfqpoint{4.744608in}{2.061351in}}%
\pgfpathlineto{\pgfqpoint{4.752852in}{2.081892in}}%
\pgfpathlineto{\pgfqpoint{4.761092in}{2.102352in}}%
\pgfpathlineto{\pgfqpoint{4.769328in}{2.122728in}}%
\pgfpathlineto{\pgfqpoint{4.777561in}{2.143014in}}%
\pgfpathlineto{\pgfqpoint{4.762702in}{2.128945in}}%
\pgfpathlineto{\pgfqpoint{4.747862in}{2.115060in}}%
\pgfpathlineto{\pgfqpoint{4.733042in}{2.101360in}}%
\pgfpathlineto{\pgfqpoint{4.718241in}{2.087844in}}%
\pgfpathlineto{\pgfqpoint{4.710023in}{2.068057in}}%
\pgfpathlineto{\pgfqpoint{4.701802in}{2.048190in}}%
\pgfpathlineto{\pgfqpoint{4.693577in}{2.028245in}}%
\pgfpathlineto{\pgfqpoint{4.685348in}{2.008229in}}%
\pgfpathclose%
\pgfusepath{fill}%
\end{pgfscope}%
\begin{pgfscope}%
\pgfpathrectangle{\pgfqpoint{1.150000in}{0.150000in}}{\pgfqpoint{5.700000in}{5.700000in}}%
\pgfusepath{clip}%
\pgfsetbuttcap%
\pgfsetroundjoin%
\definecolor{currentfill}{rgb}{0.125394,0.574318,0.549086}%
\pgfsetfillcolor{currentfill}%
\pgfsetfillopacity{0.800000}%
\pgfsetlinewidth{0.000000pt}%
\definecolor{currentstroke}{rgb}{0.000000,0.000000,0.000000}%
\pgfsetstrokecolor{currentstroke}%
\pgfsetdash{}{0pt}%
\pgfpathmoveto{\pgfqpoint{4.843274in}{2.301554in}}%
\pgfpathlineto{\pgfqpoint{4.858184in}{2.316736in}}%
\pgfpathlineto{\pgfqpoint{4.873114in}{2.332106in}}%
\pgfpathlineto{\pgfqpoint{4.888065in}{2.347662in}}%
\pgfpathlineto{\pgfqpoint{4.903038in}{2.363407in}}%
\pgfpathlineto{\pgfqpoint{4.911247in}{2.383120in}}%
\pgfpathlineto{\pgfqpoint{4.919451in}{2.402694in}}%
\pgfpathlineto{\pgfqpoint{4.927650in}{2.422125in}}%
\pgfpathlineto{\pgfqpoint{4.935844in}{2.441408in}}%
\pgfpathlineto{\pgfqpoint{4.920857in}{2.425249in}}%
\pgfpathlineto{\pgfqpoint{4.905890in}{2.409278in}}%
\pgfpathlineto{\pgfqpoint{4.890945in}{2.393495in}}%
\pgfpathlineto{\pgfqpoint{4.876020in}{2.377901in}}%
\pgfpathlineto{\pgfqpoint{4.867841in}{2.359018in}}%
\pgfpathlineto{\pgfqpoint{4.859657in}{2.339996in}}%
\pgfpathlineto{\pgfqpoint{4.851468in}{2.320841in}}%
\pgfpathlineto{\pgfqpoint{4.843274in}{2.301554in}}%
\pgfpathclose%
\pgfusepath{fill}%
\end{pgfscope}%
\begin{pgfscope}%
\pgfpathrectangle{\pgfqpoint{1.150000in}{0.150000in}}{\pgfqpoint{5.700000in}{5.700000in}}%
\pgfusepath{clip}%
\pgfsetbuttcap%
\pgfsetroundjoin%
\definecolor{currentfill}{rgb}{0.259857,0.745492,0.444467}%
\pgfsetfillcolor{currentfill}%
\pgfsetfillopacity{0.800000}%
\pgfsetlinewidth{0.000000pt}%
\definecolor{currentstroke}{rgb}{0.000000,0.000000,0.000000}%
\pgfsetstrokecolor{currentstroke}%
\pgfsetdash{}{0pt}%
\pgfpathmoveto{\pgfqpoint{5.158872in}{2.864158in}}%
\pgfpathlineto{\pgfqpoint{5.174039in}{2.882825in}}%
\pgfpathlineto{\pgfqpoint{5.189229in}{2.901685in}}%
\pgfpathlineto{\pgfqpoint{5.204444in}{2.920739in}}%
\pgfpathlineto{\pgfqpoint{5.219682in}{2.939988in}}%
\pgfpathlineto{\pgfqpoint{5.227759in}{2.955937in}}%
\pgfpathlineto{\pgfqpoint{5.235827in}{2.971662in}}%
\pgfpathlineto{\pgfqpoint{5.243887in}{2.987164in}}%
\pgfpathlineto{\pgfqpoint{5.251937in}{3.002439in}}%
\pgfpathlineto{\pgfqpoint{5.236688in}{2.982987in}}%
\pgfpathlineto{\pgfqpoint{5.221464in}{2.963730in}}%
\pgfpathlineto{\pgfqpoint{5.206263in}{2.944667in}}%
\pgfpathlineto{\pgfqpoint{5.191087in}{2.925798in}}%
\pgfpathlineto{\pgfqpoint{5.183046in}{2.910712in}}%
\pgfpathlineto{\pgfqpoint{5.174996in}{2.895408in}}%
\pgfpathlineto{\pgfqpoint{5.166938in}{2.879890in}}%
\pgfpathlineto{\pgfqpoint{5.158872in}{2.864158in}}%
\pgfpathclose%
\pgfusepath{fill}%
\end{pgfscope}%
\begin{pgfscope}%
\pgfpathrectangle{\pgfqpoint{1.150000in}{0.150000in}}{\pgfqpoint{5.700000in}{5.700000in}}%
\pgfusepath{clip}%
\pgfsetbuttcap%
\pgfsetroundjoin%
\definecolor{currentfill}{rgb}{0.212395,0.359683,0.551710}%
\pgfsetfillcolor{currentfill}%
\pgfsetfillopacity{0.800000}%
\pgfsetlinewidth{0.000000pt}%
\definecolor{currentstroke}{rgb}{0.000000,0.000000,0.000000}%
\pgfsetstrokecolor{currentstroke}%
\pgfsetdash{}{0pt}%
\pgfpathmoveto{\pgfqpoint{4.494518in}{1.641526in}}%
\pgfpathlineto{\pgfqpoint{4.509180in}{1.651447in}}%
\pgfpathlineto{\pgfqpoint{4.523859in}{1.661547in}}%
\pgfpathlineto{\pgfqpoint{4.538554in}{1.671826in}}%
\pgfpathlineto{\pgfqpoint{4.553266in}{1.682283in}}%
\pgfpathlineto{\pgfqpoint{4.561542in}{1.702753in}}%
\pgfpathlineto{\pgfqpoint{4.569816in}{1.723237in}}%
\pgfpathlineto{\pgfqpoint{4.578086in}{1.743728in}}%
\pgfpathlineto{\pgfqpoint{4.586355in}{1.764220in}}%
\pgfpathlineto{\pgfqpoint{4.571629in}{1.753123in}}%
\pgfpathlineto{\pgfqpoint{4.556920in}{1.742207in}}%
\pgfpathlineto{\pgfqpoint{4.542229in}{1.731469in}}%
\pgfpathlineto{\pgfqpoint{4.527555in}{1.720911in}}%
\pgfpathlineto{\pgfqpoint{4.519299in}{1.701045in}}%
\pgfpathlineto{\pgfqpoint{4.511041in}{1.681188in}}%
\pgfpathlineto{\pgfqpoint{4.502781in}{1.661346in}}%
\pgfpathlineto{\pgfqpoint{4.494518in}{1.641526in}}%
\pgfpathclose%
\pgfusepath{fill}%
\end{pgfscope}%
\begin{pgfscope}%
\pgfpathrectangle{\pgfqpoint{1.150000in}{0.150000in}}{\pgfqpoint{5.700000in}{5.700000in}}%
\pgfusepath{clip}%
\pgfsetbuttcap%
\pgfsetroundjoin%
\definecolor{currentfill}{rgb}{0.140210,0.665859,0.513427}%
\pgfsetfillcolor{currentfill}%
\pgfsetfillopacity{0.800000}%
\pgfsetlinewidth{0.000000pt}%
\definecolor{currentstroke}{rgb}{0.000000,0.000000,0.000000}%
\pgfsetstrokecolor{currentstroke}%
\pgfsetdash{}{0pt}%
\pgfpathmoveto{\pgfqpoint{5.001193in}{2.589983in}}%
\pgfpathlineto{\pgfqpoint{5.016232in}{2.607057in}}%
\pgfpathlineto{\pgfqpoint{5.031292in}{2.624322in}}%
\pgfpathlineto{\pgfqpoint{5.046375in}{2.641778in}}%
\pgfpathlineto{\pgfqpoint{5.061481in}{2.659425in}}%
\pgfpathlineto{\pgfqpoint{5.069636in}{2.677562in}}%
\pgfpathlineto{\pgfqpoint{5.077785in}{2.695512in}}%
\pgfpathlineto{\pgfqpoint{5.085927in}{2.713271in}}%
\pgfpathlineto{\pgfqpoint{5.094061in}{2.730838in}}%
\pgfpathlineto{\pgfqpoint{5.078942in}{2.712879in}}%
\pgfpathlineto{\pgfqpoint{5.063845in}{2.695112in}}%
\pgfpathlineto{\pgfqpoint{5.048771in}{2.677537in}}%
\pgfpathlineto{\pgfqpoint{5.033720in}{2.660153in}}%
\pgfpathlineto{\pgfqpoint{5.025598in}{2.642883in}}%
\pgfpathlineto{\pgfqpoint{5.017470in}{2.625430in}}%
\pgfpathlineto{\pgfqpoint{5.009335in}{2.607795in}}%
\pgfpathlineto{\pgfqpoint{5.001193in}{2.589983in}}%
\pgfpathclose%
\pgfusepath{fill}%
\end{pgfscope}%
\begin{pgfscope}%
\pgfpathrectangle{\pgfqpoint{1.150000in}{0.150000in}}{\pgfqpoint{5.700000in}{5.700000in}}%
\pgfusepath{clip}%
\pgfsetbuttcap%
\pgfsetroundjoin%
\definecolor{currentfill}{rgb}{0.275191,0.194905,0.496005}%
\pgfsetfillcolor{currentfill}%
\pgfsetfillopacity{0.800000}%
\pgfsetlinewidth{0.000000pt}%
\definecolor{currentstroke}{rgb}{0.000000,0.000000,0.000000}%
\pgfsetstrokecolor{currentstroke}%
\pgfsetdash{}{0pt}%
\pgfpathmoveto{\pgfqpoint{4.212287in}{1.215405in}}%
\pgfpathlineto{\pgfqpoint{4.226795in}{1.220295in}}%
\pgfpathlineto{\pgfqpoint{4.241317in}{1.225358in}}%
\pgfpathlineto{\pgfqpoint{4.255851in}{1.230595in}}%
\pgfpathlineto{\pgfqpoint{4.270399in}{1.236004in}}%
\pgfpathlineto{\pgfqpoint{4.278707in}{1.253238in}}%
\pgfpathlineto{\pgfqpoint{4.287013in}{1.270635in}}%
\pgfpathlineto{\pgfqpoint{4.295316in}{1.288189in}}%
\pgfpathlineto{\pgfqpoint{4.303616in}{1.305891in}}%
\pgfpathlineto{\pgfqpoint{4.289064in}{1.299696in}}%
\pgfpathlineto{\pgfqpoint{4.274526in}{1.293675in}}%
\pgfpathlineto{\pgfqpoint{4.260001in}{1.287827in}}%
\pgfpathlineto{\pgfqpoint{4.245489in}{1.282154in}}%
\pgfpathlineto{\pgfqpoint{4.237193in}{1.265225in}}%
\pgfpathlineto{\pgfqpoint{4.228895in}{1.248452in}}%
\pgfpathlineto{\pgfqpoint{4.220592in}{1.231843in}}%
\pgfpathlineto{\pgfqpoint{4.212287in}{1.215405in}}%
\pgfpathclose%
\pgfusepath{fill}%
\end{pgfscope}%
\begin{pgfscope}%
\pgfpathrectangle{\pgfqpoint{1.150000in}{0.150000in}}{\pgfqpoint{5.700000in}{5.700000in}}%
\pgfusepath{clip}%
\pgfsetbuttcap%
\pgfsetroundjoin%
\definecolor{currentfill}{rgb}{0.253935,0.265254,0.529983}%
\pgfsetfillcolor{currentfill}%
\pgfsetfillopacity{0.800000}%
\pgfsetlinewidth{0.000000pt}%
\definecolor{currentstroke}{rgb}{0.000000,0.000000,0.000000}%
\pgfsetstrokecolor{currentstroke}%
\pgfsetdash{}{0pt}%
\pgfpathmoveto{\pgfqpoint{4.336786in}{1.378042in}}%
\pgfpathlineto{\pgfqpoint{4.351358in}{1.385168in}}%
\pgfpathlineto{\pgfqpoint{4.365944in}{1.392468in}}%
\pgfpathlineto{\pgfqpoint{4.380545in}{1.399943in}}%
\pgfpathlineto{\pgfqpoint{4.395161in}{1.407594in}}%
\pgfpathlineto{\pgfqpoint{4.403454in}{1.426670in}}%
\pgfpathlineto{\pgfqpoint{4.411745in}{1.445844in}}%
\pgfpathlineto{\pgfqpoint{4.420033in}{1.465110in}}%
\pgfpathlineto{\pgfqpoint{4.428319in}{1.484461in}}%
\pgfpathlineto{\pgfqpoint{4.413695in}{1.476082in}}%
\pgfpathlineto{\pgfqpoint{4.399085in}{1.467879in}}%
\pgfpathlineto{\pgfqpoint{4.384491in}{1.459852in}}%
\pgfpathlineto{\pgfqpoint{4.369911in}{1.452001in}}%
\pgfpathlineto{\pgfqpoint{4.361634in}{1.433366in}}%
\pgfpathlineto{\pgfqpoint{4.353354in}{1.414824in}}%
\pgfpathlineto{\pgfqpoint{4.345071in}{1.396380in}}%
\pgfpathlineto{\pgfqpoint{4.336786in}{1.378042in}}%
\pgfpathclose%
\pgfusepath{fill}%
\end{pgfscope}%
\begin{pgfscope}%
\pgfpathrectangle{\pgfqpoint{1.150000in}{0.150000in}}{\pgfqpoint{5.700000in}{5.700000in}}%
\pgfusepath{clip}%
\pgfsetbuttcap%
\pgfsetroundjoin%
\definecolor{currentfill}{rgb}{0.168126,0.459988,0.558082}%
\pgfsetfillcolor{currentfill}%
\pgfsetfillopacity{0.800000}%
\pgfsetlinewidth{0.000000pt}%
\definecolor{currentstroke}{rgb}{0.000000,0.000000,0.000000}%
\pgfsetstrokecolor{currentstroke}%
\pgfsetdash{}{0pt}%
\pgfpathmoveto{\pgfqpoint{4.652400in}{1.927540in}}%
\pgfpathlineto{\pgfqpoint{4.667171in}{1.940001in}}%
\pgfpathlineto{\pgfqpoint{4.681962in}{1.952645in}}%
\pgfpathlineto{\pgfqpoint{4.696771in}{1.965471in}}%
\pgfpathlineto{\pgfqpoint{4.711599in}{1.978480in}}%
\pgfpathlineto{\pgfqpoint{4.719856in}{1.999294in}}%
\pgfpathlineto{\pgfqpoint{4.728110in}{2.020047in}}%
\pgfpathlineto{\pgfqpoint{4.736361in}{2.040734in}}%
\pgfpathlineto{\pgfqpoint{4.744608in}{2.061351in}}%
\pgfpathlineto{\pgfqpoint{4.729764in}{2.047795in}}%
\pgfpathlineto{\pgfqpoint{4.714940in}{2.034423in}}%
\pgfpathlineto{\pgfqpoint{4.700135in}{2.021234in}}%
\pgfpathlineto{\pgfqpoint{4.685348in}{2.008229in}}%
\pgfpathlineto{\pgfqpoint{4.677116in}{1.988145in}}%
\pgfpathlineto{\pgfqpoint{4.668880in}{1.967999in}}%
\pgfpathlineto{\pgfqpoint{4.660642in}{1.947796in}}%
\pgfpathlineto{\pgfqpoint{4.652400in}{1.927540in}}%
\pgfpathclose%
\pgfusepath{fill}%
\end{pgfscope}%
\begin{pgfscope}%
\pgfpathrectangle{\pgfqpoint{1.150000in}{0.150000in}}{\pgfqpoint{5.700000in}{5.700000in}}%
\pgfusepath{clip}%
\pgfsetbuttcap%
\pgfsetroundjoin%
\definecolor{currentfill}{rgb}{0.395174,0.797475,0.367757}%
\pgfsetfillcolor{currentfill}%
\pgfsetfillopacity{0.800000}%
\pgfsetlinewidth{0.000000pt}%
\definecolor{currentstroke}{rgb}{0.000000,0.000000,0.000000}%
\pgfsetstrokecolor{currentstroke}%
\pgfsetdash{}{0pt}%
\pgfpathmoveto{\pgfqpoint{5.284049in}{3.061255in}}%
\pgfpathlineto{\pgfqpoint{5.299331in}{3.081069in}}%
\pgfpathlineto{\pgfqpoint{5.314638in}{3.101079in}}%
\pgfpathlineto{\pgfqpoint{5.329971in}{3.121286in}}%
\pgfpathlineto{\pgfqpoint{5.345328in}{3.141690in}}%
\pgfpathlineto{\pgfqpoint{5.353341in}{3.155959in}}%
\pgfpathlineto{\pgfqpoint{5.361343in}{3.169987in}}%
\pgfpathlineto{\pgfqpoint{5.369336in}{3.183771in}}%
\pgfpathlineto{\pgfqpoint{5.377318in}{3.197311in}}%
\pgfpathlineto{\pgfqpoint{5.361953in}{3.176778in}}%
\pgfpathlineto{\pgfqpoint{5.346613in}{3.156442in}}%
\pgfpathlineto{\pgfqpoint{5.331298in}{3.136302in}}%
\pgfpathlineto{\pgfqpoint{5.316009in}{3.116359in}}%
\pgfpathlineto{\pgfqpoint{5.308033in}{3.102935in}}%
\pgfpathlineto{\pgfqpoint{5.300048in}{3.089275in}}%
\pgfpathlineto{\pgfqpoint{5.292053in}{3.075382in}}%
\pgfpathlineto{\pgfqpoint{5.284049in}{3.061255in}}%
\pgfpathclose%
\pgfusepath{fill}%
\end{pgfscope}%
\begin{pgfscope}%
\pgfpathrectangle{\pgfqpoint{1.150000in}{0.150000in}}{\pgfqpoint{5.700000in}{5.700000in}}%
\pgfusepath{clip}%
\pgfsetbuttcap%
\pgfsetroundjoin%
\definecolor{currentfill}{rgb}{0.535621,0.835785,0.281908}%
\pgfsetfillcolor{currentfill}%
\pgfsetfillopacity{0.800000}%
\pgfsetlinewidth{0.000000pt}%
\definecolor{currentstroke}{rgb}{0.000000,0.000000,0.000000}%
\pgfsetstrokecolor{currentstroke}%
\pgfsetdash{}{0pt}%
\pgfpathmoveto{\pgfqpoint{5.409141in}{3.249017in}}%
\pgfpathlineto{\pgfqpoint{5.424538in}{3.269840in}}%
\pgfpathlineto{\pgfqpoint{5.439960in}{3.290860in}}%
\pgfpathlineto{\pgfqpoint{5.455409in}{3.312079in}}%
\pgfpathlineto{\pgfqpoint{5.463342in}{3.324441in}}%
\pgfpathlineto{\pgfqpoint{5.471263in}{3.336549in}}%
\pgfpathlineto{\pgfqpoint{5.479173in}{3.348403in}}%
\pgfpathlineto{\pgfqpoint{5.487071in}{3.360004in}}%
\pgfpathlineto{\pgfqpoint{5.471619in}{3.338731in}}%
\pgfpathlineto{\pgfqpoint{5.456192in}{3.317657in}}%
\pgfpathlineto{\pgfqpoint{5.440792in}{3.296781in}}%
\pgfpathlineto{\pgfqpoint{5.432896in}{3.285210in}}%
\pgfpathlineto{\pgfqpoint{5.424989in}{3.273393in}}%
\pgfpathlineto{\pgfqpoint{5.417070in}{3.261328in}}%
\pgfpathlineto{\pgfqpoint{5.409141in}{3.249017in}}%
\pgfpathclose%
\pgfusepath{fill}%
\end{pgfscope}%
\begin{pgfscope}%
\pgfpathrectangle{\pgfqpoint{1.150000in}{0.150000in}}{\pgfqpoint{5.700000in}{5.700000in}}%
\pgfusepath{clip}%
\pgfsetbuttcap%
\pgfsetroundjoin%
\definecolor{currentfill}{rgb}{0.131172,0.555899,0.552459}%
\pgfsetfillcolor{currentfill}%
\pgfsetfillopacity{0.800000}%
\pgfsetlinewidth{0.000000pt}%
\definecolor{currentstroke}{rgb}{0.000000,0.000000,0.000000}%
\pgfsetstrokecolor{currentstroke}%
\pgfsetdash{}{0pt}%
\pgfpathmoveto{\pgfqpoint{4.810452in}{2.223174in}}%
\pgfpathlineto{\pgfqpoint{4.825347in}{2.237909in}}%
\pgfpathlineto{\pgfqpoint{4.840261in}{2.252831in}}%
\pgfpathlineto{\pgfqpoint{4.855197in}{2.267938in}}%
\pgfpathlineto{\pgfqpoint{4.870153in}{2.283233in}}%
\pgfpathlineto{\pgfqpoint{4.878381in}{2.303466in}}%
\pgfpathlineto{\pgfqpoint{4.886605in}{2.323576in}}%
\pgfpathlineto{\pgfqpoint{4.894824in}{2.343557in}}%
\pgfpathlineto{\pgfqpoint{4.903038in}{2.363407in}}%
\pgfpathlineto{\pgfqpoint{4.888065in}{2.347662in}}%
\pgfpathlineto{\pgfqpoint{4.873114in}{2.332106in}}%
\pgfpathlineto{\pgfqpoint{4.858184in}{2.316736in}}%
\pgfpathlineto{\pgfqpoint{4.843274in}{2.301554in}}%
\pgfpathlineto{\pgfqpoint{4.835075in}{2.282140in}}%
\pgfpathlineto{\pgfqpoint{4.826872in}{2.262603in}}%
\pgfpathlineto{\pgfqpoint{4.818664in}{2.242946in}}%
\pgfpathlineto{\pgfqpoint{4.810452in}{2.223174in}}%
\pgfpathclose%
\pgfusepath{fill}%
\end{pgfscope}%
\begin{pgfscope}%
\pgfpathrectangle{\pgfqpoint{1.150000in}{0.150000in}}{\pgfqpoint{5.700000in}{5.700000in}}%
\pgfusepath{clip}%
\pgfsetbuttcap%
\pgfsetroundjoin%
\definecolor{currentfill}{rgb}{0.221989,0.339161,0.548752}%
\pgfsetfillcolor{currentfill}%
\pgfsetfillopacity{0.800000}%
\pgfsetlinewidth{0.000000pt}%
\definecolor{currentstroke}{rgb}{0.000000,0.000000,0.000000}%
\pgfsetstrokecolor{currentstroke}%
\pgfsetdash{}{0pt}%
\pgfpathmoveto{\pgfqpoint{4.461439in}{1.562577in}}%
\pgfpathlineto{\pgfqpoint{4.476090in}{1.571831in}}%
\pgfpathlineto{\pgfqpoint{4.490756in}{1.581262in}}%
\pgfpathlineto{\pgfqpoint{4.505439in}{1.590871in}}%
\pgfpathlineto{\pgfqpoint{4.520139in}{1.600657in}}%
\pgfpathlineto{\pgfqpoint{4.528424in}{1.621013in}}%
\pgfpathlineto{\pgfqpoint{4.536707in}{1.641407in}}%
\pgfpathlineto{\pgfqpoint{4.544988in}{1.661833in}}%
\pgfpathlineto{\pgfqpoint{4.553266in}{1.682283in}}%
\pgfpathlineto{\pgfqpoint{4.538554in}{1.671826in}}%
\pgfpathlineto{\pgfqpoint{4.523859in}{1.661547in}}%
\pgfpathlineto{\pgfqpoint{4.509180in}{1.651447in}}%
\pgfpathlineto{\pgfqpoint{4.494518in}{1.641526in}}%
\pgfpathlineto{\pgfqpoint{4.486252in}{1.621732in}}%
\pgfpathlineto{\pgfqpoint{4.477983in}{1.601972in}}%
\pgfpathlineto{\pgfqpoint{4.469712in}{1.582252in}}%
\pgfpathlineto{\pgfqpoint{4.461439in}{1.562577in}}%
\pgfpathclose%
\pgfusepath{fill}%
\end{pgfscope}%
\begin{pgfscope}%
\pgfpathrectangle{\pgfqpoint{1.150000in}{0.150000in}}{\pgfqpoint{5.700000in}{5.700000in}}%
\pgfusepath{clip}%
\pgfsetbuttcap%
\pgfsetroundjoin%
\definecolor{currentfill}{rgb}{0.128087,0.647749,0.523491}%
\pgfsetfillcolor{currentfill}%
\pgfsetfillopacity{0.800000}%
\pgfsetlinewidth{0.000000pt}%
\definecolor{currentstroke}{rgb}{0.000000,0.000000,0.000000}%
\pgfsetstrokecolor{currentstroke}%
\pgfsetdash{}{0pt}%
\pgfpathmoveto{\pgfqpoint{4.968566in}{2.517004in}}%
\pgfpathlineto{\pgfqpoint{4.983590in}{2.533733in}}%
\pgfpathlineto{\pgfqpoint{4.998636in}{2.550652in}}%
\pgfpathlineto{\pgfqpoint{5.013705in}{2.567761in}}%
\pgfpathlineto{\pgfqpoint{5.028795in}{2.585061in}}%
\pgfpathlineto{\pgfqpoint{5.036976in}{2.603919in}}%
\pgfpathlineto{\pgfqpoint{5.045151in}{2.622601in}}%
\pgfpathlineto{\pgfqpoint{5.053319in}{2.641104in}}%
\pgfpathlineto{\pgfqpoint{5.061481in}{2.659425in}}%
\pgfpathlineto{\pgfqpoint{5.046375in}{2.641778in}}%
\pgfpathlineto{\pgfqpoint{5.031292in}{2.624322in}}%
\pgfpathlineto{\pgfqpoint{5.016232in}{2.607057in}}%
\pgfpathlineto{\pgfqpoint{5.001193in}{2.589983in}}%
\pgfpathlineto{\pgfqpoint{4.993046in}{2.571994in}}%
\pgfpathlineto{\pgfqpoint{4.984892in}{2.553833in}}%
\pgfpathlineto{\pgfqpoint{4.976732in}{2.535502in}}%
\pgfpathlineto{\pgfqpoint{4.968566in}{2.517004in}}%
\pgfpathclose%
\pgfusepath{fill}%
\end{pgfscope}%
\begin{pgfscope}%
\pgfpathrectangle{\pgfqpoint{1.150000in}{0.150000in}}{\pgfqpoint{5.700000in}{5.700000in}}%
\pgfusepath{clip}%
\pgfsetbuttcap%
\pgfsetroundjoin%
\definecolor{currentfill}{rgb}{0.177423,0.437527,0.557565}%
\pgfsetfillcolor{currentfill}%
\pgfsetfillopacity{0.800000}%
\pgfsetlinewidth{0.000000pt}%
\definecolor{currentstroke}{rgb}{0.000000,0.000000,0.000000}%
\pgfsetstrokecolor{currentstroke}%
\pgfsetdash{}{0pt}%
\pgfpathmoveto{\pgfqpoint{4.619400in}{1.846091in}}%
\pgfpathlineto{\pgfqpoint{4.634158in}{1.857976in}}%
\pgfpathlineto{\pgfqpoint{4.648933in}{1.870043in}}%
\pgfpathlineto{\pgfqpoint{4.663727in}{1.882291in}}%
\pgfpathlineto{\pgfqpoint{4.678539in}{1.894720in}}%
\pgfpathlineto{\pgfqpoint{4.686808in}{1.915725in}}%
\pgfpathlineto{\pgfqpoint{4.695075in}{1.936691in}}%
\pgfpathlineto{\pgfqpoint{4.703338in}{1.957611in}}%
\pgfpathlineto{\pgfqpoint{4.711599in}{1.978480in}}%
\pgfpathlineto{\pgfqpoint{4.696771in}{1.965471in}}%
\pgfpathlineto{\pgfqpoint{4.681962in}{1.952645in}}%
\pgfpathlineto{\pgfqpoint{4.667171in}{1.940001in}}%
\pgfpathlineto{\pgfqpoint{4.652400in}{1.927540in}}%
\pgfpathlineto{\pgfqpoint{4.644154in}{1.907236in}}%
\pgfpathlineto{\pgfqpoint{4.635906in}{1.886890in}}%
\pgfpathlineto{\pgfqpoint{4.627655in}{1.866506in}}%
\pgfpathlineto{\pgfqpoint{4.619400in}{1.846091in}}%
\pgfpathclose%
\pgfusepath{fill}%
\end{pgfscope}%
\begin{pgfscope}%
\pgfpathrectangle{\pgfqpoint{1.150000in}{0.150000in}}{\pgfqpoint{5.700000in}{5.700000in}}%
\pgfusepath{clip}%
\pgfsetbuttcap%
\pgfsetroundjoin%
\definecolor{currentfill}{rgb}{0.232815,0.732247,0.459277}%
\pgfsetfillcolor{currentfill}%
\pgfsetfillopacity{0.800000}%
\pgfsetlinewidth{0.000000pt}%
\definecolor{currentstroke}{rgb}{0.000000,0.000000,0.000000}%
\pgfsetstrokecolor{currentstroke}%
\pgfsetdash{}{0pt}%
\pgfpathmoveto{\pgfqpoint{5.126528in}{2.799131in}}%
\pgfpathlineto{\pgfqpoint{5.141683in}{2.817558in}}%
\pgfpathlineto{\pgfqpoint{5.156862in}{2.836179in}}%
\pgfpathlineto{\pgfqpoint{5.172064in}{2.854993in}}%
\pgfpathlineto{\pgfqpoint{5.187291in}{2.874001in}}%
\pgfpathlineto{\pgfqpoint{5.195401in}{2.890823in}}%
\pgfpathlineto{\pgfqpoint{5.203503in}{2.907429in}}%
\pgfpathlineto{\pgfqpoint{5.211597in}{2.923818in}}%
\pgfpathlineto{\pgfqpoint{5.219682in}{2.939988in}}%
\pgfpathlineto{\pgfqpoint{5.204444in}{2.920739in}}%
\pgfpathlineto{\pgfqpoint{5.189229in}{2.901685in}}%
\pgfpathlineto{\pgfqpoint{5.174039in}{2.882825in}}%
\pgfpathlineto{\pgfqpoint{5.158872in}{2.864158in}}%
\pgfpathlineto{\pgfqpoint{5.150798in}{2.848215in}}%
\pgfpathlineto{\pgfqpoint{5.142716in}{2.832061in}}%
\pgfpathlineto{\pgfqpoint{5.134625in}{2.815699in}}%
\pgfpathlineto{\pgfqpoint{5.126528in}{2.799131in}}%
\pgfpathclose%
\pgfusepath{fill}%
\end{pgfscope}%
\begin{pgfscope}%
\pgfpathrectangle{\pgfqpoint{1.150000in}{0.150000in}}{\pgfqpoint{5.700000in}{5.700000in}}%
\pgfusepath{clip}%
\pgfsetbuttcap%
\pgfsetroundjoin%
\definecolor{currentfill}{rgb}{0.262138,0.242286,0.520837}%
\pgfsetfillcolor{currentfill}%
\pgfsetfillopacity{0.800000}%
\pgfsetlinewidth{0.000000pt}%
\definecolor{currentstroke}{rgb}{0.000000,0.000000,0.000000}%
\pgfsetstrokecolor{currentstroke}%
\pgfsetdash{}{0pt}%
\pgfpathmoveto{\pgfqpoint{4.303616in}{1.305891in}}%
\pgfpathlineto{\pgfqpoint{4.318181in}{1.312260in}}%
\pgfpathlineto{\pgfqpoint{4.332761in}{1.318803in}}%
\pgfpathlineto{\pgfqpoint{4.347354in}{1.325520in}}%
\pgfpathlineto{\pgfqpoint{4.361962in}{1.332411in}}%
\pgfpathlineto{\pgfqpoint{4.370265in}{1.351024in}}%
\pgfpathlineto{\pgfqpoint{4.378566in}{1.369764in}}%
\pgfpathlineto{\pgfqpoint{4.386865in}{1.388622in}}%
\pgfpathlineto{\pgfqpoint{4.395161in}{1.407594in}}%
\pgfpathlineto{\pgfqpoint{4.380545in}{1.399943in}}%
\pgfpathlineto{\pgfqpoint{4.365944in}{1.392468in}}%
\pgfpathlineto{\pgfqpoint{4.351358in}{1.385168in}}%
\pgfpathlineto{\pgfqpoint{4.336786in}{1.378042in}}%
\pgfpathlineto{\pgfqpoint{4.328497in}{1.359818in}}%
\pgfpathlineto{\pgfqpoint{4.320206in}{1.341713in}}%
\pgfpathlineto{\pgfqpoint{4.311912in}{1.323735in}}%
\pgfpathlineto{\pgfqpoint{4.303616in}{1.305891in}}%
\pgfpathclose%
\pgfusepath{fill}%
\end{pgfscope}%
\begin{pgfscope}%
\pgfpathrectangle{\pgfqpoint{1.150000in}{0.150000in}}{\pgfqpoint{5.700000in}{5.700000in}}%
\pgfusepath{clip}%
\pgfsetbuttcap%
\pgfsetroundjoin%
\definecolor{currentfill}{rgb}{0.278826,0.175490,0.483397}%
\pgfsetfillcolor{currentfill}%
\pgfsetfillopacity{0.800000}%
\pgfsetlinewidth{0.000000pt}%
\definecolor{currentstroke}{rgb}{0.000000,0.000000,0.000000}%
\pgfsetstrokecolor{currentstroke}%
\pgfsetdash{}{0pt}%
\pgfpathmoveto{\pgfqpoint{4.179027in}{1.151518in}}%
\pgfpathlineto{\pgfqpoint{4.193535in}{1.155595in}}%
\pgfpathlineto{\pgfqpoint{4.208055in}{1.159844in}}%
\pgfpathlineto{\pgfqpoint{4.222587in}{1.164264in}}%
\pgfpathlineto{\pgfqpoint{4.237131in}{1.168857in}}%
\pgfpathlineto{\pgfqpoint{4.245453in}{1.185360in}}%
\pgfpathlineto{\pgfqpoint{4.253772in}{1.202057in}}%
\pgfpathlineto{\pgfqpoint{4.262087in}{1.218941in}}%
\pgfpathlineto{\pgfqpoint{4.270399in}{1.236004in}}%
\pgfpathlineto{\pgfqpoint{4.255851in}{1.230595in}}%
\pgfpathlineto{\pgfqpoint{4.241317in}{1.225358in}}%
\pgfpathlineto{\pgfqpoint{4.226795in}{1.220295in}}%
\pgfpathlineto{\pgfqpoint{4.212287in}{1.215405in}}%
\pgfpathlineto{\pgfqpoint{4.203977in}{1.199146in}}%
\pgfpathlineto{\pgfqpoint{4.195664in}{1.183073in}}%
\pgfpathlineto{\pgfqpoint{4.187348in}{1.167195in}}%
\pgfpathlineto{\pgfqpoint{4.179027in}{1.151518in}}%
\pgfpathclose%
\pgfusepath{fill}%
\end{pgfscope}%
\begin{pgfscope}%
\pgfpathrectangle{\pgfqpoint{1.150000in}{0.150000in}}{\pgfqpoint{5.700000in}{5.700000in}}%
\pgfusepath{clip}%
\pgfsetbuttcap%
\pgfsetroundjoin%
\definecolor{currentfill}{rgb}{0.139147,0.533812,0.555298}%
\pgfsetfillcolor{currentfill}%
\pgfsetfillopacity{0.800000}%
\pgfsetlinewidth{0.000000pt}%
\definecolor{currentstroke}{rgb}{0.000000,0.000000,0.000000}%
\pgfsetstrokecolor{currentstroke}%
\pgfsetdash{}{0pt}%
\pgfpathmoveto{\pgfqpoint{4.777561in}{2.143014in}}%
\pgfpathlineto{\pgfqpoint{4.792440in}{2.157268in}}%
\pgfpathlineto{\pgfqpoint{4.807339in}{2.171708in}}%
\pgfpathlineto{\pgfqpoint{4.822258in}{2.186333in}}%
\pgfpathlineto{\pgfqpoint{4.837197in}{2.201143in}}%
\pgfpathlineto{\pgfqpoint{4.845442in}{2.221831in}}%
\pgfpathlineto{\pgfqpoint{4.853683in}{2.242411in}}%
\pgfpathlineto{\pgfqpoint{4.861920in}{2.262880in}}%
\pgfpathlineto{\pgfqpoint{4.870153in}{2.283233in}}%
\pgfpathlineto{\pgfqpoint{4.855197in}{2.267938in}}%
\pgfpathlineto{\pgfqpoint{4.840261in}{2.252831in}}%
\pgfpathlineto{\pgfqpoint{4.825347in}{2.237909in}}%
\pgfpathlineto{\pgfqpoint{4.810452in}{2.223174in}}%
\pgfpathlineto{\pgfqpoint{4.802235in}{2.203290in}}%
\pgfpathlineto{\pgfqpoint{4.794015in}{2.183300in}}%
\pgfpathlineto{\pgfqpoint{4.785790in}{2.163206in}}%
\pgfpathlineto{\pgfqpoint{4.777561in}{2.143014in}}%
\pgfpathclose%
\pgfusepath{fill}%
\end{pgfscope}%
\begin{pgfscope}%
\pgfpathrectangle{\pgfqpoint{1.150000in}{0.150000in}}{\pgfqpoint{5.700000in}{5.700000in}}%
\pgfusepath{clip}%
\pgfsetbuttcap%
\pgfsetroundjoin%
\definecolor{currentfill}{rgb}{0.233603,0.313828,0.543914}%
\pgfsetfillcolor{currentfill}%
\pgfsetfillopacity{0.800000}%
\pgfsetlinewidth{0.000000pt}%
\definecolor{currentstroke}{rgb}{0.000000,0.000000,0.000000}%
\pgfsetstrokecolor{currentstroke}%
\pgfsetdash{}{0pt}%
\pgfpathmoveto{\pgfqpoint{4.428319in}{1.484461in}}%
\pgfpathlineto{\pgfqpoint{4.442960in}{1.493016in}}%
\pgfpathlineto{\pgfqpoint{4.457615in}{1.501747in}}%
\pgfpathlineto{\pgfqpoint{4.472287in}{1.510655in}}%
\pgfpathlineto{\pgfqpoint{4.486974in}{1.519740in}}%
\pgfpathlineto{\pgfqpoint{4.495268in}{1.539880in}}%
\pgfpathlineto{\pgfqpoint{4.503561in}{1.560084in}}%
\pgfpathlineto{\pgfqpoint{4.511851in}{1.580346in}}%
\pgfpathlineto{\pgfqpoint{4.520139in}{1.600657in}}%
\pgfpathlineto{\pgfqpoint{4.505439in}{1.590871in}}%
\pgfpathlineto{\pgfqpoint{4.490756in}{1.581262in}}%
\pgfpathlineto{\pgfqpoint{4.476090in}{1.571831in}}%
\pgfpathlineto{\pgfqpoint{4.461439in}{1.562577in}}%
\pgfpathlineto{\pgfqpoint{4.453163in}{1.542954in}}%
\pgfpathlineto{\pgfqpoint{4.444884in}{1.523389in}}%
\pgfpathlineto{\pgfqpoint{4.436603in}{1.503889in}}%
\pgfpathlineto{\pgfqpoint{4.428319in}{1.484461in}}%
\pgfpathclose%
\pgfusepath{fill}%
\end{pgfscope}%
\begin{pgfscope}%
\pgfpathrectangle{\pgfqpoint{1.150000in}{0.150000in}}{\pgfqpoint{5.700000in}{5.700000in}}%
\pgfusepath{clip}%
\pgfsetbuttcap%
\pgfsetroundjoin%
\definecolor{currentfill}{rgb}{0.187231,0.414746,0.556547}%
\pgfsetfillcolor{currentfill}%
\pgfsetfillopacity{0.800000}%
\pgfsetlinewidth{0.000000pt}%
\definecolor{currentstroke}{rgb}{0.000000,0.000000,0.000000}%
\pgfsetstrokecolor{currentstroke}%
\pgfsetdash{}{0pt}%
\pgfpathmoveto{\pgfqpoint{4.586355in}{1.764220in}}%
\pgfpathlineto{\pgfqpoint{4.601098in}{1.775497in}}%
\pgfpathlineto{\pgfqpoint{4.615858in}{1.786954in}}%
\pgfpathlineto{\pgfqpoint{4.630637in}{1.798591in}}%
\pgfpathlineto{\pgfqpoint{4.645433in}{1.810408in}}%
\pgfpathlineto{\pgfqpoint{4.653714in}{1.831519in}}%
\pgfpathlineto{\pgfqpoint{4.661991in}{1.852611in}}%
\pgfpathlineto{\pgfqpoint{4.670267in}{1.873680in}}%
\pgfpathlineto{\pgfqpoint{4.678539in}{1.894720in}}%
\pgfpathlineto{\pgfqpoint{4.663727in}{1.882291in}}%
\pgfpathlineto{\pgfqpoint{4.648933in}{1.870043in}}%
\pgfpathlineto{\pgfqpoint{4.634158in}{1.857976in}}%
\pgfpathlineto{\pgfqpoint{4.619400in}{1.846091in}}%
\pgfpathlineto{\pgfqpoint{4.611143in}{1.825649in}}%
\pgfpathlineto{\pgfqpoint{4.602883in}{1.805187in}}%
\pgfpathlineto{\pgfqpoint{4.594620in}{1.784708in}}%
\pgfpathlineto{\pgfqpoint{4.586355in}{1.764220in}}%
\pgfpathclose%
\pgfusepath{fill}%
\end{pgfscope}%
\begin{pgfscope}%
\pgfpathrectangle{\pgfqpoint{1.150000in}{0.150000in}}{\pgfqpoint{5.700000in}{5.700000in}}%
\pgfusepath{clip}%
\pgfsetbuttcap%
\pgfsetroundjoin%
\definecolor{currentfill}{rgb}{0.369214,0.788888,0.382914}%
\pgfsetfillcolor{currentfill}%
\pgfsetfillopacity{0.800000}%
\pgfsetlinewidth{0.000000pt}%
\definecolor{currentstroke}{rgb}{0.000000,0.000000,0.000000}%
\pgfsetstrokecolor{currentstroke}%
\pgfsetdash{}{0pt}%
\pgfpathmoveto{\pgfqpoint{5.251937in}{3.002439in}}%
\pgfpathlineto{\pgfqpoint{5.267211in}{3.022087in}}%
\pgfpathlineto{\pgfqpoint{5.282509in}{3.041930in}}%
\pgfpathlineto{\pgfqpoint{5.297832in}{3.061970in}}%
\pgfpathlineto{\pgfqpoint{5.313180in}{3.082206in}}%
\pgfpathlineto{\pgfqpoint{5.321231in}{3.097435in}}%
\pgfpathlineto{\pgfqpoint{5.329273in}{3.112426in}}%
\pgfpathlineto{\pgfqpoint{5.337306in}{3.127178in}}%
\pgfpathlineto{\pgfqpoint{5.345328in}{3.141690in}}%
\pgfpathlineto{\pgfqpoint{5.329971in}{3.121286in}}%
\pgfpathlineto{\pgfqpoint{5.314638in}{3.101079in}}%
\pgfpathlineto{\pgfqpoint{5.299331in}{3.081069in}}%
\pgfpathlineto{\pgfqpoint{5.284049in}{3.061255in}}%
\pgfpathlineto{\pgfqpoint{5.276035in}{3.046897in}}%
\pgfpathlineto{\pgfqpoint{5.268012in}{3.032307in}}%
\pgfpathlineto{\pgfqpoint{5.259979in}{3.017487in}}%
\pgfpathlineto{\pgfqpoint{5.251937in}{3.002439in}}%
\pgfpathclose%
\pgfusepath{fill}%
\end{pgfscope}%
\begin{pgfscope}%
\pgfpathrectangle{\pgfqpoint{1.150000in}{0.150000in}}{\pgfqpoint{5.700000in}{5.700000in}}%
\pgfusepath{clip}%
\pgfsetbuttcap%
\pgfsetroundjoin%
\definecolor{currentfill}{rgb}{0.121380,0.629492,0.531973}%
\pgfsetfillcolor{currentfill}%
\pgfsetfillopacity{0.800000}%
\pgfsetlinewidth{0.000000pt}%
\definecolor{currentstroke}{rgb}{0.000000,0.000000,0.000000}%
\pgfsetstrokecolor{currentstroke}%
\pgfsetdash{}{0pt}%
\pgfpathmoveto{\pgfqpoint{4.935844in}{2.441408in}}%
\pgfpathlineto{\pgfqpoint{4.950854in}{2.457757in}}%
\pgfpathlineto{\pgfqpoint{4.965885in}{2.474294in}}%
\pgfpathlineto{\pgfqpoint{4.980937in}{2.491020in}}%
\pgfpathlineto{\pgfqpoint{4.996012in}{2.507937in}}%
\pgfpathlineto{\pgfqpoint{5.004217in}{2.527465in}}%
\pgfpathlineto{\pgfqpoint{5.012415in}{2.546831in}}%
\pgfpathlineto{\pgfqpoint{5.020608in}{2.566031in}}%
\pgfpathlineto{\pgfqpoint{5.028795in}{2.585061in}}%
\pgfpathlineto{\pgfqpoint{5.013705in}{2.567761in}}%
\pgfpathlineto{\pgfqpoint{4.998636in}{2.550652in}}%
\pgfpathlineto{\pgfqpoint{4.983590in}{2.533733in}}%
\pgfpathlineto{\pgfqpoint{4.968566in}{2.517004in}}%
\pgfpathlineto{\pgfqpoint{4.960394in}{2.498343in}}%
\pgfpathlineto{\pgfqpoint{4.952216in}{2.479521in}}%
\pgfpathlineto{\pgfqpoint{4.944033in}{2.460542in}}%
\pgfpathlineto{\pgfqpoint{4.935844in}{2.441408in}}%
\pgfpathclose%
\pgfusepath{fill}%
\end{pgfscope}%
\begin{pgfscope}%
\pgfpathrectangle{\pgfqpoint{1.150000in}{0.150000in}}{\pgfqpoint{5.700000in}{5.700000in}}%
\pgfusepath{clip}%
\pgfsetbuttcap%
\pgfsetroundjoin%
\definecolor{currentfill}{rgb}{0.515992,0.831158,0.294279}%
\pgfsetfillcolor{currentfill}%
\pgfsetfillopacity{0.800000}%
\pgfsetlinewidth{0.000000pt}%
\definecolor{currentstroke}{rgb}{0.000000,0.000000,0.000000}%
\pgfsetstrokecolor{currentstroke}%
\pgfsetdash{}{0pt}%
\pgfpathmoveto{\pgfqpoint{5.377318in}{3.197311in}}%
\pgfpathlineto{\pgfqpoint{5.392709in}{3.218041in}}%
\pgfpathlineto{\pgfqpoint{5.408125in}{3.238970in}}%
\pgfpathlineto{\pgfqpoint{5.423567in}{3.260097in}}%
\pgfpathlineto{\pgfqpoint{5.431544in}{3.273472in}}%
\pgfpathlineto{\pgfqpoint{5.439510in}{3.286594in}}%
\pgfpathlineto{\pgfqpoint{5.447465in}{3.299463in}}%
\pgfpathlineto{\pgfqpoint{5.455409in}{3.312079in}}%
\pgfpathlineto{\pgfqpoint{5.439960in}{3.290860in}}%
\pgfpathlineto{\pgfqpoint{5.424538in}{3.269840in}}%
\pgfpathlineto{\pgfqpoint{5.409141in}{3.249017in}}%
\pgfpathlineto{\pgfqpoint{5.401201in}{3.236460in}}%
\pgfpathlineto{\pgfqpoint{5.393251in}{3.223656in}}%
\pgfpathlineto{\pgfqpoint{5.385290in}{3.210606in}}%
\pgfpathlineto{\pgfqpoint{5.377318in}{3.197311in}}%
\pgfpathclose%
\pgfusepath{fill}%
\end{pgfscope}%
\begin{pgfscope}%
\pgfpathrectangle{\pgfqpoint{1.150000in}{0.150000in}}{\pgfqpoint{5.700000in}{5.700000in}}%
\pgfusepath{clip}%
\pgfsetbuttcap%
\pgfsetroundjoin%
\definecolor{currentfill}{rgb}{0.269308,0.218818,0.509577}%
\pgfsetfillcolor{currentfill}%
\pgfsetfillopacity{0.800000}%
\pgfsetlinewidth{0.000000pt}%
\definecolor{currentstroke}{rgb}{0.000000,0.000000,0.000000}%
\pgfsetstrokecolor{currentstroke}%
\pgfsetdash{}{0pt}%
\pgfpathmoveto{\pgfqpoint{4.270399in}{1.236004in}}%
\pgfpathlineto{\pgfqpoint{4.284959in}{1.241586in}}%
\pgfpathlineto{\pgfqpoint{4.299533in}{1.247341in}}%
\pgfpathlineto{\pgfqpoint{4.314121in}{1.253269in}}%
\pgfpathlineto{\pgfqpoint{4.328722in}{1.259370in}}%
\pgfpathlineto{\pgfqpoint{4.337036in}{1.277404in}}%
\pgfpathlineto{\pgfqpoint{4.345347in}{1.295594in}}%
\pgfpathlineto{\pgfqpoint{4.353656in}{1.313932in}}%
\pgfpathlineto{\pgfqpoint{4.361962in}{1.332411in}}%
\pgfpathlineto{\pgfqpoint{4.347354in}{1.325520in}}%
\pgfpathlineto{\pgfqpoint{4.332761in}{1.318803in}}%
\pgfpathlineto{\pgfqpoint{4.318181in}{1.312260in}}%
\pgfpathlineto{\pgfqpoint{4.303616in}{1.305891in}}%
\pgfpathlineto{\pgfqpoint{4.295316in}{1.288189in}}%
\pgfpathlineto{\pgfqpoint{4.287013in}{1.270635in}}%
\pgfpathlineto{\pgfqpoint{4.278707in}{1.253238in}}%
\pgfpathlineto{\pgfqpoint{4.270399in}{1.236004in}}%
\pgfpathclose%
\pgfusepath{fill}%
\end{pgfscope}%
\begin{pgfscope}%
\pgfpathrectangle{\pgfqpoint{1.150000in}{0.150000in}}{\pgfqpoint{5.700000in}{5.700000in}}%
\pgfusepath{clip}%
\pgfsetbuttcap%
\pgfsetroundjoin%
\definecolor{currentfill}{rgb}{0.208030,0.718701,0.472873}%
\pgfsetfillcolor{currentfill}%
\pgfsetfillopacity{0.800000}%
\pgfsetlinewidth{0.000000pt}%
\definecolor{currentstroke}{rgb}{0.000000,0.000000,0.000000}%
\pgfsetstrokecolor{currentstroke}%
\pgfsetdash{}{0pt}%
\pgfpathmoveto{\pgfqpoint{5.094061in}{2.730838in}}%
\pgfpathlineto{\pgfqpoint{5.109204in}{2.748989in}}%
\pgfpathlineto{\pgfqpoint{5.124370in}{2.767333in}}%
\pgfpathlineto{\pgfqpoint{5.139559in}{2.785870in}}%
\pgfpathlineto{\pgfqpoint{5.154772in}{2.804601in}}%
\pgfpathlineto{\pgfqpoint{5.162913in}{2.822263in}}%
\pgfpathlineto{\pgfqpoint{5.171047in}{2.839719in}}%
\pgfpathlineto{\pgfqpoint{5.179173in}{2.856965in}}%
\pgfpathlineto{\pgfqpoint{5.187291in}{2.874001in}}%
\pgfpathlineto{\pgfqpoint{5.172064in}{2.854993in}}%
\pgfpathlineto{\pgfqpoint{5.156862in}{2.836179in}}%
\pgfpathlineto{\pgfqpoint{5.141683in}{2.817558in}}%
\pgfpathlineto{\pgfqpoint{5.126528in}{2.799131in}}%
\pgfpathlineto{\pgfqpoint{5.118422in}{2.782359in}}%
\pgfpathlineto{\pgfqpoint{5.110309in}{2.765384in}}%
\pgfpathlineto{\pgfqpoint{5.102189in}{2.748210in}}%
\pgfpathlineto{\pgfqpoint{5.094061in}{2.730838in}}%
\pgfpathclose%
\pgfusepath{fill}%
\end{pgfscope}%
\begin{pgfscope}%
\pgfpathrectangle{\pgfqpoint{1.150000in}{0.150000in}}{\pgfqpoint{5.700000in}{5.700000in}}%
\pgfusepath{clip}%
\pgfsetbuttcap%
\pgfsetroundjoin%
\definecolor{currentfill}{rgb}{0.147607,0.511733,0.557049}%
\pgfsetfillcolor{currentfill}%
\pgfsetfillopacity{0.800000}%
\pgfsetlinewidth{0.000000pt}%
\definecolor{currentstroke}{rgb}{0.000000,0.000000,0.000000}%
\pgfsetstrokecolor{currentstroke}%
\pgfsetdash{}{0pt}%
\pgfpathmoveto{\pgfqpoint{4.744608in}{2.061351in}}%
\pgfpathlineto{\pgfqpoint{4.759471in}{2.075090in}}%
\pgfpathlineto{\pgfqpoint{4.774353in}{2.089014in}}%
\pgfpathlineto{\pgfqpoint{4.789256in}{2.103123in}}%
\pgfpathlineto{\pgfqpoint{4.804178in}{2.117415in}}%
\pgfpathlineto{\pgfqpoint{4.812438in}{2.138485in}}%
\pgfpathlineto{\pgfqpoint{4.820695in}{2.159466in}}%
\pgfpathlineto{\pgfqpoint{4.828948in}{2.180354in}}%
\pgfpathlineto{\pgfqpoint{4.837197in}{2.201143in}}%
\pgfpathlineto{\pgfqpoint{4.822258in}{2.186333in}}%
\pgfpathlineto{\pgfqpoint{4.807339in}{2.171708in}}%
\pgfpathlineto{\pgfqpoint{4.792440in}{2.157268in}}%
\pgfpathlineto{\pgfqpoint{4.777561in}{2.143014in}}%
\pgfpathlineto{\pgfqpoint{4.769328in}{2.122728in}}%
\pgfpathlineto{\pgfqpoint{4.761092in}{2.102352in}}%
\pgfpathlineto{\pgfqpoint{4.752852in}{2.081892in}}%
\pgfpathlineto{\pgfqpoint{4.744608in}{2.061351in}}%
\pgfpathclose%
\pgfusepath{fill}%
\end{pgfscope}%
\begin{pgfscope}%
\pgfpathrectangle{\pgfqpoint{1.150000in}{0.150000in}}{\pgfqpoint{5.700000in}{5.700000in}}%
\pgfusepath{clip}%
\pgfsetbuttcap%
\pgfsetroundjoin%
\definecolor{currentfill}{rgb}{0.244972,0.287675,0.537260}%
\pgfsetfillcolor{currentfill}%
\pgfsetfillopacity{0.800000}%
\pgfsetlinewidth{0.000000pt}%
\definecolor{currentstroke}{rgb}{0.000000,0.000000,0.000000}%
\pgfsetstrokecolor{currentstroke}%
\pgfsetdash{}{0pt}%
\pgfpathmoveto{\pgfqpoint{4.395161in}{1.407594in}}%
\pgfpathlineto{\pgfqpoint{4.409791in}{1.415419in}}%
\pgfpathlineto{\pgfqpoint{4.424437in}{1.423420in}}%
\pgfpathlineto{\pgfqpoint{4.439097in}{1.431596in}}%
\pgfpathlineto{\pgfqpoint{4.453773in}{1.439947in}}%
\pgfpathlineto{\pgfqpoint{4.462077in}{1.459766in}}%
\pgfpathlineto{\pgfqpoint{4.470378in}{1.479676in}}%
\pgfpathlineto{\pgfqpoint{4.478677in}{1.499669in}}%
\pgfpathlineto{\pgfqpoint{4.486974in}{1.519740in}}%
\pgfpathlineto{\pgfqpoint{4.472287in}{1.510655in}}%
\pgfpathlineto{\pgfqpoint{4.457615in}{1.501747in}}%
\pgfpathlineto{\pgfqpoint{4.442960in}{1.493016in}}%
\pgfpathlineto{\pgfqpoint{4.428319in}{1.484461in}}%
\pgfpathlineto{\pgfqpoint{4.420033in}{1.465110in}}%
\pgfpathlineto{\pgfqpoint{4.411745in}{1.445844in}}%
\pgfpathlineto{\pgfqpoint{4.403454in}{1.426670in}}%
\pgfpathlineto{\pgfqpoint{4.395161in}{1.407594in}}%
\pgfpathclose%
\pgfusepath{fill}%
\end{pgfscope}%
\begin{pgfscope}%
\pgfpathrectangle{\pgfqpoint{1.150000in}{0.150000in}}{\pgfqpoint{5.700000in}{5.700000in}}%
\pgfusepath{clip}%
\pgfsetbuttcap%
\pgfsetroundjoin%
\definecolor{currentfill}{rgb}{0.197636,0.391528,0.554969}%
\pgfsetfillcolor{currentfill}%
\pgfsetfillopacity{0.800000}%
\pgfsetlinewidth{0.000000pt}%
\definecolor{currentstroke}{rgb}{0.000000,0.000000,0.000000}%
\pgfsetstrokecolor{currentstroke}%
\pgfsetdash{}{0pt}%
\pgfpathmoveto{\pgfqpoint{4.553266in}{1.682283in}}%
\pgfpathlineto{\pgfqpoint{4.567996in}{1.692919in}}%
\pgfpathlineto{\pgfqpoint{4.582742in}{1.703735in}}%
\pgfpathlineto{\pgfqpoint{4.597506in}{1.714729in}}%
\pgfpathlineto{\pgfqpoint{4.612287in}{1.725903in}}%
\pgfpathlineto{\pgfqpoint{4.620577in}{1.747027in}}%
\pgfpathlineto{\pgfqpoint{4.628865in}{1.768156in}}%
\pgfpathlineto{\pgfqpoint{4.637150in}{1.789285in}}%
\pgfpathlineto{\pgfqpoint{4.645433in}{1.810408in}}%
\pgfpathlineto{\pgfqpoint{4.630637in}{1.798591in}}%
\pgfpathlineto{\pgfqpoint{4.615858in}{1.786954in}}%
\pgfpathlineto{\pgfqpoint{4.601098in}{1.775497in}}%
\pgfpathlineto{\pgfqpoint{4.586355in}{1.764220in}}%
\pgfpathlineto{\pgfqpoint{4.578086in}{1.743728in}}%
\pgfpathlineto{\pgfqpoint{4.569816in}{1.723237in}}%
\pgfpathlineto{\pgfqpoint{4.561542in}{1.702753in}}%
\pgfpathlineto{\pgfqpoint{4.553266in}{1.682283in}}%
\pgfpathclose%
\pgfusepath{fill}%
\end{pgfscope}%
\begin{pgfscope}%
\pgfpathrectangle{\pgfqpoint{1.150000in}{0.150000in}}{\pgfqpoint{5.700000in}{5.700000in}}%
\pgfusepath{clip}%
\pgfsetbuttcap%
\pgfsetroundjoin%
\definecolor{currentfill}{rgb}{0.119512,0.607464,0.540218}%
\pgfsetfillcolor{currentfill}%
\pgfsetfillopacity{0.800000}%
\pgfsetlinewidth{0.000000pt}%
\definecolor{currentstroke}{rgb}{0.000000,0.000000,0.000000}%
\pgfsetstrokecolor{currentstroke}%
\pgfsetdash{}{0pt}%
\pgfpathmoveto{\pgfqpoint{4.903038in}{2.363407in}}%
\pgfpathlineto{\pgfqpoint{4.918031in}{2.379339in}}%
\pgfpathlineto{\pgfqpoint{4.933046in}{2.395459in}}%
\pgfpathlineto{\pgfqpoint{4.948082in}{2.411768in}}%
\pgfpathlineto{\pgfqpoint{4.963140in}{2.428266in}}%
\pgfpathlineto{\pgfqpoint{4.971366in}{2.448410in}}%
\pgfpathlineto{\pgfqpoint{4.979587in}{2.468405in}}%
\pgfpathlineto{\pgfqpoint{4.987802in}{2.488249in}}%
\pgfpathlineto{\pgfqpoint{4.996012in}{2.507937in}}%
\pgfpathlineto{\pgfqpoint{4.980937in}{2.491020in}}%
\pgfpathlineto{\pgfqpoint{4.965885in}{2.474294in}}%
\pgfpathlineto{\pgfqpoint{4.950854in}{2.457757in}}%
\pgfpathlineto{\pgfqpoint{4.935844in}{2.441408in}}%
\pgfpathlineto{\pgfqpoint{4.927650in}{2.422125in}}%
\pgfpathlineto{\pgfqpoint{4.919451in}{2.402694in}}%
\pgfpathlineto{\pgfqpoint{4.911247in}{2.383120in}}%
\pgfpathlineto{\pgfqpoint{4.903038in}{2.363407in}}%
\pgfpathclose%
\pgfusepath{fill}%
\end{pgfscope}%
\begin{pgfscope}%
\pgfpathrectangle{\pgfqpoint{1.150000in}{0.150000in}}{\pgfqpoint{5.700000in}{5.700000in}}%
\pgfusepath{clip}%
\pgfsetbuttcap%
\pgfsetroundjoin%
\definecolor{currentfill}{rgb}{0.156270,0.489624,0.557936}%
\pgfsetfillcolor{currentfill}%
\pgfsetfillopacity{0.800000}%
\pgfsetlinewidth{0.000000pt}%
\definecolor{currentstroke}{rgb}{0.000000,0.000000,0.000000}%
\pgfsetstrokecolor{currentstroke}%
\pgfsetdash{}{0pt}%
\pgfpathmoveto{\pgfqpoint{4.711599in}{1.978480in}}%
\pgfpathlineto{\pgfqpoint{4.726445in}{1.991672in}}%
\pgfpathlineto{\pgfqpoint{4.741311in}{2.005047in}}%
\pgfpathlineto{\pgfqpoint{4.756197in}{2.018605in}}%
\pgfpathlineto{\pgfqpoint{4.771101in}{2.032347in}}%
\pgfpathlineto{\pgfqpoint{4.779375in}{2.053723in}}%
\pgfpathlineto{\pgfqpoint{4.787646in}{2.075029in}}%
\pgfpathlineto{\pgfqpoint{4.795914in}{2.096262in}}%
\pgfpathlineto{\pgfqpoint{4.804178in}{2.117415in}}%
\pgfpathlineto{\pgfqpoint{4.789256in}{2.103123in}}%
\pgfpathlineto{\pgfqpoint{4.774353in}{2.089014in}}%
\pgfpathlineto{\pgfqpoint{4.759471in}{2.075090in}}%
\pgfpathlineto{\pgfqpoint{4.744608in}{2.061351in}}%
\pgfpathlineto{\pgfqpoint{4.736361in}{2.040734in}}%
\pgfpathlineto{\pgfqpoint{4.728110in}{2.020047in}}%
\pgfpathlineto{\pgfqpoint{4.719856in}{1.999294in}}%
\pgfpathlineto{\pgfqpoint{4.711599in}{1.978480in}}%
\pgfpathclose%
\pgfusepath{fill}%
\end{pgfscope}%
\begin{pgfscope}%
\pgfpathrectangle{\pgfqpoint{1.150000in}{0.150000in}}{\pgfqpoint{5.700000in}{5.700000in}}%
\pgfusepath{clip}%
\pgfsetbuttcap%
\pgfsetroundjoin%
\definecolor{currentfill}{rgb}{0.344074,0.780029,0.397381}%
\pgfsetfillcolor{currentfill}%
\pgfsetfillopacity{0.800000}%
\pgfsetlinewidth{0.000000pt}%
\definecolor{currentstroke}{rgb}{0.000000,0.000000,0.000000}%
\pgfsetstrokecolor{currentstroke}%
\pgfsetdash{}{0pt}%
\pgfpathmoveto{\pgfqpoint{5.219682in}{2.939988in}}%
\pgfpathlineto{\pgfqpoint{5.234945in}{2.959432in}}%
\pgfpathlineto{\pgfqpoint{5.250232in}{2.979071in}}%
\pgfpathlineto{\pgfqpoint{5.265544in}{2.998905in}}%
\pgfpathlineto{\pgfqpoint{5.280880in}{3.018936in}}%
\pgfpathlineto{\pgfqpoint{5.288969in}{3.035103in}}%
\pgfpathlineto{\pgfqpoint{5.297048in}{3.051039in}}%
\pgfpathlineto{\pgfqpoint{5.305119in}{3.066740in}}%
\pgfpathlineto{\pgfqpoint{5.313180in}{3.082206in}}%
\pgfpathlineto{\pgfqpoint{5.297832in}{3.061970in}}%
\pgfpathlineto{\pgfqpoint{5.282509in}{3.041930in}}%
\pgfpathlineto{\pgfqpoint{5.267211in}{3.022087in}}%
\pgfpathlineto{\pgfqpoint{5.251937in}{3.002439in}}%
\pgfpathlineto{\pgfqpoint{5.243887in}{2.987164in}}%
\pgfpathlineto{\pgfqpoint{5.235827in}{2.971662in}}%
\pgfpathlineto{\pgfqpoint{5.227759in}{2.955937in}}%
\pgfpathlineto{\pgfqpoint{5.219682in}{2.939988in}}%
\pgfpathclose%
\pgfusepath{fill}%
\end{pgfscope}%
\begin{pgfscope}%
\pgfpathrectangle{\pgfqpoint{1.150000in}{0.150000in}}{\pgfqpoint{5.700000in}{5.700000in}}%
\pgfusepath{clip}%
\pgfsetbuttcap%
\pgfsetroundjoin%
\definecolor{currentfill}{rgb}{0.180653,0.701402,0.488189}%
\pgfsetfillcolor{currentfill}%
\pgfsetfillopacity{0.800000}%
\pgfsetlinewidth{0.000000pt}%
\definecolor{currentstroke}{rgb}{0.000000,0.000000,0.000000}%
\pgfsetstrokecolor{currentstroke}%
\pgfsetdash{}{0pt}%
\pgfpathmoveto{\pgfqpoint{5.061481in}{2.659425in}}%
\pgfpathlineto{\pgfqpoint{5.076610in}{2.677264in}}%
\pgfpathlineto{\pgfqpoint{5.091761in}{2.695295in}}%
\pgfpathlineto{\pgfqpoint{5.106936in}{2.713518in}}%
\pgfpathlineto{\pgfqpoint{5.122134in}{2.731934in}}%
\pgfpathlineto{\pgfqpoint{5.130304in}{2.750398in}}%
\pgfpathlineto{\pgfqpoint{5.138467in}{2.768666in}}%
\pgfpathlineto{\pgfqpoint{5.146623in}{2.786734in}}%
\pgfpathlineto{\pgfqpoint{5.154772in}{2.804601in}}%
\pgfpathlineto{\pgfqpoint{5.139559in}{2.785870in}}%
\pgfpathlineto{\pgfqpoint{5.124370in}{2.767333in}}%
\pgfpathlineto{\pgfqpoint{5.109204in}{2.748989in}}%
\pgfpathlineto{\pgfqpoint{5.094061in}{2.730838in}}%
\pgfpathlineto{\pgfqpoint{5.085927in}{2.713271in}}%
\pgfpathlineto{\pgfqpoint{5.077785in}{2.695512in}}%
\pgfpathlineto{\pgfqpoint{5.069636in}{2.677562in}}%
\pgfpathlineto{\pgfqpoint{5.061481in}{2.659425in}}%
\pgfpathclose%
\pgfusepath{fill}%
\end{pgfscope}%
\begin{pgfscope}%
\pgfpathrectangle{\pgfqpoint{1.150000in}{0.150000in}}{\pgfqpoint{5.700000in}{5.700000in}}%
\pgfusepath{clip}%
\pgfsetbuttcap%
\pgfsetroundjoin%
\definecolor{currentfill}{rgb}{0.275191,0.194905,0.496005}%
\pgfsetfillcolor{currentfill}%
\pgfsetfillopacity{0.800000}%
\pgfsetlinewidth{0.000000pt}%
\definecolor{currentstroke}{rgb}{0.000000,0.000000,0.000000}%
\pgfsetstrokecolor{currentstroke}%
\pgfsetdash{}{0pt}%
\pgfpathmoveto{\pgfqpoint{4.237131in}{1.168857in}}%
\pgfpathlineto{\pgfqpoint{4.251689in}{1.173622in}}%
\pgfpathlineto{\pgfqpoint{4.266259in}{1.178559in}}%
\pgfpathlineto{\pgfqpoint{4.280842in}{1.183668in}}%
\pgfpathlineto{\pgfqpoint{4.295438in}{1.188949in}}%
\pgfpathlineto{\pgfqpoint{4.303763in}{1.206281in}}%
\pgfpathlineto{\pgfqpoint{4.312085in}{1.223801in}}%
\pgfpathlineto{\pgfqpoint{4.320405in}{1.241500in}}%
\pgfpathlineto{\pgfqpoint{4.328722in}{1.259370in}}%
\pgfpathlineto{\pgfqpoint{4.314121in}{1.253269in}}%
\pgfpathlineto{\pgfqpoint{4.299533in}{1.247341in}}%
\pgfpathlineto{\pgfqpoint{4.284959in}{1.241586in}}%
\pgfpathlineto{\pgfqpoint{4.270399in}{1.236004in}}%
\pgfpathlineto{\pgfqpoint{4.262087in}{1.218941in}}%
\pgfpathlineto{\pgfqpoint{4.253772in}{1.202057in}}%
\pgfpathlineto{\pgfqpoint{4.245453in}{1.185360in}}%
\pgfpathlineto{\pgfqpoint{4.237131in}{1.168857in}}%
\pgfpathclose%
\pgfusepath{fill}%
\end{pgfscope}%
\begin{pgfscope}%
\pgfpathrectangle{\pgfqpoint{1.150000in}{0.150000in}}{\pgfqpoint{5.700000in}{5.700000in}}%
\pgfusepath{clip}%
\pgfsetbuttcap%
\pgfsetroundjoin%
\definecolor{currentfill}{rgb}{0.208623,0.367752,0.552675}%
\pgfsetfillcolor{currentfill}%
\pgfsetfillopacity{0.800000}%
\pgfsetlinewidth{0.000000pt}%
\definecolor{currentstroke}{rgb}{0.000000,0.000000,0.000000}%
\pgfsetstrokecolor{currentstroke}%
\pgfsetdash{}{0pt}%
\pgfpathmoveto{\pgfqpoint{4.520139in}{1.600657in}}%
\pgfpathlineto{\pgfqpoint{4.534855in}{1.610622in}}%
\pgfpathlineto{\pgfqpoint{4.549587in}{1.620764in}}%
\pgfpathlineto{\pgfqpoint{4.564336in}{1.631084in}}%
\pgfpathlineto{\pgfqpoint{4.579103in}{1.641582in}}%
\pgfpathlineto{\pgfqpoint{4.587402in}{1.662624in}}%
\pgfpathlineto{\pgfqpoint{4.595699in}{1.683695in}}%
\pgfpathlineto{\pgfqpoint{4.603994in}{1.704790in}}%
\pgfpathlineto{\pgfqpoint{4.612287in}{1.725903in}}%
\pgfpathlineto{\pgfqpoint{4.597506in}{1.714729in}}%
\pgfpathlineto{\pgfqpoint{4.582742in}{1.703735in}}%
\pgfpathlineto{\pgfqpoint{4.567996in}{1.692919in}}%
\pgfpathlineto{\pgfqpoint{4.553266in}{1.682283in}}%
\pgfpathlineto{\pgfqpoint{4.544988in}{1.661833in}}%
\pgfpathlineto{\pgfqpoint{4.536707in}{1.641407in}}%
\pgfpathlineto{\pgfqpoint{4.528424in}{1.621013in}}%
\pgfpathlineto{\pgfqpoint{4.520139in}{1.600657in}}%
\pgfpathclose%
\pgfusepath{fill}%
\end{pgfscope}%
\begin{pgfscope}%
\pgfpathrectangle{\pgfqpoint{1.150000in}{0.150000in}}{\pgfqpoint{5.700000in}{5.700000in}}%
\pgfusepath{clip}%
\pgfsetbuttcap%
\pgfsetroundjoin%
\definecolor{currentfill}{rgb}{0.253935,0.265254,0.529983}%
\pgfsetfillcolor{currentfill}%
\pgfsetfillopacity{0.800000}%
\pgfsetlinewidth{0.000000pt}%
\definecolor{currentstroke}{rgb}{0.000000,0.000000,0.000000}%
\pgfsetstrokecolor{currentstroke}%
\pgfsetdash{}{0pt}%
\pgfpathmoveto{\pgfqpoint{4.361962in}{1.332411in}}%
\pgfpathlineto{\pgfqpoint{4.376584in}{1.339476in}}%
\pgfpathlineto{\pgfqpoint{4.391220in}{1.346715in}}%
\pgfpathlineto{\pgfqpoint{4.405871in}{1.354129in}}%
\pgfpathlineto{\pgfqpoint{4.420537in}{1.361717in}}%
\pgfpathlineto{\pgfqpoint{4.428850in}{1.381103in}}%
\pgfpathlineto{\pgfqpoint{4.437160in}{1.400609in}}%
\pgfpathlineto{\pgfqpoint{4.445468in}{1.420225in}}%
\pgfpathlineto{\pgfqpoint{4.453773in}{1.439947in}}%
\pgfpathlineto{\pgfqpoint{4.439097in}{1.431596in}}%
\pgfpathlineto{\pgfqpoint{4.424437in}{1.423420in}}%
\pgfpathlineto{\pgfqpoint{4.409791in}{1.415419in}}%
\pgfpathlineto{\pgfqpoint{4.395161in}{1.407594in}}%
\pgfpathlineto{\pgfqpoint{4.386865in}{1.388622in}}%
\pgfpathlineto{\pgfqpoint{4.378566in}{1.369764in}}%
\pgfpathlineto{\pgfqpoint{4.370265in}{1.351024in}}%
\pgfpathlineto{\pgfqpoint{4.361962in}{1.332411in}}%
\pgfpathclose%
\pgfusepath{fill}%
\end{pgfscope}%
\begin{pgfscope}%
\pgfpathrectangle{\pgfqpoint{1.150000in}{0.150000in}}{\pgfqpoint{5.700000in}{5.700000in}}%
\pgfusepath{clip}%
\pgfsetbuttcap%
\pgfsetroundjoin%
\definecolor{currentfill}{rgb}{0.487026,0.823929,0.312321}%
\pgfsetfillcolor{currentfill}%
\pgfsetfillopacity{0.800000}%
\pgfsetlinewidth{0.000000pt}%
\definecolor{currentstroke}{rgb}{0.000000,0.000000,0.000000}%
\pgfsetstrokecolor{currentstroke}%
\pgfsetdash{}{0pt}%
\pgfpathmoveto{\pgfqpoint{5.345328in}{3.141690in}}%
\pgfpathlineto{\pgfqpoint{5.360711in}{3.162290in}}%
\pgfpathlineto{\pgfqpoint{5.376120in}{3.183089in}}%
\pgfpathlineto{\pgfqpoint{5.391554in}{3.204086in}}%
\pgfpathlineto{\pgfqpoint{5.399573in}{3.218464in}}%
\pgfpathlineto{\pgfqpoint{5.407582in}{3.232592in}}%
\pgfpathlineto{\pgfqpoint{5.415580in}{3.246470in}}%
\pgfpathlineto{\pgfqpoint{5.423567in}{3.260097in}}%
\pgfpathlineto{\pgfqpoint{5.408125in}{3.238970in}}%
\pgfpathlineto{\pgfqpoint{5.392709in}{3.218041in}}%
\pgfpathlineto{\pgfqpoint{5.377318in}{3.197311in}}%
\pgfpathlineto{\pgfqpoint{5.369336in}{3.183771in}}%
\pgfpathlineto{\pgfqpoint{5.361343in}{3.169987in}}%
\pgfpathlineto{\pgfqpoint{5.353341in}{3.155959in}}%
\pgfpathlineto{\pgfqpoint{5.345328in}{3.141690in}}%
\pgfpathclose%
\pgfusepath{fill}%
\end{pgfscope}%
\begin{pgfscope}%
\pgfpathrectangle{\pgfqpoint{1.150000in}{0.150000in}}{\pgfqpoint{5.700000in}{5.700000in}}%
\pgfusepath{clip}%
\pgfsetbuttcap%
\pgfsetroundjoin%
\definecolor{currentfill}{rgb}{0.121831,0.589055,0.545623}%
\pgfsetfillcolor{currentfill}%
\pgfsetfillopacity{0.800000}%
\pgfsetlinewidth{0.000000pt}%
\definecolor{currentstroke}{rgb}{0.000000,0.000000,0.000000}%
\pgfsetstrokecolor{currentstroke}%
\pgfsetdash{}{0pt}%
\pgfpathmoveto{\pgfqpoint{4.870153in}{2.283233in}}%
\pgfpathlineto{\pgfqpoint{4.885130in}{2.298714in}}%
\pgfpathlineto{\pgfqpoint{4.900127in}{2.314383in}}%
\pgfpathlineto{\pgfqpoint{4.915146in}{2.330240in}}%
\pgfpathlineto{\pgfqpoint{4.930187in}{2.346284in}}%
\pgfpathlineto{\pgfqpoint{4.938432in}{2.366982in}}%
\pgfpathlineto{\pgfqpoint{4.946673in}{2.387548in}}%
\pgfpathlineto{\pgfqpoint{4.954909in}{2.407977in}}%
\pgfpathlineto{\pgfqpoint{4.963140in}{2.428266in}}%
\pgfpathlineto{\pgfqpoint{4.948082in}{2.411768in}}%
\pgfpathlineto{\pgfqpoint{4.933046in}{2.395459in}}%
\pgfpathlineto{\pgfqpoint{4.918031in}{2.379339in}}%
\pgfpathlineto{\pgfqpoint{4.903038in}{2.363407in}}%
\pgfpathlineto{\pgfqpoint{4.894824in}{2.343557in}}%
\pgfpathlineto{\pgfqpoint{4.886605in}{2.323576in}}%
\pgfpathlineto{\pgfqpoint{4.878381in}{2.303466in}}%
\pgfpathlineto{\pgfqpoint{4.870153in}{2.283233in}}%
\pgfpathclose%
\pgfusepath{fill}%
\end{pgfscope}%
\begin{pgfscope}%
\pgfpathrectangle{\pgfqpoint{1.150000in}{0.150000in}}{\pgfqpoint{5.700000in}{5.700000in}}%
\pgfusepath{clip}%
\pgfsetbuttcap%
\pgfsetroundjoin%
\definecolor{currentfill}{rgb}{0.165117,0.467423,0.558141}%
\pgfsetfillcolor{currentfill}%
\pgfsetfillopacity{0.800000}%
\pgfsetlinewidth{0.000000pt}%
\definecolor{currentstroke}{rgb}{0.000000,0.000000,0.000000}%
\pgfsetstrokecolor{currentstroke}%
\pgfsetdash{}{0pt}%
\pgfpathmoveto{\pgfqpoint{4.678539in}{1.894720in}}%
\pgfpathlineto{\pgfqpoint{4.693369in}{1.907331in}}%
\pgfpathlineto{\pgfqpoint{4.708219in}{1.920125in}}%
\pgfpathlineto{\pgfqpoint{4.723087in}{1.933100in}}%
\pgfpathlineto{\pgfqpoint{4.737974in}{1.946258in}}%
\pgfpathlineto{\pgfqpoint{4.746260in}{1.967858in}}%
\pgfpathlineto{\pgfqpoint{4.754544in}{1.989410in}}%
\pgfpathlineto{\pgfqpoint{4.762824in}{2.010908in}}%
\pgfpathlineto{\pgfqpoint{4.771101in}{2.032347in}}%
\pgfpathlineto{\pgfqpoint{4.756197in}{2.018605in}}%
\pgfpathlineto{\pgfqpoint{4.741311in}{2.005047in}}%
\pgfpathlineto{\pgfqpoint{4.726445in}{1.991672in}}%
\pgfpathlineto{\pgfqpoint{4.711599in}{1.978480in}}%
\pgfpathlineto{\pgfqpoint{4.703338in}{1.957611in}}%
\pgfpathlineto{\pgfqpoint{4.695075in}{1.936691in}}%
\pgfpathlineto{\pgfqpoint{4.686808in}{1.915725in}}%
\pgfpathlineto{\pgfqpoint{4.678539in}{1.894720in}}%
\pgfpathclose%
\pgfusepath{fill}%
\end{pgfscope}%
\begin{pgfscope}%
\pgfpathrectangle{\pgfqpoint{1.150000in}{0.150000in}}{\pgfqpoint{5.700000in}{5.700000in}}%
\pgfusepath{clip}%
\pgfsetbuttcap%
\pgfsetroundjoin%
\definecolor{currentfill}{rgb}{0.220057,0.343307,0.549413}%
\pgfsetfillcolor{currentfill}%
\pgfsetfillopacity{0.800000}%
\pgfsetlinewidth{0.000000pt}%
\definecolor{currentstroke}{rgb}{0.000000,0.000000,0.000000}%
\pgfsetstrokecolor{currentstroke}%
\pgfsetdash{}{0pt}%
\pgfpathmoveto{\pgfqpoint{4.486974in}{1.519740in}}%
\pgfpathlineto{\pgfqpoint{4.501677in}{1.529001in}}%
\pgfpathlineto{\pgfqpoint{4.516397in}{1.538438in}}%
\pgfpathlineto{\pgfqpoint{4.531132in}{1.548053in}}%
\pgfpathlineto{\pgfqpoint{4.545885in}{1.557845in}}%
\pgfpathlineto{\pgfqpoint{4.554192in}{1.578702in}}%
\pgfpathlineto{\pgfqpoint{4.562498in}{1.599615in}}%
\pgfpathlineto{\pgfqpoint{4.570801in}{1.620577in}}%
\pgfpathlineto{\pgfqpoint{4.579103in}{1.641582in}}%
\pgfpathlineto{\pgfqpoint{4.564336in}{1.631084in}}%
\pgfpathlineto{\pgfqpoint{4.549587in}{1.620764in}}%
\pgfpathlineto{\pgfqpoint{4.534855in}{1.610622in}}%
\pgfpathlineto{\pgfqpoint{4.520139in}{1.600657in}}%
\pgfpathlineto{\pgfqpoint{4.511851in}{1.580346in}}%
\pgfpathlineto{\pgfqpoint{4.503561in}{1.560084in}}%
\pgfpathlineto{\pgfqpoint{4.495268in}{1.539880in}}%
\pgfpathlineto{\pgfqpoint{4.486974in}{1.519740in}}%
\pgfpathclose%
\pgfusepath{fill}%
\end{pgfscope}%
\begin{pgfscope}%
\pgfpathrectangle{\pgfqpoint{1.150000in}{0.150000in}}{\pgfqpoint{5.700000in}{5.700000in}}%
\pgfusepath{clip}%
\pgfsetbuttcap%
\pgfsetroundjoin%
\definecolor{currentfill}{rgb}{0.157851,0.683765,0.501686}%
\pgfsetfillcolor{currentfill}%
\pgfsetfillopacity{0.800000}%
\pgfsetlinewidth{0.000000pt}%
\definecolor{currentstroke}{rgb}{0.000000,0.000000,0.000000}%
\pgfsetstrokecolor{currentstroke}%
\pgfsetdash{}{0pt}%
\pgfpathmoveto{\pgfqpoint{5.028795in}{2.585061in}}%
\pgfpathlineto{\pgfqpoint{5.043909in}{2.602552in}}%
\pgfpathlineto{\pgfqpoint{5.059045in}{2.620233in}}%
\pgfpathlineto{\pgfqpoint{5.074204in}{2.638107in}}%
\pgfpathlineto{\pgfqpoint{5.089386in}{2.656172in}}%
\pgfpathlineto{\pgfqpoint{5.097583in}{2.675393in}}%
\pgfpathlineto{\pgfqpoint{5.105773in}{2.694429in}}%
\pgfpathlineto{\pgfqpoint{5.113957in}{2.713277in}}%
\pgfpathlineto{\pgfqpoint{5.122134in}{2.731934in}}%
\pgfpathlineto{\pgfqpoint{5.106936in}{2.713518in}}%
\pgfpathlineto{\pgfqpoint{5.091761in}{2.695295in}}%
\pgfpathlineto{\pgfqpoint{5.076610in}{2.677264in}}%
\pgfpathlineto{\pgfqpoint{5.061481in}{2.659425in}}%
\pgfpathlineto{\pgfqpoint{5.053319in}{2.641104in}}%
\pgfpathlineto{\pgfqpoint{5.045151in}{2.622601in}}%
\pgfpathlineto{\pgfqpoint{5.036976in}{2.603919in}}%
\pgfpathlineto{\pgfqpoint{5.028795in}{2.585061in}}%
\pgfpathclose%
\pgfusepath{fill}%
\end{pgfscope}%
\begin{pgfscope}%
\pgfpathrectangle{\pgfqpoint{1.150000in}{0.150000in}}{\pgfqpoint{5.700000in}{5.700000in}}%
\pgfusepath{clip}%
\pgfsetbuttcap%
\pgfsetroundjoin%
\definecolor{currentfill}{rgb}{0.311925,0.767822,0.415586}%
\pgfsetfillcolor{currentfill}%
\pgfsetfillopacity{0.800000}%
\pgfsetlinewidth{0.000000pt}%
\definecolor{currentstroke}{rgb}{0.000000,0.000000,0.000000}%
\pgfsetstrokecolor{currentstroke}%
\pgfsetdash{}{0pt}%
\pgfpathmoveto{\pgfqpoint{5.187291in}{2.874001in}}%
\pgfpathlineto{\pgfqpoint{5.202541in}{2.893203in}}%
\pgfpathlineto{\pgfqpoint{5.217816in}{2.912600in}}%
\pgfpathlineto{\pgfqpoint{5.233115in}{2.932193in}}%
\pgfpathlineto{\pgfqpoint{5.248439in}{2.951981in}}%
\pgfpathlineto{\pgfqpoint{5.256562in}{2.969059in}}%
\pgfpathlineto{\pgfqpoint{5.264677in}{2.985912in}}%
\pgfpathlineto{\pgfqpoint{5.272783in}{3.002538in}}%
\pgfpathlineto{\pgfqpoint{5.280880in}{3.018936in}}%
\pgfpathlineto{\pgfqpoint{5.265544in}{2.998905in}}%
\pgfpathlineto{\pgfqpoint{5.250232in}{2.979071in}}%
\pgfpathlineto{\pgfqpoint{5.234945in}{2.959432in}}%
\pgfpathlineto{\pgfqpoint{5.219682in}{2.939988in}}%
\pgfpathlineto{\pgfqpoint{5.211597in}{2.923818in}}%
\pgfpathlineto{\pgfqpoint{5.203503in}{2.907429in}}%
\pgfpathlineto{\pgfqpoint{5.195401in}{2.890823in}}%
\pgfpathlineto{\pgfqpoint{5.187291in}{2.874001in}}%
\pgfpathclose%
\pgfusepath{fill}%
\end{pgfscope}%
\begin{pgfscope}%
\pgfpathrectangle{\pgfqpoint{1.150000in}{0.150000in}}{\pgfqpoint{5.700000in}{5.700000in}}%
\pgfusepath{clip}%
\pgfsetbuttcap%
\pgfsetroundjoin%
\definecolor{currentfill}{rgb}{0.262138,0.242286,0.520837}%
\pgfsetfillcolor{currentfill}%
\pgfsetfillopacity{0.800000}%
\pgfsetlinewidth{0.000000pt}%
\definecolor{currentstroke}{rgb}{0.000000,0.000000,0.000000}%
\pgfsetstrokecolor{currentstroke}%
\pgfsetdash{}{0pt}%
\pgfpathmoveto{\pgfqpoint{4.328722in}{1.259370in}}%
\pgfpathlineto{\pgfqpoint{4.343336in}{1.265644in}}%
\pgfpathlineto{\pgfqpoint{4.357965in}{1.272092in}}%
\pgfpathlineto{\pgfqpoint{4.372608in}{1.278713in}}%
\pgfpathlineto{\pgfqpoint{4.387266in}{1.285507in}}%
\pgfpathlineto{\pgfqpoint{4.395587in}{1.304344in}}%
\pgfpathlineto{\pgfqpoint{4.403906in}{1.323330in}}%
\pgfpathlineto{\pgfqpoint{4.412223in}{1.342457in}}%
\pgfpathlineto{\pgfqpoint{4.420537in}{1.361717in}}%
\pgfpathlineto{\pgfqpoint{4.405871in}{1.354129in}}%
\pgfpathlineto{\pgfqpoint{4.391220in}{1.346715in}}%
\pgfpathlineto{\pgfqpoint{4.376584in}{1.339476in}}%
\pgfpathlineto{\pgfqpoint{4.361962in}{1.332411in}}%
\pgfpathlineto{\pgfqpoint{4.353656in}{1.313932in}}%
\pgfpathlineto{\pgfqpoint{4.345347in}{1.295594in}}%
\pgfpathlineto{\pgfqpoint{4.337036in}{1.277404in}}%
\pgfpathlineto{\pgfqpoint{4.328722in}{1.259370in}}%
\pgfpathclose%
\pgfusepath{fill}%
\end{pgfscope}%
\begin{pgfscope}%
\pgfpathrectangle{\pgfqpoint{1.150000in}{0.150000in}}{\pgfqpoint{5.700000in}{5.700000in}}%
\pgfusepath{clip}%
\pgfsetbuttcap%
\pgfsetroundjoin%
\definecolor{currentfill}{rgb}{0.127568,0.566949,0.550556}%
\pgfsetfillcolor{currentfill}%
\pgfsetfillopacity{0.800000}%
\pgfsetlinewidth{0.000000pt}%
\definecolor{currentstroke}{rgb}{0.000000,0.000000,0.000000}%
\pgfsetstrokecolor{currentstroke}%
\pgfsetdash{}{0pt}%
\pgfpathmoveto{\pgfqpoint{4.837197in}{2.201143in}}%
\pgfpathlineto{\pgfqpoint{4.852157in}{2.216140in}}%
\pgfpathlineto{\pgfqpoint{4.867137in}{2.231323in}}%
\pgfpathlineto{\pgfqpoint{4.882138in}{2.246692in}}%
\pgfpathlineto{\pgfqpoint{4.897161in}{2.262249in}}%
\pgfpathlineto{\pgfqpoint{4.905424in}{2.283435in}}%
\pgfpathlineto{\pgfqpoint{4.913682in}{2.304506in}}%
\pgfpathlineto{\pgfqpoint{4.921937in}{2.325457in}}%
\pgfpathlineto{\pgfqpoint{4.930187in}{2.346284in}}%
\pgfpathlineto{\pgfqpoint{4.915146in}{2.330240in}}%
\pgfpathlineto{\pgfqpoint{4.900127in}{2.314383in}}%
\pgfpathlineto{\pgfqpoint{4.885130in}{2.298714in}}%
\pgfpathlineto{\pgfqpoint{4.870153in}{2.283233in}}%
\pgfpathlineto{\pgfqpoint{4.861920in}{2.262880in}}%
\pgfpathlineto{\pgfqpoint{4.853683in}{2.242411in}}%
\pgfpathlineto{\pgfqpoint{4.845442in}{2.221831in}}%
\pgfpathlineto{\pgfqpoint{4.837197in}{2.201143in}}%
\pgfpathclose%
\pgfusepath{fill}%
\end{pgfscope}%
\begin{pgfscope}%
\pgfpathrectangle{\pgfqpoint{1.150000in}{0.150000in}}{\pgfqpoint{5.700000in}{5.700000in}}%
\pgfusepath{clip}%
\pgfsetbuttcap%
\pgfsetroundjoin%
\definecolor{currentfill}{rgb}{0.175841,0.441290,0.557685}%
\pgfsetfillcolor{currentfill}%
\pgfsetfillopacity{0.800000}%
\pgfsetlinewidth{0.000000pt}%
\definecolor{currentstroke}{rgb}{0.000000,0.000000,0.000000}%
\pgfsetstrokecolor{currentstroke}%
\pgfsetdash{}{0pt}%
\pgfpathmoveto{\pgfqpoint{4.645433in}{1.810408in}}%
\pgfpathlineto{\pgfqpoint{4.660248in}{1.822406in}}%
\pgfpathlineto{\pgfqpoint{4.675081in}{1.834585in}}%
\pgfpathlineto{\pgfqpoint{4.689932in}{1.846946in}}%
\pgfpathlineto{\pgfqpoint{4.704802in}{1.859487in}}%
\pgfpathlineto{\pgfqpoint{4.713099in}{1.881225in}}%
\pgfpathlineto{\pgfqpoint{4.721393in}{1.902936in}}%
\pgfpathlineto{\pgfqpoint{4.729685in}{1.924615in}}%
\pgfpathlineto{\pgfqpoint{4.737974in}{1.946258in}}%
\pgfpathlineto{\pgfqpoint{4.723087in}{1.933100in}}%
\pgfpathlineto{\pgfqpoint{4.708219in}{1.920125in}}%
\pgfpathlineto{\pgfqpoint{4.693369in}{1.907331in}}%
\pgfpathlineto{\pgfqpoint{4.678539in}{1.894720in}}%
\pgfpathlineto{\pgfqpoint{4.670267in}{1.873680in}}%
\pgfpathlineto{\pgfqpoint{4.661991in}{1.852611in}}%
\pgfpathlineto{\pgfqpoint{4.653714in}{1.831519in}}%
\pgfpathlineto{\pgfqpoint{4.645433in}{1.810408in}}%
\pgfpathclose%
\pgfusepath{fill}%
\end{pgfscope}%
\begin{pgfscope}%
\pgfpathrectangle{\pgfqpoint{1.150000in}{0.150000in}}{\pgfqpoint{5.700000in}{5.700000in}}%
\pgfusepath{clip}%
\pgfsetbuttcap%
\pgfsetroundjoin%
\definecolor{currentfill}{rgb}{0.458674,0.816363,0.329727}%
\pgfsetfillcolor{currentfill}%
\pgfsetfillopacity{0.800000}%
\pgfsetlinewidth{0.000000pt}%
\definecolor{currentstroke}{rgb}{0.000000,0.000000,0.000000}%
\pgfsetstrokecolor{currentstroke}%
\pgfsetdash{}{0pt}%
\pgfpathmoveto{\pgfqpoint{5.313180in}{3.082206in}}%
\pgfpathlineto{\pgfqpoint{5.328553in}{3.102639in}}%
\pgfpathlineto{\pgfqpoint{5.343952in}{3.123269in}}%
\pgfpathlineto{\pgfqpoint{5.359376in}{3.144098in}}%
\pgfpathlineto{\pgfqpoint{5.367435in}{3.159464in}}%
\pgfpathlineto{\pgfqpoint{5.375485in}{3.174585in}}%
\pgfpathlineto{\pgfqpoint{5.383524in}{3.189459in}}%
\pgfpathlineto{\pgfqpoint{5.391554in}{3.204086in}}%
\pgfpathlineto{\pgfqpoint{5.376120in}{3.183089in}}%
\pgfpathlineto{\pgfqpoint{5.360711in}{3.162290in}}%
\pgfpathlineto{\pgfqpoint{5.345328in}{3.141690in}}%
\pgfpathlineto{\pgfqpoint{5.337306in}{3.127178in}}%
\pgfpathlineto{\pgfqpoint{5.329273in}{3.112426in}}%
\pgfpathlineto{\pgfqpoint{5.321231in}{3.097435in}}%
\pgfpathlineto{\pgfqpoint{5.313180in}{3.082206in}}%
\pgfpathclose%
\pgfusepath{fill}%
\end{pgfscope}%
\begin{pgfscope}%
\pgfpathrectangle{\pgfqpoint{1.150000in}{0.150000in}}{\pgfqpoint{5.700000in}{5.700000in}}%
\pgfusepath{clip}%
\pgfsetbuttcap%
\pgfsetroundjoin%
\definecolor{currentfill}{rgb}{0.231674,0.318106,0.544834}%
\pgfsetfillcolor{currentfill}%
\pgfsetfillopacity{0.800000}%
\pgfsetlinewidth{0.000000pt}%
\definecolor{currentstroke}{rgb}{0.000000,0.000000,0.000000}%
\pgfsetstrokecolor{currentstroke}%
\pgfsetdash{}{0pt}%
\pgfpathmoveto{\pgfqpoint{4.453773in}{1.439947in}}%
\pgfpathlineto{\pgfqpoint{4.468465in}{1.448474in}}%
\pgfpathlineto{\pgfqpoint{4.483172in}{1.457177in}}%
\pgfpathlineto{\pgfqpoint{4.497895in}{1.466055in}}%
\pgfpathlineto{\pgfqpoint{4.512634in}{1.475110in}}%
\pgfpathlineto{\pgfqpoint{4.520950in}{1.495676in}}%
\pgfpathlineto{\pgfqpoint{4.529263in}{1.516325in}}%
\pgfpathlineto{\pgfqpoint{4.537575in}{1.537051in}}%
\pgfpathlineto{\pgfqpoint{4.545885in}{1.557845in}}%
\pgfpathlineto{\pgfqpoint{4.531132in}{1.548053in}}%
\pgfpathlineto{\pgfqpoint{4.516397in}{1.538438in}}%
\pgfpathlineto{\pgfqpoint{4.501677in}{1.529001in}}%
\pgfpathlineto{\pgfqpoint{4.486974in}{1.519740in}}%
\pgfpathlineto{\pgfqpoint{4.478677in}{1.499669in}}%
\pgfpathlineto{\pgfqpoint{4.470378in}{1.479676in}}%
\pgfpathlineto{\pgfqpoint{4.462077in}{1.459766in}}%
\pgfpathlineto{\pgfqpoint{4.453773in}{1.439947in}}%
\pgfpathclose%
\pgfusepath{fill}%
\end{pgfscope}%
\begin{pgfscope}%
\pgfpathrectangle{\pgfqpoint{1.150000in}{0.150000in}}{\pgfqpoint{5.700000in}{5.700000in}}%
\pgfusepath{clip}%
\pgfsetbuttcap%
\pgfsetroundjoin%
\definecolor{currentfill}{rgb}{0.140210,0.665859,0.513427}%
\pgfsetfillcolor{currentfill}%
\pgfsetfillopacity{0.800000}%
\pgfsetlinewidth{0.000000pt}%
\definecolor{currentstroke}{rgb}{0.000000,0.000000,0.000000}%
\pgfsetstrokecolor{currentstroke}%
\pgfsetdash{}{0pt}%
\pgfpathmoveto{\pgfqpoint{4.996012in}{2.507937in}}%
\pgfpathlineto{\pgfqpoint{5.011109in}{2.525043in}}%
\pgfpathlineto{\pgfqpoint{5.026229in}{2.542340in}}%
\pgfpathlineto{\pgfqpoint{5.041371in}{2.559827in}}%
\pgfpathlineto{\pgfqpoint{5.056536in}{2.577506in}}%
\pgfpathlineto{\pgfqpoint{5.064757in}{2.597433in}}%
\pgfpathlineto{\pgfqpoint{5.072973in}{2.617189in}}%
\pgfpathlineto{\pgfqpoint{5.081182in}{2.636770in}}%
\pgfpathlineto{\pgfqpoint{5.089386in}{2.656172in}}%
\pgfpathlineto{\pgfqpoint{5.074204in}{2.638107in}}%
\pgfpathlineto{\pgfqpoint{5.059045in}{2.620233in}}%
\pgfpathlineto{\pgfqpoint{5.043909in}{2.602552in}}%
\pgfpathlineto{\pgfqpoint{5.028795in}{2.585061in}}%
\pgfpathlineto{\pgfqpoint{5.020608in}{2.566031in}}%
\pgfpathlineto{\pgfqpoint{5.012415in}{2.546831in}}%
\pgfpathlineto{\pgfqpoint{5.004217in}{2.527465in}}%
\pgfpathlineto{\pgfqpoint{4.996012in}{2.507937in}}%
\pgfpathclose%
\pgfusepath{fill}%
\end{pgfscope}%
\begin{pgfscope}%
\pgfpathrectangle{\pgfqpoint{1.150000in}{0.150000in}}{\pgfqpoint{5.700000in}{5.700000in}}%
\pgfusepath{clip}%
\pgfsetbuttcap%
\pgfsetroundjoin%
\definecolor{currentfill}{rgb}{0.135066,0.544853,0.554029}%
\pgfsetfillcolor{currentfill}%
\pgfsetfillopacity{0.800000}%
\pgfsetlinewidth{0.000000pt}%
\definecolor{currentstroke}{rgb}{0.000000,0.000000,0.000000}%
\pgfsetstrokecolor{currentstroke}%
\pgfsetdash{}{0pt}%
\pgfpathmoveto{\pgfqpoint{4.804178in}{2.117415in}}%
\pgfpathlineto{\pgfqpoint{4.819120in}{2.131893in}}%
\pgfpathlineto{\pgfqpoint{4.834082in}{2.146556in}}%
\pgfpathlineto{\pgfqpoint{4.849065in}{2.161405in}}%
\pgfpathlineto{\pgfqpoint{4.864069in}{2.176439in}}%
\pgfpathlineto{\pgfqpoint{4.872347in}{2.198042in}}%
\pgfpathlineto{\pgfqpoint{4.880622in}{2.219548in}}%
\pgfpathlineto{\pgfqpoint{4.888893in}{2.240951in}}%
\pgfpathlineto{\pgfqpoint{4.897161in}{2.262249in}}%
\pgfpathlineto{\pgfqpoint{4.882138in}{2.246692in}}%
\pgfpathlineto{\pgfqpoint{4.867137in}{2.231323in}}%
\pgfpathlineto{\pgfqpoint{4.852157in}{2.216140in}}%
\pgfpathlineto{\pgfqpoint{4.837197in}{2.201143in}}%
\pgfpathlineto{\pgfqpoint{4.828948in}{2.180354in}}%
\pgfpathlineto{\pgfqpoint{4.820695in}{2.159466in}}%
\pgfpathlineto{\pgfqpoint{4.812438in}{2.138485in}}%
\pgfpathlineto{\pgfqpoint{4.804178in}{2.117415in}}%
\pgfpathclose%
\pgfusepath{fill}%
\end{pgfscope}%
\begin{pgfscope}%
\pgfpathrectangle{\pgfqpoint{1.150000in}{0.150000in}}{\pgfqpoint{5.700000in}{5.700000in}}%
\pgfusepath{clip}%
\pgfsetbuttcap%
\pgfsetroundjoin%
\definecolor{currentfill}{rgb}{0.185556,0.418570,0.556753}%
\pgfsetfillcolor{currentfill}%
\pgfsetfillopacity{0.800000}%
\pgfsetlinewidth{0.000000pt}%
\definecolor{currentstroke}{rgb}{0.000000,0.000000,0.000000}%
\pgfsetstrokecolor{currentstroke}%
\pgfsetdash{}{0pt}%
\pgfpathmoveto{\pgfqpoint{4.612287in}{1.725903in}}%
\pgfpathlineto{\pgfqpoint{4.627086in}{1.737256in}}%
\pgfpathlineto{\pgfqpoint{4.641902in}{1.748789in}}%
\pgfpathlineto{\pgfqpoint{4.656737in}{1.760502in}}%
\pgfpathlineto{\pgfqpoint{4.671590in}{1.772395in}}%
\pgfpathlineto{\pgfqpoint{4.679896in}{1.794178in}}%
\pgfpathlineto{\pgfqpoint{4.688200in}{1.815958in}}%
\pgfpathlineto{\pgfqpoint{4.696502in}{1.837730in}}%
\pgfpathlineto{\pgfqpoint{4.704802in}{1.859487in}}%
\pgfpathlineto{\pgfqpoint{4.689932in}{1.846946in}}%
\pgfpathlineto{\pgfqpoint{4.675081in}{1.834585in}}%
\pgfpathlineto{\pgfqpoint{4.660248in}{1.822406in}}%
\pgfpathlineto{\pgfqpoint{4.645433in}{1.810408in}}%
\pgfpathlineto{\pgfqpoint{4.637150in}{1.789285in}}%
\pgfpathlineto{\pgfqpoint{4.628865in}{1.768156in}}%
\pgfpathlineto{\pgfqpoint{4.620577in}{1.747027in}}%
\pgfpathlineto{\pgfqpoint{4.612287in}{1.725903in}}%
\pgfpathclose%
\pgfusepath{fill}%
\end{pgfscope}%
\begin{pgfscope}%
\pgfpathrectangle{\pgfqpoint{1.150000in}{0.150000in}}{\pgfqpoint{5.700000in}{5.700000in}}%
\pgfusepath{clip}%
\pgfsetbuttcap%
\pgfsetroundjoin%
\definecolor{currentfill}{rgb}{0.269308,0.218818,0.509577}%
\pgfsetfillcolor{currentfill}%
\pgfsetfillopacity{0.800000}%
\pgfsetlinewidth{0.000000pt}%
\definecolor{currentstroke}{rgb}{0.000000,0.000000,0.000000}%
\pgfsetstrokecolor{currentstroke}%
\pgfsetdash{}{0pt}%
\pgfpathmoveto{\pgfqpoint{4.295438in}{1.188949in}}%
\pgfpathlineto{\pgfqpoint{4.310047in}{1.194402in}}%
\pgfpathlineto{\pgfqpoint{4.324670in}{1.200027in}}%
\pgfpathlineto{\pgfqpoint{4.339306in}{1.205825in}}%
\pgfpathlineto{\pgfqpoint{4.353956in}{1.211794in}}%
\pgfpathlineto{\pgfqpoint{4.362287in}{1.229961in}}%
\pgfpathlineto{\pgfqpoint{4.370616in}{1.248308in}}%
\pgfpathlineto{\pgfqpoint{4.378942in}{1.266825in}}%
\pgfpathlineto{\pgfqpoint{4.387266in}{1.285507in}}%
\pgfpathlineto{\pgfqpoint{4.372608in}{1.278713in}}%
\pgfpathlineto{\pgfqpoint{4.357965in}{1.272092in}}%
\pgfpathlineto{\pgfqpoint{4.343336in}{1.265644in}}%
\pgfpathlineto{\pgfqpoint{4.328722in}{1.259370in}}%
\pgfpathlineto{\pgfqpoint{4.320405in}{1.241500in}}%
\pgfpathlineto{\pgfqpoint{4.312085in}{1.223801in}}%
\pgfpathlineto{\pgfqpoint{4.303763in}{1.206281in}}%
\pgfpathlineto{\pgfqpoint{4.295438in}{1.188949in}}%
\pgfpathclose%
\pgfusepath{fill}%
\end{pgfscope}%
\begin{pgfscope}%
\pgfpathrectangle{\pgfqpoint{1.150000in}{0.150000in}}{\pgfqpoint{5.700000in}{5.700000in}}%
\pgfusepath{clip}%
\pgfsetbuttcap%
\pgfsetroundjoin%
\definecolor{currentfill}{rgb}{0.274149,0.751988,0.436601}%
\pgfsetfillcolor{currentfill}%
\pgfsetfillopacity{0.800000}%
\pgfsetlinewidth{0.000000pt}%
\definecolor{currentstroke}{rgb}{0.000000,0.000000,0.000000}%
\pgfsetstrokecolor{currentstroke}%
\pgfsetdash{}{0pt}%
\pgfpathmoveto{\pgfqpoint{5.154772in}{2.804601in}}%
\pgfpathlineto{\pgfqpoint{5.170009in}{2.823525in}}%
\pgfpathlineto{\pgfqpoint{5.185269in}{2.842643in}}%
\pgfpathlineto{\pgfqpoint{5.200554in}{2.861956in}}%
\pgfpathlineto{\pgfqpoint{5.215864in}{2.881464in}}%
\pgfpathlineto{\pgfqpoint{5.224019in}{2.899419in}}%
\pgfpathlineto{\pgfqpoint{5.232167in}{2.917159in}}%
\pgfpathlineto{\pgfqpoint{5.240307in}{2.934680in}}%
\pgfpathlineto{\pgfqpoint{5.248439in}{2.951981in}}%
\pgfpathlineto{\pgfqpoint{5.233115in}{2.932193in}}%
\pgfpathlineto{\pgfqpoint{5.217816in}{2.912600in}}%
\pgfpathlineto{\pgfqpoint{5.202541in}{2.893203in}}%
\pgfpathlineto{\pgfqpoint{5.187291in}{2.874001in}}%
\pgfpathlineto{\pgfqpoint{5.179173in}{2.856965in}}%
\pgfpathlineto{\pgfqpoint{5.171047in}{2.839719in}}%
\pgfpathlineto{\pgfqpoint{5.162913in}{2.822263in}}%
\pgfpathlineto{\pgfqpoint{5.154772in}{2.804601in}}%
\pgfpathclose%
\pgfusepath{fill}%
\end{pgfscope}%
\begin{pgfscope}%
\pgfpathrectangle{\pgfqpoint{1.150000in}{0.150000in}}{\pgfqpoint{5.700000in}{5.700000in}}%
\pgfusepath{clip}%
\pgfsetbuttcap%
\pgfsetroundjoin%
\definecolor{currentfill}{rgb}{0.243113,0.292092,0.538516}%
\pgfsetfillcolor{currentfill}%
\pgfsetfillopacity{0.800000}%
\pgfsetlinewidth{0.000000pt}%
\definecolor{currentstroke}{rgb}{0.000000,0.000000,0.000000}%
\pgfsetstrokecolor{currentstroke}%
\pgfsetdash{}{0pt}%
\pgfpathmoveto{\pgfqpoint{4.420537in}{1.361717in}}%
\pgfpathlineto{\pgfqpoint{4.435218in}{1.369479in}}%
\pgfpathlineto{\pgfqpoint{4.449914in}{1.377416in}}%
\pgfpathlineto{\pgfqpoint{4.464626in}{1.385528in}}%
\pgfpathlineto{\pgfqpoint{4.479353in}{1.393814in}}%
\pgfpathlineto{\pgfqpoint{4.487676in}{1.413979in}}%
\pgfpathlineto{\pgfqpoint{4.495997in}{1.434254in}}%
\pgfpathlineto{\pgfqpoint{4.504317in}{1.454633in}}%
\pgfpathlineto{\pgfqpoint{4.512634in}{1.475110in}}%
\pgfpathlineto{\pgfqpoint{4.497895in}{1.466055in}}%
\pgfpathlineto{\pgfqpoint{4.483172in}{1.457177in}}%
\pgfpathlineto{\pgfqpoint{4.468465in}{1.448474in}}%
\pgfpathlineto{\pgfqpoint{4.453773in}{1.439947in}}%
\pgfpathlineto{\pgfqpoint{4.445468in}{1.420225in}}%
\pgfpathlineto{\pgfqpoint{4.437160in}{1.400609in}}%
\pgfpathlineto{\pgfqpoint{4.428850in}{1.381103in}}%
\pgfpathlineto{\pgfqpoint{4.420537in}{1.361717in}}%
\pgfpathclose%
\pgfusepath{fill}%
\end{pgfscope}%
\begin{pgfscope}%
\pgfpathrectangle{\pgfqpoint{1.150000in}{0.150000in}}{\pgfqpoint{5.700000in}{5.700000in}}%
\pgfusepath{clip}%
\pgfsetbuttcap%
\pgfsetroundjoin%
\definecolor{currentfill}{rgb}{0.144759,0.519093,0.556572}%
\pgfsetfillcolor{currentfill}%
\pgfsetfillopacity{0.800000}%
\pgfsetlinewidth{0.000000pt}%
\definecolor{currentstroke}{rgb}{0.000000,0.000000,0.000000}%
\pgfsetstrokecolor{currentstroke}%
\pgfsetdash{}{0pt}%
\pgfpathmoveto{\pgfqpoint{4.771101in}{2.032347in}}%
\pgfpathlineto{\pgfqpoint{4.786026in}{2.046273in}}%
\pgfpathlineto{\pgfqpoint{4.800970in}{2.060383in}}%
\pgfpathlineto{\pgfqpoint{4.815934in}{2.074677in}}%
\pgfpathlineto{\pgfqpoint{4.830919in}{2.089156in}}%
\pgfpathlineto{\pgfqpoint{4.839211in}{2.111098in}}%
\pgfpathlineto{\pgfqpoint{4.847501in}{2.132962in}}%
\pgfpathlineto{\pgfqpoint{4.855786in}{2.154745in}}%
\pgfpathlineto{\pgfqpoint{4.864069in}{2.176439in}}%
\pgfpathlineto{\pgfqpoint{4.849065in}{2.161405in}}%
\pgfpathlineto{\pgfqpoint{4.834082in}{2.146556in}}%
\pgfpathlineto{\pgfqpoint{4.819120in}{2.131893in}}%
\pgfpathlineto{\pgfqpoint{4.804178in}{2.117415in}}%
\pgfpathlineto{\pgfqpoint{4.795914in}{2.096262in}}%
\pgfpathlineto{\pgfqpoint{4.787646in}{2.075029in}}%
\pgfpathlineto{\pgfqpoint{4.779375in}{2.053723in}}%
\pgfpathlineto{\pgfqpoint{4.771101in}{2.032347in}}%
\pgfpathclose%
\pgfusepath{fill}%
\end{pgfscope}%
\begin{pgfscope}%
\pgfpathrectangle{\pgfqpoint{1.150000in}{0.150000in}}{\pgfqpoint{5.700000in}{5.700000in}}%
\pgfusepath{clip}%
\pgfsetbuttcap%
\pgfsetroundjoin%
\definecolor{currentfill}{rgb}{0.195860,0.395433,0.555276}%
\pgfsetfillcolor{currentfill}%
\pgfsetfillopacity{0.800000}%
\pgfsetlinewidth{0.000000pt}%
\definecolor{currentstroke}{rgb}{0.000000,0.000000,0.000000}%
\pgfsetstrokecolor{currentstroke}%
\pgfsetdash{}{0pt}%
\pgfpathmoveto{\pgfqpoint{4.579103in}{1.641582in}}%
\pgfpathlineto{\pgfqpoint{4.593886in}{1.652259in}}%
\pgfpathlineto{\pgfqpoint{4.608687in}{1.663114in}}%
\pgfpathlineto{\pgfqpoint{4.623506in}{1.674148in}}%
\pgfpathlineto{\pgfqpoint{4.638342in}{1.685362in}}%
\pgfpathlineto{\pgfqpoint{4.646657in}{1.707093in}}%
\pgfpathlineto{\pgfqpoint{4.654970in}{1.728847in}}%
\pgfpathlineto{\pgfqpoint{4.663281in}{1.750616in}}%
\pgfpathlineto{\pgfqpoint{4.671590in}{1.772395in}}%
\pgfpathlineto{\pgfqpoint{4.656737in}{1.760502in}}%
\pgfpathlineto{\pgfqpoint{4.641902in}{1.748789in}}%
\pgfpathlineto{\pgfqpoint{4.627086in}{1.737256in}}%
\pgfpathlineto{\pgfqpoint{4.612287in}{1.725903in}}%
\pgfpathlineto{\pgfqpoint{4.603994in}{1.704790in}}%
\pgfpathlineto{\pgfqpoint{4.595699in}{1.683695in}}%
\pgfpathlineto{\pgfqpoint{4.587402in}{1.662624in}}%
\pgfpathlineto{\pgfqpoint{4.579103in}{1.641582in}}%
\pgfpathclose%
\pgfusepath{fill}%
\end{pgfscope}%
\begin{pgfscope}%
\pgfpathrectangle{\pgfqpoint{1.150000in}{0.150000in}}{\pgfqpoint{5.700000in}{5.700000in}}%
\pgfusepath{clip}%
\pgfsetbuttcap%
\pgfsetroundjoin%
\definecolor{currentfill}{rgb}{0.126326,0.644107,0.525311}%
\pgfsetfillcolor{currentfill}%
\pgfsetfillopacity{0.800000}%
\pgfsetlinewidth{0.000000pt}%
\definecolor{currentstroke}{rgb}{0.000000,0.000000,0.000000}%
\pgfsetstrokecolor{currentstroke}%
\pgfsetdash{}{0pt}%
\pgfpathmoveto{\pgfqpoint{4.963140in}{2.428266in}}%
\pgfpathlineto{\pgfqpoint{4.978220in}{2.444952in}}%
\pgfpathlineto{\pgfqpoint{4.993322in}{2.461829in}}%
\pgfpathlineto{\pgfqpoint{5.008446in}{2.478895in}}%
\pgfpathlineto{\pgfqpoint{5.023593in}{2.496151in}}%
\pgfpathlineto{\pgfqpoint{5.031837in}{2.516729in}}%
\pgfpathlineto{\pgfqpoint{5.040075in}{2.537150in}}%
\pgfpathlineto{\pgfqpoint{5.048308in}{2.557410in}}%
\pgfpathlineto{\pgfqpoint{5.056536in}{2.577506in}}%
\pgfpathlineto{\pgfqpoint{5.041371in}{2.559827in}}%
\pgfpathlineto{\pgfqpoint{5.026229in}{2.542340in}}%
\pgfpathlineto{\pgfqpoint{5.011109in}{2.525043in}}%
\pgfpathlineto{\pgfqpoint{4.996012in}{2.507937in}}%
\pgfpathlineto{\pgfqpoint{4.987802in}{2.488249in}}%
\pgfpathlineto{\pgfqpoint{4.979587in}{2.468405in}}%
\pgfpathlineto{\pgfqpoint{4.971366in}{2.448410in}}%
\pgfpathlineto{\pgfqpoint{4.963140in}{2.428266in}}%
\pgfpathclose%
\pgfusepath{fill}%
\end{pgfscope}%
\begin{pgfscope}%
\pgfpathrectangle{\pgfqpoint{1.150000in}{0.150000in}}{\pgfqpoint{5.700000in}{5.700000in}}%
\pgfusepath{clip}%
\pgfsetbuttcap%
\pgfsetroundjoin%
\definecolor{currentfill}{rgb}{0.421908,0.805774,0.351910}%
\pgfsetfillcolor{currentfill}%
\pgfsetfillopacity{0.800000}%
\pgfsetlinewidth{0.000000pt}%
\definecolor{currentstroke}{rgb}{0.000000,0.000000,0.000000}%
\pgfsetstrokecolor{currentstroke}%
\pgfsetdash{}{0pt}%
\pgfpathmoveto{\pgfqpoint{5.280880in}{3.018936in}}%
\pgfpathlineto{\pgfqpoint{5.296242in}{3.039164in}}%
\pgfpathlineto{\pgfqpoint{5.311629in}{3.059588in}}%
\pgfpathlineto{\pgfqpoint{5.327042in}{3.080210in}}%
\pgfpathlineto{\pgfqpoint{5.335139in}{3.096542in}}%
\pgfpathlineto{\pgfqpoint{5.343228in}{3.112636in}}%
\pgfpathlineto{\pgfqpoint{5.351307in}{3.128488in}}%
\pgfpathlineto{\pgfqpoint{5.359376in}{3.144098in}}%
\pgfpathlineto{\pgfqpoint{5.343952in}{3.123269in}}%
\pgfpathlineto{\pgfqpoint{5.328553in}{3.102639in}}%
\pgfpathlineto{\pgfqpoint{5.313180in}{3.082206in}}%
\pgfpathlineto{\pgfqpoint{5.305119in}{3.066740in}}%
\pgfpathlineto{\pgfqpoint{5.297048in}{3.051039in}}%
\pgfpathlineto{\pgfqpoint{5.288969in}{3.035103in}}%
\pgfpathlineto{\pgfqpoint{5.280880in}{3.018936in}}%
\pgfpathclose%
\pgfusepath{fill}%
\end{pgfscope}%
\begin{pgfscope}%
\pgfpathrectangle{\pgfqpoint{1.150000in}{0.150000in}}{\pgfqpoint{5.700000in}{5.700000in}}%
\pgfusepath{clip}%
\pgfsetbuttcap%
\pgfsetroundjoin%
\definecolor{currentfill}{rgb}{0.239374,0.735588,0.455688}%
\pgfsetfillcolor{currentfill}%
\pgfsetfillopacity{0.800000}%
\pgfsetlinewidth{0.000000pt}%
\definecolor{currentstroke}{rgb}{0.000000,0.000000,0.000000}%
\pgfsetstrokecolor{currentstroke}%
\pgfsetdash{}{0pt}%
\pgfpathmoveto{\pgfqpoint{5.122134in}{2.731934in}}%
\pgfpathlineto{\pgfqpoint{5.137356in}{2.750544in}}%
\pgfpathlineto{\pgfqpoint{5.152601in}{2.769346in}}%
\pgfpathlineto{\pgfqpoint{5.167870in}{2.788343in}}%
\pgfpathlineto{\pgfqpoint{5.183164in}{2.807533in}}%
\pgfpathlineto{\pgfqpoint{5.191350in}{2.826327in}}%
\pgfpathlineto{\pgfqpoint{5.199529in}{2.844915in}}%
\pgfpathlineto{\pgfqpoint{5.207700in}{2.863295in}}%
\pgfpathlineto{\pgfqpoint{5.215864in}{2.881464in}}%
\pgfpathlineto{\pgfqpoint{5.200554in}{2.861956in}}%
\pgfpathlineto{\pgfqpoint{5.185269in}{2.842643in}}%
\pgfpathlineto{\pgfqpoint{5.170009in}{2.823525in}}%
\pgfpathlineto{\pgfqpoint{5.154772in}{2.804601in}}%
\pgfpathlineto{\pgfqpoint{5.146623in}{2.786734in}}%
\pgfpathlineto{\pgfqpoint{5.138467in}{2.768666in}}%
\pgfpathlineto{\pgfqpoint{5.130304in}{2.750398in}}%
\pgfpathlineto{\pgfqpoint{5.122134in}{2.731934in}}%
\pgfpathclose%
\pgfusepath{fill}%
\end{pgfscope}%
\begin{pgfscope}%
\pgfpathrectangle{\pgfqpoint{1.150000in}{0.150000in}}{\pgfqpoint{5.700000in}{5.700000in}}%
\pgfusepath{clip}%
\pgfsetbuttcap%
\pgfsetroundjoin%
\definecolor{currentfill}{rgb}{0.153364,0.497000,0.557724}%
\pgfsetfillcolor{currentfill}%
\pgfsetfillopacity{0.800000}%
\pgfsetlinewidth{0.000000pt}%
\definecolor{currentstroke}{rgb}{0.000000,0.000000,0.000000}%
\pgfsetstrokecolor{currentstroke}%
\pgfsetdash{}{0pt}%
\pgfpathmoveto{\pgfqpoint{4.737974in}{1.946258in}}%
\pgfpathlineto{\pgfqpoint{4.752881in}{1.959599in}}%
\pgfpathlineto{\pgfqpoint{4.767807in}{1.973122in}}%
\pgfpathlineto{\pgfqpoint{4.782752in}{1.986829in}}%
\pgfpathlineto{\pgfqpoint{4.797717in}{2.000720in}}%
\pgfpathlineto{\pgfqpoint{4.806022in}{2.022919in}}%
\pgfpathlineto{\pgfqpoint{4.814324in}{2.045061in}}%
\pgfpathlineto{\pgfqpoint{4.822623in}{2.067142in}}%
\pgfpathlineto{\pgfqpoint{4.830919in}{2.089156in}}%
\pgfpathlineto{\pgfqpoint{4.815934in}{2.074677in}}%
\pgfpathlineto{\pgfqpoint{4.800970in}{2.060383in}}%
\pgfpathlineto{\pgfqpoint{4.786026in}{2.046273in}}%
\pgfpathlineto{\pgfqpoint{4.771101in}{2.032347in}}%
\pgfpathlineto{\pgfqpoint{4.762824in}{2.010908in}}%
\pgfpathlineto{\pgfqpoint{4.754544in}{1.989410in}}%
\pgfpathlineto{\pgfqpoint{4.746260in}{1.967858in}}%
\pgfpathlineto{\pgfqpoint{4.737974in}{1.946258in}}%
\pgfpathclose%
\pgfusepath{fill}%
\end{pgfscope}%
\begin{pgfscope}%
\pgfpathrectangle{\pgfqpoint{1.150000in}{0.150000in}}{\pgfqpoint{5.700000in}{5.700000in}}%
\pgfusepath{clip}%
\pgfsetbuttcap%
\pgfsetroundjoin%
\definecolor{currentfill}{rgb}{0.252194,0.269783,0.531579}%
\pgfsetfillcolor{currentfill}%
\pgfsetfillopacity{0.800000}%
\pgfsetlinewidth{0.000000pt}%
\definecolor{currentstroke}{rgb}{0.000000,0.000000,0.000000}%
\pgfsetstrokecolor{currentstroke}%
\pgfsetdash{}{0pt}%
\pgfpathmoveto{\pgfqpoint{4.387266in}{1.285507in}}%
\pgfpathlineto{\pgfqpoint{4.401937in}{1.292474in}}%
\pgfpathlineto{\pgfqpoint{4.416624in}{1.299615in}}%
\pgfpathlineto{\pgfqpoint{4.431325in}{1.306930in}}%
\pgfpathlineto{\pgfqpoint{4.446041in}{1.314418in}}%
\pgfpathlineto{\pgfqpoint{4.454372in}{1.334063in}}%
\pgfpathlineto{\pgfqpoint{4.462701in}{1.353849in}}%
\pgfpathlineto{\pgfqpoint{4.471028in}{1.373769in}}%
\pgfpathlineto{\pgfqpoint{4.479353in}{1.393814in}}%
\pgfpathlineto{\pgfqpoint{4.464626in}{1.385528in}}%
\pgfpathlineto{\pgfqpoint{4.449914in}{1.377416in}}%
\pgfpathlineto{\pgfqpoint{4.435218in}{1.369479in}}%
\pgfpathlineto{\pgfqpoint{4.420537in}{1.361717in}}%
\pgfpathlineto{\pgfqpoint{4.412223in}{1.342457in}}%
\pgfpathlineto{\pgfqpoint{4.403906in}{1.323330in}}%
\pgfpathlineto{\pgfqpoint{4.395587in}{1.304344in}}%
\pgfpathlineto{\pgfqpoint{4.387266in}{1.285507in}}%
\pgfpathclose%
\pgfusepath{fill}%
\end{pgfscope}%
\begin{pgfscope}%
\pgfpathrectangle{\pgfqpoint{1.150000in}{0.150000in}}{\pgfqpoint{5.700000in}{5.700000in}}%
\pgfusepath{clip}%
\pgfsetbuttcap%
\pgfsetroundjoin%
\definecolor{currentfill}{rgb}{0.208623,0.367752,0.552675}%
\pgfsetfillcolor{currentfill}%
\pgfsetfillopacity{0.800000}%
\pgfsetlinewidth{0.000000pt}%
\definecolor{currentstroke}{rgb}{0.000000,0.000000,0.000000}%
\pgfsetstrokecolor{currentstroke}%
\pgfsetdash{}{0pt}%
\pgfpathmoveto{\pgfqpoint{4.545885in}{1.557845in}}%
\pgfpathlineto{\pgfqpoint{4.560654in}{1.567814in}}%
\pgfpathlineto{\pgfqpoint{4.575439in}{1.577961in}}%
\pgfpathlineto{\pgfqpoint{4.590242in}{1.588285in}}%
\pgfpathlineto{\pgfqpoint{4.605062in}{1.598787in}}%
\pgfpathlineto{\pgfqpoint{4.613385in}{1.620365in}}%
\pgfpathlineto{\pgfqpoint{4.621706in}{1.641991in}}%
\pgfpathlineto{\pgfqpoint{4.630025in}{1.663659in}}%
\pgfpathlineto{\pgfqpoint{4.638342in}{1.685362in}}%
\pgfpathlineto{\pgfqpoint{4.623506in}{1.674148in}}%
\pgfpathlineto{\pgfqpoint{4.608687in}{1.663114in}}%
\pgfpathlineto{\pgfqpoint{4.593886in}{1.652259in}}%
\pgfpathlineto{\pgfqpoint{4.579103in}{1.641582in}}%
\pgfpathlineto{\pgfqpoint{4.570801in}{1.620577in}}%
\pgfpathlineto{\pgfqpoint{4.562498in}{1.599615in}}%
\pgfpathlineto{\pgfqpoint{4.554192in}{1.578702in}}%
\pgfpathlineto{\pgfqpoint{4.545885in}{1.557845in}}%
\pgfpathclose%
\pgfusepath{fill}%
\end{pgfscope}%
\begin{pgfscope}%
\pgfpathrectangle{\pgfqpoint{1.150000in}{0.150000in}}{\pgfqpoint{5.700000in}{5.700000in}}%
\pgfusepath{clip}%
\pgfsetbuttcap%
\pgfsetroundjoin%
\definecolor{currentfill}{rgb}{0.120081,0.622161,0.534946}%
\pgfsetfillcolor{currentfill}%
\pgfsetfillopacity{0.800000}%
\pgfsetlinewidth{0.000000pt}%
\definecolor{currentstroke}{rgb}{0.000000,0.000000,0.000000}%
\pgfsetstrokecolor{currentstroke}%
\pgfsetdash{}{0pt}%
\pgfpathmoveto{\pgfqpoint{4.930187in}{2.346284in}}%
\pgfpathlineto{\pgfqpoint{4.945249in}{2.362516in}}%
\pgfpathlineto{\pgfqpoint{4.960332in}{2.378937in}}%
\pgfpathlineto{\pgfqpoint{4.975438in}{2.395547in}}%
\pgfpathlineto{\pgfqpoint{4.990566in}{2.412346in}}%
\pgfpathlineto{\pgfqpoint{4.998830in}{2.433513in}}%
\pgfpathlineto{\pgfqpoint{5.007089in}{2.454539in}}%
\pgfpathlineto{\pgfqpoint{5.015343in}{2.475420in}}%
\pgfpathlineto{\pgfqpoint{5.023593in}{2.496151in}}%
\pgfpathlineto{\pgfqpoint{5.008446in}{2.478895in}}%
\pgfpathlineto{\pgfqpoint{4.993322in}{2.461829in}}%
\pgfpathlineto{\pgfqpoint{4.978220in}{2.444952in}}%
\pgfpathlineto{\pgfqpoint{4.963140in}{2.428266in}}%
\pgfpathlineto{\pgfqpoint{4.954909in}{2.407977in}}%
\pgfpathlineto{\pgfqpoint{4.946673in}{2.387548in}}%
\pgfpathlineto{\pgfqpoint{4.938432in}{2.366982in}}%
\pgfpathlineto{\pgfqpoint{4.930187in}{2.346284in}}%
\pgfpathclose%
\pgfusepath{fill}%
\end{pgfscope}%
\begin{pgfscope}%
\pgfpathrectangle{\pgfqpoint{1.150000in}{0.150000in}}{\pgfqpoint{5.700000in}{5.700000in}}%
\pgfusepath{clip}%
\pgfsetbuttcap%
\pgfsetroundjoin%
\definecolor{currentfill}{rgb}{0.163625,0.471133,0.558148}%
\pgfsetfillcolor{currentfill}%
\pgfsetfillopacity{0.800000}%
\pgfsetlinewidth{0.000000pt}%
\definecolor{currentstroke}{rgb}{0.000000,0.000000,0.000000}%
\pgfsetstrokecolor{currentstroke}%
\pgfsetdash{}{0pt}%
\pgfpathmoveto{\pgfqpoint{4.704802in}{1.859487in}}%
\pgfpathlineto{\pgfqpoint{4.719691in}{1.872210in}}%
\pgfpathlineto{\pgfqpoint{4.734598in}{1.885115in}}%
\pgfpathlineto{\pgfqpoint{4.749525in}{1.898202in}}%
\pgfpathlineto{\pgfqpoint{4.764471in}{1.911472in}}%
\pgfpathlineto{\pgfqpoint{4.772787in}{1.933840in}}%
\pgfpathlineto{\pgfqpoint{4.781099in}{1.956175in}}%
\pgfpathlineto{\pgfqpoint{4.789410in}{1.978470in}}%
\pgfpathlineto{\pgfqpoint{4.797717in}{2.000720in}}%
\pgfpathlineto{\pgfqpoint{4.782752in}{1.986829in}}%
\pgfpathlineto{\pgfqpoint{4.767807in}{1.973122in}}%
\pgfpathlineto{\pgfqpoint{4.752881in}{1.959599in}}%
\pgfpathlineto{\pgfqpoint{4.737974in}{1.946258in}}%
\pgfpathlineto{\pgfqpoint{4.729685in}{1.924615in}}%
\pgfpathlineto{\pgfqpoint{4.721393in}{1.902936in}}%
\pgfpathlineto{\pgfqpoint{4.713099in}{1.881225in}}%
\pgfpathlineto{\pgfqpoint{4.704802in}{1.859487in}}%
\pgfpathclose%
\pgfusepath{fill}%
\end{pgfscope}%
\begin{pgfscope}%
\pgfpathrectangle{\pgfqpoint{1.150000in}{0.150000in}}{\pgfqpoint{5.700000in}{5.700000in}}%
\pgfusepath{clip}%
\pgfsetbuttcap%
\pgfsetroundjoin%
\definecolor{currentfill}{rgb}{0.386433,0.794644,0.372886}%
\pgfsetfillcolor{currentfill}%
\pgfsetfillopacity{0.800000}%
\pgfsetlinewidth{0.000000pt}%
\definecolor{currentstroke}{rgb}{0.000000,0.000000,0.000000}%
\pgfsetstrokecolor{currentstroke}%
\pgfsetdash{}{0pt}%
\pgfpathmoveto{\pgfqpoint{5.248439in}{2.951981in}}%
\pgfpathlineto{\pgfqpoint{5.263788in}{2.971965in}}%
\pgfpathlineto{\pgfqpoint{5.279161in}{2.992145in}}%
\pgfpathlineto{\pgfqpoint{5.294560in}{3.012523in}}%
\pgfpathlineto{\pgfqpoint{5.302694in}{3.029794in}}%
\pgfpathlineto{\pgfqpoint{5.310819in}{3.046834in}}%
\pgfpathlineto{\pgfqpoint{5.318935in}{3.063640in}}%
\pgfpathlineto{\pgfqpoint{5.327042in}{3.080210in}}%
\pgfpathlineto{\pgfqpoint{5.311629in}{3.059588in}}%
\pgfpathlineto{\pgfqpoint{5.296242in}{3.039164in}}%
\pgfpathlineto{\pgfqpoint{5.280880in}{3.018936in}}%
\pgfpathlineto{\pgfqpoint{5.272783in}{3.002538in}}%
\pgfpathlineto{\pgfqpoint{5.264677in}{2.985912in}}%
\pgfpathlineto{\pgfqpoint{5.256562in}{2.969059in}}%
\pgfpathlineto{\pgfqpoint{5.248439in}{2.951981in}}%
\pgfpathclose%
\pgfusepath{fill}%
\end{pgfscope}%
\begin{pgfscope}%
\pgfpathrectangle{\pgfqpoint{1.150000in}{0.150000in}}{\pgfqpoint{5.700000in}{5.700000in}}%
\pgfusepath{clip}%
\pgfsetbuttcap%
\pgfsetroundjoin%
\definecolor{currentfill}{rgb}{0.220057,0.343307,0.549413}%
\pgfsetfillcolor{currentfill}%
\pgfsetfillopacity{0.800000}%
\pgfsetlinewidth{0.000000pt}%
\definecolor{currentstroke}{rgb}{0.000000,0.000000,0.000000}%
\pgfsetstrokecolor{currentstroke}%
\pgfsetdash{}{0pt}%
\pgfpathmoveto{\pgfqpoint{4.512634in}{1.475110in}}%
\pgfpathlineto{\pgfqpoint{4.527389in}{1.484340in}}%
\pgfpathlineto{\pgfqpoint{4.542161in}{1.493747in}}%
\pgfpathlineto{\pgfqpoint{4.556949in}{1.503330in}}%
\pgfpathlineto{\pgfqpoint{4.571754in}{1.513090in}}%
\pgfpathlineto{\pgfqpoint{4.580084in}{1.534408in}}%
\pgfpathlineto{\pgfqpoint{4.588412in}{1.555801in}}%
\pgfpathlineto{\pgfqpoint{4.596738in}{1.577263in}}%
\pgfpathlineto{\pgfqpoint{4.605062in}{1.598787in}}%
\pgfpathlineto{\pgfqpoint{4.590242in}{1.588285in}}%
\pgfpathlineto{\pgfqpoint{4.575439in}{1.577961in}}%
\pgfpathlineto{\pgfqpoint{4.560654in}{1.567814in}}%
\pgfpathlineto{\pgfqpoint{4.545885in}{1.557845in}}%
\pgfpathlineto{\pgfqpoint{4.537575in}{1.537051in}}%
\pgfpathlineto{\pgfqpoint{4.529263in}{1.516325in}}%
\pgfpathlineto{\pgfqpoint{4.520950in}{1.495676in}}%
\pgfpathlineto{\pgfqpoint{4.512634in}{1.475110in}}%
\pgfpathclose%
\pgfusepath{fill}%
\end{pgfscope}%
\begin{pgfscope}%
\pgfpathrectangle{\pgfqpoint{1.150000in}{0.150000in}}{\pgfqpoint{5.700000in}{5.700000in}}%
\pgfusepath{clip}%
\pgfsetbuttcap%
\pgfsetroundjoin%
\definecolor{currentfill}{rgb}{0.208030,0.718701,0.472873}%
\pgfsetfillcolor{currentfill}%
\pgfsetfillopacity{0.800000}%
\pgfsetlinewidth{0.000000pt}%
\definecolor{currentstroke}{rgb}{0.000000,0.000000,0.000000}%
\pgfsetstrokecolor{currentstroke}%
\pgfsetdash{}{0pt}%
\pgfpathmoveto{\pgfqpoint{5.089386in}{2.656172in}}%
\pgfpathlineto{\pgfqpoint{5.104591in}{2.674430in}}%
\pgfpathlineto{\pgfqpoint{5.119820in}{2.692880in}}%
\pgfpathlineto{\pgfqpoint{5.135072in}{2.711523in}}%
\pgfpathlineto{\pgfqpoint{5.150348in}{2.730360in}}%
\pgfpathlineto{\pgfqpoint{5.158562in}{2.749947in}}%
\pgfpathlineto{\pgfqpoint{5.166770in}{2.769341in}}%
\pgfpathlineto{\pgfqpoint{5.174970in}{2.788537in}}%
\pgfpathlineto{\pgfqpoint{5.183164in}{2.807533in}}%
\pgfpathlineto{\pgfqpoint{5.167870in}{2.788343in}}%
\pgfpathlineto{\pgfqpoint{5.152601in}{2.769346in}}%
\pgfpathlineto{\pgfqpoint{5.137356in}{2.750544in}}%
\pgfpathlineto{\pgfqpoint{5.122134in}{2.731934in}}%
\pgfpathlineto{\pgfqpoint{5.113957in}{2.713277in}}%
\pgfpathlineto{\pgfqpoint{5.105773in}{2.694429in}}%
\pgfpathlineto{\pgfqpoint{5.097583in}{2.675393in}}%
\pgfpathlineto{\pgfqpoint{5.089386in}{2.656172in}}%
\pgfpathclose%
\pgfusepath{fill}%
\end{pgfscope}%
\begin{pgfscope}%
\pgfpathrectangle{\pgfqpoint{1.150000in}{0.150000in}}{\pgfqpoint{5.700000in}{5.700000in}}%
\pgfusepath{clip}%
\pgfsetbuttcap%
\pgfsetroundjoin%
\definecolor{currentfill}{rgb}{0.120092,0.600104,0.542530}%
\pgfsetfillcolor{currentfill}%
\pgfsetfillopacity{0.800000}%
\pgfsetlinewidth{0.000000pt}%
\definecolor{currentstroke}{rgb}{0.000000,0.000000,0.000000}%
\pgfsetstrokecolor{currentstroke}%
\pgfsetdash{}{0pt}%
\pgfpathmoveto{\pgfqpoint{4.897161in}{2.262249in}}%
\pgfpathlineto{\pgfqpoint{4.912204in}{2.277992in}}%
\pgfpathlineto{\pgfqpoint{4.927268in}{2.293923in}}%
\pgfpathlineto{\pgfqpoint{4.942355in}{2.310042in}}%
\pgfpathlineto{\pgfqpoint{4.957463in}{2.326349in}}%
\pgfpathlineto{\pgfqpoint{4.965745in}{2.348038in}}%
\pgfpathlineto{\pgfqpoint{4.974023in}{2.369604in}}%
\pgfpathlineto{\pgfqpoint{4.982297in}{2.391041in}}%
\pgfpathlineto{\pgfqpoint{4.990566in}{2.412346in}}%
\pgfpathlineto{\pgfqpoint{4.975438in}{2.395547in}}%
\pgfpathlineto{\pgfqpoint{4.960332in}{2.378937in}}%
\pgfpathlineto{\pgfqpoint{4.945249in}{2.362516in}}%
\pgfpathlineto{\pgfqpoint{4.930187in}{2.346284in}}%
\pgfpathlineto{\pgfqpoint{4.921937in}{2.325457in}}%
\pgfpathlineto{\pgfqpoint{4.913682in}{2.304506in}}%
\pgfpathlineto{\pgfqpoint{4.905424in}{2.283435in}}%
\pgfpathlineto{\pgfqpoint{4.897161in}{2.262249in}}%
\pgfpathclose%
\pgfusepath{fill}%
\end{pgfscope}%
\begin{pgfscope}%
\pgfpathrectangle{\pgfqpoint{1.150000in}{0.150000in}}{\pgfqpoint{5.700000in}{5.700000in}}%
\pgfusepath{clip}%
\pgfsetbuttcap%
\pgfsetroundjoin%
\definecolor{currentfill}{rgb}{0.262138,0.242286,0.520837}%
\pgfsetfillcolor{currentfill}%
\pgfsetfillopacity{0.800000}%
\pgfsetlinewidth{0.000000pt}%
\definecolor{currentstroke}{rgb}{0.000000,0.000000,0.000000}%
\pgfsetstrokecolor{currentstroke}%
\pgfsetdash{}{0pt}%
\pgfpathmoveto{\pgfqpoint{4.353956in}{1.211794in}}%
\pgfpathlineto{\pgfqpoint{4.368620in}{1.217937in}}%
\pgfpathlineto{\pgfqpoint{4.383298in}{1.224251in}}%
\pgfpathlineto{\pgfqpoint{4.397991in}{1.230739in}}%
\pgfpathlineto{\pgfqpoint{4.412697in}{1.237399in}}%
\pgfpathlineto{\pgfqpoint{4.421036in}{1.256404in}}%
\pgfpathlineto{\pgfqpoint{4.429373in}{1.275580in}}%
\pgfpathlineto{\pgfqpoint{4.437708in}{1.294921in}}%
\pgfpathlineto{\pgfqpoint{4.446041in}{1.314418in}}%
\pgfpathlineto{\pgfqpoint{4.431325in}{1.306930in}}%
\pgfpathlineto{\pgfqpoint{4.416624in}{1.299615in}}%
\pgfpathlineto{\pgfqpoint{4.401937in}{1.292474in}}%
\pgfpathlineto{\pgfqpoint{4.387266in}{1.285507in}}%
\pgfpathlineto{\pgfqpoint{4.378942in}{1.266825in}}%
\pgfpathlineto{\pgfqpoint{4.370616in}{1.248308in}}%
\pgfpathlineto{\pgfqpoint{4.362287in}{1.229961in}}%
\pgfpathlineto{\pgfqpoint{4.353956in}{1.211794in}}%
\pgfpathclose%
\pgfusepath{fill}%
\end{pgfscope}%
\begin{pgfscope}%
\pgfpathrectangle{\pgfqpoint{1.150000in}{0.150000in}}{\pgfqpoint{5.700000in}{5.700000in}}%
\pgfusepath{clip}%
\pgfsetbuttcap%
\pgfsetroundjoin%
\definecolor{currentfill}{rgb}{0.172719,0.448791,0.557885}%
\pgfsetfillcolor{currentfill}%
\pgfsetfillopacity{0.800000}%
\pgfsetlinewidth{0.000000pt}%
\definecolor{currentstroke}{rgb}{0.000000,0.000000,0.000000}%
\pgfsetstrokecolor{currentstroke}%
\pgfsetdash{}{0pt}%
\pgfpathmoveto{\pgfqpoint{4.671590in}{1.772395in}}%
\pgfpathlineto{\pgfqpoint{4.686461in}{1.784469in}}%
\pgfpathlineto{\pgfqpoint{4.701350in}{1.796723in}}%
\pgfpathlineto{\pgfqpoint{4.716259in}{1.809158in}}%
\pgfpathlineto{\pgfqpoint{4.731186in}{1.821774in}}%
\pgfpathlineto{\pgfqpoint{4.739510in}{1.844220in}}%
\pgfpathlineto{\pgfqpoint{4.747833in}{1.866655in}}%
\pgfpathlineto{\pgfqpoint{4.756153in}{1.889075in}}%
\pgfpathlineto{\pgfqpoint{4.764471in}{1.911472in}}%
\pgfpathlineto{\pgfqpoint{4.749525in}{1.898202in}}%
\pgfpathlineto{\pgfqpoint{4.734598in}{1.885115in}}%
\pgfpathlineto{\pgfqpoint{4.719691in}{1.872210in}}%
\pgfpathlineto{\pgfqpoint{4.704802in}{1.859487in}}%
\pgfpathlineto{\pgfqpoint{4.696502in}{1.837730in}}%
\pgfpathlineto{\pgfqpoint{4.688200in}{1.815958in}}%
\pgfpathlineto{\pgfqpoint{4.679896in}{1.794178in}}%
\pgfpathlineto{\pgfqpoint{4.671590in}{1.772395in}}%
\pgfpathclose%
\pgfusepath{fill}%
\end{pgfscope}%
\begin{pgfscope}%
\pgfpathrectangle{\pgfqpoint{1.150000in}{0.150000in}}{\pgfqpoint{5.700000in}{5.700000in}}%
\pgfusepath{clip}%
\pgfsetbuttcap%
\pgfsetroundjoin%
\definecolor{currentfill}{rgb}{0.231674,0.318106,0.544834}%
\pgfsetfillcolor{currentfill}%
\pgfsetfillopacity{0.800000}%
\pgfsetlinewidth{0.000000pt}%
\definecolor{currentstroke}{rgb}{0.000000,0.000000,0.000000}%
\pgfsetstrokecolor{currentstroke}%
\pgfsetdash{}{0pt}%
\pgfpathmoveto{\pgfqpoint{4.479353in}{1.393814in}}%
\pgfpathlineto{\pgfqpoint{4.494095in}{1.402276in}}%
\pgfpathlineto{\pgfqpoint{4.508854in}{1.410912in}}%
\pgfpathlineto{\pgfqpoint{4.523628in}{1.419724in}}%
\pgfpathlineto{\pgfqpoint{4.538419in}{1.428711in}}%
\pgfpathlineto{\pgfqpoint{4.546755in}{1.449658in}}%
\pgfpathlineto{\pgfqpoint{4.555090in}{1.470708in}}%
\pgfpathlineto{\pgfqpoint{4.563423in}{1.491854in}}%
\pgfpathlineto{\pgfqpoint{4.571754in}{1.513090in}}%
\pgfpathlineto{\pgfqpoint{4.556949in}{1.503330in}}%
\pgfpathlineto{\pgfqpoint{4.542161in}{1.493747in}}%
\pgfpathlineto{\pgfqpoint{4.527389in}{1.484340in}}%
\pgfpathlineto{\pgfqpoint{4.512634in}{1.475110in}}%
\pgfpathlineto{\pgfqpoint{4.504317in}{1.454633in}}%
\pgfpathlineto{\pgfqpoint{4.495997in}{1.434254in}}%
\pgfpathlineto{\pgfqpoint{4.487676in}{1.413979in}}%
\pgfpathlineto{\pgfqpoint{4.479353in}{1.393814in}}%
\pgfpathclose%
\pgfusepath{fill}%
\end{pgfscope}%
\begin{pgfscope}%
\pgfpathrectangle{\pgfqpoint{1.150000in}{0.150000in}}{\pgfqpoint{5.700000in}{5.700000in}}%
\pgfusepath{clip}%
\pgfsetbuttcap%
\pgfsetroundjoin%
\definecolor{currentfill}{rgb}{0.124395,0.578002,0.548287}%
\pgfsetfillcolor{currentfill}%
\pgfsetfillopacity{0.800000}%
\pgfsetlinewidth{0.000000pt}%
\definecolor{currentstroke}{rgb}{0.000000,0.000000,0.000000}%
\pgfsetstrokecolor{currentstroke}%
\pgfsetdash{}{0pt}%
\pgfpathmoveto{\pgfqpoint{4.864069in}{2.176439in}}%
\pgfpathlineto{\pgfqpoint{4.879093in}{2.191660in}}%
\pgfpathlineto{\pgfqpoint{4.894138in}{2.207067in}}%
\pgfpathlineto{\pgfqpoint{4.909204in}{2.222660in}}%
\pgfpathlineto{\pgfqpoint{4.924292in}{2.238441in}}%
\pgfpathlineto{\pgfqpoint{4.932591in}{2.260581in}}%
\pgfpathlineto{\pgfqpoint{4.940885in}{2.282615in}}%
\pgfpathlineto{\pgfqpoint{4.949176in}{2.304539in}}%
\pgfpathlineto{\pgfqpoint{4.957463in}{2.326349in}}%
\pgfpathlineto{\pgfqpoint{4.942355in}{2.310042in}}%
\pgfpathlineto{\pgfqpoint{4.927268in}{2.293923in}}%
\pgfpathlineto{\pgfqpoint{4.912204in}{2.277992in}}%
\pgfpathlineto{\pgfqpoint{4.897161in}{2.262249in}}%
\pgfpathlineto{\pgfqpoint{4.888893in}{2.240951in}}%
\pgfpathlineto{\pgfqpoint{4.880622in}{2.219548in}}%
\pgfpathlineto{\pgfqpoint{4.872347in}{2.198042in}}%
\pgfpathlineto{\pgfqpoint{4.864069in}{2.176439in}}%
\pgfpathclose%
\pgfusepath{fill}%
\end{pgfscope}%
\begin{pgfscope}%
\pgfpathrectangle{\pgfqpoint{1.150000in}{0.150000in}}{\pgfqpoint{5.700000in}{5.700000in}}%
\pgfusepath{clip}%
\pgfsetbuttcap%
\pgfsetroundjoin%
\definecolor{currentfill}{rgb}{0.180653,0.701402,0.488189}%
\pgfsetfillcolor{currentfill}%
\pgfsetfillopacity{0.800000}%
\pgfsetlinewidth{0.000000pt}%
\definecolor{currentstroke}{rgb}{0.000000,0.000000,0.000000}%
\pgfsetstrokecolor{currentstroke}%
\pgfsetdash{}{0pt}%
\pgfpathmoveto{\pgfqpoint{5.056536in}{2.577506in}}%
\pgfpathlineto{\pgfqpoint{5.071723in}{2.595376in}}%
\pgfpathlineto{\pgfqpoint{5.086934in}{2.613438in}}%
\pgfpathlineto{\pgfqpoint{5.102169in}{2.631692in}}%
\pgfpathlineto{\pgfqpoint{5.117426in}{2.650139in}}%
\pgfpathlineto{\pgfqpoint{5.125666in}{2.670469in}}%
\pgfpathlineto{\pgfqpoint{5.133900in}{2.690618in}}%
\pgfpathlineto{\pgfqpoint{5.142127in}{2.710583in}}%
\pgfpathlineto{\pgfqpoint{5.150348in}{2.730360in}}%
\pgfpathlineto{\pgfqpoint{5.135072in}{2.711523in}}%
\pgfpathlineto{\pgfqpoint{5.119820in}{2.692880in}}%
\pgfpathlineto{\pgfqpoint{5.104591in}{2.674430in}}%
\pgfpathlineto{\pgfqpoint{5.089386in}{2.656172in}}%
\pgfpathlineto{\pgfqpoint{5.081182in}{2.636770in}}%
\pgfpathlineto{\pgfqpoint{5.072973in}{2.617189in}}%
\pgfpathlineto{\pgfqpoint{5.064757in}{2.597433in}}%
\pgfpathlineto{\pgfqpoint{5.056536in}{2.577506in}}%
\pgfpathclose%
\pgfusepath{fill}%
\end{pgfscope}%
\begin{pgfscope}%
\pgfpathrectangle{\pgfqpoint{1.150000in}{0.150000in}}{\pgfqpoint{5.700000in}{5.700000in}}%
\pgfusepath{clip}%
\pgfsetbuttcap%
\pgfsetroundjoin%
\definecolor{currentfill}{rgb}{0.352360,0.783011,0.392636}%
\pgfsetfillcolor{currentfill}%
\pgfsetfillopacity{0.800000}%
\pgfsetlinewidth{0.000000pt}%
\definecolor{currentstroke}{rgb}{0.000000,0.000000,0.000000}%
\pgfsetstrokecolor{currentstroke}%
\pgfsetdash{}{0pt}%
\pgfpathmoveto{\pgfqpoint{5.215864in}{2.881464in}}%
\pgfpathlineto{\pgfqpoint{5.231198in}{2.901167in}}%
\pgfpathlineto{\pgfqpoint{5.246556in}{2.921066in}}%
\pgfpathlineto{\pgfqpoint{5.261940in}{2.941162in}}%
\pgfpathlineto{\pgfqpoint{5.270107in}{2.959339in}}%
\pgfpathlineto{\pgfqpoint{5.278267in}{2.977293in}}%
\pgfpathlineto{\pgfqpoint{5.286418in}{2.995022in}}%
\pgfpathlineto{\pgfqpoint{5.294560in}{3.012523in}}%
\pgfpathlineto{\pgfqpoint{5.279161in}{2.992145in}}%
\pgfpathlineto{\pgfqpoint{5.263788in}{2.971965in}}%
\pgfpathlineto{\pgfqpoint{5.248439in}{2.951981in}}%
\pgfpathlineto{\pgfqpoint{5.240307in}{2.934680in}}%
\pgfpathlineto{\pgfqpoint{5.232167in}{2.917159in}}%
\pgfpathlineto{\pgfqpoint{5.224019in}{2.899419in}}%
\pgfpathlineto{\pgfqpoint{5.215864in}{2.881464in}}%
\pgfpathclose%
\pgfusepath{fill}%
\end{pgfscope}%
\begin{pgfscope}%
\pgfpathrectangle{\pgfqpoint{1.150000in}{0.150000in}}{\pgfqpoint{5.700000in}{5.700000in}}%
\pgfusepath{clip}%
\pgfsetbuttcap%
\pgfsetroundjoin%
\definecolor{currentfill}{rgb}{0.183898,0.422383,0.556944}%
\pgfsetfillcolor{currentfill}%
\pgfsetfillopacity{0.800000}%
\pgfsetlinewidth{0.000000pt}%
\definecolor{currentstroke}{rgb}{0.000000,0.000000,0.000000}%
\pgfsetstrokecolor{currentstroke}%
\pgfsetdash{}{0pt}%
\pgfpathmoveto{\pgfqpoint{4.638342in}{1.685362in}}%
\pgfpathlineto{\pgfqpoint{4.653196in}{1.696754in}}%
\pgfpathlineto{\pgfqpoint{4.668068in}{1.708325in}}%
\pgfpathlineto{\pgfqpoint{4.682958in}{1.720077in}}%
\pgfpathlineto{\pgfqpoint{4.697866in}{1.732008in}}%
\pgfpathlineto{\pgfqpoint{4.706199in}{1.754434in}}%
\pgfpathlineto{\pgfqpoint{4.714530in}{1.776875in}}%
\pgfpathlineto{\pgfqpoint{4.722859in}{1.799323in}}%
\pgfpathlineto{\pgfqpoint{4.731186in}{1.821774in}}%
\pgfpathlineto{\pgfqpoint{4.716259in}{1.809158in}}%
\pgfpathlineto{\pgfqpoint{4.701350in}{1.796723in}}%
\pgfpathlineto{\pgfqpoint{4.686461in}{1.784469in}}%
\pgfpathlineto{\pgfqpoint{4.671590in}{1.772395in}}%
\pgfpathlineto{\pgfqpoint{4.663281in}{1.750616in}}%
\pgfpathlineto{\pgfqpoint{4.654970in}{1.728847in}}%
\pgfpathlineto{\pgfqpoint{4.646657in}{1.707093in}}%
\pgfpathlineto{\pgfqpoint{4.638342in}{1.685362in}}%
\pgfpathclose%
\pgfusepath{fill}%
\end{pgfscope}%
\begin{pgfscope}%
\pgfpathrectangle{\pgfqpoint{1.150000in}{0.150000in}}{\pgfqpoint{5.700000in}{5.700000in}}%
\pgfusepath{clip}%
\pgfsetbuttcap%
\pgfsetroundjoin%
\definecolor{currentfill}{rgb}{0.132444,0.552216,0.553018}%
\pgfsetfillcolor{currentfill}%
\pgfsetfillopacity{0.800000}%
\pgfsetlinewidth{0.000000pt}%
\definecolor{currentstroke}{rgb}{0.000000,0.000000,0.000000}%
\pgfsetstrokecolor{currentstroke}%
\pgfsetdash{}{0pt}%
\pgfpathmoveto{\pgfqpoint{4.830919in}{2.089156in}}%
\pgfpathlineto{\pgfqpoint{4.845923in}{2.103820in}}%
\pgfpathlineto{\pgfqpoint{4.860949in}{2.118669in}}%
\pgfpathlineto{\pgfqpoint{4.875995in}{2.133704in}}%
\pgfpathlineto{\pgfqpoint{4.891062in}{2.148924in}}%
\pgfpathlineto{\pgfqpoint{4.899375in}{2.171437in}}%
\pgfpathlineto{\pgfqpoint{4.907684in}{2.193864in}}%
\pgfpathlineto{\pgfqpoint{4.915990in}{2.216200in}}%
\pgfpathlineto{\pgfqpoint{4.924292in}{2.238441in}}%
\pgfpathlineto{\pgfqpoint{4.909204in}{2.222660in}}%
\pgfpathlineto{\pgfqpoint{4.894138in}{2.207067in}}%
\pgfpathlineto{\pgfqpoint{4.879093in}{2.191660in}}%
\pgfpathlineto{\pgfqpoint{4.864069in}{2.176439in}}%
\pgfpathlineto{\pgfqpoint{4.855786in}{2.154745in}}%
\pgfpathlineto{\pgfqpoint{4.847501in}{2.132962in}}%
\pgfpathlineto{\pgfqpoint{4.839211in}{2.111098in}}%
\pgfpathlineto{\pgfqpoint{4.830919in}{2.089156in}}%
\pgfpathclose%
\pgfusepath{fill}%
\end{pgfscope}%
\begin{pgfscope}%
\pgfpathrectangle{\pgfqpoint{1.150000in}{0.150000in}}{\pgfqpoint{5.700000in}{5.700000in}}%
\pgfusepath{clip}%
\pgfsetbuttcap%
\pgfsetroundjoin%
\definecolor{currentfill}{rgb}{0.243113,0.292092,0.538516}%
\pgfsetfillcolor{currentfill}%
\pgfsetfillopacity{0.800000}%
\pgfsetlinewidth{0.000000pt}%
\definecolor{currentstroke}{rgb}{0.000000,0.000000,0.000000}%
\pgfsetstrokecolor{currentstroke}%
\pgfsetdash{}{0pt}%
\pgfpathmoveto{\pgfqpoint{4.446041in}{1.314418in}}%
\pgfpathlineto{\pgfqpoint{4.460772in}{1.322080in}}%
\pgfpathlineto{\pgfqpoint{4.475518in}{1.329916in}}%
\pgfpathlineto{\pgfqpoint{4.490280in}{1.337926in}}%
\pgfpathlineto{\pgfqpoint{4.505058in}{1.346110in}}%
\pgfpathlineto{\pgfqpoint{4.513400in}{1.366568in}}%
\pgfpathlineto{\pgfqpoint{4.521741in}{1.387159in}}%
\pgfpathlineto{\pgfqpoint{4.530081in}{1.407876in}}%
\pgfpathlineto{\pgfqpoint{4.538419in}{1.428711in}}%
\pgfpathlineto{\pgfqpoint{4.523628in}{1.419724in}}%
\pgfpathlineto{\pgfqpoint{4.508854in}{1.410912in}}%
\pgfpathlineto{\pgfqpoint{4.494095in}{1.402276in}}%
\pgfpathlineto{\pgfqpoint{4.479353in}{1.393814in}}%
\pgfpathlineto{\pgfqpoint{4.471028in}{1.373769in}}%
\pgfpathlineto{\pgfqpoint{4.462701in}{1.353849in}}%
\pgfpathlineto{\pgfqpoint{4.454372in}{1.334063in}}%
\pgfpathlineto{\pgfqpoint{4.446041in}{1.314418in}}%
\pgfpathclose%
\pgfusepath{fill}%
\end{pgfscope}%
\begin{pgfscope}%
\pgfpathrectangle{\pgfqpoint{1.150000in}{0.150000in}}{\pgfqpoint{5.700000in}{5.700000in}}%
\pgfusepath{clip}%
\pgfsetbuttcap%
\pgfsetroundjoin%
\definecolor{currentfill}{rgb}{0.153894,0.680203,0.504172}%
\pgfsetfillcolor{currentfill}%
\pgfsetfillopacity{0.800000}%
\pgfsetlinewidth{0.000000pt}%
\definecolor{currentstroke}{rgb}{0.000000,0.000000,0.000000}%
\pgfsetstrokecolor{currentstroke}%
\pgfsetdash{}{0pt}%
\pgfpathmoveto{\pgfqpoint{5.023593in}{2.496151in}}%
\pgfpathlineto{\pgfqpoint{5.038762in}{2.513598in}}%
\pgfpathlineto{\pgfqpoint{5.053954in}{2.531236in}}%
\pgfpathlineto{\pgfqpoint{5.069169in}{2.549065in}}%
\pgfpathlineto{\pgfqpoint{5.084407in}{2.567086in}}%
\pgfpathlineto{\pgfqpoint{5.092671in}{2.588102in}}%
\pgfpathlineto{\pgfqpoint{5.100928in}{2.608952in}}%
\pgfpathlineto{\pgfqpoint{5.109180in}{2.629632in}}%
\pgfpathlineto{\pgfqpoint{5.117426in}{2.650139in}}%
\pgfpathlineto{\pgfqpoint{5.102169in}{2.631692in}}%
\pgfpathlineto{\pgfqpoint{5.086934in}{2.613438in}}%
\pgfpathlineto{\pgfqpoint{5.071723in}{2.595376in}}%
\pgfpathlineto{\pgfqpoint{5.056536in}{2.577506in}}%
\pgfpathlineto{\pgfqpoint{5.048308in}{2.557410in}}%
\pgfpathlineto{\pgfqpoint{5.040075in}{2.537150in}}%
\pgfpathlineto{\pgfqpoint{5.031837in}{2.516729in}}%
\pgfpathlineto{\pgfqpoint{5.023593in}{2.496151in}}%
\pgfpathclose%
\pgfusepath{fill}%
\end{pgfscope}%
\begin{pgfscope}%
\pgfpathrectangle{\pgfqpoint{1.150000in}{0.150000in}}{\pgfqpoint{5.700000in}{5.700000in}}%
\pgfusepath{clip}%
\pgfsetbuttcap%
\pgfsetroundjoin%
\definecolor{currentfill}{rgb}{0.194100,0.399323,0.555565}%
\pgfsetfillcolor{currentfill}%
\pgfsetfillopacity{0.800000}%
\pgfsetlinewidth{0.000000pt}%
\definecolor{currentstroke}{rgb}{0.000000,0.000000,0.000000}%
\pgfsetstrokecolor{currentstroke}%
\pgfsetdash{}{0pt}%
\pgfpathmoveto{\pgfqpoint{4.605062in}{1.598787in}}%
\pgfpathlineto{\pgfqpoint{4.619900in}{1.609466in}}%
\pgfpathlineto{\pgfqpoint{4.634755in}{1.620324in}}%
\pgfpathlineto{\pgfqpoint{4.649627in}{1.631361in}}%
\pgfpathlineto{\pgfqpoint{4.664518in}{1.642575in}}%
\pgfpathlineto{\pgfqpoint{4.672857in}{1.664879in}}%
\pgfpathlineto{\pgfqpoint{4.681196in}{1.687224in}}%
\pgfpathlineto{\pgfqpoint{4.689532in}{1.709602in}}%
\pgfpathlineto{\pgfqpoint{4.697866in}{1.732008in}}%
\pgfpathlineto{\pgfqpoint{4.682958in}{1.720077in}}%
\pgfpathlineto{\pgfqpoint{4.668068in}{1.708325in}}%
\pgfpathlineto{\pgfqpoint{4.653196in}{1.696754in}}%
\pgfpathlineto{\pgfqpoint{4.638342in}{1.685362in}}%
\pgfpathlineto{\pgfqpoint{4.630025in}{1.663659in}}%
\pgfpathlineto{\pgfqpoint{4.621706in}{1.641991in}}%
\pgfpathlineto{\pgfqpoint{4.613385in}{1.620365in}}%
\pgfpathlineto{\pgfqpoint{4.605062in}{1.598787in}}%
\pgfpathclose%
\pgfusepath{fill}%
\end{pgfscope}%
\begin{pgfscope}%
\pgfpathrectangle{\pgfqpoint{1.150000in}{0.150000in}}{\pgfqpoint{5.700000in}{5.700000in}}%
\pgfusepath{clip}%
\pgfsetbuttcap%
\pgfsetroundjoin%
\definecolor{currentfill}{rgb}{0.311925,0.767822,0.415586}%
\pgfsetfillcolor{currentfill}%
\pgfsetfillopacity{0.800000}%
\pgfsetlinewidth{0.000000pt}%
\definecolor{currentstroke}{rgb}{0.000000,0.000000,0.000000}%
\pgfsetstrokecolor{currentstroke}%
\pgfsetdash{}{0pt}%
\pgfpathmoveto{\pgfqpoint{5.183164in}{2.807533in}}%
\pgfpathlineto{\pgfqpoint{5.198481in}{2.826919in}}%
\pgfpathlineto{\pgfqpoint{5.213824in}{2.846499in}}%
\pgfpathlineto{\pgfqpoint{5.229191in}{2.866276in}}%
\pgfpathlineto{\pgfqpoint{5.237390in}{2.885319in}}%
\pgfpathlineto{\pgfqpoint{5.245581in}{2.904149in}}%
\pgfpathlineto{\pgfqpoint{5.253764in}{2.922765in}}%
\pgfpathlineto{\pgfqpoint{5.261940in}{2.941162in}}%
\pgfpathlineto{\pgfqpoint{5.246556in}{2.921066in}}%
\pgfpathlineto{\pgfqpoint{5.231198in}{2.901167in}}%
\pgfpathlineto{\pgfqpoint{5.215864in}{2.881464in}}%
\pgfpathlineto{\pgfqpoint{5.207700in}{2.863295in}}%
\pgfpathlineto{\pgfqpoint{5.199529in}{2.844915in}}%
\pgfpathlineto{\pgfqpoint{5.191350in}{2.826327in}}%
\pgfpathlineto{\pgfqpoint{5.183164in}{2.807533in}}%
\pgfpathclose%
\pgfusepath{fill}%
\end{pgfscope}%
\begin{pgfscope}%
\pgfpathrectangle{\pgfqpoint{1.150000in}{0.150000in}}{\pgfqpoint{5.700000in}{5.700000in}}%
\pgfusepath{clip}%
\pgfsetbuttcap%
\pgfsetroundjoin%
\definecolor{currentfill}{rgb}{0.141935,0.526453,0.555991}%
\pgfsetfillcolor{currentfill}%
\pgfsetfillopacity{0.800000}%
\pgfsetlinewidth{0.000000pt}%
\definecolor{currentstroke}{rgb}{0.000000,0.000000,0.000000}%
\pgfsetstrokecolor{currentstroke}%
\pgfsetdash{}{0pt}%
\pgfpathmoveto{\pgfqpoint{4.797717in}{2.000720in}}%
\pgfpathlineto{\pgfqpoint{4.812703in}{2.014794in}}%
\pgfpathlineto{\pgfqpoint{4.827708in}{2.029052in}}%
\pgfpathlineto{\pgfqpoint{4.842733in}{2.043494in}}%
\pgfpathlineto{\pgfqpoint{4.857780in}{2.058122in}}%
\pgfpathlineto{\pgfqpoint{4.866105in}{2.080924in}}%
\pgfpathlineto{\pgfqpoint{4.874427in}{2.103663in}}%
\pgfpathlineto{\pgfqpoint{4.882746in}{2.126331in}}%
\pgfpathlineto{\pgfqpoint{4.891062in}{2.148924in}}%
\pgfpathlineto{\pgfqpoint{4.875995in}{2.133704in}}%
\pgfpathlineto{\pgfqpoint{4.860949in}{2.118669in}}%
\pgfpathlineto{\pgfqpoint{4.845923in}{2.103820in}}%
\pgfpathlineto{\pgfqpoint{4.830919in}{2.089156in}}%
\pgfpathlineto{\pgfqpoint{4.822623in}{2.067142in}}%
\pgfpathlineto{\pgfqpoint{4.814324in}{2.045061in}}%
\pgfpathlineto{\pgfqpoint{4.806022in}{2.022919in}}%
\pgfpathlineto{\pgfqpoint{4.797717in}{2.000720in}}%
\pgfpathclose%
\pgfusepath{fill}%
\end{pgfscope}%
\begin{pgfscope}%
\pgfpathrectangle{\pgfqpoint{1.150000in}{0.150000in}}{\pgfqpoint{5.700000in}{5.700000in}}%
\pgfusepath{clip}%
\pgfsetbuttcap%
\pgfsetroundjoin%
\definecolor{currentfill}{rgb}{0.252194,0.269783,0.531579}%
\pgfsetfillcolor{currentfill}%
\pgfsetfillopacity{0.800000}%
\pgfsetlinewidth{0.000000pt}%
\definecolor{currentstroke}{rgb}{0.000000,0.000000,0.000000}%
\pgfsetstrokecolor{currentstroke}%
\pgfsetdash{}{0pt}%
\pgfpathmoveto{\pgfqpoint{4.412697in}{1.237399in}}%
\pgfpathlineto{\pgfqpoint{4.427418in}{1.244231in}}%
\pgfpathlineto{\pgfqpoint{4.442154in}{1.251236in}}%
\pgfpathlineto{\pgfqpoint{4.456905in}{1.258415in}}%
\pgfpathlineto{\pgfqpoint{4.471671in}{1.265766in}}%
\pgfpathlineto{\pgfqpoint{4.480020in}{1.285613in}}%
\pgfpathlineto{\pgfqpoint{4.488367in}{1.305625in}}%
\pgfpathlineto{\pgfqpoint{4.496713in}{1.325793in}}%
\pgfpathlineto{\pgfqpoint{4.505058in}{1.346110in}}%
\pgfpathlineto{\pgfqpoint{4.490280in}{1.337926in}}%
\pgfpathlineto{\pgfqpoint{4.475518in}{1.329916in}}%
\pgfpathlineto{\pgfqpoint{4.460772in}{1.322080in}}%
\pgfpathlineto{\pgfqpoint{4.446041in}{1.314418in}}%
\pgfpathlineto{\pgfqpoint{4.437708in}{1.294921in}}%
\pgfpathlineto{\pgfqpoint{4.429373in}{1.275580in}}%
\pgfpathlineto{\pgfqpoint{4.421036in}{1.256404in}}%
\pgfpathlineto{\pgfqpoint{4.412697in}{1.237399in}}%
\pgfpathclose%
\pgfusepath{fill}%
\end{pgfscope}%
\begin{pgfscope}%
\pgfpathrectangle{\pgfqpoint{1.150000in}{0.150000in}}{\pgfqpoint{5.700000in}{5.700000in}}%
\pgfusepath{clip}%
\pgfsetbuttcap%
\pgfsetroundjoin%
\definecolor{currentfill}{rgb}{0.134692,0.658636,0.517649}%
\pgfsetfillcolor{currentfill}%
\pgfsetfillopacity{0.800000}%
\pgfsetlinewidth{0.000000pt}%
\definecolor{currentstroke}{rgb}{0.000000,0.000000,0.000000}%
\pgfsetstrokecolor{currentstroke}%
\pgfsetdash{}{0pt}%
\pgfpathmoveto{\pgfqpoint{4.990566in}{2.412346in}}%
\pgfpathlineto{\pgfqpoint{5.005715in}{2.429334in}}%
\pgfpathlineto{\pgfqpoint{5.020888in}{2.446513in}}%
\pgfpathlineto{\pgfqpoint{5.036083in}{2.463881in}}%
\pgfpathlineto{\pgfqpoint{5.051301in}{2.481441in}}%
\pgfpathlineto{\pgfqpoint{5.059585in}{2.503081in}}%
\pgfpathlineto{\pgfqpoint{5.067864in}{2.524571in}}%
\pgfpathlineto{\pgfqpoint{5.076139in}{2.545908in}}%
\pgfpathlineto{\pgfqpoint{5.084407in}{2.567086in}}%
\pgfpathlineto{\pgfqpoint{5.069169in}{2.549065in}}%
\pgfpathlineto{\pgfqpoint{5.053954in}{2.531236in}}%
\pgfpathlineto{\pgfqpoint{5.038762in}{2.513598in}}%
\pgfpathlineto{\pgfqpoint{5.023593in}{2.496151in}}%
\pgfpathlineto{\pgfqpoint{5.015343in}{2.475420in}}%
\pgfpathlineto{\pgfqpoint{5.007089in}{2.454539in}}%
\pgfpathlineto{\pgfqpoint{4.998830in}{2.433513in}}%
\pgfpathlineto{\pgfqpoint{4.990566in}{2.412346in}}%
\pgfpathclose%
\pgfusepath{fill}%
\end{pgfscope}%
\begin{pgfscope}%
\pgfpathrectangle{\pgfqpoint{1.150000in}{0.150000in}}{\pgfqpoint{5.700000in}{5.700000in}}%
\pgfusepath{clip}%
\pgfsetbuttcap%
\pgfsetroundjoin%
\definecolor{currentfill}{rgb}{0.206756,0.371758,0.553117}%
\pgfsetfillcolor{currentfill}%
\pgfsetfillopacity{0.800000}%
\pgfsetlinewidth{0.000000pt}%
\definecolor{currentstroke}{rgb}{0.000000,0.000000,0.000000}%
\pgfsetstrokecolor{currentstroke}%
\pgfsetdash{}{0pt}%
\pgfpathmoveto{\pgfqpoint{4.571754in}{1.513090in}}%
\pgfpathlineto{\pgfqpoint{4.586576in}{1.523026in}}%
\pgfpathlineto{\pgfqpoint{4.601414in}{1.533140in}}%
\pgfpathlineto{\pgfqpoint{4.616270in}{1.543430in}}%
\pgfpathlineto{\pgfqpoint{4.631143in}{1.553898in}}%
\pgfpathlineto{\pgfqpoint{4.639489in}{1.575973in}}%
\pgfpathlineto{\pgfqpoint{4.647833in}{1.598115in}}%
\pgfpathlineto{\pgfqpoint{4.656176in}{1.620318in}}%
\pgfpathlineto{\pgfqpoint{4.664518in}{1.642575in}}%
\pgfpathlineto{\pgfqpoint{4.649627in}{1.631361in}}%
\pgfpathlineto{\pgfqpoint{4.634755in}{1.620324in}}%
\pgfpathlineto{\pgfqpoint{4.619900in}{1.609466in}}%
\pgfpathlineto{\pgfqpoint{4.605062in}{1.598787in}}%
\pgfpathlineto{\pgfqpoint{4.596738in}{1.577263in}}%
\pgfpathlineto{\pgfqpoint{4.588412in}{1.555801in}}%
\pgfpathlineto{\pgfqpoint{4.580084in}{1.534408in}}%
\pgfpathlineto{\pgfqpoint{4.571754in}{1.513090in}}%
\pgfpathclose%
\pgfusepath{fill}%
\end{pgfscope}%
\begin{pgfscope}%
\pgfpathrectangle{\pgfqpoint{1.150000in}{0.150000in}}{\pgfqpoint{5.700000in}{5.700000in}}%
\pgfusepath{clip}%
\pgfsetbuttcap%
\pgfsetroundjoin%
\definecolor{currentfill}{rgb}{0.150476,0.504369,0.557430}%
\pgfsetfillcolor{currentfill}%
\pgfsetfillopacity{0.800000}%
\pgfsetlinewidth{0.000000pt}%
\definecolor{currentstroke}{rgb}{0.000000,0.000000,0.000000}%
\pgfsetstrokecolor{currentstroke}%
\pgfsetdash{}{0pt}%
\pgfpathmoveto{\pgfqpoint{4.764471in}{1.911472in}}%
\pgfpathlineto{\pgfqpoint{4.779437in}{1.924923in}}%
\pgfpathlineto{\pgfqpoint{4.794422in}{1.938558in}}%
\pgfpathlineto{\pgfqpoint{4.809427in}{1.952376in}}%
\pgfpathlineto{\pgfqpoint{4.824452in}{1.966377in}}%
\pgfpathlineto{\pgfqpoint{4.832788in}{1.989382in}}%
\pgfpathlineto{\pgfqpoint{4.841121in}{2.012345in}}%
\pgfpathlineto{\pgfqpoint{4.849452in}{2.035260in}}%
\pgfpathlineto{\pgfqpoint{4.857780in}{2.058122in}}%
\pgfpathlineto{\pgfqpoint{4.842733in}{2.043494in}}%
\pgfpathlineto{\pgfqpoint{4.827708in}{2.029052in}}%
\pgfpathlineto{\pgfqpoint{4.812703in}{2.014794in}}%
\pgfpathlineto{\pgfqpoint{4.797717in}{2.000720in}}%
\pgfpathlineto{\pgfqpoint{4.789410in}{1.978470in}}%
\pgfpathlineto{\pgfqpoint{4.781099in}{1.956175in}}%
\pgfpathlineto{\pgfqpoint{4.772787in}{1.933840in}}%
\pgfpathlineto{\pgfqpoint{4.764471in}{1.911472in}}%
\pgfpathclose%
\pgfusepath{fill}%
\end{pgfscope}%
\begin{pgfscope}%
\pgfpathrectangle{\pgfqpoint{1.150000in}{0.150000in}}{\pgfqpoint{5.700000in}{5.700000in}}%
\pgfusepath{clip}%
\pgfsetbuttcap%
\pgfsetroundjoin%
\definecolor{currentfill}{rgb}{0.266941,0.748751,0.440573}%
\pgfsetfillcolor{currentfill}%
\pgfsetfillopacity{0.800000}%
\pgfsetlinewidth{0.000000pt}%
\definecolor{currentstroke}{rgb}{0.000000,0.000000,0.000000}%
\pgfsetstrokecolor{currentstroke}%
\pgfsetdash{}{0pt}%
\pgfpathmoveto{\pgfqpoint{5.150348in}{2.730360in}}%
\pgfpathlineto{\pgfqpoint{5.165648in}{2.749391in}}%
\pgfpathlineto{\pgfqpoint{5.180973in}{2.768616in}}%
\pgfpathlineto{\pgfqpoint{5.196322in}{2.788036in}}%
\pgfpathlineto{\pgfqpoint{5.204550in}{2.807900in}}%
\pgfpathlineto{\pgfqpoint{5.212771in}{2.827563in}}%
\pgfpathlineto{\pgfqpoint{5.220984in}{2.847023in}}%
\pgfpathlineto{\pgfqpoint{5.229191in}{2.866276in}}%
\pgfpathlineto{\pgfqpoint{5.213824in}{2.846499in}}%
\pgfpathlineto{\pgfqpoint{5.198481in}{2.826919in}}%
\pgfpathlineto{\pgfqpoint{5.183164in}{2.807533in}}%
\pgfpathlineto{\pgfqpoint{5.174970in}{2.788537in}}%
\pgfpathlineto{\pgfqpoint{5.166770in}{2.769341in}}%
\pgfpathlineto{\pgfqpoint{5.158562in}{2.749947in}}%
\pgfpathlineto{\pgfqpoint{5.150348in}{2.730360in}}%
\pgfpathclose%
\pgfusepath{fill}%
\end{pgfscope}%
\begin{pgfscope}%
\pgfpathrectangle{\pgfqpoint{1.150000in}{0.150000in}}{\pgfqpoint{5.700000in}{5.700000in}}%
\pgfusepath{clip}%
\pgfsetbuttcap%
\pgfsetroundjoin%
\definecolor{currentfill}{rgb}{0.123444,0.636809,0.528763}%
\pgfsetfillcolor{currentfill}%
\pgfsetfillopacity{0.800000}%
\pgfsetlinewidth{0.000000pt}%
\definecolor{currentstroke}{rgb}{0.000000,0.000000,0.000000}%
\pgfsetstrokecolor{currentstroke}%
\pgfsetdash{}{0pt}%
\pgfpathmoveto{\pgfqpoint{4.957463in}{2.326349in}}%
\pgfpathlineto{\pgfqpoint{4.972592in}{2.342844in}}%
\pgfpathlineto{\pgfqpoint{4.987744in}{2.359528in}}%
\pgfpathlineto{\pgfqpoint{5.002918in}{2.376401in}}%
\pgfpathlineto{\pgfqpoint{5.018115in}{2.393464in}}%
\pgfpathlineto{\pgfqpoint{5.026418in}{2.415662in}}%
\pgfpathlineto{\pgfqpoint{5.034717in}{2.437727in}}%
\pgfpathlineto{\pgfqpoint{5.043011in}{2.459654in}}%
\pgfpathlineto{\pgfqpoint{5.051301in}{2.481441in}}%
\pgfpathlineto{\pgfqpoint{5.036083in}{2.463881in}}%
\pgfpathlineto{\pgfqpoint{5.020888in}{2.446513in}}%
\pgfpathlineto{\pgfqpoint{5.005715in}{2.429334in}}%
\pgfpathlineto{\pgfqpoint{4.990566in}{2.412346in}}%
\pgfpathlineto{\pgfqpoint{4.982297in}{2.391041in}}%
\pgfpathlineto{\pgfqpoint{4.974023in}{2.369604in}}%
\pgfpathlineto{\pgfqpoint{4.965745in}{2.348038in}}%
\pgfpathlineto{\pgfqpoint{4.957463in}{2.326349in}}%
\pgfpathclose%
\pgfusepath{fill}%
\end{pgfscope}%
\begin{pgfscope}%
\pgfpathrectangle{\pgfqpoint{1.150000in}{0.150000in}}{\pgfqpoint{5.700000in}{5.700000in}}%
\pgfusepath{clip}%
\pgfsetbuttcap%
\pgfsetroundjoin%
\definecolor{currentfill}{rgb}{0.218130,0.347432,0.550038}%
\pgfsetfillcolor{currentfill}%
\pgfsetfillopacity{0.800000}%
\pgfsetlinewidth{0.000000pt}%
\definecolor{currentstroke}{rgb}{0.000000,0.000000,0.000000}%
\pgfsetstrokecolor{currentstroke}%
\pgfsetdash{}{0pt}%
\pgfpathmoveto{\pgfqpoint{4.538419in}{1.428711in}}%
\pgfpathlineto{\pgfqpoint{4.553225in}{1.437874in}}%
\pgfpathlineto{\pgfqpoint{4.568049in}{1.447212in}}%
\pgfpathlineto{\pgfqpoint{4.582889in}{1.456726in}}%
\pgfpathlineto{\pgfqpoint{4.597745in}{1.466416in}}%
\pgfpathlineto{\pgfqpoint{4.606097in}{1.488150in}}%
\pgfpathlineto{\pgfqpoint{4.614447in}{1.509979in}}%
\pgfpathlineto{\pgfqpoint{4.622796in}{1.531898in}}%
\pgfpathlineto{\pgfqpoint{4.631143in}{1.553898in}}%
\pgfpathlineto{\pgfqpoint{4.616270in}{1.543430in}}%
\pgfpathlineto{\pgfqpoint{4.601414in}{1.533140in}}%
\pgfpathlineto{\pgfqpoint{4.586576in}{1.523026in}}%
\pgfpathlineto{\pgfqpoint{4.571754in}{1.513090in}}%
\pgfpathlineto{\pgfqpoint{4.563423in}{1.491854in}}%
\pgfpathlineto{\pgfqpoint{4.555090in}{1.470708in}}%
\pgfpathlineto{\pgfqpoint{4.546755in}{1.449658in}}%
\pgfpathlineto{\pgfqpoint{4.538419in}{1.428711in}}%
\pgfpathclose%
\pgfusepath{fill}%
\end{pgfscope}%
\begin{pgfscope}%
\pgfpathrectangle{\pgfqpoint{1.150000in}{0.150000in}}{\pgfqpoint{5.700000in}{5.700000in}}%
\pgfusepath{clip}%
\pgfsetbuttcap%
\pgfsetroundjoin%
\definecolor{currentfill}{rgb}{0.160665,0.478540,0.558115}%
\pgfsetfillcolor{currentfill}%
\pgfsetfillopacity{0.800000}%
\pgfsetlinewidth{0.000000pt}%
\definecolor{currentstroke}{rgb}{0.000000,0.000000,0.000000}%
\pgfsetstrokecolor{currentstroke}%
\pgfsetdash{}{0pt}%
\pgfpathmoveto{\pgfqpoint{4.731186in}{1.821774in}}%
\pgfpathlineto{\pgfqpoint{4.746132in}{1.834571in}}%
\pgfpathlineto{\pgfqpoint{4.761097in}{1.847550in}}%
\pgfpathlineto{\pgfqpoint{4.776082in}{1.860710in}}%
\pgfpathlineto{\pgfqpoint{4.791086in}{1.874053in}}%
\pgfpathlineto{\pgfqpoint{4.799431in}{1.897168in}}%
\pgfpathlineto{\pgfqpoint{4.807774in}{1.920264in}}%
\pgfpathlineto{\pgfqpoint{4.816114in}{1.943335in}}%
\pgfpathlineto{\pgfqpoint{4.824452in}{1.966377in}}%
\pgfpathlineto{\pgfqpoint{4.809427in}{1.952376in}}%
\pgfpathlineto{\pgfqpoint{4.794422in}{1.938558in}}%
\pgfpathlineto{\pgfqpoint{4.779437in}{1.924923in}}%
\pgfpathlineto{\pgfqpoint{4.764471in}{1.911472in}}%
\pgfpathlineto{\pgfqpoint{4.756153in}{1.889075in}}%
\pgfpathlineto{\pgfqpoint{4.747833in}{1.866655in}}%
\pgfpathlineto{\pgfqpoint{4.739510in}{1.844220in}}%
\pgfpathlineto{\pgfqpoint{4.731186in}{1.821774in}}%
\pgfpathclose%
\pgfusepath{fill}%
\end{pgfscope}%
\begin{pgfscope}%
\pgfpathrectangle{\pgfqpoint{1.150000in}{0.150000in}}{\pgfqpoint{5.700000in}{5.700000in}}%
\pgfusepath{clip}%
\pgfsetbuttcap%
\pgfsetroundjoin%
\definecolor{currentfill}{rgb}{0.232815,0.732247,0.459277}%
\pgfsetfillcolor{currentfill}%
\pgfsetfillopacity{0.800000}%
\pgfsetlinewidth{0.000000pt}%
\definecolor{currentstroke}{rgb}{0.000000,0.000000,0.000000}%
\pgfsetstrokecolor{currentstroke}%
\pgfsetdash{}{0pt}%
\pgfpathmoveto{\pgfqpoint{5.117426in}{2.650139in}}%
\pgfpathlineto{\pgfqpoint{5.132708in}{2.668779in}}%
\pgfpathlineto{\pgfqpoint{5.148013in}{2.687612in}}%
\pgfpathlineto{\pgfqpoint{5.163343in}{2.706639in}}%
\pgfpathlineto{\pgfqpoint{5.171597in}{2.727273in}}%
\pgfpathlineto{\pgfqpoint{5.179845in}{2.747719in}}%
\pgfpathlineto{\pgfqpoint{5.188087in}{2.767975in}}%
\pgfpathlineto{\pgfqpoint{5.196322in}{2.788036in}}%
\pgfpathlineto{\pgfqpoint{5.180973in}{2.768616in}}%
\pgfpathlineto{\pgfqpoint{5.165648in}{2.749391in}}%
\pgfpathlineto{\pgfqpoint{5.150348in}{2.730360in}}%
\pgfpathlineto{\pgfqpoint{5.142127in}{2.710583in}}%
\pgfpathlineto{\pgfqpoint{5.133900in}{2.690618in}}%
\pgfpathlineto{\pgfqpoint{5.125666in}{2.670469in}}%
\pgfpathlineto{\pgfqpoint{5.117426in}{2.650139in}}%
\pgfpathclose%
\pgfusepath{fill}%
\end{pgfscope}%
\begin{pgfscope}%
\pgfpathrectangle{\pgfqpoint{1.150000in}{0.150000in}}{\pgfqpoint{5.700000in}{5.700000in}}%
\pgfusepath{clip}%
\pgfsetbuttcap%
\pgfsetroundjoin%
\definecolor{currentfill}{rgb}{0.119423,0.611141,0.538982}%
\pgfsetfillcolor{currentfill}%
\pgfsetfillopacity{0.800000}%
\pgfsetlinewidth{0.000000pt}%
\definecolor{currentstroke}{rgb}{0.000000,0.000000,0.000000}%
\pgfsetstrokecolor{currentstroke}%
\pgfsetdash{}{0pt}%
\pgfpathmoveto{\pgfqpoint{4.924292in}{2.238441in}}%
\pgfpathlineto{\pgfqpoint{4.939401in}{2.254409in}}%
\pgfpathlineto{\pgfqpoint{4.954532in}{2.270564in}}%
\pgfpathlineto{\pgfqpoint{4.969685in}{2.286908in}}%
\pgfpathlineto{\pgfqpoint{4.984859in}{2.303439in}}%
\pgfpathlineto{\pgfqpoint{4.993179in}{2.326121in}}%
\pgfpathlineto{\pgfqpoint{5.001495in}{2.348689in}}%
\pgfpathlineto{\pgfqpoint{5.009807in}{2.371138in}}%
\pgfpathlineto{\pgfqpoint{5.018115in}{2.393464in}}%
\pgfpathlineto{\pgfqpoint{5.002918in}{2.376401in}}%
\pgfpathlineto{\pgfqpoint{4.987744in}{2.359528in}}%
\pgfpathlineto{\pgfqpoint{4.972592in}{2.342844in}}%
\pgfpathlineto{\pgfqpoint{4.957463in}{2.326349in}}%
\pgfpathlineto{\pgfqpoint{4.949176in}{2.304539in}}%
\pgfpathlineto{\pgfqpoint{4.940885in}{2.282615in}}%
\pgfpathlineto{\pgfqpoint{4.932591in}{2.260581in}}%
\pgfpathlineto{\pgfqpoint{4.924292in}{2.238441in}}%
\pgfpathclose%
\pgfusepath{fill}%
\end{pgfscope}%
\begin{pgfscope}%
\pgfpathrectangle{\pgfqpoint{1.150000in}{0.150000in}}{\pgfqpoint{5.700000in}{5.700000in}}%
\pgfusepath{clip}%
\pgfsetbuttcap%
\pgfsetroundjoin%
\definecolor{currentfill}{rgb}{0.171176,0.452530,0.557965}%
\pgfsetfillcolor{currentfill}%
\pgfsetfillopacity{0.800000}%
\pgfsetlinewidth{0.000000pt}%
\definecolor{currentstroke}{rgb}{0.000000,0.000000,0.000000}%
\pgfsetstrokecolor{currentstroke}%
\pgfsetdash{}{0pt}%
\pgfpathmoveto{\pgfqpoint{4.697866in}{1.732008in}}%
\pgfpathlineto{\pgfqpoint{4.712794in}{1.744119in}}%
\pgfpathlineto{\pgfqpoint{4.727739in}{1.756410in}}%
\pgfpathlineto{\pgfqpoint{4.742704in}{1.768882in}}%
\pgfpathlineto{\pgfqpoint{4.757688in}{1.781534in}}%
\pgfpathlineto{\pgfqpoint{4.766040in}{1.804661in}}%
\pgfpathlineto{\pgfqpoint{4.774391in}{1.827793in}}%
\pgfpathlineto{\pgfqpoint{4.782739in}{1.850926in}}%
\pgfpathlineto{\pgfqpoint{4.791086in}{1.874053in}}%
\pgfpathlineto{\pgfqpoint{4.776082in}{1.860710in}}%
\pgfpathlineto{\pgfqpoint{4.761097in}{1.847550in}}%
\pgfpathlineto{\pgfqpoint{4.746132in}{1.834571in}}%
\pgfpathlineto{\pgfqpoint{4.731186in}{1.821774in}}%
\pgfpathlineto{\pgfqpoint{4.722859in}{1.799323in}}%
\pgfpathlineto{\pgfqpoint{4.714530in}{1.776875in}}%
\pgfpathlineto{\pgfqpoint{4.706199in}{1.754434in}}%
\pgfpathlineto{\pgfqpoint{4.697866in}{1.732008in}}%
\pgfpathclose%
\pgfusepath{fill}%
\end{pgfscope}%
\begin{pgfscope}%
\pgfpathrectangle{\pgfqpoint{1.150000in}{0.150000in}}{\pgfqpoint{5.700000in}{5.700000in}}%
\pgfusepath{clip}%
\pgfsetbuttcap%
\pgfsetroundjoin%
\definecolor{currentfill}{rgb}{0.229739,0.322361,0.545706}%
\pgfsetfillcolor{currentfill}%
\pgfsetfillopacity{0.800000}%
\pgfsetlinewidth{0.000000pt}%
\definecolor{currentstroke}{rgb}{0.000000,0.000000,0.000000}%
\pgfsetstrokecolor{currentstroke}%
\pgfsetdash{}{0pt}%
\pgfpathmoveto{\pgfqpoint{4.505058in}{1.346110in}}%
\pgfpathlineto{\pgfqpoint{4.519851in}{1.354468in}}%
\pgfpathlineto{\pgfqpoint{4.534660in}{1.363001in}}%
\pgfpathlineto{\pgfqpoint{4.549485in}{1.371709in}}%
\pgfpathlineto{\pgfqpoint{4.564326in}{1.380591in}}%
\pgfpathlineto{\pgfqpoint{4.572683in}{1.401866in}}%
\pgfpathlineto{\pgfqpoint{4.581038in}{1.423267in}}%
\pgfpathlineto{\pgfqpoint{4.589393in}{1.444786in}}%
\pgfpathlineto{\pgfqpoint{4.597745in}{1.466416in}}%
\pgfpathlineto{\pgfqpoint{4.582889in}{1.456726in}}%
\pgfpathlineto{\pgfqpoint{4.568049in}{1.447212in}}%
\pgfpathlineto{\pgfqpoint{4.553225in}{1.437874in}}%
\pgfpathlineto{\pgfqpoint{4.538419in}{1.428711in}}%
\pgfpathlineto{\pgfqpoint{4.530081in}{1.407876in}}%
\pgfpathlineto{\pgfqpoint{4.521741in}{1.387159in}}%
\pgfpathlineto{\pgfqpoint{4.513400in}{1.366568in}}%
\pgfpathlineto{\pgfqpoint{4.505058in}{1.346110in}}%
\pgfpathclose%
\pgfusepath{fill}%
\end{pgfscope}%
\begin{pgfscope}%
\pgfpathrectangle{\pgfqpoint{1.150000in}{0.150000in}}{\pgfqpoint{5.700000in}{5.700000in}}%
\pgfusepath{clip}%
\pgfsetbuttcap%
\pgfsetroundjoin%
\definecolor{currentfill}{rgb}{0.122606,0.585371,0.546557}%
\pgfsetfillcolor{currentfill}%
\pgfsetfillopacity{0.800000}%
\pgfsetlinewidth{0.000000pt}%
\definecolor{currentstroke}{rgb}{0.000000,0.000000,0.000000}%
\pgfsetstrokecolor{currentstroke}%
\pgfsetdash{}{0pt}%
\pgfpathmoveto{\pgfqpoint{4.891062in}{2.148924in}}%
\pgfpathlineto{\pgfqpoint{4.906150in}{2.164331in}}%
\pgfpathlineto{\pgfqpoint{4.921259in}{2.179924in}}%
\pgfpathlineto{\pgfqpoint{4.936390in}{2.195703in}}%
\pgfpathlineto{\pgfqpoint{4.951542in}{2.211670in}}%
\pgfpathlineto{\pgfqpoint{4.959877in}{2.234759in}}%
\pgfpathlineto{\pgfqpoint{4.968208in}{2.257753in}}%
\pgfpathlineto{\pgfqpoint{4.976536in}{2.280648in}}%
\pgfpathlineto{\pgfqpoint{4.984859in}{2.303439in}}%
\pgfpathlineto{\pgfqpoint{4.969685in}{2.286908in}}%
\pgfpathlineto{\pgfqpoint{4.954532in}{2.270564in}}%
\pgfpathlineto{\pgfqpoint{4.939401in}{2.254409in}}%
\pgfpathlineto{\pgfqpoint{4.924292in}{2.238441in}}%
\pgfpathlineto{\pgfqpoint{4.915990in}{2.216200in}}%
\pgfpathlineto{\pgfqpoint{4.907684in}{2.193864in}}%
\pgfpathlineto{\pgfqpoint{4.899375in}{2.171437in}}%
\pgfpathlineto{\pgfqpoint{4.891062in}{2.148924in}}%
\pgfpathclose%
\pgfusepath{fill}%
\end{pgfscope}%
\begin{pgfscope}%
\pgfpathrectangle{\pgfqpoint{1.150000in}{0.150000in}}{\pgfqpoint{5.700000in}{5.700000in}}%
\pgfusepath{clip}%
\pgfsetbuttcap%
\pgfsetroundjoin%
\definecolor{currentfill}{rgb}{0.182256,0.426184,0.557120}%
\pgfsetfillcolor{currentfill}%
\pgfsetfillopacity{0.800000}%
\pgfsetlinewidth{0.000000pt}%
\definecolor{currentstroke}{rgb}{0.000000,0.000000,0.000000}%
\pgfsetstrokecolor{currentstroke}%
\pgfsetdash{}{0pt}%
\pgfpathmoveto{\pgfqpoint{4.664518in}{1.642575in}}%
\pgfpathlineto{\pgfqpoint{4.679426in}{1.653969in}}%
\pgfpathlineto{\pgfqpoint{4.694353in}{1.665541in}}%
\pgfpathlineto{\pgfqpoint{4.709298in}{1.677293in}}%
\pgfpathlineto{\pgfqpoint{4.724262in}{1.689224in}}%
\pgfpathlineto{\pgfqpoint{4.732620in}{1.712259in}}%
\pgfpathlineto{\pgfqpoint{4.740978in}{1.735327in}}%
\pgfpathlineto{\pgfqpoint{4.749334in}{1.758421in}}%
\pgfpathlineto{\pgfqpoint{4.757688in}{1.781534in}}%
\pgfpathlineto{\pgfqpoint{4.742704in}{1.768882in}}%
\pgfpathlineto{\pgfqpoint{4.727739in}{1.756410in}}%
\pgfpathlineto{\pgfqpoint{4.712794in}{1.744119in}}%
\pgfpathlineto{\pgfqpoint{4.697866in}{1.732008in}}%
\pgfpathlineto{\pgfqpoint{4.689532in}{1.709602in}}%
\pgfpathlineto{\pgfqpoint{4.681196in}{1.687224in}}%
\pgfpathlineto{\pgfqpoint{4.672857in}{1.664879in}}%
\pgfpathlineto{\pgfqpoint{4.664518in}{1.642575in}}%
\pgfpathclose%
\pgfusepath{fill}%
\end{pgfscope}%
\begin{pgfscope}%
\pgfpathrectangle{\pgfqpoint{1.150000in}{0.150000in}}{\pgfqpoint{5.700000in}{5.700000in}}%
\pgfusepath{clip}%
\pgfsetbuttcap%
\pgfsetroundjoin%
\definecolor{currentfill}{rgb}{0.196571,0.711827,0.479221}%
\pgfsetfillcolor{currentfill}%
\pgfsetfillopacity{0.800000}%
\pgfsetlinewidth{0.000000pt}%
\definecolor{currentstroke}{rgb}{0.000000,0.000000,0.000000}%
\pgfsetstrokecolor{currentstroke}%
\pgfsetdash{}{0pt}%
\pgfpathmoveto{\pgfqpoint{5.084407in}{2.567086in}}%
\pgfpathlineto{\pgfqpoint{5.099669in}{2.585299in}}%
\pgfpathlineto{\pgfqpoint{5.114954in}{2.603704in}}%
\pgfpathlineto{\pgfqpoint{5.130263in}{2.622302in}}%
\pgfpathlineto{\pgfqpoint{5.138542in}{2.643649in}}%
\pgfpathlineto{\pgfqpoint{5.146815in}{2.664823in}}%
\pgfpathlineto{\pgfqpoint{5.155082in}{2.685821in}}%
\pgfpathlineto{\pgfqpoint{5.163343in}{2.706639in}}%
\pgfpathlineto{\pgfqpoint{5.148013in}{2.687612in}}%
\pgfpathlineto{\pgfqpoint{5.132708in}{2.668779in}}%
\pgfpathlineto{\pgfqpoint{5.117426in}{2.650139in}}%
\pgfpathlineto{\pgfqpoint{5.109180in}{2.629632in}}%
\pgfpathlineto{\pgfqpoint{5.100928in}{2.608952in}}%
\pgfpathlineto{\pgfqpoint{5.092671in}{2.588102in}}%
\pgfpathlineto{\pgfqpoint{5.084407in}{2.567086in}}%
\pgfpathclose%
\pgfusepath{fill}%
\end{pgfscope}%
\begin{pgfscope}%
\pgfpathrectangle{\pgfqpoint{1.150000in}{0.150000in}}{\pgfqpoint{5.700000in}{5.700000in}}%
\pgfusepath{clip}%
\pgfsetbuttcap%
\pgfsetroundjoin%
\definecolor{currentfill}{rgb}{0.241237,0.296485,0.539709}%
\pgfsetfillcolor{currentfill}%
\pgfsetfillopacity{0.800000}%
\pgfsetlinewidth{0.000000pt}%
\definecolor{currentstroke}{rgb}{0.000000,0.000000,0.000000}%
\pgfsetstrokecolor{currentstroke}%
\pgfsetdash{}{0pt}%
\pgfpathmoveto{\pgfqpoint{4.471671in}{1.265766in}}%
\pgfpathlineto{\pgfqpoint{4.486452in}{1.273290in}}%
\pgfpathlineto{\pgfqpoint{4.501248in}{1.280988in}}%
\pgfpathlineto{\pgfqpoint{4.516060in}{1.288859in}}%
\pgfpathlineto{\pgfqpoint{4.530887in}{1.296903in}}%
\pgfpathlineto{\pgfqpoint{4.539249in}{1.317598in}}%
\pgfpathlineto{\pgfqpoint{4.547609in}{1.338449in}}%
\pgfpathlineto{\pgfqpoint{4.555969in}{1.359450in}}%
\pgfpathlineto{\pgfqpoint{4.564326in}{1.380591in}}%
\pgfpathlineto{\pgfqpoint{4.549485in}{1.371709in}}%
\pgfpathlineto{\pgfqpoint{4.534660in}{1.363001in}}%
\pgfpathlineto{\pgfqpoint{4.519851in}{1.354468in}}%
\pgfpathlineto{\pgfqpoint{4.505058in}{1.346110in}}%
\pgfpathlineto{\pgfqpoint{4.496713in}{1.325793in}}%
\pgfpathlineto{\pgfqpoint{4.488367in}{1.305625in}}%
\pgfpathlineto{\pgfqpoint{4.480020in}{1.285613in}}%
\pgfpathlineto{\pgfqpoint{4.471671in}{1.265766in}}%
\pgfpathclose%
\pgfusepath{fill}%
\end{pgfscope}%
\begin{pgfscope}%
\pgfpathrectangle{\pgfqpoint{1.150000in}{0.150000in}}{\pgfqpoint{5.700000in}{5.700000in}}%
\pgfusepath{clip}%
\pgfsetbuttcap%
\pgfsetroundjoin%
\definecolor{currentfill}{rgb}{0.129933,0.559582,0.551864}%
\pgfsetfillcolor{currentfill}%
\pgfsetfillopacity{0.800000}%
\pgfsetlinewidth{0.000000pt}%
\definecolor{currentstroke}{rgb}{0.000000,0.000000,0.000000}%
\pgfsetstrokecolor{currentstroke}%
\pgfsetdash{}{0pt}%
\pgfpathmoveto{\pgfqpoint{4.857780in}{2.058122in}}%
\pgfpathlineto{\pgfqpoint{4.872846in}{2.072934in}}%
\pgfpathlineto{\pgfqpoint{4.887934in}{2.087931in}}%
\pgfpathlineto{\pgfqpoint{4.903042in}{2.103113in}}%
\pgfpathlineto{\pgfqpoint{4.918172in}{2.118482in}}%
\pgfpathlineto{\pgfqpoint{4.926519in}{2.141894in}}%
\pgfpathlineto{\pgfqpoint{4.934863in}{2.165232in}}%
\pgfpathlineto{\pgfqpoint{4.943205in}{2.188493in}}%
\pgfpathlineto{\pgfqpoint{4.951542in}{2.211670in}}%
\pgfpathlineto{\pgfqpoint{4.936390in}{2.195703in}}%
\pgfpathlineto{\pgfqpoint{4.921259in}{2.179924in}}%
\pgfpathlineto{\pgfqpoint{4.906150in}{2.164331in}}%
\pgfpathlineto{\pgfqpoint{4.891062in}{2.148924in}}%
\pgfpathlineto{\pgfqpoint{4.882746in}{2.126331in}}%
\pgfpathlineto{\pgfqpoint{4.874427in}{2.103663in}}%
\pgfpathlineto{\pgfqpoint{4.866105in}{2.080924in}}%
\pgfpathlineto{\pgfqpoint{4.857780in}{2.058122in}}%
\pgfpathclose%
\pgfusepath{fill}%
\end{pgfscope}%
\begin{pgfscope}%
\pgfpathrectangle{\pgfqpoint{1.150000in}{0.150000in}}{\pgfqpoint{5.700000in}{5.700000in}}%
\pgfusepath{clip}%
\pgfsetbuttcap%
\pgfsetroundjoin%
\definecolor{currentfill}{rgb}{0.194100,0.399323,0.555565}%
\pgfsetfillcolor{currentfill}%
\pgfsetfillopacity{0.800000}%
\pgfsetlinewidth{0.000000pt}%
\definecolor{currentstroke}{rgb}{0.000000,0.000000,0.000000}%
\pgfsetstrokecolor{currentstroke}%
\pgfsetdash{}{0pt}%
\pgfpathmoveto{\pgfqpoint{4.631143in}{1.553898in}}%
\pgfpathlineto{\pgfqpoint{4.646034in}{1.564543in}}%
\pgfpathlineto{\pgfqpoint{4.660942in}{1.575366in}}%
\pgfpathlineto{\pgfqpoint{4.675868in}{1.586366in}}%
\pgfpathlineto{\pgfqpoint{4.690812in}{1.597545in}}%
\pgfpathlineto{\pgfqpoint{4.699177in}{1.620382in}}%
\pgfpathlineto{\pgfqpoint{4.707540in}{1.643279in}}%
\pgfpathlineto{\pgfqpoint{4.715901in}{1.666228in}}%
\pgfpathlineto{\pgfqpoint{4.724262in}{1.689224in}}%
\pgfpathlineto{\pgfqpoint{4.709298in}{1.677293in}}%
\pgfpathlineto{\pgfqpoint{4.694353in}{1.665541in}}%
\pgfpathlineto{\pgfqpoint{4.679426in}{1.653969in}}%
\pgfpathlineto{\pgfqpoint{4.664518in}{1.642575in}}%
\pgfpathlineto{\pgfqpoint{4.656176in}{1.620318in}}%
\pgfpathlineto{\pgfqpoint{4.647833in}{1.598115in}}%
\pgfpathlineto{\pgfqpoint{4.639489in}{1.575973in}}%
\pgfpathlineto{\pgfqpoint{4.631143in}{1.553898in}}%
\pgfpathclose%
\pgfusepath{fill}%
\end{pgfscope}%
\begin{pgfscope}%
\pgfpathrectangle{\pgfqpoint{1.150000in}{0.150000in}}{\pgfqpoint{5.700000in}{5.700000in}}%
\pgfusepath{clip}%
\pgfsetbuttcap%
\pgfsetroundjoin%
\definecolor{currentfill}{rgb}{0.166383,0.690856,0.496502}%
\pgfsetfillcolor{currentfill}%
\pgfsetfillopacity{0.800000}%
\pgfsetlinewidth{0.000000pt}%
\definecolor{currentstroke}{rgb}{0.000000,0.000000,0.000000}%
\pgfsetstrokecolor{currentstroke}%
\pgfsetdash{}{0pt}%
\pgfpathmoveto{\pgfqpoint{5.051301in}{2.481441in}}%
\pgfpathlineto{\pgfqpoint{5.066541in}{2.499191in}}%
\pgfpathlineto{\pgfqpoint{5.081805in}{2.517132in}}%
\pgfpathlineto{\pgfqpoint{5.097093in}{2.535266in}}%
\pgfpathlineto{\pgfqpoint{5.105393in}{2.557264in}}%
\pgfpathlineto{\pgfqpoint{5.113689in}{2.579105in}}%
\pgfpathlineto{\pgfqpoint{5.121979in}{2.600786in}}%
\pgfpathlineto{\pgfqpoint{5.130263in}{2.622302in}}%
\pgfpathlineto{\pgfqpoint{5.114954in}{2.603704in}}%
\pgfpathlineto{\pgfqpoint{5.099669in}{2.585299in}}%
\pgfpathlineto{\pgfqpoint{5.084407in}{2.567086in}}%
\pgfpathlineto{\pgfqpoint{5.076139in}{2.545908in}}%
\pgfpathlineto{\pgfqpoint{5.067864in}{2.524571in}}%
\pgfpathlineto{\pgfqpoint{5.059585in}{2.503081in}}%
\pgfpathlineto{\pgfqpoint{5.051301in}{2.481441in}}%
\pgfpathclose%
\pgfusepath{fill}%
\end{pgfscope}%
\begin{pgfscope}%
\pgfpathrectangle{\pgfqpoint{1.150000in}{0.150000in}}{\pgfqpoint{5.700000in}{5.700000in}}%
\pgfusepath{clip}%
\pgfsetbuttcap%
\pgfsetroundjoin%
\definecolor{currentfill}{rgb}{0.139147,0.533812,0.555298}%
\pgfsetfillcolor{currentfill}%
\pgfsetfillopacity{0.800000}%
\pgfsetlinewidth{0.000000pt}%
\definecolor{currentstroke}{rgb}{0.000000,0.000000,0.000000}%
\pgfsetstrokecolor{currentstroke}%
\pgfsetdash{}{0pt}%
\pgfpathmoveto{\pgfqpoint{4.824452in}{1.966377in}}%
\pgfpathlineto{\pgfqpoint{4.839498in}{1.980561in}}%
\pgfpathlineto{\pgfqpoint{4.854563in}{1.994930in}}%
\pgfpathlineto{\pgfqpoint{4.869649in}{2.009483in}}%
\pgfpathlineto{\pgfqpoint{4.884756in}{2.024220in}}%
\pgfpathlineto{\pgfqpoint{4.893114in}{2.047867in}}%
\pgfpathlineto{\pgfqpoint{4.901470in}{2.071463in}}%
\pgfpathlineto{\pgfqpoint{4.909822in}{2.095003in}}%
\pgfpathlineto{\pgfqpoint{4.918172in}{2.118482in}}%
\pgfpathlineto{\pgfqpoint{4.903042in}{2.103113in}}%
\pgfpathlineto{\pgfqpoint{4.887934in}{2.087931in}}%
\pgfpathlineto{\pgfqpoint{4.872846in}{2.072934in}}%
\pgfpathlineto{\pgfqpoint{4.857780in}{2.058122in}}%
\pgfpathlineto{\pgfqpoint{4.849452in}{2.035260in}}%
\pgfpathlineto{\pgfqpoint{4.841121in}{2.012345in}}%
\pgfpathlineto{\pgfqpoint{4.832788in}{1.989382in}}%
\pgfpathlineto{\pgfqpoint{4.824452in}{1.966377in}}%
\pgfpathclose%
\pgfusepath{fill}%
\end{pgfscope}%
\begin{pgfscope}%
\pgfpathrectangle{\pgfqpoint{1.150000in}{0.150000in}}{\pgfqpoint{5.700000in}{5.700000in}}%
\pgfusepath{clip}%
\pgfsetbuttcap%
\pgfsetroundjoin%
\definecolor{currentfill}{rgb}{0.204903,0.375746,0.553533}%
\pgfsetfillcolor{currentfill}%
\pgfsetfillopacity{0.800000}%
\pgfsetlinewidth{0.000000pt}%
\definecolor{currentstroke}{rgb}{0.000000,0.000000,0.000000}%
\pgfsetstrokecolor{currentstroke}%
\pgfsetdash{}{0pt}%
\pgfpathmoveto{\pgfqpoint{4.597745in}{1.466416in}}%
\pgfpathlineto{\pgfqpoint{4.612619in}{1.476282in}}%
\pgfpathlineto{\pgfqpoint{4.627510in}{1.486325in}}%
\pgfpathlineto{\pgfqpoint{4.642418in}{1.496544in}}%
\pgfpathlineto{\pgfqpoint{4.657344in}{1.506940in}}%
\pgfpathlineto{\pgfqpoint{4.665712in}{1.529466in}}%
\pgfpathlineto{\pgfqpoint{4.674080in}{1.552080in}}%
\pgfpathlineto{\pgfqpoint{4.682447in}{1.574776in}}%
\pgfpathlineto{\pgfqpoint{4.690812in}{1.597545in}}%
\pgfpathlineto{\pgfqpoint{4.675868in}{1.586366in}}%
\pgfpathlineto{\pgfqpoint{4.660942in}{1.575366in}}%
\pgfpathlineto{\pgfqpoint{4.646034in}{1.564543in}}%
\pgfpathlineto{\pgfqpoint{4.631143in}{1.553898in}}%
\pgfpathlineto{\pgfqpoint{4.622796in}{1.531898in}}%
\pgfpathlineto{\pgfqpoint{4.614447in}{1.509979in}}%
\pgfpathlineto{\pgfqpoint{4.606097in}{1.488150in}}%
\pgfpathlineto{\pgfqpoint{4.597745in}{1.466416in}}%
\pgfpathclose%
\pgfusepath{fill}%
\end{pgfscope}%
\begin{pgfscope}%
\pgfpathrectangle{\pgfqpoint{1.150000in}{0.150000in}}{\pgfqpoint{5.700000in}{5.700000in}}%
\pgfusepath{clip}%
\pgfsetbuttcap%
\pgfsetroundjoin%
\definecolor{currentfill}{rgb}{0.140210,0.665859,0.513427}%
\pgfsetfillcolor{currentfill}%
\pgfsetfillopacity{0.800000}%
\pgfsetlinewidth{0.000000pt}%
\definecolor{currentstroke}{rgb}{0.000000,0.000000,0.000000}%
\pgfsetstrokecolor{currentstroke}%
\pgfsetdash{}{0pt}%
\pgfpathmoveto{\pgfqpoint{5.018115in}{2.393464in}}%
\pgfpathlineto{\pgfqpoint{5.033334in}{2.410717in}}%
\pgfpathlineto{\pgfqpoint{5.048576in}{2.428160in}}%
\pgfpathlineto{\pgfqpoint{5.063841in}{2.445793in}}%
\pgfpathlineto{\pgfqpoint{5.072161in}{2.468374in}}%
\pgfpathlineto{\pgfqpoint{5.080477in}{2.490817in}}%
\pgfpathlineto{\pgfqpoint{5.088787in}{2.513115in}}%
\pgfpathlineto{\pgfqpoint{5.097093in}{2.535266in}}%
\pgfpathlineto{\pgfqpoint{5.081805in}{2.517132in}}%
\pgfpathlineto{\pgfqpoint{5.066541in}{2.499191in}}%
\pgfpathlineto{\pgfqpoint{5.051301in}{2.481441in}}%
\pgfpathlineto{\pgfqpoint{5.043011in}{2.459654in}}%
\pgfpathlineto{\pgfqpoint{5.034717in}{2.437727in}}%
\pgfpathlineto{\pgfqpoint{5.026418in}{2.415662in}}%
\pgfpathlineto{\pgfqpoint{5.018115in}{2.393464in}}%
\pgfpathclose%
\pgfusepath{fill}%
\end{pgfscope}%
\begin{pgfscope}%
\pgfpathrectangle{\pgfqpoint{1.150000in}{0.150000in}}{\pgfqpoint{5.700000in}{5.700000in}}%
\pgfusepath{clip}%
\pgfsetbuttcap%
\pgfsetroundjoin%
\definecolor{currentfill}{rgb}{0.149039,0.508051,0.557250}%
\pgfsetfillcolor{currentfill}%
\pgfsetfillopacity{0.800000}%
\pgfsetlinewidth{0.000000pt}%
\definecolor{currentstroke}{rgb}{0.000000,0.000000,0.000000}%
\pgfsetstrokecolor{currentstroke}%
\pgfsetdash{}{0pt}%
\pgfpathmoveto{\pgfqpoint{4.791086in}{1.874053in}}%
\pgfpathlineto{\pgfqpoint{4.806110in}{1.887578in}}%
\pgfpathlineto{\pgfqpoint{4.821154in}{1.901286in}}%
\pgfpathlineto{\pgfqpoint{4.836218in}{1.915176in}}%
\pgfpathlineto{\pgfqpoint{4.851302in}{1.929250in}}%
\pgfpathlineto{\pgfqpoint{4.859669in}{1.953038in}}%
\pgfpathlineto{\pgfqpoint{4.868034in}{1.976800in}}%
\pgfpathlineto{\pgfqpoint{4.876396in}{2.000529in}}%
\pgfpathlineto{\pgfqpoint{4.884756in}{2.024220in}}%
\pgfpathlineto{\pgfqpoint{4.869649in}{2.009483in}}%
\pgfpathlineto{\pgfqpoint{4.854563in}{1.994930in}}%
\pgfpathlineto{\pgfqpoint{4.839498in}{1.980561in}}%
\pgfpathlineto{\pgfqpoint{4.824452in}{1.966377in}}%
\pgfpathlineto{\pgfqpoint{4.816114in}{1.943335in}}%
\pgfpathlineto{\pgfqpoint{4.807774in}{1.920264in}}%
\pgfpathlineto{\pgfqpoint{4.799431in}{1.897168in}}%
\pgfpathlineto{\pgfqpoint{4.791086in}{1.874053in}}%
\pgfpathclose%
\pgfusepath{fill}%
\end{pgfscope}%
\begin{pgfscope}%
\pgfpathrectangle{\pgfqpoint{1.150000in}{0.150000in}}{\pgfqpoint{5.700000in}{5.700000in}}%
\pgfusepath{clip}%
\pgfsetbuttcap%
\pgfsetroundjoin%
\definecolor{currentfill}{rgb}{0.218130,0.347432,0.550038}%
\pgfsetfillcolor{currentfill}%
\pgfsetfillopacity{0.800000}%
\pgfsetlinewidth{0.000000pt}%
\definecolor{currentstroke}{rgb}{0.000000,0.000000,0.000000}%
\pgfsetstrokecolor{currentstroke}%
\pgfsetdash{}{0pt}%
\pgfpathmoveto{\pgfqpoint{4.564326in}{1.380591in}}%
\pgfpathlineto{\pgfqpoint{4.579184in}{1.389648in}}%
\pgfpathlineto{\pgfqpoint{4.594059in}{1.398881in}}%
\pgfpathlineto{\pgfqpoint{4.608950in}{1.408288in}}%
\pgfpathlineto{\pgfqpoint{4.623858in}{1.417871in}}%
\pgfpathlineto{\pgfqpoint{4.632231in}{1.439968in}}%
\pgfpathlineto{\pgfqpoint{4.640603in}{1.462184in}}%
\pgfpathlineto{\pgfqpoint{4.648974in}{1.484510in}}%
\pgfpathlineto{\pgfqpoint{4.657344in}{1.506940in}}%
\pgfpathlineto{\pgfqpoint{4.642418in}{1.496544in}}%
\pgfpathlineto{\pgfqpoint{4.627510in}{1.486325in}}%
\pgfpathlineto{\pgfqpoint{4.612619in}{1.476282in}}%
\pgfpathlineto{\pgfqpoint{4.597745in}{1.466416in}}%
\pgfpathlineto{\pgfqpoint{4.589393in}{1.444786in}}%
\pgfpathlineto{\pgfqpoint{4.581038in}{1.423267in}}%
\pgfpathlineto{\pgfqpoint{4.572683in}{1.401866in}}%
\pgfpathlineto{\pgfqpoint{4.564326in}{1.380591in}}%
\pgfpathclose%
\pgfusepath{fill}%
\end{pgfscope}%
\begin{pgfscope}%
\pgfpathrectangle{\pgfqpoint{1.150000in}{0.150000in}}{\pgfqpoint{5.700000in}{5.700000in}}%
\pgfusepath{clip}%
\pgfsetbuttcap%
\pgfsetroundjoin%
\definecolor{currentfill}{rgb}{0.124780,0.640461,0.527068}%
\pgfsetfillcolor{currentfill}%
\pgfsetfillopacity{0.800000}%
\pgfsetlinewidth{0.000000pt}%
\definecolor{currentstroke}{rgb}{0.000000,0.000000,0.000000}%
\pgfsetstrokecolor{currentstroke}%
\pgfsetdash{}{0pt}%
\pgfpathmoveto{\pgfqpoint{4.984859in}{2.303439in}}%
\pgfpathlineto{\pgfqpoint{5.000056in}{2.320160in}}%
\pgfpathlineto{\pgfqpoint{5.015276in}{2.337069in}}%
\pgfpathlineto{\pgfqpoint{5.030518in}{2.354168in}}%
\pgfpathlineto{\pgfqpoint{5.038855in}{2.377260in}}%
\pgfpathlineto{\pgfqpoint{5.047188in}{2.400231in}}%
\pgfpathlineto{\pgfqpoint{5.055517in}{2.423077in}}%
\pgfpathlineto{\pgfqpoint{5.063841in}{2.445793in}}%
\pgfpathlineto{\pgfqpoint{5.048576in}{2.428160in}}%
\pgfpathlineto{\pgfqpoint{5.033334in}{2.410717in}}%
\pgfpathlineto{\pgfqpoint{5.018115in}{2.393464in}}%
\pgfpathlineto{\pgfqpoint{5.009807in}{2.371138in}}%
\pgfpathlineto{\pgfqpoint{5.001495in}{2.348689in}}%
\pgfpathlineto{\pgfqpoint{4.993179in}{2.326121in}}%
\pgfpathlineto{\pgfqpoint{4.984859in}{2.303439in}}%
\pgfpathclose%
\pgfusepath{fill}%
\end{pgfscope}%
\begin{pgfscope}%
\pgfpathrectangle{\pgfqpoint{1.150000in}{0.150000in}}{\pgfqpoint{5.700000in}{5.700000in}}%
\pgfusepath{clip}%
\pgfsetbuttcap%
\pgfsetroundjoin%
\definecolor{currentfill}{rgb}{0.159194,0.482237,0.558073}%
\pgfsetfillcolor{currentfill}%
\pgfsetfillopacity{0.800000}%
\pgfsetlinewidth{0.000000pt}%
\definecolor{currentstroke}{rgb}{0.000000,0.000000,0.000000}%
\pgfsetstrokecolor{currentstroke}%
\pgfsetdash{}{0pt}%
\pgfpathmoveto{\pgfqpoint{4.757688in}{1.781534in}}%
\pgfpathlineto{\pgfqpoint{4.772691in}{1.794368in}}%
\pgfpathlineto{\pgfqpoint{4.787713in}{1.807383in}}%
\pgfpathlineto{\pgfqpoint{4.802755in}{1.820579in}}%
\pgfpathlineto{\pgfqpoint{4.817817in}{1.833957in}}%
\pgfpathlineto{\pgfqpoint{4.826191in}{1.857789in}}%
\pgfpathlineto{\pgfqpoint{4.834563in}{1.881619in}}%
\pgfpathlineto{\pgfqpoint{4.842934in}{1.905441in}}%
\pgfpathlineto{\pgfqpoint{4.851302in}{1.929250in}}%
\pgfpathlineto{\pgfqpoint{4.836218in}{1.915176in}}%
\pgfpathlineto{\pgfqpoint{4.821154in}{1.901286in}}%
\pgfpathlineto{\pgfqpoint{4.806110in}{1.887578in}}%
\pgfpathlineto{\pgfqpoint{4.791086in}{1.874053in}}%
\pgfpathlineto{\pgfqpoint{4.782739in}{1.850926in}}%
\pgfpathlineto{\pgfqpoint{4.774391in}{1.827793in}}%
\pgfpathlineto{\pgfqpoint{4.766040in}{1.804661in}}%
\pgfpathlineto{\pgfqpoint{4.757688in}{1.781534in}}%
\pgfpathclose%
\pgfusepath{fill}%
\end{pgfscope}%
\begin{pgfscope}%
\pgfpathrectangle{\pgfqpoint{1.150000in}{0.150000in}}{\pgfqpoint{5.700000in}{5.700000in}}%
\pgfusepath{clip}%
\pgfsetbuttcap%
\pgfsetroundjoin%
\definecolor{currentfill}{rgb}{0.119699,0.618490,0.536347}%
\pgfsetfillcolor{currentfill}%
\pgfsetfillopacity{0.800000}%
\pgfsetlinewidth{0.000000pt}%
\definecolor{currentstroke}{rgb}{0.000000,0.000000,0.000000}%
\pgfsetstrokecolor{currentstroke}%
\pgfsetdash{}{0pt}%
\pgfpathmoveto{\pgfqpoint{4.951542in}{2.211670in}}%
\pgfpathlineto{\pgfqpoint{4.966717in}{2.227825in}}%
\pgfpathlineto{\pgfqpoint{4.981913in}{2.244167in}}%
\pgfpathlineto{\pgfqpoint{4.997131in}{2.260697in}}%
\pgfpathlineto{\pgfqpoint{5.005483in}{2.284220in}}%
\pgfpathlineto{\pgfqpoint{5.013832in}{2.307644in}}%
\pgfpathlineto{\pgfqpoint{5.022177in}{2.330961in}}%
\pgfpathlineto{\pgfqpoint{5.030518in}{2.354168in}}%
\pgfpathlineto{\pgfqpoint{5.015276in}{2.337069in}}%
\pgfpathlineto{\pgfqpoint{5.000056in}{2.320160in}}%
\pgfpathlineto{\pgfqpoint{4.984859in}{2.303439in}}%
\pgfpathlineto{\pgfqpoint{4.976536in}{2.280648in}}%
\pgfpathlineto{\pgfqpoint{4.968208in}{2.257753in}}%
\pgfpathlineto{\pgfqpoint{4.959877in}{2.234759in}}%
\pgfpathlineto{\pgfqpoint{4.951542in}{2.211670in}}%
\pgfpathclose%
\pgfusepath{fill}%
\end{pgfscope}%
\begin{pgfscope}%
\pgfpathrectangle{\pgfqpoint{1.150000in}{0.150000in}}{\pgfqpoint{5.700000in}{5.700000in}}%
\pgfusepath{clip}%
\pgfsetbuttcap%
\pgfsetroundjoin%
\definecolor{currentfill}{rgb}{0.229739,0.322361,0.545706}%
\pgfsetfillcolor{currentfill}%
\pgfsetfillopacity{0.800000}%
\pgfsetlinewidth{0.000000pt}%
\definecolor{currentstroke}{rgb}{0.000000,0.000000,0.000000}%
\pgfsetstrokecolor{currentstroke}%
\pgfsetdash{}{0pt}%
\pgfpathmoveto{\pgfqpoint{4.530887in}{1.296903in}}%
\pgfpathlineto{\pgfqpoint{4.545731in}{1.305121in}}%
\pgfpathlineto{\pgfqpoint{4.560590in}{1.313513in}}%
\pgfpathlineto{\pgfqpoint{4.575465in}{1.322079in}}%
\pgfpathlineto{\pgfqpoint{4.590357in}{1.330819in}}%
\pgfpathlineto{\pgfqpoint{4.598734in}{1.352366in}}%
\pgfpathlineto{\pgfqpoint{4.607109in}{1.374062in}}%
\pgfpathlineto{\pgfqpoint{4.615484in}{1.395900in}}%
\pgfpathlineto{\pgfqpoint{4.623858in}{1.417871in}}%
\pgfpathlineto{\pgfqpoint{4.608950in}{1.408288in}}%
\pgfpathlineto{\pgfqpoint{4.594059in}{1.398881in}}%
\pgfpathlineto{\pgfqpoint{4.579184in}{1.389648in}}%
\pgfpathlineto{\pgfqpoint{4.564326in}{1.380591in}}%
\pgfpathlineto{\pgfqpoint{4.555969in}{1.359450in}}%
\pgfpathlineto{\pgfqpoint{4.547609in}{1.338449in}}%
\pgfpathlineto{\pgfqpoint{4.539249in}{1.317598in}}%
\pgfpathlineto{\pgfqpoint{4.530887in}{1.296903in}}%
\pgfpathclose%
\pgfusepath{fill}%
\end{pgfscope}%
\begin{pgfscope}%
\pgfpathrectangle{\pgfqpoint{1.150000in}{0.150000in}}{\pgfqpoint{5.700000in}{5.700000in}}%
\pgfusepath{clip}%
\pgfsetbuttcap%
\pgfsetroundjoin%
\definecolor{currentfill}{rgb}{0.169646,0.456262,0.558030}%
\pgfsetfillcolor{currentfill}%
\pgfsetfillopacity{0.800000}%
\pgfsetlinewidth{0.000000pt}%
\definecolor{currentstroke}{rgb}{0.000000,0.000000,0.000000}%
\pgfsetstrokecolor{currentstroke}%
\pgfsetdash{}{0pt}%
\pgfpathmoveto{\pgfqpoint{4.724262in}{1.689224in}}%
\pgfpathlineto{\pgfqpoint{4.739244in}{1.701335in}}%
\pgfpathlineto{\pgfqpoint{4.754245in}{1.713626in}}%
\pgfpathlineto{\pgfqpoint{4.769265in}{1.726096in}}%
\pgfpathlineto{\pgfqpoint{4.784305in}{1.738748in}}%
\pgfpathlineto{\pgfqpoint{4.792685in}{1.762519in}}%
\pgfpathlineto{\pgfqpoint{4.801064in}{1.786316in}}%
\pgfpathlineto{\pgfqpoint{4.809441in}{1.810131in}}%
\pgfpathlineto{\pgfqpoint{4.817817in}{1.833957in}}%
\pgfpathlineto{\pgfqpoint{4.802755in}{1.820579in}}%
\pgfpathlineto{\pgfqpoint{4.787713in}{1.807383in}}%
\pgfpathlineto{\pgfqpoint{4.772691in}{1.794368in}}%
\pgfpathlineto{\pgfqpoint{4.757688in}{1.781534in}}%
\pgfpathlineto{\pgfqpoint{4.749334in}{1.758421in}}%
\pgfpathlineto{\pgfqpoint{4.740978in}{1.735327in}}%
\pgfpathlineto{\pgfqpoint{4.732620in}{1.712259in}}%
\pgfpathlineto{\pgfqpoint{4.724262in}{1.689224in}}%
\pgfpathclose%
\pgfusepath{fill}%
\end{pgfscope}%
\begin{pgfscope}%
\pgfpathrectangle{\pgfqpoint{1.150000in}{0.150000in}}{\pgfqpoint{5.700000in}{5.700000in}}%
\pgfusepath{clip}%
\pgfsetbuttcap%
\pgfsetroundjoin%
\definecolor{currentfill}{rgb}{0.121831,0.589055,0.545623}%
\pgfsetfillcolor{currentfill}%
\pgfsetfillopacity{0.800000}%
\pgfsetlinewidth{0.000000pt}%
\definecolor{currentstroke}{rgb}{0.000000,0.000000,0.000000}%
\pgfsetstrokecolor{currentstroke}%
\pgfsetdash{}{0pt}%
\pgfpathmoveto{\pgfqpoint{4.918172in}{2.118482in}}%
\pgfpathlineto{\pgfqpoint{4.933323in}{2.134037in}}%
\pgfpathlineto{\pgfqpoint{4.948496in}{2.149778in}}%
\pgfpathlineto{\pgfqpoint{4.963690in}{2.165706in}}%
\pgfpathlineto{\pgfqpoint{4.972055in}{2.189578in}}%
\pgfpathlineto{\pgfqpoint{4.980417in}{2.213370in}}%
\pgfpathlineto{\pgfqpoint{4.988776in}{2.237079in}}%
\pgfpathlineto{\pgfqpoint{4.997131in}{2.260697in}}%
\pgfpathlineto{\pgfqpoint{4.981913in}{2.244167in}}%
\pgfpathlineto{\pgfqpoint{4.966717in}{2.227825in}}%
\pgfpathlineto{\pgfqpoint{4.951542in}{2.211670in}}%
\pgfpathlineto{\pgfqpoint{4.943205in}{2.188493in}}%
\pgfpathlineto{\pgfqpoint{4.934863in}{2.165232in}}%
\pgfpathlineto{\pgfqpoint{4.926519in}{2.141894in}}%
\pgfpathlineto{\pgfqpoint{4.918172in}{2.118482in}}%
\pgfpathclose%
\pgfusepath{fill}%
\end{pgfscope}%
\begin{pgfscope}%
\pgfpathrectangle{\pgfqpoint{1.150000in}{0.150000in}}{\pgfqpoint{5.700000in}{5.700000in}}%
\pgfusepath{clip}%
\pgfsetbuttcap%
\pgfsetroundjoin%
\definecolor{currentfill}{rgb}{0.180629,0.429975,0.557282}%
\pgfsetfillcolor{currentfill}%
\pgfsetfillopacity{0.800000}%
\pgfsetlinewidth{0.000000pt}%
\definecolor{currentstroke}{rgb}{0.000000,0.000000,0.000000}%
\pgfsetstrokecolor{currentstroke}%
\pgfsetdash{}{0pt}%
\pgfpathmoveto{\pgfqpoint{4.690812in}{1.597545in}}%
\pgfpathlineto{\pgfqpoint{4.705775in}{1.608902in}}%
\pgfpathlineto{\pgfqpoint{4.720755in}{1.620438in}}%
\pgfpathlineto{\pgfqpoint{4.735755in}{1.632152in}}%
\pgfpathlineto{\pgfqpoint{4.750773in}{1.644046in}}%
\pgfpathlineto{\pgfqpoint{4.759157in}{1.667650in}}%
\pgfpathlineto{\pgfqpoint{4.767541in}{1.691306in}}%
\pgfpathlineto{\pgfqpoint{4.775924in}{1.715007in}}%
\pgfpathlineto{\pgfqpoint{4.784305in}{1.738748in}}%
\pgfpathlineto{\pgfqpoint{4.769265in}{1.726096in}}%
\pgfpathlineto{\pgfqpoint{4.754245in}{1.713626in}}%
\pgfpathlineto{\pgfqpoint{4.739244in}{1.701335in}}%
\pgfpathlineto{\pgfqpoint{4.724262in}{1.689224in}}%
\pgfpathlineto{\pgfqpoint{4.715901in}{1.666228in}}%
\pgfpathlineto{\pgfqpoint{4.707540in}{1.643279in}}%
\pgfpathlineto{\pgfqpoint{4.699177in}{1.620382in}}%
\pgfpathlineto{\pgfqpoint{4.690812in}{1.597545in}}%
\pgfpathclose%
\pgfusepath{fill}%
\end{pgfscope}%
\begin{pgfscope}%
\pgfpathrectangle{\pgfqpoint{1.150000in}{0.150000in}}{\pgfqpoint{5.700000in}{5.700000in}}%
\pgfusepath{clip}%
\pgfsetbuttcap%
\pgfsetroundjoin%
\definecolor{currentfill}{rgb}{0.128729,0.563265,0.551229}%
\pgfsetfillcolor{currentfill}%
\pgfsetfillopacity{0.800000}%
\pgfsetlinewidth{0.000000pt}%
\definecolor{currentstroke}{rgb}{0.000000,0.000000,0.000000}%
\pgfsetstrokecolor{currentstroke}%
\pgfsetdash{}{0pt}%
\pgfpathmoveto{\pgfqpoint{4.884756in}{2.024220in}}%
\pgfpathlineto{\pgfqpoint{4.899884in}{2.039142in}}%
\pgfpathlineto{\pgfqpoint{4.915033in}{2.054249in}}%
\pgfpathlineto{\pgfqpoint{4.930204in}{2.069542in}}%
\pgfpathlineto{\pgfqpoint{4.938579in}{2.093673in}}%
\pgfpathlineto{\pgfqpoint{4.946952in}{2.117748in}}%
\pgfpathlineto{\pgfqpoint{4.955323in}{2.141761in}}%
\pgfpathlineto{\pgfqpoint{4.963690in}{2.165706in}}%
\pgfpathlineto{\pgfqpoint{4.948496in}{2.149778in}}%
\pgfpathlineto{\pgfqpoint{4.933323in}{2.134037in}}%
\pgfpathlineto{\pgfqpoint{4.918172in}{2.118482in}}%
\pgfpathlineto{\pgfqpoint{4.909822in}{2.095003in}}%
\pgfpathlineto{\pgfqpoint{4.901470in}{2.071463in}}%
\pgfpathlineto{\pgfqpoint{4.893114in}{2.047867in}}%
\pgfpathlineto{\pgfqpoint{4.884756in}{2.024220in}}%
\pgfpathclose%
\pgfusepath{fill}%
\end{pgfscope}%
\begin{pgfscope}%
\pgfpathrectangle{\pgfqpoint{1.150000in}{0.150000in}}{\pgfqpoint{5.700000in}{5.700000in}}%
\pgfusepath{clip}%
\pgfsetbuttcap%
\pgfsetroundjoin%
\definecolor{currentfill}{rgb}{0.192357,0.403199,0.555836}%
\pgfsetfillcolor{currentfill}%
\pgfsetfillopacity{0.800000}%
\pgfsetlinewidth{0.000000pt}%
\definecolor{currentstroke}{rgb}{0.000000,0.000000,0.000000}%
\pgfsetstrokecolor{currentstroke}%
\pgfsetdash{}{0pt}%
\pgfpathmoveto{\pgfqpoint{4.657344in}{1.506940in}}%
\pgfpathlineto{\pgfqpoint{4.672287in}{1.517513in}}%
\pgfpathlineto{\pgfqpoint{4.687248in}{1.528263in}}%
\pgfpathlineto{\pgfqpoint{4.702227in}{1.539190in}}%
\pgfpathlineto{\pgfqpoint{4.717224in}{1.550295in}}%
\pgfpathlineto{\pgfqpoint{4.725613in}{1.573619in}}%
\pgfpathlineto{\pgfqpoint{4.734000in}{1.597023in}}%
\pgfpathlineto{\pgfqpoint{4.742387in}{1.620501in}}%
\pgfpathlineto{\pgfqpoint{4.750773in}{1.644046in}}%
\pgfpathlineto{\pgfqpoint{4.735755in}{1.632152in}}%
\pgfpathlineto{\pgfqpoint{4.720755in}{1.620438in}}%
\pgfpathlineto{\pgfqpoint{4.705775in}{1.608902in}}%
\pgfpathlineto{\pgfqpoint{4.690812in}{1.597545in}}%
\pgfpathlineto{\pgfqpoint{4.682447in}{1.574776in}}%
\pgfpathlineto{\pgfqpoint{4.674080in}{1.552080in}}%
\pgfpathlineto{\pgfqpoint{4.665712in}{1.529466in}}%
\pgfpathlineto{\pgfqpoint{4.657344in}{1.506940in}}%
\pgfpathclose%
\pgfusepath{fill}%
\end{pgfscope}%
\begin{pgfscope}%
\pgfpathrectangle{\pgfqpoint{1.150000in}{0.150000in}}{\pgfqpoint{5.700000in}{5.700000in}}%
\pgfusepath{clip}%
\pgfsetbuttcap%
\pgfsetroundjoin%
\definecolor{currentfill}{rgb}{0.137770,0.537492,0.554906}%
\pgfsetfillcolor{currentfill}%
\pgfsetfillopacity{0.800000}%
\pgfsetlinewidth{0.000000pt}%
\definecolor{currentstroke}{rgb}{0.000000,0.000000,0.000000}%
\pgfsetstrokecolor{currentstroke}%
\pgfsetdash{}{0pt}%
\pgfpathmoveto{\pgfqpoint{4.851302in}{1.929250in}}%
\pgfpathlineto{\pgfqpoint{4.866407in}{1.943507in}}%
\pgfpathlineto{\pgfqpoint{4.881532in}{1.957948in}}%
\pgfpathlineto{\pgfqpoint{4.896678in}{1.972573in}}%
\pgfpathlineto{\pgfqpoint{4.905063in}{1.996870in}}%
\pgfpathlineto{\pgfqpoint{4.913445in}{2.021134in}}%
\pgfpathlineto{\pgfqpoint{4.921826in}{2.045360in}}%
\pgfpathlineto{\pgfqpoint{4.930204in}{2.069542in}}%
\pgfpathlineto{\pgfqpoint{4.915033in}{2.054249in}}%
\pgfpathlineto{\pgfqpoint{4.899884in}{2.039142in}}%
\pgfpathlineto{\pgfqpoint{4.884756in}{2.024220in}}%
\pgfpathlineto{\pgfqpoint{4.876396in}{2.000529in}}%
\pgfpathlineto{\pgfqpoint{4.868034in}{1.976800in}}%
\pgfpathlineto{\pgfqpoint{4.859669in}{1.953038in}}%
\pgfpathlineto{\pgfqpoint{4.851302in}{1.929250in}}%
\pgfpathclose%
\pgfusepath{fill}%
\end{pgfscope}%
\begin{pgfscope}%
\pgfpathrectangle{\pgfqpoint{1.150000in}{0.150000in}}{\pgfqpoint{5.700000in}{5.700000in}}%
\pgfusepath{clip}%
\pgfsetbuttcap%
\pgfsetroundjoin%
\definecolor{currentfill}{rgb}{0.204903,0.375746,0.553533}%
\pgfsetfillcolor{currentfill}%
\pgfsetfillopacity{0.800000}%
\pgfsetlinewidth{0.000000pt}%
\definecolor{currentstroke}{rgb}{0.000000,0.000000,0.000000}%
\pgfsetstrokecolor{currentstroke}%
\pgfsetdash{}{0pt}%
\pgfpathmoveto{\pgfqpoint{4.623858in}{1.417871in}}%
\pgfpathlineto{\pgfqpoint{4.638783in}{1.427629in}}%
\pgfpathlineto{\pgfqpoint{4.653726in}{1.437564in}}%
\pgfpathlineto{\pgfqpoint{4.668685in}{1.447674in}}%
\pgfpathlineto{\pgfqpoint{4.683663in}{1.457961in}}%
\pgfpathlineto{\pgfqpoint{4.692054in}{1.480886in}}%
\pgfpathlineto{\pgfqpoint{4.700445in}{1.503921in}}%
\pgfpathlineto{\pgfqpoint{4.708835in}{1.527060in}}%
\pgfpathlineto{\pgfqpoint{4.717224in}{1.550295in}}%
\pgfpathlineto{\pgfqpoint{4.702227in}{1.539190in}}%
\pgfpathlineto{\pgfqpoint{4.687248in}{1.528263in}}%
\pgfpathlineto{\pgfqpoint{4.672287in}{1.517513in}}%
\pgfpathlineto{\pgfqpoint{4.657344in}{1.506940in}}%
\pgfpathlineto{\pgfqpoint{4.648974in}{1.484510in}}%
\pgfpathlineto{\pgfqpoint{4.640603in}{1.462184in}}%
\pgfpathlineto{\pgfqpoint{4.632231in}{1.439968in}}%
\pgfpathlineto{\pgfqpoint{4.623858in}{1.417871in}}%
\pgfpathclose%
\pgfusepath{fill}%
\end{pgfscope}%
\begin{pgfscope}%
\pgfpathrectangle{\pgfqpoint{1.150000in}{0.150000in}}{\pgfqpoint{5.700000in}{5.700000in}}%
\pgfusepath{clip}%
\pgfsetbuttcap%
\pgfsetroundjoin%
\definecolor{currentfill}{rgb}{0.149039,0.508051,0.557250}%
\pgfsetfillcolor{currentfill}%
\pgfsetfillopacity{0.800000}%
\pgfsetlinewidth{0.000000pt}%
\definecolor{currentstroke}{rgb}{0.000000,0.000000,0.000000}%
\pgfsetstrokecolor{currentstroke}%
\pgfsetdash{}{0pt}%
\pgfpathmoveto{\pgfqpoint{4.817817in}{1.833957in}}%
\pgfpathlineto{\pgfqpoint{4.832898in}{1.847517in}}%
\pgfpathlineto{\pgfqpoint{4.848000in}{1.861260in}}%
\pgfpathlineto{\pgfqpoint{4.863122in}{1.875185in}}%
\pgfpathlineto{\pgfqpoint{4.871514in}{1.899549in}}%
\pgfpathlineto{\pgfqpoint{4.879904in}{1.923906in}}%
\pgfpathlineto{\pgfqpoint{4.888292in}{1.948250in}}%
\pgfpathlineto{\pgfqpoint{4.896678in}{1.972573in}}%
\pgfpathlineto{\pgfqpoint{4.881532in}{1.957948in}}%
\pgfpathlineto{\pgfqpoint{4.866407in}{1.943507in}}%
\pgfpathlineto{\pgfqpoint{4.851302in}{1.929250in}}%
\pgfpathlineto{\pgfqpoint{4.842934in}{1.905441in}}%
\pgfpathlineto{\pgfqpoint{4.834563in}{1.881619in}}%
\pgfpathlineto{\pgfqpoint{4.826191in}{1.857789in}}%
\pgfpathlineto{\pgfqpoint{4.817817in}{1.833957in}}%
\pgfpathclose%
\pgfusepath{fill}%
\end{pgfscope}%
\begin{pgfscope}%
\pgfpathrectangle{\pgfqpoint{1.150000in}{0.150000in}}{\pgfqpoint{5.700000in}{5.700000in}}%
\pgfusepath{clip}%
\pgfsetbuttcap%
\pgfsetroundjoin%
\definecolor{currentfill}{rgb}{0.218130,0.347432,0.550038}%
\pgfsetfillcolor{currentfill}%
\pgfsetfillopacity{0.800000}%
\pgfsetlinewidth{0.000000pt}%
\definecolor{currentstroke}{rgb}{0.000000,0.000000,0.000000}%
\pgfsetstrokecolor{currentstroke}%
\pgfsetdash{}{0pt}%
\pgfpathmoveto{\pgfqpoint{4.590357in}{1.330819in}}%
\pgfpathlineto{\pgfqpoint{4.605266in}{1.339734in}}%
\pgfpathlineto{\pgfqpoint{4.620191in}{1.348822in}}%
\pgfpathlineto{\pgfqpoint{4.635132in}{1.358086in}}%
\pgfpathlineto{\pgfqpoint{4.650091in}{1.367524in}}%
\pgfpathlineto{\pgfqpoint{4.658485in}{1.389928in}}%
\pgfpathlineto{\pgfqpoint{4.666878in}{1.412474in}}%
\pgfpathlineto{\pgfqpoint{4.675271in}{1.435154in}}%
\pgfpathlineto{\pgfqpoint{4.683663in}{1.457961in}}%
\pgfpathlineto{\pgfqpoint{4.668685in}{1.447674in}}%
\pgfpathlineto{\pgfqpoint{4.653726in}{1.437564in}}%
\pgfpathlineto{\pgfqpoint{4.638783in}{1.427629in}}%
\pgfpathlineto{\pgfqpoint{4.623858in}{1.417871in}}%
\pgfpathlineto{\pgfqpoint{4.615484in}{1.395900in}}%
\pgfpathlineto{\pgfqpoint{4.607109in}{1.374062in}}%
\pgfpathlineto{\pgfqpoint{4.598734in}{1.352366in}}%
\pgfpathlineto{\pgfqpoint{4.590357in}{1.330819in}}%
\pgfpathclose%
\pgfusepath{fill}%
\end{pgfscope}%
\begin{pgfscope}%
\pgfpathrectangle{\pgfqpoint{1.150000in}{0.150000in}}{\pgfqpoint{5.700000in}{5.700000in}}%
\pgfusepath{clip}%
\pgfsetbuttcap%
\pgfsetroundjoin%
\definecolor{currentfill}{rgb}{0.159194,0.482237,0.558073}%
\pgfsetfillcolor{currentfill}%
\pgfsetfillopacity{0.800000}%
\pgfsetlinewidth{0.000000pt}%
\definecolor{currentstroke}{rgb}{0.000000,0.000000,0.000000}%
\pgfsetstrokecolor{currentstroke}%
\pgfsetdash{}{0pt}%
\pgfpathmoveto{\pgfqpoint{4.784305in}{1.738748in}}%
\pgfpathlineto{\pgfqpoint{4.799364in}{1.751579in}}%
\pgfpathlineto{\pgfqpoint{4.814442in}{1.764592in}}%
\pgfpathlineto{\pgfqpoint{4.829541in}{1.777786in}}%
\pgfpathlineto{\pgfqpoint{4.837938in}{1.802114in}}%
\pgfpathlineto{\pgfqpoint{4.846334in}{1.826461in}}%
\pgfpathlineto{\pgfqpoint{4.854729in}{1.850820in}}%
\pgfpathlineto{\pgfqpoint{4.863122in}{1.875185in}}%
\pgfpathlineto{\pgfqpoint{4.848000in}{1.861260in}}%
\pgfpathlineto{\pgfqpoint{4.832898in}{1.847517in}}%
\pgfpathlineto{\pgfqpoint{4.817817in}{1.833957in}}%
\pgfpathlineto{\pgfqpoint{4.809441in}{1.810131in}}%
\pgfpathlineto{\pgfqpoint{4.801064in}{1.786316in}}%
\pgfpathlineto{\pgfqpoint{4.792685in}{1.762519in}}%
\pgfpathlineto{\pgfqpoint{4.784305in}{1.738748in}}%
\pgfpathclose%
\pgfusepath{fill}%
\end{pgfscope}%
\begin{pgfscope}%
\pgfpathrectangle{\pgfqpoint{1.150000in}{0.150000in}}{\pgfqpoint{5.700000in}{5.700000in}}%
\pgfusepath{clip}%
\pgfsetbuttcap%
\pgfsetroundjoin%
\definecolor{currentfill}{rgb}{0.171176,0.452530,0.557965}%
\pgfsetfillcolor{currentfill}%
\pgfsetfillopacity{0.800000}%
\pgfsetlinewidth{0.000000pt}%
\definecolor{currentstroke}{rgb}{0.000000,0.000000,0.000000}%
\pgfsetstrokecolor{currentstroke}%
\pgfsetdash{}{0pt}%
\pgfpathmoveto{\pgfqpoint{4.750773in}{1.644046in}}%
\pgfpathlineto{\pgfqpoint{4.765810in}{1.656118in}}%
\pgfpathlineto{\pgfqpoint{4.780865in}{1.668370in}}%
\pgfpathlineto{\pgfqpoint{4.795941in}{1.680802in}}%
\pgfpathlineto{\pgfqpoint{4.804342in}{1.704985in}}%
\pgfpathlineto{\pgfqpoint{4.812743in}{1.729215in}}%
\pgfpathlineto{\pgfqpoint{4.821142in}{1.753484in}}%
\pgfpathlineto{\pgfqpoint{4.829541in}{1.777786in}}%
\pgfpathlineto{\pgfqpoint{4.814442in}{1.764592in}}%
\pgfpathlineto{\pgfqpoint{4.799364in}{1.751579in}}%
\pgfpathlineto{\pgfqpoint{4.784305in}{1.738748in}}%
\pgfpathlineto{\pgfqpoint{4.775924in}{1.715007in}}%
\pgfpathlineto{\pgfqpoint{4.767541in}{1.691306in}}%
\pgfpathlineto{\pgfqpoint{4.759157in}{1.667650in}}%
\pgfpathlineto{\pgfqpoint{4.750773in}{1.644046in}}%
\pgfpathclose%
\pgfusepath{fill}%
\end{pgfscope}%
\begin{pgfscope}%
\pgfpathrectangle{\pgfqpoint{1.150000in}{0.150000in}}{\pgfqpoint{5.700000in}{5.700000in}}%
\pgfusepath{clip}%
\pgfsetbuttcap%
\pgfsetroundjoin%
\definecolor{currentfill}{rgb}{0.182256,0.426184,0.557120}%
\pgfsetfillcolor{currentfill}%
\pgfsetfillopacity{0.800000}%
\pgfsetlinewidth{0.000000pt}%
\definecolor{currentstroke}{rgb}{0.000000,0.000000,0.000000}%
\pgfsetstrokecolor{currentstroke}%
\pgfsetdash{}{0pt}%
\pgfpathmoveto{\pgfqpoint{4.717224in}{1.550295in}}%
\pgfpathlineto{\pgfqpoint{4.732240in}{1.561578in}}%
\pgfpathlineto{\pgfqpoint{4.747274in}{1.573039in}}%
\pgfpathlineto{\pgfqpoint{4.762327in}{1.584678in}}%
\pgfpathlineto{\pgfqpoint{4.770731in}{1.608604in}}%
\pgfpathlineto{\pgfqpoint{4.779135in}{1.632604in}}%
\pgfpathlineto{\pgfqpoint{4.787538in}{1.656673in}}%
\pgfpathlineto{\pgfqpoint{4.795941in}{1.680802in}}%
\pgfpathlineto{\pgfqpoint{4.780865in}{1.668370in}}%
\pgfpathlineto{\pgfqpoint{4.765810in}{1.656118in}}%
\pgfpathlineto{\pgfqpoint{4.750773in}{1.644046in}}%
\pgfpathlineto{\pgfqpoint{4.742387in}{1.620501in}}%
\pgfpathlineto{\pgfqpoint{4.734000in}{1.597023in}}%
\pgfpathlineto{\pgfqpoint{4.725613in}{1.573619in}}%
\pgfpathlineto{\pgfqpoint{4.717224in}{1.550295in}}%
\pgfpathclose%
\pgfusepath{fill}%
\end{pgfscope}%
\begin{pgfscope}%
\pgfpathrectangle{\pgfqpoint{1.150000in}{0.150000in}}{\pgfqpoint{5.700000in}{5.700000in}}%
\pgfusepath{clip}%
\pgfsetbuttcap%
\pgfsetroundjoin%
\definecolor{currentfill}{rgb}{0.194100,0.399323,0.555565}%
\pgfsetfillcolor{currentfill}%
\pgfsetfillopacity{0.800000}%
\pgfsetlinewidth{0.000000pt}%
\definecolor{currentstroke}{rgb}{0.000000,0.000000,0.000000}%
\pgfsetstrokecolor{currentstroke}%
\pgfsetdash{}{0pt}%
\pgfpathmoveto{\pgfqpoint{4.683663in}{1.457961in}}%
\pgfpathlineto{\pgfqpoint{4.698658in}{1.468424in}}%
\pgfpathlineto{\pgfqpoint{4.713671in}{1.479063in}}%
\pgfpathlineto{\pgfqpoint{4.728702in}{1.489880in}}%
\pgfpathlineto{\pgfqpoint{4.737109in}{1.513429in}}%
\pgfpathlineto{\pgfqpoint{4.745516in}{1.537084in}}%
\pgfpathlineto{\pgfqpoint{4.753921in}{1.560836in}}%
\pgfpathlineto{\pgfqpoint{4.762327in}{1.584678in}}%
\pgfpathlineto{\pgfqpoint{4.747274in}{1.573039in}}%
\pgfpathlineto{\pgfqpoint{4.732240in}{1.561578in}}%
\pgfpathlineto{\pgfqpoint{4.717224in}{1.550295in}}%
\pgfpathlineto{\pgfqpoint{4.708835in}{1.527060in}}%
\pgfpathlineto{\pgfqpoint{4.700445in}{1.503921in}}%
\pgfpathlineto{\pgfqpoint{4.692054in}{1.480886in}}%
\pgfpathlineto{\pgfqpoint{4.683663in}{1.457961in}}%
\pgfpathclose%
\pgfusepath{fill}%
\end{pgfscope}%
\begin{pgfscope}%
\pgfpathrectangle{\pgfqpoint{1.150000in}{0.150000in}}{\pgfqpoint{5.700000in}{5.700000in}}%
\pgfusepath{clip}%
\pgfsetbuttcap%
\pgfsetroundjoin%
\definecolor{currentfill}{rgb}{0.206756,0.371758,0.553117}%
\pgfsetfillcolor{currentfill}%
\pgfsetfillopacity{0.800000}%
\pgfsetlinewidth{0.000000pt}%
\definecolor{currentstroke}{rgb}{0.000000,0.000000,0.000000}%
\pgfsetstrokecolor{currentstroke}%
\pgfsetdash{}{0pt}%
\pgfpathmoveto{\pgfqpoint{4.650091in}{1.367524in}}%
\pgfpathlineto{\pgfqpoint{4.665067in}{1.377137in}}%
\pgfpathlineto{\pgfqpoint{4.680061in}{1.386926in}}%
\pgfpathlineto{\pgfqpoint{4.695072in}{1.396890in}}%
\pgfpathlineto{\pgfqpoint{4.703480in}{1.419941in}}%
\pgfpathlineto{\pgfqpoint{4.711888in}{1.443128in}}%
\pgfpathlineto{\pgfqpoint{4.720295in}{1.466443in}}%
\pgfpathlineto{\pgfqpoint{4.728702in}{1.489880in}}%
\pgfpathlineto{\pgfqpoint{4.713671in}{1.479063in}}%
\pgfpathlineto{\pgfqpoint{4.698658in}{1.468424in}}%
\pgfpathlineto{\pgfqpoint{4.683663in}{1.457961in}}%
\pgfpathlineto{\pgfqpoint{4.675271in}{1.435154in}}%
\pgfpathlineto{\pgfqpoint{4.666878in}{1.412474in}}%
\pgfpathlineto{\pgfqpoint{4.658485in}{1.389928in}}%
\pgfpathlineto{\pgfqpoint{4.650091in}{1.367524in}}%
\pgfpathclose%
\pgfusepath{fill}%
\end{pgfscope}%
\end{pgfpicture}%
\makeatother%
\endgroup%
}
   \caption{Pohľad na graf funkcie v $\mathbb{R}^3$.}
   \label{fig:graph_surface}
\end{figure}
\vspace*{\fill}
\newpage

\begin{figure}[h]
    \centering
    \includegraphics[width=0.8\textwidth]{grafy/Pohlady.png}
    \caption{Rôzne 3D pohľady na funkciu $f(x,y)$.}
    \label{fig:moj_obrazok}
\end{figure}
\newpage
%%%%%%%%%%%%%%%%%%%%%%%%%%%%%%%%%%%%%%%%%%%%%%%%%%%%%%%%%%%%%%%%%%%%%


\subsection{Newtonova metóda}

Táto metóda patrí medzi metódy druhého rádu, pretože pri hľadaní smeru poklesu využíva nielen gradient (prvé derivácie), ale aj Hessovu maticu (druhé derivácie). Hlavná myšlienka spočíva v tom, že v každom kroku funkciu $f$ aproximujeme Taylorovým polynómom druhého rádu a hľadáme jeho minimum.

Členy minimalizujúcej postupnosti $\{x^{[k]}\}$ počítame iteračným vzťahom:
$$ x^{[k+1]} = x^{[k]} - [\nabla^2 f(x^{[k]})]^{-1} \cdot \nabla f(x^{[k]}), $$
kde $\nabla f(x^{[k]})$ je gradient funkcie a $\nabla^2 f(x^{[k]})$ je Hessova matica funkcie v bode $x^{[k]}$.
Vďaka využitiu informácií o zakrivení funkcie táto metóda konverguje spravidla veľmi rýchlo (kvadraticky) v blízkosti lokálneho minima.

\subsubsection{Analýza metódy pri rôznych hodnotách počiatočnej aproximácie}

Na analýzu metódy pri rôznych počiatočných bodoch $x^{[0]}$ opäť potrebujeme vhodné ukončovacie kritérium. V súlade s teóriou volíme podmienku pre zmenu funkčnej hodnoty
$$ |f(x^{[k]}) - f(x^{[k-1]})|=\epsilon < 0{,}001. $$
Budeme teda sledovať, po koľkých krokoch sa Newtonova metóda pre rôzne $x^{[0]}$ „ukončí“.

\newpage
\noindent \textbf{Počiatočný bod} $x^{[0]} = [0; 0]$ 
\vspace{0.5cm}


Ako prvý volíme počiatok súradnicovej sústavy. Pre Newtonovu metódu očakávame veľmi rýchlu konvergenciu.

\begin{table}[h!]
    \centering
    \begin{tabular}{ccccc}
        \toprule
        \textbf{Iterácia} & \textbf{Bod } $x^{[k]}$ & \textbf{Hodnota } $f(x^{[k]})$ & \textbf{Rozdiel } $|f_k - f_{k-1}|$ \\
        \midrule
        0 & $[0.000000; 0.000000]$ & $2.000000$ & $0.367879$ \\
        1 & $[0.000000; -1.000000]$ & $2.367879$ & $0.777743$ \\
        2 & $[0.483674; -0.862755]$ & $1.590136$ & $0.020643$ \\
        3 & $[0.400337; -0.759623]$ & $1.569493$ & $0.000496$ \\
        \bottomrule
    \end{tabular}
    \caption{Priebeh NM pre $x^{[0]}=[0;0]$. Minimum nájdené v $(0.400337, -0.759623)$.}
\end{table}


\begin{figure}[H]
    \centering
    
    % --- ĽAVÝ OBRÁZOK ---
    \begin{subfigure}[b]{0.48\textwidth}
        \centering
        \resizebox{\linewidth}{!}{%% Creator: Matplotlib, PGF backend
%%
%% To include the figure in your LaTeX document, write
%%   \input{<filename>.pgf}
%%
%% Make sure the required packages are loaded in your preamble
%%   \usepackage{pgf}
%%
%% Also ensure that all the required font packages are loaded; for instance,
%% the lmodern package is sometimes necessary when using math font.
%%   \usepackage{lmodern}
%%
%% Figures using additional raster images can only be included by \input if
%% they are in the same directory as the main LaTeX file. For loading figures
%% from other directories you can use the `import` package
%%   \usepackage{import}
%%
%% and then include the figures with
%%   \import{<path to file>}{<filename>.pgf}
%%
%% Matplotlib used the following preamble
%%   
%%   \usepackage{fontspec}
%%   \setmainfont{DejaVuSerif.ttf}[Path=\detokenize{/home/radimek/Documents/projekt_mat_prog/mat_prog_kernel/lib/python3.12/site-packages/matplotlib/mpl-data/fonts/ttf/}]
%%   \setsansfont{DejaVuSans.ttf}[Path=\detokenize{/home/radimek/Documents/projekt_mat_prog/mat_prog_kernel/lib/python3.12/site-packages/matplotlib/mpl-data/fonts/ttf/}]
%%   \setmonofont{DejaVuSansMono.ttf}[Path=\detokenize{/home/radimek/Documents/projekt_mat_prog/mat_prog_kernel/lib/python3.12/site-packages/matplotlib/mpl-data/fonts/ttf/}]
%%   \makeatletter\@ifpackageloaded{underscore}{}{\usepackage[strings]{underscore}}\makeatother
%%
\begingroup%
\makeatletter%
\begin{pgfpicture}%
\pgfpathrectangle{\pgfpointorigin}{\pgfqpoint{8.000000in}{6.000000in}}%
\pgfusepath{use as bounding box, clip}%
\begin{pgfscope}%
\pgfsetbuttcap%
\pgfsetmiterjoin%
\definecolor{currentfill}{rgb}{1.000000,1.000000,1.000000}%
\pgfsetfillcolor{currentfill}%
\pgfsetlinewidth{0.000000pt}%
\definecolor{currentstroke}{rgb}{1.000000,1.000000,1.000000}%
\pgfsetstrokecolor{currentstroke}%
\pgfsetdash{}{0pt}%
\pgfpathmoveto{\pgfqpoint{0.000000in}{0.000000in}}%
\pgfpathlineto{\pgfqpoint{8.000000in}{0.000000in}}%
\pgfpathlineto{\pgfqpoint{8.000000in}{6.000000in}}%
\pgfpathlineto{\pgfqpoint{0.000000in}{6.000000in}}%
\pgfpathlineto{\pgfqpoint{0.000000in}{0.000000in}}%
\pgfpathclose%
\pgfusepath{fill}%
\end{pgfscope}%
\begin{pgfscope}%
\pgfsetbuttcap%
\pgfsetmiterjoin%
\definecolor{currentfill}{rgb}{1.000000,1.000000,1.000000}%
\pgfsetfillcolor{currentfill}%
\pgfsetlinewidth{0.000000pt}%
\definecolor{currentstroke}{rgb}{0.000000,0.000000,0.000000}%
\pgfsetstrokecolor{currentstroke}%
\pgfsetstrokeopacity{0.000000}%
\pgfsetdash{}{0pt}%
\pgfpathmoveto{\pgfqpoint{0.854460in}{0.571603in}}%
\pgfpathlineto{\pgfqpoint{7.739560in}{0.571603in}}%
\pgfpathlineto{\pgfqpoint{7.739560in}{5.797238in}}%
\pgfpathlineto{\pgfqpoint{0.854460in}{5.797238in}}%
\pgfpathlineto{\pgfqpoint{0.854460in}{0.571603in}}%
\pgfpathclose%
\pgfusepath{fill}%
\end{pgfscope}%
\begin{pgfscope}%
\pgfpathrectangle{\pgfqpoint{0.854460in}{0.571603in}}{\pgfqpoint{6.885100in}{5.225635in}}%
\pgfusepath{clip}%
\pgfsetbuttcap%
\pgfsetroundjoin%
\definecolor{currentfill}{rgb}{1.000000,0.000000,0.000000}%
\pgfsetfillcolor{currentfill}%
\pgfsetlinewidth{1.003750pt}%
\definecolor{currentstroke}{rgb}{1.000000,0.000000,0.000000}%
\pgfsetstrokecolor{currentstroke}%
\pgfsetdash{}{0pt}%
\pgfsys@defobject{currentmarker}{\pgfqpoint{-0.041667in}{-0.041667in}}{\pgfqpoint{0.041667in}{0.041667in}}{%
\pgfpathmoveto{\pgfqpoint{0.000000in}{-0.041667in}}%
\pgfpathcurveto{\pgfqpoint{0.011050in}{-0.041667in}}{\pgfqpoint{0.021649in}{-0.037276in}}{\pgfqpoint{0.029463in}{-0.029463in}}%
\pgfpathcurveto{\pgfqpoint{0.037276in}{-0.021649in}}{\pgfqpoint{0.041667in}{-0.011050in}}{\pgfqpoint{0.041667in}{0.000000in}}%
\pgfpathcurveto{\pgfqpoint{0.041667in}{0.011050in}}{\pgfqpoint{0.037276in}{0.021649in}}{\pgfqpoint{0.029463in}{0.029463in}}%
\pgfpathcurveto{\pgfqpoint{0.021649in}{0.037276in}}{\pgfqpoint{0.011050in}{0.041667in}}{\pgfqpoint{0.000000in}{0.041667in}}%
\pgfpathcurveto{\pgfqpoint{-0.011050in}{0.041667in}}{\pgfqpoint{-0.021649in}{0.037276in}}{\pgfqpoint{-0.029463in}{0.029463in}}%
\pgfpathcurveto{\pgfqpoint{-0.037276in}{0.021649in}}{\pgfqpoint{-0.041667in}{0.011050in}}{\pgfqpoint{-0.041667in}{0.000000in}}%
\pgfpathcurveto{\pgfqpoint{-0.041667in}{-0.011050in}}{\pgfqpoint{-0.037276in}{-0.021649in}}{\pgfqpoint{-0.029463in}{-0.029463in}}%
\pgfpathcurveto{\pgfqpoint{-0.021649in}{-0.037276in}}{\pgfqpoint{-0.011050in}{-0.041667in}}{\pgfqpoint{0.000000in}{-0.041667in}}%
\pgfpathlineto{\pgfqpoint{0.000000in}{-0.041667in}}%
\pgfpathclose%
\pgfusepath{stroke,fill}%
}%
\begin{pgfscope}%
\pgfsys@transformshift{3.149494in}{4.490830in}%
\pgfsys@useobject{currentmarker}{}%
\end{pgfscope}%
\begin{pgfscope}%
\pgfsys@transformshift{3.149494in}{1.878012in}%
\pgfsys@useobject{currentmarker}{}%
\end{pgfscope}%
\begin{pgfscope}%
\pgfsys@transformshift{4.259541in}{2.236607in}%
\pgfsys@useobject{currentmarker}{}%
\end{pgfscope}%
\begin{pgfscope}%
\pgfsys@transformshift{4.068280in}{2.506074in}%
\pgfsys@useobject{currentmarker}{}%
\end{pgfscope}%
\end{pgfscope}%
\begin{pgfscope}%
\pgfsetbuttcap%
\pgfsetroundjoin%
\definecolor{currentfill}{rgb}{0.000000,0.000000,0.000000}%
\pgfsetfillcolor{currentfill}%
\pgfsetlinewidth{0.803000pt}%
\definecolor{currentstroke}{rgb}{0.000000,0.000000,0.000000}%
\pgfsetstrokecolor{currentstroke}%
\pgfsetdash{}{0pt}%
\pgfsys@defobject{currentmarker}{\pgfqpoint{0.000000in}{-0.048611in}}{\pgfqpoint{0.000000in}{0.000000in}}{%
\pgfpathmoveto{\pgfqpoint{0.000000in}{0.000000in}}%
\pgfpathlineto{\pgfqpoint{0.000000in}{-0.048611in}}%
\pgfusepath{stroke,fill}%
}%
\begin{pgfscope}%
\pgfsys@transformshift{0.854460in}{0.571603in}%
\pgfsys@useobject{currentmarker}{}%
\end{pgfscope}%
\end{pgfscope}%
\begin{pgfscope}%
\definecolor{textcolor}{rgb}{0.000000,0.000000,0.000000}%
\pgfsetstrokecolor{textcolor}%
\pgfsetfillcolor{textcolor}%
\pgftext[x=0.854460in,y=0.474381in,,top]{\color{textcolor}\sffamily\fontsize{10.000000}{12.000000}\selectfont \ensuremath{-}1.0}%
\end{pgfscope}%
\begin{pgfscope}%
\pgfsetbuttcap%
\pgfsetroundjoin%
\definecolor{currentfill}{rgb}{0.000000,0.000000,0.000000}%
\pgfsetfillcolor{currentfill}%
\pgfsetlinewidth{0.803000pt}%
\definecolor{currentstroke}{rgb}{0.000000,0.000000,0.000000}%
\pgfsetstrokecolor{currentstroke}%
\pgfsetdash{}{0pt}%
\pgfsys@defobject{currentmarker}{\pgfqpoint{0.000000in}{-0.048611in}}{\pgfqpoint{0.000000in}{0.000000in}}{%
\pgfpathmoveto{\pgfqpoint{0.000000in}{0.000000in}}%
\pgfpathlineto{\pgfqpoint{0.000000in}{-0.048611in}}%
\pgfusepath{stroke,fill}%
}%
\begin{pgfscope}%
\pgfsys@transformshift{2.001977in}{0.571603in}%
\pgfsys@useobject{currentmarker}{}%
\end{pgfscope}%
\end{pgfscope}%
\begin{pgfscope}%
\definecolor{textcolor}{rgb}{0.000000,0.000000,0.000000}%
\pgfsetstrokecolor{textcolor}%
\pgfsetfillcolor{textcolor}%
\pgftext[x=2.001977in,y=0.474381in,,top]{\color{textcolor}\sffamily\fontsize{10.000000}{12.000000}\selectfont \ensuremath{-}0.5}%
\end{pgfscope}%
\begin{pgfscope}%
\pgfsetbuttcap%
\pgfsetroundjoin%
\definecolor{currentfill}{rgb}{0.000000,0.000000,0.000000}%
\pgfsetfillcolor{currentfill}%
\pgfsetlinewidth{0.803000pt}%
\definecolor{currentstroke}{rgb}{0.000000,0.000000,0.000000}%
\pgfsetstrokecolor{currentstroke}%
\pgfsetdash{}{0pt}%
\pgfsys@defobject{currentmarker}{\pgfqpoint{0.000000in}{-0.048611in}}{\pgfqpoint{0.000000in}{0.000000in}}{%
\pgfpathmoveto{\pgfqpoint{0.000000in}{0.000000in}}%
\pgfpathlineto{\pgfqpoint{0.000000in}{-0.048611in}}%
\pgfusepath{stroke,fill}%
}%
\begin{pgfscope}%
\pgfsys@transformshift{3.149494in}{0.571603in}%
\pgfsys@useobject{currentmarker}{}%
\end{pgfscope}%
\end{pgfscope}%
\begin{pgfscope}%
\definecolor{textcolor}{rgb}{0.000000,0.000000,0.000000}%
\pgfsetstrokecolor{textcolor}%
\pgfsetfillcolor{textcolor}%
\pgftext[x=3.149494in,y=0.474381in,,top]{\color{textcolor}\sffamily\fontsize{10.000000}{12.000000}\selectfont 0.0}%
\end{pgfscope}%
\begin{pgfscope}%
\pgfsetbuttcap%
\pgfsetroundjoin%
\definecolor{currentfill}{rgb}{0.000000,0.000000,0.000000}%
\pgfsetfillcolor{currentfill}%
\pgfsetlinewidth{0.803000pt}%
\definecolor{currentstroke}{rgb}{0.000000,0.000000,0.000000}%
\pgfsetstrokecolor{currentstroke}%
\pgfsetdash{}{0pt}%
\pgfsys@defobject{currentmarker}{\pgfqpoint{0.000000in}{-0.048611in}}{\pgfqpoint{0.000000in}{0.000000in}}{%
\pgfpathmoveto{\pgfqpoint{0.000000in}{0.000000in}}%
\pgfpathlineto{\pgfqpoint{0.000000in}{-0.048611in}}%
\pgfusepath{stroke,fill}%
}%
\begin{pgfscope}%
\pgfsys@transformshift{4.297010in}{0.571603in}%
\pgfsys@useobject{currentmarker}{}%
\end{pgfscope}%
\end{pgfscope}%
\begin{pgfscope}%
\definecolor{textcolor}{rgb}{0.000000,0.000000,0.000000}%
\pgfsetstrokecolor{textcolor}%
\pgfsetfillcolor{textcolor}%
\pgftext[x=4.297010in,y=0.474381in,,top]{\color{textcolor}\sffamily\fontsize{10.000000}{12.000000}\selectfont 0.5}%
\end{pgfscope}%
\begin{pgfscope}%
\pgfsetbuttcap%
\pgfsetroundjoin%
\definecolor{currentfill}{rgb}{0.000000,0.000000,0.000000}%
\pgfsetfillcolor{currentfill}%
\pgfsetlinewidth{0.803000pt}%
\definecolor{currentstroke}{rgb}{0.000000,0.000000,0.000000}%
\pgfsetstrokecolor{currentstroke}%
\pgfsetdash{}{0pt}%
\pgfsys@defobject{currentmarker}{\pgfqpoint{0.000000in}{-0.048611in}}{\pgfqpoint{0.000000in}{0.000000in}}{%
\pgfpathmoveto{\pgfqpoint{0.000000in}{0.000000in}}%
\pgfpathlineto{\pgfqpoint{0.000000in}{-0.048611in}}%
\pgfusepath{stroke,fill}%
}%
\begin{pgfscope}%
\pgfsys@transformshift{5.444527in}{0.571603in}%
\pgfsys@useobject{currentmarker}{}%
\end{pgfscope}%
\end{pgfscope}%
\begin{pgfscope}%
\definecolor{textcolor}{rgb}{0.000000,0.000000,0.000000}%
\pgfsetstrokecolor{textcolor}%
\pgfsetfillcolor{textcolor}%
\pgftext[x=5.444527in,y=0.474381in,,top]{\color{textcolor}\sffamily\fontsize{10.000000}{12.000000}\selectfont 1.0}%
\end{pgfscope}%
\begin{pgfscope}%
\pgfsetbuttcap%
\pgfsetroundjoin%
\definecolor{currentfill}{rgb}{0.000000,0.000000,0.000000}%
\pgfsetfillcolor{currentfill}%
\pgfsetlinewidth{0.803000pt}%
\definecolor{currentstroke}{rgb}{0.000000,0.000000,0.000000}%
\pgfsetstrokecolor{currentstroke}%
\pgfsetdash{}{0pt}%
\pgfsys@defobject{currentmarker}{\pgfqpoint{0.000000in}{-0.048611in}}{\pgfqpoint{0.000000in}{0.000000in}}{%
\pgfpathmoveto{\pgfqpoint{0.000000in}{0.000000in}}%
\pgfpathlineto{\pgfqpoint{0.000000in}{-0.048611in}}%
\pgfusepath{stroke,fill}%
}%
\begin{pgfscope}%
\pgfsys@transformshift{6.592044in}{0.571603in}%
\pgfsys@useobject{currentmarker}{}%
\end{pgfscope}%
\end{pgfscope}%
\begin{pgfscope}%
\definecolor{textcolor}{rgb}{0.000000,0.000000,0.000000}%
\pgfsetstrokecolor{textcolor}%
\pgfsetfillcolor{textcolor}%
\pgftext[x=6.592044in,y=0.474381in,,top]{\color{textcolor}\sffamily\fontsize{10.000000}{12.000000}\selectfont 1.5}%
\end{pgfscope}%
\begin{pgfscope}%
\pgfsetbuttcap%
\pgfsetroundjoin%
\definecolor{currentfill}{rgb}{0.000000,0.000000,0.000000}%
\pgfsetfillcolor{currentfill}%
\pgfsetlinewidth{0.803000pt}%
\definecolor{currentstroke}{rgb}{0.000000,0.000000,0.000000}%
\pgfsetstrokecolor{currentstroke}%
\pgfsetdash{}{0pt}%
\pgfsys@defobject{currentmarker}{\pgfqpoint{0.000000in}{-0.048611in}}{\pgfqpoint{0.000000in}{0.000000in}}{%
\pgfpathmoveto{\pgfqpoint{0.000000in}{0.000000in}}%
\pgfpathlineto{\pgfqpoint{0.000000in}{-0.048611in}}%
\pgfusepath{stroke,fill}%
}%
\begin{pgfscope}%
\pgfsys@transformshift{7.739560in}{0.571603in}%
\pgfsys@useobject{currentmarker}{}%
\end{pgfscope}%
\end{pgfscope}%
\begin{pgfscope}%
\definecolor{textcolor}{rgb}{0.000000,0.000000,0.000000}%
\pgfsetstrokecolor{textcolor}%
\pgfsetfillcolor{textcolor}%
\pgftext[x=7.739560in,y=0.474381in,,top]{\color{textcolor}\sffamily\fontsize{10.000000}{12.000000}\selectfont 2.0}%
\end{pgfscope}%
\begin{pgfscope}%
\definecolor{textcolor}{rgb}{0.000000,0.000000,0.000000}%
\pgfsetstrokecolor{textcolor}%
\pgfsetfillcolor{textcolor}%
\pgftext[x=4.297010in,y=0.284413in,,top]{\color{textcolor}\sffamily\fontsize{10.000000}{12.000000}\selectfont x}%
\end{pgfscope}%
\begin{pgfscope}%
\pgfsetbuttcap%
\pgfsetroundjoin%
\definecolor{currentfill}{rgb}{0.000000,0.000000,0.000000}%
\pgfsetfillcolor{currentfill}%
\pgfsetlinewidth{0.803000pt}%
\definecolor{currentstroke}{rgb}{0.000000,0.000000,0.000000}%
\pgfsetstrokecolor{currentstroke}%
\pgfsetdash{}{0pt}%
\pgfsys@defobject{currentmarker}{\pgfqpoint{-0.048611in}{0.000000in}}{\pgfqpoint{-0.000000in}{0.000000in}}{%
\pgfpathmoveto{\pgfqpoint{-0.000000in}{0.000000in}}%
\pgfpathlineto{\pgfqpoint{-0.048611in}{0.000000in}}%
\pgfusepath{stroke,fill}%
}%
\begin{pgfscope}%
\pgfsys@transformshift{0.854460in}{0.571603in}%
\pgfsys@useobject{currentmarker}{}%
\end{pgfscope}%
\end{pgfscope}%
\begin{pgfscope}%
\definecolor{textcolor}{rgb}{0.000000,0.000000,0.000000}%
\pgfsetstrokecolor{textcolor}%
\pgfsetfillcolor{textcolor}%
\pgftext[x=0.339968in, y=0.518842in, left, base]{\color{textcolor}\sffamily\fontsize{10.000000}{12.000000}\selectfont \ensuremath{-}1.50}%
\end{pgfscope}%
\begin{pgfscope}%
\pgfsetbuttcap%
\pgfsetroundjoin%
\definecolor{currentfill}{rgb}{0.000000,0.000000,0.000000}%
\pgfsetfillcolor{currentfill}%
\pgfsetlinewidth{0.803000pt}%
\definecolor{currentstroke}{rgb}{0.000000,0.000000,0.000000}%
\pgfsetstrokecolor{currentstroke}%
\pgfsetdash{}{0pt}%
\pgfsys@defobject{currentmarker}{\pgfqpoint{-0.048611in}{0.000000in}}{\pgfqpoint{-0.000000in}{0.000000in}}{%
\pgfpathmoveto{\pgfqpoint{-0.000000in}{0.000000in}}%
\pgfpathlineto{\pgfqpoint{-0.048611in}{0.000000in}}%
\pgfusepath{stroke,fill}%
}%
\begin{pgfscope}%
\pgfsys@transformshift{0.854460in}{1.224808in}%
\pgfsys@useobject{currentmarker}{}%
\end{pgfscope}%
\end{pgfscope}%
\begin{pgfscope}%
\definecolor{textcolor}{rgb}{0.000000,0.000000,0.000000}%
\pgfsetstrokecolor{textcolor}%
\pgfsetfillcolor{textcolor}%
\pgftext[x=0.339968in, y=1.172046in, left, base]{\color{textcolor}\sffamily\fontsize{10.000000}{12.000000}\selectfont \ensuremath{-}1.25}%
\end{pgfscope}%
\begin{pgfscope}%
\pgfsetbuttcap%
\pgfsetroundjoin%
\definecolor{currentfill}{rgb}{0.000000,0.000000,0.000000}%
\pgfsetfillcolor{currentfill}%
\pgfsetlinewidth{0.803000pt}%
\definecolor{currentstroke}{rgb}{0.000000,0.000000,0.000000}%
\pgfsetstrokecolor{currentstroke}%
\pgfsetdash{}{0pt}%
\pgfsys@defobject{currentmarker}{\pgfqpoint{-0.048611in}{0.000000in}}{\pgfqpoint{-0.000000in}{0.000000in}}{%
\pgfpathmoveto{\pgfqpoint{-0.000000in}{0.000000in}}%
\pgfpathlineto{\pgfqpoint{-0.048611in}{0.000000in}}%
\pgfusepath{stroke,fill}%
}%
\begin{pgfscope}%
\pgfsys@transformshift{0.854460in}{1.878012in}%
\pgfsys@useobject{currentmarker}{}%
\end{pgfscope}%
\end{pgfscope}%
\begin{pgfscope}%
\definecolor{textcolor}{rgb}{0.000000,0.000000,0.000000}%
\pgfsetstrokecolor{textcolor}%
\pgfsetfillcolor{textcolor}%
\pgftext[x=0.339968in, y=1.825251in, left, base]{\color{textcolor}\sffamily\fontsize{10.000000}{12.000000}\selectfont \ensuremath{-}1.00}%
\end{pgfscope}%
\begin{pgfscope}%
\pgfsetbuttcap%
\pgfsetroundjoin%
\definecolor{currentfill}{rgb}{0.000000,0.000000,0.000000}%
\pgfsetfillcolor{currentfill}%
\pgfsetlinewidth{0.803000pt}%
\definecolor{currentstroke}{rgb}{0.000000,0.000000,0.000000}%
\pgfsetstrokecolor{currentstroke}%
\pgfsetdash{}{0pt}%
\pgfsys@defobject{currentmarker}{\pgfqpoint{-0.048611in}{0.000000in}}{\pgfqpoint{-0.000000in}{0.000000in}}{%
\pgfpathmoveto{\pgfqpoint{-0.000000in}{0.000000in}}%
\pgfpathlineto{\pgfqpoint{-0.048611in}{0.000000in}}%
\pgfusepath{stroke,fill}%
}%
\begin{pgfscope}%
\pgfsys@transformshift{0.854460in}{2.531217in}%
\pgfsys@useobject{currentmarker}{}%
\end{pgfscope}%
\end{pgfscope}%
\begin{pgfscope}%
\definecolor{textcolor}{rgb}{0.000000,0.000000,0.000000}%
\pgfsetstrokecolor{textcolor}%
\pgfsetfillcolor{textcolor}%
\pgftext[x=0.339968in, y=2.478455in, left, base]{\color{textcolor}\sffamily\fontsize{10.000000}{12.000000}\selectfont \ensuremath{-}0.75}%
\end{pgfscope}%
\begin{pgfscope}%
\pgfsetbuttcap%
\pgfsetroundjoin%
\definecolor{currentfill}{rgb}{0.000000,0.000000,0.000000}%
\pgfsetfillcolor{currentfill}%
\pgfsetlinewidth{0.803000pt}%
\definecolor{currentstroke}{rgb}{0.000000,0.000000,0.000000}%
\pgfsetstrokecolor{currentstroke}%
\pgfsetdash{}{0pt}%
\pgfsys@defobject{currentmarker}{\pgfqpoint{-0.048611in}{0.000000in}}{\pgfqpoint{-0.000000in}{0.000000in}}{%
\pgfpathmoveto{\pgfqpoint{-0.000000in}{0.000000in}}%
\pgfpathlineto{\pgfqpoint{-0.048611in}{0.000000in}}%
\pgfusepath{stroke,fill}%
}%
\begin{pgfscope}%
\pgfsys@transformshift{0.854460in}{3.184421in}%
\pgfsys@useobject{currentmarker}{}%
\end{pgfscope}%
\end{pgfscope}%
\begin{pgfscope}%
\definecolor{textcolor}{rgb}{0.000000,0.000000,0.000000}%
\pgfsetstrokecolor{textcolor}%
\pgfsetfillcolor{textcolor}%
\pgftext[x=0.339968in, y=3.131659in, left, base]{\color{textcolor}\sffamily\fontsize{10.000000}{12.000000}\selectfont \ensuremath{-}0.50}%
\end{pgfscope}%
\begin{pgfscope}%
\pgfsetbuttcap%
\pgfsetroundjoin%
\definecolor{currentfill}{rgb}{0.000000,0.000000,0.000000}%
\pgfsetfillcolor{currentfill}%
\pgfsetlinewidth{0.803000pt}%
\definecolor{currentstroke}{rgb}{0.000000,0.000000,0.000000}%
\pgfsetstrokecolor{currentstroke}%
\pgfsetdash{}{0pt}%
\pgfsys@defobject{currentmarker}{\pgfqpoint{-0.048611in}{0.000000in}}{\pgfqpoint{-0.000000in}{0.000000in}}{%
\pgfpathmoveto{\pgfqpoint{-0.000000in}{0.000000in}}%
\pgfpathlineto{\pgfqpoint{-0.048611in}{0.000000in}}%
\pgfusepath{stroke,fill}%
}%
\begin{pgfscope}%
\pgfsys@transformshift{0.854460in}{3.837625in}%
\pgfsys@useobject{currentmarker}{}%
\end{pgfscope}%
\end{pgfscope}%
\begin{pgfscope}%
\definecolor{textcolor}{rgb}{0.000000,0.000000,0.000000}%
\pgfsetstrokecolor{textcolor}%
\pgfsetfillcolor{textcolor}%
\pgftext[x=0.339968in, y=3.784864in, left, base]{\color{textcolor}\sffamily\fontsize{10.000000}{12.000000}\selectfont \ensuremath{-}0.25}%
\end{pgfscope}%
\begin{pgfscope}%
\pgfsetbuttcap%
\pgfsetroundjoin%
\definecolor{currentfill}{rgb}{0.000000,0.000000,0.000000}%
\pgfsetfillcolor{currentfill}%
\pgfsetlinewidth{0.803000pt}%
\definecolor{currentstroke}{rgb}{0.000000,0.000000,0.000000}%
\pgfsetstrokecolor{currentstroke}%
\pgfsetdash{}{0pt}%
\pgfsys@defobject{currentmarker}{\pgfqpoint{-0.048611in}{0.000000in}}{\pgfqpoint{-0.000000in}{0.000000in}}{%
\pgfpathmoveto{\pgfqpoint{-0.000000in}{0.000000in}}%
\pgfpathlineto{\pgfqpoint{-0.048611in}{0.000000in}}%
\pgfusepath{stroke,fill}%
}%
\begin{pgfscope}%
\pgfsys@transformshift{0.854460in}{4.490830in}%
\pgfsys@useobject{currentmarker}{}%
\end{pgfscope}%
\end{pgfscope}%
\begin{pgfscope}%
\definecolor{textcolor}{rgb}{0.000000,0.000000,0.000000}%
\pgfsetstrokecolor{textcolor}%
\pgfsetfillcolor{textcolor}%
\pgftext[x=0.447993in, y=4.438068in, left, base]{\color{textcolor}\sffamily\fontsize{10.000000}{12.000000}\selectfont 0.00}%
\end{pgfscope}%
\begin{pgfscope}%
\pgfsetbuttcap%
\pgfsetroundjoin%
\definecolor{currentfill}{rgb}{0.000000,0.000000,0.000000}%
\pgfsetfillcolor{currentfill}%
\pgfsetlinewidth{0.803000pt}%
\definecolor{currentstroke}{rgb}{0.000000,0.000000,0.000000}%
\pgfsetstrokecolor{currentstroke}%
\pgfsetdash{}{0pt}%
\pgfsys@defobject{currentmarker}{\pgfqpoint{-0.048611in}{0.000000in}}{\pgfqpoint{-0.000000in}{0.000000in}}{%
\pgfpathmoveto{\pgfqpoint{-0.000000in}{0.000000in}}%
\pgfpathlineto{\pgfqpoint{-0.048611in}{0.000000in}}%
\pgfusepath{stroke,fill}%
}%
\begin{pgfscope}%
\pgfsys@transformshift{0.854460in}{5.144034in}%
\pgfsys@useobject{currentmarker}{}%
\end{pgfscope}%
\end{pgfscope}%
\begin{pgfscope}%
\definecolor{textcolor}{rgb}{0.000000,0.000000,0.000000}%
\pgfsetstrokecolor{textcolor}%
\pgfsetfillcolor{textcolor}%
\pgftext[x=0.447993in, y=5.091273in, left, base]{\color{textcolor}\sffamily\fontsize{10.000000}{12.000000}\selectfont 0.25}%
\end{pgfscope}%
\begin{pgfscope}%
\pgfsetbuttcap%
\pgfsetroundjoin%
\definecolor{currentfill}{rgb}{0.000000,0.000000,0.000000}%
\pgfsetfillcolor{currentfill}%
\pgfsetlinewidth{0.803000pt}%
\definecolor{currentstroke}{rgb}{0.000000,0.000000,0.000000}%
\pgfsetstrokecolor{currentstroke}%
\pgfsetdash{}{0pt}%
\pgfsys@defobject{currentmarker}{\pgfqpoint{-0.048611in}{0.000000in}}{\pgfqpoint{-0.000000in}{0.000000in}}{%
\pgfpathmoveto{\pgfqpoint{-0.000000in}{0.000000in}}%
\pgfpathlineto{\pgfqpoint{-0.048611in}{0.000000in}}%
\pgfusepath{stroke,fill}%
}%
\begin{pgfscope}%
\pgfsys@transformshift{0.854460in}{5.797238in}%
\pgfsys@useobject{currentmarker}{}%
\end{pgfscope}%
\end{pgfscope}%
\begin{pgfscope}%
\definecolor{textcolor}{rgb}{0.000000,0.000000,0.000000}%
\pgfsetstrokecolor{textcolor}%
\pgfsetfillcolor{textcolor}%
\pgftext[x=0.447993in, y=5.744477in, left, base]{\color{textcolor}\sffamily\fontsize{10.000000}{12.000000}\selectfont 0.50}%
\end{pgfscope}%
\begin{pgfscope}%
\definecolor{textcolor}{rgb}{0.000000,0.000000,0.000000}%
\pgfsetstrokecolor{textcolor}%
\pgfsetfillcolor{textcolor}%
\pgftext[x=0.284413in,y=3.184421in,,bottom,rotate=90.000000]{\color{textcolor}\sffamily\fontsize{10.000000}{12.000000}\selectfont y}%
\end{pgfscope}%
\begin{pgfscope}%
\pgfpathrectangle{\pgfqpoint{0.854460in}{0.571603in}}{\pgfqpoint{6.885100in}{5.225635in}}%
\pgfusepath{clip}%
\pgfsetbuttcap%
\pgfsetroundjoin%
\pgfsetlinewidth{1.505625pt}%
\definecolor{currentstroke}{rgb}{0.273809,0.031497,0.358853}%
\pgfsetstrokecolor{currentstroke}%
\pgfsetdash{}{0pt}%
\pgfpathmoveto{\pgfqpoint{4.693490in}{2.752196in}}%
\pgfpathlineto{\pgfqpoint{4.487302in}{2.947666in}}%
\pgfpathlineto{\pgfqpoint{4.301452in}{3.118772in}}%
\pgfpathlineto{\pgfqpoint{4.210514in}{3.200173in}}%
\pgfpathlineto{\pgfqpoint{4.091697in}{3.302589in}}%
\pgfpathlineto{\pgfqpoint{4.028469in}{3.355107in}}%
\pgfpathlineto{\pgfqpoint{3.962943in}{3.407626in}}%
\pgfpathlineto{\pgfqpoint{3.894284in}{3.460145in}}%
\pgfpathlineto{\pgfqpoint{3.821267in}{3.512664in}}%
\pgfpathlineto{\pgfqpoint{3.760734in}{3.553416in}}%
\pgfpathlineto{\pgfqpoint{3.691537in}{3.596085in}}%
\pgfpathlineto{\pgfqpoint{3.622340in}{3.634110in}}%
\pgfpathlineto{\pgfqpoint{3.587741in}{3.651215in}}%
\pgfpathlineto{\pgfqpoint{3.545095in}{3.670221in}}%
\pgfpathlineto{\pgfqpoint{3.518544in}{3.681140in}}%
\pgfpathlineto{\pgfqpoint{3.475541in}{3.696481in}}%
\pgfpathlineto{\pgfqpoint{3.449347in}{3.704798in}}%
\pgfpathlineto{\pgfqpoint{3.414749in}{3.714023in}}%
\pgfpathlineto{\pgfqpoint{3.371528in}{3.722740in}}%
\pgfpathlineto{\pgfqpoint{3.345552in}{3.726750in}}%
\pgfpathlineto{\pgfqpoint{3.310953in}{3.729951in}}%
\pgfpathlineto{\pgfqpoint{3.276355in}{3.730878in}}%
\pgfpathlineto{\pgfqpoint{3.241756in}{3.729348in}}%
\pgfpathlineto{\pgfqpoint{3.195109in}{3.722740in}}%
\pgfpathlineto{\pgfqpoint{3.172559in}{3.717967in}}%
\pgfpathlineto{\pgfqpoint{3.137961in}{3.707471in}}%
\pgfpathlineto{\pgfqpoint{3.103362in}{3.693333in}}%
\pgfpathlineto{\pgfqpoint{3.061734in}{3.670221in}}%
\pgfpathlineto{\pgfqpoint{3.025626in}{3.643962in}}%
\pgfpathlineto{\pgfqpoint{2.996678in}{3.617702in}}%
\pgfpathlineto{\pgfqpoint{2.964968in}{3.581237in}}%
\pgfpathlineto{\pgfqpoint{2.953486in}{3.565183in}}%
\pgfpathlineto{\pgfqpoint{2.930370in}{3.528032in}}%
\pgfpathlineto{\pgfqpoint{2.922461in}{3.512664in}}%
\pgfpathlineto{\pgfqpoint{2.910527in}{3.486405in}}%
\pgfpathlineto{\pgfqpoint{2.895771in}{3.446805in}}%
\pgfpathlineto{\pgfqpoint{2.891835in}{3.433886in}}%
\pgfpathlineto{\pgfqpoint{2.885124in}{3.407626in}}%
\pgfpathlineto{\pgfqpoint{2.879744in}{3.381367in}}%
\pgfpathlineto{\pgfqpoint{2.875643in}{3.355107in}}%
\pgfpathlineto{\pgfqpoint{2.872771in}{3.328848in}}%
\pgfpathlineto{\pgfqpoint{2.871076in}{3.302589in}}%
\pgfpathlineto{\pgfqpoint{2.871032in}{3.250070in}}%
\pgfpathlineto{\pgfqpoint{2.875144in}{3.197551in}}%
\pgfpathlineto{\pgfqpoint{2.883077in}{3.145032in}}%
\pgfpathlineto{\pgfqpoint{2.894521in}{3.092513in}}%
\pgfpathlineto{\pgfqpoint{2.901764in}{3.066253in}}%
\pgfpathlineto{\pgfqpoint{2.918736in}{3.013734in}}%
\pgfpathlineto{\pgfqpoint{2.939027in}{2.961215in}}%
\pgfpathlineto{\pgfqpoint{2.964968in}{2.903945in}}%
\pgfpathlineto{\pgfqpoint{2.989774in}{2.856177in}}%
\pgfpathlineto{\pgfqpoint{3.020190in}{2.803659in}}%
\pgfpathlineto{\pgfqpoint{3.053941in}{2.751140in}}%
\pgfpathlineto{\pgfqpoint{3.091070in}{2.698621in}}%
\pgfpathlineto{\pgfqpoint{3.131592in}{2.646102in}}%
\pgfpathlineto{\pgfqpoint{3.175628in}{2.593583in}}%
\pgfpathlineto{\pgfqpoint{3.223430in}{2.541064in}}%
\pgfpathlineto{\pgfqpoint{3.276355in}{2.486863in}}%
\pgfpathlineto{\pgfqpoint{3.329939in}{2.436026in}}%
\pgfpathlineto{\pgfqpoint{3.388981in}{2.383507in}}%
\pgfpathlineto{\pgfqpoint{3.452065in}{2.330988in}}%
\pgfpathlineto{\pgfqpoint{3.519478in}{2.278469in}}%
\pgfpathlineto{\pgfqpoint{3.591473in}{2.225950in}}%
\pgfpathlineto{\pgfqpoint{3.668262in}{2.173431in}}%
\pgfpathlineto{\pgfqpoint{3.750011in}{2.120912in}}%
\pgfpathlineto{\pgfqpoint{3.837165in}{2.068393in}}%
\pgfpathlineto{\pgfqpoint{3.933726in}{2.013964in}}%
\pgfpathlineto{\pgfqpoint{4.037522in}{1.959499in}}%
\pgfpathlineto{\pgfqpoint{4.141317in}{1.908695in}}%
\pgfpathlineto{\pgfqpoint{4.252160in}{1.858318in}}%
\pgfpathlineto{\pgfqpoint{4.348908in}{1.817507in}}%
\pgfpathlineto{\pgfqpoint{4.418105in}{1.789905in}}%
\pgfpathlineto{\pgfqpoint{4.487302in}{1.763688in}}%
\pgfpathlineto{\pgfqpoint{4.556499in}{1.738849in}}%
\pgfpathlineto{\pgfqpoint{4.625696in}{1.715384in}}%
\pgfpathlineto{\pgfqpoint{4.694893in}{1.693287in}}%
\pgfpathlineto{\pgfqpoint{4.798688in}{1.662976in}}%
\pgfpathlineto{\pgfqpoint{4.867885in}{1.644559in}}%
\pgfpathlineto{\pgfqpoint{4.971681in}{1.620247in}}%
\pgfpathlineto{\pgfqpoint{5.040878in}{1.606670in}}%
\pgfpathlineto{\pgfqpoint{5.110075in}{1.594890in}}%
\pgfpathlineto{\pgfqpoint{5.179272in}{1.586225in}}%
\pgfpathlineto{\pgfqpoint{5.248469in}{1.580612in}}%
\pgfpathlineto{\pgfqpoint{5.283067in}{1.579307in}}%
\pgfpathlineto{\pgfqpoint{5.317666in}{1.579274in}}%
\pgfpathlineto{\pgfqpoint{5.352264in}{1.580787in}}%
\pgfpathlineto{\pgfqpoint{5.386863in}{1.584210in}}%
\pgfpathlineto{\pgfqpoint{5.421461in}{1.590032in}}%
\pgfpathlineto{\pgfqpoint{5.456060in}{1.599516in}}%
\pgfpathlineto{\pgfqpoint{5.490658in}{1.615167in}}%
\pgfpathlineto{\pgfqpoint{5.501855in}{1.621982in}}%
\pgfpathlineto{\pgfqpoint{5.529755in}{1.648242in}}%
\pgfpathlineto{\pgfqpoint{5.543775in}{1.674501in}}%
\pgfpathlineto{\pgfqpoint{5.549927in}{1.700761in}}%
\pgfpathlineto{\pgfqpoint{5.550404in}{1.727020in}}%
\pgfpathlineto{\pgfqpoint{5.546669in}{1.753280in}}%
\pgfpathlineto{\pgfqpoint{5.539737in}{1.779539in}}%
\pgfpathlineto{\pgfqpoint{5.525257in}{1.817145in}}%
\pgfpathlineto{\pgfqpoint{5.518642in}{1.832058in}}%
\pgfpathlineto{\pgfqpoint{5.490333in}{1.884577in}}%
\pgfpathlineto{\pgfqpoint{5.456060in}{1.937834in}}%
\pgfpathlineto{\pgfqpoint{5.418770in}{1.989615in}}%
\pgfpathlineto{\pgfqpoint{5.377757in}{2.042134in}}%
\pgfpathlineto{\pgfqpoint{5.334286in}{2.094653in}}%
\pgfpathlineto{\pgfqpoint{5.283067in}{2.153704in}}%
\pgfpathlineto{\pgfqpoint{5.213870in}{2.230065in}}%
\pgfpathlineto{\pgfqpoint{5.118405in}{2.330988in}}%
\pgfpathlineto{\pgfqpoint{5.006279in}{2.445450in}}%
\pgfpathlineto{\pgfqpoint{4.971767in}{2.480062in}}%
\pgfpathlineto{\pgfqpoint{4.971767in}{2.480062in}}%
\pgfusepath{stroke}%
\end{pgfscope}%
\begin{pgfscope}%
\pgfpathrectangle{\pgfqpoint{0.854460in}{0.571603in}}{\pgfqpoint{6.885100in}{5.225635in}}%
\pgfusepath{clip}%
\pgfsetbuttcap%
\pgfsetroundjoin%
\pgfsetlinewidth{1.505625pt}%
\definecolor{currentstroke}{rgb}{0.279566,0.067836,0.391917}%
\pgfsetstrokecolor{currentstroke}%
\pgfsetdash{}{0pt}%
\pgfpathmoveto{\pgfqpoint{5.464147in}{2.465205in}}%
\pgfpathlineto{\pgfqpoint{5.276975in}{2.646102in}}%
\pgfpathlineto{\pgfqpoint{5.110075in}{2.810910in}}%
\pgfpathlineto{\pgfqpoint{4.986978in}{2.934956in}}%
\pgfpathlineto{\pgfqpoint{4.809030in}{3.118772in}}%
\pgfpathlineto{\pgfqpoint{4.611664in}{3.328848in}}%
\pgfpathlineto{\pgfqpoint{4.383507in}{3.578670in}}%
\pgfpathlineto{\pgfqpoint{4.229098in}{3.749000in}}%
\pgfpathlineto{\pgfqpoint{4.132948in}{3.854037in}}%
\pgfpathlineto{\pgfqpoint{4.034497in}{3.959075in}}%
\pgfpathlineto{\pgfqpoint{3.957779in}{4.037854in}}%
\pgfpathlineto{\pgfqpoint{3.899128in}{4.095939in}}%
\pgfpathlineto{\pgfqpoint{3.820687in}{4.169151in}}%
\pgfpathlineto{\pgfqpoint{3.760662in}{4.221670in}}%
\pgfpathlineto{\pgfqpoint{3.691537in}{4.277257in}}%
\pgfpathlineto{\pgfqpoint{3.656938in}{4.302993in}}%
\pgfpathlineto{\pgfqpoint{3.622340in}{4.327249in}}%
\pgfpathlineto{\pgfqpoint{3.582807in}{4.352967in}}%
\pgfpathlineto{\pgfqpoint{3.538659in}{4.379227in}}%
\pgfpathlineto{\pgfqpoint{3.518544in}{4.390548in}}%
\pgfpathlineto{\pgfqpoint{3.483946in}{4.408361in}}%
\pgfpathlineto{\pgfqpoint{3.431848in}{4.431746in}}%
\pgfpathlineto{\pgfqpoint{3.414749in}{4.438840in}}%
\pgfpathlineto{\pgfqpoint{3.359489in}{4.458005in}}%
\pgfpathlineto{\pgfqpoint{3.345552in}{4.462380in}}%
\pgfpathlineto{\pgfqpoint{3.310953in}{4.471506in}}%
\pgfpathlineto{\pgfqpoint{3.276355in}{4.478889in}}%
\pgfpathlineto{\pgfqpoint{3.241756in}{4.484516in}}%
\pgfpathlineto{\pgfqpoint{3.207158in}{4.488346in}}%
\pgfpathlineto{\pgfqpoint{3.172559in}{4.490411in}}%
\pgfpathlineto{\pgfqpoint{3.137961in}{4.490706in}}%
\pgfpathlineto{\pgfqpoint{3.103362in}{4.489220in}}%
\pgfpathlineto{\pgfqpoint{3.057285in}{4.484265in}}%
\pgfpathlineto{\pgfqpoint{3.034165in}{4.480836in}}%
\pgfpathlineto{\pgfqpoint{2.999567in}{4.473872in}}%
\pgfpathlineto{\pgfqpoint{2.964968in}{4.465061in}}%
\pgfpathlineto{\pgfqpoint{2.930370in}{4.454341in}}%
\pgfpathlineto{\pgfqpoint{2.895771in}{4.441614in}}%
\pgfpathlineto{\pgfqpoint{2.861173in}{4.426884in}}%
\pgfpathlineto{\pgfqpoint{2.818279in}{4.405486in}}%
\pgfpathlineto{\pgfqpoint{2.773051in}{4.379227in}}%
\pgfpathlineto{\pgfqpoint{2.757377in}{4.369383in}}%
\pgfpathlineto{\pgfqpoint{2.722779in}{4.345348in}}%
\pgfpathlineto{\pgfqpoint{2.688180in}{4.318534in}}%
\pgfpathlineto{\pgfqpoint{2.653582in}{4.288658in}}%
\pgfpathlineto{\pgfqpoint{2.611862in}{4.247930in}}%
\pgfpathlineto{\pgfqpoint{2.584385in}{4.218194in}}%
\pgfpathlineto{\pgfqpoint{2.544322in}{4.169151in}}%
\pgfpathlineto{\pgfqpoint{2.507062in}{4.116632in}}%
\pgfpathlineto{\pgfqpoint{2.474731in}{4.064113in}}%
\pgfpathlineto{\pgfqpoint{2.445991in}{4.010162in}}%
\pgfpathlineto{\pgfqpoint{2.422827in}{3.959075in}}%
\pgfpathlineto{\pgfqpoint{2.402303in}{3.906556in}}%
\pgfpathlineto{\pgfqpoint{2.385058in}{3.854037in}}%
\pgfpathlineto{\pgfqpoint{2.370848in}{3.801519in}}%
\pgfpathlineto{\pgfqpoint{2.359632in}{3.749000in}}%
\pgfpathlineto{\pgfqpoint{2.351051in}{3.696481in}}%
\pgfpathlineto{\pgfqpoint{2.345084in}{3.643962in}}%
\pgfpathlineto{\pgfqpoint{2.341718in}{3.591443in}}%
\pgfpathlineto{\pgfqpoint{2.340906in}{3.538924in}}%
\pgfpathlineto{\pgfqpoint{2.342508in}{3.486405in}}%
\pgfpathlineto{\pgfqpoint{2.346598in}{3.433886in}}%
\pgfpathlineto{\pgfqpoint{2.353065in}{3.381367in}}%
\pgfpathlineto{\pgfqpoint{2.361847in}{3.328848in}}%
\pgfpathlineto{\pgfqpoint{2.372875in}{3.276329in}}%
\pgfpathlineto{\pgfqpoint{2.386394in}{3.223810in}}%
\pgfpathlineto{\pgfqpoint{2.402204in}{3.171291in}}%
\pgfpathlineto{\pgfqpoint{2.420376in}{3.118772in}}%
\pgfpathlineto{\pgfqpoint{2.445991in}{3.054164in}}%
\pgfpathlineto{\pgfqpoint{2.463913in}{3.013734in}}%
\pgfpathlineto{\pgfqpoint{2.489258in}{2.961215in}}%
\pgfpathlineto{\pgfqpoint{2.517024in}{2.908696in}}%
\pgfpathlineto{\pgfqpoint{2.549787in}{2.852211in}}%
\pgfpathlineto{\pgfqpoint{2.584385in}{2.797401in}}%
\pgfpathlineto{\pgfqpoint{2.618983in}{2.746528in}}%
\pgfpathlineto{\pgfqpoint{2.653778in}{2.698621in}}%
\pgfpathlineto{\pgfqpoint{2.694650in}{2.646102in}}%
\pgfpathlineto{\pgfqpoint{2.738253in}{2.593583in}}%
\pgfpathlineto{\pgfqpoint{2.791976in}{2.533028in}}%
\pgfpathlineto{\pgfqpoint{2.833868in}{2.488545in}}%
\pgfpathlineto{\pgfqpoint{2.895771in}{2.426703in}}%
\pgfpathlineto{\pgfqpoint{2.941465in}{2.383507in}}%
\pgfpathlineto{\pgfqpoint{2.999829in}{2.330988in}}%
\pgfpathlineto{\pgfqpoint{3.068764in}{2.272609in}}%
\pgfpathlineto{\pgfqpoint{3.137961in}{2.217212in}}%
\pgfpathlineto{\pgfqpoint{3.207158in}{2.164667in}}%
\pgfpathlineto{\pgfqpoint{3.276355in}{2.114656in}}%
\pgfpathlineto{\pgfqpoint{3.345552in}{2.066891in}}%
\pgfpathlineto{\pgfqpoint{3.423179in}{2.015874in}}%
\pgfpathlineto{\pgfqpoint{3.518544in}{1.956401in}}%
\pgfpathlineto{\pgfqpoint{3.595150in}{1.910836in}}%
\pgfpathlineto{\pgfqpoint{3.691537in}{1.856197in}}%
\pgfpathlineto{\pgfqpoint{3.795332in}{1.800400in}}%
\pgfpathlineto{\pgfqpoint{3.899128in}{1.747369in}}%
\pgfpathlineto{\pgfqpoint{4.002923in}{1.696872in}}%
\pgfpathlineto{\pgfqpoint{4.107745in}{1.648242in}}%
\pgfpathlineto{\pgfqpoint{4.245113in}{1.588081in}}%
\pgfpathlineto{\pgfqpoint{4.352861in}{1.543204in}}%
\pgfpathlineto{\pgfqpoint{4.487302in}{1.490185in}}%
\pgfpathlineto{\pgfqpoint{4.627391in}{1.438166in}}%
\pgfpathlineto{\pgfqpoint{4.778371in}{1.385647in}}%
\pgfpathlineto{\pgfqpoint{4.937082in}{1.334137in}}%
\pgfpathlineto{\pgfqpoint{5.040878in}{1.302608in}}%
\pgfpathlineto{\pgfqpoint{5.179272in}{1.263098in}}%
\pgfpathlineto{\pgfqpoint{5.283067in}{1.235362in}}%
\pgfpathlineto{\pgfqpoint{5.386863in}{1.209382in}}%
\pgfpathlineto{\pgfqpoint{5.490658in}{1.185245in}}%
\pgfpathlineto{\pgfqpoint{5.594454in}{1.163051in}}%
\pgfpathlineto{\pgfqpoint{5.698249in}{1.142919in}}%
\pgfpathlineto{\pgfqpoint{5.815310in}{1.123052in}}%
\pgfpathlineto{\pgfqpoint{5.905840in}{1.110326in}}%
\pgfpathlineto{\pgfqpoint{6.009636in}{1.098465in}}%
\pgfpathlineto{\pgfqpoint{6.078833in}{1.092968in}}%
\pgfpathlineto{\pgfqpoint{6.148030in}{1.089682in}}%
\pgfpathlineto{\pgfqpoint{6.217227in}{1.089078in}}%
\pgfpathlineto{\pgfqpoint{6.286424in}{1.092004in}}%
\pgfpathlineto{\pgfqpoint{6.333897in}{1.096793in}}%
\pgfpathlineto{\pgfqpoint{6.355621in}{1.100091in}}%
\pgfpathlineto{\pgfqpoint{6.390219in}{1.107191in}}%
\pgfpathlineto{\pgfqpoint{6.424818in}{1.116718in}}%
\pgfpathlineto{\pgfqpoint{6.442960in}{1.123052in}}%
\pgfpathlineto{\pgfqpoint{6.459416in}{1.130437in}}%
\pgfpathlineto{\pgfqpoint{6.494015in}{1.150629in}}%
\pgfpathlineto{\pgfqpoint{6.519808in}{1.175571in}}%
\pgfpathlineto{\pgfqpoint{6.535420in}{1.201831in}}%
\pgfpathlineto{\pgfqpoint{6.542910in}{1.228090in}}%
\pgfpathlineto{\pgfqpoint{6.544695in}{1.254350in}}%
\pgfpathlineto{\pgfqpoint{6.542073in}{1.280609in}}%
\pgfpathlineto{\pgfqpoint{6.535991in}{1.306869in}}%
\pgfpathlineto{\pgfqpoint{6.527090in}{1.333128in}}%
\pgfpathlineto{\pgfqpoint{6.515584in}{1.359388in}}%
\pgfpathlineto{\pgfqpoint{6.494015in}{1.399894in}}%
\pgfpathlineto{\pgfqpoint{6.487179in}{1.411906in}}%
\pgfpathlineto{\pgfqpoint{6.452752in}{1.464425in}}%
\pgfpathlineto{\pgfqpoint{6.413729in}{1.516944in}}%
\pgfpathlineto{\pgfqpoint{6.371167in}{1.569463in}}%
\pgfpathlineto{\pgfqpoint{6.321022in}{1.627324in}}%
\pgfpathlineto{\pgfqpoint{6.251825in}{1.702752in}}%
\pgfpathlineto{\pgfqpoint{6.177834in}{1.779539in}}%
\pgfpathlineto{\pgfqpoint{6.072639in}{1.884577in}}%
\pgfpathlineto{\pgfqpoint{5.964404in}{1.989615in}}%
\pgfpathlineto{\pgfqpoint{5.799151in}{2.147172in}}%
\pgfpathlineto{\pgfqpoint{5.746269in}{2.197200in}}%
\pgfpathlineto{\pgfqpoint{5.746269in}{2.197200in}}%
\pgfusepath{stroke}%
\end{pgfscope}%
\begin{pgfscope}%
\pgfpathrectangle{\pgfqpoint{0.854460in}{0.571603in}}{\pgfqpoint{6.885100in}{5.225635in}}%
\pgfusepath{clip}%
\pgfsetbuttcap%
\pgfsetroundjoin%
\pgfsetlinewidth{1.505625pt}%
\definecolor{currentstroke}{rgb}{0.282656,0.100196,0.422160}%
\pgfsetstrokecolor{currentstroke}%
\pgfsetdash{}{0pt}%
\pgfpathmoveto{\pgfqpoint{6.376639in}{1.960181in}}%
\pgfpathlineto{\pgfqpoint{6.148030in}{2.162336in}}%
\pgfpathlineto{\pgfqpoint{5.990007in}{2.304729in}}%
\pgfpathlineto{\pgfqpoint{5.836643in}{2.446027in}}%
\pgfpathlineto{\pgfqpoint{5.708101in}{2.567323in}}%
\pgfpathlineto{\pgfqpoint{5.572622in}{2.698621in}}%
\pgfpathlineto{\pgfqpoint{5.441210in}{2.829918in}}%
\pgfpathlineto{\pgfqpoint{5.314114in}{2.961215in}}%
\pgfpathlineto{\pgfqpoint{5.213870in}{3.068103in}}%
\pgfpathlineto{\pgfqpoint{5.119787in}{3.171291in}}%
\pgfpathlineto{\pgfqpoint{5.004175in}{3.302589in}}%
\pgfpathlineto{\pgfqpoint{4.914737in}{3.407626in}}%
\pgfpathlineto{\pgfqpoint{4.798688in}{3.549181in}}%
\pgfpathlineto{\pgfqpoint{4.723487in}{3.643962in}}%
\pgfpathlineto{\pgfqpoint{4.642473in}{3.749000in}}%
\pgfpathlineto{\pgfqpoint{4.525106in}{3.906556in}}%
\pgfpathlineto{\pgfqpoint{4.337035in}{4.169151in}}%
\pgfpathlineto{\pgfqpoint{4.175916in}{4.396456in}}%
\pgfpathlineto{\pgfqpoint{4.071933in}{4.536784in}}%
\pgfpathlineto{\pgfqpoint{4.002923in}{4.624135in}}%
\pgfpathlineto{\pgfqpoint{3.943264in}{4.694341in}}%
\pgfpathlineto{\pgfqpoint{3.919702in}{4.720600in}}%
\pgfpathlineto{\pgfqpoint{3.869991in}{4.773119in}}%
\pgfpathlineto{\pgfqpoint{3.843325in}{4.799378in}}%
\pgfpathlineto{\pgfqpoint{3.815344in}{4.825638in}}%
\pgfpathlineto{\pgfqpoint{3.785744in}{4.851897in}}%
\pgfpathlineto{\pgfqpoint{3.754129in}{4.878157in}}%
\pgfpathlineto{\pgfqpoint{3.719979in}{4.904416in}}%
\pgfpathlineto{\pgfqpoint{3.682592in}{4.930676in}}%
\pgfpathlineto{\pgfqpoint{3.641003in}{4.956935in}}%
\pgfpathlineto{\pgfqpoint{3.622340in}{4.967922in}}%
\pgfpathlineto{\pgfqpoint{3.587741in}{4.986405in}}%
\pgfpathlineto{\pgfqpoint{3.537642in}{5.009454in}}%
\pgfpathlineto{\pgfqpoint{3.518544in}{5.017407in}}%
\pgfpathlineto{\pgfqpoint{3.466252in}{5.035714in}}%
\pgfpathlineto{\pgfqpoint{3.449347in}{5.040973in}}%
\pgfpathlineto{\pgfqpoint{3.414749in}{5.050110in}}%
\pgfpathlineto{\pgfqpoint{3.380150in}{5.057610in}}%
\pgfpathlineto{\pgfqpoint{3.345552in}{5.063497in}}%
\pgfpathlineto{\pgfqpoint{3.310953in}{5.067800in}}%
\pgfpathlineto{\pgfqpoint{3.276355in}{5.070612in}}%
\pgfpathlineto{\pgfqpoint{3.241756in}{5.071974in}}%
\pgfpathlineto{\pgfqpoint{3.207158in}{5.071927in}}%
\pgfpathlineto{\pgfqpoint{3.172559in}{5.070507in}}%
\pgfpathlineto{\pgfqpoint{3.137961in}{5.067750in}}%
\pgfpathlineto{\pgfqpoint{3.092280in}{5.061973in}}%
\pgfpathlineto{\pgfqpoint{3.068764in}{5.058310in}}%
\pgfpathlineto{\pgfqpoint{3.034165in}{5.051642in}}%
\pgfpathlineto{\pgfqpoint{2.969073in}{5.035714in}}%
\pgfpathlineto{\pgfqpoint{2.964968in}{5.034626in}}%
\pgfpathlineto{\pgfqpoint{2.930370in}{5.024185in}}%
\pgfpathlineto{\pgfqpoint{2.887385in}{5.009454in}}%
\pgfpathlineto{\pgfqpoint{2.826574in}{4.985518in}}%
\pgfpathlineto{\pgfqpoint{2.764799in}{4.956935in}}%
\pgfpathlineto{\pgfqpoint{2.757377in}{4.953325in}}%
\pgfpathlineto{\pgfqpoint{2.714605in}{4.930676in}}%
\pgfpathlineto{\pgfqpoint{2.669262in}{4.904416in}}%
\pgfpathlineto{\pgfqpoint{2.627547in}{4.878157in}}%
\pgfpathlineto{\pgfqpoint{2.588938in}{4.851897in}}%
\pgfpathlineto{\pgfqpoint{2.553008in}{4.825638in}}%
\pgfpathlineto{\pgfqpoint{2.519403in}{4.799378in}}%
\pgfpathlineto{\pgfqpoint{2.487826in}{4.773119in}}%
\pgfpathlineto{\pgfqpoint{2.458031in}{4.746860in}}%
\pgfpathlineto{\pgfqpoint{2.429805in}{4.720600in}}%
\pgfpathlineto{\pgfqpoint{2.402972in}{4.694341in}}%
\pgfpathlineto{\pgfqpoint{2.353386in}{4.641822in}}%
\pgfpathlineto{\pgfqpoint{2.330370in}{4.615562in}}%
\pgfpathlineto{\pgfqpoint{2.287543in}{4.563043in}}%
\pgfpathlineto{\pgfqpoint{2.267524in}{4.536784in}}%
\pgfpathlineto{\pgfqpoint{2.230399in}{4.484265in}}%
\pgfpathlineto{\pgfqpoint{2.196589in}{4.431746in}}%
\pgfpathlineto{\pgfqpoint{2.165831in}{4.379227in}}%
\pgfpathlineto{\pgfqpoint{2.134605in}{4.319938in}}%
\pgfpathlineto{\pgfqpoint{2.112976in}{4.274189in}}%
\pgfpathlineto{\pgfqpoint{2.090391in}{4.221670in}}%
\pgfpathlineto{\pgfqpoint{2.065408in}{4.155646in}}%
\pgfpathlineto{\pgfqpoint{2.052498in}{4.116632in}}%
\pgfpathlineto{\pgfqpoint{2.036912in}{4.064113in}}%
\pgfpathlineto{\pgfqpoint{2.023570in}{4.011594in}}%
\pgfpathlineto{\pgfqpoint{2.012385in}{3.959075in}}%
\pgfpathlineto{\pgfqpoint{2.003177in}{3.906556in}}%
\pgfpathlineto{\pgfqpoint{1.995968in}{3.854037in}}%
\pgfpathlineto{\pgfqpoint{1.990906in}{3.801519in}}%
\pgfpathlineto{\pgfqpoint{1.987780in}{3.749000in}}%
\pgfpathlineto{\pgfqpoint{1.986587in}{3.696481in}}%
\pgfpathlineto{\pgfqpoint{1.987320in}{3.643962in}}%
\pgfpathlineto{\pgfqpoint{1.989968in}{3.591443in}}%
\pgfpathlineto{\pgfqpoint{1.996211in}{3.524015in}}%
\pgfpathlineto{\pgfqpoint{2.001074in}{3.486405in}}%
\pgfpathlineto{\pgfqpoint{2.009586in}{3.433886in}}%
\pgfpathlineto{\pgfqpoint{2.019970in}{3.381367in}}%
\pgfpathlineto{\pgfqpoint{2.032234in}{3.328848in}}%
\pgfpathlineto{\pgfqpoint{2.046648in}{3.276329in}}%
\pgfpathlineto{\pgfqpoint{2.065408in}{3.216257in}}%
\pgfpathlineto{\pgfqpoint{2.081263in}{3.171291in}}%
\pgfpathlineto{\pgfqpoint{2.101516in}{3.118772in}}%
\pgfpathlineto{\pgfqpoint{2.134605in}{3.042848in}}%
\pgfpathlineto{\pgfqpoint{2.148528in}{3.013734in}}%
\pgfpathlineto{\pgfqpoint{2.175124in}{2.961215in}}%
\pgfpathlineto{\pgfqpoint{2.203872in}{2.908696in}}%
\pgfpathlineto{\pgfqpoint{2.238400in}{2.850573in}}%
\pgfpathlineto{\pgfqpoint{2.272999in}{2.796444in}}%
\pgfpathlineto{\pgfqpoint{2.307597in}{2.745754in}}%
\pgfpathlineto{\pgfqpoint{2.342196in}{2.697945in}}%
\pgfpathlineto{\pgfqpoint{2.382077in}{2.646102in}}%
\pgfpathlineto{\pgfqpoint{2.424928in}{2.593583in}}%
\pgfpathlineto{\pgfqpoint{2.480590in}{2.529510in}}%
\pgfpathlineto{\pgfqpoint{2.518105in}{2.488545in}}%
\pgfpathlineto{\pgfqpoint{2.584385in}{2.420356in}}%
\pgfpathlineto{\pgfqpoint{2.621981in}{2.383507in}}%
\pgfpathlineto{\pgfqpoint{2.688180in}{2.321896in}}%
\pgfpathlineto{\pgfqpoint{2.757377in}{2.261053in}}%
\pgfpathlineto{\pgfqpoint{2.799058in}{2.225950in}}%
\pgfpathlineto{\pgfqpoint{2.864027in}{2.173431in}}%
\pgfpathlineto{\pgfqpoint{2.932168in}{2.120912in}}%
\pgfpathlineto{\pgfqpoint{3.003589in}{2.068393in}}%
\pgfpathlineto{\pgfqpoint{3.078379in}{2.015874in}}%
\pgfpathlineto{\pgfqpoint{3.172559in}{1.952974in}}%
\pgfpathlineto{\pgfqpoint{3.241756in}{1.908701in}}%
\pgfpathlineto{\pgfqpoint{3.345552in}{1.845430in}}%
\pgfpathlineto{\pgfqpoint{3.414749in}{1.804884in}}%
\pgfpathlineto{\pgfqpoint{3.518544in}{1.746730in}}%
\pgfpathlineto{\pgfqpoint{3.622340in}{1.691201in}}%
\pgfpathlineto{\pgfqpoint{3.726135in}{1.638089in}}%
\pgfpathlineto{\pgfqpoint{3.829931in}{1.587198in}}%
\pgfpathlineto{\pgfqpoint{3.933726in}{1.538339in}}%
\pgfpathlineto{\pgfqpoint{4.039029in}{1.490685in}}%
\pgfpathlineto{\pgfqpoint{4.175916in}{1.431619in}}%
\pgfpathlineto{\pgfqpoint{4.314310in}{1.374728in}}%
\pgfpathlineto{\pgfqpoint{4.419570in}{1.333128in}}%
\pgfpathlineto{\pgfqpoint{4.591098in}{1.268761in}}%
\pgfpathlineto{\pgfqpoint{4.704352in}{1.228090in}}%
\pgfpathlineto{\pgfqpoint{4.867885in}{1.172168in}}%
\pgfpathlineto{\pgfqpoint{5.040878in}{1.116280in}}%
\pgfpathlineto{\pgfqpoint{5.190141in}{1.070533in}}%
\pgfpathlineto{\pgfqpoint{5.372070in}{1.018014in}}%
\pgfpathlineto{\pgfqpoint{5.525257in}{0.976551in}}%
\pgfpathlineto{\pgfqpoint{5.663651in}{0.940994in}}%
\pgfpathlineto{\pgfqpoint{5.779128in}{0.912976in}}%
\pgfpathlineto{\pgfqpoint{5.940439in}{0.876419in}}%
\pgfpathlineto{\pgfqpoint{6.078833in}{0.847386in}}%
\pgfpathlineto{\pgfqpoint{6.182628in}{0.827143in}}%
\pgfpathlineto{\pgfqpoint{6.321022in}{0.802476in}}%
\pgfpathlineto{\pgfqpoint{6.459416in}{0.780678in}}%
\pgfpathlineto{\pgfqpoint{6.597810in}{0.762578in}}%
\pgfpathlineto{\pgfqpoint{6.701606in}{0.751567in}}%
\pgfpathlineto{\pgfqpoint{6.805401in}{0.743622in}}%
\pgfpathlineto{\pgfqpoint{6.874598in}{0.740208in}}%
\pgfpathlineto{\pgfqpoint{6.943795in}{0.738729in}}%
\pgfpathlineto{\pgfqpoint{7.012992in}{0.739704in}}%
\pgfpathlineto{\pgfqpoint{7.082189in}{0.743855in}}%
\pgfpathlineto{\pgfqpoint{7.116787in}{0.747432in}}%
\pgfpathlineto{\pgfqpoint{7.169601in}{0.755420in}}%
\pgfpathlineto{\pgfqpoint{7.185984in}{0.758873in}}%
\pgfpathlineto{\pgfqpoint{7.220583in}{0.767986in}}%
\pgfpathlineto{\pgfqpoint{7.260330in}{0.781679in}}%
\pgfpathlineto{\pgfqpoint{7.289780in}{0.796930in}}%
\pgfpathlineto{\pgfqpoint{7.307122in}{0.807939in}}%
\pgfpathlineto{\pgfqpoint{7.324378in}{0.823402in}}%
\pgfpathlineto{\pgfqpoint{7.334561in}{0.834198in}}%
\pgfpathlineto{\pgfqpoint{7.350173in}{0.860458in}}%
\pgfpathlineto{\pgfqpoint{7.357813in}{0.886717in}}%
\pgfpathlineto{\pgfqpoint{7.358977in}{0.916254in}}%
\pgfpathlineto{\pgfqpoint{7.356267in}{0.939236in}}%
\pgfpathlineto{\pgfqpoint{7.349433in}{0.965495in}}%
\pgfpathlineto{\pgfqpoint{7.339633in}{0.991755in}}%
\pgfpathlineto{\pgfqpoint{7.324378in}{1.023441in}}%
\pgfpathlineto{\pgfqpoint{7.312902in}{1.044274in}}%
\pgfpathlineto{\pgfqpoint{7.289780in}{1.080485in}}%
\pgfpathlineto{\pgfqpoint{7.278734in}{1.096793in}}%
\pgfpathlineto{\pgfqpoint{7.238939in}{1.149312in}}%
\pgfpathlineto{\pgfqpoint{7.194838in}{1.201831in}}%
\pgfpathlineto{\pgfqpoint{7.122442in}{1.280609in}}%
\pgfpathlineto{\pgfqpoint{7.044471in}{1.359388in}}%
\pgfpathlineto{\pgfqpoint{6.962369in}{1.438166in}}%
\pgfpathlineto{\pgfqpoint{6.839999in}{1.550968in}}%
\pgfpathlineto{\pgfqpoint{6.731616in}{1.648242in}}%
\pgfpathlineto{\pgfqpoint{6.668926in}{1.703726in}}%
\pgfpathlineto{\pgfqpoint{6.668926in}{1.703726in}}%
\pgfusepath{stroke}%
\end{pgfscope}%
\begin{pgfscope}%
\pgfpathrectangle{\pgfqpoint{0.854460in}{0.571603in}}{\pgfqpoint{6.885100in}{5.225635in}}%
\pgfusepath{clip}%
\pgfsetbuttcap%
\pgfsetroundjoin%
\pgfsetlinewidth{1.505625pt}%
\definecolor{currentstroke}{rgb}{0.283187,0.125848,0.444960}%
\pgfsetstrokecolor{currentstroke}%
\pgfsetdash{}{0pt}%
\pgfpathmoveto{\pgfqpoint{7.739560in}{1.088432in}}%
\pgfpathlineto{\pgfqpoint{7.730423in}{1.096793in}}%
\pgfpathlineto{\pgfqpoint{7.704962in}{1.119498in}}%
\pgfpathlineto{\pgfqpoint{7.701029in}{1.123052in}}%
\pgfpathlineto{\pgfqpoint{7.671234in}{1.149312in}}%
\pgfpathlineto{\pgfqpoint{7.670363in}{1.150059in}}%
\pgfpathlineto{\pgfqpoint{7.641018in}{1.175571in}}%
\pgfpathlineto{\pgfqpoint{7.635765in}{1.180048in}}%
\pgfpathlineto{\pgfqpoint{7.610499in}{1.201831in}}%
\pgfpathlineto{\pgfqpoint{7.601166in}{1.209727in}}%
\pgfpathlineto{\pgfqpoint{7.579707in}{1.228090in}}%
\pgfpathlineto{\pgfqpoint{7.566568in}{1.239137in}}%
\pgfpathlineto{\pgfqpoint{7.548671in}{1.254350in}}%
\pgfpathlineto{\pgfqpoint{7.531969in}{1.268313in}}%
\pgfpathlineto{\pgfqpoint{7.517417in}{1.280609in}}%
\pgfpathlineto{\pgfqpoint{7.497371in}{1.297286in}}%
\pgfpathlineto{\pgfqpoint{7.485969in}{1.306869in}}%
\pgfpathlineto{\pgfqpoint{7.462772in}{1.326084in}}%
\pgfpathlineto{\pgfqpoint{7.454352in}{1.333128in}}%
\pgfpathlineto{\pgfqpoint{7.428174in}{1.354732in}}%
\pgfpathlineto{\pgfqpoint{7.422586in}{1.359388in}}%
\pgfpathlineto{\pgfqpoint{7.393575in}{1.383252in}}%
\pgfpathlineto{\pgfqpoint{7.390691in}{1.385647in}}%
\pgfpathlineto{\pgfqpoint{7.358977in}{1.411667in}}%
\pgfpathlineto{\pgfqpoint{7.358687in}{1.411906in}}%
\pgfpathlineto{\pgfqpoint{7.326550in}{1.438166in}}%
\pgfpathlineto{\pgfqpoint{7.324378in}{1.439923in}}%
\pgfpathlineto{\pgfqpoint{7.302202in}{1.458009in}}%
\pgfusepath{stroke}%
\end{pgfscope}%
\begin{pgfscope}%
\pgfpathrectangle{\pgfqpoint{0.854460in}{0.571603in}}{\pgfqpoint{6.885100in}{5.225635in}}%
\pgfusepath{clip}%
\pgfsetbuttcap%
\pgfsetroundjoin%
\pgfsetlinewidth{1.505625pt}%
\definecolor{currentstroke}{rgb}{0.283187,0.125848,0.444960}%
\pgfsetstrokecolor{currentstroke}%
\pgfsetdash{}{0pt}%
\pgfpathmoveto{\pgfqpoint{7.000596in}{1.703014in}}%
\pgfpathlineto{\pgfqpoint{6.770803in}{1.891399in}}%
\pgfpathlineto{\pgfqpoint{6.590464in}{2.042134in}}%
\pgfpathlineto{\pgfqpoint{6.424818in}{2.183801in}}%
\pgfpathlineto{\pgfqpoint{6.286424in}{2.305209in}}%
\pgfpathlineto{\pgfqpoint{6.170001in}{2.409766in}}%
\pgfpathlineto{\pgfqpoint{6.044234in}{2.525806in}}%
\pgfpathlineto{\pgfqpoint{5.940439in}{2.624234in}}%
\pgfpathlineto{\pgfqpoint{5.836643in}{2.725393in}}%
\pgfpathlineto{\pgfqpoint{5.732548in}{2.829918in}}%
\pgfpathlineto{\pgfqpoint{5.629052in}{2.937293in}}%
\pgfpathlineto{\pgfqpoint{5.533330in}{3.039994in}}%
\pgfpathlineto{\pgfqpoint{5.438859in}{3.145032in}}%
\pgfpathlineto{\pgfqpoint{5.347915in}{3.250070in}}%
\pgfpathlineto{\pgfqpoint{5.260364in}{3.355107in}}%
\pgfpathlineto{\pgfqpoint{5.176350in}{3.460145in}}%
\pgfpathlineto{\pgfqpoint{5.095663in}{3.565183in}}%
\pgfpathlineto{\pgfqpoint{5.018398in}{3.670221in}}%
\pgfpathlineto{\pgfqpoint{4.937082in}{3.786059in}}%
\pgfpathlineto{\pgfqpoint{4.891012in}{3.854037in}}%
\pgfpathlineto{\pgfqpoint{4.822482in}{3.959075in}}%
\pgfpathlineto{\pgfqpoint{4.756855in}{4.064113in}}%
\pgfpathlineto{\pgfqpoint{4.693922in}{4.169151in}}%
\pgfpathlineto{\pgfqpoint{4.633265in}{4.274189in}}%
\pgfpathlineto{\pgfqpoint{4.531791in}{4.458005in}}%
\pgfpathlineto{\pgfqpoint{4.334929in}{4.825638in}}%
\pgfpathlineto{\pgfqpoint{4.279711in}{4.923693in}}%
\pgfpathlineto{\pgfqpoint{4.228054in}{5.009454in}}%
\pgfpathlineto{\pgfqpoint{4.194370in}{5.061973in}}%
\pgfpathlineto{\pgfqpoint{4.158471in}{5.114492in}}%
\pgfpathlineto{\pgfqpoint{4.119705in}{5.167011in}}%
\pgfpathlineto{\pgfqpoint{4.099011in}{5.193271in}}%
\pgfpathlineto{\pgfqpoint{4.054064in}{5.245790in}}%
\pgfpathlineto{\pgfqpoint{4.029576in}{5.272049in}}%
\pgfpathlineto{\pgfqpoint{4.002923in}{5.298808in}}%
\pgfpathlineto{\pgfqpoint{3.968325in}{5.330513in}}%
\pgfpathlineto{\pgfqpoint{3.933726in}{5.359207in}}%
\pgfpathlineto{\pgfqpoint{3.899128in}{5.385068in}}%
\pgfpathlineto{\pgfqpoint{3.864529in}{5.408287in}}%
\pgfpathlineto{\pgfqpoint{3.828920in}{5.429606in}}%
\pgfpathlineto{\pgfqpoint{3.777759in}{5.455865in}}%
\pgfpathlineto{\pgfqpoint{3.760734in}{5.463747in}}%
\pgfpathlineto{\pgfqpoint{3.715104in}{5.482125in}}%
\pgfpathlineto{\pgfqpoint{3.691537in}{5.490496in}}%
\pgfpathlineto{\pgfqpoint{3.656938in}{5.501210in}}%
\pgfpathlineto{\pgfqpoint{3.622340in}{5.510323in}}%
\pgfpathlineto{\pgfqpoint{3.587741in}{5.517840in}}%
\pgfpathlineto{\pgfqpoint{3.553143in}{5.523955in}}%
\pgfpathlineto{\pgfqpoint{3.518544in}{5.528730in}}%
\pgfpathlineto{\pgfqpoint{3.483946in}{5.532224in}}%
\pgfpathlineto{\pgfqpoint{3.444620in}{5.534644in}}%
\pgfpathlineto{\pgfqpoint{3.380150in}{5.535537in}}%
\pgfpathlineto{\pgfqpoint{3.310953in}{5.532290in}}%
\pgfpathlineto{\pgfqpoint{3.241756in}{5.525031in}}%
\pgfpathlineto{\pgfqpoint{3.172559in}{5.514071in}}%
\pgfpathlineto{\pgfqpoint{3.103362in}{5.499499in}}%
\pgfpathlineto{\pgfqpoint{3.034165in}{5.481527in}}%
\pgfpathlineto{\pgfqpoint{2.964968in}{5.460082in}}%
\pgfpathlineto{\pgfqpoint{2.895771in}{5.435294in}}%
\pgfpathlineto{\pgfqpoint{2.826574in}{5.407149in}}%
\pgfpathlineto{\pgfqpoint{2.757377in}{5.375577in}}%
\pgfpathlineto{\pgfqpoint{2.688180in}{5.340405in}}%
\pgfpathlineto{\pgfqpoint{2.618983in}{5.301629in}}%
\pgfpathlineto{\pgfqpoint{2.570533in}{5.272049in}}%
\pgfpathlineto{\pgfqpoint{2.515188in}{5.236002in}}%
\pgfpathlineto{\pgfqpoint{2.445991in}{5.186932in}}%
\pgfpathlineto{\pgfqpoint{2.376794in}{5.133091in}}%
\pgfpathlineto{\pgfqpoint{2.323816in}{5.088233in}}%
\pgfpathlineto{\pgfqpoint{2.272999in}{5.042206in}}%
\pgfpathlineto{\pgfqpoint{2.238400in}{5.008877in}}%
\pgfpathlineto{\pgfqpoint{2.187996in}{4.956935in}}%
\pgfpathlineto{\pgfqpoint{2.140530in}{4.904416in}}%
\pgfpathlineto{\pgfqpoint{2.100006in}{4.856234in}}%
\pgfpathlineto{\pgfqpoint{2.065408in}{4.812229in}}%
\pgfpathlineto{\pgfqpoint{2.030809in}{4.765354in}}%
\pgfpathlineto{\pgfqpoint{1.996211in}{4.715127in}}%
\pgfpathlineto{\pgfqpoint{1.961612in}{4.660952in}}%
\pgfpathlineto{\pgfqpoint{1.927014in}{4.602073in}}%
\pgfpathlineto{\pgfqpoint{1.892043in}{4.536784in}}%
\pgfpathlineto{\pgfqpoint{1.857817in}{4.465384in}}%
\pgfpathlineto{\pgfqpoint{1.832148in}{4.405486in}}%
\pgfpathlineto{\pgfqpoint{1.811778in}{4.352967in}}%
\pgfpathlineto{\pgfqpoint{1.788620in}{4.286158in}}%
\pgfpathlineto{\pgfqpoint{1.776856in}{4.247930in}}%
\pgfpathlineto{\pgfqpoint{1.762198in}{4.195411in}}%
\pgfpathlineto{\pgfqpoint{1.749360in}{4.142892in}}%
\pgfpathlineto{\pgfqpoint{1.738411in}{4.090373in}}%
\pgfpathlineto{\pgfqpoint{1.729146in}{4.037854in}}%
\pgfpathlineto{\pgfqpoint{1.719423in}{3.967291in}}%
\pgfpathlineto{\pgfqpoint{1.715829in}{3.932816in}}%
\pgfpathlineto{\pgfqpoint{1.711808in}{3.880297in}}%
\pgfpathlineto{\pgfqpoint{1.709470in}{3.827778in}}%
\pgfpathlineto{\pgfqpoint{1.708816in}{3.775259in}}%
\pgfpathlineto{\pgfqpoint{1.709848in}{3.722740in}}%
\pgfpathlineto{\pgfqpoint{1.712562in}{3.670221in}}%
\pgfpathlineto{\pgfqpoint{1.716951in}{3.617702in}}%
\pgfpathlineto{\pgfqpoint{1.723097in}{3.565183in}}%
\pgfpathlineto{\pgfqpoint{1.730999in}{3.512664in}}%
\pgfpathlineto{\pgfqpoint{1.740578in}{3.460145in}}%
\pgfpathlineto{\pgfqpoint{1.754021in}{3.398390in}}%
\pgfpathlineto{\pgfqpoint{1.764955in}{3.355107in}}%
\pgfpathlineto{\pgfqpoint{1.779794in}{3.302589in}}%
\pgfpathlineto{\pgfqpoint{1.796439in}{3.250070in}}%
\pgfpathlineto{\pgfqpoint{1.823218in}{3.175431in}}%
\pgfpathlineto{\pgfqpoint{1.835268in}{3.145032in}}%
\pgfpathlineto{\pgfqpoint{1.857817in}{3.091573in}}%
\pgfpathlineto{\pgfqpoint{1.894337in}{3.013734in}}%
\pgfpathlineto{\pgfqpoint{1.927014in}{2.950905in}}%
\pgfpathlineto{\pgfqpoint{1.961612in}{2.889598in}}%
\pgfpathlineto{\pgfqpoint{1.996211in}{2.832706in}}%
\pgfpathlineto{\pgfqpoint{2.030809in}{2.779538in}}%
\pgfpathlineto{\pgfqpoint{2.065408in}{2.729524in}}%
\pgfpathlineto{\pgfqpoint{2.100006in}{2.682188in}}%
\pgfpathlineto{\pgfqpoint{2.134605in}{2.637130in}}%
\pgfpathlineto{\pgfqpoint{2.169568in}{2.593583in}}%
\pgfpathlineto{\pgfqpoint{2.214061in}{2.541064in}}%
\pgfpathlineto{\pgfqpoint{2.272999in}{2.475421in}}%
\pgfpathlineto{\pgfqpoint{2.310122in}{2.436026in}}%
\pgfpathlineto{\pgfqpoint{2.376794in}{2.369010in}}%
\pgfpathlineto{\pgfqpoint{2.416384in}{2.330988in}}%
\pgfpathlineto{\pgfqpoint{2.480590in}{2.272134in}}%
\pgfpathlineto{\pgfqpoint{2.549787in}{2.211970in}}%
\pgfpathlineto{\pgfqpoint{2.618983in}{2.154761in}}%
\pgfpathlineto{\pgfqpoint{2.688180in}{2.100224in}}%
\pgfpathlineto{\pgfqpoint{2.757377in}{2.048100in}}%
\pgfpathlineto{\pgfqpoint{2.826574in}{1.998151in}}%
\pgfpathlineto{\pgfqpoint{2.895771in}{1.950160in}}%
\pgfpathlineto{\pgfqpoint{2.964968in}{1.903925in}}%
\pgfpathlineto{\pgfqpoint{3.035699in}{1.858318in}}%
\pgfpathlineto{\pgfqpoint{3.137961in}{1.795349in}}%
\pgfpathlineto{\pgfqpoint{3.208886in}{1.753280in}}%
\pgfpathlineto{\pgfqpoint{3.310953in}{1.695235in}}%
\pgfpathlineto{\pgfqpoint{3.414749in}{1.638724in}}%
\pgfpathlineto{\pgfqpoint{3.518544in}{1.584532in}}%
\pgfpathlineto{\pgfqpoint{3.622340in}{1.532477in}}%
\pgfpathlineto{\pgfqpoint{3.726135in}{1.482384in}}%
\pgfpathlineto{\pgfqpoint{3.829931in}{1.434089in}}%
\pgfpathlineto{\pgfqpoint{3.968325in}{1.372411in}}%
\pgfpathlineto{\pgfqpoint{4.072120in}{1.327891in}}%
\pgfpathlineto{\pgfqpoint{4.210514in}{1.270854in}}%
\pgfpathlineto{\pgfqpoint{4.348908in}{1.216170in}}%
\pgfpathlineto{\pgfqpoint{4.455196in}{1.175571in}}%
\pgfpathlineto{\pgfqpoint{4.625696in}{1.113263in}}%
\pgfpathlineto{\pgfqpoint{4.764090in}{1.064757in}}%
\pgfpathlineto{\pgfqpoint{4.937082in}{1.006764in}}%
\pgfpathlineto{\pgfqpoint{5.075476in}{0.962226in}}%
\pgfpathlineto{\pgfqpoint{5.248469in}{0.908966in}}%
\pgfpathlineto{\pgfqpoint{5.421461in}{0.858148in}}%
\pgfpathlineto{\pgfqpoint{5.629052in}{0.800424in}}%
\pgfpathlineto{\pgfqpoint{5.802045in}{0.754750in}}%
\pgfpathlineto{\pgfqpoint{6.010562in}{0.702901in}}%
\pgfpathlineto{\pgfqpoint{6.237683in}{0.650382in}}%
\pgfpathlineto{\pgfqpoint{6.424818in}{0.610392in}}%
\pgfpathlineto{\pgfqpoint{6.563212in}{0.582692in}}%
\pgfpathlineto{\pgfqpoint{6.620572in}{0.571603in}}%
\pgfpathlineto{\pgfqpoint{6.620572in}{0.571603in}}%
\pgfusepath{stroke}%
\end{pgfscope}%
\begin{pgfscope}%
\pgfpathrectangle{\pgfqpoint{0.854460in}{0.571603in}}{\pgfqpoint{6.885100in}{5.225635in}}%
\pgfusepath{clip}%
\pgfsetbuttcap%
\pgfsetroundjoin%
\pgfsetlinewidth{1.505625pt}%
\definecolor{currentstroke}{rgb}{0.281412,0.155834,0.469201}%
\pgfsetstrokecolor{currentstroke}%
\pgfsetdash{}{0pt}%
\pgfpathmoveto{\pgfqpoint{3.000700in}{5.797238in}}%
\pgfpathlineto{\pgfqpoint{2.964968in}{5.784422in}}%
\pgfpathlineto{\pgfqpoint{2.895771in}{5.757885in}}%
\pgfpathlineto{\pgfqpoint{2.826574in}{5.728910in}}%
\pgfpathlineto{\pgfqpoint{2.746429in}{5.692201in}}%
\pgfpathlineto{\pgfqpoint{2.688180in}{5.663434in}}%
\pgfpathlineto{\pgfqpoint{2.618983in}{5.626668in}}%
\pgfpathlineto{\pgfqpoint{2.549711in}{5.587163in}}%
\pgfpathlineto{\pgfqpoint{2.480590in}{5.544644in}}%
\pgfpathlineto{\pgfqpoint{2.411393in}{5.498971in}}%
\pgfpathlineto{\pgfqpoint{2.342196in}{5.449884in}}%
\pgfpathlineto{\pgfqpoint{2.272999in}{5.397043in}}%
\pgfpathlineto{\pgfqpoint{2.203802in}{5.340064in}}%
\pgfpathlineto{\pgfqpoint{2.156342in}{5.298308in}}%
\pgfpathlineto{\pgfqpoint{2.099939in}{5.245790in}}%
\pgfpathlineto{\pgfqpoint{2.047395in}{5.193271in}}%
\pgfpathlineto{\pgfqpoint{1.996211in}{5.138887in}}%
\pgfpathlineto{\pgfqpoint{1.951698in}{5.088233in}}%
\pgfpathlineto{\pgfqpoint{1.908293in}{5.035714in}}%
\pgfpathlineto{\pgfqpoint{1.867540in}{4.983195in}}%
\pgfpathlineto{\pgfqpoint{1.829331in}{4.930676in}}%
\pgfpathlineto{\pgfqpoint{1.793556in}{4.878157in}}%
\pgfpathlineto{\pgfqpoint{1.760104in}{4.825638in}}%
\pgfpathlineto{\pgfqpoint{1.728865in}{4.773119in}}%
\pgfpathlineto{\pgfqpoint{1.699733in}{4.720600in}}%
\pgfpathlineto{\pgfqpoint{1.672599in}{4.668081in}}%
\pgfpathlineto{\pgfqpoint{1.647363in}{4.615562in}}%
\pgfpathlineto{\pgfqpoint{1.613148in}{4.536784in}}%
\pgfpathlineto{\pgfqpoint{1.583016in}{4.458005in}}%
\pgfpathlineto{\pgfqpoint{1.565212in}{4.405486in}}%
\pgfpathlineto{\pgfqpoint{1.546430in}{4.344211in}}%
\pgfpathlineto{\pgfqpoint{1.527896in}{4.274189in}}%
\pgfpathlineto{\pgfqpoint{1.511832in}{4.202691in}}%
\pgfpathlineto{\pgfqpoint{1.500797in}{4.142892in}}%
\pgfpathlineto{\pgfqpoint{1.492766in}{4.090373in}}%
\pgfpathlineto{\pgfqpoint{1.486271in}{4.037854in}}%
\pgfpathlineto{\pgfqpoint{1.481325in}{3.985335in}}%
\pgfpathlineto{\pgfqpoint{1.477233in}{3.916245in}}%
\pgfpathlineto{\pgfqpoint{1.476132in}{3.880297in}}%
\pgfpathlineto{\pgfqpoint{1.475877in}{3.827778in}}%
\pgfpathlineto{\pgfqpoint{1.477233in}{3.773520in}}%
\pgfpathlineto{\pgfqpoint{1.480023in}{3.722740in}}%
\pgfpathlineto{\pgfqpoint{1.484459in}{3.670221in}}%
\pgfpathlineto{\pgfqpoint{1.490454in}{3.617702in}}%
\pgfpathlineto{\pgfqpoint{1.498000in}{3.565183in}}%
\pgfpathlineto{\pgfqpoint{1.511832in}{3.488344in}}%
\pgfpathlineto{\pgfqpoint{1.517839in}{3.460145in}}%
\pgfpathlineto{\pgfqpoint{1.530244in}{3.407626in}}%
\pgfpathlineto{\pgfqpoint{1.546430in}{3.347308in}}%
\pgfpathlineto{\pgfqpoint{1.559921in}{3.302589in}}%
\pgfpathlineto{\pgfqpoint{1.581029in}{3.239300in}}%
\pgfpathlineto{\pgfqpoint{1.596366in}{3.197551in}}%
\pgfpathlineto{\pgfqpoint{1.617113in}{3.145032in}}%
\pgfpathlineto{\pgfqpoint{1.651684in}{3.066253in}}%
\pgfpathlineto{\pgfqpoint{1.690393in}{2.987475in}}%
\pgfpathlineto{\pgfqpoint{1.719423in}{2.933267in}}%
\pgfpathlineto{\pgfqpoint{1.764336in}{2.856177in}}%
\pgfpathlineto{\pgfqpoint{1.797322in}{2.803659in}}%
\pgfpathlineto{\pgfqpoint{1.832342in}{2.751140in}}%
\pgfpathlineto{\pgfqpoint{1.869440in}{2.698621in}}%
\pgfpathlineto{\pgfqpoint{1.908656in}{2.646102in}}%
\pgfpathlineto{\pgfqpoint{1.961612in}{2.579375in}}%
\pgfpathlineto{\pgfqpoint{1.996211in}{2.537986in}}%
\pgfpathlineto{\pgfqpoint{2.039487in}{2.488545in}}%
\pgfpathlineto{\pgfqpoint{2.100006in}{2.423062in}}%
\pgfpathlineto{\pgfqpoint{2.138276in}{2.383507in}}%
\pgfpathlineto{\pgfqpoint{2.203802in}{2.319079in}}%
\pgfpathlineto{\pgfqpoint{2.272999in}{2.254650in}}%
\pgfpathlineto{\pgfqpoint{2.307597in}{2.223760in}}%
\pgfpathlineto{\pgfqpoint{2.376794in}{2.164372in}}%
\pgfpathlineto{\pgfqpoint{2.445991in}{2.107741in}}%
\pgfpathlineto{\pgfqpoint{2.515188in}{2.053610in}}%
\pgfpathlineto{\pgfqpoint{2.584385in}{2.001747in}}%
\pgfpathlineto{\pgfqpoint{2.653582in}{1.951936in}}%
\pgfpathlineto{\pgfqpoint{2.722779in}{1.903978in}}%
\pgfpathlineto{\pgfqpoint{2.791976in}{1.857690in}}%
\pgfpathlineto{\pgfqpoint{2.895771in}{1.791378in}}%
\pgfpathlineto{\pgfqpoint{2.964968in}{1.748852in}}%
\pgfpathlineto{\pgfqpoint{3.068764in}{1.687596in}}%
\pgfpathlineto{\pgfqpoint{3.172559in}{1.628934in}}%
\pgfpathlineto{\pgfqpoint{3.241756in}{1.591175in}}%
\pgfpathlineto{\pgfqpoint{3.345552in}{1.536457in}}%
\pgfpathlineto{\pgfqpoint{3.449347in}{1.483798in}}%
\pgfpathlineto{\pgfqpoint{3.553143in}{1.433036in}}%
\pgfpathlineto{\pgfqpoint{3.691537in}{1.368125in}}%
\pgfpathlineto{\pgfqpoint{3.795332in}{1.321316in}}%
\pgfpathlineto{\pgfqpoint{3.899128in}{1.275948in}}%
\pgfpathlineto{\pgfqpoint{4.037522in}{1.217708in}}%
\pgfpathlineto{\pgfqpoint{4.175916in}{1.161718in}}%
\pgfpathlineto{\pgfqpoint{4.279711in}{1.121044in}}%
\pgfpathlineto{\pgfqpoint{4.452704in}{1.055920in}}%
\pgfpathlineto{\pgfqpoint{4.591098in}{1.005839in}}%
\pgfpathlineto{\pgfqpoint{4.729491in}{0.957462in}}%
\pgfpathlineto{\pgfqpoint{4.902484in}{0.899298in}}%
\pgfpathlineto{\pgfqpoint{5.040878in}{0.854437in}}%
\pgfpathlineto{\pgfqpoint{5.132884in}{0.825451in}}%
\pgfpathlineto{\pgfqpoint{5.132884in}{0.825451in}}%
\pgfusepath{stroke}%
\end{pgfscope}%
\begin{pgfscope}%
\pgfpathrectangle{\pgfqpoint{0.854460in}{0.571603in}}{\pgfqpoint{6.885100in}{5.225635in}}%
\pgfusepath{clip}%
\pgfsetbuttcap%
\pgfsetroundjoin%
\pgfsetlinewidth{1.505625pt}%
\definecolor{currentstroke}{rgb}{0.281412,0.155834,0.469201}%
\pgfsetstrokecolor{currentstroke}%
\pgfsetdash{}{0pt}%
\pgfpathmoveto{\pgfqpoint{5.503072in}{0.714892in}}%
\pgfpathlineto{\pgfqpoint{5.525257in}{0.708532in}}%
\pgfpathlineto{\pgfqpoint{5.544913in}{0.702901in}}%
\pgfpathlineto{\pgfqpoint{5.559855in}{0.698710in}}%
\pgfpathlineto{\pgfqpoint{5.594454in}{0.689038in}}%
\pgfpathlineto{\pgfqpoint{5.629052in}{0.679314in}}%
\pgfpathlineto{\pgfqpoint{5.638612in}{0.676641in}}%
\pgfpathlineto{\pgfqpoint{5.663651in}{0.669792in}}%
\pgfpathlineto{\pgfqpoint{5.698249in}{0.660318in}}%
\pgfpathlineto{\pgfqpoint{5.732848in}{0.650794in}}%
\pgfpathlineto{\pgfqpoint{5.734360in}{0.650382in}}%
\pgfpathlineto{\pgfqpoint{5.767446in}{0.641555in}}%
\pgfpathlineto{\pgfqpoint{5.802045in}{0.632286in}}%
\pgfpathlineto{\pgfqpoint{5.832416in}{0.624122in}}%
\pgfpathlineto{\pgfqpoint{5.836643in}{0.623011in}}%
\pgfpathlineto{\pgfqpoint{5.871242in}{0.613995in}}%
\pgfpathlineto{\pgfqpoint{5.905840in}{0.604936in}}%
\pgfpathlineto{\pgfqpoint{5.932812in}{0.597863in}}%
\pgfpathlineto{\pgfqpoint{5.940439in}{0.595908in}}%
\pgfpathlineto{\pgfqpoint{5.975037in}{0.587105in}}%
\pgfpathlineto{\pgfqpoint{6.009636in}{0.578264in}}%
\pgfpathlineto{\pgfqpoint{6.035677in}{0.571603in}}%
\pgfusepath{stroke}%
\end{pgfscope}%
\begin{pgfscope}%
\pgfpathrectangle{\pgfqpoint{0.854460in}{0.571603in}}{\pgfqpoint{6.885100in}{5.225635in}}%
\pgfusepath{clip}%
\pgfsetbuttcap%
\pgfsetroundjoin%
\pgfsetlinewidth{1.505625pt}%
\definecolor{currentstroke}{rgb}{0.281412,0.155834,0.469201}%
\pgfsetstrokecolor{currentstroke}%
\pgfsetdash{}{0pt}%
\pgfpathmoveto{\pgfqpoint{7.739560in}{1.355279in}}%
\pgfpathlineto{\pgfqpoint{7.734163in}{1.359388in}}%
\pgfpathlineto{\pgfqpoint{7.704962in}{1.381436in}}%
\pgfpathlineto{\pgfqpoint{7.699429in}{1.385647in}}%
\pgfpathlineto{\pgfqpoint{7.686713in}{1.395253in}}%
\pgfusepath{stroke}%
\end{pgfscope}%
\begin{pgfscope}%
\pgfpathrectangle{\pgfqpoint{0.854460in}{0.571603in}}{\pgfqpoint{6.885100in}{5.225635in}}%
\pgfusepath{clip}%
\pgfsetbuttcap%
\pgfsetroundjoin%
\pgfsetlinewidth{1.505625pt}%
\definecolor{currentstroke}{rgb}{0.281412,0.155834,0.469201}%
\pgfsetstrokecolor{currentstroke}%
\pgfsetdash{}{0pt}%
\pgfpathmoveto{\pgfqpoint{7.377693in}{1.630477in}}%
\pgfpathlineto{\pgfqpoint{7.185522in}{1.779539in}}%
\pgfpathlineto{\pgfqpoint{7.012992in}{1.916153in}}%
\pgfpathlineto{\pgfqpoint{6.857604in}{2.042134in}}%
\pgfpathlineto{\pgfqpoint{6.731121in}{2.147172in}}%
\pgfpathlineto{\pgfqpoint{6.597810in}{2.260626in}}%
\pgfpathlineto{\pgfqpoint{6.487173in}{2.357248in}}%
\pgfpathlineto{\pgfqpoint{6.355621in}{2.475476in}}%
\pgfpathlineto{\pgfqpoint{6.251825in}{2.571643in}}%
\pgfpathlineto{\pgfqpoint{6.146279in}{2.672361in}}%
\pgfpathlineto{\pgfqpoint{6.039743in}{2.777399in}}%
\pgfpathlineto{\pgfqpoint{5.936930in}{2.882437in}}%
\pgfpathlineto{\pgfqpoint{5.837888in}{2.987475in}}%
\pgfpathlineto{\pgfqpoint{5.742598in}{3.092513in}}%
\pgfpathlineto{\pgfqpoint{5.651189in}{3.197551in}}%
\pgfpathlineto{\pgfqpoint{5.559855in}{3.307335in}}%
\pgfpathlineto{\pgfqpoint{5.490658in}{3.394051in}}%
\pgfpathlineto{\pgfqpoint{5.439598in}{3.460145in}}%
\pgfpathlineto{\pgfqpoint{5.361735in}{3.565183in}}%
\pgfpathlineto{\pgfqpoint{5.283067in}{3.677055in}}%
\pgfpathlineto{\pgfqpoint{5.234610in}{3.749000in}}%
\pgfpathlineto{\pgfqpoint{5.167122in}{3.854037in}}%
\pgfpathlineto{\pgfqpoint{5.103292in}{3.959075in}}%
\pgfpathlineto{\pgfqpoint{5.057701in}{4.037854in}}%
\pgfpathlineto{\pgfqpoint{4.999972in}{4.142892in}}%
\pgfpathlineto{\pgfqpoint{4.958784in}{4.221670in}}%
\pgfpathlineto{\pgfqpoint{4.906660in}{4.326708in}}%
\pgfpathlineto{\pgfqpoint{4.845456in}{4.458005in}}%
\pgfpathlineto{\pgfqpoint{4.788244in}{4.589303in}}%
\pgfpathlineto{\pgfqpoint{4.744815in}{4.694341in}}%
\pgfpathlineto{\pgfqpoint{4.682487in}{4.851897in}}%
\pgfpathlineto{\pgfqpoint{4.569604in}{5.140752in}}%
\pgfpathlineto{\pgfqpoint{4.536814in}{5.219530in}}%
\pgfpathlineto{\pgfqpoint{4.501893in}{5.298308in}}%
\pgfpathlineto{\pgfqpoint{4.476948in}{5.350827in}}%
\pgfpathlineto{\pgfqpoint{4.450273in}{5.403346in}}%
\pgfpathlineto{\pgfqpoint{4.418105in}{5.461317in}}%
\pgfpathlineto{\pgfqpoint{4.383507in}{5.517400in}}%
\pgfpathlineto{\pgfqpoint{4.348908in}{5.567459in}}%
\pgfpathlineto{\pgfqpoint{4.313143in}{5.613422in}}%
\pgfpathlineto{\pgfqpoint{4.279711in}{5.651540in}}%
\pgfpathlineto{\pgfqpoint{4.239229in}{5.692201in}}%
\pgfpathlineto{\pgfqpoint{4.209857in}{5.718460in}}%
\pgfpathlineto{\pgfqpoint{4.175916in}{5.745580in}}%
\pgfpathlineto{\pgfqpoint{4.140013in}{5.770979in}}%
\pgfpathlineto{\pgfqpoint{4.097299in}{5.797238in}}%
\pgfpathlineto{\pgfqpoint{4.097299in}{5.797238in}}%
\pgfusepath{stroke}%
\end{pgfscope}%
\begin{pgfscope}%
\pgfpathrectangle{\pgfqpoint{0.854460in}{0.571603in}}{\pgfqpoint{6.885100in}{5.225635in}}%
\pgfusepath{clip}%
\pgfsetbuttcap%
\pgfsetroundjoin%
\pgfsetlinewidth{1.505625pt}%
\definecolor{currentstroke}{rgb}{0.277134,0.185228,0.489898}%
\pgfsetstrokecolor{currentstroke}%
\pgfsetdash{}{0pt}%
\pgfpathmoveto{\pgfqpoint{2.462846in}{5.797238in}}%
\pgfpathlineto{\pgfqpoint{2.411393in}{5.765731in}}%
\pgfpathlineto{\pgfqpoint{2.338356in}{5.718460in}}%
\pgfpathlineto{\pgfqpoint{2.262431in}{5.665941in}}%
\pgfpathlineto{\pgfqpoint{2.191307in}{5.613422in}}%
\pgfpathlineto{\pgfqpoint{2.124544in}{5.560903in}}%
\pgfpathlineto{\pgfqpoint{2.061737in}{5.508384in}}%
\pgfpathlineto{\pgfqpoint{1.996211in}{5.449926in}}%
\pgfpathlineto{\pgfqpoint{1.927014in}{5.383704in}}%
\pgfpathlineto{\pgfqpoint{1.869440in}{5.324568in}}%
\pgfpathlineto{\pgfqpoint{1.821249in}{5.272049in}}%
\pgfpathlineto{\pgfqpoint{1.775931in}{5.219530in}}%
\pgfpathlineto{\pgfqpoint{1.733099in}{5.167011in}}%
\pgfpathlineto{\pgfqpoint{1.684824in}{5.103954in}}%
\pgfpathlineto{\pgfqpoint{1.636419in}{5.035714in}}%
\pgfpathlineto{\pgfqpoint{1.601630in}{4.983195in}}%
\pgfpathlineto{\pgfqpoint{1.568918in}{4.930676in}}%
\pgfpathlineto{\pgfqpoint{1.538194in}{4.878157in}}%
\pgfpathlineto{\pgfqpoint{1.509374in}{4.825638in}}%
\pgfpathlineto{\pgfqpoint{1.469743in}{4.746860in}}%
\pgfpathlineto{\pgfqpoint{1.434090in}{4.668081in}}%
\pgfpathlineto{\pgfqpoint{1.402262in}{4.589303in}}%
\pgfpathlineto{\pgfqpoint{1.373438in}{4.508461in}}%
\pgfpathlineto{\pgfqpoint{1.349703in}{4.431746in}}%
\pgfpathlineto{\pgfqpoint{1.335275in}{4.379227in}}%
\pgfpathlineto{\pgfqpoint{1.316627in}{4.300449in}}%
\pgfpathlineto{\pgfqpoint{1.304241in}{4.238453in}}%
\pgfpathlineto{\pgfqpoint{1.292942in}{4.169151in}}%
\pgfpathlineto{\pgfqpoint{1.283157in}{4.090373in}}%
\pgfpathlineto{\pgfqpoint{1.276612in}{4.011594in}}%
\pgfpathlineto{\pgfqpoint{1.273331in}{3.932816in}}%
\pgfpathlineto{\pgfqpoint{1.273332in}{3.854037in}}%
\pgfpathlineto{\pgfqpoint{1.276621in}{3.775259in}}%
\pgfpathlineto{\pgfqpoint{1.283200in}{3.696481in}}%
\pgfpathlineto{\pgfqpoint{1.293058in}{3.617702in}}%
\pgfpathlineto{\pgfqpoint{1.304241in}{3.549712in}}%
\pgfpathlineto{\pgfqpoint{1.316989in}{3.486405in}}%
\pgfpathlineto{\pgfqpoint{1.335868in}{3.407626in}}%
\pgfpathlineto{\pgfqpoint{1.350489in}{3.355107in}}%
\pgfpathlineto{\pgfqpoint{1.375263in}{3.276329in}}%
\pgfpathlineto{\pgfqpoint{1.403766in}{3.197551in}}%
\pgfpathlineto{\pgfqpoint{1.424877in}{3.145032in}}%
\pgfpathlineto{\pgfqpoint{1.447589in}{3.092513in}}%
\pgfpathlineto{\pgfqpoint{1.484986in}{3.013734in}}%
\pgfpathlineto{\pgfqpoint{1.512064in}{2.961215in}}%
\pgfpathlineto{\pgfqpoint{1.556244in}{2.882437in}}%
\pgfpathlineto{\pgfqpoint{1.587973in}{2.829918in}}%
\pgfpathlineto{\pgfqpoint{1.621627in}{2.777399in}}%
\pgfpathlineto{\pgfqpoint{1.657250in}{2.724880in}}%
\pgfpathlineto{\pgfqpoint{1.694887in}{2.672361in}}%
\pgfpathlineto{\pgfqpoint{1.734575in}{2.619842in}}%
\pgfpathlineto{\pgfqpoint{1.788620in}{2.552402in}}%
\pgfpathlineto{\pgfqpoint{1.823218in}{2.511343in}}%
\pgfpathlineto{\pgfqpoint{1.890277in}{2.436026in}}%
\pgfpathlineto{\pgfqpoint{1.939890in}{2.383507in}}%
\pgfpathlineto{\pgfqpoint{1.996211in}{2.326616in}}%
\pgfpathlineto{\pgfqpoint{2.065408in}{2.260388in}}%
\pgfpathlineto{\pgfqpoint{2.132210in}{2.199691in}}%
\pgfpathlineto{\pgfqpoint{2.203802in}{2.137987in}}%
\pgfpathlineto{\pgfqpoint{2.272999in}{2.081090in}}%
\pgfpathlineto{\pgfqpoint{2.342196in}{2.026649in}}%
\pgfpathlineto{\pgfqpoint{2.411393in}{1.974440in}}%
\pgfpathlineto{\pgfqpoint{2.480590in}{1.924259in}}%
\pgfpathlineto{\pgfqpoint{2.549787in}{1.875916in}}%
\pgfpathlineto{\pgfqpoint{2.618983in}{1.829235in}}%
\pgfpathlineto{\pgfqpoint{2.722779in}{1.762232in}}%
\pgfpathlineto{\pgfqpoint{2.791976in}{1.719294in}}%
\pgfpathlineto{\pgfqpoint{2.895771in}{1.657310in}}%
\pgfpathlineto{\pgfqpoint{2.964968in}{1.617434in}}%
\pgfpathlineto{\pgfqpoint{3.068764in}{1.559739in}}%
\pgfpathlineto{\pgfqpoint{3.172559in}{1.504268in}}%
\pgfpathlineto{\pgfqpoint{3.276355in}{1.450855in}}%
\pgfpathlineto{\pgfqpoint{3.380150in}{1.399339in}}%
\pgfpathlineto{\pgfqpoint{3.483946in}{1.349569in}}%
\pgfpathlineto{\pgfqpoint{3.587741in}{1.301404in}}%
\pgfpathlineto{\pgfqpoint{3.726135in}{1.239582in}}%
\pgfpathlineto{\pgfqpoint{3.829931in}{1.194796in}}%
\pgfpathlineto{\pgfqpoint{3.968325in}{1.137124in}}%
\pgfpathlineto{\pgfqpoint{4.106719in}{1.081548in}}%
\pgfpathlineto{\pgfqpoint{4.245113in}{1.027930in}}%
\pgfpathlineto{\pgfqpoint{4.383507in}{0.976139in}}%
\pgfpathlineto{\pgfqpoint{4.521901in}{0.926048in}}%
\pgfpathlineto{\pgfqpoint{4.604713in}{0.896826in}}%
\pgfpathlineto{\pgfqpoint{4.604713in}{0.896826in}}%
\pgfusepath{stroke}%
\end{pgfscope}%
\begin{pgfscope}%
\pgfpathrectangle{\pgfqpoint{0.854460in}{0.571603in}}{\pgfqpoint{6.885100in}{5.225635in}}%
\pgfusepath{clip}%
\pgfsetbuttcap%
\pgfsetroundjoin%
\pgfsetlinewidth{1.505625pt}%
\definecolor{currentstroke}{rgb}{0.277134,0.185228,0.489898}%
\pgfsetstrokecolor{currentstroke}%
\pgfsetdash{}{0pt}%
\pgfpathmoveto{\pgfqpoint{4.971131in}{0.773930in}}%
\pgfpathlineto{\pgfqpoint{4.971681in}{0.773754in}}%
\pgfpathlineto{\pgfqpoint{5.006279in}{0.762647in}}%
\pgfpathlineto{\pgfqpoint{5.028755in}{0.755420in}}%
\pgfpathlineto{\pgfqpoint{5.040878in}{0.751595in}}%
\pgfpathlineto{\pgfqpoint{5.075476in}{0.740706in}}%
\pgfpathlineto{\pgfqpoint{5.110075in}{0.729751in}}%
\pgfpathlineto{\pgfqpoint{5.111953in}{0.729160in}}%
\pgfpathlineto{\pgfqpoint{5.144673in}{0.719061in}}%
\pgfpathlineto{\pgfqpoint{5.179272in}{0.708326in}}%
\pgfpathlineto{\pgfqpoint{5.196764in}{0.702901in}}%
\pgfpathlineto{\pgfqpoint{5.213870in}{0.697698in}}%
\pgfpathlineto{\pgfqpoint{5.248469in}{0.687181in}}%
\pgfpathlineto{\pgfqpoint{5.282943in}{0.676641in}}%
\pgfpathlineto{\pgfqpoint{5.283067in}{0.676604in}}%
\pgfpathlineto{\pgfqpoint{5.317666in}{0.666305in}}%
\pgfpathlineto{\pgfqpoint{5.352264in}{0.655947in}}%
\pgfpathlineto{\pgfqpoint{5.370855in}{0.650382in}}%
\pgfpathlineto{\pgfqpoint{5.386863in}{0.645685in}}%
\pgfpathlineto{\pgfqpoint{5.421461in}{0.635546in}}%
\pgfpathlineto{\pgfqpoint{5.456060in}{0.625347in}}%
\pgfpathlineto{\pgfqpoint{5.460240in}{0.624122in}}%
\pgfpathlineto{\pgfqpoint{5.490658in}{0.615386in}}%
\pgfpathlineto{\pgfqpoint{5.525257in}{0.605409in}}%
\pgfpathlineto{\pgfqpoint{5.551359in}{0.597863in}}%
\pgfpathlineto{\pgfqpoint{5.559855in}{0.595456in}}%
\pgfpathlineto{\pgfqpoint{5.594454in}{0.585699in}}%
\pgfpathlineto{\pgfqpoint{5.629052in}{0.575887in}}%
\pgfpathlineto{\pgfqpoint{5.644183in}{0.571603in}}%
\pgfusepath{stroke}%
\end{pgfscope}%
\begin{pgfscope}%
\pgfpathrectangle{\pgfqpoint{0.854460in}{0.571603in}}{\pgfqpoint{6.885100in}{5.225635in}}%
\pgfusepath{clip}%
\pgfsetbuttcap%
\pgfsetroundjoin%
\pgfsetlinewidth{1.505625pt}%
\definecolor{currentstroke}{rgb}{0.277134,0.185228,0.489898}%
\pgfsetstrokecolor{currentstroke}%
\pgfsetdash{}{0pt}%
\pgfpathmoveto{\pgfqpoint{7.739560in}{1.551858in}}%
\pgfpathlineto{\pgfqpoint{7.715568in}{1.569463in}}%
\pgfpathlineto{\pgfqpoint{7.704962in}{1.577232in}}%
\pgfpathlineto{\pgfqpoint{7.679852in}{1.595723in}}%
\pgfpathlineto{\pgfqpoint{7.670363in}{1.602700in}}%
\pgfpathlineto{\pgfqpoint{7.644276in}{1.621982in}}%
\pgfpathlineto{\pgfqpoint{7.635765in}{1.628267in}}%
\pgfpathlineto{\pgfqpoint{7.608845in}{1.648242in}}%
\pgfpathlineto{\pgfqpoint{7.601166in}{1.653936in}}%
\pgfpathlineto{\pgfqpoint{7.573564in}{1.674501in}}%
\pgfpathlineto{\pgfqpoint{7.566568in}{1.679713in}}%
\pgfpathlineto{\pgfqpoint{7.538437in}{1.700761in}}%
\pgfpathlineto{\pgfqpoint{7.531969in}{1.705601in}}%
\pgfpathlineto{\pgfqpoint{7.503469in}{1.727020in}}%
\pgfpathlineto{\pgfqpoint{7.497371in}{1.731605in}}%
\pgfpathlineto{\pgfqpoint{7.468663in}{1.753280in}}%
\pgfpathlineto{\pgfqpoint{7.462772in}{1.757731in}}%
\pgfpathlineto{\pgfqpoint{7.434024in}{1.779539in}}%
\pgfpathlineto{\pgfqpoint{7.428174in}{1.783982in}}%
\pgfpathlineto{\pgfqpoint{7.399555in}{1.805799in}}%
\pgfpathlineto{\pgfqpoint{7.393575in}{1.810364in}}%
\pgfpathlineto{\pgfqpoint{7.365261in}{1.832058in}}%
\pgfpathlineto{\pgfqpoint{7.358977in}{1.836882in}}%
\pgfpathlineto{\pgfqpoint{7.331144in}{1.858318in}}%
\pgfpathlineto{\pgfqpoint{7.324378in}{1.863540in}}%
\pgfpathlineto{\pgfqpoint{7.297210in}{1.884577in}}%
\pgfpathlineto{\pgfqpoint{7.289780in}{1.890345in}}%
\pgfpathlineto{\pgfqpoint{7.269215in}{1.906356in}}%
\pgfusepath{stroke}%
\end{pgfscope}%
\begin{pgfscope}%
\pgfpathrectangle{\pgfqpoint{0.854460in}{0.571603in}}{\pgfqpoint{6.885100in}{5.225635in}}%
\pgfusepath{clip}%
\pgfsetbuttcap%
\pgfsetroundjoin%
\pgfsetlinewidth{1.505625pt}%
\definecolor{currentstroke}{rgb}{0.277134,0.185228,0.489898}%
\pgfsetstrokecolor{currentstroke}%
\pgfsetdash{}{0pt}%
\pgfpathmoveto{\pgfqpoint{6.966019in}{2.149296in}}%
\pgfpathlineto{\pgfqpoint{6.839999in}{2.254756in}}%
\pgfpathlineto{\pgfqpoint{6.720885in}{2.357248in}}%
\pgfpathlineto{\pgfqpoint{6.597810in}{2.466539in}}%
\pgfpathlineto{\pgfqpoint{6.487869in}{2.567323in}}%
\pgfpathlineto{\pgfqpoint{6.377073in}{2.672361in}}%
\pgfpathlineto{\pgfqpoint{6.270230in}{2.777399in}}%
\pgfpathlineto{\pgfqpoint{6.167390in}{2.882437in}}%
\pgfpathlineto{\pgfqpoint{6.068610in}{2.987475in}}%
\pgfpathlineto{\pgfqpoint{5.973956in}{3.092513in}}%
\pgfpathlineto{\pgfqpoint{5.905646in}{3.171291in}}%
\pgfpathlineto{\pgfqpoint{5.836643in}{3.253751in}}%
\pgfpathlineto{\pgfqpoint{5.767446in}{3.339698in}}%
\pgfpathlineto{\pgfqpoint{5.714641in}{3.407626in}}%
\pgfpathlineto{\pgfqpoint{5.636544in}{3.512664in}}%
\pgfpathlineto{\pgfqpoint{5.562590in}{3.617702in}}%
\pgfpathlineto{\pgfqpoint{5.490658in}{3.725992in}}%
\pgfpathlineto{\pgfqpoint{5.442963in}{3.801519in}}%
\pgfpathlineto{\pgfqpoint{5.386863in}{3.895212in}}%
\pgfpathlineto{\pgfqpoint{5.350337in}{3.959075in}}%
\pgfpathlineto{\pgfqpoint{5.307286in}{4.037854in}}%
\pgfpathlineto{\pgfqpoint{5.266430in}{4.116632in}}%
\pgfpathlineto{\pgfqpoint{5.227704in}{4.195411in}}%
\pgfpathlineto{\pgfqpoint{5.179272in}{4.300579in}}%
\pgfpathlineto{\pgfqpoint{5.144673in}{4.380773in}}%
\pgfpathlineto{\pgfqpoint{5.110075in}{4.465985in}}%
\pgfpathlineto{\pgfqpoint{5.082848in}{4.536784in}}%
\pgfpathlineto{\pgfqpoint{5.045100in}{4.641822in}}%
\pgfpathlineto{\pgfqpoint{5.006279in}{4.758962in}}%
\pgfpathlineto{\pgfqpoint{4.977494in}{4.851897in}}%
\pgfpathlineto{\pgfqpoint{4.937082in}{4.992203in}}%
\pgfpathlineto{\pgfqpoint{4.902484in}{5.119275in}}%
\pgfpathlineto{\pgfqpoint{4.817661in}{5.429606in}}%
\pgfpathlineto{\pgfqpoint{4.793615in}{5.508384in}}%
\pgfpathlineto{\pgfqpoint{4.764090in}{5.595658in}}%
\pgfpathlineto{\pgfqpoint{4.737079in}{5.665941in}}%
\pgfpathlineto{\pgfqpoint{4.714428in}{5.718460in}}%
\pgfpathlineto{\pgfqpoint{4.689141in}{5.770979in}}%
\pgfpathlineto{\pgfqpoint{4.675185in}{5.797238in}}%
\pgfpathlineto{\pgfqpoint{4.675185in}{5.797238in}}%
\pgfusepath{stroke}%
\end{pgfscope}%
\begin{pgfscope}%
\pgfpathrectangle{\pgfqpoint{0.854460in}{0.571603in}}{\pgfqpoint{6.885100in}{5.225635in}}%
\pgfusepath{clip}%
\pgfsetbuttcap%
\pgfsetroundjoin%
\pgfsetlinewidth{1.505625pt}%
\definecolor{currentstroke}{rgb}{0.271828,0.209303,0.504434}%
\pgfsetstrokecolor{currentstroke}%
\pgfsetdash{}{0pt}%
\pgfpathmoveto{\pgfqpoint{2.125051in}{5.797238in}}%
\pgfpathlineto{\pgfqpoint{2.055339in}{5.744720in}}%
\pgfpathlineto{\pgfqpoint{1.989467in}{5.692201in}}%
\pgfpathlineto{\pgfqpoint{1.927014in}{5.639577in}}%
\pgfpathlineto{\pgfqpoint{1.857817in}{5.577523in}}%
\pgfpathlineto{\pgfqpoint{1.788620in}{5.511302in}}%
\pgfpathlineto{\pgfqpoint{1.734327in}{5.455865in}}%
\pgfpathlineto{\pgfqpoint{1.684824in}{5.402590in}}%
\pgfpathlineto{\pgfqpoint{1.639432in}{5.350827in}}%
\pgfpathlineto{\pgfqpoint{1.595728in}{5.298308in}}%
\pgfpathlineto{\pgfqpoint{1.546430in}{5.235472in}}%
\pgfpathlineto{\pgfqpoint{1.496409in}{5.167011in}}%
\pgfpathlineto{\pgfqpoint{1.460420in}{5.114492in}}%
\pgfpathlineto{\pgfqpoint{1.426442in}{5.061973in}}%
\pgfpathlineto{\pgfqpoint{1.394399in}{5.009454in}}%
\pgfpathlineto{\pgfqpoint{1.364216in}{4.956935in}}%
\pgfpathlineto{\pgfqpoint{1.335817in}{4.904416in}}%
\pgfpathlineto{\pgfqpoint{1.296600in}{4.825638in}}%
\pgfpathlineto{\pgfqpoint{1.261141in}{4.746860in}}%
\pgfpathlineto{\pgfqpoint{1.229305in}{4.668081in}}%
\pgfpathlineto{\pgfqpoint{1.200445in}{4.587752in}}%
\pgfpathlineto{\pgfqpoint{1.176158in}{4.510524in}}%
\pgfpathlineto{\pgfqpoint{1.161392in}{4.458005in}}%
\pgfpathlineto{\pgfqpoint{1.142060in}{4.379227in}}%
\pgfpathlineto{\pgfqpoint{1.125876in}{4.300449in}}%
\pgfpathlineto{\pgfqpoint{1.112932in}{4.221670in}}%
\pgfpathlineto{\pgfqpoint{1.103039in}{4.142892in}}%
\pgfpathlineto{\pgfqpoint{1.096237in}{4.064113in}}%
\pgfpathlineto{\pgfqpoint{1.092607in}{3.985335in}}%
\pgfpathlineto{\pgfqpoint{1.092038in}{3.906556in}}%
\pgfpathlineto{\pgfqpoint{1.094540in}{3.827778in}}%
\pgfpathlineto{\pgfqpoint{1.100189in}{3.749000in}}%
\pgfpathlineto{\pgfqpoint{1.109015in}{3.670221in}}%
\pgfpathlineto{\pgfqpoint{1.120967in}{3.591443in}}%
\pgfpathlineto{\pgfqpoint{1.136124in}{3.512664in}}%
\pgfpathlineto{\pgfqpoint{1.154626in}{3.433886in}}%
\pgfpathlineto{\pgfqpoint{1.168747in}{3.381367in}}%
\pgfpathlineto{\pgfqpoint{1.192870in}{3.302589in}}%
\pgfpathlineto{\pgfqpoint{1.210870in}{3.250070in}}%
\pgfpathlineto{\pgfqpoint{1.240784in}{3.171291in}}%
\pgfpathlineto{\pgfqpoint{1.274347in}{3.092513in}}%
\pgfpathlineto{\pgfqpoint{1.311671in}{3.013734in}}%
\pgfpathlineto{\pgfqpoint{1.338839in}{2.960813in}}%
\pgfpathlineto{\pgfqpoint{1.382477in}{2.882437in}}%
\pgfpathlineto{\pgfqpoint{1.413906in}{2.829918in}}%
\pgfpathlineto{\pgfqpoint{1.447188in}{2.777399in}}%
\pgfpathlineto{\pgfqpoint{1.482371in}{2.724880in}}%
\pgfpathlineto{\pgfqpoint{1.519496in}{2.672361in}}%
\pgfpathlineto{\pgfqpoint{1.558603in}{2.619842in}}%
\pgfpathlineto{\pgfqpoint{1.615627in}{2.547675in}}%
\pgfpathlineto{\pgfqpoint{1.665310in}{2.488545in}}%
\pgfpathlineto{\pgfqpoint{1.719423in}{2.427543in}}%
\pgfpathlineto{\pgfqpoint{1.785439in}{2.357248in}}%
\pgfpathlineto{\pgfqpoint{1.837523in}{2.304729in}}%
\pgfpathlineto{\pgfqpoint{1.892415in}{2.251678in}}%
\pgfpathlineto{\pgfqpoint{1.961612in}{2.188111in}}%
\pgfpathlineto{\pgfqpoint{2.030809in}{2.127561in}}%
\pgfpathlineto{\pgfqpoint{2.101676in}{2.068393in}}%
\pgfpathlineto{\pgfqpoint{2.169203in}{2.014511in}}%
\pgfpathlineto{\pgfqpoint{2.238400in}{1.961521in}}%
\pgfpathlineto{\pgfqpoint{2.307597in}{1.910554in}}%
\pgfpathlineto{\pgfqpoint{2.411393in}{1.837653in}}%
\pgfpathlineto{\pgfqpoint{2.480590in}{1.791135in}}%
\pgfpathlineto{\pgfqpoint{2.549787in}{1.746096in}}%
\pgfpathlineto{\pgfqpoint{2.653582in}{1.681176in}}%
\pgfpathlineto{\pgfqpoint{2.722779in}{1.639516in}}%
\pgfpathlineto{\pgfqpoint{2.826574in}{1.579211in}}%
\pgfpathlineto{\pgfqpoint{2.930370in}{1.521289in}}%
\pgfpathlineto{\pgfqpoint{3.034165in}{1.465576in}}%
\pgfpathlineto{\pgfqpoint{3.137961in}{1.411906in}}%
\pgfpathlineto{\pgfqpoint{3.276355in}{1.343323in}}%
\pgfpathlineto{\pgfqpoint{3.380150in}{1.293800in}}%
\pgfpathlineto{\pgfqpoint{3.483946in}{1.245820in}}%
\pgfpathlineto{\pgfqpoint{3.622340in}{1.184098in}}%
\pgfpathlineto{\pgfqpoint{3.726135in}{1.139370in}}%
\pgfpathlineto{\pgfqpoint{3.864529in}{1.081647in}}%
\pgfpathlineto{\pgfqpoint{3.910398in}{1.062948in}}%
\pgfpathlineto{\pgfqpoint{3.910398in}{1.062948in}}%
\pgfusepath{stroke}%
\end{pgfscope}%
\begin{pgfscope}%
\pgfpathrectangle{\pgfqpoint{0.854460in}{0.571603in}}{\pgfqpoint{6.885100in}{5.225635in}}%
\pgfusepath{clip}%
\pgfsetbuttcap%
\pgfsetroundjoin%
\pgfsetlinewidth{1.505625pt}%
\definecolor{currentstroke}{rgb}{0.271828,0.209303,0.504434}%
\pgfsetstrokecolor{currentstroke}%
\pgfsetdash{}{0pt}%
\pgfpathmoveto{\pgfqpoint{4.270983in}{0.923306in}}%
\pgfpathlineto{\pgfqpoint{4.279711in}{0.920064in}}%
\pgfpathlineto{\pgfqpoint{4.298770in}{0.912976in}}%
\pgfpathlineto{\pgfqpoint{4.314310in}{0.907297in}}%
\pgfpathlineto{\pgfqpoint{4.348908in}{0.894655in}}%
\pgfpathlineto{\pgfqpoint{4.370595in}{0.886717in}}%
\pgfpathlineto{\pgfqpoint{4.383507in}{0.882072in}}%
\pgfpathlineto{\pgfqpoint{4.418105in}{0.869641in}}%
\pgfpathlineto{\pgfqpoint{4.443588in}{0.860458in}}%
\pgfpathlineto{\pgfqpoint{4.452704in}{0.857229in}}%
\pgfpathlineto{\pgfqpoint{4.487302in}{0.845007in}}%
\pgfpathlineto{\pgfqpoint{4.517760in}{0.834198in}}%
\pgfpathlineto{\pgfqpoint{4.521901in}{0.832754in}}%
\pgfpathlineto{\pgfqpoint{4.556499in}{0.820740in}}%
\pgfpathlineto{\pgfqpoint{4.591098in}{0.808654in}}%
\pgfpathlineto{\pgfqpoint{4.593154in}{0.807939in}}%
\pgfpathlineto{\pgfqpoint{4.625696in}{0.796825in}}%
\pgfpathlineto{\pgfqpoint{4.660295in}{0.784947in}}%
\pgfpathlineto{\pgfqpoint{4.669834in}{0.781679in}}%
\pgfpathlineto{\pgfqpoint{4.694893in}{0.773249in}}%
\pgfpathlineto{\pgfqpoint{4.729491in}{0.761577in}}%
\pgfpathlineto{\pgfqpoint{4.747734in}{0.755420in}}%
\pgfpathlineto{\pgfqpoint{4.764090in}{0.749998in}}%
\pgfpathlineto{\pgfqpoint{4.798688in}{0.738532in}}%
\pgfpathlineto{\pgfqpoint{4.826859in}{0.729160in}}%
\pgfpathlineto{\pgfqpoint{4.833287in}{0.727060in}}%
\pgfpathlineto{\pgfqpoint{4.867885in}{0.715798in}}%
\pgfpathlineto{\pgfqpoint{4.902484in}{0.704469in}}%
\pgfpathlineto{\pgfqpoint{4.907292in}{0.702901in}}%
\pgfpathlineto{\pgfqpoint{4.937082in}{0.693363in}}%
\pgfpathlineto{\pgfqpoint{4.971681in}{0.682238in}}%
\pgfpathlineto{\pgfqpoint{4.989086in}{0.676641in}}%
\pgfpathlineto{\pgfqpoint{5.006279in}{0.671214in}}%
\pgfpathlineto{\pgfqpoint{5.040878in}{0.660292in}}%
\pgfpathlineto{\pgfqpoint{5.072125in}{0.650382in}}%
\pgfpathlineto{\pgfqpoint{5.075476in}{0.649338in}}%
\pgfpathlineto{\pgfqpoint{5.110075in}{0.638619in}}%
\pgfpathlineto{\pgfqpoint{5.144673in}{0.627837in}}%
\pgfpathlineto{\pgfqpoint{5.156616in}{0.624122in}}%
\pgfpathlineto{\pgfqpoint{5.179272in}{0.617206in}}%
\pgfpathlineto{\pgfqpoint{5.213870in}{0.606627in}}%
\pgfpathlineto{\pgfqpoint{5.242424in}{0.597863in}}%
\pgfpathlineto{\pgfqpoint{5.248469in}{0.596042in}}%
\pgfpathlineto{\pgfqpoint{5.283067in}{0.585664in}}%
\pgfpathlineto{\pgfqpoint{5.317666in}{0.575226in}}%
\pgfpathlineto{\pgfqpoint{5.329698in}{0.571603in}}%
\pgfusepath{stroke}%
\end{pgfscope}%
\begin{pgfscope}%
\pgfpathrectangle{\pgfqpoint{0.854460in}{0.571603in}}{\pgfqpoint{6.885100in}{5.225635in}}%
\pgfusepath{clip}%
\pgfsetbuttcap%
\pgfsetroundjoin%
\pgfsetlinewidth{1.505625pt}%
\definecolor{currentstroke}{rgb}{0.271828,0.209303,0.504434}%
\pgfsetstrokecolor{currentstroke}%
\pgfsetdash{}{0pt}%
\pgfpathmoveto{\pgfqpoint{7.739560in}{1.721295in}}%
\pgfpathlineto{\pgfqpoint{7.731768in}{1.727020in}}%
\pgfpathlineto{\pgfqpoint{7.704962in}{1.746740in}}%
\pgfpathlineto{\pgfqpoint{7.696106in}{1.753280in}}%
\pgfpathlineto{\pgfqpoint{7.687376in}{1.759738in}}%
\pgfusepath{stroke}%
\end{pgfscope}%
\begin{pgfscope}%
\pgfpathrectangle{\pgfqpoint{0.854460in}{0.571603in}}{\pgfqpoint{6.885100in}{5.225635in}}%
\pgfusepath{clip}%
\pgfsetbuttcap%
\pgfsetroundjoin%
\pgfsetlinewidth{1.505625pt}%
\definecolor{currentstroke}{rgb}{0.271828,0.209303,0.504434}%
\pgfsetstrokecolor{currentstroke}%
\pgfsetdash{}{0pt}%
\pgfpathmoveto{\pgfqpoint{7.378082in}{1.994571in}}%
\pgfpathlineto{\pgfqpoint{7.251563in}{2.094653in}}%
\pgfpathlineto{\pgfqpoint{7.116787in}{2.204301in}}%
\pgfpathlineto{\pgfqpoint{6.996943in}{2.304729in}}%
\pgfpathlineto{\pgfqpoint{6.874598in}{2.410562in}}%
\pgfpathlineto{\pgfqpoint{6.757965in}{2.514804in}}%
\pgfpathlineto{\pgfqpoint{6.644551in}{2.619842in}}%
\pgfpathlineto{\pgfqpoint{6.535296in}{2.724880in}}%
\pgfpathlineto{\pgfqpoint{6.430236in}{2.829918in}}%
\pgfpathlineto{\pgfqpoint{6.329415in}{2.934956in}}%
\pgfpathlineto{\pgfqpoint{6.232886in}{3.039994in}}%
\pgfpathlineto{\pgfqpoint{6.140716in}{3.145032in}}%
\pgfpathlineto{\pgfqpoint{6.074442in}{3.223810in}}%
\pgfpathlineto{\pgfqpoint{6.009636in}{3.303849in}}%
\pgfpathlineto{\pgfqpoint{5.940439in}{3.392983in}}%
\pgfpathlineto{\pgfqpoint{5.890289in}{3.460145in}}%
\pgfpathlineto{\pgfqpoint{5.833867in}{3.538924in}}%
\pgfpathlineto{\pgfqpoint{5.779822in}{3.617702in}}%
\pgfpathlineto{\pgfqpoint{5.728278in}{3.696481in}}%
\pgfpathlineto{\pgfqpoint{5.679107in}{3.775259in}}%
\pgfpathlineto{\pgfqpoint{5.629052in}{3.859935in}}%
\pgfpathlineto{\pgfqpoint{5.588060in}{3.932816in}}%
\pgfpathlineto{\pgfqpoint{5.546065in}{4.011594in}}%
\pgfpathlineto{\pgfqpoint{5.506425in}{4.090373in}}%
\pgfpathlineto{\pgfqpoint{5.469089in}{4.169151in}}%
\pgfpathlineto{\pgfqpoint{5.434011in}{4.247930in}}%
\pgfpathlineto{\pgfqpoint{5.401140in}{4.326708in}}%
\pgfpathlineto{\pgfqpoint{5.370427in}{4.405486in}}%
\pgfpathlineto{\pgfqpoint{5.341820in}{4.484265in}}%
\pgfpathlineto{\pgfqpoint{5.315266in}{4.563043in}}%
\pgfpathlineto{\pgfqpoint{5.290594in}{4.641822in}}%
\pgfpathlineto{\pgfqpoint{5.260586in}{4.746860in}}%
\pgfpathlineto{\pgfqpoint{5.233634in}{4.851897in}}%
\pgfpathlineto{\pgfqpoint{5.209477in}{4.956935in}}%
\pgfpathlineto{\pgfqpoint{5.187668in}{5.061973in}}%
\pgfpathlineto{\pgfqpoint{5.163151in}{5.193271in}}%
\pgfpathlineto{\pgfqpoint{5.136429in}{5.350827in}}%
\pgfpathlineto{\pgfqpoint{5.096657in}{5.587163in}}%
\pgfpathlineto{\pgfqpoint{5.075476in}{5.697519in}}%
\pgfpathlineto{\pgfqpoint{5.059112in}{5.770979in}}%
\pgfpathlineto{\pgfqpoint{5.052729in}{5.797238in}}%
\pgfpathlineto{\pgfqpoint{5.052729in}{5.797238in}}%
\pgfusepath{stroke}%
\end{pgfscope}%
\begin{pgfscope}%
\pgfpathrectangle{\pgfqpoint{0.854460in}{0.571603in}}{\pgfqpoint{6.885100in}{5.225635in}}%
\pgfusepath{clip}%
\pgfsetbuttcap%
\pgfsetroundjoin%
\pgfsetlinewidth{1.505625pt}%
\definecolor{currentstroke}{rgb}{0.263663,0.237631,0.518762}%
\pgfsetstrokecolor{currentstroke}%
\pgfsetdash{}{0pt}%
\pgfpathmoveto{\pgfqpoint{1.859943in}{5.797238in}}%
\pgfpathlineto{\pgfqpoint{1.798134in}{5.744720in}}%
\pgfpathlineto{\pgfqpoint{1.739407in}{5.692201in}}%
\pgfpathlineto{\pgfqpoint{1.683525in}{5.639682in}}%
\pgfpathlineto{\pgfqpoint{1.615627in}{5.571860in}}%
\pgfpathlineto{\pgfqpoint{1.555989in}{5.508384in}}%
\pgfpathlineto{\pgfqpoint{1.509312in}{5.455865in}}%
\pgfpathlineto{\pgfqpoint{1.465096in}{5.403346in}}%
\pgfpathlineto{\pgfqpoint{1.423074in}{5.350827in}}%
\pgfpathlineto{\pgfqpoint{1.373438in}{5.285062in}}%
\pgfpathlineto{\pgfqpoint{1.327242in}{5.219530in}}%
\pgfpathlineto{\pgfqpoint{1.292404in}{5.167011in}}%
\pgfpathlineto{\pgfqpoint{1.259448in}{5.114492in}}%
\pgfpathlineto{\pgfqpoint{1.228305in}{5.061973in}}%
\pgfpathlineto{\pgfqpoint{1.184951in}{4.983195in}}%
\pgfpathlineto{\pgfqpoint{1.158123in}{4.930676in}}%
\pgfpathlineto{\pgfqpoint{1.121001in}{4.851897in}}%
\pgfpathlineto{\pgfqpoint{1.087422in}{4.773119in}}%
\pgfpathlineto{\pgfqpoint{1.057256in}{4.694341in}}%
\pgfpathlineto{\pgfqpoint{1.030434in}{4.615562in}}%
\pgfpathlineto{\pgfqpoint{1.014420in}{4.563043in}}%
\pgfpathlineto{\pgfqpoint{0.992854in}{4.484138in}}%
\pgfpathlineto{\pgfqpoint{0.974637in}{4.405486in}}%
\pgfpathlineto{\pgfqpoint{0.959291in}{4.326708in}}%
\pgfpathlineto{\pgfqpoint{0.950839in}{4.274189in}}%
\pgfpathlineto{\pgfqpoint{0.940611in}{4.195411in}}%
\pgfpathlineto{\pgfqpoint{0.933314in}{4.116632in}}%
\pgfpathlineto{\pgfqpoint{0.928969in}{4.037854in}}%
\pgfpathlineto{\pgfqpoint{0.927594in}{3.959075in}}%
\pgfpathlineto{\pgfqpoint{0.929201in}{3.880297in}}%
\pgfpathlineto{\pgfqpoint{0.933797in}{3.801519in}}%
\pgfpathlineto{\pgfqpoint{0.941385in}{3.722740in}}%
\pgfpathlineto{\pgfqpoint{0.951961in}{3.643962in}}%
\pgfpathlineto{\pgfqpoint{0.960713in}{3.591443in}}%
\pgfpathlineto{\pgfqpoint{0.976537in}{3.512664in}}%
\pgfpathlineto{\pgfqpoint{0.995412in}{3.433886in}}%
\pgfpathlineto{\pgfqpoint{1.017620in}{3.355107in}}%
\pgfpathlineto{\pgfqpoint{1.034215in}{3.302589in}}%
\pgfpathlineto{\pgfqpoint{1.062051in}{3.223384in}}%
\pgfpathlineto{\pgfqpoint{1.093089in}{3.145032in}}%
\pgfpathlineto{\pgfqpoint{1.115882in}{3.092513in}}%
\pgfpathlineto{\pgfqpoint{1.140218in}{3.039994in}}%
\pgfpathlineto{\pgfqpoint{1.179847in}{2.961215in}}%
\pgfpathlineto{\pgfqpoint{1.208344in}{2.908696in}}%
\pgfpathlineto{\pgfqpoint{1.238564in}{2.856177in}}%
\pgfpathlineto{\pgfqpoint{1.287306in}{2.777399in}}%
\pgfpathlineto{\pgfqpoint{1.322046in}{2.724880in}}%
\pgfpathlineto{\pgfqpoint{1.358657in}{2.672361in}}%
\pgfpathlineto{\pgfqpoint{1.408036in}{2.605521in}}%
\pgfpathlineto{\pgfqpoint{1.458678in}{2.541064in}}%
\pgfpathlineto{\pgfqpoint{1.511832in}{2.477270in}}%
\pgfpathlineto{\pgfqpoint{1.571417in}{2.409766in}}%
\pgfpathlineto{\pgfqpoint{1.620241in}{2.357248in}}%
\pgfpathlineto{\pgfqpoint{1.684824in}{2.291271in}}%
\pgfpathlineto{\pgfqpoint{1.752196in}{2.225950in}}%
\pgfpathlineto{\pgfqpoint{1.823218in}{2.160791in}}%
\pgfpathlineto{\pgfqpoint{1.899015in}{2.094653in}}%
\pgfpathlineto{\pgfqpoint{1.962058in}{2.042134in}}%
\pgfpathlineto{\pgfqpoint{2.030809in}{1.987232in}}%
\pgfpathlineto{\pgfqpoint{2.100006in}{1.934163in}}%
\pgfpathlineto{\pgfqpoint{2.169203in}{1.883093in}}%
\pgfpathlineto{\pgfqpoint{2.241012in}{1.832058in}}%
\pgfpathlineto{\pgfqpoint{2.342196in}{1.763284in}}%
\pgfpathlineto{\pgfqpoint{2.411393in}{1.718080in}}%
\pgfpathlineto{\pgfqpoint{2.515188in}{1.652846in}}%
\pgfpathlineto{\pgfqpoint{2.584385in}{1.611000in}}%
\pgfpathlineto{\pgfqpoint{2.688180in}{1.550347in}}%
\pgfpathlineto{\pgfqpoint{2.791976in}{1.492062in}}%
\pgfpathlineto{\pgfqpoint{2.891723in}{1.438166in}}%
\pgfpathlineto{\pgfqpoint{2.999567in}{1.382016in}}%
\pgfpathlineto{\pgfqpoint{3.137961in}{1.312849in}}%
\pgfpathlineto{\pgfqpoint{3.241756in}{1.262937in}}%
\pgfpathlineto{\pgfqpoint{3.273361in}{1.248038in}}%
\pgfpathlineto{\pgfqpoint{3.273361in}{1.248038in}}%
\pgfusepath{stroke}%
\end{pgfscope}%
\begin{pgfscope}%
\pgfpathrectangle{\pgfqpoint{0.854460in}{0.571603in}}{\pgfqpoint{6.885100in}{5.225635in}}%
\pgfusepath{clip}%
\pgfsetbuttcap%
\pgfsetroundjoin%
\pgfsetlinewidth{1.505625pt}%
\definecolor{currentstroke}{rgb}{0.263663,0.237631,0.518762}%
\pgfsetstrokecolor{currentstroke}%
\pgfsetdash{}{0pt}%
\pgfpathmoveto{\pgfqpoint{3.626733in}{1.090444in}}%
\pgfpathlineto{\pgfqpoint{3.656938in}{1.077676in}}%
\pgfpathlineto{\pgfqpoint{3.673829in}{1.070533in}}%
\pgfpathlineto{\pgfqpoint{3.691537in}{1.063165in}}%
\pgfpathlineto{\pgfqpoint{3.726135in}{1.048759in}}%
\pgfpathlineto{\pgfqpoint{3.736922in}{1.044274in}}%
\pgfpathlineto{\pgfqpoint{3.760734in}{1.034533in}}%
\pgfpathlineto{\pgfqpoint{3.795332in}{1.020344in}}%
\pgfpathlineto{\pgfqpoint{3.801030in}{1.018014in}}%
\pgfpathlineto{\pgfqpoint{3.829931in}{1.006388in}}%
\pgfpathlineto{\pgfqpoint{3.864529in}{0.992414in}}%
\pgfpathlineto{\pgfqpoint{3.866169in}{0.991755in}}%
\pgfpathlineto{\pgfqpoint{3.899128in}{0.978715in}}%
\pgfpathlineto{\pgfqpoint{3.932379in}{0.965495in}}%
\pgfpathlineto{\pgfqpoint{3.933726in}{0.964968in}}%
\pgfpathlineto{\pgfqpoint{3.968325in}{0.951497in}}%
\pgfpathlineto{\pgfqpoint{3.999663in}{0.939236in}}%
\pgfpathlineto{\pgfqpoint{4.002923in}{0.937981in}}%
\pgfpathlineto{\pgfqpoint{4.037522in}{0.924718in}}%
\pgfpathlineto{\pgfqpoint{4.068011in}{0.912976in}}%
\pgfpathlineto{\pgfqpoint{4.072120in}{0.911420in}}%
\pgfpathlineto{\pgfqpoint{4.106719in}{0.898363in}}%
\pgfpathlineto{\pgfqpoint{4.137438in}{0.886717in}}%
\pgfpathlineto{\pgfqpoint{4.141317in}{0.885271in}}%
\pgfpathlineto{\pgfqpoint{4.175916in}{0.872418in}}%
\pgfpathlineto{\pgfqpoint{4.207954in}{0.860458in}}%
\pgfpathlineto{\pgfqpoint{4.210514in}{0.859518in}}%
\pgfpathlineto{\pgfqpoint{4.245113in}{0.846868in}}%
\pgfpathlineto{\pgfqpoint{4.279570in}{0.834198in}}%
\pgfpathlineto{\pgfqpoint{4.279711in}{0.834147in}}%
\pgfpathlineto{\pgfqpoint{4.314310in}{0.821699in}}%
\pgfpathlineto{\pgfqpoint{4.348908in}{0.809179in}}%
\pgfpathlineto{\pgfqpoint{4.352350in}{0.807939in}}%
\pgfpathlineto{\pgfqpoint{4.383507in}{0.796897in}}%
\pgfpathlineto{\pgfqpoint{4.418105in}{0.784578in}}%
\pgfpathlineto{\pgfqpoint{4.426267in}{0.781679in}}%
\pgfpathlineto{\pgfqpoint{4.452704in}{0.772448in}}%
\pgfpathlineto{\pgfqpoint{4.487302in}{0.760329in}}%
\pgfpathlineto{\pgfqpoint{4.501326in}{0.755420in}}%
\pgfpathlineto{\pgfqpoint{4.521901in}{0.748340in}}%
\pgfpathlineto{\pgfqpoint{4.556499in}{0.736419in}}%
\pgfpathlineto{\pgfqpoint{4.577532in}{0.729160in}}%
\pgfpathlineto{\pgfqpoint{4.591098in}{0.724559in}}%
\pgfpathlineto{\pgfqpoint{4.625696in}{0.712834in}}%
\pgfpathlineto{\pgfqpoint{4.654888in}{0.702901in}}%
\pgfpathlineto{\pgfqpoint{4.660295in}{0.701093in}}%
\pgfpathlineto{\pgfqpoint{4.694893in}{0.689562in}}%
\pgfpathlineto{\pgfqpoint{4.729491in}{0.677966in}}%
\pgfpathlineto{\pgfqpoint{4.733458in}{0.676641in}}%
\pgfpathlineto{\pgfqpoint{4.764090in}{0.666591in}}%
\pgfpathlineto{\pgfqpoint{4.798688in}{0.655190in}}%
\pgfpathlineto{\pgfqpoint{4.813285in}{0.650382in}}%
\pgfpathlineto{\pgfqpoint{4.833287in}{0.643909in}}%
\pgfpathlineto{\pgfqpoint{4.867885in}{0.632701in}}%
\pgfpathlineto{\pgfqpoint{4.894282in}{0.624122in}}%
\pgfpathlineto{\pgfqpoint{4.902484in}{0.621504in}}%
\pgfpathlineto{\pgfqpoint{4.937082in}{0.610487in}}%
\pgfpathlineto{\pgfqpoint{4.971681in}{0.599407in}}%
\pgfpathlineto{\pgfqpoint{4.976522in}{0.597863in}}%
\pgfpathlineto{\pgfqpoint{5.006279in}{0.588537in}}%
\pgfpathlineto{\pgfqpoint{5.040878in}{0.577649in}}%
\pgfpathlineto{\pgfqpoint{5.060074in}{0.571603in}}%
\pgfusepath{stroke}%
\end{pgfscope}%
\begin{pgfscope}%
\pgfpathrectangle{\pgfqpoint{0.854460in}{0.571603in}}{\pgfqpoint{6.885100in}{5.225635in}}%
\pgfusepath{clip}%
\pgfsetbuttcap%
\pgfsetroundjoin%
\pgfsetlinewidth{1.505625pt}%
\definecolor{currentstroke}{rgb}{0.263663,0.237631,0.518762}%
\pgfsetstrokecolor{currentstroke}%
\pgfsetdash{}{0pt}%
\pgfpathmoveto{\pgfqpoint{7.739560in}{1.876059in}}%
\pgfpathlineto{\pgfqpoint{7.601166in}{1.980756in}}%
\pgfpathlineto{\pgfqpoint{7.488235in}{2.068393in}}%
\pgfpathlineto{\pgfqpoint{7.356519in}{2.173431in}}%
\pgfpathlineto{\pgfqpoint{7.220583in}{2.285349in}}%
\pgfpathlineto{\pgfqpoint{7.105186in}{2.383507in}}%
\pgfpathlineto{\pgfqpoint{6.985805in}{2.488545in}}%
\pgfpathlineto{\pgfqpoint{6.870696in}{2.593583in}}%
\pgfpathlineto{\pgfqpoint{6.759883in}{2.698621in}}%
\pgfpathlineto{\pgfqpoint{6.653472in}{2.803659in}}%
\pgfpathlineto{\pgfqpoint{6.551496in}{2.908696in}}%
\pgfpathlineto{\pgfqpoint{6.459416in}{3.007811in}}%
\pgfpathlineto{\pgfqpoint{6.406907in}{3.066253in}}%
\pgfpathlineto{\pgfqpoint{6.321022in}{3.165584in}}%
\pgfpathlineto{\pgfqpoint{6.272464in}{3.223810in}}%
\pgfpathlineto{\pgfqpoint{6.209076in}{3.302589in}}%
\pgfpathlineto{\pgfqpoint{6.148030in}{3.381679in}}%
\pgfpathlineto{\pgfqpoint{6.089927in}{3.460145in}}%
\pgfpathlineto{\pgfqpoint{6.034188in}{3.538924in}}%
\pgfpathlineto{\pgfqpoint{5.975037in}{3.626868in}}%
\pgfpathlineto{\pgfqpoint{5.930339in}{3.696481in}}%
\pgfpathlineto{\pgfqpoint{5.882211in}{3.775259in}}%
\pgfpathlineto{\pgfqpoint{5.836643in}{3.854041in}}%
\pgfpathlineto{\pgfqpoint{5.793494in}{3.932816in}}%
\pgfpathlineto{\pgfqpoint{5.752845in}{4.011594in}}%
\pgfpathlineto{\pgfqpoint{5.714658in}{4.090373in}}%
\pgfpathlineto{\pgfqpoint{5.678895in}{4.169151in}}%
\pgfpathlineto{\pgfqpoint{5.645519in}{4.247930in}}%
\pgfpathlineto{\pgfqpoint{5.614492in}{4.326708in}}%
\pgfpathlineto{\pgfqpoint{5.585780in}{4.405486in}}%
\pgfpathlineto{\pgfqpoint{5.559346in}{4.484265in}}%
\pgfpathlineto{\pgfqpoint{5.535027in}{4.563043in}}%
\pgfpathlineto{\pgfqpoint{5.512874in}{4.641822in}}%
\pgfpathlineto{\pgfqpoint{5.490658in}{4.729769in}}%
\pgfpathlineto{\pgfqpoint{5.474667in}{4.799378in}}%
\pgfpathlineto{\pgfqpoint{5.456060in}{4.890946in}}%
\pgfpathlineto{\pgfqpoint{5.443991in}{4.956935in}}%
\pgfpathlineto{\pgfqpoint{5.427313in}{5.061973in}}%
\pgfpathlineto{\pgfqpoint{5.413244in}{5.167011in}}%
\pgfpathlineto{\pgfqpoint{5.405186in}{5.236740in}}%
\pgfpathlineto{\pgfqpoint{5.405186in}{5.236740in}}%
\pgfusepath{stroke}%
\end{pgfscope}%
\begin{pgfscope}%
\pgfpathrectangle{\pgfqpoint{0.854460in}{0.571603in}}{\pgfqpoint{6.885100in}{5.225635in}}%
\pgfusepath{clip}%
\pgfsetbuttcap%
\pgfsetroundjoin%
\pgfsetlinewidth{1.505625pt}%
\definecolor{currentstroke}{rgb}{0.263663,0.237631,0.518762}%
\pgfsetstrokecolor{currentstroke}%
\pgfsetdash{}{0pt}%
\pgfpathmoveto{\pgfqpoint{5.371696in}{5.628172in}}%
\pgfpathlineto{\pgfqpoint{5.370817in}{5.639682in}}%
\pgfpathlineto{\pgfqpoint{5.368776in}{5.665941in}}%
\pgfpathlineto{\pgfqpoint{5.366688in}{5.692201in}}%
\pgfpathlineto{\pgfqpoint{5.364534in}{5.718460in}}%
\pgfpathlineto{\pgfqpoint{5.362299in}{5.744720in}}%
\pgfpathlineto{\pgfqpoint{5.359964in}{5.770979in}}%
\pgfpathlineto{\pgfqpoint{5.357509in}{5.797238in}}%
\pgfusepath{stroke}%
\end{pgfscope}%
\begin{pgfscope}%
\pgfpathrectangle{\pgfqpoint{0.854460in}{0.571603in}}{\pgfqpoint{6.885100in}{5.225635in}}%
\pgfusepath{clip}%
\pgfsetbuttcap%
\pgfsetroundjoin%
\pgfsetlinewidth{1.505625pt}%
\definecolor{currentstroke}{rgb}{0.253935,0.265254,0.529983}%
\pgfsetstrokecolor{currentstroke}%
\pgfsetdash{}{0pt}%
\pgfpathmoveto{\pgfqpoint{1.636206in}{5.797238in}}%
\pgfpathlineto{\pgfqpoint{1.615627in}{5.778503in}}%
\pgfpathlineto{\pgfqpoint{1.607480in}{5.770979in}}%
\pgfpathlineto{\pgfqpoint{1.586560in}{5.751415in}}%
\pgfusepath{stroke}%
\end{pgfscope}%
\begin{pgfscope}%
\pgfpathrectangle{\pgfqpoint{0.854460in}{0.571603in}}{\pgfqpoint{6.885100in}{5.225635in}}%
\pgfusepath{clip}%
\pgfsetbuttcap%
\pgfsetroundjoin%
\pgfsetlinewidth{1.505625pt}%
\definecolor{currentstroke}{rgb}{0.253935,0.265254,0.529983}%
\pgfsetstrokecolor{currentstroke}%
\pgfsetdash{}{0pt}%
\pgfpathmoveto{\pgfqpoint{1.321460in}{5.466357in}}%
\pgfpathlineto{\pgfqpoint{1.312947in}{5.455865in}}%
\pgfpathlineto{\pgfqpoint{1.304241in}{5.444969in}}%
\pgfpathlineto{\pgfqpoint{1.292160in}{5.429606in}}%
\pgfpathlineto{\pgfqpoint{1.271820in}{5.403346in}}%
\pgfpathlineto{\pgfqpoint{1.269642in}{5.400484in}}%
\pgfpathlineto{\pgfqpoint{1.252120in}{5.377087in}}%
\pgfpathlineto{\pgfqpoint{1.235044in}{5.353936in}}%
\pgfpathlineto{\pgfqpoint{1.232788in}{5.350827in}}%
\pgfpathlineto{\pgfqpoint{1.214077in}{5.324568in}}%
\pgfpathlineto{\pgfqpoint{1.200445in}{5.305123in}}%
\pgfpathlineto{\pgfqpoint{1.195746in}{5.298308in}}%
\pgfpathlineto{\pgfqpoint{1.177971in}{5.272049in}}%
\pgfpathlineto{\pgfqpoint{1.165847in}{5.253823in}}%
\pgfpathlineto{\pgfqpoint{1.160590in}{5.245790in}}%
\pgfpathlineto{\pgfqpoint{1.143739in}{5.219530in}}%
\pgfpathlineto{\pgfqpoint{1.131248in}{5.199708in}}%
\pgfpathlineto{\pgfqpoint{1.127260in}{5.193271in}}%
\pgfpathlineto{\pgfqpoint{1.111319in}{5.167011in}}%
\pgfpathlineto{\pgfqpoint{1.096650in}{5.142388in}}%
\pgfpathlineto{\pgfqpoint{1.095691in}{5.140752in}}%
\pgfpathlineto{\pgfqpoint{1.080646in}{5.114492in}}%
\pgfpathlineto{\pgfqpoint{1.065898in}{5.088233in}}%
\pgfpathlineto{\pgfqpoint{1.062051in}{5.081229in}}%
\pgfpathlineto{\pgfqpoint{1.051657in}{5.061973in}}%
\pgfpathlineto{\pgfqpoint{1.037791in}{5.035714in}}%
\pgfpathlineto{\pgfqpoint{1.027453in}{5.015691in}}%
\pgfpathlineto{\pgfqpoint{1.024288in}{5.009454in}}%
\pgfpathlineto{\pgfqpoint{1.011286in}{4.983195in}}%
\pgfpathlineto{\pgfqpoint{0.998586in}{4.956935in}}%
\pgfpathlineto{\pgfqpoint{0.992854in}{4.944782in}}%
\pgfpathlineto{\pgfqpoint{0.986319in}{4.930676in}}%
\pgfpathlineto{\pgfqpoint{0.974465in}{4.904416in}}%
\pgfpathlineto{\pgfqpoint{0.962916in}{4.878157in}}%
\pgfpathlineto{\pgfqpoint{0.958256in}{4.867263in}}%
\pgfpathlineto{\pgfqpoint{0.951800in}{4.851897in}}%
\pgfpathlineto{\pgfqpoint{0.941078in}{4.825638in}}%
\pgfpathlineto{\pgfqpoint{0.930661in}{4.799378in}}%
\pgfpathlineto{\pgfqpoint{0.923657in}{4.781178in}}%
\pgfpathlineto{\pgfqpoint{0.920612in}{4.773119in}}%
\pgfpathlineto{\pgfqpoint{0.911001in}{4.746860in}}%
\pgfpathlineto{\pgfqpoint{0.901697in}{4.720600in}}%
\pgfpathlineto{\pgfqpoint{0.892701in}{4.694341in}}%
\pgfpathlineto{\pgfqpoint{0.889059in}{4.683323in}}%
\pgfpathlineto{\pgfqpoint{0.884112in}{4.668081in}}%
\pgfpathlineto{\pgfqpoint{0.875898in}{4.641822in}}%
\pgfpathlineto{\pgfqpoint{0.867992in}{4.615562in}}%
\pgfpathlineto{\pgfqpoint{0.860396in}{4.589303in}}%
\pgfpathlineto{\pgfqpoint{0.854460in}{4.567906in}}%
\pgfusepath{stroke}%
\end{pgfscope}%
\begin{pgfscope}%
\pgfpathrectangle{\pgfqpoint{0.854460in}{0.571603in}}{\pgfqpoint{6.885100in}{5.225635in}}%
\pgfusepath{clip}%
\pgfsetbuttcap%
\pgfsetroundjoin%
\pgfsetlinewidth{1.505625pt}%
\definecolor{currentstroke}{rgb}{0.253935,0.265254,0.529983}%
\pgfsetstrokecolor{currentstroke}%
\pgfsetdash{}{0pt}%
\pgfpathmoveto{\pgfqpoint{0.854460in}{3.412197in}}%
\pgfpathlineto{\pgfqpoint{0.871042in}{3.355107in}}%
\pgfpathlineto{\pgfqpoint{0.896605in}{3.276329in}}%
\pgfpathlineto{\pgfqpoint{0.925446in}{3.197551in}}%
\pgfpathlineto{\pgfqpoint{0.958256in}{3.117482in}}%
\pgfpathlineto{\pgfqpoint{0.993428in}{3.039994in}}%
\pgfpathlineto{\pgfqpoint{1.032779in}{2.961215in}}%
\pgfpathlineto{\pgfqpoint{1.075828in}{2.882437in}}%
\pgfpathlineto{\pgfqpoint{1.106612in}{2.829918in}}%
\pgfpathlineto{\pgfqpoint{1.139134in}{2.777399in}}%
\pgfpathlineto{\pgfqpoint{1.173441in}{2.724880in}}%
\pgfpathlineto{\pgfqpoint{1.209576in}{2.672361in}}%
\pgfpathlineto{\pgfqpoint{1.247577in}{2.619842in}}%
\pgfpathlineto{\pgfqpoint{1.304241in}{2.545965in}}%
\pgfpathlineto{\pgfqpoint{1.351007in}{2.488545in}}%
\pgfpathlineto{\pgfqpoint{1.408036in}{2.422234in}}%
\pgfpathlineto{\pgfqpoint{1.467123in}{2.357248in}}%
\pgfpathlineto{\pgfqpoint{1.517287in}{2.304729in}}%
\pgfpathlineto{\pgfqpoint{1.581029in}{2.241184in}}%
\pgfpathlineto{\pgfqpoint{1.652554in}{2.173431in}}%
\pgfpathlineto{\pgfqpoint{1.719423in}{2.113325in}}%
\pgfpathlineto{\pgfqpoint{1.788620in}{2.053903in}}%
\pgfpathlineto{\pgfqpoint{1.867027in}{1.989615in}}%
\pgfpathlineto{\pgfqpoint{1.933980in}{1.937096in}}%
\pgfpathlineto{\pgfqpoint{2.030809in}{1.864611in}}%
\pgfpathlineto{\pgfqpoint{2.113152in}{1.805799in}}%
\pgfpathlineto{\pgfqpoint{2.203802in}{1.743845in}}%
\pgfpathlineto{\pgfqpoint{2.272999in}{1.698240in}}%
\pgfpathlineto{\pgfqpoint{2.376794in}{1.632556in}}%
\pgfpathlineto{\pgfqpoint{2.445991in}{1.590304in}}%
\pgfpathlineto{\pgfqpoint{2.549787in}{1.529181in}}%
\pgfpathlineto{\pgfqpoint{2.653582in}{1.470418in}}%
\pgfpathlineto{\pgfqpoint{2.760985in}{1.411906in}}%
\pgfpathlineto{\pgfqpoint{2.895771in}{1.341620in}}%
\pgfpathlineto{\pgfqpoint{2.999567in}{1.289545in}}%
\pgfpathlineto{\pgfqpoint{3.103362in}{1.239146in}}%
\pgfpathlineto{\pgfqpoint{3.207158in}{1.190293in}}%
\pgfpathlineto{\pgfqpoint{3.310953in}{1.142861in}}%
\pgfpathlineto{\pgfqpoint{3.449347in}{1.081745in}}%
\pgfpathlineto{\pgfqpoint{3.587741in}{1.022769in}}%
\pgfpathlineto{\pgfqpoint{3.726874in}{0.965495in}}%
\pgfpathlineto{\pgfqpoint{3.933726in}{0.883880in}}%
\pgfpathlineto{\pgfqpoint{4.135122in}{0.807939in}}%
\pgfpathlineto{\pgfqpoint{4.171699in}{0.794569in}}%
\pgfpathlineto{\pgfqpoint{4.171699in}{0.794569in}}%
\pgfusepath{stroke}%
\end{pgfscope}%
\begin{pgfscope}%
\pgfpathrectangle{\pgfqpoint{0.854460in}{0.571603in}}{\pgfqpoint{6.885100in}{5.225635in}}%
\pgfusepath{clip}%
\pgfsetbuttcap%
\pgfsetroundjoin%
\pgfsetlinewidth{1.505625pt}%
\definecolor{currentstroke}{rgb}{0.253935,0.265254,0.529983}%
\pgfsetstrokecolor{currentstroke}%
\pgfsetdash{}{0pt}%
\pgfpathmoveto{\pgfqpoint{4.536226in}{0.665967in}}%
\pgfpathlineto{\pgfqpoint{4.556499in}{0.659082in}}%
\pgfpathlineto{\pgfqpoint{4.582045in}{0.650382in}}%
\pgfpathlineto{\pgfqpoint{4.591098in}{0.647351in}}%
\pgfpathlineto{\pgfqpoint{4.625696in}{0.635788in}}%
\pgfpathlineto{\pgfqpoint{4.660295in}{0.624162in}}%
\pgfpathlineto{\pgfqpoint{4.660413in}{0.624122in}}%
\pgfpathlineto{\pgfqpoint{4.694893in}{0.612784in}}%
\pgfpathlineto{\pgfqpoint{4.729491in}{0.601344in}}%
\pgfpathlineto{\pgfqpoint{4.740037in}{0.597863in}}%
\pgfpathlineto{\pgfqpoint{4.764090in}{0.590057in}}%
\pgfpathlineto{\pgfqpoint{4.798688in}{0.578803in}}%
\pgfpathlineto{\pgfqpoint{4.820780in}{0.571603in}}%
\pgfusepath{stroke}%
\end{pgfscope}%
\begin{pgfscope}%
\pgfpathrectangle{\pgfqpoint{0.854460in}{0.571603in}}{\pgfqpoint{6.885100in}{5.225635in}}%
\pgfusepath{clip}%
\pgfsetbuttcap%
\pgfsetroundjoin%
\pgfsetlinewidth{1.505625pt}%
\definecolor{currentstroke}{rgb}{0.253935,0.265254,0.529983}%
\pgfsetstrokecolor{currentstroke}%
\pgfsetdash{}{0pt}%
\pgfpathmoveto{\pgfqpoint{7.739560in}{2.022039in}}%
\pgfpathlineto{\pgfqpoint{7.713406in}{2.042134in}}%
\pgfpathlineto{\pgfqpoint{7.704962in}{2.048660in}}%
\pgfpathlineto{\pgfqpoint{7.679471in}{2.068393in}}%
\pgfpathlineto{\pgfqpoint{7.670363in}{2.075487in}}%
\pgfpathlineto{\pgfqpoint{7.645795in}{2.094653in}}%
\pgfpathlineto{\pgfqpoint{7.635765in}{2.102527in}}%
\pgfpathlineto{\pgfqpoint{7.612380in}{2.120912in}}%
\pgfpathlineto{\pgfqpoint{7.601166in}{2.129787in}}%
\pgfpathlineto{\pgfqpoint{7.579227in}{2.147172in}}%
\pgfpathlineto{\pgfqpoint{7.566568in}{2.157273in}}%
\pgfpathlineto{\pgfqpoint{7.546339in}{2.173431in}}%
\pgfpathlineto{\pgfqpoint{7.531969in}{2.184992in}}%
\pgfpathlineto{\pgfqpoint{7.513716in}{2.199691in}}%
\pgfpathlineto{\pgfqpoint{7.497371in}{2.212951in}}%
\pgfpathlineto{\pgfqpoint{7.481361in}{2.225950in}}%
\pgfpathlineto{\pgfqpoint{7.462772in}{2.241159in}}%
\pgfpathlineto{\pgfqpoint{7.449276in}{2.252210in}}%
\pgfpathlineto{\pgfqpoint{7.428174in}{2.269624in}}%
\pgfpathlineto{\pgfqpoint{7.417461in}{2.278469in}}%
\pgfpathlineto{\pgfqpoint{7.393575in}{2.298353in}}%
\pgfpathlineto{\pgfqpoint{7.385920in}{2.304729in}}%
\pgfpathlineto{\pgfqpoint{7.358977in}{2.327356in}}%
\pgfpathlineto{\pgfqpoint{7.354653in}{2.330988in}}%
\pgfpathlineto{\pgfqpoint{7.324378in}{2.356641in}}%
\pgfpathlineto{\pgfqpoint{7.323663in}{2.357248in}}%
\pgfpathlineto{\pgfqpoint{7.292919in}{2.383507in}}%
\pgfpathlineto{\pgfqpoint{7.289780in}{2.386212in}}%
\pgfpathlineto{\pgfqpoint{7.289444in}{2.386502in}}%
\pgfusepath{stroke}%
\end{pgfscope}%
\begin{pgfscope}%
\pgfpathrectangle{\pgfqpoint{0.854460in}{0.571603in}}{\pgfqpoint{6.885100in}{5.225635in}}%
\pgfusepath{clip}%
\pgfsetbuttcap%
\pgfsetroundjoin%
\pgfsetlinewidth{1.505625pt}%
\definecolor{currentstroke}{rgb}{0.253935,0.265254,0.529983}%
\pgfsetstrokecolor{currentstroke}%
\pgfsetdash{}{0pt}%
\pgfpathmoveto{\pgfqpoint{7.000300in}{2.646572in}}%
\pgfpathlineto{\pgfqpoint{6.891822in}{2.751140in}}%
\pgfpathlineto{\pgfqpoint{6.805401in}{2.837804in}}%
\pgfpathlineto{\pgfqpoint{6.736204in}{2.909516in}}%
\pgfpathlineto{\pgfqpoint{6.663410in}{2.987475in}}%
\pgfpathlineto{\pgfqpoint{6.592455in}{3.066253in}}%
\pgfpathlineto{\pgfqpoint{6.524127in}{3.145032in}}%
\pgfpathlineto{\pgfqpoint{6.458435in}{3.223810in}}%
\pgfpathlineto{\pgfqpoint{6.390219in}{3.309182in}}%
\pgfpathlineto{\pgfqpoint{6.334870in}{3.381367in}}%
\pgfpathlineto{\pgfqpoint{6.277063in}{3.460145in}}%
\pgfpathlineto{\pgfqpoint{6.217227in}{3.545827in}}%
\pgfpathlineto{\pgfqpoint{6.169313in}{3.617702in}}%
\pgfpathlineto{\pgfqpoint{6.113431in}{3.706260in}}%
\pgfpathlineto{\pgfqpoint{6.072061in}{3.775259in}}%
\pgfpathlineto{\pgfqpoint{6.027313in}{3.854037in}}%
\pgfpathlineto{\pgfqpoint{5.985174in}{3.932816in}}%
\pgfpathlineto{\pgfqpoint{5.940439in}{4.022390in}}%
\pgfpathlineto{\pgfqpoint{5.905840in}{4.096501in}}%
\pgfpathlineto{\pgfqpoint{5.871242in}{4.175925in}}%
\pgfpathlineto{\pgfqpoint{5.842004in}{4.247930in}}%
\pgfpathlineto{\pgfqpoint{5.812405in}{4.326708in}}%
\pgfpathlineto{\pgfqpoint{5.785228in}{4.405486in}}%
\pgfpathlineto{\pgfqpoint{5.760448in}{4.484265in}}%
\pgfpathlineto{\pgfqpoint{5.737979in}{4.563043in}}%
\pgfpathlineto{\pgfqpoint{5.717746in}{4.641822in}}%
\pgfpathlineto{\pgfqpoint{5.698249in}{4.727962in}}%
\pgfpathlineto{\pgfqpoint{5.683878in}{4.799378in}}%
\pgfpathlineto{\pgfqpoint{5.670092in}{4.878157in}}%
\pgfpathlineto{\pgfqpoint{5.658267in}{4.956935in}}%
\pgfpathlineto{\pgfqpoint{5.648292in}{5.035714in}}%
\pgfpathlineto{\pgfqpoint{5.640124in}{5.114492in}}%
\pgfpathlineto{\pgfqpoint{5.631810in}{5.219530in}}%
\pgfpathlineto{\pgfqpoint{5.626084in}{5.324568in}}%
\pgfpathlineto{\pgfqpoint{5.622588in}{5.429606in}}%
\pgfpathlineto{\pgfqpoint{5.620713in}{5.560903in}}%
\pgfpathlineto{\pgfqpoint{5.620785in}{5.744720in}}%
\pgfpathlineto{\pgfqpoint{5.620933in}{5.797238in}}%
\pgfpathlineto{\pgfqpoint{5.620933in}{5.797238in}}%
\pgfusepath{stroke}%
\end{pgfscope}%
\begin{pgfscope}%
\pgfpathrectangle{\pgfqpoint{0.854460in}{0.571603in}}{\pgfqpoint{6.885100in}{5.225635in}}%
\pgfusepath{clip}%
\pgfsetbuttcap%
\pgfsetroundjoin%
\pgfsetlinewidth{1.505625pt}%
\definecolor{currentstroke}{rgb}{0.244972,0.287675,0.537260}%
\pgfsetstrokecolor{currentstroke}%
\pgfsetdash{}{0pt}%
\pgfpathmoveto{\pgfqpoint{1.439859in}{5.797238in}}%
\pgfpathlineto{\pgfqpoint{1.413089in}{5.770979in}}%
\pgfpathlineto{\pgfqpoint{1.408036in}{5.765955in}}%
\pgfpathlineto{\pgfqpoint{1.386970in}{5.744720in}}%
\pgfpathlineto{\pgfqpoint{1.373438in}{5.730900in}}%
\pgfpathlineto{\pgfqpoint{1.361424in}{5.718460in}}%
\pgfpathlineto{\pgfqpoint{1.338839in}{5.694765in}}%
\pgfpathlineto{\pgfqpoint{1.336428in}{5.692201in}}%
\pgfpathlineto{\pgfqpoint{1.312115in}{5.665941in}}%
\pgfpathlineto{\pgfqpoint{1.304241in}{5.657317in}}%
\pgfpathlineto{\pgfqpoint{1.288366in}{5.639682in}}%
\pgfpathlineto{\pgfqpoint{1.269642in}{5.618598in}}%
\pgfpathlineto{\pgfqpoint{1.265111in}{5.613422in}}%
\pgfpathlineto{\pgfqpoint{1.242474in}{5.587163in}}%
\pgfpathlineto{\pgfqpoint{1.235044in}{5.578415in}}%
\pgfpathlineto{\pgfqpoint{1.220386in}{5.560903in}}%
\pgfpathlineto{\pgfqpoint{1.200445in}{5.536742in}}%
\pgfpathlineto{\pgfqpoint{1.198739in}{5.534644in}}%
\pgfpathlineto{\pgfqpoint{1.177739in}{5.508384in}}%
\pgfpathlineto{\pgfqpoint{1.165847in}{5.493290in}}%
\pgfpathlineto{\pgfqpoint{1.157179in}{5.482125in}}%
\pgfpathlineto{\pgfqpoint{1.137118in}{5.455865in}}%
\pgfpathlineto{\pgfqpoint{1.131248in}{5.448051in}}%
\pgfpathlineto{\pgfqpoint{1.117600in}{5.429606in}}%
\pgfpathlineto{\pgfqpoint{1.098469in}{5.403346in}}%
\pgfpathlineto{\pgfqpoint{1.096650in}{5.400802in}}%
\pgfpathlineto{\pgfqpoint{1.079945in}{5.377087in}}%
\pgfpathlineto{\pgfqpoint{1.066951in}{5.358346in}}%
\pgfusepath{stroke}%
\end{pgfscope}%
\begin{pgfscope}%
\pgfpathrectangle{\pgfqpoint{0.854460in}{0.571603in}}{\pgfqpoint{6.885100in}{5.225635in}}%
\pgfusepath{clip}%
\pgfsetbuttcap%
\pgfsetroundjoin%
\pgfsetlinewidth{1.505625pt}%
\definecolor{currentstroke}{rgb}{0.244972,0.287675,0.537260}%
\pgfsetstrokecolor{currentstroke}%
\pgfsetdash{}{0pt}%
\pgfpathmoveto{\pgfqpoint{0.870850in}{5.020421in}}%
\pgfpathlineto{\pgfqpoint{0.865562in}{5.009454in}}%
\pgfpathlineto{\pgfqpoint{0.854460in}{4.985876in}}%
\pgfusepath{stroke}%
\end{pgfscope}%
\begin{pgfscope}%
\pgfpathrectangle{\pgfqpoint{0.854460in}{0.571603in}}{\pgfqpoint{6.885100in}{5.225635in}}%
\pgfusepath{clip}%
\pgfsetbuttcap%
\pgfsetroundjoin%
\pgfsetlinewidth{1.505625pt}%
\definecolor{currentstroke}{rgb}{0.244972,0.287675,0.537260}%
\pgfsetstrokecolor{currentstroke}%
\pgfsetdash{}{0pt}%
\pgfpathmoveto{\pgfqpoint{0.854460in}{3.044474in}}%
\pgfpathlineto{\pgfqpoint{0.856557in}{3.039994in}}%
\pgfpathlineto{\pgfqpoint{0.869215in}{3.013734in}}%
\pgfpathlineto{\pgfqpoint{0.882187in}{2.987475in}}%
\pgfpathlineto{\pgfqpoint{0.889059in}{2.973927in}}%
\pgfpathlineto{\pgfqpoint{0.895577in}{2.961215in}}%
\pgfpathlineto{\pgfqpoint{0.909395in}{2.934956in}}%
\pgfpathlineto{\pgfqpoint{0.923526in}{2.908696in}}%
\pgfpathlineto{\pgfqpoint{0.923657in}{2.908458in}}%
\pgfpathlineto{\pgfqpoint{0.938200in}{2.882437in}}%
\pgfpathlineto{\pgfqpoint{0.953187in}{2.856177in}}%
\pgfpathlineto{\pgfqpoint{0.958256in}{2.847510in}}%
\pgfpathlineto{\pgfqpoint{0.968649in}{2.829918in}}%
\pgfpathlineto{\pgfqpoint{0.984504in}{2.803659in}}%
\pgfpathlineto{\pgfqpoint{0.992854in}{2.790131in}}%
\pgfpathlineto{\pgfqpoint{1.000792in}{2.777399in}}%
\pgfpathlineto{\pgfqpoint{1.017523in}{2.751140in}}%
\pgfpathlineto{\pgfqpoint{1.027453in}{2.735873in}}%
\pgfpathlineto{\pgfqpoint{1.034673in}{2.724880in}}%
\pgfpathlineto{\pgfqpoint{1.052288in}{2.698621in}}%
\pgfpathlineto{\pgfqpoint{1.062051in}{2.684353in}}%
\pgfpathlineto{\pgfqpoint{1.070335in}{2.672361in}}%
\pgfpathlineto{\pgfqpoint{1.088840in}{2.646102in}}%
\pgfpathlineto{\pgfqpoint{1.096650in}{2.635238in}}%
\pgfpathlineto{\pgfqpoint{1.107818in}{2.619842in}}%
\pgfpathlineto{\pgfqpoint{1.127218in}{2.593583in}}%
\pgfpathlineto{\pgfqpoint{1.131248in}{2.588239in}}%
\pgfpathlineto{\pgfqpoint{1.147160in}{2.567323in}}%
\pgfpathlineto{\pgfqpoint{1.165847in}{2.543162in}}%
\pgfpathlineto{\pgfqpoint{1.167484in}{2.541064in}}%
\pgfpathlineto{\pgfqpoint{1.188395in}{2.514804in}}%
\pgfpathlineto{\pgfqpoint{1.200445in}{2.499923in}}%
\pgfpathlineto{\pgfqpoint{1.209736in}{2.488545in}}%
\pgfpathlineto{\pgfqpoint{1.231552in}{2.462285in}}%
\pgfpathlineto{\pgfqpoint{1.235044in}{2.458162in}}%
\pgfpathlineto{\pgfqpoint{1.253940in}{2.436026in}}%
\pgfpathlineto{\pgfqpoint{1.269642in}{2.417920in}}%
\pgfpathlineto{\pgfqpoint{1.276770in}{2.409766in}}%
\pgfpathlineto{\pgfqpoint{1.300120in}{2.383507in}}%
\pgfpathlineto{\pgfqpoint{1.304241in}{2.378956in}}%
\pgfpathlineto{\pgfqpoint{1.324043in}{2.357248in}}%
\pgfpathlineto{\pgfqpoint{1.338839in}{2.341279in}}%
\pgfpathlineto{\pgfqpoint{1.348445in}{2.330988in}}%
\pgfpathlineto{\pgfqpoint{1.373339in}{2.304729in}}%
\pgfpathlineto{\pgfqpoint{1.373438in}{2.304626in}}%
\pgfpathlineto{\pgfqpoint{1.398876in}{2.278469in}}%
\pgfpathlineto{\pgfqpoint{1.408036in}{2.269194in}}%
\pgfpathlineto{\pgfqpoint{1.424924in}{2.252210in}}%
\pgfpathlineto{\pgfqpoint{1.442635in}{2.234669in}}%
\pgfpathlineto{\pgfqpoint{1.451497in}{2.225950in}}%
\pgfpathlineto{\pgfqpoint{1.477233in}{2.201013in}}%
\pgfpathlineto{\pgfqpoint{1.478607in}{2.199691in}}%
\pgfpathlineto{\pgfqpoint{1.506350in}{2.173431in}}%
\pgfpathlineto{\pgfqpoint{1.511832in}{2.168322in}}%
\pgfpathlineto{\pgfqpoint{1.534667in}{2.147172in}}%
\pgfpathlineto{\pgfqpoint{1.546430in}{2.136440in}}%
\pgfpathlineto{\pgfqpoint{1.563551in}{2.120912in}}%
\pgfpathlineto{\pgfqpoint{1.581029in}{2.105298in}}%
\pgfpathlineto{\pgfqpoint{1.593013in}{2.094653in}}%
\pgfpathlineto{\pgfqpoint{1.615627in}{2.074865in}}%
\pgfpathlineto{\pgfqpoint{1.623065in}{2.068393in}}%
\pgfpathlineto{\pgfqpoint{1.650226in}{2.045111in}}%
\pgfpathlineto{\pgfqpoint{1.653718in}{2.042134in}}%
\pgfpathlineto{\pgfqpoint{1.684824in}{2.016007in}}%
\pgfpathlineto{\pgfqpoint{1.684983in}{2.015874in}}%
\pgfpathlineto{\pgfqpoint{1.716908in}{1.989615in}}%
\pgfpathlineto{\pgfqpoint{1.719423in}{1.987577in}}%
\pgfpathlineto{\pgfqpoint{1.749458in}{1.963355in}}%
\pgfpathlineto{\pgfqpoint{1.754021in}{1.959730in}}%
\pgfpathlineto{\pgfqpoint{1.782642in}{1.937096in}}%
\pgfpathlineto{\pgfqpoint{1.788620in}{1.932438in}}%
\pgfpathlineto{\pgfqpoint{1.816467in}{1.910836in}}%
\pgfpathlineto{\pgfqpoint{1.823218in}{1.905677in}}%
\pgfpathlineto{\pgfqpoint{1.850944in}{1.884577in}}%
\pgfpathlineto{\pgfqpoint{1.857817in}{1.879423in}}%
\pgfpathlineto{\pgfqpoint{1.886081in}{1.858318in}}%
\pgfpathlineto{\pgfqpoint{1.892415in}{1.853656in}}%
\pgfpathlineto{\pgfqpoint{1.921884in}{1.832058in}}%
\pgfpathlineto{\pgfqpoint{1.927014in}{1.828353in}}%
\pgfpathlineto{\pgfqpoint{1.958362in}{1.805799in}}%
\pgfpathlineto{\pgfqpoint{1.961612in}{1.803495in}}%
\pgfpathlineto{\pgfqpoint{1.995522in}{1.779539in}}%
\pgfpathlineto{\pgfqpoint{1.996211in}{1.779060in}}%
\pgfpathlineto{\pgfqpoint{2.030809in}{1.755074in}}%
\pgfpathlineto{\pgfqpoint{2.033407in}{1.753280in}}%
\pgfpathlineto{\pgfqpoint{2.065408in}{1.731497in}}%
\pgfpathlineto{\pgfqpoint{2.072006in}{1.727020in}}%
\pgfpathlineto{\pgfqpoint{2.100006in}{1.708298in}}%
\pgfpathlineto{\pgfqpoint{2.111313in}{1.700761in}}%
\pgfpathlineto{\pgfqpoint{2.134605in}{1.685460in}}%
\pgfpathlineto{\pgfqpoint{2.151334in}{1.674501in}}%
\pgfpathlineto{\pgfqpoint{2.169203in}{1.662966in}}%
\pgfpathlineto{\pgfqpoint{2.192072in}{1.648242in}}%
\pgfpathlineto{\pgfqpoint{2.203802in}{1.640799in}}%
\pgfpathlineto{\pgfqpoint{2.233531in}{1.621982in}}%
\pgfpathlineto{\pgfqpoint{2.238400in}{1.618945in}}%
\pgfpathlineto{\pgfqpoint{2.272999in}{1.597429in}}%
\pgfpathlineto{\pgfqpoint{2.275752in}{1.595723in}}%
\pgfpathlineto{\pgfqpoint{2.307597in}{1.576278in}}%
\pgfpathlineto{\pgfqpoint{2.318779in}{1.569463in}}%
\pgfpathlineto{\pgfqpoint{2.342196in}{1.555401in}}%
\pgfpathlineto{\pgfqpoint{2.362542in}{1.543204in}}%
\pgfpathlineto{\pgfqpoint{2.376794in}{1.534785in}}%
\pgfpathlineto{\pgfqpoint{2.407043in}{1.516944in}}%
\pgfpathlineto{\pgfqpoint{2.411393in}{1.514417in}}%
\pgfpathlineto{\pgfqpoint{2.445991in}{1.494372in}}%
\pgfpathlineto{\pgfqpoint{2.452371in}{1.490685in}}%
\pgfpathlineto{\pgfqpoint{2.480590in}{1.474617in}}%
\pgfpathlineto{\pgfqpoint{2.498507in}{1.464425in}}%
\pgfpathlineto{\pgfqpoint{2.515188in}{1.455077in}}%
\pgfpathlineto{\pgfqpoint{2.545388in}{1.438166in}}%
\pgfpathlineto{\pgfqpoint{2.549787in}{1.435739in}}%
\pgfpathlineto{\pgfqpoint{2.584385in}{1.416710in}}%
\pgfpathlineto{\pgfqpoint{2.593134in}{1.411906in}}%
\pgfpathlineto{\pgfqpoint{2.618983in}{1.397924in}}%
\pgfpathlineto{\pgfqpoint{2.641691in}{1.385647in}}%
\pgfpathlineto{\pgfqpoint{2.653582in}{1.379313in}}%
\pgfpathlineto{\pgfqpoint{2.688180in}{1.360901in}}%
\pgfpathlineto{\pgfqpoint{2.691034in}{1.359388in}}%
\pgfpathlineto{\pgfqpoint{2.722779in}{1.342802in}}%
\pgfpathlineto{\pgfqpoint{2.741293in}{1.333128in}}%
\pgfpathlineto{\pgfqpoint{2.757377in}{1.324849in}}%
\pgfpathlineto{\pgfqpoint{2.791976in}{1.307036in}}%
\pgfpathlineto{\pgfqpoint{2.792303in}{1.306869in}}%
\pgfpathlineto{\pgfqpoint{2.825837in}{1.289931in}}%
\pgfusepath{stroke}%
\end{pgfscope}%
\begin{pgfscope}%
\pgfpathrectangle{\pgfqpoint{0.854460in}{0.571603in}}{\pgfqpoint{6.885100in}{5.225635in}}%
\pgfusepath{clip}%
\pgfsetbuttcap%
\pgfsetroundjoin%
\pgfsetlinewidth{1.505625pt}%
\definecolor{currentstroke}{rgb}{0.244972,0.287675,0.537260}%
\pgfsetstrokecolor{currentstroke}%
\pgfsetdash{}{0pt}%
\pgfpathmoveto{\pgfqpoint{3.174941in}{1.122769in}}%
\pgfpathlineto{\pgfqpoint{3.207158in}{1.108252in}}%
\pgfpathlineto{\pgfqpoint{3.232532in}{1.096793in}}%
\pgfpathlineto{\pgfqpoint{3.241756in}{1.092690in}}%
\pgfpathlineto{\pgfqpoint{3.276355in}{1.077329in}}%
\pgfpathlineto{\pgfqpoint{3.291661in}{1.070533in}}%
\pgfpathlineto{\pgfqpoint{3.310953in}{1.062098in}}%
\pgfpathlineto{\pgfqpoint{3.345552in}{1.046955in}}%
\pgfpathlineto{\pgfqpoint{3.351694in}{1.044274in}}%
\pgfpathlineto{\pgfqpoint{3.380150in}{1.032040in}}%
\pgfpathlineto{\pgfqpoint{3.412679in}{1.018014in}}%
\pgfpathlineto{\pgfqpoint{3.414749in}{1.017135in}}%
\pgfpathlineto{\pgfqpoint{3.449347in}{1.002498in}}%
\pgfpathlineto{\pgfqpoint{3.474676in}{0.991755in}}%
\pgfpathlineto{\pgfqpoint{3.483946in}{0.987883in}}%
\pgfpathlineto{\pgfqpoint{3.518544in}{0.973457in}}%
\pgfpathlineto{\pgfqpoint{3.537619in}{0.965495in}}%
\pgfpathlineto{\pgfqpoint{3.553143in}{0.959116in}}%
\pgfpathlineto{\pgfqpoint{3.587741in}{0.944898in}}%
\pgfpathlineto{\pgfqpoint{3.601524in}{0.939236in}}%
\pgfpathlineto{\pgfqpoint{3.622340in}{0.930818in}}%
\pgfpathlineto{\pgfqpoint{3.656938in}{0.916805in}}%
\pgfpathlineto{\pgfqpoint{3.666407in}{0.912976in}}%
\pgfpathlineto{\pgfqpoint{3.691537in}{0.902973in}}%
\pgfpathlineto{\pgfqpoint{3.726135in}{0.889164in}}%
\pgfpathlineto{\pgfqpoint{3.732282in}{0.886717in}}%
\pgfpathlineto{\pgfqpoint{3.760734in}{0.875566in}}%
\pgfpathlineto{\pgfqpoint{3.795332in}{0.861959in}}%
\pgfpathlineto{\pgfqpoint{3.799161in}{0.860458in}}%
\pgfpathlineto{\pgfqpoint{3.829931in}{0.848582in}}%
\pgfpathlineto{\pgfqpoint{3.864529in}{0.835174in}}%
\pgfpathlineto{\pgfqpoint{3.867056in}{0.834198in}}%
\pgfpathlineto{\pgfqpoint{3.899128in}{0.822007in}}%
\pgfpathlineto{\pgfqpoint{3.933726in}{0.808796in}}%
\pgfpathlineto{\pgfqpoint{3.935979in}{0.807939in}}%
\pgfpathlineto{\pgfqpoint{3.968325in}{0.795826in}}%
\pgfpathlineto{\pgfqpoint{4.002923in}{0.782809in}}%
\pgfpathlineto{\pgfqpoint{4.005938in}{0.781679in}}%
\pgfpathlineto{\pgfqpoint{4.037522in}{0.770024in}}%
\pgfpathlineto{\pgfqpoint{4.072120in}{0.757201in}}%
\pgfpathlineto{\pgfqpoint{4.076941in}{0.755420in}}%
\pgfpathlineto{\pgfqpoint{4.106719in}{0.744590in}}%
\pgfpathlineto{\pgfqpoint{4.141317in}{0.731958in}}%
\pgfpathlineto{\pgfqpoint{4.148996in}{0.729160in}}%
\pgfpathlineto{\pgfqpoint{4.175916in}{0.719509in}}%
\pgfpathlineto{\pgfqpoint{4.210514in}{0.707066in}}%
\pgfpathlineto{\pgfqpoint{4.222109in}{0.702901in}}%
\pgfpathlineto{\pgfqpoint{4.245113in}{0.694769in}}%
\pgfpathlineto{\pgfqpoint{4.279711in}{0.682514in}}%
\pgfpathlineto{\pgfqpoint{4.296283in}{0.676641in}}%
\pgfpathlineto{\pgfqpoint{4.314310in}{0.670357in}}%
\pgfpathlineto{\pgfqpoint{4.348908in}{0.658287in}}%
\pgfpathlineto{\pgfqpoint{4.371523in}{0.650382in}}%
\pgfpathlineto{\pgfqpoint{4.383507in}{0.646261in}}%
\pgfpathlineto{\pgfqpoint{4.418105in}{0.634374in}}%
\pgfpathlineto{\pgfqpoint{4.447828in}{0.624122in}}%
\pgfpathlineto{\pgfqpoint{4.452704in}{0.622468in}}%
\pgfpathlineto{\pgfqpoint{4.487302in}{0.610763in}}%
\pgfpathlineto{\pgfqpoint{4.521901in}{0.598997in}}%
\pgfpathlineto{\pgfqpoint{4.525246in}{0.597863in}}%
\pgfpathlineto{\pgfqpoint{4.556499in}{0.587443in}}%
\pgfpathlineto{\pgfqpoint{4.591098in}{0.575858in}}%
\pgfpathlineto{\pgfqpoint{4.603813in}{0.571603in}}%
\pgfusepath{stroke}%
\end{pgfscope}%
\begin{pgfscope}%
\pgfpathrectangle{\pgfqpoint{0.854460in}{0.571603in}}{\pgfqpoint{6.885100in}{5.225635in}}%
\pgfusepath{clip}%
\pgfsetbuttcap%
\pgfsetroundjoin%
\pgfsetlinewidth{1.505625pt}%
\definecolor{currentstroke}{rgb}{0.244972,0.287675,0.537260}%
\pgfsetstrokecolor{currentstroke}%
\pgfsetdash{}{0pt}%
\pgfpathmoveto{\pgfqpoint{7.739560in}{2.162628in}}%
\pgfpathlineto{\pgfqpoint{7.725923in}{2.173431in}}%
\pgfpathlineto{\pgfqpoint{7.704962in}{2.190159in}}%
\pgfpathlineto{\pgfqpoint{7.693030in}{2.199691in}}%
\pgfpathlineto{\pgfqpoint{7.683000in}{2.207764in}}%
\pgfusepath{stroke}%
\end{pgfscope}%
\begin{pgfscope}%
\pgfpathrectangle{\pgfqpoint{0.854460in}{0.571603in}}{\pgfqpoint{6.885100in}{5.225635in}}%
\pgfusepath{clip}%
\pgfsetbuttcap%
\pgfsetroundjoin%
\pgfsetlinewidth{1.505625pt}%
\definecolor{currentstroke}{rgb}{0.244972,0.287675,0.537260}%
\pgfsetstrokecolor{currentstroke}%
\pgfsetdash{}{0pt}%
\pgfpathmoveto{\pgfqpoint{7.385101in}{2.457385in}}%
\pgfpathlineto{\pgfqpoint{7.320212in}{2.514804in}}%
\pgfpathlineto{\pgfqpoint{7.220583in}{2.605464in}}%
\pgfpathlineto{\pgfqpoint{7.149257in}{2.672361in}}%
\pgfpathlineto{\pgfqpoint{7.047590in}{2.770991in}}%
\pgfpathlineto{\pgfqpoint{6.978393in}{2.840532in}}%
\pgfpathlineto{\pgfqpoint{6.909196in}{2.912243in}}%
\pgfpathlineto{\pgfqpoint{6.838971in}{2.987475in}}%
\pgfpathlineto{\pgfqpoint{6.768058in}{3.066253in}}%
\pgfpathlineto{\pgfqpoint{6.699841in}{3.145032in}}%
\pgfpathlineto{\pgfqpoint{6.632409in}{3.226166in}}%
\pgfpathlineto{\pgfqpoint{6.563212in}{3.313252in}}%
\pgfpathlineto{\pgfqpoint{6.511305in}{3.381367in}}%
\pgfpathlineto{\pgfqpoint{6.453888in}{3.460145in}}%
\pgfpathlineto{\pgfqpoint{6.399136in}{3.538924in}}%
\pgfpathlineto{\pgfqpoint{6.347088in}{3.617702in}}%
\pgfpathlineto{\pgfqpoint{6.297720in}{3.696481in}}%
\pgfpathlineto{\pgfqpoint{6.251071in}{3.775259in}}%
\pgfpathlineto{\pgfqpoint{6.207020in}{3.854037in}}%
\pgfpathlineto{\pgfqpoint{6.165649in}{3.932816in}}%
\pgfpathlineto{\pgfqpoint{6.126923in}{4.011594in}}%
\pgfpathlineto{\pgfqpoint{6.090814in}{4.090373in}}%
\pgfpathlineto{\pgfqpoint{6.057296in}{4.169151in}}%
\pgfpathlineto{\pgfqpoint{6.026343in}{4.247930in}}%
\pgfpathlineto{\pgfqpoint{5.997936in}{4.326708in}}%
\pgfpathlineto{\pgfqpoint{5.972055in}{4.405486in}}%
\pgfpathlineto{\pgfqpoint{5.948593in}{4.484265in}}%
\pgfpathlineto{\pgfqpoint{5.927562in}{4.563043in}}%
\pgfpathlineto{\pgfqpoint{5.908945in}{4.641822in}}%
\pgfpathlineto{\pgfqpoint{5.892589in}{4.720600in}}%
\pgfpathlineto{\pgfqpoint{5.878538in}{4.799378in}}%
\pgfpathlineto{\pgfqpoint{5.866679in}{4.878157in}}%
\pgfpathlineto{\pgfqpoint{5.856907in}{4.956935in}}%
\pgfpathlineto{\pgfqpoint{5.849205in}{5.035714in}}%
\pgfpathlineto{\pgfqpoint{5.843467in}{5.114492in}}%
\pgfpathlineto{\pgfqpoint{5.839583in}{5.193271in}}%
\pgfpathlineto{\pgfqpoint{5.837437in}{5.272049in}}%
\pgfpathlineto{\pgfqpoint{5.837059in}{5.377087in}}%
\pgfpathlineto{\pgfqpoint{5.839213in}{5.482125in}}%
\pgfpathlineto{\pgfqpoint{5.843522in}{5.587163in}}%
\pgfpathlineto{\pgfqpoint{5.851262in}{5.718460in}}%
\pgfpathlineto{\pgfqpoint{5.856766in}{5.797238in}}%
\pgfpathlineto{\pgfqpoint{5.856766in}{5.797238in}}%
\pgfusepath{stroke}%
\end{pgfscope}%
\begin{pgfscope}%
\pgfpathrectangle{\pgfqpoint{0.854460in}{0.571603in}}{\pgfqpoint{6.885100in}{5.225635in}}%
\pgfusepath{clip}%
\pgfsetbuttcap%
\pgfsetroundjoin%
\pgfsetlinewidth{1.505625pt}%
\definecolor{currentstroke}{rgb}{0.233603,0.313828,0.543914}%
\pgfsetstrokecolor{currentstroke}%
\pgfsetdash{}{0pt}%
\pgfpathmoveto{\pgfqpoint{1.263661in}{5.797238in}}%
\pgfpathlineto{\pgfqpoint{1.238346in}{5.770979in}}%
\pgfpathlineto{\pgfqpoint{1.235044in}{5.767504in}}%
\pgfpathlineto{\pgfqpoint{1.225584in}{5.757415in}}%
\pgfusepath{stroke}%
\end{pgfscope}%
\begin{pgfscope}%
\pgfpathrectangle{\pgfqpoint{0.854460in}{0.571603in}}{\pgfqpoint{6.885100in}{5.225635in}}%
\pgfusepath{clip}%
\pgfsetbuttcap%
\pgfsetroundjoin%
\pgfsetlinewidth{1.505625pt}%
\definecolor{currentstroke}{rgb}{0.233603,0.313828,0.543914}%
\pgfsetstrokecolor{currentstroke}%
\pgfsetdash{}{0pt}%
\pgfpathmoveto{\pgfqpoint{0.977724in}{5.456699in}}%
\pgfpathlineto{\pgfqpoint{0.977124in}{5.455865in}}%
\pgfpathlineto{\pgfqpoint{0.958502in}{5.429606in}}%
\pgfpathlineto{\pgfqpoint{0.958256in}{5.429252in}}%
\pgfpathlineto{\pgfqpoint{0.940491in}{5.403346in}}%
\pgfpathlineto{\pgfqpoint{0.923657in}{5.378406in}}%
\pgfpathlineto{\pgfqpoint{0.922780in}{5.377087in}}%
\pgfpathlineto{\pgfqpoint{0.905654in}{5.350827in}}%
\pgfpathlineto{\pgfqpoint{0.889059in}{5.324955in}}%
\pgfpathlineto{\pgfqpoint{0.888815in}{5.324568in}}%
\pgfpathlineto{\pgfqpoint{0.872557in}{5.298308in}}%
\pgfpathlineto{\pgfqpoint{0.856584in}{5.272049in}}%
\pgfpathlineto{\pgfqpoint{0.854460in}{5.268485in}}%
\pgfusepath{stroke}%
\end{pgfscope}%
\begin{pgfscope}%
\pgfpathrectangle{\pgfqpoint{0.854460in}{0.571603in}}{\pgfqpoint{6.885100in}{5.225635in}}%
\pgfusepath{clip}%
\pgfsetbuttcap%
\pgfsetroundjoin%
\pgfsetlinewidth{1.505625pt}%
\definecolor{currentstroke}{rgb}{0.233603,0.313828,0.543914}%
\pgfsetstrokecolor{currentstroke}%
\pgfsetdash{}{0pt}%
\pgfpathmoveto{\pgfqpoint{0.854460in}{2.803958in}}%
\pgfpathlineto{\pgfqpoint{0.889059in}{2.748354in}}%
\pgfpathlineto{\pgfqpoint{0.939474in}{2.672361in}}%
\pgfpathlineto{\pgfqpoint{0.992854in}{2.597282in}}%
\pgfpathlineto{\pgfqpoint{1.035313in}{2.541064in}}%
\pgfpathlineto{\pgfqpoint{1.096650in}{2.464417in}}%
\pgfpathlineto{\pgfqpoint{1.142972in}{2.409766in}}%
\pgfpathlineto{\pgfqpoint{1.200445in}{2.345239in}}%
\pgfpathlineto{\pgfqpoint{1.263147in}{2.278469in}}%
\pgfpathlineto{\pgfqpoint{1.314936in}{2.225950in}}%
\pgfpathlineto{\pgfqpoint{1.373438in}{2.169110in}}%
\pgfpathlineto{\pgfqpoint{1.454096in}{2.094653in}}%
\pgfpathlineto{\pgfqpoint{1.513758in}{2.042134in}}%
\pgfpathlineto{\pgfqpoint{1.581029in}{1.985370in}}%
\pgfpathlineto{\pgfqpoint{1.650226in}{1.929357in}}%
\pgfpathlineto{\pgfqpoint{1.719423in}{1.875526in}}%
\pgfpathlineto{\pgfqpoint{1.788620in}{1.823695in}}%
\pgfpathlineto{\pgfqpoint{1.857817in}{1.773693in}}%
\pgfpathlineto{\pgfqpoint{1.927014in}{1.725365in}}%
\pgfpathlineto{\pgfqpoint{2.030809in}{1.655875in}}%
\pgfpathlineto{\pgfqpoint{2.124612in}{1.595723in}}%
\pgfpathlineto{\pgfqpoint{2.238400in}{1.525959in}}%
\pgfpathlineto{\pgfqpoint{2.343060in}{1.464425in}}%
\pgfpathlineto{\pgfqpoint{2.480590in}{1.387250in}}%
\pgfpathlineto{\pgfqpoint{2.584385in}{1.331435in}}%
\pgfpathlineto{\pgfqpoint{2.722779in}{1.259992in}}%
\pgfpathlineto{\pgfqpoint{2.839992in}{1.201831in}}%
\pgfpathlineto{\pgfqpoint{2.999567in}{1.125888in}}%
\pgfpathlineto{\pgfqpoint{3.120581in}{1.070533in}}%
\pgfpathlineto{\pgfqpoint{3.276355in}{1.001921in}}%
\pgfpathlineto{\pgfqpoint{3.414749in}{0.943125in}}%
\pgfpathlineto{\pgfqpoint{3.477551in}{0.917103in}}%
\pgfpathlineto{\pgfqpoint{3.477551in}{0.917103in}}%
\pgfusepath{stroke}%
\end{pgfscope}%
\begin{pgfscope}%
\pgfpathrectangle{\pgfqpoint{0.854460in}{0.571603in}}{\pgfqpoint{6.885100in}{5.225635in}}%
\pgfusepath{clip}%
\pgfsetbuttcap%
\pgfsetroundjoin%
\pgfsetlinewidth{1.505625pt}%
\definecolor{currentstroke}{rgb}{0.233603,0.313828,0.543914}%
\pgfsetstrokecolor{currentstroke}%
\pgfsetdash{}{0pt}%
\pgfpathmoveto{\pgfqpoint{3.837246in}{0.775096in}}%
\pgfpathlineto{\pgfqpoint{3.864529in}{0.764789in}}%
\pgfpathlineto{\pgfqpoint{3.889272in}{0.755420in}}%
\pgfpathlineto{\pgfqpoint{3.899128in}{0.751745in}}%
\pgfpathlineto{\pgfqpoint{3.933726in}{0.738865in}}%
\pgfpathlineto{\pgfqpoint{3.959722in}{0.729160in}}%
\pgfpathlineto{\pgfqpoint{3.968325in}{0.725998in}}%
\pgfpathlineto{\pgfqpoint{4.002923in}{0.713305in}}%
\pgfpathlineto{\pgfqpoint{4.031188in}{0.702901in}}%
\pgfpathlineto{\pgfqpoint{4.037522in}{0.700605in}}%
\pgfpathlineto{\pgfqpoint{4.072120in}{0.688098in}}%
\pgfpathlineto{\pgfqpoint{4.103676in}{0.676641in}}%
\pgfpathlineto{\pgfqpoint{4.106719in}{0.675554in}}%
\pgfpathlineto{\pgfqpoint{4.141317in}{0.663229in}}%
\pgfpathlineto{\pgfqpoint{4.175916in}{0.650842in}}%
\pgfpathlineto{\pgfqpoint{4.177207in}{0.650382in}}%
\pgfpathlineto{\pgfqpoint{4.210514in}{0.638688in}}%
\pgfpathlineto{\pgfqpoint{4.245113in}{0.626483in}}%
\pgfpathlineto{\pgfqpoint{4.251820in}{0.624122in}}%
\pgfpathlineto{\pgfqpoint{4.279711in}{0.614461in}}%
\pgfpathlineto{\pgfqpoint{4.314310in}{0.602437in}}%
\pgfpathlineto{\pgfqpoint{4.327478in}{0.597863in}}%
\pgfpathlineto{\pgfqpoint{4.348908in}{0.590537in}}%
\pgfpathlineto{\pgfqpoint{4.383507in}{0.578692in}}%
\pgfpathlineto{\pgfqpoint{4.404180in}{0.571603in}}%
\pgfusepath{stroke}%
\end{pgfscope}%
\begin{pgfscope}%
\pgfpathrectangle{\pgfqpoint{0.854460in}{0.571603in}}{\pgfqpoint{6.885100in}{5.225635in}}%
\pgfusepath{clip}%
\pgfsetbuttcap%
\pgfsetroundjoin%
\pgfsetlinewidth{1.505625pt}%
\definecolor{currentstroke}{rgb}{0.233603,0.313828,0.543914}%
\pgfsetstrokecolor{currentstroke}%
\pgfsetdash{}{0pt}%
\pgfpathmoveto{\pgfqpoint{7.739560in}{2.300148in}}%
\pgfpathlineto{\pgfqpoint{7.733984in}{2.304729in}}%
\pgfpathlineto{\pgfqpoint{7.704962in}{2.328780in}}%
\pgfpathlineto{\pgfqpoint{7.702298in}{2.330988in}}%
\pgfpathlineto{\pgfqpoint{7.670901in}{2.357248in}}%
\pgfpathlineto{\pgfqpoint{7.670363in}{2.357701in}}%
\pgfpathlineto{\pgfqpoint{7.639773in}{2.383507in}}%
\pgfpathlineto{\pgfqpoint{7.635765in}{2.386920in}}%
\pgfpathlineto{\pgfqpoint{7.608939in}{2.409766in}}%
\pgfpathlineto{\pgfqpoint{7.601166in}{2.416451in}}%
\pgfpathlineto{\pgfqpoint{7.578402in}{2.436026in}}%
\pgfpathlineto{\pgfqpoint{7.566568in}{2.446305in}}%
\pgfpathlineto{\pgfqpoint{7.548163in}{2.462285in}}%
\pgfpathlineto{\pgfqpoint{7.531969in}{2.476492in}}%
\pgfpathlineto{\pgfqpoint{7.518224in}{2.488545in}}%
\pgfpathlineto{\pgfqpoint{7.497371in}{2.507024in}}%
\pgfpathlineto{\pgfqpoint{7.488587in}{2.514804in}}%
\pgfpathlineto{\pgfqpoint{7.462772in}{2.537915in}}%
\pgfpathlineto{\pgfqpoint{7.459253in}{2.541064in}}%
\pgfpathlineto{\pgfqpoint{7.430205in}{2.567323in}}%
\pgfpathlineto{\pgfqpoint{7.428174in}{2.569179in}}%
\pgfpathlineto{\pgfqpoint{7.401430in}{2.593583in}}%
\pgfpathlineto{\pgfqpoint{7.393575in}{2.600833in}}%
\pgfpathlineto{\pgfqpoint{7.372961in}{2.619842in}}%
\pgfpathlineto{\pgfqpoint{7.358977in}{2.632890in}}%
\pgfpathlineto{\pgfqpoint{7.344801in}{2.646102in}}%
\pgfpathlineto{\pgfqpoint{7.324378in}{2.665364in}}%
\pgfpathlineto{\pgfqpoint{7.316951in}{2.672361in}}%
\pgfpathlineto{\pgfqpoint{7.289780in}{2.698273in}}%
\pgfpathlineto{\pgfqpoint{7.289415in}{2.698621in}}%
\pgfpathlineto{\pgfqpoint{7.262128in}{2.724880in}}%
\pgfpathlineto{\pgfqpoint{7.255181in}{2.731651in}}%
\pgfpathlineto{\pgfqpoint{7.235153in}{2.751140in}}%
\pgfpathlineto{\pgfqpoint{7.220583in}{2.765503in}}%
\pgfpathlineto{\pgfqpoint{7.208495in}{2.777399in}}%
\pgfpathlineto{\pgfqpoint{7.185984in}{2.799848in}}%
\pgfpathlineto{\pgfqpoint{7.182156in}{2.803659in}}%
\pgfpathlineto{\pgfqpoint{7.156095in}{2.829918in}}%
\pgfpathlineto{\pgfqpoint{7.151386in}{2.834723in}}%
\pgfpathlineto{\pgfqpoint{7.130319in}{2.856177in}}%
\pgfpathlineto{\pgfqpoint{7.116787in}{2.870154in}}%
\pgfpathlineto{\pgfqpoint{7.104868in}{2.882437in}}%
\pgfpathlineto{\pgfqpoint{7.082189in}{2.906146in}}%
\pgfpathlineto{\pgfqpoint{7.079743in}{2.908696in}}%
\pgfpathlineto{\pgfqpoint{7.054880in}{2.934956in}}%
\pgfpathlineto{\pgfqpoint{7.047590in}{2.942765in}}%
\pgfpathlineto{\pgfqpoint{7.030323in}{2.961215in}}%
\pgfpathlineto{\pgfqpoint{7.012992in}{2.980016in}}%
\pgfpathlineto{\pgfqpoint{7.006098in}{2.987475in}}%
\pgfpathlineto{\pgfqpoint{6.982172in}{3.013734in}}%
\pgfpathlineto{\pgfqpoint{6.978393in}{3.017940in}}%
\pgfpathlineto{\pgfqpoint{6.967581in}{3.029936in}}%
\pgfusepath{stroke}%
\end{pgfscope}%
\begin{pgfscope}%
\pgfpathrectangle{\pgfqpoint{0.854460in}{0.571603in}}{\pgfqpoint{6.885100in}{5.225635in}}%
\pgfusepath{clip}%
\pgfsetbuttcap%
\pgfsetroundjoin%
\pgfsetlinewidth{1.505625pt}%
\definecolor{currentstroke}{rgb}{0.233603,0.313828,0.543914}%
\pgfsetstrokecolor{currentstroke}%
\pgfsetdash{}{0pt}%
\pgfpathmoveto{\pgfqpoint{6.718173in}{3.329579in}}%
\pgfpathlineto{\pgfqpoint{6.701606in}{3.351417in}}%
\pgfpathlineto{\pgfqpoint{6.698793in}{3.355107in}}%
\pgfpathlineto{\pgfqpoint{6.679111in}{3.381367in}}%
\pgfpathlineto{\pgfqpoint{6.667007in}{3.397829in}}%
\pgfpathlineto{\pgfqpoint{6.659770in}{3.407626in}}%
\pgfpathlineto{\pgfqpoint{6.640722in}{3.433886in}}%
\pgfpathlineto{\pgfqpoint{6.632409in}{3.445565in}}%
\pgfpathlineto{\pgfqpoint{6.621978in}{3.460145in}}%
\pgfpathlineto{\pgfqpoint{6.603557in}{3.486405in}}%
\pgfpathlineto{\pgfqpoint{6.597810in}{3.494753in}}%
\pgfpathlineto{\pgfqpoint{6.585415in}{3.512664in}}%
\pgfpathlineto{\pgfqpoint{6.567613in}{3.538924in}}%
\pgfpathlineto{\pgfqpoint{6.563212in}{3.545542in}}%
\pgfpathlineto{\pgfqpoint{6.550077in}{3.565183in}}%
\pgfpathlineto{\pgfqpoint{6.532887in}{3.591443in}}%
\pgfpathlineto{\pgfqpoint{6.528613in}{3.598102in}}%
\pgfpathlineto{\pgfqpoint{6.515961in}{3.617702in}}%
\pgfpathlineto{\pgfqpoint{6.499377in}{3.643962in}}%
\pgfpathlineto{\pgfqpoint{6.494015in}{3.652630in}}%
\pgfpathlineto{\pgfqpoint{6.483067in}{3.670221in}}%
\pgfpathlineto{\pgfqpoint{6.467082in}{3.696481in}}%
\pgfpathlineto{\pgfqpoint{6.459416in}{3.709354in}}%
\pgfpathlineto{\pgfqpoint{6.451394in}{3.722740in}}%
\pgfpathlineto{\pgfqpoint{6.436003in}{3.749000in}}%
\pgfpathlineto{\pgfqpoint{6.424818in}{3.768535in}}%
\pgfpathlineto{\pgfqpoint{6.420942in}{3.775259in}}%
\pgfpathlineto{\pgfqpoint{6.406139in}{3.801519in}}%
\pgfpathlineto{\pgfqpoint{6.391698in}{3.827778in}}%
\pgfpathlineto{\pgfqpoint{6.390219in}{3.830529in}}%
\pgfpathlineto{\pgfqpoint{6.377492in}{3.854037in}}%
\pgfpathlineto{\pgfqpoint{6.363629in}{3.880297in}}%
\pgfpathlineto{\pgfqpoint{6.355621in}{3.895847in}}%
\pgfpathlineto{\pgfqpoint{6.350065in}{3.906556in}}%
\pgfpathlineto{\pgfqpoint{6.336775in}{3.932816in}}%
\pgfpathlineto{\pgfqpoint{6.323833in}{3.959075in}}%
\pgfpathlineto{\pgfqpoint{6.321022in}{3.964925in}}%
\pgfpathlineto{\pgfqpoint{6.311139in}{3.985335in}}%
\pgfpathlineto{\pgfqpoint{6.298761in}{4.011594in}}%
\pgfpathlineto{\pgfqpoint{6.286723in}{4.037854in}}%
\pgfpathlineto{\pgfqpoint{6.286424in}{4.038525in}}%
\pgfpathlineto{\pgfqpoint{6.274907in}{4.064113in}}%
\pgfpathlineto{\pgfqpoint{6.263425in}{4.090373in}}%
\pgfpathlineto{\pgfqpoint{6.252274in}{4.116632in}}%
\pgfpathlineto{\pgfqpoint{6.251825in}{4.117719in}}%
\pgfpathlineto{\pgfqpoint{6.241347in}{4.142892in}}%
\pgfpathlineto{\pgfqpoint{6.230743in}{4.169151in}}%
\pgfpathlineto{\pgfqpoint{6.220464in}{4.195411in}}%
\pgfpathlineto{\pgfqpoint{6.217227in}{4.203938in}}%
\pgfpathlineto{\pgfqpoint{6.210437in}{4.221670in}}%
\pgfpathlineto{\pgfqpoint{6.200697in}{4.247930in}}%
\pgfpathlineto{\pgfqpoint{6.191274in}{4.274189in}}%
\pgfpathlineto{\pgfqpoint{6.182628in}{4.299110in}}%
\pgfpathlineto{\pgfqpoint{6.182160in}{4.300449in}}%
\pgfpathlineto{\pgfqpoint{6.173270in}{4.326708in}}%
\pgfpathlineto{\pgfqpoint{6.164689in}{4.352967in}}%
\pgfpathlineto{\pgfqpoint{6.156415in}{4.379227in}}%
\pgfpathlineto{\pgfqpoint{6.148443in}{4.405486in}}%
\pgfpathlineto{\pgfqpoint{6.148030in}{4.406901in}}%
\pgfpathlineto{\pgfqpoint{6.140696in}{4.431746in}}%
\pgfpathlineto{\pgfqpoint{6.133244in}{4.458005in}}%
\pgfpathlineto{\pgfqpoint{6.126087in}{4.484265in}}%
\pgfpathlineto{\pgfqpoint{6.119223in}{4.510524in}}%
\pgfpathlineto{\pgfqpoint{6.113431in}{4.533658in}}%
\pgfpathlineto{\pgfqpoint{6.112640in}{4.536784in}}%
\pgfpathlineto{\pgfqpoint{6.106286in}{4.563043in}}%
\pgfpathlineto{\pgfqpoint{6.100218in}{4.589303in}}%
\pgfpathlineto{\pgfqpoint{6.094433in}{4.615562in}}%
\pgfpathlineto{\pgfqpoint{6.088927in}{4.641822in}}%
\pgfpathlineto{\pgfqpoint{6.083697in}{4.668081in}}%
\pgfpathlineto{\pgfqpoint{6.078833in}{4.693856in}}%
\pgfpathlineto{\pgfqpoint{6.078740in}{4.694341in}}%
\pgfpathlineto{\pgfqpoint{6.074005in}{4.720600in}}%
\pgfpathlineto{\pgfqpoint{6.069539in}{4.746860in}}%
\pgfpathlineto{\pgfqpoint{6.065339in}{4.773119in}}%
\pgfpathlineto{\pgfqpoint{6.061404in}{4.799378in}}%
\pgfpathlineto{\pgfqpoint{6.057728in}{4.825638in}}%
\pgfpathlineto{\pgfqpoint{6.054310in}{4.851897in}}%
\pgfpathlineto{\pgfqpoint{6.051146in}{4.878157in}}%
\pgfpathlineto{\pgfqpoint{6.048233in}{4.904416in}}%
\pgfpathlineto{\pgfqpoint{6.045568in}{4.930676in}}%
\pgfpathlineto{\pgfqpoint{6.044234in}{4.945160in}}%
\pgfpathlineto{\pgfqpoint{6.043136in}{4.956935in}}%
\pgfpathlineto{\pgfqpoint{6.040932in}{4.983195in}}%
\pgfpathlineto{\pgfqpoint{6.038970in}{5.009454in}}%
\pgfpathlineto{\pgfqpoint{6.037244in}{5.035714in}}%
\pgfpathlineto{\pgfqpoint{6.035753in}{5.061973in}}%
\pgfpathlineto{\pgfqpoint{6.034491in}{5.088233in}}%
\pgfpathlineto{\pgfqpoint{6.033457in}{5.114492in}}%
\pgfpathlineto{\pgfqpoint{6.032647in}{5.140752in}}%
\pgfpathlineto{\pgfqpoint{6.032056in}{5.167011in}}%
\pgfpathlineto{\pgfqpoint{6.031682in}{5.193271in}}%
\pgfpathlineto{\pgfqpoint{6.031520in}{5.219530in}}%
\pgfpathlineto{\pgfqpoint{6.031567in}{5.245790in}}%
\pgfpathlineto{\pgfqpoint{6.031819in}{5.272049in}}%
\pgfpathlineto{\pgfqpoint{6.032272in}{5.298308in}}%
\pgfpathlineto{\pgfqpoint{6.032922in}{5.324568in}}%
\pgfpathlineto{\pgfqpoint{6.033765in}{5.350827in}}%
\pgfpathlineto{\pgfqpoint{6.034797in}{5.377087in}}%
\pgfpathlineto{\pgfqpoint{6.036012in}{5.403346in}}%
\pgfpathlineto{\pgfqpoint{6.037407in}{5.429606in}}%
\pgfpathlineto{\pgfqpoint{6.038977in}{5.455865in}}%
\pgfpathlineto{\pgfqpoint{6.040717in}{5.482125in}}%
\pgfpathlineto{\pgfqpoint{6.042622in}{5.508384in}}%
\pgfpathlineto{\pgfqpoint{6.044234in}{5.528904in}}%
\pgfpathlineto{\pgfqpoint{6.044682in}{5.534644in}}%
\pgfpathlineto{\pgfqpoint{6.046876in}{5.560903in}}%
\pgfpathlineto{\pgfqpoint{6.049217in}{5.587163in}}%
\pgfpathlineto{\pgfqpoint{6.051699in}{5.613422in}}%
\pgfpathlineto{\pgfqpoint{6.054316in}{5.639682in}}%
\pgfpathlineto{\pgfqpoint{6.057063in}{5.665941in}}%
\pgfpathlineto{\pgfqpoint{6.059932in}{5.692201in}}%
\pgfpathlineto{\pgfqpoint{6.062917in}{5.718460in}}%
\pgfpathlineto{\pgfqpoint{6.066012in}{5.744720in}}%
\pgfpathlineto{\pgfqpoint{6.069208in}{5.770979in}}%
\pgfpathlineto{\pgfqpoint{6.072498in}{5.797238in}}%
\pgfusepath{stroke}%
\end{pgfscope}%
\begin{pgfscope}%
\pgfpathrectangle{\pgfqpoint{0.854460in}{0.571603in}}{\pgfqpoint{6.885100in}{5.225635in}}%
\pgfusepath{clip}%
\pgfsetbuttcap%
\pgfsetroundjoin%
\pgfsetlinewidth{1.505625pt}%
\definecolor{currentstroke}{rgb}{0.221989,0.339161,0.548752}%
\pgfsetstrokecolor{currentstroke}%
\pgfsetdash{}{0pt}%
\pgfpathmoveto{\pgfqpoint{1.102919in}{5.797238in}}%
\pgfpathlineto{\pgfqpoint{1.096650in}{5.790459in}}%
\pgfpathlineto{\pgfqpoint{1.078864in}{5.770979in}}%
\pgfpathlineto{\pgfqpoint{1.062051in}{5.752315in}}%
\pgfpathlineto{\pgfqpoint{1.055296in}{5.744720in}}%
\pgfpathlineto{\pgfqpoint{1.032281in}{5.718460in}}%
\pgfpathlineto{\pgfqpoint{1.027453in}{5.712868in}}%
\pgfpathlineto{\pgfqpoint{1.009843in}{5.692201in}}%
\pgfpathlineto{\pgfqpoint{0.992854in}{5.671985in}}%
\pgfpathlineto{\pgfqpoint{0.987842in}{5.665941in}}%
\pgfpathlineto{\pgfqpoint{0.966403in}{5.639682in}}%
\pgfpathlineto{\pgfqpoint{0.958256in}{5.629549in}}%
\pgfpathlineto{\pgfqpoint{0.945461in}{5.613422in}}%
\pgfpathlineto{\pgfqpoint{0.924931in}{5.587163in}}%
\pgfpathlineto{\pgfqpoint{0.923657in}{5.585505in}}%
\pgfpathlineto{\pgfqpoint{0.905007in}{5.560903in}}%
\pgfpathlineto{\pgfqpoint{0.889059in}{5.539552in}}%
\pgfpathlineto{\pgfqpoint{0.885442in}{5.534644in}}%
\pgfpathlineto{\pgfqpoint{0.866429in}{5.508384in}}%
\pgfpathlineto{\pgfqpoint{0.854460in}{5.491588in}}%
\pgfusepath{stroke}%
\end{pgfscope}%
\begin{pgfscope}%
\pgfpathrectangle{\pgfqpoint{0.854460in}{0.571603in}}{\pgfqpoint{6.885100in}{5.225635in}}%
\pgfusepath{clip}%
\pgfsetbuttcap%
\pgfsetroundjoin%
\pgfsetlinewidth{1.505625pt}%
\definecolor{currentstroke}{rgb}{0.221989,0.339161,0.548752}%
\pgfsetstrokecolor{currentstroke}%
\pgfsetdash{}{0pt}%
\pgfpathmoveto{\pgfqpoint{0.854460in}{2.616774in}}%
\pgfpathlineto{\pgfqpoint{0.871222in}{2.593583in}}%
\pgfpathlineto{\pgfqpoint{0.889059in}{2.569307in}}%
\pgfpathlineto{\pgfqpoint{0.890529in}{2.567323in}}%
\pgfpathlineto{\pgfqpoint{0.910396in}{2.541064in}}%
\pgfpathlineto{\pgfqpoint{0.923657in}{2.523824in}}%
\pgfpathlineto{\pgfqpoint{0.930654in}{2.514804in}}%
\pgfpathlineto{\pgfqpoint{0.951398in}{2.488545in}}%
\pgfpathlineto{\pgfqpoint{0.958256in}{2.480019in}}%
\pgfpathlineto{\pgfqpoint{0.972637in}{2.462285in}}%
\pgfpathlineto{\pgfqpoint{0.992854in}{2.437743in}}%
\pgfpathlineto{\pgfqpoint{0.994281in}{2.436026in}}%
\pgfpathlineto{\pgfqpoint{1.016506in}{2.409766in}}%
\pgfpathlineto{\pgfqpoint{1.027453in}{2.397038in}}%
\pgfpathlineto{\pgfqpoint{1.039180in}{2.383507in}}%
\pgfpathlineto{\pgfqpoint{1.062051in}{2.357521in}}%
\pgfpathlineto{\pgfqpoint{1.062294in}{2.357248in}}%
\pgfpathlineto{\pgfqpoint{1.086019in}{2.330988in}}%
\pgfpathlineto{\pgfqpoint{1.096650in}{2.319400in}}%
\pgfpathlineto{\pgfqpoint{1.110208in}{2.304729in}}%
\pgfpathlineto{\pgfqpoint{1.131248in}{2.282303in}}%
\pgfpathlineto{\pgfqpoint{1.134871in}{2.278469in}}%
\pgfpathlineto{\pgfqpoint{1.160105in}{2.252210in}}%
\pgfpathlineto{\pgfqpoint{1.165847in}{2.246329in}}%
\pgfpathlineto{\pgfqpoint{1.185883in}{2.225950in}}%
\pgfpathlineto{\pgfqpoint{1.200445in}{2.211359in}}%
\pgfpathlineto{\pgfqpoint{1.212168in}{2.199691in}}%
\pgfpathlineto{\pgfqpoint{1.235044in}{2.177257in}}%
\pgfpathlineto{\pgfqpoint{1.238970in}{2.173431in}}%
\pgfpathlineto{\pgfqpoint{1.266351in}{2.147172in}}%
\pgfpathlineto{\pgfqpoint{1.269642in}{2.144065in}}%
\pgfpathlineto{\pgfqpoint{1.294322in}{2.120912in}}%
\pgfpathlineto{\pgfqpoint{1.304241in}{2.111743in}}%
\pgfpathlineto{\pgfqpoint{1.322840in}{2.094653in}}%
\pgfpathlineto{\pgfqpoint{1.338839in}{2.080165in}}%
\pgfpathlineto{\pgfqpoint{1.351916in}{2.068393in}}%
\pgfpathlineto{\pgfqpoint{1.373438in}{2.049300in}}%
\pgfpathlineto{\pgfqpoint{1.381561in}{2.042134in}}%
\pgfpathlineto{\pgfqpoint{1.408036in}{2.019118in}}%
\pgfpathlineto{\pgfqpoint{1.411788in}{2.015874in}}%
\pgfpathlineto{\pgfqpoint{1.442606in}{1.989615in}}%
\pgfpathlineto{\pgfqpoint{1.442635in}{1.989590in}}%
\pgfpathlineto{\pgfqpoint{1.474069in}{1.963355in}}%
\pgfpathlineto{\pgfqpoint{1.477233in}{1.960753in}}%
\pgfpathlineto{\pgfqpoint{1.506136in}{1.937096in}}%
\pgfpathlineto{\pgfqpoint{1.511832in}{1.932501in}}%
\pgfpathlineto{\pgfqpoint{1.538816in}{1.910836in}}%
\pgfpathlineto{\pgfqpoint{1.546430in}{1.904811in}}%
\pgfpathlineto{\pgfqpoint{1.572118in}{1.884577in}}%
\pgfpathlineto{\pgfqpoint{1.581029in}{1.877658in}}%
\pgfpathlineto{\pgfqpoint{1.606051in}{1.858318in}}%
\pgfpathlineto{\pgfqpoint{1.615627in}{1.851021in}}%
\pgfpathlineto{\pgfqpoint{1.640624in}{1.832058in}}%
\pgfpathlineto{\pgfqpoint{1.650226in}{1.824878in}}%
\pgfpathlineto{\pgfqpoint{1.675845in}{1.805799in}}%
\pgfpathlineto{\pgfqpoint{1.684824in}{1.799206in}}%
\pgfpathlineto{\pgfqpoint{1.711720in}{1.779539in}}%
\pgfpathlineto{\pgfqpoint{1.719423in}{1.773987in}}%
\pgfpathlineto{\pgfqpoint{1.748259in}{1.753280in}}%
\pgfpathlineto{\pgfqpoint{1.754021in}{1.749200in}}%
\pgfpathlineto{\pgfqpoint{1.785466in}{1.727020in}}%
\pgfpathlineto{\pgfqpoint{1.788620in}{1.724827in}}%
\pgfpathlineto{\pgfqpoint{1.823218in}{1.700852in}}%
\pgfpathlineto{\pgfqpoint{1.823351in}{1.700761in}}%
\pgfpathlineto{\pgfqpoint{1.857817in}{1.677316in}}%
\pgfpathlineto{\pgfqpoint{1.861969in}{1.674501in}}%
\pgfpathlineto{\pgfqpoint{1.892415in}{1.654150in}}%
\pgfpathlineto{\pgfqpoint{1.901282in}{1.648242in}}%
\pgfpathlineto{\pgfqpoint{1.927014in}{1.631337in}}%
\pgfpathlineto{\pgfqpoint{1.941295in}{1.621982in}}%
\pgfpathlineto{\pgfqpoint{1.961612in}{1.608862in}}%
\pgfpathlineto{\pgfqpoint{1.982012in}{1.595723in}}%
\pgfpathlineto{\pgfqpoint{1.996211in}{1.586707in}}%
\pgfpathlineto{\pgfqpoint{2.023437in}{1.569463in}}%
\pgfpathlineto{\pgfqpoint{2.030809in}{1.564860in}}%
\pgfpathlineto{\pgfqpoint{2.065408in}{1.543307in}}%
\pgfpathlineto{\pgfqpoint{2.065575in}{1.543204in}}%
\pgfpathlineto{\pgfqpoint{2.071908in}{1.539331in}}%
\pgfusepath{stroke}%
\end{pgfscope}%
\begin{pgfscope}%
\pgfpathrectangle{\pgfqpoint{0.854460in}{0.571603in}}{\pgfqpoint{6.885100in}{5.225635in}}%
\pgfusepath{clip}%
\pgfsetbuttcap%
\pgfsetroundjoin%
\pgfsetlinewidth{1.505625pt}%
\definecolor{currentstroke}{rgb}{0.221989,0.339161,0.548752}%
\pgfsetstrokecolor{currentstroke}%
\pgfsetdash{}{0pt}%
\pgfpathmoveto{\pgfqpoint{2.407485in}{1.345594in}}%
\pgfpathlineto{\pgfqpoint{2.411393in}{1.343472in}}%
\pgfpathlineto{\pgfqpoint{2.430456in}{1.333128in}}%
\pgfpathlineto{\pgfqpoint{2.445991in}{1.324819in}}%
\pgfpathlineto{\pgfqpoint{2.479567in}{1.306869in}}%
\pgfpathlineto{\pgfqpoint{2.480590in}{1.306330in}}%
\pgfpathlineto{\pgfqpoint{2.515188in}{1.288168in}}%
\pgfpathlineto{\pgfqpoint{2.529594in}{1.280609in}}%
\pgfpathlineto{\pgfqpoint{2.549787in}{1.270166in}}%
\pgfpathlineto{\pgfqpoint{2.580369in}{1.254350in}}%
\pgfpathlineto{\pgfqpoint{2.584385in}{1.252303in}}%
\pgfpathlineto{\pgfqpoint{2.618983in}{1.234720in}}%
\pgfpathlineto{\pgfqpoint{2.632041in}{1.228090in}}%
\pgfpathlineto{\pgfqpoint{2.653582in}{1.217309in}}%
\pgfpathlineto{\pgfqpoint{2.684497in}{1.201831in}}%
\pgfpathlineto{\pgfqpoint{2.688180in}{1.200013in}}%
\pgfpathlineto{\pgfqpoint{2.722779in}{1.182993in}}%
\pgfpathlineto{\pgfqpoint{2.737871in}{1.175571in}}%
\pgfpathlineto{\pgfqpoint{2.757377in}{1.166116in}}%
\pgfpathlineto{\pgfqpoint{2.791976in}{1.149332in}}%
\pgfpathlineto{\pgfqpoint{2.792018in}{1.149312in}}%
\pgfpathlineto{\pgfqpoint{2.826574in}{1.132857in}}%
\pgfpathlineto{\pgfqpoint{2.847142in}{1.123052in}}%
\pgfpathlineto{\pgfqpoint{2.861173in}{1.116460in}}%
\pgfpathlineto{\pgfqpoint{2.895771in}{1.100214in}}%
\pgfpathlineto{\pgfqpoint{2.903074in}{1.096793in}}%
\pgfpathlineto{\pgfqpoint{2.930370in}{1.084189in}}%
\pgfpathlineto{\pgfqpoint{2.959900in}{1.070533in}}%
\pgfpathlineto{\pgfqpoint{2.964968in}{1.068223in}}%
\pgfpathlineto{\pgfqpoint{2.999567in}{1.052499in}}%
\pgfpathlineto{\pgfqpoint{3.017656in}{1.044274in}}%
\pgfpathlineto{\pgfqpoint{3.034165in}{1.036876in}}%
\pgfpathlineto{\pgfqpoint{3.068764in}{1.021371in}}%
\pgfpathlineto{\pgfqpoint{3.076270in}{1.018014in}}%
\pgfpathlineto{\pgfqpoint{3.103362in}{1.006075in}}%
\pgfpathlineto{\pgfqpoint{3.135787in}{0.991755in}}%
\pgfpathlineto{\pgfqpoint{3.137961in}{0.990809in}}%
\pgfpathlineto{\pgfqpoint{3.172559in}{0.975804in}}%
\pgfpathlineto{\pgfqpoint{3.196281in}{0.965495in}}%
\pgfpathlineto{\pgfqpoint{3.207158in}{0.960838in}}%
\pgfpathlineto{\pgfqpoint{3.241756in}{0.946043in}}%
\pgfpathlineto{\pgfqpoint{3.257674in}{0.939236in}}%
\pgfpathlineto{\pgfqpoint{3.276355in}{0.931365in}}%
\pgfpathlineto{\pgfqpoint{3.310953in}{0.916777in}}%
\pgfpathlineto{\pgfqpoint{3.319981in}{0.912976in}}%
\pgfpathlineto{\pgfqpoint{3.345552in}{0.902372in}}%
\pgfpathlineto{\pgfqpoint{3.380150in}{0.887988in}}%
\pgfpathlineto{\pgfqpoint{3.383218in}{0.886717in}}%
\pgfpathlineto{\pgfqpoint{3.414749in}{0.873844in}}%
\pgfpathlineto{\pgfqpoint{3.447423in}{0.860458in}}%
\pgfpathlineto{\pgfqpoint{3.449347in}{0.859681in}}%
\pgfpathlineto{\pgfqpoint{3.483946in}{0.845764in}}%
\pgfpathlineto{\pgfqpoint{3.512609in}{0.834198in}}%
\pgfpathlineto{\pgfqpoint{3.518544in}{0.831839in}}%
\pgfpathlineto{\pgfqpoint{3.553143in}{0.818119in}}%
\pgfpathlineto{\pgfqpoint{3.578751in}{0.807939in}}%
\pgfpathlineto{\pgfqpoint{3.587741in}{0.804418in}}%
\pgfpathlineto{\pgfqpoint{3.622340in}{0.790892in}}%
\pgfpathlineto{\pgfqpoint{3.645859in}{0.781679in}}%
\pgfpathlineto{\pgfqpoint{3.656938in}{0.777404in}}%
\pgfpathlineto{\pgfqpoint{3.691537in}{0.764070in}}%
\pgfpathlineto{\pgfqpoint{3.713943in}{0.755420in}}%
\pgfpathlineto{\pgfqpoint{3.726135in}{0.750784in}}%
\pgfpathlineto{\pgfqpoint{3.760734in}{0.737639in}}%
\pgfpathlineto{\pgfqpoint{3.783011in}{0.729160in}}%
\pgfpathlineto{\pgfqpoint{3.795332in}{0.724542in}}%
\pgfpathlineto{\pgfqpoint{3.829931in}{0.711584in}}%
\pgfpathlineto{\pgfqpoint{3.853071in}{0.702901in}}%
\pgfpathlineto{\pgfqpoint{3.864529in}{0.698666in}}%
\pgfpathlineto{\pgfqpoint{3.899128in}{0.685893in}}%
\pgfpathlineto{\pgfqpoint{3.924127in}{0.676641in}}%
\pgfpathlineto{\pgfqpoint{3.933726in}{0.673143in}}%
\pgfpathlineto{\pgfqpoint{3.968325in}{0.660553in}}%
\pgfpathlineto{\pgfqpoint{3.996186in}{0.650382in}}%
\pgfpathlineto{\pgfqpoint{4.002923in}{0.647960in}}%
\pgfpathlineto{\pgfqpoint{4.037522in}{0.635550in}}%
\pgfpathlineto{\pgfqpoint{4.069252in}{0.624122in}}%
\pgfpathlineto{\pgfqpoint{4.072120in}{0.623105in}}%
\pgfpathlineto{\pgfqpoint{4.106719in}{0.610874in}}%
\pgfpathlineto{\pgfqpoint{4.141317in}{0.598583in}}%
\pgfpathlineto{\pgfqpoint{4.143352in}{0.597863in}}%
\pgfpathlineto{\pgfqpoint{4.175916in}{0.586511in}}%
\pgfpathlineto{\pgfqpoint{4.210514in}{0.574398in}}%
\pgfpathlineto{\pgfqpoint{4.218510in}{0.571603in}}%
\pgfusepath{stroke}%
\end{pgfscope}%
\begin{pgfscope}%
\pgfpathrectangle{\pgfqpoint{0.854460in}{0.571603in}}{\pgfqpoint{6.885100in}{5.225635in}}%
\pgfusepath{clip}%
\pgfsetbuttcap%
\pgfsetroundjoin%
\pgfsetlinewidth{1.505625pt}%
\definecolor{currentstroke}{rgb}{0.221989,0.339161,0.548752}%
\pgfsetstrokecolor{currentstroke}%
\pgfsetdash{}{0pt}%
\pgfpathmoveto{\pgfqpoint{7.739560in}{2.436346in}}%
\pgfpathlineto{\pgfqpoint{7.709537in}{2.462285in}}%
\pgfpathlineto{\pgfqpoint{7.704962in}{2.466279in}}%
\pgfpathlineto{\pgfqpoint{7.679449in}{2.488545in}}%
\pgfpathlineto{\pgfqpoint{7.670363in}{2.496558in}}%
\pgfpathlineto{\pgfqpoint{7.649667in}{2.514804in}}%
\pgfpathlineto{\pgfqpoint{7.635765in}{2.527195in}}%
\pgfpathlineto{\pgfqpoint{7.620195in}{2.541064in}}%
\pgfpathlineto{\pgfqpoint{7.601166in}{2.558203in}}%
\pgfpathlineto{\pgfqpoint{7.591033in}{2.567323in}}%
\pgfpathlineto{\pgfqpoint{7.566568in}{2.589595in}}%
\pgfpathlineto{\pgfqpoint{7.562184in}{2.593583in}}%
\pgfpathlineto{\pgfqpoint{7.533634in}{2.619842in}}%
\pgfpathlineto{\pgfqpoint{7.531969in}{2.621390in}}%
\pgfpathlineto{\pgfqpoint{7.505359in}{2.646102in}}%
\pgfpathlineto{\pgfqpoint{7.497371in}{2.653610in}}%
\pgfpathlineto{\pgfqpoint{7.477399in}{2.672361in}}%
\pgfpathlineto{\pgfqpoint{7.462772in}{2.686265in}}%
\pgfpathlineto{\pgfqpoint{7.449757in}{2.698621in}}%
\pgfpathlineto{\pgfqpoint{7.428174in}{2.719370in}}%
\pgfpathlineto{\pgfqpoint{7.422434in}{2.724880in}}%
\pgfpathlineto{\pgfqpoint{7.395417in}{2.751140in}}%
\pgfpathlineto{\pgfqpoint{7.393575in}{2.752951in}}%
\pgfpathlineto{\pgfqpoint{7.368670in}{2.777399in}}%
\pgfpathlineto{\pgfqpoint{7.358977in}{2.787043in}}%
\pgfpathlineto{\pgfqpoint{7.342247in}{2.803659in}}%
\pgfpathlineto{\pgfqpoint{7.324378in}{2.821649in}}%
\pgfpathlineto{\pgfqpoint{7.320753in}{2.825292in}}%
\pgfusepath{stroke}%
\end{pgfscope}%
\begin{pgfscope}%
\pgfpathrectangle{\pgfqpoint{0.854460in}{0.571603in}}{\pgfqpoint{6.885100in}{5.225635in}}%
\pgfusepath{clip}%
\pgfsetbuttcap%
\pgfsetroundjoin%
\pgfsetlinewidth{1.505625pt}%
\definecolor{currentstroke}{rgb}{0.221989,0.339161,0.548752}%
\pgfsetstrokecolor{currentstroke}%
\pgfsetdash{}{0pt}%
\pgfpathmoveto{\pgfqpoint{7.056066in}{3.111046in}}%
\pgfpathlineto{\pgfqpoint{7.004905in}{3.171291in}}%
\pgfpathlineto{\pgfqpoint{6.940552in}{3.250070in}}%
\pgfpathlineto{\pgfqpoint{6.874598in}{3.334651in}}%
\pgfpathlineto{\pgfqpoint{6.820207in}{3.407626in}}%
\pgfpathlineto{\pgfqpoint{6.764268in}{3.486405in}}%
\pgfpathlineto{\pgfqpoint{6.711110in}{3.565183in}}%
\pgfpathlineto{\pgfqpoint{6.660764in}{3.643962in}}%
\pgfpathlineto{\pgfqpoint{6.613186in}{3.722740in}}%
\pgfpathlineto{\pgfqpoint{6.563212in}{3.811118in}}%
\pgfpathlineto{\pgfqpoint{6.526429in}{3.880297in}}%
\pgfpathlineto{\pgfqpoint{6.487166in}{3.959075in}}%
\pgfpathlineto{\pgfqpoint{6.450659in}{4.037854in}}%
\pgfpathlineto{\pgfqpoint{6.416886in}{4.116632in}}%
\pgfpathlineto{\pgfqpoint{6.385829in}{4.195411in}}%
\pgfpathlineto{\pgfqpoint{6.355621in}{4.279647in}}%
\pgfpathlineto{\pgfqpoint{6.331703in}{4.352967in}}%
\pgfpathlineto{\pgfqpoint{6.308604in}{4.431746in}}%
\pgfpathlineto{\pgfqpoint{6.286424in}{4.517739in}}%
\pgfpathlineto{\pgfqpoint{6.270161in}{4.589303in}}%
\pgfpathlineto{\pgfqpoint{6.254771in}{4.668081in}}%
\pgfpathlineto{\pgfqpoint{6.241806in}{4.746860in}}%
\pgfpathlineto{\pgfqpoint{6.231283in}{4.825638in}}%
\pgfpathlineto{\pgfqpoint{6.223157in}{4.904416in}}%
\pgfpathlineto{\pgfqpoint{6.217227in}{4.985489in}}%
\pgfpathlineto{\pgfqpoint{6.213755in}{5.061973in}}%
\pgfpathlineto{\pgfqpoint{6.212347in}{5.140752in}}%
\pgfpathlineto{\pgfqpoint{6.213045in}{5.219530in}}%
\pgfpathlineto{\pgfqpoint{6.215767in}{5.298308in}}%
\pgfpathlineto{\pgfqpoint{6.220388in}{5.377087in}}%
\pgfpathlineto{\pgfqpoint{6.226809in}{5.455865in}}%
\pgfpathlineto{\pgfqpoint{6.238016in}{5.560903in}}%
\pgfpathlineto{\pgfqpoint{6.251992in}{5.665941in}}%
\pgfpathlineto{\pgfqpoint{6.268241in}{5.770979in}}%
\pgfpathlineto{\pgfqpoint{6.272641in}{5.797238in}}%
\pgfpathlineto{\pgfqpoint{6.272641in}{5.797238in}}%
\pgfusepath{stroke}%
\end{pgfscope}%
\begin{pgfscope}%
\pgfpathrectangle{\pgfqpoint{0.854460in}{0.571603in}}{\pgfqpoint{6.885100in}{5.225635in}}%
\pgfusepath{clip}%
\pgfsetbuttcap%
\pgfsetroundjoin%
\pgfsetlinewidth{1.505625pt}%
\definecolor{currentstroke}{rgb}{0.212395,0.359683,0.551710}%
\pgfsetstrokecolor{currentstroke}%
\pgfsetdash{}{0pt}%
\pgfpathmoveto{\pgfqpoint{0.954572in}{5.797238in}}%
\pgfpathlineto{\pgfqpoint{0.931525in}{5.770979in}}%
\pgfpathlineto{\pgfqpoint{0.923657in}{5.761884in}}%
\pgfpathlineto{\pgfqpoint{0.908993in}{5.744720in}}%
\pgfpathlineto{\pgfqpoint{0.889059in}{5.721062in}}%
\pgfpathlineto{\pgfqpoint{0.886894in}{5.718460in}}%
\pgfpathlineto{\pgfqpoint{0.865397in}{5.692201in}}%
\pgfpathlineto{\pgfqpoint{0.854460in}{5.678646in}}%
\pgfusepath{stroke}%
\end{pgfscope}%
\begin{pgfscope}%
\pgfpathrectangle{\pgfqpoint{0.854460in}{0.571603in}}{\pgfqpoint{6.885100in}{5.225635in}}%
\pgfusepath{clip}%
\pgfsetbuttcap%
\pgfsetroundjoin%
\pgfsetlinewidth{1.505625pt}%
\definecolor{currentstroke}{rgb}{0.212395,0.359683,0.551710}%
\pgfsetstrokecolor{currentstroke}%
\pgfsetdash{}{0pt}%
\pgfpathmoveto{\pgfqpoint{7.739560in}{2.572723in}}%
\pgfpathlineto{\pgfqpoint{7.716538in}{2.593583in}}%
\pgfpathlineto{\pgfqpoint{7.704962in}{2.604195in}}%
\pgfpathlineto{\pgfqpoint{7.690558in}{2.617389in}}%
\pgfusepath{stroke}%
\end{pgfscope}%
\begin{pgfscope}%
\pgfpathrectangle{\pgfqpoint{0.854460in}{0.571603in}}{\pgfqpoint{6.885100in}{5.225635in}}%
\pgfusepath{clip}%
\pgfsetbuttcap%
\pgfsetroundjoin%
\pgfsetlinewidth{1.505625pt}%
\definecolor{currentstroke}{rgb}{0.212395,0.359683,0.551710}%
\pgfsetstrokecolor{currentstroke}%
\pgfsetdash{}{0pt}%
\pgfpathmoveto{\pgfqpoint{7.412030in}{2.889179in}}%
\pgfpathlineto{\pgfqpoint{7.343787in}{2.961215in}}%
\pgfpathlineto{\pgfqpoint{7.271920in}{3.039994in}}%
\pgfpathlineto{\pgfqpoint{7.202894in}{3.118772in}}%
\pgfpathlineto{\pgfqpoint{7.136715in}{3.197551in}}%
\pgfpathlineto{\pgfqpoint{7.073391in}{3.276329in}}%
\pgfpathlineto{\pgfqpoint{7.012937in}{3.355107in}}%
\pgfpathlineto{\pgfqpoint{6.955271in}{3.433886in}}%
\pgfpathlineto{\pgfqpoint{6.900491in}{3.512664in}}%
\pgfpathlineto{\pgfqpoint{6.848551in}{3.591443in}}%
\pgfpathlineto{\pgfqpoint{6.799457in}{3.670221in}}%
\pgfpathlineto{\pgfqpoint{6.753173in}{3.749000in}}%
\pgfpathlineto{\pgfqpoint{6.709740in}{3.827778in}}%
\pgfpathlineto{\pgfqpoint{6.667007in}{3.910883in}}%
\pgfpathlineto{\pgfqpoint{6.631303in}{3.985335in}}%
\pgfpathlineto{\pgfqpoint{6.596259in}{4.064113in}}%
\pgfpathlineto{\pgfqpoint{6.563212in}{4.144910in}}%
\pgfpathlineto{\pgfqpoint{6.534445in}{4.221670in}}%
\pgfpathlineto{\pgfqpoint{6.507632in}{4.300449in}}%
\pgfpathlineto{\pgfqpoint{6.483540in}{4.379227in}}%
\pgfpathlineto{\pgfqpoint{6.462141in}{4.458005in}}%
\pgfpathlineto{\pgfqpoint{6.443334in}{4.536784in}}%
\pgfpathlineto{\pgfqpoint{6.427194in}{4.615562in}}%
\pgfpathlineto{\pgfqpoint{6.413570in}{4.694341in}}%
\pgfpathlineto{\pgfqpoint{6.402497in}{4.773119in}}%
\pgfpathlineto{\pgfqpoint{6.393931in}{4.851897in}}%
\pgfpathlineto{\pgfqpoint{6.387780in}{4.930676in}}%
\pgfpathlineto{\pgfqpoint{6.383987in}{5.009454in}}%
\pgfpathlineto{\pgfqpoint{6.382520in}{5.088233in}}%
\pgfpathlineto{\pgfqpoint{6.383310in}{5.167011in}}%
\pgfpathlineto{\pgfqpoint{6.386286in}{5.245790in}}%
\pgfpathlineto{\pgfqpoint{6.391363in}{5.324568in}}%
\pgfpathlineto{\pgfqpoint{6.398415in}{5.403346in}}%
\pgfpathlineto{\pgfqpoint{6.407398in}{5.482125in}}%
\pgfpathlineto{\pgfqpoint{6.418224in}{5.560903in}}%
\pgfpathlineto{\pgfqpoint{6.435248in}{5.665941in}}%
\pgfpathlineto{\pgfqpoint{6.455018in}{5.770979in}}%
\pgfpathlineto{\pgfqpoint{6.460356in}{5.797238in}}%
\pgfpathlineto{\pgfqpoint{6.460356in}{5.797238in}}%
\pgfusepath{stroke}%
\end{pgfscope}%
\begin{pgfscope}%
\pgfpathrectangle{\pgfqpoint{0.854460in}{0.571603in}}{\pgfqpoint{6.885100in}{5.225635in}}%
\pgfusepath{clip}%
\pgfsetbuttcap%
\pgfsetroundjoin%
\pgfsetlinewidth{1.505625pt}%
\definecolor{currentstroke}{rgb}{0.212395,0.359683,0.551710}%
\pgfsetstrokecolor{currentstroke}%
\pgfsetdash{}{0pt}%
\pgfpathmoveto{\pgfqpoint{0.854460in}{2.460697in}}%
\pgfpathlineto{\pgfqpoint{0.874618in}{2.436026in}}%
\pgfpathlineto{\pgfqpoint{0.889059in}{2.418625in}}%
\pgfpathlineto{\pgfqpoint{0.896469in}{2.409766in}}%
\pgfpathlineto{\pgfqpoint{0.918807in}{2.383507in}}%
\pgfpathlineto{\pgfqpoint{0.923657in}{2.377904in}}%
\pgfpathlineto{\pgfqpoint{0.941676in}{2.357248in}}%
\pgfpathlineto{\pgfqpoint{0.958256in}{2.338528in}}%
\pgfpathlineto{\pgfqpoint{0.964984in}{2.330988in}}%
\pgfpathlineto{\pgfqpoint{0.988802in}{2.304729in}}%
\pgfpathlineto{\pgfqpoint{0.992854in}{2.300336in}}%
\pgfpathlineto{\pgfqpoint{1.013173in}{2.278469in}}%
\pgfpathlineto{\pgfqpoint{1.027453in}{2.263330in}}%
\pgfpathlineto{\pgfqpoint{1.038016in}{2.252210in}}%
\pgfpathlineto{\pgfqpoint{1.062051in}{2.227280in}}%
\pgfpathlineto{\pgfqpoint{1.063342in}{2.225950in}}%
\pgfpathlineto{\pgfqpoint{1.089270in}{2.199691in}}%
\pgfpathlineto{\pgfqpoint{1.096650in}{2.192329in}}%
\pgfpathlineto{\pgfqpoint{1.115718in}{2.173431in}}%
\pgfpathlineto{\pgfqpoint{1.131248in}{2.158264in}}%
\pgfpathlineto{\pgfqpoint{1.142680in}{2.147172in}}%
\pgfpathlineto{\pgfqpoint{1.165847in}{2.125019in}}%
\pgfpathlineto{\pgfqpoint{1.170169in}{2.120912in}}%
\pgfpathlineto{\pgfqpoint{1.198227in}{2.094653in}}%
\pgfpathlineto{\pgfqpoint{1.200445in}{2.092609in}}%
\pgfpathlineto{\pgfqpoint{1.226886in}{2.068393in}}%
\pgfpathlineto{\pgfqpoint{1.235044in}{2.061030in}}%
\pgfpathlineto{\pgfqpoint{1.256099in}{2.042134in}}%
\pgfpathlineto{\pgfqpoint{1.269642in}{2.030154in}}%
\pgfpathlineto{\pgfqpoint{1.285876in}{2.015874in}}%
\pgfpathlineto{\pgfqpoint{1.304241in}{1.999952in}}%
\pgfpathlineto{\pgfqpoint{1.316229in}{1.989615in}}%
\pgfpathlineto{\pgfqpoint{1.338839in}{1.970396in}}%
\pgfpathlineto{\pgfqpoint{1.347167in}{1.963355in}}%
\pgfpathlineto{\pgfqpoint{1.373438in}{1.941460in}}%
\pgfpathlineto{\pgfqpoint{1.378701in}{1.937096in}}%
\pgfpathlineto{\pgfqpoint{1.408036in}{1.913118in}}%
\pgfpathlineto{\pgfqpoint{1.410841in}{1.910836in}}%
\pgfpathlineto{\pgfqpoint{1.442635in}{1.885345in}}%
\pgfpathlineto{\pgfqpoint{1.443597in}{1.884577in}}%
\pgfpathlineto{\pgfqpoint{1.476980in}{1.858318in}}%
\pgfpathlineto{\pgfqpoint{1.477233in}{1.858121in}}%
\pgfpathlineto{\pgfqpoint{1.511001in}{1.832058in}}%
\pgfpathlineto{\pgfqpoint{1.511832in}{1.831426in}}%
\pgfpathlineto{\pgfqpoint{1.545655in}{1.805799in}}%
\pgfpathlineto{\pgfqpoint{1.546430in}{1.805219in}}%
\pgfpathlineto{\pgfqpoint{1.580950in}{1.779539in}}%
\pgfpathlineto{\pgfqpoint{1.581029in}{1.779481in}}%
\pgfpathlineto{\pgfqpoint{1.615627in}{1.754213in}}%
\pgfpathlineto{\pgfqpoint{1.616910in}{1.753280in}}%
\pgfpathlineto{\pgfqpoint{1.650226in}{1.729385in}}%
\pgfpathlineto{\pgfqpoint{1.653536in}{1.727020in}}%
\pgfpathlineto{\pgfqpoint{1.684824in}{1.704978in}}%
\pgfpathlineto{\pgfqpoint{1.690833in}{1.700761in}}%
\pgfpathlineto{\pgfqpoint{1.719423in}{1.680973in}}%
\pgfpathlineto{\pgfqpoint{1.728806in}{1.674501in}}%
\pgfpathlineto{\pgfqpoint{1.754021in}{1.657353in}}%
\pgfpathlineto{\pgfqpoint{1.767462in}{1.648242in}}%
\pgfpathlineto{\pgfqpoint{1.788620in}{1.634101in}}%
\pgfpathlineto{\pgfqpoint{1.806806in}{1.621982in}}%
\pgfpathlineto{\pgfqpoint{1.823218in}{1.611200in}}%
\pgfpathlineto{\pgfqpoint{1.846843in}{1.595723in}}%
\pgfpathlineto{\pgfqpoint{1.857817in}{1.588634in}}%
\pgfpathlineto{\pgfqpoint{1.887576in}{1.569463in}}%
\pgfpathlineto{\pgfqpoint{1.892415in}{1.566390in}}%
\pgfpathlineto{\pgfqpoint{1.927014in}{1.544479in}}%
\pgfpathlineto{\pgfqpoint{1.929036in}{1.543204in}}%
\pgfpathlineto{\pgfqpoint{1.961612in}{1.522938in}}%
\pgfpathlineto{\pgfqpoint{1.971270in}{1.516944in}}%
\pgfpathlineto{\pgfqpoint{1.996211in}{1.501682in}}%
\pgfpathlineto{\pgfqpoint{2.014219in}{1.490685in}}%
\pgfpathlineto{\pgfqpoint{2.030809in}{1.480696in}}%
\pgfpathlineto{\pgfqpoint{2.057884in}{1.464425in}}%
\pgfpathlineto{\pgfqpoint{2.065408in}{1.459967in}}%
\pgfpathlineto{\pgfqpoint{2.100006in}{1.439514in}}%
\pgfpathlineto{\pgfqpoint{2.102295in}{1.438166in}}%
\pgfpathlineto{\pgfqpoint{2.134605in}{1.419402in}}%
\pgfpathlineto{\pgfqpoint{2.147530in}{1.411906in}}%
\pgfpathlineto{\pgfqpoint{2.169203in}{1.399515in}}%
\pgfpathlineto{\pgfqpoint{2.193491in}{1.385647in}}%
\pgfpathlineto{\pgfqpoint{2.203802in}{1.379843in}}%
\pgfpathlineto{\pgfqpoint{2.238400in}{1.360396in}}%
\pgfpathlineto{\pgfqpoint{2.240201in}{1.359388in}}%
\pgfpathlineto{\pgfqpoint{2.272999in}{1.341279in}}%
\pgfpathlineto{\pgfqpoint{2.287776in}{1.333128in}}%
\pgfpathlineto{\pgfqpoint{2.307597in}{1.322348in}}%
\pgfpathlineto{\pgfqpoint{2.336081in}{1.306869in}}%
\pgfpathlineto{\pgfqpoint{2.342196in}{1.303592in}}%
\pgfpathlineto{\pgfqpoint{2.376794in}{1.285103in}}%
\pgfpathlineto{\pgfqpoint{2.385220in}{1.280609in}}%
\pgfpathlineto{\pgfqpoint{2.411393in}{1.266847in}}%
\pgfpathlineto{\pgfqpoint{2.435169in}{1.254350in}}%
\pgfpathlineto{\pgfqpoint{2.445991in}{1.248741in}}%
\pgfpathlineto{\pgfqpoint{2.480590in}{1.230837in}}%
\pgfpathlineto{\pgfqpoint{2.485913in}{1.228090in}}%
\pgfpathlineto{\pgfqpoint{2.515188in}{1.213196in}}%
\pgfpathlineto{\pgfqpoint{2.537525in}{1.201831in}}%
\pgfpathlineto{\pgfqpoint{2.549787in}{1.195680in}}%
\pgfpathlineto{\pgfqpoint{2.584385in}{1.178343in}}%
\pgfpathlineto{\pgfqpoint{2.589932in}{1.175571in}}%
\pgfpathlineto{\pgfqpoint{2.618983in}{1.161258in}}%
\pgfpathlineto{\pgfqpoint{2.643220in}{1.149312in}}%
\pgfpathlineto{\pgfqpoint{2.653582in}{1.144276in}}%
\pgfpathlineto{\pgfqpoint{2.688180in}{1.127489in}}%
\pgfpathlineto{\pgfqpoint{2.697341in}{1.123052in}}%
\pgfpathlineto{\pgfqpoint{2.722779in}{1.110907in}}%
\pgfpathlineto{\pgfqpoint{2.752312in}{1.096793in}}%
\pgfpathlineto{\pgfqpoint{2.757377in}{1.094407in}}%
\pgfpathlineto{\pgfqpoint{2.791976in}{1.078152in}}%
\pgfpathlineto{\pgfqpoint{2.808193in}{1.070533in}}%
\pgfpathlineto{\pgfqpoint{2.826574in}{1.062021in}}%
\pgfpathlineto{\pgfqpoint{2.861173in}{1.045992in}}%
\pgfpathlineto{\pgfqpoint{2.864892in}{1.044274in}}%
\pgfpathlineto{\pgfqpoint{2.895771in}{1.030215in}}%
\pgfpathlineto{\pgfqpoint{2.922525in}{1.018014in}}%
\pgfpathlineto{\pgfqpoint{2.930370in}{1.014488in}}%
\pgfpathlineto{\pgfqpoint{2.964968in}{0.998968in}}%
\pgfpathlineto{\pgfqpoint{2.981049in}{0.991755in}}%
\pgfpathlineto{\pgfqpoint{2.999567in}{0.983568in}}%
\pgfpathlineto{\pgfqpoint{3.034165in}{0.968263in}}%
\pgfpathlineto{\pgfqpoint{3.040436in}{0.965495in}}%
\pgfpathlineto{\pgfqpoint{3.068764in}{0.953174in}}%
\pgfpathlineto{\pgfqpoint{3.100735in}{0.939236in}}%
\pgfpathlineto{\pgfqpoint{3.103362in}{0.938107in}}%
\pgfpathlineto{\pgfqpoint{3.132610in}{0.925581in}}%
\pgfusepath{stroke}%
\end{pgfscope}%
\begin{pgfscope}%
\pgfpathrectangle{\pgfqpoint{0.854460in}{0.571603in}}{\pgfqpoint{6.885100in}{5.225635in}}%
\pgfusepath{clip}%
\pgfsetbuttcap%
\pgfsetroundjoin%
\pgfsetlinewidth{1.505625pt}%
\definecolor{currentstroke}{rgb}{0.212395,0.359683,0.551710}%
\pgfsetstrokecolor{currentstroke}%
\pgfsetdash{}{0pt}%
\pgfpathmoveto{\pgfqpoint{3.490315in}{0.778454in}}%
\pgfpathlineto{\pgfqpoint{3.518544in}{0.767390in}}%
\pgfpathlineto{\pgfqpoint{3.548976in}{0.755420in}}%
\pgfpathlineto{\pgfqpoint{3.553143in}{0.753805in}}%
\pgfpathlineto{\pgfqpoint{3.587741in}{0.740433in}}%
\pgfpathlineto{\pgfqpoint{3.616817in}{0.729160in}}%
\pgfpathlineto{\pgfqpoint{3.622340in}{0.727051in}}%
\pgfpathlineto{\pgfqpoint{3.656938in}{0.713866in}}%
\pgfpathlineto{\pgfqpoint{3.685625in}{0.702901in}}%
\pgfpathlineto{\pgfqpoint{3.691537in}{0.700674in}}%
\pgfpathlineto{\pgfqpoint{3.726135in}{0.687675in}}%
\pgfpathlineto{\pgfqpoint{3.755406in}{0.676641in}}%
\pgfpathlineto{\pgfqpoint{3.760734in}{0.674663in}}%
\pgfpathlineto{\pgfqpoint{3.795332in}{0.661846in}}%
\pgfpathlineto{\pgfqpoint{3.826166in}{0.650382in}}%
\pgfpathlineto{\pgfqpoint{3.829931in}{0.649003in}}%
\pgfpathlineto{\pgfqpoint{3.864529in}{0.636367in}}%
\pgfpathlineto{\pgfqpoint{3.897910in}{0.624122in}}%
\pgfpathlineto{\pgfqpoint{3.899128in}{0.623682in}}%
\pgfpathlineto{\pgfqpoint{3.933726in}{0.611224in}}%
\pgfpathlineto{\pgfqpoint{3.968325in}{0.598709in}}%
\pgfpathlineto{\pgfqpoint{3.970670in}{0.597863in}}%
\pgfpathlineto{\pgfqpoint{4.002923in}{0.586407in}}%
\pgfpathlineto{\pgfqpoint{4.037522in}{0.574068in}}%
\pgfpathlineto{\pgfqpoint{4.044446in}{0.571603in}}%
\pgfusepath{stroke}%
\end{pgfscope}%
\begin{pgfscope}%
\pgfpathrectangle{\pgfqpoint{0.854460in}{0.571603in}}{\pgfqpoint{6.885100in}{5.225635in}}%
\pgfusepath{clip}%
\pgfsetbuttcap%
\pgfsetroundjoin%
\pgfsetlinewidth{1.505625pt}%
\definecolor{currentstroke}{rgb}{0.201239,0.383670,0.554294}%
\pgfsetstrokecolor{currentstroke}%
\pgfsetdash{}{0pt}%
\pgfpathmoveto{\pgfqpoint{7.739560in}{2.710692in}}%
\pgfpathlineto{\pgfqpoint{7.724705in}{2.724880in}}%
\pgfpathlineto{\pgfqpoint{7.704962in}{2.743983in}}%
\pgfpathlineto{\pgfqpoint{7.697555in}{2.751140in}}%
\pgfpathlineto{\pgfqpoint{7.670735in}{2.777399in}}%
\pgfpathlineto{\pgfqpoint{7.670363in}{2.777767in}}%
\pgfpathlineto{\pgfqpoint{7.644186in}{2.803659in}}%
\pgfpathlineto{\pgfqpoint{7.635765in}{2.812103in}}%
\pgfpathlineto{\pgfqpoint{7.617971in}{2.829918in}}%
\pgfpathlineto{\pgfqpoint{7.601166in}{2.846980in}}%
\pgfpathlineto{\pgfqpoint{7.592091in}{2.856177in}}%
\pgfpathlineto{\pgfqpoint{7.566568in}{2.882419in}}%
\pgfpathlineto{\pgfqpoint{7.566550in}{2.882437in}}%
\pgfpathlineto{\pgfqpoint{7.541273in}{2.908696in}}%
\pgfpathlineto{\pgfqpoint{7.531969in}{2.918506in}}%
\pgfpathlineto{\pgfqpoint{7.516335in}{2.934956in}}%
\pgfpathlineto{\pgfqpoint{7.497371in}{2.955214in}}%
\pgfpathlineto{\pgfqpoint{7.491741in}{2.961215in}}%
\pgfpathlineto{\pgfqpoint{7.467453in}{2.987475in}}%
\pgfpathlineto{\pgfqpoint{7.462772in}{2.992608in}}%
\pgfpathlineto{\pgfqpoint{7.443464in}{3.013734in}}%
\pgfpathlineto{\pgfqpoint{7.428174in}{3.030732in}}%
\pgfpathlineto{\pgfqpoint{7.419822in}{3.039994in}}%
\pgfpathlineto{\pgfqpoint{7.396507in}{3.066253in}}%
\pgfpathlineto{\pgfqpoint{7.393575in}{3.069603in}}%
\pgfpathlineto{\pgfqpoint{7.373474in}{3.092513in}}%
\pgfpathlineto{\pgfqpoint{7.358977in}{3.109315in}}%
\pgfpathlineto{\pgfqpoint{7.350793in}{3.118772in}}%
\pgfpathlineto{\pgfqpoint{7.328436in}{3.145032in}}%
\pgfpathlineto{\pgfqpoint{7.324378in}{3.149872in}}%
\pgfpathlineto{\pgfqpoint{7.306368in}{3.171291in}}%
\pgfpathlineto{\pgfqpoint{7.289780in}{3.191371in}}%
\pgfpathlineto{\pgfqpoint{7.284659in}{3.197551in}}%
\pgfpathlineto{\pgfqpoint{7.263246in}{3.223810in}}%
\pgfpathlineto{\pgfqpoint{7.255181in}{3.233871in}}%
\pgfpathlineto{\pgfqpoint{7.242153in}{3.250070in}}%
\pgfpathlineto{\pgfqpoint{7.221420in}{3.276329in}}%
\pgfpathlineto{\pgfqpoint{7.220583in}{3.277406in}}%
\pgfpathlineto{\pgfqpoint{7.200945in}{3.302589in}}%
\pgfpathlineto{\pgfqpoint{7.185984in}{3.322148in}}%
\pgfpathlineto{\pgfqpoint{7.180841in}{3.328848in}}%
\pgfpathlineto{\pgfqpoint{7.161030in}{3.355107in}}%
\pgfpathlineto{\pgfqpoint{7.151386in}{3.368134in}}%
\pgfpathlineto{\pgfqpoint{7.141550in}{3.381367in}}%
\pgfpathlineto{\pgfqpoint{7.122402in}{3.407626in}}%
\pgfpathlineto{\pgfqpoint{7.116787in}{3.415469in}}%
\pgfpathlineto{\pgfqpoint{7.103550in}{3.433886in}}%
\pgfpathlineto{\pgfqpoint{7.085056in}{3.460145in}}%
\pgfpathlineto{\pgfqpoint{7.082189in}{3.464293in}}%
\pgfpathlineto{\pgfqpoint{7.066836in}{3.486405in}}%
\pgfpathlineto{\pgfqpoint{7.048990in}{3.512664in}}%
\pgfpathlineto{\pgfqpoint{7.048645in}{3.513183in}}%
\pgfusepath{stroke}%
\end{pgfscope}%
\begin{pgfscope}%
\pgfpathrectangle{\pgfqpoint{0.854460in}{0.571603in}}{\pgfqpoint{6.885100in}{5.225635in}}%
\pgfusepath{clip}%
\pgfsetbuttcap%
\pgfsetroundjoin%
\pgfsetlinewidth{1.505625pt}%
\definecolor{currentstroke}{rgb}{0.201239,0.383670,0.554294}%
\pgfsetstrokecolor{currentstroke}%
\pgfsetdash{}{0pt}%
\pgfpathmoveto{\pgfqpoint{6.848416in}{3.848793in}}%
\pgfpathlineto{\pgfqpoint{6.845677in}{3.854037in}}%
\pgfpathlineto{\pgfqpoint{6.839999in}{3.865188in}}%
\pgfpathlineto{\pgfqpoint{6.832260in}{3.880297in}}%
\pgfpathlineto{\pgfqpoint{6.819155in}{3.906556in}}%
\pgfpathlineto{\pgfqpoint{6.806407in}{3.932816in}}%
\pgfpathlineto{\pgfqpoint{6.805401in}{3.934943in}}%
\pgfpathlineto{\pgfqpoint{6.793914in}{3.959075in}}%
\pgfpathlineto{\pgfqpoint{6.781765in}{3.985335in}}%
\pgfpathlineto{\pgfqpoint{6.770803in}{4.009729in}}%
\pgfpathlineto{\pgfqpoint{6.769959in}{4.011594in}}%
\pgfpathlineto{\pgfqpoint{6.758406in}{4.037854in}}%
\pgfpathlineto{\pgfqpoint{6.747200in}{4.064113in}}%
\pgfpathlineto{\pgfqpoint{6.736335in}{4.090373in}}%
\pgfpathlineto{\pgfqpoint{6.736204in}{4.090700in}}%
\pgfpathlineto{\pgfqpoint{6.725720in}{4.116632in}}%
\pgfpathlineto{\pgfqpoint{6.715442in}{4.142892in}}%
\pgfpathlineto{\pgfqpoint{6.705500in}{4.169151in}}%
\pgfpathlineto{\pgfqpoint{6.701606in}{4.179786in}}%
\pgfpathlineto{\pgfqpoint{6.695841in}{4.195411in}}%
\pgfpathlineto{\pgfqpoint{6.686481in}{4.221670in}}%
\pgfpathlineto{\pgfqpoint{6.677449in}{4.247930in}}%
\pgfpathlineto{\pgfqpoint{6.668745in}{4.274189in}}%
\pgfpathlineto{\pgfqpoint{6.667007in}{4.279630in}}%
\pgfpathlineto{\pgfqpoint{6.660305in}{4.300449in}}%
\pgfpathlineto{\pgfqpoint{6.652175in}{4.326708in}}%
\pgfpathlineto{\pgfqpoint{6.644364in}{4.352967in}}%
\pgfpathlineto{\pgfqpoint{6.636872in}{4.379227in}}%
\pgfpathlineto{\pgfqpoint{6.632409in}{4.395557in}}%
\pgfpathlineto{\pgfqpoint{6.629671in}{4.405486in}}%
\pgfpathlineto{\pgfqpoint{6.622747in}{4.431746in}}%
\pgfpathlineto{\pgfqpoint{6.616134in}{4.458005in}}%
\pgfpathlineto{\pgfqpoint{6.609832in}{4.484265in}}%
\pgfpathlineto{\pgfqpoint{6.603838in}{4.510524in}}%
\pgfpathlineto{\pgfqpoint{6.598148in}{4.536784in}}%
\pgfpathlineto{\pgfqpoint{6.597810in}{4.538431in}}%
\pgfpathlineto{\pgfqpoint{6.592716in}{4.563043in}}%
\pgfpathlineto{\pgfqpoint{6.587583in}{4.589303in}}%
\pgfpathlineto{\pgfqpoint{6.582750in}{4.615562in}}%
\pgfpathlineto{\pgfqpoint{6.578215in}{4.641822in}}%
\pgfpathlineto{\pgfqpoint{6.573976in}{4.668081in}}%
\pgfpathlineto{\pgfqpoint{6.570029in}{4.694341in}}%
\pgfpathlineto{\pgfqpoint{6.566373in}{4.720600in}}%
\pgfpathlineto{\pgfqpoint{6.563212in}{4.745253in}}%
\pgfpathlineto{\pgfqpoint{6.563003in}{4.746860in}}%
\pgfpathlineto{\pgfqpoint{6.559894in}{4.773119in}}%
\pgfpathlineto{\pgfqpoint{6.557070in}{4.799378in}}%
\pgfpathlineto{\pgfqpoint{6.554531in}{4.825638in}}%
\pgfpathlineto{\pgfqpoint{6.552273in}{4.851897in}}%
\pgfpathlineto{\pgfqpoint{6.550294in}{4.878157in}}%
\pgfpathlineto{\pgfqpoint{6.548593in}{4.904416in}}%
\pgfpathlineto{\pgfqpoint{6.547166in}{4.930676in}}%
\pgfpathlineto{\pgfqpoint{6.546012in}{4.956935in}}%
\pgfpathlineto{\pgfqpoint{6.545129in}{4.983195in}}%
\pgfpathlineto{\pgfqpoint{6.544513in}{5.009454in}}%
\pgfpathlineto{\pgfqpoint{6.544164in}{5.035714in}}%
\pgfpathlineto{\pgfqpoint{6.544078in}{5.061973in}}%
\pgfpathlineto{\pgfqpoint{6.544254in}{5.088233in}}%
\pgfpathlineto{\pgfqpoint{6.544689in}{5.114492in}}%
\pgfpathlineto{\pgfqpoint{6.545381in}{5.140752in}}%
\pgfpathlineto{\pgfqpoint{6.546328in}{5.167011in}}%
\pgfpathlineto{\pgfqpoint{6.547528in}{5.193271in}}%
\pgfpathlineto{\pgfqpoint{6.548977in}{5.219530in}}%
\pgfpathlineto{\pgfqpoint{6.550674in}{5.245790in}}%
\pgfpathlineto{\pgfqpoint{6.552616in}{5.272049in}}%
\pgfpathlineto{\pgfqpoint{6.554801in}{5.298308in}}%
\pgfpathlineto{\pgfqpoint{6.557227in}{5.324568in}}%
\pgfpathlineto{\pgfqpoint{6.559891in}{5.350827in}}%
\pgfpathlineto{\pgfqpoint{6.562791in}{5.377087in}}%
\pgfpathlineto{\pgfqpoint{6.563212in}{5.380629in}}%
\pgfpathlineto{\pgfqpoint{6.565897in}{5.403346in}}%
\pgfpathlineto{\pgfqpoint{6.569227in}{5.429606in}}%
\pgfpathlineto{\pgfqpoint{6.572783in}{5.455865in}}%
\pgfpathlineto{\pgfqpoint{6.576561in}{5.482125in}}%
\pgfpathlineto{\pgfqpoint{6.580559in}{5.508384in}}%
\pgfpathlineto{\pgfqpoint{6.584774in}{5.534644in}}%
\pgfpathlineto{\pgfqpoint{6.589202in}{5.560903in}}%
\pgfpathlineto{\pgfqpoint{6.593842in}{5.587163in}}%
\pgfpathlineto{\pgfqpoint{6.597810in}{5.608681in}}%
\pgfpathlineto{\pgfqpoint{6.598680in}{5.613422in}}%
\pgfpathlineto{\pgfqpoint{6.603682in}{5.639682in}}%
\pgfpathlineto{\pgfqpoint{6.608883in}{5.665941in}}%
\pgfpathlineto{\pgfqpoint{6.614279in}{5.692201in}}%
\pgfpathlineto{\pgfqpoint{6.619867in}{5.718460in}}%
\pgfpathlineto{\pgfqpoint{6.625643in}{5.744720in}}%
\pgfpathlineto{\pgfqpoint{6.631604in}{5.770979in}}%
\pgfpathlineto{\pgfqpoint{6.632409in}{5.774437in}}%
\pgfpathlineto{\pgfqpoint{6.637692in}{5.797238in}}%
\pgfusepath{stroke}%
\end{pgfscope}%
\begin{pgfscope}%
\pgfpathrectangle{\pgfqpoint{0.854460in}{0.571603in}}{\pgfqpoint{6.885100in}{5.225635in}}%
\pgfusepath{clip}%
\pgfsetbuttcap%
\pgfsetroundjoin%
\pgfsetlinewidth{1.505625pt}%
\definecolor{currentstroke}{rgb}{0.201239,0.383670,0.554294}%
\pgfsetstrokecolor{currentstroke}%
\pgfsetdash{}{0pt}%
\pgfpathmoveto{\pgfqpoint{0.854460in}{2.325642in}}%
\pgfpathlineto{\pgfqpoint{0.873225in}{2.304729in}}%
\pgfpathlineto{\pgfqpoint{0.889059in}{2.287344in}}%
\pgfpathlineto{\pgfqpoint{0.897201in}{2.278469in}}%
\pgfpathlineto{\pgfqpoint{0.921667in}{2.252210in}}%
\pgfpathlineto{\pgfqpoint{0.923657in}{2.250109in}}%
\pgfpathlineto{\pgfqpoint{0.946712in}{2.225950in}}%
\pgfpathlineto{\pgfqpoint{0.958256in}{2.214030in}}%
\pgfpathlineto{\pgfqpoint{0.972238in}{2.199691in}}%
\pgfpathlineto{\pgfqpoint{0.992854in}{2.178855in}}%
\pgfpathlineto{\pgfqpoint{0.998257in}{2.173431in}}%
\pgfpathlineto{\pgfqpoint{1.024817in}{2.147172in}}%
\pgfpathlineto{\pgfqpoint{1.027453in}{2.144607in}}%
\pgfpathlineto{\pgfqpoint{1.051961in}{2.120912in}}%
\pgfpathlineto{\pgfqpoint{1.062051in}{2.111297in}}%
\pgfpathlineto{\pgfqpoint{1.079626in}{2.094653in}}%
\pgfpathlineto{\pgfqpoint{1.096650in}{2.078762in}}%
\pgfpathlineto{\pgfqpoint{1.107825in}{2.068393in}}%
\pgfpathlineto{\pgfqpoint{1.131248in}{2.046970in}}%
\pgfpathlineto{\pgfqpoint{1.136568in}{2.042134in}}%
\pgfpathlineto{\pgfqpoint{1.165847in}{2.015892in}}%
\pgfpathlineto{\pgfqpoint{1.165867in}{2.015874in}}%
\pgfpathlineto{\pgfqpoint{1.195795in}{1.989615in}}%
\pgfpathlineto{\pgfqpoint{1.200445in}{1.985593in}}%
\pgfpathlineto{\pgfqpoint{1.226294in}{1.963355in}}%
\pgfpathlineto{\pgfqpoint{1.235044in}{1.955934in}}%
\pgfpathlineto{\pgfqpoint{1.257372in}{1.937096in}}%
\pgfpathlineto{\pgfqpoint{1.269642in}{1.926889in}}%
\pgfpathlineto{\pgfqpoint{1.289039in}{1.910836in}}%
\pgfpathlineto{\pgfqpoint{1.304241in}{1.898432in}}%
\pgfpathlineto{\pgfqpoint{1.321306in}{1.884577in}}%
\pgfpathlineto{\pgfqpoint{1.338839in}{1.870541in}}%
\pgfpathlineto{\pgfqpoint{1.354181in}{1.858318in}}%
\pgfpathlineto{\pgfqpoint{1.373438in}{1.843190in}}%
\pgfpathlineto{\pgfqpoint{1.387674in}{1.832058in}}%
\pgfpathlineto{\pgfqpoint{1.408036in}{1.816359in}}%
\pgfpathlineto{\pgfqpoint{1.421794in}{1.805799in}}%
\pgfpathlineto{\pgfqpoint{1.442635in}{1.790024in}}%
\pgfpathlineto{\pgfqpoint{1.456547in}{1.779539in}}%
\pgfpathlineto{\pgfqpoint{1.477233in}{1.764166in}}%
\pgfpathlineto{\pgfqpoint{1.491943in}{1.753280in}}%
\pgfpathlineto{\pgfqpoint{1.511832in}{1.738765in}}%
\pgfpathlineto{\pgfqpoint{1.527989in}{1.727020in}}%
\pgfpathlineto{\pgfqpoint{1.546430in}{1.713801in}}%
\pgfpathlineto{\pgfqpoint{1.564691in}{1.700761in}}%
\pgfpathlineto{\pgfqpoint{1.581029in}{1.689256in}}%
\pgfpathlineto{\pgfqpoint{1.602057in}{1.674501in}}%
\pgfpathlineto{\pgfqpoint{1.615627in}{1.665111in}}%
\pgfpathlineto{\pgfqpoint{1.640091in}{1.648242in}}%
\pgfpathlineto{\pgfqpoint{1.650226in}{1.641350in}}%
\pgfpathlineto{\pgfqpoint{1.678801in}{1.621982in}}%
\pgfpathlineto{\pgfqpoint{1.684824in}{1.617956in}}%
\pgfpathlineto{\pgfqpoint{1.694803in}{1.611307in}}%
\pgfusepath{stroke}%
\end{pgfscope}%
\begin{pgfscope}%
\pgfpathrectangle{\pgfqpoint{0.854460in}{0.571603in}}{\pgfqpoint{6.885100in}{5.225635in}}%
\pgfusepath{clip}%
\pgfsetbuttcap%
\pgfsetroundjoin%
\pgfsetlinewidth{1.505625pt}%
\definecolor{currentstroke}{rgb}{0.201239,0.383670,0.554294}%
\pgfsetstrokecolor{currentstroke}%
\pgfsetdash{}{0pt}%
\pgfpathmoveto{\pgfqpoint{2.023836in}{1.406279in}}%
\pgfpathlineto{\pgfqpoint{2.030809in}{1.402224in}}%
\pgfpathlineto{\pgfqpoint{2.059368in}{1.385647in}}%
\pgfpathlineto{\pgfqpoint{2.065408in}{1.382190in}}%
\pgfpathlineto{\pgfqpoint{2.100006in}{1.362437in}}%
\pgfpathlineto{\pgfqpoint{2.105364in}{1.359388in}}%
\pgfpathlineto{\pgfqpoint{2.134605in}{1.342972in}}%
\pgfpathlineto{\pgfqpoint{2.152161in}{1.333128in}}%
\pgfpathlineto{\pgfqpoint{2.169203in}{1.323705in}}%
\pgfpathlineto{\pgfqpoint{2.199684in}{1.306869in}}%
\pgfpathlineto{\pgfqpoint{2.203802in}{1.304626in}}%
\pgfpathlineto{\pgfqpoint{2.238400in}{1.285839in}}%
\pgfpathlineto{\pgfqpoint{2.248048in}{1.280609in}}%
\pgfpathlineto{\pgfqpoint{2.272999in}{1.267273in}}%
\pgfpathlineto{\pgfqpoint{2.297191in}{1.254350in}}%
\pgfpathlineto{\pgfqpoint{2.307597in}{1.248868in}}%
\pgfpathlineto{\pgfqpoint{2.342196in}{1.230672in}}%
\pgfpathlineto{\pgfqpoint{2.347120in}{1.228090in}}%
\pgfpathlineto{\pgfqpoint{2.376794in}{1.212745in}}%
\pgfpathlineto{\pgfqpoint{2.397906in}{1.201831in}}%
\pgfpathlineto{\pgfqpoint{2.411393in}{1.194955in}}%
\pgfpathlineto{\pgfqpoint{2.445991in}{1.177332in}}%
\pgfpathlineto{\pgfqpoint{2.449459in}{1.175571in}}%
\pgfpathlineto{\pgfqpoint{2.480590in}{1.159985in}}%
\pgfpathlineto{\pgfqpoint{2.501904in}{1.149312in}}%
\pgfpathlineto{\pgfqpoint{2.515188in}{1.142752in}}%
\pgfpathlineto{\pgfqpoint{2.549787in}{1.125684in}}%
\pgfpathlineto{\pgfqpoint{2.555134in}{1.123052in}}%
\pgfpathlineto{\pgfqpoint{2.584385in}{1.108862in}}%
\pgfpathlineto{\pgfqpoint{2.609248in}{1.096793in}}%
\pgfpathlineto{\pgfqpoint{2.618983in}{1.092133in}}%
\pgfpathlineto{\pgfqpoint{2.653582in}{1.075601in}}%
\pgfpathlineto{\pgfqpoint{2.664203in}{1.070533in}}%
\pgfpathlineto{\pgfqpoint{2.688180in}{1.059253in}}%
\pgfpathlineto{\pgfqpoint{2.719989in}{1.044274in}}%
\pgfpathlineto{\pgfqpoint{2.722779in}{1.042978in}}%
\pgfpathlineto{\pgfqpoint{2.757377in}{1.026966in}}%
\pgfpathlineto{\pgfqpoint{2.776708in}{1.018014in}}%
\pgfpathlineto{\pgfqpoint{2.791976in}{1.011044in}}%
\pgfpathlineto{\pgfqpoint{2.826574in}{0.995253in}}%
\pgfpathlineto{\pgfqpoint{2.834254in}{0.991755in}}%
\pgfpathlineto{\pgfqpoint{2.861173in}{0.979667in}}%
\pgfpathlineto{\pgfqpoint{2.892680in}{0.965495in}}%
\pgfpathlineto{\pgfqpoint{2.895771in}{0.964125in}}%
\pgfpathlineto{\pgfqpoint{2.930370in}{0.948831in}}%
\pgfpathlineto{\pgfqpoint{2.952048in}{0.939236in}}%
\pgfpathlineto{\pgfqpoint{2.964968in}{0.933599in}}%
\pgfpathlineto{\pgfqpoint{2.999567in}{0.918516in}}%
\pgfpathlineto{\pgfqpoint{3.012284in}{0.912976in}}%
\pgfpathlineto{\pgfqpoint{3.034165in}{0.903581in}}%
\pgfpathlineto{\pgfqpoint{3.068764in}{0.888706in}}%
\pgfpathlineto{\pgfqpoint{3.073402in}{0.886717in}}%
\pgfpathlineto{\pgfqpoint{3.103362in}{0.874054in}}%
\pgfpathlineto{\pgfqpoint{3.135447in}{0.860458in}}%
\pgfpathlineto{\pgfqpoint{3.137961in}{0.859408in}}%
\pgfpathlineto{\pgfqpoint{3.172559in}{0.845001in}}%
\pgfpathlineto{\pgfqpoint{3.198446in}{0.834198in}}%
\pgfpathlineto{\pgfqpoint{3.207158in}{0.830615in}}%
\pgfpathlineto{\pgfqpoint{3.241756in}{0.816408in}}%
\pgfpathlineto{\pgfqpoint{3.262359in}{0.807939in}}%
\pgfpathlineto{\pgfqpoint{3.276355in}{0.802268in}}%
\pgfpathlineto{\pgfqpoint{3.310953in}{0.788257in}}%
\pgfpathlineto{\pgfqpoint{3.327195in}{0.781679in}}%
\pgfpathlineto{\pgfqpoint{3.345552in}{0.774352in}}%
\pgfpathlineto{\pgfqpoint{3.380150in}{0.760535in}}%
\pgfpathlineto{\pgfqpoint{3.392967in}{0.755420in}}%
\pgfpathlineto{\pgfqpoint{3.414749in}{0.746853in}}%
\pgfpathlineto{\pgfqpoint{3.449347in}{0.733226in}}%
\pgfpathlineto{\pgfqpoint{3.459683in}{0.729160in}}%
\pgfpathlineto{\pgfqpoint{3.483946in}{0.719755in}}%
\pgfpathlineto{\pgfqpoint{3.518544in}{0.706317in}}%
\pgfpathlineto{\pgfqpoint{3.527353in}{0.702901in}}%
\pgfpathlineto{\pgfqpoint{3.553143in}{0.693045in}}%
\pgfpathlineto{\pgfqpoint{3.587741in}{0.679793in}}%
\pgfpathlineto{\pgfqpoint{3.595983in}{0.676641in}}%
\pgfpathlineto{\pgfqpoint{3.622340in}{0.666710in}}%
\pgfpathlineto{\pgfqpoint{3.656938in}{0.653642in}}%
\pgfpathlineto{\pgfqpoint{3.665582in}{0.650382in}}%
\pgfpathlineto{\pgfqpoint{3.691537in}{0.640737in}}%
\pgfpathlineto{\pgfqpoint{3.726135in}{0.627849in}}%
\pgfpathlineto{\pgfqpoint{3.736153in}{0.624122in}}%
\pgfpathlineto{\pgfqpoint{3.760734in}{0.615112in}}%
\pgfpathlineto{\pgfqpoint{3.795332in}{0.602403in}}%
\pgfpathlineto{\pgfqpoint{3.807701in}{0.597863in}}%
\pgfpathlineto{\pgfqpoint{3.829931in}{0.589824in}}%
\pgfpathlineto{\pgfqpoint{3.864529in}{0.577291in}}%
\pgfpathlineto{\pgfqpoint{3.880231in}{0.571603in}}%
\pgfusepath{stroke}%
\end{pgfscope}%
\begin{pgfscope}%
\pgfpathrectangle{\pgfqpoint{0.854460in}{0.571603in}}{\pgfqpoint{6.885100in}{5.225635in}}%
\pgfusepath{clip}%
\pgfsetbuttcap%
\pgfsetroundjoin%
\pgfsetlinewidth{1.505625pt}%
\definecolor{currentstroke}{rgb}{0.190631,0.407061,0.556089}%
\pgfsetstrokecolor{currentstroke}%
\pgfsetdash{}{0pt}%
\pgfpathmoveto{\pgfqpoint{7.739560in}{2.851710in}}%
\pgfpathlineto{\pgfqpoint{7.735146in}{2.856177in}}%
\pgfpathlineto{\pgfqpoint{7.709538in}{2.882437in}}%
\pgfpathlineto{\pgfqpoint{7.704962in}{2.887194in}}%
\pgfpathlineto{\pgfqpoint{7.684237in}{2.908696in}}%
\pgfpathlineto{\pgfqpoint{7.670363in}{2.923307in}}%
\pgfpathlineto{\pgfqpoint{7.659280in}{2.934956in}}%
\pgfpathlineto{\pgfqpoint{7.635765in}{2.960049in}}%
\pgfpathlineto{\pgfqpoint{7.634670in}{2.961215in}}%
\pgfpathlineto{\pgfqpoint{7.610335in}{2.987475in}}%
\pgfpathlineto{\pgfqpoint{7.601166in}{2.997524in}}%
\pgfpathlineto{\pgfqpoint{7.586341in}{3.013734in}}%
\pgfpathlineto{\pgfqpoint{7.566568in}{3.035704in}}%
\pgfpathlineto{\pgfqpoint{7.562697in}{3.039994in}}%
\pgfpathlineto{\pgfqpoint{7.539349in}{3.066253in}}%
\pgfpathlineto{\pgfqpoint{7.531969in}{3.074684in}}%
\pgfpathlineto{\pgfqpoint{7.516324in}{3.092513in}}%
\pgfpathlineto{\pgfqpoint{7.497371in}{3.114478in}}%
\pgfpathlineto{\pgfqpoint{7.493655in}{3.118772in}}%
\pgfpathlineto{\pgfqpoint{7.471278in}{3.145032in}}%
\pgfpathlineto{\pgfqpoint{7.462772in}{3.155181in}}%
\pgfpathlineto{\pgfqpoint{7.449231in}{3.171291in}}%
\pgfpathlineto{\pgfqpoint{7.428174in}{3.196794in}}%
\pgfpathlineto{\pgfqpoint{7.427547in}{3.197551in}}%
\pgfpathlineto{\pgfqpoint{7.406128in}{3.223810in}}%
\pgfpathlineto{\pgfqpoint{7.393575in}{3.239483in}}%
\pgfpathlineto{\pgfqpoint{7.385069in}{3.250070in}}%
\pgfpathlineto{\pgfqpoint{7.372419in}{3.266093in}}%
\pgfusepath{stroke}%
\end{pgfscope}%
\begin{pgfscope}%
\pgfpathrectangle{\pgfqpoint{0.854460in}{0.571603in}}{\pgfqpoint{6.885100in}{5.225635in}}%
\pgfusepath{clip}%
\pgfsetbuttcap%
\pgfsetroundjoin%
\pgfsetlinewidth{1.505625pt}%
\definecolor{currentstroke}{rgb}{0.190631,0.407061,0.556089}%
\pgfsetstrokecolor{currentstroke}%
\pgfsetdash{}{0pt}%
\pgfpathmoveto{\pgfqpoint{7.145701in}{3.583810in}}%
\pgfpathlineto{\pgfqpoint{7.140794in}{3.591443in}}%
\pgfpathlineto{\pgfqpoint{7.124281in}{3.617702in}}%
\pgfpathlineto{\pgfqpoint{7.116787in}{3.629882in}}%
\pgfpathlineto{\pgfqpoint{7.108083in}{3.643962in}}%
\pgfpathlineto{\pgfqpoint{7.092209in}{3.670221in}}%
\pgfpathlineto{\pgfqpoint{7.082189in}{3.687184in}}%
\pgfpathlineto{\pgfqpoint{7.076670in}{3.696481in}}%
\pgfpathlineto{\pgfqpoint{7.061429in}{3.722740in}}%
\pgfpathlineto{\pgfqpoint{7.047590in}{3.747180in}}%
\pgfpathlineto{\pgfqpoint{7.046555in}{3.749000in}}%
\pgfpathlineto{\pgfqpoint{7.031943in}{3.775259in}}%
\pgfpathlineto{\pgfqpoint{7.017702in}{3.801519in}}%
\pgfpathlineto{\pgfqpoint{7.012992in}{3.810417in}}%
\pgfpathlineto{\pgfqpoint{7.003753in}{3.827778in}}%
\pgfpathlineto{\pgfqpoint{6.990132in}{3.854037in}}%
\pgfpathlineto{\pgfqpoint{6.978393in}{3.877281in}}%
\pgfpathlineto{\pgfqpoint{6.976862in}{3.880297in}}%
\pgfpathlineto{\pgfqpoint{6.963857in}{3.906556in}}%
\pgfpathlineto{\pgfqpoint{6.951212in}{3.932816in}}%
\pgfpathlineto{\pgfqpoint{6.943795in}{3.948651in}}%
\pgfpathlineto{\pgfqpoint{6.938882in}{3.959075in}}%
\pgfpathlineto{\pgfqpoint{6.926846in}{3.985335in}}%
\pgfpathlineto{\pgfqpoint{6.915162in}{4.011594in}}%
\pgfpathlineto{\pgfqpoint{6.909196in}{4.025403in}}%
\pgfpathlineto{\pgfqpoint{6.903782in}{4.037854in}}%
\pgfpathlineto{\pgfqpoint{6.892701in}{4.064113in}}%
\pgfpathlineto{\pgfqpoint{6.881965in}{4.090373in}}%
\pgfpathlineto{\pgfqpoint{6.874598in}{4.108978in}}%
\pgfpathlineto{\pgfqpoint{6.871546in}{4.116632in}}%
\pgfpathlineto{\pgfqpoint{6.861408in}{4.142892in}}%
\pgfpathlineto{\pgfqpoint{6.851608in}{4.169151in}}%
\pgfpathlineto{\pgfqpoint{6.842144in}{4.195411in}}%
\pgfpathlineto{\pgfqpoint{6.839999in}{4.201574in}}%
\pgfpathlineto{\pgfqpoint{6.832955in}{4.221670in}}%
\pgfpathlineto{\pgfqpoint{6.824081in}{4.247930in}}%
\pgfpathlineto{\pgfqpoint{6.815537in}{4.274189in}}%
\pgfpathlineto{\pgfqpoint{6.807321in}{4.300449in}}%
\pgfpathlineto{\pgfqpoint{6.805401in}{4.306832in}}%
\pgfpathlineto{\pgfqpoint{6.799378in}{4.326708in}}%
\pgfpathlineto{\pgfqpoint{6.791745in}{4.352967in}}%
\pgfpathlineto{\pgfqpoint{6.784433in}{4.379227in}}%
\pgfpathlineto{\pgfqpoint{6.777442in}{4.405486in}}%
\pgfpathlineto{\pgfqpoint{6.770803in}{4.431604in}}%
\pgfpathlineto{\pgfqpoint{6.770766in}{4.431746in}}%
\pgfpathlineto{\pgfqpoint{6.764352in}{4.458005in}}%
\pgfpathlineto{\pgfqpoint{6.758252in}{4.484265in}}%
\pgfpathlineto{\pgfqpoint{6.752465in}{4.510524in}}%
\pgfpathlineto{\pgfqpoint{6.746988in}{4.536784in}}%
\pgfpathlineto{\pgfqpoint{6.741818in}{4.563043in}}%
\pgfpathlineto{\pgfqpoint{6.736953in}{4.589303in}}%
\pgfpathlineto{\pgfqpoint{6.736204in}{4.593616in}}%
\pgfpathlineto{\pgfqpoint{6.732359in}{4.615562in}}%
\pgfpathlineto{\pgfqpoint{6.728061in}{4.641822in}}%
\pgfpathlineto{\pgfqpoint{6.724064in}{4.668081in}}%
\pgfpathlineto{\pgfqpoint{6.720367in}{4.694341in}}%
\pgfpathlineto{\pgfqpoint{6.716966in}{4.720600in}}%
\pgfpathlineto{\pgfqpoint{6.713859in}{4.746860in}}%
\pgfpathlineto{\pgfqpoint{6.711045in}{4.773119in}}%
\pgfpathlineto{\pgfqpoint{6.708520in}{4.799378in}}%
\pgfpathlineto{\pgfqpoint{6.706285in}{4.825638in}}%
\pgfpathlineto{\pgfqpoint{6.704335in}{4.851897in}}%
\pgfpathlineto{\pgfqpoint{6.702669in}{4.878157in}}%
\pgfpathlineto{\pgfqpoint{6.701606in}{4.898350in}}%
\pgfpathlineto{\pgfqpoint{6.701283in}{4.904416in}}%
\pgfpathlineto{\pgfqpoint{6.700169in}{4.930676in}}%
\pgfpathlineto{\pgfqpoint{6.699337in}{4.956935in}}%
\pgfpathlineto{\pgfqpoint{6.698782in}{4.983195in}}%
\pgfpathlineto{\pgfqpoint{6.698504in}{5.009454in}}%
\pgfpathlineto{\pgfqpoint{6.698501in}{5.035714in}}%
\pgfpathlineto{\pgfqpoint{6.698770in}{5.061973in}}%
\pgfpathlineto{\pgfqpoint{6.699309in}{5.088233in}}%
\pgfpathlineto{\pgfqpoint{6.700117in}{5.114492in}}%
\pgfpathlineto{\pgfqpoint{6.701191in}{5.140752in}}%
\pgfpathlineto{\pgfqpoint{6.701606in}{5.148899in}}%
\pgfpathlineto{\pgfqpoint{6.702521in}{5.167011in}}%
\pgfpathlineto{\pgfqpoint{6.704108in}{5.193271in}}%
\pgfpathlineto{\pgfqpoint{6.705953in}{5.219530in}}%
\pgfpathlineto{\pgfqpoint{6.708054in}{5.245790in}}%
\pgfpathlineto{\pgfqpoint{6.710409in}{5.272049in}}%
\pgfpathlineto{\pgfqpoint{6.713016in}{5.298308in}}%
\pgfpathlineto{\pgfqpoint{6.715873in}{5.324568in}}%
\pgfpathlineto{\pgfqpoint{6.718977in}{5.350827in}}%
\pgfpathlineto{\pgfqpoint{6.722326in}{5.377087in}}%
\pgfpathlineto{\pgfqpoint{6.725919in}{5.403346in}}%
\pgfpathlineto{\pgfqpoint{6.729753in}{5.429606in}}%
\pgfpathlineto{\pgfqpoint{6.733826in}{5.455865in}}%
\pgfpathlineto{\pgfqpoint{6.736204in}{5.470382in}}%
\pgfpathlineto{\pgfqpoint{6.738118in}{5.482125in}}%
\pgfpathlineto{\pgfqpoint{6.742620in}{5.508384in}}%
\pgfpathlineto{\pgfqpoint{6.747351in}{5.534644in}}%
\pgfpathlineto{\pgfqpoint{6.752309in}{5.560903in}}%
\pgfpathlineto{\pgfqpoint{6.757491in}{5.587163in}}%
\pgfpathlineto{\pgfqpoint{6.762895in}{5.613422in}}%
\pgfpathlineto{\pgfqpoint{6.768518in}{5.639682in}}%
\pgfpathlineto{\pgfqpoint{6.770803in}{5.649989in}}%
\pgfpathlineto{\pgfqpoint{6.774324in}{5.665941in}}%
\pgfpathlineto{\pgfqpoint{6.780319in}{5.692201in}}%
\pgfpathlineto{\pgfqpoint{6.786523in}{5.718460in}}%
\pgfpathlineto{\pgfqpoint{6.792932in}{5.744720in}}%
\pgfpathlineto{\pgfqpoint{6.799543in}{5.770979in}}%
\pgfpathlineto{\pgfqpoint{6.805401in}{5.793586in}}%
\pgfpathlineto{\pgfqpoint{6.806344in}{5.797238in}}%
\pgfusepath{stroke}%
\end{pgfscope}%
\begin{pgfscope}%
\pgfpathrectangle{\pgfqpoint{0.854460in}{0.571603in}}{\pgfqpoint{6.885100in}{5.225635in}}%
\pgfusepath{clip}%
\pgfsetbuttcap%
\pgfsetroundjoin%
\pgfsetlinewidth{1.505625pt}%
\definecolor{currentstroke}{rgb}{0.190631,0.407061,0.556089}%
\pgfsetstrokecolor{currentstroke}%
\pgfsetdash{}{0pt}%
\pgfpathmoveto{\pgfqpoint{0.854460in}{2.205838in}}%
\pgfpathlineto{\pgfqpoint{0.860377in}{2.199691in}}%
\pgfpathlineto{\pgfqpoint{0.886042in}{2.173431in}}%
\pgfpathlineto{\pgfqpoint{0.889059in}{2.170393in}}%
\pgfpathlineto{\pgfqpoint{0.912274in}{2.147172in}}%
\pgfpathlineto{\pgfqpoint{0.923657in}{2.135948in}}%
\pgfpathlineto{\pgfqpoint{0.939006in}{2.120912in}}%
\pgfpathlineto{\pgfqpoint{0.958256in}{2.102325in}}%
\pgfpathlineto{\pgfqpoint{0.966251in}{2.094653in}}%
\pgfpathlineto{\pgfqpoint{0.992854in}{2.069489in}}%
\pgfpathlineto{\pgfqpoint{0.994020in}{2.068393in}}%
\pgfpathlineto{\pgfqpoint{1.022392in}{2.042134in}}%
\pgfpathlineto{\pgfqpoint{1.027453in}{2.037517in}}%
\pgfpathlineto{\pgfqpoint{1.051318in}{2.015874in}}%
\pgfpathlineto{\pgfqpoint{1.062051in}{2.006277in}}%
\pgfpathlineto{\pgfqpoint{1.080794in}{1.989615in}}%
\pgfpathlineto{\pgfqpoint{1.096650in}{1.975716in}}%
\pgfpathlineto{\pgfqpoint{1.110830in}{1.963355in}}%
\pgfpathlineto{\pgfqpoint{1.131248in}{1.945806in}}%
\pgfpathlineto{\pgfqpoint{1.141438in}{1.937096in}}%
\pgfpathlineto{\pgfqpoint{1.165847in}{1.916521in}}%
\pgfpathlineto{\pgfqpoint{1.172626in}{1.910836in}}%
\pgfpathlineto{\pgfqpoint{1.200445in}{1.887835in}}%
\pgfpathlineto{\pgfqpoint{1.204406in}{1.884577in}}%
\pgfpathlineto{\pgfqpoint{1.235044in}{1.859724in}}%
\pgfpathlineto{\pgfqpoint{1.236786in}{1.858318in}}%
\pgfpathlineto{\pgfqpoint{1.269642in}{1.832164in}}%
\pgfpathlineto{\pgfqpoint{1.269775in}{1.832058in}}%
\pgfpathlineto{\pgfqpoint{1.287320in}{1.818355in}}%
\pgfusepath{stroke}%
\end{pgfscope}%
\begin{pgfscope}%
\pgfpathrectangle{\pgfqpoint{0.854460in}{0.571603in}}{\pgfqpoint{6.885100in}{5.225635in}}%
\pgfusepath{clip}%
\pgfsetbuttcap%
\pgfsetroundjoin%
\pgfsetlinewidth{1.505625pt}%
\definecolor{currentstroke}{rgb}{0.190631,0.407061,0.556089}%
\pgfsetstrokecolor{currentstroke}%
\pgfsetdash{}{0pt}%
\pgfpathmoveto{\pgfqpoint{1.602077in}{1.591312in}}%
\pgfpathlineto{\pgfqpoint{1.615627in}{1.582292in}}%
\pgfpathlineto{\pgfqpoint{1.634959in}{1.569463in}}%
\pgfpathlineto{\pgfqpoint{1.650226in}{1.559471in}}%
\pgfpathlineto{\pgfqpoint{1.675152in}{1.543204in}}%
\pgfpathlineto{\pgfqpoint{1.684824in}{1.536978in}}%
\pgfpathlineto{\pgfqpoint{1.716034in}{1.516944in}}%
\pgfpathlineto{\pgfqpoint{1.719423in}{1.514799in}}%
\pgfpathlineto{\pgfqpoint{1.754021in}{1.492969in}}%
\pgfpathlineto{\pgfqpoint{1.757654in}{1.490685in}}%
\pgfpathlineto{\pgfqpoint{1.788620in}{1.471479in}}%
\pgfpathlineto{\pgfqpoint{1.800018in}{1.464425in}}%
\pgfpathlineto{\pgfqpoint{1.823218in}{1.450267in}}%
\pgfpathlineto{\pgfqpoint{1.843089in}{1.438166in}}%
\pgfpathlineto{\pgfqpoint{1.857817in}{1.429320in}}%
\pgfpathlineto{\pgfqpoint{1.886868in}{1.411906in}}%
\pgfpathlineto{\pgfqpoint{1.892415in}{1.408627in}}%
\pgfpathlineto{\pgfqpoint{1.927014in}{1.388231in}}%
\pgfpathlineto{\pgfqpoint{1.931410in}{1.385647in}}%
\pgfpathlineto{\pgfqpoint{1.961612in}{1.368141in}}%
\pgfpathlineto{\pgfqpoint{1.976739in}{1.359388in}}%
\pgfpathlineto{\pgfqpoint{1.996211in}{1.348274in}}%
\pgfpathlineto{\pgfqpoint{2.022785in}{1.333128in}}%
\pgfpathlineto{\pgfqpoint{2.030809in}{1.328618in}}%
\pgfpathlineto{\pgfqpoint{2.065408in}{1.309213in}}%
\pgfpathlineto{\pgfqpoint{2.069600in}{1.306869in}}%
\pgfpathlineto{\pgfqpoint{2.100006in}{1.290101in}}%
\pgfpathlineto{\pgfqpoint{2.117237in}{1.280609in}}%
\pgfpathlineto{\pgfqpoint{2.134605in}{1.271173in}}%
\pgfpathlineto{\pgfqpoint{2.165598in}{1.254350in}}%
\pgfpathlineto{\pgfqpoint{2.169203in}{1.252420in}}%
\pgfpathlineto{\pgfqpoint{2.203802in}{1.233956in}}%
\pgfpathlineto{\pgfqpoint{2.214810in}{1.228090in}}%
\pgfpathlineto{\pgfqpoint{2.238400in}{1.215693in}}%
\pgfpathlineto{\pgfqpoint{2.264791in}{1.201831in}}%
\pgfpathlineto{\pgfqpoint{2.272999in}{1.197579in}}%
\pgfpathlineto{\pgfqpoint{2.307597in}{1.179693in}}%
\pgfpathlineto{\pgfqpoint{2.315588in}{1.175571in}}%
\pgfpathlineto{\pgfqpoint{2.342196in}{1.162035in}}%
\pgfpathlineto{\pgfqpoint{2.367208in}{1.149312in}}%
\pgfpathlineto{\pgfqpoint{2.376794in}{1.144503in}}%
\pgfpathlineto{\pgfqpoint{2.411393in}{1.127176in}}%
\pgfpathlineto{\pgfqpoint{2.419645in}{1.123052in}}%
\pgfpathlineto{\pgfqpoint{2.445991in}{1.110068in}}%
\pgfpathlineto{\pgfqpoint{2.472917in}{1.096793in}}%
\pgfpathlineto{\pgfqpoint{2.480590in}{1.093062in}}%
\pgfpathlineto{\pgfqpoint{2.515188in}{1.076277in}}%
\pgfpathlineto{\pgfqpoint{2.527042in}{1.070533in}}%
\pgfpathlineto{\pgfqpoint{2.549787in}{1.059665in}}%
\pgfpathlineto{\pgfqpoint{2.581973in}{1.044274in}}%
\pgfpathlineto{\pgfqpoint{2.584385in}{1.043137in}}%
\pgfpathlineto{\pgfqpoint{2.618983in}{1.026876in}}%
\pgfpathlineto{\pgfqpoint{2.637830in}{1.018014in}}%
\pgfpathlineto{\pgfqpoint{2.653582in}{1.010711in}}%
\pgfpathlineto{\pgfqpoint{2.688180in}{0.994674in}}%
\pgfpathlineto{\pgfqpoint{2.694493in}{0.991755in}}%
\pgfpathlineto{\pgfqpoint{2.722779in}{0.978857in}}%
\pgfpathlineto{\pgfqpoint{2.752043in}{0.965495in}}%
\pgfpathlineto{\pgfqpoint{2.757377in}{0.963094in}}%
\pgfpathlineto{\pgfqpoint{2.791976in}{0.947557in}}%
\pgfpathlineto{\pgfqpoint{2.810498in}{0.939236in}}%
\pgfpathlineto{\pgfqpoint{2.826574in}{0.932115in}}%
\pgfpathlineto{\pgfqpoint{2.861173in}{0.916792in}}%
\pgfpathlineto{\pgfqpoint{2.869802in}{0.912976in}}%
\pgfpathlineto{\pgfqpoint{2.895771in}{0.901657in}}%
\pgfpathlineto{\pgfqpoint{2.929977in}{0.886717in}}%
\pgfpathlineto{\pgfqpoint{2.930370in}{0.886548in}}%
\pgfpathlineto{\pgfqpoint{2.964968in}{0.871702in}}%
\pgfpathlineto{\pgfqpoint{2.991119in}{0.860458in}}%
\pgfpathlineto{\pgfqpoint{2.999567in}{0.856876in}}%
\pgfpathlineto{\pgfqpoint{3.034165in}{0.842234in}}%
\pgfpathlineto{\pgfqpoint{3.053144in}{0.834198in}}%
\pgfpathlineto{\pgfqpoint{3.068764in}{0.827678in}}%
\pgfpathlineto{\pgfqpoint{3.103362in}{0.813237in}}%
\pgfpathlineto{\pgfqpoint{3.116067in}{0.807939in}}%
\pgfpathlineto{\pgfqpoint{3.137961in}{0.798937in}}%
\pgfpathlineto{\pgfqpoint{3.172559in}{0.784695in}}%
\pgfpathlineto{\pgfqpoint{3.179899in}{0.781679in}}%
\pgfpathlineto{\pgfqpoint{3.207158in}{0.770639in}}%
\pgfpathlineto{\pgfqpoint{3.241756in}{0.756591in}}%
\pgfpathlineto{\pgfqpoint{3.244651in}{0.755420in}}%
\pgfpathlineto{\pgfqpoint{3.276355in}{0.742768in}}%
\pgfpathlineto{\pgfqpoint{3.310340in}{0.729160in}}%
\pgfpathlineto{\pgfqpoint{3.310953in}{0.728918in}}%
\pgfpathlineto{\pgfqpoint{3.345552in}{0.715310in}}%
\pgfpathlineto{\pgfqpoint{3.376991in}{0.702901in}}%
\pgfpathlineto{\pgfqpoint{3.380150in}{0.701672in}}%
\pgfpathlineto{\pgfqpoint{3.414749in}{0.688250in}}%
\pgfpathlineto{\pgfqpoint{3.444579in}{0.676641in}}%
\pgfpathlineto{\pgfqpoint{3.449347in}{0.674812in}}%
\pgfpathlineto{\pgfqpoint{3.483946in}{0.661575in}}%
\pgfpathlineto{\pgfqpoint{3.513112in}{0.650382in}}%
\pgfpathlineto{\pgfqpoint{3.518544in}{0.648327in}}%
\pgfpathlineto{\pgfqpoint{3.553143in}{0.635271in}}%
\pgfpathlineto{\pgfqpoint{3.582595in}{0.624122in}}%
\pgfpathlineto{\pgfqpoint{3.587741in}{0.622203in}}%
\pgfpathlineto{\pgfqpoint{3.622340in}{0.609326in}}%
\pgfpathlineto{\pgfqpoint{3.653035in}{0.597863in}}%
\pgfpathlineto{\pgfqpoint{3.656938in}{0.596426in}}%
\pgfpathlineto{\pgfqpoint{3.691537in}{0.583726in}}%
\pgfpathlineto{\pgfqpoint{3.724434in}{0.571603in}}%
\pgfusepath{stroke}%
\end{pgfscope}%
\begin{pgfscope}%
\pgfpathrectangle{\pgfqpoint{0.854460in}{0.571603in}}{\pgfqpoint{6.885100in}{5.225635in}}%
\pgfusepath{clip}%
\pgfsetbuttcap%
\pgfsetroundjoin%
\pgfsetlinewidth{1.505625pt}%
\definecolor{currentstroke}{rgb}{0.180629,0.429975,0.557282}%
\pgfsetstrokecolor{currentstroke}%
\pgfsetdash{}{0pt}%
\pgfpathmoveto{\pgfqpoint{7.739560in}{2.997487in}}%
\pgfpathlineto{\pgfqpoint{7.724690in}{3.013734in}}%
\pgfpathlineto{\pgfqpoint{7.704962in}{3.035640in}}%
\pgfpathlineto{\pgfqpoint{7.701031in}{3.039994in}}%
\pgfpathlineto{\pgfqpoint{7.693335in}{3.048646in}}%
\pgfusepath{stroke}%
\end{pgfscope}%
\begin{pgfscope}%
\pgfpathrectangle{\pgfqpoint{0.854460in}{0.571603in}}{\pgfqpoint{6.885100in}{5.225635in}}%
\pgfusepath{clip}%
\pgfsetbuttcap%
\pgfsetroundjoin%
\pgfsetlinewidth{1.505625pt}%
\definecolor{currentstroke}{rgb}{0.180629,0.429975,0.557282}%
\pgfsetstrokecolor{currentstroke}%
\pgfsetdash{}{0pt}%
\pgfpathmoveto{\pgfqpoint{7.446195in}{3.350205in}}%
\pgfpathlineto{\pgfqpoint{7.442504in}{3.355107in}}%
\pgfpathlineto{\pgfqpoint{7.428174in}{3.374533in}}%
\pgfpathlineto{\pgfqpoint{7.423114in}{3.381367in}}%
\pgfpathlineto{\pgfqpoint{7.404026in}{3.407626in}}%
\pgfpathlineto{\pgfqpoint{7.393575in}{3.422291in}}%
\pgfpathlineto{\pgfqpoint{7.385281in}{3.433886in}}%
\pgfpathlineto{\pgfqpoint{7.366861in}{3.460145in}}%
\pgfpathlineto{\pgfqpoint{7.358977in}{3.471612in}}%
\pgfpathlineto{\pgfqpoint{7.348765in}{3.486405in}}%
\pgfpathlineto{\pgfqpoint{7.331008in}{3.512664in}}%
\pgfpathlineto{\pgfqpoint{7.324378in}{3.522671in}}%
\pgfpathlineto{\pgfqpoint{7.313564in}{3.538924in}}%
\pgfpathlineto{\pgfqpoint{7.296464in}{3.565183in}}%
\pgfpathlineto{\pgfqpoint{7.289780in}{3.575667in}}%
\pgfpathlineto{\pgfqpoint{7.279677in}{3.591443in}}%
\pgfpathlineto{\pgfqpoint{7.263227in}{3.617702in}}%
\pgfpathlineto{\pgfqpoint{7.255181in}{3.630835in}}%
\pgfpathlineto{\pgfqpoint{7.247101in}{3.643962in}}%
\pgfpathlineto{\pgfqpoint{7.231298in}{3.670221in}}%
\pgfpathlineto{\pgfqpoint{7.220583in}{3.688449in}}%
\pgfpathlineto{\pgfqpoint{7.215838in}{3.696481in}}%
\pgfpathlineto{\pgfqpoint{7.200676in}{3.722740in}}%
\pgfpathlineto{\pgfqpoint{7.185984in}{3.748830in}}%
\pgfpathlineto{\pgfqpoint{7.185889in}{3.749000in}}%
\pgfpathlineto{\pgfqpoint{7.171362in}{3.775259in}}%
\pgfpathlineto{\pgfqpoint{7.157208in}{3.801519in}}%
\pgfpathlineto{\pgfqpoint{7.151386in}{3.812593in}}%
\pgfpathlineto{\pgfqpoint{7.143360in}{3.827778in}}%
\pgfpathlineto{\pgfqpoint{7.129834in}{3.854037in}}%
\pgfpathlineto{\pgfqpoint{7.116787in}{3.880066in}}%
\pgfpathlineto{\pgfqpoint{7.116671in}{3.880297in}}%
\pgfpathlineto{\pgfqpoint{7.103770in}{3.906556in}}%
\pgfpathlineto{\pgfqpoint{7.091229in}{3.932816in}}%
\pgfpathlineto{\pgfqpoint{7.082189in}{3.952290in}}%
\pgfpathlineto{\pgfqpoint{7.079021in}{3.959075in}}%
\pgfpathlineto{\pgfqpoint{7.067098in}{3.985335in}}%
\pgfpathlineto{\pgfqpoint{7.055529in}{4.011594in}}%
\pgfpathlineto{\pgfqpoint{7.047590in}{4.030166in}}%
\pgfpathlineto{\pgfqpoint{7.044284in}{4.037854in}}%
\pgfpathlineto{\pgfqpoint{7.033327in}{4.064113in}}%
\pgfpathlineto{\pgfqpoint{7.022717in}{4.090373in}}%
\pgfpathlineto{\pgfqpoint{7.012992in}{4.115247in}}%
\pgfpathlineto{\pgfqpoint{7.012447in}{4.116632in}}%
\pgfpathlineto{\pgfqpoint{7.002444in}{4.142892in}}%
\pgfpathlineto{\pgfqpoint{6.992782in}{4.169151in}}%
\pgfpathlineto{\pgfqpoint{6.983458in}{4.195411in}}%
\pgfpathlineto{\pgfqpoint{6.978393in}{4.210202in}}%
\pgfpathlineto{\pgfqpoint{6.974439in}{4.221670in}}%
\pgfpathlineto{\pgfqpoint{6.965715in}{4.247930in}}%
\pgfpathlineto{\pgfqpoint{6.957324in}{4.274189in}}%
\pgfpathlineto{\pgfqpoint{6.949264in}{4.300449in}}%
\pgfpathlineto{\pgfqpoint{6.943795in}{4.319018in}}%
\pgfpathlineto{\pgfqpoint{6.941513in}{4.326708in}}%
\pgfpathlineto{\pgfqpoint{6.934046in}{4.352967in}}%
\pgfpathlineto{\pgfqpoint{6.926905in}{4.379227in}}%
\pgfpathlineto{\pgfqpoint{6.920087in}{4.405486in}}%
\pgfpathlineto{\pgfqpoint{6.913589in}{4.431746in}}%
\pgfpathlineto{\pgfqpoint{6.909196in}{4.450412in}}%
\pgfpathlineto{\pgfqpoint{6.907395in}{4.458005in}}%
\pgfpathlineto{\pgfqpoint{6.901484in}{4.484265in}}%
\pgfpathlineto{\pgfqpoint{6.895889in}{4.510524in}}%
\pgfpathlineto{\pgfqpoint{6.890609in}{4.536784in}}%
\pgfpathlineto{\pgfqpoint{6.885640in}{4.563043in}}%
\pgfpathlineto{\pgfqpoint{6.880981in}{4.589303in}}%
\pgfpathlineto{\pgfqpoint{6.876629in}{4.615562in}}%
\pgfpathlineto{\pgfqpoint{6.874598in}{4.628745in}}%
\pgfpathlineto{\pgfqpoint{6.872566in}{4.641822in}}%
\pgfpathlineto{\pgfqpoint{6.868792in}{4.668081in}}%
\pgfpathlineto{\pgfqpoint{6.865322in}{4.694341in}}%
\pgfpathlineto{\pgfqpoint{6.862154in}{4.720600in}}%
\pgfpathlineto{\pgfqpoint{6.859285in}{4.746860in}}%
\pgfpathlineto{\pgfqpoint{6.856715in}{4.773119in}}%
\pgfpathlineto{\pgfqpoint{6.854440in}{4.799378in}}%
\pgfpathlineto{\pgfqpoint{6.852459in}{4.825638in}}%
\pgfpathlineto{\pgfqpoint{6.850770in}{4.851897in}}%
\pgfpathlineto{\pgfqpoint{6.849371in}{4.878157in}}%
\pgfpathlineto{\pgfqpoint{6.848260in}{4.904416in}}%
\pgfpathlineto{\pgfqpoint{6.847436in}{4.930676in}}%
\pgfpathlineto{\pgfqpoint{6.846897in}{4.956935in}}%
\pgfpathlineto{\pgfqpoint{6.846641in}{4.983195in}}%
\pgfpathlineto{\pgfqpoint{6.846665in}{5.009454in}}%
\pgfpathlineto{\pgfqpoint{6.846969in}{5.035714in}}%
\pgfpathlineto{\pgfqpoint{6.847550in}{5.061973in}}%
\pgfpathlineto{\pgfqpoint{6.848407in}{5.088233in}}%
\pgfpathlineto{\pgfqpoint{6.849538in}{5.114492in}}%
\pgfpathlineto{\pgfqpoint{6.850941in}{5.140752in}}%
\pgfpathlineto{\pgfqpoint{6.852615in}{5.167011in}}%
\pgfpathlineto{\pgfqpoint{6.854557in}{5.193271in}}%
\pgfpathlineto{\pgfqpoint{6.856766in}{5.219530in}}%
\pgfpathlineto{\pgfqpoint{6.859240in}{5.245790in}}%
\pgfpathlineto{\pgfqpoint{6.861977in}{5.272049in}}%
\pgfpathlineto{\pgfqpoint{6.864976in}{5.298308in}}%
\pgfpathlineto{\pgfqpoint{6.868235in}{5.324568in}}%
\pgfpathlineto{\pgfqpoint{6.871751in}{5.350827in}}%
\pgfpathlineto{\pgfqpoint{6.874598in}{5.370663in}}%
\pgfpathlineto{\pgfqpoint{6.875515in}{5.377087in}}%
\pgfpathlineto{\pgfqpoint{6.879507in}{5.403346in}}%
\pgfpathlineto{\pgfqpoint{6.883748in}{5.429606in}}%
\pgfpathlineto{\pgfqpoint{6.888237in}{5.455865in}}%
\pgfpathlineto{\pgfqpoint{6.892972in}{5.482125in}}%
\pgfpathlineto{\pgfqpoint{6.897951in}{5.508384in}}%
\pgfpathlineto{\pgfqpoint{6.903172in}{5.534644in}}%
\pgfpathlineto{\pgfqpoint{6.908633in}{5.560903in}}%
\pgfpathlineto{\pgfqpoint{6.909196in}{5.563511in}}%
\pgfpathlineto{\pgfqpoint{6.914286in}{5.587163in}}%
\pgfpathlineto{\pgfqpoint{6.920167in}{5.613422in}}%
\pgfpathlineto{\pgfqpoint{6.926279in}{5.639682in}}%
\pgfpathlineto{\pgfqpoint{6.932620in}{5.665941in}}%
\pgfpathlineto{\pgfqpoint{6.939187in}{5.692201in}}%
\pgfpathlineto{\pgfqpoint{6.943795in}{5.710049in}}%
\pgfpathlineto{\pgfqpoint{6.945958in}{5.718460in}}%
\pgfpathlineto{\pgfqpoint{6.952907in}{5.744720in}}%
\pgfpathlineto{\pgfqpoint{6.960072in}{5.770979in}}%
\pgfpathlineto{\pgfqpoint{6.967451in}{5.797238in}}%
\pgfusepath{stroke}%
\end{pgfscope}%
\begin{pgfscope}%
\pgfpathrectangle{\pgfqpoint{0.854460in}{0.571603in}}{\pgfqpoint{6.885100in}{5.225635in}}%
\pgfusepath{clip}%
\pgfsetbuttcap%
\pgfsetroundjoin%
\pgfsetlinewidth{1.505625pt}%
\definecolor{currentstroke}{rgb}{0.180629,0.429975,0.557282}%
\pgfsetstrokecolor{currentstroke}%
\pgfsetdash{}{0pt}%
\pgfpathmoveto{\pgfqpoint{0.854460in}{2.097736in}}%
\pgfpathlineto{\pgfqpoint{0.857632in}{2.094653in}}%
\pgfpathlineto{\pgfqpoint{0.885053in}{2.068393in}}%
\pgfpathlineto{\pgfqpoint{0.889059in}{2.064614in}}%
\pgfpathlineto{\pgfqpoint{0.913033in}{2.042134in}}%
\pgfpathlineto{\pgfqpoint{0.923657in}{2.032312in}}%
\pgfpathlineto{\pgfqpoint{0.941543in}{2.015874in}}%
\pgfpathlineto{\pgfqpoint{0.958256in}{2.000730in}}%
\pgfpathlineto{\pgfqpoint{0.970594in}{1.989615in}}%
\pgfpathlineto{\pgfqpoint{0.992854in}{1.969840in}}%
\pgfpathlineto{\pgfqpoint{1.000195in}{1.963355in}}%
\pgfpathlineto{\pgfqpoint{1.027453in}{1.939613in}}%
\pgfpathlineto{\pgfqpoint{1.030359in}{1.937096in}}%
\pgfpathlineto{\pgfqpoint{1.061106in}{1.910836in}}%
\pgfpathlineto{\pgfqpoint{1.062051in}{1.910041in}}%
\pgfpathlineto{\pgfqpoint{1.092465in}{1.884577in}}%
\pgfpathlineto{\pgfqpoint{1.096650in}{1.881122in}}%
\pgfpathlineto{\pgfqpoint{1.124408in}{1.858318in}}%
\pgfpathlineto{\pgfqpoint{1.131248in}{1.852775in}}%
\pgfpathlineto{\pgfqpoint{1.156944in}{1.832058in}}%
\pgfpathlineto{\pgfqpoint{1.165847in}{1.824979in}}%
\pgfpathlineto{\pgfqpoint{1.190083in}{1.805799in}}%
\pgfpathlineto{\pgfqpoint{1.200445in}{1.797710in}}%
\pgfpathlineto{\pgfqpoint{1.223832in}{1.779539in}}%
\pgfpathlineto{\pgfqpoint{1.235044in}{1.770947in}}%
\pgfpathlineto{\pgfqpoint{1.258200in}{1.753280in}}%
\pgfpathlineto{\pgfqpoint{1.269642in}{1.744669in}}%
\pgfpathlineto{\pgfqpoint{1.293196in}{1.727020in}}%
\pgfpathlineto{\pgfqpoint{1.304241in}{1.718857in}}%
\pgfpathlineto{\pgfqpoint{1.328827in}{1.700761in}}%
\pgfpathlineto{\pgfqpoint{1.338839in}{1.693491in}}%
\pgfpathlineto{\pgfqpoint{1.365099in}{1.674501in}}%
\pgfpathlineto{\pgfqpoint{1.373438in}{1.668553in}}%
\pgfpathlineto{\pgfqpoint{1.402021in}{1.648242in}}%
\pgfpathlineto{\pgfqpoint{1.408036in}{1.644025in}}%
\pgfpathlineto{\pgfqpoint{1.439597in}{1.621982in}}%
\pgfpathlineto{\pgfqpoint{1.442635in}{1.619890in}}%
\pgfpathlineto{\pgfqpoint{1.477233in}{1.596139in}}%
\pgfpathlineto{\pgfqpoint{1.477843in}{1.595723in}}%
\pgfpathlineto{\pgfqpoint{1.511832in}{1.572803in}}%
\pgfpathlineto{\pgfqpoint{1.516800in}{1.569463in}}%
\pgfpathlineto{\pgfqpoint{1.546430in}{1.549818in}}%
\pgfpathlineto{\pgfqpoint{1.556438in}{1.543204in}}%
\pgfpathlineto{\pgfqpoint{1.581029in}{1.527171in}}%
\pgfpathlineto{\pgfqpoint{1.596759in}{1.516944in}}%
\pgfpathlineto{\pgfqpoint{1.615627in}{1.504845in}}%
\pgfpathlineto{\pgfqpoint{1.637769in}{1.490685in}}%
\pgfpathlineto{\pgfqpoint{1.650226in}{1.482827in}}%
\pgfpathlineto{\pgfqpoint{1.679471in}{1.464425in}}%
\pgfpathlineto{\pgfqpoint{1.684824in}{1.461102in}}%
\pgfpathlineto{\pgfqpoint{1.719423in}{1.439692in}}%
\pgfpathlineto{\pgfqpoint{1.721897in}{1.438166in}}%
\pgfpathlineto{\pgfqpoint{1.754021in}{1.418626in}}%
\pgfpathlineto{\pgfqpoint{1.765094in}{1.411906in}}%
\pgfpathlineto{\pgfqpoint{1.788620in}{1.397822in}}%
\pgfpathlineto{\pgfqpoint{1.808997in}{1.385647in}}%
\pgfpathlineto{\pgfqpoint{1.823218in}{1.377266in}}%
\pgfpathlineto{\pgfqpoint{1.853610in}{1.359388in}}%
\pgfpathlineto{\pgfqpoint{1.857817in}{1.356946in}}%
\pgfpathlineto{\pgfqpoint{1.892415in}{1.336933in}}%
\pgfpathlineto{\pgfqpoint{1.899010in}{1.333128in}}%
\pgfpathlineto{\pgfqpoint{1.927014in}{1.317191in}}%
\pgfpathlineto{\pgfqpoint{1.945178in}{1.306869in}}%
\pgfpathlineto{\pgfqpoint{1.961612in}{1.297656in}}%
\pgfpathlineto{\pgfqpoint{1.992063in}{1.280609in}}%
\pgfpathlineto{\pgfqpoint{1.996211in}{1.278319in}}%
\pgfpathlineto{\pgfqpoint{2.030809in}{1.259272in}}%
\pgfpathlineto{\pgfqpoint{2.039770in}{1.254350in}}%
\pgfpathlineto{\pgfqpoint{2.065408in}{1.240456in}}%
\pgfpathlineto{\pgfqpoint{2.088247in}{1.228090in}}%
\pgfpathlineto{\pgfqpoint{2.100006in}{1.221810in}}%
\pgfpathlineto{\pgfqpoint{2.134605in}{1.203358in}}%
\pgfpathlineto{\pgfqpoint{2.137479in}{1.201831in}}%
\pgfpathlineto{\pgfqpoint{2.169203in}{1.185197in}}%
\pgfpathlineto{\pgfqpoint{2.187573in}{1.175571in}}%
\pgfpathlineto{\pgfqpoint{2.203802in}{1.167183in}}%
\pgfpathlineto{\pgfqpoint{2.238389in}{1.149312in}}%
\pgfpathlineto{\pgfqpoint{2.238400in}{1.149306in}}%
\pgfpathlineto{\pgfqpoint{2.272999in}{1.131742in}}%
\pgfpathlineto{\pgfqpoint{2.290118in}{1.123052in}}%
\pgfpathlineto{\pgfqpoint{2.307597in}{1.114301in}}%
\pgfpathlineto{\pgfqpoint{2.342196in}{1.096980in}}%
\pgfpathlineto{\pgfqpoint{2.342571in}{1.096793in}}%
\pgfpathlineto{\pgfqpoint{2.376794in}{1.079958in}}%
\pgfpathlineto{\pgfqpoint{2.395948in}{1.070533in}}%
\pgfpathlineto{\pgfqpoint{2.411393in}{1.063038in}}%
\pgfpathlineto{\pgfqpoint{2.445991in}{1.046255in}}%
\pgfpathlineto{\pgfqpoint{2.450086in}{1.044274in}}%
\pgfpathlineto{\pgfqpoint{2.480590in}{1.029723in}}%
\pgfpathlineto{\pgfqpoint{2.505117in}{1.018014in}}%
\pgfpathlineto{\pgfqpoint{2.515188in}{1.013273in}}%
\pgfpathlineto{\pgfqpoint{2.549787in}{0.997010in}}%
\pgfpathlineto{\pgfqpoint{2.560982in}{0.991755in}}%
\pgfpathlineto{\pgfqpoint{2.584385in}{0.980920in}}%
\pgfpathlineto{\pgfqpoint{2.617668in}{0.965495in}}%
\pgfpathlineto{\pgfqpoint{2.618983in}{0.964894in}}%
\pgfpathlineto{\pgfqpoint{2.653582in}{0.949135in}}%
\pgfpathlineto{\pgfqpoint{2.675294in}{0.939236in}}%
\pgfpathlineto{\pgfqpoint{2.688180in}{0.933441in}}%
\pgfpathlineto{\pgfqpoint{2.722779in}{0.917899in}}%
\pgfpathlineto{\pgfqpoint{2.733750in}{0.912976in}}%
\pgfpathlineto{\pgfqpoint{2.737108in}{0.911491in}}%
\pgfusepath{stroke}%
\end{pgfscope}%
\begin{pgfscope}%
\pgfpathrectangle{\pgfqpoint{0.854460in}{0.571603in}}{\pgfqpoint{6.885100in}{5.225635in}}%
\pgfusepath{clip}%
\pgfsetbuttcap%
\pgfsetroundjoin%
\pgfsetlinewidth{1.505625pt}%
\definecolor{currentstroke}{rgb}{0.180629,0.429975,0.557282}%
\pgfsetstrokecolor{currentstroke}%
\pgfsetdash{}{0pt}%
\pgfpathmoveto{\pgfqpoint{3.092909in}{0.759652in}}%
\pgfpathlineto{\pgfqpoint{3.103181in}{0.755420in}}%
\pgfpathlineto{\pgfqpoint{3.103362in}{0.755346in}}%
\pgfpathlineto{\pgfqpoint{3.137961in}{0.741340in}}%
\pgfpathlineto{\pgfqpoint{3.167958in}{0.729160in}}%
\pgfpathlineto{\pgfqpoint{3.172559in}{0.727318in}}%
\pgfpathlineto{\pgfqpoint{3.207158in}{0.713503in}}%
\pgfpathlineto{\pgfqpoint{3.233651in}{0.702901in}}%
\pgfpathlineto{\pgfqpoint{3.241756in}{0.699703in}}%
\pgfpathlineto{\pgfqpoint{3.276355in}{0.686076in}}%
\pgfpathlineto{\pgfqpoint{3.300268in}{0.676641in}}%
\pgfpathlineto{\pgfqpoint{3.310953in}{0.672485in}}%
\pgfpathlineto{\pgfqpoint{3.345552in}{0.659044in}}%
\pgfpathlineto{\pgfqpoint{3.367818in}{0.650382in}}%
\pgfpathlineto{\pgfqpoint{3.380150in}{0.645652in}}%
\pgfpathlineto{\pgfqpoint{3.414749in}{0.632394in}}%
\pgfpathlineto{\pgfqpoint{3.436306in}{0.624122in}}%
\pgfpathlineto{\pgfqpoint{3.449347in}{0.619190in}}%
\pgfpathlineto{\pgfqpoint{3.483946in}{0.606111in}}%
\pgfpathlineto{\pgfqpoint{3.505739in}{0.597863in}}%
\pgfpathlineto{\pgfqpoint{3.518544in}{0.593085in}}%
\pgfpathlineto{\pgfqpoint{3.553143in}{0.580185in}}%
\pgfpathlineto{\pgfqpoint{3.576121in}{0.571603in}}%
\pgfusepath{stroke}%
\end{pgfscope}%
\begin{pgfscope}%
\pgfpathrectangle{\pgfqpoint{0.854460in}{0.571603in}}{\pgfqpoint{6.885100in}{5.225635in}}%
\pgfusepath{clip}%
\pgfsetbuttcap%
\pgfsetroundjoin%
\pgfsetlinewidth{1.505625pt}%
\definecolor{currentstroke}{rgb}{0.172719,0.448791,0.557885}%
\pgfsetstrokecolor{currentstroke}%
\pgfsetdash{}{0pt}%
\pgfpathmoveto{\pgfqpoint{7.739560in}{3.150090in}}%
\pgfpathlineto{\pgfqpoint{7.721746in}{3.171291in}}%
\pgfpathlineto{\pgfqpoint{7.704962in}{3.191629in}}%
\pgfpathlineto{\pgfqpoint{7.700061in}{3.197551in}}%
\pgfpathlineto{\pgfqpoint{7.678683in}{3.223810in}}%
\pgfpathlineto{\pgfqpoint{7.670363in}{3.234210in}}%
\pgfpathlineto{\pgfqpoint{7.657638in}{3.250070in}}%
\pgfpathlineto{\pgfqpoint{7.636958in}{3.276329in}}%
\pgfpathlineto{\pgfqpoint{7.635765in}{3.277870in}}%
\pgfpathlineto{\pgfqpoint{7.616557in}{3.302589in}}%
\pgfpathlineto{\pgfqpoint{7.601166in}{3.322787in}}%
\pgfpathlineto{\pgfqpoint{7.596532in}{3.328848in}}%
\pgfpathlineto{\pgfqpoint{7.576811in}{3.355107in}}%
\pgfpathlineto{\pgfqpoint{7.566568in}{3.369013in}}%
\pgfpathlineto{\pgfqpoint{7.557435in}{3.381367in}}%
\pgfpathlineto{\pgfqpoint{7.538395in}{3.407626in}}%
\pgfpathlineto{\pgfqpoint{7.531969in}{3.416660in}}%
\pgfpathlineto{\pgfqpoint{7.519671in}{3.433886in}}%
\pgfpathlineto{\pgfqpoint{7.501305in}{3.460145in}}%
\pgfpathlineto{\pgfqpoint{7.497371in}{3.465881in}}%
\pgfpathlineto{\pgfqpoint{7.483237in}{3.486405in}}%
\pgfpathlineto{\pgfqpoint{7.465539in}{3.512664in}}%
\pgfpathlineto{\pgfqpoint{7.462772in}{3.516852in}}%
\pgfpathlineto{\pgfqpoint{7.448129in}{3.538924in}}%
\pgfpathlineto{\pgfqpoint{7.431094in}{3.565183in}}%
\pgfpathlineto{\pgfqpoint{7.428174in}{3.569778in}}%
\pgfpathlineto{\pgfqpoint{7.414347in}{3.591443in}}%
\pgfpathlineto{\pgfqpoint{7.405028in}{3.606384in}}%
\pgfusepath{stroke}%
\end{pgfscope}%
\begin{pgfscope}%
\pgfpathrectangle{\pgfqpoint{0.854460in}{0.571603in}}{\pgfqpoint{6.885100in}{5.225635in}}%
\pgfusepath{clip}%
\pgfsetbuttcap%
\pgfsetroundjoin%
\pgfsetlinewidth{1.505625pt}%
\definecolor{currentstroke}{rgb}{0.172719,0.448791,0.557885}%
\pgfsetstrokecolor{currentstroke}%
\pgfsetdash{}{0pt}%
\pgfpathmoveto{\pgfqpoint{7.218888in}{3.950244in}}%
\pgfpathlineto{\pgfqpoint{7.214799in}{3.959075in}}%
\pgfpathlineto{\pgfqpoint{7.202988in}{3.985335in}}%
\pgfpathlineto{\pgfqpoint{7.191533in}{4.011594in}}%
\pgfpathlineto{\pgfqpoint{7.185984in}{4.024707in}}%
\pgfpathlineto{\pgfqpoint{7.180388in}{4.037854in}}%
\pgfpathlineto{\pgfqpoint{7.169552in}{4.064113in}}%
\pgfpathlineto{\pgfqpoint{7.159065in}{4.090373in}}%
\pgfpathlineto{\pgfqpoint{7.151386in}{4.110251in}}%
\pgfpathlineto{\pgfqpoint{7.148905in}{4.116632in}}%
\pgfpathlineto{\pgfqpoint{7.139033in}{4.142892in}}%
\pgfpathlineto{\pgfqpoint{7.129504in}{4.169151in}}%
\pgfpathlineto{\pgfqpoint{7.120316in}{4.195411in}}%
\pgfpathlineto{\pgfqpoint{7.116787in}{4.205874in}}%
\pgfpathlineto{\pgfqpoint{7.111425in}{4.221670in}}%
\pgfpathlineto{\pgfqpoint{7.102844in}{4.247930in}}%
\pgfpathlineto{\pgfqpoint{7.094599in}{4.274189in}}%
\pgfpathlineto{\pgfqpoint{7.086688in}{4.300449in}}%
\pgfpathlineto{\pgfqpoint{7.082189in}{4.316027in}}%
\pgfpathlineto{\pgfqpoint{7.079082in}{4.326708in}}%
\pgfpathlineto{\pgfqpoint{7.071773in}{4.352967in}}%
\pgfpathlineto{\pgfqpoint{7.064792in}{4.379227in}}%
\pgfpathlineto{\pgfqpoint{7.058137in}{4.405486in}}%
\pgfpathlineto{\pgfqpoint{7.051806in}{4.431746in}}%
\pgfpathlineto{\pgfqpoint{7.047590in}{4.450163in}}%
\pgfpathlineto{\pgfqpoint{7.045782in}{4.458005in}}%
\pgfpathlineto{\pgfqpoint{7.040047in}{4.484265in}}%
\pgfpathlineto{\pgfqpoint{7.034631in}{4.510524in}}%
\pgfpathlineto{\pgfqpoint{7.029533in}{4.536784in}}%
\pgfpathlineto{\pgfqpoint{7.024751in}{4.563043in}}%
\pgfpathlineto{\pgfqpoint{7.020282in}{4.589303in}}%
\pgfpathlineto{\pgfqpoint{7.016124in}{4.615562in}}%
\pgfpathlineto{\pgfqpoint{7.012992in}{4.636939in}}%
\pgfpathlineto{\pgfqpoint{7.012270in}{4.641822in}}%
\pgfpathlineto{\pgfqpoint{7.008702in}{4.668081in}}%
\pgfpathlineto{\pgfqpoint{7.005441in}{4.694341in}}%
\pgfpathlineto{\pgfqpoint{7.002486in}{4.720600in}}%
\pgfpathlineto{\pgfqpoint{6.999836in}{4.746860in}}%
\pgfpathlineto{\pgfqpoint{6.997487in}{4.773119in}}%
\pgfpathlineto{\pgfqpoint{6.995440in}{4.799378in}}%
\pgfpathlineto{\pgfqpoint{6.993691in}{4.825638in}}%
\pgfpathlineto{\pgfqpoint{6.992240in}{4.851897in}}%
\pgfpathlineto{\pgfqpoint{6.991083in}{4.878157in}}%
\pgfpathlineto{\pgfqpoint{6.990221in}{4.904416in}}%
\pgfpathlineto{\pgfqpoint{6.989650in}{4.930676in}}%
\pgfpathlineto{\pgfqpoint{6.989370in}{4.956935in}}%
\pgfpathlineto{\pgfqpoint{6.989378in}{4.983195in}}%
\pgfpathlineto{\pgfqpoint{6.989674in}{5.009454in}}%
\pgfpathlineto{\pgfqpoint{6.990255in}{5.035714in}}%
\pgfpathlineto{\pgfqpoint{6.991120in}{5.061973in}}%
\pgfpathlineto{\pgfqpoint{6.992267in}{5.088233in}}%
\pgfpathlineto{\pgfqpoint{6.993694in}{5.114492in}}%
\pgfpathlineto{\pgfqpoint{6.995401in}{5.140752in}}%
\pgfpathlineto{\pgfqpoint{6.997385in}{5.167011in}}%
\pgfpathlineto{\pgfqpoint{6.999645in}{5.193271in}}%
\pgfpathlineto{\pgfqpoint{7.002180in}{5.219530in}}%
\pgfpathlineto{\pgfqpoint{7.004987in}{5.245790in}}%
\pgfpathlineto{\pgfqpoint{7.008066in}{5.272049in}}%
\pgfpathlineto{\pgfqpoint{7.011414in}{5.298308in}}%
\pgfpathlineto{\pgfqpoint{7.012992in}{5.309784in}}%
\pgfpathlineto{\pgfqpoint{7.015014in}{5.324568in}}%
\pgfpathlineto{\pgfqpoint{7.018865in}{5.350827in}}%
\pgfpathlineto{\pgfqpoint{7.022978in}{5.377087in}}%
\pgfpathlineto{\pgfqpoint{7.027352in}{5.403346in}}%
\pgfpathlineto{\pgfqpoint{7.031986in}{5.429606in}}%
\pgfpathlineto{\pgfqpoint{7.036877in}{5.455865in}}%
\pgfpathlineto{\pgfqpoint{7.042025in}{5.482125in}}%
\pgfpathlineto{\pgfqpoint{7.047426in}{5.508384in}}%
\pgfpathlineto{\pgfqpoint{7.047590in}{5.509150in}}%
\pgfpathlineto{\pgfqpoint{7.053034in}{5.534644in}}%
\pgfpathlineto{\pgfqpoint{7.058890in}{5.560903in}}%
\pgfpathlineto{\pgfqpoint{7.064991in}{5.587163in}}%
\pgfpathlineto{\pgfqpoint{7.071338in}{5.613422in}}%
\pgfpathlineto{\pgfqpoint{7.077927in}{5.639682in}}%
\pgfpathlineto{\pgfqpoint{7.082189in}{5.656093in}}%
\pgfpathlineto{\pgfqpoint{7.084736in}{5.665941in}}%
\pgfpathlineto{\pgfqpoint{7.091746in}{5.692201in}}%
\pgfpathlineto{\pgfqpoint{7.098989in}{5.718460in}}%
\pgfpathlineto{\pgfqpoint{7.106466in}{5.744720in}}%
\pgfpathlineto{\pgfqpoint{7.114173in}{5.770979in}}%
\pgfpathlineto{\pgfqpoint{7.116787in}{5.779662in}}%
\pgfpathlineto{\pgfqpoint{7.122062in}{5.797238in}}%
\pgfusepath{stroke}%
\end{pgfscope}%
\begin{pgfscope}%
\pgfpathrectangle{\pgfqpoint{0.854460in}{0.571603in}}{\pgfqpoint{6.885100in}{5.225635in}}%
\pgfusepath{clip}%
\pgfsetbuttcap%
\pgfsetroundjoin%
\pgfsetlinewidth{1.505625pt}%
\definecolor{currentstroke}{rgb}{0.172719,0.448791,0.557885}%
\pgfsetstrokecolor{currentstroke}%
\pgfsetdash{}{0pt}%
\pgfpathmoveto{\pgfqpoint{0.854460in}{1.999034in}}%
\pgfpathlineto{\pgfqpoint{0.864777in}{1.989615in}}%
\pgfpathlineto{\pgfqpoint{0.889059in}{1.967751in}}%
\pgfpathlineto{\pgfqpoint{0.893969in}{1.963355in}}%
\pgfpathlineto{\pgfqpoint{0.923657in}{1.937145in}}%
\pgfpathlineto{\pgfqpoint{0.923714in}{1.937096in}}%
\pgfpathlineto{\pgfqpoint{0.954074in}{1.910836in}}%
\pgfpathlineto{\pgfqpoint{0.958256in}{1.907269in}}%
\pgfpathlineto{\pgfqpoint{0.984999in}{1.884577in}}%
\pgfpathlineto{\pgfqpoint{0.992854in}{1.878003in}}%
\pgfpathlineto{\pgfqpoint{1.016499in}{1.858318in}}%
\pgfpathlineto{\pgfqpoint{1.027453in}{1.849323in}}%
\pgfpathlineto{\pgfqpoint{1.048584in}{1.832058in}}%
\pgfpathlineto{\pgfqpoint{1.062051in}{1.821205in}}%
\pgfpathlineto{\pgfqpoint{1.081262in}{1.805799in}}%
\pgfpathlineto{\pgfqpoint{1.096650in}{1.793626in}}%
\pgfpathlineto{\pgfqpoint{1.114542in}{1.779539in}}%
\pgfpathlineto{\pgfqpoint{1.131248in}{1.766565in}}%
\pgfpathlineto{\pgfqpoint{1.148433in}{1.753280in}}%
\pgfpathlineto{\pgfqpoint{1.165847in}{1.740000in}}%
\pgfpathlineto{\pgfqpoint{1.182943in}{1.727020in}}%
\pgfpathlineto{\pgfqpoint{1.200445in}{1.713912in}}%
\pgfpathlineto{\pgfqpoint{1.218080in}{1.700761in}}%
\pgfpathlineto{\pgfqpoint{1.235044in}{1.688281in}}%
\pgfpathlineto{\pgfqpoint{1.253852in}{1.674501in}}%
\pgfpathlineto{\pgfqpoint{1.269642in}{1.663088in}}%
\pgfpathlineto{\pgfqpoint{1.290265in}{1.648242in}}%
\pgfpathlineto{\pgfqpoint{1.304241in}{1.638316in}}%
\pgfpathlineto{\pgfqpoint{1.327327in}{1.621982in}}%
\pgfpathlineto{\pgfqpoint{1.338839in}{1.613946in}}%
\pgfpathlineto{\pgfqpoint{1.365043in}{1.595723in}}%
\pgfpathlineto{\pgfqpoint{1.373438in}{1.589963in}}%
\pgfpathlineto{\pgfqpoint{1.403419in}{1.569463in}}%
\pgfpathlineto{\pgfqpoint{1.408036in}{1.566349in}}%
\pgfpathlineto{\pgfqpoint{1.442461in}{1.543204in}}%
\pgfpathlineto{\pgfqpoint{1.442635in}{1.543089in}}%
\pgfpathlineto{\pgfqpoint{1.477233in}{1.520237in}}%
\pgfpathlineto{\pgfqpoint{1.482234in}{1.516944in}}%
\pgfpathlineto{\pgfqpoint{1.511832in}{1.497719in}}%
\pgfpathlineto{\pgfqpoint{1.522692in}{1.490685in}}%
\pgfpathlineto{\pgfqpoint{1.546430in}{1.475517in}}%
\pgfpathlineto{\pgfqpoint{1.563838in}{1.464425in}}%
\pgfpathlineto{\pgfqpoint{1.581029in}{1.453618in}}%
\pgfpathlineto{\pgfqpoint{1.605674in}{1.438166in}}%
\pgfpathlineto{\pgfqpoint{1.615627in}{1.432009in}}%
\pgfpathlineto{\pgfqpoint{1.648204in}{1.411906in}}%
\pgfpathlineto{\pgfqpoint{1.650226in}{1.410675in}}%
\pgfpathlineto{\pgfqpoint{1.684824in}{1.389690in}}%
\pgfpathlineto{\pgfqpoint{1.691508in}{1.385647in}}%
\pgfpathlineto{\pgfqpoint{1.719423in}{1.368988in}}%
\pgfpathlineto{\pgfqpoint{1.735543in}{1.359388in}}%
\pgfpathlineto{\pgfqpoint{1.754021in}{1.348530in}}%
\pgfpathlineto{\pgfqpoint{1.780285in}{1.333128in}}%
\pgfpathlineto{\pgfqpoint{1.788620in}{1.328305in}}%
\pgfpathlineto{\pgfqpoint{1.823218in}{1.308333in}}%
\pgfpathlineto{\pgfqpoint{1.825764in}{1.306869in}}%
\pgfpathlineto{\pgfqpoint{1.857817in}{1.288679in}}%
\pgfpathlineto{\pgfqpoint{1.872058in}{1.280609in}}%
\pgfpathlineto{\pgfqpoint{1.892415in}{1.269229in}}%
\pgfpathlineto{\pgfqpoint{1.919067in}{1.254350in}}%
\pgfpathlineto{\pgfqpoint{1.927014in}{1.249973in}}%
\pgfpathlineto{\pgfqpoint{1.961612in}{1.230961in}}%
\pgfpathlineto{\pgfqpoint{1.966852in}{1.228090in}}%
\pgfpathlineto{\pgfqpoint{1.996211in}{1.212220in}}%
\pgfpathlineto{\pgfqpoint{2.015450in}{1.201831in}}%
\pgfpathlineto{\pgfqpoint{2.030809in}{1.193648in}}%
\pgfpathlineto{\pgfqpoint{2.064766in}{1.175571in}}%
\pgfpathlineto{\pgfqpoint{2.065408in}{1.175234in}}%
\pgfpathlineto{\pgfqpoint{2.100006in}{1.157134in}}%
\pgfpathlineto{\pgfqpoint{2.114968in}{1.149312in}}%
\pgfpathlineto{\pgfqpoint{2.134605in}{1.139184in}}%
\pgfpathlineto{\pgfqpoint{2.165898in}{1.123052in}}%
\pgfpathlineto{\pgfqpoint{2.169203in}{1.121371in}}%
\pgfpathlineto{\pgfqpoint{2.203802in}{1.103832in}}%
\pgfpathlineto{\pgfqpoint{2.217699in}{1.096793in}}%
\pgfpathlineto{\pgfqpoint{2.238400in}{1.086450in}}%
\pgfpathlineto{\pgfqpoint{2.258888in}{1.076213in}}%
\pgfusepath{stroke}%
\end{pgfscope}%
\begin{pgfscope}%
\pgfpathrectangle{\pgfqpoint{0.854460in}{0.571603in}}{\pgfqpoint{6.885100in}{5.225635in}}%
\pgfusepath{clip}%
\pgfsetbuttcap%
\pgfsetroundjoin%
\pgfsetlinewidth{1.505625pt}%
\definecolor{currentstroke}{rgb}{0.172719,0.448791,0.557885}%
\pgfsetstrokecolor{currentstroke}%
\pgfsetdash{}{0pt}%
\pgfpathmoveto{\pgfqpoint{2.608710in}{0.910607in}}%
\pgfpathlineto{\pgfqpoint{2.618983in}{0.905994in}}%
\pgfpathlineto{\pgfqpoint{2.653582in}{0.890465in}}%
\pgfpathlineto{\pgfqpoint{2.661947in}{0.886717in}}%
\pgfpathlineto{\pgfqpoint{2.688180in}{0.875125in}}%
\pgfpathlineto{\pgfqpoint{2.721320in}{0.860458in}}%
\pgfpathlineto{\pgfqpoint{2.722779in}{0.859821in}}%
\pgfpathlineto{\pgfqpoint{2.757377in}{0.844768in}}%
\pgfpathlineto{\pgfqpoint{2.781635in}{0.834198in}}%
\pgfpathlineto{\pgfqpoint{2.791976in}{0.829754in}}%
\pgfpathlineto{\pgfqpoint{2.826574in}{0.814907in}}%
\pgfpathlineto{\pgfqpoint{2.842814in}{0.807939in}}%
\pgfpathlineto{\pgfqpoint{2.861173in}{0.800170in}}%
\pgfpathlineto{\pgfqpoint{2.895771in}{0.785525in}}%
\pgfpathlineto{\pgfqpoint{2.904871in}{0.781679in}}%
\pgfpathlineto{\pgfqpoint{2.930370in}{0.771052in}}%
\pgfpathlineto{\pgfqpoint{2.964968in}{0.756605in}}%
\pgfpathlineto{\pgfqpoint{2.967817in}{0.755420in}}%
\pgfpathlineto{\pgfqpoint{2.999567in}{0.742384in}}%
\pgfpathlineto{\pgfqpoint{3.031690in}{0.729160in}}%
\pgfpathlineto{\pgfqpoint{3.034165in}{0.728155in}}%
\pgfpathlineto{\pgfqpoint{3.068764in}{0.714152in}}%
\pgfpathlineto{\pgfqpoint{3.096495in}{0.702901in}}%
\pgfpathlineto{\pgfqpoint{3.103362in}{0.700153in}}%
\pgfpathlineto{\pgfqpoint{3.137961in}{0.686340in}}%
\pgfpathlineto{\pgfqpoint{3.162211in}{0.676641in}}%
\pgfpathlineto{\pgfqpoint{3.172559in}{0.672560in}}%
\pgfpathlineto{\pgfqpoint{3.207158in}{0.658934in}}%
\pgfpathlineto{\pgfqpoint{3.228846in}{0.650382in}}%
\pgfpathlineto{\pgfqpoint{3.241756in}{0.645362in}}%
\pgfpathlineto{\pgfqpoint{3.276355in}{0.631920in}}%
\pgfpathlineto{\pgfqpoint{3.296408in}{0.624122in}}%
\pgfpathlineto{\pgfqpoint{3.310953in}{0.618546in}}%
\pgfpathlineto{\pgfqpoint{3.345552in}{0.605285in}}%
\pgfpathlineto{\pgfqpoint{3.364903in}{0.597863in}}%
\pgfpathlineto{\pgfqpoint{3.380150in}{0.592097in}}%
\pgfpathlineto{\pgfqpoint{3.414749in}{0.579015in}}%
\pgfpathlineto{\pgfqpoint{3.434337in}{0.571603in}}%
\pgfusepath{stroke}%
\end{pgfscope}%
\begin{pgfscope}%
\pgfpathrectangle{\pgfqpoint{0.854460in}{0.571603in}}{\pgfqpoint{6.885100in}{5.225635in}}%
\pgfusepath{clip}%
\pgfsetbuttcap%
\pgfsetroundjoin%
\pgfsetlinewidth{1.505625pt}%
\definecolor{currentstroke}{rgb}{0.163625,0.471133,0.558148}%
\pgfsetstrokecolor{currentstroke}%
\pgfsetdash{}{0pt}%
\pgfpathmoveto{\pgfqpoint{7.739560in}{3.312425in}}%
\pgfpathlineto{\pgfqpoint{7.727017in}{3.328848in}}%
\pgfpathlineto{\pgfqpoint{7.707349in}{3.355107in}}%
\pgfpathlineto{\pgfqpoint{7.704962in}{3.358353in}}%
\pgfpathlineto{\pgfqpoint{7.687972in}{3.381367in}}%
\pgfpathlineto{\pgfqpoint{7.670363in}{3.405720in}}%
\pgfpathlineto{\pgfqpoint{7.668980in}{3.407626in}}%
\pgfpathlineto{\pgfqpoint{7.650270in}{3.433886in}}%
\pgfpathlineto{\pgfqpoint{7.635765in}{3.454681in}}%
\pgfpathlineto{\pgfqpoint{7.631939in}{3.460145in}}%
\pgfpathlineto{\pgfqpoint{7.613908in}{3.486405in}}%
\pgfpathlineto{\pgfqpoint{7.601166in}{3.505364in}}%
\pgfpathlineto{\pgfqpoint{7.596241in}{3.512664in}}%
\pgfpathlineto{\pgfqpoint{7.578882in}{3.538924in}}%
\pgfpathlineto{\pgfqpoint{7.566568in}{3.557969in}}%
\pgfpathlineto{\pgfqpoint{7.561884in}{3.565183in}}%
\pgfpathlineto{\pgfqpoint{7.545192in}{3.591443in}}%
\pgfpathlineto{\pgfqpoint{7.531969in}{3.612729in}}%
\pgfpathlineto{\pgfqpoint{7.528867in}{3.617702in}}%
\pgfpathlineto{\pgfqpoint{7.512836in}{3.643962in}}%
\pgfpathlineto{\pgfqpoint{7.497371in}{3.669917in}}%
\pgfpathlineto{\pgfqpoint{7.497189in}{3.670221in}}%
\pgfpathlineto{\pgfqpoint{7.481814in}{3.696481in}}%
\pgfpathlineto{\pgfqpoint{7.466821in}{3.722740in}}%
\pgfpathlineto{\pgfqpoint{7.462772in}{3.730001in}}%
\pgfpathlineto{\pgfqpoint{7.452128in}{3.749000in}}%
\pgfpathlineto{\pgfqpoint{7.437783in}{3.775259in}}%
\pgfpathlineto{\pgfqpoint{7.428174in}{3.793306in}}%
\pgfpathlineto{\pgfqpoint{7.423780in}{3.801519in}}%
\pgfpathlineto{\pgfqpoint{7.410079in}{3.827778in}}%
\pgfpathlineto{\pgfqpoint{7.396748in}{3.854037in}}%
\pgfpathlineto{\pgfqpoint{7.393575in}{3.860453in}}%
\pgfpathlineto{\pgfqpoint{7.383711in}{3.880297in}}%
\pgfpathlineto{\pgfqpoint{7.371018in}{3.906556in}}%
\pgfpathlineto{\pgfqpoint{7.358977in}{3.932198in}}%
\pgfpathlineto{\pgfqpoint{7.358685in}{3.932816in}}%
\pgfpathlineto{\pgfqpoint{7.346626in}{3.959075in}}%
\pgfpathlineto{\pgfqpoint{7.334926in}{3.985335in}}%
\pgfpathlineto{\pgfqpoint{7.324378in}{4.009750in}}%
\pgfpathlineto{\pgfqpoint{7.323577in}{4.011594in}}%
\pgfpathlineto{\pgfqpoint{7.312506in}{4.037854in}}%
\pgfpathlineto{\pgfqpoint{7.301789in}{4.064113in}}%
\pgfpathlineto{\pgfqpoint{7.291422in}{4.090373in}}%
\pgfpathlineto{\pgfqpoint{7.289780in}{4.094672in}}%
\pgfpathlineto{\pgfqpoint{7.281341in}{4.116632in}}%
\pgfpathlineto{\pgfqpoint{7.271595in}{4.142892in}}%
\pgfpathlineto{\pgfqpoint{7.262195in}{4.169151in}}%
\pgfpathlineto{\pgfqpoint{7.255181in}{4.189478in}}%
\pgfpathlineto{\pgfqpoint{7.253121in}{4.195411in}}%
\pgfpathlineto{\pgfqpoint{7.244338in}{4.221670in}}%
\pgfpathlineto{\pgfqpoint{7.235895in}{4.247930in}}%
\pgfpathlineto{\pgfqpoint{7.229631in}{4.268222in}}%
\pgfusepath{stroke}%
\end{pgfscope}%
\begin{pgfscope}%
\pgfpathrectangle{\pgfqpoint{0.854460in}{0.571603in}}{\pgfqpoint{6.885100in}{5.225635in}}%
\pgfusepath{clip}%
\pgfsetbuttcap%
\pgfsetroundjoin%
\pgfsetlinewidth{1.505625pt}%
\definecolor{currentstroke}{rgb}{0.163625,0.471133,0.558148}%
\pgfsetstrokecolor{currentstroke}%
\pgfsetdash{}{0pt}%
\pgfpathmoveto{\pgfqpoint{7.146407in}{4.651415in}}%
\pgfpathlineto{\pgfqpoint{7.144262in}{4.668081in}}%
\pgfpathlineto{\pgfqpoint{7.141194in}{4.694341in}}%
\pgfpathlineto{\pgfqpoint{7.138436in}{4.720600in}}%
\pgfpathlineto{\pgfqpoint{7.135986in}{4.746860in}}%
\pgfpathlineto{\pgfqpoint{7.133842in}{4.773119in}}%
\pgfpathlineto{\pgfqpoint{7.132003in}{4.799378in}}%
\pgfpathlineto{\pgfqpoint{7.130466in}{4.825638in}}%
\pgfpathlineto{\pgfqpoint{7.129232in}{4.851897in}}%
\pgfpathlineto{\pgfqpoint{7.128297in}{4.878157in}}%
\pgfpathlineto{\pgfqpoint{7.127660in}{4.904416in}}%
\pgfpathlineto{\pgfqpoint{7.127320in}{4.930676in}}%
\pgfpathlineto{\pgfqpoint{7.127275in}{4.956935in}}%
\pgfpathlineto{\pgfqpoint{7.127524in}{4.983195in}}%
\pgfpathlineto{\pgfqpoint{7.128065in}{5.009454in}}%
\pgfpathlineto{\pgfqpoint{7.128897in}{5.035714in}}%
\pgfpathlineto{\pgfqpoint{7.130018in}{5.061973in}}%
\pgfpathlineto{\pgfqpoint{7.131427in}{5.088233in}}%
\pgfpathlineto{\pgfqpoint{7.133122in}{5.114492in}}%
\pgfpathlineto{\pgfqpoint{7.135102in}{5.140752in}}%
\pgfpathlineto{\pgfqpoint{7.137366in}{5.167011in}}%
\pgfpathlineto{\pgfqpoint{7.139912in}{5.193271in}}%
\pgfpathlineto{\pgfqpoint{7.142739in}{5.219530in}}%
\pgfpathlineto{\pgfqpoint{7.145846in}{5.245790in}}%
\pgfpathlineto{\pgfqpoint{7.149230in}{5.272049in}}%
\pgfpathlineto{\pgfqpoint{7.151386in}{5.287527in}}%
\pgfpathlineto{\pgfqpoint{7.152880in}{5.298308in}}%
\pgfpathlineto{\pgfqpoint{7.156785in}{5.324568in}}%
\pgfpathlineto{\pgfqpoint{7.160962in}{5.350827in}}%
\pgfpathlineto{\pgfqpoint{7.165409in}{5.377087in}}%
\pgfpathlineto{\pgfqpoint{7.170125in}{5.403346in}}%
\pgfpathlineto{\pgfqpoint{7.175109in}{5.429606in}}%
\pgfpathlineto{\pgfqpoint{7.180359in}{5.455865in}}%
\pgfpathlineto{\pgfqpoint{7.185873in}{5.482125in}}%
\pgfpathlineto{\pgfqpoint{7.185984in}{5.482633in}}%
\pgfpathlineto{\pgfqpoint{7.191605in}{5.508384in}}%
\pgfpathlineto{\pgfqpoint{7.197595in}{5.534644in}}%
\pgfpathlineto{\pgfqpoint{7.203843in}{5.560903in}}%
\pgfpathlineto{\pgfqpoint{7.210348in}{5.587163in}}%
\pgfpathlineto{\pgfqpoint{7.217107in}{5.613422in}}%
\pgfpathlineto{\pgfqpoint{7.220583in}{5.626465in}}%
\pgfpathlineto{\pgfqpoint{7.224091in}{5.639682in}}%
\pgfpathlineto{\pgfqpoint{7.231296in}{5.665941in}}%
\pgfpathlineto{\pgfqpoint{7.238749in}{5.692201in}}%
\pgfpathlineto{\pgfqpoint{7.246448in}{5.718460in}}%
\pgfpathlineto{\pgfqpoint{7.254391in}{5.744720in}}%
\pgfpathlineto{\pgfqpoint{7.255181in}{5.747268in}}%
\pgfpathlineto{\pgfqpoint{7.262515in}{5.770979in}}%
\pgfpathlineto{\pgfqpoint{7.270871in}{5.797238in}}%
\pgfusepath{stroke}%
\end{pgfscope}%
\begin{pgfscope}%
\pgfpathrectangle{\pgfqpoint{0.854460in}{0.571603in}}{\pgfqpoint{6.885100in}{5.225635in}}%
\pgfusepath{clip}%
\pgfsetbuttcap%
\pgfsetroundjoin%
\pgfsetlinewidth{1.505625pt}%
\definecolor{currentstroke}{rgb}{0.163625,0.471133,0.558148}%
\pgfsetstrokecolor{currentstroke}%
\pgfsetdash{}{0pt}%
\pgfpathmoveto{\pgfqpoint{0.854460in}{1.907901in}}%
\pgfpathlineto{\pgfqpoint{0.881581in}{1.884577in}}%
\pgfpathlineto{\pgfqpoint{0.889059in}{1.878234in}}%
\pgfpathlineto{\pgfqpoint{0.912662in}{1.858318in}}%
\pgfpathlineto{\pgfqpoint{0.923657in}{1.849165in}}%
\pgfpathlineto{\pgfqpoint{0.944317in}{1.832058in}}%
\pgfpathlineto{\pgfqpoint{0.958256in}{1.820672in}}%
\pgfpathlineto{\pgfqpoint{0.976556in}{1.805799in}}%
\pgfpathlineto{\pgfqpoint{0.992854in}{1.792731in}}%
\pgfpathlineto{\pgfqpoint{1.009388in}{1.779539in}}%
\pgfpathlineto{\pgfqpoint{1.027453in}{1.765320in}}%
\pgfpathlineto{\pgfqpoint{1.042822in}{1.753280in}}%
\pgfpathlineto{\pgfqpoint{1.062051in}{1.738418in}}%
\pgfpathlineto{\pgfqpoint{1.076866in}{1.727020in}}%
\pgfpathlineto{\pgfqpoint{1.096650in}{1.712004in}}%
\pgfpathlineto{\pgfqpoint{1.111529in}{1.700761in}}%
\pgfpathlineto{\pgfqpoint{1.131248in}{1.686060in}}%
\pgfpathlineto{\pgfqpoint{1.146818in}{1.674501in}}%
\pgfpathlineto{\pgfqpoint{1.165847in}{1.660565in}}%
\pgfpathlineto{\pgfqpoint{1.182741in}{1.648242in}}%
\pgfpathlineto{\pgfqpoint{1.200445in}{1.635501in}}%
\pgfpathlineto{\pgfqpoint{1.219305in}{1.621982in}}%
\pgfpathlineto{\pgfqpoint{1.235044in}{1.610851in}}%
\pgfpathlineto{\pgfqpoint{1.256516in}{1.595723in}}%
\pgfpathlineto{\pgfqpoint{1.269642in}{1.586598in}}%
\pgfpathlineto{\pgfqpoint{1.294380in}{1.569463in}}%
\pgfpathlineto{\pgfqpoint{1.304241in}{1.562724in}}%
\pgfpathlineto{\pgfqpoint{1.332903in}{1.543204in}}%
\pgfpathlineto{\pgfqpoint{1.338839in}{1.539215in}}%
\pgfpathlineto{\pgfqpoint{1.372092in}{1.516944in}}%
\pgfpathlineto{\pgfqpoint{1.373438in}{1.516055in}}%
\pgfpathlineto{\pgfqpoint{1.408036in}{1.493282in}}%
\pgfpathlineto{\pgfqpoint{1.411996in}{1.490685in}}%
\pgfpathlineto{\pgfqpoint{1.442635in}{1.470855in}}%
\pgfpathlineto{\pgfqpoint{1.452598in}{1.464425in}}%
\pgfpathlineto{\pgfqpoint{1.477233in}{1.448740in}}%
\pgfpathlineto{\pgfqpoint{1.493886in}{1.438166in}}%
\pgfpathlineto{\pgfqpoint{1.511832in}{1.426923in}}%
\pgfpathlineto{\pgfqpoint{1.535863in}{1.411906in}}%
\pgfpathlineto{\pgfqpoint{1.546430in}{1.405391in}}%
\pgfpathlineto{\pgfqpoint{1.578532in}{1.385647in}}%
\pgfpathlineto{\pgfqpoint{1.581029in}{1.384132in}}%
\pgfpathlineto{\pgfqpoint{1.615627in}{1.363211in}}%
\pgfpathlineto{\pgfqpoint{1.621969in}{1.359388in}}%
\pgfpathlineto{\pgfqpoint{1.650226in}{1.342576in}}%
\pgfpathlineto{\pgfqpoint{1.666139in}{1.333128in}}%
\pgfpathlineto{\pgfqpoint{1.684824in}{1.322182in}}%
\pgfpathlineto{\pgfqpoint{1.711015in}{1.306869in}}%
\pgfpathlineto{\pgfqpoint{1.719423in}{1.302018in}}%
\pgfpathlineto{\pgfqpoint{1.754021in}{1.282104in}}%
\pgfpathlineto{\pgfqpoint{1.756627in}{1.280609in}}%
\pgfpathlineto{\pgfqpoint{1.788620in}{1.262503in}}%
\pgfpathlineto{\pgfqpoint{1.791726in}{1.260748in}}%
\pgfusepath{stroke}%
\end{pgfscope}%
\begin{pgfscope}%
\pgfpathrectangle{\pgfqpoint{0.854460in}{0.571603in}}{\pgfqpoint{6.885100in}{5.225635in}}%
\pgfusepath{clip}%
\pgfsetbuttcap%
\pgfsetroundjoin%
\pgfsetlinewidth{1.505625pt}%
\definecolor{currentstroke}{rgb}{0.163625,0.471133,0.558148}%
\pgfsetstrokecolor{currentstroke}%
\pgfsetdash{}{0pt}%
\pgfpathmoveto{\pgfqpoint{2.133532in}{1.078630in}}%
\pgfpathlineto{\pgfqpoint{2.134605in}{1.078088in}}%
\pgfpathlineto{\pgfqpoint{2.149552in}{1.070533in}}%
\pgfpathlineto{\pgfqpoint{2.169203in}{1.060735in}}%
\pgfpathlineto{\pgfqpoint{2.202219in}{1.044274in}}%
\pgfpathlineto{\pgfqpoint{2.203802in}{1.043495in}}%
\pgfpathlineto{\pgfqpoint{2.238400in}{1.026535in}}%
\pgfpathlineto{\pgfqpoint{2.255786in}{1.018014in}}%
\pgfpathlineto{\pgfqpoint{2.272999in}{1.009692in}}%
\pgfpathlineto{\pgfqpoint{2.307597in}{0.992966in}}%
\pgfpathlineto{\pgfqpoint{2.310111in}{0.991755in}}%
\pgfpathlineto{\pgfqpoint{2.342196in}{0.976503in}}%
\pgfpathlineto{\pgfqpoint{2.365338in}{0.965495in}}%
\pgfpathlineto{\pgfqpoint{2.376794in}{0.960120in}}%
\pgfpathlineto{\pgfqpoint{2.411393in}{0.943908in}}%
\pgfpathlineto{\pgfqpoint{2.421379in}{0.939236in}}%
\pgfpathlineto{\pgfqpoint{2.445991in}{0.927877in}}%
\pgfpathlineto{\pgfqpoint{2.478248in}{0.912976in}}%
\pgfpathlineto{\pgfqpoint{2.480590in}{0.911909in}}%
\pgfpathlineto{\pgfqpoint{2.515188in}{0.896193in}}%
\pgfpathlineto{\pgfqpoint{2.536036in}{0.886717in}}%
\pgfpathlineto{\pgfqpoint{2.549787in}{0.880551in}}%
\pgfpathlineto{\pgfqpoint{2.584385in}{0.865050in}}%
\pgfpathlineto{\pgfqpoint{2.594651in}{0.860458in}}%
\pgfpathlineto{\pgfqpoint{2.618983in}{0.849720in}}%
\pgfpathlineto{\pgfqpoint{2.653582in}{0.834430in}}%
\pgfpathlineto{\pgfqpoint{2.654109in}{0.834198in}}%
\pgfpathlineto{\pgfqpoint{2.688180in}{0.819397in}}%
\pgfpathlineto{\pgfqpoint{2.714512in}{0.807939in}}%
\pgfpathlineto{\pgfqpoint{2.722779in}{0.804390in}}%
\pgfpathlineto{\pgfqpoint{2.757377in}{0.789566in}}%
\pgfpathlineto{\pgfqpoint{2.775782in}{0.781679in}}%
\pgfpathlineto{\pgfqpoint{2.791976in}{0.774834in}}%
\pgfpathlineto{\pgfqpoint{2.826574in}{0.760211in}}%
\pgfpathlineto{\pgfqpoint{2.837925in}{0.755420in}}%
\pgfpathlineto{\pgfqpoint{2.861173in}{0.745740in}}%
\pgfpathlineto{\pgfqpoint{2.895771in}{0.731316in}}%
\pgfpathlineto{\pgfqpoint{2.900955in}{0.729160in}}%
\pgfpathlineto{\pgfqpoint{2.930370in}{0.717094in}}%
\pgfpathlineto{\pgfqpoint{2.964882in}{0.702901in}}%
\pgfpathlineto{\pgfqpoint{2.964968in}{0.702865in}}%
\pgfpathlineto{\pgfqpoint{2.999567in}{0.688880in}}%
\pgfpathlineto{\pgfqpoint{3.029759in}{0.676641in}}%
\pgfpathlineto{\pgfqpoint{3.034165in}{0.674880in}}%
\pgfpathlineto{\pgfqpoint{3.068764in}{0.661083in}}%
\pgfpathlineto{\pgfqpoint{3.095543in}{0.650382in}}%
\pgfpathlineto{\pgfqpoint{3.103362in}{0.647300in}}%
\pgfpathlineto{\pgfqpoint{3.137961in}{0.633689in}}%
\pgfpathlineto{\pgfqpoint{3.162240in}{0.624122in}}%
\pgfpathlineto{\pgfqpoint{3.172559in}{0.620112in}}%
\pgfpathlineto{\pgfqpoint{3.207158in}{0.606685in}}%
\pgfpathlineto{\pgfqpoint{3.229859in}{0.597863in}}%
\pgfpathlineto{\pgfqpoint{3.241756in}{0.593303in}}%
\pgfpathlineto{\pgfqpoint{3.276355in}{0.580056in}}%
\pgfpathlineto{\pgfqpoint{3.298404in}{0.571603in}}%
\pgfusepath{stroke}%
\end{pgfscope}%
\begin{pgfscope}%
\pgfpathrectangle{\pgfqpoint{0.854460in}{0.571603in}}{\pgfqpoint{6.885100in}{5.225635in}}%
\pgfusepath{clip}%
\pgfsetbuttcap%
\pgfsetroundjoin%
\pgfsetlinewidth{1.505625pt}%
\definecolor{currentstroke}{rgb}{0.154815,0.493313,0.557840}%
\pgfsetstrokecolor{currentstroke}%
\pgfsetdash{}{0pt}%
\pgfpathmoveto{\pgfqpoint{7.739560in}{3.488721in}}%
\pgfpathlineto{\pgfqpoint{7.723442in}{3.512664in}}%
\pgfpathlineto{\pgfqpoint{7.706160in}{3.538924in}}%
\pgfpathlineto{\pgfqpoint{7.704962in}{3.540781in}}%
\pgfpathlineto{\pgfqpoint{7.689162in}{3.565183in}}%
\pgfpathlineto{\pgfqpoint{7.672550in}{3.591443in}}%
\pgfpathlineto{\pgfqpoint{7.670363in}{3.594973in}}%
\pgfpathlineto{\pgfqpoint{7.656229in}{3.617702in}}%
\pgfpathlineto{\pgfqpoint{7.640282in}{3.643962in}}%
\pgfpathlineto{\pgfqpoint{7.635765in}{3.651570in}}%
\pgfpathlineto{\pgfqpoint{7.624644in}{3.670221in}}%
\pgfpathlineto{\pgfqpoint{7.609358in}{3.696481in}}%
\pgfpathlineto{\pgfqpoint{7.601166in}{3.710896in}}%
\pgfpathlineto{\pgfqpoint{7.594406in}{3.722740in}}%
\pgfpathlineto{\pgfqpoint{7.579777in}{3.749000in}}%
\pgfpathlineto{\pgfqpoint{7.566568in}{3.773335in}}%
\pgfpathlineto{\pgfqpoint{7.565518in}{3.775259in}}%
\pgfpathlineto{\pgfqpoint{7.551542in}{3.801519in}}%
\pgfpathlineto{\pgfqpoint{7.537940in}{3.827778in}}%
\pgfpathlineto{\pgfqpoint{7.531969in}{3.839612in}}%
\pgfpathlineto{\pgfqpoint{7.524656in}{3.854037in}}%
\pgfpathlineto{\pgfqpoint{7.511699in}{3.880297in}}%
\pgfpathlineto{\pgfqpoint{7.499109in}{3.906556in}}%
\pgfpathlineto{\pgfqpoint{7.497371in}{3.910284in}}%
\pgfpathlineto{\pgfqpoint{7.486808in}{3.932816in}}%
\pgfpathlineto{\pgfqpoint{7.474857in}{3.959075in}}%
\pgfpathlineto{\pgfqpoint{7.463267in}{3.985335in}}%
\pgfpathlineto{\pgfqpoint{7.462772in}{3.986490in}}%
\pgfpathlineto{\pgfqpoint{7.462673in}{3.986719in}}%
\pgfusepath{stroke}%
\end{pgfscope}%
\begin{pgfscope}%
\pgfpathrectangle{\pgfqpoint{0.854460in}{0.571603in}}{\pgfqpoint{6.885100in}{5.225635in}}%
\pgfusepath{clip}%
\pgfsetbuttcap%
\pgfsetroundjoin%
\pgfsetlinewidth{1.505625pt}%
\definecolor{currentstroke}{rgb}{0.154815,0.493313,0.557840}%
\pgfsetstrokecolor{currentstroke}%
\pgfsetdash{}{0pt}%
\pgfpathmoveto{\pgfqpoint{7.334218in}{4.356792in}}%
\pgfpathlineto{\pgfqpoint{7.328506in}{4.379227in}}%
\pgfpathlineto{\pgfqpoint{7.324378in}{4.396277in}}%
\pgfpathlineto{\pgfqpoint{7.322134in}{4.405486in}}%
\pgfpathlineto{\pgfqpoint{7.316062in}{4.431746in}}%
\pgfpathlineto{\pgfqpoint{7.310319in}{4.458005in}}%
\pgfpathlineto{\pgfqpoint{7.304903in}{4.484265in}}%
\pgfpathlineto{\pgfqpoint{7.299812in}{4.510524in}}%
\pgfpathlineto{\pgfqpoint{7.295044in}{4.536784in}}%
\pgfpathlineto{\pgfqpoint{7.290597in}{4.563043in}}%
\pgfpathlineto{\pgfqpoint{7.289780in}{4.568243in}}%
\pgfpathlineto{\pgfqpoint{7.286445in}{4.589303in}}%
\pgfpathlineto{\pgfqpoint{7.282606in}{4.615562in}}%
\pgfpathlineto{\pgfqpoint{7.279086in}{4.641822in}}%
\pgfpathlineto{\pgfqpoint{7.275882in}{4.668081in}}%
\pgfpathlineto{\pgfqpoint{7.272992in}{4.694341in}}%
\pgfpathlineto{\pgfqpoint{7.270415in}{4.720600in}}%
\pgfpathlineto{\pgfqpoint{7.268149in}{4.746860in}}%
\pgfpathlineto{\pgfqpoint{7.266194in}{4.773119in}}%
\pgfpathlineto{\pgfqpoint{7.264546in}{4.799378in}}%
\pgfpathlineto{\pgfqpoint{7.263205in}{4.825638in}}%
\pgfpathlineto{\pgfqpoint{7.262169in}{4.851897in}}%
\pgfpathlineto{\pgfqpoint{7.261436in}{4.878157in}}%
\pgfpathlineto{\pgfqpoint{7.261006in}{4.904416in}}%
\pgfpathlineto{\pgfqpoint{7.260877in}{4.930676in}}%
\pgfpathlineto{\pgfqpoint{7.261046in}{4.956935in}}%
\pgfpathlineto{\pgfqpoint{7.261514in}{4.983195in}}%
\pgfpathlineto{\pgfqpoint{7.262279in}{5.009454in}}%
\pgfpathlineto{\pgfqpoint{7.263338in}{5.035714in}}%
\pgfpathlineto{\pgfqpoint{7.264692in}{5.061973in}}%
\pgfpathlineto{\pgfqpoint{7.266338in}{5.088233in}}%
\pgfpathlineto{\pgfqpoint{7.268276in}{5.114492in}}%
\pgfpathlineto{\pgfqpoint{7.270504in}{5.140752in}}%
\pgfpathlineto{\pgfqpoint{7.273020in}{5.167011in}}%
\pgfpathlineto{\pgfqpoint{7.275825in}{5.193271in}}%
\pgfpathlineto{\pgfqpoint{7.278915in}{5.219530in}}%
\pgfpathlineto{\pgfqpoint{7.282291in}{5.245790in}}%
\pgfpathlineto{\pgfqpoint{7.285951in}{5.272049in}}%
\pgfpathlineto{\pgfqpoint{7.289780in}{5.297553in}}%
\pgfpathlineto{\pgfqpoint{7.289893in}{5.298308in}}%
\pgfpathlineto{\pgfqpoint{7.294085in}{5.324568in}}%
\pgfpathlineto{\pgfqpoint{7.298554in}{5.350827in}}%
\pgfpathlineto{\pgfqpoint{7.303301in}{5.377087in}}%
\pgfpathlineto{\pgfqpoint{7.308324in}{5.403346in}}%
\pgfpathlineto{\pgfqpoint{7.313621in}{5.429606in}}%
\pgfpathlineto{\pgfqpoint{7.319191in}{5.455865in}}%
\pgfpathlineto{\pgfqpoint{7.324378in}{5.479191in}}%
\pgfpathlineto{\pgfqpoint{7.325028in}{5.482125in}}%
\pgfpathlineto{\pgfqpoint{7.331094in}{5.508384in}}%
\pgfpathlineto{\pgfqpoint{7.337427in}{5.534644in}}%
\pgfpathlineto{\pgfqpoint{7.344025in}{5.560903in}}%
\pgfpathlineto{\pgfqpoint{7.350889in}{5.587163in}}%
\pgfpathlineto{\pgfqpoint{7.358016in}{5.613422in}}%
\pgfpathlineto{\pgfqpoint{7.358977in}{5.616848in}}%
\pgfpathlineto{\pgfqpoint{7.365355in}{5.639682in}}%
\pgfpathlineto{\pgfqpoint{7.372945in}{5.665941in}}%
\pgfpathlineto{\pgfqpoint{7.380792in}{5.692201in}}%
\pgfpathlineto{\pgfqpoint{7.388895in}{5.718460in}}%
\pgfpathlineto{\pgfqpoint{7.393575in}{5.733198in}}%
\pgfpathlineto{\pgfqpoint{7.397223in}{5.744720in}}%
\pgfpathlineto{\pgfqpoint{7.405764in}{5.770979in}}%
\pgfpathlineto{\pgfqpoint{7.414554in}{5.797238in}}%
\pgfusepath{stroke}%
\end{pgfscope}%
\begin{pgfscope}%
\pgfpathrectangle{\pgfqpoint{0.854460in}{0.571603in}}{\pgfqpoint{6.885100in}{5.225635in}}%
\pgfusepath{clip}%
\pgfsetbuttcap%
\pgfsetroundjoin%
\pgfsetlinewidth{1.505625pt}%
\definecolor{currentstroke}{rgb}{0.154815,0.493313,0.557840}%
\pgfsetstrokecolor{currentstroke}%
\pgfsetdash{}{0pt}%
\pgfpathmoveto{\pgfqpoint{0.854460in}{1.823232in}}%
\pgfpathlineto{\pgfqpoint{0.875623in}{1.805799in}}%
\pgfpathlineto{\pgfqpoint{0.889059in}{1.794878in}}%
\pgfpathlineto{\pgfqpoint{0.908027in}{1.779539in}}%
\pgfpathlineto{\pgfqpoint{0.923657in}{1.767069in}}%
\pgfpathlineto{\pgfqpoint{0.941025in}{1.753280in}}%
\pgfpathlineto{\pgfqpoint{0.958256in}{1.739781in}}%
\pgfpathlineto{\pgfqpoint{0.974623in}{1.727020in}}%
\pgfpathlineto{\pgfqpoint{0.992854in}{1.712995in}}%
\pgfpathlineto{\pgfqpoint{1.008831in}{1.700761in}}%
\pgfpathlineto{\pgfqpoint{1.027453in}{1.686690in}}%
\pgfpathlineto{\pgfqpoint{1.043656in}{1.674501in}}%
\pgfpathlineto{\pgfqpoint{1.062051in}{1.660847in}}%
\pgfpathlineto{\pgfqpoint{1.079107in}{1.648242in}}%
\pgfpathlineto{\pgfqpoint{1.096650in}{1.635448in}}%
\pgfpathlineto{\pgfqpoint{1.115190in}{1.621982in}}%
\pgfpathlineto{\pgfqpoint{1.131248in}{1.610474in}}%
\pgfpathlineto{\pgfqpoint{1.151912in}{1.595723in}}%
\pgfpathlineto{\pgfqpoint{1.165847in}{1.585907in}}%
\pgfpathlineto{\pgfqpoint{1.189281in}{1.569463in}}%
\pgfpathlineto{\pgfqpoint{1.200445in}{1.561732in}}%
\pgfpathlineto{\pgfqpoint{1.227301in}{1.543204in}}%
\pgfpathlineto{\pgfqpoint{1.235044in}{1.537932in}}%
\pgfpathlineto{\pgfqpoint{1.265979in}{1.516944in}}%
\pgfpathlineto{\pgfqpoint{1.269642in}{1.514492in}}%
\pgfpathlineto{\pgfqpoint{1.304241in}{1.491411in}}%
\pgfpathlineto{\pgfqpoint{1.305333in}{1.490685in}}%
\pgfpathlineto{\pgfqpoint{1.338839in}{1.468717in}}%
\pgfpathlineto{\pgfqpoint{1.345406in}{1.464425in}}%
\pgfpathlineto{\pgfqpoint{1.347669in}{1.462965in}}%
\pgfusepath{stroke}%
\end{pgfscope}%
\begin{pgfscope}%
\pgfpathrectangle{\pgfqpoint{0.854460in}{0.571603in}}{\pgfqpoint{6.885100in}{5.225635in}}%
\pgfusepath{clip}%
\pgfsetbuttcap%
\pgfsetroundjoin%
\pgfsetlinewidth{1.505625pt}%
\definecolor{currentstroke}{rgb}{0.154815,0.493313,0.557840}%
\pgfsetstrokecolor{currentstroke}%
\pgfsetdash{}{0pt}%
\pgfpathmoveto{\pgfqpoint{1.678814in}{1.261478in}}%
\pgfpathlineto{\pgfqpoint{1.684824in}{1.258031in}}%
\pgfpathlineto{\pgfqpoint{1.691261in}{1.254350in}}%
\pgfpathlineto{\pgfqpoint{1.719423in}{1.238455in}}%
\pgfpathlineto{\pgfqpoint{1.737817in}{1.228090in}}%
\pgfpathlineto{\pgfqpoint{1.754021in}{1.219079in}}%
\pgfpathlineto{\pgfqpoint{1.785084in}{1.201831in}}%
\pgfpathlineto{\pgfqpoint{1.788620in}{1.199894in}}%
\pgfpathlineto{\pgfqpoint{1.823218in}{1.180998in}}%
\pgfpathlineto{\pgfqpoint{1.833173in}{1.175571in}}%
\pgfpathlineto{\pgfqpoint{1.857817in}{1.162316in}}%
\pgfpathlineto{\pgfqpoint{1.882018in}{1.149312in}}%
\pgfpathlineto{\pgfqpoint{1.892415in}{1.143799in}}%
\pgfpathlineto{\pgfqpoint{1.927014in}{1.125487in}}%
\pgfpathlineto{\pgfqpoint{1.931628in}{1.123052in}}%
\pgfpathlineto{\pgfqpoint{1.961612in}{1.107439in}}%
\pgfpathlineto{\pgfqpoint{1.982072in}{1.096793in}}%
\pgfpathlineto{\pgfqpoint{1.996211in}{1.089533in}}%
\pgfpathlineto{\pgfqpoint{2.030809in}{1.071786in}}%
\pgfpathlineto{\pgfqpoint{2.033259in}{1.070533in}}%
\pgfpathlineto{\pgfqpoint{2.065408in}{1.054316in}}%
\pgfpathlineto{\pgfqpoint{2.085320in}{1.044274in}}%
\pgfpathlineto{\pgfqpoint{2.100006in}{1.036966in}}%
\pgfpathlineto{\pgfqpoint{2.134605in}{1.019764in}}%
\pgfpathlineto{\pgfqpoint{2.138134in}{1.018014in}}%
\pgfpathlineto{\pgfqpoint{2.169203in}{1.002819in}}%
\pgfpathlineto{\pgfqpoint{2.191824in}{0.991755in}}%
\pgfpathlineto{\pgfqpoint{2.203802in}{0.985974in}}%
\pgfpathlineto{\pgfqpoint{2.238400in}{0.969299in}}%
\pgfpathlineto{\pgfqpoint{2.246308in}{0.965495in}}%
\pgfpathlineto{\pgfqpoint{2.272999in}{0.952830in}}%
\pgfpathlineto{\pgfqpoint{2.301632in}{0.939236in}}%
\pgfpathlineto{\pgfqpoint{2.307597in}{0.936442in}}%
\pgfpathlineto{\pgfqpoint{2.342196in}{0.920276in}}%
\pgfpathlineto{\pgfqpoint{2.357825in}{0.912976in}}%
\pgfpathlineto{\pgfqpoint{2.376794in}{0.904236in}}%
\pgfpathlineto{\pgfqpoint{2.411393in}{0.888292in}}%
\pgfpathlineto{\pgfqpoint{2.414820in}{0.886717in}}%
\pgfpathlineto{\pgfqpoint{2.445991in}{0.872586in}}%
\pgfpathlineto{\pgfqpoint{2.472717in}{0.860458in}}%
\pgfpathlineto{\pgfqpoint{2.480590in}{0.856933in}}%
\pgfpathlineto{\pgfqpoint{2.515188in}{0.841473in}}%
\pgfpathlineto{\pgfqpoint{2.531475in}{0.834198in}}%
\pgfpathlineto{\pgfqpoint{2.549787in}{0.826128in}}%
\pgfpathlineto{\pgfqpoint{2.584385in}{0.810880in}}%
\pgfpathlineto{\pgfqpoint{2.591073in}{0.807939in}}%
\pgfpathlineto{\pgfqpoint{2.618983in}{0.795830in}}%
\pgfpathlineto{\pgfqpoint{2.651546in}{0.781679in}}%
\pgfpathlineto{\pgfqpoint{2.653582in}{0.780806in}}%
\pgfpathlineto{\pgfqpoint{2.688180in}{0.766020in}}%
\pgfpathlineto{\pgfqpoint{2.712947in}{0.755420in}}%
\pgfpathlineto{\pgfqpoint{2.722779in}{0.751268in}}%
\pgfpathlineto{\pgfqpoint{2.757377in}{0.736682in}}%
\pgfpathlineto{\pgfqpoint{2.775217in}{0.729160in}}%
\pgfpathlineto{\pgfqpoint{2.791976in}{0.722190in}}%
\pgfpathlineto{\pgfqpoint{2.826574in}{0.707801in}}%
\pgfpathlineto{\pgfqpoint{2.838369in}{0.702901in}}%
\pgfpathlineto{\pgfqpoint{2.861173in}{0.693555in}}%
\pgfpathlineto{\pgfqpoint{2.895771in}{0.679360in}}%
\pgfpathlineto{\pgfqpoint{2.902412in}{0.676641in}}%
\pgfpathlineto{\pgfqpoint{2.930370in}{0.665350in}}%
\pgfpathlineto{\pgfqpoint{2.964968in}{0.651346in}}%
\pgfpathlineto{\pgfqpoint{2.967358in}{0.650382in}}%
\pgfpathlineto{\pgfqpoint{2.999567in}{0.637559in}}%
\pgfpathlineto{\pgfqpoint{3.033223in}{0.624122in}}%
\pgfpathlineto{\pgfqpoint{3.034165in}{0.623751in}}%
\pgfpathlineto{\pgfqpoint{3.068764in}{0.610169in}}%
\pgfpathlineto{\pgfqpoint{3.100021in}{0.597863in}}%
\pgfpathlineto{\pgfqpoint{3.103362in}{0.596565in}}%
\pgfpathlineto{\pgfqpoint{3.137961in}{0.583165in}}%
\pgfpathlineto{\pgfqpoint{3.167734in}{0.571603in}}%
\pgfusepath{stroke}%
\end{pgfscope}%
\begin{pgfscope}%
\pgfpathrectangle{\pgfqpoint{0.854460in}{0.571603in}}{\pgfqpoint{6.885100in}{5.225635in}}%
\pgfusepath{clip}%
\pgfsetbuttcap%
\pgfsetroundjoin%
\pgfsetlinewidth{1.505625pt}%
\definecolor{currentstroke}{rgb}{0.147607,0.511733,0.557049}%
\pgfsetstrokecolor{currentstroke}%
\pgfsetdash{}{0pt}%
\pgfpathmoveto{\pgfqpoint{7.739560in}{3.686319in}}%
\pgfpathlineto{\pgfqpoint{7.733649in}{3.696481in}}%
\pgfpathlineto{\pgfqpoint{7.718734in}{3.722740in}}%
\pgfpathlineto{\pgfqpoint{7.713312in}{3.732536in}}%
\pgfusepath{stroke}%
\end{pgfscope}%
\begin{pgfscope}%
\pgfpathrectangle{\pgfqpoint{0.854460in}{0.571603in}}{\pgfqpoint{6.885100in}{5.225635in}}%
\pgfusepath{clip}%
\pgfsetbuttcap%
\pgfsetroundjoin%
\pgfsetlinewidth{1.505625pt}%
\definecolor{currentstroke}{rgb}{0.147607,0.511733,0.557049}%
\pgfsetstrokecolor{currentstroke}%
\pgfsetdash{}{0pt}%
\pgfpathmoveto{\pgfqpoint{7.546867in}{4.086647in}}%
\pgfpathlineto{\pgfqpoint{7.545419in}{4.090373in}}%
\pgfpathlineto{\pgfqpoint{7.535564in}{4.116632in}}%
\pgfpathlineto{\pgfqpoint{7.531969in}{4.126557in}}%
\pgfpathlineto{\pgfqpoint{7.526018in}{4.142892in}}%
\pgfpathlineto{\pgfqpoint{7.516797in}{4.169151in}}%
\pgfpathlineto{\pgfqpoint{7.507924in}{4.195411in}}%
\pgfpathlineto{\pgfqpoint{7.499395in}{4.221670in}}%
\pgfpathlineto{\pgfqpoint{7.497371in}{4.228160in}}%
\pgfpathlineto{\pgfqpoint{7.491166in}{4.247930in}}%
\pgfpathlineto{\pgfqpoint{7.483266in}{4.274189in}}%
\pgfpathlineto{\pgfqpoint{7.475706in}{4.300449in}}%
\pgfpathlineto{\pgfqpoint{7.468484in}{4.326708in}}%
\pgfpathlineto{\pgfqpoint{7.462772in}{4.348492in}}%
\pgfpathlineto{\pgfqpoint{7.461591in}{4.352967in}}%
\pgfpathlineto{\pgfqpoint{7.454994in}{4.379227in}}%
\pgfpathlineto{\pgfqpoint{7.448732in}{4.405486in}}%
\pgfpathlineto{\pgfqpoint{7.442802in}{4.431746in}}%
\pgfpathlineto{\pgfqpoint{7.437202in}{4.458005in}}%
\pgfpathlineto{\pgfqpoint{7.431931in}{4.484265in}}%
\pgfpathlineto{\pgfqpoint{7.428174in}{4.504224in}}%
\pgfpathlineto{\pgfqpoint{7.426979in}{4.510524in}}%
\pgfpathlineto{\pgfqpoint{7.422327in}{4.536784in}}%
\pgfpathlineto{\pgfqpoint{7.418001in}{4.563043in}}%
\pgfpathlineto{\pgfqpoint{7.413998in}{4.589303in}}%
\pgfpathlineto{\pgfqpoint{7.410318in}{4.615562in}}%
\pgfpathlineto{\pgfqpoint{7.406958in}{4.641822in}}%
\pgfpathlineto{\pgfqpoint{7.403916in}{4.668081in}}%
\pgfpathlineto{\pgfqpoint{7.401192in}{4.694341in}}%
\pgfpathlineto{\pgfqpoint{7.398783in}{4.720600in}}%
\pgfpathlineto{\pgfqpoint{7.396688in}{4.746860in}}%
\pgfpathlineto{\pgfqpoint{7.394906in}{4.773119in}}%
\pgfpathlineto{\pgfqpoint{7.393575in}{4.796874in}}%
\pgfpathlineto{\pgfqpoint{7.393434in}{4.799378in}}%
\pgfpathlineto{\pgfqpoint{7.392264in}{4.825638in}}%
\pgfpathlineto{\pgfqpoint{7.391405in}{4.851897in}}%
\pgfpathlineto{\pgfqpoint{7.390854in}{4.878157in}}%
\pgfpathlineto{\pgfqpoint{7.390611in}{4.904416in}}%
\pgfpathlineto{\pgfqpoint{7.390675in}{4.930676in}}%
\pgfpathlineto{\pgfqpoint{7.391044in}{4.956935in}}%
\pgfpathlineto{\pgfqpoint{7.391717in}{4.983195in}}%
\pgfpathlineto{\pgfqpoint{7.392692in}{5.009454in}}%
\pgfpathlineto{\pgfqpoint{7.393575in}{5.027628in}}%
\pgfpathlineto{\pgfqpoint{7.393966in}{5.035714in}}%
\pgfpathlineto{\pgfqpoint{7.395532in}{5.061973in}}%
\pgfpathlineto{\pgfqpoint{7.397394in}{5.088233in}}%
\pgfpathlineto{\pgfqpoint{7.399552in}{5.114492in}}%
\pgfpathlineto{\pgfqpoint{7.402004in}{5.140752in}}%
\pgfpathlineto{\pgfqpoint{7.404750in}{5.167011in}}%
\pgfpathlineto{\pgfqpoint{7.407788in}{5.193271in}}%
\pgfpathlineto{\pgfqpoint{7.411117in}{5.219530in}}%
\pgfpathlineto{\pgfqpoint{7.414737in}{5.245790in}}%
\pgfpathlineto{\pgfqpoint{7.418645in}{5.272049in}}%
\pgfpathlineto{\pgfqpoint{7.422841in}{5.298308in}}%
\pgfpathlineto{\pgfqpoint{7.427325in}{5.324568in}}%
\pgfpathlineto{\pgfqpoint{7.428174in}{5.329255in}}%
\pgfpathlineto{\pgfqpoint{7.432065in}{5.350827in}}%
\pgfpathlineto{\pgfqpoint{7.437081in}{5.377087in}}%
\pgfpathlineto{\pgfqpoint{7.442379in}{5.403346in}}%
\pgfpathlineto{\pgfqpoint{7.447958in}{5.429606in}}%
\pgfpathlineto{\pgfqpoint{7.453816in}{5.455865in}}%
\pgfpathlineto{\pgfqpoint{7.459952in}{5.482125in}}%
\pgfpathlineto{\pgfqpoint{7.462772in}{5.493693in}}%
\pgfpathlineto{\pgfqpoint{7.466339in}{5.508384in}}%
\pgfpathlineto{\pgfqpoint{7.472979in}{5.534644in}}%
\pgfpathlineto{\pgfqpoint{7.479892in}{5.560903in}}%
\pgfpathlineto{\pgfqpoint{7.487077in}{5.587163in}}%
\pgfpathlineto{\pgfqpoint{7.494533in}{5.613422in}}%
\pgfpathlineto{\pgfqpoint{7.497371in}{5.623092in}}%
\pgfpathlineto{\pgfqpoint{7.502222in}{5.639682in}}%
\pgfpathlineto{\pgfqpoint{7.510156in}{5.665941in}}%
\pgfpathlineto{\pgfqpoint{7.518354in}{5.692201in}}%
\pgfpathlineto{\pgfqpoint{7.526817in}{5.718460in}}%
\pgfpathlineto{\pgfqpoint{7.531969in}{5.733992in}}%
\pgfpathlineto{\pgfqpoint{7.535516in}{5.744720in}}%
\pgfpathlineto{\pgfqpoint{7.544435in}{5.770979in}}%
\pgfpathlineto{\pgfqpoint{7.553612in}{5.797238in}}%
\pgfusepath{stroke}%
\end{pgfscope}%
\begin{pgfscope}%
\pgfpathrectangle{\pgfqpoint{0.854460in}{0.571603in}}{\pgfqpoint{6.885100in}{5.225635in}}%
\pgfusepath{clip}%
\pgfsetbuttcap%
\pgfsetroundjoin%
\pgfsetlinewidth{1.505625pt}%
\definecolor{currentstroke}{rgb}{0.147607,0.511733,0.557049}%
\pgfsetstrokecolor{currentstroke}%
\pgfsetdash{}{0pt}%
\pgfpathmoveto{\pgfqpoint{0.854460in}{1.743965in}}%
\pgfpathlineto{\pgfqpoint{0.875903in}{1.727020in}}%
\pgfpathlineto{\pgfqpoint{0.889059in}{1.716761in}}%
\pgfpathlineto{\pgfqpoint{0.906171in}{1.703480in}}%
\pgfusepath{stroke}%
\end{pgfscope}%
\begin{pgfscope}%
\pgfpathrectangle{\pgfqpoint{0.854460in}{0.571603in}}{\pgfqpoint{6.885100in}{5.225635in}}%
\pgfusepath{clip}%
\pgfsetbuttcap%
\pgfsetroundjoin%
\pgfsetlinewidth{1.505625pt}%
\definecolor{currentstroke}{rgb}{0.147607,0.511733,0.557049}%
\pgfsetstrokecolor{currentstroke}%
\pgfsetdash{}{0pt}%
\pgfpathmoveto{\pgfqpoint{1.221850in}{1.477757in}}%
\pgfpathlineto{\pgfqpoint{1.235044in}{1.468994in}}%
\pgfpathlineto{\pgfqpoint{1.241944in}{1.464425in}}%
\pgfpathlineto{\pgfqpoint{1.269642in}{1.446328in}}%
\pgfpathlineto{\pgfqpoint{1.282175in}{1.438166in}}%
\pgfpathlineto{\pgfqpoint{1.304241in}{1.423982in}}%
\pgfpathlineto{\pgfqpoint{1.323082in}{1.411906in}}%
\pgfpathlineto{\pgfqpoint{1.338839in}{1.401940in}}%
\pgfpathlineto{\pgfqpoint{1.364672in}{1.385647in}}%
\pgfpathlineto{\pgfqpoint{1.373438in}{1.380190in}}%
\pgfpathlineto{\pgfqpoint{1.406946in}{1.359388in}}%
\pgfpathlineto{\pgfqpoint{1.408036in}{1.358720in}}%
\pgfpathlineto{\pgfqpoint{1.442635in}{1.337605in}}%
\pgfpathlineto{\pgfqpoint{1.449991in}{1.333128in}}%
\pgfpathlineto{\pgfqpoint{1.477233in}{1.316764in}}%
\pgfpathlineto{\pgfqpoint{1.493746in}{1.306869in}}%
\pgfpathlineto{\pgfqpoint{1.511832in}{1.296171in}}%
\pgfpathlineto{\pgfqpoint{1.538200in}{1.280609in}}%
\pgfpathlineto{\pgfqpoint{1.546430in}{1.275815in}}%
\pgfpathlineto{\pgfqpoint{1.581029in}{1.255712in}}%
\pgfpathlineto{\pgfqpoint{1.583381in}{1.254350in}}%
\pgfpathlineto{\pgfqpoint{1.615627in}{1.235925in}}%
\pgfpathlineto{\pgfqpoint{1.629364in}{1.228090in}}%
\pgfpathlineto{\pgfqpoint{1.650226in}{1.216347in}}%
\pgfpathlineto{\pgfqpoint{1.676056in}{1.201831in}}%
\pgfpathlineto{\pgfqpoint{1.684824in}{1.196967in}}%
\pgfpathlineto{\pgfqpoint{1.719423in}{1.177821in}}%
\pgfpathlineto{\pgfqpoint{1.723501in}{1.175571in}}%
\pgfpathlineto{\pgfqpoint{1.754021in}{1.158956in}}%
\pgfpathlineto{\pgfqpoint{1.771758in}{1.149312in}}%
\pgfpathlineto{\pgfqpoint{1.788620in}{1.140264in}}%
\pgfpathlineto{\pgfqpoint{1.820730in}{1.123052in}}%
\pgfpathlineto{\pgfqpoint{1.823218in}{1.121736in}}%
\pgfpathlineto{\pgfqpoint{1.857817in}{1.103497in}}%
\pgfpathlineto{\pgfqpoint{1.870551in}{1.096793in}}%
\pgfpathlineto{\pgfqpoint{1.892415in}{1.085433in}}%
\pgfpathlineto{\pgfqpoint{1.921115in}{1.070533in}}%
\pgfpathlineto{\pgfqpoint{1.927014in}{1.067511in}}%
\pgfpathlineto{\pgfqpoint{1.961612in}{1.049832in}}%
\pgfpathlineto{\pgfqpoint{1.972509in}{1.044274in}}%
\pgfpathlineto{\pgfqpoint{1.996211in}{1.032342in}}%
\pgfpathlineto{\pgfqpoint{2.024683in}{1.018014in}}%
\pgfpathlineto{\pgfqpoint{2.030809in}{1.014972in}}%
\pgfpathlineto{\pgfqpoint{2.065408in}{0.997834in}}%
\pgfpathlineto{\pgfqpoint{2.077698in}{0.991755in}}%
\pgfpathlineto{\pgfqpoint{2.100006in}{0.980866in}}%
\pgfpathlineto{\pgfqpoint{2.131493in}{0.965495in}}%
\pgfpathlineto{\pgfqpoint{2.134605in}{0.963996in}}%
\pgfpathlineto{\pgfqpoint{2.169203in}{0.947382in}}%
\pgfpathlineto{\pgfqpoint{2.186173in}{0.939236in}}%
\pgfpathlineto{\pgfqpoint{2.203802in}{0.930885in}}%
\pgfpathlineto{\pgfqpoint{2.238400in}{0.914500in}}%
\pgfpathlineto{\pgfqpoint{2.241628in}{0.912976in}}%
\pgfpathlineto{\pgfqpoint{2.272999in}{0.898361in}}%
\pgfpathlineto{\pgfqpoint{2.297976in}{0.886717in}}%
\pgfpathlineto{\pgfqpoint{2.307597in}{0.882291in}}%
\pgfpathlineto{\pgfqpoint{2.342196in}{0.866403in}}%
\pgfpathlineto{\pgfqpoint{2.355157in}{0.860458in}}%
\pgfpathlineto{\pgfqpoint{2.376794in}{0.850664in}}%
\pgfpathlineto{\pgfqpoint{2.411393in}{0.834994in}}%
\pgfpathlineto{\pgfqpoint{2.413155in}{0.834198in}}%
\pgfpathlineto{\pgfqpoint{2.445991in}{0.819571in}}%
\pgfpathlineto{\pgfqpoint{2.472073in}{0.807939in}}%
\pgfpathlineto{\pgfqpoint{2.480590in}{0.804191in}}%
\pgfpathlineto{\pgfqpoint{2.515188in}{0.788994in}}%
\pgfpathlineto{\pgfqpoint{2.531846in}{0.781679in}}%
\pgfpathlineto{\pgfqpoint{2.549787in}{0.773906in}}%
\pgfpathlineto{\pgfqpoint{2.584385in}{0.758916in}}%
\pgfpathlineto{\pgfqpoint{2.592469in}{0.755420in}}%
\pgfpathlineto{\pgfqpoint{2.618983in}{0.744107in}}%
\pgfpathlineto{\pgfqpoint{2.653582in}{0.729320in}}%
\pgfpathlineto{\pgfqpoint{2.653956in}{0.729160in}}%
\pgfpathlineto{\pgfqpoint{2.688180in}{0.714777in}}%
\pgfpathlineto{\pgfqpoint{2.716384in}{0.702901in}}%
\pgfpathlineto{\pgfqpoint{2.722779in}{0.700244in}}%
\pgfpathlineto{\pgfqpoint{2.757377in}{0.685901in}}%
\pgfpathlineto{\pgfqpoint{2.779692in}{0.676641in}}%
\pgfpathlineto{\pgfqpoint{2.791976in}{0.671612in}}%
\pgfpathlineto{\pgfqpoint{2.826574in}{0.657462in}}%
\pgfpathlineto{\pgfqpoint{2.843887in}{0.650382in}}%
\pgfpathlineto{\pgfqpoint{2.861173in}{0.643407in}}%
\pgfpathlineto{\pgfqpoint{2.895771in}{0.629447in}}%
\pgfpathlineto{\pgfqpoint{2.908979in}{0.624122in}}%
\pgfpathlineto{\pgfqpoint{2.930370in}{0.615614in}}%
\pgfpathlineto{\pgfqpoint{2.964968in}{0.601841in}}%
\pgfpathlineto{\pgfqpoint{2.974976in}{0.597863in}}%
\pgfpathlineto{\pgfqpoint{2.999567in}{0.588219in}}%
\pgfpathlineto{\pgfqpoint{3.034165in}{0.574630in}}%
\pgfpathlineto{\pgfqpoint{3.041885in}{0.571603in}}%
\pgfusepath{stroke}%
\end{pgfscope}%
\begin{pgfscope}%
\pgfpathrectangle{\pgfqpoint{0.854460in}{0.571603in}}{\pgfqpoint{6.885100in}{5.225635in}}%
\pgfusepath{clip}%
\pgfsetbuttcap%
\pgfsetroundjoin%
\pgfsetlinewidth{1.505625pt}%
\definecolor{currentstroke}{rgb}{0.139147,0.533812,0.555298}%
\pgfsetstrokecolor{currentstroke}%
\pgfsetdash{}{0pt}%
\pgfpathmoveto{\pgfqpoint{7.739560in}{3.919632in}}%
\pgfpathlineto{\pgfqpoint{7.733449in}{3.932816in}}%
\pgfpathlineto{\pgfqpoint{7.721632in}{3.959075in}}%
\pgfpathlineto{\pgfqpoint{7.710178in}{3.985335in}}%
\pgfpathlineto{\pgfqpoint{7.704962in}{3.997674in}}%
\pgfpathlineto{\pgfqpoint{7.699046in}{4.011594in}}%
\pgfpathlineto{\pgfqpoint{7.688239in}{4.037854in}}%
\pgfpathlineto{\pgfqpoint{7.677790in}{4.064113in}}%
\pgfpathlineto{\pgfqpoint{7.670363in}{4.083430in}}%
\pgfpathlineto{\pgfqpoint{7.667679in}{4.090373in}}%
\pgfpathlineto{\pgfqpoint{7.657873in}{4.116632in}}%
\pgfpathlineto{\pgfqpoint{7.648419in}{4.142892in}}%
\pgfpathlineto{\pgfqpoint{7.639316in}{4.169151in}}%
\pgfpathlineto{\pgfqpoint{7.635765in}{4.179797in}}%
\pgfpathlineto{\pgfqpoint{7.630526in}{4.195411in}}%
\pgfpathlineto{\pgfqpoint{7.622060in}{4.221670in}}%
\pgfpathlineto{\pgfqpoint{7.613940in}{4.247930in}}%
\pgfpathlineto{\pgfqpoint{7.606164in}{4.274189in}}%
\pgfpathlineto{\pgfqpoint{7.601166in}{4.291837in}}%
\pgfpathlineto{\pgfqpoint{7.598713in}{4.300449in}}%
\pgfpathlineto{\pgfqpoint{7.591569in}{4.326708in}}%
\pgfpathlineto{\pgfqpoint{7.584765in}{4.352967in}}%
\pgfpathlineto{\pgfqpoint{7.578299in}{4.379227in}}%
\pgfpathlineto{\pgfqpoint{7.572169in}{4.405486in}}%
\pgfpathlineto{\pgfqpoint{7.566568in}{4.430863in}}%
\pgfpathlineto{\pgfqpoint{7.566372in}{4.431746in}}%
\pgfpathlineto{\pgfqpoint{7.560870in}{4.458005in}}%
\pgfpathlineto{\pgfqpoint{7.555700in}{4.484265in}}%
\pgfpathlineto{\pgfqpoint{7.550862in}{4.510524in}}%
\pgfpathlineto{\pgfqpoint{7.546352in}{4.536784in}}%
\pgfpathlineto{\pgfqpoint{7.542170in}{4.563043in}}%
\pgfpathlineto{\pgfqpoint{7.538313in}{4.589303in}}%
\pgfpathlineto{\pgfqpoint{7.534780in}{4.615562in}}%
\pgfpathlineto{\pgfqpoint{7.531969in}{4.638550in}}%
\pgfpathlineto{\pgfqpoint{7.531566in}{4.641822in}}%
\pgfpathlineto{\pgfqpoint{7.528656in}{4.668081in}}%
\pgfpathlineto{\pgfqpoint{7.526068in}{4.694341in}}%
\pgfpathlineto{\pgfqpoint{7.523799in}{4.720600in}}%
\pgfpathlineto{\pgfqpoint{7.521849in}{4.746860in}}%
\pgfpathlineto{\pgfqpoint{7.520216in}{4.773119in}}%
\pgfpathlineto{\pgfqpoint{7.518899in}{4.799378in}}%
\pgfpathlineto{\pgfqpoint{7.517897in}{4.825638in}}%
\pgfpathlineto{\pgfqpoint{7.517207in}{4.851897in}}%
\pgfpathlineto{\pgfqpoint{7.516829in}{4.878157in}}%
\pgfpathlineto{\pgfqpoint{7.516762in}{4.904416in}}%
\pgfpathlineto{\pgfqpoint{7.517004in}{4.930676in}}%
\pgfpathlineto{\pgfqpoint{7.517554in}{4.956935in}}%
\pgfpathlineto{\pgfqpoint{7.518412in}{4.983195in}}%
\pgfpathlineto{\pgfqpoint{7.519575in}{5.009454in}}%
\pgfpathlineto{\pgfqpoint{7.521042in}{5.035714in}}%
\pgfpathlineto{\pgfqpoint{7.522814in}{5.061973in}}%
\pgfpathlineto{\pgfqpoint{7.524888in}{5.088233in}}%
\pgfpathlineto{\pgfqpoint{7.526284in}{5.103665in}}%
\pgfusepath{stroke}%
\end{pgfscope}%
\begin{pgfscope}%
\pgfpathrectangle{\pgfqpoint{0.854460in}{0.571603in}}{\pgfqpoint{6.885100in}{5.225635in}}%
\pgfusepath{clip}%
\pgfsetbuttcap%
\pgfsetroundjoin%
\pgfsetlinewidth{1.505625pt}%
\definecolor{currentstroke}{rgb}{0.139147,0.533812,0.555298}%
\pgfsetstrokecolor{currentstroke}%
\pgfsetdash{}{0pt}%
\pgfpathmoveto{\pgfqpoint{7.593087in}{5.490242in}}%
\pgfpathlineto{\pgfqpoint{7.597706in}{5.508384in}}%
\pgfpathlineto{\pgfqpoint{7.601166in}{5.521452in}}%
\pgfpathlineto{\pgfqpoint{7.604647in}{5.534644in}}%
\pgfpathlineto{\pgfqpoint{7.611842in}{5.560903in}}%
\pgfpathlineto{\pgfqpoint{7.619315in}{5.587163in}}%
\pgfpathlineto{\pgfqpoint{7.627066in}{5.613422in}}%
\pgfpathlineto{\pgfqpoint{7.635093in}{5.639682in}}%
\pgfpathlineto{\pgfqpoint{7.635765in}{5.641813in}}%
\pgfpathlineto{\pgfqpoint{7.643340in}{5.665941in}}%
\pgfpathlineto{\pgfqpoint{7.651854in}{5.692201in}}%
\pgfpathlineto{\pgfqpoint{7.660639in}{5.718460in}}%
\pgfpathlineto{\pgfqpoint{7.669695in}{5.744720in}}%
\pgfpathlineto{\pgfqpoint{7.670363in}{5.746608in}}%
\pgfpathlineto{\pgfqpoint{7.678956in}{5.770979in}}%
\pgfpathlineto{\pgfqpoint{7.688479in}{5.797238in}}%
\pgfusepath{stroke}%
\end{pgfscope}%
\begin{pgfscope}%
\pgfpathrectangle{\pgfqpoint{0.854460in}{0.571603in}}{\pgfqpoint{6.885100in}{5.225635in}}%
\pgfusepath{clip}%
\pgfsetbuttcap%
\pgfsetroundjoin%
\pgfsetlinewidth{1.505625pt}%
\definecolor{currentstroke}{rgb}{0.139147,0.533812,0.555298}%
\pgfsetstrokecolor{currentstroke}%
\pgfsetdash{}{0pt}%
\pgfpathmoveto{\pgfqpoint{0.854460in}{1.669367in}}%
\pgfpathlineto{\pgfqpoint{0.882288in}{1.648242in}}%
\pgfpathlineto{\pgfqpoint{0.889059in}{1.643169in}}%
\pgfpathlineto{\pgfqpoint{0.917462in}{1.621982in}}%
\pgfpathlineto{\pgfqpoint{0.923657in}{1.617421in}}%
\pgfpathlineto{\pgfqpoint{0.953258in}{1.595723in}}%
\pgfpathlineto{\pgfqpoint{0.958256in}{1.592107in}}%
\pgfpathlineto{\pgfqpoint{0.989683in}{1.569463in}}%
\pgfpathlineto{\pgfqpoint{0.992854in}{1.567208in}}%
\pgfpathlineto{\pgfqpoint{1.026745in}{1.543204in}}%
\pgfpathlineto{\pgfqpoint{1.027453in}{1.542709in}}%
\pgfpathlineto{\pgfqpoint{1.062051in}{1.518626in}}%
\pgfpathlineto{\pgfqpoint{1.064476in}{1.516944in}}%
\pgfpathlineto{\pgfqpoint{1.096650in}{1.494927in}}%
\pgfpathlineto{\pgfqpoint{1.102871in}{1.490685in}}%
\pgfpathlineto{\pgfqpoint{1.131248in}{1.471587in}}%
\pgfpathlineto{\pgfqpoint{1.141927in}{1.464425in}}%
\pgfpathlineto{\pgfqpoint{1.165847in}{1.448592in}}%
\pgfpathlineto{\pgfqpoint{1.181649in}{1.438166in}}%
\pgfpathlineto{\pgfqpoint{1.200445in}{1.425926in}}%
\pgfpathlineto{\pgfqpoint{1.222042in}{1.411906in}}%
\pgfpathlineto{\pgfqpoint{1.235044in}{1.403576in}}%
\pgfpathlineto{\pgfqpoint{1.263111in}{1.385647in}}%
\pgfpathlineto{\pgfqpoint{1.269642in}{1.381529in}}%
\pgfpathlineto{\pgfqpoint{1.304241in}{1.359778in}}%
\pgfpathlineto{\pgfqpoint{1.304865in}{1.359388in}}%
\pgfpathlineto{\pgfqpoint{1.338839in}{1.338393in}}%
\pgfpathlineto{\pgfqpoint{1.347383in}{1.333128in}}%
\pgfpathlineto{\pgfqpoint{1.373438in}{1.317278in}}%
\pgfpathlineto{\pgfqpoint{1.390593in}{1.306869in}}%
\pgfpathlineto{\pgfqpoint{1.408036in}{1.296421in}}%
\pgfpathlineto{\pgfqpoint{1.434497in}{1.280609in}}%
\pgfpathlineto{\pgfqpoint{1.442635in}{1.275810in}}%
\pgfpathlineto{\pgfqpoint{1.477233in}{1.255455in}}%
\pgfpathlineto{\pgfqpoint{1.479120in}{1.254350in}}%
\pgfpathlineto{\pgfqpoint{1.511832in}{1.235426in}}%
\pgfpathlineto{\pgfqpoint{1.524538in}{1.228090in}}%
\pgfpathlineto{\pgfqpoint{1.546430in}{1.215614in}}%
\pgfpathlineto{\pgfqpoint{1.570661in}{1.201831in}}%
\pgfpathlineto{\pgfqpoint{1.581029in}{1.196010in}}%
\pgfpathlineto{\pgfqpoint{1.615627in}{1.176623in}}%
\pgfpathlineto{\pgfqpoint{1.617512in}{1.175571in}}%
\pgfpathlineto{\pgfqpoint{1.650226in}{1.157545in}}%
\pgfpathlineto{\pgfqpoint{1.665189in}{1.149312in}}%
\pgfpathlineto{\pgfqpoint{1.684824in}{1.138648in}}%
\pgfpathlineto{\pgfqpoint{1.713578in}{1.123052in}}%
\pgfpathlineto{\pgfqpoint{1.719423in}{1.119923in}}%
\pgfpathlineto{\pgfqpoint{1.754021in}{1.101454in}}%
\pgfpathlineto{\pgfqpoint{1.762771in}{1.096793in}}%
\pgfpathlineto{\pgfqpoint{1.788620in}{1.083204in}}%
\pgfpathlineto{\pgfqpoint{1.812744in}{1.070533in}}%
\pgfpathlineto{\pgfqpoint{1.823218in}{1.065103in}}%
\pgfpathlineto{\pgfqpoint{1.857817in}{1.047201in}}%
\pgfpathlineto{\pgfqpoint{1.863489in}{1.044274in}}%
\pgfpathlineto{\pgfqpoint{1.892415in}{1.029542in}}%
\pgfpathlineto{\pgfqpoint{1.915063in}{1.018014in}}%
\pgfpathlineto{\pgfqpoint{1.927014in}{1.012011in}}%
\pgfpathlineto{\pgfqpoint{1.961612in}{0.994656in}}%
\pgfpathlineto{\pgfqpoint{1.967411in}{0.991755in}}%
\pgfpathlineto{\pgfqpoint{1.996211in}{0.977535in}}%
\pgfpathlineto{\pgfqpoint{2.020600in}{0.965495in}}%
\pgfpathlineto{\pgfqpoint{2.030809in}{0.960521in}}%
\pgfpathlineto{\pgfqpoint{2.065408in}{0.943695in}}%
\pgfpathlineto{\pgfqpoint{2.074595in}{0.939236in}}%
\pgfpathlineto{\pgfqpoint{2.100006in}{0.927063in}}%
\pgfpathlineto{\pgfqpoint{2.129408in}{0.912976in}}%
\pgfpathlineto{\pgfqpoint{2.134605in}{0.910519in}}%
\pgfpathlineto{\pgfqpoint{2.169203in}{0.894204in}}%
\pgfpathlineto{\pgfqpoint{2.185089in}{0.886717in}}%
\pgfpathlineto{\pgfqpoint{2.203802in}{0.878013in}}%
\pgfpathlineto{\pgfqpoint{2.238400in}{0.861923in}}%
\pgfpathlineto{\pgfqpoint{2.241560in}{0.860458in}}%
\pgfpathlineto{\pgfqpoint{2.272999in}{0.846072in}}%
\pgfpathlineto{\pgfqpoint{2.298927in}{0.834198in}}%
\pgfpathlineto{\pgfqpoint{2.307597in}{0.830280in}}%
\pgfpathlineto{\pgfqpoint{2.342196in}{0.814675in}}%
\pgfpathlineto{\pgfqpoint{2.357140in}{0.807939in}}%
\pgfpathlineto{\pgfqpoint{2.359741in}{0.806782in}}%
\pgfusepath{stroke}%
\end{pgfscope}%
\begin{pgfscope}%
\pgfpathrectangle{\pgfqpoint{0.854460in}{0.571603in}}{\pgfqpoint{6.885100in}{5.225635in}}%
\pgfusepath{clip}%
\pgfsetbuttcap%
\pgfsetroundjoin%
\pgfsetlinewidth{1.505625pt}%
\definecolor{currentstroke}{rgb}{0.139147,0.533812,0.555298}%
\pgfsetstrokecolor{currentstroke}%
\pgfsetdash{}{0pt}%
\pgfpathmoveto{\pgfqpoint{2.715258in}{0.654256in}}%
\pgfpathlineto{\pgfqpoint{2.722779in}{0.651140in}}%
\pgfpathlineto{\pgfqpoint{2.724615in}{0.650382in}}%
\pgfpathlineto{\pgfqpoint{2.757377in}{0.637029in}}%
\pgfpathlineto{\pgfqpoint{2.788973in}{0.624122in}}%
\pgfpathlineto{\pgfqpoint{2.791976in}{0.622912in}}%
\pgfpathlineto{\pgfqpoint{2.826574in}{0.609005in}}%
\pgfpathlineto{\pgfqpoint{2.854241in}{0.597863in}}%
\pgfpathlineto{\pgfqpoint{2.861173in}{0.595108in}}%
\pgfpathlineto{\pgfqpoint{2.895771in}{0.581387in}}%
\pgfpathlineto{\pgfqpoint{2.920408in}{0.571603in}}%
\pgfusepath{stroke}%
\end{pgfscope}%
\begin{pgfscope}%
\pgfpathrectangle{\pgfqpoint{0.854460in}{0.571603in}}{\pgfqpoint{6.885100in}{5.225635in}}%
\pgfusepath{clip}%
\pgfsetbuttcap%
\pgfsetroundjoin%
\pgfsetlinewidth{1.505625pt}%
\definecolor{currentstroke}{rgb}{0.131172,0.555899,0.552459}%
\pgfsetstrokecolor{currentstroke}%
\pgfsetdash{}{0pt}%
\pgfpathmoveto{\pgfqpoint{7.739560in}{4.229240in}}%
\pgfpathlineto{\pgfqpoint{7.733829in}{4.247930in}}%
\pgfpathlineto{\pgfqpoint{7.726120in}{4.274189in}}%
\pgfpathlineto{\pgfqpoint{7.718757in}{4.300449in}}%
\pgfpathlineto{\pgfqpoint{7.713838in}{4.318845in}}%
\pgfusepath{stroke}%
\end{pgfscope}%
\begin{pgfscope}%
\pgfpathrectangle{\pgfqpoint{0.854460in}{0.571603in}}{\pgfqpoint{6.885100in}{5.225635in}}%
\pgfusepath{clip}%
\pgfsetbuttcap%
\pgfsetroundjoin%
\pgfsetlinewidth{1.505625pt}%
\definecolor{currentstroke}{rgb}{0.131172,0.555899,0.552459}%
\pgfsetstrokecolor{currentstroke}%
\pgfsetdash{}{0pt}%
\pgfpathmoveto{\pgfqpoint{7.646984in}{4.705288in}}%
\pgfpathlineto{\pgfqpoint{7.645747in}{4.720600in}}%
\pgfpathlineto{\pgfqpoint{7.643945in}{4.746860in}}%
\pgfpathlineto{\pgfqpoint{7.642463in}{4.773119in}}%
\pgfpathlineto{\pgfqpoint{7.641299in}{4.799378in}}%
\pgfpathlineto{\pgfqpoint{7.640451in}{4.825638in}}%
\pgfpathlineto{\pgfqpoint{7.639918in}{4.851897in}}%
\pgfpathlineto{\pgfqpoint{7.639700in}{4.878157in}}%
\pgfpathlineto{\pgfqpoint{7.639794in}{4.904416in}}%
\pgfpathlineto{\pgfqpoint{7.640201in}{4.930676in}}%
\pgfpathlineto{\pgfqpoint{7.640918in}{4.956935in}}%
\pgfpathlineto{\pgfqpoint{7.641944in}{4.983195in}}%
\pgfpathlineto{\pgfqpoint{7.643280in}{5.009454in}}%
\pgfpathlineto{\pgfqpoint{7.644922in}{5.035714in}}%
\pgfpathlineto{\pgfqpoint{7.646872in}{5.061973in}}%
\pgfpathlineto{\pgfqpoint{7.649127in}{5.088233in}}%
\pgfpathlineto{\pgfqpoint{7.651686in}{5.114492in}}%
\pgfpathlineto{\pgfqpoint{7.654549in}{5.140752in}}%
\pgfpathlineto{\pgfqpoint{7.657715in}{5.167011in}}%
\pgfpathlineto{\pgfqpoint{7.661183in}{5.193271in}}%
\pgfpathlineto{\pgfqpoint{7.664952in}{5.219530in}}%
\pgfpathlineto{\pgfqpoint{7.669021in}{5.245790in}}%
\pgfpathlineto{\pgfqpoint{7.670363in}{5.253871in}}%
\pgfpathlineto{\pgfqpoint{7.673369in}{5.272049in}}%
\pgfpathlineto{\pgfqpoint{7.678003in}{5.298308in}}%
\pgfpathlineto{\pgfqpoint{7.682933in}{5.324568in}}%
\pgfpathlineto{\pgfqpoint{7.688158in}{5.350827in}}%
\pgfpathlineto{\pgfqpoint{7.693677in}{5.377087in}}%
\pgfpathlineto{\pgfqpoint{7.699489in}{5.403346in}}%
\pgfpathlineto{\pgfqpoint{7.704962in}{5.426894in}}%
\pgfpathlineto{\pgfqpoint{7.705590in}{5.429606in}}%
\pgfpathlineto{\pgfqpoint{7.711941in}{5.455865in}}%
\pgfpathlineto{\pgfqpoint{7.718581in}{5.482125in}}%
\pgfpathlineto{\pgfqpoint{7.725510in}{5.508384in}}%
\pgfpathlineto{\pgfqpoint{7.732725in}{5.534644in}}%
\pgfpathlineto{\pgfqpoint{7.739560in}{5.558575in}}%
\pgfusepath{stroke}%
\end{pgfscope}%
\begin{pgfscope}%
\pgfpathrectangle{\pgfqpoint{0.854460in}{0.571603in}}{\pgfqpoint{6.885100in}{5.225635in}}%
\pgfusepath{clip}%
\pgfsetbuttcap%
\pgfsetroundjoin%
\pgfsetlinewidth{1.505625pt}%
\definecolor{currentstroke}{rgb}{0.131172,0.555899,0.552459}%
\pgfsetstrokecolor{currentstroke}%
\pgfsetdash{}{0pt}%
\pgfpathmoveto{\pgfqpoint{0.854460in}{1.598867in}}%
\pgfpathlineto{\pgfqpoint{0.858712in}{1.595723in}}%
\pgfpathlineto{\pgfqpoint{0.889059in}{1.573569in}}%
\pgfpathlineto{\pgfqpoint{0.894706in}{1.569463in}}%
\pgfpathlineto{\pgfqpoint{0.923657in}{1.548690in}}%
\pgfpathlineto{\pgfqpoint{0.931335in}{1.543204in}}%
\pgfpathlineto{\pgfqpoint{0.958256in}{1.524215in}}%
\pgfpathlineto{\pgfqpoint{0.968605in}{1.516944in}}%
\pgfpathlineto{\pgfqpoint{0.992854in}{1.500127in}}%
\pgfpathlineto{\pgfqpoint{1.006521in}{1.490685in}}%
\pgfpathlineto{\pgfqpoint{1.027453in}{1.476410in}}%
\pgfpathlineto{\pgfqpoint{1.045091in}{1.464425in}}%
\pgfpathlineto{\pgfqpoint{1.062051in}{1.453050in}}%
\pgfpathlineto{\pgfqpoint{1.084319in}{1.438166in}}%
\pgfpathlineto{\pgfqpoint{1.096650in}{1.430031in}}%
\pgfpathlineto{\pgfqpoint{1.124211in}{1.411906in}}%
\pgfpathlineto{\pgfqpoint{1.131248in}{1.407339in}}%
\pgfpathlineto{\pgfqpoint{1.164771in}{1.385647in}}%
\pgfpathlineto{\pgfqpoint{1.165847in}{1.384960in}}%
\pgfpathlineto{\pgfqpoint{1.200445in}{1.362952in}}%
\pgfpathlineto{\pgfqpoint{1.206066in}{1.359388in}}%
\pgfpathlineto{\pgfqpoint{1.235044in}{1.341250in}}%
\pgfpathlineto{\pgfqpoint{1.248056in}{1.333128in}}%
\pgfpathlineto{\pgfqpoint{1.269642in}{1.319829in}}%
\pgfpathlineto{\pgfqpoint{1.290733in}{1.306869in}}%
\pgfpathlineto{\pgfqpoint{1.304241in}{1.298675in}}%
\pgfpathlineto{\pgfqpoint{1.334100in}{1.280609in}}%
\pgfpathlineto{\pgfqpoint{1.338839in}{1.277778in}}%
\pgfpathlineto{\pgfqpoint{1.373438in}{1.257181in}}%
\pgfpathlineto{\pgfqpoint{1.378210in}{1.254350in}}%
\pgfpathlineto{\pgfqpoint{1.408036in}{1.236879in}}%
\pgfpathlineto{\pgfqpoint{1.423072in}{1.228090in}}%
\pgfpathlineto{\pgfqpoint{1.442635in}{1.216803in}}%
\pgfpathlineto{\pgfqpoint{1.468637in}{1.201831in}}%
\pgfpathlineto{\pgfqpoint{1.477233in}{1.196944in}}%
\pgfpathlineto{\pgfqpoint{1.511832in}{1.177326in}}%
\pgfpathlineto{\pgfqpoint{1.514938in}{1.175571in}}%
\pgfpathlineto{\pgfqpoint{1.546430in}{1.158005in}}%
\pgfpathlineto{\pgfqpoint{1.562042in}{1.149312in}}%
\pgfpathlineto{\pgfqpoint{1.581029in}{1.138875in}}%
\pgfpathlineto{\pgfqpoint{1.609855in}{1.123052in}}%
\pgfpathlineto{\pgfqpoint{1.615627in}{1.119925in}}%
\pgfpathlineto{\pgfqpoint{1.650226in}{1.101233in}}%
\pgfpathlineto{\pgfqpoint{1.658465in}{1.096793in}}%
\pgfpathlineto{\pgfqpoint{1.684824in}{1.082770in}}%
\pgfpathlineto{\pgfqpoint{1.707851in}{1.070533in}}%
\pgfpathlineto{\pgfqpoint{1.719423in}{1.064463in}}%
\pgfpathlineto{\pgfqpoint{1.754021in}{1.046346in}}%
\pgfpathlineto{\pgfqpoint{1.757990in}{1.044274in}}%
\pgfpathlineto{\pgfqpoint{1.788620in}{1.028491in}}%
\pgfpathlineto{\pgfqpoint{1.808968in}{1.018014in}}%
\pgfpathlineto{\pgfqpoint{1.823218in}{1.010772in}}%
\pgfpathlineto{\pgfqpoint{1.857817in}{0.993208in}}%
\pgfpathlineto{\pgfqpoint{1.860689in}{0.991755in}}%
\pgfpathlineto{\pgfqpoint{1.892415in}{0.975910in}}%
\pgfpathlineto{\pgfqpoint{1.913276in}{0.965495in}}%
\pgfpathlineto{\pgfqpoint{1.927014in}{0.958726in}}%
\pgfpathlineto{\pgfqpoint{1.961612in}{0.941697in}}%
\pgfpathlineto{\pgfqpoint{1.966625in}{0.939236in}}%
\pgfpathlineto{\pgfqpoint{1.996211in}{0.924903in}}%
\pgfpathlineto{\pgfqpoint{2.020831in}{0.912976in}}%
\pgfpathlineto{\pgfqpoint{2.030809in}{0.908205in}}%
\pgfpathlineto{\pgfqpoint{2.065408in}{0.891692in}}%
\pgfpathlineto{\pgfqpoint{2.075849in}{0.886717in}}%
\pgfpathlineto{\pgfqpoint{2.100006in}{0.875357in}}%
\pgfpathlineto{\pgfqpoint{2.131678in}{0.860458in}}%
\pgfpathlineto{\pgfqpoint{2.134605in}{0.859099in}}%
\pgfpathlineto{\pgfqpoint{2.169203in}{0.843084in}}%
\pgfpathlineto{\pgfqpoint{2.188401in}{0.834198in}}%
\pgfpathlineto{\pgfqpoint{2.203802in}{0.827162in}}%
\pgfpathlineto{\pgfqpoint{2.238400in}{0.811368in}}%
\pgfpathlineto{\pgfqpoint{2.245928in}{0.807939in}}%
\pgfpathlineto{\pgfqpoint{2.272999in}{0.795768in}}%
\pgfpathlineto{\pgfqpoint{2.304308in}{0.781679in}}%
\pgfpathlineto{\pgfqpoint{2.307597in}{0.780219in}}%
\pgfpathlineto{\pgfqpoint{2.342196in}{0.764898in}}%
\pgfpathlineto{\pgfqpoint{2.363589in}{0.755420in}}%
\pgfpathlineto{\pgfqpoint{2.368870in}{0.753110in}}%
\pgfusepath{stroke}%
\end{pgfscope}%
\begin{pgfscope}%
\pgfpathrectangle{\pgfqpoint{0.854460in}{0.571603in}}{\pgfqpoint{6.885100in}{5.225635in}}%
\pgfusepath{clip}%
\pgfsetbuttcap%
\pgfsetroundjoin%
\pgfsetlinewidth{1.505625pt}%
\definecolor{currentstroke}{rgb}{0.131172,0.555899,0.552459}%
\pgfsetstrokecolor{currentstroke}%
\pgfsetdash{}{0pt}%
\pgfpathmoveto{\pgfqpoint{2.725308in}{0.602827in}}%
\pgfpathlineto{\pgfqpoint{2.737514in}{0.597863in}}%
\pgfpathlineto{\pgfqpoint{2.757377in}{0.589892in}}%
\pgfpathlineto{\pgfqpoint{2.791976in}{0.576002in}}%
\pgfpathlineto{\pgfqpoint{2.802946in}{0.571603in}}%
\pgfusepath{stroke}%
\end{pgfscope}%
\begin{pgfscope}%
\pgfpathrectangle{\pgfqpoint{0.854460in}{0.571603in}}{\pgfqpoint{6.885100in}{5.225635in}}%
\pgfusepath{clip}%
\pgfsetbuttcap%
\pgfsetroundjoin%
\pgfsetlinewidth{1.505625pt}%
\definecolor{currentstroke}{rgb}{0.125394,0.574318,0.549086}%
\pgfsetstrokecolor{currentstroke}%
\pgfsetdash{}{0pt}%
\pgfpathmoveto{\pgfqpoint{0.854460in}{1.532026in}}%
\pgfpathlineto{\pgfqpoint{0.875642in}{1.516944in}}%
\pgfpathlineto{\pgfqpoint{0.889059in}{1.507514in}}%
\pgfpathlineto{\pgfqpoint{0.913096in}{1.490685in}}%
\pgfpathlineto{\pgfqpoint{0.923657in}{1.483385in}}%
\pgfpathlineto{\pgfqpoint{0.951194in}{1.464425in}}%
\pgfpathlineto{\pgfqpoint{0.958256in}{1.459626in}}%
\pgfpathlineto{\pgfqpoint{0.989943in}{1.438166in}}%
\pgfpathlineto{\pgfqpoint{0.992854in}{1.436220in}}%
\pgfpathlineto{\pgfqpoint{1.027453in}{1.413178in}}%
\pgfpathlineto{\pgfqpoint{1.029369in}{1.411906in}}%
\pgfpathlineto{\pgfqpoint{1.062051in}{1.390506in}}%
\pgfpathlineto{\pgfqpoint{1.069496in}{1.385647in}}%
\pgfpathlineto{\pgfqpoint{1.096650in}{1.368151in}}%
\pgfpathlineto{\pgfqpoint{1.110294in}{1.359388in}}%
\pgfpathlineto{\pgfqpoint{1.131248in}{1.346101in}}%
\pgfpathlineto{\pgfqpoint{1.151770in}{1.333128in}}%
\pgfpathlineto{\pgfqpoint{1.165847in}{1.324343in}}%
\pgfpathlineto{\pgfqpoint{1.193925in}{1.306869in}}%
\pgfpathlineto{\pgfqpoint{1.200445in}{1.302863in}}%
\pgfpathlineto{\pgfqpoint{1.235044in}{1.281671in}}%
\pgfpathlineto{\pgfqpoint{1.236783in}{1.280609in}}%
\pgfpathlineto{\pgfqpoint{1.269642in}{1.260817in}}%
\pgfpathlineto{\pgfqpoint{1.280406in}{1.254350in}}%
\pgfpathlineto{\pgfqpoint{1.304241in}{1.240212in}}%
\pgfpathlineto{\pgfqpoint{1.324724in}{1.228090in}}%
\pgfpathlineto{\pgfqpoint{1.338839in}{1.219844in}}%
\pgfpathlineto{\pgfqpoint{1.369739in}{1.201831in}}%
\pgfpathlineto{\pgfqpoint{1.373438in}{1.199702in}}%
\pgfpathlineto{\pgfqpoint{1.408036in}{1.179859in}}%
\pgfpathlineto{\pgfqpoint{1.415533in}{1.175571in}}%
\pgfpathlineto{\pgfqpoint{1.442635in}{1.160268in}}%
\pgfpathlineto{\pgfqpoint{1.462073in}{1.149312in}}%
\pgfpathlineto{\pgfqpoint{1.477233in}{1.140876in}}%
\pgfpathlineto{\pgfqpoint{1.509319in}{1.123052in}}%
\pgfpathlineto{\pgfqpoint{1.511832in}{1.121674in}}%
\pgfpathlineto{\pgfqpoint{1.546430in}{1.102768in}}%
\pgfpathlineto{\pgfqpoint{1.557385in}{1.096793in}}%
\pgfpathlineto{\pgfqpoint{1.581029in}{1.084063in}}%
\pgfpathlineto{\pgfqpoint{1.606191in}{1.070533in}}%
\pgfpathlineto{\pgfqpoint{1.615627in}{1.065524in}}%
\pgfpathlineto{\pgfqpoint{1.650226in}{1.047199in}}%
\pgfpathlineto{\pgfqpoint{1.655764in}{1.044274in}}%
\pgfpathlineto{\pgfqpoint{1.684824in}{1.029123in}}%
\pgfpathlineto{\pgfqpoint{1.706151in}{1.018014in}}%
\pgfpathlineto{\pgfqpoint{1.719423in}{1.011190in}}%
\pgfpathlineto{\pgfqpoint{1.754021in}{0.993425in}}%
\pgfpathlineto{\pgfqpoint{1.757285in}{0.991755in}}%
\pgfpathlineto{\pgfqpoint{1.788620in}{0.975924in}}%
\pgfpathlineto{\pgfqpoint{1.809272in}{0.965495in}}%
\pgfpathlineto{\pgfqpoint{1.823218in}{0.958544in}}%
\pgfpathlineto{\pgfqpoint{1.856503in}{0.941973in}}%
\pgfusepath{stroke}%
\end{pgfscope}%
\begin{pgfscope}%
\pgfpathrectangle{\pgfqpoint{0.854460in}{0.571603in}}{\pgfqpoint{6.885100in}{5.225635in}}%
\pgfusepath{clip}%
\pgfsetbuttcap%
\pgfsetroundjoin%
\pgfsetlinewidth{1.505625pt}%
\definecolor{currentstroke}{rgb}{0.125394,0.574318,0.549086}%
\pgfsetstrokecolor{currentstroke}%
\pgfsetdash{}{0pt}%
\pgfpathmoveto{\pgfqpoint{2.206575in}{0.776912in}}%
\pgfpathlineto{\pgfqpoint{2.238400in}{0.762656in}}%
\pgfpathlineto{\pgfqpoint{2.254565in}{0.755420in}}%
\pgfpathlineto{\pgfqpoint{2.272999in}{0.747274in}}%
\pgfpathlineto{\pgfqpoint{2.307597in}{0.731988in}}%
\pgfpathlineto{\pgfqpoint{2.314013in}{0.729160in}}%
\pgfpathlineto{\pgfqpoint{2.342196in}{0.716898in}}%
\pgfpathlineto{\pgfqpoint{2.374329in}{0.702901in}}%
\pgfpathlineto{\pgfqpoint{2.376794in}{0.701841in}}%
\pgfpathlineto{\pgfqpoint{2.411393in}{0.687009in}}%
\pgfpathlineto{\pgfqpoint{2.435555in}{0.676641in}}%
\pgfpathlineto{\pgfqpoint{2.445991in}{0.672221in}}%
\pgfpathlineto{\pgfqpoint{2.480590in}{0.657591in}}%
\pgfpathlineto{\pgfqpoint{2.497641in}{0.650382in}}%
\pgfpathlineto{\pgfqpoint{2.515188in}{0.643060in}}%
\pgfpathlineto{\pgfqpoint{2.549787in}{0.628627in}}%
\pgfpathlineto{\pgfqpoint{2.560598in}{0.624122in}}%
\pgfpathlineto{\pgfqpoint{2.584385in}{0.614342in}}%
\pgfpathlineto{\pgfqpoint{2.618983in}{0.600102in}}%
\pgfpathlineto{\pgfqpoint{2.624435in}{0.597863in}}%
\pgfpathlineto{\pgfqpoint{2.653582in}{0.586051in}}%
\pgfpathlineto{\pgfqpoint{2.688180in}{0.572001in}}%
\pgfpathlineto{\pgfqpoint{2.689162in}{0.571603in}}%
\pgfusepath{stroke}%
\end{pgfscope}%
\begin{pgfscope}%
\pgfpathrectangle{\pgfqpoint{0.854460in}{0.571603in}}{\pgfqpoint{6.885100in}{5.225635in}}%
\pgfusepath{clip}%
\pgfsetbuttcap%
\pgfsetroundjoin%
\pgfsetlinewidth{1.505625pt}%
\definecolor{currentstroke}{rgb}{0.120565,0.596422,0.543611}%
\pgfsetstrokecolor{currentstroke}%
\pgfsetdash{}{0pt}%
\pgfpathmoveto{\pgfqpoint{0.854460in}{1.468327in}}%
\pgfpathlineto{\pgfqpoint{0.860076in}{1.464425in}}%
\pgfpathlineto{\pgfqpoint{0.889059in}{1.444549in}}%
\pgfpathlineto{\pgfqpoint{0.898401in}{1.438166in}}%
\pgfpathlineto{\pgfqpoint{0.923657in}{1.421130in}}%
\pgfpathlineto{\pgfqpoint{0.937381in}{1.411906in}}%
\pgfpathlineto{\pgfqpoint{0.958256in}{1.398055in}}%
\pgfpathlineto{\pgfqpoint{0.977021in}{1.385647in}}%
\pgfpathlineto{\pgfqpoint{0.992854in}{1.375310in}}%
\pgfpathlineto{\pgfqpoint{1.017325in}{1.359388in}}%
\pgfpathlineto{\pgfqpoint{1.027453in}{1.352881in}}%
\pgfpathlineto{\pgfqpoint{1.058298in}{1.333128in}}%
\pgfpathlineto{\pgfqpoint{1.062051in}{1.330755in}}%
\pgfpathlineto{\pgfqpoint{1.096650in}{1.308960in}}%
\pgfpathlineto{\pgfqpoint{1.099981in}{1.306869in}}%
\pgfpathlineto{\pgfqpoint{1.131248in}{1.287493in}}%
\pgfpathlineto{\pgfqpoint{1.142389in}{1.280609in}}%
\pgfpathlineto{\pgfqpoint{1.165847in}{1.266298in}}%
\pgfpathlineto{\pgfqpoint{1.185484in}{1.254350in}}%
\pgfpathlineto{\pgfqpoint{1.200445in}{1.245362in}}%
\pgfpathlineto{\pgfqpoint{1.229268in}{1.228090in}}%
\pgfpathlineto{\pgfqpoint{1.235044in}{1.224673in}}%
\pgfpathlineto{\pgfqpoint{1.269642in}{1.204268in}}%
\pgfpathlineto{\pgfqpoint{1.273788in}{1.201831in}}%
\pgfpathlineto{\pgfqpoint{1.304241in}{1.184159in}}%
\pgfpathlineto{\pgfqpoint{1.319071in}{1.175571in}}%
\pgfpathlineto{\pgfqpoint{1.338839in}{1.164270in}}%
\pgfpathlineto{\pgfqpoint{1.365055in}{1.149312in}}%
\pgfpathlineto{\pgfqpoint{1.373438in}{1.144590in}}%
\pgfpathlineto{\pgfqpoint{1.408036in}{1.125150in}}%
\pgfpathlineto{\pgfqpoint{1.411781in}{1.123052in}}%
\pgfpathlineto{\pgfqpoint{1.442635in}{1.105994in}}%
\pgfpathlineto{\pgfqpoint{1.459305in}{1.096793in}}%
\pgfpathlineto{\pgfqpoint{1.477233in}{1.087023in}}%
\pgfpathlineto{\pgfqpoint{1.507538in}{1.070533in}}%
\pgfpathlineto{\pgfqpoint{1.511832in}{1.068226in}}%
\pgfpathlineto{\pgfqpoint{1.546430in}{1.049699in}}%
\pgfpathlineto{\pgfqpoint{1.556583in}{1.044274in}}%
\pgfpathlineto{\pgfqpoint{1.581029in}{1.031377in}}%
\pgfpathlineto{\pgfqpoint{1.606386in}{1.018014in}}%
\pgfpathlineto{\pgfqpoint{1.615627in}{1.013206in}}%
\pgfpathlineto{\pgfqpoint{1.650226in}{0.995246in}}%
\pgfpathlineto{\pgfqpoint{1.656968in}{0.991755in}}%
\pgfpathlineto{\pgfqpoint{1.684824in}{0.977517in}}%
\pgfpathlineto{\pgfqpoint{1.708361in}{0.965495in}}%
\pgfpathlineto{\pgfqpoint{1.719423in}{0.959918in}}%
\pgfpathlineto{\pgfqpoint{1.754021in}{0.942504in}}%
\pgfpathlineto{\pgfqpoint{1.760532in}{0.939236in}}%
\pgfpathlineto{\pgfqpoint{1.788620in}{0.925316in}}%
\pgfpathlineto{\pgfqpoint{1.813529in}{0.912976in}}%
\pgfpathlineto{\pgfqpoint{1.823218in}{0.908238in}}%
\pgfpathlineto{\pgfqpoint{1.857817in}{0.891352in}}%
\pgfpathlineto{\pgfqpoint{1.858811in}{0.890867in}}%
\pgfusepath{stroke}%
\end{pgfscope}%
\begin{pgfscope}%
\pgfpathrectangle{\pgfqpoint{0.854460in}{0.571603in}}{\pgfqpoint{6.885100in}{5.225635in}}%
\pgfusepath{clip}%
\pgfsetbuttcap%
\pgfsetroundjoin%
\pgfsetlinewidth{1.505625pt}%
\definecolor{currentstroke}{rgb}{0.120565,0.596422,0.543611}%
\pgfsetstrokecolor{currentstroke}%
\pgfsetdash{}{0pt}%
\pgfpathmoveto{\pgfqpoint{2.209936in}{0.728139in}}%
\pgfpathlineto{\pgfqpoint{2.238400in}{0.715624in}}%
\pgfpathlineto{\pgfqpoint{2.267310in}{0.702901in}}%
\pgfpathlineto{\pgfqpoint{2.272999in}{0.700429in}}%
\pgfpathlineto{\pgfqpoint{2.307597in}{0.685435in}}%
\pgfpathlineto{\pgfqpoint{2.327884in}{0.676641in}}%
\pgfpathlineto{\pgfqpoint{2.342196in}{0.670518in}}%
\pgfpathlineto{\pgfqpoint{2.376794in}{0.655727in}}%
\pgfpathlineto{\pgfqpoint{2.389311in}{0.650382in}}%
\pgfpathlineto{\pgfqpoint{2.411393in}{0.641074in}}%
\pgfpathlineto{\pgfqpoint{2.445991in}{0.626483in}}%
\pgfpathlineto{\pgfqpoint{2.451601in}{0.624122in}}%
\pgfpathlineto{\pgfqpoint{2.480590in}{0.612083in}}%
\pgfpathlineto{\pgfqpoint{2.514770in}{0.597863in}}%
\pgfpathlineto{\pgfqpoint{2.515188in}{0.597691in}}%
\pgfpathlineto{\pgfqpoint{2.549787in}{0.583529in}}%
\pgfpathlineto{\pgfqpoint{2.578866in}{0.571603in}}%
\pgfusepath{stroke}%
\end{pgfscope}%
\begin{pgfscope}%
\pgfpathrectangle{\pgfqpoint{0.854460in}{0.571603in}}{\pgfqpoint{6.885100in}{5.225635in}}%
\pgfusepath{clip}%
\pgfsetbuttcap%
\pgfsetroundjoin%
\pgfsetlinewidth{1.505625pt}%
\definecolor{currentstroke}{rgb}{0.119699,0.618490,0.536347}%
\pgfsetstrokecolor{currentstroke}%
\pgfsetdash{}{0pt}%
\pgfpathmoveto{\pgfqpoint{0.854460in}{1.407535in}}%
\pgfpathlineto{\pgfqpoint{0.887127in}{1.385647in}}%
\pgfpathlineto{\pgfqpoint{0.889059in}{1.384369in}}%
\pgfpathlineto{\pgfqpoint{0.923657in}{1.361573in}}%
\pgfpathlineto{\pgfqpoint{0.926986in}{1.359388in}}%
\pgfpathlineto{\pgfqpoint{0.958256in}{1.339122in}}%
\pgfpathlineto{\pgfqpoint{0.967535in}{1.333128in}}%
\pgfpathlineto{\pgfqpoint{0.992854in}{1.316980in}}%
\pgfpathlineto{\pgfqpoint{1.008757in}{1.306869in}}%
\pgfpathlineto{\pgfqpoint{1.027453in}{1.295132in}}%
\pgfpathlineto{\pgfqpoint{1.050656in}{1.280609in}}%
\pgfpathlineto{\pgfqpoint{1.062051in}{1.273566in}}%
\pgfpathlineto{\pgfqpoint{1.093234in}{1.254350in}}%
\pgfpathlineto{\pgfqpoint{1.096650in}{1.252271in}}%
\pgfpathlineto{\pgfqpoint{1.131248in}{1.231296in}}%
\pgfpathlineto{\pgfqpoint{1.136553in}{1.228090in}}%
\pgfpathlineto{\pgfqpoint{1.165847in}{1.210612in}}%
\pgfpathlineto{\pgfqpoint{1.180600in}{1.201831in}}%
\pgfpathlineto{\pgfqpoint{1.200445in}{1.190168in}}%
\pgfpathlineto{\pgfqpoint{1.225343in}{1.175571in}}%
\pgfpathlineto{\pgfqpoint{1.235044in}{1.169955in}}%
\pgfpathlineto{\pgfqpoint{1.269642in}{1.149974in}}%
\pgfpathlineto{\pgfqpoint{1.270794in}{1.149312in}}%
\pgfpathlineto{\pgfqpoint{1.304241in}{1.130314in}}%
\pgfpathlineto{\pgfqpoint{1.317052in}{1.123052in}}%
\pgfpathlineto{\pgfqpoint{1.338839in}{1.110858in}}%
\pgfpathlineto{\pgfqpoint{1.364015in}{1.096793in}}%
\pgfpathlineto{\pgfqpoint{1.373438in}{1.091595in}}%
\pgfpathlineto{\pgfqpoint{1.388783in}{1.083150in}}%
\pgfusepath{stroke}%
\end{pgfscope}%
\begin{pgfscope}%
\pgfpathrectangle{\pgfqpoint{0.854460in}{0.571603in}}{\pgfqpoint{6.885100in}{5.225635in}}%
\pgfusepath{clip}%
\pgfsetbuttcap%
\pgfsetroundjoin%
\pgfsetlinewidth{1.505625pt}%
\definecolor{currentstroke}{rgb}{0.119699,0.618490,0.536347}%
\pgfsetstrokecolor{currentstroke}%
\pgfsetdash{}{0pt}%
\pgfpathmoveto{\pgfqpoint{1.732191in}{0.904177in}}%
\pgfpathlineto{\pgfqpoint{1.754021in}{0.893410in}}%
\pgfpathlineto{\pgfqpoint{1.767610in}{0.886717in}}%
\pgfpathlineto{\pgfqpoint{1.788620in}{0.876500in}}%
\pgfpathlineto{\pgfqpoint{1.821617in}{0.860458in}}%
\pgfpathlineto{\pgfqpoint{1.823218in}{0.859689in}}%
\pgfpathlineto{\pgfqpoint{1.857817in}{0.843138in}}%
\pgfpathlineto{\pgfqpoint{1.876513in}{0.834198in}}%
\pgfpathlineto{\pgfqpoint{1.892415in}{0.826691in}}%
\pgfpathlineto{\pgfqpoint{1.927014in}{0.810371in}}%
\pgfpathlineto{\pgfqpoint{1.932184in}{0.807939in}}%
\pgfpathlineto{\pgfqpoint{1.961612in}{0.794271in}}%
\pgfpathlineto{\pgfqpoint{1.988718in}{0.781679in}}%
\pgfpathlineto{\pgfqpoint{1.996211in}{0.778243in}}%
\pgfpathlineto{\pgfqpoint{2.030809in}{0.762410in}}%
\pgfpathlineto{\pgfqpoint{2.046099in}{0.755420in}}%
\pgfpathlineto{\pgfqpoint{2.065408in}{0.746704in}}%
\pgfpathlineto{\pgfqpoint{2.100006in}{0.731090in}}%
\pgfpathlineto{\pgfqpoint{2.104294in}{0.729160in}}%
\pgfpathlineto{\pgfqpoint{2.134605in}{0.715692in}}%
\pgfpathlineto{\pgfqpoint{2.163371in}{0.702901in}}%
\pgfpathlineto{\pgfqpoint{2.169203in}{0.700340in}}%
\pgfpathlineto{\pgfqpoint{2.203802in}{0.685189in}}%
\pgfpathlineto{\pgfqpoint{2.223321in}{0.676641in}}%
\pgfpathlineto{\pgfqpoint{2.238400in}{0.670122in}}%
\pgfpathlineto{\pgfqpoint{2.272999in}{0.655177in}}%
\pgfpathlineto{\pgfqpoint{2.284116in}{0.650382in}}%
\pgfpathlineto{\pgfqpoint{2.307597in}{0.640383in}}%
\pgfpathlineto{\pgfqpoint{2.342196in}{0.625640in}}%
\pgfpathlineto{\pgfqpoint{2.345767in}{0.624122in}}%
\pgfpathlineto{\pgfqpoint{2.376794in}{0.611106in}}%
\pgfpathlineto{\pgfqpoint{2.408314in}{0.597863in}}%
\pgfpathlineto{\pgfqpoint{2.411393in}{0.596586in}}%
\pgfpathlineto{\pgfqpoint{2.445991in}{0.582276in}}%
\pgfpathlineto{\pgfqpoint{2.471764in}{0.571603in}}%
\pgfusepath{stroke}%
\end{pgfscope}%
\begin{pgfscope}%
\pgfpathrectangle{\pgfqpoint{0.854460in}{0.571603in}}{\pgfqpoint{6.885100in}{5.225635in}}%
\pgfusepath{clip}%
\pgfsetbuttcap%
\pgfsetroundjoin%
\pgfsetlinewidth{1.505625pt}%
\definecolor{currentstroke}{rgb}{0.123444,0.636809,0.528763}%
\pgfsetstrokecolor{currentstroke}%
\pgfsetdash{}{0pt}%
\pgfpathmoveto{\pgfqpoint{0.854460in}{1.349356in}}%
\pgfpathlineto{\pgfqpoint{0.879255in}{1.333128in}}%
\pgfpathlineto{\pgfqpoint{0.889059in}{1.326792in}}%
\pgfpathlineto{\pgfqpoint{0.919988in}{1.306869in}}%
\pgfpathlineto{\pgfqpoint{0.923657in}{1.304534in}}%
\pgfpathlineto{\pgfqpoint{0.925001in}{1.303683in}}%
\pgfusepath{stroke}%
\end{pgfscope}%
\begin{pgfscope}%
\pgfpathrectangle{\pgfqpoint{0.854460in}{0.571603in}}{\pgfqpoint{6.885100in}{5.225635in}}%
\pgfusepath{clip}%
\pgfsetbuttcap%
\pgfsetroundjoin%
\pgfsetlinewidth{1.505625pt}%
\definecolor{currentstroke}{rgb}{0.123444,0.636809,0.528763}%
\pgfsetstrokecolor{currentstroke}%
\pgfsetdash{}{0pt}%
\pgfpathmoveto{\pgfqpoint{1.257542in}{1.104589in}}%
\pgfpathlineto{\pgfqpoint{1.269642in}{1.097748in}}%
\pgfpathlineto{\pgfqpoint{1.271338in}{1.096793in}}%
\pgfpathlineto{\pgfqpoint{1.304241in}{1.078493in}}%
\pgfpathlineto{\pgfqpoint{1.318579in}{1.070533in}}%
\pgfpathlineto{\pgfqpoint{1.338839in}{1.059427in}}%
\pgfpathlineto{\pgfqpoint{1.366528in}{1.044274in}}%
\pgfpathlineto{\pgfqpoint{1.373438in}{1.040540in}}%
\pgfpathlineto{\pgfqpoint{1.408036in}{1.021895in}}%
\pgfpathlineto{\pgfqpoint{1.415257in}{1.018014in}}%
\pgfpathlineto{\pgfqpoint{1.442635in}{1.003485in}}%
\pgfpathlineto{\pgfqpoint{1.464769in}{0.991755in}}%
\pgfpathlineto{\pgfqpoint{1.477233in}{0.985232in}}%
\pgfpathlineto{\pgfqpoint{1.511832in}{0.967157in}}%
\pgfpathlineto{\pgfqpoint{1.515023in}{0.965495in}}%
\pgfpathlineto{\pgfqpoint{1.546430in}{0.949348in}}%
\pgfpathlineto{\pgfqpoint{1.566119in}{0.939236in}}%
\pgfpathlineto{\pgfqpoint{1.581029in}{0.931674in}}%
\pgfpathlineto{\pgfqpoint{1.615627in}{0.914149in}}%
\pgfpathlineto{\pgfqpoint{1.617950in}{0.912976in}}%
\pgfpathlineto{\pgfqpoint{1.650226in}{0.896888in}}%
\pgfpathlineto{\pgfqpoint{1.670644in}{0.886717in}}%
\pgfpathlineto{\pgfqpoint{1.684824in}{0.879742in}}%
\pgfpathlineto{\pgfqpoint{1.719423in}{0.862747in}}%
\pgfpathlineto{\pgfqpoint{1.724097in}{0.860458in}}%
\pgfpathlineto{\pgfqpoint{1.754021in}{0.845986in}}%
\pgfpathlineto{\pgfqpoint{1.778402in}{0.834198in}}%
\pgfpathlineto{\pgfqpoint{1.788620in}{0.829320in}}%
\pgfpathlineto{\pgfqpoint{1.823218in}{0.812836in}}%
\pgfpathlineto{\pgfqpoint{1.833515in}{0.807939in}}%
\pgfpathlineto{\pgfqpoint{1.857817in}{0.796528in}}%
\pgfpathlineto{\pgfqpoint{1.889437in}{0.781679in}}%
\pgfpathlineto{\pgfqpoint{1.892415in}{0.780298in}}%
\pgfpathlineto{\pgfqpoint{1.927014in}{0.764306in}}%
\pgfpathlineto{\pgfqpoint{1.946245in}{0.755420in}}%
\pgfpathlineto{\pgfqpoint{1.961612in}{0.748408in}}%
\pgfpathlineto{\pgfqpoint{1.996211in}{0.732637in}}%
\pgfpathlineto{\pgfqpoint{2.003855in}{0.729160in}}%
\pgfpathlineto{\pgfqpoint{2.030809in}{0.717056in}}%
\pgfpathlineto{\pgfqpoint{2.062313in}{0.702901in}}%
\pgfpathlineto{\pgfqpoint{2.065408in}{0.701528in}}%
\pgfpathlineto{\pgfqpoint{2.100006in}{0.686224in}}%
\pgfpathlineto{\pgfqpoint{2.121666in}{0.676641in}}%
\pgfpathlineto{\pgfqpoint{2.134605in}{0.670989in}}%
\pgfpathlineto{\pgfqpoint{2.169203in}{0.655895in}}%
\pgfpathlineto{\pgfqpoint{2.181855in}{0.650382in}}%
\pgfpathlineto{\pgfqpoint{2.203802in}{0.640940in}}%
\pgfpathlineto{\pgfqpoint{2.238400in}{0.626051in}}%
\pgfpathlineto{\pgfqpoint{2.242892in}{0.624122in}}%
\pgfpathlineto{\pgfqpoint{2.272999in}{0.611364in}}%
\pgfpathlineto{\pgfqpoint{2.304816in}{0.597863in}}%
\pgfpathlineto{\pgfqpoint{2.307597in}{0.596698in}}%
\pgfpathlineto{\pgfqpoint{2.342196in}{0.582245in}}%
\pgfpathlineto{\pgfqpoint{2.367643in}{0.571603in}}%
\pgfusepath{stroke}%
\end{pgfscope}%
\begin{pgfscope}%
\pgfpathrectangle{\pgfqpoint{0.854460in}{0.571603in}}{\pgfqpoint{6.885100in}{5.225635in}}%
\pgfusepath{clip}%
\pgfsetbuttcap%
\pgfsetroundjoin%
\pgfsetlinewidth{1.505625pt}%
\definecolor{currentstroke}{rgb}{0.134692,0.658636,0.517649}%
\pgfsetstrokecolor{currentstroke}%
\pgfsetdash{}{0pt}%
\pgfpathmoveto{\pgfqpoint{0.854460in}{1.293507in}}%
\pgfpathlineto{\pgfqpoint{0.874625in}{1.280609in}}%
\pgfpathlineto{\pgfqpoint{0.889059in}{1.271492in}}%
\pgfpathlineto{\pgfqpoint{0.916283in}{1.254350in}}%
\pgfpathlineto{\pgfqpoint{0.923657in}{1.249764in}}%
\pgfpathlineto{\pgfqpoint{0.958256in}{1.228315in}}%
\pgfpathlineto{\pgfqpoint{0.958621in}{1.228090in}}%
\pgfpathlineto{\pgfqpoint{0.992854in}{1.207222in}}%
\pgfpathlineto{\pgfqpoint{1.001722in}{1.201831in}}%
\pgfpathlineto{\pgfqpoint{1.027453in}{1.186384in}}%
\pgfpathlineto{\pgfqpoint{1.045513in}{1.175571in}}%
\pgfpathlineto{\pgfqpoint{1.062051in}{1.165793in}}%
\pgfpathlineto{\pgfqpoint{1.089995in}{1.149312in}}%
\pgfpathlineto{\pgfqpoint{1.096650in}{1.145436in}}%
\pgfpathlineto{\pgfqpoint{1.131248in}{1.125345in}}%
\pgfpathlineto{\pgfqpoint{1.135211in}{1.123052in}}%
\pgfpathlineto{\pgfqpoint{1.165847in}{1.105546in}}%
\pgfpathlineto{\pgfqpoint{1.181198in}{1.096793in}}%
\pgfpathlineto{\pgfqpoint{1.200445in}{1.085954in}}%
\pgfpathlineto{\pgfqpoint{1.227886in}{1.070533in}}%
\pgfpathlineto{\pgfqpoint{1.235044in}{1.066561in}}%
\pgfpathlineto{\pgfqpoint{1.269642in}{1.047415in}}%
\pgfpathlineto{\pgfqpoint{1.275336in}{1.044274in}}%
\pgfpathlineto{\pgfqpoint{1.304241in}{1.028526in}}%
\pgfpathlineto{\pgfqpoint{1.323567in}{1.018014in}}%
\pgfpathlineto{\pgfqpoint{1.338839in}{1.009811in}}%
\pgfpathlineto{\pgfqpoint{1.372508in}{0.991755in}}%
\pgfpathlineto{\pgfqpoint{1.373438in}{0.991262in}}%
\pgfpathlineto{\pgfqpoint{1.404004in}{0.975136in}}%
\pgfusepath{stroke}%
\end{pgfscope}%
\begin{pgfscope}%
\pgfpathrectangle{\pgfqpoint{0.854460in}{0.571603in}}{\pgfqpoint{6.885100in}{5.225635in}}%
\pgfusepath{clip}%
\pgfsetbuttcap%
\pgfsetroundjoin%
\pgfsetlinewidth{1.505625pt}%
\definecolor{currentstroke}{rgb}{0.134692,0.658636,0.517649}%
\pgfsetstrokecolor{currentstroke}%
\pgfsetdash{}{0pt}%
\pgfpathmoveto{\pgfqpoint{1.750234in}{0.801885in}}%
\pgfpathlineto{\pgfqpoint{1.754021in}{0.800087in}}%
\pgfpathlineto{\pgfqpoint{1.788620in}{0.783674in}}%
\pgfpathlineto{\pgfqpoint{1.792837in}{0.781679in}}%
\pgfpathlineto{\pgfqpoint{1.823218in}{0.767489in}}%
\pgfpathlineto{\pgfqpoint{1.849058in}{0.755420in}}%
\pgfpathlineto{\pgfqpoint{1.857817in}{0.751380in}}%
\pgfpathlineto{\pgfqpoint{1.892415in}{0.735456in}}%
\pgfpathlineto{\pgfqpoint{1.906112in}{0.729160in}}%
\pgfpathlineto{\pgfqpoint{1.927014in}{0.719673in}}%
\pgfpathlineto{\pgfqpoint{1.961612in}{0.703969in}}%
\pgfpathlineto{\pgfqpoint{1.963974in}{0.702901in}}%
\pgfpathlineto{\pgfqpoint{1.996211in}{0.688498in}}%
\pgfpathlineto{\pgfqpoint{2.022733in}{0.676641in}}%
\pgfpathlineto{\pgfqpoint{2.030809in}{0.673077in}}%
\pgfpathlineto{\pgfqpoint{2.065408in}{0.657837in}}%
\pgfpathlineto{\pgfqpoint{2.082342in}{0.650382in}}%
\pgfpathlineto{\pgfqpoint{2.100006in}{0.642704in}}%
\pgfpathlineto{\pgfqpoint{2.134605in}{0.627672in}}%
\pgfpathlineto{\pgfqpoint{2.142791in}{0.624122in}}%
\pgfpathlineto{\pgfqpoint{2.169203in}{0.612815in}}%
\pgfpathlineto{\pgfqpoint{2.203802in}{0.597987in}}%
\pgfpathlineto{\pgfqpoint{2.204093in}{0.597863in}}%
\pgfpathlineto{\pgfqpoint{2.238400in}{0.583393in}}%
\pgfpathlineto{\pgfqpoint{2.266318in}{0.571603in}}%
\pgfusepath{stroke}%
\end{pgfscope}%
\begin{pgfscope}%
\pgfpathrectangle{\pgfqpoint{0.854460in}{0.571603in}}{\pgfqpoint{6.885100in}{5.225635in}}%
\pgfusepath{clip}%
\pgfsetbuttcap%
\pgfsetroundjoin%
\pgfsetlinewidth{1.505625pt}%
\definecolor{currentstroke}{rgb}{0.153894,0.680203,0.504172}%
\pgfsetstrokecolor{currentstroke}%
\pgfsetdash{}{0pt}%
\pgfpathmoveto{\pgfqpoint{0.854460in}{1.239796in}}%
\pgfpathlineto{\pgfqpoint{0.873179in}{1.228090in}}%
\pgfpathlineto{\pgfqpoint{0.889059in}{1.218283in}}%
\pgfpathlineto{\pgfqpoint{0.915776in}{1.201831in}}%
\pgfpathlineto{\pgfqpoint{0.923657in}{1.197038in}}%
\pgfpathlineto{\pgfqpoint{0.958256in}{1.176059in}}%
\pgfpathlineto{\pgfqpoint{0.959063in}{1.175571in}}%
\pgfpathlineto{\pgfqpoint{0.992854in}{1.155420in}}%
\pgfpathlineto{\pgfqpoint{1.003123in}{1.149312in}}%
\pgfpathlineto{\pgfqpoint{1.027453in}{1.135020in}}%
\pgfpathlineto{\pgfqpoint{1.047877in}{1.123052in}}%
\pgfpathlineto{\pgfqpoint{1.062051in}{1.114850in}}%
\pgfpathlineto{\pgfqpoint{1.093327in}{1.096793in}}%
\pgfpathlineto{\pgfqpoint{1.096650in}{1.094898in}}%
\pgfpathlineto{\pgfqpoint{1.131248in}{1.075241in}}%
\pgfpathlineto{\pgfqpoint{1.139557in}{1.070533in}}%
\pgfpathlineto{\pgfqpoint{1.165847in}{1.055822in}}%
\pgfpathlineto{\pgfqpoint{1.186525in}{1.044274in}}%
\pgfpathlineto{\pgfqpoint{1.200445in}{1.036596in}}%
\pgfpathlineto{\pgfqpoint{1.234199in}{1.018014in}}%
\pgfpathlineto{\pgfqpoint{1.235044in}{1.017555in}}%
\pgfpathlineto{\pgfqpoint{1.269642in}{0.998817in}}%
\pgfpathlineto{\pgfqpoint{1.282706in}{0.991755in}}%
\pgfpathlineto{\pgfqpoint{1.304241in}{0.980258in}}%
\pgfpathlineto{\pgfqpoint{1.331933in}{0.965495in}}%
\pgfpathlineto{\pgfqpoint{1.338839in}{0.961859in}}%
\pgfpathlineto{\pgfqpoint{1.373438in}{0.943695in}}%
\pgfpathlineto{\pgfqpoint{1.381952in}{0.939236in}}%
\pgfpathlineto{\pgfqpoint{1.408036in}{0.925745in}}%
\pgfpathlineto{\pgfqpoint{1.408139in}{0.925692in}}%
\pgfusepath{stroke}%
\end{pgfscope}%
\begin{pgfscope}%
\pgfpathrectangle{\pgfqpoint{0.854460in}{0.571603in}}{\pgfqpoint{6.885100in}{5.225635in}}%
\pgfusepath{clip}%
\pgfsetbuttcap%
\pgfsetroundjoin%
\pgfsetlinewidth{1.505625pt}%
\definecolor{currentstroke}{rgb}{0.153894,0.680203,0.504172}%
\pgfsetstrokecolor{currentstroke}%
\pgfsetdash{}{0pt}%
\pgfpathmoveto{\pgfqpoint{1.755537in}{0.754879in}}%
\pgfpathlineto{\pgfqpoint{1.788620in}{0.739509in}}%
\pgfpathlineto{\pgfqpoint{1.810893in}{0.729160in}}%
\pgfpathlineto{\pgfqpoint{1.823218in}{0.723505in}}%
\pgfpathlineto{\pgfqpoint{1.857817in}{0.707655in}}%
\pgfpathlineto{\pgfqpoint{1.868214in}{0.702901in}}%
\pgfpathlineto{\pgfqpoint{1.892415in}{0.691972in}}%
\pgfpathlineto{\pgfqpoint{1.926353in}{0.676641in}}%
\pgfpathlineto{\pgfqpoint{1.927014in}{0.676346in}}%
\pgfpathlineto{\pgfqpoint{1.961612in}{0.660965in}}%
\pgfpathlineto{\pgfqpoint{1.985406in}{0.650382in}}%
\pgfpathlineto{\pgfqpoint{1.996211in}{0.645636in}}%
\pgfpathlineto{\pgfqpoint{2.030809in}{0.630465in}}%
\pgfpathlineto{\pgfqpoint{2.045290in}{0.624122in}}%
\pgfpathlineto{\pgfqpoint{2.065408in}{0.615421in}}%
\pgfpathlineto{\pgfqpoint{2.100006in}{0.600457in}}%
\pgfpathlineto{\pgfqpoint{2.106018in}{0.597863in}}%
\pgfpathlineto{\pgfqpoint{2.134605in}{0.585683in}}%
\pgfpathlineto{\pgfqpoint{2.167616in}{0.571603in}}%
\pgfusepath{stroke}%
\end{pgfscope}%
\begin{pgfscope}%
\pgfpathrectangle{\pgfqpoint{0.854460in}{0.571603in}}{\pgfqpoint{6.885100in}{5.225635in}}%
\pgfusepath{clip}%
\pgfsetbuttcap%
\pgfsetroundjoin%
\pgfsetlinewidth{1.505625pt}%
\definecolor{currentstroke}{rgb}{0.175707,0.697900,0.491033}%
\pgfsetstrokecolor{currentstroke}%
\pgfsetdash{}{0pt}%
\pgfpathmoveto{\pgfqpoint{0.854460in}{1.188050in}}%
\pgfpathlineto{\pgfqpoint{0.874861in}{1.175571in}}%
\pgfpathlineto{\pgfqpoint{0.889059in}{1.166994in}}%
\pgfpathlineto{\pgfqpoint{0.918409in}{1.149312in}}%
\pgfpathlineto{\pgfqpoint{0.923657in}{1.146189in}}%
\pgfpathlineto{\pgfqpoint{0.958256in}{1.125671in}}%
\pgfpathlineto{\pgfqpoint{0.962688in}{1.123052in}}%
\pgfpathlineto{\pgfqpoint{0.992854in}{1.105446in}}%
\pgfpathlineto{\pgfqpoint{1.007717in}{1.096793in}}%
\pgfpathlineto{\pgfqpoint{1.027453in}{1.085444in}}%
\pgfpathlineto{\pgfqpoint{1.053443in}{1.070533in}}%
\pgfpathlineto{\pgfqpoint{1.062051in}{1.065655in}}%
\pgfpathlineto{\pgfqpoint{1.096650in}{1.046104in}}%
\pgfpathlineto{\pgfqpoint{1.099900in}{1.044274in}}%
\pgfpathlineto{\pgfqpoint{1.131248in}{1.026840in}}%
\pgfpathlineto{\pgfqpoint{1.147150in}{1.018014in}}%
\pgfpathlineto{\pgfqpoint{1.165847in}{1.007765in}}%
\pgfpathlineto{\pgfqpoint{1.195106in}{0.991755in}}%
\pgfpathlineto{\pgfqpoint{1.200445in}{0.988869in}}%
\pgfpathlineto{\pgfqpoint{1.235044in}{0.970230in}}%
\pgfpathlineto{\pgfqpoint{1.243855in}{0.965495in}}%
\pgfpathlineto{\pgfqpoint{1.269642in}{0.951809in}}%
\pgfpathlineto{\pgfqpoint{1.293369in}{0.939236in}}%
\pgfpathlineto{\pgfqpoint{1.304241in}{0.933546in}}%
\pgfpathlineto{\pgfqpoint{1.338839in}{0.915476in}}%
\pgfpathlineto{\pgfqpoint{1.343640in}{0.912976in}}%
\pgfpathlineto{\pgfqpoint{1.373438in}{0.897654in}}%
\pgfpathlineto{\pgfqpoint{1.394732in}{0.886717in}}%
\pgfpathlineto{\pgfqpoint{1.407581in}{0.880199in}}%
\pgfusepath{stroke}%
\end{pgfscope}%
\begin{pgfscope}%
\pgfpathrectangle{\pgfqpoint{0.854460in}{0.571603in}}{\pgfqpoint{6.885100in}{5.225635in}}%
\pgfusepath{clip}%
\pgfsetbuttcap%
\pgfsetroundjoin%
\pgfsetlinewidth{1.505625pt}%
\definecolor{currentstroke}{rgb}{0.175707,0.697900,0.491033}%
\pgfsetstrokecolor{currentstroke}%
\pgfsetdash{}{0pt}%
\pgfpathmoveto{\pgfqpoint{1.756021in}{0.711599in}}%
\pgfpathlineto{\pgfqpoint{1.774841in}{0.702901in}}%
\pgfpathlineto{\pgfqpoint{1.788620in}{0.696612in}}%
\pgfpathlineto{\pgfqpoint{1.823218in}{0.680840in}}%
\pgfpathlineto{\pgfqpoint{1.832448in}{0.676641in}}%
\pgfpathlineto{\pgfqpoint{1.857817in}{0.665244in}}%
\pgfpathlineto{\pgfqpoint{1.890885in}{0.650382in}}%
\pgfpathlineto{\pgfqpoint{1.892415in}{0.649702in}}%
\pgfpathlineto{\pgfqpoint{1.927014in}{0.634396in}}%
\pgfpathlineto{\pgfqpoint{1.950227in}{0.624122in}}%
\pgfpathlineto{\pgfqpoint{1.961612in}{0.619147in}}%
\pgfpathlineto{\pgfqpoint{1.996211in}{0.604050in}}%
\pgfpathlineto{\pgfqpoint{2.010405in}{0.597863in}}%
\pgfpathlineto{\pgfqpoint{2.030809in}{0.589080in}}%
\pgfpathlineto{\pgfqpoint{2.065408in}{0.574189in}}%
\pgfpathlineto{\pgfqpoint{2.071430in}{0.571603in}}%
\pgfusepath{stroke}%
\end{pgfscope}%
\begin{pgfscope}%
\pgfpathrectangle{\pgfqpoint{0.854460in}{0.571603in}}{\pgfqpoint{6.885100in}{5.225635in}}%
\pgfusepath{clip}%
\pgfsetbuttcap%
\pgfsetroundjoin%
\pgfsetlinewidth{1.505625pt}%
\definecolor{currentstroke}{rgb}{0.208030,0.718701,0.472873}%
\pgfsetstrokecolor{currentstroke}%
\pgfsetdash{}{0pt}%
\pgfpathmoveto{\pgfqpoint{0.854460in}{1.138108in}}%
\pgfpathlineto{\pgfqpoint{0.879613in}{1.123052in}}%
\pgfpathlineto{\pgfqpoint{0.889059in}{1.117468in}}%
\pgfpathlineto{\pgfqpoint{0.923657in}{1.097068in}}%
\pgfpathlineto{\pgfqpoint{0.924126in}{1.096793in}}%
\pgfpathlineto{\pgfqpoint{0.958256in}{1.076999in}}%
\pgfpathlineto{\pgfqpoint{0.969432in}{1.070533in}}%
\pgfpathlineto{\pgfqpoint{0.992854in}{1.057149in}}%
\pgfpathlineto{\pgfqpoint{1.015437in}{1.044274in}}%
\pgfpathlineto{\pgfqpoint{1.027453in}{1.037507in}}%
\pgfpathlineto{\pgfqpoint{1.062051in}{1.018066in}}%
\pgfpathlineto{\pgfqpoint{1.062144in}{1.018014in}}%
\pgfpathlineto{\pgfqpoint{1.096650in}{0.998941in}}%
\pgfpathlineto{\pgfqpoint{1.109677in}{0.991755in}}%
\pgfpathlineto{\pgfqpoint{1.131248in}{0.980001in}}%
\pgfpathlineto{\pgfqpoint{1.157918in}{0.965495in}}%
\pgfpathlineto{\pgfqpoint{1.165847in}{0.961236in}}%
\pgfpathlineto{\pgfqpoint{1.200445in}{0.942700in}}%
\pgfpathlineto{\pgfqpoint{1.206930in}{0.939236in}}%
\pgfpathlineto{\pgfqpoint{1.235044in}{0.924402in}}%
\pgfpathlineto{\pgfqpoint{1.256732in}{0.912976in}}%
\pgfpathlineto{\pgfqpoint{1.269642in}{0.906258in}}%
\pgfpathlineto{\pgfqpoint{1.304241in}{0.888287in}}%
\pgfpathlineto{\pgfqpoint{1.307274in}{0.886717in}}%
\pgfpathlineto{\pgfqpoint{1.338839in}{0.870578in}}%
\pgfpathlineto{\pgfqpoint{1.358658in}{0.860458in}}%
\pgfpathlineto{\pgfqpoint{1.373438in}{0.853002in}}%
\pgfpathlineto{\pgfqpoint{1.407073in}{0.836061in}}%
\pgfusepath{stroke}%
\end{pgfscope}%
\begin{pgfscope}%
\pgfpathrectangle{\pgfqpoint{0.854460in}{0.571603in}}{\pgfqpoint{6.885100in}{5.225635in}}%
\pgfusepath{clip}%
\pgfsetbuttcap%
\pgfsetroundjoin%
\pgfsetlinewidth{1.505625pt}%
\definecolor{currentstroke}{rgb}{0.208030,0.718701,0.472873}%
\pgfsetstrokecolor{currentstroke}%
\pgfsetdash{}{0pt}%
\pgfpathmoveto{\pgfqpoint{1.756473in}{0.669531in}}%
\pgfpathlineto{\pgfqpoint{1.788620in}{0.654953in}}%
\pgfpathlineto{\pgfqpoint{1.798720in}{0.650382in}}%
\pgfpathlineto{\pgfqpoint{1.823218in}{0.639431in}}%
\pgfpathlineto{\pgfqpoint{1.857453in}{0.624122in}}%
\pgfpathlineto{\pgfqpoint{1.857817in}{0.623961in}}%
\pgfpathlineto{\pgfqpoint{1.892415in}{0.608734in}}%
\pgfpathlineto{\pgfqpoint{1.917102in}{0.597863in}}%
\pgfpathlineto{\pgfqpoint{1.927014in}{0.593552in}}%
\pgfpathlineto{\pgfqpoint{1.961612in}{0.578533in}}%
\pgfpathlineto{\pgfqpoint{1.977589in}{0.571603in}}%
\pgfusepath{stroke}%
\end{pgfscope}%
\begin{pgfscope}%
\pgfpathrectangle{\pgfqpoint{0.854460in}{0.571603in}}{\pgfqpoint{6.885100in}{5.225635in}}%
\pgfusepath{clip}%
\pgfsetbuttcap%
\pgfsetroundjoin%
\pgfsetlinewidth{1.505625pt}%
\definecolor{currentstroke}{rgb}{0.246070,0.738910,0.452024}%
\pgfsetstrokecolor{currentstroke}%
\pgfsetdash{}{0pt}%
\pgfpathmoveto{\pgfqpoint{0.854460in}{1.089822in}}%
\pgfpathlineto{\pgfqpoint{0.887379in}{1.070533in}}%
\pgfpathlineto{\pgfqpoint{0.889059in}{1.069561in}}%
\pgfpathlineto{\pgfqpoint{0.923657in}{1.049615in}}%
\pgfpathlineto{\pgfqpoint{0.932948in}{1.044274in}}%
\pgfpathlineto{\pgfqpoint{0.958256in}{1.029901in}}%
\pgfpathlineto{\pgfqpoint{0.979234in}{1.018014in}}%
\pgfpathlineto{\pgfqpoint{0.992854in}{1.010391in}}%
\pgfpathlineto{\pgfqpoint{1.026222in}{0.991755in}}%
\pgfpathlineto{\pgfqpoint{1.027453in}{0.991076in}}%
\pgfpathlineto{\pgfqpoint{1.062051in}{0.972063in}}%
\pgfpathlineto{\pgfqpoint{1.074029in}{0.965495in}}%
\pgfpathlineto{\pgfqpoint{1.096650in}{0.953242in}}%
\pgfpathlineto{\pgfqpoint{1.122556in}{0.939236in}}%
\pgfpathlineto{\pgfqpoint{1.131248in}{0.934593in}}%
\pgfpathlineto{\pgfqpoint{1.165847in}{0.916164in}}%
\pgfpathlineto{\pgfqpoint{1.171849in}{0.912976in}}%
\pgfpathlineto{\pgfqpoint{1.200445in}{0.897976in}}%
\pgfpathlineto{\pgfqpoint{1.221940in}{0.886717in}}%
\pgfpathlineto{\pgfqpoint{1.235044in}{0.879937in}}%
\pgfpathlineto{\pgfqpoint{1.269642in}{0.862069in}}%
\pgfpathlineto{\pgfqpoint{1.272774in}{0.860458in}}%
\pgfpathlineto{\pgfqpoint{1.304241in}{0.844460in}}%
\pgfpathlineto{\pgfqpoint{1.324449in}{0.834198in}}%
\pgfpathlineto{\pgfqpoint{1.338839in}{0.826979in}}%
\pgfpathlineto{\pgfqpoint{1.373438in}{0.809652in}}%
\pgfpathlineto{\pgfqpoint{1.376870in}{0.807939in}}%
\pgfpathlineto{\pgfqpoint{1.408036in}{0.792571in}}%
\pgfpathlineto{\pgfqpoint{1.420264in}{0.786547in}}%
\pgfusepath{stroke}%
\end{pgfscope}%
\begin{pgfscope}%
\pgfpathrectangle{\pgfqpoint{0.854460in}{0.571603in}}{\pgfqpoint{6.885100in}{5.225635in}}%
\pgfusepath{clip}%
\pgfsetbuttcap%
\pgfsetroundjoin%
\pgfsetlinewidth{1.505625pt}%
\definecolor{currentstroke}{rgb}{0.246070,0.738910,0.452024}%
\pgfsetstrokecolor{currentstroke}%
\pgfsetdash{}{0pt}%
\pgfpathmoveto{\pgfqpoint{1.770761in}{0.622423in}}%
\pgfpathlineto{\pgfqpoint{1.788620in}{0.614480in}}%
\pgfpathlineto{\pgfqpoint{1.823218in}{0.599093in}}%
\pgfpathlineto{\pgfqpoint{1.825992in}{0.597863in}}%
\pgfpathlineto{\pgfqpoint{1.857817in}{0.583926in}}%
\pgfpathlineto{\pgfqpoint{1.885938in}{0.571603in}}%
\pgfusepath{stroke}%
\end{pgfscope}%
\begin{pgfscope}%
\pgfpathrectangle{\pgfqpoint{0.854460in}{0.571603in}}{\pgfqpoint{6.885100in}{5.225635in}}%
\pgfusepath{clip}%
\pgfsetbuttcap%
\pgfsetroundjoin%
\pgfsetlinewidth{1.505625pt}%
\definecolor{currentstroke}{rgb}{0.288921,0.758394,0.428426}%
\pgfsetstrokecolor{currentstroke}%
\pgfsetdash{}{0pt}%
\pgfpathmoveto{\pgfqpoint{0.854460in}{1.043056in}}%
\pgfpathlineto{\pgfqpoint{0.889059in}{1.023232in}}%
\pgfpathlineto{\pgfqpoint{0.898191in}{1.018014in}}%
\pgfpathlineto{\pgfqpoint{0.923657in}{1.003640in}}%
\pgfpathlineto{\pgfqpoint{0.926750in}{1.001898in}}%
\pgfusepath{stroke}%
\end{pgfscope}%
\begin{pgfscope}%
\pgfpathrectangle{\pgfqpoint{0.854460in}{0.571603in}}{\pgfqpoint{6.885100in}{5.225635in}}%
\pgfusepath{clip}%
\pgfsetbuttcap%
\pgfsetroundjoin%
\pgfsetlinewidth{1.505625pt}%
\definecolor{currentstroke}{rgb}{0.288921,0.758394,0.428426}%
\pgfsetstrokecolor{currentstroke}%
\pgfsetdash{}{0pt}%
\pgfpathmoveto{\pgfqpoint{1.268562in}{0.819788in}}%
\pgfpathlineto{\pgfqpoint{1.269642in}{0.819242in}}%
\pgfpathlineto{\pgfqpoint{1.292027in}{0.807939in}}%
\pgfpathlineto{\pgfqpoint{1.304241in}{0.801846in}}%
\pgfpathlineto{\pgfqpoint{1.338839in}{0.784620in}}%
\pgfpathlineto{\pgfqpoint{1.344763in}{0.781679in}}%
\pgfpathlineto{\pgfqpoint{1.373438in}{0.767617in}}%
\pgfpathlineto{\pgfqpoint{1.398330in}{0.755420in}}%
\pgfpathlineto{\pgfqpoint{1.408036in}{0.750721in}}%
\pgfpathlineto{\pgfqpoint{1.442635in}{0.734012in}}%
\pgfpathlineto{\pgfqpoint{1.452703in}{0.729160in}}%
\pgfpathlineto{\pgfqpoint{1.477233in}{0.717482in}}%
\pgfpathlineto{\pgfqpoint{1.507880in}{0.702901in}}%
\pgfpathlineto{\pgfqpoint{1.511832in}{0.701043in}}%
\pgfpathlineto{\pgfqpoint{1.546430in}{0.684831in}}%
\pgfpathlineto{\pgfqpoint{1.563925in}{0.676641in}}%
\pgfpathlineto{\pgfqpoint{1.581029in}{0.668732in}}%
\pgfpathlineto{\pgfqpoint{1.615627in}{0.652747in}}%
\pgfpathlineto{\pgfqpoint{1.620762in}{0.650382in}}%
\pgfpathlineto{\pgfqpoint{1.650226in}{0.636973in}}%
\pgfpathlineto{\pgfqpoint{1.678464in}{0.624122in}}%
\pgfpathlineto{\pgfqpoint{1.684824in}{0.621263in}}%
\pgfpathlineto{\pgfqpoint{1.719423in}{0.605749in}}%
\pgfpathlineto{\pgfqpoint{1.737025in}{0.597863in}}%
\pgfpathlineto{\pgfqpoint{1.754021in}{0.590341in}}%
\pgfpathlineto{\pgfqpoint{1.788620in}{0.575042in}}%
\pgfpathlineto{\pgfqpoint{1.796413in}{0.571603in}}%
\pgfusepath{stroke}%
\end{pgfscope}%
\begin{pgfscope}%
\pgfpathrectangle{\pgfqpoint{0.854460in}{0.571603in}}{\pgfqpoint{6.885100in}{5.225635in}}%
\pgfusepath{clip}%
\pgfsetbuttcap%
\pgfsetroundjoin%
\pgfsetlinewidth{1.505625pt}%
\definecolor{currentstroke}{rgb}{0.327796,0.773980,0.406640}%
\pgfsetstrokecolor{currentstroke}%
\pgfsetdash{}{0pt}%
\pgfpathmoveto{\pgfqpoint{0.854460in}{0.997793in}}%
\pgfpathlineto{\pgfqpoint{0.865092in}{0.991755in}}%
\pgfpathlineto{\pgfqpoint{0.889059in}{0.978308in}}%
\pgfpathlineto{\pgfqpoint{0.911947in}{0.965495in}}%
\pgfpathlineto{\pgfqpoint{0.923657in}{0.959019in}}%
\pgfpathlineto{\pgfqpoint{0.926120in}{0.957660in}}%
\pgfusepath{stroke}%
\end{pgfscope}%
\begin{pgfscope}%
\pgfpathrectangle{\pgfqpoint{0.854460in}{0.571603in}}{\pgfqpoint{6.885100in}{5.225635in}}%
\pgfusepath{clip}%
\pgfsetbuttcap%
\pgfsetroundjoin%
\pgfsetlinewidth{1.505625pt}%
\definecolor{currentstroke}{rgb}{0.327796,0.773980,0.406640}%
\pgfsetstrokecolor{currentstroke}%
\pgfsetdash{}{0pt}%
\pgfpathmoveto{\pgfqpoint{1.269092in}{0.777823in}}%
\pgfpathlineto{\pgfqpoint{1.269642in}{0.777550in}}%
\pgfpathlineto{\pgfqpoint{1.304241in}{0.760429in}}%
\pgfpathlineto{\pgfqpoint{1.314386in}{0.755420in}}%
\pgfpathlineto{\pgfqpoint{1.338839in}{0.743492in}}%
\pgfpathlineto{\pgfqpoint{1.368246in}{0.729160in}}%
\pgfpathlineto{\pgfqpoint{1.373438in}{0.726660in}}%
\pgfpathlineto{\pgfqpoint{1.408036in}{0.710052in}}%
\pgfpathlineto{\pgfqpoint{1.422954in}{0.702901in}}%
\pgfpathlineto{\pgfqpoint{1.442635in}{0.693581in}}%
\pgfpathlineto{\pgfqpoint{1.477233in}{0.677208in}}%
\pgfpathlineto{\pgfqpoint{1.478435in}{0.676641in}}%
\pgfpathlineto{\pgfqpoint{1.511832in}{0.661083in}}%
\pgfpathlineto{\pgfqpoint{1.534810in}{0.650382in}}%
\pgfpathlineto{\pgfqpoint{1.546430in}{0.645035in}}%
\pgfpathlineto{\pgfqpoint{1.581029in}{0.629147in}}%
\pgfpathlineto{\pgfqpoint{1.591991in}{0.624122in}}%
\pgfpathlineto{\pgfqpoint{1.615627in}{0.613420in}}%
\pgfpathlineto{\pgfqpoint{1.649986in}{0.597863in}}%
\pgfpathlineto{\pgfqpoint{1.650226in}{0.597755in}}%
\pgfpathlineto{\pgfqpoint{1.684824in}{0.582334in}}%
\pgfpathlineto{\pgfqpoint{1.708894in}{0.571603in}}%
\pgfusepath{stroke}%
\end{pgfscope}%
\begin{pgfscope}%
\pgfpathrectangle{\pgfqpoint{0.854460in}{0.571603in}}{\pgfqpoint{6.885100in}{5.225635in}}%
\pgfusepath{clip}%
\pgfsetbuttcap%
\pgfsetroundjoin%
\pgfsetlinewidth{1.505625pt}%
\definecolor{currentstroke}{rgb}{0.377779,0.791781,0.377939}%
\pgfsetstrokecolor{currentstroke}%
\pgfsetdash{}{0pt}%
\pgfpathmoveto{\pgfqpoint{0.854460in}{0.953853in}}%
\pgfpathlineto{\pgfqpoint{0.880727in}{0.939236in}}%
\pgfpathlineto{\pgfqpoint{0.889059in}{0.934655in}}%
\pgfpathlineto{\pgfqpoint{0.923657in}{0.915687in}}%
\pgfpathlineto{\pgfqpoint{0.928619in}{0.912976in}}%
\pgfpathlineto{\pgfqpoint{0.958256in}{0.896978in}}%
\pgfpathlineto{\pgfqpoint{0.959224in}{0.896457in}}%
\pgfusepath{stroke}%
\end{pgfscope}%
\begin{pgfscope}%
\pgfpathrectangle{\pgfqpoint{0.854460in}{0.571603in}}{\pgfqpoint{6.885100in}{5.225635in}}%
\pgfusepath{clip}%
\pgfsetbuttcap%
\pgfsetroundjoin%
\pgfsetlinewidth{1.505625pt}%
\definecolor{currentstroke}{rgb}{0.377779,0.791781,0.377939}%
\pgfsetstrokecolor{currentstroke}%
\pgfsetdash{}{0pt}%
\pgfpathmoveto{\pgfqpoint{1.375003in}{0.686119in}}%
\pgfpathlineto{\pgfqpoint{1.394878in}{0.676641in}}%
\pgfpathlineto{\pgfqpoint{1.408036in}{0.670442in}}%
\pgfpathlineto{\pgfqpoint{1.442635in}{0.654171in}}%
\pgfpathlineto{\pgfqpoint{1.450713in}{0.650382in}}%
\pgfpathlineto{\pgfqpoint{1.477233in}{0.638090in}}%
\pgfpathlineto{\pgfqpoint{1.507380in}{0.624122in}}%
\pgfpathlineto{\pgfqpoint{1.511832in}{0.622084in}}%
\pgfpathlineto{\pgfqpoint{1.546430in}{0.606294in}}%
\pgfpathlineto{\pgfqpoint{1.564918in}{0.597863in}}%
\pgfpathlineto{\pgfqpoint{1.581029in}{0.590605in}}%
\pgfpathlineto{\pgfqpoint{1.615627in}{0.575035in}}%
\pgfpathlineto{\pgfqpoint{1.623270in}{0.571603in}}%
\pgfusepath{stroke}%
\end{pgfscope}%
\begin{pgfscope}%
\pgfpathrectangle{\pgfqpoint{0.854460in}{0.571603in}}{\pgfqpoint{6.885100in}{5.225635in}}%
\pgfusepath{clip}%
\pgfsetbuttcap%
\pgfsetroundjoin%
\pgfsetlinewidth{1.505625pt}%
\definecolor{currentstroke}{rgb}{0.430983,0.808473,0.346476}%
\pgfsetstrokecolor{currentstroke}%
\pgfsetdash{}{0pt}%
\pgfpathmoveto{\pgfqpoint{0.854460in}{0.911107in}}%
\pgfpathlineto{\pgfqpoint{0.889059in}{0.892263in}}%
\pgfpathlineto{\pgfqpoint{0.899269in}{0.886717in}}%
\pgfpathlineto{\pgfqpoint{0.923657in}{0.873627in}}%
\pgfpathlineto{\pgfqpoint{0.948241in}{0.860458in}}%
\pgfpathlineto{\pgfqpoint{0.958256in}{0.855157in}}%
\pgfpathlineto{\pgfqpoint{0.958586in}{0.854983in}}%
\pgfusepath{stroke}%
\end{pgfscope}%
\begin{pgfscope}%
\pgfpathrectangle{\pgfqpoint{0.854460in}{0.571603in}}{\pgfqpoint{6.885100in}{5.225635in}}%
\pgfusepath{clip}%
\pgfsetbuttcap%
\pgfsetroundjoin%
\pgfsetlinewidth{1.505625pt}%
\definecolor{currentstroke}{rgb}{0.430983,0.808473,0.346476}%
\pgfsetstrokecolor{currentstroke}%
\pgfsetdash{}{0pt}%
\pgfpathmoveto{\pgfqpoint{1.375531in}{0.647045in}}%
\pgfpathlineto{\pgfqpoint{1.408036in}{0.631856in}}%
\pgfpathlineto{\pgfqpoint{1.424606in}{0.624122in}}%
\pgfpathlineto{\pgfqpoint{1.442635in}{0.615809in}}%
\pgfpathlineto{\pgfqpoint{1.477233in}{0.599870in}}%
\pgfpathlineto{\pgfqpoint{1.481603in}{0.597863in}}%
\pgfpathlineto{\pgfqpoint{1.511832in}{0.584145in}}%
\pgfpathlineto{\pgfqpoint{1.539472in}{0.571603in}}%
\pgfusepath{stroke}%
\end{pgfscope}%
\begin{pgfscope}%
\pgfpathrectangle{\pgfqpoint{0.854460in}{0.571603in}}{\pgfqpoint{6.885100in}{5.225635in}}%
\pgfusepath{clip}%
\pgfsetbuttcap%
\pgfsetroundjoin%
\pgfsetlinewidth{1.505625pt}%
\definecolor{currentstroke}{rgb}{0.477504,0.821444,0.318195}%
\pgfsetstrokecolor{currentstroke}%
\pgfsetdash{}{0pt}%
\pgfpathmoveto{\pgfqpoint{0.854460in}{0.869610in}}%
\pgfpathlineto{\pgfqpoint{0.871405in}{0.860458in}}%
\pgfpathlineto{\pgfqpoint{0.889059in}{0.851036in}}%
\pgfpathlineto{\pgfqpoint{0.920668in}{0.834198in}}%
\pgfpathlineto{\pgfqpoint{0.923657in}{0.832625in}}%
\pgfpathlineto{\pgfqpoint{0.957994in}{0.814618in}}%
\pgfusepath{stroke}%
\end{pgfscope}%
\begin{pgfscope}%
\pgfpathrectangle{\pgfqpoint{0.854460in}{0.571603in}}{\pgfqpoint{6.885100in}{5.225635in}}%
\pgfusepath{clip}%
\pgfsetbuttcap%
\pgfsetroundjoin%
\pgfsetlinewidth{1.505625pt}%
\definecolor{currentstroke}{rgb}{0.477504,0.821444,0.318195}%
\pgfsetstrokecolor{currentstroke}%
\pgfsetdash{}{0pt}%
\pgfpathmoveto{\pgfqpoint{1.376051in}{0.609006in}}%
\pgfpathlineto{\pgfqpoint{1.400049in}{0.597863in}}%
\pgfpathlineto{\pgfqpoint{1.408036in}{0.594198in}}%
\pgfpathlineto{\pgfqpoint{1.442635in}{0.578364in}}%
\pgfpathlineto{\pgfqpoint{1.457427in}{0.571603in}}%
\pgfusepath{stroke}%
\end{pgfscope}%
\begin{pgfscope}%
\pgfpathrectangle{\pgfqpoint{0.854460in}{0.571603in}}{\pgfqpoint{6.885100in}{5.225635in}}%
\pgfusepath{clip}%
\pgfsetbuttcap%
\pgfsetroundjoin%
\pgfsetlinewidth{1.505625pt}%
\definecolor{currentstroke}{rgb}{0.535621,0.835785,0.281908}%
\pgfsetstrokecolor{currentstroke}%
\pgfsetdash{}{0pt}%
\pgfpathmoveto{\pgfqpoint{0.854460in}{0.829161in}}%
\pgfpathlineto{\pgfqpoint{0.889059in}{0.810849in}}%
\pgfpathlineto{\pgfqpoint{0.894575in}{0.807939in}}%
\pgfpathlineto{\pgfqpoint{0.923244in}{0.792994in}}%
\pgfusepath{stroke}%
\end{pgfscope}%
\begin{pgfscope}%
\pgfpathrectangle{\pgfqpoint{0.854460in}{0.571603in}}{\pgfqpoint{6.885100in}{5.225635in}}%
\pgfusepath{clip}%
\pgfsetbuttcap%
\pgfsetroundjoin%
\pgfsetlinewidth{1.505625pt}%
\definecolor{currentstroke}{rgb}{0.535621,0.835785,0.281908}%
\pgfsetstrokecolor{currentstroke}%
\pgfsetdash{}{0pt}%
\pgfpathmoveto{\pgfqpoint{1.341576in}{0.587961in}}%
\pgfpathlineto{\pgfqpoint{1.373438in}{0.573248in}}%
\pgfpathlineto{\pgfqpoint{1.377011in}{0.571603in}}%
\pgfusepath{stroke}%
\end{pgfscope}%
\begin{pgfscope}%
\pgfpathrectangle{\pgfqpoint{0.854460in}{0.571603in}}{\pgfqpoint{6.885100in}{5.225635in}}%
\pgfusepath{clip}%
\pgfsetbuttcap%
\pgfsetroundjoin%
\pgfsetlinewidth{1.505625pt}%
\definecolor{currentstroke}{rgb}{0.595839,0.848717,0.243329}%
\pgfsetstrokecolor{currentstroke}%
\pgfsetdash{}{0pt}%
\pgfpathmoveto{\pgfqpoint{0.854460in}{0.789777in}}%
\pgfpathlineto{\pgfqpoint{0.869663in}{0.781801in}}%
\pgfusepath{stroke}%
\end{pgfscope}%
\begin{pgfscope}%
\pgfpathrectangle{\pgfqpoint{0.854460in}{0.571603in}}{\pgfqpoint{6.885100in}{5.225635in}}%
\pgfusepath{clip}%
\pgfsetbuttcap%
\pgfsetroundjoin%
\pgfsetlinewidth{1.505625pt}%
\definecolor{currentstroke}{rgb}{0.595839,0.848717,0.243329}%
\pgfsetstrokecolor{currentstroke}%
\pgfsetdash{}{0pt}%
\pgfpathmoveto{\pgfqpoint{1.287838in}{0.576436in}}%
\pgfpathlineto{\pgfqpoint{1.298220in}{0.571603in}}%
\pgfusepath{stroke}%
\end{pgfscope}%
\begin{pgfscope}%
\pgfpathrectangle{\pgfqpoint{0.854460in}{0.571603in}}{\pgfqpoint{6.885100in}{5.225635in}}%
\pgfusepath{clip}%
\pgfsetbuttcap%
\pgfsetroundjoin%
\pgfsetlinewidth{1.505625pt}%
\definecolor{currentstroke}{rgb}{0.647257,0.858400,0.209861}%
\pgfsetstrokecolor{currentstroke}%
\pgfsetdash{}{0pt}%
\pgfpathmoveto{\pgfqpoint{0.854460in}{0.751340in}}%
\pgfpathlineto{\pgfqpoint{0.889059in}{0.733551in}}%
\pgfpathlineto{\pgfqpoint{0.897623in}{0.729160in}}%
\pgfpathlineto{\pgfqpoint{0.923657in}{0.715969in}}%
\pgfpathlineto{\pgfqpoint{0.949488in}{0.702901in}}%
\pgfpathlineto{\pgfqpoint{0.958256in}{0.698517in}}%
\pgfpathlineto{\pgfqpoint{0.992854in}{0.681266in}}%
\pgfpathlineto{\pgfqpoint{1.002154in}{0.676641in}}%
\pgfpathlineto{\pgfqpoint{1.027453in}{0.664208in}}%
\pgfpathlineto{\pgfqpoint{1.055619in}{0.650382in}}%
\pgfpathlineto{\pgfqpoint{1.062051in}{0.647261in}}%
\pgfpathlineto{\pgfqpoint{1.096650in}{0.630528in}}%
\pgfpathlineto{\pgfqpoint{1.109919in}{0.624122in}}%
\pgfpathlineto{\pgfqpoint{1.131248in}{0.613947in}}%
\pgfpathlineto{\pgfqpoint{1.164996in}{0.597863in}}%
\pgfpathlineto{\pgfqpoint{1.165847in}{0.597462in}}%
\pgfpathlineto{\pgfqpoint{1.200445in}{0.581226in}}%
\pgfpathlineto{\pgfqpoint{1.220966in}{0.571603in}}%
\pgfusepath{stroke}%
\end{pgfscope}%
\begin{pgfscope}%
\pgfpathrectangle{\pgfqpoint{0.854460in}{0.571603in}}{\pgfqpoint{6.885100in}{5.225635in}}%
\pgfusepath{clip}%
\pgfsetbuttcap%
\pgfsetroundjoin%
\pgfsetlinewidth{1.505625pt}%
\definecolor{currentstroke}{rgb}{0.709898,0.868751,0.169257}%
\pgfsetstrokecolor{currentstroke}%
\pgfsetdash{}{0pt}%
\pgfpathmoveto{\pgfqpoint{0.854460in}{0.713868in}}%
\pgfpathlineto{\pgfqpoint{0.875965in}{0.702901in}}%
\pgfpathlineto{\pgfqpoint{0.889059in}{0.696301in}}%
\pgfpathlineto{\pgfqpoint{0.923657in}{0.678901in}}%
\pgfpathlineto{\pgfqpoint{0.928165in}{0.676641in}}%
\pgfpathlineto{\pgfqpoint{0.958256in}{0.661735in}}%
\pgfpathlineto{\pgfqpoint{0.981204in}{0.650382in}}%
\pgfpathlineto{\pgfqpoint{0.992854in}{0.644685in}}%
\pgfpathlineto{\pgfqpoint{1.027453in}{0.627808in}}%
\pgfpathlineto{\pgfqpoint{1.035029in}{0.624122in}}%
\pgfpathlineto{\pgfqpoint{1.062051in}{0.611131in}}%
\pgfpathlineto{\pgfqpoint{1.089678in}{0.597863in}}%
\pgfpathlineto{\pgfqpoint{1.096650in}{0.594554in}}%
\pgfpathlineto{\pgfqpoint{1.131248in}{0.578179in}}%
\pgfpathlineto{\pgfqpoint{1.145167in}{0.571603in}}%
\pgfusepath{stroke}%
\end{pgfscope}%
\begin{pgfscope}%
\pgfpathrectangle{\pgfqpoint{0.854460in}{0.571603in}}{\pgfqpoint{6.885100in}{5.225635in}}%
\pgfusepath{clip}%
\pgfsetbuttcap%
\pgfsetroundjoin%
\pgfsetlinewidth{1.505625pt}%
\definecolor{currentstroke}{rgb}{0.772852,0.877868,0.131109}%
\pgfsetstrokecolor{currentstroke}%
\pgfsetdash{}{0pt}%
\pgfpathmoveto{\pgfqpoint{0.854460in}{0.677197in}}%
\pgfpathlineto{\pgfqpoint{0.855560in}{0.676641in}}%
\pgfpathlineto{\pgfqpoint{0.889059in}{0.659913in}}%
\pgfpathlineto{\pgfqpoint{0.908173in}{0.650382in}}%
\pgfpathlineto{\pgfqpoint{0.923657in}{0.642750in}}%
\pgfpathlineto{\pgfqpoint{0.958256in}{0.625729in}}%
\pgfpathlineto{\pgfqpoint{0.961534in}{0.624122in}}%
\pgfpathlineto{\pgfqpoint{0.992854in}{0.608947in}}%
\pgfpathlineto{\pgfqpoint{1.015756in}{0.597863in}}%
\pgfpathlineto{\pgfqpoint{1.027453in}{0.592268in}}%
\pgfpathlineto{\pgfqpoint{1.062051in}{0.575755in}}%
\pgfpathlineto{\pgfqpoint{1.070773in}{0.571603in}}%
\pgfusepath{stroke}%
\end{pgfscope}%
\begin{pgfscope}%
\pgfpathrectangle{\pgfqpoint{0.854460in}{0.571603in}}{\pgfqpoint{6.885100in}{5.225635in}}%
\pgfusepath{clip}%
\pgfsetbuttcap%
\pgfsetroundjoin%
\pgfsetlinewidth{1.505625pt}%
\definecolor{currentstroke}{rgb}{0.824940,0.884720,0.106217}%
\pgfsetstrokecolor{currentstroke}%
\pgfsetdash{}{0pt}%
\pgfpathmoveto{\pgfqpoint{0.854460in}{0.641448in}}%
\pgfpathlineto{\pgfqpoint{0.889059in}{0.624285in}}%
\pgfpathlineto{\pgfqpoint{0.889387in}{0.624122in}}%
\pgfpathlineto{\pgfqpoint{0.923657in}{0.607385in}}%
\pgfpathlineto{\pgfqpoint{0.943180in}{0.597863in}}%
\pgfpathlineto{\pgfqpoint{0.958256in}{0.590595in}}%
\pgfpathlineto{\pgfqpoint{0.992854in}{0.573946in}}%
\pgfpathlineto{\pgfqpoint{0.997737in}{0.571603in}}%
\pgfusepath{stroke}%
\end{pgfscope}%
\begin{pgfscope}%
\pgfpathrectangle{\pgfqpoint{0.854460in}{0.571603in}}{\pgfqpoint{6.885100in}{5.225635in}}%
\pgfusepath{clip}%
\pgfsetbuttcap%
\pgfsetroundjoin%
\pgfsetlinewidth{1.505625pt}%
\definecolor{currentstroke}{rgb}{0.886271,0.892374,0.095374}%
\pgfsetstrokecolor{currentstroke}%
\pgfsetdash{}{0pt}%
\pgfpathmoveto{\pgfqpoint{0.854460in}{0.606439in}}%
\pgfpathlineto{\pgfqpoint{0.871906in}{0.597863in}}%
\pgfpathlineto{\pgfqpoint{0.889059in}{0.589528in}}%
\pgfpathlineto{\pgfqpoint{0.923657in}{0.572742in}}%
\pgfpathlineto{\pgfqpoint{0.926014in}{0.571603in}}%
\pgfusepath{stroke}%
\end{pgfscope}%
\begin{pgfscope}%
\pgfpathrectangle{\pgfqpoint{0.854460in}{0.571603in}}{\pgfqpoint{6.885100in}{5.225635in}}%
\pgfusepath{clip}%
\pgfsetbuttcap%
\pgfsetroundjoin%
\pgfsetlinewidth{1.505625pt}%
\definecolor{currentstroke}{rgb}{0.945636,0.899815,0.112838}%
\pgfsetstrokecolor{currentstroke}%
\pgfsetdash{}{0pt}%
\pgfpathmoveto{\pgfqpoint{0.854460in}{0.572138in}}%
\pgfpathlineto{\pgfqpoint{0.855558in}{0.571603in}}%
\pgfusepath{stroke}%
\end{pgfscope}%
\begin{pgfscope}%
\pgfpathrectangle{\pgfqpoint{0.854460in}{0.571603in}}{\pgfqpoint{6.885100in}{5.225635in}}%
\pgfusepath{clip}%
\pgfsetrectcap%
\pgfsetroundjoin%
\pgfsetlinewidth{1.505625pt}%
\definecolor{currentstroke}{rgb}{0.000000,0.000000,0.000000}%
\pgfsetstrokecolor{currentstroke}%
\pgfsetdash{}{0pt}%
\pgfpathmoveto{\pgfqpoint{3.149494in}{4.490830in}}%
\pgfpathlineto{\pgfqpoint{3.149494in}{1.878012in}}%
\pgfpathlineto{\pgfqpoint{4.259541in}{2.236607in}}%
\pgfpathlineto{\pgfqpoint{4.068280in}{2.506074in}}%
\pgfusepath{stroke}%
\end{pgfscope}%
\begin{pgfscope}%
\pgfsetrectcap%
\pgfsetmiterjoin%
\pgfsetlinewidth{0.803000pt}%
\definecolor{currentstroke}{rgb}{0.000000,0.000000,0.000000}%
\pgfsetstrokecolor{currentstroke}%
\pgfsetdash{}{0pt}%
\pgfpathmoveto{\pgfqpoint{0.854460in}{0.571603in}}%
\pgfpathlineto{\pgfqpoint{0.854460in}{5.797238in}}%
\pgfusepath{stroke}%
\end{pgfscope}%
\begin{pgfscope}%
\pgfsetrectcap%
\pgfsetmiterjoin%
\pgfsetlinewidth{0.803000pt}%
\definecolor{currentstroke}{rgb}{0.000000,0.000000,0.000000}%
\pgfsetstrokecolor{currentstroke}%
\pgfsetdash{}{0pt}%
\pgfpathmoveto{\pgfqpoint{7.739560in}{0.571603in}}%
\pgfpathlineto{\pgfqpoint{7.739560in}{5.797238in}}%
\pgfusepath{stroke}%
\end{pgfscope}%
\begin{pgfscope}%
\pgfsetrectcap%
\pgfsetmiterjoin%
\pgfsetlinewidth{0.803000pt}%
\definecolor{currentstroke}{rgb}{0.000000,0.000000,0.000000}%
\pgfsetstrokecolor{currentstroke}%
\pgfsetdash{}{0pt}%
\pgfpathmoveto{\pgfqpoint{0.854460in}{0.571603in}}%
\pgfpathlineto{\pgfqpoint{7.739560in}{0.571603in}}%
\pgfusepath{stroke}%
\end{pgfscope}%
\begin{pgfscope}%
\pgfsetrectcap%
\pgfsetmiterjoin%
\pgfsetlinewidth{0.803000pt}%
\definecolor{currentstroke}{rgb}{0.000000,0.000000,0.000000}%
\pgfsetstrokecolor{currentstroke}%
\pgfsetdash{}{0pt}%
\pgfpathmoveto{\pgfqpoint{0.854460in}{5.797238in}}%
\pgfpathlineto{\pgfqpoint{7.739560in}{5.797238in}}%
\pgfusepath{stroke}%
\end{pgfscope}%
\begin{pgfscope}%
\definecolor{textcolor}{rgb}{0.273809,0.031497,0.358853}%
\pgfsetstrokecolor{textcolor}%
\pgfsetfillcolor{textcolor}%
\pgftext[x=4.722791in, y=2.680320in, left, base,rotate=316.199517]{\color{textcolor}\sffamily\fontsize{8.000000}{9.600000}\selectfont 1.75}%
\end{pgfscope}%
\begin{pgfscope}%
\definecolor{textcolor}{rgb}{0.279566,0.067836,0.391917}%
\pgfsetstrokecolor{textcolor}%
\pgfsetfillcolor{textcolor}%
\pgftext[x=5.493637in, y=2.392928in, left, base,rotate=317.001797]{\color{textcolor}\sffamily\fontsize{8.000000}{9.600000}\selectfont 2.00}%
\end{pgfscope}%
\begin{pgfscope}%
\definecolor{textcolor}{rgb}{0.282656,0.100196,0.422160}%
\pgfsetstrokecolor{textcolor}%
\pgfsetfillcolor{textcolor}%
\pgftext[x=6.409130in, y=1.889536in, left, base,rotate=319.272549]{\color{textcolor}\sffamily\fontsize{8.000000}{9.600000}\selectfont 2.25}%
\end{pgfscope}%
\begin{pgfscope}%
\definecolor{textcolor}{rgb}{0.283187,0.125848,0.444960}%
\pgfsetstrokecolor{textcolor}%
\pgfsetfillcolor{textcolor}%
\pgftext[x=7.035547in, y=1.633619in, left, base,rotate=321.445610]{\color{textcolor}\sffamily\fontsize{8.000000}{9.600000}\selectfont 2.50}%
\end{pgfscope}%
\begin{pgfscope}%
\definecolor{textcolor}{rgb}{0.281412,0.155834,0.469201}%
\pgfsetstrokecolor{textcolor}%
\pgfsetfillcolor{textcolor}%
\pgftext[x=5.190341in, y=0.774483in, left, base,rotate=343.655611]{\color{textcolor}\sffamily\fontsize{8.000000}{9.600000}\selectfont 2.75}%
\end{pgfscope}%
\begin{pgfscope}%
\definecolor{textcolor}{rgb}{0.281412,0.155834,0.469201}%
\pgfsetstrokecolor{textcolor}%
\pgfsetfillcolor{textcolor}%
\pgftext[x=7.414526in, y=1.562023in, left, base,rotate=323.237694]{\color{textcolor}\sffamily\fontsize{8.000000}{9.600000}\selectfont 2.75}%
\end{pgfscope}%
\begin{pgfscope}%
\definecolor{textcolor}{rgb}{0.277134,0.185228,0.489898}%
\pgfsetstrokecolor{textcolor}%
\pgfsetfillcolor{textcolor}%
\pgftext[x=4.660461in, y=0.843764in, left, base,rotate=341.776993]{\color{textcolor}\sffamily\fontsize{8.000000}{9.600000}\selectfont 3.00}%
\end{pgfscope}%
\begin{pgfscope}%
\definecolor{textcolor}{rgb}{0.277134,0.185228,0.489898}%
\pgfsetstrokecolor{textcolor}%
\pgfsetfillcolor{textcolor}%
\pgftext[x=7.000608in, y=2.079128in, left, base,rotate=321.815252]{\color{textcolor}\sffamily\fontsize{8.000000}{9.600000}\selectfont 3.00}%
\end{pgfscope}%
\begin{pgfscope}%
\definecolor{textcolor}{rgb}{0.271828,0.209303,0.504434}%
\pgfsetstrokecolor{textcolor}%
\pgfsetfillcolor{textcolor}%
\pgftext[x=3.963660in, y=1.007051in, left, base,rotate=339.188224]{\color{textcolor}\sffamily\fontsize{8.000000}{9.600000}\selectfont 3.25}%
\end{pgfscope}%
\begin{pgfscope}%
\definecolor{textcolor}{rgb}{0.271828,0.209303,0.504434}%
\pgfsetstrokecolor{textcolor}%
\pgfsetfillcolor{textcolor}%
\pgftext[x=7.414468in, y=1.925454in, left, base,rotate=323.305243]{\color{textcolor}\sffamily\fontsize{8.000000}{9.600000}\selectfont 3.25}%
\end{pgfscope}%
\begin{pgfscope}%
\definecolor{textcolor}{rgb}{0.263663,0.237631,0.518762}%
\pgfsetstrokecolor{textcolor}%
\pgfsetfillcolor{textcolor}%
\pgftext[x=3.323735in, y=1.189117in, left, base,rotate=336.367739]{\color{textcolor}\sffamily\fontsize{8.000000}{9.600000}\selectfont 3.50}%
\end{pgfscope}%
\begin{pgfscope}%
\definecolor{textcolor}{rgb}{0.263663,0.237631,0.518762}%
\pgfsetstrokecolor{textcolor}%
\pgfsetfillcolor{textcolor}%
\pgftext[x=5.345802in, y=5.552956in, left, base,rotate=274.877744]{\color{textcolor}\sffamily\fontsize{8.000000}{9.600000}\selectfont 3.50}%
\end{pgfscope}%
\begin{pgfscope}%
\definecolor{textcolor}{rgb}{0.253935,0.265254,0.529983}%
\pgfsetstrokecolor{textcolor}%
\pgfsetfillcolor{textcolor}%
\pgftext[x=1.386454in, y=5.502559in, left, base,rotate=46.532648]{\color{textcolor}\sffamily\fontsize{8.000000}{9.600000}\selectfont 3.75}%
\end{pgfscope}%
\begin{pgfscope}%
\definecolor{textcolor}{rgb}{0.253935,0.265254,0.529983}%
\pgfsetstrokecolor{textcolor}%
\pgfsetfillcolor{textcolor}%
\pgftext[x=4.226667in, y=0.740665in, left, base,rotate=340.902480]{\color{textcolor}\sffamily\fontsize{8.000000}{9.600000}\selectfont 3.75}%
\end{pgfscope}%
\begin{pgfscope}%
\definecolor{textcolor}{rgb}{0.253935,0.265254,0.529983}%
\pgfsetstrokecolor{textcolor}%
\pgfsetfillcolor{textcolor}%
\pgftext[x=7.030352in, y=2.573693in, left, base,rotate=318.560069]{\color{textcolor}\sffamily\fontsize{8.000000}{9.600000}\selectfont 3.75}%
\end{pgfscope}%
\begin{pgfscope}%
\definecolor{textcolor}{rgb}{0.244972,0.287675,0.537260}%
\pgfsetstrokecolor{textcolor}%
\pgfsetfillcolor{textcolor}%
\pgftext[x=0.925837in, y=5.071201in, left, base,rotate=59.390645]{\color{textcolor}\sffamily\fontsize{8.000000}{9.600000}\selectfont 4.00}%
\end{pgfscope}%
\begin{pgfscope}%
\definecolor{textcolor}{rgb}{0.244972,0.287675,0.537260}%
\pgfsetstrokecolor{textcolor}%
\pgfsetfillcolor{textcolor}%
\pgftext[x=2.874575in, y=1.229425in, left, base,rotate=334.835338]{\color{textcolor}\sffamily\fontsize{8.000000}{9.600000}\selectfont 4.00}%
\end{pgfscope}%
\begin{pgfscope}%
\definecolor{textcolor}{rgb}{0.244972,0.287675,0.537260}%
\pgfsetstrokecolor{textcolor}%
\pgfsetfillcolor{textcolor}%
\pgftext[x=7.417760in, y=2.385887in, left, base,rotate=320.564740]{\color{textcolor}\sffamily\fontsize{8.000000}{9.600000}\selectfont 4.00}%
\end{pgfscope}%
\begin{pgfscope}%
\definecolor{textcolor}{rgb}{0.233603,0.313828,0.543914}%
\pgfsetstrokecolor{textcolor}%
\pgfsetfillcolor{textcolor}%
\pgftext[x=1.040564in, y=5.496899in, left, base,rotate=49.971998]{\color{textcolor}\sffamily\fontsize{8.000000}{9.600000}\selectfont 4.25}%
\end{pgfscope}%
\begin{pgfscope}%
\definecolor{textcolor}{rgb}{0.233603,0.313828,0.543914}%
\pgfsetstrokecolor{textcolor}%
\pgfsetfillcolor{textcolor}%
\pgftext[x=3.530522in, y=0.861005in, left, base,rotate=338.817134]{\color{textcolor}\sffamily\fontsize{8.000000}{9.600000}\selectfont 4.25}%
\end{pgfscope}%
\begin{pgfscope}%
\definecolor{textcolor}{rgb}{0.233603,0.313828,0.543914}%
\pgfsetstrokecolor{textcolor}%
\pgfsetfillcolor{textcolor}%
\pgftext[x=6.736610in, y=3.251782in, left, base,rotate=310.303324]{\color{textcolor}\sffamily\fontsize{8.000000}{9.600000}\selectfont 4.25}%
\end{pgfscope}%
\begin{pgfscope}%
\definecolor{textcolor}{rgb}{0.221989,0.339161,0.548752}%
\pgfsetstrokecolor{textcolor}%
\pgfsetfillcolor{textcolor}%
\pgftext[x=2.115667in, y=1.474448in, left, base,rotate=330.462100]{\color{textcolor}\sffamily\fontsize{8.000000}{9.600000}\selectfont 4.50}%
\end{pgfscope}%
\begin{pgfscope}%
\definecolor{textcolor}{rgb}{0.221989,0.339161,0.548752}%
\pgfsetstrokecolor{textcolor}%
\pgfsetfillcolor{textcolor}%
\pgftext[x=7.078771in, y=3.034732in, left, base,rotate=313.358552]{\color{textcolor}\sffamily\fontsize{8.000000}{9.600000}\selectfont 4.50}%
\end{pgfscope}%
\begin{pgfscope}%
\definecolor{textcolor}{rgb}{0.212395,0.359683,0.551710}%
\pgfsetstrokecolor{textcolor}%
\pgfsetfillcolor{textcolor}%
\pgftext[x=7.438728in, y=2.814543in, left, base,rotate=316.246023]{\color{textcolor}\sffamily\fontsize{8.000000}{9.600000}\selectfont 4.75}%
\end{pgfscope}%
\begin{pgfscope}%
\definecolor{textcolor}{rgb}{0.212395,0.359683,0.551710}%
\pgfsetstrokecolor{textcolor}%
\pgfsetfillcolor{textcolor}%
\pgftext[x=3.184773in, y=0.868603in, left, base,rotate=338.021835]{\color{textcolor}\sffamily\fontsize{8.000000}{9.600000}\selectfont 4.75}%
\end{pgfscope}%
\begin{pgfscope}%
\definecolor{textcolor}{rgb}{0.201239,0.383670,0.554294}%
\pgfsetstrokecolor{textcolor}%
\pgfsetfillcolor{textcolor}%
\pgftext[x=6.853318in, y=3.767808in, left, base,rotate=301.313671]{\color{textcolor}\sffamily\fontsize{8.000000}{9.600000}\selectfont 5.00}%
\end{pgfscope}%
\begin{pgfscope}%
\definecolor{textcolor}{rgb}{0.201239,0.383670,0.554294}%
\pgfsetstrokecolor{textcolor}%
\pgfsetfillcolor{textcolor}%
\pgftext[x=1.736294in, y=1.544671in, left, base,rotate=328.555006]{\color{textcolor}\sffamily\fontsize{8.000000}{9.600000}\selectfont 5.00}%
\end{pgfscope}%
\begin{pgfscope}%
\definecolor{textcolor}{rgb}{0.190631,0.407061,0.556089}%
\pgfsetstrokecolor{textcolor}%
\pgfsetfillcolor{textcolor}%
\pgftext[x=7.157618in, y=3.504072in, left, base,rotate=306.039222]{\color{textcolor}\sffamily\fontsize{8.000000}{9.600000}\selectfont 5.25}%
\end{pgfscope}%
\begin{pgfscope}%
\definecolor{textcolor}{rgb}{0.190631,0.407061,0.556089}%
\pgfsetstrokecolor{textcolor}%
\pgfsetfillcolor{textcolor}%
\pgftext[x=1.323958in, y=1.748308in, left, base,rotate=324.710769]{\color{textcolor}\sffamily\fontsize{8.000000}{9.600000}\selectfont 5.25}%
\end{pgfscope}%
\begin{pgfscope}%
\definecolor{textcolor}{rgb}{0.180629,0.429975,0.557282}%
\pgfsetstrokecolor{textcolor}%
\pgfsetfillcolor{textcolor}%
\pgftext[x=7.463743in, y=3.271995in, left, base,rotate=309.871623]{\color{textcolor}\sffamily\fontsize{8.000000}{9.600000}\selectfont 5.50}%
\end{pgfscope}%
\begin{pgfscope}%
\definecolor{textcolor}{rgb}{0.180629,0.429975,0.557282}%
\pgfsetstrokecolor{textcolor}%
\pgfsetfillcolor{textcolor}%
\pgftext[x=2.788507in, y=0.853716in, left, base,rotate=337.272447]{\color{textcolor}\sffamily\fontsize{8.000000}{9.600000}\selectfont 5.50}%
\end{pgfscope}%
\begin{pgfscope}%
\definecolor{textcolor}{rgb}{0.172719,0.448791,0.557885}%
\pgfsetstrokecolor{textcolor}%
\pgfsetfillcolor{textcolor}%
\pgftext[x=7.219888in, y=3.868733in, left, base,rotate=298.903294]{\color{textcolor}\sffamily\fontsize{8.000000}{9.600000}\selectfont 5.75}%
\end{pgfscope}%
\begin{pgfscope}%
\definecolor{textcolor}{rgb}{0.172719,0.448791,0.557885}%
\pgfsetstrokecolor{textcolor}%
\pgfsetfillcolor{textcolor}%
\pgftext[x=2.307947in, y=1.016080in, left, base,rotate=335.078245]{\color{textcolor}\sffamily\fontsize{8.000000}{9.600000}\selectfont 5.75}%
\end{pgfscope}%
\begin{pgfscope}%
\definecolor{textcolor}{rgb}{0.163625,0.471133,0.558148}%
\pgfsetstrokecolor{textcolor}%
\pgfsetfillcolor{textcolor}%
\pgftext[x=7.123285in, y=4.572149in, left, base,rotate=282.487157]{\color{textcolor}\sffamily\fontsize{8.000000}{9.600000}\selectfont 6.00}%
\end{pgfscope}%
\begin{pgfscope}%
\definecolor{textcolor}{rgb}{0.163625,0.471133,0.558148}%
\pgfsetstrokecolor{textcolor}%
\pgfsetfillcolor{textcolor}%
\pgftext[x=1.837792in, y=1.197864in, left, base,rotate=332.395189]{\color{textcolor}\sffamily\fontsize{8.000000}{9.600000}\selectfont 6.00}%
\end{pgfscope}%
\begin{pgfscope}%
\definecolor{textcolor}{rgb}{0.154815,0.493313,0.557840}%
\pgfsetstrokecolor{textcolor}%
\pgfsetfillcolor{textcolor}%
\pgftext[x=7.321028in, y=4.275529in, left, base,rotate=289.494063]{\color{textcolor}\sffamily\fontsize{8.000000}{9.600000}\selectfont 6.25}%
\end{pgfscope}%
\begin{pgfscope}%
\definecolor{textcolor}{rgb}{0.154815,0.493313,0.557840}%
\pgfsetstrokecolor{textcolor}%
\pgfsetfillcolor{textcolor}%
\pgftext[x=1.389912in, y=1.396917in, left, base,rotate=329.157911]{\color{textcolor}\sffamily\fontsize{8.000000}{9.600000}\selectfont 6.25}%
\end{pgfscope}%
\begin{pgfscope}%
\definecolor{textcolor}{rgb}{0.147607,0.511733,0.557049}%
\pgfsetstrokecolor{textcolor}%
\pgfsetfillcolor{textcolor}%
\pgftext[x=7.542751in, y=4.004854in, left, base,rotate=295.608374]{\color{textcolor}\sffamily\fontsize{8.000000}{9.600000}\selectfont 6.50}%
\end{pgfscope}%
\begin{pgfscope}%
\definecolor{textcolor}{rgb}{0.147607,0.511733,0.557049}%
\pgfsetstrokecolor{textcolor}%
\pgfsetfillcolor{textcolor}%
\pgftext[x=0.943187in, y=1.633734in, left, base,rotate=324.944088]{\color{textcolor}\sffamily\fontsize{8.000000}{9.600000}\selectfont 6.50}%
\end{pgfscope}%
\begin{pgfscope}%
\definecolor{textcolor}{rgb}{0.139147,0.533812,0.555298}%
\pgfsetstrokecolor{textcolor}%
\pgfsetfillcolor{textcolor}%
\pgftext[x=7.560789in, y=5.171165in, left, base,rotate=79.990114]{\color{textcolor}\sffamily\fontsize{8.000000}{9.600000}\selectfont 6.75}%
\end{pgfscope}%
\begin{pgfscope}%
\definecolor{textcolor}{rgb}{0.139147,0.533812,0.555298}%
\pgfsetstrokecolor{textcolor}%
\pgfsetfillcolor{textcolor}%
\pgftext[x=2.411038in, y=0.748916in, left, base,rotate=337.163979]{\color{textcolor}\sffamily\fontsize{8.000000}{9.600000}\selectfont 6.75}%
\end{pgfscope}%
\begin{pgfscope}%
\definecolor{textcolor}{rgb}{0.131172,0.555899,0.552459}%
\pgfsetstrokecolor{textcolor}%
\pgfsetfillcolor{textcolor}%
\pgftext[x=7.620120in, y=4.627013in, left, base,rotate=280.008219]{\color{textcolor}\sffamily\fontsize{8.000000}{9.600000}\selectfont 7.00}%
\end{pgfscope}%
\begin{pgfscope}%
\definecolor{textcolor}{rgb}{0.131172,0.555899,0.552459}%
\pgfsetstrokecolor{textcolor}%
\pgfsetfillcolor{textcolor}%
\pgftext[x=2.420532in, y=0.695624in, left, base,rotate=337.515664]{\color{textcolor}\sffamily\fontsize{8.000000}{9.600000}\selectfont 7.00}%
\end{pgfscope}%
\begin{pgfscope}%
\definecolor{textcolor}{rgb}{0.125394,0.574318,0.549086}%
\pgfsetstrokecolor{textcolor}%
\pgfsetfillcolor{textcolor}%
\pgftext[x=1.905675in, y=0.881968in, left, base,rotate=335.166869]{\color{textcolor}\sffamily\fontsize{8.000000}{9.600000}\selectfont 7.25}%
\end{pgfscope}%
\begin{pgfscope}%
\definecolor{textcolor}{rgb}{0.120565,0.596422,0.543611}%
\pgfsetstrokecolor{textcolor}%
\pgfsetfillcolor{textcolor}%
\pgftext[x=1.908384in, y=0.831247in, left, base,rotate=335.543220]{\color{textcolor}\sffamily\fontsize{8.000000}{9.600000}\selectfont 7.50}%
\end{pgfscope}%
\begin{pgfscope}%
\definecolor{textcolor}{rgb}{0.119699,0.618490,0.536347}%
\pgfsetstrokecolor{textcolor}%
\pgfsetfillcolor{textcolor}%
\pgftext[x=1.435441in, y=1.020790in, left, base,rotate=332.912315]{\color{textcolor}\sffamily\fontsize{8.000000}{9.600000}\selectfont 7.75}%
\end{pgfscope}%
\begin{pgfscope}%
\definecolor{textcolor}{rgb}{0.123444,0.636809,0.528763}%
\pgfsetstrokecolor{textcolor}%
\pgfsetfillcolor{textcolor}%
\pgftext[x=0.967761in, y=1.238063in, left, base,rotate=329.564287]{\color{textcolor}\sffamily\fontsize{8.000000}{9.600000}\selectfont 8.00}%
\end{pgfscope}%
\begin{pgfscope}%
\definecolor{textcolor}{rgb}{0.134692,0.658636,0.517649}%
\pgfsetstrokecolor{textcolor}%
\pgfsetfillcolor{textcolor}%
\pgftext[x=1.451710in, y=0.913731in, left, base,rotate=333.843732]{\color{textcolor}\sffamily\fontsize{8.000000}{9.600000}\selectfont 8.25}%
\end{pgfscope}%
\begin{pgfscope}%
\definecolor{textcolor}{rgb}{0.153894,0.680203,0.504172}%
\pgfsetstrokecolor{textcolor}%
\pgfsetfillcolor{textcolor}%
\pgftext[x=1.456300in, y=0.864723in, left, base,rotate=334.243368]{\color{textcolor}\sffamily\fontsize{8.000000}{9.600000}\selectfont 8.50}%
\end{pgfscope}%
\begin{pgfscope}%
\definecolor{textcolor}{rgb}{0.175707,0.697900,0.491033}%
\pgfsetstrokecolor{textcolor}%
\pgfsetfillcolor{textcolor}%
\pgftext[x=1.456142in, y=0.819621in, left, base,rotate=334.595918]{\color{textcolor}\sffamily\fontsize{8.000000}{9.600000}\selectfont 8.75}%
\end{pgfscope}%
\begin{pgfscope}%
\definecolor{textcolor}{rgb}{0.208030,0.718701,0.472873}%
\pgfsetstrokecolor{textcolor}%
\pgfsetfillcolor{textcolor}%
\pgftext[x=1.455996in, y=0.775826in, left, base,rotate=334.929559]{\color{textcolor}\sffamily\fontsize{8.000000}{9.600000}\selectfont 9.00}%
\end{pgfscope}%
\begin{pgfscope}%
\definecolor{textcolor}{rgb}{0.246070,0.738910,0.452024}%
\pgfsetstrokecolor{textcolor}%
\pgfsetfillcolor{textcolor}%
\pgftext[x=1.469600in, y=0.726700in, left, base,rotate=335.318745]{\color{textcolor}\sffamily\fontsize{8.000000}{9.600000}\selectfont 9.25}%
\end{pgfscope}%
\begin{pgfscope}%
\definecolor{textcolor}{rgb}{0.288921,0.758394,0.428426}%
\pgfsetstrokecolor{textcolor}%
\pgfsetfillcolor{textcolor}%
\pgftext[x=0.972828in, y=0.939035in, left, base,rotate=332.397059]{\color{textcolor}\sffamily\fontsize{8.000000}{9.600000}\selectfont 9.50}%
\end{pgfscope}%
\begin{pgfscope}%
\definecolor{textcolor}{rgb}{0.327796,0.773980,0.406640}%
\pgfsetstrokecolor{textcolor}%
\pgfsetfillcolor{textcolor}%
\pgftext[x=0.972634in, y=0.895191in, left, base,rotate=332.771157]{\color{textcolor}\sffamily\fontsize{8.000000}{9.600000}\selectfont 9.75}%
\end{pgfscope}%
\begin{pgfscope}%
\definecolor{textcolor}{rgb}{0.377779,0.791781,0.377939}%
\pgfsetstrokecolor{textcolor}%
\pgfsetfillcolor{textcolor}%
\pgftext[x=1.009762in, y=0.831961in, left, base,rotate=333.596964]{\color{textcolor}\sffamily\fontsize{8.000000}{9.600000}\selectfont 10.00}%
\end{pgfscope}%
\begin{pgfscope}%
\definecolor{textcolor}{rgb}{0.430983,0.808473,0.346476}%
\pgfsetstrokecolor{textcolor}%
\pgfsetfillcolor{textcolor}%
\pgftext[x=1.009517in, y=0.790872in, left, base,rotate=333.924265]{\color{textcolor}\sffamily\fontsize{8.000000}{9.600000}\selectfont 10.25}%
\end{pgfscope}%
\begin{pgfscope}%
\definecolor{textcolor}{rgb}{0.477504,0.821444,0.318195}%
\pgfsetstrokecolor{textcolor}%
\pgfsetfillcolor{textcolor}%
\pgftext[x=1.009291in, y=0.750864in, left, base,rotate=334.233719]{\color{textcolor}\sffamily\fontsize{8.000000}{9.600000}\selectfont 10.50}%
\end{pgfscope}%
\begin{pgfscope}%
\definecolor{textcolor}{rgb}{0.535621,0.835785,0.281908}%
\pgfsetstrokecolor{textcolor}%
\pgfsetfillcolor{textcolor}%
\pgftext[x=0.974634in, y=0.729332in, left, base,rotate=334.313760]{\color{textcolor}\sffamily\fontsize{8.000000}{9.600000}\selectfont 10.75}%
\end{pgfscope}%
\begin{pgfscope}%
\definecolor{textcolor}{rgb}{0.595839,0.848717,0.243329}%
\pgfsetstrokecolor{textcolor}%
\pgfsetfillcolor{textcolor}%
\pgftext[x=0.920996in, y=0.718079in, left, base,rotate=334.267925]{\color{textcolor}\sffamily\fontsize{8.000000}{9.600000}\selectfont 11.00}%
\end{pgfscope}%
\end{pgfpicture}%
\makeatother%
\endgroup%
}
        \caption{Pohľad zhora (Vrstevnice)}
        \label{fig:newton_vlavo}
    \end{subfigure}
    \hfill
    % --- PRAVÝ OBRÁZOK ---
    \begin{subfigure}[b]{0.48\textwidth}
        \centering
        \resizebox{\linewidth}{!}{%% Creator: Matplotlib, PGF backend
%%
%% To include the figure in your LaTeX document, write
%%   \input{<filename>.pgf}
%%
%% Make sure the required packages are loaded in your preamble
%%   \usepackage{pgf}
%%
%% Also ensure that all the required font packages are loaded; for instance,
%% the lmodern package is sometimes necessary when using math font.
%%   \usepackage{lmodern}
%%
%% Figures using additional raster images can only be included by \input if
%% they are in the same directory as the main LaTeX file. For loading figures
%% from other directories you can use the `import` package
%%   \usepackage{import}
%%
%% and then include the figures with
%%   \import{<path to file>}{<filename>.pgf}
%%
%% Matplotlib used the following preamble
%%   
%%   \usepackage{fontspec}
%%   \setmainfont{DejaVuSerif.ttf}[Path=\detokenize{/home/radimek/Documents/projekt_mat_prog/mat_prog_kernel/lib/python3.12/site-packages/matplotlib/mpl-data/fonts/ttf/}]
%%   \setsansfont{DejaVuSans.ttf}[Path=\detokenize{/home/radimek/Documents/projekt_mat_prog/mat_prog_kernel/lib/python3.12/site-packages/matplotlib/mpl-data/fonts/ttf/}]
%%   \setmonofont{DejaVuSansMono.ttf}[Path=\detokenize{/home/radimek/Documents/projekt_mat_prog/mat_prog_kernel/lib/python3.12/site-packages/matplotlib/mpl-data/fonts/ttf/}]
%%   \makeatletter\@ifpackageloaded{underscore}{}{\usepackage[strings]{underscore}}\makeatother
%%
\begingroup%
\makeatletter%
\begin{pgfpicture}%
\pgfpathrectangle{\pgfpointorigin}{\pgfqpoint{8.000000in}{6.000000in}}%
\pgfusepath{use as bounding box, clip}%
\begin{pgfscope}%
\pgfsetbuttcap%
\pgfsetmiterjoin%
\definecolor{currentfill}{rgb}{1.000000,1.000000,1.000000}%
\pgfsetfillcolor{currentfill}%
\pgfsetlinewidth{0.000000pt}%
\definecolor{currentstroke}{rgb}{1.000000,1.000000,1.000000}%
\pgfsetstrokecolor{currentstroke}%
\pgfsetdash{}{0pt}%
\pgfpathmoveto{\pgfqpoint{0.000000in}{0.000000in}}%
\pgfpathlineto{\pgfqpoint{8.000000in}{0.000000in}}%
\pgfpathlineto{\pgfqpoint{8.000000in}{6.000000in}}%
\pgfpathlineto{\pgfqpoint{0.000000in}{6.000000in}}%
\pgfpathlineto{\pgfqpoint{0.000000in}{0.000000in}}%
\pgfpathclose%
\pgfusepath{fill}%
\end{pgfscope}%
\begin{pgfscope}%
\pgfsetbuttcap%
\pgfsetmiterjoin%
\definecolor{currentfill}{rgb}{1.000000,1.000000,1.000000}%
\pgfsetfillcolor{currentfill}%
\pgfsetlinewidth{0.000000pt}%
\definecolor{currentstroke}{rgb}{0.000000,0.000000,0.000000}%
\pgfsetstrokecolor{currentstroke}%
\pgfsetstrokeopacity{0.000000}%
\pgfsetdash{}{0pt}%
\pgfpathmoveto{\pgfqpoint{1.150000in}{0.150000in}}%
\pgfpathlineto{\pgfqpoint{6.850000in}{0.150000in}}%
\pgfpathlineto{\pgfqpoint{6.850000in}{5.850000in}}%
\pgfpathlineto{\pgfqpoint{1.150000in}{5.850000in}}%
\pgfpathlineto{\pgfqpoint{1.150000in}{0.150000in}}%
\pgfpathclose%
\pgfusepath{fill}%
\end{pgfscope}%
\begin{pgfscope}%
\pgfsetbuttcap%
\pgfsetmiterjoin%
\definecolor{currentfill}{rgb}{0.950000,0.950000,0.950000}%
\pgfsetfillcolor{currentfill}%
\pgfsetfillopacity{0.500000}%
\pgfsetlinewidth{1.003750pt}%
\definecolor{currentstroke}{rgb}{0.950000,0.950000,0.950000}%
\pgfsetstrokecolor{currentstroke}%
\pgfsetstrokeopacity{0.500000}%
\pgfsetdash{}{0pt}%
\pgfpathmoveto{\pgfqpoint{1.580389in}{1.555437in}}%
\pgfpathlineto{\pgfqpoint{3.462715in}{3.133240in}}%
\pgfpathlineto{\pgfqpoint{3.436549in}{5.408715in}}%
\pgfpathlineto{\pgfqpoint{1.464144in}{3.969343in}}%
\pgfusepath{stroke,fill}%
\end{pgfscope}%
\begin{pgfscope}%
\pgfsetbuttcap%
\pgfsetmiterjoin%
\definecolor{currentfill}{rgb}{0.900000,0.900000,0.900000}%
\pgfsetfillcolor{currentfill}%
\pgfsetfillopacity{0.500000}%
\pgfsetlinewidth{1.003750pt}%
\definecolor{currentstroke}{rgb}{0.900000,0.900000,0.900000}%
\pgfsetstrokecolor{currentstroke}%
\pgfsetstrokeopacity{0.500000}%
\pgfsetdash{}{0pt}%
\pgfpathmoveto{\pgfqpoint{3.462715in}{3.133240in}}%
\pgfpathlineto{\pgfqpoint{6.483177in}{2.255311in}}%
\pgfpathlineto{\pgfqpoint{6.590967in}{4.609162in}}%
\pgfpathlineto{\pgfqpoint{3.436549in}{5.408715in}}%
\pgfusepath{stroke,fill}%
\end{pgfscope}%
\begin{pgfscope}%
\pgfsetbuttcap%
\pgfsetmiterjoin%
\definecolor{currentfill}{rgb}{0.925000,0.925000,0.925000}%
\pgfsetfillcolor{currentfill}%
\pgfsetfillopacity{0.500000}%
\pgfsetlinewidth{1.003750pt}%
\definecolor{currentstroke}{rgb}{0.925000,0.925000,0.925000}%
\pgfsetstrokecolor{currentstroke}%
\pgfsetstrokeopacity{0.500000}%
\pgfsetdash{}{0pt}%
\pgfpathmoveto{\pgfqpoint{1.580389in}{1.555437in}}%
\pgfpathlineto{\pgfqpoint{4.782226in}{0.509717in}}%
\pgfpathlineto{\pgfqpoint{6.483177in}{2.255311in}}%
\pgfpathlineto{\pgfqpoint{3.462715in}{3.133240in}}%
\pgfusepath{stroke,fill}%
\end{pgfscope}%
\begin{pgfscope}%
\pgfsetrectcap%
\pgfsetroundjoin%
\pgfsetlinewidth{0.803000pt}%
\definecolor{currentstroke}{rgb}{0.000000,0.000000,0.000000}%
\pgfsetstrokecolor{currentstroke}%
\pgfsetdash{}{0pt}%
\pgfpathmoveto{\pgfqpoint{1.580389in}{1.555437in}}%
\pgfpathlineto{\pgfqpoint{4.782226in}{0.509717in}}%
\pgfusepath{stroke}%
\end{pgfscope}%
\begin{pgfscope}%
\definecolor{textcolor}{rgb}{0.000000,0.000000,0.000000}%
\pgfsetstrokecolor{textcolor}%
\pgfsetfillcolor{textcolor}%
\pgftext[x=2.913491in,y=0.557898in,,]{\color{textcolor}\sffamily\fontsize{10.000000}{12.000000}\selectfont x}%
\end{pgfscope}%
\begin{pgfscope}%
\pgfsetbuttcap%
\pgfsetroundjoin%
\pgfsetlinewidth{0.803000pt}%
\definecolor{currentstroke}{rgb}{0.690196,0.690196,0.690196}%
\pgfsetstrokecolor{currentstroke}%
\pgfsetdash{}{0pt}%
\pgfpathmoveto{\pgfqpoint{1.774309in}{1.492103in}}%
\pgfpathlineto{\pgfqpoint{3.646411in}{3.079847in}}%
\pgfpathlineto{\pgfqpoint{3.628011in}{5.360185in}}%
\pgfusepath{stroke}%
\end{pgfscope}%
\begin{pgfscope}%
\pgfsetbuttcap%
\pgfsetroundjoin%
\pgfsetlinewidth{0.803000pt}%
\definecolor{currentstroke}{rgb}{0.690196,0.690196,0.690196}%
\pgfsetstrokecolor{currentstroke}%
\pgfsetdash{}{0pt}%
\pgfpathmoveto{\pgfqpoint{2.222368in}{1.345767in}}%
\pgfpathlineto{\pgfqpoint{4.070468in}{2.956591in}}%
\pgfpathlineto{\pgfqpoint{4.070186in}{5.248106in}}%
\pgfusepath{stroke}%
\end{pgfscope}%
\begin{pgfscope}%
\pgfsetbuttcap%
\pgfsetroundjoin%
\pgfsetlinewidth{0.803000pt}%
\definecolor{currentstroke}{rgb}{0.690196,0.690196,0.690196}%
\pgfsetstrokecolor{currentstroke}%
\pgfsetdash{}{0pt}%
\pgfpathmoveto{\pgfqpoint{2.677247in}{1.197204in}}%
\pgfpathlineto{\pgfqpoint{4.500444in}{2.831614in}}%
\pgfpathlineto{\pgfqpoint{4.518800in}{5.134396in}}%
\pgfusepath{stroke}%
\end{pgfscope}%
\begin{pgfscope}%
\pgfsetbuttcap%
\pgfsetroundjoin%
\pgfsetlinewidth{0.803000pt}%
\definecolor{currentstroke}{rgb}{0.690196,0.690196,0.690196}%
\pgfsetstrokecolor{currentstroke}%
\pgfsetdash{}{0pt}%
\pgfpathmoveto{\pgfqpoint{3.139103in}{1.046362in}}%
\pgfpathlineto{\pgfqpoint{4.936464in}{2.704880in}}%
\pgfpathlineto{\pgfqpoint{4.973994in}{5.019017in}}%
\pgfusepath{stroke}%
\end{pgfscope}%
\begin{pgfscope}%
\pgfsetbuttcap%
\pgfsetroundjoin%
\pgfsetlinewidth{0.803000pt}%
\definecolor{currentstroke}{rgb}{0.690196,0.690196,0.690196}%
\pgfsetstrokecolor{currentstroke}%
\pgfsetdash{}{0pt}%
\pgfpathmoveto{\pgfqpoint{3.608098in}{0.893188in}}%
\pgfpathlineto{\pgfqpoint{5.378655in}{2.576352in}}%
\pgfpathlineto{\pgfqpoint{5.435914in}{4.901934in}}%
\pgfusepath{stroke}%
\end{pgfscope}%
\begin{pgfscope}%
\pgfsetbuttcap%
\pgfsetroundjoin%
\pgfsetlinewidth{0.803000pt}%
\definecolor{currentstroke}{rgb}{0.690196,0.690196,0.690196}%
\pgfsetstrokecolor{currentstroke}%
\pgfsetdash{}{0pt}%
\pgfpathmoveto{\pgfqpoint{4.084398in}{0.737628in}}%
\pgfpathlineto{\pgfqpoint{5.827149in}{2.445993in}}%
\pgfpathlineto{\pgfqpoint{5.904712in}{4.783107in}}%
\pgfusepath{stroke}%
\end{pgfscope}%
\begin{pgfscope}%
\pgfsetbuttcap%
\pgfsetroundjoin%
\pgfsetlinewidth{0.803000pt}%
\definecolor{currentstroke}{rgb}{0.690196,0.690196,0.690196}%
\pgfsetstrokecolor{currentstroke}%
\pgfsetdash{}{0pt}%
\pgfpathmoveto{\pgfqpoint{4.568177in}{0.579626in}}%
\pgfpathlineto{\pgfqpoint{6.282083in}{2.313762in}}%
\pgfpathlineto{\pgfqpoint{6.380540in}{4.662499in}}%
\pgfusepath{stroke}%
\end{pgfscope}%
\begin{pgfscope}%
\pgfsetrectcap%
\pgfsetroundjoin%
\pgfsetlinewidth{0.803000pt}%
\definecolor{currentstroke}{rgb}{0.000000,0.000000,0.000000}%
\pgfsetstrokecolor{currentstroke}%
\pgfsetdash{}{0pt}%
\pgfpathmoveto{\pgfqpoint{1.790612in}{1.505929in}}%
\pgfpathlineto{\pgfqpoint{1.741635in}{1.464392in}}%
\pgfusepath{stroke}%
\end{pgfscope}%
\begin{pgfscope}%
\definecolor{textcolor}{rgb}{0.000000,0.000000,0.000000}%
\pgfsetstrokecolor{textcolor}%
\pgfsetfillcolor{textcolor}%
\pgftext[x=1.669876in,y=1.274184in,,top]{\color{textcolor}\sffamily\fontsize{10.000000}{12.000000}\selectfont \ensuremath{-}1.0}%
\end{pgfscope}%
\begin{pgfscope}%
\pgfsetrectcap%
\pgfsetroundjoin%
\pgfsetlinewidth{0.803000pt}%
\definecolor{currentstroke}{rgb}{0.000000,0.000000,0.000000}%
\pgfsetstrokecolor{currentstroke}%
\pgfsetdash{}{0pt}%
\pgfpathmoveto{\pgfqpoint{2.238471in}{1.359803in}}%
\pgfpathlineto{\pgfqpoint{2.190092in}{1.317635in}}%
\pgfusepath{stroke}%
\end{pgfscope}%
\begin{pgfscope}%
\definecolor{textcolor}{rgb}{0.000000,0.000000,0.000000}%
\pgfsetstrokecolor{textcolor}%
\pgfsetfillcolor{textcolor}%
\pgftext[x=2.118230in,y=1.125843in,,top]{\color{textcolor}\sffamily\fontsize{10.000000}{12.000000}\selectfont \ensuremath{-}0.5}%
\end{pgfscope}%
\begin{pgfscope}%
\pgfsetrectcap%
\pgfsetroundjoin%
\pgfsetlinewidth{0.803000pt}%
\definecolor{currentstroke}{rgb}{0.000000,0.000000,0.000000}%
\pgfsetstrokecolor{currentstroke}%
\pgfsetdash{}{0pt}%
\pgfpathmoveto{\pgfqpoint{2.693143in}{1.211454in}}%
\pgfpathlineto{\pgfqpoint{2.645386in}{1.168642in}}%
\pgfusepath{stroke}%
\end{pgfscope}%
\begin{pgfscope}%
\definecolor{textcolor}{rgb}{0.000000,0.000000,0.000000}%
\pgfsetstrokecolor{textcolor}%
\pgfsetfillcolor{textcolor}%
\pgftext[x=2.573426in,y=0.975238in,,top]{\color{textcolor}\sffamily\fontsize{10.000000}{12.000000}\selectfont 0.0}%
\end{pgfscope}%
\begin{pgfscope}%
\pgfsetrectcap%
\pgfsetroundjoin%
\pgfsetlinewidth{0.803000pt}%
\definecolor{currentstroke}{rgb}{0.000000,0.000000,0.000000}%
\pgfsetstrokecolor{currentstroke}%
\pgfsetdash{}{0pt}%
\pgfpathmoveto{\pgfqpoint{3.154783in}{1.060831in}}%
\pgfpathlineto{\pgfqpoint{3.107673in}{1.017359in}}%
\pgfusepath{stroke}%
\end{pgfscope}%
\begin{pgfscope}%
\definecolor{textcolor}{rgb}{0.000000,0.000000,0.000000}%
\pgfsetstrokecolor{textcolor}%
\pgfsetfillcolor{textcolor}%
\pgftext[x=3.035622in,y=0.822318in,,top]{\color{textcolor}\sffamily\fontsize{10.000000}{12.000000}\selectfont 0.5}%
\end{pgfscope}%
\begin{pgfscope}%
\pgfsetrectcap%
\pgfsetroundjoin%
\pgfsetlinewidth{0.803000pt}%
\definecolor{currentstroke}{rgb}{0.000000,0.000000,0.000000}%
\pgfsetstrokecolor{currentstroke}%
\pgfsetdash{}{0pt}%
\pgfpathmoveto{\pgfqpoint{3.623554in}{0.907881in}}%
\pgfpathlineto{\pgfqpoint{3.577116in}{0.863735in}}%
\pgfusepath{stroke}%
\end{pgfscope}%
\begin{pgfscope}%
\definecolor{textcolor}{rgb}{0.000000,0.000000,0.000000}%
\pgfsetstrokecolor{textcolor}%
\pgfsetfillcolor{textcolor}%
\pgftext[x=3.504979in,y=0.667028in,,top]{\color{textcolor}\sffamily\fontsize{10.000000}{12.000000}\selectfont 1.0}%
\end{pgfscope}%
\begin{pgfscope}%
\pgfsetrectcap%
\pgfsetroundjoin%
\pgfsetlinewidth{0.803000pt}%
\definecolor{currentstroke}{rgb}{0.000000,0.000000,0.000000}%
\pgfsetstrokecolor{currentstroke}%
\pgfsetdash{}{0pt}%
\pgfpathmoveto{\pgfqpoint{4.099622in}{0.752551in}}%
\pgfpathlineto{\pgfqpoint{4.053883in}{0.707715in}}%
\pgfusepath{stroke}%
\end{pgfscope}%
\begin{pgfscope}%
\definecolor{textcolor}{rgb}{0.000000,0.000000,0.000000}%
\pgfsetstrokecolor{textcolor}%
\pgfsetfillcolor{textcolor}%
\pgftext[x=3.981665in,y=0.509314in,,top]{\color{textcolor}\sffamily\fontsize{10.000000}{12.000000}\selectfont 1.5}%
\end{pgfscope}%
\begin{pgfscope}%
\pgfsetrectcap%
\pgfsetroundjoin%
\pgfsetlinewidth{0.803000pt}%
\definecolor{currentstroke}{rgb}{0.000000,0.000000,0.000000}%
\pgfsetstrokecolor{currentstroke}%
\pgfsetdash{}{0pt}%
\pgfpathmoveto{\pgfqpoint{4.583158in}{0.594784in}}%
\pgfpathlineto{\pgfqpoint{4.538146in}{0.549241in}}%
\pgfusepath{stroke}%
\end{pgfscope}%
\begin{pgfscope}%
\definecolor{textcolor}{rgb}{0.000000,0.000000,0.000000}%
\pgfsetstrokecolor{textcolor}%
\pgfsetfillcolor{textcolor}%
\pgftext[x=4.465855in,y=0.349116in,,top]{\color{textcolor}\sffamily\fontsize{10.000000}{12.000000}\selectfont 2.0}%
\end{pgfscope}%
\begin{pgfscope}%
\pgfsetrectcap%
\pgfsetroundjoin%
\pgfsetlinewidth{0.803000pt}%
\definecolor{currentstroke}{rgb}{0.000000,0.000000,0.000000}%
\pgfsetstrokecolor{currentstroke}%
\pgfsetdash{}{0pt}%
\pgfpathmoveto{\pgfqpoint{6.483177in}{2.255311in}}%
\pgfpathlineto{\pgfqpoint{4.782226in}{0.509717in}}%
\pgfusepath{stroke}%
\end{pgfscope}%
\begin{pgfscope}%
\definecolor{textcolor}{rgb}{0.000000,0.000000,0.000000}%
\pgfsetstrokecolor{textcolor}%
\pgfsetfillcolor{textcolor}%
\pgftext[x=6.045209in,y=1.032725in,,]{\color{textcolor}\sffamily\fontsize{10.000000}{12.000000}\selectfont y}%
\end{pgfscope}%
\begin{pgfscope}%
\pgfsetbuttcap%
\pgfsetroundjoin%
\pgfsetlinewidth{0.803000pt}%
\definecolor{currentstroke}{rgb}{0.690196,0.690196,0.690196}%
\pgfsetstrokecolor{currentstroke}%
\pgfsetdash{}{0pt}%
\pgfpathmoveto{\pgfqpoint{1.600541in}{4.068879in}}%
\pgfpathlineto{\pgfqpoint{1.710097in}{1.664161in}}%
\pgfpathlineto{\pgfqpoint{4.899919in}{0.630499in}}%
\pgfusepath{stroke}%
\end{pgfscope}%
\begin{pgfscope}%
\pgfsetbuttcap%
\pgfsetroundjoin%
\pgfsetlinewidth{0.803000pt}%
\definecolor{currentstroke}{rgb}{0.690196,0.690196,0.690196}%
\pgfsetstrokecolor{currentstroke}%
\pgfsetdash{}{0pt}%
\pgfpathmoveto{\pgfqpoint{1.830662in}{4.236811in}}%
\pgfpathlineto{\pgfqpoint{1.929089in}{1.847724in}}%
\pgfpathlineto{\pgfqpoint{5.098461in}{0.834252in}}%
\pgfusepath{stroke}%
\end{pgfscope}%
\begin{pgfscope}%
\pgfsetbuttcap%
\pgfsetroundjoin%
\pgfsetlinewidth{0.803000pt}%
\definecolor{currentstroke}{rgb}{0.690196,0.690196,0.690196}%
\pgfsetstrokecolor{currentstroke}%
\pgfsetdash{}{0pt}%
\pgfpathmoveto{\pgfqpoint{2.056228in}{4.401419in}}%
\pgfpathlineto{\pgfqpoint{2.143933in}{2.027810in}}%
\pgfpathlineto{\pgfqpoint{5.293045in}{1.033943in}}%
\pgfusepath{stroke}%
\end{pgfscope}%
\begin{pgfscope}%
\pgfsetbuttcap%
\pgfsetroundjoin%
\pgfsetlinewidth{0.803000pt}%
\definecolor{currentstroke}{rgb}{0.690196,0.690196,0.690196}%
\pgfsetstrokecolor{currentstroke}%
\pgfsetdash{}{0pt}%
\pgfpathmoveto{\pgfqpoint{2.277372in}{4.562799in}}%
\pgfpathlineto{\pgfqpoint{2.354746in}{2.204518in}}%
\pgfpathlineto{\pgfqpoint{5.483787in}{1.229691in}}%
\pgfusepath{stroke}%
\end{pgfscope}%
\begin{pgfscope}%
\pgfsetbuttcap%
\pgfsetroundjoin%
\pgfsetlinewidth{0.803000pt}%
\definecolor{currentstroke}{rgb}{0.690196,0.690196,0.690196}%
\pgfsetstrokecolor{currentstroke}%
\pgfsetdash{}{0pt}%
\pgfpathmoveto{\pgfqpoint{2.494222in}{4.721047in}}%
\pgfpathlineto{\pgfqpoint{2.561641in}{2.377942in}}%
\pgfpathlineto{\pgfqpoint{5.670801in}{1.421614in}}%
\pgfusepath{stroke}%
\end{pgfscope}%
\begin{pgfscope}%
\pgfsetbuttcap%
\pgfsetroundjoin%
\pgfsetlinewidth{0.803000pt}%
\definecolor{currentstroke}{rgb}{0.690196,0.690196,0.690196}%
\pgfsetstrokecolor{currentstroke}%
\pgfsetdash{}{0pt}%
\pgfpathmoveto{\pgfqpoint{2.706903in}{4.876252in}}%
\pgfpathlineto{\pgfqpoint{2.764725in}{2.548171in}}%
\pgfpathlineto{\pgfqpoint{5.854194in}{1.609820in}}%
\pgfusepath{stroke}%
\end{pgfscope}%
\begin{pgfscope}%
\pgfsetbuttcap%
\pgfsetroundjoin%
\pgfsetlinewidth{0.803000pt}%
\definecolor{currentstroke}{rgb}{0.690196,0.690196,0.690196}%
\pgfsetstrokecolor{currentstroke}%
\pgfsetdash{}{0pt}%
\pgfpathmoveto{\pgfqpoint{2.915534in}{5.028501in}}%
\pgfpathlineto{\pgfqpoint{2.964104in}{2.715295in}}%
\pgfpathlineto{\pgfqpoint{6.034072in}{1.794419in}}%
\pgfusepath{stroke}%
\end{pgfscope}%
\begin{pgfscope}%
\pgfsetbuttcap%
\pgfsetroundjoin%
\pgfsetlinewidth{0.803000pt}%
\definecolor{currentstroke}{rgb}{0.690196,0.690196,0.690196}%
\pgfsetstrokecolor{currentstroke}%
\pgfsetdash{}{0pt}%
\pgfpathmoveto{\pgfqpoint{3.120229in}{5.177879in}}%
\pgfpathlineto{\pgfqpoint{3.159878in}{2.879396in}}%
\pgfpathlineto{\pgfqpoint{6.210533in}{1.975511in}}%
\pgfusepath{stroke}%
\end{pgfscope}%
\begin{pgfscope}%
\pgfsetbuttcap%
\pgfsetroundjoin%
\pgfsetlinewidth{0.803000pt}%
\definecolor{currentstroke}{rgb}{0.690196,0.690196,0.690196}%
\pgfsetstrokecolor{currentstroke}%
\pgfsetdash{}{0pt}%
\pgfpathmoveto{\pgfqpoint{3.321099in}{5.324464in}}%
\pgfpathlineto{\pgfqpoint{3.352144in}{3.040557in}}%
\pgfpathlineto{\pgfqpoint{6.383674in}{2.153197in}}%
\pgfusepath{stroke}%
\end{pgfscope}%
\begin{pgfscope}%
\pgfsetrectcap%
\pgfsetroundjoin%
\pgfsetlinewidth{0.803000pt}%
\definecolor{currentstroke}{rgb}{0.000000,0.000000,0.000000}%
\pgfsetstrokecolor{currentstroke}%
\pgfsetdash{}{0pt}%
\pgfpathmoveto{\pgfqpoint{4.873038in}{0.639210in}}%
\pgfpathlineto{\pgfqpoint{4.953750in}{0.613055in}}%
\pgfusepath{stroke}%
\end{pgfscope}%
\begin{pgfscope}%
\definecolor{textcolor}{rgb}{0.000000,0.000000,0.000000}%
\pgfsetstrokecolor{textcolor}%
\pgfsetfillcolor{textcolor}%
\pgftext[x=5.078779in,y=0.444104in,,top]{\color{textcolor}\sffamily\fontsize{10.000000}{12.000000}\selectfont \ensuremath{-}1.50}%
\end{pgfscope}%
\begin{pgfscope}%
\pgfsetrectcap%
\pgfsetroundjoin%
\pgfsetlinewidth{0.803000pt}%
\definecolor{currentstroke}{rgb}{0.000000,0.000000,0.000000}%
\pgfsetstrokecolor{currentstroke}%
\pgfsetdash{}{0pt}%
\pgfpathmoveto{\pgfqpoint{5.071766in}{0.842788in}}%
\pgfpathlineto{\pgfqpoint{5.151919in}{0.817158in}}%
\pgfusepath{stroke}%
\end{pgfscope}%
\begin{pgfscope}%
\definecolor{textcolor}{rgb}{0.000000,0.000000,0.000000}%
\pgfsetstrokecolor{textcolor}%
\pgfsetfillcolor{textcolor}%
\pgftext[x=5.275596in,y=0.649813in,,top]{\color{textcolor}\sffamily\fontsize{10.000000}{12.000000}\selectfont \ensuremath{-}1.25}%
\end{pgfscope}%
\begin{pgfscope}%
\pgfsetrectcap%
\pgfsetroundjoin%
\pgfsetlinewidth{0.803000pt}%
\definecolor{currentstroke}{rgb}{0.000000,0.000000,0.000000}%
\pgfsetstrokecolor{currentstroke}%
\pgfsetdash{}{0pt}%
\pgfpathmoveto{\pgfqpoint{5.266534in}{1.042310in}}%
\pgfpathlineto{\pgfqpoint{5.346134in}{1.017188in}}%
\pgfusepath{stroke}%
\end{pgfscope}%
\begin{pgfscope}%
\definecolor{textcolor}{rgb}{0.000000,0.000000,0.000000}%
\pgfsetstrokecolor{textcolor}%
\pgfsetfillcolor{textcolor}%
\pgftext[x=5.468487in,y=0.851419in,,top]{\color{textcolor}\sffamily\fontsize{10.000000}{12.000000}\selectfont \ensuremath{-}1.00}%
\end{pgfscope}%
\begin{pgfscope}%
\pgfsetrectcap%
\pgfsetroundjoin%
\pgfsetlinewidth{0.803000pt}%
\definecolor{currentstroke}{rgb}{0.000000,0.000000,0.000000}%
\pgfsetstrokecolor{currentstroke}%
\pgfsetdash{}{0pt}%
\pgfpathmoveto{\pgfqpoint{5.457458in}{1.237894in}}%
\pgfpathlineto{\pgfqpoint{5.536511in}{1.213266in}}%
\pgfusepath{stroke}%
\end{pgfscope}%
\begin{pgfscope}%
\definecolor{textcolor}{rgb}{0.000000,0.000000,0.000000}%
\pgfsetstrokecolor{textcolor}%
\pgfsetfillcolor{textcolor}%
\pgftext[x=5.657569in,y=1.049043in,,top]{\color{textcolor}\sffamily\fontsize{10.000000}{12.000000}\selectfont \ensuremath{-}0.75}%
\end{pgfscope}%
\begin{pgfscope}%
\pgfsetrectcap%
\pgfsetroundjoin%
\pgfsetlinewidth{0.803000pt}%
\definecolor{currentstroke}{rgb}{0.000000,0.000000,0.000000}%
\pgfsetstrokecolor{currentstroke}%
\pgfsetdash{}{0pt}%
\pgfpathmoveto{\pgfqpoint{5.644652in}{1.429657in}}%
\pgfpathlineto{\pgfqpoint{5.723164in}{1.405507in}}%
\pgfusepath{stroke}%
\end{pgfscope}%
\begin{pgfscope}%
\definecolor{textcolor}{rgb}{0.000000,0.000000,0.000000}%
\pgfsetstrokecolor{textcolor}%
\pgfsetfillcolor{textcolor}%
\pgftext[x=5.842953in,y=1.242803in,,top]{\color{textcolor}\sffamily\fontsize{10.000000}{12.000000}\selectfont \ensuremath{-}0.50}%
\end{pgfscope}%
\begin{pgfscope}%
\pgfsetrectcap%
\pgfsetroundjoin%
\pgfsetlinewidth{0.803000pt}%
\definecolor{currentstroke}{rgb}{0.000000,0.000000,0.000000}%
\pgfsetstrokecolor{currentstroke}%
\pgfsetdash{}{0pt}%
\pgfpathmoveto{\pgfqpoint{5.828223in}{1.617708in}}%
\pgfpathlineto{\pgfqpoint{5.906201in}{1.594025in}}%
\pgfusepath{stroke}%
\end{pgfscope}%
\begin{pgfscope}%
\definecolor{textcolor}{rgb}{0.000000,0.000000,0.000000}%
\pgfsetstrokecolor{textcolor}%
\pgfsetfillcolor{textcolor}%
\pgftext[x=6.024748in,y=1.432811in,,top]{\color{textcolor}\sffamily\fontsize{10.000000}{12.000000}\selectfont \ensuremath{-}0.25}%
\end{pgfscope}%
\begin{pgfscope}%
\pgfsetrectcap%
\pgfsetroundjoin%
\pgfsetlinewidth{0.803000pt}%
\definecolor{currentstroke}{rgb}{0.000000,0.000000,0.000000}%
\pgfsetstrokecolor{currentstroke}%
\pgfsetdash{}{0pt}%
\pgfpathmoveto{\pgfqpoint{6.008276in}{1.802156in}}%
\pgfpathlineto{\pgfqpoint{6.085725in}{1.778924in}}%
\pgfusepath{stroke}%
\end{pgfscope}%
\begin{pgfscope}%
\definecolor{textcolor}{rgb}{0.000000,0.000000,0.000000}%
\pgfsetstrokecolor{textcolor}%
\pgfsetfillcolor{textcolor}%
\pgftext[x=6.203055in,y=1.619174in,,top]{\color{textcolor}\sffamily\fontsize{10.000000}{12.000000}\selectfont 0.00}%
\end{pgfscope}%
\begin{pgfscope}%
\pgfsetrectcap%
\pgfsetroundjoin%
\pgfsetlinewidth{0.803000pt}%
\definecolor{currentstroke}{rgb}{0.000000,0.000000,0.000000}%
\pgfsetstrokecolor{currentstroke}%
\pgfsetdash{}{0pt}%
\pgfpathmoveto{\pgfqpoint{6.184911in}{1.983102in}}%
\pgfpathlineto{\pgfqpoint{6.261837in}{1.960310in}}%
\pgfusepath{stroke}%
\end{pgfscope}%
\begin{pgfscope}%
\definecolor{textcolor}{rgb}{0.000000,0.000000,0.000000}%
\pgfsetstrokecolor{textcolor}%
\pgfsetfillcolor{textcolor}%
\pgftext[x=6.377975in,y=1.801997in,,top]{\color{textcolor}\sffamily\fontsize{10.000000}{12.000000}\selectfont 0.25}%
\end{pgfscope}%
\begin{pgfscope}%
\pgfsetrectcap%
\pgfsetroundjoin%
\pgfsetlinewidth{0.803000pt}%
\definecolor{currentstroke}{rgb}{0.000000,0.000000,0.000000}%
\pgfsetstrokecolor{currentstroke}%
\pgfsetdash{}{0pt}%
\pgfpathmoveto{\pgfqpoint{6.358225in}{2.160646in}}%
\pgfpathlineto{\pgfqpoint{6.434634in}{2.138280in}}%
\pgfusepath{stroke}%
\end{pgfscope}%
\begin{pgfscope}%
\definecolor{textcolor}{rgb}{0.000000,0.000000,0.000000}%
\pgfsetstrokecolor{textcolor}%
\pgfsetfillcolor{textcolor}%
\pgftext[x=6.549603in,y=1.981379in,,top]{\color{textcolor}\sffamily\fontsize{10.000000}{12.000000}\selectfont 0.50}%
\end{pgfscope}%
\begin{pgfscope}%
\pgfsetrectcap%
\pgfsetroundjoin%
\pgfsetlinewidth{0.803000pt}%
\definecolor{currentstroke}{rgb}{0.000000,0.000000,0.000000}%
\pgfsetstrokecolor{currentstroke}%
\pgfsetdash{}{0pt}%
\pgfpathmoveto{\pgfqpoint{6.483177in}{2.255311in}}%
\pgfpathlineto{\pgfqpoint{6.590967in}{4.609162in}}%
\pgfusepath{stroke}%
\end{pgfscope}%
\begin{pgfscope}%
\definecolor{textcolor}{rgb}{0.000000,0.000000,0.000000}%
\pgfsetstrokecolor{textcolor}%
\pgfsetfillcolor{textcolor}%
\pgftext[x=7.097978in,y=3.481758in,,,rotate=87.378092]{\color{textcolor}\sffamily\fontsize{10.000000}{12.000000}\selectfont f(x,y)}%
\end{pgfscope}%
\begin{pgfscope}%
\pgfsetbuttcap%
\pgfsetroundjoin%
\pgfsetlinewidth{0.803000pt}%
\definecolor{currentstroke}{rgb}{0.690196,0.690196,0.690196}%
\pgfsetstrokecolor{currentstroke}%
\pgfsetdash{}{0pt}%
\pgfpathmoveto{\pgfqpoint{6.493334in}{2.477102in}}%
\pgfpathlineto{\pgfqpoint{3.460245in}{3.348056in}}%
\pgfpathlineto{\pgfqpoint{1.569453in}{1.782539in}}%
\pgfusepath{stroke}%
\end{pgfscope}%
\begin{pgfscope}%
\pgfsetbuttcap%
\pgfsetroundjoin%
\pgfsetlinewidth{0.803000pt}%
\definecolor{currentstroke}{rgb}{0.690196,0.690196,0.690196}%
\pgfsetstrokecolor{currentstroke}%
\pgfsetdash{}{0pt}%
\pgfpathmoveto{\pgfqpoint{6.510056in}{2.842272in}}%
\pgfpathlineto{\pgfqpoint{3.456180in}{3.701557in}}%
\pgfpathlineto{\pgfqpoint{1.551439in}{2.156612in}}%
\pgfusepath{stroke}%
\end{pgfscope}%
\begin{pgfscope}%
\pgfsetbuttcap%
\pgfsetroundjoin%
\pgfsetlinewidth{0.803000pt}%
\definecolor{currentstroke}{rgb}{0.690196,0.690196,0.690196}%
\pgfsetstrokecolor{currentstroke}%
\pgfsetdash{}{0pt}%
\pgfpathmoveto{\pgfqpoint{6.527011in}{3.212532in}}%
\pgfpathlineto{\pgfqpoint{3.452061in}{4.059748in}}%
\pgfpathlineto{\pgfqpoint{1.533164in}{2.536098in}}%
\pgfusepath{stroke}%
\end{pgfscope}%
\begin{pgfscope}%
\pgfsetbuttcap%
\pgfsetroundjoin%
\pgfsetlinewidth{0.803000pt}%
\definecolor{currentstroke}{rgb}{0.690196,0.690196,0.690196}%
\pgfsetstrokecolor{currentstroke}%
\pgfsetdash{}{0pt}%
\pgfpathmoveto{\pgfqpoint{6.544204in}{3.587988in}}%
\pgfpathlineto{\pgfqpoint{3.447887in}{4.422723in}}%
\pgfpathlineto{\pgfqpoint{1.514623in}{2.921116in}}%
\pgfusepath{stroke}%
\end{pgfscope}%
\begin{pgfscope}%
\pgfsetbuttcap%
\pgfsetroundjoin%
\pgfsetlinewidth{0.803000pt}%
\definecolor{currentstroke}{rgb}{0.690196,0.690196,0.690196}%
\pgfsetstrokecolor{currentstroke}%
\pgfsetdash{}{0pt}%
\pgfpathmoveto{\pgfqpoint{6.561641in}{3.968752in}}%
\pgfpathlineto{\pgfqpoint{3.443657in}{4.790580in}}%
\pgfpathlineto{\pgfqpoint{1.495810in}{3.311788in}}%
\pgfusepath{stroke}%
\end{pgfscope}%
\begin{pgfscope}%
\pgfsetbuttcap%
\pgfsetroundjoin%
\pgfsetlinewidth{0.803000pt}%
\definecolor{currentstroke}{rgb}{0.690196,0.690196,0.690196}%
\pgfsetstrokecolor{currentstroke}%
\pgfsetdash{}{0pt}%
\pgfpathmoveto{\pgfqpoint{6.579325in}{4.354936in}}%
\pgfpathlineto{\pgfqpoint{3.439370in}{5.163416in}}%
\pgfpathlineto{\pgfqpoint{1.476718in}{3.708239in}}%
\pgfusepath{stroke}%
\end{pgfscope}%
\begin{pgfscope}%
\pgfsetrectcap%
\pgfsetroundjoin%
\pgfsetlinewidth{0.803000pt}%
\definecolor{currentstroke}{rgb}{0.000000,0.000000,0.000000}%
\pgfsetstrokecolor{currentstroke}%
\pgfsetdash{}{0pt}%
\pgfpathmoveto{\pgfqpoint{6.467873in}{2.484413in}}%
\pgfpathlineto{\pgfqpoint{6.544317in}{2.462462in}}%
\pgfusepath{stroke}%
\end{pgfscope}%
\begin{pgfscope}%
\definecolor{textcolor}{rgb}{0.000000,0.000000,0.000000}%
\pgfsetstrokecolor{textcolor}%
\pgfsetfillcolor{textcolor}%
\pgftext[x=6.748104in,y=2.513039in,,top]{\color{textcolor}\sffamily\fontsize{10.000000}{12.000000}\selectfont 2}%
\end{pgfscope}%
\begin{pgfscope}%
\pgfsetrectcap%
\pgfsetroundjoin%
\pgfsetlinewidth{0.803000pt}%
\definecolor{currentstroke}{rgb}{0.000000,0.000000,0.000000}%
\pgfsetstrokecolor{currentstroke}%
\pgfsetdash{}{0pt}%
\pgfpathmoveto{\pgfqpoint{6.484412in}{2.849487in}}%
\pgfpathlineto{\pgfqpoint{6.561405in}{2.827823in}}%
\pgfusepath{stroke}%
\end{pgfscope}%
\begin{pgfscope}%
\definecolor{textcolor}{rgb}{0.000000,0.000000,0.000000}%
\pgfsetstrokecolor{textcolor}%
\pgfsetfillcolor{textcolor}%
\pgftext[x=6.766553in,y=2.877739in,,top]{\color{textcolor}\sffamily\fontsize{10.000000}{12.000000}\selectfont 4}%
\end{pgfscope}%
\begin{pgfscope}%
\pgfsetrectcap%
\pgfsetroundjoin%
\pgfsetlinewidth{0.803000pt}%
\definecolor{currentstroke}{rgb}{0.000000,0.000000,0.000000}%
\pgfsetstrokecolor{currentstroke}%
\pgfsetdash{}{0pt}%
\pgfpathmoveto{\pgfqpoint{6.501182in}{3.219648in}}%
\pgfpathlineto{\pgfqpoint{6.578732in}{3.198281in}}%
\pgfusepath{stroke}%
\end{pgfscope}%
\begin{pgfscope}%
\definecolor{textcolor}{rgb}{0.000000,0.000000,0.000000}%
\pgfsetstrokecolor{textcolor}%
\pgfsetfillcolor{textcolor}%
\pgftext[x=6.785258in,y=3.247512in,,top]{\color{textcolor}\sffamily\fontsize{10.000000}{12.000000}\selectfont 6}%
\end{pgfscope}%
\begin{pgfscope}%
\pgfsetrectcap%
\pgfsetroundjoin%
\pgfsetlinewidth{0.803000pt}%
\definecolor{currentstroke}{rgb}{0.000000,0.000000,0.000000}%
\pgfsetstrokecolor{currentstroke}%
\pgfsetdash{}{0pt}%
\pgfpathmoveto{\pgfqpoint{6.518187in}{3.595002in}}%
\pgfpathlineto{\pgfqpoint{6.596303in}{3.573943in}}%
\pgfusepath{stroke}%
\end{pgfscope}%
\begin{pgfscope}%
\definecolor{textcolor}{rgb}{0.000000,0.000000,0.000000}%
\pgfsetstrokecolor{textcolor}%
\pgfsetfillcolor{textcolor}%
\pgftext[x=6.804226in,y=3.622465in,,top]{\color{textcolor}\sffamily\fontsize{10.000000}{12.000000}\selectfont 8}%
\end{pgfscope}%
\begin{pgfscope}%
\pgfsetrectcap%
\pgfsetroundjoin%
\pgfsetlinewidth{0.803000pt}%
\definecolor{currentstroke}{rgb}{0.000000,0.000000,0.000000}%
\pgfsetstrokecolor{currentstroke}%
\pgfsetdash{}{0pt}%
\pgfpathmoveto{\pgfqpoint{6.535432in}{3.975660in}}%
\pgfpathlineto{\pgfqpoint{6.614122in}{3.954919in}}%
\pgfusepath{stroke}%
\end{pgfscope}%
\begin{pgfscope}%
\definecolor{textcolor}{rgb}{0.000000,0.000000,0.000000}%
\pgfsetstrokecolor{textcolor}%
\pgfsetfillcolor{textcolor}%
\pgftext[x=6.823461in,y=4.002706in,,top]{\color{textcolor}\sffamily\fontsize{10.000000}{12.000000}\selectfont 10}%
\end{pgfscope}%
\begin{pgfscope}%
\pgfsetrectcap%
\pgfsetroundjoin%
\pgfsetlinewidth{0.803000pt}%
\definecolor{currentstroke}{rgb}{0.000000,0.000000,0.000000}%
\pgfsetstrokecolor{currentstroke}%
\pgfsetdash{}{0pt}%
\pgfpathmoveto{\pgfqpoint{6.552923in}{4.361734in}}%
\pgfpathlineto{\pgfqpoint{6.632195in}{4.341323in}}%
\pgfusepath{stroke}%
\end{pgfscope}%
\begin{pgfscope}%
\definecolor{textcolor}{rgb}{0.000000,0.000000,0.000000}%
\pgfsetstrokecolor{textcolor}%
\pgfsetfillcolor{textcolor}%
\pgftext[x=6.842969in,y=4.388350in,,top]{\color{textcolor}\sffamily\fontsize{10.000000}{12.000000}\selectfont 12}%
\end{pgfscope}%
\begin{pgfscope}%
\pgfpathrectangle{\pgfqpoint{1.150000in}{0.150000in}}{\pgfqpoint{5.700000in}{5.700000in}}%
\pgfusepath{clip}%
\pgfsetrectcap%
\pgfsetroundjoin%
\pgfsetlinewidth{2.007500pt}%
\definecolor{currentstroke}{rgb}{0.000000,0.000000,0.000000}%
\pgfsetstrokecolor{currentstroke}%
\pgfsetdash{}{0pt}%
\pgfpathmoveto{\pgfqpoint{4.017786in}{2.619826in}}%
\pgfpathlineto{\pgfqpoint{3.218836in}{1.981271in}}%
\pgfpathlineto{\pgfqpoint{3.773310in}{1.801422in}}%
\pgfpathlineto{\pgfqpoint{3.780380in}{1.896982in}}%
\pgfusepath{stroke}%
\end{pgfscope}%
\begin{pgfscope}%
\pgfpathrectangle{\pgfqpoint{1.150000in}{0.150000in}}{\pgfqpoint{5.700000in}{5.700000in}}%
\pgfusepath{clip}%
\pgfsetbuttcap%
\pgfsetroundjoin%
\definecolor{currentfill}{rgb}{1.000000,0.000000,0.000000}%
\pgfsetfillcolor{currentfill}%
\pgfsetfillopacity{0.300000}%
\pgfsetlinewidth{1.003750pt}%
\definecolor{currentstroke}{rgb}{1.000000,0.000000,0.000000}%
\pgfsetstrokecolor{currentstroke}%
\pgfsetstrokeopacity{0.300000}%
\pgfsetdash{}{0pt}%
\pgfpathmoveto{\pgfqpoint{4.017786in}{2.570722in}}%
\pgfpathcurveto{\pgfqpoint{4.030809in}{2.570722in}}{\pgfqpoint{4.043300in}{2.575896in}}{\pgfqpoint{4.052509in}{2.585104in}}%
\pgfpathcurveto{\pgfqpoint{4.061717in}{2.594312in}}{\pgfqpoint{4.066891in}{2.606804in}}{\pgfqpoint{4.066891in}{2.619826in}}%
\pgfpathcurveto{\pgfqpoint{4.066891in}{2.632849in}}{\pgfqpoint{4.061717in}{2.645340in}}{\pgfqpoint{4.052509in}{2.654548in}}%
\pgfpathcurveto{\pgfqpoint{4.043300in}{2.663757in}}{\pgfqpoint{4.030809in}{2.668931in}}{\pgfqpoint{4.017786in}{2.668931in}}%
\pgfpathcurveto{\pgfqpoint{4.004764in}{2.668931in}}{\pgfqpoint{3.992273in}{2.663757in}}{\pgfqpoint{3.983064in}{2.654548in}}%
\pgfpathcurveto{\pgfqpoint{3.973856in}{2.645340in}}{\pgfqpoint{3.968682in}{2.632849in}}{\pgfqpoint{3.968682in}{2.619826in}}%
\pgfpathcurveto{\pgfqpoint{3.968682in}{2.606804in}}{\pgfqpoint{3.973856in}{2.594312in}}{\pgfqpoint{3.983064in}{2.585104in}}%
\pgfpathcurveto{\pgfqpoint{3.992273in}{2.575896in}}{\pgfqpoint{4.004764in}{2.570722in}}{\pgfqpoint{4.017786in}{2.570722in}}%
\pgfpathlineto{\pgfqpoint{4.017786in}{2.570722in}}%
\pgfpathclose%
\pgfusepath{stroke,fill}%
\end{pgfscope}%
\begin{pgfscope}%
\pgfpathrectangle{\pgfqpoint{1.150000in}{0.150000in}}{\pgfqpoint{5.700000in}{5.700000in}}%
\pgfusepath{clip}%
\pgfsetbuttcap%
\pgfsetroundjoin%
\definecolor{currentfill}{rgb}{1.000000,0.000000,0.000000}%
\pgfsetfillcolor{currentfill}%
\pgfsetfillopacity{0.888484}%
\pgfsetlinewidth{1.003750pt}%
\definecolor{currentstroke}{rgb}{1.000000,0.000000,0.000000}%
\pgfsetstrokecolor{currentstroke}%
\pgfsetstrokeopacity{0.888484}%
\pgfsetdash{}{0pt}%
\pgfpathmoveto{\pgfqpoint{3.780380in}{1.847877in}}%
\pgfpathcurveto{\pgfqpoint{3.793402in}{1.847877in}}{\pgfqpoint{3.805894in}{1.853051in}}{\pgfqpoint{3.815102in}{1.862259in}}%
\pgfpathcurveto{\pgfqpoint{3.824310in}{1.871468in}}{\pgfqpoint{3.829484in}{1.883959in}}{\pgfqpoint{3.829484in}{1.896982in}}%
\pgfpathcurveto{\pgfqpoint{3.829484in}{1.910004in}}{\pgfqpoint{3.824310in}{1.922495in}}{\pgfqpoint{3.815102in}{1.931704in}}%
\pgfpathcurveto{\pgfqpoint{3.805894in}{1.940912in}}{\pgfqpoint{3.793402in}{1.946086in}}{\pgfqpoint{3.780380in}{1.946086in}}%
\pgfpathcurveto{\pgfqpoint{3.767357in}{1.946086in}}{\pgfqpoint{3.754866in}{1.940912in}}{\pgfqpoint{3.745658in}{1.931704in}}%
\pgfpathcurveto{\pgfqpoint{3.736449in}{1.922495in}}{\pgfqpoint{3.731275in}{1.910004in}}{\pgfqpoint{3.731275in}{1.896982in}}%
\pgfpathcurveto{\pgfqpoint{3.731275in}{1.883959in}}{\pgfqpoint{3.736449in}{1.871468in}}{\pgfqpoint{3.745658in}{1.862259in}}%
\pgfpathcurveto{\pgfqpoint{3.754866in}{1.853051in}}{\pgfqpoint{3.767357in}{1.847877in}}{\pgfqpoint{3.780380in}{1.847877in}}%
\pgfpathlineto{\pgfqpoint{3.780380in}{1.847877in}}%
\pgfpathclose%
\pgfusepath{stroke,fill}%
\end{pgfscope}%
\begin{pgfscope}%
\pgfpathrectangle{\pgfqpoint{1.150000in}{0.150000in}}{\pgfqpoint{5.700000in}{5.700000in}}%
\pgfusepath{clip}%
\pgfsetbuttcap%
\pgfsetroundjoin%
\definecolor{currentfill}{rgb}{1.000000,0.000000,0.000000}%
\pgfsetfillcolor{currentfill}%
\pgfsetfillopacity{0.984374}%
\pgfsetlinewidth{1.003750pt}%
\definecolor{currentstroke}{rgb}{1.000000,0.000000,0.000000}%
\pgfsetstrokecolor{currentstroke}%
\pgfsetstrokeopacity{0.984374}%
\pgfsetdash{}{0pt}%
\pgfpathmoveto{\pgfqpoint{3.773310in}{1.752318in}}%
\pgfpathcurveto{\pgfqpoint{3.786332in}{1.752318in}}{\pgfqpoint{3.798823in}{1.757492in}}{\pgfqpoint{3.808032in}{1.766700in}}%
\pgfpathcurveto{\pgfqpoint{3.817240in}{1.775909in}}{\pgfqpoint{3.822414in}{1.788400in}}{\pgfqpoint{3.822414in}{1.801422in}}%
\pgfpathcurveto{\pgfqpoint{3.822414in}{1.814445in}}{\pgfqpoint{3.817240in}{1.826936in}}{\pgfqpoint{3.808032in}{1.836145in}}%
\pgfpathcurveto{\pgfqpoint{3.798823in}{1.845353in}}{\pgfqpoint{3.786332in}{1.850527in}}{\pgfqpoint{3.773310in}{1.850527in}}%
\pgfpathcurveto{\pgfqpoint{3.760287in}{1.850527in}}{\pgfqpoint{3.747796in}{1.845353in}}{\pgfqpoint{3.738587in}{1.836145in}}%
\pgfpathcurveto{\pgfqpoint{3.729379in}{1.826936in}}{\pgfqpoint{3.724205in}{1.814445in}}{\pgfqpoint{3.724205in}{1.801422in}}%
\pgfpathcurveto{\pgfqpoint{3.724205in}{1.788400in}}{\pgfqpoint{3.729379in}{1.775909in}}{\pgfqpoint{3.738587in}{1.766700in}}%
\pgfpathcurveto{\pgfqpoint{3.747796in}{1.757492in}}{\pgfqpoint{3.760287in}{1.752318in}}{\pgfqpoint{3.773310in}{1.752318in}}%
\pgfpathlineto{\pgfqpoint{3.773310in}{1.752318in}}%
\pgfpathclose%
\pgfusepath{stroke,fill}%
\end{pgfscope}%
\begin{pgfscope}%
\pgfpathrectangle{\pgfqpoint{1.150000in}{0.150000in}}{\pgfqpoint{5.700000in}{5.700000in}}%
\pgfusepath{clip}%
\pgfsetbuttcap%
\pgfsetroundjoin%
\definecolor{currentfill}{rgb}{1.000000,0.000000,0.000000}%
\pgfsetfillcolor{currentfill}%
\pgfsetlinewidth{1.003750pt}%
\definecolor{currentstroke}{rgb}{1.000000,0.000000,0.000000}%
\pgfsetstrokecolor{currentstroke}%
\pgfsetdash{}{0pt}%
\pgfpathmoveto{\pgfqpoint{3.218836in}{1.932166in}}%
\pgfpathcurveto{\pgfqpoint{3.231859in}{1.932166in}}{\pgfqpoint{3.244350in}{1.937340in}}{\pgfqpoint{3.253559in}{1.946548in}}%
\pgfpathcurveto{\pgfqpoint{3.262767in}{1.955757in}}{\pgfqpoint{3.267941in}{1.968248in}}{\pgfqpoint{3.267941in}{1.981271in}}%
\pgfpathcurveto{\pgfqpoint{3.267941in}{1.994293in}}{\pgfqpoint{3.262767in}{2.006784in}}{\pgfqpoint{3.253559in}{2.015993in}}%
\pgfpathcurveto{\pgfqpoint{3.244350in}{2.025201in}}{\pgfqpoint{3.231859in}{2.030375in}}{\pgfqpoint{3.218836in}{2.030375in}}%
\pgfpathcurveto{\pgfqpoint{3.205814in}{2.030375in}}{\pgfqpoint{3.193323in}{2.025201in}}{\pgfqpoint{3.184114in}{2.015993in}}%
\pgfpathcurveto{\pgfqpoint{3.174906in}{2.006784in}}{\pgfqpoint{3.169732in}{1.994293in}}{\pgfqpoint{3.169732in}{1.981271in}}%
\pgfpathcurveto{\pgfqpoint{3.169732in}{1.968248in}}{\pgfqpoint{3.174906in}{1.955757in}}{\pgfqpoint{3.184114in}{1.946548in}}%
\pgfpathcurveto{\pgfqpoint{3.193323in}{1.937340in}}{\pgfqpoint{3.205814in}{1.932166in}}{\pgfqpoint{3.218836in}{1.932166in}}%
\pgfpathlineto{\pgfqpoint{3.218836in}{1.932166in}}%
\pgfpathclose%
\pgfusepath{stroke,fill}%
\end{pgfscope}%
\begin{pgfscope}%
\pgfpathrectangle{\pgfqpoint{1.150000in}{0.150000in}}{\pgfqpoint{5.700000in}{5.700000in}}%
\pgfusepath{clip}%
\pgfsetbuttcap%
\pgfsetroundjoin%
\definecolor{currentfill}{rgb}{0.204903,0.375746,0.553533}%
\pgfsetfillcolor{currentfill}%
\pgfsetfillopacity{0.700000}%
\pgfsetlinewidth{0.000000pt}%
\definecolor{currentstroke}{rgb}{0.000000,0.000000,0.000000}%
\pgfsetstrokecolor{currentstroke}%
\pgfsetdash{}{0pt}%
\pgfpathmoveto{\pgfqpoint{3.528976in}{3.724924in}}%
\pgfpathlineto{\pgfqpoint{3.542007in}{3.709946in}}%
\pgfpathlineto{\pgfqpoint{3.555037in}{3.695160in}}%
\pgfpathlineto{\pgfqpoint{3.568065in}{3.680563in}}%
\pgfpathlineto{\pgfqpoint{3.581091in}{3.666153in}}%
\pgfpathlineto{\pgfqpoint{3.573450in}{3.652645in}}%
\pgfpathlineto{\pgfqpoint{3.565803in}{3.639280in}}%
\pgfpathlineto{\pgfqpoint{3.558152in}{3.626057in}}%
\pgfpathlineto{\pgfqpoint{3.550495in}{3.612974in}}%
\pgfpathlineto{\pgfqpoint{3.537459in}{3.627188in}}%
\pgfpathlineto{\pgfqpoint{3.524421in}{3.641590in}}%
\pgfpathlineto{\pgfqpoint{3.511381in}{3.656181in}}%
\pgfpathlineto{\pgfqpoint{3.498340in}{3.670963in}}%
\pgfpathlineto{\pgfqpoint{3.506007in}{3.684236in}}%
\pgfpathlineto{\pgfqpoint{3.513668in}{3.697651in}}%
\pgfpathlineto{\pgfqpoint{3.521324in}{3.711213in}}%
\pgfpathlineto{\pgfqpoint{3.528976in}{3.724924in}}%
\pgfpathclose%
\pgfusepath{fill}%
\end{pgfscope}%
\begin{pgfscope}%
\pgfpathrectangle{\pgfqpoint{1.150000in}{0.150000in}}{\pgfqpoint{5.700000in}{5.700000in}}%
\pgfusepath{clip}%
\pgfsetbuttcap%
\pgfsetroundjoin%
\definecolor{currentfill}{rgb}{0.216210,0.351535,0.550627}%
\pgfsetfillcolor{currentfill}%
\pgfsetfillopacity{0.700000}%
\pgfsetlinewidth{0.000000pt}%
\definecolor{currentstroke}{rgb}{0.000000,0.000000,0.000000}%
\pgfsetstrokecolor{currentstroke}%
\pgfsetdash{}{0pt}%
\pgfpathmoveto{\pgfqpoint{3.581091in}{3.666153in}}%
\pgfpathlineto{\pgfqpoint{3.594117in}{3.651930in}}%
\pgfpathlineto{\pgfqpoint{3.607141in}{3.637891in}}%
\pgfpathlineto{\pgfqpoint{3.620163in}{3.624034in}}%
\pgfpathlineto{\pgfqpoint{3.633185in}{3.610359in}}%
\pgfpathlineto{\pgfqpoint{3.625553in}{3.597052in}}%
\pgfpathlineto{\pgfqpoint{3.617917in}{3.583884in}}%
\pgfpathlineto{\pgfqpoint{3.610275in}{3.570854in}}%
\pgfpathlineto{\pgfqpoint{3.602629in}{3.557958in}}%
\pgfpathlineto{\pgfqpoint{3.589597in}{3.571439in}}%
\pgfpathlineto{\pgfqpoint{3.576564in}{3.585101in}}%
\pgfpathlineto{\pgfqpoint{3.563530in}{3.598945in}}%
\pgfpathlineto{\pgfqpoint{3.550495in}{3.612974in}}%
\pgfpathlineto{\pgfqpoint{3.558152in}{3.626057in}}%
\pgfpathlineto{\pgfqpoint{3.565803in}{3.639280in}}%
\pgfpathlineto{\pgfqpoint{3.573450in}{3.652645in}}%
\pgfpathlineto{\pgfqpoint{3.581091in}{3.666153in}}%
\pgfpathclose%
\pgfusepath{fill}%
\end{pgfscope}%
\begin{pgfscope}%
\pgfpathrectangle{\pgfqpoint{1.150000in}{0.150000in}}{\pgfqpoint{5.700000in}{5.700000in}}%
\pgfusepath{clip}%
\pgfsetbuttcap%
\pgfsetroundjoin%
\definecolor{currentfill}{rgb}{0.225863,0.330805,0.547314}%
\pgfsetfillcolor{currentfill}%
\pgfsetfillopacity{0.700000}%
\pgfsetlinewidth{0.000000pt}%
\definecolor{currentstroke}{rgb}{0.000000,0.000000,0.000000}%
\pgfsetstrokecolor{currentstroke}%
\pgfsetdash{}{0pt}%
\pgfpathmoveto{\pgfqpoint{3.633185in}{3.610359in}}%
\pgfpathlineto{\pgfqpoint{3.646207in}{3.596862in}}%
\pgfpathlineto{\pgfqpoint{3.659227in}{3.583544in}}%
\pgfpathlineto{\pgfqpoint{3.672247in}{3.570402in}}%
\pgfpathlineto{\pgfqpoint{3.685267in}{3.557434in}}%
\pgfpathlineto{\pgfqpoint{3.677645in}{3.544328in}}%
\pgfpathlineto{\pgfqpoint{3.670018in}{3.531357in}}%
\pgfpathlineto{\pgfqpoint{3.662386in}{3.518519in}}%
\pgfpathlineto{\pgfqpoint{3.654750in}{3.505810in}}%
\pgfpathlineto{\pgfqpoint{3.641720in}{3.518584in}}%
\pgfpathlineto{\pgfqpoint{3.628690in}{3.531532in}}%
\pgfpathlineto{\pgfqpoint{3.615660in}{3.544656in}}%
\pgfpathlineto{\pgfqpoint{3.602629in}{3.557958in}}%
\pgfpathlineto{\pgfqpoint{3.610275in}{3.570854in}}%
\pgfpathlineto{\pgfqpoint{3.617917in}{3.583884in}}%
\pgfpathlineto{\pgfqpoint{3.625553in}{3.597052in}}%
\pgfpathlineto{\pgfqpoint{3.633185in}{3.610359in}}%
\pgfpathclose%
\pgfusepath{fill}%
\end{pgfscope}%
\begin{pgfscope}%
\pgfpathrectangle{\pgfqpoint{1.150000in}{0.150000in}}{\pgfqpoint{5.700000in}{5.700000in}}%
\pgfusepath{clip}%
\pgfsetbuttcap%
\pgfsetroundjoin%
\definecolor{currentfill}{rgb}{0.235526,0.309527,0.542944}%
\pgfsetfillcolor{currentfill}%
\pgfsetfillopacity{0.700000}%
\pgfsetlinewidth{0.000000pt}%
\definecolor{currentstroke}{rgb}{0.000000,0.000000,0.000000}%
\pgfsetstrokecolor{currentstroke}%
\pgfsetdash{}{0pt}%
\pgfpathmoveto{\pgfqpoint{3.685267in}{3.557434in}}%
\pgfpathlineto{\pgfqpoint{3.698286in}{3.544639in}}%
\pgfpathlineto{\pgfqpoint{3.711306in}{3.532016in}}%
\pgfpathlineto{\pgfqpoint{3.724325in}{3.519563in}}%
\pgfpathlineto{\pgfqpoint{3.737345in}{3.507279in}}%
\pgfpathlineto{\pgfqpoint{3.729732in}{3.494374in}}%
\pgfpathlineto{\pgfqpoint{3.722115in}{3.481600in}}%
\pgfpathlineto{\pgfqpoint{3.714494in}{3.468953in}}%
\pgfpathlineto{\pgfqpoint{3.706867in}{3.456432in}}%
\pgfpathlineto{\pgfqpoint{3.693838in}{3.468522in}}%
\pgfpathlineto{\pgfqpoint{3.680808in}{3.480781in}}%
\pgfpathlineto{\pgfqpoint{3.667779in}{3.493210in}}%
\pgfpathlineto{\pgfqpoint{3.654750in}{3.505810in}}%
\pgfpathlineto{\pgfqpoint{3.662386in}{3.518519in}}%
\pgfpathlineto{\pgfqpoint{3.670018in}{3.531357in}}%
\pgfpathlineto{\pgfqpoint{3.677645in}{3.544328in}}%
\pgfpathlineto{\pgfqpoint{3.685267in}{3.557434in}}%
\pgfpathclose%
\pgfusepath{fill}%
\end{pgfscope}%
\begin{pgfscope}%
\pgfpathrectangle{\pgfqpoint{1.150000in}{0.150000in}}{\pgfqpoint{5.700000in}{5.700000in}}%
\pgfusepath{clip}%
\pgfsetbuttcap%
\pgfsetroundjoin%
\definecolor{currentfill}{rgb}{0.244972,0.287675,0.537260}%
\pgfsetfillcolor{currentfill}%
\pgfsetfillopacity{0.700000}%
\pgfsetlinewidth{0.000000pt}%
\definecolor{currentstroke}{rgb}{0.000000,0.000000,0.000000}%
\pgfsetstrokecolor{currentstroke}%
\pgfsetdash{}{0pt}%
\pgfpathmoveto{\pgfqpoint{3.737345in}{3.507279in}}%
\pgfpathlineto{\pgfqpoint{3.750365in}{3.495163in}}%
\pgfpathlineto{\pgfqpoint{3.763385in}{3.483212in}}%
\pgfpathlineto{\pgfqpoint{3.776406in}{3.471425in}}%
\pgfpathlineto{\pgfqpoint{3.789428in}{3.459802in}}%
\pgfpathlineto{\pgfqpoint{3.781825in}{3.447098in}}%
\pgfpathlineto{\pgfqpoint{3.774218in}{3.434519in}}%
\pgfpathlineto{\pgfqpoint{3.766606in}{3.422064in}}%
\pgfpathlineto{\pgfqpoint{3.758990in}{3.409730in}}%
\pgfpathlineto{\pgfqpoint{3.745958in}{3.421159in}}%
\pgfpathlineto{\pgfqpoint{3.732927in}{3.432752in}}%
\pgfpathlineto{\pgfqpoint{3.719897in}{3.444509in}}%
\pgfpathlineto{\pgfqpoint{3.706867in}{3.456432in}}%
\pgfpathlineto{\pgfqpoint{3.714494in}{3.468953in}}%
\pgfpathlineto{\pgfqpoint{3.722115in}{3.481600in}}%
\pgfpathlineto{\pgfqpoint{3.729732in}{3.494374in}}%
\pgfpathlineto{\pgfqpoint{3.737345in}{3.507279in}}%
\pgfpathclose%
\pgfusepath{fill}%
\end{pgfscope}%
\begin{pgfscope}%
\pgfpathrectangle{\pgfqpoint{1.150000in}{0.150000in}}{\pgfqpoint{5.700000in}{5.700000in}}%
\pgfusepath{clip}%
\pgfsetbuttcap%
\pgfsetroundjoin%
\definecolor{currentfill}{rgb}{0.252194,0.269783,0.531579}%
\pgfsetfillcolor{currentfill}%
\pgfsetfillopacity{0.700000}%
\pgfsetlinewidth{0.000000pt}%
\definecolor{currentstroke}{rgb}{0.000000,0.000000,0.000000}%
\pgfsetstrokecolor{currentstroke}%
\pgfsetdash{}{0pt}%
\pgfpathmoveto{\pgfqpoint{3.789428in}{3.459802in}}%
\pgfpathlineto{\pgfqpoint{3.802450in}{3.448341in}}%
\pgfpathlineto{\pgfqpoint{3.815474in}{3.437041in}}%
\pgfpathlineto{\pgfqpoint{3.828498in}{3.425899in}}%
\pgfpathlineto{\pgfqpoint{3.841524in}{3.414916in}}%
\pgfpathlineto{\pgfqpoint{3.833931in}{3.402412in}}%
\pgfpathlineto{\pgfqpoint{3.826334in}{3.390029in}}%
\pgfpathlineto{\pgfqpoint{3.818732in}{3.377765in}}%
\pgfpathlineto{\pgfqpoint{3.811126in}{3.365618in}}%
\pgfpathlineto{\pgfqpoint{3.798090in}{3.376408in}}%
\pgfpathlineto{\pgfqpoint{3.785056in}{3.387355in}}%
\pgfpathlineto{\pgfqpoint{3.772022in}{3.398462in}}%
\pgfpathlineto{\pgfqpoint{3.758990in}{3.409730in}}%
\pgfpathlineto{\pgfqpoint{3.766606in}{3.422064in}}%
\pgfpathlineto{\pgfqpoint{3.774218in}{3.434519in}}%
\pgfpathlineto{\pgfqpoint{3.781825in}{3.447098in}}%
\pgfpathlineto{\pgfqpoint{3.789428in}{3.459802in}}%
\pgfpathclose%
\pgfusepath{fill}%
\end{pgfscope}%
\begin{pgfscope}%
\pgfpathrectangle{\pgfqpoint{1.150000in}{0.150000in}}{\pgfqpoint{5.700000in}{5.700000in}}%
\pgfusepath{clip}%
\pgfsetbuttcap%
\pgfsetroundjoin%
\definecolor{currentfill}{rgb}{0.212395,0.359683,0.551710}%
\pgfsetfillcolor{currentfill}%
\pgfsetfillopacity{0.700000}%
\pgfsetlinewidth{0.000000pt}%
\definecolor{currentstroke}{rgb}{0.000000,0.000000,0.000000}%
\pgfsetstrokecolor{currentstroke}%
\pgfsetdash{}{0pt}%
\pgfpathmoveto{\pgfqpoint{3.498340in}{3.670963in}}%
\pgfpathlineto{\pgfqpoint{3.511381in}{3.656181in}}%
\pgfpathlineto{\pgfqpoint{3.524421in}{3.641590in}}%
\pgfpathlineto{\pgfqpoint{3.537459in}{3.627188in}}%
\pgfpathlineto{\pgfqpoint{3.550495in}{3.612974in}}%
\pgfpathlineto{\pgfqpoint{3.542833in}{3.600027in}}%
\pgfpathlineto{\pgfqpoint{3.535166in}{3.587214in}}%
\pgfpathlineto{\pgfqpoint{3.527494in}{3.574534in}}%
\pgfpathlineto{\pgfqpoint{3.519817in}{3.561984in}}%
\pgfpathlineto{\pgfqpoint{3.506770in}{3.576022in}}%
\pgfpathlineto{\pgfqpoint{3.493722in}{3.590247in}}%
\pgfpathlineto{\pgfqpoint{3.480672in}{3.604661in}}%
\pgfpathlineto{\pgfqpoint{3.467620in}{3.619266in}}%
\pgfpathlineto{\pgfqpoint{3.475308in}{3.631986in}}%
\pgfpathlineto{\pgfqpoint{3.482990in}{3.644841in}}%
\pgfpathlineto{\pgfqpoint{3.490668in}{3.657833in}}%
\pgfpathlineto{\pgfqpoint{3.498340in}{3.670963in}}%
\pgfpathclose%
\pgfusepath{fill}%
\end{pgfscope}%
\begin{pgfscope}%
\pgfpathrectangle{\pgfqpoint{1.150000in}{0.150000in}}{\pgfqpoint{5.700000in}{5.700000in}}%
\pgfusepath{clip}%
\pgfsetbuttcap%
\pgfsetroundjoin%
\definecolor{currentfill}{rgb}{0.221989,0.339161,0.548752}%
\pgfsetfillcolor{currentfill}%
\pgfsetfillopacity{0.700000}%
\pgfsetlinewidth{0.000000pt}%
\definecolor{currentstroke}{rgb}{0.000000,0.000000,0.000000}%
\pgfsetstrokecolor{currentstroke}%
\pgfsetdash{}{0pt}%
\pgfpathmoveto{\pgfqpoint{3.550495in}{3.612974in}}%
\pgfpathlineto{\pgfqpoint{3.563530in}{3.598945in}}%
\pgfpathlineto{\pgfqpoint{3.576564in}{3.585101in}}%
\pgfpathlineto{\pgfqpoint{3.589597in}{3.571439in}}%
\pgfpathlineto{\pgfqpoint{3.602629in}{3.557958in}}%
\pgfpathlineto{\pgfqpoint{3.594977in}{3.545194in}}%
\pgfpathlineto{\pgfqpoint{3.587320in}{3.532561in}}%
\pgfpathlineto{\pgfqpoint{3.579659in}{3.520055in}}%
\pgfpathlineto{\pgfqpoint{3.571992in}{3.507675in}}%
\pgfpathlineto{\pgfqpoint{3.558950in}{3.520980in}}%
\pgfpathlineto{\pgfqpoint{3.545907in}{3.534465in}}%
\pgfpathlineto{\pgfqpoint{3.532862in}{3.548133in}}%
\pgfpathlineto{\pgfqpoint{3.519817in}{3.561984in}}%
\pgfpathlineto{\pgfqpoint{3.527494in}{3.574534in}}%
\pgfpathlineto{\pgfqpoint{3.535166in}{3.587214in}}%
\pgfpathlineto{\pgfqpoint{3.542833in}{3.600027in}}%
\pgfpathlineto{\pgfqpoint{3.550495in}{3.612974in}}%
\pgfpathclose%
\pgfusepath{fill}%
\end{pgfscope}%
\begin{pgfscope}%
\pgfpathrectangle{\pgfqpoint{1.150000in}{0.150000in}}{\pgfqpoint{5.700000in}{5.700000in}}%
\pgfusepath{clip}%
\pgfsetbuttcap%
\pgfsetroundjoin%
\definecolor{currentfill}{rgb}{0.258965,0.251537,0.524736}%
\pgfsetfillcolor{currentfill}%
\pgfsetfillopacity{0.700000}%
\pgfsetlinewidth{0.000000pt}%
\definecolor{currentstroke}{rgb}{0.000000,0.000000,0.000000}%
\pgfsetstrokecolor{currentstroke}%
\pgfsetdash{}{0pt}%
\pgfpathmoveto{\pgfqpoint{3.841524in}{3.414916in}}%
\pgfpathlineto{\pgfqpoint{3.854551in}{3.404090in}}%
\pgfpathlineto{\pgfqpoint{3.867580in}{3.393419in}}%
\pgfpathlineto{\pgfqpoint{3.880610in}{3.382903in}}%
\pgfpathlineto{\pgfqpoint{3.893641in}{3.372540in}}%
\pgfpathlineto{\pgfqpoint{3.886059in}{3.360235in}}%
\pgfpathlineto{\pgfqpoint{3.878471in}{3.348048in}}%
\pgfpathlineto{\pgfqpoint{3.870880in}{3.335975in}}%
\pgfpathlineto{\pgfqpoint{3.863283in}{3.324015in}}%
\pgfpathlineto{\pgfqpoint{3.850242in}{3.334185in}}%
\pgfpathlineto{\pgfqpoint{3.837202in}{3.344508in}}%
\pgfpathlineto{\pgfqpoint{3.824163in}{3.354985in}}%
\pgfpathlineto{\pgfqpoint{3.811126in}{3.365618in}}%
\pgfpathlineto{\pgfqpoint{3.818732in}{3.377765in}}%
\pgfpathlineto{\pgfqpoint{3.826334in}{3.390029in}}%
\pgfpathlineto{\pgfqpoint{3.833931in}{3.402412in}}%
\pgfpathlineto{\pgfqpoint{3.841524in}{3.414916in}}%
\pgfpathclose%
\pgfusepath{fill}%
\end{pgfscope}%
\begin{pgfscope}%
\pgfpathrectangle{\pgfqpoint{1.150000in}{0.150000in}}{\pgfqpoint{5.700000in}{5.700000in}}%
\pgfusepath{clip}%
\pgfsetbuttcap%
\pgfsetroundjoin%
\definecolor{currentfill}{rgb}{0.233603,0.313828,0.543914}%
\pgfsetfillcolor{currentfill}%
\pgfsetfillopacity{0.700000}%
\pgfsetlinewidth{0.000000pt}%
\definecolor{currentstroke}{rgb}{0.000000,0.000000,0.000000}%
\pgfsetstrokecolor{currentstroke}%
\pgfsetdash{}{0pt}%
\pgfpathmoveto{\pgfqpoint{3.602629in}{3.557958in}}%
\pgfpathlineto{\pgfqpoint{3.615660in}{3.544656in}}%
\pgfpathlineto{\pgfqpoint{3.628690in}{3.531532in}}%
\pgfpathlineto{\pgfqpoint{3.641720in}{3.518584in}}%
\pgfpathlineto{\pgfqpoint{3.654750in}{3.505810in}}%
\pgfpathlineto{\pgfqpoint{3.647108in}{3.493230in}}%
\pgfpathlineto{\pgfqpoint{3.639462in}{3.480775in}}%
\pgfpathlineto{\pgfqpoint{3.631811in}{3.468443in}}%
\pgfpathlineto{\pgfqpoint{3.624154in}{3.456233in}}%
\pgfpathlineto{\pgfqpoint{3.611114in}{3.468830in}}%
\pgfpathlineto{\pgfqpoint{3.598074in}{3.481602in}}%
\pgfpathlineto{\pgfqpoint{3.585033in}{3.494550in}}%
\pgfpathlineto{\pgfqpoint{3.571992in}{3.507675in}}%
\pgfpathlineto{\pgfqpoint{3.579659in}{3.520055in}}%
\pgfpathlineto{\pgfqpoint{3.587320in}{3.532561in}}%
\pgfpathlineto{\pgfqpoint{3.594977in}{3.545194in}}%
\pgfpathlineto{\pgfqpoint{3.602629in}{3.557958in}}%
\pgfpathclose%
\pgfusepath{fill}%
\end{pgfscope}%
\begin{pgfscope}%
\pgfpathrectangle{\pgfqpoint{1.150000in}{0.150000in}}{\pgfqpoint{5.700000in}{5.700000in}}%
\pgfusepath{clip}%
\pgfsetbuttcap%
\pgfsetroundjoin%
\definecolor{currentfill}{rgb}{0.241237,0.296485,0.539709}%
\pgfsetfillcolor{currentfill}%
\pgfsetfillopacity{0.700000}%
\pgfsetlinewidth{0.000000pt}%
\definecolor{currentstroke}{rgb}{0.000000,0.000000,0.000000}%
\pgfsetstrokecolor{currentstroke}%
\pgfsetdash{}{0pt}%
\pgfpathmoveto{\pgfqpoint{3.654750in}{3.505810in}}%
\pgfpathlineto{\pgfqpoint{3.667779in}{3.493210in}}%
\pgfpathlineto{\pgfqpoint{3.680808in}{3.480781in}}%
\pgfpathlineto{\pgfqpoint{3.693838in}{3.468522in}}%
\pgfpathlineto{\pgfqpoint{3.706867in}{3.456432in}}%
\pgfpathlineto{\pgfqpoint{3.699236in}{3.444034in}}%
\pgfpathlineto{\pgfqpoint{3.691600in}{3.431758in}}%
\pgfpathlineto{\pgfqpoint{3.683959in}{3.419600in}}%
\pgfpathlineto{\pgfqpoint{3.676313in}{3.407559in}}%
\pgfpathlineto{\pgfqpoint{3.663273in}{3.419473in}}%
\pgfpathlineto{\pgfqpoint{3.650234in}{3.431556in}}%
\pgfpathlineto{\pgfqpoint{3.637194in}{3.443809in}}%
\pgfpathlineto{\pgfqpoint{3.624154in}{3.456233in}}%
\pgfpathlineto{\pgfqpoint{3.631811in}{3.468443in}}%
\pgfpathlineto{\pgfqpoint{3.639462in}{3.480775in}}%
\pgfpathlineto{\pgfqpoint{3.647108in}{3.493230in}}%
\pgfpathlineto{\pgfqpoint{3.654750in}{3.505810in}}%
\pgfpathclose%
\pgfusepath{fill}%
\end{pgfscope}%
\begin{pgfscope}%
\pgfpathrectangle{\pgfqpoint{1.150000in}{0.150000in}}{\pgfqpoint{5.700000in}{5.700000in}}%
\pgfusepath{clip}%
\pgfsetbuttcap%
\pgfsetroundjoin%
\definecolor{currentfill}{rgb}{0.263663,0.237631,0.518762}%
\pgfsetfillcolor{currentfill}%
\pgfsetfillopacity{0.700000}%
\pgfsetlinewidth{0.000000pt}%
\definecolor{currentstroke}{rgb}{0.000000,0.000000,0.000000}%
\pgfsetstrokecolor{currentstroke}%
\pgfsetdash{}{0pt}%
\pgfpathmoveto{\pgfqpoint{3.893641in}{3.372540in}}%
\pgfpathlineto{\pgfqpoint{3.906675in}{3.362329in}}%
\pgfpathlineto{\pgfqpoint{3.919710in}{3.352269in}}%
\pgfpathlineto{\pgfqpoint{3.932748in}{3.342359in}}%
\pgfpathlineto{\pgfqpoint{3.945788in}{3.332598in}}%
\pgfpathlineto{\pgfqpoint{3.938215in}{3.320493in}}%
\pgfpathlineto{\pgfqpoint{3.930637in}{3.308501in}}%
\pgfpathlineto{\pgfqpoint{3.923056in}{3.296620in}}%
\pgfpathlineto{\pgfqpoint{3.915470in}{3.284846in}}%
\pgfpathlineto{\pgfqpoint{3.902420in}{3.294414in}}%
\pgfpathlineto{\pgfqpoint{3.889373in}{3.304131in}}%
\pgfpathlineto{\pgfqpoint{3.876327in}{3.313997in}}%
\pgfpathlineto{\pgfqpoint{3.863283in}{3.324015in}}%
\pgfpathlineto{\pgfqpoint{3.870880in}{3.335975in}}%
\pgfpathlineto{\pgfqpoint{3.878471in}{3.348048in}}%
\pgfpathlineto{\pgfqpoint{3.886059in}{3.360235in}}%
\pgfpathlineto{\pgfqpoint{3.893641in}{3.372540in}}%
\pgfpathclose%
\pgfusepath{fill}%
\end{pgfscope}%
\begin{pgfscope}%
\pgfpathrectangle{\pgfqpoint{1.150000in}{0.150000in}}{\pgfqpoint{5.700000in}{5.700000in}}%
\pgfusepath{clip}%
\pgfsetbuttcap%
\pgfsetroundjoin%
\definecolor{currentfill}{rgb}{0.250425,0.274290,0.533103}%
\pgfsetfillcolor{currentfill}%
\pgfsetfillopacity{0.700000}%
\pgfsetlinewidth{0.000000pt}%
\definecolor{currentstroke}{rgb}{0.000000,0.000000,0.000000}%
\pgfsetstrokecolor{currentstroke}%
\pgfsetdash{}{0pt}%
\pgfpathmoveto{\pgfqpoint{3.706867in}{3.456432in}}%
\pgfpathlineto{\pgfqpoint{3.719897in}{3.444509in}}%
\pgfpathlineto{\pgfqpoint{3.732927in}{3.432752in}}%
\pgfpathlineto{\pgfqpoint{3.745958in}{3.421159in}}%
\pgfpathlineto{\pgfqpoint{3.758990in}{3.409730in}}%
\pgfpathlineto{\pgfqpoint{3.751369in}{3.397515in}}%
\pgfpathlineto{\pgfqpoint{3.743743in}{3.385416in}}%
\pgfpathlineto{\pgfqpoint{3.736113in}{3.373432in}}%
\pgfpathlineto{\pgfqpoint{3.728477in}{3.361561in}}%
\pgfpathlineto{\pgfqpoint{3.715435in}{3.372814in}}%
\pgfpathlineto{\pgfqpoint{3.702394in}{3.384231in}}%
\pgfpathlineto{\pgfqpoint{3.689353in}{3.395812in}}%
\pgfpathlineto{\pgfqpoint{3.676313in}{3.407559in}}%
\pgfpathlineto{\pgfqpoint{3.683959in}{3.419600in}}%
\pgfpathlineto{\pgfqpoint{3.691600in}{3.431758in}}%
\pgfpathlineto{\pgfqpoint{3.699236in}{3.444034in}}%
\pgfpathlineto{\pgfqpoint{3.706867in}{3.456432in}}%
\pgfpathclose%
\pgfusepath{fill}%
\end{pgfscope}%
\begin{pgfscope}%
\pgfpathrectangle{\pgfqpoint{1.150000in}{0.150000in}}{\pgfqpoint{5.700000in}{5.700000in}}%
\pgfusepath{clip}%
\pgfsetbuttcap%
\pgfsetroundjoin%
\definecolor{currentfill}{rgb}{0.257322,0.256130,0.526563}%
\pgfsetfillcolor{currentfill}%
\pgfsetfillopacity{0.700000}%
\pgfsetlinewidth{0.000000pt}%
\definecolor{currentstroke}{rgb}{0.000000,0.000000,0.000000}%
\pgfsetstrokecolor{currentstroke}%
\pgfsetdash{}{0pt}%
\pgfpathmoveto{\pgfqpoint{3.758990in}{3.409730in}}%
\pgfpathlineto{\pgfqpoint{3.772022in}{3.398462in}}%
\pgfpathlineto{\pgfqpoint{3.785056in}{3.387355in}}%
\pgfpathlineto{\pgfqpoint{3.798090in}{3.376408in}}%
\pgfpathlineto{\pgfqpoint{3.811126in}{3.365618in}}%
\pgfpathlineto{\pgfqpoint{3.803515in}{3.353585in}}%
\pgfpathlineto{\pgfqpoint{3.795900in}{3.341665in}}%
\pgfpathlineto{\pgfqpoint{3.788279in}{3.329854in}}%
\pgfpathlineto{\pgfqpoint{3.780655in}{3.318152in}}%
\pgfpathlineto{\pgfqpoint{3.767609in}{3.328766in}}%
\pgfpathlineto{\pgfqpoint{3.754564in}{3.339538in}}%
\pgfpathlineto{\pgfqpoint{3.741520in}{3.350469in}}%
\pgfpathlineto{\pgfqpoint{3.728477in}{3.361561in}}%
\pgfpathlineto{\pgfqpoint{3.736113in}{3.373432in}}%
\pgfpathlineto{\pgfqpoint{3.743743in}{3.385416in}}%
\pgfpathlineto{\pgfqpoint{3.751369in}{3.397515in}}%
\pgfpathlineto{\pgfqpoint{3.758990in}{3.409730in}}%
\pgfpathclose%
\pgfusepath{fill}%
\end{pgfscope}%
\begin{pgfscope}%
\pgfpathrectangle{\pgfqpoint{1.150000in}{0.150000in}}{\pgfqpoint{5.700000in}{5.700000in}}%
\pgfusepath{clip}%
\pgfsetbuttcap%
\pgfsetroundjoin%
\definecolor{currentfill}{rgb}{0.269308,0.218818,0.509577}%
\pgfsetfillcolor{currentfill}%
\pgfsetfillopacity{0.700000}%
\pgfsetlinewidth{0.000000pt}%
\definecolor{currentstroke}{rgb}{0.000000,0.000000,0.000000}%
\pgfsetstrokecolor{currentstroke}%
\pgfsetdash{}{0pt}%
\pgfpathmoveto{\pgfqpoint{3.945788in}{3.332598in}}%
\pgfpathlineto{\pgfqpoint{3.958829in}{3.322984in}}%
\pgfpathlineto{\pgfqpoint{3.971874in}{3.313517in}}%
\pgfpathlineto{\pgfqpoint{3.984920in}{3.304196in}}%
\pgfpathlineto{\pgfqpoint{3.997970in}{3.295019in}}%
\pgfpathlineto{\pgfqpoint{3.990407in}{3.283114in}}%
\pgfpathlineto{\pgfqpoint{3.982840in}{3.271318in}}%
\pgfpathlineto{\pgfqpoint{3.975268in}{3.259627in}}%
\pgfpathlineto{\pgfqpoint{3.967693in}{3.248040in}}%
\pgfpathlineto{\pgfqpoint{3.954633in}{3.257024in}}%
\pgfpathlineto{\pgfqpoint{3.941576in}{3.266152in}}%
\pgfpathlineto{\pgfqpoint{3.928522in}{3.275426in}}%
\pgfpathlineto{\pgfqpoint{3.915470in}{3.284846in}}%
\pgfpathlineto{\pgfqpoint{3.923056in}{3.296620in}}%
\pgfpathlineto{\pgfqpoint{3.930637in}{3.308501in}}%
\pgfpathlineto{\pgfqpoint{3.938215in}{3.320493in}}%
\pgfpathlineto{\pgfqpoint{3.945788in}{3.332598in}}%
\pgfpathclose%
\pgfusepath{fill}%
\end{pgfscope}%
\begin{pgfscope}%
\pgfpathrectangle{\pgfqpoint{1.150000in}{0.150000in}}{\pgfqpoint{5.700000in}{5.700000in}}%
\pgfusepath{clip}%
\pgfsetbuttcap%
\pgfsetroundjoin%
\definecolor{currentfill}{rgb}{0.218130,0.347432,0.550038}%
\pgfsetfillcolor{currentfill}%
\pgfsetfillopacity{0.700000}%
\pgfsetlinewidth{0.000000pt}%
\definecolor{currentstroke}{rgb}{0.000000,0.000000,0.000000}%
\pgfsetstrokecolor{currentstroke}%
\pgfsetdash{}{0pt}%
\pgfpathmoveto{\pgfqpoint{3.467620in}{3.619266in}}%
\pgfpathlineto{\pgfqpoint{3.480672in}{3.604661in}}%
\pgfpathlineto{\pgfqpoint{3.493722in}{3.590247in}}%
\pgfpathlineto{\pgfqpoint{3.506770in}{3.576022in}}%
\pgfpathlineto{\pgfqpoint{3.519817in}{3.561984in}}%
\pgfpathlineto{\pgfqpoint{3.512134in}{3.549562in}}%
\pgfpathlineto{\pgfqpoint{3.504446in}{3.537266in}}%
\pgfpathlineto{\pgfqpoint{3.496753in}{3.525094in}}%
\pgfpathlineto{\pgfqpoint{3.489054in}{3.513043in}}%
\pgfpathlineto{\pgfqpoint{3.475997in}{3.526922in}}%
\pgfpathlineto{\pgfqpoint{3.462937in}{3.540988in}}%
\pgfpathlineto{\pgfqpoint{3.449876in}{3.555244in}}%
\pgfpathlineto{\pgfqpoint{3.436813in}{3.569690in}}%
\pgfpathlineto{\pgfqpoint{3.444523in}{3.581892in}}%
\pgfpathlineto{\pgfqpoint{3.452228in}{3.594221in}}%
\pgfpathlineto{\pgfqpoint{3.459926in}{3.606679in}}%
\pgfpathlineto{\pgfqpoint{3.467620in}{3.619266in}}%
\pgfpathclose%
\pgfusepath{fill}%
\end{pgfscope}%
\begin{pgfscope}%
\pgfpathrectangle{\pgfqpoint{1.150000in}{0.150000in}}{\pgfqpoint{5.700000in}{5.700000in}}%
\pgfusepath{clip}%
\pgfsetbuttcap%
\pgfsetroundjoin%
\definecolor{currentfill}{rgb}{0.229739,0.322361,0.545706}%
\pgfsetfillcolor{currentfill}%
\pgfsetfillopacity{0.700000}%
\pgfsetlinewidth{0.000000pt}%
\definecolor{currentstroke}{rgb}{0.000000,0.000000,0.000000}%
\pgfsetstrokecolor{currentstroke}%
\pgfsetdash{}{0pt}%
\pgfpathmoveto{\pgfqpoint{3.519817in}{3.561984in}}%
\pgfpathlineto{\pgfqpoint{3.532862in}{3.548133in}}%
\pgfpathlineto{\pgfqpoint{3.545907in}{3.534465in}}%
\pgfpathlineto{\pgfqpoint{3.558950in}{3.520980in}}%
\pgfpathlineto{\pgfqpoint{3.571992in}{3.507675in}}%
\pgfpathlineto{\pgfqpoint{3.564320in}{3.495418in}}%
\pgfpathlineto{\pgfqpoint{3.556643in}{3.483283in}}%
\pgfpathlineto{\pgfqpoint{3.548961in}{3.471267in}}%
\pgfpathlineto{\pgfqpoint{3.541273in}{3.459369in}}%
\pgfpathlineto{\pgfqpoint{3.528220in}{3.472515in}}%
\pgfpathlineto{\pgfqpoint{3.515166in}{3.485842in}}%
\pgfpathlineto{\pgfqpoint{3.502111in}{3.499350in}}%
\pgfpathlineto{\pgfqpoint{3.489054in}{3.513043in}}%
\pgfpathlineto{\pgfqpoint{3.496753in}{3.525094in}}%
\pgfpathlineto{\pgfqpoint{3.504446in}{3.537266in}}%
\pgfpathlineto{\pgfqpoint{3.512134in}{3.549562in}}%
\pgfpathlineto{\pgfqpoint{3.519817in}{3.561984in}}%
\pgfpathclose%
\pgfusepath{fill}%
\end{pgfscope}%
\begin{pgfscope}%
\pgfpathrectangle{\pgfqpoint{1.150000in}{0.150000in}}{\pgfqpoint{5.700000in}{5.700000in}}%
\pgfusepath{clip}%
\pgfsetbuttcap%
\pgfsetroundjoin%
\definecolor{currentfill}{rgb}{0.239346,0.300855,0.540844}%
\pgfsetfillcolor{currentfill}%
\pgfsetfillopacity{0.700000}%
\pgfsetlinewidth{0.000000pt}%
\definecolor{currentstroke}{rgb}{0.000000,0.000000,0.000000}%
\pgfsetstrokecolor{currentstroke}%
\pgfsetdash{}{0pt}%
\pgfpathmoveto{\pgfqpoint{3.571992in}{3.507675in}}%
\pgfpathlineto{\pgfqpoint{3.585033in}{3.494550in}}%
\pgfpathlineto{\pgfqpoint{3.598074in}{3.481602in}}%
\pgfpathlineto{\pgfqpoint{3.611114in}{3.468830in}}%
\pgfpathlineto{\pgfqpoint{3.624154in}{3.456233in}}%
\pgfpathlineto{\pgfqpoint{3.616493in}{3.444142in}}%
\pgfpathlineto{\pgfqpoint{3.608827in}{3.432167in}}%
\pgfpathlineto{\pgfqpoint{3.601155in}{3.420307in}}%
\pgfpathlineto{\pgfqpoint{3.593479in}{3.408561in}}%
\pgfpathlineto{\pgfqpoint{3.580428in}{3.421000in}}%
\pgfpathlineto{\pgfqpoint{3.567377in}{3.433613in}}%
\pgfpathlineto{\pgfqpoint{3.554325in}{3.446402in}}%
\pgfpathlineto{\pgfqpoint{3.541273in}{3.459369in}}%
\pgfpathlineto{\pgfqpoint{3.548961in}{3.471267in}}%
\pgfpathlineto{\pgfqpoint{3.556643in}{3.483283in}}%
\pgfpathlineto{\pgfqpoint{3.564320in}{3.495418in}}%
\pgfpathlineto{\pgfqpoint{3.571992in}{3.507675in}}%
\pgfpathclose%
\pgfusepath{fill}%
\end{pgfscope}%
\begin{pgfscope}%
\pgfpathrectangle{\pgfqpoint{1.150000in}{0.150000in}}{\pgfqpoint{5.700000in}{5.700000in}}%
\pgfusepath{clip}%
\pgfsetbuttcap%
\pgfsetroundjoin%
\definecolor{currentfill}{rgb}{0.263663,0.237631,0.518762}%
\pgfsetfillcolor{currentfill}%
\pgfsetfillopacity{0.700000}%
\pgfsetlinewidth{0.000000pt}%
\definecolor{currentstroke}{rgb}{0.000000,0.000000,0.000000}%
\pgfsetstrokecolor{currentstroke}%
\pgfsetdash{}{0pt}%
\pgfpathmoveto{\pgfqpoint{3.811126in}{3.365618in}}%
\pgfpathlineto{\pgfqpoint{3.824163in}{3.354985in}}%
\pgfpathlineto{\pgfqpoint{3.837202in}{3.344508in}}%
\pgfpathlineto{\pgfqpoint{3.850242in}{3.334185in}}%
\pgfpathlineto{\pgfqpoint{3.863283in}{3.324015in}}%
\pgfpathlineto{\pgfqpoint{3.855683in}{3.312165in}}%
\pgfpathlineto{\pgfqpoint{3.848078in}{3.300422in}}%
\pgfpathlineto{\pgfqpoint{3.840468in}{3.288785in}}%
\pgfpathlineto{\pgfqpoint{3.832854in}{3.277252in}}%
\pgfpathlineto{\pgfqpoint{3.819801in}{3.287246in}}%
\pgfpathlineto{\pgfqpoint{3.806751in}{3.297393in}}%
\pgfpathlineto{\pgfqpoint{3.793702in}{3.307695in}}%
\pgfpathlineto{\pgfqpoint{3.780655in}{3.318152in}}%
\pgfpathlineto{\pgfqpoint{3.788279in}{3.329854in}}%
\pgfpathlineto{\pgfqpoint{3.795900in}{3.341665in}}%
\pgfpathlineto{\pgfqpoint{3.803515in}{3.353585in}}%
\pgfpathlineto{\pgfqpoint{3.811126in}{3.365618in}}%
\pgfpathclose%
\pgfusepath{fill}%
\end{pgfscope}%
\begin{pgfscope}%
\pgfpathrectangle{\pgfqpoint{1.150000in}{0.150000in}}{\pgfqpoint{5.700000in}{5.700000in}}%
\pgfusepath{clip}%
\pgfsetbuttcap%
\pgfsetroundjoin%
\definecolor{currentfill}{rgb}{0.273006,0.204520,0.501721}%
\pgfsetfillcolor{currentfill}%
\pgfsetfillopacity{0.700000}%
\pgfsetlinewidth{0.000000pt}%
\definecolor{currentstroke}{rgb}{0.000000,0.000000,0.000000}%
\pgfsetstrokecolor{currentstroke}%
\pgfsetdash{}{0pt}%
\pgfpathmoveto{\pgfqpoint{3.997970in}{3.295019in}}%
\pgfpathlineto{\pgfqpoint{4.011022in}{3.285986in}}%
\pgfpathlineto{\pgfqpoint{4.024077in}{3.277095in}}%
\pgfpathlineto{\pgfqpoint{4.037134in}{3.268347in}}%
\pgfpathlineto{\pgfqpoint{4.050195in}{3.259738in}}%
\pgfpathlineto{\pgfqpoint{4.042642in}{3.248033in}}%
\pgfpathlineto{\pgfqpoint{4.035085in}{3.236432in}}%
\pgfpathlineto{\pgfqpoint{4.027524in}{3.224933in}}%
\pgfpathlineto{\pgfqpoint{4.019959in}{3.213532in}}%
\pgfpathlineto{\pgfqpoint{4.006888in}{3.221947in}}%
\pgfpathlineto{\pgfqpoint{3.993820in}{3.230503in}}%
\pgfpathlineto{\pgfqpoint{3.980755in}{3.239200in}}%
\pgfpathlineto{\pgfqpoint{3.967693in}{3.248040in}}%
\pgfpathlineto{\pgfqpoint{3.975268in}{3.259627in}}%
\pgfpathlineto{\pgfqpoint{3.982840in}{3.271318in}}%
\pgfpathlineto{\pgfqpoint{3.990407in}{3.283114in}}%
\pgfpathlineto{\pgfqpoint{3.997970in}{3.295019in}}%
\pgfpathclose%
\pgfusepath{fill}%
\end{pgfscope}%
\begin{pgfscope}%
\pgfpathrectangle{\pgfqpoint{1.150000in}{0.150000in}}{\pgfqpoint{5.700000in}{5.700000in}}%
\pgfusepath{clip}%
\pgfsetbuttcap%
\pgfsetroundjoin%
\definecolor{currentfill}{rgb}{0.248629,0.278775,0.534556}%
\pgfsetfillcolor{currentfill}%
\pgfsetfillopacity{0.700000}%
\pgfsetlinewidth{0.000000pt}%
\definecolor{currentstroke}{rgb}{0.000000,0.000000,0.000000}%
\pgfsetstrokecolor{currentstroke}%
\pgfsetdash{}{0pt}%
\pgfpathmoveto{\pgfqpoint{3.624154in}{3.456233in}}%
\pgfpathlineto{\pgfqpoint{3.637194in}{3.443809in}}%
\pgfpathlineto{\pgfqpoint{3.650234in}{3.431556in}}%
\pgfpathlineto{\pgfqpoint{3.663273in}{3.419473in}}%
\pgfpathlineto{\pgfqpoint{3.676313in}{3.407559in}}%
\pgfpathlineto{\pgfqpoint{3.668663in}{3.395632in}}%
\pgfpathlineto{\pgfqpoint{3.661007in}{3.383818in}}%
\pgfpathlineto{\pgfqpoint{3.653346in}{3.372115in}}%
\pgfpathlineto{\pgfqpoint{3.645680in}{3.360520in}}%
\pgfpathlineto{\pgfqpoint{3.632630in}{3.372276in}}%
\pgfpathlineto{\pgfqpoint{3.619579in}{3.384200in}}%
\pgfpathlineto{\pgfqpoint{3.606529in}{3.396295in}}%
\pgfpathlineto{\pgfqpoint{3.593479in}{3.408561in}}%
\pgfpathlineto{\pgfqpoint{3.601155in}{3.420307in}}%
\pgfpathlineto{\pgfqpoint{3.608827in}{3.432167in}}%
\pgfpathlineto{\pgfqpoint{3.616493in}{3.444142in}}%
\pgfpathlineto{\pgfqpoint{3.624154in}{3.456233in}}%
\pgfpathclose%
\pgfusepath{fill}%
\end{pgfscope}%
\begin{pgfscope}%
\pgfpathrectangle{\pgfqpoint{1.150000in}{0.150000in}}{\pgfqpoint{5.700000in}{5.700000in}}%
\pgfusepath{clip}%
\pgfsetbuttcap%
\pgfsetroundjoin%
\definecolor{currentfill}{rgb}{0.267968,0.223549,0.512008}%
\pgfsetfillcolor{currentfill}%
\pgfsetfillopacity{0.700000}%
\pgfsetlinewidth{0.000000pt}%
\definecolor{currentstroke}{rgb}{0.000000,0.000000,0.000000}%
\pgfsetstrokecolor{currentstroke}%
\pgfsetdash{}{0pt}%
\pgfpathmoveto{\pgfqpoint{3.863283in}{3.324015in}}%
\pgfpathlineto{\pgfqpoint{3.876327in}{3.313997in}}%
\pgfpathlineto{\pgfqpoint{3.889373in}{3.304131in}}%
\pgfpathlineto{\pgfqpoint{3.902420in}{3.294414in}}%
\pgfpathlineto{\pgfqpoint{3.915470in}{3.284846in}}%
\pgfpathlineto{\pgfqpoint{3.907880in}{3.273178in}}%
\pgfpathlineto{\pgfqpoint{3.900285in}{3.261613in}}%
\pgfpathlineto{\pgfqpoint{3.892685in}{3.250150in}}%
\pgfpathlineto{\pgfqpoint{3.885082in}{3.238786in}}%
\pgfpathlineto{\pgfqpoint{3.872021in}{3.248178in}}%
\pgfpathlineto{\pgfqpoint{3.858963in}{3.257719in}}%
\pgfpathlineto{\pgfqpoint{3.845908in}{3.267410in}}%
\pgfpathlineto{\pgfqpoint{3.832854in}{3.277252in}}%
\pgfpathlineto{\pgfqpoint{3.840468in}{3.288785in}}%
\pgfpathlineto{\pgfqpoint{3.848078in}{3.300422in}}%
\pgfpathlineto{\pgfqpoint{3.855683in}{3.312165in}}%
\pgfpathlineto{\pgfqpoint{3.863283in}{3.324015in}}%
\pgfpathclose%
\pgfusepath{fill}%
\end{pgfscope}%
\begin{pgfscope}%
\pgfpathrectangle{\pgfqpoint{1.150000in}{0.150000in}}{\pgfqpoint{5.700000in}{5.700000in}}%
\pgfusepath{clip}%
\pgfsetbuttcap%
\pgfsetroundjoin%
\definecolor{currentfill}{rgb}{0.255645,0.260703,0.528312}%
\pgfsetfillcolor{currentfill}%
\pgfsetfillopacity{0.700000}%
\pgfsetlinewidth{0.000000pt}%
\definecolor{currentstroke}{rgb}{0.000000,0.000000,0.000000}%
\pgfsetstrokecolor{currentstroke}%
\pgfsetdash{}{0pt}%
\pgfpathmoveto{\pgfqpoint{3.676313in}{3.407559in}}%
\pgfpathlineto{\pgfqpoint{3.689353in}{3.395812in}}%
\pgfpathlineto{\pgfqpoint{3.702394in}{3.384231in}}%
\pgfpathlineto{\pgfqpoint{3.715435in}{3.372814in}}%
\pgfpathlineto{\pgfqpoint{3.728477in}{3.361561in}}%
\pgfpathlineto{\pgfqpoint{3.720837in}{3.349799in}}%
\pgfpathlineto{\pgfqpoint{3.713192in}{3.338145in}}%
\pgfpathlineto{\pgfqpoint{3.705542in}{3.326598in}}%
\pgfpathlineto{\pgfqpoint{3.697887in}{3.315154in}}%
\pgfpathlineto{\pgfqpoint{3.684835in}{3.326250in}}%
\pgfpathlineto{\pgfqpoint{3.671783in}{3.337508in}}%
\pgfpathlineto{\pgfqpoint{3.658731in}{3.348931in}}%
\pgfpathlineto{\pgfqpoint{3.645680in}{3.360520in}}%
\pgfpathlineto{\pgfqpoint{3.653346in}{3.372115in}}%
\pgfpathlineto{\pgfqpoint{3.661007in}{3.383818in}}%
\pgfpathlineto{\pgfqpoint{3.668663in}{3.395632in}}%
\pgfpathlineto{\pgfqpoint{3.676313in}{3.407559in}}%
\pgfpathclose%
\pgfusepath{fill}%
\end{pgfscope}%
\begin{pgfscope}%
\pgfpathrectangle{\pgfqpoint{1.150000in}{0.150000in}}{\pgfqpoint{5.700000in}{5.700000in}}%
\pgfusepath{clip}%
\pgfsetbuttcap%
\pgfsetroundjoin%
\definecolor{currentfill}{rgb}{0.275191,0.194905,0.496005}%
\pgfsetfillcolor{currentfill}%
\pgfsetfillopacity{0.700000}%
\pgfsetlinewidth{0.000000pt}%
\definecolor{currentstroke}{rgb}{0.000000,0.000000,0.000000}%
\pgfsetstrokecolor{currentstroke}%
\pgfsetdash{}{0pt}%
\pgfpathmoveto{\pgfqpoint{4.050195in}{3.259738in}}%
\pgfpathlineto{\pgfqpoint{4.063259in}{3.251270in}}%
\pgfpathlineto{\pgfqpoint{4.076326in}{3.242940in}}%
\pgfpathlineto{\pgfqpoint{4.089396in}{3.234748in}}%
\pgfpathlineto{\pgfqpoint{4.102470in}{3.226693in}}%
\pgfpathlineto{\pgfqpoint{4.094928in}{3.215188in}}%
\pgfpathlineto{\pgfqpoint{4.087381in}{3.203783in}}%
\pgfpathlineto{\pgfqpoint{4.079831in}{3.192474in}}%
\pgfpathlineto{\pgfqpoint{4.072276in}{3.181261in}}%
\pgfpathlineto{\pgfqpoint{4.059192in}{3.189122in}}%
\pgfpathlineto{\pgfqpoint{4.046111in}{3.197121in}}%
\pgfpathlineto{\pgfqpoint{4.033033in}{3.205257in}}%
\pgfpathlineto{\pgfqpoint{4.019959in}{3.213532in}}%
\pgfpathlineto{\pgfqpoint{4.027524in}{3.224933in}}%
\pgfpathlineto{\pgfqpoint{4.035085in}{3.236432in}}%
\pgfpathlineto{\pgfqpoint{4.042642in}{3.248033in}}%
\pgfpathlineto{\pgfqpoint{4.050195in}{3.259738in}}%
\pgfpathclose%
\pgfusepath{fill}%
\end{pgfscope}%
\begin{pgfscope}%
\pgfpathrectangle{\pgfqpoint{1.150000in}{0.150000in}}{\pgfqpoint{5.700000in}{5.700000in}}%
\pgfusepath{clip}%
\pgfsetbuttcap%
\pgfsetroundjoin%
\definecolor{currentfill}{rgb}{0.262138,0.242286,0.520837}%
\pgfsetfillcolor{currentfill}%
\pgfsetfillopacity{0.700000}%
\pgfsetlinewidth{0.000000pt}%
\definecolor{currentstroke}{rgb}{0.000000,0.000000,0.000000}%
\pgfsetstrokecolor{currentstroke}%
\pgfsetdash{}{0pt}%
\pgfpathmoveto{\pgfqpoint{3.728477in}{3.361561in}}%
\pgfpathlineto{\pgfqpoint{3.741520in}{3.350469in}}%
\pgfpathlineto{\pgfqpoint{3.754564in}{3.339538in}}%
\pgfpathlineto{\pgfqpoint{3.767609in}{3.328766in}}%
\pgfpathlineto{\pgfqpoint{3.780655in}{3.318152in}}%
\pgfpathlineto{\pgfqpoint{3.773025in}{3.306555in}}%
\pgfpathlineto{\pgfqpoint{3.765391in}{3.295062in}}%
\pgfpathlineto{\pgfqpoint{3.757751in}{3.283670in}}%
\pgfpathlineto{\pgfqpoint{3.750107in}{3.272378in}}%
\pgfpathlineto{\pgfqpoint{3.737051in}{3.282834in}}%
\pgfpathlineto{\pgfqpoint{3.723995in}{3.293448in}}%
\pgfpathlineto{\pgfqpoint{3.710941in}{3.304221in}}%
\pgfpathlineto{\pgfqpoint{3.697887in}{3.315154in}}%
\pgfpathlineto{\pgfqpoint{3.705542in}{3.326598in}}%
\pgfpathlineto{\pgfqpoint{3.713192in}{3.338145in}}%
\pgfpathlineto{\pgfqpoint{3.720837in}{3.349799in}}%
\pgfpathlineto{\pgfqpoint{3.728477in}{3.361561in}}%
\pgfpathclose%
\pgfusepath{fill}%
\end{pgfscope}%
\begin{pgfscope}%
\pgfpathrectangle{\pgfqpoint{1.150000in}{0.150000in}}{\pgfqpoint{5.700000in}{5.700000in}}%
\pgfusepath{clip}%
\pgfsetbuttcap%
\pgfsetroundjoin%
\definecolor{currentfill}{rgb}{0.271828,0.209303,0.504434}%
\pgfsetfillcolor{currentfill}%
\pgfsetfillopacity{0.700000}%
\pgfsetlinewidth{0.000000pt}%
\definecolor{currentstroke}{rgb}{0.000000,0.000000,0.000000}%
\pgfsetstrokecolor{currentstroke}%
\pgfsetdash{}{0pt}%
\pgfpathmoveto{\pgfqpoint{3.915470in}{3.284846in}}%
\pgfpathlineto{\pgfqpoint{3.928522in}{3.275426in}}%
\pgfpathlineto{\pgfqpoint{3.941576in}{3.266152in}}%
\pgfpathlineto{\pgfqpoint{3.954633in}{3.257024in}}%
\pgfpathlineto{\pgfqpoint{3.967693in}{3.248040in}}%
\pgfpathlineto{\pgfqpoint{3.960113in}{3.236555in}}%
\pgfpathlineto{\pgfqpoint{3.952528in}{3.225168in}}%
\pgfpathlineto{\pgfqpoint{3.944940in}{3.213878in}}%
\pgfpathlineto{\pgfqpoint{3.937346in}{3.202683in}}%
\pgfpathlineto{\pgfqpoint{3.924276in}{3.211491in}}%
\pgfpathlineto{\pgfqpoint{3.911209in}{3.220443in}}%
\pgfpathlineto{\pgfqpoint{3.898144in}{3.229541in}}%
\pgfpathlineto{\pgfqpoint{3.885082in}{3.238786in}}%
\pgfpathlineto{\pgfqpoint{3.892685in}{3.250150in}}%
\pgfpathlineto{\pgfqpoint{3.900285in}{3.261613in}}%
\pgfpathlineto{\pgfqpoint{3.907880in}{3.273178in}}%
\pgfpathlineto{\pgfqpoint{3.915470in}{3.284846in}}%
\pgfpathclose%
\pgfusepath{fill}%
\end{pgfscope}%
\begin{pgfscope}%
\pgfpathrectangle{\pgfqpoint{1.150000in}{0.150000in}}{\pgfqpoint{5.700000in}{5.700000in}}%
\pgfusepath{clip}%
\pgfsetbuttcap%
\pgfsetroundjoin%
\definecolor{currentfill}{rgb}{0.223925,0.334994,0.548053}%
\pgfsetfillcolor{currentfill}%
\pgfsetfillopacity{0.700000}%
\pgfsetlinewidth{0.000000pt}%
\definecolor{currentstroke}{rgb}{0.000000,0.000000,0.000000}%
\pgfsetstrokecolor{currentstroke}%
\pgfsetdash{}{0pt}%
\pgfpathmoveto{\pgfqpoint{3.436813in}{3.569690in}}%
\pgfpathlineto{\pgfqpoint{3.449876in}{3.555244in}}%
\pgfpathlineto{\pgfqpoint{3.462937in}{3.540988in}}%
\pgfpathlineto{\pgfqpoint{3.475997in}{3.526922in}}%
\pgfpathlineto{\pgfqpoint{3.489054in}{3.513043in}}%
\pgfpathlineto{\pgfqpoint{3.481350in}{3.501113in}}%
\pgfpathlineto{\pgfqpoint{3.473641in}{3.489301in}}%
\pgfpathlineto{\pgfqpoint{3.465926in}{3.477604in}}%
\pgfpathlineto{\pgfqpoint{3.458205in}{3.466022in}}%
\pgfpathlineto{\pgfqpoint{3.445136in}{3.479760in}}%
\pgfpathlineto{\pgfqpoint{3.432065in}{3.493686in}}%
\pgfpathlineto{\pgfqpoint{3.418992in}{3.507800in}}%
\pgfpathlineto{\pgfqpoint{3.405918in}{3.522105in}}%
\pgfpathlineto{\pgfqpoint{3.413650in}{3.533821in}}%
\pgfpathlineto{\pgfqpoint{3.421377in}{3.545656in}}%
\pgfpathlineto{\pgfqpoint{3.429098in}{3.557612in}}%
\pgfpathlineto{\pgfqpoint{3.436813in}{3.569690in}}%
\pgfpathclose%
\pgfusepath{fill}%
\end{pgfscope}%
\begin{pgfscope}%
\pgfpathrectangle{\pgfqpoint{1.150000in}{0.150000in}}{\pgfqpoint{5.700000in}{5.700000in}}%
\pgfusepath{clip}%
\pgfsetbuttcap%
\pgfsetroundjoin%
\definecolor{currentfill}{rgb}{0.235526,0.309527,0.542944}%
\pgfsetfillcolor{currentfill}%
\pgfsetfillopacity{0.700000}%
\pgfsetlinewidth{0.000000pt}%
\definecolor{currentstroke}{rgb}{0.000000,0.000000,0.000000}%
\pgfsetstrokecolor{currentstroke}%
\pgfsetdash{}{0pt}%
\pgfpathmoveto{\pgfqpoint{3.489054in}{3.513043in}}%
\pgfpathlineto{\pgfqpoint{3.502111in}{3.499350in}}%
\pgfpathlineto{\pgfqpoint{3.515166in}{3.485842in}}%
\pgfpathlineto{\pgfqpoint{3.528220in}{3.472515in}}%
\pgfpathlineto{\pgfqpoint{3.541273in}{3.459369in}}%
\pgfpathlineto{\pgfqpoint{3.533580in}{3.447586in}}%
\pgfpathlineto{\pgfqpoint{3.525882in}{3.435916in}}%
\pgfpathlineto{\pgfqpoint{3.518178in}{3.424359in}}%
\pgfpathlineto{\pgfqpoint{3.510469in}{3.412911in}}%
\pgfpathlineto{\pgfqpoint{3.497405in}{3.425916in}}%
\pgfpathlineto{\pgfqpoint{3.484339in}{3.439102in}}%
\pgfpathlineto{\pgfqpoint{3.471273in}{3.452470in}}%
\pgfpathlineto{\pgfqpoint{3.458205in}{3.466022in}}%
\pgfpathlineto{\pgfqpoint{3.465926in}{3.477604in}}%
\pgfpathlineto{\pgfqpoint{3.473641in}{3.489301in}}%
\pgfpathlineto{\pgfqpoint{3.481350in}{3.501113in}}%
\pgfpathlineto{\pgfqpoint{3.489054in}{3.513043in}}%
\pgfpathclose%
\pgfusepath{fill}%
\end{pgfscope}%
\begin{pgfscope}%
\pgfpathrectangle{\pgfqpoint{1.150000in}{0.150000in}}{\pgfqpoint{5.700000in}{5.700000in}}%
\pgfusepath{clip}%
\pgfsetbuttcap%
\pgfsetroundjoin%
\definecolor{currentfill}{rgb}{0.244972,0.287675,0.537260}%
\pgfsetfillcolor{currentfill}%
\pgfsetfillopacity{0.700000}%
\pgfsetlinewidth{0.000000pt}%
\definecolor{currentstroke}{rgb}{0.000000,0.000000,0.000000}%
\pgfsetstrokecolor{currentstroke}%
\pgfsetdash{}{0pt}%
\pgfpathmoveto{\pgfqpoint{3.541273in}{3.459369in}}%
\pgfpathlineto{\pgfqpoint{3.554325in}{3.446402in}}%
\pgfpathlineto{\pgfqpoint{3.567377in}{3.433613in}}%
\pgfpathlineto{\pgfqpoint{3.580428in}{3.421000in}}%
\pgfpathlineto{\pgfqpoint{3.593479in}{3.408561in}}%
\pgfpathlineto{\pgfqpoint{3.585797in}{3.396925in}}%
\pgfpathlineto{\pgfqpoint{3.578110in}{3.385398in}}%
\pgfpathlineto{\pgfqpoint{3.570417in}{3.373979in}}%
\pgfpathlineto{\pgfqpoint{3.562720in}{3.362665in}}%
\pgfpathlineto{\pgfqpoint{3.549658in}{3.374963in}}%
\pgfpathlineto{\pgfqpoint{3.536595in}{3.387436in}}%
\pgfpathlineto{\pgfqpoint{3.523533in}{3.400085in}}%
\pgfpathlineto{\pgfqpoint{3.510469in}{3.412911in}}%
\pgfpathlineto{\pgfqpoint{3.518178in}{3.424359in}}%
\pgfpathlineto{\pgfqpoint{3.525882in}{3.435916in}}%
\pgfpathlineto{\pgfqpoint{3.533580in}{3.447586in}}%
\pgfpathlineto{\pgfqpoint{3.541273in}{3.459369in}}%
\pgfpathclose%
\pgfusepath{fill}%
\end{pgfscope}%
\begin{pgfscope}%
\pgfpathrectangle{\pgfqpoint{1.150000in}{0.150000in}}{\pgfqpoint{5.700000in}{5.700000in}}%
\pgfusepath{clip}%
\pgfsetbuttcap%
\pgfsetroundjoin%
\definecolor{currentfill}{rgb}{0.267968,0.223549,0.512008}%
\pgfsetfillcolor{currentfill}%
\pgfsetfillopacity{0.700000}%
\pgfsetlinewidth{0.000000pt}%
\definecolor{currentstroke}{rgb}{0.000000,0.000000,0.000000}%
\pgfsetstrokecolor{currentstroke}%
\pgfsetdash{}{0pt}%
\pgfpathmoveto{\pgfqpoint{3.780655in}{3.318152in}}%
\pgfpathlineto{\pgfqpoint{3.793702in}{3.307695in}}%
\pgfpathlineto{\pgfqpoint{3.806751in}{3.297393in}}%
\pgfpathlineto{\pgfqpoint{3.819801in}{3.287246in}}%
\pgfpathlineto{\pgfqpoint{3.832854in}{3.277252in}}%
\pgfpathlineto{\pgfqpoint{3.825234in}{3.265820in}}%
\pgfpathlineto{\pgfqpoint{3.817611in}{3.254487in}}%
\pgfpathlineto{\pgfqpoint{3.809982in}{3.243251in}}%
\pgfpathlineto{\pgfqpoint{3.802349in}{3.232111in}}%
\pgfpathlineto{\pgfqpoint{3.789286in}{3.241947in}}%
\pgfpathlineto{\pgfqpoint{3.776225in}{3.251936in}}%
\pgfpathlineto{\pgfqpoint{3.763166in}{3.262079in}}%
\pgfpathlineto{\pgfqpoint{3.750107in}{3.272378in}}%
\pgfpathlineto{\pgfqpoint{3.757751in}{3.283670in}}%
\pgfpathlineto{\pgfqpoint{3.765391in}{3.295062in}}%
\pgfpathlineto{\pgfqpoint{3.773025in}{3.306555in}}%
\pgfpathlineto{\pgfqpoint{3.780655in}{3.318152in}}%
\pgfpathclose%
\pgfusepath{fill}%
\end{pgfscope}%
\begin{pgfscope}%
\pgfpathrectangle{\pgfqpoint{1.150000in}{0.150000in}}{\pgfqpoint{5.700000in}{5.700000in}}%
\pgfusepath{clip}%
\pgfsetbuttcap%
\pgfsetroundjoin%
\definecolor{currentfill}{rgb}{0.277134,0.185228,0.489898}%
\pgfsetfillcolor{currentfill}%
\pgfsetfillopacity{0.700000}%
\pgfsetlinewidth{0.000000pt}%
\definecolor{currentstroke}{rgb}{0.000000,0.000000,0.000000}%
\pgfsetstrokecolor{currentstroke}%
\pgfsetdash{}{0pt}%
\pgfpathmoveto{\pgfqpoint{4.102470in}{3.226693in}}%
\pgfpathlineto{\pgfqpoint{4.115548in}{3.218775in}}%
\pgfpathlineto{\pgfqpoint{4.128629in}{3.210991in}}%
\pgfpathlineto{\pgfqpoint{4.141713in}{3.203342in}}%
\pgfpathlineto{\pgfqpoint{4.154802in}{3.195827in}}%
\pgfpathlineto{\pgfqpoint{4.147270in}{3.184522in}}%
\pgfpathlineto{\pgfqpoint{4.139734in}{3.173312in}}%
\pgfpathlineto{\pgfqpoint{4.132194in}{3.162195in}}%
\pgfpathlineto{\pgfqpoint{4.124650in}{3.151168in}}%
\pgfpathlineto{\pgfqpoint{4.111551in}{3.158490in}}%
\pgfpathlineto{\pgfqpoint{4.098455in}{3.165945in}}%
\pgfpathlineto{\pgfqpoint{4.085364in}{3.173535in}}%
\pgfpathlineto{\pgfqpoint{4.072276in}{3.181261in}}%
\pgfpathlineto{\pgfqpoint{4.079831in}{3.192474in}}%
\pgfpathlineto{\pgfqpoint{4.087381in}{3.203783in}}%
\pgfpathlineto{\pgfqpoint{4.094928in}{3.215188in}}%
\pgfpathlineto{\pgfqpoint{4.102470in}{3.226693in}}%
\pgfpathclose%
\pgfusepath{fill}%
\end{pgfscope}%
\begin{pgfscope}%
\pgfpathrectangle{\pgfqpoint{1.150000in}{0.150000in}}{\pgfqpoint{5.700000in}{5.700000in}}%
\pgfusepath{clip}%
\pgfsetbuttcap%
\pgfsetroundjoin%
\definecolor{currentfill}{rgb}{0.275191,0.194905,0.496005}%
\pgfsetfillcolor{currentfill}%
\pgfsetfillopacity{0.700000}%
\pgfsetlinewidth{0.000000pt}%
\definecolor{currentstroke}{rgb}{0.000000,0.000000,0.000000}%
\pgfsetstrokecolor{currentstroke}%
\pgfsetdash{}{0pt}%
\pgfpathmoveto{\pgfqpoint{3.967693in}{3.248040in}}%
\pgfpathlineto{\pgfqpoint{3.980755in}{3.239200in}}%
\pgfpathlineto{\pgfqpoint{3.993820in}{3.230503in}}%
\pgfpathlineto{\pgfqpoint{4.006888in}{3.221947in}}%
\pgfpathlineto{\pgfqpoint{4.019959in}{3.213532in}}%
\pgfpathlineto{\pgfqpoint{4.012390in}{3.202229in}}%
\pgfpathlineto{\pgfqpoint{4.004816in}{3.191020in}}%
\pgfpathlineto{\pgfqpoint{3.997237in}{3.179904in}}%
\pgfpathlineto{\pgfqpoint{3.989655in}{3.168878in}}%
\pgfpathlineto{\pgfqpoint{3.976573in}{3.177117in}}%
\pgfpathlineto{\pgfqpoint{3.963494in}{3.185497in}}%
\pgfpathlineto{\pgfqpoint{3.950419in}{3.194019in}}%
\pgfpathlineto{\pgfqpoint{3.937346in}{3.202683in}}%
\pgfpathlineto{\pgfqpoint{3.944940in}{3.213878in}}%
\pgfpathlineto{\pgfqpoint{3.952528in}{3.225168in}}%
\pgfpathlineto{\pgfqpoint{3.960113in}{3.236555in}}%
\pgfpathlineto{\pgfqpoint{3.967693in}{3.248040in}}%
\pgfpathclose%
\pgfusepath{fill}%
\end{pgfscope}%
\begin{pgfscope}%
\pgfpathrectangle{\pgfqpoint{1.150000in}{0.150000in}}{\pgfqpoint{5.700000in}{5.700000in}}%
\pgfusepath{clip}%
\pgfsetbuttcap%
\pgfsetroundjoin%
\definecolor{currentfill}{rgb}{0.252194,0.269783,0.531579}%
\pgfsetfillcolor{currentfill}%
\pgfsetfillopacity{0.700000}%
\pgfsetlinewidth{0.000000pt}%
\definecolor{currentstroke}{rgb}{0.000000,0.000000,0.000000}%
\pgfsetstrokecolor{currentstroke}%
\pgfsetdash{}{0pt}%
\pgfpathmoveto{\pgfqpoint{3.593479in}{3.408561in}}%
\pgfpathlineto{\pgfqpoint{3.606529in}{3.396295in}}%
\pgfpathlineto{\pgfqpoint{3.619579in}{3.384200in}}%
\pgfpathlineto{\pgfqpoint{3.632630in}{3.372276in}}%
\pgfpathlineto{\pgfqpoint{3.645680in}{3.360520in}}%
\pgfpathlineto{\pgfqpoint{3.638010in}{3.349031in}}%
\pgfpathlineto{\pgfqpoint{3.630334in}{3.337647in}}%
\pgfpathlineto{\pgfqpoint{3.622653in}{3.326366in}}%
\pgfpathlineto{\pgfqpoint{3.614966in}{3.315186in}}%
\pgfpathlineto{\pgfqpoint{3.601905in}{3.326802in}}%
\pgfpathlineto{\pgfqpoint{3.588843in}{3.338585in}}%
\pgfpathlineto{\pgfqpoint{3.575781in}{3.350539in}}%
\pgfpathlineto{\pgfqpoint{3.562720in}{3.362665in}}%
\pgfpathlineto{\pgfqpoint{3.570417in}{3.373979in}}%
\pgfpathlineto{\pgfqpoint{3.578110in}{3.385398in}}%
\pgfpathlineto{\pgfqpoint{3.585797in}{3.396925in}}%
\pgfpathlineto{\pgfqpoint{3.593479in}{3.408561in}}%
\pgfpathclose%
\pgfusepath{fill}%
\end{pgfscope}%
\begin{pgfscope}%
\pgfpathrectangle{\pgfqpoint{1.150000in}{0.150000in}}{\pgfqpoint{5.700000in}{5.700000in}}%
\pgfusepath{clip}%
\pgfsetbuttcap%
\pgfsetroundjoin%
\definecolor{currentfill}{rgb}{0.271828,0.209303,0.504434}%
\pgfsetfillcolor{currentfill}%
\pgfsetfillopacity{0.700000}%
\pgfsetlinewidth{0.000000pt}%
\definecolor{currentstroke}{rgb}{0.000000,0.000000,0.000000}%
\pgfsetstrokecolor{currentstroke}%
\pgfsetdash{}{0pt}%
\pgfpathmoveto{\pgfqpoint{3.832854in}{3.277252in}}%
\pgfpathlineto{\pgfqpoint{3.845908in}{3.267410in}}%
\pgfpathlineto{\pgfqpoint{3.858963in}{3.257719in}}%
\pgfpathlineto{\pgfqpoint{3.872021in}{3.248178in}}%
\pgfpathlineto{\pgfqpoint{3.885082in}{3.238786in}}%
\pgfpathlineto{\pgfqpoint{3.877473in}{3.227518in}}%
\pgfpathlineto{\pgfqpoint{3.869860in}{3.216346in}}%
\pgfpathlineto{\pgfqpoint{3.862242in}{3.205266in}}%
\pgfpathlineto{\pgfqpoint{3.854620in}{3.194277in}}%
\pgfpathlineto{\pgfqpoint{3.841549in}{3.203511in}}%
\pgfpathlineto{\pgfqpoint{3.828480in}{3.212894in}}%
\pgfpathlineto{\pgfqpoint{3.815414in}{3.222427in}}%
\pgfpathlineto{\pgfqpoint{3.802349in}{3.232111in}}%
\pgfpathlineto{\pgfqpoint{3.809982in}{3.243251in}}%
\pgfpathlineto{\pgfqpoint{3.817611in}{3.254487in}}%
\pgfpathlineto{\pgfqpoint{3.825234in}{3.265820in}}%
\pgfpathlineto{\pgfqpoint{3.832854in}{3.277252in}}%
\pgfpathclose%
\pgfusepath{fill}%
\end{pgfscope}%
\begin{pgfscope}%
\pgfpathrectangle{\pgfqpoint{1.150000in}{0.150000in}}{\pgfqpoint{5.700000in}{5.700000in}}%
\pgfusepath{clip}%
\pgfsetbuttcap%
\pgfsetroundjoin%
\definecolor{currentfill}{rgb}{0.260571,0.246922,0.522828}%
\pgfsetfillcolor{currentfill}%
\pgfsetfillopacity{0.700000}%
\pgfsetlinewidth{0.000000pt}%
\definecolor{currentstroke}{rgb}{0.000000,0.000000,0.000000}%
\pgfsetstrokecolor{currentstroke}%
\pgfsetdash{}{0pt}%
\pgfpathmoveto{\pgfqpoint{3.645680in}{3.360520in}}%
\pgfpathlineto{\pgfqpoint{3.658731in}{3.348931in}}%
\pgfpathlineto{\pgfqpoint{3.671783in}{3.337508in}}%
\pgfpathlineto{\pgfqpoint{3.684835in}{3.326250in}}%
\pgfpathlineto{\pgfqpoint{3.697887in}{3.315154in}}%
\pgfpathlineto{\pgfqpoint{3.690227in}{3.303813in}}%
\pgfpathlineto{\pgfqpoint{3.682563in}{3.292572in}}%
\pgfpathlineto{\pgfqpoint{3.674893in}{3.281429in}}%
\pgfpathlineto{\pgfqpoint{3.667218in}{3.270383in}}%
\pgfpathlineto{\pgfqpoint{3.654154in}{3.281338in}}%
\pgfpathlineto{\pgfqpoint{3.641091in}{3.292456in}}%
\pgfpathlineto{\pgfqpoint{3.628028in}{3.303738in}}%
\pgfpathlineto{\pgfqpoint{3.614966in}{3.315186in}}%
\pgfpathlineto{\pgfqpoint{3.622653in}{3.326366in}}%
\pgfpathlineto{\pgfqpoint{3.630334in}{3.337647in}}%
\pgfpathlineto{\pgfqpoint{3.638010in}{3.349031in}}%
\pgfpathlineto{\pgfqpoint{3.645680in}{3.360520in}}%
\pgfpathclose%
\pgfusepath{fill}%
\end{pgfscope}%
\begin{pgfscope}%
\pgfpathrectangle{\pgfqpoint{1.150000in}{0.150000in}}{\pgfqpoint{5.700000in}{5.700000in}}%
\pgfusepath{clip}%
\pgfsetbuttcap%
\pgfsetroundjoin%
\definecolor{currentfill}{rgb}{0.278012,0.180367,0.486697}%
\pgfsetfillcolor{currentfill}%
\pgfsetfillopacity{0.700000}%
\pgfsetlinewidth{0.000000pt}%
\definecolor{currentstroke}{rgb}{0.000000,0.000000,0.000000}%
\pgfsetstrokecolor{currentstroke}%
\pgfsetdash{}{0pt}%
\pgfpathmoveto{\pgfqpoint{4.019959in}{3.213532in}}%
\pgfpathlineto{\pgfqpoint{4.033033in}{3.205257in}}%
\pgfpathlineto{\pgfqpoint{4.046111in}{3.197121in}}%
\pgfpathlineto{\pgfqpoint{4.059192in}{3.189122in}}%
\pgfpathlineto{\pgfqpoint{4.072276in}{3.181261in}}%
\pgfpathlineto{\pgfqpoint{4.064717in}{3.170140in}}%
\pgfpathlineto{\pgfqpoint{4.057153in}{3.159109in}}%
\pgfpathlineto{\pgfqpoint{4.049586in}{3.148166in}}%
\pgfpathlineto{\pgfqpoint{4.042014in}{3.137310in}}%
\pgfpathlineto{\pgfqpoint{4.028919in}{3.144995in}}%
\pgfpathlineto{\pgfqpoint{4.015827in}{3.152818in}}%
\pgfpathlineto{\pgfqpoint{4.002739in}{3.160779in}}%
\pgfpathlineto{\pgfqpoint{3.989655in}{3.168878in}}%
\pgfpathlineto{\pgfqpoint{3.997237in}{3.179904in}}%
\pgfpathlineto{\pgfqpoint{4.004816in}{3.191020in}}%
\pgfpathlineto{\pgfqpoint{4.012390in}{3.202229in}}%
\pgfpathlineto{\pgfqpoint{4.019959in}{3.213532in}}%
\pgfpathclose%
\pgfusepath{fill}%
\end{pgfscope}%
\begin{pgfscope}%
\pgfpathrectangle{\pgfqpoint{1.150000in}{0.150000in}}{\pgfqpoint{5.700000in}{5.700000in}}%
\pgfusepath{clip}%
\pgfsetbuttcap%
\pgfsetroundjoin%
\definecolor{currentfill}{rgb}{0.278826,0.175490,0.483397}%
\pgfsetfillcolor{currentfill}%
\pgfsetfillopacity{0.700000}%
\pgfsetlinewidth{0.000000pt}%
\definecolor{currentstroke}{rgb}{0.000000,0.000000,0.000000}%
\pgfsetstrokecolor{currentstroke}%
\pgfsetdash{}{0pt}%
\pgfpathmoveto{\pgfqpoint{4.154802in}{3.195827in}}%
\pgfpathlineto{\pgfqpoint{4.167895in}{3.188445in}}%
\pgfpathlineto{\pgfqpoint{4.180991in}{3.181194in}}%
\pgfpathlineto{\pgfqpoint{4.194092in}{3.174075in}}%
\pgfpathlineto{\pgfqpoint{4.207197in}{3.167086in}}%
\pgfpathlineto{\pgfqpoint{4.199676in}{3.155981in}}%
\pgfpathlineto{\pgfqpoint{4.192150in}{3.144966in}}%
\pgfpathlineto{\pgfqpoint{4.184621in}{3.134041in}}%
\pgfpathlineto{\pgfqpoint{4.177087in}{3.123201in}}%
\pgfpathlineto{\pgfqpoint{4.163972in}{3.129996in}}%
\pgfpathlineto{\pgfqpoint{4.150860in}{3.136922in}}%
\pgfpathlineto{\pgfqpoint{4.137753in}{3.143979in}}%
\pgfpathlineto{\pgfqpoint{4.124650in}{3.151168in}}%
\pgfpathlineto{\pgfqpoint{4.132194in}{3.162195in}}%
\pgfpathlineto{\pgfqpoint{4.139734in}{3.173312in}}%
\pgfpathlineto{\pgfqpoint{4.147270in}{3.184522in}}%
\pgfpathlineto{\pgfqpoint{4.154802in}{3.195827in}}%
\pgfpathclose%
\pgfusepath{fill}%
\end{pgfscope}%
\begin{pgfscope}%
\pgfpathrectangle{\pgfqpoint{1.150000in}{0.150000in}}{\pgfqpoint{5.700000in}{5.700000in}}%
\pgfusepath{clip}%
\pgfsetbuttcap%
\pgfsetroundjoin%
\definecolor{currentfill}{rgb}{0.266580,0.228262,0.514349}%
\pgfsetfillcolor{currentfill}%
\pgfsetfillopacity{0.700000}%
\pgfsetlinewidth{0.000000pt}%
\definecolor{currentstroke}{rgb}{0.000000,0.000000,0.000000}%
\pgfsetstrokecolor{currentstroke}%
\pgfsetdash{}{0pt}%
\pgfpathmoveto{\pgfqpoint{3.697887in}{3.315154in}}%
\pgfpathlineto{\pgfqpoint{3.710941in}{3.304221in}}%
\pgfpathlineto{\pgfqpoint{3.723995in}{3.293448in}}%
\pgfpathlineto{\pgfqpoint{3.737051in}{3.282834in}}%
\pgfpathlineto{\pgfqpoint{3.750107in}{3.272378in}}%
\pgfpathlineto{\pgfqpoint{3.742458in}{3.261184in}}%
\pgfpathlineto{\pgfqpoint{3.734805in}{3.250085in}}%
\pgfpathlineto{\pgfqpoint{3.727146in}{3.239081in}}%
\pgfpathlineto{\pgfqpoint{3.719482in}{3.228169in}}%
\pgfpathlineto{\pgfqpoint{3.706414in}{3.238484in}}%
\pgfpathlineto{\pgfqpoint{3.693348in}{3.248957in}}%
\pgfpathlineto{\pgfqpoint{3.680282in}{3.259590in}}%
\pgfpathlineto{\pgfqpoint{3.667218in}{3.270383in}}%
\pgfpathlineto{\pgfqpoint{3.674893in}{3.281429in}}%
\pgfpathlineto{\pgfqpoint{3.682563in}{3.292572in}}%
\pgfpathlineto{\pgfqpoint{3.690227in}{3.303813in}}%
\pgfpathlineto{\pgfqpoint{3.697887in}{3.315154in}}%
\pgfpathclose%
\pgfusepath{fill}%
\end{pgfscope}%
\begin{pgfscope}%
\pgfpathrectangle{\pgfqpoint{1.150000in}{0.150000in}}{\pgfqpoint{5.700000in}{5.700000in}}%
\pgfusepath{clip}%
\pgfsetbuttcap%
\pgfsetroundjoin%
\definecolor{currentfill}{rgb}{0.229739,0.322361,0.545706}%
\pgfsetfillcolor{currentfill}%
\pgfsetfillopacity{0.700000}%
\pgfsetlinewidth{0.000000pt}%
\definecolor{currentstroke}{rgb}{0.000000,0.000000,0.000000}%
\pgfsetstrokecolor{currentstroke}%
\pgfsetdash{}{0pt}%
\pgfpathmoveto{\pgfqpoint{3.405918in}{3.522105in}}%
\pgfpathlineto{\pgfqpoint{3.418992in}{3.507800in}}%
\pgfpathlineto{\pgfqpoint{3.432065in}{3.493686in}}%
\pgfpathlineto{\pgfqpoint{3.445136in}{3.479760in}}%
\pgfpathlineto{\pgfqpoint{3.458205in}{3.466022in}}%
\pgfpathlineto{\pgfqpoint{3.450479in}{3.454553in}}%
\pgfpathlineto{\pgfqpoint{3.442747in}{3.443195in}}%
\pgfpathlineto{\pgfqpoint{3.435009in}{3.431946in}}%
\pgfpathlineto{\pgfqpoint{3.427266in}{3.420804in}}%
\pgfpathlineto{\pgfqpoint{3.414185in}{3.434419in}}%
\pgfpathlineto{\pgfqpoint{3.401102in}{3.448222in}}%
\pgfpathlineto{\pgfqpoint{3.388018in}{3.462213in}}%
\pgfpathlineto{\pgfqpoint{3.374931in}{3.476396in}}%
\pgfpathlineto{\pgfqpoint{3.382686in}{3.487653in}}%
\pgfpathlineto{\pgfqpoint{3.390436in}{3.499023in}}%
\pgfpathlineto{\pgfqpoint{3.398180in}{3.510506in}}%
\pgfpathlineto{\pgfqpoint{3.405918in}{3.522105in}}%
\pgfpathclose%
\pgfusepath{fill}%
\end{pgfscope}%
\begin{pgfscope}%
\pgfpathrectangle{\pgfqpoint{1.150000in}{0.150000in}}{\pgfqpoint{5.700000in}{5.700000in}}%
\pgfusepath{clip}%
\pgfsetbuttcap%
\pgfsetroundjoin%
\definecolor{currentfill}{rgb}{0.275191,0.194905,0.496005}%
\pgfsetfillcolor{currentfill}%
\pgfsetfillopacity{0.700000}%
\pgfsetlinewidth{0.000000pt}%
\definecolor{currentstroke}{rgb}{0.000000,0.000000,0.000000}%
\pgfsetstrokecolor{currentstroke}%
\pgfsetdash{}{0pt}%
\pgfpathmoveto{\pgfqpoint{3.885082in}{3.238786in}}%
\pgfpathlineto{\pgfqpoint{3.898144in}{3.229541in}}%
\pgfpathlineto{\pgfqpoint{3.911209in}{3.220443in}}%
\pgfpathlineto{\pgfqpoint{3.924276in}{3.211491in}}%
\pgfpathlineto{\pgfqpoint{3.937346in}{3.202683in}}%
\pgfpathlineto{\pgfqpoint{3.929748in}{3.191580in}}%
\pgfpathlineto{\pgfqpoint{3.922146in}{3.180568in}}%
\pgfpathlineto{\pgfqpoint{3.914539in}{3.169645in}}%
\pgfpathlineto{\pgfqpoint{3.906927in}{3.158807in}}%
\pgfpathlineto{\pgfqpoint{3.893847in}{3.167457in}}%
\pgfpathlineto{\pgfqpoint{3.880768in}{3.176251in}}%
\pgfpathlineto{\pgfqpoint{3.867693in}{3.185191in}}%
\pgfpathlineto{\pgfqpoint{3.854620in}{3.194277in}}%
\pgfpathlineto{\pgfqpoint{3.862242in}{3.205266in}}%
\pgfpathlineto{\pgfqpoint{3.869860in}{3.216346in}}%
\pgfpathlineto{\pgfqpoint{3.877473in}{3.227518in}}%
\pgfpathlineto{\pgfqpoint{3.885082in}{3.238786in}}%
\pgfpathclose%
\pgfusepath{fill}%
\end{pgfscope}%
\begin{pgfscope}%
\pgfpathrectangle{\pgfqpoint{1.150000in}{0.150000in}}{\pgfqpoint{5.700000in}{5.700000in}}%
\pgfusepath{clip}%
\pgfsetbuttcap%
\pgfsetroundjoin%
\definecolor{currentfill}{rgb}{0.239346,0.300855,0.540844}%
\pgfsetfillcolor{currentfill}%
\pgfsetfillopacity{0.700000}%
\pgfsetlinewidth{0.000000pt}%
\definecolor{currentstroke}{rgb}{0.000000,0.000000,0.000000}%
\pgfsetstrokecolor{currentstroke}%
\pgfsetdash{}{0pt}%
\pgfpathmoveto{\pgfqpoint{3.458205in}{3.466022in}}%
\pgfpathlineto{\pgfqpoint{3.471273in}{3.452470in}}%
\pgfpathlineto{\pgfqpoint{3.484339in}{3.439102in}}%
\pgfpathlineto{\pgfqpoint{3.497405in}{3.425916in}}%
\pgfpathlineto{\pgfqpoint{3.510469in}{3.412911in}}%
\pgfpathlineto{\pgfqpoint{3.502754in}{3.401571in}}%
\pgfpathlineto{\pgfqpoint{3.495034in}{3.390338in}}%
\pgfpathlineto{\pgfqpoint{3.487309in}{3.379209in}}%
\pgfpathlineto{\pgfqpoint{3.479578in}{3.368184in}}%
\pgfpathlineto{\pgfqpoint{3.466502in}{3.381066in}}%
\pgfpathlineto{\pgfqpoint{3.453424in}{3.394129in}}%
\pgfpathlineto{\pgfqpoint{3.440346in}{3.407375in}}%
\pgfpathlineto{\pgfqpoint{3.427266in}{3.420804in}}%
\pgfpathlineto{\pgfqpoint{3.435009in}{3.431946in}}%
\pgfpathlineto{\pgfqpoint{3.442747in}{3.443195in}}%
\pgfpathlineto{\pgfqpoint{3.450479in}{3.454553in}}%
\pgfpathlineto{\pgfqpoint{3.458205in}{3.466022in}}%
\pgfpathclose%
\pgfusepath{fill}%
\end{pgfscope}%
\begin{pgfscope}%
\pgfpathrectangle{\pgfqpoint{1.150000in}{0.150000in}}{\pgfqpoint{5.700000in}{5.700000in}}%
\pgfusepath{clip}%
\pgfsetbuttcap%
\pgfsetroundjoin%
\definecolor{currentfill}{rgb}{0.248629,0.278775,0.534556}%
\pgfsetfillcolor{currentfill}%
\pgfsetfillopacity{0.700000}%
\pgfsetlinewidth{0.000000pt}%
\definecolor{currentstroke}{rgb}{0.000000,0.000000,0.000000}%
\pgfsetstrokecolor{currentstroke}%
\pgfsetdash{}{0pt}%
\pgfpathmoveto{\pgfqpoint{3.510469in}{3.412911in}}%
\pgfpathlineto{\pgfqpoint{3.523533in}{3.400085in}}%
\pgfpathlineto{\pgfqpoint{3.536595in}{3.387436in}}%
\pgfpathlineto{\pgfqpoint{3.549658in}{3.374963in}}%
\pgfpathlineto{\pgfqpoint{3.562720in}{3.362665in}}%
\pgfpathlineto{\pgfqpoint{3.555017in}{3.351455in}}%
\pgfpathlineto{\pgfqpoint{3.547308in}{3.340346in}}%
\pgfpathlineto{\pgfqpoint{3.539595in}{3.329338in}}%
\pgfpathlineto{\pgfqpoint{3.531876in}{3.318429in}}%
\pgfpathlineto{\pgfqpoint{3.518802in}{3.330605in}}%
\pgfpathlineto{\pgfqpoint{3.505728in}{3.342955in}}%
\pgfpathlineto{\pgfqpoint{3.492653in}{3.355480in}}%
\pgfpathlineto{\pgfqpoint{3.479578in}{3.368184in}}%
\pgfpathlineto{\pgfqpoint{3.487309in}{3.379209in}}%
\pgfpathlineto{\pgfqpoint{3.495034in}{3.390338in}}%
\pgfpathlineto{\pgfqpoint{3.502754in}{3.401571in}}%
\pgfpathlineto{\pgfqpoint{3.510469in}{3.412911in}}%
\pgfpathclose%
\pgfusepath{fill}%
\end{pgfscope}%
\begin{pgfscope}%
\pgfpathrectangle{\pgfqpoint{1.150000in}{0.150000in}}{\pgfqpoint{5.700000in}{5.700000in}}%
\pgfusepath{clip}%
\pgfsetbuttcap%
\pgfsetroundjoin%
\definecolor{currentfill}{rgb}{0.270595,0.214069,0.507052}%
\pgfsetfillcolor{currentfill}%
\pgfsetfillopacity{0.700000}%
\pgfsetlinewidth{0.000000pt}%
\definecolor{currentstroke}{rgb}{0.000000,0.000000,0.000000}%
\pgfsetstrokecolor{currentstroke}%
\pgfsetdash{}{0pt}%
\pgfpathmoveto{\pgfqpoint{3.750107in}{3.272378in}}%
\pgfpathlineto{\pgfqpoint{3.763166in}{3.262079in}}%
\pgfpathlineto{\pgfqpoint{3.776225in}{3.251936in}}%
\pgfpathlineto{\pgfqpoint{3.789286in}{3.241947in}}%
\pgfpathlineto{\pgfqpoint{3.802349in}{3.232111in}}%
\pgfpathlineto{\pgfqpoint{3.794711in}{3.221064in}}%
\pgfpathlineto{\pgfqpoint{3.787068in}{3.210108in}}%
\pgfpathlineto{\pgfqpoint{3.779420in}{3.199242in}}%
\pgfpathlineto{\pgfqpoint{3.771767in}{3.188464in}}%
\pgfpathlineto{\pgfqpoint{3.758694in}{3.198159in}}%
\pgfpathlineto{\pgfqpoint{3.745622in}{3.208007in}}%
\pgfpathlineto{\pgfqpoint{3.732551in}{3.218010in}}%
\pgfpathlineto{\pgfqpoint{3.719482in}{3.228169in}}%
\pgfpathlineto{\pgfqpoint{3.727146in}{3.239081in}}%
\pgfpathlineto{\pgfqpoint{3.734805in}{3.250085in}}%
\pgfpathlineto{\pgfqpoint{3.742458in}{3.261184in}}%
\pgfpathlineto{\pgfqpoint{3.750107in}{3.272378in}}%
\pgfpathclose%
\pgfusepath{fill}%
\end{pgfscope}%
\begin{pgfscope}%
\pgfpathrectangle{\pgfqpoint{1.150000in}{0.150000in}}{\pgfqpoint{5.700000in}{5.700000in}}%
\pgfusepath{clip}%
\pgfsetbuttcap%
\pgfsetroundjoin%
\definecolor{currentfill}{rgb}{0.279574,0.170599,0.479997}%
\pgfsetfillcolor{currentfill}%
\pgfsetfillopacity{0.700000}%
\pgfsetlinewidth{0.000000pt}%
\definecolor{currentstroke}{rgb}{0.000000,0.000000,0.000000}%
\pgfsetstrokecolor{currentstroke}%
\pgfsetdash{}{0pt}%
\pgfpathmoveto{\pgfqpoint{4.072276in}{3.181261in}}%
\pgfpathlineto{\pgfqpoint{4.085364in}{3.173535in}}%
\pgfpathlineto{\pgfqpoint{4.098455in}{3.165945in}}%
\pgfpathlineto{\pgfqpoint{4.111551in}{3.158490in}}%
\pgfpathlineto{\pgfqpoint{4.124650in}{3.151168in}}%
\pgfpathlineto{\pgfqpoint{4.117101in}{3.140229in}}%
\pgfpathlineto{\pgfqpoint{4.109549in}{3.129377in}}%
\pgfpathlineto{\pgfqpoint{4.101992in}{3.118608in}}%
\pgfpathlineto{\pgfqpoint{4.094430in}{3.107921in}}%
\pgfpathlineto{\pgfqpoint{4.081320in}{3.115067in}}%
\pgfpathlineto{\pgfqpoint{4.068214in}{3.122346in}}%
\pgfpathlineto{\pgfqpoint{4.055112in}{3.129760in}}%
\pgfpathlineto{\pgfqpoint{4.042014in}{3.137310in}}%
\pgfpathlineto{\pgfqpoint{4.049586in}{3.148166in}}%
\pgfpathlineto{\pgfqpoint{4.057153in}{3.159109in}}%
\pgfpathlineto{\pgfqpoint{4.064717in}{3.170140in}}%
\pgfpathlineto{\pgfqpoint{4.072276in}{3.181261in}}%
\pgfpathclose%
\pgfusepath{fill}%
\end{pgfscope}%
\begin{pgfscope}%
\pgfpathrectangle{\pgfqpoint{1.150000in}{0.150000in}}{\pgfqpoint{5.700000in}{5.700000in}}%
\pgfusepath{clip}%
\pgfsetbuttcap%
\pgfsetroundjoin%
\definecolor{currentfill}{rgb}{0.257322,0.256130,0.526563}%
\pgfsetfillcolor{currentfill}%
\pgfsetfillopacity{0.700000}%
\pgfsetlinewidth{0.000000pt}%
\definecolor{currentstroke}{rgb}{0.000000,0.000000,0.000000}%
\pgfsetstrokecolor{currentstroke}%
\pgfsetdash{}{0pt}%
\pgfpathmoveto{\pgfqpoint{3.562720in}{3.362665in}}%
\pgfpathlineto{\pgfqpoint{3.575781in}{3.350539in}}%
\pgfpathlineto{\pgfqpoint{3.588843in}{3.338585in}}%
\pgfpathlineto{\pgfqpoint{3.601905in}{3.326802in}}%
\pgfpathlineto{\pgfqpoint{3.614966in}{3.315186in}}%
\pgfpathlineto{\pgfqpoint{3.607275in}{3.304105in}}%
\pgfpathlineto{\pgfqpoint{3.599578in}{3.293122in}}%
\pgfpathlineto{\pgfqpoint{3.591876in}{3.282235in}}%
\pgfpathlineto{\pgfqpoint{3.584169in}{3.271442in}}%
\pgfpathlineto{\pgfqpoint{3.571095in}{3.282934in}}%
\pgfpathlineto{\pgfqpoint{3.558022in}{3.294595in}}%
\pgfpathlineto{\pgfqpoint{3.544949in}{3.306427in}}%
\pgfpathlineto{\pgfqpoint{3.531876in}{3.318429in}}%
\pgfpathlineto{\pgfqpoint{3.539595in}{3.329338in}}%
\pgfpathlineto{\pgfqpoint{3.547308in}{3.340346in}}%
\pgfpathlineto{\pgfqpoint{3.555017in}{3.351455in}}%
\pgfpathlineto{\pgfqpoint{3.562720in}{3.362665in}}%
\pgfpathclose%
\pgfusepath{fill}%
\end{pgfscope}%
\begin{pgfscope}%
\pgfpathrectangle{\pgfqpoint{1.150000in}{0.150000in}}{\pgfqpoint{5.700000in}{5.700000in}}%
\pgfusepath{clip}%
\pgfsetbuttcap%
\pgfsetroundjoin%
\definecolor{currentfill}{rgb}{0.278012,0.180367,0.486697}%
\pgfsetfillcolor{currentfill}%
\pgfsetfillopacity{0.700000}%
\pgfsetlinewidth{0.000000pt}%
\definecolor{currentstroke}{rgb}{0.000000,0.000000,0.000000}%
\pgfsetstrokecolor{currentstroke}%
\pgfsetdash{}{0pt}%
\pgfpathmoveto{\pgfqpoint{3.937346in}{3.202683in}}%
\pgfpathlineto{\pgfqpoint{3.950419in}{3.194019in}}%
\pgfpathlineto{\pgfqpoint{3.963494in}{3.185497in}}%
\pgfpathlineto{\pgfqpoint{3.976573in}{3.177117in}}%
\pgfpathlineto{\pgfqpoint{3.989655in}{3.168878in}}%
\pgfpathlineto{\pgfqpoint{3.982067in}{3.157940in}}%
\pgfpathlineto{\pgfqpoint{3.974476in}{3.147089in}}%
\pgfpathlineto{\pgfqpoint{3.966879in}{3.136321in}}%
\pgfpathlineto{\pgfqpoint{3.959278in}{3.125636in}}%
\pgfpathlineto{\pgfqpoint{3.946186in}{3.133717in}}%
\pgfpathlineto{\pgfqpoint{3.933097in}{3.141938in}}%
\pgfpathlineto{\pgfqpoint{3.920011in}{3.150302in}}%
\pgfpathlineto{\pgfqpoint{3.906927in}{3.158807in}}%
\pgfpathlineto{\pgfqpoint{3.914539in}{3.169645in}}%
\pgfpathlineto{\pgfqpoint{3.922146in}{3.180568in}}%
\pgfpathlineto{\pgfqpoint{3.929748in}{3.191580in}}%
\pgfpathlineto{\pgfqpoint{3.937346in}{3.202683in}}%
\pgfpathclose%
\pgfusepath{fill}%
\end{pgfscope}%
\begin{pgfscope}%
\pgfpathrectangle{\pgfqpoint{1.150000in}{0.150000in}}{\pgfqpoint{5.700000in}{5.700000in}}%
\pgfusepath{clip}%
\pgfsetbuttcap%
\pgfsetroundjoin%
\definecolor{currentfill}{rgb}{0.280255,0.165693,0.476498}%
\pgfsetfillcolor{currentfill}%
\pgfsetfillopacity{0.700000}%
\pgfsetlinewidth{0.000000pt}%
\definecolor{currentstroke}{rgb}{0.000000,0.000000,0.000000}%
\pgfsetstrokecolor{currentstroke}%
\pgfsetdash{}{0pt}%
\pgfpathmoveto{\pgfqpoint{4.207197in}{3.167086in}}%
\pgfpathlineto{\pgfqpoint{4.220306in}{3.160227in}}%
\pgfpathlineto{\pgfqpoint{4.233420in}{3.153497in}}%
\pgfpathlineto{\pgfqpoint{4.246538in}{3.146895in}}%
\pgfpathlineto{\pgfqpoint{4.259661in}{3.140420in}}%
\pgfpathlineto{\pgfqpoint{4.252151in}{3.129515in}}%
\pgfpathlineto{\pgfqpoint{4.244636in}{3.118697in}}%
\pgfpathlineto{\pgfqpoint{4.237117in}{3.107962in}}%
\pgfpathlineto{\pgfqpoint{4.229595in}{3.097310in}}%
\pgfpathlineto{\pgfqpoint{4.216461in}{3.103590in}}%
\pgfpathlineto{\pgfqpoint{4.203332in}{3.109999in}}%
\pgfpathlineto{\pgfqpoint{4.190207in}{3.116535in}}%
\pgfpathlineto{\pgfqpoint{4.177087in}{3.123201in}}%
\pgfpathlineto{\pgfqpoint{4.184621in}{3.134041in}}%
\pgfpathlineto{\pgfqpoint{4.192150in}{3.144966in}}%
\pgfpathlineto{\pgfqpoint{4.199676in}{3.155981in}}%
\pgfpathlineto{\pgfqpoint{4.207197in}{3.167086in}}%
\pgfpathclose%
\pgfusepath{fill}%
\end{pgfscope}%
\begin{pgfscope}%
\pgfpathrectangle{\pgfqpoint{1.150000in}{0.150000in}}{\pgfqpoint{5.700000in}{5.700000in}}%
\pgfusepath{clip}%
\pgfsetbuttcap%
\pgfsetroundjoin%
\definecolor{currentfill}{rgb}{0.275191,0.194905,0.496005}%
\pgfsetfillcolor{currentfill}%
\pgfsetfillopacity{0.700000}%
\pgfsetlinewidth{0.000000pt}%
\definecolor{currentstroke}{rgb}{0.000000,0.000000,0.000000}%
\pgfsetstrokecolor{currentstroke}%
\pgfsetdash{}{0pt}%
\pgfpathmoveto{\pgfqpoint{3.802349in}{3.232111in}}%
\pgfpathlineto{\pgfqpoint{3.815414in}{3.222427in}}%
\pgfpathlineto{\pgfqpoint{3.828480in}{3.212894in}}%
\pgfpathlineto{\pgfqpoint{3.841549in}{3.203511in}}%
\pgfpathlineto{\pgfqpoint{3.854620in}{3.194277in}}%
\pgfpathlineto{\pgfqpoint{3.846993in}{3.183377in}}%
\pgfpathlineto{\pgfqpoint{3.839361in}{3.172565in}}%
\pgfpathlineto{\pgfqpoint{3.831724in}{3.161837in}}%
\pgfpathlineto{\pgfqpoint{3.824082in}{3.151193in}}%
\pgfpathlineto{\pgfqpoint{3.811000in}{3.160286in}}%
\pgfpathlineto{\pgfqpoint{3.797921in}{3.169528in}}%
\pgfpathlineto{\pgfqpoint{3.784843in}{3.178920in}}%
\pgfpathlineto{\pgfqpoint{3.771767in}{3.188464in}}%
\pgfpathlineto{\pgfqpoint{3.779420in}{3.199242in}}%
\pgfpathlineto{\pgfqpoint{3.787068in}{3.210108in}}%
\pgfpathlineto{\pgfqpoint{3.794711in}{3.221064in}}%
\pgfpathlineto{\pgfqpoint{3.802349in}{3.232111in}}%
\pgfpathclose%
\pgfusepath{fill}%
\end{pgfscope}%
\begin{pgfscope}%
\pgfpathrectangle{\pgfqpoint{1.150000in}{0.150000in}}{\pgfqpoint{5.700000in}{5.700000in}}%
\pgfusepath{clip}%
\pgfsetbuttcap%
\pgfsetroundjoin%
\definecolor{currentfill}{rgb}{0.263663,0.237631,0.518762}%
\pgfsetfillcolor{currentfill}%
\pgfsetfillopacity{0.700000}%
\pgfsetlinewidth{0.000000pt}%
\definecolor{currentstroke}{rgb}{0.000000,0.000000,0.000000}%
\pgfsetstrokecolor{currentstroke}%
\pgfsetdash{}{0pt}%
\pgfpathmoveto{\pgfqpoint{3.614966in}{3.315186in}}%
\pgfpathlineto{\pgfqpoint{3.628028in}{3.303738in}}%
\pgfpathlineto{\pgfqpoint{3.641091in}{3.292456in}}%
\pgfpathlineto{\pgfqpoint{3.654154in}{3.281338in}}%
\pgfpathlineto{\pgfqpoint{3.667218in}{3.270383in}}%
\pgfpathlineto{\pgfqpoint{3.659537in}{3.259431in}}%
\pgfpathlineto{\pgfqpoint{3.651852in}{3.248573in}}%
\pgfpathlineto{\pgfqpoint{3.644162in}{3.237806in}}%
\pgfpathlineto{\pgfqpoint{3.636466in}{3.227130in}}%
\pgfpathlineto{\pgfqpoint{3.623391in}{3.237962in}}%
\pgfpathlineto{\pgfqpoint{3.610316in}{3.248957in}}%
\pgfpathlineto{\pgfqpoint{3.597242in}{3.260116in}}%
\pgfpathlineto{\pgfqpoint{3.584169in}{3.271442in}}%
\pgfpathlineto{\pgfqpoint{3.591876in}{3.282235in}}%
\pgfpathlineto{\pgfqpoint{3.599578in}{3.293122in}}%
\pgfpathlineto{\pgfqpoint{3.607275in}{3.304105in}}%
\pgfpathlineto{\pgfqpoint{3.614966in}{3.315186in}}%
\pgfpathclose%
\pgfusepath{fill}%
\end{pgfscope}%
\begin{pgfscope}%
\pgfpathrectangle{\pgfqpoint{1.150000in}{0.150000in}}{\pgfqpoint{5.700000in}{5.700000in}}%
\pgfusepath{clip}%
\pgfsetbuttcap%
\pgfsetroundjoin%
\definecolor{currentfill}{rgb}{0.279574,0.170599,0.479997}%
\pgfsetfillcolor{currentfill}%
\pgfsetfillopacity{0.700000}%
\pgfsetlinewidth{0.000000pt}%
\definecolor{currentstroke}{rgb}{0.000000,0.000000,0.000000}%
\pgfsetstrokecolor{currentstroke}%
\pgfsetdash{}{0pt}%
\pgfpathmoveto{\pgfqpoint{3.989655in}{3.168878in}}%
\pgfpathlineto{\pgfqpoint{4.002739in}{3.160779in}}%
\pgfpathlineto{\pgfqpoint{4.015827in}{3.152818in}}%
\pgfpathlineto{\pgfqpoint{4.028919in}{3.144995in}}%
\pgfpathlineto{\pgfqpoint{4.042014in}{3.137310in}}%
\pgfpathlineto{\pgfqpoint{4.034437in}{3.126537in}}%
\pgfpathlineto{\pgfqpoint{4.026856in}{3.115846in}}%
\pgfpathlineto{\pgfqpoint{4.019271in}{3.105235in}}%
\pgfpathlineto{\pgfqpoint{4.011681in}{3.094702in}}%
\pgfpathlineto{\pgfqpoint{3.998575in}{3.102229in}}%
\pgfpathlineto{\pgfqpoint{3.985473in}{3.109893in}}%
\pgfpathlineto{\pgfqpoint{3.972374in}{3.117695in}}%
\pgfpathlineto{\pgfqpoint{3.959278in}{3.125636in}}%
\pgfpathlineto{\pgfqpoint{3.966879in}{3.136321in}}%
\pgfpathlineto{\pgfqpoint{3.974476in}{3.147089in}}%
\pgfpathlineto{\pgfqpoint{3.982067in}{3.157940in}}%
\pgfpathlineto{\pgfqpoint{3.989655in}{3.168878in}}%
\pgfpathclose%
\pgfusepath{fill}%
\end{pgfscope}%
\begin{pgfscope}%
\pgfpathrectangle{\pgfqpoint{1.150000in}{0.150000in}}{\pgfqpoint{5.700000in}{5.700000in}}%
\pgfusepath{clip}%
\pgfsetbuttcap%
\pgfsetroundjoin%
\definecolor{currentfill}{rgb}{0.280868,0.160771,0.472899}%
\pgfsetfillcolor{currentfill}%
\pgfsetfillopacity{0.700000}%
\pgfsetlinewidth{0.000000pt}%
\definecolor{currentstroke}{rgb}{0.000000,0.000000,0.000000}%
\pgfsetstrokecolor{currentstroke}%
\pgfsetdash{}{0pt}%
\pgfpathmoveto{\pgfqpoint{4.124650in}{3.151168in}}%
\pgfpathlineto{\pgfqpoint{4.137753in}{3.143979in}}%
\pgfpathlineto{\pgfqpoint{4.150860in}{3.136922in}}%
\pgfpathlineto{\pgfqpoint{4.163972in}{3.129996in}}%
\pgfpathlineto{\pgfqpoint{4.177087in}{3.123201in}}%
\pgfpathlineto{\pgfqpoint{4.169549in}{3.112445in}}%
\pgfpathlineto{\pgfqpoint{4.162008in}{3.101771in}}%
\pgfpathlineto{\pgfqpoint{4.154461in}{3.091176in}}%
\pgfpathlineto{\pgfqpoint{4.146911in}{3.080659in}}%
\pgfpathlineto{\pgfqpoint{4.133784in}{3.087278in}}%
\pgfpathlineto{\pgfqpoint{4.120662in}{3.094027in}}%
\pgfpathlineto{\pgfqpoint{4.107544in}{3.100908in}}%
\pgfpathlineto{\pgfqpoint{4.094430in}{3.107921in}}%
\pgfpathlineto{\pgfqpoint{4.101992in}{3.118608in}}%
\pgfpathlineto{\pgfqpoint{4.109549in}{3.129377in}}%
\pgfpathlineto{\pgfqpoint{4.117101in}{3.140229in}}%
\pgfpathlineto{\pgfqpoint{4.124650in}{3.151168in}}%
\pgfpathclose%
\pgfusepath{fill}%
\end{pgfscope}%
\begin{pgfscope}%
\pgfpathrectangle{\pgfqpoint{1.150000in}{0.150000in}}{\pgfqpoint{5.700000in}{5.700000in}}%
\pgfusepath{clip}%
\pgfsetbuttcap%
\pgfsetroundjoin%
\definecolor{currentfill}{rgb}{0.269308,0.218818,0.509577}%
\pgfsetfillcolor{currentfill}%
\pgfsetfillopacity{0.700000}%
\pgfsetlinewidth{0.000000pt}%
\definecolor{currentstroke}{rgb}{0.000000,0.000000,0.000000}%
\pgfsetstrokecolor{currentstroke}%
\pgfsetdash{}{0pt}%
\pgfpathmoveto{\pgfqpoint{3.667218in}{3.270383in}}%
\pgfpathlineto{\pgfqpoint{3.680282in}{3.259590in}}%
\pgfpathlineto{\pgfqpoint{3.693348in}{3.248957in}}%
\pgfpathlineto{\pgfqpoint{3.706414in}{3.238484in}}%
\pgfpathlineto{\pgfqpoint{3.719482in}{3.228169in}}%
\pgfpathlineto{\pgfqpoint{3.711813in}{3.217347in}}%
\pgfpathlineto{\pgfqpoint{3.704139in}{3.206614in}}%
\pgfpathlineto{\pgfqpoint{3.696460in}{3.195968in}}%
\pgfpathlineto{\pgfqpoint{3.688776in}{3.185407in}}%
\pgfpathlineto{\pgfqpoint{3.675697in}{3.195600in}}%
\pgfpathlineto{\pgfqpoint{3.662619in}{3.205950in}}%
\pgfpathlineto{\pgfqpoint{3.649542in}{3.216460in}}%
\pgfpathlineto{\pgfqpoint{3.636466in}{3.227130in}}%
\pgfpathlineto{\pgfqpoint{3.644162in}{3.237806in}}%
\pgfpathlineto{\pgfqpoint{3.651852in}{3.248573in}}%
\pgfpathlineto{\pgfqpoint{3.659537in}{3.259431in}}%
\pgfpathlineto{\pgfqpoint{3.667218in}{3.270383in}}%
\pgfpathclose%
\pgfusepath{fill}%
\end{pgfscope}%
\begin{pgfscope}%
\pgfpathrectangle{\pgfqpoint{1.150000in}{0.150000in}}{\pgfqpoint{5.700000in}{5.700000in}}%
\pgfusepath{clip}%
\pgfsetbuttcap%
\pgfsetroundjoin%
\definecolor{currentfill}{rgb}{0.235526,0.309527,0.542944}%
\pgfsetfillcolor{currentfill}%
\pgfsetfillopacity{0.700000}%
\pgfsetlinewidth{0.000000pt}%
\definecolor{currentstroke}{rgb}{0.000000,0.000000,0.000000}%
\pgfsetstrokecolor{currentstroke}%
\pgfsetdash{}{0pt}%
\pgfpathmoveto{\pgfqpoint{3.374931in}{3.476396in}}%
\pgfpathlineto{\pgfqpoint{3.388018in}{3.462213in}}%
\pgfpathlineto{\pgfqpoint{3.401102in}{3.448222in}}%
\pgfpathlineto{\pgfqpoint{3.414185in}{3.434419in}}%
\pgfpathlineto{\pgfqpoint{3.427266in}{3.420804in}}%
\pgfpathlineto{\pgfqpoint{3.419517in}{3.409769in}}%
\pgfpathlineto{\pgfqpoint{3.411763in}{3.398838in}}%
\pgfpathlineto{\pgfqpoint{3.404002in}{3.388011in}}%
\pgfpathlineto{\pgfqpoint{3.396236in}{3.377285in}}%
\pgfpathlineto{\pgfqpoint{3.383142in}{3.390795in}}%
\pgfpathlineto{\pgfqpoint{3.370047in}{3.404492in}}%
\pgfpathlineto{\pgfqpoint{3.356950in}{3.418379in}}%
\pgfpathlineto{\pgfqpoint{3.343851in}{3.432456in}}%
\pgfpathlineto{\pgfqpoint{3.351630in}{3.443280in}}%
\pgfpathlineto{\pgfqpoint{3.359403in}{3.454211in}}%
\pgfpathlineto{\pgfqpoint{3.367170in}{3.465249in}}%
\pgfpathlineto{\pgfqpoint{3.374931in}{3.476396in}}%
\pgfpathclose%
\pgfusepath{fill}%
\end{pgfscope}%
\begin{pgfscope}%
\pgfpathrectangle{\pgfqpoint{1.150000in}{0.150000in}}{\pgfqpoint{5.700000in}{5.700000in}}%
\pgfusepath{clip}%
\pgfsetbuttcap%
\pgfsetroundjoin%
\definecolor{currentfill}{rgb}{0.277134,0.185228,0.489898}%
\pgfsetfillcolor{currentfill}%
\pgfsetfillopacity{0.700000}%
\pgfsetlinewidth{0.000000pt}%
\definecolor{currentstroke}{rgb}{0.000000,0.000000,0.000000}%
\pgfsetstrokecolor{currentstroke}%
\pgfsetdash{}{0pt}%
\pgfpathmoveto{\pgfqpoint{3.854620in}{3.194277in}}%
\pgfpathlineto{\pgfqpoint{3.867693in}{3.185191in}}%
\pgfpathlineto{\pgfqpoint{3.880768in}{3.176251in}}%
\pgfpathlineto{\pgfqpoint{3.893847in}{3.167457in}}%
\pgfpathlineto{\pgfqpoint{3.906927in}{3.158807in}}%
\pgfpathlineto{\pgfqpoint{3.899311in}{3.148055in}}%
\pgfpathlineto{\pgfqpoint{3.891690in}{3.137385in}}%
\pgfpathlineto{\pgfqpoint{3.884064in}{3.126796in}}%
\pgfpathlineto{\pgfqpoint{3.876433in}{3.116286in}}%
\pgfpathlineto{\pgfqpoint{3.863342in}{3.124795in}}%
\pgfpathlineto{\pgfqpoint{3.850253in}{3.133448in}}%
\pgfpathlineto{\pgfqpoint{3.837166in}{3.142247in}}%
\pgfpathlineto{\pgfqpoint{3.824082in}{3.151193in}}%
\pgfpathlineto{\pgfqpoint{3.831724in}{3.161837in}}%
\pgfpathlineto{\pgfqpoint{3.839361in}{3.172565in}}%
\pgfpathlineto{\pgfqpoint{3.846993in}{3.183377in}}%
\pgfpathlineto{\pgfqpoint{3.854620in}{3.194277in}}%
\pgfpathclose%
\pgfusepath{fill}%
\end{pgfscope}%
\begin{pgfscope}%
\pgfpathrectangle{\pgfqpoint{1.150000in}{0.150000in}}{\pgfqpoint{5.700000in}{5.700000in}}%
\pgfusepath{clip}%
\pgfsetbuttcap%
\pgfsetroundjoin%
\definecolor{currentfill}{rgb}{0.280868,0.160771,0.472899}%
\pgfsetfillcolor{currentfill}%
\pgfsetfillopacity{0.700000}%
\pgfsetlinewidth{0.000000pt}%
\definecolor{currentstroke}{rgb}{0.000000,0.000000,0.000000}%
\pgfsetstrokecolor{currentstroke}%
\pgfsetdash{}{0pt}%
\pgfpathmoveto{\pgfqpoint{4.259661in}{3.140420in}}%
\pgfpathlineto{\pgfqpoint{4.272789in}{3.134072in}}%
\pgfpathlineto{\pgfqpoint{4.285921in}{3.127851in}}%
\pgfpathlineto{\pgfqpoint{4.299059in}{3.121754in}}%
\pgfpathlineto{\pgfqpoint{4.312201in}{3.115783in}}%
\pgfpathlineto{\pgfqpoint{4.304701in}{3.105078in}}%
\pgfpathlineto{\pgfqpoint{4.297198in}{3.094456in}}%
\pgfpathlineto{\pgfqpoint{4.289690in}{3.083913in}}%
\pgfpathlineto{\pgfqpoint{4.282178in}{3.073448in}}%
\pgfpathlineto{\pgfqpoint{4.269025in}{3.079226in}}%
\pgfpathlineto{\pgfqpoint{4.255877in}{3.085128in}}%
\pgfpathlineto{\pgfqpoint{4.242733in}{3.091156in}}%
\pgfpathlineto{\pgfqpoint{4.229595in}{3.097310in}}%
\pgfpathlineto{\pgfqpoint{4.237117in}{3.107962in}}%
\pgfpathlineto{\pgfqpoint{4.244636in}{3.118697in}}%
\pgfpathlineto{\pgfqpoint{4.252151in}{3.129515in}}%
\pgfpathlineto{\pgfqpoint{4.259661in}{3.140420in}}%
\pgfpathclose%
\pgfusepath{fill}%
\end{pgfscope}%
\begin{pgfscope}%
\pgfpathrectangle{\pgfqpoint{1.150000in}{0.150000in}}{\pgfqpoint{5.700000in}{5.700000in}}%
\pgfusepath{clip}%
\pgfsetbuttcap%
\pgfsetroundjoin%
\definecolor{currentfill}{rgb}{0.244972,0.287675,0.537260}%
\pgfsetfillcolor{currentfill}%
\pgfsetfillopacity{0.700000}%
\pgfsetlinewidth{0.000000pt}%
\definecolor{currentstroke}{rgb}{0.000000,0.000000,0.000000}%
\pgfsetstrokecolor{currentstroke}%
\pgfsetdash{}{0pt}%
\pgfpathmoveto{\pgfqpoint{3.427266in}{3.420804in}}%
\pgfpathlineto{\pgfqpoint{3.440346in}{3.407375in}}%
\pgfpathlineto{\pgfqpoint{3.453424in}{3.394129in}}%
\pgfpathlineto{\pgfqpoint{3.466502in}{3.381066in}}%
\pgfpathlineto{\pgfqpoint{3.479578in}{3.368184in}}%
\pgfpathlineto{\pgfqpoint{3.471841in}{3.357260in}}%
\pgfpathlineto{\pgfqpoint{3.464099in}{3.346437in}}%
\pgfpathlineto{\pgfqpoint{3.456351in}{3.335712in}}%
\pgfpathlineto{\pgfqpoint{3.448597in}{3.325084in}}%
\pgfpathlineto{\pgfqpoint{3.435509in}{3.337862in}}%
\pgfpathlineto{\pgfqpoint{3.422419in}{3.350820in}}%
\pgfpathlineto{\pgfqpoint{3.409328in}{3.363960in}}%
\pgfpathlineto{\pgfqpoint{3.396236in}{3.377285in}}%
\pgfpathlineto{\pgfqpoint{3.404002in}{3.388011in}}%
\pgfpathlineto{\pgfqpoint{3.411763in}{3.398838in}}%
\pgfpathlineto{\pgfqpoint{3.419517in}{3.409769in}}%
\pgfpathlineto{\pgfqpoint{3.427266in}{3.420804in}}%
\pgfpathclose%
\pgfusepath{fill}%
\end{pgfscope}%
\begin{pgfscope}%
\pgfpathrectangle{\pgfqpoint{1.150000in}{0.150000in}}{\pgfqpoint{5.700000in}{5.700000in}}%
\pgfusepath{clip}%
\pgfsetbuttcap%
\pgfsetroundjoin%
\definecolor{currentfill}{rgb}{0.253935,0.265254,0.529983}%
\pgfsetfillcolor{currentfill}%
\pgfsetfillopacity{0.700000}%
\pgfsetlinewidth{0.000000pt}%
\definecolor{currentstroke}{rgb}{0.000000,0.000000,0.000000}%
\pgfsetstrokecolor{currentstroke}%
\pgfsetdash{}{0pt}%
\pgfpathmoveto{\pgfqpoint{3.479578in}{3.368184in}}%
\pgfpathlineto{\pgfqpoint{3.492653in}{3.355480in}}%
\pgfpathlineto{\pgfqpoint{3.505728in}{3.342955in}}%
\pgfpathlineto{\pgfqpoint{3.518802in}{3.330605in}}%
\pgfpathlineto{\pgfqpoint{3.531876in}{3.318429in}}%
\pgfpathlineto{\pgfqpoint{3.524151in}{3.307617in}}%
\pgfpathlineto{\pgfqpoint{3.516421in}{3.296901in}}%
\pgfpathlineto{\pgfqpoint{3.508685in}{3.286279in}}%
\pgfpathlineto{\pgfqpoint{3.500944in}{3.275749in}}%
\pgfpathlineto{\pgfqpoint{3.487858in}{3.287820in}}%
\pgfpathlineto{\pgfqpoint{3.474772in}{3.300065in}}%
\pgfpathlineto{\pgfqpoint{3.461685in}{3.312486in}}%
\pgfpathlineto{\pgfqpoint{3.448597in}{3.325084in}}%
\pgfpathlineto{\pgfqpoint{3.456351in}{3.335712in}}%
\pgfpathlineto{\pgfqpoint{3.464099in}{3.346437in}}%
\pgfpathlineto{\pgfqpoint{3.471841in}{3.357260in}}%
\pgfpathlineto{\pgfqpoint{3.479578in}{3.368184in}}%
\pgfpathclose%
\pgfusepath{fill}%
\end{pgfscope}%
\begin{pgfscope}%
\pgfpathrectangle{\pgfqpoint{1.150000in}{0.150000in}}{\pgfqpoint{5.700000in}{5.700000in}}%
\pgfusepath{clip}%
\pgfsetbuttcap%
\pgfsetroundjoin%
\definecolor{currentfill}{rgb}{0.274128,0.199721,0.498911}%
\pgfsetfillcolor{currentfill}%
\pgfsetfillopacity{0.700000}%
\pgfsetlinewidth{0.000000pt}%
\definecolor{currentstroke}{rgb}{0.000000,0.000000,0.000000}%
\pgfsetstrokecolor{currentstroke}%
\pgfsetdash{}{0pt}%
\pgfpathmoveto{\pgfqpoint{3.719482in}{3.228169in}}%
\pgfpathlineto{\pgfqpoint{3.732551in}{3.218010in}}%
\pgfpathlineto{\pgfqpoint{3.745622in}{3.208007in}}%
\pgfpathlineto{\pgfqpoint{3.758694in}{3.198159in}}%
\pgfpathlineto{\pgfqpoint{3.771767in}{3.188464in}}%
\pgfpathlineto{\pgfqpoint{3.764110in}{3.177771in}}%
\pgfpathlineto{\pgfqpoint{3.756447in}{3.167163in}}%
\pgfpathlineto{\pgfqpoint{3.748779in}{3.156638in}}%
\pgfpathlineto{\pgfqpoint{3.741107in}{3.146194in}}%
\pgfpathlineto{\pgfqpoint{3.728022in}{3.155766in}}%
\pgfpathlineto{\pgfqpoint{3.714938in}{3.165492in}}%
\pgfpathlineto{\pgfqpoint{3.701856in}{3.175372in}}%
\pgfpathlineto{\pgfqpoint{3.688776in}{3.185407in}}%
\pgfpathlineto{\pgfqpoint{3.696460in}{3.195968in}}%
\pgfpathlineto{\pgfqpoint{3.704139in}{3.206614in}}%
\pgfpathlineto{\pgfqpoint{3.711813in}{3.217347in}}%
\pgfpathlineto{\pgfqpoint{3.719482in}{3.228169in}}%
\pgfpathclose%
\pgfusepath{fill}%
\end{pgfscope}%
\begin{pgfscope}%
\pgfpathrectangle{\pgfqpoint{1.150000in}{0.150000in}}{\pgfqpoint{5.700000in}{5.700000in}}%
\pgfusepath{clip}%
\pgfsetbuttcap%
\pgfsetroundjoin%
\definecolor{currentfill}{rgb}{0.280868,0.160771,0.472899}%
\pgfsetfillcolor{currentfill}%
\pgfsetfillopacity{0.700000}%
\pgfsetlinewidth{0.000000pt}%
\definecolor{currentstroke}{rgb}{0.000000,0.000000,0.000000}%
\pgfsetstrokecolor{currentstroke}%
\pgfsetdash{}{0pt}%
\pgfpathmoveto{\pgfqpoint{4.042014in}{3.137310in}}%
\pgfpathlineto{\pgfqpoint{4.055112in}{3.129760in}}%
\pgfpathlineto{\pgfqpoint{4.068214in}{3.122346in}}%
\pgfpathlineto{\pgfqpoint{4.081320in}{3.115067in}}%
\pgfpathlineto{\pgfqpoint{4.094430in}{3.107921in}}%
\pgfpathlineto{\pgfqpoint{4.086864in}{3.097314in}}%
\pgfpathlineto{\pgfqpoint{4.079294in}{3.086783in}}%
\pgfpathlineto{\pgfqpoint{4.071720in}{3.076329in}}%
\pgfpathlineto{\pgfqpoint{4.064141in}{3.065947in}}%
\pgfpathlineto{\pgfqpoint{4.051020in}{3.072935in}}%
\pgfpathlineto{\pgfqpoint{4.037903in}{3.080055in}}%
\pgfpathlineto{\pgfqpoint{4.024790in}{3.087311in}}%
\pgfpathlineto{\pgfqpoint{4.011681in}{3.094702in}}%
\pgfpathlineto{\pgfqpoint{4.019271in}{3.105235in}}%
\pgfpathlineto{\pgfqpoint{4.026856in}{3.115846in}}%
\pgfpathlineto{\pgfqpoint{4.034437in}{3.126537in}}%
\pgfpathlineto{\pgfqpoint{4.042014in}{3.137310in}}%
\pgfpathclose%
\pgfusepath{fill}%
\end{pgfscope}%
\begin{pgfscope}%
\pgfpathrectangle{\pgfqpoint{1.150000in}{0.150000in}}{\pgfqpoint{5.700000in}{5.700000in}}%
\pgfusepath{clip}%
\pgfsetbuttcap%
\pgfsetroundjoin%
\definecolor{currentfill}{rgb}{0.260571,0.246922,0.522828}%
\pgfsetfillcolor{currentfill}%
\pgfsetfillopacity{0.700000}%
\pgfsetlinewidth{0.000000pt}%
\definecolor{currentstroke}{rgb}{0.000000,0.000000,0.000000}%
\pgfsetstrokecolor{currentstroke}%
\pgfsetdash{}{0pt}%
\pgfpathmoveto{\pgfqpoint{3.531876in}{3.318429in}}%
\pgfpathlineto{\pgfqpoint{3.544949in}{3.306427in}}%
\pgfpathlineto{\pgfqpoint{3.558022in}{3.294595in}}%
\pgfpathlineto{\pgfqpoint{3.571095in}{3.282934in}}%
\pgfpathlineto{\pgfqpoint{3.584169in}{3.271442in}}%
\pgfpathlineto{\pgfqpoint{3.576456in}{3.260741in}}%
\pgfpathlineto{\pgfqpoint{3.568738in}{3.250132in}}%
\pgfpathlineto{\pgfqpoint{3.561014in}{3.239613in}}%
\pgfpathlineto{\pgfqpoint{3.553285in}{3.229182in}}%
\pgfpathlineto{\pgfqpoint{3.540200in}{3.240570in}}%
\pgfpathlineto{\pgfqpoint{3.527114in}{3.252126in}}%
\pgfpathlineto{\pgfqpoint{3.514029in}{3.263852in}}%
\pgfpathlineto{\pgfqpoint{3.500944in}{3.275749in}}%
\pgfpathlineto{\pgfqpoint{3.508685in}{3.286279in}}%
\pgfpathlineto{\pgfqpoint{3.516421in}{3.296901in}}%
\pgfpathlineto{\pgfqpoint{3.524151in}{3.307617in}}%
\pgfpathlineto{\pgfqpoint{3.531876in}{3.318429in}}%
\pgfpathclose%
\pgfusepath{fill}%
\end{pgfscope}%
\begin{pgfscope}%
\pgfpathrectangle{\pgfqpoint{1.150000in}{0.150000in}}{\pgfqpoint{5.700000in}{5.700000in}}%
\pgfusepath{clip}%
\pgfsetbuttcap%
\pgfsetroundjoin%
\definecolor{currentfill}{rgb}{0.279574,0.170599,0.479997}%
\pgfsetfillcolor{currentfill}%
\pgfsetfillopacity{0.700000}%
\pgfsetlinewidth{0.000000pt}%
\definecolor{currentstroke}{rgb}{0.000000,0.000000,0.000000}%
\pgfsetstrokecolor{currentstroke}%
\pgfsetdash{}{0pt}%
\pgfpathmoveto{\pgfqpoint{3.906927in}{3.158807in}}%
\pgfpathlineto{\pgfqpoint{3.920011in}{3.150302in}}%
\pgfpathlineto{\pgfqpoint{3.933097in}{3.141938in}}%
\pgfpathlineto{\pgfqpoint{3.946186in}{3.133717in}}%
\pgfpathlineto{\pgfqpoint{3.959278in}{3.125636in}}%
\pgfpathlineto{\pgfqpoint{3.951673in}{3.115031in}}%
\pgfpathlineto{\pgfqpoint{3.944063in}{3.104504in}}%
\pgfpathlineto{\pgfqpoint{3.936448in}{3.094054in}}%
\pgfpathlineto{\pgfqpoint{3.928828in}{3.083678in}}%
\pgfpathlineto{\pgfqpoint{3.915725in}{3.091618in}}%
\pgfpathlineto{\pgfqpoint{3.902625in}{3.099698in}}%
\pgfpathlineto{\pgfqpoint{3.889528in}{3.107921in}}%
\pgfpathlineto{\pgfqpoint{3.876433in}{3.116286in}}%
\pgfpathlineto{\pgfqpoint{3.884064in}{3.126796in}}%
\pgfpathlineto{\pgfqpoint{3.891690in}{3.137385in}}%
\pgfpathlineto{\pgfqpoint{3.899311in}{3.148055in}}%
\pgfpathlineto{\pgfqpoint{3.906927in}{3.158807in}}%
\pgfpathclose%
\pgfusepath{fill}%
\end{pgfscope}%
\begin{pgfscope}%
\pgfpathrectangle{\pgfqpoint{1.150000in}{0.150000in}}{\pgfqpoint{5.700000in}{5.700000in}}%
\pgfusepath{clip}%
\pgfsetbuttcap%
\pgfsetroundjoin%
\definecolor{currentfill}{rgb}{0.281412,0.155834,0.469201}%
\pgfsetfillcolor{currentfill}%
\pgfsetfillopacity{0.700000}%
\pgfsetlinewidth{0.000000pt}%
\definecolor{currentstroke}{rgb}{0.000000,0.000000,0.000000}%
\pgfsetstrokecolor{currentstroke}%
\pgfsetdash{}{0pt}%
\pgfpathmoveto{\pgfqpoint{4.177087in}{3.123201in}}%
\pgfpathlineto{\pgfqpoint{4.190207in}{3.116535in}}%
\pgfpathlineto{\pgfqpoint{4.203332in}{3.109999in}}%
\pgfpathlineto{\pgfqpoint{4.216461in}{3.103590in}}%
\pgfpathlineto{\pgfqpoint{4.229595in}{3.097310in}}%
\pgfpathlineto{\pgfqpoint{4.222068in}{3.086737in}}%
\pgfpathlineto{\pgfqpoint{4.214537in}{3.076241in}}%
\pgfpathlineto{\pgfqpoint{4.207001in}{3.065821in}}%
\pgfpathlineto{\pgfqpoint{4.199462in}{3.055473in}}%
\pgfpathlineto{\pgfqpoint{4.186317in}{3.061578in}}%
\pgfpathlineto{\pgfqpoint{4.173177in}{3.067810in}}%
\pgfpathlineto{\pgfqpoint{4.160042in}{3.074170in}}%
\pgfpathlineto{\pgfqpoint{4.146911in}{3.080659in}}%
\pgfpathlineto{\pgfqpoint{4.154461in}{3.091176in}}%
\pgfpathlineto{\pgfqpoint{4.162008in}{3.101771in}}%
\pgfpathlineto{\pgfqpoint{4.169549in}{3.112445in}}%
\pgfpathlineto{\pgfqpoint{4.177087in}{3.123201in}}%
\pgfpathclose%
\pgfusepath{fill}%
\end{pgfscope}%
\begin{pgfscope}%
\pgfpathrectangle{\pgfqpoint{1.150000in}{0.150000in}}{\pgfqpoint{5.700000in}{5.700000in}}%
\pgfusepath{clip}%
\pgfsetbuttcap%
\pgfsetroundjoin%
\definecolor{currentfill}{rgb}{0.281887,0.150881,0.465405}%
\pgfsetfillcolor{currentfill}%
\pgfsetfillopacity{0.700000}%
\pgfsetlinewidth{0.000000pt}%
\definecolor{currentstroke}{rgb}{0.000000,0.000000,0.000000}%
\pgfsetstrokecolor{currentstroke}%
\pgfsetdash{}{0pt}%
\pgfpathmoveto{\pgfqpoint{4.312201in}{3.115783in}}%
\pgfpathlineto{\pgfqpoint{4.325348in}{3.109935in}}%
\pgfpathlineto{\pgfqpoint{4.338501in}{3.104211in}}%
\pgfpathlineto{\pgfqpoint{4.351659in}{3.098610in}}%
\pgfpathlineto{\pgfqpoint{4.364822in}{3.093131in}}%
\pgfpathlineto{\pgfqpoint{4.357334in}{3.082627in}}%
\pgfpathlineto{\pgfqpoint{4.349841in}{3.072201in}}%
\pgfpathlineto{\pgfqpoint{4.342345in}{3.061850in}}%
\pgfpathlineto{\pgfqpoint{4.334844in}{3.051573in}}%
\pgfpathlineto{\pgfqpoint{4.321669in}{3.056858in}}%
\pgfpathlineto{\pgfqpoint{4.308500in}{3.062265in}}%
\pgfpathlineto{\pgfqpoint{4.295337in}{3.067795in}}%
\pgfpathlineto{\pgfqpoint{4.282178in}{3.073448in}}%
\pgfpathlineto{\pgfqpoint{4.289690in}{3.083913in}}%
\pgfpathlineto{\pgfqpoint{4.297198in}{3.094456in}}%
\pgfpathlineto{\pgfqpoint{4.304701in}{3.105078in}}%
\pgfpathlineto{\pgfqpoint{4.312201in}{3.115783in}}%
\pgfpathclose%
\pgfusepath{fill}%
\end{pgfscope}%
\begin{pgfscope}%
\pgfpathrectangle{\pgfqpoint{1.150000in}{0.150000in}}{\pgfqpoint{5.700000in}{5.700000in}}%
\pgfusepath{clip}%
\pgfsetbuttcap%
\pgfsetroundjoin%
\definecolor{currentfill}{rgb}{0.277134,0.185228,0.489898}%
\pgfsetfillcolor{currentfill}%
\pgfsetfillopacity{0.700000}%
\pgfsetlinewidth{0.000000pt}%
\definecolor{currentstroke}{rgb}{0.000000,0.000000,0.000000}%
\pgfsetstrokecolor{currentstroke}%
\pgfsetdash{}{0pt}%
\pgfpathmoveto{\pgfqpoint{3.771767in}{3.188464in}}%
\pgfpathlineto{\pgfqpoint{3.784843in}{3.178920in}}%
\pgfpathlineto{\pgfqpoint{3.797921in}{3.169528in}}%
\pgfpathlineto{\pgfqpoint{3.811000in}{3.160286in}}%
\pgfpathlineto{\pgfqpoint{3.824082in}{3.151193in}}%
\pgfpathlineto{\pgfqpoint{3.816436in}{3.140630in}}%
\pgfpathlineto{\pgfqpoint{3.808784in}{3.130148in}}%
\pgfpathlineto{\pgfqpoint{3.801128in}{3.119743in}}%
\pgfpathlineto{\pgfqpoint{3.793466in}{3.109416in}}%
\pgfpathlineto{\pgfqpoint{3.780373in}{3.118386in}}%
\pgfpathlineto{\pgfqpoint{3.767282in}{3.127505in}}%
\pgfpathlineto{\pgfqpoint{3.754194in}{3.136774in}}%
\pgfpathlineto{\pgfqpoint{3.741107in}{3.146194in}}%
\pgfpathlineto{\pgfqpoint{3.748779in}{3.156638in}}%
\pgfpathlineto{\pgfqpoint{3.756447in}{3.167163in}}%
\pgfpathlineto{\pgfqpoint{3.764110in}{3.177771in}}%
\pgfpathlineto{\pgfqpoint{3.771767in}{3.188464in}}%
\pgfpathclose%
\pgfusepath{fill}%
\end{pgfscope}%
\begin{pgfscope}%
\pgfpathrectangle{\pgfqpoint{1.150000in}{0.150000in}}{\pgfqpoint{5.700000in}{5.700000in}}%
\pgfusepath{clip}%
\pgfsetbuttcap%
\pgfsetroundjoin%
\definecolor{currentfill}{rgb}{0.267968,0.223549,0.512008}%
\pgfsetfillcolor{currentfill}%
\pgfsetfillopacity{0.700000}%
\pgfsetlinewidth{0.000000pt}%
\definecolor{currentstroke}{rgb}{0.000000,0.000000,0.000000}%
\pgfsetstrokecolor{currentstroke}%
\pgfsetdash{}{0pt}%
\pgfpathmoveto{\pgfqpoint{3.584169in}{3.271442in}}%
\pgfpathlineto{\pgfqpoint{3.597242in}{3.260116in}}%
\pgfpathlineto{\pgfqpoint{3.610316in}{3.248957in}}%
\pgfpathlineto{\pgfqpoint{3.623391in}{3.237962in}}%
\pgfpathlineto{\pgfqpoint{3.636466in}{3.227130in}}%
\pgfpathlineto{\pgfqpoint{3.628765in}{3.216541in}}%
\pgfpathlineto{\pgfqpoint{3.621058in}{3.206040in}}%
\pgfpathlineto{\pgfqpoint{3.613347in}{3.195623in}}%
\pgfpathlineto{\pgfqpoint{3.605630in}{3.185291in}}%
\pgfpathlineto{\pgfqpoint{3.592543in}{3.196018in}}%
\pgfpathlineto{\pgfqpoint{3.579456in}{3.206908in}}%
\pgfpathlineto{\pgfqpoint{3.566370in}{3.217962in}}%
\pgfpathlineto{\pgfqpoint{3.553285in}{3.229182in}}%
\pgfpathlineto{\pgfqpoint{3.561014in}{3.239613in}}%
\pgfpathlineto{\pgfqpoint{3.568738in}{3.250132in}}%
\pgfpathlineto{\pgfqpoint{3.576456in}{3.260741in}}%
\pgfpathlineto{\pgfqpoint{3.584169in}{3.271442in}}%
\pgfpathclose%
\pgfusepath{fill}%
\end{pgfscope}%
\begin{pgfscope}%
\pgfpathrectangle{\pgfqpoint{1.150000in}{0.150000in}}{\pgfqpoint{5.700000in}{5.700000in}}%
\pgfusepath{clip}%
\pgfsetbuttcap%
\pgfsetroundjoin%
\definecolor{currentfill}{rgb}{0.280868,0.160771,0.472899}%
\pgfsetfillcolor{currentfill}%
\pgfsetfillopacity{0.700000}%
\pgfsetlinewidth{0.000000pt}%
\definecolor{currentstroke}{rgb}{0.000000,0.000000,0.000000}%
\pgfsetstrokecolor{currentstroke}%
\pgfsetdash{}{0pt}%
\pgfpathmoveto{\pgfqpoint{3.959278in}{3.125636in}}%
\pgfpathlineto{\pgfqpoint{3.972374in}{3.117695in}}%
\pgfpathlineto{\pgfqpoint{3.985473in}{3.109893in}}%
\pgfpathlineto{\pgfqpoint{3.998575in}{3.102229in}}%
\pgfpathlineto{\pgfqpoint{4.011681in}{3.094702in}}%
\pgfpathlineto{\pgfqpoint{4.004086in}{3.084244in}}%
\pgfpathlineto{\pgfqpoint{3.996487in}{3.073860in}}%
\pgfpathlineto{\pgfqpoint{3.988883in}{3.063549in}}%
\pgfpathlineto{\pgfqpoint{3.981274in}{3.053308in}}%
\pgfpathlineto{\pgfqpoint{3.968158in}{3.060694in}}%
\pgfpathlineto{\pgfqpoint{3.955045in}{3.068217in}}%
\pgfpathlineto{\pgfqpoint{3.941935in}{3.075878in}}%
\pgfpathlineto{\pgfqpoint{3.928828in}{3.083678in}}%
\pgfpathlineto{\pgfqpoint{3.936448in}{3.094054in}}%
\pgfpathlineto{\pgfqpoint{3.944063in}{3.104504in}}%
\pgfpathlineto{\pgfqpoint{3.951673in}{3.115031in}}%
\pgfpathlineto{\pgfqpoint{3.959278in}{3.125636in}}%
\pgfpathclose%
\pgfusepath{fill}%
\end{pgfscope}%
\begin{pgfscope}%
\pgfpathrectangle{\pgfqpoint{1.150000in}{0.150000in}}{\pgfqpoint{5.700000in}{5.700000in}}%
\pgfusepath{clip}%
\pgfsetbuttcap%
\pgfsetroundjoin%
\definecolor{currentfill}{rgb}{0.281887,0.150881,0.465405}%
\pgfsetfillcolor{currentfill}%
\pgfsetfillopacity{0.700000}%
\pgfsetlinewidth{0.000000pt}%
\definecolor{currentstroke}{rgb}{0.000000,0.000000,0.000000}%
\pgfsetstrokecolor{currentstroke}%
\pgfsetdash{}{0pt}%
\pgfpathmoveto{\pgfqpoint{4.094430in}{3.107921in}}%
\pgfpathlineto{\pgfqpoint{4.107544in}{3.100908in}}%
\pgfpathlineto{\pgfqpoint{4.120662in}{3.094027in}}%
\pgfpathlineto{\pgfqpoint{4.133784in}{3.087278in}}%
\pgfpathlineto{\pgfqpoint{4.146911in}{3.080659in}}%
\pgfpathlineto{\pgfqpoint{4.139356in}{3.070217in}}%
\pgfpathlineto{\pgfqpoint{4.131797in}{3.059848in}}%
\pgfpathlineto{\pgfqpoint{4.124233in}{3.049550in}}%
\pgfpathlineto{\pgfqpoint{4.116665in}{3.039321in}}%
\pgfpathlineto{\pgfqpoint{4.103527in}{3.045781in}}%
\pgfpathlineto{\pgfqpoint{4.090394in}{3.052371in}}%
\pgfpathlineto{\pgfqpoint{4.077265in}{3.059093in}}%
\pgfpathlineto{\pgfqpoint{4.064141in}{3.065947in}}%
\pgfpathlineto{\pgfqpoint{4.071720in}{3.076329in}}%
\pgfpathlineto{\pgfqpoint{4.079294in}{3.086783in}}%
\pgfpathlineto{\pgfqpoint{4.086864in}{3.097314in}}%
\pgfpathlineto{\pgfqpoint{4.094430in}{3.107921in}}%
\pgfpathclose%
\pgfusepath{fill}%
\end{pgfscope}%
\begin{pgfscope}%
\pgfpathrectangle{\pgfqpoint{1.150000in}{0.150000in}}{\pgfqpoint{5.700000in}{5.700000in}}%
\pgfusepath{clip}%
\pgfsetbuttcap%
\pgfsetroundjoin%
\definecolor{currentfill}{rgb}{0.271828,0.209303,0.504434}%
\pgfsetfillcolor{currentfill}%
\pgfsetfillopacity{0.700000}%
\pgfsetlinewidth{0.000000pt}%
\definecolor{currentstroke}{rgb}{0.000000,0.000000,0.000000}%
\pgfsetstrokecolor{currentstroke}%
\pgfsetdash{}{0pt}%
\pgfpathmoveto{\pgfqpoint{3.636466in}{3.227130in}}%
\pgfpathlineto{\pgfqpoint{3.649542in}{3.216460in}}%
\pgfpathlineto{\pgfqpoint{3.662619in}{3.205950in}}%
\pgfpathlineto{\pgfqpoint{3.675697in}{3.195600in}}%
\pgfpathlineto{\pgfqpoint{3.688776in}{3.185407in}}%
\pgfpathlineto{\pgfqpoint{3.681086in}{3.174931in}}%
\pgfpathlineto{\pgfqpoint{3.673392in}{3.164537in}}%
\pgfpathlineto{\pgfqpoint{3.665692in}{3.154224in}}%
\pgfpathlineto{\pgfqpoint{3.657987in}{3.143990in}}%
\pgfpathlineto{\pgfqpoint{3.644896in}{3.154077in}}%
\pgfpathlineto{\pgfqpoint{3.631806in}{3.164322in}}%
\pgfpathlineto{\pgfqpoint{3.618718in}{3.174726in}}%
\pgfpathlineto{\pgfqpoint{3.605630in}{3.185291in}}%
\pgfpathlineto{\pgfqpoint{3.613347in}{3.195623in}}%
\pgfpathlineto{\pgfqpoint{3.621058in}{3.206040in}}%
\pgfpathlineto{\pgfqpoint{3.628765in}{3.216541in}}%
\pgfpathlineto{\pgfqpoint{3.636466in}{3.227130in}}%
\pgfpathclose%
\pgfusepath{fill}%
\end{pgfscope}%
\begin{pgfscope}%
\pgfpathrectangle{\pgfqpoint{1.150000in}{0.150000in}}{\pgfqpoint{5.700000in}{5.700000in}}%
\pgfusepath{clip}%
\pgfsetbuttcap%
\pgfsetroundjoin%
\definecolor{currentfill}{rgb}{0.239346,0.300855,0.540844}%
\pgfsetfillcolor{currentfill}%
\pgfsetfillopacity{0.700000}%
\pgfsetlinewidth{0.000000pt}%
\definecolor{currentstroke}{rgb}{0.000000,0.000000,0.000000}%
\pgfsetstrokecolor{currentstroke}%
\pgfsetdash{}{0pt}%
\pgfpathmoveto{\pgfqpoint{3.343851in}{3.432456in}}%
\pgfpathlineto{\pgfqpoint{3.356950in}{3.418379in}}%
\pgfpathlineto{\pgfqpoint{3.370047in}{3.404492in}}%
\pgfpathlineto{\pgfqpoint{3.383142in}{3.390795in}}%
\pgfpathlineto{\pgfqpoint{3.396236in}{3.377285in}}%
\pgfpathlineto{\pgfqpoint{3.388464in}{3.366659in}}%
\pgfpathlineto{\pgfqpoint{3.380686in}{3.356132in}}%
\pgfpathlineto{\pgfqpoint{3.372902in}{3.345704in}}%
\pgfpathlineto{\pgfqpoint{3.365112in}{3.335371in}}%
\pgfpathlineto{\pgfqpoint{3.352006in}{3.348794in}}%
\pgfpathlineto{\pgfqpoint{3.338897in}{3.362405in}}%
\pgfpathlineto{\pgfqpoint{3.325787in}{3.376204in}}%
\pgfpathlineto{\pgfqpoint{3.312674in}{3.390195in}}%
\pgfpathlineto{\pgfqpoint{3.320477in}{3.400607in}}%
\pgfpathlineto{\pgfqpoint{3.328275in}{3.411121in}}%
\pgfpathlineto{\pgfqpoint{3.336066in}{3.421737in}}%
\pgfpathlineto{\pgfqpoint{3.343851in}{3.432456in}}%
\pgfpathclose%
\pgfusepath{fill}%
\end{pgfscope}%
\begin{pgfscope}%
\pgfpathrectangle{\pgfqpoint{1.150000in}{0.150000in}}{\pgfqpoint{5.700000in}{5.700000in}}%
\pgfusepath{clip}%
\pgfsetbuttcap%
\pgfsetroundjoin%
\definecolor{currentfill}{rgb}{0.279574,0.170599,0.479997}%
\pgfsetfillcolor{currentfill}%
\pgfsetfillopacity{0.700000}%
\pgfsetlinewidth{0.000000pt}%
\definecolor{currentstroke}{rgb}{0.000000,0.000000,0.000000}%
\pgfsetstrokecolor{currentstroke}%
\pgfsetdash{}{0pt}%
\pgfpathmoveto{\pgfqpoint{3.824082in}{3.151193in}}%
\pgfpathlineto{\pgfqpoint{3.837166in}{3.142247in}}%
\pgfpathlineto{\pgfqpoint{3.850253in}{3.133448in}}%
\pgfpathlineto{\pgfqpoint{3.863342in}{3.124795in}}%
\pgfpathlineto{\pgfqpoint{3.876433in}{3.116286in}}%
\pgfpathlineto{\pgfqpoint{3.868798in}{3.105853in}}%
\pgfpathlineto{\pgfqpoint{3.861158in}{3.095496in}}%
\pgfpathlineto{\pgfqpoint{3.853512in}{3.085213in}}%
\pgfpathlineto{\pgfqpoint{3.845862in}{3.075002in}}%
\pgfpathlineto{\pgfqpoint{3.832760in}{3.083388in}}%
\pgfpathlineto{\pgfqpoint{3.819659in}{3.091918in}}%
\pgfpathlineto{\pgfqpoint{3.806562in}{3.100593in}}%
\pgfpathlineto{\pgfqpoint{3.793466in}{3.109416in}}%
\pgfpathlineto{\pgfqpoint{3.801128in}{3.119743in}}%
\pgfpathlineto{\pgfqpoint{3.808784in}{3.130148in}}%
\pgfpathlineto{\pgfqpoint{3.816436in}{3.140630in}}%
\pgfpathlineto{\pgfqpoint{3.824082in}{3.151193in}}%
\pgfpathclose%
\pgfusepath{fill}%
\end{pgfscope}%
\begin{pgfscope}%
\pgfpathrectangle{\pgfqpoint{1.150000in}{0.150000in}}{\pgfqpoint{5.700000in}{5.700000in}}%
\pgfusepath{clip}%
\pgfsetbuttcap%
\pgfsetroundjoin%
\definecolor{currentfill}{rgb}{0.282290,0.145912,0.461510}%
\pgfsetfillcolor{currentfill}%
\pgfsetfillopacity{0.700000}%
\pgfsetlinewidth{0.000000pt}%
\definecolor{currentstroke}{rgb}{0.000000,0.000000,0.000000}%
\pgfsetstrokecolor{currentstroke}%
\pgfsetdash{}{0pt}%
\pgfpathmoveto{\pgfqpoint{4.229595in}{3.097310in}}%
\pgfpathlineto{\pgfqpoint{4.242733in}{3.091156in}}%
\pgfpathlineto{\pgfqpoint{4.255877in}{3.085128in}}%
\pgfpathlineto{\pgfqpoint{4.269025in}{3.079226in}}%
\pgfpathlineto{\pgfqpoint{4.282178in}{3.073448in}}%
\pgfpathlineto{\pgfqpoint{4.274662in}{3.063059in}}%
\pgfpathlineto{\pgfqpoint{4.267142in}{3.052742in}}%
\pgfpathlineto{\pgfqpoint{4.259618in}{3.042496in}}%
\pgfpathlineto{\pgfqpoint{4.252089in}{3.032319in}}%
\pgfpathlineto{\pgfqpoint{4.238925in}{3.037919in}}%
\pgfpathlineto{\pgfqpoint{4.225766in}{3.043645in}}%
\pgfpathlineto{\pgfqpoint{4.212611in}{3.049496in}}%
\pgfpathlineto{\pgfqpoint{4.199462in}{3.055473in}}%
\pgfpathlineto{\pgfqpoint{4.207001in}{3.065821in}}%
\pgfpathlineto{\pgfqpoint{4.214537in}{3.076241in}}%
\pgfpathlineto{\pgfqpoint{4.222068in}{3.086737in}}%
\pgfpathlineto{\pgfqpoint{4.229595in}{3.097310in}}%
\pgfpathclose%
\pgfusepath{fill}%
\end{pgfscope}%
\begin{pgfscope}%
\pgfpathrectangle{\pgfqpoint{1.150000in}{0.150000in}}{\pgfqpoint{5.700000in}{5.700000in}}%
\pgfusepath{clip}%
\pgfsetbuttcap%
\pgfsetroundjoin%
\definecolor{currentfill}{rgb}{0.248629,0.278775,0.534556}%
\pgfsetfillcolor{currentfill}%
\pgfsetfillopacity{0.700000}%
\pgfsetlinewidth{0.000000pt}%
\definecolor{currentstroke}{rgb}{0.000000,0.000000,0.000000}%
\pgfsetstrokecolor{currentstroke}%
\pgfsetdash{}{0pt}%
\pgfpathmoveto{\pgfqpoint{3.396236in}{3.377285in}}%
\pgfpathlineto{\pgfqpoint{3.409328in}{3.363960in}}%
\pgfpathlineto{\pgfqpoint{3.422419in}{3.350820in}}%
\pgfpathlineto{\pgfqpoint{3.435509in}{3.337862in}}%
\pgfpathlineto{\pgfqpoint{3.448597in}{3.325084in}}%
\pgfpathlineto{\pgfqpoint{3.440838in}{3.314552in}}%
\pgfpathlineto{\pgfqpoint{3.433073in}{3.304115in}}%
\pgfpathlineto{\pgfqpoint{3.425302in}{3.293771in}}%
\pgfpathlineto{\pgfqpoint{3.417525in}{3.283519in}}%
\pgfpathlineto{\pgfqpoint{3.404424in}{3.296209in}}%
\pgfpathlineto{\pgfqpoint{3.391321in}{3.309080in}}%
\pgfpathlineto{\pgfqpoint{3.378217in}{3.322134in}}%
\pgfpathlineto{\pgfqpoint{3.365112in}{3.335371in}}%
\pgfpathlineto{\pgfqpoint{3.372902in}{3.345704in}}%
\pgfpathlineto{\pgfqpoint{3.380686in}{3.356132in}}%
\pgfpathlineto{\pgfqpoint{3.388464in}{3.366659in}}%
\pgfpathlineto{\pgfqpoint{3.396236in}{3.377285in}}%
\pgfpathclose%
\pgfusepath{fill}%
\end{pgfscope}%
\begin{pgfscope}%
\pgfpathrectangle{\pgfqpoint{1.150000in}{0.150000in}}{\pgfqpoint{5.700000in}{5.700000in}}%
\pgfusepath{clip}%
\pgfsetbuttcap%
\pgfsetroundjoin%
\definecolor{currentfill}{rgb}{0.257322,0.256130,0.526563}%
\pgfsetfillcolor{currentfill}%
\pgfsetfillopacity{0.700000}%
\pgfsetlinewidth{0.000000pt}%
\definecolor{currentstroke}{rgb}{0.000000,0.000000,0.000000}%
\pgfsetstrokecolor{currentstroke}%
\pgfsetdash{}{0pt}%
\pgfpathmoveto{\pgfqpoint{3.448597in}{3.325084in}}%
\pgfpathlineto{\pgfqpoint{3.461685in}{3.312486in}}%
\pgfpathlineto{\pgfqpoint{3.474772in}{3.300065in}}%
\pgfpathlineto{\pgfqpoint{3.487858in}{3.287820in}}%
\pgfpathlineto{\pgfqpoint{3.500944in}{3.275749in}}%
\pgfpathlineto{\pgfqpoint{3.493197in}{3.265312in}}%
\pgfpathlineto{\pgfqpoint{3.485444in}{3.254964in}}%
\pgfpathlineto{\pgfqpoint{3.477686in}{3.244705in}}%
\pgfpathlineto{\pgfqpoint{3.469922in}{3.234533in}}%
\pgfpathlineto{\pgfqpoint{3.456824in}{3.246517in}}%
\pgfpathlineto{\pgfqpoint{3.443725in}{3.258674in}}%
\pgfpathlineto{\pgfqpoint{3.430625in}{3.271008in}}%
\pgfpathlineto{\pgfqpoint{3.417525in}{3.283519in}}%
\pgfpathlineto{\pgfqpoint{3.425302in}{3.293771in}}%
\pgfpathlineto{\pgfqpoint{3.433073in}{3.304115in}}%
\pgfpathlineto{\pgfqpoint{3.440838in}{3.314552in}}%
\pgfpathlineto{\pgfqpoint{3.448597in}{3.325084in}}%
\pgfpathclose%
\pgfusepath{fill}%
\end{pgfscope}%
\begin{pgfscope}%
\pgfpathrectangle{\pgfqpoint{1.150000in}{0.150000in}}{\pgfqpoint{5.700000in}{5.700000in}}%
\pgfusepath{clip}%
\pgfsetbuttcap%
\pgfsetroundjoin%
\definecolor{currentfill}{rgb}{0.281887,0.150881,0.465405}%
\pgfsetfillcolor{currentfill}%
\pgfsetfillopacity{0.700000}%
\pgfsetlinewidth{0.000000pt}%
\definecolor{currentstroke}{rgb}{0.000000,0.000000,0.000000}%
\pgfsetstrokecolor{currentstroke}%
\pgfsetdash{}{0pt}%
\pgfpathmoveto{\pgfqpoint{4.364822in}{3.093131in}}%
\pgfpathlineto{\pgfqpoint{4.377991in}{3.087774in}}%
\pgfpathlineto{\pgfqpoint{4.391165in}{3.082537in}}%
\pgfpathlineto{\pgfqpoint{4.404345in}{3.077421in}}%
\pgfpathlineto{\pgfqpoint{4.417531in}{3.072424in}}%
\pgfpathlineto{\pgfqpoint{4.410054in}{3.062121in}}%
\pgfpathlineto{\pgfqpoint{4.402572in}{3.051892in}}%
\pgfpathlineto{\pgfqpoint{4.395087in}{3.041734in}}%
\pgfpathlineto{\pgfqpoint{4.387598in}{3.031644in}}%
\pgfpathlineto{\pgfqpoint{4.374401in}{3.036446in}}%
\pgfpathlineto{\pgfqpoint{4.361209in}{3.041368in}}%
\pgfpathlineto{\pgfqpoint{4.348024in}{3.046410in}}%
\pgfpathlineto{\pgfqpoint{4.334844in}{3.051573in}}%
\pgfpathlineto{\pgfqpoint{4.342345in}{3.061850in}}%
\pgfpathlineto{\pgfqpoint{4.349841in}{3.072201in}}%
\pgfpathlineto{\pgfqpoint{4.357334in}{3.082627in}}%
\pgfpathlineto{\pgfqpoint{4.364822in}{3.093131in}}%
\pgfpathclose%
\pgfusepath{fill}%
\end{pgfscope}%
\begin{pgfscope}%
\pgfpathrectangle{\pgfqpoint{1.150000in}{0.150000in}}{\pgfqpoint{5.700000in}{5.700000in}}%
\pgfusepath{clip}%
\pgfsetbuttcap%
\pgfsetroundjoin%
\definecolor{currentfill}{rgb}{0.276194,0.190074,0.493001}%
\pgfsetfillcolor{currentfill}%
\pgfsetfillopacity{0.700000}%
\pgfsetlinewidth{0.000000pt}%
\definecolor{currentstroke}{rgb}{0.000000,0.000000,0.000000}%
\pgfsetstrokecolor{currentstroke}%
\pgfsetdash{}{0pt}%
\pgfpathmoveto{\pgfqpoint{3.688776in}{3.185407in}}%
\pgfpathlineto{\pgfqpoint{3.701856in}{3.175372in}}%
\pgfpathlineto{\pgfqpoint{3.714938in}{3.165492in}}%
\pgfpathlineto{\pgfqpoint{3.728022in}{3.155766in}}%
\pgfpathlineto{\pgfqpoint{3.741107in}{3.146194in}}%
\pgfpathlineto{\pgfqpoint{3.733429in}{3.135830in}}%
\pgfpathlineto{\pgfqpoint{3.725746in}{3.125543in}}%
\pgfpathlineto{\pgfqpoint{3.718058in}{3.115334in}}%
\pgfpathlineto{\pgfqpoint{3.710364in}{3.105199in}}%
\pgfpathlineto{\pgfqpoint{3.697268in}{3.114665in}}%
\pgfpathlineto{\pgfqpoint{3.684173in}{3.124285in}}%
\pgfpathlineto{\pgfqpoint{3.671079in}{3.134060in}}%
\pgfpathlineto{\pgfqpoint{3.657987in}{3.143990in}}%
\pgfpathlineto{\pgfqpoint{3.665692in}{3.154224in}}%
\pgfpathlineto{\pgfqpoint{3.673392in}{3.164537in}}%
\pgfpathlineto{\pgfqpoint{3.681086in}{3.174931in}}%
\pgfpathlineto{\pgfqpoint{3.688776in}{3.185407in}}%
\pgfpathclose%
\pgfusepath{fill}%
\end{pgfscope}%
\begin{pgfscope}%
\pgfpathrectangle{\pgfqpoint{1.150000in}{0.150000in}}{\pgfqpoint{5.700000in}{5.700000in}}%
\pgfusepath{clip}%
\pgfsetbuttcap%
\pgfsetroundjoin%
\definecolor{currentfill}{rgb}{0.281887,0.150881,0.465405}%
\pgfsetfillcolor{currentfill}%
\pgfsetfillopacity{0.700000}%
\pgfsetlinewidth{0.000000pt}%
\definecolor{currentstroke}{rgb}{0.000000,0.000000,0.000000}%
\pgfsetstrokecolor{currentstroke}%
\pgfsetdash{}{0pt}%
\pgfpathmoveto{\pgfqpoint{4.011681in}{3.094702in}}%
\pgfpathlineto{\pgfqpoint{4.024790in}{3.087311in}}%
\pgfpathlineto{\pgfqpoint{4.037903in}{3.080055in}}%
\pgfpathlineto{\pgfqpoint{4.051020in}{3.072935in}}%
\pgfpathlineto{\pgfqpoint{4.064141in}{3.065947in}}%
\pgfpathlineto{\pgfqpoint{4.056557in}{3.055638in}}%
\pgfpathlineto{\pgfqpoint{4.048969in}{3.045398in}}%
\pgfpathlineto{\pgfqpoint{4.041376in}{3.035225in}}%
\pgfpathlineto{\pgfqpoint{4.033778in}{3.025119in}}%
\pgfpathlineto{\pgfqpoint{4.020647in}{3.031965in}}%
\pgfpathlineto{\pgfqpoint{4.007519in}{3.038944in}}%
\pgfpathlineto{\pgfqpoint{3.994395in}{3.046058in}}%
\pgfpathlineto{\pgfqpoint{3.981274in}{3.053308in}}%
\pgfpathlineto{\pgfqpoint{3.988883in}{3.063549in}}%
\pgfpathlineto{\pgfqpoint{3.996487in}{3.073860in}}%
\pgfpathlineto{\pgfqpoint{4.004086in}{3.084244in}}%
\pgfpathlineto{\pgfqpoint{4.011681in}{3.094702in}}%
\pgfpathclose%
\pgfusepath{fill}%
\end{pgfscope}%
\begin{pgfscope}%
\pgfpathrectangle{\pgfqpoint{1.150000in}{0.150000in}}{\pgfqpoint{5.700000in}{5.700000in}}%
\pgfusepath{clip}%
\pgfsetbuttcap%
\pgfsetroundjoin%
\definecolor{currentfill}{rgb}{0.265145,0.232956,0.516599}%
\pgfsetfillcolor{currentfill}%
\pgfsetfillopacity{0.700000}%
\pgfsetlinewidth{0.000000pt}%
\definecolor{currentstroke}{rgb}{0.000000,0.000000,0.000000}%
\pgfsetstrokecolor{currentstroke}%
\pgfsetdash{}{0pt}%
\pgfpathmoveto{\pgfqpoint{3.500944in}{3.275749in}}%
\pgfpathlineto{\pgfqpoint{3.514029in}{3.263852in}}%
\pgfpathlineto{\pgfqpoint{3.527114in}{3.252126in}}%
\pgfpathlineto{\pgfqpoint{3.540200in}{3.240570in}}%
\pgfpathlineto{\pgfqpoint{3.553285in}{3.229182in}}%
\pgfpathlineto{\pgfqpoint{3.545550in}{3.218838in}}%
\pgfpathlineto{\pgfqpoint{3.537810in}{3.208580in}}%
\pgfpathlineto{\pgfqpoint{3.530065in}{3.198407in}}%
\pgfpathlineto{\pgfqpoint{3.522313in}{3.188316in}}%
\pgfpathlineto{\pgfqpoint{3.509216in}{3.199616in}}%
\pgfpathlineto{\pgfqpoint{3.496118in}{3.211085in}}%
\pgfpathlineto{\pgfqpoint{3.483020in}{3.222723in}}%
\pgfpathlineto{\pgfqpoint{3.469922in}{3.234533in}}%
\pgfpathlineto{\pgfqpoint{3.477686in}{3.244705in}}%
\pgfpathlineto{\pgfqpoint{3.485444in}{3.254964in}}%
\pgfpathlineto{\pgfqpoint{3.493197in}{3.265312in}}%
\pgfpathlineto{\pgfqpoint{3.500944in}{3.275749in}}%
\pgfpathclose%
\pgfusepath{fill}%
\end{pgfscope}%
\begin{pgfscope}%
\pgfpathrectangle{\pgfqpoint{1.150000in}{0.150000in}}{\pgfqpoint{5.700000in}{5.700000in}}%
\pgfusepath{clip}%
\pgfsetbuttcap%
\pgfsetroundjoin%
\definecolor{currentfill}{rgb}{0.280868,0.160771,0.472899}%
\pgfsetfillcolor{currentfill}%
\pgfsetfillopacity{0.700000}%
\pgfsetlinewidth{0.000000pt}%
\definecolor{currentstroke}{rgb}{0.000000,0.000000,0.000000}%
\pgfsetstrokecolor{currentstroke}%
\pgfsetdash{}{0pt}%
\pgfpathmoveto{\pgfqpoint{3.876433in}{3.116286in}}%
\pgfpathlineto{\pgfqpoint{3.889528in}{3.107921in}}%
\pgfpathlineto{\pgfqpoint{3.902625in}{3.099698in}}%
\pgfpathlineto{\pgfqpoint{3.915725in}{3.091618in}}%
\pgfpathlineto{\pgfqpoint{3.928828in}{3.083678in}}%
\pgfpathlineto{\pgfqpoint{3.921204in}{3.073375in}}%
\pgfpathlineto{\pgfqpoint{3.913575in}{3.063144in}}%
\pgfpathlineto{\pgfqpoint{3.905941in}{3.052982in}}%
\pgfpathlineto{\pgfqpoint{3.898302in}{3.042888in}}%
\pgfpathlineto{\pgfqpoint{3.885188in}{3.050705in}}%
\pgfpathlineto{\pgfqpoint{3.872076in}{3.058662in}}%
\pgfpathlineto{\pgfqpoint{3.858968in}{3.066761in}}%
\pgfpathlineto{\pgfqpoint{3.845862in}{3.075002in}}%
\pgfpathlineto{\pgfqpoint{3.853512in}{3.085213in}}%
\pgfpathlineto{\pgfqpoint{3.861158in}{3.095496in}}%
\pgfpathlineto{\pgfqpoint{3.868798in}{3.105853in}}%
\pgfpathlineto{\pgfqpoint{3.876433in}{3.116286in}}%
\pgfpathclose%
\pgfusepath{fill}%
\end{pgfscope}%
\begin{pgfscope}%
\pgfpathrectangle{\pgfqpoint{1.150000in}{0.150000in}}{\pgfqpoint{5.700000in}{5.700000in}}%
\pgfusepath{clip}%
\pgfsetbuttcap%
\pgfsetroundjoin%
\definecolor{currentfill}{rgb}{0.282290,0.145912,0.461510}%
\pgfsetfillcolor{currentfill}%
\pgfsetfillopacity{0.700000}%
\pgfsetlinewidth{0.000000pt}%
\definecolor{currentstroke}{rgb}{0.000000,0.000000,0.000000}%
\pgfsetstrokecolor{currentstroke}%
\pgfsetdash{}{0pt}%
\pgfpathmoveto{\pgfqpoint{4.146911in}{3.080659in}}%
\pgfpathlineto{\pgfqpoint{4.160042in}{3.074170in}}%
\pgfpathlineto{\pgfqpoint{4.173177in}{3.067810in}}%
\pgfpathlineto{\pgfqpoint{4.186317in}{3.061578in}}%
\pgfpathlineto{\pgfqpoint{4.199462in}{3.055473in}}%
\pgfpathlineto{\pgfqpoint{4.191918in}{3.045197in}}%
\pgfpathlineto{\pgfqpoint{4.184369in}{3.034989in}}%
\pgfpathlineto{\pgfqpoint{4.176817in}{3.024848in}}%
\pgfpathlineto{\pgfqpoint{4.169259in}{3.014772in}}%
\pgfpathlineto{\pgfqpoint{4.156104in}{3.020717in}}%
\pgfpathlineto{\pgfqpoint{4.142953in}{3.026789in}}%
\pgfpathlineto{\pgfqpoint{4.129806in}{3.032990in}}%
\pgfpathlineto{\pgfqpoint{4.116665in}{3.039321in}}%
\pgfpathlineto{\pgfqpoint{4.124233in}{3.049550in}}%
\pgfpathlineto{\pgfqpoint{4.131797in}{3.059848in}}%
\pgfpathlineto{\pgfqpoint{4.139356in}{3.070217in}}%
\pgfpathlineto{\pgfqpoint{4.146911in}{3.080659in}}%
\pgfpathclose%
\pgfusepath{fill}%
\end{pgfscope}%
\begin{pgfscope}%
\pgfpathrectangle{\pgfqpoint{1.150000in}{0.150000in}}{\pgfqpoint{5.700000in}{5.700000in}}%
\pgfusepath{clip}%
\pgfsetbuttcap%
\pgfsetroundjoin%
\definecolor{currentfill}{rgb}{0.282623,0.140926,0.457517}%
\pgfsetfillcolor{currentfill}%
\pgfsetfillopacity{0.700000}%
\pgfsetlinewidth{0.000000pt}%
\definecolor{currentstroke}{rgb}{0.000000,0.000000,0.000000}%
\pgfsetstrokecolor{currentstroke}%
\pgfsetdash{}{0pt}%
\pgfpathmoveto{\pgfqpoint{4.282178in}{3.073448in}}%
\pgfpathlineto{\pgfqpoint{4.295337in}{3.067795in}}%
\pgfpathlineto{\pgfqpoint{4.308500in}{3.062265in}}%
\pgfpathlineto{\pgfqpoint{4.321669in}{3.056858in}}%
\pgfpathlineto{\pgfqpoint{4.334844in}{3.051573in}}%
\pgfpathlineto{\pgfqpoint{4.327339in}{3.041367in}}%
\pgfpathlineto{\pgfqpoint{4.319830in}{3.031229in}}%
\pgfpathlineto{\pgfqpoint{4.312317in}{3.021158in}}%
\pgfpathlineto{\pgfqpoint{4.304800in}{3.011151in}}%
\pgfpathlineto{\pgfqpoint{4.291614in}{3.016259in}}%
\pgfpathlineto{\pgfqpoint{4.278434in}{3.021489in}}%
\pgfpathlineto{\pgfqpoint{4.265259in}{3.026842in}}%
\pgfpathlineto{\pgfqpoint{4.252089in}{3.032319in}}%
\pgfpathlineto{\pgfqpoint{4.259618in}{3.042496in}}%
\pgfpathlineto{\pgfqpoint{4.267142in}{3.052742in}}%
\pgfpathlineto{\pgfqpoint{4.274662in}{3.063059in}}%
\pgfpathlineto{\pgfqpoint{4.282178in}{3.073448in}}%
\pgfpathclose%
\pgfusepath{fill}%
\end{pgfscope}%
\begin{pgfscope}%
\pgfpathrectangle{\pgfqpoint{1.150000in}{0.150000in}}{\pgfqpoint{5.700000in}{5.700000in}}%
\pgfusepath{clip}%
\pgfsetbuttcap%
\pgfsetroundjoin%
\definecolor{currentfill}{rgb}{0.278826,0.175490,0.483397}%
\pgfsetfillcolor{currentfill}%
\pgfsetfillopacity{0.700000}%
\pgfsetlinewidth{0.000000pt}%
\definecolor{currentstroke}{rgb}{0.000000,0.000000,0.000000}%
\pgfsetstrokecolor{currentstroke}%
\pgfsetdash{}{0pt}%
\pgfpathmoveto{\pgfqpoint{3.741107in}{3.146194in}}%
\pgfpathlineto{\pgfqpoint{3.754194in}{3.136774in}}%
\pgfpathlineto{\pgfqpoint{3.767282in}{3.127505in}}%
\pgfpathlineto{\pgfqpoint{3.780373in}{3.118386in}}%
\pgfpathlineto{\pgfqpoint{3.793466in}{3.109416in}}%
\pgfpathlineto{\pgfqpoint{3.785800in}{3.099164in}}%
\pgfpathlineto{\pgfqpoint{3.778128in}{3.088985in}}%
\pgfpathlineto{\pgfqpoint{3.770452in}{3.078879in}}%
\pgfpathlineto{\pgfqpoint{3.762770in}{3.068843in}}%
\pgfpathlineto{\pgfqpoint{3.749666in}{3.077708in}}%
\pgfpathlineto{\pgfqpoint{3.736563in}{3.086721in}}%
\pgfpathlineto{\pgfqpoint{3.723463in}{3.095884in}}%
\pgfpathlineto{\pgfqpoint{3.710364in}{3.105199in}}%
\pgfpathlineto{\pgfqpoint{3.718058in}{3.115334in}}%
\pgfpathlineto{\pgfqpoint{3.725746in}{3.125543in}}%
\pgfpathlineto{\pgfqpoint{3.733429in}{3.135830in}}%
\pgfpathlineto{\pgfqpoint{3.741107in}{3.146194in}}%
\pgfpathclose%
\pgfusepath{fill}%
\end{pgfscope}%
\begin{pgfscope}%
\pgfpathrectangle{\pgfqpoint{1.150000in}{0.150000in}}{\pgfqpoint{5.700000in}{5.700000in}}%
\pgfusepath{clip}%
\pgfsetbuttcap%
\pgfsetroundjoin%
\definecolor{currentfill}{rgb}{0.270595,0.214069,0.507052}%
\pgfsetfillcolor{currentfill}%
\pgfsetfillopacity{0.700000}%
\pgfsetlinewidth{0.000000pt}%
\definecolor{currentstroke}{rgb}{0.000000,0.000000,0.000000}%
\pgfsetstrokecolor{currentstroke}%
\pgfsetdash{}{0pt}%
\pgfpathmoveto{\pgfqpoint{3.553285in}{3.229182in}}%
\pgfpathlineto{\pgfqpoint{3.566370in}{3.217962in}}%
\pgfpathlineto{\pgfqpoint{3.579456in}{3.206908in}}%
\pgfpathlineto{\pgfqpoint{3.592543in}{3.196018in}}%
\pgfpathlineto{\pgfqpoint{3.605630in}{3.185291in}}%
\pgfpathlineto{\pgfqpoint{3.597907in}{3.175041in}}%
\pgfpathlineto{\pgfqpoint{3.590180in}{3.164873in}}%
\pgfpathlineto{\pgfqpoint{3.582446in}{3.154785in}}%
\pgfpathlineto{\pgfqpoint{3.574707in}{3.144775in}}%
\pgfpathlineto{\pgfqpoint{3.561608in}{3.155414in}}%
\pgfpathlineto{\pgfqpoint{3.548510in}{3.166216in}}%
\pgfpathlineto{\pgfqpoint{3.535411in}{3.177183in}}%
\pgfpathlineto{\pgfqpoint{3.522313in}{3.188316in}}%
\pgfpathlineto{\pgfqpoint{3.530065in}{3.198407in}}%
\pgfpathlineto{\pgfqpoint{3.537810in}{3.208580in}}%
\pgfpathlineto{\pgfqpoint{3.545550in}{3.218838in}}%
\pgfpathlineto{\pgfqpoint{3.553285in}{3.229182in}}%
\pgfpathclose%
\pgfusepath{fill}%
\end{pgfscope}%
\begin{pgfscope}%
\pgfpathrectangle{\pgfqpoint{1.150000in}{0.150000in}}{\pgfqpoint{5.700000in}{5.700000in}}%
\pgfusepath{clip}%
\pgfsetbuttcap%
\pgfsetroundjoin%
\definecolor{currentfill}{rgb}{0.282290,0.145912,0.461510}%
\pgfsetfillcolor{currentfill}%
\pgfsetfillopacity{0.700000}%
\pgfsetlinewidth{0.000000pt}%
\definecolor{currentstroke}{rgb}{0.000000,0.000000,0.000000}%
\pgfsetstrokecolor{currentstroke}%
\pgfsetdash{}{0pt}%
\pgfpathmoveto{\pgfqpoint{4.417531in}{3.072424in}}%
\pgfpathlineto{\pgfqpoint{4.430722in}{3.067547in}}%
\pgfpathlineto{\pgfqpoint{4.443920in}{3.062788in}}%
\pgfpathlineto{\pgfqpoint{4.457123in}{3.058148in}}%
\pgfpathlineto{\pgfqpoint{4.470333in}{3.053625in}}%
\pgfpathlineto{\pgfqpoint{4.462867in}{3.043523in}}%
\pgfpathlineto{\pgfqpoint{4.455397in}{3.033491in}}%
\pgfpathlineto{\pgfqpoint{4.447923in}{3.023526in}}%
\pgfpathlineto{\pgfqpoint{4.440445in}{3.013625in}}%
\pgfpathlineto{\pgfqpoint{4.427224in}{3.017953in}}%
\pgfpathlineto{\pgfqpoint{4.414010in}{3.022398in}}%
\pgfpathlineto{\pgfqpoint{4.400801in}{3.026962in}}%
\pgfpathlineto{\pgfqpoint{4.387598in}{3.031644in}}%
\pgfpathlineto{\pgfqpoint{4.395087in}{3.041734in}}%
\pgfpathlineto{\pgfqpoint{4.402572in}{3.051892in}}%
\pgfpathlineto{\pgfqpoint{4.410054in}{3.062121in}}%
\pgfpathlineto{\pgfqpoint{4.417531in}{3.072424in}}%
\pgfpathclose%
\pgfusepath{fill}%
\end{pgfscope}%
\begin{pgfscope}%
\pgfpathrectangle{\pgfqpoint{1.150000in}{0.150000in}}{\pgfqpoint{5.700000in}{5.700000in}}%
\pgfusepath{clip}%
\pgfsetbuttcap%
\pgfsetroundjoin%
\definecolor{currentfill}{rgb}{0.281887,0.150881,0.465405}%
\pgfsetfillcolor{currentfill}%
\pgfsetfillopacity{0.700000}%
\pgfsetlinewidth{0.000000pt}%
\definecolor{currentstroke}{rgb}{0.000000,0.000000,0.000000}%
\pgfsetstrokecolor{currentstroke}%
\pgfsetdash{}{0pt}%
\pgfpathmoveto{\pgfqpoint{3.928828in}{3.083678in}}%
\pgfpathlineto{\pgfqpoint{3.941935in}{3.075878in}}%
\pgfpathlineto{\pgfqpoint{3.955045in}{3.068217in}}%
\pgfpathlineto{\pgfqpoint{3.968158in}{3.060694in}}%
\pgfpathlineto{\pgfqpoint{3.981274in}{3.053308in}}%
\pgfpathlineto{\pgfqpoint{3.973661in}{3.043135in}}%
\pgfpathlineto{\pgfqpoint{3.966043in}{3.033030in}}%
\pgfpathlineto{\pgfqpoint{3.958420in}{3.022990in}}%
\pgfpathlineto{\pgfqpoint{3.950793in}{3.013013in}}%
\pgfpathlineto{\pgfqpoint{3.937665in}{3.020275in}}%
\pgfpathlineto{\pgfqpoint{3.924541in}{3.027674in}}%
\pgfpathlineto{\pgfqpoint{3.911420in}{3.035212in}}%
\pgfpathlineto{\pgfqpoint{3.898302in}{3.042888in}}%
\pgfpathlineto{\pgfqpoint{3.905941in}{3.052982in}}%
\pgfpathlineto{\pgfqpoint{3.913575in}{3.063144in}}%
\pgfpathlineto{\pgfqpoint{3.921204in}{3.073375in}}%
\pgfpathlineto{\pgfqpoint{3.928828in}{3.083678in}}%
\pgfpathclose%
\pgfusepath{fill}%
\end{pgfscope}%
\begin{pgfscope}%
\pgfpathrectangle{\pgfqpoint{1.150000in}{0.150000in}}{\pgfqpoint{5.700000in}{5.700000in}}%
\pgfusepath{clip}%
\pgfsetbuttcap%
\pgfsetroundjoin%
\definecolor{currentfill}{rgb}{0.282623,0.140926,0.457517}%
\pgfsetfillcolor{currentfill}%
\pgfsetfillopacity{0.700000}%
\pgfsetlinewidth{0.000000pt}%
\definecolor{currentstroke}{rgb}{0.000000,0.000000,0.000000}%
\pgfsetstrokecolor{currentstroke}%
\pgfsetdash{}{0pt}%
\pgfpathmoveto{\pgfqpoint{4.064141in}{3.065947in}}%
\pgfpathlineto{\pgfqpoint{4.077265in}{3.059093in}}%
\pgfpathlineto{\pgfqpoint{4.090394in}{3.052371in}}%
\pgfpathlineto{\pgfqpoint{4.103527in}{3.045781in}}%
\pgfpathlineto{\pgfqpoint{4.116665in}{3.039321in}}%
\pgfpathlineto{\pgfqpoint{4.109092in}{3.029159in}}%
\pgfpathlineto{\pgfqpoint{4.101515in}{3.019063in}}%
\pgfpathlineto{\pgfqpoint{4.093933in}{3.009030in}}%
\pgfpathlineto{\pgfqpoint{4.086346in}{2.999058in}}%
\pgfpathlineto{\pgfqpoint{4.073198in}{3.005377in}}%
\pgfpathlineto{\pgfqpoint{4.060054in}{3.011826in}}%
\pgfpathlineto{\pgfqpoint{4.046914in}{3.018406in}}%
\pgfpathlineto{\pgfqpoint{4.033778in}{3.025119in}}%
\pgfpathlineto{\pgfqpoint{4.041376in}{3.035225in}}%
\pgfpathlineto{\pgfqpoint{4.048969in}{3.045398in}}%
\pgfpathlineto{\pgfqpoint{4.056557in}{3.055638in}}%
\pgfpathlineto{\pgfqpoint{4.064141in}{3.065947in}}%
\pgfpathclose%
\pgfusepath{fill}%
\end{pgfscope}%
\begin{pgfscope}%
\pgfpathrectangle{\pgfqpoint{1.150000in}{0.150000in}}{\pgfqpoint{5.700000in}{5.700000in}}%
\pgfusepath{clip}%
\pgfsetbuttcap%
\pgfsetroundjoin%
\definecolor{currentfill}{rgb}{0.243113,0.292092,0.538516}%
\pgfsetfillcolor{currentfill}%
\pgfsetfillopacity{0.700000}%
\pgfsetlinewidth{0.000000pt}%
\definecolor{currentstroke}{rgb}{0.000000,0.000000,0.000000}%
\pgfsetstrokecolor{currentstroke}%
\pgfsetdash{}{0pt}%
\pgfpathmoveto{\pgfqpoint{3.312674in}{3.390195in}}%
\pgfpathlineto{\pgfqpoint{3.325787in}{3.376204in}}%
\pgfpathlineto{\pgfqpoint{3.338897in}{3.362405in}}%
\pgfpathlineto{\pgfqpoint{3.352006in}{3.348794in}}%
\pgfpathlineto{\pgfqpoint{3.365112in}{3.335371in}}%
\pgfpathlineto{\pgfqpoint{3.357316in}{3.325134in}}%
\pgfpathlineto{\pgfqpoint{3.349515in}{3.314991in}}%
\pgfpathlineto{\pgfqpoint{3.341707in}{3.304941in}}%
\pgfpathlineto{\pgfqpoint{3.333893in}{3.294984in}}%
\pgfpathlineto{\pgfqpoint{3.320772in}{3.308337in}}%
\pgfpathlineto{\pgfqpoint{3.307650in}{3.321879in}}%
\pgfpathlineto{\pgfqpoint{3.294526in}{3.335609in}}%
\pgfpathlineto{\pgfqpoint{3.281400in}{3.349531in}}%
\pgfpathlineto{\pgfqpoint{3.289228in}{3.359551in}}%
\pgfpathlineto{\pgfqpoint{3.297049in}{3.369668in}}%
\pgfpathlineto{\pgfqpoint{3.304865in}{3.379882in}}%
\pgfpathlineto{\pgfqpoint{3.312674in}{3.390195in}}%
\pgfpathclose%
\pgfusepath{fill}%
\end{pgfscope}%
\begin{pgfscope}%
\pgfpathrectangle{\pgfqpoint{1.150000in}{0.150000in}}{\pgfqpoint{5.700000in}{5.700000in}}%
\pgfusepath{clip}%
\pgfsetbuttcap%
\pgfsetroundjoin%
\definecolor{currentfill}{rgb}{0.274128,0.199721,0.498911}%
\pgfsetfillcolor{currentfill}%
\pgfsetfillopacity{0.700000}%
\pgfsetlinewidth{0.000000pt}%
\definecolor{currentstroke}{rgb}{0.000000,0.000000,0.000000}%
\pgfsetstrokecolor{currentstroke}%
\pgfsetdash{}{0pt}%
\pgfpathmoveto{\pgfqpoint{3.605630in}{3.185291in}}%
\pgfpathlineto{\pgfqpoint{3.618718in}{3.174726in}}%
\pgfpathlineto{\pgfqpoint{3.631806in}{3.164322in}}%
\pgfpathlineto{\pgfqpoint{3.644896in}{3.154077in}}%
\pgfpathlineto{\pgfqpoint{3.657987in}{3.143990in}}%
\pgfpathlineto{\pgfqpoint{3.650276in}{3.133835in}}%
\pgfpathlineto{\pgfqpoint{3.642561in}{3.123756in}}%
\pgfpathlineto{\pgfqpoint{3.634840in}{3.113753in}}%
\pgfpathlineto{\pgfqpoint{3.627113in}{3.103825in}}%
\pgfpathlineto{\pgfqpoint{3.614010in}{3.113824in}}%
\pgfpathlineto{\pgfqpoint{3.600908in}{3.123981in}}%
\pgfpathlineto{\pgfqpoint{3.587807in}{3.134298in}}%
\pgfpathlineto{\pgfqpoint{3.574707in}{3.144775in}}%
\pgfpathlineto{\pgfqpoint{3.582446in}{3.154785in}}%
\pgfpathlineto{\pgfqpoint{3.590180in}{3.164873in}}%
\pgfpathlineto{\pgfqpoint{3.597907in}{3.175041in}}%
\pgfpathlineto{\pgfqpoint{3.605630in}{3.185291in}}%
\pgfpathclose%
\pgfusepath{fill}%
\end{pgfscope}%
\begin{pgfscope}%
\pgfpathrectangle{\pgfqpoint{1.150000in}{0.150000in}}{\pgfqpoint{5.700000in}{5.700000in}}%
\pgfusepath{clip}%
\pgfsetbuttcap%
\pgfsetroundjoin%
\definecolor{currentfill}{rgb}{0.280868,0.160771,0.472899}%
\pgfsetfillcolor{currentfill}%
\pgfsetfillopacity{0.700000}%
\pgfsetlinewidth{0.000000pt}%
\definecolor{currentstroke}{rgb}{0.000000,0.000000,0.000000}%
\pgfsetstrokecolor{currentstroke}%
\pgfsetdash{}{0pt}%
\pgfpathmoveto{\pgfqpoint{3.793466in}{3.109416in}}%
\pgfpathlineto{\pgfqpoint{3.806562in}{3.100593in}}%
\pgfpathlineto{\pgfqpoint{3.819659in}{3.091918in}}%
\pgfpathlineto{\pgfqpoint{3.832760in}{3.083388in}}%
\pgfpathlineto{\pgfqpoint{3.845862in}{3.075002in}}%
\pgfpathlineto{\pgfqpoint{3.838207in}{3.064862in}}%
\pgfpathlineto{\pgfqpoint{3.830547in}{3.054792in}}%
\pgfpathlineto{\pgfqpoint{3.822882in}{3.044789in}}%
\pgfpathlineto{\pgfqpoint{3.815212in}{3.034853in}}%
\pgfpathlineto{\pgfqpoint{3.802098in}{3.043132in}}%
\pgfpathlineto{\pgfqpoint{3.788986in}{3.051556in}}%
\pgfpathlineto{\pgfqpoint{3.775877in}{3.060126in}}%
\pgfpathlineto{\pgfqpoint{3.762770in}{3.068843in}}%
\pgfpathlineto{\pgfqpoint{3.770452in}{3.078879in}}%
\pgfpathlineto{\pgfqpoint{3.778128in}{3.088985in}}%
\pgfpathlineto{\pgfqpoint{3.785800in}{3.099164in}}%
\pgfpathlineto{\pgfqpoint{3.793466in}{3.109416in}}%
\pgfpathclose%
\pgfusepath{fill}%
\end{pgfscope}%
\begin{pgfscope}%
\pgfpathrectangle{\pgfqpoint{1.150000in}{0.150000in}}{\pgfqpoint{5.700000in}{5.700000in}}%
\pgfusepath{clip}%
\pgfsetbuttcap%
\pgfsetroundjoin%
\definecolor{currentfill}{rgb}{0.282884,0.135920,0.453427}%
\pgfsetfillcolor{currentfill}%
\pgfsetfillopacity{0.700000}%
\pgfsetlinewidth{0.000000pt}%
\definecolor{currentstroke}{rgb}{0.000000,0.000000,0.000000}%
\pgfsetstrokecolor{currentstroke}%
\pgfsetdash{}{0pt}%
\pgfpathmoveto{\pgfqpoint{4.199462in}{3.055473in}}%
\pgfpathlineto{\pgfqpoint{4.212611in}{3.049496in}}%
\pgfpathlineto{\pgfqpoint{4.225766in}{3.043645in}}%
\pgfpathlineto{\pgfqpoint{4.238925in}{3.037919in}}%
\pgfpathlineto{\pgfqpoint{4.252089in}{3.032319in}}%
\pgfpathlineto{\pgfqpoint{4.244556in}{3.022208in}}%
\pgfpathlineto{\pgfqpoint{4.237019in}{3.012162in}}%
\pgfpathlineto{\pgfqpoint{4.229477in}{3.002178in}}%
\pgfpathlineto{\pgfqpoint{4.221931in}{2.992254in}}%
\pgfpathlineto{\pgfqpoint{4.208756in}{2.997696in}}%
\pgfpathlineto{\pgfqpoint{4.195585in}{3.003262in}}%
\pgfpathlineto{\pgfqpoint{4.182420in}{3.008954in}}%
\pgfpathlineto{\pgfqpoint{4.169259in}{3.014772in}}%
\pgfpathlineto{\pgfqpoint{4.176817in}{3.024848in}}%
\pgfpathlineto{\pgfqpoint{4.184369in}{3.034989in}}%
\pgfpathlineto{\pgfqpoint{4.191918in}{3.045197in}}%
\pgfpathlineto{\pgfqpoint{4.199462in}{3.055473in}}%
\pgfpathclose%
\pgfusepath{fill}%
\end{pgfscope}%
\begin{pgfscope}%
\pgfpathrectangle{\pgfqpoint{1.150000in}{0.150000in}}{\pgfqpoint{5.700000in}{5.700000in}}%
\pgfusepath{clip}%
\pgfsetbuttcap%
\pgfsetroundjoin%
\definecolor{currentfill}{rgb}{0.252194,0.269783,0.531579}%
\pgfsetfillcolor{currentfill}%
\pgfsetfillopacity{0.700000}%
\pgfsetlinewidth{0.000000pt}%
\definecolor{currentstroke}{rgb}{0.000000,0.000000,0.000000}%
\pgfsetstrokecolor{currentstroke}%
\pgfsetdash{}{0pt}%
\pgfpathmoveto{\pgfqpoint{3.365112in}{3.335371in}}%
\pgfpathlineto{\pgfqpoint{3.378217in}{3.322134in}}%
\pgfpathlineto{\pgfqpoint{3.391321in}{3.309080in}}%
\pgfpathlineto{\pgfqpoint{3.404424in}{3.296209in}}%
\pgfpathlineto{\pgfqpoint{3.417525in}{3.283519in}}%
\pgfpathlineto{\pgfqpoint{3.409742in}{3.273358in}}%
\pgfpathlineto{\pgfqpoint{3.401954in}{3.263287in}}%
\pgfpathlineto{\pgfqpoint{3.394159in}{3.253304in}}%
\pgfpathlineto{\pgfqpoint{3.386359in}{3.243409in}}%
\pgfpathlineto{\pgfqpoint{3.373244in}{3.256030in}}%
\pgfpathlineto{\pgfqpoint{3.360129in}{3.268831in}}%
\pgfpathlineto{\pgfqpoint{3.347011in}{3.281815in}}%
\pgfpathlineto{\pgfqpoint{3.333893in}{3.294984in}}%
\pgfpathlineto{\pgfqpoint{3.341707in}{3.304941in}}%
\pgfpathlineto{\pgfqpoint{3.349515in}{3.314991in}}%
\pgfpathlineto{\pgfqpoint{3.357316in}{3.325134in}}%
\pgfpathlineto{\pgfqpoint{3.365112in}{3.335371in}}%
\pgfpathclose%
\pgfusepath{fill}%
\end{pgfscope}%
\begin{pgfscope}%
\pgfpathrectangle{\pgfqpoint{1.150000in}{0.150000in}}{\pgfqpoint{5.700000in}{5.700000in}}%
\pgfusepath{clip}%
\pgfsetbuttcap%
\pgfsetroundjoin%
\definecolor{currentfill}{rgb}{0.260571,0.246922,0.522828}%
\pgfsetfillcolor{currentfill}%
\pgfsetfillopacity{0.700000}%
\pgfsetlinewidth{0.000000pt}%
\definecolor{currentstroke}{rgb}{0.000000,0.000000,0.000000}%
\pgfsetstrokecolor{currentstroke}%
\pgfsetdash{}{0pt}%
\pgfpathmoveto{\pgfqpoint{3.417525in}{3.283519in}}%
\pgfpathlineto{\pgfqpoint{3.430625in}{3.271008in}}%
\pgfpathlineto{\pgfqpoint{3.443725in}{3.258674in}}%
\pgfpathlineto{\pgfqpoint{3.456824in}{3.246517in}}%
\pgfpathlineto{\pgfqpoint{3.469922in}{3.234533in}}%
\pgfpathlineto{\pgfqpoint{3.462153in}{3.224449in}}%
\pgfpathlineto{\pgfqpoint{3.454377in}{3.214449in}}%
\pgfpathlineto{\pgfqpoint{3.446596in}{3.204533in}}%
\pgfpathlineto{\pgfqpoint{3.438809in}{3.194701in}}%
\pgfpathlineto{\pgfqpoint{3.425698in}{3.206615in}}%
\pgfpathlineto{\pgfqpoint{3.412586in}{3.218703in}}%
\pgfpathlineto{\pgfqpoint{3.399473in}{3.230967in}}%
\pgfpathlineto{\pgfqpoint{3.386359in}{3.243409in}}%
\pgfpathlineto{\pgfqpoint{3.394159in}{3.253304in}}%
\pgfpathlineto{\pgfqpoint{3.401954in}{3.263287in}}%
\pgfpathlineto{\pgfqpoint{3.409742in}{3.273358in}}%
\pgfpathlineto{\pgfqpoint{3.417525in}{3.283519in}}%
\pgfpathclose%
\pgfusepath{fill}%
\end{pgfscope}%
\begin{pgfscope}%
\pgfpathrectangle{\pgfqpoint{1.150000in}{0.150000in}}{\pgfqpoint{5.700000in}{5.700000in}}%
\pgfusepath{clip}%
\pgfsetbuttcap%
\pgfsetroundjoin%
\definecolor{currentfill}{rgb}{0.282623,0.140926,0.457517}%
\pgfsetfillcolor{currentfill}%
\pgfsetfillopacity{0.700000}%
\pgfsetlinewidth{0.000000pt}%
\definecolor{currentstroke}{rgb}{0.000000,0.000000,0.000000}%
\pgfsetstrokecolor{currentstroke}%
\pgfsetdash{}{0pt}%
\pgfpathmoveto{\pgfqpoint{4.334844in}{3.051573in}}%
\pgfpathlineto{\pgfqpoint{4.348024in}{3.046410in}}%
\pgfpathlineto{\pgfqpoint{4.361209in}{3.041368in}}%
\pgfpathlineto{\pgfqpoint{4.374401in}{3.036446in}}%
\pgfpathlineto{\pgfqpoint{4.387598in}{3.031644in}}%
\pgfpathlineto{\pgfqpoint{4.380104in}{3.021622in}}%
\pgfpathlineto{\pgfqpoint{4.372607in}{3.011664in}}%
\pgfpathlineto{\pgfqpoint{4.365105in}{3.001768in}}%
\pgfpathlineto{\pgfqpoint{4.357598in}{2.991932in}}%
\pgfpathlineto{\pgfqpoint{4.344390in}{2.996556in}}%
\pgfpathlineto{\pgfqpoint{4.331188in}{3.001301in}}%
\pgfpathlineto{\pgfqpoint{4.317991in}{3.006165in}}%
\pgfpathlineto{\pgfqpoint{4.304800in}{3.011151in}}%
\pgfpathlineto{\pgfqpoint{4.312317in}{3.021158in}}%
\pgfpathlineto{\pgfqpoint{4.319830in}{3.031229in}}%
\pgfpathlineto{\pgfqpoint{4.327339in}{3.041367in}}%
\pgfpathlineto{\pgfqpoint{4.334844in}{3.051573in}}%
\pgfpathclose%
\pgfusepath{fill}%
\end{pgfscope}%
\begin{pgfscope}%
\pgfpathrectangle{\pgfqpoint{1.150000in}{0.150000in}}{\pgfqpoint{5.700000in}{5.700000in}}%
\pgfusepath{clip}%
\pgfsetbuttcap%
\pgfsetroundjoin%
\definecolor{currentfill}{rgb}{0.278012,0.180367,0.486697}%
\pgfsetfillcolor{currentfill}%
\pgfsetfillopacity{0.700000}%
\pgfsetlinewidth{0.000000pt}%
\definecolor{currentstroke}{rgb}{0.000000,0.000000,0.000000}%
\pgfsetstrokecolor{currentstroke}%
\pgfsetdash{}{0pt}%
\pgfpathmoveto{\pgfqpoint{3.657987in}{3.143990in}}%
\pgfpathlineto{\pgfqpoint{3.671079in}{3.134060in}}%
\pgfpathlineto{\pgfqpoint{3.684173in}{3.124285in}}%
\pgfpathlineto{\pgfqpoint{3.697268in}{3.114665in}}%
\pgfpathlineto{\pgfqpoint{3.710364in}{3.105199in}}%
\pgfpathlineto{\pgfqpoint{3.702666in}{3.095138in}}%
\pgfpathlineto{\pgfqpoint{3.694962in}{3.085150in}}%
\pgfpathlineto{\pgfqpoint{3.687253in}{3.075233in}}%
\pgfpathlineto{\pgfqpoint{3.679539in}{3.065385in}}%
\pgfpathlineto{\pgfqpoint{3.666430in}{3.074764in}}%
\pgfpathlineto{\pgfqpoint{3.653323in}{3.084296in}}%
\pgfpathlineto{\pgfqpoint{3.640217in}{3.093983in}}%
\pgfpathlineto{\pgfqpoint{3.627113in}{3.103825in}}%
\pgfpathlineto{\pgfqpoint{3.634840in}{3.113753in}}%
\pgfpathlineto{\pgfqpoint{3.642561in}{3.123756in}}%
\pgfpathlineto{\pgfqpoint{3.650276in}{3.133835in}}%
\pgfpathlineto{\pgfqpoint{3.657987in}{3.143990in}}%
\pgfpathclose%
\pgfusepath{fill}%
\end{pgfscope}%
\begin{pgfscope}%
\pgfpathrectangle{\pgfqpoint{1.150000in}{0.150000in}}{\pgfqpoint{5.700000in}{5.700000in}}%
\pgfusepath{clip}%
\pgfsetbuttcap%
\pgfsetroundjoin%
\definecolor{currentfill}{rgb}{0.282623,0.140926,0.457517}%
\pgfsetfillcolor{currentfill}%
\pgfsetfillopacity{0.700000}%
\pgfsetlinewidth{0.000000pt}%
\definecolor{currentstroke}{rgb}{0.000000,0.000000,0.000000}%
\pgfsetstrokecolor{currentstroke}%
\pgfsetdash{}{0pt}%
\pgfpathmoveto{\pgfqpoint{3.981274in}{3.053308in}}%
\pgfpathlineto{\pgfqpoint{3.994395in}{3.046058in}}%
\pgfpathlineto{\pgfqpoint{4.007519in}{3.038944in}}%
\pgfpathlineto{\pgfqpoint{4.020647in}{3.031965in}}%
\pgfpathlineto{\pgfqpoint{4.033778in}{3.025119in}}%
\pgfpathlineto{\pgfqpoint{4.026176in}{3.015077in}}%
\pgfpathlineto{\pgfqpoint{4.018569in}{3.005098in}}%
\pgfpathlineto{\pgfqpoint{4.010958in}{2.995179in}}%
\pgfpathlineto{\pgfqpoint{4.003341in}{2.985320in}}%
\pgfpathlineto{\pgfqpoint{3.990198in}{2.992042in}}%
\pgfpathlineto{\pgfqpoint{3.977059in}{2.998897in}}%
\pgfpathlineto{\pgfqpoint{3.963924in}{3.005887in}}%
\pgfpathlineto{\pgfqpoint{3.950793in}{3.013013in}}%
\pgfpathlineto{\pgfqpoint{3.958420in}{3.022990in}}%
\pgfpathlineto{\pgfqpoint{3.966043in}{3.033030in}}%
\pgfpathlineto{\pgfqpoint{3.973661in}{3.043135in}}%
\pgfpathlineto{\pgfqpoint{3.981274in}{3.053308in}}%
\pgfpathclose%
\pgfusepath{fill}%
\end{pgfscope}%
\begin{pgfscope}%
\pgfpathrectangle{\pgfqpoint{1.150000in}{0.150000in}}{\pgfqpoint{5.700000in}{5.700000in}}%
\pgfusepath{clip}%
\pgfsetbuttcap%
\pgfsetroundjoin%
\definecolor{currentfill}{rgb}{0.266580,0.228262,0.514349}%
\pgfsetfillcolor{currentfill}%
\pgfsetfillopacity{0.700000}%
\pgfsetlinewidth{0.000000pt}%
\definecolor{currentstroke}{rgb}{0.000000,0.000000,0.000000}%
\pgfsetstrokecolor{currentstroke}%
\pgfsetdash{}{0pt}%
\pgfpathmoveto{\pgfqpoint{3.469922in}{3.234533in}}%
\pgfpathlineto{\pgfqpoint{3.483020in}{3.222723in}}%
\pgfpathlineto{\pgfqpoint{3.496118in}{3.211085in}}%
\pgfpathlineto{\pgfqpoint{3.509216in}{3.199616in}}%
\pgfpathlineto{\pgfqpoint{3.522313in}{3.188316in}}%
\pgfpathlineto{\pgfqpoint{3.514557in}{3.178307in}}%
\pgfpathlineto{\pgfqpoint{3.506794in}{3.168380in}}%
\pgfpathlineto{\pgfqpoint{3.499026in}{3.158532in}}%
\pgfpathlineto{\pgfqpoint{3.491252in}{3.148762in}}%
\pgfpathlineto{\pgfqpoint{3.478141in}{3.159993in}}%
\pgfpathlineto{\pgfqpoint{3.465031in}{3.171391in}}%
\pgfpathlineto{\pgfqpoint{3.451920in}{3.182960in}}%
\pgfpathlineto{\pgfqpoint{3.438809in}{3.194701in}}%
\pgfpathlineto{\pgfqpoint{3.446596in}{3.204533in}}%
\pgfpathlineto{\pgfqpoint{3.454377in}{3.214449in}}%
\pgfpathlineto{\pgfqpoint{3.462153in}{3.224449in}}%
\pgfpathlineto{\pgfqpoint{3.469922in}{3.234533in}}%
\pgfpathclose%
\pgfusepath{fill}%
\end{pgfscope}%
\begin{pgfscope}%
\pgfpathrectangle{\pgfqpoint{1.150000in}{0.150000in}}{\pgfqpoint{5.700000in}{5.700000in}}%
\pgfusepath{clip}%
\pgfsetbuttcap%
\pgfsetroundjoin%
\definecolor{currentfill}{rgb}{0.281887,0.150881,0.465405}%
\pgfsetfillcolor{currentfill}%
\pgfsetfillopacity{0.700000}%
\pgfsetlinewidth{0.000000pt}%
\definecolor{currentstroke}{rgb}{0.000000,0.000000,0.000000}%
\pgfsetstrokecolor{currentstroke}%
\pgfsetdash{}{0pt}%
\pgfpathmoveto{\pgfqpoint{3.845862in}{3.075002in}}%
\pgfpathlineto{\pgfqpoint{3.858968in}{3.066761in}}%
\pgfpathlineto{\pgfqpoint{3.872076in}{3.058662in}}%
\pgfpathlineto{\pgfqpoint{3.885188in}{3.050705in}}%
\pgfpathlineto{\pgfqpoint{3.898302in}{3.042888in}}%
\pgfpathlineto{\pgfqpoint{3.890658in}{3.032861in}}%
\pgfpathlineto{\pgfqpoint{3.883010in}{3.022899in}}%
\pgfpathlineto{\pgfqpoint{3.875356in}{3.013000in}}%
\pgfpathlineto{\pgfqpoint{3.867698in}{3.003163in}}%
\pgfpathlineto{\pgfqpoint{3.854572in}{3.010873in}}%
\pgfpathlineto{\pgfqpoint{3.841449in}{3.018724in}}%
\pgfpathlineto{\pgfqpoint{3.828329in}{3.026717in}}%
\pgfpathlineto{\pgfqpoint{3.815212in}{3.034853in}}%
\pgfpathlineto{\pgfqpoint{3.822882in}{3.044789in}}%
\pgfpathlineto{\pgfqpoint{3.830547in}{3.054792in}}%
\pgfpathlineto{\pgfqpoint{3.838207in}{3.064862in}}%
\pgfpathlineto{\pgfqpoint{3.845862in}{3.075002in}}%
\pgfpathclose%
\pgfusepath{fill}%
\end{pgfscope}%
\begin{pgfscope}%
\pgfpathrectangle{\pgfqpoint{1.150000in}{0.150000in}}{\pgfqpoint{5.700000in}{5.700000in}}%
\pgfusepath{clip}%
\pgfsetbuttcap%
\pgfsetroundjoin%
\definecolor{currentfill}{rgb}{0.282884,0.135920,0.453427}%
\pgfsetfillcolor{currentfill}%
\pgfsetfillopacity{0.700000}%
\pgfsetlinewidth{0.000000pt}%
\definecolor{currentstroke}{rgb}{0.000000,0.000000,0.000000}%
\pgfsetstrokecolor{currentstroke}%
\pgfsetdash{}{0pt}%
\pgfpathmoveto{\pgfqpoint{4.116665in}{3.039321in}}%
\pgfpathlineto{\pgfqpoint{4.129806in}{3.032990in}}%
\pgfpathlineto{\pgfqpoint{4.142953in}{3.026789in}}%
\pgfpathlineto{\pgfqpoint{4.156104in}{3.020717in}}%
\pgfpathlineto{\pgfqpoint{4.169259in}{3.014772in}}%
\pgfpathlineto{\pgfqpoint{4.161698in}{3.004758in}}%
\pgfpathlineto{\pgfqpoint{4.154132in}{2.994806in}}%
\pgfpathlineto{\pgfqpoint{4.146561in}{2.984913in}}%
\pgfpathlineto{\pgfqpoint{4.138985in}{2.975077in}}%
\pgfpathlineto{\pgfqpoint{4.125819in}{2.980880in}}%
\pgfpathlineto{\pgfqpoint{4.112657in}{2.986811in}}%
\pgfpathlineto{\pgfqpoint{4.099499in}{2.992870in}}%
\pgfpathlineto{\pgfqpoint{4.086346in}{2.999058in}}%
\pgfpathlineto{\pgfqpoint{4.093933in}{3.009030in}}%
\pgfpathlineto{\pgfqpoint{4.101515in}{3.019063in}}%
\pgfpathlineto{\pgfqpoint{4.109092in}{3.029159in}}%
\pgfpathlineto{\pgfqpoint{4.116665in}{3.039321in}}%
\pgfpathclose%
\pgfusepath{fill}%
\end{pgfscope}%
\begin{pgfscope}%
\pgfpathrectangle{\pgfqpoint{1.150000in}{0.150000in}}{\pgfqpoint{5.700000in}{5.700000in}}%
\pgfusepath{clip}%
\pgfsetbuttcap%
\pgfsetroundjoin%
\definecolor{currentfill}{rgb}{0.282290,0.145912,0.461510}%
\pgfsetfillcolor{currentfill}%
\pgfsetfillopacity{0.700000}%
\pgfsetlinewidth{0.000000pt}%
\definecolor{currentstroke}{rgb}{0.000000,0.000000,0.000000}%
\pgfsetstrokecolor{currentstroke}%
\pgfsetdash{}{0pt}%
\pgfpathmoveto{\pgfqpoint{4.470333in}{3.053625in}}%
\pgfpathlineto{\pgfqpoint{4.483548in}{3.049219in}}%
\pgfpathlineto{\pgfqpoint{4.496770in}{3.044930in}}%
\pgfpathlineto{\pgfqpoint{4.509999in}{3.040757in}}%
\pgfpathlineto{\pgfqpoint{4.523233in}{3.036699in}}%
\pgfpathlineto{\pgfqpoint{4.515779in}{3.026799in}}%
\pgfpathlineto{\pgfqpoint{4.508321in}{3.016964in}}%
\pgfpathlineto{\pgfqpoint{4.500859in}{3.007192in}}%
\pgfpathlineto{\pgfqpoint{4.493393in}{2.997480in}}%
\pgfpathlineto{\pgfqpoint{4.480146in}{3.001342in}}%
\pgfpathlineto{\pgfqpoint{4.466906in}{3.005320in}}%
\pgfpathlineto{\pgfqpoint{4.453673in}{3.009414in}}%
\pgfpathlineto{\pgfqpoint{4.440445in}{3.013625in}}%
\pgfpathlineto{\pgfqpoint{4.447923in}{3.023526in}}%
\pgfpathlineto{\pgfqpoint{4.455397in}{3.033491in}}%
\pgfpathlineto{\pgfqpoint{4.462867in}{3.043523in}}%
\pgfpathlineto{\pgfqpoint{4.470333in}{3.053625in}}%
\pgfpathclose%
\pgfusepath{fill}%
\end{pgfscope}%
\begin{pgfscope}%
\pgfpathrectangle{\pgfqpoint{1.150000in}{0.150000in}}{\pgfqpoint{5.700000in}{5.700000in}}%
\pgfusepath{clip}%
\pgfsetbuttcap%
\pgfsetroundjoin%
\definecolor{currentfill}{rgb}{0.283072,0.130895,0.449241}%
\pgfsetfillcolor{currentfill}%
\pgfsetfillopacity{0.700000}%
\pgfsetlinewidth{0.000000pt}%
\definecolor{currentstroke}{rgb}{0.000000,0.000000,0.000000}%
\pgfsetstrokecolor{currentstroke}%
\pgfsetdash{}{0pt}%
\pgfpathmoveto{\pgfqpoint{4.252089in}{3.032319in}}%
\pgfpathlineto{\pgfqpoint{4.265259in}{3.026842in}}%
\pgfpathlineto{\pgfqpoint{4.278434in}{3.021489in}}%
\pgfpathlineto{\pgfqpoint{4.291614in}{3.016259in}}%
\pgfpathlineto{\pgfqpoint{4.304800in}{3.011151in}}%
\pgfpathlineto{\pgfqpoint{4.297278in}{3.001207in}}%
\pgfpathlineto{\pgfqpoint{4.289752in}{2.991322in}}%
\pgfpathlineto{\pgfqpoint{4.282221in}{2.981496in}}%
\pgfpathlineto{\pgfqpoint{4.274686in}{2.971726in}}%
\pgfpathlineto{\pgfqpoint{4.261489in}{2.976674in}}%
\pgfpathlineto{\pgfqpoint{4.248298in}{2.981744in}}%
\pgfpathlineto{\pgfqpoint{4.235112in}{2.986938in}}%
\pgfpathlineto{\pgfqpoint{4.221931in}{2.992254in}}%
\pgfpathlineto{\pgfqpoint{4.229477in}{3.002178in}}%
\pgfpathlineto{\pgfqpoint{4.237019in}{3.012162in}}%
\pgfpathlineto{\pgfqpoint{4.244556in}{3.022208in}}%
\pgfpathlineto{\pgfqpoint{4.252089in}{3.032319in}}%
\pgfpathclose%
\pgfusepath{fill}%
\end{pgfscope}%
\begin{pgfscope}%
\pgfpathrectangle{\pgfqpoint{1.150000in}{0.150000in}}{\pgfqpoint{5.700000in}{5.700000in}}%
\pgfusepath{clip}%
\pgfsetbuttcap%
\pgfsetroundjoin%
\definecolor{currentfill}{rgb}{0.280255,0.165693,0.476498}%
\pgfsetfillcolor{currentfill}%
\pgfsetfillopacity{0.700000}%
\pgfsetlinewidth{0.000000pt}%
\definecolor{currentstroke}{rgb}{0.000000,0.000000,0.000000}%
\pgfsetstrokecolor{currentstroke}%
\pgfsetdash{}{0pt}%
\pgfpathmoveto{\pgfqpoint{3.710364in}{3.105199in}}%
\pgfpathlineto{\pgfqpoint{3.723463in}{3.095884in}}%
\pgfpathlineto{\pgfqpoint{3.736563in}{3.086721in}}%
\pgfpathlineto{\pgfqpoint{3.749666in}{3.077708in}}%
\pgfpathlineto{\pgfqpoint{3.762770in}{3.068843in}}%
\pgfpathlineto{\pgfqpoint{3.755084in}{3.058877in}}%
\pgfpathlineto{\pgfqpoint{3.747392in}{3.048979in}}%
\pgfpathlineto{\pgfqpoint{3.739694in}{3.039148in}}%
\pgfpathlineto{\pgfqpoint{3.731992in}{3.029382in}}%
\pgfpathlineto{\pgfqpoint{3.718876in}{3.038159in}}%
\pgfpathlineto{\pgfqpoint{3.705761in}{3.047084in}}%
\pgfpathlineto{\pgfqpoint{3.692649in}{3.056159in}}%
\pgfpathlineto{\pgfqpoint{3.679539in}{3.065385in}}%
\pgfpathlineto{\pgfqpoint{3.687253in}{3.075233in}}%
\pgfpathlineto{\pgfqpoint{3.694962in}{3.085150in}}%
\pgfpathlineto{\pgfqpoint{3.702666in}{3.095138in}}%
\pgfpathlineto{\pgfqpoint{3.710364in}{3.105199in}}%
\pgfpathclose%
\pgfusepath{fill}%
\end{pgfscope}%
\begin{pgfscope}%
\pgfpathrectangle{\pgfqpoint{1.150000in}{0.150000in}}{\pgfqpoint{5.700000in}{5.700000in}}%
\pgfusepath{clip}%
\pgfsetbuttcap%
\pgfsetroundjoin%
\definecolor{currentfill}{rgb}{0.273006,0.204520,0.501721}%
\pgfsetfillcolor{currentfill}%
\pgfsetfillopacity{0.700000}%
\pgfsetlinewidth{0.000000pt}%
\definecolor{currentstroke}{rgb}{0.000000,0.000000,0.000000}%
\pgfsetstrokecolor{currentstroke}%
\pgfsetdash{}{0pt}%
\pgfpathmoveto{\pgfqpoint{3.522313in}{3.188316in}}%
\pgfpathlineto{\pgfqpoint{3.535411in}{3.177183in}}%
\pgfpathlineto{\pgfqpoint{3.548510in}{3.166216in}}%
\pgfpathlineto{\pgfqpoint{3.561608in}{3.155414in}}%
\pgfpathlineto{\pgfqpoint{3.574707in}{3.144775in}}%
\pgfpathlineto{\pgfqpoint{3.566963in}{3.134843in}}%
\pgfpathlineto{\pgfqpoint{3.559213in}{3.124987in}}%
\pgfpathlineto{\pgfqpoint{3.551458in}{3.115207in}}%
\pgfpathlineto{\pgfqpoint{3.543697in}{3.105501in}}%
\pgfpathlineto{\pgfqpoint{3.530585in}{3.116070in}}%
\pgfpathlineto{\pgfqpoint{3.517474in}{3.126803in}}%
\pgfpathlineto{\pgfqpoint{3.504363in}{3.137699in}}%
\pgfpathlineto{\pgfqpoint{3.491252in}{3.148762in}}%
\pgfpathlineto{\pgfqpoint{3.499026in}{3.158532in}}%
\pgfpathlineto{\pgfqpoint{3.506794in}{3.168380in}}%
\pgfpathlineto{\pgfqpoint{3.514557in}{3.178307in}}%
\pgfpathlineto{\pgfqpoint{3.522313in}{3.188316in}}%
\pgfpathclose%
\pgfusepath{fill}%
\end{pgfscope}%
\begin{pgfscope}%
\pgfpathrectangle{\pgfqpoint{1.150000in}{0.150000in}}{\pgfqpoint{5.700000in}{5.700000in}}%
\pgfusepath{clip}%
\pgfsetbuttcap%
\pgfsetroundjoin%
\definecolor{currentfill}{rgb}{0.282884,0.135920,0.453427}%
\pgfsetfillcolor{currentfill}%
\pgfsetfillopacity{0.700000}%
\pgfsetlinewidth{0.000000pt}%
\definecolor{currentstroke}{rgb}{0.000000,0.000000,0.000000}%
\pgfsetstrokecolor{currentstroke}%
\pgfsetdash{}{0pt}%
\pgfpathmoveto{\pgfqpoint{4.387598in}{3.031644in}}%
\pgfpathlineto{\pgfqpoint{4.400801in}{3.026962in}}%
\pgfpathlineto{\pgfqpoint{4.414010in}{3.022398in}}%
\pgfpathlineto{\pgfqpoint{4.427224in}{3.017953in}}%
\pgfpathlineto{\pgfqpoint{4.440445in}{3.013625in}}%
\pgfpathlineto{\pgfqpoint{4.432963in}{3.003786in}}%
\pgfpathlineto{\pgfqpoint{4.425477in}{2.994008in}}%
\pgfpathlineto{\pgfqpoint{4.417987in}{2.984288in}}%
\pgfpathlineto{\pgfqpoint{4.410492in}{2.974623in}}%
\pgfpathlineto{\pgfqpoint{4.397259in}{2.978773in}}%
\pgfpathlineto{\pgfqpoint{4.384033in}{2.983041in}}%
\pgfpathlineto{\pgfqpoint{4.370813in}{2.987427in}}%
\pgfpathlineto{\pgfqpoint{4.357598in}{2.991932in}}%
\pgfpathlineto{\pgfqpoint{4.365105in}{3.001768in}}%
\pgfpathlineto{\pgfqpoint{4.372607in}{3.011664in}}%
\pgfpathlineto{\pgfqpoint{4.380104in}{3.021622in}}%
\pgfpathlineto{\pgfqpoint{4.387598in}{3.031644in}}%
\pgfpathclose%
\pgfusepath{fill}%
\end{pgfscope}%
\begin{pgfscope}%
\pgfpathrectangle{\pgfqpoint{1.150000in}{0.150000in}}{\pgfqpoint{5.700000in}{5.700000in}}%
\pgfusepath{clip}%
\pgfsetbuttcap%
\pgfsetroundjoin%
\definecolor{currentfill}{rgb}{0.282623,0.140926,0.457517}%
\pgfsetfillcolor{currentfill}%
\pgfsetfillopacity{0.700000}%
\pgfsetlinewidth{0.000000pt}%
\definecolor{currentstroke}{rgb}{0.000000,0.000000,0.000000}%
\pgfsetstrokecolor{currentstroke}%
\pgfsetdash{}{0pt}%
\pgfpathmoveto{\pgfqpoint{3.898302in}{3.042888in}}%
\pgfpathlineto{\pgfqpoint{3.911420in}{3.035212in}}%
\pgfpathlineto{\pgfqpoint{3.924541in}{3.027674in}}%
\pgfpathlineto{\pgfqpoint{3.937665in}{3.020275in}}%
\pgfpathlineto{\pgfqpoint{3.950793in}{3.013013in}}%
\pgfpathlineto{\pgfqpoint{3.943160in}{3.003099in}}%
\pgfpathlineto{\pgfqpoint{3.935523in}{2.993245in}}%
\pgfpathlineto{\pgfqpoint{3.927881in}{2.983450in}}%
\pgfpathlineto{\pgfqpoint{3.920234in}{2.973713in}}%
\pgfpathlineto{\pgfqpoint{3.907095in}{2.980869in}}%
\pgfpathlineto{\pgfqpoint{3.893959in}{2.988162in}}%
\pgfpathlineto{\pgfqpoint{3.880827in}{2.995593in}}%
\pgfpathlineto{\pgfqpoint{3.867698in}{3.003163in}}%
\pgfpathlineto{\pgfqpoint{3.875356in}{3.013000in}}%
\pgfpathlineto{\pgfqpoint{3.883010in}{3.022899in}}%
\pgfpathlineto{\pgfqpoint{3.890658in}{3.032861in}}%
\pgfpathlineto{\pgfqpoint{3.898302in}{3.042888in}}%
\pgfpathclose%
\pgfusepath{fill}%
\end{pgfscope}%
\begin{pgfscope}%
\pgfpathrectangle{\pgfqpoint{1.150000in}{0.150000in}}{\pgfqpoint{5.700000in}{5.700000in}}%
\pgfusepath{clip}%
\pgfsetbuttcap%
\pgfsetroundjoin%
\definecolor{currentfill}{rgb}{0.283072,0.130895,0.449241}%
\pgfsetfillcolor{currentfill}%
\pgfsetfillopacity{0.700000}%
\pgfsetlinewidth{0.000000pt}%
\definecolor{currentstroke}{rgb}{0.000000,0.000000,0.000000}%
\pgfsetstrokecolor{currentstroke}%
\pgfsetdash{}{0pt}%
\pgfpathmoveto{\pgfqpoint{4.033778in}{3.025119in}}%
\pgfpathlineto{\pgfqpoint{4.046914in}{3.018406in}}%
\pgfpathlineto{\pgfqpoint{4.060054in}{3.011826in}}%
\pgfpathlineto{\pgfqpoint{4.073198in}{3.005377in}}%
\pgfpathlineto{\pgfqpoint{4.086346in}{2.999058in}}%
\pgfpathlineto{\pgfqpoint{4.078755in}{2.989147in}}%
\pgfpathlineto{\pgfqpoint{4.071160in}{2.979294in}}%
\pgfpathlineto{\pgfqpoint{4.063559in}{2.969498in}}%
\pgfpathlineto{\pgfqpoint{4.055954in}{2.959756in}}%
\pgfpathlineto{\pgfqpoint{4.042794in}{2.965950in}}%
\pgfpathlineto{\pgfqpoint{4.029639in}{2.972275in}}%
\pgfpathlineto{\pgfqpoint{4.016488in}{2.978731in}}%
\pgfpathlineto{\pgfqpoint{4.003341in}{2.985320in}}%
\pgfpathlineto{\pgfqpoint{4.010958in}{2.995179in}}%
\pgfpathlineto{\pgfqpoint{4.018569in}{3.005098in}}%
\pgfpathlineto{\pgfqpoint{4.026176in}{3.015077in}}%
\pgfpathlineto{\pgfqpoint{4.033778in}{3.025119in}}%
\pgfpathclose%
\pgfusepath{fill}%
\end{pgfscope}%
\begin{pgfscope}%
\pgfpathrectangle{\pgfqpoint{1.150000in}{0.150000in}}{\pgfqpoint{5.700000in}{5.700000in}}%
\pgfusepath{clip}%
\pgfsetbuttcap%
\pgfsetroundjoin%
\definecolor{currentfill}{rgb}{0.246811,0.283237,0.535941}%
\pgfsetfillcolor{currentfill}%
\pgfsetfillopacity{0.700000}%
\pgfsetlinewidth{0.000000pt}%
\definecolor{currentstroke}{rgb}{0.000000,0.000000,0.000000}%
\pgfsetstrokecolor{currentstroke}%
\pgfsetdash{}{0pt}%
\pgfpathmoveto{\pgfqpoint{3.281400in}{3.349531in}}%
\pgfpathlineto{\pgfqpoint{3.294526in}{3.335609in}}%
\pgfpathlineto{\pgfqpoint{3.307650in}{3.321879in}}%
\pgfpathlineto{\pgfqpoint{3.320772in}{3.308337in}}%
\pgfpathlineto{\pgfqpoint{3.333893in}{3.294984in}}%
\pgfpathlineto{\pgfqpoint{3.326072in}{3.285117in}}%
\pgfpathlineto{\pgfqpoint{3.318246in}{3.275340in}}%
\pgfpathlineto{\pgfqpoint{3.310414in}{3.265653in}}%
\pgfpathlineto{\pgfqpoint{3.302575in}{3.256054in}}%
\pgfpathlineto{\pgfqpoint{3.289440in}{3.269356in}}%
\pgfpathlineto{\pgfqpoint{3.276304in}{3.282846in}}%
\pgfpathlineto{\pgfqpoint{3.263166in}{3.296526in}}%
\pgfpathlineto{\pgfqpoint{3.250025in}{3.310396in}}%
\pgfpathlineto{\pgfqpoint{3.257878in}{3.320040in}}%
\pgfpathlineto{\pgfqpoint{3.265725in}{3.329776in}}%
\pgfpathlineto{\pgfqpoint{3.273566in}{3.339606in}}%
\pgfpathlineto{\pgfqpoint{3.281400in}{3.349531in}}%
\pgfpathclose%
\pgfusepath{fill}%
\end{pgfscope}%
\begin{pgfscope}%
\pgfpathrectangle{\pgfqpoint{1.150000in}{0.150000in}}{\pgfqpoint{5.700000in}{5.700000in}}%
\pgfusepath{clip}%
\pgfsetbuttcap%
\pgfsetroundjoin%
\definecolor{currentfill}{rgb}{0.276194,0.190074,0.493001}%
\pgfsetfillcolor{currentfill}%
\pgfsetfillopacity{0.700000}%
\pgfsetlinewidth{0.000000pt}%
\definecolor{currentstroke}{rgb}{0.000000,0.000000,0.000000}%
\pgfsetstrokecolor{currentstroke}%
\pgfsetdash{}{0pt}%
\pgfpathmoveto{\pgfqpoint{3.574707in}{3.144775in}}%
\pgfpathlineto{\pgfqpoint{3.587807in}{3.134298in}}%
\pgfpathlineto{\pgfqpoint{3.600908in}{3.123981in}}%
\pgfpathlineto{\pgfqpoint{3.614010in}{3.113824in}}%
\pgfpathlineto{\pgfqpoint{3.627113in}{3.103825in}}%
\pgfpathlineto{\pgfqpoint{3.619381in}{3.093969in}}%
\pgfpathlineto{\pgfqpoint{3.611644in}{3.084186in}}%
\pgfpathlineto{\pgfqpoint{3.603901in}{3.074474in}}%
\pgfpathlineto{\pgfqpoint{3.596153in}{3.064831in}}%
\pgfpathlineto{\pgfqpoint{3.583037in}{3.074760in}}%
\pgfpathlineto{\pgfqpoint{3.569923in}{3.084847in}}%
\pgfpathlineto{\pgfqpoint{3.556810in}{3.095094in}}%
\pgfpathlineto{\pgfqpoint{3.543697in}{3.105501in}}%
\pgfpathlineto{\pgfqpoint{3.551458in}{3.115207in}}%
\pgfpathlineto{\pgfqpoint{3.559213in}{3.124987in}}%
\pgfpathlineto{\pgfqpoint{3.566963in}{3.134843in}}%
\pgfpathlineto{\pgfqpoint{3.574707in}{3.144775in}}%
\pgfpathclose%
\pgfusepath{fill}%
\end{pgfscope}%
\begin{pgfscope}%
\pgfpathrectangle{\pgfqpoint{1.150000in}{0.150000in}}{\pgfqpoint{5.700000in}{5.700000in}}%
\pgfusepath{clip}%
\pgfsetbuttcap%
\pgfsetroundjoin%
\definecolor{currentfill}{rgb}{0.281887,0.150881,0.465405}%
\pgfsetfillcolor{currentfill}%
\pgfsetfillopacity{0.700000}%
\pgfsetlinewidth{0.000000pt}%
\definecolor{currentstroke}{rgb}{0.000000,0.000000,0.000000}%
\pgfsetstrokecolor{currentstroke}%
\pgfsetdash{}{0pt}%
\pgfpathmoveto{\pgfqpoint{3.762770in}{3.068843in}}%
\pgfpathlineto{\pgfqpoint{3.775877in}{3.060126in}}%
\pgfpathlineto{\pgfqpoint{3.788986in}{3.051556in}}%
\pgfpathlineto{\pgfqpoint{3.802098in}{3.043132in}}%
\pgfpathlineto{\pgfqpoint{3.815212in}{3.034853in}}%
\pgfpathlineto{\pgfqpoint{3.807537in}{3.024982in}}%
\pgfpathlineto{\pgfqpoint{3.799857in}{3.015174in}}%
\pgfpathlineto{\pgfqpoint{3.792172in}{3.005429in}}%
\pgfpathlineto{\pgfqpoint{3.784481in}{2.995746in}}%
\pgfpathlineto{\pgfqpoint{3.771355in}{3.003937in}}%
\pgfpathlineto{\pgfqpoint{3.758232in}{3.012272in}}%
\pgfpathlineto{\pgfqpoint{3.745111in}{3.020754in}}%
\pgfpathlineto{\pgfqpoint{3.731992in}{3.029382in}}%
\pgfpathlineto{\pgfqpoint{3.739694in}{3.039148in}}%
\pgfpathlineto{\pgfqpoint{3.747392in}{3.048979in}}%
\pgfpathlineto{\pgfqpoint{3.755084in}{3.058877in}}%
\pgfpathlineto{\pgfqpoint{3.762770in}{3.068843in}}%
\pgfpathclose%
\pgfusepath{fill}%
\end{pgfscope}%
\begin{pgfscope}%
\pgfpathrectangle{\pgfqpoint{1.150000in}{0.150000in}}{\pgfqpoint{5.700000in}{5.700000in}}%
\pgfusepath{clip}%
\pgfsetbuttcap%
\pgfsetroundjoin%
\definecolor{currentfill}{rgb}{0.283072,0.130895,0.449241}%
\pgfsetfillcolor{currentfill}%
\pgfsetfillopacity{0.700000}%
\pgfsetlinewidth{0.000000pt}%
\definecolor{currentstroke}{rgb}{0.000000,0.000000,0.000000}%
\pgfsetstrokecolor{currentstroke}%
\pgfsetdash{}{0pt}%
\pgfpathmoveto{\pgfqpoint{4.169259in}{3.014772in}}%
\pgfpathlineto{\pgfqpoint{4.182420in}{3.008954in}}%
\pgfpathlineto{\pgfqpoint{4.195585in}{3.003262in}}%
\pgfpathlineto{\pgfqpoint{4.208756in}{2.997696in}}%
\pgfpathlineto{\pgfqpoint{4.221931in}{2.992254in}}%
\pgfpathlineto{\pgfqpoint{4.214381in}{2.982390in}}%
\pgfpathlineto{\pgfqpoint{4.206826in}{2.972582in}}%
\pgfpathlineto{\pgfqpoint{4.199266in}{2.962828in}}%
\pgfpathlineto{\pgfqpoint{4.191702in}{2.953128in}}%
\pgfpathlineto{\pgfqpoint{4.178515in}{2.958427in}}%
\pgfpathlineto{\pgfqpoint{4.165334in}{2.963851in}}%
\pgfpathlineto{\pgfqpoint{4.152157in}{2.969401in}}%
\pgfpathlineto{\pgfqpoint{4.138985in}{2.975077in}}%
\pgfpathlineto{\pgfqpoint{4.146561in}{2.984913in}}%
\pgfpathlineto{\pgfqpoint{4.154132in}{2.994806in}}%
\pgfpathlineto{\pgfqpoint{4.161698in}{3.004758in}}%
\pgfpathlineto{\pgfqpoint{4.169259in}{3.014772in}}%
\pgfpathclose%
\pgfusepath{fill}%
\end{pgfscope}%
\begin{pgfscope}%
\pgfpathrectangle{\pgfqpoint{1.150000in}{0.150000in}}{\pgfqpoint{5.700000in}{5.700000in}}%
\pgfusepath{clip}%
\pgfsetbuttcap%
\pgfsetroundjoin%
\definecolor{currentfill}{rgb}{0.255645,0.260703,0.528312}%
\pgfsetfillcolor{currentfill}%
\pgfsetfillopacity{0.700000}%
\pgfsetlinewidth{0.000000pt}%
\definecolor{currentstroke}{rgb}{0.000000,0.000000,0.000000}%
\pgfsetstrokecolor{currentstroke}%
\pgfsetdash{}{0pt}%
\pgfpathmoveto{\pgfqpoint{3.333893in}{3.294984in}}%
\pgfpathlineto{\pgfqpoint{3.347011in}{3.281815in}}%
\pgfpathlineto{\pgfqpoint{3.360129in}{3.268831in}}%
\pgfpathlineto{\pgfqpoint{3.373244in}{3.256030in}}%
\pgfpathlineto{\pgfqpoint{3.386359in}{3.243409in}}%
\pgfpathlineto{\pgfqpoint{3.378553in}{3.233600in}}%
\pgfpathlineto{\pgfqpoint{3.370740in}{3.223877in}}%
\pgfpathlineto{\pgfqpoint{3.362922in}{3.214239in}}%
\pgfpathlineto{\pgfqpoint{3.355097in}{3.204685in}}%
\pgfpathlineto{\pgfqpoint{3.341969in}{3.217254in}}%
\pgfpathlineto{\pgfqpoint{3.328839in}{3.230004in}}%
\pgfpathlineto{\pgfqpoint{3.315708in}{3.242937in}}%
\pgfpathlineto{\pgfqpoint{3.302575in}{3.256054in}}%
\pgfpathlineto{\pgfqpoint{3.310414in}{3.265653in}}%
\pgfpathlineto{\pgfqpoint{3.318246in}{3.275340in}}%
\pgfpathlineto{\pgfqpoint{3.326072in}{3.285117in}}%
\pgfpathlineto{\pgfqpoint{3.333893in}{3.294984in}}%
\pgfpathclose%
\pgfusepath{fill}%
\end{pgfscope}%
\begin{pgfscope}%
\pgfpathrectangle{\pgfqpoint{1.150000in}{0.150000in}}{\pgfqpoint{5.700000in}{5.700000in}}%
\pgfusepath{clip}%
\pgfsetbuttcap%
\pgfsetroundjoin%
\definecolor{currentfill}{rgb}{0.263663,0.237631,0.518762}%
\pgfsetfillcolor{currentfill}%
\pgfsetfillopacity{0.700000}%
\pgfsetlinewidth{0.000000pt}%
\definecolor{currentstroke}{rgb}{0.000000,0.000000,0.000000}%
\pgfsetstrokecolor{currentstroke}%
\pgfsetdash{}{0pt}%
\pgfpathmoveto{\pgfqpoint{3.386359in}{3.243409in}}%
\pgfpathlineto{\pgfqpoint{3.399473in}{3.230967in}}%
\pgfpathlineto{\pgfqpoint{3.412586in}{3.218703in}}%
\pgfpathlineto{\pgfqpoint{3.425698in}{3.206615in}}%
\pgfpathlineto{\pgfqpoint{3.438809in}{3.194701in}}%
\pgfpathlineto{\pgfqpoint{3.431016in}{3.184951in}}%
\pgfpathlineto{\pgfqpoint{3.423217in}{3.175282in}}%
\pgfpathlineto{\pgfqpoint{3.415413in}{3.165693in}}%
\pgfpathlineto{\pgfqpoint{3.407602in}{3.156184in}}%
\pgfpathlineto{\pgfqpoint{3.394477in}{3.168046in}}%
\pgfpathlineto{\pgfqpoint{3.381351in}{3.180082in}}%
\pgfpathlineto{\pgfqpoint{3.368225in}{3.192295in}}%
\pgfpathlineto{\pgfqpoint{3.355097in}{3.204685in}}%
\pgfpathlineto{\pgfqpoint{3.362922in}{3.214239in}}%
\pgfpathlineto{\pgfqpoint{3.370740in}{3.223877in}}%
\pgfpathlineto{\pgfqpoint{3.378553in}{3.233600in}}%
\pgfpathlineto{\pgfqpoint{3.386359in}{3.243409in}}%
\pgfpathclose%
\pgfusepath{fill}%
\end{pgfscope}%
\begin{pgfscope}%
\pgfpathrectangle{\pgfqpoint{1.150000in}{0.150000in}}{\pgfqpoint{5.700000in}{5.700000in}}%
\pgfusepath{clip}%
\pgfsetbuttcap%
\pgfsetroundjoin%
\definecolor{currentfill}{rgb}{0.282290,0.145912,0.461510}%
\pgfsetfillcolor{currentfill}%
\pgfsetfillopacity{0.700000}%
\pgfsetlinewidth{0.000000pt}%
\definecolor{currentstroke}{rgb}{0.000000,0.000000,0.000000}%
\pgfsetstrokecolor{currentstroke}%
\pgfsetdash{}{0pt}%
\pgfpathmoveto{\pgfqpoint{4.523233in}{3.036699in}}%
\pgfpathlineto{\pgfqpoint{4.536475in}{3.032756in}}%
\pgfpathlineto{\pgfqpoint{4.549723in}{3.028928in}}%
\pgfpathlineto{\pgfqpoint{4.562977in}{3.025214in}}%
\pgfpathlineto{\pgfqpoint{4.576239in}{3.021613in}}%
\pgfpathlineto{\pgfqpoint{4.568796in}{3.011915in}}%
\pgfpathlineto{\pgfqpoint{4.561350in}{3.002278in}}%
\pgfpathlineto{\pgfqpoint{4.553900in}{2.992699in}}%
\pgfpathlineto{\pgfqpoint{4.546445in}{2.983177in}}%
\pgfpathlineto{\pgfqpoint{4.533172in}{2.986581in}}%
\pgfpathlineto{\pgfqpoint{4.519906in}{2.990100in}}%
\pgfpathlineto{\pgfqpoint{4.506646in}{2.993732in}}%
\pgfpathlineto{\pgfqpoint{4.493393in}{2.997480in}}%
\pgfpathlineto{\pgfqpoint{4.500859in}{3.007192in}}%
\pgfpathlineto{\pgfqpoint{4.508321in}{3.016964in}}%
\pgfpathlineto{\pgfqpoint{4.515779in}{3.026799in}}%
\pgfpathlineto{\pgfqpoint{4.523233in}{3.036699in}}%
\pgfpathclose%
\pgfusepath{fill}%
\end{pgfscope}%
\begin{pgfscope}%
\pgfpathrectangle{\pgfqpoint{1.150000in}{0.150000in}}{\pgfqpoint{5.700000in}{5.700000in}}%
\pgfusepath{clip}%
\pgfsetbuttcap%
\pgfsetroundjoin%
\definecolor{currentfill}{rgb}{0.283072,0.130895,0.449241}%
\pgfsetfillcolor{currentfill}%
\pgfsetfillopacity{0.700000}%
\pgfsetlinewidth{0.000000pt}%
\definecolor{currentstroke}{rgb}{0.000000,0.000000,0.000000}%
\pgfsetstrokecolor{currentstroke}%
\pgfsetdash{}{0pt}%
\pgfpathmoveto{\pgfqpoint{4.304800in}{3.011151in}}%
\pgfpathlineto{\pgfqpoint{4.317991in}{3.006165in}}%
\pgfpathlineto{\pgfqpoint{4.331188in}{3.001301in}}%
\pgfpathlineto{\pgfqpoint{4.344390in}{2.996556in}}%
\pgfpathlineto{\pgfqpoint{4.357598in}{2.991932in}}%
\pgfpathlineto{\pgfqpoint{4.350088in}{2.982154in}}%
\pgfpathlineto{\pgfqpoint{4.342573in}{2.972432in}}%
\pgfpathlineto{\pgfqpoint{4.335054in}{2.962763in}}%
\pgfpathlineto{\pgfqpoint{4.327530in}{2.953147in}}%
\pgfpathlineto{\pgfqpoint{4.314310in}{2.957611in}}%
\pgfpathlineto{\pgfqpoint{4.301097in}{2.962195in}}%
\pgfpathlineto{\pgfqpoint{4.287889in}{2.966900in}}%
\pgfpathlineto{\pgfqpoint{4.274686in}{2.971726in}}%
\pgfpathlineto{\pgfqpoint{4.282221in}{2.981496in}}%
\pgfpathlineto{\pgfqpoint{4.289752in}{2.991322in}}%
\pgfpathlineto{\pgfqpoint{4.297278in}{3.001207in}}%
\pgfpathlineto{\pgfqpoint{4.304800in}{3.011151in}}%
\pgfpathclose%
\pgfusepath{fill}%
\end{pgfscope}%
\begin{pgfscope}%
\pgfpathrectangle{\pgfqpoint{1.150000in}{0.150000in}}{\pgfqpoint{5.700000in}{5.700000in}}%
\pgfusepath{clip}%
\pgfsetbuttcap%
\pgfsetroundjoin%
\definecolor{currentfill}{rgb}{0.279574,0.170599,0.479997}%
\pgfsetfillcolor{currentfill}%
\pgfsetfillopacity{0.700000}%
\pgfsetlinewidth{0.000000pt}%
\definecolor{currentstroke}{rgb}{0.000000,0.000000,0.000000}%
\pgfsetstrokecolor{currentstroke}%
\pgfsetdash{}{0pt}%
\pgfpathmoveto{\pgfqpoint{3.627113in}{3.103825in}}%
\pgfpathlineto{\pgfqpoint{3.640217in}{3.093983in}}%
\pgfpathlineto{\pgfqpoint{3.653323in}{3.084296in}}%
\pgfpathlineto{\pgfqpoint{3.666430in}{3.074764in}}%
\pgfpathlineto{\pgfqpoint{3.679539in}{3.065385in}}%
\pgfpathlineto{\pgfqpoint{3.671819in}{3.055607in}}%
\pgfpathlineto{\pgfqpoint{3.664094in}{3.045896in}}%
\pgfpathlineto{\pgfqpoint{3.656364in}{3.036252in}}%
\pgfpathlineto{\pgfqpoint{3.648628in}{3.026674in}}%
\pgfpathlineto{\pgfqpoint{3.635507in}{3.035982in}}%
\pgfpathlineto{\pgfqpoint{3.622387in}{3.045443in}}%
\pgfpathlineto{\pgfqpoint{3.609269in}{3.055060in}}%
\pgfpathlineto{\pgfqpoint{3.596153in}{3.064831in}}%
\pgfpathlineto{\pgfqpoint{3.603901in}{3.074474in}}%
\pgfpathlineto{\pgfqpoint{3.611644in}{3.084186in}}%
\pgfpathlineto{\pgfqpoint{3.619381in}{3.093969in}}%
\pgfpathlineto{\pgfqpoint{3.627113in}{3.103825in}}%
\pgfpathclose%
\pgfusepath{fill}%
\end{pgfscope}%
\begin{pgfscope}%
\pgfpathrectangle{\pgfqpoint{1.150000in}{0.150000in}}{\pgfqpoint{5.700000in}{5.700000in}}%
\pgfusepath{clip}%
\pgfsetbuttcap%
\pgfsetroundjoin%
\definecolor{currentfill}{rgb}{0.283072,0.130895,0.449241}%
\pgfsetfillcolor{currentfill}%
\pgfsetfillopacity{0.700000}%
\pgfsetlinewidth{0.000000pt}%
\definecolor{currentstroke}{rgb}{0.000000,0.000000,0.000000}%
\pgfsetstrokecolor{currentstroke}%
\pgfsetdash{}{0pt}%
\pgfpathmoveto{\pgfqpoint{3.950793in}{3.013013in}}%
\pgfpathlineto{\pgfqpoint{3.963924in}{3.005887in}}%
\pgfpathlineto{\pgfqpoint{3.977059in}{2.998897in}}%
\pgfpathlineto{\pgfqpoint{3.990198in}{2.992042in}}%
\pgfpathlineto{\pgfqpoint{4.003341in}{2.985320in}}%
\pgfpathlineto{\pgfqpoint{3.995720in}{2.975519in}}%
\pgfpathlineto{\pgfqpoint{3.988094in}{2.965774in}}%
\pgfpathlineto{\pgfqpoint{3.980463in}{2.956083in}}%
\pgfpathlineto{\pgfqpoint{3.972827in}{2.946447in}}%
\pgfpathlineto{\pgfqpoint{3.959673in}{2.953061in}}%
\pgfpathlineto{\pgfqpoint{3.946523in}{2.959810in}}%
\pgfpathlineto{\pgfqpoint{3.933376in}{2.966694in}}%
\pgfpathlineto{\pgfqpoint{3.920234in}{2.973713in}}%
\pgfpathlineto{\pgfqpoint{3.927881in}{2.983450in}}%
\pgfpathlineto{\pgfqpoint{3.935523in}{2.993245in}}%
\pgfpathlineto{\pgfqpoint{3.943160in}{3.003099in}}%
\pgfpathlineto{\pgfqpoint{3.950793in}{3.013013in}}%
\pgfpathclose%
\pgfusepath{fill}%
\end{pgfscope}%
\begin{pgfscope}%
\pgfpathrectangle{\pgfqpoint{1.150000in}{0.150000in}}{\pgfqpoint{5.700000in}{5.700000in}}%
\pgfusepath{clip}%
\pgfsetbuttcap%
\pgfsetroundjoin%
\definecolor{currentfill}{rgb}{0.269308,0.218818,0.509577}%
\pgfsetfillcolor{currentfill}%
\pgfsetfillopacity{0.700000}%
\pgfsetlinewidth{0.000000pt}%
\definecolor{currentstroke}{rgb}{0.000000,0.000000,0.000000}%
\pgfsetstrokecolor{currentstroke}%
\pgfsetdash{}{0pt}%
\pgfpathmoveto{\pgfqpoint{3.438809in}{3.194701in}}%
\pgfpathlineto{\pgfqpoint{3.451920in}{3.182960in}}%
\pgfpathlineto{\pgfqpoint{3.465031in}{3.171391in}}%
\pgfpathlineto{\pgfqpoint{3.478141in}{3.159993in}}%
\pgfpathlineto{\pgfqpoint{3.491252in}{3.148762in}}%
\pgfpathlineto{\pgfqpoint{3.483472in}{3.139071in}}%
\pgfpathlineto{\pgfqpoint{3.475687in}{3.129456in}}%
\pgfpathlineto{\pgfqpoint{3.467896in}{3.119917in}}%
\pgfpathlineto{\pgfqpoint{3.460099in}{3.110453in}}%
\pgfpathlineto{\pgfqpoint{3.446975in}{3.121631in}}%
\pgfpathlineto{\pgfqpoint{3.433851in}{3.132978in}}%
\pgfpathlineto{\pgfqpoint{3.420727in}{3.144495in}}%
\pgfpathlineto{\pgfqpoint{3.407602in}{3.156184in}}%
\pgfpathlineto{\pgfqpoint{3.415413in}{3.165693in}}%
\pgfpathlineto{\pgfqpoint{3.423217in}{3.175282in}}%
\pgfpathlineto{\pgfqpoint{3.431016in}{3.184951in}}%
\pgfpathlineto{\pgfqpoint{3.438809in}{3.194701in}}%
\pgfpathclose%
\pgfusepath{fill}%
\end{pgfscope}%
\begin{pgfscope}%
\pgfpathrectangle{\pgfqpoint{1.150000in}{0.150000in}}{\pgfqpoint{5.700000in}{5.700000in}}%
\pgfusepath{clip}%
\pgfsetbuttcap%
\pgfsetroundjoin%
\definecolor{currentfill}{rgb}{0.282623,0.140926,0.457517}%
\pgfsetfillcolor{currentfill}%
\pgfsetfillopacity{0.700000}%
\pgfsetlinewidth{0.000000pt}%
\definecolor{currentstroke}{rgb}{0.000000,0.000000,0.000000}%
\pgfsetstrokecolor{currentstroke}%
\pgfsetdash{}{0pt}%
\pgfpathmoveto{\pgfqpoint{3.815212in}{3.034853in}}%
\pgfpathlineto{\pgfqpoint{3.828329in}{3.026717in}}%
\pgfpathlineto{\pgfqpoint{3.841449in}{3.018724in}}%
\pgfpathlineto{\pgfqpoint{3.854572in}{3.010873in}}%
\pgfpathlineto{\pgfqpoint{3.867698in}{3.003163in}}%
\pgfpathlineto{\pgfqpoint{3.860034in}{2.993387in}}%
\pgfpathlineto{\pgfqpoint{3.852365in}{2.983671in}}%
\pgfpathlineto{\pgfqpoint{3.844692in}{2.974012in}}%
\pgfpathlineto{\pgfqpoint{3.837013in}{2.964410in}}%
\pgfpathlineto{\pgfqpoint{3.823876in}{2.972032in}}%
\pgfpathlineto{\pgfqpoint{3.810741in}{2.979794in}}%
\pgfpathlineto{\pgfqpoint{3.797610in}{2.987698in}}%
\pgfpathlineto{\pgfqpoint{3.784481in}{2.995746in}}%
\pgfpathlineto{\pgfqpoint{3.792172in}{3.005429in}}%
\pgfpathlineto{\pgfqpoint{3.799857in}{3.015174in}}%
\pgfpathlineto{\pgfqpoint{3.807537in}{3.024982in}}%
\pgfpathlineto{\pgfqpoint{3.815212in}{3.034853in}}%
\pgfpathclose%
\pgfusepath{fill}%
\end{pgfscope}%
\begin{pgfscope}%
\pgfpathrectangle{\pgfqpoint{1.150000in}{0.150000in}}{\pgfqpoint{5.700000in}{5.700000in}}%
\pgfusepath{clip}%
\pgfsetbuttcap%
\pgfsetroundjoin%
\definecolor{currentfill}{rgb}{0.283187,0.125848,0.444960}%
\pgfsetfillcolor{currentfill}%
\pgfsetfillopacity{0.700000}%
\pgfsetlinewidth{0.000000pt}%
\definecolor{currentstroke}{rgb}{0.000000,0.000000,0.000000}%
\pgfsetstrokecolor{currentstroke}%
\pgfsetdash{}{0pt}%
\pgfpathmoveto{\pgfqpoint{4.086346in}{2.999058in}}%
\pgfpathlineto{\pgfqpoint{4.099499in}{2.992870in}}%
\pgfpathlineto{\pgfqpoint{4.112657in}{2.986811in}}%
\pgfpathlineto{\pgfqpoint{4.125819in}{2.980880in}}%
\pgfpathlineto{\pgfqpoint{4.138985in}{2.975077in}}%
\pgfpathlineto{\pgfqpoint{4.131405in}{2.965296in}}%
\pgfpathlineto{\pgfqpoint{4.123821in}{2.955570in}}%
\pgfpathlineto{\pgfqpoint{4.116232in}{2.945896in}}%
\pgfpathlineto{\pgfqpoint{4.108637in}{2.936273in}}%
\pgfpathlineto{\pgfqpoint{4.095460in}{2.941951in}}%
\pgfpathlineto{\pgfqpoint{4.082286in}{2.947757in}}%
\pgfpathlineto{\pgfqpoint{4.069118in}{2.953692in}}%
\pgfpathlineto{\pgfqpoint{4.055954in}{2.959756in}}%
\pgfpathlineto{\pgfqpoint{4.063559in}{2.969498in}}%
\pgfpathlineto{\pgfqpoint{4.071160in}{2.979294in}}%
\pgfpathlineto{\pgfqpoint{4.078755in}{2.989147in}}%
\pgfpathlineto{\pgfqpoint{4.086346in}{2.999058in}}%
\pgfpathclose%
\pgfusepath{fill}%
\end{pgfscope}%
\begin{pgfscope}%
\pgfpathrectangle{\pgfqpoint{1.150000in}{0.150000in}}{\pgfqpoint{5.700000in}{5.700000in}}%
\pgfusepath{clip}%
\pgfsetbuttcap%
\pgfsetroundjoin%
\definecolor{currentfill}{rgb}{0.282884,0.135920,0.453427}%
\pgfsetfillcolor{currentfill}%
\pgfsetfillopacity{0.700000}%
\pgfsetlinewidth{0.000000pt}%
\definecolor{currentstroke}{rgb}{0.000000,0.000000,0.000000}%
\pgfsetstrokecolor{currentstroke}%
\pgfsetdash{}{0pt}%
\pgfpathmoveto{\pgfqpoint{4.440445in}{3.013625in}}%
\pgfpathlineto{\pgfqpoint{4.453673in}{3.009414in}}%
\pgfpathlineto{\pgfqpoint{4.466906in}{3.005320in}}%
\pgfpathlineto{\pgfqpoint{4.480146in}{3.001342in}}%
\pgfpathlineto{\pgfqpoint{4.493393in}{2.997480in}}%
\pgfpathlineto{\pgfqpoint{4.485922in}{2.987826in}}%
\pgfpathlineto{\pgfqpoint{4.478448in}{2.978228in}}%
\pgfpathlineto{\pgfqpoint{4.470969in}{2.968683in}}%
\pgfpathlineto{\pgfqpoint{4.463486in}{2.959190in}}%
\pgfpathlineto{\pgfqpoint{4.450227in}{2.962874in}}%
\pgfpathlineto{\pgfqpoint{4.436976in}{2.966674in}}%
\pgfpathlineto{\pgfqpoint{4.423731in}{2.970590in}}%
\pgfpathlineto{\pgfqpoint{4.410492in}{2.974623in}}%
\pgfpathlineto{\pgfqpoint{4.417987in}{2.984288in}}%
\pgfpathlineto{\pgfqpoint{4.425477in}{2.994008in}}%
\pgfpathlineto{\pgfqpoint{4.432963in}{3.003786in}}%
\pgfpathlineto{\pgfqpoint{4.440445in}{3.013625in}}%
\pgfpathclose%
\pgfusepath{fill}%
\end{pgfscope}%
\begin{pgfscope}%
\pgfpathrectangle{\pgfqpoint{1.150000in}{0.150000in}}{\pgfqpoint{5.700000in}{5.700000in}}%
\pgfusepath{clip}%
\pgfsetbuttcap%
\pgfsetroundjoin%
\definecolor{currentfill}{rgb}{0.283187,0.125848,0.444960}%
\pgfsetfillcolor{currentfill}%
\pgfsetfillopacity{0.700000}%
\pgfsetlinewidth{0.000000pt}%
\definecolor{currentstroke}{rgb}{0.000000,0.000000,0.000000}%
\pgfsetstrokecolor{currentstroke}%
\pgfsetdash{}{0pt}%
\pgfpathmoveto{\pgfqpoint{4.221931in}{2.992254in}}%
\pgfpathlineto{\pgfqpoint{4.235112in}{2.986938in}}%
\pgfpathlineto{\pgfqpoint{4.248298in}{2.981744in}}%
\pgfpathlineto{\pgfqpoint{4.261489in}{2.976674in}}%
\pgfpathlineto{\pgfqpoint{4.274686in}{2.971726in}}%
\pgfpathlineto{\pgfqpoint{4.267147in}{2.962010in}}%
\pgfpathlineto{\pgfqpoint{4.259603in}{2.952347in}}%
\pgfpathlineto{\pgfqpoint{4.252054in}{2.942734in}}%
\pgfpathlineto{\pgfqpoint{4.244501in}{2.933169in}}%
\pgfpathlineto{\pgfqpoint{4.231293in}{2.937975in}}%
\pgfpathlineto{\pgfqpoint{4.218091in}{2.942902in}}%
\pgfpathlineto{\pgfqpoint{4.204894in}{2.947953in}}%
\pgfpathlineto{\pgfqpoint{4.191702in}{2.953128in}}%
\pgfpathlineto{\pgfqpoint{4.199266in}{2.962828in}}%
\pgfpathlineto{\pgfqpoint{4.206826in}{2.972582in}}%
\pgfpathlineto{\pgfqpoint{4.214381in}{2.982390in}}%
\pgfpathlineto{\pgfqpoint{4.221931in}{2.992254in}}%
\pgfpathclose%
\pgfusepath{fill}%
\end{pgfscope}%
\begin{pgfscope}%
\pgfpathrectangle{\pgfqpoint{1.150000in}{0.150000in}}{\pgfqpoint{5.700000in}{5.700000in}}%
\pgfusepath{clip}%
\pgfsetbuttcap%
\pgfsetroundjoin%
\definecolor{currentfill}{rgb}{0.281412,0.155834,0.469201}%
\pgfsetfillcolor{currentfill}%
\pgfsetfillopacity{0.700000}%
\pgfsetlinewidth{0.000000pt}%
\definecolor{currentstroke}{rgb}{0.000000,0.000000,0.000000}%
\pgfsetstrokecolor{currentstroke}%
\pgfsetdash{}{0pt}%
\pgfpathmoveto{\pgfqpoint{3.679539in}{3.065385in}}%
\pgfpathlineto{\pgfqpoint{3.692649in}{3.056159in}}%
\pgfpathlineto{\pgfqpoint{3.705761in}{3.047084in}}%
\pgfpathlineto{\pgfqpoint{3.718876in}{3.038159in}}%
\pgfpathlineto{\pgfqpoint{3.731992in}{3.029382in}}%
\pgfpathlineto{\pgfqpoint{3.724284in}{3.019681in}}%
\pgfpathlineto{\pgfqpoint{3.716572in}{3.010043in}}%
\pgfpathlineto{\pgfqpoint{3.708853in}{3.000468in}}%
\pgfpathlineto{\pgfqpoint{3.701130in}{2.990953in}}%
\pgfpathlineto{\pgfqpoint{3.688001in}{2.999659in}}%
\pgfpathlineto{\pgfqpoint{3.674875in}{3.008513in}}%
\pgfpathlineto{\pgfqpoint{3.661750in}{3.017518in}}%
\pgfpathlineto{\pgfqpoint{3.648628in}{3.026674in}}%
\pgfpathlineto{\pgfqpoint{3.656364in}{3.036252in}}%
\pgfpathlineto{\pgfqpoint{3.664094in}{3.045896in}}%
\pgfpathlineto{\pgfqpoint{3.671819in}{3.055607in}}%
\pgfpathlineto{\pgfqpoint{3.679539in}{3.065385in}}%
\pgfpathclose%
\pgfusepath{fill}%
\end{pgfscope}%
\begin{pgfscope}%
\pgfpathrectangle{\pgfqpoint{1.150000in}{0.150000in}}{\pgfqpoint{5.700000in}{5.700000in}}%
\pgfusepath{clip}%
\pgfsetbuttcap%
\pgfsetroundjoin%
\definecolor{currentfill}{rgb}{0.274128,0.199721,0.498911}%
\pgfsetfillcolor{currentfill}%
\pgfsetfillopacity{0.700000}%
\pgfsetlinewidth{0.000000pt}%
\definecolor{currentstroke}{rgb}{0.000000,0.000000,0.000000}%
\pgfsetstrokecolor{currentstroke}%
\pgfsetdash{}{0pt}%
\pgfpathmoveto{\pgfqpoint{3.491252in}{3.148762in}}%
\pgfpathlineto{\pgfqpoint{3.504363in}{3.137699in}}%
\pgfpathlineto{\pgfqpoint{3.517474in}{3.126803in}}%
\pgfpathlineto{\pgfqpoint{3.530585in}{3.116070in}}%
\pgfpathlineto{\pgfqpoint{3.543697in}{3.105501in}}%
\pgfpathlineto{\pgfqpoint{3.535930in}{3.095868in}}%
\pgfpathlineto{\pgfqpoint{3.528158in}{3.086308in}}%
\pgfpathlineto{\pgfqpoint{3.520380in}{3.076819in}}%
\pgfpathlineto{\pgfqpoint{3.512596in}{3.067401in}}%
\pgfpathlineto{\pgfqpoint{3.499471in}{3.077918in}}%
\pgfpathlineto{\pgfqpoint{3.486347in}{3.088598in}}%
\pgfpathlineto{\pgfqpoint{3.473223in}{3.099443in}}%
\pgfpathlineto{\pgfqpoint{3.460099in}{3.110453in}}%
\pgfpathlineto{\pgfqpoint{3.467896in}{3.119917in}}%
\pgfpathlineto{\pgfqpoint{3.475687in}{3.129456in}}%
\pgfpathlineto{\pgfqpoint{3.483472in}{3.139071in}}%
\pgfpathlineto{\pgfqpoint{3.491252in}{3.148762in}}%
\pgfpathclose%
\pgfusepath{fill}%
\end{pgfscope}%
\begin{pgfscope}%
\pgfpathrectangle{\pgfqpoint{1.150000in}{0.150000in}}{\pgfqpoint{5.700000in}{5.700000in}}%
\pgfusepath{clip}%
\pgfsetbuttcap%
\pgfsetroundjoin%
\definecolor{currentfill}{rgb}{0.283187,0.125848,0.444960}%
\pgfsetfillcolor{currentfill}%
\pgfsetfillopacity{0.700000}%
\pgfsetlinewidth{0.000000pt}%
\definecolor{currentstroke}{rgb}{0.000000,0.000000,0.000000}%
\pgfsetstrokecolor{currentstroke}%
\pgfsetdash{}{0pt}%
\pgfpathmoveto{\pgfqpoint{4.357598in}{2.991932in}}%
\pgfpathlineto{\pgfqpoint{4.370813in}{2.987427in}}%
\pgfpathlineto{\pgfqpoint{4.384033in}{2.983041in}}%
\pgfpathlineto{\pgfqpoint{4.397259in}{2.978773in}}%
\pgfpathlineto{\pgfqpoint{4.410492in}{2.974623in}}%
\pgfpathlineto{\pgfqpoint{4.402993in}{2.965012in}}%
\pgfpathlineto{\pgfqpoint{4.395489in}{2.955452in}}%
\pgfpathlineto{\pgfqpoint{4.387981in}{2.945942in}}%
\pgfpathlineto{\pgfqpoint{4.380469in}{2.936480in}}%
\pgfpathlineto{\pgfqpoint{4.367225in}{2.940469in}}%
\pgfpathlineto{\pgfqpoint{4.353987in}{2.944577in}}%
\pgfpathlineto{\pgfqpoint{4.340756in}{2.948802in}}%
\pgfpathlineto{\pgfqpoint{4.327530in}{2.953147in}}%
\pgfpathlineto{\pgfqpoint{4.335054in}{2.962763in}}%
\pgfpathlineto{\pgfqpoint{4.342573in}{2.972432in}}%
\pgfpathlineto{\pgfqpoint{4.350088in}{2.982154in}}%
\pgfpathlineto{\pgfqpoint{4.357598in}{2.991932in}}%
\pgfpathclose%
\pgfusepath{fill}%
\end{pgfscope}%
\begin{pgfscope}%
\pgfpathrectangle{\pgfqpoint{1.150000in}{0.150000in}}{\pgfqpoint{5.700000in}{5.700000in}}%
\pgfusepath{clip}%
\pgfsetbuttcap%
\pgfsetroundjoin%
\definecolor{currentfill}{rgb}{0.283072,0.130895,0.449241}%
\pgfsetfillcolor{currentfill}%
\pgfsetfillopacity{0.700000}%
\pgfsetlinewidth{0.000000pt}%
\definecolor{currentstroke}{rgb}{0.000000,0.000000,0.000000}%
\pgfsetstrokecolor{currentstroke}%
\pgfsetdash{}{0pt}%
\pgfpathmoveto{\pgfqpoint{3.867698in}{3.003163in}}%
\pgfpathlineto{\pgfqpoint{3.880827in}{2.995593in}}%
\pgfpathlineto{\pgfqpoint{3.893959in}{2.988162in}}%
\pgfpathlineto{\pgfqpoint{3.907095in}{2.980869in}}%
\pgfpathlineto{\pgfqpoint{3.920234in}{2.973713in}}%
\pgfpathlineto{\pgfqpoint{3.912582in}{2.964033in}}%
\pgfpathlineto{\pgfqpoint{3.904924in}{2.954407in}}%
\pgfpathlineto{\pgfqpoint{3.897262in}{2.944835in}}%
\pgfpathlineto{\pgfqpoint{3.889595in}{2.935316in}}%
\pgfpathlineto{\pgfqpoint{3.876445in}{2.942383in}}%
\pgfpathlineto{\pgfqpoint{3.863297in}{2.949587in}}%
\pgfpathlineto{\pgfqpoint{3.850154in}{2.956929in}}%
\pgfpathlineto{\pgfqpoint{3.837013in}{2.964410in}}%
\pgfpathlineto{\pgfqpoint{3.844692in}{2.974012in}}%
\pgfpathlineto{\pgfqpoint{3.852365in}{2.983671in}}%
\pgfpathlineto{\pgfqpoint{3.860034in}{2.993387in}}%
\pgfpathlineto{\pgfqpoint{3.867698in}{3.003163in}}%
\pgfpathclose%
\pgfusepath{fill}%
\end{pgfscope}%
\begin{pgfscope}%
\pgfpathrectangle{\pgfqpoint{1.150000in}{0.150000in}}{\pgfqpoint{5.700000in}{5.700000in}}%
\pgfusepath{clip}%
\pgfsetbuttcap%
\pgfsetroundjoin%
\definecolor{currentfill}{rgb}{0.283229,0.120777,0.440584}%
\pgfsetfillcolor{currentfill}%
\pgfsetfillopacity{0.700000}%
\pgfsetlinewidth{0.000000pt}%
\definecolor{currentstroke}{rgb}{0.000000,0.000000,0.000000}%
\pgfsetstrokecolor{currentstroke}%
\pgfsetdash{}{0pt}%
\pgfpathmoveto{\pgfqpoint{4.003341in}{2.985320in}}%
\pgfpathlineto{\pgfqpoint{4.016488in}{2.978731in}}%
\pgfpathlineto{\pgfqpoint{4.029639in}{2.972275in}}%
\pgfpathlineto{\pgfqpoint{4.042794in}{2.965950in}}%
\pgfpathlineto{\pgfqpoint{4.055954in}{2.959756in}}%
\pgfpathlineto{\pgfqpoint{4.048344in}{2.950068in}}%
\pgfpathlineto{\pgfqpoint{4.040729in}{2.940432in}}%
\pgfpathlineto{\pgfqpoint{4.033109in}{2.930847in}}%
\pgfpathlineto{\pgfqpoint{4.025485in}{2.921310in}}%
\pgfpathlineto{\pgfqpoint{4.012314in}{2.927397in}}%
\pgfpathlineto{\pgfqpoint{3.999148in}{2.933615in}}%
\pgfpathlineto{\pgfqpoint{3.985985in}{2.939965in}}%
\pgfpathlineto{\pgfqpoint{3.972827in}{2.946447in}}%
\pgfpathlineto{\pgfqpoint{3.980463in}{2.956083in}}%
\pgfpathlineto{\pgfqpoint{3.988094in}{2.965774in}}%
\pgfpathlineto{\pgfqpoint{3.995720in}{2.975519in}}%
\pgfpathlineto{\pgfqpoint{4.003341in}{2.985320in}}%
\pgfpathclose%
\pgfusepath{fill}%
\end{pgfscope}%
\begin{pgfscope}%
\pgfpathrectangle{\pgfqpoint{1.150000in}{0.150000in}}{\pgfqpoint{5.700000in}{5.700000in}}%
\pgfusepath{clip}%
\pgfsetbuttcap%
\pgfsetroundjoin%
\definecolor{currentfill}{rgb}{0.282290,0.145912,0.461510}%
\pgfsetfillcolor{currentfill}%
\pgfsetfillopacity{0.700000}%
\pgfsetlinewidth{0.000000pt}%
\definecolor{currentstroke}{rgb}{0.000000,0.000000,0.000000}%
\pgfsetstrokecolor{currentstroke}%
\pgfsetdash{}{0pt}%
\pgfpathmoveto{\pgfqpoint{4.576239in}{3.021613in}}%
\pgfpathlineto{\pgfqpoint{4.589507in}{3.018126in}}%
\pgfpathlineto{\pgfqpoint{4.602782in}{3.014751in}}%
\pgfpathlineto{\pgfqpoint{4.616064in}{3.011489in}}%
\pgfpathlineto{\pgfqpoint{4.629354in}{3.008338in}}%
\pgfpathlineto{\pgfqpoint{4.621923in}{2.998843in}}%
\pgfpathlineto{\pgfqpoint{4.614489in}{2.989404in}}%
\pgfpathlineto{\pgfqpoint{4.607051in}{2.980019in}}%
\pgfpathlineto{\pgfqpoint{4.599609in}{2.970687in}}%
\pgfpathlineto{\pgfqpoint{4.586307in}{2.973641in}}%
\pgfpathlineto{\pgfqpoint{4.573013in}{2.976707in}}%
\pgfpathlineto{\pgfqpoint{4.559726in}{2.979885in}}%
\pgfpathlineto{\pgfqpoint{4.546445in}{2.983177in}}%
\pgfpathlineto{\pgfqpoint{4.553900in}{2.992699in}}%
\pgfpathlineto{\pgfqpoint{4.561350in}{3.002278in}}%
\pgfpathlineto{\pgfqpoint{4.568796in}{3.011915in}}%
\pgfpathlineto{\pgfqpoint{4.576239in}{3.021613in}}%
\pgfpathclose%
\pgfusepath{fill}%
\end{pgfscope}%
\begin{pgfscope}%
\pgfpathrectangle{\pgfqpoint{1.150000in}{0.150000in}}{\pgfqpoint{5.700000in}{5.700000in}}%
\pgfusepath{clip}%
\pgfsetbuttcap%
\pgfsetroundjoin%
\definecolor{currentfill}{rgb}{0.250425,0.274290,0.533103}%
\pgfsetfillcolor{currentfill}%
\pgfsetfillopacity{0.700000}%
\pgfsetlinewidth{0.000000pt}%
\definecolor{currentstroke}{rgb}{0.000000,0.000000,0.000000}%
\pgfsetstrokecolor{currentstroke}%
\pgfsetdash{}{0pt}%
\pgfpathmoveto{\pgfqpoint{3.250025in}{3.310396in}}%
\pgfpathlineto{\pgfqpoint{3.263166in}{3.296526in}}%
\pgfpathlineto{\pgfqpoint{3.276304in}{3.282846in}}%
\pgfpathlineto{\pgfqpoint{3.289440in}{3.269356in}}%
\pgfpathlineto{\pgfqpoint{3.302575in}{3.256054in}}%
\pgfpathlineto{\pgfqpoint{3.294730in}{3.246542in}}%
\pgfpathlineto{\pgfqpoint{3.286879in}{3.237117in}}%
\pgfpathlineto{\pgfqpoint{3.279021in}{3.227779in}}%
\pgfpathlineto{\pgfqpoint{3.271157in}{3.218525in}}%
\pgfpathlineto{\pgfqpoint{3.258008in}{3.231794in}}%
\pgfpathlineto{\pgfqpoint{3.244857in}{3.245251in}}%
\pgfpathlineto{\pgfqpoint{3.231703in}{3.258897in}}%
\pgfpathlineto{\pgfqpoint{3.218547in}{3.272734in}}%
\pgfpathlineto{\pgfqpoint{3.226427in}{3.282014in}}%
\pgfpathlineto{\pgfqpoint{3.234299in}{3.291384in}}%
\pgfpathlineto{\pgfqpoint{3.242165in}{3.300844in}}%
\pgfpathlineto{\pgfqpoint{3.250025in}{3.310396in}}%
\pgfpathclose%
\pgfusepath{fill}%
\end{pgfscope}%
\begin{pgfscope}%
\pgfpathrectangle{\pgfqpoint{1.150000in}{0.150000in}}{\pgfqpoint{5.700000in}{5.700000in}}%
\pgfusepath{clip}%
\pgfsetbuttcap%
\pgfsetroundjoin%
\definecolor{currentfill}{rgb}{0.278012,0.180367,0.486697}%
\pgfsetfillcolor{currentfill}%
\pgfsetfillopacity{0.700000}%
\pgfsetlinewidth{0.000000pt}%
\definecolor{currentstroke}{rgb}{0.000000,0.000000,0.000000}%
\pgfsetstrokecolor{currentstroke}%
\pgfsetdash{}{0pt}%
\pgfpathmoveto{\pgfqpoint{3.543697in}{3.105501in}}%
\pgfpathlineto{\pgfqpoint{3.556810in}{3.095094in}}%
\pgfpathlineto{\pgfqpoint{3.569923in}{3.084847in}}%
\pgfpathlineto{\pgfqpoint{3.583037in}{3.074760in}}%
\pgfpathlineto{\pgfqpoint{3.596153in}{3.064831in}}%
\pgfpathlineto{\pgfqpoint{3.588399in}{3.055258in}}%
\pgfpathlineto{\pgfqpoint{3.580640in}{3.045752in}}%
\pgfpathlineto{\pgfqpoint{3.572875in}{3.036314in}}%
\pgfpathlineto{\pgfqpoint{3.565104in}{3.026942in}}%
\pgfpathlineto{\pgfqpoint{3.551976in}{3.036818in}}%
\pgfpathlineto{\pgfqpoint{3.538848in}{3.046852in}}%
\pgfpathlineto{\pgfqpoint{3.525722in}{3.057046in}}%
\pgfpathlineto{\pgfqpoint{3.512596in}{3.067401in}}%
\pgfpathlineto{\pgfqpoint{3.520380in}{3.076819in}}%
\pgfpathlineto{\pgfqpoint{3.528158in}{3.086308in}}%
\pgfpathlineto{\pgfqpoint{3.535930in}{3.095868in}}%
\pgfpathlineto{\pgfqpoint{3.543697in}{3.105501in}}%
\pgfpathclose%
\pgfusepath{fill}%
\end{pgfscope}%
\begin{pgfscope}%
\pgfpathrectangle{\pgfqpoint{1.150000in}{0.150000in}}{\pgfqpoint{5.700000in}{5.700000in}}%
\pgfusepath{clip}%
\pgfsetbuttcap%
\pgfsetroundjoin%
\definecolor{currentfill}{rgb}{0.282290,0.145912,0.461510}%
\pgfsetfillcolor{currentfill}%
\pgfsetfillopacity{0.700000}%
\pgfsetlinewidth{0.000000pt}%
\definecolor{currentstroke}{rgb}{0.000000,0.000000,0.000000}%
\pgfsetstrokecolor{currentstroke}%
\pgfsetdash{}{0pt}%
\pgfpathmoveto{\pgfqpoint{3.731992in}{3.029382in}}%
\pgfpathlineto{\pgfqpoint{3.745111in}{3.020754in}}%
\pgfpathlineto{\pgfqpoint{3.758232in}{3.012272in}}%
\pgfpathlineto{\pgfqpoint{3.771355in}{3.003937in}}%
\pgfpathlineto{\pgfqpoint{3.784481in}{2.995746in}}%
\pgfpathlineto{\pgfqpoint{3.776785in}{2.986122in}}%
\pgfpathlineto{\pgfqpoint{3.769085in}{2.976557in}}%
\pgfpathlineto{\pgfqpoint{3.761378in}{2.967050in}}%
\pgfpathlineto{\pgfqpoint{3.753667in}{2.957600in}}%
\pgfpathlineto{\pgfqpoint{3.740529in}{2.965720in}}%
\pgfpathlineto{\pgfqpoint{3.727394in}{2.973985in}}%
\pgfpathlineto{\pgfqpoint{3.714261in}{2.982395in}}%
\pgfpathlineto{\pgfqpoint{3.701130in}{2.990953in}}%
\pgfpathlineto{\pgfqpoint{3.708853in}{3.000468in}}%
\pgfpathlineto{\pgfqpoint{3.716572in}{3.010043in}}%
\pgfpathlineto{\pgfqpoint{3.724284in}{3.019681in}}%
\pgfpathlineto{\pgfqpoint{3.731992in}{3.029382in}}%
\pgfpathclose%
\pgfusepath{fill}%
\end{pgfscope}%
\begin{pgfscope}%
\pgfpathrectangle{\pgfqpoint{1.150000in}{0.150000in}}{\pgfqpoint{5.700000in}{5.700000in}}%
\pgfusepath{clip}%
\pgfsetbuttcap%
\pgfsetroundjoin%
\definecolor{currentfill}{rgb}{0.283229,0.120777,0.440584}%
\pgfsetfillcolor{currentfill}%
\pgfsetfillopacity{0.700000}%
\pgfsetlinewidth{0.000000pt}%
\definecolor{currentstroke}{rgb}{0.000000,0.000000,0.000000}%
\pgfsetstrokecolor{currentstroke}%
\pgfsetdash{}{0pt}%
\pgfpathmoveto{\pgfqpoint{4.138985in}{2.975077in}}%
\pgfpathlineto{\pgfqpoint{4.152157in}{2.969401in}}%
\pgfpathlineto{\pgfqpoint{4.165334in}{2.963851in}}%
\pgfpathlineto{\pgfqpoint{4.178515in}{2.958427in}}%
\pgfpathlineto{\pgfqpoint{4.191702in}{2.953128in}}%
\pgfpathlineto{\pgfqpoint{4.184133in}{2.943479in}}%
\pgfpathlineto{\pgfqpoint{4.176559in}{2.933880in}}%
\pgfpathlineto{\pgfqpoint{4.168981in}{2.924328in}}%
\pgfpathlineto{\pgfqpoint{4.161398in}{2.914823in}}%
\pgfpathlineto{\pgfqpoint{4.148201in}{2.919997in}}%
\pgfpathlineto{\pgfqpoint{4.135008in}{2.925296in}}%
\pgfpathlineto{\pgfqpoint{4.121820in}{2.930721in}}%
\pgfpathlineto{\pgfqpoint{4.108637in}{2.936273in}}%
\pgfpathlineto{\pgfqpoint{4.116232in}{2.945896in}}%
\pgfpathlineto{\pgfqpoint{4.123821in}{2.955570in}}%
\pgfpathlineto{\pgfqpoint{4.131405in}{2.965296in}}%
\pgfpathlineto{\pgfqpoint{4.138985in}{2.975077in}}%
\pgfpathclose%
\pgfusepath{fill}%
\end{pgfscope}%
\begin{pgfscope}%
\pgfpathrectangle{\pgfqpoint{1.150000in}{0.150000in}}{\pgfqpoint{5.700000in}{5.700000in}}%
\pgfusepath{clip}%
\pgfsetbuttcap%
\pgfsetroundjoin%
\definecolor{currentfill}{rgb}{0.258965,0.251537,0.524736}%
\pgfsetfillcolor{currentfill}%
\pgfsetfillopacity{0.700000}%
\pgfsetlinewidth{0.000000pt}%
\definecolor{currentstroke}{rgb}{0.000000,0.000000,0.000000}%
\pgfsetstrokecolor{currentstroke}%
\pgfsetdash{}{0pt}%
\pgfpathmoveto{\pgfqpoint{3.302575in}{3.256054in}}%
\pgfpathlineto{\pgfqpoint{3.315708in}{3.242937in}}%
\pgfpathlineto{\pgfqpoint{3.328839in}{3.230004in}}%
\pgfpathlineto{\pgfqpoint{3.341969in}{3.217254in}}%
\pgfpathlineto{\pgfqpoint{3.355097in}{3.204685in}}%
\pgfpathlineto{\pgfqpoint{3.347267in}{3.195214in}}%
\pgfpathlineto{\pgfqpoint{3.339430in}{3.185825in}}%
\pgfpathlineto{\pgfqpoint{3.331587in}{3.176517in}}%
\pgfpathlineto{\pgfqpoint{3.323738in}{3.167291in}}%
\pgfpathlineto{\pgfqpoint{3.310595in}{3.179827in}}%
\pgfpathlineto{\pgfqpoint{3.297450in}{3.192543in}}%
\pgfpathlineto{\pgfqpoint{3.284304in}{3.205442in}}%
\pgfpathlineto{\pgfqpoint{3.271157in}{3.218525in}}%
\pgfpathlineto{\pgfqpoint{3.279021in}{3.227779in}}%
\pgfpathlineto{\pgfqpoint{3.286879in}{3.237117in}}%
\pgfpathlineto{\pgfqpoint{3.294730in}{3.246542in}}%
\pgfpathlineto{\pgfqpoint{3.302575in}{3.256054in}}%
\pgfpathclose%
\pgfusepath{fill}%
\end{pgfscope}%
\begin{pgfscope}%
\pgfpathrectangle{\pgfqpoint{1.150000in}{0.150000in}}{\pgfqpoint{5.700000in}{5.700000in}}%
\pgfusepath{clip}%
\pgfsetbuttcap%
\pgfsetroundjoin%
\definecolor{currentfill}{rgb}{0.265145,0.232956,0.516599}%
\pgfsetfillcolor{currentfill}%
\pgfsetfillopacity{0.700000}%
\pgfsetlinewidth{0.000000pt}%
\definecolor{currentstroke}{rgb}{0.000000,0.000000,0.000000}%
\pgfsetstrokecolor{currentstroke}%
\pgfsetdash{}{0pt}%
\pgfpathmoveto{\pgfqpoint{3.355097in}{3.204685in}}%
\pgfpathlineto{\pgfqpoint{3.368225in}{3.192295in}}%
\pgfpathlineto{\pgfqpoint{3.381351in}{3.180082in}}%
\pgfpathlineto{\pgfqpoint{3.394477in}{3.168046in}}%
\pgfpathlineto{\pgfqpoint{3.407602in}{3.156184in}}%
\pgfpathlineto{\pgfqpoint{3.399785in}{3.146753in}}%
\pgfpathlineto{\pgfqpoint{3.391963in}{3.137401in}}%
\pgfpathlineto{\pgfqpoint{3.384134in}{3.128125in}}%
\pgfpathlineto{\pgfqpoint{3.376299in}{3.118926in}}%
\pgfpathlineto{\pgfqpoint{3.363160in}{3.130754in}}%
\pgfpathlineto{\pgfqpoint{3.350020in}{3.142756in}}%
\pgfpathlineto{\pgfqpoint{3.336879in}{3.154935in}}%
\pgfpathlineto{\pgfqpoint{3.323738in}{3.167291in}}%
\pgfpathlineto{\pgfqpoint{3.331587in}{3.176517in}}%
\pgfpathlineto{\pgfqpoint{3.339430in}{3.185825in}}%
\pgfpathlineto{\pgfqpoint{3.347267in}{3.195214in}}%
\pgfpathlineto{\pgfqpoint{3.355097in}{3.204685in}}%
\pgfpathclose%
\pgfusepath{fill}%
\end{pgfscope}%
\begin{pgfscope}%
\pgfpathrectangle{\pgfqpoint{1.150000in}{0.150000in}}{\pgfqpoint{5.700000in}{5.700000in}}%
\pgfusepath{clip}%
\pgfsetbuttcap%
\pgfsetroundjoin%
\definecolor{currentfill}{rgb}{0.282884,0.135920,0.453427}%
\pgfsetfillcolor{currentfill}%
\pgfsetfillopacity{0.700000}%
\pgfsetlinewidth{0.000000pt}%
\definecolor{currentstroke}{rgb}{0.000000,0.000000,0.000000}%
\pgfsetstrokecolor{currentstroke}%
\pgfsetdash{}{0pt}%
\pgfpathmoveto{\pgfqpoint{4.493393in}{2.997480in}}%
\pgfpathlineto{\pgfqpoint{4.506646in}{2.993732in}}%
\pgfpathlineto{\pgfqpoint{4.519906in}{2.990100in}}%
\pgfpathlineto{\pgfqpoint{4.533172in}{2.986581in}}%
\pgfpathlineto{\pgfqpoint{4.546445in}{2.983177in}}%
\pgfpathlineto{\pgfqpoint{4.538987in}{2.973708in}}%
\pgfpathlineto{\pgfqpoint{4.531524in}{2.964290in}}%
\pgfpathlineto{\pgfqpoint{4.524057in}{2.954922in}}%
\pgfpathlineto{\pgfqpoint{4.516585in}{2.945601in}}%
\pgfpathlineto{\pgfqpoint{4.503300in}{2.948827in}}%
\pgfpathlineto{\pgfqpoint{4.490022in}{2.952167in}}%
\pgfpathlineto{\pgfqpoint{4.476750in}{2.955621in}}%
\pgfpathlineto{\pgfqpoint{4.463486in}{2.959190in}}%
\pgfpathlineto{\pgfqpoint{4.470969in}{2.968683in}}%
\pgfpathlineto{\pgfqpoint{4.478448in}{2.978228in}}%
\pgfpathlineto{\pgfqpoint{4.485922in}{2.987826in}}%
\pgfpathlineto{\pgfqpoint{4.493393in}{2.997480in}}%
\pgfpathclose%
\pgfusepath{fill}%
\end{pgfscope}%
\begin{pgfscope}%
\pgfpathrectangle{\pgfqpoint{1.150000in}{0.150000in}}{\pgfqpoint{5.700000in}{5.700000in}}%
\pgfusepath{clip}%
\pgfsetbuttcap%
\pgfsetroundjoin%
\definecolor{currentfill}{rgb}{0.283229,0.120777,0.440584}%
\pgfsetfillcolor{currentfill}%
\pgfsetfillopacity{0.700000}%
\pgfsetlinewidth{0.000000pt}%
\definecolor{currentstroke}{rgb}{0.000000,0.000000,0.000000}%
\pgfsetstrokecolor{currentstroke}%
\pgfsetdash{}{0pt}%
\pgfpathmoveto{\pgfqpoint{4.274686in}{2.971726in}}%
\pgfpathlineto{\pgfqpoint{4.287889in}{2.966900in}}%
\pgfpathlineto{\pgfqpoint{4.301097in}{2.962195in}}%
\pgfpathlineto{\pgfqpoint{4.314310in}{2.957611in}}%
\pgfpathlineto{\pgfqpoint{4.327530in}{2.953147in}}%
\pgfpathlineto{\pgfqpoint{4.320002in}{2.943580in}}%
\pgfpathlineto{\pgfqpoint{4.312469in}{2.934062in}}%
\pgfpathlineto{\pgfqpoint{4.304932in}{2.924589in}}%
\pgfpathlineto{\pgfqpoint{4.297390in}{2.915161in}}%
\pgfpathlineto{\pgfqpoint{4.284159in}{2.919483in}}%
\pgfpathlineto{\pgfqpoint{4.270934in}{2.923924in}}%
\pgfpathlineto{\pgfqpoint{4.257715in}{2.928486in}}%
\pgfpathlineto{\pgfqpoint{4.244501in}{2.933169in}}%
\pgfpathlineto{\pgfqpoint{4.252054in}{2.942734in}}%
\pgfpathlineto{\pgfqpoint{4.259603in}{2.952347in}}%
\pgfpathlineto{\pgfqpoint{4.267147in}{2.962010in}}%
\pgfpathlineto{\pgfqpoint{4.274686in}{2.971726in}}%
\pgfpathclose%
\pgfusepath{fill}%
\end{pgfscope}%
\begin{pgfscope}%
\pgfpathrectangle{\pgfqpoint{1.150000in}{0.150000in}}{\pgfqpoint{5.700000in}{5.700000in}}%
\pgfusepath{clip}%
\pgfsetbuttcap%
\pgfsetroundjoin%
\definecolor{currentfill}{rgb}{0.280255,0.165693,0.476498}%
\pgfsetfillcolor{currentfill}%
\pgfsetfillopacity{0.700000}%
\pgfsetlinewidth{0.000000pt}%
\definecolor{currentstroke}{rgb}{0.000000,0.000000,0.000000}%
\pgfsetstrokecolor{currentstroke}%
\pgfsetdash{}{0pt}%
\pgfpathmoveto{\pgfqpoint{3.596153in}{3.064831in}}%
\pgfpathlineto{\pgfqpoint{3.609269in}{3.055060in}}%
\pgfpathlineto{\pgfqpoint{3.622387in}{3.045443in}}%
\pgfpathlineto{\pgfqpoint{3.635507in}{3.035982in}}%
\pgfpathlineto{\pgfqpoint{3.648628in}{3.026674in}}%
\pgfpathlineto{\pgfqpoint{3.640887in}{3.017160in}}%
\pgfpathlineto{\pgfqpoint{3.633140in}{3.007709in}}%
\pgfpathlineto{\pgfqpoint{3.625388in}{2.998321in}}%
\pgfpathlineto{\pgfqpoint{3.617630in}{2.988995in}}%
\pgfpathlineto{\pgfqpoint{3.604496in}{2.998250in}}%
\pgfpathlineto{\pgfqpoint{3.591364in}{3.007659in}}%
\pgfpathlineto{\pgfqpoint{3.578233in}{3.017223in}}%
\pgfpathlineto{\pgfqpoint{3.565104in}{3.026942in}}%
\pgfpathlineto{\pgfqpoint{3.572875in}{3.036314in}}%
\pgfpathlineto{\pgfqpoint{3.580640in}{3.045752in}}%
\pgfpathlineto{\pgfqpoint{3.588399in}{3.055258in}}%
\pgfpathlineto{\pgfqpoint{3.596153in}{3.064831in}}%
\pgfpathclose%
\pgfusepath{fill}%
\end{pgfscope}%
\begin{pgfscope}%
\pgfpathrectangle{\pgfqpoint{1.150000in}{0.150000in}}{\pgfqpoint{5.700000in}{5.700000in}}%
\pgfusepath{clip}%
\pgfsetbuttcap%
\pgfsetroundjoin%
\definecolor{currentfill}{rgb}{0.283229,0.120777,0.440584}%
\pgfsetfillcolor{currentfill}%
\pgfsetfillopacity{0.700000}%
\pgfsetlinewidth{0.000000pt}%
\definecolor{currentstroke}{rgb}{0.000000,0.000000,0.000000}%
\pgfsetstrokecolor{currentstroke}%
\pgfsetdash{}{0pt}%
\pgfpathmoveto{\pgfqpoint{3.920234in}{2.973713in}}%
\pgfpathlineto{\pgfqpoint{3.933376in}{2.966694in}}%
\pgfpathlineto{\pgfqpoint{3.946523in}{2.959810in}}%
\pgfpathlineto{\pgfqpoint{3.959673in}{2.953061in}}%
\pgfpathlineto{\pgfqpoint{3.972827in}{2.946447in}}%
\pgfpathlineto{\pgfqpoint{3.965186in}{2.936862in}}%
\pgfpathlineto{\pgfqpoint{3.957541in}{2.927327in}}%
\pgfpathlineto{\pgfqpoint{3.949890in}{2.917843in}}%
\pgfpathlineto{\pgfqpoint{3.942235in}{2.908406in}}%
\pgfpathlineto{\pgfqpoint{3.929069in}{2.914932in}}%
\pgfpathlineto{\pgfqpoint{3.915907in}{2.921591in}}%
\pgfpathlineto{\pgfqpoint{3.902749in}{2.928386in}}%
\pgfpathlineto{\pgfqpoint{3.889595in}{2.935316in}}%
\pgfpathlineto{\pgfqpoint{3.897262in}{2.944835in}}%
\pgfpathlineto{\pgfqpoint{3.904924in}{2.954407in}}%
\pgfpathlineto{\pgfqpoint{3.912582in}{2.964033in}}%
\pgfpathlineto{\pgfqpoint{3.920234in}{2.973713in}}%
\pgfpathclose%
\pgfusepath{fill}%
\end{pgfscope}%
\begin{pgfscope}%
\pgfpathrectangle{\pgfqpoint{1.150000in}{0.150000in}}{\pgfqpoint{5.700000in}{5.700000in}}%
\pgfusepath{clip}%
\pgfsetbuttcap%
\pgfsetroundjoin%
\definecolor{currentfill}{rgb}{0.271828,0.209303,0.504434}%
\pgfsetfillcolor{currentfill}%
\pgfsetfillopacity{0.700000}%
\pgfsetlinewidth{0.000000pt}%
\definecolor{currentstroke}{rgb}{0.000000,0.000000,0.000000}%
\pgfsetstrokecolor{currentstroke}%
\pgfsetdash{}{0pt}%
\pgfpathmoveto{\pgfqpoint{3.407602in}{3.156184in}}%
\pgfpathlineto{\pgfqpoint{3.420727in}{3.144495in}}%
\pgfpathlineto{\pgfqpoint{3.433851in}{3.132978in}}%
\pgfpathlineto{\pgfqpoint{3.446975in}{3.121631in}}%
\pgfpathlineto{\pgfqpoint{3.460099in}{3.110453in}}%
\pgfpathlineto{\pgfqpoint{3.452296in}{3.101064in}}%
\pgfpathlineto{\pgfqpoint{3.444487in}{3.091748in}}%
\pgfpathlineto{\pgfqpoint{3.436672in}{3.082504in}}%
\pgfpathlineto{\pgfqpoint{3.428851in}{3.073333in}}%
\pgfpathlineto{\pgfqpoint{3.415714in}{3.084476in}}%
\pgfpathlineto{\pgfqpoint{3.402576in}{3.095789in}}%
\pgfpathlineto{\pgfqpoint{3.389438in}{3.107272in}}%
\pgfpathlineto{\pgfqpoint{3.376299in}{3.118926in}}%
\pgfpathlineto{\pgfqpoint{3.384134in}{3.128125in}}%
\pgfpathlineto{\pgfqpoint{3.391963in}{3.137401in}}%
\pgfpathlineto{\pgfqpoint{3.399785in}{3.146753in}}%
\pgfpathlineto{\pgfqpoint{3.407602in}{3.156184in}}%
\pgfpathclose%
\pgfusepath{fill}%
\end{pgfscope}%
\begin{pgfscope}%
\pgfpathrectangle{\pgfqpoint{1.150000in}{0.150000in}}{\pgfqpoint{5.700000in}{5.700000in}}%
\pgfusepath{clip}%
\pgfsetbuttcap%
\pgfsetroundjoin%
\definecolor{currentfill}{rgb}{0.283072,0.130895,0.449241}%
\pgfsetfillcolor{currentfill}%
\pgfsetfillopacity{0.700000}%
\pgfsetlinewidth{0.000000pt}%
\definecolor{currentstroke}{rgb}{0.000000,0.000000,0.000000}%
\pgfsetstrokecolor{currentstroke}%
\pgfsetdash{}{0pt}%
\pgfpathmoveto{\pgfqpoint{3.784481in}{2.995746in}}%
\pgfpathlineto{\pgfqpoint{3.797610in}{2.987698in}}%
\pgfpathlineto{\pgfqpoint{3.810741in}{2.979794in}}%
\pgfpathlineto{\pgfqpoint{3.823876in}{2.972032in}}%
\pgfpathlineto{\pgfqpoint{3.837013in}{2.964410in}}%
\pgfpathlineto{\pgfqpoint{3.829329in}{2.954864in}}%
\pgfpathlineto{\pgfqpoint{3.821640in}{2.945373in}}%
\pgfpathlineto{\pgfqpoint{3.813946in}{2.935935in}}%
\pgfpathlineto{\pgfqpoint{3.806247in}{2.926550in}}%
\pgfpathlineto{\pgfqpoint{3.793097in}{2.934100in}}%
\pgfpathlineto{\pgfqpoint{3.779951in}{2.941791in}}%
\pgfpathlineto{\pgfqpoint{3.766808in}{2.949624in}}%
\pgfpathlineto{\pgfqpoint{3.753667in}{2.957600in}}%
\pgfpathlineto{\pgfqpoint{3.761378in}{2.967050in}}%
\pgfpathlineto{\pgfqpoint{3.769085in}{2.976557in}}%
\pgfpathlineto{\pgfqpoint{3.776785in}{2.986122in}}%
\pgfpathlineto{\pgfqpoint{3.784481in}{2.995746in}}%
\pgfpathclose%
\pgfusepath{fill}%
\end{pgfscope}%
\begin{pgfscope}%
\pgfpathrectangle{\pgfqpoint{1.150000in}{0.150000in}}{\pgfqpoint{5.700000in}{5.700000in}}%
\pgfusepath{clip}%
\pgfsetbuttcap%
\pgfsetroundjoin%
\definecolor{currentfill}{rgb}{0.283197,0.115680,0.436115}%
\pgfsetfillcolor{currentfill}%
\pgfsetfillopacity{0.700000}%
\pgfsetlinewidth{0.000000pt}%
\definecolor{currentstroke}{rgb}{0.000000,0.000000,0.000000}%
\pgfsetstrokecolor{currentstroke}%
\pgfsetdash{}{0pt}%
\pgfpathmoveto{\pgfqpoint{4.055954in}{2.959756in}}%
\pgfpathlineto{\pgfqpoint{4.069118in}{2.953692in}}%
\pgfpathlineto{\pgfqpoint{4.082286in}{2.947757in}}%
\pgfpathlineto{\pgfqpoint{4.095460in}{2.941951in}}%
\pgfpathlineto{\pgfqpoint{4.108637in}{2.936273in}}%
\pgfpathlineto{\pgfqpoint{4.101039in}{2.926698in}}%
\pgfpathlineto{\pgfqpoint{4.093435in}{2.917172in}}%
\pgfpathlineto{\pgfqpoint{4.085827in}{2.907691in}}%
\pgfpathlineto{\pgfqpoint{4.078214in}{2.898255in}}%
\pgfpathlineto{\pgfqpoint{4.065024in}{2.903826in}}%
\pgfpathlineto{\pgfqpoint{4.051840in}{2.909525in}}%
\pgfpathlineto{\pgfqpoint{4.038660in}{2.915353in}}%
\pgfpathlineto{\pgfqpoint{4.025485in}{2.921310in}}%
\pgfpathlineto{\pgfqpoint{4.033109in}{2.930847in}}%
\pgfpathlineto{\pgfqpoint{4.040729in}{2.940432in}}%
\pgfpathlineto{\pgfqpoint{4.048344in}{2.950068in}}%
\pgfpathlineto{\pgfqpoint{4.055954in}{2.959756in}}%
\pgfpathclose%
\pgfusepath{fill}%
\end{pgfscope}%
\begin{pgfscope}%
\pgfpathrectangle{\pgfqpoint{1.150000in}{0.150000in}}{\pgfqpoint{5.700000in}{5.700000in}}%
\pgfusepath{clip}%
\pgfsetbuttcap%
\pgfsetroundjoin%
\definecolor{currentfill}{rgb}{0.283187,0.125848,0.444960}%
\pgfsetfillcolor{currentfill}%
\pgfsetfillopacity{0.700000}%
\pgfsetlinewidth{0.000000pt}%
\definecolor{currentstroke}{rgb}{0.000000,0.000000,0.000000}%
\pgfsetstrokecolor{currentstroke}%
\pgfsetdash{}{0pt}%
\pgfpathmoveto{\pgfqpoint{4.410492in}{2.974623in}}%
\pgfpathlineto{\pgfqpoint{4.423731in}{2.970590in}}%
\pgfpathlineto{\pgfqpoint{4.436976in}{2.966674in}}%
\pgfpathlineto{\pgfqpoint{4.450227in}{2.962874in}}%
\pgfpathlineto{\pgfqpoint{4.463486in}{2.959190in}}%
\pgfpathlineto{\pgfqpoint{4.455998in}{2.949746in}}%
\pgfpathlineto{\pgfqpoint{4.448506in}{2.940350in}}%
\pgfpathlineto{\pgfqpoint{4.441010in}{2.930998in}}%
\pgfpathlineto{\pgfqpoint{4.433509in}{2.921690in}}%
\pgfpathlineto{\pgfqpoint{4.420239in}{2.925213in}}%
\pgfpathlineto{\pgfqpoint{4.406976in}{2.928852in}}%
\pgfpathlineto{\pgfqpoint{4.393719in}{2.932608in}}%
\pgfpathlineto{\pgfqpoint{4.380469in}{2.936480in}}%
\pgfpathlineto{\pgfqpoint{4.387981in}{2.945942in}}%
\pgfpathlineto{\pgfqpoint{4.395489in}{2.955452in}}%
\pgfpathlineto{\pgfqpoint{4.402993in}{2.965012in}}%
\pgfpathlineto{\pgfqpoint{4.410492in}{2.974623in}}%
\pgfpathclose%
\pgfusepath{fill}%
\end{pgfscope}%
\begin{pgfscope}%
\pgfpathrectangle{\pgfqpoint{1.150000in}{0.150000in}}{\pgfqpoint{5.700000in}{5.700000in}}%
\pgfusepath{clip}%
\pgfsetbuttcap%
\pgfsetroundjoin%
\definecolor{currentfill}{rgb}{0.283197,0.115680,0.436115}%
\pgfsetfillcolor{currentfill}%
\pgfsetfillopacity{0.700000}%
\pgfsetlinewidth{0.000000pt}%
\definecolor{currentstroke}{rgb}{0.000000,0.000000,0.000000}%
\pgfsetstrokecolor{currentstroke}%
\pgfsetdash{}{0pt}%
\pgfpathmoveto{\pgfqpoint{4.191702in}{2.953128in}}%
\pgfpathlineto{\pgfqpoint{4.204894in}{2.947953in}}%
\pgfpathlineto{\pgfqpoint{4.218091in}{2.942902in}}%
\pgfpathlineto{\pgfqpoint{4.231293in}{2.937975in}}%
\pgfpathlineto{\pgfqpoint{4.244501in}{2.933169in}}%
\pgfpathlineto{\pgfqpoint{4.236944in}{2.923652in}}%
\pgfpathlineto{\pgfqpoint{4.229381in}{2.914180in}}%
\pgfpathlineto{\pgfqpoint{4.221814in}{2.904752in}}%
\pgfpathlineto{\pgfqpoint{4.214243in}{2.895365in}}%
\pgfpathlineto{\pgfqpoint{4.201024in}{2.900045in}}%
\pgfpathlineto{\pgfqpoint{4.187810in}{2.904848in}}%
\pgfpathlineto{\pgfqpoint{4.174601in}{2.909774in}}%
\pgfpathlineto{\pgfqpoint{4.161398in}{2.914823in}}%
\pgfpathlineto{\pgfqpoint{4.168981in}{2.924328in}}%
\pgfpathlineto{\pgfqpoint{4.176559in}{2.933880in}}%
\pgfpathlineto{\pgfqpoint{4.184133in}{2.943479in}}%
\pgfpathlineto{\pgfqpoint{4.191702in}{2.953128in}}%
\pgfpathclose%
\pgfusepath{fill}%
\end{pgfscope}%
\begin{pgfscope}%
\pgfpathrectangle{\pgfqpoint{1.150000in}{0.150000in}}{\pgfqpoint{5.700000in}{5.700000in}}%
\pgfusepath{clip}%
\pgfsetbuttcap%
\pgfsetroundjoin%
\definecolor{currentfill}{rgb}{0.281887,0.150881,0.465405}%
\pgfsetfillcolor{currentfill}%
\pgfsetfillopacity{0.700000}%
\pgfsetlinewidth{0.000000pt}%
\definecolor{currentstroke}{rgb}{0.000000,0.000000,0.000000}%
\pgfsetstrokecolor{currentstroke}%
\pgfsetdash{}{0pt}%
\pgfpathmoveto{\pgfqpoint{3.648628in}{3.026674in}}%
\pgfpathlineto{\pgfqpoint{3.661750in}{3.017518in}}%
\pgfpathlineto{\pgfqpoint{3.674875in}{3.008513in}}%
\pgfpathlineto{\pgfqpoint{3.688001in}{2.999659in}}%
\pgfpathlineto{\pgfqpoint{3.701130in}{2.990953in}}%
\pgfpathlineto{\pgfqpoint{3.693401in}{2.981499in}}%
\pgfpathlineto{\pgfqpoint{3.685667in}{2.972104in}}%
\pgfpathlineto{\pgfqpoint{3.677927in}{2.962767in}}%
\pgfpathlineto{\pgfqpoint{3.670182in}{2.953487in}}%
\pgfpathlineto{\pgfqpoint{3.657041in}{2.962140in}}%
\pgfpathlineto{\pgfqpoint{3.643902in}{2.970941in}}%
\pgfpathlineto{\pgfqpoint{3.630765in}{2.979892in}}%
\pgfpathlineto{\pgfqpoint{3.617630in}{2.988995in}}%
\pgfpathlineto{\pgfqpoint{3.625388in}{2.998321in}}%
\pgfpathlineto{\pgfqpoint{3.633140in}{3.007709in}}%
\pgfpathlineto{\pgfqpoint{3.640887in}{3.017160in}}%
\pgfpathlineto{\pgfqpoint{3.648628in}{3.026674in}}%
\pgfpathclose%
\pgfusepath{fill}%
\end{pgfscope}%
\begin{pgfscope}%
\pgfpathrectangle{\pgfqpoint{1.150000in}{0.150000in}}{\pgfqpoint{5.700000in}{5.700000in}}%
\pgfusepath{clip}%
\pgfsetbuttcap%
\pgfsetroundjoin%
\definecolor{currentfill}{rgb}{0.282290,0.145912,0.461510}%
\pgfsetfillcolor{currentfill}%
\pgfsetfillopacity{0.700000}%
\pgfsetlinewidth{0.000000pt}%
\definecolor{currentstroke}{rgb}{0.000000,0.000000,0.000000}%
\pgfsetstrokecolor{currentstroke}%
\pgfsetdash{}{0pt}%
\pgfpathmoveto{\pgfqpoint{4.629354in}{3.008338in}}%
\pgfpathlineto{\pgfqpoint{4.642650in}{3.005299in}}%
\pgfpathlineto{\pgfqpoint{4.655954in}{3.002371in}}%
\pgfpathlineto{\pgfqpoint{4.669265in}{2.999554in}}%
\pgfpathlineto{\pgfqpoint{4.682584in}{2.996847in}}%
\pgfpathlineto{\pgfqpoint{4.675166in}{2.987554in}}%
\pgfpathlineto{\pgfqpoint{4.667744in}{2.978314in}}%
\pgfpathlineto{\pgfqpoint{4.660319in}{2.969124in}}%
\pgfpathlineto{\pgfqpoint{4.652888in}{2.959982in}}%
\pgfpathlineto{\pgfqpoint{4.639557in}{2.962492in}}%
\pgfpathlineto{\pgfqpoint{4.626234in}{2.965113in}}%
\pgfpathlineto{\pgfqpoint{4.612918in}{2.967844in}}%
\pgfpathlineto{\pgfqpoint{4.599609in}{2.970687in}}%
\pgfpathlineto{\pgfqpoint{4.607051in}{2.980019in}}%
\pgfpathlineto{\pgfqpoint{4.614489in}{2.989404in}}%
\pgfpathlineto{\pgfqpoint{4.621923in}{2.998843in}}%
\pgfpathlineto{\pgfqpoint{4.629354in}{3.008338in}}%
\pgfpathclose%
\pgfusepath{fill}%
\end{pgfscope}%
\begin{pgfscope}%
\pgfpathrectangle{\pgfqpoint{1.150000in}{0.150000in}}{\pgfqpoint{5.700000in}{5.700000in}}%
\pgfusepath{clip}%
\pgfsetbuttcap%
\pgfsetroundjoin%
\definecolor{currentfill}{rgb}{0.276194,0.190074,0.493001}%
\pgfsetfillcolor{currentfill}%
\pgfsetfillopacity{0.700000}%
\pgfsetlinewidth{0.000000pt}%
\definecolor{currentstroke}{rgb}{0.000000,0.000000,0.000000}%
\pgfsetstrokecolor{currentstroke}%
\pgfsetdash{}{0pt}%
\pgfpathmoveto{\pgfqpoint{3.460099in}{3.110453in}}%
\pgfpathlineto{\pgfqpoint{3.473223in}{3.099443in}}%
\pgfpathlineto{\pgfqpoint{3.486347in}{3.088598in}}%
\pgfpathlineto{\pgfqpoint{3.499471in}{3.077918in}}%
\pgfpathlineto{\pgfqpoint{3.512596in}{3.067401in}}%
\pgfpathlineto{\pgfqpoint{3.504807in}{3.058053in}}%
\pgfpathlineto{\pgfqpoint{3.497012in}{3.048773in}}%
\pgfpathlineto{\pgfqpoint{3.489211in}{3.039562in}}%
\pgfpathlineto{\pgfqpoint{3.481404in}{3.030419in}}%
\pgfpathlineto{\pgfqpoint{3.468265in}{3.040901in}}%
\pgfpathlineto{\pgfqpoint{3.455127in}{3.051546in}}%
\pgfpathlineto{\pgfqpoint{3.441989in}{3.062356in}}%
\pgfpathlineto{\pgfqpoint{3.428851in}{3.073333in}}%
\pgfpathlineto{\pgfqpoint{3.436672in}{3.082504in}}%
\pgfpathlineto{\pgfqpoint{3.444487in}{3.091748in}}%
\pgfpathlineto{\pgfqpoint{3.452296in}{3.101064in}}%
\pgfpathlineto{\pgfqpoint{3.460099in}{3.110453in}}%
\pgfpathclose%
\pgfusepath{fill}%
\end{pgfscope}%
\begin{pgfscope}%
\pgfpathrectangle{\pgfqpoint{1.150000in}{0.150000in}}{\pgfqpoint{5.700000in}{5.700000in}}%
\pgfusepath{clip}%
\pgfsetbuttcap%
\pgfsetroundjoin%
\definecolor{currentfill}{rgb}{0.283229,0.120777,0.440584}%
\pgfsetfillcolor{currentfill}%
\pgfsetfillopacity{0.700000}%
\pgfsetlinewidth{0.000000pt}%
\definecolor{currentstroke}{rgb}{0.000000,0.000000,0.000000}%
\pgfsetstrokecolor{currentstroke}%
\pgfsetdash{}{0pt}%
\pgfpathmoveto{\pgfqpoint{4.327530in}{2.953147in}}%
\pgfpathlineto{\pgfqpoint{4.340756in}{2.948802in}}%
\pgfpathlineto{\pgfqpoint{4.353987in}{2.944577in}}%
\pgfpathlineto{\pgfqpoint{4.367225in}{2.940469in}}%
\pgfpathlineto{\pgfqpoint{4.380469in}{2.936480in}}%
\pgfpathlineto{\pgfqpoint{4.372952in}{2.927063in}}%
\pgfpathlineto{\pgfqpoint{4.365431in}{2.917690in}}%
\pgfpathlineto{\pgfqpoint{4.357905in}{2.908358in}}%
\pgfpathlineto{\pgfqpoint{4.350375in}{2.899067in}}%
\pgfpathlineto{\pgfqpoint{4.337119in}{2.902913in}}%
\pgfpathlineto{\pgfqpoint{4.323870in}{2.906877in}}%
\pgfpathlineto{\pgfqpoint{4.310627in}{2.910960in}}%
\pgfpathlineto{\pgfqpoint{4.297390in}{2.915161in}}%
\pgfpathlineto{\pgfqpoint{4.304932in}{2.924589in}}%
\pgfpathlineto{\pgfqpoint{4.312469in}{2.934062in}}%
\pgfpathlineto{\pgfqpoint{4.320002in}{2.943580in}}%
\pgfpathlineto{\pgfqpoint{4.327530in}{2.953147in}}%
\pgfpathclose%
\pgfusepath{fill}%
\end{pgfscope}%
\begin{pgfscope}%
\pgfpathrectangle{\pgfqpoint{1.150000in}{0.150000in}}{\pgfqpoint{5.700000in}{5.700000in}}%
\pgfusepath{clip}%
\pgfsetbuttcap%
\pgfsetroundjoin%
\definecolor{currentfill}{rgb}{0.283229,0.120777,0.440584}%
\pgfsetfillcolor{currentfill}%
\pgfsetfillopacity{0.700000}%
\pgfsetlinewidth{0.000000pt}%
\definecolor{currentstroke}{rgb}{0.000000,0.000000,0.000000}%
\pgfsetstrokecolor{currentstroke}%
\pgfsetdash{}{0pt}%
\pgfpathmoveto{\pgfqpoint{3.837013in}{2.964410in}}%
\pgfpathlineto{\pgfqpoint{3.850154in}{2.956929in}}%
\pgfpathlineto{\pgfqpoint{3.863297in}{2.949587in}}%
\pgfpathlineto{\pgfqpoint{3.876445in}{2.942383in}}%
\pgfpathlineto{\pgfqpoint{3.889595in}{2.935316in}}%
\pgfpathlineto{\pgfqpoint{3.881923in}{2.925848in}}%
\pgfpathlineto{\pgfqpoint{3.874246in}{2.916431in}}%
\pgfpathlineto{\pgfqpoint{3.866563in}{2.907062in}}%
\pgfpathlineto{\pgfqpoint{3.858876in}{2.897741in}}%
\pgfpathlineto{\pgfqpoint{3.845713in}{2.904736in}}%
\pgfpathlineto{\pgfqpoint{3.832555in}{2.911869in}}%
\pgfpathlineto{\pgfqpoint{3.819399in}{2.919140in}}%
\pgfpathlineto{\pgfqpoint{3.806247in}{2.926550in}}%
\pgfpathlineto{\pgfqpoint{3.813946in}{2.935935in}}%
\pgfpathlineto{\pgfqpoint{3.821640in}{2.945373in}}%
\pgfpathlineto{\pgfqpoint{3.829329in}{2.954864in}}%
\pgfpathlineto{\pgfqpoint{3.837013in}{2.964410in}}%
\pgfpathclose%
\pgfusepath{fill}%
\end{pgfscope}%
\begin{pgfscope}%
\pgfpathrectangle{\pgfqpoint{1.150000in}{0.150000in}}{\pgfqpoint{5.700000in}{5.700000in}}%
\pgfusepath{clip}%
\pgfsetbuttcap%
\pgfsetroundjoin%
\definecolor{currentfill}{rgb}{0.283197,0.115680,0.436115}%
\pgfsetfillcolor{currentfill}%
\pgfsetfillopacity{0.700000}%
\pgfsetlinewidth{0.000000pt}%
\definecolor{currentstroke}{rgb}{0.000000,0.000000,0.000000}%
\pgfsetstrokecolor{currentstroke}%
\pgfsetdash{}{0pt}%
\pgfpathmoveto{\pgfqpoint{3.972827in}{2.946447in}}%
\pgfpathlineto{\pgfqpoint{3.985985in}{2.939965in}}%
\pgfpathlineto{\pgfqpoint{3.999148in}{2.933615in}}%
\pgfpathlineto{\pgfqpoint{4.012314in}{2.927397in}}%
\pgfpathlineto{\pgfqpoint{4.025485in}{2.921310in}}%
\pgfpathlineto{\pgfqpoint{4.017856in}{2.911821in}}%
\pgfpathlineto{\pgfqpoint{4.010221in}{2.902379in}}%
\pgfpathlineto{\pgfqpoint{4.002582in}{2.892981in}}%
\pgfpathlineto{\pgfqpoint{3.994938in}{2.883628in}}%
\pgfpathlineto{\pgfqpoint{3.981756in}{2.889625in}}%
\pgfpathlineto{\pgfqpoint{3.968578in}{2.895754in}}%
\pgfpathlineto{\pgfqpoint{3.955404in}{2.902014in}}%
\pgfpathlineto{\pgfqpoint{3.942235in}{2.908406in}}%
\pgfpathlineto{\pgfqpoint{3.949890in}{2.917843in}}%
\pgfpathlineto{\pgfqpoint{3.957541in}{2.927327in}}%
\pgfpathlineto{\pgfqpoint{3.965186in}{2.936862in}}%
\pgfpathlineto{\pgfqpoint{3.972827in}{2.946447in}}%
\pgfpathclose%
\pgfusepath{fill}%
\end{pgfscope}%
\begin{pgfscope}%
\pgfpathrectangle{\pgfqpoint{1.150000in}{0.150000in}}{\pgfqpoint{5.700000in}{5.700000in}}%
\pgfusepath{clip}%
\pgfsetbuttcap%
\pgfsetroundjoin%
\definecolor{currentfill}{rgb}{0.282884,0.135920,0.453427}%
\pgfsetfillcolor{currentfill}%
\pgfsetfillopacity{0.700000}%
\pgfsetlinewidth{0.000000pt}%
\definecolor{currentstroke}{rgb}{0.000000,0.000000,0.000000}%
\pgfsetstrokecolor{currentstroke}%
\pgfsetdash{}{0pt}%
\pgfpathmoveto{\pgfqpoint{4.546445in}{2.983177in}}%
\pgfpathlineto{\pgfqpoint{4.559726in}{2.979885in}}%
\pgfpathlineto{\pgfqpoint{4.573013in}{2.976707in}}%
\pgfpathlineto{\pgfqpoint{4.586307in}{2.973641in}}%
\pgfpathlineto{\pgfqpoint{4.599609in}{2.970687in}}%
\pgfpathlineto{\pgfqpoint{4.592162in}{2.961403in}}%
\pgfpathlineto{\pgfqpoint{4.584711in}{2.952167in}}%
\pgfpathlineto{\pgfqpoint{4.577256in}{2.942976in}}%
\pgfpathlineto{\pgfqpoint{4.569796in}{2.933827in}}%
\pgfpathlineto{\pgfqpoint{4.556483in}{2.936602in}}%
\pgfpathlineto{\pgfqpoint{4.543177in}{2.939489in}}%
\pgfpathlineto{\pgfqpoint{4.529877in}{2.942489in}}%
\pgfpathlineto{\pgfqpoint{4.516585in}{2.945601in}}%
\pgfpathlineto{\pgfqpoint{4.524057in}{2.954922in}}%
\pgfpathlineto{\pgfqpoint{4.531524in}{2.964290in}}%
\pgfpathlineto{\pgfqpoint{4.538987in}{2.973708in}}%
\pgfpathlineto{\pgfqpoint{4.546445in}{2.983177in}}%
\pgfpathclose%
\pgfusepath{fill}%
\end{pgfscope}%
\begin{pgfscope}%
\pgfpathrectangle{\pgfqpoint{1.150000in}{0.150000in}}{\pgfqpoint{5.700000in}{5.700000in}}%
\pgfusepath{clip}%
\pgfsetbuttcap%
\pgfsetroundjoin%
\definecolor{currentfill}{rgb}{0.252194,0.269783,0.531579}%
\pgfsetfillcolor{currentfill}%
\pgfsetfillopacity{0.700000}%
\pgfsetlinewidth{0.000000pt}%
\definecolor{currentstroke}{rgb}{0.000000,0.000000,0.000000}%
\pgfsetstrokecolor{currentstroke}%
\pgfsetdash{}{0pt}%
\pgfpathmoveto{\pgfqpoint{3.218547in}{3.272734in}}%
\pgfpathlineto{\pgfqpoint{3.231703in}{3.258897in}}%
\pgfpathlineto{\pgfqpoint{3.244857in}{3.245251in}}%
\pgfpathlineto{\pgfqpoint{3.258008in}{3.231794in}}%
\pgfpathlineto{\pgfqpoint{3.271157in}{3.218525in}}%
\pgfpathlineto{\pgfqpoint{3.263287in}{3.209357in}}%
\pgfpathlineto{\pgfqpoint{3.255410in}{3.200272in}}%
\pgfpathlineto{\pgfqpoint{3.247526in}{3.191272in}}%
\pgfpathlineto{\pgfqpoint{3.239637in}{3.182354in}}%
\pgfpathlineto{\pgfqpoint{3.226472in}{3.195608in}}%
\pgfpathlineto{\pgfqpoint{3.213305in}{3.209049in}}%
\pgfpathlineto{\pgfqpoint{3.200136in}{3.222680in}}%
\pgfpathlineto{\pgfqpoint{3.186965in}{3.236502in}}%
\pgfpathlineto{\pgfqpoint{3.194871in}{3.245428in}}%
\pgfpathlineto{\pgfqpoint{3.202770in}{3.254442in}}%
\pgfpathlineto{\pgfqpoint{3.210662in}{3.263544in}}%
\pgfpathlineto{\pgfqpoint{3.218547in}{3.272734in}}%
\pgfpathclose%
\pgfusepath{fill}%
\end{pgfscope}%
\begin{pgfscope}%
\pgfpathrectangle{\pgfqpoint{1.150000in}{0.150000in}}{\pgfqpoint{5.700000in}{5.700000in}}%
\pgfusepath{clip}%
\pgfsetbuttcap%
\pgfsetroundjoin%
\definecolor{currentfill}{rgb}{0.278826,0.175490,0.483397}%
\pgfsetfillcolor{currentfill}%
\pgfsetfillopacity{0.700000}%
\pgfsetlinewidth{0.000000pt}%
\definecolor{currentstroke}{rgb}{0.000000,0.000000,0.000000}%
\pgfsetstrokecolor{currentstroke}%
\pgfsetdash{}{0pt}%
\pgfpathmoveto{\pgfqpoint{3.512596in}{3.067401in}}%
\pgfpathlineto{\pgfqpoint{3.525722in}{3.057046in}}%
\pgfpathlineto{\pgfqpoint{3.538848in}{3.046852in}}%
\pgfpathlineto{\pgfqpoint{3.551976in}{3.036818in}}%
\pgfpathlineto{\pgfqpoint{3.565104in}{3.026942in}}%
\pgfpathlineto{\pgfqpoint{3.557328in}{3.017635in}}%
\pgfpathlineto{\pgfqpoint{3.549546in}{3.008393in}}%
\pgfpathlineto{\pgfqpoint{3.541758in}{2.999214in}}%
\pgfpathlineto{\pgfqpoint{3.533965in}{2.990099in}}%
\pgfpathlineto{\pgfqpoint{3.520823in}{2.999940in}}%
\pgfpathlineto{\pgfqpoint{3.507683in}{3.009940in}}%
\pgfpathlineto{\pgfqpoint{3.494543in}{3.020099in}}%
\pgfpathlineto{\pgfqpoint{3.481404in}{3.030419in}}%
\pgfpathlineto{\pgfqpoint{3.489211in}{3.039562in}}%
\pgfpathlineto{\pgfqpoint{3.497012in}{3.048773in}}%
\pgfpathlineto{\pgfqpoint{3.504807in}{3.058053in}}%
\pgfpathlineto{\pgfqpoint{3.512596in}{3.067401in}}%
\pgfpathclose%
\pgfusepath{fill}%
\end{pgfscope}%
\begin{pgfscope}%
\pgfpathrectangle{\pgfqpoint{1.150000in}{0.150000in}}{\pgfqpoint{5.700000in}{5.700000in}}%
\pgfusepath{clip}%
\pgfsetbuttcap%
\pgfsetroundjoin%
\definecolor{currentfill}{rgb}{0.282884,0.135920,0.453427}%
\pgfsetfillcolor{currentfill}%
\pgfsetfillopacity{0.700000}%
\pgfsetlinewidth{0.000000pt}%
\definecolor{currentstroke}{rgb}{0.000000,0.000000,0.000000}%
\pgfsetstrokecolor{currentstroke}%
\pgfsetdash{}{0pt}%
\pgfpathmoveto{\pgfqpoint{3.701130in}{2.990953in}}%
\pgfpathlineto{\pgfqpoint{3.714261in}{2.982395in}}%
\pgfpathlineto{\pgfqpoint{3.727394in}{2.973985in}}%
\pgfpathlineto{\pgfqpoint{3.740529in}{2.965720in}}%
\pgfpathlineto{\pgfqpoint{3.753667in}{2.957600in}}%
\pgfpathlineto{\pgfqpoint{3.745950in}{2.948206in}}%
\pgfpathlineto{\pgfqpoint{3.738228in}{2.938866in}}%
\pgfpathlineto{\pgfqpoint{3.730501in}{2.929581in}}%
\pgfpathlineto{\pgfqpoint{3.722768in}{2.920348in}}%
\pgfpathlineto{\pgfqpoint{3.709618in}{2.928414in}}%
\pgfpathlineto{\pgfqpoint{3.696470in}{2.936626in}}%
\pgfpathlineto{\pgfqpoint{3.683325in}{2.944983in}}%
\pgfpathlineto{\pgfqpoint{3.670182in}{2.953487in}}%
\pgfpathlineto{\pgfqpoint{3.677927in}{2.962767in}}%
\pgfpathlineto{\pgfqpoint{3.685667in}{2.972104in}}%
\pgfpathlineto{\pgfqpoint{3.693401in}{2.981499in}}%
\pgfpathlineto{\pgfqpoint{3.701130in}{2.990953in}}%
\pgfpathclose%
\pgfusepath{fill}%
\end{pgfscope}%
\begin{pgfscope}%
\pgfpathrectangle{\pgfqpoint{1.150000in}{0.150000in}}{\pgfqpoint{5.700000in}{5.700000in}}%
\pgfusepath{clip}%
\pgfsetbuttcap%
\pgfsetroundjoin%
\definecolor{currentfill}{rgb}{0.283091,0.110553,0.431554}%
\pgfsetfillcolor{currentfill}%
\pgfsetfillopacity{0.700000}%
\pgfsetlinewidth{0.000000pt}%
\definecolor{currentstroke}{rgb}{0.000000,0.000000,0.000000}%
\pgfsetstrokecolor{currentstroke}%
\pgfsetdash{}{0pt}%
\pgfpathmoveto{\pgfqpoint{4.108637in}{2.936273in}}%
\pgfpathlineto{\pgfqpoint{4.121820in}{2.930721in}}%
\pgfpathlineto{\pgfqpoint{4.135008in}{2.925296in}}%
\pgfpathlineto{\pgfqpoint{4.148201in}{2.919997in}}%
\pgfpathlineto{\pgfqpoint{4.161398in}{2.914823in}}%
\pgfpathlineto{\pgfqpoint{4.153811in}{2.905363in}}%
\pgfpathlineto{\pgfqpoint{4.146218in}{2.895946in}}%
\pgfpathlineto{\pgfqpoint{4.138621in}{2.886571in}}%
\pgfpathlineto{\pgfqpoint{4.131019in}{2.877236in}}%
\pgfpathlineto{\pgfqpoint{4.117810in}{2.882302in}}%
\pgfpathlineto{\pgfqpoint{4.104606in}{2.887494in}}%
\pgfpathlineto{\pgfqpoint{4.091407in}{2.892811in}}%
\pgfpathlineto{\pgfqpoint{4.078214in}{2.898255in}}%
\pgfpathlineto{\pgfqpoint{4.085827in}{2.907691in}}%
\pgfpathlineto{\pgfqpoint{4.093435in}{2.917172in}}%
\pgfpathlineto{\pgfqpoint{4.101039in}{2.926698in}}%
\pgfpathlineto{\pgfqpoint{4.108637in}{2.936273in}}%
\pgfpathclose%
\pgfusepath{fill}%
\end{pgfscope}%
\begin{pgfscope}%
\pgfpathrectangle{\pgfqpoint{1.150000in}{0.150000in}}{\pgfqpoint{5.700000in}{5.700000in}}%
\pgfusepath{clip}%
\pgfsetbuttcap%
\pgfsetroundjoin%
\definecolor{currentfill}{rgb}{0.260571,0.246922,0.522828}%
\pgfsetfillcolor{currentfill}%
\pgfsetfillopacity{0.700000}%
\pgfsetlinewidth{0.000000pt}%
\definecolor{currentstroke}{rgb}{0.000000,0.000000,0.000000}%
\pgfsetstrokecolor{currentstroke}%
\pgfsetdash{}{0pt}%
\pgfpathmoveto{\pgfqpoint{3.271157in}{3.218525in}}%
\pgfpathlineto{\pgfqpoint{3.284304in}{3.205442in}}%
\pgfpathlineto{\pgfqpoint{3.297450in}{3.192543in}}%
\pgfpathlineto{\pgfqpoint{3.310595in}{3.179827in}}%
\pgfpathlineto{\pgfqpoint{3.323738in}{3.167291in}}%
\pgfpathlineto{\pgfqpoint{3.315882in}{3.158145in}}%
\pgfpathlineto{\pgfqpoint{3.308020in}{3.149079in}}%
\pgfpathlineto{\pgfqpoint{3.300152in}{3.140091in}}%
\pgfpathlineto{\pgfqpoint{3.292278in}{3.131183in}}%
\pgfpathlineto{\pgfqpoint{3.279120in}{3.143703in}}%
\pgfpathlineto{\pgfqpoint{3.265960in}{3.156403in}}%
\pgfpathlineto{\pgfqpoint{3.252799in}{3.169287in}}%
\pgfpathlineto{\pgfqpoint{3.239637in}{3.182354in}}%
\pgfpathlineto{\pgfqpoint{3.247526in}{3.191272in}}%
\pgfpathlineto{\pgfqpoint{3.255410in}{3.200272in}}%
\pgfpathlineto{\pgfqpoint{3.263287in}{3.209357in}}%
\pgfpathlineto{\pgfqpoint{3.271157in}{3.218525in}}%
\pgfpathclose%
\pgfusepath{fill}%
\end{pgfscope}%
\begin{pgfscope}%
\pgfpathrectangle{\pgfqpoint{1.150000in}{0.150000in}}{\pgfqpoint{5.700000in}{5.700000in}}%
\pgfusepath{clip}%
\pgfsetbuttcap%
\pgfsetroundjoin%
\definecolor{currentfill}{rgb}{0.267968,0.223549,0.512008}%
\pgfsetfillcolor{currentfill}%
\pgfsetfillopacity{0.700000}%
\pgfsetlinewidth{0.000000pt}%
\definecolor{currentstroke}{rgb}{0.000000,0.000000,0.000000}%
\pgfsetstrokecolor{currentstroke}%
\pgfsetdash{}{0pt}%
\pgfpathmoveto{\pgfqpoint{3.323738in}{3.167291in}}%
\pgfpathlineto{\pgfqpoint{3.336879in}{3.154935in}}%
\pgfpathlineto{\pgfqpoint{3.350020in}{3.142756in}}%
\pgfpathlineto{\pgfqpoint{3.363160in}{3.130754in}}%
\pgfpathlineto{\pgfqpoint{3.376299in}{3.118926in}}%
\pgfpathlineto{\pgfqpoint{3.368458in}{3.109803in}}%
\pgfpathlineto{\pgfqpoint{3.360611in}{3.100755in}}%
\pgfpathlineto{\pgfqpoint{3.352758in}{3.091782in}}%
\pgfpathlineto{\pgfqpoint{3.344898in}{3.082883in}}%
\pgfpathlineto{\pgfqpoint{3.331744in}{3.094694in}}%
\pgfpathlineto{\pgfqpoint{3.318590in}{3.106680in}}%
\pgfpathlineto{\pgfqpoint{3.305434in}{3.118843in}}%
\pgfpathlineto{\pgfqpoint{3.292278in}{3.131183in}}%
\pgfpathlineto{\pgfqpoint{3.300152in}{3.140091in}}%
\pgfpathlineto{\pgfqpoint{3.308020in}{3.149079in}}%
\pgfpathlineto{\pgfqpoint{3.315882in}{3.158145in}}%
\pgfpathlineto{\pgfqpoint{3.323738in}{3.167291in}}%
\pgfpathclose%
\pgfusepath{fill}%
\end{pgfscope}%
\begin{pgfscope}%
\pgfpathrectangle{\pgfqpoint{1.150000in}{0.150000in}}{\pgfqpoint{5.700000in}{5.700000in}}%
\pgfusepath{clip}%
\pgfsetbuttcap%
\pgfsetroundjoin%
\definecolor{currentfill}{rgb}{0.283197,0.115680,0.436115}%
\pgfsetfillcolor{currentfill}%
\pgfsetfillopacity{0.700000}%
\pgfsetlinewidth{0.000000pt}%
\definecolor{currentstroke}{rgb}{0.000000,0.000000,0.000000}%
\pgfsetstrokecolor{currentstroke}%
\pgfsetdash{}{0pt}%
\pgfpathmoveto{\pgfqpoint{4.244501in}{2.933169in}}%
\pgfpathlineto{\pgfqpoint{4.257715in}{2.928486in}}%
\pgfpathlineto{\pgfqpoint{4.270934in}{2.923924in}}%
\pgfpathlineto{\pgfqpoint{4.284159in}{2.919483in}}%
\pgfpathlineto{\pgfqpoint{4.297390in}{2.915161in}}%
\pgfpathlineto{\pgfqpoint{4.289844in}{2.905776in}}%
\pgfpathlineto{\pgfqpoint{4.282293in}{2.896432in}}%
\pgfpathlineto{\pgfqpoint{4.274737in}{2.887127in}}%
\pgfpathlineto{\pgfqpoint{4.267177in}{2.877859in}}%
\pgfpathlineto{\pgfqpoint{4.253935in}{2.882055in}}%
\pgfpathlineto{\pgfqpoint{4.240698in}{2.886371in}}%
\pgfpathlineto{\pgfqpoint{4.227468in}{2.890807in}}%
\pgfpathlineto{\pgfqpoint{4.214243in}{2.895365in}}%
\pgfpathlineto{\pgfqpoint{4.221814in}{2.904752in}}%
\pgfpathlineto{\pgfqpoint{4.229381in}{2.914180in}}%
\pgfpathlineto{\pgfqpoint{4.236944in}{2.923652in}}%
\pgfpathlineto{\pgfqpoint{4.244501in}{2.933169in}}%
\pgfpathclose%
\pgfusepath{fill}%
\end{pgfscope}%
\begin{pgfscope}%
\pgfpathrectangle{\pgfqpoint{1.150000in}{0.150000in}}{\pgfqpoint{5.700000in}{5.700000in}}%
\pgfusepath{clip}%
\pgfsetbuttcap%
\pgfsetroundjoin%
\definecolor{currentfill}{rgb}{0.283072,0.130895,0.449241}%
\pgfsetfillcolor{currentfill}%
\pgfsetfillopacity{0.700000}%
\pgfsetlinewidth{0.000000pt}%
\definecolor{currentstroke}{rgb}{0.000000,0.000000,0.000000}%
\pgfsetstrokecolor{currentstroke}%
\pgfsetdash{}{0pt}%
\pgfpathmoveto{\pgfqpoint{4.463486in}{2.959190in}}%
\pgfpathlineto{\pgfqpoint{4.476750in}{2.955621in}}%
\pgfpathlineto{\pgfqpoint{4.490022in}{2.952167in}}%
\pgfpathlineto{\pgfqpoint{4.503300in}{2.948827in}}%
\pgfpathlineto{\pgfqpoint{4.516585in}{2.945601in}}%
\pgfpathlineto{\pgfqpoint{4.509109in}{2.936325in}}%
\pgfpathlineto{\pgfqpoint{4.501629in}{2.927092in}}%
\pgfpathlineto{\pgfqpoint{4.494144in}{2.917900in}}%
\pgfpathlineto{\pgfqpoint{4.486655in}{2.908747in}}%
\pgfpathlineto{\pgfqpoint{4.473358in}{2.911811in}}%
\pgfpathlineto{\pgfqpoint{4.460068in}{2.914990in}}%
\pgfpathlineto{\pgfqpoint{4.446785in}{2.918283in}}%
\pgfpathlineto{\pgfqpoint{4.433509in}{2.921690in}}%
\pgfpathlineto{\pgfqpoint{4.441010in}{2.930998in}}%
\pgfpathlineto{\pgfqpoint{4.448506in}{2.940350in}}%
\pgfpathlineto{\pgfqpoint{4.455998in}{2.949746in}}%
\pgfpathlineto{\pgfqpoint{4.463486in}{2.959190in}}%
\pgfpathclose%
\pgfusepath{fill}%
\end{pgfscope}%
\begin{pgfscope}%
\pgfpathrectangle{\pgfqpoint{1.150000in}{0.150000in}}{\pgfqpoint{5.700000in}{5.700000in}}%
\pgfusepath{clip}%
\pgfsetbuttcap%
\pgfsetroundjoin%
\definecolor{currentfill}{rgb}{0.281412,0.155834,0.469201}%
\pgfsetfillcolor{currentfill}%
\pgfsetfillopacity{0.700000}%
\pgfsetlinewidth{0.000000pt}%
\definecolor{currentstroke}{rgb}{0.000000,0.000000,0.000000}%
\pgfsetstrokecolor{currentstroke}%
\pgfsetdash{}{0pt}%
\pgfpathmoveto{\pgfqpoint{3.565104in}{3.026942in}}%
\pgfpathlineto{\pgfqpoint{3.578233in}{3.017223in}}%
\pgfpathlineto{\pgfqpoint{3.591364in}{3.007659in}}%
\pgfpathlineto{\pgfqpoint{3.604496in}{2.998250in}}%
\pgfpathlineto{\pgfqpoint{3.617630in}{2.988995in}}%
\pgfpathlineto{\pgfqpoint{3.609866in}{2.979730in}}%
\pgfpathlineto{\pgfqpoint{3.602098in}{2.970526in}}%
\pgfpathlineto{\pgfqpoint{3.594323in}{2.961380in}}%
\pgfpathlineto{\pgfqpoint{3.586543in}{2.952294in}}%
\pgfpathlineto{\pgfqpoint{3.573396in}{2.961514in}}%
\pgfpathlineto{\pgfqpoint{3.560251in}{2.970887in}}%
\pgfpathlineto{\pgfqpoint{3.547107in}{2.980415in}}%
\pgfpathlineto{\pgfqpoint{3.533965in}{2.990099in}}%
\pgfpathlineto{\pgfqpoint{3.541758in}{2.999214in}}%
\pgfpathlineto{\pgfqpoint{3.549546in}{3.008393in}}%
\pgfpathlineto{\pgfqpoint{3.557328in}{3.017635in}}%
\pgfpathlineto{\pgfqpoint{3.565104in}{3.026942in}}%
\pgfpathclose%
\pgfusepath{fill}%
\end{pgfscope}%
\begin{pgfscope}%
\pgfpathrectangle{\pgfqpoint{1.150000in}{0.150000in}}{\pgfqpoint{5.700000in}{5.700000in}}%
\pgfusepath{clip}%
\pgfsetbuttcap%
\pgfsetroundjoin%
\definecolor{currentfill}{rgb}{0.283197,0.115680,0.436115}%
\pgfsetfillcolor{currentfill}%
\pgfsetfillopacity{0.700000}%
\pgfsetlinewidth{0.000000pt}%
\definecolor{currentstroke}{rgb}{0.000000,0.000000,0.000000}%
\pgfsetstrokecolor{currentstroke}%
\pgfsetdash{}{0pt}%
\pgfpathmoveto{\pgfqpoint{3.889595in}{2.935316in}}%
\pgfpathlineto{\pgfqpoint{3.902749in}{2.928386in}}%
\pgfpathlineto{\pgfqpoint{3.915907in}{2.921591in}}%
\pgfpathlineto{\pgfqpoint{3.929069in}{2.914932in}}%
\pgfpathlineto{\pgfqpoint{3.942235in}{2.908406in}}%
\pgfpathlineto{\pgfqpoint{3.934574in}{2.899017in}}%
\pgfpathlineto{\pgfqpoint{3.926908in}{2.889673in}}%
\pgfpathlineto{\pgfqpoint{3.919238in}{2.880374in}}%
\pgfpathlineto{\pgfqpoint{3.911562in}{2.871119in}}%
\pgfpathlineto{\pgfqpoint{3.898385in}{2.877572in}}%
\pgfpathlineto{\pgfqpoint{3.885211in}{2.884160in}}%
\pgfpathlineto{\pgfqpoint{3.872042in}{2.890883in}}%
\pgfpathlineto{\pgfqpoint{3.858876in}{2.897741in}}%
\pgfpathlineto{\pgfqpoint{3.866563in}{2.907062in}}%
\pgfpathlineto{\pgfqpoint{3.874246in}{2.916431in}}%
\pgfpathlineto{\pgfqpoint{3.881923in}{2.925848in}}%
\pgfpathlineto{\pgfqpoint{3.889595in}{2.935316in}}%
\pgfpathclose%
\pgfusepath{fill}%
\end{pgfscope}%
\begin{pgfscope}%
\pgfpathrectangle{\pgfqpoint{1.150000in}{0.150000in}}{\pgfqpoint{5.700000in}{5.700000in}}%
\pgfusepath{clip}%
\pgfsetbuttcap%
\pgfsetroundjoin%
\definecolor{currentfill}{rgb}{0.281887,0.150881,0.465405}%
\pgfsetfillcolor{currentfill}%
\pgfsetfillopacity{0.700000}%
\pgfsetlinewidth{0.000000pt}%
\definecolor{currentstroke}{rgb}{0.000000,0.000000,0.000000}%
\pgfsetstrokecolor{currentstroke}%
\pgfsetdash{}{0pt}%
\pgfpathmoveto{\pgfqpoint{4.682584in}{2.996847in}}%
\pgfpathlineto{\pgfqpoint{4.695910in}{2.994249in}}%
\pgfpathlineto{\pgfqpoint{4.709244in}{2.991761in}}%
\pgfpathlineto{\pgfqpoint{4.722586in}{2.989382in}}%
\pgfpathlineto{\pgfqpoint{4.735935in}{2.987112in}}%
\pgfpathlineto{\pgfqpoint{4.728530in}{2.978024in}}%
\pgfpathlineto{\pgfqpoint{4.721121in}{2.968983in}}%
\pgfpathlineto{\pgfqpoint{4.713707in}{2.959988in}}%
\pgfpathlineto{\pgfqpoint{4.706290in}{2.951036in}}%
\pgfpathlineto{\pgfqpoint{4.692928in}{2.953109in}}%
\pgfpathlineto{\pgfqpoint{4.679573in}{2.955290in}}%
\pgfpathlineto{\pgfqpoint{4.666227in}{2.957581in}}%
\pgfpathlineto{\pgfqpoint{4.652888in}{2.959982in}}%
\pgfpathlineto{\pgfqpoint{4.660319in}{2.969124in}}%
\pgfpathlineto{\pgfqpoint{4.667744in}{2.978314in}}%
\pgfpathlineto{\pgfqpoint{4.675166in}{2.987554in}}%
\pgfpathlineto{\pgfqpoint{4.682584in}{2.996847in}}%
\pgfpathclose%
\pgfusepath{fill}%
\end{pgfscope}%
\begin{pgfscope}%
\pgfpathrectangle{\pgfqpoint{1.150000in}{0.150000in}}{\pgfqpoint{5.700000in}{5.700000in}}%
\pgfusepath{clip}%
\pgfsetbuttcap%
\pgfsetroundjoin%
\definecolor{currentfill}{rgb}{0.283187,0.125848,0.444960}%
\pgfsetfillcolor{currentfill}%
\pgfsetfillopacity{0.700000}%
\pgfsetlinewidth{0.000000pt}%
\definecolor{currentstroke}{rgb}{0.000000,0.000000,0.000000}%
\pgfsetstrokecolor{currentstroke}%
\pgfsetdash{}{0pt}%
\pgfpathmoveto{\pgfqpoint{3.753667in}{2.957600in}}%
\pgfpathlineto{\pgfqpoint{3.766808in}{2.949624in}}%
\pgfpathlineto{\pgfqpoint{3.779951in}{2.941791in}}%
\pgfpathlineto{\pgfqpoint{3.793097in}{2.934100in}}%
\pgfpathlineto{\pgfqpoint{3.806247in}{2.926550in}}%
\pgfpathlineto{\pgfqpoint{3.798542in}{2.917216in}}%
\pgfpathlineto{\pgfqpoint{3.790832in}{2.907932in}}%
\pgfpathlineto{\pgfqpoint{3.783117in}{2.898698in}}%
\pgfpathlineto{\pgfqpoint{3.775397in}{2.889512in}}%
\pgfpathlineto{\pgfqpoint{3.762235in}{2.897009in}}%
\pgfpathlineto{\pgfqpoint{3.749077in}{2.904646in}}%
\pgfpathlineto{\pgfqpoint{3.735921in}{2.912426in}}%
\pgfpathlineto{\pgfqpoint{3.722768in}{2.920348in}}%
\pgfpathlineto{\pgfqpoint{3.730501in}{2.929581in}}%
\pgfpathlineto{\pgfqpoint{3.738228in}{2.938866in}}%
\pgfpathlineto{\pgfqpoint{3.745950in}{2.948206in}}%
\pgfpathlineto{\pgfqpoint{3.753667in}{2.957600in}}%
\pgfpathclose%
\pgfusepath{fill}%
\end{pgfscope}%
\begin{pgfscope}%
\pgfpathrectangle{\pgfqpoint{1.150000in}{0.150000in}}{\pgfqpoint{5.700000in}{5.700000in}}%
\pgfusepath{clip}%
\pgfsetbuttcap%
\pgfsetroundjoin%
\definecolor{currentfill}{rgb}{0.273006,0.204520,0.501721}%
\pgfsetfillcolor{currentfill}%
\pgfsetfillopacity{0.700000}%
\pgfsetlinewidth{0.000000pt}%
\definecolor{currentstroke}{rgb}{0.000000,0.000000,0.000000}%
\pgfsetstrokecolor{currentstroke}%
\pgfsetdash{}{0pt}%
\pgfpathmoveto{\pgfqpoint{3.376299in}{3.118926in}}%
\pgfpathlineto{\pgfqpoint{3.389438in}{3.107272in}}%
\pgfpathlineto{\pgfqpoint{3.402576in}{3.095789in}}%
\pgfpathlineto{\pgfqpoint{3.415714in}{3.084476in}}%
\pgfpathlineto{\pgfqpoint{3.428851in}{3.073333in}}%
\pgfpathlineto{\pgfqpoint{3.421025in}{3.064233in}}%
\pgfpathlineto{\pgfqpoint{3.413192in}{3.055204in}}%
\pgfpathlineto{\pgfqpoint{3.405353in}{3.046245in}}%
\pgfpathlineto{\pgfqpoint{3.397508in}{3.037356in}}%
\pgfpathlineto{\pgfqpoint{3.384356in}{3.048483in}}%
\pgfpathlineto{\pgfqpoint{3.371204in}{3.059779in}}%
\pgfpathlineto{\pgfqpoint{3.358051in}{3.071245in}}%
\pgfpathlineto{\pgfqpoint{3.344898in}{3.082883in}}%
\pgfpathlineto{\pgfqpoint{3.352758in}{3.091782in}}%
\pgfpathlineto{\pgfqpoint{3.360611in}{3.100755in}}%
\pgfpathlineto{\pgfqpoint{3.368458in}{3.109803in}}%
\pgfpathlineto{\pgfqpoint{3.376299in}{3.118926in}}%
\pgfpathclose%
\pgfusepath{fill}%
\end{pgfscope}%
\begin{pgfscope}%
\pgfpathrectangle{\pgfqpoint{1.150000in}{0.150000in}}{\pgfqpoint{5.700000in}{5.700000in}}%
\pgfusepath{clip}%
\pgfsetbuttcap%
\pgfsetroundjoin%
\definecolor{currentfill}{rgb}{0.283091,0.110553,0.431554}%
\pgfsetfillcolor{currentfill}%
\pgfsetfillopacity{0.700000}%
\pgfsetlinewidth{0.000000pt}%
\definecolor{currentstroke}{rgb}{0.000000,0.000000,0.000000}%
\pgfsetstrokecolor{currentstroke}%
\pgfsetdash{}{0pt}%
\pgfpathmoveto{\pgfqpoint{4.025485in}{2.921310in}}%
\pgfpathlineto{\pgfqpoint{4.038660in}{2.915353in}}%
\pgfpathlineto{\pgfqpoint{4.051840in}{2.909525in}}%
\pgfpathlineto{\pgfqpoint{4.065024in}{2.903826in}}%
\pgfpathlineto{\pgfqpoint{4.078214in}{2.898255in}}%
\pgfpathlineto{\pgfqpoint{4.070595in}{2.888863in}}%
\pgfpathlineto{\pgfqpoint{4.062973in}{2.879512in}}%
\pgfpathlineto{\pgfqpoint{4.055345in}{2.870202in}}%
\pgfpathlineto{\pgfqpoint{4.047712in}{2.860932in}}%
\pgfpathlineto{\pgfqpoint{4.034511in}{2.866413in}}%
\pgfpathlineto{\pgfqpoint{4.021316in}{2.872023in}}%
\pgfpathlineto{\pgfqpoint{4.008125in}{2.877760in}}%
\pgfpathlineto{\pgfqpoint{3.994938in}{2.883628in}}%
\pgfpathlineto{\pgfqpoint{4.002582in}{2.892981in}}%
\pgfpathlineto{\pgfqpoint{4.010221in}{2.902379in}}%
\pgfpathlineto{\pgfqpoint{4.017856in}{2.911821in}}%
\pgfpathlineto{\pgfqpoint{4.025485in}{2.921310in}}%
\pgfpathclose%
\pgfusepath{fill}%
\end{pgfscope}%
\begin{pgfscope}%
\pgfpathrectangle{\pgfqpoint{1.150000in}{0.150000in}}{\pgfqpoint{5.700000in}{5.700000in}}%
\pgfusepath{clip}%
\pgfsetbuttcap%
\pgfsetroundjoin%
\definecolor{currentfill}{rgb}{0.283229,0.120777,0.440584}%
\pgfsetfillcolor{currentfill}%
\pgfsetfillopacity{0.700000}%
\pgfsetlinewidth{0.000000pt}%
\definecolor{currentstroke}{rgb}{0.000000,0.000000,0.000000}%
\pgfsetstrokecolor{currentstroke}%
\pgfsetdash{}{0pt}%
\pgfpathmoveto{\pgfqpoint{4.380469in}{2.936480in}}%
\pgfpathlineto{\pgfqpoint{4.393719in}{2.932608in}}%
\pgfpathlineto{\pgfqpoint{4.406976in}{2.928852in}}%
\pgfpathlineto{\pgfqpoint{4.420239in}{2.925213in}}%
\pgfpathlineto{\pgfqpoint{4.433509in}{2.921690in}}%
\pgfpathlineto{\pgfqpoint{4.426004in}{2.912424in}}%
\pgfpathlineto{\pgfqpoint{4.418494in}{2.903196in}}%
\pgfpathlineto{\pgfqpoint{4.410979in}{2.894006in}}%
\pgfpathlineto{\pgfqpoint{4.403460in}{2.884852in}}%
\pgfpathlineto{\pgfqpoint{4.390179in}{2.888232in}}%
\pgfpathlineto{\pgfqpoint{4.376904in}{2.891727in}}%
\pgfpathlineto{\pgfqpoint{4.363636in}{2.895339in}}%
\pgfpathlineto{\pgfqpoint{4.350375in}{2.899067in}}%
\pgfpathlineto{\pgfqpoint{4.357905in}{2.908358in}}%
\pgfpathlineto{\pgfqpoint{4.365431in}{2.917690in}}%
\pgfpathlineto{\pgfqpoint{4.372952in}{2.927063in}}%
\pgfpathlineto{\pgfqpoint{4.380469in}{2.936480in}}%
\pgfpathclose%
\pgfusepath{fill}%
\end{pgfscope}%
\begin{pgfscope}%
\pgfpathrectangle{\pgfqpoint{1.150000in}{0.150000in}}{\pgfqpoint{5.700000in}{5.700000in}}%
\pgfusepath{clip}%
\pgfsetbuttcap%
\pgfsetroundjoin%
\definecolor{currentfill}{rgb}{0.283091,0.110553,0.431554}%
\pgfsetfillcolor{currentfill}%
\pgfsetfillopacity{0.700000}%
\pgfsetlinewidth{0.000000pt}%
\definecolor{currentstroke}{rgb}{0.000000,0.000000,0.000000}%
\pgfsetstrokecolor{currentstroke}%
\pgfsetdash{}{0pt}%
\pgfpathmoveto{\pgfqpoint{4.161398in}{2.914823in}}%
\pgfpathlineto{\pgfqpoint{4.174601in}{2.909774in}}%
\pgfpathlineto{\pgfqpoint{4.187810in}{2.904848in}}%
\pgfpathlineto{\pgfqpoint{4.201024in}{2.900045in}}%
\pgfpathlineto{\pgfqpoint{4.214243in}{2.895365in}}%
\pgfpathlineto{\pgfqpoint{4.206666in}{2.886019in}}%
\pgfpathlineto{\pgfqpoint{4.199085in}{2.876712in}}%
\pgfpathlineto{\pgfqpoint{4.191499in}{2.867443in}}%
\pgfpathlineto{\pgfqpoint{4.183909in}{2.858209in}}%
\pgfpathlineto{\pgfqpoint{4.170678in}{2.862781in}}%
\pgfpathlineto{\pgfqpoint{4.157453in}{2.867476in}}%
\pgfpathlineto{\pgfqpoint{4.144234in}{2.872294in}}%
\pgfpathlineto{\pgfqpoint{4.131019in}{2.877236in}}%
\pgfpathlineto{\pgfqpoint{4.138621in}{2.886571in}}%
\pgfpathlineto{\pgfqpoint{4.146218in}{2.895946in}}%
\pgfpathlineto{\pgfqpoint{4.153811in}{2.905363in}}%
\pgfpathlineto{\pgfqpoint{4.161398in}{2.914823in}}%
\pgfpathclose%
\pgfusepath{fill}%
\end{pgfscope}%
\begin{pgfscope}%
\pgfpathrectangle{\pgfqpoint{1.150000in}{0.150000in}}{\pgfqpoint{5.700000in}{5.700000in}}%
\pgfusepath{clip}%
\pgfsetbuttcap%
\pgfsetroundjoin%
\definecolor{currentfill}{rgb}{0.282623,0.140926,0.457517}%
\pgfsetfillcolor{currentfill}%
\pgfsetfillopacity{0.700000}%
\pgfsetlinewidth{0.000000pt}%
\definecolor{currentstroke}{rgb}{0.000000,0.000000,0.000000}%
\pgfsetstrokecolor{currentstroke}%
\pgfsetdash{}{0pt}%
\pgfpathmoveto{\pgfqpoint{3.617630in}{2.988995in}}%
\pgfpathlineto{\pgfqpoint{3.630765in}{2.979892in}}%
\pgfpathlineto{\pgfqpoint{3.643902in}{2.970941in}}%
\pgfpathlineto{\pgfqpoint{3.657041in}{2.962140in}}%
\pgfpathlineto{\pgfqpoint{3.670182in}{2.953487in}}%
\pgfpathlineto{\pgfqpoint{3.662431in}{2.944264in}}%
\pgfpathlineto{\pgfqpoint{3.654675in}{2.935097in}}%
\pgfpathlineto{\pgfqpoint{3.646914in}{2.925986in}}%
\pgfpathlineto{\pgfqpoint{3.639147in}{2.916928in}}%
\pgfpathlineto{\pgfqpoint{3.625993in}{2.925545in}}%
\pgfpathlineto{\pgfqpoint{3.612841in}{2.934310in}}%
\pgfpathlineto{\pgfqpoint{3.599691in}{2.943226in}}%
\pgfpathlineto{\pgfqpoint{3.586543in}{2.952294in}}%
\pgfpathlineto{\pgfqpoint{3.594323in}{2.961380in}}%
\pgfpathlineto{\pgfqpoint{3.602098in}{2.970526in}}%
\pgfpathlineto{\pgfqpoint{3.609866in}{2.979730in}}%
\pgfpathlineto{\pgfqpoint{3.617630in}{2.988995in}}%
\pgfpathclose%
\pgfusepath{fill}%
\end{pgfscope}%
\begin{pgfscope}%
\pgfpathrectangle{\pgfqpoint{1.150000in}{0.150000in}}{\pgfqpoint{5.700000in}{5.700000in}}%
\pgfusepath{clip}%
\pgfsetbuttcap%
\pgfsetroundjoin%
\definecolor{currentfill}{rgb}{0.282623,0.140926,0.457517}%
\pgfsetfillcolor{currentfill}%
\pgfsetfillopacity{0.700000}%
\pgfsetlinewidth{0.000000pt}%
\definecolor{currentstroke}{rgb}{0.000000,0.000000,0.000000}%
\pgfsetstrokecolor{currentstroke}%
\pgfsetdash{}{0pt}%
\pgfpathmoveto{\pgfqpoint{4.599609in}{2.970687in}}%
\pgfpathlineto{\pgfqpoint{4.612918in}{2.967844in}}%
\pgfpathlineto{\pgfqpoint{4.626234in}{2.965113in}}%
\pgfpathlineto{\pgfqpoint{4.639557in}{2.962492in}}%
\pgfpathlineto{\pgfqpoint{4.652888in}{2.959982in}}%
\pgfpathlineto{\pgfqpoint{4.645454in}{2.950884in}}%
\pgfpathlineto{\pgfqpoint{4.638015in}{2.941829in}}%
\pgfpathlineto{\pgfqpoint{4.630572in}{2.932816in}}%
\pgfpathlineto{\pgfqpoint{4.623125in}{2.923840in}}%
\pgfpathlineto{\pgfqpoint{4.609781in}{2.926171in}}%
\pgfpathlineto{\pgfqpoint{4.596446in}{2.928612in}}%
\pgfpathlineto{\pgfqpoint{4.583117in}{2.931164in}}%
\pgfpathlineto{\pgfqpoint{4.569796in}{2.933827in}}%
\pgfpathlineto{\pgfqpoint{4.577256in}{2.942976in}}%
\pgfpathlineto{\pgfqpoint{4.584711in}{2.952167in}}%
\pgfpathlineto{\pgfqpoint{4.592162in}{2.961403in}}%
\pgfpathlineto{\pgfqpoint{4.599609in}{2.970687in}}%
\pgfpathclose%
\pgfusepath{fill}%
\end{pgfscope}%
\begin{pgfscope}%
\pgfpathrectangle{\pgfqpoint{1.150000in}{0.150000in}}{\pgfqpoint{5.700000in}{5.700000in}}%
\pgfusepath{clip}%
\pgfsetbuttcap%
\pgfsetroundjoin%
\definecolor{currentfill}{rgb}{0.277134,0.185228,0.489898}%
\pgfsetfillcolor{currentfill}%
\pgfsetfillopacity{0.700000}%
\pgfsetlinewidth{0.000000pt}%
\definecolor{currentstroke}{rgb}{0.000000,0.000000,0.000000}%
\pgfsetstrokecolor{currentstroke}%
\pgfsetdash{}{0pt}%
\pgfpathmoveto{\pgfqpoint{3.428851in}{3.073333in}}%
\pgfpathlineto{\pgfqpoint{3.441989in}{3.062356in}}%
\pgfpathlineto{\pgfqpoint{3.455127in}{3.051546in}}%
\pgfpathlineto{\pgfqpoint{3.468265in}{3.040901in}}%
\pgfpathlineto{\pgfqpoint{3.481404in}{3.030419in}}%
\pgfpathlineto{\pgfqpoint{3.473591in}{3.021342in}}%
\pgfpathlineto{\pgfqpoint{3.465772in}{3.012332in}}%
\pgfpathlineto{\pgfqpoint{3.457948in}{3.003388in}}%
\pgfpathlineto{\pgfqpoint{3.450117in}{2.994510in}}%
\pgfpathlineto{\pgfqpoint{3.436964in}{3.004975in}}%
\pgfpathlineto{\pgfqpoint{3.423812in}{3.015603in}}%
\pgfpathlineto{\pgfqpoint{3.410660in}{3.026396in}}%
\pgfpathlineto{\pgfqpoint{3.397508in}{3.037356in}}%
\pgfpathlineto{\pgfqpoint{3.405353in}{3.046245in}}%
\pgfpathlineto{\pgfqpoint{3.413192in}{3.055204in}}%
\pgfpathlineto{\pgfqpoint{3.421025in}{3.064233in}}%
\pgfpathlineto{\pgfqpoint{3.428851in}{3.073333in}}%
\pgfpathclose%
\pgfusepath{fill}%
\end{pgfscope}%
\begin{pgfscope}%
\pgfpathrectangle{\pgfqpoint{1.150000in}{0.150000in}}{\pgfqpoint{5.700000in}{5.700000in}}%
\pgfusepath{clip}%
\pgfsetbuttcap%
\pgfsetroundjoin%
\definecolor{currentfill}{rgb}{0.283091,0.110553,0.431554}%
\pgfsetfillcolor{currentfill}%
\pgfsetfillopacity{0.700000}%
\pgfsetlinewidth{0.000000pt}%
\definecolor{currentstroke}{rgb}{0.000000,0.000000,0.000000}%
\pgfsetstrokecolor{currentstroke}%
\pgfsetdash{}{0pt}%
\pgfpathmoveto{\pgfqpoint{4.297390in}{2.915161in}}%
\pgfpathlineto{\pgfqpoint{4.310627in}{2.910960in}}%
\pgfpathlineto{\pgfqpoint{4.323870in}{2.906877in}}%
\pgfpathlineto{\pgfqpoint{4.337119in}{2.902913in}}%
\pgfpathlineto{\pgfqpoint{4.350375in}{2.899067in}}%
\pgfpathlineto{\pgfqpoint{4.342840in}{2.889814in}}%
\pgfpathlineto{\pgfqpoint{4.335300in}{2.880598in}}%
\pgfpathlineto{\pgfqpoint{4.327755in}{2.871417in}}%
\pgfpathlineto{\pgfqpoint{4.320206in}{2.862269in}}%
\pgfpathlineto{\pgfqpoint{4.306940in}{2.865989in}}%
\pgfpathlineto{\pgfqpoint{4.293679in}{2.869827in}}%
\pgfpathlineto{\pgfqpoint{4.280425in}{2.873784in}}%
\pgfpathlineto{\pgfqpoint{4.267177in}{2.877859in}}%
\pgfpathlineto{\pgfqpoint{4.274737in}{2.887127in}}%
\pgfpathlineto{\pgfqpoint{4.282293in}{2.896432in}}%
\pgfpathlineto{\pgfqpoint{4.289844in}{2.905776in}}%
\pgfpathlineto{\pgfqpoint{4.297390in}{2.915161in}}%
\pgfpathclose%
\pgfusepath{fill}%
\end{pgfscope}%
\begin{pgfscope}%
\pgfpathrectangle{\pgfqpoint{1.150000in}{0.150000in}}{\pgfqpoint{5.700000in}{5.700000in}}%
\pgfusepath{clip}%
\pgfsetbuttcap%
\pgfsetroundjoin%
\definecolor{currentfill}{rgb}{0.283197,0.115680,0.436115}%
\pgfsetfillcolor{currentfill}%
\pgfsetfillopacity{0.700000}%
\pgfsetlinewidth{0.000000pt}%
\definecolor{currentstroke}{rgb}{0.000000,0.000000,0.000000}%
\pgfsetstrokecolor{currentstroke}%
\pgfsetdash{}{0pt}%
\pgfpathmoveto{\pgfqpoint{3.806247in}{2.926550in}}%
\pgfpathlineto{\pgfqpoint{3.819399in}{2.919140in}}%
\pgfpathlineto{\pgfqpoint{3.832555in}{2.911869in}}%
\pgfpathlineto{\pgfqpoint{3.845713in}{2.904736in}}%
\pgfpathlineto{\pgfqpoint{3.858876in}{2.897741in}}%
\pgfpathlineto{\pgfqpoint{3.851183in}{2.888468in}}%
\pgfpathlineto{\pgfqpoint{3.843485in}{2.879240in}}%
\pgfpathlineto{\pgfqpoint{3.835782in}{2.870058in}}%
\pgfpathlineto{\pgfqpoint{3.828074in}{2.860920in}}%
\pgfpathlineto{\pgfqpoint{3.814899in}{2.867861in}}%
\pgfpathlineto{\pgfqpoint{3.801729in}{2.874940in}}%
\pgfpathlineto{\pgfqpoint{3.788561in}{2.882156in}}%
\pgfpathlineto{\pgfqpoint{3.775397in}{2.889512in}}%
\pgfpathlineto{\pgfqpoint{3.783117in}{2.898698in}}%
\pgfpathlineto{\pgfqpoint{3.790832in}{2.907932in}}%
\pgfpathlineto{\pgfqpoint{3.798542in}{2.917216in}}%
\pgfpathlineto{\pgfqpoint{3.806247in}{2.926550in}}%
\pgfpathclose%
\pgfusepath{fill}%
\end{pgfscope}%
\begin{pgfscope}%
\pgfpathrectangle{\pgfqpoint{1.150000in}{0.150000in}}{\pgfqpoint{5.700000in}{5.700000in}}%
\pgfusepath{clip}%
\pgfsetbuttcap%
\pgfsetroundjoin%
\definecolor{currentfill}{rgb}{0.282910,0.105393,0.426902}%
\pgfsetfillcolor{currentfill}%
\pgfsetfillopacity{0.700000}%
\pgfsetlinewidth{0.000000pt}%
\definecolor{currentstroke}{rgb}{0.000000,0.000000,0.000000}%
\pgfsetstrokecolor{currentstroke}%
\pgfsetdash{}{0pt}%
\pgfpathmoveto{\pgfqpoint{3.942235in}{2.908406in}}%
\pgfpathlineto{\pgfqpoint{3.955404in}{2.902014in}}%
\pgfpathlineto{\pgfqpoint{3.968578in}{2.895754in}}%
\pgfpathlineto{\pgfqpoint{3.981756in}{2.889625in}}%
\pgfpathlineto{\pgfqpoint{3.994938in}{2.883628in}}%
\pgfpathlineto{\pgfqpoint{3.987289in}{2.874317in}}%
\pgfpathlineto{\pgfqpoint{3.979635in}{2.865047in}}%
\pgfpathlineto{\pgfqpoint{3.971975in}{2.855818in}}%
\pgfpathlineto{\pgfqpoint{3.964311in}{2.846629in}}%
\pgfpathlineto{\pgfqpoint{3.951118in}{2.852554in}}%
\pgfpathlineto{\pgfqpoint{3.937928in}{2.858610in}}%
\pgfpathlineto{\pgfqpoint{3.924743in}{2.864798in}}%
\pgfpathlineto{\pgfqpoint{3.911562in}{2.871119in}}%
\pgfpathlineto{\pgfqpoint{3.919238in}{2.880374in}}%
\pgfpathlineto{\pgfqpoint{3.926908in}{2.889673in}}%
\pgfpathlineto{\pgfqpoint{3.934574in}{2.899017in}}%
\pgfpathlineto{\pgfqpoint{3.942235in}{2.908406in}}%
\pgfpathclose%
\pgfusepath{fill}%
\end{pgfscope}%
\begin{pgfscope}%
\pgfpathrectangle{\pgfqpoint{1.150000in}{0.150000in}}{\pgfqpoint{5.700000in}{5.700000in}}%
\pgfusepath{clip}%
\pgfsetbuttcap%
\pgfsetroundjoin%
\definecolor{currentfill}{rgb}{0.283072,0.130895,0.449241}%
\pgfsetfillcolor{currentfill}%
\pgfsetfillopacity{0.700000}%
\pgfsetlinewidth{0.000000pt}%
\definecolor{currentstroke}{rgb}{0.000000,0.000000,0.000000}%
\pgfsetstrokecolor{currentstroke}%
\pgfsetdash{}{0pt}%
\pgfpathmoveto{\pgfqpoint{4.516585in}{2.945601in}}%
\pgfpathlineto{\pgfqpoint{4.529877in}{2.942489in}}%
\pgfpathlineto{\pgfqpoint{4.543177in}{2.939489in}}%
\pgfpathlineto{\pgfqpoint{4.556483in}{2.936602in}}%
\pgfpathlineto{\pgfqpoint{4.569796in}{2.933827in}}%
\pgfpathlineto{\pgfqpoint{4.562332in}{2.924719in}}%
\pgfpathlineto{\pgfqpoint{4.554864in}{2.915650in}}%
\pgfpathlineto{\pgfqpoint{4.547391in}{2.906618in}}%
\pgfpathlineto{\pgfqpoint{4.539913in}{2.897620in}}%
\pgfpathlineto{\pgfqpoint{4.526588in}{2.900233in}}%
\pgfpathlineto{\pgfqpoint{4.513270in}{2.902958in}}%
\pgfpathlineto{\pgfqpoint{4.499959in}{2.905796in}}%
\pgfpathlineto{\pgfqpoint{4.486655in}{2.908747in}}%
\pgfpathlineto{\pgfqpoint{4.494144in}{2.917900in}}%
\pgfpathlineto{\pgfqpoint{4.501629in}{2.927092in}}%
\pgfpathlineto{\pgfqpoint{4.509109in}{2.936325in}}%
\pgfpathlineto{\pgfqpoint{4.516585in}{2.945601in}}%
\pgfpathclose%
\pgfusepath{fill}%
\end{pgfscope}%
\begin{pgfscope}%
\pgfpathrectangle{\pgfqpoint{1.150000in}{0.150000in}}{\pgfqpoint{5.700000in}{5.700000in}}%
\pgfusepath{clip}%
\pgfsetbuttcap%
\pgfsetroundjoin%
\definecolor{currentfill}{rgb}{0.280255,0.165693,0.476498}%
\pgfsetfillcolor{currentfill}%
\pgfsetfillopacity{0.700000}%
\pgfsetlinewidth{0.000000pt}%
\definecolor{currentstroke}{rgb}{0.000000,0.000000,0.000000}%
\pgfsetstrokecolor{currentstroke}%
\pgfsetdash{}{0pt}%
\pgfpathmoveto{\pgfqpoint{3.481404in}{3.030419in}}%
\pgfpathlineto{\pgfqpoint{3.494543in}{3.020099in}}%
\pgfpathlineto{\pgfqpoint{3.507683in}{3.009940in}}%
\pgfpathlineto{\pgfqpoint{3.520823in}{2.999940in}}%
\pgfpathlineto{\pgfqpoint{3.533965in}{2.990099in}}%
\pgfpathlineto{\pgfqpoint{3.526166in}{2.981047in}}%
\pgfpathlineto{\pgfqpoint{3.518361in}{2.972056in}}%
\pgfpathlineto{\pgfqpoint{3.510550in}{2.963127in}}%
\pgfpathlineto{\pgfqpoint{3.502733in}{2.954259in}}%
\pgfpathlineto{\pgfqpoint{3.489578in}{2.964083in}}%
\pgfpathlineto{\pgfqpoint{3.476424in}{2.974065in}}%
\pgfpathlineto{\pgfqpoint{3.463270in}{2.984207in}}%
\pgfpathlineto{\pgfqpoint{3.450117in}{2.994510in}}%
\pgfpathlineto{\pgfqpoint{3.457948in}{3.003388in}}%
\pgfpathlineto{\pgfqpoint{3.465772in}{3.012332in}}%
\pgfpathlineto{\pgfqpoint{3.473591in}{3.021342in}}%
\pgfpathlineto{\pgfqpoint{3.481404in}{3.030419in}}%
\pgfpathclose%
\pgfusepath{fill}%
\end{pgfscope}%
\begin{pgfscope}%
\pgfpathrectangle{\pgfqpoint{1.150000in}{0.150000in}}{\pgfqpoint{5.700000in}{5.700000in}}%
\pgfusepath{clip}%
\pgfsetbuttcap%
\pgfsetroundjoin%
\definecolor{currentfill}{rgb}{0.253935,0.265254,0.529983}%
\pgfsetfillcolor{currentfill}%
\pgfsetfillopacity{0.700000}%
\pgfsetlinewidth{0.000000pt}%
\definecolor{currentstroke}{rgb}{0.000000,0.000000,0.000000}%
\pgfsetstrokecolor{currentstroke}%
\pgfsetdash{}{0pt}%
\pgfpathmoveto{\pgfqpoint{3.186965in}{3.236502in}}%
\pgfpathlineto{\pgfqpoint{3.200136in}{3.222680in}}%
\pgfpathlineto{\pgfqpoint{3.213305in}{3.209049in}}%
\pgfpathlineto{\pgfqpoint{3.226472in}{3.195608in}}%
\pgfpathlineto{\pgfqpoint{3.239637in}{3.182354in}}%
\pgfpathlineto{\pgfqpoint{3.231740in}{3.173519in}}%
\pgfpathlineto{\pgfqpoint{3.223837in}{3.164767in}}%
\pgfpathlineto{\pgfqpoint{3.215927in}{3.156097in}}%
\pgfpathlineto{\pgfqpoint{3.208011in}{3.147508in}}%
\pgfpathlineto{\pgfqpoint{3.194831in}{3.160764in}}%
\pgfpathlineto{\pgfqpoint{3.181648in}{3.174208in}}%
\pgfpathlineto{\pgfqpoint{3.168463in}{3.187841in}}%
\pgfpathlineto{\pgfqpoint{3.155275in}{3.201665in}}%
\pgfpathlineto{\pgfqpoint{3.163208in}{3.210245in}}%
\pgfpathlineto{\pgfqpoint{3.171134in}{3.218910in}}%
\pgfpathlineto{\pgfqpoint{3.179053in}{3.227663in}}%
\pgfpathlineto{\pgfqpoint{3.186965in}{3.236502in}}%
\pgfpathclose%
\pgfusepath{fill}%
\end{pgfscope}%
\begin{pgfscope}%
\pgfpathrectangle{\pgfqpoint{1.150000in}{0.150000in}}{\pgfqpoint{5.700000in}{5.700000in}}%
\pgfusepath{clip}%
\pgfsetbuttcap%
\pgfsetroundjoin%
\definecolor{currentfill}{rgb}{0.282910,0.105393,0.426902}%
\pgfsetfillcolor{currentfill}%
\pgfsetfillopacity{0.700000}%
\pgfsetlinewidth{0.000000pt}%
\definecolor{currentstroke}{rgb}{0.000000,0.000000,0.000000}%
\pgfsetstrokecolor{currentstroke}%
\pgfsetdash{}{0pt}%
\pgfpathmoveto{\pgfqpoint{4.078214in}{2.898255in}}%
\pgfpathlineto{\pgfqpoint{4.091407in}{2.892811in}}%
\pgfpathlineto{\pgfqpoint{4.104606in}{2.887494in}}%
\pgfpathlineto{\pgfqpoint{4.117810in}{2.882302in}}%
\pgfpathlineto{\pgfqpoint{4.131019in}{2.877236in}}%
\pgfpathlineto{\pgfqpoint{4.123413in}{2.867940in}}%
\pgfpathlineto{\pgfqpoint{4.115801in}{2.858682in}}%
\pgfpathlineto{\pgfqpoint{4.108184in}{2.849460in}}%
\pgfpathlineto{\pgfqpoint{4.100563in}{2.840273in}}%
\pgfpathlineto{\pgfqpoint{4.087343in}{2.845249in}}%
\pgfpathlineto{\pgfqpoint{4.074127in}{2.850351in}}%
\pgfpathlineto{\pgfqpoint{4.060917in}{2.855578in}}%
\pgfpathlineto{\pgfqpoint{4.047712in}{2.860932in}}%
\pgfpathlineto{\pgfqpoint{4.055345in}{2.870202in}}%
\pgfpathlineto{\pgfqpoint{4.062973in}{2.879512in}}%
\pgfpathlineto{\pgfqpoint{4.070595in}{2.888863in}}%
\pgfpathlineto{\pgfqpoint{4.078214in}{2.898255in}}%
\pgfpathclose%
\pgfusepath{fill}%
\end{pgfscope}%
\begin{pgfscope}%
\pgfpathrectangle{\pgfqpoint{1.150000in}{0.150000in}}{\pgfqpoint{5.700000in}{5.700000in}}%
\pgfusepath{clip}%
\pgfsetbuttcap%
\pgfsetroundjoin%
\definecolor{currentfill}{rgb}{0.283072,0.130895,0.449241}%
\pgfsetfillcolor{currentfill}%
\pgfsetfillopacity{0.700000}%
\pgfsetlinewidth{0.000000pt}%
\definecolor{currentstroke}{rgb}{0.000000,0.000000,0.000000}%
\pgfsetstrokecolor{currentstroke}%
\pgfsetdash{}{0pt}%
\pgfpathmoveto{\pgfqpoint{3.670182in}{2.953487in}}%
\pgfpathlineto{\pgfqpoint{3.683325in}{2.944983in}}%
\pgfpathlineto{\pgfqpoint{3.696470in}{2.936626in}}%
\pgfpathlineto{\pgfqpoint{3.709618in}{2.928414in}}%
\pgfpathlineto{\pgfqpoint{3.722768in}{2.920348in}}%
\pgfpathlineto{\pgfqpoint{3.715030in}{2.911168in}}%
\pgfpathlineto{\pgfqpoint{3.707287in}{2.902039in}}%
\pgfpathlineto{\pgfqpoint{3.699538in}{2.892961in}}%
\pgfpathlineto{\pgfqpoint{3.691784in}{2.883932in}}%
\pgfpathlineto{\pgfqpoint{3.678621in}{2.891963in}}%
\pgfpathlineto{\pgfqpoint{3.665460in}{2.900138in}}%
\pgfpathlineto{\pgfqpoint{3.652302in}{2.908460in}}%
\pgfpathlineto{\pgfqpoint{3.639147in}{2.916928in}}%
\pgfpathlineto{\pgfqpoint{3.646914in}{2.925986in}}%
\pgfpathlineto{\pgfqpoint{3.654675in}{2.935097in}}%
\pgfpathlineto{\pgfqpoint{3.662431in}{2.944264in}}%
\pgfpathlineto{\pgfqpoint{3.670182in}{2.953487in}}%
\pgfpathclose%
\pgfusepath{fill}%
\end{pgfscope}%
\begin{pgfscope}%
\pgfpathrectangle{\pgfqpoint{1.150000in}{0.150000in}}{\pgfqpoint{5.700000in}{5.700000in}}%
\pgfusepath{clip}%
\pgfsetbuttcap%
\pgfsetroundjoin%
\definecolor{currentfill}{rgb}{0.262138,0.242286,0.520837}%
\pgfsetfillcolor{currentfill}%
\pgfsetfillopacity{0.700000}%
\pgfsetlinewidth{0.000000pt}%
\definecolor{currentstroke}{rgb}{0.000000,0.000000,0.000000}%
\pgfsetstrokecolor{currentstroke}%
\pgfsetdash{}{0pt}%
\pgfpathmoveto{\pgfqpoint{3.239637in}{3.182354in}}%
\pgfpathlineto{\pgfqpoint{3.252799in}{3.169287in}}%
\pgfpathlineto{\pgfqpoint{3.265960in}{3.156403in}}%
\pgfpathlineto{\pgfqpoint{3.279120in}{3.143703in}}%
\pgfpathlineto{\pgfqpoint{3.292278in}{3.131183in}}%
\pgfpathlineto{\pgfqpoint{3.284397in}{3.122353in}}%
\pgfpathlineto{\pgfqpoint{3.276509in}{3.113601in}}%
\pgfpathlineto{\pgfqpoint{3.268616in}{3.104926in}}%
\pgfpathlineto{\pgfqpoint{3.260715in}{3.096329in}}%
\pgfpathlineto{\pgfqpoint{3.247542in}{3.108850in}}%
\pgfpathlineto{\pgfqpoint{3.234367in}{3.121553in}}%
\pgfpathlineto{\pgfqpoint{3.221190in}{3.134438in}}%
\pgfpathlineto{\pgfqpoint{3.208011in}{3.147508in}}%
\pgfpathlineto{\pgfqpoint{3.215927in}{3.156097in}}%
\pgfpathlineto{\pgfqpoint{3.223837in}{3.164767in}}%
\pgfpathlineto{\pgfqpoint{3.231740in}{3.173519in}}%
\pgfpathlineto{\pgfqpoint{3.239637in}{3.182354in}}%
\pgfpathclose%
\pgfusepath{fill}%
\end{pgfscope}%
\begin{pgfscope}%
\pgfpathrectangle{\pgfqpoint{1.150000in}{0.150000in}}{\pgfqpoint{5.700000in}{5.700000in}}%
\pgfusepath{clip}%
\pgfsetbuttcap%
\pgfsetroundjoin%
\definecolor{currentfill}{rgb}{0.281412,0.155834,0.469201}%
\pgfsetfillcolor{currentfill}%
\pgfsetfillopacity{0.700000}%
\pgfsetlinewidth{0.000000pt}%
\definecolor{currentstroke}{rgb}{0.000000,0.000000,0.000000}%
\pgfsetstrokecolor{currentstroke}%
\pgfsetdash{}{0pt}%
\pgfpathmoveto{\pgfqpoint{4.735935in}{2.987112in}}%
\pgfpathlineto{\pgfqpoint{4.749292in}{2.984951in}}%
\pgfpathlineto{\pgfqpoint{4.762657in}{2.982897in}}%
\pgfpathlineto{\pgfqpoint{4.776031in}{2.980951in}}%
\pgfpathlineto{\pgfqpoint{4.789412in}{2.979112in}}%
\pgfpathlineto{\pgfqpoint{4.782020in}{2.970228in}}%
\pgfpathlineto{\pgfqpoint{4.774624in}{2.961387in}}%
\pgfpathlineto{\pgfqpoint{4.767223in}{2.952588in}}%
\pgfpathlineto{\pgfqpoint{4.759818in}{2.943827in}}%
\pgfpathlineto{\pgfqpoint{4.746424in}{2.945468in}}%
\pgfpathlineto{\pgfqpoint{4.733038in}{2.947216in}}%
\pgfpathlineto{\pgfqpoint{4.719660in}{2.949072in}}%
\pgfpathlineto{\pgfqpoint{4.706290in}{2.951036in}}%
\pgfpathlineto{\pgfqpoint{4.713707in}{2.959988in}}%
\pgfpathlineto{\pgfqpoint{4.721121in}{2.968983in}}%
\pgfpathlineto{\pgfqpoint{4.728530in}{2.978024in}}%
\pgfpathlineto{\pgfqpoint{4.735935in}{2.987112in}}%
\pgfpathclose%
\pgfusepath{fill}%
\end{pgfscope}%
\begin{pgfscope}%
\pgfpathrectangle{\pgfqpoint{1.150000in}{0.150000in}}{\pgfqpoint{5.700000in}{5.700000in}}%
\pgfusepath{clip}%
\pgfsetbuttcap%
\pgfsetroundjoin%
\definecolor{currentfill}{rgb}{0.282910,0.105393,0.426902}%
\pgfsetfillcolor{currentfill}%
\pgfsetfillopacity{0.700000}%
\pgfsetlinewidth{0.000000pt}%
\definecolor{currentstroke}{rgb}{0.000000,0.000000,0.000000}%
\pgfsetstrokecolor{currentstroke}%
\pgfsetdash{}{0pt}%
\pgfpathmoveto{\pgfqpoint{4.214243in}{2.895365in}}%
\pgfpathlineto{\pgfqpoint{4.227468in}{2.890807in}}%
\pgfpathlineto{\pgfqpoint{4.240698in}{2.886371in}}%
\pgfpathlineto{\pgfqpoint{4.253935in}{2.882055in}}%
\pgfpathlineto{\pgfqpoint{4.267177in}{2.877859in}}%
\pgfpathlineto{\pgfqpoint{4.259612in}{2.868628in}}%
\pgfpathlineto{\pgfqpoint{4.252042in}{2.859432in}}%
\pgfpathlineto{\pgfqpoint{4.244467in}{2.850268in}}%
\pgfpathlineto{\pgfqpoint{4.236887in}{2.841136in}}%
\pgfpathlineto{\pgfqpoint{4.223634in}{2.845224in}}%
\pgfpathlineto{\pgfqpoint{4.210386in}{2.849431in}}%
\pgfpathlineto{\pgfqpoint{4.197145in}{2.853759in}}%
\pgfpathlineto{\pgfqpoint{4.183909in}{2.858209in}}%
\pgfpathlineto{\pgfqpoint{4.191499in}{2.867443in}}%
\pgfpathlineto{\pgfqpoint{4.199085in}{2.876712in}}%
\pgfpathlineto{\pgfqpoint{4.206666in}{2.886019in}}%
\pgfpathlineto{\pgfqpoint{4.214243in}{2.895365in}}%
\pgfpathclose%
\pgfusepath{fill}%
\end{pgfscope}%
\begin{pgfscope}%
\pgfpathrectangle{\pgfqpoint{1.150000in}{0.150000in}}{\pgfqpoint{5.700000in}{5.700000in}}%
\pgfusepath{clip}%
\pgfsetbuttcap%
\pgfsetroundjoin%
\definecolor{currentfill}{rgb}{0.283229,0.120777,0.440584}%
\pgfsetfillcolor{currentfill}%
\pgfsetfillopacity{0.700000}%
\pgfsetlinewidth{0.000000pt}%
\definecolor{currentstroke}{rgb}{0.000000,0.000000,0.000000}%
\pgfsetstrokecolor{currentstroke}%
\pgfsetdash{}{0pt}%
\pgfpathmoveto{\pgfqpoint{4.433509in}{2.921690in}}%
\pgfpathlineto{\pgfqpoint{4.446785in}{2.918283in}}%
\pgfpathlineto{\pgfqpoint{4.460068in}{2.914990in}}%
\pgfpathlineto{\pgfqpoint{4.473358in}{2.911811in}}%
\pgfpathlineto{\pgfqpoint{4.486655in}{2.908747in}}%
\pgfpathlineto{\pgfqpoint{4.479161in}{2.899631in}}%
\pgfpathlineto{\pgfqpoint{4.471663in}{2.890550in}}%
\pgfpathlineto{\pgfqpoint{4.464160in}{2.881502in}}%
\pgfpathlineto{\pgfqpoint{4.456653in}{2.872485in}}%
\pgfpathlineto{\pgfqpoint{4.443344in}{2.875405in}}%
\pgfpathlineto{\pgfqpoint{4.430043in}{2.878440in}}%
\pgfpathlineto{\pgfqpoint{4.416748in}{2.881589in}}%
\pgfpathlineto{\pgfqpoint{4.403460in}{2.884852in}}%
\pgfpathlineto{\pgfqpoint{4.410979in}{2.894006in}}%
\pgfpathlineto{\pgfqpoint{4.418494in}{2.903196in}}%
\pgfpathlineto{\pgfqpoint{4.426004in}{2.912424in}}%
\pgfpathlineto{\pgfqpoint{4.433509in}{2.921690in}}%
\pgfpathclose%
\pgfusepath{fill}%
\end{pgfscope}%
\begin{pgfscope}%
\pgfpathrectangle{\pgfqpoint{1.150000in}{0.150000in}}{\pgfqpoint{5.700000in}{5.700000in}}%
\pgfusepath{clip}%
\pgfsetbuttcap%
\pgfsetroundjoin%
\definecolor{currentfill}{rgb}{0.269308,0.218818,0.509577}%
\pgfsetfillcolor{currentfill}%
\pgfsetfillopacity{0.700000}%
\pgfsetlinewidth{0.000000pt}%
\definecolor{currentstroke}{rgb}{0.000000,0.000000,0.000000}%
\pgfsetstrokecolor{currentstroke}%
\pgfsetdash{}{0pt}%
\pgfpathmoveto{\pgfqpoint{3.292278in}{3.131183in}}%
\pgfpathlineto{\pgfqpoint{3.305434in}{3.118843in}}%
\pgfpathlineto{\pgfqpoint{3.318590in}{3.106680in}}%
\pgfpathlineto{\pgfqpoint{3.331744in}{3.094694in}}%
\pgfpathlineto{\pgfqpoint{3.344898in}{3.082883in}}%
\pgfpathlineto{\pgfqpoint{3.337032in}{3.074058in}}%
\pgfpathlineto{\pgfqpoint{3.329160in}{3.065306in}}%
\pgfpathlineto{\pgfqpoint{3.321282in}{3.056628in}}%
\pgfpathlineto{\pgfqpoint{3.313397in}{3.048022in}}%
\pgfpathlineto{\pgfqpoint{3.300228in}{3.059835in}}%
\pgfpathlineto{\pgfqpoint{3.287058in}{3.071822in}}%
\pgfpathlineto{\pgfqpoint{3.273887in}{3.083987in}}%
\pgfpathlineto{\pgfqpoint{3.260715in}{3.096329in}}%
\pgfpathlineto{\pgfqpoint{3.268616in}{3.104926in}}%
\pgfpathlineto{\pgfqpoint{3.276509in}{3.113601in}}%
\pgfpathlineto{\pgfqpoint{3.284397in}{3.122353in}}%
\pgfpathlineto{\pgfqpoint{3.292278in}{3.131183in}}%
\pgfpathclose%
\pgfusepath{fill}%
\end{pgfscope}%
\begin{pgfscope}%
\pgfpathrectangle{\pgfqpoint{1.150000in}{0.150000in}}{\pgfqpoint{5.700000in}{5.700000in}}%
\pgfusepath{clip}%
\pgfsetbuttcap%
\pgfsetroundjoin%
\definecolor{currentfill}{rgb}{0.281887,0.150881,0.465405}%
\pgfsetfillcolor{currentfill}%
\pgfsetfillopacity{0.700000}%
\pgfsetlinewidth{0.000000pt}%
\definecolor{currentstroke}{rgb}{0.000000,0.000000,0.000000}%
\pgfsetstrokecolor{currentstroke}%
\pgfsetdash{}{0pt}%
\pgfpathmoveto{\pgfqpoint{3.533965in}{2.990099in}}%
\pgfpathlineto{\pgfqpoint{3.547107in}{2.980415in}}%
\pgfpathlineto{\pgfqpoint{3.560251in}{2.970887in}}%
\pgfpathlineto{\pgfqpoint{3.573396in}{2.961514in}}%
\pgfpathlineto{\pgfqpoint{3.586543in}{2.952294in}}%
\pgfpathlineto{\pgfqpoint{3.578757in}{2.943266in}}%
\pgfpathlineto{\pgfqpoint{3.570966in}{2.934295in}}%
\pgfpathlineto{\pgfqpoint{3.563168in}{2.925381in}}%
\pgfpathlineto{\pgfqpoint{3.555366in}{2.916525in}}%
\pgfpathlineto{\pgfqpoint{3.542205in}{2.925727in}}%
\pgfpathlineto{\pgfqpoint{3.529047in}{2.935082in}}%
\pgfpathlineto{\pgfqpoint{3.515889in}{2.944593in}}%
\pgfpathlineto{\pgfqpoint{3.502733in}{2.954259in}}%
\pgfpathlineto{\pgfqpoint{3.510550in}{2.963127in}}%
\pgfpathlineto{\pgfqpoint{3.518361in}{2.972056in}}%
\pgfpathlineto{\pgfqpoint{3.526166in}{2.981047in}}%
\pgfpathlineto{\pgfqpoint{3.533965in}{2.990099in}}%
\pgfpathclose%
\pgfusepath{fill}%
\end{pgfscope}%
\begin{pgfscope}%
\pgfpathrectangle{\pgfqpoint{1.150000in}{0.150000in}}{\pgfqpoint{5.700000in}{5.700000in}}%
\pgfusepath{clip}%
\pgfsetbuttcap%
\pgfsetroundjoin%
\definecolor{currentfill}{rgb}{0.282910,0.105393,0.426902}%
\pgfsetfillcolor{currentfill}%
\pgfsetfillopacity{0.700000}%
\pgfsetlinewidth{0.000000pt}%
\definecolor{currentstroke}{rgb}{0.000000,0.000000,0.000000}%
\pgfsetstrokecolor{currentstroke}%
\pgfsetdash{}{0pt}%
\pgfpathmoveto{\pgfqpoint{3.858876in}{2.897741in}}%
\pgfpathlineto{\pgfqpoint{3.872042in}{2.890883in}}%
\pgfpathlineto{\pgfqpoint{3.885211in}{2.884160in}}%
\pgfpathlineto{\pgfqpoint{3.898385in}{2.877572in}}%
\pgfpathlineto{\pgfqpoint{3.911562in}{2.871119in}}%
\pgfpathlineto{\pgfqpoint{3.903881in}{2.861906in}}%
\pgfpathlineto{\pgfqpoint{3.896195in}{2.852735in}}%
\pgfpathlineto{\pgfqpoint{3.888503in}{2.843605in}}%
\pgfpathlineto{\pgfqpoint{3.880807in}{2.834515in}}%
\pgfpathlineto{\pgfqpoint{3.867618in}{2.840914in}}%
\pgfpathlineto{\pgfqpoint{3.854433in}{2.847448in}}%
\pgfpathlineto{\pgfqpoint{3.841252in}{2.854116in}}%
\pgfpathlineto{\pgfqpoint{3.828074in}{2.860920in}}%
\pgfpathlineto{\pgfqpoint{3.835782in}{2.870058in}}%
\pgfpathlineto{\pgfqpoint{3.843485in}{2.879240in}}%
\pgfpathlineto{\pgfqpoint{3.851183in}{2.888468in}}%
\pgfpathlineto{\pgfqpoint{3.858876in}{2.897741in}}%
\pgfpathclose%
\pgfusepath{fill}%
\end{pgfscope}%
\begin{pgfscope}%
\pgfpathrectangle{\pgfqpoint{1.150000in}{0.150000in}}{\pgfqpoint{5.700000in}{5.700000in}}%
\pgfusepath{clip}%
\pgfsetbuttcap%
\pgfsetroundjoin%
\definecolor{currentfill}{rgb}{0.282290,0.145912,0.461510}%
\pgfsetfillcolor{currentfill}%
\pgfsetfillopacity{0.700000}%
\pgfsetlinewidth{0.000000pt}%
\definecolor{currentstroke}{rgb}{0.000000,0.000000,0.000000}%
\pgfsetstrokecolor{currentstroke}%
\pgfsetdash{}{0pt}%
\pgfpathmoveto{\pgfqpoint{4.652888in}{2.959982in}}%
\pgfpathlineto{\pgfqpoint{4.666227in}{2.957581in}}%
\pgfpathlineto{\pgfqpoint{4.679573in}{2.955290in}}%
\pgfpathlineto{\pgfqpoint{4.692928in}{2.953109in}}%
\pgfpathlineto{\pgfqpoint{4.706290in}{2.951036in}}%
\pgfpathlineto{\pgfqpoint{4.698868in}{2.942125in}}%
\pgfpathlineto{\pgfqpoint{4.691441in}{2.933253in}}%
\pgfpathlineto{\pgfqpoint{4.684010in}{2.924417in}}%
\pgfpathlineto{\pgfqpoint{4.676575in}{2.915615in}}%
\pgfpathlineto{\pgfqpoint{4.663201in}{2.917508in}}%
\pgfpathlineto{\pgfqpoint{4.649834in}{2.919509in}}%
\pgfpathlineto{\pgfqpoint{4.636476in}{2.921620in}}%
\pgfpathlineto{\pgfqpoint{4.623125in}{2.923840in}}%
\pgfpathlineto{\pgfqpoint{4.630572in}{2.932816in}}%
\pgfpathlineto{\pgfqpoint{4.638015in}{2.941829in}}%
\pgfpathlineto{\pgfqpoint{4.645454in}{2.950884in}}%
\pgfpathlineto{\pgfqpoint{4.652888in}{2.959982in}}%
\pgfpathclose%
\pgfusepath{fill}%
\end{pgfscope}%
\begin{pgfscope}%
\pgfpathrectangle{\pgfqpoint{1.150000in}{0.150000in}}{\pgfqpoint{5.700000in}{5.700000in}}%
\pgfusepath{clip}%
\pgfsetbuttcap%
\pgfsetroundjoin%
\definecolor{currentfill}{rgb}{0.282656,0.100196,0.422160}%
\pgfsetfillcolor{currentfill}%
\pgfsetfillopacity{0.700000}%
\pgfsetlinewidth{0.000000pt}%
\definecolor{currentstroke}{rgb}{0.000000,0.000000,0.000000}%
\pgfsetstrokecolor{currentstroke}%
\pgfsetdash{}{0pt}%
\pgfpathmoveto{\pgfqpoint{3.994938in}{2.883628in}}%
\pgfpathlineto{\pgfqpoint{4.008125in}{2.877760in}}%
\pgfpathlineto{\pgfqpoint{4.021316in}{2.872023in}}%
\pgfpathlineto{\pgfqpoint{4.034511in}{2.866413in}}%
\pgfpathlineto{\pgfqpoint{4.047712in}{2.860932in}}%
\pgfpathlineto{\pgfqpoint{4.040074in}{2.851700in}}%
\pgfpathlineto{\pgfqpoint{4.032431in}{2.842505in}}%
\pgfpathlineto{\pgfqpoint{4.024784in}{2.833346in}}%
\pgfpathlineto{\pgfqpoint{4.017131in}{2.824223in}}%
\pgfpathlineto{\pgfqpoint{4.003919in}{2.829631in}}%
\pgfpathlineto{\pgfqpoint{3.990712in}{2.835168in}}%
\pgfpathlineto{\pgfqpoint{3.977509in}{2.840834in}}%
\pgfpathlineto{\pgfqpoint{3.964311in}{2.846629in}}%
\pgfpathlineto{\pgfqpoint{3.971975in}{2.855818in}}%
\pgfpathlineto{\pgfqpoint{3.979635in}{2.865047in}}%
\pgfpathlineto{\pgfqpoint{3.987289in}{2.874317in}}%
\pgfpathlineto{\pgfqpoint{3.994938in}{2.883628in}}%
\pgfpathclose%
\pgfusepath{fill}%
\end{pgfscope}%
\begin{pgfscope}%
\pgfpathrectangle{\pgfqpoint{1.150000in}{0.150000in}}{\pgfqpoint{5.700000in}{5.700000in}}%
\pgfusepath{clip}%
\pgfsetbuttcap%
\pgfsetroundjoin%
\definecolor{currentfill}{rgb}{0.283229,0.120777,0.440584}%
\pgfsetfillcolor{currentfill}%
\pgfsetfillopacity{0.700000}%
\pgfsetlinewidth{0.000000pt}%
\definecolor{currentstroke}{rgb}{0.000000,0.000000,0.000000}%
\pgfsetstrokecolor{currentstroke}%
\pgfsetdash{}{0pt}%
\pgfpathmoveto{\pgfqpoint{3.722768in}{2.920348in}}%
\pgfpathlineto{\pgfqpoint{3.735921in}{2.912426in}}%
\pgfpathlineto{\pgfqpoint{3.749077in}{2.904646in}}%
\pgfpathlineto{\pgfqpoint{3.762235in}{2.897009in}}%
\pgfpathlineto{\pgfqpoint{3.775397in}{2.889512in}}%
\pgfpathlineto{\pgfqpoint{3.767671in}{2.880375in}}%
\pgfpathlineto{\pgfqpoint{3.759940in}{2.871284in}}%
\pgfpathlineto{\pgfqpoint{3.752203in}{2.862240in}}%
\pgfpathlineto{\pgfqpoint{3.744461in}{2.853242in}}%
\pgfpathlineto{\pgfqpoint{3.731288in}{2.860702in}}%
\pgfpathlineto{\pgfqpoint{3.718117in}{2.868303in}}%
\pgfpathlineto{\pgfqpoint{3.704949in}{2.876046in}}%
\pgfpathlineto{\pgfqpoint{3.691784in}{2.883932in}}%
\pgfpathlineto{\pgfqpoint{3.699538in}{2.892961in}}%
\pgfpathlineto{\pgfqpoint{3.707287in}{2.902039in}}%
\pgfpathlineto{\pgfqpoint{3.715030in}{2.911168in}}%
\pgfpathlineto{\pgfqpoint{3.722768in}{2.920348in}}%
\pgfpathclose%
\pgfusepath{fill}%
\end{pgfscope}%
\begin{pgfscope}%
\pgfpathrectangle{\pgfqpoint{1.150000in}{0.150000in}}{\pgfqpoint{5.700000in}{5.700000in}}%
\pgfusepath{clip}%
\pgfsetbuttcap%
\pgfsetroundjoin%
\definecolor{currentfill}{rgb}{0.274128,0.199721,0.498911}%
\pgfsetfillcolor{currentfill}%
\pgfsetfillopacity{0.700000}%
\pgfsetlinewidth{0.000000pt}%
\definecolor{currentstroke}{rgb}{0.000000,0.000000,0.000000}%
\pgfsetstrokecolor{currentstroke}%
\pgfsetdash{}{0pt}%
\pgfpathmoveto{\pgfqpoint{3.344898in}{3.082883in}}%
\pgfpathlineto{\pgfqpoint{3.358051in}{3.071245in}}%
\pgfpathlineto{\pgfqpoint{3.371204in}{3.059779in}}%
\pgfpathlineto{\pgfqpoint{3.384356in}{3.048483in}}%
\pgfpathlineto{\pgfqpoint{3.397508in}{3.037356in}}%
\pgfpathlineto{\pgfqpoint{3.389657in}{3.028536in}}%
\pgfpathlineto{\pgfqpoint{3.381800in}{3.019786in}}%
\pgfpathlineto{\pgfqpoint{3.373937in}{3.011104in}}%
\pgfpathlineto{\pgfqpoint{3.366067in}{3.002490in}}%
\pgfpathlineto{\pgfqpoint{3.352900in}{3.013618in}}%
\pgfpathlineto{\pgfqpoint{3.339733in}{3.024915in}}%
\pgfpathlineto{\pgfqpoint{3.326565in}{3.036383in}}%
\pgfpathlineto{\pgfqpoint{3.313397in}{3.048022in}}%
\pgfpathlineto{\pgfqpoint{3.321282in}{3.056628in}}%
\pgfpathlineto{\pgfqpoint{3.329160in}{3.065306in}}%
\pgfpathlineto{\pgfqpoint{3.337032in}{3.074058in}}%
\pgfpathlineto{\pgfqpoint{3.344898in}{3.082883in}}%
\pgfpathclose%
\pgfusepath{fill}%
\end{pgfscope}%
\begin{pgfscope}%
\pgfpathrectangle{\pgfqpoint{1.150000in}{0.150000in}}{\pgfqpoint{5.700000in}{5.700000in}}%
\pgfusepath{clip}%
\pgfsetbuttcap%
\pgfsetroundjoin%
\definecolor{currentfill}{rgb}{0.283091,0.110553,0.431554}%
\pgfsetfillcolor{currentfill}%
\pgfsetfillopacity{0.700000}%
\pgfsetlinewidth{0.000000pt}%
\definecolor{currentstroke}{rgb}{0.000000,0.000000,0.000000}%
\pgfsetstrokecolor{currentstroke}%
\pgfsetdash{}{0pt}%
\pgfpathmoveto{\pgfqpoint{4.350375in}{2.899067in}}%
\pgfpathlineto{\pgfqpoint{4.363636in}{2.895339in}}%
\pgfpathlineto{\pgfqpoint{4.376904in}{2.891727in}}%
\pgfpathlineto{\pgfqpoint{4.390179in}{2.888232in}}%
\pgfpathlineto{\pgfqpoint{4.403460in}{2.884852in}}%
\pgfpathlineto{\pgfqpoint{4.395936in}{2.875732in}}%
\pgfpathlineto{\pgfqpoint{4.388408in}{2.866645in}}%
\pgfpathlineto{\pgfqpoint{4.380875in}{2.857587in}}%
\pgfpathlineto{\pgfqpoint{4.373337in}{2.848559in}}%
\pgfpathlineto{\pgfqpoint{4.360045in}{2.851812in}}%
\pgfpathlineto{\pgfqpoint{4.346759in}{2.855181in}}%
\pgfpathlineto{\pgfqpoint{4.333479in}{2.858666in}}%
\pgfpathlineto{\pgfqpoint{4.320206in}{2.862269in}}%
\pgfpathlineto{\pgfqpoint{4.327755in}{2.871417in}}%
\pgfpathlineto{\pgfqpoint{4.335300in}{2.880598in}}%
\pgfpathlineto{\pgfqpoint{4.342840in}{2.889814in}}%
\pgfpathlineto{\pgfqpoint{4.350375in}{2.899067in}}%
\pgfpathclose%
\pgfusepath{fill}%
\end{pgfscope}%
\begin{pgfscope}%
\pgfpathrectangle{\pgfqpoint{1.150000in}{0.150000in}}{\pgfqpoint{5.700000in}{5.700000in}}%
\pgfusepath{clip}%
\pgfsetbuttcap%
\pgfsetroundjoin%
\definecolor{currentfill}{rgb}{0.282656,0.100196,0.422160}%
\pgfsetfillcolor{currentfill}%
\pgfsetfillopacity{0.700000}%
\pgfsetlinewidth{0.000000pt}%
\definecolor{currentstroke}{rgb}{0.000000,0.000000,0.000000}%
\pgfsetstrokecolor{currentstroke}%
\pgfsetdash{}{0pt}%
\pgfpathmoveto{\pgfqpoint{4.131019in}{2.877236in}}%
\pgfpathlineto{\pgfqpoint{4.144234in}{2.872294in}}%
\pgfpathlineto{\pgfqpoint{4.157453in}{2.867476in}}%
\pgfpathlineto{\pgfqpoint{4.170678in}{2.862781in}}%
\pgfpathlineto{\pgfqpoint{4.183909in}{2.858209in}}%
\pgfpathlineto{\pgfqpoint{4.176313in}{2.849011in}}%
\pgfpathlineto{\pgfqpoint{4.168713in}{2.839845in}}%
\pgfpathlineto{\pgfqpoint{4.161107in}{2.830712in}}%
\pgfpathlineto{\pgfqpoint{4.153497in}{2.821609in}}%
\pgfpathlineto{\pgfqpoint{4.140255in}{2.826090in}}%
\pgfpathlineto{\pgfqpoint{4.127019in}{2.830694in}}%
\pgfpathlineto{\pgfqpoint{4.113788in}{2.835422in}}%
\pgfpathlineto{\pgfqpoint{4.100563in}{2.840273in}}%
\pgfpathlineto{\pgfqpoint{4.108184in}{2.849460in}}%
\pgfpathlineto{\pgfqpoint{4.115801in}{2.858682in}}%
\pgfpathlineto{\pgfqpoint{4.123413in}{2.867940in}}%
\pgfpathlineto{\pgfqpoint{4.131019in}{2.877236in}}%
\pgfpathclose%
\pgfusepath{fill}%
\end{pgfscope}%
\begin{pgfscope}%
\pgfpathrectangle{\pgfqpoint{1.150000in}{0.150000in}}{\pgfqpoint{5.700000in}{5.700000in}}%
\pgfusepath{clip}%
\pgfsetbuttcap%
\pgfsetroundjoin%
\definecolor{currentfill}{rgb}{0.282884,0.135920,0.453427}%
\pgfsetfillcolor{currentfill}%
\pgfsetfillopacity{0.700000}%
\pgfsetlinewidth{0.000000pt}%
\definecolor{currentstroke}{rgb}{0.000000,0.000000,0.000000}%
\pgfsetstrokecolor{currentstroke}%
\pgfsetdash{}{0pt}%
\pgfpathmoveto{\pgfqpoint{3.586543in}{2.952294in}}%
\pgfpathlineto{\pgfqpoint{3.599691in}{2.943226in}}%
\pgfpathlineto{\pgfqpoint{3.612841in}{2.934310in}}%
\pgfpathlineto{\pgfqpoint{3.625993in}{2.925545in}}%
\pgfpathlineto{\pgfqpoint{3.639147in}{2.916928in}}%
\pgfpathlineto{\pgfqpoint{3.631374in}{2.907925in}}%
\pgfpathlineto{\pgfqpoint{3.623596in}{2.898974in}}%
\pgfpathlineto{\pgfqpoint{3.615812in}{2.890076in}}%
\pgfpathlineto{\pgfqpoint{3.608022in}{2.881231in}}%
\pgfpathlineto{\pgfqpoint{3.594855in}{2.889829in}}%
\pgfpathlineto{\pgfqpoint{3.581690in}{2.898577in}}%
\pgfpathlineto{\pgfqpoint{3.568527in}{2.907475in}}%
\pgfpathlineto{\pgfqpoint{3.555366in}{2.916525in}}%
\pgfpathlineto{\pgfqpoint{3.563168in}{2.925381in}}%
\pgfpathlineto{\pgfqpoint{3.570966in}{2.934295in}}%
\pgfpathlineto{\pgfqpoint{3.578757in}{2.943266in}}%
\pgfpathlineto{\pgfqpoint{3.586543in}{2.952294in}}%
\pgfpathclose%
\pgfusepath{fill}%
\end{pgfscope}%
\begin{pgfscope}%
\pgfpathrectangle{\pgfqpoint{1.150000in}{0.150000in}}{\pgfqpoint{5.700000in}{5.700000in}}%
\pgfusepath{clip}%
\pgfsetbuttcap%
\pgfsetroundjoin%
\definecolor{currentfill}{rgb}{0.282884,0.135920,0.453427}%
\pgfsetfillcolor{currentfill}%
\pgfsetfillopacity{0.700000}%
\pgfsetlinewidth{0.000000pt}%
\definecolor{currentstroke}{rgb}{0.000000,0.000000,0.000000}%
\pgfsetstrokecolor{currentstroke}%
\pgfsetdash{}{0pt}%
\pgfpathmoveto{\pgfqpoint{4.569796in}{2.933827in}}%
\pgfpathlineto{\pgfqpoint{4.583117in}{2.931164in}}%
\pgfpathlineto{\pgfqpoint{4.596446in}{2.928612in}}%
\pgfpathlineto{\pgfqpoint{4.609781in}{2.926171in}}%
\pgfpathlineto{\pgfqpoint{4.623125in}{2.923840in}}%
\pgfpathlineto{\pgfqpoint{4.615673in}{2.914901in}}%
\pgfpathlineto{\pgfqpoint{4.608216in}{2.905997in}}%
\pgfpathlineto{\pgfqpoint{4.600755in}{2.897124in}}%
\pgfpathlineto{\pgfqpoint{4.593289in}{2.888282in}}%
\pgfpathlineto{\pgfqpoint{4.579934in}{2.890450in}}%
\pgfpathlineto{\pgfqpoint{4.566586in}{2.892729in}}%
\pgfpathlineto{\pgfqpoint{4.553246in}{2.895119in}}%
\pgfpathlineto{\pgfqpoint{4.539913in}{2.897620in}}%
\pgfpathlineto{\pgfqpoint{4.547391in}{2.906618in}}%
\pgfpathlineto{\pgfqpoint{4.554864in}{2.915650in}}%
\pgfpathlineto{\pgfqpoint{4.562332in}{2.924719in}}%
\pgfpathlineto{\pgfqpoint{4.569796in}{2.933827in}}%
\pgfpathclose%
\pgfusepath{fill}%
\end{pgfscope}%
\begin{pgfscope}%
\pgfpathrectangle{\pgfqpoint{1.150000in}{0.150000in}}{\pgfqpoint{5.700000in}{5.700000in}}%
\pgfusepath{clip}%
\pgfsetbuttcap%
\pgfsetroundjoin%
\definecolor{currentfill}{rgb}{0.278012,0.180367,0.486697}%
\pgfsetfillcolor{currentfill}%
\pgfsetfillopacity{0.700000}%
\pgfsetlinewidth{0.000000pt}%
\definecolor{currentstroke}{rgb}{0.000000,0.000000,0.000000}%
\pgfsetstrokecolor{currentstroke}%
\pgfsetdash{}{0pt}%
\pgfpathmoveto{\pgfqpoint{3.397508in}{3.037356in}}%
\pgfpathlineto{\pgfqpoint{3.410660in}{3.026396in}}%
\pgfpathlineto{\pgfqpoint{3.423812in}{3.015603in}}%
\pgfpathlineto{\pgfqpoint{3.436964in}{3.004975in}}%
\pgfpathlineto{\pgfqpoint{3.450117in}{2.994510in}}%
\pgfpathlineto{\pgfqpoint{3.442280in}{2.985696in}}%
\pgfpathlineto{\pgfqpoint{3.434438in}{2.976947in}}%
\pgfpathlineto{\pgfqpoint{3.426589in}{2.968262in}}%
\pgfpathlineto{\pgfqpoint{3.418734in}{2.959641in}}%
\pgfpathlineto{\pgfqpoint{3.405567in}{2.970106in}}%
\pgfpathlineto{\pgfqpoint{3.392400in}{2.980736in}}%
\pgfpathlineto{\pgfqpoint{3.379234in}{2.991530in}}%
\pgfpathlineto{\pgfqpoint{3.366067in}{3.002490in}}%
\pgfpathlineto{\pgfqpoint{3.373937in}{3.011104in}}%
\pgfpathlineto{\pgfqpoint{3.381800in}{3.019786in}}%
\pgfpathlineto{\pgfqpoint{3.389657in}{3.028536in}}%
\pgfpathlineto{\pgfqpoint{3.397508in}{3.037356in}}%
\pgfpathclose%
\pgfusepath{fill}%
\end{pgfscope}%
\begin{pgfscope}%
\pgfpathrectangle{\pgfqpoint{1.150000in}{0.150000in}}{\pgfqpoint{5.700000in}{5.700000in}}%
\pgfusepath{clip}%
\pgfsetbuttcap%
\pgfsetroundjoin%
\definecolor{currentfill}{rgb}{0.282910,0.105393,0.426902}%
\pgfsetfillcolor{currentfill}%
\pgfsetfillopacity{0.700000}%
\pgfsetlinewidth{0.000000pt}%
\definecolor{currentstroke}{rgb}{0.000000,0.000000,0.000000}%
\pgfsetstrokecolor{currentstroke}%
\pgfsetdash{}{0pt}%
\pgfpathmoveto{\pgfqpoint{4.267177in}{2.877859in}}%
\pgfpathlineto{\pgfqpoint{4.280425in}{2.873784in}}%
\pgfpathlineto{\pgfqpoint{4.293679in}{2.869827in}}%
\pgfpathlineto{\pgfqpoint{4.306940in}{2.865989in}}%
\pgfpathlineto{\pgfqpoint{4.320206in}{2.862269in}}%
\pgfpathlineto{\pgfqpoint{4.312652in}{2.853153in}}%
\pgfpathlineto{\pgfqpoint{4.305094in}{2.844067in}}%
\pgfpathlineto{\pgfqpoint{4.297530in}{2.835010in}}%
\pgfpathlineto{\pgfqpoint{4.289962in}{2.825980in}}%
\pgfpathlineto{\pgfqpoint{4.276684in}{2.829592in}}%
\pgfpathlineto{\pgfqpoint{4.263412in}{2.833321in}}%
\pgfpathlineto{\pgfqpoint{4.250147in}{2.837169in}}%
\pgfpathlineto{\pgfqpoint{4.236887in}{2.841136in}}%
\pgfpathlineto{\pgfqpoint{4.244467in}{2.850268in}}%
\pgfpathlineto{\pgfqpoint{4.252042in}{2.859432in}}%
\pgfpathlineto{\pgfqpoint{4.259612in}{2.868628in}}%
\pgfpathlineto{\pgfqpoint{4.267177in}{2.877859in}}%
\pgfpathclose%
\pgfusepath{fill}%
\end{pgfscope}%
\begin{pgfscope}%
\pgfpathrectangle{\pgfqpoint{1.150000in}{0.150000in}}{\pgfqpoint{5.700000in}{5.700000in}}%
\pgfusepath{clip}%
\pgfsetbuttcap%
\pgfsetroundjoin%
\definecolor{currentfill}{rgb}{0.282656,0.100196,0.422160}%
\pgfsetfillcolor{currentfill}%
\pgfsetfillopacity{0.700000}%
\pgfsetlinewidth{0.000000pt}%
\definecolor{currentstroke}{rgb}{0.000000,0.000000,0.000000}%
\pgfsetstrokecolor{currentstroke}%
\pgfsetdash{}{0pt}%
\pgfpathmoveto{\pgfqpoint{3.911562in}{2.871119in}}%
\pgfpathlineto{\pgfqpoint{3.924743in}{2.864798in}}%
\pgfpathlineto{\pgfqpoint{3.937928in}{2.858610in}}%
\pgfpathlineto{\pgfqpoint{3.951118in}{2.852554in}}%
\pgfpathlineto{\pgfqpoint{3.964311in}{2.846629in}}%
\pgfpathlineto{\pgfqpoint{3.956642in}{2.837477in}}%
\pgfpathlineto{\pgfqpoint{3.948967in}{2.828363in}}%
\pgfpathlineto{\pgfqpoint{3.941288in}{2.819286in}}%
\pgfpathlineto{\pgfqpoint{3.933603in}{2.810244in}}%
\pgfpathlineto{\pgfqpoint{3.920398in}{2.816114in}}%
\pgfpathlineto{\pgfqpoint{3.907197in}{2.822116in}}%
\pgfpathlineto{\pgfqpoint{3.894000in}{2.828249in}}%
\pgfpathlineto{\pgfqpoint{3.880807in}{2.834515in}}%
\pgfpathlineto{\pgfqpoint{3.888503in}{2.843605in}}%
\pgfpathlineto{\pgfqpoint{3.896195in}{2.852735in}}%
\pgfpathlineto{\pgfqpoint{3.903881in}{2.861906in}}%
\pgfpathlineto{\pgfqpoint{3.911562in}{2.871119in}}%
\pgfpathclose%
\pgfusepath{fill}%
\end{pgfscope}%
\begin{pgfscope}%
\pgfpathrectangle{\pgfqpoint{1.150000in}{0.150000in}}{\pgfqpoint{5.700000in}{5.700000in}}%
\pgfusepath{clip}%
\pgfsetbuttcap%
\pgfsetroundjoin%
\definecolor{currentfill}{rgb}{0.283091,0.110553,0.431554}%
\pgfsetfillcolor{currentfill}%
\pgfsetfillopacity{0.700000}%
\pgfsetlinewidth{0.000000pt}%
\definecolor{currentstroke}{rgb}{0.000000,0.000000,0.000000}%
\pgfsetstrokecolor{currentstroke}%
\pgfsetdash{}{0pt}%
\pgfpathmoveto{\pgfqpoint{3.775397in}{2.889512in}}%
\pgfpathlineto{\pgfqpoint{3.788561in}{2.882156in}}%
\pgfpathlineto{\pgfqpoint{3.801729in}{2.874940in}}%
\pgfpathlineto{\pgfqpoint{3.814899in}{2.867861in}}%
\pgfpathlineto{\pgfqpoint{3.828074in}{2.860920in}}%
\pgfpathlineto{\pgfqpoint{3.820360in}{2.851826in}}%
\pgfpathlineto{\pgfqpoint{3.812641in}{2.842774in}}%
\pgfpathlineto{\pgfqpoint{3.804917in}{2.833764in}}%
\pgfpathlineto{\pgfqpoint{3.797188in}{2.824796in}}%
\pgfpathlineto{\pgfqpoint{3.784001in}{2.831700in}}%
\pgfpathlineto{\pgfqpoint{3.770818in}{2.838742in}}%
\pgfpathlineto{\pgfqpoint{3.757638in}{2.845922in}}%
\pgfpathlineto{\pgfqpoint{3.744461in}{2.853242in}}%
\pgfpathlineto{\pgfqpoint{3.752203in}{2.862240in}}%
\pgfpathlineto{\pgfqpoint{3.759940in}{2.871284in}}%
\pgfpathlineto{\pgfqpoint{3.767671in}{2.880375in}}%
\pgfpathlineto{\pgfqpoint{3.775397in}{2.889512in}}%
\pgfpathclose%
\pgfusepath{fill}%
\end{pgfscope}%
\begin{pgfscope}%
\pgfpathrectangle{\pgfqpoint{1.150000in}{0.150000in}}{\pgfqpoint{5.700000in}{5.700000in}}%
\pgfusepath{clip}%
\pgfsetbuttcap%
\pgfsetroundjoin%
\definecolor{currentfill}{rgb}{0.283187,0.125848,0.444960}%
\pgfsetfillcolor{currentfill}%
\pgfsetfillopacity{0.700000}%
\pgfsetlinewidth{0.000000pt}%
\definecolor{currentstroke}{rgb}{0.000000,0.000000,0.000000}%
\pgfsetstrokecolor{currentstroke}%
\pgfsetdash{}{0pt}%
\pgfpathmoveto{\pgfqpoint{4.486655in}{2.908747in}}%
\pgfpathlineto{\pgfqpoint{4.499959in}{2.905796in}}%
\pgfpathlineto{\pgfqpoint{4.513270in}{2.902958in}}%
\pgfpathlineto{\pgfqpoint{4.526588in}{2.900233in}}%
\pgfpathlineto{\pgfqpoint{4.539913in}{2.897620in}}%
\pgfpathlineto{\pgfqpoint{4.532431in}{2.888655in}}%
\pgfpathlineto{\pgfqpoint{4.524945in}{2.879720in}}%
\pgfpathlineto{\pgfqpoint{4.517453in}{2.870815in}}%
\pgfpathlineto{\pgfqpoint{4.509957in}{2.861936in}}%
\pgfpathlineto{\pgfqpoint{4.496620in}{2.864405in}}%
\pgfpathlineto{\pgfqpoint{4.483291in}{2.866985in}}%
\pgfpathlineto{\pgfqpoint{4.469968in}{2.869679in}}%
\pgfpathlineto{\pgfqpoint{4.456653in}{2.872485in}}%
\pgfpathlineto{\pgfqpoint{4.464160in}{2.881502in}}%
\pgfpathlineto{\pgfqpoint{4.471663in}{2.890550in}}%
\pgfpathlineto{\pgfqpoint{4.479161in}{2.899631in}}%
\pgfpathlineto{\pgfqpoint{4.486655in}{2.908747in}}%
\pgfpathclose%
\pgfusepath{fill}%
\end{pgfscope}%
\begin{pgfscope}%
\pgfpathrectangle{\pgfqpoint{1.150000in}{0.150000in}}{\pgfqpoint{5.700000in}{5.700000in}}%
\pgfusepath{clip}%
\pgfsetbuttcap%
\pgfsetroundjoin%
\definecolor{currentfill}{rgb}{0.280868,0.160771,0.472899}%
\pgfsetfillcolor{currentfill}%
\pgfsetfillopacity{0.700000}%
\pgfsetlinewidth{0.000000pt}%
\definecolor{currentstroke}{rgb}{0.000000,0.000000,0.000000}%
\pgfsetstrokecolor{currentstroke}%
\pgfsetdash{}{0pt}%
\pgfpathmoveto{\pgfqpoint{4.789412in}{2.979112in}}%
\pgfpathlineto{\pgfqpoint{4.802802in}{2.977380in}}%
\pgfpathlineto{\pgfqpoint{4.816200in}{2.975756in}}%
\pgfpathlineto{\pgfqpoint{4.829606in}{2.974237in}}%
\pgfpathlineto{\pgfqpoint{4.843021in}{2.972824in}}%
\pgfpathlineto{\pgfqpoint{4.835642in}{2.964145in}}%
\pgfpathlineto{\pgfqpoint{4.828258in}{2.955505in}}%
\pgfpathlineto{\pgfqpoint{4.820871in}{2.946902in}}%
\pgfpathlineto{\pgfqpoint{4.813479in}{2.938334in}}%
\pgfpathlineto{\pgfqpoint{4.800051in}{2.939547in}}%
\pgfpathlineto{\pgfqpoint{4.786631in}{2.940867in}}%
\pgfpathlineto{\pgfqpoint{4.773220in}{2.942294in}}%
\pgfpathlineto{\pgfqpoint{4.759818in}{2.943827in}}%
\pgfpathlineto{\pgfqpoint{4.767223in}{2.952588in}}%
\pgfpathlineto{\pgfqpoint{4.774624in}{2.961387in}}%
\pgfpathlineto{\pgfqpoint{4.782020in}{2.970228in}}%
\pgfpathlineto{\pgfqpoint{4.789412in}{2.979112in}}%
\pgfpathclose%
\pgfusepath{fill}%
\end{pgfscope}%
\begin{pgfscope}%
\pgfpathrectangle{\pgfqpoint{1.150000in}{0.150000in}}{\pgfqpoint{5.700000in}{5.700000in}}%
\pgfusepath{clip}%
\pgfsetbuttcap%
\pgfsetroundjoin%
\definecolor{currentfill}{rgb}{0.280868,0.160771,0.472899}%
\pgfsetfillcolor{currentfill}%
\pgfsetfillopacity{0.700000}%
\pgfsetlinewidth{0.000000pt}%
\definecolor{currentstroke}{rgb}{0.000000,0.000000,0.000000}%
\pgfsetstrokecolor{currentstroke}%
\pgfsetdash{}{0pt}%
\pgfpathmoveto{\pgfqpoint{3.450117in}{2.994510in}}%
\pgfpathlineto{\pgfqpoint{3.463270in}{2.984207in}}%
\pgfpathlineto{\pgfqpoint{3.476424in}{2.974065in}}%
\pgfpathlineto{\pgfqpoint{3.489578in}{2.964083in}}%
\pgfpathlineto{\pgfqpoint{3.502733in}{2.954259in}}%
\pgfpathlineto{\pgfqpoint{3.494911in}{2.945452in}}%
\pgfpathlineto{\pgfqpoint{3.487082in}{2.936705in}}%
\pgfpathlineto{\pgfqpoint{3.479248in}{2.928017in}}%
\pgfpathlineto{\pgfqpoint{3.471408in}{2.919389in}}%
\pgfpathlineto{\pgfqpoint{3.458238in}{2.929213in}}%
\pgfpathlineto{\pgfqpoint{3.445070in}{2.939195in}}%
\pgfpathlineto{\pgfqpoint{3.431902in}{2.949338in}}%
\pgfpathlineto{\pgfqpoint{3.418734in}{2.959641in}}%
\pgfpathlineto{\pgfqpoint{3.426589in}{2.968262in}}%
\pgfpathlineto{\pgfqpoint{3.434438in}{2.976947in}}%
\pgfpathlineto{\pgfqpoint{3.442280in}{2.985696in}}%
\pgfpathlineto{\pgfqpoint{3.450117in}{2.994510in}}%
\pgfpathclose%
\pgfusepath{fill}%
\end{pgfscope}%
\begin{pgfscope}%
\pgfpathrectangle{\pgfqpoint{1.150000in}{0.150000in}}{\pgfqpoint{5.700000in}{5.700000in}}%
\pgfusepath{clip}%
\pgfsetbuttcap%
\pgfsetroundjoin%
\definecolor{currentfill}{rgb}{0.255645,0.260703,0.528312}%
\pgfsetfillcolor{currentfill}%
\pgfsetfillopacity{0.700000}%
\pgfsetlinewidth{0.000000pt}%
\definecolor{currentstroke}{rgb}{0.000000,0.000000,0.000000}%
\pgfsetstrokecolor{currentstroke}%
\pgfsetdash{}{0pt}%
\pgfpathmoveto{\pgfqpoint{3.155275in}{3.201665in}}%
\pgfpathlineto{\pgfqpoint{3.168463in}{3.187841in}}%
\pgfpathlineto{\pgfqpoint{3.181648in}{3.174208in}}%
\pgfpathlineto{\pgfqpoint{3.194831in}{3.160764in}}%
\pgfpathlineto{\pgfqpoint{3.208011in}{3.147508in}}%
\pgfpathlineto{\pgfqpoint{3.200088in}{3.139001in}}%
\pgfpathlineto{\pgfqpoint{3.192158in}{3.130575in}}%
\pgfpathlineto{\pgfqpoint{3.184222in}{3.122230in}}%
\pgfpathlineto{\pgfqpoint{3.176278in}{3.113966in}}%
\pgfpathlineto{\pgfqpoint{3.163081in}{3.127243in}}%
\pgfpathlineto{\pgfqpoint{3.149881in}{3.140707in}}%
\pgfpathlineto{\pgfqpoint{3.136679in}{3.154361in}}%
\pgfpathlineto{\pgfqpoint{3.123475in}{3.168206in}}%
\pgfpathlineto{\pgfqpoint{3.131435in}{3.176442in}}%
\pgfpathlineto{\pgfqpoint{3.139389in}{3.184764in}}%
\pgfpathlineto{\pgfqpoint{3.147335in}{3.193172in}}%
\pgfpathlineto{\pgfqpoint{3.155275in}{3.201665in}}%
\pgfpathclose%
\pgfusepath{fill}%
\end{pgfscope}%
\begin{pgfscope}%
\pgfpathrectangle{\pgfqpoint{1.150000in}{0.150000in}}{\pgfqpoint{5.700000in}{5.700000in}}%
\pgfusepath{clip}%
\pgfsetbuttcap%
\pgfsetroundjoin%
\definecolor{currentfill}{rgb}{0.282327,0.094955,0.417331}%
\pgfsetfillcolor{currentfill}%
\pgfsetfillopacity{0.700000}%
\pgfsetlinewidth{0.000000pt}%
\definecolor{currentstroke}{rgb}{0.000000,0.000000,0.000000}%
\pgfsetstrokecolor{currentstroke}%
\pgfsetdash{}{0pt}%
\pgfpathmoveto{\pgfqpoint{4.047712in}{2.860932in}}%
\pgfpathlineto{\pgfqpoint{4.060917in}{2.855578in}}%
\pgfpathlineto{\pgfqpoint{4.074127in}{2.850351in}}%
\pgfpathlineto{\pgfqpoint{4.087343in}{2.845249in}}%
\pgfpathlineto{\pgfqpoint{4.100563in}{2.840273in}}%
\pgfpathlineto{\pgfqpoint{4.092936in}{2.831121in}}%
\pgfpathlineto{\pgfqpoint{4.085305in}{2.822001in}}%
\pgfpathlineto{\pgfqpoint{4.077669in}{2.812912in}}%
\pgfpathlineto{\pgfqpoint{4.070027in}{2.803855in}}%
\pgfpathlineto{\pgfqpoint{4.056796in}{2.808758in}}%
\pgfpathlineto{\pgfqpoint{4.043569in}{2.813787in}}%
\pgfpathlineto{\pgfqpoint{4.030348in}{2.818941in}}%
\pgfpathlineto{\pgfqpoint{4.017131in}{2.824223in}}%
\pgfpathlineto{\pgfqpoint{4.024784in}{2.833346in}}%
\pgfpathlineto{\pgfqpoint{4.032431in}{2.842505in}}%
\pgfpathlineto{\pgfqpoint{4.040074in}{2.851700in}}%
\pgfpathlineto{\pgfqpoint{4.047712in}{2.860932in}}%
\pgfpathclose%
\pgfusepath{fill}%
\end{pgfscope}%
\begin{pgfscope}%
\pgfpathrectangle{\pgfqpoint{1.150000in}{0.150000in}}{\pgfqpoint{5.700000in}{5.700000in}}%
\pgfusepath{clip}%
\pgfsetbuttcap%
\pgfsetroundjoin%
\definecolor{currentfill}{rgb}{0.283187,0.125848,0.444960}%
\pgfsetfillcolor{currentfill}%
\pgfsetfillopacity{0.700000}%
\pgfsetlinewidth{0.000000pt}%
\definecolor{currentstroke}{rgb}{0.000000,0.000000,0.000000}%
\pgfsetstrokecolor{currentstroke}%
\pgfsetdash{}{0pt}%
\pgfpathmoveto{\pgfqpoint{3.639147in}{2.916928in}}%
\pgfpathlineto{\pgfqpoint{3.652302in}{2.908460in}}%
\pgfpathlineto{\pgfqpoint{3.665460in}{2.900138in}}%
\pgfpathlineto{\pgfqpoint{3.678621in}{2.891963in}}%
\pgfpathlineto{\pgfqpoint{3.691784in}{2.883932in}}%
\pgfpathlineto{\pgfqpoint{3.684024in}{2.874954in}}%
\pgfpathlineto{\pgfqpoint{3.676258in}{2.866024in}}%
\pgfpathlineto{\pgfqpoint{3.668488in}{2.857142in}}%
\pgfpathlineto{\pgfqpoint{3.660711in}{2.848309in}}%
\pgfpathlineto{\pgfqpoint{3.647535in}{2.856321in}}%
\pgfpathlineto{\pgfqpoint{3.634362in}{2.864478in}}%
\pgfpathlineto{\pgfqpoint{3.621191in}{2.872781in}}%
\pgfpathlineto{\pgfqpoint{3.608022in}{2.881231in}}%
\pgfpathlineto{\pgfqpoint{3.615812in}{2.890076in}}%
\pgfpathlineto{\pgfqpoint{3.623596in}{2.898974in}}%
\pgfpathlineto{\pgfqpoint{3.631374in}{2.907925in}}%
\pgfpathlineto{\pgfqpoint{3.639147in}{2.916928in}}%
\pgfpathclose%
\pgfusepath{fill}%
\end{pgfscope}%
\begin{pgfscope}%
\pgfpathrectangle{\pgfqpoint{1.150000in}{0.150000in}}{\pgfqpoint{5.700000in}{5.700000in}}%
\pgfusepath{clip}%
\pgfsetbuttcap%
\pgfsetroundjoin%
\definecolor{currentfill}{rgb}{0.263663,0.237631,0.518762}%
\pgfsetfillcolor{currentfill}%
\pgfsetfillopacity{0.700000}%
\pgfsetlinewidth{0.000000pt}%
\definecolor{currentstroke}{rgb}{0.000000,0.000000,0.000000}%
\pgfsetstrokecolor{currentstroke}%
\pgfsetdash{}{0pt}%
\pgfpathmoveto{\pgfqpoint{3.208011in}{3.147508in}}%
\pgfpathlineto{\pgfqpoint{3.221190in}{3.134438in}}%
\pgfpathlineto{\pgfqpoint{3.234367in}{3.121553in}}%
\pgfpathlineto{\pgfqpoint{3.247542in}{3.108850in}}%
\pgfpathlineto{\pgfqpoint{3.260715in}{3.096329in}}%
\pgfpathlineto{\pgfqpoint{3.252808in}{3.087808in}}%
\pgfpathlineto{\pgfqpoint{3.244895in}{3.079365in}}%
\pgfpathlineto{\pgfqpoint{3.236975in}{3.070998in}}%
\pgfpathlineto{\pgfqpoint{3.229048in}{3.062707in}}%
\pgfpathlineto{\pgfqpoint{3.215858in}{3.075248in}}%
\pgfpathlineto{\pgfqpoint{3.202667in}{3.087971in}}%
\pgfpathlineto{\pgfqpoint{3.189473in}{3.100876in}}%
\pgfpathlineto{\pgfqpoint{3.176278in}{3.113966in}}%
\pgfpathlineto{\pgfqpoint{3.184222in}{3.122230in}}%
\pgfpathlineto{\pgfqpoint{3.192158in}{3.130575in}}%
\pgfpathlineto{\pgfqpoint{3.200088in}{3.139001in}}%
\pgfpathlineto{\pgfqpoint{3.208011in}{3.147508in}}%
\pgfpathclose%
\pgfusepath{fill}%
\end{pgfscope}%
\begin{pgfscope}%
\pgfpathrectangle{\pgfqpoint{1.150000in}{0.150000in}}{\pgfqpoint{5.700000in}{5.700000in}}%
\pgfusepath{clip}%
\pgfsetbuttcap%
\pgfsetroundjoin%
\definecolor{currentfill}{rgb}{0.281887,0.150881,0.465405}%
\pgfsetfillcolor{currentfill}%
\pgfsetfillopacity{0.700000}%
\pgfsetlinewidth{0.000000pt}%
\definecolor{currentstroke}{rgb}{0.000000,0.000000,0.000000}%
\pgfsetstrokecolor{currentstroke}%
\pgfsetdash{}{0pt}%
\pgfpathmoveto{\pgfqpoint{4.706290in}{2.951036in}}%
\pgfpathlineto{\pgfqpoint{4.719660in}{2.949072in}}%
\pgfpathlineto{\pgfqpoint{4.733038in}{2.947216in}}%
\pgfpathlineto{\pgfqpoint{4.746424in}{2.945468in}}%
\pgfpathlineto{\pgfqpoint{4.759818in}{2.943827in}}%
\pgfpathlineto{\pgfqpoint{4.752409in}{2.935104in}}%
\pgfpathlineto{\pgfqpoint{4.744995in}{2.926414in}}%
\pgfpathlineto{\pgfqpoint{4.737576in}{2.917757in}}%
\pgfpathlineto{\pgfqpoint{4.730154in}{2.909129in}}%
\pgfpathlineto{\pgfqpoint{4.716747in}{2.910589in}}%
\pgfpathlineto{\pgfqpoint{4.703348in}{2.912156in}}%
\pgfpathlineto{\pgfqpoint{4.689958in}{2.913832in}}%
\pgfpathlineto{\pgfqpoint{4.676575in}{2.915615in}}%
\pgfpathlineto{\pgfqpoint{4.684010in}{2.924417in}}%
\pgfpathlineto{\pgfqpoint{4.691441in}{2.933253in}}%
\pgfpathlineto{\pgfqpoint{4.698868in}{2.942125in}}%
\pgfpathlineto{\pgfqpoint{4.706290in}{2.951036in}}%
\pgfpathclose%
\pgfusepath{fill}%
\end{pgfscope}%
\begin{pgfscope}%
\pgfpathrectangle{\pgfqpoint{1.150000in}{0.150000in}}{\pgfqpoint{5.700000in}{5.700000in}}%
\pgfusepath{clip}%
\pgfsetbuttcap%
\pgfsetroundjoin%
\definecolor{currentfill}{rgb}{0.282656,0.100196,0.422160}%
\pgfsetfillcolor{currentfill}%
\pgfsetfillopacity{0.700000}%
\pgfsetlinewidth{0.000000pt}%
\definecolor{currentstroke}{rgb}{0.000000,0.000000,0.000000}%
\pgfsetstrokecolor{currentstroke}%
\pgfsetdash{}{0pt}%
\pgfpathmoveto{\pgfqpoint{4.183909in}{2.858209in}}%
\pgfpathlineto{\pgfqpoint{4.197145in}{2.853759in}}%
\pgfpathlineto{\pgfqpoint{4.210386in}{2.849431in}}%
\pgfpathlineto{\pgfqpoint{4.223634in}{2.845224in}}%
\pgfpathlineto{\pgfqpoint{4.236887in}{2.841136in}}%
\pgfpathlineto{\pgfqpoint{4.229303in}{2.832035in}}%
\pgfpathlineto{\pgfqpoint{4.221714in}{2.822963in}}%
\pgfpathlineto{\pgfqpoint{4.214120in}{2.813918in}}%
\pgfpathlineto{\pgfqpoint{4.206521in}{2.804900in}}%
\pgfpathlineto{\pgfqpoint{4.193256in}{2.808896in}}%
\pgfpathlineto{\pgfqpoint{4.179997in}{2.813012in}}%
\pgfpathlineto{\pgfqpoint{4.166744in}{2.817250in}}%
\pgfpathlineto{\pgfqpoint{4.153497in}{2.821609in}}%
\pgfpathlineto{\pgfqpoint{4.161107in}{2.830712in}}%
\pgfpathlineto{\pgfqpoint{4.168713in}{2.839845in}}%
\pgfpathlineto{\pgfqpoint{4.176313in}{2.849011in}}%
\pgfpathlineto{\pgfqpoint{4.183909in}{2.858209in}}%
\pgfpathclose%
\pgfusepath{fill}%
\end{pgfscope}%
\begin{pgfscope}%
\pgfpathrectangle{\pgfqpoint{1.150000in}{0.150000in}}{\pgfqpoint{5.700000in}{5.700000in}}%
\pgfusepath{clip}%
\pgfsetbuttcap%
\pgfsetroundjoin%
\definecolor{currentfill}{rgb}{0.283197,0.115680,0.436115}%
\pgfsetfillcolor{currentfill}%
\pgfsetfillopacity{0.700000}%
\pgfsetlinewidth{0.000000pt}%
\definecolor{currentstroke}{rgb}{0.000000,0.000000,0.000000}%
\pgfsetstrokecolor{currentstroke}%
\pgfsetdash{}{0pt}%
\pgfpathmoveto{\pgfqpoint{4.403460in}{2.884852in}}%
\pgfpathlineto{\pgfqpoint{4.416748in}{2.881589in}}%
\pgfpathlineto{\pgfqpoint{4.430043in}{2.878440in}}%
\pgfpathlineto{\pgfqpoint{4.443344in}{2.875405in}}%
\pgfpathlineto{\pgfqpoint{4.456653in}{2.872485in}}%
\pgfpathlineto{\pgfqpoint{4.449140in}{2.863498in}}%
\pgfpathlineto{\pgfqpoint{4.441623in}{2.854540in}}%
\pgfpathlineto{\pgfqpoint{4.434102in}{2.845607in}}%
\pgfpathlineto{\pgfqpoint{4.426575in}{2.836700in}}%
\pgfpathlineto{\pgfqpoint{4.413255in}{2.839493in}}%
\pgfpathlineto{\pgfqpoint{4.399942in}{2.842400in}}%
\pgfpathlineto{\pgfqpoint{4.386636in}{2.845422in}}%
\pgfpathlineto{\pgfqpoint{4.373337in}{2.848559in}}%
\pgfpathlineto{\pgfqpoint{4.380875in}{2.857587in}}%
\pgfpathlineto{\pgfqpoint{4.388408in}{2.866645in}}%
\pgfpathlineto{\pgfqpoint{4.395936in}{2.875732in}}%
\pgfpathlineto{\pgfqpoint{4.403460in}{2.884852in}}%
\pgfpathclose%
\pgfusepath{fill}%
\end{pgfscope}%
\begin{pgfscope}%
\pgfpathrectangle{\pgfqpoint{1.150000in}{0.150000in}}{\pgfqpoint{5.700000in}{5.700000in}}%
\pgfusepath{clip}%
\pgfsetbuttcap%
\pgfsetroundjoin%
\definecolor{currentfill}{rgb}{0.270595,0.214069,0.507052}%
\pgfsetfillcolor{currentfill}%
\pgfsetfillopacity{0.700000}%
\pgfsetlinewidth{0.000000pt}%
\definecolor{currentstroke}{rgb}{0.000000,0.000000,0.000000}%
\pgfsetstrokecolor{currentstroke}%
\pgfsetdash{}{0pt}%
\pgfpathmoveto{\pgfqpoint{3.260715in}{3.096329in}}%
\pgfpathlineto{\pgfqpoint{3.273887in}{3.083987in}}%
\pgfpathlineto{\pgfqpoint{3.287058in}{3.071822in}}%
\pgfpathlineto{\pgfqpoint{3.300228in}{3.059835in}}%
\pgfpathlineto{\pgfqpoint{3.313397in}{3.048022in}}%
\pgfpathlineto{\pgfqpoint{3.305506in}{3.039489in}}%
\pgfpathlineto{\pgfqpoint{3.297608in}{3.031028in}}%
\pgfpathlineto{\pgfqpoint{3.289704in}{3.022639in}}%
\pgfpathlineto{\pgfqpoint{3.281794in}{3.014323in}}%
\pgfpathlineto{\pgfqpoint{3.268609in}{3.026155in}}%
\pgfpathlineto{\pgfqpoint{3.255423in}{3.038162in}}%
\pgfpathlineto{\pgfqpoint{3.242236in}{3.050345in}}%
\pgfpathlineto{\pgfqpoint{3.229048in}{3.062707in}}%
\pgfpathlineto{\pgfqpoint{3.236975in}{3.070998in}}%
\pgfpathlineto{\pgfqpoint{3.244895in}{3.079365in}}%
\pgfpathlineto{\pgfqpoint{3.252808in}{3.087808in}}%
\pgfpathlineto{\pgfqpoint{3.260715in}{3.096329in}}%
\pgfpathclose%
\pgfusepath{fill}%
\end{pgfscope}%
\begin{pgfscope}%
\pgfpathrectangle{\pgfqpoint{1.150000in}{0.150000in}}{\pgfqpoint{5.700000in}{5.700000in}}%
\pgfusepath{clip}%
\pgfsetbuttcap%
\pgfsetroundjoin%
\definecolor{currentfill}{rgb}{0.282290,0.145912,0.461510}%
\pgfsetfillcolor{currentfill}%
\pgfsetfillopacity{0.700000}%
\pgfsetlinewidth{0.000000pt}%
\definecolor{currentstroke}{rgb}{0.000000,0.000000,0.000000}%
\pgfsetstrokecolor{currentstroke}%
\pgfsetdash{}{0pt}%
\pgfpathmoveto{\pgfqpoint{3.502733in}{2.954259in}}%
\pgfpathlineto{\pgfqpoint{3.515889in}{2.944593in}}%
\pgfpathlineto{\pgfqpoint{3.529047in}{2.935082in}}%
\pgfpathlineto{\pgfqpoint{3.542205in}{2.925727in}}%
\pgfpathlineto{\pgfqpoint{3.555366in}{2.916525in}}%
\pgfpathlineto{\pgfqpoint{3.547557in}{2.907724in}}%
\pgfpathlineto{\pgfqpoint{3.539742in}{2.898979in}}%
\pgfpathlineto{\pgfqpoint{3.531922in}{2.890289in}}%
\pgfpathlineto{\pgfqpoint{3.524096in}{2.881655in}}%
\pgfpathlineto{\pgfqpoint{3.510922in}{2.890856in}}%
\pgfpathlineto{\pgfqpoint{3.497749in}{2.900212in}}%
\pgfpathlineto{\pgfqpoint{3.484578in}{2.909722in}}%
\pgfpathlineto{\pgfqpoint{3.471408in}{2.919389in}}%
\pgfpathlineto{\pgfqpoint{3.479248in}{2.928017in}}%
\pgfpathlineto{\pgfqpoint{3.487082in}{2.936705in}}%
\pgfpathlineto{\pgfqpoint{3.494911in}{2.945452in}}%
\pgfpathlineto{\pgfqpoint{3.502733in}{2.954259in}}%
\pgfpathclose%
\pgfusepath{fill}%
\end{pgfscope}%
\begin{pgfscope}%
\pgfpathrectangle{\pgfqpoint{1.150000in}{0.150000in}}{\pgfqpoint{5.700000in}{5.700000in}}%
\pgfusepath{clip}%
\pgfsetbuttcap%
\pgfsetroundjoin%
\definecolor{currentfill}{rgb}{0.282656,0.100196,0.422160}%
\pgfsetfillcolor{currentfill}%
\pgfsetfillopacity{0.700000}%
\pgfsetlinewidth{0.000000pt}%
\definecolor{currentstroke}{rgb}{0.000000,0.000000,0.000000}%
\pgfsetstrokecolor{currentstroke}%
\pgfsetdash{}{0pt}%
\pgfpathmoveto{\pgfqpoint{3.828074in}{2.860920in}}%
\pgfpathlineto{\pgfqpoint{3.841252in}{2.854116in}}%
\pgfpathlineto{\pgfqpoint{3.854433in}{2.847448in}}%
\pgfpathlineto{\pgfqpoint{3.867618in}{2.840914in}}%
\pgfpathlineto{\pgfqpoint{3.880807in}{2.834515in}}%
\pgfpathlineto{\pgfqpoint{3.873105in}{2.825464in}}%
\pgfpathlineto{\pgfqpoint{3.865399in}{2.816451in}}%
\pgfpathlineto{\pgfqpoint{3.857686in}{2.807476in}}%
\pgfpathlineto{\pgfqpoint{3.849969in}{2.798538in}}%
\pgfpathlineto{\pgfqpoint{3.836768in}{2.804901in}}%
\pgfpathlineto{\pgfqpoint{3.823571in}{2.811397in}}%
\pgfpathlineto{\pgfqpoint{3.810378in}{2.818029in}}%
\pgfpathlineto{\pgfqpoint{3.797188in}{2.824796in}}%
\pgfpathlineto{\pgfqpoint{3.804917in}{2.833764in}}%
\pgfpathlineto{\pgfqpoint{3.812641in}{2.842774in}}%
\pgfpathlineto{\pgfqpoint{3.820360in}{2.851826in}}%
\pgfpathlineto{\pgfqpoint{3.828074in}{2.860920in}}%
\pgfpathclose%
\pgfusepath{fill}%
\end{pgfscope}%
\begin{pgfscope}%
\pgfpathrectangle{\pgfqpoint{1.150000in}{0.150000in}}{\pgfqpoint{5.700000in}{5.700000in}}%
\pgfusepath{clip}%
\pgfsetbuttcap%
\pgfsetroundjoin%
\definecolor{currentfill}{rgb}{0.282623,0.140926,0.457517}%
\pgfsetfillcolor{currentfill}%
\pgfsetfillopacity{0.700000}%
\pgfsetlinewidth{0.000000pt}%
\definecolor{currentstroke}{rgb}{0.000000,0.000000,0.000000}%
\pgfsetstrokecolor{currentstroke}%
\pgfsetdash{}{0pt}%
\pgfpathmoveto{\pgfqpoint{4.623125in}{2.923840in}}%
\pgfpathlineto{\pgfqpoint{4.636476in}{2.921620in}}%
\pgfpathlineto{\pgfqpoint{4.649834in}{2.919509in}}%
\pgfpathlineto{\pgfqpoint{4.663201in}{2.917508in}}%
\pgfpathlineto{\pgfqpoint{4.676575in}{2.915615in}}%
\pgfpathlineto{\pgfqpoint{4.669135in}{2.906846in}}%
\pgfpathlineto{\pgfqpoint{4.661691in}{2.898106in}}%
\pgfpathlineto{\pgfqpoint{4.654242in}{2.889395in}}%
\pgfpathlineto{\pgfqpoint{4.646789in}{2.880709in}}%
\pgfpathlineto{\pgfqpoint{4.633402in}{2.882438in}}%
\pgfpathlineto{\pgfqpoint{4.620023in}{2.884277in}}%
\pgfpathlineto{\pgfqpoint{4.606652in}{2.886225in}}%
\pgfpathlineto{\pgfqpoint{4.593289in}{2.888282in}}%
\pgfpathlineto{\pgfqpoint{4.600755in}{2.897124in}}%
\pgfpathlineto{\pgfqpoint{4.608216in}{2.905997in}}%
\pgfpathlineto{\pgfqpoint{4.615673in}{2.914901in}}%
\pgfpathlineto{\pgfqpoint{4.623125in}{2.923840in}}%
\pgfpathclose%
\pgfusepath{fill}%
\end{pgfscope}%
\begin{pgfscope}%
\pgfpathrectangle{\pgfqpoint{1.150000in}{0.150000in}}{\pgfqpoint{5.700000in}{5.700000in}}%
\pgfusepath{clip}%
\pgfsetbuttcap%
\pgfsetroundjoin%
\definecolor{currentfill}{rgb}{0.282327,0.094955,0.417331}%
\pgfsetfillcolor{currentfill}%
\pgfsetfillopacity{0.700000}%
\pgfsetlinewidth{0.000000pt}%
\definecolor{currentstroke}{rgb}{0.000000,0.000000,0.000000}%
\pgfsetstrokecolor{currentstroke}%
\pgfsetdash{}{0pt}%
\pgfpathmoveto{\pgfqpoint{3.964311in}{2.846629in}}%
\pgfpathlineto{\pgfqpoint{3.977509in}{2.840834in}}%
\pgfpathlineto{\pgfqpoint{3.990712in}{2.835168in}}%
\pgfpathlineto{\pgfqpoint{4.003919in}{2.829631in}}%
\pgfpathlineto{\pgfqpoint{4.017131in}{2.824223in}}%
\pgfpathlineto{\pgfqpoint{4.009473in}{2.815133in}}%
\pgfpathlineto{\pgfqpoint{4.001810in}{2.806076in}}%
\pgfpathlineto{\pgfqpoint{3.994142in}{2.797051in}}%
\pgfpathlineto{\pgfqpoint{3.986469in}{2.788058in}}%
\pgfpathlineto{\pgfqpoint{3.973246in}{2.793411in}}%
\pgfpathlineto{\pgfqpoint{3.960027in}{2.798893in}}%
\pgfpathlineto{\pgfqpoint{3.946813in}{2.804504in}}%
\pgfpathlineto{\pgfqpoint{3.933603in}{2.810244in}}%
\pgfpathlineto{\pgfqpoint{3.941288in}{2.819286in}}%
\pgfpathlineto{\pgfqpoint{3.948967in}{2.828363in}}%
\pgfpathlineto{\pgfqpoint{3.956642in}{2.837477in}}%
\pgfpathlineto{\pgfqpoint{3.964311in}{2.846629in}}%
\pgfpathclose%
\pgfusepath{fill}%
\end{pgfscope}%
\begin{pgfscope}%
\pgfpathrectangle{\pgfqpoint{1.150000in}{0.150000in}}{\pgfqpoint{5.700000in}{5.700000in}}%
\pgfusepath{clip}%
\pgfsetbuttcap%
\pgfsetroundjoin%
\definecolor{currentfill}{rgb}{0.283091,0.110553,0.431554}%
\pgfsetfillcolor{currentfill}%
\pgfsetfillopacity{0.700000}%
\pgfsetlinewidth{0.000000pt}%
\definecolor{currentstroke}{rgb}{0.000000,0.000000,0.000000}%
\pgfsetstrokecolor{currentstroke}%
\pgfsetdash{}{0pt}%
\pgfpathmoveto{\pgfqpoint{3.691784in}{2.883932in}}%
\pgfpathlineto{\pgfqpoint{3.704949in}{2.876046in}}%
\pgfpathlineto{\pgfqpoint{3.718117in}{2.868303in}}%
\pgfpathlineto{\pgfqpoint{3.731288in}{2.860702in}}%
\pgfpathlineto{\pgfqpoint{3.744461in}{2.853242in}}%
\pgfpathlineto{\pgfqpoint{3.736714in}{2.844289in}}%
\pgfpathlineto{\pgfqpoint{3.728962in}{2.835380in}}%
\pgfpathlineto{\pgfqpoint{3.721203in}{2.826515in}}%
\pgfpathlineto{\pgfqpoint{3.713440in}{2.817693in}}%
\pgfpathlineto{\pgfqpoint{3.700254in}{2.825134in}}%
\pgfpathlineto{\pgfqpoint{3.687070in}{2.832716in}}%
\pgfpathlineto{\pgfqpoint{3.673889in}{2.840441in}}%
\pgfpathlineto{\pgfqpoint{3.660711in}{2.848309in}}%
\pgfpathlineto{\pgfqpoint{3.668488in}{2.857142in}}%
\pgfpathlineto{\pgfqpoint{3.676258in}{2.866024in}}%
\pgfpathlineto{\pgfqpoint{3.684024in}{2.874954in}}%
\pgfpathlineto{\pgfqpoint{3.691784in}{2.883932in}}%
\pgfpathclose%
\pgfusepath{fill}%
\end{pgfscope}%
\begin{pgfscope}%
\pgfpathrectangle{\pgfqpoint{1.150000in}{0.150000in}}{\pgfqpoint{5.700000in}{5.700000in}}%
\pgfusepath{clip}%
\pgfsetbuttcap%
\pgfsetroundjoin%
\definecolor{currentfill}{rgb}{0.276194,0.190074,0.493001}%
\pgfsetfillcolor{currentfill}%
\pgfsetfillopacity{0.700000}%
\pgfsetlinewidth{0.000000pt}%
\definecolor{currentstroke}{rgb}{0.000000,0.000000,0.000000}%
\pgfsetstrokecolor{currentstroke}%
\pgfsetdash{}{0pt}%
\pgfpathmoveto{\pgfqpoint{3.313397in}{3.048022in}}%
\pgfpathlineto{\pgfqpoint{3.326565in}{3.036383in}}%
\pgfpathlineto{\pgfqpoint{3.339733in}{3.024915in}}%
\pgfpathlineto{\pgfqpoint{3.352900in}{3.013618in}}%
\pgfpathlineto{\pgfqpoint{3.366067in}{3.002490in}}%
\pgfpathlineto{\pgfqpoint{3.358191in}{2.993945in}}%
\pgfpathlineto{\pgfqpoint{3.350309in}{2.985467in}}%
\pgfpathlineto{\pgfqpoint{3.342421in}{2.977057in}}%
\pgfpathlineto{\pgfqpoint{3.334526in}{2.968715in}}%
\pgfpathlineto{\pgfqpoint{3.321343in}{2.979862in}}%
\pgfpathlineto{\pgfqpoint{3.308161in}{2.991178in}}%
\pgfpathlineto{\pgfqpoint{3.294978in}{3.002664in}}%
\pgfpathlineto{\pgfqpoint{3.281794in}{3.014323in}}%
\pgfpathlineto{\pgfqpoint{3.289704in}{3.022639in}}%
\pgfpathlineto{\pgfqpoint{3.297608in}{3.031028in}}%
\pgfpathlineto{\pgfqpoint{3.305506in}{3.039489in}}%
\pgfpathlineto{\pgfqpoint{3.313397in}{3.048022in}}%
\pgfpathclose%
\pgfusepath{fill}%
\end{pgfscope}%
\begin{pgfscope}%
\pgfpathrectangle{\pgfqpoint{1.150000in}{0.150000in}}{\pgfqpoint{5.700000in}{5.700000in}}%
\pgfusepath{clip}%
\pgfsetbuttcap%
\pgfsetroundjoin%
\definecolor{currentfill}{rgb}{0.282910,0.105393,0.426902}%
\pgfsetfillcolor{currentfill}%
\pgfsetfillopacity{0.700000}%
\pgfsetlinewidth{0.000000pt}%
\definecolor{currentstroke}{rgb}{0.000000,0.000000,0.000000}%
\pgfsetstrokecolor{currentstroke}%
\pgfsetdash{}{0pt}%
\pgfpathmoveto{\pgfqpoint{4.320206in}{2.862269in}}%
\pgfpathlineto{\pgfqpoint{4.333479in}{2.858666in}}%
\pgfpathlineto{\pgfqpoint{4.346759in}{2.855181in}}%
\pgfpathlineto{\pgfqpoint{4.360045in}{2.851812in}}%
\pgfpathlineto{\pgfqpoint{4.373337in}{2.848559in}}%
\pgfpathlineto{\pgfqpoint{4.365794in}{2.839559in}}%
\pgfpathlineto{\pgfqpoint{4.358247in}{2.830584in}}%
\pgfpathlineto{\pgfqpoint{4.350695in}{2.821634in}}%
\pgfpathlineto{\pgfqpoint{4.343138in}{2.812707in}}%
\pgfpathlineto{\pgfqpoint{4.329834in}{2.815851in}}%
\pgfpathlineto{\pgfqpoint{4.316537in}{2.819110in}}%
\pgfpathlineto{\pgfqpoint{4.303246in}{2.822487in}}%
\pgfpathlineto{\pgfqpoint{4.289962in}{2.825980in}}%
\pgfpathlineto{\pgfqpoint{4.297530in}{2.835010in}}%
\pgfpathlineto{\pgfqpoint{4.305094in}{2.844067in}}%
\pgfpathlineto{\pgfqpoint{4.312652in}{2.853153in}}%
\pgfpathlineto{\pgfqpoint{4.320206in}{2.862269in}}%
\pgfpathclose%
\pgfusepath{fill}%
\end{pgfscope}%
\begin{pgfscope}%
\pgfpathrectangle{\pgfqpoint{1.150000in}{0.150000in}}{\pgfqpoint{5.700000in}{5.700000in}}%
\pgfusepath{clip}%
\pgfsetbuttcap%
\pgfsetroundjoin%
\definecolor{currentfill}{rgb}{0.282327,0.094955,0.417331}%
\pgfsetfillcolor{currentfill}%
\pgfsetfillopacity{0.700000}%
\pgfsetlinewidth{0.000000pt}%
\definecolor{currentstroke}{rgb}{0.000000,0.000000,0.000000}%
\pgfsetstrokecolor{currentstroke}%
\pgfsetdash{}{0pt}%
\pgfpathmoveto{\pgfqpoint{4.100563in}{2.840273in}}%
\pgfpathlineto{\pgfqpoint{4.113788in}{2.835422in}}%
\pgfpathlineto{\pgfqpoint{4.127019in}{2.830694in}}%
\pgfpathlineto{\pgfqpoint{4.140255in}{2.826090in}}%
\pgfpathlineto{\pgfqpoint{4.153497in}{2.821609in}}%
\pgfpathlineto{\pgfqpoint{4.145882in}{2.812536in}}%
\pgfpathlineto{\pgfqpoint{4.138262in}{2.803491in}}%
\pgfpathlineto{\pgfqpoint{4.130637in}{2.794474in}}%
\pgfpathlineto{\pgfqpoint{4.123007in}{2.785483in}}%
\pgfpathlineto{\pgfqpoint{4.109754in}{2.789891in}}%
\pgfpathlineto{\pgfqpoint{4.096506in}{2.794422in}}%
\pgfpathlineto{\pgfqpoint{4.083264in}{2.799076in}}%
\pgfpathlineto{\pgfqpoint{4.070027in}{2.803855in}}%
\pgfpathlineto{\pgfqpoint{4.077669in}{2.812912in}}%
\pgfpathlineto{\pgfqpoint{4.085305in}{2.822001in}}%
\pgfpathlineto{\pgfqpoint{4.092936in}{2.831121in}}%
\pgfpathlineto{\pgfqpoint{4.100563in}{2.840273in}}%
\pgfpathclose%
\pgfusepath{fill}%
\end{pgfscope}%
\begin{pgfscope}%
\pgfpathrectangle{\pgfqpoint{1.150000in}{0.150000in}}{\pgfqpoint{5.700000in}{5.700000in}}%
\pgfusepath{clip}%
\pgfsetbuttcap%
\pgfsetroundjoin%
\definecolor{currentfill}{rgb}{0.283072,0.130895,0.449241}%
\pgfsetfillcolor{currentfill}%
\pgfsetfillopacity{0.700000}%
\pgfsetlinewidth{0.000000pt}%
\definecolor{currentstroke}{rgb}{0.000000,0.000000,0.000000}%
\pgfsetstrokecolor{currentstroke}%
\pgfsetdash{}{0pt}%
\pgfpathmoveto{\pgfqpoint{3.555366in}{2.916525in}}%
\pgfpathlineto{\pgfqpoint{3.568527in}{2.907475in}}%
\pgfpathlineto{\pgfqpoint{3.581690in}{2.898577in}}%
\pgfpathlineto{\pgfqpoint{3.594855in}{2.889829in}}%
\pgfpathlineto{\pgfqpoint{3.608022in}{2.881231in}}%
\pgfpathlineto{\pgfqpoint{3.600227in}{2.872437in}}%
\pgfpathlineto{\pgfqpoint{3.592426in}{2.863695in}}%
\pgfpathlineto{\pgfqpoint{3.584619in}{2.855003in}}%
\pgfpathlineto{\pgfqpoint{3.576807in}{2.846363in}}%
\pgfpathlineto{\pgfqpoint{3.563626in}{2.854960in}}%
\pgfpathlineto{\pgfqpoint{3.550448in}{2.863708in}}%
\pgfpathlineto{\pgfqpoint{3.537271in}{2.872605in}}%
\pgfpathlineto{\pgfqpoint{3.524096in}{2.881655in}}%
\pgfpathlineto{\pgfqpoint{3.531922in}{2.890289in}}%
\pgfpathlineto{\pgfqpoint{3.539742in}{2.898979in}}%
\pgfpathlineto{\pgfqpoint{3.547557in}{2.907724in}}%
\pgfpathlineto{\pgfqpoint{3.555366in}{2.916525in}}%
\pgfpathclose%
\pgfusepath{fill}%
\end{pgfscope}%
\begin{pgfscope}%
\pgfpathrectangle{\pgfqpoint{1.150000in}{0.150000in}}{\pgfqpoint{5.700000in}{5.700000in}}%
\pgfusepath{clip}%
\pgfsetbuttcap%
\pgfsetroundjoin%
\definecolor{currentfill}{rgb}{0.283187,0.125848,0.444960}%
\pgfsetfillcolor{currentfill}%
\pgfsetfillopacity{0.700000}%
\pgfsetlinewidth{0.000000pt}%
\definecolor{currentstroke}{rgb}{0.000000,0.000000,0.000000}%
\pgfsetstrokecolor{currentstroke}%
\pgfsetdash{}{0pt}%
\pgfpathmoveto{\pgfqpoint{4.539913in}{2.897620in}}%
\pgfpathlineto{\pgfqpoint{4.553246in}{2.895119in}}%
\pgfpathlineto{\pgfqpoint{4.566586in}{2.892729in}}%
\pgfpathlineto{\pgfqpoint{4.579934in}{2.890450in}}%
\pgfpathlineto{\pgfqpoint{4.593289in}{2.888282in}}%
\pgfpathlineto{\pgfqpoint{4.585819in}{2.879469in}}%
\pgfpathlineto{\pgfqpoint{4.578344in}{2.870681in}}%
\pgfpathlineto{\pgfqpoint{4.570865in}{2.861919in}}%
\pgfpathlineto{\pgfqpoint{4.563380in}{2.853179in}}%
\pgfpathlineto{\pgfqpoint{4.550013in}{2.855202in}}%
\pgfpathlineto{\pgfqpoint{4.536654in}{2.857335in}}%
\pgfpathlineto{\pgfqpoint{4.523302in}{2.859580in}}%
\pgfpathlineto{\pgfqpoint{4.509957in}{2.861936in}}%
\pgfpathlineto{\pgfqpoint{4.517453in}{2.870815in}}%
\pgfpathlineto{\pgfqpoint{4.524945in}{2.879720in}}%
\pgfpathlineto{\pgfqpoint{4.532431in}{2.888655in}}%
\pgfpathlineto{\pgfqpoint{4.539913in}{2.897620in}}%
\pgfpathclose%
\pgfusepath{fill}%
\end{pgfscope}%
\begin{pgfscope}%
\pgfpathrectangle{\pgfqpoint{1.150000in}{0.150000in}}{\pgfqpoint{5.700000in}{5.700000in}}%
\pgfusepath{clip}%
\pgfsetbuttcap%
\pgfsetroundjoin%
\definecolor{currentfill}{rgb}{0.278826,0.175490,0.483397}%
\pgfsetfillcolor{currentfill}%
\pgfsetfillopacity{0.700000}%
\pgfsetlinewidth{0.000000pt}%
\definecolor{currentstroke}{rgb}{0.000000,0.000000,0.000000}%
\pgfsetstrokecolor{currentstroke}%
\pgfsetdash{}{0pt}%
\pgfpathmoveto{\pgfqpoint{3.366067in}{3.002490in}}%
\pgfpathlineto{\pgfqpoint{3.379234in}{2.991530in}}%
\pgfpathlineto{\pgfqpoint{3.392400in}{2.980736in}}%
\pgfpathlineto{\pgfqpoint{3.405567in}{2.970106in}}%
\pgfpathlineto{\pgfqpoint{3.418734in}{2.959641in}}%
\pgfpathlineto{\pgfqpoint{3.410873in}{2.951084in}}%
\pgfpathlineto{\pgfqpoint{3.403006in}{2.942590in}}%
\pgfpathlineto{\pgfqpoint{3.395133in}{2.934159in}}%
\pgfpathlineto{\pgfqpoint{3.387253in}{2.925792in}}%
\pgfpathlineto{\pgfqpoint{3.374071in}{2.936276in}}%
\pgfpathlineto{\pgfqpoint{3.360890in}{2.946923in}}%
\pgfpathlineto{\pgfqpoint{3.347708in}{2.957736in}}%
\pgfpathlineto{\pgfqpoint{3.334526in}{2.968715in}}%
\pgfpathlineto{\pgfqpoint{3.342421in}{2.977057in}}%
\pgfpathlineto{\pgfqpoint{3.350309in}{2.985467in}}%
\pgfpathlineto{\pgfqpoint{3.358191in}{2.993945in}}%
\pgfpathlineto{\pgfqpoint{3.366067in}{3.002490in}}%
\pgfpathclose%
\pgfusepath{fill}%
\end{pgfscope}%
\begin{pgfscope}%
\pgfpathrectangle{\pgfqpoint{1.150000in}{0.150000in}}{\pgfqpoint{5.700000in}{5.700000in}}%
\pgfusepath{clip}%
\pgfsetbuttcap%
\pgfsetroundjoin%
\definecolor{currentfill}{rgb}{0.279574,0.170599,0.479997}%
\pgfsetfillcolor{currentfill}%
\pgfsetfillopacity{0.700000}%
\pgfsetlinewidth{0.000000pt}%
\definecolor{currentstroke}{rgb}{0.000000,0.000000,0.000000}%
\pgfsetstrokecolor{currentstroke}%
\pgfsetdash{}{0pt}%
\pgfpathmoveto{\pgfqpoint{4.843021in}{2.972824in}}%
\pgfpathlineto{\pgfqpoint{4.856444in}{2.971518in}}%
\pgfpathlineto{\pgfqpoint{4.869876in}{2.970316in}}%
\pgfpathlineto{\pgfqpoint{4.883316in}{2.969220in}}%
\pgfpathlineto{\pgfqpoint{4.896766in}{2.968229in}}%
\pgfpathlineto{\pgfqpoint{4.889400in}{2.959755in}}%
\pgfpathlineto{\pgfqpoint{4.882030in}{2.951317in}}%
\pgfpathlineto{\pgfqpoint{4.874656in}{2.942911in}}%
\pgfpathlineto{\pgfqpoint{4.867277in}{2.934535in}}%
\pgfpathlineto{\pgfqpoint{4.853814in}{2.935327in}}%
\pgfpathlineto{\pgfqpoint{4.840360in}{2.936224in}}%
\pgfpathlineto{\pgfqpoint{4.826915in}{2.937226in}}%
\pgfpathlineto{\pgfqpoint{4.813479in}{2.938334in}}%
\pgfpathlineto{\pgfqpoint{4.820871in}{2.946902in}}%
\pgfpathlineto{\pgfqpoint{4.828258in}{2.955505in}}%
\pgfpathlineto{\pgfqpoint{4.835642in}{2.964145in}}%
\pgfpathlineto{\pgfqpoint{4.843021in}{2.972824in}}%
\pgfpathclose%
\pgfusepath{fill}%
\end{pgfscope}%
\begin{pgfscope}%
\pgfpathrectangle{\pgfqpoint{1.150000in}{0.150000in}}{\pgfqpoint{5.700000in}{5.700000in}}%
\pgfusepath{clip}%
\pgfsetbuttcap%
\pgfsetroundjoin%
\definecolor{currentfill}{rgb}{0.282656,0.100196,0.422160}%
\pgfsetfillcolor{currentfill}%
\pgfsetfillopacity{0.700000}%
\pgfsetlinewidth{0.000000pt}%
\definecolor{currentstroke}{rgb}{0.000000,0.000000,0.000000}%
\pgfsetstrokecolor{currentstroke}%
\pgfsetdash{}{0pt}%
\pgfpathmoveto{\pgfqpoint{4.236887in}{2.841136in}}%
\pgfpathlineto{\pgfqpoint{4.250147in}{2.837169in}}%
\pgfpathlineto{\pgfqpoint{4.263412in}{2.833321in}}%
\pgfpathlineto{\pgfqpoint{4.276684in}{2.829592in}}%
\pgfpathlineto{\pgfqpoint{4.289962in}{2.825980in}}%
\pgfpathlineto{\pgfqpoint{4.282389in}{2.816977in}}%
\pgfpathlineto{\pgfqpoint{4.274811in}{2.807998in}}%
\pgfpathlineto{\pgfqpoint{4.267228in}{2.799042in}}%
\pgfpathlineto{\pgfqpoint{4.259640in}{2.790109in}}%
\pgfpathlineto{\pgfqpoint{4.246351in}{2.793628in}}%
\pgfpathlineto{\pgfqpoint{4.233068in}{2.797266in}}%
\pgfpathlineto{\pgfqpoint{4.219791in}{2.801023in}}%
\pgfpathlineto{\pgfqpoint{4.206521in}{2.804900in}}%
\pgfpathlineto{\pgfqpoint{4.214120in}{2.813918in}}%
\pgfpathlineto{\pgfqpoint{4.221714in}{2.822963in}}%
\pgfpathlineto{\pgfqpoint{4.229303in}{2.832035in}}%
\pgfpathlineto{\pgfqpoint{4.236887in}{2.841136in}}%
\pgfpathclose%
\pgfusepath{fill}%
\end{pgfscope}%
\begin{pgfscope}%
\pgfpathrectangle{\pgfqpoint{1.150000in}{0.150000in}}{\pgfqpoint{5.700000in}{5.700000in}}%
\pgfusepath{clip}%
\pgfsetbuttcap%
\pgfsetroundjoin%
\definecolor{currentfill}{rgb}{0.282327,0.094955,0.417331}%
\pgfsetfillcolor{currentfill}%
\pgfsetfillopacity{0.700000}%
\pgfsetlinewidth{0.000000pt}%
\definecolor{currentstroke}{rgb}{0.000000,0.000000,0.000000}%
\pgfsetstrokecolor{currentstroke}%
\pgfsetdash{}{0pt}%
\pgfpathmoveto{\pgfqpoint{3.880807in}{2.834515in}}%
\pgfpathlineto{\pgfqpoint{3.894000in}{2.828249in}}%
\pgfpathlineto{\pgfqpoint{3.907197in}{2.822116in}}%
\pgfpathlineto{\pgfqpoint{3.920398in}{2.816114in}}%
\pgfpathlineto{\pgfqpoint{3.933603in}{2.810244in}}%
\pgfpathlineto{\pgfqpoint{3.925913in}{2.801237in}}%
\pgfpathlineto{\pgfqpoint{3.918218in}{2.792263in}}%
\pgfpathlineto{\pgfqpoint{3.910518in}{2.783323in}}%
\pgfpathlineto{\pgfqpoint{3.902813in}{2.774416in}}%
\pgfpathlineto{\pgfqpoint{3.889596in}{2.780249in}}%
\pgfpathlineto{\pgfqpoint{3.876383in}{2.786213in}}%
\pgfpathlineto{\pgfqpoint{3.863174in}{2.792309in}}%
\pgfpathlineto{\pgfqpoint{3.849969in}{2.798538in}}%
\pgfpathlineto{\pgfqpoint{3.857686in}{2.807476in}}%
\pgfpathlineto{\pgfqpoint{3.865399in}{2.816451in}}%
\pgfpathlineto{\pgfqpoint{3.873105in}{2.825464in}}%
\pgfpathlineto{\pgfqpoint{3.880807in}{2.834515in}}%
\pgfpathclose%
\pgfusepath{fill}%
\end{pgfscope}%
\begin{pgfscope}%
\pgfpathrectangle{\pgfqpoint{1.150000in}{0.150000in}}{\pgfqpoint{5.700000in}{5.700000in}}%
\pgfusepath{clip}%
\pgfsetbuttcap%
\pgfsetroundjoin%
\definecolor{currentfill}{rgb}{0.282656,0.100196,0.422160}%
\pgfsetfillcolor{currentfill}%
\pgfsetfillopacity{0.700000}%
\pgfsetlinewidth{0.000000pt}%
\definecolor{currentstroke}{rgb}{0.000000,0.000000,0.000000}%
\pgfsetstrokecolor{currentstroke}%
\pgfsetdash{}{0pt}%
\pgfpathmoveto{\pgfqpoint{3.744461in}{2.853242in}}%
\pgfpathlineto{\pgfqpoint{3.757638in}{2.845922in}}%
\pgfpathlineto{\pgfqpoint{3.770818in}{2.838742in}}%
\pgfpathlineto{\pgfqpoint{3.784001in}{2.831700in}}%
\pgfpathlineto{\pgfqpoint{3.797188in}{2.824796in}}%
\pgfpathlineto{\pgfqpoint{3.789453in}{2.815868in}}%
\pgfpathlineto{\pgfqpoint{3.781713in}{2.806981in}}%
\pgfpathlineto{\pgfqpoint{3.773967in}{2.798133in}}%
\pgfpathlineto{\pgfqpoint{3.766216in}{2.789323in}}%
\pgfpathlineto{\pgfqpoint{3.753017in}{2.796208in}}%
\pgfpathlineto{\pgfqpoint{3.739822in}{2.803231in}}%
\pgfpathlineto{\pgfqpoint{3.726629in}{2.810392in}}%
\pgfpathlineto{\pgfqpoint{3.713440in}{2.817693in}}%
\pgfpathlineto{\pgfqpoint{3.721203in}{2.826515in}}%
\pgfpathlineto{\pgfqpoint{3.728962in}{2.835380in}}%
\pgfpathlineto{\pgfqpoint{3.736714in}{2.844289in}}%
\pgfpathlineto{\pgfqpoint{3.744461in}{2.853242in}}%
\pgfpathclose%
\pgfusepath{fill}%
\end{pgfscope}%
\begin{pgfscope}%
\pgfpathrectangle{\pgfqpoint{1.150000in}{0.150000in}}{\pgfqpoint{5.700000in}{5.700000in}}%
\pgfusepath{clip}%
\pgfsetbuttcap%
\pgfsetroundjoin%
\definecolor{currentfill}{rgb}{0.283197,0.115680,0.436115}%
\pgfsetfillcolor{currentfill}%
\pgfsetfillopacity{0.700000}%
\pgfsetlinewidth{0.000000pt}%
\definecolor{currentstroke}{rgb}{0.000000,0.000000,0.000000}%
\pgfsetstrokecolor{currentstroke}%
\pgfsetdash{}{0pt}%
\pgfpathmoveto{\pgfqpoint{4.456653in}{2.872485in}}%
\pgfpathlineto{\pgfqpoint{4.469968in}{2.869679in}}%
\pgfpathlineto{\pgfqpoint{4.483291in}{2.866985in}}%
\pgfpathlineto{\pgfqpoint{4.496620in}{2.864405in}}%
\pgfpathlineto{\pgfqpoint{4.509957in}{2.861936in}}%
\pgfpathlineto{\pgfqpoint{4.502457in}{2.853084in}}%
\pgfpathlineto{\pgfqpoint{4.494951in}{2.844254in}}%
\pgfpathlineto{\pgfqpoint{4.487441in}{2.835447in}}%
\pgfpathlineto{\pgfqpoint{4.479926in}{2.826660in}}%
\pgfpathlineto{\pgfqpoint{4.466577in}{2.829001in}}%
\pgfpathlineto{\pgfqpoint{4.453236in}{2.831454in}}%
\pgfpathlineto{\pgfqpoint{4.439902in}{2.834020in}}%
\pgfpathlineto{\pgfqpoint{4.426575in}{2.836700in}}%
\pgfpathlineto{\pgfqpoint{4.434102in}{2.845607in}}%
\pgfpathlineto{\pgfqpoint{4.441623in}{2.854540in}}%
\pgfpathlineto{\pgfqpoint{4.449140in}{2.863498in}}%
\pgfpathlineto{\pgfqpoint{4.456653in}{2.872485in}}%
\pgfpathclose%
\pgfusepath{fill}%
\end{pgfscope}%
\begin{pgfscope}%
\pgfpathrectangle{\pgfqpoint{1.150000in}{0.150000in}}{\pgfqpoint{5.700000in}{5.700000in}}%
\pgfusepath{clip}%
\pgfsetbuttcap%
\pgfsetroundjoin%
\definecolor{currentfill}{rgb}{0.281412,0.155834,0.469201}%
\pgfsetfillcolor{currentfill}%
\pgfsetfillopacity{0.700000}%
\pgfsetlinewidth{0.000000pt}%
\definecolor{currentstroke}{rgb}{0.000000,0.000000,0.000000}%
\pgfsetstrokecolor{currentstroke}%
\pgfsetdash{}{0pt}%
\pgfpathmoveto{\pgfqpoint{4.759818in}{2.943827in}}%
\pgfpathlineto{\pgfqpoint{4.773220in}{2.942294in}}%
\pgfpathlineto{\pgfqpoint{4.786631in}{2.940867in}}%
\pgfpathlineto{\pgfqpoint{4.800051in}{2.939547in}}%
\pgfpathlineto{\pgfqpoint{4.813479in}{2.938334in}}%
\pgfpathlineto{\pgfqpoint{4.806082in}{2.929798in}}%
\pgfpathlineto{\pgfqpoint{4.798681in}{2.921291in}}%
\pgfpathlineto{\pgfqpoint{4.791276in}{2.912813in}}%
\pgfpathlineto{\pgfqpoint{4.783865in}{2.904360in}}%
\pgfpathlineto{\pgfqpoint{4.770425in}{2.905393in}}%
\pgfpathlineto{\pgfqpoint{4.756993in}{2.906531in}}%
\pgfpathlineto{\pgfqpoint{4.743569in}{2.907777in}}%
\pgfpathlineto{\pgfqpoint{4.730154in}{2.909129in}}%
\pgfpathlineto{\pgfqpoint{4.737576in}{2.917757in}}%
\pgfpathlineto{\pgfqpoint{4.744995in}{2.926414in}}%
\pgfpathlineto{\pgfqpoint{4.752409in}{2.935104in}}%
\pgfpathlineto{\pgfqpoint{4.759818in}{2.943827in}}%
\pgfpathclose%
\pgfusepath{fill}%
\end{pgfscope}%
\begin{pgfscope}%
\pgfpathrectangle{\pgfqpoint{1.150000in}{0.150000in}}{\pgfqpoint{5.700000in}{5.700000in}}%
\pgfusepath{clip}%
\pgfsetbuttcap%
\pgfsetroundjoin%
\definecolor{currentfill}{rgb}{0.281412,0.155834,0.469201}%
\pgfsetfillcolor{currentfill}%
\pgfsetfillopacity{0.700000}%
\pgfsetlinewidth{0.000000pt}%
\definecolor{currentstroke}{rgb}{0.000000,0.000000,0.000000}%
\pgfsetstrokecolor{currentstroke}%
\pgfsetdash{}{0pt}%
\pgfpathmoveto{\pgfqpoint{3.418734in}{2.959641in}}%
\pgfpathlineto{\pgfqpoint{3.431902in}{2.949338in}}%
\pgfpathlineto{\pgfqpoint{3.445070in}{2.939195in}}%
\pgfpathlineto{\pgfqpoint{3.458238in}{2.929213in}}%
\pgfpathlineto{\pgfqpoint{3.471408in}{2.919389in}}%
\pgfpathlineto{\pgfqpoint{3.463561in}{2.910820in}}%
\pgfpathlineto{\pgfqpoint{3.455709in}{2.902311in}}%
\pgfpathlineto{\pgfqpoint{3.447850in}{2.893860in}}%
\pgfpathlineto{\pgfqpoint{3.439986in}{2.885468in}}%
\pgfpathlineto{\pgfqpoint{3.426802in}{2.895310in}}%
\pgfpathlineto{\pgfqpoint{3.413618in}{2.905310in}}%
\pgfpathlineto{\pgfqpoint{3.400436in}{2.915470in}}%
\pgfpathlineto{\pgfqpoint{3.387253in}{2.925792in}}%
\pgfpathlineto{\pgfqpoint{3.395133in}{2.934159in}}%
\pgfpathlineto{\pgfqpoint{3.403006in}{2.942590in}}%
\pgfpathlineto{\pgfqpoint{3.410873in}{2.951084in}}%
\pgfpathlineto{\pgfqpoint{3.418734in}{2.959641in}}%
\pgfpathclose%
\pgfusepath{fill}%
\end{pgfscope}%
\begin{pgfscope}%
\pgfpathrectangle{\pgfqpoint{1.150000in}{0.150000in}}{\pgfqpoint{5.700000in}{5.700000in}}%
\pgfusepath{clip}%
\pgfsetbuttcap%
\pgfsetroundjoin%
\definecolor{currentfill}{rgb}{0.281924,0.089666,0.412415}%
\pgfsetfillcolor{currentfill}%
\pgfsetfillopacity{0.700000}%
\pgfsetlinewidth{0.000000pt}%
\definecolor{currentstroke}{rgb}{0.000000,0.000000,0.000000}%
\pgfsetstrokecolor{currentstroke}%
\pgfsetdash{}{0pt}%
\pgfpathmoveto{\pgfqpoint{4.017131in}{2.824223in}}%
\pgfpathlineto{\pgfqpoint{4.030348in}{2.818941in}}%
\pgfpathlineto{\pgfqpoint{4.043569in}{2.813787in}}%
\pgfpathlineto{\pgfqpoint{4.056796in}{2.808758in}}%
\pgfpathlineto{\pgfqpoint{4.070027in}{2.803855in}}%
\pgfpathlineto{\pgfqpoint{4.062381in}{2.794827in}}%
\pgfpathlineto{\pgfqpoint{4.054729in}{2.785828in}}%
\pgfpathlineto{\pgfqpoint{4.047073in}{2.776856in}}%
\pgfpathlineto{\pgfqpoint{4.039411in}{2.767912in}}%
\pgfpathlineto{\pgfqpoint{4.026168in}{2.772759in}}%
\pgfpathlineto{\pgfqpoint{4.012930in}{2.777732in}}%
\pgfpathlineto{\pgfqpoint{3.999697in}{2.782832in}}%
\pgfpathlineto{\pgfqpoint{3.986469in}{2.788058in}}%
\pgfpathlineto{\pgfqpoint{3.994142in}{2.797051in}}%
\pgfpathlineto{\pgfqpoint{4.001810in}{2.806076in}}%
\pgfpathlineto{\pgfqpoint{4.009473in}{2.815133in}}%
\pgfpathlineto{\pgfqpoint{4.017131in}{2.824223in}}%
\pgfpathclose%
\pgfusepath{fill}%
\end{pgfscope}%
\begin{pgfscope}%
\pgfpathrectangle{\pgfqpoint{1.150000in}{0.150000in}}{\pgfqpoint{5.700000in}{5.700000in}}%
\pgfusepath{clip}%
\pgfsetbuttcap%
\pgfsetroundjoin%
\definecolor{currentfill}{rgb}{0.283197,0.115680,0.436115}%
\pgfsetfillcolor{currentfill}%
\pgfsetfillopacity{0.700000}%
\pgfsetlinewidth{0.000000pt}%
\definecolor{currentstroke}{rgb}{0.000000,0.000000,0.000000}%
\pgfsetstrokecolor{currentstroke}%
\pgfsetdash{}{0pt}%
\pgfpathmoveto{\pgfqpoint{3.608022in}{2.881231in}}%
\pgfpathlineto{\pgfqpoint{3.621191in}{2.872781in}}%
\pgfpathlineto{\pgfqpoint{3.634362in}{2.864478in}}%
\pgfpathlineto{\pgfqpoint{3.647535in}{2.856321in}}%
\pgfpathlineto{\pgfqpoint{3.660711in}{2.848309in}}%
\pgfpathlineto{\pgfqpoint{3.652929in}{2.839522in}}%
\pgfpathlineto{\pgfqpoint{3.645141in}{2.830783in}}%
\pgfpathlineto{\pgfqpoint{3.637348in}{2.822090in}}%
\pgfpathlineto{\pgfqpoint{3.629549in}{2.813444in}}%
\pgfpathlineto{\pgfqpoint{3.616360in}{2.821455in}}%
\pgfpathlineto{\pgfqpoint{3.603174in}{2.829611in}}%
\pgfpathlineto{\pgfqpoint{3.589989in}{2.837913in}}%
\pgfpathlineto{\pgfqpoint{3.576807in}{2.846363in}}%
\pgfpathlineto{\pgfqpoint{3.584619in}{2.855003in}}%
\pgfpathlineto{\pgfqpoint{3.592426in}{2.863695in}}%
\pgfpathlineto{\pgfqpoint{3.600227in}{2.872437in}}%
\pgfpathlineto{\pgfqpoint{3.608022in}{2.881231in}}%
\pgfpathclose%
\pgfusepath{fill}%
\end{pgfscope}%
\begin{pgfscope}%
\pgfpathrectangle{\pgfqpoint{1.150000in}{0.150000in}}{\pgfqpoint{5.700000in}{5.700000in}}%
\pgfusepath{clip}%
\pgfsetbuttcap%
\pgfsetroundjoin%
\definecolor{currentfill}{rgb}{0.257322,0.256130,0.526563}%
\pgfsetfillcolor{currentfill}%
\pgfsetfillopacity{0.700000}%
\pgfsetlinewidth{0.000000pt}%
\definecolor{currentstroke}{rgb}{0.000000,0.000000,0.000000}%
\pgfsetstrokecolor{currentstroke}%
\pgfsetdash{}{0pt}%
\pgfpathmoveto{\pgfqpoint{3.123475in}{3.168206in}}%
\pgfpathlineto{\pgfqpoint{3.136679in}{3.154361in}}%
\pgfpathlineto{\pgfqpoint{3.149881in}{3.140707in}}%
\pgfpathlineto{\pgfqpoint{3.163081in}{3.127243in}}%
\pgfpathlineto{\pgfqpoint{3.176278in}{3.113966in}}%
\pgfpathlineto{\pgfqpoint{3.168328in}{3.105784in}}%
\pgfpathlineto{\pgfqpoint{3.160371in}{3.097682in}}%
\pgfpathlineto{\pgfqpoint{3.152406in}{3.089661in}}%
\pgfpathlineto{\pgfqpoint{3.144435in}{3.081721in}}%
\pgfpathlineto{\pgfqpoint{3.131220in}{3.095036in}}%
\pgfpathlineto{\pgfqpoint{3.118003in}{3.108538in}}%
\pgfpathlineto{\pgfqpoint{3.104784in}{3.122231in}}%
\pgfpathlineto{\pgfqpoint{3.091561in}{3.136115in}}%
\pgfpathlineto{\pgfqpoint{3.099550in}{3.144009in}}%
\pgfpathlineto{\pgfqpoint{3.107532in}{3.151989in}}%
\pgfpathlineto{\pgfqpoint{3.115507in}{3.160055in}}%
\pgfpathlineto{\pgfqpoint{3.123475in}{3.168206in}}%
\pgfpathclose%
\pgfusepath{fill}%
\end{pgfscope}%
\begin{pgfscope}%
\pgfpathrectangle{\pgfqpoint{1.150000in}{0.150000in}}{\pgfqpoint{5.700000in}{5.700000in}}%
\pgfusepath{clip}%
\pgfsetbuttcap%
\pgfsetroundjoin%
\definecolor{currentfill}{rgb}{0.265145,0.232956,0.516599}%
\pgfsetfillcolor{currentfill}%
\pgfsetfillopacity{0.700000}%
\pgfsetlinewidth{0.000000pt}%
\definecolor{currentstroke}{rgb}{0.000000,0.000000,0.000000}%
\pgfsetstrokecolor{currentstroke}%
\pgfsetdash{}{0pt}%
\pgfpathmoveto{\pgfqpoint{3.176278in}{3.113966in}}%
\pgfpathlineto{\pgfqpoint{3.189473in}{3.100876in}}%
\pgfpathlineto{\pgfqpoint{3.202667in}{3.087971in}}%
\pgfpathlineto{\pgfqpoint{3.215858in}{3.075248in}}%
\pgfpathlineto{\pgfqpoint{3.229048in}{3.062707in}}%
\pgfpathlineto{\pgfqpoint{3.221114in}{3.054493in}}%
\pgfpathlineto{\pgfqpoint{3.213174in}{3.046356in}}%
\pgfpathlineto{\pgfqpoint{3.205227in}{3.038295in}}%
\pgfpathlineto{\pgfqpoint{3.197273in}{3.030310in}}%
\pgfpathlineto{\pgfqpoint{3.184066in}{3.042889in}}%
\pgfpathlineto{\pgfqpoint{3.170858in}{3.055650in}}%
\pgfpathlineto{\pgfqpoint{3.157647in}{3.068593in}}%
\pgfpathlineto{\pgfqpoint{3.144435in}{3.081721in}}%
\pgfpathlineto{\pgfqpoint{3.152406in}{3.089661in}}%
\pgfpathlineto{\pgfqpoint{3.160371in}{3.097682in}}%
\pgfpathlineto{\pgfqpoint{3.168328in}{3.105784in}}%
\pgfpathlineto{\pgfqpoint{3.176278in}{3.113966in}}%
\pgfpathclose%
\pgfusepath{fill}%
\end{pgfscope}%
\begin{pgfscope}%
\pgfpathrectangle{\pgfqpoint{1.150000in}{0.150000in}}{\pgfqpoint{5.700000in}{5.700000in}}%
\pgfusepath{clip}%
\pgfsetbuttcap%
\pgfsetroundjoin%
\definecolor{currentfill}{rgb}{0.282290,0.145912,0.461510}%
\pgfsetfillcolor{currentfill}%
\pgfsetfillopacity{0.700000}%
\pgfsetlinewidth{0.000000pt}%
\definecolor{currentstroke}{rgb}{0.000000,0.000000,0.000000}%
\pgfsetstrokecolor{currentstroke}%
\pgfsetdash{}{0pt}%
\pgfpathmoveto{\pgfqpoint{4.676575in}{2.915615in}}%
\pgfpathlineto{\pgfqpoint{4.689958in}{2.913832in}}%
\pgfpathlineto{\pgfqpoint{4.703348in}{2.912156in}}%
\pgfpathlineto{\pgfqpoint{4.716747in}{2.910589in}}%
\pgfpathlineto{\pgfqpoint{4.730154in}{2.909129in}}%
\pgfpathlineto{\pgfqpoint{4.722726in}{2.900529in}}%
\pgfpathlineto{\pgfqpoint{4.715294in}{2.891955in}}%
\pgfpathlineto{\pgfqpoint{4.707858in}{2.883405in}}%
\pgfpathlineto{\pgfqpoint{4.700416in}{2.874877in}}%
\pgfpathlineto{\pgfqpoint{4.686997in}{2.876173in}}%
\pgfpathlineto{\pgfqpoint{4.673586in}{2.877576in}}%
\pgfpathlineto{\pgfqpoint{4.660183in}{2.879088in}}%
\pgfpathlineto{\pgfqpoint{4.646789in}{2.880709in}}%
\pgfpathlineto{\pgfqpoint{4.654242in}{2.889395in}}%
\pgfpathlineto{\pgfqpoint{4.661691in}{2.898106in}}%
\pgfpathlineto{\pgfqpoint{4.669135in}{2.906846in}}%
\pgfpathlineto{\pgfqpoint{4.676575in}{2.915615in}}%
\pgfpathclose%
\pgfusepath{fill}%
\end{pgfscope}%
\begin{pgfscope}%
\pgfpathrectangle{\pgfqpoint{1.150000in}{0.150000in}}{\pgfqpoint{5.700000in}{5.700000in}}%
\pgfusepath{clip}%
\pgfsetbuttcap%
\pgfsetroundjoin%
\definecolor{currentfill}{rgb}{0.282327,0.094955,0.417331}%
\pgfsetfillcolor{currentfill}%
\pgfsetfillopacity{0.700000}%
\pgfsetlinewidth{0.000000pt}%
\definecolor{currentstroke}{rgb}{0.000000,0.000000,0.000000}%
\pgfsetstrokecolor{currentstroke}%
\pgfsetdash{}{0pt}%
\pgfpathmoveto{\pgfqpoint{4.153497in}{2.821609in}}%
\pgfpathlineto{\pgfqpoint{4.166744in}{2.817250in}}%
\pgfpathlineto{\pgfqpoint{4.179997in}{2.813012in}}%
\pgfpathlineto{\pgfqpoint{4.193256in}{2.808896in}}%
\pgfpathlineto{\pgfqpoint{4.206521in}{2.804900in}}%
\pgfpathlineto{\pgfqpoint{4.198917in}{2.795907in}}%
\pgfpathlineto{\pgfqpoint{4.191308in}{2.786938in}}%
\pgfpathlineto{\pgfqpoint{4.183694in}{2.777992in}}%
\pgfpathlineto{\pgfqpoint{4.176075in}{2.769068in}}%
\pgfpathlineto{\pgfqpoint{4.162799in}{2.772990in}}%
\pgfpathlineto{\pgfqpoint{4.149529in}{2.777033in}}%
\pgfpathlineto{\pgfqpoint{4.136265in}{2.781197in}}%
\pgfpathlineto{\pgfqpoint{4.123007in}{2.785483in}}%
\pgfpathlineto{\pgfqpoint{4.130637in}{2.794474in}}%
\pgfpathlineto{\pgfqpoint{4.138262in}{2.803491in}}%
\pgfpathlineto{\pgfqpoint{4.145882in}{2.812536in}}%
\pgfpathlineto{\pgfqpoint{4.153497in}{2.821609in}}%
\pgfpathclose%
\pgfusepath{fill}%
\end{pgfscope}%
\begin{pgfscope}%
\pgfpathrectangle{\pgfqpoint{1.150000in}{0.150000in}}{\pgfqpoint{5.700000in}{5.700000in}}%
\pgfusepath{clip}%
\pgfsetbuttcap%
\pgfsetroundjoin%
\definecolor{currentfill}{rgb}{0.283091,0.110553,0.431554}%
\pgfsetfillcolor{currentfill}%
\pgfsetfillopacity{0.700000}%
\pgfsetlinewidth{0.000000pt}%
\definecolor{currentstroke}{rgb}{0.000000,0.000000,0.000000}%
\pgfsetstrokecolor{currentstroke}%
\pgfsetdash{}{0pt}%
\pgfpathmoveto{\pgfqpoint{4.373337in}{2.848559in}}%
\pgfpathlineto{\pgfqpoint{4.386636in}{2.845422in}}%
\pgfpathlineto{\pgfqpoint{4.399942in}{2.842400in}}%
\pgfpathlineto{\pgfqpoint{4.413255in}{2.839493in}}%
\pgfpathlineto{\pgfqpoint{4.426575in}{2.836700in}}%
\pgfpathlineto{\pgfqpoint{4.419044in}{2.827815in}}%
\pgfpathlineto{\pgfqpoint{4.411508in}{2.818952in}}%
\pgfpathlineto{\pgfqpoint{4.403967in}{2.810109in}}%
\pgfpathlineto{\pgfqpoint{4.396421in}{2.801285in}}%
\pgfpathlineto{\pgfqpoint{4.383090in}{2.803969in}}%
\pgfpathlineto{\pgfqpoint{4.369766in}{2.806766in}}%
\pgfpathlineto{\pgfqpoint{4.356448in}{2.809679in}}%
\pgfpathlineto{\pgfqpoint{4.343138in}{2.812707in}}%
\pgfpathlineto{\pgfqpoint{4.350695in}{2.821634in}}%
\pgfpathlineto{\pgfqpoint{4.358247in}{2.830584in}}%
\pgfpathlineto{\pgfqpoint{4.365794in}{2.839559in}}%
\pgfpathlineto{\pgfqpoint{4.373337in}{2.848559in}}%
\pgfpathclose%
\pgfusepath{fill}%
\end{pgfscope}%
\begin{pgfscope}%
\pgfpathrectangle{\pgfqpoint{1.150000in}{0.150000in}}{\pgfqpoint{5.700000in}{5.700000in}}%
\pgfusepath{clip}%
\pgfsetbuttcap%
\pgfsetroundjoin%
\definecolor{currentfill}{rgb}{0.271828,0.209303,0.504434}%
\pgfsetfillcolor{currentfill}%
\pgfsetfillopacity{0.700000}%
\pgfsetlinewidth{0.000000pt}%
\definecolor{currentstroke}{rgb}{0.000000,0.000000,0.000000}%
\pgfsetstrokecolor{currentstroke}%
\pgfsetdash{}{0pt}%
\pgfpathmoveto{\pgfqpoint{3.229048in}{3.062707in}}%
\pgfpathlineto{\pgfqpoint{3.242236in}{3.050345in}}%
\pgfpathlineto{\pgfqpoint{3.255423in}{3.038162in}}%
\pgfpathlineto{\pgfqpoint{3.268609in}{3.026155in}}%
\pgfpathlineto{\pgfqpoint{3.281794in}{3.014323in}}%
\pgfpathlineto{\pgfqpoint{3.273876in}{3.006078in}}%
\pgfpathlineto{\pgfqpoint{3.265953in}{2.997906in}}%
\pgfpathlineto{\pgfqpoint{3.258022in}{2.989805in}}%
\pgfpathlineto{\pgfqpoint{3.250085in}{2.981776in}}%
\pgfpathlineto{\pgfqpoint{3.236884in}{2.993646in}}%
\pgfpathlineto{\pgfqpoint{3.223682in}{3.005690in}}%
\pgfpathlineto{\pgfqpoint{3.210478in}{3.017911in}}%
\pgfpathlineto{\pgfqpoint{3.197273in}{3.030310in}}%
\pgfpathlineto{\pgfqpoint{3.205227in}{3.038295in}}%
\pgfpathlineto{\pgfqpoint{3.213174in}{3.046356in}}%
\pgfpathlineto{\pgfqpoint{3.221114in}{3.054493in}}%
\pgfpathlineto{\pgfqpoint{3.229048in}{3.062707in}}%
\pgfpathclose%
\pgfusepath{fill}%
\end{pgfscope}%
\begin{pgfscope}%
\pgfpathrectangle{\pgfqpoint{1.150000in}{0.150000in}}{\pgfqpoint{5.700000in}{5.700000in}}%
\pgfusepath{clip}%
\pgfsetbuttcap%
\pgfsetroundjoin%
\definecolor{currentfill}{rgb}{0.282623,0.140926,0.457517}%
\pgfsetfillcolor{currentfill}%
\pgfsetfillopacity{0.700000}%
\pgfsetlinewidth{0.000000pt}%
\definecolor{currentstroke}{rgb}{0.000000,0.000000,0.000000}%
\pgfsetstrokecolor{currentstroke}%
\pgfsetdash{}{0pt}%
\pgfpathmoveto{\pgfqpoint{3.471408in}{2.919389in}}%
\pgfpathlineto{\pgfqpoint{3.484578in}{2.909722in}}%
\pgfpathlineto{\pgfqpoint{3.497749in}{2.900212in}}%
\pgfpathlineto{\pgfqpoint{3.510922in}{2.890856in}}%
\pgfpathlineto{\pgfqpoint{3.524096in}{2.881655in}}%
\pgfpathlineto{\pgfqpoint{3.516264in}{2.873075in}}%
\pgfpathlineto{\pgfqpoint{3.508425in}{2.864550in}}%
\pgfpathlineto{\pgfqpoint{3.500582in}{2.856079in}}%
\pgfpathlineto{\pgfqpoint{3.492732in}{2.847663in}}%
\pgfpathlineto{\pgfqpoint{3.479543in}{2.856882in}}%
\pgfpathlineto{\pgfqpoint{3.466356in}{2.866255in}}%
\pgfpathlineto{\pgfqpoint{3.453171in}{2.875783in}}%
\pgfpathlineto{\pgfqpoint{3.439986in}{2.885468in}}%
\pgfpathlineto{\pgfqpoint{3.447850in}{2.893860in}}%
\pgfpathlineto{\pgfqpoint{3.455709in}{2.902311in}}%
\pgfpathlineto{\pgfqpoint{3.463561in}{2.910820in}}%
\pgfpathlineto{\pgfqpoint{3.471408in}{2.919389in}}%
\pgfpathclose%
\pgfusepath{fill}%
\end{pgfscope}%
\begin{pgfscope}%
\pgfpathrectangle{\pgfqpoint{1.150000in}{0.150000in}}{\pgfqpoint{5.700000in}{5.700000in}}%
\pgfusepath{clip}%
\pgfsetbuttcap%
\pgfsetroundjoin%
\definecolor{currentfill}{rgb}{0.282327,0.094955,0.417331}%
\pgfsetfillcolor{currentfill}%
\pgfsetfillopacity{0.700000}%
\pgfsetlinewidth{0.000000pt}%
\definecolor{currentstroke}{rgb}{0.000000,0.000000,0.000000}%
\pgfsetstrokecolor{currentstroke}%
\pgfsetdash{}{0pt}%
\pgfpathmoveto{\pgfqpoint{3.797188in}{2.824796in}}%
\pgfpathlineto{\pgfqpoint{3.810378in}{2.818029in}}%
\pgfpathlineto{\pgfqpoint{3.823571in}{2.811397in}}%
\pgfpathlineto{\pgfqpoint{3.836768in}{2.804901in}}%
\pgfpathlineto{\pgfqpoint{3.849969in}{2.798538in}}%
\pgfpathlineto{\pgfqpoint{3.842246in}{2.789637in}}%
\pgfpathlineto{\pgfqpoint{3.834518in}{2.780770in}}%
\pgfpathlineto{\pgfqpoint{3.826785in}{2.771940in}}%
\pgfpathlineto{\pgfqpoint{3.819047in}{2.763143in}}%
\pgfpathlineto{\pgfqpoint{3.805833in}{2.769486in}}%
\pgfpathlineto{\pgfqpoint{3.792624in}{2.775963in}}%
\pgfpathlineto{\pgfqpoint{3.779418in}{2.782575in}}%
\pgfpathlineto{\pgfqpoint{3.766216in}{2.789323in}}%
\pgfpathlineto{\pgfqpoint{3.773967in}{2.798133in}}%
\pgfpathlineto{\pgfqpoint{3.781713in}{2.806981in}}%
\pgfpathlineto{\pgfqpoint{3.789453in}{2.815868in}}%
\pgfpathlineto{\pgfqpoint{3.797188in}{2.824796in}}%
\pgfpathclose%
\pgfusepath{fill}%
\end{pgfscope}%
\begin{pgfscope}%
\pgfpathrectangle{\pgfqpoint{1.150000in}{0.150000in}}{\pgfqpoint{5.700000in}{5.700000in}}%
\pgfusepath{clip}%
\pgfsetbuttcap%
\pgfsetroundjoin%
\definecolor{currentfill}{rgb}{0.281924,0.089666,0.412415}%
\pgfsetfillcolor{currentfill}%
\pgfsetfillopacity{0.700000}%
\pgfsetlinewidth{0.000000pt}%
\definecolor{currentstroke}{rgb}{0.000000,0.000000,0.000000}%
\pgfsetstrokecolor{currentstroke}%
\pgfsetdash{}{0pt}%
\pgfpathmoveto{\pgfqpoint{3.933603in}{2.810244in}}%
\pgfpathlineto{\pgfqpoint{3.946813in}{2.804504in}}%
\pgfpathlineto{\pgfqpoint{3.960027in}{2.798893in}}%
\pgfpathlineto{\pgfqpoint{3.973246in}{2.793411in}}%
\pgfpathlineto{\pgfqpoint{3.986469in}{2.788058in}}%
\pgfpathlineto{\pgfqpoint{3.978791in}{2.779095in}}%
\pgfpathlineto{\pgfqpoint{3.971108in}{2.770161in}}%
\pgfpathlineto{\pgfqpoint{3.963419in}{2.761257in}}%
\pgfpathlineto{\pgfqpoint{3.955726in}{2.752380in}}%
\pgfpathlineto{\pgfqpoint{3.942491in}{2.757696in}}%
\pgfpathlineto{\pgfqpoint{3.929260in}{2.763140in}}%
\pgfpathlineto{\pgfqpoint{3.916034in}{2.768713in}}%
\pgfpathlineto{\pgfqpoint{3.902813in}{2.774416in}}%
\pgfpathlineto{\pgfqpoint{3.910518in}{2.783323in}}%
\pgfpathlineto{\pgfqpoint{3.918218in}{2.792263in}}%
\pgfpathlineto{\pgfqpoint{3.925913in}{2.801237in}}%
\pgfpathlineto{\pgfqpoint{3.933603in}{2.810244in}}%
\pgfpathclose%
\pgfusepath{fill}%
\end{pgfscope}%
\begin{pgfscope}%
\pgfpathrectangle{\pgfqpoint{1.150000in}{0.150000in}}{\pgfqpoint{5.700000in}{5.700000in}}%
\pgfusepath{clip}%
\pgfsetbuttcap%
\pgfsetroundjoin%
\definecolor{currentfill}{rgb}{0.283072,0.130895,0.449241}%
\pgfsetfillcolor{currentfill}%
\pgfsetfillopacity{0.700000}%
\pgfsetlinewidth{0.000000pt}%
\definecolor{currentstroke}{rgb}{0.000000,0.000000,0.000000}%
\pgfsetstrokecolor{currentstroke}%
\pgfsetdash{}{0pt}%
\pgfpathmoveto{\pgfqpoint{4.593289in}{2.888282in}}%
\pgfpathlineto{\pgfqpoint{4.606652in}{2.886225in}}%
\pgfpathlineto{\pgfqpoint{4.620023in}{2.884277in}}%
\pgfpathlineto{\pgfqpoint{4.633402in}{2.882438in}}%
\pgfpathlineto{\pgfqpoint{4.646789in}{2.880709in}}%
\pgfpathlineto{\pgfqpoint{4.639330in}{2.872047in}}%
\pgfpathlineto{\pgfqpoint{4.631867in}{2.863408in}}%
\pgfpathlineto{\pgfqpoint{4.624400in}{2.854789in}}%
\pgfpathlineto{\pgfqpoint{4.616927in}{2.846188in}}%
\pgfpathlineto{\pgfqpoint{4.603529in}{2.847772in}}%
\pgfpathlineto{\pgfqpoint{4.590138in}{2.849464in}}%
\pgfpathlineto{\pgfqpoint{4.576755in}{2.851267in}}%
\pgfpathlineto{\pgfqpoint{4.563380in}{2.853179in}}%
\pgfpathlineto{\pgfqpoint{4.570865in}{2.861919in}}%
\pgfpathlineto{\pgfqpoint{4.578344in}{2.870681in}}%
\pgfpathlineto{\pgfqpoint{4.585819in}{2.879469in}}%
\pgfpathlineto{\pgfqpoint{4.593289in}{2.888282in}}%
\pgfpathclose%
\pgfusepath{fill}%
\end{pgfscope}%
\begin{pgfscope}%
\pgfpathrectangle{\pgfqpoint{1.150000in}{0.150000in}}{\pgfqpoint{5.700000in}{5.700000in}}%
\pgfusepath{clip}%
\pgfsetbuttcap%
\pgfsetroundjoin%
\definecolor{currentfill}{rgb}{0.282910,0.105393,0.426902}%
\pgfsetfillcolor{currentfill}%
\pgfsetfillopacity{0.700000}%
\pgfsetlinewidth{0.000000pt}%
\definecolor{currentstroke}{rgb}{0.000000,0.000000,0.000000}%
\pgfsetstrokecolor{currentstroke}%
\pgfsetdash{}{0pt}%
\pgfpathmoveto{\pgfqpoint{3.660711in}{2.848309in}}%
\pgfpathlineto{\pgfqpoint{3.673889in}{2.840441in}}%
\pgfpathlineto{\pgfqpoint{3.687070in}{2.832716in}}%
\pgfpathlineto{\pgfqpoint{3.700254in}{2.825134in}}%
\pgfpathlineto{\pgfqpoint{3.713440in}{2.817693in}}%
\pgfpathlineto{\pgfqpoint{3.705671in}{2.808915in}}%
\pgfpathlineto{\pgfqpoint{3.697896in}{2.800178in}}%
\pgfpathlineto{\pgfqpoint{3.690116in}{2.791485in}}%
\pgfpathlineto{\pgfqpoint{3.682330in}{2.782833in}}%
\pgfpathlineto{\pgfqpoint{3.669131in}{2.790272in}}%
\pgfpathlineto{\pgfqpoint{3.655934in}{2.797854in}}%
\pgfpathlineto{\pgfqpoint{3.642740in}{2.805577in}}%
\pgfpathlineto{\pgfqpoint{3.629549in}{2.813444in}}%
\pgfpathlineto{\pgfqpoint{3.637348in}{2.822090in}}%
\pgfpathlineto{\pgfqpoint{3.645141in}{2.830783in}}%
\pgfpathlineto{\pgfqpoint{3.652929in}{2.839522in}}%
\pgfpathlineto{\pgfqpoint{3.660711in}{2.848309in}}%
\pgfpathclose%
\pgfusepath{fill}%
\end{pgfscope}%
\begin{pgfscope}%
\pgfpathrectangle{\pgfqpoint{1.150000in}{0.150000in}}{\pgfqpoint{5.700000in}{5.700000in}}%
\pgfusepath{clip}%
\pgfsetbuttcap%
\pgfsetroundjoin%
\definecolor{currentfill}{rgb}{0.276194,0.190074,0.493001}%
\pgfsetfillcolor{currentfill}%
\pgfsetfillopacity{0.700000}%
\pgfsetlinewidth{0.000000pt}%
\definecolor{currentstroke}{rgb}{0.000000,0.000000,0.000000}%
\pgfsetstrokecolor{currentstroke}%
\pgfsetdash{}{0pt}%
\pgfpathmoveto{\pgfqpoint{3.281794in}{3.014323in}}%
\pgfpathlineto{\pgfqpoint{3.294978in}{3.002664in}}%
\pgfpathlineto{\pgfqpoint{3.308161in}{2.991178in}}%
\pgfpathlineto{\pgfqpoint{3.321343in}{2.979862in}}%
\pgfpathlineto{\pgfqpoint{3.334526in}{2.968715in}}%
\pgfpathlineto{\pgfqpoint{3.326624in}{2.960440in}}%
\pgfpathlineto{\pgfqpoint{3.318717in}{2.952233in}}%
\pgfpathlineto{\pgfqpoint{3.310802in}{2.944093in}}%
\pgfpathlineto{\pgfqpoint{3.302882in}{2.936021in}}%
\pgfpathlineto{\pgfqpoint{3.289684in}{2.947205in}}%
\pgfpathlineto{\pgfqpoint{3.276485in}{2.958557in}}%
\pgfpathlineto{\pgfqpoint{3.263285in}{2.970081in}}%
\pgfpathlineto{\pgfqpoint{3.250085in}{2.981776in}}%
\pgfpathlineto{\pgfqpoint{3.258022in}{2.989805in}}%
\pgfpathlineto{\pgfqpoint{3.265953in}{2.997906in}}%
\pgfpathlineto{\pgfqpoint{3.273876in}{3.006078in}}%
\pgfpathlineto{\pgfqpoint{3.281794in}{3.014323in}}%
\pgfpathclose%
\pgfusepath{fill}%
\end{pgfscope}%
\begin{pgfscope}%
\pgfpathrectangle{\pgfqpoint{1.150000in}{0.150000in}}{\pgfqpoint{5.700000in}{5.700000in}}%
\pgfusepath{clip}%
\pgfsetbuttcap%
\pgfsetroundjoin%
\definecolor{currentfill}{rgb}{0.282656,0.100196,0.422160}%
\pgfsetfillcolor{currentfill}%
\pgfsetfillopacity{0.700000}%
\pgfsetlinewidth{0.000000pt}%
\definecolor{currentstroke}{rgb}{0.000000,0.000000,0.000000}%
\pgfsetstrokecolor{currentstroke}%
\pgfsetdash{}{0pt}%
\pgfpathmoveto{\pgfqpoint{4.289962in}{2.825980in}}%
\pgfpathlineto{\pgfqpoint{4.303246in}{2.822487in}}%
\pgfpathlineto{\pgfqpoint{4.316537in}{2.819110in}}%
\pgfpathlineto{\pgfqpoint{4.329834in}{2.815851in}}%
\pgfpathlineto{\pgfqpoint{4.343138in}{2.812707in}}%
\pgfpathlineto{\pgfqpoint{4.335576in}{2.803802in}}%
\pgfpathlineto{\pgfqpoint{4.328009in}{2.794916in}}%
\pgfpathlineto{\pgfqpoint{4.320437in}{2.786050in}}%
\pgfpathlineto{\pgfqpoint{4.312861in}{2.777202in}}%
\pgfpathlineto{\pgfqpoint{4.299546in}{2.780254in}}%
\pgfpathlineto{\pgfqpoint{4.286237in}{2.783422in}}%
\pgfpathlineto{\pgfqpoint{4.272935in}{2.786706in}}%
\pgfpathlineto{\pgfqpoint{4.259640in}{2.790109in}}%
\pgfpathlineto{\pgfqpoint{4.267228in}{2.799042in}}%
\pgfpathlineto{\pgfqpoint{4.274811in}{2.807998in}}%
\pgfpathlineto{\pgfqpoint{4.282389in}{2.816977in}}%
\pgfpathlineto{\pgfqpoint{4.289962in}{2.825980in}}%
\pgfpathclose%
\pgfusepath{fill}%
\end{pgfscope}%
\begin{pgfscope}%
\pgfpathrectangle{\pgfqpoint{1.150000in}{0.150000in}}{\pgfqpoint{5.700000in}{5.700000in}}%
\pgfusepath{clip}%
\pgfsetbuttcap%
\pgfsetroundjoin%
\definecolor{currentfill}{rgb}{0.278826,0.175490,0.483397}%
\pgfsetfillcolor{currentfill}%
\pgfsetfillopacity{0.700000}%
\pgfsetlinewidth{0.000000pt}%
\definecolor{currentstroke}{rgb}{0.000000,0.000000,0.000000}%
\pgfsetstrokecolor{currentstroke}%
\pgfsetdash{}{0pt}%
\pgfpathmoveto{\pgfqpoint{4.896766in}{2.968229in}}%
\pgfpathlineto{\pgfqpoint{4.910224in}{2.967343in}}%
\pgfpathlineto{\pgfqpoint{4.923691in}{2.966561in}}%
\pgfpathlineto{\pgfqpoint{4.937167in}{2.965883in}}%
\pgfpathlineto{\pgfqpoint{4.950653in}{2.965308in}}%
\pgfpathlineto{\pgfqpoint{4.943301in}{2.957041in}}%
\pgfpathlineto{\pgfqpoint{4.935944in}{2.948804in}}%
\pgfpathlineto{\pgfqpoint{4.928584in}{2.940596in}}%
\pgfpathlineto{\pgfqpoint{4.921218in}{2.932414in}}%
\pgfpathlineto{\pgfqpoint{4.907719in}{2.932788in}}%
\pgfpathlineto{\pgfqpoint{4.894229in}{2.933266in}}%
\pgfpathlineto{\pgfqpoint{4.880749in}{2.933848in}}%
\pgfpathlineto{\pgfqpoint{4.867277in}{2.934535in}}%
\pgfpathlineto{\pgfqpoint{4.874656in}{2.942911in}}%
\pgfpathlineto{\pgfqpoint{4.882030in}{2.951317in}}%
\pgfpathlineto{\pgfqpoint{4.889400in}{2.959755in}}%
\pgfpathlineto{\pgfqpoint{4.896766in}{2.968229in}}%
\pgfpathclose%
\pgfusepath{fill}%
\end{pgfscope}%
\begin{pgfscope}%
\pgfpathrectangle{\pgfqpoint{1.150000in}{0.150000in}}{\pgfqpoint{5.700000in}{5.700000in}}%
\pgfusepath{clip}%
\pgfsetbuttcap%
\pgfsetroundjoin%
\definecolor{currentfill}{rgb}{0.281924,0.089666,0.412415}%
\pgfsetfillcolor{currentfill}%
\pgfsetfillopacity{0.700000}%
\pgfsetlinewidth{0.000000pt}%
\definecolor{currentstroke}{rgb}{0.000000,0.000000,0.000000}%
\pgfsetstrokecolor{currentstroke}%
\pgfsetdash{}{0pt}%
\pgfpathmoveto{\pgfqpoint{4.070027in}{2.803855in}}%
\pgfpathlineto{\pgfqpoint{4.083264in}{2.799076in}}%
\pgfpathlineto{\pgfqpoint{4.096506in}{2.794422in}}%
\pgfpathlineto{\pgfqpoint{4.109754in}{2.789891in}}%
\pgfpathlineto{\pgfqpoint{4.123007in}{2.785483in}}%
\pgfpathlineto{\pgfqpoint{4.115372in}{2.776517in}}%
\pgfpathlineto{\pgfqpoint{4.107732in}{2.767576in}}%
\pgfpathlineto{\pgfqpoint{4.100086in}{2.758658in}}%
\pgfpathlineto{\pgfqpoint{4.092436in}{2.749763in}}%
\pgfpathlineto{\pgfqpoint{4.079172in}{2.754115in}}%
\pgfpathlineto{\pgfqpoint{4.065913in}{2.758590in}}%
\pgfpathlineto{\pgfqpoint{4.052660in}{2.763189in}}%
\pgfpathlineto{\pgfqpoint{4.039411in}{2.767912in}}%
\pgfpathlineto{\pgfqpoint{4.047073in}{2.776856in}}%
\pgfpathlineto{\pgfqpoint{4.054729in}{2.785828in}}%
\pgfpathlineto{\pgfqpoint{4.062381in}{2.794827in}}%
\pgfpathlineto{\pgfqpoint{4.070027in}{2.803855in}}%
\pgfpathclose%
\pgfusepath{fill}%
\end{pgfscope}%
\begin{pgfscope}%
\pgfpathrectangle{\pgfqpoint{1.150000in}{0.150000in}}{\pgfqpoint{5.700000in}{5.700000in}}%
\pgfusepath{clip}%
\pgfsetbuttcap%
\pgfsetroundjoin%
\definecolor{currentfill}{rgb}{0.283187,0.125848,0.444960}%
\pgfsetfillcolor{currentfill}%
\pgfsetfillopacity{0.700000}%
\pgfsetlinewidth{0.000000pt}%
\definecolor{currentstroke}{rgb}{0.000000,0.000000,0.000000}%
\pgfsetstrokecolor{currentstroke}%
\pgfsetdash{}{0pt}%
\pgfpathmoveto{\pgfqpoint{3.524096in}{2.881655in}}%
\pgfpathlineto{\pgfqpoint{3.537271in}{2.872605in}}%
\pgfpathlineto{\pgfqpoint{3.550448in}{2.863708in}}%
\pgfpathlineto{\pgfqpoint{3.563626in}{2.854960in}}%
\pgfpathlineto{\pgfqpoint{3.576807in}{2.846363in}}%
\pgfpathlineto{\pgfqpoint{3.568989in}{2.837772in}}%
\pgfpathlineto{\pgfqpoint{3.561165in}{2.829232in}}%
\pgfpathlineto{\pgfqpoint{3.553335in}{2.820742in}}%
\pgfpathlineto{\pgfqpoint{3.545499in}{2.812302in}}%
\pgfpathlineto{\pgfqpoint{3.532305in}{2.820917in}}%
\pgfpathlineto{\pgfqpoint{3.519112in}{2.829681in}}%
\pgfpathlineto{\pgfqpoint{3.505921in}{2.838596in}}%
\pgfpathlineto{\pgfqpoint{3.492732in}{2.847663in}}%
\pgfpathlineto{\pgfqpoint{3.500582in}{2.856079in}}%
\pgfpathlineto{\pgfqpoint{3.508425in}{2.864550in}}%
\pgfpathlineto{\pgfqpoint{3.516264in}{2.873075in}}%
\pgfpathlineto{\pgfqpoint{3.524096in}{2.881655in}}%
\pgfpathclose%
\pgfusepath{fill}%
\end{pgfscope}%
\begin{pgfscope}%
\pgfpathrectangle{\pgfqpoint{1.150000in}{0.150000in}}{\pgfqpoint{5.700000in}{5.700000in}}%
\pgfusepath{clip}%
\pgfsetbuttcap%
\pgfsetroundjoin%
\definecolor{currentfill}{rgb}{0.283229,0.120777,0.440584}%
\pgfsetfillcolor{currentfill}%
\pgfsetfillopacity{0.700000}%
\pgfsetlinewidth{0.000000pt}%
\definecolor{currentstroke}{rgb}{0.000000,0.000000,0.000000}%
\pgfsetstrokecolor{currentstroke}%
\pgfsetdash{}{0pt}%
\pgfpathmoveto{\pgfqpoint{4.509957in}{2.861936in}}%
\pgfpathlineto{\pgfqpoint{4.523302in}{2.859580in}}%
\pgfpathlineto{\pgfqpoint{4.536654in}{2.857335in}}%
\pgfpathlineto{\pgfqpoint{4.550013in}{2.855202in}}%
\pgfpathlineto{\pgfqpoint{4.563380in}{2.853179in}}%
\pgfpathlineto{\pgfqpoint{4.555891in}{2.844460in}}%
\pgfpathlineto{\pgfqpoint{4.548398in}{2.835761in}}%
\pgfpathlineto{\pgfqpoint{4.540899in}{2.827079in}}%
\pgfpathlineto{\pgfqpoint{4.533395in}{2.818414in}}%
\pgfpathlineto{\pgfqpoint{4.520017in}{2.820308in}}%
\pgfpathlineto{\pgfqpoint{4.506646in}{2.822314in}}%
\pgfpathlineto{\pgfqpoint{4.493282in}{2.824431in}}%
\pgfpathlineto{\pgfqpoint{4.479926in}{2.826660in}}%
\pgfpathlineto{\pgfqpoint{4.487441in}{2.835447in}}%
\pgfpathlineto{\pgfqpoint{4.494951in}{2.844254in}}%
\pgfpathlineto{\pgfqpoint{4.502457in}{2.853084in}}%
\pgfpathlineto{\pgfqpoint{4.509957in}{2.861936in}}%
\pgfpathclose%
\pgfusepath{fill}%
\end{pgfscope}%
\begin{pgfscope}%
\pgfpathrectangle{\pgfqpoint{1.150000in}{0.150000in}}{\pgfqpoint{5.700000in}{5.700000in}}%
\pgfusepath{clip}%
\pgfsetbuttcap%
\pgfsetroundjoin%
\definecolor{currentfill}{rgb}{0.279574,0.170599,0.479997}%
\pgfsetfillcolor{currentfill}%
\pgfsetfillopacity{0.700000}%
\pgfsetlinewidth{0.000000pt}%
\definecolor{currentstroke}{rgb}{0.000000,0.000000,0.000000}%
\pgfsetstrokecolor{currentstroke}%
\pgfsetdash{}{0pt}%
\pgfpathmoveto{\pgfqpoint{3.334526in}{2.968715in}}%
\pgfpathlineto{\pgfqpoint{3.347708in}{2.957736in}}%
\pgfpathlineto{\pgfqpoint{3.360890in}{2.946923in}}%
\pgfpathlineto{\pgfqpoint{3.374071in}{2.936276in}}%
\pgfpathlineto{\pgfqpoint{3.387253in}{2.925792in}}%
\pgfpathlineto{\pgfqpoint{3.379368in}{2.917487in}}%
\pgfpathlineto{\pgfqpoint{3.371476in}{2.909246in}}%
\pgfpathlineto{\pgfqpoint{3.363577in}{2.901068in}}%
\pgfpathlineto{\pgfqpoint{3.355672in}{2.892953in}}%
\pgfpathlineto{\pgfqpoint{3.342475in}{2.903473in}}%
\pgfpathlineto{\pgfqpoint{3.329277in}{2.914157in}}%
\pgfpathlineto{\pgfqpoint{3.316080in}{2.925006in}}%
\pgfpathlineto{\pgfqpoint{3.302882in}{2.936021in}}%
\pgfpathlineto{\pgfqpoint{3.310802in}{2.944093in}}%
\pgfpathlineto{\pgfqpoint{3.318717in}{2.952233in}}%
\pgfpathlineto{\pgfqpoint{3.326624in}{2.960440in}}%
\pgfpathlineto{\pgfqpoint{3.334526in}{2.968715in}}%
\pgfpathclose%
\pgfusepath{fill}%
\end{pgfscope}%
\begin{pgfscope}%
\pgfpathrectangle{\pgfqpoint{1.150000in}{0.150000in}}{\pgfqpoint{5.700000in}{5.700000in}}%
\pgfusepath{clip}%
\pgfsetbuttcap%
\pgfsetroundjoin%
\definecolor{currentfill}{rgb}{0.280255,0.165693,0.476498}%
\pgfsetfillcolor{currentfill}%
\pgfsetfillopacity{0.700000}%
\pgfsetlinewidth{0.000000pt}%
\definecolor{currentstroke}{rgb}{0.000000,0.000000,0.000000}%
\pgfsetstrokecolor{currentstroke}%
\pgfsetdash{}{0pt}%
\pgfpathmoveto{\pgfqpoint{4.813479in}{2.938334in}}%
\pgfpathlineto{\pgfqpoint{4.826915in}{2.937226in}}%
\pgfpathlineto{\pgfqpoint{4.840360in}{2.936224in}}%
\pgfpathlineto{\pgfqpoint{4.853814in}{2.935327in}}%
\pgfpathlineto{\pgfqpoint{4.867277in}{2.934535in}}%
\pgfpathlineto{\pgfqpoint{4.859893in}{2.926187in}}%
\pgfpathlineto{\pgfqpoint{4.852506in}{2.917865in}}%
\pgfpathlineto{\pgfqpoint{4.845113in}{2.909567in}}%
\pgfpathlineto{\pgfqpoint{4.837716in}{2.901290in}}%
\pgfpathlineto{\pgfqpoint{4.824240in}{2.901899in}}%
\pgfpathlineto{\pgfqpoint{4.810773in}{2.902614in}}%
\pgfpathlineto{\pgfqpoint{4.797315in}{2.903434in}}%
\pgfpathlineto{\pgfqpoint{4.783865in}{2.904360in}}%
\pgfpathlineto{\pgfqpoint{4.791276in}{2.912813in}}%
\pgfpathlineto{\pgfqpoint{4.798681in}{2.921291in}}%
\pgfpathlineto{\pgfqpoint{4.806082in}{2.929798in}}%
\pgfpathlineto{\pgfqpoint{4.813479in}{2.938334in}}%
\pgfpathclose%
\pgfusepath{fill}%
\end{pgfscope}%
\begin{pgfscope}%
\pgfpathrectangle{\pgfqpoint{1.150000in}{0.150000in}}{\pgfqpoint{5.700000in}{5.700000in}}%
\pgfusepath{clip}%
\pgfsetbuttcap%
\pgfsetroundjoin%
\definecolor{currentfill}{rgb}{0.282327,0.094955,0.417331}%
\pgfsetfillcolor{currentfill}%
\pgfsetfillopacity{0.700000}%
\pgfsetlinewidth{0.000000pt}%
\definecolor{currentstroke}{rgb}{0.000000,0.000000,0.000000}%
\pgfsetstrokecolor{currentstroke}%
\pgfsetdash{}{0pt}%
\pgfpathmoveto{\pgfqpoint{4.206521in}{2.804900in}}%
\pgfpathlineto{\pgfqpoint{4.219791in}{2.801023in}}%
\pgfpathlineto{\pgfqpoint{4.233068in}{2.797266in}}%
\pgfpathlineto{\pgfqpoint{4.246351in}{2.793628in}}%
\pgfpathlineto{\pgfqpoint{4.259640in}{2.790109in}}%
\pgfpathlineto{\pgfqpoint{4.252047in}{2.781196in}}%
\pgfpathlineto{\pgfqpoint{4.244450in}{2.772303in}}%
\pgfpathlineto{\pgfqpoint{4.236847in}{2.763429in}}%
\pgfpathlineto{\pgfqpoint{4.229239in}{2.754572in}}%
\pgfpathlineto{\pgfqpoint{4.215939in}{2.758018in}}%
\pgfpathlineto{\pgfqpoint{4.202645in}{2.761582in}}%
\pgfpathlineto{\pgfqpoint{4.189357in}{2.765265in}}%
\pgfpathlineto{\pgfqpoint{4.176075in}{2.769068in}}%
\pgfpathlineto{\pgfqpoint{4.183694in}{2.777992in}}%
\pgfpathlineto{\pgfqpoint{4.191308in}{2.786938in}}%
\pgfpathlineto{\pgfqpoint{4.198917in}{2.795907in}}%
\pgfpathlineto{\pgfqpoint{4.206521in}{2.804900in}}%
\pgfpathclose%
\pgfusepath{fill}%
\end{pgfscope}%
\begin{pgfscope}%
\pgfpathrectangle{\pgfqpoint{1.150000in}{0.150000in}}{\pgfqpoint{5.700000in}{5.700000in}}%
\pgfusepath{clip}%
\pgfsetbuttcap%
\pgfsetroundjoin%
\definecolor{currentfill}{rgb}{0.281446,0.084320,0.407414}%
\pgfsetfillcolor{currentfill}%
\pgfsetfillopacity{0.700000}%
\pgfsetlinewidth{0.000000pt}%
\definecolor{currentstroke}{rgb}{0.000000,0.000000,0.000000}%
\pgfsetstrokecolor{currentstroke}%
\pgfsetdash{}{0pt}%
\pgfpathmoveto{\pgfqpoint{3.849969in}{2.798538in}}%
\pgfpathlineto{\pgfqpoint{3.863174in}{2.792309in}}%
\pgfpathlineto{\pgfqpoint{3.876383in}{2.786213in}}%
\pgfpathlineto{\pgfqpoint{3.889596in}{2.780249in}}%
\pgfpathlineto{\pgfqpoint{3.902813in}{2.774416in}}%
\pgfpathlineto{\pgfqpoint{3.895102in}{2.765541in}}%
\pgfpathlineto{\pgfqpoint{3.887386in}{2.756697in}}%
\pgfpathlineto{\pgfqpoint{3.879665in}{2.747883in}}%
\pgfpathlineto{\pgfqpoint{3.871939in}{2.739100in}}%
\pgfpathlineto{\pgfqpoint{3.858709in}{2.744913in}}%
\pgfpathlineto{\pgfqpoint{3.845484in}{2.750858in}}%
\pgfpathlineto{\pgfqpoint{3.832263in}{2.756934in}}%
\pgfpathlineto{\pgfqpoint{3.819047in}{2.763143in}}%
\pgfpathlineto{\pgfqpoint{3.826785in}{2.771940in}}%
\pgfpathlineto{\pgfqpoint{3.834518in}{2.780770in}}%
\pgfpathlineto{\pgfqpoint{3.842246in}{2.789637in}}%
\pgfpathlineto{\pgfqpoint{3.849969in}{2.798538in}}%
\pgfpathclose%
\pgfusepath{fill}%
\end{pgfscope}%
\begin{pgfscope}%
\pgfpathrectangle{\pgfqpoint{1.150000in}{0.150000in}}{\pgfqpoint{5.700000in}{5.700000in}}%
\pgfusepath{clip}%
\pgfsetbuttcap%
\pgfsetroundjoin%
\definecolor{currentfill}{rgb}{0.282327,0.094955,0.417331}%
\pgfsetfillcolor{currentfill}%
\pgfsetfillopacity{0.700000}%
\pgfsetlinewidth{0.000000pt}%
\definecolor{currentstroke}{rgb}{0.000000,0.000000,0.000000}%
\pgfsetstrokecolor{currentstroke}%
\pgfsetdash{}{0pt}%
\pgfpathmoveto{\pgfqpoint{3.713440in}{2.817693in}}%
\pgfpathlineto{\pgfqpoint{3.726629in}{2.810392in}}%
\pgfpathlineto{\pgfqpoint{3.739822in}{2.803231in}}%
\pgfpathlineto{\pgfqpoint{3.753017in}{2.796208in}}%
\pgfpathlineto{\pgfqpoint{3.766216in}{2.789323in}}%
\pgfpathlineto{\pgfqpoint{3.758459in}{2.780553in}}%
\pgfpathlineto{\pgfqpoint{3.750698in}{2.771821in}}%
\pgfpathlineto{\pgfqpoint{3.742930in}{2.763126in}}%
\pgfpathlineto{\pgfqpoint{3.735157in}{2.754469in}}%
\pgfpathlineto{\pgfqpoint{3.721946in}{2.761353in}}%
\pgfpathlineto{\pgfqpoint{3.708737in}{2.768374in}}%
\pgfpathlineto{\pgfqpoint{3.695532in}{2.775533in}}%
\pgfpathlineto{\pgfqpoint{3.682330in}{2.782833in}}%
\pgfpathlineto{\pgfqpoint{3.690116in}{2.791485in}}%
\pgfpathlineto{\pgfqpoint{3.697896in}{2.800178in}}%
\pgfpathlineto{\pgfqpoint{3.705671in}{2.808915in}}%
\pgfpathlineto{\pgfqpoint{3.713440in}{2.817693in}}%
\pgfpathclose%
\pgfusepath{fill}%
\end{pgfscope}%
\begin{pgfscope}%
\pgfpathrectangle{\pgfqpoint{1.150000in}{0.150000in}}{\pgfqpoint{5.700000in}{5.700000in}}%
\pgfusepath{clip}%
\pgfsetbuttcap%
\pgfsetroundjoin%
\definecolor{currentfill}{rgb}{0.283091,0.110553,0.431554}%
\pgfsetfillcolor{currentfill}%
\pgfsetfillopacity{0.700000}%
\pgfsetlinewidth{0.000000pt}%
\definecolor{currentstroke}{rgb}{0.000000,0.000000,0.000000}%
\pgfsetstrokecolor{currentstroke}%
\pgfsetdash{}{0pt}%
\pgfpathmoveto{\pgfqpoint{4.426575in}{2.836700in}}%
\pgfpathlineto{\pgfqpoint{4.439902in}{2.834020in}}%
\pgfpathlineto{\pgfqpoint{4.453236in}{2.831454in}}%
\pgfpathlineto{\pgfqpoint{4.466577in}{2.829001in}}%
\pgfpathlineto{\pgfqpoint{4.479926in}{2.826660in}}%
\pgfpathlineto{\pgfqpoint{4.472406in}{2.817892in}}%
\pgfpathlineto{\pgfqpoint{4.464881in}{2.809141in}}%
\pgfpathlineto{\pgfqpoint{4.457352in}{2.800406in}}%
\pgfpathlineto{\pgfqpoint{4.449817in}{2.791685in}}%
\pgfpathlineto{\pgfqpoint{4.436457in}{2.793916in}}%
\pgfpathlineto{\pgfqpoint{4.423105in}{2.796259in}}%
\pgfpathlineto{\pgfqpoint{4.409759in}{2.798715in}}%
\pgfpathlineto{\pgfqpoint{4.396421in}{2.801285in}}%
\pgfpathlineto{\pgfqpoint{4.403967in}{2.810109in}}%
\pgfpathlineto{\pgfqpoint{4.411508in}{2.818952in}}%
\pgfpathlineto{\pgfqpoint{4.419044in}{2.827815in}}%
\pgfpathlineto{\pgfqpoint{4.426575in}{2.836700in}}%
\pgfpathclose%
\pgfusepath{fill}%
\end{pgfscope}%
\begin{pgfscope}%
\pgfpathrectangle{\pgfqpoint{1.150000in}{0.150000in}}{\pgfqpoint{5.700000in}{5.700000in}}%
\pgfusepath{clip}%
\pgfsetbuttcap%
\pgfsetroundjoin%
\definecolor{currentfill}{rgb}{0.281887,0.150881,0.465405}%
\pgfsetfillcolor{currentfill}%
\pgfsetfillopacity{0.700000}%
\pgfsetlinewidth{0.000000pt}%
\definecolor{currentstroke}{rgb}{0.000000,0.000000,0.000000}%
\pgfsetstrokecolor{currentstroke}%
\pgfsetdash{}{0pt}%
\pgfpathmoveto{\pgfqpoint{4.730154in}{2.909129in}}%
\pgfpathlineto{\pgfqpoint{4.743569in}{2.907777in}}%
\pgfpathlineto{\pgfqpoint{4.756993in}{2.906531in}}%
\pgfpathlineto{\pgfqpoint{4.770425in}{2.905393in}}%
\pgfpathlineto{\pgfqpoint{4.783865in}{2.904360in}}%
\pgfpathlineto{\pgfqpoint{4.776451in}{2.895931in}}%
\pgfpathlineto{\pgfqpoint{4.769031in}{2.887524in}}%
\pgfpathlineto{\pgfqpoint{4.761607in}{2.879135in}}%
\pgfpathlineto{\pgfqpoint{4.754178in}{2.870764in}}%
\pgfpathlineto{\pgfqpoint{4.740725in}{2.871632in}}%
\pgfpathlineto{\pgfqpoint{4.727280in}{2.872607in}}%
\pgfpathlineto{\pgfqpoint{4.713844in}{2.873688in}}%
\pgfpathlineto{\pgfqpoint{4.700416in}{2.874877in}}%
\pgfpathlineto{\pgfqpoint{4.707858in}{2.883405in}}%
\pgfpathlineto{\pgfqpoint{4.715294in}{2.891955in}}%
\pgfpathlineto{\pgfqpoint{4.722726in}{2.900529in}}%
\pgfpathlineto{\pgfqpoint{4.730154in}{2.909129in}}%
\pgfpathclose%
\pgfusepath{fill}%
\end{pgfscope}%
\begin{pgfscope}%
\pgfpathrectangle{\pgfqpoint{1.150000in}{0.150000in}}{\pgfqpoint{5.700000in}{5.700000in}}%
\pgfusepath{clip}%
\pgfsetbuttcap%
\pgfsetroundjoin%
\definecolor{currentfill}{rgb}{0.281446,0.084320,0.407414}%
\pgfsetfillcolor{currentfill}%
\pgfsetfillopacity{0.700000}%
\pgfsetlinewidth{0.000000pt}%
\definecolor{currentstroke}{rgb}{0.000000,0.000000,0.000000}%
\pgfsetstrokecolor{currentstroke}%
\pgfsetdash{}{0pt}%
\pgfpathmoveto{\pgfqpoint{3.986469in}{2.788058in}}%
\pgfpathlineto{\pgfqpoint{3.999697in}{2.782832in}}%
\pgfpathlineto{\pgfqpoint{4.012930in}{2.777732in}}%
\pgfpathlineto{\pgfqpoint{4.026168in}{2.772759in}}%
\pgfpathlineto{\pgfqpoint{4.039411in}{2.767912in}}%
\pgfpathlineto{\pgfqpoint{4.031745in}{2.758993in}}%
\pgfpathlineto{\pgfqpoint{4.024073in}{2.750100in}}%
\pgfpathlineto{\pgfqpoint{4.016396in}{2.741231in}}%
\pgfpathlineto{\pgfqpoint{4.008714in}{2.732386in}}%
\pgfpathlineto{\pgfqpoint{3.995459in}{2.737196in}}%
\pgfpathlineto{\pgfqpoint{3.982210in}{2.742131in}}%
\pgfpathlineto{\pgfqpoint{3.968965in}{2.747192in}}%
\pgfpathlineto{\pgfqpoint{3.955726in}{2.752380in}}%
\pgfpathlineto{\pgfqpoint{3.963419in}{2.761257in}}%
\pgfpathlineto{\pgfqpoint{3.971108in}{2.770161in}}%
\pgfpathlineto{\pgfqpoint{3.978791in}{2.779095in}}%
\pgfpathlineto{\pgfqpoint{3.986469in}{2.788058in}}%
\pgfpathclose%
\pgfusepath{fill}%
\end{pgfscope}%
\begin{pgfscope}%
\pgfpathrectangle{\pgfqpoint{1.150000in}{0.150000in}}{\pgfqpoint{5.700000in}{5.700000in}}%
\pgfusepath{clip}%
\pgfsetbuttcap%
\pgfsetroundjoin%
\definecolor{currentfill}{rgb}{0.281887,0.150881,0.465405}%
\pgfsetfillcolor{currentfill}%
\pgfsetfillopacity{0.700000}%
\pgfsetlinewidth{0.000000pt}%
\definecolor{currentstroke}{rgb}{0.000000,0.000000,0.000000}%
\pgfsetstrokecolor{currentstroke}%
\pgfsetdash{}{0pt}%
\pgfpathmoveto{\pgfqpoint{3.387253in}{2.925792in}}%
\pgfpathlineto{\pgfqpoint{3.400436in}{2.915470in}}%
\pgfpathlineto{\pgfqpoint{3.413618in}{2.905310in}}%
\pgfpathlineto{\pgfqpoint{3.426802in}{2.895310in}}%
\pgfpathlineto{\pgfqpoint{3.439986in}{2.885468in}}%
\pgfpathlineto{\pgfqpoint{3.432115in}{2.877134in}}%
\pgfpathlineto{\pgfqpoint{3.424238in}{2.868860in}}%
\pgfpathlineto{\pgfqpoint{3.416355in}{2.860643in}}%
\pgfpathlineto{\pgfqpoint{3.408466in}{2.852486in}}%
\pgfpathlineto{\pgfqpoint{3.395267in}{2.862363in}}%
\pgfpathlineto{\pgfqpoint{3.382068in}{2.872400in}}%
\pgfpathlineto{\pgfqpoint{3.368870in}{2.882596in}}%
\pgfpathlineto{\pgfqpoint{3.355672in}{2.892953in}}%
\pgfpathlineto{\pgfqpoint{3.363577in}{2.901068in}}%
\pgfpathlineto{\pgfqpoint{3.371476in}{2.909246in}}%
\pgfpathlineto{\pgfqpoint{3.379368in}{2.917487in}}%
\pgfpathlineto{\pgfqpoint{3.387253in}{2.925792in}}%
\pgfpathclose%
\pgfusepath{fill}%
\end{pgfscope}%
\begin{pgfscope}%
\pgfpathrectangle{\pgfqpoint{1.150000in}{0.150000in}}{\pgfqpoint{5.700000in}{5.700000in}}%
\pgfusepath{clip}%
\pgfsetbuttcap%
\pgfsetroundjoin%
\definecolor{currentfill}{rgb}{0.283091,0.110553,0.431554}%
\pgfsetfillcolor{currentfill}%
\pgfsetfillopacity{0.700000}%
\pgfsetlinewidth{0.000000pt}%
\definecolor{currentstroke}{rgb}{0.000000,0.000000,0.000000}%
\pgfsetstrokecolor{currentstroke}%
\pgfsetdash{}{0pt}%
\pgfpathmoveto{\pgfqpoint{3.576807in}{2.846363in}}%
\pgfpathlineto{\pgfqpoint{3.589989in}{2.837913in}}%
\pgfpathlineto{\pgfqpoint{3.603174in}{2.829611in}}%
\pgfpathlineto{\pgfqpoint{3.616360in}{2.821455in}}%
\pgfpathlineto{\pgfqpoint{3.629549in}{2.813444in}}%
\pgfpathlineto{\pgfqpoint{3.621744in}{2.804843in}}%
\pgfpathlineto{\pgfqpoint{3.613934in}{2.796288in}}%
\pgfpathlineto{\pgfqpoint{3.606118in}{2.787779in}}%
\pgfpathlineto{\pgfqpoint{3.598296in}{2.779316in}}%
\pgfpathlineto{\pgfqpoint{3.585094in}{2.787343in}}%
\pgfpathlineto{\pgfqpoint{3.571893in}{2.795516in}}%
\pgfpathlineto{\pgfqpoint{3.558695in}{2.803835in}}%
\pgfpathlineto{\pgfqpoint{3.545499in}{2.812302in}}%
\pgfpathlineto{\pgfqpoint{3.553335in}{2.820742in}}%
\pgfpathlineto{\pgfqpoint{3.561165in}{2.829232in}}%
\pgfpathlineto{\pgfqpoint{3.568989in}{2.837772in}}%
\pgfpathlineto{\pgfqpoint{3.576807in}{2.846363in}}%
\pgfpathclose%
\pgfusepath{fill}%
\end{pgfscope}%
\begin{pgfscope}%
\pgfpathrectangle{\pgfqpoint{1.150000in}{0.150000in}}{\pgfqpoint{5.700000in}{5.700000in}}%
\pgfusepath{clip}%
\pgfsetbuttcap%
\pgfsetroundjoin%
\definecolor{currentfill}{rgb}{0.258965,0.251537,0.524736}%
\pgfsetfillcolor{currentfill}%
\pgfsetfillopacity{0.700000}%
\pgfsetlinewidth{0.000000pt}%
\definecolor{currentstroke}{rgb}{0.000000,0.000000,0.000000}%
\pgfsetstrokecolor{currentstroke}%
\pgfsetdash{}{0pt}%
\pgfpathmoveto{\pgfqpoint{3.091561in}{3.136115in}}%
\pgfpathlineto{\pgfqpoint{3.104784in}{3.122231in}}%
\pgfpathlineto{\pgfqpoint{3.118003in}{3.108538in}}%
\pgfpathlineto{\pgfqpoint{3.131220in}{3.095036in}}%
\pgfpathlineto{\pgfqpoint{3.144435in}{3.081721in}}%
\pgfpathlineto{\pgfqpoint{3.136457in}{3.073863in}}%
\pgfpathlineto{\pgfqpoint{3.128471in}{3.066085in}}%
\pgfpathlineto{\pgfqpoint{3.120479in}{3.058390in}}%
\pgfpathlineto{\pgfqpoint{3.112479in}{3.050776in}}%
\pgfpathlineto{\pgfqpoint{3.099246in}{3.064147in}}%
\pgfpathlineto{\pgfqpoint{3.086011in}{3.077706in}}%
\pgfpathlineto{\pgfqpoint{3.072773in}{3.091455in}}%
\pgfpathlineto{\pgfqpoint{3.059532in}{3.105396in}}%
\pgfpathlineto{\pgfqpoint{3.067550in}{3.112946in}}%
\pgfpathlineto{\pgfqpoint{3.075561in}{3.120583in}}%
\pgfpathlineto{\pgfqpoint{3.083565in}{3.128306in}}%
\pgfpathlineto{\pgfqpoint{3.091561in}{3.136115in}}%
\pgfpathclose%
\pgfusepath{fill}%
\end{pgfscope}%
\begin{pgfscope}%
\pgfpathrectangle{\pgfqpoint{1.150000in}{0.150000in}}{\pgfqpoint{5.700000in}{5.700000in}}%
\pgfusepath{clip}%
\pgfsetbuttcap%
\pgfsetroundjoin%
\definecolor{currentfill}{rgb}{0.266580,0.228262,0.514349}%
\pgfsetfillcolor{currentfill}%
\pgfsetfillopacity{0.700000}%
\pgfsetlinewidth{0.000000pt}%
\definecolor{currentstroke}{rgb}{0.000000,0.000000,0.000000}%
\pgfsetstrokecolor{currentstroke}%
\pgfsetdash{}{0pt}%
\pgfpathmoveto{\pgfqpoint{3.144435in}{3.081721in}}%
\pgfpathlineto{\pgfqpoint{3.157647in}{3.068593in}}%
\pgfpathlineto{\pgfqpoint{3.170858in}{3.055650in}}%
\pgfpathlineto{\pgfqpoint{3.184066in}{3.042889in}}%
\pgfpathlineto{\pgfqpoint{3.197273in}{3.030310in}}%
\pgfpathlineto{\pgfqpoint{3.189312in}{3.022402in}}%
\pgfpathlineto{\pgfqpoint{3.181344in}{3.014572in}}%
\pgfpathlineto{\pgfqpoint{3.173369in}{3.006818in}}%
\pgfpathlineto{\pgfqpoint{3.165388in}{2.999141in}}%
\pgfpathlineto{\pgfqpoint{3.152163in}{3.011776in}}%
\pgfpathlineto{\pgfqpoint{3.138937in}{3.024592in}}%
\pgfpathlineto{\pgfqpoint{3.125709in}{3.037592in}}%
\pgfpathlineto{\pgfqpoint{3.112479in}{3.050776in}}%
\pgfpathlineto{\pgfqpoint{3.120479in}{3.058390in}}%
\pgfpathlineto{\pgfqpoint{3.128471in}{3.066085in}}%
\pgfpathlineto{\pgfqpoint{3.136457in}{3.073863in}}%
\pgfpathlineto{\pgfqpoint{3.144435in}{3.081721in}}%
\pgfpathclose%
\pgfusepath{fill}%
\end{pgfscope}%
\begin{pgfscope}%
\pgfpathrectangle{\pgfqpoint{1.150000in}{0.150000in}}{\pgfqpoint{5.700000in}{5.700000in}}%
\pgfusepath{clip}%
\pgfsetbuttcap%
\pgfsetroundjoin%
\definecolor{currentfill}{rgb}{0.282623,0.140926,0.457517}%
\pgfsetfillcolor{currentfill}%
\pgfsetfillopacity{0.700000}%
\pgfsetlinewidth{0.000000pt}%
\definecolor{currentstroke}{rgb}{0.000000,0.000000,0.000000}%
\pgfsetstrokecolor{currentstroke}%
\pgfsetdash{}{0pt}%
\pgfpathmoveto{\pgfqpoint{4.646789in}{2.880709in}}%
\pgfpathlineto{\pgfqpoint{4.660183in}{2.879088in}}%
\pgfpathlineto{\pgfqpoint{4.673586in}{2.877576in}}%
\pgfpathlineto{\pgfqpoint{4.686997in}{2.876173in}}%
\pgfpathlineto{\pgfqpoint{4.700416in}{2.874877in}}%
\pgfpathlineto{\pgfqpoint{4.692970in}{2.866368in}}%
\pgfpathlineto{\pgfqpoint{4.685519in}{2.857876in}}%
\pgfpathlineto{\pgfqpoint{4.678063in}{2.849401in}}%
\pgfpathlineto{\pgfqpoint{4.670603in}{2.840940in}}%
\pgfpathlineto{\pgfqpoint{4.657172in}{2.842090in}}%
\pgfpathlineto{\pgfqpoint{4.643749in}{2.843348in}}%
\pgfpathlineto{\pgfqpoint{4.630334in}{2.844714in}}%
\pgfpathlineto{\pgfqpoint{4.616927in}{2.846188in}}%
\pgfpathlineto{\pgfqpoint{4.624400in}{2.854789in}}%
\pgfpathlineto{\pgfqpoint{4.631867in}{2.863408in}}%
\pgfpathlineto{\pgfqpoint{4.639330in}{2.872047in}}%
\pgfpathlineto{\pgfqpoint{4.646789in}{2.880709in}}%
\pgfpathclose%
\pgfusepath{fill}%
\end{pgfscope}%
\begin{pgfscope}%
\pgfpathrectangle{\pgfqpoint{1.150000in}{0.150000in}}{\pgfqpoint{5.700000in}{5.700000in}}%
\pgfusepath{clip}%
\pgfsetbuttcap%
\pgfsetroundjoin%
\definecolor{currentfill}{rgb}{0.281446,0.084320,0.407414}%
\pgfsetfillcolor{currentfill}%
\pgfsetfillopacity{0.700000}%
\pgfsetlinewidth{0.000000pt}%
\definecolor{currentstroke}{rgb}{0.000000,0.000000,0.000000}%
\pgfsetstrokecolor{currentstroke}%
\pgfsetdash{}{0pt}%
\pgfpathmoveto{\pgfqpoint{4.123007in}{2.785483in}}%
\pgfpathlineto{\pgfqpoint{4.136265in}{2.781197in}}%
\pgfpathlineto{\pgfqpoint{4.149529in}{2.777033in}}%
\pgfpathlineto{\pgfqpoint{4.162799in}{2.772990in}}%
\pgfpathlineto{\pgfqpoint{4.176075in}{2.769068in}}%
\pgfpathlineto{\pgfqpoint{4.168451in}{2.760165in}}%
\pgfpathlineto{\pgfqpoint{4.160823in}{2.751282in}}%
\pgfpathlineto{\pgfqpoint{4.153189in}{2.742418in}}%
\pgfpathlineto{\pgfqpoint{4.145550in}{2.733572in}}%
\pgfpathlineto{\pgfqpoint{4.132263in}{2.737438in}}%
\pgfpathlineto{\pgfqpoint{4.118981in}{2.741425in}}%
\pgfpathlineto{\pgfqpoint{4.105706in}{2.745533in}}%
\pgfpathlineto{\pgfqpoint{4.092436in}{2.749763in}}%
\pgfpathlineto{\pgfqpoint{4.100086in}{2.758658in}}%
\pgfpathlineto{\pgfqpoint{4.107732in}{2.767576in}}%
\pgfpathlineto{\pgfqpoint{4.115372in}{2.776517in}}%
\pgfpathlineto{\pgfqpoint{4.123007in}{2.785483in}}%
\pgfpathclose%
\pgfusepath{fill}%
\end{pgfscope}%
\begin{pgfscope}%
\pgfpathrectangle{\pgfqpoint{1.150000in}{0.150000in}}{\pgfqpoint{5.700000in}{5.700000in}}%
\pgfusepath{clip}%
\pgfsetbuttcap%
\pgfsetroundjoin%
\definecolor{currentfill}{rgb}{0.282656,0.100196,0.422160}%
\pgfsetfillcolor{currentfill}%
\pgfsetfillopacity{0.700000}%
\pgfsetlinewidth{0.000000pt}%
\definecolor{currentstroke}{rgb}{0.000000,0.000000,0.000000}%
\pgfsetstrokecolor{currentstroke}%
\pgfsetdash{}{0pt}%
\pgfpathmoveto{\pgfqpoint{4.343138in}{2.812707in}}%
\pgfpathlineto{\pgfqpoint{4.356448in}{2.809679in}}%
\pgfpathlineto{\pgfqpoint{4.369766in}{2.806766in}}%
\pgfpathlineto{\pgfqpoint{4.383090in}{2.803969in}}%
\pgfpathlineto{\pgfqpoint{4.396421in}{2.801285in}}%
\pgfpathlineto{\pgfqpoint{4.388870in}{2.792478in}}%
\pgfpathlineto{\pgfqpoint{4.381315in}{2.783687in}}%
\pgfpathlineto{\pgfqpoint{4.373754in}{2.774911in}}%
\pgfpathlineto{\pgfqpoint{4.366189in}{2.766148in}}%
\pgfpathlineto{\pgfqpoint{4.352846in}{2.768740in}}%
\pgfpathlineto{\pgfqpoint{4.339511in}{2.771445in}}%
\pgfpathlineto{\pgfqpoint{4.326182in}{2.774266in}}%
\pgfpathlineto{\pgfqpoint{4.312861in}{2.777202in}}%
\pgfpathlineto{\pgfqpoint{4.320437in}{2.786050in}}%
\pgfpathlineto{\pgfqpoint{4.328009in}{2.794916in}}%
\pgfpathlineto{\pgfqpoint{4.335576in}{2.803802in}}%
\pgfpathlineto{\pgfqpoint{4.343138in}{2.812707in}}%
\pgfpathclose%
\pgfusepath{fill}%
\end{pgfscope}%
\begin{pgfscope}%
\pgfpathrectangle{\pgfqpoint{1.150000in}{0.150000in}}{\pgfqpoint{5.700000in}{5.700000in}}%
\pgfusepath{clip}%
\pgfsetbuttcap%
\pgfsetroundjoin%
\definecolor{currentfill}{rgb}{0.282884,0.135920,0.453427}%
\pgfsetfillcolor{currentfill}%
\pgfsetfillopacity{0.700000}%
\pgfsetlinewidth{0.000000pt}%
\definecolor{currentstroke}{rgb}{0.000000,0.000000,0.000000}%
\pgfsetstrokecolor{currentstroke}%
\pgfsetdash{}{0pt}%
\pgfpathmoveto{\pgfqpoint{3.439986in}{2.885468in}}%
\pgfpathlineto{\pgfqpoint{3.453171in}{2.875783in}}%
\pgfpathlineto{\pgfqpoint{3.466356in}{2.866255in}}%
\pgfpathlineto{\pgfqpoint{3.479543in}{2.856882in}}%
\pgfpathlineto{\pgfqpoint{3.492732in}{2.847663in}}%
\pgfpathlineto{\pgfqpoint{3.484876in}{2.839301in}}%
\pgfpathlineto{\pgfqpoint{3.477013in}{2.830993in}}%
\pgfpathlineto{\pgfqpoint{3.469145in}{2.822739in}}%
\pgfpathlineto{\pgfqpoint{3.461271in}{2.814540in}}%
\pgfpathlineto{\pgfqpoint{3.448068in}{2.823794in}}%
\pgfpathlineto{\pgfqpoint{3.434866in}{2.833203in}}%
\pgfpathlineto{\pgfqpoint{3.421666in}{2.842766in}}%
\pgfpathlineto{\pgfqpoint{3.408466in}{2.852486in}}%
\pgfpathlineto{\pgfqpoint{3.416355in}{2.860643in}}%
\pgfpathlineto{\pgfqpoint{3.424238in}{2.868860in}}%
\pgfpathlineto{\pgfqpoint{3.432115in}{2.877134in}}%
\pgfpathlineto{\pgfqpoint{3.439986in}{2.885468in}}%
\pgfpathclose%
\pgfusepath{fill}%
\end{pgfscope}%
\begin{pgfscope}%
\pgfpathrectangle{\pgfqpoint{1.150000in}{0.150000in}}{\pgfqpoint{5.700000in}{5.700000in}}%
\pgfusepath{clip}%
\pgfsetbuttcap%
\pgfsetroundjoin%
\definecolor{currentfill}{rgb}{0.273006,0.204520,0.501721}%
\pgfsetfillcolor{currentfill}%
\pgfsetfillopacity{0.700000}%
\pgfsetlinewidth{0.000000pt}%
\definecolor{currentstroke}{rgb}{0.000000,0.000000,0.000000}%
\pgfsetstrokecolor{currentstroke}%
\pgfsetdash{}{0pt}%
\pgfpathmoveto{\pgfqpoint{3.197273in}{3.030310in}}%
\pgfpathlineto{\pgfqpoint{3.210478in}{3.017911in}}%
\pgfpathlineto{\pgfqpoint{3.223682in}{3.005690in}}%
\pgfpathlineto{\pgfqpoint{3.236884in}{2.993646in}}%
\pgfpathlineto{\pgfqpoint{3.250085in}{2.981776in}}%
\pgfpathlineto{\pgfqpoint{3.242141in}{2.973820in}}%
\pgfpathlineto{\pgfqpoint{3.234191in}{2.965936in}}%
\pgfpathlineto{\pgfqpoint{3.226233in}{2.958125in}}%
\pgfpathlineto{\pgfqpoint{3.218269in}{2.950386in}}%
\pgfpathlineto{\pgfqpoint{3.205051in}{2.962311in}}%
\pgfpathlineto{\pgfqpoint{3.191831in}{2.974410in}}%
\pgfpathlineto{\pgfqpoint{3.178610in}{2.986687in}}%
\pgfpathlineto{\pgfqpoint{3.165388in}{2.999141in}}%
\pgfpathlineto{\pgfqpoint{3.173369in}{3.006818in}}%
\pgfpathlineto{\pgfqpoint{3.181344in}{3.014572in}}%
\pgfpathlineto{\pgfqpoint{3.189312in}{3.022402in}}%
\pgfpathlineto{\pgfqpoint{3.197273in}{3.030310in}}%
\pgfpathclose%
\pgfusepath{fill}%
\end{pgfscope}%
\begin{pgfscope}%
\pgfpathrectangle{\pgfqpoint{1.150000in}{0.150000in}}{\pgfqpoint{5.700000in}{5.700000in}}%
\pgfusepath{clip}%
\pgfsetbuttcap%
\pgfsetroundjoin%
\definecolor{currentfill}{rgb}{0.281924,0.089666,0.412415}%
\pgfsetfillcolor{currentfill}%
\pgfsetfillopacity{0.700000}%
\pgfsetlinewidth{0.000000pt}%
\definecolor{currentstroke}{rgb}{0.000000,0.000000,0.000000}%
\pgfsetstrokecolor{currentstroke}%
\pgfsetdash{}{0pt}%
\pgfpathmoveto{\pgfqpoint{3.766216in}{2.789323in}}%
\pgfpathlineto{\pgfqpoint{3.779418in}{2.782575in}}%
\pgfpathlineto{\pgfqpoint{3.792624in}{2.775963in}}%
\pgfpathlineto{\pgfqpoint{3.805833in}{2.769486in}}%
\pgfpathlineto{\pgfqpoint{3.819047in}{2.763143in}}%
\pgfpathlineto{\pgfqpoint{3.811302in}{2.754381in}}%
\pgfpathlineto{\pgfqpoint{3.803553in}{2.745653in}}%
\pgfpathlineto{\pgfqpoint{3.795798in}{2.736958in}}%
\pgfpathlineto{\pgfqpoint{3.788038in}{2.728297in}}%
\pgfpathlineto{\pgfqpoint{3.774812in}{2.734637in}}%
\pgfpathlineto{\pgfqpoint{3.761591in}{2.741112in}}%
\pgfpathlineto{\pgfqpoint{3.748372in}{2.747723in}}%
\pgfpathlineto{\pgfqpoint{3.735157in}{2.754469in}}%
\pgfpathlineto{\pgfqpoint{3.742930in}{2.763126in}}%
\pgfpathlineto{\pgfqpoint{3.750698in}{2.771821in}}%
\pgfpathlineto{\pgfqpoint{3.758459in}{2.780553in}}%
\pgfpathlineto{\pgfqpoint{3.766216in}{2.789323in}}%
\pgfpathclose%
\pgfusepath{fill}%
\end{pgfscope}%
\begin{pgfscope}%
\pgfpathrectangle{\pgfqpoint{1.150000in}{0.150000in}}{\pgfqpoint{5.700000in}{5.700000in}}%
\pgfusepath{clip}%
\pgfsetbuttcap%
\pgfsetroundjoin%
\definecolor{currentfill}{rgb}{0.277134,0.185228,0.489898}%
\pgfsetfillcolor{currentfill}%
\pgfsetfillopacity{0.700000}%
\pgfsetlinewidth{0.000000pt}%
\definecolor{currentstroke}{rgb}{0.000000,0.000000,0.000000}%
\pgfsetstrokecolor{currentstroke}%
\pgfsetdash{}{0pt}%
\pgfpathmoveto{\pgfqpoint{4.950653in}{2.965308in}}%
\pgfpathlineto{\pgfqpoint{4.964147in}{2.964838in}}%
\pgfpathlineto{\pgfqpoint{4.977651in}{2.964471in}}%
\pgfpathlineto{\pgfqpoint{4.991164in}{2.964207in}}%
\pgfpathlineto{\pgfqpoint{5.004687in}{2.964045in}}%
\pgfpathlineto{\pgfqpoint{4.997349in}{2.955985in}}%
\pgfpathlineto{\pgfqpoint{4.990006in}{2.947951in}}%
\pgfpathlineto{\pgfqpoint{4.982659in}{2.939941in}}%
\pgfpathlineto{\pgfqpoint{4.975307in}{2.931953in}}%
\pgfpathlineto{\pgfqpoint{4.961771in}{2.931913in}}%
\pgfpathlineto{\pgfqpoint{4.948244in}{2.931977in}}%
\pgfpathlineto{\pgfqpoint{4.934726in}{2.932143in}}%
\pgfpathlineto{\pgfqpoint{4.921218in}{2.932414in}}%
\pgfpathlineto{\pgfqpoint{4.928584in}{2.940596in}}%
\pgfpathlineto{\pgfqpoint{4.935944in}{2.948804in}}%
\pgfpathlineto{\pgfqpoint{4.943301in}{2.957041in}}%
\pgfpathlineto{\pgfqpoint{4.950653in}{2.965308in}}%
\pgfpathclose%
\pgfusepath{fill}%
\end{pgfscope}%
\begin{pgfscope}%
\pgfpathrectangle{\pgfqpoint{1.150000in}{0.150000in}}{\pgfqpoint{5.700000in}{5.700000in}}%
\pgfusepath{clip}%
\pgfsetbuttcap%
\pgfsetroundjoin%
\definecolor{currentfill}{rgb}{0.280894,0.078907,0.402329}%
\pgfsetfillcolor{currentfill}%
\pgfsetfillopacity{0.700000}%
\pgfsetlinewidth{0.000000pt}%
\definecolor{currentstroke}{rgb}{0.000000,0.000000,0.000000}%
\pgfsetstrokecolor{currentstroke}%
\pgfsetdash{}{0pt}%
\pgfpathmoveto{\pgfqpoint{3.902813in}{2.774416in}}%
\pgfpathlineto{\pgfqpoint{3.916034in}{2.768713in}}%
\pgfpathlineto{\pgfqpoint{3.929260in}{2.763140in}}%
\pgfpathlineto{\pgfqpoint{3.942491in}{2.757696in}}%
\pgfpathlineto{\pgfqpoint{3.955726in}{2.752380in}}%
\pgfpathlineto{\pgfqpoint{3.948027in}{2.743532in}}%
\pgfpathlineto{\pgfqpoint{3.940323in}{2.734710in}}%
\pgfpathlineto{\pgfqpoint{3.932613in}{2.725914in}}%
\pgfpathlineto{\pgfqpoint{3.924899in}{2.717145in}}%
\pgfpathlineto{\pgfqpoint{3.911652in}{2.722441in}}%
\pgfpathlineto{\pgfqpoint{3.898410in}{2.727865in}}%
\pgfpathlineto{\pgfqpoint{3.885172in}{2.733417in}}%
\pgfpathlineto{\pgfqpoint{3.871939in}{2.739100in}}%
\pgfpathlineto{\pgfqpoint{3.879665in}{2.747883in}}%
\pgfpathlineto{\pgfqpoint{3.887386in}{2.756697in}}%
\pgfpathlineto{\pgfqpoint{3.895102in}{2.765541in}}%
\pgfpathlineto{\pgfqpoint{3.902813in}{2.774416in}}%
\pgfpathclose%
\pgfusepath{fill}%
\end{pgfscope}%
\begin{pgfscope}%
\pgfpathrectangle{\pgfqpoint{1.150000in}{0.150000in}}{\pgfqpoint{5.700000in}{5.700000in}}%
\pgfusepath{clip}%
\pgfsetbuttcap%
\pgfsetroundjoin%
\definecolor{currentfill}{rgb}{0.283187,0.125848,0.444960}%
\pgfsetfillcolor{currentfill}%
\pgfsetfillopacity{0.700000}%
\pgfsetlinewidth{0.000000pt}%
\definecolor{currentstroke}{rgb}{0.000000,0.000000,0.000000}%
\pgfsetstrokecolor{currentstroke}%
\pgfsetdash{}{0pt}%
\pgfpathmoveto{\pgfqpoint{4.563380in}{2.853179in}}%
\pgfpathlineto{\pgfqpoint{4.576755in}{2.851267in}}%
\pgfpathlineto{\pgfqpoint{4.590138in}{2.849464in}}%
\pgfpathlineto{\pgfqpoint{4.603529in}{2.847772in}}%
\pgfpathlineto{\pgfqpoint{4.616927in}{2.846188in}}%
\pgfpathlineto{\pgfqpoint{4.609450in}{2.837604in}}%
\pgfpathlineto{\pgfqpoint{4.601968in}{2.829035in}}%
\pgfpathlineto{\pgfqpoint{4.594481in}{2.820480in}}%
\pgfpathlineto{\pgfqpoint{4.586989in}{2.811936in}}%
\pgfpathlineto{\pgfqpoint{4.573578in}{2.813391in}}%
\pgfpathlineto{\pgfqpoint{4.560176in}{2.814955in}}%
\pgfpathlineto{\pgfqpoint{4.546782in}{2.816629in}}%
\pgfpathlineto{\pgfqpoint{4.533395in}{2.818414in}}%
\pgfpathlineto{\pgfqpoint{4.540899in}{2.827079in}}%
\pgfpathlineto{\pgfqpoint{4.548398in}{2.835761in}}%
\pgfpathlineto{\pgfqpoint{4.555891in}{2.844460in}}%
\pgfpathlineto{\pgfqpoint{4.563380in}{2.853179in}}%
\pgfpathclose%
\pgfusepath{fill}%
\end{pgfscope}%
\begin{pgfscope}%
\pgfpathrectangle{\pgfqpoint{1.150000in}{0.150000in}}{\pgfqpoint{5.700000in}{5.700000in}}%
\pgfusepath{clip}%
\pgfsetbuttcap%
\pgfsetroundjoin%
\definecolor{currentfill}{rgb}{0.282656,0.100196,0.422160}%
\pgfsetfillcolor{currentfill}%
\pgfsetfillopacity{0.700000}%
\pgfsetlinewidth{0.000000pt}%
\definecolor{currentstroke}{rgb}{0.000000,0.000000,0.000000}%
\pgfsetstrokecolor{currentstroke}%
\pgfsetdash{}{0pt}%
\pgfpathmoveto{\pgfqpoint{3.629549in}{2.813444in}}%
\pgfpathlineto{\pgfqpoint{3.642740in}{2.805577in}}%
\pgfpathlineto{\pgfqpoint{3.655934in}{2.797854in}}%
\pgfpathlineto{\pgfqpoint{3.669131in}{2.790272in}}%
\pgfpathlineto{\pgfqpoint{3.682330in}{2.782833in}}%
\pgfpathlineto{\pgfqpoint{3.674539in}{2.774222in}}%
\pgfpathlineto{\pgfqpoint{3.666742in}{2.765653in}}%
\pgfpathlineto{\pgfqpoint{3.658939in}{2.757126in}}%
\pgfpathlineto{\pgfqpoint{3.651131in}{2.748639in}}%
\pgfpathlineto{\pgfqpoint{3.637918in}{2.756095in}}%
\pgfpathlineto{\pgfqpoint{3.624708in}{2.763693in}}%
\pgfpathlineto{\pgfqpoint{3.611501in}{2.771433in}}%
\pgfpathlineto{\pgfqpoint{3.598296in}{2.779316in}}%
\pgfpathlineto{\pgfqpoint{3.606118in}{2.787779in}}%
\pgfpathlineto{\pgfqpoint{3.613934in}{2.796288in}}%
\pgfpathlineto{\pgfqpoint{3.621744in}{2.804843in}}%
\pgfpathlineto{\pgfqpoint{3.629549in}{2.813444in}}%
\pgfpathclose%
\pgfusepath{fill}%
\end{pgfscope}%
\begin{pgfscope}%
\pgfpathrectangle{\pgfqpoint{1.150000in}{0.150000in}}{\pgfqpoint{5.700000in}{5.700000in}}%
\pgfusepath{clip}%
\pgfsetbuttcap%
\pgfsetroundjoin%
\definecolor{currentfill}{rgb}{0.282327,0.094955,0.417331}%
\pgfsetfillcolor{currentfill}%
\pgfsetfillopacity{0.700000}%
\pgfsetlinewidth{0.000000pt}%
\definecolor{currentstroke}{rgb}{0.000000,0.000000,0.000000}%
\pgfsetstrokecolor{currentstroke}%
\pgfsetdash{}{0pt}%
\pgfpathmoveto{\pgfqpoint{4.259640in}{2.790109in}}%
\pgfpathlineto{\pgfqpoint{4.272935in}{2.786706in}}%
\pgfpathlineto{\pgfqpoint{4.286237in}{2.783422in}}%
\pgfpathlineto{\pgfqpoint{4.299546in}{2.780254in}}%
\pgfpathlineto{\pgfqpoint{4.312861in}{2.777202in}}%
\pgfpathlineto{\pgfqpoint{4.305279in}{2.768370in}}%
\pgfpathlineto{\pgfqpoint{4.297693in}{2.759554in}}%
\pgfpathlineto{\pgfqpoint{4.290101in}{2.750752in}}%
\pgfpathlineto{\pgfqpoint{4.282504in}{2.741963in}}%
\pgfpathlineto{\pgfqpoint{4.269178in}{2.744940in}}%
\pgfpathlineto{\pgfqpoint{4.255859in}{2.748034in}}%
\pgfpathlineto{\pgfqpoint{4.242546in}{2.751244in}}%
\pgfpathlineto{\pgfqpoint{4.229239in}{2.754572in}}%
\pgfpathlineto{\pgfqpoint{4.236847in}{2.763429in}}%
\pgfpathlineto{\pgfqpoint{4.244450in}{2.772303in}}%
\pgfpathlineto{\pgfqpoint{4.252047in}{2.781196in}}%
\pgfpathlineto{\pgfqpoint{4.259640in}{2.790109in}}%
\pgfpathclose%
\pgfusepath{fill}%
\end{pgfscope}%
\begin{pgfscope}%
\pgfpathrectangle{\pgfqpoint{1.150000in}{0.150000in}}{\pgfqpoint{5.700000in}{5.700000in}}%
\pgfusepath{clip}%
\pgfsetbuttcap%
\pgfsetroundjoin%
\definecolor{currentfill}{rgb}{0.277134,0.185228,0.489898}%
\pgfsetfillcolor{currentfill}%
\pgfsetfillopacity{0.700000}%
\pgfsetlinewidth{0.000000pt}%
\definecolor{currentstroke}{rgb}{0.000000,0.000000,0.000000}%
\pgfsetstrokecolor{currentstroke}%
\pgfsetdash{}{0pt}%
\pgfpathmoveto{\pgfqpoint{3.250085in}{2.981776in}}%
\pgfpathlineto{\pgfqpoint{3.263285in}{2.970081in}}%
\pgfpathlineto{\pgfqpoint{3.276485in}{2.958557in}}%
\pgfpathlineto{\pgfqpoint{3.289684in}{2.947205in}}%
\pgfpathlineto{\pgfqpoint{3.302882in}{2.936021in}}%
\pgfpathlineto{\pgfqpoint{3.294954in}{2.928017in}}%
\pgfpathlineto{\pgfqpoint{3.287020in}{2.920081in}}%
\pgfpathlineto{\pgfqpoint{3.279080in}{2.912213in}}%
\pgfpathlineto{\pgfqpoint{3.271133in}{2.904413in}}%
\pgfpathlineto{\pgfqpoint{3.257918in}{2.915651in}}%
\pgfpathlineto{\pgfqpoint{3.244702in}{2.927058in}}%
\pgfpathlineto{\pgfqpoint{3.231486in}{2.938636in}}%
\pgfpathlineto{\pgfqpoint{3.218269in}{2.950386in}}%
\pgfpathlineto{\pgfqpoint{3.226233in}{2.958125in}}%
\pgfpathlineto{\pgfqpoint{3.234191in}{2.965936in}}%
\pgfpathlineto{\pgfqpoint{3.242141in}{2.973820in}}%
\pgfpathlineto{\pgfqpoint{3.250085in}{2.981776in}}%
\pgfpathclose%
\pgfusepath{fill}%
\end{pgfscope}%
\begin{pgfscope}%
\pgfpathrectangle{\pgfqpoint{1.150000in}{0.150000in}}{\pgfqpoint{5.700000in}{5.700000in}}%
\pgfusepath{clip}%
\pgfsetbuttcap%
\pgfsetroundjoin%
\definecolor{currentfill}{rgb}{0.280894,0.078907,0.402329}%
\pgfsetfillcolor{currentfill}%
\pgfsetfillopacity{0.700000}%
\pgfsetlinewidth{0.000000pt}%
\definecolor{currentstroke}{rgb}{0.000000,0.000000,0.000000}%
\pgfsetstrokecolor{currentstroke}%
\pgfsetdash{}{0pt}%
\pgfpathmoveto{\pgfqpoint{4.039411in}{2.767912in}}%
\pgfpathlineto{\pgfqpoint{4.052660in}{2.763189in}}%
\pgfpathlineto{\pgfqpoint{4.065913in}{2.758590in}}%
\pgfpathlineto{\pgfqpoint{4.079172in}{2.754115in}}%
\pgfpathlineto{\pgfqpoint{4.092436in}{2.749763in}}%
\pgfpathlineto{\pgfqpoint{4.084781in}{2.740889in}}%
\pgfpathlineto{\pgfqpoint{4.077120in}{2.732036in}}%
\pgfpathlineto{\pgfqpoint{4.069455in}{2.723203in}}%
\pgfpathlineto{\pgfqpoint{4.061784in}{2.714390in}}%
\pgfpathlineto{\pgfqpoint{4.048509in}{2.718704in}}%
\pgfpathlineto{\pgfqpoint{4.035239in}{2.723141in}}%
\pgfpathlineto{\pgfqpoint{4.021974in}{2.727701in}}%
\pgfpathlineto{\pgfqpoint{4.008714in}{2.732386in}}%
\pgfpathlineto{\pgfqpoint{4.016396in}{2.741231in}}%
\pgfpathlineto{\pgfqpoint{4.024073in}{2.750100in}}%
\pgfpathlineto{\pgfqpoint{4.031745in}{2.758993in}}%
\pgfpathlineto{\pgfqpoint{4.039411in}{2.767912in}}%
\pgfpathclose%
\pgfusepath{fill}%
\end{pgfscope}%
\begin{pgfscope}%
\pgfpathrectangle{\pgfqpoint{1.150000in}{0.150000in}}{\pgfqpoint{5.700000in}{5.700000in}}%
\pgfusepath{clip}%
\pgfsetbuttcap%
\pgfsetroundjoin%
\definecolor{currentfill}{rgb}{0.279574,0.170599,0.479997}%
\pgfsetfillcolor{currentfill}%
\pgfsetfillopacity{0.700000}%
\pgfsetlinewidth{0.000000pt}%
\definecolor{currentstroke}{rgb}{0.000000,0.000000,0.000000}%
\pgfsetstrokecolor{currentstroke}%
\pgfsetdash{}{0pt}%
\pgfpathmoveto{\pgfqpoint{4.867277in}{2.934535in}}%
\pgfpathlineto{\pgfqpoint{4.880749in}{2.933848in}}%
\pgfpathlineto{\pgfqpoint{4.894229in}{2.933266in}}%
\pgfpathlineto{\pgfqpoint{4.907719in}{2.932788in}}%
\pgfpathlineto{\pgfqpoint{4.921218in}{2.932414in}}%
\pgfpathlineto{\pgfqpoint{4.913848in}{2.924255in}}%
\pgfpathlineto{\pgfqpoint{4.906473in}{2.916118in}}%
\pgfpathlineto{\pgfqpoint{4.899094in}{2.908000in}}%
\pgfpathlineto{\pgfqpoint{4.891710in}{2.899899in}}%
\pgfpathlineto{\pgfqpoint{4.878197in}{2.900090in}}%
\pgfpathlineto{\pgfqpoint{4.864694in}{2.900385in}}%
\pgfpathlineto{\pgfqpoint{4.851200in}{2.900785in}}%
\pgfpathlineto{\pgfqpoint{4.837716in}{2.901290in}}%
\pgfpathlineto{\pgfqpoint{4.845113in}{2.909567in}}%
\pgfpathlineto{\pgfqpoint{4.852506in}{2.917865in}}%
\pgfpathlineto{\pgfqpoint{4.859893in}{2.926187in}}%
\pgfpathlineto{\pgfqpoint{4.867277in}{2.934535in}}%
\pgfpathclose%
\pgfusepath{fill}%
\end{pgfscope}%
\begin{pgfscope}%
\pgfpathrectangle{\pgfqpoint{1.150000in}{0.150000in}}{\pgfqpoint{5.700000in}{5.700000in}}%
\pgfusepath{clip}%
\pgfsetbuttcap%
\pgfsetroundjoin%
\definecolor{currentfill}{rgb}{0.283229,0.120777,0.440584}%
\pgfsetfillcolor{currentfill}%
\pgfsetfillopacity{0.700000}%
\pgfsetlinewidth{0.000000pt}%
\definecolor{currentstroke}{rgb}{0.000000,0.000000,0.000000}%
\pgfsetstrokecolor{currentstroke}%
\pgfsetdash{}{0pt}%
\pgfpathmoveto{\pgfqpoint{3.492732in}{2.847663in}}%
\pgfpathlineto{\pgfqpoint{3.505921in}{2.838596in}}%
\pgfpathlineto{\pgfqpoint{3.519112in}{2.829681in}}%
\pgfpathlineto{\pgfqpoint{3.532305in}{2.820917in}}%
\pgfpathlineto{\pgfqpoint{3.545499in}{2.812302in}}%
\pgfpathlineto{\pgfqpoint{3.537657in}{2.803912in}}%
\pgfpathlineto{\pgfqpoint{3.529810in}{2.795571in}}%
\pgfpathlineto{\pgfqpoint{3.521956in}{2.787280in}}%
\pgfpathlineto{\pgfqpoint{3.514097in}{2.779039in}}%
\pgfpathlineto{\pgfqpoint{3.500888in}{2.787689in}}%
\pgfpathlineto{\pgfqpoint{3.487681in}{2.796488in}}%
\pgfpathlineto{\pgfqpoint{3.474475in}{2.805438in}}%
\pgfpathlineto{\pgfqpoint{3.461271in}{2.814540in}}%
\pgfpathlineto{\pgfqpoint{3.469145in}{2.822739in}}%
\pgfpathlineto{\pgfqpoint{3.477013in}{2.830993in}}%
\pgfpathlineto{\pgfqpoint{3.484876in}{2.839301in}}%
\pgfpathlineto{\pgfqpoint{3.492732in}{2.847663in}}%
\pgfpathclose%
\pgfusepath{fill}%
\end{pgfscope}%
\begin{pgfscope}%
\pgfpathrectangle{\pgfqpoint{1.150000in}{0.150000in}}{\pgfqpoint{5.700000in}{5.700000in}}%
\pgfusepath{clip}%
\pgfsetbuttcap%
\pgfsetroundjoin%
\definecolor{currentfill}{rgb}{0.283197,0.115680,0.436115}%
\pgfsetfillcolor{currentfill}%
\pgfsetfillopacity{0.700000}%
\pgfsetlinewidth{0.000000pt}%
\definecolor{currentstroke}{rgb}{0.000000,0.000000,0.000000}%
\pgfsetstrokecolor{currentstroke}%
\pgfsetdash{}{0pt}%
\pgfpathmoveto{\pgfqpoint{4.479926in}{2.826660in}}%
\pgfpathlineto{\pgfqpoint{4.493282in}{2.824431in}}%
\pgfpathlineto{\pgfqpoint{4.506646in}{2.822314in}}%
\pgfpathlineto{\pgfqpoint{4.520017in}{2.820308in}}%
\pgfpathlineto{\pgfqpoint{4.533395in}{2.818414in}}%
\pgfpathlineto{\pgfqpoint{4.525887in}{2.809762in}}%
\pgfpathlineto{\pgfqpoint{4.518374in}{2.801124in}}%
\pgfpathlineto{\pgfqpoint{4.510855in}{2.792497in}}%
\pgfpathlineto{\pgfqpoint{4.503332in}{2.783880in}}%
\pgfpathlineto{\pgfqpoint{4.489942in}{2.785665in}}%
\pgfpathlineto{\pgfqpoint{4.476560in}{2.787560in}}%
\pgfpathlineto{\pgfqpoint{4.463185in}{2.789567in}}%
\pgfpathlineto{\pgfqpoint{4.449817in}{2.791685in}}%
\pgfpathlineto{\pgfqpoint{4.457352in}{2.800406in}}%
\pgfpathlineto{\pgfqpoint{4.464881in}{2.809141in}}%
\pgfpathlineto{\pgfqpoint{4.472406in}{2.817892in}}%
\pgfpathlineto{\pgfqpoint{4.479926in}{2.826660in}}%
\pgfpathclose%
\pgfusepath{fill}%
\end{pgfscope}%
\begin{pgfscope}%
\pgfpathrectangle{\pgfqpoint{1.150000in}{0.150000in}}{\pgfqpoint{5.700000in}{5.700000in}}%
\pgfusepath{clip}%
\pgfsetbuttcap%
\pgfsetroundjoin%
\definecolor{currentfill}{rgb}{0.280255,0.165693,0.476498}%
\pgfsetfillcolor{currentfill}%
\pgfsetfillopacity{0.700000}%
\pgfsetlinewidth{0.000000pt}%
\definecolor{currentstroke}{rgb}{0.000000,0.000000,0.000000}%
\pgfsetstrokecolor{currentstroke}%
\pgfsetdash{}{0pt}%
\pgfpathmoveto{\pgfqpoint{3.302882in}{2.936021in}}%
\pgfpathlineto{\pgfqpoint{3.316080in}{2.925006in}}%
\pgfpathlineto{\pgfqpoint{3.329277in}{2.914157in}}%
\pgfpathlineto{\pgfqpoint{3.342475in}{2.903473in}}%
\pgfpathlineto{\pgfqpoint{3.355672in}{2.892953in}}%
\pgfpathlineto{\pgfqpoint{3.347761in}{2.884902in}}%
\pgfpathlineto{\pgfqpoint{3.339843in}{2.876913in}}%
\pgfpathlineto{\pgfqpoint{3.331919in}{2.868989in}}%
\pgfpathlineto{\pgfqpoint{3.323988in}{2.861128in}}%
\pgfpathlineto{\pgfqpoint{3.310775in}{2.871702in}}%
\pgfpathlineto{\pgfqpoint{3.297561in}{2.882440in}}%
\pgfpathlineto{\pgfqpoint{3.284347in}{2.893343in}}%
\pgfpathlineto{\pgfqpoint{3.271133in}{2.904413in}}%
\pgfpathlineto{\pgfqpoint{3.279080in}{2.912213in}}%
\pgfpathlineto{\pgfqpoint{3.287020in}{2.920081in}}%
\pgfpathlineto{\pgfqpoint{3.294954in}{2.928017in}}%
\pgfpathlineto{\pgfqpoint{3.302882in}{2.936021in}}%
\pgfpathclose%
\pgfusepath{fill}%
\end{pgfscope}%
\begin{pgfscope}%
\pgfpathrectangle{\pgfqpoint{1.150000in}{0.150000in}}{\pgfqpoint{5.700000in}{5.700000in}}%
\pgfusepath{clip}%
\pgfsetbuttcap%
\pgfsetroundjoin%
\definecolor{currentfill}{rgb}{0.280868,0.160771,0.472899}%
\pgfsetfillcolor{currentfill}%
\pgfsetfillopacity{0.700000}%
\pgfsetlinewidth{0.000000pt}%
\definecolor{currentstroke}{rgb}{0.000000,0.000000,0.000000}%
\pgfsetstrokecolor{currentstroke}%
\pgfsetdash{}{0pt}%
\pgfpathmoveto{\pgfqpoint{4.783865in}{2.904360in}}%
\pgfpathlineto{\pgfqpoint{4.797315in}{2.903434in}}%
\pgfpathlineto{\pgfqpoint{4.810773in}{2.902614in}}%
\pgfpathlineto{\pgfqpoint{4.824240in}{2.901899in}}%
\pgfpathlineto{\pgfqpoint{4.837716in}{2.901290in}}%
\pgfpathlineto{\pgfqpoint{4.830314in}{2.893032in}}%
\pgfpathlineto{\pgfqpoint{4.822907in}{2.884791in}}%
\pgfpathlineto{\pgfqpoint{4.815495in}{2.876565in}}%
\pgfpathlineto{\pgfqpoint{4.808079in}{2.868352in}}%
\pgfpathlineto{\pgfqpoint{4.794590in}{2.868797in}}%
\pgfpathlineto{\pgfqpoint{4.781110in}{2.869347in}}%
\pgfpathlineto{\pgfqpoint{4.767640in}{2.870002in}}%
\pgfpathlineto{\pgfqpoint{4.754178in}{2.870764in}}%
\pgfpathlineto{\pgfqpoint{4.761607in}{2.879135in}}%
\pgfpathlineto{\pgfqpoint{4.769031in}{2.887524in}}%
\pgfpathlineto{\pgfqpoint{4.776451in}{2.895931in}}%
\pgfpathlineto{\pgfqpoint{4.783865in}{2.904360in}}%
\pgfpathclose%
\pgfusepath{fill}%
\end{pgfscope}%
\begin{pgfscope}%
\pgfpathrectangle{\pgfqpoint{1.150000in}{0.150000in}}{\pgfqpoint{5.700000in}{5.700000in}}%
\pgfusepath{clip}%
\pgfsetbuttcap%
\pgfsetroundjoin%
\definecolor{currentfill}{rgb}{0.281446,0.084320,0.407414}%
\pgfsetfillcolor{currentfill}%
\pgfsetfillopacity{0.700000}%
\pgfsetlinewidth{0.000000pt}%
\definecolor{currentstroke}{rgb}{0.000000,0.000000,0.000000}%
\pgfsetstrokecolor{currentstroke}%
\pgfsetdash{}{0pt}%
\pgfpathmoveto{\pgfqpoint{4.176075in}{2.769068in}}%
\pgfpathlineto{\pgfqpoint{4.189357in}{2.765265in}}%
\pgfpathlineto{\pgfqpoint{4.202645in}{2.761582in}}%
\pgfpathlineto{\pgfqpoint{4.215939in}{2.758018in}}%
\pgfpathlineto{\pgfqpoint{4.229239in}{2.754572in}}%
\pgfpathlineto{\pgfqpoint{4.221627in}{2.745732in}}%
\pgfpathlineto{\pgfqpoint{4.214009in}{2.736908in}}%
\pgfpathlineto{\pgfqpoint{4.206386in}{2.728098in}}%
\pgfpathlineto{\pgfqpoint{4.198758in}{2.719302in}}%
\pgfpathlineto{\pgfqpoint{4.185447in}{2.722691in}}%
\pgfpathlineto{\pgfqpoint{4.172142in}{2.726199in}}%
\pgfpathlineto{\pgfqpoint{4.158843in}{2.729825in}}%
\pgfpathlineto{\pgfqpoint{4.145550in}{2.733572in}}%
\pgfpathlineto{\pgfqpoint{4.153189in}{2.742418in}}%
\pgfpathlineto{\pgfqpoint{4.160823in}{2.751282in}}%
\pgfpathlineto{\pgfqpoint{4.168451in}{2.760165in}}%
\pgfpathlineto{\pgfqpoint{4.176075in}{2.769068in}}%
\pgfpathclose%
\pgfusepath{fill}%
\end{pgfscope}%
\begin{pgfscope}%
\pgfpathrectangle{\pgfqpoint{1.150000in}{0.150000in}}{\pgfqpoint{5.700000in}{5.700000in}}%
\pgfusepath{clip}%
\pgfsetbuttcap%
\pgfsetroundjoin%
\definecolor{currentfill}{rgb}{0.280894,0.078907,0.402329}%
\pgfsetfillcolor{currentfill}%
\pgfsetfillopacity{0.700000}%
\pgfsetlinewidth{0.000000pt}%
\definecolor{currentstroke}{rgb}{0.000000,0.000000,0.000000}%
\pgfsetstrokecolor{currentstroke}%
\pgfsetdash{}{0pt}%
\pgfpathmoveto{\pgfqpoint{3.819047in}{2.763143in}}%
\pgfpathlineto{\pgfqpoint{3.832263in}{2.756934in}}%
\pgfpathlineto{\pgfqpoint{3.845484in}{2.750858in}}%
\pgfpathlineto{\pgfqpoint{3.858709in}{2.744913in}}%
\pgfpathlineto{\pgfqpoint{3.871939in}{2.739100in}}%
\pgfpathlineto{\pgfqpoint{3.864207in}{2.730347in}}%
\pgfpathlineto{\pgfqpoint{3.856470in}{2.721623in}}%
\pgfpathlineto{\pgfqpoint{3.848727in}{2.712928in}}%
\pgfpathlineto{\pgfqpoint{3.840979in}{2.704262in}}%
\pgfpathlineto{\pgfqpoint{3.827738in}{2.710073in}}%
\pgfpathlineto{\pgfqpoint{3.814501in}{2.716015in}}%
\pgfpathlineto{\pgfqpoint{3.801267in}{2.722090in}}%
\pgfpathlineto{\pgfqpoint{3.788038in}{2.728297in}}%
\pgfpathlineto{\pgfqpoint{3.795798in}{2.736958in}}%
\pgfpathlineto{\pgfqpoint{3.803553in}{2.745653in}}%
\pgfpathlineto{\pgfqpoint{3.811302in}{2.754381in}}%
\pgfpathlineto{\pgfqpoint{3.819047in}{2.763143in}}%
\pgfpathclose%
\pgfusepath{fill}%
\end{pgfscope}%
\begin{pgfscope}%
\pgfpathrectangle{\pgfqpoint{1.150000in}{0.150000in}}{\pgfqpoint{5.700000in}{5.700000in}}%
\pgfusepath{clip}%
\pgfsetbuttcap%
\pgfsetroundjoin%
\definecolor{currentfill}{rgb}{0.281924,0.089666,0.412415}%
\pgfsetfillcolor{currentfill}%
\pgfsetfillopacity{0.700000}%
\pgfsetlinewidth{0.000000pt}%
\definecolor{currentstroke}{rgb}{0.000000,0.000000,0.000000}%
\pgfsetstrokecolor{currentstroke}%
\pgfsetdash{}{0pt}%
\pgfpathmoveto{\pgfqpoint{3.682330in}{2.782833in}}%
\pgfpathlineto{\pgfqpoint{3.695532in}{2.775533in}}%
\pgfpathlineto{\pgfqpoint{3.708737in}{2.768374in}}%
\pgfpathlineto{\pgfqpoint{3.721946in}{2.761353in}}%
\pgfpathlineto{\pgfqpoint{3.735157in}{2.754469in}}%
\pgfpathlineto{\pgfqpoint{3.727379in}{2.745849in}}%
\pgfpathlineto{\pgfqpoint{3.719595in}{2.737267in}}%
\pgfpathlineto{\pgfqpoint{3.711805in}{2.728721in}}%
\pgfpathlineto{\pgfqpoint{3.704010in}{2.720212in}}%
\pgfpathlineto{\pgfqpoint{3.690786in}{2.727111in}}%
\pgfpathlineto{\pgfqpoint{3.677564in}{2.734148in}}%
\pgfpathlineto{\pgfqpoint{3.664346in}{2.741324in}}%
\pgfpathlineto{\pgfqpoint{3.651131in}{2.748639in}}%
\pgfpathlineto{\pgfqpoint{3.658939in}{2.757126in}}%
\pgfpathlineto{\pgfqpoint{3.666742in}{2.765653in}}%
\pgfpathlineto{\pgfqpoint{3.674539in}{2.774222in}}%
\pgfpathlineto{\pgfqpoint{3.682330in}{2.782833in}}%
\pgfpathclose%
\pgfusepath{fill}%
\end{pgfscope}%
\begin{pgfscope}%
\pgfpathrectangle{\pgfqpoint{1.150000in}{0.150000in}}{\pgfqpoint{5.700000in}{5.700000in}}%
\pgfusepath{clip}%
\pgfsetbuttcap%
\pgfsetroundjoin%
\definecolor{currentfill}{rgb}{0.282910,0.105393,0.426902}%
\pgfsetfillcolor{currentfill}%
\pgfsetfillopacity{0.700000}%
\pgfsetlinewidth{0.000000pt}%
\definecolor{currentstroke}{rgb}{0.000000,0.000000,0.000000}%
\pgfsetstrokecolor{currentstroke}%
\pgfsetdash{}{0pt}%
\pgfpathmoveto{\pgfqpoint{4.396421in}{2.801285in}}%
\pgfpathlineto{\pgfqpoint{4.409759in}{2.798715in}}%
\pgfpathlineto{\pgfqpoint{4.423105in}{2.796259in}}%
\pgfpathlineto{\pgfqpoint{4.436457in}{2.793916in}}%
\pgfpathlineto{\pgfqpoint{4.449817in}{2.791685in}}%
\pgfpathlineto{\pgfqpoint{4.442278in}{2.782977in}}%
\pgfpathlineto{\pgfqpoint{4.434733in}{2.774281in}}%
\pgfpathlineto{\pgfqpoint{4.427184in}{2.765595in}}%
\pgfpathlineto{\pgfqpoint{4.419630in}{2.756919in}}%
\pgfpathlineto{\pgfqpoint{4.406259in}{2.759057in}}%
\pgfpathlineto{\pgfqpoint{4.392895in}{2.761307in}}%
\pgfpathlineto{\pgfqpoint{4.379538in}{2.763671in}}%
\pgfpathlineto{\pgfqpoint{4.366189in}{2.766148in}}%
\pgfpathlineto{\pgfqpoint{4.373754in}{2.774911in}}%
\pgfpathlineto{\pgfqpoint{4.381315in}{2.783687in}}%
\pgfpathlineto{\pgfqpoint{4.388870in}{2.792478in}}%
\pgfpathlineto{\pgfqpoint{4.396421in}{2.801285in}}%
\pgfpathclose%
\pgfusepath{fill}%
\end{pgfscope}%
\begin{pgfscope}%
\pgfpathrectangle{\pgfqpoint{1.150000in}{0.150000in}}{\pgfqpoint{5.700000in}{5.700000in}}%
\pgfusepath{clip}%
\pgfsetbuttcap%
\pgfsetroundjoin%
\definecolor{currentfill}{rgb}{0.280894,0.078907,0.402329}%
\pgfsetfillcolor{currentfill}%
\pgfsetfillopacity{0.700000}%
\pgfsetlinewidth{0.000000pt}%
\definecolor{currentstroke}{rgb}{0.000000,0.000000,0.000000}%
\pgfsetstrokecolor{currentstroke}%
\pgfsetdash{}{0pt}%
\pgfpathmoveto{\pgfqpoint{3.955726in}{2.752380in}}%
\pgfpathlineto{\pgfqpoint{3.968965in}{2.747192in}}%
\pgfpathlineto{\pgfqpoint{3.982210in}{2.742131in}}%
\pgfpathlineto{\pgfqpoint{3.995459in}{2.737196in}}%
\pgfpathlineto{\pgfqpoint{4.008714in}{2.732386in}}%
\pgfpathlineto{\pgfqpoint{4.001027in}{2.723564in}}%
\pgfpathlineto{\pgfqpoint{3.993334in}{2.714765in}}%
\pgfpathlineto{\pgfqpoint{3.985637in}{2.705988in}}%
\pgfpathlineto{\pgfqpoint{3.977934in}{2.697232in}}%
\pgfpathlineto{\pgfqpoint{3.964668in}{2.702021in}}%
\pgfpathlineto{\pgfqpoint{3.951407in}{2.706936in}}%
\pgfpathlineto{\pgfqpoint{3.938150in}{2.711977in}}%
\pgfpathlineto{\pgfqpoint{3.924899in}{2.717145in}}%
\pgfpathlineto{\pgfqpoint{3.932613in}{2.725914in}}%
\pgfpathlineto{\pgfqpoint{3.940323in}{2.734710in}}%
\pgfpathlineto{\pgfqpoint{3.948027in}{2.743532in}}%
\pgfpathlineto{\pgfqpoint{3.955726in}{2.752380in}}%
\pgfpathclose%
\pgfusepath{fill}%
\end{pgfscope}%
\begin{pgfscope}%
\pgfpathrectangle{\pgfqpoint{1.150000in}{0.150000in}}{\pgfqpoint{5.700000in}{5.700000in}}%
\pgfusepath{clip}%
\pgfsetbuttcap%
\pgfsetroundjoin%
\definecolor{currentfill}{rgb}{0.282290,0.145912,0.461510}%
\pgfsetfillcolor{currentfill}%
\pgfsetfillopacity{0.700000}%
\pgfsetlinewidth{0.000000pt}%
\definecolor{currentstroke}{rgb}{0.000000,0.000000,0.000000}%
\pgfsetstrokecolor{currentstroke}%
\pgfsetdash{}{0pt}%
\pgfpathmoveto{\pgfqpoint{4.700416in}{2.874877in}}%
\pgfpathlineto{\pgfqpoint{4.713844in}{2.873688in}}%
\pgfpathlineto{\pgfqpoint{4.727280in}{2.872607in}}%
\pgfpathlineto{\pgfqpoint{4.740725in}{2.871632in}}%
\pgfpathlineto{\pgfqpoint{4.754178in}{2.870764in}}%
\pgfpathlineto{\pgfqpoint{4.746744in}{2.862408in}}%
\pgfpathlineto{\pgfqpoint{4.739305in}{2.854066in}}%
\pgfpathlineto{\pgfqpoint{4.731862in}{2.845736in}}%
\pgfpathlineto{\pgfqpoint{4.724413in}{2.837415in}}%
\pgfpathlineto{\pgfqpoint{4.710948in}{2.838136in}}%
\pgfpathlineto{\pgfqpoint{4.697491in}{2.838964in}}%
\pgfpathlineto{\pgfqpoint{4.684043in}{2.839898in}}%
\pgfpathlineto{\pgfqpoint{4.670603in}{2.840940in}}%
\pgfpathlineto{\pgfqpoint{4.678063in}{2.849401in}}%
\pgfpathlineto{\pgfqpoint{4.685519in}{2.857876in}}%
\pgfpathlineto{\pgfqpoint{4.692970in}{2.866368in}}%
\pgfpathlineto{\pgfqpoint{4.700416in}{2.874877in}}%
\pgfpathclose%
\pgfusepath{fill}%
\end{pgfscope}%
\begin{pgfscope}%
\pgfpathrectangle{\pgfqpoint{1.150000in}{0.150000in}}{\pgfqpoint{5.700000in}{5.700000in}}%
\pgfusepath{clip}%
\pgfsetbuttcap%
\pgfsetroundjoin%
\definecolor{currentfill}{rgb}{0.282290,0.145912,0.461510}%
\pgfsetfillcolor{currentfill}%
\pgfsetfillopacity{0.700000}%
\pgfsetlinewidth{0.000000pt}%
\definecolor{currentstroke}{rgb}{0.000000,0.000000,0.000000}%
\pgfsetstrokecolor{currentstroke}%
\pgfsetdash{}{0pt}%
\pgfpathmoveto{\pgfqpoint{3.355672in}{2.892953in}}%
\pgfpathlineto{\pgfqpoint{3.368870in}{2.882596in}}%
\pgfpathlineto{\pgfqpoint{3.382068in}{2.872400in}}%
\pgfpathlineto{\pgfqpoint{3.395267in}{2.862363in}}%
\pgfpathlineto{\pgfqpoint{3.408466in}{2.852486in}}%
\pgfpathlineto{\pgfqpoint{3.400570in}{2.844388in}}%
\pgfpathlineto{\pgfqpoint{3.392668in}{2.836348in}}%
\pgfpathlineto{\pgfqpoint{3.384760in}{2.828368in}}%
\pgfpathlineto{\pgfqpoint{3.376845in}{2.820447in}}%
\pgfpathlineto{\pgfqpoint{3.363630in}{2.830378in}}%
\pgfpathlineto{\pgfqpoint{3.350416in}{2.840467in}}%
\pgfpathlineto{\pgfqpoint{3.337202in}{2.850717in}}%
\pgfpathlineto{\pgfqpoint{3.323988in}{2.861128in}}%
\pgfpathlineto{\pgfqpoint{3.331919in}{2.868989in}}%
\pgfpathlineto{\pgfqpoint{3.339843in}{2.876913in}}%
\pgfpathlineto{\pgfqpoint{3.347761in}{2.884902in}}%
\pgfpathlineto{\pgfqpoint{3.355672in}{2.892953in}}%
\pgfpathclose%
\pgfusepath{fill}%
\end{pgfscope}%
\begin{pgfscope}%
\pgfpathrectangle{\pgfqpoint{1.150000in}{0.150000in}}{\pgfqpoint{5.700000in}{5.700000in}}%
\pgfusepath{clip}%
\pgfsetbuttcap%
\pgfsetroundjoin%
\definecolor{currentfill}{rgb}{0.282910,0.105393,0.426902}%
\pgfsetfillcolor{currentfill}%
\pgfsetfillopacity{0.700000}%
\pgfsetlinewidth{0.000000pt}%
\definecolor{currentstroke}{rgb}{0.000000,0.000000,0.000000}%
\pgfsetstrokecolor{currentstroke}%
\pgfsetdash{}{0pt}%
\pgfpathmoveto{\pgfqpoint{3.545499in}{2.812302in}}%
\pgfpathlineto{\pgfqpoint{3.558695in}{2.803835in}}%
\pgfpathlineto{\pgfqpoint{3.571893in}{2.795516in}}%
\pgfpathlineto{\pgfqpoint{3.585094in}{2.787343in}}%
\pgfpathlineto{\pgfqpoint{3.598296in}{2.779316in}}%
\pgfpathlineto{\pgfqpoint{3.590468in}{2.770898in}}%
\pgfpathlineto{\pgfqpoint{3.582635in}{2.762525in}}%
\pgfpathlineto{\pgfqpoint{3.574795in}{2.754197in}}%
\pgfpathlineto{\pgfqpoint{3.566950in}{2.745916in}}%
\pgfpathlineto{\pgfqpoint{3.553734in}{2.753977in}}%
\pgfpathlineto{\pgfqpoint{3.540519in}{2.762185in}}%
\pgfpathlineto{\pgfqpoint{3.527307in}{2.770538in}}%
\pgfpathlineto{\pgfqpoint{3.514097in}{2.779039in}}%
\pgfpathlineto{\pgfqpoint{3.521956in}{2.787280in}}%
\pgfpathlineto{\pgfqpoint{3.529810in}{2.795571in}}%
\pgfpathlineto{\pgfqpoint{3.537657in}{2.803912in}}%
\pgfpathlineto{\pgfqpoint{3.545499in}{2.812302in}}%
\pgfpathclose%
\pgfusepath{fill}%
\end{pgfscope}%
\begin{pgfscope}%
\pgfpathrectangle{\pgfqpoint{1.150000in}{0.150000in}}{\pgfqpoint{5.700000in}{5.700000in}}%
\pgfusepath{clip}%
\pgfsetbuttcap%
\pgfsetroundjoin%
\definecolor{currentfill}{rgb}{0.258965,0.251537,0.524736}%
\pgfsetfillcolor{currentfill}%
\pgfsetfillopacity{0.700000}%
\pgfsetlinewidth{0.000000pt}%
\definecolor{currentstroke}{rgb}{0.000000,0.000000,0.000000}%
\pgfsetstrokecolor{currentstroke}%
\pgfsetdash{}{0pt}%
\pgfpathmoveto{\pgfqpoint{3.059532in}{3.105396in}}%
\pgfpathlineto{\pgfqpoint{3.072773in}{3.091455in}}%
\pgfpathlineto{\pgfqpoint{3.086011in}{3.077706in}}%
\pgfpathlineto{\pgfqpoint{3.099246in}{3.064147in}}%
\pgfpathlineto{\pgfqpoint{3.112479in}{3.050776in}}%
\pgfpathlineto{\pgfqpoint{3.104472in}{3.043244in}}%
\pgfpathlineto{\pgfqpoint{3.096457in}{3.035795in}}%
\pgfpathlineto{\pgfqpoint{3.088435in}{3.028429in}}%
\pgfpathlineto{\pgfqpoint{3.080406in}{3.021146in}}%
\pgfpathlineto{\pgfqpoint{3.067154in}{3.034591in}}%
\pgfpathlineto{\pgfqpoint{3.053900in}{3.048225in}}%
\pgfpathlineto{\pgfqpoint{3.040643in}{3.062049in}}%
\pgfpathlineto{\pgfqpoint{3.027383in}{3.076064in}}%
\pgfpathlineto{\pgfqpoint{3.035431in}{3.083266in}}%
\pgfpathlineto{\pgfqpoint{3.043472in}{3.090555in}}%
\pgfpathlineto{\pgfqpoint{3.051506in}{3.097932in}}%
\pgfpathlineto{\pgfqpoint{3.059532in}{3.105396in}}%
\pgfpathclose%
\pgfusepath{fill}%
\end{pgfscope}%
\begin{pgfscope}%
\pgfpathrectangle{\pgfqpoint{1.150000in}{0.150000in}}{\pgfqpoint{5.700000in}{5.700000in}}%
\pgfusepath{clip}%
\pgfsetbuttcap%
\pgfsetroundjoin%
\definecolor{currentfill}{rgb}{0.280894,0.078907,0.402329}%
\pgfsetfillcolor{currentfill}%
\pgfsetfillopacity{0.700000}%
\pgfsetlinewidth{0.000000pt}%
\definecolor{currentstroke}{rgb}{0.000000,0.000000,0.000000}%
\pgfsetstrokecolor{currentstroke}%
\pgfsetdash{}{0pt}%
\pgfpathmoveto{\pgfqpoint{4.092436in}{2.749763in}}%
\pgfpathlineto{\pgfqpoint{4.105706in}{2.745533in}}%
\pgfpathlineto{\pgfqpoint{4.118981in}{2.741425in}}%
\pgfpathlineto{\pgfqpoint{4.132263in}{2.737438in}}%
\pgfpathlineto{\pgfqpoint{4.145550in}{2.733572in}}%
\pgfpathlineto{\pgfqpoint{4.137906in}{2.724743in}}%
\pgfpathlineto{\pgfqpoint{4.130257in}{2.715931in}}%
\pgfpathlineto{\pgfqpoint{4.122602in}{2.707135in}}%
\pgfpathlineto{\pgfqpoint{4.114943in}{2.698354in}}%
\pgfpathlineto{\pgfqpoint{4.101645in}{2.702181in}}%
\pgfpathlineto{\pgfqpoint{4.088352in}{2.706129in}}%
\pgfpathlineto{\pgfqpoint{4.075065in}{2.710199in}}%
\pgfpathlineto{\pgfqpoint{4.061784in}{2.714390in}}%
\pgfpathlineto{\pgfqpoint{4.069455in}{2.723203in}}%
\pgfpathlineto{\pgfqpoint{4.077120in}{2.732036in}}%
\pgfpathlineto{\pgfqpoint{4.084781in}{2.740889in}}%
\pgfpathlineto{\pgfqpoint{4.092436in}{2.749763in}}%
\pgfpathclose%
\pgfusepath{fill}%
\end{pgfscope}%
\begin{pgfscope}%
\pgfpathrectangle{\pgfqpoint{1.150000in}{0.150000in}}{\pgfqpoint{5.700000in}{5.700000in}}%
\pgfusepath{clip}%
\pgfsetbuttcap%
\pgfsetroundjoin%
\definecolor{currentfill}{rgb}{0.282884,0.135920,0.453427}%
\pgfsetfillcolor{currentfill}%
\pgfsetfillopacity{0.700000}%
\pgfsetlinewidth{0.000000pt}%
\definecolor{currentstroke}{rgb}{0.000000,0.000000,0.000000}%
\pgfsetstrokecolor{currentstroke}%
\pgfsetdash{}{0pt}%
\pgfpathmoveto{\pgfqpoint{4.616927in}{2.846188in}}%
\pgfpathlineto{\pgfqpoint{4.630334in}{2.844714in}}%
\pgfpathlineto{\pgfqpoint{4.643749in}{2.843348in}}%
\pgfpathlineto{\pgfqpoint{4.657172in}{2.842090in}}%
\pgfpathlineto{\pgfqpoint{4.670603in}{2.840940in}}%
\pgfpathlineto{\pgfqpoint{4.663137in}{2.832492in}}%
\pgfpathlineto{\pgfqpoint{4.655667in}{2.824054in}}%
\pgfpathlineto{\pgfqpoint{4.648192in}{2.815625in}}%
\pgfpathlineto{\pgfqpoint{4.640712in}{2.807203in}}%
\pgfpathlineto{\pgfqpoint{4.627268in}{2.808224in}}%
\pgfpathlineto{\pgfqpoint{4.613834in}{2.809353in}}%
\pgfpathlineto{\pgfqpoint{4.600407in}{2.810590in}}%
\pgfpathlineto{\pgfqpoint{4.586989in}{2.811936in}}%
\pgfpathlineto{\pgfqpoint{4.594481in}{2.820480in}}%
\pgfpathlineto{\pgfqpoint{4.601968in}{2.829035in}}%
\pgfpathlineto{\pgfqpoint{4.609450in}{2.837604in}}%
\pgfpathlineto{\pgfqpoint{4.616927in}{2.846188in}}%
\pgfpathclose%
\pgfusepath{fill}%
\end{pgfscope}%
\begin{pgfscope}%
\pgfpathrectangle{\pgfqpoint{1.150000in}{0.150000in}}{\pgfqpoint{5.700000in}{5.700000in}}%
\pgfusepath{clip}%
\pgfsetbuttcap%
\pgfsetroundjoin%
\definecolor{currentfill}{rgb}{0.282327,0.094955,0.417331}%
\pgfsetfillcolor{currentfill}%
\pgfsetfillopacity{0.700000}%
\pgfsetlinewidth{0.000000pt}%
\definecolor{currentstroke}{rgb}{0.000000,0.000000,0.000000}%
\pgfsetstrokecolor{currentstroke}%
\pgfsetdash{}{0pt}%
\pgfpathmoveto{\pgfqpoint{4.312861in}{2.777202in}}%
\pgfpathlineto{\pgfqpoint{4.326182in}{2.774266in}}%
\pgfpathlineto{\pgfqpoint{4.339511in}{2.771445in}}%
\pgfpathlineto{\pgfqpoint{4.352846in}{2.768740in}}%
\pgfpathlineto{\pgfqpoint{4.366189in}{2.766148in}}%
\pgfpathlineto{\pgfqpoint{4.358618in}{2.757398in}}%
\pgfpathlineto{\pgfqpoint{4.351043in}{2.748658in}}%
\pgfpathlineto{\pgfqpoint{4.343462in}{2.739929in}}%
\pgfpathlineto{\pgfqpoint{4.335877in}{2.731208in}}%
\pgfpathlineto{\pgfqpoint{4.322523in}{2.733724in}}%
\pgfpathlineto{\pgfqpoint{4.309177in}{2.736355in}}%
\pgfpathlineto{\pgfqpoint{4.295837in}{2.739101in}}%
\pgfpathlineto{\pgfqpoint{4.282504in}{2.741963in}}%
\pgfpathlineto{\pgfqpoint{4.290101in}{2.750752in}}%
\pgfpathlineto{\pgfqpoint{4.297693in}{2.759554in}}%
\pgfpathlineto{\pgfqpoint{4.305279in}{2.768370in}}%
\pgfpathlineto{\pgfqpoint{4.312861in}{2.777202in}}%
\pgfpathclose%
\pgfusepath{fill}%
\end{pgfscope}%
\begin{pgfscope}%
\pgfpathrectangle{\pgfqpoint{1.150000in}{0.150000in}}{\pgfqpoint{5.700000in}{5.700000in}}%
\pgfusepath{clip}%
\pgfsetbuttcap%
\pgfsetroundjoin%
\definecolor{currentfill}{rgb}{0.267968,0.223549,0.512008}%
\pgfsetfillcolor{currentfill}%
\pgfsetfillopacity{0.700000}%
\pgfsetlinewidth{0.000000pt}%
\definecolor{currentstroke}{rgb}{0.000000,0.000000,0.000000}%
\pgfsetstrokecolor{currentstroke}%
\pgfsetdash{}{0pt}%
\pgfpathmoveto{\pgfqpoint{3.112479in}{3.050776in}}%
\pgfpathlineto{\pgfqpoint{3.125709in}{3.037592in}}%
\pgfpathlineto{\pgfqpoint{3.138937in}{3.024592in}}%
\pgfpathlineto{\pgfqpoint{3.152163in}{3.011776in}}%
\pgfpathlineto{\pgfqpoint{3.165388in}{2.999141in}}%
\pgfpathlineto{\pgfqpoint{3.157399in}{2.991542in}}%
\pgfpathlineto{\pgfqpoint{3.149403in}{2.984022in}}%
\pgfpathlineto{\pgfqpoint{3.141399in}{2.976579in}}%
\pgfpathlineto{\pgfqpoint{3.133389in}{2.969216in}}%
\pgfpathlineto{\pgfqpoint{3.120146in}{2.981924in}}%
\pgfpathlineto{\pgfqpoint{3.106902in}{2.994814in}}%
\pgfpathlineto{\pgfqpoint{3.093655in}{3.007888in}}%
\pgfpathlineto{\pgfqpoint{3.080406in}{3.021146in}}%
\pgfpathlineto{\pgfqpoint{3.088435in}{3.028429in}}%
\pgfpathlineto{\pgfqpoint{3.096457in}{3.035795in}}%
\pgfpathlineto{\pgfqpoint{3.104472in}{3.043244in}}%
\pgfpathlineto{\pgfqpoint{3.112479in}{3.050776in}}%
\pgfpathclose%
\pgfusepath{fill}%
\end{pgfscope}%
\begin{pgfscope}%
\pgfpathrectangle{\pgfqpoint{1.150000in}{0.150000in}}{\pgfqpoint{5.700000in}{5.700000in}}%
\pgfusepath{clip}%
\pgfsetbuttcap%
\pgfsetroundjoin%
\definecolor{currentfill}{rgb}{0.275191,0.194905,0.496005}%
\pgfsetfillcolor{currentfill}%
\pgfsetfillopacity{0.700000}%
\pgfsetlinewidth{0.000000pt}%
\definecolor{currentstroke}{rgb}{0.000000,0.000000,0.000000}%
\pgfsetstrokecolor{currentstroke}%
\pgfsetdash{}{0pt}%
\pgfpathmoveto{\pgfqpoint{5.004687in}{2.964045in}}%
\pgfpathlineto{\pgfqpoint{5.018219in}{2.963987in}}%
\pgfpathlineto{\pgfqpoint{5.031760in}{2.964031in}}%
\pgfpathlineto{\pgfqpoint{5.045312in}{2.964177in}}%
\pgfpathlineto{\pgfqpoint{5.058873in}{2.964425in}}%
\pgfpathlineto{\pgfqpoint{5.051549in}{2.956572in}}%
\pgfpathlineto{\pgfqpoint{5.044221in}{2.948742in}}%
\pgfpathlineto{\pgfqpoint{5.036888in}{2.940931in}}%
\pgfpathlineto{\pgfqpoint{5.029550in}{2.933138in}}%
\pgfpathlineto{\pgfqpoint{5.015975in}{2.932688in}}%
\pgfpathlineto{\pgfqpoint{5.002409in}{2.932341in}}%
\pgfpathlineto{\pgfqpoint{4.988854in}{2.932095in}}%
\pgfpathlineto{\pgfqpoint{4.975307in}{2.931953in}}%
\pgfpathlineto{\pgfqpoint{4.982659in}{2.939941in}}%
\pgfpathlineto{\pgfqpoint{4.990006in}{2.947951in}}%
\pgfpathlineto{\pgfqpoint{4.997349in}{2.955985in}}%
\pgfpathlineto{\pgfqpoint{5.004687in}{2.964045in}}%
\pgfpathclose%
\pgfusepath{fill}%
\end{pgfscope}%
\begin{pgfscope}%
\pgfpathrectangle{\pgfqpoint{1.150000in}{0.150000in}}{\pgfqpoint{5.700000in}{5.700000in}}%
\pgfusepath{clip}%
\pgfsetbuttcap%
\pgfsetroundjoin%
\definecolor{currentfill}{rgb}{0.283072,0.130895,0.449241}%
\pgfsetfillcolor{currentfill}%
\pgfsetfillopacity{0.700000}%
\pgfsetlinewidth{0.000000pt}%
\definecolor{currentstroke}{rgb}{0.000000,0.000000,0.000000}%
\pgfsetstrokecolor{currentstroke}%
\pgfsetdash{}{0pt}%
\pgfpathmoveto{\pgfqpoint{3.408466in}{2.852486in}}%
\pgfpathlineto{\pgfqpoint{3.421666in}{2.842766in}}%
\pgfpathlineto{\pgfqpoint{3.434866in}{2.833203in}}%
\pgfpathlineto{\pgfqpoint{3.448068in}{2.823794in}}%
\pgfpathlineto{\pgfqpoint{3.461271in}{2.814540in}}%
\pgfpathlineto{\pgfqpoint{3.453391in}{2.806395in}}%
\pgfpathlineto{\pgfqpoint{3.445504in}{2.798304in}}%
\pgfpathlineto{\pgfqpoint{3.437611in}{2.790269in}}%
\pgfpathlineto{\pgfqpoint{3.429712in}{2.782288in}}%
\pgfpathlineto{\pgfqpoint{3.416494in}{2.791596in}}%
\pgfpathlineto{\pgfqpoint{3.403277in}{2.801057in}}%
\pgfpathlineto{\pgfqpoint{3.390061in}{2.810674in}}%
\pgfpathlineto{\pgfqpoint{3.376845in}{2.820447in}}%
\pgfpathlineto{\pgfqpoint{3.384760in}{2.828368in}}%
\pgfpathlineto{\pgfqpoint{3.392668in}{2.836348in}}%
\pgfpathlineto{\pgfqpoint{3.400570in}{2.844388in}}%
\pgfpathlineto{\pgfqpoint{3.408466in}{2.852486in}}%
\pgfpathclose%
\pgfusepath{fill}%
\end{pgfscope}%
\begin{pgfscope}%
\pgfpathrectangle{\pgfqpoint{1.150000in}{0.150000in}}{\pgfqpoint{5.700000in}{5.700000in}}%
\pgfusepath{clip}%
\pgfsetbuttcap%
\pgfsetroundjoin%
\definecolor{currentfill}{rgb}{0.281446,0.084320,0.407414}%
\pgfsetfillcolor{currentfill}%
\pgfsetfillopacity{0.700000}%
\pgfsetlinewidth{0.000000pt}%
\definecolor{currentstroke}{rgb}{0.000000,0.000000,0.000000}%
\pgfsetstrokecolor{currentstroke}%
\pgfsetdash{}{0pt}%
\pgfpathmoveto{\pgfqpoint{3.735157in}{2.754469in}}%
\pgfpathlineto{\pgfqpoint{3.748372in}{2.747723in}}%
\pgfpathlineto{\pgfqpoint{3.761591in}{2.741112in}}%
\pgfpathlineto{\pgfqpoint{3.774812in}{2.734637in}}%
\pgfpathlineto{\pgfqpoint{3.788038in}{2.728297in}}%
\pgfpathlineto{\pgfqpoint{3.780272in}{2.719668in}}%
\pgfpathlineto{\pgfqpoint{3.772501in}{2.711072in}}%
\pgfpathlineto{\pgfqpoint{3.764724in}{2.702508in}}%
\pgfpathlineto{\pgfqpoint{3.756942in}{2.693976in}}%
\pgfpathlineto{\pgfqpoint{3.743704in}{2.700332in}}%
\pgfpathlineto{\pgfqpoint{3.730469in}{2.706823in}}%
\pgfpathlineto{\pgfqpoint{3.717238in}{2.713449in}}%
\pgfpathlineto{\pgfqpoint{3.704010in}{2.720212in}}%
\pgfpathlineto{\pgfqpoint{3.711805in}{2.728721in}}%
\pgfpathlineto{\pgfqpoint{3.719595in}{2.737267in}}%
\pgfpathlineto{\pgfqpoint{3.727379in}{2.745849in}}%
\pgfpathlineto{\pgfqpoint{3.735157in}{2.754469in}}%
\pgfpathclose%
\pgfusepath{fill}%
\end{pgfscope}%
\begin{pgfscope}%
\pgfpathrectangle{\pgfqpoint{1.150000in}{0.150000in}}{\pgfqpoint{5.700000in}{5.700000in}}%
\pgfusepath{clip}%
\pgfsetbuttcap%
\pgfsetroundjoin%
\definecolor{currentfill}{rgb}{0.273006,0.204520,0.501721}%
\pgfsetfillcolor{currentfill}%
\pgfsetfillopacity{0.700000}%
\pgfsetlinewidth{0.000000pt}%
\definecolor{currentstroke}{rgb}{0.000000,0.000000,0.000000}%
\pgfsetstrokecolor{currentstroke}%
\pgfsetdash{}{0pt}%
\pgfpathmoveto{\pgfqpoint{3.165388in}{2.999141in}}%
\pgfpathlineto{\pgfqpoint{3.178610in}{2.986687in}}%
\pgfpathlineto{\pgfqpoint{3.191831in}{2.974410in}}%
\pgfpathlineto{\pgfqpoint{3.205051in}{2.962311in}}%
\pgfpathlineto{\pgfqpoint{3.218269in}{2.950386in}}%
\pgfpathlineto{\pgfqpoint{3.210298in}{2.942721in}}%
\pgfpathlineto{\pgfqpoint{3.202320in}{2.935129in}}%
\pgfpathlineto{\pgfqpoint{3.194334in}{2.927612in}}%
\pgfpathlineto{\pgfqpoint{3.186342in}{2.920168in}}%
\pgfpathlineto{\pgfqpoint{3.173106in}{2.932165in}}%
\pgfpathlineto{\pgfqpoint{3.159869in}{2.944338in}}%
\pgfpathlineto{\pgfqpoint{3.146630in}{2.956688in}}%
\pgfpathlineto{\pgfqpoint{3.133389in}{2.969216in}}%
\pgfpathlineto{\pgfqpoint{3.141399in}{2.976579in}}%
\pgfpathlineto{\pgfqpoint{3.149403in}{2.984022in}}%
\pgfpathlineto{\pgfqpoint{3.157399in}{2.991542in}}%
\pgfpathlineto{\pgfqpoint{3.165388in}{2.999141in}}%
\pgfpathclose%
\pgfusepath{fill}%
\end{pgfscope}%
\begin{pgfscope}%
\pgfpathrectangle{\pgfqpoint{1.150000in}{0.150000in}}{\pgfqpoint{5.700000in}{5.700000in}}%
\pgfusepath{clip}%
\pgfsetbuttcap%
\pgfsetroundjoin%
\definecolor{currentfill}{rgb}{0.280267,0.073417,0.397163}%
\pgfsetfillcolor{currentfill}%
\pgfsetfillopacity{0.700000}%
\pgfsetlinewidth{0.000000pt}%
\definecolor{currentstroke}{rgb}{0.000000,0.000000,0.000000}%
\pgfsetstrokecolor{currentstroke}%
\pgfsetdash{}{0pt}%
\pgfpathmoveto{\pgfqpoint{3.871939in}{2.739100in}}%
\pgfpathlineto{\pgfqpoint{3.885172in}{2.733417in}}%
\pgfpathlineto{\pgfqpoint{3.898410in}{2.727865in}}%
\pgfpathlineto{\pgfqpoint{3.911652in}{2.722441in}}%
\pgfpathlineto{\pgfqpoint{3.924899in}{2.717145in}}%
\pgfpathlineto{\pgfqpoint{3.917179in}{2.708401in}}%
\pgfpathlineto{\pgfqpoint{3.909454in}{2.699682in}}%
\pgfpathlineto{\pgfqpoint{3.901723in}{2.690988in}}%
\pgfpathlineto{\pgfqpoint{3.893988in}{2.682318in}}%
\pgfpathlineto{\pgfqpoint{3.880729in}{2.687610in}}%
\pgfpathlineto{\pgfqpoint{3.867474in}{2.693032in}}%
\pgfpathlineto{\pgfqpoint{3.854225in}{2.698582in}}%
\pgfpathlineto{\pgfqpoint{3.840979in}{2.704262in}}%
\pgfpathlineto{\pgfqpoint{3.848727in}{2.712928in}}%
\pgfpathlineto{\pgfqpoint{3.856470in}{2.721623in}}%
\pgfpathlineto{\pgfqpoint{3.864207in}{2.730347in}}%
\pgfpathlineto{\pgfqpoint{3.871939in}{2.739100in}}%
\pgfpathclose%
\pgfusepath{fill}%
\end{pgfscope}%
\begin{pgfscope}%
\pgfpathrectangle{\pgfqpoint{1.150000in}{0.150000in}}{\pgfqpoint{5.700000in}{5.700000in}}%
\pgfusepath{clip}%
\pgfsetbuttcap%
\pgfsetroundjoin%
\definecolor{currentfill}{rgb}{0.278012,0.180367,0.486697}%
\pgfsetfillcolor{currentfill}%
\pgfsetfillopacity{0.700000}%
\pgfsetlinewidth{0.000000pt}%
\definecolor{currentstroke}{rgb}{0.000000,0.000000,0.000000}%
\pgfsetstrokecolor{currentstroke}%
\pgfsetdash{}{0pt}%
\pgfpathmoveto{\pgfqpoint{4.921218in}{2.932414in}}%
\pgfpathlineto{\pgfqpoint{4.934726in}{2.932143in}}%
\pgfpathlineto{\pgfqpoint{4.948244in}{2.931977in}}%
\pgfpathlineto{\pgfqpoint{4.961771in}{2.931913in}}%
\pgfpathlineto{\pgfqpoint{4.975307in}{2.931953in}}%
\pgfpathlineto{\pgfqpoint{4.967951in}{2.923984in}}%
\pgfpathlineto{\pgfqpoint{4.960590in}{2.916033in}}%
\pgfpathlineto{\pgfqpoint{4.953224in}{2.908096in}}%
\pgfpathlineto{\pgfqpoint{4.945853in}{2.900172in}}%
\pgfpathlineto{\pgfqpoint{4.932303in}{2.899948in}}%
\pgfpathlineto{\pgfqpoint{4.918762in}{2.899828in}}%
\pgfpathlineto{\pgfqpoint{4.905231in}{2.899812in}}%
\pgfpathlineto{\pgfqpoint{4.891710in}{2.899899in}}%
\pgfpathlineto{\pgfqpoint{4.899094in}{2.908000in}}%
\pgfpathlineto{\pgfqpoint{4.906473in}{2.916118in}}%
\pgfpathlineto{\pgfqpoint{4.913848in}{2.924255in}}%
\pgfpathlineto{\pgfqpoint{4.921218in}{2.932414in}}%
\pgfpathclose%
\pgfusepath{fill}%
\end{pgfscope}%
\begin{pgfscope}%
\pgfpathrectangle{\pgfqpoint{1.150000in}{0.150000in}}{\pgfqpoint{5.700000in}{5.700000in}}%
\pgfusepath{clip}%
\pgfsetbuttcap%
\pgfsetroundjoin%
\definecolor{currentfill}{rgb}{0.283229,0.120777,0.440584}%
\pgfsetfillcolor{currentfill}%
\pgfsetfillopacity{0.700000}%
\pgfsetlinewidth{0.000000pt}%
\definecolor{currentstroke}{rgb}{0.000000,0.000000,0.000000}%
\pgfsetstrokecolor{currentstroke}%
\pgfsetdash{}{0pt}%
\pgfpathmoveto{\pgfqpoint{4.533395in}{2.818414in}}%
\pgfpathlineto{\pgfqpoint{4.546782in}{2.816629in}}%
\pgfpathlineto{\pgfqpoint{4.560176in}{2.814955in}}%
\pgfpathlineto{\pgfqpoint{4.573578in}{2.813391in}}%
\pgfpathlineto{\pgfqpoint{4.586989in}{2.811936in}}%
\pgfpathlineto{\pgfqpoint{4.579492in}{2.803402in}}%
\pgfpathlineto{\pgfqpoint{4.571990in}{2.794877in}}%
\pgfpathlineto{\pgfqpoint{4.564483in}{2.786359in}}%
\pgfpathlineto{\pgfqpoint{4.556971in}{2.777846in}}%
\pgfpathlineto{\pgfqpoint{4.543550in}{2.779190in}}%
\pgfpathlineto{\pgfqpoint{4.530136in}{2.780643in}}%
\pgfpathlineto{\pgfqpoint{4.516730in}{2.782207in}}%
\pgfpathlineto{\pgfqpoint{4.503332in}{2.783880in}}%
\pgfpathlineto{\pgfqpoint{4.510855in}{2.792497in}}%
\pgfpathlineto{\pgfqpoint{4.518374in}{2.801124in}}%
\pgfpathlineto{\pgfqpoint{4.525887in}{2.809762in}}%
\pgfpathlineto{\pgfqpoint{4.533395in}{2.818414in}}%
\pgfpathclose%
\pgfusepath{fill}%
\end{pgfscope}%
\begin{pgfscope}%
\pgfpathrectangle{\pgfqpoint{1.150000in}{0.150000in}}{\pgfqpoint{5.700000in}{5.700000in}}%
\pgfusepath{clip}%
\pgfsetbuttcap%
\pgfsetroundjoin%
\definecolor{currentfill}{rgb}{0.282327,0.094955,0.417331}%
\pgfsetfillcolor{currentfill}%
\pgfsetfillopacity{0.700000}%
\pgfsetlinewidth{0.000000pt}%
\definecolor{currentstroke}{rgb}{0.000000,0.000000,0.000000}%
\pgfsetstrokecolor{currentstroke}%
\pgfsetdash{}{0pt}%
\pgfpathmoveto{\pgfqpoint{3.598296in}{2.779316in}}%
\pgfpathlineto{\pgfqpoint{3.611501in}{2.771433in}}%
\pgfpathlineto{\pgfqpoint{3.624708in}{2.763693in}}%
\pgfpathlineto{\pgfqpoint{3.637918in}{2.756095in}}%
\pgfpathlineto{\pgfqpoint{3.651131in}{2.748639in}}%
\pgfpathlineto{\pgfqpoint{3.643317in}{2.740194in}}%
\pgfpathlineto{\pgfqpoint{3.635497in}{2.731789in}}%
\pgfpathlineto{\pgfqpoint{3.627671in}{2.723425in}}%
\pgfpathlineto{\pgfqpoint{3.619840in}{2.715103in}}%
\pgfpathlineto{\pgfqpoint{3.606614in}{2.722593in}}%
\pgfpathlineto{\pgfqpoint{3.593390in}{2.730224in}}%
\pgfpathlineto{\pgfqpoint{3.580169in}{2.737998in}}%
\pgfpathlineto{\pgfqpoint{3.566950in}{2.745916in}}%
\pgfpathlineto{\pgfqpoint{3.574795in}{2.754197in}}%
\pgfpathlineto{\pgfqpoint{3.582635in}{2.762525in}}%
\pgfpathlineto{\pgfqpoint{3.590468in}{2.770898in}}%
\pgfpathlineto{\pgfqpoint{3.598296in}{2.779316in}}%
\pgfpathclose%
\pgfusepath{fill}%
\end{pgfscope}%
\begin{pgfscope}%
\pgfpathrectangle{\pgfqpoint{1.150000in}{0.150000in}}{\pgfqpoint{5.700000in}{5.700000in}}%
\pgfusepath{clip}%
\pgfsetbuttcap%
\pgfsetroundjoin%
\definecolor{currentfill}{rgb}{0.281924,0.089666,0.412415}%
\pgfsetfillcolor{currentfill}%
\pgfsetfillopacity{0.700000}%
\pgfsetlinewidth{0.000000pt}%
\definecolor{currentstroke}{rgb}{0.000000,0.000000,0.000000}%
\pgfsetstrokecolor{currentstroke}%
\pgfsetdash{}{0pt}%
\pgfpathmoveto{\pgfqpoint{4.229239in}{2.754572in}}%
\pgfpathlineto{\pgfqpoint{4.242546in}{2.751244in}}%
\pgfpathlineto{\pgfqpoint{4.255859in}{2.748034in}}%
\pgfpathlineto{\pgfqpoint{4.269178in}{2.744940in}}%
\pgfpathlineto{\pgfqpoint{4.282504in}{2.741963in}}%
\pgfpathlineto{\pgfqpoint{4.274903in}{2.733186in}}%
\pgfpathlineto{\pgfqpoint{4.267296in}{2.724421in}}%
\pgfpathlineto{\pgfqpoint{4.259684in}{2.715666in}}%
\pgfpathlineto{\pgfqpoint{4.252068in}{2.706920in}}%
\pgfpathlineto{\pgfqpoint{4.238731in}{2.709840in}}%
\pgfpathlineto{\pgfqpoint{4.225400in}{2.712877in}}%
\pgfpathlineto{\pgfqpoint{4.212076in}{2.716031in}}%
\pgfpathlineto{\pgfqpoint{4.198758in}{2.719302in}}%
\pgfpathlineto{\pgfqpoint{4.206386in}{2.728098in}}%
\pgfpathlineto{\pgfqpoint{4.214009in}{2.736908in}}%
\pgfpathlineto{\pgfqpoint{4.221627in}{2.745732in}}%
\pgfpathlineto{\pgfqpoint{4.229239in}{2.754572in}}%
\pgfpathclose%
\pgfusepath{fill}%
\end{pgfscope}%
\begin{pgfscope}%
\pgfpathrectangle{\pgfqpoint{1.150000in}{0.150000in}}{\pgfqpoint{5.700000in}{5.700000in}}%
\pgfusepath{clip}%
\pgfsetbuttcap%
\pgfsetroundjoin%
\definecolor{currentfill}{rgb}{0.280267,0.073417,0.397163}%
\pgfsetfillcolor{currentfill}%
\pgfsetfillopacity{0.700000}%
\pgfsetlinewidth{0.000000pt}%
\definecolor{currentstroke}{rgb}{0.000000,0.000000,0.000000}%
\pgfsetstrokecolor{currentstroke}%
\pgfsetdash{}{0pt}%
\pgfpathmoveto{\pgfqpoint{4.008714in}{2.732386in}}%
\pgfpathlineto{\pgfqpoint{4.021974in}{2.727701in}}%
\pgfpathlineto{\pgfqpoint{4.035239in}{2.723141in}}%
\pgfpathlineto{\pgfqpoint{4.048509in}{2.718704in}}%
\pgfpathlineto{\pgfqpoint{4.061784in}{2.714390in}}%
\pgfpathlineto{\pgfqpoint{4.054109in}{2.705596in}}%
\pgfpathlineto{\pgfqpoint{4.046428in}{2.696819in}}%
\pgfpathlineto{\pgfqpoint{4.038741in}{2.688061in}}%
\pgfpathlineto{\pgfqpoint{4.031050in}{2.679319in}}%
\pgfpathlineto{\pgfqpoint{4.017763in}{2.683612in}}%
\pgfpathlineto{\pgfqpoint{4.004481in}{2.688028in}}%
\pgfpathlineto{\pgfqpoint{3.991205in}{2.692568in}}%
\pgfpathlineto{\pgfqpoint{3.977934in}{2.697232in}}%
\pgfpathlineto{\pgfqpoint{3.985637in}{2.705988in}}%
\pgfpathlineto{\pgfqpoint{3.993334in}{2.714765in}}%
\pgfpathlineto{\pgfqpoint{4.001027in}{2.723564in}}%
\pgfpathlineto{\pgfqpoint{4.008714in}{2.732386in}}%
\pgfpathclose%
\pgfusepath{fill}%
\end{pgfscope}%
\begin{pgfscope}%
\pgfpathrectangle{\pgfqpoint{1.150000in}{0.150000in}}{\pgfqpoint{5.700000in}{5.700000in}}%
\pgfusepath{clip}%
\pgfsetbuttcap%
\pgfsetroundjoin%
\definecolor{currentfill}{rgb}{0.278012,0.180367,0.486697}%
\pgfsetfillcolor{currentfill}%
\pgfsetfillopacity{0.700000}%
\pgfsetlinewidth{0.000000pt}%
\definecolor{currentstroke}{rgb}{0.000000,0.000000,0.000000}%
\pgfsetstrokecolor{currentstroke}%
\pgfsetdash{}{0pt}%
\pgfpathmoveto{\pgfqpoint{3.218269in}{2.950386in}}%
\pgfpathlineto{\pgfqpoint{3.231486in}{2.938636in}}%
\pgfpathlineto{\pgfqpoint{3.244702in}{2.927058in}}%
\pgfpathlineto{\pgfqpoint{3.257918in}{2.915651in}}%
\pgfpathlineto{\pgfqpoint{3.271133in}{2.904413in}}%
\pgfpathlineto{\pgfqpoint{3.263179in}{2.896682in}}%
\pgfpathlineto{\pgfqpoint{3.255218in}{2.889020in}}%
\pgfpathlineto{\pgfqpoint{3.247250in}{2.881427in}}%
\pgfpathlineto{\pgfqpoint{3.239276in}{2.873904in}}%
\pgfpathlineto{\pgfqpoint{3.226044in}{2.885214in}}%
\pgfpathlineto{\pgfqpoint{3.212811in}{2.896694in}}%
\pgfpathlineto{\pgfqpoint{3.199577in}{2.908345in}}%
\pgfpathlineto{\pgfqpoint{3.186342in}{2.920168in}}%
\pgfpathlineto{\pgfqpoint{3.194334in}{2.927612in}}%
\pgfpathlineto{\pgfqpoint{3.202320in}{2.935129in}}%
\pgfpathlineto{\pgfqpoint{3.210298in}{2.942721in}}%
\pgfpathlineto{\pgfqpoint{3.218269in}{2.950386in}}%
\pgfpathclose%
\pgfusepath{fill}%
\end{pgfscope}%
\begin{pgfscope}%
\pgfpathrectangle{\pgfqpoint{1.150000in}{0.150000in}}{\pgfqpoint{5.700000in}{5.700000in}}%
\pgfusepath{clip}%
\pgfsetbuttcap%
\pgfsetroundjoin%
\definecolor{currentfill}{rgb}{0.279574,0.170599,0.479997}%
\pgfsetfillcolor{currentfill}%
\pgfsetfillopacity{0.700000}%
\pgfsetlinewidth{0.000000pt}%
\definecolor{currentstroke}{rgb}{0.000000,0.000000,0.000000}%
\pgfsetstrokecolor{currentstroke}%
\pgfsetdash{}{0pt}%
\pgfpathmoveto{\pgfqpoint{4.837716in}{2.901290in}}%
\pgfpathlineto{\pgfqpoint{4.851200in}{2.900785in}}%
\pgfpathlineto{\pgfqpoint{4.864694in}{2.900385in}}%
\pgfpathlineto{\pgfqpoint{4.878197in}{2.900090in}}%
\pgfpathlineto{\pgfqpoint{4.891710in}{2.899899in}}%
\pgfpathlineto{\pgfqpoint{4.884321in}{2.891813in}}%
\pgfpathlineto{\pgfqpoint{4.876927in}{2.883739in}}%
\pgfpathlineto{\pgfqpoint{4.869528in}{2.875677in}}%
\pgfpathlineto{\pgfqpoint{4.862124in}{2.867623in}}%
\pgfpathlineto{\pgfqpoint{4.848599in}{2.867648in}}%
\pgfpathlineto{\pgfqpoint{4.835083in}{2.867778in}}%
\pgfpathlineto{\pgfqpoint{4.821576in}{2.868013in}}%
\pgfpathlineto{\pgfqpoint{4.808079in}{2.868352in}}%
\pgfpathlineto{\pgfqpoint{4.815495in}{2.876565in}}%
\pgfpathlineto{\pgfqpoint{4.822907in}{2.884791in}}%
\pgfpathlineto{\pgfqpoint{4.830314in}{2.893032in}}%
\pgfpathlineto{\pgfqpoint{4.837716in}{2.901290in}}%
\pgfpathclose%
\pgfusepath{fill}%
\end{pgfscope}%
\begin{pgfscope}%
\pgfpathrectangle{\pgfqpoint{1.150000in}{0.150000in}}{\pgfqpoint{5.700000in}{5.700000in}}%
\pgfusepath{clip}%
\pgfsetbuttcap%
\pgfsetroundjoin%
\definecolor{currentfill}{rgb}{0.283091,0.110553,0.431554}%
\pgfsetfillcolor{currentfill}%
\pgfsetfillopacity{0.700000}%
\pgfsetlinewidth{0.000000pt}%
\definecolor{currentstroke}{rgb}{0.000000,0.000000,0.000000}%
\pgfsetstrokecolor{currentstroke}%
\pgfsetdash{}{0pt}%
\pgfpathmoveto{\pgfqpoint{4.449817in}{2.791685in}}%
\pgfpathlineto{\pgfqpoint{4.463185in}{2.789567in}}%
\pgfpathlineto{\pgfqpoint{4.476560in}{2.787560in}}%
\pgfpathlineto{\pgfqpoint{4.489942in}{2.785665in}}%
\pgfpathlineto{\pgfqpoint{4.503332in}{2.783880in}}%
\pgfpathlineto{\pgfqpoint{4.495804in}{2.775272in}}%
\pgfpathlineto{\pgfqpoint{4.488271in}{2.766671in}}%
\pgfpathlineto{\pgfqpoint{4.480733in}{2.758076in}}%
\pgfpathlineto{\pgfqpoint{4.473189in}{2.749486in}}%
\pgfpathlineto{\pgfqpoint{4.459788in}{2.751177in}}%
\pgfpathlineto{\pgfqpoint{4.446394in}{2.752979in}}%
\pgfpathlineto{\pgfqpoint{4.433008in}{2.754893in}}%
\pgfpathlineto{\pgfqpoint{4.419630in}{2.756919in}}%
\pgfpathlineto{\pgfqpoint{4.427184in}{2.765595in}}%
\pgfpathlineto{\pgfqpoint{4.434733in}{2.774281in}}%
\pgfpathlineto{\pgfqpoint{4.442278in}{2.782977in}}%
\pgfpathlineto{\pgfqpoint{4.449817in}{2.791685in}}%
\pgfpathclose%
\pgfusepath{fill}%
\end{pgfscope}%
\begin{pgfscope}%
\pgfpathrectangle{\pgfqpoint{1.150000in}{0.150000in}}{\pgfqpoint{5.700000in}{5.700000in}}%
\pgfusepath{clip}%
\pgfsetbuttcap%
\pgfsetroundjoin%
\definecolor{currentfill}{rgb}{0.283197,0.115680,0.436115}%
\pgfsetfillcolor{currentfill}%
\pgfsetfillopacity{0.700000}%
\pgfsetlinewidth{0.000000pt}%
\definecolor{currentstroke}{rgb}{0.000000,0.000000,0.000000}%
\pgfsetstrokecolor{currentstroke}%
\pgfsetdash{}{0pt}%
\pgfpathmoveto{\pgfqpoint{3.461271in}{2.814540in}}%
\pgfpathlineto{\pgfqpoint{3.474475in}{2.805438in}}%
\pgfpathlineto{\pgfqpoint{3.487681in}{2.796488in}}%
\pgfpathlineto{\pgfqpoint{3.500888in}{2.787689in}}%
\pgfpathlineto{\pgfqpoint{3.514097in}{2.779039in}}%
\pgfpathlineto{\pgfqpoint{3.506231in}{2.770849in}}%
\pgfpathlineto{\pgfqpoint{3.498359in}{2.762708in}}%
\pgfpathlineto{\pgfqpoint{3.490482in}{2.754618in}}%
\pgfpathlineto{\pgfqpoint{3.482598in}{2.746578in}}%
\pgfpathlineto{\pgfqpoint{3.469374in}{2.755280in}}%
\pgfpathlineto{\pgfqpoint{3.456152in}{2.764131in}}%
\pgfpathlineto{\pgfqpoint{3.442932in}{2.773134in}}%
\pgfpathlineto{\pgfqpoint{3.429712in}{2.782288in}}%
\pgfpathlineto{\pgfqpoint{3.437611in}{2.790269in}}%
\pgfpathlineto{\pgfqpoint{3.445504in}{2.798304in}}%
\pgfpathlineto{\pgfqpoint{3.453391in}{2.806395in}}%
\pgfpathlineto{\pgfqpoint{3.461271in}{2.814540in}}%
\pgfpathclose%
\pgfusepath{fill}%
\end{pgfscope}%
\begin{pgfscope}%
\pgfpathrectangle{\pgfqpoint{1.150000in}{0.150000in}}{\pgfqpoint{5.700000in}{5.700000in}}%
\pgfusepath{clip}%
\pgfsetbuttcap%
\pgfsetroundjoin%
\definecolor{currentfill}{rgb}{0.281412,0.155834,0.469201}%
\pgfsetfillcolor{currentfill}%
\pgfsetfillopacity{0.700000}%
\pgfsetlinewidth{0.000000pt}%
\definecolor{currentstroke}{rgb}{0.000000,0.000000,0.000000}%
\pgfsetstrokecolor{currentstroke}%
\pgfsetdash{}{0pt}%
\pgfpathmoveto{\pgfqpoint{4.754178in}{2.870764in}}%
\pgfpathlineto{\pgfqpoint{4.767640in}{2.870002in}}%
\pgfpathlineto{\pgfqpoint{4.781110in}{2.869347in}}%
\pgfpathlineto{\pgfqpoint{4.794590in}{2.868797in}}%
\pgfpathlineto{\pgfqpoint{4.808079in}{2.868352in}}%
\pgfpathlineto{\pgfqpoint{4.800657in}{2.860150in}}%
\pgfpathlineto{\pgfqpoint{4.793231in}{2.851958in}}%
\pgfpathlineto{\pgfqpoint{4.785800in}{2.843772in}}%
\pgfpathlineto{\pgfqpoint{4.778363in}{2.835592in}}%
\pgfpathlineto{\pgfqpoint{4.764862in}{2.835889in}}%
\pgfpathlineto{\pgfqpoint{4.751371in}{2.836292in}}%
\pgfpathlineto{\pgfqpoint{4.737887in}{2.836800in}}%
\pgfpathlineto{\pgfqpoint{4.724413in}{2.837415in}}%
\pgfpathlineto{\pgfqpoint{4.731862in}{2.845736in}}%
\pgfpathlineto{\pgfqpoint{4.739305in}{2.854066in}}%
\pgfpathlineto{\pgfqpoint{4.746744in}{2.862408in}}%
\pgfpathlineto{\pgfqpoint{4.754178in}{2.870764in}}%
\pgfpathclose%
\pgfusepath{fill}%
\end{pgfscope}%
\begin{pgfscope}%
\pgfpathrectangle{\pgfqpoint{1.150000in}{0.150000in}}{\pgfqpoint{5.700000in}{5.700000in}}%
\pgfusepath{clip}%
\pgfsetbuttcap%
\pgfsetroundjoin%
\definecolor{currentfill}{rgb}{0.280868,0.160771,0.472899}%
\pgfsetfillcolor{currentfill}%
\pgfsetfillopacity{0.700000}%
\pgfsetlinewidth{0.000000pt}%
\definecolor{currentstroke}{rgb}{0.000000,0.000000,0.000000}%
\pgfsetstrokecolor{currentstroke}%
\pgfsetdash{}{0pt}%
\pgfpathmoveto{\pgfqpoint{3.271133in}{2.904413in}}%
\pgfpathlineto{\pgfqpoint{3.284347in}{2.893343in}}%
\pgfpathlineto{\pgfqpoint{3.297561in}{2.882440in}}%
\pgfpathlineto{\pgfqpoint{3.310775in}{2.871702in}}%
\pgfpathlineto{\pgfqpoint{3.323988in}{2.861128in}}%
\pgfpathlineto{\pgfqpoint{3.316051in}{2.853332in}}%
\pgfpathlineto{\pgfqpoint{3.308107in}{2.845600in}}%
\pgfpathlineto{\pgfqpoint{3.300156in}{2.837933in}}%
\pgfpathlineto{\pgfqpoint{3.292199in}{2.830332in}}%
\pgfpathlineto{\pgfqpoint{3.278969in}{2.840977in}}%
\pgfpathlineto{\pgfqpoint{3.265738in}{2.851787in}}%
\pgfpathlineto{\pgfqpoint{3.252507in}{2.862762in}}%
\pgfpathlineto{\pgfqpoint{3.239276in}{2.873904in}}%
\pgfpathlineto{\pgfqpoint{3.247250in}{2.881427in}}%
\pgfpathlineto{\pgfqpoint{3.255218in}{2.889020in}}%
\pgfpathlineto{\pgfqpoint{3.263179in}{2.896682in}}%
\pgfpathlineto{\pgfqpoint{3.271133in}{2.904413in}}%
\pgfpathclose%
\pgfusepath{fill}%
\end{pgfscope}%
\begin{pgfscope}%
\pgfpathrectangle{\pgfqpoint{1.150000in}{0.150000in}}{\pgfqpoint{5.700000in}{5.700000in}}%
\pgfusepath{clip}%
\pgfsetbuttcap%
\pgfsetroundjoin%
\definecolor{currentfill}{rgb}{0.280894,0.078907,0.402329}%
\pgfsetfillcolor{currentfill}%
\pgfsetfillopacity{0.700000}%
\pgfsetlinewidth{0.000000pt}%
\definecolor{currentstroke}{rgb}{0.000000,0.000000,0.000000}%
\pgfsetstrokecolor{currentstroke}%
\pgfsetdash{}{0pt}%
\pgfpathmoveto{\pgfqpoint{4.145550in}{2.733572in}}%
\pgfpathlineto{\pgfqpoint{4.158843in}{2.729825in}}%
\pgfpathlineto{\pgfqpoint{4.172142in}{2.726199in}}%
\pgfpathlineto{\pgfqpoint{4.185447in}{2.722691in}}%
\pgfpathlineto{\pgfqpoint{4.198758in}{2.719302in}}%
\pgfpathlineto{\pgfqpoint{4.191125in}{2.710519in}}%
\pgfpathlineto{\pgfqpoint{4.183487in}{2.701748in}}%
\pgfpathlineto{\pgfqpoint{4.175844in}{2.692989in}}%
\pgfpathlineto{\pgfqpoint{4.168196in}{2.684240in}}%
\pgfpathlineto{\pgfqpoint{4.154874in}{2.687590in}}%
\pgfpathlineto{\pgfqpoint{4.141557in}{2.691058in}}%
\pgfpathlineto{\pgfqpoint{4.128247in}{2.694646in}}%
\pgfpathlineto{\pgfqpoint{4.114943in}{2.698354in}}%
\pgfpathlineto{\pgfqpoint{4.122602in}{2.707135in}}%
\pgfpathlineto{\pgfqpoint{4.130257in}{2.715931in}}%
\pgfpathlineto{\pgfqpoint{4.137906in}{2.724743in}}%
\pgfpathlineto{\pgfqpoint{4.145550in}{2.733572in}}%
\pgfpathclose%
\pgfusepath{fill}%
\end{pgfscope}%
\begin{pgfscope}%
\pgfpathrectangle{\pgfqpoint{1.150000in}{0.150000in}}{\pgfqpoint{5.700000in}{5.700000in}}%
\pgfusepath{clip}%
\pgfsetbuttcap%
\pgfsetroundjoin%
\definecolor{currentfill}{rgb}{0.280267,0.073417,0.397163}%
\pgfsetfillcolor{currentfill}%
\pgfsetfillopacity{0.700000}%
\pgfsetlinewidth{0.000000pt}%
\definecolor{currentstroke}{rgb}{0.000000,0.000000,0.000000}%
\pgfsetstrokecolor{currentstroke}%
\pgfsetdash{}{0pt}%
\pgfpathmoveto{\pgfqpoint{3.788038in}{2.728297in}}%
\pgfpathlineto{\pgfqpoint{3.801267in}{2.722090in}}%
\pgfpathlineto{\pgfqpoint{3.814501in}{2.716015in}}%
\pgfpathlineto{\pgfqpoint{3.827738in}{2.710073in}}%
\pgfpathlineto{\pgfqpoint{3.840979in}{2.704262in}}%
\pgfpathlineto{\pgfqpoint{3.833226in}{2.695625in}}%
\pgfpathlineto{\pgfqpoint{3.825467in}{2.687015in}}%
\pgfpathlineto{\pgfqpoint{3.817703in}{2.678434in}}%
\pgfpathlineto{\pgfqpoint{3.809933in}{2.669881in}}%
\pgfpathlineto{\pgfqpoint{3.796680in}{2.675707in}}%
\pgfpathlineto{\pgfqpoint{3.783430in}{2.681664in}}%
\pgfpathlineto{\pgfqpoint{3.770184in}{2.687754in}}%
\pgfpathlineto{\pgfqpoint{3.756942in}{2.693976in}}%
\pgfpathlineto{\pgfqpoint{3.764724in}{2.702508in}}%
\pgfpathlineto{\pgfqpoint{3.772501in}{2.711072in}}%
\pgfpathlineto{\pgfqpoint{3.780272in}{2.719668in}}%
\pgfpathlineto{\pgfqpoint{3.788038in}{2.728297in}}%
\pgfpathclose%
\pgfusepath{fill}%
\end{pgfscope}%
\begin{pgfscope}%
\pgfpathrectangle{\pgfqpoint{1.150000in}{0.150000in}}{\pgfqpoint{5.700000in}{5.700000in}}%
\pgfusepath{clip}%
\pgfsetbuttcap%
\pgfsetroundjoin%
\definecolor{currentfill}{rgb}{0.281446,0.084320,0.407414}%
\pgfsetfillcolor{currentfill}%
\pgfsetfillopacity{0.700000}%
\pgfsetlinewidth{0.000000pt}%
\definecolor{currentstroke}{rgb}{0.000000,0.000000,0.000000}%
\pgfsetstrokecolor{currentstroke}%
\pgfsetdash{}{0pt}%
\pgfpathmoveto{\pgfqpoint{3.651131in}{2.748639in}}%
\pgfpathlineto{\pgfqpoint{3.664346in}{2.741324in}}%
\pgfpathlineto{\pgfqpoint{3.677564in}{2.734148in}}%
\pgfpathlineto{\pgfqpoint{3.690786in}{2.727111in}}%
\pgfpathlineto{\pgfqpoint{3.704010in}{2.720212in}}%
\pgfpathlineto{\pgfqpoint{3.696210in}{2.711739in}}%
\pgfpathlineto{\pgfqpoint{3.688403in}{2.703303in}}%
\pgfpathlineto{\pgfqpoint{3.680591in}{2.694903in}}%
\pgfpathlineto{\pgfqpoint{3.672773in}{2.686541in}}%
\pgfpathlineto{\pgfqpoint{3.659536in}{2.693473in}}%
\pgfpathlineto{\pgfqpoint{3.646301in}{2.700544in}}%
\pgfpathlineto{\pgfqpoint{3.633069in}{2.707754in}}%
\pgfpathlineto{\pgfqpoint{3.619840in}{2.715103in}}%
\pgfpathlineto{\pgfqpoint{3.627671in}{2.723425in}}%
\pgfpathlineto{\pgfqpoint{3.635497in}{2.731789in}}%
\pgfpathlineto{\pgfqpoint{3.643317in}{2.740194in}}%
\pgfpathlineto{\pgfqpoint{3.651131in}{2.748639in}}%
\pgfpathclose%
\pgfusepath{fill}%
\end{pgfscope}%
\begin{pgfscope}%
\pgfpathrectangle{\pgfqpoint{1.150000in}{0.150000in}}{\pgfqpoint{5.700000in}{5.700000in}}%
\pgfusepath{clip}%
\pgfsetbuttcap%
\pgfsetroundjoin%
\definecolor{currentfill}{rgb}{0.282656,0.100196,0.422160}%
\pgfsetfillcolor{currentfill}%
\pgfsetfillopacity{0.700000}%
\pgfsetlinewidth{0.000000pt}%
\definecolor{currentstroke}{rgb}{0.000000,0.000000,0.000000}%
\pgfsetstrokecolor{currentstroke}%
\pgfsetdash{}{0pt}%
\pgfpathmoveto{\pgfqpoint{4.366189in}{2.766148in}}%
\pgfpathlineto{\pgfqpoint{4.379538in}{2.763671in}}%
\pgfpathlineto{\pgfqpoint{4.392895in}{2.761307in}}%
\pgfpathlineto{\pgfqpoint{4.406259in}{2.759057in}}%
\pgfpathlineto{\pgfqpoint{4.419630in}{2.756919in}}%
\pgfpathlineto{\pgfqpoint{4.412070in}{2.748250in}}%
\pgfpathlineto{\pgfqpoint{4.404506in}{2.739587in}}%
\pgfpathlineto{\pgfqpoint{4.396936in}{2.730931in}}%
\pgfpathlineto{\pgfqpoint{4.389362in}{2.722279in}}%
\pgfpathlineto{\pgfqpoint{4.375980in}{2.724341in}}%
\pgfpathlineto{\pgfqpoint{4.362605in}{2.726517in}}%
\pgfpathlineto{\pgfqpoint{4.349237in}{2.728806in}}%
\pgfpathlineto{\pgfqpoint{4.335877in}{2.731208in}}%
\pgfpathlineto{\pgfqpoint{4.343462in}{2.739929in}}%
\pgfpathlineto{\pgfqpoint{4.351043in}{2.748658in}}%
\pgfpathlineto{\pgfqpoint{4.358618in}{2.757398in}}%
\pgfpathlineto{\pgfqpoint{4.366189in}{2.766148in}}%
\pgfpathclose%
\pgfusepath{fill}%
\end{pgfscope}%
\begin{pgfscope}%
\pgfpathrectangle{\pgfqpoint{1.150000in}{0.150000in}}{\pgfqpoint{5.700000in}{5.700000in}}%
\pgfusepath{clip}%
\pgfsetbuttcap%
\pgfsetroundjoin%
\definecolor{currentfill}{rgb}{0.280267,0.073417,0.397163}%
\pgfsetfillcolor{currentfill}%
\pgfsetfillopacity{0.700000}%
\pgfsetlinewidth{0.000000pt}%
\definecolor{currentstroke}{rgb}{0.000000,0.000000,0.000000}%
\pgfsetstrokecolor{currentstroke}%
\pgfsetdash{}{0pt}%
\pgfpathmoveto{\pgfqpoint{3.924899in}{2.717145in}}%
\pgfpathlineto{\pgfqpoint{3.938150in}{2.711977in}}%
\pgfpathlineto{\pgfqpoint{3.951407in}{2.706936in}}%
\pgfpathlineto{\pgfqpoint{3.964668in}{2.702021in}}%
\pgfpathlineto{\pgfqpoint{3.977934in}{2.697232in}}%
\pgfpathlineto{\pgfqpoint{3.970226in}{2.688498in}}%
\pgfpathlineto{\pgfqpoint{3.962512in}{2.679784in}}%
\pgfpathlineto{\pgfqpoint{3.954794in}{2.671090in}}%
\pgfpathlineto{\pgfqpoint{3.947070in}{2.662416in}}%
\pgfpathlineto{\pgfqpoint{3.933792in}{2.667202in}}%
\pgfpathlineto{\pgfqpoint{3.920519in}{2.672114in}}%
\pgfpathlineto{\pgfqpoint{3.907251in}{2.677152in}}%
\pgfpathlineto{\pgfqpoint{3.893988in}{2.682318in}}%
\pgfpathlineto{\pgfqpoint{3.901723in}{2.690988in}}%
\pgfpathlineto{\pgfqpoint{3.909454in}{2.699682in}}%
\pgfpathlineto{\pgfqpoint{3.917179in}{2.708401in}}%
\pgfpathlineto{\pgfqpoint{3.924899in}{2.717145in}}%
\pgfpathclose%
\pgfusepath{fill}%
\end{pgfscope}%
\begin{pgfscope}%
\pgfpathrectangle{\pgfqpoint{1.150000in}{0.150000in}}{\pgfqpoint{5.700000in}{5.700000in}}%
\pgfusepath{clip}%
\pgfsetbuttcap%
\pgfsetroundjoin%
\definecolor{currentfill}{rgb}{0.282623,0.140926,0.457517}%
\pgfsetfillcolor{currentfill}%
\pgfsetfillopacity{0.700000}%
\pgfsetlinewidth{0.000000pt}%
\definecolor{currentstroke}{rgb}{0.000000,0.000000,0.000000}%
\pgfsetstrokecolor{currentstroke}%
\pgfsetdash{}{0pt}%
\pgfpathmoveto{\pgfqpoint{4.670603in}{2.840940in}}%
\pgfpathlineto{\pgfqpoint{4.684043in}{2.839898in}}%
\pgfpathlineto{\pgfqpoint{4.697491in}{2.838964in}}%
\pgfpathlineto{\pgfqpoint{4.710948in}{2.838136in}}%
\pgfpathlineto{\pgfqpoint{4.724413in}{2.837415in}}%
\pgfpathlineto{\pgfqpoint{4.716960in}{2.829102in}}%
\pgfpathlineto{\pgfqpoint{4.709501in}{2.820796in}}%
\pgfpathlineto{\pgfqpoint{4.702038in}{2.812494in}}%
\pgfpathlineto{\pgfqpoint{4.694569in}{2.804195in}}%
\pgfpathlineto{\pgfqpoint{4.681092in}{2.804787in}}%
\pgfpathlineto{\pgfqpoint{4.667623in}{2.805485in}}%
\pgfpathlineto{\pgfqpoint{4.654163in}{2.806290in}}%
\pgfpathlineto{\pgfqpoint{4.640712in}{2.807203in}}%
\pgfpathlineto{\pgfqpoint{4.648192in}{2.815625in}}%
\pgfpathlineto{\pgfqpoint{4.655667in}{2.824054in}}%
\pgfpathlineto{\pgfqpoint{4.663137in}{2.832492in}}%
\pgfpathlineto{\pgfqpoint{4.670603in}{2.840940in}}%
\pgfpathclose%
\pgfusepath{fill}%
\end{pgfscope}%
\begin{pgfscope}%
\pgfpathrectangle{\pgfqpoint{1.150000in}{0.150000in}}{\pgfqpoint{5.700000in}{5.700000in}}%
\pgfusepath{clip}%
\pgfsetbuttcap%
\pgfsetroundjoin%
\definecolor{currentfill}{rgb}{0.282910,0.105393,0.426902}%
\pgfsetfillcolor{currentfill}%
\pgfsetfillopacity{0.700000}%
\pgfsetlinewidth{0.000000pt}%
\definecolor{currentstroke}{rgb}{0.000000,0.000000,0.000000}%
\pgfsetstrokecolor{currentstroke}%
\pgfsetdash{}{0pt}%
\pgfpathmoveto{\pgfqpoint{3.514097in}{2.779039in}}%
\pgfpathlineto{\pgfqpoint{3.527307in}{2.770538in}}%
\pgfpathlineto{\pgfqpoint{3.540519in}{2.762185in}}%
\pgfpathlineto{\pgfqpoint{3.553734in}{2.753977in}}%
\pgfpathlineto{\pgfqpoint{3.566950in}{2.745916in}}%
\pgfpathlineto{\pgfqpoint{3.559099in}{2.737679in}}%
\pgfpathlineto{\pgfqpoint{3.551242in}{2.729489in}}%
\pgfpathlineto{\pgfqpoint{3.543379in}{2.721344in}}%
\pgfpathlineto{\pgfqpoint{3.535510in}{2.713245in}}%
\pgfpathlineto{\pgfqpoint{3.522279in}{2.721359in}}%
\pgfpathlineto{\pgfqpoint{3.509050in}{2.729618in}}%
\pgfpathlineto{\pgfqpoint{3.495823in}{2.738024in}}%
\pgfpathlineto{\pgfqpoint{3.482598in}{2.746578in}}%
\pgfpathlineto{\pgfqpoint{3.490482in}{2.754618in}}%
\pgfpathlineto{\pgfqpoint{3.498359in}{2.762708in}}%
\pgfpathlineto{\pgfqpoint{3.506231in}{2.770849in}}%
\pgfpathlineto{\pgfqpoint{3.514097in}{2.779039in}}%
\pgfpathclose%
\pgfusepath{fill}%
\end{pgfscope}%
\begin{pgfscope}%
\pgfpathrectangle{\pgfqpoint{1.150000in}{0.150000in}}{\pgfqpoint{5.700000in}{5.700000in}}%
\pgfusepath{clip}%
\pgfsetbuttcap%
\pgfsetroundjoin%
\definecolor{currentfill}{rgb}{0.282290,0.145912,0.461510}%
\pgfsetfillcolor{currentfill}%
\pgfsetfillopacity{0.700000}%
\pgfsetlinewidth{0.000000pt}%
\definecolor{currentstroke}{rgb}{0.000000,0.000000,0.000000}%
\pgfsetstrokecolor{currentstroke}%
\pgfsetdash{}{0pt}%
\pgfpathmoveto{\pgfqpoint{3.323988in}{2.861128in}}%
\pgfpathlineto{\pgfqpoint{3.337202in}{2.850717in}}%
\pgfpathlineto{\pgfqpoint{3.350416in}{2.840467in}}%
\pgfpathlineto{\pgfqpoint{3.363630in}{2.830378in}}%
\pgfpathlineto{\pgfqpoint{3.376845in}{2.820447in}}%
\pgfpathlineto{\pgfqpoint{3.368924in}{2.812586in}}%
\pgfpathlineto{\pgfqpoint{3.360997in}{2.804785in}}%
\pgfpathlineto{\pgfqpoint{3.353063in}{2.797044in}}%
\pgfpathlineto{\pgfqpoint{3.345122in}{2.789365in}}%
\pgfpathlineto{\pgfqpoint{3.331891in}{2.799367in}}%
\pgfpathlineto{\pgfqpoint{3.318660in}{2.809528in}}%
\pgfpathlineto{\pgfqpoint{3.305430in}{2.819849in}}%
\pgfpathlineto{\pgfqpoint{3.292199in}{2.830332in}}%
\pgfpathlineto{\pgfqpoint{3.300156in}{2.837933in}}%
\pgfpathlineto{\pgfqpoint{3.308107in}{2.845600in}}%
\pgfpathlineto{\pgfqpoint{3.316051in}{2.853332in}}%
\pgfpathlineto{\pgfqpoint{3.323988in}{2.861128in}}%
\pgfpathclose%
\pgfusepath{fill}%
\end{pgfscope}%
\begin{pgfscope}%
\pgfpathrectangle{\pgfqpoint{1.150000in}{0.150000in}}{\pgfqpoint{5.700000in}{5.700000in}}%
\pgfusepath{clip}%
\pgfsetbuttcap%
\pgfsetroundjoin%
\definecolor{currentfill}{rgb}{0.273006,0.204520,0.501721}%
\pgfsetfillcolor{currentfill}%
\pgfsetfillopacity{0.700000}%
\pgfsetlinewidth{0.000000pt}%
\definecolor{currentstroke}{rgb}{0.000000,0.000000,0.000000}%
\pgfsetstrokecolor{currentstroke}%
\pgfsetdash{}{0pt}%
\pgfpathmoveto{\pgfqpoint{5.058873in}{2.964425in}}%
\pgfpathlineto{\pgfqpoint{5.072444in}{2.964775in}}%
\pgfpathlineto{\pgfqpoint{5.086025in}{2.965227in}}%
\pgfpathlineto{\pgfqpoint{5.099615in}{2.965780in}}%
\pgfpathlineto{\pgfqpoint{5.113216in}{2.966434in}}%
\pgfpathlineto{\pgfqpoint{5.105907in}{2.958790in}}%
\pgfpathlineto{\pgfqpoint{5.098594in}{2.951164in}}%
\pgfpathlineto{\pgfqpoint{5.091275in}{2.943553in}}%
\pgfpathlineto{\pgfqpoint{5.083951in}{2.935956in}}%
\pgfpathlineto{\pgfqpoint{5.070336in}{2.935099in}}%
\pgfpathlineto{\pgfqpoint{5.056731in}{2.934344in}}%
\pgfpathlineto{\pgfqpoint{5.043135in}{2.933690in}}%
\pgfpathlineto{\pgfqpoint{5.029550in}{2.933138in}}%
\pgfpathlineto{\pgfqpoint{5.036888in}{2.940931in}}%
\pgfpathlineto{\pgfqpoint{5.044221in}{2.948742in}}%
\pgfpathlineto{\pgfqpoint{5.051549in}{2.956572in}}%
\pgfpathlineto{\pgfqpoint{5.058873in}{2.964425in}}%
\pgfpathclose%
\pgfusepath{fill}%
\end{pgfscope}%
\begin{pgfscope}%
\pgfpathrectangle{\pgfqpoint{1.150000in}{0.150000in}}{\pgfqpoint{5.700000in}{5.700000in}}%
\pgfusepath{clip}%
\pgfsetbuttcap%
\pgfsetroundjoin%
\definecolor{currentfill}{rgb}{0.280267,0.073417,0.397163}%
\pgfsetfillcolor{currentfill}%
\pgfsetfillopacity{0.700000}%
\pgfsetlinewidth{0.000000pt}%
\definecolor{currentstroke}{rgb}{0.000000,0.000000,0.000000}%
\pgfsetstrokecolor{currentstroke}%
\pgfsetdash{}{0pt}%
\pgfpathmoveto{\pgfqpoint{4.061784in}{2.714390in}}%
\pgfpathlineto{\pgfqpoint{4.075065in}{2.710199in}}%
\pgfpathlineto{\pgfqpoint{4.088352in}{2.706129in}}%
\pgfpathlineto{\pgfqpoint{4.101645in}{2.702181in}}%
\pgfpathlineto{\pgfqpoint{4.114943in}{2.698354in}}%
\pgfpathlineto{\pgfqpoint{4.107278in}{2.689587in}}%
\pgfpathlineto{\pgfqpoint{4.099609in}{2.680834in}}%
\pgfpathlineto{\pgfqpoint{4.091934in}{2.672094in}}%
\pgfpathlineto{\pgfqpoint{4.084254in}{2.663367in}}%
\pgfpathlineto{\pgfqpoint{4.070944in}{2.667174in}}%
\pgfpathlineto{\pgfqpoint{4.057641in}{2.671101in}}%
\pgfpathlineto{\pgfqpoint{4.044343in}{2.675149in}}%
\pgfpathlineto{\pgfqpoint{4.031050in}{2.679319in}}%
\pgfpathlineto{\pgfqpoint{4.038741in}{2.688061in}}%
\pgfpathlineto{\pgfqpoint{4.046428in}{2.696819in}}%
\pgfpathlineto{\pgfqpoint{4.054109in}{2.705596in}}%
\pgfpathlineto{\pgfqpoint{4.061784in}{2.714390in}}%
\pgfpathclose%
\pgfusepath{fill}%
\end{pgfscope}%
\begin{pgfscope}%
\pgfpathrectangle{\pgfqpoint{1.150000in}{0.150000in}}{\pgfqpoint{5.700000in}{5.700000in}}%
\pgfusepath{clip}%
\pgfsetbuttcap%
\pgfsetroundjoin%
\definecolor{currentfill}{rgb}{0.260571,0.246922,0.522828}%
\pgfsetfillcolor{currentfill}%
\pgfsetfillopacity{0.700000}%
\pgfsetlinewidth{0.000000pt}%
\definecolor{currentstroke}{rgb}{0.000000,0.000000,0.000000}%
\pgfsetstrokecolor{currentstroke}%
\pgfsetdash{}{0pt}%
\pgfpathmoveto{\pgfqpoint{3.027383in}{3.076064in}}%
\pgfpathlineto{\pgfqpoint{3.040643in}{3.062049in}}%
\pgfpathlineto{\pgfqpoint{3.053900in}{3.048225in}}%
\pgfpathlineto{\pgfqpoint{3.067154in}{3.034591in}}%
\pgfpathlineto{\pgfqpoint{3.080406in}{3.021146in}}%
\pgfpathlineto{\pgfqpoint{3.072369in}{3.013947in}}%
\pgfpathlineto{\pgfqpoint{3.064325in}{3.006833in}}%
\pgfpathlineto{\pgfqpoint{3.056273in}{2.999803in}}%
\pgfpathlineto{\pgfqpoint{3.048213in}{2.992860in}}%
\pgfpathlineto{\pgfqpoint{3.034942in}{3.006397in}}%
\pgfpathlineto{\pgfqpoint{3.021668in}{3.020124in}}%
\pgfpathlineto{\pgfqpoint{3.008391in}{3.034040in}}%
\pgfpathlineto{\pgfqpoint{2.995110in}{3.048149in}}%
\pgfpathlineto{\pgfqpoint{3.003190in}{3.054993in}}%
\pgfpathlineto{\pgfqpoint{3.011262in}{3.061928in}}%
\pgfpathlineto{\pgfqpoint{3.019326in}{3.068952in}}%
\pgfpathlineto{\pgfqpoint{3.027383in}{3.076064in}}%
\pgfpathclose%
\pgfusepath{fill}%
\end{pgfscope}%
\begin{pgfscope}%
\pgfpathrectangle{\pgfqpoint{1.150000in}{0.150000in}}{\pgfqpoint{5.700000in}{5.700000in}}%
\pgfusepath{clip}%
\pgfsetbuttcap%
\pgfsetroundjoin%
\definecolor{currentfill}{rgb}{0.281924,0.089666,0.412415}%
\pgfsetfillcolor{currentfill}%
\pgfsetfillopacity{0.700000}%
\pgfsetlinewidth{0.000000pt}%
\definecolor{currentstroke}{rgb}{0.000000,0.000000,0.000000}%
\pgfsetstrokecolor{currentstroke}%
\pgfsetdash{}{0pt}%
\pgfpathmoveto{\pgfqpoint{4.282504in}{2.741963in}}%
\pgfpathlineto{\pgfqpoint{4.295837in}{2.739101in}}%
\pgfpathlineto{\pgfqpoint{4.309177in}{2.736355in}}%
\pgfpathlineto{\pgfqpoint{4.322523in}{2.733724in}}%
\pgfpathlineto{\pgfqpoint{4.335877in}{2.731208in}}%
\pgfpathlineto{\pgfqpoint{4.328286in}{2.722495in}}%
\pgfpathlineto{\pgfqpoint{4.320690in}{2.713789in}}%
\pgfpathlineto{\pgfqpoint{4.313090in}{2.705089in}}%
\pgfpathlineto{\pgfqpoint{4.305484in}{2.696394in}}%
\pgfpathlineto{\pgfqpoint{4.292120in}{2.698853in}}%
\pgfpathlineto{\pgfqpoint{4.278762in}{2.701427in}}%
\pgfpathlineto{\pgfqpoint{4.265411in}{2.704115in}}%
\pgfpathlineto{\pgfqpoint{4.252068in}{2.706920in}}%
\pgfpathlineto{\pgfqpoint{4.259684in}{2.715666in}}%
\pgfpathlineto{\pgfqpoint{4.267296in}{2.724421in}}%
\pgfpathlineto{\pgfqpoint{4.274903in}{2.733186in}}%
\pgfpathlineto{\pgfqpoint{4.282504in}{2.741963in}}%
\pgfpathclose%
\pgfusepath{fill}%
\end{pgfscope}%
\begin{pgfscope}%
\pgfpathrectangle{\pgfqpoint{1.150000in}{0.150000in}}{\pgfqpoint{5.700000in}{5.700000in}}%
\pgfusepath{clip}%
\pgfsetbuttcap%
\pgfsetroundjoin%
\definecolor{currentfill}{rgb}{0.283072,0.130895,0.449241}%
\pgfsetfillcolor{currentfill}%
\pgfsetfillopacity{0.700000}%
\pgfsetlinewidth{0.000000pt}%
\definecolor{currentstroke}{rgb}{0.000000,0.000000,0.000000}%
\pgfsetstrokecolor{currentstroke}%
\pgfsetdash{}{0pt}%
\pgfpathmoveto{\pgfqpoint{4.586989in}{2.811936in}}%
\pgfpathlineto{\pgfqpoint{4.600407in}{2.810590in}}%
\pgfpathlineto{\pgfqpoint{4.613834in}{2.809353in}}%
\pgfpathlineto{\pgfqpoint{4.627268in}{2.808224in}}%
\pgfpathlineto{\pgfqpoint{4.640712in}{2.807203in}}%
\pgfpathlineto{\pgfqpoint{4.633226in}{2.798787in}}%
\pgfpathlineto{\pgfqpoint{4.625736in}{2.790376in}}%
\pgfpathlineto{\pgfqpoint{4.618241in}{2.781967in}}%
\pgfpathlineto{\pgfqpoint{4.610740in}{2.773559in}}%
\pgfpathlineto{\pgfqpoint{4.597286in}{2.774468in}}%
\pgfpathlineto{\pgfqpoint{4.583839in}{2.775486in}}%
\pgfpathlineto{\pgfqpoint{4.570401in}{2.776611in}}%
\pgfpathlineto{\pgfqpoint{4.556971in}{2.777846in}}%
\pgfpathlineto{\pgfqpoint{4.564483in}{2.786359in}}%
\pgfpathlineto{\pgfqpoint{4.571990in}{2.794877in}}%
\pgfpathlineto{\pgfqpoint{4.579492in}{2.803402in}}%
\pgfpathlineto{\pgfqpoint{4.586989in}{2.811936in}}%
\pgfpathclose%
\pgfusepath{fill}%
\end{pgfscope}%
\begin{pgfscope}%
\pgfpathrectangle{\pgfqpoint{1.150000in}{0.150000in}}{\pgfqpoint{5.700000in}{5.700000in}}%
\pgfusepath{clip}%
\pgfsetbuttcap%
\pgfsetroundjoin%
\definecolor{currentfill}{rgb}{0.275191,0.194905,0.496005}%
\pgfsetfillcolor{currentfill}%
\pgfsetfillopacity{0.700000}%
\pgfsetlinewidth{0.000000pt}%
\definecolor{currentstroke}{rgb}{0.000000,0.000000,0.000000}%
\pgfsetstrokecolor{currentstroke}%
\pgfsetdash{}{0pt}%
\pgfpathmoveto{\pgfqpoint{4.975307in}{2.931953in}}%
\pgfpathlineto{\pgfqpoint{4.988854in}{2.932095in}}%
\pgfpathlineto{\pgfqpoint{5.002409in}{2.932341in}}%
\pgfpathlineto{\pgfqpoint{5.015975in}{2.932688in}}%
\pgfpathlineto{\pgfqpoint{5.029550in}{2.933138in}}%
\pgfpathlineto{\pgfqpoint{5.022208in}{2.925360in}}%
\pgfpathlineto{\pgfqpoint{5.014860in}{2.917595in}}%
\pgfpathlineto{\pgfqpoint{5.007508in}{2.909840in}}%
\pgfpathlineto{\pgfqpoint{5.000150in}{2.902094in}}%
\pgfpathlineto{\pgfqpoint{4.986561in}{2.901459in}}%
\pgfpathlineto{\pgfqpoint{4.972982in}{2.900927in}}%
\pgfpathlineto{\pgfqpoint{4.959413in}{2.900498in}}%
\pgfpathlineto{\pgfqpoint{4.945853in}{2.900172in}}%
\pgfpathlineto{\pgfqpoint{4.953224in}{2.908096in}}%
\pgfpathlineto{\pgfqpoint{4.960590in}{2.916033in}}%
\pgfpathlineto{\pgfqpoint{4.967951in}{2.923984in}}%
\pgfpathlineto{\pgfqpoint{4.975307in}{2.931953in}}%
\pgfpathclose%
\pgfusepath{fill}%
\end{pgfscope}%
\begin{pgfscope}%
\pgfpathrectangle{\pgfqpoint{1.150000in}{0.150000in}}{\pgfqpoint{5.700000in}{5.700000in}}%
\pgfusepath{clip}%
\pgfsetbuttcap%
\pgfsetroundjoin%
\definecolor{currentfill}{rgb}{0.267968,0.223549,0.512008}%
\pgfsetfillcolor{currentfill}%
\pgfsetfillopacity{0.700000}%
\pgfsetlinewidth{0.000000pt}%
\definecolor{currentstroke}{rgb}{0.000000,0.000000,0.000000}%
\pgfsetstrokecolor{currentstroke}%
\pgfsetdash{}{0pt}%
\pgfpathmoveto{\pgfqpoint{3.080406in}{3.021146in}}%
\pgfpathlineto{\pgfqpoint{3.093655in}{3.007888in}}%
\pgfpathlineto{\pgfqpoint{3.106902in}{2.994814in}}%
\pgfpathlineto{\pgfqpoint{3.120146in}{2.981924in}}%
\pgfpathlineto{\pgfqpoint{3.133389in}{2.969216in}}%
\pgfpathlineto{\pgfqpoint{3.125371in}{2.961932in}}%
\pgfpathlineto{\pgfqpoint{3.117346in}{2.954727in}}%
\pgfpathlineto{\pgfqpoint{3.109314in}{2.947604in}}%
\pgfpathlineto{\pgfqpoint{3.101274in}{2.940561in}}%
\pgfpathlineto{\pgfqpoint{3.088012in}{2.953361in}}%
\pgfpathlineto{\pgfqpoint{3.074748in}{2.966343in}}%
\pgfpathlineto{\pgfqpoint{3.061482in}{2.979509in}}%
\pgfpathlineto{\pgfqpoint{3.048213in}{2.992860in}}%
\pgfpathlineto{\pgfqpoint{3.056273in}{2.999803in}}%
\pgfpathlineto{\pgfqpoint{3.064325in}{3.006833in}}%
\pgfpathlineto{\pgfqpoint{3.072369in}{3.013947in}}%
\pgfpathlineto{\pgfqpoint{3.080406in}{3.021146in}}%
\pgfpathclose%
\pgfusepath{fill}%
\end{pgfscope}%
\begin{pgfscope}%
\pgfpathrectangle{\pgfqpoint{1.150000in}{0.150000in}}{\pgfqpoint{5.700000in}{5.700000in}}%
\pgfusepath{clip}%
\pgfsetbuttcap%
\pgfsetroundjoin%
\definecolor{currentfill}{rgb}{0.280894,0.078907,0.402329}%
\pgfsetfillcolor{currentfill}%
\pgfsetfillopacity{0.700000}%
\pgfsetlinewidth{0.000000pt}%
\definecolor{currentstroke}{rgb}{0.000000,0.000000,0.000000}%
\pgfsetstrokecolor{currentstroke}%
\pgfsetdash{}{0pt}%
\pgfpathmoveto{\pgfqpoint{3.704010in}{2.720212in}}%
\pgfpathlineto{\pgfqpoint{3.717238in}{2.713449in}}%
\pgfpathlineto{\pgfqpoint{3.730469in}{2.706823in}}%
\pgfpathlineto{\pgfqpoint{3.743704in}{2.700332in}}%
\pgfpathlineto{\pgfqpoint{3.756942in}{2.693976in}}%
\pgfpathlineto{\pgfqpoint{3.749154in}{2.685477in}}%
\pgfpathlineto{\pgfqpoint{3.741361in}{2.677010in}}%
\pgfpathlineto{\pgfqpoint{3.733562in}{2.668575in}}%
\pgfpathlineto{\pgfqpoint{3.725758in}{2.660172in}}%
\pgfpathlineto{\pgfqpoint{3.712506in}{2.666562in}}%
\pgfpathlineto{\pgfqpoint{3.699259in}{2.673086in}}%
\pgfpathlineto{\pgfqpoint{3.686014in}{2.679745in}}%
\pgfpathlineto{\pgfqpoint{3.672773in}{2.686541in}}%
\pgfpathlineto{\pgfqpoint{3.680591in}{2.694903in}}%
\pgfpathlineto{\pgfqpoint{3.688403in}{2.703303in}}%
\pgfpathlineto{\pgfqpoint{3.696210in}{2.711739in}}%
\pgfpathlineto{\pgfqpoint{3.704010in}{2.720212in}}%
\pgfpathclose%
\pgfusepath{fill}%
\end{pgfscope}%
\begin{pgfscope}%
\pgfpathrectangle{\pgfqpoint{1.150000in}{0.150000in}}{\pgfqpoint{5.700000in}{5.700000in}}%
\pgfusepath{clip}%
\pgfsetbuttcap%
\pgfsetroundjoin%
\definecolor{currentfill}{rgb}{0.283187,0.125848,0.444960}%
\pgfsetfillcolor{currentfill}%
\pgfsetfillopacity{0.700000}%
\pgfsetlinewidth{0.000000pt}%
\definecolor{currentstroke}{rgb}{0.000000,0.000000,0.000000}%
\pgfsetstrokecolor{currentstroke}%
\pgfsetdash{}{0pt}%
\pgfpathmoveto{\pgfqpoint{3.376845in}{2.820447in}}%
\pgfpathlineto{\pgfqpoint{3.390061in}{2.810674in}}%
\pgfpathlineto{\pgfqpoint{3.403277in}{2.801057in}}%
\pgfpathlineto{\pgfqpoint{3.416494in}{2.791596in}}%
\pgfpathlineto{\pgfqpoint{3.429712in}{2.782288in}}%
\pgfpathlineto{\pgfqpoint{3.421807in}{2.774363in}}%
\pgfpathlineto{\pgfqpoint{3.413895in}{2.766493in}}%
\pgfpathlineto{\pgfqpoint{3.405977in}{2.758680in}}%
\pgfpathlineto{\pgfqpoint{3.398053in}{2.750923in}}%
\pgfpathlineto{\pgfqpoint{3.384819in}{2.760301in}}%
\pgfpathlineto{\pgfqpoint{3.371586in}{2.769833in}}%
\pgfpathlineto{\pgfqpoint{3.358354in}{2.779520in}}%
\pgfpathlineto{\pgfqpoint{3.345122in}{2.789365in}}%
\pgfpathlineto{\pgfqpoint{3.353063in}{2.797044in}}%
\pgfpathlineto{\pgfqpoint{3.360997in}{2.804785in}}%
\pgfpathlineto{\pgfqpoint{3.368924in}{2.812586in}}%
\pgfpathlineto{\pgfqpoint{3.376845in}{2.820447in}}%
\pgfpathclose%
\pgfusepath{fill}%
\end{pgfscope}%
\begin{pgfscope}%
\pgfpathrectangle{\pgfqpoint{1.150000in}{0.150000in}}{\pgfqpoint{5.700000in}{5.700000in}}%
\pgfusepath{clip}%
\pgfsetbuttcap%
\pgfsetroundjoin%
\definecolor{currentfill}{rgb}{0.279566,0.067836,0.391917}%
\pgfsetfillcolor{currentfill}%
\pgfsetfillopacity{0.700000}%
\pgfsetlinewidth{0.000000pt}%
\definecolor{currentstroke}{rgb}{0.000000,0.000000,0.000000}%
\pgfsetstrokecolor{currentstroke}%
\pgfsetdash{}{0pt}%
\pgfpathmoveto{\pgfqpoint{3.840979in}{2.704262in}}%
\pgfpathlineto{\pgfqpoint{3.854225in}{2.698582in}}%
\pgfpathlineto{\pgfqpoint{3.867474in}{2.693032in}}%
\pgfpathlineto{\pgfqpoint{3.880729in}{2.687610in}}%
\pgfpathlineto{\pgfqpoint{3.893988in}{2.682318in}}%
\pgfpathlineto{\pgfqpoint{3.886246in}{2.673672in}}%
\pgfpathlineto{\pgfqpoint{3.878500in}{2.665049in}}%
\pgfpathlineto{\pgfqpoint{3.870748in}{2.656451in}}%
\pgfpathlineto{\pgfqpoint{3.862991in}{2.647876in}}%
\pgfpathlineto{\pgfqpoint{3.849720in}{2.653184in}}%
\pgfpathlineto{\pgfqpoint{3.836453in}{2.658620in}}%
\pgfpathlineto{\pgfqpoint{3.823191in}{2.664185in}}%
\pgfpathlineto{\pgfqpoint{3.809933in}{2.669881in}}%
\pgfpathlineto{\pgfqpoint{3.817703in}{2.678434in}}%
\pgfpathlineto{\pgfqpoint{3.825467in}{2.687015in}}%
\pgfpathlineto{\pgfqpoint{3.833226in}{2.695625in}}%
\pgfpathlineto{\pgfqpoint{3.840979in}{2.704262in}}%
\pgfpathclose%
\pgfusepath{fill}%
\end{pgfscope}%
\begin{pgfscope}%
\pgfpathrectangle{\pgfqpoint{1.150000in}{0.150000in}}{\pgfqpoint{5.700000in}{5.700000in}}%
\pgfusepath{clip}%
\pgfsetbuttcap%
\pgfsetroundjoin%
\definecolor{currentfill}{rgb}{0.283197,0.115680,0.436115}%
\pgfsetfillcolor{currentfill}%
\pgfsetfillopacity{0.700000}%
\pgfsetlinewidth{0.000000pt}%
\definecolor{currentstroke}{rgb}{0.000000,0.000000,0.000000}%
\pgfsetstrokecolor{currentstroke}%
\pgfsetdash{}{0pt}%
\pgfpathmoveto{\pgfqpoint{4.503332in}{2.783880in}}%
\pgfpathlineto{\pgfqpoint{4.516730in}{2.782207in}}%
\pgfpathlineto{\pgfqpoint{4.530136in}{2.780643in}}%
\pgfpathlineto{\pgfqpoint{4.543550in}{2.779190in}}%
\pgfpathlineto{\pgfqpoint{4.556971in}{2.777846in}}%
\pgfpathlineto{\pgfqpoint{4.549454in}{2.769338in}}%
\pgfpathlineto{\pgfqpoint{4.541933in}{2.760833in}}%
\pgfpathlineto{\pgfqpoint{4.534406in}{2.752329in}}%
\pgfpathlineto{\pgfqpoint{4.526873in}{2.743826in}}%
\pgfpathlineto{\pgfqpoint{4.513440in}{2.745076in}}%
\pgfpathlineto{\pgfqpoint{4.500016in}{2.746436in}}%
\pgfpathlineto{\pgfqpoint{4.486599in}{2.747906in}}%
\pgfpathlineto{\pgfqpoint{4.473189in}{2.749486in}}%
\pgfpathlineto{\pgfqpoint{4.480733in}{2.758076in}}%
\pgfpathlineto{\pgfqpoint{4.488271in}{2.766671in}}%
\pgfpathlineto{\pgfqpoint{4.495804in}{2.775272in}}%
\pgfpathlineto{\pgfqpoint{4.503332in}{2.783880in}}%
\pgfpathclose%
\pgfusepath{fill}%
\end{pgfscope}%
\begin{pgfscope}%
\pgfpathrectangle{\pgfqpoint{1.150000in}{0.150000in}}{\pgfqpoint{5.700000in}{5.700000in}}%
\pgfusepath{clip}%
\pgfsetbuttcap%
\pgfsetroundjoin%
\definecolor{currentfill}{rgb}{0.274128,0.199721,0.498911}%
\pgfsetfillcolor{currentfill}%
\pgfsetfillopacity{0.700000}%
\pgfsetlinewidth{0.000000pt}%
\definecolor{currentstroke}{rgb}{0.000000,0.000000,0.000000}%
\pgfsetstrokecolor{currentstroke}%
\pgfsetdash{}{0pt}%
\pgfpathmoveto{\pgfqpoint{3.133389in}{2.969216in}}%
\pgfpathlineto{\pgfqpoint{3.146630in}{2.956688in}}%
\pgfpathlineto{\pgfqpoint{3.159869in}{2.944338in}}%
\pgfpathlineto{\pgfqpoint{3.173106in}{2.932165in}}%
\pgfpathlineto{\pgfqpoint{3.186342in}{2.920168in}}%
\pgfpathlineto{\pgfqpoint{3.178343in}{2.912799in}}%
\pgfpathlineto{\pgfqpoint{3.170336in}{2.905506in}}%
\pgfpathlineto{\pgfqpoint{3.162323in}{2.898289in}}%
\pgfpathlineto{\pgfqpoint{3.154302in}{2.891148in}}%
\pgfpathlineto{\pgfqpoint{3.141047in}{2.903237in}}%
\pgfpathlineto{\pgfqpoint{3.127791in}{2.915500in}}%
\pgfpathlineto{\pgfqpoint{3.114533in}{2.927941in}}%
\pgfpathlineto{\pgfqpoint{3.101274in}{2.940561in}}%
\pgfpathlineto{\pgfqpoint{3.109314in}{2.947604in}}%
\pgfpathlineto{\pgfqpoint{3.117346in}{2.954727in}}%
\pgfpathlineto{\pgfqpoint{3.125371in}{2.961932in}}%
\pgfpathlineto{\pgfqpoint{3.133389in}{2.969216in}}%
\pgfpathclose%
\pgfusepath{fill}%
\end{pgfscope}%
\begin{pgfscope}%
\pgfpathrectangle{\pgfqpoint{1.150000in}{0.150000in}}{\pgfqpoint{5.700000in}{5.700000in}}%
\pgfusepath{clip}%
\pgfsetbuttcap%
\pgfsetroundjoin%
\definecolor{currentfill}{rgb}{0.278012,0.180367,0.486697}%
\pgfsetfillcolor{currentfill}%
\pgfsetfillopacity{0.700000}%
\pgfsetlinewidth{0.000000pt}%
\definecolor{currentstroke}{rgb}{0.000000,0.000000,0.000000}%
\pgfsetstrokecolor{currentstroke}%
\pgfsetdash{}{0pt}%
\pgfpathmoveto{\pgfqpoint{4.891710in}{2.899899in}}%
\pgfpathlineto{\pgfqpoint{4.905231in}{2.899812in}}%
\pgfpathlineto{\pgfqpoint{4.918762in}{2.899828in}}%
\pgfpathlineto{\pgfqpoint{4.932303in}{2.899948in}}%
\pgfpathlineto{\pgfqpoint{4.945853in}{2.900172in}}%
\pgfpathlineto{\pgfqpoint{4.938477in}{2.892258in}}%
\pgfpathlineto{\pgfqpoint{4.931096in}{2.884353in}}%
\pgfpathlineto{\pgfqpoint{4.923710in}{2.876454in}}%
\pgfpathlineto{\pgfqpoint{4.916320in}{2.868560in}}%
\pgfpathlineto{\pgfqpoint{4.902756in}{2.868170in}}%
\pgfpathlineto{\pgfqpoint{4.889203in}{2.867884in}}%
\pgfpathlineto{\pgfqpoint{4.875659in}{2.867701in}}%
\pgfpathlineto{\pgfqpoint{4.862124in}{2.867623in}}%
\pgfpathlineto{\pgfqpoint{4.869528in}{2.875677in}}%
\pgfpathlineto{\pgfqpoint{4.876927in}{2.883739in}}%
\pgfpathlineto{\pgfqpoint{4.884321in}{2.891813in}}%
\pgfpathlineto{\pgfqpoint{4.891710in}{2.899899in}}%
\pgfpathclose%
\pgfusepath{fill}%
\end{pgfscope}%
\begin{pgfscope}%
\pgfpathrectangle{\pgfqpoint{1.150000in}{0.150000in}}{\pgfqpoint{5.700000in}{5.700000in}}%
\pgfusepath{clip}%
\pgfsetbuttcap%
\pgfsetroundjoin%
\definecolor{currentfill}{rgb}{0.281924,0.089666,0.412415}%
\pgfsetfillcolor{currentfill}%
\pgfsetfillopacity{0.700000}%
\pgfsetlinewidth{0.000000pt}%
\definecolor{currentstroke}{rgb}{0.000000,0.000000,0.000000}%
\pgfsetstrokecolor{currentstroke}%
\pgfsetdash{}{0pt}%
\pgfpathmoveto{\pgfqpoint{3.566950in}{2.745916in}}%
\pgfpathlineto{\pgfqpoint{3.580169in}{2.737998in}}%
\pgfpathlineto{\pgfqpoint{3.593390in}{2.730224in}}%
\pgfpathlineto{\pgfqpoint{3.606614in}{2.722593in}}%
\pgfpathlineto{\pgfqpoint{3.619840in}{2.715103in}}%
\pgfpathlineto{\pgfqpoint{3.612003in}{2.706821in}}%
\pgfpathlineto{\pgfqpoint{3.604160in}{2.698581in}}%
\pgfpathlineto{\pgfqpoint{3.596311in}{2.690383in}}%
\pgfpathlineto{\pgfqpoint{3.588456in}{2.682226in}}%
\pgfpathlineto{\pgfqpoint{3.575216in}{2.689767in}}%
\pgfpathlineto{\pgfqpoint{3.561978in}{2.697450in}}%
\pgfpathlineto{\pgfqpoint{3.548743in}{2.705276in}}%
\pgfpathlineto{\pgfqpoint{3.535510in}{2.713245in}}%
\pgfpathlineto{\pgfqpoint{3.543379in}{2.721344in}}%
\pgfpathlineto{\pgfqpoint{3.551242in}{2.729489in}}%
\pgfpathlineto{\pgfqpoint{3.559099in}{2.737679in}}%
\pgfpathlineto{\pgfqpoint{3.566950in}{2.745916in}}%
\pgfpathclose%
\pgfusepath{fill}%
\end{pgfscope}%
\begin{pgfscope}%
\pgfpathrectangle{\pgfqpoint{1.150000in}{0.150000in}}{\pgfqpoint{5.700000in}{5.700000in}}%
\pgfusepath{clip}%
\pgfsetbuttcap%
\pgfsetroundjoin%
\definecolor{currentfill}{rgb}{0.281446,0.084320,0.407414}%
\pgfsetfillcolor{currentfill}%
\pgfsetfillopacity{0.700000}%
\pgfsetlinewidth{0.000000pt}%
\definecolor{currentstroke}{rgb}{0.000000,0.000000,0.000000}%
\pgfsetstrokecolor{currentstroke}%
\pgfsetdash{}{0pt}%
\pgfpathmoveto{\pgfqpoint{4.198758in}{2.719302in}}%
\pgfpathlineto{\pgfqpoint{4.212076in}{2.716031in}}%
\pgfpathlineto{\pgfqpoint{4.225400in}{2.712877in}}%
\pgfpathlineto{\pgfqpoint{4.238731in}{2.709840in}}%
\pgfpathlineto{\pgfqpoint{4.252068in}{2.706920in}}%
\pgfpathlineto{\pgfqpoint{4.244446in}{2.698183in}}%
\pgfpathlineto{\pgfqpoint{4.236819in}{2.689453in}}%
\pgfpathlineto{\pgfqpoint{4.229187in}{2.680731in}}%
\pgfpathlineto{\pgfqpoint{4.221550in}{2.672015in}}%
\pgfpathlineto{\pgfqpoint{4.208201in}{2.674896in}}%
\pgfpathlineto{\pgfqpoint{4.194860in}{2.677893in}}%
\pgfpathlineto{\pgfqpoint{4.181525in}{2.681008in}}%
\pgfpathlineto{\pgfqpoint{4.168196in}{2.684240in}}%
\pgfpathlineto{\pgfqpoint{4.175844in}{2.692989in}}%
\pgfpathlineto{\pgfqpoint{4.183487in}{2.701748in}}%
\pgfpathlineto{\pgfqpoint{4.191125in}{2.710519in}}%
\pgfpathlineto{\pgfqpoint{4.198758in}{2.719302in}}%
\pgfpathclose%
\pgfusepath{fill}%
\end{pgfscope}%
\begin{pgfscope}%
\pgfpathrectangle{\pgfqpoint{1.150000in}{0.150000in}}{\pgfqpoint{5.700000in}{5.700000in}}%
\pgfusepath{clip}%
\pgfsetbuttcap%
\pgfsetroundjoin%
\definecolor{currentfill}{rgb}{0.279566,0.067836,0.391917}%
\pgfsetfillcolor{currentfill}%
\pgfsetfillopacity{0.700000}%
\pgfsetlinewidth{0.000000pt}%
\definecolor{currentstroke}{rgb}{0.000000,0.000000,0.000000}%
\pgfsetstrokecolor{currentstroke}%
\pgfsetdash{}{0pt}%
\pgfpathmoveto{\pgfqpoint{3.977934in}{2.697232in}}%
\pgfpathlineto{\pgfqpoint{3.991205in}{2.692568in}}%
\pgfpathlineto{\pgfqpoint{4.004481in}{2.688028in}}%
\pgfpathlineto{\pgfqpoint{4.017763in}{2.683612in}}%
\pgfpathlineto{\pgfqpoint{4.031050in}{2.679319in}}%
\pgfpathlineto{\pgfqpoint{4.023354in}{2.670595in}}%
\pgfpathlineto{\pgfqpoint{4.015652in}{2.661886in}}%
\pgfpathlineto{\pgfqpoint{4.007945in}{2.653194in}}%
\pgfpathlineto{\pgfqpoint{4.000233in}{2.644517in}}%
\pgfpathlineto{\pgfqpoint{3.986934in}{2.648806in}}%
\pgfpathlineto{\pgfqpoint{3.973641in}{2.653219in}}%
\pgfpathlineto{\pgfqpoint{3.960353in}{2.657755in}}%
\pgfpathlineto{\pgfqpoint{3.947070in}{2.662416in}}%
\pgfpathlineto{\pgfqpoint{3.954794in}{2.671090in}}%
\pgfpathlineto{\pgfqpoint{3.962512in}{2.679784in}}%
\pgfpathlineto{\pgfqpoint{3.970226in}{2.688498in}}%
\pgfpathlineto{\pgfqpoint{3.977934in}{2.697232in}}%
\pgfpathclose%
\pgfusepath{fill}%
\end{pgfscope}%
\begin{pgfscope}%
\pgfpathrectangle{\pgfqpoint{1.150000in}{0.150000in}}{\pgfqpoint{5.700000in}{5.700000in}}%
\pgfusepath{clip}%
\pgfsetbuttcap%
\pgfsetroundjoin%
\definecolor{currentfill}{rgb}{0.280255,0.165693,0.476498}%
\pgfsetfillcolor{currentfill}%
\pgfsetfillopacity{0.700000}%
\pgfsetlinewidth{0.000000pt}%
\definecolor{currentstroke}{rgb}{0.000000,0.000000,0.000000}%
\pgfsetstrokecolor{currentstroke}%
\pgfsetdash{}{0pt}%
\pgfpathmoveto{\pgfqpoint{4.808079in}{2.868352in}}%
\pgfpathlineto{\pgfqpoint{4.821576in}{2.868013in}}%
\pgfpathlineto{\pgfqpoint{4.835083in}{2.867778in}}%
\pgfpathlineto{\pgfqpoint{4.848599in}{2.867648in}}%
\pgfpathlineto{\pgfqpoint{4.862124in}{2.867623in}}%
\pgfpathlineto{\pgfqpoint{4.854715in}{2.859575in}}%
\pgfpathlineto{\pgfqpoint{4.847302in}{2.851533in}}%
\pgfpathlineto{\pgfqpoint{4.839883in}{2.843493in}}%
\pgfpathlineto{\pgfqpoint{4.832459in}{2.835455in}}%
\pgfpathlineto{\pgfqpoint{4.818921in}{2.835332in}}%
\pgfpathlineto{\pgfqpoint{4.805393in}{2.835314in}}%
\pgfpathlineto{\pgfqpoint{4.791874in}{2.835401in}}%
\pgfpathlineto{\pgfqpoint{4.778363in}{2.835592in}}%
\pgfpathlineto{\pgfqpoint{4.785800in}{2.843772in}}%
\pgfpathlineto{\pgfqpoint{4.793231in}{2.851958in}}%
\pgfpathlineto{\pgfqpoint{4.800657in}{2.860150in}}%
\pgfpathlineto{\pgfqpoint{4.808079in}{2.868352in}}%
\pgfpathclose%
\pgfusepath{fill}%
\end{pgfscope}%
\begin{pgfscope}%
\pgfpathrectangle{\pgfqpoint{1.150000in}{0.150000in}}{\pgfqpoint{5.700000in}{5.700000in}}%
\pgfusepath{clip}%
\pgfsetbuttcap%
\pgfsetroundjoin%
\definecolor{currentfill}{rgb}{0.278012,0.180367,0.486697}%
\pgfsetfillcolor{currentfill}%
\pgfsetfillopacity{0.700000}%
\pgfsetlinewidth{0.000000pt}%
\definecolor{currentstroke}{rgb}{0.000000,0.000000,0.000000}%
\pgfsetstrokecolor{currentstroke}%
\pgfsetdash{}{0pt}%
\pgfpathmoveto{\pgfqpoint{3.186342in}{2.920168in}}%
\pgfpathlineto{\pgfqpoint{3.199577in}{2.908345in}}%
\pgfpathlineto{\pgfqpoint{3.212811in}{2.896694in}}%
\pgfpathlineto{\pgfqpoint{3.226044in}{2.885214in}}%
\pgfpathlineto{\pgfqpoint{3.239276in}{2.873904in}}%
\pgfpathlineto{\pgfqpoint{3.231294in}{2.866452in}}%
\pgfpathlineto{\pgfqpoint{3.223306in}{2.859070in}}%
\pgfpathlineto{\pgfqpoint{3.215310in}{2.851760in}}%
\pgfpathlineto{\pgfqpoint{3.207308in}{2.844522in}}%
\pgfpathlineto{\pgfqpoint{3.194058in}{2.855923in}}%
\pgfpathlineto{\pgfqpoint{3.180807in}{2.867493in}}%
\pgfpathlineto{\pgfqpoint{3.167555in}{2.879234in}}%
\pgfpathlineto{\pgfqpoint{3.154302in}{2.891148in}}%
\pgfpathlineto{\pgfqpoint{3.162323in}{2.898289in}}%
\pgfpathlineto{\pgfqpoint{3.170336in}{2.905506in}}%
\pgfpathlineto{\pgfqpoint{3.178343in}{2.912799in}}%
\pgfpathlineto{\pgfqpoint{3.186342in}{2.920168in}}%
\pgfpathclose%
\pgfusepath{fill}%
\end{pgfscope}%
\begin{pgfscope}%
\pgfpathrectangle{\pgfqpoint{1.150000in}{0.150000in}}{\pgfqpoint{5.700000in}{5.700000in}}%
\pgfusepath{clip}%
\pgfsetbuttcap%
\pgfsetroundjoin%
\definecolor{currentfill}{rgb}{0.282910,0.105393,0.426902}%
\pgfsetfillcolor{currentfill}%
\pgfsetfillopacity{0.700000}%
\pgfsetlinewidth{0.000000pt}%
\definecolor{currentstroke}{rgb}{0.000000,0.000000,0.000000}%
\pgfsetstrokecolor{currentstroke}%
\pgfsetdash{}{0pt}%
\pgfpathmoveto{\pgfqpoint{4.419630in}{2.756919in}}%
\pgfpathlineto{\pgfqpoint{4.433008in}{2.754893in}}%
\pgfpathlineto{\pgfqpoint{4.446394in}{2.752979in}}%
\pgfpathlineto{\pgfqpoint{4.459788in}{2.751177in}}%
\pgfpathlineto{\pgfqpoint{4.473189in}{2.749486in}}%
\pgfpathlineto{\pgfqpoint{4.465641in}{2.740899in}}%
\pgfpathlineto{\pgfqpoint{4.458088in}{2.732315in}}%
\pgfpathlineto{\pgfqpoint{4.450529in}{2.723731in}}%
\pgfpathlineto{\pgfqpoint{4.442966in}{2.715148in}}%
\pgfpathlineto{\pgfqpoint{4.429553in}{2.716763in}}%
\pgfpathlineto{\pgfqpoint{4.416149in}{2.718490in}}%
\pgfpathlineto{\pgfqpoint{4.402752in}{2.720328in}}%
\pgfpathlineto{\pgfqpoint{4.389362in}{2.722279in}}%
\pgfpathlineto{\pgfqpoint{4.396936in}{2.730931in}}%
\pgfpathlineto{\pgfqpoint{4.404506in}{2.739587in}}%
\pgfpathlineto{\pgfqpoint{4.412070in}{2.748250in}}%
\pgfpathlineto{\pgfqpoint{4.419630in}{2.756919in}}%
\pgfpathclose%
\pgfusepath{fill}%
\end{pgfscope}%
\begin{pgfscope}%
\pgfpathrectangle{\pgfqpoint{1.150000in}{0.150000in}}{\pgfqpoint{5.700000in}{5.700000in}}%
\pgfusepath{clip}%
\pgfsetbuttcap%
\pgfsetroundjoin%
\definecolor{currentfill}{rgb}{0.283197,0.115680,0.436115}%
\pgfsetfillcolor{currentfill}%
\pgfsetfillopacity{0.700000}%
\pgfsetlinewidth{0.000000pt}%
\definecolor{currentstroke}{rgb}{0.000000,0.000000,0.000000}%
\pgfsetstrokecolor{currentstroke}%
\pgfsetdash{}{0pt}%
\pgfpathmoveto{\pgfqpoint{3.429712in}{2.782288in}}%
\pgfpathlineto{\pgfqpoint{3.442932in}{2.773134in}}%
\pgfpathlineto{\pgfqpoint{3.456152in}{2.764131in}}%
\pgfpathlineto{\pgfqpoint{3.469374in}{2.755280in}}%
\pgfpathlineto{\pgfqpoint{3.482598in}{2.746578in}}%
\pgfpathlineto{\pgfqpoint{3.474708in}{2.738589in}}%
\pgfpathlineto{\pgfqpoint{3.466811in}{2.730651in}}%
\pgfpathlineto{\pgfqpoint{3.458909in}{2.722765in}}%
\pgfpathlineto{\pgfqpoint{3.451000in}{2.714931in}}%
\pgfpathlineto{\pgfqpoint{3.437761in}{2.723703in}}%
\pgfpathlineto{\pgfqpoint{3.424524in}{2.732625in}}%
\pgfpathlineto{\pgfqpoint{3.411288in}{2.741698in}}%
\pgfpathlineto{\pgfqpoint{3.398053in}{2.750923in}}%
\pgfpathlineto{\pgfqpoint{3.405977in}{2.758680in}}%
\pgfpathlineto{\pgfqpoint{3.413895in}{2.766493in}}%
\pgfpathlineto{\pgfqpoint{3.421807in}{2.774363in}}%
\pgfpathlineto{\pgfqpoint{3.429712in}{2.782288in}}%
\pgfpathclose%
\pgfusepath{fill}%
\end{pgfscope}%
\begin{pgfscope}%
\pgfpathrectangle{\pgfqpoint{1.150000in}{0.150000in}}{\pgfqpoint{5.700000in}{5.700000in}}%
\pgfusepath{clip}%
\pgfsetbuttcap%
\pgfsetroundjoin%
\definecolor{currentfill}{rgb}{0.281887,0.150881,0.465405}%
\pgfsetfillcolor{currentfill}%
\pgfsetfillopacity{0.700000}%
\pgfsetlinewidth{0.000000pt}%
\definecolor{currentstroke}{rgb}{0.000000,0.000000,0.000000}%
\pgfsetstrokecolor{currentstroke}%
\pgfsetdash{}{0pt}%
\pgfpathmoveto{\pgfqpoint{4.724413in}{2.837415in}}%
\pgfpathlineto{\pgfqpoint{4.737887in}{2.836800in}}%
\pgfpathlineto{\pgfqpoint{4.751371in}{2.836292in}}%
\pgfpathlineto{\pgfqpoint{4.764862in}{2.835889in}}%
\pgfpathlineto{\pgfqpoint{4.778363in}{2.835592in}}%
\pgfpathlineto{\pgfqpoint{4.770922in}{2.827416in}}%
\pgfpathlineto{\pgfqpoint{4.763476in}{2.819242in}}%
\pgfpathlineto{\pgfqpoint{4.756024in}{2.811068in}}%
\pgfpathlineto{\pgfqpoint{4.748568in}{2.802893in}}%
\pgfpathlineto{\pgfqpoint{4.735055in}{2.803060in}}%
\pgfpathlineto{\pgfqpoint{4.721551in}{2.803332in}}%
\pgfpathlineto{\pgfqpoint{4.708056in}{2.803711in}}%
\pgfpathlineto{\pgfqpoint{4.694569in}{2.804195in}}%
\pgfpathlineto{\pgfqpoint{4.702038in}{2.812494in}}%
\pgfpathlineto{\pgfqpoint{4.709501in}{2.820796in}}%
\pgfpathlineto{\pgfqpoint{4.716960in}{2.829102in}}%
\pgfpathlineto{\pgfqpoint{4.724413in}{2.837415in}}%
\pgfpathclose%
\pgfusepath{fill}%
\end{pgfscope}%
\begin{pgfscope}%
\pgfpathrectangle{\pgfqpoint{1.150000in}{0.150000in}}{\pgfqpoint{5.700000in}{5.700000in}}%
\pgfusepath{clip}%
\pgfsetbuttcap%
\pgfsetroundjoin%
\definecolor{currentfill}{rgb}{0.280267,0.073417,0.397163}%
\pgfsetfillcolor{currentfill}%
\pgfsetfillopacity{0.700000}%
\pgfsetlinewidth{0.000000pt}%
\definecolor{currentstroke}{rgb}{0.000000,0.000000,0.000000}%
\pgfsetstrokecolor{currentstroke}%
\pgfsetdash{}{0pt}%
\pgfpathmoveto{\pgfqpoint{4.114943in}{2.698354in}}%
\pgfpathlineto{\pgfqpoint{4.128247in}{2.694646in}}%
\pgfpathlineto{\pgfqpoint{4.141557in}{2.691058in}}%
\pgfpathlineto{\pgfqpoint{4.154874in}{2.687590in}}%
\pgfpathlineto{\pgfqpoint{4.168196in}{2.684240in}}%
\pgfpathlineto{\pgfqpoint{4.160543in}{2.675501in}}%
\pgfpathlineto{\pgfqpoint{4.152884in}{2.666772in}}%
\pgfpathlineto{\pgfqpoint{4.145221in}{2.658052in}}%
\pgfpathlineto{\pgfqpoint{4.137552in}{2.649340in}}%
\pgfpathlineto{\pgfqpoint{4.124218in}{2.652668in}}%
\pgfpathlineto{\pgfqpoint{4.110891in}{2.656115in}}%
\pgfpathlineto{\pgfqpoint{4.097569in}{2.659681in}}%
\pgfpathlineto{\pgfqpoint{4.084254in}{2.663367in}}%
\pgfpathlineto{\pgfqpoint{4.091934in}{2.672094in}}%
\pgfpathlineto{\pgfqpoint{4.099609in}{2.680834in}}%
\pgfpathlineto{\pgfqpoint{4.107278in}{2.689587in}}%
\pgfpathlineto{\pgfqpoint{4.114943in}{2.698354in}}%
\pgfpathclose%
\pgfusepath{fill}%
\end{pgfscope}%
\begin{pgfscope}%
\pgfpathrectangle{\pgfqpoint{1.150000in}{0.150000in}}{\pgfqpoint{5.700000in}{5.700000in}}%
\pgfusepath{clip}%
\pgfsetbuttcap%
\pgfsetroundjoin%
\definecolor{currentfill}{rgb}{0.280868,0.160771,0.472899}%
\pgfsetfillcolor{currentfill}%
\pgfsetfillopacity{0.700000}%
\pgfsetlinewidth{0.000000pt}%
\definecolor{currentstroke}{rgb}{0.000000,0.000000,0.000000}%
\pgfsetstrokecolor{currentstroke}%
\pgfsetdash{}{0pt}%
\pgfpathmoveto{\pgfqpoint{3.239276in}{2.873904in}}%
\pgfpathlineto{\pgfqpoint{3.252507in}{2.862762in}}%
\pgfpathlineto{\pgfqpoint{3.265738in}{2.851787in}}%
\pgfpathlineto{\pgfqpoint{3.278969in}{2.840977in}}%
\pgfpathlineto{\pgfqpoint{3.292199in}{2.830332in}}%
\pgfpathlineto{\pgfqpoint{3.284235in}{2.822796in}}%
\pgfpathlineto{\pgfqpoint{3.276264in}{2.815327in}}%
\pgfpathlineto{\pgfqpoint{3.268286in}{2.807925in}}%
\pgfpathlineto{\pgfqpoint{3.260302in}{2.800590in}}%
\pgfpathlineto{\pgfqpoint{3.247054in}{2.811325in}}%
\pgfpathlineto{\pgfqpoint{3.233806in}{2.822225in}}%
\pgfpathlineto{\pgfqpoint{3.220557in}{2.833290in}}%
\pgfpathlineto{\pgfqpoint{3.207308in}{2.844522in}}%
\pgfpathlineto{\pgfqpoint{3.215310in}{2.851760in}}%
\pgfpathlineto{\pgfqpoint{3.223306in}{2.859070in}}%
\pgfpathlineto{\pgfqpoint{3.231294in}{2.866452in}}%
\pgfpathlineto{\pgfqpoint{3.239276in}{2.873904in}}%
\pgfpathclose%
\pgfusepath{fill}%
\end{pgfscope}%
\begin{pgfscope}%
\pgfpathrectangle{\pgfqpoint{1.150000in}{0.150000in}}{\pgfqpoint{5.700000in}{5.700000in}}%
\pgfusepath{clip}%
\pgfsetbuttcap%
\pgfsetroundjoin%
\definecolor{currentfill}{rgb}{0.279566,0.067836,0.391917}%
\pgfsetfillcolor{currentfill}%
\pgfsetfillopacity{0.700000}%
\pgfsetlinewidth{0.000000pt}%
\definecolor{currentstroke}{rgb}{0.000000,0.000000,0.000000}%
\pgfsetstrokecolor{currentstroke}%
\pgfsetdash{}{0pt}%
\pgfpathmoveto{\pgfqpoint{3.756942in}{2.693976in}}%
\pgfpathlineto{\pgfqpoint{3.770184in}{2.687754in}}%
\pgfpathlineto{\pgfqpoint{3.783430in}{2.681664in}}%
\pgfpathlineto{\pgfqpoint{3.796680in}{2.675707in}}%
\pgfpathlineto{\pgfqpoint{3.809933in}{2.669881in}}%
\pgfpathlineto{\pgfqpoint{3.802158in}{2.661355in}}%
\pgfpathlineto{\pgfqpoint{3.794378in}{2.652857in}}%
\pgfpathlineto{\pgfqpoint{3.786592in}{2.644387in}}%
\pgfpathlineto{\pgfqpoint{3.778800in}{2.635945in}}%
\pgfpathlineto{\pgfqpoint{3.765534in}{2.641804in}}%
\pgfpathlineto{\pgfqpoint{3.752271in}{2.647794in}}%
\pgfpathlineto{\pgfqpoint{3.739013in}{2.653917in}}%
\pgfpathlineto{\pgfqpoint{3.725758in}{2.660172in}}%
\pgfpathlineto{\pgfqpoint{3.733562in}{2.668575in}}%
\pgfpathlineto{\pgfqpoint{3.741361in}{2.677010in}}%
\pgfpathlineto{\pgfqpoint{3.749154in}{2.685477in}}%
\pgfpathlineto{\pgfqpoint{3.756942in}{2.693976in}}%
\pgfpathclose%
\pgfusepath{fill}%
\end{pgfscope}%
\begin{pgfscope}%
\pgfpathrectangle{\pgfqpoint{1.150000in}{0.150000in}}{\pgfqpoint{5.700000in}{5.700000in}}%
\pgfusepath{clip}%
\pgfsetbuttcap%
\pgfsetroundjoin%
\definecolor{currentfill}{rgb}{0.280894,0.078907,0.402329}%
\pgfsetfillcolor{currentfill}%
\pgfsetfillopacity{0.700000}%
\pgfsetlinewidth{0.000000pt}%
\definecolor{currentstroke}{rgb}{0.000000,0.000000,0.000000}%
\pgfsetstrokecolor{currentstroke}%
\pgfsetdash{}{0pt}%
\pgfpathmoveto{\pgfqpoint{3.619840in}{2.715103in}}%
\pgfpathlineto{\pgfqpoint{3.633069in}{2.707754in}}%
\pgfpathlineto{\pgfqpoint{3.646301in}{2.700544in}}%
\pgfpathlineto{\pgfqpoint{3.659536in}{2.693473in}}%
\pgfpathlineto{\pgfqpoint{3.672773in}{2.686541in}}%
\pgfpathlineto{\pgfqpoint{3.664950in}{2.678215in}}%
\pgfpathlineto{\pgfqpoint{3.657121in}{2.669925in}}%
\pgfpathlineto{\pgfqpoint{3.649286in}{2.661673in}}%
\pgfpathlineto{\pgfqpoint{3.641445in}{2.653458in}}%
\pgfpathlineto{\pgfqpoint{3.628193in}{2.660442in}}%
\pgfpathlineto{\pgfqpoint{3.614945in}{2.667564in}}%
\pgfpathlineto{\pgfqpoint{3.601699in}{2.674825in}}%
\pgfpathlineto{\pgfqpoint{3.588456in}{2.682226in}}%
\pgfpathlineto{\pgfqpoint{3.596311in}{2.690383in}}%
\pgfpathlineto{\pgfqpoint{3.604160in}{2.698581in}}%
\pgfpathlineto{\pgfqpoint{3.612003in}{2.706821in}}%
\pgfpathlineto{\pgfqpoint{3.619840in}{2.715103in}}%
\pgfpathclose%
\pgfusepath{fill}%
\end{pgfscope}%
\begin{pgfscope}%
\pgfpathrectangle{\pgfqpoint{1.150000in}{0.150000in}}{\pgfqpoint{5.700000in}{5.700000in}}%
\pgfusepath{clip}%
\pgfsetbuttcap%
\pgfsetroundjoin%
\definecolor{currentfill}{rgb}{0.282327,0.094955,0.417331}%
\pgfsetfillcolor{currentfill}%
\pgfsetfillopacity{0.700000}%
\pgfsetlinewidth{0.000000pt}%
\definecolor{currentstroke}{rgb}{0.000000,0.000000,0.000000}%
\pgfsetstrokecolor{currentstroke}%
\pgfsetdash{}{0pt}%
\pgfpathmoveto{\pgfqpoint{4.335877in}{2.731208in}}%
\pgfpathlineto{\pgfqpoint{4.349237in}{2.728806in}}%
\pgfpathlineto{\pgfqpoint{4.362605in}{2.726517in}}%
\pgfpathlineto{\pgfqpoint{4.375980in}{2.724341in}}%
\pgfpathlineto{\pgfqpoint{4.389362in}{2.722279in}}%
\pgfpathlineto{\pgfqpoint{4.381782in}{2.713630in}}%
\pgfpathlineto{\pgfqpoint{4.374198in}{2.704984in}}%
\pgfpathlineto{\pgfqpoint{4.366608in}{2.696339in}}%
\pgfpathlineto{\pgfqpoint{4.359013in}{2.687695in}}%
\pgfpathlineto{\pgfqpoint{4.345620in}{2.689700in}}%
\pgfpathlineto{\pgfqpoint{4.332234in}{2.691818in}}%
\pgfpathlineto{\pgfqpoint{4.318855in}{2.694049in}}%
\pgfpathlineto{\pgfqpoint{4.305484in}{2.696394in}}%
\pgfpathlineto{\pgfqpoint{4.313090in}{2.705089in}}%
\pgfpathlineto{\pgfqpoint{4.320690in}{2.713789in}}%
\pgfpathlineto{\pgfqpoint{4.328286in}{2.722495in}}%
\pgfpathlineto{\pgfqpoint{4.335877in}{2.731208in}}%
\pgfpathclose%
\pgfusepath{fill}%
\end{pgfscope}%
\begin{pgfscope}%
\pgfpathrectangle{\pgfqpoint{1.150000in}{0.150000in}}{\pgfqpoint{5.700000in}{5.700000in}}%
\pgfusepath{clip}%
\pgfsetbuttcap%
\pgfsetroundjoin%
\definecolor{currentfill}{rgb}{0.279566,0.067836,0.391917}%
\pgfsetfillcolor{currentfill}%
\pgfsetfillopacity{0.700000}%
\pgfsetlinewidth{0.000000pt}%
\definecolor{currentstroke}{rgb}{0.000000,0.000000,0.000000}%
\pgfsetstrokecolor{currentstroke}%
\pgfsetdash{}{0pt}%
\pgfpathmoveto{\pgfqpoint{3.893988in}{2.682318in}}%
\pgfpathlineto{\pgfqpoint{3.907251in}{2.677152in}}%
\pgfpathlineto{\pgfqpoint{3.920519in}{2.672114in}}%
\pgfpathlineto{\pgfqpoint{3.933792in}{2.667202in}}%
\pgfpathlineto{\pgfqpoint{3.947070in}{2.662416in}}%
\pgfpathlineto{\pgfqpoint{3.939341in}{2.653762in}}%
\pgfpathlineto{\pgfqpoint{3.931606in}{2.645128in}}%
\pgfpathlineto{\pgfqpoint{3.923866in}{2.636513in}}%
\pgfpathlineto{\pgfqpoint{3.916121in}{2.627916in}}%
\pgfpathlineto{\pgfqpoint{3.902831in}{2.632717in}}%
\pgfpathlineto{\pgfqpoint{3.889546in}{2.637643in}}%
\pgfpathlineto{\pgfqpoint{3.876266in}{2.642696in}}%
\pgfpathlineto{\pgfqpoint{3.862991in}{2.647876in}}%
\pgfpathlineto{\pgfqpoint{3.870748in}{2.656451in}}%
\pgfpathlineto{\pgfqpoint{3.878500in}{2.665049in}}%
\pgfpathlineto{\pgfqpoint{3.886246in}{2.673672in}}%
\pgfpathlineto{\pgfqpoint{3.893988in}{2.682318in}}%
\pgfpathclose%
\pgfusepath{fill}%
\end{pgfscope}%
\begin{pgfscope}%
\pgfpathrectangle{\pgfqpoint{1.150000in}{0.150000in}}{\pgfqpoint{5.700000in}{5.700000in}}%
\pgfusepath{clip}%
\pgfsetbuttcap%
\pgfsetroundjoin%
\definecolor{currentfill}{rgb}{0.282884,0.135920,0.453427}%
\pgfsetfillcolor{currentfill}%
\pgfsetfillopacity{0.700000}%
\pgfsetlinewidth{0.000000pt}%
\definecolor{currentstroke}{rgb}{0.000000,0.000000,0.000000}%
\pgfsetstrokecolor{currentstroke}%
\pgfsetdash{}{0pt}%
\pgfpathmoveto{\pgfqpoint{4.640712in}{2.807203in}}%
\pgfpathlineto{\pgfqpoint{4.654163in}{2.806290in}}%
\pgfpathlineto{\pgfqpoint{4.667623in}{2.805485in}}%
\pgfpathlineto{\pgfqpoint{4.681092in}{2.804787in}}%
\pgfpathlineto{\pgfqpoint{4.694569in}{2.804195in}}%
\pgfpathlineto{\pgfqpoint{4.687096in}{2.795898in}}%
\pgfpathlineto{\pgfqpoint{4.679617in}{2.787601in}}%
\pgfpathlineto{\pgfqpoint{4.672133in}{2.779302in}}%
\pgfpathlineto{\pgfqpoint{4.664644in}{2.770999in}}%
\pgfpathlineto{\pgfqpoint{4.651155in}{2.771479in}}%
\pgfpathlineto{\pgfqpoint{4.637675in}{2.772065in}}%
\pgfpathlineto{\pgfqpoint{4.624203in}{2.772758in}}%
\pgfpathlineto{\pgfqpoint{4.610740in}{2.773559in}}%
\pgfpathlineto{\pgfqpoint{4.618241in}{2.781967in}}%
\pgfpathlineto{\pgfqpoint{4.625736in}{2.790376in}}%
\pgfpathlineto{\pgfqpoint{4.633226in}{2.798787in}}%
\pgfpathlineto{\pgfqpoint{4.640712in}{2.807203in}}%
\pgfpathclose%
\pgfusepath{fill}%
\end{pgfscope}%
\begin{pgfscope}%
\pgfpathrectangle{\pgfqpoint{1.150000in}{0.150000in}}{\pgfqpoint{5.700000in}{5.700000in}}%
\pgfusepath{clip}%
\pgfsetbuttcap%
\pgfsetroundjoin%
\definecolor{currentfill}{rgb}{0.269308,0.218818,0.509577}%
\pgfsetfillcolor{currentfill}%
\pgfsetfillopacity{0.700000}%
\pgfsetlinewidth{0.000000pt}%
\definecolor{currentstroke}{rgb}{0.000000,0.000000,0.000000}%
\pgfsetstrokecolor{currentstroke}%
\pgfsetdash{}{0pt}%
\pgfpathmoveto{\pgfqpoint{5.113216in}{2.966434in}}%
\pgfpathlineto{\pgfqpoint{5.126827in}{2.967189in}}%
\pgfpathlineto{\pgfqpoint{5.140449in}{2.968046in}}%
\pgfpathlineto{\pgfqpoint{5.154081in}{2.969002in}}%
\pgfpathlineto{\pgfqpoint{5.167723in}{2.970060in}}%
\pgfpathlineto{\pgfqpoint{5.160428in}{2.962625in}}%
\pgfpathlineto{\pgfqpoint{5.153129in}{2.955205in}}%
\pgfpathlineto{\pgfqpoint{5.145825in}{2.947795in}}%
\pgfpathlineto{\pgfqpoint{5.138517in}{2.940394in}}%
\pgfpathlineto{\pgfqpoint{5.124860in}{2.939133in}}%
\pgfpathlineto{\pgfqpoint{5.111213in}{2.937973in}}%
\pgfpathlineto{\pgfqpoint{5.097577in}{2.936914in}}%
\pgfpathlineto{\pgfqpoint{5.083951in}{2.935956in}}%
\pgfpathlineto{\pgfqpoint{5.091275in}{2.943553in}}%
\pgfpathlineto{\pgfqpoint{5.098594in}{2.951164in}}%
\pgfpathlineto{\pgfqpoint{5.105907in}{2.958790in}}%
\pgfpathlineto{\pgfqpoint{5.113216in}{2.966434in}}%
\pgfpathclose%
\pgfusepath{fill}%
\end{pgfscope}%
\begin{pgfscope}%
\pgfpathrectangle{\pgfqpoint{1.150000in}{0.150000in}}{\pgfqpoint{5.700000in}{5.700000in}}%
\pgfusepath{clip}%
\pgfsetbuttcap%
\pgfsetroundjoin%
\definecolor{currentfill}{rgb}{0.282656,0.100196,0.422160}%
\pgfsetfillcolor{currentfill}%
\pgfsetfillopacity{0.700000}%
\pgfsetlinewidth{0.000000pt}%
\definecolor{currentstroke}{rgb}{0.000000,0.000000,0.000000}%
\pgfsetstrokecolor{currentstroke}%
\pgfsetdash{}{0pt}%
\pgfpathmoveto{\pgfqpoint{3.482598in}{2.746578in}}%
\pgfpathlineto{\pgfqpoint{3.495823in}{2.738024in}}%
\pgfpathlineto{\pgfqpoint{3.509050in}{2.729618in}}%
\pgfpathlineto{\pgfqpoint{3.522279in}{2.721359in}}%
\pgfpathlineto{\pgfqpoint{3.535510in}{2.713245in}}%
\pgfpathlineto{\pgfqpoint{3.527635in}{2.705193in}}%
\pgfpathlineto{\pgfqpoint{3.519753in}{2.697188in}}%
\pgfpathlineto{\pgfqpoint{3.511866in}{2.689230in}}%
\pgfpathlineto{\pgfqpoint{3.503973in}{2.681319in}}%
\pgfpathlineto{\pgfqpoint{3.490727in}{2.689502in}}%
\pgfpathlineto{\pgfqpoint{3.477483in}{2.697832in}}%
\pgfpathlineto{\pgfqpoint{3.464241in}{2.706307in}}%
\pgfpathlineto{\pgfqpoint{3.451000in}{2.714931in}}%
\pgfpathlineto{\pgfqpoint{3.458909in}{2.722765in}}%
\pgfpathlineto{\pgfqpoint{3.466811in}{2.730651in}}%
\pgfpathlineto{\pgfqpoint{3.474708in}{2.738589in}}%
\pgfpathlineto{\pgfqpoint{3.482598in}{2.746578in}}%
\pgfpathclose%
\pgfusepath{fill}%
\end{pgfscope}%
\begin{pgfscope}%
\pgfpathrectangle{\pgfqpoint{1.150000in}{0.150000in}}{\pgfqpoint{5.700000in}{5.700000in}}%
\pgfusepath{clip}%
\pgfsetbuttcap%
\pgfsetroundjoin%
\definecolor{currentfill}{rgb}{0.282623,0.140926,0.457517}%
\pgfsetfillcolor{currentfill}%
\pgfsetfillopacity{0.700000}%
\pgfsetlinewidth{0.000000pt}%
\definecolor{currentstroke}{rgb}{0.000000,0.000000,0.000000}%
\pgfsetstrokecolor{currentstroke}%
\pgfsetdash{}{0pt}%
\pgfpathmoveto{\pgfqpoint{3.292199in}{2.830332in}}%
\pgfpathlineto{\pgfqpoint{3.305430in}{2.819849in}}%
\pgfpathlineto{\pgfqpoint{3.318660in}{2.809528in}}%
\pgfpathlineto{\pgfqpoint{3.331891in}{2.799367in}}%
\pgfpathlineto{\pgfqpoint{3.345122in}{2.789365in}}%
\pgfpathlineto{\pgfqpoint{3.337175in}{2.781746in}}%
\pgfpathlineto{\pgfqpoint{3.329221in}{2.774190in}}%
\pgfpathlineto{\pgfqpoint{3.321260in}{2.766697in}}%
\pgfpathlineto{\pgfqpoint{3.313293in}{2.759266in}}%
\pgfpathlineto{\pgfqpoint{3.300045in}{2.769357in}}%
\pgfpathlineto{\pgfqpoint{3.286797in}{2.779607in}}%
\pgfpathlineto{\pgfqpoint{3.273549in}{2.790018in}}%
\pgfpathlineto{\pgfqpoint{3.260302in}{2.800590in}}%
\pgfpathlineto{\pgfqpoint{3.268286in}{2.807925in}}%
\pgfpathlineto{\pgfqpoint{3.276264in}{2.815327in}}%
\pgfpathlineto{\pgfqpoint{3.284235in}{2.822796in}}%
\pgfpathlineto{\pgfqpoint{3.292199in}{2.830332in}}%
\pgfpathclose%
\pgfusepath{fill}%
\end{pgfscope}%
\begin{pgfscope}%
\pgfpathrectangle{\pgfqpoint{1.150000in}{0.150000in}}{\pgfqpoint{5.700000in}{5.700000in}}%
\pgfusepath{clip}%
\pgfsetbuttcap%
\pgfsetroundjoin%
\definecolor{currentfill}{rgb}{0.273006,0.204520,0.501721}%
\pgfsetfillcolor{currentfill}%
\pgfsetfillopacity{0.700000}%
\pgfsetlinewidth{0.000000pt}%
\definecolor{currentstroke}{rgb}{0.000000,0.000000,0.000000}%
\pgfsetstrokecolor{currentstroke}%
\pgfsetdash{}{0pt}%
\pgfpathmoveto{\pgfqpoint{5.029550in}{2.933138in}}%
\pgfpathlineto{\pgfqpoint{5.043135in}{2.933690in}}%
\pgfpathlineto{\pgfqpoint{5.056731in}{2.934344in}}%
\pgfpathlineto{\pgfqpoint{5.070336in}{2.935099in}}%
\pgfpathlineto{\pgfqpoint{5.083951in}{2.935956in}}%
\pgfpathlineto{\pgfqpoint{5.076623in}{2.928369in}}%
\pgfpathlineto{\pgfqpoint{5.069290in}{2.920791in}}%
\pgfpathlineto{\pgfqpoint{5.061951in}{2.913219in}}%
\pgfpathlineto{\pgfqpoint{5.054608in}{2.905651in}}%
\pgfpathlineto{\pgfqpoint{5.040978in}{2.904609in}}%
\pgfpathlineto{\pgfqpoint{5.027359in}{2.903669in}}%
\pgfpathlineto{\pgfqpoint{5.013750in}{2.902830in}}%
\pgfpathlineto{\pgfqpoint{5.000150in}{2.902094in}}%
\pgfpathlineto{\pgfqpoint{5.007508in}{2.909840in}}%
\pgfpathlineto{\pgfqpoint{5.014860in}{2.917595in}}%
\pgfpathlineto{\pgfqpoint{5.022208in}{2.925360in}}%
\pgfpathlineto{\pgfqpoint{5.029550in}{2.933138in}}%
\pgfpathclose%
\pgfusepath{fill}%
\end{pgfscope}%
\begin{pgfscope}%
\pgfpathrectangle{\pgfqpoint{1.150000in}{0.150000in}}{\pgfqpoint{5.700000in}{5.700000in}}%
\pgfusepath{clip}%
\pgfsetbuttcap%
\pgfsetroundjoin%
\definecolor{currentfill}{rgb}{0.279566,0.067836,0.391917}%
\pgfsetfillcolor{currentfill}%
\pgfsetfillopacity{0.700000}%
\pgfsetlinewidth{0.000000pt}%
\definecolor{currentstroke}{rgb}{0.000000,0.000000,0.000000}%
\pgfsetstrokecolor{currentstroke}%
\pgfsetdash{}{0pt}%
\pgfpathmoveto{\pgfqpoint{4.031050in}{2.679319in}}%
\pgfpathlineto{\pgfqpoint{4.044343in}{2.675149in}}%
\pgfpathlineto{\pgfqpoint{4.057641in}{2.671101in}}%
\pgfpathlineto{\pgfqpoint{4.070944in}{2.667174in}}%
\pgfpathlineto{\pgfqpoint{4.084254in}{2.663367in}}%
\pgfpathlineto{\pgfqpoint{4.076569in}{2.654653in}}%
\pgfpathlineto{\pgfqpoint{4.068878in}{2.645950in}}%
\pgfpathlineto{\pgfqpoint{4.061183in}{2.637259in}}%
\pgfpathlineto{\pgfqpoint{4.053482in}{2.628579in}}%
\pgfpathlineto{\pgfqpoint{4.040161in}{2.632382in}}%
\pgfpathlineto{\pgfqpoint{4.026846in}{2.636305in}}%
\pgfpathlineto{\pgfqpoint{4.013537in}{2.640350in}}%
\pgfpathlineto{\pgfqpoint{4.000233in}{2.644517in}}%
\pgfpathlineto{\pgfqpoint{4.007945in}{2.653194in}}%
\pgfpathlineto{\pgfqpoint{4.015652in}{2.661886in}}%
\pgfpathlineto{\pgfqpoint{4.023354in}{2.670595in}}%
\pgfpathlineto{\pgfqpoint{4.031050in}{2.679319in}}%
\pgfpathclose%
\pgfusepath{fill}%
\end{pgfscope}%
\begin{pgfscope}%
\pgfpathrectangle{\pgfqpoint{1.150000in}{0.150000in}}{\pgfqpoint{5.700000in}{5.700000in}}%
\pgfusepath{clip}%
\pgfsetbuttcap%
\pgfsetroundjoin%
\definecolor{currentfill}{rgb}{0.281446,0.084320,0.407414}%
\pgfsetfillcolor{currentfill}%
\pgfsetfillopacity{0.700000}%
\pgfsetlinewidth{0.000000pt}%
\definecolor{currentstroke}{rgb}{0.000000,0.000000,0.000000}%
\pgfsetstrokecolor{currentstroke}%
\pgfsetdash{}{0pt}%
\pgfpathmoveto{\pgfqpoint{4.252068in}{2.706920in}}%
\pgfpathlineto{\pgfqpoint{4.265411in}{2.704115in}}%
\pgfpathlineto{\pgfqpoint{4.278762in}{2.701427in}}%
\pgfpathlineto{\pgfqpoint{4.292120in}{2.698853in}}%
\pgfpathlineto{\pgfqpoint{4.305484in}{2.696394in}}%
\pgfpathlineto{\pgfqpoint{4.297873in}{2.687703in}}%
\pgfpathlineto{\pgfqpoint{4.290257in}{2.679016in}}%
\pgfpathlineto{\pgfqpoint{4.282636in}{2.670331in}}%
\pgfpathlineto{\pgfqpoint{4.275009in}{2.661648in}}%
\pgfpathlineto{\pgfqpoint{4.261634in}{2.664067in}}%
\pgfpathlineto{\pgfqpoint{4.248266in}{2.666601in}}%
\pgfpathlineto{\pgfqpoint{4.234904in}{2.669250in}}%
\pgfpathlineto{\pgfqpoint{4.221550in}{2.672015in}}%
\pgfpathlineto{\pgfqpoint{4.229187in}{2.680731in}}%
\pgfpathlineto{\pgfqpoint{4.236819in}{2.689453in}}%
\pgfpathlineto{\pgfqpoint{4.244446in}{2.698183in}}%
\pgfpathlineto{\pgfqpoint{4.252068in}{2.706920in}}%
\pgfpathclose%
\pgfusepath{fill}%
\end{pgfscope}%
\begin{pgfscope}%
\pgfpathrectangle{\pgfqpoint{1.150000in}{0.150000in}}{\pgfqpoint{5.700000in}{5.700000in}}%
\pgfusepath{clip}%
\pgfsetbuttcap%
\pgfsetroundjoin%
\definecolor{currentfill}{rgb}{0.283187,0.125848,0.444960}%
\pgfsetfillcolor{currentfill}%
\pgfsetfillopacity{0.700000}%
\pgfsetlinewidth{0.000000pt}%
\definecolor{currentstroke}{rgb}{0.000000,0.000000,0.000000}%
\pgfsetstrokecolor{currentstroke}%
\pgfsetdash{}{0pt}%
\pgfpathmoveto{\pgfqpoint{4.556971in}{2.777846in}}%
\pgfpathlineto{\pgfqpoint{4.570401in}{2.776611in}}%
\pgfpathlineto{\pgfqpoint{4.583839in}{2.775486in}}%
\pgfpathlineto{\pgfqpoint{4.597286in}{2.774468in}}%
\pgfpathlineto{\pgfqpoint{4.610740in}{2.773559in}}%
\pgfpathlineto{\pgfqpoint{4.603235in}{2.765152in}}%
\pgfpathlineto{\pgfqpoint{4.595724in}{2.756743in}}%
\pgfpathlineto{\pgfqpoint{4.588208in}{2.748331in}}%
\pgfpathlineto{\pgfqpoint{4.580687in}{2.739915in}}%
\pgfpathlineto{\pgfqpoint{4.567221in}{2.740730in}}%
\pgfpathlineto{\pgfqpoint{4.553764in}{2.741653in}}%
\pgfpathlineto{\pgfqpoint{4.540315in}{2.742685in}}%
\pgfpathlineto{\pgfqpoint{4.526873in}{2.743826in}}%
\pgfpathlineto{\pgfqpoint{4.534406in}{2.752329in}}%
\pgfpathlineto{\pgfqpoint{4.541933in}{2.760833in}}%
\pgfpathlineto{\pgfqpoint{4.549454in}{2.769338in}}%
\pgfpathlineto{\pgfqpoint{4.556971in}{2.777846in}}%
\pgfpathclose%
\pgfusepath{fill}%
\end{pgfscope}%
\begin{pgfscope}%
\pgfpathrectangle{\pgfqpoint{1.150000in}{0.150000in}}{\pgfqpoint{5.700000in}{5.700000in}}%
\pgfusepath{clip}%
\pgfsetbuttcap%
\pgfsetroundjoin%
\definecolor{currentfill}{rgb}{0.260571,0.246922,0.522828}%
\pgfsetfillcolor{currentfill}%
\pgfsetfillopacity{0.700000}%
\pgfsetlinewidth{0.000000pt}%
\definecolor{currentstroke}{rgb}{0.000000,0.000000,0.000000}%
\pgfsetstrokecolor{currentstroke}%
\pgfsetdash{}{0pt}%
\pgfpathmoveto{\pgfqpoint{2.995110in}{3.048149in}}%
\pgfpathlineto{\pgfqpoint{3.008391in}{3.034040in}}%
\pgfpathlineto{\pgfqpoint{3.021668in}{3.020124in}}%
\pgfpathlineto{\pgfqpoint{3.034942in}{3.006397in}}%
\pgfpathlineto{\pgfqpoint{3.048213in}{2.992860in}}%
\pgfpathlineto{\pgfqpoint{3.040146in}{2.986002in}}%
\pgfpathlineto{\pgfqpoint{3.032071in}{2.979232in}}%
\pgfpathlineto{\pgfqpoint{3.023988in}{2.972550in}}%
\pgfpathlineto{\pgfqpoint{3.015897in}{2.965956in}}%
\pgfpathlineto{\pgfqpoint{3.002605in}{2.979605in}}%
\pgfpathlineto{\pgfqpoint{2.989310in}{2.993442in}}%
\pgfpathlineto{\pgfqpoint{2.976012in}{3.007470in}}%
\pgfpathlineto{\pgfqpoint{2.962711in}{3.021690in}}%
\pgfpathlineto{\pgfqpoint{2.970823in}{3.028166in}}%
\pgfpathlineto{\pgfqpoint{2.978927in}{3.034735in}}%
\pgfpathlineto{\pgfqpoint{2.987023in}{3.041396in}}%
\pgfpathlineto{\pgfqpoint{2.995110in}{3.048149in}}%
\pgfpathclose%
\pgfusepath{fill}%
\end{pgfscope}%
\begin{pgfscope}%
\pgfpathrectangle{\pgfqpoint{1.150000in}{0.150000in}}{\pgfqpoint{5.700000in}{5.700000in}}%
\pgfusepath{clip}%
\pgfsetbuttcap%
\pgfsetroundjoin%
\definecolor{currentfill}{rgb}{0.276194,0.190074,0.493001}%
\pgfsetfillcolor{currentfill}%
\pgfsetfillopacity{0.700000}%
\pgfsetlinewidth{0.000000pt}%
\definecolor{currentstroke}{rgb}{0.000000,0.000000,0.000000}%
\pgfsetstrokecolor{currentstroke}%
\pgfsetdash{}{0pt}%
\pgfpathmoveto{\pgfqpoint{4.945853in}{2.900172in}}%
\pgfpathlineto{\pgfqpoint{4.959413in}{2.900498in}}%
\pgfpathlineto{\pgfqpoint{4.972982in}{2.900927in}}%
\pgfpathlineto{\pgfqpoint{4.986561in}{2.901459in}}%
\pgfpathlineto{\pgfqpoint{5.000150in}{2.902094in}}%
\pgfpathlineto{\pgfqpoint{4.992788in}{2.894353in}}%
\pgfpathlineto{\pgfqpoint{4.985421in}{2.886617in}}%
\pgfpathlineto{\pgfqpoint{4.978048in}{2.878883in}}%
\pgfpathlineto{\pgfqpoint{4.970670in}{2.871149in}}%
\pgfpathlineto{\pgfqpoint{4.957068in}{2.870347in}}%
\pgfpathlineto{\pgfqpoint{4.943475in}{2.869649in}}%
\pgfpathlineto{\pgfqpoint{4.929893in}{2.869053in}}%
\pgfpathlineto{\pgfqpoint{4.916320in}{2.868560in}}%
\pgfpathlineto{\pgfqpoint{4.923710in}{2.876454in}}%
\pgfpathlineto{\pgfqpoint{4.931096in}{2.884353in}}%
\pgfpathlineto{\pgfqpoint{4.938477in}{2.892258in}}%
\pgfpathlineto{\pgfqpoint{4.945853in}{2.900172in}}%
\pgfpathclose%
\pgfusepath{fill}%
\end{pgfscope}%
\begin{pgfscope}%
\pgfpathrectangle{\pgfqpoint{1.150000in}{0.150000in}}{\pgfqpoint{5.700000in}{5.700000in}}%
\pgfusepath{clip}%
\pgfsetbuttcap%
\pgfsetroundjoin%
\definecolor{currentfill}{rgb}{0.280267,0.073417,0.397163}%
\pgfsetfillcolor{currentfill}%
\pgfsetfillopacity{0.700000}%
\pgfsetlinewidth{0.000000pt}%
\definecolor{currentstroke}{rgb}{0.000000,0.000000,0.000000}%
\pgfsetstrokecolor{currentstroke}%
\pgfsetdash{}{0pt}%
\pgfpathmoveto{\pgfqpoint{3.672773in}{2.686541in}}%
\pgfpathlineto{\pgfqpoint{3.686014in}{2.679745in}}%
\pgfpathlineto{\pgfqpoint{3.699259in}{2.673086in}}%
\pgfpathlineto{\pgfqpoint{3.712506in}{2.666562in}}%
\pgfpathlineto{\pgfqpoint{3.725758in}{2.660172in}}%
\pgfpathlineto{\pgfqpoint{3.717948in}{2.651802in}}%
\pgfpathlineto{\pgfqpoint{3.710132in}{2.643464in}}%
\pgfpathlineto{\pgfqpoint{3.702310in}{2.635159in}}%
\pgfpathlineto{\pgfqpoint{3.694483in}{2.626886in}}%
\pgfpathlineto{\pgfqpoint{3.681219in}{2.633326in}}%
\pgfpathlineto{\pgfqpoint{3.667957in}{2.639901in}}%
\pgfpathlineto{\pgfqpoint{3.654700in}{2.646612in}}%
\pgfpathlineto{\pgfqpoint{3.641445in}{2.653458in}}%
\pgfpathlineto{\pgfqpoint{3.649286in}{2.661673in}}%
\pgfpathlineto{\pgfqpoint{3.657121in}{2.669925in}}%
\pgfpathlineto{\pgfqpoint{3.664950in}{2.678215in}}%
\pgfpathlineto{\pgfqpoint{3.672773in}{2.686541in}}%
\pgfpathclose%
\pgfusepath{fill}%
\end{pgfscope}%
\begin{pgfscope}%
\pgfpathrectangle{\pgfqpoint{1.150000in}{0.150000in}}{\pgfqpoint{5.700000in}{5.700000in}}%
\pgfusepath{clip}%
\pgfsetbuttcap%
\pgfsetroundjoin%
\definecolor{currentfill}{rgb}{0.267968,0.223549,0.512008}%
\pgfsetfillcolor{currentfill}%
\pgfsetfillopacity{0.700000}%
\pgfsetlinewidth{0.000000pt}%
\definecolor{currentstroke}{rgb}{0.000000,0.000000,0.000000}%
\pgfsetstrokecolor{currentstroke}%
\pgfsetdash{}{0pt}%
\pgfpathmoveto{\pgfqpoint{3.048213in}{2.992860in}}%
\pgfpathlineto{\pgfqpoint{3.061482in}{2.979509in}}%
\pgfpathlineto{\pgfqpoint{3.074748in}{2.966343in}}%
\pgfpathlineto{\pgfqpoint{3.088012in}{2.953361in}}%
\pgfpathlineto{\pgfqpoint{3.101274in}{2.940561in}}%
\pgfpathlineto{\pgfqpoint{3.093226in}{2.933600in}}%
\pgfpathlineto{\pgfqpoint{3.085171in}{2.926722in}}%
\pgfpathlineto{\pgfqpoint{3.077108in}{2.919928in}}%
\pgfpathlineto{\pgfqpoint{3.069038in}{2.913217in}}%
\pgfpathlineto{\pgfqpoint{3.055756in}{2.926127in}}%
\pgfpathlineto{\pgfqpoint{3.042472in}{2.939219in}}%
\pgfpathlineto{\pgfqpoint{3.029186in}{2.952495in}}%
\pgfpathlineto{\pgfqpoint{3.015897in}{2.965956in}}%
\pgfpathlineto{\pgfqpoint{3.023988in}{2.972550in}}%
\pgfpathlineto{\pgfqpoint{3.032071in}{2.979232in}}%
\pgfpathlineto{\pgfqpoint{3.040146in}{2.986002in}}%
\pgfpathlineto{\pgfqpoint{3.048213in}{2.992860in}}%
\pgfpathclose%
\pgfusepath{fill}%
\end{pgfscope}%
\begin{pgfscope}%
\pgfpathrectangle{\pgfqpoint{1.150000in}{0.150000in}}{\pgfqpoint{5.700000in}{5.700000in}}%
\pgfusepath{clip}%
\pgfsetbuttcap%
\pgfsetroundjoin%
\definecolor{currentfill}{rgb}{0.278791,0.062145,0.386592}%
\pgfsetfillcolor{currentfill}%
\pgfsetfillopacity{0.700000}%
\pgfsetlinewidth{0.000000pt}%
\definecolor{currentstroke}{rgb}{0.000000,0.000000,0.000000}%
\pgfsetstrokecolor{currentstroke}%
\pgfsetdash{}{0pt}%
\pgfpathmoveto{\pgfqpoint{3.809933in}{2.669881in}}%
\pgfpathlineto{\pgfqpoint{3.823191in}{2.664185in}}%
\pgfpathlineto{\pgfqpoint{3.836453in}{2.658620in}}%
\pgfpathlineto{\pgfqpoint{3.849720in}{2.653184in}}%
\pgfpathlineto{\pgfqpoint{3.862991in}{2.647876in}}%
\pgfpathlineto{\pgfqpoint{3.855228in}{2.639324in}}%
\pgfpathlineto{\pgfqpoint{3.847460in}{2.630796in}}%
\pgfpathlineto{\pgfqpoint{3.839686in}{2.622291in}}%
\pgfpathlineto{\pgfqpoint{3.831907in}{2.613810in}}%
\pgfpathlineto{\pgfqpoint{3.818624in}{2.619150in}}%
\pgfpathlineto{\pgfqpoint{3.805345in}{2.624619in}}%
\pgfpathlineto{\pgfqpoint{3.792071in}{2.630217in}}%
\pgfpathlineto{\pgfqpoint{3.778800in}{2.635945in}}%
\pgfpathlineto{\pgfqpoint{3.786592in}{2.644387in}}%
\pgfpathlineto{\pgfqpoint{3.794378in}{2.652857in}}%
\pgfpathlineto{\pgfqpoint{3.802158in}{2.661355in}}%
\pgfpathlineto{\pgfqpoint{3.809933in}{2.669881in}}%
\pgfpathclose%
\pgfusepath{fill}%
\end{pgfscope}%
\begin{pgfscope}%
\pgfpathrectangle{\pgfqpoint{1.150000in}{0.150000in}}{\pgfqpoint{5.700000in}{5.700000in}}%
\pgfusepath{clip}%
\pgfsetbuttcap%
\pgfsetroundjoin%
\definecolor{currentfill}{rgb}{0.283091,0.110553,0.431554}%
\pgfsetfillcolor{currentfill}%
\pgfsetfillopacity{0.700000}%
\pgfsetlinewidth{0.000000pt}%
\definecolor{currentstroke}{rgb}{0.000000,0.000000,0.000000}%
\pgfsetstrokecolor{currentstroke}%
\pgfsetdash{}{0pt}%
\pgfpathmoveto{\pgfqpoint{4.473189in}{2.749486in}}%
\pgfpathlineto{\pgfqpoint{4.486599in}{2.747906in}}%
\pgfpathlineto{\pgfqpoint{4.500016in}{2.746436in}}%
\pgfpathlineto{\pgfqpoint{4.513440in}{2.745076in}}%
\pgfpathlineto{\pgfqpoint{4.526873in}{2.743826in}}%
\pgfpathlineto{\pgfqpoint{4.519336in}{2.735321in}}%
\pgfpathlineto{\pgfqpoint{4.511794in}{2.726815in}}%
\pgfpathlineto{\pgfqpoint{4.504246in}{2.718305in}}%
\pgfpathlineto{\pgfqpoint{4.496694in}{2.709792in}}%
\pgfpathlineto{\pgfqpoint{4.483250in}{2.710966in}}%
\pgfpathlineto{\pgfqpoint{4.469814in}{2.712250in}}%
\pgfpathlineto{\pgfqpoint{4.456386in}{2.713644in}}%
\pgfpathlineto{\pgfqpoint{4.442966in}{2.715148in}}%
\pgfpathlineto{\pgfqpoint{4.450529in}{2.723731in}}%
\pgfpathlineto{\pgfqpoint{4.458088in}{2.732315in}}%
\pgfpathlineto{\pgfqpoint{4.465641in}{2.740899in}}%
\pgfpathlineto{\pgfqpoint{4.473189in}{2.749486in}}%
\pgfpathclose%
\pgfusepath{fill}%
\end{pgfscope}%
\begin{pgfscope}%
\pgfpathrectangle{\pgfqpoint{1.150000in}{0.150000in}}{\pgfqpoint{5.700000in}{5.700000in}}%
\pgfusepath{clip}%
\pgfsetbuttcap%
\pgfsetroundjoin%
\definecolor{currentfill}{rgb}{0.283187,0.125848,0.444960}%
\pgfsetfillcolor{currentfill}%
\pgfsetfillopacity{0.700000}%
\pgfsetlinewidth{0.000000pt}%
\definecolor{currentstroke}{rgb}{0.000000,0.000000,0.000000}%
\pgfsetstrokecolor{currentstroke}%
\pgfsetdash{}{0pt}%
\pgfpathmoveto{\pgfqpoint{3.345122in}{2.789365in}}%
\pgfpathlineto{\pgfqpoint{3.358354in}{2.779520in}}%
\pgfpathlineto{\pgfqpoint{3.371586in}{2.769833in}}%
\pgfpathlineto{\pgfqpoint{3.384819in}{2.760301in}}%
\pgfpathlineto{\pgfqpoint{3.398053in}{2.750923in}}%
\pgfpathlineto{\pgfqpoint{3.390122in}{2.743223in}}%
\pgfpathlineto{\pgfqpoint{3.382184in}{2.735580in}}%
\pgfpathlineto{\pgfqpoint{3.374241in}{2.727995in}}%
\pgfpathlineto{\pgfqpoint{3.366290in}{2.720469in}}%
\pgfpathlineto{\pgfqpoint{3.353040in}{2.729936in}}%
\pgfpathlineto{\pgfqpoint{3.339790in}{2.739557in}}%
\pgfpathlineto{\pgfqpoint{3.326541in}{2.749333in}}%
\pgfpathlineto{\pgfqpoint{3.313293in}{2.759266in}}%
\pgfpathlineto{\pgfqpoint{3.321260in}{2.766697in}}%
\pgfpathlineto{\pgfqpoint{3.329221in}{2.774190in}}%
\pgfpathlineto{\pgfqpoint{3.337175in}{2.781746in}}%
\pgfpathlineto{\pgfqpoint{3.345122in}{2.789365in}}%
\pgfpathclose%
\pgfusepath{fill}%
\end{pgfscope}%
\begin{pgfscope}%
\pgfpathrectangle{\pgfqpoint{1.150000in}{0.150000in}}{\pgfqpoint{5.700000in}{5.700000in}}%
\pgfusepath{clip}%
\pgfsetbuttcap%
\pgfsetroundjoin%
\definecolor{currentfill}{rgb}{0.278826,0.175490,0.483397}%
\pgfsetfillcolor{currentfill}%
\pgfsetfillopacity{0.700000}%
\pgfsetlinewidth{0.000000pt}%
\definecolor{currentstroke}{rgb}{0.000000,0.000000,0.000000}%
\pgfsetstrokecolor{currentstroke}%
\pgfsetdash{}{0pt}%
\pgfpathmoveto{\pgfqpoint{4.862124in}{2.867623in}}%
\pgfpathlineto{\pgfqpoint{4.875659in}{2.867701in}}%
\pgfpathlineto{\pgfqpoint{4.889203in}{2.867884in}}%
\pgfpathlineto{\pgfqpoint{4.902756in}{2.868170in}}%
\pgfpathlineto{\pgfqpoint{4.916320in}{2.868560in}}%
\pgfpathlineto{\pgfqpoint{4.908924in}{2.860668in}}%
\pgfpathlineto{\pgfqpoint{4.901523in}{2.852776in}}%
\pgfpathlineto{\pgfqpoint{4.894117in}{2.844883in}}%
\pgfpathlineto{\pgfqpoint{4.886705in}{2.836987in}}%
\pgfpathlineto{\pgfqpoint{4.873129in}{2.836448in}}%
\pgfpathlineto{\pgfqpoint{4.859563in}{2.836013in}}%
\pgfpathlineto{\pgfqpoint{4.846006in}{2.835682in}}%
\pgfpathlineto{\pgfqpoint{4.832459in}{2.835455in}}%
\pgfpathlineto{\pgfqpoint{4.839883in}{2.843493in}}%
\pgfpathlineto{\pgfqpoint{4.847302in}{2.851533in}}%
\pgfpathlineto{\pgfqpoint{4.854715in}{2.859575in}}%
\pgfpathlineto{\pgfqpoint{4.862124in}{2.867623in}}%
\pgfpathclose%
\pgfusepath{fill}%
\end{pgfscope}%
\begin{pgfscope}%
\pgfpathrectangle{\pgfqpoint{1.150000in}{0.150000in}}{\pgfqpoint{5.700000in}{5.700000in}}%
\pgfusepath{clip}%
\pgfsetbuttcap%
\pgfsetroundjoin%
\definecolor{currentfill}{rgb}{0.280894,0.078907,0.402329}%
\pgfsetfillcolor{currentfill}%
\pgfsetfillopacity{0.700000}%
\pgfsetlinewidth{0.000000pt}%
\definecolor{currentstroke}{rgb}{0.000000,0.000000,0.000000}%
\pgfsetstrokecolor{currentstroke}%
\pgfsetdash{}{0pt}%
\pgfpathmoveto{\pgfqpoint{4.168196in}{2.684240in}}%
\pgfpathlineto{\pgfqpoint{4.181525in}{2.681008in}}%
\pgfpathlineto{\pgfqpoint{4.194860in}{2.677893in}}%
\pgfpathlineto{\pgfqpoint{4.208201in}{2.674896in}}%
\pgfpathlineto{\pgfqpoint{4.221550in}{2.672015in}}%
\pgfpathlineto{\pgfqpoint{4.213907in}{2.663305in}}%
\pgfpathlineto{\pgfqpoint{4.206260in}{2.654600in}}%
\pgfpathlineto{\pgfqpoint{4.198607in}{2.645899in}}%
\pgfpathlineto{\pgfqpoint{4.190949in}{2.637202in}}%
\pgfpathlineto{\pgfqpoint{4.177590in}{2.640061in}}%
\pgfpathlineto{\pgfqpoint{4.164238in}{2.643037in}}%
\pgfpathlineto{\pgfqpoint{4.150891in}{2.646129in}}%
\pgfpathlineto{\pgfqpoint{4.137552in}{2.649340in}}%
\pgfpathlineto{\pgfqpoint{4.145221in}{2.658052in}}%
\pgfpathlineto{\pgfqpoint{4.152884in}{2.666772in}}%
\pgfpathlineto{\pgfqpoint{4.160543in}{2.675501in}}%
\pgfpathlineto{\pgfqpoint{4.168196in}{2.684240in}}%
\pgfpathclose%
\pgfusepath{fill}%
\end{pgfscope}%
\begin{pgfscope}%
\pgfpathrectangle{\pgfqpoint{1.150000in}{0.150000in}}{\pgfqpoint{5.700000in}{5.700000in}}%
\pgfusepath{clip}%
\pgfsetbuttcap%
\pgfsetroundjoin%
\definecolor{currentfill}{rgb}{0.281924,0.089666,0.412415}%
\pgfsetfillcolor{currentfill}%
\pgfsetfillopacity{0.700000}%
\pgfsetlinewidth{0.000000pt}%
\definecolor{currentstroke}{rgb}{0.000000,0.000000,0.000000}%
\pgfsetstrokecolor{currentstroke}%
\pgfsetdash{}{0pt}%
\pgfpathmoveto{\pgfqpoint{3.535510in}{2.713245in}}%
\pgfpathlineto{\pgfqpoint{3.548743in}{2.705276in}}%
\pgfpathlineto{\pgfqpoint{3.561978in}{2.697450in}}%
\pgfpathlineto{\pgfqpoint{3.575216in}{2.689767in}}%
\pgfpathlineto{\pgfqpoint{3.588456in}{2.682226in}}%
\pgfpathlineto{\pgfqpoint{3.580596in}{2.674111in}}%
\pgfpathlineto{\pgfqpoint{3.572729in}{2.666038in}}%
\pgfpathlineto{\pgfqpoint{3.564856in}{2.658008in}}%
\pgfpathlineto{\pgfqpoint{3.556978in}{2.650022in}}%
\pgfpathlineto{\pgfqpoint{3.543723in}{2.657632in}}%
\pgfpathlineto{\pgfqpoint{3.530471in}{2.665385in}}%
\pgfpathlineto{\pgfqpoint{3.517221in}{2.673280in}}%
\pgfpathlineto{\pgfqpoint{3.503973in}{2.681319in}}%
\pgfpathlineto{\pgfqpoint{3.511866in}{2.689230in}}%
\pgfpathlineto{\pgfqpoint{3.519753in}{2.697188in}}%
\pgfpathlineto{\pgfqpoint{3.527635in}{2.705193in}}%
\pgfpathlineto{\pgfqpoint{3.535510in}{2.713245in}}%
\pgfpathclose%
\pgfusepath{fill}%
\end{pgfscope}%
\begin{pgfscope}%
\pgfpathrectangle{\pgfqpoint{1.150000in}{0.150000in}}{\pgfqpoint{5.700000in}{5.700000in}}%
\pgfusepath{clip}%
\pgfsetbuttcap%
\pgfsetroundjoin%
\definecolor{currentfill}{rgb}{0.274128,0.199721,0.498911}%
\pgfsetfillcolor{currentfill}%
\pgfsetfillopacity{0.700000}%
\pgfsetlinewidth{0.000000pt}%
\definecolor{currentstroke}{rgb}{0.000000,0.000000,0.000000}%
\pgfsetstrokecolor{currentstroke}%
\pgfsetdash{}{0pt}%
\pgfpathmoveto{\pgfqpoint{3.101274in}{2.940561in}}%
\pgfpathlineto{\pgfqpoint{3.114533in}{2.927941in}}%
\pgfpathlineto{\pgfqpoint{3.127791in}{2.915500in}}%
\pgfpathlineto{\pgfqpoint{3.141047in}{2.903237in}}%
\pgfpathlineto{\pgfqpoint{3.154302in}{2.891148in}}%
\pgfpathlineto{\pgfqpoint{3.146273in}{2.884085in}}%
\pgfpathlineto{\pgfqpoint{3.138238in}{2.877100in}}%
\pgfpathlineto{\pgfqpoint{3.130194in}{2.870193in}}%
\pgfpathlineto{\pgfqpoint{3.122144in}{2.863367in}}%
\pgfpathlineto{\pgfqpoint{3.108870in}{2.875564in}}%
\pgfpathlineto{\pgfqpoint{3.095594in}{2.887937in}}%
\pgfpathlineto{\pgfqpoint{3.082317in}{2.900488in}}%
\pgfpathlineto{\pgfqpoint{3.069038in}{2.913217in}}%
\pgfpathlineto{\pgfqpoint{3.077108in}{2.919928in}}%
\pgfpathlineto{\pgfqpoint{3.085171in}{2.926722in}}%
\pgfpathlineto{\pgfqpoint{3.093226in}{2.933600in}}%
\pgfpathlineto{\pgfqpoint{3.101274in}{2.940561in}}%
\pgfpathclose%
\pgfusepath{fill}%
\end{pgfscope}%
\begin{pgfscope}%
\pgfpathrectangle{\pgfqpoint{1.150000in}{0.150000in}}{\pgfqpoint{5.700000in}{5.700000in}}%
\pgfusepath{clip}%
\pgfsetbuttcap%
\pgfsetroundjoin%
\definecolor{currentfill}{rgb}{0.278791,0.062145,0.386592}%
\pgfsetfillcolor{currentfill}%
\pgfsetfillopacity{0.700000}%
\pgfsetlinewidth{0.000000pt}%
\definecolor{currentstroke}{rgb}{0.000000,0.000000,0.000000}%
\pgfsetstrokecolor{currentstroke}%
\pgfsetdash{}{0pt}%
\pgfpathmoveto{\pgfqpoint{3.947070in}{2.662416in}}%
\pgfpathlineto{\pgfqpoint{3.960353in}{2.657755in}}%
\pgfpathlineto{\pgfqpoint{3.973641in}{2.653219in}}%
\pgfpathlineto{\pgfqpoint{3.986934in}{2.648806in}}%
\pgfpathlineto{\pgfqpoint{4.000233in}{2.644517in}}%
\pgfpathlineto{\pgfqpoint{3.992515in}{2.635855in}}%
\pgfpathlineto{\pgfqpoint{3.984792in}{2.627208in}}%
\pgfpathlineto{\pgfqpoint{3.977064in}{2.618576in}}%
\pgfpathlineto{\pgfqpoint{3.969331in}{2.609959in}}%
\pgfpathlineto{\pgfqpoint{3.956020in}{2.614263in}}%
\pgfpathlineto{\pgfqpoint{3.942716in}{2.618690in}}%
\pgfpathlineto{\pgfqpoint{3.929416in}{2.623241in}}%
\pgfpathlineto{\pgfqpoint{3.916121in}{2.627916in}}%
\pgfpathlineto{\pgfqpoint{3.923866in}{2.636513in}}%
\pgfpathlineto{\pgfqpoint{3.931606in}{2.645128in}}%
\pgfpathlineto{\pgfqpoint{3.939341in}{2.653762in}}%
\pgfpathlineto{\pgfqpoint{3.947070in}{2.662416in}}%
\pgfpathclose%
\pgfusepath{fill}%
\end{pgfscope}%
\begin{pgfscope}%
\pgfpathrectangle{\pgfqpoint{1.150000in}{0.150000in}}{\pgfqpoint{5.700000in}{5.700000in}}%
\pgfusepath{clip}%
\pgfsetbuttcap%
\pgfsetroundjoin%
\definecolor{currentfill}{rgb}{0.280868,0.160771,0.472899}%
\pgfsetfillcolor{currentfill}%
\pgfsetfillopacity{0.700000}%
\pgfsetlinewidth{0.000000pt}%
\definecolor{currentstroke}{rgb}{0.000000,0.000000,0.000000}%
\pgfsetstrokecolor{currentstroke}%
\pgfsetdash{}{0pt}%
\pgfpathmoveto{\pgfqpoint{4.778363in}{2.835592in}}%
\pgfpathlineto{\pgfqpoint{4.791874in}{2.835401in}}%
\pgfpathlineto{\pgfqpoint{4.805393in}{2.835314in}}%
\pgfpathlineto{\pgfqpoint{4.818921in}{2.835332in}}%
\pgfpathlineto{\pgfqpoint{4.832459in}{2.835455in}}%
\pgfpathlineto{\pgfqpoint{4.825030in}{2.827416in}}%
\pgfpathlineto{\pgfqpoint{4.817596in}{2.819375in}}%
\pgfpathlineto{\pgfqpoint{4.810156in}{2.811329in}}%
\pgfpathlineto{\pgfqpoint{4.802712in}{2.803278in}}%
\pgfpathlineto{\pgfqpoint{4.789162in}{2.803024in}}%
\pgfpathlineto{\pgfqpoint{4.775621in}{2.802875in}}%
\pgfpathlineto{\pgfqpoint{4.762090in}{2.802831in}}%
\pgfpathlineto{\pgfqpoint{4.748568in}{2.802893in}}%
\pgfpathlineto{\pgfqpoint{4.756024in}{2.811068in}}%
\pgfpathlineto{\pgfqpoint{4.763476in}{2.819242in}}%
\pgfpathlineto{\pgfqpoint{4.770922in}{2.827416in}}%
\pgfpathlineto{\pgfqpoint{4.778363in}{2.835592in}}%
\pgfpathclose%
\pgfusepath{fill}%
\end{pgfscope}%
\begin{pgfscope}%
\pgfpathrectangle{\pgfqpoint{1.150000in}{0.150000in}}{\pgfqpoint{5.700000in}{5.700000in}}%
\pgfusepath{clip}%
\pgfsetbuttcap%
\pgfsetroundjoin%
\definecolor{currentfill}{rgb}{0.282656,0.100196,0.422160}%
\pgfsetfillcolor{currentfill}%
\pgfsetfillopacity{0.700000}%
\pgfsetlinewidth{0.000000pt}%
\definecolor{currentstroke}{rgb}{0.000000,0.000000,0.000000}%
\pgfsetstrokecolor{currentstroke}%
\pgfsetdash{}{0pt}%
\pgfpathmoveto{\pgfqpoint{4.389362in}{2.722279in}}%
\pgfpathlineto{\pgfqpoint{4.402752in}{2.720328in}}%
\pgfpathlineto{\pgfqpoint{4.416149in}{2.718490in}}%
\pgfpathlineto{\pgfqpoint{4.429553in}{2.716763in}}%
\pgfpathlineto{\pgfqpoint{4.442966in}{2.715148in}}%
\pgfpathlineto{\pgfqpoint{4.435397in}{2.706564in}}%
\pgfpathlineto{\pgfqpoint{4.427823in}{2.697978in}}%
\pgfpathlineto{\pgfqpoint{4.420244in}{2.689389in}}%
\pgfpathlineto{\pgfqpoint{4.412660in}{2.680797in}}%
\pgfpathlineto{\pgfqpoint{4.399237in}{2.682354in}}%
\pgfpathlineto{\pgfqpoint{4.385821in}{2.684022in}}%
\pgfpathlineto{\pgfqpoint{4.372413in}{2.685803in}}%
\pgfpathlineto{\pgfqpoint{4.359013in}{2.687695in}}%
\pgfpathlineto{\pgfqpoint{4.366608in}{2.696339in}}%
\pgfpathlineto{\pgfqpoint{4.374198in}{2.704984in}}%
\pgfpathlineto{\pgfqpoint{4.381782in}{2.713630in}}%
\pgfpathlineto{\pgfqpoint{4.389362in}{2.722279in}}%
\pgfpathclose%
\pgfusepath{fill}%
\end{pgfscope}%
\begin{pgfscope}%
\pgfpathrectangle{\pgfqpoint{1.150000in}{0.150000in}}{\pgfqpoint{5.700000in}{5.700000in}}%
\pgfusepath{clip}%
\pgfsetbuttcap%
\pgfsetroundjoin%
\definecolor{currentfill}{rgb}{0.278012,0.180367,0.486697}%
\pgfsetfillcolor{currentfill}%
\pgfsetfillopacity{0.700000}%
\pgfsetlinewidth{0.000000pt}%
\definecolor{currentstroke}{rgb}{0.000000,0.000000,0.000000}%
\pgfsetstrokecolor{currentstroke}%
\pgfsetdash{}{0pt}%
\pgfpathmoveto{\pgfqpoint{3.154302in}{2.891148in}}%
\pgfpathlineto{\pgfqpoint{3.167555in}{2.879234in}}%
\pgfpathlineto{\pgfqpoint{3.180807in}{2.867493in}}%
\pgfpathlineto{\pgfqpoint{3.194058in}{2.855923in}}%
\pgfpathlineto{\pgfqpoint{3.207308in}{2.844522in}}%
\pgfpathlineto{\pgfqpoint{3.199298in}{2.837357in}}%
\pgfpathlineto{\pgfqpoint{3.191281in}{2.830266in}}%
\pgfpathlineto{\pgfqpoint{3.183257in}{2.823248in}}%
\pgfpathlineto{\pgfqpoint{3.175226in}{2.816306in}}%
\pgfpathlineto{\pgfqpoint{3.161957in}{2.827815in}}%
\pgfpathlineto{\pgfqpoint{3.148687in}{2.839494in}}%
\pgfpathlineto{\pgfqpoint{3.135416in}{2.851344in}}%
\pgfpathlineto{\pgfqpoint{3.122144in}{2.863367in}}%
\pgfpathlineto{\pgfqpoint{3.130194in}{2.870193in}}%
\pgfpathlineto{\pgfqpoint{3.138238in}{2.877100in}}%
\pgfpathlineto{\pgfqpoint{3.146273in}{2.884085in}}%
\pgfpathlineto{\pgfqpoint{3.154302in}{2.891148in}}%
\pgfpathclose%
\pgfusepath{fill}%
\end{pgfscope}%
\begin{pgfscope}%
\pgfpathrectangle{\pgfqpoint{1.150000in}{0.150000in}}{\pgfqpoint{5.700000in}{5.700000in}}%
\pgfusepath{clip}%
\pgfsetbuttcap%
\pgfsetroundjoin%
\definecolor{currentfill}{rgb}{0.283091,0.110553,0.431554}%
\pgfsetfillcolor{currentfill}%
\pgfsetfillopacity{0.700000}%
\pgfsetlinewidth{0.000000pt}%
\definecolor{currentstroke}{rgb}{0.000000,0.000000,0.000000}%
\pgfsetstrokecolor{currentstroke}%
\pgfsetdash{}{0pt}%
\pgfpathmoveto{\pgfqpoint{3.398053in}{2.750923in}}%
\pgfpathlineto{\pgfqpoint{3.411288in}{2.741698in}}%
\pgfpathlineto{\pgfqpoint{3.424524in}{2.732625in}}%
\pgfpathlineto{\pgfqpoint{3.437761in}{2.723703in}}%
\pgfpathlineto{\pgfqpoint{3.451000in}{2.714931in}}%
\pgfpathlineto{\pgfqpoint{3.443085in}{2.707149in}}%
\pgfpathlineto{\pgfqpoint{3.435164in}{2.699421in}}%
\pgfpathlineto{\pgfqpoint{3.427236in}{2.691746in}}%
\pgfpathlineto{\pgfqpoint{3.419302in}{2.684125in}}%
\pgfpathlineto{\pgfqpoint{3.406047in}{2.692985in}}%
\pgfpathlineto{\pgfqpoint{3.392794in}{2.701995in}}%
\pgfpathlineto{\pgfqpoint{3.379541in}{2.711156in}}%
\pgfpathlineto{\pgfqpoint{3.366290in}{2.720469in}}%
\pgfpathlineto{\pgfqpoint{3.374241in}{2.727995in}}%
\pgfpathlineto{\pgfqpoint{3.382184in}{2.735580in}}%
\pgfpathlineto{\pgfqpoint{3.390122in}{2.743223in}}%
\pgfpathlineto{\pgfqpoint{3.398053in}{2.750923in}}%
\pgfpathclose%
\pgfusepath{fill}%
\end{pgfscope}%
\begin{pgfscope}%
\pgfpathrectangle{\pgfqpoint{1.150000in}{0.150000in}}{\pgfqpoint{5.700000in}{5.700000in}}%
\pgfusepath{clip}%
\pgfsetbuttcap%
\pgfsetroundjoin%
\definecolor{currentfill}{rgb}{0.279566,0.067836,0.391917}%
\pgfsetfillcolor{currentfill}%
\pgfsetfillopacity{0.700000}%
\pgfsetlinewidth{0.000000pt}%
\definecolor{currentstroke}{rgb}{0.000000,0.000000,0.000000}%
\pgfsetstrokecolor{currentstroke}%
\pgfsetdash{}{0pt}%
\pgfpathmoveto{\pgfqpoint{4.084254in}{2.663367in}}%
\pgfpathlineto{\pgfqpoint{4.097569in}{2.659681in}}%
\pgfpathlineto{\pgfqpoint{4.110891in}{2.656115in}}%
\pgfpathlineto{\pgfqpoint{4.124218in}{2.652668in}}%
\pgfpathlineto{\pgfqpoint{4.137552in}{2.649340in}}%
\pgfpathlineto{\pgfqpoint{4.129878in}{2.640636in}}%
\pgfpathlineto{\pgfqpoint{4.122198in}{2.631939in}}%
\pgfpathlineto{\pgfqpoint{4.114514in}{2.623249in}}%
\pgfpathlineto{\pgfqpoint{4.106824in}{2.614567in}}%
\pgfpathlineto{\pgfqpoint{4.093480in}{2.617891in}}%
\pgfpathlineto{\pgfqpoint{4.080141in}{2.621334in}}%
\pgfpathlineto{\pgfqpoint{4.066809in}{2.624897in}}%
\pgfpathlineto{\pgfqpoint{4.053482in}{2.628579in}}%
\pgfpathlineto{\pgfqpoint{4.061183in}{2.637259in}}%
\pgfpathlineto{\pgfqpoint{4.068878in}{2.645950in}}%
\pgfpathlineto{\pgfqpoint{4.076569in}{2.654653in}}%
\pgfpathlineto{\pgfqpoint{4.084254in}{2.663367in}}%
\pgfpathclose%
\pgfusepath{fill}%
\end{pgfscope}%
\begin{pgfscope}%
\pgfpathrectangle{\pgfqpoint{1.150000in}{0.150000in}}{\pgfqpoint{5.700000in}{5.700000in}}%
\pgfusepath{clip}%
\pgfsetbuttcap%
\pgfsetroundjoin%
\definecolor{currentfill}{rgb}{0.282290,0.145912,0.461510}%
\pgfsetfillcolor{currentfill}%
\pgfsetfillopacity{0.700000}%
\pgfsetlinewidth{0.000000pt}%
\definecolor{currentstroke}{rgb}{0.000000,0.000000,0.000000}%
\pgfsetstrokecolor{currentstroke}%
\pgfsetdash{}{0pt}%
\pgfpathmoveto{\pgfqpoint{4.694569in}{2.804195in}}%
\pgfpathlineto{\pgfqpoint{4.708056in}{2.803711in}}%
\pgfpathlineto{\pgfqpoint{4.721551in}{2.803332in}}%
\pgfpathlineto{\pgfqpoint{4.735055in}{2.803060in}}%
\pgfpathlineto{\pgfqpoint{4.748568in}{2.802893in}}%
\pgfpathlineto{\pgfqpoint{4.741106in}{2.794715in}}%
\pgfpathlineto{\pgfqpoint{4.733639in}{2.786532in}}%
\pgfpathlineto{\pgfqpoint{4.726167in}{2.778343in}}%
\pgfpathlineto{\pgfqpoint{4.718689in}{2.770147in}}%
\pgfpathlineto{\pgfqpoint{4.705165in}{2.770201in}}%
\pgfpathlineto{\pgfqpoint{4.691649in}{2.770361in}}%
\pgfpathlineto{\pgfqpoint{4.678142in}{2.770627in}}%
\pgfpathlineto{\pgfqpoint{4.664644in}{2.770999in}}%
\pgfpathlineto{\pgfqpoint{4.672133in}{2.779302in}}%
\pgfpathlineto{\pgfqpoint{4.679617in}{2.787601in}}%
\pgfpathlineto{\pgfqpoint{4.687096in}{2.795898in}}%
\pgfpathlineto{\pgfqpoint{4.694569in}{2.804195in}}%
\pgfpathclose%
\pgfusepath{fill}%
\end{pgfscope}%
\begin{pgfscope}%
\pgfpathrectangle{\pgfqpoint{1.150000in}{0.150000in}}{\pgfqpoint{5.700000in}{5.700000in}}%
\pgfusepath{clip}%
\pgfsetbuttcap%
\pgfsetroundjoin%
\definecolor{currentfill}{rgb}{0.279566,0.067836,0.391917}%
\pgfsetfillcolor{currentfill}%
\pgfsetfillopacity{0.700000}%
\pgfsetlinewidth{0.000000pt}%
\definecolor{currentstroke}{rgb}{0.000000,0.000000,0.000000}%
\pgfsetstrokecolor{currentstroke}%
\pgfsetdash{}{0pt}%
\pgfpathmoveto{\pgfqpoint{3.725758in}{2.660172in}}%
\pgfpathlineto{\pgfqpoint{3.739013in}{2.653917in}}%
\pgfpathlineto{\pgfqpoint{3.752271in}{2.647794in}}%
\pgfpathlineto{\pgfqpoint{3.765534in}{2.641804in}}%
\pgfpathlineto{\pgfqpoint{3.778800in}{2.635945in}}%
\pgfpathlineto{\pgfqpoint{3.771003in}{2.627531in}}%
\pgfpathlineto{\pgfqpoint{3.763200in}{2.619145in}}%
\pgfpathlineto{\pgfqpoint{3.755392in}{2.610787in}}%
\pgfpathlineto{\pgfqpoint{3.747578in}{2.602457in}}%
\pgfpathlineto{\pgfqpoint{3.734299in}{2.608366in}}%
\pgfpathlineto{\pgfqpoint{3.721023in}{2.614407in}}%
\pgfpathlineto{\pgfqpoint{3.707751in}{2.620580in}}%
\pgfpathlineto{\pgfqpoint{3.694483in}{2.626886in}}%
\pgfpathlineto{\pgfqpoint{3.702310in}{2.635159in}}%
\pgfpathlineto{\pgfqpoint{3.710132in}{2.643464in}}%
\pgfpathlineto{\pgfqpoint{3.717948in}{2.651802in}}%
\pgfpathlineto{\pgfqpoint{3.725758in}{2.660172in}}%
\pgfpathclose%
\pgfusepath{fill}%
\end{pgfscope}%
\begin{pgfscope}%
\pgfpathrectangle{\pgfqpoint{1.150000in}{0.150000in}}{\pgfqpoint{5.700000in}{5.700000in}}%
\pgfusepath{clip}%
\pgfsetbuttcap%
\pgfsetroundjoin%
\definecolor{currentfill}{rgb}{0.281924,0.089666,0.412415}%
\pgfsetfillcolor{currentfill}%
\pgfsetfillopacity{0.700000}%
\pgfsetlinewidth{0.000000pt}%
\definecolor{currentstroke}{rgb}{0.000000,0.000000,0.000000}%
\pgfsetstrokecolor{currentstroke}%
\pgfsetdash{}{0pt}%
\pgfpathmoveto{\pgfqpoint{4.305484in}{2.696394in}}%
\pgfpathlineto{\pgfqpoint{4.318855in}{2.694049in}}%
\pgfpathlineto{\pgfqpoint{4.332234in}{2.691818in}}%
\pgfpathlineto{\pgfqpoint{4.345620in}{2.689700in}}%
\pgfpathlineto{\pgfqpoint{4.359013in}{2.687695in}}%
\pgfpathlineto{\pgfqpoint{4.351413in}{2.679051in}}%
\pgfpathlineto{\pgfqpoint{4.343808in}{2.670406in}}%
\pgfpathlineto{\pgfqpoint{4.336197in}{2.661759in}}%
\pgfpathlineto{\pgfqpoint{4.328582in}{2.653109in}}%
\pgfpathlineto{\pgfqpoint{4.315178in}{2.655074in}}%
\pgfpathlineto{\pgfqpoint{4.301781in}{2.657152in}}%
\pgfpathlineto{\pgfqpoint{4.288392in}{2.659343in}}%
\pgfpathlineto{\pgfqpoint{4.275009in}{2.661648in}}%
\pgfpathlineto{\pgfqpoint{4.282636in}{2.670331in}}%
\pgfpathlineto{\pgfqpoint{4.290257in}{2.679016in}}%
\pgfpathlineto{\pgfqpoint{4.297873in}{2.687703in}}%
\pgfpathlineto{\pgfqpoint{4.305484in}{2.696394in}}%
\pgfpathclose%
\pgfusepath{fill}%
\end{pgfscope}%
\begin{pgfscope}%
\pgfpathrectangle{\pgfqpoint{1.150000in}{0.150000in}}{\pgfqpoint{5.700000in}{5.700000in}}%
\pgfusepath{clip}%
\pgfsetbuttcap%
\pgfsetroundjoin%
\definecolor{currentfill}{rgb}{0.265145,0.232956,0.516599}%
\pgfsetfillcolor{currentfill}%
\pgfsetfillopacity{0.700000}%
\pgfsetlinewidth{0.000000pt}%
\definecolor{currentstroke}{rgb}{0.000000,0.000000,0.000000}%
\pgfsetstrokecolor{currentstroke}%
\pgfsetdash{}{0pt}%
\pgfpathmoveto{\pgfqpoint{5.167723in}{2.970060in}}%
\pgfpathlineto{\pgfqpoint{5.181375in}{2.971218in}}%
\pgfpathlineto{\pgfqpoint{5.195038in}{2.972476in}}%
\pgfpathlineto{\pgfqpoint{5.208712in}{2.973834in}}%
\pgfpathlineto{\pgfqpoint{5.222397in}{2.975292in}}%
\pgfpathlineto{\pgfqpoint{5.215118in}{2.968068in}}%
\pgfpathlineto{\pgfqpoint{5.207834in}{2.960853in}}%
\pgfpathlineto{\pgfqpoint{5.200545in}{2.953645in}}%
\pgfpathlineto{\pgfqpoint{5.193251in}{2.946442in}}%
\pgfpathlineto{\pgfqpoint{5.179551in}{2.944780in}}%
\pgfpathlineto{\pgfqpoint{5.165862in}{2.943217in}}%
\pgfpathlineto{\pgfqpoint{5.152184in}{2.941756in}}%
\pgfpathlineto{\pgfqpoint{5.138517in}{2.940394in}}%
\pgfpathlineto{\pgfqpoint{5.145825in}{2.947795in}}%
\pgfpathlineto{\pgfqpoint{5.153129in}{2.955205in}}%
\pgfpathlineto{\pgfqpoint{5.160428in}{2.962625in}}%
\pgfpathlineto{\pgfqpoint{5.167723in}{2.970060in}}%
\pgfpathclose%
\pgfusepath{fill}%
\end{pgfscope}%
\begin{pgfscope}%
\pgfpathrectangle{\pgfqpoint{1.150000in}{0.150000in}}{\pgfqpoint{5.700000in}{5.700000in}}%
\pgfusepath{clip}%
\pgfsetbuttcap%
\pgfsetroundjoin%
\definecolor{currentfill}{rgb}{0.280894,0.078907,0.402329}%
\pgfsetfillcolor{currentfill}%
\pgfsetfillopacity{0.700000}%
\pgfsetlinewidth{0.000000pt}%
\definecolor{currentstroke}{rgb}{0.000000,0.000000,0.000000}%
\pgfsetstrokecolor{currentstroke}%
\pgfsetdash{}{0pt}%
\pgfpathmoveto{\pgfqpoint{3.588456in}{2.682226in}}%
\pgfpathlineto{\pgfqpoint{3.601699in}{2.674825in}}%
\pgfpathlineto{\pgfqpoint{3.614945in}{2.667564in}}%
\pgfpathlineto{\pgfqpoint{3.628193in}{2.660442in}}%
\pgfpathlineto{\pgfqpoint{3.641445in}{2.653458in}}%
\pgfpathlineto{\pgfqpoint{3.633598in}{2.645281in}}%
\pgfpathlineto{\pgfqpoint{3.625746in}{2.637142in}}%
\pgfpathlineto{\pgfqpoint{3.617888in}{2.629041in}}%
\pgfpathlineto{\pgfqpoint{3.610023in}{2.620978in}}%
\pgfpathlineto{\pgfqpoint{3.596758in}{2.628031in}}%
\pgfpathlineto{\pgfqpoint{3.583495in}{2.635222in}}%
\pgfpathlineto{\pgfqpoint{3.570235in}{2.642552in}}%
\pgfpathlineto{\pgfqpoint{3.556978in}{2.650022in}}%
\pgfpathlineto{\pgfqpoint{3.564856in}{2.658008in}}%
\pgfpathlineto{\pgfqpoint{3.572729in}{2.666038in}}%
\pgfpathlineto{\pgfqpoint{3.580596in}{2.674111in}}%
\pgfpathlineto{\pgfqpoint{3.588456in}{2.682226in}}%
\pgfpathclose%
\pgfusepath{fill}%
\end{pgfscope}%
\begin{pgfscope}%
\pgfpathrectangle{\pgfqpoint{1.150000in}{0.150000in}}{\pgfqpoint{5.700000in}{5.700000in}}%
\pgfusepath{clip}%
\pgfsetbuttcap%
\pgfsetroundjoin%
\definecolor{currentfill}{rgb}{0.280868,0.160771,0.472899}%
\pgfsetfillcolor{currentfill}%
\pgfsetfillopacity{0.700000}%
\pgfsetlinewidth{0.000000pt}%
\definecolor{currentstroke}{rgb}{0.000000,0.000000,0.000000}%
\pgfsetstrokecolor{currentstroke}%
\pgfsetdash{}{0pt}%
\pgfpathmoveto{\pgfqpoint{3.207308in}{2.844522in}}%
\pgfpathlineto{\pgfqpoint{3.220557in}{2.833290in}}%
\pgfpathlineto{\pgfqpoint{3.233806in}{2.822225in}}%
\pgfpathlineto{\pgfqpoint{3.247054in}{2.811325in}}%
\pgfpathlineto{\pgfqpoint{3.260302in}{2.800590in}}%
\pgfpathlineto{\pgfqpoint{3.252310in}{2.793324in}}%
\pgfpathlineto{\pgfqpoint{3.244312in}{2.786127in}}%
\pgfpathlineto{\pgfqpoint{3.236306in}{2.778999in}}%
\pgfpathlineto{\pgfqpoint{3.228293in}{2.771943in}}%
\pgfpathlineto{\pgfqpoint{3.215027in}{2.782785in}}%
\pgfpathlineto{\pgfqpoint{3.201761in}{2.793793in}}%
\pgfpathlineto{\pgfqpoint{3.188494in}{2.804966in}}%
\pgfpathlineto{\pgfqpoint{3.175226in}{2.816306in}}%
\pgfpathlineto{\pgfqpoint{3.183257in}{2.823248in}}%
\pgfpathlineto{\pgfqpoint{3.191281in}{2.830266in}}%
\pgfpathlineto{\pgfqpoint{3.199298in}{2.837357in}}%
\pgfpathlineto{\pgfqpoint{3.207308in}{2.844522in}}%
\pgfpathclose%
\pgfusepath{fill}%
\end{pgfscope}%
\begin{pgfscope}%
\pgfpathrectangle{\pgfqpoint{1.150000in}{0.150000in}}{\pgfqpoint{5.700000in}{5.700000in}}%
\pgfusepath{clip}%
\pgfsetbuttcap%
\pgfsetroundjoin%
\definecolor{currentfill}{rgb}{0.278791,0.062145,0.386592}%
\pgfsetfillcolor{currentfill}%
\pgfsetfillopacity{0.700000}%
\pgfsetlinewidth{0.000000pt}%
\definecolor{currentstroke}{rgb}{0.000000,0.000000,0.000000}%
\pgfsetstrokecolor{currentstroke}%
\pgfsetdash{}{0pt}%
\pgfpathmoveto{\pgfqpoint{3.862991in}{2.647876in}}%
\pgfpathlineto{\pgfqpoint{3.876266in}{2.642696in}}%
\pgfpathlineto{\pgfqpoint{3.889546in}{2.637643in}}%
\pgfpathlineto{\pgfqpoint{3.902831in}{2.632717in}}%
\pgfpathlineto{\pgfqpoint{3.916121in}{2.627916in}}%
\pgfpathlineto{\pgfqpoint{3.908370in}{2.619339in}}%
\pgfpathlineto{\pgfqpoint{3.900615in}{2.610781in}}%
\pgfpathlineto{\pgfqpoint{3.892853in}{2.602242in}}%
\pgfpathlineto{\pgfqpoint{3.885086in}{2.593721in}}%
\pgfpathlineto{\pgfqpoint{3.871785in}{2.598554in}}%
\pgfpathlineto{\pgfqpoint{3.858488in}{2.603513in}}%
\pgfpathlineto{\pgfqpoint{3.845195in}{2.608598in}}%
\pgfpathlineto{\pgfqpoint{3.831907in}{2.613810in}}%
\pgfpathlineto{\pgfqpoint{3.839686in}{2.622291in}}%
\pgfpathlineto{\pgfqpoint{3.847460in}{2.630796in}}%
\pgfpathlineto{\pgfqpoint{3.855228in}{2.639324in}}%
\pgfpathlineto{\pgfqpoint{3.862991in}{2.647876in}}%
\pgfpathclose%
\pgfusepath{fill}%
\end{pgfscope}%
\begin{pgfscope}%
\pgfpathrectangle{\pgfqpoint{1.150000in}{0.150000in}}{\pgfqpoint{5.700000in}{5.700000in}}%
\pgfusepath{clip}%
\pgfsetbuttcap%
\pgfsetroundjoin%
\definecolor{currentfill}{rgb}{0.283072,0.130895,0.449241}%
\pgfsetfillcolor{currentfill}%
\pgfsetfillopacity{0.700000}%
\pgfsetlinewidth{0.000000pt}%
\definecolor{currentstroke}{rgb}{0.000000,0.000000,0.000000}%
\pgfsetstrokecolor{currentstroke}%
\pgfsetdash{}{0pt}%
\pgfpathmoveto{\pgfqpoint{4.610740in}{2.773559in}}%
\pgfpathlineto{\pgfqpoint{4.624203in}{2.772758in}}%
\pgfpathlineto{\pgfqpoint{4.637675in}{2.772065in}}%
\pgfpathlineto{\pgfqpoint{4.651155in}{2.771479in}}%
\pgfpathlineto{\pgfqpoint{4.664644in}{2.770999in}}%
\pgfpathlineto{\pgfqpoint{4.657150in}{2.762693in}}%
\pgfpathlineto{\pgfqpoint{4.649651in}{2.754380in}}%
\pgfpathlineto{\pgfqpoint{4.642147in}{2.746061in}}%
\pgfpathlineto{\pgfqpoint{4.634637in}{2.737733in}}%
\pgfpathlineto{\pgfqpoint{4.621136in}{2.738117in}}%
\pgfpathlineto{\pgfqpoint{4.607645in}{2.738609in}}%
\pgfpathlineto{\pgfqpoint{4.594162in}{2.739208in}}%
\pgfpathlineto{\pgfqpoint{4.580687in}{2.739915in}}%
\pgfpathlineto{\pgfqpoint{4.588208in}{2.748331in}}%
\pgfpathlineto{\pgfqpoint{4.595724in}{2.756743in}}%
\pgfpathlineto{\pgfqpoint{4.603235in}{2.765152in}}%
\pgfpathlineto{\pgfqpoint{4.610740in}{2.773559in}}%
\pgfpathclose%
\pgfusepath{fill}%
\end{pgfscope}%
\begin{pgfscope}%
\pgfpathrectangle{\pgfqpoint{1.150000in}{0.150000in}}{\pgfqpoint{5.700000in}{5.700000in}}%
\pgfusepath{clip}%
\pgfsetbuttcap%
\pgfsetroundjoin%
\definecolor{currentfill}{rgb}{0.269308,0.218818,0.509577}%
\pgfsetfillcolor{currentfill}%
\pgfsetfillopacity{0.700000}%
\pgfsetlinewidth{0.000000pt}%
\definecolor{currentstroke}{rgb}{0.000000,0.000000,0.000000}%
\pgfsetstrokecolor{currentstroke}%
\pgfsetdash{}{0pt}%
\pgfpathmoveto{\pgfqpoint{5.083951in}{2.935956in}}%
\pgfpathlineto{\pgfqpoint{5.097577in}{2.936914in}}%
\pgfpathlineto{\pgfqpoint{5.111213in}{2.937973in}}%
\pgfpathlineto{\pgfqpoint{5.124860in}{2.939133in}}%
\pgfpathlineto{\pgfqpoint{5.138517in}{2.940394in}}%
\pgfpathlineto{\pgfqpoint{5.131203in}{2.933000in}}%
\pgfpathlineto{\pgfqpoint{5.123883in}{2.925609in}}%
\pgfpathlineto{\pgfqpoint{5.116559in}{2.918221in}}%
\pgfpathlineto{\pgfqpoint{5.109230in}{2.910832in}}%
\pgfpathlineto{\pgfqpoint{5.095558in}{2.909385in}}%
\pgfpathlineto{\pgfqpoint{5.081898in}{2.908039in}}%
\pgfpathlineto{\pgfqpoint{5.068247in}{2.906795in}}%
\pgfpathlineto{\pgfqpoint{5.054608in}{2.905651in}}%
\pgfpathlineto{\pgfqpoint{5.061951in}{2.913219in}}%
\pgfpathlineto{\pgfqpoint{5.069290in}{2.920791in}}%
\pgfpathlineto{\pgfqpoint{5.076623in}{2.928369in}}%
\pgfpathlineto{\pgfqpoint{5.083951in}{2.935956in}}%
\pgfpathclose%
\pgfusepath{fill}%
\end{pgfscope}%
\begin{pgfscope}%
\pgfpathrectangle{\pgfqpoint{1.150000in}{0.150000in}}{\pgfqpoint{5.700000in}{5.700000in}}%
\pgfusepath{clip}%
\pgfsetbuttcap%
\pgfsetroundjoin%
\definecolor{currentfill}{rgb}{0.282327,0.094955,0.417331}%
\pgfsetfillcolor{currentfill}%
\pgfsetfillopacity{0.700000}%
\pgfsetlinewidth{0.000000pt}%
\definecolor{currentstroke}{rgb}{0.000000,0.000000,0.000000}%
\pgfsetstrokecolor{currentstroke}%
\pgfsetdash{}{0pt}%
\pgfpathmoveto{\pgfqpoint{3.451000in}{2.714931in}}%
\pgfpathlineto{\pgfqpoint{3.464241in}{2.706307in}}%
\pgfpathlineto{\pgfqpoint{3.477483in}{2.697832in}}%
\pgfpathlineto{\pgfqpoint{3.490727in}{2.689502in}}%
\pgfpathlineto{\pgfqpoint{3.503973in}{2.681319in}}%
\pgfpathlineto{\pgfqpoint{3.496073in}{2.673456in}}%
\pgfpathlineto{\pgfqpoint{3.488167in}{2.665642in}}%
\pgfpathlineto{\pgfqpoint{3.480255in}{2.657878in}}%
\pgfpathlineto{\pgfqpoint{3.472337in}{2.650162in}}%
\pgfpathlineto{\pgfqpoint{3.459075in}{2.658433in}}%
\pgfpathlineto{\pgfqpoint{3.445816in}{2.666850in}}%
\pgfpathlineto{\pgfqpoint{3.432558in}{2.675414in}}%
\pgfpathlineto{\pgfqpoint{3.419302in}{2.684125in}}%
\pgfpathlineto{\pgfqpoint{3.427236in}{2.691746in}}%
\pgfpathlineto{\pgfqpoint{3.435164in}{2.699421in}}%
\pgfpathlineto{\pgfqpoint{3.443085in}{2.707149in}}%
\pgfpathlineto{\pgfqpoint{3.451000in}{2.714931in}}%
\pgfpathclose%
\pgfusepath{fill}%
\end{pgfscope}%
\begin{pgfscope}%
\pgfpathrectangle{\pgfqpoint{1.150000in}{0.150000in}}{\pgfqpoint{5.700000in}{5.700000in}}%
\pgfusepath{clip}%
\pgfsetbuttcap%
\pgfsetroundjoin%
\definecolor{currentfill}{rgb}{0.278791,0.062145,0.386592}%
\pgfsetfillcolor{currentfill}%
\pgfsetfillopacity{0.700000}%
\pgfsetlinewidth{0.000000pt}%
\definecolor{currentstroke}{rgb}{0.000000,0.000000,0.000000}%
\pgfsetstrokecolor{currentstroke}%
\pgfsetdash{}{0pt}%
\pgfpathmoveto{\pgfqpoint{4.000233in}{2.644517in}}%
\pgfpathlineto{\pgfqpoint{4.013537in}{2.640350in}}%
\pgfpathlineto{\pgfqpoint{4.026846in}{2.636305in}}%
\pgfpathlineto{\pgfqpoint{4.040161in}{2.632382in}}%
\pgfpathlineto{\pgfqpoint{4.053482in}{2.628579in}}%
\pgfpathlineto{\pgfqpoint{4.045776in}{2.619910in}}%
\pgfpathlineto{\pgfqpoint{4.038065in}{2.611251in}}%
\pgfpathlineto{\pgfqpoint{4.030348in}{2.602603in}}%
\pgfpathlineto{\pgfqpoint{4.022626in}{2.593966in}}%
\pgfpathlineto{\pgfqpoint{4.009294in}{2.597782in}}%
\pgfpathlineto{\pgfqpoint{3.995967in}{2.601719in}}%
\pgfpathlineto{\pgfqpoint{3.982646in}{2.605778in}}%
\pgfpathlineto{\pgfqpoint{3.969331in}{2.609959in}}%
\pgfpathlineto{\pgfqpoint{3.977064in}{2.618576in}}%
\pgfpathlineto{\pgfqpoint{3.984792in}{2.627208in}}%
\pgfpathlineto{\pgfqpoint{3.992515in}{2.635855in}}%
\pgfpathlineto{\pgfqpoint{4.000233in}{2.644517in}}%
\pgfpathclose%
\pgfusepath{fill}%
\end{pgfscope}%
\begin{pgfscope}%
\pgfpathrectangle{\pgfqpoint{1.150000in}{0.150000in}}{\pgfqpoint{5.700000in}{5.700000in}}%
\pgfusepath{clip}%
\pgfsetbuttcap%
\pgfsetroundjoin%
\definecolor{currentfill}{rgb}{0.280894,0.078907,0.402329}%
\pgfsetfillcolor{currentfill}%
\pgfsetfillopacity{0.700000}%
\pgfsetlinewidth{0.000000pt}%
\definecolor{currentstroke}{rgb}{0.000000,0.000000,0.000000}%
\pgfsetstrokecolor{currentstroke}%
\pgfsetdash{}{0pt}%
\pgfpathmoveto{\pgfqpoint{4.221550in}{2.672015in}}%
\pgfpathlineto{\pgfqpoint{4.234904in}{2.669250in}}%
\pgfpathlineto{\pgfqpoint{4.248266in}{2.666601in}}%
\pgfpathlineto{\pgfqpoint{4.261634in}{2.664067in}}%
\pgfpathlineto{\pgfqpoint{4.275009in}{2.661648in}}%
\pgfpathlineto{\pgfqpoint{4.267378in}{2.652966in}}%
\pgfpathlineto{\pgfqpoint{4.259741in}{2.644286in}}%
\pgfpathlineto{\pgfqpoint{4.252100in}{2.635605in}}%
\pgfpathlineto{\pgfqpoint{4.244453in}{2.626924in}}%
\pgfpathlineto{\pgfqpoint{4.231066in}{2.629321in}}%
\pgfpathlineto{\pgfqpoint{4.217687in}{2.631832in}}%
\pgfpathlineto{\pgfqpoint{4.204315in}{2.634459in}}%
\pgfpathlineto{\pgfqpoint{4.190949in}{2.637202in}}%
\pgfpathlineto{\pgfqpoint{4.198607in}{2.645899in}}%
\pgfpathlineto{\pgfqpoint{4.206260in}{2.654600in}}%
\pgfpathlineto{\pgfqpoint{4.213907in}{2.663305in}}%
\pgfpathlineto{\pgfqpoint{4.221550in}{2.672015in}}%
\pgfpathclose%
\pgfusepath{fill}%
\end{pgfscope}%
\begin{pgfscope}%
\pgfpathrectangle{\pgfqpoint{1.150000in}{0.150000in}}{\pgfqpoint{5.700000in}{5.700000in}}%
\pgfusepath{clip}%
\pgfsetbuttcap%
\pgfsetroundjoin%
\definecolor{currentfill}{rgb}{0.282623,0.140926,0.457517}%
\pgfsetfillcolor{currentfill}%
\pgfsetfillopacity{0.700000}%
\pgfsetlinewidth{0.000000pt}%
\definecolor{currentstroke}{rgb}{0.000000,0.000000,0.000000}%
\pgfsetstrokecolor{currentstroke}%
\pgfsetdash{}{0pt}%
\pgfpathmoveto{\pgfqpoint{3.260302in}{2.800590in}}%
\pgfpathlineto{\pgfqpoint{3.273549in}{2.790018in}}%
\pgfpathlineto{\pgfqpoint{3.286797in}{2.779607in}}%
\pgfpathlineto{\pgfqpoint{3.300045in}{2.769357in}}%
\pgfpathlineto{\pgfqpoint{3.313293in}{2.759266in}}%
\pgfpathlineto{\pgfqpoint{3.305319in}{2.751899in}}%
\pgfpathlineto{\pgfqpoint{3.297338in}{2.744597in}}%
\pgfpathlineto{\pgfqpoint{3.289350in}{2.737360in}}%
\pgfpathlineto{\pgfqpoint{3.281355in}{2.730190in}}%
\pgfpathlineto{\pgfqpoint{3.268090in}{2.740388in}}%
\pgfpathlineto{\pgfqpoint{3.254824in}{2.750745in}}%
\pgfpathlineto{\pgfqpoint{3.241559in}{2.761263in}}%
\pgfpathlineto{\pgfqpoint{3.228293in}{2.771943in}}%
\pgfpathlineto{\pgfqpoint{3.236306in}{2.778999in}}%
\pgfpathlineto{\pgfqpoint{3.244312in}{2.786127in}}%
\pgfpathlineto{\pgfqpoint{3.252310in}{2.793324in}}%
\pgfpathlineto{\pgfqpoint{3.260302in}{2.800590in}}%
\pgfpathclose%
\pgfusepath{fill}%
\end{pgfscope}%
\begin{pgfscope}%
\pgfpathrectangle{\pgfqpoint{1.150000in}{0.150000in}}{\pgfqpoint{5.700000in}{5.700000in}}%
\pgfusepath{clip}%
\pgfsetbuttcap%
\pgfsetroundjoin%
\definecolor{currentfill}{rgb}{0.274128,0.199721,0.498911}%
\pgfsetfillcolor{currentfill}%
\pgfsetfillopacity{0.700000}%
\pgfsetlinewidth{0.000000pt}%
\definecolor{currentstroke}{rgb}{0.000000,0.000000,0.000000}%
\pgfsetstrokecolor{currentstroke}%
\pgfsetdash{}{0pt}%
\pgfpathmoveto{\pgfqpoint{5.000150in}{2.902094in}}%
\pgfpathlineto{\pgfqpoint{5.013750in}{2.902830in}}%
\pgfpathlineto{\pgfqpoint{5.027359in}{2.903669in}}%
\pgfpathlineto{\pgfqpoint{5.040978in}{2.904609in}}%
\pgfpathlineto{\pgfqpoint{5.054608in}{2.905651in}}%
\pgfpathlineto{\pgfqpoint{5.047259in}{2.898085in}}%
\pgfpathlineto{\pgfqpoint{5.039905in}{2.890519in}}%
\pgfpathlineto{\pgfqpoint{5.032546in}{2.882950in}}%
\pgfpathlineto{\pgfqpoint{5.025182in}{2.875377in}}%
\pgfpathlineto{\pgfqpoint{5.011538in}{2.874167in}}%
\pgfpathlineto{\pgfqpoint{4.997906in}{2.873059in}}%
\pgfpathlineto{\pgfqpoint{4.984283in}{2.872053in}}%
\pgfpathlineto{\pgfqpoint{4.970670in}{2.871149in}}%
\pgfpathlineto{\pgfqpoint{4.978048in}{2.878883in}}%
\pgfpathlineto{\pgfqpoint{4.985421in}{2.886617in}}%
\pgfpathlineto{\pgfqpoint{4.992788in}{2.894353in}}%
\pgfpathlineto{\pgfqpoint{5.000150in}{2.902094in}}%
\pgfpathclose%
\pgfusepath{fill}%
\end{pgfscope}%
\begin{pgfscope}%
\pgfpathrectangle{\pgfqpoint{1.150000in}{0.150000in}}{\pgfqpoint{5.700000in}{5.700000in}}%
\pgfusepath{clip}%
\pgfsetbuttcap%
\pgfsetroundjoin%
\definecolor{currentfill}{rgb}{0.283229,0.120777,0.440584}%
\pgfsetfillcolor{currentfill}%
\pgfsetfillopacity{0.700000}%
\pgfsetlinewidth{0.000000pt}%
\definecolor{currentstroke}{rgb}{0.000000,0.000000,0.000000}%
\pgfsetstrokecolor{currentstroke}%
\pgfsetdash{}{0pt}%
\pgfpathmoveto{\pgfqpoint{4.526873in}{2.743826in}}%
\pgfpathlineto{\pgfqpoint{4.540315in}{2.742685in}}%
\pgfpathlineto{\pgfqpoint{4.553764in}{2.741653in}}%
\pgfpathlineto{\pgfqpoint{4.567221in}{2.740730in}}%
\pgfpathlineto{\pgfqpoint{4.580687in}{2.739915in}}%
\pgfpathlineto{\pgfqpoint{4.573161in}{2.731493in}}%
\pgfpathlineto{\pgfqpoint{4.565630in}{2.723066in}}%
\pgfpathlineto{\pgfqpoint{4.558094in}{2.714631in}}%
\pgfpathlineto{\pgfqpoint{4.550552in}{2.706187in}}%
\pgfpathlineto{\pgfqpoint{4.537075in}{2.706925in}}%
\pgfpathlineto{\pgfqpoint{4.523606in}{2.707772in}}%
\pgfpathlineto{\pgfqpoint{4.510146in}{2.708727in}}%
\pgfpathlineto{\pgfqpoint{4.496694in}{2.709792in}}%
\pgfpathlineto{\pgfqpoint{4.504246in}{2.718305in}}%
\pgfpathlineto{\pgfqpoint{4.511794in}{2.726815in}}%
\pgfpathlineto{\pgfqpoint{4.519336in}{2.735321in}}%
\pgfpathlineto{\pgfqpoint{4.526873in}{2.743826in}}%
\pgfpathclose%
\pgfusepath{fill}%
\end{pgfscope}%
\begin{pgfscope}%
\pgfpathrectangle{\pgfqpoint{1.150000in}{0.150000in}}{\pgfqpoint{5.700000in}{5.700000in}}%
\pgfusepath{clip}%
\pgfsetbuttcap%
\pgfsetroundjoin%
\definecolor{currentfill}{rgb}{0.277134,0.185228,0.489898}%
\pgfsetfillcolor{currentfill}%
\pgfsetfillopacity{0.700000}%
\pgfsetlinewidth{0.000000pt}%
\definecolor{currentstroke}{rgb}{0.000000,0.000000,0.000000}%
\pgfsetstrokecolor{currentstroke}%
\pgfsetdash{}{0pt}%
\pgfpathmoveto{\pgfqpoint{4.916320in}{2.868560in}}%
\pgfpathlineto{\pgfqpoint{4.929893in}{2.869053in}}%
\pgfpathlineto{\pgfqpoint{4.943475in}{2.869649in}}%
\pgfpathlineto{\pgfqpoint{4.957068in}{2.870347in}}%
\pgfpathlineto{\pgfqpoint{4.970670in}{2.871149in}}%
\pgfpathlineto{\pgfqpoint{4.963287in}{2.863413in}}%
\pgfpathlineto{\pgfqpoint{4.955899in}{2.855673in}}%
\pgfpathlineto{\pgfqpoint{4.948506in}{2.847927in}}%
\pgfpathlineto{\pgfqpoint{4.941107in}{2.840174in}}%
\pgfpathlineto{\pgfqpoint{4.927492in}{2.839223in}}%
\pgfpathlineto{\pgfqpoint{4.913886in}{2.838374in}}%
\pgfpathlineto{\pgfqpoint{4.900291in}{2.837629in}}%
\pgfpathlineto{\pgfqpoint{4.886705in}{2.836987in}}%
\pgfpathlineto{\pgfqpoint{4.894117in}{2.844883in}}%
\pgfpathlineto{\pgfqpoint{4.901523in}{2.852776in}}%
\pgfpathlineto{\pgfqpoint{4.908924in}{2.860668in}}%
\pgfpathlineto{\pgfqpoint{4.916320in}{2.868560in}}%
\pgfpathclose%
\pgfusepath{fill}%
\end{pgfscope}%
\begin{pgfscope}%
\pgfpathrectangle{\pgfqpoint{1.150000in}{0.150000in}}{\pgfqpoint{5.700000in}{5.700000in}}%
\pgfusepath{clip}%
\pgfsetbuttcap%
\pgfsetroundjoin%
\definecolor{currentfill}{rgb}{0.260571,0.246922,0.522828}%
\pgfsetfillcolor{currentfill}%
\pgfsetfillopacity{0.700000}%
\pgfsetlinewidth{0.000000pt}%
\definecolor{currentstroke}{rgb}{0.000000,0.000000,0.000000}%
\pgfsetstrokecolor{currentstroke}%
\pgfsetdash{}{0pt}%
\pgfpathmoveto{\pgfqpoint{2.962711in}{3.021690in}}%
\pgfpathlineto{\pgfqpoint{2.976012in}{3.007470in}}%
\pgfpathlineto{\pgfqpoint{2.989310in}{2.993442in}}%
\pgfpathlineto{\pgfqpoint{3.002605in}{2.979605in}}%
\pgfpathlineto{\pgfqpoint{3.015897in}{2.965956in}}%
\pgfpathlineto{\pgfqpoint{3.007798in}{2.959452in}}%
\pgfpathlineto{\pgfqpoint{2.999691in}{2.953039in}}%
\pgfpathlineto{\pgfqpoint{2.991576in}{2.946717in}}%
\pgfpathlineto{\pgfqpoint{2.983452in}{2.940488in}}%
\pgfpathlineto{\pgfqpoint{2.970139in}{2.954266in}}%
\pgfpathlineto{\pgfqpoint{2.956823in}{2.968232in}}%
\pgfpathlineto{\pgfqpoint{2.943503in}{2.982390in}}%
\pgfpathlineto{\pgfqpoint{2.930179in}{2.996740in}}%
\pgfpathlineto{\pgfqpoint{2.938325in}{3.002833in}}%
\pgfpathlineto{\pgfqpoint{2.946462in}{3.009023in}}%
\pgfpathlineto{\pgfqpoint{2.954590in}{3.015309in}}%
\pgfpathlineto{\pgfqpoint{2.962711in}{3.021690in}}%
\pgfpathclose%
\pgfusepath{fill}%
\end{pgfscope}%
\begin{pgfscope}%
\pgfpathrectangle{\pgfqpoint{1.150000in}{0.150000in}}{\pgfqpoint{5.700000in}{5.700000in}}%
\pgfusepath{clip}%
\pgfsetbuttcap%
\pgfsetroundjoin%
\definecolor{currentfill}{rgb}{0.279566,0.067836,0.391917}%
\pgfsetfillcolor{currentfill}%
\pgfsetfillopacity{0.700000}%
\pgfsetlinewidth{0.000000pt}%
\definecolor{currentstroke}{rgb}{0.000000,0.000000,0.000000}%
\pgfsetstrokecolor{currentstroke}%
\pgfsetdash{}{0pt}%
\pgfpathmoveto{\pgfqpoint{3.641445in}{2.653458in}}%
\pgfpathlineto{\pgfqpoint{3.654700in}{2.646612in}}%
\pgfpathlineto{\pgfqpoint{3.667957in}{2.639901in}}%
\pgfpathlineto{\pgfqpoint{3.681219in}{2.633326in}}%
\pgfpathlineto{\pgfqpoint{3.694483in}{2.626886in}}%
\pgfpathlineto{\pgfqpoint{3.686650in}{2.618647in}}%
\pgfpathlineto{\pgfqpoint{3.678812in}{2.610441in}}%
\pgfpathlineto{\pgfqpoint{3.670967in}{2.602270in}}%
\pgfpathlineto{\pgfqpoint{3.663117in}{2.594132in}}%
\pgfpathlineto{\pgfqpoint{3.649839in}{2.600641in}}%
\pgfpathlineto{\pgfqpoint{3.636564in}{2.607284in}}%
\pgfpathlineto{\pgfqpoint{3.623292in}{2.614063in}}%
\pgfpathlineto{\pgfqpoint{3.610023in}{2.620978in}}%
\pgfpathlineto{\pgfqpoint{3.617888in}{2.629041in}}%
\pgfpathlineto{\pgfqpoint{3.625746in}{2.637142in}}%
\pgfpathlineto{\pgfqpoint{3.633598in}{2.645281in}}%
\pgfpathlineto{\pgfqpoint{3.641445in}{2.653458in}}%
\pgfpathclose%
\pgfusepath{fill}%
\end{pgfscope}%
\begin{pgfscope}%
\pgfpathrectangle{\pgfqpoint{1.150000in}{0.150000in}}{\pgfqpoint{5.700000in}{5.700000in}}%
\pgfusepath{clip}%
\pgfsetbuttcap%
\pgfsetroundjoin%
\definecolor{currentfill}{rgb}{0.277941,0.056324,0.381191}%
\pgfsetfillcolor{currentfill}%
\pgfsetfillopacity{0.700000}%
\pgfsetlinewidth{0.000000pt}%
\definecolor{currentstroke}{rgb}{0.000000,0.000000,0.000000}%
\pgfsetstrokecolor{currentstroke}%
\pgfsetdash{}{0pt}%
\pgfpathmoveto{\pgfqpoint{3.778800in}{2.635945in}}%
\pgfpathlineto{\pgfqpoint{3.792071in}{2.630217in}}%
\pgfpathlineto{\pgfqpoint{3.805345in}{2.624619in}}%
\pgfpathlineto{\pgfqpoint{3.818624in}{2.619150in}}%
\pgfpathlineto{\pgfqpoint{3.831907in}{2.613810in}}%
\pgfpathlineto{\pgfqpoint{3.824123in}{2.605352in}}%
\pgfpathlineto{\pgfqpoint{3.816333in}{2.596918in}}%
\pgfpathlineto{\pgfqpoint{3.808537in}{2.588508in}}%
\pgfpathlineto{\pgfqpoint{3.800736in}{2.580121in}}%
\pgfpathlineto{\pgfqpoint{3.787440in}{2.585511in}}%
\pgfpathlineto{\pgfqpoint{3.774149in}{2.591030in}}%
\pgfpathlineto{\pgfqpoint{3.760861in}{2.596679in}}%
\pgfpathlineto{\pgfqpoint{3.747578in}{2.602457in}}%
\pgfpathlineto{\pgfqpoint{3.755392in}{2.610787in}}%
\pgfpathlineto{\pgfqpoint{3.763200in}{2.619145in}}%
\pgfpathlineto{\pgfqpoint{3.771003in}{2.627531in}}%
\pgfpathlineto{\pgfqpoint{3.778800in}{2.635945in}}%
\pgfpathclose%
\pgfusepath{fill}%
\end{pgfscope}%
\begin{pgfscope}%
\pgfpathrectangle{\pgfqpoint{1.150000in}{0.150000in}}{\pgfqpoint{5.700000in}{5.700000in}}%
\pgfusepath{clip}%
\pgfsetbuttcap%
\pgfsetroundjoin%
\definecolor{currentfill}{rgb}{0.282910,0.105393,0.426902}%
\pgfsetfillcolor{currentfill}%
\pgfsetfillopacity{0.700000}%
\pgfsetlinewidth{0.000000pt}%
\definecolor{currentstroke}{rgb}{0.000000,0.000000,0.000000}%
\pgfsetstrokecolor{currentstroke}%
\pgfsetdash{}{0pt}%
\pgfpathmoveto{\pgfqpoint{4.442966in}{2.715148in}}%
\pgfpathlineto{\pgfqpoint{4.456386in}{2.713644in}}%
\pgfpathlineto{\pgfqpoint{4.469814in}{2.712250in}}%
\pgfpathlineto{\pgfqpoint{4.483250in}{2.710966in}}%
\pgfpathlineto{\pgfqpoint{4.496694in}{2.709792in}}%
\pgfpathlineto{\pgfqpoint{4.489136in}{2.701273in}}%
\pgfpathlineto{\pgfqpoint{4.481573in}{2.692747in}}%
\pgfpathlineto{\pgfqpoint{4.474005in}{2.684215in}}%
\pgfpathlineto{\pgfqpoint{4.466432in}{2.675674in}}%
\pgfpathlineto{\pgfqpoint{4.452977in}{2.676790in}}%
\pgfpathlineto{\pgfqpoint{4.439530in}{2.678015in}}%
\pgfpathlineto{\pgfqpoint{4.426091in}{2.679350in}}%
\pgfpathlineto{\pgfqpoint{4.412660in}{2.680797in}}%
\pgfpathlineto{\pgfqpoint{4.420244in}{2.689389in}}%
\pgfpathlineto{\pgfqpoint{4.427823in}{2.697978in}}%
\pgfpathlineto{\pgfqpoint{4.435397in}{2.706564in}}%
\pgfpathlineto{\pgfqpoint{4.442966in}{2.715148in}}%
\pgfpathclose%
\pgfusepath{fill}%
\end{pgfscope}%
\begin{pgfscope}%
\pgfpathrectangle{\pgfqpoint{1.150000in}{0.150000in}}{\pgfqpoint{5.700000in}{5.700000in}}%
\pgfusepath{clip}%
\pgfsetbuttcap%
\pgfsetroundjoin%
\definecolor{currentfill}{rgb}{0.279566,0.067836,0.391917}%
\pgfsetfillcolor{currentfill}%
\pgfsetfillopacity{0.700000}%
\pgfsetlinewidth{0.000000pt}%
\definecolor{currentstroke}{rgb}{0.000000,0.000000,0.000000}%
\pgfsetstrokecolor{currentstroke}%
\pgfsetdash{}{0pt}%
\pgfpathmoveto{\pgfqpoint{4.137552in}{2.649340in}}%
\pgfpathlineto{\pgfqpoint{4.150891in}{2.646129in}}%
\pgfpathlineto{\pgfqpoint{4.164238in}{2.643037in}}%
\pgfpathlineto{\pgfqpoint{4.177590in}{2.640061in}}%
\pgfpathlineto{\pgfqpoint{4.190949in}{2.637202in}}%
\pgfpathlineto{\pgfqpoint{4.183286in}{2.628509in}}%
\pgfpathlineto{\pgfqpoint{4.175618in}{2.619819in}}%
\pgfpathlineto{\pgfqpoint{4.167945in}{2.611131in}}%
\pgfpathlineto{\pgfqpoint{4.160266in}{2.602446in}}%
\pgfpathlineto{\pgfqpoint{4.146896in}{2.605301in}}%
\pgfpathlineto{\pgfqpoint{4.133532in}{2.608272in}}%
\pgfpathlineto{\pgfqpoint{4.120175in}{2.611361in}}%
\pgfpathlineto{\pgfqpoint{4.106824in}{2.614567in}}%
\pgfpathlineto{\pgfqpoint{4.114514in}{2.623249in}}%
\pgfpathlineto{\pgfqpoint{4.122198in}{2.631939in}}%
\pgfpathlineto{\pgfqpoint{4.129878in}{2.640636in}}%
\pgfpathlineto{\pgfqpoint{4.137552in}{2.649340in}}%
\pgfpathclose%
\pgfusepath{fill}%
\end{pgfscope}%
\begin{pgfscope}%
\pgfpathrectangle{\pgfqpoint{1.150000in}{0.150000in}}{\pgfqpoint{5.700000in}{5.700000in}}%
\pgfusepath{clip}%
\pgfsetbuttcap%
\pgfsetroundjoin%
\definecolor{currentfill}{rgb}{0.283187,0.125848,0.444960}%
\pgfsetfillcolor{currentfill}%
\pgfsetfillopacity{0.700000}%
\pgfsetlinewidth{0.000000pt}%
\definecolor{currentstroke}{rgb}{0.000000,0.000000,0.000000}%
\pgfsetstrokecolor{currentstroke}%
\pgfsetdash{}{0pt}%
\pgfpathmoveto{\pgfqpoint{3.313293in}{2.759266in}}%
\pgfpathlineto{\pgfqpoint{3.326541in}{2.749333in}}%
\pgfpathlineto{\pgfqpoint{3.339790in}{2.739557in}}%
\pgfpathlineto{\pgfqpoint{3.353040in}{2.729936in}}%
\pgfpathlineto{\pgfqpoint{3.366290in}{2.720469in}}%
\pgfpathlineto{\pgfqpoint{3.358333in}{2.713003in}}%
\pgfpathlineto{\pgfqpoint{3.350369in}{2.705596in}}%
\pgfpathlineto{\pgfqpoint{3.342399in}{2.698251in}}%
\pgfpathlineto{\pgfqpoint{3.334421in}{2.690967in}}%
\pgfpathlineto{\pgfqpoint{3.321154in}{2.700539in}}%
\pgfpathlineto{\pgfqpoint{3.307887in}{2.710267in}}%
\pgfpathlineto{\pgfqpoint{3.294621in}{2.720150in}}%
\pgfpathlineto{\pgfqpoint{3.281355in}{2.730190in}}%
\pgfpathlineto{\pgfqpoint{3.289350in}{2.737360in}}%
\pgfpathlineto{\pgfqpoint{3.297338in}{2.744597in}}%
\pgfpathlineto{\pgfqpoint{3.305319in}{2.751899in}}%
\pgfpathlineto{\pgfqpoint{3.313293in}{2.759266in}}%
\pgfpathclose%
\pgfusepath{fill}%
\end{pgfscope}%
\begin{pgfscope}%
\pgfpathrectangle{\pgfqpoint{1.150000in}{0.150000in}}{\pgfqpoint{5.700000in}{5.700000in}}%
\pgfusepath{clip}%
\pgfsetbuttcap%
\pgfsetroundjoin%
\definecolor{currentfill}{rgb}{0.267968,0.223549,0.512008}%
\pgfsetfillcolor{currentfill}%
\pgfsetfillopacity{0.700000}%
\pgfsetlinewidth{0.000000pt}%
\definecolor{currentstroke}{rgb}{0.000000,0.000000,0.000000}%
\pgfsetstrokecolor{currentstroke}%
\pgfsetdash{}{0pt}%
\pgfpathmoveto{\pgfqpoint{3.015897in}{2.965956in}}%
\pgfpathlineto{\pgfqpoint{3.029186in}{2.952495in}}%
\pgfpathlineto{\pgfqpoint{3.042472in}{2.939219in}}%
\pgfpathlineto{\pgfqpoint{3.055756in}{2.926127in}}%
\pgfpathlineto{\pgfqpoint{3.069038in}{2.913217in}}%
\pgfpathlineto{\pgfqpoint{3.060959in}{2.906592in}}%
\pgfpathlineto{\pgfqpoint{3.052873in}{2.900052in}}%
\pgfpathlineto{\pgfqpoint{3.044779in}{2.893600in}}%
\pgfpathlineto{\pgfqpoint{3.036677in}{2.887235in}}%
\pgfpathlineto{\pgfqpoint{3.023375in}{2.900273in}}%
\pgfpathlineto{\pgfqpoint{3.010070in}{2.913494in}}%
\pgfpathlineto{\pgfqpoint{2.996763in}{2.926898in}}%
\pgfpathlineto{\pgfqpoint{2.983452in}{2.940488in}}%
\pgfpathlineto{\pgfqpoint{2.991576in}{2.946717in}}%
\pgfpathlineto{\pgfqpoint{2.999691in}{2.953039in}}%
\pgfpathlineto{\pgfqpoint{3.007798in}{2.959452in}}%
\pgfpathlineto{\pgfqpoint{3.015897in}{2.965956in}}%
\pgfpathclose%
\pgfusepath{fill}%
\end{pgfscope}%
\begin{pgfscope}%
\pgfpathrectangle{\pgfqpoint{1.150000in}{0.150000in}}{\pgfqpoint{5.700000in}{5.700000in}}%
\pgfusepath{clip}%
\pgfsetbuttcap%
\pgfsetroundjoin%
\definecolor{currentfill}{rgb}{0.279574,0.170599,0.479997}%
\pgfsetfillcolor{currentfill}%
\pgfsetfillopacity{0.700000}%
\pgfsetlinewidth{0.000000pt}%
\definecolor{currentstroke}{rgb}{0.000000,0.000000,0.000000}%
\pgfsetstrokecolor{currentstroke}%
\pgfsetdash{}{0pt}%
\pgfpathmoveto{\pgfqpoint{4.832459in}{2.835455in}}%
\pgfpathlineto{\pgfqpoint{4.846006in}{2.835682in}}%
\pgfpathlineto{\pgfqpoint{4.859563in}{2.836013in}}%
\pgfpathlineto{\pgfqpoint{4.873129in}{2.836448in}}%
\pgfpathlineto{\pgfqpoint{4.886705in}{2.836987in}}%
\pgfpathlineto{\pgfqpoint{4.879288in}{2.829086in}}%
\pgfpathlineto{\pgfqpoint{4.871867in}{2.821178in}}%
\pgfpathlineto{\pgfqpoint{4.864439in}{2.813262in}}%
\pgfpathlineto{\pgfqpoint{4.857007in}{2.805335in}}%
\pgfpathlineto{\pgfqpoint{4.843419in}{2.804665in}}%
\pgfpathlineto{\pgfqpoint{4.829840in}{2.804098in}}%
\pgfpathlineto{\pgfqpoint{4.816271in}{2.803636in}}%
\pgfpathlineto{\pgfqpoint{4.802712in}{2.803278in}}%
\pgfpathlineto{\pgfqpoint{4.810156in}{2.811329in}}%
\pgfpathlineto{\pgfqpoint{4.817596in}{2.819375in}}%
\pgfpathlineto{\pgfqpoint{4.825030in}{2.827416in}}%
\pgfpathlineto{\pgfqpoint{4.832459in}{2.835455in}}%
\pgfpathclose%
\pgfusepath{fill}%
\end{pgfscope}%
\begin{pgfscope}%
\pgfpathrectangle{\pgfqpoint{1.150000in}{0.150000in}}{\pgfqpoint{5.700000in}{5.700000in}}%
\pgfusepath{clip}%
\pgfsetbuttcap%
\pgfsetroundjoin%
\definecolor{currentfill}{rgb}{0.281446,0.084320,0.407414}%
\pgfsetfillcolor{currentfill}%
\pgfsetfillopacity{0.700000}%
\pgfsetlinewidth{0.000000pt}%
\definecolor{currentstroke}{rgb}{0.000000,0.000000,0.000000}%
\pgfsetstrokecolor{currentstroke}%
\pgfsetdash{}{0pt}%
\pgfpathmoveto{\pgfqpoint{3.503973in}{2.681319in}}%
\pgfpathlineto{\pgfqpoint{3.517221in}{2.673280in}}%
\pgfpathlineto{\pgfqpoint{3.530471in}{2.665385in}}%
\pgfpathlineto{\pgfqpoint{3.543723in}{2.657632in}}%
\pgfpathlineto{\pgfqpoint{3.556978in}{2.650022in}}%
\pgfpathlineto{\pgfqpoint{3.549093in}{2.642079in}}%
\pgfpathlineto{\pgfqpoint{3.541202in}{2.634180in}}%
\pgfpathlineto{\pgfqpoint{3.533306in}{2.626326in}}%
\pgfpathlineto{\pgfqpoint{3.525402in}{2.618517in}}%
\pgfpathlineto{\pgfqpoint{3.512133in}{2.626214in}}%
\pgfpathlineto{\pgfqpoint{3.498865in}{2.634054in}}%
\pgfpathlineto{\pgfqpoint{3.485600in}{2.642036in}}%
\pgfpathlineto{\pgfqpoint{3.472337in}{2.650162in}}%
\pgfpathlineto{\pgfqpoint{3.480255in}{2.657878in}}%
\pgfpathlineto{\pgfqpoint{3.488167in}{2.665642in}}%
\pgfpathlineto{\pgfqpoint{3.496073in}{2.673456in}}%
\pgfpathlineto{\pgfqpoint{3.503973in}{2.681319in}}%
\pgfpathclose%
\pgfusepath{fill}%
\end{pgfscope}%
\begin{pgfscope}%
\pgfpathrectangle{\pgfqpoint{1.150000in}{0.150000in}}{\pgfqpoint{5.700000in}{5.700000in}}%
\pgfusepath{clip}%
\pgfsetbuttcap%
\pgfsetroundjoin%
\definecolor{currentfill}{rgb}{0.277941,0.056324,0.381191}%
\pgfsetfillcolor{currentfill}%
\pgfsetfillopacity{0.700000}%
\pgfsetlinewidth{0.000000pt}%
\definecolor{currentstroke}{rgb}{0.000000,0.000000,0.000000}%
\pgfsetstrokecolor{currentstroke}%
\pgfsetdash{}{0pt}%
\pgfpathmoveto{\pgfqpoint{3.916121in}{2.627916in}}%
\pgfpathlineto{\pgfqpoint{3.929416in}{2.623241in}}%
\pgfpathlineto{\pgfqpoint{3.942716in}{2.618690in}}%
\pgfpathlineto{\pgfqpoint{3.956020in}{2.614263in}}%
\pgfpathlineto{\pgfqpoint{3.969331in}{2.609959in}}%
\pgfpathlineto{\pgfqpoint{3.961592in}{2.601357in}}%
\pgfpathlineto{\pgfqpoint{3.953848in}{2.592769in}}%
\pgfpathlineto{\pgfqpoint{3.946098in}{2.584196in}}%
\pgfpathlineto{\pgfqpoint{3.938344in}{2.575637in}}%
\pgfpathlineto{\pgfqpoint{3.925022in}{2.579972in}}%
\pgfpathlineto{\pgfqpoint{3.911705in}{2.584431in}}%
\pgfpathlineto{\pgfqpoint{3.898393in}{2.589014in}}%
\pgfpathlineto{\pgfqpoint{3.885086in}{2.593721in}}%
\pgfpathlineto{\pgfqpoint{3.892853in}{2.602242in}}%
\pgfpathlineto{\pgfqpoint{3.900615in}{2.610781in}}%
\pgfpathlineto{\pgfqpoint{3.908370in}{2.619339in}}%
\pgfpathlineto{\pgfqpoint{3.916121in}{2.627916in}}%
\pgfpathclose%
\pgfusepath{fill}%
\end{pgfscope}%
\begin{pgfscope}%
\pgfpathrectangle{\pgfqpoint{1.150000in}{0.150000in}}{\pgfqpoint{5.700000in}{5.700000in}}%
\pgfusepath{clip}%
\pgfsetbuttcap%
\pgfsetroundjoin%
\definecolor{currentfill}{rgb}{0.274128,0.199721,0.498911}%
\pgfsetfillcolor{currentfill}%
\pgfsetfillopacity{0.700000}%
\pgfsetlinewidth{0.000000pt}%
\definecolor{currentstroke}{rgb}{0.000000,0.000000,0.000000}%
\pgfsetstrokecolor{currentstroke}%
\pgfsetdash{}{0pt}%
\pgfpathmoveto{\pgfqpoint{3.069038in}{2.913217in}}%
\pgfpathlineto{\pgfqpoint{3.082317in}{2.900488in}}%
\pgfpathlineto{\pgfqpoint{3.095594in}{2.887937in}}%
\pgfpathlineto{\pgfqpoint{3.108870in}{2.875564in}}%
\pgfpathlineto{\pgfqpoint{3.122144in}{2.863367in}}%
\pgfpathlineto{\pgfqpoint{3.114086in}{2.856621in}}%
\pgfpathlineto{\pgfqpoint{3.106020in}{2.849956in}}%
\pgfpathlineto{\pgfqpoint{3.097946in}{2.843373in}}%
\pgfpathlineto{\pgfqpoint{3.089865in}{2.836875in}}%
\pgfpathlineto{\pgfqpoint{3.076571in}{2.849199in}}%
\pgfpathlineto{\pgfqpoint{3.063275in}{2.861700in}}%
\pgfpathlineto{\pgfqpoint{3.049977in}{2.874378in}}%
\pgfpathlineto{\pgfqpoint{3.036677in}{2.887235in}}%
\pgfpathlineto{\pgfqpoint{3.044779in}{2.893600in}}%
\pgfpathlineto{\pgfqpoint{3.052873in}{2.900052in}}%
\pgfpathlineto{\pgfqpoint{3.060959in}{2.906592in}}%
\pgfpathlineto{\pgfqpoint{3.069038in}{2.913217in}}%
\pgfpathclose%
\pgfusepath{fill}%
\end{pgfscope}%
\begin{pgfscope}%
\pgfpathrectangle{\pgfqpoint{1.150000in}{0.150000in}}{\pgfqpoint{5.700000in}{5.700000in}}%
\pgfusepath{clip}%
\pgfsetbuttcap%
\pgfsetroundjoin%
\definecolor{currentfill}{rgb}{0.282327,0.094955,0.417331}%
\pgfsetfillcolor{currentfill}%
\pgfsetfillopacity{0.700000}%
\pgfsetlinewidth{0.000000pt}%
\definecolor{currentstroke}{rgb}{0.000000,0.000000,0.000000}%
\pgfsetstrokecolor{currentstroke}%
\pgfsetdash{}{0pt}%
\pgfpathmoveto{\pgfqpoint{4.359013in}{2.687695in}}%
\pgfpathlineto{\pgfqpoint{4.372413in}{2.685803in}}%
\pgfpathlineto{\pgfqpoint{4.385821in}{2.684022in}}%
\pgfpathlineto{\pgfqpoint{4.399237in}{2.682354in}}%
\pgfpathlineto{\pgfqpoint{4.412660in}{2.680797in}}%
\pgfpathlineto{\pgfqpoint{4.405071in}{2.672200in}}%
\pgfpathlineto{\pgfqpoint{4.397476in}{2.663597in}}%
\pgfpathlineto{\pgfqpoint{4.389877in}{2.654988in}}%
\pgfpathlineto{\pgfqpoint{4.382272in}{2.646373in}}%
\pgfpathlineto{\pgfqpoint{4.368838in}{2.647890in}}%
\pgfpathlineto{\pgfqpoint{4.355412in}{2.649518in}}%
\pgfpathlineto{\pgfqpoint{4.341993in}{2.651257in}}%
\pgfpathlineto{\pgfqpoint{4.328582in}{2.653109in}}%
\pgfpathlineto{\pgfqpoint{4.336197in}{2.661759in}}%
\pgfpathlineto{\pgfqpoint{4.343808in}{2.670406in}}%
\pgfpathlineto{\pgfqpoint{4.351413in}{2.679051in}}%
\pgfpathlineto{\pgfqpoint{4.359013in}{2.687695in}}%
\pgfpathclose%
\pgfusepath{fill}%
\end{pgfscope}%
\begin{pgfscope}%
\pgfpathrectangle{\pgfqpoint{1.150000in}{0.150000in}}{\pgfqpoint{5.700000in}{5.700000in}}%
\pgfusepath{clip}%
\pgfsetbuttcap%
\pgfsetroundjoin%
\definecolor{currentfill}{rgb}{0.281412,0.155834,0.469201}%
\pgfsetfillcolor{currentfill}%
\pgfsetfillopacity{0.700000}%
\pgfsetlinewidth{0.000000pt}%
\definecolor{currentstroke}{rgb}{0.000000,0.000000,0.000000}%
\pgfsetstrokecolor{currentstroke}%
\pgfsetdash{}{0pt}%
\pgfpathmoveto{\pgfqpoint{4.748568in}{2.802893in}}%
\pgfpathlineto{\pgfqpoint{4.762090in}{2.802831in}}%
\pgfpathlineto{\pgfqpoint{4.775621in}{2.802875in}}%
\pgfpathlineto{\pgfqpoint{4.789162in}{2.803024in}}%
\pgfpathlineto{\pgfqpoint{4.802712in}{2.803278in}}%
\pgfpathlineto{\pgfqpoint{4.795262in}{2.795219in}}%
\pgfpathlineto{\pgfqpoint{4.787807in}{2.787152in}}%
\pgfpathlineto{\pgfqpoint{4.780346in}{2.779074in}}%
\pgfpathlineto{\pgfqpoint{4.772880in}{2.770985in}}%
\pgfpathlineto{\pgfqpoint{4.759319in}{2.770618in}}%
\pgfpathlineto{\pgfqpoint{4.745766in}{2.770356in}}%
\pgfpathlineto{\pgfqpoint{4.732223in}{2.770199in}}%
\pgfpathlineto{\pgfqpoint{4.718689in}{2.770147in}}%
\pgfpathlineto{\pgfqpoint{4.726167in}{2.778343in}}%
\pgfpathlineto{\pgfqpoint{4.733639in}{2.786532in}}%
\pgfpathlineto{\pgfqpoint{4.741106in}{2.794715in}}%
\pgfpathlineto{\pgfqpoint{4.748568in}{2.802893in}}%
\pgfpathclose%
\pgfusepath{fill}%
\end{pgfscope}%
\begin{pgfscope}%
\pgfpathrectangle{\pgfqpoint{1.150000in}{0.150000in}}{\pgfqpoint{5.700000in}{5.700000in}}%
\pgfusepath{clip}%
\pgfsetbuttcap%
\pgfsetroundjoin%
\definecolor{currentfill}{rgb}{0.278012,0.180367,0.486697}%
\pgfsetfillcolor{currentfill}%
\pgfsetfillopacity{0.700000}%
\pgfsetlinewidth{0.000000pt}%
\definecolor{currentstroke}{rgb}{0.000000,0.000000,0.000000}%
\pgfsetstrokecolor{currentstroke}%
\pgfsetdash{}{0pt}%
\pgfpathmoveto{\pgfqpoint{3.122144in}{2.863367in}}%
\pgfpathlineto{\pgfqpoint{3.135416in}{2.851344in}}%
\pgfpathlineto{\pgfqpoint{3.148687in}{2.839494in}}%
\pgfpathlineto{\pgfqpoint{3.161957in}{2.827815in}}%
\pgfpathlineto{\pgfqpoint{3.175226in}{2.816306in}}%
\pgfpathlineto{\pgfqpoint{3.167187in}{2.809440in}}%
\pgfpathlineto{\pgfqpoint{3.159141in}{2.802651in}}%
\pgfpathlineto{\pgfqpoint{3.151087in}{2.795940in}}%
\pgfpathlineto{\pgfqpoint{3.143026in}{2.789307in}}%
\pgfpathlineto{\pgfqpoint{3.129738in}{2.800942in}}%
\pgfpathlineto{\pgfqpoint{3.116448in}{2.812748in}}%
\pgfpathlineto{\pgfqpoint{3.103157in}{2.824725in}}%
\pgfpathlineto{\pgfqpoint{3.089865in}{2.836875in}}%
\pgfpathlineto{\pgfqpoint{3.097946in}{2.843373in}}%
\pgfpathlineto{\pgfqpoint{3.106020in}{2.849956in}}%
\pgfpathlineto{\pgfqpoint{3.114086in}{2.856621in}}%
\pgfpathlineto{\pgfqpoint{3.122144in}{2.863367in}}%
\pgfpathclose%
\pgfusepath{fill}%
\end{pgfscope}%
\begin{pgfscope}%
\pgfpathrectangle{\pgfqpoint{1.150000in}{0.150000in}}{\pgfqpoint{5.700000in}{5.700000in}}%
\pgfusepath{clip}%
\pgfsetbuttcap%
\pgfsetroundjoin%
\definecolor{currentfill}{rgb}{0.278791,0.062145,0.386592}%
\pgfsetfillcolor{currentfill}%
\pgfsetfillopacity{0.700000}%
\pgfsetlinewidth{0.000000pt}%
\definecolor{currentstroke}{rgb}{0.000000,0.000000,0.000000}%
\pgfsetstrokecolor{currentstroke}%
\pgfsetdash{}{0pt}%
\pgfpathmoveto{\pgfqpoint{4.053482in}{2.628579in}}%
\pgfpathlineto{\pgfqpoint{4.066809in}{2.624897in}}%
\pgfpathlineto{\pgfqpoint{4.080141in}{2.621334in}}%
\pgfpathlineto{\pgfqpoint{4.093480in}{2.617891in}}%
\pgfpathlineto{\pgfqpoint{4.106824in}{2.614567in}}%
\pgfpathlineto{\pgfqpoint{4.099130in}{2.605891in}}%
\pgfpathlineto{\pgfqpoint{4.091430in}{2.597221in}}%
\pgfpathlineto{\pgfqpoint{4.083724in}{2.588557in}}%
\pgfpathlineto{\pgfqpoint{4.076014in}{2.579899in}}%
\pgfpathlineto{\pgfqpoint{4.062658in}{2.583237in}}%
\pgfpathlineto{\pgfqpoint{4.049308in}{2.586694in}}%
\pgfpathlineto{\pgfqpoint{4.035964in}{2.590270in}}%
\pgfpathlineto{\pgfqpoint{4.022626in}{2.593966in}}%
\pgfpathlineto{\pgfqpoint{4.030348in}{2.602603in}}%
\pgfpathlineto{\pgfqpoint{4.038065in}{2.611251in}}%
\pgfpathlineto{\pgfqpoint{4.045776in}{2.619910in}}%
\pgfpathlineto{\pgfqpoint{4.053482in}{2.628579in}}%
\pgfpathclose%
\pgfusepath{fill}%
\end{pgfscope}%
\begin{pgfscope}%
\pgfpathrectangle{\pgfqpoint{1.150000in}{0.150000in}}{\pgfqpoint{5.700000in}{5.700000in}}%
\pgfusepath{clip}%
\pgfsetbuttcap%
\pgfsetroundjoin%
\definecolor{currentfill}{rgb}{0.283091,0.110553,0.431554}%
\pgfsetfillcolor{currentfill}%
\pgfsetfillopacity{0.700000}%
\pgfsetlinewidth{0.000000pt}%
\definecolor{currentstroke}{rgb}{0.000000,0.000000,0.000000}%
\pgfsetstrokecolor{currentstroke}%
\pgfsetdash{}{0pt}%
\pgfpathmoveto{\pgfqpoint{3.366290in}{2.720469in}}%
\pgfpathlineto{\pgfqpoint{3.379541in}{2.711156in}}%
\pgfpathlineto{\pgfqpoint{3.392794in}{2.701995in}}%
\pgfpathlineto{\pgfqpoint{3.406047in}{2.692985in}}%
\pgfpathlineto{\pgfqpoint{3.419302in}{2.684125in}}%
\pgfpathlineto{\pgfqpoint{3.411361in}{2.676559in}}%
\pgfpathlineto{\pgfqpoint{3.403414in}{2.669049in}}%
\pgfpathlineto{\pgfqpoint{3.395460in}{2.661595in}}%
\pgfpathlineto{\pgfqpoint{3.387500in}{2.654198in}}%
\pgfpathlineto{\pgfqpoint{3.374229in}{2.663164in}}%
\pgfpathlineto{\pgfqpoint{3.360959in}{2.672280in}}%
\pgfpathlineto{\pgfqpoint{3.347690in}{2.681547in}}%
\pgfpathlineto{\pgfqpoint{3.334421in}{2.690967in}}%
\pgfpathlineto{\pgfqpoint{3.342399in}{2.698251in}}%
\pgfpathlineto{\pgfqpoint{3.350369in}{2.705596in}}%
\pgfpathlineto{\pgfqpoint{3.358333in}{2.713003in}}%
\pgfpathlineto{\pgfqpoint{3.366290in}{2.720469in}}%
\pgfpathclose%
\pgfusepath{fill}%
\end{pgfscope}%
\begin{pgfscope}%
\pgfpathrectangle{\pgfqpoint{1.150000in}{0.150000in}}{\pgfqpoint{5.700000in}{5.700000in}}%
\pgfusepath{clip}%
\pgfsetbuttcap%
\pgfsetroundjoin%
\definecolor{currentfill}{rgb}{0.282623,0.140926,0.457517}%
\pgfsetfillcolor{currentfill}%
\pgfsetfillopacity{0.700000}%
\pgfsetlinewidth{0.000000pt}%
\definecolor{currentstroke}{rgb}{0.000000,0.000000,0.000000}%
\pgfsetstrokecolor{currentstroke}%
\pgfsetdash{}{0pt}%
\pgfpathmoveto{\pgfqpoint{4.664644in}{2.770999in}}%
\pgfpathlineto{\pgfqpoint{4.678142in}{2.770627in}}%
\pgfpathlineto{\pgfqpoint{4.691649in}{2.770361in}}%
\pgfpathlineto{\pgfqpoint{4.705165in}{2.770201in}}%
\pgfpathlineto{\pgfqpoint{4.718689in}{2.770147in}}%
\pgfpathlineto{\pgfqpoint{4.711207in}{2.761942in}}%
\pgfpathlineto{\pgfqpoint{4.703719in}{2.753727in}}%
\pgfpathlineto{\pgfqpoint{4.696226in}{2.745500in}}%
\pgfpathlineto{\pgfqpoint{4.688727in}{2.737260in}}%
\pgfpathlineto{\pgfqpoint{4.675191in}{2.737219in}}%
\pgfpathlineto{\pgfqpoint{4.661664in}{2.737284in}}%
\pgfpathlineto{\pgfqpoint{4.648146in}{2.737455in}}%
\pgfpathlineto{\pgfqpoint{4.634637in}{2.737733in}}%
\pgfpathlineto{\pgfqpoint{4.642147in}{2.746061in}}%
\pgfpathlineto{\pgfqpoint{4.649651in}{2.754380in}}%
\pgfpathlineto{\pgfqpoint{4.657150in}{2.762693in}}%
\pgfpathlineto{\pgfqpoint{4.664644in}{2.770999in}}%
\pgfpathclose%
\pgfusepath{fill}%
\end{pgfscope}%
\begin{pgfscope}%
\pgfpathrectangle{\pgfqpoint{1.150000in}{0.150000in}}{\pgfqpoint{5.700000in}{5.700000in}}%
\pgfusepath{clip}%
\pgfsetbuttcap%
\pgfsetroundjoin%
\definecolor{currentfill}{rgb}{0.260571,0.246922,0.522828}%
\pgfsetfillcolor{currentfill}%
\pgfsetfillopacity{0.700000}%
\pgfsetlinewidth{0.000000pt}%
\definecolor{currentstroke}{rgb}{0.000000,0.000000,0.000000}%
\pgfsetstrokecolor{currentstroke}%
\pgfsetdash{}{0pt}%
\pgfpathmoveto{\pgfqpoint{5.222397in}{2.975292in}}%
\pgfpathlineto{\pgfqpoint{5.236092in}{2.976849in}}%
\pgfpathlineto{\pgfqpoint{5.249798in}{2.978507in}}%
\pgfpathlineto{\pgfqpoint{5.263515in}{2.980264in}}%
\pgfpathlineto{\pgfqpoint{5.277243in}{2.982120in}}%
\pgfpathlineto{\pgfqpoint{5.269980in}{2.975107in}}%
\pgfpathlineto{\pgfqpoint{5.262712in}{2.968100in}}%
\pgfpathlineto{\pgfqpoint{5.255438in}{2.961095in}}%
\pgfpathlineto{\pgfqpoint{5.248159in}{2.954090in}}%
\pgfpathlineto{\pgfqpoint{5.234415in}{2.952028in}}%
\pgfpathlineto{\pgfqpoint{5.220683in}{2.950066in}}%
\pgfpathlineto{\pgfqpoint{5.206961in}{2.948204in}}%
\pgfpathlineto{\pgfqpoint{5.193251in}{2.946442in}}%
\pgfpathlineto{\pgfqpoint{5.200545in}{2.953645in}}%
\pgfpathlineto{\pgfqpoint{5.207834in}{2.960853in}}%
\pgfpathlineto{\pgfqpoint{5.215118in}{2.968068in}}%
\pgfpathlineto{\pgfqpoint{5.222397in}{2.975292in}}%
\pgfpathclose%
\pgfusepath{fill}%
\end{pgfscope}%
\begin{pgfscope}%
\pgfpathrectangle{\pgfqpoint{1.150000in}{0.150000in}}{\pgfqpoint{5.700000in}{5.700000in}}%
\pgfusepath{clip}%
\pgfsetbuttcap%
\pgfsetroundjoin%
\definecolor{currentfill}{rgb}{0.278791,0.062145,0.386592}%
\pgfsetfillcolor{currentfill}%
\pgfsetfillopacity{0.700000}%
\pgfsetlinewidth{0.000000pt}%
\definecolor{currentstroke}{rgb}{0.000000,0.000000,0.000000}%
\pgfsetstrokecolor{currentstroke}%
\pgfsetdash{}{0pt}%
\pgfpathmoveto{\pgfqpoint{3.694483in}{2.626886in}}%
\pgfpathlineto{\pgfqpoint{3.707751in}{2.620580in}}%
\pgfpathlineto{\pgfqpoint{3.721023in}{2.614407in}}%
\pgfpathlineto{\pgfqpoint{3.734299in}{2.608366in}}%
\pgfpathlineto{\pgfqpoint{3.747578in}{2.602457in}}%
\pgfpathlineto{\pgfqpoint{3.739758in}{2.594156in}}%
\pgfpathlineto{\pgfqpoint{3.731933in}{2.585885in}}%
\pgfpathlineto{\pgfqpoint{3.724102in}{2.577642in}}%
\pgfpathlineto{\pgfqpoint{3.716265in}{2.569430in}}%
\pgfpathlineto{\pgfqpoint{3.702973in}{2.575407in}}%
\pgfpathlineto{\pgfqpoint{3.689684in}{2.581516in}}%
\pgfpathlineto{\pgfqpoint{3.676399in}{2.587757in}}%
\pgfpathlineto{\pgfqpoint{3.663117in}{2.594132in}}%
\pgfpathlineto{\pgfqpoint{3.670967in}{2.602270in}}%
\pgfpathlineto{\pgfqpoint{3.678812in}{2.610441in}}%
\pgfpathlineto{\pgfqpoint{3.686650in}{2.618647in}}%
\pgfpathlineto{\pgfqpoint{3.694483in}{2.626886in}}%
\pgfpathclose%
\pgfusepath{fill}%
\end{pgfscope}%
\begin{pgfscope}%
\pgfpathrectangle{\pgfqpoint{1.150000in}{0.150000in}}{\pgfqpoint{5.700000in}{5.700000in}}%
\pgfusepath{clip}%
\pgfsetbuttcap%
\pgfsetroundjoin%
\definecolor{currentfill}{rgb}{0.281446,0.084320,0.407414}%
\pgfsetfillcolor{currentfill}%
\pgfsetfillopacity{0.700000}%
\pgfsetlinewidth{0.000000pt}%
\definecolor{currentstroke}{rgb}{0.000000,0.000000,0.000000}%
\pgfsetstrokecolor{currentstroke}%
\pgfsetdash{}{0pt}%
\pgfpathmoveto{\pgfqpoint{4.275009in}{2.661648in}}%
\pgfpathlineto{\pgfqpoint{4.288392in}{2.659343in}}%
\pgfpathlineto{\pgfqpoint{4.301781in}{2.657152in}}%
\pgfpathlineto{\pgfqpoint{4.315178in}{2.655074in}}%
\pgfpathlineto{\pgfqpoint{4.328582in}{2.653109in}}%
\pgfpathlineto{\pgfqpoint{4.320961in}{2.644457in}}%
\pgfpathlineto{\pgfqpoint{4.313335in}{2.635801in}}%
\pgfpathlineto{\pgfqpoint{4.305704in}{2.627140in}}%
\pgfpathlineto{\pgfqpoint{4.298068in}{2.618475in}}%
\pgfpathlineto{\pgfqpoint{4.284653in}{2.620417in}}%
\pgfpathlineto{\pgfqpoint{4.271246in}{2.622472in}}%
\pgfpathlineto{\pgfqpoint{4.257846in}{2.624641in}}%
\pgfpathlineto{\pgfqpoint{4.244453in}{2.626924in}}%
\pgfpathlineto{\pgfqpoint{4.252100in}{2.635605in}}%
\pgfpathlineto{\pgfqpoint{4.259741in}{2.644286in}}%
\pgfpathlineto{\pgfqpoint{4.267378in}{2.652966in}}%
\pgfpathlineto{\pgfqpoint{4.275009in}{2.661648in}}%
\pgfpathclose%
\pgfusepath{fill}%
\end{pgfscope}%
\begin{pgfscope}%
\pgfpathrectangle{\pgfqpoint{1.150000in}{0.150000in}}{\pgfqpoint{5.700000in}{5.700000in}}%
\pgfusepath{clip}%
\pgfsetbuttcap%
\pgfsetroundjoin%
\definecolor{currentfill}{rgb}{0.280267,0.073417,0.397163}%
\pgfsetfillcolor{currentfill}%
\pgfsetfillopacity{0.700000}%
\pgfsetlinewidth{0.000000pt}%
\definecolor{currentstroke}{rgb}{0.000000,0.000000,0.000000}%
\pgfsetstrokecolor{currentstroke}%
\pgfsetdash{}{0pt}%
\pgfpathmoveto{\pgfqpoint{3.556978in}{2.650022in}}%
\pgfpathlineto{\pgfqpoint{3.570235in}{2.642552in}}%
\pgfpathlineto{\pgfqpoint{3.583495in}{2.635222in}}%
\pgfpathlineto{\pgfqpoint{3.596758in}{2.628031in}}%
\pgfpathlineto{\pgfqpoint{3.610023in}{2.620978in}}%
\pgfpathlineto{\pgfqpoint{3.602153in}{2.612955in}}%
\pgfpathlineto{\pgfqpoint{3.594277in}{2.604972in}}%
\pgfpathlineto{\pgfqpoint{3.586395in}{2.597029in}}%
\pgfpathlineto{\pgfqpoint{3.578507in}{2.589126in}}%
\pgfpathlineto{\pgfqpoint{3.565227in}{2.596266in}}%
\pgfpathlineto{\pgfqpoint{3.551949in}{2.603543in}}%
\pgfpathlineto{\pgfqpoint{3.538675in}{2.610960in}}%
\pgfpathlineto{\pgfqpoint{3.525402in}{2.618517in}}%
\pgfpathlineto{\pgfqpoint{3.533306in}{2.626326in}}%
\pgfpathlineto{\pgfqpoint{3.541202in}{2.634180in}}%
\pgfpathlineto{\pgfqpoint{3.549093in}{2.642079in}}%
\pgfpathlineto{\pgfqpoint{3.556978in}{2.650022in}}%
\pgfpathclose%
\pgfusepath{fill}%
\end{pgfscope}%
\begin{pgfscope}%
\pgfpathrectangle{\pgfqpoint{1.150000in}{0.150000in}}{\pgfqpoint{5.700000in}{5.700000in}}%
\pgfusepath{clip}%
\pgfsetbuttcap%
\pgfsetroundjoin%
\definecolor{currentfill}{rgb}{0.277941,0.056324,0.381191}%
\pgfsetfillcolor{currentfill}%
\pgfsetfillopacity{0.700000}%
\pgfsetlinewidth{0.000000pt}%
\definecolor{currentstroke}{rgb}{0.000000,0.000000,0.000000}%
\pgfsetstrokecolor{currentstroke}%
\pgfsetdash{}{0pt}%
\pgfpathmoveto{\pgfqpoint{3.831907in}{2.613810in}}%
\pgfpathlineto{\pgfqpoint{3.845195in}{2.608598in}}%
\pgfpathlineto{\pgfqpoint{3.858488in}{2.603513in}}%
\pgfpathlineto{\pgfqpoint{3.871785in}{2.598554in}}%
\pgfpathlineto{\pgfqpoint{3.885086in}{2.593721in}}%
\pgfpathlineto{\pgfqpoint{3.877314in}{2.585220in}}%
\pgfpathlineto{\pgfqpoint{3.869537in}{2.576738in}}%
\pgfpathlineto{\pgfqpoint{3.861753in}{2.568276in}}%
\pgfpathlineto{\pgfqpoint{3.853965in}{2.559833in}}%
\pgfpathlineto{\pgfqpoint{3.840651in}{2.564716in}}%
\pgfpathlineto{\pgfqpoint{3.827341in}{2.569724in}}%
\pgfpathlineto{\pgfqpoint{3.814037in}{2.574859in}}%
\pgfpathlineto{\pgfqpoint{3.800736in}{2.580121in}}%
\pgfpathlineto{\pgfqpoint{3.808537in}{2.588508in}}%
\pgfpathlineto{\pgfqpoint{3.816333in}{2.596918in}}%
\pgfpathlineto{\pgfqpoint{3.824123in}{2.605352in}}%
\pgfpathlineto{\pgfqpoint{3.831907in}{2.613810in}}%
\pgfpathclose%
\pgfusepath{fill}%
\end{pgfscope}%
\begin{pgfscope}%
\pgfpathrectangle{\pgfqpoint{1.150000in}{0.150000in}}{\pgfqpoint{5.700000in}{5.700000in}}%
\pgfusepath{clip}%
\pgfsetbuttcap%
\pgfsetroundjoin%
\definecolor{currentfill}{rgb}{0.266580,0.228262,0.514349}%
\pgfsetfillcolor{currentfill}%
\pgfsetfillopacity{0.700000}%
\pgfsetlinewidth{0.000000pt}%
\definecolor{currentstroke}{rgb}{0.000000,0.000000,0.000000}%
\pgfsetstrokecolor{currentstroke}%
\pgfsetdash{}{0pt}%
\pgfpathmoveto{\pgfqpoint{5.138517in}{2.940394in}}%
\pgfpathlineto{\pgfqpoint{5.152184in}{2.941756in}}%
\pgfpathlineto{\pgfqpoint{5.165862in}{2.943217in}}%
\pgfpathlineto{\pgfqpoint{5.179551in}{2.944780in}}%
\pgfpathlineto{\pgfqpoint{5.193251in}{2.946442in}}%
\pgfpathlineto{\pgfqpoint{5.185951in}{2.939241in}}%
\pgfpathlineto{\pgfqpoint{5.178647in}{2.932039in}}%
\pgfpathlineto{\pgfqpoint{5.171337in}{2.924835in}}%
\pgfpathlineto{\pgfqpoint{5.164022in}{2.917627in}}%
\pgfpathlineto{\pgfqpoint{5.150308in}{2.915777in}}%
\pgfpathlineto{\pgfqpoint{5.136604in}{2.914028in}}%
\pgfpathlineto{\pgfqpoint{5.122912in}{2.912380in}}%
\pgfpathlineto{\pgfqpoint{5.109230in}{2.910832in}}%
\pgfpathlineto{\pgfqpoint{5.116559in}{2.918221in}}%
\pgfpathlineto{\pgfqpoint{5.123883in}{2.925609in}}%
\pgfpathlineto{\pgfqpoint{5.131203in}{2.933000in}}%
\pgfpathlineto{\pgfqpoint{5.138517in}{2.940394in}}%
\pgfpathclose%
\pgfusepath{fill}%
\end{pgfscope}%
\begin{pgfscope}%
\pgfpathrectangle{\pgfqpoint{1.150000in}{0.150000in}}{\pgfqpoint{5.700000in}{5.700000in}}%
\pgfusepath{clip}%
\pgfsetbuttcap%
\pgfsetroundjoin%
\definecolor{currentfill}{rgb}{0.280868,0.160771,0.472899}%
\pgfsetfillcolor{currentfill}%
\pgfsetfillopacity{0.700000}%
\pgfsetlinewidth{0.000000pt}%
\definecolor{currentstroke}{rgb}{0.000000,0.000000,0.000000}%
\pgfsetstrokecolor{currentstroke}%
\pgfsetdash{}{0pt}%
\pgfpathmoveto{\pgfqpoint{3.175226in}{2.816306in}}%
\pgfpathlineto{\pgfqpoint{3.188494in}{2.804966in}}%
\pgfpathlineto{\pgfqpoint{3.201761in}{2.793793in}}%
\pgfpathlineto{\pgfqpoint{3.215027in}{2.782785in}}%
\pgfpathlineto{\pgfqpoint{3.228293in}{2.771943in}}%
\pgfpathlineto{\pgfqpoint{3.220273in}{2.764957in}}%
\pgfpathlineto{\pgfqpoint{3.212246in}{2.758044in}}%
\pgfpathlineto{\pgfqpoint{3.204212in}{2.751205in}}%
\pgfpathlineto{\pgfqpoint{3.196170in}{2.744440in}}%
\pgfpathlineto{\pgfqpoint{3.182885in}{2.755408in}}%
\pgfpathlineto{\pgfqpoint{3.169599in}{2.766541in}}%
\pgfpathlineto{\pgfqpoint{3.156313in}{2.777841in}}%
\pgfpathlineto{\pgfqpoint{3.143026in}{2.789307in}}%
\pgfpathlineto{\pgfqpoint{3.151087in}{2.795940in}}%
\pgfpathlineto{\pgfqpoint{3.159141in}{2.802651in}}%
\pgfpathlineto{\pgfqpoint{3.167187in}{2.809440in}}%
\pgfpathlineto{\pgfqpoint{3.175226in}{2.816306in}}%
\pgfpathclose%
\pgfusepath{fill}%
\end{pgfscope}%
\begin{pgfscope}%
\pgfpathrectangle{\pgfqpoint{1.150000in}{0.150000in}}{\pgfqpoint{5.700000in}{5.700000in}}%
\pgfusepath{clip}%
\pgfsetbuttcap%
\pgfsetroundjoin%
\definecolor{currentfill}{rgb}{0.283072,0.130895,0.449241}%
\pgfsetfillcolor{currentfill}%
\pgfsetfillopacity{0.700000}%
\pgfsetlinewidth{0.000000pt}%
\definecolor{currentstroke}{rgb}{0.000000,0.000000,0.000000}%
\pgfsetstrokecolor{currentstroke}%
\pgfsetdash{}{0pt}%
\pgfpathmoveto{\pgfqpoint{4.580687in}{2.739915in}}%
\pgfpathlineto{\pgfqpoint{4.594162in}{2.739208in}}%
\pgfpathlineto{\pgfqpoint{4.607645in}{2.738609in}}%
\pgfpathlineto{\pgfqpoint{4.621136in}{2.738117in}}%
\pgfpathlineto{\pgfqpoint{4.634637in}{2.737733in}}%
\pgfpathlineto{\pgfqpoint{4.627122in}{2.729395in}}%
\pgfpathlineto{\pgfqpoint{4.619602in}{2.721046in}}%
\pgfpathlineto{\pgfqpoint{4.612076in}{2.712686in}}%
\pgfpathlineto{\pgfqpoint{4.604545in}{2.704312in}}%
\pgfpathlineto{\pgfqpoint{4.591034in}{2.704620in}}%
\pgfpathlineto{\pgfqpoint{4.577531in}{2.705035in}}%
\pgfpathlineto{\pgfqpoint{4.564037in}{2.705557in}}%
\pgfpathlineto{\pgfqpoint{4.550552in}{2.706187in}}%
\pgfpathlineto{\pgfqpoint{4.558094in}{2.714631in}}%
\pgfpathlineto{\pgfqpoint{4.565630in}{2.723066in}}%
\pgfpathlineto{\pgfqpoint{4.573161in}{2.731493in}}%
\pgfpathlineto{\pgfqpoint{4.580687in}{2.739915in}}%
\pgfpathclose%
\pgfusepath{fill}%
\end{pgfscope}%
\begin{pgfscope}%
\pgfpathrectangle{\pgfqpoint{1.150000in}{0.150000in}}{\pgfqpoint{5.700000in}{5.700000in}}%
\pgfusepath{clip}%
\pgfsetbuttcap%
\pgfsetroundjoin%
\definecolor{currentfill}{rgb}{0.270595,0.214069,0.507052}%
\pgfsetfillcolor{currentfill}%
\pgfsetfillopacity{0.700000}%
\pgfsetlinewidth{0.000000pt}%
\definecolor{currentstroke}{rgb}{0.000000,0.000000,0.000000}%
\pgfsetstrokecolor{currentstroke}%
\pgfsetdash{}{0pt}%
\pgfpathmoveto{\pgfqpoint{5.054608in}{2.905651in}}%
\pgfpathlineto{\pgfqpoint{5.068247in}{2.906795in}}%
\pgfpathlineto{\pgfqpoint{5.081898in}{2.908039in}}%
\pgfpathlineto{\pgfqpoint{5.095558in}{2.909385in}}%
\pgfpathlineto{\pgfqpoint{5.109230in}{2.910832in}}%
\pgfpathlineto{\pgfqpoint{5.101895in}{2.903441in}}%
\pgfpathlineto{\pgfqpoint{5.094555in}{2.896045in}}%
\pgfpathlineto{\pgfqpoint{5.087209in}{2.888643in}}%
\pgfpathlineto{\pgfqpoint{5.079859in}{2.881232in}}%
\pgfpathlineto{\pgfqpoint{5.066173in}{2.879616in}}%
\pgfpathlineto{\pgfqpoint{5.052499in}{2.878102in}}%
\pgfpathlineto{\pgfqpoint{5.038835in}{2.876688in}}%
\pgfpathlineto{\pgfqpoint{5.025182in}{2.875377in}}%
\pgfpathlineto{\pgfqpoint{5.032546in}{2.882950in}}%
\pgfpathlineto{\pgfqpoint{5.039905in}{2.890519in}}%
\pgfpathlineto{\pgfqpoint{5.047259in}{2.898085in}}%
\pgfpathlineto{\pgfqpoint{5.054608in}{2.905651in}}%
\pgfpathclose%
\pgfusepath{fill}%
\end{pgfscope}%
\begin{pgfscope}%
\pgfpathrectangle{\pgfqpoint{1.150000in}{0.150000in}}{\pgfqpoint{5.700000in}{5.700000in}}%
\pgfusepath{clip}%
\pgfsetbuttcap%
\pgfsetroundjoin%
\definecolor{currentfill}{rgb}{0.277941,0.056324,0.381191}%
\pgfsetfillcolor{currentfill}%
\pgfsetfillopacity{0.700000}%
\pgfsetlinewidth{0.000000pt}%
\definecolor{currentstroke}{rgb}{0.000000,0.000000,0.000000}%
\pgfsetstrokecolor{currentstroke}%
\pgfsetdash{}{0pt}%
\pgfpathmoveto{\pgfqpoint{3.969331in}{2.609959in}}%
\pgfpathlineto{\pgfqpoint{3.982646in}{2.605778in}}%
\pgfpathlineto{\pgfqpoint{3.995967in}{2.601719in}}%
\pgfpathlineto{\pgfqpoint{4.009294in}{2.597782in}}%
\pgfpathlineto{\pgfqpoint{4.022626in}{2.593966in}}%
\pgfpathlineto{\pgfqpoint{4.014899in}{2.585338in}}%
\pgfpathlineto{\pgfqpoint{4.007166in}{2.576721in}}%
\pgfpathlineto{\pgfqpoint{3.999429in}{2.568114in}}%
\pgfpathlineto{\pgfqpoint{3.991686in}{2.559517in}}%
\pgfpathlineto{\pgfqpoint{3.978342in}{2.563365in}}%
\pgfpathlineto{\pgfqpoint{3.965004in}{2.567334in}}%
\pgfpathlineto{\pgfqpoint{3.951671in}{2.571424in}}%
\pgfpathlineto{\pgfqpoint{3.938344in}{2.575637in}}%
\pgfpathlineto{\pgfqpoint{3.946098in}{2.584196in}}%
\pgfpathlineto{\pgfqpoint{3.953848in}{2.592769in}}%
\pgfpathlineto{\pgfqpoint{3.961592in}{2.601357in}}%
\pgfpathlineto{\pgfqpoint{3.969331in}{2.609959in}}%
\pgfpathclose%
\pgfusepath{fill}%
\end{pgfscope}%
\begin{pgfscope}%
\pgfpathrectangle{\pgfqpoint{1.150000in}{0.150000in}}{\pgfqpoint{5.700000in}{5.700000in}}%
\pgfusepath{clip}%
\pgfsetbuttcap%
\pgfsetroundjoin%
\definecolor{currentfill}{rgb}{0.282327,0.094955,0.417331}%
\pgfsetfillcolor{currentfill}%
\pgfsetfillopacity{0.700000}%
\pgfsetlinewidth{0.000000pt}%
\definecolor{currentstroke}{rgb}{0.000000,0.000000,0.000000}%
\pgfsetstrokecolor{currentstroke}%
\pgfsetdash{}{0pt}%
\pgfpathmoveto{\pgfqpoint{3.419302in}{2.684125in}}%
\pgfpathlineto{\pgfqpoint{3.432558in}{2.675414in}}%
\pgfpathlineto{\pgfqpoint{3.445816in}{2.666850in}}%
\pgfpathlineto{\pgfqpoint{3.459075in}{2.658433in}}%
\pgfpathlineto{\pgfqpoint{3.472337in}{2.650162in}}%
\pgfpathlineto{\pgfqpoint{3.464412in}{2.642498in}}%
\pgfpathlineto{\pgfqpoint{3.456481in}{2.634884in}}%
\pgfpathlineto{\pgfqpoint{3.448544in}{2.627323in}}%
\pgfpathlineto{\pgfqpoint{3.440600in}{2.619814in}}%
\pgfpathlineto{\pgfqpoint{3.427322in}{2.628190in}}%
\pgfpathlineto{\pgfqpoint{3.414047in}{2.636712in}}%
\pgfpathlineto{\pgfqpoint{3.400773in}{2.645381in}}%
\pgfpathlineto{\pgfqpoint{3.387500in}{2.654198in}}%
\pgfpathlineto{\pgfqpoint{3.395460in}{2.661595in}}%
\pgfpathlineto{\pgfqpoint{3.403414in}{2.669049in}}%
\pgfpathlineto{\pgfqpoint{3.411361in}{2.676559in}}%
\pgfpathlineto{\pgfqpoint{3.419302in}{2.684125in}}%
\pgfpathclose%
\pgfusepath{fill}%
\end{pgfscope}%
\begin{pgfscope}%
\pgfpathrectangle{\pgfqpoint{1.150000in}{0.150000in}}{\pgfqpoint{5.700000in}{5.700000in}}%
\pgfusepath{clip}%
\pgfsetbuttcap%
\pgfsetroundjoin%
\definecolor{currentfill}{rgb}{0.280267,0.073417,0.397163}%
\pgfsetfillcolor{currentfill}%
\pgfsetfillopacity{0.700000}%
\pgfsetlinewidth{0.000000pt}%
\definecolor{currentstroke}{rgb}{0.000000,0.000000,0.000000}%
\pgfsetstrokecolor{currentstroke}%
\pgfsetdash{}{0pt}%
\pgfpathmoveto{\pgfqpoint{4.190949in}{2.637202in}}%
\pgfpathlineto{\pgfqpoint{4.204315in}{2.634459in}}%
\pgfpathlineto{\pgfqpoint{4.217687in}{2.631832in}}%
\pgfpathlineto{\pgfqpoint{4.231066in}{2.629321in}}%
\pgfpathlineto{\pgfqpoint{4.244453in}{2.626924in}}%
\pgfpathlineto{\pgfqpoint{4.236800in}{2.618242in}}%
\pgfpathlineto{\pgfqpoint{4.229143in}{2.609558in}}%
\pgfpathlineto{\pgfqpoint{4.221480in}{2.600873in}}%
\pgfpathlineto{\pgfqpoint{4.213813in}{2.592186in}}%
\pgfpathlineto{\pgfqpoint{4.200416in}{2.594578in}}%
\pgfpathlineto{\pgfqpoint{4.187026in}{2.597086in}}%
\pgfpathlineto{\pgfqpoint{4.173643in}{2.599708in}}%
\pgfpathlineto{\pgfqpoint{4.160266in}{2.602446in}}%
\pgfpathlineto{\pgfqpoint{4.167945in}{2.611131in}}%
\pgfpathlineto{\pgfqpoint{4.175618in}{2.619819in}}%
\pgfpathlineto{\pgfqpoint{4.183286in}{2.628509in}}%
\pgfpathlineto{\pgfqpoint{4.190949in}{2.637202in}}%
\pgfpathclose%
\pgfusepath{fill}%
\end{pgfscope}%
\begin{pgfscope}%
\pgfpathrectangle{\pgfqpoint{1.150000in}{0.150000in}}{\pgfqpoint{5.700000in}{5.700000in}}%
\pgfusepath{clip}%
\pgfsetbuttcap%
\pgfsetroundjoin%
\definecolor{currentfill}{rgb}{0.283197,0.115680,0.436115}%
\pgfsetfillcolor{currentfill}%
\pgfsetfillopacity{0.700000}%
\pgfsetlinewidth{0.000000pt}%
\definecolor{currentstroke}{rgb}{0.000000,0.000000,0.000000}%
\pgfsetstrokecolor{currentstroke}%
\pgfsetdash{}{0pt}%
\pgfpathmoveto{\pgfqpoint{4.496694in}{2.709792in}}%
\pgfpathlineto{\pgfqpoint{4.510146in}{2.708727in}}%
\pgfpathlineto{\pgfqpoint{4.523606in}{2.707772in}}%
\pgfpathlineto{\pgfqpoint{4.537075in}{2.706925in}}%
\pgfpathlineto{\pgfqpoint{4.550552in}{2.706187in}}%
\pgfpathlineto{\pgfqpoint{4.543005in}{2.697733in}}%
\pgfpathlineto{\pgfqpoint{4.535453in}{2.689269in}}%
\pgfpathlineto{\pgfqpoint{4.527895in}{2.680793in}}%
\pgfpathlineto{\pgfqpoint{4.520333in}{2.672305in}}%
\pgfpathlineto{\pgfqpoint{4.506845in}{2.672984in}}%
\pgfpathlineto{\pgfqpoint{4.493366in}{2.673772in}}%
\pgfpathlineto{\pgfqpoint{4.479894in}{2.674668in}}%
\pgfpathlineto{\pgfqpoint{4.466432in}{2.675674in}}%
\pgfpathlineto{\pgfqpoint{4.474005in}{2.684215in}}%
\pgfpathlineto{\pgfqpoint{4.481573in}{2.692747in}}%
\pgfpathlineto{\pgfqpoint{4.489136in}{2.701273in}}%
\pgfpathlineto{\pgfqpoint{4.496694in}{2.709792in}}%
\pgfpathclose%
\pgfusepath{fill}%
\end{pgfscope}%
\begin{pgfscope}%
\pgfpathrectangle{\pgfqpoint{1.150000in}{0.150000in}}{\pgfqpoint{5.700000in}{5.700000in}}%
\pgfusepath{clip}%
\pgfsetbuttcap%
\pgfsetroundjoin%
\definecolor{currentfill}{rgb}{0.274128,0.199721,0.498911}%
\pgfsetfillcolor{currentfill}%
\pgfsetfillopacity{0.700000}%
\pgfsetlinewidth{0.000000pt}%
\definecolor{currentstroke}{rgb}{0.000000,0.000000,0.000000}%
\pgfsetstrokecolor{currentstroke}%
\pgfsetdash{}{0pt}%
\pgfpathmoveto{\pgfqpoint{4.970670in}{2.871149in}}%
\pgfpathlineto{\pgfqpoint{4.984283in}{2.872053in}}%
\pgfpathlineto{\pgfqpoint{4.997906in}{2.873059in}}%
\pgfpathlineto{\pgfqpoint{5.011538in}{2.874167in}}%
\pgfpathlineto{\pgfqpoint{5.025182in}{2.875377in}}%
\pgfpathlineto{\pgfqpoint{5.017812in}{2.867797in}}%
\pgfpathlineto{\pgfqpoint{5.010437in}{2.860210in}}%
\pgfpathlineto{\pgfqpoint{5.003056in}{2.852612in}}%
\pgfpathlineto{\pgfqpoint{4.995670in}{2.845002in}}%
\pgfpathlineto{\pgfqpoint{4.982014in}{2.843642in}}%
\pgfpathlineto{\pgfqpoint{4.968368in}{2.842384in}}%
\pgfpathlineto{\pgfqpoint{4.954733in}{2.841228in}}%
\pgfpathlineto{\pgfqpoint{4.941107in}{2.840174in}}%
\pgfpathlineto{\pgfqpoint{4.948506in}{2.847927in}}%
\pgfpathlineto{\pgfqpoint{4.955899in}{2.855673in}}%
\pgfpathlineto{\pgfqpoint{4.963287in}{2.863413in}}%
\pgfpathlineto{\pgfqpoint{4.970670in}{2.871149in}}%
\pgfpathclose%
\pgfusepath{fill}%
\end{pgfscope}%
\begin{pgfscope}%
\pgfpathrectangle{\pgfqpoint{1.150000in}{0.150000in}}{\pgfqpoint{5.700000in}{5.700000in}}%
\pgfusepath{clip}%
\pgfsetbuttcap%
\pgfsetroundjoin%
\definecolor{currentfill}{rgb}{0.282623,0.140926,0.457517}%
\pgfsetfillcolor{currentfill}%
\pgfsetfillopacity{0.700000}%
\pgfsetlinewidth{0.000000pt}%
\definecolor{currentstroke}{rgb}{0.000000,0.000000,0.000000}%
\pgfsetstrokecolor{currentstroke}%
\pgfsetdash{}{0pt}%
\pgfpathmoveto{\pgfqpoint{3.228293in}{2.771943in}}%
\pgfpathlineto{\pgfqpoint{3.241559in}{2.761263in}}%
\pgfpathlineto{\pgfqpoint{3.254824in}{2.750745in}}%
\pgfpathlineto{\pgfqpoint{3.268090in}{2.740388in}}%
\pgfpathlineto{\pgfqpoint{3.281355in}{2.730190in}}%
\pgfpathlineto{\pgfqpoint{3.273354in}{2.723086in}}%
\pgfpathlineto{\pgfqpoint{3.265345in}{2.716050in}}%
\pgfpathlineto{\pgfqpoint{3.257329in}{2.709083in}}%
\pgfpathlineto{\pgfqpoint{3.249306in}{2.702186in}}%
\pgfpathlineto{\pgfqpoint{3.236022in}{2.712509in}}%
\pgfpathlineto{\pgfqpoint{3.222738in}{2.722991in}}%
\pgfpathlineto{\pgfqpoint{3.209454in}{2.733635in}}%
\pgfpathlineto{\pgfqpoint{3.196170in}{2.744440in}}%
\pgfpathlineto{\pgfqpoint{3.204212in}{2.751205in}}%
\pgfpathlineto{\pgfqpoint{3.212246in}{2.758044in}}%
\pgfpathlineto{\pgfqpoint{3.220273in}{2.764957in}}%
\pgfpathlineto{\pgfqpoint{3.228293in}{2.771943in}}%
\pgfpathclose%
\pgfusepath{fill}%
\end{pgfscope}%
\begin{pgfscope}%
\pgfpathrectangle{\pgfqpoint{1.150000in}{0.150000in}}{\pgfqpoint{5.700000in}{5.700000in}}%
\pgfusepath{clip}%
\pgfsetbuttcap%
\pgfsetroundjoin%
\definecolor{currentfill}{rgb}{0.277134,0.185228,0.489898}%
\pgfsetfillcolor{currentfill}%
\pgfsetfillopacity{0.700000}%
\pgfsetlinewidth{0.000000pt}%
\definecolor{currentstroke}{rgb}{0.000000,0.000000,0.000000}%
\pgfsetstrokecolor{currentstroke}%
\pgfsetdash{}{0pt}%
\pgfpathmoveto{\pgfqpoint{4.886705in}{2.836987in}}%
\pgfpathlineto{\pgfqpoint{4.900291in}{2.837629in}}%
\pgfpathlineto{\pgfqpoint{4.913886in}{2.838374in}}%
\pgfpathlineto{\pgfqpoint{4.927492in}{2.839223in}}%
\pgfpathlineto{\pgfqpoint{4.941107in}{2.840174in}}%
\pgfpathlineto{\pgfqpoint{4.933703in}{2.832411in}}%
\pgfpathlineto{\pgfqpoint{4.926294in}{2.824637in}}%
\pgfpathlineto{\pgfqpoint{4.918879in}{2.816851in}}%
\pgfpathlineto{\pgfqpoint{4.911459in}{2.809050in}}%
\pgfpathlineto{\pgfqpoint{4.897831in}{2.807966in}}%
\pgfpathlineto{\pgfqpoint{4.884213in}{2.806986in}}%
\pgfpathlineto{\pgfqpoint{4.870605in}{2.806109in}}%
\pgfpathlineto{\pgfqpoint{4.857007in}{2.805335in}}%
\pgfpathlineto{\pgfqpoint{4.864439in}{2.813262in}}%
\pgfpathlineto{\pgfqpoint{4.871867in}{2.821178in}}%
\pgfpathlineto{\pgfqpoint{4.879288in}{2.829086in}}%
\pgfpathlineto{\pgfqpoint{4.886705in}{2.836987in}}%
\pgfpathclose%
\pgfusepath{fill}%
\end{pgfscope}%
\begin{pgfscope}%
\pgfpathrectangle{\pgfqpoint{1.150000in}{0.150000in}}{\pgfqpoint{5.700000in}{5.700000in}}%
\pgfusepath{clip}%
\pgfsetbuttcap%
\pgfsetroundjoin%
\definecolor{currentfill}{rgb}{0.277941,0.056324,0.381191}%
\pgfsetfillcolor{currentfill}%
\pgfsetfillopacity{0.700000}%
\pgfsetlinewidth{0.000000pt}%
\definecolor{currentstroke}{rgb}{0.000000,0.000000,0.000000}%
\pgfsetstrokecolor{currentstroke}%
\pgfsetdash{}{0pt}%
\pgfpathmoveto{\pgfqpoint{3.747578in}{2.602457in}}%
\pgfpathlineto{\pgfqpoint{3.760861in}{2.596679in}}%
\pgfpathlineto{\pgfqpoint{3.774149in}{2.591030in}}%
\pgfpathlineto{\pgfqpoint{3.787440in}{2.585511in}}%
\pgfpathlineto{\pgfqpoint{3.800736in}{2.580121in}}%
\pgfpathlineto{\pgfqpoint{3.792930in}{2.571759in}}%
\pgfpathlineto{\pgfqpoint{3.785117in}{2.563422in}}%
\pgfpathlineto{\pgfqpoint{3.777299in}{2.555109in}}%
\pgfpathlineto{\pgfqpoint{3.769476in}{2.546823in}}%
\pgfpathlineto{\pgfqpoint{3.756167in}{2.552281in}}%
\pgfpathlineto{\pgfqpoint{3.742862in}{2.557867in}}%
\pgfpathlineto{\pgfqpoint{3.729562in}{2.563583in}}%
\pgfpathlineto{\pgfqpoint{3.716265in}{2.569430in}}%
\pgfpathlineto{\pgfqpoint{3.724102in}{2.577642in}}%
\pgfpathlineto{\pgfqpoint{3.731933in}{2.585885in}}%
\pgfpathlineto{\pgfqpoint{3.739758in}{2.594156in}}%
\pgfpathlineto{\pgfqpoint{3.747578in}{2.602457in}}%
\pgfpathclose%
\pgfusepath{fill}%
\end{pgfscope}%
\begin{pgfscope}%
\pgfpathrectangle{\pgfqpoint{1.150000in}{0.150000in}}{\pgfqpoint{5.700000in}{5.700000in}}%
\pgfusepath{clip}%
\pgfsetbuttcap%
\pgfsetroundjoin%
\definecolor{currentfill}{rgb}{0.278791,0.062145,0.386592}%
\pgfsetfillcolor{currentfill}%
\pgfsetfillopacity{0.700000}%
\pgfsetlinewidth{0.000000pt}%
\definecolor{currentstroke}{rgb}{0.000000,0.000000,0.000000}%
\pgfsetstrokecolor{currentstroke}%
\pgfsetdash{}{0pt}%
\pgfpathmoveto{\pgfqpoint{3.610023in}{2.620978in}}%
\pgfpathlineto{\pgfqpoint{3.623292in}{2.614063in}}%
\pgfpathlineto{\pgfqpoint{3.636564in}{2.607284in}}%
\pgfpathlineto{\pgfqpoint{3.649839in}{2.600641in}}%
\pgfpathlineto{\pgfqpoint{3.663117in}{2.594132in}}%
\pgfpathlineto{\pgfqpoint{3.655261in}{2.586029in}}%
\pgfpathlineto{\pgfqpoint{3.647399in}{2.577962in}}%
\pgfpathlineto{\pgfqpoint{3.639531in}{2.569930in}}%
\pgfpathlineto{\pgfqpoint{3.631657in}{2.561935in}}%
\pgfpathlineto{\pgfqpoint{3.618365in}{2.568530in}}%
\pgfpathlineto{\pgfqpoint{3.605076in}{2.575259in}}%
\pgfpathlineto{\pgfqpoint{3.591790in}{2.582124in}}%
\pgfpathlineto{\pgfqpoint{3.578507in}{2.589126in}}%
\pgfpathlineto{\pgfqpoint{3.586395in}{2.597029in}}%
\pgfpathlineto{\pgfqpoint{3.594277in}{2.604972in}}%
\pgfpathlineto{\pgfqpoint{3.602153in}{2.612955in}}%
\pgfpathlineto{\pgfqpoint{3.610023in}{2.620978in}}%
\pgfpathclose%
\pgfusepath{fill}%
\end{pgfscope}%
\begin{pgfscope}%
\pgfpathrectangle{\pgfqpoint{1.150000in}{0.150000in}}{\pgfqpoint{5.700000in}{5.700000in}}%
\pgfusepath{clip}%
\pgfsetbuttcap%
\pgfsetroundjoin%
\definecolor{currentfill}{rgb}{0.282656,0.100196,0.422160}%
\pgfsetfillcolor{currentfill}%
\pgfsetfillopacity{0.700000}%
\pgfsetlinewidth{0.000000pt}%
\definecolor{currentstroke}{rgb}{0.000000,0.000000,0.000000}%
\pgfsetstrokecolor{currentstroke}%
\pgfsetdash{}{0pt}%
\pgfpathmoveto{\pgfqpoint{4.412660in}{2.680797in}}%
\pgfpathlineto{\pgfqpoint{4.426091in}{2.679350in}}%
\pgfpathlineto{\pgfqpoint{4.439530in}{2.678015in}}%
\pgfpathlineto{\pgfqpoint{4.452977in}{2.676790in}}%
\pgfpathlineto{\pgfqpoint{4.466432in}{2.675674in}}%
\pgfpathlineto{\pgfqpoint{4.458853in}{2.667124in}}%
\pgfpathlineto{\pgfqpoint{4.451269in}{2.658565in}}%
\pgfpathlineto{\pgfqpoint{4.443680in}{2.649995in}}%
\pgfpathlineto{\pgfqpoint{4.436086in}{2.641414in}}%
\pgfpathlineto{\pgfqpoint{4.422620in}{2.642488in}}%
\pgfpathlineto{\pgfqpoint{4.409163in}{2.643673in}}%
\pgfpathlineto{\pgfqpoint{4.395714in}{2.644968in}}%
\pgfpathlineto{\pgfqpoint{4.382272in}{2.646373in}}%
\pgfpathlineto{\pgfqpoint{4.389877in}{2.654988in}}%
\pgfpathlineto{\pgfqpoint{4.397476in}{2.663597in}}%
\pgfpathlineto{\pgfqpoint{4.405071in}{2.672200in}}%
\pgfpathlineto{\pgfqpoint{4.412660in}{2.680797in}}%
\pgfpathclose%
\pgfusepath{fill}%
\end{pgfscope}%
\begin{pgfscope}%
\pgfpathrectangle{\pgfqpoint{1.150000in}{0.150000in}}{\pgfqpoint{5.700000in}{5.700000in}}%
\pgfusepath{clip}%
\pgfsetbuttcap%
\pgfsetroundjoin%
\definecolor{currentfill}{rgb}{0.278791,0.062145,0.386592}%
\pgfsetfillcolor{currentfill}%
\pgfsetfillopacity{0.700000}%
\pgfsetlinewidth{0.000000pt}%
\definecolor{currentstroke}{rgb}{0.000000,0.000000,0.000000}%
\pgfsetstrokecolor{currentstroke}%
\pgfsetdash{}{0pt}%
\pgfpathmoveto{\pgfqpoint{4.106824in}{2.614567in}}%
\pgfpathlineto{\pgfqpoint{4.120175in}{2.611361in}}%
\pgfpathlineto{\pgfqpoint{4.133532in}{2.608272in}}%
\pgfpathlineto{\pgfqpoint{4.146896in}{2.605301in}}%
\pgfpathlineto{\pgfqpoint{4.160266in}{2.602446in}}%
\pgfpathlineto{\pgfqpoint{4.152582in}{2.593763in}}%
\pgfpathlineto{\pgfqpoint{4.144893in}{2.585082in}}%
\pgfpathlineto{\pgfqpoint{4.137199in}{2.576403in}}%
\pgfpathlineto{\pgfqpoint{4.129499in}{2.567725in}}%
\pgfpathlineto{\pgfqpoint{4.116118in}{2.570593in}}%
\pgfpathlineto{\pgfqpoint{4.102744in}{2.573577in}}%
\pgfpathlineto{\pgfqpoint{4.089376in}{2.576679in}}%
\pgfpathlineto{\pgfqpoint{4.076014in}{2.579899in}}%
\pgfpathlineto{\pgfqpoint{4.083724in}{2.588557in}}%
\pgfpathlineto{\pgfqpoint{4.091430in}{2.597221in}}%
\pgfpathlineto{\pgfqpoint{4.099130in}{2.605891in}}%
\pgfpathlineto{\pgfqpoint{4.106824in}{2.614567in}}%
\pgfpathclose%
\pgfusepath{fill}%
\end{pgfscope}%
\begin{pgfscope}%
\pgfpathrectangle{\pgfqpoint{1.150000in}{0.150000in}}{\pgfqpoint{5.700000in}{5.700000in}}%
\pgfusepath{clip}%
\pgfsetbuttcap%
\pgfsetroundjoin%
\definecolor{currentfill}{rgb}{0.258965,0.251537,0.524736}%
\pgfsetfillcolor{currentfill}%
\pgfsetfillopacity{0.700000}%
\pgfsetlinewidth{0.000000pt}%
\definecolor{currentstroke}{rgb}{0.000000,0.000000,0.000000}%
\pgfsetstrokecolor{currentstroke}%
\pgfsetdash{}{0pt}%
\pgfpathmoveto{\pgfqpoint{2.930179in}{2.996740in}}%
\pgfpathlineto{\pgfqpoint{2.943503in}{2.982390in}}%
\pgfpathlineto{\pgfqpoint{2.956823in}{2.968232in}}%
\pgfpathlineto{\pgfqpoint{2.970139in}{2.954266in}}%
\pgfpathlineto{\pgfqpoint{2.983452in}{2.940488in}}%
\pgfpathlineto{\pgfqpoint{2.975321in}{2.934353in}}%
\pgfpathlineto{\pgfqpoint{2.967181in}{2.928312in}}%
\pgfpathlineto{\pgfqpoint{2.959032in}{2.922367in}}%
\pgfpathlineto{\pgfqpoint{2.950875in}{2.916520in}}%
\pgfpathlineto{\pgfqpoint{2.937540in}{2.930444in}}%
\pgfpathlineto{\pgfqpoint{2.924201in}{2.944559in}}%
\pgfpathlineto{\pgfqpoint{2.910858in}{2.958864in}}%
\pgfpathlineto{\pgfqpoint{2.897512in}{2.973363in}}%
\pgfpathlineto{\pgfqpoint{2.905692in}{2.979056in}}%
\pgfpathlineto{\pgfqpoint{2.913863in}{2.984850in}}%
\pgfpathlineto{\pgfqpoint{2.922025in}{2.990745in}}%
\pgfpathlineto{\pgfqpoint{2.930179in}{2.996740in}}%
\pgfpathclose%
\pgfusepath{fill}%
\end{pgfscope}%
\begin{pgfscope}%
\pgfpathrectangle{\pgfqpoint{1.150000in}{0.150000in}}{\pgfqpoint{5.700000in}{5.700000in}}%
\pgfusepath{clip}%
\pgfsetbuttcap%
\pgfsetroundjoin%
\definecolor{currentfill}{rgb}{0.279574,0.170599,0.479997}%
\pgfsetfillcolor{currentfill}%
\pgfsetfillopacity{0.700000}%
\pgfsetlinewidth{0.000000pt}%
\definecolor{currentstroke}{rgb}{0.000000,0.000000,0.000000}%
\pgfsetstrokecolor{currentstroke}%
\pgfsetdash{}{0pt}%
\pgfpathmoveto{\pgfqpoint{4.802712in}{2.803278in}}%
\pgfpathlineto{\pgfqpoint{4.816271in}{2.803636in}}%
\pgfpathlineto{\pgfqpoint{4.829840in}{2.804098in}}%
\pgfpathlineto{\pgfqpoint{4.843419in}{2.804665in}}%
\pgfpathlineto{\pgfqpoint{4.857007in}{2.805335in}}%
\pgfpathlineto{\pgfqpoint{4.849569in}{2.797397in}}%
\pgfpathlineto{\pgfqpoint{4.842126in}{2.789445in}}%
\pgfpathlineto{\pgfqpoint{4.834677in}{2.781479in}}%
\pgfpathlineto{\pgfqpoint{4.827223in}{2.773497in}}%
\pgfpathlineto{\pgfqpoint{4.813623in}{2.772713in}}%
\pgfpathlineto{\pgfqpoint{4.800032in}{2.772032in}}%
\pgfpathlineto{\pgfqpoint{4.786452in}{2.771456in}}%
\pgfpathlineto{\pgfqpoint{4.772880in}{2.770985in}}%
\pgfpathlineto{\pgfqpoint{4.780346in}{2.779074in}}%
\pgfpathlineto{\pgfqpoint{4.787807in}{2.787152in}}%
\pgfpathlineto{\pgfqpoint{4.795262in}{2.795219in}}%
\pgfpathlineto{\pgfqpoint{4.802712in}{2.803278in}}%
\pgfpathclose%
\pgfusepath{fill}%
\end{pgfscope}%
\begin{pgfscope}%
\pgfpathrectangle{\pgfqpoint{1.150000in}{0.150000in}}{\pgfqpoint{5.700000in}{5.700000in}}%
\pgfusepath{clip}%
\pgfsetbuttcap%
\pgfsetroundjoin%
\definecolor{currentfill}{rgb}{0.277018,0.050344,0.375715}%
\pgfsetfillcolor{currentfill}%
\pgfsetfillopacity{0.700000}%
\pgfsetlinewidth{0.000000pt}%
\definecolor{currentstroke}{rgb}{0.000000,0.000000,0.000000}%
\pgfsetstrokecolor{currentstroke}%
\pgfsetdash{}{0pt}%
\pgfpathmoveto{\pgfqpoint{3.885086in}{2.593721in}}%
\pgfpathlineto{\pgfqpoint{3.898393in}{2.589014in}}%
\pgfpathlineto{\pgfqpoint{3.911705in}{2.584431in}}%
\pgfpathlineto{\pgfqpoint{3.925022in}{2.579972in}}%
\pgfpathlineto{\pgfqpoint{3.938344in}{2.575637in}}%
\pgfpathlineto{\pgfqpoint{3.930583in}{2.567093in}}%
\pgfpathlineto{\pgfqpoint{3.922818in}{2.558564in}}%
\pgfpathlineto{\pgfqpoint{3.915047in}{2.550049in}}%
\pgfpathlineto{\pgfqpoint{3.907270in}{2.541550in}}%
\pgfpathlineto{\pgfqpoint{3.893936in}{2.545935in}}%
\pgfpathlineto{\pgfqpoint{3.880608in}{2.550444in}}%
\pgfpathlineto{\pgfqpoint{3.867284in}{2.555076in}}%
\pgfpathlineto{\pgfqpoint{3.853965in}{2.559833in}}%
\pgfpathlineto{\pgfqpoint{3.861753in}{2.568276in}}%
\pgfpathlineto{\pgfqpoint{3.869537in}{2.576738in}}%
\pgfpathlineto{\pgfqpoint{3.877314in}{2.585220in}}%
\pgfpathlineto{\pgfqpoint{3.885086in}{2.593721in}}%
\pgfpathclose%
\pgfusepath{fill}%
\end{pgfscope}%
\begin{pgfscope}%
\pgfpathrectangle{\pgfqpoint{1.150000in}{0.150000in}}{\pgfqpoint{5.700000in}{5.700000in}}%
\pgfusepath{clip}%
\pgfsetbuttcap%
\pgfsetroundjoin%
\definecolor{currentfill}{rgb}{0.283229,0.120777,0.440584}%
\pgfsetfillcolor{currentfill}%
\pgfsetfillopacity{0.700000}%
\pgfsetlinewidth{0.000000pt}%
\definecolor{currentstroke}{rgb}{0.000000,0.000000,0.000000}%
\pgfsetstrokecolor{currentstroke}%
\pgfsetdash{}{0pt}%
\pgfpathmoveto{\pgfqpoint{3.281355in}{2.730190in}}%
\pgfpathlineto{\pgfqpoint{3.294621in}{2.720150in}}%
\pgfpathlineto{\pgfqpoint{3.307887in}{2.710267in}}%
\pgfpathlineto{\pgfqpoint{3.321154in}{2.700539in}}%
\pgfpathlineto{\pgfqpoint{3.334421in}{2.690967in}}%
\pgfpathlineto{\pgfqpoint{3.326437in}{2.683745in}}%
\pgfpathlineto{\pgfqpoint{3.318446in}{2.676587in}}%
\pgfpathlineto{\pgfqpoint{3.310449in}{2.669493in}}%
\pgfpathlineto{\pgfqpoint{3.302444in}{2.662465in}}%
\pgfpathlineto{\pgfqpoint{3.289159in}{2.672162in}}%
\pgfpathlineto{\pgfqpoint{3.275874in}{2.682014in}}%
\pgfpathlineto{\pgfqpoint{3.262590in}{2.692021in}}%
\pgfpathlineto{\pgfqpoint{3.249306in}{2.702186in}}%
\pgfpathlineto{\pgfqpoint{3.257329in}{2.709083in}}%
\pgfpathlineto{\pgfqpoint{3.265345in}{2.716050in}}%
\pgfpathlineto{\pgfqpoint{3.273354in}{2.723086in}}%
\pgfpathlineto{\pgfqpoint{3.281355in}{2.730190in}}%
\pgfpathclose%
\pgfusepath{fill}%
\end{pgfscope}%
\begin{pgfscope}%
\pgfpathrectangle{\pgfqpoint{1.150000in}{0.150000in}}{\pgfqpoint{5.700000in}{5.700000in}}%
\pgfusepath{clip}%
\pgfsetbuttcap%
\pgfsetroundjoin%
\definecolor{currentfill}{rgb}{0.281446,0.084320,0.407414}%
\pgfsetfillcolor{currentfill}%
\pgfsetfillopacity{0.700000}%
\pgfsetlinewidth{0.000000pt}%
\definecolor{currentstroke}{rgb}{0.000000,0.000000,0.000000}%
\pgfsetstrokecolor{currentstroke}%
\pgfsetdash{}{0pt}%
\pgfpathmoveto{\pgfqpoint{3.472337in}{2.650162in}}%
\pgfpathlineto{\pgfqpoint{3.485600in}{2.642036in}}%
\pgfpathlineto{\pgfqpoint{3.498865in}{2.634054in}}%
\pgfpathlineto{\pgfqpoint{3.512133in}{2.626214in}}%
\pgfpathlineto{\pgfqpoint{3.525402in}{2.618517in}}%
\pgfpathlineto{\pgfqpoint{3.517493in}{2.610754in}}%
\pgfpathlineto{\pgfqpoint{3.509578in}{2.603037in}}%
\pgfpathlineto{\pgfqpoint{3.501656in}{2.595369in}}%
\pgfpathlineto{\pgfqpoint{3.493728in}{2.587748in}}%
\pgfpathlineto{\pgfqpoint{3.480443in}{2.595551in}}%
\pgfpathlineto{\pgfqpoint{3.467160in}{2.603495in}}%
\pgfpathlineto{\pgfqpoint{3.453879in}{2.611582in}}%
\pgfpathlineto{\pgfqpoint{3.440600in}{2.619814in}}%
\pgfpathlineto{\pgfqpoint{3.448544in}{2.627323in}}%
\pgfpathlineto{\pgfqpoint{3.456481in}{2.634884in}}%
\pgfpathlineto{\pgfqpoint{3.464412in}{2.642498in}}%
\pgfpathlineto{\pgfqpoint{3.472337in}{2.650162in}}%
\pgfpathclose%
\pgfusepath{fill}%
\end{pgfscope}%
\begin{pgfscope}%
\pgfpathrectangle{\pgfqpoint{1.150000in}{0.150000in}}{\pgfqpoint{5.700000in}{5.700000in}}%
\pgfusepath{clip}%
\pgfsetbuttcap%
\pgfsetroundjoin%
\definecolor{currentfill}{rgb}{0.267968,0.223549,0.512008}%
\pgfsetfillcolor{currentfill}%
\pgfsetfillopacity{0.700000}%
\pgfsetlinewidth{0.000000pt}%
\definecolor{currentstroke}{rgb}{0.000000,0.000000,0.000000}%
\pgfsetstrokecolor{currentstroke}%
\pgfsetdash{}{0pt}%
\pgfpathmoveto{\pgfqpoint{2.983452in}{2.940488in}}%
\pgfpathlineto{\pgfqpoint{2.996763in}{2.926898in}}%
\pgfpathlineto{\pgfqpoint{3.010070in}{2.913494in}}%
\pgfpathlineto{\pgfqpoint{3.023375in}{2.900273in}}%
\pgfpathlineto{\pgfqpoint{3.036677in}{2.887235in}}%
\pgfpathlineto{\pgfqpoint{3.028567in}{2.880960in}}%
\pgfpathlineto{\pgfqpoint{3.020449in}{2.874775in}}%
\pgfpathlineto{\pgfqpoint{3.012323in}{2.868681in}}%
\pgfpathlineto{\pgfqpoint{3.004188in}{2.862679in}}%
\pgfpathlineto{\pgfqpoint{2.990864in}{2.875864in}}%
\pgfpathlineto{\pgfqpoint{2.977538in}{2.889231in}}%
\pgfpathlineto{\pgfqpoint{2.964208in}{2.902782in}}%
\pgfpathlineto{\pgfqpoint{2.950875in}{2.916520in}}%
\pgfpathlineto{\pgfqpoint{2.959032in}{2.922367in}}%
\pgfpathlineto{\pgfqpoint{2.967181in}{2.928312in}}%
\pgfpathlineto{\pgfqpoint{2.975321in}{2.934353in}}%
\pgfpathlineto{\pgfqpoint{2.983452in}{2.940488in}}%
\pgfpathclose%
\pgfusepath{fill}%
\end{pgfscope}%
\begin{pgfscope}%
\pgfpathrectangle{\pgfqpoint{1.150000in}{0.150000in}}{\pgfqpoint{5.700000in}{5.700000in}}%
\pgfusepath{clip}%
\pgfsetbuttcap%
\pgfsetroundjoin%
\definecolor{currentfill}{rgb}{0.281924,0.089666,0.412415}%
\pgfsetfillcolor{currentfill}%
\pgfsetfillopacity{0.700000}%
\pgfsetlinewidth{0.000000pt}%
\definecolor{currentstroke}{rgb}{0.000000,0.000000,0.000000}%
\pgfsetstrokecolor{currentstroke}%
\pgfsetdash{}{0pt}%
\pgfpathmoveto{\pgfqpoint{4.328582in}{2.653109in}}%
\pgfpathlineto{\pgfqpoint{4.341993in}{2.651257in}}%
\pgfpathlineto{\pgfqpoint{4.355412in}{2.649518in}}%
\pgfpathlineto{\pgfqpoint{4.368838in}{2.647890in}}%
\pgfpathlineto{\pgfqpoint{4.382272in}{2.646373in}}%
\pgfpathlineto{\pgfqpoint{4.374662in}{2.637750in}}%
\pgfpathlineto{\pgfqpoint{4.367046in}{2.629119in}}%
\pgfpathlineto{\pgfqpoint{4.359426in}{2.620479in}}%
\pgfpathlineto{\pgfqpoint{4.351800in}{2.611830in}}%
\pgfpathlineto{\pgfqpoint{4.338356in}{2.613324in}}%
\pgfpathlineto{\pgfqpoint{4.324919in}{2.614929in}}%
\pgfpathlineto{\pgfqpoint{4.311490in}{2.616646in}}%
\pgfpathlineto{\pgfqpoint{4.298068in}{2.618475in}}%
\pgfpathlineto{\pgfqpoint{4.305704in}{2.627140in}}%
\pgfpathlineto{\pgfqpoint{4.313335in}{2.635801in}}%
\pgfpathlineto{\pgfqpoint{4.320961in}{2.644457in}}%
\pgfpathlineto{\pgfqpoint{4.328582in}{2.653109in}}%
\pgfpathclose%
\pgfusepath{fill}%
\end{pgfscope}%
\begin{pgfscope}%
\pgfpathrectangle{\pgfqpoint{1.150000in}{0.150000in}}{\pgfqpoint{5.700000in}{5.700000in}}%
\pgfusepath{clip}%
\pgfsetbuttcap%
\pgfsetroundjoin%
\definecolor{currentfill}{rgb}{0.281412,0.155834,0.469201}%
\pgfsetfillcolor{currentfill}%
\pgfsetfillopacity{0.700000}%
\pgfsetlinewidth{0.000000pt}%
\definecolor{currentstroke}{rgb}{0.000000,0.000000,0.000000}%
\pgfsetstrokecolor{currentstroke}%
\pgfsetdash{}{0pt}%
\pgfpathmoveto{\pgfqpoint{4.718689in}{2.770147in}}%
\pgfpathlineto{\pgfqpoint{4.732223in}{2.770199in}}%
\pgfpathlineto{\pgfqpoint{4.745766in}{2.770356in}}%
\pgfpathlineto{\pgfqpoint{4.759319in}{2.770618in}}%
\pgfpathlineto{\pgfqpoint{4.772880in}{2.770985in}}%
\pgfpathlineto{\pgfqpoint{4.765409in}{2.762882in}}%
\pgfpathlineto{\pgfqpoint{4.757933in}{2.754764in}}%
\pgfpathlineto{\pgfqpoint{4.750451in}{2.746631in}}%
\pgfpathlineto{\pgfqpoint{4.742964in}{2.738480in}}%
\pgfpathlineto{\pgfqpoint{4.729391in}{2.738018in}}%
\pgfpathlineto{\pgfqpoint{4.715827in}{2.737660in}}%
\pgfpathlineto{\pgfqpoint{4.702272in}{2.737407in}}%
\pgfpathlineto{\pgfqpoint{4.688727in}{2.737260in}}%
\pgfpathlineto{\pgfqpoint{4.696226in}{2.745500in}}%
\pgfpathlineto{\pgfqpoint{4.703719in}{2.753727in}}%
\pgfpathlineto{\pgfqpoint{4.711207in}{2.761942in}}%
\pgfpathlineto{\pgfqpoint{4.718689in}{2.770147in}}%
\pgfpathclose%
\pgfusepath{fill}%
\end{pgfscope}%
\begin{pgfscope}%
\pgfpathrectangle{\pgfqpoint{1.150000in}{0.150000in}}{\pgfqpoint{5.700000in}{5.700000in}}%
\pgfusepath{clip}%
\pgfsetbuttcap%
\pgfsetroundjoin%
\definecolor{currentfill}{rgb}{0.274128,0.199721,0.498911}%
\pgfsetfillcolor{currentfill}%
\pgfsetfillopacity{0.700000}%
\pgfsetlinewidth{0.000000pt}%
\definecolor{currentstroke}{rgb}{0.000000,0.000000,0.000000}%
\pgfsetstrokecolor{currentstroke}%
\pgfsetdash{}{0pt}%
\pgfpathmoveto{\pgfqpoint{3.036677in}{2.887235in}}%
\pgfpathlineto{\pgfqpoint{3.049977in}{2.874378in}}%
\pgfpathlineto{\pgfqpoint{3.063275in}{2.861700in}}%
\pgfpathlineto{\pgfqpoint{3.076571in}{2.849199in}}%
\pgfpathlineto{\pgfqpoint{3.089865in}{2.836875in}}%
\pgfpathlineto{\pgfqpoint{3.081776in}{2.830460in}}%
\pgfpathlineto{\pgfqpoint{3.073679in}{2.824131in}}%
\pgfpathlineto{\pgfqpoint{3.065574in}{2.817889in}}%
\pgfpathlineto{\pgfqpoint{3.057461in}{2.811735in}}%
\pgfpathlineto{\pgfqpoint{3.044146in}{2.824205in}}%
\pgfpathlineto{\pgfqpoint{3.030829in}{2.836852in}}%
\pgfpathlineto{\pgfqpoint{3.017510in}{2.849676in}}%
\pgfpathlineto{\pgfqpoint{3.004188in}{2.862679in}}%
\pgfpathlineto{\pgfqpoint{3.012323in}{2.868681in}}%
\pgfpathlineto{\pgfqpoint{3.020449in}{2.874775in}}%
\pgfpathlineto{\pgfqpoint{3.028567in}{2.880960in}}%
\pgfpathlineto{\pgfqpoint{3.036677in}{2.887235in}}%
\pgfpathclose%
\pgfusepath{fill}%
\end{pgfscope}%
\begin{pgfscope}%
\pgfpathrectangle{\pgfqpoint{1.150000in}{0.150000in}}{\pgfqpoint{5.700000in}{5.700000in}}%
\pgfusepath{clip}%
\pgfsetbuttcap%
\pgfsetroundjoin%
\definecolor{currentfill}{rgb}{0.277941,0.056324,0.381191}%
\pgfsetfillcolor{currentfill}%
\pgfsetfillopacity{0.700000}%
\pgfsetlinewidth{0.000000pt}%
\definecolor{currentstroke}{rgb}{0.000000,0.000000,0.000000}%
\pgfsetstrokecolor{currentstroke}%
\pgfsetdash{}{0pt}%
\pgfpathmoveto{\pgfqpoint{4.022626in}{2.593966in}}%
\pgfpathlineto{\pgfqpoint{4.035964in}{2.590270in}}%
\pgfpathlineto{\pgfqpoint{4.049308in}{2.586694in}}%
\pgfpathlineto{\pgfqpoint{4.062658in}{2.583237in}}%
\pgfpathlineto{\pgfqpoint{4.076014in}{2.579899in}}%
\pgfpathlineto{\pgfqpoint{4.068298in}{2.571247in}}%
\pgfpathlineto{\pgfqpoint{4.060577in}{2.562600in}}%
\pgfpathlineto{\pgfqpoint{4.052850in}{2.553960in}}%
\pgfpathlineto{\pgfqpoint{4.045118in}{2.545325in}}%
\pgfpathlineto{\pgfqpoint{4.031751in}{2.548694in}}%
\pgfpathlineto{\pgfqpoint{4.018390in}{2.552182in}}%
\pgfpathlineto{\pgfqpoint{4.005035in}{2.555789in}}%
\pgfpathlineto{\pgfqpoint{3.991686in}{2.559517in}}%
\pgfpathlineto{\pgfqpoint{3.999429in}{2.568114in}}%
\pgfpathlineto{\pgfqpoint{4.007166in}{2.576721in}}%
\pgfpathlineto{\pgfqpoint{4.014899in}{2.585338in}}%
\pgfpathlineto{\pgfqpoint{4.022626in}{2.593966in}}%
\pgfpathclose%
\pgfusepath{fill}%
\end{pgfscope}%
\begin{pgfscope}%
\pgfpathrectangle{\pgfqpoint{1.150000in}{0.150000in}}{\pgfqpoint{5.700000in}{5.700000in}}%
\pgfusepath{clip}%
\pgfsetbuttcap%
\pgfsetroundjoin%
\definecolor{currentfill}{rgb}{0.255645,0.260703,0.528312}%
\pgfsetfillcolor{currentfill}%
\pgfsetfillopacity{0.700000}%
\pgfsetlinewidth{0.000000pt}%
\definecolor{currentstroke}{rgb}{0.000000,0.000000,0.000000}%
\pgfsetstrokecolor{currentstroke}%
\pgfsetdash{}{0pt}%
\pgfpathmoveto{\pgfqpoint{5.277243in}{2.982120in}}%
\pgfpathlineto{\pgfqpoint{5.290983in}{2.984075in}}%
\pgfpathlineto{\pgfqpoint{5.304733in}{2.986130in}}%
\pgfpathlineto{\pgfqpoint{5.318495in}{2.988283in}}%
\pgfpathlineto{\pgfqpoint{5.332268in}{2.990536in}}%
\pgfpathlineto{\pgfqpoint{5.325021in}{2.983735in}}%
\pgfpathlineto{\pgfqpoint{5.317768in}{2.976936in}}%
\pgfpathlineto{\pgfqpoint{5.310510in}{2.970135in}}%
\pgfpathlineto{\pgfqpoint{5.303247in}{2.963329in}}%
\pgfpathlineto{\pgfqpoint{5.289458in}{2.960870in}}%
\pgfpathlineto{\pgfqpoint{5.275680in}{2.958511in}}%
\pgfpathlineto{\pgfqpoint{5.261914in}{2.956251in}}%
\pgfpathlineto{\pgfqpoint{5.248159in}{2.954090in}}%
\pgfpathlineto{\pgfqpoint{5.255438in}{2.961095in}}%
\pgfpathlineto{\pgfqpoint{5.262712in}{2.968100in}}%
\pgfpathlineto{\pgfqpoint{5.269980in}{2.975107in}}%
\pgfpathlineto{\pgfqpoint{5.277243in}{2.982120in}}%
\pgfpathclose%
\pgfusepath{fill}%
\end{pgfscope}%
\begin{pgfscope}%
\pgfpathrectangle{\pgfqpoint{1.150000in}{0.150000in}}{\pgfqpoint{5.700000in}{5.700000in}}%
\pgfusepath{clip}%
\pgfsetbuttcap%
\pgfsetroundjoin%
\definecolor{currentfill}{rgb}{0.282623,0.140926,0.457517}%
\pgfsetfillcolor{currentfill}%
\pgfsetfillopacity{0.700000}%
\pgfsetlinewidth{0.000000pt}%
\definecolor{currentstroke}{rgb}{0.000000,0.000000,0.000000}%
\pgfsetstrokecolor{currentstroke}%
\pgfsetdash{}{0pt}%
\pgfpathmoveto{\pgfqpoint{4.634637in}{2.737733in}}%
\pgfpathlineto{\pgfqpoint{4.648146in}{2.737455in}}%
\pgfpathlineto{\pgfqpoint{4.661664in}{2.737284in}}%
\pgfpathlineto{\pgfqpoint{4.675191in}{2.737219in}}%
\pgfpathlineto{\pgfqpoint{4.688727in}{2.737260in}}%
\pgfpathlineto{\pgfqpoint{4.681223in}{2.729007in}}%
\pgfpathlineto{\pgfqpoint{4.673714in}{2.720738in}}%
\pgfpathlineto{\pgfqpoint{4.666200in}{2.712453in}}%
\pgfpathlineto{\pgfqpoint{4.658680in}{2.704150in}}%
\pgfpathlineto{\pgfqpoint{4.645133in}{2.704031in}}%
\pgfpathlineto{\pgfqpoint{4.631595in}{2.704018in}}%
\pgfpathlineto{\pgfqpoint{4.618066in}{2.704112in}}%
\pgfpathlineto{\pgfqpoint{4.604545in}{2.704312in}}%
\pgfpathlineto{\pgfqpoint{4.612076in}{2.712686in}}%
\pgfpathlineto{\pgfqpoint{4.619602in}{2.721046in}}%
\pgfpathlineto{\pgfqpoint{4.627122in}{2.729395in}}%
\pgfpathlineto{\pgfqpoint{4.634637in}{2.737733in}}%
\pgfpathclose%
\pgfusepath{fill}%
\end{pgfscope}%
\begin{pgfscope}%
\pgfpathrectangle{\pgfqpoint{1.150000in}{0.150000in}}{\pgfqpoint{5.700000in}{5.700000in}}%
\pgfusepath{clip}%
\pgfsetbuttcap%
\pgfsetroundjoin%
\definecolor{currentfill}{rgb}{0.282910,0.105393,0.426902}%
\pgfsetfillcolor{currentfill}%
\pgfsetfillopacity{0.700000}%
\pgfsetlinewidth{0.000000pt}%
\definecolor{currentstroke}{rgb}{0.000000,0.000000,0.000000}%
\pgfsetstrokecolor{currentstroke}%
\pgfsetdash{}{0pt}%
\pgfpathmoveto{\pgfqpoint{3.334421in}{2.690967in}}%
\pgfpathlineto{\pgfqpoint{3.347690in}{2.681547in}}%
\pgfpathlineto{\pgfqpoint{3.360959in}{2.672280in}}%
\pgfpathlineto{\pgfqpoint{3.374229in}{2.663164in}}%
\pgfpathlineto{\pgfqpoint{3.387500in}{2.654198in}}%
\pgfpathlineto{\pgfqpoint{3.379533in}{2.646859in}}%
\pgfpathlineto{\pgfqpoint{3.371559in}{2.639580in}}%
\pgfpathlineto{\pgfqpoint{3.363579in}{2.632360in}}%
\pgfpathlineto{\pgfqpoint{3.355592in}{2.625200in}}%
\pgfpathlineto{\pgfqpoint{3.342303in}{2.634290in}}%
\pgfpathlineto{\pgfqpoint{3.329016in}{2.643530in}}%
\pgfpathlineto{\pgfqpoint{3.315730in}{2.652921in}}%
\pgfpathlineto{\pgfqpoint{3.302444in}{2.662465in}}%
\pgfpathlineto{\pgfqpoint{3.310449in}{2.669493in}}%
\pgfpathlineto{\pgfqpoint{3.318446in}{2.676587in}}%
\pgfpathlineto{\pgfqpoint{3.326437in}{2.683745in}}%
\pgfpathlineto{\pgfqpoint{3.334421in}{2.690967in}}%
\pgfpathclose%
\pgfusepath{fill}%
\end{pgfscope}%
\begin{pgfscope}%
\pgfpathrectangle{\pgfqpoint{1.150000in}{0.150000in}}{\pgfqpoint{5.700000in}{5.700000in}}%
\pgfusepath{clip}%
\pgfsetbuttcap%
\pgfsetroundjoin%
\definecolor{currentfill}{rgb}{0.280894,0.078907,0.402329}%
\pgfsetfillcolor{currentfill}%
\pgfsetfillopacity{0.700000}%
\pgfsetlinewidth{0.000000pt}%
\definecolor{currentstroke}{rgb}{0.000000,0.000000,0.000000}%
\pgfsetstrokecolor{currentstroke}%
\pgfsetdash{}{0pt}%
\pgfpathmoveto{\pgfqpoint{4.244453in}{2.626924in}}%
\pgfpathlineto{\pgfqpoint{4.257846in}{2.624641in}}%
\pgfpathlineto{\pgfqpoint{4.271246in}{2.622472in}}%
\pgfpathlineto{\pgfqpoint{4.284653in}{2.620417in}}%
\pgfpathlineto{\pgfqpoint{4.298068in}{2.618475in}}%
\pgfpathlineto{\pgfqpoint{4.290426in}{2.609805in}}%
\pgfpathlineto{\pgfqpoint{4.282779in}{2.601128in}}%
\pgfpathlineto{\pgfqpoint{4.275127in}{2.592446in}}%
\pgfpathlineto{\pgfqpoint{4.267470in}{2.583757in}}%
\pgfpathlineto{\pgfqpoint{4.254045in}{2.585694in}}%
\pgfpathlineto{\pgfqpoint{4.240627in}{2.587744in}}%
\pgfpathlineto{\pgfqpoint{4.227216in}{2.589908in}}%
\pgfpathlineto{\pgfqpoint{4.213813in}{2.592186in}}%
\pgfpathlineto{\pgfqpoint{4.221480in}{2.600873in}}%
\pgfpathlineto{\pgfqpoint{4.229143in}{2.609558in}}%
\pgfpathlineto{\pgfqpoint{4.236800in}{2.618242in}}%
\pgfpathlineto{\pgfqpoint{4.244453in}{2.626924in}}%
\pgfpathclose%
\pgfusepath{fill}%
\end{pgfscope}%
\begin{pgfscope}%
\pgfpathrectangle{\pgfqpoint{1.150000in}{0.150000in}}{\pgfqpoint{5.700000in}{5.700000in}}%
\pgfusepath{clip}%
\pgfsetbuttcap%
\pgfsetroundjoin%
\definecolor{currentfill}{rgb}{0.278012,0.180367,0.486697}%
\pgfsetfillcolor{currentfill}%
\pgfsetfillopacity{0.700000}%
\pgfsetlinewidth{0.000000pt}%
\definecolor{currentstroke}{rgb}{0.000000,0.000000,0.000000}%
\pgfsetstrokecolor{currentstroke}%
\pgfsetdash{}{0pt}%
\pgfpathmoveto{\pgfqpoint{3.089865in}{2.836875in}}%
\pgfpathlineto{\pgfqpoint{3.103157in}{2.824725in}}%
\pgfpathlineto{\pgfqpoint{3.116448in}{2.812748in}}%
\pgfpathlineto{\pgfqpoint{3.129738in}{2.800942in}}%
\pgfpathlineto{\pgfqpoint{3.143026in}{2.789307in}}%
\pgfpathlineto{\pgfqpoint{3.134957in}{2.782755in}}%
\pgfpathlineto{\pgfqpoint{3.126881in}{2.776283in}}%
\pgfpathlineto{\pgfqpoint{3.118796in}{2.769894in}}%
\pgfpathlineto{\pgfqpoint{3.110704in}{2.763588in}}%
\pgfpathlineto{\pgfqpoint{3.097396in}{2.775368in}}%
\pgfpathlineto{\pgfqpoint{3.084086in}{2.787318in}}%
\pgfpathlineto{\pgfqpoint{3.070774in}{2.799440in}}%
\pgfpathlineto{\pgfqpoint{3.057461in}{2.811735in}}%
\pgfpathlineto{\pgfqpoint{3.065574in}{2.817889in}}%
\pgfpathlineto{\pgfqpoint{3.073679in}{2.824131in}}%
\pgfpathlineto{\pgfqpoint{3.081776in}{2.830460in}}%
\pgfpathlineto{\pgfqpoint{3.089865in}{2.836875in}}%
\pgfpathclose%
\pgfusepath{fill}%
\end{pgfscope}%
\begin{pgfscope}%
\pgfpathrectangle{\pgfqpoint{1.150000in}{0.150000in}}{\pgfqpoint{5.700000in}{5.700000in}}%
\pgfusepath{clip}%
\pgfsetbuttcap%
\pgfsetroundjoin%
\definecolor{currentfill}{rgb}{0.277941,0.056324,0.381191}%
\pgfsetfillcolor{currentfill}%
\pgfsetfillopacity{0.700000}%
\pgfsetlinewidth{0.000000pt}%
\definecolor{currentstroke}{rgb}{0.000000,0.000000,0.000000}%
\pgfsetstrokecolor{currentstroke}%
\pgfsetdash{}{0pt}%
\pgfpathmoveto{\pgfqpoint{3.663117in}{2.594132in}}%
\pgfpathlineto{\pgfqpoint{3.676399in}{2.587757in}}%
\pgfpathlineto{\pgfqpoint{3.689684in}{2.581516in}}%
\pgfpathlineto{\pgfqpoint{3.702973in}{2.575407in}}%
\pgfpathlineto{\pgfqpoint{3.716265in}{2.569430in}}%
\pgfpathlineto{\pgfqpoint{3.708423in}{2.561248in}}%
\pgfpathlineto{\pgfqpoint{3.700575in}{2.553097in}}%
\pgfpathlineto{\pgfqpoint{3.692721in}{2.544977in}}%
\pgfpathlineto{\pgfqpoint{3.684861in}{2.536889in}}%
\pgfpathlineto{\pgfqpoint{3.671555in}{2.542952in}}%
\pgfpathlineto{\pgfqpoint{3.658252in}{2.549147in}}%
\pgfpathlineto{\pgfqpoint{3.644953in}{2.555474in}}%
\pgfpathlineto{\pgfqpoint{3.631657in}{2.561935in}}%
\pgfpathlineto{\pgfqpoint{3.639531in}{2.569930in}}%
\pgfpathlineto{\pgfqpoint{3.647399in}{2.577962in}}%
\pgfpathlineto{\pgfqpoint{3.655261in}{2.586029in}}%
\pgfpathlineto{\pgfqpoint{3.663117in}{2.594132in}}%
\pgfpathclose%
\pgfusepath{fill}%
\end{pgfscope}%
\begin{pgfscope}%
\pgfpathrectangle{\pgfqpoint{1.150000in}{0.150000in}}{\pgfqpoint{5.700000in}{5.700000in}}%
\pgfusepath{clip}%
\pgfsetbuttcap%
\pgfsetroundjoin%
\definecolor{currentfill}{rgb}{0.260571,0.246922,0.522828}%
\pgfsetfillcolor{currentfill}%
\pgfsetfillopacity{0.700000}%
\pgfsetlinewidth{0.000000pt}%
\definecolor{currentstroke}{rgb}{0.000000,0.000000,0.000000}%
\pgfsetstrokecolor{currentstroke}%
\pgfsetdash{}{0pt}%
\pgfpathmoveto{\pgfqpoint{5.193251in}{2.946442in}}%
\pgfpathlineto{\pgfqpoint{5.206961in}{2.948204in}}%
\pgfpathlineto{\pgfqpoint{5.220683in}{2.950066in}}%
\pgfpathlineto{\pgfqpoint{5.234415in}{2.952028in}}%
\pgfpathlineto{\pgfqpoint{5.248159in}{2.954090in}}%
\pgfpathlineto{\pgfqpoint{5.240875in}{2.947082in}}%
\pgfpathlineto{\pgfqpoint{5.233585in}{2.940071in}}%
\pgfpathlineto{\pgfqpoint{5.226290in}{2.933052in}}%
\pgfpathlineto{\pgfqpoint{5.218989in}{2.926025in}}%
\pgfpathlineto{\pgfqpoint{5.205231in}{2.923775in}}%
\pgfpathlineto{\pgfqpoint{5.191483in}{2.921626in}}%
\pgfpathlineto{\pgfqpoint{5.177747in}{2.919576in}}%
\pgfpathlineto{\pgfqpoint{5.164022in}{2.917627in}}%
\pgfpathlineto{\pgfqpoint{5.171337in}{2.924835in}}%
\pgfpathlineto{\pgfqpoint{5.178647in}{2.932039in}}%
\pgfpathlineto{\pgfqpoint{5.185951in}{2.939241in}}%
\pgfpathlineto{\pgfqpoint{5.193251in}{2.946442in}}%
\pgfpathclose%
\pgfusepath{fill}%
\end{pgfscope}%
\begin{pgfscope}%
\pgfpathrectangle{\pgfqpoint{1.150000in}{0.150000in}}{\pgfqpoint{5.700000in}{5.700000in}}%
\pgfusepath{clip}%
\pgfsetbuttcap%
\pgfsetroundjoin%
\definecolor{currentfill}{rgb}{0.277018,0.050344,0.375715}%
\pgfsetfillcolor{currentfill}%
\pgfsetfillopacity{0.700000}%
\pgfsetlinewidth{0.000000pt}%
\definecolor{currentstroke}{rgb}{0.000000,0.000000,0.000000}%
\pgfsetstrokecolor{currentstroke}%
\pgfsetdash{}{0pt}%
\pgfpathmoveto{\pgfqpoint{3.800736in}{2.580121in}}%
\pgfpathlineto{\pgfqpoint{3.814037in}{2.574859in}}%
\pgfpathlineto{\pgfqpoint{3.827341in}{2.569724in}}%
\pgfpathlineto{\pgfqpoint{3.840651in}{2.564716in}}%
\pgfpathlineto{\pgfqpoint{3.853965in}{2.559833in}}%
\pgfpathlineto{\pgfqpoint{3.846171in}{2.551410in}}%
\pgfpathlineto{\pgfqpoint{3.838371in}{2.543007in}}%
\pgfpathlineto{\pgfqpoint{3.830566in}{2.534625in}}%
\pgfpathlineto{\pgfqpoint{3.822755in}{2.526264in}}%
\pgfpathlineto{\pgfqpoint{3.809429in}{2.531214in}}%
\pgfpathlineto{\pgfqpoint{3.796107in}{2.536290in}}%
\pgfpathlineto{\pgfqpoint{3.782789in}{2.541493in}}%
\pgfpathlineto{\pgfqpoint{3.769476in}{2.546823in}}%
\pgfpathlineto{\pgfqpoint{3.777299in}{2.555109in}}%
\pgfpathlineto{\pgfqpoint{3.785117in}{2.563422in}}%
\pgfpathlineto{\pgfqpoint{3.792930in}{2.571759in}}%
\pgfpathlineto{\pgfqpoint{3.800736in}{2.580121in}}%
\pgfpathclose%
\pgfusepath{fill}%
\end{pgfscope}%
\begin{pgfscope}%
\pgfpathrectangle{\pgfqpoint{1.150000in}{0.150000in}}{\pgfqpoint{5.700000in}{5.700000in}}%
\pgfusepath{clip}%
\pgfsetbuttcap%
\pgfsetroundjoin%
\definecolor{currentfill}{rgb}{0.280267,0.073417,0.397163}%
\pgfsetfillcolor{currentfill}%
\pgfsetfillopacity{0.700000}%
\pgfsetlinewidth{0.000000pt}%
\definecolor{currentstroke}{rgb}{0.000000,0.000000,0.000000}%
\pgfsetstrokecolor{currentstroke}%
\pgfsetdash{}{0pt}%
\pgfpathmoveto{\pgfqpoint{3.525402in}{2.618517in}}%
\pgfpathlineto{\pgfqpoint{3.538675in}{2.610960in}}%
\pgfpathlineto{\pgfqpoint{3.551949in}{2.603543in}}%
\pgfpathlineto{\pgfqpoint{3.565227in}{2.596266in}}%
\pgfpathlineto{\pgfqpoint{3.578507in}{2.589126in}}%
\pgfpathlineto{\pgfqpoint{3.570613in}{2.581266in}}%
\pgfpathlineto{\pgfqpoint{3.562712in}{2.573447in}}%
\pgfpathlineto{\pgfqpoint{3.554806in}{2.565672in}}%
\pgfpathlineto{\pgfqpoint{3.546893in}{2.557940in}}%
\pgfpathlineto{\pgfqpoint{3.533598in}{2.565183in}}%
\pgfpathlineto{\pgfqpoint{3.520306in}{2.572565in}}%
\pgfpathlineto{\pgfqpoint{3.507016in}{2.580087in}}%
\pgfpathlineto{\pgfqpoint{3.493728in}{2.587748in}}%
\pgfpathlineto{\pgfqpoint{3.501656in}{2.595369in}}%
\pgfpathlineto{\pgfqpoint{3.509578in}{2.603037in}}%
\pgfpathlineto{\pgfqpoint{3.517493in}{2.610754in}}%
\pgfpathlineto{\pgfqpoint{3.525402in}{2.618517in}}%
\pgfpathclose%
\pgfusepath{fill}%
\end{pgfscope}%
\begin{pgfscope}%
\pgfpathrectangle{\pgfqpoint{1.150000in}{0.150000in}}{\pgfqpoint{5.700000in}{5.700000in}}%
\pgfusepath{clip}%
\pgfsetbuttcap%
\pgfsetroundjoin%
\definecolor{currentfill}{rgb}{0.283187,0.125848,0.444960}%
\pgfsetfillcolor{currentfill}%
\pgfsetfillopacity{0.700000}%
\pgfsetlinewidth{0.000000pt}%
\definecolor{currentstroke}{rgb}{0.000000,0.000000,0.000000}%
\pgfsetstrokecolor{currentstroke}%
\pgfsetdash{}{0pt}%
\pgfpathmoveto{\pgfqpoint{4.550552in}{2.706187in}}%
\pgfpathlineto{\pgfqpoint{4.564037in}{2.705557in}}%
\pgfpathlineto{\pgfqpoint{4.577531in}{2.705035in}}%
\pgfpathlineto{\pgfqpoint{4.591034in}{2.704620in}}%
\pgfpathlineto{\pgfqpoint{4.604545in}{2.704312in}}%
\pgfpathlineto{\pgfqpoint{4.597009in}{2.695925in}}%
\pgfpathlineto{\pgfqpoint{4.589468in}{2.687522in}}%
\pgfpathlineto{\pgfqpoint{4.581921in}{2.679103in}}%
\pgfpathlineto{\pgfqpoint{4.574369in}{2.670668in}}%
\pgfpathlineto{\pgfqpoint{4.560847in}{2.670916in}}%
\pgfpathlineto{\pgfqpoint{4.547334in}{2.671271in}}%
\pgfpathlineto{\pgfqpoint{4.533829in}{2.671734in}}%
\pgfpathlineto{\pgfqpoint{4.520333in}{2.672305in}}%
\pgfpathlineto{\pgfqpoint{4.527895in}{2.680793in}}%
\pgfpathlineto{\pgfqpoint{4.535453in}{2.689269in}}%
\pgfpathlineto{\pgfqpoint{4.543005in}{2.697733in}}%
\pgfpathlineto{\pgfqpoint{4.550552in}{2.706187in}}%
\pgfpathclose%
\pgfusepath{fill}%
\end{pgfscope}%
\begin{pgfscope}%
\pgfpathrectangle{\pgfqpoint{1.150000in}{0.150000in}}{\pgfqpoint{5.700000in}{5.700000in}}%
\pgfusepath{clip}%
\pgfsetbuttcap%
\pgfsetroundjoin%
\definecolor{currentfill}{rgb}{0.266580,0.228262,0.514349}%
\pgfsetfillcolor{currentfill}%
\pgfsetfillopacity{0.700000}%
\pgfsetlinewidth{0.000000pt}%
\definecolor{currentstroke}{rgb}{0.000000,0.000000,0.000000}%
\pgfsetstrokecolor{currentstroke}%
\pgfsetdash{}{0pt}%
\pgfpathmoveto{\pgfqpoint{5.109230in}{2.910832in}}%
\pgfpathlineto{\pgfqpoint{5.122912in}{2.912380in}}%
\pgfpathlineto{\pgfqpoint{5.136604in}{2.914028in}}%
\pgfpathlineto{\pgfqpoint{5.150308in}{2.915777in}}%
\pgfpathlineto{\pgfqpoint{5.164022in}{2.917627in}}%
\pgfpathlineto{\pgfqpoint{5.156701in}{2.910411in}}%
\pgfpathlineto{\pgfqpoint{5.149375in}{2.903187in}}%
\pgfpathlineto{\pgfqpoint{5.142044in}{2.895951in}}%
\pgfpathlineto{\pgfqpoint{5.134707in}{2.888703in}}%
\pgfpathlineto{\pgfqpoint{5.120978in}{2.886684in}}%
\pgfpathlineto{\pgfqpoint{5.107261in}{2.884766in}}%
\pgfpathlineto{\pgfqpoint{5.093554in}{2.882948in}}%
\pgfpathlineto{\pgfqpoint{5.079859in}{2.881232in}}%
\pgfpathlineto{\pgfqpoint{5.087209in}{2.888643in}}%
\pgfpathlineto{\pgfqpoint{5.094555in}{2.896045in}}%
\pgfpathlineto{\pgfqpoint{5.101895in}{2.903441in}}%
\pgfpathlineto{\pgfqpoint{5.109230in}{2.910832in}}%
\pgfpathclose%
\pgfusepath{fill}%
\end{pgfscope}%
\begin{pgfscope}%
\pgfpathrectangle{\pgfqpoint{1.150000in}{0.150000in}}{\pgfqpoint{5.700000in}{5.700000in}}%
\pgfusepath{clip}%
\pgfsetbuttcap%
\pgfsetroundjoin%
\definecolor{currentfill}{rgb}{0.280868,0.160771,0.472899}%
\pgfsetfillcolor{currentfill}%
\pgfsetfillopacity{0.700000}%
\pgfsetlinewidth{0.000000pt}%
\definecolor{currentstroke}{rgb}{0.000000,0.000000,0.000000}%
\pgfsetstrokecolor{currentstroke}%
\pgfsetdash{}{0pt}%
\pgfpathmoveto{\pgfqpoint{3.143026in}{2.789307in}}%
\pgfpathlineto{\pgfqpoint{3.156313in}{2.777841in}}%
\pgfpathlineto{\pgfqpoint{3.169599in}{2.766541in}}%
\pgfpathlineto{\pgfqpoint{3.182885in}{2.755408in}}%
\pgfpathlineto{\pgfqpoint{3.196170in}{2.744440in}}%
\pgfpathlineto{\pgfqpoint{3.188121in}{2.737750in}}%
\pgfpathlineto{\pgfqpoint{3.180064in}{2.731137in}}%
\pgfpathlineto{\pgfqpoint{3.172000in}{2.724601in}}%
\pgfpathlineto{\pgfqpoint{3.163928in}{2.718145in}}%
\pgfpathlineto{\pgfqpoint{3.150623in}{2.729257in}}%
\pgfpathlineto{\pgfqpoint{3.137318in}{2.740534in}}%
\pgfpathlineto{\pgfqpoint{3.124012in}{2.751977in}}%
\pgfpathlineto{\pgfqpoint{3.110704in}{2.763588in}}%
\pgfpathlineto{\pgfqpoint{3.118796in}{2.769894in}}%
\pgfpathlineto{\pgfqpoint{3.126881in}{2.776283in}}%
\pgfpathlineto{\pgfqpoint{3.134957in}{2.782755in}}%
\pgfpathlineto{\pgfqpoint{3.143026in}{2.789307in}}%
\pgfpathclose%
\pgfusepath{fill}%
\end{pgfscope}%
\begin{pgfscope}%
\pgfpathrectangle{\pgfqpoint{1.150000in}{0.150000in}}{\pgfqpoint{5.700000in}{5.700000in}}%
\pgfusepath{clip}%
\pgfsetbuttcap%
\pgfsetroundjoin%
\definecolor{currentfill}{rgb}{0.277018,0.050344,0.375715}%
\pgfsetfillcolor{currentfill}%
\pgfsetfillopacity{0.700000}%
\pgfsetlinewidth{0.000000pt}%
\definecolor{currentstroke}{rgb}{0.000000,0.000000,0.000000}%
\pgfsetstrokecolor{currentstroke}%
\pgfsetdash{}{0pt}%
\pgfpathmoveto{\pgfqpoint{3.938344in}{2.575637in}}%
\pgfpathlineto{\pgfqpoint{3.951671in}{2.571424in}}%
\pgfpathlineto{\pgfqpoint{3.965004in}{2.567334in}}%
\pgfpathlineto{\pgfqpoint{3.978342in}{2.563365in}}%
\pgfpathlineto{\pgfqpoint{3.991686in}{2.559517in}}%
\pgfpathlineto{\pgfqpoint{3.983937in}{2.550930in}}%
\pgfpathlineto{\pgfqpoint{3.976183in}{2.542354in}}%
\pgfpathlineto{\pgfqpoint{3.968424in}{2.533788in}}%
\pgfpathlineto{\pgfqpoint{3.960659in}{2.525233in}}%
\pgfpathlineto{\pgfqpoint{3.947304in}{2.529130in}}%
\pgfpathlineto{\pgfqpoint{3.933954in}{2.533148in}}%
\pgfpathlineto{\pgfqpoint{3.920610in}{2.537288in}}%
\pgfpathlineto{\pgfqpoint{3.907270in}{2.541550in}}%
\pgfpathlineto{\pgfqpoint{3.915047in}{2.550049in}}%
\pgfpathlineto{\pgfqpoint{3.922818in}{2.558564in}}%
\pgfpathlineto{\pgfqpoint{3.930583in}{2.567093in}}%
\pgfpathlineto{\pgfqpoint{3.938344in}{2.575637in}}%
\pgfpathclose%
\pgfusepath{fill}%
\end{pgfscope}%
\begin{pgfscope}%
\pgfpathrectangle{\pgfqpoint{1.150000in}{0.150000in}}{\pgfqpoint{5.700000in}{5.700000in}}%
\pgfusepath{clip}%
\pgfsetbuttcap%
\pgfsetroundjoin%
\definecolor{currentfill}{rgb}{0.279566,0.067836,0.391917}%
\pgfsetfillcolor{currentfill}%
\pgfsetfillopacity{0.700000}%
\pgfsetlinewidth{0.000000pt}%
\definecolor{currentstroke}{rgb}{0.000000,0.000000,0.000000}%
\pgfsetstrokecolor{currentstroke}%
\pgfsetdash{}{0pt}%
\pgfpathmoveto{\pgfqpoint{4.160266in}{2.602446in}}%
\pgfpathlineto{\pgfqpoint{4.173643in}{2.599708in}}%
\pgfpathlineto{\pgfqpoint{4.187026in}{2.597086in}}%
\pgfpathlineto{\pgfqpoint{4.200416in}{2.594578in}}%
\pgfpathlineto{\pgfqpoint{4.213813in}{2.592186in}}%
\pgfpathlineto{\pgfqpoint{4.206140in}{2.583497in}}%
\pgfpathlineto{\pgfqpoint{4.198461in}{2.574805in}}%
\pgfpathlineto{\pgfqpoint{4.190778in}{2.566110in}}%
\pgfpathlineto{\pgfqpoint{4.183089in}{2.557412in}}%
\pgfpathlineto{\pgfqpoint{4.169681in}{2.559818in}}%
\pgfpathlineto{\pgfqpoint{4.156281in}{2.562338in}}%
\pgfpathlineto{\pgfqpoint{4.142887in}{2.564973in}}%
\pgfpathlineto{\pgfqpoint{4.129499in}{2.567725in}}%
\pgfpathlineto{\pgfqpoint{4.137199in}{2.576403in}}%
\pgfpathlineto{\pgfqpoint{4.144893in}{2.585082in}}%
\pgfpathlineto{\pgfqpoint{4.152582in}{2.593763in}}%
\pgfpathlineto{\pgfqpoint{4.160266in}{2.602446in}}%
\pgfpathclose%
\pgfusepath{fill}%
\end{pgfscope}%
\begin{pgfscope}%
\pgfpathrectangle{\pgfqpoint{1.150000in}{0.150000in}}{\pgfqpoint{5.700000in}{5.700000in}}%
\pgfusepath{clip}%
\pgfsetbuttcap%
\pgfsetroundjoin%
\definecolor{currentfill}{rgb}{0.270595,0.214069,0.507052}%
\pgfsetfillcolor{currentfill}%
\pgfsetfillopacity{0.700000}%
\pgfsetlinewidth{0.000000pt}%
\definecolor{currentstroke}{rgb}{0.000000,0.000000,0.000000}%
\pgfsetstrokecolor{currentstroke}%
\pgfsetdash{}{0pt}%
\pgfpathmoveto{\pgfqpoint{5.025182in}{2.875377in}}%
\pgfpathlineto{\pgfqpoint{5.038835in}{2.876688in}}%
\pgfpathlineto{\pgfqpoint{5.052499in}{2.878102in}}%
\pgfpathlineto{\pgfqpoint{5.066173in}{2.879616in}}%
\pgfpathlineto{\pgfqpoint{5.079859in}{2.881232in}}%
\pgfpathlineto{\pgfqpoint{5.072502in}{2.873810in}}%
\pgfpathlineto{\pgfqpoint{5.065140in}{2.866375in}}%
\pgfpathlineto{\pgfqpoint{5.057773in}{2.858926in}}%
\pgfpathlineto{\pgfqpoint{5.050400in}{2.851461in}}%
\pgfpathlineto{\pgfqpoint{5.036702in}{2.849694in}}%
\pgfpathlineto{\pgfqpoint{5.023014in}{2.848029in}}%
\pgfpathlineto{\pgfqpoint{5.009337in}{2.846465in}}%
\pgfpathlineto{\pgfqpoint{4.995670in}{2.845002in}}%
\pgfpathlineto{\pgfqpoint{5.003056in}{2.852612in}}%
\pgfpathlineto{\pgfqpoint{5.010437in}{2.860210in}}%
\pgfpathlineto{\pgfqpoint{5.017812in}{2.867797in}}%
\pgfpathlineto{\pgfqpoint{5.025182in}{2.875377in}}%
\pgfpathclose%
\pgfusepath{fill}%
\end{pgfscope}%
\begin{pgfscope}%
\pgfpathrectangle{\pgfqpoint{1.150000in}{0.150000in}}{\pgfqpoint{5.700000in}{5.700000in}}%
\pgfusepath{clip}%
\pgfsetbuttcap%
\pgfsetroundjoin%
\definecolor{currentfill}{rgb}{0.283091,0.110553,0.431554}%
\pgfsetfillcolor{currentfill}%
\pgfsetfillopacity{0.700000}%
\pgfsetlinewidth{0.000000pt}%
\definecolor{currentstroke}{rgb}{0.000000,0.000000,0.000000}%
\pgfsetstrokecolor{currentstroke}%
\pgfsetdash{}{0pt}%
\pgfpathmoveto{\pgfqpoint{4.466432in}{2.675674in}}%
\pgfpathlineto{\pgfqpoint{4.479894in}{2.674668in}}%
\pgfpathlineto{\pgfqpoint{4.493366in}{2.673772in}}%
\pgfpathlineto{\pgfqpoint{4.506845in}{2.672984in}}%
\pgfpathlineto{\pgfqpoint{4.520333in}{2.672305in}}%
\pgfpathlineto{\pgfqpoint{4.512765in}{2.663803in}}%
\pgfpathlineto{\pgfqpoint{4.505191in}{2.655287in}}%
\pgfpathlineto{\pgfqpoint{4.497613in}{2.646756in}}%
\pgfpathlineto{\pgfqpoint{4.490029in}{2.638210in}}%
\pgfpathlineto{\pgfqpoint{4.476531in}{2.638847in}}%
\pgfpathlineto{\pgfqpoint{4.463041in}{2.639594in}}%
\pgfpathlineto{\pgfqpoint{4.449559in}{2.640449in}}%
\pgfpathlineto{\pgfqpoint{4.436086in}{2.641414in}}%
\pgfpathlineto{\pgfqpoint{4.443680in}{2.649995in}}%
\pgfpathlineto{\pgfqpoint{4.451269in}{2.658565in}}%
\pgfpathlineto{\pgfqpoint{4.458853in}{2.667124in}}%
\pgfpathlineto{\pgfqpoint{4.466432in}{2.675674in}}%
\pgfpathclose%
\pgfusepath{fill}%
\end{pgfscope}%
\begin{pgfscope}%
\pgfpathrectangle{\pgfqpoint{1.150000in}{0.150000in}}{\pgfqpoint{5.700000in}{5.700000in}}%
\pgfusepath{clip}%
\pgfsetbuttcap%
\pgfsetroundjoin%
\definecolor{currentfill}{rgb}{0.282327,0.094955,0.417331}%
\pgfsetfillcolor{currentfill}%
\pgfsetfillopacity{0.700000}%
\pgfsetlinewidth{0.000000pt}%
\definecolor{currentstroke}{rgb}{0.000000,0.000000,0.000000}%
\pgfsetstrokecolor{currentstroke}%
\pgfsetdash{}{0pt}%
\pgfpathmoveto{\pgfqpoint{3.387500in}{2.654198in}}%
\pgfpathlineto{\pgfqpoint{3.400773in}{2.645381in}}%
\pgfpathlineto{\pgfqpoint{3.414047in}{2.636712in}}%
\pgfpathlineto{\pgfqpoint{3.427322in}{2.628190in}}%
\pgfpathlineto{\pgfqpoint{3.440600in}{2.619814in}}%
\pgfpathlineto{\pgfqpoint{3.432649in}{2.612358in}}%
\pgfpathlineto{\pgfqpoint{3.424692in}{2.604957in}}%
\pgfpathlineto{\pgfqpoint{3.416729in}{2.597612in}}%
\pgfpathlineto{\pgfqpoint{3.408759in}{2.590323in}}%
\pgfpathlineto{\pgfqpoint{3.395465in}{2.598822in}}%
\pgfpathlineto{\pgfqpoint{3.382172in}{2.607467in}}%
\pgfpathlineto{\pgfqpoint{3.368881in}{2.616260in}}%
\pgfpathlineto{\pgfqpoint{3.355592in}{2.625200in}}%
\pgfpathlineto{\pgfqpoint{3.363579in}{2.632360in}}%
\pgfpathlineto{\pgfqpoint{3.371559in}{2.639580in}}%
\pgfpathlineto{\pgfqpoint{3.379533in}{2.646859in}}%
\pgfpathlineto{\pgfqpoint{3.387500in}{2.654198in}}%
\pgfpathclose%
\pgfusepath{fill}%
\end{pgfscope}%
\begin{pgfscope}%
\pgfpathrectangle{\pgfqpoint{1.150000in}{0.150000in}}{\pgfqpoint{5.700000in}{5.700000in}}%
\pgfusepath{clip}%
\pgfsetbuttcap%
\pgfsetroundjoin%
\definecolor{currentfill}{rgb}{0.275191,0.194905,0.496005}%
\pgfsetfillcolor{currentfill}%
\pgfsetfillopacity{0.700000}%
\pgfsetlinewidth{0.000000pt}%
\definecolor{currentstroke}{rgb}{0.000000,0.000000,0.000000}%
\pgfsetstrokecolor{currentstroke}%
\pgfsetdash{}{0pt}%
\pgfpathmoveto{\pgfqpoint{4.941107in}{2.840174in}}%
\pgfpathlineto{\pgfqpoint{4.954733in}{2.841228in}}%
\pgfpathlineto{\pgfqpoint{4.968368in}{2.842384in}}%
\pgfpathlineto{\pgfqpoint{4.982014in}{2.843642in}}%
\pgfpathlineto{\pgfqpoint{4.995670in}{2.845002in}}%
\pgfpathlineto{\pgfqpoint{4.988279in}{2.837379in}}%
\pgfpathlineto{\pgfqpoint{4.980882in}{2.829740in}}%
\pgfpathlineto{\pgfqpoint{4.973480in}{2.822084in}}%
\pgfpathlineto{\pgfqpoint{4.966072in}{2.814409in}}%
\pgfpathlineto{\pgfqpoint{4.952403in}{2.812915in}}%
\pgfpathlineto{\pgfqpoint{4.938745in}{2.811524in}}%
\pgfpathlineto{\pgfqpoint{4.925097in}{2.810236in}}%
\pgfpathlineto{\pgfqpoint{4.911459in}{2.809050in}}%
\pgfpathlineto{\pgfqpoint{4.918879in}{2.816851in}}%
\pgfpathlineto{\pgfqpoint{4.926294in}{2.824637in}}%
\pgfpathlineto{\pgfqpoint{4.933703in}{2.832411in}}%
\pgfpathlineto{\pgfqpoint{4.941107in}{2.840174in}}%
\pgfpathclose%
\pgfusepath{fill}%
\end{pgfscope}%
\begin{pgfscope}%
\pgfpathrectangle{\pgfqpoint{1.150000in}{0.150000in}}{\pgfqpoint{5.700000in}{5.700000in}}%
\pgfusepath{clip}%
\pgfsetbuttcap%
\pgfsetroundjoin%
\definecolor{currentfill}{rgb}{0.282623,0.140926,0.457517}%
\pgfsetfillcolor{currentfill}%
\pgfsetfillopacity{0.700000}%
\pgfsetlinewidth{0.000000pt}%
\definecolor{currentstroke}{rgb}{0.000000,0.000000,0.000000}%
\pgfsetstrokecolor{currentstroke}%
\pgfsetdash{}{0pt}%
\pgfpathmoveto{\pgfqpoint{3.196170in}{2.744440in}}%
\pgfpathlineto{\pgfqpoint{3.209454in}{2.733635in}}%
\pgfpathlineto{\pgfqpoint{3.222738in}{2.722991in}}%
\pgfpathlineto{\pgfqpoint{3.236022in}{2.712509in}}%
\pgfpathlineto{\pgfqpoint{3.249306in}{2.702186in}}%
\pgfpathlineto{\pgfqpoint{3.241276in}{2.695360in}}%
\pgfpathlineto{\pgfqpoint{3.233238in}{2.688605in}}%
\pgfpathlineto{\pgfqpoint{3.225194in}{2.681924in}}%
\pgfpathlineto{\pgfqpoint{3.217141in}{2.675318in}}%
\pgfpathlineto{\pgfqpoint{3.203838in}{2.685784in}}%
\pgfpathlineto{\pgfqpoint{3.190535in}{2.696409in}}%
\pgfpathlineto{\pgfqpoint{3.177232in}{2.707196in}}%
\pgfpathlineto{\pgfqpoint{3.163928in}{2.718145in}}%
\pgfpathlineto{\pgfqpoint{3.172000in}{2.724601in}}%
\pgfpathlineto{\pgfqpoint{3.180064in}{2.731137in}}%
\pgfpathlineto{\pgfqpoint{3.188121in}{2.737750in}}%
\pgfpathlineto{\pgfqpoint{3.196170in}{2.744440in}}%
\pgfpathclose%
\pgfusepath{fill}%
\end{pgfscope}%
\begin{pgfscope}%
\pgfpathrectangle{\pgfqpoint{1.150000in}{0.150000in}}{\pgfqpoint{5.700000in}{5.700000in}}%
\pgfusepath{clip}%
\pgfsetbuttcap%
\pgfsetroundjoin%
\definecolor{currentfill}{rgb}{0.282327,0.094955,0.417331}%
\pgfsetfillcolor{currentfill}%
\pgfsetfillopacity{0.700000}%
\pgfsetlinewidth{0.000000pt}%
\definecolor{currentstroke}{rgb}{0.000000,0.000000,0.000000}%
\pgfsetstrokecolor{currentstroke}%
\pgfsetdash{}{0pt}%
\pgfpathmoveto{\pgfqpoint{4.382272in}{2.646373in}}%
\pgfpathlineto{\pgfqpoint{4.395714in}{2.644968in}}%
\pgfpathlineto{\pgfqpoint{4.409163in}{2.643673in}}%
\pgfpathlineto{\pgfqpoint{4.422620in}{2.642488in}}%
\pgfpathlineto{\pgfqpoint{4.436086in}{2.641414in}}%
\pgfpathlineto{\pgfqpoint{4.428486in}{2.632821in}}%
\pgfpathlineto{\pgfqpoint{4.420881in}{2.624215in}}%
\pgfpathlineto{\pgfqpoint{4.413271in}{2.615596in}}%
\pgfpathlineto{\pgfqpoint{4.405656in}{2.606964in}}%
\pgfpathlineto{\pgfqpoint{4.392180in}{2.608015in}}%
\pgfpathlineto{\pgfqpoint{4.378712in}{2.609176in}}%
\pgfpathlineto{\pgfqpoint{4.365252in}{2.610448in}}%
\pgfpathlineto{\pgfqpoint{4.351800in}{2.611830in}}%
\pgfpathlineto{\pgfqpoint{4.359426in}{2.620479in}}%
\pgfpathlineto{\pgfqpoint{4.367046in}{2.629119in}}%
\pgfpathlineto{\pgfqpoint{4.374662in}{2.637750in}}%
\pgfpathlineto{\pgfqpoint{4.382272in}{2.646373in}}%
\pgfpathclose%
\pgfusepath{fill}%
\end{pgfscope}%
\begin{pgfscope}%
\pgfpathrectangle{\pgfqpoint{1.150000in}{0.150000in}}{\pgfqpoint{5.700000in}{5.700000in}}%
\pgfusepath{clip}%
\pgfsetbuttcap%
\pgfsetroundjoin%
\definecolor{currentfill}{rgb}{0.278012,0.180367,0.486697}%
\pgfsetfillcolor{currentfill}%
\pgfsetfillopacity{0.700000}%
\pgfsetlinewidth{0.000000pt}%
\definecolor{currentstroke}{rgb}{0.000000,0.000000,0.000000}%
\pgfsetstrokecolor{currentstroke}%
\pgfsetdash{}{0pt}%
\pgfpathmoveto{\pgfqpoint{4.857007in}{2.805335in}}%
\pgfpathlineto{\pgfqpoint{4.870605in}{2.806109in}}%
\pgfpathlineto{\pgfqpoint{4.884213in}{2.806986in}}%
\pgfpathlineto{\pgfqpoint{4.897831in}{2.807966in}}%
\pgfpathlineto{\pgfqpoint{4.911459in}{2.809050in}}%
\pgfpathlineto{\pgfqpoint{4.904033in}{2.801232in}}%
\pgfpathlineto{\pgfqpoint{4.896602in}{2.793397in}}%
\pgfpathlineto{\pgfqpoint{4.889165in}{2.785543in}}%
\pgfpathlineto{\pgfqpoint{4.881723in}{2.777669in}}%
\pgfpathlineto{\pgfqpoint{4.868083in}{2.776471in}}%
\pgfpathlineto{\pgfqpoint{4.854453in}{2.775376in}}%
\pgfpathlineto{\pgfqpoint{4.840833in}{2.774385in}}%
\pgfpathlineto{\pgfqpoint{4.827223in}{2.773497in}}%
\pgfpathlineto{\pgfqpoint{4.834677in}{2.781479in}}%
\pgfpathlineto{\pgfqpoint{4.842126in}{2.789445in}}%
\pgfpathlineto{\pgfqpoint{4.849569in}{2.797397in}}%
\pgfpathlineto{\pgfqpoint{4.857007in}{2.805335in}}%
\pgfpathclose%
\pgfusepath{fill}%
\end{pgfscope}%
\begin{pgfscope}%
\pgfpathrectangle{\pgfqpoint{1.150000in}{0.150000in}}{\pgfqpoint{5.700000in}{5.700000in}}%
\pgfusepath{clip}%
\pgfsetbuttcap%
\pgfsetroundjoin%
\definecolor{currentfill}{rgb}{0.277018,0.050344,0.375715}%
\pgfsetfillcolor{currentfill}%
\pgfsetfillopacity{0.700000}%
\pgfsetlinewidth{0.000000pt}%
\definecolor{currentstroke}{rgb}{0.000000,0.000000,0.000000}%
\pgfsetstrokecolor{currentstroke}%
\pgfsetdash{}{0pt}%
\pgfpathmoveto{\pgfqpoint{3.716265in}{2.569430in}}%
\pgfpathlineto{\pgfqpoint{3.729562in}{2.563583in}}%
\pgfpathlineto{\pgfqpoint{3.742862in}{2.557867in}}%
\pgfpathlineto{\pgfqpoint{3.756167in}{2.552281in}}%
\pgfpathlineto{\pgfqpoint{3.769476in}{2.546823in}}%
\pgfpathlineto{\pgfqpoint{3.761647in}{2.538562in}}%
\pgfpathlineto{\pgfqpoint{3.753812in}{2.530327in}}%
\pgfpathlineto{\pgfqpoint{3.745971in}{2.522119in}}%
\pgfpathlineto{\pgfqpoint{3.738125in}{2.513939in}}%
\pgfpathlineto{\pgfqpoint{3.724803in}{2.519483in}}%
\pgfpathlineto{\pgfqpoint{3.711485in}{2.525155in}}%
\pgfpathlineto{\pgfqpoint{3.698171in}{2.530957in}}%
\pgfpathlineto{\pgfqpoint{3.684861in}{2.536889in}}%
\pgfpathlineto{\pgfqpoint{3.692721in}{2.544977in}}%
\pgfpathlineto{\pgfqpoint{3.700575in}{2.553097in}}%
\pgfpathlineto{\pgfqpoint{3.708423in}{2.561248in}}%
\pgfpathlineto{\pgfqpoint{3.716265in}{2.569430in}}%
\pgfpathclose%
\pgfusepath{fill}%
\end{pgfscope}%
\begin{pgfscope}%
\pgfpathrectangle{\pgfqpoint{1.150000in}{0.150000in}}{\pgfqpoint{5.700000in}{5.700000in}}%
\pgfusepath{clip}%
\pgfsetbuttcap%
\pgfsetroundjoin%
\definecolor{currentfill}{rgb}{0.278791,0.062145,0.386592}%
\pgfsetfillcolor{currentfill}%
\pgfsetfillopacity{0.700000}%
\pgfsetlinewidth{0.000000pt}%
\definecolor{currentstroke}{rgb}{0.000000,0.000000,0.000000}%
\pgfsetstrokecolor{currentstroke}%
\pgfsetdash{}{0pt}%
\pgfpathmoveto{\pgfqpoint{3.578507in}{2.589126in}}%
\pgfpathlineto{\pgfqpoint{3.591790in}{2.582124in}}%
\pgfpathlineto{\pgfqpoint{3.605076in}{2.575259in}}%
\pgfpathlineto{\pgfqpoint{3.618365in}{2.568530in}}%
\pgfpathlineto{\pgfqpoint{3.631657in}{2.561935in}}%
\pgfpathlineto{\pgfqpoint{3.623778in}{2.553977in}}%
\pgfpathlineto{\pgfqpoint{3.615892in}{2.546056in}}%
\pgfpathlineto{\pgfqpoint{3.608000in}{2.538174in}}%
\pgfpathlineto{\pgfqpoint{3.600102in}{2.530332in}}%
\pgfpathlineto{\pgfqpoint{3.586796in}{2.537030in}}%
\pgfpathlineto{\pgfqpoint{3.573492in}{2.543864in}}%
\pgfpathlineto{\pgfqpoint{3.560191in}{2.550834in}}%
\pgfpathlineto{\pgfqpoint{3.546893in}{2.557940in}}%
\pgfpathlineto{\pgfqpoint{3.554806in}{2.565672in}}%
\pgfpathlineto{\pgfqpoint{3.562712in}{2.573447in}}%
\pgfpathlineto{\pgfqpoint{3.570613in}{2.581266in}}%
\pgfpathlineto{\pgfqpoint{3.578507in}{2.589126in}}%
\pgfpathclose%
\pgfusepath{fill}%
\end{pgfscope}%
\begin{pgfscope}%
\pgfpathrectangle{\pgfqpoint{1.150000in}{0.150000in}}{\pgfqpoint{5.700000in}{5.700000in}}%
\pgfusepath{clip}%
\pgfsetbuttcap%
\pgfsetroundjoin%
\definecolor{currentfill}{rgb}{0.278791,0.062145,0.386592}%
\pgfsetfillcolor{currentfill}%
\pgfsetfillopacity{0.700000}%
\pgfsetlinewidth{0.000000pt}%
\definecolor{currentstroke}{rgb}{0.000000,0.000000,0.000000}%
\pgfsetstrokecolor{currentstroke}%
\pgfsetdash{}{0pt}%
\pgfpathmoveto{\pgfqpoint{4.076014in}{2.579899in}}%
\pgfpathlineto{\pgfqpoint{4.089376in}{2.576679in}}%
\pgfpathlineto{\pgfqpoint{4.102744in}{2.573577in}}%
\pgfpathlineto{\pgfqpoint{4.116118in}{2.570593in}}%
\pgfpathlineto{\pgfqpoint{4.129499in}{2.567725in}}%
\pgfpathlineto{\pgfqpoint{4.121794in}{2.559048in}}%
\pgfpathlineto{\pgfqpoint{4.114084in}{2.550373in}}%
\pgfpathlineto{\pgfqpoint{4.106369in}{2.541699in}}%
\pgfpathlineto{\pgfqpoint{4.098648in}{2.533027in}}%
\pgfpathlineto{\pgfqpoint{4.085256in}{2.535925in}}%
\pgfpathlineto{\pgfqpoint{4.071871in}{2.538941in}}%
\pgfpathlineto{\pgfqpoint{4.058491in}{2.542074in}}%
\pgfpathlineto{\pgfqpoint{4.045118in}{2.545325in}}%
\pgfpathlineto{\pgfqpoint{4.052850in}{2.553960in}}%
\pgfpathlineto{\pgfqpoint{4.060577in}{2.562600in}}%
\pgfpathlineto{\pgfqpoint{4.068298in}{2.571247in}}%
\pgfpathlineto{\pgfqpoint{4.076014in}{2.579899in}}%
\pgfpathclose%
\pgfusepath{fill}%
\end{pgfscope}%
\begin{pgfscope}%
\pgfpathrectangle{\pgfqpoint{1.150000in}{0.150000in}}{\pgfqpoint{5.700000in}{5.700000in}}%
\pgfusepath{clip}%
\pgfsetbuttcap%
\pgfsetroundjoin%
\definecolor{currentfill}{rgb}{0.276022,0.044167,0.370164}%
\pgfsetfillcolor{currentfill}%
\pgfsetfillopacity{0.700000}%
\pgfsetlinewidth{0.000000pt}%
\definecolor{currentstroke}{rgb}{0.000000,0.000000,0.000000}%
\pgfsetstrokecolor{currentstroke}%
\pgfsetdash{}{0pt}%
\pgfpathmoveto{\pgfqpoint{3.853965in}{2.559833in}}%
\pgfpathlineto{\pgfqpoint{3.867284in}{2.555076in}}%
\pgfpathlineto{\pgfqpoint{3.880608in}{2.550444in}}%
\pgfpathlineto{\pgfqpoint{3.893936in}{2.545935in}}%
\pgfpathlineto{\pgfqpoint{3.907270in}{2.541550in}}%
\pgfpathlineto{\pgfqpoint{3.899489in}{2.533067in}}%
\pgfpathlineto{\pgfqpoint{3.891701in}{2.524599in}}%
\pgfpathlineto{\pgfqpoint{3.883908in}{2.516147in}}%
\pgfpathlineto{\pgfqpoint{3.876110in}{2.507712in}}%
\pgfpathlineto{\pgfqpoint{3.862764in}{2.512165in}}%
\pgfpathlineto{\pgfqpoint{3.849423in}{2.516740in}}%
\pgfpathlineto{\pgfqpoint{3.836087in}{2.521440in}}%
\pgfpathlineto{\pgfqpoint{3.822755in}{2.526264in}}%
\pgfpathlineto{\pgfqpoint{3.830566in}{2.534625in}}%
\pgfpathlineto{\pgfqpoint{3.838371in}{2.543007in}}%
\pgfpathlineto{\pgfqpoint{3.846171in}{2.551410in}}%
\pgfpathlineto{\pgfqpoint{3.853965in}{2.559833in}}%
\pgfpathclose%
\pgfusepath{fill}%
\end{pgfscope}%
\begin{pgfscope}%
\pgfpathrectangle{\pgfqpoint{1.150000in}{0.150000in}}{\pgfqpoint{5.700000in}{5.700000in}}%
\pgfusepath{clip}%
\pgfsetbuttcap%
\pgfsetroundjoin%
\definecolor{currentfill}{rgb}{0.280255,0.165693,0.476498}%
\pgfsetfillcolor{currentfill}%
\pgfsetfillopacity{0.700000}%
\pgfsetlinewidth{0.000000pt}%
\definecolor{currentstroke}{rgb}{0.000000,0.000000,0.000000}%
\pgfsetstrokecolor{currentstroke}%
\pgfsetdash{}{0pt}%
\pgfpathmoveto{\pgfqpoint{4.772880in}{2.770985in}}%
\pgfpathlineto{\pgfqpoint{4.786452in}{2.771456in}}%
\pgfpathlineto{\pgfqpoint{4.800032in}{2.772032in}}%
\pgfpathlineto{\pgfqpoint{4.813623in}{2.772713in}}%
\pgfpathlineto{\pgfqpoint{4.827223in}{2.773497in}}%
\pgfpathlineto{\pgfqpoint{4.819764in}{2.765497in}}%
\pgfpathlineto{\pgfqpoint{4.812299in}{2.757477in}}%
\pgfpathlineto{\pgfqpoint{4.804828in}{2.749438in}}%
\pgfpathlineto{\pgfqpoint{4.797352in}{2.741377in}}%
\pgfpathlineto{\pgfqpoint{4.783741in}{2.740496in}}%
\pgfpathlineto{\pgfqpoint{4.770139in}{2.739720in}}%
\pgfpathlineto{\pgfqpoint{4.756546in}{2.739048in}}%
\pgfpathlineto{\pgfqpoint{4.742964in}{2.738480in}}%
\pgfpathlineto{\pgfqpoint{4.750451in}{2.746631in}}%
\pgfpathlineto{\pgfqpoint{4.757933in}{2.754764in}}%
\pgfpathlineto{\pgfqpoint{4.765409in}{2.762882in}}%
\pgfpathlineto{\pgfqpoint{4.772880in}{2.770985in}}%
\pgfpathclose%
\pgfusepath{fill}%
\end{pgfscope}%
\begin{pgfscope}%
\pgfpathrectangle{\pgfqpoint{1.150000in}{0.150000in}}{\pgfqpoint{5.700000in}{5.700000in}}%
\pgfusepath{clip}%
\pgfsetbuttcap%
\pgfsetroundjoin%
\definecolor{currentfill}{rgb}{0.280894,0.078907,0.402329}%
\pgfsetfillcolor{currentfill}%
\pgfsetfillopacity{0.700000}%
\pgfsetlinewidth{0.000000pt}%
\definecolor{currentstroke}{rgb}{0.000000,0.000000,0.000000}%
\pgfsetstrokecolor{currentstroke}%
\pgfsetdash{}{0pt}%
\pgfpathmoveto{\pgfqpoint{3.440600in}{2.619814in}}%
\pgfpathlineto{\pgfqpoint{3.453879in}{2.611582in}}%
\pgfpathlineto{\pgfqpoint{3.467160in}{2.603495in}}%
\pgfpathlineto{\pgfqpoint{3.480443in}{2.595551in}}%
\pgfpathlineto{\pgfqpoint{3.493728in}{2.587748in}}%
\pgfpathlineto{\pgfqpoint{3.485794in}{2.580177in}}%
\pgfpathlineto{\pgfqpoint{3.477853in}{2.572655in}}%
\pgfpathlineto{\pgfqpoint{3.469906in}{2.565184in}}%
\pgfpathlineto{\pgfqpoint{3.461952in}{2.557766in}}%
\pgfpathlineto{\pgfqpoint{3.448651in}{2.565691in}}%
\pgfpathlineto{\pgfqpoint{3.435351in}{2.573758in}}%
\pgfpathlineto{\pgfqpoint{3.422054in}{2.581968in}}%
\pgfpathlineto{\pgfqpoint{3.408759in}{2.590323in}}%
\pgfpathlineto{\pgfqpoint{3.416729in}{2.597612in}}%
\pgfpathlineto{\pgfqpoint{3.424692in}{2.604957in}}%
\pgfpathlineto{\pgfqpoint{3.432649in}{2.612358in}}%
\pgfpathlineto{\pgfqpoint{3.440600in}{2.619814in}}%
\pgfpathclose%
\pgfusepath{fill}%
\end{pgfscope}%
\begin{pgfscope}%
\pgfpathrectangle{\pgfqpoint{1.150000in}{0.150000in}}{\pgfqpoint{5.700000in}{5.700000in}}%
\pgfusepath{clip}%
\pgfsetbuttcap%
\pgfsetroundjoin%
\definecolor{currentfill}{rgb}{0.281446,0.084320,0.407414}%
\pgfsetfillcolor{currentfill}%
\pgfsetfillopacity{0.700000}%
\pgfsetlinewidth{0.000000pt}%
\definecolor{currentstroke}{rgb}{0.000000,0.000000,0.000000}%
\pgfsetstrokecolor{currentstroke}%
\pgfsetdash{}{0pt}%
\pgfpathmoveto{\pgfqpoint{4.298068in}{2.618475in}}%
\pgfpathlineto{\pgfqpoint{4.311490in}{2.616646in}}%
\pgfpathlineto{\pgfqpoint{4.324919in}{2.614929in}}%
\pgfpathlineto{\pgfqpoint{4.338356in}{2.613324in}}%
\pgfpathlineto{\pgfqpoint{4.351800in}{2.611830in}}%
\pgfpathlineto{\pgfqpoint{4.344169in}{2.603172in}}%
\pgfpathlineto{\pgfqpoint{4.336533in}{2.594503in}}%
\pgfpathlineto{\pgfqpoint{4.328891in}{2.585823in}}%
\pgfpathlineto{\pgfqpoint{4.321245in}{2.577133in}}%
\pgfpathlineto{\pgfqpoint{4.307790in}{2.578621in}}%
\pgfpathlineto{\pgfqpoint{4.294342in}{2.580221in}}%
\pgfpathlineto{\pgfqpoint{4.280903in}{2.581933in}}%
\pgfpathlineto{\pgfqpoint{4.267470in}{2.583757in}}%
\pgfpathlineto{\pgfqpoint{4.275127in}{2.592446in}}%
\pgfpathlineto{\pgfqpoint{4.282779in}{2.601128in}}%
\pgfpathlineto{\pgfqpoint{4.290426in}{2.609805in}}%
\pgfpathlineto{\pgfqpoint{4.298068in}{2.618475in}}%
\pgfpathclose%
\pgfusepath{fill}%
\end{pgfscope}%
\begin{pgfscope}%
\pgfpathrectangle{\pgfqpoint{1.150000in}{0.150000in}}{\pgfqpoint{5.700000in}{5.700000in}}%
\pgfusepath{clip}%
\pgfsetbuttcap%
\pgfsetroundjoin%
\definecolor{currentfill}{rgb}{0.283229,0.120777,0.440584}%
\pgfsetfillcolor{currentfill}%
\pgfsetfillopacity{0.700000}%
\pgfsetlinewidth{0.000000pt}%
\definecolor{currentstroke}{rgb}{0.000000,0.000000,0.000000}%
\pgfsetstrokecolor{currentstroke}%
\pgfsetdash{}{0pt}%
\pgfpathmoveto{\pgfqpoint{3.249306in}{2.702186in}}%
\pgfpathlineto{\pgfqpoint{3.262590in}{2.692021in}}%
\pgfpathlineto{\pgfqpoint{3.275874in}{2.682014in}}%
\pgfpathlineto{\pgfqpoint{3.289159in}{2.672162in}}%
\pgfpathlineto{\pgfqpoint{3.302444in}{2.662465in}}%
\pgfpathlineto{\pgfqpoint{3.294432in}{2.655503in}}%
\pgfpathlineto{\pgfqpoint{3.286413in}{2.648608in}}%
\pgfpathlineto{\pgfqpoint{3.278387in}{2.641782in}}%
\pgfpathlineto{\pgfqpoint{3.270354in}{2.635026in}}%
\pgfpathlineto{\pgfqpoint{3.257050in}{2.644865in}}%
\pgfpathlineto{\pgfqpoint{3.243747in}{2.654860in}}%
\pgfpathlineto{\pgfqpoint{3.230444in}{2.665010in}}%
\pgfpathlineto{\pgfqpoint{3.217141in}{2.675318in}}%
\pgfpathlineto{\pgfqpoint{3.225194in}{2.681924in}}%
\pgfpathlineto{\pgfqpoint{3.233238in}{2.688605in}}%
\pgfpathlineto{\pgfqpoint{3.241276in}{2.695360in}}%
\pgfpathlineto{\pgfqpoint{3.249306in}{2.702186in}}%
\pgfpathclose%
\pgfusepath{fill}%
\end{pgfscope}%
\begin{pgfscope}%
\pgfpathrectangle{\pgfqpoint{1.150000in}{0.150000in}}{\pgfqpoint{5.700000in}{5.700000in}}%
\pgfusepath{clip}%
\pgfsetbuttcap%
\pgfsetroundjoin%
\definecolor{currentfill}{rgb}{0.258965,0.251537,0.524736}%
\pgfsetfillcolor{currentfill}%
\pgfsetfillopacity{0.700000}%
\pgfsetlinewidth{0.000000pt}%
\definecolor{currentstroke}{rgb}{0.000000,0.000000,0.000000}%
\pgfsetstrokecolor{currentstroke}%
\pgfsetdash{}{0pt}%
\pgfpathmoveto{\pgfqpoint{2.897512in}{2.973363in}}%
\pgfpathlineto{\pgfqpoint{2.910858in}{2.958864in}}%
\pgfpathlineto{\pgfqpoint{2.924201in}{2.944559in}}%
\pgfpathlineto{\pgfqpoint{2.937540in}{2.930444in}}%
\pgfpathlineto{\pgfqpoint{2.950875in}{2.916520in}}%
\pgfpathlineto{\pgfqpoint{2.942710in}{2.910770in}}%
\pgfpathlineto{\pgfqpoint{2.934536in}{2.905121in}}%
\pgfpathlineto{\pgfqpoint{2.926353in}{2.899573in}}%
\pgfpathlineto{\pgfqpoint{2.918161in}{2.894127in}}%
\pgfpathlineto{\pgfqpoint{2.904801in}{2.908218in}}%
\pgfpathlineto{\pgfqpoint{2.891439in}{2.922498in}}%
\pgfpathlineto{\pgfqpoint{2.878072in}{2.936971in}}%
\pgfpathlineto{\pgfqpoint{2.864702in}{2.951636in}}%
\pgfpathlineto{\pgfqpoint{2.872918in}{2.956908in}}%
\pgfpathlineto{\pgfqpoint{2.881125in}{2.962288in}}%
\pgfpathlineto{\pgfqpoint{2.889323in}{2.967773in}}%
\pgfpathlineto{\pgfqpoint{2.897512in}{2.973363in}}%
\pgfpathclose%
\pgfusepath{fill}%
\end{pgfscope}%
\begin{pgfscope}%
\pgfpathrectangle{\pgfqpoint{1.150000in}{0.150000in}}{\pgfqpoint{5.700000in}{5.700000in}}%
\pgfusepath{clip}%
\pgfsetbuttcap%
\pgfsetroundjoin%
\definecolor{currentfill}{rgb}{0.281887,0.150881,0.465405}%
\pgfsetfillcolor{currentfill}%
\pgfsetfillopacity{0.700000}%
\pgfsetlinewidth{0.000000pt}%
\definecolor{currentstroke}{rgb}{0.000000,0.000000,0.000000}%
\pgfsetstrokecolor{currentstroke}%
\pgfsetdash{}{0pt}%
\pgfpathmoveto{\pgfqpoint{4.688727in}{2.737260in}}%
\pgfpathlineto{\pgfqpoint{4.702272in}{2.737407in}}%
\pgfpathlineto{\pgfqpoint{4.715827in}{2.737660in}}%
\pgfpathlineto{\pgfqpoint{4.729391in}{2.738018in}}%
\pgfpathlineto{\pgfqpoint{4.742964in}{2.738480in}}%
\pgfpathlineto{\pgfqpoint{4.735471in}{2.730311in}}%
\pgfpathlineto{\pgfqpoint{4.727973in}{2.722123in}}%
\pgfpathlineto{\pgfqpoint{4.720470in}{2.713913in}}%
\pgfpathlineto{\pgfqpoint{4.712961in}{2.705682in}}%
\pgfpathlineto{\pgfqpoint{4.699376in}{2.705141in}}%
\pgfpathlineto{\pgfqpoint{4.685801in}{2.704705in}}%
\pgfpathlineto{\pgfqpoint{4.672236in}{2.704375in}}%
\pgfpathlineto{\pgfqpoint{4.658680in}{2.704150in}}%
\pgfpathlineto{\pgfqpoint{4.666200in}{2.712453in}}%
\pgfpathlineto{\pgfqpoint{4.673714in}{2.720738in}}%
\pgfpathlineto{\pgfqpoint{4.681223in}{2.729007in}}%
\pgfpathlineto{\pgfqpoint{4.688727in}{2.737260in}}%
\pgfpathclose%
\pgfusepath{fill}%
\end{pgfscope}%
\begin{pgfscope}%
\pgfpathrectangle{\pgfqpoint{1.150000in}{0.150000in}}{\pgfqpoint{5.700000in}{5.700000in}}%
\pgfusepath{clip}%
\pgfsetbuttcap%
\pgfsetroundjoin%
\definecolor{currentfill}{rgb}{0.266580,0.228262,0.514349}%
\pgfsetfillcolor{currentfill}%
\pgfsetfillopacity{0.700000}%
\pgfsetlinewidth{0.000000pt}%
\definecolor{currentstroke}{rgb}{0.000000,0.000000,0.000000}%
\pgfsetstrokecolor{currentstroke}%
\pgfsetdash{}{0pt}%
\pgfpathmoveto{\pgfqpoint{2.950875in}{2.916520in}}%
\pgfpathlineto{\pgfqpoint{2.964208in}{2.902782in}}%
\pgfpathlineto{\pgfqpoint{2.977538in}{2.889231in}}%
\pgfpathlineto{\pgfqpoint{2.990864in}{2.875864in}}%
\pgfpathlineto{\pgfqpoint{3.004188in}{2.862679in}}%
\pgfpathlineto{\pgfqpoint{2.996045in}{2.856772in}}%
\pgfpathlineto{\pgfqpoint{2.987894in}{2.850960in}}%
\pgfpathlineto{\pgfqpoint{2.979734in}{2.845244in}}%
\pgfpathlineto{\pgfqpoint{2.971566in}{2.839626in}}%
\pgfpathlineto{\pgfqpoint{2.958219in}{2.852975in}}%
\pgfpathlineto{\pgfqpoint{2.944869in}{2.866507in}}%
\pgfpathlineto{\pgfqpoint{2.931517in}{2.880224in}}%
\pgfpathlineto{\pgfqpoint{2.918161in}{2.894127in}}%
\pgfpathlineto{\pgfqpoint{2.926353in}{2.899573in}}%
\pgfpathlineto{\pgfqpoint{2.934536in}{2.905121in}}%
\pgfpathlineto{\pgfqpoint{2.942710in}{2.910770in}}%
\pgfpathlineto{\pgfqpoint{2.950875in}{2.916520in}}%
\pgfpathclose%
\pgfusepath{fill}%
\end{pgfscope}%
\begin{pgfscope}%
\pgfpathrectangle{\pgfqpoint{1.150000in}{0.150000in}}{\pgfqpoint{5.700000in}{5.700000in}}%
\pgfusepath{clip}%
\pgfsetbuttcap%
\pgfsetroundjoin%
\definecolor{currentfill}{rgb}{0.250425,0.274290,0.533103}%
\pgfsetfillcolor{currentfill}%
\pgfsetfillopacity{0.700000}%
\pgfsetlinewidth{0.000000pt}%
\definecolor{currentstroke}{rgb}{0.000000,0.000000,0.000000}%
\pgfsetstrokecolor{currentstroke}%
\pgfsetdash{}{0pt}%
\pgfpathmoveto{\pgfqpoint{5.332268in}{2.990536in}}%
\pgfpathlineto{\pgfqpoint{5.346053in}{2.992887in}}%
\pgfpathlineto{\pgfqpoint{5.359849in}{2.995337in}}%
\pgfpathlineto{\pgfqpoint{5.373657in}{2.997885in}}%
\pgfpathlineto{\pgfqpoint{5.387477in}{3.000532in}}%
\pgfpathlineto{\pgfqpoint{5.380246in}{2.993945in}}%
\pgfpathlineto{\pgfqpoint{5.373009in}{2.987355in}}%
\pgfpathlineto{\pgfqpoint{5.365767in}{2.980758in}}%
\pgfpathlineto{\pgfqpoint{5.358519in}{2.974153in}}%
\pgfpathlineto{\pgfqpoint{5.344684in}{2.971299in}}%
\pgfpathlineto{\pgfqpoint{5.330860in}{2.968543in}}%
\pgfpathlineto{\pgfqpoint{5.317047in}{2.965887in}}%
\pgfpathlineto{\pgfqpoint{5.303247in}{2.963329in}}%
\pgfpathlineto{\pgfqpoint{5.310510in}{2.970135in}}%
\pgfpathlineto{\pgfqpoint{5.317768in}{2.976936in}}%
\pgfpathlineto{\pgfqpoint{5.325021in}{2.983735in}}%
\pgfpathlineto{\pgfqpoint{5.332268in}{2.990536in}}%
\pgfpathclose%
\pgfusepath{fill}%
\end{pgfscope}%
\begin{pgfscope}%
\pgfpathrectangle{\pgfqpoint{1.150000in}{0.150000in}}{\pgfqpoint{5.700000in}{5.700000in}}%
\pgfusepath{clip}%
\pgfsetbuttcap%
\pgfsetroundjoin%
\definecolor{currentfill}{rgb}{0.277018,0.050344,0.375715}%
\pgfsetfillcolor{currentfill}%
\pgfsetfillopacity{0.700000}%
\pgfsetlinewidth{0.000000pt}%
\definecolor{currentstroke}{rgb}{0.000000,0.000000,0.000000}%
\pgfsetstrokecolor{currentstroke}%
\pgfsetdash{}{0pt}%
\pgfpathmoveto{\pgfqpoint{3.991686in}{2.559517in}}%
\pgfpathlineto{\pgfqpoint{4.005035in}{2.555789in}}%
\pgfpathlineto{\pgfqpoint{4.018390in}{2.552182in}}%
\pgfpathlineto{\pgfqpoint{4.031751in}{2.548694in}}%
\pgfpathlineto{\pgfqpoint{4.045118in}{2.545325in}}%
\pgfpathlineto{\pgfqpoint{4.037381in}{2.536696in}}%
\pgfpathlineto{\pgfqpoint{4.029639in}{2.528073in}}%
\pgfpathlineto{\pgfqpoint{4.021891in}{2.519456in}}%
\pgfpathlineto{\pgfqpoint{4.014138in}{2.510845in}}%
\pgfpathlineto{\pgfqpoint{4.000760in}{2.514263in}}%
\pgfpathlineto{\pgfqpoint{3.987387in}{2.517800in}}%
\pgfpathlineto{\pgfqpoint{3.974020in}{2.521457in}}%
\pgfpathlineto{\pgfqpoint{3.960659in}{2.525233in}}%
\pgfpathlineto{\pgfqpoint{3.968424in}{2.533788in}}%
\pgfpathlineto{\pgfqpoint{3.976183in}{2.542354in}}%
\pgfpathlineto{\pgfqpoint{3.983937in}{2.550930in}}%
\pgfpathlineto{\pgfqpoint{3.991686in}{2.559517in}}%
\pgfpathclose%
\pgfusepath{fill}%
\end{pgfscope}%
\begin{pgfscope}%
\pgfpathrectangle{\pgfqpoint{1.150000in}{0.150000in}}{\pgfqpoint{5.700000in}{5.700000in}}%
\pgfusepath{clip}%
\pgfsetbuttcap%
\pgfsetroundjoin%
\definecolor{currentfill}{rgb}{0.282884,0.135920,0.453427}%
\pgfsetfillcolor{currentfill}%
\pgfsetfillopacity{0.700000}%
\pgfsetlinewidth{0.000000pt}%
\definecolor{currentstroke}{rgb}{0.000000,0.000000,0.000000}%
\pgfsetstrokecolor{currentstroke}%
\pgfsetdash{}{0pt}%
\pgfpathmoveto{\pgfqpoint{4.604545in}{2.704312in}}%
\pgfpathlineto{\pgfqpoint{4.618066in}{2.704112in}}%
\pgfpathlineto{\pgfqpoint{4.631595in}{2.704018in}}%
\pgfpathlineto{\pgfqpoint{4.645133in}{2.704031in}}%
\pgfpathlineto{\pgfqpoint{4.658680in}{2.704150in}}%
\pgfpathlineto{\pgfqpoint{4.651154in}{2.695829in}}%
\pgfpathlineto{\pgfqpoint{4.643624in}{2.687488in}}%
\pgfpathlineto{\pgfqpoint{4.636088in}{2.679127in}}%
\pgfpathlineto{\pgfqpoint{4.628546in}{2.670744in}}%
\pgfpathlineto{\pgfqpoint{4.614989in}{2.670566in}}%
\pgfpathlineto{\pgfqpoint{4.601440in}{2.670493in}}%
\pgfpathlineto{\pgfqpoint{4.587900in}{2.670527in}}%
\pgfpathlineto{\pgfqpoint{4.574369in}{2.670668in}}%
\pgfpathlineto{\pgfqpoint{4.581921in}{2.679103in}}%
\pgfpathlineto{\pgfqpoint{4.589468in}{2.687522in}}%
\pgfpathlineto{\pgfqpoint{4.597009in}{2.695925in}}%
\pgfpathlineto{\pgfqpoint{4.604545in}{2.704312in}}%
\pgfpathclose%
\pgfusepath{fill}%
\end{pgfscope}%
\begin{pgfscope}%
\pgfpathrectangle{\pgfqpoint{1.150000in}{0.150000in}}{\pgfqpoint{5.700000in}{5.700000in}}%
\pgfusepath{clip}%
\pgfsetbuttcap%
\pgfsetroundjoin%
\definecolor{currentfill}{rgb}{0.273006,0.204520,0.501721}%
\pgfsetfillcolor{currentfill}%
\pgfsetfillopacity{0.700000}%
\pgfsetlinewidth{0.000000pt}%
\definecolor{currentstroke}{rgb}{0.000000,0.000000,0.000000}%
\pgfsetstrokecolor{currentstroke}%
\pgfsetdash{}{0pt}%
\pgfpathmoveto{\pgfqpoint{3.004188in}{2.862679in}}%
\pgfpathlineto{\pgfqpoint{3.017510in}{2.849676in}}%
\pgfpathlineto{\pgfqpoint{3.030829in}{2.836852in}}%
\pgfpathlineto{\pgfqpoint{3.044146in}{2.824205in}}%
\pgfpathlineto{\pgfqpoint{3.057461in}{2.811735in}}%
\pgfpathlineto{\pgfqpoint{3.049340in}{2.805671in}}%
\pgfpathlineto{\pgfqpoint{3.041211in}{2.799696in}}%
\pgfpathlineto{\pgfqpoint{3.033073in}{2.793814in}}%
\pgfpathlineto{\pgfqpoint{3.024927in}{2.788025in}}%
\pgfpathlineto{\pgfqpoint{3.011590in}{2.800659in}}%
\pgfpathlineto{\pgfqpoint{2.998251in}{2.813469in}}%
\pgfpathlineto{\pgfqpoint{2.984910in}{2.826458in}}%
\pgfpathlineto{\pgfqpoint{2.971566in}{2.839626in}}%
\pgfpathlineto{\pgfqpoint{2.979734in}{2.845244in}}%
\pgfpathlineto{\pgfqpoint{2.987894in}{2.850960in}}%
\pgfpathlineto{\pgfqpoint{2.996045in}{2.856772in}}%
\pgfpathlineto{\pgfqpoint{3.004188in}{2.862679in}}%
\pgfpathclose%
\pgfusepath{fill}%
\end{pgfscope}%
\begin{pgfscope}%
\pgfpathrectangle{\pgfqpoint{1.150000in}{0.150000in}}{\pgfqpoint{5.700000in}{5.700000in}}%
\pgfusepath{clip}%
\pgfsetbuttcap%
\pgfsetroundjoin%
\definecolor{currentfill}{rgb}{0.280267,0.073417,0.397163}%
\pgfsetfillcolor{currentfill}%
\pgfsetfillopacity{0.700000}%
\pgfsetlinewidth{0.000000pt}%
\definecolor{currentstroke}{rgb}{0.000000,0.000000,0.000000}%
\pgfsetstrokecolor{currentstroke}%
\pgfsetdash{}{0pt}%
\pgfpathmoveto{\pgfqpoint{4.213813in}{2.592186in}}%
\pgfpathlineto{\pgfqpoint{4.227216in}{2.589908in}}%
\pgfpathlineto{\pgfqpoint{4.240627in}{2.587744in}}%
\pgfpathlineto{\pgfqpoint{4.254045in}{2.585694in}}%
\pgfpathlineto{\pgfqpoint{4.267470in}{2.583757in}}%
\pgfpathlineto{\pgfqpoint{4.259808in}{2.575062in}}%
\pgfpathlineto{\pgfqpoint{4.252140in}{2.566359in}}%
\pgfpathlineto{\pgfqpoint{4.244467in}{2.557650in}}%
\pgfpathlineto{\pgfqpoint{4.236789in}{2.548932in}}%
\pgfpathlineto{\pgfqpoint{4.223353in}{2.550882in}}%
\pgfpathlineto{\pgfqpoint{4.209925in}{2.552945in}}%
\pgfpathlineto{\pgfqpoint{4.196503in}{2.555122in}}%
\pgfpathlineto{\pgfqpoint{4.183089in}{2.557412in}}%
\pgfpathlineto{\pgfqpoint{4.190778in}{2.566110in}}%
\pgfpathlineto{\pgfqpoint{4.198461in}{2.574805in}}%
\pgfpathlineto{\pgfqpoint{4.206140in}{2.583497in}}%
\pgfpathlineto{\pgfqpoint{4.213813in}{2.592186in}}%
\pgfpathclose%
\pgfusepath{fill}%
\end{pgfscope}%
\begin{pgfscope}%
\pgfpathrectangle{\pgfqpoint{1.150000in}{0.150000in}}{\pgfqpoint{5.700000in}{5.700000in}}%
\pgfusepath{clip}%
\pgfsetbuttcap%
\pgfsetroundjoin%
\definecolor{currentfill}{rgb}{0.255645,0.260703,0.528312}%
\pgfsetfillcolor{currentfill}%
\pgfsetfillopacity{0.700000}%
\pgfsetlinewidth{0.000000pt}%
\definecolor{currentstroke}{rgb}{0.000000,0.000000,0.000000}%
\pgfsetstrokecolor{currentstroke}%
\pgfsetdash{}{0pt}%
\pgfpathmoveto{\pgfqpoint{5.248159in}{2.954090in}}%
\pgfpathlineto{\pgfqpoint{5.261914in}{2.956251in}}%
\pgfpathlineto{\pgfqpoint{5.275680in}{2.958511in}}%
\pgfpathlineto{\pgfqpoint{5.289458in}{2.960870in}}%
\pgfpathlineto{\pgfqpoint{5.303247in}{2.963329in}}%
\pgfpathlineto{\pgfqpoint{5.295978in}{2.956517in}}%
\pgfpathlineto{\pgfqpoint{5.288703in}{2.949696in}}%
\pgfpathlineto{\pgfqpoint{5.281423in}{2.942864in}}%
\pgfpathlineto{\pgfqpoint{5.274138in}{2.936018in}}%
\pgfpathlineto{\pgfqpoint{5.260333in}{2.933371in}}%
\pgfpathlineto{\pgfqpoint{5.246541in}{2.930822in}}%
\pgfpathlineto{\pgfqpoint{5.232759in}{2.928374in}}%
\pgfpathlineto{\pgfqpoint{5.218989in}{2.926025in}}%
\pgfpathlineto{\pgfqpoint{5.226290in}{2.933052in}}%
\pgfpathlineto{\pgfqpoint{5.233585in}{2.940071in}}%
\pgfpathlineto{\pgfqpoint{5.240875in}{2.947082in}}%
\pgfpathlineto{\pgfqpoint{5.248159in}{2.954090in}}%
\pgfpathclose%
\pgfusepath{fill}%
\end{pgfscope}%
\begin{pgfscope}%
\pgfpathrectangle{\pgfqpoint{1.150000in}{0.150000in}}{\pgfqpoint{5.700000in}{5.700000in}}%
\pgfusepath{clip}%
\pgfsetbuttcap%
\pgfsetroundjoin%
\definecolor{currentfill}{rgb}{0.277018,0.050344,0.375715}%
\pgfsetfillcolor{currentfill}%
\pgfsetfillopacity{0.700000}%
\pgfsetlinewidth{0.000000pt}%
\definecolor{currentstroke}{rgb}{0.000000,0.000000,0.000000}%
\pgfsetstrokecolor{currentstroke}%
\pgfsetdash{}{0pt}%
\pgfpathmoveto{\pgfqpoint{3.631657in}{2.561935in}}%
\pgfpathlineto{\pgfqpoint{3.644953in}{2.555474in}}%
\pgfpathlineto{\pgfqpoint{3.658252in}{2.549147in}}%
\pgfpathlineto{\pgfqpoint{3.671555in}{2.542952in}}%
\pgfpathlineto{\pgfqpoint{3.684861in}{2.536889in}}%
\pgfpathlineto{\pgfqpoint{3.676995in}{2.528834in}}%
\pgfpathlineto{\pgfqpoint{3.669124in}{2.520812in}}%
\pgfpathlineto{\pgfqpoint{3.661246in}{2.512824in}}%
\pgfpathlineto{\pgfqpoint{3.653363in}{2.504871in}}%
\pgfpathlineto{\pgfqpoint{3.640043in}{2.511038in}}%
\pgfpathlineto{\pgfqpoint{3.626726in}{2.517336in}}%
\pgfpathlineto{\pgfqpoint{3.613412in}{2.523767in}}%
\pgfpathlineto{\pgfqpoint{3.600102in}{2.530332in}}%
\pgfpathlineto{\pgfqpoint{3.608000in}{2.538174in}}%
\pgfpathlineto{\pgfqpoint{3.615892in}{2.546056in}}%
\pgfpathlineto{\pgfqpoint{3.623778in}{2.553977in}}%
\pgfpathlineto{\pgfqpoint{3.631657in}{2.561935in}}%
\pgfpathclose%
\pgfusepath{fill}%
\end{pgfscope}%
\begin{pgfscope}%
\pgfpathrectangle{\pgfqpoint{1.150000in}{0.150000in}}{\pgfqpoint{5.700000in}{5.700000in}}%
\pgfusepath{clip}%
\pgfsetbuttcap%
\pgfsetroundjoin%
\definecolor{currentfill}{rgb}{0.282910,0.105393,0.426902}%
\pgfsetfillcolor{currentfill}%
\pgfsetfillopacity{0.700000}%
\pgfsetlinewidth{0.000000pt}%
\definecolor{currentstroke}{rgb}{0.000000,0.000000,0.000000}%
\pgfsetstrokecolor{currentstroke}%
\pgfsetdash{}{0pt}%
\pgfpathmoveto{\pgfqpoint{3.302444in}{2.662465in}}%
\pgfpathlineto{\pgfqpoint{3.315730in}{2.652921in}}%
\pgfpathlineto{\pgfqpoint{3.329016in}{2.643530in}}%
\pgfpathlineto{\pgfqpoint{3.342303in}{2.634290in}}%
\pgfpathlineto{\pgfqpoint{3.355592in}{2.625200in}}%
\pgfpathlineto{\pgfqpoint{3.347598in}{2.618103in}}%
\pgfpathlineto{\pgfqpoint{3.339597in}{2.611069in}}%
\pgfpathlineto{\pgfqpoint{3.331589in}{2.604099in}}%
\pgfpathlineto{\pgfqpoint{3.323574in}{2.597194in}}%
\pgfpathlineto{\pgfqpoint{3.310268in}{2.606425in}}%
\pgfpathlineto{\pgfqpoint{3.296962in}{2.615807in}}%
\pgfpathlineto{\pgfqpoint{3.283658in}{2.625340in}}%
\pgfpathlineto{\pgfqpoint{3.270354in}{2.635026in}}%
\pgfpathlineto{\pgfqpoint{3.278387in}{2.641782in}}%
\pgfpathlineto{\pgfqpoint{3.286413in}{2.648608in}}%
\pgfpathlineto{\pgfqpoint{3.294432in}{2.655503in}}%
\pgfpathlineto{\pgfqpoint{3.302444in}{2.662465in}}%
\pgfpathclose%
\pgfusepath{fill}%
\end{pgfscope}%
\begin{pgfscope}%
\pgfpathrectangle{\pgfqpoint{1.150000in}{0.150000in}}{\pgfqpoint{5.700000in}{5.700000in}}%
\pgfusepath{clip}%
\pgfsetbuttcap%
\pgfsetroundjoin%
\definecolor{currentfill}{rgb}{0.276022,0.044167,0.370164}%
\pgfsetfillcolor{currentfill}%
\pgfsetfillopacity{0.700000}%
\pgfsetlinewidth{0.000000pt}%
\definecolor{currentstroke}{rgb}{0.000000,0.000000,0.000000}%
\pgfsetstrokecolor{currentstroke}%
\pgfsetdash{}{0pt}%
\pgfpathmoveto{\pgfqpoint{3.769476in}{2.546823in}}%
\pgfpathlineto{\pgfqpoint{3.782789in}{2.541493in}}%
\pgfpathlineto{\pgfqpoint{3.796107in}{2.536290in}}%
\pgfpathlineto{\pgfqpoint{3.809429in}{2.531214in}}%
\pgfpathlineto{\pgfqpoint{3.822755in}{2.526264in}}%
\pgfpathlineto{\pgfqpoint{3.814939in}{2.517925in}}%
\pgfpathlineto{\pgfqpoint{3.807117in}{2.509607in}}%
\pgfpathlineto{\pgfqpoint{3.799290in}{2.501312in}}%
\pgfpathlineto{\pgfqpoint{3.791456in}{2.493040in}}%
\pgfpathlineto{\pgfqpoint{3.778117in}{2.498075in}}%
\pgfpathlineto{\pgfqpoint{3.764782in}{2.503236in}}%
\pgfpathlineto{\pgfqpoint{3.751451in}{2.508524in}}%
\pgfpathlineto{\pgfqpoint{3.738125in}{2.513939in}}%
\pgfpathlineto{\pgfqpoint{3.745971in}{2.522119in}}%
\pgfpathlineto{\pgfqpoint{3.753812in}{2.530327in}}%
\pgfpathlineto{\pgfqpoint{3.761647in}{2.538562in}}%
\pgfpathlineto{\pgfqpoint{3.769476in}{2.546823in}}%
\pgfpathclose%
\pgfusepath{fill}%
\end{pgfscope}%
\begin{pgfscope}%
\pgfpathrectangle{\pgfqpoint{1.150000in}{0.150000in}}{\pgfqpoint{5.700000in}{5.700000in}}%
\pgfusepath{clip}%
\pgfsetbuttcap%
\pgfsetroundjoin%
\definecolor{currentfill}{rgb}{0.262138,0.242286,0.520837}%
\pgfsetfillcolor{currentfill}%
\pgfsetfillopacity{0.700000}%
\pgfsetlinewidth{0.000000pt}%
\definecolor{currentstroke}{rgb}{0.000000,0.000000,0.000000}%
\pgfsetstrokecolor{currentstroke}%
\pgfsetdash{}{0pt}%
\pgfpathmoveto{\pgfqpoint{5.164022in}{2.917627in}}%
\pgfpathlineto{\pgfqpoint{5.177747in}{2.919576in}}%
\pgfpathlineto{\pgfqpoint{5.191483in}{2.921626in}}%
\pgfpathlineto{\pgfqpoint{5.205231in}{2.923775in}}%
\pgfpathlineto{\pgfqpoint{5.218989in}{2.926025in}}%
\pgfpathlineto{\pgfqpoint{5.211683in}{2.918986in}}%
\pgfpathlineto{\pgfqpoint{5.204371in}{2.911934in}}%
\pgfpathlineto{\pgfqpoint{5.197054in}{2.904866in}}%
\pgfpathlineto{\pgfqpoint{5.189731in}{2.897781in}}%
\pgfpathlineto{\pgfqpoint{5.175958in}{2.895361in}}%
\pgfpathlineto{\pgfqpoint{5.162196in}{2.893042in}}%
\pgfpathlineto{\pgfqpoint{5.148446in}{2.890822in}}%
\pgfpathlineto{\pgfqpoint{5.134707in}{2.888703in}}%
\pgfpathlineto{\pgfqpoint{5.142044in}{2.895951in}}%
\pgfpathlineto{\pgfqpoint{5.149375in}{2.903187in}}%
\pgfpathlineto{\pgfqpoint{5.156701in}{2.910411in}}%
\pgfpathlineto{\pgfqpoint{5.164022in}{2.917627in}}%
\pgfpathclose%
\pgfusepath{fill}%
\end{pgfscope}%
\begin{pgfscope}%
\pgfpathrectangle{\pgfqpoint{1.150000in}{0.150000in}}{\pgfqpoint{5.700000in}{5.700000in}}%
\pgfusepath{clip}%
\pgfsetbuttcap%
\pgfsetroundjoin%
\definecolor{currentfill}{rgb}{0.283229,0.120777,0.440584}%
\pgfsetfillcolor{currentfill}%
\pgfsetfillopacity{0.700000}%
\pgfsetlinewidth{0.000000pt}%
\definecolor{currentstroke}{rgb}{0.000000,0.000000,0.000000}%
\pgfsetstrokecolor{currentstroke}%
\pgfsetdash{}{0pt}%
\pgfpathmoveto{\pgfqpoint{4.520333in}{2.672305in}}%
\pgfpathlineto{\pgfqpoint{4.533829in}{2.671734in}}%
\pgfpathlineto{\pgfqpoint{4.547334in}{2.671271in}}%
\pgfpathlineto{\pgfqpoint{4.560847in}{2.670916in}}%
\pgfpathlineto{\pgfqpoint{4.574369in}{2.670668in}}%
\pgfpathlineto{\pgfqpoint{4.566812in}{2.662214in}}%
\pgfpathlineto{\pgfqpoint{4.559249in}{2.653742in}}%
\pgfpathlineto{\pgfqpoint{4.551681in}{2.645251in}}%
\pgfpathlineto{\pgfqpoint{4.544107in}{2.636739in}}%
\pgfpathlineto{\pgfqpoint{4.530575in}{2.636945in}}%
\pgfpathlineto{\pgfqpoint{4.517051in}{2.637259in}}%
\pgfpathlineto{\pgfqpoint{4.503536in}{2.637680in}}%
\pgfpathlineto{\pgfqpoint{4.490029in}{2.638210in}}%
\pgfpathlineto{\pgfqpoint{4.497613in}{2.646756in}}%
\pgfpathlineto{\pgfqpoint{4.505191in}{2.655287in}}%
\pgfpathlineto{\pgfqpoint{4.512765in}{2.663803in}}%
\pgfpathlineto{\pgfqpoint{4.520333in}{2.672305in}}%
\pgfpathclose%
\pgfusepath{fill}%
\end{pgfscope}%
\begin{pgfscope}%
\pgfpathrectangle{\pgfqpoint{1.150000in}{0.150000in}}{\pgfqpoint{5.700000in}{5.700000in}}%
\pgfusepath{clip}%
\pgfsetbuttcap%
\pgfsetroundjoin%
\definecolor{currentfill}{rgb}{0.278012,0.180367,0.486697}%
\pgfsetfillcolor{currentfill}%
\pgfsetfillopacity{0.700000}%
\pgfsetlinewidth{0.000000pt}%
\definecolor{currentstroke}{rgb}{0.000000,0.000000,0.000000}%
\pgfsetstrokecolor{currentstroke}%
\pgfsetdash{}{0pt}%
\pgfpathmoveto{\pgfqpoint{3.057461in}{2.811735in}}%
\pgfpathlineto{\pgfqpoint{3.070774in}{2.799440in}}%
\pgfpathlineto{\pgfqpoint{3.084086in}{2.787318in}}%
\pgfpathlineto{\pgfqpoint{3.097396in}{2.775368in}}%
\pgfpathlineto{\pgfqpoint{3.110704in}{2.763588in}}%
\pgfpathlineto{\pgfqpoint{3.102604in}{2.757367in}}%
\pgfpathlineto{\pgfqpoint{3.094497in}{2.751232in}}%
\pgfpathlineto{\pgfqpoint{3.086381in}{2.745184in}}%
\pgfpathlineto{\pgfqpoint{3.078257in}{2.739225in}}%
\pgfpathlineto{\pgfqpoint{3.064927in}{2.751168in}}%
\pgfpathlineto{\pgfqpoint{3.051595in}{2.763281in}}%
\pgfpathlineto{\pgfqpoint{3.038262in}{2.775566in}}%
\pgfpathlineto{\pgfqpoint{3.024927in}{2.788025in}}%
\pgfpathlineto{\pgfqpoint{3.033073in}{2.793814in}}%
\pgfpathlineto{\pgfqpoint{3.041211in}{2.799696in}}%
\pgfpathlineto{\pgfqpoint{3.049340in}{2.805671in}}%
\pgfpathlineto{\pgfqpoint{3.057461in}{2.811735in}}%
\pgfpathclose%
\pgfusepath{fill}%
\end{pgfscope}%
\begin{pgfscope}%
\pgfpathrectangle{\pgfqpoint{1.150000in}{0.150000in}}{\pgfqpoint{5.700000in}{5.700000in}}%
\pgfusepath{clip}%
\pgfsetbuttcap%
\pgfsetroundjoin%
\definecolor{currentfill}{rgb}{0.279566,0.067836,0.391917}%
\pgfsetfillcolor{currentfill}%
\pgfsetfillopacity{0.700000}%
\pgfsetlinewidth{0.000000pt}%
\definecolor{currentstroke}{rgb}{0.000000,0.000000,0.000000}%
\pgfsetstrokecolor{currentstroke}%
\pgfsetdash{}{0pt}%
\pgfpathmoveto{\pgfqpoint{3.493728in}{2.587748in}}%
\pgfpathlineto{\pgfqpoint{3.507016in}{2.580087in}}%
\pgfpathlineto{\pgfqpoint{3.520306in}{2.572565in}}%
\pgfpathlineto{\pgfqpoint{3.533598in}{2.565183in}}%
\pgfpathlineto{\pgfqpoint{3.546893in}{2.557940in}}%
\pgfpathlineto{\pgfqpoint{3.538974in}{2.550253in}}%
\pgfpathlineto{\pgfqpoint{3.531049in}{2.542611in}}%
\pgfpathlineto{\pgfqpoint{3.523118in}{2.535016in}}%
\pgfpathlineto{\pgfqpoint{3.515180in}{2.527468in}}%
\pgfpathlineto{\pgfqpoint{3.501869in}{2.534834in}}%
\pgfpathlineto{\pgfqpoint{3.488561in}{2.542338in}}%
\pgfpathlineto{\pgfqpoint{3.475256in}{2.549982in}}%
\pgfpathlineto{\pgfqpoint{3.461952in}{2.557766in}}%
\pgfpathlineto{\pgfqpoint{3.469906in}{2.565184in}}%
\pgfpathlineto{\pgfqpoint{3.477853in}{2.572655in}}%
\pgfpathlineto{\pgfqpoint{3.485794in}{2.580177in}}%
\pgfpathlineto{\pgfqpoint{3.493728in}{2.587748in}}%
\pgfpathclose%
\pgfusepath{fill}%
\end{pgfscope}%
\begin{pgfscope}%
\pgfpathrectangle{\pgfqpoint{1.150000in}{0.150000in}}{\pgfqpoint{5.700000in}{5.700000in}}%
\pgfusepath{clip}%
\pgfsetbuttcap%
\pgfsetroundjoin%
\definecolor{currentfill}{rgb}{0.266580,0.228262,0.514349}%
\pgfsetfillcolor{currentfill}%
\pgfsetfillopacity{0.700000}%
\pgfsetlinewidth{0.000000pt}%
\definecolor{currentstroke}{rgb}{0.000000,0.000000,0.000000}%
\pgfsetstrokecolor{currentstroke}%
\pgfsetdash{}{0pt}%
\pgfpathmoveto{\pgfqpoint{5.079859in}{2.881232in}}%
\pgfpathlineto{\pgfqpoint{5.093554in}{2.882948in}}%
\pgfpathlineto{\pgfqpoint{5.107261in}{2.884766in}}%
\pgfpathlineto{\pgfqpoint{5.120978in}{2.886684in}}%
\pgfpathlineto{\pgfqpoint{5.134707in}{2.888703in}}%
\pgfpathlineto{\pgfqpoint{5.127364in}{2.881439in}}%
\pgfpathlineto{\pgfqpoint{5.120016in}{2.874159in}}%
\pgfpathlineto{\pgfqpoint{5.112662in}{2.866859in}}%
\pgfpathlineto{\pgfqpoint{5.105302in}{2.859539in}}%
\pgfpathlineto{\pgfqpoint{5.091560in}{2.857368in}}%
\pgfpathlineto{\pgfqpoint{5.077829in}{2.855298in}}%
\pgfpathlineto{\pgfqpoint{5.064109in}{2.853329in}}%
\pgfpathlineto{\pgfqpoint{5.050400in}{2.851461in}}%
\pgfpathlineto{\pgfqpoint{5.057773in}{2.858926in}}%
\pgfpathlineto{\pgfqpoint{5.065140in}{2.866375in}}%
\pgfpathlineto{\pgfqpoint{5.072502in}{2.873810in}}%
\pgfpathlineto{\pgfqpoint{5.079859in}{2.881232in}}%
\pgfpathclose%
\pgfusepath{fill}%
\end{pgfscope}%
\begin{pgfscope}%
\pgfpathrectangle{\pgfqpoint{1.150000in}{0.150000in}}{\pgfqpoint{5.700000in}{5.700000in}}%
\pgfusepath{clip}%
\pgfsetbuttcap%
\pgfsetroundjoin%
\definecolor{currentfill}{rgb}{0.276022,0.044167,0.370164}%
\pgfsetfillcolor{currentfill}%
\pgfsetfillopacity{0.700000}%
\pgfsetlinewidth{0.000000pt}%
\definecolor{currentstroke}{rgb}{0.000000,0.000000,0.000000}%
\pgfsetstrokecolor{currentstroke}%
\pgfsetdash{}{0pt}%
\pgfpathmoveto{\pgfqpoint{3.907270in}{2.541550in}}%
\pgfpathlineto{\pgfqpoint{3.920610in}{2.537288in}}%
\pgfpathlineto{\pgfqpoint{3.933954in}{2.533148in}}%
\pgfpathlineto{\pgfqpoint{3.947304in}{2.529130in}}%
\pgfpathlineto{\pgfqpoint{3.960659in}{2.525233in}}%
\pgfpathlineto{\pgfqpoint{3.952889in}{2.516689in}}%
\pgfpathlineto{\pgfqpoint{3.945114in}{2.508157in}}%
\pgfpathlineto{\pgfqpoint{3.937333in}{2.499636in}}%
\pgfpathlineto{\pgfqpoint{3.929547in}{2.491128in}}%
\pgfpathlineto{\pgfqpoint{3.916180in}{2.495091in}}%
\pgfpathlineto{\pgfqpoint{3.902818in}{2.499176in}}%
\pgfpathlineto{\pgfqpoint{3.889461in}{2.503383in}}%
\pgfpathlineto{\pgfqpoint{3.876110in}{2.507712in}}%
\pgfpathlineto{\pgfqpoint{3.883908in}{2.516147in}}%
\pgfpathlineto{\pgfqpoint{3.891701in}{2.524599in}}%
\pgfpathlineto{\pgfqpoint{3.899489in}{2.533067in}}%
\pgfpathlineto{\pgfqpoint{3.907270in}{2.541550in}}%
\pgfpathclose%
\pgfusepath{fill}%
\end{pgfscope}%
\begin{pgfscope}%
\pgfpathrectangle{\pgfqpoint{1.150000in}{0.150000in}}{\pgfqpoint{5.700000in}{5.700000in}}%
\pgfusepath{clip}%
\pgfsetbuttcap%
\pgfsetroundjoin%
\definecolor{currentfill}{rgb}{0.278791,0.062145,0.386592}%
\pgfsetfillcolor{currentfill}%
\pgfsetfillopacity{0.700000}%
\pgfsetlinewidth{0.000000pt}%
\definecolor{currentstroke}{rgb}{0.000000,0.000000,0.000000}%
\pgfsetstrokecolor{currentstroke}%
\pgfsetdash{}{0pt}%
\pgfpathmoveto{\pgfqpoint{4.129499in}{2.567725in}}%
\pgfpathlineto{\pgfqpoint{4.142887in}{2.564973in}}%
\pgfpathlineto{\pgfqpoint{4.156281in}{2.562338in}}%
\pgfpathlineto{\pgfqpoint{4.169681in}{2.559818in}}%
\pgfpathlineto{\pgfqpoint{4.183089in}{2.557412in}}%
\pgfpathlineto{\pgfqpoint{4.175395in}{2.548712in}}%
\pgfpathlineto{\pgfqpoint{4.167696in}{2.540008in}}%
\pgfpathlineto{\pgfqpoint{4.159991in}{2.531301in}}%
\pgfpathlineto{\pgfqpoint{4.152281in}{2.522591in}}%
\pgfpathlineto{\pgfqpoint{4.138863in}{2.525027in}}%
\pgfpathlineto{\pgfqpoint{4.125452in}{2.527578in}}%
\pgfpathlineto{\pgfqpoint{4.112047in}{2.530244in}}%
\pgfpathlineto{\pgfqpoint{4.098648in}{2.533027in}}%
\pgfpathlineto{\pgfqpoint{4.106369in}{2.541699in}}%
\pgfpathlineto{\pgfqpoint{4.114084in}{2.550373in}}%
\pgfpathlineto{\pgfqpoint{4.121794in}{2.559048in}}%
\pgfpathlineto{\pgfqpoint{4.129499in}{2.567725in}}%
\pgfpathclose%
\pgfusepath{fill}%
\end{pgfscope}%
\begin{pgfscope}%
\pgfpathrectangle{\pgfqpoint{1.150000in}{0.150000in}}{\pgfqpoint{5.700000in}{5.700000in}}%
\pgfusepath{clip}%
\pgfsetbuttcap%
\pgfsetroundjoin%
\definecolor{currentfill}{rgb}{0.282910,0.105393,0.426902}%
\pgfsetfillcolor{currentfill}%
\pgfsetfillopacity{0.700000}%
\pgfsetlinewidth{0.000000pt}%
\definecolor{currentstroke}{rgb}{0.000000,0.000000,0.000000}%
\pgfsetstrokecolor{currentstroke}%
\pgfsetdash{}{0pt}%
\pgfpathmoveto{\pgfqpoint{4.436086in}{2.641414in}}%
\pgfpathlineto{\pgfqpoint{4.449559in}{2.640449in}}%
\pgfpathlineto{\pgfqpoint{4.463041in}{2.639594in}}%
\pgfpathlineto{\pgfqpoint{4.476531in}{2.638847in}}%
\pgfpathlineto{\pgfqpoint{4.490029in}{2.638210in}}%
\pgfpathlineto{\pgfqpoint{4.482440in}{2.629647in}}%
\pgfpathlineto{\pgfqpoint{4.474846in}{2.621067in}}%
\pgfpathlineto{\pgfqpoint{4.467246in}{2.612470in}}%
\pgfpathlineto{\pgfqpoint{4.459641in}{2.603854in}}%
\pgfpathlineto{\pgfqpoint{4.446132in}{2.604468in}}%
\pgfpathlineto{\pgfqpoint{4.432632in}{2.605191in}}%
\pgfpathlineto{\pgfqpoint{4.419140in}{2.606023in}}%
\pgfpathlineto{\pgfqpoint{4.405656in}{2.606964in}}%
\pgfpathlineto{\pgfqpoint{4.413271in}{2.615596in}}%
\pgfpathlineto{\pgfqpoint{4.420881in}{2.624215in}}%
\pgfpathlineto{\pgfqpoint{4.428486in}{2.632821in}}%
\pgfpathlineto{\pgfqpoint{4.436086in}{2.641414in}}%
\pgfpathclose%
\pgfusepath{fill}%
\end{pgfscope}%
\begin{pgfscope}%
\pgfpathrectangle{\pgfqpoint{1.150000in}{0.150000in}}{\pgfqpoint{5.700000in}{5.700000in}}%
\pgfusepath{clip}%
\pgfsetbuttcap%
\pgfsetroundjoin%
\definecolor{currentfill}{rgb}{0.271828,0.209303,0.504434}%
\pgfsetfillcolor{currentfill}%
\pgfsetfillopacity{0.700000}%
\pgfsetlinewidth{0.000000pt}%
\definecolor{currentstroke}{rgb}{0.000000,0.000000,0.000000}%
\pgfsetstrokecolor{currentstroke}%
\pgfsetdash{}{0pt}%
\pgfpathmoveto{\pgfqpoint{4.995670in}{2.845002in}}%
\pgfpathlineto{\pgfqpoint{5.009337in}{2.846465in}}%
\pgfpathlineto{\pgfqpoint{5.023014in}{2.848029in}}%
\pgfpathlineto{\pgfqpoint{5.036702in}{2.849694in}}%
\pgfpathlineto{\pgfqpoint{5.050400in}{2.851461in}}%
\pgfpathlineto{\pgfqpoint{5.043022in}{2.843977in}}%
\pgfpathlineto{\pgfqpoint{5.035638in}{2.836474in}}%
\pgfpathlineto{\pgfqpoint{5.028248in}{2.828949in}}%
\pgfpathlineto{\pgfqpoint{5.020853in}{2.821400in}}%
\pgfpathlineto{\pgfqpoint{5.007142in}{2.819500in}}%
\pgfpathlineto{\pgfqpoint{4.993441in}{2.817701in}}%
\pgfpathlineto{\pgfqpoint{4.979751in}{2.816004in}}%
\pgfpathlineto{\pgfqpoint{4.966072in}{2.814409in}}%
\pgfpathlineto{\pgfqpoint{4.973480in}{2.822084in}}%
\pgfpathlineto{\pgfqpoint{4.980882in}{2.829740in}}%
\pgfpathlineto{\pgfqpoint{4.988279in}{2.837379in}}%
\pgfpathlineto{\pgfqpoint{4.995670in}{2.845002in}}%
\pgfpathclose%
\pgfusepath{fill}%
\end{pgfscope}%
\begin{pgfscope}%
\pgfpathrectangle{\pgfqpoint{1.150000in}{0.150000in}}{\pgfqpoint{5.700000in}{5.700000in}}%
\pgfusepath{clip}%
\pgfsetbuttcap%
\pgfsetroundjoin%
\definecolor{currentfill}{rgb}{0.280868,0.160771,0.472899}%
\pgfsetfillcolor{currentfill}%
\pgfsetfillopacity{0.700000}%
\pgfsetlinewidth{0.000000pt}%
\definecolor{currentstroke}{rgb}{0.000000,0.000000,0.000000}%
\pgfsetstrokecolor{currentstroke}%
\pgfsetdash{}{0pt}%
\pgfpathmoveto{\pgfqpoint{3.110704in}{2.763588in}}%
\pgfpathlineto{\pgfqpoint{3.124012in}{2.751977in}}%
\pgfpathlineto{\pgfqpoint{3.137318in}{2.740534in}}%
\pgfpathlineto{\pgfqpoint{3.150623in}{2.729257in}}%
\pgfpathlineto{\pgfqpoint{3.163928in}{2.718145in}}%
\pgfpathlineto{\pgfqpoint{3.155848in}{2.711768in}}%
\pgfpathlineto{\pgfqpoint{3.147761in}{2.705473in}}%
\pgfpathlineto{\pgfqpoint{3.139666in}{2.699261in}}%
\pgfpathlineto{\pgfqpoint{3.131563in}{2.693132in}}%
\pgfpathlineto{\pgfqpoint{3.118238in}{2.704407in}}%
\pgfpathlineto{\pgfqpoint{3.104912in}{2.715846in}}%
\pgfpathlineto{\pgfqpoint{3.091585in}{2.727452in}}%
\pgfpathlineto{\pgfqpoint{3.078257in}{2.739225in}}%
\pgfpathlineto{\pgfqpoint{3.086381in}{2.745184in}}%
\pgfpathlineto{\pgfqpoint{3.094497in}{2.751232in}}%
\pgfpathlineto{\pgfqpoint{3.102604in}{2.757367in}}%
\pgfpathlineto{\pgfqpoint{3.110704in}{2.763588in}}%
\pgfpathclose%
\pgfusepath{fill}%
\end{pgfscope}%
\begin{pgfscope}%
\pgfpathrectangle{\pgfqpoint{1.150000in}{0.150000in}}{\pgfqpoint{5.700000in}{5.700000in}}%
\pgfusepath{clip}%
\pgfsetbuttcap%
\pgfsetroundjoin%
\definecolor{currentfill}{rgb}{0.281924,0.089666,0.412415}%
\pgfsetfillcolor{currentfill}%
\pgfsetfillopacity{0.700000}%
\pgfsetlinewidth{0.000000pt}%
\definecolor{currentstroke}{rgb}{0.000000,0.000000,0.000000}%
\pgfsetstrokecolor{currentstroke}%
\pgfsetdash{}{0pt}%
\pgfpathmoveto{\pgfqpoint{3.355592in}{2.625200in}}%
\pgfpathlineto{\pgfqpoint{3.368881in}{2.616260in}}%
\pgfpathlineto{\pgfqpoint{3.382172in}{2.607467in}}%
\pgfpathlineto{\pgfqpoint{3.395465in}{2.598822in}}%
\pgfpathlineto{\pgfqpoint{3.408759in}{2.590323in}}%
\pgfpathlineto{\pgfqpoint{3.400782in}{2.583091in}}%
\pgfpathlineto{\pgfqpoint{3.392798in}{2.575918in}}%
\pgfpathlineto{\pgfqpoint{3.384808in}{2.568804in}}%
\pgfpathlineto{\pgfqpoint{3.376811in}{2.561751in}}%
\pgfpathlineto{\pgfqpoint{3.363500in}{2.570391in}}%
\pgfpathlineto{\pgfqpoint{3.350190in}{2.579178in}}%
\pgfpathlineto{\pgfqpoint{3.336881in}{2.588112in}}%
\pgfpathlineto{\pgfqpoint{3.323574in}{2.597194in}}%
\pgfpathlineto{\pgfqpoint{3.331589in}{2.604099in}}%
\pgfpathlineto{\pgfqpoint{3.339597in}{2.611069in}}%
\pgfpathlineto{\pgfqpoint{3.347598in}{2.618103in}}%
\pgfpathlineto{\pgfqpoint{3.355592in}{2.625200in}}%
\pgfpathclose%
\pgfusepath{fill}%
\end{pgfscope}%
\begin{pgfscope}%
\pgfpathrectangle{\pgfqpoint{1.150000in}{0.150000in}}{\pgfqpoint{5.700000in}{5.700000in}}%
\pgfusepath{clip}%
\pgfsetbuttcap%
\pgfsetroundjoin%
\definecolor{currentfill}{rgb}{0.275191,0.194905,0.496005}%
\pgfsetfillcolor{currentfill}%
\pgfsetfillopacity{0.700000}%
\pgfsetlinewidth{0.000000pt}%
\definecolor{currentstroke}{rgb}{0.000000,0.000000,0.000000}%
\pgfsetstrokecolor{currentstroke}%
\pgfsetdash{}{0pt}%
\pgfpathmoveto{\pgfqpoint{4.911459in}{2.809050in}}%
\pgfpathlineto{\pgfqpoint{4.925097in}{2.810236in}}%
\pgfpathlineto{\pgfqpoint{4.938745in}{2.811524in}}%
\pgfpathlineto{\pgfqpoint{4.952403in}{2.812915in}}%
\pgfpathlineto{\pgfqpoint{4.966072in}{2.814409in}}%
\pgfpathlineto{\pgfqpoint{4.958659in}{2.806713in}}%
\pgfpathlineto{\pgfqpoint{4.951239in}{2.798995in}}%
\pgfpathlineto{\pgfqpoint{4.943815in}{2.791254in}}%
\pgfpathlineto{\pgfqpoint{4.936384in}{2.783488in}}%
\pgfpathlineto{\pgfqpoint{4.922703in}{2.781879in}}%
\pgfpathlineto{\pgfqpoint{4.909033in}{2.780373in}}%
\pgfpathlineto{\pgfqpoint{4.895373in}{2.778969in}}%
\pgfpathlineto{\pgfqpoint{4.881723in}{2.777669in}}%
\pgfpathlineto{\pgfqpoint{4.889165in}{2.785543in}}%
\pgfpathlineto{\pgfqpoint{4.896602in}{2.793397in}}%
\pgfpathlineto{\pgfqpoint{4.904033in}{2.801232in}}%
\pgfpathlineto{\pgfqpoint{4.911459in}{2.809050in}}%
\pgfpathclose%
\pgfusepath{fill}%
\end{pgfscope}%
\begin{pgfscope}%
\pgfpathrectangle{\pgfqpoint{1.150000in}{0.150000in}}{\pgfqpoint{5.700000in}{5.700000in}}%
\pgfusepath{clip}%
\pgfsetbuttcap%
\pgfsetroundjoin%
\definecolor{currentfill}{rgb}{0.282327,0.094955,0.417331}%
\pgfsetfillcolor{currentfill}%
\pgfsetfillopacity{0.700000}%
\pgfsetlinewidth{0.000000pt}%
\definecolor{currentstroke}{rgb}{0.000000,0.000000,0.000000}%
\pgfsetstrokecolor{currentstroke}%
\pgfsetdash{}{0pt}%
\pgfpathmoveto{\pgfqpoint{4.351800in}{2.611830in}}%
\pgfpathlineto{\pgfqpoint{4.365252in}{2.610448in}}%
\pgfpathlineto{\pgfqpoint{4.378712in}{2.609176in}}%
\pgfpathlineto{\pgfqpoint{4.392180in}{2.608015in}}%
\pgfpathlineto{\pgfqpoint{4.405656in}{2.606964in}}%
\pgfpathlineto{\pgfqpoint{4.398035in}{2.598318in}}%
\pgfpathlineto{\pgfqpoint{4.390409in}{2.589656in}}%
\pgfpathlineto{\pgfqpoint{4.382778in}{2.580980in}}%
\pgfpathlineto{\pgfqpoint{4.375142in}{2.572289in}}%
\pgfpathlineto{\pgfqpoint{4.361656in}{2.573335in}}%
\pgfpathlineto{\pgfqpoint{4.348178in}{2.574490in}}%
\pgfpathlineto{\pgfqpoint{4.334707in}{2.575756in}}%
\pgfpathlineto{\pgfqpoint{4.321245in}{2.577133in}}%
\pgfpathlineto{\pgfqpoint{4.328891in}{2.585823in}}%
\pgfpathlineto{\pgfqpoint{4.336533in}{2.594503in}}%
\pgfpathlineto{\pgfqpoint{4.344169in}{2.603172in}}%
\pgfpathlineto{\pgfqpoint{4.351800in}{2.611830in}}%
\pgfpathclose%
\pgfusepath{fill}%
\end{pgfscope}%
\begin{pgfscope}%
\pgfpathrectangle{\pgfqpoint{1.150000in}{0.150000in}}{\pgfqpoint{5.700000in}{5.700000in}}%
\pgfusepath{clip}%
\pgfsetbuttcap%
\pgfsetroundjoin%
\definecolor{currentfill}{rgb}{0.278012,0.180367,0.486697}%
\pgfsetfillcolor{currentfill}%
\pgfsetfillopacity{0.700000}%
\pgfsetlinewidth{0.000000pt}%
\definecolor{currentstroke}{rgb}{0.000000,0.000000,0.000000}%
\pgfsetstrokecolor{currentstroke}%
\pgfsetdash{}{0pt}%
\pgfpathmoveto{\pgfqpoint{4.827223in}{2.773497in}}%
\pgfpathlineto{\pgfqpoint{4.840833in}{2.774385in}}%
\pgfpathlineto{\pgfqpoint{4.854453in}{2.775376in}}%
\pgfpathlineto{\pgfqpoint{4.868083in}{2.776471in}}%
\pgfpathlineto{\pgfqpoint{4.881723in}{2.777669in}}%
\pgfpathlineto{\pgfqpoint{4.874275in}{2.769772in}}%
\pgfpathlineto{\pgfqpoint{4.866821in}{2.761852in}}%
\pgfpathlineto{\pgfqpoint{4.859362in}{2.753907in}}%
\pgfpathlineto{\pgfqpoint{4.851898in}{2.745936in}}%
\pgfpathlineto{\pgfqpoint{4.838246in}{2.744641in}}%
\pgfpathlineto{\pgfqpoint{4.824605in}{2.743449in}}%
\pgfpathlineto{\pgfqpoint{4.810974in}{2.742361in}}%
\pgfpathlineto{\pgfqpoint{4.797352in}{2.741377in}}%
\pgfpathlineto{\pgfqpoint{4.804828in}{2.749438in}}%
\pgfpathlineto{\pgfqpoint{4.812299in}{2.757477in}}%
\pgfpathlineto{\pgfqpoint{4.819764in}{2.765497in}}%
\pgfpathlineto{\pgfqpoint{4.827223in}{2.773497in}}%
\pgfpathclose%
\pgfusepath{fill}%
\end{pgfscope}%
\begin{pgfscope}%
\pgfpathrectangle{\pgfqpoint{1.150000in}{0.150000in}}{\pgfqpoint{5.700000in}{5.700000in}}%
\pgfusepath{clip}%
\pgfsetbuttcap%
\pgfsetroundjoin%
\definecolor{currentfill}{rgb}{0.276022,0.044167,0.370164}%
\pgfsetfillcolor{currentfill}%
\pgfsetfillopacity{0.700000}%
\pgfsetlinewidth{0.000000pt}%
\definecolor{currentstroke}{rgb}{0.000000,0.000000,0.000000}%
\pgfsetstrokecolor{currentstroke}%
\pgfsetdash{}{0pt}%
\pgfpathmoveto{\pgfqpoint{3.684861in}{2.536889in}}%
\pgfpathlineto{\pgfqpoint{3.698171in}{2.530957in}}%
\pgfpathlineto{\pgfqpoint{3.711485in}{2.525155in}}%
\pgfpathlineto{\pgfqpoint{3.724803in}{2.519483in}}%
\pgfpathlineto{\pgfqpoint{3.738125in}{2.513939in}}%
\pgfpathlineto{\pgfqpoint{3.730273in}{2.505787in}}%
\pgfpathlineto{\pgfqpoint{3.722415in}{2.497664in}}%
\pgfpathlineto{\pgfqpoint{3.714552in}{2.489571in}}%
\pgfpathlineto{\pgfqpoint{3.706682in}{2.481508in}}%
\pgfpathlineto{\pgfqpoint{3.693347in}{2.487155in}}%
\pgfpathlineto{\pgfqpoint{3.680015in}{2.492930in}}%
\pgfpathlineto{\pgfqpoint{3.666687in}{2.498835in}}%
\pgfpathlineto{\pgfqpoint{3.653363in}{2.504871in}}%
\pgfpathlineto{\pgfqpoint{3.661246in}{2.512824in}}%
\pgfpathlineto{\pgfqpoint{3.669124in}{2.520812in}}%
\pgfpathlineto{\pgfqpoint{3.676995in}{2.528834in}}%
\pgfpathlineto{\pgfqpoint{3.684861in}{2.536889in}}%
\pgfpathclose%
\pgfusepath{fill}%
\end{pgfscope}%
\begin{pgfscope}%
\pgfpathrectangle{\pgfqpoint{1.150000in}{0.150000in}}{\pgfqpoint{5.700000in}{5.700000in}}%
\pgfusepath{clip}%
\pgfsetbuttcap%
\pgfsetroundjoin%
\definecolor{currentfill}{rgb}{0.277941,0.056324,0.381191}%
\pgfsetfillcolor{currentfill}%
\pgfsetfillopacity{0.700000}%
\pgfsetlinewidth{0.000000pt}%
\definecolor{currentstroke}{rgb}{0.000000,0.000000,0.000000}%
\pgfsetstrokecolor{currentstroke}%
\pgfsetdash{}{0pt}%
\pgfpathmoveto{\pgfqpoint{4.045118in}{2.545325in}}%
\pgfpathlineto{\pgfqpoint{4.058491in}{2.542074in}}%
\pgfpathlineto{\pgfqpoint{4.071871in}{2.538941in}}%
\pgfpathlineto{\pgfqpoint{4.085256in}{2.535925in}}%
\pgfpathlineto{\pgfqpoint{4.098648in}{2.533027in}}%
\pgfpathlineto{\pgfqpoint{4.090922in}{2.524355in}}%
\pgfpathlineto{\pgfqpoint{4.083191in}{2.515686in}}%
\pgfpathlineto{\pgfqpoint{4.075455in}{2.507018in}}%
\pgfpathlineto{\pgfqpoint{4.067713in}{2.498352in}}%
\pgfpathlineto{\pgfqpoint{4.054310in}{2.501299in}}%
\pgfpathlineto{\pgfqpoint{4.040913in}{2.504364in}}%
\pgfpathlineto{\pgfqpoint{4.027523in}{2.507545in}}%
\pgfpathlineto{\pgfqpoint{4.014138in}{2.510845in}}%
\pgfpathlineto{\pgfqpoint{4.021891in}{2.519456in}}%
\pgfpathlineto{\pgfqpoint{4.029639in}{2.528073in}}%
\pgfpathlineto{\pgfqpoint{4.037381in}{2.536696in}}%
\pgfpathlineto{\pgfqpoint{4.045118in}{2.545325in}}%
\pgfpathclose%
\pgfusepath{fill}%
\end{pgfscope}%
\begin{pgfscope}%
\pgfpathrectangle{\pgfqpoint{1.150000in}{0.150000in}}{\pgfqpoint{5.700000in}{5.700000in}}%
\pgfusepath{clip}%
\pgfsetbuttcap%
\pgfsetroundjoin%
\definecolor{currentfill}{rgb}{0.282623,0.140926,0.457517}%
\pgfsetfillcolor{currentfill}%
\pgfsetfillopacity{0.700000}%
\pgfsetlinewidth{0.000000pt}%
\definecolor{currentstroke}{rgb}{0.000000,0.000000,0.000000}%
\pgfsetstrokecolor{currentstroke}%
\pgfsetdash{}{0pt}%
\pgfpathmoveto{\pgfqpoint{3.163928in}{2.718145in}}%
\pgfpathlineto{\pgfqpoint{3.177232in}{2.707196in}}%
\pgfpathlineto{\pgfqpoint{3.190535in}{2.696409in}}%
\pgfpathlineto{\pgfqpoint{3.203838in}{2.685784in}}%
\pgfpathlineto{\pgfqpoint{3.217141in}{2.675318in}}%
\pgfpathlineto{\pgfqpoint{3.209082in}{2.668786in}}%
\pgfpathlineto{\pgfqpoint{3.201015in}{2.662332in}}%
\pgfpathlineto{\pgfqpoint{3.192940in}{2.655956in}}%
\pgfpathlineto{\pgfqpoint{3.184857in}{2.649659in}}%
\pgfpathlineto{\pgfqpoint{3.171535in}{2.660286in}}%
\pgfpathlineto{\pgfqpoint{3.158211in}{2.671074in}}%
\pgfpathlineto{\pgfqpoint{3.144888in}{2.682022in}}%
\pgfpathlineto{\pgfqpoint{3.131563in}{2.693132in}}%
\pgfpathlineto{\pgfqpoint{3.139666in}{2.699261in}}%
\pgfpathlineto{\pgfqpoint{3.147761in}{2.705473in}}%
\pgfpathlineto{\pgfqpoint{3.155848in}{2.711768in}}%
\pgfpathlineto{\pgfqpoint{3.163928in}{2.718145in}}%
\pgfpathclose%
\pgfusepath{fill}%
\end{pgfscope}%
\begin{pgfscope}%
\pgfpathrectangle{\pgfqpoint{1.150000in}{0.150000in}}{\pgfqpoint{5.700000in}{5.700000in}}%
\pgfusepath{clip}%
\pgfsetbuttcap%
\pgfsetroundjoin%
\definecolor{currentfill}{rgb}{0.277941,0.056324,0.381191}%
\pgfsetfillcolor{currentfill}%
\pgfsetfillopacity{0.700000}%
\pgfsetlinewidth{0.000000pt}%
\definecolor{currentstroke}{rgb}{0.000000,0.000000,0.000000}%
\pgfsetstrokecolor{currentstroke}%
\pgfsetdash{}{0pt}%
\pgfpathmoveto{\pgfqpoint{3.546893in}{2.557940in}}%
\pgfpathlineto{\pgfqpoint{3.560191in}{2.550834in}}%
\pgfpathlineto{\pgfqpoint{3.573492in}{2.543864in}}%
\pgfpathlineto{\pgfqpoint{3.586796in}{2.537030in}}%
\pgfpathlineto{\pgfqpoint{3.600102in}{2.530332in}}%
\pgfpathlineto{\pgfqpoint{3.592199in}{2.522529in}}%
\pgfpathlineto{\pgfqpoint{3.584289in}{2.514768in}}%
\pgfpathlineto{\pgfqpoint{3.576372in}{2.507049in}}%
\pgfpathlineto{\pgfqpoint{3.568450in}{2.499372in}}%
\pgfpathlineto{\pgfqpoint{3.555128in}{2.506193in}}%
\pgfpathlineto{\pgfqpoint{3.541809in}{2.513148in}}%
\pgfpathlineto{\pgfqpoint{3.528493in}{2.520240in}}%
\pgfpathlineto{\pgfqpoint{3.515180in}{2.527468in}}%
\pgfpathlineto{\pgfqpoint{3.523118in}{2.535016in}}%
\pgfpathlineto{\pgfqpoint{3.531049in}{2.542611in}}%
\pgfpathlineto{\pgfqpoint{3.538974in}{2.550253in}}%
\pgfpathlineto{\pgfqpoint{3.546893in}{2.557940in}}%
\pgfpathclose%
\pgfusepath{fill}%
\end{pgfscope}%
\begin{pgfscope}%
\pgfpathrectangle{\pgfqpoint{1.150000in}{0.150000in}}{\pgfqpoint{5.700000in}{5.700000in}}%
\pgfusepath{clip}%
\pgfsetbuttcap%
\pgfsetroundjoin%
\definecolor{currentfill}{rgb}{0.276022,0.044167,0.370164}%
\pgfsetfillcolor{currentfill}%
\pgfsetfillopacity{0.700000}%
\pgfsetlinewidth{0.000000pt}%
\definecolor{currentstroke}{rgb}{0.000000,0.000000,0.000000}%
\pgfsetstrokecolor{currentstroke}%
\pgfsetdash{}{0pt}%
\pgfpathmoveto{\pgfqpoint{3.822755in}{2.526264in}}%
\pgfpathlineto{\pgfqpoint{3.836087in}{2.521440in}}%
\pgfpathlineto{\pgfqpoint{3.849423in}{2.516740in}}%
\pgfpathlineto{\pgfqpoint{3.862764in}{2.512165in}}%
\pgfpathlineto{\pgfqpoint{3.876110in}{2.507712in}}%
\pgfpathlineto{\pgfqpoint{3.868306in}{2.499295in}}%
\pgfpathlineto{\pgfqpoint{3.860497in}{2.490894in}}%
\pgfpathlineto{\pgfqpoint{3.852682in}{2.482512in}}%
\pgfpathlineto{\pgfqpoint{3.844862in}{2.474148in}}%
\pgfpathlineto{\pgfqpoint{3.831503in}{2.478685in}}%
\pgfpathlineto{\pgfqpoint{3.818149in}{2.483346in}}%
\pgfpathlineto{\pgfqpoint{3.804801in}{2.488130in}}%
\pgfpathlineto{\pgfqpoint{3.791456in}{2.493040in}}%
\pgfpathlineto{\pgfqpoint{3.799290in}{2.501312in}}%
\pgfpathlineto{\pgfqpoint{3.807117in}{2.509607in}}%
\pgfpathlineto{\pgfqpoint{3.814939in}{2.517925in}}%
\pgfpathlineto{\pgfqpoint{3.822755in}{2.526264in}}%
\pgfpathclose%
\pgfusepath{fill}%
\end{pgfscope}%
\begin{pgfscope}%
\pgfpathrectangle{\pgfqpoint{1.150000in}{0.150000in}}{\pgfqpoint{5.700000in}{5.700000in}}%
\pgfusepath{clip}%
\pgfsetbuttcap%
\pgfsetroundjoin%
\definecolor{currentfill}{rgb}{0.280868,0.160771,0.472899}%
\pgfsetfillcolor{currentfill}%
\pgfsetfillopacity{0.700000}%
\pgfsetlinewidth{0.000000pt}%
\definecolor{currentstroke}{rgb}{0.000000,0.000000,0.000000}%
\pgfsetstrokecolor{currentstroke}%
\pgfsetdash{}{0pt}%
\pgfpathmoveto{\pgfqpoint{4.742964in}{2.738480in}}%
\pgfpathlineto{\pgfqpoint{4.756546in}{2.739048in}}%
\pgfpathlineto{\pgfqpoint{4.770139in}{2.739720in}}%
\pgfpathlineto{\pgfqpoint{4.783741in}{2.740496in}}%
\pgfpathlineto{\pgfqpoint{4.797352in}{2.741377in}}%
\pgfpathlineto{\pgfqpoint{4.789871in}{2.733293in}}%
\pgfpathlineto{\pgfqpoint{4.782384in}{2.725185in}}%
\pgfpathlineto{\pgfqpoint{4.774891in}{2.717052in}}%
\pgfpathlineto{\pgfqpoint{4.767393in}{2.708892in}}%
\pgfpathlineto{\pgfqpoint{4.753770in}{2.707933in}}%
\pgfpathlineto{\pgfqpoint{4.740158in}{2.707078in}}%
\pgfpathlineto{\pgfqpoint{4.726554in}{2.706327in}}%
\pgfpathlineto{\pgfqpoint{4.712961in}{2.705682in}}%
\pgfpathlineto{\pgfqpoint{4.720470in}{2.713913in}}%
\pgfpathlineto{\pgfqpoint{4.727973in}{2.722123in}}%
\pgfpathlineto{\pgfqpoint{4.735471in}{2.730311in}}%
\pgfpathlineto{\pgfqpoint{4.742964in}{2.738480in}}%
\pgfpathclose%
\pgfusepath{fill}%
\end{pgfscope}%
\begin{pgfscope}%
\pgfpathrectangle{\pgfqpoint{1.150000in}{0.150000in}}{\pgfqpoint{5.700000in}{5.700000in}}%
\pgfusepath{clip}%
\pgfsetbuttcap%
\pgfsetroundjoin%
\definecolor{currentfill}{rgb}{0.280894,0.078907,0.402329}%
\pgfsetfillcolor{currentfill}%
\pgfsetfillopacity{0.700000}%
\pgfsetlinewidth{0.000000pt}%
\definecolor{currentstroke}{rgb}{0.000000,0.000000,0.000000}%
\pgfsetstrokecolor{currentstroke}%
\pgfsetdash{}{0pt}%
\pgfpathmoveto{\pgfqpoint{4.267470in}{2.583757in}}%
\pgfpathlineto{\pgfqpoint{4.280903in}{2.581933in}}%
\pgfpathlineto{\pgfqpoint{4.294342in}{2.580221in}}%
\pgfpathlineto{\pgfqpoint{4.307790in}{2.578621in}}%
\pgfpathlineto{\pgfqpoint{4.321245in}{2.577133in}}%
\pgfpathlineto{\pgfqpoint{4.313593in}{2.568432in}}%
\pgfpathlineto{\pgfqpoint{4.305935in}{2.559719in}}%
\pgfpathlineto{\pgfqpoint{4.298273in}{2.550995in}}%
\pgfpathlineto{\pgfqpoint{4.290605in}{2.542258in}}%
\pgfpathlineto{\pgfqpoint{4.277140in}{2.543759in}}%
\pgfpathlineto{\pgfqpoint{4.263682in}{2.545371in}}%
\pgfpathlineto{\pgfqpoint{4.250232in}{2.547096in}}%
\pgfpathlineto{\pgfqpoint{4.236789in}{2.548932in}}%
\pgfpathlineto{\pgfqpoint{4.244467in}{2.557650in}}%
\pgfpathlineto{\pgfqpoint{4.252140in}{2.566359in}}%
\pgfpathlineto{\pgfqpoint{4.259808in}{2.575062in}}%
\pgfpathlineto{\pgfqpoint{4.267470in}{2.583757in}}%
\pgfpathclose%
\pgfusepath{fill}%
\end{pgfscope}%
\begin{pgfscope}%
\pgfpathrectangle{\pgfqpoint{1.150000in}{0.150000in}}{\pgfqpoint{5.700000in}{5.700000in}}%
\pgfusepath{clip}%
\pgfsetbuttcap%
\pgfsetroundjoin%
\definecolor{currentfill}{rgb}{0.280894,0.078907,0.402329}%
\pgfsetfillcolor{currentfill}%
\pgfsetfillopacity{0.700000}%
\pgfsetlinewidth{0.000000pt}%
\definecolor{currentstroke}{rgb}{0.000000,0.000000,0.000000}%
\pgfsetstrokecolor{currentstroke}%
\pgfsetdash{}{0pt}%
\pgfpathmoveto{\pgfqpoint{3.408759in}{2.590323in}}%
\pgfpathlineto{\pgfqpoint{3.422054in}{2.581968in}}%
\pgfpathlineto{\pgfqpoint{3.435351in}{2.573758in}}%
\pgfpathlineto{\pgfqpoint{3.448651in}{2.565691in}}%
\pgfpathlineto{\pgfqpoint{3.461952in}{2.557766in}}%
\pgfpathlineto{\pgfqpoint{3.453992in}{2.550400in}}%
\pgfpathlineto{\pgfqpoint{3.446025in}{2.543088in}}%
\pgfpathlineto{\pgfqpoint{3.438052in}{2.535831in}}%
\pgfpathlineto{\pgfqpoint{3.430072in}{2.528631in}}%
\pgfpathlineto{\pgfqpoint{3.416754in}{2.536696in}}%
\pgfpathlineto{\pgfqpoint{3.403438in}{2.544904in}}%
\pgfpathlineto{\pgfqpoint{3.390123in}{2.553256in}}%
\pgfpathlineto{\pgfqpoint{3.376811in}{2.561751in}}%
\pgfpathlineto{\pgfqpoint{3.384808in}{2.568804in}}%
\pgfpathlineto{\pgfqpoint{3.392798in}{2.575918in}}%
\pgfpathlineto{\pgfqpoint{3.400782in}{2.583091in}}%
\pgfpathlineto{\pgfqpoint{3.408759in}{2.590323in}}%
\pgfpathclose%
\pgfusepath{fill}%
\end{pgfscope}%
\begin{pgfscope}%
\pgfpathrectangle{\pgfqpoint{1.150000in}{0.150000in}}{\pgfqpoint{5.700000in}{5.700000in}}%
\pgfusepath{clip}%
\pgfsetbuttcap%
\pgfsetroundjoin%
\definecolor{currentfill}{rgb}{0.282290,0.145912,0.461510}%
\pgfsetfillcolor{currentfill}%
\pgfsetfillopacity{0.700000}%
\pgfsetlinewidth{0.000000pt}%
\definecolor{currentstroke}{rgb}{0.000000,0.000000,0.000000}%
\pgfsetstrokecolor{currentstroke}%
\pgfsetdash{}{0pt}%
\pgfpathmoveto{\pgfqpoint{4.658680in}{2.704150in}}%
\pgfpathlineto{\pgfqpoint{4.672236in}{2.704375in}}%
\pgfpathlineto{\pgfqpoint{4.685801in}{2.704705in}}%
\pgfpathlineto{\pgfqpoint{4.699376in}{2.705141in}}%
\pgfpathlineto{\pgfqpoint{4.712961in}{2.705682in}}%
\pgfpathlineto{\pgfqpoint{4.705446in}{2.697427in}}%
\pgfpathlineto{\pgfqpoint{4.697926in}{2.689149in}}%
\pgfpathlineto{\pgfqpoint{4.690401in}{2.680846in}}%
\pgfpathlineto{\pgfqpoint{4.682870in}{2.672517in}}%
\pgfpathlineto{\pgfqpoint{4.669275in}{2.671916in}}%
\pgfpathlineto{\pgfqpoint{4.655689in}{2.671420in}}%
\pgfpathlineto{\pgfqpoint{4.642113in}{2.671029in}}%
\pgfpathlineto{\pgfqpoint{4.628546in}{2.670744in}}%
\pgfpathlineto{\pgfqpoint{4.636088in}{2.679127in}}%
\pgfpathlineto{\pgfqpoint{4.643624in}{2.687488in}}%
\pgfpathlineto{\pgfqpoint{4.651154in}{2.695829in}}%
\pgfpathlineto{\pgfqpoint{4.658680in}{2.704150in}}%
\pgfpathclose%
\pgfusepath{fill}%
\end{pgfscope}%
\begin{pgfscope}%
\pgfpathrectangle{\pgfqpoint{1.150000in}{0.150000in}}{\pgfqpoint{5.700000in}{5.700000in}}%
\pgfusepath{clip}%
\pgfsetbuttcap%
\pgfsetroundjoin%
\definecolor{currentfill}{rgb}{0.283229,0.120777,0.440584}%
\pgfsetfillcolor{currentfill}%
\pgfsetfillopacity{0.700000}%
\pgfsetlinewidth{0.000000pt}%
\definecolor{currentstroke}{rgb}{0.000000,0.000000,0.000000}%
\pgfsetstrokecolor{currentstroke}%
\pgfsetdash{}{0pt}%
\pgfpathmoveto{\pgfqpoint{3.217141in}{2.675318in}}%
\pgfpathlineto{\pgfqpoint{3.230444in}{2.665010in}}%
\pgfpathlineto{\pgfqpoint{3.243747in}{2.654860in}}%
\pgfpathlineto{\pgfqpoint{3.257050in}{2.644865in}}%
\pgfpathlineto{\pgfqpoint{3.270354in}{2.635026in}}%
\pgfpathlineto{\pgfqpoint{3.262313in}{2.628341in}}%
\pgfpathlineto{\pgfqpoint{3.254266in}{2.621728in}}%
\pgfpathlineto{\pgfqpoint{3.246210in}{2.615189in}}%
\pgfpathlineto{\pgfqpoint{3.238148in}{2.608725in}}%
\pgfpathlineto{\pgfqpoint{3.224825in}{2.618725in}}%
\pgfpathlineto{\pgfqpoint{3.211502in}{2.628880in}}%
\pgfpathlineto{\pgfqpoint{3.198180in}{2.639191in}}%
\pgfpathlineto{\pgfqpoint{3.184857in}{2.649659in}}%
\pgfpathlineto{\pgfqpoint{3.192940in}{2.655956in}}%
\pgfpathlineto{\pgfqpoint{3.201015in}{2.662332in}}%
\pgfpathlineto{\pgfqpoint{3.209082in}{2.668786in}}%
\pgfpathlineto{\pgfqpoint{3.217141in}{2.675318in}}%
\pgfpathclose%
\pgfusepath{fill}%
\end{pgfscope}%
\begin{pgfscope}%
\pgfpathrectangle{\pgfqpoint{1.150000in}{0.150000in}}{\pgfqpoint{5.700000in}{5.700000in}}%
\pgfusepath{clip}%
\pgfsetbuttcap%
\pgfsetroundjoin%
\definecolor{currentfill}{rgb}{0.243113,0.292092,0.538516}%
\pgfsetfillcolor{currentfill}%
\pgfsetfillopacity{0.700000}%
\pgfsetlinewidth{0.000000pt}%
\definecolor{currentstroke}{rgb}{0.000000,0.000000,0.000000}%
\pgfsetstrokecolor{currentstroke}%
\pgfsetdash{}{0pt}%
\pgfpathmoveto{\pgfqpoint{5.387477in}{3.000532in}}%
\pgfpathlineto{\pgfqpoint{5.401308in}{3.003277in}}%
\pgfpathlineto{\pgfqpoint{5.415152in}{3.006121in}}%
\pgfpathlineto{\pgfqpoint{5.429007in}{3.009062in}}%
\pgfpathlineto{\pgfqpoint{5.442874in}{3.012102in}}%
\pgfpathlineto{\pgfqpoint{5.435659in}{3.005730in}}%
\pgfpathlineto{\pgfqpoint{5.428439in}{2.999350in}}%
\pgfpathlineto{\pgfqpoint{5.421214in}{2.992959in}}%
\pgfpathlineto{\pgfqpoint{5.413982in}{2.986555in}}%
\pgfpathlineto{\pgfqpoint{5.400098in}{2.983307in}}%
\pgfpathlineto{\pgfqpoint{5.386227in}{2.980157in}}%
\pgfpathlineto{\pgfqpoint{5.372367in}{2.977106in}}%
\pgfpathlineto{\pgfqpoint{5.358519in}{2.974153in}}%
\pgfpathlineto{\pgfqpoint{5.365767in}{2.980758in}}%
\pgfpathlineto{\pgfqpoint{5.373009in}{2.987355in}}%
\pgfpathlineto{\pgfqpoint{5.380246in}{2.993945in}}%
\pgfpathlineto{\pgfqpoint{5.387477in}{3.000532in}}%
\pgfpathclose%
\pgfusepath{fill}%
\end{pgfscope}%
\begin{pgfscope}%
\pgfpathrectangle{\pgfqpoint{1.150000in}{0.150000in}}{\pgfqpoint{5.700000in}{5.700000in}}%
\pgfusepath{clip}%
\pgfsetbuttcap%
\pgfsetroundjoin%
\definecolor{currentfill}{rgb}{0.276022,0.044167,0.370164}%
\pgfsetfillcolor{currentfill}%
\pgfsetfillopacity{0.700000}%
\pgfsetlinewidth{0.000000pt}%
\definecolor{currentstroke}{rgb}{0.000000,0.000000,0.000000}%
\pgfsetstrokecolor{currentstroke}%
\pgfsetdash{}{0pt}%
\pgfpathmoveto{\pgfqpoint{3.960659in}{2.525233in}}%
\pgfpathlineto{\pgfqpoint{3.974020in}{2.521457in}}%
\pgfpathlineto{\pgfqpoint{3.987387in}{2.517800in}}%
\pgfpathlineto{\pgfqpoint{4.000760in}{2.514263in}}%
\pgfpathlineto{\pgfqpoint{4.014138in}{2.510845in}}%
\pgfpathlineto{\pgfqpoint{4.006380in}{2.502241in}}%
\pgfpathlineto{\pgfqpoint{3.998616in}{2.493644in}}%
\pgfpathlineto{\pgfqpoint{3.990847in}{2.485055in}}%
\pgfpathlineto{\pgfqpoint{3.983072in}{2.476473in}}%
\pgfpathlineto{\pgfqpoint{3.969682in}{2.479957in}}%
\pgfpathlineto{\pgfqpoint{3.956298in}{2.483561in}}%
\pgfpathlineto{\pgfqpoint{3.942920in}{2.487284in}}%
\pgfpathlineto{\pgfqpoint{3.929547in}{2.491128in}}%
\pgfpathlineto{\pgfqpoint{3.937333in}{2.499636in}}%
\pgfpathlineto{\pgfqpoint{3.945114in}{2.508157in}}%
\pgfpathlineto{\pgfqpoint{3.952889in}{2.516689in}}%
\pgfpathlineto{\pgfqpoint{3.960659in}{2.525233in}}%
\pgfpathclose%
\pgfusepath{fill}%
\end{pgfscope}%
\begin{pgfscope}%
\pgfpathrectangle{\pgfqpoint{1.150000in}{0.150000in}}{\pgfqpoint{5.700000in}{5.700000in}}%
\pgfusepath{clip}%
\pgfsetbuttcap%
\pgfsetroundjoin%
\definecolor{currentfill}{rgb}{0.257322,0.256130,0.526563}%
\pgfsetfillcolor{currentfill}%
\pgfsetfillopacity{0.700000}%
\pgfsetlinewidth{0.000000pt}%
\definecolor{currentstroke}{rgb}{0.000000,0.000000,0.000000}%
\pgfsetstrokecolor{currentstroke}%
\pgfsetdash{}{0pt}%
\pgfpathmoveto{\pgfqpoint{2.864702in}{2.951636in}}%
\pgfpathlineto{\pgfqpoint{2.878072in}{2.936971in}}%
\pgfpathlineto{\pgfqpoint{2.891439in}{2.922498in}}%
\pgfpathlineto{\pgfqpoint{2.904801in}{2.908218in}}%
\pgfpathlineto{\pgfqpoint{2.918161in}{2.894127in}}%
\pgfpathlineto{\pgfqpoint{2.909960in}{2.888785in}}%
\pgfpathlineto{\pgfqpoint{2.901750in}{2.883549in}}%
\pgfpathlineto{\pgfqpoint{2.893531in}{2.878420in}}%
\pgfpathlineto{\pgfqpoint{2.885303in}{2.873400in}}%
\pgfpathlineto{\pgfqpoint{2.871919in}{2.887675in}}%
\pgfpathlineto{\pgfqpoint{2.858531in}{2.902140in}}%
\pgfpathlineto{\pgfqpoint{2.845140in}{2.916798in}}%
\pgfpathlineto{\pgfqpoint{2.831744in}{2.931649in}}%
\pgfpathlineto{\pgfqpoint{2.839998in}{2.936477in}}%
\pgfpathlineto{\pgfqpoint{2.848242in}{2.941419in}}%
\pgfpathlineto{\pgfqpoint{2.856477in}{2.946472in}}%
\pgfpathlineto{\pgfqpoint{2.864702in}{2.951636in}}%
\pgfpathclose%
\pgfusepath{fill}%
\end{pgfscope}%
\begin{pgfscope}%
\pgfpathrectangle{\pgfqpoint{1.150000in}{0.150000in}}{\pgfqpoint{5.700000in}{5.700000in}}%
\pgfusepath{clip}%
\pgfsetbuttcap%
\pgfsetroundjoin%
\definecolor{currentfill}{rgb}{0.283072,0.130895,0.449241}%
\pgfsetfillcolor{currentfill}%
\pgfsetfillopacity{0.700000}%
\pgfsetlinewidth{0.000000pt}%
\definecolor{currentstroke}{rgb}{0.000000,0.000000,0.000000}%
\pgfsetstrokecolor{currentstroke}%
\pgfsetdash{}{0pt}%
\pgfpathmoveto{\pgfqpoint{4.574369in}{2.670668in}}%
\pgfpathlineto{\pgfqpoint{4.587900in}{2.670527in}}%
\pgfpathlineto{\pgfqpoint{4.601440in}{2.670493in}}%
\pgfpathlineto{\pgfqpoint{4.614989in}{2.670566in}}%
\pgfpathlineto{\pgfqpoint{4.628546in}{2.670744in}}%
\pgfpathlineto{\pgfqpoint{4.620999in}{2.662340in}}%
\pgfpathlineto{\pgfqpoint{4.613447in}{2.653912in}}%
\pgfpathlineto{\pgfqpoint{4.605889in}{2.645460in}}%
\pgfpathlineto{\pgfqpoint{4.598326in}{2.636984in}}%
\pgfpathlineto{\pgfqpoint{4.584758in}{2.636763in}}%
\pgfpathlineto{\pgfqpoint{4.571199in}{2.636648in}}%
\pgfpathlineto{\pgfqpoint{4.557649in}{2.636640in}}%
\pgfpathlineto{\pgfqpoint{4.544107in}{2.636739in}}%
\pgfpathlineto{\pgfqpoint{4.551681in}{2.645251in}}%
\pgfpathlineto{\pgfqpoint{4.559249in}{2.653742in}}%
\pgfpathlineto{\pgfqpoint{4.566812in}{2.662214in}}%
\pgfpathlineto{\pgfqpoint{4.574369in}{2.670668in}}%
\pgfpathclose%
\pgfusepath{fill}%
\end{pgfscope}%
\begin{pgfscope}%
\pgfpathrectangle{\pgfqpoint{1.150000in}{0.150000in}}{\pgfqpoint{5.700000in}{5.700000in}}%
\pgfusepath{clip}%
\pgfsetbuttcap%
\pgfsetroundjoin%
\definecolor{currentfill}{rgb}{0.250425,0.274290,0.533103}%
\pgfsetfillcolor{currentfill}%
\pgfsetfillopacity{0.700000}%
\pgfsetlinewidth{0.000000pt}%
\definecolor{currentstroke}{rgb}{0.000000,0.000000,0.000000}%
\pgfsetstrokecolor{currentstroke}%
\pgfsetdash{}{0pt}%
\pgfpathmoveto{\pgfqpoint{5.303247in}{2.963329in}}%
\pgfpathlineto{\pgfqpoint{5.317047in}{2.965887in}}%
\pgfpathlineto{\pgfqpoint{5.330860in}{2.968543in}}%
\pgfpathlineto{\pgfqpoint{5.344684in}{2.971299in}}%
\pgfpathlineto{\pgfqpoint{5.358519in}{2.974153in}}%
\pgfpathlineto{\pgfqpoint{5.351266in}{2.967537in}}%
\pgfpathlineto{\pgfqpoint{5.344007in}{2.960908in}}%
\pgfpathlineto{\pgfqpoint{5.336742in}{2.954263in}}%
\pgfpathlineto{\pgfqpoint{5.329472in}{2.947600in}}%
\pgfpathlineto{\pgfqpoint{5.315620in}{2.944556in}}%
\pgfpathlineto{\pgfqpoint{5.301781in}{2.941611in}}%
\pgfpathlineto{\pgfqpoint{5.287953in}{2.938765in}}%
\pgfpathlineto{\pgfqpoint{5.274138in}{2.936018in}}%
\pgfpathlineto{\pgfqpoint{5.281423in}{2.942864in}}%
\pgfpathlineto{\pgfqpoint{5.288703in}{2.949696in}}%
\pgfpathlineto{\pgfqpoint{5.295978in}{2.956517in}}%
\pgfpathlineto{\pgfqpoint{5.303247in}{2.963329in}}%
\pgfpathclose%
\pgfusepath{fill}%
\end{pgfscope}%
\begin{pgfscope}%
\pgfpathrectangle{\pgfqpoint{1.150000in}{0.150000in}}{\pgfqpoint{5.700000in}{5.700000in}}%
\pgfusepath{clip}%
\pgfsetbuttcap%
\pgfsetroundjoin%
\definecolor{currentfill}{rgb}{0.279566,0.067836,0.391917}%
\pgfsetfillcolor{currentfill}%
\pgfsetfillopacity{0.700000}%
\pgfsetlinewidth{0.000000pt}%
\definecolor{currentstroke}{rgb}{0.000000,0.000000,0.000000}%
\pgfsetstrokecolor{currentstroke}%
\pgfsetdash{}{0pt}%
\pgfpathmoveto{\pgfqpoint{4.183089in}{2.557412in}}%
\pgfpathlineto{\pgfqpoint{4.196503in}{2.555122in}}%
\pgfpathlineto{\pgfqpoint{4.209925in}{2.552945in}}%
\pgfpathlineto{\pgfqpoint{4.223353in}{2.550882in}}%
\pgfpathlineto{\pgfqpoint{4.236789in}{2.548932in}}%
\pgfpathlineto{\pgfqpoint{4.229106in}{2.540208in}}%
\pgfpathlineto{\pgfqpoint{4.221417in}{2.531476in}}%
\pgfpathlineto{\pgfqpoint{4.213723in}{2.522736in}}%
\pgfpathlineto{\pgfqpoint{4.206024in}{2.513989in}}%
\pgfpathlineto{\pgfqpoint{4.192578in}{2.515969in}}%
\pgfpathlineto{\pgfqpoint{4.179139in}{2.518063in}}%
\pgfpathlineto{\pgfqpoint{4.165707in}{2.520270in}}%
\pgfpathlineto{\pgfqpoint{4.152281in}{2.522591in}}%
\pgfpathlineto{\pgfqpoint{4.159991in}{2.531301in}}%
\pgfpathlineto{\pgfqpoint{4.167696in}{2.540008in}}%
\pgfpathlineto{\pgfqpoint{4.175395in}{2.548712in}}%
\pgfpathlineto{\pgfqpoint{4.183089in}{2.557412in}}%
\pgfpathclose%
\pgfusepath{fill}%
\end{pgfscope}%
\begin{pgfscope}%
\pgfpathrectangle{\pgfqpoint{1.150000in}{0.150000in}}{\pgfqpoint{5.700000in}{5.700000in}}%
\pgfusepath{clip}%
\pgfsetbuttcap%
\pgfsetroundjoin%
\definecolor{currentfill}{rgb}{0.266580,0.228262,0.514349}%
\pgfsetfillcolor{currentfill}%
\pgfsetfillopacity{0.700000}%
\pgfsetlinewidth{0.000000pt}%
\definecolor{currentstroke}{rgb}{0.000000,0.000000,0.000000}%
\pgfsetstrokecolor{currentstroke}%
\pgfsetdash{}{0pt}%
\pgfpathmoveto{\pgfqpoint{2.918161in}{2.894127in}}%
\pgfpathlineto{\pgfqpoint{2.931517in}{2.880224in}}%
\pgfpathlineto{\pgfqpoint{2.944869in}{2.866507in}}%
\pgfpathlineto{\pgfqpoint{2.958219in}{2.852975in}}%
\pgfpathlineto{\pgfqpoint{2.971566in}{2.839626in}}%
\pgfpathlineto{\pgfqpoint{2.963388in}{2.834107in}}%
\pgfpathlineto{\pgfqpoint{2.955202in}{2.828690in}}%
\pgfpathlineto{\pgfqpoint{2.947007in}{2.823374in}}%
\pgfpathlineto{\pgfqpoint{2.938804in}{2.818163in}}%
\pgfpathlineto{\pgfqpoint{2.925433in}{2.831696in}}%
\pgfpathlineto{\pgfqpoint{2.912060in}{2.845412in}}%
\pgfpathlineto{\pgfqpoint{2.898683in}{2.859312in}}%
\pgfpathlineto{\pgfqpoint{2.885303in}{2.873400in}}%
\pgfpathlineto{\pgfqpoint{2.893531in}{2.878420in}}%
\pgfpathlineto{\pgfqpoint{2.901750in}{2.883549in}}%
\pgfpathlineto{\pgfqpoint{2.909960in}{2.888785in}}%
\pgfpathlineto{\pgfqpoint{2.918161in}{2.894127in}}%
\pgfpathclose%
\pgfusepath{fill}%
\end{pgfscope}%
\begin{pgfscope}%
\pgfpathrectangle{\pgfqpoint{1.150000in}{0.150000in}}{\pgfqpoint{5.700000in}{5.700000in}}%
\pgfusepath{clip}%
\pgfsetbuttcap%
\pgfsetroundjoin%
\definecolor{currentfill}{rgb}{0.255645,0.260703,0.528312}%
\pgfsetfillcolor{currentfill}%
\pgfsetfillopacity{0.700000}%
\pgfsetlinewidth{0.000000pt}%
\definecolor{currentstroke}{rgb}{0.000000,0.000000,0.000000}%
\pgfsetstrokecolor{currentstroke}%
\pgfsetdash{}{0pt}%
\pgfpathmoveto{\pgfqpoint{5.218989in}{2.926025in}}%
\pgfpathlineto{\pgfqpoint{5.232759in}{2.928374in}}%
\pgfpathlineto{\pgfqpoint{5.246541in}{2.930822in}}%
\pgfpathlineto{\pgfqpoint{5.260333in}{2.933371in}}%
\pgfpathlineto{\pgfqpoint{5.274138in}{2.936018in}}%
\pgfpathlineto{\pgfqpoint{5.266846in}{2.929157in}}%
\pgfpathlineto{\pgfqpoint{5.259549in}{2.922278in}}%
\pgfpathlineto{\pgfqpoint{5.252246in}{2.915379in}}%
\pgfpathlineto{\pgfqpoint{5.244937in}{2.908459in}}%
\pgfpathlineto{\pgfqpoint{5.231118in}{2.905640in}}%
\pgfpathlineto{\pgfqpoint{5.217311in}{2.902921in}}%
\pgfpathlineto{\pgfqpoint{5.203515in}{2.900301in}}%
\pgfpathlineto{\pgfqpoint{5.189731in}{2.897781in}}%
\pgfpathlineto{\pgfqpoint{5.197054in}{2.904866in}}%
\pgfpathlineto{\pgfqpoint{5.204371in}{2.911934in}}%
\pgfpathlineto{\pgfqpoint{5.211683in}{2.918986in}}%
\pgfpathlineto{\pgfqpoint{5.218989in}{2.926025in}}%
\pgfpathclose%
\pgfusepath{fill}%
\end{pgfscope}%
\begin{pgfscope}%
\pgfpathrectangle{\pgfqpoint{1.150000in}{0.150000in}}{\pgfqpoint{5.700000in}{5.700000in}}%
\pgfusepath{clip}%
\pgfsetbuttcap%
\pgfsetroundjoin%
\definecolor{currentfill}{rgb}{0.277018,0.050344,0.375715}%
\pgfsetfillcolor{currentfill}%
\pgfsetfillopacity{0.700000}%
\pgfsetlinewidth{0.000000pt}%
\definecolor{currentstroke}{rgb}{0.000000,0.000000,0.000000}%
\pgfsetstrokecolor{currentstroke}%
\pgfsetdash{}{0pt}%
\pgfpathmoveto{\pgfqpoint{3.600102in}{2.530332in}}%
\pgfpathlineto{\pgfqpoint{3.613412in}{2.523767in}}%
\pgfpathlineto{\pgfqpoint{3.626726in}{2.517336in}}%
\pgfpathlineto{\pgfqpoint{3.640043in}{2.511038in}}%
\pgfpathlineto{\pgfqpoint{3.653363in}{2.504871in}}%
\pgfpathlineto{\pgfqpoint{3.645474in}{2.496954in}}%
\pgfpathlineto{\pgfqpoint{3.637578in}{2.489073in}}%
\pgfpathlineto{\pgfqpoint{3.629677in}{2.481230in}}%
\pgfpathlineto{\pgfqpoint{3.621770in}{2.473425in}}%
\pgfpathlineto{\pgfqpoint{3.608435in}{2.479713in}}%
\pgfpathlineto{\pgfqpoint{3.595103in}{2.486133in}}%
\pgfpathlineto{\pgfqpoint{3.581775in}{2.492686in}}%
\pgfpathlineto{\pgfqpoint{3.568450in}{2.499372in}}%
\pgfpathlineto{\pgfqpoint{3.576372in}{2.507049in}}%
\pgfpathlineto{\pgfqpoint{3.584289in}{2.514768in}}%
\pgfpathlineto{\pgfqpoint{3.592199in}{2.522529in}}%
\pgfpathlineto{\pgfqpoint{3.600102in}{2.530332in}}%
\pgfpathclose%
\pgfusepath{fill}%
\end{pgfscope}%
\begin{pgfscope}%
\pgfpathrectangle{\pgfqpoint{1.150000in}{0.150000in}}{\pgfqpoint{5.700000in}{5.700000in}}%
\pgfusepath{clip}%
\pgfsetbuttcap%
\pgfsetroundjoin%
\definecolor{currentfill}{rgb}{0.274952,0.037752,0.364543}%
\pgfsetfillcolor{currentfill}%
\pgfsetfillopacity{0.700000}%
\pgfsetlinewidth{0.000000pt}%
\definecolor{currentstroke}{rgb}{0.000000,0.000000,0.000000}%
\pgfsetstrokecolor{currentstroke}%
\pgfsetdash{}{0pt}%
\pgfpathmoveto{\pgfqpoint{3.738125in}{2.513939in}}%
\pgfpathlineto{\pgfqpoint{3.751451in}{2.508524in}}%
\pgfpathlineto{\pgfqpoint{3.764782in}{2.503236in}}%
\pgfpathlineto{\pgfqpoint{3.778117in}{2.498075in}}%
\pgfpathlineto{\pgfqpoint{3.791456in}{2.493040in}}%
\pgfpathlineto{\pgfqpoint{3.783618in}{2.484792in}}%
\pgfpathlineto{\pgfqpoint{3.775773in}{2.476568in}}%
\pgfpathlineto{\pgfqpoint{3.767923in}{2.468369in}}%
\pgfpathlineto{\pgfqpoint{3.760067in}{2.460197in}}%
\pgfpathlineto{\pgfqpoint{3.746714in}{2.465334in}}%
\pgfpathlineto{\pgfqpoint{3.733366in}{2.470599in}}%
\pgfpathlineto{\pgfqpoint{3.720022in}{2.475989in}}%
\pgfpathlineto{\pgfqpoint{3.706682in}{2.481508in}}%
\pgfpathlineto{\pgfqpoint{3.714552in}{2.489571in}}%
\pgfpathlineto{\pgfqpoint{3.722415in}{2.497664in}}%
\pgfpathlineto{\pgfqpoint{3.730273in}{2.505787in}}%
\pgfpathlineto{\pgfqpoint{3.738125in}{2.513939in}}%
\pgfpathclose%
\pgfusepath{fill}%
\end{pgfscope}%
\begin{pgfscope}%
\pgfpathrectangle{\pgfqpoint{1.150000in}{0.150000in}}{\pgfqpoint{5.700000in}{5.700000in}}%
\pgfusepath{clip}%
\pgfsetbuttcap%
\pgfsetroundjoin%
\definecolor{currentfill}{rgb}{0.273006,0.204520,0.501721}%
\pgfsetfillcolor{currentfill}%
\pgfsetfillopacity{0.700000}%
\pgfsetlinewidth{0.000000pt}%
\definecolor{currentstroke}{rgb}{0.000000,0.000000,0.000000}%
\pgfsetstrokecolor{currentstroke}%
\pgfsetdash{}{0pt}%
\pgfpathmoveto{\pgfqpoint{2.971566in}{2.839626in}}%
\pgfpathlineto{\pgfqpoint{2.984910in}{2.826458in}}%
\pgfpathlineto{\pgfqpoint{2.998251in}{2.813469in}}%
\pgfpathlineto{\pgfqpoint{3.011590in}{2.800659in}}%
\pgfpathlineto{\pgfqpoint{3.024927in}{2.788025in}}%
\pgfpathlineto{\pgfqpoint{3.016773in}{2.782331in}}%
\pgfpathlineto{\pgfqpoint{3.008610in}{2.776733in}}%
\pgfpathlineto{\pgfqpoint{3.000439in}{2.771233in}}%
\pgfpathlineto{\pgfqpoint{2.992259in}{2.765832in}}%
\pgfpathlineto{\pgfqpoint{2.978899in}{2.778648in}}%
\pgfpathlineto{\pgfqpoint{2.965536in}{2.791641in}}%
\pgfpathlineto{\pgfqpoint{2.952171in}{2.804812in}}%
\pgfpathlineto{\pgfqpoint{2.938804in}{2.818163in}}%
\pgfpathlineto{\pgfqpoint{2.947007in}{2.823374in}}%
\pgfpathlineto{\pgfqpoint{2.955202in}{2.828690in}}%
\pgfpathlineto{\pgfqpoint{2.963388in}{2.834107in}}%
\pgfpathlineto{\pgfqpoint{2.971566in}{2.839626in}}%
\pgfpathclose%
\pgfusepath{fill}%
\end{pgfscope}%
\begin{pgfscope}%
\pgfpathrectangle{\pgfqpoint{1.150000in}{0.150000in}}{\pgfqpoint{5.700000in}{5.700000in}}%
\pgfusepath{clip}%
\pgfsetbuttcap%
\pgfsetroundjoin%
\definecolor{currentfill}{rgb}{0.283197,0.115680,0.436115}%
\pgfsetfillcolor{currentfill}%
\pgfsetfillopacity{0.700000}%
\pgfsetlinewidth{0.000000pt}%
\definecolor{currentstroke}{rgb}{0.000000,0.000000,0.000000}%
\pgfsetstrokecolor{currentstroke}%
\pgfsetdash{}{0pt}%
\pgfpathmoveto{\pgfqpoint{4.490029in}{2.638210in}}%
\pgfpathlineto{\pgfqpoint{4.503536in}{2.637680in}}%
\pgfpathlineto{\pgfqpoint{4.517051in}{2.637259in}}%
\pgfpathlineto{\pgfqpoint{4.530575in}{2.636945in}}%
\pgfpathlineto{\pgfqpoint{4.544107in}{2.636739in}}%
\pgfpathlineto{\pgfqpoint{4.536529in}{2.628207in}}%
\pgfpathlineto{\pgfqpoint{4.528944in}{2.619653in}}%
\pgfpathlineto{\pgfqpoint{4.521355in}{2.611078in}}%
\pgfpathlineto{\pgfqpoint{4.513760in}{2.602480in}}%
\pgfpathlineto{\pgfqpoint{4.500217in}{2.602662in}}%
\pgfpathlineto{\pgfqpoint{4.486683in}{2.602951in}}%
\pgfpathlineto{\pgfqpoint{4.473158in}{2.603349in}}%
\pgfpathlineto{\pgfqpoint{4.459641in}{2.603854in}}%
\pgfpathlineto{\pgfqpoint{4.467246in}{2.612470in}}%
\pgfpathlineto{\pgfqpoint{4.474846in}{2.621067in}}%
\pgfpathlineto{\pgfqpoint{4.482440in}{2.629647in}}%
\pgfpathlineto{\pgfqpoint{4.490029in}{2.638210in}}%
\pgfpathclose%
\pgfusepath{fill}%
\end{pgfscope}%
\begin{pgfscope}%
\pgfpathrectangle{\pgfqpoint{1.150000in}{0.150000in}}{\pgfqpoint{5.700000in}{5.700000in}}%
\pgfusepath{clip}%
\pgfsetbuttcap%
\pgfsetroundjoin%
\definecolor{currentfill}{rgb}{0.282910,0.105393,0.426902}%
\pgfsetfillcolor{currentfill}%
\pgfsetfillopacity{0.700000}%
\pgfsetlinewidth{0.000000pt}%
\definecolor{currentstroke}{rgb}{0.000000,0.000000,0.000000}%
\pgfsetstrokecolor{currentstroke}%
\pgfsetdash{}{0pt}%
\pgfpathmoveto{\pgfqpoint{3.270354in}{2.635026in}}%
\pgfpathlineto{\pgfqpoint{3.283658in}{2.625340in}}%
\pgfpathlineto{\pgfqpoint{3.296962in}{2.615807in}}%
\pgfpathlineto{\pgfqpoint{3.310268in}{2.606425in}}%
\pgfpathlineto{\pgfqpoint{3.323574in}{2.597194in}}%
\pgfpathlineto{\pgfqpoint{3.315552in}{2.590356in}}%
\pgfpathlineto{\pgfqpoint{3.307523in}{2.583585in}}%
\pgfpathlineto{\pgfqpoint{3.299487in}{2.576884in}}%
\pgfpathlineto{\pgfqpoint{3.291443in}{2.570253in}}%
\pgfpathlineto{\pgfqpoint{3.278119in}{2.579644in}}%
\pgfpathlineto{\pgfqpoint{3.264794in}{2.589186in}}%
\pgfpathlineto{\pgfqpoint{3.251471in}{2.598879in}}%
\pgfpathlineto{\pgfqpoint{3.238148in}{2.608725in}}%
\pgfpathlineto{\pgfqpoint{3.246210in}{2.615189in}}%
\pgfpathlineto{\pgfqpoint{3.254266in}{2.621728in}}%
\pgfpathlineto{\pgfqpoint{3.262313in}{2.628341in}}%
\pgfpathlineto{\pgfqpoint{3.270354in}{2.635026in}}%
\pgfpathclose%
\pgfusepath{fill}%
\end{pgfscope}%
\begin{pgfscope}%
\pgfpathrectangle{\pgfqpoint{1.150000in}{0.150000in}}{\pgfqpoint{5.700000in}{5.700000in}}%
\pgfusepath{clip}%
\pgfsetbuttcap%
\pgfsetroundjoin%
\definecolor{currentfill}{rgb}{0.262138,0.242286,0.520837}%
\pgfsetfillcolor{currentfill}%
\pgfsetfillopacity{0.700000}%
\pgfsetlinewidth{0.000000pt}%
\definecolor{currentstroke}{rgb}{0.000000,0.000000,0.000000}%
\pgfsetstrokecolor{currentstroke}%
\pgfsetdash{}{0pt}%
\pgfpathmoveto{\pgfqpoint{5.134707in}{2.888703in}}%
\pgfpathlineto{\pgfqpoint{5.148446in}{2.890822in}}%
\pgfpathlineto{\pgfqpoint{5.162196in}{2.893042in}}%
\pgfpathlineto{\pgfqpoint{5.175958in}{2.895361in}}%
\pgfpathlineto{\pgfqpoint{5.189731in}{2.897781in}}%
\pgfpathlineto{\pgfqpoint{5.182402in}{2.890677in}}%
\pgfpathlineto{\pgfqpoint{5.175068in}{2.883551in}}%
\pgfpathlineto{\pgfqpoint{5.167727in}{2.876402in}}%
\pgfpathlineto{\pgfqpoint{5.160381in}{2.869228in}}%
\pgfpathlineto{\pgfqpoint{5.146594in}{2.866655in}}%
\pgfpathlineto{\pgfqpoint{5.132819in}{2.864182in}}%
\pgfpathlineto{\pgfqpoint{5.119055in}{2.861810in}}%
\pgfpathlineto{\pgfqpoint{5.105302in}{2.859539in}}%
\pgfpathlineto{\pgfqpoint{5.112662in}{2.866859in}}%
\pgfpathlineto{\pgfqpoint{5.120016in}{2.874159in}}%
\pgfpathlineto{\pgfqpoint{5.127364in}{2.881439in}}%
\pgfpathlineto{\pgfqpoint{5.134707in}{2.888703in}}%
\pgfpathclose%
\pgfusepath{fill}%
\end{pgfscope}%
\begin{pgfscope}%
\pgfpathrectangle{\pgfqpoint{1.150000in}{0.150000in}}{\pgfqpoint{5.700000in}{5.700000in}}%
\pgfusepath{clip}%
\pgfsetbuttcap%
\pgfsetroundjoin%
\definecolor{currentfill}{rgb}{0.279566,0.067836,0.391917}%
\pgfsetfillcolor{currentfill}%
\pgfsetfillopacity{0.700000}%
\pgfsetlinewidth{0.000000pt}%
\definecolor{currentstroke}{rgb}{0.000000,0.000000,0.000000}%
\pgfsetstrokecolor{currentstroke}%
\pgfsetdash{}{0pt}%
\pgfpathmoveto{\pgfqpoint{3.461952in}{2.557766in}}%
\pgfpathlineto{\pgfqpoint{3.475256in}{2.549982in}}%
\pgfpathlineto{\pgfqpoint{3.488561in}{2.542338in}}%
\pgfpathlineto{\pgfqpoint{3.501869in}{2.534834in}}%
\pgfpathlineto{\pgfqpoint{3.515180in}{2.527468in}}%
\pgfpathlineto{\pgfqpoint{3.507236in}{2.519968in}}%
\pgfpathlineto{\pgfqpoint{3.499285in}{2.512519in}}%
\pgfpathlineto{\pgfqpoint{3.491328in}{2.505119in}}%
\pgfpathlineto{\pgfqpoint{3.483365in}{2.497772in}}%
\pgfpathlineto{\pgfqpoint{3.470038in}{2.505278in}}%
\pgfpathlineto{\pgfqpoint{3.456714in}{2.512922in}}%
\pgfpathlineto{\pgfqpoint{3.443392in}{2.520706in}}%
\pgfpathlineto{\pgfqpoint{3.430072in}{2.528631in}}%
\pgfpathlineto{\pgfqpoint{3.438052in}{2.535831in}}%
\pgfpathlineto{\pgfqpoint{3.446025in}{2.543088in}}%
\pgfpathlineto{\pgfqpoint{3.453992in}{2.550400in}}%
\pgfpathlineto{\pgfqpoint{3.461952in}{2.557766in}}%
\pgfpathclose%
\pgfusepath{fill}%
\end{pgfscope}%
\begin{pgfscope}%
\pgfpathrectangle{\pgfqpoint{1.150000in}{0.150000in}}{\pgfqpoint{5.700000in}{5.700000in}}%
\pgfusepath{clip}%
\pgfsetbuttcap%
\pgfsetroundjoin%
\definecolor{currentfill}{rgb}{0.277941,0.056324,0.381191}%
\pgfsetfillcolor{currentfill}%
\pgfsetfillopacity{0.700000}%
\pgfsetlinewidth{0.000000pt}%
\definecolor{currentstroke}{rgb}{0.000000,0.000000,0.000000}%
\pgfsetstrokecolor{currentstroke}%
\pgfsetdash{}{0pt}%
\pgfpathmoveto{\pgfqpoint{4.098648in}{2.533027in}}%
\pgfpathlineto{\pgfqpoint{4.112047in}{2.530244in}}%
\pgfpathlineto{\pgfqpoint{4.125452in}{2.527578in}}%
\pgfpathlineto{\pgfqpoint{4.138863in}{2.525027in}}%
\pgfpathlineto{\pgfqpoint{4.152281in}{2.522591in}}%
\pgfpathlineto{\pgfqpoint{4.144566in}{2.513878in}}%
\pgfpathlineto{\pgfqpoint{4.136846in}{2.505162in}}%
\pgfpathlineto{\pgfqpoint{4.129120in}{2.496444in}}%
\pgfpathlineto{\pgfqpoint{4.121389in}{2.487723in}}%
\pgfpathlineto{\pgfqpoint{4.107960in}{2.490207in}}%
\pgfpathlineto{\pgfqpoint{4.094538in}{2.492806in}}%
\pgfpathlineto{\pgfqpoint{4.081122in}{2.495521in}}%
\pgfpathlineto{\pgfqpoint{4.067713in}{2.498352in}}%
\pgfpathlineto{\pgfqpoint{4.075455in}{2.507018in}}%
\pgfpathlineto{\pgfqpoint{4.083191in}{2.515686in}}%
\pgfpathlineto{\pgfqpoint{4.090922in}{2.524355in}}%
\pgfpathlineto{\pgfqpoint{4.098648in}{2.533027in}}%
\pgfpathclose%
\pgfusepath{fill}%
\end{pgfscope}%
\begin{pgfscope}%
\pgfpathrectangle{\pgfqpoint{1.150000in}{0.150000in}}{\pgfqpoint{5.700000in}{5.700000in}}%
\pgfusepath{clip}%
\pgfsetbuttcap%
\pgfsetroundjoin%
\definecolor{currentfill}{rgb}{0.267968,0.223549,0.512008}%
\pgfsetfillcolor{currentfill}%
\pgfsetfillopacity{0.700000}%
\pgfsetlinewidth{0.000000pt}%
\definecolor{currentstroke}{rgb}{0.000000,0.000000,0.000000}%
\pgfsetstrokecolor{currentstroke}%
\pgfsetdash{}{0pt}%
\pgfpathmoveto{\pgfqpoint{5.050400in}{2.851461in}}%
\pgfpathlineto{\pgfqpoint{5.064109in}{2.853329in}}%
\pgfpathlineto{\pgfqpoint{5.077829in}{2.855298in}}%
\pgfpathlineto{\pgfqpoint{5.091560in}{2.857368in}}%
\pgfpathlineto{\pgfqpoint{5.105302in}{2.859539in}}%
\pgfpathlineto{\pgfqpoint{5.097937in}{2.852196in}}%
\pgfpathlineto{\pgfqpoint{5.090566in}{2.844829in}}%
\pgfpathlineto{\pgfqpoint{5.083189in}{2.837436in}}%
\pgfpathlineto{\pgfqpoint{5.075806in}{2.830015in}}%
\pgfpathlineto{\pgfqpoint{5.062051in}{2.827710in}}%
\pgfpathlineto{\pgfqpoint{5.048308in}{2.825505in}}%
\pgfpathlineto{\pgfqpoint{5.034575in}{2.823402in}}%
\pgfpathlineto{\pgfqpoint{5.020853in}{2.821400in}}%
\pgfpathlineto{\pgfqpoint{5.028248in}{2.828949in}}%
\pgfpathlineto{\pgfqpoint{5.035638in}{2.836474in}}%
\pgfpathlineto{\pgfqpoint{5.043022in}{2.843977in}}%
\pgfpathlineto{\pgfqpoint{5.050400in}{2.851461in}}%
\pgfpathclose%
\pgfusepath{fill}%
\end{pgfscope}%
\begin{pgfscope}%
\pgfpathrectangle{\pgfqpoint{1.150000in}{0.150000in}}{\pgfqpoint{5.700000in}{5.700000in}}%
\pgfusepath{clip}%
\pgfsetbuttcap%
\pgfsetroundjoin%
\definecolor{currentfill}{rgb}{0.274952,0.037752,0.364543}%
\pgfsetfillcolor{currentfill}%
\pgfsetfillopacity{0.700000}%
\pgfsetlinewidth{0.000000pt}%
\definecolor{currentstroke}{rgb}{0.000000,0.000000,0.000000}%
\pgfsetstrokecolor{currentstroke}%
\pgfsetdash{}{0pt}%
\pgfpathmoveto{\pgfqpoint{3.876110in}{2.507712in}}%
\pgfpathlineto{\pgfqpoint{3.889461in}{2.503383in}}%
\pgfpathlineto{\pgfqpoint{3.902818in}{2.499176in}}%
\pgfpathlineto{\pgfqpoint{3.916180in}{2.495091in}}%
\pgfpathlineto{\pgfqpoint{3.929547in}{2.491128in}}%
\pgfpathlineto{\pgfqpoint{3.921755in}{2.482632in}}%
\pgfpathlineto{\pgfqpoint{3.913958in}{2.474149in}}%
\pgfpathlineto{\pgfqpoint{3.906155in}{2.465679in}}%
\pgfpathlineto{\pgfqpoint{3.898347in}{2.457224in}}%
\pgfpathlineto{\pgfqpoint{3.884968in}{2.461273in}}%
\pgfpathlineto{\pgfqpoint{3.871594in}{2.465443in}}%
\pgfpathlineto{\pgfqpoint{3.858225in}{2.469734in}}%
\pgfpathlineto{\pgfqpoint{3.844862in}{2.474148in}}%
\pgfpathlineto{\pgfqpoint{3.852682in}{2.482512in}}%
\pgfpathlineto{\pgfqpoint{3.860497in}{2.490894in}}%
\pgfpathlineto{\pgfqpoint{3.868306in}{2.499295in}}%
\pgfpathlineto{\pgfqpoint{3.876110in}{2.507712in}}%
\pgfpathclose%
\pgfusepath{fill}%
\end{pgfscope}%
\begin{pgfscope}%
\pgfpathrectangle{\pgfqpoint{1.150000in}{0.150000in}}{\pgfqpoint{5.700000in}{5.700000in}}%
\pgfusepath{clip}%
\pgfsetbuttcap%
\pgfsetroundjoin%
\definecolor{currentfill}{rgb}{0.277134,0.185228,0.489898}%
\pgfsetfillcolor{currentfill}%
\pgfsetfillopacity{0.700000}%
\pgfsetlinewidth{0.000000pt}%
\definecolor{currentstroke}{rgb}{0.000000,0.000000,0.000000}%
\pgfsetstrokecolor{currentstroke}%
\pgfsetdash{}{0pt}%
\pgfpathmoveto{\pgfqpoint{3.024927in}{2.788025in}}%
\pgfpathlineto{\pgfqpoint{3.038262in}{2.775566in}}%
\pgfpathlineto{\pgfqpoint{3.051595in}{2.763281in}}%
\pgfpathlineto{\pgfqpoint{3.064927in}{2.751168in}}%
\pgfpathlineto{\pgfqpoint{3.078257in}{2.739225in}}%
\pgfpathlineto{\pgfqpoint{3.070124in}{2.733356in}}%
\pgfpathlineto{\pgfqpoint{3.061984in}{2.727579in}}%
\pgfpathlineto{\pgfqpoint{3.053835in}{2.721895in}}%
\pgfpathlineto{\pgfqpoint{3.045678in}{2.716305in}}%
\pgfpathlineto{\pgfqpoint{3.032326in}{2.728429in}}%
\pgfpathlineto{\pgfqpoint{3.018972in}{2.740724in}}%
\pgfpathlineto{\pgfqpoint{3.005616in}{2.753191in}}%
\pgfpathlineto{\pgfqpoint{2.992259in}{2.765832in}}%
\pgfpathlineto{\pgfqpoint{3.000439in}{2.771233in}}%
\pgfpathlineto{\pgfqpoint{3.008610in}{2.776733in}}%
\pgfpathlineto{\pgfqpoint{3.016773in}{2.782331in}}%
\pgfpathlineto{\pgfqpoint{3.024927in}{2.788025in}}%
\pgfpathclose%
\pgfusepath{fill}%
\end{pgfscope}%
\begin{pgfscope}%
\pgfpathrectangle{\pgfqpoint{1.150000in}{0.150000in}}{\pgfqpoint{5.700000in}{5.700000in}}%
\pgfusepath{clip}%
\pgfsetbuttcap%
\pgfsetroundjoin%
\definecolor{currentfill}{rgb}{0.282656,0.100196,0.422160}%
\pgfsetfillcolor{currentfill}%
\pgfsetfillopacity{0.700000}%
\pgfsetlinewidth{0.000000pt}%
\definecolor{currentstroke}{rgb}{0.000000,0.000000,0.000000}%
\pgfsetstrokecolor{currentstroke}%
\pgfsetdash{}{0pt}%
\pgfpathmoveto{\pgfqpoint{4.405656in}{2.606964in}}%
\pgfpathlineto{\pgfqpoint{4.419140in}{2.606023in}}%
\pgfpathlineto{\pgfqpoint{4.432632in}{2.605191in}}%
\pgfpathlineto{\pgfqpoint{4.446132in}{2.604468in}}%
\pgfpathlineto{\pgfqpoint{4.459641in}{2.603854in}}%
\pgfpathlineto{\pgfqpoint{4.452030in}{2.595220in}}%
\pgfpathlineto{\pgfqpoint{4.444415in}{2.586567in}}%
\pgfpathlineto{\pgfqpoint{4.436793in}{2.577895in}}%
\pgfpathlineto{\pgfqpoint{4.429167in}{2.569203in}}%
\pgfpathlineto{\pgfqpoint{4.415648in}{2.569811in}}%
\pgfpathlineto{\pgfqpoint{4.402138in}{2.570528in}}%
\pgfpathlineto{\pgfqpoint{4.388636in}{2.571354in}}%
\pgfpathlineto{\pgfqpoint{4.375142in}{2.572289in}}%
\pgfpathlineto{\pgfqpoint{4.382778in}{2.580980in}}%
\pgfpathlineto{\pgfqpoint{4.390409in}{2.589656in}}%
\pgfpathlineto{\pgfqpoint{4.398035in}{2.598318in}}%
\pgfpathlineto{\pgfqpoint{4.405656in}{2.606964in}}%
\pgfpathclose%
\pgfusepath{fill}%
\end{pgfscope}%
\begin{pgfscope}%
\pgfpathrectangle{\pgfqpoint{1.150000in}{0.150000in}}{\pgfqpoint{5.700000in}{5.700000in}}%
\pgfusepath{clip}%
\pgfsetbuttcap%
\pgfsetroundjoin%
\definecolor{currentfill}{rgb}{0.271828,0.209303,0.504434}%
\pgfsetfillcolor{currentfill}%
\pgfsetfillopacity{0.700000}%
\pgfsetlinewidth{0.000000pt}%
\definecolor{currentstroke}{rgb}{0.000000,0.000000,0.000000}%
\pgfsetstrokecolor{currentstroke}%
\pgfsetdash{}{0pt}%
\pgfpathmoveto{\pgfqpoint{4.966072in}{2.814409in}}%
\pgfpathlineto{\pgfqpoint{4.979751in}{2.816004in}}%
\pgfpathlineto{\pgfqpoint{4.993441in}{2.817701in}}%
\pgfpathlineto{\pgfqpoint{5.007142in}{2.819500in}}%
\pgfpathlineto{\pgfqpoint{5.020853in}{2.821400in}}%
\pgfpathlineto{\pgfqpoint{5.013452in}{2.813827in}}%
\pgfpathlineto{\pgfqpoint{5.006045in}{2.806227in}}%
\pgfpathlineto{\pgfqpoint{4.998632in}{2.798600in}}%
\pgfpathlineto{\pgfqpoint{4.991214in}{2.790942in}}%
\pgfpathlineto{\pgfqpoint{4.977491in}{2.788926in}}%
\pgfpathlineto{\pgfqpoint{4.963778in}{2.787011in}}%
\pgfpathlineto{\pgfqpoint{4.950076in}{2.785199in}}%
\pgfpathlineto{\pgfqpoint{4.936384in}{2.783488in}}%
\pgfpathlineto{\pgfqpoint{4.943815in}{2.791254in}}%
\pgfpathlineto{\pgfqpoint{4.951239in}{2.798995in}}%
\pgfpathlineto{\pgfqpoint{4.958659in}{2.806713in}}%
\pgfpathlineto{\pgfqpoint{4.966072in}{2.814409in}}%
\pgfpathclose%
\pgfusepath{fill}%
\end{pgfscope}%
\begin{pgfscope}%
\pgfpathrectangle{\pgfqpoint{1.150000in}{0.150000in}}{\pgfqpoint{5.700000in}{5.700000in}}%
\pgfusepath{clip}%
\pgfsetbuttcap%
\pgfsetroundjoin%
\definecolor{currentfill}{rgb}{0.276194,0.190074,0.493001}%
\pgfsetfillcolor{currentfill}%
\pgfsetfillopacity{0.700000}%
\pgfsetlinewidth{0.000000pt}%
\definecolor{currentstroke}{rgb}{0.000000,0.000000,0.000000}%
\pgfsetstrokecolor{currentstroke}%
\pgfsetdash{}{0pt}%
\pgfpathmoveto{\pgfqpoint{4.881723in}{2.777669in}}%
\pgfpathlineto{\pgfqpoint{4.895373in}{2.778969in}}%
\pgfpathlineto{\pgfqpoint{4.909033in}{2.780373in}}%
\pgfpathlineto{\pgfqpoint{4.922703in}{2.781879in}}%
\pgfpathlineto{\pgfqpoint{4.936384in}{2.783488in}}%
\pgfpathlineto{\pgfqpoint{4.928948in}{2.775695in}}%
\pgfpathlineto{\pgfqpoint{4.921507in}{2.767874in}}%
\pgfpathlineto{\pgfqpoint{4.914059in}{2.760025in}}%
\pgfpathlineto{\pgfqpoint{4.906606in}{2.752144in}}%
\pgfpathlineto{\pgfqpoint{4.892913in}{2.750438in}}%
\pgfpathlineto{\pgfqpoint{4.879231in}{2.748835in}}%
\pgfpathlineto{\pgfqpoint{4.865559in}{2.747334in}}%
\pgfpathlineto{\pgfqpoint{4.851898in}{2.745936in}}%
\pgfpathlineto{\pgfqpoint{4.859362in}{2.753907in}}%
\pgfpathlineto{\pgfqpoint{4.866821in}{2.761852in}}%
\pgfpathlineto{\pgfqpoint{4.874275in}{2.769772in}}%
\pgfpathlineto{\pgfqpoint{4.881723in}{2.777669in}}%
\pgfpathclose%
\pgfusepath{fill}%
\end{pgfscope}%
\begin{pgfscope}%
\pgfpathrectangle{\pgfqpoint{1.150000in}{0.150000in}}{\pgfqpoint{5.700000in}{5.700000in}}%
\pgfusepath{clip}%
\pgfsetbuttcap%
\pgfsetroundjoin%
\definecolor{currentfill}{rgb}{0.280868,0.160771,0.472899}%
\pgfsetfillcolor{currentfill}%
\pgfsetfillopacity{0.700000}%
\pgfsetlinewidth{0.000000pt}%
\definecolor{currentstroke}{rgb}{0.000000,0.000000,0.000000}%
\pgfsetstrokecolor{currentstroke}%
\pgfsetdash{}{0pt}%
\pgfpathmoveto{\pgfqpoint{3.078257in}{2.739225in}}%
\pgfpathlineto{\pgfqpoint{3.091585in}{2.727452in}}%
\pgfpathlineto{\pgfqpoint{3.104912in}{2.715846in}}%
\pgfpathlineto{\pgfqpoint{3.118238in}{2.704407in}}%
\pgfpathlineto{\pgfqpoint{3.131563in}{2.693132in}}%
\pgfpathlineto{\pgfqpoint{3.123453in}{2.687090in}}%
\pgfpathlineto{\pgfqpoint{3.115334in}{2.681134in}}%
\pgfpathlineto{\pgfqpoint{3.107207in}{2.675267in}}%
\pgfpathlineto{\pgfqpoint{3.099072in}{2.669490in}}%
\pgfpathlineto{\pgfqpoint{3.085725in}{2.680945in}}%
\pgfpathlineto{\pgfqpoint{3.072377in}{2.692564in}}%
\pgfpathlineto{\pgfqpoint{3.059028in}{2.704351in}}%
\pgfpathlineto{\pgfqpoint{3.045678in}{2.716305in}}%
\pgfpathlineto{\pgfqpoint{3.053835in}{2.721895in}}%
\pgfpathlineto{\pgfqpoint{3.061984in}{2.727579in}}%
\pgfpathlineto{\pgfqpoint{3.070124in}{2.733356in}}%
\pgfpathlineto{\pgfqpoint{3.078257in}{2.739225in}}%
\pgfpathclose%
\pgfusepath{fill}%
\end{pgfscope}%
\begin{pgfscope}%
\pgfpathrectangle{\pgfqpoint{1.150000in}{0.150000in}}{\pgfqpoint{5.700000in}{5.700000in}}%
\pgfusepath{clip}%
\pgfsetbuttcap%
\pgfsetroundjoin%
\definecolor{currentfill}{rgb}{0.281924,0.089666,0.412415}%
\pgfsetfillcolor{currentfill}%
\pgfsetfillopacity{0.700000}%
\pgfsetlinewidth{0.000000pt}%
\definecolor{currentstroke}{rgb}{0.000000,0.000000,0.000000}%
\pgfsetstrokecolor{currentstroke}%
\pgfsetdash{}{0pt}%
\pgfpathmoveto{\pgfqpoint{3.323574in}{2.597194in}}%
\pgfpathlineto{\pgfqpoint{3.336881in}{2.588112in}}%
\pgfpathlineto{\pgfqpoint{3.350190in}{2.579178in}}%
\pgfpathlineto{\pgfqpoint{3.363500in}{2.570391in}}%
\pgfpathlineto{\pgfqpoint{3.376811in}{2.561751in}}%
\pgfpathlineto{\pgfqpoint{3.368807in}{2.554760in}}%
\pgfpathlineto{\pgfqpoint{3.360796in}{2.547832in}}%
\pgfpathlineto{\pgfqpoint{3.352778in}{2.540969in}}%
\pgfpathlineto{\pgfqpoint{3.344753in}{2.534172in}}%
\pgfpathlineto{\pgfqpoint{3.331424in}{2.542972in}}%
\pgfpathlineto{\pgfqpoint{3.318096in}{2.551918in}}%
\pgfpathlineto{\pgfqpoint{3.304769in}{2.561011in}}%
\pgfpathlineto{\pgfqpoint{3.291443in}{2.570253in}}%
\pgfpathlineto{\pgfqpoint{3.299487in}{2.576884in}}%
\pgfpathlineto{\pgfqpoint{3.307523in}{2.583585in}}%
\pgfpathlineto{\pgfqpoint{3.315552in}{2.590356in}}%
\pgfpathlineto{\pgfqpoint{3.323574in}{2.597194in}}%
\pgfpathclose%
\pgfusepath{fill}%
\end{pgfscope}%
\begin{pgfscope}%
\pgfpathrectangle{\pgfqpoint{1.150000in}{0.150000in}}{\pgfqpoint{5.700000in}{5.700000in}}%
\pgfusepath{clip}%
\pgfsetbuttcap%
\pgfsetroundjoin%
\definecolor{currentfill}{rgb}{0.281924,0.089666,0.412415}%
\pgfsetfillcolor{currentfill}%
\pgfsetfillopacity{0.700000}%
\pgfsetlinewidth{0.000000pt}%
\definecolor{currentstroke}{rgb}{0.000000,0.000000,0.000000}%
\pgfsetstrokecolor{currentstroke}%
\pgfsetdash{}{0pt}%
\pgfpathmoveto{\pgfqpoint{4.321245in}{2.577133in}}%
\pgfpathlineto{\pgfqpoint{4.334707in}{2.575756in}}%
\pgfpathlineto{\pgfqpoint{4.348178in}{2.574490in}}%
\pgfpathlineto{\pgfqpoint{4.361656in}{2.573335in}}%
\pgfpathlineto{\pgfqpoint{4.375142in}{2.572289in}}%
\pgfpathlineto{\pgfqpoint{4.367500in}{2.563582in}}%
\pgfpathlineto{\pgfqpoint{4.359853in}{2.554859in}}%
\pgfpathlineto{\pgfqpoint{4.352201in}{2.546121in}}%
\pgfpathlineto{\pgfqpoint{4.344543in}{2.537366in}}%
\pgfpathlineto{\pgfqpoint{4.331047in}{2.538423in}}%
\pgfpathlineto{\pgfqpoint{4.317558in}{2.539591in}}%
\pgfpathlineto{\pgfqpoint{4.304078in}{2.540869in}}%
\pgfpathlineto{\pgfqpoint{4.290605in}{2.542258in}}%
\pgfpathlineto{\pgfqpoint{4.298273in}{2.550995in}}%
\pgfpathlineto{\pgfqpoint{4.305935in}{2.559719in}}%
\pgfpathlineto{\pgfqpoint{4.313593in}{2.568432in}}%
\pgfpathlineto{\pgfqpoint{4.321245in}{2.577133in}}%
\pgfpathclose%
\pgfusepath{fill}%
\end{pgfscope}%
\begin{pgfscope}%
\pgfpathrectangle{\pgfqpoint{1.150000in}{0.150000in}}{\pgfqpoint{5.700000in}{5.700000in}}%
\pgfusepath{clip}%
\pgfsetbuttcap%
\pgfsetroundjoin%
\definecolor{currentfill}{rgb}{0.277018,0.050344,0.375715}%
\pgfsetfillcolor{currentfill}%
\pgfsetfillopacity{0.700000}%
\pgfsetlinewidth{0.000000pt}%
\definecolor{currentstroke}{rgb}{0.000000,0.000000,0.000000}%
\pgfsetstrokecolor{currentstroke}%
\pgfsetdash{}{0pt}%
\pgfpathmoveto{\pgfqpoint{4.014138in}{2.510845in}}%
\pgfpathlineto{\pgfqpoint{4.027523in}{2.507545in}}%
\pgfpathlineto{\pgfqpoint{4.040913in}{2.504364in}}%
\pgfpathlineto{\pgfqpoint{4.054310in}{2.501299in}}%
\pgfpathlineto{\pgfqpoint{4.067713in}{2.498352in}}%
\pgfpathlineto{\pgfqpoint{4.059966in}{2.489688in}}%
\pgfpathlineto{\pgfqpoint{4.052213in}{2.481027in}}%
\pgfpathlineto{\pgfqpoint{4.044455in}{2.472369in}}%
\pgfpathlineto{\pgfqpoint{4.036692in}{2.463714in}}%
\pgfpathlineto{\pgfqpoint{4.023278in}{2.466728in}}%
\pgfpathlineto{\pgfqpoint{4.009870in}{2.469858in}}%
\pgfpathlineto{\pgfqpoint{3.996468in}{2.473107in}}%
\pgfpathlineto{\pgfqpoint{3.983072in}{2.476473in}}%
\pgfpathlineto{\pgfqpoint{3.990847in}{2.485055in}}%
\pgfpathlineto{\pgfqpoint{3.998616in}{2.493644in}}%
\pgfpathlineto{\pgfqpoint{4.006380in}{2.502241in}}%
\pgfpathlineto{\pgfqpoint{4.014138in}{2.510845in}}%
\pgfpathclose%
\pgfusepath{fill}%
\end{pgfscope}%
\begin{pgfscope}%
\pgfpathrectangle{\pgfqpoint{1.150000in}{0.150000in}}{\pgfqpoint{5.700000in}{5.700000in}}%
\pgfusepath{clip}%
\pgfsetbuttcap%
\pgfsetroundjoin%
\definecolor{currentfill}{rgb}{0.278826,0.175490,0.483397}%
\pgfsetfillcolor{currentfill}%
\pgfsetfillopacity{0.700000}%
\pgfsetlinewidth{0.000000pt}%
\definecolor{currentstroke}{rgb}{0.000000,0.000000,0.000000}%
\pgfsetstrokecolor{currentstroke}%
\pgfsetdash{}{0pt}%
\pgfpathmoveto{\pgfqpoint{4.797352in}{2.741377in}}%
\pgfpathlineto{\pgfqpoint{4.810974in}{2.742361in}}%
\pgfpathlineto{\pgfqpoint{4.824605in}{2.743449in}}%
\pgfpathlineto{\pgfqpoint{4.838246in}{2.744641in}}%
\pgfpathlineto{\pgfqpoint{4.851898in}{2.745936in}}%
\pgfpathlineto{\pgfqpoint{4.844428in}{2.737937in}}%
\pgfpathlineto{\pgfqpoint{4.836952in}{2.729911in}}%
\pgfpathlineto{\pgfqpoint{4.829470in}{2.721854in}}%
\pgfpathlineto{\pgfqpoint{4.821983in}{2.713767in}}%
\pgfpathlineto{\pgfqpoint{4.808321in}{2.712393in}}%
\pgfpathlineto{\pgfqpoint{4.794668in}{2.711122in}}%
\pgfpathlineto{\pgfqpoint{4.781026in}{2.709955in}}%
\pgfpathlineto{\pgfqpoint{4.767393in}{2.708892in}}%
\pgfpathlineto{\pgfqpoint{4.774891in}{2.717052in}}%
\pgfpathlineto{\pgfqpoint{4.782384in}{2.725185in}}%
\pgfpathlineto{\pgfqpoint{4.789871in}{2.733293in}}%
\pgfpathlineto{\pgfqpoint{4.797352in}{2.741377in}}%
\pgfpathclose%
\pgfusepath{fill}%
\end{pgfscope}%
\begin{pgfscope}%
\pgfpathrectangle{\pgfqpoint{1.150000in}{0.150000in}}{\pgfqpoint{5.700000in}{5.700000in}}%
\pgfusepath{clip}%
\pgfsetbuttcap%
\pgfsetroundjoin%
\definecolor{currentfill}{rgb}{0.276022,0.044167,0.370164}%
\pgfsetfillcolor{currentfill}%
\pgfsetfillopacity{0.700000}%
\pgfsetlinewidth{0.000000pt}%
\definecolor{currentstroke}{rgb}{0.000000,0.000000,0.000000}%
\pgfsetstrokecolor{currentstroke}%
\pgfsetdash{}{0pt}%
\pgfpathmoveto{\pgfqpoint{3.653363in}{2.504871in}}%
\pgfpathlineto{\pgfqpoint{3.666687in}{2.498835in}}%
\pgfpathlineto{\pgfqpoint{3.680015in}{2.492930in}}%
\pgfpathlineto{\pgfqpoint{3.693347in}{2.487155in}}%
\pgfpathlineto{\pgfqpoint{3.706682in}{2.481508in}}%
\pgfpathlineto{\pgfqpoint{3.698807in}{2.473476in}}%
\pgfpathlineto{\pgfqpoint{3.690926in}{2.465477in}}%
\pgfpathlineto{\pgfqpoint{3.683039in}{2.457510in}}%
\pgfpathlineto{\pgfqpoint{3.675146in}{2.449578in}}%
\pgfpathlineto{\pgfqpoint{3.661796in}{2.455345in}}%
\pgfpathlineto{\pgfqpoint{3.648450in}{2.461242in}}%
\pgfpathlineto{\pgfqpoint{3.635108in}{2.467269in}}%
\pgfpathlineto{\pgfqpoint{3.621770in}{2.473425in}}%
\pgfpathlineto{\pgfqpoint{3.629677in}{2.481230in}}%
\pgfpathlineto{\pgfqpoint{3.637578in}{2.489073in}}%
\pgfpathlineto{\pgfqpoint{3.645474in}{2.496954in}}%
\pgfpathlineto{\pgfqpoint{3.653363in}{2.504871in}}%
\pgfpathclose%
\pgfusepath{fill}%
\end{pgfscope}%
\begin{pgfscope}%
\pgfpathrectangle{\pgfqpoint{1.150000in}{0.150000in}}{\pgfqpoint{5.700000in}{5.700000in}}%
\pgfusepath{clip}%
\pgfsetbuttcap%
\pgfsetroundjoin%
\definecolor{currentfill}{rgb}{0.277941,0.056324,0.381191}%
\pgfsetfillcolor{currentfill}%
\pgfsetfillopacity{0.700000}%
\pgfsetlinewidth{0.000000pt}%
\definecolor{currentstroke}{rgb}{0.000000,0.000000,0.000000}%
\pgfsetstrokecolor{currentstroke}%
\pgfsetdash{}{0pt}%
\pgfpathmoveto{\pgfqpoint{3.515180in}{2.527468in}}%
\pgfpathlineto{\pgfqpoint{3.528493in}{2.520240in}}%
\pgfpathlineto{\pgfqpoint{3.541809in}{2.513148in}}%
\pgfpathlineto{\pgfqpoint{3.555128in}{2.506193in}}%
\pgfpathlineto{\pgfqpoint{3.568450in}{2.499372in}}%
\pgfpathlineto{\pgfqpoint{3.560522in}{2.491740in}}%
\pgfpathlineto{\pgfqpoint{3.552587in}{2.484152in}}%
\pgfpathlineto{\pgfqpoint{3.544646in}{2.476611in}}%
\pgfpathlineto{\pgfqpoint{3.536698in}{2.469117in}}%
\pgfpathlineto{\pgfqpoint{3.523361in}{2.476077in}}%
\pgfpathlineto{\pgfqpoint{3.510026in}{2.483173in}}%
\pgfpathlineto{\pgfqpoint{3.496694in}{2.490404in}}%
\pgfpathlineto{\pgfqpoint{3.483365in}{2.497772in}}%
\pgfpathlineto{\pgfqpoint{3.491328in}{2.505119in}}%
\pgfpathlineto{\pgfqpoint{3.499285in}{2.512519in}}%
\pgfpathlineto{\pgfqpoint{3.507236in}{2.519968in}}%
\pgfpathlineto{\pgfqpoint{3.515180in}{2.527468in}}%
\pgfpathclose%
\pgfusepath{fill}%
\end{pgfscope}%
\begin{pgfscope}%
\pgfpathrectangle{\pgfqpoint{1.150000in}{0.150000in}}{\pgfqpoint{5.700000in}{5.700000in}}%
\pgfusepath{clip}%
\pgfsetbuttcap%
\pgfsetroundjoin%
\definecolor{currentfill}{rgb}{0.274952,0.037752,0.364543}%
\pgfsetfillcolor{currentfill}%
\pgfsetfillopacity{0.700000}%
\pgfsetlinewidth{0.000000pt}%
\definecolor{currentstroke}{rgb}{0.000000,0.000000,0.000000}%
\pgfsetstrokecolor{currentstroke}%
\pgfsetdash{}{0pt}%
\pgfpathmoveto{\pgfqpoint{3.791456in}{2.493040in}}%
\pgfpathlineto{\pgfqpoint{3.804801in}{2.488130in}}%
\pgfpathlineto{\pgfqpoint{3.818149in}{2.483346in}}%
\pgfpathlineto{\pgfqpoint{3.831503in}{2.478685in}}%
\pgfpathlineto{\pgfqpoint{3.844862in}{2.474148in}}%
\pgfpathlineto{\pgfqpoint{3.837036in}{2.465804in}}%
\pgfpathlineto{\pgfqpoint{3.829204in}{2.457480in}}%
\pgfpathlineto{\pgfqpoint{3.821367in}{2.449176in}}%
\pgfpathlineto{\pgfqpoint{3.813524in}{2.440894in}}%
\pgfpathlineto{\pgfqpoint{3.800153in}{2.445534in}}%
\pgfpathlineto{\pgfqpoint{3.786786in}{2.450297in}}%
\pgfpathlineto{\pgfqpoint{3.773424in}{2.455184in}}%
\pgfpathlineto{\pgfqpoint{3.760067in}{2.460197in}}%
\pgfpathlineto{\pgfqpoint{3.767923in}{2.468369in}}%
\pgfpathlineto{\pgfqpoint{3.775773in}{2.476568in}}%
\pgfpathlineto{\pgfqpoint{3.783618in}{2.484792in}}%
\pgfpathlineto{\pgfqpoint{3.791456in}{2.493040in}}%
\pgfpathclose%
\pgfusepath{fill}%
\end{pgfscope}%
\begin{pgfscope}%
\pgfpathrectangle{\pgfqpoint{1.150000in}{0.150000in}}{\pgfqpoint{5.700000in}{5.700000in}}%
\pgfusepath{clip}%
\pgfsetbuttcap%
\pgfsetroundjoin%
\definecolor{currentfill}{rgb}{0.281412,0.155834,0.469201}%
\pgfsetfillcolor{currentfill}%
\pgfsetfillopacity{0.700000}%
\pgfsetlinewidth{0.000000pt}%
\definecolor{currentstroke}{rgb}{0.000000,0.000000,0.000000}%
\pgfsetstrokecolor{currentstroke}%
\pgfsetdash{}{0pt}%
\pgfpathmoveto{\pgfqpoint{4.712961in}{2.705682in}}%
\pgfpathlineto{\pgfqpoint{4.726554in}{2.706327in}}%
\pgfpathlineto{\pgfqpoint{4.740158in}{2.707078in}}%
\pgfpathlineto{\pgfqpoint{4.753770in}{2.707933in}}%
\pgfpathlineto{\pgfqpoint{4.767393in}{2.708892in}}%
\pgfpathlineto{\pgfqpoint{4.759890in}{2.700705in}}%
\pgfpathlineto{\pgfqpoint{4.752380in}{2.692490in}}%
\pgfpathlineto{\pgfqpoint{4.744866in}{2.684245in}}%
\pgfpathlineto{\pgfqpoint{4.737345in}{2.675970in}}%
\pgfpathlineto{\pgfqpoint{4.723712in}{2.674950in}}%
\pgfpathlineto{\pgfqpoint{4.710088in}{2.674035in}}%
\pgfpathlineto{\pgfqpoint{4.696474in}{2.673223in}}%
\pgfpathlineto{\pgfqpoint{4.682870in}{2.672517in}}%
\pgfpathlineto{\pgfqpoint{4.690401in}{2.680846in}}%
\pgfpathlineto{\pgfqpoint{4.697926in}{2.689149in}}%
\pgfpathlineto{\pgfqpoint{4.705446in}{2.697427in}}%
\pgfpathlineto{\pgfqpoint{4.712961in}{2.705682in}}%
\pgfpathclose%
\pgfusepath{fill}%
\end{pgfscope}%
\begin{pgfscope}%
\pgfpathrectangle{\pgfqpoint{1.150000in}{0.150000in}}{\pgfqpoint{5.700000in}{5.700000in}}%
\pgfusepath{clip}%
\pgfsetbuttcap%
\pgfsetroundjoin%
\definecolor{currentfill}{rgb}{0.282623,0.140926,0.457517}%
\pgfsetfillcolor{currentfill}%
\pgfsetfillopacity{0.700000}%
\pgfsetlinewidth{0.000000pt}%
\definecolor{currentstroke}{rgb}{0.000000,0.000000,0.000000}%
\pgfsetstrokecolor{currentstroke}%
\pgfsetdash{}{0pt}%
\pgfpathmoveto{\pgfqpoint{3.131563in}{2.693132in}}%
\pgfpathlineto{\pgfqpoint{3.144888in}{2.682022in}}%
\pgfpathlineto{\pgfqpoint{3.158211in}{2.671074in}}%
\pgfpathlineto{\pgfqpoint{3.171535in}{2.660286in}}%
\pgfpathlineto{\pgfqpoint{3.184857in}{2.649659in}}%
\pgfpathlineto{\pgfqpoint{3.176767in}{2.643444in}}%
\pgfpathlineto{\pgfqpoint{3.168669in}{2.637311in}}%
\pgfpathlineto{\pgfqpoint{3.160564in}{2.631262in}}%
\pgfpathlineto{\pgfqpoint{3.152450in}{2.625298in}}%
\pgfpathlineto{\pgfqpoint{3.139106in}{2.636105in}}%
\pgfpathlineto{\pgfqpoint{3.125762in}{2.647071in}}%
\pgfpathlineto{\pgfqpoint{3.112417in}{2.658199in}}%
\pgfpathlineto{\pgfqpoint{3.099072in}{2.669490in}}%
\pgfpathlineto{\pgfqpoint{3.107207in}{2.675267in}}%
\pgfpathlineto{\pgfqpoint{3.115334in}{2.681134in}}%
\pgfpathlineto{\pgfqpoint{3.123453in}{2.687090in}}%
\pgfpathlineto{\pgfqpoint{3.131563in}{2.693132in}}%
\pgfpathclose%
\pgfusepath{fill}%
\end{pgfscope}%
\begin{pgfscope}%
\pgfpathrectangle{\pgfqpoint{1.150000in}{0.150000in}}{\pgfqpoint{5.700000in}{5.700000in}}%
\pgfusepath{clip}%
\pgfsetbuttcap%
\pgfsetroundjoin%
\definecolor{currentfill}{rgb}{0.280267,0.073417,0.397163}%
\pgfsetfillcolor{currentfill}%
\pgfsetfillopacity{0.700000}%
\pgfsetlinewidth{0.000000pt}%
\definecolor{currentstroke}{rgb}{0.000000,0.000000,0.000000}%
\pgfsetstrokecolor{currentstroke}%
\pgfsetdash{}{0pt}%
\pgfpathmoveto{\pgfqpoint{4.236789in}{2.548932in}}%
\pgfpathlineto{\pgfqpoint{4.250232in}{2.547096in}}%
\pgfpathlineto{\pgfqpoint{4.263682in}{2.545371in}}%
\pgfpathlineto{\pgfqpoint{4.277140in}{2.543759in}}%
\pgfpathlineto{\pgfqpoint{4.290605in}{2.542258in}}%
\pgfpathlineto{\pgfqpoint{4.282932in}{2.533510in}}%
\pgfpathlineto{\pgfqpoint{4.275254in}{2.524750in}}%
\pgfpathlineto{\pgfqpoint{4.267570in}{2.515978in}}%
\pgfpathlineto{\pgfqpoint{4.259881in}{2.507194in}}%
\pgfpathlineto{\pgfqpoint{4.246406in}{2.508725in}}%
\pgfpathlineto{\pgfqpoint{4.232938in}{2.510368in}}%
\pgfpathlineto{\pgfqpoint{4.219477in}{2.512122in}}%
\pgfpathlineto{\pgfqpoint{4.206024in}{2.513989in}}%
\pgfpathlineto{\pgfqpoint{4.213723in}{2.522736in}}%
\pgfpathlineto{\pgfqpoint{4.221417in}{2.531476in}}%
\pgfpathlineto{\pgfqpoint{4.229106in}{2.540208in}}%
\pgfpathlineto{\pgfqpoint{4.236789in}{2.548932in}}%
\pgfpathclose%
\pgfusepath{fill}%
\end{pgfscope}%
\begin{pgfscope}%
\pgfpathrectangle{\pgfqpoint{1.150000in}{0.150000in}}{\pgfqpoint{5.700000in}{5.700000in}}%
\pgfusepath{clip}%
\pgfsetbuttcap%
\pgfsetroundjoin%
\definecolor{currentfill}{rgb}{0.282623,0.140926,0.457517}%
\pgfsetfillcolor{currentfill}%
\pgfsetfillopacity{0.700000}%
\pgfsetlinewidth{0.000000pt}%
\definecolor{currentstroke}{rgb}{0.000000,0.000000,0.000000}%
\pgfsetstrokecolor{currentstroke}%
\pgfsetdash{}{0pt}%
\pgfpathmoveto{\pgfqpoint{4.628546in}{2.670744in}}%
\pgfpathlineto{\pgfqpoint{4.642113in}{2.671029in}}%
\pgfpathlineto{\pgfqpoint{4.655689in}{2.671420in}}%
\pgfpathlineto{\pgfqpoint{4.669275in}{2.671916in}}%
\pgfpathlineto{\pgfqpoint{4.682870in}{2.672517in}}%
\pgfpathlineto{\pgfqpoint{4.675333in}{2.664162in}}%
\pgfpathlineto{\pgfqpoint{4.667792in}{2.655779in}}%
\pgfpathlineto{\pgfqpoint{4.660244in}{2.647367in}}%
\pgfpathlineto{\pgfqpoint{4.652691in}{2.638927in}}%
\pgfpathlineto{\pgfqpoint{4.639086in}{2.638283in}}%
\pgfpathlineto{\pgfqpoint{4.625490in}{2.637744in}}%
\pgfpathlineto{\pgfqpoint{4.611904in}{2.637311in}}%
\pgfpathlineto{\pgfqpoint{4.598326in}{2.636984in}}%
\pgfpathlineto{\pgfqpoint{4.605889in}{2.645460in}}%
\pgfpathlineto{\pgfqpoint{4.613447in}{2.653912in}}%
\pgfpathlineto{\pgfqpoint{4.620999in}{2.662340in}}%
\pgfpathlineto{\pgfqpoint{4.628546in}{2.670744in}}%
\pgfpathclose%
\pgfusepath{fill}%
\end{pgfscope}%
\begin{pgfscope}%
\pgfpathrectangle{\pgfqpoint{1.150000in}{0.150000in}}{\pgfqpoint{5.700000in}{5.700000in}}%
\pgfusepath{clip}%
\pgfsetbuttcap%
\pgfsetroundjoin%
\definecolor{currentfill}{rgb}{0.280894,0.078907,0.402329}%
\pgfsetfillcolor{currentfill}%
\pgfsetfillopacity{0.700000}%
\pgfsetlinewidth{0.000000pt}%
\definecolor{currentstroke}{rgb}{0.000000,0.000000,0.000000}%
\pgfsetstrokecolor{currentstroke}%
\pgfsetdash{}{0pt}%
\pgfpathmoveto{\pgfqpoint{3.376811in}{2.561751in}}%
\pgfpathlineto{\pgfqpoint{3.390123in}{2.553256in}}%
\pgfpathlineto{\pgfqpoint{3.403438in}{2.544904in}}%
\pgfpathlineto{\pgfqpoint{3.416754in}{2.536696in}}%
\pgfpathlineto{\pgfqpoint{3.430072in}{2.528631in}}%
\pgfpathlineto{\pgfqpoint{3.422085in}{2.521488in}}%
\pgfpathlineto{\pgfqpoint{3.414092in}{2.514404in}}%
\pgfpathlineto{\pgfqpoint{3.406091in}{2.507380in}}%
\pgfpathlineto{\pgfqpoint{3.398084in}{2.500417in}}%
\pgfpathlineto{\pgfqpoint{3.384749in}{2.508642in}}%
\pgfpathlineto{\pgfqpoint{3.371415in}{2.517008in}}%
\pgfpathlineto{\pgfqpoint{3.358083in}{2.525518in}}%
\pgfpathlineto{\pgfqpoint{3.344753in}{2.534172in}}%
\pgfpathlineto{\pgfqpoint{3.352778in}{2.540969in}}%
\pgfpathlineto{\pgfqpoint{3.360796in}{2.547832in}}%
\pgfpathlineto{\pgfqpoint{3.368807in}{2.554760in}}%
\pgfpathlineto{\pgfqpoint{3.376811in}{2.561751in}}%
\pgfpathclose%
\pgfusepath{fill}%
\end{pgfscope}%
\begin{pgfscope}%
\pgfpathrectangle{\pgfqpoint{1.150000in}{0.150000in}}{\pgfqpoint{5.700000in}{5.700000in}}%
\pgfusepath{clip}%
\pgfsetbuttcap%
\pgfsetroundjoin%
\definecolor{currentfill}{rgb}{0.235526,0.309527,0.542944}%
\pgfsetfillcolor{currentfill}%
\pgfsetfillopacity{0.700000}%
\pgfsetlinewidth{0.000000pt}%
\definecolor{currentstroke}{rgb}{0.000000,0.000000,0.000000}%
\pgfsetstrokecolor{currentstroke}%
\pgfsetdash{}{0pt}%
\pgfpathmoveto{\pgfqpoint{5.442874in}{3.012102in}}%
\pgfpathlineto{\pgfqpoint{5.456753in}{3.015240in}}%
\pgfpathlineto{\pgfqpoint{5.470645in}{3.018476in}}%
\pgfpathlineto{\pgfqpoint{5.484549in}{3.021809in}}%
\pgfpathlineto{\pgfqpoint{5.498465in}{3.025241in}}%
\pgfpathlineto{\pgfqpoint{5.491267in}{3.019084in}}%
\pgfpathlineto{\pgfqpoint{5.484064in}{3.012915in}}%
\pgfpathlineto{\pgfqpoint{5.476855in}{3.006732in}}%
\pgfpathlineto{\pgfqpoint{5.469640in}{3.000530in}}%
\pgfpathlineto{\pgfqpoint{5.455707in}{2.996889in}}%
\pgfpathlineto{\pgfqpoint{5.441786in}{2.993346in}}%
\pgfpathlineto{\pgfqpoint{5.427878in}{2.989902in}}%
\pgfpathlineto{\pgfqpoint{5.413982in}{2.986555in}}%
\pgfpathlineto{\pgfqpoint{5.421214in}{2.992959in}}%
\pgfpathlineto{\pgfqpoint{5.428439in}{2.999350in}}%
\pgfpathlineto{\pgfqpoint{5.435659in}{3.005730in}}%
\pgfpathlineto{\pgfqpoint{5.442874in}{3.012102in}}%
\pgfpathclose%
\pgfusepath{fill}%
\end{pgfscope}%
\begin{pgfscope}%
\pgfpathrectangle{\pgfqpoint{1.150000in}{0.150000in}}{\pgfqpoint{5.700000in}{5.700000in}}%
\pgfusepath{clip}%
\pgfsetbuttcap%
\pgfsetroundjoin%
\definecolor{currentfill}{rgb}{0.274952,0.037752,0.364543}%
\pgfsetfillcolor{currentfill}%
\pgfsetfillopacity{0.700000}%
\pgfsetlinewidth{0.000000pt}%
\definecolor{currentstroke}{rgb}{0.000000,0.000000,0.000000}%
\pgfsetstrokecolor{currentstroke}%
\pgfsetdash{}{0pt}%
\pgfpathmoveto{\pgfqpoint{3.929547in}{2.491128in}}%
\pgfpathlineto{\pgfqpoint{3.942920in}{2.487284in}}%
\pgfpathlineto{\pgfqpoint{3.956298in}{2.483561in}}%
\pgfpathlineto{\pgfqpoint{3.969682in}{2.479957in}}%
\pgfpathlineto{\pgfqpoint{3.983072in}{2.476473in}}%
\pgfpathlineto{\pgfqpoint{3.975292in}{2.467899in}}%
\pgfpathlineto{\pgfqpoint{3.967507in}{2.459334in}}%
\pgfpathlineto{\pgfqpoint{3.959716in}{2.450778in}}%
\pgfpathlineto{\pgfqpoint{3.951920in}{2.442232in}}%
\pgfpathlineto{\pgfqpoint{3.938518in}{2.445801in}}%
\pgfpathlineto{\pgfqpoint{3.925122in}{2.449489in}}%
\pgfpathlineto{\pgfqpoint{3.911732in}{2.453297in}}%
\pgfpathlineto{\pgfqpoint{3.898347in}{2.457224in}}%
\pgfpathlineto{\pgfqpoint{3.906155in}{2.465679in}}%
\pgfpathlineto{\pgfqpoint{3.913958in}{2.474149in}}%
\pgfpathlineto{\pgfqpoint{3.921755in}{2.482632in}}%
\pgfpathlineto{\pgfqpoint{3.929547in}{2.491128in}}%
\pgfpathclose%
\pgfusepath{fill}%
\end{pgfscope}%
\begin{pgfscope}%
\pgfpathrectangle{\pgfqpoint{1.150000in}{0.150000in}}{\pgfqpoint{5.700000in}{5.700000in}}%
\pgfusepath{clip}%
\pgfsetbuttcap%
\pgfsetroundjoin%
\definecolor{currentfill}{rgb}{0.283187,0.125848,0.444960}%
\pgfsetfillcolor{currentfill}%
\pgfsetfillopacity{0.700000}%
\pgfsetlinewidth{0.000000pt}%
\definecolor{currentstroke}{rgb}{0.000000,0.000000,0.000000}%
\pgfsetstrokecolor{currentstroke}%
\pgfsetdash{}{0pt}%
\pgfpathmoveto{\pgfqpoint{4.544107in}{2.636739in}}%
\pgfpathlineto{\pgfqpoint{4.557649in}{2.636640in}}%
\pgfpathlineto{\pgfqpoint{4.571199in}{2.636648in}}%
\pgfpathlineto{\pgfqpoint{4.584758in}{2.636763in}}%
\pgfpathlineto{\pgfqpoint{4.598326in}{2.636984in}}%
\pgfpathlineto{\pgfqpoint{4.590758in}{2.628483in}}%
\pgfpathlineto{\pgfqpoint{4.583184in}{2.619956in}}%
\pgfpathlineto{\pgfqpoint{4.575604in}{2.611402in}}%
\pgfpathlineto{\pgfqpoint{4.568019in}{2.602822in}}%
\pgfpathlineto{\pgfqpoint{4.554441in}{2.602576in}}%
\pgfpathlineto{\pgfqpoint{4.540872in}{2.602437in}}%
\pgfpathlineto{\pgfqpoint{4.527311in}{2.602405in}}%
\pgfpathlineto{\pgfqpoint{4.513760in}{2.602480in}}%
\pgfpathlineto{\pgfqpoint{4.521355in}{2.611078in}}%
\pgfpathlineto{\pgfqpoint{4.528944in}{2.619653in}}%
\pgfpathlineto{\pgfqpoint{4.536529in}{2.628207in}}%
\pgfpathlineto{\pgfqpoint{4.544107in}{2.636739in}}%
\pgfpathclose%
\pgfusepath{fill}%
\end{pgfscope}%
\begin{pgfscope}%
\pgfpathrectangle{\pgfqpoint{1.150000in}{0.150000in}}{\pgfqpoint{5.700000in}{5.700000in}}%
\pgfusepath{clip}%
\pgfsetbuttcap%
\pgfsetroundjoin%
\definecolor{currentfill}{rgb}{0.283187,0.125848,0.444960}%
\pgfsetfillcolor{currentfill}%
\pgfsetfillopacity{0.700000}%
\pgfsetlinewidth{0.000000pt}%
\definecolor{currentstroke}{rgb}{0.000000,0.000000,0.000000}%
\pgfsetstrokecolor{currentstroke}%
\pgfsetdash{}{0pt}%
\pgfpathmoveto{\pgfqpoint{3.184857in}{2.649659in}}%
\pgfpathlineto{\pgfqpoint{3.198180in}{2.639191in}}%
\pgfpathlineto{\pgfqpoint{3.211502in}{2.628880in}}%
\pgfpathlineto{\pgfqpoint{3.224825in}{2.618725in}}%
\pgfpathlineto{\pgfqpoint{3.238148in}{2.608725in}}%
\pgfpathlineto{\pgfqpoint{3.230078in}{2.602337in}}%
\pgfpathlineto{\pgfqpoint{3.222000in}{2.596028in}}%
\pgfpathlineto{\pgfqpoint{3.213915in}{2.589797in}}%
\pgfpathlineto{\pgfqpoint{3.205821in}{2.583648in}}%
\pgfpathlineto{\pgfqpoint{3.192479in}{2.593827in}}%
\pgfpathlineto{\pgfqpoint{3.179136in}{2.604160in}}%
\pgfpathlineto{\pgfqpoint{3.165793in}{2.614650in}}%
\pgfpathlineto{\pgfqpoint{3.152450in}{2.625298in}}%
\pgfpathlineto{\pgfqpoint{3.160564in}{2.631262in}}%
\pgfpathlineto{\pgfqpoint{3.168669in}{2.637311in}}%
\pgfpathlineto{\pgfqpoint{3.176767in}{2.643444in}}%
\pgfpathlineto{\pgfqpoint{3.184857in}{2.649659in}}%
\pgfpathclose%
\pgfusepath{fill}%
\end{pgfscope}%
\begin{pgfscope}%
\pgfpathrectangle{\pgfqpoint{1.150000in}{0.150000in}}{\pgfqpoint{5.700000in}{5.700000in}}%
\pgfusepath{clip}%
\pgfsetbuttcap%
\pgfsetroundjoin%
\definecolor{currentfill}{rgb}{0.278791,0.062145,0.386592}%
\pgfsetfillcolor{currentfill}%
\pgfsetfillopacity{0.700000}%
\pgfsetlinewidth{0.000000pt}%
\definecolor{currentstroke}{rgb}{0.000000,0.000000,0.000000}%
\pgfsetstrokecolor{currentstroke}%
\pgfsetdash{}{0pt}%
\pgfpathmoveto{\pgfqpoint{4.152281in}{2.522591in}}%
\pgfpathlineto{\pgfqpoint{4.165707in}{2.520270in}}%
\pgfpathlineto{\pgfqpoint{4.179139in}{2.518063in}}%
\pgfpathlineto{\pgfqpoint{4.192578in}{2.515969in}}%
\pgfpathlineto{\pgfqpoint{4.206024in}{2.513989in}}%
\pgfpathlineto{\pgfqpoint{4.198319in}{2.505235in}}%
\pgfpathlineto{\pgfqpoint{4.190609in}{2.496473in}}%
\pgfpathlineto{\pgfqpoint{4.182894in}{2.487704in}}%
\pgfpathlineto{\pgfqpoint{4.175174in}{2.478928in}}%
\pgfpathlineto{\pgfqpoint{4.161717in}{2.480956in}}%
\pgfpathlineto{\pgfqpoint{4.148268in}{2.483098in}}%
\pgfpathlineto{\pgfqpoint{4.134825in}{2.485353in}}%
\pgfpathlineto{\pgfqpoint{4.121389in}{2.487723in}}%
\pgfpathlineto{\pgfqpoint{4.129120in}{2.496444in}}%
\pgfpathlineto{\pgfqpoint{4.136846in}{2.505162in}}%
\pgfpathlineto{\pgfqpoint{4.144566in}{2.513878in}}%
\pgfpathlineto{\pgfqpoint{4.152281in}{2.522591in}}%
\pgfpathclose%
\pgfusepath{fill}%
\end{pgfscope}%
\begin{pgfscope}%
\pgfpathrectangle{\pgfqpoint{1.150000in}{0.150000in}}{\pgfqpoint{5.700000in}{5.700000in}}%
\pgfusepath{clip}%
\pgfsetbuttcap%
\pgfsetroundjoin%
\definecolor{currentfill}{rgb}{0.243113,0.292092,0.538516}%
\pgfsetfillcolor{currentfill}%
\pgfsetfillopacity{0.700000}%
\pgfsetlinewidth{0.000000pt}%
\definecolor{currentstroke}{rgb}{0.000000,0.000000,0.000000}%
\pgfsetstrokecolor{currentstroke}%
\pgfsetdash{}{0pt}%
\pgfpathmoveto{\pgfqpoint{5.358519in}{2.974153in}}%
\pgfpathlineto{\pgfqpoint{5.372367in}{2.977106in}}%
\pgfpathlineto{\pgfqpoint{5.386227in}{2.980157in}}%
\pgfpathlineto{\pgfqpoint{5.400098in}{2.983307in}}%
\pgfpathlineto{\pgfqpoint{5.413982in}{2.986555in}}%
\pgfpathlineto{\pgfqpoint{5.406745in}{2.980136in}}%
\pgfpathlineto{\pgfqpoint{5.399501in}{2.973700in}}%
\pgfpathlineto{\pgfqpoint{5.392252in}{2.967243in}}%
\pgfpathlineto{\pgfqpoint{5.384997in}{2.960764in}}%
\pgfpathlineto{\pgfqpoint{5.371097in}{2.957325in}}%
\pgfpathlineto{\pgfqpoint{5.357210in}{2.953985in}}%
\pgfpathlineto{\pgfqpoint{5.343335in}{2.950743in}}%
\pgfpathlineto{\pgfqpoint{5.329472in}{2.947600in}}%
\pgfpathlineto{\pgfqpoint{5.336742in}{2.954263in}}%
\pgfpathlineto{\pgfqpoint{5.344007in}{2.960908in}}%
\pgfpathlineto{\pgfqpoint{5.351266in}{2.967537in}}%
\pgfpathlineto{\pgfqpoint{5.358519in}{2.974153in}}%
\pgfpathclose%
\pgfusepath{fill}%
\end{pgfscope}%
\begin{pgfscope}%
\pgfpathrectangle{\pgfqpoint{1.150000in}{0.150000in}}{\pgfqpoint{5.700000in}{5.700000in}}%
\pgfusepath{clip}%
\pgfsetbuttcap%
\pgfsetroundjoin%
\definecolor{currentfill}{rgb}{0.250425,0.274290,0.533103}%
\pgfsetfillcolor{currentfill}%
\pgfsetfillopacity{0.700000}%
\pgfsetlinewidth{0.000000pt}%
\definecolor{currentstroke}{rgb}{0.000000,0.000000,0.000000}%
\pgfsetstrokecolor{currentstroke}%
\pgfsetdash{}{0pt}%
\pgfpathmoveto{\pgfqpoint{5.274138in}{2.936018in}}%
\pgfpathlineto{\pgfqpoint{5.287953in}{2.938765in}}%
\pgfpathlineto{\pgfqpoint{5.301781in}{2.941611in}}%
\pgfpathlineto{\pgfqpoint{5.315620in}{2.944556in}}%
\pgfpathlineto{\pgfqpoint{5.329472in}{2.947600in}}%
\pgfpathlineto{\pgfqpoint{5.322195in}{2.940917in}}%
\pgfpathlineto{\pgfqpoint{5.314913in}{2.934213in}}%
\pgfpathlineto{\pgfqpoint{5.307625in}{2.927484in}}%
\pgfpathlineto{\pgfqpoint{5.300330in}{2.920728in}}%
\pgfpathlineto{\pgfqpoint{5.286464in}{2.917512in}}%
\pgfpathlineto{\pgfqpoint{5.272610in}{2.914395in}}%
\pgfpathlineto{\pgfqpoint{5.258768in}{2.911377in}}%
\pgfpathlineto{\pgfqpoint{5.244937in}{2.908459in}}%
\pgfpathlineto{\pgfqpoint{5.252246in}{2.915379in}}%
\pgfpathlineto{\pgfqpoint{5.259549in}{2.922278in}}%
\pgfpathlineto{\pgfqpoint{5.266846in}{2.929157in}}%
\pgfpathlineto{\pgfqpoint{5.274138in}{2.936018in}}%
\pgfpathclose%
\pgfusepath{fill}%
\end{pgfscope}%
\begin{pgfscope}%
\pgfpathrectangle{\pgfqpoint{1.150000in}{0.150000in}}{\pgfqpoint{5.700000in}{5.700000in}}%
\pgfusepath{clip}%
\pgfsetbuttcap%
\pgfsetroundjoin%
\definecolor{currentfill}{rgb}{0.283091,0.110553,0.431554}%
\pgfsetfillcolor{currentfill}%
\pgfsetfillopacity{0.700000}%
\pgfsetlinewidth{0.000000pt}%
\definecolor{currentstroke}{rgb}{0.000000,0.000000,0.000000}%
\pgfsetstrokecolor{currentstroke}%
\pgfsetdash{}{0pt}%
\pgfpathmoveto{\pgfqpoint{4.459641in}{2.603854in}}%
\pgfpathlineto{\pgfqpoint{4.473158in}{2.603349in}}%
\pgfpathlineto{\pgfqpoint{4.486683in}{2.602951in}}%
\pgfpathlineto{\pgfqpoint{4.500217in}{2.602662in}}%
\pgfpathlineto{\pgfqpoint{4.513760in}{2.602480in}}%
\pgfpathlineto{\pgfqpoint{4.506160in}{2.593858in}}%
\pgfpathlineto{\pgfqpoint{4.498554in}{2.585214in}}%
\pgfpathlineto{\pgfqpoint{4.490943in}{2.576545in}}%
\pgfpathlineto{\pgfqpoint{4.483326in}{2.567853in}}%
\pgfpathlineto{\pgfqpoint{4.469774in}{2.568029in}}%
\pgfpathlineto{\pgfqpoint{4.456230in}{2.568312in}}%
\pgfpathlineto{\pgfqpoint{4.442694in}{2.568703in}}%
\pgfpathlineto{\pgfqpoint{4.429167in}{2.569203in}}%
\pgfpathlineto{\pgfqpoint{4.436793in}{2.577895in}}%
\pgfpathlineto{\pgfqpoint{4.444415in}{2.586567in}}%
\pgfpathlineto{\pgfqpoint{4.452030in}{2.595220in}}%
\pgfpathlineto{\pgfqpoint{4.459641in}{2.603854in}}%
\pgfpathclose%
\pgfusepath{fill}%
\end{pgfscope}%
\begin{pgfscope}%
\pgfpathrectangle{\pgfqpoint{1.150000in}{0.150000in}}{\pgfqpoint{5.700000in}{5.700000in}}%
\pgfusepath{clip}%
\pgfsetbuttcap%
\pgfsetroundjoin%
\definecolor{currentfill}{rgb}{0.276022,0.044167,0.370164}%
\pgfsetfillcolor{currentfill}%
\pgfsetfillopacity{0.700000}%
\pgfsetlinewidth{0.000000pt}%
\definecolor{currentstroke}{rgb}{0.000000,0.000000,0.000000}%
\pgfsetstrokecolor{currentstroke}%
\pgfsetdash{}{0pt}%
\pgfpathmoveto{\pgfqpoint{3.568450in}{2.499372in}}%
\pgfpathlineto{\pgfqpoint{3.581775in}{2.492686in}}%
\pgfpathlineto{\pgfqpoint{3.595103in}{2.486133in}}%
\pgfpathlineto{\pgfqpoint{3.608435in}{2.479713in}}%
\pgfpathlineto{\pgfqpoint{3.621770in}{2.473425in}}%
\pgfpathlineto{\pgfqpoint{3.613856in}{2.465660in}}%
\pgfpathlineto{\pgfqpoint{3.605937in}{2.457936in}}%
\pgfpathlineto{\pgfqpoint{3.598011in}{2.450253in}}%
\pgfpathlineto{\pgfqpoint{3.590079in}{2.442613in}}%
\pgfpathlineto{\pgfqpoint{3.576729in}{2.449040in}}%
\pgfpathlineto{\pgfqpoint{3.563382in}{2.455599in}}%
\pgfpathlineto{\pgfqpoint{3.550039in}{2.462291in}}%
\pgfpathlineto{\pgfqpoint{3.536698in}{2.469117in}}%
\pgfpathlineto{\pgfqpoint{3.544646in}{2.476611in}}%
\pgfpathlineto{\pgfqpoint{3.552587in}{2.484152in}}%
\pgfpathlineto{\pgfqpoint{3.560522in}{2.491740in}}%
\pgfpathlineto{\pgfqpoint{3.568450in}{2.499372in}}%
\pgfpathclose%
\pgfusepath{fill}%
\end{pgfscope}%
\begin{pgfscope}%
\pgfpathrectangle{\pgfqpoint{1.150000in}{0.150000in}}{\pgfqpoint{5.700000in}{5.700000in}}%
\pgfusepath{clip}%
\pgfsetbuttcap%
\pgfsetroundjoin%
\definecolor{currentfill}{rgb}{0.255645,0.260703,0.528312}%
\pgfsetfillcolor{currentfill}%
\pgfsetfillopacity{0.700000}%
\pgfsetlinewidth{0.000000pt}%
\definecolor{currentstroke}{rgb}{0.000000,0.000000,0.000000}%
\pgfsetstrokecolor{currentstroke}%
\pgfsetdash{}{0pt}%
\pgfpathmoveto{\pgfqpoint{2.831744in}{2.931649in}}%
\pgfpathlineto{\pgfqpoint{2.845140in}{2.916798in}}%
\pgfpathlineto{\pgfqpoint{2.858531in}{2.902140in}}%
\pgfpathlineto{\pgfqpoint{2.871919in}{2.887675in}}%
\pgfpathlineto{\pgfqpoint{2.885303in}{2.873400in}}%
\pgfpathlineto{\pgfqpoint{2.877065in}{2.868489in}}%
\pgfpathlineto{\pgfqpoint{2.868818in}{2.863692in}}%
\pgfpathlineto{\pgfqpoint{2.860561in}{2.859007in}}%
\pgfpathlineto{\pgfqpoint{2.852295in}{2.854439in}}%
\pgfpathlineto{\pgfqpoint{2.838886in}{2.868918in}}%
\pgfpathlineto{\pgfqpoint{2.825472in}{2.883587in}}%
\pgfpathlineto{\pgfqpoint{2.812055in}{2.898448in}}%
\pgfpathlineto{\pgfqpoint{2.798633in}{2.913503in}}%
\pgfpathlineto{\pgfqpoint{2.806926in}{2.917861in}}%
\pgfpathlineto{\pgfqpoint{2.815208in}{2.922339in}}%
\pgfpathlineto{\pgfqpoint{2.823481in}{2.926935in}}%
\pgfpathlineto{\pgfqpoint{2.831744in}{2.931649in}}%
\pgfpathclose%
\pgfusepath{fill}%
\end{pgfscope}%
\begin{pgfscope}%
\pgfpathrectangle{\pgfqpoint{1.150000in}{0.150000in}}{\pgfqpoint{5.700000in}{5.700000in}}%
\pgfusepath{clip}%
\pgfsetbuttcap%
\pgfsetroundjoin%
\definecolor{currentfill}{rgb}{0.274952,0.037752,0.364543}%
\pgfsetfillcolor{currentfill}%
\pgfsetfillopacity{0.700000}%
\pgfsetlinewidth{0.000000pt}%
\definecolor{currentstroke}{rgb}{0.000000,0.000000,0.000000}%
\pgfsetstrokecolor{currentstroke}%
\pgfsetdash{}{0pt}%
\pgfpathmoveto{\pgfqpoint{3.706682in}{2.481508in}}%
\pgfpathlineto{\pgfqpoint{3.720022in}{2.475989in}}%
\pgfpathlineto{\pgfqpoint{3.733366in}{2.470599in}}%
\pgfpathlineto{\pgfqpoint{3.746714in}{2.465334in}}%
\pgfpathlineto{\pgfqpoint{3.760067in}{2.460197in}}%
\pgfpathlineto{\pgfqpoint{3.752205in}{2.452051in}}%
\pgfpathlineto{\pgfqpoint{3.744338in}{2.443933in}}%
\pgfpathlineto{\pgfqpoint{3.736465in}{2.435843in}}%
\pgfpathlineto{\pgfqpoint{3.728585in}{2.427783in}}%
\pgfpathlineto{\pgfqpoint{3.715219in}{2.433042in}}%
\pgfpathlineto{\pgfqpoint{3.701857in}{2.438427in}}%
\pgfpathlineto{\pgfqpoint{3.688499in}{2.443938in}}%
\pgfpathlineto{\pgfqpoint{3.675146in}{2.449578in}}%
\pgfpathlineto{\pgfqpoint{3.683039in}{2.457510in}}%
\pgfpathlineto{\pgfqpoint{3.690926in}{2.465477in}}%
\pgfpathlineto{\pgfqpoint{3.698807in}{2.473476in}}%
\pgfpathlineto{\pgfqpoint{3.706682in}{2.481508in}}%
\pgfpathclose%
\pgfusepath{fill}%
\end{pgfscope}%
\begin{pgfscope}%
\pgfpathrectangle{\pgfqpoint{1.150000in}{0.150000in}}{\pgfqpoint{5.700000in}{5.700000in}}%
\pgfusepath{clip}%
\pgfsetbuttcap%
\pgfsetroundjoin%
\definecolor{currentfill}{rgb}{0.255645,0.260703,0.528312}%
\pgfsetfillcolor{currentfill}%
\pgfsetfillopacity{0.700000}%
\pgfsetlinewidth{0.000000pt}%
\definecolor{currentstroke}{rgb}{0.000000,0.000000,0.000000}%
\pgfsetstrokecolor{currentstroke}%
\pgfsetdash{}{0pt}%
\pgfpathmoveto{\pgfqpoint{5.189731in}{2.897781in}}%
\pgfpathlineto{\pgfqpoint{5.203515in}{2.900301in}}%
\pgfpathlineto{\pgfqpoint{5.217311in}{2.902921in}}%
\pgfpathlineto{\pgfqpoint{5.231118in}{2.905640in}}%
\pgfpathlineto{\pgfqpoint{5.244937in}{2.908459in}}%
\pgfpathlineto{\pgfqpoint{5.237622in}{2.901514in}}%
\pgfpathlineto{\pgfqpoint{5.230302in}{2.894544in}}%
\pgfpathlineto{\pgfqpoint{5.222975in}{2.887546in}}%
\pgfpathlineto{\pgfqpoint{5.215643in}{2.880519in}}%
\pgfpathlineto{\pgfqpoint{5.201810in}{2.877546in}}%
\pgfpathlineto{\pgfqpoint{5.187989in}{2.874673in}}%
\pgfpathlineto{\pgfqpoint{5.174179in}{2.871900in}}%
\pgfpathlineto{\pgfqpoint{5.160381in}{2.869228in}}%
\pgfpathlineto{\pgfqpoint{5.167727in}{2.876402in}}%
\pgfpathlineto{\pgfqpoint{5.175068in}{2.883551in}}%
\pgfpathlineto{\pgfqpoint{5.182402in}{2.890677in}}%
\pgfpathlineto{\pgfqpoint{5.189731in}{2.897781in}}%
\pgfpathclose%
\pgfusepath{fill}%
\end{pgfscope}%
\begin{pgfscope}%
\pgfpathrectangle{\pgfqpoint{1.150000in}{0.150000in}}{\pgfqpoint{5.700000in}{5.700000in}}%
\pgfusepath{clip}%
\pgfsetbuttcap%
\pgfsetroundjoin%
\definecolor{currentfill}{rgb}{0.265145,0.232956,0.516599}%
\pgfsetfillcolor{currentfill}%
\pgfsetfillopacity{0.700000}%
\pgfsetlinewidth{0.000000pt}%
\definecolor{currentstroke}{rgb}{0.000000,0.000000,0.000000}%
\pgfsetstrokecolor{currentstroke}%
\pgfsetdash{}{0pt}%
\pgfpathmoveto{\pgfqpoint{2.885303in}{2.873400in}}%
\pgfpathlineto{\pgfqpoint{2.898683in}{2.859312in}}%
\pgfpathlineto{\pgfqpoint{2.912060in}{2.845412in}}%
\pgfpathlineto{\pgfqpoint{2.925433in}{2.831696in}}%
\pgfpathlineto{\pgfqpoint{2.938804in}{2.818163in}}%
\pgfpathlineto{\pgfqpoint{2.930591in}{2.813058in}}%
\pgfpathlineto{\pgfqpoint{2.922369in}{2.808060in}}%
\pgfpathlineto{\pgfqpoint{2.914138in}{2.803171in}}%
\pgfpathlineto{\pgfqpoint{2.905897in}{2.798393in}}%
\pgfpathlineto{\pgfqpoint{2.892502in}{2.812127in}}%
\pgfpathlineto{\pgfqpoint{2.879103in}{2.826045in}}%
\pgfpathlineto{\pgfqpoint{2.865701in}{2.840149in}}%
\pgfpathlineto{\pgfqpoint{2.852295in}{2.854439in}}%
\pgfpathlineto{\pgfqpoint{2.860561in}{2.859007in}}%
\pgfpathlineto{\pgfqpoint{2.868818in}{2.863692in}}%
\pgfpathlineto{\pgfqpoint{2.877065in}{2.868489in}}%
\pgfpathlineto{\pgfqpoint{2.885303in}{2.873400in}}%
\pgfpathclose%
\pgfusepath{fill}%
\end{pgfscope}%
\begin{pgfscope}%
\pgfpathrectangle{\pgfqpoint{1.150000in}{0.150000in}}{\pgfqpoint{5.700000in}{5.700000in}}%
\pgfusepath{clip}%
\pgfsetbuttcap%
\pgfsetroundjoin%
\definecolor{currentfill}{rgb}{0.262138,0.242286,0.520837}%
\pgfsetfillcolor{currentfill}%
\pgfsetfillopacity{0.700000}%
\pgfsetlinewidth{0.000000pt}%
\definecolor{currentstroke}{rgb}{0.000000,0.000000,0.000000}%
\pgfsetstrokecolor{currentstroke}%
\pgfsetdash{}{0pt}%
\pgfpathmoveto{\pgfqpoint{5.105302in}{2.859539in}}%
\pgfpathlineto{\pgfqpoint{5.119055in}{2.861810in}}%
\pgfpathlineto{\pgfqpoint{5.132819in}{2.864182in}}%
\pgfpathlineto{\pgfqpoint{5.146594in}{2.866655in}}%
\pgfpathlineto{\pgfqpoint{5.160381in}{2.869228in}}%
\pgfpathlineto{\pgfqpoint{5.153029in}{2.862026in}}%
\pgfpathlineto{\pgfqpoint{5.145671in}{2.854796in}}%
\pgfpathlineto{\pgfqpoint{5.138307in}{2.847536in}}%
\pgfpathlineto{\pgfqpoint{5.130937in}{2.840243in}}%
\pgfpathlineto{\pgfqpoint{5.117137in}{2.837535in}}%
\pgfpathlineto{\pgfqpoint{5.103349in}{2.834928in}}%
\pgfpathlineto{\pgfqpoint{5.089572in}{2.832421in}}%
\pgfpathlineto{\pgfqpoint{5.075806in}{2.830015in}}%
\pgfpathlineto{\pgfqpoint{5.083189in}{2.837436in}}%
\pgfpathlineto{\pgfqpoint{5.090566in}{2.844829in}}%
\pgfpathlineto{\pgfqpoint{5.097937in}{2.852196in}}%
\pgfpathlineto{\pgfqpoint{5.105302in}{2.859539in}}%
\pgfpathclose%
\pgfusepath{fill}%
\end{pgfscope}%
\begin{pgfscope}%
\pgfpathrectangle{\pgfqpoint{1.150000in}{0.150000in}}{\pgfqpoint{5.700000in}{5.700000in}}%
\pgfusepath{clip}%
\pgfsetbuttcap%
\pgfsetroundjoin%
\definecolor{currentfill}{rgb}{0.277018,0.050344,0.375715}%
\pgfsetfillcolor{currentfill}%
\pgfsetfillopacity{0.700000}%
\pgfsetlinewidth{0.000000pt}%
\definecolor{currentstroke}{rgb}{0.000000,0.000000,0.000000}%
\pgfsetstrokecolor{currentstroke}%
\pgfsetdash{}{0pt}%
\pgfpathmoveto{\pgfqpoint{4.067713in}{2.498352in}}%
\pgfpathlineto{\pgfqpoint{4.081122in}{2.495521in}}%
\pgfpathlineto{\pgfqpoint{4.094538in}{2.492806in}}%
\pgfpathlineto{\pgfqpoint{4.107960in}{2.490207in}}%
\pgfpathlineto{\pgfqpoint{4.121389in}{2.487723in}}%
\pgfpathlineto{\pgfqpoint{4.113653in}{2.479000in}}%
\pgfpathlineto{\pgfqpoint{4.105911in}{2.470274in}}%
\pgfpathlineto{\pgfqpoint{4.098165in}{2.461548in}}%
\pgfpathlineto{\pgfqpoint{4.090412in}{2.452820in}}%
\pgfpathlineto{\pgfqpoint{4.076972in}{2.455370in}}%
\pgfpathlineto{\pgfqpoint{4.063539in}{2.458036in}}%
\pgfpathlineto{\pgfqpoint{4.050112in}{2.460817in}}%
\pgfpathlineto{\pgfqpoint{4.036692in}{2.463714in}}%
\pgfpathlineto{\pgfqpoint{4.044455in}{2.472369in}}%
\pgfpathlineto{\pgfqpoint{4.052213in}{2.481027in}}%
\pgfpathlineto{\pgfqpoint{4.059966in}{2.489688in}}%
\pgfpathlineto{\pgfqpoint{4.067713in}{2.498352in}}%
\pgfpathclose%
\pgfusepath{fill}%
\end{pgfscope}%
\begin{pgfscope}%
\pgfpathrectangle{\pgfqpoint{1.150000in}{0.150000in}}{\pgfqpoint{5.700000in}{5.700000in}}%
\pgfusepath{clip}%
\pgfsetbuttcap%
\pgfsetroundjoin%
\definecolor{currentfill}{rgb}{0.282910,0.105393,0.426902}%
\pgfsetfillcolor{currentfill}%
\pgfsetfillopacity{0.700000}%
\pgfsetlinewidth{0.000000pt}%
\definecolor{currentstroke}{rgb}{0.000000,0.000000,0.000000}%
\pgfsetstrokecolor{currentstroke}%
\pgfsetdash{}{0pt}%
\pgfpathmoveto{\pgfqpoint{3.238148in}{2.608725in}}%
\pgfpathlineto{\pgfqpoint{3.251471in}{2.598879in}}%
\pgfpathlineto{\pgfqpoint{3.264794in}{2.589186in}}%
\pgfpathlineto{\pgfqpoint{3.278119in}{2.579644in}}%
\pgfpathlineto{\pgfqpoint{3.291443in}{2.570253in}}%
\pgfpathlineto{\pgfqpoint{3.283393in}{2.563694in}}%
\pgfpathlineto{\pgfqpoint{3.275334in}{2.557208in}}%
\pgfpathlineto{\pgfqpoint{3.267269in}{2.550798in}}%
\pgfpathlineto{\pgfqpoint{3.259196in}{2.544464in}}%
\pgfpathlineto{\pgfqpoint{3.245851in}{2.554033in}}%
\pgfpathlineto{\pgfqpoint{3.232508in}{2.563752in}}%
\pgfpathlineto{\pgfqpoint{3.219164in}{2.573624in}}%
\pgfpathlineto{\pgfqpoint{3.205821in}{2.583648in}}%
\pgfpathlineto{\pgfqpoint{3.213915in}{2.589797in}}%
\pgfpathlineto{\pgfqpoint{3.222000in}{2.596028in}}%
\pgfpathlineto{\pgfqpoint{3.230078in}{2.602337in}}%
\pgfpathlineto{\pgfqpoint{3.238148in}{2.608725in}}%
\pgfpathclose%
\pgfusepath{fill}%
\end{pgfscope}%
\begin{pgfscope}%
\pgfpathrectangle{\pgfqpoint{1.150000in}{0.150000in}}{\pgfqpoint{5.700000in}{5.700000in}}%
\pgfusepath{clip}%
\pgfsetbuttcap%
\pgfsetroundjoin%
\definecolor{currentfill}{rgb}{0.274952,0.037752,0.364543}%
\pgfsetfillcolor{currentfill}%
\pgfsetfillopacity{0.700000}%
\pgfsetlinewidth{0.000000pt}%
\definecolor{currentstroke}{rgb}{0.000000,0.000000,0.000000}%
\pgfsetstrokecolor{currentstroke}%
\pgfsetdash{}{0pt}%
\pgfpathmoveto{\pgfqpoint{3.844862in}{2.474148in}}%
\pgfpathlineto{\pgfqpoint{3.858225in}{2.469734in}}%
\pgfpathlineto{\pgfqpoint{3.871594in}{2.465443in}}%
\pgfpathlineto{\pgfqpoint{3.884968in}{2.461273in}}%
\pgfpathlineto{\pgfqpoint{3.898347in}{2.457224in}}%
\pgfpathlineto{\pgfqpoint{3.890534in}{2.448784in}}%
\pgfpathlineto{\pgfqpoint{3.882714in}{2.440360in}}%
\pgfpathlineto{\pgfqpoint{3.874890in}{2.431952in}}%
\pgfpathlineto{\pgfqpoint{3.867059in}{2.423560in}}%
\pgfpathlineto{\pgfqpoint{3.853668in}{2.427711in}}%
\pgfpathlineto{\pgfqpoint{3.840281in}{2.431983in}}%
\pgfpathlineto{\pgfqpoint{3.826900in}{2.436378in}}%
\pgfpathlineto{\pgfqpoint{3.813524in}{2.440894in}}%
\pgfpathlineto{\pgfqpoint{3.821367in}{2.449176in}}%
\pgfpathlineto{\pgfqpoint{3.829204in}{2.457480in}}%
\pgfpathlineto{\pgfqpoint{3.837036in}{2.465804in}}%
\pgfpathlineto{\pgfqpoint{3.844862in}{2.474148in}}%
\pgfpathclose%
\pgfusepath{fill}%
\end{pgfscope}%
\begin{pgfscope}%
\pgfpathrectangle{\pgfqpoint{1.150000in}{0.150000in}}{\pgfqpoint{5.700000in}{5.700000in}}%
\pgfusepath{clip}%
\pgfsetbuttcap%
\pgfsetroundjoin%
\definecolor{currentfill}{rgb}{0.279566,0.067836,0.391917}%
\pgfsetfillcolor{currentfill}%
\pgfsetfillopacity{0.700000}%
\pgfsetlinewidth{0.000000pt}%
\definecolor{currentstroke}{rgb}{0.000000,0.000000,0.000000}%
\pgfsetstrokecolor{currentstroke}%
\pgfsetdash{}{0pt}%
\pgfpathmoveto{\pgfqpoint{3.430072in}{2.528631in}}%
\pgfpathlineto{\pgfqpoint{3.443392in}{2.520706in}}%
\pgfpathlineto{\pgfqpoint{3.456714in}{2.512922in}}%
\pgfpathlineto{\pgfqpoint{3.470038in}{2.505278in}}%
\pgfpathlineto{\pgfqpoint{3.483365in}{2.497772in}}%
\pgfpathlineto{\pgfqpoint{3.475395in}{2.490478in}}%
\pgfpathlineto{\pgfqpoint{3.467418in}{2.483238in}}%
\pgfpathlineto{\pgfqpoint{3.459435in}{2.476053in}}%
\pgfpathlineto{\pgfqpoint{3.451445in}{2.468926in}}%
\pgfpathlineto{\pgfqpoint{3.438102in}{2.476590in}}%
\pgfpathlineto{\pgfqpoint{3.424760in}{2.484392in}}%
\pgfpathlineto{\pgfqpoint{3.411421in}{2.492334in}}%
\pgfpathlineto{\pgfqpoint{3.398084in}{2.500417in}}%
\pgfpathlineto{\pgfqpoint{3.406091in}{2.507380in}}%
\pgfpathlineto{\pgfqpoint{3.414092in}{2.514404in}}%
\pgfpathlineto{\pgfqpoint{3.422085in}{2.521488in}}%
\pgfpathlineto{\pgfqpoint{3.430072in}{2.528631in}}%
\pgfpathclose%
\pgfusepath{fill}%
\end{pgfscope}%
\begin{pgfscope}%
\pgfpathrectangle{\pgfqpoint{1.150000in}{0.150000in}}{\pgfqpoint{5.700000in}{5.700000in}}%
\pgfusepath{clip}%
\pgfsetbuttcap%
\pgfsetroundjoin%
\definecolor{currentfill}{rgb}{0.282327,0.094955,0.417331}%
\pgfsetfillcolor{currentfill}%
\pgfsetfillopacity{0.700000}%
\pgfsetlinewidth{0.000000pt}%
\definecolor{currentstroke}{rgb}{0.000000,0.000000,0.000000}%
\pgfsetstrokecolor{currentstroke}%
\pgfsetdash{}{0pt}%
\pgfpathmoveto{\pgfqpoint{4.375142in}{2.572289in}}%
\pgfpathlineto{\pgfqpoint{4.388636in}{2.571354in}}%
\pgfpathlineto{\pgfqpoint{4.402138in}{2.570528in}}%
\pgfpathlineto{\pgfqpoint{4.415648in}{2.569811in}}%
\pgfpathlineto{\pgfqpoint{4.429167in}{2.569203in}}%
\pgfpathlineto{\pgfqpoint{4.421535in}{2.560490in}}%
\pgfpathlineto{\pgfqpoint{4.413898in}{2.551758in}}%
\pgfpathlineto{\pgfqpoint{4.406256in}{2.543005in}}%
\pgfpathlineto{\pgfqpoint{4.398608in}{2.534231in}}%
\pgfpathlineto{\pgfqpoint{4.385080in}{2.534851in}}%
\pgfpathlineto{\pgfqpoint{4.371559in}{2.535580in}}%
\pgfpathlineto{\pgfqpoint{4.358047in}{2.536418in}}%
\pgfpathlineto{\pgfqpoint{4.344543in}{2.537366in}}%
\pgfpathlineto{\pgfqpoint{4.352201in}{2.546121in}}%
\pgfpathlineto{\pgfqpoint{4.359853in}{2.554859in}}%
\pgfpathlineto{\pgfqpoint{4.367500in}{2.563582in}}%
\pgfpathlineto{\pgfqpoint{4.375142in}{2.572289in}}%
\pgfpathclose%
\pgfusepath{fill}%
\end{pgfscope}%
\begin{pgfscope}%
\pgfpathrectangle{\pgfqpoint{1.150000in}{0.150000in}}{\pgfqpoint{5.700000in}{5.700000in}}%
\pgfusepath{clip}%
\pgfsetbuttcap%
\pgfsetroundjoin%
\definecolor{currentfill}{rgb}{0.271828,0.209303,0.504434}%
\pgfsetfillcolor{currentfill}%
\pgfsetfillopacity{0.700000}%
\pgfsetlinewidth{0.000000pt}%
\definecolor{currentstroke}{rgb}{0.000000,0.000000,0.000000}%
\pgfsetstrokecolor{currentstroke}%
\pgfsetdash{}{0pt}%
\pgfpathmoveto{\pgfqpoint{2.938804in}{2.818163in}}%
\pgfpathlineto{\pgfqpoint{2.952171in}{2.804812in}}%
\pgfpathlineto{\pgfqpoint{2.965536in}{2.791641in}}%
\pgfpathlineto{\pgfqpoint{2.978899in}{2.778648in}}%
\pgfpathlineto{\pgfqpoint{2.992259in}{2.765832in}}%
\pgfpathlineto{\pgfqpoint{2.984070in}{2.760532in}}%
\pgfpathlineto{\pgfqpoint{2.975872in}{2.755335in}}%
\pgfpathlineto{\pgfqpoint{2.967665in}{2.750243in}}%
\pgfpathlineto{\pgfqpoint{2.959449in}{2.745256in}}%
\pgfpathlineto{\pgfqpoint{2.946065in}{2.758273in}}%
\pgfpathlineto{\pgfqpoint{2.932679in}{2.771467in}}%
\pgfpathlineto{\pgfqpoint{2.919289in}{2.784840in}}%
\pgfpathlineto{\pgfqpoint{2.905897in}{2.798393in}}%
\pgfpathlineto{\pgfqpoint{2.914138in}{2.803171in}}%
\pgfpathlineto{\pgfqpoint{2.922369in}{2.808060in}}%
\pgfpathlineto{\pgfqpoint{2.930591in}{2.813058in}}%
\pgfpathlineto{\pgfqpoint{2.938804in}{2.818163in}}%
\pgfpathclose%
\pgfusepath{fill}%
\end{pgfscope}%
\begin{pgfscope}%
\pgfpathrectangle{\pgfqpoint{1.150000in}{0.150000in}}{\pgfqpoint{5.700000in}{5.700000in}}%
\pgfusepath{clip}%
\pgfsetbuttcap%
\pgfsetroundjoin%
\definecolor{currentfill}{rgb}{0.267968,0.223549,0.512008}%
\pgfsetfillcolor{currentfill}%
\pgfsetfillopacity{0.700000}%
\pgfsetlinewidth{0.000000pt}%
\definecolor{currentstroke}{rgb}{0.000000,0.000000,0.000000}%
\pgfsetstrokecolor{currentstroke}%
\pgfsetdash{}{0pt}%
\pgfpathmoveto{\pgfqpoint{5.020853in}{2.821400in}}%
\pgfpathlineto{\pgfqpoint{5.034575in}{2.823402in}}%
\pgfpathlineto{\pgfqpoint{5.048308in}{2.825505in}}%
\pgfpathlineto{\pgfqpoint{5.062051in}{2.827710in}}%
\pgfpathlineto{\pgfqpoint{5.075806in}{2.830015in}}%
\pgfpathlineto{\pgfqpoint{5.068417in}{2.822565in}}%
\pgfpathlineto{\pgfqpoint{5.061023in}{2.815084in}}%
\pgfpathlineto{\pgfqpoint{5.053623in}{2.807570in}}%
\pgfpathlineto{\pgfqpoint{5.046216in}{2.800023in}}%
\pgfpathlineto{\pgfqpoint{5.032449in}{2.797601in}}%
\pgfpathlineto{\pgfqpoint{5.018693in}{2.795280in}}%
\pgfpathlineto{\pgfqpoint{5.004948in}{2.793061in}}%
\pgfpathlineto{\pgfqpoint{4.991214in}{2.790942in}}%
\pgfpathlineto{\pgfqpoint{4.998632in}{2.798600in}}%
\pgfpathlineto{\pgfqpoint{5.006045in}{2.806227in}}%
\pgfpathlineto{\pgfqpoint{5.013452in}{2.813827in}}%
\pgfpathlineto{\pgfqpoint{5.020853in}{2.821400in}}%
\pgfpathclose%
\pgfusepath{fill}%
\end{pgfscope}%
\begin{pgfscope}%
\pgfpathrectangle{\pgfqpoint{1.150000in}{0.150000in}}{\pgfqpoint{5.700000in}{5.700000in}}%
\pgfusepath{clip}%
\pgfsetbuttcap%
\pgfsetroundjoin%
\definecolor{currentfill}{rgb}{0.273006,0.204520,0.501721}%
\pgfsetfillcolor{currentfill}%
\pgfsetfillopacity{0.700000}%
\pgfsetlinewidth{0.000000pt}%
\definecolor{currentstroke}{rgb}{0.000000,0.000000,0.000000}%
\pgfsetstrokecolor{currentstroke}%
\pgfsetdash{}{0pt}%
\pgfpathmoveto{\pgfqpoint{4.936384in}{2.783488in}}%
\pgfpathlineto{\pgfqpoint{4.950076in}{2.785199in}}%
\pgfpathlineto{\pgfqpoint{4.963778in}{2.787011in}}%
\pgfpathlineto{\pgfqpoint{4.977491in}{2.788926in}}%
\pgfpathlineto{\pgfqpoint{4.991214in}{2.790942in}}%
\pgfpathlineto{\pgfqpoint{4.983790in}{2.783254in}}%
\pgfpathlineto{\pgfqpoint{4.976360in}{2.775534in}}%
\pgfpathlineto{\pgfqpoint{4.968924in}{2.767780in}}%
\pgfpathlineto{\pgfqpoint{4.961482in}{2.759991in}}%
\pgfpathlineto{\pgfqpoint{4.947747in}{2.757876in}}%
\pgfpathlineto{\pgfqpoint{4.934023in}{2.755863in}}%
\pgfpathlineto{\pgfqpoint{4.920309in}{2.753953in}}%
\pgfpathlineto{\pgfqpoint{4.906606in}{2.752144in}}%
\pgfpathlineto{\pgfqpoint{4.914059in}{2.760025in}}%
\pgfpathlineto{\pgfqpoint{4.921507in}{2.767874in}}%
\pgfpathlineto{\pgfqpoint{4.928948in}{2.775695in}}%
\pgfpathlineto{\pgfqpoint{4.936384in}{2.783488in}}%
\pgfpathclose%
\pgfusepath{fill}%
\end{pgfscope}%
\begin{pgfscope}%
\pgfpathrectangle{\pgfqpoint{1.150000in}{0.150000in}}{\pgfqpoint{5.700000in}{5.700000in}}%
\pgfusepath{clip}%
\pgfsetbuttcap%
\pgfsetroundjoin%
\definecolor{currentfill}{rgb}{0.277134,0.185228,0.489898}%
\pgfsetfillcolor{currentfill}%
\pgfsetfillopacity{0.700000}%
\pgfsetlinewidth{0.000000pt}%
\definecolor{currentstroke}{rgb}{0.000000,0.000000,0.000000}%
\pgfsetstrokecolor{currentstroke}%
\pgfsetdash{}{0pt}%
\pgfpathmoveto{\pgfqpoint{2.992259in}{2.765832in}}%
\pgfpathlineto{\pgfqpoint{3.005616in}{2.753191in}}%
\pgfpathlineto{\pgfqpoint{3.018972in}{2.740724in}}%
\pgfpathlineto{\pgfqpoint{3.032326in}{2.728429in}}%
\pgfpathlineto{\pgfqpoint{3.045678in}{2.716305in}}%
\pgfpathlineto{\pgfqpoint{3.037512in}{2.710813in}}%
\pgfpathlineto{\pgfqpoint{3.029338in}{2.705418in}}%
\pgfpathlineto{\pgfqpoint{3.021155in}{2.700123in}}%
\pgfpathlineto{\pgfqpoint{3.012963in}{2.694929in}}%
\pgfpathlineto{\pgfqpoint{2.999587in}{2.707253in}}%
\pgfpathlineto{\pgfqpoint{2.986210in}{2.719748in}}%
\pgfpathlineto{\pgfqpoint{2.972831in}{2.732415in}}%
\pgfpathlineto{\pgfqpoint{2.959449in}{2.745256in}}%
\pgfpathlineto{\pgfqpoint{2.967665in}{2.750243in}}%
\pgfpathlineto{\pgfqpoint{2.975872in}{2.755335in}}%
\pgfpathlineto{\pgfqpoint{2.984070in}{2.760532in}}%
\pgfpathlineto{\pgfqpoint{2.992259in}{2.765832in}}%
\pgfpathclose%
\pgfusepath{fill}%
\end{pgfscope}%
\begin{pgfscope}%
\pgfpathrectangle{\pgfqpoint{1.150000in}{0.150000in}}{\pgfqpoint{5.700000in}{5.700000in}}%
\pgfusepath{clip}%
\pgfsetbuttcap%
\pgfsetroundjoin%
\definecolor{currentfill}{rgb}{0.281446,0.084320,0.407414}%
\pgfsetfillcolor{currentfill}%
\pgfsetfillopacity{0.700000}%
\pgfsetlinewidth{0.000000pt}%
\definecolor{currentstroke}{rgb}{0.000000,0.000000,0.000000}%
\pgfsetstrokecolor{currentstroke}%
\pgfsetdash{}{0pt}%
\pgfpathmoveto{\pgfqpoint{4.290605in}{2.542258in}}%
\pgfpathlineto{\pgfqpoint{4.304078in}{2.540869in}}%
\pgfpathlineto{\pgfqpoint{4.317558in}{2.539591in}}%
\pgfpathlineto{\pgfqpoint{4.331047in}{2.538423in}}%
\pgfpathlineto{\pgfqpoint{4.344543in}{2.537366in}}%
\pgfpathlineto{\pgfqpoint{4.336880in}{2.528594in}}%
\pgfpathlineto{\pgfqpoint{4.329212in}{2.519806in}}%
\pgfpathlineto{\pgfqpoint{4.321538in}{2.511002in}}%
\pgfpathlineto{\pgfqpoint{4.313859in}{2.502182in}}%
\pgfpathlineto{\pgfqpoint{4.300353in}{2.503269in}}%
\pgfpathlineto{\pgfqpoint{4.286855in}{2.504467in}}%
\pgfpathlineto{\pgfqpoint{4.273364in}{2.505775in}}%
\pgfpathlineto{\pgfqpoint{4.259881in}{2.507194in}}%
\pgfpathlineto{\pgfqpoint{4.267570in}{2.515978in}}%
\pgfpathlineto{\pgfqpoint{4.275254in}{2.524750in}}%
\pgfpathlineto{\pgfqpoint{4.282932in}{2.533510in}}%
\pgfpathlineto{\pgfqpoint{4.290605in}{2.542258in}}%
\pgfpathclose%
\pgfusepath{fill}%
\end{pgfscope}%
\begin{pgfscope}%
\pgfpathrectangle{\pgfqpoint{1.150000in}{0.150000in}}{\pgfqpoint{5.700000in}{5.700000in}}%
\pgfusepath{clip}%
\pgfsetbuttcap%
\pgfsetroundjoin%
\definecolor{currentfill}{rgb}{0.276194,0.190074,0.493001}%
\pgfsetfillcolor{currentfill}%
\pgfsetfillopacity{0.700000}%
\pgfsetlinewidth{0.000000pt}%
\definecolor{currentstroke}{rgb}{0.000000,0.000000,0.000000}%
\pgfsetstrokecolor{currentstroke}%
\pgfsetdash{}{0pt}%
\pgfpathmoveto{\pgfqpoint{4.851898in}{2.745936in}}%
\pgfpathlineto{\pgfqpoint{4.865559in}{2.747334in}}%
\pgfpathlineto{\pgfqpoint{4.879231in}{2.748835in}}%
\pgfpathlineto{\pgfqpoint{4.892913in}{2.750438in}}%
\pgfpathlineto{\pgfqpoint{4.906606in}{2.752144in}}%
\pgfpathlineto{\pgfqpoint{4.899147in}{2.744232in}}%
\pgfpathlineto{\pgfqpoint{4.891682in}{2.736287in}}%
\pgfpathlineto{\pgfqpoint{4.884212in}{2.728308in}}%
\pgfpathlineto{\pgfqpoint{4.876736in}{2.720294in}}%
\pgfpathlineto{\pgfqpoint{4.863032in}{2.718508in}}%
\pgfpathlineto{\pgfqpoint{4.849339in}{2.716825in}}%
\pgfpathlineto{\pgfqpoint{4.835656in}{2.715244in}}%
\pgfpathlineto{\pgfqpoint{4.821983in}{2.713767in}}%
\pgfpathlineto{\pgfqpoint{4.829470in}{2.721854in}}%
\pgfpathlineto{\pgfqpoint{4.836952in}{2.729911in}}%
\pgfpathlineto{\pgfqpoint{4.844428in}{2.737937in}}%
\pgfpathlineto{\pgfqpoint{4.851898in}{2.745936in}}%
\pgfpathclose%
\pgfusepath{fill}%
\end{pgfscope}%
\begin{pgfscope}%
\pgfpathrectangle{\pgfqpoint{1.150000in}{0.150000in}}{\pgfqpoint{5.700000in}{5.700000in}}%
\pgfusepath{clip}%
\pgfsetbuttcap%
\pgfsetroundjoin%
\definecolor{currentfill}{rgb}{0.276022,0.044167,0.370164}%
\pgfsetfillcolor{currentfill}%
\pgfsetfillopacity{0.700000}%
\pgfsetlinewidth{0.000000pt}%
\definecolor{currentstroke}{rgb}{0.000000,0.000000,0.000000}%
\pgfsetstrokecolor{currentstroke}%
\pgfsetdash{}{0pt}%
\pgfpathmoveto{\pgfqpoint{3.983072in}{2.476473in}}%
\pgfpathlineto{\pgfqpoint{3.996468in}{2.473107in}}%
\pgfpathlineto{\pgfqpoint{4.009870in}{2.469858in}}%
\pgfpathlineto{\pgfqpoint{4.023278in}{2.466728in}}%
\pgfpathlineto{\pgfqpoint{4.036692in}{2.463714in}}%
\pgfpathlineto{\pgfqpoint{4.028923in}{2.455063in}}%
\pgfpathlineto{\pgfqpoint{4.021149in}{2.446416in}}%
\pgfpathlineto{\pgfqpoint{4.013370in}{2.437773in}}%
\pgfpathlineto{\pgfqpoint{4.005585in}{2.429136in}}%
\pgfpathlineto{\pgfqpoint{3.992160in}{2.432234in}}%
\pgfpathlineto{\pgfqpoint{3.978740in}{2.435449in}}%
\pgfpathlineto{\pgfqpoint{3.965327in}{2.438781in}}%
\pgfpathlineto{\pgfqpoint{3.951920in}{2.442232in}}%
\pgfpathlineto{\pgfqpoint{3.959716in}{2.450778in}}%
\pgfpathlineto{\pgfqpoint{3.967507in}{2.459334in}}%
\pgfpathlineto{\pgfqpoint{3.975292in}{2.467899in}}%
\pgfpathlineto{\pgfqpoint{3.983072in}{2.476473in}}%
\pgfpathclose%
\pgfusepath{fill}%
\end{pgfscope}%
\begin{pgfscope}%
\pgfpathrectangle{\pgfqpoint{1.150000in}{0.150000in}}{\pgfqpoint{5.700000in}{5.700000in}}%
\pgfusepath{clip}%
\pgfsetbuttcap%
\pgfsetroundjoin%
\definecolor{currentfill}{rgb}{0.279574,0.170599,0.479997}%
\pgfsetfillcolor{currentfill}%
\pgfsetfillopacity{0.700000}%
\pgfsetlinewidth{0.000000pt}%
\definecolor{currentstroke}{rgb}{0.000000,0.000000,0.000000}%
\pgfsetstrokecolor{currentstroke}%
\pgfsetdash{}{0pt}%
\pgfpathmoveto{\pgfqpoint{4.767393in}{2.708892in}}%
\pgfpathlineto{\pgfqpoint{4.781026in}{2.709955in}}%
\pgfpathlineto{\pgfqpoint{4.794668in}{2.711122in}}%
\pgfpathlineto{\pgfqpoint{4.808321in}{2.712393in}}%
\pgfpathlineto{\pgfqpoint{4.821983in}{2.713767in}}%
\pgfpathlineto{\pgfqpoint{4.814490in}{2.705648in}}%
\pgfpathlineto{\pgfqpoint{4.806992in}{2.697496in}}%
\pgfpathlineto{\pgfqpoint{4.799488in}{2.689311in}}%
\pgfpathlineto{\pgfqpoint{4.791978in}{2.681090in}}%
\pgfpathlineto{\pgfqpoint{4.778305in}{2.679655in}}%
\pgfpathlineto{\pgfqpoint{4.764642in}{2.678323in}}%
\pgfpathlineto{\pgfqpoint{4.750988in}{2.677095in}}%
\pgfpathlineto{\pgfqpoint{4.737345in}{2.675970in}}%
\pgfpathlineto{\pgfqpoint{4.744866in}{2.684245in}}%
\pgfpathlineto{\pgfqpoint{4.752380in}{2.692490in}}%
\pgfpathlineto{\pgfqpoint{4.759890in}{2.700705in}}%
\pgfpathlineto{\pgfqpoint{4.767393in}{2.708892in}}%
\pgfpathclose%
\pgfusepath{fill}%
\end{pgfscope}%
\begin{pgfscope}%
\pgfpathrectangle{\pgfqpoint{1.150000in}{0.150000in}}{\pgfqpoint{5.700000in}{5.700000in}}%
\pgfusepath{clip}%
\pgfsetbuttcap%
\pgfsetroundjoin%
\definecolor{currentfill}{rgb}{0.282327,0.094955,0.417331}%
\pgfsetfillcolor{currentfill}%
\pgfsetfillopacity{0.700000}%
\pgfsetlinewidth{0.000000pt}%
\definecolor{currentstroke}{rgb}{0.000000,0.000000,0.000000}%
\pgfsetstrokecolor{currentstroke}%
\pgfsetdash{}{0pt}%
\pgfpathmoveto{\pgfqpoint{3.291443in}{2.570253in}}%
\pgfpathlineto{\pgfqpoint{3.304769in}{2.561011in}}%
\pgfpathlineto{\pgfqpoint{3.318096in}{2.551918in}}%
\pgfpathlineto{\pgfqpoint{3.331424in}{2.542972in}}%
\pgfpathlineto{\pgfqpoint{3.344753in}{2.534172in}}%
\pgfpathlineto{\pgfqpoint{3.336721in}{2.527443in}}%
\pgfpathlineto{\pgfqpoint{3.328681in}{2.520782in}}%
\pgfpathlineto{\pgfqpoint{3.320635in}{2.514192in}}%
\pgfpathlineto{\pgfqpoint{3.312581in}{2.507673in}}%
\pgfpathlineto{\pgfqpoint{3.299233in}{2.516650in}}%
\pgfpathlineto{\pgfqpoint{3.285886in}{2.525773in}}%
\pgfpathlineto{\pgfqpoint{3.272541in}{2.535044in}}%
\pgfpathlineto{\pgfqpoint{3.259196in}{2.544464in}}%
\pgfpathlineto{\pgfqpoint{3.267269in}{2.550798in}}%
\pgfpathlineto{\pgfqpoint{3.275334in}{2.557208in}}%
\pgfpathlineto{\pgfqpoint{3.283393in}{2.563694in}}%
\pgfpathlineto{\pgfqpoint{3.291443in}{2.570253in}}%
\pgfpathclose%
\pgfusepath{fill}%
\end{pgfscope}%
\begin{pgfscope}%
\pgfpathrectangle{\pgfqpoint{1.150000in}{0.150000in}}{\pgfqpoint{5.700000in}{5.700000in}}%
\pgfusepath{clip}%
\pgfsetbuttcap%
\pgfsetroundjoin%
\definecolor{currentfill}{rgb}{0.274952,0.037752,0.364543}%
\pgfsetfillcolor{currentfill}%
\pgfsetfillopacity{0.700000}%
\pgfsetlinewidth{0.000000pt}%
\definecolor{currentstroke}{rgb}{0.000000,0.000000,0.000000}%
\pgfsetstrokecolor{currentstroke}%
\pgfsetdash{}{0pt}%
\pgfpathmoveto{\pgfqpoint{3.621770in}{2.473425in}}%
\pgfpathlineto{\pgfqpoint{3.635108in}{2.467269in}}%
\pgfpathlineto{\pgfqpoint{3.648450in}{2.461242in}}%
\pgfpathlineto{\pgfqpoint{3.661796in}{2.455345in}}%
\pgfpathlineto{\pgfqpoint{3.675146in}{2.449578in}}%
\pgfpathlineto{\pgfqpoint{3.667247in}{2.441680in}}%
\pgfpathlineto{\pgfqpoint{3.659342in}{2.433819in}}%
\pgfpathlineto{\pgfqpoint{3.651431in}{2.425995in}}%
\pgfpathlineto{\pgfqpoint{3.643514in}{2.418209in}}%
\pgfpathlineto{\pgfqpoint{3.630149in}{2.424116in}}%
\pgfpathlineto{\pgfqpoint{3.616789in}{2.430151in}}%
\pgfpathlineto{\pgfqpoint{3.603432in}{2.436317in}}%
\pgfpathlineto{\pgfqpoint{3.590079in}{2.442613in}}%
\pgfpathlineto{\pgfqpoint{3.598011in}{2.450253in}}%
\pgfpathlineto{\pgfqpoint{3.605937in}{2.457936in}}%
\pgfpathlineto{\pgfqpoint{3.613856in}{2.465660in}}%
\pgfpathlineto{\pgfqpoint{3.621770in}{2.473425in}}%
\pgfpathclose%
\pgfusepath{fill}%
\end{pgfscope}%
\begin{pgfscope}%
\pgfpathrectangle{\pgfqpoint{1.150000in}{0.150000in}}{\pgfqpoint{5.700000in}{5.700000in}}%
\pgfusepath{clip}%
\pgfsetbuttcap%
\pgfsetroundjoin%
\definecolor{currentfill}{rgb}{0.280255,0.165693,0.476498}%
\pgfsetfillcolor{currentfill}%
\pgfsetfillopacity{0.700000}%
\pgfsetlinewidth{0.000000pt}%
\definecolor{currentstroke}{rgb}{0.000000,0.000000,0.000000}%
\pgfsetstrokecolor{currentstroke}%
\pgfsetdash{}{0pt}%
\pgfpathmoveto{\pgfqpoint{3.045678in}{2.716305in}}%
\pgfpathlineto{\pgfqpoint{3.059028in}{2.704351in}}%
\pgfpathlineto{\pgfqpoint{3.072377in}{2.692564in}}%
\pgfpathlineto{\pgfqpoint{3.085725in}{2.680945in}}%
\pgfpathlineto{\pgfqpoint{3.099072in}{2.669490in}}%
\pgfpathlineto{\pgfqpoint{3.090928in}{2.663805in}}%
\pgfpathlineto{\pgfqpoint{3.082776in}{2.658213in}}%
\pgfpathlineto{\pgfqpoint{3.074616in}{2.652717in}}%
\pgfpathlineto{\pgfqpoint{3.066448in}{2.647318in}}%
\pgfpathlineto{\pgfqpoint{3.053078in}{2.658971in}}%
\pgfpathlineto{\pgfqpoint{3.039708in}{2.670790in}}%
\pgfpathlineto{\pgfqpoint{3.026336in}{2.682775in}}%
\pgfpathlineto{\pgfqpoint{3.012963in}{2.694929in}}%
\pgfpathlineto{\pgfqpoint{3.021155in}{2.700123in}}%
\pgfpathlineto{\pgfqpoint{3.029338in}{2.705418in}}%
\pgfpathlineto{\pgfqpoint{3.037512in}{2.710813in}}%
\pgfpathlineto{\pgfqpoint{3.045678in}{2.716305in}}%
\pgfpathclose%
\pgfusepath{fill}%
\end{pgfscope}%
\begin{pgfscope}%
\pgfpathrectangle{\pgfqpoint{1.150000in}{0.150000in}}{\pgfqpoint{5.700000in}{5.700000in}}%
\pgfusepath{clip}%
\pgfsetbuttcap%
\pgfsetroundjoin%
\definecolor{currentfill}{rgb}{0.273809,0.031497,0.358853}%
\pgfsetfillcolor{currentfill}%
\pgfsetfillopacity{0.700000}%
\pgfsetlinewidth{0.000000pt}%
\definecolor{currentstroke}{rgb}{0.000000,0.000000,0.000000}%
\pgfsetstrokecolor{currentstroke}%
\pgfsetdash{}{0pt}%
\pgfpathmoveto{\pgfqpoint{3.760067in}{2.460197in}}%
\pgfpathlineto{\pgfqpoint{3.773424in}{2.455184in}}%
\pgfpathlineto{\pgfqpoint{3.786786in}{2.450297in}}%
\pgfpathlineto{\pgfqpoint{3.800153in}{2.445534in}}%
\pgfpathlineto{\pgfqpoint{3.813524in}{2.440894in}}%
\pgfpathlineto{\pgfqpoint{3.805675in}{2.432634in}}%
\pgfpathlineto{\pgfqpoint{3.797821in}{2.424398in}}%
\pgfpathlineto{\pgfqpoint{3.789961in}{2.416186in}}%
\pgfpathlineto{\pgfqpoint{3.782095in}{2.407998in}}%
\pgfpathlineto{\pgfqpoint{3.768711in}{2.412758in}}%
\pgfpathlineto{\pgfqpoint{3.755331in}{2.417642in}}%
\pgfpathlineto{\pgfqpoint{3.741956in}{2.422650in}}%
\pgfpathlineto{\pgfqpoint{3.728585in}{2.427783in}}%
\pgfpathlineto{\pgfqpoint{3.736465in}{2.435843in}}%
\pgfpathlineto{\pgfqpoint{3.744338in}{2.443933in}}%
\pgfpathlineto{\pgfqpoint{3.752205in}{2.452051in}}%
\pgfpathlineto{\pgfqpoint{3.760067in}{2.460197in}}%
\pgfpathclose%
\pgfusepath{fill}%
\end{pgfscope}%
\begin{pgfscope}%
\pgfpathrectangle{\pgfqpoint{1.150000in}{0.150000in}}{\pgfqpoint{5.700000in}{5.700000in}}%
\pgfusepath{clip}%
\pgfsetbuttcap%
\pgfsetroundjoin%
\definecolor{currentfill}{rgb}{0.281412,0.155834,0.469201}%
\pgfsetfillcolor{currentfill}%
\pgfsetfillopacity{0.700000}%
\pgfsetlinewidth{0.000000pt}%
\definecolor{currentstroke}{rgb}{0.000000,0.000000,0.000000}%
\pgfsetstrokecolor{currentstroke}%
\pgfsetdash{}{0pt}%
\pgfpathmoveto{\pgfqpoint{4.682870in}{2.672517in}}%
\pgfpathlineto{\pgfqpoint{4.696474in}{2.673223in}}%
\pgfpathlineto{\pgfqpoint{4.710088in}{2.674035in}}%
\pgfpathlineto{\pgfqpoint{4.723712in}{2.674950in}}%
\pgfpathlineto{\pgfqpoint{4.737345in}{2.675970in}}%
\pgfpathlineto{\pgfqpoint{4.729819in}{2.667665in}}%
\pgfpathlineto{\pgfqpoint{4.722288in}{2.659327in}}%
\pgfpathlineto{\pgfqpoint{4.714751in}{2.650956in}}%
\pgfpathlineto{\pgfqpoint{4.707208in}{2.642552in}}%
\pgfpathlineto{\pgfqpoint{4.693564in}{2.641489in}}%
\pgfpathlineto{\pgfqpoint{4.679930in}{2.640530in}}%
\pgfpathlineto{\pgfqpoint{4.666306in}{2.639676in}}%
\pgfpathlineto{\pgfqpoint{4.652691in}{2.638927in}}%
\pgfpathlineto{\pgfqpoint{4.660244in}{2.647367in}}%
\pgfpathlineto{\pgfqpoint{4.667792in}{2.655779in}}%
\pgfpathlineto{\pgfqpoint{4.675333in}{2.664162in}}%
\pgfpathlineto{\pgfqpoint{4.682870in}{2.672517in}}%
\pgfpathclose%
\pgfusepath{fill}%
\end{pgfscope}%
\begin{pgfscope}%
\pgfpathrectangle{\pgfqpoint{1.150000in}{0.150000in}}{\pgfqpoint{5.700000in}{5.700000in}}%
\pgfusepath{clip}%
\pgfsetbuttcap%
\pgfsetroundjoin%
\definecolor{currentfill}{rgb}{0.279566,0.067836,0.391917}%
\pgfsetfillcolor{currentfill}%
\pgfsetfillopacity{0.700000}%
\pgfsetlinewidth{0.000000pt}%
\definecolor{currentstroke}{rgb}{0.000000,0.000000,0.000000}%
\pgfsetstrokecolor{currentstroke}%
\pgfsetdash{}{0pt}%
\pgfpathmoveto{\pgfqpoint{4.206024in}{2.513989in}}%
\pgfpathlineto{\pgfqpoint{4.219477in}{2.512122in}}%
\pgfpathlineto{\pgfqpoint{4.232938in}{2.510368in}}%
\pgfpathlineto{\pgfqpoint{4.246406in}{2.508725in}}%
\pgfpathlineto{\pgfqpoint{4.259881in}{2.507194in}}%
\pgfpathlineto{\pgfqpoint{4.252187in}{2.498399in}}%
\pgfpathlineto{\pgfqpoint{4.244487in}{2.489591in}}%
\pgfpathlineto{\pgfqpoint{4.236783in}{2.480772in}}%
\pgfpathlineto{\pgfqpoint{4.229072in}{2.471941in}}%
\pgfpathlineto{\pgfqpoint{4.215587in}{2.473520in}}%
\pgfpathlineto{\pgfqpoint{4.202109in}{2.475210in}}%
\pgfpathlineto{\pgfqpoint{4.188638in}{2.477013in}}%
\pgfpathlineto{\pgfqpoint{4.175174in}{2.478928in}}%
\pgfpathlineto{\pgfqpoint{4.182894in}{2.487704in}}%
\pgfpathlineto{\pgfqpoint{4.190609in}{2.496473in}}%
\pgfpathlineto{\pgfqpoint{4.198319in}{2.505235in}}%
\pgfpathlineto{\pgfqpoint{4.206024in}{2.513989in}}%
\pgfpathclose%
\pgfusepath{fill}%
\end{pgfscope}%
\begin{pgfscope}%
\pgfpathrectangle{\pgfqpoint{1.150000in}{0.150000in}}{\pgfqpoint{5.700000in}{5.700000in}}%
\pgfusepath{clip}%
\pgfsetbuttcap%
\pgfsetroundjoin%
\definecolor{currentfill}{rgb}{0.277941,0.056324,0.381191}%
\pgfsetfillcolor{currentfill}%
\pgfsetfillopacity{0.700000}%
\pgfsetlinewidth{0.000000pt}%
\definecolor{currentstroke}{rgb}{0.000000,0.000000,0.000000}%
\pgfsetstrokecolor{currentstroke}%
\pgfsetdash{}{0pt}%
\pgfpathmoveto{\pgfqpoint{3.483365in}{2.497772in}}%
\pgfpathlineto{\pgfqpoint{3.496694in}{2.490404in}}%
\pgfpathlineto{\pgfqpoint{3.510026in}{2.483173in}}%
\pgfpathlineto{\pgfqpoint{3.523361in}{2.476077in}}%
\pgfpathlineto{\pgfqpoint{3.536698in}{2.469117in}}%
\pgfpathlineto{\pgfqpoint{3.528744in}{2.461672in}}%
\pgfpathlineto{\pgfqpoint{3.520784in}{2.454276in}}%
\pgfpathlineto{\pgfqpoint{3.512817in}{2.446932in}}%
\pgfpathlineto{\pgfqpoint{3.504844in}{2.439640in}}%
\pgfpathlineto{\pgfqpoint{3.491490in}{2.446757in}}%
\pgfpathlineto{\pgfqpoint{3.478139in}{2.454010in}}%
\pgfpathlineto{\pgfqpoint{3.464791in}{2.461400in}}%
\pgfpathlineto{\pgfqpoint{3.451445in}{2.468926in}}%
\pgfpathlineto{\pgfqpoint{3.459435in}{2.476053in}}%
\pgfpathlineto{\pgfqpoint{3.467418in}{2.483238in}}%
\pgfpathlineto{\pgfqpoint{3.475395in}{2.490478in}}%
\pgfpathlineto{\pgfqpoint{3.483365in}{2.497772in}}%
\pgfpathclose%
\pgfusepath{fill}%
\end{pgfscope}%
\begin{pgfscope}%
\pgfpathrectangle{\pgfqpoint{1.150000in}{0.150000in}}{\pgfqpoint{5.700000in}{5.700000in}}%
\pgfusepath{clip}%
\pgfsetbuttcap%
\pgfsetroundjoin%
\definecolor{currentfill}{rgb}{0.282884,0.135920,0.453427}%
\pgfsetfillcolor{currentfill}%
\pgfsetfillopacity{0.700000}%
\pgfsetlinewidth{0.000000pt}%
\definecolor{currentstroke}{rgb}{0.000000,0.000000,0.000000}%
\pgfsetstrokecolor{currentstroke}%
\pgfsetdash{}{0pt}%
\pgfpathmoveto{\pgfqpoint{4.598326in}{2.636984in}}%
\pgfpathlineto{\pgfqpoint{4.611904in}{2.637311in}}%
\pgfpathlineto{\pgfqpoint{4.625490in}{2.637744in}}%
\pgfpathlineto{\pgfqpoint{4.639086in}{2.638283in}}%
\pgfpathlineto{\pgfqpoint{4.652691in}{2.638927in}}%
\pgfpathlineto{\pgfqpoint{4.645133in}{2.630457in}}%
\pgfpathlineto{\pgfqpoint{4.637569in}{2.621957in}}%
\pgfpathlineto{\pgfqpoint{4.629999in}{2.613426in}}%
\pgfpathlineto{\pgfqpoint{4.622424in}{2.604863in}}%
\pgfpathlineto{\pgfqpoint{4.608809in}{2.604195in}}%
\pgfpathlineto{\pgfqpoint{4.595203in}{2.603631in}}%
\pgfpathlineto{\pgfqpoint{4.581607in}{2.603173in}}%
\pgfpathlineto{\pgfqpoint{4.568019in}{2.602822in}}%
\pgfpathlineto{\pgfqpoint{4.575604in}{2.611402in}}%
\pgfpathlineto{\pgfqpoint{4.583184in}{2.619956in}}%
\pgfpathlineto{\pgfqpoint{4.590758in}{2.628483in}}%
\pgfpathlineto{\pgfqpoint{4.598326in}{2.636984in}}%
\pgfpathclose%
\pgfusepath{fill}%
\end{pgfscope}%
\begin{pgfscope}%
\pgfpathrectangle{\pgfqpoint{1.150000in}{0.150000in}}{\pgfqpoint{5.700000in}{5.700000in}}%
\pgfusepath{clip}%
\pgfsetbuttcap%
\pgfsetroundjoin%
\definecolor{currentfill}{rgb}{0.282290,0.145912,0.461510}%
\pgfsetfillcolor{currentfill}%
\pgfsetfillopacity{0.700000}%
\pgfsetlinewidth{0.000000pt}%
\definecolor{currentstroke}{rgb}{0.000000,0.000000,0.000000}%
\pgfsetstrokecolor{currentstroke}%
\pgfsetdash{}{0pt}%
\pgfpathmoveto{\pgfqpoint{3.099072in}{2.669490in}}%
\pgfpathlineto{\pgfqpoint{3.112417in}{2.658199in}}%
\pgfpathlineto{\pgfqpoint{3.125762in}{2.647071in}}%
\pgfpathlineto{\pgfqpoint{3.139106in}{2.636105in}}%
\pgfpathlineto{\pgfqpoint{3.152450in}{2.625298in}}%
\pgfpathlineto{\pgfqpoint{3.144328in}{2.619422in}}%
\pgfpathlineto{\pgfqpoint{3.136198in}{2.613635in}}%
\pgfpathlineto{\pgfqpoint{3.128060in}{2.607938in}}%
\pgfpathlineto{\pgfqpoint{3.119914in}{2.602334in}}%
\pgfpathlineto{\pgfqpoint{3.106548in}{2.613338in}}%
\pgfpathlineto{\pgfqpoint{3.093182in}{2.624502in}}%
\pgfpathlineto{\pgfqpoint{3.079815in}{2.635829in}}%
\pgfpathlineto{\pgfqpoint{3.066448in}{2.647318in}}%
\pgfpathlineto{\pgfqpoint{3.074616in}{2.652717in}}%
\pgfpathlineto{\pgfqpoint{3.082776in}{2.658213in}}%
\pgfpathlineto{\pgfqpoint{3.090928in}{2.663805in}}%
\pgfpathlineto{\pgfqpoint{3.099072in}{2.669490in}}%
\pgfpathclose%
\pgfusepath{fill}%
\end{pgfscope}%
\begin{pgfscope}%
\pgfpathrectangle{\pgfqpoint{1.150000in}{0.150000in}}{\pgfqpoint{5.700000in}{5.700000in}}%
\pgfusepath{clip}%
\pgfsetbuttcap%
\pgfsetroundjoin%
\definecolor{currentfill}{rgb}{0.274952,0.037752,0.364543}%
\pgfsetfillcolor{currentfill}%
\pgfsetfillopacity{0.700000}%
\pgfsetlinewidth{0.000000pt}%
\definecolor{currentstroke}{rgb}{0.000000,0.000000,0.000000}%
\pgfsetstrokecolor{currentstroke}%
\pgfsetdash{}{0pt}%
\pgfpathmoveto{\pgfqpoint{3.898347in}{2.457224in}}%
\pgfpathlineto{\pgfqpoint{3.911732in}{2.453297in}}%
\pgfpathlineto{\pgfqpoint{3.925122in}{2.449489in}}%
\pgfpathlineto{\pgfqpoint{3.938518in}{2.445801in}}%
\pgfpathlineto{\pgfqpoint{3.951920in}{2.442232in}}%
\pgfpathlineto{\pgfqpoint{3.944118in}{2.433696in}}%
\pgfpathlineto{\pgfqpoint{3.936311in}{2.425172in}}%
\pgfpathlineto{\pgfqpoint{3.928498in}{2.416659in}}%
\pgfpathlineto{\pgfqpoint{3.920680in}{2.408159in}}%
\pgfpathlineto{\pgfqpoint{3.907266in}{2.411830in}}%
\pgfpathlineto{\pgfqpoint{3.893859in}{2.415620in}}%
\pgfpathlineto{\pgfqpoint{3.880456in}{2.419530in}}%
\pgfpathlineto{\pgfqpoint{3.867059in}{2.423560in}}%
\pgfpathlineto{\pgfqpoint{3.874890in}{2.431952in}}%
\pgfpathlineto{\pgfqpoint{3.882714in}{2.440360in}}%
\pgfpathlineto{\pgfqpoint{3.890534in}{2.448784in}}%
\pgfpathlineto{\pgfqpoint{3.898347in}{2.457224in}}%
\pgfpathclose%
\pgfusepath{fill}%
\end{pgfscope}%
\begin{pgfscope}%
\pgfpathrectangle{\pgfqpoint{1.150000in}{0.150000in}}{\pgfqpoint{5.700000in}{5.700000in}}%
\pgfusepath{clip}%
\pgfsetbuttcap%
\pgfsetroundjoin%
\definecolor{currentfill}{rgb}{0.227802,0.326594,0.546532}%
\pgfsetfillcolor{currentfill}%
\pgfsetfillopacity{0.700000}%
\pgfsetlinewidth{0.000000pt}%
\definecolor{currentstroke}{rgb}{0.000000,0.000000,0.000000}%
\pgfsetstrokecolor{currentstroke}%
\pgfsetdash{}{0pt}%
\pgfpathmoveto{\pgfqpoint{5.498465in}{3.025241in}}%
\pgfpathlineto{\pgfqpoint{5.512394in}{3.028770in}}%
\pgfpathlineto{\pgfqpoint{5.526335in}{3.032397in}}%
\pgfpathlineto{\pgfqpoint{5.540289in}{3.036122in}}%
\pgfpathlineto{\pgfqpoint{5.554255in}{3.039944in}}%
\pgfpathlineto{\pgfqpoint{5.547075in}{3.034005in}}%
\pgfpathlineto{\pgfqpoint{5.539889in}{3.028048in}}%
\pgfpathlineto{\pgfqpoint{5.532697in}{3.022072in}}%
\pgfpathlineto{\pgfqpoint{5.525498in}{3.016075in}}%
\pgfpathlineto{\pgfqpoint{5.511515in}{3.012042in}}%
\pgfpathlineto{\pgfqpoint{5.497544in}{3.008107in}}%
\pgfpathlineto{\pgfqpoint{5.483585in}{3.004270in}}%
\pgfpathlineto{\pgfqpoint{5.469640in}{3.000530in}}%
\pgfpathlineto{\pgfqpoint{5.476855in}{3.006732in}}%
\pgfpathlineto{\pgfqpoint{5.484064in}{3.012915in}}%
\pgfpathlineto{\pgfqpoint{5.491267in}{3.019084in}}%
\pgfpathlineto{\pgfqpoint{5.498465in}{3.025241in}}%
\pgfpathclose%
\pgfusepath{fill}%
\end{pgfscope}%
\begin{pgfscope}%
\pgfpathrectangle{\pgfqpoint{1.150000in}{0.150000in}}{\pgfqpoint{5.700000in}{5.700000in}}%
\pgfusepath{clip}%
\pgfsetbuttcap%
\pgfsetroundjoin%
\definecolor{currentfill}{rgb}{0.283229,0.120777,0.440584}%
\pgfsetfillcolor{currentfill}%
\pgfsetfillopacity{0.700000}%
\pgfsetlinewidth{0.000000pt}%
\definecolor{currentstroke}{rgb}{0.000000,0.000000,0.000000}%
\pgfsetstrokecolor{currentstroke}%
\pgfsetdash{}{0pt}%
\pgfpathmoveto{\pgfqpoint{4.513760in}{2.602480in}}%
\pgfpathlineto{\pgfqpoint{4.527311in}{2.602405in}}%
\pgfpathlineto{\pgfqpoint{4.540872in}{2.602437in}}%
\pgfpathlineto{\pgfqpoint{4.554441in}{2.602576in}}%
\pgfpathlineto{\pgfqpoint{4.568019in}{2.602822in}}%
\pgfpathlineto{\pgfqpoint{4.560429in}{2.594214in}}%
\pgfpathlineto{\pgfqpoint{4.552833in}{2.585578in}}%
\pgfpathlineto{\pgfqpoint{4.545232in}{2.576914in}}%
\pgfpathlineto{\pgfqpoint{4.537625in}{2.568221in}}%
\pgfpathlineto{\pgfqpoint{4.524037in}{2.567969in}}%
\pgfpathlineto{\pgfqpoint{4.510458in}{2.567823in}}%
\pgfpathlineto{\pgfqpoint{4.496888in}{2.567785in}}%
\pgfpathlineto{\pgfqpoint{4.483326in}{2.567853in}}%
\pgfpathlineto{\pgfqpoint{4.490943in}{2.576545in}}%
\pgfpathlineto{\pgfqpoint{4.498554in}{2.585214in}}%
\pgfpathlineto{\pgfqpoint{4.506160in}{2.593858in}}%
\pgfpathlineto{\pgfqpoint{4.513760in}{2.602480in}}%
\pgfpathclose%
\pgfusepath{fill}%
\end{pgfscope}%
\begin{pgfscope}%
\pgfpathrectangle{\pgfqpoint{1.150000in}{0.150000in}}{\pgfqpoint{5.700000in}{5.700000in}}%
\pgfusepath{clip}%
\pgfsetbuttcap%
\pgfsetroundjoin%
\definecolor{currentfill}{rgb}{0.277941,0.056324,0.381191}%
\pgfsetfillcolor{currentfill}%
\pgfsetfillopacity{0.700000}%
\pgfsetlinewidth{0.000000pt}%
\definecolor{currentstroke}{rgb}{0.000000,0.000000,0.000000}%
\pgfsetstrokecolor{currentstroke}%
\pgfsetdash{}{0pt}%
\pgfpathmoveto{\pgfqpoint{4.121389in}{2.487723in}}%
\pgfpathlineto{\pgfqpoint{4.134825in}{2.485353in}}%
\pgfpathlineto{\pgfqpoint{4.148268in}{2.483098in}}%
\pgfpathlineto{\pgfqpoint{4.161717in}{2.480956in}}%
\pgfpathlineto{\pgfqpoint{4.175174in}{2.478928in}}%
\pgfpathlineto{\pgfqpoint{4.167448in}{2.470146in}}%
\pgfpathlineto{\pgfqpoint{4.159717in}{2.461357in}}%
\pgfpathlineto{\pgfqpoint{4.151981in}{2.452562in}}%
\pgfpathlineto{\pgfqpoint{4.144239in}{2.443761in}}%
\pgfpathlineto{\pgfqpoint{4.130772in}{2.445855in}}%
\pgfpathlineto{\pgfqpoint{4.117312in}{2.448063in}}%
\pgfpathlineto{\pgfqpoint{4.103859in}{2.450384in}}%
\pgfpathlineto{\pgfqpoint{4.090412in}{2.452820in}}%
\pgfpathlineto{\pgfqpoint{4.098165in}{2.461548in}}%
\pgfpathlineto{\pgfqpoint{4.105911in}{2.470274in}}%
\pgfpathlineto{\pgfqpoint{4.113653in}{2.479000in}}%
\pgfpathlineto{\pgfqpoint{4.121389in}{2.487723in}}%
\pgfpathclose%
\pgfusepath{fill}%
\end{pgfscope}%
\begin{pgfscope}%
\pgfpathrectangle{\pgfqpoint{1.150000in}{0.150000in}}{\pgfqpoint{5.700000in}{5.700000in}}%
\pgfusepath{clip}%
\pgfsetbuttcap%
\pgfsetroundjoin%
\definecolor{currentfill}{rgb}{0.280894,0.078907,0.402329}%
\pgfsetfillcolor{currentfill}%
\pgfsetfillopacity{0.700000}%
\pgfsetlinewidth{0.000000pt}%
\definecolor{currentstroke}{rgb}{0.000000,0.000000,0.000000}%
\pgfsetstrokecolor{currentstroke}%
\pgfsetdash{}{0pt}%
\pgfpathmoveto{\pgfqpoint{3.344753in}{2.534172in}}%
\pgfpathlineto{\pgfqpoint{3.358083in}{2.525518in}}%
\pgfpathlineto{\pgfqpoint{3.371415in}{2.517008in}}%
\pgfpathlineto{\pgfqpoint{3.384749in}{2.508642in}}%
\pgfpathlineto{\pgfqpoint{3.398084in}{2.500417in}}%
\pgfpathlineto{\pgfqpoint{3.390070in}{2.493518in}}%
\pgfpathlineto{\pgfqpoint{3.382049in}{2.486682in}}%
\pgfpathlineto{\pgfqpoint{3.374021in}{2.479913in}}%
\pgfpathlineto{\pgfqpoint{3.365986in}{2.473211in}}%
\pgfpathlineto{\pgfqpoint{3.352632in}{2.481612in}}%
\pgfpathlineto{\pgfqpoint{3.339280in}{2.490155in}}%
\pgfpathlineto{\pgfqpoint{3.325930in}{2.498842in}}%
\pgfpathlineto{\pgfqpoint{3.312581in}{2.507673in}}%
\pgfpathlineto{\pgfqpoint{3.320635in}{2.514192in}}%
\pgfpathlineto{\pgfqpoint{3.328681in}{2.520782in}}%
\pgfpathlineto{\pgfqpoint{3.336721in}{2.527443in}}%
\pgfpathlineto{\pgfqpoint{3.344753in}{2.534172in}}%
\pgfpathclose%
\pgfusepath{fill}%
\end{pgfscope}%
\begin{pgfscope}%
\pgfpathrectangle{\pgfqpoint{1.150000in}{0.150000in}}{\pgfqpoint{5.700000in}{5.700000in}}%
\pgfusepath{clip}%
\pgfsetbuttcap%
\pgfsetroundjoin%
\definecolor{currentfill}{rgb}{0.235526,0.309527,0.542944}%
\pgfsetfillcolor{currentfill}%
\pgfsetfillopacity{0.700000}%
\pgfsetlinewidth{0.000000pt}%
\definecolor{currentstroke}{rgb}{0.000000,0.000000,0.000000}%
\pgfsetstrokecolor{currentstroke}%
\pgfsetdash{}{0pt}%
\pgfpathmoveto{\pgfqpoint{5.413982in}{2.986555in}}%
\pgfpathlineto{\pgfqpoint{5.427878in}{2.989902in}}%
\pgfpathlineto{\pgfqpoint{5.441786in}{2.993346in}}%
\pgfpathlineto{\pgfqpoint{5.455707in}{2.996889in}}%
\pgfpathlineto{\pgfqpoint{5.469640in}{3.000530in}}%
\pgfpathlineto{\pgfqpoint{5.462419in}{2.994310in}}%
\pgfpathlineto{\pgfqpoint{5.455192in}{2.988067in}}%
\pgfpathlineto{\pgfqpoint{5.447958in}{2.981800in}}%
\pgfpathlineto{\pgfqpoint{5.440719in}{2.975506in}}%
\pgfpathlineto{\pgfqpoint{5.426770in}{2.971673in}}%
\pgfpathlineto{\pgfqpoint{5.412833in}{2.967938in}}%
\pgfpathlineto{\pgfqpoint{5.398909in}{2.964302in}}%
\pgfpathlineto{\pgfqpoint{5.384997in}{2.960764in}}%
\pgfpathlineto{\pgfqpoint{5.392252in}{2.967243in}}%
\pgfpathlineto{\pgfqpoint{5.399501in}{2.973700in}}%
\pgfpathlineto{\pgfqpoint{5.406745in}{2.980136in}}%
\pgfpathlineto{\pgfqpoint{5.413982in}{2.986555in}}%
\pgfpathclose%
\pgfusepath{fill}%
\end{pgfscope}%
\begin{pgfscope}%
\pgfpathrectangle{\pgfqpoint{1.150000in}{0.150000in}}{\pgfqpoint{5.700000in}{5.700000in}}%
\pgfusepath{clip}%
\pgfsetbuttcap%
\pgfsetroundjoin%
\definecolor{currentfill}{rgb}{0.243113,0.292092,0.538516}%
\pgfsetfillcolor{currentfill}%
\pgfsetfillopacity{0.700000}%
\pgfsetlinewidth{0.000000pt}%
\definecolor{currentstroke}{rgb}{0.000000,0.000000,0.000000}%
\pgfsetstrokecolor{currentstroke}%
\pgfsetdash{}{0pt}%
\pgfpathmoveto{\pgfqpoint{5.329472in}{2.947600in}}%
\pgfpathlineto{\pgfqpoint{5.343335in}{2.950743in}}%
\pgfpathlineto{\pgfqpoint{5.357210in}{2.953985in}}%
\pgfpathlineto{\pgfqpoint{5.371097in}{2.957325in}}%
\pgfpathlineto{\pgfqpoint{5.384997in}{2.960764in}}%
\pgfpathlineto{\pgfqpoint{5.377736in}{2.954261in}}%
\pgfpathlineto{\pgfqpoint{5.370469in}{2.947732in}}%
\pgfpathlineto{\pgfqpoint{5.363195in}{2.941173in}}%
\pgfpathlineto{\pgfqpoint{5.355916in}{2.934584in}}%
\pgfpathlineto{\pgfqpoint{5.342001in}{2.930972in}}%
\pgfpathlineto{\pgfqpoint{5.328099in}{2.927458in}}%
\pgfpathlineto{\pgfqpoint{5.314208in}{2.924044in}}%
\pgfpathlineto{\pgfqpoint{5.300330in}{2.920728in}}%
\pgfpathlineto{\pgfqpoint{5.307625in}{2.927484in}}%
\pgfpathlineto{\pgfqpoint{5.314913in}{2.934213in}}%
\pgfpathlineto{\pgfqpoint{5.322195in}{2.940917in}}%
\pgfpathlineto{\pgfqpoint{5.329472in}{2.947600in}}%
\pgfpathclose%
\pgfusepath{fill}%
\end{pgfscope}%
\begin{pgfscope}%
\pgfpathrectangle{\pgfqpoint{1.150000in}{0.150000in}}{\pgfqpoint{5.700000in}{5.700000in}}%
\pgfusepath{clip}%
\pgfsetbuttcap%
\pgfsetroundjoin%
\definecolor{currentfill}{rgb}{0.283187,0.125848,0.444960}%
\pgfsetfillcolor{currentfill}%
\pgfsetfillopacity{0.700000}%
\pgfsetlinewidth{0.000000pt}%
\definecolor{currentstroke}{rgb}{0.000000,0.000000,0.000000}%
\pgfsetstrokecolor{currentstroke}%
\pgfsetdash{}{0pt}%
\pgfpathmoveto{\pgfqpoint{3.152450in}{2.625298in}}%
\pgfpathlineto{\pgfqpoint{3.165793in}{2.614650in}}%
\pgfpathlineto{\pgfqpoint{3.179136in}{2.604160in}}%
\pgfpathlineto{\pgfqpoint{3.192479in}{2.593827in}}%
\pgfpathlineto{\pgfqpoint{3.205821in}{2.583648in}}%
\pgfpathlineto{\pgfqpoint{3.197721in}{2.577582in}}%
\pgfpathlineto{\pgfqpoint{3.189612in}{2.571599in}}%
\pgfpathlineto{\pgfqpoint{3.181495in}{2.565703in}}%
\pgfpathlineto{\pgfqpoint{3.173370in}{2.559895in}}%
\pgfpathlineto{\pgfqpoint{3.160007in}{2.570270in}}%
\pgfpathlineto{\pgfqpoint{3.146643in}{2.580801in}}%
\pgfpathlineto{\pgfqpoint{3.133278in}{2.591488in}}%
\pgfpathlineto{\pgfqpoint{3.119914in}{2.602334in}}%
\pgfpathlineto{\pgfqpoint{3.128060in}{2.607938in}}%
\pgfpathlineto{\pgfqpoint{3.136198in}{2.613635in}}%
\pgfpathlineto{\pgfqpoint{3.144328in}{2.619422in}}%
\pgfpathlineto{\pgfqpoint{3.152450in}{2.625298in}}%
\pgfpathclose%
\pgfusepath{fill}%
\end{pgfscope}%
\begin{pgfscope}%
\pgfpathrectangle{\pgfqpoint{1.150000in}{0.150000in}}{\pgfqpoint{5.700000in}{5.700000in}}%
\pgfusepath{clip}%
\pgfsetbuttcap%
\pgfsetroundjoin%
\definecolor{currentfill}{rgb}{0.282910,0.105393,0.426902}%
\pgfsetfillcolor{currentfill}%
\pgfsetfillopacity{0.700000}%
\pgfsetlinewidth{0.000000pt}%
\definecolor{currentstroke}{rgb}{0.000000,0.000000,0.000000}%
\pgfsetstrokecolor{currentstroke}%
\pgfsetdash{}{0pt}%
\pgfpathmoveto{\pgfqpoint{4.429167in}{2.569203in}}%
\pgfpathlineto{\pgfqpoint{4.442694in}{2.568703in}}%
\pgfpathlineto{\pgfqpoint{4.456230in}{2.568312in}}%
\pgfpathlineto{\pgfqpoint{4.469774in}{2.568029in}}%
\pgfpathlineto{\pgfqpoint{4.483326in}{2.567853in}}%
\pgfpathlineto{\pgfqpoint{4.475705in}{2.559136in}}%
\pgfpathlineto{\pgfqpoint{4.468077in}{2.550394in}}%
\pgfpathlineto{\pgfqpoint{4.460445in}{2.541627in}}%
\pgfpathlineto{\pgfqpoint{4.452807in}{2.532835in}}%
\pgfpathlineto{\pgfqpoint{4.439244in}{2.533022in}}%
\pgfpathlineto{\pgfqpoint{4.425690in}{2.533317in}}%
\pgfpathlineto{\pgfqpoint{4.412145in}{2.533720in}}%
\pgfpathlineto{\pgfqpoint{4.398608in}{2.534231in}}%
\pgfpathlineto{\pgfqpoint{4.406256in}{2.543005in}}%
\pgfpathlineto{\pgfqpoint{4.413898in}{2.551758in}}%
\pgfpathlineto{\pgfqpoint{4.421535in}{2.560490in}}%
\pgfpathlineto{\pgfqpoint{4.429167in}{2.569203in}}%
\pgfpathclose%
\pgfusepath{fill}%
\end{pgfscope}%
\begin{pgfscope}%
\pgfpathrectangle{\pgfqpoint{1.150000in}{0.150000in}}{\pgfqpoint{5.700000in}{5.700000in}}%
\pgfusepath{clip}%
\pgfsetbuttcap%
\pgfsetroundjoin%
\definecolor{currentfill}{rgb}{0.250425,0.274290,0.533103}%
\pgfsetfillcolor{currentfill}%
\pgfsetfillopacity{0.700000}%
\pgfsetlinewidth{0.000000pt}%
\definecolor{currentstroke}{rgb}{0.000000,0.000000,0.000000}%
\pgfsetstrokecolor{currentstroke}%
\pgfsetdash{}{0pt}%
\pgfpathmoveto{\pgfqpoint{5.244937in}{2.908459in}}%
\pgfpathlineto{\pgfqpoint{5.258768in}{2.911377in}}%
\pgfpathlineto{\pgfqpoint{5.272610in}{2.914395in}}%
\pgfpathlineto{\pgfqpoint{5.286464in}{2.917512in}}%
\pgfpathlineto{\pgfqpoint{5.300330in}{2.920728in}}%
\pgfpathlineto{\pgfqpoint{5.293030in}{2.913945in}}%
\pgfpathlineto{\pgfqpoint{5.285723in}{2.907131in}}%
\pgfpathlineto{\pgfqpoint{5.278411in}{2.900285in}}%
\pgfpathlineto{\pgfqpoint{5.271092in}{2.893406in}}%
\pgfpathlineto{\pgfqpoint{5.257212in}{2.890035in}}%
\pgfpathlineto{\pgfqpoint{5.243343in}{2.886763in}}%
\pgfpathlineto{\pgfqpoint{5.229487in}{2.883591in}}%
\pgfpathlineto{\pgfqpoint{5.215643in}{2.880519in}}%
\pgfpathlineto{\pgfqpoint{5.222975in}{2.887546in}}%
\pgfpathlineto{\pgfqpoint{5.230302in}{2.894544in}}%
\pgfpathlineto{\pgfqpoint{5.237622in}{2.901514in}}%
\pgfpathlineto{\pgfqpoint{5.244937in}{2.908459in}}%
\pgfpathclose%
\pgfusepath{fill}%
\end{pgfscope}%
\begin{pgfscope}%
\pgfpathrectangle{\pgfqpoint{1.150000in}{0.150000in}}{\pgfqpoint{5.700000in}{5.700000in}}%
\pgfusepath{clip}%
\pgfsetbuttcap%
\pgfsetroundjoin%
\definecolor{currentfill}{rgb}{0.273809,0.031497,0.358853}%
\pgfsetfillcolor{currentfill}%
\pgfsetfillopacity{0.700000}%
\pgfsetlinewidth{0.000000pt}%
\definecolor{currentstroke}{rgb}{0.000000,0.000000,0.000000}%
\pgfsetstrokecolor{currentstroke}%
\pgfsetdash{}{0pt}%
\pgfpathmoveto{\pgfqpoint{3.675146in}{2.449578in}}%
\pgfpathlineto{\pgfqpoint{3.688499in}{2.443938in}}%
\pgfpathlineto{\pgfqpoint{3.701857in}{2.438427in}}%
\pgfpathlineto{\pgfqpoint{3.715219in}{2.433042in}}%
\pgfpathlineto{\pgfqpoint{3.728585in}{2.427783in}}%
\pgfpathlineto{\pgfqpoint{3.720700in}{2.419754in}}%
\pgfpathlineto{\pgfqpoint{3.712809in}{2.411756in}}%
\pgfpathlineto{\pgfqpoint{3.704913in}{2.403791in}}%
\pgfpathlineto{\pgfqpoint{3.697010in}{2.395860in}}%
\pgfpathlineto{\pgfqpoint{3.683630in}{2.401257in}}%
\pgfpathlineto{\pgfqpoint{3.670254in}{2.406780in}}%
\pgfpathlineto{\pgfqpoint{3.656882in}{2.412431in}}%
\pgfpathlineto{\pgfqpoint{3.643514in}{2.418209in}}%
\pgfpathlineto{\pgfqpoint{3.651431in}{2.425995in}}%
\pgfpathlineto{\pgfqpoint{3.659342in}{2.433819in}}%
\pgfpathlineto{\pgfqpoint{3.667247in}{2.441680in}}%
\pgfpathlineto{\pgfqpoint{3.675146in}{2.449578in}}%
\pgfpathclose%
\pgfusepath{fill}%
\end{pgfscope}%
\begin{pgfscope}%
\pgfpathrectangle{\pgfqpoint{1.150000in}{0.150000in}}{\pgfqpoint{5.700000in}{5.700000in}}%
\pgfusepath{clip}%
\pgfsetbuttcap%
\pgfsetroundjoin%
\definecolor{currentfill}{rgb}{0.276022,0.044167,0.370164}%
\pgfsetfillcolor{currentfill}%
\pgfsetfillopacity{0.700000}%
\pgfsetlinewidth{0.000000pt}%
\definecolor{currentstroke}{rgb}{0.000000,0.000000,0.000000}%
\pgfsetstrokecolor{currentstroke}%
\pgfsetdash{}{0pt}%
\pgfpathmoveto{\pgfqpoint{3.536698in}{2.469117in}}%
\pgfpathlineto{\pgfqpoint{3.550039in}{2.462291in}}%
\pgfpathlineto{\pgfqpoint{3.563382in}{2.455599in}}%
\pgfpathlineto{\pgfqpoint{3.576729in}{2.449040in}}%
\pgfpathlineto{\pgfqpoint{3.590079in}{2.442613in}}%
\pgfpathlineto{\pgfqpoint{3.582140in}{2.435017in}}%
\pgfpathlineto{\pgfqpoint{3.574196in}{2.427466in}}%
\pgfpathlineto{\pgfqpoint{3.566245in}{2.419962in}}%
\pgfpathlineto{\pgfqpoint{3.558288in}{2.412506in}}%
\pgfpathlineto{\pgfqpoint{3.544922in}{2.419091in}}%
\pgfpathlineto{\pgfqpoint{3.531560in}{2.425807in}}%
\pgfpathlineto{\pgfqpoint{3.518200in}{2.432657in}}%
\pgfpathlineto{\pgfqpoint{3.504844in}{2.439640in}}%
\pgfpathlineto{\pgfqpoint{3.512817in}{2.446932in}}%
\pgfpathlineto{\pgfqpoint{3.520784in}{2.454276in}}%
\pgfpathlineto{\pgfqpoint{3.528744in}{2.461672in}}%
\pgfpathlineto{\pgfqpoint{3.536698in}{2.469117in}}%
\pgfpathclose%
\pgfusepath{fill}%
\end{pgfscope}%
\begin{pgfscope}%
\pgfpathrectangle{\pgfqpoint{1.150000in}{0.150000in}}{\pgfqpoint{5.700000in}{5.700000in}}%
\pgfusepath{clip}%
\pgfsetbuttcap%
\pgfsetroundjoin%
\definecolor{currentfill}{rgb}{0.257322,0.256130,0.526563}%
\pgfsetfillcolor{currentfill}%
\pgfsetfillopacity{0.700000}%
\pgfsetlinewidth{0.000000pt}%
\definecolor{currentstroke}{rgb}{0.000000,0.000000,0.000000}%
\pgfsetstrokecolor{currentstroke}%
\pgfsetdash{}{0pt}%
\pgfpathmoveto{\pgfqpoint{5.160381in}{2.869228in}}%
\pgfpathlineto{\pgfqpoint{5.174179in}{2.871900in}}%
\pgfpathlineto{\pgfqpoint{5.187989in}{2.874673in}}%
\pgfpathlineto{\pgfqpoint{5.201810in}{2.877546in}}%
\pgfpathlineto{\pgfqpoint{5.215643in}{2.880519in}}%
\pgfpathlineto{\pgfqpoint{5.208304in}{2.873460in}}%
\pgfpathlineto{\pgfqpoint{5.200959in}{2.866368in}}%
\pgfpathlineto{\pgfqpoint{5.193609in}{2.859241in}}%
\pgfpathlineto{\pgfqpoint{5.186252in}{2.852077in}}%
\pgfpathlineto{\pgfqpoint{5.172406in}{2.848968in}}%
\pgfpathlineto{\pgfqpoint{5.158571in}{2.845960in}}%
\pgfpathlineto{\pgfqpoint{5.144748in}{2.843051in}}%
\pgfpathlineto{\pgfqpoint{5.130937in}{2.840243in}}%
\pgfpathlineto{\pgfqpoint{5.138307in}{2.847536in}}%
\pgfpathlineto{\pgfqpoint{5.145671in}{2.854796in}}%
\pgfpathlineto{\pgfqpoint{5.153029in}{2.862026in}}%
\pgfpathlineto{\pgfqpoint{5.160381in}{2.869228in}}%
\pgfpathclose%
\pgfusepath{fill}%
\end{pgfscope}%
\begin{pgfscope}%
\pgfpathrectangle{\pgfqpoint{1.150000in}{0.150000in}}{\pgfqpoint{5.700000in}{5.700000in}}%
\pgfusepath{clip}%
\pgfsetbuttcap%
\pgfsetroundjoin%
\definecolor{currentfill}{rgb}{0.276022,0.044167,0.370164}%
\pgfsetfillcolor{currentfill}%
\pgfsetfillopacity{0.700000}%
\pgfsetlinewidth{0.000000pt}%
\definecolor{currentstroke}{rgb}{0.000000,0.000000,0.000000}%
\pgfsetstrokecolor{currentstroke}%
\pgfsetdash{}{0pt}%
\pgfpathmoveto{\pgfqpoint{4.036692in}{2.463714in}}%
\pgfpathlineto{\pgfqpoint{4.050112in}{2.460817in}}%
\pgfpathlineto{\pgfqpoint{4.063539in}{2.458036in}}%
\pgfpathlineto{\pgfqpoint{4.076972in}{2.455370in}}%
\pgfpathlineto{\pgfqpoint{4.090412in}{2.452820in}}%
\pgfpathlineto{\pgfqpoint{4.082655in}{2.444091in}}%
\pgfpathlineto{\pgfqpoint{4.074892in}{2.435362in}}%
\pgfpathlineto{\pgfqpoint{4.067123in}{2.426634in}}%
\pgfpathlineto{\pgfqpoint{4.059350in}{2.417906in}}%
\pgfpathlineto{\pgfqpoint{4.045899in}{2.420541in}}%
\pgfpathlineto{\pgfqpoint{4.032455in}{2.423290in}}%
\pgfpathlineto{\pgfqpoint{4.019017in}{2.426155in}}%
\pgfpathlineto{\pgfqpoint{4.005585in}{2.429136in}}%
\pgfpathlineto{\pgfqpoint{4.013370in}{2.437773in}}%
\pgfpathlineto{\pgfqpoint{4.021149in}{2.446416in}}%
\pgfpathlineto{\pgfqpoint{4.028923in}{2.455063in}}%
\pgfpathlineto{\pgfqpoint{4.036692in}{2.463714in}}%
\pgfpathclose%
\pgfusepath{fill}%
\end{pgfscope}%
\begin{pgfscope}%
\pgfpathrectangle{\pgfqpoint{1.150000in}{0.150000in}}{\pgfqpoint{5.700000in}{5.700000in}}%
\pgfusepath{clip}%
\pgfsetbuttcap%
\pgfsetroundjoin%
\definecolor{currentfill}{rgb}{0.281924,0.089666,0.412415}%
\pgfsetfillcolor{currentfill}%
\pgfsetfillopacity{0.700000}%
\pgfsetlinewidth{0.000000pt}%
\definecolor{currentstroke}{rgb}{0.000000,0.000000,0.000000}%
\pgfsetstrokecolor{currentstroke}%
\pgfsetdash{}{0pt}%
\pgfpathmoveto{\pgfqpoint{4.344543in}{2.537366in}}%
\pgfpathlineto{\pgfqpoint{4.358047in}{2.536418in}}%
\pgfpathlineto{\pgfqpoint{4.371559in}{2.535580in}}%
\pgfpathlineto{\pgfqpoint{4.385080in}{2.534851in}}%
\pgfpathlineto{\pgfqpoint{4.398608in}{2.534231in}}%
\pgfpathlineto{\pgfqpoint{4.390955in}{2.525437in}}%
\pgfpathlineto{\pgfqpoint{4.383297in}{2.516622in}}%
\pgfpathlineto{\pgfqpoint{4.375633in}{2.507786in}}%
\pgfpathlineto{\pgfqpoint{4.367964in}{2.498929in}}%
\pgfpathlineto{\pgfqpoint{4.354426in}{2.499578in}}%
\pgfpathlineto{\pgfqpoint{4.340895in}{2.500337in}}%
\pgfpathlineto{\pgfqpoint{4.327373in}{2.501204in}}%
\pgfpathlineto{\pgfqpoint{4.313859in}{2.502182in}}%
\pgfpathlineto{\pgfqpoint{4.321538in}{2.511002in}}%
\pgfpathlineto{\pgfqpoint{4.329212in}{2.519806in}}%
\pgfpathlineto{\pgfqpoint{4.336880in}{2.528594in}}%
\pgfpathlineto{\pgfqpoint{4.344543in}{2.537366in}}%
\pgfpathclose%
\pgfusepath{fill}%
\end{pgfscope}%
\begin{pgfscope}%
\pgfpathrectangle{\pgfqpoint{1.150000in}{0.150000in}}{\pgfqpoint{5.700000in}{5.700000in}}%
\pgfusepath{clip}%
\pgfsetbuttcap%
\pgfsetroundjoin%
\definecolor{currentfill}{rgb}{0.273809,0.031497,0.358853}%
\pgfsetfillcolor{currentfill}%
\pgfsetfillopacity{0.700000}%
\pgfsetlinewidth{0.000000pt}%
\definecolor{currentstroke}{rgb}{0.000000,0.000000,0.000000}%
\pgfsetstrokecolor{currentstroke}%
\pgfsetdash{}{0pt}%
\pgfpathmoveto{\pgfqpoint{3.813524in}{2.440894in}}%
\pgfpathlineto{\pgfqpoint{3.826900in}{2.436378in}}%
\pgfpathlineto{\pgfqpoint{3.840281in}{2.431983in}}%
\pgfpathlineto{\pgfqpoint{3.853668in}{2.427711in}}%
\pgfpathlineto{\pgfqpoint{3.867059in}{2.423560in}}%
\pgfpathlineto{\pgfqpoint{3.859223in}{2.415187in}}%
\pgfpathlineto{\pgfqpoint{3.851382in}{2.406833in}}%
\pgfpathlineto{\pgfqpoint{3.843535in}{2.398498in}}%
\pgfpathlineto{\pgfqpoint{3.835682in}{2.390184in}}%
\pgfpathlineto{\pgfqpoint{3.822278in}{2.394455in}}%
\pgfpathlineto{\pgfqpoint{3.808879in}{2.398847in}}%
\pgfpathlineto{\pgfqpoint{3.795484in}{2.403361in}}%
\pgfpathlineto{\pgfqpoint{3.782095in}{2.407998in}}%
\pgfpathlineto{\pgfqpoint{3.789961in}{2.416186in}}%
\pgfpathlineto{\pgfqpoint{3.797821in}{2.424398in}}%
\pgfpathlineto{\pgfqpoint{3.805675in}{2.432634in}}%
\pgfpathlineto{\pgfqpoint{3.813524in}{2.440894in}}%
\pgfpathclose%
\pgfusepath{fill}%
\end{pgfscope}%
\begin{pgfscope}%
\pgfpathrectangle{\pgfqpoint{1.150000in}{0.150000in}}{\pgfqpoint{5.700000in}{5.700000in}}%
\pgfusepath{clip}%
\pgfsetbuttcap%
\pgfsetroundjoin%
\definecolor{currentfill}{rgb}{0.262138,0.242286,0.520837}%
\pgfsetfillcolor{currentfill}%
\pgfsetfillopacity{0.700000}%
\pgfsetlinewidth{0.000000pt}%
\definecolor{currentstroke}{rgb}{0.000000,0.000000,0.000000}%
\pgfsetstrokecolor{currentstroke}%
\pgfsetdash{}{0pt}%
\pgfpathmoveto{\pgfqpoint{5.075806in}{2.830015in}}%
\pgfpathlineto{\pgfqpoint{5.089572in}{2.832421in}}%
\pgfpathlineto{\pgfqpoint{5.103349in}{2.834928in}}%
\pgfpathlineto{\pgfqpoint{5.117137in}{2.837535in}}%
\pgfpathlineto{\pgfqpoint{5.130937in}{2.840243in}}%
\pgfpathlineto{\pgfqpoint{5.123561in}{2.832917in}}%
\pgfpathlineto{\pgfqpoint{5.116179in}{2.825555in}}%
\pgfpathlineto{\pgfqpoint{5.108791in}{2.818156in}}%
\pgfpathlineto{\pgfqpoint{5.101397in}{2.810719in}}%
\pgfpathlineto{\pgfqpoint{5.087585in}{2.807894in}}%
\pgfpathlineto{\pgfqpoint{5.073784in}{2.805169in}}%
\pgfpathlineto{\pgfqpoint{5.059995in}{2.802546in}}%
\pgfpathlineto{\pgfqpoint{5.046216in}{2.800023in}}%
\pgfpathlineto{\pgfqpoint{5.053623in}{2.807570in}}%
\pgfpathlineto{\pgfqpoint{5.061023in}{2.815084in}}%
\pgfpathlineto{\pgfqpoint{5.068417in}{2.822565in}}%
\pgfpathlineto{\pgfqpoint{5.075806in}{2.830015in}}%
\pgfpathclose%
\pgfusepath{fill}%
\end{pgfscope}%
\begin{pgfscope}%
\pgfpathrectangle{\pgfqpoint{1.150000in}{0.150000in}}{\pgfqpoint{5.700000in}{5.700000in}}%
\pgfusepath{clip}%
\pgfsetbuttcap%
\pgfsetroundjoin%
\definecolor{currentfill}{rgb}{0.253935,0.265254,0.529983}%
\pgfsetfillcolor{currentfill}%
\pgfsetfillopacity{0.700000}%
\pgfsetlinewidth{0.000000pt}%
\definecolor{currentstroke}{rgb}{0.000000,0.000000,0.000000}%
\pgfsetstrokecolor{currentstroke}%
\pgfsetdash{}{0pt}%
\pgfpathmoveto{\pgfqpoint{2.798633in}{2.913503in}}%
\pgfpathlineto{\pgfqpoint{2.812055in}{2.898448in}}%
\pgfpathlineto{\pgfqpoint{2.825472in}{2.883587in}}%
\pgfpathlineto{\pgfqpoint{2.838886in}{2.868918in}}%
\pgfpathlineto{\pgfqpoint{2.852295in}{2.854439in}}%
\pgfpathlineto{\pgfqpoint{2.844019in}{2.849988in}}%
\pgfpathlineto{\pgfqpoint{2.835733in}{2.845656in}}%
\pgfpathlineto{\pgfqpoint{2.827438in}{2.841446in}}%
\pgfpathlineto{\pgfqpoint{2.819132in}{2.837359in}}%
\pgfpathlineto{\pgfqpoint{2.805695in}{2.852060in}}%
\pgfpathlineto{\pgfqpoint{2.792255in}{2.866952in}}%
\pgfpathlineto{\pgfqpoint{2.778810in}{2.882036in}}%
\pgfpathlineto{\pgfqpoint{2.765360in}{2.897314in}}%
\pgfpathlineto{\pgfqpoint{2.773694in}{2.901171in}}%
\pgfpathlineto{\pgfqpoint{2.782017in}{2.905156in}}%
\pgfpathlineto{\pgfqpoint{2.790330in}{2.909268in}}%
\pgfpathlineto{\pgfqpoint{2.798633in}{2.913503in}}%
\pgfpathclose%
\pgfusepath{fill}%
\end{pgfscope}%
\begin{pgfscope}%
\pgfpathrectangle{\pgfqpoint{1.150000in}{0.150000in}}{\pgfqpoint{5.700000in}{5.700000in}}%
\pgfusepath{clip}%
\pgfsetbuttcap%
\pgfsetroundjoin%
\definecolor{currentfill}{rgb}{0.279566,0.067836,0.391917}%
\pgfsetfillcolor{currentfill}%
\pgfsetfillopacity{0.700000}%
\pgfsetlinewidth{0.000000pt}%
\definecolor{currentstroke}{rgb}{0.000000,0.000000,0.000000}%
\pgfsetstrokecolor{currentstroke}%
\pgfsetdash{}{0pt}%
\pgfpathmoveto{\pgfqpoint{3.398084in}{2.500417in}}%
\pgfpathlineto{\pgfqpoint{3.411421in}{2.492334in}}%
\pgfpathlineto{\pgfqpoint{3.424760in}{2.484392in}}%
\pgfpathlineto{\pgfqpoint{3.438102in}{2.476590in}}%
\pgfpathlineto{\pgfqpoint{3.451445in}{2.468926in}}%
\pgfpathlineto{\pgfqpoint{3.443449in}{2.461857in}}%
\pgfpathlineto{\pgfqpoint{3.435445in}{2.454847in}}%
\pgfpathlineto{\pgfqpoint{3.427435in}{2.447899in}}%
\pgfpathlineto{\pgfqpoint{3.419417in}{2.441014in}}%
\pgfpathlineto{\pgfqpoint{3.406056in}{2.448854in}}%
\pgfpathlineto{\pgfqpoint{3.392698in}{2.456833in}}%
\pgfpathlineto{\pgfqpoint{3.379341in}{2.464952in}}%
\pgfpathlineto{\pgfqpoint{3.365986in}{2.473211in}}%
\pgfpathlineto{\pgfqpoint{3.374021in}{2.479913in}}%
\pgfpathlineto{\pgfqpoint{3.382049in}{2.486682in}}%
\pgfpathlineto{\pgfqpoint{3.390070in}{2.493518in}}%
\pgfpathlineto{\pgfqpoint{3.398084in}{2.500417in}}%
\pgfpathclose%
\pgfusepath{fill}%
\end{pgfscope}%
\begin{pgfscope}%
\pgfpathrectangle{\pgfqpoint{1.150000in}{0.150000in}}{\pgfqpoint{5.700000in}{5.700000in}}%
\pgfusepath{clip}%
\pgfsetbuttcap%
\pgfsetroundjoin%
\definecolor{currentfill}{rgb}{0.267968,0.223549,0.512008}%
\pgfsetfillcolor{currentfill}%
\pgfsetfillopacity{0.700000}%
\pgfsetlinewidth{0.000000pt}%
\definecolor{currentstroke}{rgb}{0.000000,0.000000,0.000000}%
\pgfsetstrokecolor{currentstroke}%
\pgfsetdash{}{0pt}%
\pgfpathmoveto{\pgfqpoint{4.991214in}{2.790942in}}%
\pgfpathlineto{\pgfqpoint{5.004948in}{2.793061in}}%
\pgfpathlineto{\pgfqpoint{5.018693in}{2.795280in}}%
\pgfpathlineto{\pgfqpoint{5.032449in}{2.797601in}}%
\pgfpathlineto{\pgfqpoint{5.046216in}{2.800023in}}%
\pgfpathlineto{\pgfqpoint{5.038804in}{2.792440in}}%
\pgfpathlineto{\pgfqpoint{5.031386in}{2.784820in}}%
\pgfpathlineto{\pgfqpoint{5.023962in}{2.777162in}}%
\pgfpathlineto{\pgfqpoint{5.016532in}{2.769465in}}%
\pgfpathlineto{\pgfqpoint{5.002753in}{2.766945in}}%
\pgfpathlineto{\pgfqpoint{4.988985in}{2.764525in}}%
\pgfpathlineto{\pgfqpoint{4.975228in}{2.762207in}}%
\pgfpathlineto{\pgfqpoint{4.961482in}{2.759991in}}%
\pgfpathlineto{\pgfqpoint{4.968924in}{2.767780in}}%
\pgfpathlineto{\pgfqpoint{4.976360in}{2.775534in}}%
\pgfpathlineto{\pgfqpoint{4.983790in}{2.783254in}}%
\pgfpathlineto{\pgfqpoint{4.991214in}{2.790942in}}%
\pgfpathclose%
\pgfusepath{fill}%
\end{pgfscope}%
\begin{pgfscope}%
\pgfpathrectangle{\pgfqpoint{1.150000in}{0.150000in}}{\pgfqpoint{5.700000in}{5.700000in}}%
\pgfusepath{clip}%
\pgfsetbuttcap%
\pgfsetroundjoin%
\definecolor{currentfill}{rgb}{0.283091,0.110553,0.431554}%
\pgfsetfillcolor{currentfill}%
\pgfsetfillopacity{0.700000}%
\pgfsetlinewidth{0.000000pt}%
\definecolor{currentstroke}{rgb}{0.000000,0.000000,0.000000}%
\pgfsetstrokecolor{currentstroke}%
\pgfsetdash{}{0pt}%
\pgfpathmoveto{\pgfqpoint{3.205821in}{2.583648in}}%
\pgfpathlineto{\pgfqpoint{3.219164in}{2.573624in}}%
\pgfpathlineto{\pgfqpoint{3.232508in}{2.563752in}}%
\pgfpathlineto{\pgfqpoint{3.245851in}{2.554033in}}%
\pgfpathlineto{\pgfqpoint{3.259196in}{2.544464in}}%
\pgfpathlineto{\pgfqpoint{3.251115in}{2.538208in}}%
\pgfpathlineto{\pgfqpoint{3.243027in}{2.532031in}}%
\pgfpathlineto{\pgfqpoint{3.234931in}{2.525936in}}%
\pgfpathlineto{\pgfqpoint{3.226827in}{2.519925in}}%
\pgfpathlineto{\pgfqpoint{3.213462in}{2.529690in}}%
\pgfpathlineto{\pgfqpoint{3.200098in}{2.539606in}}%
\pgfpathlineto{\pgfqpoint{3.186734in}{2.549674in}}%
\pgfpathlineto{\pgfqpoint{3.173370in}{2.559895in}}%
\pgfpathlineto{\pgfqpoint{3.181495in}{2.565703in}}%
\pgfpathlineto{\pgfqpoint{3.189612in}{2.571599in}}%
\pgfpathlineto{\pgfqpoint{3.197721in}{2.577582in}}%
\pgfpathlineto{\pgfqpoint{3.205821in}{2.583648in}}%
\pgfpathclose%
\pgfusepath{fill}%
\end{pgfscope}%
\begin{pgfscope}%
\pgfpathrectangle{\pgfqpoint{1.150000in}{0.150000in}}{\pgfqpoint{5.700000in}{5.700000in}}%
\pgfusepath{clip}%
\pgfsetbuttcap%
\pgfsetroundjoin%
\definecolor{currentfill}{rgb}{0.262138,0.242286,0.520837}%
\pgfsetfillcolor{currentfill}%
\pgfsetfillopacity{0.700000}%
\pgfsetlinewidth{0.000000pt}%
\definecolor{currentstroke}{rgb}{0.000000,0.000000,0.000000}%
\pgfsetstrokecolor{currentstroke}%
\pgfsetdash{}{0pt}%
\pgfpathmoveto{\pgfqpoint{2.852295in}{2.854439in}}%
\pgfpathlineto{\pgfqpoint{2.865701in}{2.840149in}}%
\pgfpathlineto{\pgfqpoint{2.879103in}{2.826045in}}%
\pgfpathlineto{\pgfqpoint{2.892502in}{2.812127in}}%
\pgfpathlineto{\pgfqpoint{2.905897in}{2.798393in}}%
\pgfpathlineto{\pgfqpoint{2.897647in}{2.793727in}}%
\pgfpathlineto{\pgfqpoint{2.889387in}{2.789177in}}%
\pgfpathlineto{\pgfqpoint{2.881118in}{2.784743in}}%
\pgfpathlineto{\pgfqpoint{2.872839in}{2.780428in}}%
\pgfpathlineto{\pgfqpoint{2.859418in}{2.794383in}}%
\pgfpathlineto{\pgfqpoint{2.845993in}{2.808522in}}%
\pgfpathlineto{\pgfqpoint{2.832564in}{2.822847in}}%
\pgfpathlineto{\pgfqpoint{2.819132in}{2.837359in}}%
\pgfpathlineto{\pgfqpoint{2.827438in}{2.841446in}}%
\pgfpathlineto{\pgfqpoint{2.835733in}{2.845656in}}%
\pgfpathlineto{\pgfqpoint{2.844019in}{2.849988in}}%
\pgfpathlineto{\pgfqpoint{2.852295in}{2.854439in}}%
\pgfpathclose%
\pgfusepath{fill}%
\end{pgfscope}%
\begin{pgfscope}%
\pgfpathrectangle{\pgfqpoint{1.150000in}{0.150000in}}{\pgfqpoint{5.700000in}{5.700000in}}%
\pgfusepath{clip}%
\pgfsetbuttcap%
\pgfsetroundjoin%
\definecolor{currentfill}{rgb}{0.273006,0.204520,0.501721}%
\pgfsetfillcolor{currentfill}%
\pgfsetfillopacity{0.700000}%
\pgfsetlinewidth{0.000000pt}%
\definecolor{currentstroke}{rgb}{0.000000,0.000000,0.000000}%
\pgfsetstrokecolor{currentstroke}%
\pgfsetdash{}{0pt}%
\pgfpathmoveto{\pgfqpoint{4.906606in}{2.752144in}}%
\pgfpathlineto{\pgfqpoint{4.920309in}{2.753953in}}%
\pgfpathlineto{\pgfqpoint{4.934023in}{2.755863in}}%
\pgfpathlineto{\pgfqpoint{4.947747in}{2.757876in}}%
\pgfpathlineto{\pgfqpoint{4.961482in}{2.759991in}}%
\pgfpathlineto{\pgfqpoint{4.954035in}{2.752166in}}%
\pgfpathlineto{\pgfqpoint{4.946581in}{2.744303in}}%
\pgfpathlineto{\pgfqpoint{4.939122in}{2.736402in}}%
\pgfpathlineto{\pgfqpoint{4.931657in}{2.728461in}}%
\pgfpathlineto{\pgfqpoint{4.917910in}{2.726266in}}%
\pgfpathlineto{\pgfqpoint{4.904175in}{2.724173in}}%
\pgfpathlineto{\pgfqpoint{4.890450in}{2.722182in}}%
\pgfpathlineto{\pgfqpoint{4.876736in}{2.720294in}}%
\pgfpathlineto{\pgfqpoint{4.884212in}{2.728308in}}%
\pgfpathlineto{\pgfqpoint{4.891682in}{2.736287in}}%
\pgfpathlineto{\pgfqpoint{4.899147in}{2.744232in}}%
\pgfpathlineto{\pgfqpoint{4.906606in}{2.752144in}}%
\pgfpathclose%
\pgfusepath{fill}%
\end{pgfscope}%
\begin{pgfscope}%
\pgfpathrectangle{\pgfqpoint{1.150000in}{0.150000in}}{\pgfqpoint{5.700000in}{5.700000in}}%
\pgfusepath{clip}%
\pgfsetbuttcap%
\pgfsetroundjoin%
\definecolor{currentfill}{rgb}{0.280894,0.078907,0.402329}%
\pgfsetfillcolor{currentfill}%
\pgfsetfillopacity{0.700000}%
\pgfsetlinewidth{0.000000pt}%
\definecolor{currentstroke}{rgb}{0.000000,0.000000,0.000000}%
\pgfsetstrokecolor{currentstroke}%
\pgfsetdash{}{0pt}%
\pgfpathmoveto{\pgfqpoint{4.259881in}{2.507194in}}%
\pgfpathlineto{\pgfqpoint{4.273364in}{2.505775in}}%
\pgfpathlineto{\pgfqpoint{4.286855in}{2.504467in}}%
\pgfpathlineto{\pgfqpoint{4.300353in}{2.503269in}}%
\pgfpathlineto{\pgfqpoint{4.313859in}{2.502182in}}%
\pgfpathlineto{\pgfqpoint{4.306175in}{2.493345in}}%
\pgfpathlineto{\pgfqpoint{4.298486in}{2.484492in}}%
\pgfpathlineto{\pgfqpoint{4.290791in}{2.475622in}}%
\pgfpathlineto{\pgfqpoint{4.283091in}{2.466737in}}%
\pgfpathlineto{\pgfqpoint{4.269575in}{2.467873in}}%
\pgfpathlineto{\pgfqpoint{4.256066in}{2.469118in}}%
\pgfpathlineto{\pgfqpoint{4.242566in}{2.470474in}}%
\pgfpathlineto{\pgfqpoint{4.229072in}{2.471941in}}%
\pgfpathlineto{\pgfqpoint{4.236783in}{2.480772in}}%
\pgfpathlineto{\pgfqpoint{4.244487in}{2.489591in}}%
\pgfpathlineto{\pgfqpoint{4.252187in}{2.498399in}}%
\pgfpathlineto{\pgfqpoint{4.259881in}{2.507194in}}%
\pgfpathclose%
\pgfusepath{fill}%
\end{pgfscope}%
\begin{pgfscope}%
\pgfpathrectangle{\pgfqpoint{1.150000in}{0.150000in}}{\pgfqpoint{5.700000in}{5.700000in}}%
\pgfusepath{clip}%
\pgfsetbuttcap%
\pgfsetroundjoin%
\definecolor{currentfill}{rgb}{0.270595,0.214069,0.507052}%
\pgfsetfillcolor{currentfill}%
\pgfsetfillopacity{0.700000}%
\pgfsetlinewidth{0.000000pt}%
\definecolor{currentstroke}{rgb}{0.000000,0.000000,0.000000}%
\pgfsetstrokecolor{currentstroke}%
\pgfsetdash{}{0pt}%
\pgfpathmoveto{\pgfqpoint{2.905897in}{2.798393in}}%
\pgfpathlineto{\pgfqpoint{2.919289in}{2.784840in}}%
\pgfpathlineto{\pgfqpoint{2.932679in}{2.771467in}}%
\pgfpathlineto{\pgfqpoint{2.946065in}{2.758273in}}%
\pgfpathlineto{\pgfqpoint{2.959449in}{2.745256in}}%
\pgfpathlineto{\pgfqpoint{2.951224in}{2.740378in}}%
\pgfpathlineto{\pgfqpoint{2.942990in}{2.735610in}}%
\pgfpathlineto{\pgfqpoint{2.934746in}{2.730954in}}%
\pgfpathlineto{\pgfqpoint{2.926493in}{2.726411in}}%
\pgfpathlineto{\pgfqpoint{2.913084in}{2.739648in}}%
\pgfpathlineto{\pgfqpoint{2.899672in}{2.753062in}}%
\pgfpathlineto{\pgfqpoint{2.886257in}{2.766654in}}%
\pgfpathlineto{\pgfqpoint{2.872839in}{2.780428in}}%
\pgfpathlineto{\pgfqpoint{2.881118in}{2.784743in}}%
\pgfpathlineto{\pgfqpoint{2.889387in}{2.789177in}}%
\pgfpathlineto{\pgfqpoint{2.897647in}{2.793727in}}%
\pgfpathlineto{\pgfqpoint{2.905897in}{2.798393in}}%
\pgfpathclose%
\pgfusepath{fill}%
\end{pgfscope}%
\begin{pgfscope}%
\pgfpathrectangle{\pgfqpoint{1.150000in}{0.150000in}}{\pgfqpoint{5.700000in}{5.700000in}}%
\pgfusepath{clip}%
\pgfsetbuttcap%
\pgfsetroundjoin%
\definecolor{currentfill}{rgb}{0.277134,0.185228,0.489898}%
\pgfsetfillcolor{currentfill}%
\pgfsetfillopacity{0.700000}%
\pgfsetlinewidth{0.000000pt}%
\definecolor{currentstroke}{rgb}{0.000000,0.000000,0.000000}%
\pgfsetstrokecolor{currentstroke}%
\pgfsetdash{}{0pt}%
\pgfpathmoveto{\pgfqpoint{4.821983in}{2.713767in}}%
\pgfpathlineto{\pgfqpoint{4.835656in}{2.715244in}}%
\pgfpathlineto{\pgfqpoint{4.849339in}{2.716825in}}%
\pgfpathlineto{\pgfqpoint{4.863032in}{2.718508in}}%
\pgfpathlineto{\pgfqpoint{4.876736in}{2.720294in}}%
\pgfpathlineto{\pgfqpoint{4.869254in}{2.712243in}}%
\pgfpathlineto{\pgfqpoint{4.861766in}{2.704155in}}%
\pgfpathlineto{\pgfqpoint{4.854273in}{2.696029in}}%
\pgfpathlineto{\pgfqpoint{4.846773in}{2.687864in}}%
\pgfpathlineto{\pgfqpoint{4.833059in}{2.686016in}}%
\pgfpathlineto{\pgfqpoint{4.819355in}{2.684271in}}%
\pgfpathlineto{\pgfqpoint{4.805661in}{2.682629in}}%
\pgfpathlineto{\pgfqpoint{4.791978in}{2.681090in}}%
\pgfpathlineto{\pgfqpoint{4.799488in}{2.689311in}}%
\pgfpathlineto{\pgfqpoint{4.806992in}{2.697496in}}%
\pgfpathlineto{\pgfqpoint{4.814490in}{2.705648in}}%
\pgfpathlineto{\pgfqpoint{4.821983in}{2.713767in}}%
\pgfpathclose%
\pgfusepath{fill}%
\end{pgfscope}%
\begin{pgfscope}%
\pgfpathrectangle{\pgfqpoint{1.150000in}{0.150000in}}{\pgfqpoint{5.700000in}{5.700000in}}%
\pgfusepath{clip}%
\pgfsetbuttcap%
\pgfsetroundjoin%
\definecolor{currentfill}{rgb}{0.274952,0.037752,0.364543}%
\pgfsetfillcolor{currentfill}%
\pgfsetfillopacity{0.700000}%
\pgfsetlinewidth{0.000000pt}%
\definecolor{currentstroke}{rgb}{0.000000,0.000000,0.000000}%
\pgfsetstrokecolor{currentstroke}%
\pgfsetdash{}{0pt}%
\pgfpathmoveto{\pgfqpoint{3.951920in}{2.442232in}}%
\pgfpathlineto{\pgfqpoint{3.965327in}{2.438781in}}%
\pgfpathlineto{\pgfqpoint{3.978740in}{2.435449in}}%
\pgfpathlineto{\pgfqpoint{3.992160in}{2.432234in}}%
\pgfpathlineto{\pgfqpoint{4.005585in}{2.429136in}}%
\pgfpathlineto{\pgfqpoint{3.997795in}{2.420505in}}%
\pgfpathlineto{\pgfqpoint{3.989999in}{2.411881in}}%
\pgfpathlineto{\pgfqpoint{3.982198in}{2.403264in}}%
\pgfpathlineto{\pgfqpoint{3.974392in}{2.394655in}}%
\pgfpathlineto{\pgfqpoint{3.960955in}{2.397855in}}%
\pgfpathlineto{\pgfqpoint{3.947524in}{2.401172in}}%
\pgfpathlineto{\pgfqpoint{3.934099in}{2.404606in}}%
\pgfpathlineto{\pgfqpoint{3.920680in}{2.408159in}}%
\pgfpathlineto{\pgfqpoint{3.928498in}{2.416659in}}%
\pgfpathlineto{\pgfqpoint{3.936311in}{2.425172in}}%
\pgfpathlineto{\pgfqpoint{3.944118in}{2.433696in}}%
\pgfpathlineto{\pgfqpoint{3.951920in}{2.442232in}}%
\pgfpathclose%
\pgfusepath{fill}%
\end{pgfscope}%
\begin{pgfscope}%
\pgfpathrectangle{\pgfqpoint{1.150000in}{0.150000in}}{\pgfqpoint{5.700000in}{5.700000in}}%
\pgfusepath{clip}%
\pgfsetbuttcap%
\pgfsetroundjoin%
\definecolor{currentfill}{rgb}{0.280255,0.165693,0.476498}%
\pgfsetfillcolor{currentfill}%
\pgfsetfillopacity{0.700000}%
\pgfsetlinewidth{0.000000pt}%
\definecolor{currentstroke}{rgb}{0.000000,0.000000,0.000000}%
\pgfsetstrokecolor{currentstroke}%
\pgfsetdash{}{0pt}%
\pgfpathmoveto{\pgfqpoint{4.737345in}{2.675970in}}%
\pgfpathlineto{\pgfqpoint{4.750988in}{2.677095in}}%
\pgfpathlineto{\pgfqpoint{4.764642in}{2.678323in}}%
\pgfpathlineto{\pgfqpoint{4.778305in}{2.679655in}}%
\pgfpathlineto{\pgfqpoint{4.791978in}{2.681090in}}%
\pgfpathlineto{\pgfqpoint{4.784462in}{2.672834in}}%
\pgfpathlineto{\pgfqpoint{4.776941in}{2.664542in}}%
\pgfpathlineto{\pgfqpoint{4.769414in}{2.656213in}}%
\pgfpathlineto{\pgfqpoint{4.761881in}{2.647845in}}%
\pgfpathlineto{\pgfqpoint{4.748198in}{2.646366in}}%
\pgfpathlineto{\pgfqpoint{4.734525in}{2.644991in}}%
\pgfpathlineto{\pgfqpoint{4.720861in}{2.643720in}}%
\pgfpathlineto{\pgfqpoint{4.707208in}{2.642552in}}%
\pgfpathlineto{\pgfqpoint{4.714751in}{2.650956in}}%
\pgfpathlineto{\pgfqpoint{4.722288in}{2.659327in}}%
\pgfpathlineto{\pgfqpoint{4.729819in}{2.667665in}}%
\pgfpathlineto{\pgfqpoint{4.737345in}{2.675970in}}%
\pgfpathclose%
\pgfusepath{fill}%
\end{pgfscope}%
\begin{pgfscope}%
\pgfpathrectangle{\pgfqpoint{1.150000in}{0.150000in}}{\pgfqpoint{5.700000in}{5.700000in}}%
\pgfusepath{clip}%
\pgfsetbuttcap%
\pgfsetroundjoin%
\definecolor{currentfill}{rgb}{0.276194,0.190074,0.493001}%
\pgfsetfillcolor{currentfill}%
\pgfsetfillopacity{0.700000}%
\pgfsetlinewidth{0.000000pt}%
\definecolor{currentstroke}{rgb}{0.000000,0.000000,0.000000}%
\pgfsetstrokecolor{currentstroke}%
\pgfsetdash{}{0pt}%
\pgfpathmoveto{\pgfqpoint{2.959449in}{2.745256in}}%
\pgfpathlineto{\pgfqpoint{2.972831in}{2.732415in}}%
\pgfpathlineto{\pgfqpoint{2.986210in}{2.719748in}}%
\pgfpathlineto{\pgfqpoint{2.999587in}{2.707253in}}%
\pgfpathlineto{\pgfqpoint{3.012963in}{2.694929in}}%
\pgfpathlineto{\pgfqpoint{3.004762in}{2.689839in}}%
\pgfpathlineto{\pgfqpoint{2.996552in}{2.684855in}}%
\pgfpathlineto{\pgfqpoint{2.988333in}{2.679978in}}%
\pgfpathlineto{\pgfqpoint{2.980105in}{2.675210in}}%
\pgfpathlineto{\pgfqpoint{2.966706in}{2.687752in}}%
\pgfpathlineto{\pgfqpoint{2.953304in}{2.700465in}}%
\pgfpathlineto{\pgfqpoint{2.939900in}{2.713351in}}%
\pgfpathlineto{\pgfqpoint{2.926493in}{2.726411in}}%
\pgfpathlineto{\pgfqpoint{2.934746in}{2.730954in}}%
\pgfpathlineto{\pgfqpoint{2.942990in}{2.735610in}}%
\pgfpathlineto{\pgfqpoint{2.951224in}{2.740378in}}%
\pgfpathlineto{\pgfqpoint{2.959449in}{2.745256in}}%
\pgfpathclose%
\pgfusepath{fill}%
\end{pgfscope}%
\begin{pgfscope}%
\pgfpathrectangle{\pgfqpoint{1.150000in}{0.150000in}}{\pgfqpoint{5.700000in}{5.700000in}}%
\pgfusepath{clip}%
\pgfsetbuttcap%
\pgfsetroundjoin%
\definecolor{currentfill}{rgb}{0.274952,0.037752,0.364543}%
\pgfsetfillcolor{currentfill}%
\pgfsetfillopacity{0.700000}%
\pgfsetlinewidth{0.000000pt}%
\definecolor{currentstroke}{rgb}{0.000000,0.000000,0.000000}%
\pgfsetstrokecolor{currentstroke}%
\pgfsetdash{}{0pt}%
\pgfpathmoveto{\pgfqpoint{3.590079in}{2.442613in}}%
\pgfpathlineto{\pgfqpoint{3.603432in}{2.436317in}}%
\pgfpathlineto{\pgfqpoint{3.616789in}{2.430151in}}%
\pgfpathlineto{\pgfqpoint{3.630149in}{2.424116in}}%
\pgfpathlineto{\pgfqpoint{3.643514in}{2.418209in}}%
\pgfpathlineto{\pgfqpoint{3.635590in}{2.410463in}}%
\pgfpathlineto{\pgfqpoint{3.627661in}{2.402758in}}%
\pgfpathlineto{\pgfqpoint{3.619725in}{2.395095in}}%
\pgfpathlineto{\pgfqpoint{3.611783in}{2.387475in}}%
\pgfpathlineto{\pgfqpoint{3.598404in}{2.393538in}}%
\pgfpathlineto{\pgfqpoint{3.585028in}{2.399731in}}%
\pgfpathlineto{\pgfqpoint{3.571656in}{2.406053in}}%
\pgfpathlineto{\pgfqpoint{3.558288in}{2.412506in}}%
\pgfpathlineto{\pgfqpoint{3.566245in}{2.419962in}}%
\pgfpathlineto{\pgfqpoint{3.574196in}{2.427466in}}%
\pgfpathlineto{\pgfqpoint{3.582140in}{2.435017in}}%
\pgfpathlineto{\pgfqpoint{3.590079in}{2.442613in}}%
\pgfpathclose%
\pgfusepath{fill}%
\end{pgfscope}%
\begin{pgfscope}%
\pgfpathrectangle{\pgfqpoint{1.150000in}{0.150000in}}{\pgfqpoint{5.700000in}{5.700000in}}%
\pgfusepath{clip}%
\pgfsetbuttcap%
\pgfsetroundjoin%
\definecolor{currentfill}{rgb}{0.278791,0.062145,0.386592}%
\pgfsetfillcolor{currentfill}%
\pgfsetfillopacity{0.700000}%
\pgfsetlinewidth{0.000000pt}%
\definecolor{currentstroke}{rgb}{0.000000,0.000000,0.000000}%
\pgfsetstrokecolor{currentstroke}%
\pgfsetdash{}{0pt}%
\pgfpathmoveto{\pgfqpoint{4.175174in}{2.478928in}}%
\pgfpathlineto{\pgfqpoint{4.188638in}{2.477013in}}%
\pgfpathlineto{\pgfqpoint{4.202109in}{2.475210in}}%
\pgfpathlineto{\pgfqpoint{4.215587in}{2.473520in}}%
\pgfpathlineto{\pgfqpoint{4.229072in}{2.471941in}}%
\pgfpathlineto{\pgfqpoint{4.221357in}{2.463100in}}%
\pgfpathlineto{\pgfqpoint{4.213636in}{2.454247in}}%
\pgfpathlineto{\pgfqpoint{4.205910in}{2.445384in}}%
\pgfpathlineto{\pgfqpoint{4.198179in}{2.436511in}}%
\pgfpathlineto{\pgfqpoint{4.184683in}{2.438155in}}%
\pgfpathlineto{\pgfqpoint{4.171195in}{2.439912in}}%
\pgfpathlineto{\pgfqpoint{4.157714in}{2.441780in}}%
\pgfpathlineto{\pgfqpoint{4.144239in}{2.443761in}}%
\pgfpathlineto{\pgfqpoint{4.151981in}{2.452562in}}%
\pgfpathlineto{\pgfqpoint{4.159717in}{2.461357in}}%
\pgfpathlineto{\pgfqpoint{4.167448in}{2.470146in}}%
\pgfpathlineto{\pgfqpoint{4.175174in}{2.478928in}}%
\pgfpathclose%
\pgfusepath{fill}%
\end{pgfscope}%
\begin{pgfscope}%
\pgfpathrectangle{\pgfqpoint{1.150000in}{0.150000in}}{\pgfqpoint{5.700000in}{5.700000in}}%
\pgfusepath{clip}%
\pgfsetbuttcap%
\pgfsetroundjoin%
\definecolor{currentfill}{rgb}{0.281887,0.150881,0.465405}%
\pgfsetfillcolor{currentfill}%
\pgfsetfillopacity{0.700000}%
\pgfsetlinewidth{0.000000pt}%
\definecolor{currentstroke}{rgb}{0.000000,0.000000,0.000000}%
\pgfsetstrokecolor{currentstroke}%
\pgfsetdash{}{0pt}%
\pgfpathmoveto{\pgfqpoint{4.652691in}{2.638927in}}%
\pgfpathlineto{\pgfqpoint{4.666306in}{2.639676in}}%
\pgfpathlineto{\pgfqpoint{4.679930in}{2.640530in}}%
\pgfpathlineto{\pgfqpoint{4.693564in}{2.641489in}}%
\pgfpathlineto{\pgfqpoint{4.707208in}{2.642552in}}%
\pgfpathlineto{\pgfqpoint{4.699659in}{2.634114in}}%
\pgfpathlineto{\pgfqpoint{4.692105in}{2.625641in}}%
\pgfpathlineto{\pgfqpoint{4.684546in}{2.617133in}}%
\pgfpathlineto{\pgfqpoint{4.676981in}{2.608589in}}%
\pgfpathlineto{\pgfqpoint{4.663327in}{2.607500in}}%
\pgfpathlineto{\pgfqpoint{4.649683in}{2.606516in}}%
\pgfpathlineto{\pgfqpoint{4.636049in}{2.605637in}}%
\pgfpathlineto{\pgfqpoint{4.622424in}{2.604863in}}%
\pgfpathlineto{\pgfqpoint{4.629999in}{2.613426in}}%
\pgfpathlineto{\pgfqpoint{4.637569in}{2.621957in}}%
\pgfpathlineto{\pgfqpoint{4.645133in}{2.630457in}}%
\pgfpathlineto{\pgfqpoint{4.652691in}{2.638927in}}%
\pgfpathclose%
\pgfusepath{fill}%
\end{pgfscope}%
\begin{pgfscope}%
\pgfpathrectangle{\pgfqpoint{1.150000in}{0.150000in}}{\pgfqpoint{5.700000in}{5.700000in}}%
\pgfusepath{clip}%
\pgfsetbuttcap%
\pgfsetroundjoin%
\definecolor{currentfill}{rgb}{0.282327,0.094955,0.417331}%
\pgfsetfillcolor{currentfill}%
\pgfsetfillopacity{0.700000}%
\pgfsetlinewidth{0.000000pt}%
\definecolor{currentstroke}{rgb}{0.000000,0.000000,0.000000}%
\pgfsetstrokecolor{currentstroke}%
\pgfsetdash{}{0pt}%
\pgfpathmoveto{\pgfqpoint{3.259196in}{2.544464in}}%
\pgfpathlineto{\pgfqpoint{3.272541in}{2.535044in}}%
\pgfpathlineto{\pgfqpoint{3.285886in}{2.525773in}}%
\pgfpathlineto{\pgfqpoint{3.299233in}{2.516650in}}%
\pgfpathlineto{\pgfqpoint{3.312581in}{2.507673in}}%
\pgfpathlineto{\pgfqpoint{3.304520in}{2.501228in}}%
\pgfpathlineto{\pgfqpoint{3.296451in}{2.494859in}}%
\pgfpathlineto{\pgfqpoint{3.288375in}{2.488566in}}%
\pgfpathlineto{\pgfqpoint{3.280291in}{2.482352in}}%
\pgfpathlineto{\pgfqpoint{3.266924in}{2.491524in}}%
\pgfpathlineto{\pgfqpoint{3.253557in}{2.500843in}}%
\pgfpathlineto{\pgfqpoint{3.240192in}{2.510309in}}%
\pgfpathlineto{\pgfqpoint{3.226827in}{2.519925in}}%
\pgfpathlineto{\pgfqpoint{3.234931in}{2.525936in}}%
\pgfpathlineto{\pgfqpoint{3.243027in}{2.532031in}}%
\pgfpathlineto{\pgfqpoint{3.251115in}{2.538208in}}%
\pgfpathlineto{\pgfqpoint{3.259196in}{2.544464in}}%
\pgfpathclose%
\pgfusepath{fill}%
\end{pgfscope}%
\begin{pgfscope}%
\pgfpathrectangle{\pgfqpoint{1.150000in}{0.150000in}}{\pgfqpoint{5.700000in}{5.700000in}}%
\pgfusepath{clip}%
\pgfsetbuttcap%
\pgfsetroundjoin%
\definecolor{currentfill}{rgb}{0.273809,0.031497,0.358853}%
\pgfsetfillcolor{currentfill}%
\pgfsetfillopacity{0.700000}%
\pgfsetlinewidth{0.000000pt}%
\definecolor{currentstroke}{rgb}{0.000000,0.000000,0.000000}%
\pgfsetstrokecolor{currentstroke}%
\pgfsetdash{}{0pt}%
\pgfpathmoveto{\pgfqpoint{3.728585in}{2.427783in}}%
\pgfpathlineto{\pgfqpoint{3.741956in}{2.422650in}}%
\pgfpathlineto{\pgfqpoint{3.755331in}{2.417642in}}%
\pgfpathlineto{\pgfqpoint{3.768711in}{2.412758in}}%
\pgfpathlineto{\pgfqpoint{3.782095in}{2.407998in}}%
\pgfpathlineto{\pgfqpoint{3.774224in}{2.399837in}}%
\pgfpathlineto{\pgfqpoint{3.766346in}{2.391703in}}%
\pgfpathlineto{\pgfqpoint{3.758463in}{2.383598in}}%
\pgfpathlineto{\pgfqpoint{3.750574in}{2.375522in}}%
\pgfpathlineto{\pgfqpoint{3.737176in}{2.380420in}}%
\pgfpathlineto{\pgfqpoint{3.723783in}{2.385442in}}%
\pgfpathlineto{\pgfqpoint{3.710394in}{2.390588in}}%
\pgfpathlineto{\pgfqpoint{3.697010in}{2.395860in}}%
\pgfpathlineto{\pgfqpoint{3.704913in}{2.403791in}}%
\pgfpathlineto{\pgfqpoint{3.712809in}{2.411756in}}%
\pgfpathlineto{\pgfqpoint{3.720700in}{2.419754in}}%
\pgfpathlineto{\pgfqpoint{3.728585in}{2.427783in}}%
\pgfpathclose%
\pgfusepath{fill}%
\end{pgfscope}%
\begin{pgfscope}%
\pgfpathrectangle{\pgfqpoint{1.150000in}{0.150000in}}{\pgfqpoint{5.700000in}{5.700000in}}%
\pgfusepath{clip}%
\pgfsetbuttcap%
\pgfsetroundjoin%
\definecolor{currentfill}{rgb}{0.277941,0.056324,0.381191}%
\pgfsetfillcolor{currentfill}%
\pgfsetfillopacity{0.700000}%
\pgfsetlinewidth{0.000000pt}%
\definecolor{currentstroke}{rgb}{0.000000,0.000000,0.000000}%
\pgfsetstrokecolor{currentstroke}%
\pgfsetdash{}{0pt}%
\pgfpathmoveto{\pgfqpoint{3.451445in}{2.468926in}}%
\pgfpathlineto{\pgfqpoint{3.464791in}{2.461400in}}%
\pgfpathlineto{\pgfqpoint{3.478139in}{2.454010in}}%
\pgfpathlineto{\pgfqpoint{3.491490in}{2.446757in}}%
\pgfpathlineto{\pgfqpoint{3.504844in}{2.439640in}}%
\pgfpathlineto{\pgfqpoint{3.496864in}{2.432402in}}%
\pgfpathlineto{\pgfqpoint{3.488877in}{2.425219in}}%
\pgfpathlineto{\pgfqpoint{3.480884in}{2.418093in}}%
\pgfpathlineto{\pgfqpoint{3.472884in}{2.411025in}}%
\pgfpathlineto{\pgfqpoint{3.459514in}{2.418319in}}%
\pgfpathlineto{\pgfqpoint{3.446146in}{2.425747in}}%
\pgfpathlineto{\pgfqpoint{3.432781in}{2.433312in}}%
\pgfpathlineto{\pgfqpoint{3.419417in}{2.441014in}}%
\pgfpathlineto{\pgfqpoint{3.427435in}{2.447899in}}%
\pgfpathlineto{\pgfqpoint{3.435445in}{2.454847in}}%
\pgfpathlineto{\pgfqpoint{3.443449in}{2.461857in}}%
\pgfpathlineto{\pgfqpoint{3.451445in}{2.468926in}}%
\pgfpathclose%
\pgfusepath{fill}%
\end{pgfscope}%
\begin{pgfscope}%
\pgfpathrectangle{\pgfqpoint{1.150000in}{0.150000in}}{\pgfqpoint{5.700000in}{5.700000in}}%
\pgfusepath{clip}%
\pgfsetbuttcap%
\pgfsetroundjoin%
\definecolor{currentfill}{rgb}{0.279574,0.170599,0.479997}%
\pgfsetfillcolor{currentfill}%
\pgfsetfillopacity{0.700000}%
\pgfsetlinewidth{0.000000pt}%
\definecolor{currentstroke}{rgb}{0.000000,0.000000,0.000000}%
\pgfsetstrokecolor{currentstroke}%
\pgfsetdash{}{0pt}%
\pgfpathmoveto{\pgfqpoint{3.012963in}{2.694929in}}%
\pgfpathlineto{\pgfqpoint{3.026336in}{2.682775in}}%
\pgfpathlineto{\pgfqpoint{3.039708in}{2.670790in}}%
\pgfpathlineto{\pgfqpoint{3.053078in}{2.658971in}}%
\pgfpathlineto{\pgfqpoint{3.066448in}{2.647318in}}%
\pgfpathlineto{\pgfqpoint{3.058270in}{2.642017in}}%
\pgfpathlineto{\pgfqpoint{3.050084in}{2.636817in}}%
\pgfpathlineto{\pgfqpoint{3.041889in}{2.631720in}}%
\pgfpathlineto{\pgfqpoint{3.033685in}{2.626728in}}%
\pgfpathlineto{\pgfqpoint{3.020293in}{2.638598in}}%
\pgfpathlineto{\pgfqpoint{3.006899in}{2.650634in}}%
\pgfpathlineto{\pgfqpoint{2.993503in}{2.662838in}}%
\pgfpathlineto{\pgfqpoint{2.980105in}{2.675210in}}%
\pgfpathlineto{\pgfqpoint{2.988333in}{2.679978in}}%
\pgfpathlineto{\pgfqpoint{2.996552in}{2.684855in}}%
\pgfpathlineto{\pgfqpoint{3.004762in}{2.689839in}}%
\pgfpathlineto{\pgfqpoint{3.012963in}{2.694929in}}%
\pgfpathclose%
\pgfusepath{fill}%
\end{pgfscope}%
\begin{pgfscope}%
\pgfpathrectangle{\pgfqpoint{1.150000in}{0.150000in}}{\pgfqpoint{5.700000in}{5.700000in}}%
\pgfusepath{clip}%
\pgfsetbuttcap%
\pgfsetroundjoin%
\definecolor{currentfill}{rgb}{0.283072,0.130895,0.449241}%
\pgfsetfillcolor{currentfill}%
\pgfsetfillopacity{0.700000}%
\pgfsetlinewidth{0.000000pt}%
\definecolor{currentstroke}{rgb}{0.000000,0.000000,0.000000}%
\pgfsetstrokecolor{currentstroke}%
\pgfsetdash{}{0pt}%
\pgfpathmoveto{\pgfqpoint{4.568019in}{2.602822in}}%
\pgfpathlineto{\pgfqpoint{4.581607in}{2.603173in}}%
\pgfpathlineto{\pgfqpoint{4.595203in}{2.603631in}}%
\pgfpathlineto{\pgfqpoint{4.608809in}{2.604195in}}%
\pgfpathlineto{\pgfqpoint{4.622424in}{2.604863in}}%
\pgfpathlineto{\pgfqpoint{4.614844in}{2.596269in}}%
\pgfpathlineto{\pgfqpoint{4.607258in}{2.587642in}}%
\pgfpathlineto{\pgfqpoint{4.599666in}{2.578983in}}%
\pgfpathlineto{\pgfqpoint{4.592069in}{2.570290in}}%
\pgfpathlineto{\pgfqpoint{4.578444in}{2.569614in}}%
\pgfpathlineto{\pgfqpoint{4.564829in}{2.569044in}}%
\pgfpathlineto{\pgfqpoint{4.551222in}{2.568580in}}%
\pgfpathlineto{\pgfqpoint{4.537625in}{2.568221in}}%
\pgfpathlineto{\pgfqpoint{4.545232in}{2.576914in}}%
\pgfpathlineto{\pgfqpoint{4.552833in}{2.585578in}}%
\pgfpathlineto{\pgfqpoint{4.560429in}{2.594214in}}%
\pgfpathlineto{\pgfqpoint{4.568019in}{2.602822in}}%
\pgfpathclose%
\pgfusepath{fill}%
\end{pgfscope}%
\begin{pgfscope}%
\pgfpathrectangle{\pgfqpoint{1.150000in}{0.150000in}}{\pgfqpoint{5.700000in}{5.700000in}}%
\pgfusepath{clip}%
\pgfsetbuttcap%
\pgfsetroundjoin%
\definecolor{currentfill}{rgb}{0.273809,0.031497,0.358853}%
\pgfsetfillcolor{currentfill}%
\pgfsetfillopacity{0.700000}%
\pgfsetlinewidth{0.000000pt}%
\definecolor{currentstroke}{rgb}{0.000000,0.000000,0.000000}%
\pgfsetstrokecolor{currentstroke}%
\pgfsetdash{}{0pt}%
\pgfpathmoveto{\pgfqpoint{3.867059in}{2.423560in}}%
\pgfpathlineto{\pgfqpoint{3.880456in}{2.419530in}}%
\pgfpathlineto{\pgfqpoint{3.893859in}{2.415620in}}%
\pgfpathlineto{\pgfqpoint{3.907266in}{2.411830in}}%
\pgfpathlineto{\pgfqpoint{3.920680in}{2.408159in}}%
\pgfpathlineto{\pgfqpoint{3.912856in}{2.399672in}}%
\pgfpathlineto{\pgfqpoint{3.905027in}{2.391200in}}%
\pgfpathlineto{\pgfqpoint{3.897192in}{2.382743in}}%
\pgfpathlineto{\pgfqpoint{3.889352in}{2.374302in}}%
\pgfpathlineto{\pgfqpoint{3.875926in}{2.378093in}}%
\pgfpathlineto{\pgfqpoint{3.862506in}{2.382004in}}%
\pgfpathlineto{\pgfqpoint{3.849091in}{2.386034in}}%
\pgfpathlineto{\pgfqpoint{3.835682in}{2.390184in}}%
\pgfpathlineto{\pgfqpoint{3.843535in}{2.398498in}}%
\pgfpathlineto{\pgfqpoint{3.851382in}{2.406833in}}%
\pgfpathlineto{\pgfqpoint{3.859223in}{2.415187in}}%
\pgfpathlineto{\pgfqpoint{3.867059in}{2.423560in}}%
\pgfpathclose%
\pgfusepath{fill}%
\end{pgfscope}%
\begin{pgfscope}%
\pgfpathrectangle{\pgfqpoint{1.150000in}{0.150000in}}{\pgfqpoint{5.700000in}{5.700000in}}%
\pgfusepath{clip}%
\pgfsetbuttcap%
\pgfsetroundjoin%
\definecolor{currentfill}{rgb}{0.283197,0.115680,0.436115}%
\pgfsetfillcolor{currentfill}%
\pgfsetfillopacity{0.700000}%
\pgfsetlinewidth{0.000000pt}%
\definecolor{currentstroke}{rgb}{0.000000,0.000000,0.000000}%
\pgfsetstrokecolor{currentstroke}%
\pgfsetdash{}{0pt}%
\pgfpathmoveto{\pgfqpoint{4.483326in}{2.567853in}}%
\pgfpathlineto{\pgfqpoint{4.496888in}{2.567785in}}%
\pgfpathlineto{\pgfqpoint{4.510458in}{2.567823in}}%
\pgfpathlineto{\pgfqpoint{4.524037in}{2.567969in}}%
\pgfpathlineto{\pgfqpoint{4.537625in}{2.568221in}}%
\pgfpathlineto{\pgfqpoint{4.530013in}{2.559499in}}%
\pgfpathlineto{\pgfqpoint{4.522396in}{2.550748in}}%
\pgfpathlineto{\pgfqpoint{4.514773in}{2.541968in}}%
\pgfpathlineto{\pgfqpoint{4.507144in}{2.533158in}}%
\pgfpathlineto{\pgfqpoint{4.493547in}{2.532917in}}%
\pgfpathlineto{\pgfqpoint{4.479958in}{2.532783in}}%
\pgfpathlineto{\pgfqpoint{4.466378in}{2.532755in}}%
\pgfpathlineto{\pgfqpoint{4.452807in}{2.532835in}}%
\pgfpathlineto{\pgfqpoint{4.460445in}{2.541627in}}%
\pgfpathlineto{\pgfqpoint{4.468077in}{2.550394in}}%
\pgfpathlineto{\pgfqpoint{4.475705in}{2.559136in}}%
\pgfpathlineto{\pgfqpoint{4.483326in}{2.567853in}}%
\pgfpathclose%
\pgfusepath{fill}%
\end{pgfscope}%
\begin{pgfscope}%
\pgfpathrectangle{\pgfqpoint{1.150000in}{0.150000in}}{\pgfqpoint{5.700000in}{5.700000in}}%
\pgfusepath{clip}%
\pgfsetbuttcap%
\pgfsetroundjoin%
\definecolor{currentfill}{rgb}{0.277018,0.050344,0.375715}%
\pgfsetfillcolor{currentfill}%
\pgfsetfillopacity{0.700000}%
\pgfsetlinewidth{0.000000pt}%
\definecolor{currentstroke}{rgb}{0.000000,0.000000,0.000000}%
\pgfsetstrokecolor{currentstroke}%
\pgfsetdash{}{0pt}%
\pgfpathmoveto{\pgfqpoint{4.090412in}{2.452820in}}%
\pgfpathlineto{\pgfqpoint{4.103859in}{2.450384in}}%
\pgfpathlineto{\pgfqpoint{4.117312in}{2.448063in}}%
\pgfpathlineto{\pgfqpoint{4.130772in}{2.445855in}}%
\pgfpathlineto{\pgfqpoint{4.144239in}{2.443761in}}%
\pgfpathlineto{\pgfqpoint{4.136492in}{2.434955in}}%
\pgfpathlineto{\pgfqpoint{4.128740in}{2.426145in}}%
\pgfpathlineto{\pgfqpoint{4.120983in}{2.417330in}}%
\pgfpathlineto{\pgfqpoint{4.113220in}{2.408512in}}%
\pgfpathlineto{\pgfqpoint{4.099742in}{2.410690in}}%
\pgfpathlineto{\pgfqpoint{4.086271in}{2.412981in}}%
\pgfpathlineto{\pgfqpoint{4.072807in}{2.415387in}}%
\pgfpathlineto{\pgfqpoint{4.059350in}{2.417906in}}%
\pgfpathlineto{\pgfqpoint{4.067123in}{2.426634in}}%
\pgfpathlineto{\pgfqpoint{4.074892in}{2.435362in}}%
\pgfpathlineto{\pgfqpoint{4.082655in}{2.444091in}}%
\pgfpathlineto{\pgfqpoint{4.090412in}{2.452820in}}%
\pgfpathclose%
\pgfusepath{fill}%
\end{pgfscope}%
\begin{pgfscope}%
\pgfpathrectangle{\pgfqpoint{1.150000in}{0.150000in}}{\pgfqpoint{5.700000in}{5.700000in}}%
\pgfusepath{clip}%
\pgfsetbuttcap%
\pgfsetroundjoin%
\definecolor{currentfill}{rgb}{0.220057,0.343307,0.549413}%
\pgfsetfillcolor{currentfill}%
\pgfsetfillopacity{0.700000}%
\pgfsetlinewidth{0.000000pt}%
\definecolor{currentstroke}{rgb}{0.000000,0.000000,0.000000}%
\pgfsetstrokecolor{currentstroke}%
\pgfsetdash{}{0pt}%
\pgfpathmoveto{\pgfqpoint{5.554255in}{3.039944in}}%
\pgfpathlineto{\pgfqpoint{5.568234in}{3.043864in}}%
\pgfpathlineto{\pgfqpoint{5.582227in}{3.047882in}}%
\pgfpathlineto{\pgfqpoint{5.596232in}{3.051997in}}%
\pgfpathlineto{\pgfqpoint{5.610250in}{3.056209in}}%
\pgfpathlineto{\pgfqpoint{5.603087in}{3.050487in}}%
\pgfpathlineto{\pgfqpoint{5.595919in}{3.044744in}}%
\pgfpathlineto{\pgfqpoint{5.588744in}{3.038978in}}%
\pgfpathlineto{\pgfqpoint{5.581563in}{3.033185in}}%
\pgfpathlineto{\pgfqpoint{5.567527in}{3.028761in}}%
\pgfpathlineto{\pgfqpoint{5.553505in}{3.024434in}}%
\pgfpathlineto{\pgfqpoint{5.539495in}{3.020206in}}%
\pgfpathlineto{\pgfqpoint{5.525498in}{3.016075in}}%
\pgfpathlineto{\pgfqpoint{5.532697in}{3.022072in}}%
\pgfpathlineto{\pgfqpoint{5.539889in}{3.028048in}}%
\pgfpathlineto{\pgfqpoint{5.547075in}{3.034005in}}%
\pgfpathlineto{\pgfqpoint{5.554255in}{3.039944in}}%
\pgfpathclose%
\pgfusepath{fill}%
\end{pgfscope}%
\begin{pgfscope}%
\pgfpathrectangle{\pgfqpoint{1.150000in}{0.150000in}}{\pgfqpoint{5.700000in}{5.700000in}}%
\pgfusepath{clip}%
\pgfsetbuttcap%
\pgfsetroundjoin%
\definecolor{currentfill}{rgb}{0.281887,0.150881,0.465405}%
\pgfsetfillcolor{currentfill}%
\pgfsetfillopacity{0.700000}%
\pgfsetlinewidth{0.000000pt}%
\definecolor{currentstroke}{rgb}{0.000000,0.000000,0.000000}%
\pgfsetstrokecolor{currentstroke}%
\pgfsetdash{}{0pt}%
\pgfpathmoveto{\pgfqpoint{3.066448in}{2.647318in}}%
\pgfpathlineto{\pgfqpoint{3.079815in}{2.635829in}}%
\pgfpathlineto{\pgfqpoint{3.093182in}{2.624502in}}%
\pgfpathlineto{\pgfqpoint{3.106548in}{2.613338in}}%
\pgfpathlineto{\pgfqpoint{3.119914in}{2.602334in}}%
\pgfpathlineto{\pgfqpoint{3.111759in}{2.596823in}}%
\pgfpathlineto{\pgfqpoint{3.103596in}{2.591409in}}%
\pgfpathlineto{\pgfqpoint{3.095424in}{2.586093in}}%
\pgfpathlineto{\pgfqpoint{3.087244in}{2.580877in}}%
\pgfpathlineto{\pgfqpoint{3.073856in}{2.592098in}}%
\pgfpathlineto{\pgfqpoint{3.060467in}{2.603479in}}%
\pgfpathlineto{\pgfqpoint{3.047077in}{2.615022in}}%
\pgfpathlineto{\pgfqpoint{3.033685in}{2.626728in}}%
\pgfpathlineto{\pgfqpoint{3.041889in}{2.631720in}}%
\pgfpathlineto{\pgfqpoint{3.050084in}{2.636817in}}%
\pgfpathlineto{\pgfqpoint{3.058270in}{2.642017in}}%
\pgfpathlineto{\pgfqpoint{3.066448in}{2.647318in}}%
\pgfpathclose%
\pgfusepath{fill}%
\end{pgfscope}%
\begin{pgfscope}%
\pgfpathrectangle{\pgfqpoint{1.150000in}{0.150000in}}{\pgfqpoint{5.700000in}{5.700000in}}%
\pgfusepath{clip}%
\pgfsetbuttcap%
\pgfsetroundjoin%
\definecolor{currentfill}{rgb}{0.227802,0.326594,0.546532}%
\pgfsetfillcolor{currentfill}%
\pgfsetfillopacity{0.700000}%
\pgfsetlinewidth{0.000000pt}%
\definecolor{currentstroke}{rgb}{0.000000,0.000000,0.000000}%
\pgfsetstrokecolor{currentstroke}%
\pgfsetdash{}{0pt}%
\pgfpathmoveto{\pgfqpoint{5.469640in}{3.000530in}}%
\pgfpathlineto{\pgfqpoint{5.483585in}{3.004270in}}%
\pgfpathlineto{\pgfqpoint{5.497544in}{3.008107in}}%
\pgfpathlineto{\pgfqpoint{5.511515in}{3.012042in}}%
\pgfpathlineto{\pgfqpoint{5.525498in}{3.016075in}}%
\pgfpathlineto{\pgfqpoint{5.518294in}{3.010053in}}%
\pgfpathlineto{\pgfqpoint{5.511083in}{3.004005in}}%
\pgfpathlineto{\pgfqpoint{5.503866in}{2.997929in}}%
\pgfpathlineto{\pgfqpoint{5.496643in}{2.991821in}}%
\pgfpathlineto{\pgfqpoint{5.482643in}{2.987595in}}%
\pgfpathlineto{\pgfqpoint{5.468655in}{2.983467in}}%
\pgfpathlineto{\pgfqpoint{5.454681in}{2.979437in}}%
\pgfpathlineto{\pgfqpoint{5.440719in}{2.975506in}}%
\pgfpathlineto{\pgfqpoint{5.447958in}{2.981800in}}%
\pgfpathlineto{\pgfqpoint{5.455192in}{2.988067in}}%
\pgfpathlineto{\pgfqpoint{5.462419in}{2.994310in}}%
\pgfpathlineto{\pgfqpoint{5.469640in}{3.000530in}}%
\pgfpathclose%
\pgfusepath{fill}%
\end{pgfscope}%
\begin{pgfscope}%
\pgfpathrectangle{\pgfqpoint{1.150000in}{0.150000in}}{\pgfqpoint{5.700000in}{5.700000in}}%
\pgfusepath{clip}%
\pgfsetbuttcap%
\pgfsetroundjoin%
\definecolor{currentfill}{rgb}{0.280894,0.078907,0.402329}%
\pgfsetfillcolor{currentfill}%
\pgfsetfillopacity{0.700000}%
\pgfsetlinewidth{0.000000pt}%
\definecolor{currentstroke}{rgb}{0.000000,0.000000,0.000000}%
\pgfsetstrokecolor{currentstroke}%
\pgfsetdash{}{0pt}%
\pgfpathmoveto{\pgfqpoint{3.312581in}{2.507673in}}%
\pgfpathlineto{\pgfqpoint{3.325930in}{2.498842in}}%
\pgfpathlineto{\pgfqpoint{3.339280in}{2.490155in}}%
\pgfpathlineto{\pgfqpoint{3.352632in}{2.481612in}}%
\pgfpathlineto{\pgfqpoint{3.365986in}{2.473211in}}%
\pgfpathlineto{\pgfqpoint{3.357943in}{2.466578in}}%
\pgfpathlineto{\pgfqpoint{3.349894in}{2.460015in}}%
\pgfpathlineto{\pgfqpoint{3.341837in}{2.453525in}}%
\pgfpathlineto{\pgfqpoint{3.333772in}{2.447109in}}%
\pgfpathlineto{\pgfqpoint{3.320400in}{2.455705in}}%
\pgfpathlineto{\pgfqpoint{3.307029in}{2.464443in}}%
\pgfpathlineto{\pgfqpoint{3.293660in}{2.473325in}}%
\pgfpathlineto{\pgfqpoint{3.280291in}{2.482352in}}%
\pgfpathlineto{\pgfqpoint{3.288375in}{2.488566in}}%
\pgfpathlineto{\pgfqpoint{3.296451in}{2.494859in}}%
\pgfpathlineto{\pgfqpoint{3.304520in}{2.501228in}}%
\pgfpathlineto{\pgfqpoint{3.312581in}{2.507673in}}%
\pgfpathclose%
\pgfusepath{fill}%
\end{pgfscope}%
\begin{pgfscope}%
\pgfpathrectangle{\pgfqpoint{1.150000in}{0.150000in}}{\pgfqpoint{5.700000in}{5.700000in}}%
\pgfusepath{clip}%
\pgfsetbuttcap%
\pgfsetroundjoin%
\definecolor{currentfill}{rgb}{0.282656,0.100196,0.422160}%
\pgfsetfillcolor{currentfill}%
\pgfsetfillopacity{0.700000}%
\pgfsetlinewidth{0.000000pt}%
\definecolor{currentstroke}{rgb}{0.000000,0.000000,0.000000}%
\pgfsetstrokecolor{currentstroke}%
\pgfsetdash{}{0pt}%
\pgfpathmoveto{\pgfqpoint{4.398608in}{2.534231in}}%
\pgfpathlineto{\pgfqpoint{4.412145in}{2.533720in}}%
\pgfpathlineto{\pgfqpoint{4.425690in}{2.533317in}}%
\pgfpathlineto{\pgfqpoint{4.439244in}{2.533022in}}%
\pgfpathlineto{\pgfqpoint{4.452807in}{2.532835in}}%
\pgfpathlineto{\pgfqpoint{4.445163in}{2.524018in}}%
\pgfpathlineto{\pgfqpoint{4.437515in}{2.515176in}}%
\pgfpathlineto{\pgfqpoint{4.429861in}{2.506308in}}%
\pgfpathlineto{\pgfqpoint{4.422201in}{2.497414in}}%
\pgfpathlineto{\pgfqpoint{4.408629in}{2.497631in}}%
\pgfpathlineto{\pgfqpoint{4.395066in}{2.497955in}}%
\pgfpathlineto{\pgfqpoint{4.381511in}{2.498388in}}%
\pgfpathlineto{\pgfqpoint{4.367964in}{2.498929in}}%
\pgfpathlineto{\pgfqpoint{4.375633in}{2.507786in}}%
\pgfpathlineto{\pgfqpoint{4.383297in}{2.516622in}}%
\pgfpathlineto{\pgfqpoint{4.390955in}{2.525437in}}%
\pgfpathlineto{\pgfqpoint{4.398608in}{2.534231in}}%
\pgfpathclose%
\pgfusepath{fill}%
\end{pgfscope}%
\begin{pgfscope}%
\pgfpathrectangle{\pgfqpoint{1.150000in}{0.150000in}}{\pgfqpoint{5.700000in}{5.700000in}}%
\pgfusepath{clip}%
\pgfsetbuttcap%
\pgfsetroundjoin%
\definecolor{currentfill}{rgb}{0.235526,0.309527,0.542944}%
\pgfsetfillcolor{currentfill}%
\pgfsetfillopacity{0.700000}%
\pgfsetlinewidth{0.000000pt}%
\definecolor{currentstroke}{rgb}{0.000000,0.000000,0.000000}%
\pgfsetstrokecolor{currentstroke}%
\pgfsetdash{}{0pt}%
\pgfpathmoveto{\pgfqpoint{5.384997in}{2.960764in}}%
\pgfpathlineto{\pgfqpoint{5.398909in}{2.964302in}}%
\pgfpathlineto{\pgfqpoint{5.412833in}{2.967938in}}%
\pgfpathlineto{\pgfqpoint{5.426770in}{2.971673in}}%
\pgfpathlineto{\pgfqpoint{5.440719in}{2.975506in}}%
\pgfpathlineto{\pgfqpoint{5.433473in}{2.969183in}}%
\pgfpathlineto{\pgfqpoint{5.426222in}{2.962830in}}%
\pgfpathlineto{\pgfqpoint{5.418963in}{2.956443in}}%
\pgfpathlineto{\pgfqpoint{5.411699in}{2.950022in}}%
\pgfpathlineto{\pgfqpoint{5.397734in}{2.946014in}}%
\pgfpathlineto{\pgfqpoint{5.383782in}{2.942106in}}%
\pgfpathlineto{\pgfqpoint{5.369843in}{2.938296in}}%
\pgfpathlineto{\pgfqpoint{5.355916in}{2.934584in}}%
\pgfpathlineto{\pgfqpoint{5.363195in}{2.941173in}}%
\pgfpathlineto{\pgfqpoint{5.370469in}{2.947732in}}%
\pgfpathlineto{\pgfqpoint{5.377736in}{2.954261in}}%
\pgfpathlineto{\pgfqpoint{5.384997in}{2.960764in}}%
\pgfpathclose%
\pgfusepath{fill}%
\end{pgfscope}%
\begin{pgfscope}%
\pgfpathrectangle{\pgfqpoint{1.150000in}{0.150000in}}{\pgfqpoint{5.700000in}{5.700000in}}%
\pgfusepath{clip}%
\pgfsetbuttcap%
\pgfsetroundjoin%
\definecolor{currentfill}{rgb}{0.243113,0.292092,0.538516}%
\pgfsetfillcolor{currentfill}%
\pgfsetfillopacity{0.700000}%
\pgfsetlinewidth{0.000000pt}%
\definecolor{currentstroke}{rgb}{0.000000,0.000000,0.000000}%
\pgfsetstrokecolor{currentstroke}%
\pgfsetdash{}{0pt}%
\pgfpathmoveto{\pgfqpoint{5.300330in}{2.920728in}}%
\pgfpathlineto{\pgfqpoint{5.314208in}{2.924044in}}%
\pgfpathlineto{\pgfqpoint{5.328099in}{2.927458in}}%
\pgfpathlineto{\pgfqpoint{5.342001in}{2.930972in}}%
\pgfpathlineto{\pgfqpoint{5.355916in}{2.934584in}}%
\pgfpathlineto{\pgfqpoint{5.348630in}{2.927963in}}%
\pgfpathlineto{\pgfqpoint{5.341338in}{2.921306in}}%
\pgfpathlineto{\pgfqpoint{5.334039in}{2.914614in}}%
\pgfpathlineto{\pgfqpoint{5.326735in}{2.907882in}}%
\pgfpathlineto{\pgfqpoint{5.312806in}{2.904114in}}%
\pgfpathlineto{\pgfqpoint{5.298889in}{2.900445in}}%
\pgfpathlineto{\pgfqpoint{5.284984in}{2.896876in}}%
\pgfpathlineto{\pgfqpoint{5.271092in}{2.893406in}}%
\pgfpathlineto{\pgfqpoint{5.278411in}{2.900285in}}%
\pgfpathlineto{\pgfqpoint{5.285723in}{2.907131in}}%
\pgfpathlineto{\pgfqpoint{5.293030in}{2.913945in}}%
\pgfpathlineto{\pgfqpoint{5.300330in}{2.920728in}}%
\pgfpathclose%
\pgfusepath{fill}%
\end{pgfscope}%
\begin{pgfscope}%
\pgfpathrectangle{\pgfqpoint{1.150000in}{0.150000in}}{\pgfqpoint{5.700000in}{5.700000in}}%
\pgfusepath{clip}%
\pgfsetbuttcap%
\pgfsetroundjoin%
\definecolor{currentfill}{rgb}{0.273809,0.031497,0.358853}%
\pgfsetfillcolor{currentfill}%
\pgfsetfillopacity{0.700000}%
\pgfsetlinewidth{0.000000pt}%
\definecolor{currentstroke}{rgb}{0.000000,0.000000,0.000000}%
\pgfsetstrokecolor{currentstroke}%
\pgfsetdash{}{0pt}%
\pgfpathmoveto{\pgfqpoint{3.643514in}{2.418209in}}%
\pgfpathlineto{\pgfqpoint{3.656882in}{2.412431in}}%
\pgfpathlineto{\pgfqpoint{3.670254in}{2.406780in}}%
\pgfpathlineto{\pgfqpoint{3.683630in}{2.401257in}}%
\pgfpathlineto{\pgfqpoint{3.697010in}{2.395860in}}%
\pgfpathlineto{\pgfqpoint{3.689101in}{2.387964in}}%
\pgfpathlineto{\pgfqpoint{3.681186in}{2.380104in}}%
\pgfpathlineto{\pgfqpoint{3.673265in}{2.372282in}}%
\pgfpathlineto{\pgfqpoint{3.665338in}{2.364499in}}%
\pgfpathlineto{\pgfqpoint{3.651944in}{2.370053in}}%
\pgfpathlineto{\pgfqpoint{3.638553in}{2.375733in}}%
\pgfpathlineto{\pgfqpoint{3.625166in}{2.381540in}}%
\pgfpathlineto{\pgfqpoint{3.611783in}{2.387475in}}%
\pgfpathlineto{\pgfqpoint{3.619725in}{2.395095in}}%
\pgfpathlineto{\pgfqpoint{3.627661in}{2.402758in}}%
\pgfpathlineto{\pgfqpoint{3.635590in}{2.410463in}}%
\pgfpathlineto{\pgfqpoint{3.643514in}{2.418209in}}%
\pgfpathclose%
\pgfusepath{fill}%
\end{pgfscope}%
\begin{pgfscope}%
\pgfpathrectangle{\pgfqpoint{1.150000in}{0.150000in}}{\pgfqpoint{5.700000in}{5.700000in}}%
\pgfusepath{clip}%
\pgfsetbuttcap%
\pgfsetroundjoin%
\definecolor{currentfill}{rgb}{0.250425,0.274290,0.533103}%
\pgfsetfillcolor{currentfill}%
\pgfsetfillopacity{0.700000}%
\pgfsetlinewidth{0.000000pt}%
\definecolor{currentstroke}{rgb}{0.000000,0.000000,0.000000}%
\pgfsetstrokecolor{currentstroke}%
\pgfsetdash{}{0pt}%
\pgfpathmoveto{\pgfqpoint{5.215643in}{2.880519in}}%
\pgfpathlineto{\pgfqpoint{5.229487in}{2.883591in}}%
\pgfpathlineto{\pgfqpoint{5.243343in}{2.886763in}}%
\pgfpathlineto{\pgfqpoint{5.257212in}{2.890035in}}%
\pgfpathlineto{\pgfqpoint{5.271092in}{2.893406in}}%
\pgfpathlineto{\pgfqpoint{5.263767in}{2.886490in}}%
\pgfpathlineto{\pgfqpoint{5.256436in}{2.879536in}}%
\pgfpathlineto{\pgfqpoint{5.249098in}{2.872544in}}%
\pgfpathlineto{\pgfqpoint{5.241755in}{2.865510in}}%
\pgfpathlineto{\pgfqpoint{5.227861in}{2.862002in}}%
\pgfpathlineto{\pgfqpoint{5.213979in}{2.858594in}}%
\pgfpathlineto{\pgfqpoint{5.200109in}{2.855286in}}%
\pgfpathlineto{\pgfqpoint{5.186252in}{2.852077in}}%
\pgfpathlineto{\pgfqpoint{5.193609in}{2.859241in}}%
\pgfpathlineto{\pgfqpoint{5.200959in}{2.866368in}}%
\pgfpathlineto{\pgfqpoint{5.208304in}{2.873460in}}%
\pgfpathlineto{\pgfqpoint{5.215643in}{2.880519in}}%
\pgfpathclose%
\pgfusepath{fill}%
\end{pgfscope}%
\begin{pgfscope}%
\pgfpathrectangle{\pgfqpoint{1.150000in}{0.150000in}}{\pgfqpoint{5.700000in}{5.700000in}}%
\pgfusepath{clip}%
\pgfsetbuttcap%
\pgfsetroundjoin%
\definecolor{currentfill}{rgb}{0.274952,0.037752,0.364543}%
\pgfsetfillcolor{currentfill}%
\pgfsetfillopacity{0.700000}%
\pgfsetlinewidth{0.000000pt}%
\definecolor{currentstroke}{rgb}{0.000000,0.000000,0.000000}%
\pgfsetstrokecolor{currentstroke}%
\pgfsetdash{}{0pt}%
\pgfpathmoveto{\pgfqpoint{4.005585in}{2.429136in}}%
\pgfpathlineto{\pgfqpoint{4.019017in}{2.426155in}}%
\pgfpathlineto{\pgfqpoint{4.032455in}{2.423290in}}%
\pgfpathlineto{\pgfqpoint{4.045899in}{2.420541in}}%
\pgfpathlineto{\pgfqpoint{4.059350in}{2.417906in}}%
\pgfpathlineto{\pgfqpoint{4.051571in}{2.409180in}}%
\pgfpathlineto{\pgfqpoint{4.043786in}{2.400456in}}%
\pgfpathlineto{\pgfqpoint{4.035997in}{2.391735in}}%
\pgfpathlineto{\pgfqpoint{4.028201in}{2.383018in}}%
\pgfpathlineto{\pgfqpoint{4.014740in}{2.385754in}}%
\pgfpathlineto{\pgfqpoint{4.001284in}{2.388605in}}%
\pgfpathlineto{\pgfqpoint{3.987835in}{2.391572in}}%
\pgfpathlineto{\pgfqpoint{3.974392in}{2.394655in}}%
\pgfpathlineto{\pgfqpoint{3.982198in}{2.403264in}}%
\pgfpathlineto{\pgfqpoint{3.989999in}{2.411881in}}%
\pgfpathlineto{\pgfqpoint{3.997795in}{2.420505in}}%
\pgfpathlineto{\pgfqpoint{4.005585in}{2.429136in}}%
\pgfpathclose%
\pgfusepath{fill}%
\end{pgfscope}%
\begin{pgfscope}%
\pgfpathrectangle{\pgfqpoint{1.150000in}{0.150000in}}{\pgfqpoint{5.700000in}{5.700000in}}%
\pgfusepath{clip}%
\pgfsetbuttcap%
\pgfsetroundjoin%
\definecolor{currentfill}{rgb}{0.276022,0.044167,0.370164}%
\pgfsetfillcolor{currentfill}%
\pgfsetfillopacity{0.700000}%
\pgfsetlinewidth{0.000000pt}%
\definecolor{currentstroke}{rgb}{0.000000,0.000000,0.000000}%
\pgfsetstrokecolor{currentstroke}%
\pgfsetdash{}{0pt}%
\pgfpathmoveto{\pgfqpoint{3.504844in}{2.439640in}}%
\pgfpathlineto{\pgfqpoint{3.518200in}{2.432657in}}%
\pgfpathlineto{\pgfqpoint{3.531560in}{2.425807in}}%
\pgfpathlineto{\pgfqpoint{3.544922in}{2.419091in}}%
\pgfpathlineto{\pgfqpoint{3.558288in}{2.412506in}}%
\pgfpathlineto{\pgfqpoint{3.550324in}{2.405100in}}%
\pgfpathlineto{\pgfqpoint{3.542354in}{2.397744in}}%
\pgfpathlineto{\pgfqpoint{3.534377in}{2.390440in}}%
\pgfpathlineto{\pgfqpoint{3.526394in}{2.383191in}}%
\pgfpathlineto{\pgfqpoint{3.513012in}{2.389950in}}%
\pgfpathlineto{\pgfqpoint{3.499633in}{2.396842in}}%
\pgfpathlineto{\pgfqpoint{3.486257in}{2.403867in}}%
\pgfpathlineto{\pgfqpoint{3.472884in}{2.411025in}}%
\pgfpathlineto{\pgfqpoint{3.480884in}{2.418093in}}%
\pgfpathlineto{\pgfqpoint{3.488877in}{2.425219in}}%
\pgfpathlineto{\pgfqpoint{3.496864in}{2.432402in}}%
\pgfpathlineto{\pgfqpoint{3.504844in}{2.439640in}}%
\pgfpathclose%
\pgfusepath{fill}%
\end{pgfscope}%
\begin{pgfscope}%
\pgfpathrectangle{\pgfqpoint{1.150000in}{0.150000in}}{\pgfqpoint{5.700000in}{5.700000in}}%
\pgfusepath{clip}%
\pgfsetbuttcap%
\pgfsetroundjoin%
\definecolor{currentfill}{rgb}{0.283072,0.130895,0.449241}%
\pgfsetfillcolor{currentfill}%
\pgfsetfillopacity{0.700000}%
\pgfsetlinewidth{0.000000pt}%
\definecolor{currentstroke}{rgb}{0.000000,0.000000,0.000000}%
\pgfsetstrokecolor{currentstroke}%
\pgfsetdash{}{0pt}%
\pgfpathmoveto{\pgfqpoint{3.119914in}{2.602334in}}%
\pgfpathlineto{\pgfqpoint{3.133278in}{2.591488in}}%
\pgfpathlineto{\pgfqpoint{3.146643in}{2.580801in}}%
\pgfpathlineto{\pgfqpoint{3.160007in}{2.570270in}}%
\pgfpathlineto{\pgfqpoint{3.173370in}{2.559895in}}%
\pgfpathlineto{\pgfqpoint{3.165238in}{2.554176in}}%
\pgfpathlineto{\pgfqpoint{3.157096in}{2.548549in}}%
\pgfpathlineto{\pgfqpoint{3.148947in}{2.543015in}}%
\pgfpathlineto{\pgfqpoint{3.140789in}{2.537576in}}%
\pgfpathlineto{\pgfqpoint{3.127404in}{2.548167in}}%
\pgfpathlineto{\pgfqpoint{3.114018in}{2.558913in}}%
\pgfpathlineto{\pgfqpoint{3.100631in}{2.569816in}}%
\pgfpathlineto{\pgfqpoint{3.087244in}{2.580877in}}%
\pgfpathlineto{\pgfqpoint{3.095424in}{2.586093in}}%
\pgfpathlineto{\pgfqpoint{3.103596in}{2.591409in}}%
\pgfpathlineto{\pgfqpoint{3.111759in}{2.596823in}}%
\pgfpathlineto{\pgfqpoint{3.119914in}{2.602334in}}%
\pgfpathclose%
\pgfusepath{fill}%
\end{pgfscope}%
\begin{pgfscope}%
\pgfpathrectangle{\pgfqpoint{1.150000in}{0.150000in}}{\pgfqpoint{5.700000in}{5.700000in}}%
\pgfusepath{clip}%
\pgfsetbuttcap%
\pgfsetroundjoin%
\definecolor{currentfill}{rgb}{0.281446,0.084320,0.407414}%
\pgfsetfillcolor{currentfill}%
\pgfsetfillopacity{0.700000}%
\pgfsetlinewidth{0.000000pt}%
\definecolor{currentstroke}{rgb}{0.000000,0.000000,0.000000}%
\pgfsetstrokecolor{currentstroke}%
\pgfsetdash{}{0pt}%
\pgfpathmoveto{\pgfqpoint{4.313859in}{2.502182in}}%
\pgfpathlineto{\pgfqpoint{4.327373in}{2.501204in}}%
\pgfpathlineto{\pgfqpoint{4.340895in}{2.500337in}}%
\pgfpathlineto{\pgfqpoint{4.354426in}{2.499578in}}%
\pgfpathlineto{\pgfqpoint{4.367964in}{2.498929in}}%
\pgfpathlineto{\pgfqpoint{4.360290in}{2.490051in}}%
\pgfpathlineto{\pgfqpoint{4.352610in}{2.481152in}}%
\pgfpathlineto{\pgfqpoint{4.344925in}{2.472233in}}%
\pgfpathlineto{\pgfqpoint{4.337235in}{2.463294in}}%
\pgfpathlineto{\pgfqpoint{4.323687in}{2.463991in}}%
\pgfpathlineto{\pgfqpoint{4.310147in}{2.464797in}}%
\pgfpathlineto{\pgfqpoint{4.296615in}{2.465712in}}%
\pgfpathlineto{\pgfqpoint{4.283091in}{2.466737in}}%
\pgfpathlineto{\pgfqpoint{4.290791in}{2.475622in}}%
\pgfpathlineto{\pgfqpoint{4.298486in}{2.484492in}}%
\pgfpathlineto{\pgfqpoint{4.306175in}{2.493345in}}%
\pgfpathlineto{\pgfqpoint{4.313859in}{2.502182in}}%
\pgfpathclose%
\pgfusepath{fill}%
\end{pgfscope}%
\begin{pgfscope}%
\pgfpathrectangle{\pgfqpoint{1.150000in}{0.150000in}}{\pgfqpoint{5.700000in}{5.700000in}}%
\pgfusepath{clip}%
\pgfsetbuttcap%
\pgfsetroundjoin%
\definecolor{currentfill}{rgb}{0.257322,0.256130,0.526563}%
\pgfsetfillcolor{currentfill}%
\pgfsetfillopacity{0.700000}%
\pgfsetlinewidth{0.000000pt}%
\definecolor{currentstroke}{rgb}{0.000000,0.000000,0.000000}%
\pgfsetstrokecolor{currentstroke}%
\pgfsetdash{}{0pt}%
\pgfpathmoveto{\pgfqpoint{5.130937in}{2.840243in}}%
\pgfpathlineto{\pgfqpoint{5.144748in}{2.843051in}}%
\pgfpathlineto{\pgfqpoint{5.158571in}{2.845960in}}%
\pgfpathlineto{\pgfqpoint{5.172406in}{2.848968in}}%
\pgfpathlineto{\pgfqpoint{5.186252in}{2.852077in}}%
\pgfpathlineto{\pgfqpoint{5.178889in}{2.844875in}}%
\pgfpathlineto{\pgfqpoint{5.171519in}{2.837634in}}%
\pgfpathlineto{\pgfqpoint{5.164144in}{2.830351in}}%
\pgfpathlineto{\pgfqpoint{5.156762in}{2.823025in}}%
\pgfpathlineto{\pgfqpoint{5.142903in}{2.819798in}}%
\pgfpathlineto{\pgfqpoint{5.129056in}{2.816671in}}%
\pgfpathlineto{\pgfqpoint{5.115221in}{2.813645in}}%
\pgfpathlineto{\pgfqpoint{5.101397in}{2.810719in}}%
\pgfpathlineto{\pgfqpoint{5.108791in}{2.818156in}}%
\pgfpathlineto{\pgfqpoint{5.116179in}{2.825555in}}%
\pgfpathlineto{\pgfqpoint{5.123561in}{2.832917in}}%
\pgfpathlineto{\pgfqpoint{5.130937in}{2.840243in}}%
\pgfpathclose%
\pgfusepath{fill}%
\end{pgfscope}%
\begin{pgfscope}%
\pgfpathrectangle{\pgfqpoint{1.150000in}{0.150000in}}{\pgfqpoint{5.700000in}{5.700000in}}%
\pgfusepath{clip}%
\pgfsetbuttcap%
\pgfsetroundjoin%
\definecolor{currentfill}{rgb}{0.272594,0.025563,0.353093}%
\pgfsetfillcolor{currentfill}%
\pgfsetfillopacity{0.700000}%
\pgfsetlinewidth{0.000000pt}%
\definecolor{currentstroke}{rgb}{0.000000,0.000000,0.000000}%
\pgfsetstrokecolor{currentstroke}%
\pgfsetdash{}{0pt}%
\pgfpathmoveto{\pgfqpoint{3.782095in}{2.407998in}}%
\pgfpathlineto{\pgfqpoint{3.795484in}{2.403361in}}%
\pgfpathlineto{\pgfqpoint{3.808879in}{2.398847in}}%
\pgfpathlineto{\pgfqpoint{3.822278in}{2.394455in}}%
\pgfpathlineto{\pgfqpoint{3.835682in}{2.390184in}}%
\pgfpathlineto{\pgfqpoint{3.827823in}{2.381891in}}%
\pgfpathlineto{\pgfqpoint{3.819959in}{2.373621in}}%
\pgfpathlineto{\pgfqpoint{3.812089in}{2.365376in}}%
\pgfpathlineto{\pgfqpoint{3.804213in}{2.357155in}}%
\pgfpathlineto{\pgfqpoint{3.790796in}{2.361564in}}%
\pgfpathlineto{\pgfqpoint{3.777384in}{2.366094in}}%
\pgfpathlineto{\pgfqpoint{3.763977in}{2.370747in}}%
\pgfpathlineto{\pgfqpoint{3.750574in}{2.375522in}}%
\pgfpathlineto{\pgfqpoint{3.758463in}{2.383598in}}%
\pgfpathlineto{\pgfqpoint{3.766346in}{2.391703in}}%
\pgfpathlineto{\pgfqpoint{3.774224in}{2.399837in}}%
\pgfpathlineto{\pgfqpoint{3.782095in}{2.407998in}}%
\pgfpathclose%
\pgfusepath{fill}%
\end{pgfscope}%
\begin{pgfscope}%
\pgfpathrectangle{\pgfqpoint{1.150000in}{0.150000in}}{\pgfqpoint{5.700000in}{5.700000in}}%
\pgfusepath{clip}%
\pgfsetbuttcap%
\pgfsetroundjoin%
\definecolor{currentfill}{rgb}{0.263663,0.237631,0.518762}%
\pgfsetfillcolor{currentfill}%
\pgfsetfillopacity{0.700000}%
\pgfsetlinewidth{0.000000pt}%
\definecolor{currentstroke}{rgb}{0.000000,0.000000,0.000000}%
\pgfsetstrokecolor{currentstroke}%
\pgfsetdash{}{0pt}%
\pgfpathmoveto{\pgfqpoint{5.046216in}{2.800023in}}%
\pgfpathlineto{\pgfqpoint{5.059995in}{2.802546in}}%
\pgfpathlineto{\pgfqpoint{5.073784in}{2.805169in}}%
\pgfpathlineto{\pgfqpoint{5.087585in}{2.807894in}}%
\pgfpathlineto{\pgfqpoint{5.101397in}{2.810719in}}%
\pgfpathlineto{\pgfqpoint{5.093997in}{2.803242in}}%
\pgfpathlineto{\pgfqpoint{5.086591in}{2.795724in}}%
\pgfpathlineto{\pgfqpoint{5.079179in}{2.788164in}}%
\pgfpathlineto{\pgfqpoint{5.071760in}{2.780559in}}%
\pgfpathlineto{\pgfqpoint{5.057936in}{2.777634in}}%
\pgfpathlineto{\pgfqpoint{5.044123in}{2.774810in}}%
\pgfpathlineto{\pgfqpoint{5.030322in}{2.772087in}}%
\pgfpathlineto{\pgfqpoint{5.016532in}{2.769465in}}%
\pgfpathlineto{\pgfqpoint{5.023962in}{2.777162in}}%
\pgfpathlineto{\pgfqpoint{5.031386in}{2.784820in}}%
\pgfpathlineto{\pgfqpoint{5.038804in}{2.792440in}}%
\pgfpathlineto{\pgfqpoint{5.046216in}{2.800023in}}%
\pgfpathclose%
\pgfusepath{fill}%
\end{pgfscope}%
\begin{pgfscope}%
\pgfpathrectangle{\pgfqpoint{1.150000in}{0.150000in}}{\pgfqpoint{5.700000in}{5.700000in}}%
\pgfusepath{clip}%
\pgfsetbuttcap%
\pgfsetroundjoin%
\definecolor{currentfill}{rgb}{0.269308,0.218818,0.509577}%
\pgfsetfillcolor{currentfill}%
\pgfsetfillopacity{0.700000}%
\pgfsetlinewidth{0.000000pt}%
\definecolor{currentstroke}{rgb}{0.000000,0.000000,0.000000}%
\pgfsetstrokecolor{currentstroke}%
\pgfsetdash{}{0pt}%
\pgfpathmoveto{\pgfqpoint{4.961482in}{2.759991in}}%
\pgfpathlineto{\pgfqpoint{4.975228in}{2.762207in}}%
\pgfpathlineto{\pgfqpoint{4.988985in}{2.764525in}}%
\pgfpathlineto{\pgfqpoint{5.002753in}{2.766945in}}%
\pgfpathlineto{\pgfqpoint{5.016532in}{2.769465in}}%
\pgfpathlineto{\pgfqpoint{5.009096in}{2.761728in}}%
\pgfpathlineto{\pgfqpoint{5.001654in}{2.753948in}}%
\pgfpathlineto{\pgfqpoint{4.994205in}{2.746125in}}%
\pgfpathlineto{\pgfqpoint{4.986751in}{2.738258in}}%
\pgfpathlineto{\pgfqpoint{4.972961in}{2.735657in}}%
\pgfpathlineto{\pgfqpoint{4.959182in}{2.733156in}}%
\pgfpathlineto{\pgfqpoint{4.945414in}{2.730758in}}%
\pgfpathlineto{\pgfqpoint{4.931657in}{2.728461in}}%
\pgfpathlineto{\pgfqpoint{4.939122in}{2.736402in}}%
\pgfpathlineto{\pgfqpoint{4.946581in}{2.744303in}}%
\pgfpathlineto{\pgfqpoint{4.954035in}{2.752166in}}%
\pgfpathlineto{\pgfqpoint{4.961482in}{2.759991in}}%
\pgfpathclose%
\pgfusepath{fill}%
\end{pgfscope}%
\begin{pgfscope}%
\pgfpathrectangle{\pgfqpoint{1.150000in}{0.150000in}}{\pgfqpoint{5.700000in}{5.700000in}}%
\pgfusepath{clip}%
\pgfsetbuttcap%
\pgfsetroundjoin%
\definecolor{currentfill}{rgb}{0.279566,0.067836,0.391917}%
\pgfsetfillcolor{currentfill}%
\pgfsetfillopacity{0.700000}%
\pgfsetlinewidth{0.000000pt}%
\definecolor{currentstroke}{rgb}{0.000000,0.000000,0.000000}%
\pgfsetstrokecolor{currentstroke}%
\pgfsetdash{}{0pt}%
\pgfpathmoveto{\pgfqpoint{3.365986in}{2.473211in}}%
\pgfpathlineto{\pgfqpoint{3.379341in}{2.464952in}}%
\pgfpathlineto{\pgfqpoint{3.392698in}{2.456833in}}%
\pgfpathlineto{\pgfqpoint{3.406056in}{2.448854in}}%
\pgfpathlineto{\pgfqpoint{3.419417in}{2.441014in}}%
\pgfpathlineto{\pgfqpoint{3.411393in}{2.434194in}}%
\pgfpathlineto{\pgfqpoint{3.403362in}{2.427439in}}%
\pgfpathlineto{\pgfqpoint{3.395323in}{2.420752in}}%
\pgfpathlineto{\pgfqpoint{3.387278in}{2.414135in}}%
\pgfpathlineto{\pgfqpoint{3.373899in}{2.422169in}}%
\pgfpathlineto{\pgfqpoint{3.360522in}{2.430343in}}%
\pgfpathlineto{\pgfqpoint{3.347146in}{2.438656in}}%
\pgfpathlineto{\pgfqpoint{3.333772in}{2.447109in}}%
\pgfpathlineto{\pgfqpoint{3.341837in}{2.453525in}}%
\pgfpathlineto{\pgfqpoint{3.349894in}{2.460015in}}%
\pgfpathlineto{\pgfqpoint{3.357943in}{2.466578in}}%
\pgfpathlineto{\pgfqpoint{3.365986in}{2.473211in}}%
\pgfpathclose%
\pgfusepath{fill}%
\end{pgfscope}%
\begin{pgfscope}%
\pgfpathrectangle{\pgfqpoint{1.150000in}{0.150000in}}{\pgfqpoint{5.700000in}{5.700000in}}%
\pgfusepath{clip}%
\pgfsetbuttcap%
\pgfsetroundjoin%
\definecolor{currentfill}{rgb}{0.279566,0.067836,0.391917}%
\pgfsetfillcolor{currentfill}%
\pgfsetfillopacity{0.700000}%
\pgfsetlinewidth{0.000000pt}%
\definecolor{currentstroke}{rgb}{0.000000,0.000000,0.000000}%
\pgfsetstrokecolor{currentstroke}%
\pgfsetdash{}{0pt}%
\pgfpathmoveto{\pgfqpoint{4.229072in}{2.471941in}}%
\pgfpathlineto{\pgfqpoint{4.242566in}{2.470474in}}%
\pgfpathlineto{\pgfqpoint{4.256066in}{2.469118in}}%
\pgfpathlineto{\pgfqpoint{4.269575in}{2.467873in}}%
\pgfpathlineto{\pgfqpoint{4.283091in}{2.466737in}}%
\pgfpathlineto{\pgfqpoint{4.275385in}{2.457837in}}%
\pgfpathlineto{\pgfqpoint{4.267675in}{2.448921in}}%
\pgfpathlineto{\pgfqpoint{4.259959in}{2.439990in}}%
\pgfpathlineto{\pgfqpoint{4.252237in}{2.431045in}}%
\pgfpathlineto{\pgfqpoint{4.238711in}{2.432245in}}%
\pgfpathlineto{\pgfqpoint{4.225193in}{2.433556in}}%
\pgfpathlineto{\pgfqpoint{4.211682in}{2.434978in}}%
\pgfpathlineto{\pgfqpoint{4.198179in}{2.436511in}}%
\pgfpathlineto{\pgfqpoint{4.205910in}{2.445384in}}%
\pgfpathlineto{\pgfqpoint{4.213636in}{2.454247in}}%
\pgfpathlineto{\pgfqpoint{4.221357in}{2.463100in}}%
\pgfpathlineto{\pgfqpoint{4.229072in}{2.471941in}}%
\pgfpathclose%
\pgfusepath{fill}%
\end{pgfscope}%
\begin{pgfscope}%
\pgfpathrectangle{\pgfqpoint{1.150000in}{0.150000in}}{\pgfqpoint{5.700000in}{5.700000in}}%
\pgfusepath{clip}%
\pgfsetbuttcap%
\pgfsetroundjoin%
\definecolor{currentfill}{rgb}{0.274128,0.199721,0.498911}%
\pgfsetfillcolor{currentfill}%
\pgfsetfillopacity{0.700000}%
\pgfsetlinewidth{0.000000pt}%
\definecolor{currentstroke}{rgb}{0.000000,0.000000,0.000000}%
\pgfsetstrokecolor{currentstroke}%
\pgfsetdash{}{0pt}%
\pgfpathmoveto{\pgfqpoint{4.876736in}{2.720294in}}%
\pgfpathlineto{\pgfqpoint{4.890450in}{2.722182in}}%
\pgfpathlineto{\pgfqpoint{4.904175in}{2.724173in}}%
\pgfpathlineto{\pgfqpoint{4.917910in}{2.726266in}}%
\pgfpathlineto{\pgfqpoint{4.931657in}{2.728461in}}%
\pgfpathlineto{\pgfqpoint{4.924186in}{2.720479in}}%
\pgfpathlineto{\pgfqpoint{4.916709in}{2.712456in}}%
\pgfpathlineto{\pgfqpoint{4.909226in}{2.704390in}}%
\pgfpathlineto{\pgfqpoint{4.901737in}{2.696280in}}%
\pgfpathlineto{\pgfqpoint{4.887980in}{2.694023in}}%
\pgfpathlineto{\pgfqpoint{4.874234in}{2.691867in}}%
\pgfpathlineto{\pgfqpoint{4.860498in}{2.689814in}}%
\pgfpathlineto{\pgfqpoint{4.846773in}{2.687864in}}%
\pgfpathlineto{\pgfqpoint{4.854273in}{2.696029in}}%
\pgfpathlineto{\pgfqpoint{4.861766in}{2.704155in}}%
\pgfpathlineto{\pgfqpoint{4.869254in}{2.712243in}}%
\pgfpathlineto{\pgfqpoint{4.876736in}{2.720294in}}%
\pgfpathclose%
\pgfusepath{fill}%
\end{pgfscope}%
\begin{pgfscope}%
\pgfpathrectangle{\pgfqpoint{1.150000in}{0.150000in}}{\pgfqpoint{5.700000in}{5.700000in}}%
\pgfusepath{clip}%
\pgfsetbuttcap%
\pgfsetroundjoin%
\definecolor{currentfill}{rgb}{0.278012,0.180367,0.486697}%
\pgfsetfillcolor{currentfill}%
\pgfsetfillopacity{0.700000}%
\pgfsetlinewidth{0.000000pt}%
\definecolor{currentstroke}{rgb}{0.000000,0.000000,0.000000}%
\pgfsetstrokecolor{currentstroke}%
\pgfsetdash{}{0pt}%
\pgfpathmoveto{\pgfqpoint{4.791978in}{2.681090in}}%
\pgfpathlineto{\pgfqpoint{4.805661in}{2.682629in}}%
\pgfpathlineto{\pgfqpoint{4.819355in}{2.684271in}}%
\pgfpathlineto{\pgfqpoint{4.833059in}{2.686016in}}%
\pgfpathlineto{\pgfqpoint{4.846773in}{2.687864in}}%
\pgfpathlineto{\pgfqpoint{4.839268in}{2.679659in}}%
\pgfpathlineto{\pgfqpoint{4.831757in}{2.671412in}}%
\pgfpathlineto{\pgfqpoint{4.824240in}{2.663125in}}%
\pgfpathlineto{\pgfqpoint{4.816718in}{2.654794in}}%
\pgfpathlineto{\pgfqpoint{4.802993in}{2.652903in}}%
\pgfpathlineto{\pgfqpoint{4.789279in}{2.651114in}}%
\pgfpathlineto{\pgfqpoint{4.775575in}{2.649428in}}%
\pgfpathlineto{\pgfqpoint{4.761881in}{2.647845in}}%
\pgfpathlineto{\pgfqpoint{4.769414in}{2.656213in}}%
\pgfpathlineto{\pgfqpoint{4.776941in}{2.664542in}}%
\pgfpathlineto{\pgfqpoint{4.784462in}{2.672834in}}%
\pgfpathlineto{\pgfqpoint{4.791978in}{2.681090in}}%
\pgfpathclose%
\pgfusepath{fill}%
\end{pgfscope}%
\begin{pgfscope}%
\pgfpathrectangle{\pgfqpoint{1.150000in}{0.150000in}}{\pgfqpoint{5.700000in}{5.700000in}}%
\pgfusepath{clip}%
\pgfsetbuttcap%
\pgfsetroundjoin%
\definecolor{currentfill}{rgb}{0.283091,0.110553,0.431554}%
\pgfsetfillcolor{currentfill}%
\pgfsetfillopacity{0.700000}%
\pgfsetlinewidth{0.000000pt}%
\definecolor{currentstroke}{rgb}{0.000000,0.000000,0.000000}%
\pgfsetstrokecolor{currentstroke}%
\pgfsetdash{}{0pt}%
\pgfpathmoveto{\pgfqpoint{3.173370in}{2.559895in}}%
\pgfpathlineto{\pgfqpoint{3.186734in}{2.549674in}}%
\pgfpathlineto{\pgfqpoint{3.200098in}{2.539606in}}%
\pgfpathlineto{\pgfqpoint{3.213462in}{2.529690in}}%
\pgfpathlineto{\pgfqpoint{3.226827in}{2.519925in}}%
\pgfpathlineto{\pgfqpoint{3.218715in}{2.513998in}}%
\pgfpathlineto{\pgfqpoint{3.210595in}{2.508158in}}%
\pgfpathlineto{\pgfqpoint{3.202468in}{2.502407in}}%
\pgfpathlineto{\pgfqpoint{3.194332in}{2.496747in}}%
\pgfpathlineto{\pgfqpoint{3.180946in}{2.506726in}}%
\pgfpathlineto{\pgfqpoint{3.167560in}{2.516857in}}%
\pgfpathlineto{\pgfqpoint{3.154175in}{2.527140in}}%
\pgfpathlineto{\pgfqpoint{3.140789in}{2.537576in}}%
\pgfpathlineto{\pgfqpoint{3.148947in}{2.543015in}}%
\pgfpathlineto{\pgfqpoint{3.157096in}{2.548549in}}%
\pgfpathlineto{\pgfqpoint{3.165238in}{2.554176in}}%
\pgfpathlineto{\pgfqpoint{3.173370in}{2.559895in}}%
\pgfpathclose%
\pgfusepath{fill}%
\end{pgfscope}%
\begin{pgfscope}%
\pgfpathrectangle{\pgfqpoint{1.150000in}{0.150000in}}{\pgfqpoint{5.700000in}{5.700000in}}%
\pgfusepath{clip}%
\pgfsetbuttcap%
\pgfsetroundjoin%
\definecolor{currentfill}{rgb}{0.273809,0.031497,0.358853}%
\pgfsetfillcolor{currentfill}%
\pgfsetfillopacity{0.700000}%
\pgfsetlinewidth{0.000000pt}%
\definecolor{currentstroke}{rgb}{0.000000,0.000000,0.000000}%
\pgfsetstrokecolor{currentstroke}%
\pgfsetdash{}{0pt}%
\pgfpathmoveto{\pgfqpoint{3.920680in}{2.408159in}}%
\pgfpathlineto{\pgfqpoint{3.934099in}{2.404606in}}%
\pgfpathlineto{\pgfqpoint{3.947524in}{2.401172in}}%
\pgfpathlineto{\pgfqpoint{3.960955in}{2.397855in}}%
\pgfpathlineto{\pgfqpoint{3.974392in}{2.394655in}}%
\pgfpathlineto{\pgfqpoint{3.966580in}{2.386056in}}%
\pgfpathlineto{\pgfqpoint{3.958763in}{2.377466in}}%
\pgfpathlineto{\pgfqpoint{3.950940in}{2.368887in}}%
\pgfpathlineto{\pgfqpoint{3.943111in}{2.360319in}}%
\pgfpathlineto{\pgfqpoint{3.929663in}{2.363639in}}%
\pgfpathlineto{\pgfqpoint{3.916220in}{2.367075in}}%
\pgfpathlineto{\pgfqpoint{3.902783in}{2.370630in}}%
\pgfpathlineto{\pgfqpoint{3.889352in}{2.374302in}}%
\pgfpathlineto{\pgfqpoint{3.897192in}{2.382743in}}%
\pgfpathlineto{\pgfqpoint{3.905027in}{2.391200in}}%
\pgfpathlineto{\pgfqpoint{3.912856in}{2.399672in}}%
\pgfpathlineto{\pgfqpoint{3.920680in}{2.408159in}}%
\pgfpathclose%
\pgfusepath{fill}%
\end{pgfscope}%
\begin{pgfscope}%
\pgfpathrectangle{\pgfqpoint{1.150000in}{0.150000in}}{\pgfqpoint{5.700000in}{5.700000in}}%
\pgfusepath{clip}%
\pgfsetbuttcap%
\pgfsetroundjoin%
\definecolor{currentfill}{rgb}{0.250425,0.274290,0.533103}%
\pgfsetfillcolor{currentfill}%
\pgfsetfillopacity{0.700000}%
\pgfsetlinewidth{0.000000pt}%
\definecolor{currentstroke}{rgb}{0.000000,0.000000,0.000000}%
\pgfsetstrokecolor{currentstroke}%
\pgfsetdash{}{0pt}%
\pgfpathmoveto{\pgfqpoint{2.765360in}{2.897314in}}%
\pgfpathlineto{\pgfqpoint{2.778810in}{2.882036in}}%
\pgfpathlineto{\pgfqpoint{2.792255in}{2.866952in}}%
\pgfpathlineto{\pgfqpoint{2.805695in}{2.852060in}}%
\pgfpathlineto{\pgfqpoint{2.819132in}{2.837359in}}%
\pgfpathlineto{\pgfqpoint{2.810816in}{2.833398in}}%
\pgfpathlineto{\pgfqpoint{2.802489in}{2.829564in}}%
\pgfpathlineto{\pgfqpoint{2.794152in}{2.825860in}}%
\pgfpathlineto{\pgfqpoint{2.785805in}{2.822287in}}%
\pgfpathlineto{\pgfqpoint{2.772340in}{2.837230in}}%
\pgfpathlineto{\pgfqpoint{2.758871in}{2.852363in}}%
\pgfpathlineto{\pgfqpoint{2.745397in}{2.867689in}}%
\pgfpathlineto{\pgfqpoint{2.731919in}{2.883210in}}%
\pgfpathlineto{\pgfqpoint{2.740296in}{2.886533in}}%
\pgfpathlineto{\pgfqpoint{2.748661in}{2.889993in}}%
\pgfpathlineto{\pgfqpoint{2.757016in}{2.893587in}}%
\pgfpathlineto{\pgfqpoint{2.765360in}{2.897314in}}%
\pgfpathclose%
\pgfusepath{fill}%
\end{pgfscope}%
\begin{pgfscope}%
\pgfpathrectangle{\pgfqpoint{1.150000in}{0.150000in}}{\pgfqpoint{5.700000in}{5.700000in}}%
\pgfusepath{clip}%
\pgfsetbuttcap%
\pgfsetroundjoin%
\definecolor{currentfill}{rgb}{0.280255,0.165693,0.476498}%
\pgfsetfillcolor{currentfill}%
\pgfsetfillopacity{0.700000}%
\pgfsetlinewidth{0.000000pt}%
\definecolor{currentstroke}{rgb}{0.000000,0.000000,0.000000}%
\pgfsetstrokecolor{currentstroke}%
\pgfsetdash{}{0pt}%
\pgfpathmoveto{\pgfqpoint{4.707208in}{2.642552in}}%
\pgfpathlineto{\pgfqpoint{4.720861in}{2.643720in}}%
\pgfpathlineto{\pgfqpoint{4.734525in}{2.644991in}}%
\pgfpathlineto{\pgfqpoint{4.748198in}{2.646366in}}%
\pgfpathlineto{\pgfqpoint{4.761881in}{2.647845in}}%
\pgfpathlineto{\pgfqpoint{4.754343in}{2.639440in}}%
\pgfpathlineto{\pgfqpoint{4.746799in}{2.630994in}}%
\pgfpathlineto{\pgfqpoint{4.739249in}{2.622509in}}%
\pgfpathlineto{\pgfqpoint{4.731694in}{2.613984in}}%
\pgfpathlineto{\pgfqpoint{4.718001in}{2.612479in}}%
\pgfpathlineto{\pgfqpoint{4.704317in}{2.611078in}}%
\pgfpathlineto{\pgfqpoint{4.690644in}{2.609782in}}%
\pgfpathlineto{\pgfqpoint{4.676981in}{2.608589in}}%
\pgfpathlineto{\pgfqpoint{4.684546in}{2.617133in}}%
\pgfpathlineto{\pgfqpoint{4.692105in}{2.625641in}}%
\pgfpathlineto{\pgfqpoint{4.699659in}{2.634114in}}%
\pgfpathlineto{\pgfqpoint{4.707208in}{2.642552in}}%
\pgfpathclose%
\pgfusepath{fill}%
\end{pgfscope}%
\begin{pgfscope}%
\pgfpathrectangle{\pgfqpoint{1.150000in}{0.150000in}}{\pgfqpoint{5.700000in}{5.700000in}}%
\pgfusepath{clip}%
\pgfsetbuttcap%
\pgfsetroundjoin%
\definecolor{currentfill}{rgb}{0.260571,0.246922,0.522828}%
\pgfsetfillcolor{currentfill}%
\pgfsetfillopacity{0.700000}%
\pgfsetlinewidth{0.000000pt}%
\definecolor{currentstroke}{rgb}{0.000000,0.000000,0.000000}%
\pgfsetstrokecolor{currentstroke}%
\pgfsetdash{}{0pt}%
\pgfpathmoveto{\pgfqpoint{2.819132in}{2.837359in}}%
\pgfpathlineto{\pgfqpoint{2.832564in}{2.822847in}}%
\pgfpathlineto{\pgfqpoint{2.845993in}{2.808522in}}%
\pgfpathlineto{\pgfqpoint{2.859418in}{2.794383in}}%
\pgfpathlineto{\pgfqpoint{2.872839in}{2.780428in}}%
\pgfpathlineto{\pgfqpoint{2.864550in}{2.776233in}}%
\pgfpathlineto{\pgfqpoint{2.856251in}{2.772161in}}%
\pgfpathlineto{\pgfqpoint{2.847942in}{2.768214in}}%
\pgfpathlineto{\pgfqpoint{2.839622in}{2.764395in}}%
\pgfpathlineto{\pgfqpoint{2.826174in}{2.778590in}}%
\pgfpathlineto{\pgfqpoint{2.812721in}{2.792969in}}%
\pgfpathlineto{\pgfqpoint{2.799265in}{2.807535in}}%
\pgfpathlineto{\pgfqpoint{2.785805in}{2.822287in}}%
\pgfpathlineto{\pgfqpoint{2.794152in}{2.825860in}}%
\pgfpathlineto{\pgfqpoint{2.802489in}{2.829564in}}%
\pgfpathlineto{\pgfqpoint{2.810816in}{2.833398in}}%
\pgfpathlineto{\pgfqpoint{2.819132in}{2.837359in}}%
\pgfpathclose%
\pgfusepath{fill}%
\end{pgfscope}%
\begin{pgfscope}%
\pgfpathrectangle{\pgfqpoint{1.150000in}{0.150000in}}{\pgfqpoint{5.700000in}{5.700000in}}%
\pgfusepath{clip}%
\pgfsetbuttcap%
\pgfsetroundjoin%
\definecolor{currentfill}{rgb}{0.277941,0.056324,0.381191}%
\pgfsetfillcolor{currentfill}%
\pgfsetfillopacity{0.700000}%
\pgfsetlinewidth{0.000000pt}%
\definecolor{currentstroke}{rgb}{0.000000,0.000000,0.000000}%
\pgfsetstrokecolor{currentstroke}%
\pgfsetdash{}{0pt}%
\pgfpathmoveto{\pgfqpoint{4.144239in}{2.443761in}}%
\pgfpathlineto{\pgfqpoint{4.157714in}{2.441780in}}%
\pgfpathlineto{\pgfqpoint{4.171195in}{2.439912in}}%
\pgfpathlineto{\pgfqpoint{4.184683in}{2.438155in}}%
\pgfpathlineto{\pgfqpoint{4.198179in}{2.436511in}}%
\pgfpathlineto{\pgfqpoint{4.190443in}{2.427628in}}%
\pgfpathlineto{\pgfqpoint{4.182701in}{2.418737in}}%
\pgfpathlineto{\pgfqpoint{4.174953in}{2.409836in}}%
\pgfpathlineto{\pgfqpoint{4.167201in}{2.400928in}}%
\pgfpathlineto{\pgfqpoint{4.153695in}{2.402656in}}%
\pgfpathlineto{\pgfqpoint{4.140196in}{2.404495in}}%
\pgfpathlineto{\pgfqpoint{4.126704in}{2.406448in}}%
\pgfpathlineto{\pgfqpoint{4.113220in}{2.408512in}}%
\pgfpathlineto{\pgfqpoint{4.120983in}{2.417330in}}%
\pgfpathlineto{\pgfqpoint{4.128740in}{2.426145in}}%
\pgfpathlineto{\pgfqpoint{4.136492in}{2.434955in}}%
\pgfpathlineto{\pgfqpoint{4.144239in}{2.443761in}}%
\pgfpathclose%
\pgfusepath{fill}%
\end{pgfscope}%
\begin{pgfscope}%
\pgfpathrectangle{\pgfqpoint{1.150000in}{0.150000in}}{\pgfqpoint{5.700000in}{5.700000in}}%
\pgfusepath{clip}%
\pgfsetbuttcap%
\pgfsetroundjoin%
\definecolor{currentfill}{rgb}{0.282290,0.145912,0.461510}%
\pgfsetfillcolor{currentfill}%
\pgfsetfillopacity{0.700000}%
\pgfsetlinewidth{0.000000pt}%
\definecolor{currentstroke}{rgb}{0.000000,0.000000,0.000000}%
\pgfsetstrokecolor{currentstroke}%
\pgfsetdash{}{0pt}%
\pgfpathmoveto{\pgfqpoint{4.622424in}{2.604863in}}%
\pgfpathlineto{\pgfqpoint{4.636049in}{2.605637in}}%
\pgfpathlineto{\pgfqpoint{4.649683in}{2.606516in}}%
\pgfpathlineto{\pgfqpoint{4.663327in}{2.607500in}}%
\pgfpathlineto{\pgfqpoint{4.676981in}{2.608589in}}%
\pgfpathlineto{\pgfqpoint{4.669410in}{2.600008in}}%
\pgfpathlineto{\pgfqpoint{4.661834in}{2.591391in}}%
\pgfpathlineto{\pgfqpoint{4.654252in}{2.582736in}}%
\pgfpathlineto{\pgfqpoint{4.646664in}{2.574044in}}%
\pgfpathlineto{\pgfqpoint{4.633001in}{2.572948in}}%
\pgfpathlineto{\pgfqpoint{4.619347in}{2.571957in}}%
\pgfpathlineto{\pgfqpoint{4.605704in}{2.571071in}}%
\pgfpathlineto{\pgfqpoint{4.592069in}{2.570290in}}%
\pgfpathlineto{\pgfqpoint{4.599666in}{2.578983in}}%
\pgfpathlineto{\pgfqpoint{4.607258in}{2.587642in}}%
\pgfpathlineto{\pgfqpoint{4.614844in}{2.596269in}}%
\pgfpathlineto{\pgfqpoint{4.622424in}{2.604863in}}%
\pgfpathclose%
\pgfusepath{fill}%
\end{pgfscope}%
\begin{pgfscope}%
\pgfpathrectangle{\pgfqpoint{1.150000in}{0.150000in}}{\pgfqpoint{5.700000in}{5.700000in}}%
\pgfusepath{clip}%
\pgfsetbuttcap%
\pgfsetroundjoin%
\definecolor{currentfill}{rgb}{0.267968,0.223549,0.512008}%
\pgfsetfillcolor{currentfill}%
\pgfsetfillopacity{0.700000}%
\pgfsetlinewidth{0.000000pt}%
\definecolor{currentstroke}{rgb}{0.000000,0.000000,0.000000}%
\pgfsetstrokecolor{currentstroke}%
\pgfsetdash{}{0pt}%
\pgfpathmoveto{\pgfqpoint{2.872839in}{2.780428in}}%
\pgfpathlineto{\pgfqpoint{2.886257in}{2.766654in}}%
\pgfpathlineto{\pgfqpoint{2.899672in}{2.753062in}}%
\pgfpathlineto{\pgfqpoint{2.913084in}{2.739648in}}%
\pgfpathlineto{\pgfqpoint{2.926493in}{2.726411in}}%
\pgfpathlineto{\pgfqpoint{2.918230in}{2.721985in}}%
\pgfpathlineto{\pgfqpoint{2.909958in}{2.717677in}}%
\pgfpathlineto{\pgfqpoint{2.901676in}{2.713489in}}%
\pgfpathlineto{\pgfqpoint{2.893384in}{2.709423in}}%
\pgfpathlineto{\pgfqpoint{2.879948in}{2.722898in}}%
\pgfpathlineto{\pgfqpoint{2.866509in}{2.736550in}}%
\pgfpathlineto{\pgfqpoint{2.853068in}{2.750382in}}%
\pgfpathlineto{\pgfqpoint{2.839622in}{2.764395in}}%
\pgfpathlineto{\pgfqpoint{2.847942in}{2.768214in}}%
\pgfpathlineto{\pgfqpoint{2.856251in}{2.772161in}}%
\pgfpathlineto{\pgfqpoint{2.864550in}{2.776233in}}%
\pgfpathlineto{\pgfqpoint{2.872839in}{2.780428in}}%
\pgfpathclose%
\pgfusepath{fill}%
\end{pgfscope}%
\begin{pgfscope}%
\pgfpathrectangle{\pgfqpoint{1.150000in}{0.150000in}}{\pgfqpoint{5.700000in}{5.700000in}}%
\pgfusepath{clip}%
\pgfsetbuttcap%
\pgfsetroundjoin%
\definecolor{currentfill}{rgb}{0.274952,0.037752,0.364543}%
\pgfsetfillcolor{currentfill}%
\pgfsetfillopacity{0.700000}%
\pgfsetlinewidth{0.000000pt}%
\definecolor{currentstroke}{rgb}{0.000000,0.000000,0.000000}%
\pgfsetstrokecolor{currentstroke}%
\pgfsetdash{}{0pt}%
\pgfpathmoveto{\pgfqpoint{3.558288in}{2.412506in}}%
\pgfpathlineto{\pgfqpoint{3.571656in}{2.406053in}}%
\pgfpathlineto{\pgfqpoint{3.585028in}{2.399731in}}%
\pgfpathlineto{\pgfqpoint{3.598404in}{2.393538in}}%
\pgfpathlineto{\pgfqpoint{3.611783in}{2.387475in}}%
\pgfpathlineto{\pgfqpoint{3.603835in}{2.379900in}}%
\pgfpathlineto{\pgfqpoint{3.595881in}{2.372372in}}%
\pgfpathlineto{\pgfqpoint{3.587920in}{2.364891in}}%
\pgfpathlineto{\pgfqpoint{3.579953in}{2.357460in}}%
\pgfpathlineto{\pgfqpoint{3.566558in}{2.363698in}}%
\pgfpathlineto{\pgfqpoint{3.553166in}{2.370065in}}%
\pgfpathlineto{\pgfqpoint{3.539778in}{2.376563in}}%
\pgfpathlineto{\pgfqpoint{3.526394in}{2.383191in}}%
\pgfpathlineto{\pgfqpoint{3.534377in}{2.390440in}}%
\pgfpathlineto{\pgfqpoint{3.542354in}{2.397744in}}%
\pgfpathlineto{\pgfqpoint{3.550324in}{2.405100in}}%
\pgfpathlineto{\pgfqpoint{3.558288in}{2.412506in}}%
\pgfpathclose%
\pgfusepath{fill}%
\end{pgfscope}%
\begin{pgfscope}%
\pgfpathrectangle{\pgfqpoint{1.150000in}{0.150000in}}{\pgfqpoint{5.700000in}{5.700000in}}%
\pgfusepath{clip}%
\pgfsetbuttcap%
\pgfsetroundjoin%
\definecolor{currentfill}{rgb}{0.272594,0.025563,0.353093}%
\pgfsetfillcolor{currentfill}%
\pgfsetfillopacity{0.700000}%
\pgfsetlinewidth{0.000000pt}%
\definecolor{currentstroke}{rgb}{0.000000,0.000000,0.000000}%
\pgfsetstrokecolor{currentstroke}%
\pgfsetdash{}{0pt}%
\pgfpathmoveto{\pgfqpoint{3.697010in}{2.395860in}}%
\pgfpathlineto{\pgfqpoint{3.710394in}{2.390588in}}%
\pgfpathlineto{\pgfqpoint{3.723783in}{2.385442in}}%
\pgfpathlineto{\pgfqpoint{3.737176in}{2.380420in}}%
\pgfpathlineto{\pgfqpoint{3.750574in}{2.375522in}}%
\pgfpathlineto{\pgfqpoint{3.742679in}{2.367476in}}%
\pgfpathlineto{\pgfqpoint{3.734779in}{2.359462in}}%
\pgfpathlineto{\pgfqpoint{3.726872in}{2.351482in}}%
\pgfpathlineto{\pgfqpoint{3.718959in}{2.343536in}}%
\pgfpathlineto{\pgfqpoint{3.705547in}{2.348591in}}%
\pgfpathlineto{\pgfqpoint{3.692140in}{2.353769in}}%
\pgfpathlineto{\pgfqpoint{3.678737in}{2.359072in}}%
\pgfpathlineto{\pgfqpoint{3.665338in}{2.364499in}}%
\pgfpathlineto{\pgfqpoint{3.673265in}{2.372282in}}%
\pgfpathlineto{\pgfqpoint{3.681186in}{2.380104in}}%
\pgfpathlineto{\pgfqpoint{3.689101in}{2.387964in}}%
\pgfpathlineto{\pgfqpoint{3.697010in}{2.395860in}}%
\pgfpathclose%
\pgfusepath{fill}%
\end{pgfscope}%
\begin{pgfscope}%
\pgfpathrectangle{\pgfqpoint{1.150000in}{0.150000in}}{\pgfqpoint{5.700000in}{5.700000in}}%
\pgfusepath{clip}%
\pgfsetbuttcap%
\pgfsetroundjoin%
\definecolor{currentfill}{rgb}{0.283187,0.125848,0.444960}%
\pgfsetfillcolor{currentfill}%
\pgfsetfillopacity{0.700000}%
\pgfsetlinewidth{0.000000pt}%
\definecolor{currentstroke}{rgb}{0.000000,0.000000,0.000000}%
\pgfsetstrokecolor{currentstroke}%
\pgfsetdash{}{0pt}%
\pgfpathmoveto{\pgfqpoint{4.537625in}{2.568221in}}%
\pgfpathlineto{\pgfqpoint{4.551222in}{2.568580in}}%
\pgfpathlineto{\pgfqpoint{4.564829in}{2.569044in}}%
\pgfpathlineto{\pgfqpoint{4.578444in}{2.569614in}}%
\pgfpathlineto{\pgfqpoint{4.592069in}{2.570290in}}%
\pgfpathlineto{\pgfqpoint{4.584467in}{2.561564in}}%
\pgfpathlineto{\pgfqpoint{4.576859in}{2.552804in}}%
\pgfpathlineto{\pgfqpoint{4.569245in}{2.544010in}}%
\pgfpathlineto{\pgfqpoint{4.561626in}{2.535183in}}%
\pgfpathlineto{\pgfqpoint{4.547992in}{2.534518in}}%
\pgfpathlineto{\pgfqpoint{4.534367in}{2.533959in}}%
\pgfpathlineto{\pgfqpoint{4.520751in}{2.533505in}}%
\pgfpathlineto{\pgfqpoint{4.507144in}{2.533158in}}%
\pgfpathlineto{\pgfqpoint{4.514773in}{2.541968in}}%
\pgfpathlineto{\pgfqpoint{4.522396in}{2.550748in}}%
\pgfpathlineto{\pgfqpoint{4.530013in}{2.559499in}}%
\pgfpathlineto{\pgfqpoint{4.537625in}{2.568221in}}%
\pgfpathclose%
\pgfusepath{fill}%
\end{pgfscope}%
\begin{pgfscope}%
\pgfpathrectangle{\pgfqpoint{1.150000in}{0.150000in}}{\pgfqpoint{5.700000in}{5.700000in}}%
\pgfusepath{clip}%
\pgfsetbuttcap%
\pgfsetroundjoin%
\definecolor{currentfill}{rgb}{0.274128,0.199721,0.498911}%
\pgfsetfillcolor{currentfill}%
\pgfsetfillopacity{0.700000}%
\pgfsetlinewidth{0.000000pt}%
\definecolor{currentstroke}{rgb}{0.000000,0.000000,0.000000}%
\pgfsetstrokecolor{currentstroke}%
\pgfsetdash{}{0pt}%
\pgfpathmoveto{\pgfqpoint{2.926493in}{2.726411in}}%
\pgfpathlineto{\pgfqpoint{2.939900in}{2.713351in}}%
\pgfpathlineto{\pgfqpoint{2.953304in}{2.700465in}}%
\pgfpathlineto{\pgfqpoint{2.966706in}{2.687752in}}%
\pgfpathlineto{\pgfqpoint{2.980105in}{2.675210in}}%
\pgfpathlineto{\pgfqpoint{2.971868in}{2.670553in}}%
\pgfpathlineto{\pgfqpoint{2.963621in}{2.666010in}}%
\pgfpathlineto{\pgfqpoint{2.955365in}{2.661582in}}%
\pgfpathlineto{\pgfqpoint{2.947099in}{2.657272in}}%
\pgfpathlineto{\pgfqpoint{2.933674in}{2.670050in}}%
\pgfpathlineto{\pgfqpoint{2.920246in}{2.683001in}}%
\pgfpathlineto{\pgfqpoint{2.906816in}{2.696125in}}%
\pgfpathlineto{\pgfqpoint{2.893384in}{2.709423in}}%
\pgfpathlineto{\pgfqpoint{2.901676in}{2.713489in}}%
\pgfpathlineto{\pgfqpoint{2.909958in}{2.717677in}}%
\pgfpathlineto{\pgfqpoint{2.918230in}{2.721985in}}%
\pgfpathlineto{\pgfqpoint{2.926493in}{2.726411in}}%
\pgfpathclose%
\pgfusepath{fill}%
\end{pgfscope}%
\begin{pgfscope}%
\pgfpathrectangle{\pgfqpoint{1.150000in}{0.150000in}}{\pgfqpoint{5.700000in}{5.700000in}}%
\pgfusepath{clip}%
\pgfsetbuttcap%
\pgfsetroundjoin%
\definecolor{currentfill}{rgb}{0.282327,0.094955,0.417331}%
\pgfsetfillcolor{currentfill}%
\pgfsetfillopacity{0.700000}%
\pgfsetlinewidth{0.000000pt}%
\definecolor{currentstroke}{rgb}{0.000000,0.000000,0.000000}%
\pgfsetstrokecolor{currentstroke}%
\pgfsetdash{}{0pt}%
\pgfpathmoveto{\pgfqpoint{3.226827in}{2.519925in}}%
\pgfpathlineto{\pgfqpoint{3.240192in}{2.510309in}}%
\pgfpathlineto{\pgfqpoint{3.253557in}{2.500843in}}%
\pgfpathlineto{\pgfqpoint{3.266924in}{2.491524in}}%
\pgfpathlineto{\pgfqpoint{3.280291in}{2.482352in}}%
\pgfpathlineto{\pgfqpoint{3.272200in}{2.476218in}}%
\pgfpathlineto{\pgfqpoint{3.264101in}{2.470167in}}%
\pgfpathlineto{\pgfqpoint{3.255994in}{2.464199in}}%
\pgfpathlineto{\pgfqpoint{3.247879in}{2.458318in}}%
\pgfpathlineto{\pgfqpoint{3.234491in}{2.467704in}}%
\pgfpathlineto{\pgfqpoint{3.221104in}{2.477237in}}%
\pgfpathlineto{\pgfqpoint{3.207718in}{2.486917in}}%
\pgfpathlineto{\pgfqpoint{3.194332in}{2.496747in}}%
\pgfpathlineto{\pgfqpoint{3.202468in}{2.502407in}}%
\pgfpathlineto{\pgfqpoint{3.210595in}{2.508158in}}%
\pgfpathlineto{\pgfqpoint{3.218715in}{2.513998in}}%
\pgfpathlineto{\pgfqpoint{3.226827in}{2.519925in}}%
\pgfpathclose%
\pgfusepath{fill}%
\end{pgfscope}%
\begin{pgfscope}%
\pgfpathrectangle{\pgfqpoint{1.150000in}{0.150000in}}{\pgfqpoint{5.700000in}{5.700000in}}%
\pgfusepath{clip}%
\pgfsetbuttcap%
\pgfsetroundjoin%
\definecolor{currentfill}{rgb}{0.277941,0.056324,0.381191}%
\pgfsetfillcolor{currentfill}%
\pgfsetfillopacity{0.700000}%
\pgfsetlinewidth{0.000000pt}%
\definecolor{currentstroke}{rgb}{0.000000,0.000000,0.000000}%
\pgfsetstrokecolor{currentstroke}%
\pgfsetdash{}{0pt}%
\pgfpathmoveto{\pgfqpoint{3.419417in}{2.441014in}}%
\pgfpathlineto{\pgfqpoint{3.432781in}{2.433312in}}%
\pgfpathlineto{\pgfqpoint{3.446146in}{2.425747in}}%
\pgfpathlineto{\pgfqpoint{3.459514in}{2.418319in}}%
\pgfpathlineto{\pgfqpoint{3.472884in}{2.411025in}}%
\pgfpathlineto{\pgfqpoint{3.464877in}{2.404018in}}%
\pgfpathlineto{\pgfqpoint{3.456864in}{2.397072in}}%
\pgfpathlineto{\pgfqpoint{3.448843in}{2.390189in}}%
\pgfpathlineto{\pgfqpoint{3.440816in}{2.383371in}}%
\pgfpathlineto{\pgfqpoint{3.427428in}{2.390858in}}%
\pgfpathlineto{\pgfqpoint{3.414042in}{2.398480in}}%
\pgfpathlineto{\pgfqpoint{3.400659in}{2.406239in}}%
\pgfpathlineto{\pgfqpoint{3.387278in}{2.414135in}}%
\pgfpathlineto{\pgfqpoint{3.395323in}{2.420752in}}%
\pgfpathlineto{\pgfqpoint{3.403362in}{2.427439in}}%
\pgfpathlineto{\pgfqpoint{3.411393in}{2.434194in}}%
\pgfpathlineto{\pgfqpoint{3.419417in}{2.441014in}}%
\pgfpathclose%
\pgfusepath{fill}%
\end{pgfscope}%
\begin{pgfscope}%
\pgfpathrectangle{\pgfqpoint{1.150000in}{0.150000in}}{\pgfqpoint{5.700000in}{5.700000in}}%
\pgfusepath{clip}%
\pgfsetbuttcap%
\pgfsetroundjoin%
\definecolor{currentfill}{rgb}{0.272594,0.025563,0.353093}%
\pgfsetfillcolor{currentfill}%
\pgfsetfillopacity{0.700000}%
\pgfsetlinewidth{0.000000pt}%
\definecolor{currentstroke}{rgb}{0.000000,0.000000,0.000000}%
\pgfsetstrokecolor{currentstroke}%
\pgfsetdash{}{0pt}%
\pgfpathmoveto{\pgfqpoint{3.835682in}{2.390184in}}%
\pgfpathlineto{\pgfqpoint{3.849091in}{2.386034in}}%
\pgfpathlineto{\pgfqpoint{3.862506in}{2.382004in}}%
\pgfpathlineto{\pgfqpoint{3.875926in}{2.378093in}}%
\pgfpathlineto{\pgfqpoint{3.889352in}{2.374302in}}%
\pgfpathlineto{\pgfqpoint{3.881506in}{2.365878in}}%
\pgfpathlineto{\pgfqpoint{3.873654in}{2.357473in}}%
\pgfpathlineto{\pgfqpoint{3.865797in}{2.349087in}}%
\pgfpathlineto{\pgfqpoint{3.857934in}{2.340721in}}%
\pgfpathlineto{\pgfqpoint{3.844496in}{2.344650in}}%
\pgfpathlineto{\pgfqpoint{3.831063in}{2.348698in}}%
\pgfpathlineto{\pgfqpoint{3.817636in}{2.352866in}}%
\pgfpathlineto{\pgfqpoint{3.804213in}{2.357155in}}%
\pgfpathlineto{\pgfqpoint{3.812089in}{2.365376in}}%
\pgfpathlineto{\pgfqpoint{3.819959in}{2.373621in}}%
\pgfpathlineto{\pgfqpoint{3.827823in}{2.381891in}}%
\pgfpathlineto{\pgfqpoint{3.835682in}{2.390184in}}%
\pgfpathclose%
\pgfusepath{fill}%
\end{pgfscope}%
\begin{pgfscope}%
\pgfpathrectangle{\pgfqpoint{1.150000in}{0.150000in}}{\pgfqpoint{5.700000in}{5.700000in}}%
\pgfusepath{clip}%
\pgfsetbuttcap%
\pgfsetroundjoin%
\definecolor{currentfill}{rgb}{0.283091,0.110553,0.431554}%
\pgfsetfillcolor{currentfill}%
\pgfsetfillopacity{0.700000}%
\pgfsetlinewidth{0.000000pt}%
\definecolor{currentstroke}{rgb}{0.000000,0.000000,0.000000}%
\pgfsetstrokecolor{currentstroke}%
\pgfsetdash{}{0pt}%
\pgfpathmoveto{\pgfqpoint{4.452807in}{2.532835in}}%
\pgfpathlineto{\pgfqpoint{4.466378in}{2.532755in}}%
\pgfpathlineto{\pgfqpoint{4.479958in}{2.532783in}}%
\pgfpathlineto{\pgfqpoint{4.493547in}{2.532917in}}%
\pgfpathlineto{\pgfqpoint{4.507144in}{2.533158in}}%
\pgfpathlineto{\pgfqpoint{4.499510in}{2.524318in}}%
\pgfpathlineto{\pgfqpoint{4.491871in}{2.515449in}}%
\pgfpathlineto{\pgfqpoint{4.484226in}{2.506549in}}%
\pgfpathlineto{\pgfqpoint{4.476576in}{2.497620in}}%
\pgfpathlineto{\pgfqpoint{4.462969in}{2.497409in}}%
\pgfpathlineto{\pgfqpoint{4.449371in}{2.497304in}}%
\pgfpathlineto{\pgfqpoint{4.435782in}{2.497305in}}%
\pgfpathlineto{\pgfqpoint{4.422201in}{2.497414in}}%
\pgfpathlineto{\pgfqpoint{4.429861in}{2.506308in}}%
\pgfpathlineto{\pgfqpoint{4.437515in}{2.515176in}}%
\pgfpathlineto{\pgfqpoint{4.445163in}{2.524018in}}%
\pgfpathlineto{\pgfqpoint{4.452807in}{2.532835in}}%
\pgfpathclose%
\pgfusepath{fill}%
\end{pgfscope}%
\begin{pgfscope}%
\pgfpathrectangle{\pgfqpoint{1.150000in}{0.150000in}}{\pgfqpoint{5.700000in}{5.700000in}}%
\pgfusepath{clip}%
\pgfsetbuttcap%
\pgfsetroundjoin%
\definecolor{currentfill}{rgb}{0.276022,0.044167,0.370164}%
\pgfsetfillcolor{currentfill}%
\pgfsetfillopacity{0.700000}%
\pgfsetlinewidth{0.000000pt}%
\definecolor{currentstroke}{rgb}{0.000000,0.000000,0.000000}%
\pgfsetstrokecolor{currentstroke}%
\pgfsetdash{}{0pt}%
\pgfpathmoveto{\pgfqpoint{4.059350in}{2.417906in}}%
\pgfpathlineto{\pgfqpoint{4.072807in}{2.415387in}}%
\pgfpathlineto{\pgfqpoint{4.086271in}{2.412981in}}%
\pgfpathlineto{\pgfqpoint{4.099742in}{2.410690in}}%
\pgfpathlineto{\pgfqpoint{4.113220in}{2.408512in}}%
\pgfpathlineto{\pgfqpoint{4.105452in}{2.399691in}}%
\pgfpathlineto{\pgfqpoint{4.097678in}{2.390868in}}%
\pgfpathlineto{\pgfqpoint{4.089899in}{2.382043in}}%
\pgfpathlineto{\pgfqpoint{4.082115in}{2.373217in}}%
\pgfpathlineto{\pgfqpoint{4.068626in}{2.375497in}}%
\pgfpathlineto{\pgfqpoint{4.055145in}{2.377890in}}%
\pgfpathlineto{\pgfqpoint{4.041670in}{2.380397in}}%
\pgfpathlineto{\pgfqpoint{4.028201in}{2.383018in}}%
\pgfpathlineto{\pgfqpoint{4.035997in}{2.391735in}}%
\pgfpathlineto{\pgfqpoint{4.043786in}{2.400456in}}%
\pgfpathlineto{\pgfqpoint{4.051571in}{2.409180in}}%
\pgfpathlineto{\pgfqpoint{4.059350in}{2.417906in}}%
\pgfpathclose%
\pgfusepath{fill}%
\end{pgfscope}%
\begin{pgfscope}%
\pgfpathrectangle{\pgfqpoint{1.150000in}{0.150000in}}{\pgfqpoint{5.700000in}{5.700000in}}%
\pgfusepath{clip}%
\pgfsetbuttcap%
\pgfsetroundjoin%
\definecolor{currentfill}{rgb}{0.278826,0.175490,0.483397}%
\pgfsetfillcolor{currentfill}%
\pgfsetfillopacity{0.700000}%
\pgfsetlinewidth{0.000000pt}%
\definecolor{currentstroke}{rgb}{0.000000,0.000000,0.000000}%
\pgfsetstrokecolor{currentstroke}%
\pgfsetdash{}{0pt}%
\pgfpathmoveto{\pgfqpoint{2.980105in}{2.675210in}}%
\pgfpathlineto{\pgfqpoint{2.993503in}{2.662838in}}%
\pgfpathlineto{\pgfqpoint{3.006899in}{2.650634in}}%
\pgfpathlineto{\pgfqpoint{3.020293in}{2.638598in}}%
\pgfpathlineto{\pgfqpoint{3.033685in}{2.626728in}}%
\pgfpathlineto{\pgfqpoint{3.025473in}{2.621842in}}%
\pgfpathlineto{\pgfqpoint{3.017251in}{2.617064in}}%
\pgfpathlineto{\pgfqpoint{3.009020in}{2.612398in}}%
\pgfpathlineto{\pgfqpoint{3.000779in}{2.607844in}}%
\pgfpathlineto{\pgfqpoint{2.987362in}{2.619951in}}%
\pgfpathlineto{\pgfqpoint{2.973943in}{2.632223in}}%
\pgfpathlineto{\pgfqpoint{2.960522in}{2.644663in}}%
\pgfpathlineto{\pgfqpoint{2.947099in}{2.657272in}}%
\pgfpathlineto{\pgfqpoint{2.955365in}{2.661582in}}%
\pgfpathlineto{\pgfqpoint{2.963621in}{2.666010in}}%
\pgfpathlineto{\pgfqpoint{2.971868in}{2.670553in}}%
\pgfpathlineto{\pgfqpoint{2.980105in}{2.675210in}}%
\pgfpathclose%
\pgfusepath{fill}%
\end{pgfscope}%
\begin{pgfscope}%
\pgfpathrectangle{\pgfqpoint{1.150000in}{0.150000in}}{\pgfqpoint{5.700000in}{5.700000in}}%
\pgfusepath{clip}%
\pgfsetbuttcap%
\pgfsetroundjoin%
\definecolor{currentfill}{rgb}{0.212395,0.359683,0.551710}%
\pgfsetfillcolor{currentfill}%
\pgfsetfillopacity{0.700000}%
\pgfsetlinewidth{0.000000pt}%
\definecolor{currentstroke}{rgb}{0.000000,0.000000,0.000000}%
\pgfsetstrokecolor{currentstroke}%
\pgfsetdash{}{0pt}%
\pgfpathmoveto{\pgfqpoint{5.610250in}{3.056209in}}%
\pgfpathlineto{\pgfqpoint{5.624281in}{3.060519in}}%
\pgfpathlineto{\pgfqpoint{5.638326in}{3.064926in}}%
\pgfpathlineto{\pgfqpoint{5.652383in}{3.069430in}}%
\pgfpathlineto{\pgfqpoint{5.666454in}{3.074032in}}%
\pgfpathlineto{\pgfqpoint{5.659310in}{3.068529in}}%
\pgfpathlineto{\pgfqpoint{5.652160in}{3.063001in}}%
\pgfpathlineto{\pgfqpoint{5.645003in}{3.057445in}}%
\pgfpathlineto{\pgfqpoint{5.637839in}{3.051858in}}%
\pgfpathlineto{\pgfqpoint{5.623750in}{3.047044in}}%
\pgfpathlineto{\pgfqpoint{5.609674in}{3.042326in}}%
\pgfpathlineto{\pgfqpoint{5.595612in}{3.037707in}}%
\pgfpathlineto{\pgfqpoint{5.581563in}{3.033185in}}%
\pgfpathlineto{\pgfqpoint{5.588744in}{3.038978in}}%
\pgfpathlineto{\pgfqpoint{5.595919in}{3.044744in}}%
\pgfpathlineto{\pgfqpoint{5.603087in}{3.050487in}}%
\pgfpathlineto{\pgfqpoint{5.610250in}{3.056209in}}%
\pgfpathclose%
\pgfusepath{fill}%
\end{pgfscope}%
\begin{pgfscope}%
\pgfpathrectangle{\pgfqpoint{1.150000in}{0.150000in}}{\pgfqpoint{5.700000in}{5.700000in}}%
\pgfusepath{clip}%
\pgfsetbuttcap%
\pgfsetroundjoin%
\definecolor{currentfill}{rgb}{0.282327,0.094955,0.417331}%
\pgfsetfillcolor{currentfill}%
\pgfsetfillopacity{0.700000}%
\pgfsetlinewidth{0.000000pt}%
\definecolor{currentstroke}{rgb}{0.000000,0.000000,0.000000}%
\pgfsetstrokecolor{currentstroke}%
\pgfsetdash{}{0pt}%
\pgfpathmoveto{\pgfqpoint{4.367964in}{2.498929in}}%
\pgfpathlineto{\pgfqpoint{4.381511in}{2.498388in}}%
\pgfpathlineto{\pgfqpoint{4.395066in}{2.497955in}}%
\pgfpathlineto{\pgfqpoint{4.408629in}{2.497631in}}%
\pgfpathlineto{\pgfqpoint{4.422201in}{2.497414in}}%
\pgfpathlineto{\pgfqpoint{4.414536in}{2.488496in}}%
\pgfpathlineto{\pgfqpoint{4.406866in}{2.479552in}}%
\pgfpathlineto{\pgfqpoint{4.399191in}{2.470584in}}%
\pgfpathlineto{\pgfqpoint{4.391510in}{2.461590in}}%
\pgfpathlineto{\pgfqpoint{4.377928in}{2.461854in}}%
\pgfpathlineto{\pgfqpoint{4.364355in}{2.462226in}}%
\pgfpathlineto{\pgfqpoint{4.350791in}{2.462706in}}%
\pgfpathlineto{\pgfqpoint{4.337235in}{2.463294in}}%
\pgfpathlineto{\pgfqpoint{4.344925in}{2.472233in}}%
\pgfpathlineto{\pgfqpoint{4.352610in}{2.481152in}}%
\pgfpathlineto{\pgfqpoint{4.360290in}{2.490051in}}%
\pgfpathlineto{\pgfqpoint{4.367964in}{2.498929in}}%
\pgfpathclose%
\pgfusepath{fill}%
\end{pgfscope}%
\begin{pgfscope}%
\pgfpathrectangle{\pgfqpoint{1.150000in}{0.150000in}}{\pgfqpoint{5.700000in}{5.700000in}}%
\pgfusepath{clip}%
\pgfsetbuttcap%
\pgfsetroundjoin%
\definecolor{currentfill}{rgb}{0.220057,0.343307,0.549413}%
\pgfsetfillcolor{currentfill}%
\pgfsetfillopacity{0.700000}%
\pgfsetlinewidth{0.000000pt}%
\definecolor{currentstroke}{rgb}{0.000000,0.000000,0.000000}%
\pgfsetstrokecolor{currentstroke}%
\pgfsetdash{}{0pt}%
\pgfpathmoveto{\pgfqpoint{5.525498in}{3.016075in}}%
\pgfpathlineto{\pgfqpoint{5.539495in}{3.020206in}}%
\pgfpathlineto{\pgfqpoint{5.553505in}{3.024434in}}%
\pgfpathlineto{\pgfqpoint{5.567527in}{3.028761in}}%
\pgfpathlineto{\pgfqpoint{5.581563in}{3.033185in}}%
\pgfpathlineto{\pgfqpoint{5.574376in}{3.027364in}}%
\pgfpathlineto{\pgfqpoint{5.567182in}{3.021512in}}%
\pgfpathlineto{\pgfqpoint{5.559981in}{3.015627in}}%
\pgfpathlineto{\pgfqpoint{5.552774in}{3.009706in}}%
\pgfpathlineto{\pgfqpoint{5.538722in}{3.005088in}}%
\pgfpathlineto{\pgfqpoint{5.524682in}{3.000567in}}%
\pgfpathlineto{\pgfqpoint{5.510656in}{2.996145in}}%
\pgfpathlineto{\pgfqpoint{5.496643in}{2.991821in}}%
\pgfpathlineto{\pgfqpoint{5.503866in}{2.997929in}}%
\pgfpathlineto{\pgfqpoint{5.511083in}{3.004005in}}%
\pgfpathlineto{\pgfqpoint{5.518294in}{3.010053in}}%
\pgfpathlineto{\pgfqpoint{5.525498in}{3.016075in}}%
\pgfpathclose%
\pgfusepath{fill}%
\end{pgfscope}%
\begin{pgfscope}%
\pgfpathrectangle{\pgfqpoint{1.150000in}{0.150000in}}{\pgfqpoint{5.700000in}{5.700000in}}%
\pgfusepath{clip}%
\pgfsetbuttcap%
\pgfsetroundjoin%
\definecolor{currentfill}{rgb}{0.227802,0.326594,0.546532}%
\pgfsetfillcolor{currentfill}%
\pgfsetfillopacity{0.700000}%
\pgfsetlinewidth{0.000000pt}%
\definecolor{currentstroke}{rgb}{0.000000,0.000000,0.000000}%
\pgfsetstrokecolor{currentstroke}%
\pgfsetdash{}{0pt}%
\pgfpathmoveto{\pgfqpoint{5.440719in}{2.975506in}}%
\pgfpathlineto{\pgfqpoint{5.454681in}{2.979437in}}%
\pgfpathlineto{\pgfqpoint{5.468655in}{2.983467in}}%
\pgfpathlineto{\pgfqpoint{5.482643in}{2.987595in}}%
\pgfpathlineto{\pgfqpoint{5.496643in}{2.991821in}}%
\pgfpathlineto{\pgfqpoint{5.489413in}{2.985680in}}%
\pgfpathlineto{\pgfqpoint{5.482177in}{2.979504in}}%
\pgfpathlineto{\pgfqpoint{5.474934in}{2.973290in}}%
\pgfpathlineto{\pgfqpoint{5.467685in}{2.967037in}}%
\pgfpathlineto{\pgfqpoint{5.453669in}{2.962635in}}%
\pgfpathlineto{\pgfqpoint{5.439666in}{2.958332in}}%
\pgfpathlineto{\pgfqpoint{5.425676in}{2.954128in}}%
\pgfpathlineto{\pgfqpoint{5.411699in}{2.950022in}}%
\pgfpathlineto{\pgfqpoint{5.418963in}{2.956443in}}%
\pgfpathlineto{\pgfqpoint{5.426222in}{2.962830in}}%
\pgfpathlineto{\pgfqpoint{5.433473in}{2.969183in}}%
\pgfpathlineto{\pgfqpoint{5.440719in}{2.975506in}}%
\pgfpathclose%
\pgfusepath{fill}%
\end{pgfscope}%
\begin{pgfscope}%
\pgfpathrectangle{\pgfqpoint{1.150000in}{0.150000in}}{\pgfqpoint{5.700000in}{5.700000in}}%
\pgfusepath{clip}%
\pgfsetbuttcap%
\pgfsetroundjoin%
\definecolor{currentfill}{rgb}{0.280894,0.078907,0.402329}%
\pgfsetfillcolor{currentfill}%
\pgfsetfillopacity{0.700000}%
\pgfsetlinewidth{0.000000pt}%
\definecolor{currentstroke}{rgb}{0.000000,0.000000,0.000000}%
\pgfsetstrokecolor{currentstroke}%
\pgfsetdash{}{0pt}%
\pgfpathmoveto{\pgfqpoint{3.280291in}{2.482352in}}%
\pgfpathlineto{\pgfqpoint{3.293660in}{2.473325in}}%
\pgfpathlineto{\pgfqpoint{3.307029in}{2.464443in}}%
\pgfpathlineto{\pgfqpoint{3.320400in}{2.455705in}}%
\pgfpathlineto{\pgfqpoint{3.333772in}{2.447109in}}%
\pgfpathlineto{\pgfqpoint{3.325701in}{2.440769in}}%
\pgfpathlineto{\pgfqpoint{3.317621in}{2.434507in}}%
\pgfpathlineto{\pgfqpoint{3.309535in}{2.428324in}}%
\pgfpathlineto{\pgfqpoint{3.301440in}{2.422223in}}%
\pgfpathlineto{\pgfqpoint{3.288048in}{2.431032in}}%
\pgfpathlineto{\pgfqpoint{3.274657in}{2.439983in}}%
\pgfpathlineto{\pgfqpoint{3.261268in}{2.449078in}}%
\pgfpathlineto{\pgfqpoint{3.247879in}{2.458318in}}%
\pgfpathlineto{\pgfqpoint{3.255994in}{2.464199in}}%
\pgfpathlineto{\pgfqpoint{3.264101in}{2.470167in}}%
\pgfpathlineto{\pgfqpoint{3.272200in}{2.476218in}}%
\pgfpathlineto{\pgfqpoint{3.280291in}{2.482352in}}%
\pgfpathclose%
\pgfusepath{fill}%
\end{pgfscope}%
\begin{pgfscope}%
\pgfpathrectangle{\pgfqpoint{1.150000in}{0.150000in}}{\pgfqpoint{5.700000in}{5.700000in}}%
\pgfusepath{clip}%
\pgfsetbuttcap%
\pgfsetroundjoin%
\definecolor{currentfill}{rgb}{0.281412,0.155834,0.469201}%
\pgfsetfillcolor{currentfill}%
\pgfsetfillopacity{0.700000}%
\pgfsetlinewidth{0.000000pt}%
\definecolor{currentstroke}{rgb}{0.000000,0.000000,0.000000}%
\pgfsetstrokecolor{currentstroke}%
\pgfsetdash{}{0pt}%
\pgfpathmoveto{\pgfqpoint{3.033685in}{2.626728in}}%
\pgfpathlineto{\pgfqpoint{3.047077in}{2.615022in}}%
\pgfpathlineto{\pgfqpoint{3.060467in}{2.603479in}}%
\pgfpathlineto{\pgfqpoint{3.073856in}{2.592098in}}%
\pgfpathlineto{\pgfqpoint{3.087244in}{2.580877in}}%
\pgfpathlineto{\pgfqpoint{3.079055in}{2.575763in}}%
\pgfpathlineto{\pgfqpoint{3.070857in}{2.570753in}}%
\pgfpathlineto{\pgfqpoint{3.062650in}{2.565849in}}%
\pgfpathlineto{\pgfqpoint{3.054434in}{2.561053in}}%
\pgfpathlineto{\pgfqpoint{3.041022in}{2.572509in}}%
\pgfpathlineto{\pgfqpoint{3.027609in}{2.584125in}}%
\pgfpathlineto{\pgfqpoint{3.014195in}{2.595903in}}%
\pgfpathlineto{\pgfqpoint{3.000779in}{2.607844in}}%
\pgfpathlineto{\pgfqpoint{3.009020in}{2.612398in}}%
\pgfpathlineto{\pgfqpoint{3.017251in}{2.617064in}}%
\pgfpathlineto{\pgfqpoint{3.025473in}{2.621842in}}%
\pgfpathlineto{\pgfqpoint{3.033685in}{2.626728in}}%
\pgfpathclose%
\pgfusepath{fill}%
\end{pgfscope}%
\begin{pgfscope}%
\pgfpathrectangle{\pgfqpoint{1.150000in}{0.150000in}}{\pgfqpoint{5.700000in}{5.700000in}}%
\pgfusepath{clip}%
\pgfsetbuttcap%
\pgfsetroundjoin%
\definecolor{currentfill}{rgb}{0.235526,0.309527,0.542944}%
\pgfsetfillcolor{currentfill}%
\pgfsetfillopacity{0.700000}%
\pgfsetlinewidth{0.000000pt}%
\definecolor{currentstroke}{rgb}{0.000000,0.000000,0.000000}%
\pgfsetstrokecolor{currentstroke}%
\pgfsetdash{}{0pt}%
\pgfpathmoveto{\pgfqpoint{5.355916in}{2.934584in}}%
\pgfpathlineto{\pgfqpoint{5.369843in}{2.938296in}}%
\pgfpathlineto{\pgfqpoint{5.383782in}{2.942106in}}%
\pgfpathlineto{\pgfqpoint{5.397734in}{2.946014in}}%
\pgfpathlineto{\pgfqpoint{5.411699in}{2.950022in}}%
\pgfpathlineto{\pgfqpoint{5.404428in}{2.943563in}}%
\pgfpathlineto{\pgfqpoint{5.397150in}{2.937065in}}%
\pgfpathlineto{\pgfqpoint{5.389866in}{2.930526in}}%
\pgfpathlineto{\pgfqpoint{5.382576in}{2.923944in}}%
\pgfpathlineto{\pgfqpoint{5.368597in}{2.919780in}}%
\pgfpathlineto{\pgfqpoint{5.354630in}{2.915715in}}%
\pgfpathlineto{\pgfqpoint{5.340676in}{2.911749in}}%
\pgfpathlineto{\pgfqpoint{5.326735in}{2.907882in}}%
\pgfpathlineto{\pgfqpoint{5.334039in}{2.914614in}}%
\pgfpathlineto{\pgfqpoint{5.341338in}{2.921306in}}%
\pgfpathlineto{\pgfqpoint{5.348630in}{2.927963in}}%
\pgfpathlineto{\pgfqpoint{5.355916in}{2.934584in}}%
\pgfpathclose%
\pgfusepath{fill}%
\end{pgfscope}%
\begin{pgfscope}%
\pgfpathrectangle{\pgfqpoint{1.150000in}{0.150000in}}{\pgfqpoint{5.700000in}{5.700000in}}%
\pgfusepath{clip}%
\pgfsetbuttcap%
\pgfsetroundjoin%
\definecolor{currentfill}{rgb}{0.273809,0.031497,0.358853}%
\pgfsetfillcolor{currentfill}%
\pgfsetfillopacity{0.700000}%
\pgfsetlinewidth{0.000000pt}%
\definecolor{currentstroke}{rgb}{0.000000,0.000000,0.000000}%
\pgfsetstrokecolor{currentstroke}%
\pgfsetdash{}{0pt}%
\pgfpathmoveto{\pgfqpoint{3.974392in}{2.394655in}}%
\pgfpathlineto{\pgfqpoint{3.987835in}{2.391572in}}%
\pgfpathlineto{\pgfqpoint{4.001284in}{2.388605in}}%
\pgfpathlineto{\pgfqpoint{4.014740in}{2.385754in}}%
\pgfpathlineto{\pgfqpoint{4.028201in}{2.383018in}}%
\pgfpathlineto{\pgfqpoint{4.020401in}{2.374305in}}%
\pgfpathlineto{\pgfqpoint{4.012595in}{2.365598in}}%
\pgfpathlineto{\pgfqpoint{4.004783in}{2.356897in}}%
\pgfpathlineto{\pgfqpoint{3.996967in}{2.348203in}}%
\pgfpathlineto{\pgfqpoint{3.983493in}{2.351059in}}%
\pgfpathlineto{\pgfqpoint{3.970027in}{2.354030in}}%
\pgfpathlineto{\pgfqpoint{3.956566in}{2.357116in}}%
\pgfpathlineto{\pgfqpoint{3.943111in}{2.360319in}}%
\pgfpathlineto{\pgfqpoint{3.950940in}{2.368887in}}%
\pgfpathlineto{\pgfqpoint{3.958763in}{2.377466in}}%
\pgfpathlineto{\pgfqpoint{3.966580in}{2.386056in}}%
\pgfpathlineto{\pgfqpoint{3.974392in}{2.394655in}}%
\pgfpathclose%
\pgfusepath{fill}%
\end{pgfscope}%
\begin{pgfscope}%
\pgfpathrectangle{\pgfqpoint{1.150000in}{0.150000in}}{\pgfqpoint{5.700000in}{5.700000in}}%
\pgfusepath{clip}%
\pgfsetbuttcap%
\pgfsetroundjoin%
\definecolor{currentfill}{rgb}{0.280894,0.078907,0.402329}%
\pgfsetfillcolor{currentfill}%
\pgfsetfillopacity{0.700000}%
\pgfsetlinewidth{0.000000pt}%
\definecolor{currentstroke}{rgb}{0.000000,0.000000,0.000000}%
\pgfsetstrokecolor{currentstroke}%
\pgfsetdash{}{0pt}%
\pgfpathmoveto{\pgfqpoint{4.283091in}{2.466737in}}%
\pgfpathlineto{\pgfqpoint{4.296615in}{2.465712in}}%
\pgfpathlineto{\pgfqpoint{4.310147in}{2.464797in}}%
\pgfpathlineto{\pgfqpoint{4.323687in}{2.463991in}}%
\pgfpathlineto{\pgfqpoint{4.337235in}{2.463294in}}%
\pgfpathlineto{\pgfqpoint{4.329539in}{2.454335in}}%
\pgfpathlineto{\pgfqpoint{4.321838in}{2.445356in}}%
\pgfpathlineto{\pgfqpoint{4.314132in}{2.436357in}}%
\pgfpathlineto{\pgfqpoint{4.306420in}{2.427340in}}%
\pgfpathlineto{\pgfqpoint{4.292862in}{2.428102in}}%
\pgfpathlineto{\pgfqpoint{4.279313in}{2.428973in}}%
\pgfpathlineto{\pgfqpoint{4.265771in}{2.429954in}}%
\pgfpathlineto{\pgfqpoint{4.252237in}{2.431045in}}%
\pgfpathlineto{\pgfqpoint{4.259959in}{2.439990in}}%
\pgfpathlineto{\pgfqpoint{4.267675in}{2.448921in}}%
\pgfpathlineto{\pgfqpoint{4.275385in}{2.457837in}}%
\pgfpathlineto{\pgfqpoint{4.283091in}{2.466737in}}%
\pgfpathclose%
\pgfusepath{fill}%
\end{pgfscope}%
\begin{pgfscope}%
\pgfpathrectangle{\pgfqpoint{1.150000in}{0.150000in}}{\pgfqpoint{5.700000in}{5.700000in}}%
\pgfusepath{clip}%
\pgfsetbuttcap%
\pgfsetroundjoin%
\definecolor{currentfill}{rgb}{0.272594,0.025563,0.353093}%
\pgfsetfillcolor{currentfill}%
\pgfsetfillopacity{0.700000}%
\pgfsetlinewidth{0.000000pt}%
\definecolor{currentstroke}{rgb}{0.000000,0.000000,0.000000}%
\pgfsetstrokecolor{currentstroke}%
\pgfsetdash{}{0pt}%
\pgfpathmoveto{\pgfqpoint{3.611783in}{2.387475in}}%
\pgfpathlineto{\pgfqpoint{3.625166in}{2.381540in}}%
\pgfpathlineto{\pgfqpoint{3.638553in}{2.375733in}}%
\pgfpathlineto{\pgfqpoint{3.651944in}{2.370053in}}%
\pgfpathlineto{\pgfqpoint{3.665338in}{2.364499in}}%
\pgfpathlineto{\pgfqpoint{3.657405in}{2.356757in}}%
\pgfpathlineto{\pgfqpoint{3.649466in}{2.349056in}}%
\pgfpathlineto{\pgfqpoint{3.641520in}{2.341399in}}%
\pgfpathlineto{\pgfqpoint{3.633568in}{2.333786in}}%
\pgfpathlineto{\pgfqpoint{3.620159in}{2.339514in}}%
\pgfpathlineto{\pgfqpoint{3.606753in}{2.345369in}}%
\pgfpathlineto{\pgfqpoint{3.593351in}{2.351350in}}%
\pgfpathlineto{\pgfqpoint{3.579953in}{2.357460in}}%
\pgfpathlineto{\pgfqpoint{3.587920in}{2.364891in}}%
\pgfpathlineto{\pgfqpoint{3.595881in}{2.372372in}}%
\pgfpathlineto{\pgfqpoint{3.603835in}{2.379900in}}%
\pgfpathlineto{\pgfqpoint{3.611783in}{2.387475in}}%
\pgfpathclose%
\pgfusepath{fill}%
\end{pgfscope}%
\begin{pgfscope}%
\pgfpathrectangle{\pgfqpoint{1.150000in}{0.150000in}}{\pgfqpoint{5.700000in}{5.700000in}}%
\pgfusepath{clip}%
\pgfsetbuttcap%
\pgfsetroundjoin%
\definecolor{currentfill}{rgb}{0.243113,0.292092,0.538516}%
\pgfsetfillcolor{currentfill}%
\pgfsetfillopacity{0.700000}%
\pgfsetlinewidth{0.000000pt}%
\definecolor{currentstroke}{rgb}{0.000000,0.000000,0.000000}%
\pgfsetstrokecolor{currentstroke}%
\pgfsetdash{}{0pt}%
\pgfpathmoveto{\pgfqpoint{5.271092in}{2.893406in}}%
\pgfpathlineto{\pgfqpoint{5.284984in}{2.896876in}}%
\pgfpathlineto{\pgfqpoint{5.298889in}{2.900445in}}%
\pgfpathlineto{\pgfqpoint{5.312806in}{2.904114in}}%
\pgfpathlineto{\pgfqpoint{5.326735in}{2.907882in}}%
\pgfpathlineto{\pgfqpoint{5.319424in}{2.901111in}}%
\pgfpathlineto{\pgfqpoint{5.312106in}{2.894297in}}%
\pgfpathlineto{\pgfqpoint{5.304782in}{2.887439in}}%
\pgfpathlineto{\pgfqpoint{5.297452in}{2.880536in}}%
\pgfpathlineto{\pgfqpoint{5.283509in}{2.876631in}}%
\pgfpathlineto{\pgfqpoint{5.269579in}{2.872824in}}%
\pgfpathlineto{\pgfqpoint{5.255661in}{2.869117in}}%
\pgfpathlineto{\pgfqpoint{5.241755in}{2.865510in}}%
\pgfpathlineto{\pgfqpoint{5.249098in}{2.872544in}}%
\pgfpathlineto{\pgfqpoint{5.256436in}{2.879536in}}%
\pgfpathlineto{\pgfqpoint{5.263767in}{2.886490in}}%
\pgfpathlineto{\pgfqpoint{5.271092in}{2.893406in}}%
\pgfpathclose%
\pgfusepath{fill}%
\end{pgfscope}%
\begin{pgfscope}%
\pgfpathrectangle{\pgfqpoint{1.150000in}{0.150000in}}{\pgfqpoint{5.700000in}{5.700000in}}%
\pgfusepath{clip}%
\pgfsetbuttcap%
\pgfsetroundjoin%
\definecolor{currentfill}{rgb}{0.250425,0.274290,0.533103}%
\pgfsetfillcolor{currentfill}%
\pgfsetfillopacity{0.700000}%
\pgfsetlinewidth{0.000000pt}%
\definecolor{currentstroke}{rgb}{0.000000,0.000000,0.000000}%
\pgfsetstrokecolor{currentstroke}%
\pgfsetdash{}{0pt}%
\pgfpathmoveto{\pgfqpoint{5.186252in}{2.852077in}}%
\pgfpathlineto{\pgfqpoint{5.200109in}{2.855286in}}%
\pgfpathlineto{\pgfqpoint{5.213979in}{2.858594in}}%
\pgfpathlineto{\pgfqpoint{5.227861in}{2.862002in}}%
\pgfpathlineto{\pgfqpoint{5.241755in}{2.865510in}}%
\pgfpathlineto{\pgfqpoint{5.234405in}{2.858433in}}%
\pgfpathlineto{\pgfqpoint{5.227048in}{2.851313in}}%
\pgfpathlineto{\pgfqpoint{5.219685in}{2.844146in}}%
\pgfpathlineto{\pgfqpoint{5.212316in}{2.836932in}}%
\pgfpathlineto{\pgfqpoint{5.198410in}{2.833305in}}%
\pgfpathlineto{\pgfqpoint{5.184515in}{2.829779in}}%
\pgfpathlineto{\pgfqpoint{5.170633in}{2.826352in}}%
\pgfpathlineto{\pgfqpoint{5.156762in}{2.823025in}}%
\pgfpathlineto{\pgfqpoint{5.164144in}{2.830351in}}%
\pgfpathlineto{\pgfqpoint{5.171519in}{2.837634in}}%
\pgfpathlineto{\pgfqpoint{5.178889in}{2.844875in}}%
\pgfpathlineto{\pgfqpoint{5.186252in}{2.852077in}}%
\pgfpathclose%
\pgfusepath{fill}%
\end{pgfscope}%
\begin{pgfscope}%
\pgfpathrectangle{\pgfqpoint{1.150000in}{0.150000in}}{\pgfqpoint{5.700000in}{5.700000in}}%
\pgfusepath{clip}%
\pgfsetbuttcap%
\pgfsetroundjoin%
\definecolor{currentfill}{rgb}{0.276022,0.044167,0.370164}%
\pgfsetfillcolor{currentfill}%
\pgfsetfillopacity{0.700000}%
\pgfsetlinewidth{0.000000pt}%
\definecolor{currentstroke}{rgb}{0.000000,0.000000,0.000000}%
\pgfsetstrokecolor{currentstroke}%
\pgfsetdash{}{0pt}%
\pgfpathmoveto{\pgfqpoint{3.472884in}{2.411025in}}%
\pgfpathlineto{\pgfqpoint{3.486257in}{2.403867in}}%
\pgfpathlineto{\pgfqpoint{3.499633in}{2.396842in}}%
\pgfpathlineto{\pgfqpoint{3.513012in}{2.389950in}}%
\pgfpathlineto{\pgfqpoint{3.526394in}{2.383191in}}%
\pgfpathlineto{\pgfqpoint{3.518404in}{2.375997in}}%
\pgfpathlineto{\pgfqpoint{3.510407in}{2.368859in}}%
\pgfpathlineto{\pgfqpoint{3.502404in}{2.361781in}}%
\pgfpathlineto{\pgfqpoint{3.494394in}{2.354763in}}%
\pgfpathlineto{\pgfqpoint{3.480995in}{2.361716in}}%
\pgfpathlineto{\pgfqpoint{3.467599in}{2.368801in}}%
\pgfpathlineto{\pgfqpoint{3.454206in}{2.376019in}}%
\pgfpathlineto{\pgfqpoint{3.440816in}{2.383371in}}%
\pgfpathlineto{\pgfqpoint{3.448843in}{2.390189in}}%
\pgfpathlineto{\pgfqpoint{3.456864in}{2.397072in}}%
\pgfpathlineto{\pgfqpoint{3.464877in}{2.404018in}}%
\pgfpathlineto{\pgfqpoint{3.472884in}{2.411025in}}%
\pgfpathclose%
\pgfusepath{fill}%
\end{pgfscope}%
\begin{pgfscope}%
\pgfpathrectangle{\pgfqpoint{1.150000in}{0.150000in}}{\pgfqpoint{5.700000in}{5.700000in}}%
\pgfusepath{clip}%
\pgfsetbuttcap%
\pgfsetroundjoin%
\definecolor{currentfill}{rgb}{0.272594,0.025563,0.353093}%
\pgfsetfillcolor{currentfill}%
\pgfsetfillopacity{0.700000}%
\pgfsetlinewidth{0.000000pt}%
\definecolor{currentstroke}{rgb}{0.000000,0.000000,0.000000}%
\pgfsetstrokecolor{currentstroke}%
\pgfsetdash{}{0pt}%
\pgfpathmoveto{\pgfqpoint{3.750574in}{2.375522in}}%
\pgfpathlineto{\pgfqpoint{3.763977in}{2.370747in}}%
\pgfpathlineto{\pgfqpoint{3.777384in}{2.366094in}}%
\pgfpathlineto{\pgfqpoint{3.790796in}{2.361564in}}%
\pgfpathlineto{\pgfqpoint{3.804213in}{2.357155in}}%
\pgfpathlineto{\pgfqpoint{3.796332in}{2.348960in}}%
\pgfpathlineto{\pgfqpoint{3.788445in}{2.340792in}}%
\pgfpathlineto{\pgfqpoint{3.780552in}{2.332654in}}%
\pgfpathlineto{\pgfqpoint{3.772653in}{2.324545in}}%
\pgfpathlineto{\pgfqpoint{3.759222in}{2.329110in}}%
\pgfpathlineto{\pgfqpoint{3.745796in}{2.333796in}}%
\pgfpathlineto{\pgfqpoint{3.732375in}{2.338605in}}%
\pgfpathlineto{\pgfqpoint{3.718959in}{2.343536in}}%
\pgfpathlineto{\pgfqpoint{3.726872in}{2.351482in}}%
\pgfpathlineto{\pgfqpoint{3.734779in}{2.359462in}}%
\pgfpathlineto{\pgfqpoint{3.742679in}{2.367476in}}%
\pgfpathlineto{\pgfqpoint{3.750574in}{2.375522in}}%
\pgfpathclose%
\pgfusepath{fill}%
\end{pgfscope}%
\begin{pgfscope}%
\pgfpathrectangle{\pgfqpoint{1.150000in}{0.150000in}}{\pgfqpoint{5.700000in}{5.700000in}}%
\pgfusepath{clip}%
\pgfsetbuttcap%
\pgfsetroundjoin%
\definecolor{currentfill}{rgb}{0.257322,0.256130,0.526563}%
\pgfsetfillcolor{currentfill}%
\pgfsetfillopacity{0.700000}%
\pgfsetlinewidth{0.000000pt}%
\definecolor{currentstroke}{rgb}{0.000000,0.000000,0.000000}%
\pgfsetstrokecolor{currentstroke}%
\pgfsetdash{}{0pt}%
\pgfpathmoveto{\pgfqpoint{5.101397in}{2.810719in}}%
\pgfpathlineto{\pgfqpoint{5.115221in}{2.813645in}}%
\pgfpathlineto{\pgfqpoint{5.129056in}{2.816671in}}%
\pgfpathlineto{\pgfqpoint{5.142903in}{2.819798in}}%
\pgfpathlineto{\pgfqpoint{5.156762in}{2.823025in}}%
\pgfpathlineto{\pgfqpoint{5.149374in}{2.815655in}}%
\pgfpathlineto{\pgfqpoint{5.141980in}{2.808239in}}%
\pgfpathlineto{\pgfqpoint{5.134580in}{2.800776in}}%
\pgfpathlineto{\pgfqpoint{5.127173in}{2.793264in}}%
\pgfpathlineto{\pgfqpoint{5.113302in}{2.789937in}}%
\pgfpathlineto{\pgfqpoint{5.099443in}{2.786711in}}%
\pgfpathlineto{\pgfqpoint{5.085596in}{2.783585in}}%
\pgfpathlineto{\pgfqpoint{5.071760in}{2.780559in}}%
\pgfpathlineto{\pgfqpoint{5.079179in}{2.788164in}}%
\pgfpathlineto{\pgfqpoint{5.086591in}{2.795724in}}%
\pgfpathlineto{\pgfqpoint{5.093997in}{2.803242in}}%
\pgfpathlineto{\pgfqpoint{5.101397in}{2.810719in}}%
\pgfpathclose%
\pgfusepath{fill}%
\end{pgfscope}%
\begin{pgfscope}%
\pgfpathrectangle{\pgfqpoint{1.150000in}{0.150000in}}{\pgfqpoint{5.700000in}{5.700000in}}%
\pgfusepath{clip}%
\pgfsetbuttcap%
\pgfsetroundjoin%
\definecolor{currentfill}{rgb}{0.263663,0.237631,0.518762}%
\pgfsetfillcolor{currentfill}%
\pgfsetfillopacity{0.700000}%
\pgfsetlinewidth{0.000000pt}%
\definecolor{currentstroke}{rgb}{0.000000,0.000000,0.000000}%
\pgfsetstrokecolor{currentstroke}%
\pgfsetdash{}{0pt}%
\pgfpathmoveto{\pgfqpoint{5.016532in}{2.769465in}}%
\pgfpathlineto{\pgfqpoint{5.030322in}{2.772087in}}%
\pgfpathlineto{\pgfqpoint{5.044123in}{2.774810in}}%
\pgfpathlineto{\pgfqpoint{5.057936in}{2.777634in}}%
\pgfpathlineto{\pgfqpoint{5.071760in}{2.780559in}}%
\pgfpathlineto{\pgfqpoint{5.064336in}{2.772909in}}%
\pgfpathlineto{\pgfqpoint{5.056905in}{2.765213in}}%
\pgfpathlineto{\pgfqpoint{5.049468in}{2.757470in}}%
\pgfpathlineto{\pgfqpoint{5.042025in}{2.749677in}}%
\pgfpathlineto{\pgfqpoint{5.028189in}{2.746671in}}%
\pgfpathlineto{\pgfqpoint{5.014365in}{2.743766in}}%
\pgfpathlineto{\pgfqpoint{5.000553in}{2.740961in}}%
\pgfpathlineto{\pgfqpoint{4.986751in}{2.738258in}}%
\pgfpathlineto{\pgfqpoint{4.994205in}{2.746125in}}%
\pgfpathlineto{\pgfqpoint{5.001654in}{2.753948in}}%
\pgfpathlineto{\pgfqpoint{5.009096in}{2.761728in}}%
\pgfpathlineto{\pgfqpoint{5.016532in}{2.769465in}}%
\pgfpathclose%
\pgfusepath{fill}%
\end{pgfscope}%
\begin{pgfscope}%
\pgfpathrectangle{\pgfqpoint{1.150000in}{0.150000in}}{\pgfqpoint{5.700000in}{5.700000in}}%
\pgfusepath{clip}%
\pgfsetbuttcap%
\pgfsetroundjoin%
\definecolor{currentfill}{rgb}{0.269308,0.218818,0.509577}%
\pgfsetfillcolor{currentfill}%
\pgfsetfillopacity{0.700000}%
\pgfsetlinewidth{0.000000pt}%
\definecolor{currentstroke}{rgb}{0.000000,0.000000,0.000000}%
\pgfsetstrokecolor{currentstroke}%
\pgfsetdash{}{0pt}%
\pgfpathmoveto{\pgfqpoint{4.931657in}{2.728461in}}%
\pgfpathlineto{\pgfqpoint{4.945414in}{2.730758in}}%
\pgfpathlineto{\pgfqpoint{4.959182in}{2.733156in}}%
\pgfpathlineto{\pgfqpoint{4.972961in}{2.735657in}}%
\pgfpathlineto{\pgfqpoint{4.986751in}{2.738258in}}%
\pgfpathlineto{\pgfqpoint{4.979291in}{2.730346in}}%
\pgfpathlineto{\pgfqpoint{4.971825in}{2.722388in}}%
\pgfpathlineto{\pgfqpoint{4.964352in}{2.714382in}}%
\pgfpathlineto{\pgfqpoint{4.956874in}{2.706329in}}%
\pgfpathlineto{\pgfqpoint{4.943073in}{2.703664in}}%
\pgfpathlineto{\pgfqpoint{4.929283in}{2.701101in}}%
\pgfpathlineto{\pgfqpoint{4.915505in}{2.698640in}}%
\pgfpathlineto{\pgfqpoint{4.901737in}{2.696280in}}%
\pgfpathlineto{\pgfqpoint{4.909226in}{2.704390in}}%
\pgfpathlineto{\pgfqpoint{4.916709in}{2.712456in}}%
\pgfpathlineto{\pgfqpoint{4.924186in}{2.720479in}}%
\pgfpathlineto{\pgfqpoint{4.931657in}{2.728461in}}%
\pgfpathclose%
\pgfusepath{fill}%
\end{pgfscope}%
\begin{pgfscope}%
\pgfpathrectangle{\pgfqpoint{1.150000in}{0.150000in}}{\pgfqpoint{5.700000in}{5.700000in}}%
\pgfusepath{clip}%
\pgfsetbuttcap%
\pgfsetroundjoin%
\definecolor{currentfill}{rgb}{0.282884,0.135920,0.453427}%
\pgfsetfillcolor{currentfill}%
\pgfsetfillopacity{0.700000}%
\pgfsetlinewidth{0.000000pt}%
\definecolor{currentstroke}{rgb}{0.000000,0.000000,0.000000}%
\pgfsetstrokecolor{currentstroke}%
\pgfsetdash{}{0pt}%
\pgfpathmoveto{\pgfqpoint{3.087244in}{2.580877in}}%
\pgfpathlineto{\pgfqpoint{3.100631in}{2.569816in}}%
\pgfpathlineto{\pgfqpoint{3.114018in}{2.558913in}}%
\pgfpathlineto{\pgfqpoint{3.127404in}{2.548167in}}%
\pgfpathlineto{\pgfqpoint{3.140789in}{2.537576in}}%
\pgfpathlineto{\pgfqpoint{3.132623in}{2.532235in}}%
\pgfpathlineto{\pgfqpoint{3.124449in}{2.526993in}}%
\pgfpathlineto{\pgfqpoint{3.116265in}{2.521852in}}%
\pgfpathlineto{\pgfqpoint{3.108073in}{2.516816in}}%
\pgfpathlineto{\pgfqpoint{3.094664in}{2.527640in}}%
\pgfpathlineto{\pgfqpoint{3.081255in}{2.538620in}}%
\pgfpathlineto{\pgfqpoint{3.067845in}{2.549758in}}%
\pgfpathlineto{\pgfqpoint{3.054434in}{2.561053in}}%
\pgfpathlineto{\pgfqpoint{3.062650in}{2.565849in}}%
\pgfpathlineto{\pgfqpoint{3.070857in}{2.570753in}}%
\pgfpathlineto{\pgfqpoint{3.079055in}{2.575763in}}%
\pgfpathlineto{\pgfqpoint{3.087244in}{2.580877in}}%
\pgfpathclose%
\pgfusepath{fill}%
\end{pgfscope}%
\begin{pgfscope}%
\pgfpathrectangle{\pgfqpoint{1.150000in}{0.150000in}}{\pgfqpoint{5.700000in}{5.700000in}}%
\pgfusepath{clip}%
\pgfsetbuttcap%
\pgfsetroundjoin%
\definecolor{currentfill}{rgb}{0.278791,0.062145,0.386592}%
\pgfsetfillcolor{currentfill}%
\pgfsetfillopacity{0.700000}%
\pgfsetlinewidth{0.000000pt}%
\definecolor{currentstroke}{rgb}{0.000000,0.000000,0.000000}%
\pgfsetstrokecolor{currentstroke}%
\pgfsetdash{}{0pt}%
\pgfpathmoveto{\pgfqpoint{4.198179in}{2.436511in}}%
\pgfpathlineto{\pgfqpoint{4.211682in}{2.434978in}}%
\pgfpathlineto{\pgfqpoint{4.225193in}{2.433556in}}%
\pgfpathlineto{\pgfqpoint{4.238711in}{2.432245in}}%
\pgfpathlineto{\pgfqpoint{4.252237in}{2.431045in}}%
\pgfpathlineto{\pgfqpoint{4.244511in}{2.422085in}}%
\pgfpathlineto{\pgfqpoint{4.236779in}{2.413112in}}%
\pgfpathlineto{\pgfqpoint{4.229042in}{2.404126in}}%
\pgfpathlineto{\pgfqpoint{4.221299in}{2.395127in}}%
\pgfpathlineto{\pgfqpoint{4.207763in}{2.396411in}}%
\pgfpathlineto{\pgfqpoint{4.194235in}{2.397806in}}%
\pgfpathlineto{\pgfqpoint{4.180714in}{2.399311in}}%
\pgfpathlineto{\pgfqpoint{4.167201in}{2.400928in}}%
\pgfpathlineto{\pgfqpoint{4.174953in}{2.409836in}}%
\pgfpathlineto{\pgfqpoint{4.182701in}{2.418737in}}%
\pgfpathlineto{\pgfqpoint{4.190443in}{2.427628in}}%
\pgfpathlineto{\pgfqpoint{4.198179in}{2.436511in}}%
\pgfpathclose%
\pgfusepath{fill}%
\end{pgfscope}%
\begin{pgfscope}%
\pgfpathrectangle{\pgfqpoint{1.150000in}{0.150000in}}{\pgfqpoint{5.700000in}{5.700000in}}%
\pgfusepath{clip}%
\pgfsetbuttcap%
\pgfsetroundjoin%
\definecolor{currentfill}{rgb}{0.274128,0.199721,0.498911}%
\pgfsetfillcolor{currentfill}%
\pgfsetfillopacity{0.700000}%
\pgfsetlinewidth{0.000000pt}%
\definecolor{currentstroke}{rgb}{0.000000,0.000000,0.000000}%
\pgfsetstrokecolor{currentstroke}%
\pgfsetdash{}{0pt}%
\pgfpathmoveto{\pgfqpoint{4.846773in}{2.687864in}}%
\pgfpathlineto{\pgfqpoint{4.860498in}{2.689814in}}%
\pgfpathlineto{\pgfqpoint{4.874234in}{2.691867in}}%
\pgfpathlineto{\pgfqpoint{4.887980in}{2.694023in}}%
\pgfpathlineto{\pgfqpoint{4.901737in}{2.696280in}}%
\pgfpathlineto{\pgfqpoint{4.894242in}{2.688126in}}%
\pgfpathlineto{\pgfqpoint{4.886741in}{2.679926in}}%
\pgfpathlineto{\pgfqpoint{4.879234in}{2.671680in}}%
\pgfpathlineto{\pgfqpoint{4.871722in}{2.663388in}}%
\pgfpathlineto{\pgfqpoint{4.857955in}{2.661086in}}%
\pgfpathlineto{\pgfqpoint{4.844198in}{2.658886in}}%
\pgfpathlineto{\pgfqpoint{4.830453in}{2.656789in}}%
\pgfpathlineto{\pgfqpoint{4.816718in}{2.654794in}}%
\pgfpathlineto{\pgfqpoint{4.824240in}{2.663125in}}%
\pgfpathlineto{\pgfqpoint{4.831757in}{2.671412in}}%
\pgfpathlineto{\pgfqpoint{4.839268in}{2.679659in}}%
\pgfpathlineto{\pgfqpoint{4.846773in}{2.687864in}}%
\pgfpathclose%
\pgfusepath{fill}%
\end{pgfscope}%
\begin{pgfscope}%
\pgfpathrectangle{\pgfqpoint{1.150000in}{0.150000in}}{\pgfqpoint{5.700000in}{5.700000in}}%
\pgfusepath{clip}%
\pgfsetbuttcap%
\pgfsetroundjoin%
\definecolor{currentfill}{rgb}{0.278012,0.180367,0.486697}%
\pgfsetfillcolor{currentfill}%
\pgfsetfillopacity{0.700000}%
\pgfsetlinewidth{0.000000pt}%
\definecolor{currentstroke}{rgb}{0.000000,0.000000,0.000000}%
\pgfsetstrokecolor{currentstroke}%
\pgfsetdash{}{0pt}%
\pgfpathmoveto{\pgfqpoint{4.761881in}{2.647845in}}%
\pgfpathlineto{\pgfqpoint{4.775575in}{2.649428in}}%
\pgfpathlineto{\pgfqpoint{4.789279in}{2.651114in}}%
\pgfpathlineto{\pgfqpoint{4.802993in}{2.652903in}}%
\pgfpathlineto{\pgfqpoint{4.816718in}{2.654794in}}%
\pgfpathlineto{\pgfqpoint{4.809189in}{2.646421in}}%
\pgfpathlineto{\pgfqpoint{4.801655in}{2.638004in}}%
\pgfpathlineto{\pgfqpoint{4.794115in}{2.629543in}}%
\pgfpathlineto{\pgfqpoint{4.786569in}{2.621036in}}%
\pgfpathlineto{\pgfqpoint{4.772835in}{2.619118in}}%
\pgfpathlineto{\pgfqpoint{4.759111in}{2.617304in}}%
\pgfpathlineto{\pgfqpoint{4.745397in}{2.615592in}}%
\pgfpathlineto{\pgfqpoint{4.731694in}{2.613984in}}%
\pgfpathlineto{\pgfqpoint{4.739249in}{2.622509in}}%
\pgfpathlineto{\pgfqpoint{4.746799in}{2.630994in}}%
\pgfpathlineto{\pgfqpoint{4.754343in}{2.639440in}}%
\pgfpathlineto{\pgfqpoint{4.761881in}{2.647845in}}%
\pgfpathclose%
\pgfusepath{fill}%
\end{pgfscope}%
\begin{pgfscope}%
\pgfpathrectangle{\pgfqpoint{1.150000in}{0.150000in}}{\pgfqpoint{5.700000in}{5.700000in}}%
\pgfusepath{clip}%
\pgfsetbuttcap%
\pgfsetroundjoin%
\definecolor{currentfill}{rgb}{0.272594,0.025563,0.353093}%
\pgfsetfillcolor{currentfill}%
\pgfsetfillopacity{0.700000}%
\pgfsetlinewidth{0.000000pt}%
\definecolor{currentstroke}{rgb}{0.000000,0.000000,0.000000}%
\pgfsetstrokecolor{currentstroke}%
\pgfsetdash{}{0pt}%
\pgfpathmoveto{\pgfqpoint{3.889352in}{2.374302in}}%
\pgfpathlineto{\pgfqpoint{3.902783in}{2.370630in}}%
\pgfpathlineto{\pgfqpoint{3.916220in}{2.367075in}}%
\pgfpathlineto{\pgfqpoint{3.929663in}{2.363639in}}%
\pgfpathlineto{\pgfqpoint{3.943111in}{2.360319in}}%
\pgfpathlineto{\pgfqpoint{3.935277in}{2.351765in}}%
\pgfpathlineto{\pgfqpoint{3.927438in}{2.343224in}}%
\pgfpathlineto{\pgfqpoint{3.919593in}{2.334698in}}%
\pgfpathlineto{\pgfqpoint{3.911742in}{2.326188in}}%
\pgfpathlineto{\pgfqpoint{3.898282in}{2.329645in}}%
\pgfpathlineto{\pgfqpoint{3.884827in}{2.333219in}}%
\pgfpathlineto{\pgfqpoint{3.871378in}{2.336911in}}%
\pgfpathlineto{\pgfqpoint{3.857934in}{2.340721in}}%
\pgfpathlineto{\pgfqpoint{3.865797in}{2.349087in}}%
\pgfpathlineto{\pgfqpoint{3.873654in}{2.357473in}}%
\pgfpathlineto{\pgfqpoint{3.881506in}{2.365878in}}%
\pgfpathlineto{\pgfqpoint{3.889352in}{2.374302in}}%
\pgfpathclose%
\pgfusepath{fill}%
\end{pgfscope}%
\begin{pgfscope}%
\pgfpathrectangle{\pgfqpoint{1.150000in}{0.150000in}}{\pgfqpoint{5.700000in}{5.700000in}}%
\pgfusepath{clip}%
\pgfsetbuttcap%
\pgfsetroundjoin%
\definecolor{currentfill}{rgb}{0.279566,0.067836,0.391917}%
\pgfsetfillcolor{currentfill}%
\pgfsetfillopacity{0.700000}%
\pgfsetlinewidth{0.000000pt}%
\definecolor{currentstroke}{rgb}{0.000000,0.000000,0.000000}%
\pgfsetstrokecolor{currentstroke}%
\pgfsetdash{}{0pt}%
\pgfpathmoveto{\pgfqpoint{3.333772in}{2.447109in}}%
\pgfpathlineto{\pgfqpoint{3.347146in}{2.438656in}}%
\pgfpathlineto{\pgfqpoint{3.360522in}{2.430343in}}%
\pgfpathlineto{\pgfqpoint{3.373899in}{2.422169in}}%
\pgfpathlineto{\pgfqpoint{3.387278in}{2.414135in}}%
\pgfpathlineto{\pgfqpoint{3.379225in}{2.407589in}}%
\pgfpathlineto{\pgfqpoint{3.371165in}{2.401117in}}%
\pgfpathlineto{\pgfqpoint{3.363098in}{2.394719in}}%
\pgfpathlineto{\pgfqpoint{3.355023in}{2.388399in}}%
\pgfpathlineto{\pgfqpoint{3.341625in}{2.396645in}}%
\pgfpathlineto{\pgfqpoint{3.328228in}{2.405031in}}%
\pgfpathlineto{\pgfqpoint{3.314833in}{2.413557in}}%
\pgfpathlineto{\pgfqpoint{3.301440in}{2.422223in}}%
\pgfpathlineto{\pgfqpoint{3.309535in}{2.428324in}}%
\pgfpathlineto{\pgfqpoint{3.317621in}{2.434507in}}%
\pgfpathlineto{\pgfqpoint{3.325701in}{2.440769in}}%
\pgfpathlineto{\pgfqpoint{3.333772in}{2.447109in}}%
\pgfpathclose%
\pgfusepath{fill}%
\end{pgfscope}%
\begin{pgfscope}%
\pgfpathrectangle{\pgfqpoint{1.150000in}{0.150000in}}{\pgfqpoint{5.700000in}{5.700000in}}%
\pgfusepath{clip}%
\pgfsetbuttcap%
\pgfsetroundjoin%
\definecolor{currentfill}{rgb}{0.280868,0.160771,0.472899}%
\pgfsetfillcolor{currentfill}%
\pgfsetfillopacity{0.700000}%
\pgfsetlinewidth{0.000000pt}%
\definecolor{currentstroke}{rgb}{0.000000,0.000000,0.000000}%
\pgfsetstrokecolor{currentstroke}%
\pgfsetdash{}{0pt}%
\pgfpathmoveto{\pgfqpoint{4.676981in}{2.608589in}}%
\pgfpathlineto{\pgfqpoint{4.690644in}{2.609782in}}%
\pgfpathlineto{\pgfqpoint{4.704317in}{2.611078in}}%
\pgfpathlineto{\pgfqpoint{4.718001in}{2.612479in}}%
\pgfpathlineto{\pgfqpoint{4.731694in}{2.613984in}}%
\pgfpathlineto{\pgfqpoint{4.724133in}{2.605418in}}%
\pgfpathlineto{\pgfqpoint{4.716566in}{2.596810in}}%
\pgfpathlineto{\pgfqpoint{4.708993in}{2.588161in}}%
\pgfpathlineto{\pgfqpoint{4.701415in}{2.579469in}}%
\pgfpathlineto{\pgfqpoint{4.687712in}{2.577957in}}%
\pgfpathlineto{\pgfqpoint{4.674020in}{2.576549in}}%
\pgfpathlineto{\pgfqpoint{4.660337in}{2.575244in}}%
\pgfpathlineto{\pgfqpoint{4.646664in}{2.574044in}}%
\pgfpathlineto{\pgfqpoint{4.654252in}{2.582736in}}%
\pgfpathlineto{\pgfqpoint{4.661834in}{2.591391in}}%
\pgfpathlineto{\pgfqpoint{4.669410in}{2.600008in}}%
\pgfpathlineto{\pgfqpoint{4.676981in}{2.608589in}}%
\pgfpathclose%
\pgfusepath{fill}%
\end{pgfscope}%
\begin{pgfscope}%
\pgfpathrectangle{\pgfqpoint{1.150000in}{0.150000in}}{\pgfqpoint{5.700000in}{5.700000in}}%
\pgfusepath{clip}%
\pgfsetbuttcap%
\pgfsetroundjoin%
\definecolor{currentfill}{rgb}{0.277018,0.050344,0.375715}%
\pgfsetfillcolor{currentfill}%
\pgfsetfillopacity{0.700000}%
\pgfsetlinewidth{0.000000pt}%
\definecolor{currentstroke}{rgb}{0.000000,0.000000,0.000000}%
\pgfsetstrokecolor{currentstroke}%
\pgfsetdash{}{0pt}%
\pgfpathmoveto{\pgfqpoint{4.113220in}{2.408512in}}%
\pgfpathlineto{\pgfqpoint{4.126704in}{2.406448in}}%
\pgfpathlineto{\pgfqpoint{4.140196in}{2.404495in}}%
\pgfpathlineto{\pgfqpoint{4.153695in}{2.402656in}}%
\pgfpathlineto{\pgfqpoint{4.167201in}{2.400928in}}%
\pgfpathlineto{\pgfqpoint{4.159443in}{2.392012in}}%
\pgfpathlineto{\pgfqpoint{4.151680in}{2.383089in}}%
\pgfpathlineto{\pgfqpoint{4.143911in}{2.374161in}}%
\pgfpathlineto{\pgfqpoint{4.136138in}{2.365227in}}%
\pgfpathlineto{\pgfqpoint{4.122621in}{2.367056in}}%
\pgfpathlineto{\pgfqpoint{4.109112in}{2.368997in}}%
\pgfpathlineto{\pgfqpoint{4.095610in}{2.371051in}}%
\pgfpathlineto{\pgfqpoint{4.082115in}{2.373217in}}%
\pgfpathlineto{\pgfqpoint{4.089899in}{2.382043in}}%
\pgfpathlineto{\pgfqpoint{4.097678in}{2.390868in}}%
\pgfpathlineto{\pgfqpoint{4.105452in}{2.399691in}}%
\pgfpathlineto{\pgfqpoint{4.113220in}{2.408512in}}%
\pgfpathclose%
\pgfusepath{fill}%
\end{pgfscope}%
\begin{pgfscope}%
\pgfpathrectangle{\pgfqpoint{1.150000in}{0.150000in}}{\pgfqpoint{5.700000in}{5.700000in}}%
\pgfusepath{clip}%
\pgfsetbuttcap%
\pgfsetroundjoin%
\definecolor{currentfill}{rgb}{0.282623,0.140926,0.457517}%
\pgfsetfillcolor{currentfill}%
\pgfsetfillopacity{0.700000}%
\pgfsetlinewidth{0.000000pt}%
\definecolor{currentstroke}{rgb}{0.000000,0.000000,0.000000}%
\pgfsetstrokecolor{currentstroke}%
\pgfsetdash{}{0pt}%
\pgfpathmoveto{\pgfqpoint{4.592069in}{2.570290in}}%
\pgfpathlineto{\pgfqpoint{4.605704in}{2.571071in}}%
\pgfpathlineto{\pgfqpoint{4.619347in}{2.571957in}}%
\pgfpathlineto{\pgfqpoint{4.633001in}{2.572948in}}%
\pgfpathlineto{\pgfqpoint{4.646664in}{2.574044in}}%
\pgfpathlineto{\pgfqpoint{4.639071in}{2.565314in}}%
\pgfpathlineto{\pgfqpoint{4.631472in}{2.556546in}}%
\pgfpathlineto{\pgfqpoint{4.623868in}{2.547739in}}%
\pgfpathlineto{\pgfqpoint{4.616258in}{2.538893in}}%
\pgfpathlineto{\pgfqpoint{4.602586in}{2.537808in}}%
\pgfpathlineto{\pgfqpoint{4.588923in}{2.536828in}}%
\pgfpathlineto{\pgfqpoint{4.575270in}{2.535953in}}%
\pgfpathlineto{\pgfqpoint{4.561626in}{2.535183in}}%
\pgfpathlineto{\pgfqpoint{4.569245in}{2.544010in}}%
\pgfpathlineto{\pgfqpoint{4.576859in}{2.552804in}}%
\pgfpathlineto{\pgfqpoint{4.584467in}{2.561564in}}%
\pgfpathlineto{\pgfqpoint{4.592069in}{2.570290in}}%
\pgfpathclose%
\pgfusepath{fill}%
\end{pgfscope}%
\begin{pgfscope}%
\pgfpathrectangle{\pgfqpoint{1.150000in}{0.150000in}}{\pgfqpoint{5.700000in}{5.700000in}}%
\pgfusepath{clip}%
\pgfsetbuttcap%
\pgfsetroundjoin%
\definecolor{currentfill}{rgb}{0.283197,0.115680,0.436115}%
\pgfsetfillcolor{currentfill}%
\pgfsetfillopacity{0.700000}%
\pgfsetlinewidth{0.000000pt}%
\definecolor{currentstroke}{rgb}{0.000000,0.000000,0.000000}%
\pgfsetstrokecolor{currentstroke}%
\pgfsetdash{}{0pt}%
\pgfpathmoveto{\pgfqpoint{3.140789in}{2.537576in}}%
\pgfpathlineto{\pgfqpoint{3.154175in}{2.527140in}}%
\pgfpathlineto{\pgfqpoint{3.167560in}{2.516857in}}%
\pgfpathlineto{\pgfqpoint{3.180946in}{2.506726in}}%
\pgfpathlineto{\pgfqpoint{3.194332in}{2.496747in}}%
\pgfpathlineto{\pgfqpoint{3.186188in}{2.491179in}}%
\pgfpathlineto{\pgfqpoint{3.178035in}{2.485707in}}%
\pgfpathlineto{\pgfqpoint{3.169874in}{2.480331in}}%
\pgfpathlineto{\pgfqpoint{3.161705in}{2.475054in}}%
\pgfpathlineto{\pgfqpoint{3.148297in}{2.485266in}}%
\pgfpathlineto{\pgfqpoint{3.134889in}{2.495630in}}%
\pgfpathlineto{\pgfqpoint{3.121481in}{2.506146in}}%
\pgfpathlineto{\pgfqpoint{3.108073in}{2.516816in}}%
\pgfpathlineto{\pgfqpoint{3.116265in}{2.521852in}}%
\pgfpathlineto{\pgfqpoint{3.124449in}{2.526993in}}%
\pgfpathlineto{\pgfqpoint{3.132623in}{2.532235in}}%
\pgfpathlineto{\pgfqpoint{3.140789in}{2.537576in}}%
\pgfpathclose%
\pgfusepath{fill}%
\end{pgfscope}%
\begin{pgfscope}%
\pgfpathrectangle{\pgfqpoint{1.150000in}{0.150000in}}{\pgfqpoint{5.700000in}{5.700000in}}%
\pgfusepath{clip}%
\pgfsetbuttcap%
\pgfsetroundjoin%
\definecolor{currentfill}{rgb}{0.283229,0.120777,0.440584}%
\pgfsetfillcolor{currentfill}%
\pgfsetfillopacity{0.700000}%
\pgfsetlinewidth{0.000000pt}%
\definecolor{currentstroke}{rgb}{0.000000,0.000000,0.000000}%
\pgfsetstrokecolor{currentstroke}%
\pgfsetdash{}{0pt}%
\pgfpathmoveto{\pgfqpoint{4.507144in}{2.533158in}}%
\pgfpathlineto{\pgfqpoint{4.520751in}{2.533505in}}%
\pgfpathlineto{\pgfqpoint{4.534367in}{2.533959in}}%
\pgfpathlineto{\pgfqpoint{4.547992in}{2.534518in}}%
\pgfpathlineto{\pgfqpoint{4.561626in}{2.535183in}}%
\pgfpathlineto{\pgfqpoint{4.554001in}{2.526321in}}%
\pgfpathlineto{\pgfqpoint{4.546371in}{2.517424in}}%
\pgfpathlineto{\pgfqpoint{4.538736in}{2.508494in}}%
\pgfpathlineto{\pgfqpoint{4.531095in}{2.499529in}}%
\pgfpathlineto{\pgfqpoint{4.517451in}{2.498893in}}%
\pgfpathlineto{\pgfqpoint{4.503817in}{2.498363in}}%
\pgfpathlineto{\pgfqpoint{4.490192in}{2.497938in}}%
\pgfpathlineto{\pgfqpoint{4.476576in}{2.497620in}}%
\pgfpathlineto{\pgfqpoint{4.484226in}{2.506549in}}%
\pgfpathlineto{\pgfqpoint{4.491871in}{2.515449in}}%
\pgfpathlineto{\pgfqpoint{4.499510in}{2.524318in}}%
\pgfpathlineto{\pgfqpoint{4.507144in}{2.533158in}}%
\pgfpathclose%
\pgfusepath{fill}%
\end{pgfscope}%
\begin{pgfscope}%
\pgfpathrectangle{\pgfqpoint{1.150000in}{0.150000in}}{\pgfqpoint{5.700000in}{5.700000in}}%
\pgfusepath{clip}%
\pgfsetbuttcap%
\pgfsetroundjoin%
\definecolor{currentfill}{rgb}{0.272594,0.025563,0.353093}%
\pgfsetfillcolor{currentfill}%
\pgfsetfillopacity{0.700000}%
\pgfsetlinewidth{0.000000pt}%
\definecolor{currentstroke}{rgb}{0.000000,0.000000,0.000000}%
\pgfsetstrokecolor{currentstroke}%
\pgfsetdash{}{0pt}%
\pgfpathmoveto{\pgfqpoint{3.665338in}{2.364499in}}%
\pgfpathlineto{\pgfqpoint{3.678737in}{2.359072in}}%
\pgfpathlineto{\pgfqpoint{3.692140in}{2.353769in}}%
\pgfpathlineto{\pgfqpoint{3.705547in}{2.348591in}}%
\pgfpathlineto{\pgfqpoint{3.718959in}{2.343536in}}%
\pgfpathlineto{\pgfqpoint{3.711040in}{2.335626in}}%
\pgfpathlineto{\pgfqpoint{3.703116in}{2.327753in}}%
\pgfpathlineto{\pgfqpoint{3.695185in}{2.319920in}}%
\pgfpathlineto{\pgfqpoint{3.687248in}{2.312126in}}%
\pgfpathlineto{\pgfqpoint{3.673822in}{2.317355in}}%
\pgfpathlineto{\pgfqpoint{3.660400in}{2.322707in}}%
\pgfpathlineto{\pgfqpoint{3.646982in}{2.328184in}}%
\pgfpathlineto{\pgfqpoint{3.633568in}{2.333786in}}%
\pgfpathlineto{\pgfqpoint{3.641520in}{2.341399in}}%
\pgfpathlineto{\pgfqpoint{3.649466in}{2.349056in}}%
\pgfpathlineto{\pgfqpoint{3.657405in}{2.356757in}}%
\pgfpathlineto{\pgfqpoint{3.665338in}{2.364499in}}%
\pgfpathclose%
\pgfusepath{fill}%
\end{pgfscope}%
\begin{pgfscope}%
\pgfpathrectangle{\pgfqpoint{1.150000in}{0.150000in}}{\pgfqpoint{5.700000in}{5.700000in}}%
\pgfusepath{clip}%
\pgfsetbuttcap%
\pgfsetroundjoin%
\definecolor{currentfill}{rgb}{0.273809,0.031497,0.358853}%
\pgfsetfillcolor{currentfill}%
\pgfsetfillopacity{0.700000}%
\pgfsetlinewidth{0.000000pt}%
\definecolor{currentstroke}{rgb}{0.000000,0.000000,0.000000}%
\pgfsetstrokecolor{currentstroke}%
\pgfsetdash{}{0pt}%
\pgfpathmoveto{\pgfqpoint{3.526394in}{2.383191in}}%
\pgfpathlineto{\pgfqpoint{3.539778in}{2.376563in}}%
\pgfpathlineto{\pgfqpoint{3.553166in}{2.370065in}}%
\pgfpathlineto{\pgfqpoint{3.566558in}{2.363698in}}%
\pgfpathlineto{\pgfqpoint{3.579953in}{2.357460in}}%
\pgfpathlineto{\pgfqpoint{3.571979in}{2.350080in}}%
\pgfpathlineto{\pgfqpoint{3.563999in}{2.342752in}}%
\pgfpathlineto{\pgfqpoint{3.556012in}{2.335478in}}%
\pgfpathlineto{\pgfqpoint{3.548019in}{2.328261in}}%
\pgfpathlineto{\pgfqpoint{3.534608in}{2.334692in}}%
\pgfpathlineto{\pgfqpoint{3.521200in}{2.341252in}}%
\pgfpathlineto{\pgfqpoint{3.507795in}{2.347942in}}%
\pgfpathlineto{\pgfqpoint{3.494394in}{2.354763in}}%
\pgfpathlineto{\pgfqpoint{3.502404in}{2.361781in}}%
\pgfpathlineto{\pgfqpoint{3.510407in}{2.368859in}}%
\pgfpathlineto{\pgfqpoint{3.518404in}{2.375997in}}%
\pgfpathlineto{\pgfqpoint{3.526394in}{2.383191in}}%
\pgfpathclose%
\pgfusepath{fill}%
\end{pgfscope}%
\begin{pgfscope}%
\pgfpathrectangle{\pgfqpoint{1.150000in}{0.150000in}}{\pgfqpoint{5.700000in}{5.700000in}}%
\pgfusepath{clip}%
\pgfsetbuttcap%
\pgfsetroundjoin%
\definecolor{currentfill}{rgb}{0.246811,0.283237,0.535941}%
\pgfsetfillcolor{currentfill}%
\pgfsetfillopacity{0.700000}%
\pgfsetlinewidth{0.000000pt}%
\definecolor{currentstroke}{rgb}{0.000000,0.000000,0.000000}%
\pgfsetstrokecolor{currentstroke}%
\pgfsetdash{}{0pt}%
\pgfpathmoveto{\pgfqpoint{2.731919in}{2.883210in}}%
\pgfpathlineto{\pgfqpoint{2.745397in}{2.867689in}}%
\pgfpathlineto{\pgfqpoint{2.758871in}{2.852363in}}%
\pgfpathlineto{\pgfqpoint{2.772340in}{2.837230in}}%
\pgfpathlineto{\pgfqpoint{2.785805in}{2.822287in}}%
\pgfpathlineto{\pgfqpoint{2.777447in}{2.818849in}}%
\pgfpathlineto{\pgfqpoint{2.769077in}{2.815548in}}%
\pgfpathlineto{\pgfqpoint{2.760697in}{2.812385in}}%
\pgfpathlineto{\pgfqpoint{2.752306in}{2.809364in}}%
\pgfpathlineto{\pgfqpoint{2.738812in}{2.824567in}}%
\pgfpathlineto{\pgfqpoint{2.725313in}{2.839961in}}%
\pgfpathlineto{\pgfqpoint{2.711809in}{2.855548in}}%
\pgfpathlineto{\pgfqpoint{2.698301in}{2.871331in}}%
\pgfpathlineto{\pgfqpoint{2.706722in}{2.874084in}}%
\pgfpathlineto{\pgfqpoint{2.715132in}{2.876983in}}%
\pgfpathlineto{\pgfqpoint{2.723531in}{2.880026in}}%
\pgfpathlineto{\pgfqpoint{2.731919in}{2.883210in}}%
\pgfpathclose%
\pgfusepath{fill}%
\end{pgfscope}%
\begin{pgfscope}%
\pgfpathrectangle{\pgfqpoint{1.150000in}{0.150000in}}{\pgfqpoint{5.700000in}{5.700000in}}%
\pgfusepath{clip}%
\pgfsetbuttcap%
\pgfsetroundjoin%
\definecolor{currentfill}{rgb}{0.257322,0.256130,0.526563}%
\pgfsetfillcolor{currentfill}%
\pgfsetfillopacity{0.700000}%
\pgfsetlinewidth{0.000000pt}%
\definecolor{currentstroke}{rgb}{0.000000,0.000000,0.000000}%
\pgfsetstrokecolor{currentstroke}%
\pgfsetdash{}{0pt}%
\pgfpathmoveto{\pgfqpoint{2.785805in}{2.822287in}}%
\pgfpathlineto{\pgfqpoint{2.799265in}{2.807535in}}%
\pgfpathlineto{\pgfqpoint{2.812721in}{2.792969in}}%
\pgfpathlineto{\pgfqpoint{2.826174in}{2.778590in}}%
\pgfpathlineto{\pgfqpoint{2.839622in}{2.764395in}}%
\pgfpathlineto{\pgfqpoint{2.831293in}{2.760705in}}%
\pgfpathlineto{\pgfqpoint{2.822952in}{2.757146in}}%
\pgfpathlineto{\pgfqpoint{2.814602in}{2.753722in}}%
\pgfpathlineto{\pgfqpoint{2.806240in}{2.750434in}}%
\pgfpathlineto{\pgfqpoint{2.792763in}{2.764888in}}%
\pgfpathlineto{\pgfqpoint{2.779281in}{2.779526in}}%
\pgfpathlineto{\pgfqpoint{2.765796in}{2.794351in}}%
\pgfpathlineto{\pgfqpoint{2.752306in}{2.809364in}}%
\pgfpathlineto{\pgfqpoint{2.760697in}{2.812385in}}%
\pgfpathlineto{\pgfqpoint{2.769077in}{2.815548in}}%
\pgfpathlineto{\pgfqpoint{2.777447in}{2.818849in}}%
\pgfpathlineto{\pgfqpoint{2.785805in}{2.822287in}}%
\pgfpathclose%
\pgfusepath{fill}%
\end{pgfscope}%
\begin{pgfscope}%
\pgfpathrectangle{\pgfqpoint{1.150000in}{0.150000in}}{\pgfqpoint{5.700000in}{5.700000in}}%
\pgfusepath{clip}%
\pgfsetbuttcap%
\pgfsetroundjoin%
\definecolor{currentfill}{rgb}{0.282910,0.105393,0.426902}%
\pgfsetfillcolor{currentfill}%
\pgfsetfillopacity{0.700000}%
\pgfsetlinewidth{0.000000pt}%
\definecolor{currentstroke}{rgb}{0.000000,0.000000,0.000000}%
\pgfsetstrokecolor{currentstroke}%
\pgfsetdash{}{0pt}%
\pgfpathmoveto{\pgfqpoint{4.422201in}{2.497414in}}%
\pgfpathlineto{\pgfqpoint{4.435782in}{2.497305in}}%
\pgfpathlineto{\pgfqpoint{4.449371in}{2.497304in}}%
\pgfpathlineto{\pgfqpoint{4.462969in}{2.497409in}}%
\pgfpathlineto{\pgfqpoint{4.476576in}{2.497620in}}%
\pgfpathlineto{\pgfqpoint{4.468921in}{2.488661in}}%
\pgfpathlineto{\pgfqpoint{4.461260in}{2.479673in}}%
\pgfpathlineto{\pgfqpoint{4.453593in}{2.470655in}}%
\pgfpathlineto{\pgfqpoint{4.445922in}{2.461608in}}%
\pgfpathlineto{\pgfqpoint{4.432306in}{2.461443in}}%
\pgfpathlineto{\pgfqpoint{4.418698in}{2.461385in}}%
\pgfpathlineto{\pgfqpoint{4.405100in}{2.461434in}}%
\pgfpathlineto{\pgfqpoint{4.391510in}{2.461590in}}%
\pgfpathlineto{\pgfqpoint{4.399191in}{2.470584in}}%
\pgfpathlineto{\pgfqpoint{4.406866in}{2.479552in}}%
\pgfpathlineto{\pgfqpoint{4.414536in}{2.488496in}}%
\pgfpathlineto{\pgfqpoint{4.422201in}{2.497414in}}%
\pgfpathclose%
\pgfusepath{fill}%
\end{pgfscope}%
\begin{pgfscope}%
\pgfpathrectangle{\pgfqpoint{1.150000in}{0.150000in}}{\pgfqpoint{5.700000in}{5.700000in}}%
\pgfusepath{clip}%
\pgfsetbuttcap%
\pgfsetroundjoin%
\definecolor{currentfill}{rgb}{0.274952,0.037752,0.364543}%
\pgfsetfillcolor{currentfill}%
\pgfsetfillopacity{0.700000}%
\pgfsetlinewidth{0.000000pt}%
\definecolor{currentstroke}{rgb}{0.000000,0.000000,0.000000}%
\pgfsetstrokecolor{currentstroke}%
\pgfsetdash{}{0pt}%
\pgfpathmoveto{\pgfqpoint{4.028201in}{2.383018in}}%
\pgfpathlineto{\pgfqpoint{4.041670in}{2.380397in}}%
\pgfpathlineto{\pgfqpoint{4.055145in}{2.377890in}}%
\pgfpathlineto{\pgfqpoint{4.068626in}{2.375497in}}%
\pgfpathlineto{\pgfqpoint{4.082115in}{2.373217in}}%
\pgfpathlineto{\pgfqpoint{4.074325in}{2.364392in}}%
\pgfpathlineto{\pgfqpoint{4.066530in}{2.355567in}}%
\pgfpathlineto{\pgfqpoint{4.058730in}{2.346745in}}%
\pgfpathlineto{\pgfqpoint{4.050924in}{2.337924in}}%
\pgfpathlineto{\pgfqpoint{4.037425in}{2.340323in}}%
\pgfpathlineto{\pgfqpoint{4.023932in}{2.342836in}}%
\pgfpathlineto{\pgfqpoint{4.010446in}{2.345462in}}%
\pgfpathlineto{\pgfqpoint{3.996967in}{2.348203in}}%
\pgfpathlineto{\pgfqpoint{4.004783in}{2.356897in}}%
\pgfpathlineto{\pgfqpoint{4.012595in}{2.365598in}}%
\pgfpathlineto{\pgfqpoint{4.020401in}{2.374305in}}%
\pgfpathlineto{\pgfqpoint{4.028201in}{2.383018in}}%
\pgfpathclose%
\pgfusepath{fill}%
\end{pgfscope}%
\begin{pgfscope}%
\pgfpathrectangle{\pgfqpoint{1.150000in}{0.150000in}}{\pgfqpoint{5.700000in}{5.700000in}}%
\pgfusepath{clip}%
\pgfsetbuttcap%
\pgfsetroundjoin%
\definecolor{currentfill}{rgb}{0.271305,0.019942,0.347269}%
\pgfsetfillcolor{currentfill}%
\pgfsetfillopacity{0.700000}%
\pgfsetlinewidth{0.000000pt}%
\definecolor{currentstroke}{rgb}{0.000000,0.000000,0.000000}%
\pgfsetstrokecolor{currentstroke}%
\pgfsetdash{}{0pt}%
\pgfpathmoveto{\pgfqpoint{3.804213in}{2.357155in}}%
\pgfpathlineto{\pgfqpoint{3.817636in}{2.352866in}}%
\pgfpathlineto{\pgfqpoint{3.831063in}{2.348698in}}%
\pgfpathlineto{\pgfqpoint{3.844496in}{2.344650in}}%
\pgfpathlineto{\pgfqpoint{3.857934in}{2.340721in}}%
\pgfpathlineto{\pgfqpoint{3.850065in}{2.332378in}}%
\pgfpathlineto{\pgfqpoint{3.842191in}{2.324057in}}%
\pgfpathlineto{\pgfqpoint{3.834311in}{2.315760in}}%
\pgfpathlineto{\pgfqpoint{3.826425in}{2.307489in}}%
\pgfpathlineto{\pgfqpoint{3.812974in}{2.311573in}}%
\pgfpathlineto{\pgfqpoint{3.799529in}{2.315777in}}%
\pgfpathlineto{\pgfqpoint{3.786088in}{2.320101in}}%
\pgfpathlineto{\pgfqpoint{3.772653in}{2.324545in}}%
\pgfpathlineto{\pgfqpoint{3.780552in}{2.332654in}}%
\pgfpathlineto{\pgfqpoint{3.788445in}{2.340792in}}%
\pgfpathlineto{\pgfqpoint{3.796332in}{2.348960in}}%
\pgfpathlineto{\pgfqpoint{3.804213in}{2.357155in}}%
\pgfpathclose%
\pgfusepath{fill}%
\end{pgfscope}%
\begin{pgfscope}%
\pgfpathrectangle{\pgfqpoint{1.150000in}{0.150000in}}{\pgfqpoint{5.700000in}{5.700000in}}%
\pgfusepath{clip}%
\pgfsetbuttcap%
\pgfsetroundjoin%
\definecolor{currentfill}{rgb}{0.266580,0.228262,0.514349}%
\pgfsetfillcolor{currentfill}%
\pgfsetfillopacity{0.700000}%
\pgfsetlinewidth{0.000000pt}%
\definecolor{currentstroke}{rgb}{0.000000,0.000000,0.000000}%
\pgfsetstrokecolor{currentstroke}%
\pgfsetdash{}{0pt}%
\pgfpathmoveto{\pgfqpoint{2.839622in}{2.764395in}}%
\pgfpathlineto{\pgfqpoint{2.853068in}{2.750382in}}%
\pgfpathlineto{\pgfqpoint{2.866509in}{2.736550in}}%
\pgfpathlineto{\pgfqpoint{2.879948in}{2.722898in}}%
\pgfpathlineto{\pgfqpoint{2.893384in}{2.709423in}}%
\pgfpathlineto{\pgfqpoint{2.885081in}{2.705482in}}%
\pgfpathlineto{\pgfqpoint{2.876769in}{2.701669in}}%
\pgfpathlineto{\pgfqpoint{2.868446in}{2.697984in}}%
\pgfpathlineto{\pgfqpoint{2.860113in}{2.694431in}}%
\pgfpathlineto{\pgfqpoint{2.846650in}{2.708163in}}%
\pgfpathlineto{\pgfqpoint{2.833183in}{2.722073in}}%
\pgfpathlineto{\pgfqpoint{2.819713in}{2.736163in}}%
\pgfpathlineto{\pgfqpoint{2.806240in}{2.750434in}}%
\pgfpathlineto{\pgfqpoint{2.814602in}{2.753722in}}%
\pgfpathlineto{\pgfqpoint{2.822952in}{2.757146in}}%
\pgfpathlineto{\pgfqpoint{2.831293in}{2.760705in}}%
\pgfpathlineto{\pgfqpoint{2.839622in}{2.764395in}}%
\pgfpathclose%
\pgfusepath{fill}%
\end{pgfscope}%
\begin{pgfscope}%
\pgfpathrectangle{\pgfqpoint{1.150000in}{0.150000in}}{\pgfqpoint{5.700000in}{5.700000in}}%
\pgfusepath{clip}%
\pgfsetbuttcap%
\pgfsetroundjoin%
\definecolor{currentfill}{rgb}{0.277941,0.056324,0.381191}%
\pgfsetfillcolor{currentfill}%
\pgfsetfillopacity{0.700000}%
\pgfsetlinewidth{0.000000pt}%
\definecolor{currentstroke}{rgb}{0.000000,0.000000,0.000000}%
\pgfsetstrokecolor{currentstroke}%
\pgfsetdash{}{0pt}%
\pgfpathmoveto{\pgfqpoint{3.387278in}{2.414135in}}%
\pgfpathlineto{\pgfqpoint{3.400659in}{2.406239in}}%
\pgfpathlineto{\pgfqpoint{3.414042in}{2.398480in}}%
\pgfpathlineto{\pgfqpoint{3.427428in}{2.390858in}}%
\pgfpathlineto{\pgfqpoint{3.440816in}{2.383371in}}%
\pgfpathlineto{\pgfqpoint{3.432781in}{2.376620in}}%
\pgfpathlineto{\pgfqpoint{3.424740in}{2.369938in}}%
\pgfpathlineto{\pgfqpoint{3.416691in}{2.363326in}}%
\pgfpathlineto{\pgfqpoint{3.408635in}{2.356787in}}%
\pgfpathlineto{\pgfqpoint{3.395229in}{2.364486in}}%
\pgfpathlineto{\pgfqpoint{3.381825in}{2.372320in}}%
\pgfpathlineto{\pgfqpoint{3.368423in}{2.380291in}}%
\pgfpathlineto{\pgfqpoint{3.355023in}{2.388399in}}%
\pgfpathlineto{\pgfqpoint{3.363098in}{2.394719in}}%
\pgfpathlineto{\pgfqpoint{3.371165in}{2.401117in}}%
\pgfpathlineto{\pgfqpoint{3.379225in}{2.407589in}}%
\pgfpathlineto{\pgfqpoint{3.387278in}{2.414135in}}%
\pgfpathclose%
\pgfusepath{fill}%
\end{pgfscope}%
\begin{pgfscope}%
\pgfpathrectangle{\pgfqpoint{1.150000in}{0.150000in}}{\pgfqpoint{5.700000in}{5.700000in}}%
\pgfusepath{clip}%
\pgfsetbuttcap%
\pgfsetroundjoin%
\definecolor{currentfill}{rgb}{0.282656,0.100196,0.422160}%
\pgfsetfillcolor{currentfill}%
\pgfsetfillopacity{0.700000}%
\pgfsetlinewidth{0.000000pt}%
\definecolor{currentstroke}{rgb}{0.000000,0.000000,0.000000}%
\pgfsetstrokecolor{currentstroke}%
\pgfsetdash{}{0pt}%
\pgfpathmoveto{\pgfqpoint{3.194332in}{2.496747in}}%
\pgfpathlineto{\pgfqpoint{3.207718in}{2.486917in}}%
\pgfpathlineto{\pgfqpoint{3.221104in}{2.477237in}}%
\pgfpathlineto{\pgfqpoint{3.234491in}{2.467704in}}%
\pgfpathlineto{\pgfqpoint{3.247879in}{2.458318in}}%
\pgfpathlineto{\pgfqpoint{3.239756in}{2.452525in}}%
\pgfpathlineto{\pgfqpoint{3.231626in}{2.446823in}}%
\pgfpathlineto{\pgfqpoint{3.223487in}{2.441212in}}%
\pgfpathlineto{\pgfqpoint{3.215339in}{2.435696in}}%
\pgfpathlineto{\pgfqpoint{3.201930in}{2.445314in}}%
\pgfpathlineto{\pgfqpoint{3.188522in}{2.455079in}}%
\pgfpathlineto{\pgfqpoint{3.175113in}{2.464992in}}%
\pgfpathlineto{\pgfqpoint{3.161705in}{2.475054in}}%
\pgfpathlineto{\pgfqpoint{3.169874in}{2.480331in}}%
\pgfpathlineto{\pgfqpoint{3.178035in}{2.485707in}}%
\pgfpathlineto{\pgfqpoint{3.186188in}{2.491179in}}%
\pgfpathlineto{\pgfqpoint{3.194332in}{2.496747in}}%
\pgfpathclose%
\pgfusepath{fill}%
\end{pgfscope}%
\begin{pgfscope}%
\pgfpathrectangle{\pgfqpoint{1.150000in}{0.150000in}}{\pgfqpoint{5.700000in}{5.700000in}}%
\pgfusepath{clip}%
\pgfsetbuttcap%
\pgfsetroundjoin%
\definecolor{currentfill}{rgb}{0.281924,0.089666,0.412415}%
\pgfsetfillcolor{currentfill}%
\pgfsetfillopacity{0.700000}%
\pgfsetlinewidth{0.000000pt}%
\definecolor{currentstroke}{rgb}{0.000000,0.000000,0.000000}%
\pgfsetstrokecolor{currentstroke}%
\pgfsetdash{}{0pt}%
\pgfpathmoveto{\pgfqpoint{4.337235in}{2.463294in}}%
\pgfpathlineto{\pgfqpoint{4.350791in}{2.462706in}}%
\pgfpathlineto{\pgfqpoint{4.364355in}{2.462226in}}%
\pgfpathlineto{\pgfqpoint{4.377928in}{2.461854in}}%
\pgfpathlineto{\pgfqpoint{4.391510in}{2.461590in}}%
\pgfpathlineto{\pgfqpoint{4.383823in}{2.452573in}}%
\pgfpathlineto{\pgfqpoint{4.376132in}{2.443530in}}%
\pgfpathlineto{\pgfqpoint{4.368435in}{2.434464in}}%
\pgfpathlineto{\pgfqpoint{4.360733in}{2.425375in}}%
\pgfpathlineto{\pgfqpoint{4.347142in}{2.425704in}}%
\pgfpathlineto{\pgfqpoint{4.333560in}{2.426141in}}%
\pgfpathlineto{\pgfqpoint{4.319986in}{2.426686in}}%
\pgfpathlineto{\pgfqpoint{4.306420in}{2.427340in}}%
\pgfpathlineto{\pgfqpoint{4.314132in}{2.436357in}}%
\pgfpathlineto{\pgfqpoint{4.321838in}{2.445356in}}%
\pgfpathlineto{\pgfqpoint{4.329539in}{2.454335in}}%
\pgfpathlineto{\pgfqpoint{4.337235in}{2.463294in}}%
\pgfpathclose%
\pgfusepath{fill}%
\end{pgfscope}%
\begin{pgfscope}%
\pgfpathrectangle{\pgfqpoint{1.150000in}{0.150000in}}{\pgfqpoint{5.700000in}{5.700000in}}%
\pgfusepath{clip}%
\pgfsetbuttcap%
\pgfsetroundjoin%
\definecolor{currentfill}{rgb}{0.273006,0.204520,0.501721}%
\pgfsetfillcolor{currentfill}%
\pgfsetfillopacity{0.700000}%
\pgfsetlinewidth{0.000000pt}%
\definecolor{currentstroke}{rgb}{0.000000,0.000000,0.000000}%
\pgfsetstrokecolor{currentstroke}%
\pgfsetdash{}{0pt}%
\pgfpathmoveto{\pgfqpoint{2.893384in}{2.709423in}}%
\pgfpathlineto{\pgfqpoint{2.906816in}{2.696125in}}%
\pgfpathlineto{\pgfqpoint{2.920246in}{2.683001in}}%
\pgfpathlineto{\pgfqpoint{2.933674in}{2.670050in}}%
\pgfpathlineto{\pgfqpoint{2.947099in}{2.657272in}}%
\pgfpathlineto{\pgfqpoint{2.938823in}{2.653081in}}%
\pgfpathlineto{\pgfqpoint{2.930538in}{2.649013in}}%
\pgfpathlineto{\pgfqpoint{2.922243in}{2.645070in}}%
\pgfpathlineto{\pgfqpoint{2.913937in}{2.641253in}}%
\pgfpathlineto{\pgfqpoint{2.900485in}{2.654288in}}%
\pgfpathlineto{\pgfqpoint{2.887031in}{2.667495in}}%
\pgfpathlineto{\pgfqpoint{2.873573in}{2.680875in}}%
\pgfpathlineto{\pgfqpoint{2.860113in}{2.694431in}}%
\pgfpathlineto{\pgfqpoint{2.868446in}{2.697984in}}%
\pgfpathlineto{\pgfqpoint{2.876769in}{2.701669in}}%
\pgfpathlineto{\pgfqpoint{2.885081in}{2.705482in}}%
\pgfpathlineto{\pgfqpoint{2.893384in}{2.709423in}}%
\pgfpathclose%
\pgfusepath{fill}%
\end{pgfscope}%
\begin{pgfscope}%
\pgfpathrectangle{\pgfqpoint{1.150000in}{0.150000in}}{\pgfqpoint{5.700000in}{5.700000in}}%
\pgfusepath{clip}%
\pgfsetbuttcap%
\pgfsetroundjoin%
\definecolor{currentfill}{rgb}{0.203063,0.379716,0.553925}%
\pgfsetfillcolor{currentfill}%
\pgfsetfillopacity{0.700000}%
\pgfsetlinewidth{0.000000pt}%
\definecolor{currentstroke}{rgb}{0.000000,0.000000,0.000000}%
\pgfsetstrokecolor{currentstroke}%
\pgfsetdash{}{0pt}%
\pgfpathmoveto{\pgfqpoint{5.666454in}{3.074032in}}%
\pgfpathlineto{\pgfqpoint{5.680539in}{3.078731in}}%
\pgfpathlineto{\pgfqpoint{5.694637in}{3.083527in}}%
\pgfpathlineto{\pgfqpoint{5.708749in}{3.088421in}}%
\pgfpathlineto{\pgfqpoint{5.722874in}{3.093412in}}%
\pgfpathlineto{\pgfqpoint{5.715749in}{3.088130in}}%
\pgfpathlineto{\pgfqpoint{5.708616in}{3.082818in}}%
\pgfpathlineto{\pgfqpoint{5.701478in}{3.077473in}}%
\pgfpathlineto{\pgfqpoint{5.694332in}{3.072094in}}%
\pgfpathlineto{\pgfqpoint{5.680188in}{3.066889in}}%
\pgfpathlineto{\pgfqpoint{5.666058in}{3.061781in}}%
\pgfpathlineto{\pgfqpoint{5.651942in}{3.056771in}}%
\pgfpathlineto{\pgfqpoint{5.637839in}{3.051858in}}%
\pgfpathlineto{\pgfqpoint{5.645003in}{3.057445in}}%
\pgfpathlineto{\pgfqpoint{5.652160in}{3.063001in}}%
\pgfpathlineto{\pgfqpoint{5.659310in}{3.068529in}}%
\pgfpathlineto{\pgfqpoint{5.666454in}{3.074032in}}%
\pgfpathclose%
\pgfusepath{fill}%
\end{pgfscope}%
\begin{pgfscope}%
\pgfpathrectangle{\pgfqpoint{1.150000in}{0.150000in}}{\pgfqpoint{5.700000in}{5.700000in}}%
\pgfusepath{clip}%
\pgfsetbuttcap%
\pgfsetroundjoin%
\definecolor{currentfill}{rgb}{0.210503,0.363727,0.552206}%
\pgfsetfillcolor{currentfill}%
\pgfsetfillopacity{0.700000}%
\pgfsetlinewidth{0.000000pt}%
\definecolor{currentstroke}{rgb}{0.000000,0.000000,0.000000}%
\pgfsetstrokecolor{currentstroke}%
\pgfsetdash{}{0pt}%
\pgfpathmoveto{\pgfqpoint{5.581563in}{3.033185in}}%
\pgfpathlineto{\pgfqpoint{5.595612in}{3.037707in}}%
\pgfpathlineto{\pgfqpoint{5.609674in}{3.042326in}}%
\pgfpathlineto{\pgfqpoint{5.623750in}{3.047044in}}%
\pgfpathlineto{\pgfqpoint{5.637839in}{3.051858in}}%
\pgfpathlineto{\pgfqpoint{5.630669in}{3.046239in}}%
\pgfpathlineto{\pgfqpoint{5.623492in}{3.040584in}}%
\pgfpathlineto{\pgfqpoint{5.616309in}{3.034892in}}%
\pgfpathlineto{\pgfqpoint{5.609118in}{3.029160in}}%
\pgfpathlineto{\pgfqpoint{5.595012in}{3.024149in}}%
\pgfpathlineto{\pgfqpoint{5.580919in}{3.019237in}}%
\pgfpathlineto{\pgfqpoint{5.566840in}{3.014423in}}%
\pgfpathlineto{\pgfqpoint{5.552774in}{3.009706in}}%
\pgfpathlineto{\pgfqpoint{5.559981in}{3.015627in}}%
\pgfpathlineto{\pgfqpoint{5.567182in}{3.021512in}}%
\pgfpathlineto{\pgfqpoint{5.574376in}{3.027364in}}%
\pgfpathlineto{\pgfqpoint{5.581563in}{3.033185in}}%
\pgfpathclose%
\pgfusepath{fill}%
\end{pgfscope}%
\begin{pgfscope}%
\pgfpathrectangle{\pgfqpoint{1.150000in}{0.150000in}}{\pgfqpoint{5.700000in}{5.700000in}}%
\pgfusepath{clip}%
\pgfsetbuttcap%
\pgfsetroundjoin%
\definecolor{currentfill}{rgb}{0.280267,0.073417,0.397163}%
\pgfsetfillcolor{currentfill}%
\pgfsetfillopacity{0.700000}%
\pgfsetlinewidth{0.000000pt}%
\definecolor{currentstroke}{rgb}{0.000000,0.000000,0.000000}%
\pgfsetstrokecolor{currentstroke}%
\pgfsetdash{}{0pt}%
\pgfpathmoveto{\pgfqpoint{4.252237in}{2.431045in}}%
\pgfpathlineto{\pgfqpoint{4.265771in}{2.429954in}}%
\pgfpathlineto{\pgfqpoint{4.279313in}{2.428973in}}%
\pgfpathlineto{\pgfqpoint{4.292862in}{2.428102in}}%
\pgfpathlineto{\pgfqpoint{4.306420in}{2.427340in}}%
\pgfpathlineto{\pgfqpoint{4.298703in}{2.418304in}}%
\pgfpathlineto{\pgfqpoint{4.290981in}{2.409249in}}%
\pgfpathlineto{\pgfqpoint{4.283253in}{2.400178in}}%
\pgfpathlineto{\pgfqpoint{4.275520in}{2.391089in}}%
\pgfpathlineto{\pgfqpoint{4.261953in}{2.391934in}}%
\pgfpathlineto{\pgfqpoint{4.248394in}{2.392889in}}%
\pgfpathlineto{\pgfqpoint{4.234843in}{2.393953in}}%
\pgfpathlineto{\pgfqpoint{4.221299in}{2.395127in}}%
\pgfpathlineto{\pgfqpoint{4.229042in}{2.404126in}}%
\pgfpathlineto{\pgfqpoint{4.236779in}{2.413112in}}%
\pgfpathlineto{\pgfqpoint{4.244511in}{2.422085in}}%
\pgfpathlineto{\pgfqpoint{4.252237in}{2.431045in}}%
\pgfpathclose%
\pgfusepath{fill}%
\end{pgfscope}%
\begin{pgfscope}%
\pgfpathrectangle{\pgfqpoint{1.150000in}{0.150000in}}{\pgfqpoint{5.700000in}{5.700000in}}%
\pgfusepath{clip}%
\pgfsetbuttcap%
\pgfsetroundjoin%
\definecolor{currentfill}{rgb}{0.278012,0.180367,0.486697}%
\pgfsetfillcolor{currentfill}%
\pgfsetfillopacity{0.700000}%
\pgfsetlinewidth{0.000000pt}%
\definecolor{currentstroke}{rgb}{0.000000,0.000000,0.000000}%
\pgfsetstrokecolor{currentstroke}%
\pgfsetdash{}{0pt}%
\pgfpathmoveto{\pgfqpoint{2.947099in}{2.657272in}}%
\pgfpathlineto{\pgfqpoint{2.960522in}{2.644663in}}%
\pgfpathlineto{\pgfqpoint{2.973943in}{2.632223in}}%
\pgfpathlineto{\pgfqpoint{2.987362in}{2.619951in}}%
\pgfpathlineto{\pgfqpoint{3.000779in}{2.607844in}}%
\pgfpathlineto{\pgfqpoint{2.992529in}{2.603406in}}%
\pgfpathlineto{\pgfqpoint{2.984270in}{2.599085in}}%
\pgfpathlineto{\pgfqpoint{2.976001in}{2.594885in}}%
\pgfpathlineto{\pgfqpoint{2.967722in}{2.590806in}}%
\pgfpathlineto{\pgfqpoint{2.954279in}{2.603167in}}%
\pgfpathlineto{\pgfqpoint{2.940834in}{2.615694in}}%
\pgfpathlineto{\pgfqpoint{2.927387in}{2.628389in}}%
\pgfpathlineto{\pgfqpoint{2.913937in}{2.641253in}}%
\pgfpathlineto{\pgfqpoint{2.922243in}{2.645070in}}%
\pgfpathlineto{\pgfqpoint{2.930538in}{2.649013in}}%
\pgfpathlineto{\pgfqpoint{2.938823in}{2.653081in}}%
\pgfpathlineto{\pgfqpoint{2.947099in}{2.657272in}}%
\pgfpathclose%
\pgfusepath{fill}%
\end{pgfscope}%
\begin{pgfscope}%
\pgfpathrectangle{\pgfqpoint{1.150000in}{0.150000in}}{\pgfqpoint{5.700000in}{5.700000in}}%
\pgfusepath{clip}%
\pgfsetbuttcap%
\pgfsetroundjoin%
\definecolor{currentfill}{rgb}{0.273809,0.031497,0.358853}%
\pgfsetfillcolor{currentfill}%
\pgfsetfillopacity{0.700000}%
\pgfsetlinewidth{0.000000pt}%
\definecolor{currentstroke}{rgb}{0.000000,0.000000,0.000000}%
\pgfsetstrokecolor{currentstroke}%
\pgfsetdash{}{0pt}%
\pgfpathmoveto{\pgfqpoint{3.943111in}{2.360319in}}%
\pgfpathlineto{\pgfqpoint{3.956566in}{2.357116in}}%
\pgfpathlineto{\pgfqpoint{3.970027in}{2.354030in}}%
\pgfpathlineto{\pgfqpoint{3.983493in}{2.351059in}}%
\pgfpathlineto{\pgfqpoint{3.996967in}{2.348203in}}%
\pgfpathlineto{\pgfqpoint{3.989144in}{2.339518in}}%
\pgfpathlineto{\pgfqpoint{3.981316in}{2.330841in}}%
\pgfpathlineto{\pgfqpoint{3.973483in}{2.322176in}}%
\pgfpathlineto{\pgfqpoint{3.965644in}{2.313521in}}%
\pgfpathlineto{\pgfqpoint{3.952160in}{2.316515in}}%
\pgfpathlineto{\pgfqpoint{3.938681in}{2.319623in}}%
\pgfpathlineto{\pgfqpoint{3.925209in}{2.322847in}}%
\pgfpathlineto{\pgfqpoint{3.911742in}{2.326188in}}%
\pgfpathlineto{\pgfqpoint{3.919593in}{2.334698in}}%
\pgfpathlineto{\pgfqpoint{3.927438in}{2.343224in}}%
\pgfpathlineto{\pgfqpoint{3.935277in}{2.351765in}}%
\pgfpathlineto{\pgfqpoint{3.943111in}{2.360319in}}%
\pgfpathclose%
\pgfusepath{fill}%
\end{pgfscope}%
\begin{pgfscope}%
\pgfpathrectangle{\pgfqpoint{1.150000in}{0.150000in}}{\pgfqpoint{5.700000in}{5.700000in}}%
\pgfusepath{clip}%
\pgfsetbuttcap%
\pgfsetroundjoin%
\definecolor{currentfill}{rgb}{0.218130,0.347432,0.550038}%
\pgfsetfillcolor{currentfill}%
\pgfsetfillopacity{0.700000}%
\pgfsetlinewidth{0.000000pt}%
\definecolor{currentstroke}{rgb}{0.000000,0.000000,0.000000}%
\pgfsetstrokecolor{currentstroke}%
\pgfsetdash{}{0pt}%
\pgfpathmoveto{\pgfqpoint{5.496643in}{2.991821in}}%
\pgfpathlineto{\pgfqpoint{5.510656in}{2.996145in}}%
\pgfpathlineto{\pgfqpoint{5.524682in}{3.000567in}}%
\pgfpathlineto{\pgfqpoint{5.538722in}{3.005088in}}%
\pgfpathlineto{\pgfqpoint{5.552774in}{3.009706in}}%
\pgfpathlineto{\pgfqpoint{5.545561in}{3.003748in}}%
\pgfpathlineto{\pgfqpoint{5.538340in}{2.997750in}}%
\pgfpathlineto{\pgfqpoint{5.531114in}{2.991710in}}%
\pgfpathlineto{\pgfqpoint{5.523880in}{2.985626in}}%
\pgfpathlineto{\pgfqpoint{5.509811in}{2.980831in}}%
\pgfpathlineto{\pgfqpoint{5.495756in}{2.976135in}}%
\pgfpathlineto{\pgfqpoint{5.481714in}{2.971536in}}%
\pgfpathlineto{\pgfqpoint{5.467685in}{2.967037in}}%
\pgfpathlineto{\pgfqpoint{5.474934in}{2.973290in}}%
\pgfpathlineto{\pgfqpoint{5.482177in}{2.979504in}}%
\pgfpathlineto{\pgfqpoint{5.489413in}{2.985680in}}%
\pgfpathlineto{\pgfqpoint{5.496643in}{2.991821in}}%
\pgfpathclose%
\pgfusepath{fill}%
\end{pgfscope}%
\begin{pgfscope}%
\pgfpathrectangle{\pgfqpoint{1.150000in}{0.150000in}}{\pgfqpoint{5.700000in}{5.700000in}}%
\pgfusepath{clip}%
\pgfsetbuttcap%
\pgfsetroundjoin%
\definecolor{currentfill}{rgb}{0.225863,0.330805,0.547314}%
\pgfsetfillcolor{currentfill}%
\pgfsetfillopacity{0.700000}%
\pgfsetlinewidth{0.000000pt}%
\definecolor{currentstroke}{rgb}{0.000000,0.000000,0.000000}%
\pgfsetstrokecolor{currentstroke}%
\pgfsetdash{}{0pt}%
\pgfpathmoveto{\pgfqpoint{5.411699in}{2.950022in}}%
\pgfpathlineto{\pgfqpoint{5.425676in}{2.954128in}}%
\pgfpathlineto{\pgfqpoint{5.439666in}{2.958332in}}%
\pgfpathlineto{\pgfqpoint{5.453669in}{2.962635in}}%
\pgfpathlineto{\pgfqpoint{5.467685in}{2.967037in}}%
\pgfpathlineto{\pgfqpoint{5.460429in}{2.960742in}}%
\pgfpathlineto{\pgfqpoint{5.453167in}{2.954403in}}%
\pgfpathlineto{\pgfqpoint{5.445897in}{2.948019in}}%
\pgfpathlineto{\pgfqpoint{5.438621in}{2.941588in}}%
\pgfpathlineto{\pgfqpoint{5.424591in}{2.937029in}}%
\pgfpathlineto{\pgfqpoint{5.410573in}{2.932568in}}%
\pgfpathlineto{\pgfqpoint{5.396568in}{2.928207in}}%
\pgfpathlineto{\pgfqpoint{5.382576in}{2.923944in}}%
\pgfpathlineto{\pgfqpoint{5.389866in}{2.930526in}}%
\pgfpathlineto{\pgfqpoint{5.397150in}{2.937065in}}%
\pgfpathlineto{\pgfqpoint{5.404428in}{2.943563in}}%
\pgfpathlineto{\pgfqpoint{5.411699in}{2.950022in}}%
\pgfpathclose%
\pgfusepath{fill}%
\end{pgfscope}%
\begin{pgfscope}%
\pgfpathrectangle{\pgfqpoint{1.150000in}{0.150000in}}{\pgfqpoint{5.700000in}{5.700000in}}%
\pgfusepath{clip}%
\pgfsetbuttcap%
\pgfsetroundjoin%
\definecolor{currentfill}{rgb}{0.235526,0.309527,0.542944}%
\pgfsetfillcolor{currentfill}%
\pgfsetfillopacity{0.700000}%
\pgfsetlinewidth{0.000000pt}%
\definecolor{currentstroke}{rgb}{0.000000,0.000000,0.000000}%
\pgfsetstrokecolor{currentstroke}%
\pgfsetdash{}{0pt}%
\pgfpathmoveto{\pgfqpoint{5.326735in}{2.907882in}}%
\pgfpathlineto{\pgfqpoint{5.340676in}{2.911749in}}%
\pgfpathlineto{\pgfqpoint{5.354630in}{2.915715in}}%
\pgfpathlineto{\pgfqpoint{5.368597in}{2.919780in}}%
\pgfpathlineto{\pgfqpoint{5.382576in}{2.923944in}}%
\pgfpathlineto{\pgfqpoint{5.375279in}{2.917318in}}%
\pgfpathlineto{\pgfqpoint{5.367975in}{2.910644in}}%
\pgfpathlineto{\pgfqpoint{5.360665in}{2.903923in}}%
\pgfpathlineto{\pgfqpoint{5.353348in}{2.897151in}}%
\pgfpathlineto{\pgfqpoint{5.339355in}{2.892849in}}%
\pgfpathlineto{\pgfqpoint{5.325375in}{2.888645in}}%
\pgfpathlineto{\pgfqpoint{5.311407in}{2.884541in}}%
\pgfpathlineto{\pgfqpoint{5.297452in}{2.880536in}}%
\pgfpathlineto{\pgfqpoint{5.304782in}{2.887439in}}%
\pgfpathlineto{\pgfqpoint{5.312106in}{2.894297in}}%
\pgfpathlineto{\pgfqpoint{5.319424in}{2.901111in}}%
\pgfpathlineto{\pgfqpoint{5.326735in}{2.907882in}}%
\pgfpathclose%
\pgfusepath{fill}%
\end{pgfscope}%
\begin{pgfscope}%
\pgfpathrectangle{\pgfqpoint{1.150000in}{0.150000in}}{\pgfqpoint{5.700000in}{5.700000in}}%
\pgfusepath{clip}%
\pgfsetbuttcap%
\pgfsetroundjoin%
\definecolor{currentfill}{rgb}{0.272594,0.025563,0.353093}%
\pgfsetfillcolor{currentfill}%
\pgfsetfillopacity{0.700000}%
\pgfsetlinewidth{0.000000pt}%
\definecolor{currentstroke}{rgb}{0.000000,0.000000,0.000000}%
\pgfsetstrokecolor{currentstroke}%
\pgfsetdash{}{0pt}%
\pgfpathmoveto{\pgfqpoint{3.579953in}{2.357460in}}%
\pgfpathlineto{\pgfqpoint{3.593351in}{2.351350in}}%
\pgfpathlineto{\pgfqpoint{3.606753in}{2.345369in}}%
\pgfpathlineto{\pgfqpoint{3.620159in}{2.339514in}}%
\pgfpathlineto{\pgfqpoint{3.633568in}{2.333786in}}%
\pgfpathlineto{\pgfqpoint{3.625610in}{2.326220in}}%
\pgfpathlineto{\pgfqpoint{3.617646in}{2.318702in}}%
\pgfpathlineto{\pgfqpoint{3.609675in}{2.311234in}}%
\pgfpathlineto{\pgfqpoint{3.601698in}{2.303817in}}%
\pgfpathlineto{\pgfqpoint{3.588272in}{2.309738in}}%
\pgfpathlineto{\pgfqpoint{3.574851in}{2.315785in}}%
\pgfpathlineto{\pgfqpoint{3.561433in}{2.321959in}}%
\pgfpathlineto{\pgfqpoint{3.548019in}{2.328261in}}%
\pgfpathlineto{\pgfqpoint{3.556012in}{2.335478in}}%
\pgfpathlineto{\pgfqpoint{3.563999in}{2.342752in}}%
\pgfpathlineto{\pgfqpoint{3.571979in}{2.350080in}}%
\pgfpathlineto{\pgfqpoint{3.579953in}{2.357460in}}%
\pgfpathclose%
\pgfusepath{fill}%
\end{pgfscope}%
\begin{pgfscope}%
\pgfpathrectangle{\pgfqpoint{1.150000in}{0.150000in}}{\pgfqpoint{5.700000in}{5.700000in}}%
\pgfusepath{clip}%
\pgfsetbuttcap%
\pgfsetroundjoin%
\definecolor{currentfill}{rgb}{0.243113,0.292092,0.538516}%
\pgfsetfillcolor{currentfill}%
\pgfsetfillopacity{0.700000}%
\pgfsetlinewidth{0.000000pt}%
\definecolor{currentstroke}{rgb}{0.000000,0.000000,0.000000}%
\pgfsetstrokecolor{currentstroke}%
\pgfsetdash{}{0pt}%
\pgfpathmoveto{\pgfqpoint{5.241755in}{2.865510in}}%
\pgfpathlineto{\pgfqpoint{5.255661in}{2.869117in}}%
\pgfpathlineto{\pgfqpoint{5.269579in}{2.872824in}}%
\pgfpathlineto{\pgfqpoint{5.283509in}{2.876631in}}%
\pgfpathlineto{\pgfqpoint{5.297452in}{2.880536in}}%
\pgfpathlineto{\pgfqpoint{5.290115in}{2.873586in}}%
\pgfpathlineto{\pgfqpoint{5.282771in}{2.866587in}}%
\pgfpathlineto{\pgfqpoint{5.275421in}{2.859537in}}%
\pgfpathlineto{\pgfqpoint{5.268065in}{2.852436in}}%
\pgfpathlineto{\pgfqpoint{5.254109in}{2.848411in}}%
\pgfpathlineto{\pgfqpoint{5.240166in}{2.844485in}}%
\pgfpathlineto{\pgfqpoint{5.226235in}{2.840659in}}%
\pgfpathlineto{\pgfqpoint{5.212316in}{2.836932in}}%
\pgfpathlineto{\pgfqpoint{5.219685in}{2.844146in}}%
\pgfpathlineto{\pgfqpoint{5.227048in}{2.851313in}}%
\pgfpathlineto{\pgfqpoint{5.234405in}{2.858433in}}%
\pgfpathlineto{\pgfqpoint{5.241755in}{2.865510in}}%
\pgfpathclose%
\pgfusepath{fill}%
\end{pgfscope}%
\begin{pgfscope}%
\pgfpathrectangle{\pgfqpoint{1.150000in}{0.150000in}}{\pgfqpoint{5.700000in}{5.700000in}}%
\pgfusepath{clip}%
\pgfsetbuttcap%
\pgfsetroundjoin%
\definecolor{currentfill}{rgb}{0.281446,0.084320,0.407414}%
\pgfsetfillcolor{currentfill}%
\pgfsetfillopacity{0.700000}%
\pgfsetlinewidth{0.000000pt}%
\definecolor{currentstroke}{rgb}{0.000000,0.000000,0.000000}%
\pgfsetstrokecolor{currentstroke}%
\pgfsetdash{}{0pt}%
\pgfpathmoveto{\pgfqpoint{3.247879in}{2.458318in}}%
\pgfpathlineto{\pgfqpoint{3.261268in}{2.449078in}}%
\pgfpathlineto{\pgfqpoint{3.274657in}{2.439983in}}%
\pgfpathlineto{\pgfqpoint{3.288048in}{2.431032in}}%
\pgfpathlineto{\pgfqpoint{3.301440in}{2.422223in}}%
\pgfpathlineto{\pgfqpoint{3.293338in}{2.416206in}}%
\pgfpathlineto{\pgfqpoint{3.285228in}{2.410274in}}%
\pgfpathlineto{\pgfqpoint{3.277110in}{2.404429in}}%
\pgfpathlineto{\pgfqpoint{3.268984in}{2.398675in}}%
\pgfpathlineto{\pgfqpoint{3.255572in}{2.407715in}}%
\pgfpathlineto{\pgfqpoint{3.242160in}{2.416897in}}%
\pgfpathlineto{\pgfqpoint{3.228749in}{2.426224in}}%
\pgfpathlineto{\pgfqpoint{3.215339in}{2.435696in}}%
\pgfpathlineto{\pgfqpoint{3.223487in}{2.441212in}}%
\pgfpathlineto{\pgfqpoint{3.231626in}{2.446823in}}%
\pgfpathlineto{\pgfqpoint{3.239756in}{2.452525in}}%
\pgfpathlineto{\pgfqpoint{3.247879in}{2.458318in}}%
\pgfpathclose%
\pgfusepath{fill}%
\end{pgfscope}%
\begin{pgfscope}%
\pgfpathrectangle{\pgfqpoint{1.150000in}{0.150000in}}{\pgfqpoint{5.700000in}{5.700000in}}%
\pgfusepath{clip}%
\pgfsetbuttcap%
\pgfsetroundjoin%
\definecolor{currentfill}{rgb}{0.250425,0.274290,0.533103}%
\pgfsetfillcolor{currentfill}%
\pgfsetfillopacity{0.700000}%
\pgfsetlinewidth{0.000000pt}%
\definecolor{currentstroke}{rgb}{0.000000,0.000000,0.000000}%
\pgfsetstrokecolor{currentstroke}%
\pgfsetdash{}{0pt}%
\pgfpathmoveto{\pgfqpoint{5.156762in}{2.823025in}}%
\pgfpathlineto{\pgfqpoint{5.170633in}{2.826352in}}%
\pgfpathlineto{\pgfqpoint{5.184515in}{2.829779in}}%
\pgfpathlineto{\pgfqpoint{5.198410in}{2.833305in}}%
\pgfpathlineto{\pgfqpoint{5.212316in}{2.836932in}}%
\pgfpathlineto{\pgfqpoint{5.204941in}{2.829669in}}%
\pgfpathlineto{\pgfqpoint{5.197558in}{2.822356in}}%
\pgfpathlineto{\pgfqpoint{5.190170in}{2.814992in}}%
\pgfpathlineto{\pgfqpoint{5.182775in}{2.807575in}}%
\pgfpathlineto{\pgfqpoint{5.168856in}{2.803847in}}%
\pgfpathlineto{\pgfqpoint{5.154950in}{2.800219in}}%
\pgfpathlineto{\pgfqpoint{5.141055in}{2.796692in}}%
\pgfpathlineto{\pgfqpoint{5.127173in}{2.793264in}}%
\pgfpathlineto{\pgfqpoint{5.134580in}{2.800776in}}%
\pgfpathlineto{\pgfqpoint{5.141980in}{2.808239in}}%
\pgfpathlineto{\pgfqpoint{5.149374in}{2.815655in}}%
\pgfpathlineto{\pgfqpoint{5.156762in}{2.823025in}}%
\pgfpathclose%
\pgfusepath{fill}%
\end{pgfscope}%
\begin{pgfscope}%
\pgfpathrectangle{\pgfqpoint{1.150000in}{0.150000in}}{\pgfqpoint{5.700000in}{5.700000in}}%
\pgfusepath{clip}%
\pgfsetbuttcap%
\pgfsetroundjoin%
\definecolor{currentfill}{rgb}{0.271305,0.019942,0.347269}%
\pgfsetfillcolor{currentfill}%
\pgfsetfillopacity{0.700000}%
\pgfsetlinewidth{0.000000pt}%
\definecolor{currentstroke}{rgb}{0.000000,0.000000,0.000000}%
\pgfsetstrokecolor{currentstroke}%
\pgfsetdash{}{0pt}%
\pgfpathmoveto{\pgfqpoint{3.718959in}{2.343536in}}%
\pgfpathlineto{\pgfqpoint{3.732375in}{2.338605in}}%
\pgfpathlineto{\pgfqpoint{3.745796in}{2.333796in}}%
\pgfpathlineto{\pgfqpoint{3.759222in}{2.329110in}}%
\pgfpathlineto{\pgfqpoint{3.772653in}{2.324545in}}%
\pgfpathlineto{\pgfqpoint{3.764748in}{2.316468in}}%
\pgfpathlineto{\pgfqpoint{3.756837in}{2.308424in}}%
\pgfpathlineto{\pgfqpoint{3.748920in}{2.300414in}}%
\pgfpathlineto{\pgfqpoint{3.740998in}{2.292439in}}%
\pgfpathlineto{\pgfqpoint{3.727553in}{2.297178in}}%
\pgfpathlineto{\pgfqpoint{3.714113in}{2.302039in}}%
\pgfpathlineto{\pgfqpoint{3.700678in}{2.307021in}}%
\pgfpathlineto{\pgfqpoint{3.687248in}{2.312126in}}%
\pgfpathlineto{\pgfqpoint{3.695185in}{2.319920in}}%
\pgfpathlineto{\pgfqpoint{3.703116in}{2.327753in}}%
\pgfpathlineto{\pgfqpoint{3.711040in}{2.335626in}}%
\pgfpathlineto{\pgfqpoint{3.718959in}{2.343536in}}%
\pgfpathclose%
\pgfusepath{fill}%
\end{pgfscope}%
\begin{pgfscope}%
\pgfpathrectangle{\pgfqpoint{1.150000in}{0.150000in}}{\pgfqpoint{5.700000in}{5.700000in}}%
\pgfusepath{clip}%
\pgfsetbuttcap%
\pgfsetroundjoin%
\definecolor{currentfill}{rgb}{0.258965,0.251537,0.524736}%
\pgfsetfillcolor{currentfill}%
\pgfsetfillopacity{0.700000}%
\pgfsetlinewidth{0.000000pt}%
\definecolor{currentstroke}{rgb}{0.000000,0.000000,0.000000}%
\pgfsetstrokecolor{currentstroke}%
\pgfsetdash{}{0pt}%
\pgfpathmoveto{\pgfqpoint{5.071760in}{2.780559in}}%
\pgfpathlineto{\pgfqpoint{5.085596in}{2.783585in}}%
\pgfpathlineto{\pgfqpoint{5.099443in}{2.786711in}}%
\pgfpathlineto{\pgfqpoint{5.113302in}{2.789937in}}%
\pgfpathlineto{\pgfqpoint{5.127173in}{2.793264in}}%
\pgfpathlineto{\pgfqpoint{5.119760in}{2.785703in}}%
\pgfpathlineto{\pgfqpoint{5.112340in}{2.778092in}}%
\pgfpathlineto{\pgfqpoint{5.104915in}{2.770428in}}%
\pgfpathlineto{\pgfqpoint{5.097483in}{2.762711in}}%
\pgfpathlineto{\pgfqpoint{5.083601in}{2.759301in}}%
\pgfpathlineto{\pgfqpoint{5.069730in}{2.755993in}}%
\pgfpathlineto{\pgfqpoint{5.055872in}{2.752785in}}%
\pgfpathlineto{\pgfqpoint{5.042025in}{2.749677in}}%
\pgfpathlineto{\pgfqpoint{5.049468in}{2.757470in}}%
\pgfpathlineto{\pgfqpoint{5.056905in}{2.765213in}}%
\pgfpathlineto{\pgfqpoint{5.064336in}{2.772909in}}%
\pgfpathlineto{\pgfqpoint{5.071760in}{2.780559in}}%
\pgfpathclose%
\pgfusepath{fill}%
\end{pgfscope}%
\begin{pgfscope}%
\pgfpathrectangle{\pgfqpoint{1.150000in}{0.150000in}}{\pgfqpoint{5.700000in}{5.700000in}}%
\pgfusepath{clip}%
\pgfsetbuttcap%
\pgfsetroundjoin%
\definecolor{currentfill}{rgb}{0.280868,0.160771,0.472899}%
\pgfsetfillcolor{currentfill}%
\pgfsetfillopacity{0.700000}%
\pgfsetlinewidth{0.000000pt}%
\definecolor{currentstroke}{rgb}{0.000000,0.000000,0.000000}%
\pgfsetstrokecolor{currentstroke}%
\pgfsetdash{}{0pt}%
\pgfpathmoveto{\pgfqpoint{3.000779in}{2.607844in}}%
\pgfpathlineto{\pgfqpoint{3.014195in}{2.595903in}}%
\pgfpathlineto{\pgfqpoint{3.027609in}{2.584125in}}%
\pgfpathlineto{\pgfqpoint{3.041022in}{2.572509in}}%
\pgfpathlineto{\pgfqpoint{3.054434in}{2.561053in}}%
\pgfpathlineto{\pgfqpoint{3.046209in}{2.556368in}}%
\pgfpathlineto{\pgfqpoint{3.037975in}{2.551796in}}%
\pgfpathlineto{\pgfqpoint{3.029731in}{2.547339in}}%
\pgfpathlineto{\pgfqpoint{3.021478in}{2.542999in}}%
\pgfpathlineto{\pgfqpoint{3.008041in}{2.554708in}}%
\pgfpathlineto{\pgfqpoint{2.994603in}{2.566578in}}%
\pgfpathlineto{\pgfqpoint{2.981163in}{2.578610in}}%
\pgfpathlineto{\pgfqpoint{2.967722in}{2.590806in}}%
\pgfpathlineto{\pgfqpoint{2.976001in}{2.594885in}}%
\pgfpathlineto{\pgfqpoint{2.984270in}{2.599085in}}%
\pgfpathlineto{\pgfqpoint{2.992529in}{2.603406in}}%
\pgfpathlineto{\pgfqpoint{3.000779in}{2.607844in}}%
\pgfpathclose%
\pgfusepath{fill}%
\end{pgfscope}%
\begin{pgfscope}%
\pgfpathrectangle{\pgfqpoint{1.150000in}{0.150000in}}{\pgfqpoint{5.700000in}{5.700000in}}%
\pgfusepath{clip}%
\pgfsetbuttcap%
\pgfsetroundjoin%
\definecolor{currentfill}{rgb}{0.265145,0.232956,0.516599}%
\pgfsetfillcolor{currentfill}%
\pgfsetfillopacity{0.700000}%
\pgfsetlinewidth{0.000000pt}%
\definecolor{currentstroke}{rgb}{0.000000,0.000000,0.000000}%
\pgfsetstrokecolor{currentstroke}%
\pgfsetdash{}{0pt}%
\pgfpathmoveto{\pgfqpoint{4.986751in}{2.738258in}}%
\pgfpathlineto{\pgfqpoint{5.000553in}{2.740961in}}%
\pgfpathlineto{\pgfqpoint{5.014365in}{2.743766in}}%
\pgfpathlineto{\pgfqpoint{5.028189in}{2.746671in}}%
\pgfpathlineto{\pgfqpoint{5.042025in}{2.749677in}}%
\pgfpathlineto{\pgfqpoint{5.034575in}{2.741836in}}%
\pgfpathlineto{\pgfqpoint{5.027120in}{2.733943in}}%
\pgfpathlineto{\pgfqpoint{5.019658in}{2.725998in}}%
\pgfpathlineto{\pgfqpoint{5.012190in}{2.718001in}}%
\pgfpathlineto{\pgfqpoint{4.998344in}{2.714931in}}%
\pgfpathlineto{\pgfqpoint{4.984509in}{2.711962in}}%
\pgfpathlineto{\pgfqpoint{4.970686in}{2.709095in}}%
\pgfpathlineto{\pgfqpoint{4.956874in}{2.706329in}}%
\pgfpathlineto{\pgfqpoint{4.964352in}{2.714382in}}%
\pgfpathlineto{\pgfqpoint{4.971825in}{2.722388in}}%
\pgfpathlineto{\pgfqpoint{4.979291in}{2.730346in}}%
\pgfpathlineto{\pgfqpoint{4.986751in}{2.738258in}}%
\pgfpathclose%
\pgfusepath{fill}%
\end{pgfscope}%
\begin{pgfscope}%
\pgfpathrectangle{\pgfqpoint{1.150000in}{0.150000in}}{\pgfqpoint{5.700000in}{5.700000in}}%
\pgfusepath{clip}%
\pgfsetbuttcap%
\pgfsetroundjoin%
\definecolor{currentfill}{rgb}{0.276022,0.044167,0.370164}%
\pgfsetfillcolor{currentfill}%
\pgfsetfillopacity{0.700000}%
\pgfsetlinewidth{0.000000pt}%
\definecolor{currentstroke}{rgb}{0.000000,0.000000,0.000000}%
\pgfsetstrokecolor{currentstroke}%
\pgfsetdash{}{0pt}%
\pgfpathmoveto{\pgfqpoint{3.440816in}{2.383371in}}%
\pgfpathlineto{\pgfqpoint{3.454206in}{2.376019in}}%
\pgfpathlineto{\pgfqpoint{3.467599in}{2.368801in}}%
\pgfpathlineto{\pgfqpoint{3.480995in}{2.361716in}}%
\pgfpathlineto{\pgfqpoint{3.494394in}{2.354763in}}%
\pgfpathlineto{\pgfqpoint{3.486377in}{2.347808in}}%
\pgfpathlineto{\pgfqpoint{3.478353in}{2.340916in}}%
\pgfpathlineto{\pgfqpoint{3.470322in}{2.334091in}}%
\pgfpathlineto{\pgfqpoint{3.462284in}{2.327333in}}%
\pgfpathlineto{\pgfqpoint{3.448868in}{2.334497in}}%
\pgfpathlineto{\pgfqpoint{3.435455in}{2.341794in}}%
\pgfpathlineto{\pgfqpoint{3.422044in}{2.349223in}}%
\pgfpathlineto{\pgfqpoint{3.408635in}{2.356787in}}%
\pgfpathlineto{\pgfqpoint{3.416691in}{2.363326in}}%
\pgfpathlineto{\pgfqpoint{3.424740in}{2.369938in}}%
\pgfpathlineto{\pgfqpoint{3.432781in}{2.376620in}}%
\pgfpathlineto{\pgfqpoint{3.440816in}{2.383371in}}%
\pgfpathclose%
\pgfusepath{fill}%
\end{pgfscope}%
\begin{pgfscope}%
\pgfpathrectangle{\pgfqpoint{1.150000in}{0.150000in}}{\pgfqpoint{5.700000in}{5.700000in}}%
\pgfusepath{clip}%
\pgfsetbuttcap%
\pgfsetroundjoin%
\definecolor{currentfill}{rgb}{0.277941,0.056324,0.381191}%
\pgfsetfillcolor{currentfill}%
\pgfsetfillopacity{0.700000}%
\pgfsetlinewidth{0.000000pt}%
\definecolor{currentstroke}{rgb}{0.000000,0.000000,0.000000}%
\pgfsetstrokecolor{currentstroke}%
\pgfsetdash{}{0pt}%
\pgfpathmoveto{\pgfqpoint{4.167201in}{2.400928in}}%
\pgfpathlineto{\pgfqpoint{4.180714in}{2.399311in}}%
\pgfpathlineto{\pgfqpoint{4.194235in}{2.397806in}}%
\pgfpathlineto{\pgfqpoint{4.207763in}{2.396411in}}%
\pgfpathlineto{\pgfqpoint{4.221299in}{2.395127in}}%
\pgfpathlineto{\pgfqpoint{4.213551in}{2.386117in}}%
\pgfpathlineto{\pgfqpoint{4.205798in}{2.377095in}}%
\pgfpathlineto{\pgfqpoint{4.198040in}{2.368063in}}%
\pgfpathlineto{\pgfqpoint{4.190276in}{2.359021in}}%
\pgfpathlineto{\pgfqpoint{4.176730in}{2.360406in}}%
\pgfpathlineto{\pgfqpoint{4.163192in}{2.361902in}}%
\pgfpathlineto{\pgfqpoint{4.149661in}{2.363509in}}%
\pgfpathlineto{\pgfqpoint{4.136138in}{2.365227in}}%
\pgfpathlineto{\pgfqpoint{4.143911in}{2.374161in}}%
\pgfpathlineto{\pgfqpoint{4.151680in}{2.383089in}}%
\pgfpathlineto{\pgfqpoint{4.159443in}{2.392012in}}%
\pgfpathlineto{\pgfqpoint{4.167201in}{2.400928in}}%
\pgfpathclose%
\pgfusepath{fill}%
\end{pgfscope}%
\begin{pgfscope}%
\pgfpathrectangle{\pgfqpoint{1.150000in}{0.150000in}}{\pgfqpoint{5.700000in}{5.700000in}}%
\pgfusepath{clip}%
\pgfsetbuttcap%
\pgfsetroundjoin%
\definecolor{currentfill}{rgb}{0.270595,0.214069,0.507052}%
\pgfsetfillcolor{currentfill}%
\pgfsetfillopacity{0.700000}%
\pgfsetlinewidth{0.000000pt}%
\definecolor{currentstroke}{rgb}{0.000000,0.000000,0.000000}%
\pgfsetstrokecolor{currentstroke}%
\pgfsetdash{}{0pt}%
\pgfpathmoveto{\pgfqpoint{4.901737in}{2.696280in}}%
\pgfpathlineto{\pgfqpoint{4.915505in}{2.698640in}}%
\pgfpathlineto{\pgfqpoint{4.929283in}{2.701101in}}%
\pgfpathlineto{\pgfqpoint{4.943073in}{2.703664in}}%
\pgfpathlineto{\pgfqpoint{4.956874in}{2.706329in}}%
\pgfpathlineto{\pgfqpoint{4.949389in}{2.698226in}}%
\pgfpathlineto{\pgfqpoint{4.941899in}{2.690073in}}%
\pgfpathlineto{\pgfqpoint{4.934402in}{2.681870in}}%
\pgfpathlineto{\pgfqpoint{4.926900in}{2.673615in}}%
\pgfpathlineto{\pgfqpoint{4.913089in}{2.670906in}}%
\pgfpathlineto{\pgfqpoint{4.899289in}{2.668298in}}%
\pgfpathlineto{\pgfqpoint{4.885500in}{2.665792in}}%
\pgfpathlineto{\pgfqpoint{4.871722in}{2.663388in}}%
\pgfpathlineto{\pgfqpoint{4.879234in}{2.671680in}}%
\pgfpathlineto{\pgfqpoint{4.886741in}{2.679926in}}%
\pgfpathlineto{\pgfqpoint{4.894242in}{2.688126in}}%
\pgfpathlineto{\pgfqpoint{4.901737in}{2.696280in}}%
\pgfpathclose%
\pgfusepath{fill}%
\end{pgfscope}%
\begin{pgfscope}%
\pgfpathrectangle{\pgfqpoint{1.150000in}{0.150000in}}{\pgfqpoint{5.700000in}{5.700000in}}%
\pgfusepath{clip}%
\pgfsetbuttcap%
\pgfsetroundjoin%
\definecolor{currentfill}{rgb}{0.275191,0.194905,0.496005}%
\pgfsetfillcolor{currentfill}%
\pgfsetfillopacity{0.700000}%
\pgfsetlinewidth{0.000000pt}%
\definecolor{currentstroke}{rgb}{0.000000,0.000000,0.000000}%
\pgfsetstrokecolor{currentstroke}%
\pgfsetdash{}{0pt}%
\pgfpathmoveto{\pgfqpoint{4.816718in}{2.654794in}}%
\pgfpathlineto{\pgfqpoint{4.830453in}{2.656789in}}%
\pgfpathlineto{\pgfqpoint{4.844198in}{2.658886in}}%
\pgfpathlineto{\pgfqpoint{4.857955in}{2.661086in}}%
\pgfpathlineto{\pgfqpoint{4.871722in}{2.663388in}}%
\pgfpathlineto{\pgfqpoint{4.864203in}{2.655047in}}%
\pgfpathlineto{\pgfqpoint{4.856679in}{2.646659in}}%
\pgfpathlineto{\pgfqpoint{4.849148in}{2.638221in}}%
\pgfpathlineto{\pgfqpoint{4.841612in}{2.629734in}}%
\pgfpathlineto{\pgfqpoint{4.827835in}{2.627406in}}%
\pgfpathlineto{\pgfqpoint{4.814069in}{2.625180in}}%
\pgfpathlineto{\pgfqpoint{4.800314in}{2.623057in}}%
\pgfpathlineto{\pgfqpoint{4.786569in}{2.621036in}}%
\pgfpathlineto{\pgfqpoint{4.794115in}{2.629543in}}%
\pgfpathlineto{\pgfqpoint{4.801655in}{2.638004in}}%
\pgfpathlineto{\pgfqpoint{4.809189in}{2.646421in}}%
\pgfpathlineto{\pgfqpoint{4.816718in}{2.654794in}}%
\pgfpathclose%
\pgfusepath{fill}%
\end{pgfscope}%
\begin{pgfscope}%
\pgfpathrectangle{\pgfqpoint{1.150000in}{0.150000in}}{\pgfqpoint{5.700000in}{5.700000in}}%
\pgfusepath{clip}%
\pgfsetbuttcap%
\pgfsetroundjoin%
\definecolor{currentfill}{rgb}{0.278826,0.175490,0.483397}%
\pgfsetfillcolor{currentfill}%
\pgfsetfillopacity{0.700000}%
\pgfsetlinewidth{0.000000pt}%
\definecolor{currentstroke}{rgb}{0.000000,0.000000,0.000000}%
\pgfsetstrokecolor{currentstroke}%
\pgfsetdash{}{0pt}%
\pgfpathmoveto{\pgfqpoint{4.731694in}{2.613984in}}%
\pgfpathlineto{\pgfqpoint{4.745397in}{2.615592in}}%
\pgfpathlineto{\pgfqpoint{4.759111in}{2.617304in}}%
\pgfpathlineto{\pgfqpoint{4.772835in}{2.619118in}}%
\pgfpathlineto{\pgfqpoint{4.786569in}{2.621036in}}%
\pgfpathlineto{\pgfqpoint{4.779017in}{2.612485in}}%
\pgfpathlineto{\pgfqpoint{4.771460in}{2.603887in}}%
\pgfpathlineto{\pgfqpoint{4.763897in}{2.595243in}}%
\pgfpathlineto{\pgfqpoint{4.756328in}{2.586553in}}%
\pgfpathlineto{\pgfqpoint{4.742584in}{2.584627in}}%
\pgfpathlineto{\pgfqpoint{4.728851in}{2.582804in}}%
\pgfpathlineto{\pgfqpoint{4.715128in}{2.581085in}}%
\pgfpathlineto{\pgfqpoint{4.701415in}{2.579469in}}%
\pgfpathlineto{\pgfqpoint{4.708993in}{2.588161in}}%
\pgfpathlineto{\pgfqpoint{4.716566in}{2.596810in}}%
\pgfpathlineto{\pgfqpoint{4.724133in}{2.605418in}}%
\pgfpathlineto{\pgfqpoint{4.731694in}{2.613984in}}%
\pgfpathclose%
\pgfusepath{fill}%
\end{pgfscope}%
\begin{pgfscope}%
\pgfpathrectangle{\pgfqpoint{1.150000in}{0.150000in}}{\pgfqpoint{5.700000in}{5.700000in}}%
\pgfusepath{clip}%
\pgfsetbuttcap%
\pgfsetroundjoin%
\definecolor{currentfill}{rgb}{0.271305,0.019942,0.347269}%
\pgfsetfillcolor{currentfill}%
\pgfsetfillopacity{0.700000}%
\pgfsetlinewidth{0.000000pt}%
\definecolor{currentstroke}{rgb}{0.000000,0.000000,0.000000}%
\pgfsetstrokecolor{currentstroke}%
\pgfsetdash{}{0pt}%
\pgfpathmoveto{\pgfqpoint{3.857934in}{2.340721in}}%
\pgfpathlineto{\pgfqpoint{3.871378in}{2.336911in}}%
\pgfpathlineto{\pgfqpoint{3.884827in}{2.333219in}}%
\pgfpathlineto{\pgfqpoint{3.898282in}{2.329645in}}%
\pgfpathlineto{\pgfqpoint{3.911742in}{2.326188in}}%
\pgfpathlineto{\pgfqpoint{3.903886in}{2.317695in}}%
\pgfpathlineto{\pgfqpoint{3.896024in}{2.309221in}}%
\pgfpathlineto{\pgfqpoint{3.888157in}{2.300766in}}%
\pgfpathlineto{\pgfqpoint{3.880284in}{2.292333in}}%
\pgfpathlineto{\pgfqpoint{3.866811in}{2.295945in}}%
\pgfpathlineto{\pgfqpoint{3.853343in}{2.299675in}}%
\pgfpathlineto{\pgfqpoint{3.839882in}{2.303523in}}%
\pgfpathlineto{\pgfqpoint{3.826425in}{2.307489in}}%
\pgfpathlineto{\pgfqpoint{3.834311in}{2.315760in}}%
\pgfpathlineto{\pgfqpoint{3.842191in}{2.324057in}}%
\pgfpathlineto{\pgfqpoint{3.850065in}{2.332378in}}%
\pgfpathlineto{\pgfqpoint{3.857934in}{2.340721in}}%
\pgfpathclose%
\pgfusepath{fill}%
\end{pgfscope}%
\begin{pgfscope}%
\pgfpathrectangle{\pgfqpoint{1.150000in}{0.150000in}}{\pgfqpoint{5.700000in}{5.700000in}}%
\pgfusepath{clip}%
\pgfsetbuttcap%
\pgfsetroundjoin%
\definecolor{currentfill}{rgb}{0.281412,0.155834,0.469201}%
\pgfsetfillcolor{currentfill}%
\pgfsetfillopacity{0.700000}%
\pgfsetlinewidth{0.000000pt}%
\definecolor{currentstroke}{rgb}{0.000000,0.000000,0.000000}%
\pgfsetstrokecolor{currentstroke}%
\pgfsetdash{}{0pt}%
\pgfpathmoveto{\pgfqpoint{4.646664in}{2.574044in}}%
\pgfpathlineto{\pgfqpoint{4.660337in}{2.575244in}}%
\pgfpathlineto{\pgfqpoint{4.674020in}{2.576549in}}%
\pgfpathlineto{\pgfqpoint{4.687712in}{2.577957in}}%
\pgfpathlineto{\pgfqpoint{4.701415in}{2.579469in}}%
\pgfpathlineto{\pgfqpoint{4.693831in}{2.570735in}}%
\pgfpathlineto{\pgfqpoint{4.686242in}{2.561959in}}%
\pgfpathlineto{\pgfqpoint{4.678646in}{2.553139in}}%
\pgfpathlineto{\pgfqpoint{4.671045in}{2.544277in}}%
\pgfpathlineto{\pgfqpoint{4.657334in}{2.542775in}}%
\pgfpathlineto{\pgfqpoint{4.643632in}{2.541377in}}%
\pgfpathlineto{\pgfqpoint{4.629940in}{2.540083in}}%
\pgfpathlineto{\pgfqpoint{4.616258in}{2.538893in}}%
\pgfpathlineto{\pgfqpoint{4.623868in}{2.547739in}}%
\pgfpathlineto{\pgfqpoint{4.631472in}{2.556546in}}%
\pgfpathlineto{\pgfqpoint{4.639071in}{2.565314in}}%
\pgfpathlineto{\pgfqpoint{4.646664in}{2.574044in}}%
\pgfpathclose%
\pgfusepath{fill}%
\end{pgfscope}%
\begin{pgfscope}%
\pgfpathrectangle{\pgfqpoint{1.150000in}{0.150000in}}{\pgfqpoint{5.700000in}{5.700000in}}%
\pgfusepath{clip}%
\pgfsetbuttcap%
\pgfsetroundjoin%
\definecolor{currentfill}{rgb}{0.282623,0.140926,0.457517}%
\pgfsetfillcolor{currentfill}%
\pgfsetfillopacity{0.700000}%
\pgfsetlinewidth{0.000000pt}%
\definecolor{currentstroke}{rgb}{0.000000,0.000000,0.000000}%
\pgfsetstrokecolor{currentstroke}%
\pgfsetdash{}{0pt}%
\pgfpathmoveto{\pgfqpoint{3.054434in}{2.561053in}}%
\pgfpathlineto{\pgfqpoint{3.067845in}{2.549758in}}%
\pgfpathlineto{\pgfqpoint{3.081255in}{2.538620in}}%
\pgfpathlineto{\pgfqpoint{3.094664in}{2.527640in}}%
\pgfpathlineto{\pgfqpoint{3.108073in}{2.516816in}}%
\pgfpathlineto{\pgfqpoint{3.099872in}{2.511885in}}%
\pgfpathlineto{\pgfqpoint{3.091662in}{2.507062in}}%
\pgfpathlineto{\pgfqpoint{3.083443in}{2.502350in}}%
\pgfpathlineto{\pgfqpoint{3.075215in}{2.497750in}}%
\pgfpathlineto{\pgfqpoint{3.061782in}{2.508827in}}%
\pgfpathlineto{\pgfqpoint{3.048348in}{2.520060in}}%
\pgfpathlineto{\pgfqpoint{3.034914in}{2.531450in}}%
\pgfpathlineto{\pgfqpoint{3.021478in}{2.542999in}}%
\pgfpathlineto{\pgfqpoint{3.029731in}{2.547339in}}%
\pgfpathlineto{\pgfqpoint{3.037975in}{2.551796in}}%
\pgfpathlineto{\pgfqpoint{3.046209in}{2.556368in}}%
\pgfpathlineto{\pgfqpoint{3.054434in}{2.561053in}}%
\pgfpathclose%
\pgfusepath{fill}%
\end{pgfscope}%
\begin{pgfscope}%
\pgfpathrectangle{\pgfqpoint{1.150000in}{0.150000in}}{\pgfqpoint{5.700000in}{5.700000in}}%
\pgfusepath{clip}%
\pgfsetbuttcap%
\pgfsetroundjoin%
\definecolor{currentfill}{rgb}{0.276022,0.044167,0.370164}%
\pgfsetfillcolor{currentfill}%
\pgfsetfillopacity{0.700000}%
\pgfsetlinewidth{0.000000pt}%
\definecolor{currentstroke}{rgb}{0.000000,0.000000,0.000000}%
\pgfsetstrokecolor{currentstroke}%
\pgfsetdash{}{0pt}%
\pgfpathmoveto{\pgfqpoint{4.082115in}{2.373217in}}%
\pgfpathlineto{\pgfqpoint{4.095610in}{2.371051in}}%
\pgfpathlineto{\pgfqpoint{4.109112in}{2.368997in}}%
\pgfpathlineto{\pgfqpoint{4.122621in}{2.367056in}}%
\pgfpathlineto{\pgfqpoint{4.136138in}{2.365227in}}%
\pgfpathlineto{\pgfqpoint{4.128358in}{2.356289in}}%
\pgfpathlineto{\pgfqpoint{4.120574in}{2.347347in}}%
\pgfpathlineto{\pgfqpoint{4.112784in}{2.338402in}}%
\pgfpathlineto{\pgfqpoint{4.104989in}{2.329456in}}%
\pgfpathlineto{\pgfqpoint{4.091462in}{2.331405in}}%
\pgfpathlineto{\pgfqpoint{4.077942in}{2.333466in}}%
\pgfpathlineto{\pgfqpoint{4.064430in}{2.335639in}}%
\pgfpathlineto{\pgfqpoint{4.050924in}{2.337924in}}%
\pgfpathlineto{\pgfqpoint{4.058730in}{2.346745in}}%
\pgfpathlineto{\pgfqpoint{4.066530in}{2.355567in}}%
\pgfpathlineto{\pgfqpoint{4.074325in}{2.364392in}}%
\pgfpathlineto{\pgfqpoint{4.082115in}{2.373217in}}%
\pgfpathclose%
\pgfusepath{fill}%
\end{pgfscope}%
\begin{pgfscope}%
\pgfpathrectangle{\pgfqpoint{1.150000in}{0.150000in}}{\pgfqpoint{5.700000in}{5.700000in}}%
\pgfusepath{clip}%
\pgfsetbuttcap%
\pgfsetroundjoin%
\definecolor{currentfill}{rgb}{0.282884,0.135920,0.453427}%
\pgfsetfillcolor{currentfill}%
\pgfsetfillopacity{0.700000}%
\pgfsetlinewidth{0.000000pt}%
\definecolor{currentstroke}{rgb}{0.000000,0.000000,0.000000}%
\pgfsetstrokecolor{currentstroke}%
\pgfsetdash{}{0pt}%
\pgfpathmoveto{\pgfqpoint{4.561626in}{2.535183in}}%
\pgfpathlineto{\pgfqpoint{4.575270in}{2.535953in}}%
\pgfpathlineto{\pgfqpoint{4.588923in}{2.536828in}}%
\pgfpathlineto{\pgfqpoint{4.602586in}{2.537808in}}%
\pgfpathlineto{\pgfqpoint{4.616258in}{2.538893in}}%
\pgfpathlineto{\pgfqpoint{4.608642in}{2.530009in}}%
\pgfpathlineto{\pgfqpoint{4.601021in}{2.521087in}}%
\pgfpathlineto{\pgfqpoint{4.593395in}{2.512125in}}%
\pgfpathlineto{\pgfqpoint{4.585763in}{2.503124in}}%
\pgfpathlineto{\pgfqpoint{4.572082in}{2.502068in}}%
\pgfpathlineto{\pgfqpoint{4.558410in}{2.501117in}}%
\pgfpathlineto{\pgfqpoint{4.544748in}{2.500270in}}%
\pgfpathlineto{\pgfqpoint{4.531095in}{2.499529in}}%
\pgfpathlineto{\pgfqpoint{4.538736in}{2.508494in}}%
\pgfpathlineto{\pgfqpoint{4.546371in}{2.517424in}}%
\pgfpathlineto{\pgfqpoint{4.554001in}{2.526321in}}%
\pgfpathlineto{\pgfqpoint{4.561626in}{2.535183in}}%
\pgfpathclose%
\pgfusepath{fill}%
\end{pgfscope}%
\begin{pgfscope}%
\pgfpathrectangle{\pgfqpoint{1.150000in}{0.150000in}}{\pgfqpoint{5.700000in}{5.700000in}}%
\pgfusepath{clip}%
\pgfsetbuttcap%
\pgfsetroundjoin%
\definecolor{currentfill}{rgb}{0.279566,0.067836,0.391917}%
\pgfsetfillcolor{currentfill}%
\pgfsetfillopacity{0.700000}%
\pgfsetlinewidth{0.000000pt}%
\definecolor{currentstroke}{rgb}{0.000000,0.000000,0.000000}%
\pgfsetstrokecolor{currentstroke}%
\pgfsetdash{}{0pt}%
\pgfpathmoveto{\pgfqpoint{3.301440in}{2.422223in}}%
\pgfpathlineto{\pgfqpoint{3.314833in}{2.413557in}}%
\pgfpathlineto{\pgfqpoint{3.328228in}{2.405031in}}%
\pgfpathlineto{\pgfqpoint{3.341625in}{2.396645in}}%
\pgfpathlineto{\pgfqpoint{3.355023in}{2.388399in}}%
\pgfpathlineto{\pgfqpoint{3.346941in}{2.382157in}}%
\pgfpathlineto{\pgfqpoint{3.338851in}{2.375997in}}%
\pgfpathlineto{\pgfqpoint{3.330753in}{2.369919in}}%
\pgfpathlineto{\pgfqpoint{3.322648in}{2.363927in}}%
\pgfpathlineto{\pgfqpoint{3.309230in}{2.372404in}}%
\pgfpathlineto{\pgfqpoint{3.295813in}{2.381020in}}%
\pgfpathlineto{\pgfqpoint{3.282398in}{2.389777in}}%
\pgfpathlineto{\pgfqpoint{3.268984in}{2.398675in}}%
\pgfpathlineto{\pgfqpoint{3.277110in}{2.404429in}}%
\pgfpathlineto{\pgfqpoint{3.285228in}{2.410274in}}%
\pgfpathlineto{\pgfqpoint{3.293338in}{2.416206in}}%
\pgfpathlineto{\pgfqpoint{3.301440in}{2.422223in}}%
\pgfpathclose%
\pgfusepath{fill}%
\end{pgfscope}%
\begin{pgfscope}%
\pgfpathrectangle{\pgfqpoint{1.150000in}{0.150000in}}{\pgfqpoint{5.700000in}{5.700000in}}%
\pgfusepath{clip}%
\pgfsetbuttcap%
\pgfsetroundjoin%
\definecolor{currentfill}{rgb}{0.283197,0.115680,0.436115}%
\pgfsetfillcolor{currentfill}%
\pgfsetfillopacity{0.700000}%
\pgfsetlinewidth{0.000000pt}%
\definecolor{currentstroke}{rgb}{0.000000,0.000000,0.000000}%
\pgfsetstrokecolor{currentstroke}%
\pgfsetdash{}{0pt}%
\pgfpathmoveto{\pgfqpoint{4.476576in}{2.497620in}}%
\pgfpathlineto{\pgfqpoint{4.490192in}{2.497938in}}%
\pgfpathlineto{\pgfqpoint{4.503817in}{2.498363in}}%
\pgfpathlineto{\pgfqpoint{4.517451in}{2.498893in}}%
\pgfpathlineto{\pgfqpoint{4.531095in}{2.499529in}}%
\pgfpathlineto{\pgfqpoint{4.523448in}{2.490530in}}%
\pgfpathlineto{\pgfqpoint{4.515797in}{2.481496in}}%
\pgfpathlineto{\pgfqpoint{4.508139in}{2.472429in}}%
\pgfpathlineto{\pgfqpoint{4.500476in}{2.463328in}}%
\pgfpathlineto{\pgfqpoint{4.486824in}{2.462739in}}%
\pgfpathlineto{\pgfqpoint{4.473181in}{2.462256in}}%
\pgfpathlineto{\pgfqpoint{4.459547in}{2.461879in}}%
\pgfpathlineto{\pgfqpoint{4.445922in}{2.461608in}}%
\pgfpathlineto{\pgfqpoint{4.453593in}{2.470655in}}%
\pgfpathlineto{\pgfqpoint{4.461260in}{2.479673in}}%
\pgfpathlineto{\pgfqpoint{4.468921in}{2.488661in}}%
\pgfpathlineto{\pgfqpoint{4.476576in}{2.497620in}}%
\pgfpathclose%
\pgfusepath{fill}%
\end{pgfscope}%
\begin{pgfscope}%
\pgfpathrectangle{\pgfqpoint{1.150000in}{0.150000in}}{\pgfqpoint{5.700000in}{5.700000in}}%
\pgfusepath{clip}%
\pgfsetbuttcap%
\pgfsetroundjoin%
\definecolor{currentfill}{rgb}{0.271305,0.019942,0.347269}%
\pgfsetfillcolor{currentfill}%
\pgfsetfillopacity{0.700000}%
\pgfsetlinewidth{0.000000pt}%
\definecolor{currentstroke}{rgb}{0.000000,0.000000,0.000000}%
\pgfsetstrokecolor{currentstroke}%
\pgfsetdash{}{0pt}%
\pgfpathmoveto{\pgfqpoint{3.633568in}{2.333786in}}%
\pgfpathlineto{\pgfqpoint{3.646982in}{2.328184in}}%
\pgfpathlineto{\pgfqpoint{3.660400in}{2.322707in}}%
\pgfpathlineto{\pgfqpoint{3.673822in}{2.317355in}}%
\pgfpathlineto{\pgfqpoint{3.687248in}{2.312126in}}%
\pgfpathlineto{\pgfqpoint{3.679305in}{2.304375in}}%
\pgfpathlineto{\pgfqpoint{3.671355in}{2.296667in}}%
\pgfpathlineto{\pgfqpoint{3.663400in}{2.289004in}}%
\pgfpathlineto{\pgfqpoint{3.655438in}{2.281388in}}%
\pgfpathlineto{\pgfqpoint{3.641997in}{2.286809in}}%
\pgfpathlineto{\pgfqpoint{3.628560in}{2.292354in}}%
\pgfpathlineto{\pgfqpoint{3.615127in}{2.298023in}}%
\pgfpathlineto{\pgfqpoint{3.601698in}{2.303817in}}%
\pgfpathlineto{\pgfqpoint{3.609675in}{2.311234in}}%
\pgfpathlineto{\pgfqpoint{3.617646in}{2.318702in}}%
\pgfpathlineto{\pgfqpoint{3.625610in}{2.326220in}}%
\pgfpathlineto{\pgfqpoint{3.633568in}{2.333786in}}%
\pgfpathclose%
\pgfusepath{fill}%
\end{pgfscope}%
\begin{pgfscope}%
\pgfpathrectangle{\pgfqpoint{1.150000in}{0.150000in}}{\pgfqpoint{5.700000in}{5.700000in}}%
\pgfusepath{clip}%
\pgfsetbuttcap%
\pgfsetroundjoin%
\definecolor{currentfill}{rgb}{0.273809,0.031497,0.358853}%
\pgfsetfillcolor{currentfill}%
\pgfsetfillopacity{0.700000}%
\pgfsetlinewidth{0.000000pt}%
\definecolor{currentstroke}{rgb}{0.000000,0.000000,0.000000}%
\pgfsetstrokecolor{currentstroke}%
\pgfsetdash{}{0pt}%
\pgfpathmoveto{\pgfqpoint{3.494394in}{2.354763in}}%
\pgfpathlineto{\pgfqpoint{3.507795in}{2.347942in}}%
\pgfpathlineto{\pgfqpoint{3.521200in}{2.341252in}}%
\pgfpathlineto{\pgfqpoint{3.534608in}{2.334692in}}%
\pgfpathlineto{\pgfqpoint{3.548019in}{2.328261in}}%
\pgfpathlineto{\pgfqpoint{3.540019in}{2.321101in}}%
\pgfpathlineto{\pgfqpoint{3.532012in}{2.314001in}}%
\pgfpathlineto{\pgfqpoint{3.523998in}{2.306962in}}%
\pgfpathlineto{\pgfqpoint{3.515978in}{2.299987in}}%
\pgfpathlineto{\pgfqpoint{3.502550in}{2.306629in}}%
\pgfpathlineto{\pgfqpoint{3.489125in}{2.313400in}}%
\pgfpathlineto{\pgfqpoint{3.475703in}{2.320301in}}%
\pgfpathlineto{\pgfqpoint{3.462284in}{2.327333in}}%
\pgfpathlineto{\pgfqpoint{3.470322in}{2.334091in}}%
\pgfpathlineto{\pgfqpoint{3.478353in}{2.340916in}}%
\pgfpathlineto{\pgfqpoint{3.486377in}{2.347808in}}%
\pgfpathlineto{\pgfqpoint{3.494394in}{2.354763in}}%
\pgfpathclose%
\pgfusepath{fill}%
\end{pgfscope}%
\begin{pgfscope}%
\pgfpathrectangle{\pgfqpoint{1.150000in}{0.150000in}}{\pgfqpoint{5.700000in}{5.700000in}}%
\pgfusepath{clip}%
\pgfsetbuttcap%
\pgfsetroundjoin%
\definecolor{currentfill}{rgb}{0.282656,0.100196,0.422160}%
\pgfsetfillcolor{currentfill}%
\pgfsetfillopacity{0.700000}%
\pgfsetlinewidth{0.000000pt}%
\definecolor{currentstroke}{rgb}{0.000000,0.000000,0.000000}%
\pgfsetstrokecolor{currentstroke}%
\pgfsetdash{}{0pt}%
\pgfpathmoveto{\pgfqpoint{4.391510in}{2.461590in}}%
\pgfpathlineto{\pgfqpoint{4.405100in}{2.461434in}}%
\pgfpathlineto{\pgfqpoint{4.418698in}{2.461385in}}%
\pgfpathlineto{\pgfqpoint{4.432306in}{2.461443in}}%
\pgfpathlineto{\pgfqpoint{4.445922in}{2.461608in}}%
\pgfpathlineto{\pgfqpoint{4.438245in}{2.452531in}}%
\pgfpathlineto{\pgfqpoint{4.430562in}{2.443426in}}%
\pgfpathlineto{\pgfqpoint{4.422874in}{2.434293in}}%
\pgfpathlineto{\pgfqpoint{4.415181in}{2.425131in}}%
\pgfpathlineto{\pgfqpoint{4.401556in}{2.425032in}}%
\pgfpathlineto{\pgfqpoint{4.387940in}{2.425039in}}%
\pgfpathlineto{\pgfqpoint{4.374332in}{2.425153in}}%
\pgfpathlineto{\pgfqpoint{4.360733in}{2.425375in}}%
\pgfpathlineto{\pgfqpoint{4.368435in}{2.434464in}}%
\pgfpathlineto{\pgfqpoint{4.376132in}{2.443530in}}%
\pgfpathlineto{\pgfqpoint{4.383823in}{2.452573in}}%
\pgfpathlineto{\pgfqpoint{4.391510in}{2.461590in}}%
\pgfpathclose%
\pgfusepath{fill}%
\end{pgfscope}%
\begin{pgfscope}%
\pgfpathrectangle{\pgfqpoint{1.150000in}{0.150000in}}{\pgfqpoint{5.700000in}{5.700000in}}%
\pgfusepath{clip}%
\pgfsetbuttcap%
\pgfsetroundjoin%
\definecolor{currentfill}{rgb}{0.283229,0.120777,0.440584}%
\pgfsetfillcolor{currentfill}%
\pgfsetfillopacity{0.700000}%
\pgfsetlinewidth{0.000000pt}%
\definecolor{currentstroke}{rgb}{0.000000,0.000000,0.000000}%
\pgfsetstrokecolor{currentstroke}%
\pgfsetdash{}{0pt}%
\pgfpathmoveto{\pgfqpoint{3.108073in}{2.516816in}}%
\pgfpathlineto{\pgfqpoint{3.121481in}{2.506146in}}%
\pgfpathlineto{\pgfqpoint{3.134889in}{2.495630in}}%
\pgfpathlineto{\pgfqpoint{3.148297in}{2.485266in}}%
\pgfpathlineto{\pgfqpoint{3.161705in}{2.475054in}}%
\pgfpathlineto{\pgfqpoint{3.153527in}{2.469878in}}%
\pgfpathlineto{\pgfqpoint{3.145341in}{2.464806in}}%
\pgfpathlineto{\pgfqpoint{3.137146in}{2.459840in}}%
\pgfpathlineto{\pgfqpoint{3.128941in}{2.454982in}}%
\pgfpathlineto{\pgfqpoint{3.115510in}{2.465445in}}%
\pgfpathlineto{\pgfqpoint{3.102079in}{2.476061in}}%
\pgfpathlineto{\pgfqpoint{3.088647in}{2.486829in}}%
\pgfpathlineto{\pgfqpoint{3.075215in}{2.497750in}}%
\pgfpathlineto{\pgfqpoint{3.083443in}{2.502350in}}%
\pgfpathlineto{\pgfqpoint{3.091662in}{2.507062in}}%
\pgfpathlineto{\pgfqpoint{3.099872in}{2.511885in}}%
\pgfpathlineto{\pgfqpoint{3.108073in}{2.516816in}}%
\pgfpathclose%
\pgfusepath{fill}%
\end{pgfscope}%
\begin{pgfscope}%
\pgfpathrectangle{\pgfqpoint{1.150000in}{0.150000in}}{\pgfqpoint{5.700000in}{5.700000in}}%
\pgfusepath{clip}%
\pgfsetbuttcap%
\pgfsetroundjoin%
\definecolor{currentfill}{rgb}{0.273809,0.031497,0.358853}%
\pgfsetfillcolor{currentfill}%
\pgfsetfillopacity{0.700000}%
\pgfsetlinewidth{0.000000pt}%
\definecolor{currentstroke}{rgb}{0.000000,0.000000,0.000000}%
\pgfsetstrokecolor{currentstroke}%
\pgfsetdash{}{0pt}%
\pgfpathmoveto{\pgfqpoint{3.996967in}{2.348203in}}%
\pgfpathlineto{\pgfqpoint{4.010446in}{2.345462in}}%
\pgfpathlineto{\pgfqpoint{4.023932in}{2.342836in}}%
\pgfpathlineto{\pgfqpoint{4.037425in}{2.340323in}}%
\pgfpathlineto{\pgfqpoint{4.050924in}{2.337924in}}%
\pgfpathlineto{\pgfqpoint{4.043113in}{2.329108in}}%
\pgfpathlineto{\pgfqpoint{4.035296in}{2.320297in}}%
\pgfpathlineto{\pgfqpoint{4.027474in}{2.311491in}}%
\pgfpathlineto{\pgfqpoint{4.019646in}{2.302693in}}%
\pgfpathlineto{\pgfqpoint{4.006136in}{2.305229in}}%
\pgfpathlineto{\pgfqpoint{3.992632in}{2.307879in}}%
\pgfpathlineto{\pgfqpoint{3.979135in}{2.310643in}}%
\pgfpathlineto{\pgfqpoint{3.965644in}{2.313521in}}%
\pgfpathlineto{\pgfqpoint{3.973483in}{2.322176in}}%
\pgfpathlineto{\pgfqpoint{3.981316in}{2.330841in}}%
\pgfpathlineto{\pgfqpoint{3.989144in}{2.339518in}}%
\pgfpathlineto{\pgfqpoint{3.996967in}{2.348203in}}%
\pgfpathclose%
\pgfusepath{fill}%
\end{pgfscope}%
\begin{pgfscope}%
\pgfpathrectangle{\pgfqpoint{1.150000in}{0.150000in}}{\pgfqpoint{5.700000in}{5.700000in}}%
\pgfusepath{clip}%
\pgfsetbuttcap%
\pgfsetroundjoin%
\definecolor{currentfill}{rgb}{0.271305,0.019942,0.347269}%
\pgfsetfillcolor{currentfill}%
\pgfsetfillopacity{0.700000}%
\pgfsetlinewidth{0.000000pt}%
\definecolor{currentstroke}{rgb}{0.000000,0.000000,0.000000}%
\pgfsetstrokecolor{currentstroke}%
\pgfsetdash{}{0pt}%
\pgfpathmoveto{\pgfqpoint{3.772653in}{2.324545in}}%
\pgfpathlineto{\pgfqpoint{3.786088in}{2.320101in}}%
\pgfpathlineto{\pgfqpoint{3.799529in}{2.315777in}}%
\pgfpathlineto{\pgfqpoint{3.812974in}{2.311573in}}%
\pgfpathlineto{\pgfqpoint{3.826425in}{2.307489in}}%
\pgfpathlineto{\pgfqpoint{3.818534in}{2.299245in}}%
\pgfpathlineto{\pgfqpoint{3.810636in}{2.291029in}}%
\pgfpathlineto{\pgfqpoint{3.802733in}{2.282843in}}%
\pgfpathlineto{\pgfqpoint{3.794824in}{2.274688in}}%
\pgfpathlineto{\pgfqpoint{3.781360in}{2.278947in}}%
\pgfpathlineto{\pgfqpoint{3.767901in}{2.283324in}}%
\pgfpathlineto{\pgfqpoint{3.754447in}{2.287822in}}%
\pgfpathlineto{\pgfqpoint{3.740998in}{2.292439in}}%
\pgfpathlineto{\pgfqpoint{3.748920in}{2.300414in}}%
\pgfpathlineto{\pgfqpoint{3.756837in}{2.308424in}}%
\pgfpathlineto{\pgfqpoint{3.764748in}{2.316468in}}%
\pgfpathlineto{\pgfqpoint{3.772653in}{2.324545in}}%
\pgfpathclose%
\pgfusepath{fill}%
\end{pgfscope}%
\begin{pgfscope}%
\pgfpathrectangle{\pgfqpoint{1.150000in}{0.150000in}}{\pgfqpoint{5.700000in}{5.700000in}}%
\pgfusepath{clip}%
\pgfsetbuttcap%
\pgfsetroundjoin%
\definecolor{currentfill}{rgb}{0.281446,0.084320,0.407414}%
\pgfsetfillcolor{currentfill}%
\pgfsetfillopacity{0.700000}%
\pgfsetlinewidth{0.000000pt}%
\definecolor{currentstroke}{rgb}{0.000000,0.000000,0.000000}%
\pgfsetstrokecolor{currentstroke}%
\pgfsetdash{}{0pt}%
\pgfpathmoveto{\pgfqpoint{4.306420in}{2.427340in}}%
\pgfpathlineto{\pgfqpoint{4.319986in}{2.426686in}}%
\pgfpathlineto{\pgfqpoint{4.333560in}{2.426141in}}%
\pgfpathlineto{\pgfqpoint{4.347142in}{2.425704in}}%
\pgfpathlineto{\pgfqpoint{4.360733in}{2.425375in}}%
\pgfpathlineto{\pgfqpoint{4.353025in}{2.416262in}}%
\pgfpathlineto{\pgfqpoint{4.345312in}{2.407127in}}%
\pgfpathlineto{\pgfqpoint{4.337594in}{2.397970in}}%
\pgfpathlineto{\pgfqpoint{4.329870in}{2.388791in}}%
\pgfpathlineto{\pgfqpoint{4.316270in}{2.389203in}}%
\pgfpathlineto{\pgfqpoint{4.302679in}{2.389723in}}%
\pgfpathlineto{\pgfqpoint{4.289096in}{2.390352in}}%
\pgfpathlineto{\pgfqpoint{4.275520in}{2.391089in}}%
\pgfpathlineto{\pgfqpoint{4.283253in}{2.400178in}}%
\pgfpathlineto{\pgfqpoint{4.290981in}{2.409249in}}%
\pgfpathlineto{\pgfqpoint{4.298703in}{2.418304in}}%
\pgfpathlineto{\pgfqpoint{4.306420in}{2.427340in}}%
\pgfpathclose%
\pgfusepath{fill}%
\end{pgfscope}%
\begin{pgfscope}%
\pgfpathrectangle{\pgfqpoint{1.150000in}{0.150000in}}{\pgfqpoint{5.700000in}{5.700000in}}%
\pgfusepath{clip}%
\pgfsetbuttcap%
\pgfsetroundjoin%
\definecolor{currentfill}{rgb}{0.277941,0.056324,0.381191}%
\pgfsetfillcolor{currentfill}%
\pgfsetfillopacity{0.700000}%
\pgfsetlinewidth{0.000000pt}%
\definecolor{currentstroke}{rgb}{0.000000,0.000000,0.000000}%
\pgfsetstrokecolor{currentstroke}%
\pgfsetdash{}{0pt}%
\pgfpathmoveto{\pgfqpoint{3.355023in}{2.388399in}}%
\pgfpathlineto{\pgfqpoint{3.368423in}{2.380291in}}%
\pgfpathlineto{\pgfqpoint{3.381825in}{2.372320in}}%
\pgfpathlineto{\pgfqpoint{3.395229in}{2.364486in}}%
\pgfpathlineto{\pgfqpoint{3.408635in}{2.356787in}}%
\pgfpathlineto{\pgfqpoint{3.400572in}{2.350322in}}%
\pgfpathlineto{\pgfqpoint{3.392501in}{2.343934in}}%
\pgfpathlineto{\pgfqpoint{3.384423in}{2.337624in}}%
\pgfpathlineto{\pgfqpoint{3.376338in}{2.331394in}}%
\pgfpathlineto{\pgfqpoint{3.362912in}{2.339323in}}%
\pgfpathlineto{\pgfqpoint{3.349489in}{2.347387in}}%
\pgfpathlineto{\pgfqpoint{3.336067in}{2.355588in}}%
\pgfpathlineto{\pgfqpoint{3.322648in}{2.363927in}}%
\pgfpathlineto{\pgfqpoint{3.330753in}{2.369919in}}%
\pgfpathlineto{\pgfqpoint{3.338851in}{2.375997in}}%
\pgfpathlineto{\pgfqpoint{3.346941in}{2.382157in}}%
\pgfpathlineto{\pgfqpoint{3.355023in}{2.388399in}}%
\pgfpathclose%
\pgfusepath{fill}%
\end{pgfscope}%
\begin{pgfscope}%
\pgfpathrectangle{\pgfqpoint{1.150000in}{0.150000in}}{\pgfqpoint{5.700000in}{5.700000in}}%
\pgfusepath{clip}%
\pgfsetbuttcap%
\pgfsetroundjoin%
\definecolor{currentfill}{rgb}{0.243113,0.292092,0.538516}%
\pgfsetfillcolor{currentfill}%
\pgfsetfillopacity{0.700000}%
\pgfsetlinewidth{0.000000pt}%
\definecolor{currentstroke}{rgb}{0.000000,0.000000,0.000000}%
\pgfsetstrokecolor{currentstroke}%
\pgfsetdash{}{0pt}%
\pgfpathmoveto{\pgfqpoint{2.698301in}{2.871331in}}%
\pgfpathlineto{\pgfqpoint{2.711809in}{2.855548in}}%
\pgfpathlineto{\pgfqpoint{2.725313in}{2.839961in}}%
\pgfpathlineto{\pgfqpoint{2.738812in}{2.824567in}}%
\pgfpathlineto{\pgfqpoint{2.752306in}{2.809364in}}%
\pgfpathlineto{\pgfqpoint{2.743904in}{2.806486in}}%
\pgfpathlineto{\pgfqpoint{2.735490in}{2.803755in}}%
\pgfpathlineto{\pgfqpoint{2.727065in}{2.801173in}}%
\pgfpathlineto{\pgfqpoint{2.718627in}{2.798742in}}%
\pgfpathlineto{\pgfqpoint{2.705102in}{2.814224in}}%
\pgfpathlineto{\pgfqpoint{2.691572in}{2.829899in}}%
\pgfpathlineto{\pgfqpoint{2.678037in}{2.845767in}}%
\pgfpathlineto{\pgfqpoint{2.664496in}{2.861831in}}%
\pgfpathlineto{\pgfqpoint{2.672965in}{2.863974in}}%
\pgfpathlineto{\pgfqpoint{2.681422in}{2.866273in}}%
\pgfpathlineto{\pgfqpoint{2.689867in}{2.868726in}}%
\pgfpathlineto{\pgfqpoint{2.698301in}{2.871331in}}%
\pgfpathclose%
\pgfusepath{fill}%
\end{pgfscope}%
\begin{pgfscope}%
\pgfpathrectangle{\pgfqpoint{1.150000in}{0.150000in}}{\pgfqpoint{5.700000in}{5.700000in}}%
\pgfusepath{clip}%
\pgfsetbuttcap%
\pgfsetroundjoin%
\definecolor{currentfill}{rgb}{0.282910,0.105393,0.426902}%
\pgfsetfillcolor{currentfill}%
\pgfsetfillopacity{0.700000}%
\pgfsetlinewidth{0.000000pt}%
\definecolor{currentstroke}{rgb}{0.000000,0.000000,0.000000}%
\pgfsetstrokecolor{currentstroke}%
\pgfsetdash{}{0pt}%
\pgfpathmoveto{\pgfqpoint{3.161705in}{2.475054in}}%
\pgfpathlineto{\pgfqpoint{3.175113in}{2.464992in}}%
\pgfpathlineto{\pgfqpoint{3.188522in}{2.455079in}}%
\pgfpathlineto{\pgfqpoint{3.201930in}{2.445314in}}%
\pgfpathlineto{\pgfqpoint{3.215339in}{2.435696in}}%
\pgfpathlineto{\pgfqpoint{3.207184in}{2.430277in}}%
\pgfpathlineto{\pgfqpoint{3.199020in}{2.424956in}}%
\pgfpathlineto{\pgfqpoint{3.190848in}{2.419737in}}%
\pgfpathlineto{\pgfqpoint{3.182667in}{2.414621in}}%
\pgfpathlineto{\pgfqpoint{3.169235in}{2.424489in}}%
\pgfpathlineto{\pgfqpoint{3.155804in}{2.434505in}}%
\pgfpathlineto{\pgfqpoint{3.142373in}{2.444668in}}%
\pgfpathlineto{\pgfqpoint{3.128941in}{2.454982in}}%
\pgfpathlineto{\pgfqpoint{3.137146in}{2.459840in}}%
\pgfpathlineto{\pgfqpoint{3.145341in}{2.464806in}}%
\pgfpathlineto{\pgfqpoint{3.153527in}{2.469878in}}%
\pgfpathlineto{\pgfqpoint{3.161705in}{2.475054in}}%
\pgfpathclose%
\pgfusepath{fill}%
\end{pgfscope}%
\begin{pgfscope}%
\pgfpathrectangle{\pgfqpoint{1.150000in}{0.150000in}}{\pgfqpoint{5.700000in}{5.700000in}}%
\pgfusepath{clip}%
\pgfsetbuttcap%
\pgfsetroundjoin%
\definecolor{currentfill}{rgb}{0.253935,0.265254,0.529983}%
\pgfsetfillcolor{currentfill}%
\pgfsetfillopacity{0.700000}%
\pgfsetlinewidth{0.000000pt}%
\definecolor{currentstroke}{rgb}{0.000000,0.000000,0.000000}%
\pgfsetstrokecolor{currentstroke}%
\pgfsetdash{}{0pt}%
\pgfpathmoveto{\pgfqpoint{2.752306in}{2.809364in}}%
\pgfpathlineto{\pgfqpoint{2.765796in}{2.794351in}}%
\pgfpathlineto{\pgfqpoint{2.779281in}{2.779526in}}%
\pgfpathlineto{\pgfqpoint{2.792763in}{2.764888in}}%
\pgfpathlineto{\pgfqpoint{2.806240in}{2.750434in}}%
\pgfpathlineto{\pgfqpoint{2.797867in}{2.747285in}}%
\pgfpathlineto{\pgfqpoint{2.789484in}{2.744277in}}%
\pgfpathlineto{\pgfqpoint{2.781089in}{2.741414in}}%
\pgfpathlineto{\pgfqpoint{2.772683in}{2.738697in}}%
\pgfpathlineto{\pgfqpoint{2.759175in}{2.753429in}}%
\pgfpathlineto{\pgfqpoint{2.745664in}{2.768346in}}%
\pgfpathlineto{\pgfqpoint{2.732148in}{2.783450in}}%
\pgfpathlineto{\pgfqpoint{2.718627in}{2.798742in}}%
\pgfpathlineto{\pgfqpoint{2.727065in}{2.801173in}}%
\pgfpathlineto{\pgfqpoint{2.735490in}{2.803755in}}%
\pgfpathlineto{\pgfqpoint{2.743904in}{2.806486in}}%
\pgfpathlineto{\pgfqpoint{2.752306in}{2.809364in}}%
\pgfpathclose%
\pgfusepath{fill}%
\end{pgfscope}%
\begin{pgfscope}%
\pgfpathrectangle{\pgfqpoint{1.150000in}{0.150000in}}{\pgfqpoint{5.700000in}{5.700000in}}%
\pgfusepath{clip}%
\pgfsetbuttcap%
\pgfsetroundjoin%
\definecolor{currentfill}{rgb}{0.279566,0.067836,0.391917}%
\pgfsetfillcolor{currentfill}%
\pgfsetfillopacity{0.700000}%
\pgfsetlinewidth{0.000000pt}%
\definecolor{currentstroke}{rgb}{0.000000,0.000000,0.000000}%
\pgfsetstrokecolor{currentstroke}%
\pgfsetdash{}{0pt}%
\pgfpathmoveto{\pgfqpoint{4.221299in}{2.395127in}}%
\pgfpathlineto{\pgfqpoint{4.234843in}{2.393953in}}%
\pgfpathlineto{\pgfqpoint{4.248394in}{2.392889in}}%
\pgfpathlineto{\pgfqpoint{4.261953in}{2.391934in}}%
\pgfpathlineto{\pgfqpoint{4.275520in}{2.391089in}}%
\pgfpathlineto{\pgfqpoint{4.267782in}{2.381984in}}%
\pgfpathlineto{\pgfqpoint{4.260039in}{2.372863in}}%
\pgfpathlineto{\pgfqpoint{4.252290in}{2.363727in}}%
\pgfpathlineto{\pgfqpoint{4.244536in}{2.354577in}}%
\pgfpathlineto{\pgfqpoint{4.230959in}{2.355524in}}%
\pgfpathlineto{\pgfqpoint{4.217390in}{2.356580in}}%
\pgfpathlineto{\pgfqpoint{4.203829in}{2.357746in}}%
\pgfpathlineto{\pgfqpoint{4.190276in}{2.359021in}}%
\pgfpathlineto{\pgfqpoint{4.198040in}{2.368063in}}%
\pgfpathlineto{\pgfqpoint{4.205798in}{2.377095in}}%
\pgfpathlineto{\pgfqpoint{4.213551in}{2.386117in}}%
\pgfpathlineto{\pgfqpoint{4.221299in}{2.395127in}}%
\pgfpathclose%
\pgfusepath{fill}%
\end{pgfscope}%
\begin{pgfscope}%
\pgfpathrectangle{\pgfqpoint{1.150000in}{0.150000in}}{\pgfqpoint{5.700000in}{5.700000in}}%
\pgfusepath{clip}%
\pgfsetbuttcap%
\pgfsetroundjoin%
\definecolor{currentfill}{rgb}{0.272594,0.025563,0.353093}%
\pgfsetfillcolor{currentfill}%
\pgfsetfillopacity{0.700000}%
\pgfsetlinewidth{0.000000pt}%
\definecolor{currentstroke}{rgb}{0.000000,0.000000,0.000000}%
\pgfsetstrokecolor{currentstroke}%
\pgfsetdash{}{0pt}%
\pgfpathmoveto{\pgfqpoint{3.911742in}{2.326188in}}%
\pgfpathlineto{\pgfqpoint{3.925209in}{2.322847in}}%
\pgfpathlineto{\pgfqpoint{3.938681in}{2.319623in}}%
\pgfpathlineto{\pgfqpoint{3.952160in}{2.316515in}}%
\pgfpathlineto{\pgfqpoint{3.965644in}{2.313521in}}%
\pgfpathlineto{\pgfqpoint{3.957800in}{2.304880in}}%
\pgfpathlineto{\pgfqpoint{3.949950in}{2.296252in}}%
\pgfpathlineto{\pgfqpoint{3.942095in}{2.287640in}}%
\pgfpathlineto{\pgfqpoint{3.934234in}{2.279044in}}%
\pgfpathlineto{\pgfqpoint{3.920737in}{2.282193in}}%
\pgfpathlineto{\pgfqpoint{3.907247in}{2.285457in}}%
\pgfpathlineto{\pgfqpoint{3.893762in}{2.288837in}}%
\pgfpathlineto{\pgfqpoint{3.880284in}{2.292333in}}%
\pgfpathlineto{\pgfqpoint{3.888157in}{2.300766in}}%
\pgfpathlineto{\pgfqpoint{3.896024in}{2.309221in}}%
\pgfpathlineto{\pgfqpoint{3.903886in}{2.317695in}}%
\pgfpathlineto{\pgfqpoint{3.911742in}{2.326188in}}%
\pgfpathclose%
\pgfusepath{fill}%
\end{pgfscope}%
\begin{pgfscope}%
\pgfpathrectangle{\pgfqpoint{1.150000in}{0.150000in}}{\pgfqpoint{5.700000in}{5.700000in}}%
\pgfusepath{clip}%
\pgfsetbuttcap%
\pgfsetroundjoin%
\definecolor{currentfill}{rgb}{0.263663,0.237631,0.518762}%
\pgfsetfillcolor{currentfill}%
\pgfsetfillopacity{0.700000}%
\pgfsetlinewidth{0.000000pt}%
\definecolor{currentstroke}{rgb}{0.000000,0.000000,0.000000}%
\pgfsetstrokecolor{currentstroke}%
\pgfsetdash{}{0pt}%
\pgfpathmoveto{\pgfqpoint{2.806240in}{2.750434in}}%
\pgfpathlineto{\pgfqpoint{2.819713in}{2.736163in}}%
\pgfpathlineto{\pgfqpoint{2.833183in}{2.722073in}}%
\pgfpathlineto{\pgfqpoint{2.846650in}{2.708163in}}%
\pgfpathlineto{\pgfqpoint{2.860113in}{2.694431in}}%
\pgfpathlineto{\pgfqpoint{2.851770in}{2.691012in}}%
\pgfpathlineto{\pgfqpoint{2.843415in}{2.687730in}}%
\pgfpathlineto{\pgfqpoint{2.835050in}{2.684587in}}%
\pgfpathlineto{\pgfqpoint{2.826674in}{2.681586in}}%
\pgfpathlineto{\pgfqpoint{2.813182in}{2.695594in}}%
\pgfpathlineto{\pgfqpoint{2.799686in}{2.709781in}}%
\pgfpathlineto{\pgfqpoint{2.786186in}{2.724148in}}%
\pgfpathlineto{\pgfqpoint{2.772683in}{2.738697in}}%
\pgfpathlineto{\pgfqpoint{2.781089in}{2.741414in}}%
\pgfpathlineto{\pgfqpoint{2.789484in}{2.744277in}}%
\pgfpathlineto{\pgfqpoint{2.797867in}{2.747285in}}%
\pgfpathlineto{\pgfqpoint{2.806240in}{2.750434in}}%
\pgfpathclose%
\pgfusepath{fill}%
\end{pgfscope}%
\begin{pgfscope}%
\pgfpathrectangle{\pgfqpoint{1.150000in}{0.150000in}}{\pgfqpoint{5.700000in}{5.700000in}}%
\pgfusepath{clip}%
\pgfsetbuttcap%
\pgfsetroundjoin%
\definecolor{currentfill}{rgb}{0.194100,0.399323,0.555565}%
\pgfsetfillcolor{currentfill}%
\pgfsetfillopacity{0.700000}%
\pgfsetlinewidth{0.000000pt}%
\definecolor{currentstroke}{rgb}{0.000000,0.000000,0.000000}%
\pgfsetstrokecolor{currentstroke}%
\pgfsetdash{}{0pt}%
\pgfpathmoveto{\pgfqpoint{5.722874in}{3.093412in}}%
\pgfpathlineto{\pgfqpoint{5.737014in}{3.098500in}}%
\pgfpathlineto{\pgfqpoint{5.751167in}{3.103685in}}%
\pgfpathlineto{\pgfqpoint{5.765334in}{3.108968in}}%
\pgfpathlineto{\pgfqpoint{5.779515in}{3.114348in}}%
\pgfpathlineto{\pgfqpoint{5.772408in}{3.109288in}}%
\pgfpathlineto{\pgfqpoint{5.765295in}{3.104193in}}%
\pgfpathlineto{\pgfqpoint{5.758175in}{3.099062in}}%
\pgfpathlineto{\pgfqpoint{5.751047in}{3.093891in}}%
\pgfpathlineto{\pgfqpoint{5.736847in}{3.088296in}}%
\pgfpathlineto{\pgfqpoint{5.722661in}{3.082797in}}%
\pgfpathlineto{\pgfqpoint{5.708490in}{3.077397in}}%
\pgfpathlineto{\pgfqpoint{5.694332in}{3.072094in}}%
\pgfpathlineto{\pgfqpoint{5.701478in}{3.077473in}}%
\pgfpathlineto{\pgfqpoint{5.708616in}{3.082818in}}%
\pgfpathlineto{\pgfqpoint{5.715749in}{3.088130in}}%
\pgfpathlineto{\pgfqpoint{5.722874in}{3.093412in}}%
\pgfpathclose%
\pgfusepath{fill}%
\end{pgfscope}%
\begin{pgfscope}%
\pgfpathrectangle{\pgfqpoint{1.150000in}{0.150000in}}{\pgfqpoint{5.700000in}{5.700000in}}%
\pgfusepath{clip}%
\pgfsetbuttcap%
\pgfsetroundjoin%
\definecolor{currentfill}{rgb}{0.201239,0.383670,0.554294}%
\pgfsetfillcolor{currentfill}%
\pgfsetfillopacity{0.700000}%
\pgfsetlinewidth{0.000000pt}%
\definecolor{currentstroke}{rgb}{0.000000,0.000000,0.000000}%
\pgfsetstrokecolor{currentstroke}%
\pgfsetdash{}{0pt}%
\pgfpathmoveto{\pgfqpoint{5.637839in}{3.051858in}}%
\pgfpathlineto{\pgfqpoint{5.651942in}{3.056771in}}%
\pgfpathlineto{\pgfqpoint{5.666058in}{3.061781in}}%
\pgfpathlineto{\pgfqpoint{5.680188in}{3.066889in}}%
\pgfpathlineto{\pgfqpoint{5.694332in}{3.072094in}}%
\pgfpathlineto{\pgfqpoint{5.687180in}{3.066678in}}%
\pgfpathlineto{\pgfqpoint{5.680020in}{3.061221in}}%
\pgfpathlineto{\pgfqpoint{5.672854in}{3.055723in}}%
\pgfpathlineto{\pgfqpoint{5.665681in}{3.050180in}}%
\pgfpathlineto{\pgfqpoint{5.651519in}{3.044778in}}%
\pgfpathlineto{\pgfqpoint{5.637372in}{3.039474in}}%
\pgfpathlineto{\pgfqpoint{5.623238in}{3.034268in}}%
\pgfpathlineto{\pgfqpoint{5.609118in}{3.029160in}}%
\pgfpathlineto{\pgfqpoint{5.616309in}{3.034892in}}%
\pgfpathlineto{\pgfqpoint{5.623492in}{3.040584in}}%
\pgfpathlineto{\pgfqpoint{5.630669in}{3.046239in}}%
\pgfpathlineto{\pgfqpoint{5.637839in}{3.051858in}}%
\pgfpathclose%
\pgfusepath{fill}%
\end{pgfscope}%
\begin{pgfscope}%
\pgfpathrectangle{\pgfqpoint{1.150000in}{0.150000in}}{\pgfqpoint{5.700000in}{5.700000in}}%
\pgfusepath{clip}%
\pgfsetbuttcap%
\pgfsetroundjoin%
\definecolor{currentfill}{rgb}{0.210503,0.363727,0.552206}%
\pgfsetfillcolor{currentfill}%
\pgfsetfillopacity{0.700000}%
\pgfsetlinewidth{0.000000pt}%
\definecolor{currentstroke}{rgb}{0.000000,0.000000,0.000000}%
\pgfsetstrokecolor{currentstroke}%
\pgfsetdash{}{0pt}%
\pgfpathmoveto{\pgfqpoint{5.552774in}{3.009706in}}%
\pgfpathlineto{\pgfqpoint{5.566840in}{3.014423in}}%
\pgfpathlineto{\pgfqpoint{5.580919in}{3.019237in}}%
\pgfpathlineto{\pgfqpoint{5.595012in}{3.024149in}}%
\pgfpathlineto{\pgfqpoint{5.609118in}{3.029160in}}%
\pgfpathlineto{\pgfqpoint{5.601921in}{3.023385in}}%
\pgfpathlineto{\pgfqpoint{5.594717in}{3.017567in}}%
\pgfpathlineto{\pgfqpoint{5.587507in}{3.011702in}}%
\pgfpathlineto{\pgfqpoint{5.580289in}{3.005788in}}%
\pgfpathlineto{\pgfqpoint{5.566166in}{3.000600in}}%
\pgfpathlineto{\pgfqpoint{5.552057in}{2.995511in}}%
\pgfpathlineto{\pgfqpoint{5.537962in}{2.990519in}}%
\pgfpathlineto{\pgfqpoint{5.523880in}{2.985626in}}%
\pgfpathlineto{\pgfqpoint{5.531114in}{2.991710in}}%
\pgfpathlineto{\pgfqpoint{5.538340in}{2.997750in}}%
\pgfpathlineto{\pgfqpoint{5.545561in}{3.003748in}}%
\pgfpathlineto{\pgfqpoint{5.552774in}{3.009706in}}%
\pgfpathclose%
\pgfusepath{fill}%
\end{pgfscope}%
\begin{pgfscope}%
\pgfpathrectangle{\pgfqpoint{1.150000in}{0.150000in}}{\pgfqpoint{5.700000in}{5.700000in}}%
\pgfusepath{clip}%
\pgfsetbuttcap%
\pgfsetroundjoin%
\definecolor{currentfill}{rgb}{0.270595,0.214069,0.507052}%
\pgfsetfillcolor{currentfill}%
\pgfsetfillopacity{0.700000}%
\pgfsetlinewidth{0.000000pt}%
\definecolor{currentstroke}{rgb}{0.000000,0.000000,0.000000}%
\pgfsetstrokecolor{currentstroke}%
\pgfsetdash{}{0pt}%
\pgfpathmoveto{\pgfqpoint{2.860113in}{2.694431in}}%
\pgfpathlineto{\pgfqpoint{2.873573in}{2.680875in}}%
\pgfpathlineto{\pgfqpoint{2.887031in}{2.667495in}}%
\pgfpathlineto{\pgfqpoint{2.900485in}{2.654288in}}%
\pgfpathlineto{\pgfqpoint{2.913937in}{2.641253in}}%
\pgfpathlineto{\pgfqpoint{2.905621in}{2.637566in}}%
\pgfpathlineto{\pgfqpoint{2.897295in}{2.634011in}}%
\pgfpathlineto{\pgfqpoint{2.888959in}{2.630590in}}%
\pgfpathlineto{\pgfqpoint{2.880612in}{2.627306in}}%
\pgfpathlineto{\pgfqpoint{2.867132in}{2.640616in}}%
\pgfpathlineto{\pgfqpoint{2.853649in}{2.654098in}}%
\pgfpathlineto{\pgfqpoint{2.840163in}{2.667754in}}%
\pgfpathlineto{\pgfqpoint{2.826674in}{2.681586in}}%
\pgfpathlineto{\pgfqpoint{2.835050in}{2.684587in}}%
\pgfpathlineto{\pgfqpoint{2.843415in}{2.687730in}}%
\pgfpathlineto{\pgfqpoint{2.851770in}{2.691012in}}%
\pgfpathlineto{\pgfqpoint{2.860113in}{2.694431in}}%
\pgfpathclose%
\pgfusepath{fill}%
\end{pgfscope}%
\begin{pgfscope}%
\pgfpathrectangle{\pgfqpoint{1.150000in}{0.150000in}}{\pgfqpoint{5.700000in}{5.700000in}}%
\pgfusepath{clip}%
\pgfsetbuttcap%
\pgfsetroundjoin%
\definecolor{currentfill}{rgb}{0.218130,0.347432,0.550038}%
\pgfsetfillcolor{currentfill}%
\pgfsetfillopacity{0.700000}%
\pgfsetlinewidth{0.000000pt}%
\definecolor{currentstroke}{rgb}{0.000000,0.000000,0.000000}%
\pgfsetstrokecolor{currentstroke}%
\pgfsetdash{}{0pt}%
\pgfpathmoveto{\pgfqpoint{5.467685in}{2.967037in}}%
\pgfpathlineto{\pgfqpoint{5.481714in}{2.971536in}}%
\pgfpathlineto{\pgfqpoint{5.495756in}{2.976135in}}%
\pgfpathlineto{\pgfqpoint{5.509811in}{2.980831in}}%
\pgfpathlineto{\pgfqpoint{5.523880in}{2.985626in}}%
\pgfpathlineto{\pgfqpoint{5.516639in}{2.979496in}}%
\pgfpathlineto{\pgfqpoint{5.509392in}{2.973318in}}%
\pgfpathlineto{\pgfqpoint{5.502138in}{2.967090in}}%
\pgfpathlineto{\pgfqpoint{5.494877in}{2.960810in}}%
\pgfpathlineto{\pgfqpoint{5.480793in}{2.955856in}}%
\pgfpathlineto{\pgfqpoint{5.466722in}{2.951002in}}%
\pgfpathlineto{\pgfqpoint{5.452665in}{2.946245in}}%
\pgfpathlineto{\pgfqpoint{5.438621in}{2.941588in}}%
\pgfpathlineto{\pgfqpoint{5.445897in}{2.948019in}}%
\pgfpathlineto{\pgfqpoint{5.453167in}{2.954403in}}%
\pgfpathlineto{\pgfqpoint{5.460429in}{2.960742in}}%
\pgfpathlineto{\pgfqpoint{5.467685in}{2.967037in}}%
\pgfpathclose%
\pgfusepath{fill}%
\end{pgfscope}%
\begin{pgfscope}%
\pgfpathrectangle{\pgfqpoint{1.150000in}{0.150000in}}{\pgfqpoint{5.700000in}{5.700000in}}%
\pgfusepath{clip}%
\pgfsetbuttcap%
\pgfsetroundjoin%
\definecolor{currentfill}{rgb}{0.225863,0.330805,0.547314}%
\pgfsetfillcolor{currentfill}%
\pgfsetfillopacity{0.700000}%
\pgfsetlinewidth{0.000000pt}%
\definecolor{currentstroke}{rgb}{0.000000,0.000000,0.000000}%
\pgfsetstrokecolor{currentstroke}%
\pgfsetdash{}{0pt}%
\pgfpathmoveto{\pgfqpoint{5.382576in}{2.923944in}}%
\pgfpathlineto{\pgfqpoint{5.396568in}{2.928207in}}%
\pgfpathlineto{\pgfqpoint{5.410573in}{2.932568in}}%
\pgfpathlineto{\pgfqpoint{5.424591in}{2.937029in}}%
\pgfpathlineto{\pgfqpoint{5.438621in}{2.941588in}}%
\pgfpathlineto{\pgfqpoint{5.431339in}{2.935107in}}%
\pgfpathlineto{\pgfqpoint{5.424049in}{2.928575in}}%
\pgfpathlineto{\pgfqpoint{5.416753in}{2.921991in}}%
\pgfpathlineto{\pgfqpoint{5.409450in}{2.915351in}}%
\pgfpathlineto{\pgfqpoint{5.395405in}{2.910653in}}%
\pgfpathlineto{\pgfqpoint{5.381373in}{2.906053in}}%
\pgfpathlineto{\pgfqpoint{5.367354in}{2.901553in}}%
\pgfpathlineto{\pgfqpoint{5.353348in}{2.897151in}}%
\pgfpathlineto{\pgfqpoint{5.360665in}{2.903923in}}%
\pgfpathlineto{\pgfqpoint{5.367975in}{2.910644in}}%
\pgfpathlineto{\pgfqpoint{5.375279in}{2.917318in}}%
\pgfpathlineto{\pgfqpoint{5.382576in}{2.923944in}}%
\pgfpathclose%
\pgfusepath{fill}%
\end{pgfscope}%
\begin{pgfscope}%
\pgfpathrectangle{\pgfqpoint{1.150000in}{0.150000in}}{\pgfqpoint{5.700000in}{5.700000in}}%
\pgfusepath{clip}%
\pgfsetbuttcap%
\pgfsetroundjoin%
\definecolor{currentfill}{rgb}{0.272594,0.025563,0.353093}%
\pgfsetfillcolor{currentfill}%
\pgfsetfillopacity{0.700000}%
\pgfsetlinewidth{0.000000pt}%
\definecolor{currentstroke}{rgb}{0.000000,0.000000,0.000000}%
\pgfsetstrokecolor{currentstroke}%
\pgfsetdash{}{0pt}%
\pgfpathmoveto{\pgfqpoint{3.548019in}{2.328261in}}%
\pgfpathlineto{\pgfqpoint{3.561433in}{2.321959in}}%
\pgfpathlineto{\pgfqpoint{3.574851in}{2.315785in}}%
\pgfpathlineto{\pgfqpoint{3.588272in}{2.309738in}}%
\pgfpathlineto{\pgfqpoint{3.601698in}{2.303817in}}%
\pgfpathlineto{\pgfqpoint{3.593714in}{2.296454in}}%
\pgfpathlineto{\pgfqpoint{3.585724in}{2.289145in}}%
\pgfpathlineto{\pgfqpoint{3.577727in}{2.281894in}}%
\pgfpathlineto{\pgfqpoint{3.569723in}{2.274701in}}%
\pgfpathlineto{\pgfqpoint{3.556282in}{2.280832in}}%
\pgfpathlineto{\pgfqpoint{3.542844in}{2.287089in}}%
\pgfpathlineto{\pgfqpoint{3.529409in}{2.293474in}}%
\pgfpathlineto{\pgfqpoint{3.515978in}{2.299987in}}%
\pgfpathlineto{\pgfqpoint{3.523998in}{2.306962in}}%
\pgfpathlineto{\pgfqpoint{3.532012in}{2.314001in}}%
\pgfpathlineto{\pgfqpoint{3.540019in}{2.321101in}}%
\pgfpathlineto{\pgfqpoint{3.548019in}{2.328261in}}%
\pgfpathclose%
\pgfusepath{fill}%
\end{pgfscope}%
\begin{pgfscope}%
\pgfpathrectangle{\pgfqpoint{1.150000in}{0.150000in}}{\pgfqpoint{5.700000in}{5.700000in}}%
\pgfusepath{clip}%
\pgfsetbuttcap%
\pgfsetroundjoin%
\definecolor{currentfill}{rgb}{0.235526,0.309527,0.542944}%
\pgfsetfillcolor{currentfill}%
\pgfsetfillopacity{0.700000}%
\pgfsetlinewidth{0.000000pt}%
\definecolor{currentstroke}{rgb}{0.000000,0.000000,0.000000}%
\pgfsetstrokecolor{currentstroke}%
\pgfsetdash{}{0pt}%
\pgfpathmoveto{\pgfqpoint{5.297452in}{2.880536in}}%
\pgfpathlineto{\pgfqpoint{5.311407in}{2.884541in}}%
\pgfpathlineto{\pgfqpoint{5.325375in}{2.888645in}}%
\pgfpathlineto{\pgfqpoint{5.339355in}{2.892849in}}%
\pgfpathlineto{\pgfqpoint{5.353348in}{2.897151in}}%
\pgfpathlineto{\pgfqpoint{5.346025in}{2.890328in}}%
\pgfpathlineto{\pgfqpoint{5.338694in}{2.883451in}}%
\pgfpathlineto{\pgfqpoint{5.331357in}{2.876520in}}%
\pgfpathlineto{\pgfqpoint{5.324014in}{2.869532in}}%
\pgfpathlineto{\pgfqpoint{5.310007in}{2.865109in}}%
\pgfpathlineto{\pgfqpoint{5.296014in}{2.860786in}}%
\pgfpathlineto{\pgfqpoint{5.282033in}{2.856561in}}%
\pgfpathlineto{\pgfqpoint{5.268065in}{2.852436in}}%
\pgfpathlineto{\pgfqpoint{5.275421in}{2.859537in}}%
\pgfpathlineto{\pgfqpoint{5.282771in}{2.866587in}}%
\pgfpathlineto{\pgfqpoint{5.290115in}{2.873586in}}%
\pgfpathlineto{\pgfqpoint{5.297452in}{2.880536in}}%
\pgfpathclose%
\pgfusepath{fill}%
\end{pgfscope}%
\begin{pgfscope}%
\pgfpathrectangle{\pgfqpoint{1.150000in}{0.150000in}}{\pgfqpoint{5.700000in}{5.700000in}}%
\pgfusepath{clip}%
\pgfsetbuttcap%
\pgfsetroundjoin%
\definecolor{currentfill}{rgb}{0.243113,0.292092,0.538516}%
\pgfsetfillcolor{currentfill}%
\pgfsetfillopacity{0.700000}%
\pgfsetlinewidth{0.000000pt}%
\definecolor{currentstroke}{rgb}{0.000000,0.000000,0.000000}%
\pgfsetstrokecolor{currentstroke}%
\pgfsetdash{}{0pt}%
\pgfpathmoveto{\pgfqpoint{5.212316in}{2.836932in}}%
\pgfpathlineto{\pgfqpoint{5.226235in}{2.840659in}}%
\pgfpathlineto{\pgfqpoint{5.240166in}{2.844485in}}%
\pgfpathlineto{\pgfqpoint{5.254109in}{2.848411in}}%
\pgfpathlineto{\pgfqpoint{5.268065in}{2.852436in}}%
\pgfpathlineto{\pgfqpoint{5.260702in}{2.845282in}}%
\pgfpathlineto{\pgfqpoint{5.253332in}{2.838072in}}%
\pgfpathlineto{\pgfqpoint{5.245955in}{2.830807in}}%
\pgfpathlineto{\pgfqpoint{5.238572in}{2.823484in}}%
\pgfpathlineto{\pgfqpoint{5.224605in}{2.819357in}}%
\pgfpathlineto{\pgfqpoint{5.210649in}{2.815330in}}%
\pgfpathlineto{\pgfqpoint{5.196706in}{2.811402in}}%
\pgfpathlineto{\pgfqpoint{5.182775in}{2.807575in}}%
\pgfpathlineto{\pgfqpoint{5.190170in}{2.814992in}}%
\pgfpathlineto{\pgfqpoint{5.197558in}{2.822356in}}%
\pgfpathlineto{\pgfqpoint{5.204941in}{2.829669in}}%
\pgfpathlineto{\pgfqpoint{5.212316in}{2.836932in}}%
\pgfpathclose%
\pgfusepath{fill}%
\end{pgfscope}%
\begin{pgfscope}%
\pgfpathrectangle{\pgfqpoint{1.150000in}{0.150000in}}{\pgfqpoint{5.700000in}{5.700000in}}%
\pgfusepath{clip}%
\pgfsetbuttcap%
\pgfsetroundjoin%
\definecolor{currentfill}{rgb}{0.277018,0.050344,0.375715}%
\pgfsetfillcolor{currentfill}%
\pgfsetfillopacity{0.700000}%
\pgfsetlinewidth{0.000000pt}%
\definecolor{currentstroke}{rgb}{0.000000,0.000000,0.000000}%
\pgfsetstrokecolor{currentstroke}%
\pgfsetdash{}{0pt}%
\pgfpathmoveto{\pgfqpoint{4.136138in}{2.365227in}}%
\pgfpathlineto{\pgfqpoint{4.149661in}{2.363509in}}%
\pgfpathlineto{\pgfqpoint{4.163192in}{2.361902in}}%
\pgfpathlineto{\pgfqpoint{4.176730in}{2.360406in}}%
\pgfpathlineto{\pgfqpoint{4.190276in}{2.359021in}}%
\pgfpathlineto{\pgfqpoint{4.182507in}{2.349970in}}%
\pgfpathlineto{\pgfqpoint{4.174733in}{2.340911in}}%
\pgfpathlineto{\pgfqpoint{4.166953in}{2.331845in}}%
\pgfpathlineto{\pgfqpoint{4.159168in}{2.322773in}}%
\pgfpathlineto{\pgfqpoint{4.145612in}{2.324278in}}%
\pgfpathlineto{\pgfqpoint{4.132064in}{2.325893in}}%
\pgfpathlineto{\pgfqpoint{4.118523in}{2.327619in}}%
\pgfpathlineto{\pgfqpoint{4.104989in}{2.329456in}}%
\pgfpathlineto{\pgfqpoint{4.112784in}{2.338402in}}%
\pgfpathlineto{\pgfqpoint{4.120574in}{2.347347in}}%
\pgfpathlineto{\pgfqpoint{4.128358in}{2.356289in}}%
\pgfpathlineto{\pgfqpoint{4.136138in}{2.365227in}}%
\pgfpathclose%
\pgfusepath{fill}%
\end{pgfscope}%
\begin{pgfscope}%
\pgfpathrectangle{\pgfqpoint{1.150000in}{0.150000in}}{\pgfqpoint{5.700000in}{5.700000in}}%
\pgfusepath{clip}%
\pgfsetbuttcap%
\pgfsetroundjoin%
\definecolor{currentfill}{rgb}{0.271305,0.019942,0.347269}%
\pgfsetfillcolor{currentfill}%
\pgfsetfillopacity{0.700000}%
\pgfsetlinewidth{0.000000pt}%
\definecolor{currentstroke}{rgb}{0.000000,0.000000,0.000000}%
\pgfsetstrokecolor{currentstroke}%
\pgfsetdash{}{0pt}%
\pgfpathmoveto{\pgfqpoint{3.687248in}{2.312126in}}%
\pgfpathlineto{\pgfqpoint{3.700678in}{2.307021in}}%
\pgfpathlineto{\pgfqpoint{3.714113in}{2.302039in}}%
\pgfpathlineto{\pgfqpoint{3.727553in}{2.297178in}}%
\pgfpathlineto{\pgfqpoint{3.740998in}{2.292439in}}%
\pgfpathlineto{\pgfqpoint{3.733069in}{2.284503in}}%
\pgfpathlineto{\pgfqpoint{3.725134in}{2.276605in}}%
\pgfpathlineto{\pgfqpoint{3.717193in}{2.268748in}}%
\pgfpathlineto{\pgfqpoint{3.709246in}{2.260934in}}%
\pgfpathlineto{\pgfqpoint{3.695787in}{2.265865in}}%
\pgfpathlineto{\pgfqpoint{3.682333in}{2.270917in}}%
\pgfpathlineto{\pgfqpoint{3.668883in}{2.276091in}}%
\pgfpathlineto{\pgfqpoint{3.655438in}{2.281388in}}%
\pgfpathlineto{\pgfqpoint{3.663400in}{2.289004in}}%
\pgfpathlineto{\pgfqpoint{3.671355in}{2.296667in}}%
\pgfpathlineto{\pgfqpoint{3.679305in}{2.304375in}}%
\pgfpathlineto{\pgfqpoint{3.687248in}{2.312126in}}%
\pgfpathclose%
\pgfusepath{fill}%
\end{pgfscope}%
\begin{pgfscope}%
\pgfpathrectangle{\pgfqpoint{1.150000in}{0.150000in}}{\pgfqpoint{5.700000in}{5.700000in}}%
\pgfusepath{clip}%
\pgfsetbuttcap%
\pgfsetroundjoin%
\definecolor{currentfill}{rgb}{0.252194,0.269783,0.531579}%
\pgfsetfillcolor{currentfill}%
\pgfsetfillopacity{0.700000}%
\pgfsetlinewidth{0.000000pt}%
\definecolor{currentstroke}{rgb}{0.000000,0.000000,0.000000}%
\pgfsetstrokecolor{currentstroke}%
\pgfsetdash{}{0pt}%
\pgfpathmoveto{\pgfqpoint{5.127173in}{2.793264in}}%
\pgfpathlineto{\pgfqpoint{5.141055in}{2.796692in}}%
\pgfpathlineto{\pgfqpoint{5.154950in}{2.800219in}}%
\pgfpathlineto{\pgfqpoint{5.168856in}{2.803847in}}%
\pgfpathlineto{\pgfqpoint{5.182775in}{2.807575in}}%
\pgfpathlineto{\pgfqpoint{5.175374in}{2.800103in}}%
\pgfpathlineto{\pgfqpoint{5.167966in}{2.792576in}}%
\pgfpathlineto{\pgfqpoint{5.160551in}{2.784992in}}%
\pgfpathlineto{\pgfqpoint{5.153130in}{2.777351in}}%
\pgfpathlineto{\pgfqpoint{5.139200in}{2.773541in}}%
\pgfpathlineto{\pgfqpoint{5.125282in}{2.769830in}}%
\pgfpathlineto{\pgfqpoint{5.111377in}{2.766220in}}%
\pgfpathlineto{\pgfqpoint{5.097483in}{2.762711in}}%
\pgfpathlineto{\pgfqpoint{5.104915in}{2.770428in}}%
\pgfpathlineto{\pgfqpoint{5.112340in}{2.778092in}}%
\pgfpathlineto{\pgfqpoint{5.119760in}{2.785703in}}%
\pgfpathlineto{\pgfqpoint{5.127173in}{2.793264in}}%
\pgfpathclose%
\pgfusepath{fill}%
\end{pgfscope}%
\begin{pgfscope}%
\pgfpathrectangle{\pgfqpoint{1.150000in}{0.150000in}}{\pgfqpoint{5.700000in}{5.700000in}}%
\pgfusepath{clip}%
\pgfsetbuttcap%
\pgfsetroundjoin%
\definecolor{currentfill}{rgb}{0.258965,0.251537,0.524736}%
\pgfsetfillcolor{currentfill}%
\pgfsetfillopacity{0.700000}%
\pgfsetlinewidth{0.000000pt}%
\definecolor{currentstroke}{rgb}{0.000000,0.000000,0.000000}%
\pgfsetstrokecolor{currentstroke}%
\pgfsetdash{}{0pt}%
\pgfpathmoveto{\pgfqpoint{5.042025in}{2.749677in}}%
\pgfpathlineto{\pgfqpoint{5.055872in}{2.752785in}}%
\pgfpathlineto{\pgfqpoint{5.069730in}{2.755993in}}%
\pgfpathlineto{\pgfqpoint{5.083601in}{2.759301in}}%
\pgfpathlineto{\pgfqpoint{5.097483in}{2.762711in}}%
\pgfpathlineto{\pgfqpoint{5.090044in}{2.754939in}}%
\pgfpathlineto{\pgfqpoint{5.082599in}{2.747113in}}%
\pgfpathlineto{\pgfqpoint{5.075148in}{2.739230in}}%
\pgfpathlineto{\pgfqpoint{5.067691in}{2.731290in}}%
\pgfpathlineto{\pgfqpoint{5.053798in}{2.727816in}}%
\pgfpathlineto{\pgfqpoint{5.039917in}{2.724444in}}%
\pgfpathlineto{\pgfqpoint{5.026048in}{2.721172in}}%
\pgfpathlineto{\pgfqpoint{5.012190in}{2.718001in}}%
\pgfpathlineto{\pgfqpoint{5.019658in}{2.725998in}}%
\pgfpathlineto{\pgfqpoint{5.027120in}{2.733943in}}%
\pgfpathlineto{\pgfqpoint{5.034575in}{2.741836in}}%
\pgfpathlineto{\pgfqpoint{5.042025in}{2.749677in}}%
\pgfpathclose%
\pgfusepath{fill}%
\end{pgfscope}%
\begin{pgfscope}%
\pgfpathrectangle{\pgfqpoint{1.150000in}{0.150000in}}{\pgfqpoint{5.700000in}{5.700000in}}%
\pgfusepath{clip}%
\pgfsetbuttcap%
\pgfsetroundjoin%
\definecolor{currentfill}{rgb}{0.265145,0.232956,0.516599}%
\pgfsetfillcolor{currentfill}%
\pgfsetfillopacity{0.700000}%
\pgfsetlinewidth{0.000000pt}%
\definecolor{currentstroke}{rgb}{0.000000,0.000000,0.000000}%
\pgfsetstrokecolor{currentstroke}%
\pgfsetdash{}{0pt}%
\pgfpathmoveto{\pgfqpoint{4.956874in}{2.706329in}}%
\pgfpathlineto{\pgfqpoint{4.970686in}{2.709095in}}%
\pgfpathlineto{\pgfqpoint{4.984509in}{2.711962in}}%
\pgfpathlineto{\pgfqpoint{4.998344in}{2.714931in}}%
\pgfpathlineto{\pgfqpoint{5.012190in}{2.718001in}}%
\pgfpathlineto{\pgfqpoint{5.004716in}{2.709950in}}%
\pgfpathlineto{\pgfqpoint{4.997236in}{2.701845in}}%
\pgfpathlineto{\pgfqpoint{4.989749in}{2.693685in}}%
\pgfpathlineto{\pgfqpoint{4.982256in}{2.685469in}}%
\pgfpathlineto{\pgfqpoint{4.968400in}{2.682353in}}%
\pgfpathlineto{\pgfqpoint{4.954555in}{2.679339in}}%
\pgfpathlineto{\pgfqpoint{4.940722in}{2.676426in}}%
\pgfpathlineto{\pgfqpoint{4.926900in}{2.673615in}}%
\pgfpathlineto{\pgfqpoint{4.934402in}{2.681870in}}%
\pgfpathlineto{\pgfqpoint{4.941899in}{2.690073in}}%
\pgfpathlineto{\pgfqpoint{4.949389in}{2.698226in}}%
\pgfpathlineto{\pgfqpoint{4.956874in}{2.706329in}}%
\pgfpathclose%
\pgfusepath{fill}%
\end{pgfscope}%
\begin{pgfscope}%
\pgfpathrectangle{\pgfqpoint{1.150000in}{0.150000in}}{\pgfqpoint{5.700000in}{5.700000in}}%
\pgfusepath{clip}%
\pgfsetbuttcap%
\pgfsetroundjoin%
\definecolor{currentfill}{rgb}{0.271828,0.209303,0.504434}%
\pgfsetfillcolor{currentfill}%
\pgfsetfillopacity{0.700000}%
\pgfsetlinewidth{0.000000pt}%
\definecolor{currentstroke}{rgb}{0.000000,0.000000,0.000000}%
\pgfsetstrokecolor{currentstroke}%
\pgfsetdash{}{0pt}%
\pgfpathmoveto{\pgfqpoint{4.871722in}{2.663388in}}%
\pgfpathlineto{\pgfqpoint{4.885500in}{2.665792in}}%
\pgfpathlineto{\pgfqpoint{4.899289in}{2.668298in}}%
\pgfpathlineto{\pgfqpoint{4.913089in}{2.670906in}}%
\pgfpathlineto{\pgfqpoint{4.926900in}{2.673615in}}%
\pgfpathlineto{\pgfqpoint{4.919391in}{2.665308in}}%
\pgfpathlineto{\pgfqpoint{4.911876in}{2.656949in}}%
\pgfpathlineto{\pgfqpoint{4.904355in}{2.648536in}}%
\pgfpathlineto{\pgfqpoint{4.896828in}{2.640069in}}%
\pgfpathlineto{\pgfqpoint{4.883008in}{2.637332in}}%
\pgfpathlineto{\pgfqpoint{4.869198in}{2.634698in}}%
\pgfpathlineto{\pgfqpoint{4.855400in}{2.632165in}}%
\pgfpathlineto{\pgfqpoint{4.841612in}{2.629734in}}%
\pgfpathlineto{\pgfqpoint{4.849148in}{2.638221in}}%
\pgfpathlineto{\pgfqpoint{4.856679in}{2.646659in}}%
\pgfpathlineto{\pgfqpoint{4.864203in}{2.655047in}}%
\pgfpathlineto{\pgfqpoint{4.871722in}{2.663388in}}%
\pgfpathclose%
\pgfusepath{fill}%
\end{pgfscope}%
\begin{pgfscope}%
\pgfpathrectangle{\pgfqpoint{1.150000in}{0.150000in}}{\pgfqpoint{5.700000in}{5.700000in}}%
\pgfusepath{clip}%
\pgfsetbuttcap%
\pgfsetroundjoin%
\definecolor{currentfill}{rgb}{0.276194,0.190074,0.493001}%
\pgfsetfillcolor{currentfill}%
\pgfsetfillopacity{0.700000}%
\pgfsetlinewidth{0.000000pt}%
\definecolor{currentstroke}{rgb}{0.000000,0.000000,0.000000}%
\pgfsetstrokecolor{currentstroke}%
\pgfsetdash{}{0pt}%
\pgfpathmoveto{\pgfqpoint{2.913937in}{2.641253in}}%
\pgfpathlineto{\pgfqpoint{2.927387in}{2.628389in}}%
\pgfpathlineto{\pgfqpoint{2.940834in}{2.615694in}}%
\pgfpathlineto{\pgfqpoint{2.954279in}{2.603167in}}%
\pgfpathlineto{\pgfqpoint{2.967722in}{2.590806in}}%
\pgfpathlineto{\pgfqpoint{2.959433in}{2.586852in}}%
\pgfpathlineto{\pgfqpoint{2.951135in}{2.583025in}}%
\pgfpathlineto{\pgfqpoint{2.942826in}{2.579327in}}%
\pgfpathlineto{\pgfqpoint{2.934506in}{2.575762in}}%
\pgfpathlineto{\pgfqpoint{2.921036in}{2.588397in}}%
\pgfpathlineto{\pgfqpoint{2.907564in}{2.601198in}}%
\pgfpathlineto{\pgfqpoint{2.894089in}{2.614167in}}%
\pgfpathlineto{\pgfqpoint{2.880612in}{2.627306in}}%
\pgfpathlineto{\pgfqpoint{2.888959in}{2.630590in}}%
\pgfpathlineto{\pgfqpoint{2.897295in}{2.634011in}}%
\pgfpathlineto{\pgfqpoint{2.905621in}{2.637566in}}%
\pgfpathlineto{\pgfqpoint{2.913937in}{2.641253in}}%
\pgfpathclose%
\pgfusepath{fill}%
\end{pgfscope}%
\begin{pgfscope}%
\pgfpathrectangle{\pgfqpoint{1.150000in}{0.150000in}}{\pgfqpoint{5.700000in}{5.700000in}}%
\pgfusepath{clip}%
\pgfsetbuttcap%
\pgfsetroundjoin%
\definecolor{currentfill}{rgb}{0.276194,0.190074,0.493001}%
\pgfsetfillcolor{currentfill}%
\pgfsetfillopacity{0.700000}%
\pgfsetlinewidth{0.000000pt}%
\definecolor{currentstroke}{rgb}{0.000000,0.000000,0.000000}%
\pgfsetstrokecolor{currentstroke}%
\pgfsetdash{}{0pt}%
\pgfpathmoveto{\pgfqpoint{4.786569in}{2.621036in}}%
\pgfpathlineto{\pgfqpoint{4.800314in}{2.623057in}}%
\pgfpathlineto{\pgfqpoint{4.814069in}{2.625180in}}%
\pgfpathlineto{\pgfqpoint{4.827835in}{2.627406in}}%
\pgfpathlineto{\pgfqpoint{4.841612in}{2.629734in}}%
\pgfpathlineto{\pgfqpoint{4.834070in}{2.621198in}}%
\pgfpathlineto{\pgfqpoint{4.826522in}{2.612611in}}%
\pgfpathlineto{\pgfqpoint{4.818968in}{2.603973in}}%
\pgfpathlineto{\pgfqpoint{4.811408in}{2.595284in}}%
\pgfpathlineto{\pgfqpoint{4.797622in}{2.592947in}}%
\pgfpathlineto{\pgfqpoint{4.783846in}{2.590713in}}%
\pgfpathlineto{\pgfqpoint{4.770082in}{2.588581in}}%
\pgfpathlineto{\pgfqpoint{4.756328in}{2.586553in}}%
\pgfpathlineto{\pgfqpoint{4.763897in}{2.595243in}}%
\pgfpathlineto{\pgfqpoint{4.771460in}{2.603887in}}%
\pgfpathlineto{\pgfqpoint{4.779017in}{2.612485in}}%
\pgfpathlineto{\pgfqpoint{4.786569in}{2.621036in}}%
\pgfpathclose%
\pgfusepath{fill}%
\end{pgfscope}%
\begin{pgfscope}%
\pgfpathrectangle{\pgfqpoint{1.150000in}{0.150000in}}{\pgfqpoint{5.700000in}{5.700000in}}%
\pgfusepath{clip}%
\pgfsetbuttcap%
\pgfsetroundjoin%
\definecolor{currentfill}{rgb}{0.281924,0.089666,0.412415}%
\pgfsetfillcolor{currentfill}%
\pgfsetfillopacity{0.700000}%
\pgfsetlinewidth{0.000000pt}%
\definecolor{currentstroke}{rgb}{0.000000,0.000000,0.000000}%
\pgfsetstrokecolor{currentstroke}%
\pgfsetdash{}{0pt}%
\pgfpathmoveto{\pgfqpoint{3.215339in}{2.435696in}}%
\pgfpathlineto{\pgfqpoint{3.228749in}{2.426224in}}%
\pgfpathlineto{\pgfqpoint{3.242160in}{2.416897in}}%
\pgfpathlineto{\pgfqpoint{3.255572in}{2.407715in}}%
\pgfpathlineto{\pgfqpoint{3.268984in}{2.398675in}}%
\pgfpathlineto{\pgfqpoint{3.260850in}{2.393012in}}%
\pgfpathlineto{\pgfqpoint{3.252708in}{2.387444in}}%
\pgfpathlineto{\pgfqpoint{3.244558in}{2.381972in}}%
\pgfpathlineto{\pgfqpoint{3.236399in}{2.376599in}}%
\pgfpathlineto{\pgfqpoint{3.222965in}{2.385889in}}%
\pgfpathlineto{\pgfqpoint{3.209532in}{2.395321in}}%
\pgfpathlineto{\pgfqpoint{3.196099in}{2.404898in}}%
\pgfpathlineto{\pgfqpoint{3.182667in}{2.414621in}}%
\pgfpathlineto{\pgfqpoint{3.190848in}{2.419737in}}%
\pgfpathlineto{\pgfqpoint{3.199020in}{2.424956in}}%
\pgfpathlineto{\pgfqpoint{3.207184in}{2.430277in}}%
\pgfpathlineto{\pgfqpoint{3.215339in}{2.435696in}}%
\pgfpathclose%
\pgfusepath{fill}%
\end{pgfscope}%
\begin{pgfscope}%
\pgfpathrectangle{\pgfqpoint{1.150000in}{0.150000in}}{\pgfqpoint{5.700000in}{5.700000in}}%
\pgfusepath{clip}%
\pgfsetbuttcap%
\pgfsetroundjoin%
\definecolor{currentfill}{rgb}{0.279574,0.170599,0.479997}%
\pgfsetfillcolor{currentfill}%
\pgfsetfillopacity{0.700000}%
\pgfsetlinewidth{0.000000pt}%
\definecolor{currentstroke}{rgb}{0.000000,0.000000,0.000000}%
\pgfsetstrokecolor{currentstroke}%
\pgfsetdash{}{0pt}%
\pgfpathmoveto{\pgfqpoint{4.701415in}{2.579469in}}%
\pgfpathlineto{\pgfqpoint{4.715128in}{2.581085in}}%
\pgfpathlineto{\pgfqpoint{4.728851in}{2.582804in}}%
\pgfpathlineto{\pgfqpoint{4.742584in}{2.584627in}}%
\pgfpathlineto{\pgfqpoint{4.756328in}{2.586553in}}%
\pgfpathlineto{\pgfqpoint{4.748753in}{2.577815in}}%
\pgfpathlineto{\pgfqpoint{4.741172in}{2.569031in}}%
\pgfpathlineto{\pgfqpoint{4.733586in}{2.560199in}}%
\pgfpathlineto{\pgfqpoint{4.725994in}{2.551319in}}%
\pgfpathlineto{\pgfqpoint{4.712241in}{2.549404in}}%
\pgfpathlineto{\pgfqpoint{4.698499in}{2.547591in}}%
\pgfpathlineto{\pgfqpoint{4.684767in}{2.545882in}}%
\pgfpathlineto{\pgfqpoint{4.671045in}{2.544277in}}%
\pgfpathlineto{\pgfqpoint{4.678646in}{2.553139in}}%
\pgfpathlineto{\pgfqpoint{4.686242in}{2.561959in}}%
\pgfpathlineto{\pgfqpoint{4.693831in}{2.570735in}}%
\pgfpathlineto{\pgfqpoint{4.701415in}{2.579469in}}%
\pgfpathclose%
\pgfusepath{fill}%
\end{pgfscope}%
\begin{pgfscope}%
\pgfpathrectangle{\pgfqpoint{1.150000in}{0.150000in}}{\pgfqpoint{5.700000in}{5.700000in}}%
\pgfusepath{clip}%
\pgfsetbuttcap%
\pgfsetroundjoin%
\definecolor{currentfill}{rgb}{0.276022,0.044167,0.370164}%
\pgfsetfillcolor{currentfill}%
\pgfsetfillopacity{0.700000}%
\pgfsetlinewidth{0.000000pt}%
\definecolor{currentstroke}{rgb}{0.000000,0.000000,0.000000}%
\pgfsetstrokecolor{currentstroke}%
\pgfsetdash{}{0pt}%
\pgfpathmoveto{\pgfqpoint{3.408635in}{2.356787in}}%
\pgfpathlineto{\pgfqpoint{3.422044in}{2.349223in}}%
\pgfpathlineto{\pgfqpoint{3.435455in}{2.341794in}}%
\pgfpathlineto{\pgfqpoint{3.448868in}{2.334497in}}%
\pgfpathlineto{\pgfqpoint{3.462284in}{2.327333in}}%
\pgfpathlineto{\pgfqpoint{3.454239in}{2.320646in}}%
\pgfpathlineto{\pgfqpoint{3.446187in}{2.314030in}}%
\pgfpathlineto{\pgfqpoint{3.438128in}{2.307488in}}%
\pgfpathlineto{\pgfqpoint{3.430062in}{2.301022in}}%
\pgfpathlineto{\pgfqpoint{3.416627in}{2.308416in}}%
\pgfpathlineto{\pgfqpoint{3.403195in}{2.315942in}}%
\pgfpathlineto{\pgfqpoint{3.389765in}{2.323601in}}%
\pgfpathlineto{\pgfqpoint{3.376338in}{2.331394in}}%
\pgfpathlineto{\pgfqpoint{3.384423in}{2.337624in}}%
\pgfpathlineto{\pgfqpoint{3.392501in}{2.343934in}}%
\pgfpathlineto{\pgfqpoint{3.400572in}{2.350322in}}%
\pgfpathlineto{\pgfqpoint{3.408635in}{2.356787in}}%
\pgfpathclose%
\pgfusepath{fill}%
\end{pgfscope}%
\begin{pgfscope}%
\pgfpathrectangle{\pgfqpoint{1.150000in}{0.150000in}}{\pgfqpoint{5.700000in}{5.700000in}}%
\pgfusepath{clip}%
\pgfsetbuttcap%
\pgfsetroundjoin%
\definecolor{currentfill}{rgb}{0.281887,0.150881,0.465405}%
\pgfsetfillcolor{currentfill}%
\pgfsetfillopacity{0.700000}%
\pgfsetlinewidth{0.000000pt}%
\definecolor{currentstroke}{rgb}{0.000000,0.000000,0.000000}%
\pgfsetstrokecolor{currentstroke}%
\pgfsetdash{}{0pt}%
\pgfpathmoveto{\pgfqpoint{4.616258in}{2.538893in}}%
\pgfpathlineto{\pgfqpoint{4.629940in}{2.540083in}}%
\pgfpathlineto{\pgfqpoint{4.643632in}{2.541377in}}%
\pgfpathlineto{\pgfqpoint{4.657334in}{2.542775in}}%
\pgfpathlineto{\pgfqpoint{4.671045in}{2.544277in}}%
\pgfpathlineto{\pgfqpoint{4.663439in}{2.535371in}}%
\pgfpathlineto{\pgfqpoint{4.655827in}{2.526422in}}%
\pgfpathlineto{\pgfqpoint{4.648209in}{2.517429in}}%
\pgfpathlineto{\pgfqpoint{4.640585in}{2.508393in}}%
\pgfpathlineto{\pgfqpoint{4.626865in}{2.506920in}}%
\pgfpathlineto{\pgfqpoint{4.613154in}{2.505550in}}%
\pgfpathlineto{\pgfqpoint{4.599454in}{2.504285in}}%
\pgfpathlineto{\pgfqpoint{4.585763in}{2.503124in}}%
\pgfpathlineto{\pgfqpoint{4.593395in}{2.512125in}}%
\pgfpathlineto{\pgfqpoint{4.601021in}{2.521087in}}%
\pgfpathlineto{\pgfqpoint{4.608642in}{2.530009in}}%
\pgfpathlineto{\pgfqpoint{4.616258in}{2.538893in}}%
\pgfpathclose%
\pgfusepath{fill}%
\end{pgfscope}%
\begin{pgfscope}%
\pgfpathrectangle{\pgfqpoint{1.150000in}{0.150000in}}{\pgfqpoint{5.700000in}{5.700000in}}%
\pgfusepath{clip}%
\pgfsetbuttcap%
\pgfsetroundjoin%
\definecolor{currentfill}{rgb}{0.271305,0.019942,0.347269}%
\pgfsetfillcolor{currentfill}%
\pgfsetfillopacity{0.700000}%
\pgfsetlinewidth{0.000000pt}%
\definecolor{currentstroke}{rgb}{0.000000,0.000000,0.000000}%
\pgfsetstrokecolor{currentstroke}%
\pgfsetdash{}{0pt}%
\pgfpathmoveto{\pgfqpoint{3.826425in}{2.307489in}}%
\pgfpathlineto{\pgfqpoint{3.839882in}{2.303523in}}%
\pgfpathlineto{\pgfqpoint{3.853343in}{2.299675in}}%
\pgfpathlineto{\pgfqpoint{3.866811in}{2.295945in}}%
\pgfpathlineto{\pgfqpoint{3.880284in}{2.292333in}}%
\pgfpathlineto{\pgfqpoint{3.872405in}{2.283922in}}%
\pgfpathlineto{\pgfqpoint{3.864520in}{2.275534in}}%
\pgfpathlineto{\pgfqpoint{3.856630in}{2.267173in}}%
\pgfpathlineto{\pgfqpoint{3.848734in}{2.258838in}}%
\pgfpathlineto{\pgfqpoint{3.835248in}{2.262624in}}%
\pgfpathlineto{\pgfqpoint{3.821768in}{2.266528in}}%
\pgfpathlineto{\pgfqpoint{3.808294in}{2.270549in}}%
\pgfpathlineto{\pgfqpoint{3.794824in}{2.274688in}}%
\pgfpathlineto{\pgfqpoint{3.802733in}{2.282843in}}%
\pgfpathlineto{\pgfqpoint{3.810636in}{2.291029in}}%
\pgfpathlineto{\pgfqpoint{3.818534in}{2.299245in}}%
\pgfpathlineto{\pgfqpoint{3.826425in}{2.307489in}}%
\pgfpathclose%
\pgfusepath{fill}%
\end{pgfscope}%
\begin{pgfscope}%
\pgfpathrectangle{\pgfqpoint{1.150000in}{0.150000in}}{\pgfqpoint{5.700000in}{5.700000in}}%
\pgfusepath{clip}%
\pgfsetbuttcap%
\pgfsetroundjoin%
\definecolor{currentfill}{rgb}{0.274952,0.037752,0.364543}%
\pgfsetfillcolor{currentfill}%
\pgfsetfillopacity{0.700000}%
\pgfsetlinewidth{0.000000pt}%
\definecolor{currentstroke}{rgb}{0.000000,0.000000,0.000000}%
\pgfsetstrokecolor{currentstroke}%
\pgfsetdash{}{0pt}%
\pgfpathmoveto{\pgfqpoint{4.050924in}{2.337924in}}%
\pgfpathlineto{\pgfqpoint{4.064430in}{2.335639in}}%
\pgfpathlineto{\pgfqpoint{4.077942in}{2.333466in}}%
\pgfpathlineto{\pgfqpoint{4.091462in}{2.331405in}}%
\pgfpathlineto{\pgfqpoint{4.104989in}{2.329456in}}%
\pgfpathlineto{\pgfqpoint{4.097188in}{2.320509in}}%
\pgfpathlineto{\pgfqpoint{4.089382in}{2.311563in}}%
\pgfpathlineto{\pgfqpoint{4.081571in}{2.302618in}}%
\pgfpathlineto{\pgfqpoint{4.073754in}{2.293675in}}%
\pgfpathlineto{\pgfqpoint{4.060217in}{2.295761in}}%
\pgfpathlineto{\pgfqpoint{4.046687in}{2.297959in}}%
\pgfpathlineto{\pgfqpoint{4.033163in}{2.300270in}}%
\pgfpathlineto{\pgfqpoint{4.019646in}{2.302693in}}%
\pgfpathlineto{\pgfqpoint{4.027474in}{2.311491in}}%
\pgfpathlineto{\pgfqpoint{4.035296in}{2.320297in}}%
\pgfpathlineto{\pgfqpoint{4.043113in}{2.329108in}}%
\pgfpathlineto{\pgfqpoint{4.050924in}{2.337924in}}%
\pgfpathclose%
\pgfusepath{fill}%
\end{pgfscope}%
\begin{pgfscope}%
\pgfpathrectangle{\pgfqpoint{1.150000in}{0.150000in}}{\pgfqpoint{5.700000in}{5.700000in}}%
\pgfusepath{clip}%
\pgfsetbuttcap%
\pgfsetroundjoin%
\definecolor{currentfill}{rgb}{0.280255,0.165693,0.476498}%
\pgfsetfillcolor{currentfill}%
\pgfsetfillopacity{0.700000}%
\pgfsetlinewidth{0.000000pt}%
\definecolor{currentstroke}{rgb}{0.000000,0.000000,0.000000}%
\pgfsetstrokecolor{currentstroke}%
\pgfsetdash{}{0pt}%
\pgfpathmoveto{\pgfqpoint{2.967722in}{2.590806in}}%
\pgfpathlineto{\pgfqpoint{2.981163in}{2.578610in}}%
\pgfpathlineto{\pgfqpoint{2.994603in}{2.566578in}}%
\pgfpathlineto{\pgfqpoint{3.008041in}{2.554708in}}%
\pgfpathlineto{\pgfqpoint{3.021478in}{2.542999in}}%
\pgfpathlineto{\pgfqpoint{3.013215in}{2.538779in}}%
\pgfpathlineto{\pgfqpoint{3.004943in}{2.534682in}}%
\pgfpathlineto{\pgfqpoint{2.996661in}{2.530710in}}%
\pgfpathlineto{\pgfqpoint{2.988369in}{2.526865in}}%
\pgfpathlineto{\pgfqpoint{2.974906in}{2.538846in}}%
\pgfpathlineto{\pgfqpoint{2.961441in}{2.550988in}}%
\pgfpathlineto{\pgfqpoint{2.947975in}{2.563293in}}%
\pgfpathlineto{\pgfqpoint{2.934506in}{2.575762in}}%
\pgfpathlineto{\pgfqpoint{2.942826in}{2.579327in}}%
\pgfpathlineto{\pgfqpoint{2.951135in}{2.583025in}}%
\pgfpathlineto{\pgfqpoint{2.959433in}{2.586852in}}%
\pgfpathlineto{\pgfqpoint{2.967722in}{2.590806in}}%
\pgfpathclose%
\pgfusepath{fill}%
\end{pgfscope}%
\begin{pgfscope}%
\pgfpathrectangle{\pgfqpoint{1.150000in}{0.150000in}}{\pgfqpoint{5.700000in}{5.700000in}}%
\pgfusepath{clip}%
\pgfsetbuttcap%
\pgfsetroundjoin%
\definecolor{currentfill}{rgb}{0.283072,0.130895,0.449241}%
\pgfsetfillcolor{currentfill}%
\pgfsetfillopacity{0.700000}%
\pgfsetlinewidth{0.000000pt}%
\definecolor{currentstroke}{rgb}{0.000000,0.000000,0.000000}%
\pgfsetstrokecolor{currentstroke}%
\pgfsetdash{}{0pt}%
\pgfpathmoveto{\pgfqpoint{4.531095in}{2.499529in}}%
\pgfpathlineto{\pgfqpoint{4.544748in}{2.500270in}}%
\pgfpathlineto{\pgfqpoint{4.558410in}{2.501117in}}%
\pgfpathlineto{\pgfqpoint{4.572082in}{2.502068in}}%
\pgfpathlineto{\pgfqpoint{4.585763in}{2.503124in}}%
\pgfpathlineto{\pgfqpoint{4.578125in}{2.494085in}}%
\pgfpathlineto{\pgfqpoint{4.570482in}{2.485007in}}%
\pgfpathlineto{\pgfqpoint{4.562833in}{2.475891in}}%
\pgfpathlineto{\pgfqpoint{4.555179in}{2.466736in}}%
\pgfpathlineto{\pgfqpoint{4.541489in}{2.465727in}}%
\pgfpathlineto{\pgfqpoint{4.527809in}{2.464822in}}%
\pgfpathlineto{\pgfqpoint{4.514138in}{2.464022in}}%
\pgfpathlineto{\pgfqpoint{4.500476in}{2.463328in}}%
\pgfpathlineto{\pgfqpoint{4.508139in}{2.472429in}}%
\pgfpathlineto{\pgfqpoint{4.515797in}{2.481496in}}%
\pgfpathlineto{\pgfqpoint{4.523448in}{2.490530in}}%
\pgfpathlineto{\pgfqpoint{4.531095in}{2.499529in}}%
\pgfpathclose%
\pgfusepath{fill}%
\end{pgfscope}%
\begin{pgfscope}%
\pgfpathrectangle{\pgfqpoint{1.150000in}{0.150000in}}{\pgfqpoint{5.700000in}{5.700000in}}%
\pgfusepath{clip}%
\pgfsetbuttcap%
\pgfsetroundjoin%
\definecolor{currentfill}{rgb}{0.283091,0.110553,0.431554}%
\pgfsetfillcolor{currentfill}%
\pgfsetfillopacity{0.700000}%
\pgfsetlinewidth{0.000000pt}%
\definecolor{currentstroke}{rgb}{0.000000,0.000000,0.000000}%
\pgfsetstrokecolor{currentstroke}%
\pgfsetdash{}{0pt}%
\pgfpathmoveto{\pgfqpoint{4.445922in}{2.461608in}}%
\pgfpathlineto{\pgfqpoint{4.459547in}{2.461879in}}%
\pgfpathlineto{\pgfqpoint{4.473181in}{2.462256in}}%
\pgfpathlineto{\pgfqpoint{4.486824in}{2.462739in}}%
\pgfpathlineto{\pgfqpoint{4.500476in}{2.463328in}}%
\pgfpathlineto{\pgfqpoint{4.492808in}{2.454193in}}%
\pgfpathlineto{\pgfqpoint{4.485135in}{2.445026in}}%
\pgfpathlineto{\pgfqpoint{4.477455in}{2.435825in}}%
\pgfpathlineto{\pgfqpoint{4.469771in}{2.426592in}}%
\pgfpathlineto{\pgfqpoint{4.456110in}{2.426068in}}%
\pgfpathlineto{\pgfqpoint{4.442458in}{2.425649in}}%
\pgfpathlineto{\pgfqpoint{4.428815in}{2.425337in}}%
\pgfpathlineto{\pgfqpoint{4.415181in}{2.425131in}}%
\pgfpathlineto{\pgfqpoint{4.422874in}{2.434293in}}%
\pgfpathlineto{\pgfqpoint{4.430562in}{2.443426in}}%
\pgfpathlineto{\pgfqpoint{4.438245in}{2.452531in}}%
\pgfpathlineto{\pgfqpoint{4.445922in}{2.461608in}}%
\pgfpathclose%
\pgfusepath{fill}%
\end{pgfscope}%
\begin{pgfscope}%
\pgfpathrectangle{\pgfqpoint{1.150000in}{0.150000in}}{\pgfqpoint{5.700000in}{5.700000in}}%
\pgfusepath{clip}%
\pgfsetbuttcap%
\pgfsetroundjoin%
\definecolor{currentfill}{rgb}{0.280267,0.073417,0.397163}%
\pgfsetfillcolor{currentfill}%
\pgfsetfillopacity{0.700000}%
\pgfsetlinewidth{0.000000pt}%
\definecolor{currentstroke}{rgb}{0.000000,0.000000,0.000000}%
\pgfsetstrokecolor{currentstroke}%
\pgfsetdash{}{0pt}%
\pgfpathmoveto{\pgfqpoint{3.268984in}{2.398675in}}%
\pgfpathlineto{\pgfqpoint{3.282398in}{2.389777in}}%
\pgfpathlineto{\pgfqpoint{3.295813in}{2.381020in}}%
\pgfpathlineto{\pgfqpoint{3.309230in}{2.372404in}}%
\pgfpathlineto{\pgfqpoint{3.322648in}{2.363927in}}%
\pgfpathlineto{\pgfqpoint{3.314535in}{2.358022in}}%
\pgfpathlineto{\pgfqpoint{3.306413in}{2.352207in}}%
\pgfpathlineto{\pgfqpoint{3.298284in}{2.346484in}}%
\pgfpathlineto{\pgfqpoint{3.290147in}{2.340855in}}%
\pgfpathlineto{\pgfqpoint{3.276708in}{2.349581in}}%
\pgfpathlineto{\pgfqpoint{3.263271in}{2.358446in}}%
\pgfpathlineto{\pgfqpoint{3.249835in}{2.367452in}}%
\pgfpathlineto{\pgfqpoint{3.236399in}{2.376599in}}%
\pgfpathlineto{\pgfqpoint{3.244558in}{2.381972in}}%
\pgfpathlineto{\pgfqpoint{3.252708in}{2.387444in}}%
\pgfpathlineto{\pgfqpoint{3.260850in}{2.393012in}}%
\pgfpathlineto{\pgfqpoint{3.268984in}{2.398675in}}%
\pgfpathclose%
\pgfusepath{fill}%
\end{pgfscope}%
\begin{pgfscope}%
\pgfpathrectangle{\pgfqpoint{1.150000in}{0.150000in}}{\pgfqpoint{5.700000in}{5.700000in}}%
\pgfusepath{clip}%
\pgfsetbuttcap%
\pgfsetroundjoin%
\definecolor{currentfill}{rgb}{0.282327,0.094955,0.417331}%
\pgfsetfillcolor{currentfill}%
\pgfsetfillopacity{0.700000}%
\pgfsetlinewidth{0.000000pt}%
\definecolor{currentstroke}{rgb}{0.000000,0.000000,0.000000}%
\pgfsetstrokecolor{currentstroke}%
\pgfsetdash{}{0pt}%
\pgfpathmoveto{\pgfqpoint{4.360733in}{2.425375in}}%
\pgfpathlineto{\pgfqpoint{4.374332in}{2.425153in}}%
\pgfpathlineto{\pgfqpoint{4.387940in}{2.425039in}}%
\pgfpathlineto{\pgfqpoint{4.401556in}{2.425032in}}%
\pgfpathlineto{\pgfqpoint{4.415181in}{2.425131in}}%
\pgfpathlineto{\pgfqpoint{4.407483in}{2.415942in}}%
\pgfpathlineto{\pgfqpoint{4.399779in}{2.406726in}}%
\pgfpathlineto{\pgfqpoint{4.392069in}{2.397483in}}%
\pgfpathlineto{\pgfqpoint{4.384355in}{2.388215in}}%
\pgfpathlineto{\pgfqpoint{4.370721in}{2.388198in}}%
\pgfpathlineto{\pgfqpoint{4.357095in}{2.388289in}}%
\pgfpathlineto{\pgfqpoint{4.343479in}{2.388486in}}%
\pgfpathlineto{\pgfqpoint{4.329870in}{2.388791in}}%
\pgfpathlineto{\pgfqpoint{4.337594in}{2.397970in}}%
\pgfpathlineto{\pgfqpoint{4.345312in}{2.407127in}}%
\pgfpathlineto{\pgfqpoint{4.353025in}{2.416262in}}%
\pgfpathlineto{\pgfqpoint{4.360733in}{2.425375in}}%
\pgfpathclose%
\pgfusepath{fill}%
\end{pgfscope}%
\begin{pgfscope}%
\pgfpathrectangle{\pgfqpoint{1.150000in}{0.150000in}}{\pgfqpoint{5.700000in}{5.700000in}}%
\pgfusepath{clip}%
\pgfsetbuttcap%
\pgfsetroundjoin%
\definecolor{currentfill}{rgb}{0.282290,0.145912,0.461510}%
\pgfsetfillcolor{currentfill}%
\pgfsetfillopacity{0.700000}%
\pgfsetlinewidth{0.000000pt}%
\definecolor{currentstroke}{rgb}{0.000000,0.000000,0.000000}%
\pgfsetstrokecolor{currentstroke}%
\pgfsetdash{}{0pt}%
\pgfpathmoveto{\pgfqpoint{3.021478in}{2.542999in}}%
\pgfpathlineto{\pgfqpoint{3.034914in}{2.531450in}}%
\pgfpathlineto{\pgfqpoint{3.048348in}{2.520060in}}%
\pgfpathlineto{\pgfqpoint{3.061782in}{2.508827in}}%
\pgfpathlineto{\pgfqpoint{3.075215in}{2.497750in}}%
\pgfpathlineto{\pgfqpoint{3.066977in}{2.493266in}}%
\pgfpathlineto{\pgfqpoint{3.058731in}{2.488900in}}%
\pgfpathlineto{\pgfqpoint{3.050474in}{2.484653in}}%
\pgfpathlineto{\pgfqpoint{3.042208in}{2.480530in}}%
\pgfpathlineto{\pgfqpoint{3.028750in}{2.491877in}}%
\pgfpathlineto{\pgfqpoint{3.015291in}{2.503382in}}%
\pgfpathlineto{\pgfqpoint{3.001830in}{2.515044in}}%
\pgfpathlineto{\pgfqpoint{2.988369in}{2.526865in}}%
\pgfpathlineto{\pgfqpoint{2.996661in}{2.530710in}}%
\pgfpathlineto{\pgfqpoint{3.004943in}{2.534682in}}%
\pgfpathlineto{\pgfqpoint{3.013215in}{2.538779in}}%
\pgfpathlineto{\pgfqpoint{3.021478in}{2.542999in}}%
\pgfpathclose%
\pgfusepath{fill}%
\end{pgfscope}%
\begin{pgfscope}%
\pgfpathrectangle{\pgfqpoint{1.150000in}{0.150000in}}{\pgfqpoint{5.700000in}{5.700000in}}%
\pgfusepath{clip}%
\pgfsetbuttcap%
\pgfsetroundjoin%
\definecolor{currentfill}{rgb}{0.271305,0.019942,0.347269}%
\pgfsetfillcolor{currentfill}%
\pgfsetfillopacity{0.700000}%
\pgfsetlinewidth{0.000000pt}%
\definecolor{currentstroke}{rgb}{0.000000,0.000000,0.000000}%
\pgfsetstrokecolor{currentstroke}%
\pgfsetdash{}{0pt}%
\pgfpathmoveto{\pgfqpoint{3.601698in}{2.303817in}}%
\pgfpathlineto{\pgfqpoint{3.615127in}{2.298023in}}%
\pgfpathlineto{\pgfqpoint{3.628560in}{2.292354in}}%
\pgfpathlineto{\pgfqpoint{3.641997in}{2.286809in}}%
\pgfpathlineto{\pgfqpoint{3.655438in}{2.281388in}}%
\pgfpathlineto{\pgfqpoint{3.647470in}{2.273821in}}%
\pgfpathlineto{\pgfqpoint{3.639495in}{2.266305in}}%
\pgfpathlineto{\pgfqpoint{3.631514in}{2.258841in}}%
\pgfpathlineto{\pgfqpoint{3.623527in}{2.251431in}}%
\pgfpathlineto{\pgfqpoint{3.610070in}{2.257062in}}%
\pgfpathlineto{\pgfqpoint{3.596617in}{2.262817in}}%
\pgfpathlineto{\pgfqpoint{3.583168in}{2.268696in}}%
\pgfpathlineto{\pgfqpoint{3.569723in}{2.274701in}}%
\pgfpathlineto{\pgfqpoint{3.577727in}{2.281894in}}%
\pgfpathlineto{\pgfqpoint{3.585724in}{2.289145in}}%
\pgfpathlineto{\pgfqpoint{3.593714in}{2.296454in}}%
\pgfpathlineto{\pgfqpoint{3.601698in}{2.303817in}}%
\pgfpathclose%
\pgfusepath{fill}%
\end{pgfscope}%
\begin{pgfscope}%
\pgfpathrectangle{\pgfqpoint{1.150000in}{0.150000in}}{\pgfqpoint{5.700000in}{5.700000in}}%
\pgfusepath{clip}%
\pgfsetbuttcap%
\pgfsetroundjoin%
\definecolor{currentfill}{rgb}{0.272594,0.025563,0.353093}%
\pgfsetfillcolor{currentfill}%
\pgfsetfillopacity{0.700000}%
\pgfsetlinewidth{0.000000pt}%
\definecolor{currentstroke}{rgb}{0.000000,0.000000,0.000000}%
\pgfsetstrokecolor{currentstroke}%
\pgfsetdash{}{0pt}%
\pgfpathmoveto{\pgfqpoint{3.965644in}{2.313521in}}%
\pgfpathlineto{\pgfqpoint{3.979135in}{2.310643in}}%
\pgfpathlineto{\pgfqpoint{3.992632in}{2.307879in}}%
\pgfpathlineto{\pgfqpoint{4.006136in}{2.305229in}}%
\pgfpathlineto{\pgfqpoint{4.019646in}{2.302693in}}%
\pgfpathlineto{\pgfqpoint{4.011813in}{2.293903in}}%
\pgfpathlineto{\pgfqpoint{4.003975in}{2.285122in}}%
\pgfpathlineto{\pgfqpoint{3.996131in}{2.276352in}}%
\pgfpathlineto{\pgfqpoint{3.988282in}{2.267595in}}%
\pgfpathlineto{\pgfqpoint{3.974760in}{2.270286in}}%
\pgfpathlineto{\pgfqpoint{3.961245in}{2.273092in}}%
\pgfpathlineto{\pgfqpoint{3.947736in}{2.276011in}}%
\pgfpathlineto{\pgfqpoint{3.934234in}{2.279044in}}%
\pgfpathlineto{\pgfqpoint{3.942095in}{2.287640in}}%
\pgfpathlineto{\pgfqpoint{3.949950in}{2.296252in}}%
\pgfpathlineto{\pgfqpoint{3.957800in}{2.304880in}}%
\pgfpathlineto{\pgfqpoint{3.965644in}{2.313521in}}%
\pgfpathclose%
\pgfusepath{fill}%
\end{pgfscope}%
\begin{pgfscope}%
\pgfpathrectangle{\pgfqpoint{1.150000in}{0.150000in}}{\pgfqpoint{5.700000in}{5.700000in}}%
\pgfusepath{clip}%
\pgfsetbuttcap%
\pgfsetroundjoin%
\definecolor{currentfill}{rgb}{0.273809,0.031497,0.358853}%
\pgfsetfillcolor{currentfill}%
\pgfsetfillopacity{0.700000}%
\pgfsetlinewidth{0.000000pt}%
\definecolor{currentstroke}{rgb}{0.000000,0.000000,0.000000}%
\pgfsetstrokecolor{currentstroke}%
\pgfsetdash{}{0pt}%
\pgfpathmoveto{\pgfqpoint{3.462284in}{2.327333in}}%
\pgfpathlineto{\pgfqpoint{3.475703in}{2.320301in}}%
\pgfpathlineto{\pgfqpoint{3.489125in}{2.313400in}}%
\pgfpathlineto{\pgfqpoint{3.502550in}{2.306629in}}%
\pgfpathlineto{\pgfqpoint{3.515978in}{2.299987in}}%
\pgfpathlineto{\pgfqpoint{3.507951in}{2.293077in}}%
\pgfpathlineto{\pgfqpoint{3.499917in}{2.286234in}}%
\pgfpathlineto{\pgfqpoint{3.491875in}{2.279461in}}%
\pgfpathlineto{\pgfqpoint{3.483827in}{2.272759in}}%
\pgfpathlineto{\pgfqpoint{3.470381in}{2.279630in}}%
\pgfpathlineto{\pgfqpoint{3.456939in}{2.286630in}}%
\pgfpathlineto{\pgfqpoint{3.443499in}{2.293761in}}%
\pgfpathlineto{\pgfqpoint{3.430062in}{2.301022in}}%
\pgfpathlineto{\pgfqpoint{3.438128in}{2.307488in}}%
\pgfpathlineto{\pgfqpoint{3.446187in}{2.314030in}}%
\pgfpathlineto{\pgfqpoint{3.454239in}{2.320646in}}%
\pgfpathlineto{\pgfqpoint{3.462284in}{2.327333in}}%
\pgfpathclose%
\pgfusepath{fill}%
\end{pgfscope}%
\begin{pgfscope}%
\pgfpathrectangle{\pgfqpoint{1.150000in}{0.150000in}}{\pgfqpoint{5.700000in}{5.700000in}}%
\pgfusepath{clip}%
\pgfsetbuttcap%
\pgfsetroundjoin%
\definecolor{currentfill}{rgb}{0.269944,0.014625,0.341379}%
\pgfsetfillcolor{currentfill}%
\pgfsetfillopacity{0.700000}%
\pgfsetlinewidth{0.000000pt}%
\definecolor{currentstroke}{rgb}{0.000000,0.000000,0.000000}%
\pgfsetstrokecolor{currentstroke}%
\pgfsetdash{}{0pt}%
\pgfpathmoveto{\pgfqpoint{3.740998in}{2.292439in}}%
\pgfpathlineto{\pgfqpoint{3.754447in}{2.287822in}}%
\pgfpathlineto{\pgfqpoint{3.767901in}{2.283324in}}%
\pgfpathlineto{\pgfqpoint{3.781360in}{2.278947in}}%
\pgfpathlineto{\pgfqpoint{3.794824in}{2.274688in}}%
\pgfpathlineto{\pgfqpoint{3.786909in}{2.266567in}}%
\pgfpathlineto{\pgfqpoint{3.778988in}{2.258480in}}%
\pgfpathlineto{\pgfqpoint{3.771061in}{2.250429in}}%
\pgfpathlineto{\pgfqpoint{3.763128in}{2.242416in}}%
\pgfpathlineto{\pgfqpoint{3.749651in}{2.246866in}}%
\pgfpathlineto{\pgfqpoint{3.736178in}{2.251435in}}%
\pgfpathlineto{\pgfqpoint{3.722709in}{2.256124in}}%
\pgfpathlineto{\pgfqpoint{3.709246in}{2.260934in}}%
\pgfpathlineto{\pgfqpoint{3.717193in}{2.268748in}}%
\pgfpathlineto{\pgfqpoint{3.725134in}{2.276605in}}%
\pgfpathlineto{\pgfqpoint{3.733069in}{2.284503in}}%
\pgfpathlineto{\pgfqpoint{3.740998in}{2.292439in}}%
\pgfpathclose%
\pgfusepath{fill}%
\end{pgfscope}%
\begin{pgfscope}%
\pgfpathrectangle{\pgfqpoint{1.150000in}{0.150000in}}{\pgfqpoint{5.700000in}{5.700000in}}%
\pgfusepath{clip}%
\pgfsetbuttcap%
\pgfsetroundjoin%
\definecolor{currentfill}{rgb}{0.280894,0.078907,0.402329}%
\pgfsetfillcolor{currentfill}%
\pgfsetfillopacity{0.700000}%
\pgfsetlinewidth{0.000000pt}%
\definecolor{currentstroke}{rgb}{0.000000,0.000000,0.000000}%
\pgfsetstrokecolor{currentstroke}%
\pgfsetdash{}{0pt}%
\pgfpathmoveto{\pgfqpoint{4.275520in}{2.391089in}}%
\pgfpathlineto{\pgfqpoint{4.289096in}{2.390352in}}%
\pgfpathlineto{\pgfqpoint{4.302679in}{2.389723in}}%
\pgfpathlineto{\pgfqpoint{4.316270in}{2.389203in}}%
\pgfpathlineto{\pgfqpoint{4.329870in}{2.388791in}}%
\pgfpathlineto{\pgfqpoint{4.322141in}{2.379591in}}%
\pgfpathlineto{\pgfqpoint{4.314407in}{2.370372in}}%
\pgfpathlineto{\pgfqpoint{4.306668in}{2.361132in}}%
\pgfpathlineto{\pgfqpoint{4.298923in}{2.351874in}}%
\pgfpathlineto{\pgfqpoint{4.285314in}{2.352388in}}%
\pgfpathlineto{\pgfqpoint{4.271713in}{2.353009in}}%
\pgfpathlineto{\pgfqpoint{4.258121in}{2.353739in}}%
\pgfpathlineto{\pgfqpoint{4.244536in}{2.354577in}}%
\pgfpathlineto{\pgfqpoint{4.252290in}{2.363727in}}%
\pgfpathlineto{\pgfqpoint{4.260039in}{2.372863in}}%
\pgfpathlineto{\pgfqpoint{4.267782in}{2.381984in}}%
\pgfpathlineto{\pgfqpoint{4.275520in}{2.391089in}}%
\pgfpathclose%
\pgfusepath{fill}%
\end{pgfscope}%
\begin{pgfscope}%
\pgfpathrectangle{\pgfqpoint{1.150000in}{0.150000in}}{\pgfqpoint{5.700000in}{5.700000in}}%
\pgfusepath{clip}%
\pgfsetbuttcap%
\pgfsetroundjoin%
\definecolor{currentfill}{rgb}{0.283187,0.125848,0.444960}%
\pgfsetfillcolor{currentfill}%
\pgfsetfillopacity{0.700000}%
\pgfsetlinewidth{0.000000pt}%
\definecolor{currentstroke}{rgb}{0.000000,0.000000,0.000000}%
\pgfsetstrokecolor{currentstroke}%
\pgfsetdash{}{0pt}%
\pgfpathmoveto{\pgfqpoint{3.075215in}{2.497750in}}%
\pgfpathlineto{\pgfqpoint{3.088647in}{2.486829in}}%
\pgfpathlineto{\pgfqpoint{3.102079in}{2.476061in}}%
\pgfpathlineto{\pgfqpoint{3.115510in}{2.465445in}}%
\pgfpathlineto{\pgfqpoint{3.128941in}{2.454982in}}%
\pgfpathlineto{\pgfqpoint{3.120728in}{2.450234in}}%
\pgfpathlineto{\pgfqpoint{3.112506in}{2.445599in}}%
\pgfpathlineto{\pgfqpoint{3.104275in}{2.441080in}}%
\pgfpathlineto{\pgfqpoint{3.096034in}{2.436679in}}%
\pgfpathlineto{\pgfqpoint{3.082578in}{2.447413in}}%
\pgfpathlineto{\pgfqpoint{3.069122in}{2.458298in}}%
\pgfpathlineto{\pgfqpoint{3.055666in}{2.469337in}}%
\pgfpathlineto{\pgfqpoint{3.042208in}{2.480530in}}%
\pgfpathlineto{\pgfqpoint{3.050474in}{2.484653in}}%
\pgfpathlineto{\pgfqpoint{3.058731in}{2.488900in}}%
\pgfpathlineto{\pgfqpoint{3.066977in}{2.493266in}}%
\pgfpathlineto{\pgfqpoint{3.075215in}{2.497750in}}%
\pgfpathclose%
\pgfusepath{fill}%
\end{pgfscope}%
\begin{pgfscope}%
\pgfpathrectangle{\pgfqpoint{1.150000in}{0.150000in}}{\pgfqpoint{5.700000in}{5.700000in}}%
\pgfusepath{clip}%
\pgfsetbuttcap%
\pgfsetroundjoin%
\definecolor{currentfill}{rgb}{0.278791,0.062145,0.386592}%
\pgfsetfillcolor{currentfill}%
\pgfsetfillopacity{0.700000}%
\pgfsetlinewidth{0.000000pt}%
\definecolor{currentstroke}{rgb}{0.000000,0.000000,0.000000}%
\pgfsetstrokecolor{currentstroke}%
\pgfsetdash{}{0pt}%
\pgfpathmoveto{\pgfqpoint{4.190276in}{2.359021in}}%
\pgfpathlineto{\pgfqpoint{4.203829in}{2.357746in}}%
\pgfpathlineto{\pgfqpoint{4.217390in}{2.356580in}}%
\pgfpathlineto{\pgfqpoint{4.230959in}{2.355524in}}%
\pgfpathlineto{\pgfqpoint{4.244536in}{2.354577in}}%
\pgfpathlineto{\pgfqpoint{4.236777in}{2.345414in}}%
\pgfpathlineto{\pgfqpoint{4.229012in}{2.336238in}}%
\pgfpathlineto{\pgfqpoint{4.221242in}{2.327050in}}%
\pgfpathlineto{\pgfqpoint{4.213467in}{2.317852in}}%
\pgfpathlineto{\pgfqpoint{4.199880in}{2.318918in}}%
\pgfpathlineto{\pgfqpoint{4.186302in}{2.320094in}}%
\pgfpathlineto{\pgfqpoint{4.172731in}{2.321378in}}%
\pgfpathlineto{\pgfqpoint{4.159168in}{2.322773in}}%
\pgfpathlineto{\pgfqpoint{4.166953in}{2.331845in}}%
\pgfpathlineto{\pgfqpoint{4.174733in}{2.340911in}}%
\pgfpathlineto{\pgfqpoint{4.182507in}{2.349970in}}%
\pgfpathlineto{\pgfqpoint{4.190276in}{2.359021in}}%
\pgfpathclose%
\pgfusepath{fill}%
\end{pgfscope}%
\begin{pgfscope}%
\pgfpathrectangle{\pgfqpoint{1.150000in}{0.150000in}}{\pgfqpoint{5.700000in}{5.700000in}}%
\pgfusepath{clip}%
\pgfsetbuttcap%
\pgfsetroundjoin%
\definecolor{currentfill}{rgb}{0.271305,0.019942,0.347269}%
\pgfsetfillcolor{currentfill}%
\pgfsetfillopacity{0.700000}%
\pgfsetlinewidth{0.000000pt}%
\definecolor{currentstroke}{rgb}{0.000000,0.000000,0.000000}%
\pgfsetstrokecolor{currentstroke}%
\pgfsetdash{}{0pt}%
\pgfpathmoveto{\pgfqpoint{3.880284in}{2.292333in}}%
\pgfpathlineto{\pgfqpoint{3.893762in}{2.288837in}}%
\pgfpathlineto{\pgfqpoint{3.907247in}{2.285457in}}%
\pgfpathlineto{\pgfqpoint{3.920737in}{2.282193in}}%
\pgfpathlineto{\pgfqpoint{3.934234in}{2.279044in}}%
\pgfpathlineto{\pgfqpoint{3.926367in}{2.270467in}}%
\pgfpathlineto{\pgfqpoint{3.918495in}{2.261909in}}%
\pgfpathlineto{\pgfqpoint{3.910617in}{2.253371in}}%
\pgfpathlineto{\pgfqpoint{3.902733in}{2.244856in}}%
\pgfpathlineto{\pgfqpoint{3.889225in}{2.248178in}}%
\pgfpathlineto{\pgfqpoint{3.875722in}{2.251615in}}%
\pgfpathlineto{\pgfqpoint{3.862225in}{2.255168in}}%
\pgfpathlineto{\pgfqpoint{3.848734in}{2.258838in}}%
\pgfpathlineto{\pgfqpoint{3.856630in}{2.267173in}}%
\pgfpathlineto{\pgfqpoint{3.864520in}{2.275534in}}%
\pgfpathlineto{\pgfqpoint{3.872405in}{2.283922in}}%
\pgfpathlineto{\pgfqpoint{3.880284in}{2.292333in}}%
\pgfpathclose%
\pgfusepath{fill}%
\end{pgfscope}%
\begin{pgfscope}%
\pgfpathrectangle{\pgfqpoint{1.150000in}{0.150000in}}{\pgfqpoint{5.700000in}{5.700000in}}%
\pgfusepath{clip}%
\pgfsetbuttcap%
\pgfsetroundjoin%
\definecolor{currentfill}{rgb}{0.278791,0.062145,0.386592}%
\pgfsetfillcolor{currentfill}%
\pgfsetfillopacity{0.700000}%
\pgfsetlinewidth{0.000000pt}%
\definecolor{currentstroke}{rgb}{0.000000,0.000000,0.000000}%
\pgfsetstrokecolor{currentstroke}%
\pgfsetdash{}{0pt}%
\pgfpathmoveto{\pgfqpoint{3.322648in}{2.363927in}}%
\pgfpathlineto{\pgfqpoint{3.336067in}{2.355588in}}%
\pgfpathlineto{\pgfqpoint{3.349489in}{2.347387in}}%
\pgfpathlineto{\pgfqpoint{3.362912in}{2.339323in}}%
\pgfpathlineto{\pgfqpoint{3.376338in}{2.331394in}}%
\pgfpathlineto{\pgfqpoint{3.368244in}{2.325248in}}%
\pgfpathlineto{\pgfqpoint{3.360144in}{2.319187in}}%
\pgfpathlineto{\pgfqpoint{3.352035in}{2.313212in}}%
\pgfpathlineto{\pgfqpoint{3.343919in}{2.307328in}}%
\pgfpathlineto{\pgfqpoint{3.330473in}{2.315505in}}%
\pgfpathlineto{\pgfqpoint{3.317030in}{2.323818in}}%
\pgfpathlineto{\pgfqpoint{3.303588in}{2.332267in}}%
\pgfpathlineto{\pgfqpoint{3.290147in}{2.340855in}}%
\pgfpathlineto{\pgfqpoint{3.298284in}{2.346484in}}%
\pgfpathlineto{\pgfqpoint{3.306413in}{2.352207in}}%
\pgfpathlineto{\pgfqpoint{3.314535in}{2.358022in}}%
\pgfpathlineto{\pgfqpoint{3.322648in}{2.363927in}}%
\pgfpathclose%
\pgfusepath{fill}%
\end{pgfscope}%
\begin{pgfscope}%
\pgfpathrectangle{\pgfqpoint{1.150000in}{0.150000in}}{\pgfqpoint{5.700000in}{5.700000in}}%
\pgfusepath{clip}%
\pgfsetbuttcap%
\pgfsetroundjoin%
\definecolor{currentfill}{rgb}{0.283091,0.110553,0.431554}%
\pgfsetfillcolor{currentfill}%
\pgfsetfillopacity{0.700000}%
\pgfsetlinewidth{0.000000pt}%
\definecolor{currentstroke}{rgb}{0.000000,0.000000,0.000000}%
\pgfsetstrokecolor{currentstroke}%
\pgfsetdash{}{0pt}%
\pgfpathmoveto{\pgfqpoint{3.128941in}{2.454982in}}%
\pgfpathlineto{\pgfqpoint{3.142373in}{2.444668in}}%
\pgfpathlineto{\pgfqpoint{3.155804in}{2.434505in}}%
\pgfpathlineto{\pgfqpoint{3.169235in}{2.424489in}}%
\pgfpathlineto{\pgfqpoint{3.182667in}{2.414621in}}%
\pgfpathlineto{\pgfqpoint{3.174477in}{2.409610in}}%
\pgfpathlineto{\pgfqpoint{3.166279in}{2.404709in}}%
\pgfpathlineto{\pgfqpoint{3.158071in}{2.399918in}}%
\pgfpathlineto{\pgfqpoint{3.149855in}{2.395240in}}%
\pgfpathlineto{\pgfqpoint{3.136400in}{2.405377in}}%
\pgfpathlineto{\pgfqpoint{3.122944in}{2.415662in}}%
\pgfpathlineto{\pgfqpoint{3.109489in}{2.426096in}}%
\pgfpathlineto{\pgfqpoint{3.096034in}{2.436679in}}%
\pgfpathlineto{\pgfqpoint{3.104275in}{2.441080in}}%
\pgfpathlineto{\pgfqpoint{3.112506in}{2.445599in}}%
\pgfpathlineto{\pgfqpoint{3.120728in}{2.450234in}}%
\pgfpathlineto{\pgfqpoint{3.128941in}{2.454982in}}%
\pgfpathclose%
\pgfusepath{fill}%
\end{pgfscope}%
\begin{pgfscope}%
\pgfpathrectangle{\pgfqpoint{1.150000in}{0.150000in}}{\pgfqpoint{5.700000in}{5.700000in}}%
\pgfusepath{clip}%
\pgfsetbuttcap%
\pgfsetroundjoin%
\definecolor{currentfill}{rgb}{0.276022,0.044167,0.370164}%
\pgfsetfillcolor{currentfill}%
\pgfsetfillopacity{0.700000}%
\pgfsetlinewidth{0.000000pt}%
\definecolor{currentstroke}{rgb}{0.000000,0.000000,0.000000}%
\pgfsetstrokecolor{currentstroke}%
\pgfsetdash{}{0pt}%
\pgfpathmoveto{\pgfqpoint{4.104989in}{2.329456in}}%
\pgfpathlineto{\pgfqpoint{4.118523in}{2.327619in}}%
\pgfpathlineto{\pgfqpoint{4.132064in}{2.325893in}}%
\pgfpathlineto{\pgfqpoint{4.145612in}{2.324278in}}%
\pgfpathlineto{\pgfqpoint{4.159168in}{2.322773in}}%
\pgfpathlineto{\pgfqpoint{4.151378in}{2.313696in}}%
\pgfpathlineto{\pgfqpoint{4.143582in}{2.304614in}}%
\pgfpathlineto{\pgfqpoint{4.135781in}{2.295530in}}%
\pgfpathlineto{\pgfqpoint{4.127975in}{2.286443in}}%
\pgfpathlineto{\pgfqpoint{4.114409in}{2.288085in}}%
\pgfpathlineto{\pgfqpoint{4.100850in}{2.289837in}}%
\pgfpathlineto{\pgfqpoint{4.087299in}{2.291701in}}%
\pgfpathlineto{\pgfqpoint{4.073754in}{2.293675in}}%
\pgfpathlineto{\pgfqpoint{4.081571in}{2.302618in}}%
\pgfpathlineto{\pgfqpoint{4.089382in}{2.311563in}}%
\pgfpathlineto{\pgfqpoint{4.097188in}{2.320509in}}%
\pgfpathlineto{\pgfqpoint{4.104989in}{2.329456in}}%
\pgfpathclose%
\pgfusepath{fill}%
\end{pgfscope}%
\begin{pgfscope}%
\pgfpathrectangle{\pgfqpoint{1.150000in}{0.150000in}}{\pgfqpoint{5.700000in}{5.700000in}}%
\pgfusepath{clip}%
\pgfsetbuttcap%
\pgfsetroundjoin%
\definecolor{currentfill}{rgb}{0.185556,0.418570,0.556753}%
\pgfsetfillcolor{currentfill}%
\pgfsetfillopacity{0.700000}%
\pgfsetlinewidth{0.000000pt}%
\definecolor{currentstroke}{rgb}{0.000000,0.000000,0.000000}%
\pgfsetstrokecolor{currentstroke}%
\pgfsetdash{}{0pt}%
\pgfpathmoveto{\pgfqpoint{5.779515in}{3.114348in}}%
\pgfpathlineto{\pgfqpoint{5.793710in}{3.119825in}}%
\pgfpathlineto{\pgfqpoint{5.807920in}{3.125399in}}%
\pgfpathlineto{\pgfqpoint{5.822144in}{3.131071in}}%
\pgfpathlineto{\pgfqpoint{5.836382in}{3.136840in}}%
\pgfpathlineto{\pgfqpoint{5.829295in}{3.132003in}}%
\pgfpathlineto{\pgfqpoint{5.822201in}{3.127128in}}%
\pgfpathlineto{\pgfqpoint{5.815099in}{3.122211in}}%
\pgfpathlineto{\pgfqpoint{5.807991in}{3.117250in}}%
\pgfpathlineto{\pgfqpoint{5.793733in}{3.111264in}}%
\pgfpathlineto{\pgfqpoint{5.779490in}{3.105375in}}%
\pgfpathlineto{\pgfqpoint{5.765261in}{3.099584in}}%
\pgfpathlineto{\pgfqpoint{5.751047in}{3.093891in}}%
\pgfpathlineto{\pgfqpoint{5.758175in}{3.099062in}}%
\pgfpathlineto{\pgfqpoint{5.765295in}{3.104193in}}%
\pgfpathlineto{\pgfqpoint{5.772408in}{3.109288in}}%
\pgfpathlineto{\pgfqpoint{5.779515in}{3.114348in}}%
\pgfpathclose%
\pgfusepath{fill}%
\end{pgfscope}%
\begin{pgfscope}%
\pgfpathrectangle{\pgfqpoint{1.150000in}{0.150000in}}{\pgfqpoint{5.700000in}{5.700000in}}%
\pgfusepath{clip}%
\pgfsetbuttcap%
\pgfsetroundjoin%
\definecolor{currentfill}{rgb}{0.192357,0.403199,0.555836}%
\pgfsetfillcolor{currentfill}%
\pgfsetfillopacity{0.700000}%
\pgfsetlinewidth{0.000000pt}%
\definecolor{currentstroke}{rgb}{0.000000,0.000000,0.000000}%
\pgfsetstrokecolor{currentstroke}%
\pgfsetdash{}{0pt}%
\pgfpathmoveto{\pgfqpoint{5.694332in}{3.072094in}}%
\pgfpathlineto{\pgfqpoint{5.708490in}{3.077397in}}%
\pgfpathlineto{\pgfqpoint{5.722661in}{3.082797in}}%
\pgfpathlineto{\pgfqpoint{5.736847in}{3.088296in}}%
\pgfpathlineto{\pgfqpoint{5.751047in}{3.093891in}}%
\pgfpathlineto{\pgfqpoint{5.743913in}{3.088679in}}%
\pgfpathlineto{\pgfqpoint{5.736771in}{3.083422in}}%
\pgfpathlineto{\pgfqpoint{5.729623in}{3.078119in}}%
\pgfpathlineto{\pgfqpoint{5.722467in}{3.072767in}}%
\pgfpathlineto{\pgfqpoint{5.708249in}{3.066973in}}%
\pgfpathlineto{\pgfqpoint{5.694046in}{3.061278in}}%
\pgfpathlineto{\pgfqpoint{5.679856in}{3.055680in}}%
\pgfpathlineto{\pgfqpoint{5.665681in}{3.050180in}}%
\pgfpathlineto{\pgfqpoint{5.672854in}{3.055723in}}%
\pgfpathlineto{\pgfqpoint{5.680020in}{3.061221in}}%
\pgfpathlineto{\pgfqpoint{5.687180in}{3.066678in}}%
\pgfpathlineto{\pgfqpoint{5.694332in}{3.072094in}}%
\pgfpathclose%
\pgfusepath{fill}%
\end{pgfscope}%
\begin{pgfscope}%
\pgfpathrectangle{\pgfqpoint{1.150000in}{0.150000in}}{\pgfqpoint{5.700000in}{5.700000in}}%
\pgfusepath{clip}%
\pgfsetbuttcap%
\pgfsetroundjoin%
\definecolor{currentfill}{rgb}{0.201239,0.383670,0.554294}%
\pgfsetfillcolor{currentfill}%
\pgfsetfillopacity{0.700000}%
\pgfsetlinewidth{0.000000pt}%
\definecolor{currentstroke}{rgb}{0.000000,0.000000,0.000000}%
\pgfsetstrokecolor{currentstroke}%
\pgfsetdash{}{0pt}%
\pgfpathmoveto{\pgfqpoint{5.609118in}{3.029160in}}%
\pgfpathlineto{\pgfqpoint{5.623238in}{3.034268in}}%
\pgfpathlineto{\pgfqpoint{5.637372in}{3.039474in}}%
\pgfpathlineto{\pgfqpoint{5.651519in}{3.044778in}}%
\pgfpathlineto{\pgfqpoint{5.665681in}{3.050180in}}%
\pgfpathlineto{\pgfqpoint{5.658501in}{3.044591in}}%
\pgfpathlineto{\pgfqpoint{5.651313in}{3.038953in}}%
\pgfpathlineto{\pgfqpoint{5.644119in}{3.033264in}}%
\pgfpathlineto{\pgfqpoint{5.636917in}{3.027522in}}%
\pgfpathlineto{\pgfqpoint{5.622739in}{3.021941in}}%
\pgfpathlineto{\pgfqpoint{5.608575in}{3.016459in}}%
\pgfpathlineto{\pgfqpoint{5.594425in}{3.011074in}}%
\pgfpathlineto{\pgfqpoint{5.580289in}{3.005788in}}%
\pgfpathlineto{\pgfqpoint{5.587507in}{3.011702in}}%
\pgfpathlineto{\pgfqpoint{5.594717in}{3.017567in}}%
\pgfpathlineto{\pgfqpoint{5.601921in}{3.023385in}}%
\pgfpathlineto{\pgfqpoint{5.609118in}{3.029160in}}%
\pgfpathclose%
\pgfusepath{fill}%
\end{pgfscope}%
\begin{pgfscope}%
\pgfpathrectangle{\pgfqpoint{1.150000in}{0.150000in}}{\pgfqpoint{5.700000in}{5.700000in}}%
\pgfusepath{clip}%
\pgfsetbuttcap%
\pgfsetroundjoin%
\definecolor{currentfill}{rgb}{0.260571,0.246922,0.522828}%
\pgfsetfillcolor{currentfill}%
\pgfsetfillopacity{0.700000}%
\pgfsetlinewidth{0.000000pt}%
\definecolor{currentstroke}{rgb}{0.000000,0.000000,0.000000}%
\pgfsetstrokecolor{currentstroke}%
\pgfsetdash{}{0pt}%
\pgfpathmoveto{\pgfqpoint{5.012190in}{2.718001in}}%
\pgfpathlineto{\pgfqpoint{5.026048in}{2.721172in}}%
\pgfpathlineto{\pgfqpoint{5.039917in}{2.724444in}}%
\pgfpathlineto{\pgfqpoint{5.053798in}{2.727816in}}%
\pgfpathlineto{\pgfqpoint{5.067691in}{2.731290in}}%
\pgfpathlineto{\pgfqpoint{5.060227in}{2.723292in}}%
\pgfpathlineto{\pgfqpoint{5.052757in}{2.715234in}}%
\pgfpathlineto{\pgfqpoint{5.045280in}{2.707117in}}%
\pgfpathlineto{\pgfqpoint{5.037797in}{2.698940in}}%
\pgfpathlineto{\pgfqpoint{5.023895in}{2.695421in}}%
\pgfpathlineto{\pgfqpoint{5.010003in}{2.692002in}}%
\pgfpathlineto{\pgfqpoint{4.996124in}{2.688685in}}%
\pgfpathlineto{\pgfqpoint{4.982256in}{2.685469in}}%
\pgfpathlineto{\pgfqpoint{4.989749in}{2.693685in}}%
\pgfpathlineto{\pgfqpoint{4.997236in}{2.701845in}}%
\pgfpathlineto{\pgfqpoint{5.004716in}{2.709950in}}%
\pgfpathlineto{\pgfqpoint{5.012190in}{2.718001in}}%
\pgfpathclose%
\pgfusepath{fill}%
\end{pgfscope}%
\begin{pgfscope}%
\pgfpathrectangle{\pgfqpoint{1.150000in}{0.150000in}}{\pgfqpoint{5.700000in}{5.700000in}}%
\pgfusepath{clip}%
\pgfsetbuttcap%
\pgfsetroundjoin%
\definecolor{currentfill}{rgb}{0.266580,0.228262,0.514349}%
\pgfsetfillcolor{currentfill}%
\pgfsetfillopacity{0.700000}%
\pgfsetlinewidth{0.000000pt}%
\definecolor{currentstroke}{rgb}{0.000000,0.000000,0.000000}%
\pgfsetstrokecolor{currentstroke}%
\pgfsetdash{}{0pt}%
\pgfpathmoveto{\pgfqpoint{4.926900in}{2.673615in}}%
\pgfpathlineto{\pgfqpoint{4.940722in}{2.676426in}}%
\pgfpathlineto{\pgfqpoint{4.954555in}{2.679339in}}%
\pgfpathlineto{\pgfqpoint{4.968400in}{2.682353in}}%
\pgfpathlineto{\pgfqpoint{4.982256in}{2.685469in}}%
\pgfpathlineto{\pgfqpoint{4.974757in}{2.677196in}}%
\pgfpathlineto{\pgfqpoint{4.967252in}{2.668865in}}%
\pgfpathlineto{\pgfqpoint{4.959741in}{2.660477in}}%
\pgfpathlineto{\pgfqpoint{4.952223in}{2.652030in}}%
\pgfpathlineto{\pgfqpoint{4.938358in}{2.648888in}}%
\pgfpathlineto{\pgfqpoint{4.924503in}{2.645847in}}%
\pgfpathlineto{\pgfqpoint{4.910660in}{2.642907in}}%
\pgfpathlineto{\pgfqpoint{4.896828in}{2.640069in}}%
\pgfpathlineto{\pgfqpoint{4.904355in}{2.648536in}}%
\pgfpathlineto{\pgfqpoint{4.911876in}{2.656949in}}%
\pgfpathlineto{\pgfqpoint{4.919391in}{2.665308in}}%
\pgfpathlineto{\pgfqpoint{4.926900in}{2.673615in}}%
\pgfpathclose%
\pgfusepath{fill}%
\end{pgfscope}%
\begin{pgfscope}%
\pgfpathrectangle{\pgfqpoint{1.150000in}{0.150000in}}{\pgfqpoint{5.700000in}{5.700000in}}%
\pgfusepath{clip}%
\pgfsetbuttcap%
\pgfsetroundjoin%
\definecolor{currentfill}{rgb}{0.252194,0.269783,0.531579}%
\pgfsetfillcolor{currentfill}%
\pgfsetfillopacity{0.700000}%
\pgfsetlinewidth{0.000000pt}%
\definecolor{currentstroke}{rgb}{0.000000,0.000000,0.000000}%
\pgfsetstrokecolor{currentstroke}%
\pgfsetdash{}{0pt}%
\pgfpathmoveto{\pgfqpoint{5.097483in}{2.762711in}}%
\pgfpathlineto{\pgfqpoint{5.111377in}{2.766220in}}%
\pgfpathlineto{\pgfqpoint{5.125282in}{2.769830in}}%
\pgfpathlineto{\pgfqpoint{5.139200in}{2.773541in}}%
\pgfpathlineto{\pgfqpoint{5.153130in}{2.777351in}}%
\pgfpathlineto{\pgfqpoint{5.145703in}{2.769651in}}%
\pgfpathlineto{\pgfqpoint{5.138269in}{2.761891in}}%
\pgfpathlineto{\pgfqpoint{5.130829in}{2.754071in}}%
\pgfpathlineto{\pgfqpoint{5.123382in}{2.746188in}}%
\pgfpathlineto{\pgfqpoint{5.109441in}{2.742313in}}%
\pgfpathlineto{\pgfqpoint{5.095512in}{2.738538in}}%
\pgfpathlineto{\pgfqpoint{5.081596in}{2.734864in}}%
\pgfpathlineto{\pgfqpoint{5.067691in}{2.731290in}}%
\pgfpathlineto{\pgfqpoint{5.075148in}{2.739230in}}%
\pgfpathlineto{\pgfqpoint{5.082599in}{2.747113in}}%
\pgfpathlineto{\pgfqpoint{5.090044in}{2.754939in}}%
\pgfpathlineto{\pgfqpoint{5.097483in}{2.762711in}}%
\pgfpathclose%
\pgfusepath{fill}%
\end{pgfscope}%
\begin{pgfscope}%
\pgfpathrectangle{\pgfqpoint{1.150000in}{0.150000in}}{\pgfqpoint{5.700000in}{5.700000in}}%
\pgfusepath{clip}%
\pgfsetbuttcap%
\pgfsetroundjoin%
\definecolor{currentfill}{rgb}{0.208623,0.367752,0.552675}%
\pgfsetfillcolor{currentfill}%
\pgfsetfillopacity{0.700000}%
\pgfsetlinewidth{0.000000pt}%
\definecolor{currentstroke}{rgb}{0.000000,0.000000,0.000000}%
\pgfsetstrokecolor{currentstroke}%
\pgfsetdash{}{0pt}%
\pgfpathmoveto{\pgfqpoint{5.523880in}{2.985626in}}%
\pgfpathlineto{\pgfqpoint{5.537962in}{2.990519in}}%
\pgfpathlineto{\pgfqpoint{5.552057in}{2.995511in}}%
\pgfpathlineto{\pgfqpoint{5.566166in}{3.000600in}}%
\pgfpathlineto{\pgfqpoint{5.580289in}{3.005788in}}%
\pgfpathlineto{\pgfqpoint{5.573064in}{2.999824in}}%
\pgfpathlineto{\pgfqpoint{5.565832in}{2.993808in}}%
\pgfpathlineto{\pgfqpoint{5.558593in}{2.987736in}}%
\pgfpathlineto{\pgfqpoint{5.551347in}{2.981609in}}%
\pgfpathlineto{\pgfqpoint{5.537209in}{2.976261in}}%
\pgfpathlineto{\pgfqpoint{5.523084in}{2.971012in}}%
\pgfpathlineto{\pgfqpoint{5.508974in}{2.965862in}}%
\pgfpathlineto{\pgfqpoint{5.494877in}{2.960810in}}%
\pgfpathlineto{\pgfqpoint{5.502138in}{2.967090in}}%
\pgfpathlineto{\pgfqpoint{5.509392in}{2.973318in}}%
\pgfpathlineto{\pgfqpoint{5.516639in}{2.979496in}}%
\pgfpathlineto{\pgfqpoint{5.523880in}{2.985626in}}%
\pgfpathclose%
\pgfusepath{fill}%
\end{pgfscope}%
\begin{pgfscope}%
\pgfpathrectangle{\pgfqpoint{1.150000in}{0.150000in}}{\pgfqpoint{5.700000in}{5.700000in}}%
\pgfusepath{clip}%
\pgfsetbuttcap%
\pgfsetroundjoin%
\definecolor{currentfill}{rgb}{0.243113,0.292092,0.538516}%
\pgfsetfillcolor{currentfill}%
\pgfsetfillopacity{0.700000}%
\pgfsetlinewidth{0.000000pt}%
\definecolor{currentstroke}{rgb}{0.000000,0.000000,0.000000}%
\pgfsetstrokecolor{currentstroke}%
\pgfsetdash{}{0pt}%
\pgfpathmoveto{\pgfqpoint{5.182775in}{2.807575in}}%
\pgfpathlineto{\pgfqpoint{5.196706in}{2.811402in}}%
\pgfpathlineto{\pgfqpoint{5.210649in}{2.815330in}}%
\pgfpathlineto{\pgfqpoint{5.224605in}{2.819357in}}%
\pgfpathlineto{\pgfqpoint{5.238572in}{2.823484in}}%
\pgfpathlineto{\pgfqpoint{5.231183in}{2.816103in}}%
\pgfpathlineto{\pgfqpoint{5.223786in}{2.808661in}}%
\pgfpathlineto{\pgfqpoint{5.216383in}{2.801159in}}%
\pgfpathlineto{\pgfqpoint{5.208974in}{2.793594in}}%
\pgfpathlineto{\pgfqpoint{5.194994in}{2.789383in}}%
\pgfpathlineto{\pgfqpoint{5.181027in}{2.785273in}}%
\pgfpathlineto{\pgfqpoint{5.167073in}{2.781262in}}%
\pgfpathlineto{\pgfqpoint{5.153130in}{2.777351in}}%
\pgfpathlineto{\pgfqpoint{5.160551in}{2.784992in}}%
\pgfpathlineto{\pgfqpoint{5.167966in}{2.792576in}}%
\pgfpathlineto{\pgfqpoint{5.175374in}{2.800103in}}%
\pgfpathlineto{\pgfqpoint{5.182775in}{2.807575in}}%
\pgfpathclose%
\pgfusepath{fill}%
\end{pgfscope}%
\begin{pgfscope}%
\pgfpathrectangle{\pgfqpoint{1.150000in}{0.150000in}}{\pgfqpoint{5.700000in}{5.700000in}}%
\pgfusepath{clip}%
\pgfsetbuttcap%
\pgfsetroundjoin%
\definecolor{currentfill}{rgb}{0.269944,0.014625,0.341379}%
\pgfsetfillcolor{currentfill}%
\pgfsetfillopacity{0.700000}%
\pgfsetlinewidth{0.000000pt}%
\definecolor{currentstroke}{rgb}{0.000000,0.000000,0.000000}%
\pgfsetstrokecolor{currentstroke}%
\pgfsetdash{}{0pt}%
\pgfpathmoveto{\pgfqpoint{3.655438in}{2.281388in}}%
\pgfpathlineto{\pgfqpoint{3.668883in}{2.276091in}}%
\pgfpathlineto{\pgfqpoint{3.682333in}{2.270917in}}%
\pgfpathlineto{\pgfqpoint{3.695787in}{2.265865in}}%
\pgfpathlineto{\pgfqpoint{3.709246in}{2.260934in}}%
\pgfpathlineto{\pgfqpoint{3.701293in}{2.253164in}}%
\pgfpathlineto{\pgfqpoint{3.693333in}{2.245440in}}%
\pgfpathlineto{\pgfqpoint{3.685367in}{2.237763in}}%
\pgfpathlineto{\pgfqpoint{3.677395in}{2.230137in}}%
\pgfpathlineto{\pgfqpoint{3.663922in}{2.235277in}}%
\pgfpathlineto{\pgfqpoint{3.650452in}{2.240539in}}%
\pgfpathlineto{\pgfqpoint{3.636987in}{2.245924in}}%
\pgfpathlineto{\pgfqpoint{3.623527in}{2.251431in}}%
\pgfpathlineto{\pgfqpoint{3.631514in}{2.258841in}}%
\pgfpathlineto{\pgfqpoint{3.639495in}{2.266305in}}%
\pgfpathlineto{\pgfqpoint{3.647470in}{2.273821in}}%
\pgfpathlineto{\pgfqpoint{3.655438in}{2.281388in}}%
\pgfpathclose%
\pgfusepath{fill}%
\end{pgfscope}%
\begin{pgfscope}%
\pgfpathrectangle{\pgfqpoint{1.150000in}{0.150000in}}{\pgfqpoint{5.700000in}{5.700000in}}%
\pgfusepath{clip}%
\pgfsetbuttcap%
\pgfsetroundjoin%
\definecolor{currentfill}{rgb}{0.273006,0.204520,0.501721}%
\pgfsetfillcolor{currentfill}%
\pgfsetfillopacity{0.700000}%
\pgfsetlinewidth{0.000000pt}%
\definecolor{currentstroke}{rgb}{0.000000,0.000000,0.000000}%
\pgfsetstrokecolor{currentstroke}%
\pgfsetdash{}{0pt}%
\pgfpathmoveto{\pgfqpoint{4.841612in}{2.629734in}}%
\pgfpathlineto{\pgfqpoint{4.855400in}{2.632165in}}%
\pgfpathlineto{\pgfqpoint{4.869198in}{2.634698in}}%
\pgfpathlineto{\pgfqpoint{4.883008in}{2.637332in}}%
\pgfpathlineto{\pgfqpoint{4.896828in}{2.640069in}}%
\pgfpathlineto{\pgfqpoint{4.889295in}{2.631547in}}%
\pgfpathlineto{\pgfqpoint{4.881756in}{2.622971in}}%
\pgfpathlineto{\pgfqpoint{4.874212in}{2.614339in}}%
\pgfpathlineto{\pgfqpoint{4.866661in}{2.605652in}}%
\pgfpathlineto{\pgfqpoint{4.852831in}{2.602907in}}%
\pgfpathlineto{\pgfqpoint{4.839012in}{2.600264in}}%
\pgfpathlineto{\pgfqpoint{4.825205in}{2.597723in}}%
\pgfpathlineto{\pgfqpoint{4.811408in}{2.595284in}}%
\pgfpathlineto{\pgfqpoint{4.818968in}{2.603973in}}%
\pgfpathlineto{\pgfqpoint{4.826522in}{2.612611in}}%
\pgfpathlineto{\pgfqpoint{4.834070in}{2.621198in}}%
\pgfpathlineto{\pgfqpoint{4.841612in}{2.629734in}}%
\pgfpathclose%
\pgfusepath{fill}%
\end{pgfscope}%
\begin{pgfscope}%
\pgfpathrectangle{\pgfqpoint{1.150000in}{0.150000in}}{\pgfqpoint{5.700000in}{5.700000in}}%
\pgfusepath{clip}%
\pgfsetbuttcap%
\pgfsetroundjoin%
\definecolor{currentfill}{rgb}{0.235526,0.309527,0.542944}%
\pgfsetfillcolor{currentfill}%
\pgfsetfillopacity{0.700000}%
\pgfsetlinewidth{0.000000pt}%
\definecolor{currentstroke}{rgb}{0.000000,0.000000,0.000000}%
\pgfsetstrokecolor{currentstroke}%
\pgfsetdash{}{0pt}%
\pgfpathmoveto{\pgfqpoint{5.268065in}{2.852436in}}%
\pgfpathlineto{\pgfqpoint{5.282033in}{2.856561in}}%
\pgfpathlineto{\pgfqpoint{5.296014in}{2.860786in}}%
\pgfpathlineto{\pgfqpoint{5.310007in}{2.865109in}}%
\pgfpathlineto{\pgfqpoint{5.324014in}{2.869532in}}%
\pgfpathlineto{\pgfqpoint{5.316663in}{2.862487in}}%
\pgfpathlineto{\pgfqpoint{5.309306in}{2.855382in}}%
\pgfpathlineto{\pgfqpoint{5.301941in}{2.848216in}}%
\pgfpathlineto{\pgfqpoint{5.294570in}{2.840989in}}%
\pgfpathlineto{\pgfqpoint{5.280552in}{2.836464in}}%
\pgfpathlineto{\pgfqpoint{5.266546in}{2.832038in}}%
\pgfpathlineto{\pgfqpoint{5.252553in}{2.827711in}}%
\pgfpathlineto{\pgfqpoint{5.238572in}{2.823484in}}%
\pgfpathlineto{\pgfqpoint{5.245955in}{2.830807in}}%
\pgfpathlineto{\pgfqpoint{5.253332in}{2.838072in}}%
\pgfpathlineto{\pgfqpoint{5.260702in}{2.845282in}}%
\pgfpathlineto{\pgfqpoint{5.268065in}{2.852436in}}%
\pgfpathclose%
\pgfusepath{fill}%
\end{pgfscope}%
\begin{pgfscope}%
\pgfpathrectangle{\pgfqpoint{1.150000in}{0.150000in}}{\pgfqpoint{5.700000in}{5.700000in}}%
\pgfusepath{clip}%
\pgfsetbuttcap%
\pgfsetroundjoin%
\definecolor{currentfill}{rgb}{0.218130,0.347432,0.550038}%
\pgfsetfillcolor{currentfill}%
\pgfsetfillopacity{0.700000}%
\pgfsetlinewidth{0.000000pt}%
\definecolor{currentstroke}{rgb}{0.000000,0.000000,0.000000}%
\pgfsetstrokecolor{currentstroke}%
\pgfsetdash{}{0pt}%
\pgfpathmoveto{\pgfqpoint{5.438621in}{2.941588in}}%
\pgfpathlineto{\pgfqpoint{5.452665in}{2.946245in}}%
\pgfpathlineto{\pgfqpoint{5.466722in}{2.951002in}}%
\pgfpathlineto{\pgfqpoint{5.480793in}{2.955856in}}%
\pgfpathlineto{\pgfqpoint{5.494877in}{2.960810in}}%
\pgfpathlineto{\pgfqpoint{5.487608in}{2.954476in}}%
\pgfpathlineto{\pgfqpoint{5.480333in}{2.948087in}}%
\pgfpathlineto{\pgfqpoint{5.473051in}{2.941640in}}%
\pgfpathlineto{\pgfqpoint{5.465762in}{2.935134in}}%
\pgfpathlineto{\pgfqpoint{5.451664in}{2.930040in}}%
\pgfpathlineto{\pgfqpoint{5.437579in}{2.925045in}}%
\pgfpathlineto{\pgfqpoint{5.423508in}{2.920149in}}%
\pgfpathlineto{\pgfqpoint{5.409450in}{2.915351in}}%
\pgfpathlineto{\pgfqpoint{5.416753in}{2.921991in}}%
\pgfpathlineto{\pgfqpoint{5.424049in}{2.928575in}}%
\pgfpathlineto{\pgfqpoint{5.431339in}{2.935107in}}%
\pgfpathlineto{\pgfqpoint{5.438621in}{2.941588in}}%
\pgfpathclose%
\pgfusepath{fill}%
\end{pgfscope}%
\begin{pgfscope}%
\pgfpathrectangle{\pgfqpoint{1.150000in}{0.150000in}}{\pgfqpoint{5.700000in}{5.700000in}}%
\pgfusepath{clip}%
\pgfsetbuttcap%
\pgfsetroundjoin%
\definecolor{currentfill}{rgb}{0.225863,0.330805,0.547314}%
\pgfsetfillcolor{currentfill}%
\pgfsetfillopacity{0.700000}%
\pgfsetlinewidth{0.000000pt}%
\definecolor{currentstroke}{rgb}{0.000000,0.000000,0.000000}%
\pgfsetstrokecolor{currentstroke}%
\pgfsetdash{}{0pt}%
\pgfpathmoveto{\pgfqpoint{5.353348in}{2.897151in}}%
\pgfpathlineto{\pgfqpoint{5.367354in}{2.901553in}}%
\pgfpathlineto{\pgfqpoint{5.381373in}{2.906053in}}%
\pgfpathlineto{\pgfqpoint{5.395405in}{2.910653in}}%
\pgfpathlineto{\pgfqpoint{5.409450in}{2.915351in}}%
\pgfpathlineto{\pgfqpoint{5.402140in}{2.908656in}}%
\pgfpathlineto{\pgfqpoint{5.394823in}{2.901903in}}%
\pgfpathlineto{\pgfqpoint{5.387499in}{2.895090in}}%
\pgfpathlineto{\pgfqpoint{5.380168in}{2.888217in}}%
\pgfpathlineto{\pgfqpoint{5.366110in}{2.883397in}}%
\pgfpathlineto{\pgfqpoint{5.352065in}{2.878676in}}%
\pgfpathlineto{\pgfqpoint{5.338033in}{2.874055in}}%
\pgfpathlineto{\pgfqpoint{5.324014in}{2.869532in}}%
\pgfpathlineto{\pgfqpoint{5.331357in}{2.876520in}}%
\pgfpathlineto{\pgfqpoint{5.338694in}{2.883451in}}%
\pgfpathlineto{\pgfqpoint{5.346025in}{2.890328in}}%
\pgfpathlineto{\pgfqpoint{5.353348in}{2.897151in}}%
\pgfpathclose%
\pgfusepath{fill}%
\end{pgfscope}%
\begin{pgfscope}%
\pgfpathrectangle{\pgfqpoint{1.150000in}{0.150000in}}{\pgfqpoint{5.700000in}{5.700000in}}%
\pgfusepath{clip}%
\pgfsetbuttcap%
\pgfsetroundjoin%
\definecolor{currentfill}{rgb}{0.272594,0.025563,0.353093}%
\pgfsetfillcolor{currentfill}%
\pgfsetfillopacity{0.700000}%
\pgfsetlinewidth{0.000000pt}%
\definecolor{currentstroke}{rgb}{0.000000,0.000000,0.000000}%
\pgfsetstrokecolor{currentstroke}%
\pgfsetdash{}{0pt}%
\pgfpathmoveto{\pgfqpoint{3.515978in}{2.299987in}}%
\pgfpathlineto{\pgfqpoint{3.529409in}{2.293474in}}%
\pgfpathlineto{\pgfqpoint{3.542844in}{2.287089in}}%
\pgfpathlineto{\pgfqpoint{3.556282in}{2.280832in}}%
\pgfpathlineto{\pgfqpoint{3.569723in}{2.274701in}}%
\pgfpathlineto{\pgfqpoint{3.561713in}{2.267569in}}%
\pgfpathlineto{\pgfqpoint{3.553696in}{2.260500in}}%
\pgfpathlineto{\pgfqpoint{3.545672in}{2.253496in}}%
\pgfpathlineto{\pgfqpoint{3.537641in}{2.246558in}}%
\pgfpathlineto{\pgfqpoint{3.524183in}{2.252918in}}%
\pgfpathlineto{\pgfqpoint{3.510728in}{2.259404in}}%
\pgfpathlineto{\pgfqpoint{3.497276in}{2.266017in}}%
\pgfpathlineto{\pgfqpoint{3.483827in}{2.272759in}}%
\pgfpathlineto{\pgfqpoint{3.491875in}{2.279461in}}%
\pgfpathlineto{\pgfqpoint{3.499917in}{2.286234in}}%
\pgfpathlineto{\pgfqpoint{3.507951in}{2.293077in}}%
\pgfpathlineto{\pgfqpoint{3.515978in}{2.299987in}}%
\pgfpathclose%
\pgfusepath{fill}%
\end{pgfscope}%
\begin{pgfscope}%
\pgfpathrectangle{\pgfqpoint{1.150000in}{0.150000in}}{\pgfqpoint{5.700000in}{5.700000in}}%
\pgfusepath{clip}%
\pgfsetbuttcap%
\pgfsetroundjoin%
\definecolor{currentfill}{rgb}{0.277134,0.185228,0.489898}%
\pgfsetfillcolor{currentfill}%
\pgfsetfillopacity{0.700000}%
\pgfsetlinewidth{0.000000pt}%
\definecolor{currentstroke}{rgb}{0.000000,0.000000,0.000000}%
\pgfsetstrokecolor{currentstroke}%
\pgfsetdash{}{0pt}%
\pgfpathmoveto{\pgfqpoint{4.756328in}{2.586553in}}%
\pgfpathlineto{\pgfqpoint{4.770082in}{2.588581in}}%
\pgfpathlineto{\pgfqpoint{4.783846in}{2.590713in}}%
\pgfpathlineto{\pgfqpoint{4.797622in}{2.592947in}}%
\pgfpathlineto{\pgfqpoint{4.811408in}{2.595284in}}%
\pgfpathlineto{\pgfqpoint{4.803842in}{2.586543in}}%
\pgfpathlineto{\pgfqpoint{4.796270in}{2.577751in}}%
\pgfpathlineto{\pgfqpoint{4.788693in}{2.568906in}}%
\pgfpathlineto{\pgfqpoint{4.781109in}{2.560010in}}%
\pgfpathlineto{\pgfqpoint{4.767315in}{2.557683in}}%
\pgfpathlineto{\pgfqpoint{4.753530in}{2.555459in}}%
\pgfpathlineto{\pgfqpoint{4.739757in}{2.553338in}}%
\pgfpathlineto{\pgfqpoint{4.725994in}{2.551319in}}%
\pgfpathlineto{\pgfqpoint{4.733586in}{2.560199in}}%
\pgfpathlineto{\pgfqpoint{4.741172in}{2.569031in}}%
\pgfpathlineto{\pgfqpoint{4.748753in}{2.577815in}}%
\pgfpathlineto{\pgfqpoint{4.756328in}{2.586553in}}%
\pgfpathclose%
\pgfusepath{fill}%
\end{pgfscope}%
\begin{pgfscope}%
\pgfpathrectangle{\pgfqpoint{1.150000in}{0.150000in}}{\pgfqpoint{5.700000in}{5.700000in}}%
\pgfusepath{clip}%
\pgfsetbuttcap%
\pgfsetroundjoin%
\definecolor{currentfill}{rgb}{0.239346,0.300855,0.540844}%
\pgfsetfillcolor{currentfill}%
\pgfsetfillopacity{0.700000}%
\pgfsetlinewidth{0.000000pt}%
\definecolor{currentstroke}{rgb}{0.000000,0.000000,0.000000}%
\pgfsetstrokecolor{currentstroke}%
\pgfsetdash{}{0pt}%
\pgfpathmoveto{\pgfqpoint{2.664496in}{2.861831in}}%
\pgfpathlineto{\pgfqpoint{2.678037in}{2.845767in}}%
\pgfpathlineto{\pgfqpoint{2.691572in}{2.829899in}}%
\pgfpathlineto{\pgfqpoint{2.705102in}{2.814224in}}%
\pgfpathlineto{\pgfqpoint{2.718627in}{2.798742in}}%
\pgfpathlineto{\pgfqpoint{2.710178in}{2.796465in}}%
\pgfpathlineto{\pgfqpoint{2.701717in}{2.794346in}}%
\pgfpathlineto{\pgfqpoint{2.693244in}{2.792386in}}%
\pgfpathlineto{\pgfqpoint{2.684759in}{2.790588in}}%
\pgfpathlineto{\pgfqpoint{2.671201in}{2.806370in}}%
\pgfpathlineto{\pgfqpoint{2.657638in}{2.822344in}}%
\pgfpathlineto{\pgfqpoint{2.644069in}{2.838513in}}%
\pgfpathlineto{\pgfqpoint{2.630495in}{2.854878in}}%
\pgfpathlineto{\pgfqpoint{2.639014in}{2.856368in}}%
\pgfpathlineto{\pgfqpoint{2.647521in}{2.858025in}}%
\pgfpathlineto{\pgfqpoint{2.656015in}{2.859847in}}%
\pgfpathlineto{\pgfqpoint{2.664496in}{2.861831in}}%
\pgfpathclose%
\pgfusepath{fill}%
\end{pgfscope}%
\begin{pgfscope}%
\pgfpathrectangle{\pgfqpoint{1.150000in}{0.150000in}}{\pgfqpoint{5.700000in}{5.700000in}}%
\pgfusepath{clip}%
\pgfsetbuttcap%
\pgfsetroundjoin%
\definecolor{currentfill}{rgb}{0.250425,0.274290,0.533103}%
\pgfsetfillcolor{currentfill}%
\pgfsetfillopacity{0.700000}%
\pgfsetlinewidth{0.000000pt}%
\definecolor{currentstroke}{rgb}{0.000000,0.000000,0.000000}%
\pgfsetstrokecolor{currentstroke}%
\pgfsetdash{}{0pt}%
\pgfpathmoveto{\pgfqpoint{2.718627in}{2.798742in}}%
\pgfpathlineto{\pgfqpoint{2.732148in}{2.783450in}}%
\pgfpathlineto{\pgfqpoint{2.745664in}{2.768346in}}%
\pgfpathlineto{\pgfqpoint{2.759175in}{2.753429in}}%
\pgfpathlineto{\pgfqpoint{2.772683in}{2.738697in}}%
\pgfpathlineto{\pgfqpoint{2.764265in}{2.736129in}}%
\pgfpathlineto{\pgfqpoint{2.755836in}{2.733714in}}%
\pgfpathlineto{\pgfqpoint{2.747395in}{2.731453in}}%
\pgfpathlineto{\pgfqpoint{2.738942in}{2.729350in}}%
\pgfpathlineto{\pgfqpoint{2.725403in}{2.744380in}}%
\pgfpathlineto{\pgfqpoint{2.711860in}{2.759595in}}%
\pgfpathlineto{\pgfqpoint{2.698312in}{2.774997in}}%
\pgfpathlineto{\pgfqpoint{2.684759in}{2.790588in}}%
\pgfpathlineto{\pgfqpoint{2.693244in}{2.792386in}}%
\pgfpathlineto{\pgfqpoint{2.701717in}{2.794346in}}%
\pgfpathlineto{\pgfqpoint{2.710178in}{2.796465in}}%
\pgfpathlineto{\pgfqpoint{2.718627in}{2.798742in}}%
\pgfpathclose%
\pgfusepath{fill}%
\end{pgfscope}%
\begin{pgfscope}%
\pgfpathrectangle{\pgfqpoint{1.150000in}{0.150000in}}{\pgfqpoint{5.700000in}{5.700000in}}%
\pgfusepath{clip}%
\pgfsetbuttcap%
\pgfsetroundjoin%
\definecolor{currentfill}{rgb}{0.280255,0.165693,0.476498}%
\pgfsetfillcolor{currentfill}%
\pgfsetfillopacity{0.700000}%
\pgfsetlinewidth{0.000000pt}%
\definecolor{currentstroke}{rgb}{0.000000,0.000000,0.000000}%
\pgfsetstrokecolor{currentstroke}%
\pgfsetdash{}{0pt}%
\pgfpathmoveto{\pgfqpoint{4.671045in}{2.544277in}}%
\pgfpathlineto{\pgfqpoint{4.684767in}{2.545882in}}%
\pgfpathlineto{\pgfqpoint{4.698499in}{2.547591in}}%
\pgfpathlineto{\pgfqpoint{4.712241in}{2.549404in}}%
\pgfpathlineto{\pgfqpoint{4.725994in}{2.551319in}}%
\pgfpathlineto{\pgfqpoint{4.718396in}{2.542392in}}%
\pgfpathlineto{\pgfqpoint{4.710793in}{2.533416in}}%
\pgfpathlineto{\pgfqpoint{4.703183in}{2.524393in}}%
\pgfpathlineto{\pgfqpoint{4.695568in}{2.515322in}}%
\pgfpathlineto{\pgfqpoint{4.681807in}{2.513435in}}%
\pgfpathlineto{\pgfqpoint{4.668057in}{2.511651in}}%
\pgfpathlineto{\pgfqpoint{4.654316in}{2.509970in}}%
\pgfpathlineto{\pgfqpoint{4.640585in}{2.508393in}}%
\pgfpathlineto{\pgfqpoint{4.648209in}{2.517429in}}%
\pgfpathlineto{\pgfqpoint{4.655827in}{2.526422in}}%
\pgfpathlineto{\pgfqpoint{4.663439in}{2.535371in}}%
\pgfpathlineto{\pgfqpoint{4.671045in}{2.544277in}}%
\pgfpathclose%
\pgfusepath{fill}%
\end{pgfscope}%
\begin{pgfscope}%
\pgfpathrectangle{\pgfqpoint{1.150000in}{0.150000in}}{\pgfqpoint{5.700000in}{5.700000in}}%
\pgfusepath{clip}%
\pgfsetbuttcap%
\pgfsetroundjoin%
\definecolor{currentfill}{rgb}{0.260571,0.246922,0.522828}%
\pgfsetfillcolor{currentfill}%
\pgfsetfillopacity{0.700000}%
\pgfsetlinewidth{0.000000pt}%
\definecolor{currentstroke}{rgb}{0.000000,0.000000,0.000000}%
\pgfsetstrokecolor{currentstroke}%
\pgfsetdash{}{0pt}%
\pgfpathmoveto{\pgfqpoint{2.772683in}{2.738697in}}%
\pgfpathlineto{\pgfqpoint{2.786186in}{2.724148in}}%
\pgfpathlineto{\pgfqpoint{2.799686in}{2.709781in}}%
\pgfpathlineto{\pgfqpoint{2.813182in}{2.695594in}}%
\pgfpathlineto{\pgfqpoint{2.826674in}{2.681586in}}%
\pgfpathlineto{\pgfqpoint{2.818287in}{2.678729in}}%
\pgfpathlineto{\pgfqpoint{2.809888in}{2.676020in}}%
\pgfpathlineto{\pgfqpoint{2.801478in}{2.673460in}}%
\pgfpathlineto{\pgfqpoint{2.793057in}{2.671053in}}%
\pgfpathlineto{\pgfqpoint{2.779534in}{2.685357in}}%
\pgfpathlineto{\pgfqpoint{2.766007in}{2.699840in}}%
\pgfpathlineto{\pgfqpoint{2.752477in}{2.714504in}}%
\pgfpathlineto{\pgfqpoint{2.738942in}{2.729350in}}%
\pgfpathlineto{\pgfqpoint{2.747395in}{2.731453in}}%
\pgfpathlineto{\pgfqpoint{2.755836in}{2.733714in}}%
\pgfpathlineto{\pgfqpoint{2.764265in}{2.736129in}}%
\pgfpathlineto{\pgfqpoint{2.772683in}{2.738697in}}%
\pgfpathclose%
\pgfusepath{fill}%
\end{pgfscope}%
\begin{pgfscope}%
\pgfpathrectangle{\pgfqpoint{1.150000in}{0.150000in}}{\pgfqpoint{5.700000in}{5.700000in}}%
\pgfusepath{clip}%
\pgfsetbuttcap%
\pgfsetroundjoin%
\definecolor{currentfill}{rgb}{0.282290,0.145912,0.461510}%
\pgfsetfillcolor{currentfill}%
\pgfsetfillopacity{0.700000}%
\pgfsetlinewidth{0.000000pt}%
\definecolor{currentstroke}{rgb}{0.000000,0.000000,0.000000}%
\pgfsetstrokecolor{currentstroke}%
\pgfsetdash{}{0pt}%
\pgfpathmoveto{\pgfqpoint{4.585763in}{2.503124in}}%
\pgfpathlineto{\pgfqpoint{4.599454in}{2.504285in}}%
\pgfpathlineto{\pgfqpoint{4.613154in}{2.505550in}}%
\pgfpathlineto{\pgfqpoint{4.626865in}{2.506920in}}%
\pgfpathlineto{\pgfqpoint{4.640585in}{2.508393in}}%
\pgfpathlineto{\pgfqpoint{4.632956in}{2.499314in}}%
\pgfpathlineto{\pgfqpoint{4.625322in}{2.490192in}}%
\pgfpathlineto{\pgfqpoint{4.617682in}{2.481026in}}%
\pgfpathlineto{\pgfqpoint{4.610036in}{2.471818in}}%
\pgfpathlineto{\pgfqpoint{4.596307in}{2.470392in}}%
\pgfpathlineto{\pgfqpoint{4.582588in}{2.469069in}}%
\pgfpathlineto{\pgfqpoint{4.568879in}{2.467850in}}%
\pgfpathlineto{\pgfqpoint{4.555179in}{2.466736in}}%
\pgfpathlineto{\pgfqpoint{4.562833in}{2.475891in}}%
\pgfpathlineto{\pgfqpoint{4.570482in}{2.485007in}}%
\pgfpathlineto{\pgfqpoint{4.578125in}{2.494085in}}%
\pgfpathlineto{\pgfqpoint{4.585763in}{2.503124in}}%
\pgfpathclose%
\pgfusepath{fill}%
\end{pgfscope}%
\begin{pgfscope}%
\pgfpathrectangle{\pgfqpoint{1.150000in}{0.150000in}}{\pgfqpoint{5.700000in}{5.700000in}}%
\pgfusepath{clip}%
\pgfsetbuttcap%
\pgfsetroundjoin%
\definecolor{currentfill}{rgb}{0.269944,0.014625,0.341379}%
\pgfsetfillcolor{currentfill}%
\pgfsetfillopacity{0.700000}%
\pgfsetlinewidth{0.000000pt}%
\definecolor{currentstroke}{rgb}{0.000000,0.000000,0.000000}%
\pgfsetstrokecolor{currentstroke}%
\pgfsetdash{}{0pt}%
\pgfpathmoveto{\pgfqpoint{3.794824in}{2.274688in}}%
\pgfpathlineto{\pgfqpoint{3.808294in}{2.270549in}}%
\pgfpathlineto{\pgfqpoint{3.821768in}{2.266528in}}%
\pgfpathlineto{\pgfqpoint{3.835248in}{2.262624in}}%
\pgfpathlineto{\pgfqpoint{3.848734in}{2.258838in}}%
\pgfpathlineto{\pgfqpoint{3.840832in}{2.250532in}}%
\pgfpathlineto{\pgfqpoint{3.832925in}{2.242255in}}%
\pgfpathlineto{\pgfqpoint{3.825011in}{2.234011in}}%
\pgfpathlineto{\pgfqpoint{3.817092in}{2.225800in}}%
\pgfpathlineto{\pgfqpoint{3.803593in}{2.229777in}}%
\pgfpathlineto{\pgfqpoint{3.790100in}{2.233872in}}%
\pgfpathlineto{\pgfqpoint{3.776611in}{2.238085in}}%
\pgfpathlineto{\pgfqpoint{3.763128in}{2.242416in}}%
\pgfpathlineto{\pgfqpoint{3.771061in}{2.250429in}}%
\pgfpathlineto{\pgfqpoint{3.778988in}{2.258480in}}%
\pgfpathlineto{\pgfqpoint{3.786909in}{2.266567in}}%
\pgfpathlineto{\pgfqpoint{3.794824in}{2.274688in}}%
\pgfpathclose%
\pgfusepath{fill}%
\end{pgfscope}%
\begin{pgfscope}%
\pgfpathrectangle{\pgfqpoint{1.150000in}{0.150000in}}{\pgfqpoint{5.700000in}{5.700000in}}%
\pgfusepath{clip}%
\pgfsetbuttcap%
\pgfsetroundjoin%
\definecolor{currentfill}{rgb}{0.283187,0.125848,0.444960}%
\pgfsetfillcolor{currentfill}%
\pgfsetfillopacity{0.700000}%
\pgfsetlinewidth{0.000000pt}%
\definecolor{currentstroke}{rgb}{0.000000,0.000000,0.000000}%
\pgfsetstrokecolor{currentstroke}%
\pgfsetdash{}{0pt}%
\pgfpathmoveto{\pgfqpoint{4.500476in}{2.463328in}}%
\pgfpathlineto{\pgfqpoint{4.514138in}{2.464022in}}%
\pgfpathlineto{\pgfqpoint{4.527809in}{2.464822in}}%
\pgfpathlineto{\pgfqpoint{4.541489in}{2.465727in}}%
\pgfpathlineto{\pgfqpoint{4.555179in}{2.466736in}}%
\pgfpathlineto{\pgfqpoint{4.547520in}{2.457544in}}%
\pgfpathlineto{\pgfqpoint{4.539854in}{2.448313in}}%
\pgfpathlineto{\pgfqpoint{4.532184in}{2.439045in}}%
\pgfpathlineto{\pgfqpoint{4.524508in}{2.429740in}}%
\pgfpathlineto{\pgfqpoint{4.510810in}{2.428796in}}%
\pgfpathlineto{\pgfqpoint{4.497121in}{2.427956in}}%
\pgfpathlineto{\pgfqpoint{4.483441in}{2.427221in}}%
\pgfpathlineto{\pgfqpoint{4.469771in}{2.426592in}}%
\pgfpathlineto{\pgfqpoint{4.477455in}{2.435825in}}%
\pgfpathlineto{\pgfqpoint{4.485135in}{2.445026in}}%
\pgfpathlineto{\pgfqpoint{4.492808in}{2.454193in}}%
\pgfpathlineto{\pgfqpoint{4.500476in}{2.463328in}}%
\pgfpathclose%
\pgfusepath{fill}%
\end{pgfscope}%
\begin{pgfscope}%
\pgfpathrectangle{\pgfqpoint{1.150000in}{0.150000in}}{\pgfqpoint{5.700000in}{5.700000in}}%
\pgfusepath{clip}%
\pgfsetbuttcap%
\pgfsetroundjoin%
\definecolor{currentfill}{rgb}{0.273809,0.031497,0.358853}%
\pgfsetfillcolor{currentfill}%
\pgfsetfillopacity{0.700000}%
\pgfsetlinewidth{0.000000pt}%
\definecolor{currentstroke}{rgb}{0.000000,0.000000,0.000000}%
\pgfsetstrokecolor{currentstroke}%
\pgfsetdash{}{0pt}%
\pgfpathmoveto{\pgfqpoint{4.019646in}{2.302693in}}%
\pgfpathlineto{\pgfqpoint{4.033163in}{2.300270in}}%
\pgfpathlineto{\pgfqpoint{4.046687in}{2.297959in}}%
\pgfpathlineto{\pgfqpoint{4.060217in}{2.295761in}}%
\pgfpathlineto{\pgfqpoint{4.073754in}{2.293675in}}%
\pgfpathlineto{\pgfqpoint{4.065932in}{2.284737in}}%
\pgfpathlineto{\pgfqpoint{4.058105in}{2.275803in}}%
\pgfpathlineto{\pgfqpoint{4.050272in}{2.266876in}}%
\pgfpathlineto{\pgfqpoint{4.042434in}{2.257956in}}%
\pgfpathlineto{\pgfqpoint{4.028885in}{2.260197in}}%
\pgfpathlineto{\pgfqpoint{4.015344in}{2.262551in}}%
\pgfpathlineto{\pgfqpoint{4.001810in}{2.265016in}}%
\pgfpathlineto{\pgfqpoint{3.988282in}{2.267595in}}%
\pgfpathlineto{\pgfqpoint{3.996131in}{2.276352in}}%
\pgfpathlineto{\pgfqpoint{4.003975in}{2.285122in}}%
\pgfpathlineto{\pgfqpoint{4.011813in}{2.293903in}}%
\pgfpathlineto{\pgfqpoint{4.019646in}{2.302693in}}%
\pgfpathclose%
\pgfusepath{fill}%
\end{pgfscope}%
\begin{pgfscope}%
\pgfpathrectangle{\pgfqpoint{1.150000in}{0.150000in}}{\pgfqpoint{5.700000in}{5.700000in}}%
\pgfusepath{clip}%
\pgfsetbuttcap%
\pgfsetroundjoin%
\definecolor{currentfill}{rgb}{0.267968,0.223549,0.512008}%
\pgfsetfillcolor{currentfill}%
\pgfsetfillopacity{0.700000}%
\pgfsetlinewidth{0.000000pt}%
\definecolor{currentstroke}{rgb}{0.000000,0.000000,0.000000}%
\pgfsetstrokecolor{currentstroke}%
\pgfsetdash{}{0pt}%
\pgfpathmoveto{\pgfqpoint{2.826674in}{2.681586in}}%
\pgfpathlineto{\pgfqpoint{2.840163in}{2.667754in}}%
\pgfpathlineto{\pgfqpoint{2.853649in}{2.654098in}}%
\pgfpathlineto{\pgfqpoint{2.867132in}{2.640616in}}%
\pgfpathlineto{\pgfqpoint{2.880612in}{2.627306in}}%
\pgfpathlineto{\pgfqpoint{2.872254in}{2.624162in}}%
\pgfpathlineto{\pgfqpoint{2.863885in}{2.621160in}}%
\pgfpathlineto{\pgfqpoint{2.855505in}{2.618303in}}%
\pgfpathlineto{\pgfqpoint{2.847114in}{2.615594in}}%
\pgfpathlineto{\pgfqpoint{2.833604in}{2.629198in}}%
\pgfpathlineto{\pgfqpoint{2.820092in}{2.642975in}}%
\pgfpathlineto{\pgfqpoint{2.806576in}{2.656926in}}%
\pgfpathlineto{\pgfqpoint{2.793057in}{2.671053in}}%
\pgfpathlineto{\pgfqpoint{2.801478in}{2.673460in}}%
\pgfpathlineto{\pgfqpoint{2.809888in}{2.676020in}}%
\pgfpathlineto{\pgfqpoint{2.818287in}{2.678729in}}%
\pgfpathlineto{\pgfqpoint{2.826674in}{2.681586in}}%
\pgfpathclose%
\pgfusepath{fill}%
\end{pgfscope}%
\begin{pgfscope}%
\pgfpathrectangle{\pgfqpoint{1.150000in}{0.150000in}}{\pgfqpoint{5.700000in}{5.700000in}}%
\pgfusepath{clip}%
\pgfsetbuttcap%
\pgfsetroundjoin%
\definecolor{currentfill}{rgb}{0.277018,0.050344,0.375715}%
\pgfsetfillcolor{currentfill}%
\pgfsetfillopacity{0.700000}%
\pgfsetlinewidth{0.000000pt}%
\definecolor{currentstroke}{rgb}{0.000000,0.000000,0.000000}%
\pgfsetstrokecolor{currentstroke}%
\pgfsetdash{}{0pt}%
\pgfpathmoveto{\pgfqpoint{3.376338in}{2.331394in}}%
\pgfpathlineto{\pgfqpoint{3.389765in}{2.323601in}}%
\pgfpathlineto{\pgfqpoint{3.403195in}{2.315942in}}%
\pgfpathlineto{\pgfqpoint{3.416627in}{2.308416in}}%
\pgfpathlineto{\pgfqpoint{3.430062in}{2.301022in}}%
\pgfpathlineto{\pgfqpoint{3.421988in}{2.294635in}}%
\pgfpathlineto{\pgfqpoint{3.413906in}{2.288328in}}%
\pgfpathlineto{\pgfqpoint{3.405817in}{2.282103in}}%
\pgfpathlineto{\pgfqpoint{3.397721in}{2.275964in}}%
\pgfpathlineto{\pgfqpoint{3.384267in}{2.283605in}}%
\pgfpathlineto{\pgfqpoint{3.370815in}{2.291379in}}%
\pgfpathlineto{\pgfqpoint{3.357366in}{2.299286in}}%
\pgfpathlineto{\pgfqpoint{3.343919in}{2.307328in}}%
\pgfpathlineto{\pgfqpoint{3.352035in}{2.313212in}}%
\pgfpathlineto{\pgfqpoint{3.360144in}{2.319187in}}%
\pgfpathlineto{\pgfqpoint{3.368244in}{2.325248in}}%
\pgfpathlineto{\pgfqpoint{3.376338in}{2.331394in}}%
\pgfpathclose%
\pgfusepath{fill}%
\end{pgfscope}%
\begin{pgfscope}%
\pgfpathrectangle{\pgfqpoint{1.150000in}{0.150000in}}{\pgfqpoint{5.700000in}{5.700000in}}%
\pgfusepath{clip}%
\pgfsetbuttcap%
\pgfsetroundjoin%
\definecolor{currentfill}{rgb}{0.282327,0.094955,0.417331}%
\pgfsetfillcolor{currentfill}%
\pgfsetfillopacity{0.700000}%
\pgfsetlinewidth{0.000000pt}%
\definecolor{currentstroke}{rgb}{0.000000,0.000000,0.000000}%
\pgfsetstrokecolor{currentstroke}%
\pgfsetdash{}{0pt}%
\pgfpathmoveto{\pgfqpoint{3.182667in}{2.414621in}}%
\pgfpathlineto{\pgfqpoint{3.196099in}{2.404898in}}%
\pgfpathlineto{\pgfqpoint{3.209532in}{2.395321in}}%
\pgfpathlineto{\pgfqpoint{3.222965in}{2.385889in}}%
\pgfpathlineto{\pgfqpoint{3.236399in}{2.376599in}}%
\pgfpathlineto{\pgfqpoint{3.228232in}{2.371328in}}%
\pgfpathlineto{\pgfqpoint{3.220057in}{2.366160in}}%
\pgfpathlineto{\pgfqpoint{3.211872in}{2.361098in}}%
\pgfpathlineto{\pgfqpoint{3.203680in}{2.356145in}}%
\pgfpathlineto{\pgfqpoint{3.190223in}{2.365702in}}%
\pgfpathlineto{\pgfqpoint{3.176766in}{2.375403in}}%
\pgfpathlineto{\pgfqpoint{3.163310in}{2.385249in}}%
\pgfpathlineto{\pgfqpoint{3.149855in}{2.395240in}}%
\pgfpathlineto{\pgfqpoint{3.158071in}{2.399918in}}%
\pgfpathlineto{\pgfqpoint{3.166279in}{2.404709in}}%
\pgfpathlineto{\pgfqpoint{3.174477in}{2.409610in}}%
\pgfpathlineto{\pgfqpoint{3.182667in}{2.414621in}}%
\pgfpathclose%
\pgfusepath{fill}%
\end{pgfscope}%
\begin{pgfscope}%
\pgfpathrectangle{\pgfqpoint{1.150000in}{0.150000in}}{\pgfqpoint{5.700000in}{5.700000in}}%
\pgfusepath{clip}%
\pgfsetbuttcap%
\pgfsetroundjoin%
\definecolor{currentfill}{rgb}{0.282910,0.105393,0.426902}%
\pgfsetfillcolor{currentfill}%
\pgfsetfillopacity{0.700000}%
\pgfsetlinewidth{0.000000pt}%
\definecolor{currentstroke}{rgb}{0.000000,0.000000,0.000000}%
\pgfsetstrokecolor{currentstroke}%
\pgfsetdash{}{0pt}%
\pgfpathmoveto{\pgfqpoint{4.415181in}{2.425131in}}%
\pgfpathlineto{\pgfqpoint{4.428815in}{2.425337in}}%
\pgfpathlineto{\pgfqpoint{4.442458in}{2.425649in}}%
\pgfpathlineto{\pgfqpoint{4.456110in}{2.426068in}}%
\pgfpathlineto{\pgfqpoint{4.469771in}{2.426592in}}%
\pgfpathlineto{\pgfqpoint{4.462081in}{2.417326in}}%
\pgfpathlineto{\pgfqpoint{4.454386in}{2.408029in}}%
\pgfpathlineto{\pgfqpoint{4.446685in}{2.398701in}}%
\pgfpathlineto{\pgfqpoint{4.438979in}{2.389343in}}%
\pgfpathlineto{\pgfqpoint{4.425310in}{2.388902in}}%
\pgfpathlineto{\pgfqpoint{4.411649in}{2.388567in}}%
\pgfpathlineto{\pgfqpoint{4.397997in}{2.388338in}}%
\pgfpathlineto{\pgfqpoint{4.384355in}{2.388215in}}%
\pgfpathlineto{\pgfqpoint{4.392069in}{2.397483in}}%
\pgfpathlineto{\pgfqpoint{4.399779in}{2.406726in}}%
\pgfpathlineto{\pgfqpoint{4.407483in}{2.415942in}}%
\pgfpathlineto{\pgfqpoint{4.415181in}{2.425131in}}%
\pgfpathclose%
\pgfusepath{fill}%
\end{pgfscope}%
\begin{pgfscope}%
\pgfpathrectangle{\pgfqpoint{1.150000in}{0.150000in}}{\pgfqpoint{5.700000in}{5.700000in}}%
\pgfusepath{clip}%
\pgfsetbuttcap%
\pgfsetroundjoin%
\definecolor{currentfill}{rgb}{0.274128,0.199721,0.498911}%
\pgfsetfillcolor{currentfill}%
\pgfsetfillopacity{0.700000}%
\pgfsetlinewidth{0.000000pt}%
\definecolor{currentstroke}{rgb}{0.000000,0.000000,0.000000}%
\pgfsetstrokecolor{currentstroke}%
\pgfsetdash{}{0pt}%
\pgfpathmoveto{\pgfqpoint{2.880612in}{2.627306in}}%
\pgfpathlineto{\pgfqpoint{2.894089in}{2.614167in}}%
\pgfpathlineto{\pgfqpoint{2.907564in}{2.601198in}}%
\pgfpathlineto{\pgfqpoint{2.921036in}{2.588397in}}%
\pgfpathlineto{\pgfqpoint{2.934506in}{2.575762in}}%
\pgfpathlineto{\pgfqpoint{2.926177in}{2.572332in}}%
\pgfpathlineto{\pgfqpoint{2.917837in}{2.569039in}}%
\pgfpathlineto{\pgfqpoint{2.909486in}{2.565887in}}%
\pgfpathlineto{\pgfqpoint{2.901124in}{2.562877in}}%
\pgfpathlineto{\pgfqpoint{2.887626in}{2.575805in}}%
\pgfpathlineto{\pgfqpoint{2.874124in}{2.588899in}}%
\pgfpathlineto{\pgfqpoint{2.860620in}{2.602162in}}%
\pgfpathlineto{\pgfqpoint{2.847114in}{2.615594in}}%
\pgfpathlineto{\pgfqpoint{2.855505in}{2.618303in}}%
\pgfpathlineto{\pgfqpoint{2.863885in}{2.621160in}}%
\pgfpathlineto{\pgfqpoint{2.872254in}{2.624162in}}%
\pgfpathlineto{\pgfqpoint{2.880612in}{2.627306in}}%
\pgfpathclose%
\pgfusepath{fill}%
\end{pgfscope}%
\begin{pgfscope}%
\pgfpathrectangle{\pgfqpoint{1.150000in}{0.150000in}}{\pgfqpoint{5.700000in}{5.700000in}}%
\pgfusepath{clip}%
\pgfsetbuttcap%
\pgfsetroundjoin%
\definecolor{currentfill}{rgb}{0.281924,0.089666,0.412415}%
\pgfsetfillcolor{currentfill}%
\pgfsetfillopacity{0.700000}%
\pgfsetlinewidth{0.000000pt}%
\definecolor{currentstroke}{rgb}{0.000000,0.000000,0.000000}%
\pgfsetstrokecolor{currentstroke}%
\pgfsetdash{}{0pt}%
\pgfpathmoveto{\pgfqpoint{4.329870in}{2.388791in}}%
\pgfpathlineto{\pgfqpoint{4.343479in}{2.388486in}}%
\pgfpathlineto{\pgfqpoint{4.357095in}{2.388289in}}%
\pgfpathlineto{\pgfqpoint{4.370721in}{2.388198in}}%
\pgfpathlineto{\pgfqpoint{4.384355in}{2.388215in}}%
\pgfpathlineto{\pgfqpoint{4.376635in}{2.378921in}}%
\pgfpathlineto{\pgfqpoint{4.368910in}{2.369602in}}%
\pgfpathlineto{\pgfqpoint{4.361179in}{2.360259in}}%
\pgfpathlineto{\pgfqpoint{4.353443in}{2.350893in}}%
\pgfpathlineto{\pgfqpoint{4.339800in}{2.350978in}}%
\pgfpathlineto{\pgfqpoint{4.326166in}{2.351170in}}%
\pgfpathlineto{\pgfqpoint{4.312540in}{2.351468in}}%
\pgfpathlineto{\pgfqpoint{4.298923in}{2.351874in}}%
\pgfpathlineto{\pgfqpoint{4.306668in}{2.361132in}}%
\pgfpathlineto{\pgfqpoint{4.314407in}{2.370372in}}%
\pgfpathlineto{\pgfqpoint{4.322141in}{2.379591in}}%
\pgfpathlineto{\pgfqpoint{4.329870in}{2.388791in}}%
\pgfpathclose%
\pgfusepath{fill}%
\end{pgfscope}%
\begin{pgfscope}%
\pgfpathrectangle{\pgfqpoint{1.150000in}{0.150000in}}{\pgfqpoint{5.700000in}{5.700000in}}%
\pgfusepath{clip}%
\pgfsetbuttcap%
\pgfsetroundjoin%
\definecolor{currentfill}{rgb}{0.271305,0.019942,0.347269}%
\pgfsetfillcolor{currentfill}%
\pgfsetfillopacity{0.700000}%
\pgfsetlinewidth{0.000000pt}%
\definecolor{currentstroke}{rgb}{0.000000,0.000000,0.000000}%
\pgfsetstrokecolor{currentstroke}%
\pgfsetdash{}{0pt}%
\pgfpathmoveto{\pgfqpoint{3.934234in}{2.279044in}}%
\pgfpathlineto{\pgfqpoint{3.947736in}{2.276011in}}%
\pgfpathlineto{\pgfqpoint{3.961245in}{2.273092in}}%
\pgfpathlineto{\pgfqpoint{3.974760in}{2.270286in}}%
\pgfpathlineto{\pgfqpoint{3.988282in}{2.267595in}}%
\pgfpathlineto{\pgfqpoint{3.980427in}{2.258850in}}%
\pgfpathlineto{\pgfqpoint{3.972566in}{2.250121in}}%
\pgfpathlineto{\pgfqpoint{3.964700in}{2.241408in}}%
\pgfpathlineto{\pgfqpoint{3.956829in}{2.232713in}}%
\pgfpathlineto{\pgfqpoint{3.943296in}{2.235578in}}%
\pgfpathlineto{\pgfqpoint{3.929769in}{2.238557in}}%
\pgfpathlineto{\pgfqpoint{3.916248in}{2.241649in}}%
\pgfpathlineto{\pgfqpoint{3.902733in}{2.244856in}}%
\pgfpathlineto{\pgfqpoint{3.910617in}{2.253371in}}%
\pgfpathlineto{\pgfqpoint{3.918495in}{2.261909in}}%
\pgfpathlineto{\pgfqpoint{3.926367in}{2.270467in}}%
\pgfpathlineto{\pgfqpoint{3.934234in}{2.279044in}}%
\pgfpathclose%
\pgfusepath{fill}%
\end{pgfscope}%
\begin{pgfscope}%
\pgfpathrectangle{\pgfqpoint{1.150000in}{0.150000in}}{\pgfqpoint{5.700000in}{5.700000in}}%
\pgfusepath{clip}%
\pgfsetbuttcap%
\pgfsetroundjoin%
\definecolor{currentfill}{rgb}{0.278826,0.175490,0.483397}%
\pgfsetfillcolor{currentfill}%
\pgfsetfillopacity{0.700000}%
\pgfsetlinewidth{0.000000pt}%
\definecolor{currentstroke}{rgb}{0.000000,0.000000,0.000000}%
\pgfsetstrokecolor{currentstroke}%
\pgfsetdash{}{0pt}%
\pgfpathmoveto{\pgfqpoint{2.934506in}{2.575762in}}%
\pgfpathlineto{\pgfqpoint{2.947975in}{2.563293in}}%
\pgfpathlineto{\pgfqpoint{2.961441in}{2.550988in}}%
\pgfpathlineto{\pgfqpoint{2.974906in}{2.538846in}}%
\pgfpathlineto{\pgfqpoint{2.988369in}{2.526865in}}%
\pgfpathlineto{\pgfqpoint{2.980067in}{2.523150in}}%
\pgfpathlineto{\pgfqpoint{2.971754in}{2.519568in}}%
\pgfpathlineto{\pgfqpoint{2.963432in}{2.516121in}}%
\pgfpathlineto{\pgfqpoint{2.955098in}{2.512812in}}%
\pgfpathlineto{\pgfqpoint{2.941608in}{2.525085in}}%
\pgfpathlineto{\pgfqpoint{2.928115in}{2.537519in}}%
\pgfpathlineto{\pgfqpoint{2.914621in}{2.550116in}}%
\pgfpathlineto{\pgfqpoint{2.901124in}{2.562877in}}%
\pgfpathlineto{\pgfqpoint{2.909486in}{2.565887in}}%
\pgfpathlineto{\pgfqpoint{2.917837in}{2.569039in}}%
\pgfpathlineto{\pgfqpoint{2.926177in}{2.572332in}}%
\pgfpathlineto{\pgfqpoint{2.934506in}{2.575762in}}%
\pgfpathclose%
\pgfusepath{fill}%
\end{pgfscope}%
\begin{pgfscope}%
\pgfpathrectangle{\pgfqpoint{1.150000in}{0.150000in}}{\pgfqpoint{5.700000in}{5.700000in}}%
\pgfusepath{clip}%
\pgfsetbuttcap%
\pgfsetroundjoin%
\definecolor{currentfill}{rgb}{0.279566,0.067836,0.391917}%
\pgfsetfillcolor{currentfill}%
\pgfsetfillopacity{0.700000}%
\pgfsetlinewidth{0.000000pt}%
\definecolor{currentstroke}{rgb}{0.000000,0.000000,0.000000}%
\pgfsetstrokecolor{currentstroke}%
\pgfsetdash{}{0pt}%
\pgfpathmoveto{\pgfqpoint{4.244536in}{2.354577in}}%
\pgfpathlineto{\pgfqpoint{4.258121in}{2.353739in}}%
\pgfpathlineto{\pgfqpoint{4.271713in}{2.353009in}}%
\pgfpathlineto{\pgfqpoint{4.285314in}{2.352388in}}%
\pgfpathlineto{\pgfqpoint{4.298923in}{2.351874in}}%
\pgfpathlineto{\pgfqpoint{4.291173in}{2.342599in}}%
\pgfpathlineto{\pgfqpoint{4.283418in}{2.333306in}}%
\pgfpathlineto{\pgfqpoint{4.275657in}{2.323997in}}%
\pgfpathlineto{\pgfqpoint{4.267891in}{2.314672in}}%
\pgfpathlineto{\pgfqpoint{4.254273in}{2.315305in}}%
\pgfpathlineto{\pgfqpoint{4.240663in}{2.316046in}}%
\pgfpathlineto{\pgfqpoint{4.227061in}{2.316895in}}%
\pgfpathlineto{\pgfqpoint{4.213467in}{2.317852in}}%
\pgfpathlineto{\pgfqpoint{4.221242in}{2.327050in}}%
\pgfpathlineto{\pgfqpoint{4.229012in}{2.336238in}}%
\pgfpathlineto{\pgfqpoint{4.236777in}{2.345414in}}%
\pgfpathlineto{\pgfqpoint{4.244536in}{2.354577in}}%
\pgfpathclose%
\pgfusepath{fill}%
\end{pgfscope}%
\begin{pgfscope}%
\pgfpathrectangle{\pgfqpoint{1.150000in}{0.150000in}}{\pgfqpoint{5.700000in}{5.700000in}}%
\pgfusepath{clip}%
\pgfsetbuttcap%
\pgfsetroundjoin%
\definecolor{currentfill}{rgb}{0.271305,0.019942,0.347269}%
\pgfsetfillcolor{currentfill}%
\pgfsetfillopacity{0.700000}%
\pgfsetlinewidth{0.000000pt}%
\definecolor{currentstroke}{rgb}{0.000000,0.000000,0.000000}%
\pgfsetstrokecolor{currentstroke}%
\pgfsetdash{}{0pt}%
\pgfpathmoveto{\pgfqpoint{3.569723in}{2.274701in}}%
\pgfpathlineto{\pgfqpoint{3.583168in}{2.268696in}}%
\pgfpathlineto{\pgfqpoint{3.596617in}{2.262817in}}%
\pgfpathlineto{\pgfqpoint{3.610070in}{2.257062in}}%
\pgfpathlineto{\pgfqpoint{3.623527in}{2.251431in}}%
\pgfpathlineto{\pgfqpoint{3.615533in}{2.244078in}}%
\pgfpathlineto{\pgfqpoint{3.607532in}{2.236782in}}%
\pgfpathlineto{\pgfqpoint{3.599525in}{2.229548in}}%
\pgfpathlineto{\pgfqpoint{3.591511in}{2.222375in}}%
\pgfpathlineto{\pgfqpoint{3.578038in}{2.228234in}}%
\pgfpathlineto{\pgfqpoint{3.564569in}{2.234217in}}%
\pgfpathlineto{\pgfqpoint{3.551103in}{2.240325in}}%
\pgfpathlineto{\pgfqpoint{3.537641in}{2.246558in}}%
\pgfpathlineto{\pgfqpoint{3.545672in}{2.253496in}}%
\pgfpathlineto{\pgfqpoint{3.553696in}{2.260500in}}%
\pgfpathlineto{\pgfqpoint{3.561713in}{2.267569in}}%
\pgfpathlineto{\pgfqpoint{3.569723in}{2.274701in}}%
\pgfpathclose%
\pgfusepath{fill}%
\end{pgfscope}%
\begin{pgfscope}%
\pgfpathrectangle{\pgfqpoint{1.150000in}{0.150000in}}{\pgfqpoint{5.700000in}{5.700000in}}%
\pgfusepath{clip}%
\pgfsetbuttcap%
\pgfsetroundjoin%
\definecolor{currentfill}{rgb}{0.280894,0.078907,0.402329}%
\pgfsetfillcolor{currentfill}%
\pgfsetfillopacity{0.700000}%
\pgfsetlinewidth{0.000000pt}%
\definecolor{currentstroke}{rgb}{0.000000,0.000000,0.000000}%
\pgfsetstrokecolor{currentstroke}%
\pgfsetdash{}{0pt}%
\pgfpathmoveto{\pgfqpoint{3.236399in}{2.376599in}}%
\pgfpathlineto{\pgfqpoint{3.249835in}{2.367452in}}%
\pgfpathlineto{\pgfqpoint{3.263271in}{2.358446in}}%
\pgfpathlineto{\pgfqpoint{3.276708in}{2.349581in}}%
\pgfpathlineto{\pgfqpoint{3.290147in}{2.340855in}}%
\pgfpathlineto{\pgfqpoint{3.282002in}{2.335322in}}%
\pgfpathlineto{\pgfqpoint{3.273849in}{2.329889in}}%
\pgfpathlineto{\pgfqpoint{3.265687in}{2.324557in}}%
\pgfpathlineto{\pgfqpoint{3.257516in}{2.319329in}}%
\pgfpathlineto{\pgfqpoint{3.244056in}{2.328323in}}%
\pgfpathlineto{\pgfqpoint{3.230596in}{2.337456in}}%
\pgfpathlineto{\pgfqpoint{3.217137in}{2.346729in}}%
\pgfpathlineto{\pgfqpoint{3.203680in}{2.356145in}}%
\pgfpathlineto{\pgfqpoint{3.211872in}{2.361098in}}%
\pgfpathlineto{\pgfqpoint{3.220057in}{2.366160in}}%
\pgfpathlineto{\pgfqpoint{3.228232in}{2.371328in}}%
\pgfpathlineto{\pgfqpoint{3.236399in}{2.376599in}}%
\pgfpathclose%
\pgfusepath{fill}%
\end{pgfscope}%
\begin{pgfscope}%
\pgfpathrectangle{\pgfqpoint{1.150000in}{0.150000in}}{\pgfqpoint{5.700000in}{5.700000in}}%
\pgfusepath{clip}%
\pgfsetbuttcap%
\pgfsetroundjoin%
\definecolor{currentfill}{rgb}{0.269944,0.014625,0.341379}%
\pgfsetfillcolor{currentfill}%
\pgfsetfillopacity{0.700000}%
\pgfsetlinewidth{0.000000pt}%
\definecolor{currentstroke}{rgb}{0.000000,0.000000,0.000000}%
\pgfsetstrokecolor{currentstroke}%
\pgfsetdash{}{0pt}%
\pgfpathmoveto{\pgfqpoint{3.709246in}{2.260934in}}%
\pgfpathlineto{\pgfqpoint{3.722709in}{2.256124in}}%
\pgfpathlineto{\pgfqpoint{3.736178in}{2.251435in}}%
\pgfpathlineto{\pgfqpoint{3.749651in}{2.246866in}}%
\pgfpathlineto{\pgfqpoint{3.763128in}{2.242416in}}%
\pgfpathlineto{\pgfqpoint{3.755190in}{2.234443in}}%
\pgfpathlineto{\pgfqpoint{3.747244in}{2.226511in}}%
\pgfpathlineto{\pgfqpoint{3.739293in}{2.218623in}}%
\pgfpathlineto{\pgfqpoint{3.731336in}{2.210780in}}%
\pgfpathlineto{\pgfqpoint{3.717844in}{2.215439in}}%
\pgfpathlineto{\pgfqpoint{3.704356in}{2.220218in}}%
\pgfpathlineto{\pgfqpoint{3.690874in}{2.225117in}}%
\pgfpathlineto{\pgfqpoint{3.677395in}{2.230137in}}%
\pgfpathlineto{\pgfqpoint{3.685367in}{2.237763in}}%
\pgfpathlineto{\pgfqpoint{3.693333in}{2.245440in}}%
\pgfpathlineto{\pgfqpoint{3.701293in}{2.253164in}}%
\pgfpathlineto{\pgfqpoint{3.709246in}{2.260934in}}%
\pgfpathclose%
\pgfusepath{fill}%
\end{pgfscope}%
\begin{pgfscope}%
\pgfpathrectangle{\pgfqpoint{1.150000in}{0.150000in}}{\pgfqpoint{5.700000in}{5.700000in}}%
\pgfusepath{clip}%
\pgfsetbuttcap%
\pgfsetroundjoin%
\definecolor{currentfill}{rgb}{0.274952,0.037752,0.364543}%
\pgfsetfillcolor{currentfill}%
\pgfsetfillopacity{0.700000}%
\pgfsetlinewidth{0.000000pt}%
\definecolor{currentstroke}{rgb}{0.000000,0.000000,0.000000}%
\pgfsetstrokecolor{currentstroke}%
\pgfsetdash{}{0pt}%
\pgfpathmoveto{\pgfqpoint{3.430062in}{2.301022in}}%
\pgfpathlineto{\pgfqpoint{3.443499in}{2.293761in}}%
\pgfpathlineto{\pgfqpoint{3.456939in}{2.286630in}}%
\pgfpathlineto{\pgfqpoint{3.470381in}{2.279630in}}%
\pgfpathlineto{\pgfqpoint{3.483827in}{2.272759in}}%
\pgfpathlineto{\pgfqpoint{3.475772in}{2.266131in}}%
\pgfpathlineto{\pgfqpoint{3.467709in}{2.259579in}}%
\pgfpathlineto{\pgfqpoint{3.459639in}{2.253105in}}%
\pgfpathlineto{\pgfqpoint{3.451562in}{2.246712in}}%
\pgfpathlineto{\pgfqpoint{3.438098in}{2.253829in}}%
\pgfpathlineto{\pgfqpoint{3.424636in}{2.261077in}}%
\pgfpathlineto{\pgfqpoint{3.411177in}{2.268455in}}%
\pgfpathlineto{\pgfqpoint{3.397721in}{2.275964in}}%
\pgfpathlineto{\pgfqpoint{3.405817in}{2.282103in}}%
\pgfpathlineto{\pgfqpoint{3.413906in}{2.288328in}}%
\pgfpathlineto{\pgfqpoint{3.421988in}{2.294635in}}%
\pgfpathlineto{\pgfqpoint{3.430062in}{2.301022in}}%
\pgfpathclose%
\pgfusepath{fill}%
\end{pgfscope}%
\begin{pgfscope}%
\pgfpathrectangle{\pgfqpoint{1.150000in}{0.150000in}}{\pgfqpoint{5.700000in}{5.700000in}}%
\pgfusepath{clip}%
\pgfsetbuttcap%
\pgfsetroundjoin%
\definecolor{currentfill}{rgb}{0.281412,0.155834,0.469201}%
\pgfsetfillcolor{currentfill}%
\pgfsetfillopacity{0.700000}%
\pgfsetlinewidth{0.000000pt}%
\definecolor{currentstroke}{rgb}{0.000000,0.000000,0.000000}%
\pgfsetstrokecolor{currentstroke}%
\pgfsetdash{}{0pt}%
\pgfpathmoveto{\pgfqpoint{2.988369in}{2.526865in}}%
\pgfpathlineto{\pgfqpoint{3.001830in}{2.515044in}}%
\pgfpathlineto{\pgfqpoint{3.015291in}{2.503382in}}%
\pgfpathlineto{\pgfqpoint{3.028750in}{2.491877in}}%
\pgfpathlineto{\pgfqpoint{3.042208in}{2.480530in}}%
\pgfpathlineto{\pgfqpoint{3.033932in}{2.476531in}}%
\pgfpathlineto{\pgfqpoint{3.025647in}{2.472661in}}%
\pgfpathlineto{\pgfqpoint{3.017351in}{2.468921in}}%
\pgfpathlineto{\pgfqpoint{3.009046in}{2.465315in}}%
\pgfpathlineto{\pgfqpoint{2.995561in}{2.476953in}}%
\pgfpathlineto{\pgfqpoint{2.982075in}{2.488748in}}%
\pgfpathlineto{\pgfqpoint{2.968587in}{2.500701in}}%
\pgfpathlineto{\pgfqpoint{2.955098in}{2.512812in}}%
\pgfpathlineto{\pgfqpoint{2.963432in}{2.516121in}}%
\pgfpathlineto{\pgfqpoint{2.971754in}{2.519568in}}%
\pgfpathlineto{\pgfqpoint{2.980067in}{2.523150in}}%
\pgfpathlineto{\pgfqpoint{2.988369in}{2.526865in}}%
\pgfpathclose%
\pgfusepath{fill}%
\end{pgfscope}%
\begin{pgfscope}%
\pgfpathrectangle{\pgfqpoint{1.150000in}{0.150000in}}{\pgfqpoint{5.700000in}{5.700000in}}%
\pgfusepath{clip}%
\pgfsetbuttcap%
\pgfsetroundjoin%
\definecolor{currentfill}{rgb}{0.277941,0.056324,0.381191}%
\pgfsetfillcolor{currentfill}%
\pgfsetfillopacity{0.700000}%
\pgfsetlinewidth{0.000000pt}%
\definecolor{currentstroke}{rgb}{0.000000,0.000000,0.000000}%
\pgfsetstrokecolor{currentstroke}%
\pgfsetdash{}{0pt}%
\pgfpathmoveto{\pgfqpoint{4.159168in}{2.322773in}}%
\pgfpathlineto{\pgfqpoint{4.172731in}{2.321378in}}%
\pgfpathlineto{\pgfqpoint{4.186302in}{2.320094in}}%
\pgfpathlineto{\pgfqpoint{4.199880in}{2.318918in}}%
\pgfpathlineto{\pgfqpoint{4.213467in}{2.317852in}}%
\pgfpathlineto{\pgfqpoint{4.205686in}{2.308644in}}%
\pgfpathlineto{\pgfqpoint{4.197900in}{2.299428in}}%
\pgfpathlineto{\pgfqpoint{4.190109in}{2.290204in}}%
\pgfpathlineto{\pgfqpoint{4.182313in}{2.280973in}}%
\pgfpathlineto{\pgfqpoint{4.168717in}{2.282177in}}%
\pgfpathlineto{\pgfqpoint{4.155128in}{2.283489in}}%
\pgfpathlineto{\pgfqpoint{4.141548in}{2.284911in}}%
\pgfpathlineto{\pgfqpoint{4.127975in}{2.286443in}}%
\pgfpathlineto{\pgfqpoint{4.135781in}{2.295530in}}%
\pgfpathlineto{\pgfqpoint{4.143582in}{2.304614in}}%
\pgfpathlineto{\pgfqpoint{4.151378in}{2.313696in}}%
\pgfpathlineto{\pgfqpoint{4.159168in}{2.322773in}}%
\pgfpathclose%
\pgfusepath{fill}%
\end{pgfscope}%
\begin{pgfscope}%
\pgfpathrectangle{\pgfqpoint{1.150000in}{0.150000in}}{\pgfqpoint{5.700000in}{5.700000in}}%
\pgfusepath{clip}%
\pgfsetbuttcap%
\pgfsetroundjoin%
\definecolor{currentfill}{rgb}{0.269944,0.014625,0.341379}%
\pgfsetfillcolor{currentfill}%
\pgfsetfillopacity{0.700000}%
\pgfsetlinewidth{0.000000pt}%
\definecolor{currentstroke}{rgb}{0.000000,0.000000,0.000000}%
\pgfsetstrokecolor{currentstroke}%
\pgfsetdash{}{0pt}%
\pgfpathmoveto{\pgfqpoint{3.848734in}{2.258838in}}%
\pgfpathlineto{\pgfqpoint{3.862225in}{2.255168in}}%
\pgfpathlineto{\pgfqpoint{3.875722in}{2.251615in}}%
\pgfpathlineto{\pgfqpoint{3.889225in}{2.248178in}}%
\pgfpathlineto{\pgfqpoint{3.902733in}{2.244856in}}%
\pgfpathlineto{\pgfqpoint{3.894844in}{2.236365in}}%
\pgfpathlineto{\pgfqpoint{3.886949in}{2.227900in}}%
\pgfpathlineto{\pgfqpoint{3.879049in}{2.219461in}}%
\pgfpathlineto{\pgfqpoint{3.871142in}{2.211052in}}%
\pgfpathlineto{\pgfqpoint{3.857621in}{2.214565in}}%
\pgfpathlineto{\pgfqpoint{3.844106in}{2.218194in}}%
\pgfpathlineto{\pgfqpoint{3.830596in}{2.221939in}}%
\pgfpathlineto{\pgfqpoint{3.817092in}{2.225800in}}%
\pgfpathlineto{\pgfqpoint{3.825011in}{2.234011in}}%
\pgfpathlineto{\pgfqpoint{3.832925in}{2.242255in}}%
\pgfpathlineto{\pgfqpoint{3.840832in}{2.250532in}}%
\pgfpathlineto{\pgfqpoint{3.848734in}{2.258838in}}%
\pgfpathclose%
\pgfusepath{fill}%
\end{pgfscope}%
\begin{pgfscope}%
\pgfpathrectangle{\pgfqpoint{1.150000in}{0.150000in}}{\pgfqpoint{5.700000in}{5.700000in}}%
\pgfusepath{clip}%
\pgfsetbuttcap%
\pgfsetroundjoin%
\definecolor{currentfill}{rgb}{0.282884,0.135920,0.453427}%
\pgfsetfillcolor{currentfill}%
\pgfsetfillopacity{0.700000}%
\pgfsetlinewidth{0.000000pt}%
\definecolor{currentstroke}{rgb}{0.000000,0.000000,0.000000}%
\pgfsetstrokecolor{currentstroke}%
\pgfsetdash{}{0pt}%
\pgfpathmoveto{\pgfqpoint{3.042208in}{2.480530in}}%
\pgfpathlineto{\pgfqpoint{3.055666in}{2.469337in}}%
\pgfpathlineto{\pgfqpoint{3.069122in}{2.458298in}}%
\pgfpathlineto{\pgfqpoint{3.082578in}{2.447413in}}%
\pgfpathlineto{\pgfqpoint{3.096034in}{2.436679in}}%
\pgfpathlineto{\pgfqpoint{3.087784in}{2.432398in}}%
\pgfpathlineto{\pgfqpoint{3.079524in}{2.428241in}}%
\pgfpathlineto{\pgfqpoint{3.071255in}{2.424210in}}%
\pgfpathlineto{\pgfqpoint{3.062975in}{2.420307in}}%
\pgfpathlineto{\pgfqpoint{3.049494in}{2.431330in}}%
\pgfpathlineto{\pgfqpoint{3.036012in}{2.442505in}}%
\pgfpathlineto{\pgfqpoint{3.022529in}{2.453833in}}%
\pgfpathlineto{\pgfqpoint{3.009046in}{2.465315in}}%
\pgfpathlineto{\pgfqpoint{3.017351in}{2.468921in}}%
\pgfpathlineto{\pgfqpoint{3.025647in}{2.472661in}}%
\pgfpathlineto{\pgfqpoint{3.033932in}{2.476531in}}%
\pgfpathlineto{\pgfqpoint{3.042208in}{2.480530in}}%
\pgfpathclose%
\pgfusepath{fill}%
\end{pgfscope}%
\begin{pgfscope}%
\pgfpathrectangle{\pgfqpoint{1.150000in}{0.150000in}}{\pgfqpoint{5.700000in}{5.700000in}}%
\pgfusepath{clip}%
\pgfsetbuttcap%
\pgfsetroundjoin%
\definecolor{currentfill}{rgb}{0.274952,0.037752,0.364543}%
\pgfsetfillcolor{currentfill}%
\pgfsetfillopacity{0.700000}%
\pgfsetlinewidth{0.000000pt}%
\definecolor{currentstroke}{rgb}{0.000000,0.000000,0.000000}%
\pgfsetstrokecolor{currentstroke}%
\pgfsetdash{}{0pt}%
\pgfpathmoveto{\pgfqpoint{4.073754in}{2.293675in}}%
\pgfpathlineto{\pgfqpoint{4.087299in}{2.291701in}}%
\pgfpathlineto{\pgfqpoint{4.100850in}{2.289837in}}%
\pgfpathlineto{\pgfqpoint{4.114409in}{2.288085in}}%
\pgfpathlineto{\pgfqpoint{4.127975in}{2.286443in}}%
\pgfpathlineto{\pgfqpoint{4.120163in}{2.277356in}}%
\pgfpathlineto{\pgfqpoint{4.112346in}{2.268269in}}%
\pgfpathlineto{\pgfqpoint{4.104523in}{2.259184in}}%
\pgfpathlineto{\pgfqpoint{4.096696in}{2.250103in}}%
\pgfpathlineto{\pgfqpoint{4.083119in}{2.251900in}}%
\pgfpathlineto{\pgfqpoint{4.069550in}{2.253808in}}%
\pgfpathlineto{\pgfqpoint{4.055989in}{2.255826in}}%
\pgfpathlineto{\pgfqpoint{4.042434in}{2.257956in}}%
\pgfpathlineto{\pgfqpoint{4.050272in}{2.266876in}}%
\pgfpathlineto{\pgfqpoint{4.058105in}{2.275803in}}%
\pgfpathlineto{\pgfqpoint{4.065932in}{2.284737in}}%
\pgfpathlineto{\pgfqpoint{4.073754in}{2.293675in}}%
\pgfpathclose%
\pgfusepath{fill}%
\end{pgfscope}%
\begin{pgfscope}%
\pgfpathrectangle{\pgfqpoint{1.150000in}{0.150000in}}{\pgfqpoint{5.700000in}{5.700000in}}%
\pgfusepath{clip}%
\pgfsetbuttcap%
\pgfsetroundjoin%
\definecolor{currentfill}{rgb}{0.278791,0.062145,0.386592}%
\pgfsetfillcolor{currentfill}%
\pgfsetfillopacity{0.700000}%
\pgfsetlinewidth{0.000000pt}%
\definecolor{currentstroke}{rgb}{0.000000,0.000000,0.000000}%
\pgfsetstrokecolor{currentstroke}%
\pgfsetdash{}{0pt}%
\pgfpathmoveto{\pgfqpoint{3.290147in}{2.340855in}}%
\pgfpathlineto{\pgfqpoint{3.303588in}{2.332267in}}%
\pgfpathlineto{\pgfqpoint{3.317030in}{2.323818in}}%
\pgfpathlineto{\pgfqpoint{3.330473in}{2.315505in}}%
\pgfpathlineto{\pgfqpoint{3.343919in}{2.307328in}}%
\pgfpathlineto{\pgfqpoint{3.335794in}{2.301535in}}%
\pgfpathlineto{\pgfqpoint{3.327662in}{2.295837in}}%
\pgfpathlineto{\pgfqpoint{3.319522in}{2.290236in}}%
\pgfpathlineto{\pgfqpoint{3.311373in}{2.284734in}}%
\pgfpathlineto{\pgfqpoint{3.297907in}{2.293178in}}%
\pgfpathlineto{\pgfqpoint{3.284442in}{2.301758in}}%
\pgfpathlineto{\pgfqpoint{3.270978in}{2.310475in}}%
\pgfpathlineto{\pgfqpoint{3.257516in}{2.319329in}}%
\pgfpathlineto{\pgfqpoint{3.265687in}{2.324557in}}%
\pgfpathlineto{\pgfqpoint{3.273849in}{2.329889in}}%
\pgfpathlineto{\pgfqpoint{3.282002in}{2.335322in}}%
\pgfpathlineto{\pgfqpoint{3.290147in}{2.340855in}}%
\pgfpathclose%
\pgfusepath{fill}%
\end{pgfscope}%
\begin{pgfscope}%
\pgfpathrectangle{\pgfqpoint{1.150000in}{0.150000in}}{\pgfqpoint{5.700000in}{5.700000in}}%
\pgfusepath{clip}%
\pgfsetbuttcap%
\pgfsetroundjoin%
\definecolor{currentfill}{rgb}{0.274128,0.199721,0.498911}%
\pgfsetfillcolor{currentfill}%
\pgfsetfillopacity{0.700000}%
\pgfsetlinewidth{0.000000pt}%
\definecolor{currentstroke}{rgb}{0.000000,0.000000,0.000000}%
\pgfsetstrokecolor{currentstroke}%
\pgfsetdash{}{0pt}%
\pgfpathmoveto{\pgfqpoint{4.811408in}{2.595284in}}%
\pgfpathlineto{\pgfqpoint{4.825205in}{2.597723in}}%
\pgfpathlineto{\pgfqpoint{4.839012in}{2.600264in}}%
\pgfpathlineto{\pgfqpoint{4.852831in}{2.602907in}}%
\pgfpathlineto{\pgfqpoint{4.866661in}{2.605652in}}%
\pgfpathlineto{\pgfqpoint{4.859104in}{2.596908in}}%
\pgfpathlineto{\pgfqpoint{4.851541in}{2.588109in}}%
\pgfpathlineto{\pgfqpoint{4.843972in}{2.579252in}}%
\pgfpathlineto{\pgfqpoint{4.836397in}{2.570339in}}%
\pgfpathlineto{\pgfqpoint{4.822558in}{2.567604in}}%
\pgfpathlineto{\pgfqpoint{4.808731in}{2.564970in}}%
\pgfpathlineto{\pgfqpoint{4.794915in}{2.562439in}}%
\pgfpathlineto{\pgfqpoint{4.781109in}{2.560010in}}%
\pgfpathlineto{\pgfqpoint{4.788693in}{2.568906in}}%
\pgfpathlineto{\pgfqpoint{4.796270in}{2.577751in}}%
\pgfpathlineto{\pgfqpoint{4.803842in}{2.586543in}}%
\pgfpathlineto{\pgfqpoint{4.811408in}{2.595284in}}%
\pgfpathclose%
\pgfusepath{fill}%
\end{pgfscope}%
\begin{pgfscope}%
\pgfpathrectangle{\pgfqpoint{1.150000in}{0.150000in}}{\pgfqpoint{5.700000in}{5.700000in}}%
\pgfusepath{clip}%
\pgfsetbuttcap%
\pgfsetroundjoin%
\definecolor{currentfill}{rgb}{0.267968,0.223549,0.512008}%
\pgfsetfillcolor{currentfill}%
\pgfsetfillopacity{0.700000}%
\pgfsetlinewidth{0.000000pt}%
\definecolor{currentstroke}{rgb}{0.000000,0.000000,0.000000}%
\pgfsetstrokecolor{currentstroke}%
\pgfsetdash{}{0pt}%
\pgfpathmoveto{\pgfqpoint{4.896828in}{2.640069in}}%
\pgfpathlineto{\pgfqpoint{4.910660in}{2.642907in}}%
\pgfpathlineto{\pgfqpoint{4.924503in}{2.645847in}}%
\pgfpathlineto{\pgfqpoint{4.938358in}{2.648888in}}%
\pgfpathlineto{\pgfqpoint{4.952223in}{2.652030in}}%
\pgfpathlineto{\pgfqpoint{4.944700in}{2.643525in}}%
\pgfpathlineto{\pgfqpoint{4.937170in}{2.634959in}}%
\pgfpathlineto{\pgfqpoint{4.929634in}{2.626334in}}%
\pgfpathlineto{\pgfqpoint{4.922092in}{2.617649in}}%
\pgfpathlineto{\pgfqpoint{4.908217in}{2.614497in}}%
\pgfpathlineto{\pgfqpoint{4.894353in}{2.611447in}}%
\pgfpathlineto{\pgfqpoint{4.880501in}{2.608499in}}%
\pgfpathlineto{\pgfqpoint{4.866661in}{2.605652in}}%
\pgfpathlineto{\pgfqpoint{4.874212in}{2.614339in}}%
\pgfpathlineto{\pgfqpoint{4.881756in}{2.622971in}}%
\pgfpathlineto{\pgfqpoint{4.889295in}{2.631547in}}%
\pgfpathlineto{\pgfqpoint{4.896828in}{2.640069in}}%
\pgfpathclose%
\pgfusepath{fill}%
\end{pgfscope}%
\begin{pgfscope}%
\pgfpathrectangle{\pgfqpoint{1.150000in}{0.150000in}}{\pgfqpoint{5.700000in}{5.700000in}}%
\pgfusepath{clip}%
\pgfsetbuttcap%
\pgfsetroundjoin%
\definecolor{currentfill}{rgb}{0.278012,0.180367,0.486697}%
\pgfsetfillcolor{currentfill}%
\pgfsetfillopacity{0.700000}%
\pgfsetlinewidth{0.000000pt}%
\definecolor{currentstroke}{rgb}{0.000000,0.000000,0.000000}%
\pgfsetstrokecolor{currentstroke}%
\pgfsetdash{}{0pt}%
\pgfpathmoveto{\pgfqpoint{4.725994in}{2.551319in}}%
\pgfpathlineto{\pgfqpoint{4.739757in}{2.553338in}}%
\pgfpathlineto{\pgfqpoint{4.753530in}{2.555459in}}%
\pgfpathlineto{\pgfqpoint{4.767315in}{2.557683in}}%
\pgfpathlineto{\pgfqpoint{4.781109in}{2.560010in}}%
\pgfpathlineto{\pgfqpoint{4.773520in}{2.551061in}}%
\pgfpathlineto{\pgfqpoint{4.765925in}{2.542060in}}%
\pgfpathlineto{\pgfqpoint{4.758324in}{2.533006in}}%
\pgfpathlineto{\pgfqpoint{4.750717in}{2.523900in}}%
\pgfpathlineto{\pgfqpoint{4.736914in}{2.521602in}}%
\pgfpathlineto{\pgfqpoint{4.723122in}{2.519406in}}%
\pgfpathlineto{\pgfqpoint{4.709340in}{2.517313in}}%
\pgfpathlineto{\pgfqpoint{4.695568in}{2.515322in}}%
\pgfpathlineto{\pgfqpoint{4.703183in}{2.524393in}}%
\pgfpathlineto{\pgfqpoint{4.710793in}{2.533416in}}%
\pgfpathlineto{\pgfqpoint{4.718396in}{2.542392in}}%
\pgfpathlineto{\pgfqpoint{4.725994in}{2.551319in}}%
\pgfpathclose%
\pgfusepath{fill}%
\end{pgfscope}%
\begin{pgfscope}%
\pgfpathrectangle{\pgfqpoint{1.150000in}{0.150000in}}{\pgfqpoint{5.700000in}{5.700000in}}%
\pgfusepath{clip}%
\pgfsetbuttcap%
\pgfsetroundjoin%
\definecolor{currentfill}{rgb}{0.260571,0.246922,0.522828}%
\pgfsetfillcolor{currentfill}%
\pgfsetfillopacity{0.700000}%
\pgfsetlinewidth{0.000000pt}%
\definecolor{currentstroke}{rgb}{0.000000,0.000000,0.000000}%
\pgfsetstrokecolor{currentstroke}%
\pgfsetdash{}{0pt}%
\pgfpathmoveto{\pgfqpoint{4.982256in}{2.685469in}}%
\pgfpathlineto{\pgfqpoint{4.996124in}{2.688685in}}%
\pgfpathlineto{\pgfqpoint{5.010003in}{2.692002in}}%
\pgfpathlineto{\pgfqpoint{5.023895in}{2.695421in}}%
\pgfpathlineto{\pgfqpoint{5.037797in}{2.698940in}}%
\pgfpathlineto{\pgfqpoint{5.030308in}{2.690701in}}%
\pgfpathlineto{\pgfqpoint{5.022813in}{2.682401in}}%
\pgfpathlineto{\pgfqpoint{5.015311in}{2.674038in}}%
\pgfpathlineto{\pgfqpoint{5.007803in}{2.665612in}}%
\pgfpathlineto{\pgfqpoint{4.993890in}{2.662065in}}%
\pgfpathlineto{\pgfqpoint{4.979990in}{2.658619in}}%
\pgfpathlineto{\pgfqpoint{4.966101in}{2.655274in}}%
\pgfpathlineto{\pgfqpoint{4.952223in}{2.652030in}}%
\pgfpathlineto{\pgfqpoint{4.959741in}{2.660477in}}%
\pgfpathlineto{\pgfqpoint{4.967252in}{2.668865in}}%
\pgfpathlineto{\pgfqpoint{4.974757in}{2.677196in}}%
\pgfpathlineto{\pgfqpoint{4.982256in}{2.685469in}}%
\pgfpathclose%
\pgfusepath{fill}%
\end{pgfscope}%
\begin{pgfscope}%
\pgfpathrectangle{\pgfqpoint{1.150000in}{0.150000in}}{\pgfqpoint{5.700000in}{5.700000in}}%
\pgfusepath{clip}%
\pgfsetbuttcap%
\pgfsetroundjoin%
\definecolor{currentfill}{rgb}{0.253935,0.265254,0.529983}%
\pgfsetfillcolor{currentfill}%
\pgfsetfillopacity{0.700000}%
\pgfsetlinewidth{0.000000pt}%
\definecolor{currentstroke}{rgb}{0.000000,0.000000,0.000000}%
\pgfsetstrokecolor{currentstroke}%
\pgfsetdash{}{0pt}%
\pgfpathmoveto{\pgfqpoint{5.067691in}{2.731290in}}%
\pgfpathlineto{\pgfqpoint{5.081596in}{2.734864in}}%
\pgfpathlineto{\pgfqpoint{5.095512in}{2.738538in}}%
\pgfpathlineto{\pgfqpoint{5.109441in}{2.742313in}}%
\pgfpathlineto{\pgfqpoint{5.123382in}{2.746188in}}%
\pgfpathlineto{\pgfqpoint{5.115928in}{2.738243in}}%
\pgfpathlineto{\pgfqpoint{5.108468in}{2.730234in}}%
\pgfpathlineto{\pgfqpoint{5.101002in}{2.722161in}}%
\pgfpathlineto{\pgfqpoint{5.093529in}{2.714023in}}%
\pgfpathlineto{\pgfqpoint{5.079578in}{2.710102in}}%
\pgfpathlineto{\pgfqpoint{5.065639in}{2.706281in}}%
\pgfpathlineto{\pgfqpoint{5.051712in}{2.702560in}}%
\pgfpathlineto{\pgfqpoint{5.037797in}{2.698940in}}%
\pgfpathlineto{\pgfqpoint{5.045280in}{2.707117in}}%
\pgfpathlineto{\pgfqpoint{5.052757in}{2.715234in}}%
\pgfpathlineto{\pgfqpoint{5.060227in}{2.723292in}}%
\pgfpathlineto{\pgfqpoint{5.067691in}{2.731290in}}%
\pgfpathclose%
\pgfusepath{fill}%
\end{pgfscope}%
\begin{pgfscope}%
\pgfpathrectangle{\pgfqpoint{1.150000in}{0.150000in}}{\pgfqpoint{5.700000in}{5.700000in}}%
\pgfusepath{clip}%
\pgfsetbuttcap%
\pgfsetroundjoin%
\definecolor{currentfill}{rgb}{0.280868,0.160771,0.472899}%
\pgfsetfillcolor{currentfill}%
\pgfsetfillopacity{0.700000}%
\pgfsetlinewidth{0.000000pt}%
\definecolor{currentstroke}{rgb}{0.000000,0.000000,0.000000}%
\pgfsetstrokecolor{currentstroke}%
\pgfsetdash{}{0pt}%
\pgfpathmoveto{\pgfqpoint{4.640585in}{2.508393in}}%
\pgfpathlineto{\pgfqpoint{4.654316in}{2.509970in}}%
\pgfpathlineto{\pgfqpoint{4.668057in}{2.511651in}}%
\pgfpathlineto{\pgfqpoint{4.681807in}{2.513435in}}%
\pgfpathlineto{\pgfqpoint{4.695568in}{2.515322in}}%
\pgfpathlineto{\pgfqpoint{4.687948in}{2.506203in}}%
\pgfpathlineto{\pgfqpoint{4.680322in}{2.497037in}}%
\pgfpathlineto{\pgfqpoint{4.672690in}{2.487823in}}%
\pgfpathlineto{\pgfqpoint{4.665052in}{2.478561in}}%
\pgfpathlineto{\pgfqpoint{4.651283in}{2.476720in}}%
\pgfpathlineto{\pgfqpoint{4.637524in}{2.474983in}}%
\pgfpathlineto{\pgfqpoint{4.623775in}{2.473349in}}%
\pgfpathlineto{\pgfqpoint{4.610036in}{2.471818in}}%
\pgfpathlineto{\pgfqpoint{4.617682in}{2.481026in}}%
\pgfpathlineto{\pgfqpoint{4.625322in}{2.490192in}}%
\pgfpathlineto{\pgfqpoint{4.632956in}{2.499314in}}%
\pgfpathlineto{\pgfqpoint{4.640585in}{2.508393in}}%
\pgfpathclose%
\pgfusepath{fill}%
\end{pgfscope}%
\begin{pgfscope}%
\pgfpathrectangle{\pgfqpoint{1.150000in}{0.150000in}}{\pgfqpoint{5.700000in}{5.700000in}}%
\pgfusepath{clip}%
\pgfsetbuttcap%
\pgfsetroundjoin%
\definecolor{currentfill}{rgb}{0.244972,0.287675,0.537260}%
\pgfsetfillcolor{currentfill}%
\pgfsetfillopacity{0.700000}%
\pgfsetlinewidth{0.000000pt}%
\definecolor{currentstroke}{rgb}{0.000000,0.000000,0.000000}%
\pgfsetstrokecolor{currentstroke}%
\pgfsetdash{}{0pt}%
\pgfpathmoveto{\pgfqpoint{5.153130in}{2.777351in}}%
\pgfpathlineto{\pgfqpoint{5.167073in}{2.781262in}}%
\pgfpathlineto{\pgfqpoint{5.181027in}{2.785273in}}%
\pgfpathlineto{\pgfqpoint{5.194994in}{2.789383in}}%
\pgfpathlineto{\pgfqpoint{5.208974in}{2.793594in}}%
\pgfpathlineto{\pgfqpoint{5.201557in}{2.785966in}}%
\pgfpathlineto{\pgfqpoint{5.194134in}{2.778273in}}%
\pgfpathlineto{\pgfqpoint{5.186704in}{2.770515in}}%
\pgfpathlineto{\pgfqpoint{5.179268in}{2.762691in}}%
\pgfpathlineto{\pgfqpoint{5.165278in}{2.758415in}}%
\pgfpathlineto{\pgfqpoint{5.151300in}{2.754239in}}%
\pgfpathlineto{\pgfqpoint{5.137335in}{2.750164in}}%
\pgfpathlineto{\pgfqpoint{5.123382in}{2.746188in}}%
\pgfpathlineto{\pgfqpoint{5.130829in}{2.754071in}}%
\pgfpathlineto{\pgfqpoint{5.138269in}{2.761891in}}%
\pgfpathlineto{\pgfqpoint{5.145703in}{2.769651in}}%
\pgfpathlineto{\pgfqpoint{5.153130in}{2.777351in}}%
\pgfpathclose%
\pgfusepath{fill}%
\end{pgfscope}%
\begin{pgfscope}%
\pgfpathrectangle{\pgfqpoint{1.150000in}{0.150000in}}{\pgfqpoint{5.700000in}{5.700000in}}%
\pgfusepath{clip}%
\pgfsetbuttcap%
\pgfsetroundjoin%
\definecolor{currentfill}{rgb}{0.235526,0.309527,0.542944}%
\pgfsetfillcolor{currentfill}%
\pgfsetfillopacity{0.700000}%
\pgfsetlinewidth{0.000000pt}%
\definecolor{currentstroke}{rgb}{0.000000,0.000000,0.000000}%
\pgfsetstrokecolor{currentstroke}%
\pgfsetdash{}{0pt}%
\pgfpathmoveto{\pgfqpoint{5.238572in}{2.823484in}}%
\pgfpathlineto{\pgfqpoint{5.252553in}{2.827711in}}%
\pgfpathlineto{\pgfqpoint{5.266546in}{2.832038in}}%
\pgfpathlineto{\pgfqpoint{5.280552in}{2.836464in}}%
\pgfpathlineto{\pgfqpoint{5.294570in}{2.840989in}}%
\pgfpathlineto{\pgfqpoint{5.287193in}{2.833698in}}%
\pgfpathlineto{\pgfqpoint{5.279808in}{2.826343in}}%
\pgfpathlineto{\pgfqpoint{5.272416in}{2.818922in}}%
\pgfpathlineto{\pgfqpoint{5.265018in}{2.811435in}}%
\pgfpathlineto{\pgfqpoint{5.250988in}{2.806825in}}%
\pgfpathlineto{\pgfqpoint{5.236970in}{2.802315in}}%
\pgfpathlineto{\pgfqpoint{5.222966in}{2.797905in}}%
\pgfpathlineto{\pgfqpoint{5.208974in}{2.793594in}}%
\pgfpathlineto{\pgfqpoint{5.216383in}{2.801159in}}%
\pgfpathlineto{\pgfqpoint{5.223786in}{2.808661in}}%
\pgfpathlineto{\pgfqpoint{5.231183in}{2.816103in}}%
\pgfpathlineto{\pgfqpoint{5.238572in}{2.823484in}}%
\pgfpathclose%
\pgfusepath{fill}%
\end{pgfscope}%
\begin{pgfscope}%
\pgfpathrectangle{\pgfqpoint{1.150000in}{0.150000in}}{\pgfqpoint{5.700000in}{5.700000in}}%
\pgfusepath{clip}%
\pgfsetbuttcap%
\pgfsetroundjoin%
\definecolor{currentfill}{rgb}{0.282884,0.135920,0.453427}%
\pgfsetfillcolor{currentfill}%
\pgfsetfillopacity{0.700000}%
\pgfsetlinewidth{0.000000pt}%
\definecolor{currentstroke}{rgb}{0.000000,0.000000,0.000000}%
\pgfsetstrokecolor{currentstroke}%
\pgfsetdash{}{0pt}%
\pgfpathmoveto{\pgfqpoint{4.555179in}{2.466736in}}%
\pgfpathlineto{\pgfqpoint{4.568879in}{2.467850in}}%
\pgfpathlineto{\pgfqpoint{4.582588in}{2.469069in}}%
\pgfpathlineto{\pgfqpoint{4.596307in}{2.470392in}}%
\pgfpathlineto{\pgfqpoint{4.610036in}{2.471818in}}%
\pgfpathlineto{\pgfqpoint{4.602385in}{2.462567in}}%
\pgfpathlineto{\pgfqpoint{4.594728in}{2.453274in}}%
\pgfpathlineto{\pgfqpoint{4.587065in}{2.443939in}}%
\pgfpathlineto{\pgfqpoint{4.579397in}{2.434563in}}%
\pgfpathlineto{\pgfqpoint{4.565660in}{2.433201in}}%
\pgfpathlineto{\pgfqpoint{4.551933in}{2.431943in}}%
\pgfpathlineto{\pgfqpoint{4.538216in}{2.430789in}}%
\pgfpathlineto{\pgfqpoint{4.524508in}{2.429740in}}%
\pgfpathlineto{\pgfqpoint{4.532184in}{2.439045in}}%
\pgfpathlineto{\pgfqpoint{4.539854in}{2.448313in}}%
\pgfpathlineto{\pgfqpoint{4.547520in}{2.457544in}}%
\pgfpathlineto{\pgfqpoint{4.555179in}{2.466736in}}%
\pgfpathclose%
\pgfusepath{fill}%
\end{pgfscope}%
\begin{pgfscope}%
\pgfpathrectangle{\pgfqpoint{1.150000in}{0.150000in}}{\pgfqpoint{5.700000in}{5.700000in}}%
\pgfusepath{clip}%
\pgfsetbuttcap%
\pgfsetroundjoin%
\definecolor{currentfill}{rgb}{0.269944,0.014625,0.341379}%
\pgfsetfillcolor{currentfill}%
\pgfsetfillopacity{0.700000}%
\pgfsetlinewidth{0.000000pt}%
\definecolor{currentstroke}{rgb}{0.000000,0.000000,0.000000}%
\pgfsetstrokecolor{currentstroke}%
\pgfsetdash{}{0pt}%
\pgfpathmoveto{\pgfqpoint{3.623527in}{2.251431in}}%
\pgfpathlineto{\pgfqpoint{3.636987in}{2.245924in}}%
\pgfpathlineto{\pgfqpoint{3.650452in}{2.240539in}}%
\pgfpathlineto{\pgfqpoint{3.663922in}{2.235277in}}%
\pgfpathlineto{\pgfqpoint{3.677395in}{2.230137in}}%
\pgfpathlineto{\pgfqpoint{3.669417in}{2.222562in}}%
\pgfpathlineto{\pgfqpoint{3.661432in}{2.215041in}}%
\pgfpathlineto{\pgfqpoint{3.653441in}{2.207576in}}%
\pgfpathlineto{\pgfqpoint{3.645443in}{2.200169in}}%
\pgfpathlineto{\pgfqpoint{3.631954in}{2.205537in}}%
\pgfpathlineto{\pgfqpoint{3.618469in}{2.211027in}}%
\pgfpathlineto{\pgfqpoint{3.604988in}{2.216640in}}%
\pgfpathlineto{\pgfqpoint{3.591511in}{2.222375in}}%
\pgfpathlineto{\pgfqpoint{3.599525in}{2.229548in}}%
\pgfpathlineto{\pgfqpoint{3.607532in}{2.236782in}}%
\pgfpathlineto{\pgfqpoint{3.615533in}{2.244078in}}%
\pgfpathlineto{\pgfqpoint{3.623527in}{2.251431in}}%
\pgfpathclose%
\pgfusepath{fill}%
\end{pgfscope}%
\begin{pgfscope}%
\pgfpathrectangle{\pgfqpoint{1.150000in}{0.150000in}}{\pgfqpoint{5.700000in}{5.700000in}}%
\pgfusepath{clip}%
\pgfsetbuttcap%
\pgfsetroundjoin%
\definecolor{currentfill}{rgb}{0.177423,0.437527,0.557565}%
\pgfsetfillcolor{currentfill}%
\pgfsetfillopacity{0.700000}%
\pgfsetlinewidth{0.000000pt}%
\definecolor{currentstroke}{rgb}{0.000000,0.000000,0.000000}%
\pgfsetstrokecolor{currentstroke}%
\pgfsetdash{}{0pt}%
\pgfpathmoveto{\pgfqpoint{5.836382in}{3.136840in}}%
\pgfpathlineto{\pgfqpoint{5.850635in}{3.142706in}}%
\pgfpathlineto{\pgfqpoint{5.864902in}{3.148670in}}%
\pgfpathlineto{\pgfqpoint{5.879184in}{3.154731in}}%
\pgfpathlineto{\pgfqpoint{5.893481in}{3.160889in}}%
\pgfpathlineto{\pgfqpoint{5.886414in}{3.156277in}}%
\pgfpathlineto{\pgfqpoint{5.879339in}{3.151622in}}%
\pgfpathlineto{\pgfqpoint{5.872257in}{3.146921in}}%
\pgfpathlineto{\pgfqpoint{5.865167in}{3.142171in}}%
\pgfpathlineto{\pgfqpoint{5.850851in}{3.135794in}}%
\pgfpathlineto{\pgfqpoint{5.836549in}{3.129515in}}%
\pgfpathlineto{\pgfqpoint{5.822263in}{3.123334in}}%
\pgfpathlineto{\pgfqpoint{5.807991in}{3.117250in}}%
\pgfpathlineto{\pgfqpoint{5.815099in}{3.122211in}}%
\pgfpathlineto{\pgfqpoint{5.822201in}{3.127128in}}%
\pgfpathlineto{\pgfqpoint{5.829295in}{3.132003in}}%
\pgfpathlineto{\pgfqpoint{5.836382in}{3.136840in}}%
\pgfpathclose%
\pgfusepath{fill}%
\end{pgfscope}%
\begin{pgfscope}%
\pgfpathrectangle{\pgfqpoint{1.150000in}{0.150000in}}{\pgfqpoint{5.700000in}{5.700000in}}%
\pgfusepath{clip}%
\pgfsetbuttcap%
\pgfsetroundjoin%
\definecolor{currentfill}{rgb}{0.225863,0.330805,0.547314}%
\pgfsetfillcolor{currentfill}%
\pgfsetfillopacity{0.700000}%
\pgfsetlinewidth{0.000000pt}%
\definecolor{currentstroke}{rgb}{0.000000,0.000000,0.000000}%
\pgfsetstrokecolor{currentstroke}%
\pgfsetdash{}{0pt}%
\pgfpathmoveto{\pgfqpoint{5.324014in}{2.869532in}}%
\pgfpathlineto{\pgfqpoint{5.338033in}{2.874055in}}%
\pgfpathlineto{\pgfqpoint{5.352065in}{2.878676in}}%
\pgfpathlineto{\pgfqpoint{5.366110in}{2.883397in}}%
\pgfpathlineto{\pgfqpoint{5.380168in}{2.888217in}}%
\pgfpathlineto{\pgfqpoint{5.372830in}{2.881281in}}%
\pgfpathlineto{\pgfqpoint{5.365485in}{2.874282in}}%
\pgfpathlineto{\pgfqpoint{5.358133in}{2.867217in}}%
\pgfpathlineto{\pgfqpoint{5.350775in}{2.860085in}}%
\pgfpathlineto{\pgfqpoint{5.336704in}{2.855162in}}%
\pgfpathlineto{\pgfqpoint{5.322646in}{2.850338in}}%
\pgfpathlineto{\pgfqpoint{5.308602in}{2.845614in}}%
\pgfpathlineto{\pgfqpoint{5.294570in}{2.840989in}}%
\pgfpathlineto{\pgfqpoint{5.301941in}{2.848216in}}%
\pgfpathlineto{\pgfqpoint{5.309306in}{2.855382in}}%
\pgfpathlineto{\pgfqpoint{5.316663in}{2.862487in}}%
\pgfpathlineto{\pgfqpoint{5.324014in}{2.869532in}}%
\pgfpathclose%
\pgfusepath{fill}%
\end{pgfscope}%
\begin{pgfscope}%
\pgfpathrectangle{\pgfqpoint{1.150000in}{0.150000in}}{\pgfqpoint{5.700000in}{5.700000in}}%
\pgfusepath{clip}%
\pgfsetbuttcap%
\pgfsetroundjoin%
\definecolor{currentfill}{rgb}{0.183898,0.422383,0.556944}%
\pgfsetfillcolor{currentfill}%
\pgfsetfillopacity{0.700000}%
\pgfsetlinewidth{0.000000pt}%
\definecolor{currentstroke}{rgb}{0.000000,0.000000,0.000000}%
\pgfsetstrokecolor{currentstroke}%
\pgfsetdash{}{0pt}%
\pgfpathmoveto{\pgfqpoint{5.751047in}{3.093891in}}%
\pgfpathlineto{\pgfqpoint{5.765261in}{3.099584in}}%
\pgfpathlineto{\pgfqpoint{5.779490in}{3.105375in}}%
\pgfpathlineto{\pgfqpoint{5.793733in}{3.111264in}}%
\pgfpathlineto{\pgfqpoint{5.807991in}{3.117250in}}%
\pgfpathlineto{\pgfqpoint{5.800875in}{3.112243in}}%
\pgfpathlineto{\pgfqpoint{5.793751in}{3.107187in}}%
\pgfpathlineto{\pgfqpoint{5.786621in}{3.102081in}}%
\pgfpathlineto{\pgfqpoint{5.779483in}{3.096921in}}%
\pgfpathlineto{\pgfqpoint{5.765207in}{3.090735in}}%
\pgfpathlineto{\pgfqpoint{5.750946in}{3.084648in}}%
\pgfpathlineto{\pgfqpoint{5.736699in}{3.078659in}}%
\pgfpathlineto{\pgfqpoint{5.722467in}{3.072767in}}%
\pgfpathlineto{\pgfqpoint{5.729623in}{3.078119in}}%
\pgfpathlineto{\pgfqpoint{5.736771in}{3.083422in}}%
\pgfpathlineto{\pgfqpoint{5.743913in}{3.088679in}}%
\pgfpathlineto{\pgfqpoint{5.751047in}{3.093891in}}%
\pgfpathclose%
\pgfusepath{fill}%
\end{pgfscope}%
\begin{pgfscope}%
\pgfpathrectangle{\pgfqpoint{1.150000in}{0.150000in}}{\pgfqpoint{5.700000in}{5.700000in}}%
\pgfusepath{clip}%
\pgfsetbuttcap%
\pgfsetroundjoin%
\definecolor{currentfill}{rgb}{0.218130,0.347432,0.550038}%
\pgfsetfillcolor{currentfill}%
\pgfsetfillopacity{0.700000}%
\pgfsetlinewidth{0.000000pt}%
\definecolor{currentstroke}{rgb}{0.000000,0.000000,0.000000}%
\pgfsetstrokecolor{currentstroke}%
\pgfsetdash{}{0pt}%
\pgfpathmoveto{\pgfqpoint{5.409450in}{2.915351in}}%
\pgfpathlineto{\pgfqpoint{5.423508in}{2.920149in}}%
\pgfpathlineto{\pgfqpoint{5.437579in}{2.925045in}}%
\pgfpathlineto{\pgfqpoint{5.451664in}{2.930040in}}%
\pgfpathlineto{\pgfqpoint{5.465762in}{2.935134in}}%
\pgfpathlineto{\pgfqpoint{5.458465in}{2.928568in}}%
\pgfpathlineto{\pgfqpoint{5.451162in}{2.921939in}}%
\pgfpathlineto{\pgfqpoint{5.443851in}{2.915246in}}%
\pgfpathlineto{\pgfqpoint{5.436534in}{2.908488in}}%
\pgfpathlineto{\pgfqpoint{5.422422in}{2.903272in}}%
\pgfpathlineto{\pgfqpoint{5.408324in}{2.898154in}}%
\pgfpathlineto{\pgfqpoint{5.394239in}{2.893136in}}%
\pgfpathlineto{\pgfqpoint{5.380168in}{2.888217in}}%
\pgfpathlineto{\pgfqpoint{5.387499in}{2.895090in}}%
\pgfpathlineto{\pgfqpoint{5.394823in}{2.901903in}}%
\pgfpathlineto{\pgfqpoint{5.402140in}{2.908656in}}%
\pgfpathlineto{\pgfqpoint{5.409450in}{2.915351in}}%
\pgfpathclose%
\pgfusepath{fill}%
\end{pgfscope}%
\begin{pgfscope}%
\pgfpathrectangle{\pgfqpoint{1.150000in}{0.150000in}}{\pgfqpoint{5.700000in}{5.700000in}}%
\pgfusepath{clip}%
\pgfsetbuttcap%
\pgfsetroundjoin%
\definecolor{currentfill}{rgb}{0.208623,0.367752,0.552675}%
\pgfsetfillcolor{currentfill}%
\pgfsetfillopacity{0.700000}%
\pgfsetlinewidth{0.000000pt}%
\definecolor{currentstroke}{rgb}{0.000000,0.000000,0.000000}%
\pgfsetstrokecolor{currentstroke}%
\pgfsetdash{}{0pt}%
\pgfpathmoveto{\pgfqpoint{5.494877in}{2.960810in}}%
\pgfpathlineto{\pgfqpoint{5.508974in}{2.965862in}}%
\pgfpathlineto{\pgfqpoint{5.523084in}{2.971012in}}%
\pgfpathlineto{\pgfqpoint{5.537209in}{2.976261in}}%
\pgfpathlineto{\pgfqpoint{5.551347in}{2.981609in}}%
\pgfpathlineto{\pgfqpoint{5.544093in}{2.975423in}}%
\pgfpathlineto{\pgfqpoint{5.536833in}{2.969177in}}%
\pgfpathlineto{\pgfqpoint{5.529565in}{2.962869in}}%
\pgfpathlineto{\pgfqpoint{5.522290in}{2.956498in}}%
\pgfpathlineto{\pgfqpoint{5.508137in}{2.951009in}}%
\pgfpathlineto{\pgfqpoint{5.493998in}{2.945618in}}%
\pgfpathlineto{\pgfqpoint{5.479873in}{2.940327in}}%
\pgfpathlineto{\pgfqpoint{5.465762in}{2.935134in}}%
\pgfpathlineto{\pgfqpoint{5.473051in}{2.941640in}}%
\pgfpathlineto{\pgfqpoint{5.480333in}{2.948087in}}%
\pgfpathlineto{\pgfqpoint{5.487608in}{2.954476in}}%
\pgfpathlineto{\pgfqpoint{5.494877in}{2.960810in}}%
\pgfpathclose%
\pgfusepath{fill}%
\end{pgfscope}%
\begin{pgfscope}%
\pgfpathrectangle{\pgfqpoint{1.150000in}{0.150000in}}{\pgfqpoint{5.700000in}{5.700000in}}%
\pgfusepath{clip}%
\pgfsetbuttcap%
\pgfsetroundjoin%
\definecolor{currentfill}{rgb}{0.192357,0.403199,0.555836}%
\pgfsetfillcolor{currentfill}%
\pgfsetfillopacity{0.700000}%
\pgfsetlinewidth{0.000000pt}%
\definecolor{currentstroke}{rgb}{0.000000,0.000000,0.000000}%
\pgfsetstrokecolor{currentstroke}%
\pgfsetdash{}{0pt}%
\pgfpathmoveto{\pgfqpoint{5.665681in}{3.050180in}}%
\pgfpathlineto{\pgfqpoint{5.679856in}{3.055680in}}%
\pgfpathlineto{\pgfqpoint{5.694046in}{3.061278in}}%
\pgfpathlineto{\pgfqpoint{5.708249in}{3.066973in}}%
\pgfpathlineto{\pgfqpoint{5.722467in}{3.072767in}}%
\pgfpathlineto{\pgfqpoint{5.715304in}{3.067364in}}%
\pgfpathlineto{\pgfqpoint{5.708134in}{3.061908in}}%
\pgfpathlineto{\pgfqpoint{5.700956in}{3.056396in}}%
\pgfpathlineto{\pgfqpoint{5.693771in}{3.050827in}}%
\pgfpathlineto{\pgfqpoint{5.679536in}{3.044853in}}%
\pgfpathlineto{\pgfqpoint{5.665315in}{3.038978in}}%
\pgfpathlineto{\pgfqpoint{5.651109in}{3.033201in}}%
\pgfpathlineto{\pgfqpoint{5.636917in}{3.027522in}}%
\pgfpathlineto{\pgfqpoint{5.644119in}{3.033264in}}%
\pgfpathlineto{\pgfqpoint{5.651313in}{3.038953in}}%
\pgfpathlineto{\pgfqpoint{5.658501in}{3.044591in}}%
\pgfpathlineto{\pgfqpoint{5.665681in}{3.050180in}}%
\pgfpathclose%
\pgfusepath{fill}%
\end{pgfscope}%
\begin{pgfscope}%
\pgfpathrectangle{\pgfqpoint{1.150000in}{0.150000in}}{\pgfqpoint{5.700000in}{5.700000in}}%
\pgfusepath{clip}%
\pgfsetbuttcap%
\pgfsetroundjoin%
\definecolor{currentfill}{rgb}{0.272594,0.025563,0.353093}%
\pgfsetfillcolor{currentfill}%
\pgfsetfillopacity{0.700000}%
\pgfsetlinewidth{0.000000pt}%
\definecolor{currentstroke}{rgb}{0.000000,0.000000,0.000000}%
\pgfsetstrokecolor{currentstroke}%
\pgfsetdash{}{0pt}%
\pgfpathmoveto{\pgfqpoint{3.483827in}{2.272759in}}%
\pgfpathlineto{\pgfqpoint{3.497276in}{2.266017in}}%
\pgfpathlineto{\pgfqpoint{3.510728in}{2.259404in}}%
\pgfpathlineto{\pgfqpoint{3.524183in}{2.252918in}}%
\pgfpathlineto{\pgfqpoint{3.537641in}{2.246558in}}%
\pgfpathlineto{\pgfqpoint{3.529603in}{2.239690in}}%
\pgfpathlineto{\pgfqpoint{3.521559in}{2.232893in}}%
\pgfpathlineto{\pgfqpoint{3.513507in}{2.226170in}}%
\pgfpathlineto{\pgfqpoint{3.505448in}{2.219523in}}%
\pgfpathlineto{\pgfqpoint{3.491972in}{2.226129in}}%
\pgfpathlineto{\pgfqpoint{3.478499in}{2.232862in}}%
\pgfpathlineto{\pgfqpoint{3.465029in}{2.239723in}}%
\pgfpathlineto{\pgfqpoint{3.451562in}{2.246712in}}%
\pgfpathlineto{\pgfqpoint{3.459639in}{2.253105in}}%
\pgfpathlineto{\pgfqpoint{3.467709in}{2.259579in}}%
\pgfpathlineto{\pgfqpoint{3.475772in}{2.266131in}}%
\pgfpathlineto{\pgfqpoint{3.483827in}{2.272759in}}%
\pgfpathclose%
\pgfusepath{fill}%
\end{pgfscope}%
\begin{pgfscope}%
\pgfpathrectangle{\pgfqpoint{1.150000in}{0.150000in}}{\pgfqpoint{5.700000in}{5.700000in}}%
\pgfusepath{clip}%
\pgfsetbuttcap%
\pgfsetroundjoin%
\definecolor{currentfill}{rgb}{0.199430,0.387607,0.554642}%
\pgfsetfillcolor{currentfill}%
\pgfsetfillopacity{0.700000}%
\pgfsetlinewidth{0.000000pt}%
\definecolor{currentstroke}{rgb}{0.000000,0.000000,0.000000}%
\pgfsetstrokecolor{currentstroke}%
\pgfsetdash{}{0pt}%
\pgfpathmoveto{\pgfqpoint{5.580289in}{3.005788in}}%
\pgfpathlineto{\pgfqpoint{5.594425in}{3.011074in}}%
\pgfpathlineto{\pgfqpoint{5.608575in}{3.016459in}}%
\pgfpathlineto{\pgfqpoint{5.622739in}{3.021941in}}%
\pgfpathlineto{\pgfqpoint{5.636917in}{3.027522in}}%
\pgfpathlineto{\pgfqpoint{5.629708in}{3.021725in}}%
\pgfpathlineto{\pgfqpoint{5.622492in}{3.015871in}}%
\pgfpathlineto{\pgfqpoint{5.615268in}{3.009958in}}%
\pgfpathlineto{\pgfqpoint{5.608037in}{3.003984in}}%
\pgfpathlineto{\pgfqpoint{5.593844in}{2.998242in}}%
\pgfpathlineto{\pgfqpoint{5.579664in}{2.992599in}}%
\pgfpathlineto{\pgfqpoint{5.565498in}{2.987055in}}%
\pgfpathlineto{\pgfqpoint{5.551347in}{2.981609in}}%
\pgfpathlineto{\pgfqpoint{5.558593in}{2.987736in}}%
\pgfpathlineto{\pgfqpoint{5.565832in}{2.993808in}}%
\pgfpathlineto{\pgfqpoint{5.573064in}{2.999824in}}%
\pgfpathlineto{\pgfqpoint{5.580289in}{3.005788in}}%
\pgfpathclose%
\pgfusepath{fill}%
\end{pgfscope}%
\begin{pgfscope}%
\pgfpathrectangle{\pgfqpoint{1.150000in}{0.150000in}}{\pgfqpoint{5.700000in}{5.700000in}}%
\pgfusepath{clip}%
\pgfsetbuttcap%
\pgfsetroundjoin%
\definecolor{currentfill}{rgb}{0.283229,0.120777,0.440584}%
\pgfsetfillcolor{currentfill}%
\pgfsetfillopacity{0.700000}%
\pgfsetlinewidth{0.000000pt}%
\definecolor{currentstroke}{rgb}{0.000000,0.000000,0.000000}%
\pgfsetstrokecolor{currentstroke}%
\pgfsetdash{}{0pt}%
\pgfpathmoveto{\pgfqpoint{4.469771in}{2.426592in}}%
\pgfpathlineto{\pgfqpoint{4.483441in}{2.427221in}}%
\pgfpathlineto{\pgfqpoint{4.497121in}{2.427956in}}%
\pgfpathlineto{\pgfqpoint{4.510810in}{2.428796in}}%
\pgfpathlineto{\pgfqpoint{4.524508in}{2.429740in}}%
\pgfpathlineto{\pgfqpoint{4.516826in}{2.420399in}}%
\pgfpathlineto{\pgfqpoint{4.509139in}{2.411021in}}%
\pgfpathlineto{\pgfqpoint{4.501447in}{2.401608in}}%
\pgfpathlineto{\pgfqpoint{4.493749in}{2.392159in}}%
\pgfpathlineto{\pgfqpoint{4.480043in}{2.391298in}}%
\pgfpathlineto{\pgfqpoint{4.466346in}{2.390541in}}%
\pgfpathlineto{\pgfqpoint{4.452658in}{2.389889in}}%
\pgfpathlineto{\pgfqpoint{4.438979in}{2.389343in}}%
\pgfpathlineto{\pgfqpoint{4.446685in}{2.398701in}}%
\pgfpathlineto{\pgfqpoint{4.454386in}{2.408029in}}%
\pgfpathlineto{\pgfqpoint{4.462081in}{2.417326in}}%
\pgfpathlineto{\pgfqpoint{4.469771in}{2.426592in}}%
\pgfpathclose%
\pgfusepath{fill}%
\end{pgfscope}%
\begin{pgfscope}%
\pgfpathrectangle{\pgfqpoint{1.150000in}{0.150000in}}{\pgfqpoint{5.700000in}{5.700000in}}%
\pgfusepath{clip}%
\pgfsetbuttcap%
\pgfsetroundjoin%
\definecolor{currentfill}{rgb}{0.283197,0.115680,0.436115}%
\pgfsetfillcolor{currentfill}%
\pgfsetfillopacity{0.700000}%
\pgfsetlinewidth{0.000000pt}%
\definecolor{currentstroke}{rgb}{0.000000,0.000000,0.000000}%
\pgfsetstrokecolor{currentstroke}%
\pgfsetdash{}{0pt}%
\pgfpathmoveto{\pgfqpoint{3.096034in}{2.436679in}}%
\pgfpathlineto{\pgfqpoint{3.109489in}{2.426096in}}%
\pgfpathlineto{\pgfqpoint{3.122944in}{2.415662in}}%
\pgfpathlineto{\pgfqpoint{3.136400in}{2.405377in}}%
\pgfpathlineto{\pgfqpoint{3.149855in}{2.395240in}}%
\pgfpathlineto{\pgfqpoint{3.141629in}{2.390678in}}%
\pgfpathlineto{\pgfqpoint{3.133394in}{2.386235in}}%
\pgfpathlineto{\pgfqpoint{3.125150in}{2.381913in}}%
\pgfpathlineto{\pgfqpoint{3.116897in}{2.377716in}}%
\pgfpathlineto{\pgfqpoint{3.103417in}{2.388141in}}%
\pgfpathlineto{\pgfqpoint{3.089937in}{2.398714in}}%
\pgfpathlineto{\pgfqpoint{3.076456in}{2.409436in}}%
\pgfpathlineto{\pgfqpoint{3.062975in}{2.420307in}}%
\pgfpathlineto{\pgfqpoint{3.071255in}{2.424210in}}%
\pgfpathlineto{\pgfqpoint{3.079524in}{2.428241in}}%
\pgfpathlineto{\pgfqpoint{3.087784in}{2.432398in}}%
\pgfpathlineto{\pgfqpoint{3.096034in}{2.436679in}}%
\pgfpathclose%
\pgfusepath{fill}%
\end{pgfscope}%
\begin{pgfscope}%
\pgfpathrectangle{\pgfqpoint{1.150000in}{0.150000in}}{\pgfqpoint{5.700000in}{5.700000in}}%
\pgfusepath{clip}%
\pgfsetbuttcap%
\pgfsetroundjoin%
\definecolor{currentfill}{rgb}{0.272594,0.025563,0.353093}%
\pgfsetfillcolor{currentfill}%
\pgfsetfillopacity{0.700000}%
\pgfsetlinewidth{0.000000pt}%
\definecolor{currentstroke}{rgb}{0.000000,0.000000,0.000000}%
\pgfsetstrokecolor{currentstroke}%
\pgfsetdash{}{0pt}%
\pgfpathmoveto{\pgfqpoint{3.988282in}{2.267595in}}%
\pgfpathlineto{\pgfqpoint{4.001810in}{2.265016in}}%
\pgfpathlineto{\pgfqpoint{4.015344in}{2.262551in}}%
\pgfpathlineto{\pgfqpoint{4.028885in}{2.260197in}}%
\pgfpathlineto{\pgfqpoint{4.042434in}{2.257956in}}%
\pgfpathlineto{\pgfqpoint{4.034590in}{2.249045in}}%
\pgfpathlineto{\pgfqpoint{4.026741in}{2.240145in}}%
\pgfpathlineto{\pgfqpoint{4.018886in}{2.231256in}}%
\pgfpathlineto{\pgfqpoint{4.011026in}{2.222381in}}%
\pgfpathlineto{\pgfqpoint{3.997466in}{2.224796in}}%
\pgfpathlineto{\pgfqpoint{3.983914in}{2.227322in}}%
\pgfpathlineto{\pgfqpoint{3.970368in}{2.229961in}}%
\pgfpathlineto{\pgfqpoint{3.956829in}{2.232713in}}%
\pgfpathlineto{\pgfqpoint{3.964700in}{2.241408in}}%
\pgfpathlineto{\pgfqpoint{3.972566in}{2.250121in}}%
\pgfpathlineto{\pgfqpoint{3.980427in}{2.258850in}}%
\pgfpathlineto{\pgfqpoint{3.988282in}{2.267595in}}%
\pgfpathclose%
\pgfusepath{fill}%
\end{pgfscope}%
\begin{pgfscope}%
\pgfpathrectangle{\pgfqpoint{1.150000in}{0.150000in}}{\pgfqpoint{5.700000in}{5.700000in}}%
\pgfusepath{clip}%
\pgfsetbuttcap%
\pgfsetroundjoin%
\definecolor{currentfill}{rgb}{0.269944,0.014625,0.341379}%
\pgfsetfillcolor{currentfill}%
\pgfsetfillopacity{0.700000}%
\pgfsetlinewidth{0.000000pt}%
\definecolor{currentstroke}{rgb}{0.000000,0.000000,0.000000}%
\pgfsetstrokecolor{currentstroke}%
\pgfsetdash{}{0pt}%
\pgfpathmoveto{\pgfqpoint{3.763128in}{2.242416in}}%
\pgfpathlineto{\pgfqpoint{3.776611in}{2.238085in}}%
\pgfpathlineto{\pgfqpoint{3.790100in}{2.233872in}}%
\pgfpathlineto{\pgfqpoint{3.803593in}{2.229777in}}%
\pgfpathlineto{\pgfqpoint{3.817092in}{2.225800in}}%
\pgfpathlineto{\pgfqpoint{3.809167in}{2.217624in}}%
\pgfpathlineto{\pgfqpoint{3.801235in}{2.209485in}}%
\pgfpathlineto{\pgfqpoint{3.793298in}{2.201385in}}%
\pgfpathlineto{\pgfqpoint{3.785355in}{2.193325in}}%
\pgfpathlineto{\pgfqpoint{3.771843in}{2.197513in}}%
\pgfpathlineto{\pgfqpoint{3.758335in}{2.201817in}}%
\pgfpathlineto{\pgfqpoint{3.744833in}{2.206239in}}%
\pgfpathlineto{\pgfqpoint{3.731336in}{2.210780in}}%
\pgfpathlineto{\pgfqpoint{3.739293in}{2.218623in}}%
\pgfpathlineto{\pgfqpoint{3.747244in}{2.226511in}}%
\pgfpathlineto{\pgfqpoint{3.755190in}{2.234443in}}%
\pgfpathlineto{\pgfqpoint{3.763128in}{2.242416in}}%
\pgfpathclose%
\pgfusepath{fill}%
\end{pgfscope}%
\begin{pgfscope}%
\pgfpathrectangle{\pgfqpoint{1.150000in}{0.150000in}}{\pgfqpoint{5.700000in}{5.700000in}}%
\pgfusepath{clip}%
\pgfsetbuttcap%
\pgfsetroundjoin%
\definecolor{currentfill}{rgb}{0.282656,0.100196,0.422160}%
\pgfsetfillcolor{currentfill}%
\pgfsetfillopacity{0.700000}%
\pgfsetlinewidth{0.000000pt}%
\definecolor{currentstroke}{rgb}{0.000000,0.000000,0.000000}%
\pgfsetstrokecolor{currentstroke}%
\pgfsetdash{}{0pt}%
\pgfpathmoveto{\pgfqpoint{4.384355in}{2.388215in}}%
\pgfpathlineto{\pgfqpoint{4.397997in}{2.388338in}}%
\pgfpathlineto{\pgfqpoint{4.411649in}{2.388567in}}%
\pgfpathlineto{\pgfqpoint{4.425310in}{2.388902in}}%
\pgfpathlineto{\pgfqpoint{4.438979in}{2.389343in}}%
\pgfpathlineto{\pgfqpoint{4.431268in}{2.379954in}}%
\pgfpathlineto{\pgfqpoint{4.423551in}{2.370537in}}%
\pgfpathlineto{\pgfqpoint{4.415829in}{2.361091in}}%
\pgfpathlineto{\pgfqpoint{4.408102in}{2.351617in}}%
\pgfpathlineto{\pgfqpoint{4.394424in}{2.351277in}}%
\pgfpathlineto{\pgfqpoint{4.380755in}{2.351043in}}%
\pgfpathlineto{\pgfqpoint{4.367094in}{2.350915in}}%
\pgfpathlineto{\pgfqpoint{4.353443in}{2.350893in}}%
\pgfpathlineto{\pgfqpoint{4.361179in}{2.360259in}}%
\pgfpathlineto{\pgfqpoint{4.368910in}{2.369602in}}%
\pgfpathlineto{\pgfqpoint{4.376635in}{2.378921in}}%
\pgfpathlineto{\pgfqpoint{4.384355in}{2.388215in}}%
\pgfpathclose%
\pgfusepath{fill}%
\end{pgfscope}%
\begin{pgfscope}%
\pgfpathrectangle{\pgfqpoint{1.150000in}{0.150000in}}{\pgfqpoint{5.700000in}{5.700000in}}%
\pgfusepath{clip}%
\pgfsetbuttcap%
\pgfsetroundjoin%
\definecolor{currentfill}{rgb}{0.280894,0.078907,0.402329}%
\pgfsetfillcolor{currentfill}%
\pgfsetfillopacity{0.700000}%
\pgfsetlinewidth{0.000000pt}%
\definecolor{currentstroke}{rgb}{0.000000,0.000000,0.000000}%
\pgfsetstrokecolor{currentstroke}%
\pgfsetdash{}{0pt}%
\pgfpathmoveto{\pgfqpoint{4.298923in}{2.351874in}}%
\pgfpathlineto{\pgfqpoint{4.312540in}{2.351468in}}%
\pgfpathlineto{\pgfqpoint{4.326166in}{2.351170in}}%
\pgfpathlineto{\pgfqpoint{4.339800in}{2.350978in}}%
\pgfpathlineto{\pgfqpoint{4.353443in}{2.350893in}}%
\pgfpathlineto{\pgfqpoint{4.345702in}{2.341505in}}%
\pgfpathlineto{\pgfqpoint{4.337955in}{2.332095in}}%
\pgfpathlineto{\pgfqpoint{4.330204in}{2.322665in}}%
\pgfpathlineto{\pgfqpoint{4.322447in}{2.313215in}}%
\pgfpathlineto{\pgfqpoint{4.308795in}{2.313419in}}%
\pgfpathlineto{\pgfqpoint{4.295152in}{2.313730in}}%
\pgfpathlineto{\pgfqpoint{4.281517in}{2.314147in}}%
\pgfpathlineto{\pgfqpoint{4.267891in}{2.314672in}}%
\pgfpathlineto{\pgfqpoint{4.275657in}{2.323997in}}%
\pgfpathlineto{\pgfqpoint{4.283418in}{2.333306in}}%
\pgfpathlineto{\pgfqpoint{4.291173in}{2.342599in}}%
\pgfpathlineto{\pgfqpoint{4.298923in}{2.351874in}}%
\pgfpathclose%
\pgfusepath{fill}%
\end{pgfscope}%
\begin{pgfscope}%
\pgfpathrectangle{\pgfqpoint{1.150000in}{0.150000in}}{\pgfqpoint{5.700000in}{5.700000in}}%
\pgfusepath{clip}%
\pgfsetbuttcap%
\pgfsetroundjoin%
\definecolor{currentfill}{rgb}{0.277018,0.050344,0.375715}%
\pgfsetfillcolor{currentfill}%
\pgfsetfillopacity{0.700000}%
\pgfsetlinewidth{0.000000pt}%
\definecolor{currentstroke}{rgb}{0.000000,0.000000,0.000000}%
\pgfsetstrokecolor{currentstroke}%
\pgfsetdash{}{0pt}%
\pgfpathmoveto{\pgfqpoint{3.343919in}{2.307328in}}%
\pgfpathlineto{\pgfqpoint{3.357366in}{2.299286in}}%
\pgfpathlineto{\pgfqpoint{3.370815in}{2.291379in}}%
\pgfpathlineto{\pgfqpoint{3.384267in}{2.283605in}}%
\pgfpathlineto{\pgfqpoint{3.397721in}{2.275964in}}%
\pgfpathlineto{\pgfqpoint{3.389617in}{2.269912in}}%
\pgfpathlineto{\pgfqpoint{3.381505in}{2.263950in}}%
\pgfpathlineto{\pgfqpoint{3.373385in}{2.258080in}}%
\pgfpathlineto{\pgfqpoint{3.365258in}{2.252305in}}%
\pgfpathlineto{\pgfqpoint{3.351783in}{2.260212in}}%
\pgfpathlineto{\pgfqpoint{3.338311in}{2.268252in}}%
\pgfpathlineto{\pgfqpoint{3.324841in}{2.276426in}}%
\pgfpathlineto{\pgfqpoint{3.311373in}{2.284734in}}%
\pgfpathlineto{\pgfqpoint{3.319522in}{2.290236in}}%
\pgfpathlineto{\pgfqpoint{3.327662in}{2.295837in}}%
\pgfpathlineto{\pgfqpoint{3.335794in}{2.301535in}}%
\pgfpathlineto{\pgfqpoint{3.343919in}{2.307328in}}%
\pgfpathclose%
\pgfusepath{fill}%
\end{pgfscope}%
\begin{pgfscope}%
\pgfpathrectangle{\pgfqpoint{1.150000in}{0.150000in}}{\pgfqpoint{5.700000in}{5.700000in}}%
\pgfusepath{clip}%
\pgfsetbuttcap%
\pgfsetroundjoin%
\definecolor{currentfill}{rgb}{0.244972,0.287675,0.537260}%
\pgfsetfillcolor{currentfill}%
\pgfsetfillopacity{0.700000}%
\pgfsetlinewidth{0.000000pt}%
\definecolor{currentstroke}{rgb}{0.000000,0.000000,0.000000}%
\pgfsetstrokecolor{currentstroke}%
\pgfsetdash{}{0pt}%
\pgfpathmoveto{\pgfqpoint{2.684759in}{2.790588in}}%
\pgfpathlineto{\pgfqpoint{2.698312in}{2.774997in}}%
\pgfpathlineto{\pgfqpoint{2.711860in}{2.759595in}}%
\pgfpathlineto{\pgfqpoint{2.725403in}{2.744380in}}%
\pgfpathlineto{\pgfqpoint{2.738942in}{2.729350in}}%
\pgfpathlineto{\pgfqpoint{2.730477in}{2.727407in}}%
\pgfpathlineto{\pgfqpoint{2.722000in}{2.725628in}}%
\pgfpathlineto{\pgfqpoint{2.713510in}{2.724016in}}%
\pgfpathlineto{\pgfqpoint{2.705007in}{2.722573in}}%
\pgfpathlineto{\pgfqpoint{2.691435in}{2.737920in}}%
\pgfpathlineto{\pgfqpoint{2.677859in}{2.753452in}}%
\pgfpathlineto{\pgfqpoint{2.664277in}{2.769173in}}%
\pgfpathlineto{\pgfqpoint{2.650690in}{2.785083in}}%
\pgfpathlineto{\pgfqpoint{2.659227in}{2.786201in}}%
\pgfpathlineto{\pgfqpoint{2.667750in}{2.787493in}}%
\pgfpathlineto{\pgfqpoint{2.676261in}{2.788956in}}%
\pgfpathlineto{\pgfqpoint{2.684759in}{2.790588in}}%
\pgfpathclose%
\pgfusepath{fill}%
\end{pgfscope}%
\begin{pgfscope}%
\pgfpathrectangle{\pgfqpoint{1.150000in}{0.150000in}}{\pgfqpoint{5.700000in}{5.700000in}}%
\pgfusepath{clip}%
\pgfsetbuttcap%
\pgfsetroundjoin%
\definecolor{currentfill}{rgb}{0.233603,0.313828,0.543914}%
\pgfsetfillcolor{currentfill}%
\pgfsetfillopacity{0.700000}%
\pgfsetlinewidth{0.000000pt}%
\definecolor{currentstroke}{rgb}{0.000000,0.000000,0.000000}%
\pgfsetstrokecolor{currentstroke}%
\pgfsetdash{}{0pt}%
\pgfpathmoveto{\pgfqpoint{2.630495in}{2.854878in}}%
\pgfpathlineto{\pgfqpoint{2.644069in}{2.838513in}}%
\pgfpathlineto{\pgfqpoint{2.657638in}{2.822344in}}%
\pgfpathlineto{\pgfqpoint{2.671201in}{2.806370in}}%
\pgfpathlineto{\pgfqpoint{2.684759in}{2.790588in}}%
\pgfpathlineto{\pgfqpoint{2.676261in}{2.788956in}}%
\pgfpathlineto{\pgfqpoint{2.667750in}{2.787493in}}%
\pgfpathlineto{\pgfqpoint{2.659227in}{2.786201in}}%
\pgfpathlineto{\pgfqpoint{2.650690in}{2.785083in}}%
\pgfpathlineto{\pgfqpoint{2.637098in}{2.801184in}}%
\pgfpathlineto{\pgfqpoint{2.623500in}{2.817478in}}%
\pgfpathlineto{\pgfqpoint{2.609897in}{2.833967in}}%
\pgfpathlineto{\pgfqpoint{2.596288in}{2.850653in}}%
\pgfpathlineto{\pgfqpoint{2.604860in}{2.851443in}}%
\pgfpathlineto{\pgfqpoint{2.613418in}{2.852413in}}%
\pgfpathlineto{\pgfqpoint{2.621963in}{2.853559in}}%
\pgfpathlineto{\pgfqpoint{2.630495in}{2.854878in}}%
\pgfpathclose%
\pgfusepath{fill}%
\end{pgfscope}%
\begin{pgfscope}%
\pgfpathrectangle{\pgfqpoint{1.150000in}{0.150000in}}{\pgfqpoint{5.700000in}{5.700000in}}%
\pgfusepath{clip}%
\pgfsetbuttcap%
\pgfsetroundjoin%
\definecolor{currentfill}{rgb}{0.255645,0.260703,0.528312}%
\pgfsetfillcolor{currentfill}%
\pgfsetfillopacity{0.700000}%
\pgfsetlinewidth{0.000000pt}%
\definecolor{currentstroke}{rgb}{0.000000,0.000000,0.000000}%
\pgfsetstrokecolor{currentstroke}%
\pgfsetdash{}{0pt}%
\pgfpathmoveto{\pgfqpoint{2.738942in}{2.729350in}}%
\pgfpathlineto{\pgfqpoint{2.752477in}{2.714504in}}%
\pgfpathlineto{\pgfqpoint{2.766007in}{2.699840in}}%
\pgfpathlineto{\pgfqpoint{2.779534in}{2.685357in}}%
\pgfpathlineto{\pgfqpoint{2.793057in}{2.671053in}}%
\pgfpathlineto{\pgfqpoint{2.784623in}{2.668802in}}%
\pgfpathlineto{\pgfqpoint{2.776178in}{2.666709in}}%
\pgfpathlineto{\pgfqpoint{2.767721in}{2.664778in}}%
\pgfpathlineto{\pgfqpoint{2.759252in}{2.663011in}}%
\pgfpathlineto{\pgfqpoint{2.745697in}{2.677631in}}%
\pgfpathlineto{\pgfqpoint{2.732138in}{2.692430in}}%
\pgfpathlineto{\pgfqpoint{2.718575in}{2.707410in}}%
\pgfpathlineto{\pgfqpoint{2.705007in}{2.722573in}}%
\pgfpathlineto{\pgfqpoint{2.713510in}{2.724016in}}%
\pgfpathlineto{\pgfqpoint{2.722000in}{2.725628in}}%
\pgfpathlineto{\pgfqpoint{2.730477in}{2.727407in}}%
\pgfpathlineto{\pgfqpoint{2.738942in}{2.729350in}}%
\pgfpathclose%
\pgfusepath{fill}%
\end{pgfscope}%
\begin{pgfscope}%
\pgfpathrectangle{\pgfqpoint{1.150000in}{0.150000in}}{\pgfqpoint{5.700000in}{5.700000in}}%
\pgfusepath{clip}%
\pgfsetbuttcap%
\pgfsetroundjoin%
\definecolor{currentfill}{rgb}{0.282656,0.100196,0.422160}%
\pgfsetfillcolor{currentfill}%
\pgfsetfillopacity{0.700000}%
\pgfsetlinewidth{0.000000pt}%
\definecolor{currentstroke}{rgb}{0.000000,0.000000,0.000000}%
\pgfsetstrokecolor{currentstroke}%
\pgfsetdash{}{0pt}%
\pgfpathmoveto{\pgfqpoint{3.149855in}{2.395240in}}%
\pgfpathlineto{\pgfqpoint{3.163310in}{2.385249in}}%
\pgfpathlineto{\pgfqpoint{3.176766in}{2.375403in}}%
\pgfpathlineto{\pgfqpoint{3.190223in}{2.365702in}}%
\pgfpathlineto{\pgfqpoint{3.203680in}{2.356145in}}%
\pgfpathlineto{\pgfqpoint{3.195478in}{2.351303in}}%
\pgfpathlineto{\pgfqpoint{3.187267in}{2.346575in}}%
\pgfpathlineto{\pgfqpoint{3.179047in}{2.341963in}}%
\pgfpathlineto{\pgfqpoint{3.170818in}{2.337471in}}%
\pgfpathlineto{\pgfqpoint{3.157337in}{2.347316in}}%
\pgfpathlineto{\pgfqpoint{3.143857in}{2.357304in}}%
\pgfpathlineto{\pgfqpoint{3.130377in}{2.367437in}}%
\pgfpathlineto{\pgfqpoint{3.116897in}{2.377716in}}%
\pgfpathlineto{\pgfqpoint{3.125150in}{2.381913in}}%
\pgfpathlineto{\pgfqpoint{3.133394in}{2.386235in}}%
\pgfpathlineto{\pgfqpoint{3.141629in}{2.390678in}}%
\pgfpathlineto{\pgfqpoint{3.149855in}{2.395240in}}%
\pgfpathclose%
\pgfusepath{fill}%
\end{pgfscope}%
\begin{pgfscope}%
\pgfpathrectangle{\pgfqpoint{1.150000in}{0.150000in}}{\pgfqpoint{5.700000in}{5.700000in}}%
\pgfusepath{clip}%
\pgfsetbuttcap%
\pgfsetroundjoin%
\definecolor{currentfill}{rgb}{0.271305,0.019942,0.347269}%
\pgfsetfillcolor{currentfill}%
\pgfsetfillopacity{0.700000}%
\pgfsetlinewidth{0.000000pt}%
\definecolor{currentstroke}{rgb}{0.000000,0.000000,0.000000}%
\pgfsetstrokecolor{currentstroke}%
\pgfsetdash{}{0pt}%
\pgfpathmoveto{\pgfqpoint{3.902733in}{2.244856in}}%
\pgfpathlineto{\pgfqpoint{3.916248in}{2.241649in}}%
\pgfpathlineto{\pgfqpoint{3.929769in}{2.238557in}}%
\pgfpathlineto{\pgfqpoint{3.943296in}{2.235578in}}%
\pgfpathlineto{\pgfqpoint{3.956829in}{2.232713in}}%
\pgfpathlineto{\pgfqpoint{3.948951in}{2.224038in}}%
\pgfpathlineto{\pgfqpoint{3.941069in}{2.215383in}}%
\pgfpathlineto{\pgfqpoint{3.933180in}{2.206751in}}%
\pgfpathlineto{\pgfqpoint{3.925286in}{2.198144in}}%
\pgfpathlineto{\pgfqpoint{3.911741in}{2.201200in}}%
\pgfpathlineto{\pgfqpoint{3.898202in}{2.204370in}}%
\pgfpathlineto{\pgfqpoint{3.884669in}{2.207654in}}%
\pgfpathlineto{\pgfqpoint{3.871142in}{2.211052in}}%
\pgfpathlineto{\pgfqpoint{3.879049in}{2.219461in}}%
\pgfpathlineto{\pgfqpoint{3.886949in}{2.227900in}}%
\pgfpathlineto{\pgfqpoint{3.894844in}{2.236365in}}%
\pgfpathlineto{\pgfqpoint{3.902733in}{2.244856in}}%
\pgfpathclose%
\pgfusepath{fill}%
\end{pgfscope}%
\begin{pgfscope}%
\pgfpathrectangle{\pgfqpoint{1.150000in}{0.150000in}}{\pgfqpoint{5.700000in}{5.700000in}}%
\pgfusepath{clip}%
\pgfsetbuttcap%
\pgfsetroundjoin%
\definecolor{currentfill}{rgb}{0.278791,0.062145,0.386592}%
\pgfsetfillcolor{currentfill}%
\pgfsetfillopacity{0.700000}%
\pgfsetlinewidth{0.000000pt}%
\definecolor{currentstroke}{rgb}{0.000000,0.000000,0.000000}%
\pgfsetstrokecolor{currentstroke}%
\pgfsetdash{}{0pt}%
\pgfpathmoveto{\pgfqpoint{4.213467in}{2.317852in}}%
\pgfpathlineto{\pgfqpoint{4.227061in}{2.316895in}}%
\pgfpathlineto{\pgfqpoint{4.240663in}{2.316046in}}%
\pgfpathlineto{\pgfqpoint{4.254273in}{2.315305in}}%
\pgfpathlineto{\pgfqpoint{4.267891in}{2.314672in}}%
\pgfpathlineto{\pgfqpoint{4.260120in}{2.305334in}}%
\pgfpathlineto{\pgfqpoint{4.252343in}{2.295983in}}%
\pgfpathlineto{\pgfqpoint{4.244561in}{2.286619in}}%
\pgfpathlineto{\pgfqpoint{4.236774in}{2.277245in}}%
\pgfpathlineto{\pgfqpoint{4.223147in}{2.278015in}}%
\pgfpathlineto{\pgfqpoint{4.209528in}{2.278893in}}%
\pgfpathlineto{\pgfqpoint{4.195916in}{2.279879in}}%
\pgfpathlineto{\pgfqpoint{4.182313in}{2.280973in}}%
\pgfpathlineto{\pgfqpoint{4.190109in}{2.290204in}}%
\pgfpathlineto{\pgfqpoint{4.197900in}{2.299428in}}%
\pgfpathlineto{\pgfqpoint{4.205686in}{2.308644in}}%
\pgfpathlineto{\pgfqpoint{4.213467in}{2.317852in}}%
\pgfpathclose%
\pgfusepath{fill}%
\end{pgfscope}%
\begin{pgfscope}%
\pgfpathrectangle{\pgfqpoint{1.150000in}{0.150000in}}{\pgfqpoint{5.700000in}{5.700000in}}%
\pgfusepath{clip}%
\pgfsetbuttcap%
\pgfsetroundjoin%
\definecolor{currentfill}{rgb}{0.265145,0.232956,0.516599}%
\pgfsetfillcolor{currentfill}%
\pgfsetfillopacity{0.700000}%
\pgfsetlinewidth{0.000000pt}%
\definecolor{currentstroke}{rgb}{0.000000,0.000000,0.000000}%
\pgfsetstrokecolor{currentstroke}%
\pgfsetdash{}{0pt}%
\pgfpathmoveto{\pgfqpoint{2.793057in}{2.671053in}}%
\pgfpathlineto{\pgfqpoint{2.806576in}{2.656926in}}%
\pgfpathlineto{\pgfqpoint{2.820092in}{2.642975in}}%
\pgfpathlineto{\pgfqpoint{2.833604in}{2.629198in}}%
\pgfpathlineto{\pgfqpoint{2.847114in}{2.615594in}}%
\pgfpathlineto{\pgfqpoint{2.838711in}{2.613036in}}%
\pgfpathlineto{\pgfqpoint{2.830298in}{2.610632in}}%
\pgfpathlineto{\pgfqpoint{2.821872in}{2.608384in}}%
\pgfpathlineto{\pgfqpoint{2.813435in}{2.606296in}}%
\pgfpathlineto{\pgfqpoint{2.799894in}{2.620213in}}%
\pgfpathlineto{\pgfqpoint{2.786350in}{2.634304in}}%
\pgfpathlineto{\pgfqpoint{2.772803in}{2.648570in}}%
\pgfpathlineto{\pgfqpoint{2.759252in}{2.663011in}}%
\pgfpathlineto{\pgfqpoint{2.767721in}{2.664778in}}%
\pgfpathlineto{\pgfqpoint{2.776178in}{2.666709in}}%
\pgfpathlineto{\pgfqpoint{2.784623in}{2.668802in}}%
\pgfpathlineto{\pgfqpoint{2.793057in}{2.671053in}}%
\pgfpathclose%
\pgfusepath{fill}%
\end{pgfscope}%
\begin{pgfscope}%
\pgfpathrectangle{\pgfqpoint{1.150000in}{0.150000in}}{\pgfqpoint{5.700000in}{5.700000in}}%
\pgfusepath{clip}%
\pgfsetbuttcap%
\pgfsetroundjoin%
\definecolor{currentfill}{rgb}{0.271828,0.209303,0.504434}%
\pgfsetfillcolor{currentfill}%
\pgfsetfillopacity{0.700000}%
\pgfsetlinewidth{0.000000pt}%
\definecolor{currentstroke}{rgb}{0.000000,0.000000,0.000000}%
\pgfsetstrokecolor{currentstroke}%
\pgfsetdash{}{0pt}%
\pgfpathmoveto{\pgfqpoint{2.847114in}{2.615594in}}%
\pgfpathlineto{\pgfqpoint{2.860620in}{2.602162in}}%
\pgfpathlineto{\pgfqpoint{2.874124in}{2.588899in}}%
\pgfpathlineto{\pgfqpoint{2.887626in}{2.575805in}}%
\pgfpathlineto{\pgfqpoint{2.901124in}{2.562877in}}%
\pgfpathlineto{\pgfqpoint{2.892752in}{2.560014in}}%
\pgfpathlineto{\pgfqpoint{2.884368in}{2.557299in}}%
\pgfpathlineto{\pgfqpoint{2.875973in}{2.554736in}}%
\pgfpathlineto{\pgfqpoint{2.867567in}{2.552328in}}%
\pgfpathlineto{\pgfqpoint{2.854038in}{2.565568in}}%
\pgfpathlineto{\pgfqpoint{2.840506in}{2.578974in}}%
\pgfpathlineto{\pgfqpoint{2.826972in}{2.592550in}}%
\pgfpathlineto{\pgfqpoint{2.813435in}{2.606296in}}%
\pgfpathlineto{\pgfqpoint{2.821872in}{2.608384in}}%
\pgfpathlineto{\pgfqpoint{2.830298in}{2.610632in}}%
\pgfpathlineto{\pgfqpoint{2.838711in}{2.613036in}}%
\pgfpathlineto{\pgfqpoint{2.847114in}{2.615594in}}%
\pgfpathclose%
\pgfusepath{fill}%
\end{pgfscope}%
\begin{pgfscope}%
\pgfpathrectangle{\pgfqpoint{1.150000in}{0.150000in}}{\pgfqpoint{5.700000in}{5.700000in}}%
\pgfusepath{clip}%
\pgfsetbuttcap%
\pgfsetroundjoin%
\definecolor{currentfill}{rgb}{0.271305,0.019942,0.347269}%
\pgfsetfillcolor{currentfill}%
\pgfsetfillopacity{0.700000}%
\pgfsetlinewidth{0.000000pt}%
\definecolor{currentstroke}{rgb}{0.000000,0.000000,0.000000}%
\pgfsetstrokecolor{currentstroke}%
\pgfsetdash{}{0pt}%
\pgfpathmoveto{\pgfqpoint{3.537641in}{2.246558in}}%
\pgfpathlineto{\pgfqpoint{3.551103in}{2.240325in}}%
\pgfpathlineto{\pgfqpoint{3.564569in}{2.234217in}}%
\pgfpathlineto{\pgfqpoint{3.578038in}{2.228234in}}%
\pgfpathlineto{\pgfqpoint{3.591511in}{2.222375in}}%
\pgfpathlineto{\pgfqpoint{3.583490in}{2.215267in}}%
\pgfpathlineto{\pgfqpoint{3.575463in}{2.208226in}}%
\pgfpathlineto{\pgfqpoint{3.567428in}{2.201254in}}%
\pgfpathlineto{\pgfqpoint{3.559387in}{2.194354in}}%
\pgfpathlineto{\pgfqpoint{3.545897in}{2.200459in}}%
\pgfpathlineto{\pgfqpoint{3.532411in}{2.206688in}}%
\pgfpathlineto{\pgfqpoint{3.518928in}{2.213043in}}%
\pgfpathlineto{\pgfqpoint{3.505448in}{2.219523in}}%
\pgfpathlineto{\pgfqpoint{3.513507in}{2.226170in}}%
\pgfpathlineto{\pgfqpoint{3.521559in}{2.232893in}}%
\pgfpathlineto{\pgfqpoint{3.529603in}{2.239690in}}%
\pgfpathlineto{\pgfqpoint{3.537641in}{2.246558in}}%
\pgfpathclose%
\pgfusepath{fill}%
\end{pgfscope}%
\begin{pgfscope}%
\pgfpathrectangle{\pgfqpoint{1.150000in}{0.150000in}}{\pgfqpoint{5.700000in}{5.700000in}}%
\pgfusepath{clip}%
\pgfsetbuttcap%
\pgfsetroundjoin%
\definecolor{currentfill}{rgb}{0.268510,0.009605,0.335427}%
\pgfsetfillcolor{currentfill}%
\pgfsetfillopacity{0.700000}%
\pgfsetlinewidth{0.000000pt}%
\definecolor{currentstroke}{rgb}{0.000000,0.000000,0.000000}%
\pgfsetstrokecolor{currentstroke}%
\pgfsetdash{}{0pt}%
\pgfpathmoveto{\pgfqpoint{3.677395in}{2.230137in}}%
\pgfpathlineto{\pgfqpoint{3.690874in}{2.225117in}}%
\pgfpathlineto{\pgfqpoint{3.704356in}{2.220218in}}%
\pgfpathlineto{\pgfqpoint{3.717844in}{2.215439in}}%
\pgfpathlineto{\pgfqpoint{3.731336in}{2.210780in}}%
\pgfpathlineto{\pgfqpoint{3.723373in}{2.202984in}}%
\pgfpathlineto{\pgfqpoint{3.715403in}{2.195238in}}%
\pgfpathlineto{\pgfqpoint{3.707427in}{2.187542in}}%
\pgfpathlineto{\pgfqpoint{3.699444in}{2.179901in}}%
\pgfpathlineto{\pgfqpoint{3.685937in}{2.184788in}}%
\pgfpathlineto{\pgfqpoint{3.672435in}{2.189795in}}%
\pgfpathlineto{\pgfqpoint{3.658937in}{2.194921in}}%
\pgfpathlineto{\pgfqpoint{3.645443in}{2.200169in}}%
\pgfpathlineto{\pgfqpoint{3.653441in}{2.207576in}}%
\pgfpathlineto{\pgfqpoint{3.661432in}{2.215041in}}%
\pgfpathlineto{\pgfqpoint{3.669417in}{2.222562in}}%
\pgfpathlineto{\pgfqpoint{3.677395in}{2.230137in}}%
\pgfpathclose%
\pgfusepath{fill}%
\end{pgfscope}%
\begin{pgfscope}%
\pgfpathrectangle{\pgfqpoint{1.150000in}{0.150000in}}{\pgfqpoint{5.700000in}{5.700000in}}%
\pgfusepath{clip}%
\pgfsetbuttcap%
\pgfsetroundjoin%
\definecolor{currentfill}{rgb}{0.277018,0.050344,0.375715}%
\pgfsetfillcolor{currentfill}%
\pgfsetfillopacity{0.700000}%
\pgfsetlinewidth{0.000000pt}%
\definecolor{currentstroke}{rgb}{0.000000,0.000000,0.000000}%
\pgfsetstrokecolor{currentstroke}%
\pgfsetdash{}{0pt}%
\pgfpathmoveto{\pgfqpoint{4.127975in}{2.286443in}}%
\pgfpathlineto{\pgfqpoint{4.141548in}{2.284911in}}%
\pgfpathlineto{\pgfqpoint{4.155128in}{2.283489in}}%
\pgfpathlineto{\pgfqpoint{4.168717in}{2.282177in}}%
\pgfpathlineto{\pgfqpoint{4.182313in}{2.280973in}}%
\pgfpathlineto{\pgfqpoint{4.174511in}{2.271738in}}%
\pgfpathlineto{\pgfqpoint{4.166704in}{2.262498in}}%
\pgfpathlineto{\pgfqpoint{4.158891in}{2.253256in}}%
\pgfpathlineto{\pgfqpoint{4.151073in}{2.244012in}}%
\pgfpathlineto{\pgfqpoint{4.137468in}{2.245371in}}%
\pgfpathlineto{\pgfqpoint{4.123870in}{2.246838in}}%
\pgfpathlineto{\pgfqpoint{4.110279in}{2.248416in}}%
\pgfpathlineto{\pgfqpoint{4.096696in}{2.250103in}}%
\pgfpathlineto{\pgfqpoint{4.104523in}{2.259184in}}%
\pgfpathlineto{\pgfqpoint{4.112346in}{2.268269in}}%
\pgfpathlineto{\pgfqpoint{4.120163in}{2.277356in}}%
\pgfpathlineto{\pgfqpoint{4.127975in}{2.286443in}}%
\pgfpathclose%
\pgfusepath{fill}%
\end{pgfscope}%
\begin{pgfscope}%
\pgfpathrectangle{\pgfqpoint{1.150000in}{0.150000in}}{\pgfqpoint{5.700000in}{5.700000in}}%
\pgfusepath{clip}%
\pgfsetbuttcap%
\pgfsetroundjoin%
\definecolor{currentfill}{rgb}{0.277134,0.185228,0.489898}%
\pgfsetfillcolor{currentfill}%
\pgfsetfillopacity{0.700000}%
\pgfsetlinewidth{0.000000pt}%
\definecolor{currentstroke}{rgb}{0.000000,0.000000,0.000000}%
\pgfsetstrokecolor{currentstroke}%
\pgfsetdash{}{0pt}%
\pgfpathmoveto{\pgfqpoint{2.901124in}{2.562877in}}%
\pgfpathlineto{\pgfqpoint{2.914621in}{2.550116in}}%
\pgfpathlineto{\pgfqpoint{2.928115in}{2.537519in}}%
\pgfpathlineto{\pgfqpoint{2.941608in}{2.525085in}}%
\pgfpathlineto{\pgfqpoint{2.955098in}{2.512812in}}%
\pgfpathlineto{\pgfqpoint{2.946755in}{2.509645in}}%
\pgfpathlineto{\pgfqpoint{2.938400in}{2.506622in}}%
\pgfpathlineto{\pgfqpoint{2.930035in}{2.503745in}}%
\pgfpathlineto{\pgfqpoint{2.921658in}{2.501019in}}%
\pgfpathlineto{\pgfqpoint{2.908139in}{2.513602in}}%
\pgfpathlineto{\pgfqpoint{2.894617in}{2.526347in}}%
\pgfpathlineto{\pgfqpoint{2.881093in}{2.539255in}}%
\pgfpathlineto{\pgfqpoint{2.867567in}{2.552328in}}%
\pgfpathlineto{\pgfqpoint{2.875973in}{2.554736in}}%
\pgfpathlineto{\pgfqpoint{2.884368in}{2.557299in}}%
\pgfpathlineto{\pgfqpoint{2.892752in}{2.560014in}}%
\pgfpathlineto{\pgfqpoint{2.901124in}{2.562877in}}%
\pgfpathclose%
\pgfusepath{fill}%
\end{pgfscope}%
\begin{pgfscope}%
\pgfpathrectangle{\pgfqpoint{1.150000in}{0.150000in}}{\pgfqpoint{5.700000in}{5.700000in}}%
\pgfusepath{clip}%
\pgfsetbuttcap%
\pgfsetroundjoin%
\definecolor{currentfill}{rgb}{0.281446,0.084320,0.407414}%
\pgfsetfillcolor{currentfill}%
\pgfsetfillopacity{0.700000}%
\pgfsetlinewidth{0.000000pt}%
\definecolor{currentstroke}{rgb}{0.000000,0.000000,0.000000}%
\pgfsetstrokecolor{currentstroke}%
\pgfsetdash{}{0pt}%
\pgfpathmoveto{\pgfqpoint{3.203680in}{2.356145in}}%
\pgfpathlineto{\pgfqpoint{3.217137in}{2.346729in}}%
\pgfpathlineto{\pgfqpoint{3.230596in}{2.337456in}}%
\pgfpathlineto{\pgfqpoint{3.244056in}{2.328323in}}%
\pgfpathlineto{\pgfqpoint{3.257516in}{2.319329in}}%
\pgfpathlineto{\pgfqpoint{3.249337in}{2.314208in}}%
\pgfpathlineto{\pgfqpoint{3.241150in}{2.309196in}}%
\pgfpathlineto{\pgfqpoint{3.232954in}{2.304296in}}%
\pgfpathlineto{\pgfqpoint{3.224748in}{2.299510in}}%
\pgfpathlineto{\pgfqpoint{3.211265in}{2.308790in}}%
\pgfpathlineto{\pgfqpoint{3.197782in}{2.318209in}}%
\pgfpathlineto{\pgfqpoint{3.184300in}{2.327769in}}%
\pgfpathlineto{\pgfqpoint{3.170818in}{2.337471in}}%
\pgfpathlineto{\pgfqpoint{3.179047in}{2.341963in}}%
\pgfpathlineto{\pgfqpoint{3.187267in}{2.346575in}}%
\pgfpathlineto{\pgfqpoint{3.195478in}{2.351303in}}%
\pgfpathlineto{\pgfqpoint{3.203680in}{2.356145in}}%
\pgfpathclose%
\pgfusepath{fill}%
\end{pgfscope}%
\begin{pgfscope}%
\pgfpathrectangle{\pgfqpoint{1.150000in}{0.150000in}}{\pgfqpoint{5.700000in}{5.700000in}}%
\pgfusepath{clip}%
\pgfsetbuttcap%
\pgfsetroundjoin%
\definecolor{currentfill}{rgb}{0.274952,0.037752,0.364543}%
\pgfsetfillcolor{currentfill}%
\pgfsetfillopacity{0.700000}%
\pgfsetlinewidth{0.000000pt}%
\definecolor{currentstroke}{rgb}{0.000000,0.000000,0.000000}%
\pgfsetstrokecolor{currentstroke}%
\pgfsetdash{}{0pt}%
\pgfpathmoveto{\pgfqpoint{3.397721in}{2.275964in}}%
\pgfpathlineto{\pgfqpoint{3.411177in}{2.268455in}}%
\pgfpathlineto{\pgfqpoint{3.424636in}{2.261077in}}%
\pgfpathlineto{\pgfqpoint{3.438098in}{2.253829in}}%
\pgfpathlineto{\pgfqpoint{3.451562in}{2.246712in}}%
\pgfpathlineto{\pgfqpoint{3.443477in}{2.240401in}}%
\pgfpathlineto{\pgfqpoint{3.435385in}{2.234175in}}%
\pgfpathlineto{\pgfqpoint{3.427285in}{2.228037in}}%
\pgfpathlineto{\pgfqpoint{3.419177in}{2.221990in}}%
\pgfpathlineto{\pgfqpoint{3.405694in}{2.229373in}}%
\pgfpathlineto{\pgfqpoint{3.392213in}{2.236886in}}%
\pgfpathlineto{\pgfqpoint{3.378734in}{2.244530in}}%
\pgfpathlineto{\pgfqpoint{3.365258in}{2.252305in}}%
\pgfpathlineto{\pgfqpoint{3.373385in}{2.258080in}}%
\pgfpathlineto{\pgfqpoint{3.381505in}{2.263950in}}%
\pgfpathlineto{\pgfqpoint{3.389617in}{2.269912in}}%
\pgfpathlineto{\pgfqpoint{3.397721in}{2.275964in}}%
\pgfpathclose%
\pgfusepath{fill}%
\end{pgfscope}%
\begin{pgfscope}%
\pgfpathrectangle{\pgfqpoint{1.150000in}{0.150000in}}{\pgfqpoint{5.700000in}{5.700000in}}%
\pgfusepath{clip}%
\pgfsetbuttcap%
\pgfsetroundjoin%
\definecolor{currentfill}{rgb}{0.269944,0.014625,0.341379}%
\pgfsetfillcolor{currentfill}%
\pgfsetfillopacity{0.700000}%
\pgfsetlinewidth{0.000000pt}%
\definecolor{currentstroke}{rgb}{0.000000,0.000000,0.000000}%
\pgfsetstrokecolor{currentstroke}%
\pgfsetdash{}{0pt}%
\pgfpathmoveto{\pgfqpoint{3.817092in}{2.225800in}}%
\pgfpathlineto{\pgfqpoint{3.830596in}{2.221939in}}%
\pgfpathlineto{\pgfqpoint{3.844106in}{2.218194in}}%
\pgfpathlineto{\pgfqpoint{3.857621in}{2.214565in}}%
\pgfpathlineto{\pgfqpoint{3.871142in}{2.211052in}}%
\pgfpathlineto{\pgfqpoint{3.863230in}{2.202674in}}%
\pgfpathlineto{\pgfqpoint{3.855312in}{2.194328in}}%
\pgfpathlineto{\pgfqpoint{3.847388in}{2.186016in}}%
\pgfpathlineto{\pgfqpoint{3.839458in}{2.177740in}}%
\pgfpathlineto{\pgfqpoint{3.825924in}{2.181463in}}%
\pgfpathlineto{\pgfqpoint{3.812396in}{2.185301in}}%
\pgfpathlineto{\pgfqpoint{3.798873in}{2.189255in}}%
\pgfpathlineto{\pgfqpoint{3.785355in}{2.193325in}}%
\pgfpathlineto{\pgfqpoint{3.793298in}{2.201385in}}%
\pgfpathlineto{\pgfqpoint{3.801235in}{2.209485in}}%
\pgfpathlineto{\pgfqpoint{3.809167in}{2.217624in}}%
\pgfpathlineto{\pgfqpoint{3.817092in}{2.225800in}}%
\pgfpathclose%
\pgfusepath{fill}%
\end{pgfscope}%
\begin{pgfscope}%
\pgfpathrectangle{\pgfqpoint{1.150000in}{0.150000in}}{\pgfqpoint{5.700000in}{5.700000in}}%
\pgfusepath{clip}%
\pgfsetbuttcap%
\pgfsetroundjoin%
\definecolor{currentfill}{rgb}{0.273809,0.031497,0.358853}%
\pgfsetfillcolor{currentfill}%
\pgfsetfillopacity{0.700000}%
\pgfsetlinewidth{0.000000pt}%
\definecolor{currentstroke}{rgb}{0.000000,0.000000,0.000000}%
\pgfsetstrokecolor{currentstroke}%
\pgfsetdash{}{0pt}%
\pgfpathmoveto{\pgfqpoint{4.042434in}{2.257956in}}%
\pgfpathlineto{\pgfqpoint{4.055989in}{2.255826in}}%
\pgfpathlineto{\pgfqpoint{4.069550in}{2.253808in}}%
\pgfpathlineto{\pgfqpoint{4.083119in}{2.251900in}}%
\pgfpathlineto{\pgfqpoint{4.096696in}{2.250103in}}%
\pgfpathlineto{\pgfqpoint{4.088862in}{2.241025in}}%
\pgfpathlineto{\pgfqpoint{4.081024in}{2.231954in}}%
\pgfpathlineto{\pgfqpoint{4.073180in}{2.222890in}}%
\pgfpathlineto{\pgfqpoint{4.065330in}{2.213835in}}%
\pgfpathlineto{\pgfqpoint{4.051744in}{2.215806in}}%
\pgfpathlineto{\pgfqpoint{4.038164in}{2.217886in}}%
\pgfpathlineto{\pgfqpoint{4.024591in}{2.220078in}}%
\pgfpathlineto{\pgfqpoint{4.011026in}{2.222381in}}%
\pgfpathlineto{\pgfqpoint{4.018886in}{2.231256in}}%
\pgfpathlineto{\pgfqpoint{4.026741in}{2.240145in}}%
\pgfpathlineto{\pgfqpoint{4.034590in}{2.249045in}}%
\pgfpathlineto{\pgfqpoint{4.042434in}{2.257956in}}%
\pgfpathclose%
\pgfusepath{fill}%
\end{pgfscope}%
\begin{pgfscope}%
\pgfpathrectangle{\pgfqpoint{1.150000in}{0.150000in}}{\pgfqpoint{5.700000in}{5.700000in}}%
\pgfusepath{clip}%
\pgfsetbuttcap%
\pgfsetroundjoin%
\definecolor{currentfill}{rgb}{0.280255,0.165693,0.476498}%
\pgfsetfillcolor{currentfill}%
\pgfsetfillopacity{0.700000}%
\pgfsetlinewidth{0.000000pt}%
\definecolor{currentstroke}{rgb}{0.000000,0.000000,0.000000}%
\pgfsetstrokecolor{currentstroke}%
\pgfsetdash{}{0pt}%
\pgfpathmoveto{\pgfqpoint{2.955098in}{2.512812in}}%
\pgfpathlineto{\pgfqpoint{2.968587in}{2.500701in}}%
\pgfpathlineto{\pgfqpoint{2.982075in}{2.488748in}}%
\pgfpathlineto{\pgfqpoint{2.995561in}{2.476953in}}%
\pgfpathlineto{\pgfqpoint{3.009046in}{2.465315in}}%
\pgfpathlineto{\pgfqpoint{3.000730in}{2.461846in}}%
\pgfpathlineto{\pgfqpoint{2.992403in}{2.458515in}}%
\pgfpathlineto{\pgfqpoint{2.984066in}{2.455327in}}%
\pgfpathlineto{\pgfqpoint{2.975719in}{2.452283in}}%
\pgfpathlineto{\pgfqpoint{2.962206in}{2.464230in}}%
\pgfpathlineto{\pgfqpoint{2.948692in}{2.476334in}}%
\pgfpathlineto{\pgfqpoint{2.935176in}{2.488597in}}%
\pgfpathlineto{\pgfqpoint{2.921658in}{2.501019in}}%
\pgfpathlineto{\pgfqpoint{2.930035in}{2.503745in}}%
\pgfpathlineto{\pgfqpoint{2.938400in}{2.506622in}}%
\pgfpathlineto{\pgfqpoint{2.946755in}{2.509645in}}%
\pgfpathlineto{\pgfqpoint{2.955098in}{2.512812in}}%
\pgfpathclose%
\pgfusepath{fill}%
\end{pgfscope}%
\begin{pgfscope}%
\pgfpathrectangle{\pgfqpoint{1.150000in}{0.150000in}}{\pgfqpoint{5.700000in}{5.700000in}}%
\pgfusepath{clip}%
\pgfsetbuttcap%
\pgfsetroundjoin%
\definecolor{currentfill}{rgb}{0.275191,0.194905,0.496005}%
\pgfsetfillcolor{currentfill}%
\pgfsetfillopacity{0.700000}%
\pgfsetlinewidth{0.000000pt}%
\definecolor{currentstroke}{rgb}{0.000000,0.000000,0.000000}%
\pgfsetstrokecolor{currentstroke}%
\pgfsetdash{}{0pt}%
\pgfpathmoveto{\pgfqpoint{4.781109in}{2.560010in}}%
\pgfpathlineto{\pgfqpoint{4.794915in}{2.562439in}}%
\pgfpathlineto{\pgfqpoint{4.808731in}{2.564970in}}%
\pgfpathlineto{\pgfqpoint{4.822558in}{2.567604in}}%
\pgfpathlineto{\pgfqpoint{4.836397in}{2.570339in}}%
\pgfpathlineto{\pgfqpoint{4.828816in}{2.561369in}}%
\pgfpathlineto{\pgfqpoint{4.821229in}{2.552343in}}%
\pgfpathlineto{\pgfqpoint{4.813636in}{2.543259in}}%
\pgfpathlineto{\pgfqpoint{4.806038in}{2.534118in}}%
\pgfpathlineto{\pgfqpoint{4.792191in}{2.531410in}}%
\pgfpathlineto{\pgfqpoint{4.778356in}{2.528805in}}%
\pgfpathlineto{\pgfqpoint{4.764531in}{2.526302in}}%
\pgfpathlineto{\pgfqpoint{4.750717in}{2.523900in}}%
\pgfpathlineto{\pgfqpoint{4.758324in}{2.533006in}}%
\pgfpathlineto{\pgfqpoint{4.765925in}{2.542060in}}%
\pgfpathlineto{\pgfqpoint{4.773520in}{2.551061in}}%
\pgfpathlineto{\pgfqpoint{4.781109in}{2.560010in}}%
\pgfpathclose%
\pgfusepath{fill}%
\end{pgfscope}%
\begin{pgfscope}%
\pgfpathrectangle{\pgfqpoint{1.150000in}{0.150000in}}{\pgfqpoint{5.700000in}{5.700000in}}%
\pgfusepath{clip}%
\pgfsetbuttcap%
\pgfsetroundjoin%
\definecolor{currentfill}{rgb}{0.278826,0.175490,0.483397}%
\pgfsetfillcolor{currentfill}%
\pgfsetfillopacity{0.700000}%
\pgfsetlinewidth{0.000000pt}%
\definecolor{currentstroke}{rgb}{0.000000,0.000000,0.000000}%
\pgfsetstrokecolor{currentstroke}%
\pgfsetdash{}{0pt}%
\pgfpathmoveto{\pgfqpoint{4.695568in}{2.515322in}}%
\pgfpathlineto{\pgfqpoint{4.709340in}{2.517313in}}%
\pgfpathlineto{\pgfqpoint{4.723122in}{2.519406in}}%
\pgfpathlineto{\pgfqpoint{4.736914in}{2.521602in}}%
\pgfpathlineto{\pgfqpoint{4.750717in}{2.523900in}}%
\pgfpathlineto{\pgfqpoint{4.743105in}{2.514742in}}%
\pgfpathlineto{\pgfqpoint{4.735487in}{2.505531in}}%
\pgfpathlineto{\pgfqpoint{4.727863in}{2.496269in}}%
\pgfpathlineto{\pgfqpoint{4.720233in}{2.486954in}}%
\pgfpathlineto{\pgfqpoint{4.706422in}{2.484702in}}%
\pgfpathlineto{\pgfqpoint{4.692621in}{2.482552in}}%
\pgfpathlineto{\pgfqpoint{4.678831in}{2.480505in}}%
\pgfpathlineto{\pgfqpoint{4.665052in}{2.478561in}}%
\pgfpathlineto{\pgfqpoint{4.672690in}{2.487823in}}%
\pgfpathlineto{\pgfqpoint{4.680322in}{2.497037in}}%
\pgfpathlineto{\pgfqpoint{4.687948in}{2.506203in}}%
\pgfpathlineto{\pgfqpoint{4.695568in}{2.515322in}}%
\pgfpathclose%
\pgfusepath{fill}%
\end{pgfscope}%
\begin{pgfscope}%
\pgfpathrectangle{\pgfqpoint{1.150000in}{0.150000in}}{\pgfqpoint{5.700000in}{5.700000in}}%
\pgfusepath{clip}%
\pgfsetbuttcap%
\pgfsetroundjoin%
\definecolor{currentfill}{rgb}{0.269308,0.218818,0.509577}%
\pgfsetfillcolor{currentfill}%
\pgfsetfillopacity{0.700000}%
\pgfsetlinewidth{0.000000pt}%
\definecolor{currentstroke}{rgb}{0.000000,0.000000,0.000000}%
\pgfsetstrokecolor{currentstroke}%
\pgfsetdash{}{0pt}%
\pgfpathmoveto{\pgfqpoint{4.866661in}{2.605652in}}%
\pgfpathlineto{\pgfqpoint{4.880501in}{2.608499in}}%
\pgfpathlineto{\pgfqpoint{4.894353in}{2.611447in}}%
\pgfpathlineto{\pgfqpoint{4.908217in}{2.614497in}}%
\pgfpathlineto{\pgfqpoint{4.922092in}{2.617649in}}%
\pgfpathlineto{\pgfqpoint{4.914543in}{2.608903in}}%
\pgfpathlineto{\pgfqpoint{4.906989in}{2.600096in}}%
\pgfpathlineto{\pgfqpoint{4.899428in}{2.591228in}}%
\pgfpathlineto{\pgfqpoint{4.891862in}{2.582298in}}%
\pgfpathlineto{\pgfqpoint{4.877979in}{2.579156in}}%
\pgfpathlineto{\pgfqpoint{4.864107in}{2.576116in}}%
\pgfpathlineto{\pgfqpoint{4.850246in}{2.573177in}}%
\pgfpathlineto{\pgfqpoint{4.836397in}{2.570339in}}%
\pgfpathlineto{\pgfqpoint{4.843972in}{2.579252in}}%
\pgfpathlineto{\pgfqpoint{4.851541in}{2.588109in}}%
\pgfpathlineto{\pgfqpoint{4.859104in}{2.596908in}}%
\pgfpathlineto{\pgfqpoint{4.866661in}{2.605652in}}%
\pgfpathclose%
\pgfusepath{fill}%
\end{pgfscope}%
\begin{pgfscope}%
\pgfpathrectangle{\pgfqpoint{1.150000in}{0.150000in}}{\pgfqpoint{5.700000in}{5.700000in}}%
\pgfusepath{clip}%
\pgfsetbuttcap%
\pgfsetroundjoin%
\definecolor{currentfill}{rgb}{0.281887,0.150881,0.465405}%
\pgfsetfillcolor{currentfill}%
\pgfsetfillopacity{0.700000}%
\pgfsetlinewidth{0.000000pt}%
\definecolor{currentstroke}{rgb}{0.000000,0.000000,0.000000}%
\pgfsetstrokecolor{currentstroke}%
\pgfsetdash{}{0pt}%
\pgfpathmoveto{\pgfqpoint{4.610036in}{2.471818in}}%
\pgfpathlineto{\pgfqpoint{4.623775in}{2.473349in}}%
\pgfpathlineto{\pgfqpoint{4.637524in}{2.474983in}}%
\pgfpathlineto{\pgfqpoint{4.651283in}{2.476720in}}%
\pgfpathlineto{\pgfqpoint{4.665052in}{2.478561in}}%
\pgfpathlineto{\pgfqpoint{4.657409in}{2.469252in}}%
\pgfpathlineto{\pgfqpoint{4.649760in}{2.459897in}}%
\pgfpathlineto{\pgfqpoint{4.642105in}{2.450495in}}%
\pgfpathlineto{\pgfqpoint{4.634445in}{2.441047in}}%
\pgfpathlineto{\pgfqpoint{4.620668in}{2.439271in}}%
\pgfpathlineto{\pgfqpoint{4.606901in}{2.437598in}}%
\pgfpathlineto{\pgfqpoint{4.593144in}{2.436029in}}%
\pgfpathlineto{\pgfqpoint{4.579397in}{2.434563in}}%
\pgfpathlineto{\pgfqpoint{4.587065in}{2.443939in}}%
\pgfpathlineto{\pgfqpoint{4.594728in}{2.453274in}}%
\pgfpathlineto{\pgfqpoint{4.602385in}{2.462567in}}%
\pgfpathlineto{\pgfqpoint{4.610036in}{2.471818in}}%
\pgfpathclose%
\pgfusepath{fill}%
\end{pgfscope}%
\begin{pgfscope}%
\pgfpathrectangle{\pgfqpoint{1.150000in}{0.150000in}}{\pgfqpoint{5.700000in}{5.700000in}}%
\pgfusepath{clip}%
\pgfsetbuttcap%
\pgfsetroundjoin%
\definecolor{currentfill}{rgb}{0.262138,0.242286,0.520837}%
\pgfsetfillcolor{currentfill}%
\pgfsetfillopacity{0.700000}%
\pgfsetlinewidth{0.000000pt}%
\definecolor{currentstroke}{rgb}{0.000000,0.000000,0.000000}%
\pgfsetstrokecolor{currentstroke}%
\pgfsetdash{}{0pt}%
\pgfpathmoveto{\pgfqpoint{4.952223in}{2.652030in}}%
\pgfpathlineto{\pgfqpoint{4.966101in}{2.655274in}}%
\pgfpathlineto{\pgfqpoint{4.979990in}{2.658619in}}%
\pgfpathlineto{\pgfqpoint{4.993890in}{2.662065in}}%
\pgfpathlineto{\pgfqpoint{5.007803in}{2.665612in}}%
\pgfpathlineto{\pgfqpoint{5.000288in}{2.657122in}}%
\pgfpathlineto{\pgfqpoint{4.992767in}{2.648568in}}%
\pgfpathlineto{\pgfqpoint{4.985240in}{2.639950in}}%
\pgfpathlineto{\pgfqpoint{4.977707in}{2.631267in}}%
\pgfpathlineto{\pgfqpoint{4.963785in}{2.627711in}}%
\pgfpathlineto{\pgfqpoint{4.949876in}{2.624256in}}%
\pgfpathlineto{\pgfqpoint{4.935978in}{2.620902in}}%
\pgfpathlineto{\pgfqpoint{4.922092in}{2.617649in}}%
\pgfpathlineto{\pgfqpoint{4.929634in}{2.626334in}}%
\pgfpathlineto{\pgfqpoint{4.937170in}{2.634959in}}%
\pgfpathlineto{\pgfqpoint{4.944700in}{2.643525in}}%
\pgfpathlineto{\pgfqpoint{4.952223in}{2.652030in}}%
\pgfpathclose%
\pgfusepath{fill}%
\end{pgfscope}%
\begin{pgfscope}%
\pgfpathrectangle{\pgfqpoint{1.150000in}{0.150000in}}{\pgfqpoint{5.700000in}{5.700000in}}%
\pgfusepath{clip}%
\pgfsetbuttcap%
\pgfsetroundjoin%
\definecolor{currentfill}{rgb}{0.283072,0.130895,0.449241}%
\pgfsetfillcolor{currentfill}%
\pgfsetfillopacity{0.700000}%
\pgfsetlinewidth{0.000000pt}%
\definecolor{currentstroke}{rgb}{0.000000,0.000000,0.000000}%
\pgfsetstrokecolor{currentstroke}%
\pgfsetdash{}{0pt}%
\pgfpathmoveto{\pgfqpoint{4.524508in}{2.429740in}}%
\pgfpathlineto{\pgfqpoint{4.538216in}{2.430789in}}%
\pgfpathlineto{\pgfqpoint{4.551933in}{2.431943in}}%
\pgfpathlineto{\pgfqpoint{4.565660in}{2.433201in}}%
\pgfpathlineto{\pgfqpoint{4.579397in}{2.434563in}}%
\pgfpathlineto{\pgfqpoint{4.571724in}{2.425145in}}%
\pgfpathlineto{\pgfqpoint{4.564045in}{2.415687in}}%
\pgfpathlineto{\pgfqpoint{4.556361in}{2.406188in}}%
\pgfpathlineto{\pgfqpoint{4.548671in}{2.396650in}}%
\pgfpathlineto{\pgfqpoint{4.534926in}{2.395371in}}%
\pgfpathlineto{\pgfqpoint{4.521191in}{2.394196in}}%
\pgfpathlineto{\pgfqpoint{4.507465in}{2.393126in}}%
\pgfpathlineto{\pgfqpoint{4.493749in}{2.392159in}}%
\pgfpathlineto{\pgfqpoint{4.501447in}{2.401608in}}%
\pgfpathlineto{\pgfqpoint{4.509139in}{2.411021in}}%
\pgfpathlineto{\pgfqpoint{4.516826in}{2.420399in}}%
\pgfpathlineto{\pgfqpoint{4.524508in}{2.429740in}}%
\pgfpathclose%
\pgfusepath{fill}%
\end{pgfscope}%
\begin{pgfscope}%
\pgfpathrectangle{\pgfqpoint{1.150000in}{0.150000in}}{\pgfqpoint{5.700000in}{5.700000in}}%
\pgfusepath{clip}%
\pgfsetbuttcap%
\pgfsetroundjoin%
\definecolor{currentfill}{rgb}{0.253935,0.265254,0.529983}%
\pgfsetfillcolor{currentfill}%
\pgfsetfillopacity{0.700000}%
\pgfsetlinewidth{0.000000pt}%
\definecolor{currentstroke}{rgb}{0.000000,0.000000,0.000000}%
\pgfsetstrokecolor{currentstroke}%
\pgfsetdash{}{0pt}%
\pgfpathmoveto{\pgfqpoint{5.037797in}{2.698940in}}%
\pgfpathlineto{\pgfqpoint{5.051712in}{2.702560in}}%
\pgfpathlineto{\pgfqpoint{5.065639in}{2.706281in}}%
\pgfpathlineto{\pgfqpoint{5.079578in}{2.710102in}}%
\pgfpathlineto{\pgfqpoint{5.093529in}{2.714023in}}%
\pgfpathlineto{\pgfqpoint{5.086049in}{2.705819in}}%
\pgfpathlineto{\pgfqpoint{5.078563in}{2.697549in}}%
\pgfpathlineto{\pgfqpoint{5.071071in}{2.689212in}}%
\pgfpathlineto{\pgfqpoint{5.063571in}{2.680807in}}%
\pgfpathlineto{\pgfqpoint{5.049611in}{2.676857in}}%
\pgfpathlineto{\pgfqpoint{5.035663in}{2.673008in}}%
\pgfpathlineto{\pgfqpoint{5.021727in}{2.669259in}}%
\pgfpathlineto{\pgfqpoint{5.007803in}{2.665612in}}%
\pgfpathlineto{\pgfqpoint{5.015311in}{2.674038in}}%
\pgfpathlineto{\pgfqpoint{5.022813in}{2.682401in}}%
\pgfpathlineto{\pgfqpoint{5.030308in}{2.690701in}}%
\pgfpathlineto{\pgfqpoint{5.037797in}{2.698940in}}%
\pgfpathclose%
\pgfusepath{fill}%
\end{pgfscope}%
\begin{pgfscope}%
\pgfpathrectangle{\pgfqpoint{1.150000in}{0.150000in}}{\pgfqpoint{5.700000in}{5.700000in}}%
\pgfusepath{clip}%
\pgfsetbuttcap%
\pgfsetroundjoin%
\definecolor{currentfill}{rgb}{0.244972,0.287675,0.537260}%
\pgfsetfillcolor{currentfill}%
\pgfsetfillopacity{0.700000}%
\pgfsetlinewidth{0.000000pt}%
\definecolor{currentstroke}{rgb}{0.000000,0.000000,0.000000}%
\pgfsetstrokecolor{currentstroke}%
\pgfsetdash{}{0pt}%
\pgfpathmoveto{\pgfqpoint{5.123382in}{2.746188in}}%
\pgfpathlineto{\pgfqpoint{5.137335in}{2.750164in}}%
\pgfpathlineto{\pgfqpoint{5.151300in}{2.754239in}}%
\pgfpathlineto{\pgfqpoint{5.165278in}{2.758415in}}%
\pgfpathlineto{\pgfqpoint{5.179268in}{2.762691in}}%
\pgfpathlineto{\pgfqpoint{5.171825in}{2.754800in}}%
\pgfpathlineto{\pgfqpoint{5.164375in}{2.746840in}}%
\pgfpathlineto{\pgfqpoint{5.156918in}{2.738811in}}%
\pgfpathlineto{\pgfqpoint{5.149455in}{2.730713in}}%
\pgfpathlineto{\pgfqpoint{5.135455in}{2.726390in}}%
\pgfpathlineto{\pgfqpoint{5.121467in}{2.722168in}}%
\pgfpathlineto{\pgfqpoint{5.107492in}{2.718045in}}%
\pgfpathlineto{\pgfqpoint{5.093529in}{2.714023in}}%
\pgfpathlineto{\pgfqpoint{5.101002in}{2.722161in}}%
\pgfpathlineto{\pgfqpoint{5.108468in}{2.730234in}}%
\pgfpathlineto{\pgfqpoint{5.115928in}{2.738243in}}%
\pgfpathlineto{\pgfqpoint{5.123382in}{2.746188in}}%
\pgfpathclose%
\pgfusepath{fill}%
\end{pgfscope}%
\begin{pgfscope}%
\pgfpathrectangle{\pgfqpoint{1.150000in}{0.150000in}}{\pgfqpoint{5.700000in}{5.700000in}}%
\pgfusepath{clip}%
\pgfsetbuttcap%
\pgfsetroundjoin%
\definecolor{currentfill}{rgb}{0.283091,0.110553,0.431554}%
\pgfsetfillcolor{currentfill}%
\pgfsetfillopacity{0.700000}%
\pgfsetlinewidth{0.000000pt}%
\definecolor{currentstroke}{rgb}{0.000000,0.000000,0.000000}%
\pgfsetstrokecolor{currentstroke}%
\pgfsetdash{}{0pt}%
\pgfpathmoveto{\pgfqpoint{4.438979in}{2.389343in}}%
\pgfpathlineto{\pgfqpoint{4.452658in}{2.389889in}}%
\pgfpathlineto{\pgfqpoint{4.466346in}{2.390541in}}%
\pgfpathlineto{\pgfqpoint{4.480043in}{2.391298in}}%
\pgfpathlineto{\pgfqpoint{4.493749in}{2.392159in}}%
\pgfpathlineto{\pgfqpoint{4.486046in}{2.382677in}}%
\pgfpathlineto{\pgfqpoint{4.478338in}{2.373160in}}%
\pgfpathlineto{\pgfqpoint{4.470624in}{2.363611in}}%
\pgfpathlineto{\pgfqpoint{4.462905in}{2.354029in}}%
\pgfpathlineto{\pgfqpoint{4.449190in}{2.353268in}}%
\pgfpathlineto{\pgfqpoint{4.435485in}{2.352613in}}%
\pgfpathlineto{\pgfqpoint{4.421789in}{2.352062in}}%
\pgfpathlineto{\pgfqpoint{4.408102in}{2.351617in}}%
\pgfpathlineto{\pgfqpoint{4.415829in}{2.361091in}}%
\pgfpathlineto{\pgfqpoint{4.423551in}{2.370537in}}%
\pgfpathlineto{\pgfqpoint{4.431268in}{2.379954in}}%
\pgfpathlineto{\pgfqpoint{4.438979in}{2.389343in}}%
\pgfpathclose%
\pgfusepath{fill}%
\end{pgfscope}%
\begin{pgfscope}%
\pgfpathrectangle{\pgfqpoint{1.150000in}{0.150000in}}{\pgfqpoint{5.700000in}{5.700000in}}%
\pgfusepath{clip}%
\pgfsetbuttcap%
\pgfsetroundjoin%
\definecolor{currentfill}{rgb}{0.235526,0.309527,0.542944}%
\pgfsetfillcolor{currentfill}%
\pgfsetfillopacity{0.700000}%
\pgfsetlinewidth{0.000000pt}%
\definecolor{currentstroke}{rgb}{0.000000,0.000000,0.000000}%
\pgfsetstrokecolor{currentstroke}%
\pgfsetdash{}{0pt}%
\pgfpathmoveto{\pgfqpoint{5.208974in}{2.793594in}}%
\pgfpathlineto{\pgfqpoint{5.222966in}{2.797905in}}%
\pgfpathlineto{\pgfqpoint{5.236970in}{2.802315in}}%
\pgfpathlineto{\pgfqpoint{5.250988in}{2.806825in}}%
\pgfpathlineto{\pgfqpoint{5.265018in}{2.811435in}}%
\pgfpathlineto{\pgfqpoint{5.257613in}{2.803879in}}%
\pgfpathlineto{\pgfqpoint{5.250200in}{2.796254in}}%
\pgfpathlineto{\pgfqpoint{5.242781in}{2.788560in}}%
\pgfpathlineto{\pgfqpoint{5.235355in}{2.780794in}}%
\pgfpathlineto{\pgfqpoint{5.221314in}{2.776119in}}%
\pgfpathlineto{\pgfqpoint{5.207286in}{2.771543in}}%
\pgfpathlineto{\pgfqpoint{5.193271in}{2.767067in}}%
\pgfpathlineto{\pgfqpoint{5.179268in}{2.762691in}}%
\pgfpathlineto{\pgfqpoint{5.186704in}{2.770515in}}%
\pgfpathlineto{\pgfqpoint{5.194134in}{2.778273in}}%
\pgfpathlineto{\pgfqpoint{5.201557in}{2.785966in}}%
\pgfpathlineto{\pgfqpoint{5.208974in}{2.793594in}}%
\pgfpathclose%
\pgfusepath{fill}%
\end{pgfscope}%
\begin{pgfscope}%
\pgfpathrectangle{\pgfqpoint{1.150000in}{0.150000in}}{\pgfqpoint{5.700000in}{5.700000in}}%
\pgfusepath{clip}%
\pgfsetbuttcap%
\pgfsetroundjoin%
\definecolor{currentfill}{rgb}{0.279566,0.067836,0.391917}%
\pgfsetfillcolor{currentfill}%
\pgfsetfillopacity{0.700000}%
\pgfsetlinewidth{0.000000pt}%
\definecolor{currentstroke}{rgb}{0.000000,0.000000,0.000000}%
\pgfsetstrokecolor{currentstroke}%
\pgfsetdash{}{0pt}%
\pgfpathmoveto{\pgfqpoint{3.257516in}{2.319329in}}%
\pgfpathlineto{\pgfqpoint{3.270978in}{2.310475in}}%
\pgfpathlineto{\pgfqpoint{3.284442in}{2.301758in}}%
\pgfpathlineto{\pgfqpoint{3.297907in}{2.293178in}}%
\pgfpathlineto{\pgfqpoint{3.311373in}{2.284734in}}%
\pgfpathlineto{\pgfqpoint{3.303216in}{2.279335in}}%
\pgfpathlineto{\pgfqpoint{3.295051in}{2.274039in}}%
\pgfpathlineto{\pgfqpoint{3.286877in}{2.268852in}}%
\pgfpathlineto{\pgfqpoint{3.278695in}{2.263774in}}%
\pgfpathlineto{\pgfqpoint{3.265206in}{2.272502in}}%
\pgfpathlineto{\pgfqpoint{3.251719in}{2.281368in}}%
\pgfpathlineto{\pgfqpoint{3.238233in}{2.290370in}}%
\pgfpathlineto{\pgfqpoint{3.224748in}{2.299510in}}%
\pgfpathlineto{\pgfqpoint{3.232954in}{2.304296in}}%
\pgfpathlineto{\pgfqpoint{3.241150in}{2.309196in}}%
\pgfpathlineto{\pgfqpoint{3.249337in}{2.314208in}}%
\pgfpathlineto{\pgfqpoint{3.257516in}{2.319329in}}%
\pgfpathclose%
\pgfusepath{fill}%
\end{pgfscope}%
\begin{pgfscope}%
\pgfpathrectangle{\pgfqpoint{1.150000in}{0.150000in}}{\pgfqpoint{5.700000in}{5.700000in}}%
\pgfusepath{clip}%
\pgfsetbuttcap%
\pgfsetroundjoin%
\definecolor{currentfill}{rgb}{0.269944,0.014625,0.341379}%
\pgfsetfillcolor{currentfill}%
\pgfsetfillopacity{0.700000}%
\pgfsetlinewidth{0.000000pt}%
\definecolor{currentstroke}{rgb}{0.000000,0.000000,0.000000}%
\pgfsetstrokecolor{currentstroke}%
\pgfsetdash{}{0pt}%
\pgfpathmoveto{\pgfqpoint{3.591511in}{2.222375in}}%
\pgfpathlineto{\pgfqpoint{3.604988in}{2.216640in}}%
\pgfpathlineto{\pgfqpoint{3.618469in}{2.211027in}}%
\pgfpathlineto{\pgfqpoint{3.631954in}{2.205537in}}%
\pgfpathlineto{\pgfqpoint{3.645443in}{2.200169in}}%
\pgfpathlineto{\pgfqpoint{3.637439in}{2.192821in}}%
\pgfpathlineto{\pgfqpoint{3.629428in}{2.185536in}}%
\pgfpathlineto{\pgfqpoint{3.621410in}{2.178316in}}%
\pgfpathlineto{\pgfqpoint{3.613386in}{2.171162in}}%
\pgfpathlineto{\pgfqpoint{3.599880in}{2.176777in}}%
\pgfpathlineto{\pgfqpoint{3.586378in}{2.182513in}}%
\pgfpathlineto{\pgfqpoint{3.572881in}{2.188372in}}%
\pgfpathlineto{\pgfqpoint{3.559387in}{2.194354in}}%
\pgfpathlineto{\pgfqpoint{3.567428in}{2.201254in}}%
\pgfpathlineto{\pgfqpoint{3.575463in}{2.208226in}}%
\pgfpathlineto{\pgfqpoint{3.583490in}{2.215267in}}%
\pgfpathlineto{\pgfqpoint{3.591511in}{2.222375in}}%
\pgfpathclose%
\pgfusepath{fill}%
\end{pgfscope}%
\begin{pgfscope}%
\pgfpathrectangle{\pgfqpoint{1.150000in}{0.150000in}}{\pgfqpoint{5.700000in}{5.700000in}}%
\pgfusepath{clip}%
\pgfsetbuttcap%
\pgfsetroundjoin%
\definecolor{currentfill}{rgb}{0.282290,0.145912,0.461510}%
\pgfsetfillcolor{currentfill}%
\pgfsetfillopacity{0.700000}%
\pgfsetlinewidth{0.000000pt}%
\definecolor{currentstroke}{rgb}{0.000000,0.000000,0.000000}%
\pgfsetstrokecolor{currentstroke}%
\pgfsetdash{}{0pt}%
\pgfpathmoveto{\pgfqpoint{3.009046in}{2.465315in}}%
\pgfpathlineto{\pgfqpoint{3.022529in}{2.453833in}}%
\pgfpathlineto{\pgfqpoint{3.036012in}{2.442505in}}%
\pgfpathlineto{\pgfqpoint{3.049494in}{2.431330in}}%
\pgfpathlineto{\pgfqpoint{3.062975in}{2.420307in}}%
\pgfpathlineto{\pgfqpoint{3.054686in}{2.416536in}}%
\pgfpathlineto{\pgfqpoint{3.046387in}{2.412900in}}%
\pgfpathlineto{\pgfqpoint{3.038078in}{2.409401in}}%
\pgfpathlineto{\pgfqpoint{3.029758in}{2.406042in}}%
\pgfpathlineto{\pgfqpoint{3.016250in}{2.417372in}}%
\pgfpathlineto{\pgfqpoint{3.002740in}{2.428855in}}%
\pgfpathlineto{\pgfqpoint{2.989230in}{2.440492in}}%
\pgfpathlineto{\pgfqpoint{2.975719in}{2.452283in}}%
\pgfpathlineto{\pgfqpoint{2.984066in}{2.455327in}}%
\pgfpathlineto{\pgfqpoint{2.992403in}{2.458515in}}%
\pgfpathlineto{\pgfqpoint{3.000730in}{2.461846in}}%
\pgfpathlineto{\pgfqpoint{3.009046in}{2.465315in}}%
\pgfpathclose%
\pgfusepath{fill}%
\end{pgfscope}%
\begin{pgfscope}%
\pgfpathrectangle{\pgfqpoint{1.150000in}{0.150000in}}{\pgfqpoint{5.700000in}{5.700000in}}%
\pgfusepath{clip}%
\pgfsetbuttcap%
\pgfsetroundjoin%
\definecolor{currentfill}{rgb}{0.225863,0.330805,0.547314}%
\pgfsetfillcolor{currentfill}%
\pgfsetfillopacity{0.700000}%
\pgfsetlinewidth{0.000000pt}%
\definecolor{currentstroke}{rgb}{0.000000,0.000000,0.000000}%
\pgfsetstrokecolor{currentstroke}%
\pgfsetdash{}{0pt}%
\pgfpathmoveto{\pgfqpoint{5.294570in}{2.840989in}}%
\pgfpathlineto{\pgfqpoint{5.308602in}{2.845614in}}%
\pgfpathlineto{\pgfqpoint{5.322646in}{2.850338in}}%
\pgfpathlineto{\pgfqpoint{5.336704in}{2.855162in}}%
\pgfpathlineto{\pgfqpoint{5.350775in}{2.860085in}}%
\pgfpathlineto{\pgfqpoint{5.343409in}{2.852886in}}%
\pgfpathlineto{\pgfqpoint{5.336036in}{2.845618in}}%
\pgfpathlineto{\pgfqpoint{5.328656in}{2.838280in}}%
\pgfpathlineto{\pgfqpoint{5.321269in}{2.830870in}}%
\pgfpathlineto{\pgfqpoint{5.307186in}{2.825862in}}%
\pgfpathlineto{\pgfqpoint{5.293117in}{2.820953in}}%
\pgfpathlineto{\pgfqpoint{5.279061in}{2.816144in}}%
\pgfpathlineto{\pgfqpoint{5.265018in}{2.811435in}}%
\pgfpathlineto{\pgfqpoint{5.272416in}{2.818922in}}%
\pgfpathlineto{\pgfqpoint{5.279808in}{2.826343in}}%
\pgfpathlineto{\pgfqpoint{5.287193in}{2.833698in}}%
\pgfpathlineto{\pgfqpoint{5.294570in}{2.840989in}}%
\pgfpathclose%
\pgfusepath{fill}%
\end{pgfscope}%
\begin{pgfscope}%
\pgfpathrectangle{\pgfqpoint{1.150000in}{0.150000in}}{\pgfqpoint{5.700000in}{5.700000in}}%
\pgfusepath{clip}%
\pgfsetbuttcap%
\pgfsetroundjoin%
\definecolor{currentfill}{rgb}{0.271305,0.019942,0.347269}%
\pgfsetfillcolor{currentfill}%
\pgfsetfillopacity{0.700000}%
\pgfsetlinewidth{0.000000pt}%
\definecolor{currentstroke}{rgb}{0.000000,0.000000,0.000000}%
\pgfsetstrokecolor{currentstroke}%
\pgfsetdash{}{0pt}%
\pgfpathmoveto{\pgfqpoint{3.956829in}{2.232713in}}%
\pgfpathlineto{\pgfqpoint{3.970368in}{2.229961in}}%
\pgfpathlineto{\pgfqpoint{3.983914in}{2.227322in}}%
\pgfpathlineto{\pgfqpoint{3.997466in}{2.224796in}}%
\pgfpathlineto{\pgfqpoint{4.011026in}{2.222381in}}%
\pgfpathlineto{\pgfqpoint{4.003160in}{2.213521in}}%
\pgfpathlineto{\pgfqpoint{3.995289in}{2.204678in}}%
\pgfpathlineto{\pgfqpoint{3.987412in}{2.195852in}}%
\pgfpathlineto{\pgfqpoint{3.979530in}{2.187047in}}%
\pgfpathlineto{\pgfqpoint{3.965959in}{2.189653in}}%
\pgfpathlineto{\pgfqpoint{3.952395in}{2.192371in}}%
\pgfpathlineto{\pgfqpoint{3.938837in}{2.195201in}}%
\pgfpathlineto{\pgfqpoint{3.925286in}{2.198144in}}%
\pgfpathlineto{\pgfqpoint{3.933180in}{2.206751in}}%
\pgfpathlineto{\pgfqpoint{3.941069in}{2.215383in}}%
\pgfpathlineto{\pgfqpoint{3.948951in}{2.224038in}}%
\pgfpathlineto{\pgfqpoint{3.956829in}{2.232713in}}%
\pgfpathclose%
\pgfusepath{fill}%
\end{pgfscope}%
\begin{pgfscope}%
\pgfpathrectangle{\pgfqpoint{1.150000in}{0.150000in}}{\pgfqpoint{5.700000in}{5.700000in}}%
\pgfusepath{clip}%
\pgfsetbuttcap%
\pgfsetroundjoin%
\definecolor{currentfill}{rgb}{0.282327,0.094955,0.417331}%
\pgfsetfillcolor{currentfill}%
\pgfsetfillopacity{0.700000}%
\pgfsetlinewidth{0.000000pt}%
\definecolor{currentstroke}{rgb}{0.000000,0.000000,0.000000}%
\pgfsetstrokecolor{currentstroke}%
\pgfsetdash{}{0pt}%
\pgfpathmoveto{\pgfqpoint{4.353443in}{2.350893in}}%
\pgfpathlineto{\pgfqpoint{4.367094in}{2.350915in}}%
\pgfpathlineto{\pgfqpoint{4.380755in}{2.351043in}}%
\pgfpathlineto{\pgfqpoint{4.394424in}{2.351277in}}%
\pgfpathlineto{\pgfqpoint{4.408102in}{2.351617in}}%
\pgfpathlineto{\pgfqpoint{4.400369in}{2.342116in}}%
\pgfpathlineto{\pgfqpoint{4.392631in}{2.332589in}}%
\pgfpathlineto{\pgfqpoint{4.384888in}{2.323037in}}%
\pgfpathlineto{\pgfqpoint{4.377139in}{2.313461in}}%
\pgfpathlineto{\pgfqpoint{4.363453in}{2.313241in}}%
\pgfpathlineto{\pgfqpoint{4.349775in}{2.313126in}}%
\pgfpathlineto{\pgfqpoint{4.336107in}{2.313117in}}%
\pgfpathlineto{\pgfqpoint{4.322447in}{2.313215in}}%
\pgfpathlineto{\pgfqpoint{4.330204in}{2.322665in}}%
\pgfpathlineto{\pgfqpoint{4.337955in}{2.332095in}}%
\pgfpathlineto{\pgfqpoint{4.345702in}{2.341505in}}%
\pgfpathlineto{\pgfqpoint{4.353443in}{2.350893in}}%
\pgfpathclose%
\pgfusepath{fill}%
\end{pgfscope}%
\begin{pgfscope}%
\pgfpathrectangle{\pgfqpoint{1.150000in}{0.150000in}}{\pgfqpoint{5.700000in}{5.700000in}}%
\pgfusepath{clip}%
\pgfsetbuttcap%
\pgfsetroundjoin%
\definecolor{currentfill}{rgb}{0.216210,0.351535,0.550627}%
\pgfsetfillcolor{currentfill}%
\pgfsetfillopacity{0.700000}%
\pgfsetlinewidth{0.000000pt}%
\definecolor{currentstroke}{rgb}{0.000000,0.000000,0.000000}%
\pgfsetstrokecolor{currentstroke}%
\pgfsetdash{}{0pt}%
\pgfpathmoveto{\pgfqpoint{5.380168in}{2.888217in}}%
\pgfpathlineto{\pgfqpoint{5.394239in}{2.893136in}}%
\pgfpathlineto{\pgfqpoint{5.408324in}{2.898154in}}%
\pgfpathlineto{\pgfqpoint{5.422422in}{2.903272in}}%
\pgfpathlineto{\pgfqpoint{5.436534in}{2.908488in}}%
\pgfpathlineto{\pgfqpoint{5.429209in}{2.901663in}}%
\pgfpathlineto{\pgfqpoint{5.421876in}{2.894769in}}%
\pgfpathlineto{\pgfqpoint{5.414537in}{2.887806in}}%
\pgfpathlineto{\pgfqpoint{5.407190in}{2.880771in}}%
\pgfpathlineto{\pgfqpoint{5.393066in}{2.875451in}}%
\pgfpathlineto{\pgfqpoint{5.378956in}{2.870230in}}%
\pgfpathlineto{\pgfqpoint{5.364858in}{2.865108in}}%
\pgfpathlineto{\pgfqpoint{5.350775in}{2.860085in}}%
\pgfpathlineto{\pgfqpoint{5.358133in}{2.867217in}}%
\pgfpathlineto{\pgfqpoint{5.365485in}{2.874282in}}%
\pgfpathlineto{\pgfqpoint{5.372830in}{2.881281in}}%
\pgfpathlineto{\pgfqpoint{5.380168in}{2.888217in}}%
\pgfpathclose%
\pgfusepath{fill}%
\end{pgfscope}%
\begin{pgfscope}%
\pgfpathrectangle{\pgfqpoint{1.150000in}{0.150000in}}{\pgfqpoint{5.700000in}{5.700000in}}%
\pgfusepath{clip}%
\pgfsetbuttcap%
\pgfsetroundjoin%
\definecolor{currentfill}{rgb}{0.208623,0.367752,0.552675}%
\pgfsetfillcolor{currentfill}%
\pgfsetfillopacity{0.700000}%
\pgfsetlinewidth{0.000000pt}%
\definecolor{currentstroke}{rgb}{0.000000,0.000000,0.000000}%
\pgfsetstrokecolor{currentstroke}%
\pgfsetdash{}{0pt}%
\pgfpathmoveto{\pgfqpoint{5.465762in}{2.935134in}}%
\pgfpathlineto{\pgfqpoint{5.479873in}{2.940327in}}%
\pgfpathlineto{\pgfqpoint{5.493998in}{2.945618in}}%
\pgfpathlineto{\pgfqpoint{5.508137in}{2.951009in}}%
\pgfpathlineto{\pgfqpoint{5.522290in}{2.956498in}}%
\pgfpathlineto{\pgfqpoint{5.515007in}{2.950061in}}%
\pgfpathlineto{\pgfqpoint{5.507717in}{2.943557in}}%
\pgfpathlineto{\pgfqpoint{5.500420in}{2.936985in}}%
\pgfpathlineto{\pgfqpoint{5.493116in}{2.930343in}}%
\pgfpathlineto{\pgfqpoint{5.478950in}{2.924731in}}%
\pgfpathlineto{\pgfqpoint{5.464797in}{2.919217in}}%
\pgfpathlineto{\pgfqpoint{5.450659in}{2.913803in}}%
\pgfpathlineto{\pgfqpoint{5.436534in}{2.908488in}}%
\pgfpathlineto{\pgfqpoint{5.443851in}{2.915246in}}%
\pgfpathlineto{\pgfqpoint{5.451162in}{2.921939in}}%
\pgfpathlineto{\pgfqpoint{5.458465in}{2.928568in}}%
\pgfpathlineto{\pgfqpoint{5.465762in}{2.935134in}}%
\pgfpathclose%
\pgfusepath{fill}%
\end{pgfscope}%
\begin{pgfscope}%
\pgfpathrectangle{\pgfqpoint{1.150000in}{0.150000in}}{\pgfqpoint{5.700000in}{5.700000in}}%
\pgfusepath{clip}%
\pgfsetbuttcap%
\pgfsetroundjoin%
\definecolor{currentfill}{rgb}{0.268510,0.009605,0.335427}%
\pgfsetfillcolor{currentfill}%
\pgfsetfillopacity{0.700000}%
\pgfsetlinewidth{0.000000pt}%
\definecolor{currentstroke}{rgb}{0.000000,0.000000,0.000000}%
\pgfsetstrokecolor{currentstroke}%
\pgfsetdash{}{0pt}%
\pgfpathmoveto{\pgfqpoint{3.731336in}{2.210780in}}%
\pgfpathlineto{\pgfqpoint{3.744833in}{2.206239in}}%
\pgfpathlineto{\pgfqpoint{3.758335in}{2.201817in}}%
\pgfpathlineto{\pgfqpoint{3.771843in}{2.197513in}}%
\pgfpathlineto{\pgfqpoint{3.785355in}{2.193325in}}%
\pgfpathlineto{\pgfqpoint{3.777406in}{2.185309in}}%
\pgfpathlineto{\pgfqpoint{3.769450in}{2.177337in}}%
\pgfpathlineto{\pgfqpoint{3.761489in}{2.169412in}}%
\pgfpathlineto{\pgfqpoint{3.753521in}{2.161536in}}%
\pgfpathlineto{\pgfqpoint{3.739994in}{2.165950in}}%
\pgfpathlineto{\pgfqpoint{3.726473in}{2.170482in}}%
\pgfpathlineto{\pgfqpoint{3.712956in}{2.175132in}}%
\pgfpathlineto{\pgfqpoint{3.699444in}{2.179901in}}%
\pgfpathlineto{\pgfqpoint{3.707427in}{2.187542in}}%
\pgfpathlineto{\pgfqpoint{3.715403in}{2.195238in}}%
\pgfpathlineto{\pgfqpoint{3.723373in}{2.202984in}}%
\pgfpathlineto{\pgfqpoint{3.731336in}{2.210780in}}%
\pgfpathclose%
\pgfusepath{fill}%
\end{pgfscope}%
\begin{pgfscope}%
\pgfpathrectangle{\pgfqpoint{1.150000in}{0.150000in}}{\pgfqpoint{5.700000in}{5.700000in}}%
\pgfusepath{clip}%
\pgfsetbuttcap%
\pgfsetroundjoin%
\definecolor{currentfill}{rgb}{0.273809,0.031497,0.358853}%
\pgfsetfillcolor{currentfill}%
\pgfsetfillopacity{0.700000}%
\pgfsetlinewidth{0.000000pt}%
\definecolor{currentstroke}{rgb}{0.000000,0.000000,0.000000}%
\pgfsetstrokecolor{currentstroke}%
\pgfsetdash{}{0pt}%
\pgfpathmoveto{\pgfqpoint{3.451562in}{2.246712in}}%
\pgfpathlineto{\pgfqpoint{3.465029in}{2.239723in}}%
\pgfpathlineto{\pgfqpoint{3.478499in}{2.232862in}}%
\pgfpathlineto{\pgfqpoint{3.491972in}{2.226129in}}%
\pgfpathlineto{\pgfqpoint{3.505448in}{2.219523in}}%
\pgfpathlineto{\pgfqpoint{3.497382in}{2.212954in}}%
\pgfpathlineto{\pgfqpoint{3.489308in}{2.206465in}}%
\pgfpathlineto{\pgfqpoint{3.481228in}{2.200060in}}%
\pgfpathlineto{\pgfqpoint{3.473139in}{2.193740in}}%
\pgfpathlineto{\pgfqpoint{3.459645in}{2.200611in}}%
\pgfpathlineto{\pgfqpoint{3.446153in}{2.207610in}}%
\pgfpathlineto{\pgfqpoint{3.432664in}{2.214736in}}%
\pgfpathlineto{\pgfqpoint{3.419177in}{2.221990in}}%
\pgfpathlineto{\pgfqpoint{3.427285in}{2.228037in}}%
\pgfpathlineto{\pgfqpoint{3.435385in}{2.234175in}}%
\pgfpathlineto{\pgfqpoint{3.443477in}{2.240401in}}%
\pgfpathlineto{\pgfqpoint{3.451562in}{2.246712in}}%
\pgfpathclose%
\pgfusepath{fill}%
\end{pgfscope}%
\begin{pgfscope}%
\pgfpathrectangle{\pgfqpoint{1.150000in}{0.150000in}}{\pgfqpoint{5.700000in}{5.700000in}}%
\pgfusepath{clip}%
\pgfsetbuttcap%
\pgfsetroundjoin%
\definecolor{currentfill}{rgb}{0.199430,0.387607,0.554642}%
\pgfsetfillcolor{currentfill}%
\pgfsetfillopacity{0.700000}%
\pgfsetlinewidth{0.000000pt}%
\definecolor{currentstroke}{rgb}{0.000000,0.000000,0.000000}%
\pgfsetstrokecolor{currentstroke}%
\pgfsetdash{}{0pt}%
\pgfpathmoveto{\pgfqpoint{5.551347in}{2.981609in}}%
\pgfpathlineto{\pgfqpoint{5.565498in}{2.987055in}}%
\pgfpathlineto{\pgfqpoint{5.579664in}{2.992599in}}%
\pgfpathlineto{\pgfqpoint{5.593844in}{2.998242in}}%
\pgfpathlineto{\pgfqpoint{5.608037in}{3.003984in}}%
\pgfpathlineto{\pgfqpoint{5.600799in}{2.997947in}}%
\pgfpathlineto{\pgfqpoint{5.593553in}{2.991845in}}%
\pgfpathlineto{\pgfqpoint{5.586300in}{2.985677in}}%
\pgfpathlineto{\pgfqpoint{5.579039in}{2.979441in}}%
\pgfpathlineto{\pgfqpoint{5.564831in}{2.973557in}}%
\pgfpathlineto{\pgfqpoint{5.550636in}{2.967772in}}%
\pgfpathlineto{\pgfqpoint{5.536456in}{2.962085in}}%
\pgfpathlineto{\pgfqpoint{5.522290in}{2.956498in}}%
\pgfpathlineto{\pgfqpoint{5.529565in}{2.962869in}}%
\pgfpathlineto{\pgfqpoint{5.536833in}{2.969177in}}%
\pgfpathlineto{\pgfqpoint{5.544093in}{2.975423in}}%
\pgfpathlineto{\pgfqpoint{5.551347in}{2.981609in}}%
\pgfpathclose%
\pgfusepath{fill}%
\end{pgfscope}%
\begin{pgfscope}%
\pgfpathrectangle{\pgfqpoint{1.150000in}{0.150000in}}{\pgfqpoint{5.700000in}{5.700000in}}%
\pgfusepath{clip}%
\pgfsetbuttcap%
\pgfsetroundjoin%
\definecolor{currentfill}{rgb}{0.168126,0.459988,0.558082}%
\pgfsetfillcolor{currentfill}%
\pgfsetfillopacity{0.700000}%
\pgfsetlinewidth{0.000000pt}%
\definecolor{currentstroke}{rgb}{0.000000,0.000000,0.000000}%
\pgfsetstrokecolor{currentstroke}%
\pgfsetdash{}{0pt}%
\pgfpathmoveto{\pgfqpoint{5.893481in}{3.160889in}}%
\pgfpathlineto{\pgfqpoint{5.907793in}{3.167145in}}%
\pgfpathlineto{\pgfqpoint{5.922119in}{3.173498in}}%
\pgfpathlineto{\pgfqpoint{5.936461in}{3.179949in}}%
\pgfpathlineto{\pgfqpoint{5.950817in}{3.186497in}}%
\pgfpathlineto{\pgfqpoint{5.943770in}{3.182111in}}%
\pgfpathlineto{\pgfqpoint{5.936715in}{3.177678in}}%
\pgfpathlineto{\pgfqpoint{5.929653in}{3.173194in}}%
\pgfpathlineto{\pgfqpoint{5.922583in}{3.168657in}}%
\pgfpathlineto{\pgfqpoint{5.908207in}{3.161889in}}%
\pgfpathlineto{\pgfqpoint{5.893845in}{3.155219in}}%
\pgfpathlineto{\pgfqpoint{5.879499in}{3.148646in}}%
\pgfpathlineto{\pgfqpoint{5.865167in}{3.142171in}}%
\pgfpathlineto{\pgfqpoint{5.872257in}{3.146921in}}%
\pgfpathlineto{\pgfqpoint{5.879339in}{3.151622in}}%
\pgfpathlineto{\pgfqpoint{5.886414in}{3.156277in}}%
\pgfpathlineto{\pgfqpoint{5.893481in}{3.160889in}}%
\pgfpathclose%
\pgfusepath{fill}%
\end{pgfscope}%
\begin{pgfscope}%
\pgfpathrectangle{\pgfqpoint{1.150000in}{0.150000in}}{\pgfqpoint{5.700000in}{5.700000in}}%
\pgfusepath{clip}%
\pgfsetbuttcap%
\pgfsetroundjoin%
\definecolor{currentfill}{rgb}{0.190631,0.407061,0.556089}%
\pgfsetfillcolor{currentfill}%
\pgfsetfillopacity{0.700000}%
\pgfsetlinewidth{0.000000pt}%
\definecolor{currentstroke}{rgb}{0.000000,0.000000,0.000000}%
\pgfsetstrokecolor{currentstroke}%
\pgfsetdash{}{0pt}%
\pgfpathmoveto{\pgfqpoint{5.636917in}{3.027522in}}%
\pgfpathlineto{\pgfqpoint{5.651109in}{3.033201in}}%
\pgfpathlineto{\pgfqpoint{5.665315in}{3.038978in}}%
\pgfpathlineto{\pgfqpoint{5.679536in}{3.044853in}}%
\pgfpathlineto{\pgfqpoint{5.693771in}{3.050827in}}%
\pgfpathlineto{\pgfqpoint{5.686578in}{3.045198in}}%
\pgfpathlineto{\pgfqpoint{5.679378in}{3.039508in}}%
\pgfpathlineto{\pgfqpoint{5.672170in}{3.033754in}}%
\pgfpathlineto{\pgfqpoint{5.664954in}{3.027934in}}%
\pgfpathlineto{\pgfqpoint{5.650704in}{3.021799in}}%
\pgfpathlineto{\pgfqpoint{5.636467in}{3.015762in}}%
\pgfpathlineto{\pgfqpoint{5.622245in}{3.009824in}}%
\pgfpathlineto{\pgfqpoint{5.608037in}{3.003984in}}%
\pgfpathlineto{\pgfqpoint{5.615268in}{3.009958in}}%
\pgfpathlineto{\pgfqpoint{5.622492in}{3.015871in}}%
\pgfpathlineto{\pgfqpoint{5.629708in}{3.021725in}}%
\pgfpathlineto{\pgfqpoint{5.636917in}{3.027522in}}%
\pgfpathclose%
\pgfusepath{fill}%
\end{pgfscope}%
\begin{pgfscope}%
\pgfpathrectangle{\pgfqpoint{1.150000in}{0.150000in}}{\pgfqpoint{5.700000in}{5.700000in}}%
\pgfusepath{clip}%
\pgfsetbuttcap%
\pgfsetroundjoin%
\definecolor{currentfill}{rgb}{0.175841,0.441290,0.557685}%
\pgfsetfillcolor{currentfill}%
\pgfsetfillopacity{0.700000}%
\pgfsetlinewidth{0.000000pt}%
\definecolor{currentstroke}{rgb}{0.000000,0.000000,0.000000}%
\pgfsetstrokecolor{currentstroke}%
\pgfsetdash{}{0pt}%
\pgfpathmoveto{\pgfqpoint{5.807991in}{3.117250in}}%
\pgfpathlineto{\pgfqpoint{5.822263in}{3.123334in}}%
\pgfpathlineto{\pgfqpoint{5.836549in}{3.129515in}}%
\pgfpathlineto{\pgfqpoint{5.850851in}{3.135794in}}%
\pgfpathlineto{\pgfqpoint{5.865167in}{3.142171in}}%
\pgfpathlineto{\pgfqpoint{5.858070in}{3.137371in}}%
\pgfpathlineto{\pgfqpoint{5.850966in}{3.132518in}}%
\pgfpathlineto{\pgfqpoint{5.843853in}{3.127609in}}%
\pgfpathlineto{\pgfqpoint{5.836734in}{3.122643in}}%
\pgfpathlineto{\pgfqpoint{5.822399in}{3.116065in}}%
\pgfpathlineto{\pgfqpoint{5.808078in}{3.109586in}}%
\pgfpathlineto{\pgfqpoint{5.793773in}{3.103204in}}%
\pgfpathlineto{\pgfqpoint{5.779483in}{3.096921in}}%
\pgfpathlineto{\pgfqpoint{5.786621in}{3.102081in}}%
\pgfpathlineto{\pgfqpoint{5.793751in}{3.107187in}}%
\pgfpathlineto{\pgfqpoint{5.800875in}{3.112243in}}%
\pgfpathlineto{\pgfqpoint{5.807991in}{3.117250in}}%
\pgfpathclose%
\pgfusepath{fill}%
\end{pgfscope}%
\begin{pgfscope}%
\pgfpathrectangle{\pgfqpoint{1.150000in}{0.150000in}}{\pgfqpoint{5.700000in}{5.700000in}}%
\pgfusepath{clip}%
\pgfsetbuttcap%
\pgfsetroundjoin%
\definecolor{currentfill}{rgb}{0.182256,0.426184,0.557120}%
\pgfsetfillcolor{currentfill}%
\pgfsetfillopacity{0.700000}%
\pgfsetlinewidth{0.000000pt}%
\definecolor{currentstroke}{rgb}{0.000000,0.000000,0.000000}%
\pgfsetstrokecolor{currentstroke}%
\pgfsetdash{}{0pt}%
\pgfpathmoveto{\pgfqpoint{5.722467in}{3.072767in}}%
\pgfpathlineto{\pgfqpoint{5.736699in}{3.078659in}}%
\pgfpathlineto{\pgfqpoint{5.750946in}{3.084648in}}%
\pgfpathlineto{\pgfqpoint{5.765207in}{3.090735in}}%
\pgfpathlineto{\pgfqpoint{5.779483in}{3.096921in}}%
\pgfpathlineto{\pgfqpoint{5.772337in}{3.091705in}}%
\pgfpathlineto{\pgfqpoint{5.765184in}{3.086432in}}%
\pgfpathlineto{\pgfqpoint{5.758023in}{3.081099in}}%
\pgfpathlineto{\pgfqpoint{5.750855in}{3.075704in}}%
\pgfpathlineto{\pgfqpoint{5.736562in}{3.069337in}}%
\pgfpathlineto{\pgfqpoint{5.722284in}{3.063069in}}%
\pgfpathlineto{\pgfqpoint{5.708020in}{3.056899in}}%
\pgfpathlineto{\pgfqpoint{5.693771in}{3.050827in}}%
\pgfpathlineto{\pgfqpoint{5.700956in}{3.056396in}}%
\pgfpathlineto{\pgfqpoint{5.708134in}{3.061908in}}%
\pgfpathlineto{\pgfqpoint{5.715304in}{3.067364in}}%
\pgfpathlineto{\pgfqpoint{5.722467in}{3.072767in}}%
\pgfpathclose%
\pgfusepath{fill}%
\end{pgfscope}%
\begin{pgfscope}%
\pgfpathrectangle{\pgfqpoint{1.150000in}{0.150000in}}{\pgfqpoint{5.700000in}{5.700000in}}%
\pgfusepath{clip}%
\pgfsetbuttcap%
\pgfsetroundjoin%
\definecolor{currentfill}{rgb}{0.280267,0.073417,0.397163}%
\pgfsetfillcolor{currentfill}%
\pgfsetfillopacity{0.700000}%
\pgfsetlinewidth{0.000000pt}%
\definecolor{currentstroke}{rgb}{0.000000,0.000000,0.000000}%
\pgfsetstrokecolor{currentstroke}%
\pgfsetdash{}{0pt}%
\pgfpathmoveto{\pgfqpoint{4.267891in}{2.314672in}}%
\pgfpathlineto{\pgfqpoint{4.281517in}{2.314147in}}%
\pgfpathlineto{\pgfqpoint{4.295152in}{2.313730in}}%
\pgfpathlineto{\pgfqpoint{4.308795in}{2.313419in}}%
\pgfpathlineto{\pgfqpoint{4.322447in}{2.313215in}}%
\pgfpathlineto{\pgfqpoint{4.314684in}{2.303746in}}%
\pgfpathlineto{\pgfqpoint{4.306917in}{2.294259in}}%
\pgfpathlineto{\pgfqpoint{4.299144in}{2.284756in}}%
\pgfpathlineto{\pgfqpoint{4.291365in}{2.275238in}}%
\pgfpathlineto{\pgfqpoint{4.277705in}{2.275579in}}%
\pgfpathlineto{\pgfqpoint{4.264053in}{2.276027in}}%
\pgfpathlineto{\pgfqpoint{4.250410in}{2.276582in}}%
\pgfpathlineto{\pgfqpoint{4.236774in}{2.277245in}}%
\pgfpathlineto{\pgfqpoint{4.244561in}{2.286619in}}%
\pgfpathlineto{\pgfqpoint{4.252343in}{2.295983in}}%
\pgfpathlineto{\pgfqpoint{4.260120in}{2.305334in}}%
\pgfpathlineto{\pgfqpoint{4.267891in}{2.314672in}}%
\pgfpathclose%
\pgfusepath{fill}%
\end{pgfscope}%
\begin{pgfscope}%
\pgfpathrectangle{\pgfqpoint{1.150000in}{0.150000in}}{\pgfqpoint{5.700000in}{5.700000in}}%
\pgfusepath{clip}%
\pgfsetbuttcap%
\pgfsetroundjoin%
\definecolor{currentfill}{rgb}{0.283187,0.125848,0.444960}%
\pgfsetfillcolor{currentfill}%
\pgfsetfillopacity{0.700000}%
\pgfsetlinewidth{0.000000pt}%
\definecolor{currentstroke}{rgb}{0.000000,0.000000,0.000000}%
\pgfsetstrokecolor{currentstroke}%
\pgfsetdash{}{0pt}%
\pgfpathmoveto{\pgfqpoint{3.062975in}{2.420307in}}%
\pgfpathlineto{\pgfqpoint{3.076456in}{2.409436in}}%
\pgfpathlineto{\pgfqpoint{3.089937in}{2.398714in}}%
\pgfpathlineto{\pgfqpoint{3.103417in}{2.388141in}}%
\pgfpathlineto{\pgfqpoint{3.116897in}{2.377716in}}%
\pgfpathlineto{\pgfqpoint{3.108634in}{2.373645in}}%
\pgfpathlineto{\pgfqpoint{3.100361in}{2.369703in}}%
\pgfpathlineto{\pgfqpoint{3.092078in}{2.365895in}}%
\pgfpathlineto{\pgfqpoint{3.083785in}{2.362221in}}%
\pgfpathlineto{\pgfqpoint{3.070279in}{2.372953in}}%
\pgfpathlineto{\pgfqpoint{3.056772in}{2.383833in}}%
\pgfpathlineto{\pgfqpoint{3.043266in}{2.394863in}}%
\pgfpathlineto{\pgfqpoint{3.029758in}{2.406042in}}%
\pgfpathlineto{\pgfqpoint{3.038078in}{2.409401in}}%
\pgfpathlineto{\pgfqpoint{3.046387in}{2.412900in}}%
\pgfpathlineto{\pgfqpoint{3.054686in}{2.416536in}}%
\pgfpathlineto{\pgfqpoint{3.062975in}{2.420307in}}%
\pgfpathclose%
\pgfusepath{fill}%
\end{pgfscope}%
\begin{pgfscope}%
\pgfpathrectangle{\pgfqpoint{1.150000in}{0.150000in}}{\pgfqpoint{5.700000in}{5.700000in}}%
\pgfusepath{clip}%
\pgfsetbuttcap%
\pgfsetroundjoin%
\definecolor{currentfill}{rgb}{0.277941,0.056324,0.381191}%
\pgfsetfillcolor{currentfill}%
\pgfsetfillopacity{0.700000}%
\pgfsetlinewidth{0.000000pt}%
\definecolor{currentstroke}{rgb}{0.000000,0.000000,0.000000}%
\pgfsetstrokecolor{currentstroke}%
\pgfsetdash{}{0pt}%
\pgfpathmoveto{\pgfqpoint{4.182313in}{2.280973in}}%
\pgfpathlineto{\pgfqpoint{4.195916in}{2.279879in}}%
\pgfpathlineto{\pgfqpoint{4.209528in}{2.278893in}}%
\pgfpathlineto{\pgfqpoint{4.223147in}{2.278015in}}%
\pgfpathlineto{\pgfqpoint{4.236774in}{2.277245in}}%
\pgfpathlineto{\pgfqpoint{4.228982in}{2.267860in}}%
\pgfpathlineto{\pgfqpoint{4.221184in}{2.258468in}}%
\pgfpathlineto{\pgfqpoint{4.213381in}{2.249068in}}%
\pgfpathlineto{\pgfqpoint{4.205573in}{2.239662in}}%
\pgfpathlineto{\pgfqpoint{4.191936in}{2.240587in}}%
\pgfpathlineto{\pgfqpoint{4.178308in}{2.241621in}}%
\pgfpathlineto{\pgfqpoint{4.164687in}{2.242762in}}%
\pgfpathlineto{\pgfqpoint{4.151073in}{2.244012in}}%
\pgfpathlineto{\pgfqpoint{4.158891in}{2.253256in}}%
\pgfpathlineto{\pgfqpoint{4.166704in}{2.262498in}}%
\pgfpathlineto{\pgfqpoint{4.174511in}{2.271738in}}%
\pgfpathlineto{\pgfqpoint{4.182313in}{2.280973in}}%
\pgfpathclose%
\pgfusepath{fill}%
\end{pgfscope}%
\begin{pgfscope}%
\pgfpathrectangle{\pgfqpoint{1.150000in}{0.150000in}}{\pgfqpoint{5.700000in}{5.700000in}}%
\pgfusepath{clip}%
\pgfsetbuttcap%
\pgfsetroundjoin%
\definecolor{currentfill}{rgb}{0.269944,0.014625,0.341379}%
\pgfsetfillcolor{currentfill}%
\pgfsetfillopacity{0.700000}%
\pgfsetlinewidth{0.000000pt}%
\definecolor{currentstroke}{rgb}{0.000000,0.000000,0.000000}%
\pgfsetstrokecolor{currentstroke}%
\pgfsetdash{}{0pt}%
\pgfpathmoveto{\pgfqpoint{3.871142in}{2.211052in}}%
\pgfpathlineto{\pgfqpoint{3.884669in}{2.207654in}}%
\pgfpathlineto{\pgfqpoint{3.898202in}{2.204370in}}%
\pgfpathlineto{\pgfqpoint{3.911741in}{2.201200in}}%
\pgfpathlineto{\pgfqpoint{3.925286in}{2.198144in}}%
\pgfpathlineto{\pgfqpoint{3.917386in}{2.189563in}}%
\pgfpathlineto{\pgfqpoint{3.909481in}{2.181010in}}%
\pgfpathlineto{\pgfqpoint{3.901570in}{2.172486in}}%
\pgfpathlineto{\pgfqpoint{3.893653in}{2.163995in}}%
\pgfpathlineto{\pgfqpoint{3.880095in}{2.167260in}}%
\pgfpathlineto{\pgfqpoint{3.866544in}{2.170640in}}%
\pgfpathlineto{\pgfqpoint{3.852998in}{2.174133in}}%
\pgfpathlineto{\pgfqpoint{3.839458in}{2.177740in}}%
\pgfpathlineto{\pgfqpoint{3.847388in}{2.186016in}}%
\pgfpathlineto{\pgfqpoint{3.855312in}{2.194328in}}%
\pgfpathlineto{\pgfqpoint{3.863230in}{2.202674in}}%
\pgfpathlineto{\pgfqpoint{3.871142in}{2.211052in}}%
\pgfpathclose%
\pgfusepath{fill}%
\end{pgfscope}%
\begin{pgfscope}%
\pgfpathrectangle{\pgfqpoint{1.150000in}{0.150000in}}{\pgfqpoint{5.700000in}{5.700000in}}%
\pgfusepath{clip}%
\pgfsetbuttcap%
\pgfsetroundjoin%
\definecolor{currentfill}{rgb}{0.277941,0.056324,0.381191}%
\pgfsetfillcolor{currentfill}%
\pgfsetfillopacity{0.700000}%
\pgfsetlinewidth{0.000000pt}%
\definecolor{currentstroke}{rgb}{0.000000,0.000000,0.000000}%
\pgfsetstrokecolor{currentstroke}%
\pgfsetdash{}{0pt}%
\pgfpathmoveto{\pgfqpoint{3.311373in}{2.284734in}}%
\pgfpathlineto{\pgfqpoint{3.324841in}{2.276426in}}%
\pgfpathlineto{\pgfqpoint{3.338311in}{2.268252in}}%
\pgfpathlineto{\pgfqpoint{3.351783in}{2.260212in}}%
\pgfpathlineto{\pgfqpoint{3.365258in}{2.252305in}}%
\pgfpathlineto{\pgfqpoint{3.357122in}{2.246628in}}%
\pgfpathlineto{\pgfqpoint{3.348978in}{2.241050in}}%
\pgfpathlineto{\pgfqpoint{3.340826in}{2.235575in}}%
\pgfpathlineto{\pgfqpoint{3.332666in}{2.230205in}}%
\pgfpathlineto{\pgfqpoint{3.319170in}{2.238397in}}%
\pgfpathlineto{\pgfqpoint{3.305677in}{2.246722in}}%
\pgfpathlineto{\pgfqpoint{3.292185in}{2.255180in}}%
\pgfpathlineto{\pgfqpoint{3.278695in}{2.263774in}}%
\pgfpathlineto{\pgfqpoint{3.286877in}{2.268852in}}%
\pgfpathlineto{\pgfqpoint{3.295051in}{2.274039in}}%
\pgfpathlineto{\pgfqpoint{3.303216in}{2.279335in}}%
\pgfpathlineto{\pgfqpoint{3.311373in}{2.284734in}}%
\pgfpathclose%
\pgfusepath{fill}%
\end{pgfscope}%
\begin{pgfscope}%
\pgfpathrectangle{\pgfqpoint{1.150000in}{0.150000in}}{\pgfqpoint{5.700000in}{5.700000in}}%
\pgfusepath{clip}%
\pgfsetbuttcap%
\pgfsetroundjoin%
\definecolor{currentfill}{rgb}{0.274952,0.037752,0.364543}%
\pgfsetfillcolor{currentfill}%
\pgfsetfillopacity{0.700000}%
\pgfsetlinewidth{0.000000pt}%
\definecolor{currentstroke}{rgb}{0.000000,0.000000,0.000000}%
\pgfsetstrokecolor{currentstroke}%
\pgfsetdash{}{0pt}%
\pgfpathmoveto{\pgfqpoint{4.096696in}{2.250103in}}%
\pgfpathlineto{\pgfqpoint{4.110279in}{2.248416in}}%
\pgfpathlineto{\pgfqpoint{4.123870in}{2.246838in}}%
\pgfpathlineto{\pgfqpoint{4.137468in}{2.245371in}}%
\pgfpathlineto{\pgfqpoint{4.151073in}{2.244012in}}%
\pgfpathlineto{\pgfqpoint{4.143250in}{2.234768in}}%
\pgfpathlineto{\pgfqpoint{4.135422in}{2.225526in}}%
\pgfpathlineto{\pgfqpoint{4.127588in}{2.216286in}}%
\pgfpathlineto{\pgfqpoint{4.119749in}{2.207051in}}%
\pgfpathlineto{\pgfqpoint{4.106133in}{2.208583in}}%
\pgfpathlineto{\pgfqpoint{4.092525in}{2.210224in}}%
\pgfpathlineto{\pgfqpoint{4.078924in}{2.211975in}}%
\pgfpathlineto{\pgfqpoint{4.065330in}{2.213835in}}%
\pgfpathlineto{\pgfqpoint{4.073180in}{2.222890in}}%
\pgfpathlineto{\pgfqpoint{4.081024in}{2.231954in}}%
\pgfpathlineto{\pgfqpoint{4.088862in}{2.241025in}}%
\pgfpathlineto{\pgfqpoint{4.096696in}{2.250103in}}%
\pgfpathclose%
\pgfusepath{fill}%
\end{pgfscope}%
\begin{pgfscope}%
\pgfpathrectangle{\pgfqpoint{1.150000in}{0.150000in}}{\pgfqpoint{5.700000in}{5.700000in}}%
\pgfusepath{clip}%
\pgfsetbuttcap%
\pgfsetroundjoin%
\definecolor{currentfill}{rgb}{0.283091,0.110553,0.431554}%
\pgfsetfillcolor{currentfill}%
\pgfsetfillopacity{0.700000}%
\pgfsetlinewidth{0.000000pt}%
\definecolor{currentstroke}{rgb}{0.000000,0.000000,0.000000}%
\pgfsetstrokecolor{currentstroke}%
\pgfsetdash{}{0pt}%
\pgfpathmoveto{\pgfqpoint{3.116897in}{2.377716in}}%
\pgfpathlineto{\pgfqpoint{3.130377in}{2.367437in}}%
\pgfpathlineto{\pgfqpoint{3.143857in}{2.357304in}}%
\pgfpathlineto{\pgfqpoint{3.157337in}{2.347316in}}%
\pgfpathlineto{\pgfqpoint{3.170818in}{2.337471in}}%
\pgfpathlineto{\pgfqpoint{3.162580in}{2.333101in}}%
\pgfpathlineto{\pgfqpoint{3.154332in}{2.328856in}}%
\pgfpathlineto{\pgfqpoint{3.146075in}{2.324739in}}%
\pgfpathlineto{\pgfqpoint{3.137808in}{2.320753in}}%
\pgfpathlineto{\pgfqpoint{3.124302in}{2.330903in}}%
\pgfpathlineto{\pgfqpoint{3.110797in}{2.341197in}}%
\pgfpathlineto{\pgfqpoint{3.097291in}{2.351636in}}%
\pgfpathlineto{\pgfqpoint{3.083785in}{2.362221in}}%
\pgfpathlineto{\pgfqpoint{3.092078in}{2.365895in}}%
\pgfpathlineto{\pgfqpoint{3.100361in}{2.369703in}}%
\pgfpathlineto{\pgfqpoint{3.108634in}{2.373645in}}%
\pgfpathlineto{\pgfqpoint{3.116897in}{2.377716in}}%
\pgfpathclose%
\pgfusepath{fill}%
\end{pgfscope}%
\begin{pgfscope}%
\pgfpathrectangle{\pgfqpoint{1.150000in}{0.150000in}}{\pgfqpoint{5.700000in}{5.700000in}}%
\pgfusepath{clip}%
\pgfsetbuttcap%
\pgfsetroundjoin%
\definecolor{currentfill}{rgb}{0.268510,0.009605,0.335427}%
\pgfsetfillcolor{currentfill}%
\pgfsetfillopacity{0.700000}%
\pgfsetlinewidth{0.000000pt}%
\definecolor{currentstroke}{rgb}{0.000000,0.000000,0.000000}%
\pgfsetstrokecolor{currentstroke}%
\pgfsetdash{}{0pt}%
\pgfpathmoveto{\pgfqpoint{3.645443in}{2.200169in}}%
\pgfpathlineto{\pgfqpoint{3.658937in}{2.194921in}}%
\pgfpathlineto{\pgfqpoint{3.672435in}{2.189795in}}%
\pgfpathlineto{\pgfqpoint{3.685937in}{2.184788in}}%
\pgfpathlineto{\pgfqpoint{3.699444in}{2.179901in}}%
\pgfpathlineto{\pgfqpoint{3.691455in}{2.172314in}}%
\pgfpathlineto{\pgfqpoint{3.683460in}{2.164786in}}%
\pgfpathlineto{\pgfqpoint{3.675458in}{2.157317in}}%
\pgfpathlineto{\pgfqpoint{3.667450in}{2.149911in}}%
\pgfpathlineto{\pgfqpoint{3.653927in}{2.155044in}}%
\pgfpathlineto{\pgfqpoint{3.640409in}{2.160296in}}%
\pgfpathlineto{\pgfqpoint{3.626895in}{2.165669in}}%
\pgfpathlineto{\pgfqpoint{3.613386in}{2.171162in}}%
\pgfpathlineto{\pgfqpoint{3.621410in}{2.178316in}}%
\pgfpathlineto{\pgfqpoint{3.629428in}{2.185536in}}%
\pgfpathlineto{\pgfqpoint{3.637439in}{2.192821in}}%
\pgfpathlineto{\pgfqpoint{3.645443in}{2.200169in}}%
\pgfpathclose%
\pgfusepath{fill}%
\end{pgfscope}%
\begin{pgfscope}%
\pgfpathrectangle{\pgfqpoint{1.150000in}{0.150000in}}{\pgfqpoint{5.700000in}{5.700000in}}%
\pgfusepath{clip}%
\pgfsetbuttcap%
\pgfsetroundjoin%
\definecolor{currentfill}{rgb}{0.271305,0.019942,0.347269}%
\pgfsetfillcolor{currentfill}%
\pgfsetfillopacity{0.700000}%
\pgfsetlinewidth{0.000000pt}%
\definecolor{currentstroke}{rgb}{0.000000,0.000000,0.000000}%
\pgfsetstrokecolor{currentstroke}%
\pgfsetdash{}{0pt}%
\pgfpathmoveto{\pgfqpoint{3.505448in}{2.219523in}}%
\pgfpathlineto{\pgfqpoint{3.518928in}{2.213043in}}%
\pgfpathlineto{\pgfqpoint{3.532411in}{2.206688in}}%
\pgfpathlineto{\pgfqpoint{3.545897in}{2.200459in}}%
\pgfpathlineto{\pgfqpoint{3.559387in}{2.194354in}}%
\pgfpathlineto{\pgfqpoint{3.551339in}{2.187527in}}%
\pgfpathlineto{\pgfqpoint{3.543283in}{2.180776in}}%
\pgfpathlineto{\pgfqpoint{3.535221in}{2.174103in}}%
\pgfpathlineto{\pgfqpoint{3.527151in}{2.167512in}}%
\pgfpathlineto{\pgfqpoint{3.513643in}{2.173882in}}%
\pgfpathlineto{\pgfqpoint{3.500139in}{2.180376in}}%
\pgfpathlineto{\pgfqpoint{3.486637in}{2.186995in}}%
\pgfpathlineto{\pgfqpoint{3.473139in}{2.193740in}}%
\pgfpathlineto{\pgfqpoint{3.481228in}{2.200060in}}%
\pgfpathlineto{\pgfqpoint{3.489308in}{2.206465in}}%
\pgfpathlineto{\pgfqpoint{3.497382in}{2.212954in}}%
\pgfpathlineto{\pgfqpoint{3.505448in}{2.219523in}}%
\pgfpathclose%
\pgfusepath{fill}%
\end{pgfscope}%
\begin{pgfscope}%
\pgfpathrectangle{\pgfqpoint{1.150000in}{0.150000in}}{\pgfqpoint{5.700000in}{5.700000in}}%
\pgfusepath{clip}%
\pgfsetbuttcap%
\pgfsetroundjoin%
\definecolor{currentfill}{rgb}{0.239346,0.300855,0.540844}%
\pgfsetfillcolor{currentfill}%
\pgfsetfillopacity{0.700000}%
\pgfsetlinewidth{0.000000pt}%
\definecolor{currentstroke}{rgb}{0.000000,0.000000,0.000000}%
\pgfsetstrokecolor{currentstroke}%
\pgfsetdash{}{0pt}%
\pgfpathmoveto{\pgfqpoint{2.650690in}{2.785083in}}%
\pgfpathlineto{\pgfqpoint{2.664277in}{2.769173in}}%
\pgfpathlineto{\pgfqpoint{2.677859in}{2.753452in}}%
\pgfpathlineto{\pgfqpoint{2.691435in}{2.737920in}}%
\pgfpathlineto{\pgfqpoint{2.705007in}{2.722573in}}%
\pgfpathlineto{\pgfqpoint{2.696492in}{2.721302in}}%
\pgfpathlineto{\pgfqpoint{2.687964in}{2.720207in}}%
\pgfpathlineto{\pgfqpoint{2.679423in}{2.719292in}}%
\pgfpathlineto{\pgfqpoint{2.670868in}{2.718558in}}%
\pgfpathlineto{\pgfqpoint{2.657261in}{2.734242in}}%
\pgfpathlineto{\pgfqpoint{2.643649in}{2.750113in}}%
\pgfpathlineto{\pgfqpoint{2.630032in}{2.766172in}}%
\pgfpathlineto{\pgfqpoint{2.616410in}{2.782421in}}%
\pgfpathlineto{\pgfqpoint{2.625000in}{2.782809in}}%
\pgfpathlineto{\pgfqpoint{2.633577in}{2.783384in}}%
\pgfpathlineto{\pgfqpoint{2.642140in}{2.784143in}}%
\pgfpathlineto{\pgfqpoint{2.650690in}{2.785083in}}%
\pgfpathclose%
\pgfusepath{fill}%
\end{pgfscope}%
\begin{pgfscope}%
\pgfpathrectangle{\pgfqpoint{1.150000in}{0.150000in}}{\pgfqpoint{5.700000in}{5.700000in}}%
\pgfusepath{clip}%
\pgfsetbuttcap%
\pgfsetroundjoin%
\definecolor{currentfill}{rgb}{0.250425,0.274290,0.533103}%
\pgfsetfillcolor{currentfill}%
\pgfsetfillopacity{0.700000}%
\pgfsetlinewidth{0.000000pt}%
\definecolor{currentstroke}{rgb}{0.000000,0.000000,0.000000}%
\pgfsetstrokecolor{currentstroke}%
\pgfsetdash{}{0pt}%
\pgfpathmoveto{\pgfqpoint{2.705007in}{2.722573in}}%
\pgfpathlineto{\pgfqpoint{2.718575in}{2.707410in}}%
\pgfpathlineto{\pgfqpoint{2.732138in}{2.692430in}}%
\pgfpathlineto{\pgfqpoint{2.745697in}{2.677631in}}%
\pgfpathlineto{\pgfqpoint{2.759252in}{2.663011in}}%
\pgfpathlineto{\pgfqpoint{2.750770in}{2.661413in}}%
\pgfpathlineto{\pgfqpoint{2.742276in}{2.659984in}}%
\pgfpathlineto{\pgfqpoint{2.733769in}{2.658730in}}%
\pgfpathlineto{\pgfqpoint{2.725249in}{2.657653in}}%
\pgfpathlineto{\pgfqpoint{2.711661in}{2.672608in}}%
\pgfpathlineto{\pgfqpoint{2.698068in}{2.687743in}}%
\pgfpathlineto{\pgfqpoint{2.684470in}{2.703059in}}%
\pgfpathlineto{\pgfqpoint{2.670868in}{2.718558in}}%
\pgfpathlineto{\pgfqpoint{2.679423in}{2.719292in}}%
\pgfpathlineto{\pgfqpoint{2.687964in}{2.720207in}}%
\pgfpathlineto{\pgfqpoint{2.696492in}{2.721302in}}%
\pgfpathlineto{\pgfqpoint{2.705007in}{2.722573in}}%
\pgfpathclose%
\pgfusepath{fill}%
\end{pgfscope}%
\begin{pgfscope}%
\pgfpathrectangle{\pgfqpoint{1.150000in}{0.150000in}}{\pgfqpoint{5.700000in}{5.700000in}}%
\pgfusepath{clip}%
\pgfsetbuttcap%
\pgfsetroundjoin%
\definecolor{currentfill}{rgb}{0.227802,0.326594,0.546532}%
\pgfsetfillcolor{currentfill}%
\pgfsetfillopacity{0.700000}%
\pgfsetlinewidth{0.000000pt}%
\definecolor{currentstroke}{rgb}{0.000000,0.000000,0.000000}%
\pgfsetstrokecolor{currentstroke}%
\pgfsetdash{}{0pt}%
\pgfpathmoveto{\pgfqpoint{2.596288in}{2.850653in}}%
\pgfpathlineto{\pgfqpoint{2.609897in}{2.833967in}}%
\pgfpathlineto{\pgfqpoint{2.623500in}{2.817478in}}%
\pgfpathlineto{\pgfqpoint{2.637098in}{2.801184in}}%
\pgfpathlineto{\pgfqpoint{2.650690in}{2.785083in}}%
\pgfpathlineto{\pgfqpoint{2.642140in}{2.784143in}}%
\pgfpathlineto{\pgfqpoint{2.633577in}{2.783384in}}%
\pgfpathlineto{\pgfqpoint{2.625000in}{2.782809in}}%
\pgfpathlineto{\pgfqpoint{2.616410in}{2.782421in}}%
\pgfpathlineto{\pgfqpoint{2.602782in}{2.798861in}}%
\pgfpathlineto{\pgfqpoint{2.589148in}{2.815496in}}%
\pgfpathlineto{\pgfqpoint{2.575508in}{2.832325in}}%
\pgfpathlineto{\pgfqpoint{2.561862in}{2.849352in}}%
\pgfpathlineto{\pgfqpoint{2.570489in}{2.849392in}}%
\pgfpathlineto{\pgfqpoint{2.579103in}{2.849625in}}%
\pgfpathlineto{\pgfqpoint{2.587702in}{2.850046in}}%
\pgfpathlineto{\pgfqpoint{2.596288in}{2.850653in}}%
\pgfpathclose%
\pgfusepath{fill}%
\end{pgfscope}%
\begin{pgfscope}%
\pgfpathrectangle{\pgfqpoint{1.150000in}{0.150000in}}{\pgfqpoint{5.700000in}{5.700000in}}%
\pgfusepath{clip}%
\pgfsetbuttcap%
\pgfsetroundjoin%
\definecolor{currentfill}{rgb}{0.268510,0.009605,0.335427}%
\pgfsetfillcolor{currentfill}%
\pgfsetfillopacity{0.700000}%
\pgfsetlinewidth{0.000000pt}%
\definecolor{currentstroke}{rgb}{0.000000,0.000000,0.000000}%
\pgfsetstrokecolor{currentstroke}%
\pgfsetdash{}{0pt}%
\pgfpathmoveto{\pgfqpoint{3.785355in}{2.193325in}}%
\pgfpathlineto{\pgfqpoint{3.798873in}{2.189255in}}%
\pgfpathlineto{\pgfqpoint{3.812396in}{2.185301in}}%
\pgfpathlineto{\pgfqpoint{3.825924in}{2.181463in}}%
\pgfpathlineto{\pgfqpoint{3.839458in}{2.177740in}}%
\pgfpathlineto{\pgfqpoint{3.831523in}{2.169503in}}%
\pgfpathlineto{\pgfqpoint{3.823581in}{2.161306in}}%
\pgfpathlineto{\pgfqpoint{3.815633in}{2.153151in}}%
\pgfpathlineto{\pgfqpoint{3.807680in}{2.145041in}}%
\pgfpathlineto{\pgfqpoint{3.794132in}{2.148991in}}%
\pgfpathlineto{\pgfqpoint{3.780590in}{2.153056in}}%
\pgfpathlineto{\pgfqpoint{3.767053in}{2.157238in}}%
\pgfpathlineto{\pgfqpoint{3.753521in}{2.161536in}}%
\pgfpathlineto{\pgfqpoint{3.761489in}{2.169412in}}%
\pgfpathlineto{\pgfqpoint{3.769450in}{2.177337in}}%
\pgfpathlineto{\pgfqpoint{3.777406in}{2.185309in}}%
\pgfpathlineto{\pgfqpoint{3.785355in}{2.193325in}}%
\pgfpathclose%
\pgfusepath{fill}%
\end{pgfscope}%
\begin{pgfscope}%
\pgfpathrectangle{\pgfqpoint{1.150000in}{0.150000in}}{\pgfqpoint{5.700000in}{5.700000in}}%
\pgfusepath{clip}%
\pgfsetbuttcap%
\pgfsetroundjoin%
\definecolor{currentfill}{rgb}{0.260571,0.246922,0.522828}%
\pgfsetfillcolor{currentfill}%
\pgfsetfillopacity{0.700000}%
\pgfsetlinewidth{0.000000pt}%
\definecolor{currentstroke}{rgb}{0.000000,0.000000,0.000000}%
\pgfsetstrokecolor{currentstroke}%
\pgfsetdash{}{0pt}%
\pgfpathmoveto{\pgfqpoint{2.759252in}{2.663011in}}%
\pgfpathlineto{\pgfqpoint{2.772803in}{2.648570in}}%
\pgfpathlineto{\pgfqpoint{2.786350in}{2.634304in}}%
\pgfpathlineto{\pgfqpoint{2.799894in}{2.620213in}}%
\pgfpathlineto{\pgfqpoint{2.813435in}{2.606296in}}%
\pgfpathlineto{\pgfqpoint{2.804986in}{2.604370in}}%
\pgfpathlineto{\pgfqpoint{2.796524in}{2.602610in}}%
\pgfpathlineto{\pgfqpoint{2.788050in}{2.601020in}}%
\pgfpathlineto{\pgfqpoint{2.779564in}{2.599601in}}%
\pgfpathlineto{\pgfqpoint{2.765991in}{2.613852in}}%
\pgfpathlineto{\pgfqpoint{2.752414in}{2.628277in}}%
\pgfpathlineto{\pgfqpoint{2.738834in}{2.642877in}}%
\pgfpathlineto{\pgfqpoint{2.725249in}{2.657653in}}%
\pgfpathlineto{\pgfqpoint{2.733769in}{2.658730in}}%
\pgfpathlineto{\pgfqpoint{2.742276in}{2.659984in}}%
\pgfpathlineto{\pgfqpoint{2.750770in}{2.661413in}}%
\pgfpathlineto{\pgfqpoint{2.759252in}{2.663011in}}%
\pgfpathclose%
\pgfusepath{fill}%
\end{pgfscope}%
\begin{pgfscope}%
\pgfpathrectangle{\pgfqpoint{1.150000in}{0.150000in}}{\pgfqpoint{5.700000in}{5.700000in}}%
\pgfusepath{clip}%
\pgfsetbuttcap%
\pgfsetroundjoin%
\definecolor{currentfill}{rgb}{0.272594,0.025563,0.353093}%
\pgfsetfillcolor{currentfill}%
\pgfsetfillopacity{0.700000}%
\pgfsetlinewidth{0.000000pt}%
\definecolor{currentstroke}{rgb}{0.000000,0.000000,0.000000}%
\pgfsetstrokecolor{currentstroke}%
\pgfsetdash{}{0pt}%
\pgfpathmoveto{\pgfqpoint{4.011026in}{2.222381in}}%
\pgfpathlineto{\pgfqpoint{4.024591in}{2.220078in}}%
\pgfpathlineto{\pgfqpoint{4.038164in}{2.217886in}}%
\pgfpathlineto{\pgfqpoint{4.051744in}{2.215806in}}%
\pgfpathlineto{\pgfqpoint{4.065330in}{2.213835in}}%
\pgfpathlineto{\pgfqpoint{4.057476in}{2.204791in}}%
\pgfpathlineto{\pgfqpoint{4.049615in}{2.195758in}}%
\pgfpathlineto{\pgfqpoint{4.041750in}{2.186739in}}%
\pgfpathlineto{\pgfqpoint{4.033878in}{2.177736in}}%
\pgfpathlineto{\pgfqpoint{4.020281in}{2.179898in}}%
\pgfpathlineto{\pgfqpoint{4.006690in}{2.182170in}}%
\pgfpathlineto{\pgfqpoint{3.993107in}{2.184553in}}%
\pgfpathlineto{\pgfqpoint{3.979530in}{2.187047in}}%
\pgfpathlineto{\pgfqpoint{3.987412in}{2.195852in}}%
\pgfpathlineto{\pgfqpoint{3.995289in}{2.204678in}}%
\pgfpathlineto{\pgfqpoint{4.003160in}{2.213521in}}%
\pgfpathlineto{\pgfqpoint{4.011026in}{2.222381in}}%
\pgfpathclose%
\pgfusepath{fill}%
\end{pgfscope}%
\begin{pgfscope}%
\pgfpathrectangle{\pgfqpoint{1.150000in}{0.150000in}}{\pgfqpoint{5.700000in}{5.700000in}}%
\pgfusepath{clip}%
\pgfsetbuttcap%
\pgfsetroundjoin%
\definecolor{currentfill}{rgb}{0.276022,0.044167,0.370164}%
\pgfsetfillcolor{currentfill}%
\pgfsetfillopacity{0.700000}%
\pgfsetlinewidth{0.000000pt}%
\definecolor{currentstroke}{rgb}{0.000000,0.000000,0.000000}%
\pgfsetstrokecolor{currentstroke}%
\pgfsetdash{}{0pt}%
\pgfpathmoveto{\pgfqpoint{3.365258in}{2.252305in}}%
\pgfpathlineto{\pgfqpoint{3.378734in}{2.244530in}}%
\pgfpathlineto{\pgfqpoint{3.392213in}{2.236886in}}%
\pgfpathlineto{\pgfqpoint{3.405694in}{2.229373in}}%
\pgfpathlineto{\pgfqpoint{3.419177in}{2.221990in}}%
\pgfpathlineto{\pgfqpoint{3.411062in}{2.216035in}}%
\pgfpathlineto{\pgfqpoint{3.402939in}{2.210175in}}%
\pgfpathlineto{\pgfqpoint{3.394808in}{2.204414in}}%
\pgfpathlineto{\pgfqpoint{3.386669in}{2.198753in}}%
\pgfpathlineto{\pgfqpoint{3.373165in}{2.206420in}}%
\pgfpathlineto{\pgfqpoint{3.359663in}{2.214218in}}%
\pgfpathlineto{\pgfqpoint{3.346163in}{2.222146in}}%
\pgfpathlineto{\pgfqpoint{3.332666in}{2.230205in}}%
\pgfpathlineto{\pgfqpoint{3.340826in}{2.235575in}}%
\pgfpathlineto{\pgfqpoint{3.348978in}{2.241050in}}%
\pgfpathlineto{\pgfqpoint{3.357122in}{2.246628in}}%
\pgfpathlineto{\pgfqpoint{3.365258in}{2.252305in}}%
\pgfpathclose%
\pgfusepath{fill}%
\end{pgfscope}%
\begin{pgfscope}%
\pgfpathrectangle{\pgfqpoint{1.150000in}{0.150000in}}{\pgfqpoint{5.700000in}{5.700000in}}%
\pgfusepath{clip}%
\pgfsetbuttcap%
\pgfsetroundjoin%
\definecolor{currentfill}{rgb}{0.279574,0.170599,0.479997}%
\pgfsetfillcolor{currentfill}%
\pgfsetfillopacity{0.700000}%
\pgfsetlinewidth{0.000000pt}%
\definecolor{currentstroke}{rgb}{0.000000,0.000000,0.000000}%
\pgfsetstrokecolor{currentstroke}%
\pgfsetdash{}{0pt}%
\pgfpathmoveto{\pgfqpoint{4.665052in}{2.478561in}}%
\pgfpathlineto{\pgfqpoint{4.678831in}{2.480505in}}%
\pgfpathlineto{\pgfqpoint{4.692621in}{2.482552in}}%
\pgfpathlineto{\pgfqpoint{4.706422in}{2.484702in}}%
\pgfpathlineto{\pgfqpoint{4.720233in}{2.486954in}}%
\pgfpathlineto{\pgfqpoint{4.712597in}{2.477587in}}%
\pgfpathlineto{\pgfqpoint{4.704956in}{2.468169in}}%
\pgfpathlineto{\pgfqpoint{4.697309in}{2.458701in}}%
\pgfpathlineto{\pgfqpoint{4.689657in}{2.449181in}}%
\pgfpathlineto{\pgfqpoint{4.675838in}{2.446993in}}%
\pgfpathlineto{\pgfqpoint{4.662030in}{2.444908in}}%
\pgfpathlineto{\pgfqpoint{4.648233in}{2.442926in}}%
\pgfpathlineto{\pgfqpoint{4.634445in}{2.441047in}}%
\pgfpathlineto{\pgfqpoint{4.642105in}{2.450495in}}%
\pgfpathlineto{\pgfqpoint{4.649760in}{2.459897in}}%
\pgfpathlineto{\pgfqpoint{4.657409in}{2.469252in}}%
\pgfpathlineto{\pgfqpoint{4.665052in}{2.478561in}}%
\pgfpathclose%
\pgfusepath{fill}%
\end{pgfscope}%
\begin{pgfscope}%
\pgfpathrectangle{\pgfqpoint{1.150000in}{0.150000in}}{\pgfqpoint{5.700000in}{5.700000in}}%
\pgfusepath{clip}%
\pgfsetbuttcap%
\pgfsetroundjoin%
\definecolor{currentfill}{rgb}{0.276194,0.190074,0.493001}%
\pgfsetfillcolor{currentfill}%
\pgfsetfillopacity{0.700000}%
\pgfsetlinewidth{0.000000pt}%
\definecolor{currentstroke}{rgb}{0.000000,0.000000,0.000000}%
\pgfsetstrokecolor{currentstroke}%
\pgfsetdash{}{0pt}%
\pgfpathmoveto{\pgfqpoint{4.750717in}{2.523900in}}%
\pgfpathlineto{\pgfqpoint{4.764531in}{2.526302in}}%
\pgfpathlineto{\pgfqpoint{4.778356in}{2.528805in}}%
\pgfpathlineto{\pgfqpoint{4.792191in}{2.531410in}}%
\pgfpathlineto{\pgfqpoint{4.806038in}{2.534118in}}%
\pgfpathlineto{\pgfqpoint{4.798433in}{2.524920in}}%
\pgfpathlineto{\pgfqpoint{4.790823in}{2.515665in}}%
\pgfpathlineto{\pgfqpoint{4.783206in}{2.506354in}}%
\pgfpathlineto{\pgfqpoint{4.775584in}{2.496986in}}%
\pgfpathlineto{\pgfqpoint{4.761730in}{2.494325in}}%
\pgfpathlineto{\pgfqpoint{4.747887in}{2.491766in}}%
\pgfpathlineto{\pgfqpoint{4.734055in}{2.489309in}}%
\pgfpathlineto{\pgfqpoint{4.720233in}{2.486954in}}%
\pgfpathlineto{\pgfqpoint{4.727863in}{2.496269in}}%
\pgfpathlineto{\pgfqpoint{4.735487in}{2.505531in}}%
\pgfpathlineto{\pgfqpoint{4.743105in}{2.514742in}}%
\pgfpathlineto{\pgfqpoint{4.750717in}{2.523900in}}%
\pgfpathclose%
\pgfusepath{fill}%
\end{pgfscope}%
\begin{pgfscope}%
\pgfpathrectangle{\pgfqpoint{1.150000in}{0.150000in}}{\pgfqpoint{5.700000in}{5.700000in}}%
\pgfusepath{clip}%
\pgfsetbuttcap%
\pgfsetroundjoin%
\definecolor{currentfill}{rgb}{0.282290,0.145912,0.461510}%
\pgfsetfillcolor{currentfill}%
\pgfsetfillopacity{0.700000}%
\pgfsetlinewidth{0.000000pt}%
\definecolor{currentstroke}{rgb}{0.000000,0.000000,0.000000}%
\pgfsetstrokecolor{currentstroke}%
\pgfsetdash{}{0pt}%
\pgfpathmoveto{\pgfqpoint{4.579397in}{2.434563in}}%
\pgfpathlineto{\pgfqpoint{4.593144in}{2.436029in}}%
\pgfpathlineto{\pgfqpoint{4.606901in}{2.437598in}}%
\pgfpathlineto{\pgfqpoint{4.620668in}{2.439271in}}%
\pgfpathlineto{\pgfqpoint{4.634445in}{2.441047in}}%
\pgfpathlineto{\pgfqpoint{4.626780in}{2.431553in}}%
\pgfpathlineto{\pgfqpoint{4.619109in}{2.422014in}}%
\pgfpathlineto{\pgfqpoint{4.611432in}{2.412430in}}%
\pgfpathlineto{\pgfqpoint{4.603750in}{2.402802in}}%
\pgfpathlineto{\pgfqpoint{4.589965in}{2.401109in}}%
\pgfpathlineto{\pgfqpoint{4.576190in}{2.399519in}}%
\pgfpathlineto{\pgfqpoint{4.562426in}{2.398033in}}%
\pgfpathlineto{\pgfqpoint{4.548671in}{2.396650in}}%
\pgfpathlineto{\pgfqpoint{4.556361in}{2.406188in}}%
\pgfpathlineto{\pgfqpoint{4.564045in}{2.415687in}}%
\pgfpathlineto{\pgfqpoint{4.571724in}{2.425145in}}%
\pgfpathlineto{\pgfqpoint{4.579397in}{2.434563in}}%
\pgfpathclose%
\pgfusepath{fill}%
\end{pgfscope}%
\begin{pgfscope}%
\pgfpathrectangle{\pgfqpoint{1.150000in}{0.150000in}}{\pgfqpoint{5.700000in}{5.700000in}}%
\pgfusepath{clip}%
\pgfsetbuttcap%
\pgfsetroundjoin%
\definecolor{currentfill}{rgb}{0.267968,0.223549,0.512008}%
\pgfsetfillcolor{currentfill}%
\pgfsetfillopacity{0.700000}%
\pgfsetlinewidth{0.000000pt}%
\definecolor{currentstroke}{rgb}{0.000000,0.000000,0.000000}%
\pgfsetstrokecolor{currentstroke}%
\pgfsetdash{}{0pt}%
\pgfpathmoveto{\pgfqpoint{2.813435in}{2.606296in}}%
\pgfpathlineto{\pgfqpoint{2.826972in}{2.592550in}}%
\pgfpathlineto{\pgfqpoint{2.840506in}{2.578974in}}%
\pgfpathlineto{\pgfqpoint{2.854038in}{2.565568in}}%
\pgfpathlineto{\pgfqpoint{2.867567in}{2.552328in}}%
\pgfpathlineto{\pgfqpoint{2.859149in}{2.550078in}}%
\pgfpathlineto{\pgfqpoint{2.850719in}{2.547989in}}%
\pgfpathlineto{\pgfqpoint{2.842278in}{2.546063in}}%
\pgfpathlineto{\pgfqpoint{2.833824in}{2.544305in}}%
\pgfpathlineto{\pgfqpoint{2.820264in}{2.557876in}}%
\pgfpathlineto{\pgfqpoint{2.806700in}{2.571615in}}%
\pgfpathlineto{\pgfqpoint{2.793134in}{2.585523in}}%
\pgfpathlineto{\pgfqpoint{2.779564in}{2.599601in}}%
\pgfpathlineto{\pgfqpoint{2.788050in}{2.601020in}}%
\pgfpathlineto{\pgfqpoint{2.796524in}{2.602610in}}%
\pgfpathlineto{\pgfqpoint{2.804986in}{2.604370in}}%
\pgfpathlineto{\pgfqpoint{2.813435in}{2.606296in}}%
\pgfpathclose%
\pgfusepath{fill}%
\end{pgfscope}%
\begin{pgfscope}%
\pgfpathrectangle{\pgfqpoint{1.150000in}{0.150000in}}{\pgfqpoint{5.700000in}{5.700000in}}%
\pgfusepath{clip}%
\pgfsetbuttcap%
\pgfsetroundjoin%
\definecolor{currentfill}{rgb}{0.270595,0.214069,0.507052}%
\pgfsetfillcolor{currentfill}%
\pgfsetfillopacity{0.700000}%
\pgfsetlinewidth{0.000000pt}%
\definecolor{currentstroke}{rgb}{0.000000,0.000000,0.000000}%
\pgfsetstrokecolor{currentstroke}%
\pgfsetdash{}{0pt}%
\pgfpathmoveto{\pgfqpoint{4.836397in}{2.570339in}}%
\pgfpathlineto{\pgfqpoint{4.850246in}{2.573177in}}%
\pgfpathlineto{\pgfqpoint{4.864107in}{2.576116in}}%
\pgfpathlineto{\pgfqpoint{4.877979in}{2.579156in}}%
\pgfpathlineto{\pgfqpoint{4.891862in}{2.582298in}}%
\pgfpathlineto{\pgfqpoint{4.884289in}{2.573308in}}%
\pgfpathlineto{\pgfqpoint{4.876710in}{2.564255in}}%
\pgfpathlineto{\pgfqpoint{4.869126in}{2.555141in}}%
\pgfpathlineto{\pgfqpoint{4.861535in}{2.545966in}}%
\pgfpathlineto{\pgfqpoint{4.847644in}{2.542852in}}%
\pgfpathlineto{\pgfqpoint{4.833764in}{2.539839in}}%
\pgfpathlineto{\pgfqpoint{4.819895in}{2.536927in}}%
\pgfpathlineto{\pgfqpoint{4.806038in}{2.534118in}}%
\pgfpathlineto{\pgfqpoint{4.813636in}{2.543259in}}%
\pgfpathlineto{\pgfqpoint{4.821229in}{2.552343in}}%
\pgfpathlineto{\pgfqpoint{4.828816in}{2.561369in}}%
\pgfpathlineto{\pgfqpoint{4.836397in}{2.570339in}}%
\pgfpathclose%
\pgfusepath{fill}%
\end{pgfscope}%
\begin{pgfscope}%
\pgfpathrectangle{\pgfqpoint{1.150000in}{0.150000in}}{\pgfqpoint{5.700000in}{5.700000in}}%
\pgfusepath{clip}%
\pgfsetbuttcap%
\pgfsetroundjoin%
\definecolor{currentfill}{rgb}{0.283187,0.125848,0.444960}%
\pgfsetfillcolor{currentfill}%
\pgfsetfillopacity{0.700000}%
\pgfsetlinewidth{0.000000pt}%
\definecolor{currentstroke}{rgb}{0.000000,0.000000,0.000000}%
\pgfsetstrokecolor{currentstroke}%
\pgfsetdash{}{0pt}%
\pgfpathmoveto{\pgfqpoint{4.493749in}{2.392159in}}%
\pgfpathlineto{\pgfqpoint{4.507465in}{2.393126in}}%
\pgfpathlineto{\pgfqpoint{4.521191in}{2.394196in}}%
\pgfpathlineto{\pgfqpoint{4.534926in}{2.395371in}}%
\pgfpathlineto{\pgfqpoint{4.548671in}{2.396650in}}%
\pgfpathlineto{\pgfqpoint{4.540976in}{2.387073in}}%
\pgfpathlineto{\pgfqpoint{4.533275in}{2.377458in}}%
\pgfpathlineto{\pgfqpoint{4.525569in}{2.367805in}}%
\pgfpathlineto{\pgfqpoint{4.517858in}{2.358115in}}%
\pgfpathlineto{\pgfqpoint{4.504105in}{2.356937in}}%
\pgfpathlineto{\pgfqpoint{4.490362in}{2.355864in}}%
\pgfpathlineto{\pgfqpoint{4.476629in}{2.354894in}}%
\pgfpathlineto{\pgfqpoint{4.462905in}{2.354029in}}%
\pgfpathlineto{\pgfqpoint{4.470624in}{2.363611in}}%
\pgfpathlineto{\pgfqpoint{4.478338in}{2.373160in}}%
\pgfpathlineto{\pgfqpoint{4.486046in}{2.382677in}}%
\pgfpathlineto{\pgfqpoint{4.493749in}{2.392159in}}%
\pgfpathclose%
\pgfusepath{fill}%
\end{pgfscope}%
\begin{pgfscope}%
\pgfpathrectangle{\pgfqpoint{1.150000in}{0.150000in}}{\pgfqpoint{5.700000in}{5.700000in}}%
\pgfusepath{clip}%
\pgfsetbuttcap%
\pgfsetroundjoin%
\definecolor{currentfill}{rgb}{0.281924,0.089666,0.412415}%
\pgfsetfillcolor{currentfill}%
\pgfsetfillopacity{0.700000}%
\pgfsetlinewidth{0.000000pt}%
\definecolor{currentstroke}{rgb}{0.000000,0.000000,0.000000}%
\pgfsetstrokecolor{currentstroke}%
\pgfsetdash{}{0pt}%
\pgfpathmoveto{\pgfqpoint{3.170818in}{2.337471in}}%
\pgfpathlineto{\pgfqpoint{3.184300in}{2.327769in}}%
\pgfpathlineto{\pgfqpoint{3.197782in}{2.318209in}}%
\pgfpathlineto{\pgfqpoint{3.211265in}{2.308790in}}%
\pgfpathlineto{\pgfqpoint{3.224748in}{2.299510in}}%
\pgfpathlineto{\pgfqpoint{3.216534in}{2.294842in}}%
\pgfpathlineto{\pgfqpoint{3.208311in}{2.290295in}}%
\pgfpathlineto{\pgfqpoint{3.200078in}{2.285870in}}%
\pgfpathlineto{\pgfqpoint{3.191837in}{2.281571in}}%
\pgfpathlineto{\pgfqpoint{3.178329in}{2.291156in}}%
\pgfpathlineto{\pgfqpoint{3.164821in}{2.300880in}}%
\pgfpathlineto{\pgfqpoint{3.151315in}{2.310745in}}%
\pgfpathlineto{\pgfqpoint{3.137808in}{2.320753in}}%
\pgfpathlineto{\pgfqpoint{3.146075in}{2.324739in}}%
\pgfpathlineto{\pgfqpoint{3.154332in}{2.328856in}}%
\pgfpathlineto{\pgfqpoint{3.162580in}{2.333101in}}%
\pgfpathlineto{\pgfqpoint{3.170818in}{2.337471in}}%
\pgfpathclose%
\pgfusepath{fill}%
\end{pgfscope}%
\begin{pgfscope}%
\pgfpathrectangle{\pgfqpoint{1.150000in}{0.150000in}}{\pgfqpoint{5.700000in}{5.700000in}}%
\pgfusepath{clip}%
\pgfsetbuttcap%
\pgfsetroundjoin%
\definecolor{currentfill}{rgb}{0.263663,0.237631,0.518762}%
\pgfsetfillcolor{currentfill}%
\pgfsetfillopacity{0.700000}%
\pgfsetlinewidth{0.000000pt}%
\definecolor{currentstroke}{rgb}{0.000000,0.000000,0.000000}%
\pgfsetstrokecolor{currentstroke}%
\pgfsetdash{}{0pt}%
\pgfpathmoveto{\pgfqpoint{4.922092in}{2.617649in}}%
\pgfpathlineto{\pgfqpoint{4.935978in}{2.620902in}}%
\pgfpathlineto{\pgfqpoint{4.949876in}{2.624256in}}%
\pgfpathlineto{\pgfqpoint{4.963785in}{2.627711in}}%
\pgfpathlineto{\pgfqpoint{4.977707in}{2.631267in}}%
\pgfpathlineto{\pgfqpoint{4.970167in}{2.622519in}}%
\pgfpathlineto{\pgfqpoint{4.962621in}{2.613705in}}%
\pgfpathlineto{\pgfqpoint{4.955069in}{2.604825in}}%
\pgfpathlineto{\pgfqpoint{4.947510in}{2.595880in}}%
\pgfpathlineto{\pgfqpoint{4.933581in}{2.592333in}}%
\pgfpathlineto{\pgfqpoint{4.919663in}{2.588887in}}%
\pgfpathlineto{\pgfqpoint{4.905757in}{2.585542in}}%
\pgfpathlineto{\pgfqpoint{4.891862in}{2.582298in}}%
\pgfpathlineto{\pgfqpoint{4.899428in}{2.591228in}}%
\pgfpathlineto{\pgfqpoint{4.906989in}{2.600096in}}%
\pgfpathlineto{\pgfqpoint{4.914543in}{2.608903in}}%
\pgfpathlineto{\pgfqpoint{4.922092in}{2.617649in}}%
\pgfpathclose%
\pgfusepath{fill}%
\end{pgfscope}%
\begin{pgfscope}%
\pgfpathrectangle{\pgfqpoint{1.150000in}{0.150000in}}{\pgfqpoint{5.700000in}{5.700000in}}%
\pgfusepath{clip}%
\pgfsetbuttcap%
\pgfsetroundjoin%
\definecolor{currentfill}{rgb}{0.282910,0.105393,0.426902}%
\pgfsetfillcolor{currentfill}%
\pgfsetfillopacity{0.700000}%
\pgfsetlinewidth{0.000000pt}%
\definecolor{currentstroke}{rgb}{0.000000,0.000000,0.000000}%
\pgfsetstrokecolor{currentstroke}%
\pgfsetdash{}{0pt}%
\pgfpathmoveto{\pgfqpoint{4.408102in}{2.351617in}}%
\pgfpathlineto{\pgfqpoint{4.421789in}{2.352062in}}%
\pgfpathlineto{\pgfqpoint{4.435485in}{2.352613in}}%
\pgfpathlineto{\pgfqpoint{4.449190in}{2.353268in}}%
\pgfpathlineto{\pgfqpoint{4.462905in}{2.354029in}}%
\pgfpathlineto{\pgfqpoint{4.455180in}{2.344416in}}%
\pgfpathlineto{\pgfqpoint{4.447450in}{2.334772in}}%
\pgfpathlineto{\pgfqpoint{4.439715in}{2.325099in}}%
\pgfpathlineto{\pgfqpoint{4.431974in}{2.315396in}}%
\pgfpathlineto{\pgfqpoint{4.418252in}{2.314755in}}%
\pgfpathlineto{\pgfqpoint{4.404539in}{2.314219in}}%
\pgfpathlineto{\pgfqpoint{4.390834in}{2.313787in}}%
\pgfpathlineto{\pgfqpoint{4.377139in}{2.313461in}}%
\pgfpathlineto{\pgfqpoint{4.384888in}{2.323037in}}%
\pgfpathlineto{\pgfqpoint{4.392631in}{2.332589in}}%
\pgfpathlineto{\pgfqpoint{4.400369in}{2.342116in}}%
\pgfpathlineto{\pgfqpoint{4.408102in}{2.351617in}}%
\pgfpathclose%
\pgfusepath{fill}%
\end{pgfscope}%
\begin{pgfscope}%
\pgfpathrectangle{\pgfqpoint{1.150000in}{0.150000in}}{\pgfqpoint{5.700000in}{5.700000in}}%
\pgfusepath{clip}%
\pgfsetbuttcap%
\pgfsetroundjoin%
\definecolor{currentfill}{rgb}{0.255645,0.260703,0.528312}%
\pgfsetfillcolor{currentfill}%
\pgfsetfillopacity{0.700000}%
\pgfsetlinewidth{0.000000pt}%
\definecolor{currentstroke}{rgb}{0.000000,0.000000,0.000000}%
\pgfsetstrokecolor{currentstroke}%
\pgfsetdash{}{0pt}%
\pgfpathmoveto{\pgfqpoint{5.007803in}{2.665612in}}%
\pgfpathlineto{\pgfqpoint{5.021727in}{2.669259in}}%
\pgfpathlineto{\pgfqpoint{5.035663in}{2.673008in}}%
\pgfpathlineto{\pgfqpoint{5.049611in}{2.676857in}}%
\pgfpathlineto{\pgfqpoint{5.063571in}{2.680807in}}%
\pgfpathlineto{\pgfqpoint{5.056066in}{2.672333in}}%
\pgfpathlineto{\pgfqpoint{5.048554in}{2.663791in}}%
\pgfpathlineto{\pgfqpoint{5.041036in}{2.655180in}}%
\pgfpathlineto{\pgfqpoint{5.033511in}{2.646500in}}%
\pgfpathlineto{\pgfqpoint{5.019542in}{2.642541in}}%
\pgfpathlineto{\pgfqpoint{5.005585in}{2.638682in}}%
\pgfpathlineto{\pgfqpoint{4.991640in}{2.634924in}}%
\pgfpathlineto{\pgfqpoint{4.977707in}{2.631267in}}%
\pgfpathlineto{\pgfqpoint{4.985240in}{2.639950in}}%
\pgfpathlineto{\pgfqpoint{4.992767in}{2.648568in}}%
\pgfpathlineto{\pgfqpoint{5.000288in}{2.657122in}}%
\pgfpathlineto{\pgfqpoint{5.007803in}{2.665612in}}%
\pgfpathclose%
\pgfusepath{fill}%
\end{pgfscope}%
\begin{pgfscope}%
\pgfpathrectangle{\pgfqpoint{1.150000in}{0.150000in}}{\pgfqpoint{5.700000in}{5.700000in}}%
\pgfusepath{clip}%
\pgfsetbuttcap%
\pgfsetroundjoin%
\definecolor{currentfill}{rgb}{0.274128,0.199721,0.498911}%
\pgfsetfillcolor{currentfill}%
\pgfsetfillopacity{0.700000}%
\pgfsetlinewidth{0.000000pt}%
\definecolor{currentstroke}{rgb}{0.000000,0.000000,0.000000}%
\pgfsetstrokecolor{currentstroke}%
\pgfsetdash{}{0pt}%
\pgfpathmoveto{\pgfqpoint{2.867567in}{2.552328in}}%
\pgfpathlineto{\pgfqpoint{2.881093in}{2.539255in}}%
\pgfpathlineto{\pgfqpoint{2.894617in}{2.526347in}}%
\pgfpathlineto{\pgfqpoint{2.908139in}{2.513602in}}%
\pgfpathlineto{\pgfqpoint{2.921658in}{2.501019in}}%
\pgfpathlineto{\pgfqpoint{2.913270in}{2.498446in}}%
\pgfpathlineto{\pgfqpoint{2.904871in}{2.496028in}}%
\pgfpathlineto{\pgfqpoint{2.896461in}{2.493770in}}%
\pgfpathlineto{\pgfqpoint{2.888039in}{2.491674in}}%
\pgfpathlineto{\pgfqpoint{2.874489in}{2.504587in}}%
\pgfpathlineto{\pgfqpoint{2.860936in}{2.517662in}}%
\pgfpathlineto{\pgfqpoint{2.847381in}{2.530901in}}%
\pgfpathlineto{\pgfqpoint{2.833824in}{2.544305in}}%
\pgfpathlineto{\pgfqpoint{2.842278in}{2.546063in}}%
\pgfpathlineto{\pgfqpoint{2.850719in}{2.547989in}}%
\pgfpathlineto{\pgfqpoint{2.859149in}{2.550078in}}%
\pgfpathlineto{\pgfqpoint{2.867567in}{2.552328in}}%
\pgfpathclose%
\pgfusepath{fill}%
\end{pgfscope}%
\begin{pgfscope}%
\pgfpathrectangle{\pgfqpoint{1.150000in}{0.150000in}}{\pgfqpoint{5.700000in}{5.700000in}}%
\pgfusepath{clip}%
\pgfsetbuttcap%
\pgfsetroundjoin%
\definecolor{currentfill}{rgb}{0.246811,0.283237,0.535941}%
\pgfsetfillcolor{currentfill}%
\pgfsetfillopacity{0.700000}%
\pgfsetlinewidth{0.000000pt}%
\definecolor{currentstroke}{rgb}{0.000000,0.000000,0.000000}%
\pgfsetstrokecolor{currentstroke}%
\pgfsetdash{}{0pt}%
\pgfpathmoveto{\pgfqpoint{5.093529in}{2.714023in}}%
\pgfpathlineto{\pgfqpoint{5.107492in}{2.718045in}}%
\pgfpathlineto{\pgfqpoint{5.121467in}{2.722168in}}%
\pgfpathlineto{\pgfqpoint{5.135455in}{2.726390in}}%
\pgfpathlineto{\pgfqpoint{5.149455in}{2.730713in}}%
\pgfpathlineto{\pgfqpoint{5.141985in}{2.722544in}}%
\pgfpathlineto{\pgfqpoint{5.134509in}{2.714305in}}%
\pgfpathlineto{\pgfqpoint{5.127026in}{2.705993in}}%
\pgfpathlineto{\pgfqpoint{5.119536in}{2.697610in}}%
\pgfpathlineto{\pgfqpoint{5.105526in}{2.693258in}}%
\pgfpathlineto{\pgfqpoint{5.091529in}{2.689007in}}%
\pgfpathlineto{\pgfqpoint{5.077544in}{2.684857in}}%
\pgfpathlineto{\pgfqpoint{5.063571in}{2.680807in}}%
\pgfpathlineto{\pgfqpoint{5.071071in}{2.689212in}}%
\pgfpathlineto{\pgfqpoint{5.078563in}{2.697549in}}%
\pgfpathlineto{\pgfqpoint{5.086049in}{2.705819in}}%
\pgfpathlineto{\pgfqpoint{5.093529in}{2.714023in}}%
\pgfpathclose%
\pgfusepath{fill}%
\end{pgfscope}%
\begin{pgfscope}%
\pgfpathrectangle{\pgfqpoint{1.150000in}{0.150000in}}{\pgfqpoint{5.700000in}{5.700000in}}%
\pgfusepath{clip}%
\pgfsetbuttcap%
\pgfsetroundjoin%
\definecolor{currentfill}{rgb}{0.281446,0.084320,0.407414}%
\pgfsetfillcolor{currentfill}%
\pgfsetfillopacity{0.700000}%
\pgfsetlinewidth{0.000000pt}%
\definecolor{currentstroke}{rgb}{0.000000,0.000000,0.000000}%
\pgfsetstrokecolor{currentstroke}%
\pgfsetdash{}{0pt}%
\pgfpathmoveto{\pgfqpoint{4.322447in}{2.313215in}}%
\pgfpathlineto{\pgfqpoint{4.336107in}{2.313117in}}%
\pgfpathlineto{\pgfqpoint{4.349775in}{2.313126in}}%
\pgfpathlineto{\pgfqpoint{4.363453in}{2.313241in}}%
\pgfpathlineto{\pgfqpoint{4.377139in}{2.313461in}}%
\pgfpathlineto{\pgfqpoint{4.369385in}{2.303862in}}%
\pgfpathlineto{\pgfqpoint{4.361626in}{2.294240in}}%
\pgfpathlineto{\pgfqpoint{4.353862in}{2.284598in}}%
\pgfpathlineto{\pgfqpoint{4.346092in}{2.274935in}}%
\pgfpathlineto{\pgfqpoint{4.332397in}{2.274852in}}%
\pgfpathlineto{\pgfqpoint{4.318711in}{2.274874in}}%
\pgfpathlineto{\pgfqpoint{4.305034in}{2.275003in}}%
\pgfpathlineto{\pgfqpoint{4.291365in}{2.275238in}}%
\pgfpathlineto{\pgfqpoint{4.299144in}{2.284756in}}%
\pgfpathlineto{\pgfqpoint{4.306917in}{2.294259in}}%
\pgfpathlineto{\pgfqpoint{4.314684in}{2.303746in}}%
\pgfpathlineto{\pgfqpoint{4.322447in}{2.313215in}}%
\pgfpathclose%
\pgfusepath{fill}%
\end{pgfscope}%
\begin{pgfscope}%
\pgfpathrectangle{\pgfqpoint{1.150000in}{0.150000in}}{\pgfqpoint{5.700000in}{5.700000in}}%
\pgfusepath{clip}%
\pgfsetbuttcap%
\pgfsetroundjoin%
\definecolor{currentfill}{rgb}{0.237441,0.305202,0.541921}%
\pgfsetfillcolor{currentfill}%
\pgfsetfillopacity{0.700000}%
\pgfsetlinewidth{0.000000pt}%
\definecolor{currentstroke}{rgb}{0.000000,0.000000,0.000000}%
\pgfsetstrokecolor{currentstroke}%
\pgfsetdash{}{0pt}%
\pgfpathmoveto{\pgfqpoint{5.179268in}{2.762691in}}%
\pgfpathlineto{\pgfqpoint{5.193271in}{2.767067in}}%
\pgfpathlineto{\pgfqpoint{5.207286in}{2.771543in}}%
\pgfpathlineto{\pgfqpoint{5.221314in}{2.776119in}}%
\pgfpathlineto{\pgfqpoint{5.235355in}{2.780794in}}%
\pgfpathlineto{\pgfqpoint{5.227923in}{2.772957in}}%
\pgfpathlineto{\pgfqpoint{5.220483in}{2.765047in}}%
\pgfpathlineto{\pgfqpoint{5.213036in}{2.757063in}}%
\pgfpathlineto{\pgfqpoint{5.205583in}{2.749005in}}%
\pgfpathlineto{\pgfqpoint{5.191532in}{2.744282in}}%
\pgfpathlineto{\pgfqpoint{5.177494in}{2.739659in}}%
\pgfpathlineto{\pgfqpoint{5.163468in}{2.735136in}}%
\pgfpathlineto{\pgfqpoint{5.149455in}{2.730713in}}%
\pgfpathlineto{\pgfqpoint{5.156918in}{2.738811in}}%
\pgfpathlineto{\pgfqpoint{5.164375in}{2.746840in}}%
\pgfpathlineto{\pgfqpoint{5.171825in}{2.754800in}}%
\pgfpathlineto{\pgfqpoint{5.179268in}{2.762691in}}%
\pgfpathclose%
\pgfusepath{fill}%
\end{pgfscope}%
\begin{pgfscope}%
\pgfpathrectangle{\pgfqpoint{1.150000in}{0.150000in}}{\pgfqpoint{5.700000in}{5.700000in}}%
\pgfusepath{clip}%
\pgfsetbuttcap%
\pgfsetroundjoin%
\definecolor{currentfill}{rgb}{0.269944,0.014625,0.341379}%
\pgfsetfillcolor{currentfill}%
\pgfsetfillopacity{0.700000}%
\pgfsetlinewidth{0.000000pt}%
\definecolor{currentstroke}{rgb}{0.000000,0.000000,0.000000}%
\pgfsetstrokecolor{currentstroke}%
\pgfsetdash{}{0pt}%
\pgfpathmoveto{\pgfqpoint{3.925286in}{2.198144in}}%
\pgfpathlineto{\pgfqpoint{3.938837in}{2.195201in}}%
\pgfpathlineto{\pgfqpoint{3.952395in}{2.192371in}}%
\pgfpathlineto{\pgfqpoint{3.965959in}{2.189653in}}%
\pgfpathlineto{\pgfqpoint{3.979530in}{2.187047in}}%
\pgfpathlineto{\pgfqpoint{3.971642in}{2.178263in}}%
\pgfpathlineto{\pgfqpoint{3.963748in}{2.169503in}}%
\pgfpathlineto{\pgfqpoint{3.955849in}{2.160769in}}%
\pgfpathlineto{\pgfqpoint{3.947944in}{2.152061in}}%
\pgfpathlineto{\pgfqpoint{3.934362in}{2.154876in}}%
\pgfpathlineto{\pgfqpoint{3.920786in}{2.157803in}}%
\pgfpathlineto{\pgfqpoint{3.907216in}{2.160843in}}%
\pgfpathlineto{\pgfqpoint{3.893653in}{2.163995in}}%
\pgfpathlineto{\pgfqpoint{3.901570in}{2.172486in}}%
\pgfpathlineto{\pgfqpoint{3.909481in}{2.181010in}}%
\pgfpathlineto{\pgfqpoint{3.917386in}{2.189563in}}%
\pgfpathlineto{\pgfqpoint{3.925286in}{2.198144in}}%
\pgfpathclose%
\pgfusepath{fill}%
\end{pgfscope}%
\begin{pgfscope}%
\pgfpathrectangle{\pgfqpoint{1.150000in}{0.150000in}}{\pgfqpoint{5.700000in}{5.700000in}}%
\pgfusepath{clip}%
\pgfsetbuttcap%
\pgfsetroundjoin%
\definecolor{currentfill}{rgb}{0.227802,0.326594,0.546532}%
\pgfsetfillcolor{currentfill}%
\pgfsetfillopacity{0.700000}%
\pgfsetlinewidth{0.000000pt}%
\definecolor{currentstroke}{rgb}{0.000000,0.000000,0.000000}%
\pgfsetstrokecolor{currentstroke}%
\pgfsetdash{}{0pt}%
\pgfpathmoveto{\pgfqpoint{5.265018in}{2.811435in}}%
\pgfpathlineto{\pgfqpoint{5.279061in}{2.816144in}}%
\pgfpathlineto{\pgfqpoint{5.293117in}{2.820953in}}%
\pgfpathlineto{\pgfqpoint{5.307186in}{2.825862in}}%
\pgfpathlineto{\pgfqpoint{5.321269in}{2.830870in}}%
\pgfpathlineto{\pgfqpoint{5.313874in}{2.823387in}}%
\pgfpathlineto{\pgfqpoint{5.306473in}{2.815831in}}%
\pgfpathlineto{\pgfqpoint{5.299065in}{2.808200in}}%
\pgfpathlineto{\pgfqpoint{5.291650in}{2.800494in}}%
\pgfpathlineto{\pgfqpoint{5.277556in}{2.795420in}}%
\pgfpathlineto{\pgfqpoint{5.263476in}{2.790445in}}%
\pgfpathlineto{\pgfqpoint{5.249409in}{2.785570in}}%
\pgfpathlineto{\pgfqpoint{5.235355in}{2.780794in}}%
\pgfpathlineto{\pgfqpoint{5.242781in}{2.788560in}}%
\pgfpathlineto{\pgfqpoint{5.250200in}{2.796254in}}%
\pgfpathlineto{\pgfqpoint{5.257613in}{2.803879in}}%
\pgfpathlineto{\pgfqpoint{5.265018in}{2.811435in}}%
\pgfpathclose%
\pgfusepath{fill}%
\end{pgfscope}%
\begin{pgfscope}%
\pgfpathrectangle{\pgfqpoint{1.150000in}{0.150000in}}{\pgfqpoint{5.700000in}{5.700000in}}%
\pgfusepath{clip}%
\pgfsetbuttcap%
\pgfsetroundjoin%
\definecolor{currentfill}{rgb}{0.269944,0.014625,0.341379}%
\pgfsetfillcolor{currentfill}%
\pgfsetfillopacity{0.700000}%
\pgfsetlinewidth{0.000000pt}%
\definecolor{currentstroke}{rgb}{0.000000,0.000000,0.000000}%
\pgfsetstrokecolor{currentstroke}%
\pgfsetdash{}{0pt}%
\pgfpathmoveto{\pgfqpoint{3.559387in}{2.194354in}}%
\pgfpathlineto{\pgfqpoint{3.572881in}{2.188372in}}%
\pgfpathlineto{\pgfqpoint{3.586378in}{2.182513in}}%
\pgfpathlineto{\pgfqpoint{3.599880in}{2.176777in}}%
\pgfpathlineto{\pgfqpoint{3.613386in}{2.171162in}}%
\pgfpathlineto{\pgfqpoint{3.605354in}{2.164078in}}%
\pgfpathlineto{\pgfqpoint{3.597316in}{2.157065in}}%
\pgfpathlineto{\pgfqpoint{3.589271in}{2.150126in}}%
\pgfpathlineto{\pgfqpoint{3.581219in}{2.143263in}}%
\pgfpathlineto{\pgfqpoint{3.567696in}{2.149142in}}%
\pgfpathlineto{\pgfqpoint{3.554178in}{2.155142in}}%
\pgfpathlineto{\pgfqpoint{3.540663in}{2.161266in}}%
\pgfpathlineto{\pgfqpoint{3.527151in}{2.167512in}}%
\pgfpathlineto{\pgfqpoint{3.535221in}{2.174103in}}%
\pgfpathlineto{\pgfqpoint{3.543283in}{2.180776in}}%
\pgfpathlineto{\pgfqpoint{3.551339in}{2.187527in}}%
\pgfpathlineto{\pgfqpoint{3.559387in}{2.194354in}}%
\pgfpathclose%
\pgfusepath{fill}%
\end{pgfscope}%
\begin{pgfscope}%
\pgfpathrectangle{\pgfqpoint{1.150000in}{0.150000in}}{\pgfqpoint{5.700000in}{5.700000in}}%
\pgfusepath{clip}%
\pgfsetbuttcap%
\pgfsetroundjoin%
\definecolor{currentfill}{rgb}{0.278826,0.175490,0.483397}%
\pgfsetfillcolor{currentfill}%
\pgfsetfillopacity{0.700000}%
\pgfsetlinewidth{0.000000pt}%
\definecolor{currentstroke}{rgb}{0.000000,0.000000,0.000000}%
\pgfsetstrokecolor{currentstroke}%
\pgfsetdash{}{0pt}%
\pgfpathmoveto{\pgfqpoint{2.921658in}{2.501019in}}%
\pgfpathlineto{\pgfqpoint{2.935176in}{2.488597in}}%
\pgfpathlineto{\pgfqpoint{2.948692in}{2.476334in}}%
\pgfpathlineto{\pgfqpoint{2.962206in}{2.464230in}}%
\pgfpathlineto{\pgfqpoint{2.975719in}{2.452283in}}%
\pgfpathlineto{\pgfqpoint{2.967360in}{2.449388in}}%
\pgfpathlineto{\pgfqpoint{2.958991in}{2.446644in}}%
\pgfpathlineto{\pgfqpoint{2.950610in}{2.444054in}}%
\pgfpathlineto{\pgfqpoint{2.942219in}{2.441622in}}%
\pgfpathlineto{\pgfqpoint{2.928676in}{2.453898in}}%
\pgfpathlineto{\pgfqpoint{2.915132in}{2.466331in}}%
\pgfpathlineto{\pgfqpoint{2.901587in}{2.478922in}}%
\pgfpathlineto{\pgfqpoint{2.888039in}{2.491674in}}%
\pgfpathlineto{\pgfqpoint{2.896461in}{2.493770in}}%
\pgfpathlineto{\pgfqpoint{2.904871in}{2.496028in}}%
\pgfpathlineto{\pgfqpoint{2.913270in}{2.498446in}}%
\pgfpathlineto{\pgfqpoint{2.921658in}{2.501019in}}%
\pgfpathclose%
\pgfusepath{fill}%
\end{pgfscope}%
\begin{pgfscope}%
\pgfpathrectangle{\pgfqpoint{1.150000in}{0.150000in}}{\pgfqpoint{5.700000in}{5.700000in}}%
\pgfusepath{clip}%
\pgfsetbuttcap%
\pgfsetroundjoin%
\definecolor{currentfill}{rgb}{0.279566,0.067836,0.391917}%
\pgfsetfillcolor{currentfill}%
\pgfsetfillopacity{0.700000}%
\pgfsetlinewidth{0.000000pt}%
\definecolor{currentstroke}{rgb}{0.000000,0.000000,0.000000}%
\pgfsetstrokecolor{currentstroke}%
\pgfsetdash{}{0pt}%
\pgfpathmoveto{\pgfqpoint{4.236774in}{2.277245in}}%
\pgfpathlineto{\pgfqpoint{4.250410in}{2.276582in}}%
\pgfpathlineto{\pgfqpoint{4.264053in}{2.276027in}}%
\pgfpathlineto{\pgfqpoint{4.277705in}{2.275579in}}%
\pgfpathlineto{\pgfqpoint{4.291365in}{2.275238in}}%
\pgfpathlineto{\pgfqpoint{4.283582in}{2.265705in}}%
\pgfpathlineto{\pgfqpoint{4.275793in}{2.256159in}}%
\pgfpathlineto{\pgfqpoint{4.267999in}{2.246602in}}%
\pgfpathlineto{\pgfqpoint{4.260200in}{2.237034in}}%
\pgfpathlineto{\pgfqpoint{4.246531in}{2.237531in}}%
\pgfpathlineto{\pgfqpoint{4.232870in}{2.238134in}}%
\pgfpathlineto{\pgfqpoint{4.219218in}{2.238844in}}%
\pgfpathlineto{\pgfqpoint{4.205573in}{2.239662in}}%
\pgfpathlineto{\pgfqpoint{4.213381in}{2.249068in}}%
\pgfpathlineto{\pgfqpoint{4.221184in}{2.258468in}}%
\pgfpathlineto{\pgfqpoint{4.228982in}{2.267860in}}%
\pgfpathlineto{\pgfqpoint{4.236774in}{2.277245in}}%
\pgfpathclose%
\pgfusepath{fill}%
\end{pgfscope}%
\begin{pgfscope}%
\pgfpathrectangle{\pgfqpoint{1.150000in}{0.150000in}}{\pgfqpoint{5.700000in}{5.700000in}}%
\pgfusepath{clip}%
\pgfsetbuttcap%
\pgfsetroundjoin%
\definecolor{currentfill}{rgb}{0.218130,0.347432,0.550038}%
\pgfsetfillcolor{currentfill}%
\pgfsetfillopacity{0.700000}%
\pgfsetlinewidth{0.000000pt}%
\definecolor{currentstroke}{rgb}{0.000000,0.000000,0.000000}%
\pgfsetstrokecolor{currentstroke}%
\pgfsetdash{}{0pt}%
\pgfpathmoveto{\pgfqpoint{5.350775in}{2.860085in}}%
\pgfpathlineto{\pgfqpoint{5.364858in}{2.865108in}}%
\pgfpathlineto{\pgfqpoint{5.378956in}{2.870230in}}%
\pgfpathlineto{\pgfqpoint{5.393066in}{2.875451in}}%
\pgfpathlineto{\pgfqpoint{5.407190in}{2.880771in}}%
\pgfpathlineto{\pgfqpoint{5.399836in}{2.873664in}}%
\pgfpathlineto{\pgfqpoint{5.392475in}{2.866484in}}%
\pgfpathlineto{\pgfqpoint{5.385107in}{2.859228in}}%
\pgfpathlineto{\pgfqpoint{5.377731in}{2.851897in}}%
\pgfpathlineto{\pgfqpoint{5.363595in}{2.846491in}}%
\pgfpathlineto{\pgfqpoint{5.349473in}{2.841184in}}%
\pgfpathlineto{\pgfqpoint{5.335364in}{2.835977in}}%
\pgfpathlineto{\pgfqpoint{5.321269in}{2.830870in}}%
\pgfpathlineto{\pgfqpoint{5.328656in}{2.838280in}}%
\pgfpathlineto{\pgfqpoint{5.336036in}{2.845618in}}%
\pgfpathlineto{\pgfqpoint{5.343409in}{2.852886in}}%
\pgfpathlineto{\pgfqpoint{5.350775in}{2.860085in}}%
\pgfpathclose%
\pgfusepath{fill}%
\end{pgfscope}%
\begin{pgfscope}%
\pgfpathrectangle{\pgfqpoint{1.150000in}{0.150000in}}{\pgfqpoint{5.700000in}{5.700000in}}%
\pgfusepath{clip}%
\pgfsetbuttcap%
\pgfsetroundjoin%
\definecolor{currentfill}{rgb}{0.268510,0.009605,0.335427}%
\pgfsetfillcolor{currentfill}%
\pgfsetfillopacity{0.700000}%
\pgfsetlinewidth{0.000000pt}%
\definecolor{currentstroke}{rgb}{0.000000,0.000000,0.000000}%
\pgfsetstrokecolor{currentstroke}%
\pgfsetdash{}{0pt}%
\pgfpathmoveto{\pgfqpoint{3.699444in}{2.179901in}}%
\pgfpathlineto{\pgfqpoint{3.712956in}{2.175132in}}%
\pgfpathlineto{\pgfqpoint{3.726473in}{2.170482in}}%
\pgfpathlineto{\pgfqpoint{3.739994in}{2.165950in}}%
\pgfpathlineto{\pgfqpoint{3.753521in}{2.161536in}}%
\pgfpathlineto{\pgfqpoint{3.745547in}{2.153710in}}%
\pgfpathlineto{\pgfqpoint{3.737567in}{2.145938in}}%
\pgfpathlineto{\pgfqpoint{3.729580in}{2.138222in}}%
\pgfpathlineto{\pgfqpoint{3.721587in}{2.130563in}}%
\pgfpathlineto{\pgfqpoint{3.708046in}{2.135223in}}%
\pgfpathlineto{\pgfqpoint{3.694509in}{2.140001in}}%
\pgfpathlineto{\pgfqpoint{3.680977in}{2.144896in}}%
\pgfpathlineto{\pgfqpoint{3.667450in}{2.149911in}}%
\pgfpathlineto{\pgfqpoint{3.675458in}{2.157317in}}%
\pgfpathlineto{\pgfqpoint{3.683460in}{2.164786in}}%
\pgfpathlineto{\pgfqpoint{3.691455in}{2.172314in}}%
\pgfpathlineto{\pgfqpoint{3.699444in}{2.179901in}}%
\pgfpathclose%
\pgfusepath{fill}%
\end{pgfscope}%
\begin{pgfscope}%
\pgfpathrectangle{\pgfqpoint{1.150000in}{0.150000in}}{\pgfqpoint{5.700000in}{5.700000in}}%
\pgfusepath{clip}%
\pgfsetbuttcap%
\pgfsetroundjoin%
\definecolor{currentfill}{rgb}{0.208623,0.367752,0.552675}%
\pgfsetfillcolor{currentfill}%
\pgfsetfillopacity{0.700000}%
\pgfsetlinewidth{0.000000pt}%
\definecolor{currentstroke}{rgb}{0.000000,0.000000,0.000000}%
\pgfsetstrokecolor{currentstroke}%
\pgfsetdash{}{0pt}%
\pgfpathmoveto{\pgfqpoint{5.436534in}{2.908488in}}%
\pgfpathlineto{\pgfqpoint{5.450659in}{2.913803in}}%
\pgfpathlineto{\pgfqpoint{5.464797in}{2.919217in}}%
\pgfpathlineto{\pgfqpoint{5.478950in}{2.924731in}}%
\pgfpathlineto{\pgfqpoint{5.493116in}{2.930343in}}%
\pgfpathlineto{\pgfqpoint{5.485804in}{2.923629in}}%
\pgfpathlineto{\pgfqpoint{5.478484in}{2.916842in}}%
\pgfpathlineto{\pgfqpoint{5.471158in}{2.909981in}}%
\pgfpathlineto{\pgfqpoint{5.463823in}{2.903044in}}%
\pgfpathlineto{\pgfqpoint{5.449644in}{2.897327in}}%
\pgfpathlineto{\pgfqpoint{5.435479in}{2.891709in}}%
\pgfpathlineto{\pgfqpoint{5.421328in}{2.886191in}}%
\pgfpathlineto{\pgfqpoint{5.407190in}{2.880771in}}%
\pgfpathlineto{\pgfqpoint{5.414537in}{2.887806in}}%
\pgfpathlineto{\pgfqpoint{5.421876in}{2.894769in}}%
\pgfpathlineto{\pgfqpoint{5.429209in}{2.901663in}}%
\pgfpathlineto{\pgfqpoint{5.436534in}{2.908488in}}%
\pgfpathclose%
\pgfusepath{fill}%
\end{pgfscope}%
\begin{pgfscope}%
\pgfpathrectangle{\pgfqpoint{1.150000in}{0.150000in}}{\pgfqpoint{5.700000in}{5.700000in}}%
\pgfusepath{clip}%
\pgfsetbuttcap%
\pgfsetroundjoin%
\definecolor{currentfill}{rgb}{0.280894,0.078907,0.402329}%
\pgfsetfillcolor{currentfill}%
\pgfsetfillopacity{0.700000}%
\pgfsetlinewidth{0.000000pt}%
\definecolor{currentstroke}{rgb}{0.000000,0.000000,0.000000}%
\pgfsetstrokecolor{currentstroke}%
\pgfsetdash{}{0pt}%
\pgfpathmoveto{\pgfqpoint{3.224748in}{2.299510in}}%
\pgfpathlineto{\pgfqpoint{3.238233in}{2.290370in}}%
\pgfpathlineto{\pgfqpoint{3.251719in}{2.281368in}}%
\pgfpathlineto{\pgfqpoint{3.265206in}{2.272502in}}%
\pgfpathlineto{\pgfqpoint{3.278695in}{2.263774in}}%
\pgfpathlineto{\pgfqpoint{3.270504in}{2.258808in}}%
\pgfpathlineto{\pgfqpoint{3.262304in}{2.253959in}}%
\pgfpathlineto{\pgfqpoint{3.254096in}{2.249227in}}%
\pgfpathlineto{\pgfqpoint{3.245878in}{2.244618in}}%
\pgfpathlineto{\pgfqpoint{3.232366in}{2.253650in}}%
\pgfpathlineto{\pgfqpoint{3.218855in}{2.262820in}}%
\pgfpathlineto{\pgfqpoint{3.205345in}{2.272126in}}%
\pgfpathlineto{\pgfqpoint{3.191837in}{2.281571in}}%
\pgfpathlineto{\pgfqpoint{3.200078in}{2.285870in}}%
\pgfpathlineto{\pgfqpoint{3.208311in}{2.290295in}}%
\pgfpathlineto{\pgfqpoint{3.216534in}{2.294842in}}%
\pgfpathlineto{\pgfqpoint{3.224748in}{2.299510in}}%
\pgfpathclose%
\pgfusepath{fill}%
\end{pgfscope}%
\begin{pgfscope}%
\pgfpathrectangle{\pgfqpoint{1.150000in}{0.150000in}}{\pgfqpoint{5.700000in}{5.700000in}}%
\pgfusepath{clip}%
\pgfsetbuttcap%
\pgfsetroundjoin%
\definecolor{currentfill}{rgb}{0.199430,0.387607,0.554642}%
\pgfsetfillcolor{currentfill}%
\pgfsetfillopacity{0.700000}%
\pgfsetlinewidth{0.000000pt}%
\definecolor{currentstroke}{rgb}{0.000000,0.000000,0.000000}%
\pgfsetstrokecolor{currentstroke}%
\pgfsetdash{}{0pt}%
\pgfpathmoveto{\pgfqpoint{5.522290in}{2.956498in}}%
\pgfpathlineto{\pgfqpoint{5.536456in}{2.962085in}}%
\pgfpathlineto{\pgfqpoint{5.550636in}{2.967772in}}%
\pgfpathlineto{\pgfqpoint{5.564831in}{2.973557in}}%
\pgfpathlineto{\pgfqpoint{5.579039in}{2.979441in}}%
\pgfpathlineto{\pgfqpoint{5.571771in}{2.973135in}}%
\pgfpathlineto{\pgfqpoint{5.564495in}{2.966757in}}%
\pgfpathlineto{\pgfqpoint{5.557211in}{2.960307in}}%
\pgfpathlineto{\pgfqpoint{5.549920in}{2.953781in}}%
\pgfpathlineto{\pgfqpoint{5.535698in}{2.947773in}}%
\pgfpathlineto{\pgfqpoint{5.521490in}{2.941864in}}%
\pgfpathlineto{\pgfqpoint{5.507296in}{2.936054in}}%
\pgfpathlineto{\pgfqpoint{5.493116in}{2.930343in}}%
\pgfpathlineto{\pgfqpoint{5.500420in}{2.936985in}}%
\pgfpathlineto{\pgfqpoint{5.507717in}{2.943557in}}%
\pgfpathlineto{\pgfqpoint{5.515007in}{2.950061in}}%
\pgfpathlineto{\pgfqpoint{5.522290in}{2.956498in}}%
\pgfpathclose%
\pgfusepath{fill}%
\end{pgfscope}%
\begin{pgfscope}%
\pgfpathrectangle{\pgfqpoint{1.150000in}{0.150000in}}{\pgfqpoint{5.700000in}{5.700000in}}%
\pgfusepath{clip}%
\pgfsetbuttcap%
\pgfsetroundjoin%
\definecolor{currentfill}{rgb}{0.273809,0.031497,0.358853}%
\pgfsetfillcolor{currentfill}%
\pgfsetfillopacity{0.700000}%
\pgfsetlinewidth{0.000000pt}%
\definecolor{currentstroke}{rgb}{0.000000,0.000000,0.000000}%
\pgfsetstrokecolor{currentstroke}%
\pgfsetdash{}{0pt}%
\pgfpathmoveto{\pgfqpoint{3.419177in}{2.221990in}}%
\pgfpathlineto{\pgfqpoint{3.432664in}{2.214736in}}%
\pgfpathlineto{\pgfqpoint{3.446153in}{2.207610in}}%
\pgfpathlineto{\pgfqpoint{3.459645in}{2.200611in}}%
\pgfpathlineto{\pgfqpoint{3.473139in}{2.193740in}}%
\pgfpathlineto{\pgfqpoint{3.465044in}{2.187509in}}%
\pgfpathlineto{\pgfqpoint{3.456940in}{2.181368in}}%
\pgfpathlineto{\pgfqpoint{3.448829in}{2.175320in}}%
\pgfpathlineto{\pgfqpoint{3.440710in}{2.169369in}}%
\pgfpathlineto{\pgfqpoint{3.427196in}{2.176524in}}%
\pgfpathlineto{\pgfqpoint{3.413684in}{2.183805in}}%
\pgfpathlineto{\pgfqpoint{3.400175in}{2.191215in}}%
\pgfpathlineto{\pgfqpoint{3.386669in}{2.198753in}}%
\pgfpathlineto{\pgfqpoint{3.394808in}{2.204414in}}%
\pgfpathlineto{\pgfqpoint{3.402939in}{2.210175in}}%
\pgfpathlineto{\pgfqpoint{3.411062in}{2.216035in}}%
\pgfpathlineto{\pgfqpoint{3.419177in}{2.221990in}}%
\pgfpathclose%
\pgfusepath{fill}%
\end{pgfscope}%
\begin{pgfscope}%
\pgfpathrectangle{\pgfqpoint{1.150000in}{0.150000in}}{\pgfqpoint{5.700000in}{5.700000in}}%
\pgfusepath{clip}%
\pgfsetbuttcap%
\pgfsetroundjoin%
\definecolor{currentfill}{rgb}{0.190631,0.407061,0.556089}%
\pgfsetfillcolor{currentfill}%
\pgfsetfillopacity{0.700000}%
\pgfsetlinewidth{0.000000pt}%
\definecolor{currentstroke}{rgb}{0.000000,0.000000,0.000000}%
\pgfsetstrokecolor{currentstroke}%
\pgfsetdash{}{0pt}%
\pgfpathmoveto{\pgfqpoint{5.608037in}{3.003984in}}%
\pgfpathlineto{\pgfqpoint{5.622245in}{3.009824in}}%
\pgfpathlineto{\pgfqpoint{5.636467in}{3.015762in}}%
\pgfpathlineto{\pgfqpoint{5.650704in}{3.021799in}}%
\pgfpathlineto{\pgfqpoint{5.664954in}{3.027934in}}%
\pgfpathlineto{\pgfqpoint{5.657731in}{3.022047in}}%
\pgfpathlineto{\pgfqpoint{5.650501in}{3.016091in}}%
\pgfpathlineto{\pgfqpoint{5.643262in}{3.010064in}}%
\pgfpathlineto{\pgfqpoint{5.636016in}{3.003964in}}%
\pgfpathlineto{\pgfqpoint{5.621750in}{2.997685in}}%
\pgfpathlineto{\pgfqpoint{5.607499in}{2.991505in}}%
\pgfpathlineto{\pgfqpoint{5.593262in}{2.985424in}}%
\pgfpathlineto{\pgfqpoint{5.579039in}{2.979441in}}%
\pgfpathlineto{\pgfqpoint{5.586300in}{2.985677in}}%
\pgfpathlineto{\pgfqpoint{5.593553in}{2.991845in}}%
\pgfpathlineto{\pgfqpoint{5.600799in}{2.997947in}}%
\pgfpathlineto{\pgfqpoint{5.608037in}{3.003984in}}%
\pgfpathclose%
\pgfusepath{fill}%
\end{pgfscope}%
\begin{pgfscope}%
\pgfpathrectangle{\pgfqpoint{1.150000in}{0.150000in}}{\pgfqpoint{5.700000in}{5.700000in}}%
\pgfusepath{clip}%
\pgfsetbuttcap%
\pgfsetroundjoin%
\definecolor{currentfill}{rgb}{0.277018,0.050344,0.375715}%
\pgfsetfillcolor{currentfill}%
\pgfsetfillopacity{0.700000}%
\pgfsetlinewidth{0.000000pt}%
\definecolor{currentstroke}{rgb}{0.000000,0.000000,0.000000}%
\pgfsetstrokecolor{currentstroke}%
\pgfsetdash{}{0pt}%
\pgfpathmoveto{\pgfqpoint{4.151073in}{2.244012in}}%
\pgfpathlineto{\pgfqpoint{4.164687in}{2.242762in}}%
\pgfpathlineto{\pgfqpoint{4.178308in}{2.241621in}}%
\pgfpathlineto{\pgfqpoint{4.191936in}{2.240587in}}%
\pgfpathlineto{\pgfqpoint{4.205573in}{2.239662in}}%
\pgfpathlineto{\pgfqpoint{4.197760in}{2.230252in}}%
\pgfpathlineto{\pgfqpoint{4.189941in}{2.220838in}}%
\pgfpathlineto{\pgfqpoint{4.182117in}{2.211423in}}%
\pgfpathlineto{\pgfqpoint{4.174287in}{2.202008in}}%
\pgfpathlineto{\pgfqpoint{4.160641in}{2.203107in}}%
\pgfpathlineto{\pgfqpoint{4.147003in}{2.204313in}}%
\pgfpathlineto{\pgfqpoint{4.133372in}{2.205628in}}%
\pgfpathlineto{\pgfqpoint{4.119749in}{2.207051in}}%
\pgfpathlineto{\pgfqpoint{4.127588in}{2.216286in}}%
\pgfpathlineto{\pgfqpoint{4.135422in}{2.225526in}}%
\pgfpathlineto{\pgfqpoint{4.143250in}{2.234768in}}%
\pgfpathlineto{\pgfqpoint{4.151073in}{2.244012in}}%
\pgfpathclose%
\pgfusepath{fill}%
\end{pgfscope}%
\begin{pgfscope}%
\pgfpathrectangle{\pgfqpoint{1.150000in}{0.150000in}}{\pgfqpoint{5.700000in}{5.700000in}}%
\pgfusepath{clip}%
\pgfsetbuttcap%
\pgfsetroundjoin%
\definecolor{currentfill}{rgb}{0.182256,0.426184,0.557120}%
\pgfsetfillcolor{currentfill}%
\pgfsetfillopacity{0.700000}%
\pgfsetlinewidth{0.000000pt}%
\definecolor{currentstroke}{rgb}{0.000000,0.000000,0.000000}%
\pgfsetstrokecolor{currentstroke}%
\pgfsetdash{}{0pt}%
\pgfpathmoveto{\pgfqpoint{5.693771in}{3.050827in}}%
\pgfpathlineto{\pgfqpoint{5.708020in}{3.056899in}}%
\pgfpathlineto{\pgfqpoint{5.722284in}{3.063069in}}%
\pgfpathlineto{\pgfqpoint{5.736562in}{3.069337in}}%
\pgfpathlineto{\pgfqpoint{5.750855in}{3.075704in}}%
\pgfpathlineto{\pgfqpoint{5.743679in}{3.070244in}}%
\pgfpathlineto{\pgfqpoint{5.736495in}{3.064719in}}%
\pgfpathlineto{\pgfqpoint{5.729303in}{3.059125in}}%
\pgfpathlineto{\pgfqpoint{5.722103in}{3.053461in}}%
\pgfpathlineto{\pgfqpoint{5.707794in}{3.046932in}}%
\pgfpathlineto{\pgfqpoint{5.693500in}{3.040501in}}%
\pgfpathlineto{\pgfqpoint{5.679220in}{3.034168in}}%
\pgfpathlineto{\pgfqpoint{5.664954in}{3.027934in}}%
\pgfpathlineto{\pgfqpoint{5.672170in}{3.033754in}}%
\pgfpathlineto{\pgfqpoint{5.679378in}{3.039508in}}%
\pgfpathlineto{\pgfqpoint{5.686578in}{3.045198in}}%
\pgfpathlineto{\pgfqpoint{5.693771in}{3.050827in}}%
\pgfpathclose%
\pgfusepath{fill}%
\end{pgfscope}%
\begin{pgfscope}%
\pgfpathrectangle{\pgfqpoint{1.150000in}{0.150000in}}{\pgfqpoint{5.700000in}{5.700000in}}%
\pgfusepath{clip}%
\pgfsetbuttcap%
\pgfsetroundjoin%
\definecolor{currentfill}{rgb}{0.174274,0.445044,0.557792}%
\pgfsetfillcolor{currentfill}%
\pgfsetfillopacity{0.700000}%
\pgfsetlinewidth{0.000000pt}%
\definecolor{currentstroke}{rgb}{0.000000,0.000000,0.000000}%
\pgfsetstrokecolor{currentstroke}%
\pgfsetdash{}{0pt}%
\pgfpathmoveto{\pgfqpoint{5.779483in}{3.096921in}}%
\pgfpathlineto{\pgfqpoint{5.793773in}{3.103204in}}%
\pgfpathlineto{\pgfqpoint{5.808078in}{3.109586in}}%
\pgfpathlineto{\pgfqpoint{5.822399in}{3.116065in}}%
\pgfpathlineto{\pgfqpoint{5.836734in}{3.122643in}}%
\pgfpathlineto{\pgfqpoint{5.829606in}{3.117616in}}%
\pgfpathlineto{\pgfqpoint{5.822470in}{3.112527in}}%
\pgfpathlineto{\pgfqpoint{5.815327in}{3.107373in}}%
\pgfpathlineto{\pgfqpoint{5.808176in}{3.102153in}}%
\pgfpathlineto{\pgfqpoint{5.793823in}{3.095393in}}%
\pgfpathlineto{\pgfqpoint{5.779485in}{3.088732in}}%
\pgfpathlineto{\pgfqpoint{5.765163in}{3.082169in}}%
\pgfpathlineto{\pgfqpoint{5.750855in}{3.075704in}}%
\pgfpathlineto{\pgfqpoint{5.758023in}{3.081099in}}%
\pgfpathlineto{\pgfqpoint{5.765184in}{3.086432in}}%
\pgfpathlineto{\pgfqpoint{5.772337in}{3.091705in}}%
\pgfpathlineto{\pgfqpoint{5.779483in}{3.096921in}}%
\pgfpathclose%
\pgfusepath{fill}%
\end{pgfscope}%
\begin{pgfscope}%
\pgfpathrectangle{\pgfqpoint{1.150000in}{0.150000in}}{\pgfqpoint{5.700000in}{5.700000in}}%
\pgfusepath{clip}%
\pgfsetbuttcap%
\pgfsetroundjoin%
\definecolor{currentfill}{rgb}{0.160665,0.478540,0.558115}%
\pgfsetfillcolor{currentfill}%
\pgfsetfillopacity{0.700000}%
\pgfsetlinewidth{0.000000pt}%
\definecolor{currentstroke}{rgb}{0.000000,0.000000,0.000000}%
\pgfsetstrokecolor{currentstroke}%
\pgfsetdash{}{0pt}%
\pgfpathmoveto{\pgfqpoint{5.950817in}{3.186497in}}%
\pgfpathlineto{\pgfqpoint{5.965189in}{3.193143in}}%
\pgfpathlineto{\pgfqpoint{5.979576in}{3.199886in}}%
\pgfpathlineto{\pgfqpoint{5.993979in}{3.206727in}}%
\pgfpathlineto{\pgfqpoint{6.008397in}{3.213666in}}%
\pgfpathlineto{\pgfqpoint{6.001370in}{3.209508in}}%
\pgfpathlineto{\pgfqpoint{5.994336in}{3.205297in}}%
\pgfpathlineto{\pgfqpoint{5.987294in}{3.201032in}}%
\pgfpathlineto{\pgfqpoint{5.980244in}{3.196710in}}%
\pgfpathlineto{\pgfqpoint{5.965806in}{3.189550in}}%
\pgfpathlineto{\pgfqpoint{5.951383in}{3.182487in}}%
\pgfpathlineto{\pgfqpoint{5.936975in}{3.175523in}}%
\pgfpathlineto{\pgfqpoint{5.922583in}{3.168657in}}%
\pgfpathlineto{\pgfqpoint{5.929653in}{3.173194in}}%
\pgfpathlineto{\pgfqpoint{5.936715in}{3.177678in}}%
\pgfpathlineto{\pgfqpoint{5.943770in}{3.182111in}}%
\pgfpathlineto{\pgfqpoint{5.950817in}{3.186497in}}%
\pgfpathclose%
\pgfusepath{fill}%
\end{pgfscope}%
\begin{pgfscope}%
\pgfpathrectangle{\pgfqpoint{1.150000in}{0.150000in}}{\pgfqpoint{5.700000in}{5.700000in}}%
\pgfusepath{clip}%
\pgfsetbuttcap%
\pgfsetroundjoin%
\definecolor{currentfill}{rgb}{0.166617,0.463708,0.558119}%
\pgfsetfillcolor{currentfill}%
\pgfsetfillopacity{0.700000}%
\pgfsetlinewidth{0.000000pt}%
\definecolor{currentstroke}{rgb}{0.000000,0.000000,0.000000}%
\pgfsetstrokecolor{currentstroke}%
\pgfsetdash{}{0pt}%
\pgfpathmoveto{\pgfqpoint{5.865167in}{3.142171in}}%
\pgfpathlineto{\pgfqpoint{5.879499in}{3.148646in}}%
\pgfpathlineto{\pgfqpoint{5.893845in}{3.155219in}}%
\pgfpathlineto{\pgfqpoint{5.908207in}{3.161889in}}%
\pgfpathlineto{\pgfqpoint{5.922583in}{3.168657in}}%
\pgfpathlineto{\pgfqpoint{5.915506in}{3.164065in}}%
\pgfpathlineto{\pgfqpoint{5.908420in}{3.159416in}}%
\pgfpathlineto{\pgfqpoint{5.901326in}{3.154706in}}%
\pgfpathlineto{\pgfqpoint{5.894225in}{3.149934in}}%
\pgfpathlineto{\pgfqpoint{5.879829in}{3.142964in}}%
\pgfpathlineto{\pgfqpoint{5.865449in}{3.136092in}}%
\pgfpathlineto{\pgfqpoint{5.851084in}{3.129318in}}%
\pgfpathlineto{\pgfqpoint{5.836734in}{3.122643in}}%
\pgfpathlineto{\pgfqpoint{5.843853in}{3.127609in}}%
\pgfpathlineto{\pgfqpoint{5.850966in}{3.132518in}}%
\pgfpathlineto{\pgfqpoint{5.858070in}{3.137371in}}%
\pgfpathlineto{\pgfqpoint{5.865167in}{3.142171in}}%
\pgfpathclose%
\pgfusepath{fill}%
\end{pgfscope}%
\begin{pgfscope}%
\pgfpathrectangle{\pgfqpoint{1.150000in}{0.150000in}}{\pgfqpoint{5.700000in}{5.700000in}}%
\pgfusepath{clip}%
\pgfsetbuttcap%
\pgfsetroundjoin%
\definecolor{currentfill}{rgb}{0.281412,0.155834,0.469201}%
\pgfsetfillcolor{currentfill}%
\pgfsetfillopacity{0.700000}%
\pgfsetlinewidth{0.000000pt}%
\definecolor{currentstroke}{rgb}{0.000000,0.000000,0.000000}%
\pgfsetstrokecolor{currentstroke}%
\pgfsetdash{}{0pt}%
\pgfpathmoveto{\pgfqpoint{2.975719in}{2.452283in}}%
\pgfpathlineto{\pgfqpoint{2.989230in}{2.440492in}}%
\pgfpathlineto{\pgfqpoint{3.002740in}{2.428855in}}%
\pgfpathlineto{\pgfqpoint{3.016250in}{2.417372in}}%
\pgfpathlineto{\pgfqpoint{3.029758in}{2.406042in}}%
\pgfpathlineto{\pgfqpoint{3.021428in}{2.402826in}}%
\pgfpathlineto{\pgfqpoint{3.013087in}{2.399757in}}%
\pgfpathlineto{\pgfqpoint{3.004736in}{2.396838in}}%
\pgfpathlineto{\pgfqpoint{2.996373in}{2.394071in}}%
\pgfpathlineto{\pgfqpoint{2.982836in}{2.405728in}}%
\pgfpathlineto{\pgfqpoint{2.969299in}{2.417539in}}%
\pgfpathlineto{\pgfqpoint{2.955759in}{2.429503in}}%
\pgfpathlineto{\pgfqpoint{2.942219in}{2.441622in}}%
\pgfpathlineto{\pgfqpoint{2.950610in}{2.444054in}}%
\pgfpathlineto{\pgfqpoint{2.958991in}{2.446644in}}%
\pgfpathlineto{\pgfqpoint{2.967360in}{2.449388in}}%
\pgfpathlineto{\pgfqpoint{2.975719in}{2.452283in}}%
\pgfpathclose%
\pgfusepath{fill}%
\end{pgfscope}%
\begin{pgfscope}%
\pgfpathrectangle{\pgfqpoint{1.150000in}{0.150000in}}{\pgfqpoint{5.700000in}{5.700000in}}%
\pgfusepath{clip}%
\pgfsetbuttcap%
\pgfsetroundjoin%
\definecolor{currentfill}{rgb}{0.268510,0.009605,0.335427}%
\pgfsetfillcolor{currentfill}%
\pgfsetfillopacity{0.700000}%
\pgfsetlinewidth{0.000000pt}%
\definecolor{currentstroke}{rgb}{0.000000,0.000000,0.000000}%
\pgfsetstrokecolor{currentstroke}%
\pgfsetdash{}{0pt}%
\pgfpathmoveto{\pgfqpoint{3.839458in}{2.177740in}}%
\pgfpathlineto{\pgfqpoint{3.852998in}{2.174133in}}%
\pgfpathlineto{\pgfqpoint{3.866544in}{2.170640in}}%
\pgfpathlineto{\pgfqpoint{3.880095in}{2.167260in}}%
\pgfpathlineto{\pgfqpoint{3.893653in}{2.163995in}}%
\pgfpathlineto{\pgfqpoint{3.885730in}{2.155537in}}%
\pgfpathlineto{\pgfqpoint{3.877801in}{2.147115in}}%
\pgfpathlineto{\pgfqpoint{3.869867in}{2.138731in}}%
\pgfpathlineto{\pgfqpoint{3.861927in}{2.130386in}}%
\pgfpathlineto{\pgfqpoint{3.848356in}{2.133879in}}%
\pgfpathlineto{\pgfqpoint{3.834792in}{2.137485in}}%
\pgfpathlineto{\pgfqpoint{3.821233in}{2.141206in}}%
\pgfpathlineto{\pgfqpoint{3.807680in}{2.145041in}}%
\pgfpathlineto{\pgfqpoint{3.815633in}{2.153151in}}%
\pgfpathlineto{\pgfqpoint{3.823581in}{2.161306in}}%
\pgfpathlineto{\pgfqpoint{3.831523in}{2.169503in}}%
\pgfpathlineto{\pgfqpoint{3.839458in}{2.177740in}}%
\pgfpathclose%
\pgfusepath{fill}%
\end{pgfscope}%
\begin{pgfscope}%
\pgfpathrectangle{\pgfqpoint{1.150000in}{0.150000in}}{\pgfqpoint{5.700000in}{5.700000in}}%
\pgfusepath{clip}%
\pgfsetbuttcap%
\pgfsetroundjoin%
\definecolor{currentfill}{rgb}{0.273809,0.031497,0.358853}%
\pgfsetfillcolor{currentfill}%
\pgfsetfillopacity{0.700000}%
\pgfsetlinewidth{0.000000pt}%
\definecolor{currentstroke}{rgb}{0.000000,0.000000,0.000000}%
\pgfsetstrokecolor{currentstroke}%
\pgfsetdash{}{0pt}%
\pgfpathmoveto{\pgfqpoint{4.065330in}{2.213835in}}%
\pgfpathlineto{\pgfqpoint{4.078924in}{2.211975in}}%
\pgfpathlineto{\pgfqpoint{4.092525in}{2.210224in}}%
\pgfpathlineto{\pgfqpoint{4.106133in}{2.208583in}}%
\pgfpathlineto{\pgfqpoint{4.119749in}{2.207051in}}%
\pgfpathlineto{\pgfqpoint{4.111904in}{2.197822in}}%
\pgfpathlineto{\pgfqpoint{4.104055in}{2.188601in}}%
\pgfpathlineto{\pgfqpoint{4.096199in}{2.179388in}}%
\pgfpathlineto{\pgfqpoint{4.088339in}{2.170187in}}%
\pgfpathlineto{\pgfqpoint{4.074713in}{2.171910in}}%
\pgfpathlineto{\pgfqpoint{4.061094in}{2.173743in}}%
\pgfpathlineto{\pgfqpoint{4.047483in}{2.175684in}}%
\pgfpathlineto{\pgfqpoint{4.033878in}{2.177736in}}%
\pgfpathlineto{\pgfqpoint{4.041750in}{2.186739in}}%
\pgfpathlineto{\pgfqpoint{4.049615in}{2.195758in}}%
\pgfpathlineto{\pgfqpoint{4.057476in}{2.204791in}}%
\pgfpathlineto{\pgfqpoint{4.065330in}{2.213835in}}%
\pgfpathclose%
\pgfusepath{fill}%
\end{pgfscope}%
\begin{pgfscope}%
\pgfpathrectangle{\pgfqpoint{1.150000in}{0.150000in}}{\pgfqpoint{5.700000in}{5.700000in}}%
\pgfusepath{clip}%
\pgfsetbuttcap%
\pgfsetroundjoin%
\definecolor{currentfill}{rgb}{0.282884,0.135920,0.453427}%
\pgfsetfillcolor{currentfill}%
\pgfsetfillopacity{0.700000}%
\pgfsetlinewidth{0.000000pt}%
\definecolor{currentstroke}{rgb}{0.000000,0.000000,0.000000}%
\pgfsetstrokecolor{currentstroke}%
\pgfsetdash{}{0pt}%
\pgfpathmoveto{\pgfqpoint{3.029758in}{2.406042in}}%
\pgfpathlineto{\pgfqpoint{3.043266in}{2.394863in}}%
\pgfpathlineto{\pgfqpoint{3.056772in}{2.383833in}}%
\pgfpathlineto{\pgfqpoint{3.070279in}{2.372953in}}%
\pgfpathlineto{\pgfqpoint{3.083785in}{2.362221in}}%
\pgfpathlineto{\pgfqpoint{3.075482in}{2.358687in}}%
\pgfpathlineto{\pgfqpoint{3.067169in}{2.355294in}}%
\pgfpathlineto{\pgfqpoint{3.058846in}{2.352045in}}%
\pgfpathlineto{\pgfqpoint{3.050511in}{2.348945in}}%
\pgfpathlineto{\pgfqpoint{3.036978in}{2.360003in}}%
\pgfpathlineto{\pgfqpoint{3.023444in}{2.371209in}}%
\pgfpathlineto{\pgfqpoint{3.009909in}{2.382564in}}%
\pgfpathlineto{\pgfqpoint{2.996373in}{2.394071in}}%
\pgfpathlineto{\pgfqpoint{3.004736in}{2.396838in}}%
\pgfpathlineto{\pgfqpoint{3.013087in}{2.399757in}}%
\pgfpathlineto{\pgfqpoint{3.021428in}{2.402826in}}%
\pgfpathlineto{\pgfqpoint{3.029758in}{2.406042in}}%
\pgfpathclose%
\pgfusepath{fill}%
\end{pgfscope}%
\begin{pgfscope}%
\pgfpathrectangle{\pgfqpoint{1.150000in}{0.150000in}}{\pgfqpoint{5.700000in}{5.700000in}}%
\pgfusepath{clip}%
\pgfsetbuttcap%
\pgfsetroundjoin%
\definecolor{currentfill}{rgb}{0.278791,0.062145,0.386592}%
\pgfsetfillcolor{currentfill}%
\pgfsetfillopacity{0.700000}%
\pgfsetlinewidth{0.000000pt}%
\definecolor{currentstroke}{rgb}{0.000000,0.000000,0.000000}%
\pgfsetstrokecolor{currentstroke}%
\pgfsetdash{}{0pt}%
\pgfpathmoveto{\pgfqpoint{3.278695in}{2.263774in}}%
\pgfpathlineto{\pgfqpoint{3.292185in}{2.255180in}}%
\pgfpathlineto{\pgfqpoint{3.305677in}{2.246722in}}%
\pgfpathlineto{\pgfqpoint{3.319170in}{2.238397in}}%
\pgfpathlineto{\pgfqpoint{3.332666in}{2.230205in}}%
\pgfpathlineto{\pgfqpoint{3.324497in}{2.224943in}}%
\pgfpathlineto{\pgfqpoint{3.316320in}{2.219793in}}%
\pgfpathlineto{\pgfqpoint{3.308134in}{2.214756in}}%
\pgfpathlineto{\pgfqpoint{3.299940in}{2.209835in}}%
\pgfpathlineto{\pgfqpoint{3.286422in}{2.218330in}}%
\pgfpathlineto{\pgfqpoint{3.272906in}{2.226958in}}%
\pgfpathlineto{\pgfqpoint{3.259391in}{2.235721in}}%
\pgfpathlineto{\pgfqpoint{3.245878in}{2.244618in}}%
\pgfpathlineto{\pgfqpoint{3.254096in}{2.249227in}}%
\pgfpathlineto{\pgfqpoint{3.262304in}{2.253959in}}%
\pgfpathlineto{\pgfqpoint{3.270504in}{2.258808in}}%
\pgfpathlineto{\pgfqpoint{3.278695in}{2.263774in}}%
\pgfpathclose%
\pgfusepath{fill}%
\end{pgfscope}%
\begin{pgfscope}%
\pgfpathrectangle{\pgfqpoint{1.150000in}{0.150000in}}{\pgfqpoint{5.700000in}{5.700000in}}%
\pgfusepath{clip}%
\pgfsetbuttcap%
\pgfsetroundjoin%
\definecolor{currentfill}{rgb}{0.268510,0.009605,0.335427}%
\pgfsetfillcolor{currentfill}%
\pgfsetfillopacity{0.700000}%
\pgfsetlinewidth{0.000000pt}%
\definecolor{currentstroke}{rgb}{0.000000,0.000000,0.000000}%
\pgfsetstrokecolor{currentstroke}%
\pgfsetdash{}{0pt}%
\pgfpathmoveto{\pgfqpoint{3.613386in}{2.171162in}}%
\pgfpathlineto{\pgfqpoint{3.626895in}{2.165669in}}%
\pgfpathlineto{\pgfqpoint{3.640409in}{2.160296in}}%
\pgfpathlineto{\pgfqpoint{3.653927in}{2.155044in}}%
\pgfpathlineto{\pgfqpoint{3.667450in}{2.149911in}}%
\pgfpathlineto{\pgfqpoint{3.659435in}{2.142569in}}%
\pgfpathlineto{\pgfqpoint{3.651414in}{2.135294in}}%
\pgfpathlineto{\pgfqpoint{3.643385in}{2.128088in}}%
\pgfpathlineto{\pgfqpoint{3.635350in}{2.120955in}}%
\pgfpathlineto{\pgfqpoint{3.621811in}{2.126352in}}%
\pgfpathlineto{\pgfqpoint{3.608276in}{2.131868in}}%
\pgfpathlineto{\pgfqpoint{3.594746in}{2.137505in}}%
\pgfpathlineto{\pgfqpoint{3.581219in}{2.143263in}}%
\pgfpathlineto{\pgfqpoint{3.589271in}{2.150126in}}%
\pgfpathlineto{\pgfqpoint{3.597316in}{2.157065in}}%
\pgfpathlineto{\pgfqpoint{3.605354in}{2.164078in}}%
\pgfpathlineto{\pgfqpoint{3.613386in}{2.171162in}}%
\pgfpathclose%
\pgfusepath{fill}%
\end{pgfscope}%
\begin{pgfscope}%
\pgfpathrectangle{\pgfqpoint{1.150000in}{0.150000in}}{\pgfqpoint{5.700000in}{5.700000in}}%
\pgfusepath{clip}%
\pgfsetbuttcap%
\pgfsetroundjoin%
\definecolor{currentfill}{rgb}{0.272594,0.025563,0.353093}%
\pgfsetfillcolor{currentfill}%
\pgfsetfillopacity{0.700000}%
\pgfsetlinewidth{0.000000pt}%
\definecolor{currentstroke}{rgb}{0.000000,0.000000,0.000000}%
\pgfsetstrokecolor{currentstroke}%
\pgfsetdash{}{0pt}%
\pgfpathmoveto{\pgfqpoint{3.473139in}{2.193740in}}%
\pgfpathlineto{\pgfqpoint{3.486637in}{2.186995in}}%
\pgfpathlineto{\pgfqpoint{3.500139in}{2.180376in}}%
\pgfpathlineto{\pgfqpoint{3.513643in}{2.173882in}}%
\pgfpathlineto{\pgfqpoint{3.527151in}{2.167512in}}%
\pgfpathlineto{\pgfqpoint{3.519074in}{2.161004in}}%
\pgfpathlineto{\pgfqpoint{3.510990in}{2.154582in}}%
\pgfpathlineto{\pgfqpoint{3.502898in}{2.148249in}}%
\pgfpathlineto{\pgfqpoint{3.494798in}{2.142008in}}%
\pgfpathlineto{\pgfqpoint{3.481272in}{2.148661in}}%
\pgfpathlineto{\pgfqpoint{3.467748in}{2.155438in}}%
\pgfpathlineto{\pgfqpoint{3.454228in}{2.162341in}}%
\pgfpathlineto{\pgfqpoint{3.440710in}{2.169369in}}%
\pgfpathlineto{\pgfqpoint{3.448829in}{2.175320in}}%
\pgfpathlineto{\pgfqpoint{3.456940in}{2.181368in}}%
\pgfpathlineto{\pgfqpoint{3.465044in}{2.187509in}}%
\pgfpathlineto{\pgfqpoint{3.473139in}{2.193740in}}%
\pgfpathclose%
\pgfusepath{fill}%
\end{pgfscope}%
\begin{pgfscope}%
\pgfpathrectangle{\pgfqpoint{1.150000in}{0.150000in}}{\pgfqpoint{5.700000in}{5.700000in}}%
\pgfusepath{clip}%
\pgfsetbuttcap%
\pgfsetroundjoin%
\definecolor{currentfill}{rgb}{0.271305,0.019942,0.347269}%
\pgfsetfillcolor{currentfill}%
\pgfsetfillopacity{0.700000}%
\pgfsetlinewidth{0.000000pt}%
\definecolor{currentstroke}{rgb}{0.000000,0.000000,0.000000}%
\pgfsetstrokecolor{currentstroke}%
\pgfsetdash{}{0pt}%
\pgfpathmoveto{\pgfqpoint{3.979530in}{2.187047in}}%
\pgfpathlineto{\pgfqpoint{3.993107in}{2.184553in}}%
\pgfpathlineto{\pgfqpoint{4.006690in}{2.182170in}}%
\pgfpathlineto{\pgfqpoint{4.020281in}{2.179898in}}%
\pgfpathlineto{\pgfqpoint{4.033878in}{2.177736in}}%
\pgfpathlineto{\pgfqpoint{4.026002in}{2.168750in}}%
\pgfpathlineto{\pgfqpoint{4.018120in}{2.159783in}}%
\pgfpathlineto{\pgfqpoint{4.010232in}{2.150836in}}%
\pgfpathlineto{\pgfqpoint{4.002339in}{2.141913in}}%
\pgfpathlineto{\pgfqpoint{3.988730in}{2.144284in}}%
\pgfpathlineto{\pgfqpoint{3.975128in}{2.146765in}}%
\pgfpathlineto{\pgfqpoint{3.961533in}{2.149357in}}%
\pgfpathlineto{\pgfqpoint{3.947944in}{2.152061in}}%
\pgfpathlineto{\pgfqpoint{3.955849in}{2.160769in}}%
\pgfpathlineto{\pgfqpoint{3.963748in}{2.169503in}}%
\pgfpathlineto{\pgfqpoint{3.971642in}{2.178263in}}%
\pgfpathlineto{\pgfqpoint{3.979530in}{2.187047in}}%
\pgfpathclose%
\pgfusepath{fill}%
\end{pgfscope}%
\begin{pgfscope}%
\pgfpathrectangle{\pgfqpoint{1.150000in}{0.150000in}}{\pgfqpoint{5.700000in}{5.700000in}}%
\pgfusepath{clip}%
\pgfsetbuttcap%
\pgfsetroundjoin%
\definecolor{currentfill}{rgb}{0.280868,0.160771,0.472899}%
\pgfsetfillcolor{currentfill}%
\pgfsetfillopacity{0.700000}%
\pgfsetlinewidth{0.000000pt}%
\definecolor{currentstroke}{rgb}{0.000000,0.000000,0.000000}%
\pgfsetstrokecolor{currentstroke}%
\pgfsetdash{}{0pt}%
\pgfpathmoveto{\pgfqpoint{4.634445in}{2.441047in}}%
\pgfpathlineto{\pgfqpoint{4.648233in}{2.442926in}}%
\pgfpathlineto{\pgfqpoint{4.662030in}{2.444908in}}%
\pgfpathlineto{\pgfqpoint{4.675838in}{2.446993in}}%
\pgfpathlineto{\pgfqpoint{4.689657in}{2.449181in}}%
\pgfpathlineto{\pgfqpoint{4.681999in}{2.439611in}}%
\pgfpathlineto{\pgfqpoint{4.674335in}{2.429991in}}%
\pgfpathlineto{\pgfqpoint{4.666666in}{2.420322in}}%
\pgfpathlineto{\pgfqpoint{4.658991in}{2.410605in}}%
\pgfpathlineto{\pgfqpoint{4.645165in}{2.408500in}}%
\pgfpathlineto{\pgfqpoint{4.631350in}{2.406498in}}%
\pgfpathlineto{\pgfqpoint{4.617544in}{2.404599in}}%
\pgfpathlineto{\pgfqpoint{4.603750in}{2.402802in}}%
\pgfpathlineto{\pgfqpoint{4.611432in}{2.412430in}}%
\pgfpathlineto{\pgfqpoint{4.619109in}{2.422014in}}%
\pgfpathlineto{\pgfqpoint{4.626780in}{2.431553in}}%
\pgfpathlineto{\pgfqpoint{4.634445in}{2.441047in}}%
\pgfpathclose%
\pgfusepath{fill}%
\end{pgfscope}%
\begin{pgfscope}%
\pgfpathrectangle{\pgfqpoint{1.150000in}{0.150000in}}{\pgfqpoint{5.700000in}{5.700000in}}%
\pgfusepath{clip}%
\pgfsetbuttcap%
\pgfsetroundjoin%
\definecolor{currentfill}{rgb}{0.268510,0.009605,0.335427}%
\pgfsetfillcolor{currentfill}%
\pgfsetfillopacity{0.700000}%
\pgfsetlinewidth{0.000000pt}%
\definecolor{currentstroke}{rgb}{0.000000,0.000000,0.000000}%
\pgfsetstrokecolor{currentstroke}%
\pgfsetdash{}{0pt}%
\pgfpathmoveto{\pgfqpoint{3.753521in}{2.161536in}}%
\pgfpathlineto{\pgfqpoint{3.767053in}{2.157238in}}%
\pgfpathlineto{\pgfqpoint{3.780590in}{2.153056in}}%
\pgfpathlineto{\pgfqpoint{3.794132in}{2.148991in}}%
\pgfpathlineto{\pgfqpoint{3.807680in}{2.145041in}}%
\pgfpathlineto{\pgfqpoint{3.799720in}{2.136977in}}%
\pgfpathlineto{\pgfqpoint{3.791754in}{2.128962in}}%
\pgfpathlineto{\pgfqpoint{3.783782in}{2.120997in}}%
\pgfpathlineto{\pgfqpoint{3.775804in}{2.113086in}}%
\pgfpathlineto{\pgfqpoint{3.762242in}{2.117281in}}%
\pgfpathlineto{\pgfqpoint{3.748685in}{2.121592in}}%
\pgfpathlineto{\pgfqpoint{3.735134in}{2.126019in}}%
\pgfpathlineto{\pgfqpoint{3.721587in}{2.130563in}}%
\pgfpathlineto{\pgfqpoint{3.729580in}{2.138222in}}%
\pgfpathlineto{\pgfqpoint{3.737567in}{2.145938in}}%
\pgfpathlineto{\pgfqpoint{3.745547in}{2.153710in}}%
\pgfpathlineto{\pgfqpoint{3.753521in}{2.161536in}}%
\pgfpathclose%
\pgfusepath{fill}%
\end{pgfscope}%
\begin{pgfscope}%
\pgfpathrectangle{\pgfqpoint{1.150000in}{0.150000in}}{\pgfqpoint{5.700000in}{5.700000in}}%
\pgfusepath{clip}%
\pgfsetbuttcap%
\pgfsetroundjoin%
\definecolor{currentfill}{rgb}{0.282623,0.140926,0.457517}%
\pgfsetfillcolor{currentfill}%
\pgfsetfillopacity{0.700000}%
\pgfsetlinewidth{0.000000pt}%
\definecolor{currentstroke}{rgb}{0.000000,0.000000,0.000000}%
\pgfsetstrokecolor{currentstroke}%
\pgfsetdash{}{0pt}%
\pgfpathmoveto{\pgfqpoint{4.548671in}{2.396650in}}%
\pgfpathlineto{\pgfqpoint{4.562426in}{2.398033in}}%
\pgfpathlineto{\pgfqpoint{4.576190in}{2.399519in}}%
\pgfpathlineto{\pgfqpoint{4.589965in}{2.401109in}}%
\pgfpathlineto{\pgfqpoint{4.603750in}{2.402802in}}%
\pgfpathlineto{\pgfqpoint{4.596062in}{2.393131in}}%
\pgfpathlineto{\pgfqpoint{4.588369in}{2.383417in}}%
\pgfpathlineto{\pgfqpoint{4.580670in}{2.373661in}}%
\pgfpathlineto{\pgfqpoint{4.572966in}{2.363863in}}%
\pgfpathlineto{\pgfqpoint{4.559174in}{2.362271in}}%
\pgfpathlineto{\pgfqpoint{4.545392in}{2.360782in}}%
\pgfpathlineto{\pgfqpoint{4.531620in}{2.359397in}}%
\pgfpathlineto{\pgfqpoint{4.517858in}{2.358115in}}%
\pgfpathlineto{\pgfqpoint{4.525569in}{2.367805in}}%
\pgfpathlineto{\pgfqpoint{4.533275in}{2.377458in}}%
\pgfpathlineto{\pgfqpoint{4.540976in}{2.387073in}}%
\pgfpathlineto{\pgfqpoint{4.548671in}{2.396650in}}%
\pgfpathclose%
\pgfusepath{fill}%
\end{pgfscope}%
\begin{pgfscope}%
\pgfpathrectangle{\pgfqpoint{1.150000in}{0.150000in}}{\pgfqpoint{5.700000in}{5.700000in}}%
\pgfusepath{clip}%
\pgfsetbuttcap%
\pgfsetroundjoin%
\definecolor{currentfill}{rgb}{0.277134,0.185228,0.489898}%
\pgfsetfillcolor{currentfill}%
\pgfsetfillopacity{0.700000}%
\pgfsetlinewidth{0.000000pt}%
\definecolor{currentstroke}{rgb}{0.000000,0.000000,0.000000}%
\pgfsetstrokecolor{currentstroke}%
\pgfsetdash{}{0pt}%
\pgfpathmoveto{\pgfqpoint{4.720233in}{2.486954in}}%
\pgfpathlineto{\pgfqpoint{4.734055in}{2.489309in}}%
\pgfpathlineto{\pgfqpoint{4.747887in}{2.491766in}}%
\pgfpathlineto{\pgfqpoint{4.761730in}{2.494325in}}%
\pgfpathlineto{\pgfqpoint{4.775584in}{2.496986in}}%
\pgfpathlineto{\pgfqpoint{4.767956in}{2.487562in}}%
\pgfpathlineto{\pgfqpoint{4.760323in}{2.478082in}}%
\pgfpathlineto{\pgfqpoint{4.752683in}{2.468546in}}%
\pgfpathlineto{\pgfqpoint{4.745038in}{2.458955in}}%
\pgfpathlineto{\pgfqpoint{4.731177in}{2.456358in}}%
\pgfpathlineto{\pgfqpoint{4.717326in}{2.453864in}}%
\pgfpathlineto{\pgfqpoint{4.703486in}{2.451471in}}%
\pgfpathlineto{\pgfqpoint{4.689657in}{2.449181in}}%
\pgfpathlineto{\pgfqpoint{4.697309in}{2.458701in}}%
\pgfpathlineto{\pgfqpoint{4.704956in}{2.468169in}}%
\pgfpathlineto{\pgfqpoint{4.712597in}{2.477587in}}%
\pgfpathlineto{\pgfqpoint{4.720233in}{2.486954in}}%
\pgfpathclose%
\pgfusepath{fill}%
\end{pgfscope}%
\begin{pgfscope}%
\pgfpathrectangle{\pgfqpoint{1.150000in}{0.150000in}}{\pgfqpoint{5.700000in}{5.700000in}}%
\pgfusepath{clip}%
\pgfsetbuttcap%
\pgfsetroundjoin%
\definecolor{currentfill}{rgb}{0.283229,0.120777,0.440584}%
\pgfsetfillcolor{currentfill}%
\pgfsetfillopacity{0.700000}%
\pgfsetlinewidth{0.000000pt}%
\definecolor{currentstroke}{rgb}{0.000000,0.000000,0.000000}%
\pgfsetstrokecolor{currentstroke}%
\pgfsetdash{}{0pt}%
\pgfpathmoveto{\pgfqpoint{4.462905in}{2.354029in}}%
\pgfpathlineto{\pgfqpoint{4.476629in}{2.354894in}}%
\pgfpathlineto{\pgfqpoint{4.490362in}{2.355864in}}%
\pgfpathlineto{\pgfqpoint{4.504105in}{2.356937in}}%
\pgfpathlineto{\pgfqpoint{4.517858in}{2.358115in}}%
\pgfpathlineto{\pgfqpoint{4.510141in}{2.348390in}}%
\pgfpathlineto{\pgfqpoint{4.502418in}{2.338629in}}%
\pgfpathlineto{\pgfqpoint{4.494691in}{2.328834in}}%
\pgfpathlineto{\pgfqpoint{4.486958in}{2.319005in}}%
\pgfpathlineto{\pgfqpoint{4.473198in}{2.317947in}}%
\pgfpathlineto{\pgfqpoint{4.459447in}{2.316992in}}%
\pgfpathlineto{\pgfqpoint{4.445706in}{2.316142in}}%
\pgfpathlineto{\pgfqpoint{4.431974in}{2.315396in}}%
\pgfpathlineto{\pgfqpoint{4.439715in}{2.325099in}}%
\pgfpathlineto{\pgfqpoint{4.447450in}{2.334772in}}%
\pgfpathlineto{\pgfqpoint{4.455180in}{2.344416in}}%
\pgfpathlineto{\pgfqpoint{4.462905in}{2.354029in}}%
\pgfpathclose%
\pgfusepath{fill}%
\end{pgfscope}%
\begin{pgfscope}%
\pgfpathrectangle{\pgfqpoint{1.150000in}{0.150000in}}{\pgfqpoint{5.700000in}{5.700000in}}%
\pgfusepath{clip}%
\pgfsetbuttcap%
\pgfsetroundjoin%
\definecolor{currentfill}{rgb}{0.271828,0.209303,0.504434}%
\pgfsetfillcolor{currentfill}%
\pgfsetfillopacity{0.700000}%
\pgfsetlinewidth{0.000000pt}%
\definecolor{currentstroke}{rgb}{0.000000,0.000000,0.000000}%
\pgfsetstrokecolor{currentstroke}%
\pgfsetdash{}{0pt}%
\pgfpathmoveto{\pgfqpoint{4.806038in}{2.534118in}}%
\pgfpathlineto{\pgfqpoint{4.819895in}{2.536927in}}%
\pgfpathlineto{\pgfqpoint{4.833764in}{2.539839in}}%
\pgfpathlineto{\pgfqpoint{4.847644in}{2.542852in}}%
\pgfpathlineto{\pgfqpoint{4.861535in}{2.545966in}}%
\pgfpathlineto{\pgfqpoint{4.853938in}{2.536729in}}%
\pgfpathlineto{\pgfqpoint{4.846335in}{2.527430in}}%
\pgfpathlineto{\pgfqpoint{4.838727in}{2.518071in}}%
\pgfpathlineto{\pgfqpoint{4.831112in}{2.508650in}}%
\pgfpathlineto{\pgfqpoint{4.817213in}{2.505581in}}%
\pgfpathlineto{\pgfqpoint{4.803326in}{2.502615in}}%
\pgfpathlineto{\pgfqpoint{4.789450in}{2.499750in}}%
\pgfpathlineto{\pgfqpoint{4.775584in}{2.496986in}}%
\pgfpathlineto{\pgfqpoint{4.783206in}{2.506354in}}%
\pgfpathlineto{\pgfqpoint{4.790823in}{2.515665in}}%
\pgfpathlineto{\pgfqpoint{4.798433in}{2.524920in}}%
\pgfpathlineto{\pgfqpoint{4.806038in}{2.534118in}}%
\pgfpathclose%
\pgfusepath{fill}%
\end{pgfscope}%
\begin{pgfscope}%
\pgfpathrectangle{\pgfqpoint{1.150000in}{0.150000in}}{\pgfqpoint{5.700000in}{5.700000in}}%
\pgfusepath{clip}%
\pgfsetbuttcap%
\pgfsetroundjoin%
\definecolor{currentfill}{rgb}{0.283197,0.115680,0.436115}%
\pgfsetfillcolor{currentfill}%
\pgfsetfillopacity{0.700000}%
\pgfsetlinewidth{0.000000pt}%
\definecolor{currentstroke}{rgb}{0.000000,0.000000,0.000000}%
\pgfsetstrokecolor{currentstroke}%
\pgfsetdash{}{0pt}%
\pgfpathmoveto{\pgfqpoint{3.083785in}{2.362221in}}%
\pgfpathlineto{\pgfqpoint{3.097291in}{2.351636in}}%
\pgfpathlineto{\pgfqpoint{3.110797in}{2.341197in}}%
\pgfpathlineto{\pgfqpoint{3.124302in}{2.330903in}}%
\pgfpathlineto{\pgfqpoint{3.137808in}{2.320753in}}%
\pgfpathlineto{\pgfqpoint{3.129532in}{2.316900in}}%
\pgfpathlineto{\pgfqpoint{3.121245in}{2.313184in}}%
\pgfpathlineto{\pgfqpoint{3.112949in}{2.309608in}}%
\pgfpathlineto{\pgfqpoint{3.104642in}{2.306176in}}%
\pgfpathlineto{\pgfqpoint{3.091110in}{2.316651in}}%
\pgfpathlineto{\pgfqpoint{3.077577in}{2.327270in}}%
\pgfpathlineto{\pgfqpoint{3.064045in}{2.338034in}}%
\pgfpathlineto{\pgfqpoint{3.050511in}{2.348945in}}%
\pgfpathlineto{\pgfqpoint{3.058846in}{2.352045in}}%
\pgfpathlineto{\pgfqpoint{3.067169in}{2.355294in}}%
\pgfpathlineto{\pgfqpoint{3.075482in}{2.358687in}}%
\pgfpathlineto{\pgfqpoint{3.083785in}{2.362221in}}%
\pgfpathclose%
\pgfusepath{fill}%
\end{pgfscope}%
\begin{pgfscope}%
\pgfpathrectangle{\pgfqpoint{1.150000in}{0.150000in}}{\pgfqpoint{5.700000in}{5.700000in}}%
\pgfusepath{clip}%
\pgfsetbuttcap%
\pgfsetroundjoin%
\definecolor{currentfill}{rgb}{0.265145,0.232956,0.516599}%
\pgfsetfillcolor{currentfill}%
\pgfsetfillopacity{0.700000}%
\pgfsetlinewidth{0.000000pt}%
\definecolor{currentstroke}{rgb}{0.000000,0.000000,0.000000}%
\pgfsetstrokecolor{currentstroke}%
\pgfsetdash{}{0pt}%
\pgfpathmoveto{\pgfqpoint{4.891862in}{2.582298in}}%
\pgfpathlineto{\pgfqpoint{4.905757in}{2.585542in}}%
\pgfpathlineto{\pgfqpoint{4.919663in}{2.588887in}}%
\pgfpathlineto{\pgfqpoint{4.933581in}{2.592333in}}%
\pgfpathlineto{\pgfqpoint{4.947510in}{2.595880in}}%
\pgfpathlineto{\pgfqpoint{4.939946in}{2.586868in}}%
\pgfpathlineto{\pgfqpoint{4.932375in}{2.577791in}}%
\pgfpathlineto{\pgfqpoint{4.924798in}{2.568647in}}%
\pgfpathlineto{\pgfqpoint{4.917215in}{2.559437in}}%
\pgfpathlineto{\pgfqpoint{4.903277in}{2.555918in}}%
\pgfpathlineto{\pgfqpoint{4.889352in}{2.552499in}}%
\pgfpathlineto{\pgfqpoint{4.875437in}{2.549182in}}%
\pgfpathlineto{\pgfqpoint{4.861535in}{2.545966in}}%
\pgfpathlineto{\pgfqpoint{4.869126in}{2.555141in}}%
\pgfpathlineto{\pgfqpoint{4.876710in}{2.564255in}}%
\pgfpathlineto{\pgfqpoint{4.884289in}{2.573308in}}%
\pgfpathlineto{\pgfqpoint{4.891862in}{2.582298in}}%
\pgfpathclose%
\pgfusepath{fill}%
\end{pgfscope}%
\begin{pgfscope}%
\pgfpathrectangle{\pgfqpoint{1.150000in}{0.150000in}}{\pgfqpoint{5.700000in}{5.700000in}}%
\pgfusepath{clip}%
\pgfsetbuttcap%
\pgfsetroundjoin%
\definecolor{currentfill}{rgb}{0.282656,0.100196,0.422160}%
\pgfsetfillcolor{currentfill}%
\pgfsetfillopacity{0.700000}%
\pgfsetlinewidth{0.000000pt}%
\definecolor{currentstroke}{rgb}{0.000000,0.000000,0.000000}%
\pgfsetstrokecolor{currentstroke}%
\pgfsetdash{}{0pt}%
\pgfpathmoveto{\pgfqpoint{4.377139in}{2.313461in}}%
\pgfpathlineto{\pgfqpoint{4.390834in}{2.313787in}}%
\pgfpathlineto{\pgfqpoint{4.404539in}{2.314219in}}%
\pgfpathlineto{\pgfqpoint{4.418252in}{2.314755in}}%
\pgfpathlineto{\pgfqpoint{4.431974in}{2.315396in}}%
\pgfpathlineto{\pgfqpoint{4.424228in}{2.305666in}}%
\pgfpathlineto{\pgfqpoint{4.416477in}{2.295910in}}%
\pgfpathlineto{\pgfqpoint{4.408721in}{2.286127in}}%
\pgfpathlineto{\pgfqpoint{4.400959in}{2.276320in}}%
\pgfpathlineto{\pgfqpoint{4.387229in}{2.275817in}}%
\pgfpathlineto{\pgfqpoint{4.373508in}{2.275418in}}%
\pgfpathlineto{\pgfqpoint{4.359795in}{2.275124in}}%
\pgfpathlineto{\pgfqpoint{4.346092in}{2.274935in}}%
\pgfpathlineto{\pgfqpoint{4.353862in}{2.284598in}}%
\pgfpathlineto{\pgfqpoint{4.361626in}{2.294240in}}%
\pgfpathlineto{\pgfqpoint{4.369385in}{2.303862in}}%
\pgfpathlineto{\pgfqpoint{4.377139in}{2.313461in}}%
\pgfpathclose%
\pgfusepath{fill}%
\end{pgfscope}%
\begin{pgfscope}%
\pgfpathrectangle{\pgfqpoint{1.150000in}{0.150000in}}{\pgfqpoint{5.700000in}{5.700000in}}%
\pgfusepath{clip}%
\pgfsetbuttcap%
\pgfsetroundjoin%
\definecolor{currentfill}{rgb}{0.257322,0.256130,0.526563}%
\pgfsetfillcolor{currentfill}%
\pgfsetfillopacity{0.700000}%
\pgfsetlinewidth{0.000000pt}%
\definecolor{currentstroke}{rgb}{0.000000,0.000000,0.000000}%
\pgfsetstrokecolor{currentstroke}%
\pgfsetdash{}{0pt}%
\pgfpathmoveto{\pgfqpoint{4.977707in}{2.631267in}}%
\pgfpathlineto{\pgfqpoint{4.991640in}{2.634924in}}%
\pgfpathlineto{\pgfqpoint{5.005585in}{2.638682in}}%
\pgfpathlineto{\pgfqpoint{5.019542in}{2.642541in}}%
\pgfpathlineto{\pgfqpoint{5.033511in}{2.646500in}}%
\pgfpathlineto{\pgfqpoint{5.025980in}{2.637750in}}%
\pgfpathlineto{\pgfqpoint{5.018442in}{2.628929in}}%
\pgfpathlineto{\pgfqpoint{5.010898in}{2.620039in}}%
\pgfpathlineto{\pgfqpoint{5.003347in}{2.611077in}}%
\pgfpathlineto{\pgfqpoint{4.989370in}{2.607127in}}%
\pgfpathlineto{\pgfqpoint{4.975405in}{2.603277in}}%
\pgfpathlineto{\pgfqpoint{4.961452in}{2.599528in}}%
\pgfpathlineto{\pgfqpoint{4.947510in}{2.595880in}}%
\pgfpathlineto{\pgfqpoint{4.955069in}{2.604825in}}%
\pgfpathlineto{\pgfqpoint{4.962621in}{2.613705in}}%
\pgfpathlineto{\pgfqpoint{4.970167in}{2.622519in}}%
\pgfpathlineto{\pgfqpoint{4.977707in}{2.631267in}}%
\pgfpathclose%
\pgfusepath{fill}%
\end{pgfscope}%
\begin{pgfscope}%
\pgfpathrectangle{\pgfqpoint{1.150000in}{0.150000in}}{\pgfqpoint{5.700000in}{5.700000in}}%
\pgfusepath{clip}%
\pgfsetbuttcap%
\pgfsetroundjoin%
\definecolor{currentfill}{rgb}{0.280894,0.078907,0.402329}%
\pgfsetfillcolor{currentfill}%
\pgfsetfillopacity{0.700000}%
\pgfsetlinewidth{0.000000pt}%
\definecolor{currentstroke}{rgb}{0.000000,0.000000,0.000000}%
\pgfsetstrokecolor{currentstroke}%
\pgfsetdash{}{0pt}%
\pgfpathmoveto{\pgfqpoint{4.291365in}{2.275238in}}%
\pgfpathlineto{\pgfqpoint{4.305034in}{2.275003in}}%
\pgfpathlineto{\pgfqpoint{4.318711in}{2.274874in}}%
\pgfpathlineto{\pgfqpoint{4.332397in}{2.274852in}}%
\pgfpathlineto{\pgfqpoint{4.346092in}{2.274935in}}%
\pgfpathlineto{\pgfqpoint{4.338317in}{2.265254in}}%
\pgfpathlineto{\pgfqpoint{4.330537in}{2.255555in}}%
\pgfpathlineto{\pgfqpoint{4.322751in}{2.245840in}}%
\pgfpathlineto{\pgfqpoint{4.314961in}{2.236110in}}%
\pgfpathlineto{\pgfqpoint{4.301258in}{2.236182in}}%
\pgfpathlineto{\pgfqpoint{4.287563in}{2.236360in}}%
\pgfpathlineto{\pgfqpoint{4.273878in}{2.236644in}}%
\pgfpathlineto{\pgfqpoint{4.260200in}{2.237034in}}%
\pgfpathlineto{\pgfqpoint{4.267999in}{2.246602in}}%
\pgfpathlineto{\pgfqpoint{4.275793in}{2.256159in}}%
\pgfpathlineto{\pgfqpoint{4.283582in}{2.265705in}}%
\pgfpathlineto{\pgfqpoint{4.291365in}{2.275238in}}%
\pgfpathclose%
\pgfusepath{fill}%
\end{pgfscope}%
\begin{pgfscope}%
\pgfpathrectangle{\pgfqpoint{1.150000in}{0.150000in}}{\pgfqpoint{5.700000in}{5.700000in}}%
\pgfusepath{clip}%
\pgfsetbuttcap%
\pgfsetroundjoin%
\definecolor{currentfill}{rgb}{0.248629,0.278775,0.534556}%
\pgfsetfillcolor{currentfill}%
\pgfsetfillopacity{0.700000}%
\pgfsetlinewidth{0.000000pt}%
\definecolor{currentstroke}{rgb}{0.000000,0.000000,0.000000}%
\pgfsetstrokecolor{currentstroke}%
\pgfsetdash{}{0pt}%
\pgfpathmoveto{\pgfqpoint{5.063571in}{2.680807in}}%
\pgfpathlineto{\pgfqpoint{5.077544in}{2.684857in}}%
\pgfpathlineto{\pgfqpoint{5.091529in}{2.689007in}}%
\pgfpathlineto{\pgfqpoint{5.105526in}{2.693258in}}%
\pgfpathlineto{\pgfqpoint{5.119536in}{2.697610in}}%
\pgfpathlineto{\pgfqpoint{5.112039in}{2.689153in}}%
\pgfpathlineto{\pgfqpoint{5.104536in}{2.680624in}}%
\pgfpathlineto{\pgfqpoint{5.097026in}{2.672020in}}%
\pgfpathlineto{\pgfqpoint{5.089510in}{2.663343in}}%
\pgfpathlineto{\pgfqpoint{5.075492in}{2.658981in}}%
\pgfpathlineto{\pgfqpoint{5.061486in}{2.654720in}}%
\pgfpathlineto{\pgfqpoint{5.047492in}{2.650560in}}%
\pgfpathlineto{\pgfqpoint{5.033511in}{2.646500in}}%
\pgfpathlineto{\pgfqpoint{5.041036in}{2.655180in}}%
\pgfpathlineto{\pgfqpoint{5.048554in}{2.663791in}}%
\pgfpathlineto{\pgfqpoint{5.056066in}{2.672333in}}%
\pgfpathlineto{\pgfqpoint{5.063571in}{2.680807in}}%
\pgfpathclose%
\pgfusepath{fill}%
\end{pgfscope}%
\begin{pgfscope}%
\pgfpathrectangle{\pgfqpoint{1.150000in}{0.150000in}}{\pgfqpoint{5.700000in}{5.700000in}}%
\pgfusepath{clip}%
\pgfsetbuttcap%
\pgfsetroundjoin%
\definecolor{currentfill}{rgb}{0.277018,0.050344,0.375715}%
\pgfsetfillcolor{currentfill}%
\pgfsetfillopacity{0.700000}%
\pgfsetlinewidth{0.000000pt}%
\definecolor{currentstroke}{rgb}{0.000000,0.000000,0.000000}%
\pgfsetstrokecolor{currentstroke}%
\pgfsetdash{}{0pt}%
\pgfpathmoveto{\pgfqpoint{3.332666in}{2.230205in}}%
\pgfpathlineto{\pgfqpoint{3.346163in}{2.222146in}}%
\pgfpathlineto{\pgfqpoint{3.359663in}{2.214218in}}%
\pgfpathlineto{\pgfqpoint{3.373165in}{2.206420in}}%
\pgfpathlineto{\pgfqpoint{3.386669in}{2.198753in}}%
\pgfpathlineto{\pgfqpoint{3.378521in}{2.193195in}}%
\pgfpathlineto{\pgfqpoint{3.370366in}{2.187744in}}%
\pgfpathlineto{\pgfqpoint{3.362202in}{2.182402in}}%
\pgfpathlineto{\pgfqpoint{3.354030in}{2.177172in}}%
\pgfpathlineto{\pgfqpoint{3.340504in}{2.185142in}}%
\pgfpathlineto{\pgfqpoint{3.326981in}{2.193242in}}%
\pgfpathlineto{\pgfqpoint{3.313459in}{2.201473in}}%
\pgfpathlineto{\pgfqpoint{3.299940in}{2.209835in}}%
\pgfpathlineto{\pgfqpoint{3.308134in}{2.214756in}}%
\pgfpathlineto{\pgfqpoint{3.316320in}{2.219793in}}%
\pgfpathlineto{\pgfqpoint{3.324497in}{2.224943in}}%
\pgfpathlineto{\pgfqpoint{3.332666in}{2.230205in}}%
\pgfpathclose%
\pgfusepath{fill}%
\end{pgfscope}%
\begin{pgfscope}%
\pgfpathrectangle{\pgfqpoint{1.150000in}{0.150000in}}{\pgfqpoint{5.700000in}{5.700000in}}%
\pgfusepath{clip}%
\pgfsetbuttcap%
\pgfsetroundjoin%
\definecolor{currentfill}{rgb}{0.269944,0.014625,0.341379}%
\pgfsetfillcolor{currentfill}%
\pgfsetfillopacity{0.700000}%
\pgfsetlinewidth{0.000000pt}%
\definecolor{currentstroke}{rgb}{0.000000,0.000000,0.000000}%
\pgfsetstrokecolor{currentstroke}%
\pgfsetdash{}{0pt}%
\pgfpathmoveto{\pgfqpoint{3.893653in}{2.163995in}}%
\pgfpathlineto{\pgfqpoint{3.907216in}{2.160843in}}%
\pgfpathlineto{\pgfqpoint{3.920786in}{2.157803in}}%
\pgfpathlineto{\pgfqpoint{3.934362in}{2.154876in}}%
\pgfpathlineto{\pgfqpoint{3.947944in}{2.152061in}}%
\pgfpathlineto{\pgfqpoint{3.940034in}{2.143383in}}%
\pgfpathlineto{\pgfqpoint{3.932117in}{2.134735in}}%
\pgfpathlineto{\pgfqpoint{3.924196in}{2.126121in}}%
\pgfpathlineto{\pgfqpoint{3.916268in}{2.117543in}}%
\pgfpathlineto{\pgfqpoint{3.902673in}{2.120585in}}%
\pgfpathlineto{\pgfqpoint{3.889085in}{2.123739in}}%
\pgfpathlineto{\pgfqpoint{3.875503in}{2.127006in}}%
\pgfpathlineto{\pgfqpoint{3.861927in}{2.130386in}}%
\pgfpathlineto{\pgfqpoint{3.869867in}{2.138731in}}%
\pgfpathlineto{\pgfqpoint{3.877801in}{2.147115in}}%
\pgfpathlineto{\pgfqpoint{3.885730in}{2.155537in}}%
\pgfpathlineto{\pgfqpoint{3.893653in}{2.163995in}}%
\pgfpathclose%
\pgfusepath{fill}%
\end{pgfscope}%
\begin{pgfscope}%
\pgfpathrectangle{\pgfqpoint{1.150000in}{0.150000in}}{\pgfqpoint{5.700000in}{5.700000in}}%
\pgfusepath{clip}%
\pgfsetbuttcap%
\pgfsetroundjoin%
\definecolor{currentfill}{rgb}{0.237441,0.305202,0.541921}%
\pgfsetfillcolor{currentfill}%
\pgfsetfillopacity{0.700000}%
\pgfsetlinewidth{0.000000pt}%
\definecolor{currentstroke}{rgb}{0.000000,0.000000,0.000000}%
\pgfsetstrokecolor{currentstroke}%
\pgfsetdash{}{0pt}%
\pgfpathmoveto{\pgfqpoint{5.149455in}{2.730713in}}%
\pgfpathlineto{\pgfqpoint{5.163468in}{2.735136in}}%
\pgfpathlineto{\pgfqpoint{5.177494in}{2.739659in}}%
\pgfpathlineto{\pgfqpoint{5.191532in}{2.744282in}}%
\pgfpathlineto{\pgfqpoint{5.205583in}{2.749005in}}%
\pgfpathlineto{\pgfqpoint{5.198123in}{2.740872in}}%
\pgfpathlineto{\pgfqpoint{5.190656in}{2.732664in}}%
\pgfpathlineto{\pgfqpoint{5.183182in}{2.724379in}}%
\pgfpathlineto{\pgfqpoint{5.175701in}{2.716017in}}%
\pgfpathlineto{\pgfqpoint{5.161641in}{2.711265in}}%
\pgfpathlineto{\pgfqpoint{5.147593in}{2.706613in}}%
\pgfpathlineto{\pgfqpoint{5.133558in}{2.702061in}}%
\pgfpathlineto{\pgfqpoint{5.119536in}{2.697610in}}%
\pgfpathlineto{\pgfqpoint{5.127026in}{2.705993in}}%
\pgfpathlineto{\pgfqpoint{5.134509in}{2.714305in}}%
\pgfpathlineto{\pgfqpoint{5.141985in}{2.722544in}}%
\pgfpathlineto{\pgfqpoint{5.149455in}{2.730713in}}%
\pgfpathclose%
\pgfusepath{fill}%
\end{pgfscope}%
\begin{pgfscope}%
\pgfpathrectangle{\pgfqpoint{1.150000in}{0.150000in}}{\pgfqpoint{5.700000in}{5.700000in}}%
\pgfusepath{clip}%
\pgfsetbuttcap%
\pgfsetroundjoin%
\definecolor{currentfill}{rgb}{0.278791,0.062145,0.386592}%
\pgfsetfillcolor{currentfill}%
\pgfsetfillopacity{0.700000}%
\pgfsetlinewidth{0.000000pt}%
\definecolor{currentstroke}{rgb}{0.000000,0.000000,0.000000}%
\pgfsetstrokecolor{currentstroke}%
\pgfsetdash{}{0pt}%
\pgfpathmoveto{\pgfqpoint{4.205573in}{2.239662in}}%
\pgfpathlineto{\pgfqpoint{4.219218in}{2.238844in}}%
\pgfpathlineto{\pgfqpoint{4.232870in}{2.238134in}}%
\pgfpathlineto{\pgfqpoint{4.246531in}{2.237531in}}%
\pgfpathlineto{\pgfqpoint{4.260200in}{2.237034in}}%
\pgfpathlineto{\pgfqpoint{4.252396in}{2.227457in}}%
\pgfpathlineto{\pgfqpoint{4.244586in}{2.217872in}}%
\pgfpathlineto{\pgfqpoint{4.236771in}{2.208282in}}%
\pgfpathlineto{\pgfqpoint{4.228951in}{2.198686in}}%
\pgfpathlineto{\pgfqpoint{4.215273in}{2.199357in}}%
\pgfpathlineto{\pgfqpoint{4.201603in}{2.200133in}}%
\pgfpathlineto{\pgfqpoint{4.187941in}{2.201017in}}%
\pgfpathlineto{\pgfqpoint{4.174287in}{2.202008in}}%
\pgfpathlineto{\pgfqpoint{4.182117in}{2.211423in}}%
\pgfpathlineto{\pgfqpoint{4.189941in}{2.220838in}}%
\pgfpathlineto{\pgfqpoint{4.197760in}{2.230252in}}%
\pgfpathlineto{\pgfqpoint{4.205573in}{2.239662in}}%
\pgfpathclose%
\pgfusepath{fill}%
\end{pgfscope}%
\begin{pgfscope}%
\pgfpathrectangle{\pgfqpoint{1.150000in}{0.150000in}}{\pgfqpoint{5.700000in}{5.700000in}}%
\pgfusepath{clip}%
\pgfsetbuttcap%
\pgfsetroundjoin%
\definecolor{currentfill}{rgb}{0.244972,0.287675,0.537260}%
\pgfsetfillcolor{currentfill}%
\pgfsetfillopacity{0.700000}%
\pgfsetlinewidth{0.000000pt}%
\definecolor{currentstroke}{rgb}{0.000000,0.000000,0.000000}%
\pgfsetstrokecolor{currentstroke}%
\pgfsetdash{}{0pt}%
\pgfpathmoveto{\pgfqpoint{2.670868in}{2.718558in}}%
\pgfpathlineto{\pgfqpoint{2.684470in}{2.703059in}}%
\pgfpathlineto{\pgfqpoint{2.698068in}{2.687743in}}%
\pgfpathlineto{\pgfqpoint{2.711661in}{2.672608in}}%
\pgfpathlineto{\pgfqpoint{2.725249in}{2.657653in}}%
\pgfpathlineto{\pgfqpoint{2.716716in}{2.656757in}}%
\pgfpathlineto{\pgfqpoint{2.708170in}{2.656044in}}%
\pgfpathlineto{\pgfqpoint{2.699611in}{2.655519in}}%
\pgfpathlineto{\pgfqpoint{2.691037in}{2.655185in}}%
\pgfpathlineto{\pgfqpoint{2.677413in}{2.670495in}}%
\pgfpathlineto{\pgfqpoint{2.663785in}{2.685985in}}%
\pgfpathlineto{\pgfqpoint{2.650151in}{2.701658in}}%
\pgfpathlineto{\pgfqpoint{2.636513in}{2.717514in}}%
\pgfpathlineto{\pgfqpoint{2.645123in}{2.717485in}}%
\pgfpathlineto{\pgfqpoint{2.653718in}{2.717651in}}%
\pgfpathlineto{\pgfqpoint{2.662300in}{2.718010in}}%
\pgfpathlineto{\pgfqpoint{2.670868in}{2.718558in}}%
\pgfpathclose%
\pgfusepath{fill}%
\end{pgfscope}%
\begin{pgfscope}%
\pgfpathrectangle{\pgfqpoint{1.150000in}{0.150000in}}{\pgfqpoint{5.700000in}{5.700000in}}%
\pgfusepath{clip}%
\pgfsetbuttcap%
\pgfsetroundjoin%
\definecolor{currentfill}{rgb}{0.233603,0.313828,0.543914}%
\pgfsetfillcolor{currentfill}%
\pgfsetfillopacity{0.700000}%
\pgfsetlinewidth{0.000000pt}%
\definecolor{currentstroke}{rgb}{0.000000,0.000000,0.000000}%
\pgfsetstrokecolor{currentstroke}%
\pgfsetdash{}{0pt}%
\pgfpathmoveto{\pgfqpoint{2.616410in}{2.782421in}}%
\pgfpathlineto{\pgfqpoint{2.630032in}{2.766172in}}%
\pgfpathlineto{\pgfqpoint{2.643649in}{2.750113in}}%
\pgfpathlineto{\pgfqpoint{2.657261in}{2.734242in}}%
\pgfpathlineto{\pgfqpoint{2.670868in}{2.718558in}}%
\pgfpathlineto{\pgfqpoint{2.662300in}{2.718010in}}%
\pgfpathlineto{\pgfqpoint{2.653718in}{2.717651in}}%
\pgfpathlineto{\pgfqpoint{2.645123in}{2.717485in}}%
\pgfpathlineto{\pgfqpoint{2.636513in}{2.717514in}}%
\pgfpathlineto{\pgfqpoint{2.622869in}{2.733556in}}%
\pgfpathlineto{\pgfqpoint{2.609220in}{2.749784in}}%
\pgfpathlineto{\pgfqpoint{2.595566in}{2.766202in}}%
\pgfpathlineto{\pgfqpoint{2.581906in}{2.782810in}}%
\pgfpathlineto{\pgfqpoint{2.590553in}{2.782415in}}%
\pgfpathlineto{\pgfqpoint{2.599186in}{2.782220in}}%
\pgfpathlineto{\pgfqpoint{2.607805in}{2.782223in}}%
\pgfpathlineto{\pgfqpoint{2.616410in}{2.782421in}}%
\pgfpathclose%
\pgfusepath{fill}%
\end{pgfscope}%
\begin{pgfscope}%
\pgfpathrectangle{\pgfqpoint{1.150000in}{0.150000in}}{\pgfqpoint{5.700000in}{5.700000in}}%
\pgfusepath{clip}%
\pgfsetbuttcap%
\pgfsetroundjoin%
\definecolor{currentfill}{rgb}{0.282656,0.100196,0.422160}%
\pgfsetfillcolor{currentfill}%
\pgfsetfillopacity{0.700000}%
\pgfsetlinewidth{0.000000pt}%
\definecolor{currentstroke}{rgb}{0.000000,0.000000,0.000000}%
\pgfsetstrokecolor{currentstroke}%
\pgfsetdash{}{0pt}%
\pgfpathmoveto{\pgfqpoint{3.137808in}{2.320753in}}%
\pgfpathlineto{\pgfqpoint{3.151315in}{2.310745in}}%
\pgfpathlineto{\pgfqpoint{3.164821in}{2.300880in}}%
\pgfpathlineto{\pgfqpoint{3.178329in}{2.291156in}}%
\pgfpathlineto{\pgfqpoint{3.191837in}{2.281571in}}%
\pgfpathlineto{\pgfqpoint{3.183585in}{2.277402in}}%
\pgfpathlineto{\pgfqpoint{3.175325in}{2.273364in}}%
\pgfpathlineto{\pgfqpoint{3.167054in}{2.269462in}}%
\pgfpathlineto{\pgfqpoint{3.158774in}{2.265698in}}%
\pgfpathlineto{\pgfqpoint{3.145240in}{2.275606in}}%
\pgfpathlineto{\pgfqpoint{3.131707in}{2.285654in}}%
\pgfpathlineto{\pgfqpoint{3.118175in}{2.295844in}}%
\pgfpathlineto{\pgfqpoint{3.104642in}{2.306176in}}%
\pgfpathlineto{\pgfqpoint{3.112949in}{2.309608in}}%
\pgfpathlineto{\pgfqpoint{3.121245in}{2.313184in}}%
\pgfpathlineto{\pgfqpoint{3.129532in}{2.316900in}}%
\pgfpathlineto{\pgfqpoint{3.137808in}{2.320753in}}%
\pgfpathclose%
\pgfusepath{fill}%
\end{pgfscope}%
\begin{pgfscope}%
\pgfpathrectangle{\pgfqpoint{1.150000in}{0.150000in}}{\pgfqpoint{5.700000in}{5.700000in}}%
\pgfusepath{clip}%
\pgfsetbuttcap%
\pgfsetroundjoin%
\definecolor{currentfill}{rgb}{0.227802,0.326594,0.546532}%
\pgfsetfillcolor{currentfill}%
\pgfsetfillopacity{0.700000}%
\pgfsetlinewidth{0.000000pt}%
\definecolor{currentstroke}{rgb}{0.000000,0.000000,0.000000}%
\pgfsetstrokecolor{currentstroke}%
\pgfsetdash{}{0pt}%
\pgfpathmoveto{\pgfqpoint{5.235355in}{2.780794in}}%
\pgfpathlineto{\pgfqpoint{5.249409in}{2.785570in}}%
\pgfpathlineto{\pgfqpoint{5.263476in}{2.790445in}}%
\pgfpathlineto{\pgfqpoint{5.277556in}{2.795420in}}%
\pgfpathlineto{\pgfqpoint{5.291650in}{2.800494in}}%
\pgfpathlineto{\pgfqpoint{5.284227in}{2.792712in}}%
\pgfpathlineto{\pgfqpoint{5.276798in}{2.784852in}}%
\pgfpathlineto{\pgfqpoint{5.269361in}{2.776914in}}%
\pgfpathlineto{\pgfqpoint{5.261917in}{2.768896in}}%
\pgfpathlineto{\pgfqpoint{5.247814in}{2.763774in}}%
\pgfpathlineto{\pgfqpoint{5.233724in}{2.758751in}}%
\pgfpathlineto{\pgfqpoint{5.219647in}{2.753828in}}%
\pgfpathlineto{\pgfqpoint{5.205583in}{2.749005in}}%
\pgfpathlineto{\pgfqpoint{5.213036in}{2.757063in}}%
\pgfpathlineto{\pgfqpoint{5.220483in}{2.765047in}}%
\pgfpathlineto{\pgfqpoint{5.227923in}{2.772957in}}%
\pgfpathlineto{\pgfqpoint{5.235355in}{2.780794in}}%
\pgfpathclose%
\pgfusepath{fill}%
\end{pgfscope}%
\begin{pgfscope}%
\pgfpathrectangle{\pgfqpoint{1.150000in}{0.150000in}}{\pgfqpoint{5.700000in}{5.700000in}}%
\pgfusepath{clip}%
\pgfsetbuttcap%
\pgfsetroundjoin%
\definecolor{currentfill}{rgb}{0.255645,0.260703,0.528312}%
\pgfsetfillcolor{currentfill}%
\pgfsetfillopacity{0.700000}%
\pgfsetlinewidth{0.000000pt}%
\definecolor{currentstroke}{rgb}{0.000000,0.000000,0.000000}%
\pgfsetstrokecolor{currentstroke}%
\pgfsetdash{}{0pt}%
\pgfpathmoveto{\pgfqpoint{2.725249in}{2.657653in}}%
\pgfpathlineto{\pgfqpoint{2.738834in}{2.642877in}}%
\pgfpathlineto{\pgfqpoint{2.752414in}{2.628277in}}%
\pgfpathlineto{\pgfqpoint{2.765991in}{2.613852in}}%
\pgfpathlineto{\pgfqpoint{2.779564in}{2.599601in}}%
\pgfpathlineto{\pgfqpoint{2.771065in}{2.598359in}}%
\pgfpathlineto{\pgfqpoint{2.762554in}{2.597295in}}%
\pgfpathlineto{\pgfqpoint{2.754029in}{2.596413in}}%
\pgfpathlineto{\pgfqpoint{2.745491in}{2.595717in}}%
\pgfpathlineto{\pgfqpoint{2.731884in}{2.610321in}}%
\pgfpathlineto{\pgfqpoint{2.718272in}{2.625100in}}%
\pgfpathlineto{\pgfqpoint{2.704657in}{2.640054in}}%
\pgfpathlineto{\pgfqpoint{2.691037in}{2.655185in}}%
\pgfpathlineto{\pgfqpoint{2.699611in}{2.655519in}}%
\pgfpathlineto{\pgfqpoint{2.708170in}{2.656044in}}%
\pgfpathlineto{\pgfqpoint{2.716716in}{2.656757in}}%
\pgfpathlineto{\pgfqpoint{2.725249in}{2.657653in}}%
\pgfpathclose%
\pgfusepath{fill}%
\end{pgfscope}%
\begin{pgfscope}%
\pgfpathrectangle{\pgfqpoint{1.150000in}{0.150000in}}{\pgfqpoint{5.700000in}{5.700000in}}%
\pgfusepath{clip}%
\pgfsetbuttcap%
\pgfsetroundjoin%
\definecolor{currentfill}{rgb}{0.220057,0.343307,0.549413}%
\pgfsetfillcolor{currentfill}%
\pgfsetfillopacity{0.700000}%
\pgfsetlinewidth{0.000000pt}%
\definecolor{currentstroke}{rgb}{0.000000,0.000000,0.000000}%
\pgfsetstrokecolor{currentstroke}%
\pgfsetdash{}{0pt}%
\pgfpathmoveto{\pgfqpoint{2.561862in}{2.849352in}}%
\pgfpathlineto{\pgfqpoint{2.575508in}{2.832325in}}%
\pgfpathlineto{\pgfqpoint{2.589148in}{2.815496in}}%
\pgfpathlineto{\pgfqpoint{2.602782in}{2.798861in}}%
\pgfpathlineto{\pgfqpoint{2.616410in}{2.782421in}}%
\pgfpathlineto{\pgfqpoint{2.607805in}{2.782223in}}%
\pgfpathlineto{\pgfqpoint{2.599186in}{2.782220in}}%
\pgfpathlineto{\pgfqpoint{2.590553in}{2.782415in}}%
\pgfpathlineto{\pgfqpoint{2.581906in}{2.782810in}}%
\pgfpathlineto{\pgfqpoint{2.568240in}{2.799610in}}%
\pgfpathlineto{\pgfqpoint{2.554568in}{2.816605in}}%
\pgfpathlineto{\pgfqpoint{2.540889in}{2.833796in}}%
\pgfpathlineto{\pgfqpoint{2.527204in}{2.851185in}}%
\pgfpathlineto{\pgfqpoint{2.535891in}{2.850421in}}%
\pgfpathlineto{\pgfqpoint{2.544563in}{2.849863in}}%
\pgfpathlineto{\pgfqpoint{2.553220in}{2.849508in}}%
\pgfpathlineto{\pgfqpoint{2.561862in}{2.849352in}}%
\pgfpathclose%
\pgfusepath{fill}%
\end{pgfscope}%
\begin{pgfscope}%
\pgfpathrectangle{\pgfqpoint{1.150000in}{0.150000in}}{\pgfqpoint{5.700000in}{5.700000in}}%
\pgfusepath{clip}%
\pgfsetbuttcap%
\pgfsetroundjoin%
\definecolor{currentfill}{rgb}{0.218130,0.347432,0.550038}%
\pgfsetfillcolor{currentfill}%
\pgfsetfillopacity{0.700000}%
\pgfsetlinewidth{0.000000pt}%
\definecolor{currentstroke}{rgb}{0.000000,0.000000,0.000000}%
\pgfsetstrokecolor{currentstroke}%
\pgfsetdash{}{0pt}%
\pgfpathmoveto{\pgfqpoint{5.321269in}{2.830870in}}%
\pgfpathlineto{\pgfqpoint{5.335364in}{2.835977in}}%
\pgfpathlineto{\pgfqpoint{5.349473in}{2.841184in}}%
\pgfpathlineto{\pgfqpoint{5.363595in}{2.846491in}}%
\pgfpathlineto{\pgfqpoint{5.377731in}{2.851897in}}%
\pgfpathlineto{\pgfqpoint{5.370348in}{2.844488in}}%
\pgfpathlineto{\pgfqpoint{5.362958in}{2.837001in}}%
\pgfpathlineto{\pgfqpoint{5.355561in}{2.829435in}}%
\pgfpathlineto{\pgfqpoint{5.348156in}{2.821789in}}%
\pgfpathlineto{\pgfqpoint{5.334009in}{2.816316in}}%
\pgfpathlineto{\pgfqpoint{5.319876in}{2.810942in}}%
\pgfpathlineto{\pgfqpoint{5.305756in}{2.805669in}}%
\pgfpathlineto{\pgfqpoint{5.291650in}{2.800494in}}%
\pgfpathlineto{\pgfqpoint{5.299065in}{2.808200in}}%
\pgfpathlineto{\pgfqpoint{5.306473in}{2.815831in}}%
\pgfpathlineto{\pgfqpoint{5.313874in}{2.823387in}}%
\pgfpathlineto{\pgfqpoint{5.321269in}{2.830870in}}%
\pgfpathclose%
\pgfusepath{fill}%
\end{pgfscope}%
\begin{pgfscope}%
\pgfpathrectangle{\pgfqpoint{1.150000in}{0.150000in}}{\pgfqpoint{5.700000in}{5.700000in}}%
\pgfusepath{clip}%
\pgfsetbuttcap%
\pgfsetroundjoin%
\definecolor{currentfill}{rgb}{0.263663,0.237631,0.518762}%
\pgfsetfillcolor{currentfill}%
\pgfsetfillopacity{0.700000}%
\pgfsetlinewidth{0.000000pt}%
\definecolor{currentstroke}{rgb}{0.000000,0.000000,0.000000}%
\pgfsetstrokecolor{currentstroke}%
\pgfsetdash{}{0pt}%
\pgfpathmoveto{\pgfqpoint{2.779564in}{2.599601in}}%
\pgfpathlineto{\pgfqpoint{2.793134in}{2.585523in}}%
\pgfpathlineto{\pgfqpoint{2.806700in}{2.571615in}}%
\pgfpathlineto{\pgfqpoint{2.820264in}{2.557876in}}%
\pgfpathlineto{\pgfqpoint{2.833824in}{2.544305in}}%
\pgfpathlineto{\pgfqpoint{2.825358in}{2.542718in}}%
\pgfpathlineto{\pgfqpoint{2.816880in}{2.541305in}}%
\pgfpathlineto{\pgfqpoint{2.808389in}{2.540068in}}%
\pgfpathlineto{\pgfqpoint{2.799885in}{2.539013in}}%
\pgfpathlineto{\pgfqpoint{2.786292in}{2.552935in}}%
\pgfpathlineto{\pgfqpoint{2.772695in}{2.567025in}}%
\pgfpathlineto{\pgfqpoint{2.759095in}{2.581286in}}%
\pgfpathlineto{\pgfqpoint{2.745491in}{2.595717in}}%
\pgfpathlineto{\pgfqpoint{2.754029in}{2.596413in}}%
\pgfpathlineto{\pgfqpoint{2.762554in}{2.597295in}}%
\pgfpathlineto{\pgfqpoint{2.771065in}{2.598359in}}%
\pgfpathlineto{\pgfqpoint{2.779564in}{2.599601in}}%
\pgfpathclose%
\pgfusepath{fill}%
\end{pgfscope}%
\begin{pgfscope}%
\pgfpathrectangle{\pgfqpoint{1.150000in}{0.150000in}}{\pgfqpoint{5.700000in}{5.700000in}}%
\pgfusepath{clip}%
\pgfsetbuttcap%
\pgfsetroundjoin%
\definecolor{currentfill}{rgb}{0.269944,0.014625,0.341379}%
\pgfsetfillcolor{currentfill}%
\pgfsetfillopacity{0.700000}%
\pgfsetlinewidth{0.000000pt}%
\definecolor{currentstroke}{rgb}{0.000000,0.000000,0.000000}%
\pgfsetstrokecolor{currentstroke}%
\pgfsetdash{}{0pt}%
\pgfpathmoveto{\pgfqpoint{3.527151in}{2.167512in}}%
\pgfpathlineto{\pgfqpoint{3.540663in}{2.161266in}}%
\pgfpathlineto{\pgfqpoint{3.554178in}{2.155142in}}%
\pgfpathlineto{\pgfqpoint{3.567696in}{2.149142in}}%
\pgfpathlineto{\pgfqpoint{3.581219in}{2.143263in}}%
\pgfpathlineto{\pgfqpoint{3.573160in}{2.136479in}}%
\pgfpathlineto{\pgfqpoint{3.565094in}{2.129777in}}%
\pgfpathlineto{\pgfqpoint{3.557020in}{2.123159in}}%
\pgfpathlineto{\pgfqpoint{3.548940in}{2.116628in}}%
\pgfpathlineto{\pgfqpoint{3.535399in}{2.122789in}}%
\pgfpathlineto{\pgfqpoint{3.521862in}{2.129073in}}%
\pgfpathlineto{\pgfqpoint{3.508328in}{2.135479in}}%
\pgfpathlineto{\pgfqpoint{3.494798in}{2.142008in}}%
\pgfpathlineto{\pgfqpoint{3.502898in}{2.148249in}}%
\pgfpathlineto{\pgfqpoint{3.510990in}{2.154582in}}%
\pgfpathlineto{\pgfqpoint{3.519074in}{2.161004in}}%
\pgfpathlineto{\pgfqpoint{3.527151in}{2.167512in}}%
\pgfpathclose%
\pgfusepath{fill}%
\end{pgfscope}%
\begin{pgfscope}%
\pgfpathrectangle{\pgfqpoint{1.150000in}{0.150000in}}{\pgfqpoint{5.700000in}{5.700000in}}%
\pgfusepath{clip}%
\pgfsetbuttcap%
\pgfsetroundjoin%
\definecolor{currentfill}{rgb}{0.268510,0.009605,0.335427}%
\pgfsetfillcolor{currentfill}%
\pgfsetfillopacity{0.700000}%
\pgfsetlinewidth{0.000000pt}%
\definecolor{currentstroke}{rgb}{0.000000,0.000000,0.000000}%
\pgfsetstrokecolor{currentstroke}%
\pgfsetdash{}{0pt}%
\pgfpathmoveto{\pgfqpoint{3.667450in}{2.149911in}}%
\pgfpathlineto{\pgfqpoint{3.680977in}{2.144896in}}%
\pgfpathlineto{\pgfqpoint{3.694509in}{2.140001in}}%
\pgfpathlineto{\pgfqpoint{3.708046in}{2.135223in}}%
\pgfpathlineto{\pgfqpoint{3.721587in}{2.130563in}}%
\pgfpathlineto{\pgfqpoint{3.713588in}{2.122964in}}%
\pgfpathlineto{\pgfqpoint{3.705582in}{2.115427in}}%
\pgfpathlineto{\pgfqpoint{3.697570in}{2.107956in}}%
\pgfpathlineto{\pgfqpoint{3.689551in}{2.100551in}}%
\pgfpathlineto{\pgfqpoint{3.675994in}{2.105475in}}%
\pgfpathlineto{\pgfqpoint{3.662442in}{2.110517in}}%
\pgfpathlineto{\pgfqpoint{3.648894in}{2.115677in}}%
\pgfpathlineto{\pgfqpoint{3.635350in}{2.120955in}}%
\pgfpathlineto{\pgfqpoint{3.643385in}{2.128088in}}%
\pgfpathlineto{\pgfqpoint{3.651414in}{2.135294in}}%
\pgfpathlineto{\pgfqpoint{3.659435in}{2.142569in}}%
\pgfpathlineto{\pgfqpoint{3.667450in}{2.149911in}}%
\pgfpathclose%
\pgfusepath{fill}%
\end{pgfscope}%
\begin{pgfscope}%
\pgfpathrectangle{\pgfqpoint{1.150000in}{0.150000in}}{\pgfqpoint{5.700000in}{5.700000in}}%
\pgfusepath{clip}%
\pgfsetbuttcap%
\pgfsetroundjoin%
\definecolor{currentfill}{rgb}{0.276022,0.044167,0.370164}%
\pgfsetfillcolor{currentfill}%
\pgfsetfillopacity{0.700000}%
\pgfsetlinewidth{0.000000pt}%
\definecolor{currentstroke}{rgb}{0.000000,0.000000,0.000000}%
\pgfsetstrokecolor{currentstroke}%
\pgfsetdash{}{0pt}%
\pgfpathmoveto{\pgfqpoint{4.119749in}{2.207051in}}%
\pgfpathlineto{\pgfqpoint{4.133372in}{2.205628in}}%
\pgfpathlineto{\pgfqpoint{4.147003in}{2.204313in}}%
\pgfpathlineto{\pgfqpoint{4.160641in}{2.203107in}}%
\pgfpathlineto{\pgfqpoint{4.174287in}{2.202008in}}%
\pgfpathlineto{\pgfqpoint{4.166452in}{2.192595in}}%
\pgfpathlineto{\pgfqpoint{4.158612in}{2.183184in}}%
\pgfpathlineto{\pgfqpoint{4.150767in}{2.173778in}}%
\pgfpathlineto{\pgfqpoint{4.142916in}{2.164378in}}%
\pgfpathlineto{\pgfqpoint{4.129261in}{2.165668in}}%
\pgfpathlineto{\pgfqpoint{4.115613in}{2.167066in}}%
\pgfpathlineto{\pgfqpoint{4.101972in}{2.168572in}}%
\pgfpathlineto{\pgfqpoint{4.088339in}{2.170187in}}%
\pgfpathlineto{\pgfqpoint{4.096199in}{2.179388in}}%
\pgfpathlineto{\pgfqpoint{4.104055in}{2.188601in}}%
\pgfpathlineto{\pgfqpoint{4.111904in}{2.197822in}}%
\pgfpathlineto{\pgfqpoint{4.119749in}{2.207051in}}%
\pgfpathclose%
\pgfusepath{fill}%
\end{pgfscope}%
\begin{pgfscope}%
\pgfpathrectangle{\pgfqpoint{1.150000in}{0.150000in}}{\pgfqpoint{5.700000in}{5.700000in}}%
\pgfusepath{clip}%
\pgfsetbuttcap%
\pgfsetroundjoin%
\definecolor{currentfill}{rgb}{0.208623,0.367752,0.552675}%
\pgfsetfillcolor{currentfill}%
\pgfsetfillopacity{0.700000}%
\pgfsetlinewidth{0.000000pt}%
\definecolor{currentstroke}{rgb}{0.000000,0.000000,0.000000}%
\pgfsetstrokecolor{currentstroke}%
\pgfsetdash{}{0pt}%
\pgfpathmoveto{\pgfqpoint{5.407190in}{2.880771in}}%
\pgfpathlineto{\pgfqpoint{5.421328in}{2.886191in}}%
\pgfpathlineto{\pgfqpoint{5.435479in}{2.891709in}}%
\pgfpathlineto{\pgfqpoint{5.449644in}{2.897327in}}%
\pgfpathlineto{\pgfqpoint{5.463823in}{2.903044in}}%
\pgfpathlineto{\pgfqpoint{5.456482in}{2.896031in}}%
\pgfpathlineto{\pgfqpoint{5.449132in}{2.888938in}}%
\pgfpathlineto{\pgfqpoint{5.441776in}{2.881767in}}%
\pgfpathlineto{\pgfqpoint{5.434412in}{2.874514in}}%
\pgfpathlineto{\pgfqpoint{5.420221in}{2.868711in}}%
\pgfpathlineto{\pgfqpoint{5.406044in}{2.863007in}}%
\pgfpathlineto{\pgfqpoint{5.391881in}{2.857402in}}%
\pgfpathlineto{\pgfqpoint{5.377731in}{2.851897in}}%
\pgfpathlineto{\pgfqpoint{5.385107in}{2.859228in}}%
\pgfpathlineto{\pgfqpoint{5.392475in}{2.866484in}}%
\pgfpathlineto{\pgfqpoint{5.399836in}{2.873664in}}%
\pgfpathlineto{\pgfqpoint{5.407190in}{2.880771in}}%
\pgfpathclose%
\pgfusepath{fill}%
\end{pgfscope}%
\begin{pgfscope}%
\pgfpathrectangle{\pgfqpoint{1.150000in}{0.150000in}}{\pgfqpoint{5.700000in}{5.700000in}}%
\pgfusepath{clip}%
\pgfsetbuttcap%
\pgfsetroundjoin%
\definecolor{currentfill}{rgb}{0.270595,0.214069,0.507052}%
\pgfsetfillcolor{currentfill}%
\pgfsetfillopacity{0.700000}%
\pgfsetlinewidth{0.000000pt}%
\definecolor{currentstroke}{rgb}{0.000000,0.000000,0.000000}%
\pgfsetstrokecolor{currentstroke}%
\pgfsetdash{}{0pt}%
\pgfpathmoveto{\pgfqpoint{2.833824in}{2.544305in}}%
\pgfpathlineto{\pgfqpoint{2.847381in}{2.530901in}}%
\pgfpathlineto{\pgfqpoint{2.860936in}{2.517662in}}%
\pgfpathlineto{\pgfqpoint{2.874489in}{2.504587in}}%
\pgfpathlineto{\pgfqpoint{2.888039in}{2.491674in}}%
\pgfpathlineto{\pgfqpoint{2.879605in}{2.489744in}}%
\pgfpathlineto{\pgfqpoint{2.871159in}{2.487983in}}%
\pgfpathlineto{\pgfqpoint{2.862701in}{2.486394in}}%
\pgfpathlineto{\pgfqpoint{2.854230in}{2.484981in}}%
\pgfpathlineto{\pgfqpoint{2.840648in}{2.498243in}}%
\pgfpathlineto{\pgfqpoint{2.827063in}{2.511668in}}%
\pgfpathlineto{\pgfqpoint{2.813476in}{2.525258in}}%
\pgfpathlineto{\pgfqpoint{2.799885in}{2.539013in}}%
\pgfpathlineto{\pgfqpoint{2.808389in}{2.540068in}}%
\pgfpathlineto{\pgfqpoint{2.816880in}{2.541305in}}%
\pgfpathlineto{\pgfqpoint{2.825358in}{2.542718in}}%
\pgfpathlineto{\pgfqpoint{2.833824in}{2.544305in}}%
\pgfpathclose%
\pgfusepath{fill}%
\end{pgfscope}%
\begin{pgfscope}%
\pgfpathrectangle{\pgfqpoint{1.150000in}{0.150000in}}{\pgfqpoint{5.700000in}{5.700000in}}%
\pgfusepath{clip}%
\pgfsetbuttcap%
\pgfsetroundjoin%
\definecolor{currentfill}{rgb}{0.197636,0.391528,0.554969}%
\pgfsetfillcolor{currentfill}%
\pgfsetfillopacity{0.700000}%
\pgfsetlinewidth{0.000000pt}%
\definecolor{currentstroke}{rgb}{0.000000,0.000000,0.000000}%
\pgfsetstrokecolor{currentstroke}%
\pgfsetdash{}{0pt}%
\pgfpathmoveto{\pgfqpoint{5.493116in}{2.930343in}}%
\pgfpathlineto{\pgfqpoint{5.507296in}{2.936054in}}%
\pgfpathlineto{\pgfqpoint{5.521490in}{2.941864in}}%
\pgfpathlineto{\pgfqpoint{5.535698in}{2.947773in}}%
\pgfpathlineto{\pgfqpoint{5.549920in}{2.953781in}}%
\pgfpathlineto{\pgfqpoint{5.542622in}{2.947180in}}%
\pgfpathlineto{\pgfqpoint{5.535315in}{2.940501in}}%
\pgfpathlineto{\pgfqpoint{5.528001in}{2.933743in}}%
\pgfpathlineto{\pgfqpoint{5.520679in}{2.926905in}}%
\pgfpathlineto{\pgfqpoint{5.506444in}{2.920791in}}%
\pgfpathlineto{\pgfqpoint{5.492223in}{2.914776in}}%
\pgfpathlineto{\pgfqpoint{5.478016in}{2.908861in}}%
\pgfpathlineto{\pgfqpoint{5.463823in}{2.903044in}}%
\pgfpathlineto{\pgfqpoint{5.471158in}{2.909981in}}%
\pgfpathlineto{\pgfqpoint{5.478484in}{2.916842in}}%
\pgfpathlineto{\pgfqpoint{5.485804in}{2.923629in}}%
\pgfpathlineto{\pgfqpoint{5.493116in}{2.930343in}}%
\pgfpathclose%
\pgfusepath{fill}%
\end{pgfscope}%
\begin{pgfscope}%
\pgfpathrectangle{\pgfqpoint{1.150000in}{0.150000in}}{\pgfqpoint{5.700000in}{5.700000in}}%
\pgfusepath{clip}%
\pgfsetbuttcap%
\pgfsetroundjoin%
\definecolor{currentfill}{rgb}{0.188923,0.410910,0.556326}%
\pgfsetfillcolor{currentfill}%
\pgfsetfillopacity{0.700000}%
\pgfsetlinewidth{0.000000pt}%
\definecolor{currentstroke}{rgb}{0.000000,0.000000,0.000000}%
\pgfsetstrokecolor{currentstroke}%
\pgfsetdash{}{0pt}%
\pgfpathmoveto{\pgfqpoint{5.579039in}{2.979441in}}%
\pgfpathlineto{\pgfqpoint{5.593262in}{2.985424in}}%
\pgfpathlineto{\pgfqpoint{5.607499in}{2.991505in}}%
\pgfpathlineto{\pgfqpoint{5.621750in}{2.997685in}}%
\pgfpathlineto{\pgfqpoint{5.636016in}{3.003964in}}%
\pgfpathlineto{\pgfqpoint{5.628762in}{2.997790in}}%
\pgfpathlineto{\pgfqpoint{5.621500in}{2.991539in}}%
\pgfpathlineto{\pgfqpoint{5.614230in}{2.985211in}}%
\pgfpathlineto{\pgfqpoint{5.606953in}{2.978804in}}%
\pgfpathlineto{\pgfqpoint{5.592673in}{2.972400in}}%
\pgfpathlineto{\pgfqpoint{5.578408in}{2.966095in}}%
\pgfpathlineto{\pgfqpoint{5.564157in}{2.959889in}}%
\pgfpathlineto{\pgfqpoint{5.549920in}{2.953781in}}%
\pgfpathlineto{\pgfqpoint{5.557211in}{2.960307in}}%
\pgfpathlineto{\pgfqpoint{5.564495in}{2.966757in}}%
\pgfpathlineto{\pgfqpoint{5.571771in}{2.973135in}}%
\pgfpathlineto{\pgfqpoint{5.579039in}{2.979441in}}%
\pgfpathclose%
\pgfusepath{fill}%
\end{pgfscope}%
\begin{pgfscope}%
\pgfpathrectangle{\pgfqpoint{1.150000in}{0.150000in}}{\pgfqpoint{5.700000in}{5.700000in}}%
\pgfusepath{clip}%
\pgfsetbuttcap%
\pgfsetroundjoin%
\definecolor{currentfill}{rgb}{0.268510,0.009605,0.335427}%
\pgfsetfillcolor{currentfill}%
\pgfsetfillopacity{0.700000}%
\pgfsetlinewidth{0.000000pt}%
\definecolor{currentstroke}{rgb}{0.000000,0.000000,0.000000}%
\pgfsetstrokecolor{currentstroke}%
\pgfsetdash{}{0pt}%
\pgfpathmoveto{\pgfqpoint{3.807680in}{2.145041in}}%
\pgfpathlineto{\pgfqpoint{3.821233in}{2.141206in}}%
\pgfpathlineto{\pgfqpoint{3.834792in}{2.137485in}}%
\pgfpathlineto{\pgfqpoint{3.848356in}{2.133879in}}%
\pgfpathlineto{\pgfqpoint{3.861927in}{2.130386in}}%
\pgfpathlineto{\pgfqpoint{3.853980in}{2.122083in}}%
\pgfpathlineto{\pgfqpoint{3.846028in}{2.113825in}}%
\pgfpathlineto{\pgfqpoint{3.838070in}{2.105612in}}%
\pgfpathlineto{\pgfqpoint{3.830106in}{2.097449in}}%
\pgfpathlineto{\pgfqpoint{3.816522in}{2.101187in}}%
\pgfpathlineto{\pgfqpoint{3.802944in}{2.105039in}}%
\pgfpathlineto{\pgfqpoint{3.789371in}{2.109005in}}%
\pgfpathlineto{\pgfqpoint{3.775804in}{2.113086in}}%
\pgfpathlineto{\pgfqpoint{3.783782in}{2.120997in}}%
\pgfpathlineto{\pgfqpoint{3.791754in}{2.128962in}}%
\pgfpathlineto{\pgfqpoint{3.799720in}{2.136977in}}%
\pgfpathlineto{\pgfqpoint{3.807680in}{2.145041in}}%
\pgfpathclose%
\pgfusepath{fill}%
\end{pgfscope}%
\begin{pgfscope}%
\pgfpathrectangle{\pgfqpoint{1.150000in}{0.150000in}}{\pgfqpoint{5.700000in}{5.700000in}}%
\pgfusepath{clip}%
\pgfsetbuttcap%
\pgfsetroundjoin%
\definecolor{currentfill}{rgb}{0.274952,0.037752,0.364543}%
\pgfsetfillcolor{currentfill}%
\pgfsetfillopacity{0.700000}%
\pgfsetlinewidth{0.000000pt}%
\definecolor{currentstroke}{rgb}{0.000000,0.000000,0.000000}%
\pgfsetstrokecolor{currentstroke}%
\pgfsetdash{}{0pt}%
\pgfpathmoveto{\pgfqpoint{3.386669in}{2.198753in}}%
\pgfpathlineto{\pgfqpoint{3.400175in}{2.191215in}}%
\pgfpathlineto{\pgfqpoint{3.413684in}{2.183805in}}%
\pgfpathlineto{\pgfqpoint{3.427196in}{2.176524in}}%
\pgfpathlineto{\pgfqpoint{3.440710in}{2.169369in}}%
\pgfpathlineto{\pgfqpoint{3.432584in}{2.163516in}}%
\pgfpathlineto{\pgfqpoint{3.424449in}{2.157765in}}%
\pgfpathlineto{\pgfqpoint{3.416306in}{2.152119in}}%
\pgfpathlineto{\pgfqpoint{3.408155in}{2.146580in}}%
\pgfpathlineto{\pgfqpoint{3.394620in}{2.154036in}}%
\pgfpathlineto{\pgfqpoint{3.381088in}{2.161620in}}%
\pgfpathlineto{\pgfqpoint{3.367558in}{2.169332in}}%
\pgfpathlineto{\pgfqpoint{3.354030in}{2.177172in}}%
\pgfpathlineto{\pgfqpoint{3.362202in}{2.182402in}}%
\pgfpathlineto{\pgfqpoint{3.370366in}{2.187744in}}%
\pgfpathlineto{\pgfqpoint{3.378521in}{2.193195in}}%
\pgfpathlineto{\pgfqpoint{3.386669in}{2.198753in}}%
\pgfpathclose%
\pgfusepath{fill}%
\end{pgfscope}%
\begin{pgfscope}%
\pgfpathrectangle{\pgfqpoint{1.150000in}{0.150000in}}{\pgfqpoint{5.700000in}{5.700000in}}%
\pgfusepath{clip}%
\pgfsetbuttcap%
\pgfsetroundjoin%
\definecolor{currentfill}{rgb}{0.180629,0.429975,0.557282}%
\pgfsetfillcolor{currentfill}%
\pgfsetfillopacity{0.700000}%
\pgfsetlinewidth{0.000000pt}%
\definecolor{currentstroke}{rgb}{0.000000,0.000000,0.000000}%
\pgfsetstrokecolor{currentstroke}%
\pgfsetdash{}{0pt}%
\pgfpathmoveto{\pgfqpoint{5.664954in}{3.027934in}}%
\pgfpathlineto{\pgfqpoint{5.679220in}{3.034168in}}%
\pgfpathlineto{\pgfqpoint{5.693500in}{3.040501in}}%
\pgfpathlineto{\pgfqpoint{5.707794in}{3.046932in}}%
\pgfpathlineto{\pgfqpoint{5.722103in}{3.053461in}}%
\pgfpathlineto{\pgfqpoint{5.714896in}{3.047725in}}%
\pgfpathlineto{\pgfqpoint{5.707680in}{3.041916in}}%
\pgfpathlineto{\pgfqpoint{5.700457in}{3.036031in}}%
\pgfpathlineto{\pgfqpoint{5.693226in}{3.030068in}}%
\pgfpathlineto{\pgfqpoint{5.678901in}{3.023394in}}%
\pgfpathlineto{\pgfqpoint{5.664591in}{3.016818in}}%
\pgfpathlineto{\pgfqpoint{5.650296in}{3.010342in}}%
\pgfpathlineto{\pgfqpoint{5.636016in}{3.003964in}}%
\pgfpathlineto{\pgfqpoint{5.643262in}{3.010064in}}%
\pgfpathlineto{\pgfqpoint{5.650501in}{3.016091in}}%
\pgfpathlineto{\pgfqpoint{5.657731in}{3.022047in}}%
\pgfpathlineto{\pgfqpoint{5.664954in}{3.027934in}}%
\pgfpathclose%
\pgfusepath{fill}%
\end{pgfscope}%
\begin{pgfscope}%
\pgfpathrectangle{\pgfqpoint{1.150000in}{0.150000in}}{\pgfqpoint{5.700000in}{5.700000in}}%
\pgfusepath{clip}%
\pgfsetbuttcap%
\pgfsetroundjoin%
\definecolor{currentfill}{rgb}{0.281446,0.084320,0.407414}%
\pgfsetfillcolor{currentfill}%
\pgfsetfillopacity{0.700000}%
\pgfsetlinewidth{0.000000pt}%
\definecolor{currentstroke}{rgb}{0.000000,0.000000,0.000000}%
\pgfsetstrokecolor{currentstroke}%
\pgfsetdash{}{0pt}%
\pgfpathmoveto{\pgfqpoint{3.191837in}{2.281571in}}%
\pgfpathlineto{\pgfqpoint{3.205345in}{2.272126in}}%
\pgfpathlineto{\pgfqpoint{3.218855in}{2.262820in}}%
\pgfpathlineto{\pgfqpoint{3.232366in}{2.253650in}}%
\pgfpathlineto{\pgfqpoint{3.245878in}{2.244618in}}%
\pgfpathlineto{\pgfqpoint{3.237651in}{2.240132in}}%
\pgfpathlineto{\pgfqpoint{3.229415in}{2.235774in}}%
\pgfpathlineto{\pgfqpoint{3.221169in}{2.231546in}}%
\pgfpathlineto{\pgfqpoint{3.212915in}{2.227452in}}%
\pgfpathlineto{\pgfqpoint{3.199378in}{2.236808in}}%
\pgfpathlineto{\pgfqpoint{3.185843in}{2.246300in}}%
\pgfpathlineto{\pgfqpoint{3.172308in}{2.255930in}}%
\pgfpathlineto{\pgfqpoint{3.158774in}{2.265698in}}%
\pgfpathlineto{\pgfqpoint{3.167054in}{2.269462in}}%
\pgfpathlineto{\pgfqpoint{3.175325in}{2.273364in}}%
\pgfpathlineto{\pgfqpoint{3.183585in}{2.277402in}}%
\pgfpathlineto{\pgfqpoint{3.191837in}{2.281571in}}%
\pgfpathclose%
\pgfusepath{fill}%
\end{pgfscope}%
\begin{pgfscope}%
\pgfpathrectangle{\pgfqpoint{1.150000in}{0.150000in}}{\pgfqpoint{5.700000in}{5.700000in}}%
\pgfusepath{clip}%
\pgfsetbuttcap%
\pgfsetroundjoin%
\definecolor{currentfill}{rgb}{0.272594,0.025563,0.353093}%
\pgfsetfillcolor{currentfill}%
\pgfsetfillopacity{0.700000}%
\pgfsetlinewidth{0.000000pt}%
\definecolor{currentstroke}{rgb}{0.000000,0.000000,0.000000}%
\pgfsetstrokecolor{currentstroke}%
\pgfsetdash{}{0pt}%
\pgfpathmoveto{\pgfqpoint{4.033878in}{2.177736in}}%
\pgfpathlineto{\pgfqpoint{4.047483in}{2.175684in}}%
\pgfpathlineto{\pgfqpoint{4.061094in}{2.173743in}}%
\pgfpathlineto{\pgfqpoint{4.074713in}{2.171910in}}%
\pgfpathlineto{\pgfqpoint{4.088339in}{2.170187in}}%
\pgfpathlineto{\pgfqpoint{4.080473in}{2.160998in}}%
\pgfpathlineto{\pgfqpoint{4.072601in}{2.151824in}}%
\pgfpathlineto{\pgfqpoint{4.064724in}{2.142666in}}%
\pgfpathlineto{\pgfqpoint{4.056842in}{2.133526in}}%
\pgfpathlineto{\pgfqpoint{4.043206in}{2.135459in}}%
\pgfpathlineto{\pgfqpoint{4.029576in}{2.137501in}}%
\pgfpathlineto{\pgfqpoint{4.015954in}{2.139652in}}%
\pgfpathlineto{\pgfqpoint{4.002339in}{2.141913in}}%
\pgfpathlineto{\pgfqpoint{4.010232in}{2.150836in}}%
\pgfpathlineto{\pgfqpoint{4.018120in}{2.159783in}}%
\pgfpathlineto{\pgfqpoint{4.026002in}{2.168750in}}%
\pgfpathlineto{\pgfqpoint{4.033878in}{2.177736in}}%
\pgfpathclose%
\pgfusepath{fill}%
\end{pgfscope}%
\begin{pgfscope}%
\pgfpathrectangle{\pgfqpoint{1.150000in}{0.150000in}}{\pgfqpoint{5.700000in}{5.700000in}}%
\pgfusepath{clip}%
\pgfsetbuttcap%
\pgfsetroundjoin%
\definecolor{currentfill}{rgb}{0.276194,0.190074,0.493001}%
\pgfsetfillcolor{currentfill}%
\pgfsetfillopacity{0.700000}%
\pgfsetlinewidth{0.000000pt}%
\definecolor{currentstroke}{rgb}{0.000000,0.000000,0.000000}%
\pgfsetstrokecolor{currentstroke}%
\pgfsetdash{}{0pt}%
\pgfpathmoveto{\pgfqpoint{2.888039in}{2.491674in}}%
\pgfpathlineto{\pgfqpoint{2.901587in}{2.478922in}}%
\pgfpathlineto{\pgfqpoint{2.915132in}{2.466331in}}%
\pgfpathlineto{\pgfqpoint{2.928676in}{2.453898in}}%
\pgfpathlineto{\pgfqpoint{2.942219in}{2.441622in}}%
\pgfpathlineto{\pgfqpoint{2.933816in}{2.439351in}}%
\pgfpathlineto{\pgfqpoint{2.925401in}{2.437244in}}%
\pgfpathlineto{\pgfqpoint{2.916974in}{2.435304in}}%
\pgfpathlineto{\pgfqpoint{2.908536in}{2.433535in}}%
\pgfpathlineto{\pgfqpoint{2.894962in}{2.446158in}}%
\pgfpathlineto{\pgfqpoint{2.881387in}{2.458940in}}%
\pgfpathlineto{\pgfqpoint{2.867810in}{2.471880in}}%
\pgfpathlineto{\pgfqpoint{2.854230in}{2.484981in}}%
\pgfpathlineto{\pgfqpoint{2.862701in}{2.486394in}}%
\pgfpathlineto{\pgfqpoint{2.871159in}{2.487983in}}%
\pgfpathlineto{\pgfqpoint{2.879605in}{2.489744in}}%
\pgfpathlineto{\pgfqpoint{2.888039in}{2.491674in}}%
\pgfpathclose%
\pgfusepath{fill}%
\end{pgfscope}%
\begin{pgfscope}%
\pgfpathrectangle{\pgfqpoint{1.150000in}{0.150000in}}{\pgfqpoint{5.700000in}{5.700000in}}%
\pgfusepath{clip}%
\pgfsetbuttcap%
\pgfsetroundjoin%
\definecolor{currentfill}{rgb}{0.172719,0.448791,0.557885}%
\pgfsetfillcolor{currentfill}%
\pgfsetfillopacity{0.700000}%
\pgfsetlinewidth{0.000000pt}%
\definecolor{currentstroke}{rgb}{0.000000,0.000000,0.000000}%
\pgfsetstrokecolor{currentstroke}%
\pgfsetdash{}{0pt}%
\pgfpathmoveto{\pgfqpoint{5.750855in}{3.075704in}}%
\pgfpathlineto{\pgfqpoint{5.765163in}{3.082169in}}%
\pgfpathlineto{\pgfqpoint{5.779485in}{3.088732in}}%
\pgfpathlineto{\pgfqpoint{5.793823in}{3.095393in}}%
\pgfpathlineto{\pgfqpoint{5.808176in}{3.102153in}}%
\pgfpathlineto{\pgfqpoint{5.801016in}{3.096865in}}%
\pgfpathlineto{\pgfqpoint{5.793849in}{3.091505in}}%
\pgfpathlineto{\pgfqpoint{5.786673in}{3.086073in}}%
\pgfpathlineto{\pgfqpoint{5.779490in}{3.080566in}}%
\pgfpathlineto{\pgfqpoint{5.765121in}{3.073642in}}%
\pgfpathlineto{\pgfqpoint{5.750767in}{3.066816in}}%
\pgfpathlineto{\pgfqpoint{5.736428in}{3.060089in}}%
\pgfpathlineto{\pgfqpoint{5.722103in}{3.053461in}}%
\pgfpathlineto{\pgfqpoint{5.729303in}{3.059125in}}%
\pgfpathlineto{\pgfqpoint{5.736495in}{3.064719in}}%
\pgfpathlineto{\pgfqpoint{5.743679in}{3.070244in}}%
\pgfpathlineto{\pgfqpoint{5.750855in}{3.075704in}}%
\pgfpathclose%
\pgfusepath{fill}%
\end{pgfscope}%
\begin{pgfscope}%
\pgfpathrectangle{\pgfqpoint{1.150000in}{0.150000in}}{\pgfqpoint{5.700000in}{5.700000in}}%
\pgfusepath{clip}%
\pgfsetbuttcap%
\pgfsetroundjoin%
\definecolor{currentfill}{rgb}{0.165117,0.467423,0.558141}%
\pgfsetfillcolor{currentfill}%
\pgfsetfillopacity{0.700000}%
\pgfsetlinewidth{0.000000pt}%
\definecolor{currentstroke}{rgb}{0.000000,0.000000,0.000000}%
\pgfsetstrokecolor{currentstroke}%
\pgfsetdash{}{0pt}%
\pgfpathmoveto{\pgfqpoint{5.836734in}{3.122643in}}%
\pgfpathlineto{\pgfqpoint{5.851084in}{3.129318in}}%
\pgfpathlineto{\pgfqpoint{5.865449in}{3.136092in}}%
\pgfpathlineto{\pgfqpoint{5.879829in}{3.142964in}}%
\pgfpathlineto{\pgfqpoint{5.894225in}{3.149934in}}%
\pgfpathlineto{\pgfqpoint{5.887116in}{3.145098in}}%
\pgfpathlineto{\pgfqpoint{5.879998in}{3.140194in}}%
\pgfpathlineto{\pgfqpoint{5.872872in}{3.135222in}}%
\pgfpathlineto{\pgfqpoint{5.865738in}{3.130178in}}%
\pgfpathlineto{\pgfqpoint{5.851325in}{3.123024in}}%
\pgfpathlineto{\pgfqpoint{5.836926in}{3.115969in}}%
\pgfpathlineto{\pgfqpoint{5.822543in}{3.109012in}}%
\pgfpathlineto{\pgfqpoint{5.808176in}{3.102153in}}%
\pgfpathlineto{\pgfqpoint{5.815327in}{3.107373in}}%
\pgfpathlineto{\pgfqpoint{5.822470in}{3.112527in}}%
\pgfpathlineto{\pgfqpoint{5.829606in}{3.117616in}}%
\pgfpathlineto{\pgfqpoint{5.836734in}{3.122643in}}%
\pgfpathclose%
\pgfusepath{fill}%
\end{pgfscope}%
\begin{pgfscope}%
\pgfpathrectangle{\pgfqpoint{1.150000in}{0.150000in}}{\pgfqpoint{5.700000in}{5.700000in}}%
\pgfusepath{clip}%
\pgfsetbuttcap%
\pgfsetroundjoin%
\definecolor{currentfill}{rgb}{0.159194,0.482237,0.558073}%
\pgfsetfillcolor{currentfill}%
\pgfsetfillopacity{0.700000}%
\pgfsetlinewidth{0.000000pt}%
\definecolor{currentstroke}{rgb}{0.000000,0.000000,0.000000}%
\pgfsetstrokecolor{currentstroke}%
\pgfsetdash{}{0pt}%
\pgfpathmoveto{\pgfqpoint{5.922583in}{3.168657in}}%
\pgfpathlineto{\pgfqpoint{5.936975in}{3.175523in}}%
\pgfpathlineto{\pgfqpoint{5.951383in}{3.182487in}}%
\pgfpathlineto{\pgfqpoint{5.965806in}{3.189550in}}%
\pgfpathlineto{\pgfqpoint{5.980244in}{3.196710in}}%
\pgfpathlineto{\pgfqpoint{5.973186in}{3.192328in}}%
\pgfpathlineto{\pgfqpoint{5.966120in}{3.187883in}}%
\pgfpathlineto{\pgfqpoint{5.959046in}{3.183374in}}%
\pgfpathlineto{\pgfqpoint{5.951963in}{3.178798in}}%
\pgfpathlineto{\pgfqpoint{5.937505in}{3.171435in}}%
\pgfpathlineto{\pgfqpoint{5.923063in}{3.164170in}}%
\pgfpathlineto{\pgfqpoint{5.908636in}{3.157003in}}%
\pgfpathlineto{\pgfqpoint{5.894225in}{3.149934in}}%
\pgfpathlineto{\pgfqpoint{5.901326in}{3.154706in}}%
\pgfpathlineto{\pgfqpoint{5.908420in}{3.159416in}}%
\pgfpathlineto{\pgfqpoint{5.915506in}{3.164065in}}%
\pgfpathlineto{\pgfqpoint{5.922583in}{3.168657in}}%
\pgfpathclose%
\pgfusepath{fill}%
\end{pgfscope}%
\begin{pgfscope}%
\pgfpathrectangle{\pgfqpoint{1.150000in}{0.150000in}}{\pgfqpoint{5.700000in}{5.700000in}}%
\pgfusepath{clip}%
\pgfsetbuttcap%
\pgfsetroundjoin%
\definecolor{currentfill}{rgb}{0.151918,0.500685,0.557587}%
\pgfsetfillcolor{currentfill}%
\pgfsetfillopacity{0.700000}%
\pgfsetlinewidth{0.000000pt}%
\definecolor{currentstroke}{rgb}{0.000000,0.000000,0.000000}%
\pgfsetstrokecolor{currentstroke}%
\pgfsetdash{}{0pt}%
\pgfpathmoveto{\pgfqpoint{6.008397in}{3.213666in}}%
\pgfpathlineto{\pgfqpoint{6.022830in}{3.220702in}}%
\pgfpathlineto{\pgfqpoint{6.037279in}{3.227837in}}%
\pgfpathlineto{\pgfqpoint{6.051744in}{3.235069in}}%
\pgfpathlineto{\pgfqpoint{6.066225in}{3.242399in}}%
\pgfpathlineto{\pgfqpoint{6.059220in}{3.238470in}}%
\pgfpathlineto{\pgfqpoint{6.052207in}{3.234484in}}%
\pgfpathlineto{\pgfqpoint{6.045185in}{3.230439in}}%
\pgfpathlineto{\pgfqpoint{6.038156in}{3.226332in}}%
\pgfpathlineto{\pgfqpoint{6.023654in}{3.218779in}}%
\pgfpathlineto{\pgfqpoint{6.009168in}{3.211325in}}%
\pgfpathlineto{\pgfqpoint{5.994698in}{3.203968in}}%
\pgfpathlineto{\pgfqpoint{5.980244in}{3.196710in}}%
\pgfpathlineto{\pgfqpoint{5.987294in}{3.201032in}}%
\pgfpathlineto{\pgfqpoint{5.994336in}{3.205297in}}%
\pgfpathlineto{\pgfqpoint{6.001370in}{3.209508in}}%
\pgfpathlineto{\pgfqpoint{6.008397in}{3.213666in}}%
\pgfpathclose%
\pgfusepath{fill}%
\end{pgfscope}%
\begin{pgfscope}%
\pgfpathrectangle{\pgfqpoint{1.150000in}{0.150000in}}{\pgfqpoint{5.700000in}{5.700000in}}%
\pgfusepath{clip}%
\pgfsetbuttcap%
\pgfsetroundjoin%
\definecolor{currentfill}{rgb}{0.280255,0.165693,0.476498}%
\pgfsetfillcolor{currentfill}%
\pgfsetfillopacity{0.700000}%
\pgfsetlinewidth{0.000000pt}%
\definecolor{currentstroke}{rgb}{0.000000,0.000000,0.000000}%
\pgfsetstrokecolor{currentstroke}%
\pgfsetdash{}{0pt}%
\pgfpathmoveto{\pgfqpoint{2.942219in}{2.441622in}}%
\pgfpathlineto{\pgfqpoint{2.955759in}{2.429503in}}%
\pgfpathlineto{\pgfqpoint{2.969299in}{2.417539in}}%
\pgfpathlineto{\pgfqpoint{2.982836in}{2.405728in}}%
\pgfpathlineto{\pgfqpoint{2.996373in}{2.394071in}}%
\pgfpathlineto{\pgfqpoint{2.988000in}{2.391459in}}%
\pgfpathlineto{\pgfqpoint{2.979615in}{2.389008in}}%
\pgfpathlineto{\pgfqpoint{2.971219in}{2.386718in}}%
\pgfpathlineto{\pgfqpoint{2.962812in}{2.384595in}}%
\pgfpathlineto{\pgfqpoint{2.949245in}{2.396599in}}%
\pgfpathlineto{\pgfqpoint{2.935677in}{2.408756in}}%
\pgfpathlineto{\pgfqpoint{2.922107in}{2.421068in}}%
\pgfpathlineto{\pgfqpoint{2.908536in}{2.433535in}}%
\pgfpathlineto{\pgfqpoint{2.916974in}{2.435304in}}%
\pgfpathlineto{\pgfqpoint{2.925401in}{2.437244in}}%
\pgfpathlineto{\pgfqpoint{2.933816in}{2.439351in}}%
\pgfpathlineto{\pgfqpoint{2.942219in}{2.441622in}}%
\pgfpathclose%
\pgfusepath{fill}%
\end{pgfscope}%
\begin{pgfscope}%
\pgfpathrectangle{\pgfqpoint{1.150000in}{0.150000in}}{\pgfqpoint{5.700000in}{5.700000in}}%
\pgfusepath{clip}%
\pgfsetbuttcap%
\pgfsetroundjoin%
\definecolor{currentfill}{rgb}{0.281412,0.155834,0.469201}%
\pgfsetfillcolor{currentfill}%
\pgfsetfillopacity{0.700000}%
\pgfsetlinewidth{0.000000pt}%
\definecolor{currentstroke}{rgb}{0.000000,0.000000,0.000000}%
\pgfsetstrokecolor{currentstroke}%
\pgfsetdash{}{0pt}%
\pgfpathmoveto{\pgfqpoint{4.603750in}{2.402802in}}%
\pgfpathlineto{\pgfqpoint{4.617544in}{2.404599in}}%
\pgfpathlineto{\pgfqpoint{4.631350in}{2.406498in}}%
\pgfpathlineto{\pgfqpoint{4.645165in}{2.408500in}}%
\pgfpathlineto{\pgfqpoint{4.658991in}{2.410605in}}%
\pgfpathlineto{\pgfqpoint{4.651310in}{2.400839in}}%
\pgfpathlineto{\pgfqpoint{4.643624in}{2.391026in}}%
\pgfpathlineto{\pgfqpoint{4.635933in}{2.381166in}}%
\pgfpathlineto{\pgfqpoint{4.628236in}{2.371260in}}%
\pgfpathlineto{\pgfqpoint{4.614403in}{2.369257in}}%
\pgfpathlineto{\pgfqpoint{4.600580in}{2.367356in}}%
\pgfpathlineto{\pgfqpoint{4.586768in}{2.365558in}}%
\pgfpathlineto{\pgfqpoint{4.572966in}{2.363863in}}%
\pgfpathlineto{\pgfqpoint{4.580670in}{2.373661in}}%
\pgfpathlineto{\pgfqpoint{4.588369in}{2.383417in}}%
\pgfpathlineto{\pgfqpoint{4.596062in}{2.393131in}}%
\pgfpathlineto{\pgfqpoint{4.603750in}{2.402802in}}%
\pgfpathclose%
\pgfusepath{fill}%
\end{pgfscope}%
\begin{pgfscope}%
\pgfpathrectangle{\pgfqpoint{1.150000in}{0.150000in}}{\pgfqpoint{5.700000in}{5.700000in}}%
\pgfusepath{clip}%
\pgfsetbuttcap%
\pgfsetroundjoin%
\definecolor{currentfill}{rgb}{0.283072,0.130895,0.449241}%
\pgfsetfillcolor{currentfill}%
\pgfsetfillopacity{0.700000}%
\pgfsetlinewidth{0.000000pt}%
\definecolor{currentstroke}{rgb}{0.000000,0.000000,0.000000}%
\pgfsetstrokecolor{currentstroke}%
\pgfsetdash{}{0pt}%
\pgfpathmoveto{\pgfqpoint{4.517858in}{2.358115in}}%
\pgfpathlineto{\pgfqpoint{4.531620in}{2.359397in}}%
\pgfpathlineto{\pgfqpoint{4.545392in}{2.360782in}}%
\pgfpathlineto{\pgfqpoint{4.559174in}{2.362271in}}%
\pgfpathlineto{\pgfqpoint{4.572966in}{2.363863in}}%
\pgfpathlineto{\pgfqpoint{4.565256in}{2.354025in}}%
\pgfpathlineto{\pgfqpoint{4.557541in}{2.344147in}}%
\pgfpathlineto{\pgfqpoint{4.549821in}{2.334230in}}%
\pgfpathlineto{\pgfqpoint{4.542095in}{2.324275in}}%
\pgfpathlineto{\pgfqpoint{4.528296in}{2.322803in}}%
\pgfpathlineto{\pgfqpoint{4.514507in}{2.321433in}}%
\pgfpathlineto{\pgfqpoint{4.500728in}{2.320168in}}%
\pgfpathlineto{\pgfqpoint{4.486958in}{2.319005in}}%
\pgfpathlineto{\pgfqpoint{4.494691in}{2.328834in}}%
\pgfpathlineto{\pgfqpoint{4.502418in}{2.338629in}}%
\pgfpathlineto{\pgfqpoint{4.510141in}{2.348390in}}%
\pgfpathlineto{\pgfqpoint{4.517858in}{2.358115in}}%
\pgfpathclose%
\pgfusepath{fill}%
\end{pgfscope}%
\begin{pgfscope}%
\pgfpathrectangle{\pgfqpoint{1.150000in}{0.150000in}}{\pgfqpoint{5.700000in}{5.700000in}}%
\pgfusepath{clip}%
\pgfsetbuttcap%
\pgfsetroundjoin%
\definecolor{currentfill}{rgb}{0.278012,0.180367,0.486697}%
\pgfsetfillcolor{currentfill}%
\pgfsetfillopacity{0.700000}%
\pgfsetlinewidth{0.000000pt}%
\definecolor{currentstroke}{rgb}{0.000000,0.000000,0.000000}%
\pgfsetstrokecolor{currentstroke}%
\pgfsetdash{}{0pt}%
\pgfpathmoveto{\pgfqpoint{4.689657in}{2.449181in}}%
\pgfpathlineto{\pgfqpoint{4.703486in}{2.451471in}}%
\pgfpathlineto{\pgfqpoint{4.717326in}{2.453864in}}%
\pgfpathlineto{\pgfqpoint{4.731177in}{2.456358in}}%
\pgfpathlineto{\pgfqpoint{4.745038in}{2.458955in}}%
\pgfpathlineto{\pgfqpoint{4.737387in}{2.449309in}}%
\pgfpathlineto{\pgfqpoint{4.729730in}{2.439609in}}%
\pgfpathlineto{\pgfqpoint{4.722068in}{2.429855in}}%
\pgfpathlineto{\pgfqpoint{4.714400in}{2.420047in}}%
\pgfpathlineto{\pgfqpoint{4.700532in}{2.417533in}}%
\pgfpathlineto{\pgfqpoint{4.686674in}{2.415122in}}%
\pgfpathlineto{\pgfqpoint{4.672827in}{2.412812in}}%
\pgfpathlineto{\pgfqpoint{4.658991in}{2.410605in}}%
\pgfpathlineto{\pgfqpoint{4.666666in}{2.420322in}}%
\pgfpathlineto{\pgfqpoint{4.674335in}{2.429991in}}%
\pgfpathlineto{\pgfqpoint{4.681999in}{2.439611in}}%
\pgfpathlineto{\pgfqpoint{4.689657in}{2.449181in}}%
\pgfpathclose%
\pgfusepath{fill}%
\end{pgfscope}%
\begin{pgfscope}%
\pgfpathrectangle{\pgfqpoint{1.150000in}{0.150000in}}{\pgfqpoint{5.700000in}{5.700000in}}%
\pgfusepath{clip}%
\pgfsetbuttcap%
\pgfsetroundjoin%
\definecolor{currentfill}{rgb}{0.269944,0.014625,0.341379}%
\pgfsetfillcolor{currentfill}%
\pgfsetfillopacity{0.700000}%
\pgfsetlinewidth{0.000000pt}%
\definecolor{currentstroke}{rgb}{0.000000,0.000000,0.000000}%
\pgfsetstrokecolor{currentstroke}%
\pgfsetdash{}{0pt}%
\pgfpathmoveto{\pgfqpoint{3.947944in}{2.152061in}}%
\pgfpathlineto{\pgfqpoint{3.961533in}{2.149357in}}%
\pgfpathlineto{\pgfqpoint{3.975128in}{2.146765in}}%
\pgfpathlineto{\pgfqpoint{3.988730in}{2.144284in}}%
\pgfpathlineto{\pgfqpoint{4.002339in}{2.141913in}}%
\pgfpathlineto{\pgfqpoint{3.994440in}{2.133014in}}%
\pgfpathlineto{\pgfqpoint{3.986535in}{2.124141in}}%
\pgfpathlineto{\pgfqpoint{3.978625in}{2.115298in}}%
\pgfpathlineto{\pgfqpoint{3.970709in}{2.106485in}}%
\pgfpathlineto{\pgfqpoint{3.957089in}{2.109083in}}%
\pgfpathlineto{\pgfqpoint{3.943476in}{2.111792in}}%
\pgfpathlineto{\pgfqpoint{3.929869in}{2.114612in}}%
\pgfpathlineto{\pgfqpoint{3.916268in}{2.117543in}}%
\pgfpathlineto{\pgfqpoint{3.924196in}{2.126121in}}%
\pgfpathlineto{\pgfqpoint{3.932117in}{2.134735in}}%
\pgfpathlineto{\pgfqpoint{3.940034in}{2.143383in}}%
\pgfpathlineto{\pgfqpoint{3.947944in}{2.152061in}}%
\pgfpathclose%
\pgfusepath{fill}%
\end{pgfscope}%
\begin{pgfscope}%
\pgfpathrectangle{\pgfqpoint{1.150000in}{0.150000in}}{\pgfqpoint{5.700000in}{5.700000in}}%
\pgfusepath{clip}%
\pgfsetbuttcap%
\pgfsetroundjoin%
\definecolor{currentfill}{rgb}{0.283091,0.110553,0.431554}%
\pgfsetfillcolor{currentfill}%
\pgfsetfillopacity{0.700000}%
\pgfsetlinewidth{0.000000pt}%
\definecolor{currentstroke}{rgb}{0.000000,0.000000,0.000000}%
\pgfsetstrokecolor{currentstroke}%
\pgfsetdash{}{0pt}%
\pgfpathmoveto{\pgfqpoint{4.431974in}{2.315396in}}%
\pgfpathlineto{\pgfqpoint{4.445706in}{2.316142in}}%
\pgfpathlineto{\pgfqpoint{4.459447in}{2.316992in}}%
\pgfpathlineto{\pgfqpoint{4.473198in}{2.317947in}}%
\pgfpathlineto{\pgfqpoint{4.486958in}{2.319005in}}%
\pgfpathlineto{\pgfqpoint{4.479220in}{2.309145in}}%
\pgfpathlineto{\pgfqpoint{4.471476in}{2.299253in}}%
\pgfpathlineto{\pgfqpoint{4.463727in}{2.289331in}}%
\pgfpathlineto{\pgfqpoint{4.455973in}{2.279379in}}%
\pgfpathlineto{\pgfqpoint{4.442206in}{2.278459in}}%
\pgfpathlineto{\pgfqpoint{4.428447in}{2.277642in}}%
\pgfpathlineto{\pgfqpoint{4.414699in}{2.276929in}}%
\pgfpathlineto{\pgfqpoint{4.400959in}{2.276320in}}%
\pgfpathlineto{\pgfqpoint{4.408721in}{2.286127in}}%
\pgfpathlineto{\pgfqpoint{4.416477in}{2.295910in}}%
\pgfpathlineto{\pgfqpoint{4.424228in}{2.305666in}}%
\pgfpathlineto{\pgfqpoint{4.431974in}{2.315396in}}%
\pgfpathclose%
\pgfusepath{fill}%
\end{pgfscope}%
\begin{pgfscope}%
\pgfpathrectangle{\pgfqpoint{1.150000in}{0.150000in}}{\pgfqpoint{5.700000in}{5.700000in}}%
\pgfusepath{clip}%
\pgfsetbuttcap%
\pgfsetroundjoin%
\definecolor{currentfill}{rgb}{0.268510,0.009605,0.335427}%
\pgfsetfillcolor{currentfill}%
\pgfsetfillopacity{0.700000}%
\pgfsetlinewidth{0.000000pt}%
\definecolor{currentstroke}{rgb}{0.000000,0.000000,0.000000}%
\pgfsetstrokecolor{currentstroke}%
\pgfsetdash{}{0pt}%
\pgfpathmoveto{\pgfqpoint{3.581219in}{2.143263in}}%
\pgfpathlineto{\pgfqpoint{3.594746in}{2.137505in}}%
\pgfpathlineto{\pgfqpoint{3.608276in}{2.131868in}}%
\pgfpathlineto{\pgfqpoint{3.621811in}{2.126352in}}%
\pgfpathlineto{\pgfqpoint{3.635350in}{2.120955in}}%
\pgfpathlineto{\pgfqpoint{3.627308in}{2.113895in}}%
\pgfpathlineto{\pgfqpoint{3.619259in}{2.106913in}}%
\pgfpathlineto{\pgfqpoint{3.611204in}{2.100011in}}%
\pgfpathlineto{\pgfqpoint{3.603141in}{2.093190in}}%
\pgfpathlineto{\pgfqpoint{3.589584in}{2.098870in}}%
\pgfpathlineto{\pgfqpoint{3.576032in}{2.104669in}}%
\pgfpathlineto{\pgfqpoint{3.562484in}{2.110588in}}%
\pgfpathlineto{\pgfqpoint{3.548940in}{2.116628in}}%
\pgfpathlineto{\pgfqpoint{3.557020in}{2.123159in}}%
\pgfpathlineto{\pgfqpoint{3.565094in}{2.129777in}}%
\pgfpathlineto{\pgfqpoint{3.573160in}{2.136479in}}%
\pgfpathlineto{\pgfqpoint{3.581219in}{2.143263in}}%
\pgfpathclose%
\pgfusepath{fill}%
\end{pgfscope}%
\begin{pgfscope}%
\pgfpathrectangle{\pgfqpoint{1.150000in}{0.150000in}}{\pgfqpoint{5.700000in}{5.700000in}}%
\pgfusepath{clip}%
\pgfsetbuttcap%
\pgfsetroundjoin%
\definecolor{currentfill}{rgb}{0.273006,0.204520,0.501721}%
\pgfsetfillcolor{currentfill}%
\pgfsetfillopacity{0.700000}%
\pgfsetlinewidth{0.000000pt}%
\definecolor{currentstroke}{rgb}{0.000000,0.000000,0.000000}%
\pgfsetstrokecolor{currentstroke}%
\pgfsetdash{}{0pt}%
\pgfpathmoveto{\pgfqpoint{4.775584in}{2.496986in}}%
\pgfpathlineto{\pgfqpoint{4.789450in}{2.499750in}}%
\pgfpathlineto{\pgfqpoint{4.803326in}{2.502615in}}%
\pgfpathlineto{\pgfqpoint{4.817213in}{2.505581in}}%
\pgfpathlineto{\pgfqpoint{4.831112in}{2.508650in}}%
\pgfpathlineto{\pgfqpoint{4.823491in}{2.499168in}}%
\pgfpathlineto{\pgfqpoint{4.815865in}{2.489626in}}%
\pgfpathlineto{\pgfqpoint{4.808232in}{2.480023in}}%
\pgfpathlineto{\pgfqpoint{4.800594in}{2.470361in}}%
\pgfpathlineto{\pgfqpoint{4.786688in}{2.467357in}}%
\pgfpathlineto{\pgfqpoint{4.772794in}{2.464454in}}%
\pgfpathlineto{\pgfqpoint{4.758910in}{2.461654in}}%
\pgfpathlineto{\pgfqpoint{4.745038in}{2.458955in}}%
\pgfpathlineto{\pgfqpoint{4.752683in}{2.468546in}}%
\pgfpathlineto{\pgfqpoint{4.760323in}{2.478082in}}%
\pgfpathlineto{\pgfqpoint{4.767956in}{2.487562in}}%
\pgfpathlineto{\pgfqpoint{4.775584in}{2.496986in}}%
\pgfpathclose%
\pgfusepath{fill}%
\end{pgfscope}%
\begin{pgfscope}%
\pgfpathrectangle{\pgfqpoint{1.150000in}{0.150000in}}{\pgfqpoint{5.700000in}{5.700000in}}%
\pgfusepath{clip}%
\pgfsetbuttcap%
\pgfsetroundjoin%
\definecolor{currentfill}{rgb}{0.279566,0.067836,0.391917}%
\pgfsetfillcolor{currentfill}%
\pgfsetfillopacity{0.700000}%
\pgfsetlinewidth{0.000000pt}%
\definecolor{currentstroke}{rgb}{0.000000,0.000000,0.000000}%
\pgfsetstrokecolor{currentstroke}%
\pgfsetdash{}{0pt}%
\pgfpathmoveto{\pgfqpoint{3.245878in}{2.244618in}}%
\pgfpathlineto{\pgfqpoint{3.259391in}{2.235721in}}%
\pgfpathlineto{\pgfqpoint{3.272906in}{2.226958in}}%
\pgfpathlineto{\pgfqpoint{3.286422in}{2.218330in}}%
\pgfpathlineto{\pgfqpoint{3.299940in}{2.209835in}}%
\pgfpathlineto{\pgfqpoint{3.291736in}{2.205035in}}%
\pgfpathlineto{\pgfqpoint{3.283524in}{2.200357in}}%
\pgfpathlineto{\pgfqpoint{3.275303in}{2.195804in}}%
\pgfpathlineto{\pgfqpoint{3.267072in}{2.191381in}}%
\pgfpathlineto{\pgfqpoint{3.253531in}{2.200198in}}%
\pgfpathlineto{\pgfqpoint{3.239991in}{2.209148in}}%
\pgfpathlineto{\pgfqpoint{3.226452in}{2.218233in}}%
\pgfpathlineto{\pgfqpoint{3.212915in}{2.227452in}}%
\pgfpathlineto{\pgfqpoint{3.221169in}{2.231546in}}%
\pgfpathlineto{\pgfqpoint{3.229415in}{2.235774in}}%
\pgfpathlineto{\pgfqpoint{3.237651in}{2.240132in}}%
\pgfpathlineto{\pgfqpoint{3.245878in}{2.244618in}}%
\pgfpathclose%
\pgfusepath{fill}%
\end{pgfscope}%
\begin{pgfscope}%
\pgfpathrectangle{\pgfqpoint{1.150000in}{0.150000in}}{\pgfqpoint{5.700000in}{5.700000in}}%
\pgfusepath{clip}%
\pgfsetbuttcap%
\pgfsetroundjoin%
\definecolor{currentfill}{rgb}{0.281924,0.089666,0.412415}%
\pgfsetfillcolor{currentfill}%
\pgfsetfillopacity{0.700000}%
\pgfsetlinewidth{0.000000pt}%
\definecolor{currentstroke}{rgb}{0.000000,0.000000,0.000000}%
\pgfsetstrokecolor{currentstroke}%
\pgfsetdash{}{0pt}%
\pgfpathmoveto{\pgfqpoint{4.346092in}{2.274935in}}%
\pgfpathlineto{\pgfqpoint{4.359795in}{2.275124in}}%
\pgfpathlineto{\pgfqpoint{4.373508in}{2.275418in}}%
\pgfpathlineto{\pgfqpoint{4.387229in}{2.275817in}}%
\pgfpathlineto{\pgfqpoint{4.400959in}{2.276320in}}%
\pgfpathlineto{\pgfqpoint{4.393192in}{2.266490in}}%
\pgfpathlineto{\pgfqpoint{4.385420in}{2.256638in}}%
\pgfpathlineto{\pgfqpoint{4.377643in}{2.246765in}}%
\pgfpathlineto{\pgfqpoint{4.369860in}{2.236873in}}%
\pgfpathlineto{\pgfqpoint{4.356122in}{2.236525in}}%
\pgfpathlineto{\pgfqpoint{4.342393in}{2.236281in}}%
\pgfpathlineto{\pgfqpoint{4.328672in}{2.236143in}}%
\pgfpathlineto{\pgfqpoint{4.314961in}{2.236110in}}%
\pgfpathlineto{\pgfqpoint{4.322751in}{2.245840in}}%
\pgfpathlineto{\pgfqpoint{4.330537in}{2.255555in}}%
\pgfpathlineto{\pgfqpoint{4.338317in}{2.265254in}}%
\pgfpathlineto{\pgfqpoint{4.346092in}{2.274935in}}%
\pgfpathclose%
\pgfusepath{fill}%
\end{pgfscope}%
\begin{pgfscope}%
\pgfpathrectangle{\pgfqpoint{1.150000in}{0.150000in}}{\pgfqpoint{5.700000in}{5.700000in}}%
\pgfusepath{clip}%
\pgfsetbuttcap%
\pgfsetroundjoin%
\definecolor{currentfill}{rgb}{0.267004,0.004874,0.329415}%
\pgfsetfillcolor{currentfill}%
\pgfsetfillopacity{0.700000}%
\pgfsetlinewidth{0.000000pt}%
\definecolor{currentstroke}{rgb}{0.000000,0.000000,0.000000}%
\pgfsetstrokecolor{currentstroke}%
\pgfsetdash{}{0pt}%
\pgfpathmoveto{\pgfqpoint{3.721587in}{2.130563in}}%
\pgfpathlineto{\pgfqpoint{3.735134in}{2.126019in}}%
\pgfpathlineto{\pgfqpoint{3.748685in}{2.121592in}}%
\pgfpathlineto{\pgfqpoint{3.762242in}{2.117281in}}%
\pgfpathlineto{\pgfqpoint{3.775804in}{2.113086in}}%
\pgfpathlineto{\pgfqpoint{3.767819in}{2.105230in}}%
\pgfpathlineto{\pgfqpoint{3.759829in}{2.097432in}}%
\pgfpathlineto{\pgfqpoint{3.751831in}{2.089694in}}%
\pgfpathlineto{\pgfqpoint{3.743828in}{2.082019in}}%
\pgfpathlineto{\pgfqpoint{3.730251in}{2.086479in}}%
\pgfpathlineto{\pgfqpoint{3.716679in}{2.091053in}}%
\pgfpathlineto{\pgfqpoint{3.703113in}{2.095744in}}%
\pgfpathlineto{\pgfqpoint{3.689551in}{2.100551in}}%
\pgfpathlineto{\pgfqpoint{3.697570in}{2.107956in}}%
\pgfpathlineto{\pgfqpoint{3.705582in}{2.115427in}}%
\pgfpathlineto{\pgfqpoint{3.713588in}{2.122964in}}%
\pgfpathlineto{\pgfqpoint{3.721587in}{2.130563in}}%
\pgfpathclose%
\pgfusepath{fill}%
\end{pgfscope}%
\begin{pgfscope}%
\pgfpathrectangle{\pgfqpoint{1.150000in}{0.150000in}}{\pgfqpoint{5.700000in}{5.700000in}}%
\pgfusepath{clip}%
\pgfsetbuttcap%
\pgfsetroundjoin%
\definecolor{currentfill}{rgb}{0.266580,0.228262,0.514349}%
\pgfsetfillcolor{currentfill}%
\pgfsetfillopacity{0.700000}%
\pgfsetlinewidth{0.000000pt}%
\definecolor{currentstroke}{rgb}{0.000000,0.000000,0.000000}%
\pgfsetstrokecolor{currentstroke}%
\pgfsetdash{}{0pt}%
\pgfpathmoveto{\pgfqpoint{4.861535in}{2.545966in}}%
\pgfpathlineto{\pgfqpoint{4.875437in}{2.549182in}}%
\pgfpathlineto{\pgfqpoint{4.889352in}{2.552499in}}%
\pgfpathlineto{\pgfqpoint{4.903277in}{2.555918in}}%
\pgfpathlineto{\pgfqpoint{4.917215in}{2.559437in}}%
\pgfpathlineto{\pgfqpoint{4.909625in}{2.550161in}}%
\pgfpathlineto{\pgfqpoint{4.902030in}{2.540819in}}%
\pgfpathlineto{\pgfqpoint{4.894429in}{2.531411in}}%
\pgfpathlineto{\pgfqpoint{4.886821in}{2.521937in}}%
\pgfpathlineto{\pgfqpoint{4.872876in}{2.518463in}}%
\pgfpathlineto{\pgfqpoint{4.858943in}{2.515091in}}%
\pgfpathlineto{\pgfqpoint{4.845022in}{2.511820in}}%
\pgfpathlineto{\pgfqpoint{4.831112in}{2.508650in}}%
\pgfpathlineto{\pgfqpoint{4.838727in}{2.518071in}}%
\pgfpathlineto{\pgfqpoint{4.846335in}{2.527430in}}%
\pgfpathlineto{\pgfqpoint{4.853938in}{2.536729in}}%
\pgfpathlineto{\pgfqpoint{4.861535in}{2.545966in}}%
\pgfpathclose%
\pgfusepath{fill}%
\end{pgfscope}%
\begin{pgfscope}%
\pgfpathrectangle{\pgfqpoint{1.150000in}{0.150000in}}{\pgfqpoint{5.700000in}{5.700000in}}%
\pgfusepath{clip}%
\pgfsetbuttcap%
\pgfsetroundjoin%
\definecolor{currentfill}{rgb}{0.282290,0.145912,0.461510}%
\pgfsetfillcolor{currentfill}%
\pgfsetfillopacity{0.700000}%
\pgfsetlinewidth{0.000000pt}%
\definecolor{currentstroke}{rgb}{0.000000,0.000000,0.000000}%
\pgfsetstrokecolor{currentstroke}%
\pgfsetdash{}{0pt}%
\pgfpathmoveto{\pgfqpoint{2.996373in}{2.394071in}}%
\pgfpathlineto{\pgfqpoint{3.009909in}{2.382564in}}%
\pgfpathlineto{\pgfqpoint{3.023444in}{2.371209in}}%
\pgfpathlineto{\pgfqpoint{3.036978in}{2.360003in}}%
\pgfpathlineto{\pgfqpoint{3.050511in}{2.348945in}}%
\pgfpathlineto{\pgfqpoint{3.042167in}{2.345995in}}%
\pgfpathlineto{\pgfqpoint{3.033811in}{2.343200in}}%
\pgfpathlineto{\pgfqpoint{3.025445in}{2.340563in}}%
\pgfpathlineto{\pgfqpoint{3.017067in}{2.338087in}}%
\pgfpathlineto{\pgfqpoint{3.003505in}{2.349490in}}%
\pgfpathlineto{\pgfqpoint{2.989941in}{2.361041in}}%
\pgfpathlineto{\pgfqpoint{2.976377in}{2.372743in}}%
\pgfpathlineto{\pgfqpoint{2.962812in}{2.384595in}}%
\pgfpathlineto{\pgfqpoint{2.971219in}{2.386718in}}%
\pgfpathlineto{\pgfqpoint{2.979615in}{2.389008in}}%
\pgfpathlineto{\pgfqpoint{2.988000in}{2.391459in}}%
\pgfpathlineto{\pgfqpoint{2.996373in}{2.394071in}}%
\pgfpathclose%
\pgfusepath{fill}%
\end{pgfscope}%
\begin{pgfscope}%
\pgfpathrectangle{\pgfqpoint{1.150000in}{0.150000in}}{\pgfqpoint{5.700000in}{5.700000in}}%
\pgfusepath{clip}%
\pgfsetbuttcap%
\pgfsetroundjoin%
\definecolor{currentfill}{rgb}{0.272594,0.025563,0.353093}%
\pgfsetfillcolor{currentfill}%
\pgfsetfillopacity{0.700000}%
\pgfsetlinewidth{0.000000pt}%
\definecolor{currentstroke}{rgb}{0.000000,0.000000,0.000000}%
\pgfsetstrokecolor{currentstroke}%
\pgfsetdash{}{0pt}%
\pgfpathmoveto{\pgfqpoint{3.440710in}{2.169369in}}%
\pgfpathlineto{\pgfqpoint{3.454228in}{2.162341in}}%
\pgfpathlineto{\pgfqpoint{3.467748in}{2.155438in}}%
\pgfpathlineto{\pgfqpoint{3.481272in}{2.148661in}}%
\pgfpathlineto{\pgfqpoint{3.494798in}{2.142008in}}%
\pgfpathlineto{\pgfqpoint{3.486691in}{2.135861in}}%
\pgfpathlineto{\pgfqpoint{3.478577in}{2.129810in}}%
\pgfpathlineto{\pgfqpoint{3.470454in}{2.123860in}}%
\pgfpathlineto{\pgfqpoint{3.462324in}{2.118012in}}%
\pgfpathlineto{\pgfqpoint{3.448777in}{2.124967in}}%
\pgfpathlineto{\pgfqpoint{3.435234in}{2.132046in}}%
\pgfpathlineto{\pgfqpoint{3.421693in}{2.139250in}}%
\pgfpathlineto{\pgfqpoint{3.408155in}{2.146580in}}%
\pgfpathlineto{\pgfqpoint{3.416306in}{2.152119in}}%
\pgfpathlineto{\pgfqpoint{3.424449in}{2.157765in}}%
\pgfpathlineto{\pgfqpoint{3.432584in}{2.163516in}}%
\pgfpathlineto{\pgfqpoint{3.440710in}{2.169369in}}%
\pgfpathclose%
\pgfusepath{fill}%
\end{pgfscope}%
\begin{pgfscope}%
\pgfpathrectangle{\pgfqpoint{1.150000in}{0.150000in}}{\pgfqpoint{5.700000in}{5.700000in}}%
\pgfusepath{clip}%
\pgfsetbuttcap%
\pgfsetroundjoin%
\definecolor{currentfill}{rgb}{0.258965,0.251537,0.524736}%
\pgfsetfillcolor{currentfill}%
\pgfsetfillopacity{0.700000}%
\pgfsetlinewidth{0.000000pt}%
\definecolor{currentstroke}{rgb}{0.000000,0.000000,0.000000}%
\pgfsetstrokecolor{currentstroke}%
\pgfsetdash{}{0pt}%
\pgfpathmoveto{\pgfqpoint{4.947510in}{2.595880in}}%
\pgfpathlineto{\pgfqpoint{4.961452in}{2.599528in}}%
\pgfpathlineto{\pgfqpoint{4.975405in}{2.603277in}}%
\pgfpathlineto{\pgfqpoint{4.989370in}{2.607127in}}%
\pgfpathlineto{\pgfqpoint{5.003347in}{2.611077in}}%
\pgfpathlineto{\pgfqpoint{4.995791in}{2.602045in}}%
\pgfpathlineto{\pgfqpoint{4.988228in}{2.592943in}}%
\pgfpathlineto{\pgfqpoint{4.980658in}{2.583769in}}%
\pgfpathlineto{\pgfqpoint{4.973083in}{2.574525in}}%
\pgfpathlineto{\pgfqpoint{4.959098in}{2.570602in}}%
\pgfpathlineto{\pgfqpoint{4.945125in}{2.566779in}}%
\pgfpathlineto{\pgfqpoint{4.931164in}{2.563058in}}%
\pgfpathlineto{\pgfqpoint{4.917215in}{2.559437in}}%
\pgfpathlineto{\pgfqpoint{4.924798in}{2.568647in}}%
\pgfpathlineto{\pgfqpoint{4.932375in}{2.577791in}}%
\pgfpathlineto{\pgfqpoint{4.939946in}{2.586868in}}%
\pgfpathlineto{\pgfqpoint{4.947510in}{2.595880in}}%
\pgfpathclose%
\pgfusepath{fill}%
\end{pgfscope}%
\begin{pgfscope}%
\pgfpathrectangle{\pgfqpoint{1.150000in}{0.150000in}}{\pgfqpoint{5.700000in}{5.700000in}}%
\pgfusepath{clip}%
\pgfsetbuttcap%
\pgfsetroundjoin%
\definecolor{currentfill}{rgb}{0.280267,0.073417,0.397163}%
\pgfsetfillcolor{currentfill}%
\pgfsetfillopacity{0.700000}%
\pgfsetlinewidth{0.000000pt}%
\definecolor{currentstroke}{rgb}{0.000000,0.000000,0.000000}%
\pgfsetstrokecolor{currentstroke}%
\pgfsetdash{}{0pt}%
\pgfpathmoveto{\pgfqpoint{4.260200in}{2.237034in}}%
\pgfpathlineto{\pgfqpoint{4.273878in}{2.236644in}}%
\pgfpathlineto{\pgfqpoint{4.287563in}{2.236360in}}%
\pgfpathlineto{\pgfqpoint{4.301258in}{2.236182in}}%
\pgfpathlineto{\pgfqpoint{4.314961in}{2.236110in}}%
\pgfpathlineto{\pgfqpoint{4.307165in}{2.226366in}}%
\pgfpathlineto{\pgfqpoint{4.299363in}{2.216610in}}%
\pgfpathlineto{\pgfqpoint{4.291557in}{2.206844in}}%
\pgfpathlineto{\pgfqpoint{4.283745in}{2.197068in}}%
\pgfpathlineto{\pgfqpoint{4.270034in}{2.197314in}}%
\pgfpathlineto{\pgfqpoint{4.256331in}{2.197665in}}%
\pgfpathlineto{\pgfqpoint{4.242637in}{2.198123in}}%
\pgfpathlineto{\pgfqpoint{4.228951in}{2.198686in}}%
\pgfpathlineto{\pgfqpoint{4.236771in}{2.208282in}}%
\pgfpathlineto{\pgfqpoint{4.244586in}{2.217872in}}%
\pgfpathlineto{\pgfqpoint{4.252396in}{2.227457in}}%
\pgfpathlineto{\pgfqpoint{4.260200in}{2.237034in}}%
\pgfpathclose%
\pgfusepath{fill}%
\end{pgfscope}%
\begin{pgfscope}%
\pgfpathrectangle{\pgfqpoint{1.150000in}{0.150000in}}{\pgfqpoint{5.700000in}{5.700000in}}%
\pgfusepath{clip}%
\pgfsetbuttcap%
\pgfsetroundjoin%
\definecolor{currentfill}{rgb}{0.250425,0.274290,0.533103}%
\pgfsetfillcolor{currentfill}%
\pgfsetfillopacity{0.700000}%
\pgfsetlinewidth{0.000000pt}%
\definecolor{currentstroke}{rgb}{0.000000,0.000000,0.000000}%
\pgfsetstrokecolor{currentstroke}%
\pgfsetdash{}{0pt}%
\pgfpathmoveto{\pgfqpoint{5.033511in}{2.646500in}}%
\pgfpathlineto{\pgfqpoint{5.047492in}{2.650560in}}%
\pgfpathlineto{\pgfqpoint{5.061486in}{2.654720in}}%
\pgfpathlineto{\pgfqpoint{5.075492in}{2.658981in}}%
\pgfpathlineto{\pgfqpoint{5.089510in}{2.663343in}}%
\pgfpathlineto{\pgfqpoint{5.081987in}{2.654591in}}%
\pgfpathlineto{\pgfqpoint{5.074458in}{2.645764in}}%
\pgfpathlineto{\pgfqpoint{5.066922in}{2.636862in}}%
\pgfpathlineto{\pgfqpoint{5.059379in}{2.627885in}}%
\pgfpathlineto{\pgfqpoint{5.045353in}{2.623533in}}%
\pgfpathlineto{\pgfqpoint{5.031339in}{2.619280in}}%
\pgfpathlineto{\pgfqpoint{5.017337in}{2.615129in}}%
\pgfpathlineto{\pgfqpoint{5.003347in}{2.611077in}}%
\pgfpathlineto{\pgfqpoint{5.010898in}{2.620039in}}%
\pgfpathlineto{\pgfqpoint{5.018442in}{2.628929in}}%
\pgfpathlineto{\pgfqpoint{5.025980in}{2.637750in}}%
\pgfpathlineto{\pgfqpoint{5.033511in}{2.646500in}}%
\pgfpathclose%
\pgfusepath{fill}%
\end{pgfscope}%
\begin{pgfscope}%
\pgfpathrectangle{\pgfqpoint{1.150000in}{0.150000in}}{\pgfqpoint{5.700000in}{5.700000in}}%
\pgfusepath{clip}%
\pgfsetbuttcap%
\pgfsetroundjoin%
\definecolor{currentfill}{rgb}{0.277018,0.050344,0.375715}%
\pgfsetfillcolor{currentfill}%
\pgfsetfillopacity{0.700000}%
\pgfsetlinewidth{0.000000pt}%
\definecolor{currentstroke}{rgb}{0.000000,0.000000,0.000000}%
\pgfsetstrokecolor{currentstroke}%
\pgfsetdash{}{0pt}%
\pgfpathmoveto{\pgfqpoint{4.174287in}{2.202008in}}%
\pgfpathlineto{\pgfqpoint{4.187941in}{2.201017in}}%
\pgfpathlineto{\pgfqpoint{4.201603in}{2.200133in}}%
\pgfpathlineto{\pgfqpoint{4.215273in}{2.199357in}}%
\pgfpathlineto{\pgfqpoint{4.228951in}{2.198686in}}%
\pgfpathlineto{\pgfqpoint{4.221125in}{2.189088in}}%
\pgfpathlineto{\pgfqpoint{4.213294in}{2.179488in}}%
\pgfpathlineto{\pgfqpoint{4.205458in}{2.169888in}}%
\pgfpathlineto{\pgfqpoint{4.197617in}{2.160291in}}%
\pgfpathlineto{\pgfqpoint{4.183930in}{2.161152in}}%
\pgfpathlineto{\pgfqpoint{4.170251in}{2.162121in}}%
\pgfpathlineto{\pgfqpoint{4.156580in}{2.163196in}}%
\pgfpathlineto{\pgfqpoint{4.142916in}{2.164378in}}%
\pgfpathlineto{\pgfqpoint{4.150767in}{2.173778in}}%
\pgfpathlineto{\pgfqpoint{4.158612in}{2.183184in}}%
\pgfpathlineto{\pgfqpoint{4.166452in}{2.192595in}}%
\pgfpathlineto{\pgfqpoint{4.174287in}{2.202008in}}%
\pgfpathclose%
\pgfusepath{fill}%
\end{pgfscope}%
\begin{pgfscope}%
\pgfpathrectangle{\pgfqpoint{1.150000in}{0.150000in}}{\pgfqpoint{5.700000in}{5.700000in}}%
\pgfusepath{clip}%
\pgfsetbuttcap%
\pgfsetroundjoin%
\definecolor{currentfill}{rgb}{0.239346,0.300855,0.540844}%
\pgfsetfillcolor{currentfill}%
\pgfsetfillopacity{0.700000}%
\pgfsetlinewidth{0.000000pt}%
\definecolor{currentstroke}{rgb}{0.000000,0.000000,0.000000}%
\pgfsetstrokecolor{currentstroke}%
\pgfsetdash{}{0pt}%
\pgfpathmoveto{\pgfqpoint{5.119536in}{2.697610in}}%
\pgfpathlineto{\pgfqpoint{5.133558in}{2.702061in}}%
\pgfpathlineto{\pgfqpoint{5.147593in}{2.706613in}}%
\pgfpathlineto{\pgfqpoint{5.161641in}{2.711265in}}%
\pgfpathlineto{\pgfqpoint{5.175701in}{2.716017in}}%
\pgfpathlineto{\pgfqpoint{5.168213in}{2.707578in}}%
\pgfpathlineto{\pgfqpoint{5.160719in}{2.699061in}}%
\pgfpathlineto{\pgfqpoint{5.153218in}{2.690465in}}%
\pgfpathlineto{\pgfqpoint{5.145710in}{2.681791in}}%
\pgfpathlineto{\pgfqpoint{5.131641in}{2.677028in}}%
\pgfpathlineto{\pgfqpoint{5.117585in}{2.672366in}}%
\pgfpathlineto{\pgfqpoint{5.103541in}{2.667804in}}%
\pgfpathlineto{\pgfqpoint{5.089510in}{2.663343in}}%
\pgfpathlineto{\pgfqpoint{5.097026in}{2.672020in}}%
\pgfpathlineto{\pgfqpoint{5.104536in}{2.680624in}}%
\pgfpathlineto{\pgfqpoint{5.112039in}{2.689153in}}%
\pgfpathlineto{\pgfqpoint{5.119536in}{2.697610in}}%
\pgfpathclose%
\pgfusepath{fill}%
\end{pgfscope}%
\begin{pgfscope}%
\pgfpathrectangle{\pgfqpoint{1.150000in}{0.150000in}}{\pgfqpoint{5.700000in}{5.700000in}}%
\pgfusepath{clip}%
\pgfsetbuttcap%
\pgfsetroundjoin%
\definecolor{currentfill}{rgb}{0.268510,0.009605,0.335427}%
\pgfsetfillcolor{currentfill}%
\pgfsetfillopacity{0.700000}%
\pgfsetlinewidth{0.000000pt}%
\definecolor{currentstroke}{rgb}{0.000000,0.000000,0.000000}%
\pgfsetstrokecolor{currentstroke}%
\pgfsetdash{}{0pt}%
\pgfpathmoveto{\pgfqpoint{3.861927in}{2.130386in}}%
\pgfpathlineto{\pgfqpoint{3.875503in}{2.127006in}}%
\pgfpathlineto{\pgfqpoint{3.889085in}{2.123739in}}%
\pgfpathlineto{\pgfqpoint{3.902673in}{2.120585in}}%
\pgfpathlineto{\pgfqpoint{3.916268in}{2.117543in}}%
\pgfpathlineto{\pgfqpoint{3.908334in}{2.109001in}}%
\pgfpathlineto{\pgfqpoint{3.900395in}{2.100500in}}%
\pgfpathlineto{\pgfqpoint{3.892450in}{2.092040in}}%
\pgfpathlineto{\pgfqpoint{3.884499in}{2.083624in}}%
\pgfpathlineto{\pgfqpoint{3.870892in}{2.086912in}}%
\pgfpathlineto{\pgfqpoint{3.857290in}{2.090312in}}%
\pgfpathlineto{\pgfqpoint{3.843695in}{2.093824in}}%
\pgfpathlineto{\pgfqpoint{3.830106in}{2.097449in}}%
\pgfpathlineto{\pgfqpoint{3.838070in}{2.105612in}}%
\pgfpathlineto{\pgfqpoint{3.846028in}{2.113825in}}%
\pgfpathlineto{\pgfqpoint{3.853980in}{2.122083in}}%
\pgfpathlineto{\pgfqpoint{3.861927in}{2.130386in}}%
\pgfpathclose%
\pgfusepath{fill}%
\end{pgfscope}%
\begin{pgfscope}%
\pgfpathrectangle{\pgfqpoint{1.150000in}{0.150000in}}{\pgfqpoint{5.700000in}{5.700000in}}%
\pgfusepath{clip}%
\pgfsetbuttcap%
\pgfsetroundjoin%
\definecolor{currentfill}{rgb}{0.283187,0.125848,0.444960}%
\pgfsetfillcolor{currentfill}%
\pgfsetfillopacity{0.700000}%
\pgfsetlinewidth{0.000000pt}%
\definecolor{currentstroke}{rgb}{0.000000,0.000000,0.000000}%
\pgfsetstrokecolor{currentstroke}%
\pgfsetdash{}{0pt}%
\pgfpathmoveto{\pgfqpoint{3.050511in}{2.348945in}}%
\pgfpathlineto{\pgfqpoint{3.064045in}{2.338034in}}%
\pgfpathlineto{\pgfqpoint{3.077577in}{2.327270in}}%
\pgfpathlineto{\pgfqpoint{3.091110in}{2.316651in}}%
\pgfpathlineto{\pgfqpoint{3.104642in}{2.306176in}}%
\pgfpathlineto{\pgfqpoint{3.096325in}{2.302889in}}%
\pgfpathlineto{\pgfqpoint{3.087998in}{2.299752in}}%
\pgfpathlineto{\pgfqpoint{3.079660in}{2.296768in}}%
\pgfpathlineto{\pgfqpoint{3.071311in}{2.293940in}}%
\pgfpathlineto{\pgfqpoint{3.057750in}{2.304759in}}%
\pgfpathlineto{\pgfqpoint{3.044190in}{2.315723in}}%
\pgfpathlineto{\pgfqpoint{3.030629in}{2.326831in}}%
\pgfpathlineto{\pgfqpoint{3.017067in}{2.338087in}}%
\pgfpathlineto{\pgfqpoint{3.025445in}{2.340563in}}%
\pgfpathlineto{\pgfqpoint{3.033811in}{2.343200in}}%
\pgfpathlineto{\pgfqpoint{3.042167in}{2.345995in}}%
\pgfpathlineto{\pgfqpoint{3.050511in}{2.348945in}}%
\pgfpathclose%
\pgfusepath{fill}%
\end{pgfscope}%
\begin{pgfscope}%
\pgfpathrectangle{\pgfqpoint{1.150000in}{0.150000in}}{\pgfqpoint{5.700000in}{5.700000in}}%
\pgfusepath{clip}%
\pgfsetbuttcap%
\pgfsetroundjoin%
\definecolor{currentfill}{rgb}{0.229739,0.322361,0.545706}%
\pgfsetfillcolor{currentfill}%
\pgfsetfillopacity{0.700000}%
\pgfsetlinewidth{0.000000pt}%
\definecolor{currentstroke}{rgb}{0.000000,0.000000,0.000000}%
\pgfsetstrokecolor{currentstroke}%
\pgfsetdash{}{0pt}%
\pgfpathmoveto{\pgfqpoint{5.205583in}{2.749005in}}%
\pgfpathlineto{\pgfqpoint{5.219647in}{2.753828in}}%
\pgfpathlineto{\pgfqpoint{5.233724in}{2.758751in}}%
\pgfpathlineto{\pgfqpoint{5.247814in}{2.763774in}}%
\pgfpathlineto{\pgfqpoint{5.261917in}{2.768896in}}%
\pgfpathlineto{\pgfqpoint{5.254467in}{2.760800in}}%
\pgfpathlineto{\pgfqpoint{5.247009in}{2.752623in}}%
\pgfpathlineto{\pgfqpoint{5.239544in}{2.744365in}}%
\pgfpathlineto{\pgfqpoint{5.232073in}{2.736025in}}%
\pgfpathlineto{\pgfqpoint{5.217960in}{2.730873in}}%
\pgfpathlineto{\pgfqpoint{5.203861in}{2.725821in}}%
\pgfpathlineto{\pgfqpoint{5.189774in}{2.720869in}}%
\pgfpathlineto{\pgfqpoint{5.175701in}{2.716017in}}%
\pgfpathlineto{\pgfqpoint{5.183182in}{2.724379in}}%
\pgfpathlineto{\pgfqpoint{5.190656in}{2.732664in}}%
\pgfpathlineto{\pgfqpoint{5.198123in}{2.740872in}}%
\pgfpathlineto{\pgfqpoint{5.205583in}{2.749005in}}%
\pgfpathclose%
\pgfusepath{fill}%
\end{pgfscope}%
\begin{pgfscope}%
\pgfpathrectangle{\pgfqpoint{1.150000in}{0.150000in}}{\pgfqpoint{5.700000in}{5.700000in}}%
\pgfusepath{clip}%
\pgfsetbuttcap%
\pgfsetroundjoin%
\definecolor{currentfill}{rgb}{0.277941,0.056324,0.381191}%
\pgfsetfillcolor{currentfill}%
\pgfsetfillopacity{0.700000}%
\pgfsetlinewidth{0.000000pt}%
\definecolor{currentstroke}{rgb}{0.000000,0.000000,0.000000}%
\pgfsetstrokecolor{currentstroke}%
\pgfsetdash{}{0pt}%
\pgfpathmoveto{\pgfqpoint{3.299940in}{2.209835in}}%
\pgfpathlineto{\pgfqpoint{3.313459in}{2.201473in}}%
\pgfpathlineto{\pgfqpoint{3.326981in}{2.193242in}}%
\pgfpathlineto{\pgfqpoint{3.340504in}{2.185142in}}%
\pgfpathlineto{\pgfqpoint{3.354030in}{2.177172in}}%
\pgfpathlineto{\pgfqpoint{3.345849in}{2.172057in}}%
\pgfpathlineto{\pgfqpoint{3.337660in}{2.167060in}}%
\pgfpathlineto{\pgfqpoint{3.329461in}{2.162184in}}%
\pgfpathlineto{\pgfqpoint{3.321254in}{2.157432in}}%
\pgfpathlineto{\pgfqpoint{3.307706in}{2.165723in}}%
\pgfpathlineto{\pgfqpoint{3.294160in}{2.174144in}}%
\pgfpathlineto{\pgfqpoint{3.280615in}{2.182697in}}%
\pgfpathlineto{\pgfqpoint{3.267072in}{2.191381in}}%
\pgfpathlineto{\pgfqpoint{3.275303in}{2.195804in}}%
\pgfpathlineto{\pgfqpoint{3.283524in}{2.200357in}}%
\pgfpathlineto{\pgfqpoint{3.291736in}{2.205035in}}%
\pgfpathlineto{\pgfqpoint{3.299940in}{2.209835in}}%
\pgfpathclose%
\pgfusepath{fill}%
\end{pgfscope}%
\begin{pgfscope}%
\pgfpathrectangle{\pgfqpoint{1.150000in}{0.150000in}}{\pgfqpoint{5.700000in}{5.700000in}}%
\pgfusepath{clip}%
\pgfsetbuttcap%
\pgfsetroundjoin%
\definecolor{currentfill}{rgb}{0.274952,0.037752,0.364543}%
\pgfsetfillcolor{currentfill}%
\pgfsetfillopacity{0.700000}%
\pgfsetlinewidth{0.000000pt}%
\definecolor{currentstroke}{rgb}{0.000000,0.000000,0.000000}%
\pgfsetstrokecolor{currentstroke}%
\pgfsetdash{}{0pt}%
\pgfpathmoveto{\pgfqpoint{4.088339in}{2.170187in}}%
\pgfpathlineto{\pgfqpoint{4.101972in}{2.168572in}}%
\pgfpathlineto{\pgfqpoint{4.115613in}{2.167066in}}%
\pgfpathlineto{\pgfqpoint{4.129261in}{2.165668in}}%
\pgfpathlineto{\pgfqpoint{4.142916in}{2.164378in}}%
\pgfpathlineto{\pgfqpoint{4.135060in}{2.154987in}}%
\pgfpathlineto{\pgfqpoint{4.127199in}{2.145606in}}%
\pgfpathlineto{\pgfqpoint{4.119332in}{2.136236in}}%
\pgfpathlineto{\pgfqpoint{4.111460in}{2.126880in}}%
\pgfpathlineto{\pgfqpoint{4.097794in}{2.128379in}}%
\pgfpathlineto{\pgfqpoint{4.084136in}{2.129987in}}%
\pgfpathlineto{\pgfqpoint{4.070485in}{2.131702in}}%
\pgfpathlineto{\pgfqpoint{4.056842in}{2.133526in}}%
\pgfpathlineto{\pgfqpoint{4.064724in}{2.142666in}}%
\pgfpathlineto{\pgfqpoint{4.072601in}{2.151824in}}%
\pgfpathlineto{\pgfqpoint{4.080473in}{2.160998in}}%
\pgfpathlineto{\pgfqpoint{4.088339in}{2.170187in}}%
\pgfpathclose%
\pgfusepath{fill}%
\end{pgfscope}%
\begin{pgfscope}%
\pgfpathrectangle{\pgfqpoint{1.150000in}{0.150000in}}{\pgfqpoint{5.700000in}{5.700000in}}%
\pgfusepath{clip}%
\pgfsetbuttcap%
\pgfsetroundjoin%
\definecolor{currentfill}{rgb}{0.218130,0.347432,0.550038}%
\pgfsetfillcolor{currentfill}%
\pgfsetfillopacity{0.700000}%
\pgfsetlinewidth{0.000000pt}%
\definecolor{currentstroke}{rgb}{0.000000,0.000000,0.000000}%
\pgfsetstrokecolor{currentstroke}%
\pgfsetdash{}{0pt}%
\pgfpathmoveto{\pgfqpoint{5.291650in}{2.800494in}}%
\pgfpathlineto{\pgfqpoint{5.305756in}{2.805669in}}%
\pgfpathlineto{\pgfqpoint{5.319876in}{2.810942in}}%
\pgfpathlineto{\pgfqpoint{5.334009in}{2.816316in}}%
\pgfpathlineto{\pgfqpoint{5.348156in}{2.821789in}}%
\pgfpathlineto{\pgfqpoint{5.340744in}{2.814062in}}%
\pgfpathlineto{\pgfqpoint{5.333324in}{2.806252in}}%
\pgfpathlineto{\pgfqpoint{5.325898in}{2.798360in}}%
\pgfpathlineto{\pgfqpoint{5.318464in}{2.790384in}}%
\pgfpathlineto{\pgfqpoint{5.304307in}{2.784863in}}%
\pgfpathlineto{\pgfqpoint{5.290164in}{2.779441in}}%
\pgfpathlineto{\pgfqpoint{5.276034in}{2.774119in}}%
\pgfpathlineto{\pgfqpoint{5.261917in}{2.768896in}}%
\pgfpathlineto{\pgfqpoint{5.269361in}{2.776914in}}%
\pgfpathlineto{\pgfqpoint{5.276798in}{2.784852in}}%
\pgfpathlineto{\pgfqpoint{5.284227in}{2.792712in}}%
\pgfpathlineto{\pgfqpoint{5.291650in}{2.800494in}}%
\pgfpathclose%
\pgfusepath{fill}%
\end{pgfscope}%
\begin{pgfscope}%
\pgfpathrectangle{\pgfqpoint{1.150000in}{0.150000in}}{\pgfqpoint{5.700000in}{5.700000in}}%
\pgfusepath{clip}%
\pgfsetbuttcap%
\pgfsetroundjoin%
\definecolor{currentfill}{rgb}{0.268510,0.009605,0.335427}%
\pgfsetfillcolor{currentfill}%
\pgfsetfillopacity{0.700000}%
\pgfsetlinewidth{0.000000pt}%
\definecolor{currentstroke}{rgb}{0.000000,0.000000,0.000000}%
\pgfsetstrokecolor{currentstroke}%
\pgfsetdash{}{0pt}%
\pgfpathmoveto{\pgfqpoint{3.635350in}{2.120955in}}%
\pgfpathlineto{\pgfqpoint{3.648894in}{2.115677in}}%
\pgfpathlineto{\pgfqpoint{3.662442in}{2.110517in}}%
\pgfpathlineto{\pgfqpoint{3.675994in}{2.105475in}}%
\pgfpathlineto{\pgfqpoint{3.689551in}{2.100551in}}%
\pgfpathlineto{\pgfqpoint{3.681525in}{2.093217in}}%
\pgfpathlineto{\pgfqpoint{3.673493in}{2.085955in}}%
\pgfpathlineto{\pgfqpoint{3.665454in}{2.078768in}}%
\pgfpathlineto{\pgfqpoint{3.657408in}{2.071658in}}%
\pgfpathlineto{\pgfqpoint{3.643835in}{2.076864in}}%
\pgfpathlineto{\pgfqpoint{3.630266in}{2.082188in}}%
\pgfpathlineto{\pgfqpoint{3.616701in}{2.087630in}}%
\pgfpathlineto{\pgfqpoint{3.603141in}{2.093190in}}%
\pgfpathlineto{\pgfqpoint{3.611204in}{2.100011in}}%
\pgfpathlineto{\pgfqpoint{3.619259in}{2.106913in}}%
\pgfpathlineto{\pgfqpoint{3.627308in}{2.113895in}}%
\pgfpathlineto{\pgfqpoint{3.635350in}{2.120955in}}%
\pgfpathclose%
\pgfusepath{fill}%
\end{pgfscope}%
\begin{pgfscope}%
\pgfpathrectangle{\pgfqpoint{1.150000in}{0.150000in}}{\pgfqpoint{5.700000in}{5.700000in}}%
\pgfusepath{clip}%
\pgfsetbuttcap%
\pgfsetroundjoin%
\definecolor{currentfill}{rgb}{0.208623,0.367752,0.552675}%
\pgfsetfillcolor{currentfill}%
\pgfsetfillopacity{0.700000}%
\pgfsetlinewidth{0.000000pt}%
\definecolor{currentstroke}{rgb}{0.000000,0.000000,0.000000}%
\pgfsetstrokecolor{currentstroke}%
\pgfsetdash{}{0pt}%
\pgfpathmoveto{\pgfqpoint{5.377731in}{2.851897in}}%
\pgfpathlineto{\pgfqpoint{5.391881in}{2.857402in}}%
\pgfpathlineto{\pgfqpoint{5.406044in}{2.863007in}}%
\pgfpathlineto{\pgfqpoint{5.420221in}{2.868711in}}%
\pgfpathlineto{\pgfqpoint{5.434412in}{2.874514in}}%
\pgfpathlineto{\pgfqpoint{5.427040in}{2.867180in}}%
\pgfpathlineto{\pgfqpoint{5.419661in}{2.859763in}}%
\pgfpathlineto{\pgfqpoint{5.412274in}{2.852262in}}%
\pgfpathlineto{\pgfqpoint{5.404880in}{2.844677in}}%
\pgfpathlineto{\pgfqpoint{5.390678in}{2.838806in}}%
\pgfpathlineto{\pgfqpoint{5.376490in}{2.833034in}}%
\pgfpathlineto{\pgfqpoint{5.362316in}{2.827362in}}%
\pgfpathlineto{\pgfqpoint{5.348156in}{2.821789in}}%
\pgfpathlineto{\pgfqpoint{5.355561in}{2.829435in}}%
\pgfpathlineto{\pgfqpoint{5.362958in}{2.837001in}}%
\pgfpathlineto{\pgfqpoint{5.370348in}{2.844488in}}%
\pgfpathlineto{\pgfqpoint{5.377731in}{2.851897in}}%
\pgfpathclose%
\pgfusepath{fill}%
\end{pgfscope}%
\begin{pgfscope}%
\pgfpathrectangle{\pgfqpoint{1.150000in}{0.150000in}}{\pgfqpoint{5.700000in}{5.700000in}}%
\pgfusepath{clip}%
\pgfsetbuttcap%
\pgfsetroundjoin%
\definecolor{currentfill}{rgb}{0.271305,0.019942,0.347269}%
\pgfsetfillcolor{currentfill}%
\pgfsetfillopacity{0.700000}%
\pgfsetlinewidth{0.000000pt}%
\definecolor{currentstroke}{rgb}{0.000000,0.000000,0.000000}%
\pgfsetstrokecolor{currentstroke}%
\pgfsetdash{}{0pt}%
\pgfpathmoveto{\pgfqpoint{3.494798in}{2.142008in}}%
\pgfpathlineto{\pgfqpoint{3.508328in}{2.135479in}}%
\pgfpathlineto{\pgfqpoint{3.521862in}{2.129073in}}%
\pgfpathlineto{\pgfqpoint{3.535399in}{2.122789in}}%
\pgfpathlineto{\pgfqpoint{3.548940in}{2.116628in}}%
\pgfpathlineto{\pgfqpoint{3.540851in}{2.110187in}}%
\pgfpathlineto{\pgfqpoint{3.532756in}{2.103837in}}%
\pgfpathlineto{\pgfqpoint{3.524653in}{2.097584in}}%
\pgfpathlineto{\pgfqpoint{3.516542in}{2.091428in}}%
\pgfpathlineto{\pgfqpoint{3.502982in}{2.097890in}}%
\pgfpathlineto{\pgfqpoint{3.489426in}{2.104475in}}%
\pgfpathlineto{\pgfqpoint{3.475874in}{2.111182in}}%
\pgfpathlineto{\pgfqpoint{3.462324in}{2.118012in}}%
\pgfpathlineto{\pgfqpoint{3.470454in}{2.123860in}}%
\pgfpathlineto{\pgfqpoint{3.478577in}{2.129810in}}%
\pgfpathlineto{\pgfqpoint{3.486691in}{2.135861in}}%
\pgfpathlineto{\pgfqpoint{3.494798in}{2.142008in}}%
\pgfpathclose%
\pgfusepath{fill}%
\end{pgfscope}%
\begin{pgfscope}%
\pgfpathrectangle{\pgfqpoint{1.150000in}{0.150000in}}{\pgfqpoint{5.700000in}{5.700000in}}%
\pgfusepath{clip}%
\pgfsetbuttcap%
\pgfsetroundjoin%
\definecolor{currentfill}{rgb}{0.283091,0.110553,0.431554}%
\pgfsetfillcolor{currentfill}%
\pgfsetfillopacity{0.700000}%
\pgfsetlinewidth{0.000000pt}%
\definecolor{currentstroke}{rgb}{0.000000,0.000000,0.000000}%
\pgfsetstrokecolor{currentstroke}%
\pgfsetdash{}{0pt}%
\pgfpathmoveto{\pgfqpoint{3.104642in}{2.306176in}}%
\pgfpathlineto{\pgfqpoint{3.118175in}{2.295844in}}%
\pgfpathlineto{\pgfqpoint{3.131707in}{2.285654in}}%
\pgfpathlineto{\pgfqpoint{3.145240in}{2.275606in}}%
\pgfpathlineto{\pgfqpoint{3.158774in}{2.265698in}}%
\pgfpathlineto{\pgfqpoint{3.150484in}{2.262076in}}%
\pgfpathlineto{\pgfqpoint{3.142183in}{2.258598in}}%
\pgfpathlineto{\pgfqpoint{3.133872in}{2.255268in}}%
\pgfpathlineto{\pgfqpoint{3.125551in}{2.252090in}}%
\pgfpathlineto{\pgfqpoint{3.111991in}{2.262341in}}%
\pgfpathlineto{\pgfqpoint{3.098431in}{2.272732in}}%
\pgfpathlineto{\pgfqpoint{3.084871in}{2.283265in}}%
\pgfpathlineto{\pgfqpoint{3.071311in}{2.293940in}}%
\pgfpathlineto{\pgfqpoint{3.079660in}{2.296768in}}%
\pgfpathlineto{\pgfqpoint{3.087998in}{2.299752in}}%
\pgfpathlineto{\pgfqpoint{3.096325in}{2.302889in}}%
\pgfpathlineto{\pgfqpoint{3.104642in}{2.306176in}}%
\pgfpathclose%
\pgfusepath{fill}%
\end{pgfscope}%
\begin{pgfscope}%
\pgfpathrectangle{\pgfqpoint{1.150000in}{0.150000in}}{\pgfqpoint{5.700000in}{5.700000in}}%
\pgfusepath{clip}%
\pgfsetbuttcap%
\pgfsetroundjoin%
\definecolor{currentfill}{rgb}{0.197636,0.391528,0.554969}%
\pgfsetfillcolor{currentfill}%
\pgfsetfillopacity{0.700000}%
\pgfsetlinewidth{0.000000pt}%
\definecolor{currentstroke}{rgb}{0.000000,0.000000,0.000000}%
\pgfsetstrokecolor{currentstroke}%
\pgfsetdash{}{0pt}%
\pgfpathmoveto{\pgfqpoint{5.463823in}{2.903044in}}%
\pgfpathlineto{\pgfqpoint{5.478016in}{2.908861in}}%
\pgfpathlineto{\pgfqpoint{5.492223in}{2.914776in}}%
\pgfpathlineto{\pgfqpoint{5.506444in}{2.920791in}}%
\pgfpathlineto{\pgfqpoint{5.520679in}{2.926905in}}%
\pgfpathlineto{\pgfqpoint{5.513350in}{2.919985in}}%
\pgfpathlineto{\pgfqpoint{5.506013in}{2.912982in}}%
\pgfpathlineto{\pgfqpoint{5.498668in}{2.905894in}}%
\pgfpathlineto{\pgfqpoint{5.491315in}{2.898722in}}%
\pgfpathlineto{\pgfqpoint{5.477068in}{2.892521in}}%
\pgfpathlineto{\pgfqpoint{5.462835in}{2.886419in}}%
\pgfpathlineto{\pgfqpoint{5.448616in}{2.880417in}}%
\pgfpathlineto{\pgfqpoint{5.434412in}{2.874514in}}%
\pgfpathlineto{\pgfqpoint{5.441776in}{2.881767in}}%
\pgfpathlineto{\pgfqpoint{5.449132in}{2.888938in}}%
\pgfpathlineto{\pgfqpoint{5.456482in}{2.896031in}}%
\pgfpathlineto{\pgfqpoint{5.463823in}{2.903044in}}%
\pgfpathclose%
\pgfusepath{fill}%
\end{pgfscope}%
\begin{pgfscope}%
\pgfpathrectangle{\pgfqpoint{1.150000in}{0.150000in}}{\pgfqpoint{5.700000in}{5.700000in}}%
\pgfusepath{clip}%
\pgfsetbuttcap%
\pgfsetroundjoin%
\definecolor{currentfill}{rgb}{0.267004,0.004874,0.329415}%
\pgfsetfillcolor{currentfill}%
\pgfsetfillopacity{0.700000}%
\pgfsetlinewidth{0.000000pt}%
\definecolor{currentstroke}{rgb}{0.000000,0.000000,0.000000}%
\pgfsetstrokecolor{currentstroke}%
\pgfsetdash{}{0pt}%
\pgfpathmoveto{\pgfqpoint{3.775804in}{2.113086in}}%
\pgfpathlineto{\pgfqpoint{3.789371in}{2.109005in}}%
\pgfpathlineto{\pgfqpoint{3.802944in}{2.105039in}}%
\pgfpathlineto{\pgfqpoint{3.816522in}{2.101187in}}%
\pgfpathlineto{\pgfqpoint{3.830106in}{2.097449in}}%
\pgfpathlineto{\pgfqpoint{3.822135in}{2.089336in}}%
\pgfpathlineto{\pgfqpoint{3.814159in}{2.081277in}}%
\pgfpathlineto{\pgfqpoint{3.806176in}{2.073273in}}%
\pgfpathlineto{\pgfqpoint{3.798187in}{2.065328in}}%
\pgfpathlineto{\pgfqpoint{3.784589in}{2.069330in}}%
\pgfpathlineto{\pgfqpoint{3.770997in}{2.073446in}}%
\pgfpathlineto{\pgfqpoint{3.757410in}{2.077675in}}%
\pgfpathlineto{\pgfqpoint{3.743828in}{2.082019in}}%
\pgfpathlineto{\pgfqpoint{3.751831in}{2.089694in}}%
\pgfpathlineto{\pgfqpoint{3.759829in}{2.097432in}}%
\pgfpathlineto{\pgfqpoint{3.767819in}{2.105230in}}%
\pgfpathlineto{\pgfqpoint{3.775804in}{2.113086in}}%
\pgfpathclose%
\pgfusepath{fill}%
\end{pgfscope}%
\begin{pgfscope}%
\pgfpathrectangle{\pgfqpoint{1.150000in}{0.150000in}}{\pgfqpoint{5.700000in}{5.700000in}}%
\pgfusepath{clip}%
\pgfsetbuttcap%
\pgfsetroundjoin%
\definecolor{currentfill}{rgb}{0.271305,0.019942,0.347269}%
\pgfsetfillcolor{currentfill}%
\pgfsetfillopacity{0.700000}%
\pgfsetlinewidth{0.000000pt}%
\definecolor{currentstroke}{rgb}{0.000000,0.000000,0.000000}%
\pgfsetstrokecolor{currentstroke}%
\pgfsetdash{}{0pt}%
\pgfpathmoveto{\pgfqpoint{4.002339in}{2.141913in}}%
\pgfpathlineto{\pgfqpoint{4.015954in}{2.139652in}}%
\pgfpathlineto{\pgfqpoint{4.029576in}{2.137501in}}%
\pgfpathlineto{\pgfqpoint{4.043206in}{2.135459in}}%
\pgfpathlineto{\pgfqpoint{4.056842in}{2.133526in}}%
\pgfpathlineto{\pgfqpoint{4.048954in}{2.124407in}}%
\pgfpathlineto{\pgfqpoint{4.041061in}{2.115309in}}%
\pgfpathlineto{\pgfqpoint{4.033162in}{2.106236in}}%
\pgfpathlineto{\pgfqpoint{4.025257in}{2.097189in}}%
\pgfpathlineto{\pgfqpoint{4.011610in}{2.099349in}}%
\pgfpathlineto{\pgfqpoint{3.997970in}{2.101618in}}%
\pgfpathlineto{\pgfqpoint{3.984336in}{2.103997in}}%
\pgfpathlineto{\pgfqpoint{3.970709in}{2.106485in}}%
\pgfpathlineto{\pgfqpoint{3.978625in}{2.115298in}}%
\pgfpathlineto{\pgfqpoint{3.986535in}{2.124141in}}%
\pgfpathlineto{\pgfqpoint{3.994440in}{2.133014in}}%
\pgfpathlineto{\pgfqpoint{4.002339in}{2.141913in}}%
\pgfpathclose%
\pgfusepath{fill}%
\end{pgfscope}%
\begin{pgfscope}%
\pgfpathrectangle{\pgfqpoint{1.150000in}{0.150000in}}{\pgfqpoint{5.700000in}{5.700000in}}%
\pgfusepath{clip}%
\pgfsetbuttcap%
\pgfsetroundjoin%
\definecolor{currentfill}{rgb}{0.188923,0.410910,0.556326}%
\pgfsetfillcolor{currentfill}%
\pgfsetfillopacity{0.700000}%
\pgfsetlinewidth{0.000000pt}%
\definecolor{currentstroke}{rgb}{0.000000,0.000000,0.000000}%
\pgfsetstrokecolor{currentstroke}%
\pgfsetdash{}{0pt}%
\pgfpathmoveto{\pgfqpoint{5.549920in}{2.953781in}}%
\pgfpathlineto{\pgfqpoint{5.564157in}{2.959889in}}%
\pgfpathlineto{\pgfqpoint{5.578408in}{2.966095in}}%
\pgfpathlineto{\pgfqpoint{5.592673in}{2.972400in}}%
\pgfpathlineto{\pgfqpoint{5.606953in}{2.978804in}}%
\pgfpathlineto{\pgfqpoint{5.599668in}{2.972316in}}%
\pgfpathlineto{\pgfqpoint{5.592374in}{2.965745in}}%
\pgfpathlineto{\pgfqpoint{5.585073in}{2.959091in}}%
\pgfpathlineto{\pgfqpoint{5.577764in}{2.952352in}}%
\pgfpathlineto{\pgfqpoint{5.563471in}{2.945841in}}%
\pgfpathlineto{\pgfqpoint{5.549193in}{2.939430in}}%
\pgfpathlineto{\pgfqpoint{5.534929in}{2.933118in}}%
\pgfpathlineto{\pgfqpoint{5.520679in}{2.926905in}}%
\pgfpathlineto{\pgfqpoint{5.528001in}{2.933743in}}%
\pgfpathlineto{\pgfqpoint{5.535315in}{2.940501in}}%
\pgfpathlineto{\pgfqpoint{5.542622in}{2.947180in}}%
\pgfpathlineto{\pgfqpoint{5.549920in}{2.953781in}}%
\pgfpathclose%
\pgfusepath{fill}%
\end{pgfscope}%
\begin{pgfscope}%
\pgfpathrectangle{\pgfqpoint{1.150000in}{0.150000in}}{\pgfqpoint{5.700000in}{5.700000in}}%
\pgfusepath{clip}%
\pgfsetbuttcap%
\pgfsetroundjoin%
\definecolor{currentfill}{rgb}{0.237441,0.305202,0.541921}%
\pgfsetfillcolor{currentfill}%
\pgfsetfillopacity{0.700000}%
\pgfsetlinewidth{0.000000pt}%
\definecolor{currentstroke}{rgb}{0.000000,0.000000,0.000000}%
\pgfsetstrokecolor{currentstroke}%
\pgfsetdash{}{0pt}%
\pgfpathmoveto{\pgfqpoint{2.636513in}{2.717514in}}%
\pgfpathlineto{\pgfqpoint{2.650151in}{2.701658in}}%
\pgfpathlineto{\pgfqpoint{2.663785in}{2.685985in}}%
\pgfpathlineto{\pgfqpoint{2.677413in}{2.670495in}}%
\pgfpathlineto{\pgfqpoint{2.691037in}{2.655185in}}%
\pgfpathlineto{\pgfqpoint{2.682450in}{2.655045in}}%
\pgfpathlineto{\pgfqpoint{2.673849in}{2.655103in}}%
\pgfpathlineto{\pgfqpoint{2.665233in}{2.655362in}}%
\pgfpathlineto{\pgfqpoint{2.656603in}{2.655827in}}%
\pgfpathlineto{\pgfqpoint{2.642942in}{2.671513in}}%
\pgfpathlineto{\pgfqpoint{2.629276in}{2.687379in}}%
\pgfpathlineto{\pgfqpoint{2.615604in}{2.703428in}}%
\pgfpathlineto{\pgfqpoint{2.601928in}{2.719662in}}%
\pgfpathlineto{\pgfqpoint{2.610596in}{2.718813in}}%
\pgfpathlineto{\pgfqpoint{2.619250in}{2.718175in}}%
\pgfpathlineto{\pgfqpoint{2.627888in}{2.717743in}}%
\pgfpathlineto{\pgfqpoint{2.636513in}{2.717514in}}%
\pgfpathclose%
\pgfusepath{fill}%
\end{pgfscope}%
\begin{pgfscope}%
\pgfpathrectangle{\pgfqpoint{1.150000in}{0.150000in}}{\pgfqpoint{5.700000in}{5.700000in}}%
\pgfusepath{clip}%
\pgfsetbuttcap%
\pgfsetroundjoin%
\definecolor{currentfill}{rgb}{0.180629,0.429975,0.557282}%
\pgfsetfillcolor{currentfill}%
\pgfsetfillopacity{0.700000}%
\pgfsetlinewidth{0.000000pt}%
\definecolor{currentstroke}{rgb}{0.000000,0.000000,0.000000}%
\pgfsetstrokecolor{currentstroke}%
\pgfsetdash{}{0pt}%
\pgfpathmoveto{\pgfqpoint{5.636016in}{3.003964in}}%
\pgfpathlineto{\pgfqpoint{5.650296in}{3.010342in}}%
\pgfpathlineto{\pgfqpoint{5.664591in}{3.016818in}}%
\pgfpathlineto{\pgfqpoint{5.678901in}{3.023394in}}%
\pgfpathlineto{\pgfqpoint{5.693226in}{3.030068in}}%
\pgfpathlineto{\pgfqpoint{5.685986in}{3.024026in}}%
\pgfpathlineto{\pgfqpoint{5.678739in}{3.017904in}}%
\pgfpathlineto{\pgfqpoint{5.671483in}{3.011699in}}%
\pgfpathlineto{\pgfqpoint{5.664219in}{3.005411in}}%
\pgfpathlineto{\pgfqpoint{5.649880in}{2.998610in}}%
\pgfpathlineto{\pgfqpoint{5.635556in}{2.991909in}}%
\pgfpathlineto{\pgfqpoint{5.621247in}{2.985307in}}%
\pgfpathlineto{\pgfqpoint{5.606953in}{2.978804in}}%
\pgfpathlineto{\pgfqpoint{5.614230in}{2.985211in}}%
\pgfpathlineto{\pgfqpoint{5.621500in}{2.991539in}}%
\pgfpathlineto{\pgfqpoint{5.628762in}{2.997790in}}%
\pgfpathlineto{\pgfqpoint{5.636016in}{3.003964in}}%
\pgfpathclose%
\pgfusepath{fill}%
\end{pgfscope}%
\begin{pgfscope}%
\pgfpathrectangle{\pgfqpoint{1.150000in}{0.150000in}}{\pgfqpoint{5.700000in}{5.700000in}}%
\pgfusepath{clip}%
\pgfsetbuttcap%
\pgfsetroundjoin%
\definecolor{currentfill}{rgb}{0.248629,0.278775,0.534556}%
\pgfsetfillcolor{currentfill}%
\pgfsetfillopacity{0.700000}%
\pgfsetlinewidth{0.000000pt}%
\definecolor{currentstroke}{rgb}{0.000000,0.000000,0.000000}%
\pgfsetstrokecolor{currentstroke}%
\pgfsetdash{}{0pt}%
\pgfpathmoveto{\pgfqpoint{2.691037in}{2.655185in}}%
\pgfpathlineto{\pgfqpoint{2.704657in}{2.640054in}}%
\pgfpathlineto{\pgfqpoint{2.718272in}{2.625100in}}%
\pgfpathlineto{\pgfqpoint{2.731884in}{2.610321in}}%
\pgfpathlineto{\pgfqpoint{2.745491in}{2.595717in}}%
\pgfpathlineto{\pgfqpoint{2.736940in}{2.595210in}}%
\pgfpathlineto{\pgfqpoint{2.728375in}{2.594897in}}%
\pgfpathlineto{\pgfqpoint{2.719796in}{2.594779in}}%
\pgfpathlineto{\pgfqpoint{2.711204in}{2.594862in}}%
\pgfpathlineto{\pgfqpoint{2.697560in}{2.609840in}}%
\pgfpathlineto{\pgfqpoint{2.683912in}{2.624992in}}%
\pgfpathlineto{\pgfqpoint{2.670260in}{2.640321in}}%
\pgfpathlineto{\pgfqpoint{2.656603in}{2.655827in}}%
\pgfpathlineto{\pgfqpoint{2.665233in}{2.655362in}}%
\pgfpathlineto{\pgfqpoint{2.673849in}{2.655103in}}%
\pgfpathlineto{\pgfqpoint{2.682450in}{2.655045in}}%
\pgfpathlineto{\pgfqpoint{2.691037in}{2.655185in}}%
\pgfpathclose%
\pgfusepath{fill}%
\end{pgfscope}%
\begin{pgfscope}%
\pgfpathrectangle{\pgfqpoint{1.150000in}{0.150000in}}{\pgfqpoint{5.700000in}{5.700000in}}%
\pgfusepath{clip}%
\pgfsetbuttcap%
\pgfsetroundjoin%
\definecolor{currentfill}{rgb}{0.225863,0.330805,0.547314}%
\pgfsetfillcolor{currentfill}%
\pgfsetfillopacity{0.700000}%
\pgfsetlinewidth{0.000000pt}%
\definecolor{currentstroke}{rgb}{0.000000,0.000000,0.000000}%
\pgfsetstrokecolor{currentstroke}%
\pgfsetdash{}{0pt}%
\pgfpathmoveto{\pgfqpoint{2.581906in}{2.782810in}}%
\pgfpathlineto{\pgfqpoint{2.595566in}{2.766202in}}%
\pgfpathlineto{\pgfqpoint{2.609220in}{2.749784in}}%
\pgfpathlineto{\pgfqpoint{2.622869in}{2.733556in}}%
\pgfpathlineto{\pgfqpoint{2.636513in}{2.717514in}}%
\pgfpathlineto{\pgfqpoint{2.627888in}{2.717743in}}%
\pgfpathlineto{\pgfqpoint{2.619250in}{2.718175in}}%
\pgfpathlineto{\pgfqpoint{2.610596in}{2.718813in}}%
\pgfpathlineto{\pgfqpoint{2.601928in}{2.719662in}}%
\pgfpathlineto{\pgfqpoint{2.588246in}{2.736082in}}%
\pgfpathlineto{\pgfqpoint{2.574558in}{2.752689in}}%
\pgfpathlineto{\pgfqpoint{2.560864in}{2.769486in}}%
\pgfpathlineto{\pgfqpoint{2.547165in}{2.786474in}}%
\pgfpathlineto{\pgfqpoint{2.555873in}{2.785238in}}%
\pgfpathlineto{\pgfqpoint{2.564566in}{2.784218in}}%
\pgfpathlineto{\pgfqpoint{2.573243in}{2.783410in}}%
\pgfpathlineto{\pgfqpoint{2.581906in}{2.782810in}}%
\pgfpathclose%
\pgfusepath{fill}%
\end{pgfscope}%
\begin{pgfscope}%
\pgfpathrectangle{\pgfqpoint{1.150000in}{0.150000in}}{\pgfqpoint{5.700000in}{5.700000in}}%
\pgfusepath{clip}%
\pgfsetbuttcap%
\pgfsetroundjoin%
\definecolor{currentfill}{rgb}{0.276022,0.044167,0.370164}%
\pgfsetfillcolor{currentfill}%
\pgfsetfillopacity{0.700000}%
\pgfsetlinewidth{0.000000pt}%
\definecolor{currentstroke}{rgb}{0.000000,0.000000,0.000000}%
\pgfsetstrokecolor{currentstroke}%
\pgfsetdash{}{0pt}%
\pgfpathmoveto{\pgfqpoint{3.354030in}{2.177172in}}%
\pgfpathlineto{\pgfqpoint{3.367558in}{2.169332in}}%
\pgfpathlineto{\pgfqpoint{3.381088in}{2.161620in}}%
\pgfpathlineto{\pgfqpoint{3.394620in}{2.154036in}}%
\pgfpathlineto{\pgfqpoint{3.408155in}{2.146580in}}%
\pgfpathlineto{\pgfqpoint{3.399996in}{2.141151in}}%
\pgfpathlineto{\pgfqpoint{3.391829in}{2.135835in}}%
\pgfpathlineto{\pgfqpoint{3.383653in}{2.130636in}}%
\pgfpathlineto{\pgfqpoint{3.375469in}{2.125557in}}%
\pgfpathlineto{\pgfqpoint{3.361912in}{2.133334in}}%
\pgfpathlineto{\pgfqpoint{3.348357in}{2.141238in}}%
\pgfpathlineto{\pgfqpoint{3.334805in}{2.149271in}}%
\pgfpathlineto{\pgfqpoint{3.321254in}{2.157432in}}%
\pgfpathlineto{\pgfqpoint{3.329461in}{2.162184in}}%
\pgfpathlineto{\pgfqpoint{3.337660in}{2.167060in}}%
\pgfpathlineto{\pgfqpoint{3.345849in}{2.172057in}}%
\pgfpathlineto{\pgfqpoint{3.354030in}{2.177172in}}%
\pgfpathclose%
\pgfusepath{fill}%
\end{pgfscope}%
\begin{pgfscope}%
\pgfpathrectangle{\pgfqpoint{1.150000in}{0.150000in}}{\pgfqpoint{5.700000in}{5.700000in}}%
\pgfusepath{clip}%
\pgfsetbuttcap%
\pgfsetroundjoin%
\definecolor{currentfill}{rgb}{0.172719,0.448791,0.557885}%
\pgfsetfillcolor{currentfill}%
\pgfsetfillopacity{0.700000}%
\pgfsetlinewidth{0.000000pt}%
\definecolor{currentstroke}{rgb}{0.000000,0.000000,0.000000}%
\pgfsetstrokecolor{currentstroke}%
\pgfsetdash{}{0pt}%
\pgfpathmoveto{\pgfqpoint{5.722103in}{3.053461in}}%
\pgfpathlineto{\pgfqpoint{5.736428in}{3.060089in}}%
\pgfpathlineto{\pgfqpoint{5.750767in}{3.066816in}}%
\pgfpathlineto{\pgfqpoint{5.765121in}{3.073642in}}%
\pgfpathlineto{\pgfqpoint{5.779490in}{3.080566in}}%
\pgfpathlineto{\pgfqpoint{5.772298in}{3.074982in}}%
\pgfpathlineto{\pgfqpoint{5.765098in}{3.069320in}}%
\pgfpathlineto{\pgfqpoint{5.757890in}{3.063578in}}%
\pgfpathlineto{\pgfqpoint{5.750674in}{3.057754in}}%
\pgfpathlineto{\pgfqpoint{5.736289in}{3.050684in}}%
\pgfpathlineto{\pgfqpoint{5.721919in}{3.043713in}}%
\pgfpathlineto{\pgfqpoint{5.707565in}{3.036841in}}%
\pgfpathlineto{\pgfqpoint{5.693226in}{3.030068in}}%
\pgfpathlineto{\pgfqpoint{5.700457in}{3.036031in}}%
\pgfpathlineto{\pgfqpoint{5.707680in}{3.041916in}}%
\pgfpathlineto{\pgfqpoint{5.714896in}{3.047725in}}%
\pgfpathlineto{\pgfqpoint{5.722103in}{3.053461in}}%
\pgfpathclose%
\pgfusepath{fill}%
\end{pgfscope}%
\begin{pgfscope}%
\pgfpathrectangle{\pgfqpoint{1.150000in}{0.150000in}}{\pgfqpoint{5.700000in}{5.700000in}}%
\pgfusepath{clip}%
\pgfsetbuttcap%
\pgfsetroundjoin%
\definecolor{currentfill}{rgb}{0.258965,0.251537,0.524736}%
\pgfsetfillcolor{currentfill}%
\pgfsetfillopacity{0.700000}%
\pgfsetlinewidth{0.000000pt}%
\definecolor{currentstroke}{rgb}{0.000000,0.000000,0.000000}%
\pgfsetstrokecolor{currentstroke}%
\pgfsetdash{}{0pt}%
\pgfpathmoveto{\pgfqpoint{2.745491in}{2.595717in}}%
\pgfpathlineto{\pgfqpoint{2.759095in}{2.581286in}}%
\pgfpathlineto{\pgfqpoint{2.772695in}{2.567025in}}%
\pgfpathlineto{\pgfqpoint{2.786292in}{2.552935in}}%
\pgfpathlineto{\pgfqpoint{2.799885in}{2.539013in}}%
\pgfpathlineto{\pgfqpoint{2.791369in}{2.538142in}}%
\pgfpathlineto{\pgfqpoint{2.782839in}{2.537458in}}%
\pgfpathlineto{\pgfqpoint{2.774296in}{2.536967in}}%
\pgfpathlineto{\pgfqpoint{2.765739in}{2.536670in}}%
\pgfpathlineto{\pgfqpoint{2.752111in}{2.550963in}}%
\pgfpathlineto{\pgfqpoint{2.738479in}{2.565425in}}%
\pgfpathlineto{\pgfqpoint{2.724843in}{2.580058in}}%
\pgfpathlineto{\pgfqpoint{2.711204in}{2.594862in}}%
\pgfpathlineto{\pgfqpoint{2.719796in}{2.594779in}}%
\pgfpathlineto{\pgfqpoint{2.728375in}{2.594897in}}%
\pgfpathlineto{\pgfqpoint{2.736940in}{2.595210in}}%
\pgfpathlineto{\pgfqpoint{2.745491in}{2.595717in}}%
\pgfpathclose%
\pgfusepath{fill}%
\end{pgfscope}%
\begin{pgfscope}%
\pgfpathrectangle{\pgfqpoint{1.150000in}{0.150000in}}{\pgfqpoint{5.700000in}{5.700000in}}%
\pgfusepath{clip}%
\pgfsetbuttcap%
\pgfsetroundjoin%
\definecolor{currentfill}{rgb}{0.212395,0.359683,0.551710}%
\pgfsetfillcolor{currentfill}%
\pgfsetfillopacity{0.700000}%
\pgfsetlinewidth{0.000000pt}%
\definecolor{currentstroke}{rgb}{0.000000,0.000000,0.000000}%
\pgfsetstrokecolor{currentstroke}%
\pgfsetdash{}{0pt}%
\pgfpathmoveto{\pgfqpoint{2.527204in}{2.851185in}}%
\pgfpathlineto{\pgfqpoint{2.540889in}{2.833796in}}%
\pgfpathlineto{\pgfqpoint{2.554568in}{2.816605in}}%
\pgfpathlineto{\pgfqpoint{2.568240in}{2.799610in}}%
\pgfpathlineto{\pgfqpoint{2.581906in}{2.782810in}}%
\pgfpathlineto{\pgfqpoint{2.573243in}{2.783410in}}%
\pgfpathlineto{\pgfqpoint{2.564566in}{2.784218in}}%
\pgfpathlineto{\pgfqpoint{2.555873in}{2.785238in}}%
\pgfpathlineto{\pgfqpoint{2.547165in}{2.786474in}}%
\pgfpathlineto{\pgfqpoint{2.533459in}{2.803655in}}%
\pgfpathlineto{\pgfqpoint{2.519747in}{2.821031in}}%
\pgfpathlineto{\pgfqpoint{2.506028in}{2.838603in}}%
\pgfpathlineto{\pgfqpoint{2.492302in}{2.856375in}}%
\pgfpathlineto{\pgfqpoint{2.501052in}{2.854750in}}%
\pgfpathlineto{\pgfqpoint{2.509785in}{2.853345in}}%
\pgfpathlineto{\pgfqpoint{2.518502in}{2.852158in}}%
\pgfpathlineto{\pgfqpoint{2.527204in}{2.851185in}}%
\pgfpathclose%
\pgfusepath{fill}%
\end{pgfscope}%
\begin{pgfscope}%
\pgfpathrectangle{\pgfqpoint{1.150000in}{0.150000in}}{\pgfqpoint{5.700000in}{5.700000in}}%
\pgfusepath{clip}%
\pgfsetbuttcap%
\pgfsetroundjoin%
\definecolor{currentfill}{rgb}{0.282290,0.145912,0.461510}%
\pgfsetfillcolor{currentfill}%
\pgfsetfillopacity{0.700000}%
\pgfsetlinewidth{0.000000pt}%
\definecolor{currentstroke}{rgb}{0.000000,0.000000,0.000000}%
\pgfsetstrokecolor{currentstroke}%
\pgfsetdash{}{0pt}%
\pgfpathmoveto{\pgfqpoint{4.572966in}{2.363863in}}%
\pgfpathlineto{\pgfqpoint{4.586768in}{2.365558in}}%
\pgfpathlineto{\pgfqpoint{4.600580in}{2.367356in}}%
\pgfpathlineto{\pgfqpoint{4.614403in}{2.369257in}}%
\pgfpathlineto{\pgfqpoint{4.628236in}{2.371260in}}%
\pgfpathlineto{\pgfqpoint{4.620533in}{2.361310in}}%
\pgfpathlineto{\pgfqpoint{4.612825in}{2.351314in}}%
\pgfpathlineto{\pgfqpoint{4.605111in}{2.341276in}}%
\pgfpathlineto{\pgfqpoint{4.597393in}{2.331195in}}%
\pgfpathlineto{\pgfqpoint{4.583553in}{2.329311in}}%
\pgfpathlineto{\pgfqpoint{4.569724in}{2.327530in}}%
\pgfpathlineto{\pgfqpoint{4.555905in}{2.325851in}}%
\pgfpathlineto{\pgfqpoint{4.542095in}{2.324275in}}%
\pgfpathlineto{\pgfqpoint{4.549821in}{2.334230in}}%
\pgfpathlineto{\pgfqpoint{4.557541in}{2.344147in}}%
\pgfpathlineto{\pgfqpoint{4.565256in}{2.354025in}}%
\pgfpathlineto{\pgfqpoint{4.572966in}{2.363863in}}%
\pgfpathclose%
\pgfusepath{fill}%
\end{pgfscope}%
\begin{pgfscope}%
\pgfpathrectangle{\pgfqpoint{1.150000in}{0.150000in}}{\pgfqpoint{5.700000in}{5.700000in}}%
\pgfusepath{clip}%
\pgfsetbuttcap%
\pgfsetroundjoin%
\definecolor{currentfill}{rgb}{0.283187,0.125848,0.444960}%
\pgfsetfillcolor{currentfill}%
\pgfsetfillopacity{0.700000}%
\pgfsetlinewidth{0.000000pt}%
\definecolor{currentstroke}{rgb}{0.000000,0.000000,0.000000}%
\pgfsetstrokecolor{currentstroke}%
\pgfsetdash{}{0pt}%
\pgfpathmoveto{\pgfqpoint{4.486958in}{2.319005in}}%
\pgfpathlineto{\pgfqpoint{4.500728in}{2.320168in}}%
\pgfpathlineto{\pgfqpoint{4.514507in}{2.321433in}}%
\pgfpathlineto{\pgfqpoint{4.528296in}{2.322803in}}%
\pgfpathlineto{\pgfqpoint{4.542095in}{2.324275in}}%
\pgfpathlineto{\pgfqpoint{4.534364in}{2.314284in}}%
\pgfpathlineto{\pgfqpoint{4.526628in}{2.304257in}}%
\pgfpathlineto{\pgfqpoint{4.518886in}{2.294195in}}%
\pgfpathlineto{\pgfqpoint{4.511139in}{2.284099in}}%
\pgfpathlineto{\pgfqpoint{4.497333in}{2.282764in}}%
\pgfpathlineto{\pgfqpoint{4.483537in}{2.281532in}}%
\pgfpathlineto{\pgfqpoint{4.469750in}{2.280404in}}%
\pgfpathlineto{\pgfqpoint{4.455973in}{2.279379in}}%
\pgfpathlineto{\pgfqpoint{4.463727in}{2.289331in}}%
\pgfpathlineto{\pgfqpoint{4.471476in}{2.299253in}}%
\pgfpathlineto{\pgfqpoint{4.479220in}{2.309145in}}%
\pgfpathlineto{\pgfqpoint{4.486958in}{2.319005in}}%
\pgfpathclose%
\pgfusepath{fill}%
\end{pgfscope}%
\begin{pgfscope}%
\pgfpathrectangle{\pgfqpoint{1.150000in}{0.150000in}}{\pgfqpoint{5.700000in}{5.700000in}}%
\pgfusepath{clip}%
\pgfsetbuttcap%
\pgfsetroundjoin%
\definecolor{currentfill}{rgb}{0.279574,0.170599,0.479997}%
\pgfsetfillcolor{currentfill}%
\pgfsetfillopacity{0.700000}%
\pgfsetlinewidth{0.000000pt}%
\definecolor{currentstroke}{rgb}{0.000000,0.000000,0.000000}%
\pgfsetstrokecolor{currentstroke}%
\pgfsetdash{}{0pt}%
\pgfpathmoveto{\pgfqpoint{4.658991in}{2.410605in}}%
\pgfpathlineto{\pgfqpoint{4.672827in}{2.412812in}}%
\pgfpathlineto{\pgfqpoint{4.686674in}{2.415122in}}%
\pgfpathlineto{\pgfqpoint{4.700532in}{2.417533in}}%
\pgfpathlineto{\pgfqpoint{4.714400in}{2.420047in}}%
\pgfpathlineto{\pgfqpoint{4.706726in}{2.410187in}}%
\pgfpathlineto{\pgfqpoint{4.699047in}{2.400275in}}%
\pgfpathlineto{\pgfqpoint{4.691362in}{2.390312in}}%
\pgfpathlineto{\pgfqpoint{4.683672in}{2.380298in}}%
\pgfpathlineto{\pgfqpoint{4.669797in}{2.377885in}}%
\pgfpathlineto{\pgfqpoint{4.655933in}{2.375575in}}%
\pgfpathlineto{\pgfqpoint{4.642079in}{2.373366in}}%
\pgfpathlineto{\pgfqpoint{4.628236in}{2.371260in}}%
\pgfpathlineto{\pgfqpoint{4.635933in}{2.381166in}}%
\pgfpathlineto{\pgfqpoint{4.643624in}{2.391026in}}%
\pgfpathlineto{\pgfqpoint{4.651310in}{2.400839in}}%
\pgfpathlineto{\pgfqpoint{4.658991in}{2.410605in}}%
\pgfpathclose%
\pgfusepath{fill}%
\end{pgfscope}%
\begin{pgfscope}%
\pgfpathrectangle{\pgfqpoint{1.150000in}{0.150000in}}{\pgfqpoint{5.700000in}{5.700000in}}%
\pgfusepath{clip}%
\pgfsetbuttcap%
\pgfsetroundjoin%
\definecolor{currentfill}{rgb}{0.165117,0.467423,0.558141}%
\pgfsetfillcolor{currentfill}%
\pgfsetfillopacity{0.700000}%
\pgfsetlinewidth{0.000000pt}%
\definecolor{currentstroke}{rgb}{0.000000,0.000000,0.000000}%
\pgfsetstrokecolor{currentstroke}%
\pgfsetdash{}{0pt}%
\pgfpathmoveto{\pgfqpoint{5.808176in}{3.102153in}}%
\pgfpathlineto{\pgfqpoint{5.822543in}{3.109012in}}%
\pgfpathlineto{\pgfqpoint{5.836926in}{3.115969in}}%
\pgfpathlineto{\pgfqpoint{5.851325in}{3.123024in}}%
\pgfpathlineto{\pgfqpoint{5.865738in}{3.130178in}}%
\pgfpathlineto{\pgfqpoint{5.858596in}{3.125061in}}%
\pgfpathlineto{\pgfqpoint{5.851446in}{3.119869in}}%
\pgfpathlineto{\pgfqpoint{5.844287in}{3.114599in}}%
\pgfpathlineto{\pgfqpoint{5.837120in}{3.109251in}}%
\pgfpathlineto{\pgfqpoint{5.822689in}{3.101931in}}%
\pgfpathlineto{\pgfqpoint{5.808274in}{3.094711in}}%
\pgfpathlineto{\pgfqpoint{5.793874in}{3.087589in}}%
\pgfpathlineto{\pgfqpoint{5.779490in}{3.080566in}}%
\pgfpathlineto{\pgfqpoint{5.786673in}{3.086073in}}%
\pgfpathlineto{\pgfqpoint{5.793849in}{3.091505in}}%
\pgfpathlineto{\pgfqpoint{5.801016in}{3.096865in}}%
\pgfpathlineto{\pgfqpoint{5.808176in}{3.102153in}}%
\pgfpathclose%
\pgfusepath{fill}%
\end{pgfscope}%
\begin{pgfscope}%
\pgfpathrectangle{\pgfqpoint{1.150000in}{0.150000in}}{\pgfqpoint{5.700000in}{5.700000in}}%
\pgfusepath{clip}%
\pgfsetbuttcap%
\pgfsetroundjoin%
\definecolor{currentfill}{rgb}{0.282910,0.105393,0.426902}%
\pgfsetfillcolor{currentfill}%
\pgfsetfillopacity{0.700000}%
\pgfsetlinewidth{0.000000pt}%
\definecolor{currentstroke}{rgb}{0.000000,0.000000,0.000000}%
\pgfsetstrokecolor{currentstroke}%
\pgfsetdash{}{0pt}%
\pgfpathmoveto{\pgfqpoint{4.400959in}{2.276320in}}%
\pgfpathlineto{\pgfqpoint{4.414699in}{2.276929in}}%
\pgfpathlineto{\pgfqpoint{4.428447in}{2.277642in}}%
\pgfpathlineto{\pgfqpoint{4.442206in}{2.278459in}}%
\pgfpathlineto{\pgfqpoint{4.455973in}{2.279379in}}%
\pgfpathlineto{\pgfqpoint{4.448214in}{2.269400in}}%
\pgfpathlineto{\pgfqpoint{4.440449in}{2.259395in}}%
\pgfpathlineto{\pgfqpoint{4.432679in}{2.249364in}}%
\pgfpathlineto{\pgfqpoint{4.424904in}{2.239309in}}%
\pgfpathlineto{\pgfqpoint{4.411129in}{2.238544in}}%
\pgfpathlineto{\pgfqpoint{4.397364in}{2.237883in}}%
\pgfpathlineto{\pgfqpoint{4.383607in}{2.237325in}}%
\pgfpathlineto{\pgfqpoint{4.369860in}{2.236873in}}%
\pgfpathlineto{\pgfqpoint{4.377643in}{2.246765in}}%
\pgfpathlineto{\pgfqpoint{4.385420in}{2.256638in}}%
\pgfpathlineto{\pgfqpoint{4.393192in}{2.266490in}}%
\pgfpathlineto{\pgfqpoint{4.400959in}{2.276320in}}%
\pgfpathclose%
\pgfusepath{fill}%
\end{pgfscope}%
\begin{pgfscope}%
\pgfpathrectangle{\pgfqpoint{1.150000in}{0.150000in}}{\pgfqpoint{5.700000in}{5.700000in}}%
\pgfusepath{clip}%
\pgfsetbuttcap%
\pgfsetroundjoin%
\definecolor{currentfill}{rgb}{0.282327,0.094955,0.417331}%
\pgfsetfillcolor{currentfill}%
\pgfsetfillopacity{0.700000}%
\pgfsetlinewidth{0.000000pt}%
\definecolor{currentstroke}{rgb}{0.000000,0.000000,0.000000}%
\pgfsetstrokecolor{currentstroke}%
\pgfsetdash{}{0pt}%
\pgfpathmoveto{\pgfqpoint{3.158774in}{2.265698in}}%
\pgfpathlineto{\pgfqpoint{3.172308in}{2.255930in}}%
\pgfpathlineto{\pgfqpoint{3.185843in}{2.246300in}}%
\pgfpathlineto{\pgfqpoint{3.199378in}{2.236808in}}%
\pgfpathlineto{\pgfqpoint{3.212915in}{2.227452in}}%
\pgfpathlineto{\pgfqpoint{3.204650in}{2.223495in}}%
\pgfpathlineto{\pgfqpoint{3.196376in}{2.219677in}}%
\pgfpathlineto{\pgfqpoint{3.188091in}{2.216003in}}%
\pgfpathlineto{\pgfqpoint{3.179797in}{2.212476in}}%
\pgfpathlineto{\pgfqpoint{3.166235in}{2.222173in}}%
\pgfpathlineto{\pgfqpoint{3.152673in}{2.232007in}}%
\pgfpathlineto{\pgfqpoint{3.139112in}{2.241979in}}%
\pgfpathlineto{\pgfqpoint{3.125551in}{2.252090in}}%
\pgfpathlineto{\pgfqpoint{3.133872in}{2.255268in}}%
\pgfpathlineto{\pgfqpoint{3.142183in}{2.258598in}}%
\pgfpathlineto{\pgfqpoint{3.150484in}{2.262076in}}%
\pgfpathlineto{\pgfqpoint{3.158774in}{2.265698in}}%
\pgfpathclose%
\pgfusepath{fill}%
\end{pgfscope}%
\begin{pgfscope}%
\pgfpathrectangle{\pgfqpoint{1.150000in}{0.150000in}}{\pgfqpoint{5.700000in}{5.700000in}}%
\pgfusepath{clip}%
\pgfsetbuttcap%
\pgfsetroundjoin%
\definecolor{currentfill}{rgb}{0.275191,0.194905,0.496005}%
\pgfsetfillcolor{currentfill}%
\pgfsetfillopacity{0.700000}%
\pgfsetlinewidth{0.000000pt}%
\definecolor{currentstroke}{rgb}{0.000000,0.000000,0.000000}%
\pgfsetstrokecolor{currentstroke}%
\pgfsetdash{}{0pt}%
\pgfpathmoveto{\pgfqpoint{4.745038in}{2.458955in}}%
\pgfpathlineto{\pgfqpoint{4.758910in}{2.461654in}}%
\pgfpathlineto{\pgfqpoint{4.772794in}{2.464454in}}%
\pgfpathlineto{\pgfqpoint{4.786688in}{2.467357in}}%
\pgfpathlineto{\pgfqpoint{4.800594in}{2.470361in}}%
\pgfpathlineto{\pgfqpoint{4.792950in}{2.460639in}}%
\pgfpathlineto{\pgfqpoint{4.785300in}{2.450858in}}%
\pgfpathlineto{\pgfqpoint{4.777645in}{2.441018in}}%
\pgfpathlineto{\pgfqpoint{4.769983in}{2.431121in}}%
\pgfpathlineto{\pgfqpoint{4.756071in}{2.428200in}}%
\pgfpathlineto{\pgfqpoint{4.742170in}{2.425381in}}%
\pgfpathlineto{\pgfqpoint{4.728279in}{2.422663in}}%
\pgfpathlineto{\pgfqpoint{4.714400in}{2.420047in}}%
\pgfpathlineto{\pgfqpoint{4.722068in}{2.429855in}}%
\pgfpathlineto{\pgfqpoint{4.729730in}{2.439609in}}%
\pgfpathlineto{\pgfqpoint{4.737387in}{2.449309in}}%
\pgfpathlineto{\pgfqpoint{4.745038in}{2.458955in}}%
\pgfpathclose%
\pgfusepath{fill}%
\end{pgfscope}%
\begin{pgfscope}%
\pgfpathrectangle{\pgfqpoint{1.150000in}{0.150000in}}{\pgfqpoint{5.700000in}{5.700000in}}%
\pgfusepath{clip}%
\pgfsetbuttcap%
\pgfsetroundjoin%
\definecolor{currentfill}{rgb}{0.157729,0.485932,0.558013}%
\pgfsetfillcolor{currentfill}%
\pgfsetfillopacity{0.700000}%
\pgfsetlinewidth{0.000000pt}%
\definecolor{currentstroke}{rgb}{0.000000,0.000000,0.000000}%
\pgfsetstrokecolor{currentstroke}%
\pgfsetdash{}{0pt}%
\pgfpathmoveto{\pgfqpoint{5.894225in}{3.149934in}}%
\pgfpathlineto{\pgfqpoint{5.908636in}{3.157003in}}%
\pgfpathlineto{\pgfqpoint{5.923063in}{3.164170in}}%
\pgfpathlineto{\pgfqpoint{5.937505in}{3.171435in}}%
\pgfpathlineto{\pgfqpoint{5.951963in}{3.178798in}}%
\pgfpathlineto{\pgfqpoint{5.944872in}{3.174153in}}%
\pgfpathlineto{\pgfqpoint{5.937773in}{3.169437in}}%
\pgfpathlineto{\pgfqpoint{5.930665in}{3.164647in}}%
\pgfpathlineto{\pgfqpoint{5.923549in}{3.159781in}}%
\pgfpathlineto{\pgfqpoint{5.909073in}{3.152232in}}%
\pgfpathlineto{\pgfqpoint{5.894612in}{3.144782in}}%
\pgfpathlineto{\pgfqpoint{5.880168in}{3.137431in}}%
\pgfpathlineto{\pgfqpoint{5.865738in}{3.130178in}}%
\pgfpathlineto{\pgfqpoint{5.872872in}{3.135222in}}%
\pgfpathlineto{\pgfqpoint{5.879998in}{3.140194in}}%
\pgfpathlineto{\pgfqpoint{5.887116in}{3.145098in}}%
\pgfpathlineto{\pgfqpoint{5.894225in}{3.149934in}}%
\pgfpathclose%
\pgfusepath{fill}%
\end{pgfscope}%
\begin{pgfscope}%
\pgfpathrectangle{\pgfqpoint{1.150000in}{0.150000in}}{\pgfqpoint{5.700000in}{5.700000in}}%
\pgfusepath{clip}%
\pgfsetbuttcap%
\pgfsetroundjoin%
\definecolor{currentfill}{rgb}{0.266580,0.228262,0.514349}%
\pgfsetfillcolor{currentfill}%
\pgfsetfillopacity{0.700000}%
\pgfsetlinewidth{0.000000pt}%
\definecolor{currentstroke}{rgb}{0.000000,0.000000,0.000000}%
\pgfsetstrokecolor{currentstroke}%
\pgfsetdash{}{0pt}%
\pgfpathmoveto{\pgfqpoint{2.799885in}{2.539013in}}%
\pgfpathlineto{\pgfqpoint{2.813476in}{2.525258in}}%
\pgfpathlineto{\pgfqpoint{2.827063in}{2.511668in}}%
\pgfpathlineto{\pgfqpoint{2.840648in}{2.498243in}}%
\pgfpathlineto{\pgfqpoint{2.854230in}{2.484981in}}%
\pgfpathlineto{\pgfqpoint{2.845747in}{2.483747in}}%
\pgfpathlineto{\pgfqpoint{2.837251in}{2.482696in}}%
\pgfpathlineto{\pgfqpoint{2.828743in}{2.481831in}}%
\pgfpathlineto{\pgfqpoint{2.820221in}{2.481157in}}%
\pgfpathlineto{\pgfqpoint{2.806605in}{2.494789in}}%
\pgfpathlineto{\pgfqpoint{2.792986in}{2.508584in}}%
\pgfpathlineto{\pgfqpoint{2.779364in}{2.522544in}}%
\pgfpathlineto{\pgfqpoint{2.765739in}{2.536670in}}%
\pgfpathlineto{\pgfqpoint{2.774296in}{2.536967in}}%
\pgfpathlineto{\pgfqpoint{2.782839in}{2.537458in}}%
\pgfpathlineto{\pgfqpoint{2.791369in}{2.538142in}}%
\pgfpathlineto{\pgfqpoint{2.799885in}{2.539013in}}%
\pgfpathclose%
\pgfusepath{fill}%
\end{pgfscope}%
\begin{pgfscope}%
\pgfpathrectangle{\pgfqpoint{1.150000in}{0.150000in}}{\pgfqpoint{5.700000in}{5.700000in}}%
\pgfusepath{clip}%
\pgfsetbuttcap%
\pgfsetroundjoin%
\definecolor{currentfill}{rgb}{0.269944,0.014625,0.341379}%
\pgfsetfillcolor{currentfill}%
\pgfsetfillopacity{0.700000}%
\pgfsetlinewidth{0.000000pt}%
\definecolor{currentstroke}{rgb}{0.000000,0.000000,0.000000}%
\pgfsetstrokecolor{currentstroke}%
\pgfsetdash{}{0pt}%
\pgfpathmoveto{\pgfqpoint{3.916268in}{2.117543in}}%
\pgfpathlineto{\pgfqpoint{3.929869in}{2.114612in}}%
\pgfpathlineto{\pgfqpoint{3.943476in}{2.111792in}}%
\pgfpathlineto{\pgfqpoint{3.957089in}{2.109083in}}%
\pgfpathlineto{\pgfqpoint{3.970709in}{2.106485in}}%
\pgfpathlineto{\pgfqpoint{3.962788in}{2.097705in}}%
\pgfpathlineto{\pgfqpoint{3.954861in}{2.088960in}}%
\pgfpathlineto{\pgfqpoint{3.946928in}{2.080252in}}%
\pgfpathlineto{\pgfqpoint{3.938990in}{2.071584in}}%
\pgfpathlineto{\pgfqpoint{3.925357in}{2.074428in}}%
\pgfpathlineto{\pgfqpoint{3.911732in}{2.077382in}}%
\pgfpathlineto{\pgfqpoint{3.898112in}{2.080447in}}%
\pgfpathlineto{\pgfqpoint{3.884499in}{2.083624in}}%
\pgfpathlineto{\pgfqpoint{3.892450in}{2.092040in}}%
\pgfpathlineto{\pgfqpoint{3.900395in}{2.100500in}}%
\pgfpathlineto{\pgfqpoint{3.908334in}{2.109001in}}%
\pgfpathlineto{\pgfqpoint{3.916268in}{2.117543in}}%
\pgfpathclose%
\pgfusepath{fill}%
\end{pgfscope}%
\begin{pgfscope}%
\pgfpathrectangle{\pgfqpoint{1.150000in}{0.150000in}}{\pgfqpoint{5.700000in}{5.700000in}}%
\pgfusepath{clip}%
\pgfsetbuttcap%
\pgfsetroundjoin%
\definecolor{currentfill}{rgb}{0.150476,0.504369,0.557430}%
\pgfsetfillcolor{currentfill}%
\pgfsetfillopacity{0.700000}%
\pgfsetlinewidth{0.000000pt}%
\definecolor{currentstroke}{rgb}{0.000000,0.000000,0.000000}%
\pgfsetstrokecolor{currentstroke}%
\pgfsetdash{}{0pt}%
\pgfpathmoveto{\pgfqpoint{5.980244in}{3.196710in}}%
\pgfpathlineto{\pgfqpoint{5.994698in}{3.203968in}}%
\pgfpathlineto{\pgfqpoint{6.009168in}{3.211325in}}%
\pgfpathlineto{\pgfqpoint{6.023654in}{3.218779in}}%
\pgfpathlineto{\pgfqpoint{6.038156in}{3.226332in}}%
\pgfpathlineto{\pgfqpoint{6.031118in}{3.222161in}}%
\pgfpathlineto{\pgfqpoint{6.024071in}{3.217924in}}%
\pgfpathlineto{\pgfqpoint{6.017017in}{3.213617in}}%
\pgfpathlineto{\pgfqpoint{6.009953in}{3.209239in}}%
\pgfpathlineto{\pgfqpoint{5.995432in}{3.201481in}}%
\pgfpathlineto{\pgfqpoint{5.980926in}{3.193821in}}%
\pgfpathlineto{\pgfqpoint{5.966437in}{3.186261in}}%
\pgfpathlineto{\pgfqpoint{5.951963in}{3.178798in}}%
\pgfpathlineto{\pgfqpoint{5.959046in}{3.183374in}}%
\pgfpathlineto{\pgfqpoint{5.966120in}{3.187883in}}%
\pgfpathlineto{\pgfqpoint{5.973186in}{3.192328in}}%
\pgfpathlineto{\pgfqpoint{5.980244in}{3.196710in}}%
\pgfpathclose%
\pgfusepath{fill}%
\end{pgfscope}%
\begin{pgfscope}%
\pgfpathrectangle{\pgfqpoint{1.150000in}{0.150000in}}{\pgfqpoint{5.700000in}{5.700000in}}%
\pgfusepath{clip}%
\pgfsetbuttcap%
\pgfsetroundjoin%
\definecolor{currentfill}{rgb}{0.281446,0.084320,0.407414}%
\pgfsetfillcolor{currentfill}%
\pgfsetfillopacity{0.700000}%
\pgfsetlinewidth{0.000000pt}%
\definecolor{currentstroke}{rgb}{0.000000,0.000000,0.000000}%
\pgfsetstrokecolor{currentstroke}%
\pgfsetdash{}{0pt}%
\pgfpathmoveto{\pgfqpoint{4.314961in}{2.236110in}}%
\pgfpathlineto{\pgfqpoint{4.328672in}{2.236143in}}%
\pgfpathlineto{\pgfqpoint{4.342393in}{2.236281in}}%
\pgfpathlineto{\pgfqpoint{4.356122in}{2.236525in}}%
\pgfpathlineto{\pgfqpoint{4.369860in}{2.236873in}}%
\pgfpathlineto{\pgfqpoint{4.362072in}{2.226962in}}%
\pgfpathlineto{\pgfqpoint{4.354279in}{2.217035in}}%
\pgfpathlineto{\pgfqpoint{4.346481in}{2.207092in}}%
\pgfpathlineto{\pgfqpoint{4.338677in}{2.197136in}}%
\pgfpathlineto{\pgfqpoint{4.324931in}{2.196962in}}%
\pgfpathlineto{\pgfqpoint{4.311194in}{2.196892in}}%
\pgfpathlineto{\pgfqpoint{4.297465in}{2.196927in}}%
\pgfpathlineto{\pgfqpoint{4.283745in}{2.197068in}}%
\pgfpathlineto{\pgfqpoint{4.291557in}{2.206844in}}%
\pgfpathlineto{\pgfqpoint{4.299363in}{2.216610in}}%
\pgfpathlineto{\pgfqpoint{4.307165in}{2.226366in}}%
\pgfpathlineto{\pgfqpoint{4.314961in}{2.236110in}}%
\pgfpathclose%
\pgfusepath{fill}%
\end{pgfscope}%
\begin{pgfscope}%
\pgfpathrectangle{\pgfqpoint{1.150000in}{0.150000in}}{\pgfqpoint{5.700000in}{5.700000in}}%
\pgfusepath{clip}%
\pgfsetbuttcap%
\pgfsetroundjoin%
\definecolor{currentfill}{rgb}{0.143343,0.522773,0.556295}%
\pgfsetfillcolor{currentfill}%
\pgfsetfillopacity{0.700000}%
\pgfsetlinewidth{0.000000pt}%
\definecolor{currentstroke}{rgb}{0.000000,0.000000,0.000000}%
\pgfsetstrokecolor{currentstroke}%
\pgfsetdash{}{0pt}%
\pgfpathmoveto{\pgfqpoint{6.066225in}{3.242399in}}%
\pgfpathlineto{\pgfqpoint{6.080722in}{3.249827in}}%
\pgfpathlineto{\pgfqpoint{6.095234in}{3.257353in}}%
\pgfpathlineto{\pgfqpoint{6.109763in}{3.264977in}}%
\pgfpathlineto{\pgfqpoint{6.124308in}{3.272699in}}%
\pgfpathlineto{\pgfqpoint{6.117325in}{3.269001in}}%
\pgfpathlineto{\pgfqpoint{6.110333in}{3.265242in}}%
\pgfpathlineto{\pgfqpoint{6.103333in}{3.261418in}}%
\pgfpathlineto{\pgfqpoint{6.096324in}{3.257529in}}%
\pgfpathlineto{\pgfqpoint{6.081757in}{3.249582in}}%
\pgfpathlineto{\pgfqpoint{6.067207in}{3.241734in}}%
\pgfpathlineto{\pgfqpoint{6.052673in}{3.233984in}}%
\pgfpathlineto{\pgfqpoint{6.038156in}{3.226332in}}%
\pgfpathlineto{\pgfqpoint{6.045185in}{3.230439in}}%
\pgfpathlineto{\pgfqpoint{6.052207in}{3.234484in}}%
\pgfpathlineto{\pgfqpoint{6.059220in}{3.238470in}}%
\pgfpathlineto{\pgfqpoint{6.066225in}{3.242399in}}%
\pgfpathclose%
\pgfusepath{fill}%
\end{pgfscope}%
\begin{pgfscope}%
\pgfpathrectangle{\pgfqpoint{1.150000in}{0.150000in}}{\pgfqpoint{5.700000in}{5.700000in}}%
\pgfusepath{clip}%
\pgfsetbuttcap%
\pgfsetroundjoin%
\definecolor{currentfill}{rgb}{0.269308,0.218818,0.509577}%
\pgfsetfillcolor{currentfill}%
\pgfsetfillopacity{0.700000}%
\pgfsetlinewidth{0.000000pt}%
\definecolor{currentstroke}{rgb}{0.000000,0.000000,0.000000}%
\pgfsetstrokecolor{currentstroke}%
\pgfsetdash{}{0pt}%
\pgfpathmoveto{\pgfqpoint{4.831112in}{2.508650in}}%
\pgfpathlineto{\pgfqpoint{4.845022in}{2.511820in}}%
\pgfpathlineto{\pgfqpoint{4.858943in}{2.515091in}}%
\pgfpathlineto{\pgfqpoint{4.872876in}{2.518463in}}%
\pgfpathlineto{\pgfqpoint{4.886821in}{2.521937in}}%
\pgfpathlineto{\pgfqpoint{4.879208in}{2.512398in}}%
\pgfpathlineto{\pgfqpoint{4.871588in}{2.502793in}}%
\pgfpathlineto{\pgfqpoint{4.863962in}{2.493124in}}%
\pgfpathlineto{\pgfqpoint{4.856331in}{2.483390in}}%
\pgfpathlineto{\pgfqpoint{4.842379in}{2.479981in}}%
\pgfpathlineto{\pgfqpoint{4.828440in}{2.476673in}}%
\pgfpathlineto{\pgfqpoint{4.814511in}{2.473466in}}%
\pgfpathlineto{\pgfqpoint{4.800594in}{2.470361in}}%
\pgfpathlineto{\pgfqpoint{4.808232in}{2.480023in}}%
\pgfpathlineto{\pgfqpoint{4.815865in}{2.489626in}}%
\pgfpathlineto{\pgfqpoint{4.823491in}{2.499168in}}%
\pgfpathlineto{\pgfqpoint{4.831112in}{2.508650in}}%
\pgfpathclose%
\pgfusepath{fill}%
\end{pgfscope}%
\begin{pgfscope}%
\pgfpathrectangle{\pgfqpoint{1.150000in}{0.150000in}}{\pgfqpoint{5.700000in}{5.700000in}}%
\pgfusepath{clip}%
\pgfsetbuttcap%
\pgfsetroundjoin%
\definecolor{currentfill}{rgb}{0.278791,0.062145,0.386592}%
\pgfsetfillcolor{currentfill}%
\pgfsetfillopacity{0.700000}%
\pgfsetlinewidth{0.000000pt}%
\definecolor{currentstroke}{rgb}{0.000000,0.000000,0.000000}%
\pgfsetstrokecolor{currentstroke}%
\pgfsetdash{}{0pt}%
\pgfpathmoveto{\pgfqpoint{4.228951in}{2.198686in}}%
\pgfpathlineto{\pgfqpoint{4.242637in}{2.198123in}}%
\pgfpathlineto{\pgfqpoint{4.256331in}{2.197665in}}%
\pgfpathlineto{\pgfqpoint{4.270034in}{2.197314in}}%
\pgfpathlineto{\pgfqpoint{4.283745in}{2.197068in}}%
\pgfpathlineto{\pgfqpoint{4.275929in}{2.187285in}}%
\pgfpathlineto{\pgfqpoint{4.268107in}{2.177495in}}%
\pgfpathlineto{\pgfqpoint{4.260279in}{2.167702in}}%
\pgfpathlineto{\pgfqpoint{4.252447in}{2.157905in}}%
\pgfpathlineto{\pgfqpoint{4.238727in}{2.158343in}}%
\pgfpathlineto{\pgfqpoint{4.225015in}{2.158886in}}%
\pgfpathlineto{\pgfqpoint{4.211312in}{2.159535in}}%
\pgfpathlineto{\pgfqpoint{4.197617in}{2.160291in}}%
\pgfpathlineto{\pgfqpoint{4.205458in}{2.169888in}}%
\pgfpathlineto{\pgfqpoint{4.213294in}{2.179488in}}%
\pgfpathlineto{\pgfqpoint{4.221125in}{2.189088in}}%
\pgfpathlineto{\pgfqpoint{4.228951in}{2.198686in}}%
\pgfpathclose%
\pgfusepath{fill}%
\end{pgfscope}%
\begin{pgfscope}%
\pgfpathrectangle{\pgfqpoint{1.150000in}{0.150000in}}{\pgfqpoint{5.700000in}{5.700000in}}%
\pgfusepath{clip}%
\pgfsetbuttcap%
\pgfsetroundjoin%
\definecolor{currentfill}{rgb}{0.273006,0.204520,0.501721}%
\pgfsetfillcolor{currentfill}%
\pgfsetfillopacity{0.700000}%
\pgfsetlinewidth{0.000000pt}%
\definecolor{currentstroke}{rgb}{0.000000,0.000000,0.000000}%
\pgfsetstrokecolor{currentstroke}%
\pgfsetdash{}{0pt}%
\pgfpathmoveto{\pgfqpoint{2.854230in}{2.484981in}}%
\pgfpathlineto{\pgfqpoint{2.867810in}{2.471880in}}%
\pgfpathlineto{\pgfqpoint{2.881387in}{2.458940in}}%
\pgfpathlineto{\pgfqpoint{2.894962in}{2.446158in}}%
\pgfpathlineto{\pgfqpoint{2.908536in}{2.433535in}}%
\pgfpathlineto{\pgfqpoint{2.900085in}{2.431940in}}%
\pgfpathlineto{\pgfqpoint{2.891622in}{2.430523in}}%
\pgfpathlineto{\pgfqpoint{2.883147in}{2.429288in}}%
\pgfpathlineto{\pgfqpoint{2.874659in}{2.428237in}}%
\pgfpathlineto{\pgfqpoint{2.861053in}{2.441229in}}%
\pgfpathlineto{\pgfqpoint{2.847445in}{2.454378in}}%
\pgfpathlineto{\pgfqpoint{2.833834in}{2.467687in}}%
\pgfpathlineto{\pgfqpoint{2.820221in}{2.481157in}}%
\pgfpathlineto{\pgfqpoint{2.828743in}{2.481831in}}%
\pgfpathlineto{\pgfqpoint{2.837251in}{2.482696in}}%
\pgfpathlineto{\pgfqpoint{2.845747in}{2.483747in}}%
\pgfpathlineto{\pgfqpoint{2.854230in}{2.484981in}}%
\pgfpathclose%
\pgfusepath{fill}%
\end{pgfscope}%
\begin{pgfscope}%
\pgfpathrectangle{\pgfqpoint{1.150000in}{0.150000in}}{\pgfqpoint{5.700000in}{5.700000in}}%
\pgfusepath{clip}%
\pgfsetbuttcap%
\pgfsetroundjoin%
\definecolor{currentfill}{rgb}{0.260571,0.246922,0.522828}%
\pgfsetfillcolor{currentfill}%
\pgfsetfillopacity{0.700000}%
\pgfsetlinewidth{0.000000pt}%
\definecolor{currentstroke}{rgb}{0.000000,0.000000,0.000000}%
\pgfsetstrokecolor{currentstroke}%
\pgfsetdash{}{0pt}%
\pgfpathmoveto{\pgfqpoint{4.917215in}{2.559437in}}%
\pgfpathlineto{\pgfqpoint{4.931164in}{2.563058in}}%
\pgfpathlineto{\pgfqpoint{4.945125in}{2.566779in}}%
\pgfpathlineto{\pgfqpoint{4.959098in}{2.570602in}}%
\pgfpathlineto{\pgfqpoint{4.973083in}{2.574525in}}%
\pgfpathlineto{\pgfqpoint{4.965501in}{2.565210in}}%
\pgfpathlineto{\pgfqpoint{4.957912in}{2.555824in}}%
\pgfpathlineto{\pgfqpoint{4.950318in}{2.546368in}}%
\pgfpathlineto{\pgfqpoint{4.942717in}{2.536841in}}%
\pgfpathlineto{\pgfqpoint{4.928726in}{2.532964in}}%
\pgfpathlineto{\pgfqpoint{4.914746in}{2.529187in}}%
\pgfpathlineto{\pgfqpoint{4.900777in}{2.525512in}}%
\pgfpathlineto{\pgfqpoint{4.886821in}{2.521937in}}%
\pgfpathlineto{\pgfqpoint{4.894429in}{2.531411in}}%
\pgfpathlineto{\pgfqpoint{4.902030in}{2.540819in}}%
\pgfpathlineto{\pgfqpoint{4.909625in}{2.550161in}}%
\pgfpathlineto{\pgfqpoint{4.917215in}{2.559437in}}%
\pgfpathclose%
\pgfusepath{fill}%
\end{pgfscope}%
\begin{pgfscope}%
\pgfpathrectangle{\pgfqpoint{1.150000in}{0.150000in}}{\pgfqpoint{5.700000in}{5.700000in}}%
\pgfusepath{clip}%
\pgfsetbuttcap%
\pgfsetroundjoin%
\definecolor{currentfill}{rgb}{0.269944,0.014625,0.341379}%
\pgfsetfillcolor{currentfill}%
\pgfsetfillopacity{0.700000}%
\pgfsetlinewidth{0.000000pt}%
\definecolor{currentstroke}{rgb}{0.000000,0.000000,0.000000}%
\pgfsetstrokecolor{currentstroke}%
\pgfsetdash{}{0pt}%
\pgfpathmoveto{\pgfqpoint{3.548940in}{2.116628in}}%
\pgfpathlineto{\pgfqpoint{3.562484in}{2.110588in}}%
\pgfpathlineto{\pgfqpoint{3.576032in}{2.104669in}}%
\pgfpathlineto{\pgfqpoint{3.589584in}{2.098870in}}%
\pgfpathlineto{\pgfqpoint{3.603141in}{2.093190in}}%
\pgfpathlineto{\pgfqpoint{3.595070in}{2.086455in}}%
\pgfpathlineto{\pgfqpoint{3.586993in}{2.079807in}}%
\pgfpathlineto{\pgfqpoint{3.578909in}{2.073250in}}%
\pgfpathlineto{\pgfqpoint{3.570817in}{2.066787in}}%
\pgfpathlineto{\pgfqpoint{3.557242in}{2.072767in}}%
\pgfpathlineto{\pgfqpoint{3.543672in}{2.078867in}}%
\pgfpathlineto{\pgfqpoint{3.530105in}{2.085087in}}%
\pgfpathlineto{\pgfqpoint{3.516542in}{2.091428in}}%
\pgfpathlineto{\pgfqpoint{3.524653in}{2.097584in}}%
\pgfpathlineto{\pgfqpoint{3.532756in}{2.103837in}}%
\pgfpathlineto{\pgfqpoint{3.540851in}{2.110187in}}%
\pgfpathlineto{\pgfqpoint{3.548940in}{2.116628in}}%
\pgfpathclose%
\pgfusepath{fill}%
\end{pgfscope}%
\begin{pgfscope}%
\pgfpathrectangle{\pgfqpoint{1.150000in}{0.150000in}}{\pgfqpoint{5.700000in}{5.700000in}}%
\pgfusepath{clip}%
\pgfsetbuttcap%
\pgfsetroundjoin%
\definecolor{currentfill}{rgb}{0.267004,0.004874,0.329415}%
\pgfsetfillcolor{currentfill}%
\pgfsetfillopacity{0.700000}%
\pgfsetlinewidth{0.000000pt}%
\definecolor{currentstroke}{rgb}{0.000000,0.000000,0.000000}%
\pgfsetstrokecolor{currentstroke}%
\pgfsetdash{}{0pt}%
\pgfpathmoveto{\pgfqpoint{3.689551in}{2.100551in}}%
\pgfpathlineto{\pgfqpoint{3.703113in}{2.095744in}}%
\pgfpathlineto{\pgfqpoint{3.716679in}{2.091053in}}%
\pgfpathlineto{\pgfqpoint{3.730251in}{2.086479in}}%
\pgfpathlineto{\pgfqpoint{3.743828in}{2.082019in}}%
\pgfpathlineto{\pgfqpoint{3.735818in}{2.074410in}}%
\pgfpathlineto{\pgfqpoint{3.727801in}{2.066868in}}%
\pgfpathlineto{\pgfqpoint{3.719778in}{2.059397in}}%
\pgfpathlineto{\pgfqpoint{3.711748in}{2.051999in}}%
\pgfpathlineto{\pgfqpoint{3.698156in}{2.056740in}}%
\pgfpathlineto{\pgfqpoint{3.684569in}{2.061596in}}%
\pgfpathlineto{\pgfqpoint{3.670986in}{2.066569in}}%
\pgfpathlineto{\pgfqpoint{3.657408in}{2.071658in}}%
\pgfpathlineto{\pgfqpoint{3.665454in}{2.078768in}}%
\pgfpathlineto{\pgfqpoint{3.673493in}{2.085955in}}%
\pgfpathlineto{\pgfqpoint{3.681525in}{2.093217in}}%
\pgfpathlineto{\pgfqpoint{3.689551in}{2.100551in}}%
\pgfpathclose%
\pgfusepath{fill}%
\end{pgfscope}%
\begin{pgfscope}%
\pgfpathrectangle{\pgfqpoint{1.150000in}{0.150000in}}{\pgfqpoint{5.700000in}{5.700000in}}%
\pgfusepath{clip}%
\pgfsetbuttcap%
\pgfsetroundjoin%
\definecolor{currentfill}{rgb}{0.252194,0.269783,0.531579}%
\pgfsetfillcolor{currentfill}%
\pgfsetfillopacity{0.700000}%
\pgfsetlinewidth{0.000000pt}%
\definecolor{currentstroke}{rgb}{0.000000,0.000000,0.000000}%
\pgfsetstrokecolor{currentstroke}%
\pgfsetdash{}{0pt}%
\pgfpathmoveto{\pgfqpoint{5.003347in}{2.611077in}}%
\pgfpathlineto{\pgfqpoint{5.017337in}{2.615129in}}%
\pgfpathlineto{\pgfqpoint{5.031339in}{2.619280in}}%
\pgfpathlineto{\pgfqpoint{5.045353in}{2.623533in}}%
\pgfpathlineto{\pgfqpoint{5.059379in}{2.627885in}}%
\pgfpathlineto{\pgfqpoint{5.051830in}{2.618833in}}%
\pgfpathlineto{\pgfqpoint{5.044275in}{2.609706in}}%
\pgfpathlineto{\pgfqpoint{5.036713in}{2.600503in}}%
\pgfpathlineto{\pgfqpoint{5.029144in}{2.591224in}}%
\pgfpathlineto{\pgfqpoint{5.015110in}{2.586899in}}%
\pgfpathlineto{\pgfqpoint{5.001089in}{2.582674in}}%
\pgfpathlineto{\pgfqpoint{4.987080in}{2.578549in}}%
\pgfpathlineto{\pgfqpoint{4.973083in}{2.574525in}}%
\pgfpathlineto{\pgfqpoint{4.980658in}{2.583769in}}%
\pgfpathlineto{\pgfqpoint{4.988228in}{2.592943in}}%
\pgfpathlineto{\pgfqpoint{4.995791in}{2.602045in}}%
\pgfpathlineto{\pgfqpoint{5.003347in}{2.611077in}}%
\pgfpathclose%
\pgfusepath{fill}%
\end{pgfscope}%
\begin{pgfscope}%
\pgfpathrectangle{\pgfqpoint{1.150000in}{0.150000in}}{\pgfqpoint{5.700000in}{5.700000in}}%
\pgfusepath{clip}%
\pgfsetbuttcap%
\pgfsetroundjoin%
\definecolor{currentfill}{rgb}{0.276022,0.044167,0.370164}%
\pgfsetfillcolor{currentfill}%
\pgfsetfillopacity{0.700000}%
\pgfsetlinewidth{0.000000pt}%
\definecolor{currentstroke}{rgb}{0.000000,0.000000,0.000000}%
\pgfsetstrokecolor{currentstroke}%
\pgfsetdash{}{0pt}%
\pgfpathmoveto{\pgfqpoint{4.142916in}{2.164378in}}%
\pgfpathlineto{\pgfqpoint{4.156580in}{2.163196in}}%
\pgfpathlineto{\pgfqpoint{4.170251in}{2.162121in}}%
\pgfpathlineto{\pgfqpoint{4.183930in}{2.161152in}}%
\pgfpathlineto{\pgfqpoint{4.197617in}{2.160291in}}%
\pgfpathlineto{\pgfqpoint{4.189771in}{2.150696in}}%
\pgfpathlineto{\pgfqpoint{4.181919in}{2.141108in}}%
\pgfpathlineto{\pgfqpoint{4.174062in}{2.131526in}}%
\pgfpathlineto{\pgfqpoint{4.166199in}{2.121954in}}%
\pgfpathlineto{\pgfqpoint{4.152503in}{2.123025in}}%
\pgfpathlineto{\pgfqpoint{4.138814in}{2.124203in}}%
\pgfpathlineto{\pgfqpoint{4.125133in}{2.125488in}}%
\pgfpathlineto{\pgfqpoint{4.111460in}{2.126880in}}%
\pgfpathlineto{\pgfqpoint{4.119332in}{2.136236in}}%
\pgfpathlineto{\pgfqpoint{4.127199in}{2.145606in}}%
\pgfpathlineto{\pgfqpoint{4.135060in}{2.154987in}}%
\pgfpathlineto{\pgfqpoint{4.142916in}{2.164378in}}%
\pgfpathclose%
\pgfusepath{fill}%
\end{pgfscope}%
\begin{pgfscope}%
\pgfpathrectangle{\pgfqpoint{1.150000in}{0.150000in}}{\pgfqpoint{5.700000in}{5.700000in}}%
\pgfusepath{clip}%
\pgfsetbuttcap%
\pgfsetroundjoin%
\definecolor{currentfill}{rgb}{0.278012,0.180367,0.486697}%
\pgfsetfillcolor{currentfill}%
\pgfsetfillopacity{0.700000}%
\pgfsetlinewidth{0.000000pt}%
\definecolor{currentstroke}{rgb}{0.000000,0.000000,0.000000}%
\pgfsetstrokecolor{currentstroke}%
\pgfsetdash{}{0pt}%
\pgfpathmoveto{\pgfqpoint{2.908536in}{2.433535in}}%
\pgfpathlineto{\pgfqpoint{2.922107in}{2.421068in}}%
\pgfpathlineto{\pgfqpoint{2.935677in}{2.408756in}}%
\pgfpathlineto{\pgfqpoint{2.949245in}{2.396599in}}%
\pgfpathlineto{\pgfqpoint{2.962812in}{2.384595in}}%
\pgfpathlineto{\pgfqpoint{2.954392in}{2.382641in}}%
\pgfpathlineto{\pgfqpoint{2.945961in}{2.380860in}}%
\pgfpathlineto{\pgfqpoint{2.937518in}{2.379255in}}%
\pgfpathlineto{\pgfqpoint{2.929063in}{2.377831in}}%
\pgfpathlineto{\pgfqpoint{2.915465in}{2.390201in}}%
\pgfpathlineto{\pgfqpoint{2.901865in}{2.402725in}}%
\pgfpathlineto{\pgfqpoint{2.888263in}{2.415403in}}%
\pgfpathlineto{\pgfqpoint{2.874659in}{2.428237in}}%
\pgfpathlineto{\pgfqpoint{2.883147in}{2.429288in}}%
\pgfpathlineto{\pgfqpoint{2.891622in}{2.430523in}}%
\pgfpathlineto{\pgfqpoint{2.900085in}{2.431940in}}%
\pgfpathlineto{\pgfqpoint{2.908536in}{2.433535in}}%
\pgfpathclose%
\pgfusepath{fill}%
\end{pgfscope}%
\begin{pgfscope}%
\pgfpathrectangle{\pgfqpoint{1.150000in}{0.150000in}}{\pgfqpoint{5.700000in}{5.700000in}}%
\pgfusepath{clip}%
\pgfsetbuttcap%
\pgfsetroundjoin%
\definecolor{currentfill}{rgb}{0.273809,0.031497,0.358853}%
\pgfsetfillcolor{currentfill}%
\pgfsetfillopacity{0.700000}%
\pgfsetlinewidth{0.000000pt}%
\definecolor{currentstroke}{rgb}{0.000000,0.000000,0.000000}%
\pgfsetstrokecolor{currentstroke}%
\pgfsetdash{}{0pt}%
\pgfpathmoveto{\pgfqpoint{3.408155in}{2.146580in}}%
\pgfpathlineto{\pgfqpoint{3.421693in}{2.139250in}}%
\pgfpathlineto{\pgfqpoint{3.435234in}{2.132046in}}%
\pgfpathlineto{\pgfqpoint{3.448777in}{2.124967in}}%
\pgfpathlineto{\pgfqpoint{3.462324in}{2.118012in}}%
\pgfpathlineto{\pgfqpoint{3.454186in}{2.112270in}}%
\pgfpathlineto{\pgfqpoint{3.446039in}{2.106637in}}%
\pgfpathlineto{\pgfqpoint{3.437885in}{2.101115in}}%
\pgfpathlineto{\pgfqpoint{3.429722in}{2.095709in}}%
\pgfpathlineto{\pgfqpoint{3.416154in}{2.102983in}}%
\pgfpathlineto{\pgfqpoint{3.402590in}{2.110382in}}%
\pgfpathlineto{\pgfqpoint{3.389028in}{2.117906in}}%
\pgfpathlineto{\pgfqpoint{3.375469in}{2.125557in}}%
\pgfpathlineto{\pgfqpoint{3.383653in}{2.130636in}}%
\pgfpathlineto{\pgfqpoint{3.391829in}{2.135835in}}%
\pgfpathlineto{\pgfqpoint{3.399996in}{2.141151in}}%
\pgfpathlineto{\pgfqpoint{3.408155in}{2.146580in}}%
\pgfpathclose%
\pgfusepath{fill}%
\end{pgfscope}%
\begin{pgfscope}%
\pgfpathrectangle{\pgfqpoint{1.150000in}{0.150000in}}{\pgfqpoint{5.700000in}{5.700000in}}%
\pgfusepath{clip}%
\pgfsetbuttcap%
\pgfsetroundjoin%
\definecolor{currentfill}{rgb}{0.280894,0.078907,0.402329}%
\pgfsetfillcolor{currentfill}%
\pgfsetfillopacity{0.700000}%
\pgfsetlinewidth{0.000000pt}%
\definecolor{currentstroke}{rgb}{0.000000,0.000000,0.000000}%
\pgfsetstrokecolor{currentstroke}%
\pgfsetdash{}{0pt}%
\pgfpathmoveto{\pgfqpoint{3.212915in}{2.227452in}}%
\pgfpathlineto{\pgfqpoint{3.226452in}{2.218233in}}%
\pgfpathlineto{\pgfqpoint{3.239991in}{2.209148in}}%
\pgfpathlineto{\pgfqpoint{3.253531in}{2.200198in}}%
\pgfpathlineto{\pgfqpoint{3.267072in}{2.191381in}}%
\pgfpathlineto{\pgfqpoint{3.258832in}{2.187090in}}%
\pgfpathlineto{\pgfqpoint{3.250583in}{2.182933in}}%
\pgfpathlineto{\pgfqpoint{3.242324in}{2.178916in}}%
\pgfpathlineto{\pgfqpoint{3.234056in}{2.175040in}}%
\pgfpathlineto{\pgfqpoint{3.220490in}{2.184198in}}%
\pgfpathlineto{\pgfqpoint{3.206924in}{2.193489in}}%
\pgfpathlineto{\pgfqpoint{3.193360in}{2.202915in}}%
\pgfpathlineto{\pgfqpoint{3.179797in}{2.212476in}}%
\pgfpathlineto{\pgfqpoint{3.188091in}{2.216003in}}%
\pgfpathlineto{\pgfqpoint{3.196376in}{2.219677in}}%
\pgfpathlineto{\pgfqpoint{3.204650in}{2.223495in}}%
\pgfpathlineto{\pgfqpoint{3.212915in}{2.227452in}}%
\pgfpathclose%
\pgfusepath{fill}%
\end{pgfscope}%
\begin{pgfscope}%
\pgfpathrectangle{\pgfqpoint{1.150000in}{0.150000in}}{\pgfqpoint{5.700000in}{5.700000in}}%
\pgfusepath{clip}%
\pgfsetbuttcap%
\pgfsetroundjoin%
\definecolor{currentfill}{rgb}{0.241237,0.296485,0.539709}%
\pgfsetfillcolor{currentfill}%
\pgfsetfillopacity{0.700000}%
\pgfsetlinewidth{0.000000pt}%
\definecolor{currentstroke}{rgb}{0.000000,0.000000,0.000000}%
\pgfsetstrokecolor{currentstroke}%
\pgfsetdash{}{0pt}%
\pgfpathmoveto{\pgfqpoint{5.089510in}{2.663343in}}%
\pgfpathlineto{\pgfqpoint{5.103541in}{2.667804in}}%
\pgfpathlineto{\pgfqpoint{5.117585in}{2.672366in}}%
\pgfpathlineto{\pgfqpoint{5.131641in}{2.677028in}}%
\pgfpathlineto{\pgfqpoint{5.145710in}{2.681791in}}%
\pgfpathlineto{\pgfqpoint{5.138195in}{2.673037in}}%
\pgfpathlineto{\pgfqpoint{5.130674in}{2.664205in}}%
\pgfpathlineto{\pgfqpoint{5.123146in}{2.655292in}}%
\pgfpathlineto{\pgfqpoint{5.115611in}{2.646300in}}%
\pgfpathlineto{\pgfqpoint{5.101534in}{2.641546in}}%
\pgfpathlineto{\pgfqpoint{5.087470in}{2.636892in}}%
\pgfpathlineto{\pgfqpoint{5.073418in}{2.632339in}}%
\pgfpathlineto{\pgfqpoint{5.059379in}{2.627885in}}%
\pgfpathlineto{\pgfqpoint{5.066922in}{2.636862in}}%
\pgfpathlineto{\pgfqpoint{5.074458in}{2.645764in}}%
\pgfpathlineto{\pgfqpoint{5.081987in}{2.654591in}}%
\pgfpathlineto{\pgfqpoint{5.089510in}{2.663343in}}%
\pgfpathclose%
\pgfusepath{fill}%
\end{pgfscope}%
\begin{pgfscope}%
\pgfpathrectangle{\pgfqpoint{1.150000in}{0.150000in}}{\pgfqpoint{5.700000in}{5.700000in}}%
\pgfusepath{clip}%
\pgfsetbuttcap%
\pgfsetroundjoin%
\definecolor{currentfill}{rgb}{0.267004,0.004874,0.329415}%
\pgfsetfillcolor{currentfill}%
\pgfsetfillopacity{0.700000}%
\pgfsetlinewidth{0.000000pt}%
\definecolor{currentstroke}{rgb}{0.000000,0.000000,0.000000}%
\pgfsetstrokecolor{currentstroke}%
\pgfsetdash{}{0pt}%
\pgfpathmoveto{\pgfqpoint{3.830106in}{2.097449in}}%
\pgfpathlineto{\pgfqpoint{3.843695in}{2.093824in}}%
\pgfpathlineto{\pgfqpoint{3.857290in}{2.090312in}}%
\pgfpathlineto{\pgfqpoint{3.870892in}{2.086912in}}%
\pgfpathlineto{\pgfqpoint{3.884499in}{2.083624in}}%
\pgfpathlineto{\pgfqpoint{3.876542in}{2.075254in}}%
\pgfpathlineto{\pgfqpoint{3.868579in}{2.066934in}}%
\pgfpathlineto{\pgfqpoint{3.860610in}{2.058664in}}%
\pgfpathlineto{\pgfqpoint{3.852635in}{2.050449in}}%
\pgfpathlineto{\pgfqpoint{3.839014in}{2.054000in}}%
\pgfpathlineto{\pgfqpoint{3.825399in}{2.057664in}}%
\pgfpathlineto{\pgfqpoint{3.811790in}{2.061439in}}%
\pgfpathlineto{\pgfqpoint{3.798187in}{2.065328in}}%
\pgfpathlineto{\pgfqpoint{3.806176in}{2.073273in}}%
\pgfpathlineto{\pgfqpoint{3.814159in}{2.081277in}}%
\pgfpathlineto{\pgfqpoint{3.822135in}{2.089336in}}%
\pgfpathlineto{\pgfqpoint{3.830106in}{2.097449in}}%
\pgfpathclose%
\pgfusepath{fill}%
\end{pgfscope}%
\begin{pgfscope}%
\pgfpathrectangle{\pgfqpoint{1.150000in}{0.150000in}}{\pgfqpoint{5.700000in}{5.700000in}}%
\pgfusepath{clip}%
\pgfsetbuttcap%
\pgfsetroundjoin%
\definecolor{currentfill}{rgb}{0.231674,0.318106,0.544834}%
\pgfsetfillcolor{currentfill}%
\pgfsetfillopacity{0.700000}%
\pgfsetlinewidth{0.000000pt}%
\definecolor{currentstroke}{rgb}{0.000000,0.000000,0.000000}%
\pgfsetstrokecolor{currentstroke}%
\pgfsetdash{}{0pt}%
\pgfpathmoveto{\pgfqpoint{5.175701in}{2.716017in}}%
\pgfpathlineto{\pgfqpoint{5.189774in}{2.720869in}}%
\pgfpathlineto{\pgfqpoint{5.203861in}{2.725821in}}%
\pgfpathlineto{\pgfqpoint{5.217960in}{2.730873in}}%
\pgfpathlineto{\pgfqpoint{5.232073in}{2.736025in}}%
\pgfpathlineto{\pgfqpoint{5.224594in}{2.727603in}}%
\pgfpathlineto{\pgfqpoint{5.217108in}{2.719099in}}%
\pgfpathlineto{\pgfqpoint{5.209615in}{2.710512in}}%
\pgfpathlineto{\pgfqpoint{5.202116in}{2.701841in}}%
\pgfpathlineto{\pgfqpoint{5.187995in}{2.696679in}}%
\pgfpathlineto{\pgfqpoint{5.173887in}{2.691616in}}%
\pgfpathlineto{\pgfqpoint{5.159792in}{2.686653in}}%
\pgfpathlineto{\pgfqpoint{5.145710in}{2.681791in}}%
\pgfpathlineto{\pgfqpoint{5.153218in}{2.690465in}}%
\pgfpathlineto{\pgfqpoint{5.160719in}{2.699061in}}%
\pgfpathlineto{\pgfqpoint{5.168213in}{2.707578in}}%
\pgfpathlineto{\pgfqpoint{5.175701in}{2.716017in}}%
\pgfpathclose%
\pgfusepath{fill}%
\end{pgfscope}%
\begin{pgfscope}%
\pgfpathrectangle{\pgfqpoint{1.150000in}{0.150000in}}{\pgfqpoint{5.700000in}{5.700000in}}%
\pgfusepath{clip}%
\pgfsetbuttcap%
\pgfsetroundjoin%
\definecolor{currentfill}{rgb}{0.272594,0.025563,0.353093}%
\pgfsetfillcolor{currentfill}%
\pgfsetfillopacity{0.700000}%
\pgfsetlinewidth{0.000000pt}%
\definecolor{currentstroke}{rgb}{0.000000,0.000000,0.000000}%
\pgfsetstrokecolor{currentstroke}%
\pgfsetdash{}{0pt}%
\pgfpathmoveto{\pgfqpoint{4.056842in}{2.133526in}}%
\pgfpathlineto{\pgfqpoint{4.070485in}{2.131702in}}%
\pgfpathlineto{\pgfqpoint{4.084136in}{2.129987in}}%
\pgfpathlineto{\pgfqpoint{4.097794in}{2.128379in}}%
\pgfpathlineto{\pgfqpoint{4.111460in}{2.126880in}}%
\pgfpathlineto{\pgfqpoint{4.103583in}{2.117540in}}%
\pgfpathlineto{\pgfqpoint{4.095700in}{2.108217in}}%
\pgfpathlineto{\pgfqpoint{4.087811in}{2.098914in}}%
\pgfpathlineto{\pgfqpoint{4.079918in}{2.089632in}}%
\pgfpathlineto{\pgfqpoint{4.066242in}{2.091359in}}%
\pgfpathlineto{\pgfqpoint{4.052573in}{2.093194in}}%
\pgfpathlineto{\pgfqpoint{4.038912in}{2.095137in}}%
\pgfpathlineto{\pgfqpoint{4.025257in}{2.097189in}}%
\pgfpathlineto{\pgfqpoint{4.033162in}{2.106236in}}%
\pgfpathlineto{\pgfqpoint{4.041061in}{2.115309in}}%
\pgfpathlineto{\pgfqpoint{4.048954in}{2.124407in}}%
\pgfpathlineto{\pgfqpoint{4.056842in}{2.133526in}}%
\pgfpathclose%
\pgfusepath{fill}%
\end{pgfscope}%
\begin{pgfscope}%
\pgfpathrectangle{\pgfqpoint{1.150000in}{0.150000in}}{\pgfqpoint{5.700000in}{5.700000in}}%
\pgfusepath{clip}%
\pgfsetbuttcap%
\pgfsetroundjoin%
\definecolor{currentfill}{rgb}{0.280868,0.160771,0.472899}%
\pgfsetfillcolor{currentfill}%
\pgfsetfillopacity{0.700000}%
\pgfsetlinewidth{0.000000pt}%
\definecolor{currentstroke}{rgb}{0.000000,0.000000,0.000000}%
\pgfsetstrokecolor{currentstroke}%
\pgfsetdash{}{0pt}%
\pgfpathmoveto{\pgfqpoint{2.962812in}{2.384595in}}%
\pgfpathlineto{\pgfqpoint{2.976377in}{2.372743in}}%
\pgfpathlineto{\pgfqpoint{2.989941in}{2.361041in}}%
\pgfpathlineto{\pgfqpoint{3.003505in}{2.349490in}}%
\pgfpathlineto{\pgfqpoint{3.017067in}{2.338087in}}%
\pgfpathlineto{\pgfqpoint{3.008678in}{2.335775in}}%
\pgfpathlineto{\pgfqpoint{3.000278in}{2.333631in}}%
\pgfpathlineto{\pgfqpoint{2.991866in}{2.331659in}}%
\pgfpathlineto{\pgfqpoint{2.983442in}{2.329862in}}%
\pgfpathlineto{\pgfqpoint{2.969849in}{2.341630in}}%
\pgfpathlineto{\pgfqpoint{2.956255in}{2.353546in}}%
\pgfpathlineto{\pgfqpoint{2.942659in}{2.365613in}}%
\pgfpathlineto{\pgfqpoint{2.929063in}{2.377831in}}%
\pgfpathlineto{\pgfqpoint{2.937518in}{2.379255in}}%
\pgfpathlineto{\pgfqpoint{2.945961in}{2.380860in}}%
\pgfpathlineto{\pgfqpoint{2.954392in}{2.382641in}}%
\pgfpathlineto{\pgfqpoint{2.962812in}{2.384595in}}%
\pgfpathclose%
\pgfusepath{fill}%
\end{pgfscope}%
\begin{pgfscope}%
\pgfpathrectangle{\pgfqpoint{1.150000in}{0.150000in}}{\pgfqpoint{5.700000in}{5.700000in}}%
\pgfusepath{clip}%
\pgfsetbuttcap%
\pgfsetroundjoin%
\definecolor{currentfill}{rgb}{0.220057,0.343307,0.549413}%
\pgfsetfillcolor{currentfill}%
\pgfsetfillopacity{0.700000}%
\pgfsetlinewidth{0.000000pt}%
\definecolor{currentstroke}{rgb}{0.000000,0.000000,0.000000}%
\pgfsetstrokecolor{currentstroke}%
\pgfsetdash{}{0pt}%
\pgfpathmoveto{\pgfqpoint{5.261917in}{2.768896in}}%
\pgfpathlineto{\pgfqpoint{5.276034in}{2.774119in}}%
\pgfpathlineto{\pgfqpoint{5.290164in}{2.779441in}}%
\pgfpathlineto{\pgfqpoint{5.304307in}{2.784863in}}%
\pgfpathlineto{\pgfqpoint{5.318464in}{2.790384in}}%
\pgfpathlineto{\pgfqpoint{5.311023in}{2.782324in}}%
\pgfpathlineto{\pgfqpoint{5.303575in}{2.774179in}}%
\pgfpathlineto{\pgfqpoint{5.296119in}{2.765949in}}%
\pgfpathlineto{\pgfqpoint{5.288656in}{2.757632in}}%
\pgfpathlineto{\pgfqpoint{5.274490in}{2.752080in}}%
\pgfpathlineto{\pgfqpoint{5.260338in}{2.746629in}}%
\pgfpathlineto{\pgfqpoint{5.246198in}{2.741277in}}%
\pgfpathlineto{\pgfqpoint{5.232073in}{2.736025in}}%
\pgfpathlineto{\pgfqpoint{5.239544in}{2.744365in}}%
\pgfpathlineto{\pgfqpoint{5.247009in}{2.752623in}}%
\pgfpathlineto{\pgfqpoint{5.254467in}{2.760800in}}%
\pgfpathlineto{\pgfqpoint{5.261917in}{2.768896in}}%
\pgfpathclose%
\pgfusepath{fill}%
\end{pgfscope}%
\begin{pgfscope}%
\pgfpathrectangle{\pgfqpoint{1.150000in}{0.150000in}}{\pgfqpoint{5.700000in}{5.700000in}}%
\pgfusepath{clip}%
\pgfsetbuttcap%
\pgfsetroundjoin%
\definecolor{currentfill}{rgb}{0.208623,0.367752,0.552675}%
\pgfsetfillcolor{currentfill}%
\pgfsetfillopacity{0.700000}%
\pgfsetlinewidth{0.000000pt}%
\definecolor{currentstroke}{rgb}{0.000000,0.000000,0.000000}%
\pgfsetstrokecolor{currentstroke}%
\pgfsetdash{}{0pt}%
\pgfpathmoveto{\pgfqpoint{5.348156in}{2.821789in}}%
\pgfpathlineto{\pgfqpoint{5.362316in}{2.827362in}}%
\pgfpathlineto{\pgfqpoint{5.376490in}{2.833034in}}%
\pgfpathlineto{\pgfqpoint{5.390678in}{2.838806in}}%
\pgfpathlineto{\pgfqpoint{5.404880in}{2.844677in}}%
\pgfpathlineto{\pgfqpoint{5.397478in}{2.837005in}}%
\pgfpathlineto{\pgfqpoint{5.390069in}{2.829247in}}%
\pgfpathlineto{\pgfqpoint{5.382653in}{2.821402in}}%
\pgfpathlineto{\pgfqpoint{5.375229in}{2.813468in}}%
\pgfpathlineto{\pgfqpoint{5.361017in}{2.807547in}}%
\pgfpathlineto{\pgfqpoint{5.346819in}{2.801727in}}%
\pgfpathlineto{\pgfqpoint{5.332635in}{2.796006in}}%
\pgfpathlineto{\pgfqpoint{5.318464in}{2.790384in}}%
\pgfpathlineto{\pgfqpoint{5.325898in}{2.798360in}}%
\pgfpathlineto{\pgfqpoint{5.333324in}{2.806252in}}%
\pgfpathlineto{\pgfqpoint{5.340744in}{2.814062in}}%
\pgfpathlineto{\pgfqpoint{5.348156in}{2.821789in}}%
\pgfpathclose%
\pgfusepath{fill}%
\end{pgfscope}%
\begin{pgfscope}%
\pgfpathrectangle{\pgfqpoint{1.150000in}{0.150000in}}{\pgfqpoint{5.700000in}{5.700000in}}%
\pgfusepath{clip}%
\pgfsetbuttcap%
\pgfsetroundjoin%
\definecolor{currentfill}{rgb}{0.278791,0.062145,0.386592}%
\pgfsetfillcolor{currentfill}%
\pgfsetfillopacity{0.700000}%
\pgfsetlinewidth{0.000000pt}%
\definecolor{currentstroke}{rgb}{0.000000,0.000000,0.000000}%
\pgfsetstrokecolor{currentstroke}%
\pgfsetdash{}{0pt}%
\pgfpathmoveto{\pgfqpoint{3.267072in}{2.191381in}}%
\pgfpathlineto{\pgfqpoint{3.280615in}{2.182697in}}%
\pgfpathlineto{\pgfqpoint{3.294160in}{2.174144in}}%
\pgfpathlineto{\pgfqpoint{3.307706in}{2.165723in}}%
\pgfpathlineto{\pgfqpoint{3.321254in}{2.157432in}}%
\pgfpathlineto{\pgfqpoint{3.313038in}{2.152808in}}%
\pgfpathlineto{\pgfqpoint{3.304813in}{2.148314in}}%
\pgfpathlineto{\pgfqpoint{3.296579in}{2.143953in}}%
\pgfpathlineto{\pgfqpoint{3.288335in}{2.139730in}}%
\pgfpathlineto{\pgfqpoint{3.274763in}{2.148361in}}%
\pgfpathlineto{\pgfqpoint{3.261193in}{2.157123in}}%
\pgfpathlineto{\pgfqpoint{3.247624in}{2.166015in}}%
\pgfpathlineto{\pgfqpoint{3.234056in}{2.175040in}}%
\pgfpathlineto{\pgfqpoint{3.242324in}{2.178916in}}%
\pgfpathlineto{\pgfqpoint{3.250583in}{2.182933in}}%
\pgfpathlineto{\pgfqpoint{3.258832in}{2.187090in}}%
\pgfpathlineto{\pgfqpoint{3.267072in}{2.191381in}}%
\pgfpathclose%
\pgfusepath{fill}%
\end{pgfscope}%
\begin{pgfscope}%
\pgfpathrectangle{\pgfqpoint{1.150000in}{0.150000in}}{\pgfqpoint{5.700000in}{5.700000in}}%
\pgfusepath{clip}%
\pgfsetbuttcap%
\pgfsetroundjoin%
\definecolor{currentfill}{rgb}{0.268510,0.009605,0.335427}%
\pgfsetfillcolor{currentfill}%
\pgfsetfillopacity{0.700000}%
\pgfsetlinewidth{0.000000pt}%
\definecolor{currentstroke}{rgb}{0.000000,0.000000,0.000000}%
\pgfsetstrokecolor{currentstroke}%
\pgfsetdash{}{0pt}%
\pgfpathmoveto{\pgfqpoint{3.603141in}{2.093190in}}%
\pgfpathlineto{\pgfqpoint{3.616701in}{2.087630in}}%
\pgfpathlineto{\pgfqpoint{3.630266in}{2.082188in}}%
\pgfpathlineto{\pgfqpoint{3.643835in}{2.076864in}}%
\pgfpathlineto{\pgfqpoint{3.657408in}{2.071658in}}%
\pgfpathlineto{\pgfqpoint{3.649355in}{2.064629in}}%
\pgfpathlineto{\pgfqpoint{3.641295in}{2.057684in}}%
\pgfpathlineto{\pgfqpoint{3.633228in}{2.050824in}}%
\pgfpathlineto{\pgfqpoint{3.625154in}{2.044053in}}%
\pgfpathlineto{\pgfqpoint{3.611564in}{2.049560in}}%
\pgfpathlineto{\pgfqpoint{3.597977in}{2.055184in}}%
\pgfpathlineto{\pgfqpoint{3.584395in}{2.060926in}}%
\pgfpathlineto{\pgfqpoint{3.570817in}{2.066787in}}%
\pgfpathlineto{\pgfqpoint{3.578909in}{2.073250in}}%
\pgfpathlineto{\pgfqpoint{3.586993in}{2.079807in}}%
\pgfpathlineto{\pgfqpoint{3.595070in}{2.086455in}}%
\pgfpathlineto{\pgfqpoint{3.603141in}{2.093190in}}%
\pgfpathclose%
\pgfusepath{fill}%
\end{pgfscope}%
\begin{pgfscope}%
\pgfpathrectangle{\pgfqpoint{1.150000in}{0.150000in}}{\pgfqpoint{5.700000in}{5.700000in}}%
\pgfusepath{clip}%
\pgfsetbuttcap%
\pgfsetroundjoin%
\definecolor{currentfill}{rgb}{0.282623,0.140926,0.457517}%
\pgfsetfillcolor{currentfill}%
\pgfsetfillopacity{0.700000}%
\pgfsetlinewidth{0.000000pt}%
\definecolor{currentstroke}{rgb}{0.000000,0.000000,0.000000}%
\pgfsetstrokecolor{currentstroke}%
\pgfsetdash{}{0pt}%
\pgfpathmoveto{\pgfqpoint{3.017067in}{2.338087in}}%
\pgfpathlineto{\pgfqpoint{3.030629in}{2.326831in}}%
\pgfpathlineto{\pgfqpoint{3.044190in}{2.315723in}}%
\pgfpathlineto{\pgfqpoint{3.057750in}{2.304759in}}%
\pgfpathlineto{\pgfqpoint{3.071311in}{2.293940in}}%
\pgfpathlineto{\pgfqpoint{3.062951in}{2.291272in}}%
\pgfpathlineto{\pgfqpoint{3.054580in}{2.288767in}}%
\pgfpathlineto{\pgfqpoint{3.046198in}{2.286429in}}%
\pgfpathlineto{\pgfqpoint{3.037805in}{2.284261in}}%
\pgfpathlineto{\pgfqpoint{3.024215in}{2.295443in}}%
\pgfpathlineto{\pgfqpoint{3.010625in}{2.306770in}}%
\pgfpathlineto{\pgfqpoint{2.997034in}{2.318242in}}%
\pgfpathlineto{\pgfqpoint{2.983442in}{2.329862in}}%
\pgfpathlineto{\pgfqpoint{2.991866in}{2.331659in}}%
\pgfpathlineto{\pgfqpoint{3.000278in}{2.333631in}}%
\pgfpathlineto{\pgfqpoint{3.008678in}{2.335775in}}%
\pgfpathlineto{\pgfqpoint{3.017067in}{2.338087in}}%
\pgfpathclose%
\pgfusepath{fill}%
\end{pgfscope}%
\begin{pgfscope}%
\pgfpathrectangle{\pgfqpoint{1.150000in}{0.150000in}}{\pgfqpoint{5.700000in}{5.700000in}}%
\pgfusepath{clip}%
\pgfsetbuttcap%
\pgfsetroundjoin%
\definecolor{currentfill}{rgb}{0.269944,0.014625,0.341379}%
\pgfsetfillcolor{currentfill}%
\pgfsetfillopacity{0.700000}%
\pgfsetlinewidth{0.000000pt}%
\definecolor{currentstroke}{rgb}{0.000000,0.000000,0.000000}%
\pgfsetstrokecolor{currentstroke}%
\pgfsetdash{}{0pt}%
\pgfpathmoveto{\pgfqpoint{3.970709in}{2.106485in}}%
\pgfpathlineto{\pgfqpoint{3.984336in}{2.103997in}}%
\pgfpathlineto{\pgfqpoint{3.997970in}{2.101618in}}%
\pgfpathlineto{\pgfqpoint{4.011610in}{2.099349in}}%
\pgfpathlineto{\pgfqpoint{4.025257in}{2.097189in}}%
\pgfpathlineto{\pgfqpoint{4.017348in}{2.088170in}}%
\pgfpathlineto{\pgfqpoint{4.009432in}{2.079182in}}%
\pgfpathlineto{\pgfqpoint{4.001511in}{2.070226in}}%
\pgfpathlineto{\pgfqpoint{3.993584in}{2.061306in}}%
\pgfpathlineto{\pgfqpoint{3.979925in}{2.063712in}}%
\pgfpathlineto{\pgfqpoint{3.966274in}{2.066226in}}%
\pgfpathlineto{\pgfqpoint{3.952628in}{2.068851in}}%
\pgfpathlineto{\pgfqpoint{3.938990in}{2.071584in}}%
\pgfpathlineto{\pgfqpoint{3.946928in}{2.080252in}}%
\pgfpathlineto{\pgfqpoint{3.954861in}{2.088960in}}%
\pgfpathlineto{\pgfqpoint{3.962788in}{2.097705in}}%
\pgfpathlineto{\pgfqpoint{3.970709in}{2.106485in}}%
\pgfpathclose%
\pgfusepath{fill}%
\end{pgfscope}%
\begin{pgfscope}%
\pgfpathrectangle{\pgfqpoint{1.150000in}{0.150000in}}{\pgfqpoint{5.700000in}{5.700000in}}%
\pgfusepath{clip}%
\pgfsetbuttcap%
\pgfsetroundjoin%
\definecolor{currentfill}{rgb}{0.199430,0.387607,0.554642}%
\pgfsetfillcolor{currentfill}%
\pgfsetfillopacity{0.700000}%
\pgfsetlinewidth{0.000000pt}%
\definecolor{currentstroke}{rgb}{0.000000,0.000000,0.000000}%
\pgfsetstrokecolor{currentstroke}%
\pgfsetdash{}{0pt}%
\pgfpathmoveto{\pgfqpoint{5.434412in}{2.874514in}}%
\pgfpathlineto{\pgfqpoint{5.448616in}{2.880417in}}%
\pgfpathlineto{\pgfqpoint{5.462835in}{2.886419in}}%
\pgfpathlineto{\pgfqpoint{5.477068in}{2.892521in}}%
\pgfpathlineto{\pgfqpoint{5.491315in}{2.898722in}}%
\pgfpathlineto{\pgfqpoint{5.483955in}{2.891463in}}%
\pgfpathlineto{\pgfqpoint{5.476587in}{2.884117in}}%
\pgfpathlineto{\pgfqpoint{5.469211in}{2.876682in}}%
\pgfpathlineto{\pgfqpoint{5.461828in}{2.869157in}}%
\pgfpathlineto{\pgfqpoint{5.447570in}{2.862888in}}%
\pgfpathlineto{\pgfqpoint{5.433326in}{2.856718in}}%
\pgfpathlineto{\pgfqpoint{5.419096in}{2.850648in}}%
\pgfpathlineto{\pgfqpoint{5.404880in}{2.844677in}}%
\pgfpathlineto{\pgfqpoint{5.412274in}{2.852262in}}%
\pgfpathlineto{\pgfqpoint{5.419661in}{2.859763in}}%
\pgfpathlineto{\pgfqpoint{5.427040in}{2.867180in}}%
\pgfpathlineto{\pgfqpoint{5.434412in}{2.874514in}}%
\pgfpathclose%
\pgfusepath{fill}%
\end{pgfscope}%
\begin{pgfscope}%
\pgfpathrectangle{\pgfqpoint{1.150000in}{0.150000in}}{\pgfqpoint{5.700000in}{5.700000in}}%
\pgfusepath{clip}%
\pgfsetbuttcap%
\pgfsetroundjoin%
\definecolor{currentfill}{rgb}{0.267004,0.004874,0.329415}%
\pgfsetfillcolor{currentfill}%
\pgfsetfillopacity{0.700000}%
\pgfsetlinewidth{0.000000pt}%
\definecolor{currentstroke}{rgb}{0.000000,0.000000,0.000000}%
\pgfsetstrokecolor{currentstroke}%
\pgfsetdash{}{0pt}%
\pgfpathmoveto{\pgfqpoint{3.743828in}{2.082019in}}%
\pgfpathlineto{\pgfqpoint{3.757410in}{2.077675in}}%
\pgfpathlineto{\pgfqpoint{3.770997in}{2.073446in}}%
\pgfpathlineto{\pgfqpoint{3.784589in}{2.069330in}}%
\pgfpathlineto{\pgfqpoint{3.798187in}{2.065328in}}%
\pgfpathlineto{\pgfqpoint{3.790192in}{2.057444in}}%
\pgfpathlineto{\pgfqpoint{3.782190in}{2.049622in}}%
\pgfpathlineto{\pgfqpoint{3.774182in}{2.041867in}}%
\pgfpathlineto{\pgfqpoint{3.766168in}{2.034180in}}%
\pgfpathlineto{\pgfqpoint{3.752555in}{2.038464in}}%
\pgfpathlineto{\pgfqpoint{3.738948in}{2.042861in}}%
\pgfpathlineto{\pgfqpoint{3.725345in}{2.047373in}}%
\pgfpathlineto{\pgfqpoint{3.711748in}{2.051999in}}%
\pgfpathlineto{\pgfqpoint{3.719778in}{2.059397in}}%
\pgfpathlineto{\pgfqpoint{3.727801in}{2.066868in}}%
\pgfpathlineto{\pgfqpoint{3.735818in}{2.074410in}}%
\pgfpathlineto{\pgfqpoint{3.743828in}{2.082019in}}%
\pgfpathclose%
\pgfusepath{fill}%
\end{pgfscope}%
\begin{pgfscope}%
\pgfpathrectangle{\pgfqpoint{1.150000in}{0.150000in}}{\pgfqpoint{5.700000in}{5.700000in}}%
\pgfusepath{clip}%
\pgfsetbuttcap%
\pgfsetroundjoin%
\definecolor{currentfill}{rgb}{0.271305,0.019942,0.347269}%
\pgfsetfillcolor{currentfill}%
\pgfsetfillopacity{0.700000}%
\pgfsetlinewidth{0.000000pt}%
\definecolor{currentstroke}{rgb}{0.000000,0.000000,0.000000}%
\pgfsetstrokecolor{currentstroke}%
\pgfsetdash{}{0pt}%
\pgfpathmoveto{\pgfqpoint{3.462324in}{2.118012in}}%
\pgfpathlineto{\pgfqpoint{3.475874in}{2.111182in}}%
\pgfpathlineto{\pgfqpoint{3.489426in}{2.104475in}}%
\pgfpathlineto{\pgfqpoint{3.502982in}{2.097890in}}%
\pgfpathlineto{\pgfqpoint{3.516542in}{2.091428in}}%
\pgfpathlineto{\pgfqpoint{3.508424in}{2.085373in}}%
\pgfpathlineto{\pgfqpoint{3.500297in}{2.079423in}}%
\pgfpathlineto{\pgfqpoint{3.492163in}{2.073579in}}%
\pgfpathlineto{\pgfqpoint{3.484021in}{2.067846in}}%
\pgfpathlineto{\pgfqpoint{3.470442in}{2.074627in}}%
\pgfpathlineto{\pgfqpoint{3.456865in}{2.081532in}}%
\pgfpathlineto{\pgfqpoint{3.443292in}{2.088558in}}%
\pgfpathlineto{\pgfqpoint{3.429722in}{2.095709in}}%
\pgfpathlineto{\pgfqpoint{3.437885in}{2.101115in}}%
\pgfpathlineto{\pgfqpoint{3.446039in}{2.106637in}}%
\pgfpathlineto{\pgfqpoint{3.454186in}{2.112270in}}%
\pgfpathlineto{\pgfqpoint{3.462324in}{2.118012in}}%
\pgfpathclose%
\pgfusepath{fill}%
\end{pgfscope}%
\begin{pgfscope}%
\pgfpathrectangle{\pgfqpoint{1.150000in}{0.150000in}}{\pgfqpoint{5.700000in}{5.700000in}}%
\pgfusepath{clip}%
\pgfsetbuttcap%
\pgfsetroundjoin%
\definecolor{currentfill}{rgb}{0.282623,0.140926,0.457517}%
\pgfsetfillcolor{currentfill}%
\pgfsetfillopacity{0.700000}%
\pgfsetlinewidth{0.000000pt}%
\definecolor{currentstroke}{rgb}{0.000000,0.000000,0.000000}%
\pgfsetstrokecolor{currentstroke}%
\pgfsetdash{}{0pt}%
\pgfpathmoveto{\pgfqpoint{4.542095in}{2.324275in}}%
\pgfpathlineto{\pgfqpoint{4.555905in}{2.325851in}}%
\pgfpathlineto{\pgfqpoint{4.569724in}{2.327530in}}%
\pgfpathlineto{\pgfqpoint{4.583553in}{2.329311in}}%
\pgfpathlineto{\pgfqpoint{4.597393in}{2.331195in}}%
\pgfpathlineto{\pgfqpoint{4.589668in}{2.321072in}}%
\pgfpathlineto{\pgfqpoint{4.581939in}{2.310910in}}%
\pgfpathlineto{\pgfqpoint{4.574204in}{2.300708in}}%
\pgfpathlineto{\pgfqpoint{4.566463in}{2.290467in}}%
\pgfpathlineto{\pgfqpoint{4.552617in}{2.288721in}}%
\pgfpathlineto{\pgfqpoint{4.538781in}{2.287078in}}%
\pgfpathlineto{\pgfqpoint{4.524955in}{2.285537in}}%
\pgfpathlineto{\pgfqpoint{4.511139in}{2.284099in}}%
\pgfpathlineto{\pgfqpoint{4.518886in}{2.294195in}}%
\pgfpathlineto{\pgfqpoint{4.526628in}{2.304257in}}%
\pgfpathlineto{\pgfqpoint{4.534364in}{2.314284in}}%
\pgfpathlineto{\pgfqpoint{4.542095in}{2.324275in}}%
\pgfpathclose%
\pgfusepath{fill}%
\end{pgfscope}%
\begin{pgfscope}%
\pgfpathrectangle{\pgfqpoint{1.150000in}{0.150000in}}{\pgfqpoint{5.700000in}{5.700000in}}%
\pgfusepath{clip}%
\pgfsetbuttcap%
\pgfsetroundjoin%
\definecolor{currentfill}{rgb}{0.283197,0.115680,0.436115}%
\pgfsetfillcolor{currentfill}%
\pgfsetfillopacity{0.700000}%
\pgfsetlinewidth{0.000000pt}%
\definecolor{currentstroke}{rgb}{0.000000,0.000000,0.000000}%
\pgfsetstrokecolor{currentstroke}%
\pgfsetdash{}{0pt}%
\pgfpathmoveto{\pgfqpoint{4.455973in}{2.279379in}}%
\pgfpathlineto{\pgfqpoint{4.469750in}{2.280404in}}%
\pgfpathlineto{\pgfqpoint{4.483537in}{2.281532in}}%
\pgfpathlineto{\pgfqpoint{4.497333in}{2.282764in}}%
\pgfpathlineto{\pgfqpoint{4.511139in}{2.284099in}}%
\pgfpathlineto{\pgfqpoint{4.503387in}{2.273971in}}%
\pgfpathlineto{\pgfqpoint{4.495629in}{2.263812in}}%
\pgfpathlineto{\pgfqpoint{4.487866in}{2.253623in}}%
\pgfpathlineto{\pgfqpoint{4.480098in}{2.243405in}}%
\pgfpathlineto{\pgfqpoint{4.466285in}{2.242226in}}%
\pgfpathlineto{\pgfqpoint{4.452482in}{2.241150in}}%
\pgfpathlineto{\pgfqpoint{4.438688in}{2.240178in}}%
\pgfpathlineto{\pgfqpoint{4.424904in}{2.239309in}}%
\pgfpathlineto{\pgfqpoint{4.432679in}{2.249364in}}%
\pgfpathlineto{\pgfqpoint{4.440449in}{2.259395in}}%
\pgfpathlineto{\pgfqpoint{4.448214in}{2.269400in}}%
\pgfpathlineto{\pgfqpoint{4.455973in}{2.279379in}}%
\pgfpathclose%
\pgfusepath{fill}%
\end{pgfscope}%
\begin{pgfscope}%
\pgfpathrectangle{\pgfqpoint{1.150000in}{0.150000in}}{\pgfqpoint{5.700000in}{5.700000in}}%
\pgfusepath{clip}%
\pgfsetbuttcap%
\pgfsetroundjoin%
\definecolor{currentfill}{rgb}{0.188923,0.410910,0.556326}%
\pgfsetfillcolor{currentfill}%
\pgfsetfillopacity{0.700000}%
\pgfsetlinewidth{0.000000pt}%
\definecolor{currentstroke}{rgb}{0.000000,0.000000,0.000000}%
\pgfsetstrokecolor{currentstroke}%
\pgfsetdash{}{0pt}%
\pgfpathmoveto{\pgfqpoint{5.520679in}{2.926905in}}%
\pgfpathlineto{\pgfqpoint{5.534929in}{2.933118in}}%
\pgfpathlineto{\pgfqpoint{5.549193in}{2.939430in}}%
\pgfpathlineto{\pgfqpoint{5.563471in}{2.945841in}}%
\pgfpathlineto{\pgfqpoint{5.577764in}{2.952352in}}%
\pgfpathlineto{\pgfqpoint{5.570447in}{2.945527in}}%
\pgfpathlineto{\pgfqpoint{5.563122in}{2.938614in}}%
\pgfpathlineto{\pgfqpoint{5.555789in}{2.931612in}}%
\pgfpathlineto{\pgfqpoint{5.548448in}{2.924520in}}%
\pgfpathlineto{\pgfqpoint{5.534143in}{2.917921in}}%
\pgfpathlineto{\pgfqpoint{5.519853in}{2.911422in}}%
\pgfpathlineto{\pgfqpoint{5.505577in}{2.905022in}}%
\pgfpathlineto{\pgfqpoint{5.491315in}{2.898722in}}%
\pgfpathlineto{\pgfqpoint{5.498668in}{2.905894in}}%
\pgfpathlineto{\pgfqpoint{5.506013in}{2.912982in}}%
\pgfpathlineto{\pgfqpoint{5.513350in}{2.919985in}}%
\pgfpathlineto{\pgfqpoint{5.520679in}{2.926905in}}%
\pgfpathclose%
\pgfusepath{fill}%
\end{pgfscope}%
\begin{pgfscope}%
\pgfpathrectangle{\pgfqpoint{1.150000in}{0.150000in}}{\pgfqpoint{5.700000in}{5.700000in}}%
\pgfusepath{clip}%
\pgfsetbuttcap%
\pgfsetroundjoin%
\definecolor{currentfill}{rgb}{0.280255,0.165693,0.476498}%
\pgfsetfillcolor{currentfill}%
\pgfsetfillopacity{0.700000}%
\pgfsetlinewidth{0.000000pt}%
\definecolor{currentstroke}{rgb}{0.000000,0.000000,0.000000}%
\pgfsetstrokecolor{currentstroke}%
\pgfsetdash{}{0pt}%
\pgfpathmoveto{\pgfqpoint{4.628236in}{2.371260in}}%
\pgfpathlineto{\pgfqpoint{4.642079in}{2.373366in}}%
\pgfpathlineto{\pgfqpoint{4.655933in}{2.375575in}}%
\pgfpathlineto{\pgfqpoint{4.669797in}{2.377885in}}%
\pgfpathlineto{\pgfqpoint{4.683672in}{2.380298in}}%
\pgfpathlineto{\pgfqpoint{4.675976in}{2.370234in}}%
\pgfpathlineto{\pgfqpoint{4.668274in}{2.360122in}}%
\pgfpathlineto{\pgfqpoint{4.660567in}{2.349961in}}%
\pgfpathlineto{\pgfqpoint{4.652855in}{2.339754in}}%
\pgfpathlineto{\pgfqpoint{4.638973in}{2.337461in}}%
\pgfpathlineto{\pgfqpoint{4.625103in}{2.335270in}}%
\pgfpathlineto{\pgfqpoint{4.611242in}{2.333181in}}%
\pgfpathlineto{\pgfqpoint{4.597393in}{2.331195in}}%
\pgfpathlineto{\pgfqpoint{4.605111in}{2.341276in}}%
\pgfpathlineto{\pgfqpoint{4.612825in}{2.351314in}}%
\pgfpathlineto{\pgfqpoint{4.620533in}{2.361310in}}%
\pgfpathlineto{\pgfqpoint{4.628236in}{2.371260in}}%
\pgfpathclose%
\pgfusepath{fill}%
\end{pgfscope}%
\begin{pgfscope}%
\pgfpathrectangle{\pgfqpoint{1.150000in}{0.150000in}}{\pgfqpoint{5.700000in}{5.700000in}}%
\pgfusepath{clip}%
\pgfsetbuttcap%
\pgfsetroundjoin%
\definecolor{currentfill}{rgb}{0.282327,0.094955,0.417331}%
\pgfsetfillcolor{currentfill}%
\pgfsetfillopacity{0.700000}%
\pgfsetlinewidth{0.000000pt}%
\definecolor{currentstroke}{rgb}{0.000000,0.000000,0.000000}%
\pgfsetstrokecolor{currentstroke}%
\pgfsetdash{}{0pt}%
\pgfpathmoveto{\pgfqpoint{4.369860in}{2.236873in}}%
\pgfpathlineto{\pgfqpoint{4.383607in}{2.237325in}}%
\pgfpathlineto{\pgfqpoint{4.397364in}{2.237883in}}%
\pgfpathlineto{\pgfqpoint{4.411129in}{2.238544in}}%
\pgfpathlineto{\pgfqpoint{4.424904in}{2.239309in}}%
\pgfpathlineto{\pgfqpoint{4.417124in}{2.229231in}}%
\pgfpathlineto{\pgfqpoint{4.409338in}{2.219132in}}%
\pgfpathlineto{\pgfqpoint{4.401547in}{2.209014in}}%
\pgfpathlineto{\pgfqpoint{4.393751in}{2.198877in}}%
\pgfpathlineto{\pgfqpoint{4.379969in}{2.198286in}}%
\pgfpathlineto{\pgfqpoint{4.366196in}{2.197798in}}%
\pgfpathlineto{\pgfqpoint{4.352432in}{2.197415in}}%
\pgfpathlineto{\pgfqpoint{4.338677in}{2.197136in}}%
\pgfpathlineto{\pgfqpoint{4.346481in}{2.207092in}}%
\pgfpathlineto{\pgfqpoint{4.354279in}{2.217035in}}%
\pgfpathlineto{\pgfqpoint{4.362072in}{2.226962in}}%
\pgfpathlineto{\pgfqpoint{4.369860in}{2.236873in}}%
\pgfpathclose%
\pgfusepath{fill}%
\end{pgfscope}%
\begin{pgfscope}%
\pgfpathrectangle{\pgfqpoint{1.150000in}{0.150000in}}{\pgfqpoint{5.700000in}{5.700000in}}%
\pgfusepath{clip}%
\pgfsetbuttcap%
\pgfsetroundjoin%
\definecolor{currentfill}{rgb}{0.276194,0.190074,0.493001}%
\pgfsetfillcolor{currentfill}%
\pgfsetfillopacity{0.700000}%
\pgfsetlinewidth{0.000000pt}%
\definecolor{currentstroke}{rgb}{0.000000,0.000000,0.000000}%
\pgfsetstrokecolor{currentstroke}%
\pgfsetdash{}{0pt}%
\pgfpathmoveto{\pgfqpoint{4.714400in}{2.420047in}}%
\pgfpathlineto{\pgfqpoint{4.728279in}{2.422663in}}%
\pgfpathlineto{\pgfqpoint{4.742170in}{2.425381in}}%
\pgfpathlineto{\pgfqpoint{4.756071in}{2.428200in}}%
\pgfpathlineto{\pgfqpoint{4.769983in}{2.431121in}}%
\pgfpathlineto{\pgfqpoint{4.762316in}{2.421167in}}%
\pgfpathlineto{\pgfqpoint{4.754643in}{2.411155in}}%
\pgfpathlineto{\pgfqpoint{4.746965in}{2.401088in}}%
\pgfpathlineto{\pgfqpoint{4.739281in}{2.390966in}}%
\pgfpathlineto{\pgfqpoint{4.725362in}{2.388147in}}%
\pgfpathlineto{\pgfqpoint{4.711454in}{2.385429in}}%
\pgfpathlineto{\pgfqpoint{4.697558in}{2.382812in}}%
\pgfpathlineto{\pgfqpoint{4.683672in}{2.380298in}}%
\pgfpathlineto{\pgfqpoint{4.691362in}{2.390312in}}%
\pgfpathlineto{\pgfqpoint{4.699047in}{2.400275in}}%
\pgfpathlineto{\pgfqpoint{4.706726in}{2.410187in}}%
\pgfpathlineto{\pgfqpoint{4.714400in}{2.420047in}}%
\pgfpathclose%
\pgfusepath{fill}%
\end{pgfscope}%
\begin{pgfscope}%
\pgfpathrectangle{\pgfqpoint{1.150000in}{0.150000in}}{\pgfqpoint{5.700000in}{5.700000in}}%
\pgfusepath{clip}%
\pgfsetbuttcap%
\pgfsetroundjoin%
\definecolor{currentfill}{rgb}{0.180629,0.429975,0.557282}%
\pgfsetfillcolor{currentfill}%
\pgfsetfillopacity{0.700000}%
\pgfsetlinewidth{0.000000pt}%
\definecolor{currentstroke}{rgb}{0.000000,0.000000,0.000000}%
\pgfsetstrokecolor{currentstroke}%
\pgfsetdash{}{0pt}%
\pgfpathmoveto{\pgfqpoint{5.606953in}{2.978804in}}%
\pgfpathlineto{\pgfqpoint{5.621247in}{2.985307in}}%
\pgfpathlineto{\pgfqpoint{5.635556in}{2.991909in}}%
\pgfpathlineto{\pgfqpoint{5.649880in}{2.998610in}}%
\pgfpathlineto{\pgfqpoint{5.664219in}{3.005411in}}%
\pgfpathlineto{\pgfqpoint{5.656947in}{2.999037in}}%
\pgfpathlineto{\pgfqpoint{5.649667in}{2.992576in}}%
\pgfpathlineto{\pgfqpoint{5.642379in}{2.986026in}}%
\pgfpathlineto{\pgfqpoint{5.635083in}{2.979388in}}%
\pgfpathlineto{\pgfqpoint{5.620731in}{2.972480in}}%
\pgfpathlineto{\pgfqpoint{5.606394in}{2.965671in}}%
\pgfpathlineto{\pgfqpoint{5.592071in}{2.958962in}}%
\pgfpathlineto{\pgfqpoint{5.577764in}{2.952352in}}%
\pgfpathlineto{\pgfqpoint{5.585073in}{2.959091in}}%
\pgfpathlineto{\pgfqpoint{5.592374in}{2.965745in}}%
\pgfpathlineto{\pgfqpoint{5.599668in}{2.972316in}}%
\pgfpathlineto{\pgfqpoint{5.606953in}{2.978804in}}%
\pgfpathclose%
\pgfusepath{fill}%
\end{pgfscope}%
\begin{pgfscope}%
\pgfpathrectangle{\pgfqpoint{1.150000in}{0.150000in}}{\pgfqpoint{5.700000in}{5.700000in}}%
\pgfusepath{clip}%
\pgfsetbuttcap%
\pgfsetroundjoin%
\definecolor{currentfill}{rgb}{0.280267,0.073417,0.397163}%
\pgfsetfillcolor{currentfill}%
\pgfsetfillopacity{0.700000}%
\pgfsetlinewidth{0.000000pt}%
\definecolor{currentstroke}{rgb}{0.000000,0.000000,0.000000}%
\pgfsetstrokecolor{currentstroke}%
\pgfsetdash{}{0pt}%
\pgfpathmoveto{\pgfqpoint{4.283745in}{2.197068in}}%
\pgfpathlineto{\pgfqpoint{4.297465in}{2.196927in}}%
\pgfpathlineto{\pgfqpoint{4.311194in}{2.196892in}}%
\pgfpathlineto{\pgfqpoint{4.324931in}{2.196962in}}%
\pgfpathlineto{\pgfqpoint{4.338677in}{2.197136in}}%
\pgfpathlineto{\pgfqpoint{4.330868in}{2.187168in}}%
\pgfpathlineto{\pgfqpoint{4.323054in}{2.177189in}}%
\pgfpathlineto{\pgfqpoint{4.315235in}{2.167201in}}%
\pgfpathlineto{\pgfqpoint{4.307411in}{2.157206in}}%
\pgfpathlineto{\pgfqpoint{4.293657in}{2.157224in}}%
\pgfpathlineto{\pgfqpoint{4.279912in}{2.157346in}}%
\pgfpathlineto{\pgfqpoint{4.266175in}{2.157573in}}%
\pgfpathlineto{\pgfqpoint{4.252447in}{2.157905in}}%
\pgfpathlineto{\pgfqpoint{4.260279in}{2.167702in}}%
\pgfpathlineto{\pgfqpoint{4.268107in}{2.177495in}}%
\pgfpathlineto{\pgfqpoint{4.275929in}{2.187285in}}%
\pgfpathlineto{\pgfqpoint{4.283745in}{2.197068in}}%
\pgfpathclose%
\pgfusepath{fill}%
\end{pgfscope}%
\begin{pgfscope}%
\pgfpathrectangle{\pgfqpoint{1.150000in}{0.150000in}}{\pgfqpoint{5.700000in}{5.700000in}}%
\pgfusepath{clip}%
\pgfsetbuttcap%
\pgfsetroundjoin%
\definecolor{currentfill}{rgb}{0.283229,0.120777,0.440584}%
\pgfsetfillcolor{currentfill}%
\pgfsetfillopacity{0.700000}%
\pgfsetlinewidth{0.000000pt}%
\definecolor{currentstroke}{rgb}{0.000000,0.000000,0.000000}%
\pgfsetstrokecolor{currentstroke}%
\pgfsetdash{}{0pt}%
\pgfpathmoveto{\pgfqpoint{3.071311in}{2.293940in}}%
\pgfpathlineto{\pgfqpoint{3.084871in}{2.283265in}}%
\pgfpathlineto{\pgfqpoint{3.098431in}{2.272732in}}%
\pgfpathlineto{\pgfqpoint{3.111991in}{2.262341in}}%
\pgfpathlineto{\pgfqpoint{3.125551in}{2.252090in}}%
\pgfpathlineto{\pgfqpoint{3.117220in}{2.249067in}}%
\pgfpathlineto{\pgfqpoint{3.108878in}{2.246202in}}%
\pgfpathlineto{\pgfqpoint{3.100525in}{2.243499in}}%
\pgfpathlineto{\pgfqpoint{3.092160in}{2.240961in}}%
\pgfpathlineto{\pgfqpoint{3.078572in}{2.251573in}}%
\pgfpathlineto{\pgfqpoint{3.064983in}{2.262327in}}%
\pgfpathlineto{\pgfqpoint{3.051394in}{2.273222in}}%
\pgfpathlineto{\pgfqpoint{3.037805in}{2.284261in}}%
\pgfpathlineto{\pgfqpoint{3.046198in}{2.286429in}}%
\pgfpathlineto{\pgfqpoint{3.054580in}{2.288767in}}%
\pgfpathlineto{\pgfqpoint{3.062951in}{2.291272in}}%
\pgfpathlineto{\pgfqpoint{3.071311in}{2.293940in}}%
\pgfpathclose%
\pgfusepath{fill}%
\end{pgfscope}%
\begin{pgfscope}%
\pgfpathrectangle{\pgfqpoint{1.150000in}{0.150000in}}{\pgfqpoint{5.700000in}{5.700000in}}%
\pgfusepath{clip}%
\pgfsetbuttcap%
\pgfsetroundjoin%
\definecolor{currentfill}{rgb}{0.270595,0.214069,0.507052}%
\pgfsetfillcolor{currentfill}%
\pgfsetfillopacity{0.700000}%
\pgfsetlinewidth{0.000000pt}%
\definecolor{currentstroke}{rgb}{0.000000,0.000000,0.000000}%
\pgfsetstrokecolor{currentstroke}%
\pgfsetdash{}{0pt}%
\pgfpathmoveto{\pgfqpoint{4.800594in}{2.470361in}}%
\pgfpathlineto{\pgfqpoint{4.814511in}{2.473466in}}%
\pgfpathlineto{\pgfqpoint{4.828440in}{2.476673in}}%
\pgfpathlineto{\pgfqpoint{4.842379in}{2.479981in}}%
\pgfpathlineto{\pgfqpoint{4.856331in}{2.483390in}}%
\pgfpathlineto{\pgfqpoint{4.848693in}{2.473592in}}%
\pgfpathlineto{\pgfqpoint{4.841050in}{2.463730in}}%
\pgfpathlineto{\pgfqpoint{4.833401in}{2.453806in}}%
\pgfpathlineto{\pgfqpoint{4.825746in}{2.443818in}}%
\pgfpathlineto{\pgfqpoint{4.811788in}{2.440492in}}%
\pgfpathlineto{\pgfqpoint{4.797842in}{2.437267in}}%
\pgfpathlineto{\pgfqpoint{4.783907in}{2.434143in}}%
\pgfpathlineto{\pgfqpoint{4.769983in}{2.431121in}}%
\pgfpathlineto{\pgfqpoint{4.777645in}{2.441018in}}%
\pgfpathlineto{\pgfqpoint{4.785300in}{2.450858in}}%
\pgfpathlineto{\pgfqpoint{4.792950in}{2.460639in}}%
\pgfpathlineto{\pgfqpoint{4.800594in}{2.470361in}}%
\pgfpathclose%
\pgfusepath{fill}%
\end{pgfscope}%
\begin{pgfscope}%
\pgfpathrectangle{\pgfqpoint{1.150000in}{0.150000in}}{\pgfqpoint{5.700000in}{5.700000in}}%
\pgfusepath{clip}%
\pgfsetbuttcap%
\pgfsetroundjoin%
\definecolor{currentfill}{rgb}{0.268510,0.009605,0.335427}%
\pgfsetfillcolor{currentfill}%
\pgfsetfillopacity{0.700000}%
\pgfsetlinewidth{0.000000pt}%
\definecolor{currentstroke}{rgb}{0.000000,0.000000,0.000000}%
\pgfsetstrokecolor{currentstroke}%
\pgfsetdash{}{0pt}%
\pgfpathmoveto{\pgfqpoint{3.884499in}{2.083624in}}%
\pgfpathlineto{\pgfqpoint{3.898112in}{2.080447in}}%
\pgfpathlineto{\pgfqpoint{3.911732in}{2.077382in}}%
\pgfpathlineto{\pgfqpoint{3.925357in}{2.074428in}}%
\pgfpathlineto{\pgfqpoint{3.938990in}{2.071584in}}%
\pgfpathlineto{\pgfqpoint{3.931045in}{2.062958in}}%
\pgfpathlineto{\pgfqpoint{3.923095in}{2.054376in}}%
\pgfpathlineto{\pgfqpoint{3.915139in}{2.045841in}}%
\pgfpathlineto{\pgfqpoint{3.907177in}{2.037355in}}%
\pgfpathlineto{\pgfqpoint{3.893532in}{2.040462in}}%
\pgfpathlineto{\pgfqpoint{3.879894in}{2.043680in}}%
\pgfpathlineto{\pgfqpoint{3.866261in}{2.047009in}}%
\pgfpathlineto{\pgfqpoint{3.852635in}{2.050449in}}%
\pgfpathlineto{\pgfqpoint{3.860610in}{2.058664in}}%
\pgfpathlineto{\pgfqpoint{3.868579in}{2.066934in}}%
\pgfpathlineto{\pgfqpoint{3.876542in}{2.075254in}}%
\pgfpathlineto{\pgfqpoint{3.884499in}{2.083624in}}%
\pgfpathclose%
\pgfusepath{fill}%
\end{pgfscope}%
\begin{pgfscope}%
\pgfpathrectangle{\pgfqpoint{1.150000in}{0.150000in}}{\pgfqpoint{5.700000in}{5.700000in}}%
\pgfusepath{clip}%
\pgfsetbuttcap%
\pgfsetroundjoin%
\definecolor{currentfill}{rgb}{0.171176,0.452530,0.557965}%
\pgfsetfillcolor{currentfill}%
\pgfsetfillopacity{0.700000}%
\pgfsetlinewidth{0.000000pt}%
\definecolor{currentstroke}{rgb}{0.000000,0.000000,0.000000}%
\pgfsetstrokecolor{currentstroke}%
\pgfsetdash{}{0pt}%
\pgfpathmoveto{\pgfqpoint{5.693226in}{3.030068in}}%
\pgfpathlineto{\pgfqpoint{5.707565in}{3.036841in}}%
\pgfpathlineto{\pgfqpoint{5.721919in}{3.043713in}}%
\pgfpathlineto{\pgfqpoint{5.736289in}{3.050684in}}%
\pgfpathlineto{\pgfqpoint{5.750674in}{3.057754in}}%
\pgfpathlineto{\pgfqpoint{5.743449in}{3.051846in}}%
\pgfpathlineto{\pgfqpoint{5.736216in}{3.045853in}}%
\pgfpathlineto{\pgfqpoint{5.728975in}{3.039773in}}%
\pgfpathlineto{\pgfqpoint{5.721725in}{3.033604in}}%
\pgfpathlineto{\pgfqpoint{5.707326in}{3.026407in}}%
\pgfpathlineto{\pgfqpoint{5.692942in}{3.019309in}}%
\pgfpathlineto{\pgfqpoint{5.678573in}{3.012310in}}%
\pgfpathlineto{\pgfqpoint{5.664219in}{3.005411in}}%
\pgfpathlineto{\pgfqpoint{5.671483in}{3.011699in}}%
\pgfpathlineto{\pgfqpoint{5.678739in}{3.017904in}}%
\pgfpathlineto{\pgfqpoint{5.685986in}{3.024026in}}%
\pgfpathlineto{\pgfqpoint{5.693226in}{3.030068in}}%
\pgfpathclose%
\pgfusepath{fill}%
\end{pgfscope}%
\begin{pgfscope}%
\pgfpathrectangle{\pgfqpoint{1.150000in}{0.150000in}}{\pgfqpoint{5.700000in}{5.700000in}}%
\pgfusepath{clip}%
\pgfsetbuttcap%
\pgfsetroundjoin%
\definecolor{currentfill}{rgb}{0.277941,0.056324,0.381191}%
\pgfsetfillcolor{currentfill}%
\pgfsetfillopacity{0.700000}%
\pgfsetlinewidth{0.000000pt}%
\definecolor{currentstroke}{rgb}{0.000000,0.000000,0.000000}%
\pgfsetstrokecolor{currentstroke}%
\pgfsetdash{}{0pt}%
\pgfpathmoveto{\pgfqpoint{4.197617in}{2.160291in}}%
\pgfpathlineto{\pgfqpoint{4.211312in}{2.159535in}}%
\pgfpathlineto{\pgfqpoint{4.225015in}{2.158886in}}%
\pgfpathlineto{\pgfqpoint{4.238727in}{2.158343in}}%
\pgfpathlineto{\pgfqpoint{4.252447in}{2.157905in}}%
\pgfpathlineto{\pgfqpoint{4.244609in}{2.148108in}}%
\pgfpathlineto{\pgfqpoint{4.236766in}{2.138312in}}%
\pgfpathlineto{\pgfqpoint{4.228918in}{2.128518in}}%
\pgfpathlineto{\pgfqpoint{4.221064in}{2.118730in}}%
\pgfpathlineto{\pgfqpoint{4.207336in}{2.119377in}}%
\pgfpathlineto{\pgfqpoint{4.193616in}{2.120130in}}%
\pgfpathlineto{\pgfqpoint{4.179903in}{2.120989in}}%
\pgfpathlineto{\pgfqpoint{4.166199in}{2.121954in}}%
\pgfpathlineto{\pgfqpoint{4.174062in}{2.131526in}}%
\pgfpathlineto{\pgfqpoint{4.181919in}{2.141108in}}%
\pgfpathlineto{\pgfqpoint{4.189771in}{2.150696in}}%
\pgfpathlineto{\pgfqpoint{4.197617in}{2.160291in}}%
\pgfpathclose%
\pgfusepath{fill}%
\end{pgfscope}%
\begin{pgfscope}%
\pgfpathrectangle{\pgfqpoint{1.150000in}{0.150000in}}{\pgfqpoint{5.700000in}{5.700000in}}%
\pgfusepath{clip}%
\pgfsetbuttcap%
\pgfsetroundjoin%
\definecolor{currentfill}{rgb}{0.277018,0.050344,0.375715}%
\pgfsetfillcolor{currentfill}%
\pgfsetfillopacity{0.700000}%
\pgfsetlinewidth{0.000000pt}%
\definecolor{currentstroke}{rgb}{0.000000,0.000000,0.000000}%
\pgfsetstrokecolor{currentstroke}%
\pgfsetdash{}{0pt}%
\pgfpathmoveto{\pgfqpoint{3.321254in}{2.157432in}}%
\pgfpathlineto{\pgfqpoint{3.334805in}{2.149271in}}%
\pgfpathlineto{\pgfqpoint{3.348357in}{2.141238in}}%
\pgfpathlineto{\pgfqpoint{3.361912in}{2.133334in}}%
\pgfpathlineto{\pgfqpoint{3.375469in}{2.125557in}}%
\pgfpathlineto{\pgfqpoint{3.367275in}{2.120600in}}%
\pgfpathlineto{\pgfqpoint{3.359073in}{2.115768in}}%
\pgfpathlineto{\pgfqpoint{3.350863in}{2.111066in}}%
\pgfpathlineto{\pgfqpoint{3.342643in}{2.106497in}}%
\pgfpathlineto{\pgfqpoint{3.329063in}{2.114613in}}%
\pgfpathlineto{\pgfqpoint{3.315485in}{2.122857in}}%
\pgfpathlineto{\pgfqpoint{3.301909in}{2.131229in}}%
\pgfpathlineto{\pgfqpoint{3.288335in}{2.139730in}}%
\pgfpathlineto{\pgfqpoint{3.296579in}{2.143953in}}%
\pgfpathlineto{\pgfqpoint{3.304813in}{2.148314in}}%
\pgfpathlineto{\pgfqpoint{3.313038in}{2.152808in}}%
\pgfpathlineto{\pgfqpoint{3.321254in}{2.157432in}}%
\pgfpathclose%
\pgfusepath{fill}%
\end{pgfscope}%
\begin{pgfscope}%
\pgfpathrectangle{\pgfqpoint{1.150000in}{0.150000in}}{\pgfqpoint{5.700000in}{5.700000in}}%
\pgfusepath{clip}%
\pgfsetbuttcap%
\pgfsetroundjoin%
\definecolor{currentfill}{rgb}{0.263663,0.237631,0.518762}%
\pgfsetfillcolor{currentfill}%
\pgfsetfillopacity{0.700000}%
\pgfsetlinewidth{0.000000pt}%
\definecolor{currentstroke}{rgb}{0.000000,0.000000,0.000000}%
\pgfsetstrokecolor{currentstroke}%
\pgfsetdash{}{0pt}%
\pgfpathmoveto{\pgfqpoint{4.886821in}{2.521937in}}%
\pgfpathlineto{\pgfqpoint{4.900777in}{2.525512in}}%
\pgfpathlineto{\pgfqpoint{4.914746in}{2.529187in}}%
\pgfpathlineto{\pgfqpoint{4.928726in}{2.532964in}}%
\pgfpathlineto{\pgfqpoint{4.942717in}{2.536841in}}%
\pgfpathlineto{\pgfqpoint{4.935111in}{2.527244in}}%
\pgfpathlineto{\pgfqpoint{4.927498in}{2.517578in}}%
\pgfpathlineto{\pgfqpoint{4.919879in}{2.507842in}}%
\pgfpathlineto{\pgfqpoint{4.912254in}{2.498036in}}%
\pgfpathlineto{\pgfqpoint{4.898255in}{2.494223in}}%
\pgfpathlineto{\pgfqpoint{4.884269in}{2.490511in}}%
\pgfpathlineto{\pgfqpoint{4.870294in}{2.486900in}}%
\pgfpathlineto{\pgfqpoint{4.856331in}{2.483390in}}%
\pgfpathlineto{\pgfqpoint{4.863962in}{2.493124in}}%
\pgfpathlineto{\pgfqpoint{4.871588in}{2.502793in}}%
\pgfpathlineto{\pgfqpoint{4.879208in}{2.512398in}}%
\pgfpathlineto{\pgfqpoint{4.886821in}{2.521937in}}%
\pgfpathclose%
\pgfusepath{fill}%
\end{pgfscope}%
\begin{pgfscope}%
\pgfpathrectangle{\pgfqpoint{1.150000in}{0.150000in}}{\pgfqpoint{5.700000in}{5.700000in}}%
\pgfusepath{clip}%
\pgfsetbuttcap%
\pgfsetroundjoin%
\definecolor{currentfill}{rgb}{0.163625,0.471133,0.558148}%
\pgfsetfillcolor{currentfill}%
\pgfsetfillopacity{0.700000}%
\pgfsetlinewidth{0.000000pt}%
\definecolor{currentstroke}{rgb}{0.000000,0.000000,0.000000}%
\pgfsetstrokecolor{currentstroke}%
\pgfsetdash{}{0pt}%
\pgfpathmoveto{\pgfqpoint{5.779490in}{3.080566in}}%
\pgfpathlineto{\pgfqpoint{5.793874in}{3.087589in}}%
\pgfpathlineto{\pgfqpoint{5.808274in}{3.094711in}}%
\pgfpathlineto{\pgfqpoint{5.822689in}{3.101931in}}%
\pgfpathlineto{\pgfqpoint{5.837120in}{3.109251in}}%
\pgfpathlineto{\pgfqpoint{5.829944in}{3.103821in}}%
\pgfpathlineto{\pgfqpoint{5.822760in}{3.098307in}}%
\pgfpathlineto{\pgfqpoint{5.815568in}{3.092709in}}%
\pgfpathlineto{\pgfqpoint{5.808366in}{3.087025in}}%
\pgfpathlineto{\pgfqpoint{5.793920in}{3.079558in}}%
\pgfpathlineto{\pgfqpoint{5.779489in}{3.072191in}}%
\pgfpathlineto{\pgfqpoint{5.765074in}{3.064923in}}%
\pgfpathlineto{\pgfqpoint{5.750674in}{3.057754in}}%
\pgfpathlineto{\pgfqpoint{5.757890in}{3.063578in}}%
\pgfpathlineto{\pgfqpoint{5.765098in}{3.069320in}}%
\pgfpathlineto{\pgfqpoint{5.772298in}{3.074982in}}%
\pgfpathlineto{\pgfqpoint{5.779490in}{3.080566in}}%
\pgfpathclose%
\pgfusepath{fill}%
\end{pgfscope}%
\begin{pgfscope}%
\pgfpathrectangle{\pgfqpoint{1.150000in}{0.150000in}}{\pgfqpoint{5.700000in}{5.700000in}}%
\pgfusepath{clip}%
\pgfsetbuttcap%
\pgfsetroundjoin%
\definecolor{currentfill}{rgb}{0.156270,0.489624,0.557936}%
\pgfsetfillcolor{currentfill}%
\pgfsetfillopacity{0.700000}%
\pgfsetlinewidth{0.000000pt}%
\definecolor{currentstroke}{rgb}{0.000000,0.000000,0.000000}%
\pgfsetstrokecolor{currentstroke}%
\pgfsetdash{}{0pt}%
\pgfpathmoveto{\pgfqpoint{5.865738in}{3.130178in}}%
\pgfpathlineto{\pgfqpoint{5.880168in}{3.137431in}}%
\pgfpathlineto{\pgfqpoint{5.894612in}{3.144782in}}%
\pgfpathlineto{\pgfqpoint{5.909073in}{3.152232in}}%
\pgfpathlineto{\pgfqpoint{5.923549in}{3.159781in}}%
\pgfpathlineto{\pgfqpoint{5.916425in}{3.154837in}}%
\pgfpathlineto{\pgfqpoint{5.909291in}{3.149814in}}%
\pgfpathlineto{\pgfqpoint{5.902149in}{3.144708in}}%
\pgfpathlineto{\pgfqpoint{5.894999in}{3.139518in}}%
\pgfpathlineto{\pgfqpoint{5.880505in}{3.131803in}}%
\pgfpathlineto{\pgfqpoint{5.866028in}{3.124186in}}%
\pgfpathlineto{\pgfqpoint{5.851566in}{3.116669in}}%
\pgfpathlineto{\pgfqpoint{5.837120in}{3.109251in}}%
\pgfpathlineto{\pgfqpoint{5.844287in}{3.114599in}}%
\pgfpathlineto{\pgfqpoint{5.851446in}{3.119869in}}%
\pgfpathlineto{\pgfqpoint{5.858596in}{3.125061in}}%
\pgfpathlineto{\pgfqpoint{5.865738in}{3.130178in}}%
\pgfpathclose%
\pgfusepath{fill}%
\end{pgfscope}%
\begin{pgfscope}%
\pgfpathrectangle{\pgfqpoint{1.150000in}{0.150000in}}{\pgfqpoint{5.700000in}{5.700000in}}%
\pgfusepath{clip}%
\pgfsetbuttcap%
\pgfsetroundjoin%
\definecolor{currentfill}{rgb}{0.253935,0.265254,0.529983}%
\pgfsetfillcolor{currentfill}%
\pgfsetfillopacity{0.700000}%
\pgfsetlinewidth{0.000000pt}%
\definecolor{currentstroke}{rgb}{0.000000,0.000000,0.000000}%
\pgfsetstrokecolor{currentstroke}%
\pgfsetdash{}{0pt}%
\pgfpathmoveto{\pgfqpoint{4.973083in}{2.574525in}}%
\pgfpathlineto{\pgfqpoint{4.987080in}{2.578549in}}%
\pgfpathlineto{\pgfqpoint{5.001089in}{2.582674in}}%
\pgfpathlineto{\pgfqpoint{5.015110in}{2.586899in}}%
\pgfpathlineto{\pgfqpoint{5.029144in}{2.591224in}}%
\pgfpathlineto{\pgfqpoint{5.021569in}{2.581870in}}%
\pgfpathlineto{\pgfqpoint{5.013988in}{2.572441in}}%
\pgfpathlineto{\pgfqpoint{5.006401in}{2.562937in}}%
\pgfpathlineto{\pgfqpoint{4.998807in}{2.553357in}}%
\pgfpathlineto{\pgfqpoint{4.984766in}{2.549077in}}%
\pgfpathlineto{\pgfqpoint{4.970738in}{2.544898in}}%
\pgfpathlineto{\pgfqpoint{4.956721in}{2.540819in}}%
\pgfpathlineto{\pgfqpoint{4.942717in}{2.536841in}}%
\pgfpathlineto{\pgfqpoint{4.950318in}{2.546368in}}%
\pgfpathlineto{\pgfqpoint{4.957912in}{2.555824in}}%
\pgfpathlineto{\pgfqpoint{4.965501in}{2.565210in}}%
\pgfpathlineto{\pgfqpoint{4.973083in}{2.574525in}}%
\pgfpathclose%
\pgfusepath{fill}%
\end{pgfscope}%
\begin{pgfscope}%
\pgfpathrectangle{\pgfqpoint{1.150000in}{0.150000in}}{\pgfqpoint{5.700000in}{5.700000in}}%
\pgfusepath{clip}%
\pgfsetbuttcap%
\pgfsetroundjoin%
\definecolor{currentfill}{rgb}{0.149039,0.508051,0.557250}%
\pgfsetfillcolor{currentfill}%
\pgfsetfillopacity{0.700000}%
\pgfsetlinewidth{0.000000pt}%
\definecolor{currentstroke}{rgb}{0.000000,0.000000,0.000000}%
\pgfsetstrokecolor{currentstroke}%
\pgfsetdash{}{0pt}%
\pgfpathmoveto{\pgfqpoint{5.951963in}{3.178798in}}%
\pgfpathlineto{\pgfqpoint{5.966437in}{3.186261in}}%
\pgfpathlineto{\pgfqpoint{5.980926in}{3.193821in}}%
\pgfpathlineto{\pgfqpoint{5.995432in}{3.201481in}}%
\pgfpathlineto{\pgfqpoint{6.009953in}{3.209239in}}%
\pgfpathlineto{\pgfqpoint{6.002882in}{3.204786in}}%
\pgfpathlineto{\pgfqpoint{5.995801in}{3.200258in}}%
\pgfpathlineto{\pgfqpoint{5.988712in}{3.195652in}}%
\pgfpathlineto{\pgfqpoint{5.981614in}{3.190965in}}%
\pgfpathlineto{\pgfqpoint{5.967074in}{3.183021in}}%
\pgfpathlineto{\pgfqpoint{5.952549in}{3.175175in}}%
\pgfpathlineto{\pgfqpoint{5.938041in}{3.167429in}}%
\pgfpathlineto{\pgfqpoint{5.923549in}{3.159781in}}%
\pgfpathlineto{\pgfqpoint{5.930665in}{3.164647in}}%
\pgfpathlineto{\pgfqpoint{5.937773in}{3.169437in}}%
\pgfpathlineto{\pgfqpoint{5.944872in}{3.174153in}}%
\pgfpathlineto{\pgfqpoint{5.951963in}{3.178798in}}%
\pgfpathclose%
\pgfusepath{fill}%
\end{pgfscope}%
\begin{pgfscope}%
\pgfpathrectangle{\pgfqpoint{1.150000in}{0.150000in}}{\pgfqpoint{5.700000in}{5.700000in}}%
\pgfusepath{clip}%
\pgfsetbuttcap%
\pgfsetroundjoin%
\definecolor{currentfill}{rgb}{0.274952,0.037752,0.364543}%
\pgfsetfillcolor{currentfill}%
\pgfsetfillopacity{0.700000}%
\pgfsetlinewidth{0.000000pt}%
\definecolor{currentstroke}{rgb}{0.000000,0.000000,0.000000}%
\pgfsetstrokecolor{currentstroke}%
\pgfsetdash{}{0pt}%
\pgfpathmoveto{\pgfqpoint{4.111460in}{2.126880in}}%
\pgfpathlineto{\pgfqpoint{4.125133in}{2.125488in}}%
\pgfpathlineto{\pgfqpoint{4.138814in}{2.124203in}}%
\pgfpathlineto{\pgfqpoint{4.152503in}{2.123025in}}%
\pgfpathlineto{\pgfqpoint{4.166199in}{2.121954in}}%
\pgfpathlineto{\pgfqpoint{4.158331in}{2.112393in}}%
\pgfpathlineto{\pgfqpoint{4.150458in}{2.102844in}}%
\pgfpathlineto{\pgfqpoint{4.142580in}{2.093311in}}%
\pgfpathlineto{\pgfqpoint{4.134696in}{2.083795in}}%
\pgfpathlineto{\pgfqpoint{4.120990in}{2.085095in}}%
\pgfpathlineto{\pgfqpoint{4.107292in}{2.086500in}}%
\pgfpathlineto{\pgfqpoint{4.093601in}{2.088013in}}%
\pgfpathlineto{\pgfqpoint{4.079918in}{2.089632in}}%
\pgfpathlineto{\pgfqpoint{4.087811in}{2.098914in}}%
\pgfpathlineto{\pgfqpoint{4.095700in}{2.108217in}}%
\pgfpathlineto{\pgfqpoint{4.103583in}{2.117540in}}%
\pgfpathlineto{\pgfqpoint{4.111460in}{2.126880in}}%
\pgfpathclose%
\pgfusepath{fill}%
\end{pgfscope}%
\begin{pgfscope}%
\pgfpathrectangle{\pgfqpoint{1.150000in}{0.150000in}}{\pgfqpoint{5.700000in}{5.700000in}}%
\pgfusepath{clip}%
\pgfsetbuttcap%
\pgfsetroundjoin%
\definecolor{currentfill}{rgb}{0.267004,0.004874,0.329415}%
\pgfsetfillcolor{currentfill}%
\pgfsetfillopacity{0.700000}%
\pgfsetlinewidth{0.000000pt}%
\definecolor{currentstroke}{rgb}{0.000000,0.000000,0.000000}%
\pgfsetstrokecolor{currentstroke}%
\pgfsetdash{}{0pt}%
\pgfpathmoveto{\pgfqpoint{3.657408in}{2.071658in}}%
\pgfpathlineto{\pgfqpoint{3.670986in}{2.066569in}}%
\pgfpathlineto{\pgfqpoint{3.684569in}{2.061596in}}%
\pgfpathlineto{\pgfqpoint{3.698156in}{2.056740in}}%
\pgfpathlineto{\pgfqpoint{3.711748in}{2.051999in}}%
\pgfpathlineto{\pgfqpoint{3.703712in}{2.044676in}}%
\pgfpathlineto{\pgfqpoint{3.695669in}{2.037433in}}%
\pgfpathlineto{\pgfqpoint{3.687618in}{2.030271in}}%
\pgfpathlineto{\pgfqpoint{3.679561in}{2.023193in}}%
\pgfpathlineto{\pgfqpoint{3.665953in}{2.028234in}}%
\pgfpathlineto{\pgfqpoint{3.652349in}{2.033391in}}%
\pgfpathlineto{\pgfqpoint{3.638749in}{2.038664in}}%
\pgfpathlineto{\pgfqpoint{3.625154in}{2.044053in}}%
\pgfpathlineto{\pgfqpoint{3.633228in}{2.050824in}}%
\pgfpathlineto{\pgfqpoint{3.641295in}{2.057684in}}%
\pgfpathlineto{\pgfqpoint{3.649355in}{2.064629in}}%
\pgfpathlineto{\pgfqpoint{3.657408in}{2.071658in}}%
\pgfpathclose%
\pgfusepath{fill}%
\end{pgfscope}%
\begin{pgfscope}%
\pgfpathrectangle{\pgfqpoint{1.150000in}{0.150000in}}{\pgfqpoint{5.700000in}{5.700000in}}%
\pgfusepath{clip}%
\pgfsetbuttcap%
\pgfsetroundjoin%
\definecolor{currentfill}{rgb}{0.141935,0.526453,0.555991}%
\pgfsetfillcolor{currentfill}%
\pgfsetfillopacity{0.700000}%
\pgfsetlinewidth{0.000000pt}%
\definecolor{currentstroke}{rgb}{0.000000,0.000000,0.000000}%
\pgfsetstrokecolor{currentstroke}%
\pgfsetdash{}{0pt}%
\pgfpathmoveto{\pgfqpoint{6.038156in}{3.226332in}}%
\pgfpathlineto{\pgfqpoint{6.052673in}{3.233984in}}%
\pgfpathlineto{\pgfqpoint{6.067207in}{3.241734in}}%
\pgfpathlineto{\pgfqpoint{6.081757in}{3.249582in}}%
\pgfpathlineto{\pgfqpoint{6.096324in}{3.257529in}}%
\pgfpathlineto{\pgfqpoint{6.089306in}{3.253571in}}%
\pgfpathlineto{\pgfqpoint{6.082280in}{3.249541in}}%
\pgfpathlineto{\pgfqpoint{6.075246in}{3.245438in}}%
\pgfpathlineto{\pgfqpoint{6.068202in}{3.241259in}}%
\pgfpathlineto{\pgfqpoint{6.053615in}{3.233105in}}%
\pgfpathlineto{\pgfqpoint{6.039045in}{3.225051in}}%
\pgfpathlineto{\pgfqpoint{6.024491in}{3.217095in}}%
\pgfpathlineto{\pgfqpoint{6.009953in}{3.209239in}}%
\pgfpathlineto{\pgfqpoint{6.017017in}{3.213617in}}%
\pgfpathlineto{\pgfqpoint{6.024071in}{3.217924in}}%
\pgfpathlineto{\pgfqpoint{6.031118in}{3.222161in}}%
\pgfpathlineto{\pgfqpoint{6.038156in}{3.226332in}}%
\pgfpathclose%
\pgfusepath{fill}%
\end{pgfscope}%
\begin{pgfscope}%
\pgfpathrectangle{\pgfqpoint{1.150000in}{0.150000in}}{\pgfqpoint{5.700000in}{5.700000in}}%
\pgfusepath{clip}%
\pgfsetbuttcap%
\pgfsetroundjoin%
\definecolor{currentfill}{rgb}{0.282910,0.105393,0.426902}%
\pgfsetfillcolor{currentfill}%
\pgfsetfillopacity{0.700000}%
\pgfsetlinewidth{0.000000pt}%
\definecolor{currentstroke}{rgb}{0.000000,0.000000,0.000000}%
\pgfsetstrokecolor{currentstroke}%
\pgfsetdash{}{0pt}%
\pgfpathmoveto{\pgfqpoint{3.125551in}{2.252090in}}%
\pgfpathlineto{\pgfqpoint{3.139112in}{2.241979in}}%
\pgfpathlineto{\pgfqpoint{3.152673in}{2.232007in}}%
\pgfpathlineto{\pgfqpoint{3.166235in}{2.222173in}}%
\pgfpathlineto{\pgfqpoint{3.179797in}{2.212476in}}%
\pgfpathlineto{\pgfqpoint{3.171493in}{2.209098in}}%
\pgfpathlineto{\pgfqpoint{3.163178in}{2.205874in}}%
\pgfpathlineto{\pgfqpoint{3.154853in}{2.202807in}}%
\pgfpathlineto{\pgfqpoint{3.146517in}{2.199901in}}%
\pgfpathlineto{\pgfqpoint{3.132927in}{2.209959in}}%
\pgfpathlineto{\pgfqpoint{3.119338in}{2.220154in}}%
\pgfpathlineto{\pgfqpoint{3.105749in}{2.230488in}}%
\pgfpathlineto{\pgfqpoint{3.092160in}{2.240961in}}%
\pgfpathlineto{\pgfqpoint{3.100525in}{2.243499in}}%
\pgfpathlineto{\pgfqpoint{3.108878in}{2.246202in}}%
\pgfpathlineto{\pgfqpoint{3.117220in}{2.249067in}}%
\pgfpathlineto{\pgfqpoint{3.125551in}{2.252090in}}%
\pgfpathclose%
\pgfusepath{fill}%
\end{pgfscope}%
\begin{pgfscope}%
\pgfpathrectangle{\pgfqpoint{1.150000in}{0.150000in}}{\pgfqpoint{5.700000in}{5.700000in}}%
\pgfusepath{clip}%
\pgfsetbuttcap%
\pgfsetroundjoin%
\definecolor{currentfill}{rgb}{0.135066,0.544853,0.554029}%
\pgfsetfillcolor{currentfill}%
\pgfsetfillopacity{0.700000}%
\pgfsetlinewidth{0.000000pt}%
\definecolor{currentstroke}{rgb}{0.000000,0.000000,0.000000}%
\pgfsetstrokecolor{currentstroke}%
\pgfsetdash{}{0pt}%
\pgfpathmoveto{\pgfqpoint{6.124308in}{3.272699in}}%
\pgfpathlineto{\pgfqpoint{6.138869in}{3.280519in}}%
\pgfpathlineto{\pgfqpoint{6.153447in}{3.288438in}}%
\pgfpathlineto{\pgfqpoint{6.168041in}{3.296455in}}%
\pgfpathlineto{\pgfqpoint{6.182652in}{3.304571in}}%
\pgfpathlineto{\pgfqpoint{6.175690in}{3.301105in}}%
\pgfpathlineto{\pgfqpoint{6.168720in}{3.297574in}}%
\pgfpathlineto{\pgfqpoint{6.161742in}{3.293975in}}%
\pgfpathlineto{\pgfqpoint{6.154754in}{3.290304in}}%
\pgfpathlineto{\pgfqpoint{6.140122in}{3.281962in}}%
\pgfpathlineto{\pgfqpoint{6.125506in}{3.273719in}}%
\pgfpathlineto{\pgfqpoint{6.110907in}{3.265575in}}%
\pgfpathlineto{\pgfqpoint{6.096324in}{3.257529in}}%
\pgfpathlineto{\pgfqpoint{6.103333in}{3.261418in}}%
\pgfpathlineto{\pgfqpoint{6.110333in}{3.265242in}}%
\pgfpathlineto{\pgfqpoint{6.117325in}{3.269001in}}%
\pgfpathlineto{\pgfqpoint{6.124308in}{3.272699in}}%
\pgfpathclose%
\pgfusepath{fill}%
\end{pgfscope}%
\begin{pgfscope}%
\pgfpathrectangle{\pgfqpoint{1.150000in}{0.150000in}}{\pgfqpoint{5.700000in}{5.700000in}}%
\pgfusepath{clip}%
\pgfsetbuttcap%
\pgfsetroundjoin%
\definecolor{currentfill}{rgb}{0.269944,0.014625,0.341379}%
\pgfsetfillcolor{currentfill}%
\pgfsetfillopacity{0.700000}%
\pgfsetlinewidth{0.000000pt}%
\definecolor{currentstroke}{rgb}{0.000000,0.000000,0.000000}%
\pgfsetstrokecolor{currentstroke}%
\pgfsetdash{}{0pt}%
\pgfpathmoveto{\pgfqpoint{3.516542in}{2.091428in}}%
\pgfpathlineto{\pgfqpoint{3.530105in}{2.085087in}}%
\pgfpathlineto{\pgfqpoint{3.543672in}{2.078867in}}%
\pgfpathlineto{\pgfqpoint{3.557242in}{2.072767in}}%
\pgfpathlineto{\pgfqpoint{3.570817in}{2.066787in}}%
\pgfpathlineto{\pgfqpoint{3.562717in}{2.060420in}}%
\pgfpathlineto{\pgfqpoint{3.554610in}{2.054153in}}%
\pgfpathlineto{\pgfqpoint{3.546496in}{2.047987in}}%
\pgfpathlineto{\pgfqpoint{3.538373in}{2.041928in}}%
\pgfpathlineto{\pgfqpoint{3.524780in}{2.048227in}}%
\pgfpathlineto{\pgfqpoint{3.511190in}{2.054646in}}%
\pgfpathlineto{\pgfqpoint{3.497604in}{2.061185in}}%
\pgfpathlineto{\pgfqpoint{3.484021in}{2.067846in}}%
\pgfpathlineto{\pgfqpoint{3.492163in}{2.073579in}}%
\pgfpathlineto{\pgfqpoint{3.500297in}{2.079423in}}%
\pgfpathlineto{\pgfqpoint{3.508424in}{2.085373in}}%
\pgfpathlineto{\pgfqpoint{3.516542in}{2.091428in}}%
\pgfpathclose%
\pgfusepath{fill}%
\end{pgfscope}%
\begin{pgfscope}%
\pgfpathrectangle{\pgfqpoint{1.150000in}{0.150000in}}{\pgfqpoint{5.700000in}{5.700000in}}%
\pgfusepath{clip}%
\pgfsetbuttcap%
\pgfsetroundjoin%
\definecolor{currentfill}{rgb}{0.243113,0.292092,0.538516}%
\pgfsetfillcolor{currentfill}%
\pgfsetfillopacity{0.700000}%
\pgfsetlinewidth{0.000000pt}%
\definecolor{currentstroke}{rgb}{0.000000,0.000000,0.000000}%
\pgfsetstrokecolor{currentstroke}%
\pgfsetdash{}{0pt}%
\pgfpathmoveto{\pgfqpoint{5.059379in}{2.627885in}}%
\pgfpathlineto{\pgfqpoint{5.073418in}{2.632339in}}%
\pgfpathlineto{\pgfqpoint{5.087470in}{2.636892in}}%
\pgfpathlineto{\pgfqpoint{5.101534in}{2.641546in}}%
\pgfpathlineto{\pgfqpoint{5.115611in}{2.646300in}}%
\pgfpathlineto{\pgfqpoint{5.108069in}{2.637228in}}%
\pgfpathlineto{\pgfqpoint{5.100521in}{2.628076in}}%
\pgfpathlineto{\pgfqpoint{5.092967in}{2.618844in}}%
\pgfpathlineto{\pgfqpoint{5.085405in}{2.609531in}}%
\pgfpathlineto{\pgfqpoint{5.071321in}{2.604804in}}%
\pgfpathlineto{\pgfqpoint{5.057249in}{2.600177in}}%
\pgfpathlineto{\pgfqpoint{5.043191in}{2.595651in}}%
\pgfpathlineto{\pgfqpoint{5.029144in}{2.591224in}}%
\pgfpathlineto{\pgfqpoint{5.036713in}{2.600503in}}%
\pgfpathlineto{\pgfqpoint{5.044275in}{2.609706in}}%
\pgfpathlineto{\pgfqpoint{5.051830in}{2.618833in}}%
\pgfpathlineto{\pgfqpoint{5.059379in}{2.627885in}}%
\pgfpathclose%
\pgfusepath{fill}%
\end{pgfscope}%
\begin{pgfscope}%
\pgfpathrectangle{\pgfqpoint{1.150000in}{0.150000in}}{\pgfqpoint{5.700000in}{5.700000in}}%
\pgfusepath{clip}%
\pgfsetbuttcap%
\pgfsetroundjoin%
\definecolor{currentfill}{rgb}{0.241237,0.296485,0.539709}%
\pgfsetfillcolor{currentfill}%
\pgfsetfillopacity{0.700000}%
\pgfsetlinewidth{0.000000pt}%
\definecolor{currentstroke}{rgb}{0.000000,0.000000,0.000000}%
\pgfsetstrokecolor{currentstroke}%
\pgfsetdash{}{0pt}%
\pgfpathmoveto{\pgfqpoint{2.656603in}{2.655827in}}%
\pgfpathlineto{\pgfqpoint{2.670260in}{2.640321in}}%
\pgfpathlineto{\pgfqpoint{2.683912in}{2.624992in}}%
\pgfpathlineto{\pgfqpoint{2.697560in}{2.609840in}}%
\pgfpathlineto{\pgfqpoint{2.711204in}{2.594862in}}%
\pgfpathlineto{\pgfqpoint{2.702597in}{2.595149in}}%
\pgfpathlineto{\pgfqpoint{2.693975in}{2.595644in}}%
\pgfpathlineto{\pgfqpoint{2.685339in}{2.596350in}}%
\pgfpathlineto{\pgfqpoint{2.676689in}{2.597271in}}%
\pgfpathlineto{\pgfqpoint{2.663007in}{2.612643in}}%
\pgfpathlineto{\pgfqpoint{2.649321in}{2.628189in}}%
\pgfpathlineto{\pgfqpoint{2.635630in}{2.643913in}}%
\pgfpathlineto{\pgfqpoint{2.621934in}{2.659815in}}%
\pgfpathlineto{\pgfqpoint{2.630625in}{2.658491in}}%
\pgfpathlineto{\pgfqpoint{2.639299in}{2.657387in}}%
\pgfpathlineto{\pgfqpoint{2.647959in}{2.656501in}}%
\pgfpathlineto{\pgfqpoint{2.656603in}{2.655827in}}%
\pgfpathclose%
\pgfusepath{fill}%
\end{pgfscope}%
\begin{pgfscope}%
\pgfpathrectangle{\pgfqpoint{1.150000in}{0.150000in}}{\pgfqpoint{5.700000in}{5.700000in}}%
\pgfusepath{clip}%
\pgfsetbuttcap%
\pgfsetroundjoin%
\definecolor{currentfill}{rgb}{0.229739,0.322361,0.545706}%
\pgfsetfillcolor{currentfill}%
\pgfsetfillopacity{0.700000}%
\pgfsetlinewidth{0.000000pt}%
\definecolor{currentstroke}{rgb}{0.000000,0.000000,0.000000}%
\pgfsetstrokecolor{currentstroke}%
\pgfsetdash{}{0pt}%
\pgfpathmoveto{\pgfqpoint{2.601928in}{2.719662in}}%
\pgfpathlineto{\pgfqpoint{2.615604in}{2.703428in}}%
\pgfpathlineto{\pgfqpoint{2.629276in}{2.687379in}}%
\pgfpathlineto{\pgfqpoint{2.642942in}{2.671513in}}%
\pgfpathlineto{\pgfqpoint{2.656603in}{2.655827in}}%
\pgfpathlineto{\pgfqpoint{2.647959in}{2.656501in}}%
\pgfpathlineto{\pgfqpoint{2.639299in}{2.657387in}}%
\pgfpathlineto{\pgfqpoint{2.630625in}{2.658491in}}%
\pgfpathlineto{\pgfqpoint{2.621934in}{2.659815in}}%
\pgfpathlineto{\pgfqpoint{2.608234in}{2.675897in}}%
\pgfpathlineto{\pgfqpoint{2.594528in}{2.692161in}}%
\pgfpathlineto{\pgfqpoint{2.580816in}{2.708607in}}%
\pgfpathlineto{\pgfqpoint{2.567100in}{2.725239in}}%
\pgfpathlineto{\pgfqpoint{2.575830in}{2.723510in}}%
\pgfpathlineto{\pgfqpoint{2.584545in}{2.722007in}}%
\pgfpathlineto{\pgfqpoint{2.593244in}{2.720725in}}%
\pgfpathlineto{\pgfqpoint{2.601928in}{2.719662in}}%
\pgfpathclose%
\pgfusepath{fill}%
\end{pgfscope}%
\begin{pgfscope}%
\pgfpathrectangle{\pgfqpoint{1.150000in}{0.150000in}}{\pgfqpoint{5.700000in}{5.700000in}}%
\pgfusepath{clip}%
\pgfsetbuttcap%
\pgfsetroundjoin%
\definecolor{currentfill}{rgb}{0.252194,0.269783,0.531579}%
\pgfsetfillcolor{currentfill}%
\pgfsetfillopacity{0.700000}%
\pgfsetlinewidth{0.000000pt}%
\definecolor{currentstroke}{rgb}{0.000000,0.000000,0.000000}%
\pgfsetstrokecolor{currentstroke}%
\pgfsetdash{}{0pt}%
\pgfpathmoveto{\pgfqpoint{2.711204in}{2.594862in}}%
\pgfpathlineto{\pgfqpoint{2.724843in}{2.580058in}}%
\pgfpathlineto{\pgfqpoint{2.738479in}{2.565425in}}%
\pgfpathlineto{\pgfqpoint{2.752111in}{2.550963in}}%
\pgfpathlineto{\pgfqpoint{2.765739in}{2.536670in}}%
\pgfpathlineto{\pgfqpoint{2.757169in}{2.536572in}}%
\pgfpathlineto{\pgfqpoint{2.748585in}{2.536676in}}%
\pgfpathlineto{\pgfqpoint{2.739986in}{2.536987in}}%
\pgfpathlineto{\pgfqpoint{2.731373in}{2.537509in}}%
\pgfpathlineto{\pgfqpoint{2.717708in}{2.552194in}}%
\pgfpathlineto{\pgfqpoint{2.704039in}{2.567048in}}%
\pgfpathlineto{\pgfqpoint{2.690366in}{2.582074in}}%
\pgfpathlineto{\pgfqpoint{2.676689in}{2.597271in}}%
\pgfpathlineto{\pgfqpoint{2.685339in}{2.596350in}}%
\pgfpathlineto{\pgfqpoint{2.693975in}{2.595644in}}%
\pgfpathlineto{\pgfqpoint{2.702597in}{2.595149in}}%
\pgfpathlineto{\pgfqpoint{2.711204in}{2.594862in}}%
\pgfpathclose%
\pgfusepath{fill}%
\end{pgfscope}%
\begin{pgfscope}%
\pgfpathrectangle{\pgfqpoint{1.150000in}{0.150000in}}{\pgfqpoint{5.700000in}{5.700000in}}%
\pgfusepath{clip}%
\pgfsetbuttcap%
\pgfsetroundjoin%
\definecolor{currentfill}{rgb}{0.218130,0.347432,0.550038}%
\pgfsetfillcolor{currentfill}%
\pgfsetfillopacity{0.700000}%
\pgfsetlinewidth{0.000000pt}%
\definecolor{currentstroke}{rgb}{0.000000,0.000000,0.000000}%
\pgfsetstrokecolor{currentstroke}%
\pgfsetdash{}{0pt}%
\pgfpathmoveto{\pgfqpoint{2.547165in}{2.786474in}}%
\pgfpathlineto{\pgfqpoint{2.560864in}{2.769486in}}%
\pgfpathlineto{\pgfqpoint{2.574558in}{2.752689in}}%
\pgfpathlineto{\pgfqpoint{2.588246in}{2.736082in}}%
\pgfpathlineto{\pgfqpoint{2.601928in}{2.719662in}}%
\pgfpathlineto{\pgfqpoint{2.593244in}{2.720725in}}%
\pgfpathlineto{\pgfqpoint{2.584545in}{2.722007in}}%
\pgfpathlineto{\pgfqpoint{2.575830in}{2.723510in}}%
\pgfpathlineto{\pgfqpoint{2.567100in}{2.725239in}}%
\pgfpathlineto{\pgfqpoint{2.553377in}{2.742058in}}%
\pgfpathlineto{\pgfqpoint{2.539648in}{2.759065in}}%
\pgfpathlineto{\pgfqpoint{2.525913in}{2.776261in}}%
\pgfpathlineto{\pgfqpoint{2.512172in}{2.793650in}}%
\pgfpathlineto{\pgfqpoint{2.520945in}{2.791513in}}%
\pgfpathlineto{\pgfqpoint{2.529701in}{2.789607in}}%
\pgfpathlineto{\pgfqpoint{2.538441in}{2.787929in}}%
\pgfpathlineto{\pgfqpoint{2.547165in}{2.786474in}}%
\pgfpathclose%
\pgfusepath{fill}%
\end{pgfscope}%
\begin{pgfscope}%
\pgfpathrectangle{\pgfqpoint{1.150000in}{0.150000in}}{\pgfqpoint{5.700000in}{5.700000in}}%
\pgfusepath{clip}%
\pgfsetbuttcap%
\pgfsetroundjoin%
\definecolor{currentfill}{rgb}{0.267004,0.004874,0.329415}%
\pgfsetfillcolor{currentfill}%
\pgfsetfillopacity{0.700000}%
\pgfsetlinewidth{0.000000pt}%
\definecolor{currentstroke}{rgb}{0.000000,0.000000,0.000000}%
\pgfsetstrokecolor{currentstroke}%
\pgfsetdash{}{0pt}%
\pgfpathmoveto{\pgfqpoint{3.798187in}{2.065328in}}%
\pgfpathlineto{\pgfqpoint{3.811790in}{2.061439in}}%
\pgfpathlineto{\pgfqpoint{3.825399in}{2.057664in}}%
\pgfpathlineto{\pgfqpoint{3.839014in}{2.054000in}}%
\pgfpathlineto{\pgfqpoint{3.852635in}{2.050449in}}%
\pgfpathlineto{\pgfqpoint{3.844653in}{2.042289in}}%
\pgfpathlineto{\pgfqpoint{3.836666in}{2.034189in}}%
\pgfpathlineto{\pgfqpoint{3.828673in}{2.026149in}}%
\pgfpathlineto{\pgfqpoint{3.820673in}{2.018174in}}%
\pgfpathlineto{\pgfqpoint{3.807038in}{2.022007in}}%
\pgfpathlineto{\pgfqpoint{3.793409in}{2.025952in}}%
\pgfpathlineto{\pgfqpoint{3.779786in}{2.030010in}}%
\pgfpathlineto{\pgfqpoint{3.766168in}{2.034180in}}%
\pgfpathlineto{\pgfqpoint{3.774182in}{2.041867in}}%
\pgfpathlineto{\pgfqpoint{3.782190in}{2.049622in}}%
\pgfpathlineto{\pgfqpoint{3.790192in}{2.057444in}}%
\pgfpathlineto{\pgfqpoint{3.798187in}{2.065328in}}%
\pgfpathclose%
\pgfusepath{fill}%
\end{pgfscope}%
\begin{pgfscope}%
\pgfpathrectangle{\pgfqpoint{1.150000in}{0.150000in}}{\pgfqpoint{5.700000in}{5.700000in}}%
\pgfusepath{clip}%
\pgfsetbuttcap%
\pgfsetroundjoin%
\definecolor{currentfill}{rgb}{0.231674,0.318106,0.544834}%
\pgfsetfillcolor{currentfill}%
\pgfsetfillopacity{0.700000}%
\pgfsetlinewidth{0.000000pt}%
\definecolor{currentstroke}{rgb}{0.000000,0.000000,0.000000}%
\pgfsetstrokecolor{currentstroke}%
\pgfsetdash{}{0pt}%
\pgfpathmoveto{\pgfqpoint{5.145710in}{2.681791in}}%
\pgfpathlineto{\pgfqpoint{5.159792in}{2.686653in}}%
\pgfpathlineto{\pgfqpoint{5.173887in}{2.691616in}}%
\pgfpathlineto{\pgfqpoint{5.187995in}{2.696679in}}%
\pgfpathlineto{\pgfqpoint{5.202116in}{2.701841in}}%
\pgfpathlineto{\pgfqpoint{5.194609in}{2.693087in}}%
\pgfpathlineto{\pgfqpoint{5.187096in}{2.684248in}}%
\pgfpathlineto{\pgfqpoint{5.179575in}{2.675326in}}%
\pgfpathlineto{\pgfqpoint{5.172048in}{2.666318in}}%
\pgfpathlineto{\pgfqpoint{5.157919in}{2.661164in}}%
\pgfpathlineto{\pgfqpoint{5.143803in}{2.656109in}}%
\pgfpathlineto{\pgfqpoint{5.129701in}{2.651155in}}%
\pgfpathlineto{\pgfqpoint{5.115611in}{2.646300in}}%
\pgfpathlineto{\pgfqpoint{5.123146in}{2.655292in}}%
\pgfpathlineto{\pgfqpoint{5.130674in}{2.664205in}}%
\pgfpathlineto{\pgfqpoint{5.138195in}{2.673037in}}%
\pgfpathlineto{\pgfqpoint{5.145710in}{2.681791in}}%
\pgfpathclose%
\pgfusepath{fill}%
\end{pgfscope}%
\begin{pgfscope}%
\pgfpathrectangle{\pgfqpoint{1.150000in}{0.150000in}}{\pgfqpoint{5.700000in}{5.700000in}}%
\pgfusepath{clip}%
\pgfsetbuttcap%
\pgfsetroundjoin%
\definecolor{currentfill}{rgb}{0.262138,0.242286,0.520837}%
\pgfsetfillcolor{currentfill}%
\pgfsetfillopacity{0.700000}%
\pgfsetlinewidth{0.000000pt}%
\definecolor{currentstroke}{rgb}{0.000000,0.000000,0.000000}%
\pgfsetstrokecolor{currentstroke}%
\pgfsetdash{}{0pt}%
\pgfpathmoveto{\pgfqpoint{2.765739in}{2.536670in}}%
\pgfpathlineto{\pgfqpoint{2.779364in}{2.522544in}}%
\pgfpathlineto{\pgfqpoint{2.792986in}{2.508584in}}%
\pgfpathlineto{\pgfqpoint{2.806605in}{2.494789in}}%
\pgfpathlineto{\pgfqpoint{2.820221in}{2.481157in}}%
\pgfpathlineto{\pgfqpoint{2.811686in}{2.480676in}}%
\pgfpathlineto{\pgfqpoint{2.803138in}{2.480393in}}%
\pgfpathlineto{\pgfqpoint{2.794575in}{2.480311in}}%
\pgfpathlineto{\pgfqpoint{2.785999in}{2.480434in}}%
\pgfpathlineto{\pgfqpoint{2.772348in}{2.494456in}}%
\pgfpathlineto{\pgfqpoint{2.758693in}{2.508641in}}%
\pgfpathlineto{\pgfqpoint{2.745035in}{2.522991in}}%
\pgfpathlineto{\pgfqpoint{2.731373in}{2.537509in}}%
\pgfpathlineto{\pgfqpoint{2.739986in}{2.536987in}}%
\pgfpathlineto{\pgfqpoint{2.748585in}{2.536676in}}%
\pgfpathlineto{\pgfqpoint{2.757169in}{2.536572in}}%
\pgfpathlineto{\pgfqpoint{2.765739in}{2.536670in}}%
\pgfpathclose%
\pgfusepath{fill}%
\end{pgfscope}%
\begin{pgfscope}%
\pgfpathrectangle{\pgfqpoint{1.150000in}{0.150000in}}{\pgfqpoint{5.700000in}{5.700000in}}%
\pgfusepath{clip}%
\pgfsetbuttcap%
\pgfsetroundjoin%
\definecolor{currentfill}{rgb}{0.203063,0.379716,0.553925}%
\pgfsetfillcolor{currentfill}%
\pgfsetfillopacity{0.700000}%
\pgfsetlinewidth{0.000000pt}%
\definecolor{currentstroke}{rgb}{0.000000,0.000000,0.000000}%
\pgfsetstrokecolor{currentstroke}%
\pgfsetdash{}{0pt}%
\pgfpathmoveto{\pgfqpoint{2.492302in}{2.856375in}}%
\pgfpathlineto{\pgfqpoint{2.506028in}{2.838603in}}%
\pgfpathlineto{\pgfqpoint{2.519747in}{2.821031in}}%
\pgfpathlineto{\pgfqpoint{2.533459in}{2.803655in}}%
\pgfpathlineto{\pgfqpoint{2.547165in}{2.786474in}}%
\pgfpathlineto{\pgfqpoint{2.538441in}{2.787929in}}%
\pgfpathlineto{\pgfqpoint{2.529701in}{2.789607in}}%
\pgfpathlineto{\pgfqpoint{2.520945in}{2.791513in}}%
\pgfpathlineto{\pgfqpoint{2.512172in}{2.793650in}}%
\pgfpathlineto{\pgfqpoint{2.498425in}{2.811233in}}%
\pgfpathlineto{\pgfqpoint{2.484670in}{2.829011in}}%
\pgfpathlineto{\pgfqpoint{2.470909in}{2.846987in}}%
\pgfpathlineto{\pgfqpoint{2.457140in}{2.865162in}}%
\pgfpathlineto{\pgfqpoint{2.465956in}{2.862614in}}%
\pgfpathlineto{\pgfqpoint{2.474755in}{2.860303in}}%
\pgfpathlineto{\pgfqpoint{2.483537in}{2.858225in}}%
\pgfpathlineto{\pgfqpoint{2.492302in}{2.856375in}}%
\pgfpathclose%
\pgfusepath{fill}%
\end{pgfscope}%
\begin{pgfscope}%
\pgfpathrectangle{\pgfqpoint{1.150000in}{0.150000in}}{\pgfqpoint{5.700000in}{5.700000in}}%
\pgfusepath{clip}%
\pgfsetbuttcap%
\pgfsetroundjoin%
\definecolor{currentfill}{rgb}{0.271305,0.019942,0.347269}%
\pgfsetfillcolor{currentfill}%
\pgfsetfillopacity{0.700000}%
\pgfsetlinewidth{0.000000pt}%
\definecolor{currentstroke}{rgb}{0.000000,0.000000,0.000000}%
\pgfsetstrokecolor{currentstroke}%
\pgfsetdash{}{0pt}%
\pgfpathmoveto{\pgfqpoint{4.025257in}{2.097189in}}%
\pgfpathlineto{\pgfqpoint{4.038912in}{2.095137in}}%
\pgfpathlineto{\pgfqpoint{4.052573in}{2.093194in}}%
\pgfpathlineto{\pgfqpoint{4.066242in}{2.091359in}}%
\pgfpathlineto{\pgfqpoint{4.079918in}{2.089632in}}%
\pgfpathlineto{\pgfqpoint{4.072019in}{2.080375in}}%
\pgfpathlineto{\pgfqpoint{4.064114in}{2.071143in}}%
\pgfpathlineto{\pgfqpoint{4.056204in}{2.061940in}}%
\pgfpathlineto{\pgfqpoint{4.048288in}{2.052767in}}%
\pgfpathlineto{\pgfqpoint{4.034601in}{2.054740in}}%
\pgfpathlineto{\pgfqpoint{4.020922in}{2.056821in}}%
\pgfpathlineto{\pgfqpoint{4.007250in}{2.059009in}}%
\pgfpathlineto{\pgfqpoint{3.993584in}{2.061306in}}%
\pgfpathlineto{\pgfqpoint{4.001511in}{2.070226in}}%
\pgfpathlineto{\pgfqpoint{4.009432in}{2.079182in}}%
\pgfpathlineto{\pgfqpoint{4.017348in}{2.088170in}}%
\pgfpathlineto{\pgfqpoint{4.025257in}{2.097189in}}%
\pgfpathclose%
\pgfusepath{fill}%
\end{pgfscope}%
\begin{pgfscope}%
\pgfpathrectangle{\pgfqpoint{1.150000in}{0.150000in}}{\pgfqpoint{5.700000in}{5.700000in}}%
\pgfusepath{clip}%
\pgfsetbuttcap%
\pgfsetroundjoin%
\definecolor{currentfill}{rgb}{0.274952,0.037752,0.364543}%
\pgfsetfillcolor{currentfill}%
\pgfsetfillopacity{0.700000}%
\pgfsetlinewidth{0.000000pt}%
\definecolor{currentstroke}{rgb}{0.000000,0.000000,0.000000}%
\pgfsetstrokecolor{currentstroke}%
\pgfsetdash{}{0pt}%
\pgfpathmoveto{\pgfqpoint{3.375469in}{2.125557in}}%
\pgfpathlineto{\pgfqpoint{3.389028in}{2.117906in}}%
\pgfpathlineto{\pgfqpoint{3.402590in}{2.110382in}}%
\pgfpathlineto{\pgfqpoint{3.416154in}{2.102983in}}%
\pgfpathlineto{\pgfqpoint{3.429722in}{2.095709in}}%
\pgfpathlineto{\pgfqpoint{3.421551in}{2.090420in}}%
\pgfpathlineto{\pgfqpoint{3.413371in}{2.085252in}}%
\pgfpathlineto{\pgfqpoint{3.405183in}{2.080209in}}%
\pgfpathlineto{\pgfqpoint{3.396986in}{2.075293in}}%
\pgfpathlineto{\pgfqpoint{3.383396in}{2.082906in}}%
\pgfpathlineto{\pgfqpoint{3.369809in}{2.090644in}}%
\pgfpathlineto{\pgfqpoint{3.356225in}{2.098507in}}%
\pgfpathlineto{\pgfqpoint{3.342643in}{2.106497in}}%
\pgfpathlineto{\pgfqpoint{3.350863in}{2.111066in}}%
\pgfpathlineto{\pgfqpoint{3.359073in}{2.115768in}}%
\pgfpathlineto{\pgfqpoint{3.367275in}{2.120600in}}%
\pgfpathlineto{\pgfqpoint{3.375469in}{2.125557in}}%
\pgfpathclose%
\pgfusepath{fill}%
\end{pgfscope}%
\begin{pgfscope}%
\pgfpathrectangle{\pgfqpoint{1.150000in}{0.150000in}}{\pgfqpoint{5.700000in}{5.700000in}}%
\pgfusepath{clip}%
\pgfsetbuttcap%
\pgfsetroundjoin%
\definecolor{currentfill}{rgb}{0.221989,0.339161,0.548752}%
\pgfsetfillcolor{currentfill}%
\pgfsetfillopacity{0.700000}%
\pgfsetlinewidth{0.000000pt}%
\definecolor{currentstroke}{rgb}{0.000000,0.000000,0.000000}%
\pgfsetstrokecolor{currentstroke}%
\pgfsetdash{}{0pt}%
\pgfpathmoveto{\pgfqpoint{5.232073in}{2.736025in}}%
\pgfpathlineto{\pgfqpoint{5.246198in}{2.741277in}}%
\pgfpathlineto{\pgfqpoint{5.260338in}{2.746629in}}%
\pgfpathlineto{\pgfqpoint{5.274490in}{2.752080in}}%
\pgfpathlineto{\pgfqpoint{5.288656in}{2.757632in}}%
\pgfpathlineto{\pgfqpoint{5.281186in}{2.749228in}}%
\pgfpathlineto{\pgfqpoint{5.273709in}{2.740737in}}%
\pgfpathlineto{\pgfqpoint{5.266225in}{2.732159in}}%
\pgfpathlineto{\pgfqpoint{5.258734in}{2.723492in}}%
\pgfpathlineto{\pgfqpoint{5.244559in}{2.717929in}}%
\pgfpathlineto{\pgfqpoint{5.230398in}{2.712467in}}%
\pgfpathlineto{\pgfqpoint{5.216250in}{2.707104in}}%
\pgfpathlineto{\pgfqpoint{5.202116in}{2.701841in}}%
\pgfpathlineto{\pgfqpoint{5.209615in}{2.710512in}}%
\pgfpathlineto{\pgfqpoint{5.217108in}{2.719099in}}%
\pgfpathlineto{\pgfqpoint{5.224594in}{2.727603in}}%
\pgfpathlineto{\pgfqpoint{5.232073in}{2.736025in}}%
\pgfpathclose%
\pgfusepath{fill}%
\end{pgfscope}%
\begin{pgfscope}%
\pgfpathrectangle{\pgfqpoint{1.150000in}{0.150000in}}{\pgfqpoint{5.700000in}{5.700000in}}%
\pgfusepath{clip}%
\pgfsetbuttcap%
\pgfsetroundjoin%
\definecolor{currentfill}{rgb}{0.269308,0.218818,0.509577}%
\pgfsetfillcolor{currentfill}%
\pgfsetfillopacity{0.700000}%
\pgfsetlinewidth{0.000000pt}%
\definecolor{currentstroke}{rgb}{0.000000,0.000000,0.000000}%
\pgfsetstrokecolor{currentstroke}%
\pgfsetdash{}{0pt}%
\pgfpathmoveto{\pgfqpoint{2.820221in}{2.481157in}}%
\pgfpathlineto{\pgfqpoint{2.833834in}{2.467687in}}%
\pgfpathlineto{\pgfqpoint{2.847445in}{2.454378in}}%
\pgfpathlineto{\pgfqpoint{2.861053in}{2.441229in}}%
\pgfpathlineto{\pgfqpoint{2.874659in}{2.428237in}}%
\pgfpathlineto{\pgfqpoint{2.866158in}{2.427376in}}%
\pgfpathlineto{\pgfqpoint{2.857644in}{2.426707in}}%
\pgfpathlineto{\pgfqpoint{2.849117in}{2.426234in}}%
\pgfpathlineto{\pgfqpoint{2.840577in}{2.425961in}}%
\pgfpathlineto{\pgfqpoint{2.826937in}{2.439340in}}%
\pgfpathlineto{\pgfqpoint{2.813294in}{2.452878in}}%
\pgfpathlineto{\pgfqpoint{2.799648in}{2.466575in}}%
\pgfpathlineto{\pgfqpoint{2.785999in}{2.480434in}}%
\pgfpathlineto{\pgfqpoint{2.794575in}{2.480311in}}%
\pgfpathlineto{\pgfqpoint{2.803138in}{2.480393in}}%
\pgfpathlineto{\pgfqpoint{2.811686in}{2.480676in}}%
\pgfpathlineto{\pgfqpoint{2.820221in}{2.481157in}}%
\pgfpathclose%
\pgfusepath{fill}%
\end{pgfscope}%
\begin{pgfscope}%
\pgfpathrectangle{\pgfqpoint{1.150000in}{0.150000in}}{\pgfqpoint{5.700000in}{5.700000in}}%
\pgfusepath{clip}%
\pgfsetbuttcap%
\pgfsetroundjoin%
\definecolor{currentfill}{rgb}{0.281924,0.089666,0.412415}%
\pgfsetfillcolor{currentfill}%
\pgfsetfillopacity{0.700000}%
\pgfsetlinewidth{0.000000pt}%
\definecolor{currentstroke}{rgb}{0.000000,0.000000,0.000000}%
\pgfsetstrokecolor{currentstroke}%
\pgfsetdash{}{0pt}%
\pgfpathmoveto{\pgfqpoint{3.179797in}{2.212476in}}%
\pgfpathlineto{\pgfqpoint{3.193360in}{2.202915in}}%
\pgfpathlineto{\pgfqpoint{3.206924in}{2.193489in}}%
\pgfpathlineto{\pgfqpoint{3.220490in}{2.184198in}}%
\pgfpathlineto{\pgfqpoint{3.234056in}{2.175040in}}%
\pgfpathlineto{\pgfqpoint{3.225778in}{2.171310in}}%
\pgfpathlineto{\pgfqpoint{3.217489in}{2.167728in}}%
\pgfpathlineto{\pgfqpoint{3.209191in}{2.164298in}}%
\pgfpathlineto{\pgfqpoint{3.200883in}{2.161024in}}%
\pgfpathlineto{\pgfqpoint{3.187290in}{2.170542in}}%
\pgfpathlineto{\pgfqpoint{3.173698in}{2.180193in}}%
\pgfpathlineto{\pgfqpoint{3.160107in}{2.189979in}}%
\pgfpathlineto{\pgfqpoint{3.146517in}{2.199901in}}%
\pgfpathlineto{\pgfqpoint{3.154853in}{2.202807in}}%
\pgfpathlineto{\pgfqpoint{3.163178in}{2.205874in}}%
\pgfpathlineto{\pgfqpoint{3.171493in}{2.209098in}}%
\pgfpathlineto{\pgfqpoint{3.179797in}{2.212476in}}%
\pgfpathclose%
\pgfusepath{fill}%
\end{pgfscope}%
\begin{pgfscope}%
\pgfpathrectangle{\pgfqpoint{1.150000in}{0.150000in}}{\pgfqpoint{5.700000in}{5.700000in}}%
\pgfusepath{clip}%
\pgfsetbuttcap%
\pgfsetroundjoin%
\definecolor{currentfill}{rgb}{0.210503,0.363727,0.552206}%
\pgfsetfillcolor{currentfill}%
\pgfsetfillopacity{0.700000}%
\pgfsetlinewidth{0.000000pt}%
\definecolor{currentstroke}{rgb}{0.000000,0.000000,0.000000}%
\pgfsetstrokecolor{currentstroke}%
\pgfsetdash{}{0pt}%
\pgfpathmoveto{\pgfqpoint{5.318464in}{2.790384in}}%
\pgfpathlineto{\pgfqpoint{5.332635in}{2.796006in}}%
\pgfpathlineto{\pgfqpoint{5.346819in}{2.801727in}}%
\pgfpathlineto{\pgfqpoint{5.361017in}{2.807547in}}%
\pgfpathlineto{\pgfqpoint{5.375229in}{2.813468in}}%
\pgfpathlineto{\pgfqpoint{5.367797in}{2.805445in}}%
\pgfpathlineto{\pgfqpoint{5.360358in}{2.797332in}}%
\pgfpathlineto{\pgfqpoint{5.352912in}{2.789130in}}%
\pgfpathlineto{\pgfqpoint{5.345458in}{2.780836in}}%
\pgfpathlineto{\pgfqpoint{5.331237in}{2.774885in}}%
\pgfpathlineto{\pgfqpoint{5.317030in}{2.769034in}}%
\pgfpathlineto{\pgfqpoint{5.302836in}{2.763283in}}%
\pgfpathlineto{\pgfqpoint{5.288656in}{2.757632in}}%
\pgfpathlineto{\pgfqpoint{5.296119in}{2.765949in}}%
\pgfpathlineto{\pgfqpoint{5.303575in}{2.774179in}}%
\pgfpathlineto{\pgfqpoint{5.311023in}{2.782324in}}%
\pgfpathlineto{\pgfqpoint{5.318464in}{2.790384in}}%
\pgfpathclose%
\pgfusepath{fill}%
\end{pgfscope}%
\begin{pgfscope}%
\pgfpathrectangle{\pgfqpoint{1.150000in}{0.150000in}}{\pgfqpoint{5.700000in}{5.700000in}}%
\pgfusepath{clip}%
\pgfsetbuttcap%
\pgfsetroundjoin%
\definecolor{currentfill}{rgb}{0.275191,0.194905,0.496005}%
\pgfsetfillcolor{currentfill}%
\pgfsetfillopacity{0.700000}%
\pgfsetlinewidth{0.000000pt}%
\definecolor{currentstroke}{rgb}{0.000000,0.000000,0.000000}%
\pgfsetstrokecolor{currentstroke}%
\pgfsetdash{}{0pt}%
\pgfpathmoveto{\pgfqpoint{2.874659in}{2.428237in}}%
\pgfpathlineto{\pgfqpoint{2.888263in}{2.415403in}}%
\pgfpathlineto{\pgfqpoint{2.901865in}{2.402725in}}%
\pgfpathlineto{\pgfqpoint{2.915465in}{2.390201in}}%
\pgfpathlineto{\pgfqpoint{2.929063in}{2.377831in}}%
\pgfpathlineto{\pgfqpoint{2.920595in}{2.376590in}}%
\pgfpathlineto{\pgfqpoint{2.912115in}{2.375537in}}%
\pgfpathlineto{\pgfqpoint{2.903622in}{2.374675in}}%
\pgfpathlineto{\pgfqpoint{2.895116in}{2.374008in}}%
\pgfpathlineto{\pgfqpoint{2.881484in}{2.386764in}}%
\pgfpathlineto{\pgfqpoint{2.867851in}{2.399674in}}%
\pgfpathlineto{\pgfqpoint{2.854215in}{2.412740in}}%
\pgfpathlineto{\pgfqpoint{2.840577in}{2.425961in}}%
\pgfpathlineto{\pgfqpoint{2.849117in}{2.426234in}}%
\pgfpathlineto{\pgfqpoint{2.857644in}{2.426707in}}%
\pgfpathlineto{\pgfqpoint{2.866158in}{2.427376in}}%
\pgfpathlineto{\pgfqpoint{2.874659in}{2.428237in}}%
\pgfpathclose%
\pgfusepath{fill}%
\end{pgfscope}%
\begin{pgfscope}%
\pgfpathrectangle{\pgfqpoint{1.150000in}{0.150000in}}{\pgfqpoint{5.700000in}{5.700000in}}%
\pgfusepath{clip}%
\pgfsetbuttcap%
\pgfsetroundjoin%
\definecolor{currentfill}{rgb}{0.269944,0.014625,0.341379}%
\pgfsetfillcolor{currentfill}%
\pgfsetfillopacity{0.700000}%
\pgfsetlinewidth{0.000000pt}%
\definecolor{currentstroke}{rgb}{0.000000,0.000000,0.000000}%
\pgfsetstrokecolor{currentstroke}%
\pgfsetdash{}{0pt}%
\pgfpathmoveto{\pgfqpoint{3.938990in}{2.071584in}}%
\pgfpathlineto{\pgfqpoint{3.952628in}{2.068851in}}%
\pgfpathlineto{\pgfqpoint{3.966274in}{2.066226in}}%
\pgfpathlineto{\pgfqpoint{3.979925in}{2.063712in}}%
\pgfpathlineto{\pgfqpoint{3.993584in}{2.061306in}}%
\pgfpathlineto{\pgfqpoint{3.985652in}{2.052423in}}%
\pgfpathlineto{\pgfqpoint{3.977714in}{2.043580in}}%
\pgfpathlineto{\pgfqpoint{3.969770in}{2.034779in}}%
\pgfpathlineto{\pgfqpoint{3.961820in}{2.026022in}}%
\pgfpathlineto{\pgfqpoint{3.948150in}{2.028692in}}%
\pgfpathlineto{\pgfqpoint{3.934486in}{2.031470in}}%
\pgfpathlineto{\pgfqpoint{3.920828in}{2.034358in}}%
\pgfpathlineto{\pgfqpoint{3.907177in}{2.037355in}}%
\pgfpathlineto{\pgfqpoint{3.915139in}{2.045841in}}%
\pgfpathlineto{\pgfqpoint{3.923095in}{2.054376in}}%
\pgfpathlineto{\pgfqpoint{3.931045in}{2.062958in}}%
\pgfpathlineto{\pgfqpoint{3.938990in}{2.071584in}}%
\pgfpathclose%
\pgfusepath{fill}%
\end{pgfscope}%
\begin{pgfscope}%
\pgfpathrectangle{\pgfqpoint{1.150000in}{0.150000in}}{\pgfqpoint{5.700000in}{5.700000in}}%
\pgfusepath{clip}%
\pgfsetbuttcap%
\pgfsetroundjoin%
\definecolor{currentfill}{rgb}{0.283072,0.130895,0.449241}%
\pgfsetfillcolor{currentfill}%
\pgfsetfillopacity{0.700000}%
\pgfsetlinewidth{0.000000pt}%
\definecolor{currentstroke}{rgb}{0.000000,0.000000,0.000000}%
\pgfsetstrokecolor{currentstroke}%
\pgfsetdash{}{0pt}%
\pgfpathmoveto{\pgfqpoint{4.511139in}{2.284099in}}%
\pgfpathlineto{\pgfqpoint{4.524955in}{2.285537in}}%
\pgfpathlineto{\pgfqpoint{4.538781in}{2.287078in}}%
\pgfpathlineto{\pgfqpoint{4.552617in}{2.288721in}}%
\pgfpathlineto{\pgfqpoint{4.566463in}{2.290467in}}%
\pgfpathlineto{\pgfqpoint{4.558717in}{2.280190in}}%
\pgfpathlineto{\pgfqpoint{4.550966in}{2.269877in}}%
\pgfpathlineto{\pgfqpoint{4.543210in}{2.259529in}}%
\pgfpathlineto{\pgfqpoint{4.535448in}{2.249149in}}%
\pgfpathlineto{\pgfqpoint{4.521596in}{2.247559in}}%
\pgfpathlineto{\pgfqpoint{4.507754in}{2.246071in}}%
\pgfpathlineto{\pgfqpoint{4.493921in}{2.244687in}}%
\pgfpathlineto{\pgfqpoint{4.480098in}{2.243405in}}%
\pgfpathlineto{\pgfqpoint{4.487866in}{2.253623in}}%
\pgfpathlineto{\pgfqpoint{4.495629in}{2.263812in}}%
\pgfpathlineto{\pgfqpoint{4.503387in}{2.273971in}}%
\pgfpathlineto{\pgfqpoint{4.511139in}{2.284099in}}%
\pgfpathclose%
\pgfusepath{fill}%
\end{pgfscope}%
\begin{pgfscope}%
\pgfpathrectangle{\pgfqpoint{1.150000in}{0.150000in}}{\pgfqpoint{5.700000in}{5.700000in}}%
\pgfusepath{clip}%
\pgfsetbuttcap%
\pgfsetroundjoin%
\definecolor{currentfill}{rgb}{0.283091,0.110553,0.431554}%
\pgfsetfillcolor{currentfill}%
\pgfsetfillopacity{0.700000}%
\pgfsetlinewidth{0.000000pt}%
\definecolor{currentstroke}{rgb}{0.000000,0.000000,0.000000}%
\pgfsetstrokecolor{currentstroke}%
\pgfsetdash{}{0pt}%
\pgfpathmoveto{\pgfqpoint{4.424904in}{2.239309in}}%
\pgfpathlineto{\pgfqpoint{4.438688in}{2.240178in}}%
\pgfpathlineto{\pgfqpoint{4.452482in}{2.241150in}}%
\pgfpathlineto{\pgfqpoint{4.466285in}{2.242226in}}%
\pgfpathlineto{\pgfqpoint{4.480098in}{2.243405in}}%
\pgfpathlineto{\pgfqpoint{4.472325in}{2.233160in}}%
\pgfpathlineto{\pgfqpoint{4.464546in}{2.222889in}}%
\pgfpathlineto{\pgfqpoint{4.456763in}{2.212594in}}%
\pgfpathlineto{\pgfqpoint{4.448974in}{2.202276in}}%
\pgfpathlineto{\pgfqpoint{4.435154in}{2.201272in}}%
\pgfpathlineto{\pgfqpoint{4.421344in}{2.200370in}}%
\pgfpathlineto{\pgfqpoint{4.407543in}{2.199572in}}%
\pgfpathlineto{\pgfqpoint{4.393751in}{2.198877in}}%
\pgfpathlineto{\pgfqpoint{4.401547in}{2.209014in}}%
\pgfpathlineto{\pgfqpoint{4.409338in}{2.219132in}}%
\pgfpathlineto{\pgfqpoint{4.417124in}{2.229231in}}%
\pgfpathlineto{\pgfqpoint{4.424904in}{2.239309in}}%
\pgfpathclose%
\pgfusepath{fill}%
\end{pgfscope}%
\begin{pgfscope}%
\pgfpathrectangle{\pgfqpoint{1.150000in}{0.150000in}}{\pgfqpoint{5.700000in}{5.700000in}}%
\pgfusepath{clip}%
\pgfsetbuttcap%
\pgfsetroundjoin%
\definecolor{currentfill}{rgb}{0.281412,0.155834,0.469201}%
\pgfsetfillcolor{currentfill}%
\pgfsetfillopacity{0.700000}%
\pgfsetlinewidth{0.000000pt}%
\definecolor{currentstroke}{rgb}{0.000000,0.000000,0.000000}%
\pgfsetstrokecolor{currentstroke}%
\pgfsetdash{}{0pt}%
\pgfpathmoveto{\pgfqpoint{4.597393in}{2.331195in}}%
\pgfpathlineto{\pgfqpoint{4.611242in}{2.333181in}}%
\pgfpathlineto{\pgfqpoint{4.625103in}{2.335270in}}%
\pgfpathlineto{\pgfqpoint{4.638973in}{2.337461in}}%
\pgfpathlineto{\pgfqpoint{4.652855in}{2.339754in}}%
\pgfpathlineto{\pgfqpoint{4.645137in}{2.329500in}}%
\pgfpathlineto{\pgfqpoint{4.637414in}{2.319202in}}%
\pgfpathlineto{\pgfqpoint{4.629685in}{2.308859in}}%
\pgfpathlineto{\pgfqpoint{4.621950in}{2.298474in}}%
\pgfpathlineto{\pgfqpoint{4.608063in}{2.296319in}}%
\pgfpathlineto{\pgfqpoint{4.594186in}{2.294267in}}%
\pgfpathlineto{\pgfqpoint{4.580319in}{2.292316in}}%
\pgfpathlineto{\pgfqpoint{4.566463in}{2.290467in}}%
\pgfpathlineto{\pgfqpoint{4.574204in}{2.300708in}}%
\pgfpathlineto{\pgfqpoint{4.581939in}{2.310910in}}%
\pgfpathlineto{\pgfqpoint{4.589668in}{2.321072in}}%
\pgfpathlineto{\pgfqpoint{4.597393in}{2.331195in}}%
\pgfpathclose%
\pgfusepath{fill}%
\end{pgfscope}%
\begin{pgfscope}%
\pgfpathrectangle{\pgfqpoint{1.150000in}{0.150000in}}{\pgfqpoint{5.700000in}{5.700000in}}%
\pgfusepath{clip}%
\pgfsetbuttcap%
\pgfsetroundjoin%
\definecolor{currentfill}{rgb}{0.199430,0.387607,0.554642}%
\pgfsetfillcolor{currentfill}%
\pgfsetfillopacity{0.700000}%
\pgfsetlinewidth{0.000000pt}%
\definecolor{currentstroke}{rgb}{0.000000,0.000000,0.000000}%
\pgfsetstrokecolor{currentstroke}%
\pgfsetdash{}{0pt}%
\pgfpathmoveto{\pgfqpoint{5.404880in}{2.844677in}}%
\pgfpathlineto{\pgfqpoint{5.419096in}{2.850648in}}%
\pgfpathlineto{\pgfqpoint{5.433326in}{2.856718in}}%
\pgfpathlineto{\pgfqpoint{5.447570in}{2.862888in}}%
\pgfpathlineto{\pgfqpoint{5.461828in}{2.869157in}}%
\pgfpathlineto{\pgfqpoint{5.454437in}{2.861542in}}%
\pgfpathlineto{\pgfqpoint{5.447038in}{2.853836in}}%
\pgfpathlineto{\pgfqpoint{5.439631in}{2.846038in}}%
\pgfpathlineto{\pgfqpoint{5.432217in}{2.838146in}}%
\pgfpathlineto{\pgfqpoint{5.417949in}{2.831827in}}%
\pgfpathlineto{\pgfqpoint{5.403694in}{2.825608in}}%
\pgfpathlineto{\pgfqpoint{5.389455in}{2.819488in}}%
\pgfpathlineto{\pgfqpoint{5.375229in}{2.813468in}}%
\pgfpathlineto{\pgfqpoint{5.382653in}{2.821402in}}%
\pgfpathlineto{\pgfqpoint{5.390069in}{2.829247in}}%
\pgfpathlineto{\pgfqpoint{5.397478in}{2.837005in}}%
\pgfpathlineto{\pgfqpoint{5.404880in}{2.844677in}}%
\pgfpathclose%
\pgfusepath{fill}%
\end{pgfscope}%
\begin{pgfscope}%
\pgfpathrectangle{\pgfqpoint{1.150000in}{0.150000in}}{\pgfqpoint{5.700000in}{5.700000in}}%
\pgfusepath{clip}%
\pgfsetbuttcap%
\pgfsetroundjoin%
\definecolor{currentfill}{rgb}{0.268510,0.009605,0.335427}%
\pgfsetfillcolor{currentfill}%
\pgfsetfillopacity{0.700000}%
\pgfsetlinewidth{0.000000pt}%
\definecolor{currentstroke}{rgb}{0.000000,0.000000,0.000000}%
\pgfsetstrokecolor{currentstroke}%
\pgfsetdash{}{0pt}%
\pgfpathmoveto{\pgfqpoint{3.570817in}{2.066787in}}%
\pgfpathlineto{\pgfqpoint{3.584395in}{2.060926in}}%
\pgfpathlineto{\pgfqpoint{3.597977in}{2.055184in}}%
\pgfpathlineto{\pgfqpoint{3.611564in}{2.049560in}}%
\pgfpathlineto{\pgfqpoint{3.625154in}{2.044053in}}%
\pgfpathlineto{\pgfqpoint{3.617073in}{2.037374in}}%
\pgfpathlineto{\pgfqpoint{3.608984in}{2.030790in}}%
\pgfpathlineto{\pgfqpoint{3.600889in}{2.024304in}}%
\pgfpathlineto{\pgfqpoint{3.592785in}{2.017919in}}%
\pgfpathlineto{\pgfqpoint{3.579176in}{2.023744in}}%
\pgfpathlineto{\pgfqpoint{3.565571in}{2.029687in}}%
\pgfpathlineto{\pgfqpoint{3.551970in}{2.035748in}}%
\pgfpathlineto{\pgfqpoint{3.538373in}{2.041928in}}%
\pgfpathlineto{\pgfqpoint{3.546496in}{2.047987in}}%
\pgfpathlineto{\pgfqpoint{3.554610in}{2.054153in}}%
\pgfpathlineto{\pgfqpoint{3.562717in}{2.060420in}}%
\pgfpathlineto{\pgfqpoint{3.570817in}{2.066787in}}%
\pgfpathclose%
\pgfusepath{fill}%
\end{pgfscope}%
\begin{pgfscope}%
\pgfpathrectangle{\pgfqpoint{1.150000in}{0.150000in}}{\pgfqpoint{5.700000in}{5.700000in}}%
\pgfusepath{clip}%
\pgfsetbuttcap%
\pgfsetroundjoin%
\definecolor{currentfill}{rgb}{0.281924,0.089666,0.412415}%
\pgfsetfillcolor{currentfill}%
\pgfsetfillopacity{0.700000}%
\pgfsetlinewidth{0.000000pt}%
\definecolor{currentstroke}{rgb}{0.000000,0.000000,0.000000}%
\pgfsetstrokecolor{currentstroke}%
\pgfsetdash{}{0pt}%
\pgfpathmoveto{\pgfqpoint{4.338677in}{2.197136in}}%
\pgfpathlineto{\pgfqpoint{4.352432in}{2.197415in}}%
\pgfpathlineto{\pgfqpoint{4.366196in}{2.197798in}}%
\pgfpathlineto{\pgfqpoint{4.379969in}{2.198286in}}%
\pgfpathlineto{\pgfqpoint{4.393751in}{2.198877in}}%
\pgfpathlineto{\pgfqpoint{4.385950in}{2.188723in}}%
\pgfpathlineto{\pgfqpoint{4.378144in}{2.178555in}}%
\pgfpathlineto{\pgfqpoint{4.370332in}{2.168372in}}%
\pgfpathlineto{\pgfqpoint{4.362516in}{2.158179in}}%
\pgfpathlineto{\pgfqpoint{4.348726in}{2.157780in}}%
\pgfpathlineto{\pgfqpoint{4.334945in}{2.157485in}}%
\pgfpathlineto{\pgfqpoint{4.321174in}{2.157293in}}%
\pgfpathlineto{\pgfqpoint{4.307411in}{2.157206in}}%
\pgfpathlineto{\pgfqpoint{4.315235in}{2.167201in}}%
\pgfpathlineto{\pgfqpoint{4.323054in}{2.177189in}}%
\pgfpathlineto{\pgfqpoint{4.330868in}{2.187168in}}%
\pgfpathlineto{\pgfqpoint{4.338677in}{2.197136in}}%
\pgfpathclose%
\pgfusepath{fill}%
\end{pgfscope}%
\begin{pgfscope}%
\pgfpathrectangle{\pgfqpoint{1.150000in}{0.150000in}}{\pgfqpoint{5.700000in}{5.700000in}}%
\pgfusepath{clip}%
\pgfsetbuttcap%
\pgfsetroundjoin%
\definecolor{currentfill}{rgb}{0.267004,0.004874,0.329415}%
\pgfsetfillcolor{currentfill}%
\pgfsetfillopacity{0.700000}%
\pgfsetlinewidth{0.000000pt}%
\definecolor{currentstroke}{rgb}{0.000000,0.000000,0.000000}%
\pgfsetstrokecolor{currentstroke}%
\pgfsetdash{}{0pt}%
\pgfpathmoveto{\pgfqpoint{3.711748in}{2.051999in}}%
\pgfpathlineto{\pgfqpoint{3.725345in}{2.047373in}}%
\pgfpathlineto{\pgfqpoint{3.738948in}{2.042861in}}%
\pgfpathlineto{\pgfqpoint{3.752555in}{2.038464in}}%
\pgfpathlineto{\pgfqpoint{3.766168in}{2.034180in}}%
\pgfpathlineto{\pgfqpoint{3.758147in}{2.026565in}}%
\pgfpathlineto{\pgfqpoint{3.750119in}{2.019023in}}%
\pgfpathlineto{\pgfqpoint{3.742085in}{2.011559in}}%
\pgfpathlineto{\pgfqpoint{3.734045in}{2.004174in}}%
\pgfpathlineto{\pgfqpoint{3.720416in}{2.008758in}}%
\pgfpathlineto{\pgfqpoint{3.706793in}{2.013455in}}%
\pgfpathlineto{\pgfqpoint{3.693175in}{2.018267in}}%
\pgfpathlineto{\pgfqpoint{3.679561in}{2.023193in}}%
\pgfpathlineto{\pgfqpoint{3.687618in}{2.030271in}}%
\pgfpathlineto{\pgfqpoint{3.695669in}{2.037433in}}%
\pgfpathlineto{\pgfqpoint{3.703712in}{2.044676in}}%
\pgfpathlineto{\pgfqpoint{3.711748in}{2.051999in}}%
\pgfpathclose%
\pgfusepath{fill}%
\end{pgfscope}%
\begin{pgfscope}%
\pgfpathrectangle{\pgfqpoint{1.150000in}{0.150000in}}{\pgfqpoint{5.700000in}{5.700000in}}%
\pgfusepath{clip}%
\pgfsetbuttcap%
\pgfsetroundjoin%
\definecolor{currentfill}{rgb}{0.278012,0.180367,0.486697}%
\pgfsetfillcolor{currentfill}%
\pgfsetfillopacity{0.700000}%
\pgfsetlinewidth{0.000000pt}%
\definecolor{currentstroke}{rgb}{0.000000,0.000000,0.000000}%
\pgfsetstrokecolor{currentstroke}%
\pgfsetdash{}{0pt}%
\pgfpathmoveto{\pgfqpoint{4.683672in}{2.380298in}}%
\pgfpathlineto{\pgfqpoint{4.697558in}{2.382812in}}%
\pgfpathlineto{\pgfqpoint{4.711454in}{2.385429in}}%
\pgfpathlineto{\pgfqpoint{4.725362in}{2.388147in}}%
\pgfpathlineto{\pgfqpoint{4.739281in}{2.390966in}}%
\pgfpathlineto{\pgfqpoint{4.731591in}{2.380790in}}%
\pgfpathlineto{\pgfqpoint{4.723895in}{2.370560in}}%
\pgfpathlineto{\pgfqpoint{4.716194in}{2.360277in}}%
\pgfpathlineto{\pgfqpoint{4.708488in}{2.349943in}}%
\pgfpathlineto{\pgfqpoint{4.694563in}{2.347243in}}%
\pgfpathlineto{\pgfqpoint{4.680650in}{2.344645in}}%
\pgfpathlineto{\pgfqpoint{4.666747in}{2.342149in}}%
\pgfpathlineto{\pgfqpoint{4.652855in}{2.339754in}}%
\pgfpathlineto{\pgfqpoint{4.660567in}{2.349961in}}%
\pgfpathlineto{\pgfqpoint{4.668274in}{2.360122in}}%
\pgfpathlineto{\pgfqpoint{4.675976in}{2.370234in}}%
\pgfpathlineto{\pgfqpoint{4.683672in}{2.380298in}}%
\pgfpathclose%
\pgfusepath{fill}%
\end{pgfscope}%
\begin{pgfscope}%
\pgfpathrectangle{\pgfqpoint{1.150000in}{0.150000in}}{\pgfqpoint{5.700000in}{5.700000in}}%
\pgfusepath{clip}%
\pgfsetbuttcap%
\pgfsetroundjoin%
\definecolor{currentfill}{rgb}{0.278826,0.175490,0.483397}%
\pgfsetfillcolor{currentfill}%
\pgfsetfillopacity{0.700000}%
\pgfsetlinewidth{0.000000pt}%
\definecolor{currentstroke}{rgb}{0.000000,0.000000,0.000000}%
\pgfsetstrokecolor{currentstroke}%
\pgfsetdash{}{0pt}%
\pgfpathmoveto{\pgfqpoint{2.929063in}{2.377831in}}%
\pgfpathlineto{\pgfqpoint{2.942659in}{2.365613in}}%
\pgfpathlineto{\pgfqpoint{2.956255in}{2.353546in}}%
\pgfpathlineto{\pgfqpoint{2.969849in}{2.341630in}}%
\pgfpathlineto{\pgfqpoint{2.983442in}{2.329862in}}%
\pgfpathlineto{\pgfqpoint{2.975006in}{2.328244in}}%
\pgfpathlineto{\pgfqpoint{2.966558in}{2.326808in}}%
\pgfpathlineto{\pgfqpoint{2.958098in}{2.325559in}}%
\pgfpathlineto{\pgfqpoint{2.949625in}{2.324499in}}%
\pgfpathlineto{\pgfqpoint{2.936000in}{2.336651in}}%
\pgfpathlineto{\pgfqpoint{2.922373in}{2.348953in}}%
\pgfpathlineto{\pgfqpoint{2.908745in}{2.361405in}}%
\pgfpathlineto{\pgfqpoint{2.895116in}{2.374008in}}%
\pgfpathlineto{\pgfqpoint{2.903622in}{2.374675in}}%
\pgfpathlineto{\pgfqpoint{2.912115in}{2.375537in}}%
\pgfpathlineto{\pgfqpoint{2.920595in}{2.376590in}}%
\pgfpathlineto{\pgfqpoint{2.929063in}{2.377831in}}%
\pgfpathclose%
\pgfusepath{fill}%
\end{pgfscope}%
\begin{pgfscope}%
\pgfpathrectangle{\pgfqpoint{1.150000in}{0.150000in}}{\pgfqpoint{5.700000in}{5.700000in}}%
\pgfusepath{clip}%
\pgfsetbuttcap%
\pgfsetroundjoin%
\definecolor{currentfill}{rgb}{0.279566,0.067836,0.391917}%
\pgfsetfillcolor{currentfill}%
\pgfsetfillopacity{0.700000}%
\pgfsetlinewidth{0.000000pt}%
\definecolor{currentstroke}{rgb}{0.000000,0.000000,0.000000}%
\pgfsetstrokecolor{currentstroke}%
\pgfsetdash{}{0pt}%
\pgfpathmoveto{\pgfqpoint{4.252447in}{2.157905in}}%
\pgfpathlineto{\pgfqpoint{4.266175in}{2.157573in}}%
\pgfpathlineto{\pgfqpoint{4.279912in}{2.157346in}}%
\pgfpathlineto{\pgfqpoint{4.293657in}{2.157224in}}%
\pgfpathlineto{\pgfqpoint{4.307411in}{2.157206in}}%
\pgfpathlineto{\pgfqpoint{4.299582in}{2.147206in}}%
\pgfpathlineto{\pgfqpoint{4.291747in}{2.137202in}}%
\pgfpathlineto{\pgfqpoint{4.283907in}{2.127196in}}%
\pgfpathlineto{\pgfqpoint{4.276062in}{2.117191in}}%
\pgfpathlineto{\pgfqpoint{4.262300in}{2.117418in}}%
\pgfpathlineto{\pgfqpoint{4.248546in}{2.117750in}}%
\pgfpathlineto{\pgfqpoint{4.234801in}{2.118187in}}%
\pgfpathlineto{\pgfqpoint{4.221064in}{2.118730in}}%
\pgfpathlineto{\pgfqpoint{4.228918in}{2.128518in}}%
\pgfpathlineto{\pgfqpoint{4.236766in}{2.138312in}}%
\pgfpathlineto{\pgfqpoint{4.244609in}{2.148108in}}%
\pgfpathlineto{\pgfqpoint{4.252447in}{2.157905in}}%
\pgfpathclose%
\pgfusepath{fill}%
\end{pgfscope}%
\begin{pgfscope}%
\pgfpathrectangle{\pgfqpoint{1.150000in}{0.150000in}}{\pgfqpoint{5.700000in}{5.700000in}}%
\pgfusepath{clip}%
\pgfsetbuttcap%
\pgfsetroundjoin%
\definecolor{currentfill}{rgb}{0.188923,0.410910,0.556326}%
\pgfsetfillcolor{currentfill}%
\pgfsetfillopacity{0.700000}%
\pgfsetlinewidth{0.000000pt}%
\definecolor{currentstroke}{rgb}{0.000000,0.000000,0.000000}%
\pgfsetstrokecolor{currentstroke}%
\pgfsetdash{}{0pt}%
\pgfpathmoveto{\pgfqpoint{5.491315in}{2.898722in}}%
\pgfpathlineto{\pgfqpoint{5.505577in}{2.905022in}}%
\pgfpathlineto{\pgfqpoint{5.519853in}{2.911422in}}%
\pgfpathlineto{\pgfqpoint{5.534143in}{2.917921in}}%
\pgfpathlineto{\pgfqpoint{5.548448in}{2.924520in}}%
\pgfpathlineto{\pgfqpoint{5.541099in}{2.917337in}}%
\pgfpathlineto{\pgfqpoint{5.533742in}{2.910062in}}%
\pgfpathlineto{\pgfqpoint{5.526378in}{2.902693in}}%
\pgfpathlineto{\pgfqpoint{5.519005in}{2.895231in}}%
\pgfpathlineto{\pgfqpoint{5.504689in}{2.888563in}}%
\pgfpathlineto{\pgfqpoint{5.490387in}{2.881995in}}%
\pgfpathlineto{\pgfqpoint{5.476100in}{2.875526in}}%
\pgfpathlineto{\pgfqpoint{5.461828in}{2.869157in}}%
\pgfpathlineto{\pgfqpoint{5.469211in}{2.876682in}}%
\pgfpathlineto{\pgfqpoint{5.476587in}{2.884117in}}%
\pgfpathlineto{\pgfqpoint{5.483955in}{2.891463in}}%
\pgfpathlineto{\pgfqpoint{5.491315in}{2.898722in}}%
\pgfpathclose%
\pgfusepath{fill}%
\end{pgfscope}%
\begin{pgfscope}%
\pgfpathrectangle{\pgfqpoint{1.150000in}{0.150000in}}{\pgfqpoint{5.700000in}{5.700000in}}%
\pgfusepath{clip}%
\pgfsetbuttcap%
\pgfsetroundjoin%
\definecolor{currentfill}{rgb}{0.273006,0.204520,0.501721}%
\pgfsetfillcolor{currentfill}%
\pgfsetfillopacity{0.700000}%
\pgfsetlinewidth{0.000000pt}%
\definecolor{currentstroke}{rgb}{0.000000,0.000000,0.000000}%
\pgfsetstrokecolor{currentstroke}%
\pgfsetdash{}{0pt}%
\pgfpathmoveto{\pgfqpoint{4.769983in}{2.431121in}}%
\pgfpathlineto{\pgfqpoint{4.783907in}{2.434143in}}%
\pgfpathlineto{\pgfqpoint{4.797842in}{2.437267in}}%
\pgfpathlineto{\pgfqpoint{4.811788in}{2.440492in}}%
\pgfpathlineto{\pgfqpoint{4.825746in}{2.443818in}}%
\pgfpathlineto{\pgfqpoint{4.818085in}{2.433769in}}%
\pgfpathlineto{\pgfqpoint{4.810418in}{2.423659in}}%
\pgfpathlineto{\pgfqpoint{4.802746in}{2.413488in}}%
\pgfpathlineto{\pgfqpoint{4.795067in}{2.403258in}}%
\pgfpathlineto{\pgfqpoint{4.781104in}{2.400033in}}%
\pgfpathlineto{\pgfqpoint{4.767151in}{2.396910in}}%
\pgfpathlineto{\pgfqpoint{4.753210in}{2.393887in}}%
\pgfpathlineto{\pgfqpoint{4.739281in}{2.390966in}}%
\pgfpathlineto{\pgfqpoint{4.746965in}{2.401088in}}%
\pgfpathlineto{\pgfqpoint{4.754643in}{2.411155in}}%
\pgfpathlineto{\pgfqpoint{4.762316in}{2.421167in}}%
\pgfpathlineto{\pgfqpoint{4.769983in}{2.431121in}}%
\pgfpathclose%
\pgfusepath{fill}%
\end{pgfscope}%
\begin{pgfscope}%
\pgfpathrectangle{\pgfqpoint{1.150000in}{0.150000in}}{\pgfqpoint{5.700000in}{5.700000in}}%
\pgfusepath{clip}%
\pgfsetbuttcap%
\pgfsetroundjoin%
\definecolor{currentfill}{rgb}{0.280267,0.073417,0.397163}%
\pgfsetfillcolor{currentfill}%
\pgfsetfillopacity{0.700000}%
\pgfsetlinewidth{0.000000pt}%
\definecolor{currentstroke}{rgb}{0.000000,0.000000,0.000000}%
\pgfsetstrokecolor{currentstroke}%
\pgfsetdash{}{0pt}%
\pgfpathmoveto{\pgfqpoint{3.234056in}{2.175040in}}%
\pgfpathlineto{\pgfqpoint{3.247624in}{2.166015in}}%
\pgfpathlineto{\pgfqpoint{3.261193in}{2.157123in}}%
\pgfpathlineto{\pgfqpoint{3.274763in}{2.148361in}}%
\pgfpathlineto{\pgfqpoint{3.288335in}{2.139730in}}%
\pgfpathlineto{\pgfqpoint{3.280082in}{2.135648in}}%
\pgfpathlineto{\pgfqpoint{3.271819in}{2.131709in}}%
\pgfpathlineto{\pgfqpoint{3.263547in}{2.127918in}}%
\pgfpathlineto{\pgfqpoint{3.255265in}{2.124277in}}%
\pgfpathlineto{\pgfqpoint{3.241667in}{2.133267in}}%
\pgfpathlineto{\pgfqpoint{3.228071in}{2.142388in}}%
\pgfpathlineto{\pgfqpoint{3.214476in}{2.151640in}}%
\pgfpathlineto{\pgfqpoint{3.200883in}{2.161024in}}%
\pgfpathlineto{\pgfqpoint{3.209191in}{2.164298in}}%
\pgfpathlineto{\pgfqpoint{3.217489in}{2.167728in}}%
\pgfpathlineto{\pgfqpoint{3.225778in}{2.171310in}}%
\pgfpathlineto{\pgfqpoint{3.234056in}{2.175040in}}%
\pgfpathclose%
\pgfusepath{fill}%
\end{pgfscope}%
\begin{pgfscope}%
\pgfpathrectangle{\pgfqpoint{1.150000in}{0.150000in}}{\pgfqpoint{5.700000in}{5.700000in}}%
\pgfusepath{clip}%
\pgfsetbuttcap%
\pgfsetroundjoin%
\definecolor{currentfill}{rgb}{0.272594,0.025563,0.353093}%
\pgfsetfillcolor{currentfill}%
\pgfsetfillopacity{0.700000}%
\pgfsetlinewidth{0.000000pt}%
\definecolor{currentstroke}{rgb}{0.000000,0.000000,0.000000}%
\pgfsetstrokecolor{currentstroke}%
\pgfsetdash{}{0pt}%
\pgfpathmoveto{\pgfqpoint{3.429722in}{2.095709in}}%
\pgfpathlineto{\pgfqpoint{3.443292in}{2.088558in}}%
\pgfpathlineto{\pgfqpoint{3.456865in}{2.081532in}}%
\pgfpathlineto{\pgfqpoint{3.470442in}{2.074627in}}%
\pgfpathlineto{\pgfqpoint{3.484021in}{2.067846in}}%
\pgfpathlineto{\pgfqpoint{3.475871in}{2.062225in}}%
\pgfpathlineto{\pgfqpoint{3.467713in}{2.056722in}}%
\pgfpathlineto{\pgfqpoint{3.459546in}{2.051338in}}%
\pgfpathlineto{\pgfqpoint{3.451371in}{2.046077in}}%
\pgfpathlineto{\pgfqpoint{3.437770in}{2.053197in}}%
\pgfpathlineto{\pgfqpoint{3.424172in}{2.060439in}}%
\pgfpathlineto{\pgfqpoint{3.410578in}{2.067804in}}%
\pgfpathlineto{\pgfqpoint{3.396986in}{2.075293in}}%
\pgfpathlineto{\pgfqpoint{3.405183in}{2.080209in}}%
\pgfpathlineto{\pgfqpoint{3.413371in}{2.085252in}}%
\pgfpathlineto{\pgfqpoint{3.421551in}{2.090420in}}%
\pgfpathlineto{\pgfqpoint{3.429722in}{2.095709in}}%
\pgfpathclose%
\pgfusepath{fill}%
\end{pgfscope}%
\begin{pgfscope}%
\pgfpathrectangle{\pgfqpoint{1.150000in}{0.150000in}}{\pgfqpoint{5.700000in}{5.700000in}}%
\pgfusepath{clip}%
\pgfsetbuttcap%
\pgfsetroundjoin%
\definecolor{currentfill}{rgb}{0.265145,0.232956,0.516599}%
\pgfsetfillcolor{currentfill}%
\pgfsetfillopacity{0.700000}%
\pgfsetlinewidth{0.000000pt}%
\definecolor{currentstroke}{rgb}{0.000000,0.000000,0.000000}%
\pgfsetstrokecolor{currentstroke}%
\pgfsetdash{}{0pt}%
\pgfpathmoveto{\pgfqpoint{4.856331in}{2.483390in}}%
\pgfpathlineto{\pgfqpoint{4.870294in}{2.486900in}}%
\pgfpathlineto{\pgfqpoint{4.884269in}{2.490511in}}%
\pgfpathlineto{\pgfqpoint{4.898255in}{2.494223in}}%
\pgfpathlineto{\pgfqpoint{4.912254in}{2.498036in}}%
\pgfpathlineto{\pgfqpoint{4.904623in}{2.488162in}}%
\pgfpathlineto{\pgfqpoint{4.896985in}{2.478220in}}%
\pgfpathlineto{\pgfqpoint{4.889342in}{2.468210in}}%
\pgfpathlineto{\pgfqpoint{4.881693in}{2.458133in}}%
\pgfpathlineto{\pgfqpoint{4.867689in}{2.454403in}}%
\pgfpathlineto{\pgfqpoint{4.853696in}{2.450774in}}%
\pgfpathlineto{\pgfqpoint{4.839715in}{2.447246in}}%
\pgfpathlineto{\pgfqpoint{4.825746in}{2.443818in}}%
\pgfpathlineto{\pgfqpoint{4.833401in}{2.453806in}}%
\pgfpathlineto{\pgfqpoint{4.841050in}{2.463730in}}%
\pgfpathlineto{\pgfqpoint{4.848693in}{2.473592in}}%
\pgfpathlineto{\pgfqpoint{4.856331in}{2.483390in}}%
\pgfpathclose%
\pgfusepath{fill}%
\end{pgfscope}%
\begin{pgfscope}%
\pgfpathrectangle{\pgfqpoint{1.150000in}{0.150000in}}{\pgfqpoint{5.700000in}{5.700000in}}%
\pgfusepath{clip}%
\pgfsetbuttcap%
\pgfsetroundjoin%
\definecolor{currentfill}{rgb}{0.276022,0.044167,0.370164}%
\pgfsetfillcolor{currentfill}%
\pgfsetfillopacity{0.700000}%
\pgfsetlinewidth{0.000000pt}%
\definecolor{currentstroke}{rgb}{0.000000,0.000000,0.000000}%
\pgfsetstrokecolor{currentstroke}%
\pgfsetdash{}{0pt}%
\pgfpathmoveto{\pgfqpoint{4.166199in}{2.121954in}}%
\pgfpathlineto{\pgfqpoint{4.179903in}{2.120989in}}%
\pgfpathlineto{\pgfqpoint{4.193616in}{2.120130in}}%
\pgfpathlineto{\pgfqpoint{4.207336in}{2.119377in}}%
\pgfpathlineto{\pgfqpoint{4.221064in}{2.118730in}}%
\pgfpathlineto{\pgfqpoint{4.213206in}{2.108947in}}%
\pgfpathlineto{\pgfqpoint{4.205342in}{2.099173in}}%
\pgfpathlineto{\pgfqpoint{4.197473in}{2.089410in}}%
\pgfpathlineto{\pgfqpoint{4.189598in}{2.079660in}}%
\pgfpathlineto{\pgfqpoint{4.175861in}{2.080535in}}%
\pgfpathlineto{\pgfqpoint{4.162131in}{2.081516in}}%
\pgfpathlineto{\pgfqpoint{4.148410in}{2.082603in}}%
\pgfpathlineto{\pgfqpoint{4.134696in}{2.083795in}}%
\pgfpathlineto{\pgfqpoint{4.142580in}{2.093311in}}%
\pgfpathlineto{\pgfqpoint{4.150458in}{2.102844in}}%
\pgfpathlineto{\pgfqpoint{4.158331in}{2.112393in}}%
\pgfpathlineto{\pgfqpoint{4.166199in}{2.121954in}}%
\pgfpathclose%
\pgfusepath{fill}%
\end{pgfscope}%
\begin{pgfscope}%
\pgfpathrectangle{\pgfqpoint{1.150000in}{0.150000in}}{\pgfqpoint{5.700000in}{5.700000in}}%
\pgfusepath{clip}%
\pgfsetbuttcap%
\pgfsetroundjoin%
\definecolor{currentfill}{rgb}{0.179019,0.433756,0.557430}%
\pgfsetfillcolor{currentfill}%
\pgfsetfillopacity{0.700000}%
\pgfsetlinewidth{0.000000pt}%
\definecolor{currentstroke}{rgb}{0.000000,0.000000,0.000000}%
\pgfsetstrokecolor{currentstroke}%
\pgfsetdash{}{0pt}%
\pgfpathmoveto{\pgfqpoint{5.577764in}{2.952352in}}%
\pgfpathlineto{\pgfqpoint{5.592071in}{2.958962in}}%
\pgfpathlineto{\pgfqpoint{5.606394in}{2.965671in}}%
\pgfpathlineto{\pgfqpoint{5.620731in}{2.972480in}}%
\pgfpathlineto{\pgfqpoint{5.635083in}{2.979388in}}%
\pgfpathlineto{\pgfqpoint{5.627778in}{2.972658in}}%
\pgfpathlineto{\pgfqpoint{5.620466in}{2.965835in}}%
\pgfpathlineto{\pgfqpoint{5.613145in}{2.958920in}}%
\pgfpathlineto{\pgfqpoint{5.605816in}{2.951909in}}%
\pgfpathlineto{\pgfqpoint{5.591451in}{2.944913in}}%
\pgfpathlineto{\pgfqpoint{5.577102in}{2.938016in}}%
\pgfpathlineto{\pgfqpoint{5.562768in}{2.931218in}}%
\pgfpathlineto{\pgfqpoint{5.548448in}{2.924520in}}%
\pgfpathlineto{\pgfqpoint{5.555789in}{2.931612in}}%
\pgfpathlineto{\pgfqpoint{5.563122in}{2.938614in}}%
\pgfpathlineto{\pgfqpoint{5.570447in}{2.945527in}}%
\pgfpathlineto{\pgfqpoint{5.577764in}{2.952352in}}%
\pgfpathclose%
\pgfusepath{fill}%
\end{pgfscope}%
\begin{pgfscope}%
\pgfpathrectangle{\pgfqpoint{1.150000in}{0.150000in}}{\pgfqpoint{5.700000in}{5.700000in}}%
\pgfusepath{clip}%
\pgfsetbuttcap%
\pgfsetroundjoin%
\definecolor{currentfill}{rgb}{0.267004,0.004874,0.329415}%
\pgfsetfillcolor{currentfill}%
\pgfsetfillopacity{0.700000}%
\pgfsetlinewidth{0.000000pt}%
\definecolor{currentstroke}{rgb}{0.000000,0.000000,0.000000}%
\pgfsetstrokecolor{currentstroke}%
\pgfsetdash{}{0pt}%
\pgfpathmoveto{\pgfqpoint{3.852635in}{2.050449in}}%
\pgfpathlineto{\pgfqpoint{3.866261in}{2.047009in}}%
\pgfpathlineto{\pgfqpoint{3.879894in}{2.043680in}}%
\pgfpathlineto{\pgfqpoint{3.893532in}{2.040462in}}%
\pgfpathlineto{\pgfqpoint{3.907177in}{2.037355in}}%
\pgfpathlineto{\pgfqpoint{3.899209in}{2.028921in}}%
\pgfpathlineto{\pgfqpoint{3.891235in}{2.020540in}}%
\pgfpathlineto{\pgfqpoint{3.883255in}{2.012217in}}%
\pgfpathlineto{\pgfqpoint{3.875269in}{2.003953in}}%
\pgfpathlineto{\pgfqpoint{3.861611in}{2.007342in}}%
\pgfpathlineto{\pgfqpoint{3.847959in}{2.010842in}}%
\pgfpathlineto{\pgfqpoint{3.834313in}{2.014452in}}%
\pgfpathlineto{\pgfqpoint{3.820673in}{2.018174in}}%
\pgfpathlineto{\pgfqpoint{3.828673in}{2.026149in}}%
\pgfpathlineto{\pgfqpoint{3.836666in}{2.034189in}}%
\pgfpathlineto{\pgfqpoint{3.844653in}{2.042289in}}%
\pgfpathlineto{\pgfqpoint{3.852635in}{2.050449in}}%
\pgfpathclose%
\pgfusepath{fill}%
\end{pgfscope}%
\begin{pgfscope}%
\pgfpathrectangle{\pgfqpoint{1.150000in}{0.150000in}}{\pgfqpoint{5.700000in}{5.700000in}}%
\pgfusepath{clip}%
\pgfsetbuttcap%
\pgfsetroundjoin%
\definecolor{currentfill}{rgb}{0.281412,0.155834,0.469201}%
\pgfsetfillcolor{currentfill}%
\pgfsetfillopacity{0.700000}%
\pgfsetlinewidth{0.000000pt}%
\definecolor{currentstroke}{rgb}{0.000000,0.000000,0.000000}%
\pgfsetstrokecolor{currentstroke}%
\pgfsetdash{}{0pt}%
\pgfpathmoveto{\pgfqpoint{2.983442in}{2.329862in}}%
\pgfpathlineto{\pgfqpoint{2.997034in}{2.318242in}}%
\pgfpathlineto{\pgfqpoint{3.010625in}{2.306770in}}%
\pgfpathlineto{\pgfqpoint{3.024215in}{2.295443in}}%
\pgfpathlineto{\pgfqpoint{3.037805in}{2.284261in}}%
\pgfpathlineto{\pgfqpoint{3.029400in}{2.282266in}}%
\pgfpathlineto{\pgfqpoint{3.020983in}{2.280450in}}%
\pgfpathlineto{\pgfqpoint{3.012554in}{2.278814in}}%
\pgfpathlineto{\pgfqpoint{3.004114in}{2.277364in}}%
\pgfpathlineto{\pgfqpoint{2.990493in}{2.288929in}}%
\pgfpathlineto{\pgfqpoint{2.976871in}{2.300639in}}%
\pgfpathlineto{\pgfqpoint{2.963249in}{2.312496in}}%
\pgfpathlineto{\pgfqpoint{2.949625in}{2.324499in}}%
\pgfpathlineto{\pgfqpoint{2.958098in}{2.325559in}}%
\pgfpathlineto{\pgfqpoint{2.966558in}{2.326808in}}%
\pgfpathlineto{\pgfqpoint{2.975006in}{2.328244in}}%
\pgfpathlineto{\pgfqpoint{2.983442in}{2.329862in}}%
\pgfpathclose%
\pgfusepath{fill}%
\end{pgfscope}%
\begin{pgfscope}%
\pgfpathrectangle{\pgfqpoint{1.150000in}{0.150000in}}{\pgfqpoint{5.700000in}{5.700000in}}%
\pgfusepath{clip}%
\pgfsetbuttcap%
\pgfsetroundjoin%
\definecolor{currentfill}{rgb}{0.255645,0.260703,0.528312}%
\pgfsetfillcolor{currentfill}%
\pgfsetfillopacity{0.700000}%
\pgfsetlinewidth{0.000000pt}%
\definecolor{currentstroke}{rgb}{0.000000,0.000000,0.000000}%
\pgfsetstrokecolor{currentstroke}%
\pgfsetdash{}{0pt}%
\pgfpathmoveto{\pgfqpoint{4.942717in}{2.536841in}}%
\pgfpathlineto{\pgfqpoint{4.956721in}{2.540819in}}%
\pgfpathlineto{\pgfqpoint{4.970738in}{2.544898in}}%
\pgfpathlineto{\pgfqpoint{4.984766in}{2.549077in}}%
\pgfpathlineto{\pgfqpoint{4.998807in}{2.553357in}}%
\pgfpathlineto{\pgfqpoint{4.991207in}{2.543703in}}%
\pgfpathlineto{\pgfqpoint{4.983600in}{2.533974in}}%
\pgfpathlineto{\pgfqpoint{4.975987in}{2.524172in}}%
\pgfpathlineto{\pgfqpoint{4.968368in}{2.514295in}}%
\pgfpathlineto{\pgfqpoint{4.954321in}{2.510079in}}%
\pgfpathlineto{\pgfqpoint{4.940287in}{2.505964in}}%
\pgfpathlineto{\pgfqpoint{4.926264in}{2.501950in}}%
\pgfpathlineto{\pgfqpoint{4.912254in}{2.498036in}}%
\pgfpathlineto{\pgfqpoint{4.919879in}{2.507842in}}%
\pgfpathlineto{\pgfqpoint{4.927498in}{2.517578in}}%
\pgfpathlineto{\pgfqpoint{4.935111in}{2.527244in}}%
\pgfpathlineto{\pgfqpoint{4.942717in}{2.536841in}}%
\pgfpathclose%
\pgfusepath{fill}%
\end{pgfscope}%
\begin{pgfscope}%
\pgfpathrectangle{\pgfqpoint{1.150000in}{0.150000in}}{\pgfqpoint{5.700000in}{5.700000in}}%
\pgfusepath{clip}%
\pgfsetbuttcap%
\pgfsetroundjoin%
\definecolor{currentfill}{rgb}{0.171176,0.452530,0.557965}%
\pgfsetfillcolor{currentfill}%
\pgfsetfillopacity{0.700000}%
\pgfsetlinewidth{0.000000pt}%
\definecolor{currentstroke}{rgb}{0.000000,0.000000,0.000000}%
\pgfsetstrokecolor{currentstroke}%
\pgfsetdash{}{0pt}%
\pgfpathmoveto{\pgfqpoint{5.664219in}{3.005411in}}%
\pgfpathlineto{\pgfqpoint{5.678573in}{3.012310in}}%
\pgfpathlineto{\pgfqpoint{5.692942in}{3.019309in}}%
\pgfpathlineto{\pgfqpoint{5.707326in}{3.026407in}}%
\pgfpathlineto{\pgfqpoint{5.721725in}{3.033604in}}%
\pgfpathlineto{\pgfqpoint{5.714467in}{3.027345in}}%
\pgfpathlineto{\pgfqpoint{5.707200in}{3.020994in}}%
\pgfpathlineto{\pgfqpoint{5.699925in}{3.014551in}}%
\pgfpathlineto{\pgfqpoint{5.692642in}{3.008013in}}%
\pgfpathlineto{\pgfqpoint{5.678229in}{3.000707in}}%
\pgfpathlineto{\pgfqpoint{5.663832in}{2.993501in}}%
\pgfpathlineto{\pgfqpoint{5.649450in}{2.986395in}}%
\pgfpathlineto{\pgfqpoint{5.635083in}{2.979388in}}%
\pgfpathlineto{\pgfqpoint{5.642379in}{2.986026in}}%
\pgfpathlineto{\pgfqpoint{5.649667in}{2.992576in}}%
\pgfpathlineto{\pgfqpoint{5.656947in}{2.999037in}}%
\pgfpathlineto{\pgfqpoint{5.664219in}{3.005411in}}%
\pgfpathclose%
\pgfusepath{fill}%
\end{pgfscope}%
\begin{pgfscope}%
\pgfpathrectangle{\pgfqpoint{1.150000in}{0.150000in}}{\pgfqpoint{5.700000in}{5.700000in}}%
\pgfusepath{clip}%
\pgfsetbuttcap%
\pgfsetroundjoin%
\definecolor{currentfill}{rgb}{0.273809,0.031497,0.358853}%
\pgfsetfillcolor{currentfill}%
\pgfsetfillopacity{0.700000}%
\pgfsetlinewidth{0.000000pt}%
\definecolor{currentstroke}{rgb}{0.000000,0.000000,0.000000}%
\pgfsetstrokecolor{currentstroke}%
\pgfsetdash{}{0pt}%
\pgfpathmoveto{\pgfqpoint{4.079918in}{2.089632in}}%
\pgfpathlineto{\pgfqpoint{4.093601in}{2.088013in}}%
\pgfpathlineto{\pgfqpoint{4.107292in}{2.086500in}}%
\pgfpathlineto{\pgfqpoint{4.120990in}{2.085095in}}%
\pgfpathlineto{\pgfqpoint{4.134696in}{2.083795in}}%
\pgfpathlineto{\pgfqpoint{4.126807in}{2.074299in}}%
\pgfpathlineto{\pgfqpoint{4.118912in}{2.064824in}}%
\pgfpathlineto{\pgfqpoint{4.111013in}{2.055372in}}%
\pgfpathlineto{\pgfqpoint{4.103107in}{2.045947in}}%
\pgfpathlineto{\pgfqpoint{4.089391in}{2.047492in}}%
\pgfpathlineto{\pgfqpoint{4.075683in}{2.049144in}}%
\pgfpathlineto{\pgfqpoint{4.061982in}{2.050902in}}%
\pgfpathlineto{\pgfqpoint{4.048288in}{2.052767in}}%
\pgfpathlineto{\pgfqpoint{4.056204in}{2.061940in}}%
\pgfpathlineto{\pgfqpoint{4.064114in}{2.071143in}}%
\pgfpathlineto{\pgfqpoint{4.072019in}{2.080375in}}%
\pgfpathlineto{\pgfqpoint{4.079918in}{2.089632in}}%
\pgfpathclose%
\pgfusepath{fill}%
\end{pgfscope}%
\begin{pgfscope}%
\pgfpathrectangle{\pgfqpoint{1.150000in}{0.150000in}}{\pgfqpoint{5.700000in}{5.700000in}}%
\pgfusepath{clip}%
\pgfsetbuttcap%
\pgfsetroundjoin%
\definecolor{currentfill}{rgb}{0.162142,0.474838,0.558140}%
\pgfsetfillcolor{currentfill}%
\pgfsetfillopacity{0.700000}%
\pgfsetlinewidth{0.000000pt}%
\definecolor{currentstroke}{rgb}{0.000000,0.000000,0.000000}%
\pgfsetstrokecolor{currentstroke}%
\pgfsetdash{}{0pt}%
\pgfpathmoveto{\pgfqpoint{5.750674in}{3.057754in}}%
\pgfpathlineto{\pgfqpoint{5.765074in}{3.064923in}}%
\pgfpathlineto{\pgfqpoint{5.779489in}{3.072191in}}%
\pgfpathlineto{\pgfqpoint{5.793920in}{3.079558in}}%
\pgfpathlineto{\pgfqpoint{5.808366in}{3.087025in}}%
\pgfpathlineto{\pgfqpoint{5.801157in}{3.081252in}}%
\pgfpathlineto{\pgfqpoint{5.793938in}{3.075388in}}%
\pgfpathlineto{\pgfqpoint{5.786711in}{3.069434in}}%
\pgfpathlineto{\pgfqpoint{5.779476in}{3.063386in}}%
\pgfpathlineto{\pgfqpoint{5.765015in}{3.055791in}}%
\pgfpathlineto{\pgfqpoint{5.750569in}{3.048296in}}%
\pgfpathlineto{\pgfqpoint{5.736139in}{3.040900in}}%
\pgfpathlineto{\pgfqpoint{5.721725in}{3.033604in}}%
\pgfpathlineto{\pgfqpoint{5.728975in}{3.039773in}}%
\pgfpathlineto{\pgfqpoint{5.736216in}{3.045853in}}%
\pgfpathlineto{\pgfqpoint{5.743449in}{3.051846in}}%
\pgfpathlineto{\pgfqpoint{5.750674in}{3.057754in}}%
\pgfpathclose%
\pgfusepath{fill}%
\end{pgfscope}%
\begin{pgfscope}%
\pgfpathrectangle{\pgfqpoint{1.150000in}{0.150000in}}{\pgfqpoint{5.700000in}{5.700000in}}%
\pgfusepath{clip}%
\pgfsetbuttcap%
\pgfsetroundjoin%
\definecolor{currentfill}{rgb}{0.244972,0.287675,0.537260}%
\pgfsetfillcolor{currentfill}%
\pgfsetfillopacity{0.700000}%
\pgfsetlinewidth{0.000000pt}%
\definecolor{currentstroke}{rgb}{0.000000,0.000000,0.000000}%
\pgfsetstrokecolor{currentstroke}%
\pgfsetdash{}{0pt}%
\pgfpathmoveto{\pgfqpoint{5.029144in}{2.591224in}}%
\pgfpathlineto{\pgfqpoint{5.043191in}{2.595651in}}%
\pgfpathlineto{\pgfqpoint{5.057249in}{2.600177in}}%
\pgfpathlineto{\pgfqpoint{5.071321in}{2.604804in}}%
\pgfpathlineto{\pgfqpoint{5.085405in}{2.609531in}}%
\pgfpathlineto{\pgfqpoint{5.077837in}{2.600139in}}%
\pgfpathlineto{\pgfqpoint{5.070263in}{2.590666in}}%
\pgfpathlineto{\pgfqpoint{5.062682in}{2.581114in}}%
\pgfpathlineto{\pgfqpoint{5.055094in}{2.571481in}}%
\pgfpathlineto{\pgfqpoint{5.041004in}{2.566800in}}%
\pgfpathlineto{\pgfqpoint{5.026925in}{2.562219in}}%
\pgfpathlineto{\pgfqpoint{5.012860in}{2.557738in}}%
\pgfpathlineto{\pgfqpoint{4.998807in}{2.553357in}}%
\pgfpathlineto{\pgfqpoint{5.006401in}{2.562937in}}%
\pgfpathlineto{\pgfqpoint{5.013988in}{2.572441in}}%
\pgfpathlineto{\pgfqpoint{5.021569in}{2.581870in}}%
\pgfpathlineto{\pgfqpoint{5.029144in}{2.591224in}}%
\pgfpathclose%
\pgfusepath{fill}%
\end{pgfscope}%
\begin{pgfscope}%
\pgfpathrectangle{\pgfqpoint{1.150000in}{0.150000in}}{\pgfqpoint{5.700000in}{5.700000in}}%
\pgfusepath{clip}%
\pgfsetbuttcap%
\pgfsetroundjoin%
\definecolor{currentfill}{rgb}{0.278791,0.062145,0.386592}%
\pgfsetfillcolor{currentfill}%
\pgfsetfillopacity{0.700000}%
\pgfsetlinewidth{0.000000pt}%
\definecolor{currentstroke}{rgb}{0.000000,0.000000,0.000000}%
\pgfsetstrokecolor{currentstroke}%
\pgfsetdash{}{0pt}%
\pgfpathmoveto{\pgfqpoint{3.288335in}{2.139730in}}%
\pgfpathlineto{\pgfqpoint{3.301909in}{2.131229in}}%
\pgfpathlineto{\pgfqpoint{3.315485in}{2.122857in}}%
\pgfpathlineto{\pgfqpoint{3.329063in}{2.114613in}}%
\pgfpathlineto{\pgfqpoint{3.342643in}{2.106497in}}%
\pgfpathlineto{\pgfqpoint{3.334414in}{2.102063in}}%
\pgfpathlineto{\pgfqpoint{3.326176in}{2.097768in}}%
\pgfpathlineto{\pgfqpoint{3.317928in}{2.093616in}}%
\pgfpathlineto{\pgfqpoint{3.309671in}{2.089610in}}%
\pgfpathlineto{\pgfqpoint{3.296067in}{2.098084in}}%
\pgfpathlineto{\pgfqpoint{3.282464in}{2.106686in}}%
\pgfpathlineto{\pgfqpoint{3.268864in}{2.115417in}}%
\pgfpathlineto{\pgfqpoint{3.255265in}{2.124277in}}%
\pgfpathlineto{\pgfqpoint{3.263547in}{2.127918in}}%
\pgfpathlineto{\pgfqpoint{3.271819in}{2.131709in}}%
\pgfpathlineto{\pgfqpoint{3.280082in}{2.135648in}}%
\pgfpathlineto{\pgfqpoint{3.288335in}{2.139730in}}%
\pgfpathclose%
\pgfusepath{fill}%
\end{pgfscope}%
\begin{pgfscope}%
\pgfpathrectangle{\pgfqpoint{1.150000in}{0.150000in}}{\pgfqpoint{5.700000in}{5.700000in}}%
\pgfusepath{clip}%
\pgfsetbuttcap%
\pgfsetroundjoin%
\definecolor{currentfill}{rgb}{0.282884,0.135920,0.453427}%
\pgfsetfillcolor{currentfill}%
\pgfsetfillopacity{0.700000}%
\pgfsetlinewidth{0.000000pt}%
\definecolor{currentstroke}{rgb}{0.000000,0.000000,0.000000}%
\pgfsetstrokecolor{currentstroke}%
\pgfsetdash{}{0pt}%
\pgfpathmoveto{\pgfqpoint{3.037805in}{2.284261in}}%
\pgfpathlineto{\pgfqpoint{3.051394in}{2.273222in}}%
\pgfpathlineto{\pgfqpoint{3.064983in}{2.262327in}}%
\pgfpathlineto{\pgfqpoint{3.078572in}{2.251573in}}%
\pgfpathlineto{\pgfqpoint{3.092160in}{2.240961in}}%
\pgfpathlineto{\pgfqpoint{3.083785in}{2.238592in}}%
\pgfpathlineto{\pgfqpoint{3.075399in}{2.236396in}}%
\pgfpathlineto{\pgfqpoint{3.067001in}{2.234376in}}%
\pgfpathlineto{\pgfqpoint{3.058591in}{2.232536in}}%
\pgfpathlineto{\pgfqpoint{3.044972in}{2.243530in}}%
\pgfpathlineto{\pgfqpoint{3.031353in}{2.254666in}}%
\pgfpathlineto{\pgfqpoint{3.017734in}{2.265943in}}%
\pgfpathlineto{\pgfqpoint{3.004114in}{2.277364in}}%
\pgfpathlineto{\pgfqpoint{3.012554in}{2.278814in}}%
\pgfpathlineto{\pgfqpoint{3.020983in}{2.280450in}}%
\pgfpathlineto{\pgfqpoint{3.029400in}{2.282266in}}%
\pgfpathlineto{\pgfqpoint{3.037805in}{2.284261in}}%
\pgfpathclose%
\pgfusepath{fill}%
\end{pgfscope}%
\begin{pgfscope}%
\pgfpathrectangle{\pgfqpoint{1.150000in}{0.150000in}}{\pgfqpoint{5.700000in}{5.700000in}}%
\pgfusepath{clip}%
\pgfsetbuttcap%
\pgfsetroundjoin%
\definecolor{currentfill}{rgb}{0.154815,0.493313,0.557840}%
\pgfsetfillcolor{currentfill}%
\pgfsetfillopacity{0.700000}%
\pgfsetlinewidth{0.000000pt}%
\definecolor{currentstroke}{rgb}{0.000000,0.000000,0.000000}%
\pgfsetstrokecolor{currentstroke}%
\pgfsetdash{}{0pt}%
\pgfpathmoveto{\pgfqpoint{5.837120in}{3.109251in}}%
\pgfpathlineto{\pgfqpoint{5.851566in}{3.116669in}}%
\pgfpathlineto{\pgfqpoint{5.866028in}{3.124186in}}%
\pgfpathlineto{\pgfqpoint{5.880505in}{3.131803in}}%
\pgfpathlineto{\pgfqpoint{5.894999in}{3.139518in}}%
\pgfpathlineto{\pgfqpoint{5.887840in}{3.134243in}}%
\pgfpathlineto{\pgfqpoint{5.880672in}{3.128880in}}%
\pgfpathlineto{\pgfqpoint{5.873495in}{3.123427in}}%
\pgfpathlineto{\pgfqpoint{5.866309in}{3.117883in}}%
\pgfpathlineto{\pgfqpoint{5.851800in}{3.110019in}}%
\pgfpathlineto{\pgfqpoint{5.837306in}{3.102255in}}%
\pgfpathlineto{\pgfqpoint{5.822828in}{3.094590in}}%
\pgfpathlineto{\pgfqpoint{5.808366in}{3.087025in}}%
\pgfpathlineto{\pgfqpoint{5.815568in}{3.092709in}}%
\pgfpathlineto{\pgfqpoint{5.822760in}{3.098307in}}%
\pgfpathlineto{\pgfqpoint{5.829944in}{3.103821in}}%
\pgfpathlineto{\pgfqpoint{5.837120in}{3.109251in}}%
\pgfpathclose%
\pgfusepath{fill}%
\end{pgfscope}%
\begin{pgfscope}%
\pgfpathrectangle{\pgfqpoint{1.150000in}{0.150000in}}{\pgfqpoint{5.700000in}{5.700000in}}%
\pgfusepath{clip}%
\pgfsetbuttcap%
\pgfsetroundjoin%
\definecolor{currentfill}{rgb}{0.267004,0.004874,0.329415}%
\pgfsetfillcolor{currentfill}%
\pgfsetfillopacity{0.700000}%
\pgfsetlinewidth{0.000000pt}%
\definecolor{currentstroke}{rgb}{0.000000,0.000000,0.000000}%
\pgfsetstrokecolor{currentstroke}%
\pgfsetdash{}{0pt}%
\pgfpathmoveto{\pgfqpoint{3.625154in}{2.044053in}}%
\pgfpathlineto{\pgfqpoint{3.638749in}{2.038664in}}%
\pgfpathlineto{\pgfqpoint{3.652349in}{2.033391in}}%
\pgfpathlineto{\pgfqpoint{3.665953in}{2.028234in}}%
\pgfpathlineto{\pgfqpoint{3.679561in}{2.023193in}}%
\pgfpathlineto{\pgfqpoint{3.671498in}{2.016202in}}%
\pgfpathlineto{\pgfqpoint{3.663427in}{2.009302in}}%
\pgfpathlineto{\pgfqpoint{3.655348in}{2.002495in}}%
\pgfpathlineto{\pgfqpoint{3.647263in}{1.995784in}}%
\pgfpathlineto{\pgfqpoint{3.633637in}{2.001144in}}%
\pgfpathlineto{\pgfqpoint{3.620015in}{2.006619in}}%
\pgfpathlineto{\pgfqpoint{3.606398in}{2.012211in}}%
\pgfpathlineto{\pgfqpoint{3.592785in}{2.017919in}}%
\pgfpathlineto{\pgfqpoint{3.600889in}{2.024304in}}%
\pgfpathlineto{\pgfqpoint{3.608984in}{2.030790in}}%
\pgfpathlineto{\pgfqpoint{3.617073in}{2.037374in}}%
\pgfpathlineto{\pgfqpoint{3.625154in}{2.044053in}}%
\pgfpathclose%
\pgfusepath{fill}%
\end{pgfscope}%
\begin{pgfscope}%
\pgfpathrectangle{\pgfqpoint{1.150000in}{0.150000in}}{\pgfqpoint{5.700000in}{5.700000in}}%
\pgfusepath{clip}%
\pgfsetbuttcap%
\pgfsetroundjoin%
\definecolor{currentfill}{rgb}{0.127568,0.566949,0.550556}%
\pgfsetfillcolor{currentfill}%
\pgfsetfillopacity{0.700000}%
\pgfsetlinewidth{0.000000pt}%
\definecolor{currentstroke}{rgb}{0.000000,0.000000,0.000000}%
\pgfsetstrokecolor{currentstroke}%
\pgfsetdash{}{0pt}%
\pgfpathmoveto{\pgfqpoint{6.182652in}{3.304571in}}%
\pgfpathlineto{\pgfqpoint{6.197279in}{3.312785in}}%
\pgfpathlineto{\pgfqpoint{6.211923in}{3.321098in}}%
\pgfpathlineto{\pgfqpoint{6.226584in}{3.329509in}}%
\pgfpathlineto{\pgfqpoint{6.219640in}{3.326220in}}%
\pgfpathlineto{\pgfqpoint{6.212686in}{3.322860in}}%
\pgfpathlineto{\pgfqpoint{6.205724in}{3.319429in}}%
\pgfpathlineto{\pgfqpoint{6.198753in}{3.315924in}}%
\pgfpathlineto{\pgfqpoint{6.184070in}{3.307285in}}%
\pgfpathlineto{\pgfqpoint{6.169404in}{3.298745in}}%
\pgfpathlineto{\pgfqpoint{6.154754in}{3.290304in}}%
\pgfpathlineto{\pgfqpoint{6.161742in}{3.293975in}}%
\pgfpathlineto{\pgfqpoint{6.168720in}{3.297574in}}%
\pgfpathlineto{\pgfqpoint{6.175690in}{3.301105in}}%
\pgfpathlineto{\pgfqpoint{6.182652in}{3.304571in}}%
\pgfpathclose%
\pgfusepath{fill}%
\end{pgfscope}%
\begin{pgfscope}%
\pgfpathrectangle{\pgfqpoint{1.150000in}{0.150000in}}{\pgfqpoint{5.700000in}{5.700000in}}%
\pgfusepath{clip}%
\pgfsetbuttcap%
\pgfsetroundjoin%
\definecolor{currentfill}{rgb}{0.235526,0.309527,0.542944}%
\pgfsetfillcolor{currentfill}%
\pgfsetfillopacity{0.700000}%
\pgfsetlinewidth{0.000000pt}%
\definecolor{currentstroke}{rgb}{0.000000,0.000000,0.000000}%
\pgfsetstrokecolor{currentstroke}%
\pgfsetdash{}{0pt}%
\pgfpathmoveto{\pgfqpoint{5.115611in}{2.646300in}}%
\pgfpathlineto{\pgfqpoint{5.129701in}{2.651155in}}%
\pgfpathlineto{\pgfqpoint{5.143803in}{2.656109in}}%
\pgfpathlineto{\pgfqpoint{5.157919in}{2.661164in}}%
\pgfpathlineto{\pgfqpoint{5.172048in}{2.666318in}}%
\pgfpathlineto{\pgfqpoint{5.164514in}{2.657226in}}%
\pgfpathlineto{\pgfqpoint{5.156973in}{2.648050in}}%
\pgfpathlineto{\pgfqpoint{5.149426in}{2.638788in}}%
\pgfpathlineto{\pgfqpoint{5.141871in}{2.629442in}}%
\pgfpathlineto{\pgfqpoint{5.127735in}{2.624314in}}%
\pgfpathlineto{\pgfqpoint{5.113612in}{2.619286in}}%
\pgfpathlineto{\pgfqpoint{5.099502in}{2.614359in}}%
\pgfpathlineto{\pgfqpoint{5.085405in}{2.609531in}}%
\pgfpathlineto{\pgfqpoint{5.092967in}{2.618844in}}%
\pgfpathlineto{\pgfqpoint{5.100521in}{2.628076in}}%
\pgfpathlineto{\pgfqpoint{5.108069in}{2.637228in}}%
\pgfpathlineto{\pgfqpoint{5.115611in}{2.646300in}}%
\pgfpathclose%
\pgfusepath{fill}%
\end{pgfscope}%
\begin{pgfscope}%
\pgfpathrectangle{\pgfqpoint{1.150000in}{0.150000in}}{\pgfqpoint{5.700000in}{5.700000in}}%
\pgfusepath{clip}%
\pgfsetbuttcap%
\pgfsetroundjoin%
\definecolor{currentfill}{rgb}{0.146180,0.515413,0.556823}%
\pgfsetfillcolor{currentfill}%
\pgfsetfillopacity{0.700000}%
\pgfsetlinewidth{0.000000pt}%
\definecolor{currentstroke}{rgb}{0.000000,0.000000,0.000000}%
\pgfsetstrokecolor{currentstroke}%
\pgfsetdash{}{0pt}%
\pgfpathmoveto{\pgfqpoint{5.923549in}{3.159781in}}%
\pgfpathlineto{\pgfqpoint{5.938041in}{3.167429in}}%
\pgfpathlineto{\pgfqpoint{5.952549in}{3.175175in}}%
\pgfpathlineto{\pgfqpoint{5.967074in}{3.183021in}}%
\pgfpathlineto{\pgfqpoint{5.981614in}{3.190965in}}%
\pgfpathlineto{\pgfqpoint{5.974507in}{3.186196in}}%
\pgfpathlineto{\pgfqpoint{5.967391in}{3.181342in}}%
\pgfpathlineto{\pgfqpoint{5.960267in}{3.176402in}}%
\pgfpathlineto{\pgfqpoint{5.953133in}{3.171373in}}%
\pgfpathlineto{\pgfqpoint{5.938575in}{3.163260in}}%
\pgfpathlineto{\pgfqpoint{5.924034in}{3.155247in}}%
\pgfpathlineto{\pgfqpoint{5.909508in}{3.147333in}}%
\pgfpathlineto{\pgfqpoint{5.894999in}{3.139518in}}%
\pgfpathlineto{\pgfqpoint{5.902149in}{3.144708in}}%
\pgfpathlineto{\pgfqpoint{5.909291in}{3.149814in}}%
\pgfpathlineto{\pgfqpoint{5.916425in}{3.154837in}}%
\pgfpathlineto{\pgfqpoint{5.923549in}{3.159781in}}%
\pgfpathclose%
\pgfusepath{fill}%
\end{pgfscope}%
\begin{pgfscope}%
\pgfpathrectangle{\pgfqpoint{1.150000in}{0.150000in}}{\pgfqpoint{5.700000in}{5.700000in}}%
\pgfusepath{clip}%
\pgfsetbuttcap%
\pgfsetroundjoin%
\definecolor{currentfill}{rgb}{0.271305,0.019942,0.347269}%
\pgfsetfillcolor{currentfill}%
\pgfsetfillopacity{0.700000}%
\pgfsetlinewidth{0.000000pt}%
\definecolor{currentstroke}{rgb}{0.000000,0.000000,0.000000}%
\pgfsetstrokecolor{currentstroke}%
\pgfsetdash{}{0pt}%
\pgfpathmoveto{\pgfqpoint{3.484021in}{2.067846in}}%
\pgfpathlineto{\pgfqpoint{3.497604in}{2.061185in}}%
\pgfpathlineto{\pgfqpoint{3.511190in}{2.054646in}}%
\pgfpathlineto{\pgfqpoint{3.524780in}{2.048227in}}%
\pgfpathlineto{\pgfqpoint{3.538373in}{2.041928in}}%
\pgfpathlineto{\pgfqpoint{3.530243in}{2.035977in}}%
\pgfpathlineto{\pgfqpoint{3.522105in}{2.030138in}}%
\pgfpathlineto{\pgfqpoint{3.513959in}{2.024414in}}%
\pgfpathlineto{\pgfqpoint{3.505805in}{2.018808in}}%
\pgfpathlineto{\pgfqpoint{3.492192in}{2.025445in}}%
\pgfpathlineto{\pgfqpoint{3.478581in}{2.032201in}}%
\pgfpathlineto{\pgfqpoint{3.464974in}{2.039079in}}%
\pgfpathlineto{\pgfqpoint{3.451371in}{2.046077in}}%
\pgfpathlineto{\pgfqpoint{3.459546in}{2.051338in}}%
\pgfpathlineto{\pgfqpoint{3.467713in}{2.056722in}}%
\pgfpathlineto{\pgfqpoint{3.475871in}{2.062225in}}%
\pgfpathlineto{\pgfqpoint{3.484021in}{2.067846in}}%
\pgfpathclose%
\pgfusepath{fill}%
\end{pgfscope}%
\begin{pgfscope}%
\pgfpathrectangle{\pgfqpoint{1.150000in}{0.150000in}}{\pgfqpoint{5.700000in}{5.700000in}}%
\pgfusepath{clip}%
\pgfsetbuttcap%
\pgfsetroundjoin%
\definecolor{currentfill}{rgb}{0.269944,0.014625,0.341379}%
\pgfsetfillcolor{currentfill}%
\pgfsetfillopacity{0.700000}%
\pgfsetlinewidth{0.000000pt}%
\definecolor{currentstroke}{rgb}{0.000000,0.000000,0.000000}%
\pgfsetstrokecolor{currentstroke}%
\pgfsetdash{}{0pt}%
\pgfpathmoveto{\pgfqpoint{3.993584in}{2.061306in}}%
\pgfpathlineto{\pgfqpoint{4.007250in}{2.059009in}}%
\pgfpathlineto{\pgfqpoint{4.020922in}{2.056821in}}%
\pgfpathlineto{\pgfqpoint{4.034601in}{2.054740in}}%
\pgfpathlineto{\pgfqpoint{4.048288in}{2.052767in}}%
\pgfpathlineto{\pgfqpoint{4.040367in}{2.043627in}}%
\pgfpathlineto{\pgfqpoint{4.032440in}{2.034523in}}%
\pgfpathlineto{\pgfqpoint{4.024508in}{2.025455in}}%
\pgfpathlineto{\pgfqpoint{4.016570in}{2.016428in}}%
\pgfpathlineto{\pgfqpoint{4.002872in}{2.018665in}}%
\pgfpathlineto{\pgfqpoint{3.989181in}{2.021009in}}%
\pgfpathlineto{\pgfqpoint{3.975497in}{2.023462in}}%
\pgfpathlineto{\pgfqpoint{3.961820in}{2.026022in}}%
\pgfpathlineto{\pgfqpoint{3.969770in}{2.034779in}}%
\pgfpathlineto{\pgfqpoint{3.977714in}{2.043580in}}%
\pgfpathlineto{\pgfqpoint{3.985652in}{2.052423in}}%
\pgfpathlineto{\pgfqpoint{3.993584in}{2.061306in}}%
\pgfpathclose%
\pgfusepath{fill}%
\end{pgfscope}%
\begin{pgfscope}%
\pgfpathrectangle{\pgfqpoint{1.150000in}{0.150000in}}{\pgfqpoint{5.700000in}{5.700000in}}%
\pgfusepath{clip}%
\pgfsetbuttcap%
\pgfsetroundjoin%
\definecolor{currentfill}{rgb}{0.267004,0.004874,0.329415}%
\pgfsetfillcolor{currentfill}%
\pgfsetfillopacity{0.700000}%
\pgfsetlinewidth{0.000000pt}%
\definecolor{currentstroke}{rgb}{0.000000,0.000000,0.000000}%
\pgfsetstrokecolor{currentstroke}%
\pgfsetdash{}{0pt}%
\pgfpathmoveto{\pgfqpoint{3.766168in}{2.034180in}}%
\pgfpathlineto{\pgfqpoint{3.779786in}{2.030010in}}%
\pgfpathlineto{\pgfqpoint{3.793409in}{2.025952in}}%
\pgfpathlineto{\pgfqpoint{3.807038in}{2.022007in}}%
\pgfpathlineto{\pgfqpoint{3.820673in}{2.018174in}}%
\pgfpathlineto{\pgfqpoint{3.812667in}{2.010265in}}%
\pgfpathlineto{\pgfqpoint{3.804654in}{2.002426in}}%
\pgfpathlineto{\pgfqpoint{3.796635in}{1.994659in}}%
\pgfpathlineto{\pgfqpoint{3.788610in}{1.986968in}}%
\pgfpathlineto{\pgfqpoint{3.774961in}{1.991101in}}%
\pgfpathlineto{\pgfqpoint{3.761317in}{1.995346in}}%
\pgfpathlineto{\pgfqpoint{3.747678in}{1.999704in}}%
\pgfpathlineto{\pgfqpoint{3.734045in}{2.004174in}}%
\pgfpathlineto{\pgfqpoint{3.742085in}{2.011559in}}%
\pgfpathlineto{\pgfqpoint{3.750119in}{2.019023in}}%
\pgfpathlineto{\pgfqpoint{3.758147in}{2.026565in}}%
\pgfpathlineto{\pgfqpoint{3.766168in}{2.034180in}}%
\pgfpathclose%
\pgfusepath{fill}%
\end{pgfscope}%
\begin{pgfscope}%
\pgfpathrectangle{\pgfqpoint{1.150000in}{0.150000in}}{\pgfqpoint{5.700000in}{5.700000in}}%
\pgfusepath{clip}%
\pgfsetbuttcap%
\pgfsetroundjoin%
\definecolor{currentfill}{rgb}{0.139147,0.533812,0.555298}%
\pgfsetfillcolor{currentfill}%
\pgfsetfillopacity{0.700000}%
\pgfsetlinewidth{0.000000pt}%
\definecolor{currentstroke}{rgb}{0.000000,0.000000,0.000000}%
\pgfsetstrokecolor{currentstroke}%
\pgfsetdash{}{0pt}%
\pgfpathmoveto{\pgfqpoint{6.009953in}{3.209239in}}%
\pgfpathlineto{\pgfqpoint{6.024491in}{3.217095in}}%
\pgfpathlineto{\pgfqpoint{6.039045in}{3.225051in}}%
\pgfpathlineto{\pgfqpoint{6.053615in}{3.233105in}}%
\pgfpathlineto{\pgfqpoint{6.068202in}{3.241259in}}%
\pgfpathlineto{\pgfqpoint{6.061149in}{3.237001in}}%
\pgfpathlineto{\pgfqpoint{6.054088in}{3.232663in}}%
\pgfpathlineto{\pgfqpoint{6.047018in}{3.228241in}}%
\pgfpathlineto{\pgfqpoint{6.039938in}{3.223735in}}%
\pgfpathlineto{\pgfqpoint{6.025332in}{3.215394in}}%
\pgfpathlineto{\pgfqpoint{6.010743in}{3.207152in}}%
\pgfpathlineto{\pgfqpoint{5.996170in}{3.199009in}}%
\pgfpathlineto{\pgfqpoint{5.981614in}{3.190965in}}%
\pgfpathlineto{\pgfqpoint{5.988712in}{3.195652in}}%
\pgfpathlineto{\pgfqpoint{5.995801in}{3.200258in}}%
\pgfpathlineto{\pgfqpoint{6.002882in}{3.204786in}}%
\pgfpathlineto{\pgfqpoint{6.009953in}{3.209239in}}%
\pgfpathclose%
\pgfusepath{fill}%
\end{pgfscope}%
\begin{pgfscope}%
\pgfpathrectangle{\pgfqpoint{1.150000in}{0.150000in}}{\pgfqpoint{5.700000in}{5.700000in}}%
\pgfusepath{clip}%
\pgfsetbuttcap%
\pgfsetroundjoin%
\definecolor{currentfill}{rgb}{0.132444,0.552216,0.553018}%
\pgfsetfillcolor{currentfill}%
\pgfsetfillopacity{0.700000}%
\pgfsetlinewidth{0.000000pt}%
\definecolor{currentstroke}{rgb}{0.000000,0.000000,0.000000}%
\pgfsetstrokecolor{currentstroke}%
\pgfsetdash{}{0pt}%
\pgfpathmoveto{\pgfqpoint{6.096324in}{3.257529in}}%
\pgfpathlineto{\pgfqpoint{6.110907in}{3.265575in}}%
\pgfpathlineto{\pgfqpoint{6.125506in}{3.273719in}}%
\pgfpathlineto{\pgfqpoint{6.140122in}{3.281962in}}%
\pgfpathlineto{\pgfqpoint{6.154754in}{3.290304in}}%
\pgfpathlineto{\pgfqpoint{6.147758in}{3.286560in}}%
\pgfpathlineto{\pgfqpoint{6.140753in}{3.282740in}}%
\pgfpathlineto{\pgfqpoint{6.133738in}{3.278842in}}%
\pgfpathlineto{\pgfqpoint{6.126715in}{3.274863in}}%
\pgfpathlineto{\pgfqpoint{6.112061in}{3.266313in}}%
\pgfpathlineto{\pgfqpoint{6.097425in}{3.257863in}}%
\pgfpathlineto{\pgfqpoint{6.082805in}{3.249511in}}%
\pgfpathlineto{\pgfqpoint{6.068202in}{3.241259in}}%
\pgfpathlineto{\pgfqpoint{6.075246in}{3.245438in}}%
\pgfpathlineto{\pgfqpoint{6.082280in}{3.249541in}}%
\pgfpathlineto{\pgfqpoint{6.089306in}{3.253571in}}%
\pgfpathlineto{\pgfqpoint{6.096324in}{3.257529in}}%
\pgfpathclose%
\pgfusepath{fill}%
\end{pgfscope}%
\begin{pgfscope}%
\pgfpathrectangle{\pgfqpoint{1.150000in}{0.150000in}}{\pgfqpoint{5.700000in}{5.700000in}}%
\pgfusepath{clip}%
\pgfsetbuttcap%
\pgfsetroundjoin%
\definecolor{currentfill}{rgb}{0.223925,0.334994,0.548053}%
\pgfsetfillcolor{currentfill}%
\pgfsetfillopacity{0.700000}%
\pgfsetlinewidth{0.000000pt}%
\definecolor{currentstroke}{rgb}{0.000000,0.000000,0.000000}%
\pgfsetstrokecolor{currentstroke}%
\pgfsetdash{}{0pt}%
\pgfpathmoveto{\pgfqpoint{5.202116in}{2.701841in}}%
\pgfpathlineto{\pgfqpoint{5.216250in}{2.707104in}}%
\pgfpathlineto{\pgfqpoint{5.230398in}{2.712467in}}%
\pgfpathlineto{\pgfqpoint{5.244559in}{2.717929in}}%
\pgfpathlineto{\pgfqpoint{5.258734in}{2.723492in}}%
\pgfpathlineto{\pgfqpoint{5.251235in}{2.714737in}}%
\pgfpathlineto{\pgfqpoint{5.243729in}{2.705893in}}%
\pgfpathlineto{\pgfqpoint{5.236217in}{2.696960in}}%
\pgfpathlineto{\pgfqpoint{5.228697in}{2.687938in}}%
\pgfpathlineto{\pgfqpoint{5.214515in}{2.682383in}}%
\pgfpathlineto{\pgfqpoint{5.200346in}{2.676928in}}%
\pgfpathlineto{\pgfqpoint{5.186190in}{2.671573in}}%
\pgfpathlineto{\pgfqpoint{5.172048in}{2.666318in}}%
\pgfpathlineto{\pgfqpoint{5.179575in}{2.675326in}}%
\pgfpathlineto{\pgfqpoint{5.187096in}{2.684248in}}%
\pgfpathlineto{\pgfqpoint{5.194609in}{2.693087in}}%
\pgfpathlineto{\pgfqpoint{5.202116in}{2.701841in}}%
\pgfpathclose%
\pgfusepath{fill}%
\end{pgfscope}%
\begin{pgfscope}%
\pgfpathrectangle{\pgfqpoint{1.150000in}{0.150000in}}{\pgfqpoint{5.700000in}{5.700000in}}%
\pgfusepath{clip}%
\pgfsetbuttcap%
\pgfsetroundjoin%
\definecolor{currentfill}{rgb}{0.283197,0.115680,0.436115}%
\pgfsetfillcolor{currentfill}%
\pgfsetfillopacity{0.700000}%
\pgfsetlinewidth{0.000000pt}%
\definecolor{currentstroke}{rgb}{0.000000,0.000000,0.000000}%
\pgfsetstrokecolor{currentstroke}%
\pgfsetdash{}{0pt}%
\pgfpathmoveto{\pgfqpoint{3.092160in}{2.240961in}}%
\pgfpathlineto{\pgfqpoint{3.105749in}{2.230488in}}%
\pgfpathlineto{\pgfqpoint{3.119338in}{2.220154in}}%
\pgfpathlineto{\pgfqpoint{3.132927in}{2.209959in}}%
\pgfpathlineto{\pgfqpoint{3.146517in}{2.199901in}}%
\pgfpathlineto{\pgfqpoint{3.138170in}{2.197159in}}%
\pgfpathlineto{\pgfqpoint{3.129813in}{2.194584in}}%
\pgfpathlineto{\pgfqpoint{3.121444in}{2.192181in}}%
\pgfpathlineto{\pgfqpoint{3.113065in}{2.189954in}}%
\pgfpathlineto{\pgfqpoint{3.099446in}{2.200392in}}%
\pgfpathlineto{\pgfqpoint{3.085828in}{2.210968in}}%
\pgfpathlineto{\pgfqpoint{3.072209in}{2.221682in}}%
\pgfpathlineto{\pgfqpoint{3.058591in}{2.232536in}}%
\pgfpathlineto{\pgfqpoint{3.067001in}{2.234376in}}%
\pgfpathlineto{\pgfqpoint{3.075399in}{2.236396in}}%
\pgfpathlineto{\pgfqpoint{3.083785in}{2.238592in}}%
\pgfpathlineto{\pgfqpoint{3.092160in}{2.240961in}}%
\pgfpathclose%
\pgfusepath{fill}%
\end{pgfscope}%
\begin{pgfscope}%
\pgfpathrectangle{\pgfqpoint{1.150000in}{0.150000in}}{\pgfqpoint{5.700000in}{5.700000in}}%
\pgfusepath{clip}%
\pgfsetbuttcap%
\pgfsetroundjoin%
\definecolor{currentfill}{rgb}{0.212395,0.359683,0.551710}%
\pgfsetfillcolor{currentfill}%
\pgfsetfillopacity{0.700000}%
\pgfsetlinewidth{0.000000pt}%
\definecolor{currentstroke}{rgb}{0.000000,0.000000,0.000000}%
\pgfsetstrokecolor{currentstroke}%
\pgfsetdash{}{0pt}%
\pgfpathmoveto{\pgfqpoint{5.288656in}{2.757632in}}%
\pgfpathlineto{\pgfqpoint{5.302836in}{2.763283in}}%
\pgfpathlineto{\pgfqpoint{5.317030in}{2.769034in}}%
\pgfpathlineto{\pgfqpoint{5.331237in}{2.774885in}}%
\pgfpathlineto{\pgfqpoint{5.345458in}{2.780836in}}%
\pgfpathlineto{\pgfqpoint{5.337997in}{2.772450in}}%
\pgfpathlineto{\pgfqpoint{5.330528in}{2.763973in}}%
\pgfpathlineto{\pgfqpoint{5.323052in}{2.755404in}}%
\pgfpathlineto{\pgfqpoint{5.315569in}{2.746741in}}%
\pgfpathlineto{\pgfqpoint{5.301339in}{2.740779in}}%
\pgfpathlineto{\pgfqpoint{5.287124in}{2.734917in}}%
\pgfpathlineto{\pgfqpoint{5.272922in}{2.729154in}}%
\pgfpathlineto{\pgfqpoint{5.258734in}{2.723492in}}%
\pgfpathlineto{\pgfqpoint{5.266225in}{2.732159in}}%
\pgfpathlineto{\pgfqpoint{5.273709in}{2.740737in}}%
\pgfpathlineto{\pgfqpoint{5.281186in}{2.749228in}}%
\pgfpathlineto{\pgfqpoint{5.288656in}{2.757632in}}%
\pgfpathclose%
\pgfusepath{fill}%
\end{pgfscope}%
\begin{pgfscope}%
\pgfpathrectangle{\pgfqpoint{1.150000in}{0.150000in}}{\pgfqpoint{5.700000in}{5.700000in}}%
\pgfusepath{clip}%
\pgfsetbuttcap%
\pgfsetroundjoin%
\definecolor{currentfill}{rgb}{0.283187,0.125848,0.444960}%
\pgfsetfillcolor{currentfill}%
\pgfsetfillopacity{0.700000}%
\pgfsetlinewidth{0.000000pt}%
\definecolor{currentstroke}{rgb}{0.000000,0.000000,0.000000}%
\pgfsetstrokecolor{currentstroke}%
\pgfsetdash{}{0pt}%
\pgfpathmoveto{\pgfqpoint{4.480098in}{2.243405in}}%
\pgfpathlineto{\pgfqpoint{4.493921in}{2.244687in}}%
\pgfpathlineto{\pgfqpoint{4.507754in}{2.246071in}}%
\pgfpathlineto{\pgfqpoint{4.521596in}{2.247559in}}%
\pgfpathlineto{\pgfqpoint{4.535448in}{2.249149in}}%
\pgfpathlineto{\pgfqpoint{4.527682in}{2.238736in}}%
\pgfpathlineto{\pgfqpoint{4.519910in}{2.228294in}}%
\pgfpathlineto{\pgfqpoint{4.512132in}{2.217822in}}%
\pgfpathlineto{\pgfqpoint{4.504350in}{2.207323in}}%
\pgfpathlineto{\pgfqpoint{4.490491in}{2.205907in}}%
\pgfpathlineto{\pgfqpoint{4.476642in}{2.204594in}}%
\pgfpathlineto{\pgfqpoint{4.462803in}{2.203384in}}%
\pgfpathlineto{\pgfqpoint{4.448974in}{2.202276in}}%
\pgfpathlineto{\pgfqpoint{4.456763in}{2.212594in}}%
\pgfpathlineto{\pgfqpoint{4.464546in}{2.222889in}}%
\pgfpathlineto{\pgfqpoint{4.472325in}{2.233160in}}%
\pgfpathlineto{\pgfqpoint{4.480098in}{2.243405in}}%
\pgfpathclose%
\pgfusepath{fill}%
\end{pgfscope}%
\begin{pgfscope}%
\pgfpathrectangle{\pgfqpoint{1.150000in}{0.150000in}}{\pgfqpoint{5.700000in}{5.700000in}}%
\pgfusepath{clip}%
\pgfsetbuttcap%
\pgfsetroundjoin%
\definecolor{currentfill}{rgb}{0.282656,0.100196,0.422160}%
\pgfsetfillcolor{currentfill}%
\pgfsetfillopacity{0.700000}%
\pgfsetlinewidth{0.000000pt}%
\definecolor{currentstroke}{rgb}{0.000000,0.000000,0.000000}%
\pgfsetstrokecolor{currentstroke}%
\pgfsetdash{}{0pt}%
\pgfpathmoveto{\pgfqpoint{4.393751in}{2.198877in}}%
\pgfpathlineto{\pgfqpoint{4.407543in}{2.199572in}}%
\pgfpathlineto{\pgfqpoint{4.421344in}{2.200370in}}%
\pgfpathlineto{\pgfqpoint{4.435154in}{2.201272in}}%
\pgfpathlineto{\pgfqpoint{4.448974in}{2.202276in}}%
\pgfpathlineto{\pgfqpoint{4.441180in}{2.191937in}}%
\pgfpathlineto{\pgfqpoint{4.433380in}{2.181578in}}%
\pgfpathlineto{\pgfqpoint{4.425576in}{2.171202in}}%
\pgfpathlineto{\pgfqpoint{4.417766in}{2.160809in}}%
\pgfpathlineto{\pgfqpoint{4.403940in}{2.159996in}}%
\pgfpathlineto{\pgfqpoint{4.390122in}{2.159287in}}%
\pgfpathlineto{\pgfqpoint{4.376314in}{2.158681in}}%
\pgfpathlineto{\pgfqpoint{4.362516in}{2.158179in}}%
\pgfpathlineto{\pgfqpoint{4.370332in}{2.168372in}}%
\pgfpathlineto{\pgfqpoint{4.378144in}{2.178555in}}%
\pgfpathlineto{\pgfqpoint{4.385950in}{2.188723in}}%
\pgfpathlineto{\pgfqpoint{4.393751in}{2.198877in}}%
\pgfpathclose%
\pgfusepath{fill}%
\end{pgfscope}%
\begin{pgfscope}%
\pgfpathrectangle{\pgfqpoint{1.150000in}{0.150000in}}{\pgfqpoint{5.700000in}{5.700000in}}%
\pgfusepath{clip}%
\pgfsetbuttcap%
\pgfsetroundjoin%
\definecolor{currentfill}{rgb}{0.282290,0.145912,0.461510}%
\pgfsetfillcolor{currentfill}%
\pgfsetfillopacity{0.700000}%
\pgfsetlinewidth{0.000000pt}%
\definecolor{currentstroke}{rgb}{0.000000,0.000000,0.000000}%
\pgfsetstrokecolor{currentstroke}%
\pgfsetdash{}{0pt}%
\pgfpathmoveto{\pgfqpoint{4.566463in}{2.290467in}}%
\pgfpathlineto{\pgfqpoint{4.580319in}{2.292316in}}%
\pgfpathlineto{\pgfqpoint{4.594186in}{2.294267in}}%
\pgfpathlineto{\pgfqpoint{4.608063in}{2.296319in}}%
\pgfpathlineto{\pgfqpoint{4.621950in}{2.298474in}}%
\pgfpathlineto{\pgfqpoint{4.614211in}{2.288047in}}%
\pgfpathlineto{\pgfqpoint{4.606466in}{2.277580in}}%
\pgfpathlineto{\pgfqpoint{4.598716in}{2.267074in}}%
\pgfpathlineto{\pgfqpoint{4.590960in}{2.256530in}}%
\pgfpathlineto{\pgfqpoint{4.577067in}{2.254532in}}%
\pgfpathlineto{\pgfqpoint{4.563184in}{2.252635in}}%
\pgfpathlineto{\pgfqpoint{4.549311in}{2.250841in}}%
\pgfpathlineto{\pgfqpoint{4.535448in}{2.249149in}}%
\pgfpathlineto{\pgfqpoint{4.543210in}{2.259529in}}%
\pgfpathlineto{\pgfqpoint{4.550966in}{2.269877in}}%
\pgfpathlineto{\pgfqpoint{4.558717in}{2.280190in}}%
\pgfpathlineto{\pgfqpoint{4.566463in}{2.290467in}}%
\pgfpathclose%
\pgfusepath{fill}%
\end{pgfscope}%
\begin{pgfscope}%
\pgfpathrectangle{\pgfqpoint{1.150000in}{0.150000in}}{\pgfqpoint{5.700000in}{5.700000in}}%
\pgfusepath{clip}%
\pgfsetbuttcap%
\pgfsetroundjoin%
\definecolor{currentfill}{rgb}{0.276022,0.044167,0.370164}%
\pgfsetfillcolor{currentfill}%
\pgfsetfillopacity{0.700000}%
\pgfsetlinewidth{0.000000pt}%
\definecolor{currentstroke}{rgb}{0.000000,0.000000,0.000000}%
\pgfsetstrokecolor{currentstroke}%
\pgfsetdash{}{0pt}%
\pgfpathmoveto{\pgfqpoint{3.342643in}{2.106497in}}%
\pgfpathlineto{\pgfqpoint{3.356225in}{2.098507in}}%
\pgfpathlineto{\pgfqpoint{3.369809in}{2.090644in}}%
\pgfpathlineto{\pgfqpoint{3.383396in}{2.082906in}}%
\pgfpathlineto{\pgfqpoint{3.396986in}{2.075293in}}%
\pgfpathlineto{\pgfqpoint{3.388780in}{2.070509in}}%
\pgfpathlineto{\pgfqpoint{3.380565in}{2.065858in}}%
\pgfpathlineto{\pgfqpoint{3.372341in}{2.061346in}}%
\pgfpathlineto{\pgfqpoint{3.364108in}{2.056975in}}%
\pgfpathlineto{\pgfqpoint{3.350496in}{2.064946in}}%
\pgfpathlineto{\pgfqpoint{3.336885in}{2.073041in}}%
\pgfpathlineto{\pgfqpoint{3.323277in}{2.081262in}}%
\pgfpathlineto{\pgfqpoint{3.309671in}{2.089610in}}%
\pgfpathlineto{\pgfqpoint{3.317928in}{2.093616in}}%
\pgfpathlineto{\pgfqpoint{3.326176in}{2.097768in}}%
\pgfpathlineto{\pgfqpoint{3.334414in}{2.102063in}}%
\pgfpathlineto{\pgfqpoint{3.342643in}{2.106497in}}%
\pgfpathclose%
\pgfusepath{fill}%
\end{pgfscope}%
\begin{pgfscope}%
\pgfpathrectangle{\pgfqpoint{1.150000in}{0.150000in}}{\pgfqpoint{5.700000in}{5.700000in}}%
\pgfusepath{clip}%
\pgfsetbuttcap%
\pgfsetroundjoin%
\definecolor{currentfill}{rgb}{0.268510,0.009605,0.335427}%
\pgfsetfillcolor{currentfill}%
\pgfsetfillopacity{0.700000}%
\pgfsetlinewidth{0.000000pt}%
\definecolor{currentstroke}{rgb}{0.000000,0.000000,0.000000}%
\pgfsetstrokecolor{currentstroke}%
\pgfsetdash{}{0pt}%
\pgfpathmoveto{\pgfqpoint{3.907177in}{2.037355in}}%
\pgfpathlineto{\pgfqpoint{3.920828in}{2.034358in}}%
\pgfpathlineto{\pgfqpoint{3.934486in}{2.031470in}}%
\pgfpathlineto{\pgfqpoint{3.948150in}{2.028692in}}%
\pgfpathlineto{\pgfqpoint{3.961820in}{2.026022in}}%
\pgfpathlineto{\pgfqpoint{3.953865in}{2.017313in}}%
\pgfpathlineto{\pgfqpoint{3.945903in}{2.008653in}}%
\pgfpathlineto{\pgfqpoint{3.937936in}{2.000046in}}%
\pgfpathlineto{\pgfqpoint{3.929963in}{1.991493in}}%
\pgfpathlineto{\pgfqpoint{3.916280in}{1.994444in}}%
\pgfpathlineto{\pgfqpoint{3.902604in}{1.997505in}}%
\pgfpathlineto{\pgfqpoint{3.888933in}{2.000674in}}%
\pgfpathlineto{\pgfqpoint{3.875269in}{2.003953in}}%
\pgfpathlineto{\pgfqpoint{3.883255in}{2.012217in}}%
\pgfpathlineto{\pgfqpoint{3.891235in}{2.020540in}}%
\pgfpathlineto{\pgfqpoint{3.899209in}{2.028921in}}%
\pgfpathlineto{\pgfqpoint{3.907177in}{2.037355in}}%
\pgfpathclose%
\pgfusepath{fill}%
\end{pgfscope}%
\begin{pgfscope}%
\pgfpathrectangle{\pgfqpoint{1.150000in}{0.150000in}}{\pgfqpoint{5.700000in}{5.700000in}}%
\pgfusepath{clip}%
\pgfsetbuttcap%
\pgfsetroundjoin%
\definecolor{currentfill}{rgb}{0.280894,0.078907,0.402329}%
\pgfsetfillcolor{currentfill}%
\pgfsetfillopacity{0.700000}%
\pgfsetlinewidth{0.000000pt}%
\definecolor{currentstroke}{rgb}{0.000000,0.000000,0.000000}%
\pgfsetstrokecolor{currentstroke}%
\pgfsetdash{}{0pt}%
\pgfpathmoveto{\pgfqpoint{4.307411in}{2.157206in}}%
\pgfpathlineto{\pgfqpoint{4.321174in}{2.157293in}}%
\pgfpathlineto{\pgfqpoint{4.334945in}{2.157485in}}%
\pgfpathlineto{\pgfqpoint{4.348726in}{2.157780in}}%
\pgfpathlineto{\pgfqpoint{4.362516in}{2.158179in}}%
\pgfpathlineto{\pgfqpoint{4.354694in}{2.147975in}}%
\pgfpathlineto{\pgfqpoint{4.346867in}{2.137763in}}%
\pgfpathlineto{\pgfqpoint{4.339035in}{2.127545in}}%
\pgfpathlineto{\pgfqpoint{4.331198in}{2.117322in}}%
\pgfpathlineto{\pgfqpoint{4.317400in}{2.117133in}}%
\pgfpathlineto{\pgfqpoint{4.303612in}{2.117048in}}%
\pgfpathlineto{\pgfqpoint{4.289833in}{2.117067in}}%
\pgfpathlineto{\pgfqpoint{4.276062in}{2.117191in}}%
\pgfpathlineto{\pgfqpoint{4.283907in}{2.127196in}}%
\pgfpathlineto{\pgfqpoint{4.291747in}{2.137202in}}%
\pgfpathlineto{\pgfqpoint{4.299582in}{2.147206in}}%
\pgfpathlineto{\pgfqpoint{4.307411in}{2.157206in}}%
\pgfpathclose%
\pgfusepath{fill}%
\end{pgfscope}%
\begin{pgfscope}%
\pgfpathrectangle{\pgfqpoint{1.150000in}{0.150000in}}{\pgfqpoint{5.700000in}{5.700000in}}%
\pgfusepath{clip}%
\pgfsetbuttcap%
\pgfsetroundjoin%
\definecolor{currentfill}{rgb}{0.279574,0.170599,0.479997}%
\pgfsetfillcolor{currentfill}%
\pgfsetfillopacity{0.700000}%
\pgfsetlinewidth{0.000000pt}%
\definecolor{currentstroke}{rgb}{0.000000,0.000000,0.000000}%
\pgfsetstrokecolor{currentstroke}%
\pgfsetdash{}{0pt}%
\pgfpathmoveto{\pgfqpoint{4.652855in}{2.339754in}}%
\pgfpathlineto{\pgfqpoint{4.666747in}{2.342149in}}%
\pgfpathlineto{\pgfqpoint{4.680650in}{2.344645in}}%
\pgfpathlineto{\pgfqpoint{4.694563in}{2.347243in}}%
\pgfpathlineto{\pgfqpoint{4.708488in}{2.349943in}}%
\pgfpathlineto{\pgfqpoint{4.700776in}{2.339558in}}%
\pgfpathlineto{\pgfqpoint{4.693058in}{2.329124in}}%
\pgfpathlineto{\pgfqpoint{4.685335in}{2.318641in}}%
\pgfpathlineto{\pgfqpoint{4.677607in}{2.308111in}}%
\pgfpathlineto{\pgfqpoint{4.663677in}{2.305549in}}%
\pgfpathlineto{\pgfqpoint{4.649757in}{2.303089in}}%
\pgfpathlineto{\pgfqpoint{4.635849in}{2.300731in}}%
\pgfpathlineto{\pgfqpoint{4.621950in}{2.298474in}}%
\pgfpathlineto{\pgfqpoint{4.629685in}{2.308859in}}%
\pgfpathlineto{\pgfqpoint{4.637414in}{2.319202in}}%
\pgfpathlineto{\pgfqpoint{4.645137in}{2.329500in}}%
\pgfpathlineto{\pgfqpoint{4.652855in}{2.339754in}}%
\pgfpathclose%
\pgfusepath{fill}%
\end{pgfscope}%
\begin{pgfscope}%
\pgfpathrectangle{\pgfqpoint{1.150000in}{0.150000in}}{\pgfqpoint{5.700000in}{5.700000in}}%
\pgfusepath{clip}%
\pgfsetbuttcap%
\pgfsetroundjoin%
\definecolor{currentfill}{rgb}{0.201239,0.383670,0.554294}%
\pgfsetfillcolor{currentfill}%
\pgfsetfillopacity{0.700000}%
\pgfsetlinewidth{0.000000pt}%
\definecolor{currentstroke}{rgb}{0.000000,0.000000,0.000000}%
\pgfsetstrokecolor{currentstroke}%
\pgfsetdash{}{0pt}%
\pgfpathmoveto{\pgfqpoint{5.375229in}{2.813468in}}%
\pgfpathlineto{\pgfqpoint{5.389455in}{2.819488in}}%
\pgfpathlineto{\pgfqpoint{5.403694in}{2.825608in}}%
\pgfpathlineto{\pgfqpoint{5.417949in}{2.831827in}}%
\pgfpathlineto{\pgfqpoint{5.432217in}{2.838146in}}%
\pgfpathlineto{\pgfqpoint{5.424795in}{2.830161in}}%
\pgfpathlineto{\pgfqpoint{5.417365in}{2.822081in}}%
\pgfpathlineto{\pgfqpoint{5.409928in}{2.813907in}}%
\pgfpathlineto{\pgfqpoint{5.402483in}{2.805637in}}%
\pgfpathlineto{\pgfqpoint{5.388206in}{2.799287in}}%
\pgfpathlineto{\pgfqpoint{5.373942in}{2.793037in}}%
\pgfpathlineto{\pgfqpoint{5.359693in}{2.786886in}}%
\pgfpathlineto{\pgfqpoint{5.345458in}{2.780836in}}%
\pgfpathlineto{\pgfqpoint{5.352912in}{2.789130in}}%
\pgfpathlineto{\pgfqpoint{5.360358in}{2.797332in}}%
\pgfpathlineto{\pgfqpoint{5.367797in}{2.805445in}}%
\pgfpathlineto{\pgfqpoint{5.375229in}{2.813468in}}%
\pgfpathclose%
\pgfusepath{fill}%
\end{pgfscope}%
\begin{pgfscope}%
\pgfpathrectangle{\pgfqpoint{1.150000in}{0.150000in}}{\pgfqpoint{5.700000in}{5.700000in}}%
\pgfusepath{clip}%
\pgfsetbuttcap%
\pgfsetroundjoin%
\definecolor{currentfill}{rgb}{0.277941,0.056324,0.381191}%
\pgfsetfillcolor{currentfill}%
\pgfsetfillopacity{0.700000}%
\pgfsetlinewidth{0.000000pt}%
\definecolor{currentstroke}{rgb}{0.000000,0.000000,0.000000}%
\pgfsetstrokecolor{currentstroke}%
\pgfsetdash{}{0pt}%
\pgfpathmoveto{\pgfqpoint{4.221064in}{2.118730in}}%
\pgfpathlineto{\pgfqpoint{4.234801in}{2.118187in}}%
\pgfpathlineto{\pgfqpoint{4.248546in}{2.117750in}}%
\pgfpathlineto{\pgfqpoint{4.262300in}{2.117418in}}%
\pgfpathlineto{\pgfqpoint{4.276062in}{2.117191in}}%
\pgfpathlineto{\pgfqpoint{4.268212in}{2.107187in}}%
\pgfpathlineto{\pgfqpoint{4.260357in}{2.097187in}}%
\pgfpathlineto{\pgfqpoint{4.252496in}{2.087194in}}%
\pgfpathlineto{\pgfqpoint{4.244630in}{2.077208in}}%
\pgfpathlineto{\pgfqpoint{4.230860in}{2.077664in}}%
\pgfpathlineto{\pgfqpoint{4.217098in}{2.078224in}}%
\pgfpathlineto{\pgfqpoint{4.203344in}{2.078889in}}%
\pgfpathlineto{\pgfqpoint{4.189598in}{2.079660in}}%
\pgfpathlineto{\pgfqpoint{4.197473in}{2.089410in}}%
\pgfpathlineto{\pgfqpoint{4.205342in}{2.099173in}}%
\pgfpathlineto{\pgfqpoint{4.213206in}{2.108947in}}%
\pgfpathlineto{\pgfqpoint{4.221064in}{2.118730in}}%
\pgfpathclose%
\pgfusepath{fill}%
\end{pgfscope}%
\begin{pgfscope}%
\pgfpathrectangle{\pgfqpoint{1.150000in}{0.150000in}}{\pgfqpoint{5.700000in}{5.700000in}}%
\pgfusepath{clip}%
\pgfsetbuttcap%
\pgfsetroundjoin%
\definecolor{currentfill}{rgb}{0.274128,0.199721,0.498911}%
\pgfsetfillcolor{currentfill}%
\pgfsetfillopacity{0.700000}%
\pgfsetlinewidth{0.000000pt}%
\definecolor{currentstroke}{rgb}{0.000000,0.000000,0.000000}%
\pgfsetstrokecolor{currentstroke}%
\pgfsetdash{}{0pt}%
\pgfpathmoveto{\pgfqpoint{4.739281in}{2.390966in}}%
\pgfpathlineto{\pgfqpoint{4.753210in}{2.393887in}}%
\pgfpathlineto{\pgfqpoint{4.767151in}{2.396910in}}%
\pgfpathlineto{\pgfqpoint{4.781104in}{2.400033in}}%
\pgfpathlineto{\pgfqpoint{4.795067in}{2.403258in}}%
\pgfpathlineto{\pgfqpoint{4.787383in}{2.392968in}}%
\pgfpathlineto{\pgfqpoint{4.779694in}{2.382621in}}%
\pgfpathlineto{\pgfqpoint{4.771998in}{2.372216in}}%
\pgfpathlineto{\pgfqpoint{4.764297in}{2.361755in}}%
\pgfpathlineto{\pgfqpoint{4.750328in}{2.358650in}}%
\pgfpathlineto{\pgfqpoint{4.736370in}{2.355647in}}%
\pgfpathlineto{\pgfqpoint{4.722424in}{2.352744in}}%
\pgfpathlineto{\pgfqpoint{4.708488in}{2.349943in}}%
\pgfpathlineto{\pgfqpoint{4.716194in}{2.360277in}}%
\pgfpathlineto{\pgfqpoint{4.723895in}{2.370560in}}%
\pgfpathlineto{\pgfqpoint{4.731591in}{2.380790in}}%
\pgfpathlineto{\pgfqpoint{4.739281in}{2.390966in}}%
\pgfpathclose%
\pgfusepath{fill}%
\end{pgfscope}%
\begin{pgfscope}%
\pgfpathrectangle{\pgfqpoint{1.150000in}{0.150000in}}{\pgfqpoint{5.700000in}{5.700000in}}%
\pgfusepath{clip}%
\pgfsetbuttcap%
\pgfsetroundjoin%
\definecolor{currentfill}{rgb}{0.233603,0.313828,0.543914}%
\pgfsetfillcolor{currentfill}%
\pgfsetfillopacity{0.700000}%
\pgfsetlinewidth{0.000000pt}%
\definecolor{currentstroke}{rgb}{0.000000,0.000000,0.000000}%
\pgfsetstrokecolor{currentstroke}%
\pgfsetdash{}{0pt}%
\pgfpathmoveto{\pgfqpoint{2.621934in}{2.659815in}}%
\pgfpathlineto{\pgfqpoint{2.635630in}{2.643913in}}%
\pgfpathlineto{\pgfqpoint{2.649321in}{2.628189in}}%
\pgfpathlineto{\pgfqpoint{2.663007in}{2.612643in}}%
\pgfpathlineto{\pgfqpoint{2.676689in}{2.597271in}}%
\pgfpathlineto{\pgfqpoint{2.668023in}{2.598412in}}%
\pgfpathlineto{\pgfqpoint{2.659341in}{2.599777in}}%
\pgfpathlineto{\pgfqpoint{2.650645in}{2.601369in}}%
\pgfpathlineto{\pgfqpoint{2.641932in}{2.603193in}}%
\pgfpathlineto{\pgfqpoint{2.628210in}{2.618979in}}%
\pgfpathlineto{\pgfqpoint{2.614484in}{2.634942in}}%
\pgfpathlineto{\pgfqpoint{2.600752in}{2.651081in}}%
\pgfpathlineto{\pgfqpoint{2.587016in}{2.667400in}}%
\pgfpathlineto{\pgfqpoint{2.595770in}{2.665152in}}%
\pgfpathlineto{\pgfqpoint{2.604507in}{2.663142in}}%
\pgfpathlineto{\pgfqpoint{2.613229in}{2.661364in}}%
\pgfpathlineto{\pgfqpoint{2.621934in}{2.659815in}}%
\pgfpathclose%
\pgfusepath{fill}%
\end{pgfscope}%
\begin{pgfscope}%
\pgfpathrectangle{\pgfqpoint{1.150000in}{0.150000in}}{\pgfqpoint{5.700000in}{5.700000in}}%
\pgfusepath{clip}%
\pgfsetbuttcap%
\pgfsetroundjoin%
\definecolor{currentfill}{rgb}{0.244972,0.287675,0.537260}%
\pgfsetfillcolor{currentfill}%
\pgfsetfillopacity{0.700000}%
\pgfsetlinewidth{0.000000pt}%
\definecolor{currentstroke}{rgb}{0.000000,0.000000,0.000000}%
\pgfsetstrokecolor{currentstroke}%
\pgfsetdash{}{0pt}%
\pgfpathmoveto{\pgfqpoint{2.676689in}{2.597271in}}%
\pgfpathlineto{\pgfqpoint{2.690366in}{2.582074in}}%
\pgfpathlineto{\pgfqpoint{2.704039in}{2.567048in}}%
\pgfpathlineto{\pgfqpoint{2.717708in}{2.552194in}}%
\pgfpathlineto{\pgfqpoint{2.731373in}{2.537509in}}%
\pgfpathlineto{\pgfqpoint{2.722746in}{2.538244in}}%
\pgfpathlineto{\pgfqpoint{2.714104in}{2.539198in}}%
\pgfpathlineto{\pgfqpoint{2.705447in}{2.540375in}}%
\pgfpathlineto{\pgfqpoint{2.696774in}{2.541778in}}%
\pgfpathlineto{\pgfqpoint{2.683070in}{2.556875in}}%
\pgfpathlineto{\pgfqpoint{2.669362in}{2.572143in}}%
\pgfpathlineto{\pgfqpoint{2.655649in}{2.587581in}}%
\pgfpathlineto{\pgfqpoint{2.641932in}{2.603193in}}%
\pgfpathlineto{\pgfqpoint{2.650645in}{2.601369in}}%
\pgfpathlineto{\pgfqpoint{2.659341in}{2.599777in}}%
\pgfpathlineto{\pgfqpoint{2.668023in}{2.598412in}}%
\pgfpathlineto{\pgfqpoint{2.676689in}{2.597271in}}%
\pgfpathclose%
\pgfusepath{fill}%
\end{pgfscope}%
\begin{pgfscope}%
\pgfpathrectangle{\pgfqpoint{1.150000in}{0.150000in}}{\pgfqpoint{5.700000in}{5.700000in}}%
\pgfusepath{clip}%
\pgfsetbuttcap%
\pgfsetroundjoin%
\definecolor{currentfill}{rgb}{0.221989,0.339161,0.548752}%
\pgfsetfillcolor{currentfill}%
\pgfsetfillopacity{0.700000}%
\pgfsetlinewidth{0.000000pt}%
\definecolor{currentstroke}{rgb}{0.000000,0.000000,0.000000}%
\pgfsetstrokecolor{currentstroke}%
\pgfsetdash{}{0pt}%
\pgfpathmoveto{\pgfqpoint{2.567100in}{2.725239in}}%
\pgfpathlineto{\pgfqpoint{2.580816in}{2.708607in}}%
\pgfpathlineto{\pgfqpoint{2.594528in}{2.692161in}}%
\pgfpathlineto{\pgfqpoint{2.608234in}{2.675897in}}%
\pgfpathlineto{\pgfqpoint{2.621934in}{2.659815in}}%
\pgfpathlineto{\pgfqpoint{2.613229in}{2.661364in}}%
\pgfpathlineto{\pgfqpoint{2.604507in}{2.663142in}}%
\pgfpathlineto{\pgfqpoint{2.595770in}{2.665152in}}%
\pgfpathlineto{\pgfqpoint{2.587016in}{2.667400in}}%
\pgfpathlineto{\pgfqpoint{2.573274in}{2.683899in}}%
\pgfpathlineto{\pgfqpoint{2.559526in}{2.700581in}}%
\pgfpathlineto{\pgfqpoint{2.545773in}{2.717446in}}%
\pgfpathlineto{\pgfqpoint{2.532013in}{2.734498in}}%
\pgfpathlineto{\pgfqpoint{2.540810in}{2.731823in}}%
\pgfpathlineto{\pgfqpoint{2.549590in}{2.729392in}}%
\pgfpathlineto{\pgfqpoint{2.558353in}{2.727198in}}%
\pgfpathlineto{\pgfqpoint{2.567100in}{2.725239in}}%
\pgfpathclose%
\pgfusepath{fill}%
\end{pgfscope}%
\begin{pgfscope}%
\pgfpathrectangle{\pgfqpoint{1.150000in}{0.150000in}}{\pgfqpoint{5.700000in}{5.700000in}}%
\pgfusepath{clip}%
\pgfsetbuttcap%
\pgfsetroundjoin%
\definecolor{currentfill}{rgb}{0.282656,0.100196,0.422160}%
\pgfsetfillcolor{currentfill}%
\pgfsetfillopacity{0.700000}%
\pgfsetlinewidth{0.000000pt}%
\definecolor{currentstroke}{rgb}{0.000000,0.000000,0.000000}%
\pgfsetstrokecolor{currentstroke}%
\pgfsetdash{}{0pt}%
\pgfpathmoveto{\pgfqpoint{3.146517in}{2.199901in}}%
\pgfpathlineto{\pgfqpoint{3.160107in}{2.189979in}}%
\pgfpathlineto{\pgfqpoint{3.173698in}{2.180193in}}%
\pgfpathlineto{\pgfqpoint{3.187290in}{2.170542in}}%
\pgfpathlineto{\pgfqpoint{3.200883in}{2.161024in}}%
\pgfpathlineto{\pgfqpoint{3.192564in}{2.157910in}}%
\pgfpathlineto{\pgfqpoint{3.184234in}{2.154958in}}%
\pgfpathlineto{\pgfqpoint{3.175894in}{2.152173in}}%
\pgfpathlineto{\pgfqpoint{3.167543in}{2.149559in}}%
\pgfpathlineto{\pgfqpoint{3.153922in}{2.159455in}}%
\pgfpathlineto{\pgfqpoint{3.140303in}{2.169486in}}%
\pgfpathlineto{\pgfqpoint{3.126683in}{2.179652in}}%
\pgfpathlineto{\pgfqpoint{3.113065in}{2.189954in}}%
\pgfpathlineto{\pgfqpoint{3.121444in}{2.192181in}}%
\pgfpathlineto{\pgfqpoint{3.129813in}{2.194584in}}%
\pgfpathlineto{\pgfqpoint{3.138170in}{2.197159in}}%
\pgfpathlineto{\pgfqpoint{3.146517in}{2.199901in}}%
\pgfpathclose%
\pgfusepath{fill}%
\end{pgfscope}%
\begin{pgfscope}%
\pgfpathrectangle{\pgfqpoint{1.150000in}{0.150000in}}{\pgfqpoint{5.700000in}{5.700000in}}%
\pgfusepath{clip}%
\pgfsetbuttcap%
\pgfsetroundjoin%
\definecolor{currentfill}{rgb}{0.255645,0.260703,0.528312}%
\pgfsetfillcolor{currentfill}%
\pgfsetfillopacity{0.700000}%
\pgfsetlinewidth{0.000000pt}%
\definecolor{currentstroke}{rgb}{0.000000,0.000000,0.000000}%
\pgfsetstrokecolor{currentstroke}%
\pgfsetdash{}{0pt}%
\pgfpathmoveto{\pgfqpoint{2.731373in}{2.537509in}}%
\pgfpathlineto{\pgfqpoint{2.745035in}{2.522991in}}%
\pgfpathlineto{\pgfqpoint{2.758693in}{2.508641in}}%
\pgfpathlineto{\pgfqpoint{2.772348in}{2.494456in}}%
\pgfpathlineto{\pgfqpoint{2.785999in}{2.480434in}}%
\pgfpathlineto{\pgfqpoint{2.777409in}{2.480767in}}%
\pgfpathlineto{\pgfqpoint{2.768805in}{2.481312in}}%
\pgfpathlineto{\pgfqpoint{2.760186in}{2.482075in}}%
\pgfpathlineto{\pgfqpoint{2.751552in}{2.483059in}}%
\pgfpathlineto{\pgfqpoint{2.737863in}{2.497491in}}%
\pgfpathlineto{\pgfqpoint{2.724170in}{2.512087in}}%
\pgfpathlineto{\pgfqpoint{2.710474in}{2.526849in}}%
\pgfpathlineto{\pgfqpoint{2.696774in}{2.541778in}}%
\pgfpathlineto{\pgfqpoint{2.705447in}{2.540375in}}%
\pgfpathlineto{\pgfqpoint{2.714104in}{2.539198in}}%
\pgfpathlineto{\pgfqpoint{2.722746in}{2.538244in}}%
\pgfpathlineto{\pgfqpoint{2.731373in}{2.537509in}}%
\pgfpathclose%
\pgfusepath{fill}%
\end{pgfscope}%
\begin{pgfscope}%
\pgfpathrectangle{\pgfqpoint{1.150000in}{0.150000in}}{\pgfqpoint{5.700000in}{5.700000in}}%
\pgfusepath{clip}%
\pgfsetbuttcap%
\pgfsetroundjoin%
\definecolor{currentfill}{rgb}{0.267004,0.004874,0.329415}%
\pgfsetfillcolor{currentfill}%
\pgfsetfillopacity{0.700000}%
\pgfsetlinewidth{0.000000pt}%
\definecolor{currentstroke}{rgb}{0.000000,0.000000,0.000000}%
\pgfsetstrokecolor{currentstroke}%
\pgfsetdash{}{0pt}%
\pgfpathmoveto{\pgfqpoint{3.679561in}{2.023193in}}%
\pgfpathlineto{\pgfqpoint{3.693175in}{2.018267in}}%
\pgfpathlineto{\pgfqpoint{3.706793in}{2.013455in}}%
\pgfpathlineto{\pgfqpoint{3.720416in}{2.008758in}}%
\pgfpathlineto{\pgfqpoint{3.734045in}{2.004174in}}%
\pgfpathlineto{\pgfqpoint{3.725997in}{1.996872in}}%
\pgfpathlineto{\pgfqpoint{3.717943in}{1.989655in}}%
\pgfpathlineto{\pgfqpoint{3.709882in}{1.982528in}}%
\pgfpathlineto{\pgfqpoint{3.701813in}{1.975491in}}%
\pgfpathlineto{\pgfqpoint{3.688169in}{1.980393in}}%
\pgfpathlineto{\pgfqpoint{3.674529in}{1.985409in}}%
\pgfpathlineto{\pgfqpoint{3.660894in}{1.990539in}}%
\pgfpathlineto{\pgfqpoint{3.647263in}{1.995784in}}%
\pgfpathlineto{\pgfqpoint{3.655348in}{2.002495in}}%
\pgfpathlineto{\pgfqpoint{3.663427in}{2.009302in}}%
\pgfpathlineto{\pgfqpoint{3.671498in}{2.016202in}}%
\pgfpathlineto{\pgfqpoint{3.679561in}{2.023193in}}%
\pgfpathclose%
\pgfusepath{fill}%
\end{pgfscope}%
\begin{pgfscope}%
\pgfpathrectangle{\pgfqpoint{1.150000in}{0.150000in}}{\pgfqpoint{5.700000in}{5.700000in}}%
\pgfusepath{clip}%
\pgfsetbuttcap%
\pgfsetroundjoin%
\definecolor{currentfill}{rgb}{0.208623,0.367752,0.552675}%
\pgfsetfillcolor{currentfill}%
\pgfsetfillopacity{0.700000}%
\pgfsetlinewidth{0.000000pt}%
\definecolor{currentstroke}{rgb}{0.000000,0.000000,0.000000}%
\pgfsetstrokecolor{currentstroke}%
\pgfsetdash{}{0pt}%
\pgfpathmoveto{\pgfqpoint{2.512172in}{2.793650in}}%
\pgfpathlineto{\pgfqpoint{2.525913in}{2.776261in}}%
\pgfpathlineto{\pgfqpoint{2.539648in}{2.759065in}}%
\pgfpathlineto{\pgfqpoint{2.553377in}{2.742058in}}%
\pgfpathlineto{\pgfqpoint{2.567100in}{2.725239in}}%
\pgfpathlineto{\pgfqpoint{2.558353in}{2.727198in}}%
\pgfpathlineto{\pgfqpoint{2.549590in}{2.729392in}}%
\pgfpathlineto{\pgfqpoint{2.540810in}{2.731823in}}%
\pgfpathlineto{\pgfqpoint{2.532013in}{2.734498in}}%
\pgfpathlineto{\pgfqpoint{2.518248in}{2.751736in}}%
\pgfpathlineto{\pgfqpoint{2.504476in}{2.769164in}}%
\pgfpathlineto{\pgfqpoint{2.490698in}{2.786782in}}%
\pgfpathlineto{\pgfqpoint{2.476913in}{2.804594in}}%
\pgfpathlineto{\pgfqpoint{2.485754in}{2.801490in}}%
\pgfpathlineto{\pgfqpoint{2.494577in}{2.798635in}}%
\pgfpathlineto{\pgfqpoint{2.503383in}{2.796023in}}%
\pgfpathlineto{\pgfqpoint{2.512172in}{2.793650in}}%
\pgfpathclose%
\pgfusepath{fill}%
\end{pgfscope}%
\begin{pgfscope}%
\pgfpathrectangle{\pgfqpoint{1.150000in}{0.150000in}}{\pgfqpoint{5.700000in}{5.700000in}}%
\pgfusepath{clip}%
\pgfsetbuttcap%
\pgfsetroundjoin%
\definecolor{currentfill}{rgb}{0.269944,0.014625,0.341379}%
\pgfsetfillcolor{currentfill}%
\pgfsetfillopacity{0.700000}%
\pgfsetlinewidth{0.000000pt}%
\definecolor{currentstroke}{rgb}{0.000000,0.000000,0.000000}%
\pgfsetstrokecolor{currentstroke}%
\pgfsetdash{}{0pt}%
\pgfpathmoveto{\pgfqpoint{3.538373in}{2.041928in}}%
\pgfpathlineto{\pgfqpoint{3.551970in}{2.035748in}}%
\pgfpathlineto{\pgfqpoint{3.565571in}{2.029687in}}%
\pgfpathlineto{\pgfqpoint{3.579176in}{2.023744in}}%
\pgfpathlineto{\pgfqpoint{3.592785in}{2.017919in}}%
\pgfpathlineto{\pgfqpoint{3.584674in}{2.011637in}}%
\pgfpathlineto{\pgfqpoint{3.576556in}{2.005463in}}%
\pgfpathlineto{\pgfqpoint{3.568429in}{1.999399in}}%
\pgfpathlineto{\pgfqpoint{3.560295in}{1.993449in}}%
\pgfpathlineto{\pgfqpoint{3.546667in}{1.999612in}}%
\pgfpathlineto{\pgfqpoint{3.533043in}{2.005892in}}%
\pgfpathlineto{\pgfqpoint{3.519422in}{2.012291in}}%
\pgfpathlineto{\pgfqpoint{3.505805in}{2.018808in}}%
\pgfpathlineto{\pgfqpoint{3.513959in}{2.024414in}}%
\pgfpathlineto{\pgfqpoint{3.522105in}{2.030138in}}%
\pgfpathlineto{\pgfqpoint{3.530243in}{2.035977in}}%
\pgfpathlineto{\pgfqpoint{3.538373in}{2.041928in}}%
\pgfpathclose%
\pgfusepath{fill}%
\end{pgfscope}%
\begin{pgfscope}%
\pgfpathrectangle{\pgfqpoint{1.150000in}{0.150000in}}{\pgfqpoint{5.700000in}{5.700000in}}%
\pgfusepath{clip}%
\pgfsetbuttcap%
\pgfsetroundjoin%
\definecolor{currentfill}{rgb}{0.267968,0.223549,0.512008}%
\pgfsetfillcolor{currentfill}%
\pgfsetfillopacity{0.700000}%
\pgfsetlinewidth{0.000000pt}%
\definecolor{currentstroke}{rgb}{0.000000,0.000000,0.000000}%
\pgfsetstrokecolor{currentstroke}%
\pgfsetdash{}{0pt}%
\pgfpathmoveto{\pgfqpoint{4.825746in}{2.443818in}}%
\pgfpathlineto{\pgfqpoint{4.839715in}{2.447246in}}%
\pgfpathlineto{\pgfqpoint{4.853696in}{2.450774in}}%
\pgfpathlineto{\pgfqpoint{4.867689in}{2.454403in}}%
\pgfpathlineto{\pgfqpoint{4.881693in}{2.458133in}}%
\pgfpathlineto{\pgfqpoint{4.874038in}{2.447989in}}%
\pgfpathlineto{\pgfqpoint{4.866377in}{2.437780in}}%
\pgfpathlineto{\pgfqpoint{4.858710in}{2.427505in}}%
\pgfpathlineto{\pgfqpoint{4.851038in}{2.417166in}}%
\pgfpathlineto{\pgfqpoint{4.837028in}{2.413538in}}%
\pgfpathlineto{\pgfqpoint{4.823029in}{2.410010in}}%
\pgfpathlineto{\pgfqpoint{4.809042in}{2.406584in}}%
\pgfpathlineto{\pgfqpoint{4.795067in}{2.403258in}}%
\pgfpathlineto{\pgfqpoint{4.802746in}{2.413488in}}%
\pgfpathlineto{\pgfqpoint{4.810418in}{2.423659in}}%
\pgfpathlineto{\pgfqpoint{4.818085in}{2.433769in}}%
\pgfpathlineto{\pgfqpoint{4.825746in}{2.443818in}}%
\pgfpathclose%
\pgfusepath{fill}%
\end{pgfscope}%
\begin{pgfscope}%
\pgfpathrectangle{\pgfqpoint{1.150000in}{0.150000in}}{\pgfqpoint{5.700000in}{5.700000in}}%
\pgfusepath{clip}%
\pgfsetbuttcap%
\pgfsetroundjoin%
\definecolor{currentfill}{rgb}{0.190631,0.407061,0.556089}%
\pgfsetfillcolor{currentfill}%
\pgfsetfillopacity{0.700000}%
\pgfsetlinewidth{0.000000pt}%
\definecolor{currentstroke}{rgb}{0.000000,0.000000,0.000000}%
\pgfsetstrokecolor{currentstroke}%
\pgfsetdash{}{0pt}%
\pgfpathmoveto{\pgfqpoint{5.461828in}{2.869157in}}%
\pgfpathlineto{\pgfqpoint{5.476100in}{2.875526in}}%
\pgfpathlineto{\pgfqpoint{5.490387in}{2.881995in}}%
\pgfpathlineto{\pgfqpoint{5.504689in}{2.888563in}}%
\pgfpathlineto{\pgfqpoint{5.519005in}{2.895231in}}%
\pgfpathlineto{\pgfqpoint{5.511624in}{2.887673in}}%
\pgfpathlineto{\pgfqpoint{5.504236in}{2.880019in}}%
\pgfpathlineto{\pgfqpoint{5.496839in}{2.872268in}}%
\pgfpathlineto{\pgfqpoint{5.489435in}{2.864420in}}%
\pgfpathlineto{\pgfqpoint{5.475108in}{2.857702in}}%
\pgfpathlineto{\pgfqpoint{5.460797in}{2.851083in}}%
\pgfpathlineto{\pgfqpoint{5.446500in}{2.844565in}}%
\pgfpathlineto{\pgfqpoint{5.432217in}{2.838146in}}%
\pgfpathlineto{\pgfqpoint{5.439631in}{2.846038in}}%
\pgfpathlineto{\pgfqpoint{5.447038in}{2.853836in}}%
\pgfpathlineto{\pgfqpoint{5.454437in}{2.861542in}}%
\pgfpathlineto{\pgfqpoint{5.461828in}{2.869157in}}%
\pgfpathclose%
\pgfusepath{fill}%
\end{pgfscope}%
\begin{pgfscope}%
\pgfpathrectangle{\pgfqpoint{1.150000in}{0.150000in}}{\pgfqpoint{5.700000in}{5.700000in}}%
\pgfusepath{clip}%
\pgfsetbuttcap%
\pgfsetroundjoin%
\definecolor{currentfill}{rgb}{0.274952,0.037752,0.364543}%
\pgfsetfillcolor{currentfill}%
\pgfsetfillopacity{0.700000}%
\pgfsetlinewidth{0.000000pt}%
\definecolor{currentstroke}{rgb}{0.000000,0.000000,0.000000}%
\pgfsetstrokecolor{currentstroke}%
\pgfsetdash{}{0pt}%
\pgfpathmoveto{\pgfqpoint{4.134696in}{2.083795in}}%
\pgfpathlineto{\pgfqpoint{4.148410in}{2.082603in}}%
\pgfpathlineto{\pgfqpoint{4.162131in}{2.081516in}}%
\pgfpathlineto{\pgfqpoint{4.175861in}{2.080535in}}%
\pgfpathlineto{\pgfqpoint{4.189598in}{2.079660in}}%
\pgfpathlineto{\pgfqpoint{4.181719in}{2.069924in}}%
\pgfpathlineto{\pgfqpoint{4.173834in}{2.060205in}}%
\pgfpathlineto{\pgfqpoint{4.165943in}{2.050505in}}%
\pgfpathlineto{\pgfqpoint{4.158048in}{2.040827in}}%
\pgfpathlineto{\pgfqpoint{4.144301in}{2.041949in}}%
\pgfpathlineto{\pgfqpoint{4.130562in}{2.043176in}}%
\pgfpathlineto{\pgfqpoint{4.116831in}{2.044508in}}%
\pgfpathlineto{\pgfqpoint{4.103107in}{2.045947in}}%
\pgfpathlineto{\pgfqpoint{4.111013in}{2.055372in}}%
\pgfpathlineto{\pgfqpoint{4.118912in}{2.064824in}}%
\pgfpathlineto{\pgfqpoint{4.126807in}{2.074299in}}%
\pgfpathlineto{\pgfqpoint{4.134696in}{2.083795in}}%
\pgfpathclose%
\pgfusepath{fill}%
\end{pgfscope}%
\begin{pgfscope}%
\pgfpathrectangle{\pgfqpoint{1.150000in}{0.150000in}}{\pgfqpoint{5.700000in}{5.700000in}}%
\pgfusepath{clip}%
\pgfsetbuttcap%
\pgfsetroundjoin%
\definecolor{currentfill}{rgb}{0.263663,0.237631,0.518762}%
\pgfsetfillcolor{currentfill}%
\pgfsetfillopacity{0.700000}%
\pgfsetlinewidth{0.000000pt}%
\definecolor{currentstroke}{rgb}{0.000000,0.000000,0.000000}%
\pgfsetstrokecolor{currentstroke}%
\pgfsetdash{}{0pt}%
\pgfpathmoveto{\pgfqpoint{2.785999in}{2.480434in}}%
\pgfpathlineto{\pgfqpoint{2.799648in}{2.466575in}}%
\pgfpathlineto{\pgfqpoint{2.813294in}{2.452878in}}%
\pgfpathlineto{\pgfqpoint{2.826937in}{2.439340in}}%
\pgfpathlineto{\pgfqpoint{2.840577in}{2.425961in}}%
\pgfpathlineto{\pgfqpoint{2.832023in}{2.425892in}}%
\pgfpathlineto{\pgfqpoint{2.823455in}{2.426032in}}%
\pgfpathlineto{\pgfqpoint{2.814873in}{2.426383in}}%
\pgfpathlineto{\pgfqpoint{2.806277in}{2.426951in}}%
\pgfpathlineto{\pgfqpoint{2.792600in}{2.440738in}}%
\pgfpathlineto{\pgfqpoint{2.778921in}{2.454684in}}%
\pgfpathlineto{\pgfqpoint{2.765238in}{2.468791in}}%
\pgfpathlineto{\pgfqpoint{2.751552in}{2.483059in}}%
\pgfpathlineto{\pgfqpoint{2.760186in}{2.482075in}}%
\pgfpathlineto{\pgfqpoint{2.768805in}{2.481312in}}%
\pgfpathlineto{\pgfqpoint{2.777409in}{2.480767in}}%
\pgfpathlineto{\pgfqpoint{2.785999in}{2.480434in}}%
\pgfpathclose%
\pgfusepath{fill}%
\end{pgfscope}%
\begin{pgfscope}%
\pgfpathrectangle{\pgfqpoint{1.150000in}{0.150000in}}{\pgfqpoint{5.700000in}{5.700000in}}%
\pgfusepath{clip}%
\pgfsetbuttcap%
\pgfsetroundjoin%
\definecolor{currentfill}{rgb}{0.195860,0.395433,0.555276}%
\pgfsetfillcolor{currentfill}%
\pgfsetfillopacity{0.700000}%
\pgfsetlinewidth{0.000000pt}%
\definecolor{currentstroke}{rgb}{0.000000,0.000000,0.000000}%
\pgfsetstrokecolor{currentstroke}%
\pgfsetdash{}{0pt}%
\pgfpathmoveto{\pgfqpoint{2.457140in}{2.865162in}}%
\pgfpathlineto{\pgfqpoint{2.470909in}{2.846987in}}%
\pgfpathlineto{\pgfqpoint{2.484670in}{2.829011in}}%
\pgfpathlineto{\pgfqpoint{2.498425in}{2.811233in}}%
\pgfpathlineto{\pgfqpoint{2.512172in}{2.793650in}}%
\pgfpathlineto{\pgfqpoint{2.503383in}{2.796023in}}%
\pgfpathlineto{\pgfqpoint{2.494577in}{2.798635in}}%
\pgfpathlineto{\pgfqpoint{2.485754in}{2.801490in}}%
\pgfpathlineto{\pgfqpoint{2.476913in}{2.804594in}}%
\pgfpathlineto{\pgfqpoint{2.463121in}{2.822599in}}%
\pgfpathlineto{\pgfqpoint{2.449323in}{2.840802in}}%
\pgfpathlineto{\pgfqpoint{2.435516in}{2.859202in}}%
\pgfpathlineto{\pgfqpoint{2.421703in}{2.877803in}}%
\pgfpathlineto{\pgfqpoint{2.430589in}{2.874267in}}%
\pgfpathlineto{\pgfqpoint{2.439457in}{2.870984in}}%
\pgfpathlineto{\pgfqpoint{2.448307in}{2.867951in}}%
\pgfpathlineto{\pgfqpoint{2.457140in}{2.865162in}}%
\pgfpathclose%
\pgfusepath{fill}%
\end{pgfscope}%
\begin{pgfscope}%
\pgfpathrectangle{\pgfqpoint{1.150000in}{0.150000in}}{\pgfqpoint{5.700000in}{5.700000in}}%
\pgfusepath{clip}%
\pgfsetbuttcap%
\pgfsetroundjoin%
\definecolor{currentfill}{rgb}{0.258965,0.251537,0.524736}%
\pgfsetfillcolor{currentfill}%
\pgfsetfillopacity{0.700000}%
\pgfsetlinewidth{0.000000pt}%
\definecolor{currentstroke}{rgb}{0.000000,0.000000,0.000000}%
\pgfsetstrokecolor{currentstroke}%
\pgfsetdash{}{0pt}%
\pgfpathmoveto{\pgfqpoint{4.912254in}{2.498036in}}%
\pgfpathlineto{\pgfqpoint{4.926264in}{2.501950in}}%
\pgfpathlineto{\pgfqpoint{4.940287in}{2.505964in}}%
\pgfpathlineto{\pgfqpoint{4.954321in}{2.510079in}}%
\pgfpathlineto{\pgfqpoint{4.968368in}{2.514295in}}%
\pgfpathlineto{\pgfqpoint{4.960743in}{2.504345in}}%
\pgfpathlineto{\pgfqpoint{4.953112in}{2.494322in}}%
\pgfpathlineto{\pgfqpoint{4.945475in}{2.484227in}}%
\pgfpathlineto{\pgfqpoint{4.937831in}{2.474059in}}%
\pgfpathlineto{\pgfqpoint{4.923779in}{2.469927in}}%
\pgfpathlineto{\pgfqpoint{4.909738in}{2.465895in}}%
\pgfpathlineto{\pgfqpoint{4.895710in}{2.461964in}}%
\pgfpathlineto{\pgfqpoint{4.881693in}{2.458133in}}%
\pgfpathlineto{\pgfqpoint{4.889342in}{2.468210in}}%
\pgfpathlineto{\pgfqpoint{4.896985in}{2.478220in}}%
\pgfpathlineto{\pgfqpoint{4.904623in}{2.488162in}}%
\pgfpathlineto{\pgfqpoint{4.912254in}{2.498036in}}%
\pgfpathclose%
\pgfusepath{fill}%
\end{pgfscope}%
\begin{pgfscope}%
\pgfpathrectangle{\pgfqpoint{1.150000in}{0.150000in}}{\pgfqpoint{5.700000in}{5.700000in}}%
\pgfusepath{clip}%
\pgfsetbuttcap%
\pgfsetroundjoin%
\definecolor{currentfill}{rgb}{0.267004,0.004874,0.329415}%
\pgfsetfillcolor{currentfill}%
\pgfsetfillopacity{0.700000}%
\pgfsetlinewidth{0.000000pt}%
\definecolor{currentstroke}{rgb}{0.000000,0.000000,0.000000}%
\pgfsetstrokecolor{currentstroke}%
\pgfsetdash{}{0pt}%
\pgfpathmoveto{\pgfqpoint{3.820673in}{2.018174in}}%
\pgfpathlineto{\pgfqpoint{3.834313in}{2.014452in}}%
\pgfpathlineto{\pgfqpoint{3.847959in}{2.010842in}}%
\pgfpathlineto{\pgfqpoint{3.861611in}{2.007342in}}%
\pgfpathlineto{\pgfqpoint{3.875269in}{2.003953in}}%
\pgfpathlineto{\pgfqpoint{3.867277in}{1.995751in}}%
\pgfpathlineto{\pgfqpoint{3.859279in}{1.987615in}}%
\pgfpathlineto{\pgfqpoint{3.851274in}{1.979546in}}%
\pgfpathlineto{\pgfqpoint{3.843263in}{1.971547in}}%
\pgfpathlineto{\pgfqpoint{3.829591in}{1.975236in}}%
\pgfpathlineto{\pgfqpoint{3.815925in}{1.979036in}}%
\pgfpathlineto{\pgfqpoint{3.802265in}{1.982946in}}%
\pgfpathlineto{\pgfqpoint{3.788610in}{1.986968in}}%
\pgfpathlineto{\pgfqpoint{3.796635in}{1.994659in}}%
\pgfpathlineto{\pgfqpoint{3.804654in}{2.002426in}}%
\pgfpathlineto{\pgfqpoint{3.812667in}{2.010265in}}%
\pgfpathlineto{\pgfqpoint{3.820673in}{2.018174in}}%
\pgfpathclose%
\pgfusepath{fill}%
\end{pgfscope}%
\begin{pgfscope}%
\pgfpathrectangle{\pgfqpoint{1.150000in}{0.150000in}}{\pgfqpoint{5.700000in}{5.700000in}}%
\pgfusepath{clip}%
\pgfsetbuttcap%
\pgfsetroundjoin%
\definecolor{currentfill}{rgb}{0.180629,0.429975,0.557282}%
\pgfsetfillcolor{currentfill}%
\pgfsetfillopacity{0.700000}%
\pgfsetlinewidth{0.000000pt}%
\definecolor{currentstroke}{rgb}{0.000000,0.000000,0.000000}%
\pgfsetstrokecolor{currentstroke}%
\pgfsetdash{}{0pt}%
\pgfpathmoveto{\pgfqpoint{5.548448in}{2.924520in}}%
\pgfpathlineto{\pgfqpoint{5.562768in}{2.931218in}}%
\pgfpathlineto{\pgfqpoint{5.577102in}{2.938016in}}%
\pgfpathlineto{\pgfqpoint{5.591451in}{2.944913in}}%
\pgfpathlineto{\pgfqpoint{5.605816in}{2.951909in}}%
\pgfpathlineto{\pgfqpoint{5.598478in}{2.944803in}}%
\pgfpathlineto{\pgfqpoint{5.591133in}{2.937600in}}%
\pgfpathlineto{\pgfqpoint{5.583779in}{2.930299in}}%
\pgfpathlineto{\pgfqpoint{5.576417in}{2.922898in}}%
\pgfpathlineto{\pgfqpoint{5.562042in}{2.915832in}}%
\pgfpathlineto{\pgfqpoint{5.547681in}{2.908865in}}%
\pgfpathlineto{\pgfqpoint{5.533336in}{2.901998in}}%
\pgfpathlineto{\pgfqpoint{5.519005in}{2.895231in}}%
\pgfpathlineto{\pgfqpoint{5.526378in}{2.902693in}}%
\pgfpathlineto{\pgfqpoint{5.533742in}{2.910062in}}%
\pgfpathlineto{\pgfqpoint{5.541099in}{2.917337in}}%
\pgfpathlineto{\pgfqpoint{5.548448in}{2.924520in}}%
\pgfpathclose%
\pgfusepath{fill}%
\end{pgfscope}%
\begin{pgfscope}%
\pgfpathrectangle{\pgfqpoint{1.150000in}{0.150000in}}{\pgfqpoint{5.700000in}{5.700000in}}%
\pgfusepath{clip}%
\pgfsetbuttcap%
\pgfsetroundjoin%
\definecolor{currentfill}{rgb}{0.270595,0.214069,0.507052}%
\pgfsetfillcolor{currentfill}%
\pgfsetfillopacity{0.700000}%
\pgfsetlinewidth{0.000000pt}%
\definecolor{currentstroke}{rgb}{0.000000,0.000000,0.000000}%
\pgfsetstrokecolor{currentstroke}%
\pgfsetdash{}{0pt}%
\pgfpathmoveto{\pgfqpoint{2.840577in}{2.425961in}}%
\pgfpathlineto{\pgfqpoint{2.854215in}{2.412740in}}%
\pgfpathlineto{\pgfqpoint{2.867851in}{2.399674in}}%
\pgfpathlineto{\pgfqpoint{2.881484in}{2.386764in}}%
\pgfpathlineto{\pgfqpoint{2.895116in}{2.374008in}}%
\pgfpathlineto{\pgfqpoint{2.886596in}{2.373540in}}%
\pgfpathlineto{\pgfqpoint{2.878064in}{2.373275in}}%
\pgfpathlineto{\pgfqpoint{2.869518in}{2.373217in}}%
\pgfpathlineto{\pgfqpoint{2.860958in}{2.373371in}}%
\pgfpathlineto{\pgfqpoint{2.847291in}{2.386533in}}%
\pgfpathlineto{\pgfqpoint{2.833622in}{2.399850in}}%
\pgfpathlineto{\pgfqpoint{2.819951in}{2.413322in}}%
\pgfpathlineto{\pgfqpoint{2.806277in}{2.426951in}}%
\pgfpathlineto{\pgfqpoint{2.814873in}{2.426383in}}%
\pgfpathlineto{\pgfqpoint{2.823455in}{2.426032in}}%
\pgfpathlineto{\pgfqpoint{2.832023in}{2.425892in}}%
\pgfpathlineto{\pgfqpoint{2.840577in}{2.425961in}}%
\pgfpathclose%
\pgfusepath{fill}%
\end{pgfscope}%
\begin{pgfscope}%
\pgfpathrectangle{\pgfqpoint{1.150000in}{0.150000in}}{\pgfqpoint{5.700000in}{5.700000in}}%
\pgfusepath{clip}%
\pgfsetbuttcap%
\pgfsetroundjoin%
\definecolor{currentfill}{rgb}{0.274952,0.037752,0.364543}%
\pgfsetfillcolor{currentfill}%
\pgfsetfillopacity{0.700000}%
\pgfsetlinewidth{0.000000pt}%
\definecolor{currentstroke}{rgb}{0.000000,0.000000,0.000000}%
\pgfsetstrokecolor{currentstroke}%
\pgfsetdash{}{0pt}%
\pgfpathmoveto{\pgfqpoint{3.396986in}{2.075293in}}%
\pgfpathlineto{\pgfqpoint{3.410578in}{2.067804in}}%
\pgfpathlineto{\pgfqpoint{3.424172in}{2.060439in}}%
\pgfpathlineto{\pgfqpoint{3.437770in}{2.053197in}}%
\pgfpathlineto{\pgfqpoint{3.451371in}{2.046077in}}%
\pgfpathlineto{\pgfqpoint{3.443187in}{2.040942in}}%
\pgfpathlineto{\pgfqpoint{3.434995in}{2.035938in}}%
\pgfpathlineto{\pgfqpoint{3.426794in}{2.031066in}}%
\pgfpathlineto{\pgfqpoint{3.418584in}{2.026331in}}%
\pgfpathlineto{\pgfqpoint{3.404961in}{2.033808in}}%
\pgfpathlineto{\pgfqpoint{3.391341in}{2.041407in}}%
\pgfpathlineto{\pgfqpoint{3.377723in}{2.049129in}}%
\pgfpathlineto{\pgfqpoint{3.364108in}{2.056975in}}%
\pgfpathlineto{\pgfqpoint{3.372341in}{2.061346in}}%
\pgfpathlineto{\pgfqpoint{3.380565in}{2.065858in}}%
\pgfpathlineto{\pgfqpoint{3.388780in}{2.070509in}}%
\pgfpathlineto{\pgfqpoint{3.396986in}{2.075293in}}%
\pgfpathclose%
\pgfusepath{fill}%
\end{pgfscope}%
\begin{pgfscope}%
\pgfpathrectangle{\pgfqpoint{1.150000in}{0.150000in}}{\pgfqpoint{5.700000in}{5.700000in}}%
\pgfusepath{clip}%
\pgfsetbuttcap%
\pgfsetroundjoin%
\definecolor{currentfill}{rgb}{0.272594,0.025563,0.353093}%
\pgfsetfillcolor{currentfill}%
\pgfsetfillopacity{0.700000}%
\pgfsetlinewidth{0.000000pt}%
\definecolor{currentstroke}{rgb}{0.000000,0.000000,0.000000}%
\pgfsetstrokecolor{currentstroke}%
\pgfsetdash{}{0pt}%
\pgfpathmoveto{\pgfqpoint{4.048288in}{2.052767in}}%
\pgfpathlineto{\pgfqpoint{4.061982in}{2.050902in}}%
\pgfpathlineto{\pgfqpoint{4.075683in}{2.049144in}}%
\pgfpathlineto{\pgfqpoint{4.089391in}{2.047492in}}%
\pgfpathlineto{\pgfqpoint{4.103107in}{2.045947in}}%
\pgfpathlineto{\pgfqpoint{4.095197in}{2.036550in}}%
\pgfpathlineto{\pgfqpoint{4.087280in}{2.027184in}}%
\pgfpathlineto{\pgfqpoint{4.079359in}{2.017850in}}%
\pgfpathlineto{\pgfqpoint{4.071432in}{2.008553in}}%
\pgfpathlineto{\pgfqpoint{4.057705in}{2.010361in}}%
\pgfpathlineto{\pgfqpoint{4.043986in}{2.012277in}}%
\pgfpathlineto{\pgfqpoint{4.030274in}{2.014299in}}%
\pgfpathlineto{\pgfqpoint{4.016570in}{2.016428in}}%
\pgfpathlineto{\pgfqpoint{4.024508in}{2.025455in}}%
\pgfpathlineto{\pgfqpoint{4.032440in}{2.034523in}}%
\pgfpathlineto{\pgfqpoint{4.040367in}{2.043627in}}%
\pgfpathlineto{\pgfqpoint{4.048288in}{2.052767in}}%
\pgfpathclose%
\pgfusepath{fill}%
\end{pgfscope}%
\begin{pgfscope}%
\pgfpathrectangle{\pgfqpoint{1.150000in}{0.150000in}}{\pgfqpoint{5.700000in}{5.700000in}}%
\pgfusepath{clip}%
\pgfsetbuttcap%
\pgfsetroundjoin%
\definecolor{currentfill}{rgb}{0.248629,0.278775,0.534556}%
\pgfsetfillcolor{currentfill}%
\pgfsetfillopacity{0.700000}%
\pgfsetlinewidth{0.000000pt}%
\definecolor{currentstroke}{rgb}{0.000000,0.000000,0.000000}%
\pgfsetstrokecolor{currentstroke}%
\pgfsetdash{}{0pt}%
\pgfpathmoveto{\pgfqpoint{4.998807in}{2.553357in}}%
\pgfpathlineto{\pgfqpoint{5.012860in}{2.557738in}}%
\pgfpathlineto{\pgfqpoint{5.026925in}{2.562219in}}%
\pgfpathlineto{\pgfqpoint{5.041004in}{2.566800in}}%
\pgfpathlineto{\pgfqpoint{5.055094in}{2.571481in}}%
\pgfpathlineto{\pgfqpoint{5.047501in}{2.561770in}}%
\pgfpathlineto{\pgfqpoint{5.039900in}{2.551979in}}%
\pgfpathlineto{\pgfqpoint{5.032294in}{2.542109in}}%
\pgfpathlineto{\pgfqpoint{5.024681in}{2.532161in}}%
\pgfpathlineto{\pgfqpoint{5.010584in}{2.527544in}}%
\pgfpathlineto{\pgfqpoint{4.996499in}{2.523027in}}%
\pgfpathlineto{\pgfqpoint{4.982428in}{2.518611in}}%
\pgfpathlineto{\pgfqpoint{4.968368in}{2.514295in}}%
\pgfpathlineto{\pgfqpoint{4.975987in}{2.524172in}}%
\pgfpathlineto{\pgfqpoint{4.983600in}{2.533974in}}%
\pgfpathlineto{\pgfqpoint{4.991207in}{2.543703in}}%
\pgfpathlineto{\pgfqpoint{4.998807in}{2.553357in}}%
\pgfpathclose%
\pgfusepath{fill}%
\end{pgfscope}%
\begin{pgfscope}%
\pgfpathrectangle{\pgfqpoint{1.150000in}{0.150000in}}{\pgfqpoint{5.700000in}{5.700000in}}%
\pgfusepath{clip}%
\pgfsetbuttcap%
\pgfsetroundjoin%
\definecolor{currentfill}{rgb}{0.281446,0.084320,0.407414}%
\pgfsetfillcolor{currentfill}%
\pgfsetfillopacity{0.700000}%
\pgfsetlinewidth{0.000000pt}%
\definecolor{currentstroke}{rgb}{0.000000,0.000000,0.000000}%
\pgfsetstrokecolor{currentstroke}%
\pgfsetdash{}{0pt}%
\pgfpathmoveto{\pgfqpoint{3.200883in}{2.161024in}}%
\pgfpathlineto{\pgfqpoint{3.214476in}{2.151640in}}%
\pgfpathlineto{\pgfqpoint{3.228071in}{2.142388in}}%
\pgfpathlineto{\pgfqpoint{3.241667in}{2.133267in}}%
\pgfpathlineto{\pgfqpoint{3.255265in}{2.124277in}}%
\pgfpathlineto{\pgfqpoint{3.246972in}{2.120791in}}%
\pgfpathlineto{\pgfqpoint{3.238670in}{2.117464in}}%
\pgfpathlineto{\pgfqpoint{3.230357in}{2.114298in}}%
\pgfpathlineto{\pgfqpoint{3.222034in}{2.111298in}}%
\pgfpathlineto{\pgfqpoint{3.208409in}{2.120666in}}%
\pgfpathlineto{\pgfqpoint{3.194786in}{2.130165in}}%
\pgfpathlineto{\pgfqpoint{3.181164in}{2.139795in}}%
\pgfpathlineto{\pgfqpoint{3.167543in}{2.149559in}}%
\pgfpathlineto{\pgfqpoint{3.175894in}{2.152173in}}%
\pgfpathlineto{\pgfqpoint{3.184234in}{2.154958in}}%
\pgfpathlineto{\pgfqpoint{3.192564in}{2.157910in}}%
\pgfpathlineto{\pgfqpoint{3.200883in}{2.161024in}}%
\pgfpathclose%
\pgfusepath{fill}%
\end{pgfscope}%
\begin{pgfscope}%
\pgfpathrectangle{\pgfqpoint{1.150000in}{0.150000in}}{\pgfqpoint{5.700000in}{5.700000in}}%
\pgfusepath{clip}%
\pgfsetbuttcap%
\pgfsetroundjoin%
\definecolor{currentfill}{rgb}{0.171176,0.452530,0.557965}%
\pgfsetfillcolor{currentfill}%
\pgfsetfillopacity{0.700000}%
\pgfsetlinewidth{0.000000pt}%
\definecolor{currentstroke}{rgb}{0.000000,0.000000,0.000000}%
\pgfsetstrokecolor{currentstroke}%
\pgfsetdash{}{0pt}%
\pgfpathmoveto{\pgfqpoint{5.635083in}{2.979388in}}%
\pgfpathlineto{\pgfqpoint{5.649450in}{2.986395in}}%
\pgfpathlineto{\pgfqpoint{5.663832in}{2.993501in}}%
\pgfpathlineto{\pgfqpoint{5.678229in}{3.000707in}}%
\pgfpathlineto{\pgfqpoint{5.692642in}{3.008013in}}%
\pgfpathlineto{\pgfqpoint{5.685350in}{3.001379in}}%
\pgfpathlineto{\pgfqpoint{5.678050in}{2.994649in}}%
\pgfpathlineto{\pgfqpoint{5.670741in}{2.987820in}}%
\pgfpathlineto{\pgfqpoint{5.663424in}{2.980892in}}%
\pgfpathlineto{\pgfqpoint{5.648999in}{2.973497in}}%
\pgfpathlineto{\pgfqpoint{5.634589in}{2.966201in}}%
\pgfpathlineto{\pgfqpoint{5.620195in}{2.959005in}}%
\pgfpathlineto{\pgfqpoint{5.605816in}{2.951909in}}%
\pgfpathlineto{\pgfqpoint{5.613145in}{2.958920in}}%
\pgfpathlineto{\pgfqpoint{5.620466in}{2.965835in}}%
\pgfpathlineto{\pgfqpoint{5.627778in}{2.972658in}}%
\pgfpathlineto{\pgfqpoint{5.635083in}{2.979388in}}%
\pgfpathclose%
\pgfusepath{fill}%
\end{pgfscope}%
\begin{pgfscope}%
\pgfpathrectangle{\pgfqpoint{1.150000in}{0.150000in}}{\pgfqpoint{5.700000in}{5.700000in}}%
\pgfusepath{clip}%
\pgfsetbuttcap%
\pgfsetroundjoin%
\definecolor{currentfill}{rgb}{0.276194,0.190074,0.493001}%
\pgfsetfillcolor{currentfill}%
\pgfsetfillopacity{0.700000}%
\pgfsetlinewidth{0.000000pt}%
\definecolor{currentstroke}{rgb}{0.000000,0.000000,0.000000}%
\pgfsetstrokecolor{currentstroke}%
\pgfsetdash{}{0pt}%
\pgfpathmoveto{\pgfqpoint{2.895116in}{2.374008in}}%
\pgfpathlineto{\pgfqpoint{2.908745in}{2.361405in}}%
\pgfpathlineto{\pgfqpoint{2.922373in}{2.348953in}}%
\pgfpathlineto{\pgfqpoint{2.936000in}{2.336651in}}%
\pgfpathlineto{\pgfqpoint{2.949625in}{2.324499in}}%
\pgfpathlineto{\pgfqpoint{2.941139in}{2.323634in}}%
\pgfpathlineto{\pgfqpoint{2.932641in}{2.322967in}}%
\pgfpathlineto{\pgfqpoint{2.924130in}{2.322501in}}%
\pgfpathlineto{\pgfqpoint{2.915605in}{2.322242in}}%
\pgfpathlineto{\pgfqpoint{2.901946in}{2.334798in}}%
\pgfpathlineto{\pgfqpoint{2.888285in}{2.347504in}}%
\pgfpathlineto{\pgfqpoint{2.874622in}{2.360362in}}%
\pgfpathlineto{\pgfqpoint{2.860958in}{2.373371in}}%
\pgfpathlineto{\pgfqpoint{2.869518in}{2.373217in}}%
\pgfpathlineto{\pgfqpoint{2.878064in}{2.373275in}}%
\pgfpathlineto{\pgfqpoint{2.886596in}{2.373540in}}%
\pgfpathlineto{\pgfqpoint{2.895116in}{2.374008in}}%
\pgfpathclose%
\pgfusepath{fill}%
\end{pgfscope}%
\begin{pgfscope}%
\pgfpathrectangle{\pgfqpoint{1.150000in}{0.150000in}}{\pgfqpoint{5.700000in}{5.700000in}}%
\pgfusepath{clip}%
\pgfsetbuttcap%
\pgfsetroundjoin%
\definecolor{currentfill}{rgb}{0.237441,0.305202,0.541921}%
\pgfsetfillcolor{currentfill}%
\pgfsetfillopacity{0.700000}%
\pgfsetlinewidth{0.000000pt}%
\definecolor{currentstroke}{rgb}{0.000000,0.000000,0.000000}%
\pgfsetstrokecolor{currentstroke}%
\pgfsetdash{}{0pt}%
\pgfpathmoveto{\pgfqpoint{5.085405in}{2.609531in}}%
\pgfpathlineto{\pgfqpoint{5.099502in}{2.614359in}}%
\pgfpathlineto{\pgfqpoint{5.113612in}{2.619286in}}%
\pgfpathlineto{\pgfqpoint{5.127735in}{2.624314in}}%
\pgfpathlineto{\pgfqpoint{5.141871in}{2.629442in}}%
\pgfpathlineto{\pgfqpoint{5.134310in}{2.620011in}}%
\pgfpathlineto{\pgfqpoint{5.126742in}{2.610495in}}%
\pgfpathlineto{\pgfqpoint{5.119168in}{2.600895in}}%
\pgfpathlineto{\pgfqpoint{5.111586in}{2.591210in}}%
\pgfpathlineto{\pgfqpoint{5.097444in}{2.586128in}}%
\pgfpathlineto{\pgfqpoint{5.083315in}{2.581145in}}%
\pgfpathlineto{\pgfqpoint{5.069198in}{2.576263in}}%
\pgfpathlineto{\pgfqpoint{5.055094in}{2.571481in}}%
\pgfpathlineto{\pgfqpoint{5.062682in}{2.581114in}}%
\pgfpathlineto{\pgfqpoint{5.070263in}{2.590666in}}%
\pgfpathlineto{\pgfqpoint{5.077837in}{2.600139in}}%
\pgfpathlineto{\pgfqpoint{5.085405in}{2.609531in}}%
\pgfpathclose%
\pgfusepath{fill}%
\end{pgfscope}%
\begin{pgfscope}%
\pgfpathrectangle{\pgfqpoint{1.150000in}{0.150000in}}{\pgfqpoint{5.700000in}{5.700000in}}%
\pgfusepath{clip}%
\pgfsetbuttcap%
\pgfsetroundjoin%
\definecolor{currentfill}{rgb}{0.162142,0.474838,0.558140}%
\pgfsetfillcolor{currentfill}%
\pgfsetfillopacity{0.700000}%
\pgfsetlinewidth{0.000000pt}%
\definecolor{currentstroke}{rgb}{0.000000,0.000000,0.000000}%
\pgfsetstrokecolor{currentstroke}%
\pgfsetdash{}{0pt}%
\pgfpathmoveto{\pgfqpoint{5.721725in}{3.033604in}}%
\pgfpathlineto{\pgfqpoint{5.736139in}{3.040900in}}%
\pgfpathlineto{\pgfqpoint{5.750569in}{3.048296in}}%
\pgfpathlineto{\pgfqpoint{5.765015in}{3.055791in}}%
\pgfpathlineto{\pgfqpoint{5.779476in}{3.063386in}}%
\pgfpathlineto{\pgfqpoint{5.772232in}{3.057243in}}%
\pgfpathlineto{\pgfqpoint{5.764979in}{3.051004in}}%
\pgfpathlineto{\pgfqpoint{5.757717in}{3.044667in}}%
\pgfpathlineto{\pgfqpoint{5.750447in}{3.038231in}}%
\pgfpathlineto{\pgfqpoint{5.735972in}{3.030527in}}%
\pgfpathlineto{\pgfqpoint{5.721513in}{3.022923in}}%
\pgfpathlineto{\pgfqpoint{5.707070in}{3.015418in}}%
\pgfpathlineto{\pgfqpoint{5.692642in}{3.008013in}}%
\pgfpathlineto{\pgfqpoint{5.699925in}{3.014551in}}%
\pgfpathlineto{\pgfqpoint{5.707200in}{3.020994in}}%
\pgfpathlineto{\pgfqpoint{5.714467in}{3.027345in}}%
\pgfpathlineto{\pgfqpoint{5.721725in}{3.033604in}}%
\pgfpathclose%
\pgfusepath{fill}%
\end{pgfscope}%
\begin{pgfscope}%
\pgfpathrectangle{\pgfqpoint{1.150000in}{0.150000in}}{\pgfqpoint{5.700000in}{5.700000in}}%
\pgfusepath{clip}%
\pgfsetbuttcap%
\pgfsetroundjoin%
\definecolor{currentfill}{rgb}{0.269944,0.014625,0.341379}%
\pgfsetfillcolor{currentfill}%
\pgfsetfillopacity{0.700000}%
\pgfsetlinewidth{0.000000pt}%
\definecolor{currentstroke}{rgb}{0.000000,0.000000,0.000000}%
\pgfsetstrokecolor{currentstroke}%
\pgfsetdash{}{0pt}%
\pgfpathmoveto{\pgfqpoint{3.961820in}{2.026022in}}%
\pgfpathlineto{\pgfqpoint{3.975497in}{2.023462in}}%
\pgfpathlineto{\pgfqpoint{3.989181in}{2.021009in}}%
\pgfpathlineto{\pgfqpoint{4.002872in}{2.018665in}}%
\pgfpathlineto{\pgfqpoint{4.016570in}{2.016428in}}%
\pgfpathlineto{\pgfqpoint{4.008626in}{2.007444in}}%
\pgfpathlineto{\pgfqpoint{4.000677in}{1.998505in}}%
\pgfpathlineto{\pgfqpoint{3.992722in}{1.989613in}}%
\pgfpathlineto{\pgfqpoint{3.984761in}{1.980772in}}%
\pgfpathlineto{\pgfqpoint{3.971051in}{1.983290in}}%
\pgfpathlineto{\pgfqpoint{3.957349in}{1.985917in}}%
\pgfpathlineto{\pgfqpoint{3.943653in}{1.988651in}}%
\pgfpathlineto{\pgfqpoint{3.929963in}{1.991493in}}%
\pgfpathlineto{\pgfqpoint{3.937936in}{2.000046in}}%
\pgfpathlineto{\pgfqpoint{3.945903in}{2.008653in}}%
\pgfpathlineto{\pgfqpoint{3.953865in}{2.017313in}}%
\pgfpathlineto{\pgfqpoint{3.961820in}{2.026022in}}%
\pgfpathclose%
\pgfusepath{fill}%
\end{pgfscope}%
\begin{pgfscope}%
\pgfpathrectangle{\pgfqpoint{1.150000in}{0.150000in}}{\pgfqpoint{5.700000in}{5.700000in}}%
\pgfusepath{clip}%
\pgfsetbuttcap%
\pgfsetroundjoin%
\definecolor{currentfill}{rgb}{0.268510,0.009605,0.335427}%
\pgfsetfillcolor{currentfill}%
\pgfsetfillopacity{0.700000}%
\pgfsetlinewidth{0.000000pt}%
\definecolor{currentstroke}{rgb}{0.000000,0.000000,0.000000}%
\pgfsetstrokecolor{currentstroke}%
\pgfsetdash{}{0pt}%
\pgfpathmoveto{\pgfqpoint{3.592785in}{2.017919in}}%
\pgfpathlineto{\pgfqpoint{3.606398in}{2.012211in}}%
\pgfpathlineto{\pgfqpoint{3.620015in}{2.006619in}}%
\pgfpathlineto{\pgfqpoint{3.633637in}{2.001144in}}%
\pgfpathlineto{\pgfqpoint{3.647263in}{1.995784in}}%
\pgfpathlineto{\pgfqpoint{3.639170in}{1.989172in}}%
\pgfpathlineto{\pgfqpoint{3.631070in}{1.982663in}}%
\pgfpathlineto{\pgfqpoint{3.622963in}{1.976260in}}%
\pgfpathlineto{\pgfqpoint{3.614848in}{1.969966in}}%
\pgfpathlineto{\pgfqpoint{3.601204in}{1.975663in}}%
\pgfpathlineto{\pgfqpoint{3.587563in}{1.981475in}}%
\pgfpathlineto{\pgfqpoint{3.573927in}{1.987404in}}%
\pgfpathlineto{\pgfqpoint{3.560295in}{1.993449in}}%
\pgfpathlineto{\pgfqpoint{3.568429in}{1.999399in}}%
\pgfpathlineto{\pgfqpoint{3.576556in}{2.005463in}}%
\pgfpathlineto{\pgfqpoint{3.584674in}{2.011637in}}%
\pgfpathlineto{\pgfqpoint{3.592785in}{2.017919in}}%
\pgfpathclose%
\pgfusepath{fill}%
\end{pgfscope}%
\begin{pgfscope}%
\pgfpathrectangle{\pgfqpoint{1.150000in}{0.150000in}}{\pgfqpoint{5.700000in}{5.700000in}}%
\pgfusepath{clip}%
\pgfsetbuttcap%
\pgfsetroundjoin%
\definecolor{currentfill}{rgb}{0.153364,0.497000,0.557724}%
\pgfsetfillcolor{currentfill}%
\pgfsetfillopacity{0.700000}%
\pgfsetlinewidth{0.000000pt}%
\definecolor{currentstroke}{rgb}{0.000000,0.000000,0.000000}%
\pgfsetstrokecolor{currentstroke}%
\pgfsetdash{}{0pt}%
\pgfpathmoveto{\pgfqpoint{5.808366in}{3.087025in}}%
\pgfpathlineto{\pgfqpoint{5.822828in}{3.094590in}}%
\pgfpathlineto{\pgfqpoint{5.837306in}{3.102255in}}%
\pgfpathlineto{\pgfqpoint{5.851800in}{3.110019in}}%
\pgfpathlineto{\pgfqpoint{5.866309in}{3.117883in}}%
\pgfpathlineto{\pgfqpoint{5.859115in}{3.112246in}}%
\pgfpathlineto{\pgfqpoint{5.851912in}{3.106514in}}%
\pgfpathlineto{\pgfqpoint{5.844699in}{3.100686in}}%
\pgfpathlineto{\pgfqpoint{5.837478in}{3.094759in}}%
\pgfpathlineto{\pgfqpoint{5.822954in}{3.086767in}}%
\pgfpathlineto{\pgfqpoint{5.808445in}{3.078873in}}%
\pgfpathlineto{\pgfqpoint{5.793953in}{3.071080in}}%
\pgfpathlineto{\pgfqpoint{5.779476in}{3.063386in}}%
\pgfpathlineto{\pgfqpoint{5.786711in}{3.069434in}}%
\pgfpathlineto{\pgfqpoint{5.793938in}{3.075388in}}%
\pgfpathlineto{\pgfqpoint{5.801157in}{3.081252in}}%
\pgfpathlineto{\pgfqpoint{5.808366in}{3.087025in}}%
\pgfpathclose%
\pgfusepath{fill}%
\end{pgfscope}%
\begin{pgfscope}%
\pgfpathrectangle{\pgfqpoint{1.150000in}{0.150000in}}{\pgfqpoint{5.700000in}{5.700000in}}%
\pgfusepath{clip}%
\pgfsetbuttcap%
\pgfsetroundjoin%
\definecolor{currentfill}{rgb}{0.279574,0.170599,0.479997}%
\pgfsetfillcolor{currentfill}%
\pgfsetfillopacity{0.700000}%
\pgfsetlinewidth{0.000000pt}%
\definecolor{currentstroke}{rgb}{0.000000,0.000000,0.000000}%
\pgfsetstrokecolor{currentstroke}%
\pgfsetdash{}{0pt}%
\pgfpathmoveto{\pgfqpoint{2.949625in}{2.324499in}}%
\pgfpathlineto{\pgfqpoint{2.963249in}{2.312496in}}%
\pgfpathlineto{\pgfqpoint{2.976871in}{2.300639in}}%
\pgfpathlineto{\pgfqpoint{2.990493in}{2.288929in}}%
\pgfpathlineto{\pgfqpoint{3.004114in}{2.277364in}}%
\pgfpathlineto{\pgfqpoint{2.995661in}{2.276103in}}%
\pgfpathlineto{\pgfqpoint{2.987195in}{2.275035in}}%
\pgfpathlineto{\pgfqpoint{2.978717in}{2.274164in}}%
\pgfpathlineto{\pgfqpoint{2.970227in}{2.273493in}}%
\pgfpathlineto{\pgfqpoint{2.956573in}{2.285461in}}%
\pgfpathlineto{\pgfqpoint{2.942918in}{2.297574in}}%
\pgfpathlineto{\pgfqpoint{2.929262in}{2.309834in}}%
\pgfpathlineto{\pgfqpoint{2.915605in}{2.322242in}}%
\pgfpathlineto{\pgfqpoint{2.924130in}{2.322501in}}%
\pgfpathlineto{\pgfqpoint{2.932641in}{2.322967in}}%
\pgfpathlineto{\pgfqpoint{2.941139in}{2.323634in}}%
\pgfpathlineto{\pgfqpoint{2.949625in}{2.324499in}}%
\pgfpathclose%
\pgfusepath{fill}%
\end{pgfscope}%
\begin{pgfscope}%
\pgfpathrectangle{\pgfqpoint{1.150000in}{0.150000in}}{\pgfqpoint{5.700000in}{5.700000in}}%
\pgfusepath{clip}%
\pgfsetbuttcap%
\pgfsetroundjoin%
\definecolor{currentfill}{rgb}{0.225863,0.330805,0.547314}%
\pgfsetfillcolor{currentfill}%
\pgfsetfillopacity{0.700000}%
\pgfsetlinewidth{0.000000pt}%
\definecolor{currentstroke}{rgb}{0.000000,0.000000,0.000000}%
\pgfsetstrokecolor{currentstroke}%
\pgfsetdash{}{0pt}%
\pgfpathmoveto{\pgfqpoint{5.172048in}{2.666318in}}%
\pgfpathlineto{\pgfqpoint{5.186190in}{2.671573in}}%
\pgfpathlineto{\pgfqpoint{5.200346in}{2.676928in}}%
\pgfpathlineto{\pgfqpoint{5.214515in}{2.682383in}}%
\pgfpathlineto{\pgfqpoint{5.228697in}{2.687938in}}%
\pgfpathlineto{\pgfqpoint{5.221170in}{2.678826in}}%
\pgfpathlineto{\pgfqpoint{5.213636in}{2.669625in}}%
\pgfpathlineto{\pgfqpoint{5.206095in}{2.660335in}}%
\pgfpathlineto{\pgfqpoint{5.198548in}{2.650955in}}%
\pgfpathlineto{\pgfqpoint{5.184359in}{2.645426in}}%
\pgfpathlineto{\pgfqpoint{5.170183in}{2.639998in}}%
\pgfpathlineto{\pgfqpoint{5.156020in}{2.634670in}}%
\pgfpathlineto{\pgfqpoint{5.141871in}{2.629442in}}%
\pgfpathlineto{\pgfqpoint{5.149426in}{2.638788in}}%
\pgfpathlineto{\pgfqpoint{5.156973in}{2.648050in}}%
\pgfpathlineto{\pgfqpoint{5.164514in}{2.657226in}}%
\pgfpathlineto{\pgfqpoint{5.172048in}{2.666318in}}%
\pgfpathclose%
\pgfusepath{fill}%
\end{pgfscope}%
\begin{pgfscope}%
\pgfpathrectangle{\pgfqpoint{1.150000in}{0.150000in}}{\pgfqpoint{5.700000in}{5.700000in}}%
\pgfusepath{clip}%
\pgfsetbuttcap%
\pgfsetroundjoin%
\definecolor{currentfill}{rgb}{0.267004,0.004874,0.329415}%
\pgfsetfillcolor{currentfill}%
\pgfsetfillopacity{0.700000}%
\pgfsetlinewidth{0.000000pt}%
\definecolor{currentstroke}{rgb}{0.000000,0.000000,0.000000}%
\pgfsetstrokecolor{currentstroke}%
\pgfsetdash{}{0pt}%
\pgfpathmoveto{\pgfqpoint{3.734045in}{2.004174in}}%
\pgfpathlineto{\pgfqpoint{3.747678in}{1.999704in}}%
\pgfpathlineto{\pgfqpoint{3.761317in}{1.995346in}}%
\pgfpathlineto{\pgfqpoint{3.774961in}{1.991101in}}%
\pgfpathlineto{\pgfqpoint{3.788610in}{1.986968in}}%
\pgfpathlineto{\pgfqpoint{3.780578in}{1.979354in}}%
\pgfpathlineto{\pgfqpoint{3.772539in}{1.971822in}}%
\pgfpathlineto{\pgfqpoint{3.764494in}{1.964373in}}%
\pgfpathlineto{\pgfqpoint{3.756442in}{1.957012in}}%
\pgfpathlineto{\pgfqpoint{3.742777in}{1.961463in}}%
\pgfpathlineto{\pgfqpoint{3.729118in}{1.966027in}}%
\pgfpathlineto{\pgfqpoint{3.715463in}{1.970703in}}%
\pgfpathlineto{\pgfqpoint{3.701813in}{1.975491in}}%
\pgfpathlineto{\pgfqpoint{3.709882in}{1.982528in}}%
\pgfpathlineto{\pgfqpoint{3.717943in}{1.989655in}}%
\pgfpathlineto{\pgfqpoint{3.725997in}{1.996872in}}%
\pgfpathlineto{\pgfqpoint{3.734045in}{2.004174in}}%
\pgfpathclose%
\pgfusepath{fill}%
\end{pgfscope}%
\begin{pgfscope}%
\pgfpathrectangle{\pgfqpoint{1.150000in}{0.150000in}}{\pgfqpoint{5.700000in}{5.700000in}}%
\pgfusepath{clip}%
\pgfsetbuttcap%
\pgfsetroundjoin%
\definecolor{currentfill}{rgb}{0.146180,0.515413,0.556823}%
\pgfsetfillcolor{currentfill}%
\pgfsetfillopacity{0.700000}%
\pgfsetlinewidth{0.000000pt}%
\definecolor{currentstroke}{rgb}{0.000000,0.000000,0.000000}%
\pgfsetstrokecolor{currentstroke}%
\pgfsetdash{}{0pt}%
\pgfpathmoveto{\pgfqpoint{5.894999in}{3.139518in}}%
\pgfpathlineto{\pgfqpoint{5.909508in}{3.147333in}}%
\pgfpathlineto{\pgfqpoint{5.924034in}{3.155247in}}%
\pgfpathlineto{\pgfqpoint{5.938575in}{3.163260in}}%
\pgfpathlineto{\pgfqpoint{5.953133in}{3.171373in}}%
\pgfpathlineto{\pgfqpoint{5.945991in}{3.166253in}}%
\pgfpathlineto{\pgfqpoint{5.938839in}{3.161041in}}%
\pgfpathlineto{\pgfqpoint{5.931678in}{3.155735in}}%
\pgfpathlineto{\pgfqpoint{5.924508in}{3.150333in}}%
\pgfpathlineto{\pgfqpoint{5.909934in}{3.142071in}}%
\pgfpathlineto{\pgfqpoint{5.895376in}{3.133909in}}%
\pgfpathlineto{\pgfqpoint{5.880835in}{3.125846in}}%
\pgfpathlineto{\pgfqpoint{5.866309in}{3.117883in}}%
\pgfpathlineto{\pgfqpoint{5.873495in}{3.123427in}}%
\pgfpathlineto{\pgfqpoint{5.880672in}{3.128880in}}%
\pgfpathlineto{\pgfqpoint{5.887840in}{3.134243in}}%
\pgfpathlineto{\pgfqpoint{5.894999in}{3.139518in}}%
\pgfpathclose%
\pgfusepath{fill}%
\end{pgfscope}%
\begin{pgfscope}%
\pgfpathrectangle{\pgfqpoint{1.150000in}{0.150000in}}{\pgfqpoint{5.700000in}{5.700000in}}%
\pgfusepath{clip}%
\pgfsetbuttcap%
\pgfsetroundjoin%
\definecolor{currentfill}{rgb}{0.280267,0.073417,0.397163}%
\pgfsetfillcolor{currentfill}%
\pgfsetfillopacity{0.700000}%
\pgfsetlinewidth{0.000000pt}%
\definecolor{currentstroke}{rgb}{0.000000,0.000000,0.000000}%
\pgfsetstrokecolor{currentstroke}%
\pgfsetdash{}{0pt}%
\pgfpathmoveto{\pgfqpoint{3.255265in}{2.124277in}}%
\pgfpathlineto{\pgfqpoint{3.268864in}{2.115417in}}%
\pgfpathlineto{\pgfqpoint{3.282464in}{2.106686in}}%
\pgfpathlineto{\pgfqpoint{3.296067in}{2.098084in}}%
\pgfpathlineto{\pgfqpoint{3.309671in}{2.089610in}}%
\pgfpathlineto{\pgfqpoint{3.301404in}{2.085753in}}%
\pgfpathlineto{\pgfqpoint{3.293127in}{2.082051in}}%
\pgfpathlineto{\pgfqpoint{3.284841in}{2.078505in}}%
\pgfpathlineto{\pgfqpoint{3.276544in}{2.075120in}}%
\pgfpathlineto{\pgfqpoint{3.262914in}{2.083972in}}%
\pgfpathlineto{\pgfqpoint{3.249286in}{2.092951in}}%
\pgfpathlineto{\pgfqpoint{3.235659in}{2.102060in}}%
\pgfpathlineto{\pgfqpoint{3.222034in}{2.111298in}}%
\pgfpathlineto{\pgfqpoint{3.230357in}{2.114298in}}%
\pgfpathlineto{\pgfqpoint{3.238670in}{2.117464in}}%
\pgfpathlineto{\pgfqpoint{3.246972in}{2.120791in}}%
\pgfpathlineto{\pgfqpoint{3.255265in}{2.124277in}}%
\pgfpathclose%
\pgfusepath{fill}%
\end{pgfscope}%
\begin{pgfscope}%
\pgfpathrectangle{\pgfqpoint{1.150000in}{0.150000in}}{\pgfqpoint{5.700000in}{5.700000in}}%
\pgfusepath{clip}%
\pgfsetbuttcap%
\pgfsetroundjoin%
\definecolor{currentfill}{rgb}{0.283197,0.115680,0.436115}%
\pgfsetfillcolor{currentfill}%
\pgfsetfillopacity{0.700000}%
\pgfsetlinewidth{0.000000pt}%
\definecolor{currentstroke}{rgb}{0.000000,0.000000,0.000000}%
\pgfsetstrokecolor{currentstroke}%
\pgfsetdash{}{0pt}%
\pgfpathmoveto{\pgfqpoint{4.448974in}{2.202276in}}%
\pgfpathlineto{\pgfqpoint{4.462803in}{2.203384in}}%
\pgfpathlineto{\pgfqpoint{4.476642in}{2.204594in}}%
\pgfpathlineto{\pgfqpoint{4.490491in}{2.205907in}}%
\pgfpathlineto{\pgfqpoint{4.504350in}{2.207323in}}%
\pgfpathlineto{\pgfqpoint{4.496562in}{2.196798in}}%
\pgfpathlineto{\pgfqpoint{4.488769in}{2.186249in}}%
\pgfpathlineto{\pgfqpoint{4.480972in}{2.175677in}}%
\pgfpathlineto{\pgfqpoint{4.473168in}{2.165084in}}%
\pgfpathlineto{\pgfqpoint{4.459303in}{2.163862in}}%
\pgfpathlineto{\pgfqpoint{4.445448in}{2.162742in}}%
\pgfpathlineto{\pgfqpoint{4.431602in}{2.161724in}}%
\pgfpathlineto{\pgfqpoint{4.417766in}{2.160809in}}%
\pgfpathlineto{\pgfqpoint{4.425576in}{2.171202in}}%
\pgfpathlineto{\pgfqpoint{4.433380in}{2.181578in}}%
\pgfpathlineto{\pgfqpoint{4.441180in}{2.191937in}}%
\pgfpathlineto{\pgfqpoint{4.448974in}{2.202276in}}%
\pgfpathclose%
\pgfusepath{fill}%
\end{pgfscope}%
\begin{pgfscope}%
\pgfpathrectangle{\pgfqpoint{1.150000in}{0.150000in}}{\pgfqpoint{5.700000in}{5.700000in}}%
\pgfusepath{clip}%
\pgfsetbuttcap%
\pgfsetroundjoin%
\definecolor{currentfill}{rgb}{0.281924,0.089666,0.412415}%
\pgfsetfillcolor{currentfill}%
\pgfsetfillopacity{0.700000}%
\pgfsetlinewidth{0.000000pt}%
\definecolor{currentstroke}{rgb}{0.000000,0.000000,0.000000}%
\pgfsetstrokecolor{currentstroke}%
\pgfsetdash{}{0pt}%
\pgfpathmoveto{\pgfqpoint{4.362516in}{2.158179in}}%
\pgfpathlineto{\pgfqpoint{4.376314in}{2.158681in}}%
\pgfpathlineto{\pgfqpoint{4.390122in}{2.159287in}}%
\pgfpathlineto{\pgfqpoint{4.403940in}{2.159996in}}%
\pgfpathlineto{\pgfqpoint{4.417766in}{2.160809in}}%
\pgfpathlineto{\pgfqpoint{4.409952in}{2.150401in}}%
\pgfpathlineto{\pgfqpoint{4.402132in}{2.139981in}}%
\pgfpathlineto{\pgfqpoint{4.394307in}{2.129550in}}%
\pgfpathlineto{\pgfqpoint{4.386477in}{2.119109in}}%
\pgfpathlineto{\pgfqpoint{4.372643in}{2.118508in}}%
\pgfpathlineto{\pgfqpoint{4.358819in}{2.118009in}}%
\pgfpathlineto{\pgfqpoint{4.345004in}{2.117614in}}%
\pgfpathlineto{\pgfqpoint{4.331198in}{2.117322in}}%
\pgfpathlineto{\pgfqpoint{4.339035in}{2.127545in}}%
\pgfpathlineto{\pgfqpoint{4.346867in}{2.137763in}}%
\pgfpathlineto{\pgfqpoint{4.354694in}{2.147975in}}%
\pgfpathlineto{\pgfqpoint{4.362516in}{2.158179in}}%
\pgfpathclose%
\pgfusepath{fill}%
\end{pgfscope}%
\begin{pgfscope}%
\pgfpathrectangle{\pgfqpoint{1.150000in}{0.150000in}}{\pgfqpoint{5.700000in}{5.700000in}}%
\pgfusepath{clip}%
\pgfsetbuttcap%
\pgfsetroundjoin%
\definecolor{currentfill}{rgb}{0.282623,0.140926,0.457517}%
\pgfsetfillcolor{currentfill}%
\pgfsetfillopacity{0.700000}%
\pgfsetlinewidth{0.000000pt}%
\definecolor{currentstroke}{rgb}{0.000000,0.000000,0.000000}%
\pgfsetstrokecolor{currentstroke}%
\pgfsetdash{}{0pt}%
\pgfpathmoveto{\pgfqpoint{4.535448in}{2.249149in}}%
\pgfpathlineto{\pgfqpoint{4.549311in}{2.250841in}}%
\pgfpathlineto{\pgfqpoint{4.563184in}{2.252635in}}%
\pgfpathlineto{\pgfqpoint{4.577067in}{2.254532in}}%
\pgfpathlineto{\pgfqpoint{4.590960in}{2.256530in}}%
\pgfpathlineto{\pgfqpoint{4.583199in}{2.245950in}}%
\pgfpathlineto{\pgfqpoint{4.575433in}{2.235335in}}%
\pgfpathlineto{\pgfqpoint{4.567662in}{2.224686in}}%
\pgfpathlineto{\pgfqpoint{4.559886in}{2.214006in}}%
\pgfpathlineto{\pgfqpoint{4.545986in}{2.212182in}}%
\pgfpathlineto{\pgfqpoint{4.532097in}{2.210460in}}%
\pgfpathlineto{\pgfqpoint{4.518219in}{2.208840in}}%
\pgfpathlineto{\pgfqpoint{4.504350in}{2.207323in}}%
\pgfpathlineto{\pgfqpoint{4.512132in}{2.217822in}}%
\pgfpathlineto{\pgfqpoint{4.519910in}{2.228294in}}%
\pgfpathlineto{\pgfqpoint{4.527682in}{2.238736in}}%
\pgfpathlineto{\pgfqpoint{4.535448in}{2.249149in}}%
\pgfpathclose%
\pgfusepath{fill}%
\end{pgfscope}%
\begin{pgfscope}%
\pgfpathrectangle{\pgfqpoint{1.150000in}{0.150000in}}{\pgfqpoint{5.700000in}{5.700000in}}%
\pgfusepath{clip}%
\pgfsetbuttcap%
\pgfsetroundjoin%
\definecolor{currentfill}{rgb}{0.272594,0.025563,0.353093}%
\pgfsetfillcolor{currentfill}%
\pgfsetfillopacity{0.700000}%
\pgfsetlinewidth{0.000000pt}%
\definecolor{currentstroke}{rgb}{0.000000,0.000000,0.000000}%
\pgfsetstrokecolor{currentstroke}%
\pgfsetdash{}{0pt}%
\pgfpathmoveto{\pgfqpoint{3.451371in}{2.046077in}}%
\pgfpathlineto{\pgfqpoint{3.464974in}{2.039079in}}%
\pgfpathlineto{\pgfqpoint{3.478581in}{2.032201in}}%
\pgfpathlineto{\pgfqpoint{3.492192in}{2.025445in}}%
\pgfpathlineto{\pgfqpoint{3.505805in}{2.018808in}}%
\pgfpathlineto{\pgfqpoint{3.497643in}{2.013324in}}%
\pgfpathlineto{\pgfqpoint{3.489472in}{2.007965in}}%
\pgfpathlineto{\pgfqpoint{3.481293in}{2.002735in}}%
\pgfpathlineto{\pgfqpoint{3.473105in}{1.997636in}}%
\pgfpathlineto{\pgfqpoint{3.459471in}{2.004629in}}%
\pgfpathlineto{\pgfqpoint{3.445839in}{2.011742in}}%
\pgfpathlineto{\pgfqpoint{3.432210in}{2.018976in}}%
\pgfpathlineto{\pgfqpoint{3.418584in}{2.026331in}}%
\pgfpathlineto{\pgfqpoint{3.426794in}{2.031066in}}%
\pgfpathlineto{\pgfqpoint{3.434995in}{2.035938in}}%
\pgfpathlineto{\pgfqpoint{3.443187in}{2.040942in}}%
\pgfpathlineto{\pgfqpoint{3.451371in}{2.046077in}}%
\pgfpathclose%
\pgfusepath{fill}%
\end{pgfscope}%
\begin{pgfscope}%
\pgfpathrectangle{\pgfqpoint{1.150000in}{0.150000in}}{\pgfqpoint{5.700000in}{5.700000in}}%
\pgfusepath{clip}%
\pgfsetbuttcap%
\pgfsetroundjoin%
\definecolor{currentfill}{rgb}{0.126453,0.570633,0.549841}%
\pgfsetfillcolor{currentfill}%
\pgfsetfillopacity{0.700000}%
\pgfsetlinewidth{0.000000pt}%
\definecolor{currentstroke}{rgb}{0.000000,0.000000,0.000000}%
\pgfsetstrokecolor{currentstroke}%
\pgfsetdash{}{0pt}%
\pgfpathmoveto{\pgfqpoint{6.154754in}{3.290304in}}%
\pgfpathlineto{\pgfqpoint{6.169404in}{3.298745in}}%
\pgfpathlineto{\pgfqpoint{6.184070in}{3.307285in}}%
\pgfpathlineto{\pgfqpoint{6.198753in}{3.315924in}}%
\pgfpathlineto{\pgfqpoint{6.191773in}{3.312342in}}%
\pgfpathlineto{\pgfqpoint{6.184783in}{3.308681in}}%
\pgfpathlineto{\pgfqpoint{6.177784in}{3.304937in}}%
\pgfpathlineto{\pgfqpoint{6.170776in}{3.301110in}}%
\pgfpathlineto{\pgfqpoint{6.156072in}{3.292262in}}%
\pgfpathlineto{\pgfqpoint{6.141385in}{3.283513in}}%
\pgfpathlineto{\pgfqpoint{6.126715in}{3.274863in}}%
\pgfpathlineto{\pgfqpoint{6.133738in}{3.278842in}}%
\pgfpathlineto{\pgfqpoint{6.140753in}{3.282740in}}%
\pgfpathlineto{\pgfqpoint{6.147758in}{3.286560in}}%
\pgfpathlineto{\pgfqpoint{6.154754in}{3.290304in}}%
\pgfpathclose%
\pgfusepath{fill}%
\end{pgfscope}%
\begin{pgfscope}%
\pgfpathrectangle{\pgfqpoint{1.150000in}{0.150000in}}{\pgfqpoint{5.700000in}{5.700000in}}%
\pgfusepath{clip}%
\pgfsetbuttcap%
\pgfsetroundjoin%
\definecolor{currentfill}{rgb}{0.137770,0.537492,0.554906}%
\pgfsetfillcolor{currentfill}%
\pgfsetfillopacity{0.700000}%
\pgfsetlinewidth{0.000000pt}%
\definecolor{currentstroke}{rgb}{0.000000,0.000000,0.000000}%
\pgfsetstrokecolor{currentstroke}%
\pgfsetdash{}{0pt}%
\pgfpathmoveto{\pgfqpoint{5.981614in}{3.190965in}}%
\pgfpathlineto{\pgfqpoint{5.996170in}{3.199009in}}%
\pgfpathlineto{\pgfqpoint{6.010743in}{3.207152in}}%
\pgfpathlineto{\pgfqpoint{6.025332in}{3.215394in}}%
\pgfpathlineto{\pgfqpoint{6.039938in}{3.223735in}}%
\pgfpathlineto{\pgfqpoint{6.032849in}{3.219142in}}%
\pgfpathlineto{\pgfqpoint{6.025752in}{3.214459in}}%
\pgfpathlineto{\pgfqpoint{6.018645in}{3.209685in}}%
\pgfpathlineto{\pgfqpoint{6.011528in}{3.204818in}}%
\pgfpathlineto{\pgfqpoint{5.996905in}{3.196307in}}%
\pgfpathlineto{\pgfqpoint{5.982298in}{3.187896in}}%
\pgfpathlineto{\pgfqpoint{5.967707in}{3.179585in}}%
\pgfpathlineto{\pgfqpoint{5.953133in}{3.171373in}}%
\pgfpathlineto{\pgfqpoint{5.960267in}{3.176402in}}%
\pgfpathlineto{\pgfqpoint{5.967391in}{3.181342in}}%
\pgfpathlineto{\pgfqpoint{5.974507in}{3.186196in}}%
\pgfpathlineto{\pgfqpoint{5.981614in}{3.190965in}}%
\pgfpathclose%
\pgfusepath{fill}%
\end{pgfscope}%
\begin{pgfscope}%
\pgfpathrectangle{\pgfqpoint{1.150000in}{0.150000in}}{\pgfqpoint{5.700000in}{5.700000in}}%
\pgfusepath{clip}%
\pgfsetbuttcap%
\pgfsetroundjoin%
\definecolor{currentfill}{rgb}{0.280255,0.165693,0.476498}%
\pgfsetfillcolor{currentfill}%
\pgfsetfillopacity{0.700000}%
\pgfsetlinewidth{0.000000pt}%
\definecolor{currentstroke}{rgb}{0.000000,0.000000,0.000000}%
\pgfsetstrokecolor{currentstroke}%
\pgfsetdash{}{0pt}%
\pgfpathmoveto{\pgfqpoint{4.621950in}{2.298474in}}%
\pgfpathlineto{\pgfqpoint{4.635849in}{2.300731in}}%
\pgfpathlineto{\pgfqpoint{4.649757in}{2.303089in}}%
\pgfpathlineto{\pgfqpoint{4.663677in}{2.305549in}}%
\pgfpathlineto{\pgfqpoint{4.677607in}{2.308111in}}%
\pgfpathlineto{\pgfqpoint{4.669873in}{2.297534in}}%
\pgfpathlineto{\pgfqpoint{4.662134in}{2.286913in}}%
\pgfpathlineto{\pgfqpoint{4.654389in}{2.276248in}}%
\pgfpathlineto{\pgfqpoint{4.646639in}{2.265540in}}%
\pgfpathlineto{\pgfqpoint{4.632704in}{2.263136in}}%
\pgfpathlineto{\pgfqpoint{4.618779in}{2.260832in}}%
\pgfpathlineto{\pgfqpoint{4.604864in}{2.258631in}}%
\pgfpathlineto{\pgfqpoint{4.590960in}{2.256530in}}%
\pgfpathlineto{\pgfqpoint{4.598716in}{2.267074in}}%
\pgfpathlineto{\pgfqpoint{4.606466in}{2.277580in}}%
\pgfpathlineto{\pgfqpoint{4.614211in}{2.288047in}}%
\pgfpathlineto{\pgfqpoint{4.621950in}{2.298474in}}%
\pgfpathclose%
\pgfusepath{fill}%
\end{pgfscope}%
\begin{pgfscope}%
\pgfpathrectangle{\pgfqpoint{1.150000in}{0.150000in}}{\pgfqpoint{5.700000in}{5.700000in}}%
\pgfusepath{clip}%
\pgfsetbuttcap%
\pgfsetroundjoin%
\definecolor{currentfill}{rgb}{0.279566,0.067836,0.391917}%
\pgfsetfillcolor{currentfill}%
\pgfsetfillopacity{0.700000}%
\pgfsetlinewidth{0.000000pt}%
\definecolor{currentstroke}{rgb}{0.000000,0.000000,0.000000}%
\pgfsetstrokecolor{currentstroke}%
\pgfsetdash{}{0pt}%
\pgfpathmoveto{\pgfqpoint{4.276062in}{2.117191in}}%
\pgfpathlineto{\pgfqpoint{4.289833in}{2.117067in}}%
\pgfpathlineto{\pgfqpoint{4.303612in}{2.117048in}}%
\pgfpathlineto{\pgfqpoint{4.317400in}{2.117133in}}%
\pgfpathlineto{\pgfqpoint{4.331198in}{2.117322in}}%
\pgfpathlineto{\pgfqpoint{4.323355in}{2.107096in}}%
\pgfpathlineto{\pgfqpoint{4.315508in}{2.096871in}}%
\pgfpathlineto{\pgfqpoint{4.307655in}{2.086646in}}%
\pgfpathlineto{\pgfqpoint{4.299797in}{2.076425in}}%
\pgfpathlineto{\pgfqpoint{4.285993in}{2.076465in}}%
\pgfpathlineto{\pgfqpoint{4.272197in}{2.076609in}}%
\pgfpathlineto{\pgfqpoint{4.258409in}{2.076856in}}%
\pgfpathlineto{\pgfqpoint{4.244630in}{2.077208in}}%
\pgfpathlineto{\pgfqpoint{4.252496in}{2.087194in}}%
\pgfpathlineto{\pgfqpoint{4.260357in}{2.097187in}}%
\pgfpathlineto{\pgfqpoint{4.268212in}{2.107187in}}%
\pgfpathlineto{\pgfqpoint{4.276062in}{2.117191in}}%
\pgfpathclose%
\pgfusepath{fill}%
\end{pgfscope}%
\begin{pgfscope}%
\pgfpathrectangle{\pgfqpoint{1.150000in}{0.150000in}}{\pgfqpoint{5.700000in}{5.700000in}}%
\pgfusepath{clip}%
\pgfsetbuttcap%
\pgfsetroundjoin%
\definecolor{currentfill}{rgb}{0.212395,0.359683,0.551710}%
\pgfsetfillcolor{currentfill}%
\pgfsetfillopacity{0.700000}%
\pgfsetlinewidth{0.000000pt}%
\definecolor{currentstroke}{rgb}{0.000000,0.000000,0.000000}%
\pgfsetstrokecolor{currentstroke}%
\pgfsetdash{}{0pt}%
\pgfpathmoveto{\pgfqpoint{5.258734in}{2.723492in}}%
\pgfpathlineto{\pgfqpoint{5.272922in}{2.729154in}}%
\pgfpathlineto{\pgfqpoint{5.287124in}{2.734917in}}%
\pgfpathlineto{\pgfqpoint{5.301339in}{2.740779in}}%
\pgfpathlineto{\pgfqpoint{5.315569in}{2.746741in}}%
\pgfpathlineto{\pgfqpoint{5.308078in}{2.737986in}}%
\pgfpathlineto{\pgfqpoint{5.300580in}{2.729136in}}%
\pgfpathlineto{\pgfqpoint{5.293075in}{2.720194in}}%
\pgfpathlineto{\pgfqpoint{5.285563in}{2.711157in}}%
\pgfpathlineto{\pgfqpoint{5.271326in}{2.705202in}}%
\pgfpathlineto{\pgfqpoint{5.257102in}{2.699347in}}%
\pgfpathlineto{\pgfqpoint{5.242893in}{2.693593in}}%
\pgfpathlineto{\pgfqpoint{5.228697in}{2.687938in}}%
\pgfpathlineto{\pgfqpoint{5.236217in}{2.696960in}}%
\pgfpathlineto{\pgfqpoint{5.243729in}{2.705893in}}%
\pgfpathlineto{\pgfqpoint{5.251235in}{2.714737in}}%
\pgfpathlineto{\pgfqpoint{5.258734in}{2.723492in}}%
\pgfpathclose%
\pgfusepath{fill}%
\end{pgfscope}%
\begin{pgfscope}%
\pgfpathrectangle{\pgfqpoint{1.150000in}{0.150000in}}{\pgfqpoint{5.700000in}{5.700000in}}%
\pgfusepath{clip}%
\pgfsetbuttcap%
\pgfsetroundjoin%
\definecolor{currentfill}{rgb}{0.281887,0.150881,0.465405}%
\pgfsetfillcolor{currentfill}%
\pgfsetfillopacity{0.700000}%
\pgfsetlinewidth{0.000000pt}%
\definecolor{currentstroke}{rgb}{0.000000,0.000000,0.000000}%
\pgfsetstrokecolor{currentstroke}%
\pgfsetdash{}{0pt}%
\pgfpathmoveto{\pgfqpoint{3.004114in}{2.277364in}}%
\pgfpathlineto{\pgfqpoint{3.017734in}{2.265943in}}%
\pgfpathlineto{\pgfqpoint{3.031353in}{2.254666in}}%
\pgfpathlineto{\pgfqpoint{3.044972in}{2.243530in}}%
\pgfpathlineto{\pgfqpoint{3.058591in}{2.232536in}}%
\pgfpathlineto{\pgfqpoint{3.050169in}{2.230881in}}%
\pgfpathlineto{\pgfqpoint{3.041736in}{2.229413in}}%
\pgfpathlineto{\pgfqpoint{3.033290in}{2.228138in}}%
\pgfpathlineto{\pgfqpoint{3.024832in}{2.227058in}}%
\pgfpathlineto{\pgfqpoint{3.011182in}{2.238454in}}%
\pgfpathlineto{\pgfqpoint{2.997531in}{2.249991in}}%
\pgfpathlineto{\pgfqpoint{2.983879in}{2.261670in}}%
\pgfpathlineto{\pgfqpoint{2.970227in}{2.273493in}}%
\pgfpathlineto{\pgfqpoint{2.978717in}{2.274164in}}%
\pgfpathlineto{\pgfqpoint{2.987195in}{2.275035in}}%
\pgfpathlineto{\pgfqpoint{2.995661in}{2.276103in}}%
\pgfpathlineto{\pgfqpoint{3.004114in}{2.277364in}}%
\pgfpathclose%
\pgfusepath{fill}%
\end{pgfscope}%
\begin{pgfscope}%
\pgfpathrectangle{\pgfqpoint{1.150000in}{0.150000in}}{\pgfqpoint{5.700000in}{5.700000in}}%
\pgfusepath{clip}%
\pgfsetbuttcap%
\pgfsetroundjoin%
\definecolor{currentfill}{rgb}{0.131172,0.555899,0.552459}%
\pgfsetfillcolor{currentfill}%
\pgfsetfillopacity{0.700000}%
\pgfsetlinewidth{0.000000pt}%
\definecolor{currentstroke}{rgb}{0.000000,0.000000,0.000000}%
\pgfsetstrokecolor{currentstroke}%
\pgfsetdash{}{0pt}%
\pgfpathmoveto{\pgfqpoint{6.068202in}{3.241259in}}%
\pgfpathlineto{\pgfqpoint{6.082805in}{3.249511in}}%
\pgfpathlineto{\pgfqpoint{6.097425in}{3.257863in}}%
\pgfpathlineto{\pgfqpoint{6.112061in}{3.266313in}}%
\pgfpathlineto{\pgfqpoint{6.126715in}{3.274863in}}%
\pgfpathlineto{\pgfqpoint{6.119682in}{3.270802in}}%
\pgfpathlineto{\pgfqpoint{6.112640in}{3.266654in}}%
\pgfpathlineto{\pgfqpoint{6.105589in}{3.262420in}}%
\pgfpathlineto{\pgfqpoint{6.098528in}{3.258096in}}%
\pgfpathlineto{\pgfqpoint{6.083855in}{3.249356in}}%
\pgfpathlineto{\pgfqpoint{6.069199in}{3.240716in}}%
\pgfpathlineto{\pgfqpoint{6.054560in}{3.232176in}}%
\pgfpathlineto{\pgfqpoint{6.039938in}{3.223735in}}%
\pgfpathlineto{\pgfqpoint{6.047018in}{3.228241in}}%
\pgfpathlineto{\pgfqpoint{6.054088in}{3.232663in}}%
\pgfpathlineto{\pgfqpoint{6.061149in}{3.237001in}}%
\pgfpathlineto{\pgfqpoint{6.068202in}{3.241259in}}%
\pgfpathclose%
\pgfusepath{fill}%
\end{pgfscope}%
\begin{pgfscope}%
\pgfpathrectangle{\pgfqpoint{1.150000in}{0.150000in}}{\pgfqpoint{5.700000in}{5.700000in}}%
\pgfusepath{clip}%
\pgfsetbuttcap%
\pgfsetroundjoin%
\definecolor{currentfill}{rgb}{0.276194,0.190074,0.493001}%
\pgfsetfillcolor{currentfill}%
\pgfsetfillopacity{0.700000}%
\pgfsetlinewidth{0.000000pt}%
\definecolor{currentstroke}{rgb}{0.000000,0.000000,0.000000}%
\pgfsetstrokecolor{currentstroke}%
\pgfsetdash{}{0pt}%
\pgfpathmoveto{\pgfqpoint{4.708488in}{2.349943in}}%
\pgfpathlineto{\pgfqpoint{4.722424in}{2.352744in}}%
\pgfpathlineto{\pgfqpoint{4.736370in}{2.355647in}}%
\pgfpathlineto{\pgfqpoint{4.750328in}{2.358650in}}%
\pgfpathlineto{\pgfqpoint{4.764297in}{2.361755in}}%
\pgfpathlineto{\pgfqpoint{4.756591in}{2.351239in}}%
\pgfpathlineto{\pgfqpoint{4.748879in}{2.340668in}}%
\pgfpathlineto{\pgfqpoint{4.741161in}{2.330044in}}%
\pgfpathlineto{\pgfqpoint{4.733438in}{2.319369in}}%
\pgfpathlineto{\pgfqpoint{4.719464in}{2.316403in}}%
\pgfpathlineto{\pgfqpoint{4.705500in}{2.313537in}}%
\pgfpathlineto{\pgfqpoint{4.691548in}{2.310773in}}%
\pgfpathlineto{\pgfqpoint{4.677607in}{2.308111in}}%
\pgfpathlineto{\pgfqpoint{4.685335in}{2.318641in}}%
\pgfpathlineto{\pgfqpoint{4.693058in}{2.329124in}}%
\pgfpathlineto{\pgfqpoint{4.700776in}{2.339558in}}%
\pgfpathlineto{\pgfqpoint{4.708488in}{2.349943in}}%
\pgfpathclose%
\pgfusepath{fill}%
\end{pgfscope}%
\begin{pgfscope}%
\pgfpathrectangle{\pgfqpoint{1.150000in}{0.150000in}}{\pgfqpoint{5.700000in}{5.700000in}}%
\pgfusepath{clip}%
\pgfsetbuttcap%
\pgfsetroundjoin%
\definecolor{currentfill}{rgb}{0.267004,0.004874,0.329415}%
\pgfsetfillcolor{currentfill}%
\pgfsetfillopacity{0.700000}%
\pgfsetlinewidth{0.000000pt}%
\definecolor{currentstroke}{rgb}{0.000000,0.000000,0.000000}%
\pgfsetstrokecolor{currentstroke}%
\pgfsetdash{}{0pt}%
\pgfpathmoveto{\pgfqpoint{3.875269in}{2.003953in}}%
\pgfpathlineto{\pgfqpoint{3.888933in}{2.000674in}}%
\pgfpathlineto{\pgfqpoint{3.902604in}{1.997505in}}%
\pgfpathlineto{\pgfqpoint{3.916280in}{1.994444in}}%
\pgfpathlineto{\pgfqpoint{3.929963in}{1.991493in}}%
\pgfpathlineto{\pgfqpoint{3.921984in}{1.982998in}}%
\pgfpathlineto{\pgfqpoint{3.913999in}{1.974564in}}%
\pgfpathlineto{\pgfqpoint{3.906008in}{1.966193in}}%
\pgfpathlineto{\pgfqpoint{3.898011in}{1.957888in}}%
\pgfpathlineto{\pgfqpoint{3.884315in}{1.961139in}}%
\pgfpathlineto{\pgfqpoint{3.870625in}{1.964499in}}%
\pgfpathlineto{\pgfqpoint{3.856941in}{1.967968in}}%
\pgfpathlineto{\pgfqpoint{3.843263in}{1.971547in}}%
\pgfpathlineto{\pgfqpoint{3.851274in}{1.979546in}}%
\pgfpathlineto{\pgfqpoint{3.859279in}{1.987615in}}%
\pgfpathlineto{\pgfqpoint{3.867277in}{1.995751in}}%
\pgfpathlineto{\pgfqpoint{3.875269in}{2.003953in}}%
\pgfpathclose%
\pgfusepath{fill}%
\end{pgfscope}%
\begin{pgfscope}%
\pgfpathrectangle{\pgfqpoint{1.150000in}{0.150000in}}{\pgfqpoint{5.700000in}{5.700000in}}%
\pgfusepath{clip}%
\pgfsetbuttcap%
\pgfsetroundjoin%
\definecolor{currentfill}{rgb}{0.277018,0.050344,0.375715}%
\pgfsetfillcolor{currentfill}%
\pgfsetfillopacity{0.700000}%
\pgfsetlinewidth{0.000000pt}%
\definecolor{currentstroke}{rgb}{0.000000,0.000000,0.000000}%
\pgfsetstrokecolor{currentstroke}%
\pgfsetdash{}{0pt}%
\pgfpathmoveto{\pgfqpoint{4.189598in}{2.079660in}}%
\pgfpathlineto{\pgfqpoint{4.203344in}{2.078889in}}%
\pgfpathlineto{\pgfqpoint{4.217098in}{2.078224in}}%
\pgfpathlineto{\pgfqpoint{4.230860in}{2.077664in}}%
\pgfpathlineto{\pgfqpoint{4.244630in}{2.077208in}}%
\pgfpathlineto{\pgfqpoint{4.236759in}{2.067233in}}%
\pgfpathlineto{\pgfqpoint{4.228883in}{2.057270in}}%
\pgfpathlineto{\pgfqpoint{4.221002in}{2.047322in}}%
\pgfpathlineto{\pgfqpoint{4.213115in}{2.037390in}}%
\pgfpathlineto{\pgfqpoint{4.199336in}{2.038093in}}%
\pgfpathlineto{\pgfqpoint{4.185565in}{2.038899in}}%
\pgfpathlineto{\pgfqpoint{4.171803in}{2.039811in}}%
\pgfpathlineto{\pgfqpoint{4.158048in}{2.040827in}}%
\pgfpathlineto{\pgfqpoint{4.165943in}{2.050505in}}%
\pgfpathlineto{\pgfqpoint{4.173834in}{2.060205in}}%
\pgfpathlineto{\pgfqpoint{4.181719in}{2.069924in}}%
\pgfpathlineto{\pgfqpoint{4.189598in}{2.079660in}}%
\pgfpathclose%
\pgfusepath{fill}%
\end{pgfscope}%
\begin{pgfscope}%
\pgfpathrectangle{\pgfqpoint{1.150000in}{0.150000in}}{\pgfqpoint{5.700000in}{5.700000in}}%
\pgfusepath{clip}%
\pgfsetbuttcap%
\pgfsetroundjoin%
\definecolor{currentfill}{rgb}{0.201239,0.383670,0.554294}%
\pgfsetfillcolor{currentfill}%
\pgfsetfillopacity{0.700000}%
\pgfsetlinewidth{0.000000pt}%
\definecolor{currentstroke}{rgb}{0.000000,0.000000,0.000000}%
\pgfsetstrokecolor{currentstroke}%
\pgfsetdash{}{0pt}%
\pgfpathmoveto{\pgfqpoint{5.345458in}{2.780836in}}%
\pgfpathlineto{\pgfqpoint{5.359693in}{2.786886in}}%
\pgfpathlineto{\pgfqpoint{5.373942in}{2.793037in}}%
\pgfpathlineto{\pgfqpoint{5.388206in}{2.799287in}}%
\pgfpathlineto{\pgfqpoint{5.402483in}{2.805637in}}%
\pgfpathlineto{\pgfqpoint{5.395031in}{2.797270in}}%
\pgfpathlineto{\pgfqpoint{5.387570in}{2.788807in}}%
\pgfpathlineto{\pgfqpoint{5.380103in}{2.780247in}}%
\pgfpathlineto{\pgfqpoint{5.372627in}{2.771589in}}%
\pgfpathlineto{\pgfqpoint{5.358342in}{2.765227in}}%
\pgfpathlineto{\pgfqpoint{5.344070in}{2.758965in}}%
\pgfpathlineto{\pgfqpoint{5.329812in}{2.752803in}}%
\pgfpathlineto{\pgfqpoint{5.315569in}{2.746741in}}%
\pgfpathlineto{\pgfqpoint{5.323052in}{2.755404in}}%
\pgfpathlineto{\pgfqpoint{5.330528in}{2.763973in}}%
\pgfpathlineto{\pgfqpoint{5.337997in}{2.772450in}}%
\pgfpathlineto{\pgfqpoint{5.345458in}{2.780836in}}%
\pgfpathclose%
\pgfusepath{fill}%
\end{pgfscope}%
\begin{pgfscope}%
\pgfpathrectangle{\pgfqpoint{1.150000in}{0.150000in}}{\pgfqpoint{5.700000in}{5.700000in}}%
\pgfusepath{clip}%
\pgfsetbuttcap%
\pgfsetroundjoin%
\definecolor{currentfill}{rgb}{0.269308,0.218818,0.509577}%
\pgfsetfillcolor{currentfill}%
\pgfsetfillopacity{0.700000}%
\pgfsetlinewidth{0.000000pt}%
\definecolor{currentstroke}{rgb}{0.000000,0.000000,0.000000}%
\pgfsetstrokecolor{currentstroke}%
\pgfsetdash{}{0pt}%
\pgfpathmoveto{\pgfqpoint{4.795067in}{2.403258in}}%
\pgfpathlineto{\pgfqpoint{4.809042in}{2.406584in}}%
\pgfpathlineto{\pgfqpoint{4.823029in}{2.410010in}}%
\pgfpathlineto{\pgfqpoint{4.837028in}{2.413538in}}%
\pgfpathlineto{\pgfqpoint{4.851038in}{2.417166in}}%
\pgfpathlineto{\pgfqpoint{4.843359in}{2.406764in}}%
\pgfpathlineto{\pgfqpoint{4.835675in}{2.396298in}}%
\pgfpathlineto{\pgfqpoint{4.827985in}{2.385771in}}%
\pgfpathlineto{\pgfqpoint{4.820289in}{2.375183in}}%
\pgfpathlineto{\pgfqpoint{4.806274in}{2.371675in}}%
\pgfpathlineto{\pgfqpoint{4.792270in}{2.368267in}}%
\pgfpathlineto{\pgfqpoint{4.778278in}{2.364961in}}%
\pgfpathlineto{\pgfqpoint{4.764297in}{2.361755in}}%
\pgfpathlineto{\pgfqpoint{4.771998in}{2.372216in}}%
\pgfpathlineto{\pgfqpoint{4.779694in}{2.382621in}}%
\pgfpathlineto{\pgfqpoint{4.787383in}{2.392968in}}%
\pgfpathlineto{\pgfqpoint{4.795067in}{2.403258in}}%
\pgfpathclose%
\pgfusepath{fill}%
\end{pgfscope}%
\begin{pgfscope}%
\pgfpathrectangle{\pgfqpoint{1.150000in}{0.150000in}}{\pgfqpoint{5.700000in}{5.700000in}}%
\pgfusepath{clip}%
\pgfsetbuttcap%
\pgfsetroundjoin%
\definecolor{currentfill}{rgb}{0.273809,0.031497,0.358853}%
\pgfsetfillcolor{currentfill}%
\pgfsetfillopacity{0.700000}%
\pgfsetlinewidth{0.000000pt}%
\definecolor{currentstroke}{rgb}{0.000000,0.000000,0.000000}%
\pgfsetstrokecolor{currentstroke}%
\pgfsetdash{}{0pt}%
\pgfpathmoveto{\pgfqpoint{4.103107in}{2.045947in}}%
\pgfpathlineto{\pgfqpoint{4.116831in}{2.044508in}}%
\pgfpathlineto{\pgfqpoint{4.130562in}{2.043176in}}%
\pgfpathlineto{\pgfqpoint{4.144301in}{2.041949in}}%
\pgfpathlineto{\pgfqpoint{4.158048in}{2.040827in}}%
\pgfpathlineto{\pgfqpoint{4.150147in}{2.031173in}}%
\pgfpathlineto{\pgfqpoint{4.142241in}{2.021544in}}%
\pgfpathlineto{\pgfqpoint{4.134329in}{2.011945in}}%
\pgfpathlineto{\pgfqpoint{4.126412in}{2.002376in}}%
\pgfpathlineto{\pgfqpoint{4.112656in}{2.003762in}}%
\pgfpathlineto{\pgfqpoint{4.098907in}{2.005253in}}%
\pgfpathlineto{\pgfqpoint{4.085165in}{2.006850in}}%
\pgfpathlineto{\pgfqpoint{4.071432in}{2.008553in}}%
\pgfpathlineto{\pgfqpoint{4.079359in}{2.017850in}}%
\pgfpathlineto{\pgfqpoint{4.087280in}{2.027184in}}%
\pgfpathlineto{\pgfqpoint{4.095197in}{2.036550in}}%
\pgfpathlineto{\pgfqpoint{4.103107in}{2.045947in}}%
\pgfpathclose%
\pgfusepath{fill}%
\end{pgfscope}%
\begin{pgfscope}%
\pgfpathrectangle{\pgfqpoint{1.150000in}{0.150000in}}{\pgfqpoint{5.700000in}{5.700000in}}%
\pgfusepath{clip}%
\pgfsetbuttcap%
\pgfsetroundjoin%
\definecolor{currentfill}{rgb}{0.283072,0.130895,0.449241}%
\pgfsetfillcolor{currentfill}%
\pgfsetfillopacity{0.700000}%
\pgfsetlinewidth{0.000000pt}%
\definecolor{currentstroke}{rgb}{0.000000,0.000000,0.000000}%
\pgfsetstrokecolor{currentstroke}%
\pgfsetdash{}{0pt}%
\pgfpathmoveto{\pgfqpoint{3.058591in}{2.232536in}}%
\pgfpathlineto{\pgfqpoint{3.072209in}{2.221682in}}%
\pgfpathlineto{\pgfqpoint{3.085828in}{2.210968in}}%
\pgfpathlineto{\pgfqpoint{3.099446in}{2.200392in}}%
\pgfpathlineto{\pgfqpoint{3.113065in}{2.189954in}}%
\pgfpathlineto{\pgfqpoint{3.104673in}{2.187905in}}%
\pgfpathlineto{\pgfqpoint{3.096271in}{2.186040in}}%
\pgfpathlineto{\pgfqpoint{3.087856in}{2.184361in}}%
\pgfpathlineto{\pgfqpoint{3.079430in}{2.182874in}}%
\pgfpathlineto{\pgfqpoint{3.065780in}{2.193712in}}%
\pgfpathlineto{\pgfqpoint{3.052131in}{2.204688in}}%
\pgfpathlineto{\pgfqpoint{3.038482in}{2.215803in}}%
\pgfpathlineto{\pgfqpoint{3.024832in}{2.227058in}}%
\pgfpathlineto{\pgfqpoint{3.033290in}{2.228138in}}%
\pgfpathlineto{\pgfqpoint{3.041736in}{2.229413in}}%
\pgfpathlineto{\pgfqpoint{3.050169in}{2.230881in}}%
\pgfpathlineto{\pgfqpoint{3.058591in}{2.232536in}}%
\pgfpathclose%
\pgfusepath{fill}%
\end{pgfscope}%
\begin{pgfscope}%
\pgfpathrectangle{\pgfqpoint{1.150000in}{0.150000in}}{\pgfqpoint{5.700000in}{5.700000in}}%
\pgfusepath{clip}%
\pgfsetbuttcap%
\pgfsetroundjoin%
\definecolor{currentfill}{rgb}{0.277941,0.056324,0.381191}%
\pgfsetfillcolor{currentfill}%
\pgfsetfillopacity{0.700000}%
\pgfsetlinewidth{0.000000pt}%
\definecolor{currentstroke}{rgb}{0.000000,0.000000,0.000000}%
\pgfsetstrokecolor{currentstroke}%
\pgfsetdash{}{0pt}%
\pgfpathmoveto{\pgfqpoint{3.309671in}{2.089610in}}%
\pgfpathlineto{\pgfqpoint{3.323277in}{2.081262in}}%
\pgfpathlineto{\pgfqpoint{3.336885in}{2.073041in}}%
\pgfpathlineto{\pgfqpoint{3.350496in}{2.064946in}}%
\pgfpathlineto{\pgfqpoint{3.364108in}{2.056975in}}%
\pgfpathlineto{\pgfqpoint{3.355866in}{2.052750in}}%
\pgfpathlineto{\pgfqpoint{3.347614in}{2.048672in}}%
\pgfpathlineto{\pgfqpoint{3.339353in}{2.044747in}}%
\pgfpathlineto{\pgfqpoint{3.331082in}{2.040979in}}%
\pgfpathlineto{\pgfqpoint{3.317445in}{2.049325in}}%
\pgfpathlineto{\pgfqpoint{3.303809in}{2.057797in}}%
\pgfpathlineto{\pgfqpoint{3.290176in}{2.066395in}}%
\pgfpathlineto{\pgfqpoint{3.276544in}{2.075120in}}%
\pgfpathlineto{\pgfqpoint{3.284841in}{2.078505in}}%
\pgfpathlineto{\pgfqpoint{3.293127in}{2.082051in}}%
\pgfpathlineto{\pgfqpoint{3.301404in}{2.085753in}}%
\pgfpathlineto{\pgfqpoint{3.309671in}{2.089610in}}%
\pgfpathclose%
\pgfusepath{fill}%
\end{pgfscope}%
\begin{pgfscope}%
\pgfpathrectangle{\pgfqpoint{1.150000in}{0.150000in}}{\pgfqpoint{5.700000in}{5.700000in}}%
\pgfusepath{clip}%
\pgfsetbuttcap%
\pgfsetroundjoin%
\definecolor{currentfill}{rgb}{0.260571,0.246922,0.522828}%
\pgfsetfillcolor{currentfill}%
\pgfsetfillopacity{0.700000}%
\pgfsetlinewidth{0.000000pt}%
\definecolor{currentstroke}{rgb}{0.000000,0.000000,0.000000}%
\pgfsetstrokecolor{currentstroke}%
\pgfsetdash{}{0pt}%
\pgfpathmoveto{\pgfqpoint{4.881693in}{2.458133in}}%
\pgfpathlineto{\pgfqpoint{4.895710in}{2.461964in}}%
\pgfpathlineto{\pgfqpoint{4.909738in}{2.465895in}}%
\pgfpathlineto{\pgfqpoint{4.923779in}{2.469927in}}%
\pgfpathlineto{\pgfqpoint{4.937831in}{2.474059in}}%
\pgfpathlineto{\pgfqpoint{4.930182in}{2.463821in}}%
\pgfpathlineto{\pgfqpoint{4.922526in}{2.453512in}}%
\pgfpathlineto{\pgfqpoint{4.914865in}{2.443134in}}%
\pgfpathlineto{\pgfqpoint{4.907197in}{2.432686in}}%
\pgfpathlineto{\pgfqpoint{4.893139in}{2.428655in}}%
\pgfpathlineto{\pgfqpoint{4.879093in}{2.424725in}}%
\pgfpathlineto{\pgfqpoint{4.865060in}{2.420895in}}%
\pgfpathlineto{\pgfqpoint{4.851038in}{2.417166in}}%
\pgfpathlineto{\pgfqpoint{4.858710in}{2.427505in}}%
\pgfpathlineto{\pgfqpoint{4.866377in}{2.437780in}}%
\pgfpathlineto{\pgfqpoint{4.874038in}{2.447989in}}%
\pgfpathlineto{\pgfqpoint{4.881693in}{2.458133in}}%
\pgfpathclose%
\pgfusepath{fill}%
\end{pgfscope}%
\begin{pgfscope}%
\pgfpathrectangle{\pgfqpoint{1.150000in}{0.150000in}}{\pgfqpoint{5.700000in}{5.700000in}}%
\pgfusepath{clip}%
\pgfsetbuttcap%
\pgfsetroundjoin%
\definecolor{currentfill}{rgb}{0.190631,0.407061,0.556089}%
\pgfsetfillcolor{currentfill}%
\pgfsetfillopacity{0.700000}%
\pgfsetlinewidth{0.000000pt}%
\definecolor{currentstroke}{rgb}{0.000000,0.000000,0.000000}%
\pgfsetstrokecolor{currentstroke}%
\pgfsetdash{}{0pt}%
\pgfpathmoveto{\pgfqpoint{5.432217in}{2.838146in}}%
\pgfpathlineto{\pgfqpoint{5.446500in}{2.844565in}}%
\pgfpathlineto{\pgfqpoint{5.460797in}{2.851083in}}%
\pgfpathlineto{\pgfqpoint{5.475108in}{2.857702in}}%
\pgfpathlineto{\pgfqpoint{5.489435in}{2.864420in}}%
\pgfpathlineto{\pgfqpoint{5.482022in}{2.856473in}}%
\pgfpathlineto{\pgfqpoint{5.474602in}{2.848427in}}%
\pgfpathlineto{\pgfqpoint{5.467174in}{2.840281in}}%
\pgfpathlineto{\pgfqpoint{5.459738in}{2.832035in}}%
\pgfpathlineto{\pgfqpoint{5.445402in}{2.825285in}}%
\pgfpathlineto{\pgfqpoint{5.431081in}{2.818636in}}%
\pgfpathlineto{\pgfqpoint{5.416775in}{2.812086in}}%
\pgfpathlineto{\pgfqpoint{5.402483in}{2.805637in}}%
\pgfpathlineto{\pgfqpoint{5.409928in}{2.813907in}}%
\pgfpathlineto{\pgfqpoint{5.417365in}{2.822081in}}%
\pgfpathlineto{\pgfqpoint{5.424795in}{2.830161in}}%
\pgfpathlineto{\pgfqpoint{5.432217in}{2.838146in}}%
\pgfpathclose%
\pgfusepath{fill}%
\end{pgfscope}%
\begin{pgfscope}%
\pgfpathrectangle{\pgfqpoint{1.150000in}{0.150000in}}{\pgfqpoint{5.700000in}{5.700000in}}%
\pgfusepath{clip}%
\pgfsetbuttcap%
\pgfsetroundjoin%
\definecolor{currentfill}{rgb}{0.267004,0.004874,0.329415}%
\pgfsetfillcolor{currentfill}%
\pgfsetfillopacity{0.700000}%
\pgfsetlinewidth{0.000000pt}%
\definecolor{currentstroke}{rgb}{0.000000,0.000000,0.000000}%
\pgfsetstrokecolor{currentstroke}%
\pgfsetdash{}{0pt}%
\pgfpathmoveto{\pgfqpoint{3.647263in}{1.995784in}}%
\pgfpathlineto{\pgfqpoint{3.660894in}{1.990539in}}%
\pgfpathlineto{\pgfqpoint{3.674529in}{1.985409in}}%
\pgfpathlineto{\pgfqpoint{3.688169in}{1.980393in}}%
\pgfpathlineto{\pgfqpoint{3.701813in}{1.975491in}}%
\pgfpathlineto{\pgfqpoint{3.693738in}{1.968550in}}%
\pgfpathlineto{\pgfqpoint{3.685656in}{1.961707in}}%
\pgfpathlineto{\pgfqpoint{3.677566in}{1.954964in}}%
\pgfpathlineto{\pgfqpoint{3.669469in}{1.948326in}}%
\pgfpathlineto{\pgfqpoint{3.655807in}{1.953565in}}%
\pgfpathlineto{\pgfqpoint{3.642150in}{1.958918in}}%
\pgfpathlineto{\pgfqpoint{3.628497in}{1.964385in}}%
\pgfpathlineto{\pgfqpoint{3.614848in}{1.969966in}}%
\pgfpathlineto{\pgfqpoint{3.622963in}{1.976260in}}%
\pgfpathlineto{\pgfqpoint{3.631070in}{1.982663in}}%
\pgfpathlineto{\pgfqpoint{3.639170in}{1.989172in}}%
\pgfpathlineto{\pgfqpoint{3.647263in}{1.995784in}}%
\pgfpathclose%
\pgfusepath{fill}%
\end{pgfscope}%
\begin{pgfscope}%
\pgfpathrectangle{\pgfqpoint{1.150000in}{0.150000in}}{\pgfqpoint{5.700000in}{5.700000in}}%
\pgfusepath{clip}%
\pgfsetbuttcap%
\pgfsetroundjoin%
\definecolor{currentfill}{rgb}{0.271305,0.019942,0.347269}%
\pgfsetfillcolor{currentfill}%
\pgfsetfillopacity{0.700000}%
\pgfsetlinewidth{0.000000pt}%
\definecolor{currentstroke}{rgb}{0.000000,0.000000,0.000000}%
\pgfsetstrokecolor{currentstroke}%
\pgfsetdash{}{0pt}%
\pgfpathmoveto{\pgfqpoint{3.505805in}{2.018808in}}%
\pgfpathlineto{\pgfqpoint{3.519422in}{2.012291in}}%
\pgfpathlineto{\pgfqpoint{3.533043in}{2.005892in}}%
\pgfpathlineto{\pgfqpoint{3.546667in}{1.999612in}}%
\pgfpathlineto{\pgfqpoint{3.560295in}{1.993449in}}%
\pgfpathlineto{\pgfqpoint{3.552153in}{1.987616in}}%
\pgfpathlineto{\pgfqpoint{3.544003in}{1.981904in}}%
\pgfpathlineto{\pgfqpoint{3.535845in}{1.976315in}}%
\pgfpathlineto{\pgfqpoint{3.527679in}{1.970853in}}%
\pgfpathlineto{\pgfqpoint{3.514030in}{1.977372in}}%
\pgfpathlineto{\pgfqpoint{3.500385in}{1.984008in}}%
\pgfpathlineto{\pgfqpoint{3.486744in}{1.990763in}}%
\pgfpathlineto{\pgfqpoint{3.473105in}{1.997636in}}%
\pgfpathlineto{\pgfqpoint{3.481293in}{2.002735in}}%
\pgfpathlineto{\pgfqpoint{3.489472in}{2.007965in}}%
\pgfpathlineto{\pgfqpoint{3.497643in}{2.013324in}}%
\pgfpathlineto{\pgfqpoint{3.505805in}{2.018808in}}%
\pgfpathclose%
\pgfusepath{fill}%
\end{pgfscope}%
\begin{pgfscope}%
\pgfpathrectangle{\pgfqpoint{1.150000in}{0.150000in}}{\pgfqpoint{5.700000in}{5.700000in}}%
\pgfusepath{clip}%
\pgfsetbuttcap%
\pgfsetroundjoin%
\definecolor{currentfill}{rgb}{0.250425,0.274290,0.533103}%
\pgfsetfillcolor{currentfill}%
\pgfsetfillopacity{0.700000}%
\pgfsetlinewidth{0.000000pt}%
\definecolor{currentstroke}{rgb}{0.000000,0.000000,0.000000}%
\pgfsetstrokecolor{currentstroke}%
\pgfsetdash{}{0pt}%
\pgfpathmoveto{\pgfqpoint{4.968368in}{2.514295in}}%
\pgfpathlineto{\pgfqpoint{4.982428in}{2.518611in}}%
\pgfpathlineto{\pgfqpoint{4.996499in}{2.523027in}}%
\pgfpathlineto{\pgfqpoint{5.010584in}{2.527544in}}%
\pgfpathlineto{\pgfqpoint{5.024681in}{2.532161in}}%
\pgfpathlineto{\pgfqpoint{5.017061in}{2.522135in}}%
\pgfpathlineto{\pgfqpoint{5.009436in}{2.512032in}}%
\pgfpathlineto{\pgfqpoint{5.001804in}{2.501851in}}%
\pgfpathlineto{\pgfqpoint{4.994166in}{2.491593in}}%
\pgfpathlineto{\pgfqpoint{4.980063in}{2.487059in}}%
\pgfpathlineto{\pgfqpoint{4.965974in}{2.482626in}}%
\pgfpathlineto{\pgfqpoint{4.951896in}{2.478292in}}%
\pgfpathlineto{\pgfqpoint{4.937831in}{2.474059in}}%
\pgfpathlineto{\pgfqpoint{4.945475in}{2.484227in}}%
\pgfpathlineto{\pgfqpoint{4.953112in}{2.494322in}}%
\pgfpathlineto{\pgfqpoint{4.960743in}{2.504345in}}%
\pgfpathlineto{\pgfqpoint{4.968368in}{2.514295in}}%
\pgfpathclose%
\pgfusepath{fill}%
\end{pgfscope}%
\begin{pgfscope}%
\pgfpathrectangle{\pgfqpoint{1.150000in}{0.150000in}}{\pgfqpoint{5.700000in}{5.700000in}}%
\pgfusepath{clip}%
\pgfsetbuttcap%
\pgfsetroundjoin%
\definecolor{currentfill}{rgb}{0.267004,0.004874,0.329415}%
\pgfsetfillcolor{currentfill}%
\pgfsetfillopacity{0.700000}%
\pgfsetlinewidth{0.000000pt}%
\definecolor{currentstroke}{rgb}{0.000000,0.000000,0.000000}%
\pgfsetstrokecolor{currentstroke}%
\pgfsetdash{}{0pt}%
\pgfpathmoveto{\pgfqpoint{3.788610in}{1.986968in}}%
\pgfpathlineto{\pgfqpoint{3.802265in}{1.982946in}}%
\pgfpathlineto{\pgfqpoint{3.815925in}{1.979036in}}%
\pgfpathlineto{\pgfqpoint{3.829591in}{1.975236in}}%
\pgfpathlineto{\pgfqpoint{3.843263in}{1.971547in}}%
\pgfpathlineto{\pgfqpoint{3.835246in}{1.963622in}}%
\pgfpathlineto{\pgfqpoint{3.827223in}{1.955774in}}%
\pgfpathlineto{\pgfqpoint{3.819193in}{1.948005in}}%
\pgfpathlineto{\pgfqpoint{3.811156in}{1.940319in}}%
\pgfpathlineto{\pgfqpoint{3.797469in}{1.944326in}}%
\pgfpathlineto{\pgfqpoint{3.783788in}{1.948444in}}%
\pgfpathlineto{\pgfqpoint{3.770113in}{1.952672in}}%
\pgfpathlineto{\pgfqpoint{3.756442in}{1.957012in}}%
\pgfpathlineto{\pgfqpoint{3.764494in}{1.964373in}}%
\pgfpathlineto{\pgfqpoint{3.772539in}{1.971822in}}%
\pgfpathlineto{\pgfqpoint{3.780578in}{1.979354in}}%
\pgfpathlineto{\pgfqpoint{3.788610in}{1.986968in}}%
\pgfpathclose%
\pgfusepath{fill}%
\end{pgfscope}%
\begin{pgfscope}%
\pgfpathrectangle{\pgfqpoint{1.150000in}{0.150000in}}{\pgfqpoint{5.700000in}{5.700000in}}%
\pgfusepath{clip}%
\pgfsetbuttcap%
\pgfsetroundjoin%
\definecolor{currentfill}{rgb}{0.271305,0.019942,0.347269}%
\pgfsetfillcolor{currentfill}%
\pgfsetfillopacity{0.700000}%
\pgfsetlinewidth{0.000000pt}%
\definecolor{currentstroke}{rgb}{0.000000,0.000000,0.000000}%
\pgfsetstrokecolor{currentstroke}%
\pgfsetdash{}{0pt}%
\pgfpathmoveto{\pgfqpoint{4.016570in}{2.016428in}}%
\pgfpathlineto{\pgfqpoint{4.030274in}{2.014299in}}%
\pgfpathlineto{\pgfqpoint{4.043986in}{2.012277in}}%
\pgfpathlineto{\pgfqpoint{4.057705in}{2.010361in}}%
\pgfpathlineto{\pgfqpoint{4.071432in}{2.008553in}}%
\pgfpathlineto{\pgfqpoint{4.063499in}{1.999293in}}%
\pgfpathlineto{\pgfqpoint{4.055561in}{1.990074in}}%
\pgfpathlineto{\pgfqpoint{4.047617in}{1.980898in}}%
\pgfpathlineto{\pgfqpoint{4.039668in}{1.971768in}}%
\pgfpathlineto{\pgfqpoint{4.025930in}{1.973859in}}%
\pgfpathlineto{\pgfqpoint{4.012200in}{1.976056in}}%
\pgfpathlineto{\pgfqpoint{3.998477in}{1.978360in}}%
\pgfpathlineto{\pgfqpoint{3.984761in}{1.980772in}}%
\pgfpathlineto{\pgfqpoint{3.992722in}{1.989613in}}%
\pgfpathlineto{\pgfqpoint{4.000677in}{1.998505in}}%
\pgfpathlineto{\pgfqpoint{4.008626in}{2.007444in}}%
\pgfpathlineto{\pgfqpoint{4.016570in}{2.016428in}}%
\pgfpathclose%
\pgfusepath{fill}%
\end{pgfscope}%
\begin{pgfscope}%
\pgfpathrectangle{\pgfqpoint{1.150000in}{0.150000in}}{\pgfqpoint{5.700000in}{5.700000in}}%
\pgfusepath{clip}%
\pgfsetbuttcap%
\pgfsetroundjoin%
\definecolor{currentfill}{rgb}{0.180629,0.429975,0.557282}%
\pgfsetfillcolor{currentfill}%
\pgfsetfillopacity{0.700000}%
\pgfsetlinewidth{0.000000pt}%
\definecolor{currentstroke}{rgb}{0.000000,0.000000,0.000000}%
\pgfsetstrokecolor{currentstroke}%
\pgfsetdash{}{0pt}%
\pgfpathmoveto{\pgfqpoint{5.519005in}{2.895231in}}%
\pgfpathlineto{\pgfqpoint{5.533336in}{2.901998in}}%
\pgfpathlineto{\pgfqpoint{5.547681in}{2.908865in}}%
\pgfpathlineto{\pgfqpoint{5.562042in}{2.915832in}}%
\pgfpathlineto{\pgfqpoint{5.576417in}{2.922898in}}%
\pgfpathlineto{\pgfqpoint{5.569047in}{2.915398in}}%
\pgfpathlineto{\pgfqpoint{5.561669in}{2.907797in}}%
\pgfpathlineto{\pgfqpoint{5.554282in}{2.900095in}}%
\pgfpathlineto{\pgfqpoint{5.546887in}{2.892290in}}%
\pgfpathlineto{\pgfqpoint{5.532502in}{2.885173in}}%
\pgfpathlineto{\pgfqpoint{5.518131in}{2.878155in}}%
\pgfpathlineto{\pgfqpoint{5.503776in}{2.871238in}}%
\pgfpathlineto{\pgfqpoint{5.489435in}{2.864420in}}%
\pgfpathlineto{\pgfqpoint{5.496839in}{2.872268in}}%
\pgfpathlineto{\pgfqpoint{5.504236in}{2.880019in}}%
\pgfpathlineto{\pgfqpoint{5.511624in}{2.887673in}}%
\pgfpathlineto{\pgfqpoint{5.519005in}{2.895231in}}%
\pgfpathclose%
\pgfusepath{fill}%
\end{pgfscope}%
\begin{pgfscope}%
\pgfpathrectangle{\pgfqpoint{1.150000in}{0.150000in}}{\pgfqpoint{5.700000in}{5.700000in}}%
\pgfusepath{clip}%
\pgfsetbuttcap%
\pgfsetroundjoin%
\definecolor{currentfill}{rgb}{0.283197,0.115680,0.436115}%
\pgfsetfillcolor{currentfill}%
\pgfsetfillopacity{0.700000}%
\pgfsetlinewidth{0.000000pt}%
\definecolor{currentstroke}{rgb}{0.000000,0.000000,0.000000}%
\pgfsetstrokecolor{currentstroke}%
\pgfsetdash{}{0pt}%
\pgfpathmoveto{\pgfqpoint{3.113065in}{2.189954in}}%
\pgfpathlineto{\pgfqpoint{3.126683in}{2.179652in}}%
\pgfpathlineto{\pgfqpoint{3.140303in}{2.169486in}}%
\pgfpathlineto{\pgfqpoint{3.153922in}{2.159455in}}%
\pgfpathlineto{\pgfqpoint{3.167543in}{2.149559in}}%
\pgfpathlineto{\pgfqpoint{3.159181in}{2.147119in}}%
\pgfpathlineto{\pgfqpoint{3.150808in}{2.144857in}}%
\pgfpathlineto{\pgfqpoint{3.142423in}{2.142777in}}%
\pgfpathlineto{\pgfqpoint{3.134027in}{2.140883in}}%
\pgfpathlineto{\pgfqpoint{3.120377in}{2.151178in}}%
\pgfpathlineto{\pgfqpoint{3.106728in}{2.161608in}}%
\pgfpathlineto{\pgfqpoint{3.093079in}{2.172173in}}%
\pgfpathlineto{\pgfqpoint{3.079430in}{2.182874in}}%
\pgfpathlineto{\pgfqpoint{3.087856in}{2.184361in}}%
\pgfpathlineto{\pgfqpoint{3.096271in}{2.186040in}}%
\pgfpathlineto{\pgfqpoint{3.104673in}{2.187905in}}%
\pgfpathlineto{\pgfqpoint{3.113065in}{2.189954in}}%
\pgfpathclose%
\pgfusepath{fill}%
\end{pgfscope}%
\begin{pgfscope}%
\pgfpathrectangle{\pgfqpoint{1.150000in}{0.150000in}}{\pgfqpoint{5.700000in}{5.700000in}}%
\pgfusepath{clip}%
\pgfsetbuttcap%
\pgfsetroundjoin%
\definecolor{currentfill}{rgb}{0.239346,0.300855,0.540844}%
\pgfsetfillcolor{currentfill}%
\pgfsetfillopacity{0.700000}%
\pgfsetlinewidth{0.000000pt}%
\definecolor{currentstroke}{rgb}{0.000000,0.000000,0.000000}%
\pgfsetstrokecolor{currentstroke}%
\pgfsetdash{}{0pt}%
\pgfpathmoveto{\pgfqpoint{5.055094in}{2.571481in}}%
\pgfpathlineto{\pgfqpoint{5.069198in}{2.576263in}}%
\pgfpathlineto{\pgfqpoint{5.083315in}{2.581145in}}%
\pgfpathlineto{\pgfqpoint{5.097444in}{2.586128in}}%
\pgfpathlineto{\pgfqpoint{5.111586in}{2.591210in}}%
\pgfpathlineto{\pgfqpoint{5.103999in}{2.581441in}}%
\pgfpathlineto{\pgfqpoint{5.096404in}{2.571588in}}%
\pgfpathlineto{\pgfqpoint{5.088803in}{2.561652in}}%
\pgfpathlineto{\pgfqpoint{5.081196in}{2.551632in}}%
\pgfpathlineto{\pgfqpoint{5.067048in}{2.546614in}}%
\pgfpathlineto{\pgfqpoint{5.052912in}{2.541696in}}%
\pgfpathlineto{\pgfqpoint{5.038790in}{2.536879in}}%
\pgfpathlineto{\pgfqpoint{5.024681in}{2.532161in}}%
\pgfpathlineto{\pgfqpoint{5.032294in}{2.542109in}}%
\pgfpathlineto{\pgfqpoint{5.039900in}{2.551979in}}%
\pgfpathlineto{\pgfqpoint{5.047501in}{2.561770in}}%
\pgfpathlineto{\pgfqpoint{5.055094in}{2.571481in}}%
\pgfpathclose%
\pgfusepath{fill}%
\end{pgfscope}%
\begin{pgfscope}%
\pgfpathrectangle{\pgfqpoint{1.150000in}{0.150000in}}{\pgfqpoint{5.700000in}{5.700000in}}%
\pgfusepath{clip}%
\pgfsetbuttcap%
\pgfsetroundjoin%
\definecolor{currentfill}{rgb}{0.171176,0.452530,0.557965}%
\pgfsetfillcolor{currentfill}%
\pgfsetfillopacity{0.700000}%
\pgfsetlinewidth{0.000000pt}%
\definecolor{currentstroke}{rgb}{0.000000,0.000000,0.000000}%
\pgfsetstrokecolor{currentstroke}%
\pgfsetdash{}{0pt}%
\pgfpathmoveto{\pgfqpoint{5.605816in}{2.951909in}}%
\pgfpathlineto{\pgfqpoint{5.620195in}{2.959005in}}%
\pgfpathlineto{\pgfqpoint{5.634589in}{2.966201in}}%
\pgfpathlineto{\pgfqpoint{5.648999in}{2.973497in}}%
\pgfpathlineto{\pgfqpoint{5.663424in}{2.980892in}}%
\pgfpathlineto{\pgfqpoint{5.656098in}{2.973863in}}%
\pgfpathlineto{\pgfqpoint{5.648764in}{2.966732in}}%
\pgfpathlineto{\pgfqpoint{5.641421in}{2.959499in}}%
\pgfpathlineto{\pgfqpoint{5.634070in}{2.952162in}}%
\pgfpathlineto{\pgfqpoint{5.619634in}{2.944696in}}%
\pgfpathlineto{\pgfqpoint{5.605213in}{2.937331in}}%
\pgfpathlineto{\pgfqpoint{5.590808in}{2.930065in}}%
\pgfpathlineto{\pgfqpoint{5.576417in}{2.922898in}}%
\pgfpathlineto{\pgfqpoint{5.583779in}{2.930299in}}%
\pgfpathlineto{\pgfqpoint{5.591133in}{2.937600in}}%
\pgfpathlineto{\pgfqpoint{5.598478in}{2.944803in}}%
\pgfpathlineto{\pgfqpoint{5.605816in}{2.951909in}}%
\pgfpathclose%
\pgfusepath{fill}%
\end{pgfscope}%
\begin{pgfscope}%
\pgfpathrectangle{\pgfqpoint{1.150000in}{0.150000in}}{\pgfqpoint{5.700000in}{5.700000in}}%
\pgfusepath{clip}%
\pgfsetbuttcap%
\pgfsetroundjoin%
\definecolor{currentfill}{rgb}{0.276022,0.044167,0.370164}%
\pgfsetfillcolor{currentfill}%
\pgfsetfillopacity{0.700000}%
\pgfsetlinewidth{0.000000pt}%
\definecolor{currentstroke}{rgb}{0.000000,0.000000,0.000000}%
\pgfsetstrokecolor{currentstroke}%
\pgfsetdash{}{0pt}%
\pgfpathmoveto{\pgfqpoint{3.364108in}{2.056975in}}%
\pgfpathlineto{\pgfqpoint{3.377723in}{2.049129in}}%
\pgfpathlineto{\pgfqpoint{3.391341in}{2.041407in}}%
\pgfpathlineto{\pgfqpoint{3.404961in}{2.033808in}}%
\pgfpathlineto{\pgfqpoint{3.418584in}{2.026331in}}%
\pgfpathlineto{\pgfqpoint{3.410365in}{2.021736in}}%
\pgfpathlineto{\pgfqpoint{3.402138in}{2.017286in}}%
\pgfpathlineto{\pgfqpoint{3.393901in}{2.012982in}}%
\pgfpathlineto{\pgfqpoint{3.385654in}{2.008830in}}%
\pgfpathlineto{\pgfqpoint{3.372007in}{2.016683in}}%
\pgfpathlineto{\pgfqpoint{3.358363in}{2.024658in}}%
\pgfpathlineto{\pgfqpoint{3.344721in}{2.032756in}}%
\pgfpathlineto{\pgfqpoint{3.331082in}{2.040979in}}%
\pgfpathlineto{\pgfqpoint{3.339353in}{2.044747in}}%
\pgfpathlineto{\pgfqpoint{3.347614in}{2.048672in}}%
\pgfpathlineto{\pgfqpoint{3.355866in}{2.052750in}}%
\pgfpathlineto{\pgfqpoint{3.364108in}{2.056975in}}%
\pgfpathclose%
\pgfusepath{fill}%
\end{pgfscope}%
\begin{pgfscope}%
\pgfpathrectangle{\pgfqpoint{1.150000in}{0.150000in}}{\pgfqpoint{5.700000in}{5.700000in}}%
\pgfusepath{clip}%
\pgfsetbuttcap%
\pgfsetroundjoin%
\definecolor{currentfill}{rgb}{0.235526,0.309527,0.542944}%
\pgfsetfillcolor{currentfill}%
\pgfsetfillopacity{0.700000}%
\pgfsetlinewidth{0.000000pt}%
\definecolor{currentstroke}{rgb}{0.000000,0.000000,0.000000}%
\pgfsetstrokecolor{currentstroke}%
\pgfsetdash{}{0pt}%
\pgfpathmoveto{\pgfqpoint{2.641932in}{2.603193in}}%
\pgfpathlineto{\pgfqpoint{2.655649in}{2.587581in}}%
\pgfpathlineto{\pgfqpoint{2.669362in}{2.572143in}}%
\pgfpathlineto{\pgfqpoint{2.683070in}{2.556875in}}%
\pgfpathlineto{\pgfqpoint{2.696774in}{2.541778in}}%
\pgfpathlineto{\pgfqpoint{2.688086in}{2.543411in}}%
\pgfpathlineto{\pgfqpoint{2.679382in}{2.545280in}}%
\pgfpathlineto{\pgfqpoint{2.670663in}{2.547388in}}%
\pgfpathlineto{\pgfqpoint{2.661927in}{2.549740in}}%
\pgfpathlineto{\pgfqpoint{2.648182in}{2.565271in}}%
\pgfpathlineto{\pgfqpoint{2.634432in}{2.580972in}}%
\pgfpathlineto{\pgfqpoint{2.620678in}{2.596846in}}%
\pgfpathlineto{\pgfqpoint{2.606919in}{2.612893in}}%
\pgfpathlineto{\pgfqpoint{2.615697in}{2.610099in}}%
\pgfpathlineto{\pgfqpoint{2.624459in}{2.607554in}}%
\pgfpathlineto{\pgfqpoint{2.633204in}{2.605253in}}%
\pgfpathlineto{\pgfqpoint{2.641932in}{2.603193in}}%
\pgfpathclose%
\pgfusepath{fill}%
\end{pgfscope}%
\begin{pgfscope}%
\pgfpathrectangle{\pgfqpoint{1.150000in}{0.150000in}}{\pgfqpoint{5.700000in}{5.700000in}}%
\pgfusepath{clip}%
\pgfsetbuttcap%
\pgfsetroundjoin%
\definecolor{currentfill}{rgb}{0.223925,0.334994,0.548053}%
\pgfsetfillcolor{currentfill}%
\pgfsetfillopacity{0.700000}%
\pgfsetlinewidth{0.000000pt}%
\definecolor{currentstroke}{rgb}{0.000000,0.000000,0.000000}%
\pgfsetstrokecolor{currentstroke}%
\pgfsetdash{}{0pt}%
\pgfpathmoveto{\pgfqpoint{2.587016in}{2.667400in}}%
\pgfpathlineto{\pgfqpoint{2.600752in}{2.651081in}}%
\pgfpathlineto{\pgfqpoint{2.614484in}{2.634942in}}%
\pgfpathlineto{\pgfqpoint{2.628210in}{2.618979in}}%
\pgfpathlineto{\pgfqpoint{2.641932in}{2.603193in}}%
\pgfpathlineto{\pgfqpoint{2.633204in}{2.605253in}}%
\pgfpathlineto{\pgfqpoint{2.624459in}{2.607554in}}%
\pgfpathlineto{\pgfqpoint{2.615697in}{2.610099in}}%
\pgfpathlineto{\pgfqpoint{2.606919in}{2.612893in}}%
\pgfpathlineto{\pgfqpoint{2.593155in}{2.629115in}}%
\pgfpathlineto{\pgfqpoint{2.579386in}{2.645514in}}%
\pgfpathlineto{\pgfqpoint{2.565611in}{2.662091in}}%
\pgfpathlineto{\pgfqpoint{2.551832in}{2.678848in}}%
\pgfpathlineto{\pgfqpoint{2.560653in}{2.675608in}}%
\pgfpathlineto{\pgfqpoint{2.569458in}{2.672624in}}%
\pgfpathlineto{\pgfqpoint{2.578245in}{2.669889in}}%
\pgfpathlineto{\pgfqpoint{2.587016in}{2.667400in}}%
\pgfpathclose%
\pgfusepath{fill}%
\end{pgfscope}%
\begin{pgfscope}%
\pgfpathrectangle{\pgfqpoint{1.150000in}{0.150000in}}{\pgfqpoint{5.700000in}{5.700000in}}%
\pgfusepath{clip}%
\pgfsetbuttcap%
\pgfsetroundjoin%
\definecolor{currentfill}{rgb}{0.268510,0.009605,0.335427}%
\pgfsetfillcolor{currentfill}%
\pgfsetfillopacity{0.700000}%
\pgfsetlinewidth{0.000000pt}%
\definecolor{currentstroke}{rgb}{0.000000,0.000000,0.000000}%
\pgfsetstrokecolor{currentstroke}%
\pgfsetdash{}{0pt}%
\pgfpathmoveto{\pgfqpoint{3.929963in}{1.991493in}}%
\pgfpathlineto{\pgfqpoint{3.943653in}{1.988651in}}%
\pgfpathlineto{\pgfqpoint{3.957349in}{1.985917in}}%
\pgfpathlineto{\pgfqpoint{3.971051in}{1.983290in}}%
\pgfpathlineto{\pgfqpoint{3.984761in}{1.980772in}}%
\pgfpathlineto{\pgfqpoint{3.976794in}{1.971984in}}%
\pgfpathlineto{\pgfqpoint{3.968822in}{1.963251in}}%
\pgfpathlineto{\pgfqpoint{3.960844in}{1.954578in}}%
\pgfpathlineto{\pgfqpoint{3.952859in}{1.945966in}}%
\pgfpathlineto{\pgfqpoint{3.939138in}{1.948784in}}%
\pgfpathlineto{\pgfqpoint{3.925422in}{1.951711in}}%
\pgfpathlineto{\pgfqpoint{3.911714in}{1.954745in}}%
\pgfpathlineto{\pgfqpoint{3.898011in}{1.957888in}}%
\pgfpathlineto{\pgfqpoint{3.906008in}{1.966193in}}%
\pgfpathlineto{\pgfqpoint{3.913999in}{1.974564in}}%
\pgfpathlineto{\pgfqpoint{3.921984in}{1.982998in}}%
\pgfpathlineto{\pgfqpoint{3.929963in}{1.991493in}}%
\pgfpathclose%
\pgfusepath{fill}%
\end{pgfscope}%
\begin{pgfscope}%
\pgfpathrectangle{\pgfqpoint{1.150000in}{0.150000in}}{\pgfqpoint{5.700000in}{5.700000in}}%
\pgfusepath{clip}%
\pgfsetbuttcap%
\pgfsetroundjoin%
\definecolor{currentfill}{rgb}{0.246811,0.283237,0.535941}%
\pgfsetfillcolor{currentfill}%
\pgfsetfillopacity{0.700000}%
\pgfsetlinewidth{0.000000pt}%
\definecolor{currentstroke}{rgb}{0.000000,0.000000,0.000000}%
\pgfsetstrokecolor{currentstroke}%
\pgfsetdash{}{0pt}%
\pgfpathmoveto{\pgfqpoint{2.696774in}{2.541778in}}%
\pgfpathlineto{\pgfqpoint{2.710474in}{2.526849in}}%
\pgfpathlineto{\pgfqpoint{2.724170in}{2.512087in}}%
\pgfpathlineto{\pgfqpoint{2.737863in}{2.497491in}}%
\pgfpathlineto{\pgfqpoint{2.751552in}{2.483059in}}%
\pgfpathlineto{\pgfqpoint{2.742903in}{2.484269in}}%
\pgfpathlineto{\pgfqpoint{2.734240in}{2.485709in}}%
\pgfpathlineto{\pgfqpoint{2.725560in}{2.487382in}}%
\pgfpathlineto{\pgfqpoint{2.716865in}{2.489295in}}%
\pgfpathlineto{\pgfqpoint{2.703136in}{2.504157in}}%
\pgfpathlineto{\pgfqpoint{2.689404in}{2.519185in}}%
\pgfpathlineto{\pgfqpoint{2.675667in}{2.534378in}}%
\pgfpathlineto{\pgfqpoint{2.661927in}{2.549740in}}%
\pgfpathlineto{\pgfqpoint{2.670663in}{2.547388in}}%
\pgfpathlineto{\pgfqpoint{2.679382in}{2.545280in}}%
\pgfpathlineto{\pgfqpoint{2.688086in}{2.543411in}}%
\pgfpathlineto{\pgfqpoint{2.696774in}{2.541778in}}%
\pgfpathclose%
\pgfusepath{fill}%
\end{pgfscope}%
\begin{pgfscope}%
\pgfpathrectangle{\pgfqpoint{1.150000in}{0.150000in}}{\pgfqpoint{5.700000in}{5.700000in}}%
\pgfusepath{clip}%
\pgfsetbuttcap%
\pgfsetroundjoin%
\definecolor{currentfill}{rgb}{0.210503,0.363727,0.552206}%
\pgfsetfillcolor{currentfill}%
\pgfsetfillopacity{0.700000}%
\pgfsetlinewidth{0.000000pt}%
\definecolor{currentstroke}{rgb}{0.000000,0.000000,0.000000}%
\pgfsetstrokecolor{currentstroke}%
\pgfsetdash{}{0pt}%
\pgfpathmoveto{\pgfqpoint{2.532013in}{2.734498in}}%
\pgfpathlineto{\pgfqpoint{2.545773in}{2.717446in}}%
\pgfpathlineto{\pgfqpoint{2.559526in}{2.700581in}}%
\pgfpathlineto{\pgfqpoint{2.573274in}{2.683899in}}%
\pgfpathlineto{\pgfqpoint{2.587016in}{2.667400in}}%
\pgfpathlineto{\pgfqpoint{2.578245in}{2.669889in}}%
\pgfpathlineto{\pgfqpoint{2.569458in}{2.672624in}}%
\pgfpathlineto{\pgfqpoint{2.560653in}{2.675608in}}%
\pgfpathlineto{\pgfqpoint{2.551832in}{2.678848in}}%
\pgfpathlineto{\pgfqpoint{2.538046in}{2.695786in}}%
\pgfpathlineto{\pgfqpoint{2.524254in}{2.712907in}}%
\pgfpathlineto{\pgfqpoint{2.510457in}{2.730213in}}%
\pgfpathlineto{\pgfqpoint{2.496653in}{2.747705in}}%
\pgfpathlineto{\pgfqpoint{2.505519in}{2.744018in}}%
\pgfpathlineto{\pgfqpoint{2.514368in}{2.740590in}}%
\pgfpathlineto{\pgfqpoint{2.523200in}{2.737418in}}%
\pgfpathlineto{\pgfqpoint{2.532013in}{2.734498in}}%
\pgfpathclose%
\pgfusepath{fill}%
\end{pgfscope}%
\begin{pgfscope}%
\pgfpathrectangle{\pgfqpoint{1.150000in}{0.150000in}}{\pgfqpoint{5.700000in}{5.700000in}}%
\pgfusepath{clip}%
\pgfsetbuttcap%
\pgfsetroundjoin%
\definecolor{currentfill}{rgb}{0.162142,0.474838,0.558140}%
\pgfsetfillcolor{currentfill}%
\pgfsetfillopacity{0.700000}%
\pgfsetlinewidth{0.000000pt}%
\definecolor{currentstroke}{rgb}{0.000000,0.000000,0.000000}%
\pgfsetstrokecolor{currentstroke}%
\pgfsetdash{}{0pt}%
\pgfpathmoveto{\pgfqpoint{5.692642in}{3.008013in}}%
\pgfpathlineto{\pgfqpoint{5.707070in}{3.015418in}}%
\pgfpathlineto{\pgfqpoint{5.721513in}{3.022923in}}%
\pgfpathlineto{\pgfqpoint{5.735972in}{3.030527in}}%
\pgfpathlineto{\pgfqpoint{5.750447in}{3.038231in}}%
\pgfpathlineto{\pgfqpoint{5.743168in}{3.031695in}}%
\pgfpathlineto{\pgfqpoint{5.735880in}{3.025056in}}%
\pgfpathlineto{\pgfqpoint{5.728584in}{3.018315in}}%
\pgfpathlineto{\pgfqpoint{5.721279in}{3.011470in}}%
\pgfpathlineto{\pgfqpoint{5.706791in}{3.003676in}}%
\pgfpathlineto{\pgfqpoint{5.692320in}{2.995981in}}%
\pgfpathlineto{\pgfqpoint{5.677864in}{2.988387in}}%
\pgfpathlineto{\pgfqpoint{5.663424in}{2.980892in}}%
\pgfpathlineto{\pgfqpoint{5.670741in}{2.987820in}}%
\pgfpathlineto{\pgfqpoint{5.678050in}{2.994649in}}%
\pgfpathlineto{\pgfqpoint{5.685350in}{3.001379in}}%
\pgfpathlineto{\pgfqpoint{5.692642in}{3.008013in}}%
\pgfpathclose%
\pgfusepath{fill}%
\end{pgfscope}%
\begin{pgfscope}%
\pgfpathrectangle{\pgfqpoint{1.150000in}{0.150000in}}{\pgfqpoint{5.700000in}{5.700000in}}%
\pgfusepath{clip}%
\pgfsetbuttcap%
\pgfsetroundjoin%
\definecolor{currentfill}{rgb}{0.227802,0.326594,0.546532}%
\pgfsetfillcolor{currentfill}%
\pgfsetfillopacity{0.700000}%
\pgfsetlinewidth{0.000000pt}%
\definecolor{currentstroke}{rgb}{0.000000,0.000000,0.000000}%
\pgfsetstrokecolor{currentstroke}%
\pgfsetdash{}{0pt}%
\pgfpathmoveto{\pgfqpoint{5.141871in}{2.629442in}}%
\pgfpathlineto{\pgfqpoint{5.156020in}{2.634670in}}%
\pgfpathlineto{\pgfqpoint{5.170183in}{2.639998in}}%
\pgfpathlineto{\pgfqpoint{5.184359in}{2.645426in}}%
\pgfpathlineto{\pgfqpoint{5.198548in}{2.650955in}}%
\pgfpathlineto{\pgfqpoint{5.190993in}{2.641485in}}%
\pgfpathlineto{\pgfqpoint{5.183432in}{2.631926in}}%
\pgfpathlineto{\pgfqpoint{5.175863in}{2.622278in}}%
\pgfpathlineto{\pgfqpoint{5.168288in}{2.612541in}}%
\pgfpathlineto{\pgfqpoint{5.154093in}{2.607058in}}%
\pgfpathlineto{\pgfqpoint{5.139911in}{2.601675in}}%
\pgfpathlineto{\pgfqpoint{5.125742in}{2.596393in}}%
\pgfpathlineto{\pgfqpoint{5.111586in}{2.591210in}}%
\pgfpathlineto{\pgfqpoint{5.119168in}{2.600895in}}%
\pgfpathlineto{\pgfqpoint{5.126742in}{2.610495in}}%
\pgfpathlineto{\pgfqpoint{5.134310in}{2.620011in}}%
\pgfpathlineto{\pgfqpoint{5.141871in}{2.629442in}}%
\pgfpathclose%
\pgfusepath{fill}%
\end{pgfscope}%
\begin{pgfscope}%
\pgfpathrectangle{\pgfqpoint{1.150000in}{0.150000in}}{\pgfqpoint{5.700000in}{5.700000in}}%
\pgfusepath{clip}%
\pgfsetbuttcap%
\pgfsetroundjoin%
\definecolor{currentfill}{rgb}{0.282910,0.105393,0.426902}%
\pgfsetfillcolor{currentfill}%
\pgfsetfillopacity{0.700000}%
\pgfsetlinewidth{0.000000pt}%
\definecolor{currentstroke}{rgb}{0.000000,0.000000,0.000000}%
\pgfsetstrokecolor{currentstroke}%
\pgfsetdash{}{0pt}%
\pgfpathmoveto{\pgfqpoint{4.417766in}{2.160809in}}%
\pgfpathlineto{\pgfqpoint{4.431602in}{2.161724in}}%
\pgfpathlineto{\pgfqpoint{4.445448in}{2.162742in}}%
\pgfpathlineto{\pgfqpoint{4.459303in}{2.163862in}}%
\pgfpathlineto{\pgfqpoint{4.473168in}{2.165084in}}%
\pgfpathlineto{\pgfqpoint{4.465360in}{2.154473in}}%
\pgfpathlineto{\pgfqpoint{4.457547in}{2.143844in}}%
\pgfpathlineto{\pgfqpoint{4.449729in}{2.133200in}}%
\pgfpathlineto{\pgfqpoint{4.441905in}{2.122542in}}%
\pgfpathlineto{\pgfqpoint{4.428034in}{2.121530in}}%
\pgfpathlineto{\pgfqpoint{4.414172in}{2.120621in}}%
\pgfpathlineto{\pgfqpoint{4.400320in}{2.119814in}}%
\pgfpathlineto{\pgfqpoint{4.386477in}{2.119109in}}%
\pgfpathlineto{\pgfqpoint{4.394307in}{2.129550in}}%
\pgfpathlineto{\pgfqpoint{4.402132in}{2.139981in}}%
\pgfpathlineto{\pgfqpoint{4.409952in}{2.150401in}}%
\pgfpathlineto{\pgfqpoint{4.417766in}{2.160809in}}%
\pgfpathclose%
\pgfusepath{fill}%
\end{pgfscope}%
\begin{pgfscope}%
\pgfpathrectangle{\pgfqpoint{1.150000in}{0.150000in}}{\pgfqpoint{5.700000in}{5.700000in}}%
\pgfusepath{clip}%
\pgfsetbuttcap%
\pgfsetroundjoin%
\definecolor{currentfill}{rgb}{0.283072,0.130895,0.449241}%
\pgfsetfillcolor{currentfill}%
\pgfsetfillopacity{0.700000}%
\pgfsetlinewidth{0.000000pt}%
\definecolor{currentstroke}{rgb}{0.000000,0.000000,0.000000}%
\pgfsetstrokecolor{currentstroke}%
\pgfsetdash{}{0pt}%
\pgfpathmoveto{\pgfqpoint{4.504350in}{2.207323in}}%
\pgfpathlineto{\pgfqpoint{4.518219in}{2.208840in}}%
\pgfpathlineto{\pgfqpoint{4.532097in}{2.210460in}}%
\pgfpathlineto{\pgfqpoint{4.545986in}{2.212182in}}%
\pgfpathlineto{\pgfqpoint{4.559886in}{2.214006in}}%
\pgfpathlineto{\pgfqpoint{4.552104in}{2.203295in}}%
\pgfpathlineto{\pgfqpoint{4.544317in}{2.192555in}}%
\pgfpathlineto{\pgfqpoint{4.536525in}{2.181788in}}%
\pgfpathlineto{\pgfqpoint{4.528728in}{2.170995in}}%
\pgfpathlineto{\pgfqpoint{4.514823in}{2.169365in}}%
\pgfpathlineto{\pgfqpoint{4.500928in}{2.167836in}}%
\pgfpathlineto{\pgfqpoint{4.487043in}{2.166409in}}%
\pgfpathlineto{\pgfqpoint{4.473168in}{2.165084in}}%
\pgfpathlineto{\pgfqpoint{4.480972in}{2.175677in}}%
\pgfpathlineto{\pgfqpoint{4.488769in}{2.186249in}}%
\pgfpathlineto{\pgfqpoint{4.496562in}{2.196798in}}%
\pgfpathlineto{\pgfqpoint{4.504350in}{2.207323in}}%
\pgfpathclose%
\pgfusepath{fill}%
\end{pgfscope}%
\begin{pgfscope}%
\pgfpathrectangle{\pgfqpoint{1.150000in}{0.150000in}}{\pgfqpoint{5.700000in}{5.700000in}}%
\pgfusepath{clip}%
\pgfsetbuttcap%
\pgfsetroundjoin%
\definecolor{currentfill}{rgb}{0.257322,0.256130,0.526563}%
\pgfsetfillcolor{currentfill}%
\pgfsetfillopacity{0.700000}%
\pgfsetlinewidth{0.000000pt}%
\definecolor{currentstroke}{rgb}{0.000000,0.000000,0.000000}%
\pgfsetstrokecolor{currentstroke}%
\pgfsetdash{}{0pt}%
\pgfpathmoveto{\pgfqpoint{2.751552in}{2.483059in}}%
\pgfpathlineto{\pgfqpoint{2.765238in}{2.468791in}}%
\pgfpathlineto{\pgfqpoint{2.778921in}{2.454684in}}%
\pgfpathlineto{\pgfqpoint{2.792600in}{2.440738in}}%
\pgfpathlineto{\pgfqpoint{2.806277in}{2.426951in}}%
\pgfpathlineto{\pgfqpoint{2.797666in}{2.427739in}}%
\pgfpathlineto{\pgfqpoint{2.789041in}{2.428752in}}%
\pgfpathlineto{\pgfqpoint{2.780401in}{2.429993in}}%
\pgfpathlineto{\pgfqpoint{2.771745in}{2.431468in}}%
\pgfpathlineto{\pgfqpoint{2.758030in}{2.445684in}}%
\pgfpathlineto{\pgfqpoint{2.744312in}{2.460059in}}%
\pgfpathlineto{\pgfqpoint{2.730590in}{2.474596in}}%
\pgfpathlineto{\pgfqpoint{2.716865in}{2.489295in}}%
\pgfpathlineto{\pgfqpoint{2.725560in}{2.487382in}}%
\pgfpathlineto{\pgfqpoint{2.734240in}{2.485709in}}%
\pgfpathlineto{\pgfqpoint{2.742903in}{2.484269in}}%
\pgfpathlineto{\pgfqpoint{2.751552in}{2.483059in}}%
\pgfpathclose%
\pgfusepath{fill}%
\end{pgfscope}%
\begin{pgfscope}%
\pgfpathrectangle{\pgfqpoint{1.150000in}{0.150000in}}{\pgfqpoint{5.700000in}{5.700000in}}%
\pgfusepath{clip}%
\pgfsetbuttcap%
\pgfsetroundjoin%
\definecolor{currentfill}{rgb}{0.281446,0.084320,0.407414}%
\pgfsetfillcolor{currentfill}%
\pgfsetfillopacity{0.700000}%
\pgfsetlinewidth{0.000000pt}%
\definecolor{currentstroke}{rgb}{0.000000,0.000000,0.000000}%
\pgfsetstrokecolor{currentstroke}%
\pgfsetdash{}{0pt}%
\pgfpathmoveto{\pgfqpoint{4.331198in}{2.117322in}}%
\pgfpathlineto{\pgfqpoint{4.345004in}{2.117614in}}%
\pgfpathlineto{\pgfqpoint{4.358819in}{2.118009in}}%
\pgfpathlineto{\pgfqpoint{4.372643in}{2.118508in}}%
\pgfpathlineto{\pgfqpoint{4.386477in}{2.119109in}}%
\pgfpathlineto{\pgfqpoint{4.378642in}{2.108662in}}%
\pgfpathlineto{\pgfqpoint{4.370801in}{2.098210in}}%
\pgfpathlineto{\pgfqpoint{4.362956in}{2.087754in}}%
\pgfpathlineto{\pgfqpoint{4.355106in}{2.077298in}}%
\pgfpathlineto{\pgfqpoint{4.341265in}{2.076925in}}%
\pgfpathlineto{\pgfqpoint{4.327434in}{2.076656in}}%
\pgfpathlineto{\pgfqpoint{4.313611in}{2.076489in}}%
\pgfpathlineto{\pgfqpoint{4.299797in}{2.076425in}}%
\pgfpathlineto{\pgfqpoint{4.307655in}{2.086646in}}%
\pgfpathlineto{\pgfqpoint{4.315508in}{2.096871in}}%
\pgfpathlineto{\pgfqpoint{4.323355in}{2.107096in}}%
\pgfpathlineto{\pgfqpoint{4.331198in}{2.117322in}}%
\pgfpathclose%
\pgfusepath{fill}%
\end{pgfscope}%
\begin{pgfscope}%
\pgfpathrectangle{\pgfqpoint{1.150000in}{0.150000in}}{\pgfqpoint{5.700000in}{5.700000in}}%
\pgfusepath{clip}%
\pgfsetbuttcap%
\pgfsetroundjoin%
\definecolor{currentfill}{rgb}{0.197636,0.391528,0.554969}%
\pgfsetfillcolor{currentfill}%
\pgfsetfillopacity{0.700000}%
\pgfsetlinewidth{0.000000pt}%
\definecolor{currentstroke}{rgb}{0.000000,0.000000,0.000000}%
\pgfsetstrokecolor{currentstroke}%
\pgfsetdash{}{0pt}%
\pgfpathmoveto{\pgfqpoint{2.476913in}{2.804594in}}%
\pgfpathlineto{\pgfqpoint{2.490698in}{2.786782in}}%
\pgfpathlineto{\pgfqpoint{2.504476in}{2.769164in}}%
\pgfpathlineto{\pgfqpoint{2.518248in}{2.751736in}}%
\pgfpathlineto{\pgfqpoint{2.532013in}{2.734498in}}%
\pgfpathlineto{\pgfqpoint{2.523200in}{2.737418in}}%
\pgfpathlineto{\pgfqpoint{2.514368in}{2.740590in}}%
\pgfpathlineto{\pgfqpoint{2.505519in}{2.744018in}}%
\pgfpathlineto{\pgfqpoint{2.496653in}{2.747705in}}%
\pgfpathlineto{\pgfqpoint{2.482842in}{2.765385in}}%
\pgfpathlineto{\pgfqpoint{2.469025in}{2.783256in}}%
\pgfpathlineto{\pgfqpoint{2.455201in}{2.801318in}}%
\pgfpathlineto{\pgfqpoint{2.441370in}{2.819573in}}%
\pgfpathlineto{\pgfqpoint{2.450284in}{2.815434in}}%
\pgfpathlineto{\pgfqpoint{2.459178in}{2.811561in}}%
\pgfpathlineto{\pgfqpoint{2.468055in}{2.807949in}}%
\pgfpathlineto{\pgfqpoint{2.476913in}{2.804594in}}%
\pgfpathclose%
\pgfusepath{fill}%
\end{pgfscope}%
\begin{pgfscope}%
\pgfpathrectangle{\pgfqpoint{1.150000in}{0.150000in}}{\pgfqpoint{5.700000in}{5.700000in}}%
\pgfusepath{clip}%
\pgfsetbuttcap%
\pgfsetroundjoin%
\definecolor{currentfill}{rgb}{0.282656,0.100196,0.422160}%
\pgfsetfillcolor{currentfill}%
\pgfsetfillopacity{0.700000}%
\pgfsetlinewidth{0.000000pt}%
\definecolor{currentstroke}{rgb}{0.000000,0.000000,0.000000}%
\pgfsetstrokecolor{currentstroke}%
\pgfsetdash{}{0pt}%
\pgfpathmoveto{\pgfqpoint{3.167543in}{2.149559in}}%
\pgfpathlineto{\pgfqpoint{3.181164in}{2.139795in}}%
\pgfpathlineto{\pgfqpoint{3.194786in}{2.130165in}}%
\pgfpathlineto{\pgfqpoint{3.208409in}{2.120666in}}%
\pgfpathlineto{\pgfqpoint{3.222034in}{2.111298in}}%
\pgfpathlineto{\pgfqpoint{3.213700in}{2.108467in}}%
\pgfpathlineto{\pgfqpoint{3.205355in}{2.105809in}}%
\pgfpathlineto{\pgfqpoint{3.196999in}{2.103329in}}%
\pgfpathlineto{\pgfqpoint{3.188633in}{2.101030in}}%
\pgfpathlineto{\pgfqpoint{3.174980in}{2.110795in}}%
\pgfpathlineto{\pgfqpoint{3.161328in}{2.120692in}}%
\pgfpathlineto{\pgfqpoint{3.147677in}{2.130721in}}%
\pgfpathlineto{\pgfqpoint{3.134027in}{2.140883in}}%
\pgfpathlineto{\pgfqpoint{3.142423in}{2.142777in}}%
\pgfpathlineto{\pgfqpoint{3.150808in}{2.144857in}}%
\pgfpathlineto{\pgfqpoint{3.159181in}{2.147119in}}%
\pgfpathlineto{\pgfqpoint{3.167543in}{2.149559in}}%
\pgfpathclose%
\pgfusepath{fill}%
\end{pgfscope}%
\begin{pgfscope}%
\pgfpathrectangle{\pgfqpoint{1.150000in}{0.150000in}}{\pgfqpoint{5.700000in}{5.700000in}}%
\pgfusepath{clip}%
\pgfsetbuttcap%
\pgfsetroundjoin%
\definecolor{currentfill}{rgb}{0.281412,0.155834,0.469201}%
\pgfsetfillcolor{currentfill}%
\pgfsetfillopacity{0.700000}%
\pgfsetlinewidth{0.000000pt}%
\definecolor{currentstroke}{rgb}{0.000000,0.000000,0.000000}%
\pgfsetstrokecolor{currentstroke}%
\pgfsetdash{}{0pt}%
\pgfpathmoveto{\pgfqpoint{4.590960in}{2.256530in}}%
\pgfpathlineto{\pgfqpoint{4.604864in}{2.258631in}}%
\pgfpathlineto{\pgfqpoint{4.618779in}{2.260832in}}%
\pgfpathlineto{\pgfqpoint{4.632704in}{2.263136in}}%
\pgfpathlineto{\pgfqpoint{4.646639in}{2.265540in}}%
\pgfpathlineto{\pgfqpoint{4.638884in}{2.254792in}}%
\pgfpathlineto{\pgfqpoint{4.631123in}{2.244004in}}%
\pgfpathlineto{\pgfqpoint{4.623358in}{2.233178in}}%
\pgfpathlineto{\pgfqpoint{4.615587in}{2.222316in}}%
\pgfpathlineto{\pgfqpoint{4.601646in}{2.220086in}}%
\pgfpathlineto{\pgfqpoint{4.587715in}{2.217958in}}%
\pgfpathlineto{\pgfqpoint{4.573795in}{2.215931in}}%
\pgfpathlineto{\pgfqpoint{4.559886in}{2.214006in}}%
\pgfpathlineto{\pgfqpoint{4.567662in}{2.224686in}}%
\pgfpathlineto{\pgfqpoint{4.575433in}{2.235335in}}%
\pgfpathlineto{\pgfqpoint{4.583199in}{2.245950in}}%
\pgfpathlineto{\pgfqpoint{4.590960in}{2.256530in}}%
\pgfpathclose%
\pgfusepath{fill}%
\end{pgfscope}%
\begin{pgfscope}%
\pgfpathrectangle{\pgfqpoint{1.150000in}{0.150000in}}{\pgfqpoint{5.700000in}{5.700000in}}%
\pgfusepath{clip}%
\pgfsetbuttcap%
\pgfsetroundjoin%
\definecolor{currentfill}{rgb}{0.278791,0.062145,0.386592}%
\pgfsetfillcolor{currentfill}%
\pgfsetfillopacity{0.700000}%
\pgfsetlinewidth{0.000000pt}%
\definecolor{currentstroke}{rgb}{0.000000,0.000000,0.000000}%
\pgfsetstrokecolor{currentstroke}%
\pgfsetdash{}{0pt}%
\pgfpathmoveto{\pgfqpoint{4.244630in}{2.077208in}}%
\pgfpathlineto{\pgfqpoint{4.258409in}{2.076856in}}%
\pgfpathlineto{\pgfqpoint{4.272197in}{2.076609in}}%
\pgfpathlineto{\pgfqpoint{4.285993in}{2.076465in}}%
\pgfpathlineto{\pgfqpoint{4.299797in}{2.076425in}}%
\pgfpathlineto{\pgfqpoint{4.291935in}{2.066210in}}%
\pgfpathlineto{\pgfqpoint{4.284067in}{2.056003in}}%
\pgfpathlineto{\pgfqpoint{4.276193in}{2.045806in}}%
\pgfpathlineto{\pgfqpoint{4.268315in}{2.035621in}}%
\pgfpathlineto{\pgfqpoint{4.254502in}{2.035908in}}%
\pgfpathlineto{\pgfqpoint{4.240698in}{2.036298in}}%
\pgfpathlineto{\pgfqpoint{4.226903in}{2.036792in}}%
\pgfpathlineto{\pgfqpoint{4.213115in}{2.037390in}}%
\pgfpathlineto{\pgfqpoint{4.221002in}{2.047322in}}%
\pgfpathlineto{\pgfqpoint{4.228883in}{2.057270in}}%
\pgfpathlineto{\pgfqpoint{4.236759in}{2.067233in}}%
\pgfpathlineto{\pgfqpoint{4.244630in}{2.077208in}}%
\pgfpathclose%
\pgfusepath{fill}%
\end{pgfscope}%
\begin{pgfscope}%
\pgfpathrectangle{\pgfqpoint{1.150000in}{0.150000in}}{\pgfqpoint{5.700000in}{5.700000in}}%
\pgfusepath{clip}%
\pgfsetbuttcap%
\pgfsetroundjoin%
\definecolor{currentfill}{rgb}{0.153364,0.497000,0.557724}%
\pgfsetfillcolor{currentfill}%
\pgfsetfillopacity{0.700000}%
\pgfsetlinewidth{0.000000pt}%
\definecolor{currentstroke}{rgb}{0.000000,0.000000,0.000000}%
\pgfsetstrokecolor{currentstroke}%
\pgfsetdash{}{0pt}%
\pgfpathmoveto{\pgfqpoint{5.779476in}{3.063386in}}%
\pgfpathlineto{\pgfqpoint{5.793953in}{3.071080in}}%
\pgfpathlineto{\pgfqpoint{5.808445in}{3.078873in}}%
\pgfpathlineto{\pgfqpoint{5.822954in}{3.086767in}}%
\pgfpathlineto{\pgfqpoint{5.837478in}{3.094759in}}%
\pgfpathlineto{\pgfqpoint{5.830248in}{3.088734in}}%
\pgfpathlineto{\pgfqpoint{5.823009in}{3.082607in}}%
\pgfpathlineto{\pgfqpoint{5.815761in}{3.076378in}}%
\pgfpathlineto{\pgfqpoint{5.808504in}{3.070045in}}%
\pgfpathlineto{\pgfqpoint{5.793966in}{3.061941in}}%
\pgfpathlineto{\pgfqpoint{5.779444in}{3.053938in}}%
\pgfpathlineto{\pgfqpoint{5.764937in}{3.046035in}}%
\pgfpathlineto{\pgfqpoint{5.750447in}{3.038231in}}%
\pgfpathlineto{\pgfqpoint{5.757717in}{3.044667in}}%
\pgfpathlineto{\pgfqpoint{5.764979in}{3.051004in}}%
\pgfpathlineto{\pgfqpoint{5.772232in}{3.057243in}}%
\pgfpathlineto{\pgfqpoint{5.779476in}{3.063386in}}%
\pgfpathclose%
\pgfusepath{fill}%
\end{pgfscope}%
\begin{pgfscope}%
\pgfpathrectangle{\pgfqpoint{1.150000in}{0.150000in}}{\pgfqpoint{5.700000in}{5.700000in}}%
\pgfusepath{clip}%
\pgfsetbuttcap%
\pgfsetroundjoin%
\definecolor{currentfill}{rgb}{0.267004,0.004874,0.329415}%
\pgfsetfillcolor{currentfill}%
\pgfsetfillopacity{0.700000}%
\pgfsetlinewidth{0.000000pt}%
\definecolor{currentstroke}{rgb}{0.000000,0.000000,0.000000}%
\pgfsetstrokecolor{currentstroke}%
\pgfsetdash{}{0pt}%
\pgfpathmoveto{\pgfqpoint{3.701813in}{1.975491in}}%
\pgfpathlineto{\pgfqpoint{3.715463in}{1.970703in}}%
\pgfpathlineto{\pgfqpoint{3.729118in}{1.966027in}}%
\pgfpathlineto{\pgfqpoint{3.742777in}{1.961463in}}%
\pgfpathlineto{\pgfqpoint{3.756442in}{1.957012in}}%
\pgfpathlineto{\pgfqpoint{3.748384in}{1.949741in}}%
\pgfpathlineto{\pgfqpoint{3.740318in}{1.942563in}}%
\pgfpathlineto{\pgfqpoint{3.732245in}{1.935482in}}%
\pgfpathlineto{\pgfqpoint{3.724166in}{1.928501in}}%
\pgfpathlineto{\pgfqpoint{3.710484in}{1.933289in}}%
\pgfpathlineto{\pgfqpoint{3.696808in}{1.938189in}}%
\pgfpathlineto{\pgfqpoint{3.683136in}{1.943201in}}%
\pgfpathlineto{\pgfqpoint{3.669469in}{1.948326in}}%
\pgfpathlineto{\pgfqpoint{3.677566in}{1.954964in}}%
\pgfpathlineto{\pgfqpoint{3.685656in}{1.961707in}}%
\pgfpathlineto{\pgfqpoint{3.693738in}{1.968550in}}%
\pgfpathlineto{\pgfqpoint{3.701813in}{1.975491in}}%
\pgfpathclose%
\pgfusepath{fill}%
\end{pgfscope}%
\begin{pgfscope}%
\pgfpathrectangle{\pgfqpoint{1.150000in}{0.150000in}}{\pgfqpoint{5.700000in}{5.700000in}}%
\pgfusepath{clip}%
\pgfsetbuttcap%
\pgfsetroundjoin%
\definecolor{currentfill}{rgb}{0.265145,0.232956,0.516599}%
\pgfsetfillcolor{currentfill}%
\pgfsetfillopacity{0.700000}%
\pgfsetlinewidth{0.000000pt}%
\definecolor{currentstroke}{rgb}{0.000000,0.000000,0.000000}%
\pgfsetstrokecolor{currentstroke}%
\pgfsetdash{}{0pt}%
\pgfpathmoveto{\pgfqpoint{2.806277in}{2.426951in}}%
\pgfpathlineto{\pgfqpoint{2.819951in}{2.413322in}}%
\pgfpathlineto{\pgfqpoint{2.833622in}{2.399850in}}%
\pgfpathlineto{\pgfqpoint{2.847291in}{2.386533in}}%
\pgfpathlineto{\pgfqpoint{2.860958in}{2.373371in}}%
\pgfpathlineto{\pgfqpoint{2.852384in}{2.373739in}}%
\pgfpathlineto{\pgfqpoint{2.843796in}{2.374327in}}%
\pgfpathlineto{\pgfqpoint{2.835193in}{2.375139in}}%
\pgfpathlineto{\pgfqpoint{2.826576in}{2.376178in}}%
\pgfpathlineto{\pgfqpoint{2.812873in}{2.389767in}}%
\pgfpathlineto{\pgfqpoint{2.799166in}{2.403511in}}%
\pgfpathlineto{\pgfqpoint{2.785457in}{2.417411in}}%
\pgfpathlineto{\pgfqpoint{2.771745in}{2.431468in}}%
\pgfpathlineto{\pgfqpoint{2.780401in}{2.429993in}}%
\pgfpathlineto{\pgfqpoint{2.789041in}{2.428752in}}%
\pgfpathlineto{\pgfqpoint{2.797666in}{2.427739in}}%
\pgfpathlineto{\pgfqpoint{2.806277in}{2.426951in}}%
\pgfpathclose%
\pgfusepath{fill}%
\end{pgfscope}%
\begin{pgfscope}%
\pgfpathrectangle{\pgfqpoint{1.150000in}{0.150000in}}{\pgfqpoint{5.700000in}{5.700000in}}%
\pgfusepath{clip}%
\pgfsetbuttcap%
\pgfsetroundjoin%
\definecolor{currentfill}{rgb}{0.268510,0.009605,0.335427}%
\pgfsetfillcolor{currentfill}%
\pgfsetfillopacity{0.700000}%
\pgfsetlinewidth{0.000000pt}%
\definecolor{currentstroke}{rgb}{0.000000,0.000000,0.000000}%
\pgfsetstrokecolor{currentstroke}%
\pgfsetdash{}{0pt}%
\pgfpathmoveto{\pgfqpoint{3.560295in}{1.993449in}}%
\pgfpathlineto{\pgfqpoint{3.573927in}{1.987404in}}%
\pgfpathlineto{\pgfqpoint{3.587563in}{1.981475in}}%
\pgfpathlineto{\pgfqpoint{3.601204in}{1.975663in}}%
\pgfpathlineto{\pgfqpoint{3.614848in}{1.969966in}}%
\pgfpathlineto{\pgfqpoint{3.606725in}{1.963784in}}%
\pgfpathlineto{\pgfqpoint{3.598595in}{1.957718in}}%
\pgfpathlineto{\pgfqpoint{3.590457in}{1.951772in}}%
\pgfpathlineto{\pgfqpoint{3.582311in}{1.945947in}}%
\pgfpathlineto{\pgfqpoint{3.568647in}{1.952000in}}%
\pgfpathlineto{\pgfqpoint{3.554987in}{1.958168in}}%
\pgfpathlineto{\pgfqpoint{3.541331in}{1.964452in}}%
\pgfpathlineto{\pgfqpoint{3.527679in}{1.970853in}}%
\pgfpathlineto{\pgfqpoint{3.535845in}{1.976315in}}%
\pgfpathlineto{\pgfqpoint{3.544003in}{1.981904in}}%
\pgfpathlineto{\pgfqpoint{3.552153in}{1.987616in}}%
\pgfpathlineto{\pgfqpoint{3.560295in}{1.993449in}}%
\pgfpathclose%
\pgfusepath{fill}%
\end{pgfscope}%
\begin{pgfscope}%
\pgfpathrectangle{\pgfqpoint{1.150000in}{0.150000in}}{\pgfqpoint{5.700000in}{5.700000in}}%
\pgfusepath{clip}%
\pgfsetbuttcap%
\pgfsetroundjoin%
\definecolor{currentfill}{rgb}{0.278012,0.180367,0.486697}%
\pgfsetfillcolor{currentfill}%
\pgfsetfillopacity{0.700000}%
\pgfsetlinewidth{0.000000pt}%
\definecolor{currentstroke}{rgb}{0.000000,0.000000,0.000000}%
\pgfsetstrokecolor{currentstroke}%
\pgfsetdash{}{0pt}%
\pgfpathmoveto{\pgfqpoint{4.677607in}{2.308111in}}%
\pgfpathlineto{\pgfqpoint{4.691548in}{2.310773in}}%
\pgfpathlineto{\pgfqpoint{4.705500in}{2.313537in}}%
\pgfpathlineto{\pgfqpoint{4.719464in}{2.316403in}}%
\pgfpathlineto{\pgfqpoint{4.733438in}{2.319369in}}%
\pgfpathlineto{\pgfqpoint{4.725709in}{2.308642in}}%
\pgfpathlineto{\pgfqpoint{4.717975in}{2.297866in}}%
\pgfpathlineto{\pgfqpoint{4.710236in}{2.287042in}}%
\pgfpathlineto{\pgfqpoint{4.702491in}{2.276171in}}%
\pgfpathlineto{\pgfqpoint{4.688512in}{2.273362in}}%
\pgfpathlineto{\pgfqpoint{4.674543in}{2.270654in}}%
\pgfpathlineto{\pgfqpoint{4.660586in}{2.268046in}}%
\pgfpathlineto{\pgfqpoint{4.646639in}{2.265540in}}%
\pgfpathlineto{\pgfqpoint{4.654389in}{2.276248in}}%
\pgfpathlineto{\pgfqpoint{4.662134in}{2.286913in}}%
\pgfpathlineto{\pgfqpoint{4.669873in}{2.297534in}}%
\pgfpathlineto{\pgfqpoint{4.677607in}{2.308111in}}%
\pgfpathclose%
\pgfusepath{fill}%
\end{pgfscope}%
\begin{pgfscope}%
\pgfpathrectangle{\pgfqpoint{1.150000in}{0.150000in}}{\pgfqpoint{5.700000in}{5.700000in}}%
\pgfusepath{clip}%
\pgfsetbuttcap%
\pgfsetroundjoin%
\definecolor{currentfill}{rgb}{0.216210,0.351535,0.550627}%
\pgfsetfillcolor{currentfill}%
\pgfsetfillopacity{0.700000}%
\pgfsetlinewidth{0.000000pt}%
\definecolor{currentstroke}{rgb}{0.000000,0.000000,0.000000}%
\pgfsetstrokecolor{currentstroke}%
\pgfsetdash{}{0pt}%
\pgfpathmoveto{\pgfqpoint{5.228697in}{2.687938in}}%
\pgfpathlineto{\pgfqpoint{5.242893in}{2.693593in}}%
\pgfpathlineto{\pgfqpoint{5.257102in}{2.699347in}}%
\pgfpathlineto{\pgfqpoint{5.271326in}{2.705202in}}%
\pgfpathlineto{\pgfqpoint{5.285563in}{2.711157in}}%
\pgfpathlineto{\pgfqpoint{5.278043in}{2.702026in}}%
\pgfpathlineto{\pgfqpoint{5.270516in}{2.692801in}}%
\pgfpathlineto{\pgfqpoint{5.262982in}{2.683482in}}%
\pgfpathlineto{\pgfqpoint{5.255441in}{2.674068in}}%
\pgfpathlineto{\pgfqpoint{5.241197in}{2.668140in}}%
\pgfpathlineto{\pgfqpoint{5.226967in}{2.662311in}}%
\pgfpathlineto{\pgfqpoint{5.212751in}{2.656583in}}%
\pgfpathlineto{\pgfqpoint{5.198548in}{2.650955in}}%
\pgfpathlineto{\pgfqpoint{5.206095in}{2.660335in}}%
\pgfpathlineto{\pgfqpoint{5.213636in}{2.669625in}}%
\pgfpathlineto{\pgfqpoint{5.221170in}{2.678826in}}%
\pgfpathlineto{\pgfqpoint{5.228697in}{2.687938in}}%
\pgfpathclose%
\pgfusepath{fill}%
\end{pgfscope}%
\begin{pgfscope}%
\pgfpathrectangle{\pgfqpoint{1.150000in}{0.150000in}}{\pgfqpoint{5.700000in}{5.700000in}}%
\pgfusepath{clip}%
\pgfsetbuttcap%
\pgfsetroundjoin%
\definecolor{currentfill}{rgb}{0.185556,0.418570,0.556753}%
\pgfsetfillcolor{currentfill}%
\pgfsetfillopacity{0.700000}%
\pgfsetlinewidth{0.000000pt}%
\definecolor{currentstroke}{rgb}{0.000000,0.000000,0.000000}%
\pgfsetstrokecolor{currentstroke}%
\pgfsetdash{}{0pt}%
\pgfpathmoveto{\pgfqpoint{2.421703in}{2.877803in}}%
\pgfpathlineto{\pgfqpoint{2.435516in}{2.859202in}}%
\pgfpathlineto{\pgfqpoint{2.449323in}{2.840802in}}%
\pgfpathlineto{\pgfqpoint{2.463121in}{2.822599in}}%
\pgfpathlineto{\pgfqpoint{2.476913in}{2.804594in}}%
\pgfpathlineto{\pgfqpoint{2.468055in}{2.807949in}}%
\pgfpathlineto{\pgfqpoint{2.459178in}{2.811561in}}%
\pgfpathlineto{\pgfqpoint{2.450284in}{2.815434in}}%
\pgfpathlineto{\pgfqpoint{2.441370in}{2.819573in}}%
\pgfpathlineto{\pgfqpoint{2.427532in}{2.838024in}}%
\pgfpathlineto{\pgfqpoint{2.413687in}{2.856672in}}%
\pgfpathlineto{\pgfqpoint{2.399833in}{2.875519in}}%
\pgfpathlineto{\pgfqpoint{2.385972in}{2.894567in}}%
\pgfpathlineto{\pgfqpoint{2.394933in}{2.889974in}}%
\pgfpathlineto{\pgfqpoint{2.403875in}{2.885652in}}%
\pgfpathlineto{\pgfqpoint{2.412798in}{2.881596in}}%
\pgfpathlineto{\pgfqpoint{2.421703in}{2.877803in}}%
\pgfpathclose%
\pgfusepath{fill}%
\end{pgfscope}%
\begin{pgfscope}%
\pgfpathrectangle{\pgfqpoint{1.150000in}{0.150000in}}{\pgfqpoint{5.700000in}{5.700000in}}%
\pgfusepath{clip}%
\pgfsetbuttcap%
\pgfsetroundjoin%
\definecolor{currentfill}{rgb}{0.276022,0.044167,0.370164}%
\pgfsetfillcolor{currentfill}%
\pgfsetfillopacity{0.700000}%
\pgfsetlinewidth{0.000000pt}%
\definecolor{currentstroke}{rgb}{0.000000,0.000000,0.000000}%
\pgfsetstrokecolor{currentstroke}%
\pgfsetdash{}{0pt}%
\pgfpathmoveto{\pgfqpoint{4.158048in}{2.040827in}}%
\pgfpathlineto{\pgfqpoint{4.171803in}{2.039811in}}%
\pgfpathlineto{\pgfqpoint{4.185565in}{2.038899in}}%
\pgfpathlineto{\pgfqpoint{4.199336in}{2.038093in}}%
\pgfpathlineto{\pgfqpoint{4.213115in}{2.037390in}}%
\pgfpathlineto{\pgfqpoint{4.205223in}{2.027478in}}%
\pgfpathlineto{\pgfqpoint{4.197326in}{2.017588in}}%
\pgfpathlineto{\pgfqpoint{4.189424in}{2.007721in}}%
\pgfpathlineto{\pgfqpoint{4.181516in}{1.997881in}}%
\pgfpathlineto{\pgfqpoint{4.167728in}{1.998848in}}%
\pgfpathlineto{\pgfqpoint{4.153948in}{1.999919in}}%
\pgfpathlineto{\pgfqpoint{4.140176in}{2.001095in}}%
\pgfpathlineto{\pgfqpoint{4.126412in}{2.002376in}}%
\pgfpathlineto{\pgfqpoint{4.134329in}{2.011945in}}%
\pgfpathlineto{\pgfqpoint{4.142241in}{2.021544in}}%
\pgfpathlineto{\pgfqpoint{4.150147in}{2.031173in}}%
\pgfpathlineto{\pgfqpoint{4.158048in}{2.040827in}}%
\pgfpathclose%
\pgfusepath{fill}%
\end{pgfscope}%
\begin{pgfscope}%
\pgfpathrectangle{\pgfqpoint{1.150000in}{0.150000in}}{\pgfqpoint{5.700000in}{5.700000in}}%
\pgfusepath{clip}%
\pgfsetbuttcap%
\pgfsetroundjoin%
\definecolor{currentfill}{rgb}{0.144759,0.519093,0.556572}%
\pgfsetfillcolor{currentfill}%
\pgfsetfillopacity{0.700000}%
\pgfsetlinewidth{0.000000pt}%
\definecolor{currentstroke}{rgb}{0.000000,0.000000,0.000000}%
\pgfsetstrokecolor{currentstroke}%
\pgfsetdash{}{0pt}%
\pgfpathmoveto{\pgfqpoint{5.866309in}{3.117883in}}%
\pgfpathlineto{\pgfqpoint{5.880835in}{3.125846in}}%
\pgfpathlineto{\pgfqpoint{5.895376in}{3.133909in}}%
\pgfpathlineto{\pgfqpoint{5.909934in}{3.142071in}}%
\pgfpathlineto{\pgfqpoint{5.924508in}{3.150333in}}%
\pgfpathlineto{\pgfqpoint{5.917329in}{3.144833in}}%
\pgfpathlineto{\pgfqpoint{5.910141in}{3.139233in}}%
\pgfpathlineto{\pgfqpoint{5.902944in}{3.133532in}}%
\pgfpathlineto{\pgfqpoint{5.895738in}{3.127729in}}%
\pgfpathlineto{\pgfqpoint{5.881148in}{3.119337in}}%
\pgfpathlineto{\pgfqpoint{5.866575in}{3.111044in}}%
\pgfpathlineto{\pgfqpoint{5.852019in}{3.102852in}}%
\pgfpathlineto{\pgfqpoint{5.837478in}{3.094759in}}%
\pgfpathlineto{\pgfqpoint{5.844699in}{3.100686in}}%
\pgfpathlineto{\pgfqpoint{5.851912in}{3.106514in}}%
\pgfpathlineto{\pgfqpoint{5.859115in}{3.112246in}}%
\pgfpathlineto{\pgfqpoint{5.866309in}{3.117883in}}%
\pgfpathclose%
\pgfusepath{fill}%
\end{pgfscope}%
\begin{pgfscope}%
\pgfpathrectangle{\pgfqpoint{1.150000in}{0.150000in}}{\pgfqpoint{5.700000in}{5.700000in}}%
\pgfusepath{clip}%
\pgfsetbuttcap%
\pgfsetroundjoin%
\definecolor{currentfill}{rgb}{0.271828,0.209303,0.504434}%
\pgfsetfillcolor{currentfill}%
\pgfsetfillopacity{0.700000}%
\pgfsetlinewidth{0.000000pt}%
\definecolor{currentstroke}{rgb}{0.000000,0.000000,0.000000}%
\pgfsetstrokecolor{currentstroke}%
\pgfsetdash{}{0pt}%
\pgfpathmoveto{\pgfqpoint{4.764297in}{2.361755in}}%
\pgfpathlineto{\pgfqpoint{4.778278in}{2.364961in}}%
\pgfpathlineto{\pgfqpoint{4.792270in}{2.368267in}}%
\pgfpathlineto{\pgfqpoint{4.806274in}{2.371675in}}%
\pgfpathlineto{\pgfqpoint{4.820289in}{2.375183in}}%
\pgfpathlineto{\pgfqpoint{4.812588in}{2.364535in}}%
\pgfpathlineto{\pgfqpoint{4.804881in}{2.353828in}}%
\pgfpathlineto{\pgfqpoint{4.797168in}{2.343063in}}%
\pgfpathlineto{\pgfqpoint{4.789450in}{2.332242in}}%
\pgfpathlineto{\pgfqpoint{4.775430in}{2.328873in}}%
\pgfpathlineto{\pgfqpoint{4.761421in}{2.325604in}}%
\pgfpathlineto{\pgfqpoint{4.747424in}{2.322436in}}%
\pgfpathlineto{\pgfqpoint{4.733438in}{2.319369in}}%
\pgfpathlineto{\pgfqpoint{4.741161in}{2.330044in}}%
\pgfpathlineto{\pgfqpoint{4.748879in}{2.340668in}}%
\pgfpathlineto{\pgfqpoint{4.756591in}{2.351239in}}%
\pgfpathlineto{\pgfqpoint{4.764297in}{2.361755in}}%
\pgfpathclose%
\pgfusepath{fill}%
\end{pgfscope}%
\begin{pgfscope}%
\pgfpathrectangle{\pgfqpoint{1.150000in}{0.150000in}}{\pgfqpoint{5.700000in}{5.700000in}}%
\pgfusepath{clip}%
\pgfsetbuttcap%
\pgfsetroundjoin%
\definecolor{currentfill}{rgb}{0.271828,0.209303,0.504434}%
\pgfsetfillcolor{currentfill}%
\pgfsetfillopacity{0.700000}%
\pgfsetlinewidth{0.000000pt}%
\definecolor{currentstroke}{rgb}{0.000000,0.000000,0.000000}%
\pgfsetstrokecolor{currentstroke}%
\pgfsetdash{}{0pt}%
\pgfpathmoveto{\pgfqpoint{2.860958in}{2.373371in}}%
\pgfpathlineto{\pgfqpoint{2.874622in}{2.360362in}}%
\pgfpathlineto{\pgfqpoint{2.888285in}{2.347504in}}%
\pgfpathlineto{\pgfqpoint{2.901946in}{2.334798in}}%
\pgfpathlineto{\pgfqpoint{2.915605in}{2.322242in}}%
\pgfpathlineto{\pgfqpoint{2.907067in}{2.322193in}}%
\pgfpathlineto{\pgfqpoint{2.898515in}{2.322358in}}%
\pgfpathlineto{\pgfqpoint{2.889949in}{2.322741in}}%
\pgfpathlineto{\pgfqpoint{2.881369in}{2.323348in}}%
\pgfpathlineto{\pgfqpoint{2.867674in}{2.336329in}}%
\pgfpathlineto{\pgfqpoint{2.853977in}{2.349460in}}%
\pgfpathlineto{\pgfqpoint{2.840278in}{2.362743in}}%
\pgfpathlineto{\pgfqpoint{2.826576in}{2.376178in}}%
\pgfpathlineto{\pgfqpoint{2.835193in}{2.375139in}}%
\pgfpathlineto{\pgfqpoint{2.843796in}{2.374327in}}%
\pgfpathlineto{\pgfqpoint{2.852384in}{2.373739in}}%
\pgfpathlineto{\pgfqpoint{2.860958in}{2.373371in}}%
\pgfpathclose%
\pgfusepath{fill}%
\end{pgfscope}%
\begin{pgfscope}%
\pgfpathrectangle{\pgfqpoint{1.150000in}{0.150000in}}{\pgfqpoint{5.700000in}{5.700000in}}%
\pgfusepath{clip}%
\pgfsetbuttcap%
\pgfsetroundjoin%
\definecolor{currentfill}{rgb}{0.267004,0.004874,0.329415}%
\pgfsetfillcolor{currentfill}%
\pgfsetfillopacity{0.700000}%
\pgfsetlinewidth{0.000000pt}%
\definecolor{currentstroke}{rgb}{0.000000,0.000000,0.000000}%
\pgfsetstrokecolor{currentstroke}%
\pgfsetdash{}{0pt}%
\pgfpathmoveto{\pgfqpoint{3.843263in}{1.971547in}}%
\pgfpathlineto{\pgfqpoint{3.856941in}{1.967968in}}%
\pgfpathlineto{\pgfqpoint{3.870625in}{1.964499in}}%
\pgfpathlineto{\pgfqpoint{3.884315in}{1.961139in}}%
\pgfpathlineto{\pgfqpoint{3.898011in}{1.957888in}}%
\pgfpathlineto{\pgfqpoint{3.890008in}{1.949651in}}%
\pgfpathlineto{\pgfqpoint{3.881999in}{1.941487in}}%
\pgfpathlineto{\pgfqpoint{3.873983in}{1.933397in}}%
\pgfpathlineto{\pgfqpoint{3.865961in}{1.925386in}}%
\pgfpathlineto{\pgfqpoint{3.852251in}{1.928955in}}%
\pgfpathlineto{\pgfqpoint{3.838547in}{1.932634in}}%
\pgfpathlineto{\pgfqpoint{3.824849in}{1.936421in}}%
\pgfpathlineto{\pgfqpoint{3.811156in}{1.940319in}}%
\pgfpathlineto{\pgfqpoint{3.819193in}{1.948005in}}%
\pgfpathlineto{\pgfqpoint{3.827223in}{1.955774in}}%
\pgfpathlineto{\pgfqpoint{3.835246in}{1.963622in}}%
\pgfpathlineto{\pgfqpoint{3.843263in}{1.971547in}}%
\pgfpathclose%
\pgfusepath{fill}%
\end{pgfscope}%
\begin{pgfscope}%
\pgfpathrectangle{\pgfqpoint{1.150000in}{0.150000in}}{\pgfqpoint{5.700000in}{5.700000in}}%
\pgfusepath{clip}%
\pgfsetbuttcap%
\pgfsetroundjoin%
\definecolor{currentfill}{rgb}{0.203063,0.379716,0.553925}%
\pgfsetfillcolor{currentfill}%
\pgfsetfillopacity{0.700000}%
\pgfsetlinewidth{0.000000pt}%
\definecolor{currentstroke}{rgb}{0.000000,0.000000,0.000000}%
\pgfsetstrokecolor{currentstroke}%
\pgfsetdash{}{0pt}%
\pgfpathmoveto{\pgfqpoint{5.315569in}{2.746741in}}%
\pgfpathlineto{\pgfqpoint{5.329812in}{2.752803in}}%
\pgfpathlineto{\pgfqpoint{5.344070in}{2.758965in}}%
\pgfpathlineto{\pgfqpoint{5.358342in}{2.765227in}}%
\pgfpathlineto{\pgfqpoint{5.372627in}{2.771589in}}%
\pgfpathlineto{\pgfqpoint{5.365145in}{2.762833in}}%
\pgfpathlineto{\pgfqpoint{5.357654in}{2.753979in}}%
\pgfpathlineto{\pgfqpoint{5.350156in}{2.745027in}}%
\pgfpathlineto{\pgfqpoint{5.342651in}{2.735976in}}%
\pgfpathlineto{\pgfqpoint{5.328358in}{2.729621in}}%
\pgfpathlineto{\pgfqpoint{5.314079in}{2.723366in}}%
\pgfpathlineto{\pgfqpoint{5.299814in}{2.717212in}}%
\pgfpathlineto{\pgfqpoint{5.285563in}{2.711157in}}%
\pgfpathlineto{\pgfqpoint{5.293075in}{2.720194in}}%
\pgfpathlineto{\pgfqpoint{5.300580in}{2.729136in}}%
\pgfpathlineto{\pgfqpoint{5.308078in}{2.737986in}}%
\pgfpathlineto{\pgfqpoint{5.315569in}{2.746741in}}%
\pgfpathclose%
\pgfusepath{fill}%
\end{pgfscope}%
\begin{pgfscope}%
\pgfpathrectangle{\pgfqpoint{1.150000in}{0.150000in}}{\pgfqpoint{5.700000in}{5.700000in}}%
\pgfusepath{clip}%
\pgfsetbuttcap%
\pgfsetroundjoin%
\definecolor{currentfill}{rgb}{0.273809,0.031497,0.358853}%
\pgfsetfillcolor{currentfill}%
\pgfsetfillopacity{0.700000}%
\pgfsetlinewidth{0.000000pt}%
\definecolor{currentstroke}{rgb}{0.000000,0.000000,0.000000}%
\pgfsetstrokecolor{currentstroke}%
\pgfsetdash{}{0pt}%
\pgfpathmoveto{\pgfqpoint{3.418584in}{2.026331in}}%
\pgfpathlineto{\pgfqpoint{3.432210in}{2.018976in}}%
\pgfpathlineto{\pgfqpoint{3.445839in}{2.011742in}}%
\pgfpathlineto{\pgfqpoint{3.459471in}{2.004629in}}%
\pgfpathlineto{\pgfqpoint{3.473105in}{1.997636in}}%
\pgfpathlineto{\pgfqpoint{3.464909in}{1.992673in}}%
\pgfpathlineto{\pgfqpoint{3.456704in}{1.987849in}}%
\pgfpathlineto{\pgfqpoint{3.448490in}{1.983168in}}%
\pgfpathlineto{\pgfqpoint{3.440267in}{1.978634in}}%
\pgfpathlineto{\pgfqpoint{3.426610in}{1.986002in}}%
\pgfpathlineto{\pgfqpoint{3.412955in}{1.993490in}}%
\pgfpathlineto{\pgfqpoint{3.399303in}{2.001100in}}%
\pgfpathlineto{\pgfqpoint{3.385654in}{2.008830in}}%
\pgfpathlineto{\pgfqpoint{3.393901in}{2.012982in}}%
\pgfpathlineto{\pgfqpoint{3.402138in}{2.017286in}}%
\pgfpathlineto{\pgfqpoint{3.410365in}{2.021736in}}%
\pgfpathlineto{\pgfqpoint{3.418584in}{2.026331in}}%
\pgfpathclose%
\pgfusepath{fill}%
\end{pgfscope}%
\begin{pgfscope}%
\pgfpathrectangle{\pgfqpoint{1.150000in}{0.150000in}}{\pgfqpoint{5.700000in}{5.700000in}}%
\pgfusepath{clip}%
\pgfsetbuttcap%
\pgfsetroundjoin%
\definecolor{currentfill}{rgb}{0.136408,0.541173,0.554483}%
\pgfsetfillcolor{currentfill}%
\pgfsetfillopacity{0.700000}%
\pgfsetlinewidth{0.000000pt}%
\definecolor{currentstroke}{rgb}{0.000000,0.000000,0.000000}%
\pgfsetstrokecolor{currentstroke}%
\pgfsetdash{}{0pt}%
\pgfpathmoveto{\pgfqpoint{5.953133in}{3.171373in}}%
\pgfpathlineto{\pgfqpoint{5.967707in}{3.179585in}}%
\pgfpathlineto{\pgfqpoint{5.982298in}{3.187896in}}%
\pgfpathlineto{\pgfqpoint{5.996905in}{3.196307in}}%
\pgfpathlineto{\pgfqpoint{6.011528in}{3.204818in}}%
\pgfpathlineto{\pgfqpoint{6.004403in}{3.199855in}}%
\pgfpathlineto{\pgfqpoint{5.997268in}{3.194796in}}%
\pgfpathlineto{\pgfqpoint{5.990123in}{3.189637in}}%
\pgfpathlineto{\pgfqpoint{5.982970in}{3.184378in}}%
\pgfpathlineto{\pgfqpoint{5.968330in}{3.175717in}}%
\pgfpathlineto{\pgfqpoint{5.953706in}{3.167156in}}%
\pgfpathlineto{\pgfqpoint{5.939099in}{3.158694in}}%
\pgfpathlineto{\pgfqpoint{5.924508in}{3.150333in}}%
\pgfpathlineto{\pgfqpoint{5.931678in}{3.155735in}}%
\pgfpathlineto{\pgfqpoint{5.938839in}{3.161041in}}%
\pgfpathlineto{\pgfqpoint{5.945991in}{3.166253in}}%
\pgfpathlineto{\pgfqpoint{5.953133in}{3.171373in}}%
\pgfpathclose%
\pgfusepath{fill}%
\end{pgfscope}%
\begin{pgfscope}%
\pgfpathrectangle{\pgfqpoint{1.150000in}{0.150000in}}{\pgfqpoint{5.700000in}{5.700000in}}%
\pgfusepath{clip}%
\pgfsetbuttcap%
\pgfsetroundjoin%
\definecolor{currentfill}{rgb}{0.124395,0.578002,0.548287}%
\pgfsetfillcolor{currentfill}%
\pgfsetfillopacity{0.700000}%
\pgfsetlinewidth{0.000000pt}%
\definecolor{currentstroke}{rgb}{0.000000,0.000000,0.000000}%
\pgfsetstrokecolor{currentstroke}%
\pgfsetdash{}{0pt}%
\pgfpathmoveto{\pgfqpoint{6.126715in}{3.274863in}}%
\pgfpathlineto{\pgfqpoint{6.141385in}{3.283513in}}%
\pgfpathlineto{\pgfqpoint{6.156072in}{3.292262in}}%
\pgfpathlineto{\pgfqpoint{6.170776in}{3.301110in}}%
\pgfpathlineto{\pgfqpoint{6.163759in}{3.297196in}}%
\pgfpathlineto{\pgfqpoint{6.156731in}{3.293194in}}%
\pgfpathlineto{\pgfqpoint{6.149695in}{3.289100in}}%
\pgfpathlineto{\pgfqpoint{6.142649in}{3.284913in}}%
\pgfpathlineto{\pgfqpoint{6.127925in}{3.275874in}}%
\pgfpathlineto{\pgfqpoint{6.113218in}{3.266935in}}%
\pgfpathlineto{\pgfqpoint{6.098528in}{3.258096in}}%
\pgfpathlineto{\pgfqpoint{6.105589in}{3.262420in}}%
\pgfpathlineto{\pgfqpoint{6.112640in}{3.266654in}}%
\pgfpathlineto{\pgfqpoint{6.119682in}{3.270802in}}%
\pgfpathlineto{\pgfqpoint{6.126715in}{3.274863in}}%
\pgfpathclose%
\pgfusepath{fill}%
\end{pgfscope}%
\begin{pgfscope}%
\pgfpathrectangle{\pgfqpoint{1.150000in}{0.150000in}}{\pgfqpoint{5.700000in}{5.700000in}}%
\pgfusepath{clip}%
\pgfsetbuttcap%
\pgfsetroundjoin%
\definecolor{currentfill}{rgb}{0.281446,0.084320,0.407414}%
\pgfsetfillcolor{currentfill}%
\pgfsetfillopacity{0.700000}%
\pgfsetlinewidth{0.000000pt}%
\definecolor{currentstroke}{rgb}{0.000000,0.000000,0.000000}%
\pgfsetstrokecolor{currentstroke}%
\pgfsetdash{}{0pt}%
\pgfpathmoveto{\pgfqpoint{3.222034in}{2.111298in}}%
\pgfpathlineto{\pgfqpoint{3.235659in}{2.102060in}}%
\pgfpathlineto{\pgfqpoint{3.249286in}{2.092951in}}%
\pgfpathlineto{\pgfqpoint{3.262914in}{2.083972in}}%
\pgfpathlineto{\pgfqpoint{3.276544in}{2.075120in}}%
\pgfpathlineto{\pgfqpoint{3.268237in}{2.071899in}}%
\pgfpathlineto{\pgfqpoint{3.259920in}{2.068847in}}%
\pgfpathlineto{\pgfqpoint{3.251592in}{2.065968in}}%
\pgfpathlineto{\pgfqpoint{3.243254in}{2.063264in}}%
\pgfpathlineto{\pgfqpoint{3.229597in}{2.072513in}}%
\pgfpathlineto{\pgfqpoint{3.215941in}{2.081889in}}%
\pgfpathlineto{\pgfqpoint{3.202286in}{2.091394in}}%
\pgfpathlineto{\pgfqpoint{3.188633in}{2.101030in}}%
\pgfpathlineto{\pgfqpoint{3.196999in}{2.103329in}}%
\pgfpathlineto{\pgfqpoint{3.205355in}{2.105809in}}%
\pgfpathlineto{\pgfqpoint{3.213700in}{2.108467in}}%
\pgfpathlineto{\pgfqpoint{3.222034in}{2.111298in}}%
\pgfpathclose%
\pgfusepath{fill}%
\end{pgfscope}%
\begin{pgfscope}%
\pgfpathrectangle{\pgfqpoint{1.150000in}{0.150000in}}{\pgfqpoint{5.700000in}{5.700000in}}%
\pgfusepath{clip}%
\pgfsetbuttcap%
\pgfsetroundjoin%
\definecolor{currentfill}{rgb}{0.263663,0.237631,0.518762}%
\pgfsetfillcolor{currentfill}%
\pgfsetfillopacity{0.700000}%
\pgfsetlinewidth{0.000000pt}%
\definecolor{currentstroke}{rgb}{0.000000,0.000000,0.000000}%
\pgfsetstrokecolor{currentstroke}%
\pgfsetdash{}{0pt}%
\pgfpathmoveto{\pgfqpoint{4.851038in}{2.417166in}}%
\pgfpathlineto{\pgfqpoint{4.865060in}{2.420895in}}%
\pgfpathlineto{\pgfqpoint{4.879093in}{2.424725in}}%
\pgfpathlineto{\pgfqpoint{4.893139in}{2.428655in}}%
\pgfpathlineto{\pgfqpoint{4.907197in}{2.432686in}}%
\pgfpathlineto{\pgfqpoint{4.899524in}{2.422170in}}%
\pgfpathlineto{\pgfqpoint{4.891844in}{2.411586in}}%
\pgfpathlineto{\pgfqpoint{4.884159in}{2.400937in}}%
\pgfpathlineto{\pgfqpoint{4.876468in}{2.390221in}}%
\pgfpathlineto{\pgfqpoint{4.862406in}{2.386311in}}%
\pgfpathlineto{\pgfqpoint{4.848355in}{2.382501in}}%
\pgfpathlineto{\pgfqpoint{4.834316in}{2.378792in}}%
\pgfpathlineto{\pgfqpoint{4.820289in}{2.375183in}}%
\pgfpathlineto{\pgfqpoint{4.827985in}{2.385771in}}%
\pgfpathlineto{\pgfqpoint{4.835675in}{2.396298in}}%
\pgfpathlineto{\pgfqpoint{4.843359in}{2.406764in}}%
\pgfpathlineto{\pgfqpoint{4.851038in}{2.417166in}}%
\pgfpathclose%
\pgfusepath{fill}%
\end{pgfscope}%
\begin{pgfscope}%
\pgfpathrectangle{\pgfqpoint{1.150000in}{0.150000in}}{\pgfqpoint{5.700000in}{5.700000in}}%
\pgfusepath{clip}%
\pgfsetbuttcap%
\pgfsetroundjoin%
\definecolor{currentfill}{rgb}{0.272594,0.025563,0.353093}%
\pgfsetfillcolor{currentfill}%
\pgfsetfillopacity{0.700000}%
\pgfsetlinewidth{0.000000pt}%
\definecolor{currentstroke}{rgb}{0.000000,0.000000,0.000000}%
\pgfsetstrokecolor{currentstroke}%
\pgfsetdash{}{0pt}%
\pgfpathmoveto{\pgfqpoint{4.071432in}{2.008553in}}%
\pgfpathlineto{\pgfqpoint{4.085165in}{2.006850in}}%
\pgfpathlineto{\pgfqpoint{4.098907in}{2.005253in}}%
\pgfpathlineto{\pgfqpoint{4.112656in}{2.003762in}}%
\pgfpathlineto{\pgfqpoint{4.126412in}{2.002376in}}%
\pgfpathlineto{\pgfqpoint{4.118490in}{1.992841in}}%
\pgfpathlineto{\pgfqpoint{4.110562in}{1.983341in}}%
\pgfpathlineto{\pgfqpoint{4.102629in}{1.973881in}}%
\pgfpathlineto{\pgfqpoint{4.094690in}{1.964462in}}%
\pgfpathlineto{\pgfqpoint{4.080923in}{1.966130in}}%
\pgfpathlineto{\pgfqpoint{4.067164in}{1.967903in}}%
\pgfpathlineto{\pgfqpoint{4.053412in}{1.969783in}}%
\pgfpathlineto{\pgfqpoint{4.039668in}{1.971768in}}%
\pgfpathlineto{\pgfqpoint{4.047617in}{1.980898in}}%
\pgfpathlineto{\pgfqpoint{4.055561in}{1.990074in}}%
\pgfpathlineto{\pgfqpoint{4.063499in}{1.999293in}}%
\pgfpathlineto{\pgfqpoint{4.071432in}{2.008553in}}%
\pgfpathclose%
\pgfusepath{fill}%
\end{pgfscope}%
\begin{pgfscope}%
\pgfpathrectangle{\pgfqpoint{1.150000in}{0.150000in}}{\pgfqpoint{5.700000in}{5.700000in}}%
\pgfusepath{clip}%
\pgfsetbuttcap%
\pgfsetroundjoin%
\definecolor{currentfill}{rgb}{0.277134,0.185228,0.489898}%
\pgfsetfillcolor{currentfill}%
\pgfsetfillopacity{0.700000}%
\pgfsetlinewidth{0.000000pt}%
\definecolor{currentstroke}{rgb}{0.000000,0.000000,0.000000}%
\pgfsetstrokecolor{currentstroke}%
\pgfsetdash{}{0pt}%
\pgfpathmoveto{\pgfqpoint{2.915605in}{2.322242in}}%
\pgfpathlineto{\pgfqpoint{2.929262in}{2.309834in}}%
\pgfpathlineto{\pgfqpoint{2.942918in}{2.297574in}}%
\pgfpathlineto{\pgfqpoint{2.956573in}{2.285461in}}%
\pgfpathlineto{\pgfqpoint{2.970227in}{2.273493in}}%
\pgfpathlineto{\pgfqpoint{2.961723in}{2.273028in}}%
\pgfpathlineto{\pgfqpoint{2.953206in}{2.272772in}}%
\pgfpathlineto{\pgfqpoint{2.944675in}{2.272730in}}%
\pgfpathlineto{\pgfqpoint{2.936131in}{2.272905in}}%
\pgfpathlineto{\pgfqpoint{2.922443in}{2.285295in}}%
\pgfpathlineto{\pgfqpoint{2.908753in}{2.297832in}}%
\pgfpathlineto{\pgfqpoint{2.895062in}{2.310516in}}%
\pgfpathlineto{\pgfqpoint{2.881369in}{2.323348in}}%
\pgfpathlineto{\pgfqpoint{2.889949in}{2.322741in}}%
\pgfpathlineto{\pgfqpoint{2.898515in}{2.322358in}}%
\pgfpathlineto{\pgfqpoint{2.907067in}{2.322193in}}%
\pgfpathlineto{\pgfqpoint{2.915605in}{2.322242in}}%
\pgfpathclose%
\pgfusepath{fill}%
\end{pgfscope}%
\begin{pgfscope}%
\pgfpathrectangle{\pgfqpoint{1.150000in}{0.150000in}}{\pgfqpoint{5.700000in}{5.700000in}}%
\pgfusepath{clip}%
\pgfsetbuttcap%
\pgfsetroundjoin%
\definecolor{currentfill}{rgb}{0.129933,0.559582,0.551864}%
\pgfsetfillcolor{currentfill}%
\pgfsetfillopacity{0.700000}%
\pgfsetlinewidth{0.000000pt}%
\definecolor{currentstroke}{rgb}{0.000000,0.000000,0.000000}%
\pgfsetstrokecolor{currentstroke}%
\pgfsetdash{}{0pt}%
\pgfpathmoveto{\pgfqpoint{6.039938in}{3.223735in}}%
\pgfpathlineto{\pgfqpoint{6.054560in}{3.232176in}}%
\pgfpathlineto{\pgfqpoint{6.069199in}{3.240716in}}%
\pgfpathlineto{\pgfqpoint{6.083855in}{3.249356in}}%
\pgfpathlineto{\pgfqpoint{6.098528in}{3.258096in}}%
\pgfpathlineto{\pgfqpoint{6.091458in}{3.253679in}}%
\pgfpathlineto{\pgfqpoint{6.084378in}{3.249169in}}%
\pgfpathlineto{\pgfqpoint{6.077289in}{3.244563in}}%
\pgfpathlineto{\pgfqpoint{6.070191in}{3.239859in}}%
\pgfpathlineto{\pgfqpoint{6.055500in}{3.230949in}}%
\pgfpathlineto{\pgfqpoint{6.040826in}{3.222139in}}%
\pgfpathlineto{\pgfqpoint{6.026169in}{3.213428in}}%
\pgfpathlineto{\pgfqpoint{6.011528in}{3.204818in}}%
\pgfpathlineto{\pgfqpoint{6.018645in}{3.209685in}}%
\pgfpathlineto{\pgfqpoint{6.025752in}{3.214459in}}%
\pgfpathlineto{\pgfqpoint{6.032849in}{3.219142in}}%
\pgfpathlineto{\pgfqpoint{6.039938in}{3.223735in}}%
\pgfpathclose%
\pgfusepath{fill}%
\end{pgfscope}%
\begin{pgfscope}%
\pgfpathrectangle{\pgfqpoint{1.150000in}{0.150000in}}{\pgfqpoint{5.700000in}{5.700000in}}%
\pgfusepath{clip}%
\pgfsetbuttcap%
\pgfsetroundjoin%
\definecolor{currentfill}{rgb}{0.192357,0.403199,0.555836}%
\pgfsetfillcolor{currentfill}%
\pgfsetfillopacity{0.700000}%
\pgfsetlinewidth{0.000000pt}%
\definecolor{currentstroke}{rgb}{0.000000,0.000000,0.000000}%
\pgfsetstrokecolor{currentstroke}%
\pgfsetdash{}{0pt}%
\pgfpathmoveto{\pgfqpoint{5.402483in}{2.805637in}}%
\pgfpathlineto{\pgfqpoint{5.416775in}{2.812086in}}%
\pgfpathlineto{\pgfqpoint{5.431081in}{2.818636in}}%
\pgfpathlineto{\pgfqpoint{5.445402in}{2.825285in}}%
\pgfpathlineto{\pgfqpoint{5.459738in}{2.832035in}}%
\pgfpathlineto{\pgfqpoint{5.452294in}{2.823687in}}%
\pgfpathlineto{\pgfqpoint{5.444842in}{2.815239in}}%
\pgfpathlineto{\pgfqpoint{5.437382in}{2.806688in}}%
\pgfpathlineto{\pgfqpoint{5.429915in}{2.798035in}}%
\pgfpathlineto{\pgfqpoint{5.415571in}{2.791274in}}%
\pgfpathlineto{\pgfqpoint{5.401242in}{2.784612in}}%
\pgfpathlineto{\pgfqpoint{5.386928in}{2.778051in}}%
\pgfpathlineto{\pgfqpoint{5.372627in}{2.771589in}}%
\pgfpathlineto{\pgfqpoint{5.380103in}{2.780247in}}%
\pgfpathlineto{\pgfqpoint{5.387570in}{2.788807in}}%
\pgfpathlineto{\pgfqpoint{5.395031in}{2.797270in}}%
\pgfpathlineto{\pgfqpoint{5.402483in}{2.805637in}}%
\pgfpathclose%
\pgfusepath{fill}%
\end{pgfscope}%
\begin{pgfscope}%
\pgfpathrectangle{\pgfqpoint{1.150000in}{0.150000in}}{\pgfqpoint{5.700000in}{5.700000in}}%
\pgfusepath{clip}%
\pgfsetbuttcap%
\pgfsetroundjoin%
\definecolor{currentfill}{rgb}{0.253935,0.265254,0.529983}%
\pgfsetfillcolor{currentfill}%
\pgfsetfillopacity{0.700000}%
\pgfsetlinewidth{0.000000pt}%
\definecolor{currentstroke}{rgb}{0.000000,0.000000,0.000000}%
\pgfsetstrokecolor{currentstroke}%
\pgfsetdash{}{0pt}%
\pgfpathmoveto{\pgfqpoint{4.937831in}{2.474059in}}%
\pgfpathlineto{\pgfqpoint{4.951896in}{2.478292in}}%
\pgfpathlineto{\pgfqpoint{4.965974in}{2.482626in}}%
\pgfpathlineto{\pgfqpoint{4.980063in}{2.487059in}}%
\pgfpathlineto{\pgfqpoint{4.994166in}{2.491593in}}%
\pgfpathlineto{\pgfqpoint{4.986521in}{2.481260in}}%
\pgfpathlineto{\pgfqpoint{4.978871in}{2.470852in}}%
\pgfpathlineto{\pgfqpoint{4.971214in}{2.460369in}}%
\pgfpathlineto{\pgfqpoint{4.963552in}{2.449812in}}%
\pgfpathlineto{\pgfqpoint{4.949444in}{2.445380in}}%
\pgfpathlineto{\pgfqpoint{4.935350in}{2.441049in}}%
\pgfpathlineto{\pgfqpoint{4.921267in}{2.436817in}}%
\pgfpathlineto{\pgfqpoint{4.907197in}{2.432686in}}%
\pgfpathlineto{\pgfqpoint{4.914865in}{2.443134in}}%
\pgfpathlineto{\pgfqpoint{4.922526in}{2.453512in}}%
\pgfpathlineto{\pgfqpoint{4.930182in}{2.463821in}}%
\pgfpathlineto{\pgfqpoint{4.937831in}{2.474059in}}%
\pgfpathclose%
\pgfusepath{fill}%
\end{pgfscope}%
\begin{pgfscope}%
\pgfpathrectangle{\pgfqpoint{1.150000in}{0.150000in}}{\pgfqpoint{5.700000in}{5.700000in}}%
\pgfusepath{clip}%
\pgfsetbuttcap%
\pgfsetroundjoin%
\definecolor{currentfill}{rgb}{0.280255,0.165693,0.476498}%
\pgfsetfillcolor{currentfill}%
\pgfsetfillopacity{0.700000}%
\pgfsetlinewidth{0.000000pt}%
\definecolor{currentstroke}{rgb}{0.000000,0.000000,0.000000}%
\pgfsetstrokecolor{currentstroke}%
\pgfsetdash{}{0pt}%
\pgfpathmoveto{\pgfqpoint{2.970227in}{2.273493in}}%
\pgfpathlineto{\pgfqpoint{2.983879in}{2.261670in}}%
\pgfpathlineto{\pgfqpoint{2.997531in}{2.249991in}}%
\pgfpathlineto{\pgfqpoint{3.011182in}{2.238454in}}%
\pgfpathlineto{\pgfqpoint{3.024832in}{2.227058in}}%
\pgfpathlineto{\pgfqpoint{3.016361in}{2.226179in}}%
\pgfpathlineto{\pgfqpoint{3.007878in}{2.225503in}}%
\pgfpathlineto{\pgfqpoint{2.999381in}{2.225036in}}%
\pgfpathlineto{\pgfqpoint{2.990872in}{2.224782in}}%
\pgfpathlineto{\pgfqpoint{2.977188in}{2.236599in}}%
\pgfpathlineto{\pgfqpoint{2.963503in}{2.248557in}}%
\pgfpathlineto{\pgfqpoint{2.949818in}{2.260659in}}%
\pgfpathlineto{\pgfqpoint{2.936131in}{2.272905in}}%
\pgfpathlineto{\pgfqpoint{2.944675in}{2.272730in}}%
\pgfpathlineto{\pgfqpoint{2.953206in}{2.272772in}}%
\pgfpathlineto{\pgfqpoint{2.961723in}{2.273028in}}%
\pgfpathlineto{\pgfqpoint{2.970227in}{2.273493in}}%
\pgfpathclose%
\pgfusepath{fill}%
\end{pgfscope}%
\begin{pgfscope}%
\pgfpathrectangle{\pgfqpoint{1.150000in}{0.150000in}}{\pgfqpoint{5.700000in}{5.700000in}}%
\pgfusepath{clip}%
\pgfsetbuttcap%
\pgfsetroundjoin%
\definecolor{currentfill}{rgb}{0.269944,0.014625,0.341379}%
\pgfsetfillcolor{currentfill}%
\pgfsetfillopacity{0.700000}%
\pgfsetlinewidth{0.000000pt}%
\definecolor{currentstroke}{rgb}{0.000000,0.000000,0.000000}%
\pgfsetstrokecolor{currentstroke}%
\pgfsetdash{}{0pt}%
\pgfpathmoveto{\pgfqpoint{3.984761in}{1.980772in}}%
\pgfpathlineto{\pgfqpoint{3.998477in}{1.978360in}}%
\pgfpathlineto{\pgfqpoint{4.012200in}{1.976056in}}%
\pgfpathlineto{\pgfqpoint{4.025930in}{1.973859in}}%
\pgfpathlineto{\pgfqpoint{4.039668in}{1.971768in}}%
\pgfpathlineto{\pgfqpoint{4.031713in}{1.962686in}}%
\pgfpathlineto{\pgfqpoint{4.023752in}{1.953656in}}%
\pgfpathlineto{\pgfqpoint{4.015786in}{1.944680in}}%
\pgfpathlineto{\pgfqpoint{4.007814in}{1.935760in}}%
\pgfpathlineto{\pgfqpoint{3.994065in}{1.938152in}}%
\pgfpathlineto{\pgfqpoint{3.980323in}{1.940650in}}%
\pgfpathlineto{\pgfqpoint{3.966588in}{1.943254in}}%
\pgfpathlineto{\pgfqpoint{3.952859in}{1.945966in}}%
\pgfpathlineto{\pgfqpoint{3.960844in}{1.954578in}}%
\pgfpathlineto{\pgfqpoint{3.968822in}{1.963251in}}%
\pgfpathlineto{\pgfqpoint{3.976794in}{1.971984in}}%
\pgfpathlineto{\pgfqpoint{3.984761in}{1.980772in}}%
\pgfpathclose%
\pgfusepath{fill}%
\end{pgfscope}%
\begin{pgfscope}%
\pgfpathrectangle{\pgfqpoint{1.150000in}{0.150000in}}{\pgfqpoint{5.700000in}{5.700000in}}%
\pgfusepath{clip}%
\pgfsetbuttcap%
\pgfsetroundjoin%
\definecolor{currentfill}{rgb}{0.268510,0.009605,0.335427}%
\pgfsetfillcolor{currentfill}%
\pgfsetfillopacity{0.700000}%
\pgfsetlinewidth{0.000000pt}%
\definecolor{currentstroke}{rgb}{0.000000,0.000000,0.000000}%
\pgfsetstrokecolor{currentstroke}%
\pgfsetdash{}{0pt}%
\pgfpathmoveto{\pgfqpoint{3.614848in}{1.969966in}}%
\pgfpathlineto{\pgfqpoint{3.628497in}{1.964385in}}%
\pgfpathlineto{\pgfqpoint{3.642150in}{1.958918in}}%
\pgfpathlineto{\pgfqpoint{3.655807in}{1.953565in}}%
\pgfpathlineto{\pgfqpoint{3.669469in}{1.948326in}}%
\pgfpathlineto{\pgfqpoint{3.661365in}{1.941796in}}%
\pgfpathlineto{\pgfqpoint{3.653253in}{1.935377in}}%
\pgfpathlineto{\pgfqpoint{3.645134in}{1.929073in}}%
\pgfpathlineto{\pgfqpoint{3.637007in}{1.922886in}}%
\pgfpathlineto{\pgfqpoint{3.623327in}{1.928480in}}%
\pgfpathlineto{\pgfqpoint{3.609650in}{1.934188in}}%
\pgfpathlineto{\pgfqpoint{3.595978in}{1.940010in}}%
\pgfpathlineto{\pgfqpoint{3.582311in}{1.945947in}}%
\pgfpathlineto{\pgfqpoint{3.590457in}{1.951772in}}%
\pgfpathlineto{\pgfqpoint{3.598595in}{1.957718in}}%
\pgfpathlineto{\pgfqpoint{3.606725in}{1.963784in}}%
\pgfpathlineto{\pgfqpoint{3.614848in}{1.969966in}}%
\pgfpathclose%
\pgfusepath{fill}%
\end{pgfscope}%
\begin{pgfscope}%
\pgfpathrectangle{\pgfqpoint{1.150000in}{0.150000in}}{\pgfqpoint{5.700000in}{5.700000in}}%
\pgfusepath{clip}%
\pgfsetbuttcap%
\pgfsetroundjoin%
\definecolor{currentfill}{rgb}{0.180629,0.429975,0.557282}%
\pgfsetfillcolor{currentfill}%
\pgfsetfillopacity{0.700000}%
\pgfsetlinewidth{0.000000pt}%
\definecolor{currentstroke}{rgb}{0.000000,0.000000,0.000000}%
\pgfsetstrokecolor{currentstroke}%
\pgfsetdash{}{0pt}%
\pgfpathmoveto{\pgfqpoint{5.489435in}{2.864420in}}%
\pgfpathlineto{\pgfqpoint{5.503776in}{2.871238in}}%
\pgfpathlineto{\pgfqpoint{5.518131in}{2.878155in}}%
\pgfpathlineto{\pgfqpoint{5.532502in}{2.885173in}}%
\pgfpathlineto{\pgfqpoint{5.546887in}{2.892290in}}%
\pgfpathlineto{\pgfqpoint{5.539485in}{2.884382in}}%
\pgfpathlineto{\pgfqpoint{5.532074in}{2.876370in}}%
\pgfpathlineto{\pgfqpoint{5.524654in}{2.868253in}}%
\pgfpathlineto{\pgfqpoint{5.517227in}{2.860031in}}%
\pgfpathlineto{\pgfqpoint{5.502832in}{2.852882in}}%
\pgfpathlineto{\pgfqpoint{5.488453in}{2.845833in}}%
\pgfpathlineto{\pgfqpoint{5.474088in}{2.838884in}}%
\pgfpathlineto{\pgfqpoint{5.459738in}{2.832035in}}%
\pgfpathlineto{\pgfqpoint{5.467174in}{2.840281in}}%
\pgfpathlineto{\pgfqpoint{5.474602in}{2.848427in}}%
\pgfpathlineto{\pgfqpoint{5.482022in}{2.856473in}}%
\pgfpathlineto{\pgfqpoint{5.489435in}{2.864420in}}%
\pgfpathclose%
\pgfusepath{fill}%
\end{pgfscope}%
\begin{pgfscope}%
\pgfpathrectangle{\pgfqpoint{1.150000in}{0.150000in}}{\pgfqpoint{5.700000in}{5.700000in}}%
\pgfusepath{clip}%
\pgfsetbuttcap%
\pgfsetroundjoin%
\definecolor{currentfill}{rgb}{0.267004,0.004874,0.329415}%
\pgfsetfillcolor{currentfill}%
\pgfsetfillopacity{0.700000}%
\pgfsetlinewidth{0.000000pt}%
\definecolor{currentstroke}{rgb}{0.000000,0.000000,0.000000}%
\pgfsetstrokecolor{currentstroke}%
\pgfsetdash{}{0pt}%
\pgfpathmoveto{\pgfqpoint{3.756442in}{1.957012in}}%
\pgfpathlineto{\pgfqpoint{3.770113in}{1.952672in}}%
\pgfpathlineto{\pgfqpoint{3.783788in}{1.948444in}}%
\pgfpathlineto{\pgfqpoint{3.797469in}{1.944326in}}%
\pgfpathlineto{\pgfqpoint{3.811156in}{1.940319in}}%
\pgfpathlineto{\pgfqpoint{3.803113in}{1.932718in}}%
\pgfpathlineto{\pgfqpoint{3.795063in}{1.925206in}}%
\pgfpathlineto{\pgfqpoint{3.787007in}{1.917786in}}%
\pgfpathlineto{\pgfqpoint{3.778944in}{1.910461in}}%
\pgfpathlineto{\pgfqpoint{3.765241in}{1.914805in}}%
\pgfpathlineto{\pgfqpoint{3.751544in}{1.919259in}}%
\pgfpathlineto{\pgfqpoint{3.737852in}{1.923824in}}%
\pgfpathlineto{\pgfqpoint{3.724166in}{1.928501in}}%
\pgfpathlineto{\pgfqpoint{3.732245in}{1.935482in}}%
\pgfpathlineto{\pgfqpoint{3.740318in}{1.942563in}}%
\pgfpathlineto{\pgfqpoint{3.748384in}{1.949741in}}%
\pgfpathlineto{\pgfqpoint{3.756442in}{1.957012in}}%
\pgfpathclose%
\pgfusepath{fill}%
\end{pgfscope}%
\begin{pgfscope}%
\pgfpathrectangle{\pgfqpoint{1.150000in}{0.150000in}}{\pgfqpoint{5.700000in}{5.700000in}}%
\pgfusepath{clip}%
\pgfsetbuttcap%
\pgfsetroundjoin%
\definecolor{currentfill}{rgb}{0.243113,0.292092,0.538516}%
\pgfsetfillcolor{currentfill}%
\pgfsetfillopacity{0.700000}%
\pgfsetlinewidth{0.000000pt}%
\definecolor{currentstroke}{rgb}{0.000000,0.000000,0.000000}%
\pgfsetstrokecolor{currentstroke}%
\pgfsetdash{}{0pt}%
\pgfpathmoveto{\pgfqpoint{5.024681in}{2.532161in}}%
\pgfpathlineto{\pgfqpoint{5.038790in}{2.536879in}}%
\pgfpathlineto{\pgfqpoint{5.052912in}{2.541696in}}%
\pgfpathlineto{\pgfqpoint{5.067048in}{2.546614in}}%
\pgfpathlineto{\pgfqpoint{5.081196in}{2.551632in}}%
\pgfpathlineto{\pgfqpoint{5.073582in}{2.541530in}}%
\pgfpathlineto{\pgfqpoint{5.065962in}{2.531346in}}%
\pgfpathlineto{\pgfqpoint{5.058335in}{2.521079in}}%
\pgfpathlineto{\pgfqpoint{5.050702in}{2.510732in}}%
\pgfpathlineto{\pgfqpoint{5.036549in}{2.505797in}}%
\pgfpathlineto{\pgfqpoint{5.022408in}{2.500962in}}%
\pgfpathlineto{\pgfqpoint{5.008280in}{2.496228in}}%
\pgfpathlineto{\pgfqpoint{4.994166in}{2.491593in}}%
\pgfpathlineto{\pgfqpoint{5.001804in}{2.501851in}}%
\pgfpathlineto{\pgfqpoint{5.009436in}{2.512032in}}%
\pgfpathlineto{\pgfqpoint{5.017061in}{2.522135in}}%
\pgfpathlineto{\pgfqpoint{5.024681in}{2.532161in}}%
\pgfpathclose%
\pgfusepath{fill}%
\end{pgfscope}%
\begin{pgfscope}%
\pgfpathrectangle{\pgfqpoint{1.150000in}{0.150000in}}{\pgfqpoint{5.700000in}{5.700000in}}%
\pgfusepath{clip}%
\pgfsetbuttcap%
\pgfsetroundjoin%
\definecolor{currentfill}{rgb}{0.279566,0.067836,0.391917}%
\pgfsetfillcolor{currentfill}%
\pgfsetfillopacity{0.700000}%
\pgfsetlinewidth{0.000000pt}%
\definecolor{currentstroke}{rgb}{0.000000,0.000000,0.000000}%
\pgfsetstrokecolor{currentstroke}%
\pgfsetdash{}{0pt}%
\pgfpathmoveto{\pgfqpoint{3.276544in}{2.075120in}}%
\pgfpathlineto{\pgfqpoint{3.290176in}{2.066395in}}%
\pgfpathlineto{\pgfqpoint{3.303809in}{2.057797in}}%
\pgfpathlineto{\pgfqpoint{3.317445in}{2.049325in}}%
\pgfpathlineto{\pgfqpoint{3.331082in}{2.040979in}}%
\pgfpathlineto{\pgfqpoint{3.322801in}{2.037369in}}%
\pgfpathlineto{\pgfqpoint{3.314510in}{2.033924in}}%
\pgfpathlineto{\pgfqpoint{3.306209in}{2.030646in}}%
\pgfpathlineto{\pgfqpoint{3.297898in}{2.027539in}}%
\pgfpathlineto{\pgfqpoint{3.284234in}{2.036282in}}%
\pgfpathlineto{\pgfqpoint{3.270572in}{2.045150in}}%
\pgfpathlineto{\pgfqpoint{3.256912in}{2.054144in}}%
\pgfpathlineto{\pgfqpoint{3.243254in}{2.063264in}}%
\pgfpathlineto{\pgfqpoint{3.251592in}{2.065968in}}%
\pgfpathlineto{\pgfqpoint{3.259920in}{2.068847in}}%
\pgfpathlineto{\pgfqpoint{3.268237in}{2.071899in}}%
\pgfpathlineto{\pgfqpoint{3.276544in}{2.075120in}}%
\pgfpathclose%
\pgfusepath{fill}%
\end{pgfscope}%
\begin{pgfscope}%
\pgfpathrectangle{\pgfqpoint{1.150000in}{0.150000in}}{\pgfqpoint{5.700000in}{5.700000in}}%
\pgfusepath{clip}%
\pgfsetbuttcap%
\pgfsetroundjoin%
\definecolor{currentfill}{rgb}{0.272594,0.025563,0.353093}%
\pgfsetfillcolor{currentfill}%
\pgfsetfillopacity{0.700000}%
\pgfsetlinewidth{0.000000pt}%
\definecolor{currentstroke}{rgb}{0.000000,0.000000,0.000000}%
\pgfsetstrokecolor{currentstroke}%
\pgfsetdash{}{0pt}%
\pgfpathmoveto{\pgfqpoint{3.473105in}{1.997636in}}%
\pgfpathlineto{\pgfqpoint{3.486744in}{1.990763in}}%
\pgfpathlineto{\pgfqpoint{3.500385in}{1.984008in}}%
\pgfpathlineto{\pgfqpoint{3.514030in}{1.977372in}}%
\pgfpathlineto{\pgfqpoint{3.527679in}{1.970853in}}%
\pgfpathlineto{\pgfqpoint{3.519504in}{1.965523in}}%
\pgfpathlineto{\pgfqpoint{3.511321in}{1.960326in}}%
\pgfpathlineto{\pgfqpoint{3.503129in}{1.955268in}}%
\pgfpathlineto{\pgfqpoint{3.494929in}{1.950352in}}%
\pgfpathlineto{\pgfqpoint{3.481259in}{1.957245in}}%
\pgfpathlineto{\pgfqpoint{3.467592in}{1.964256in}}%
\pgfpathlineto{\pgfqpoint{3.453928in}{1.971385in}}%
\pgfpathlineto{\pgfqpoint{3.440267in}{1.978634in}}%
\pgfpathlineto{\pgfqpoint{3.448490in}{1.983168in}}%
\pgfpathlineto{\pgfqpoint{3.456704in}{1.987849in}}%
\pgfpathlineto{\pgfqpoint{3.464909in}{1.992673in}}%
\pgfpathlineto{\pgfqpoint{3.473105in}{1.997636in}}%
\pgfpathclose%
\pgfusepath{fill}%
\end{pgfscope}%
\begin{pgfscope}%
\pgfpathrectangle{\pgfqpoint{1.150000in}{0.150000in}}{\pgfqpoint{5.700000in}{5.700000in}}%
\pgfusepath{clip}%
\pgfsetbuttcap%
\pgfsetroundjoin%
\definecolor{currentfill}{rgb}{0.282290,0.145912,0.461510}%
\pgfsetfillcolor{currentfill}%
\pgfsetfillopacity{0.700000}%
\pgfsetlinewidth{0.000000pt}%
\definecolor{currentstroke}{rgb}{0.000000,0.000000,0.000000}%
\pgfsetstrokecolor{currentstroke}%
\pgfsetdash{}{0pt}%
\pgfpathmoveto{\pgfqpoint{3.024832in}{2.227058in}}%
\pgfpathlineto{\pgfqpoint{3.038482in}{2.215803in}}%
\pgfpathlineto{\pgfqpoint{3.052131in}{2.204688in}}%
\pgfpathlineto{\pgfqpoint{3.065780in}{2.193712in}}%
\pgfpathlineto{\pgfqpoint{3.079430in}{2.182874in}}%
\pgfpathlineto{\pgfqpoint{3.070991in}{2.181582in}}%
\pgfpathlineto{\pgfqpoint{3.062540in}{2.180489in}}%
\pgfpathlineto{\pgfqpoint{3.054076in}{2.179599in}}%
\pgfpathlineto{\pgfqpoint{3.045600in}{2.178917in}}%
\pgfpathlineto{\pgfqpoint{3.031919in}{2.190175in}}%
\pgfpathlineto{\pgfqpoint{3.018237in}{2.201571in}}%
\pgfpathlineto{\pgfqpoint{3.004555in}{2.213106in}}%
\pgfpathlineto{\pgfqpoint{2.990872in}{2.224782in}}%
\pgfpathlineto{\pgfqpoint{2.999381in}{2.225036in}}%
\pgfpathlineto{\pgfqpoint{3.007878in}{2.225503in}}%
\pgfpathlineto{\pgfqpoint{3.016361in}{2.226179in}}%
\pgfpathlineto{\pgfqpoint{3.024832in}{2.227058in}}%
\pgfpathclose%
\pgfusepath{fill}%
\end{pgfscope}%
\begin{pgfscope}%
\pgfpathrectangle{\pgfqpoint{1.150000in}{0.150000in}}{\pgfqpoint{5.700000in}{5.700000in}}%
\pgfusepath{clip}%
\pgfsetbuttcap%
\pgfsetroundjoin%
\definecolor{currentfill}{rgb}{0.171176,0.452530,0.557965}%
\pgfsetfillcolor{currentfill}%
\pgfsetfillopacity{0.700000}%
\pgfsetlinewidth{0.000000pt}%
\definecolor{currentstroke}{rgb}{0.000000,0.000000,0.000000}%
\pgfsetstrokecolor{currentstroke}%
\pgfsetdash{}{0pt}%
\pgfpathmoveto{\pgfqpoint{5.576417in}{2.922898in}}%
\pgfpathlineto{\pgfqpoint{5.590808in}{2.930065in}}%
\pgfpathlineto{\pgfqpoint{5.605213in}{2.937331in}}%
\pgfpathlineto{\pgfqpoint{5.619634in}{2.944696in}}%
\pgfpathlineto{\pgfqpoint{5.634070in}{2.952162in}}%
\pgfpathlineto{\pgfqpoint{5.626711in}{2.944720in}}%
\pgfpathlineto{\pgfqpoint{5.619343in}{2.937173in}}%
\pgfpathlineto{\pgfqpoint{5.611966in}{2.929520in}}%
\pgfpathlineto{\pgfqpoint{5.604581in}{2.921759in}}%
\pgfpathlineto{\pgfqpoint{5.590135in}{2.914242in}}%
\pgfpathlineto{\pgfqpoint{5.575704in}{2.906824in}}%
\pgfpathlineto{\pgfqpoint{5.561288in}{2.899507in}}%
\pgfpathlineto{\pgfqpoint{5.546887in}{2.892290in}}%
\pgfpathlineto{\pgfqpoint{5.554282in}{2.900095in}}%
\pgfpathlineto{\pgfqpoint{5.561669in}{2.907797in}}%
\pgfpathlineto{\pgfqpoint{5.569047in}{2.915398in}}%
\pgfpathlineto{\pgfqpoint{5.576417in}{2.922898in}}%
\pgfpathclose%
\pgfusepath{fill}%
\end{pgfscope}%
\begin{pgfscope}%
\pgfpathrectangle{\pgfqpoint{1.150000in}{0.150000in}}{\pgfqpoint{5.700000in}{5.700000in}}%
\pgfusepath{clip}%
\pgfsetbuttcap%
\pgfsetroundjoin%
\definecolor{currentfill}{rgb}{0.282327,0.094955,0.417331}%
\pgfsetfillcolor{currentfill}%
\pgfsetfillopacity{0.700000}%
\pgfsetlinewidth{0.000000pt}%
\definecolor{currentstroke}{rgb}{0.000000,0.000000,0.000000}%
\pgfsetstrokecolor{currentstroke}%
\pgfsetdash{}{0pt}%
\pgfpathmoveto{\pgfqpoint{4.386477in}{2.119109in}}%
\pgfpathlineto{\pgfqpoint{4.400320in}{2.119814in}}%
\pgfpathlineto{\pgfqpoint{4.414172in}{2.120621in}}%
\pgfpathlineto{\pgfqpoint{4.428034in}{2.121530in}}%
\pgfpathlineto{\pgfqpoint{4.441905in}{2.122542in}}%
\pgfpathlineto{\pgfqpoint{4.434077in}{2.111872in}}%
\pgfpathlineto{\pgfqpoint{4.426243in}{2.101193in}}%
\pgfpathlineto{\pgfqpoint{4.418404in}{2.090506in}}%
\pgfpathlineto{\pgfqpoint{4.410561in}{2.079813in}}%
\pgfpathlineto{\pgfqpoint{4.396683in}{2.079031in}}%
\pgfpathlineto{\pgfqpoint{4.382815in}{2.078351in}}%
\pgfpathlineto{\pgfqpoint{4.368956in}{2.077773in}}%
\pgfpathlineto{\pgfqpoint{4.355106in}{2.077298in}}%
\pgfpathlineto{\pgfqpoint{4.362956in}{2.087754in}}%
\pgfpathlineto{\pgfqpoint{4.370801in}{2.098210in}}%
\pgfpathlineto{\pgfqpoint{4.378642in}{2.108662in}}%
\pgfpathlineto{\pgfqpoint{4.386477in}{2.119109in}}%
\pgfpathclose%
\pgfusepath{fill}%
\end{pgfscope}%
\begin{pgfscope}%
\pgfpathrectangle{\pgfqpoint{1.150000in}{0.150000in}}{\pgfqpoint{5.700000in}{5.700000in}}%
\pgfusepath{clip}%
\pgfsetbuttcap%
\pgfsetroundjoin%
\definecolor{currentfill}{rgb}{0.229739,0.322361,0.545706}%
\pgfsetfillcolor{currentfill}%
\pgfsetfillopacity{0.700000}%
\pgfsetlinewidth{0.000000pt}%
\definecolor{currentstroke}{rgb}{0.000000,0.000000,0.000000}%
\pgfsetstrokecolor{currentstroke}%
\pgfsetdash{}{0pt}%
\pgfpathmoveto{\pgfqpoint{5.111586in}{2.591210in}}%
\pgfpathlineto{\pgfqpoint{5.125742in}{2.596393in}}%
\pgfpathlineto{\pgfqpoint{5.139911in}{2.601675in}}%
\pgfpathlineto{\pgfqpoint{5.154093in}{2.607058in}}%
\pgfpathlineto{\pgfqpoint{5.168288in}{2.612541in}}%
\pgfpathlineto{\pgfqpoint{5.160706in}{2.602715in}}%
\pgfpathlineto{\pgfqpoint{5.153117in}{2.592800in}}%
\pgfpathlineto{\pgfqpoint{5.145522in}{2.582797in}}%
\pgfpathlineto{\pgfqpoint{5.137920in}{2.572706in}}%
\pgfpathlineto{\pgfqpoint{5.123719in}{2.567287in}}%
\pgfpathlineto{\pgfqpoint{5.109531in}{2.561969in}}%
\pgfpathlineto{\pgfqpoint{5.095357in}{2.556751in}}%
\pgfpathlineto{\pgfqpoint{5.081196in}{2.551632in}}%
\pgfpathlineto{\pgfqpoint{5.088803in}{2.561652in}}%
\pgfpathlineto{\pgfqpoint{5.096404in}{2.571588in}}%
\pgfpathlineto{\pgfqpoint{5.103999in}{2.581441in}}%
\pgfpathlineto{\pgfqpoint{5.111586in}{2.591210in}}%
\pgfpathclose%
\pgfusepath{fill}%
\end{pgfscope}%
\begin{pgfscope}%
\pgfpathrectangle{\pgfqpoint{1.150000in}{0.150000in}}{\pgfqpoint{5.700000in}{5.700000in}}%
\pgfusepath{clip}%
\pgfsetbuttcap%
\pgfsetroundjoin%
\definecolor{currentfill}{rgb}{0.283229,0.120777,0.440584}%
\pgfsetfillcolor{currentfill}%
\pgfsetfillopacity{0.700000}%
\pgfsetlinewidth{0.000000pt}%
\definecolor{currentstroke}{rgb}{0.000000,0.000000,0.000000}%
\pgfsetstrokecolor{currentstroke}%
\pgfsetdash{}{0pt}%
\pgfpathmoveto{\pgfqpoint{4.473168in}{2.165084in}}%
\pgfpathlineto{\pgfqpoint{4.487043in}{2.166409in}}%
\pgfpathlineto{\pgfqpoint{4.500928in}{2.167836in}}%
\pgfpathlineto{\pgfqpoint{4.514823in}{2.169365in}}%
\pgfpathlineto{\pgfqpoint{4.528728in}{2.170995in}}%
\pgfpathlineto{\pgfqpoint{4.520926in}{2.160179in}}%
\pgfpathlineto{\pgfqpoint{4.513118in}{2.149341in}}%
\pgfpathlineto{\pgfqpoint{4.505306in}{2.138483in}}%
\pgfpathlineto{\pgfqpoint{4.497488in}{2.127607in}}%
\pgfpathlineto{\pgfqpoint{4.483578in}{2.126188in}}%
\pgfpathlineto{\pgfqpoint{4.469677in}{2.124871in}}%
\pgfpathlineto{\pgfqpoint{4.455786in}{2.123655in}}%
\pgfpathlineto{\pgfqpoint{4.441905in}{2.122542in}}%
\pgfpathlineto{\pgfqpoint{4.449729in}{2.133200in}}%
\pgfpathlineto{\pgfqpoint{4.457547in}{2.143844in}}%
\pgfpathlineto{\pgfqpoint{4.465360in}{2.154473in}}%
\pgfpathlineto{\pgfqpoint{4.473168in}{2.165084in}}%
\pgfpathclose%
\pgfusepath{fill}%
\end{pgfscope}%
\begin{pgfscope}%
\pgfpathrectangle{\pgfqpoint{1.150000in}{0.150000in}}{\pgfqpoint{5.700000in}{5.700000in}}%
\pgfusepath{clip}%
\pgfsetbuttcap%
\pgfsetroundjoin%
\definecolor{currentfill}{rgb}{0.280267,0.073417,0.397163}%
\pgfsetfillcolor{currentfill}%
\pgfsetfillopacity{0.700000}%
\pgfsetlinewidth{0.000000pt}%
\definecolor{currentstroke}{rgb}{0.000000,0.000000,0.000000}%
\pgfsetstrokecolor{currentstroke}%
\pgfsetdash{}{0pt}%
\pgfpathmoveto{\pgfqpoint{4.299797in}{2.076425in}}%
\pgfpathlineto{\pgfqpoint{4.313611in}{2.076489in}}%
\pgfpathlineto{\pgfqpoint{4.327434in}{2.076656in}}%
\pgfpathlineto{\pgfqpoint{4.341265in}{2.076925in}}%
\pgfpathlineto{\pgfqpoint{4.355106in}{2.077298in}}%
\pgfpathlineto{\pgfqpoint{4.347250in}{2.066843in}}%
\pgfpathlineto{\pgfqpoint{4.339390in}{2.056391in}}%
\pgfpathlineto{\pgfqpoint{4.331524in}{2.045945in}}%
\pgfpathlineto{\pgfqpoint{4.323654in}{2.035506in}}%
\pgfpathlineto{\pgfqpoint{4.309806in}{2.035381in}}%
\pgfpathlineto{\pgfqpoint{4.295967in}{2.035358in}}%
\pgfpathlineto{\pgfqpoint{4.282137in}{2.035438in}}%
\pgfpathlineto{\pgfqpoint{4.268315in}{2.035621in}}%
\pgfpathlineto{\pgfqpoint{4.276193in}{2.045806in}}%
\pgfpathlineto{\pgfqpoint{4.284067in}{2.056003in}}%
\pgfpathlineto{\pgfqpoint{4.291935in}{2.066210in}}%
\pgfpathlineto{\pgfqpoint{4.299797in}{2.076425in}}%
\pgfpathclose%
\pgfusepath{fill}%
\end{pgfscope}%
\begin{pgfscope}%
\pgfpathrectangle{\pgfqpoint{1.150000in}{0.150000in}}{\pgfqpoint{5.700000in}{5.700000in}}%
\pgfusepath{clip}%
\pgfsetbuttcap%
\pgfsetroundjoin%
\definecolor{currentfill}{rgb}{0.282290,0.145912,0.461510}%
\pgfsetfillcolor{currentfill}%
\pgfsetfillopacity{0.700000}%
\pgfsetlinewidth{0.000000pt}%
\definecolor{currentstroke}{rgb}{0.000000,0.000000,0.000000}%
\pgfsetstrokecolor{currentstroke}%
\pgfsetdash{}{0pt}%
\pgfpathmoveto{\pgfqpoint{4.559886in}{2.214006in}}%
\pgfpathlineto{\pgfqpoint{4.573795in}{2.215931in}}%
\pgfpathlineto{\pgfqpoint{4.587715in}{2.217958in}}%
\pgfpathlineto{\pgfqpoint{4.601646in}{2.220086in}}%
\pgfpathlineto{\pgfqpoint{4.615587in}{2.222316in}}%
\pgfpathlineto{\pgfqpoint{4.607810in}{2.211418in}}%
\pgfpathlineto{\pgfqpoint{4.600029in}{2.200487in}}%
\pgfpathlineto{\pgfqpoint{4.592242in}{2.189524in}}%
\pgfpathlineto{\pgfqpoint{4.584450in}{2.178532in}}%
\pgfpathlineto{\pgfqpoint{4.570504in}{2.176496in}}%
\pgfpathlineto{\pgfqpoint{4.556568in}{2.174561in}}%
\pgfpathlineto{\pgfqpoint{4.542643in}{2.172727in}}%
\pgfpathlineto{\pgfqpoint{4.528728in}{2.170995in}}%
\pgfpathlineto{\pgfqpoint{4.536525in}{2.181788in}}%
\pgfpathlineto{\pgfqpoint{4.544317in}{2.192555in}}%
\pgfpathlineto{\pgfqpoint{4.552104in}{2.203295in}}%
\pgfpathlineto{\pgfqpoint{4.559886in}{2.214006in}}%
\pgfpathclose%
\pgfusepath{fill}%
\end{pgfscope}%
\begin{pgfscope}%
\pgfpathrectangle{\pgfqpoint{1.150000in}{0.150000in}}{\pgfqpoint{5.700000in}{5.700000in}}%
\pgfusepath{clip}%
\pgfsetbuttcap%
\pgfsetroundjoin%
\definecolor{currentfill}{rgb}{0.267004,0.004874,0.329415}%
\pgfsetfillcolor{currentfill}%
\pgfsetfillopacity{0.700000}%
\pgfsetlinewidth{0.000000pt}%
\definecolor{currentstroke}{rgb}{0.000000,0.000000,0.000000}%
\pgfsetstrokecolor{currentstroke}%
\pgfsetdash{}{0pt}%
\pgfpathmoveto{\pgfqpoint{3.898011in}{1.957888in}}%
\pgfpathlineto{\pgfqpoint{3.911714in}{1.954745in}}%
\pgfpathlineto{\pgfqpoint{3.925422in}{1.951711in}}%
\pgfpathlineto{\pgfqpoint{3.939138in}{1.948784in}}%
\pgfpathlineto{\pgfqpoint{3.952859in}{1.945966in}}%
\pgfpathlineto{\pgfqpoint{3.944869in}{1.937418in}}%
\pgfpathlineto{\pgfqpoint{3.936873in}{1.928938in}}%
\pgfpathlineto{\pgfqpoint{3.928871in}{1.920527in}}%
\pgfpathlineto{\pgfqpoint{3.920862in}{1.912191in}}%
\pgfpathlineto{\pgfqpoint{3.907128in}{1.915328in}}%
\pgfpathlineto{\pgfqpoint{3.893399in}{1.918573in}}%
\pgfpathlineto{\pgfqpoint{3.879677in}{1.921925in}}%
\pgfpathlineto{\pgfqpoint{3.865961in}{1.925386in}}%
\pgfpathlineto{\pgfqpoint{3.873983in}{1.933397in}}%
\pgfpathlineto{\pgfqpoint{3.881999in}{1.941487in}}%
\pgfpathlineto{\pgfqpoint{3.890008in}{1.949651in}}%
\pgfpathlineto{\pgfqpoint{3.898011in}{1.957888in}}%
\pgfpathclose%
\pgfusepath{fill}%
\end{pgfscope}%
\begin{pgfscope}%
\pgfpathrectangle{\pgfqpoint{1.150000in}{0.150000in}}{\pgfqpoint{5.700000in}{5.700000in}}%
\pgfusepath{clip}%
\pgfsetbuttcap%
\pgfsetroundjoin%
\definecolor{currentfill}{rgb}{0.279574,0.170599,0.479997}%
\pgfsetfillcolor{currentfill}%
\pgfsetfillopacity{0.700000}%
\pgfsetlinewidth{0.000000pt}%
\definecolor{currentstroke}{rgb}{0.000000,0.000000,0.000000}%
\pgfsetstrokecolor{currentstroke}%
\pgfsetdash{}{0pt}%
\pgfpathmoveto{\pgfqpoint{4.646639in}{2.265540in}}%
\pgfpathlineto{\pgfqpoint{4.660586in}{2.268046in}}%
\pgfpathlineto{\pgfqpoint{4.674543in}{2.270654in}}%
\pgfpathlineto{\pgfqpoint{4.688512in}{2.273362in}}%
\pgfpathlineto{\pgfqpoint{4.702491in}{2.276171in}}%
\pgfpathlineto{\pgfqpoint{4.694741in}{2.265254in}}%
\pgfpathlineto{\pgfqpoint{4.686985in}{2.254293in}}%
\pgfpathlineto{\pgfqpoint{4.679224in}{2.243289in}}%
\pgfpathlineto{\pgfqpoint{4.671458in}{2.232245in}}%
\pgfpathlineto{\pgfqpoint{4.657474in}{2.229611in}}%
\pgfpathlineto{\pgfqpoint{4.643501in}{2.227078in}}%
\pgfpathlineto{\pgfqpoint{4.629538in}{2.224646in}}%
\pgfpathlineto{\pgfqpoint{4.615587in}{2.222316in}}%
\pgfpathlineto{\pgfqpoint{4.623358in}{2.233178in}}%
\pgfpathlineto{\pgfqpoint{4.631123in}{2.244004in}}%
\pgfpathlineto{\pgfqpoint{4.638884in}{2.254792in}}%
\pgfpathlineto{\pgfqpoint{4.646639in}{2.265540in}}%
\pgfpathclose%
\pgfusepath{fill}%
\end{pgfscope}%
\begin{pgfscope}%
\pgfpathrectangle{\pgfqpoint{1.150000in}{0.150000in}}{\pgfqpoint{5.700000in}{5.700000in}}%
\pgfusepath{clip}%
\pgfsetbuttcap%
\pgfsetroundjoin%
\definecolor{currentfill}{rgb}{0.277018,0.050344,0.375715}%
\pgfsetfillcolor{currentfill}%
\pgfsetfillopacity{0.700000}%
\pgfsetlinewidth{0.000000pt}%
\definecolor{currentstroke}{rgb}{0.000000,0.000000,0.000000}%
\pgfsetstrokecolor{currentstroke}%
\pgfsetdash{}{0pt}%
\pgfpathmoveto{\pgfqpoint{4.213115in}{2.037390in}}%
\pgfpathlineto{\pgfqpoint{4.226903in}{2.036792in}}%
\pgfpathlineto{\pgfqpoint{4.240698in}{2.036298in}}%
\pgfpathlineto{\pgfqpoint{4.254502in}{2.035908in}}%
\pgfpathlineto{\pgfqpoint{4.268315in}{2.035621in}}%
\pgfpathlineto{\pgfqpoint{4.260432in}{2.025451in}}%
\pgfpathlineto{\pgfqpoint{4.252543in}{2.015298in}}%
\pgfpathlineto{\pgfqpoint{4.244649in}{2.005165in}}%
\pgfpathlineto{\pgfqpoint{4.236751in}{1.995053in}}%
\pgfpathlineto{\pgfqpoint{4.222930in}{1.995605in}}%
\pgfpathlineto{\pgfqpoint{4.209117in}{1.996260in}}%
\pgfpathlineto{\pgfqpoint{4.195313in}{1.997019in}}%
\pgfpathlineto{\pgfqpoint{4.181516in}{1.997881in}}%
\pgfpathlineto{\pgfqpoint{4.189424in}{2.007721in}}%
\pgfpathlineto{\pgfqpoint{4.197326in}{2.017588in}}%
\pgfpathlineto{\pgfqpoint{4.205223in}{2.027478in}}%
\pgfpathlineto{\pgfqpoint{4.213115in}{2.037390in}}%
\pgfpathclose%
\pgfusepath{fill}%
\end{pgfscope}%
\begin{pgfscope}%
\pgfpathrectangle{\pgfqpoint{1.150000in}{0.150000in}}{\pgfqpoint{5.700000in}{5.700000in}}%
\pgfusepath{clip}%
\pgfsetbuttcap%
\pgfsetroundjoin%
\definecolor{currentfill}{rgb}{0.162142,0.474838,0.558140}%
\pgfsetfillcolor{currentfill}%
\pgfsetfillopacity{0.700000}%
\pgfsetlinewidth{0.000000pt}%
\definecolor{currentstroke}{rgb}{0.000000,0.000000,0.000000}%
\pgfsetstrokecolor{currentstroke}%
\pgfsetdash{}{0pt}%
\pgfpathmoveto{\pgfqpoint{5.663424in}{2.980892in}}%
\pgfpathlineto{\pgfqpoint{5.677864in}{2.988387in}}%
\pgfpathlineto{\pgfqpoint{5.692320in}{2.995981in}}%
\pgfpathlineto{\pgfqpoint{5.706791in}{3.003676in}}%
\pgfpathlineto{\pgfqpoint{5.721279in}{3.011470in}}%
\pgfpathlineto{\pgfqpoint{5.713965in}{3.004519in}}%
\pgfpathlineto{\pgfqpoint{5.706642in}{2.997462in}}%
\pgfpathlineto{\pgfqpoint{5.699310in}{2.990298in}}%
\pgfpathlineto{\pgfqpoint{5.691970in}{2.983025in}}%
\pgfpathlineto{\pgfqpoint{5.677472in}{2.975159in}}%
\pgfpathlineto{\pgfqpoint{5.662989in}{2.967393in}}%
\pgfpathlineto{\pgfqpoint{5.648522in}{2.959728in}}%
\pgfpathlineto{\pgfqpoint{5.634070in}{2.952162in}}%
\pgfpathlineto{\pgfqpoint{5.641421in}{2.959499in}}%
\pgfpathlineto{\pgfqpoint{5.648764in}{2.966732in}}%
\pgfpathlineto{\pgfqpoint{5.656098in}{2.973863in}}%
\pgfpathlineto{\pgfqpoint{5.663424in}{2.980892in}}%
\pgfpathclose%
\pgfusepath{fill}%
\end{pgfscope}%
\begin{pgfscope}%
\pgfpathrectangle{\pgfqpoint{1.150000in}{0.150000in}}{\pgfqpoint{5.700000in}{5.700000in}}%
\pgfusepath{clip}%
\pgfsetbuttcap%
\pgfsetroundjoin%
\definecolor{currentfill}{rgb}{0.218130,0.347432,0.550038}%
\pgfsetfillcolor{currentfill}%
\pgfsetfillopacity{0.700000}%
\pgfsetlinewidth{0.000000pt}%
\definecolor{currentstroke}{rgb}{0.000000,0.000000,0.000000}%
\pgfsetstrokecolor{currentstroke}%
\pgfsetdash{}{0pt}%
\pgfpathmoveto{\pgfqpoint{5.198548in}{2.650955in}}%
\pgfpathlineto{\pgfqpoint{5.212751in}{2.656583in}}%
\pgfpathlineto{\pgfqpoint{5.226967in}{2.662311in}}%
\pgfpathlineto{\pgfqpoint{5.241197in}{2.668140in}}%
\pgfpathlineto{\pgfqpoint{5.255441in}{2.674068in}}%
\pgfpathlineto{\pgfqpoint{5.247893in}{2.664560in}}%
\pgfpathlineto{\pgfqpoint{5.240337in}{2.654958in}}%
\pgfpathlineto{\pgfqpoint{5.232775in}{2.645262in}}%
\pgfpathlineto{\pgfqpoint{5.225205in}{2.635473in}}%
\pgfpathlineto{\pgfqpoint{5.210956in}{2.629590in}}%
\pgfpathlineto{\pgfqpoint{5.196719in}{2.623807in}}%
\pgfpathlineto{\pgfqpoint{5.182497in}{2.618124in}}%
\pgfpathlineto{\pgfqpoint{5.168288in}{2.612541in}}%
\pgfpathlineto{\pgfqpoint{5.175863in}{2.622278in}}%
\pgfpathlineto{\pgfqpoint{5.183432in}{2.631926in}}%
\pgfpathlineto{\pgfqpoint{5.190993in}{2.641485in}}%
\pgfpathlineto{\pgfqpoint{5.198548in}{2.650955in}}%
\pgfpathclose%
\pgfusepath{fill}%
\end{pgfscope}%
\begin{pgfscope}%
\pgfpathrectangle{\pgfqpoint{1.150000in}{0.150000in}}{\pgfqpoint{5.700000in}{5.700000in}}%
\pgfusepath{clip}%
\pgfsetbuttcap%
\pgfsetroundjoin%
\definecolor{currentfill}{rgb}{0.283072,0.130895,0.449241}%
\pgfsetfillcolor{currentfill}%
\pgfsetfillopacity{0.700000}%
\pgfsetlinewidth{0.000000pt}%
\definecolor{currentstroke}{rgb}{0.000000,0.000000,0.000000}%
\pgfsetstrokecolor{currentstroke}%
\pgfsetdash{}{0pt}%
\pgfpathmoveto{\pgfqpoint{3.079430in}{2.182874in}}%
\pgfpathlineto{\pgfqpoint{3.093079in}{2.172173in}}%
\pgfpathlineto{\pgfqpoint{3.106728in}{2.161608in}}%
\pgfpathlineto{\pgfqpoint{3.120377in}{2.151178in}}%
\pgfpathlineto{\pgfqpoint{3.134027in}{2.140883in}}%
\pgfpathlineto{\pgfqpoint{3.125619in}{2.139179in}}%
\pgfpathlineto{\pgfqpoint{3.117200in}{2.137669in}}%
\pgfpathlineto{\pgfqpoint{3.108768in}{2.136358in}}%
\pgfpathlineto{\pgfqpoint{3.100324in}{2.135250in}}%
\pgfpathlineto{\pgfqpoint{3.086643in}{2.145963in}}%
\pgfpathlineto{\pgfqpoint{3.072962in}{2.156812in}}%
\pgfpathlineto{\pgfqpoint{3.059281in}{2.167796in}}%
\pgfpathlineto{\pgfqpoint{3.045600in}{2.178917in}}%
\pgfpathlineto{\pgfqpoint{3.054076in}{2.179599in}}%
\pgfpathlineto{\pgfqpoint{3.062540in}{2.180489in}}%
\pgfpathlineto{\pgfqpoint{3.070991in}{2.181582in}}%
\pgfpathlineto{\pgfqpoint{3.079430in}{2.182874in}}%
\pgfpathclose%
\pgfusepath{fill}%
\end{pgfscope}%
\begin{pgfscope}%
\pgfpathrectangle{\pgfqpoint{1.150000in}{0.150000in}}{\pgfqpoint{5.700000in}{5.700000in}}%
\pgfusepath{clip}%
\pgfsetbuttcap%
\pgfsetroundjoin%
\definecolor{currentfill}{rgb}{0.274128,0.199721,0.498911}%
\pgfsetfillcolor{currentfill}%
\pgfsetfillopacity{0.700000}%
\pgfsetlinewidth{0.000000pt}%
\definecolor{currentstroke}{rgb}{0.000000,0.000000,0.000000}%
\pgfsetstrokecolor{currentstroke}%
\pgfsetdash{}{0pt}%
\pgfpathmoveto{\pgfqpoint{4.733438in}{2.319369in}}%
\pgfpathlineto{\pgfqpoint{4.747424in}{2.322436in}}%
\pgfpathlineto{\pgfqpoint{4.761421in}{2.325604in}}%
\pgfpathlineto{\pgfqpoint{4.775430in}{2.328873in}}%
\pgfpathlineto{\pgfqpoint{4.789450in}{2.332242in}}%
\pgfpathlineto{\pgfqpoint{4.781726in}{2.321365in}}%
\pgfpathlineto{\pgfqpoint{4.773996in}{2.310434in}}%
\pgfpathlineto{\pgfqpoint{4.766262in}{2.299450in}}%
\pgfpathlineto{\pgfqpoint{4.758521in}{2.288415in}}%
\pgfpathlineto{\pgfqpoint{4.744497in}{2.285203in}}%
\pgfpathlineto{\pgfqpoint{4.730484in}{2.282092in}}%
\pgfpathlineto{\pgfqpoint{4.716482in}{2.279081in}}%
\pgfpathlineto{\pgfqpoint{4.702491in}{2.276171in}}%
\pgfpathlineto{\pgfqpoint{4.710236in}{2.287042in}}%
\pgfpathlineto{\pgfqpoint{4.717975in}{2.297866in}}%
\pgfpathlineto{\pgfqpoint{4.725709in}{2.308642in}}%
\pgfpathlineto{\pgfqpoint{4.733438in}{2.319369in}}%
\pgfpathclose%
\pgfusepath{fill}%
\end{pgfscope}%
\begin{pgfscope}%
\pgfpathrectangle{\pgfqpoint{1.150000in}{0.150000in}}{\pgfqpoint{5.700000in}{5.700000in}}%
\pgfusepath{clip}%
\pgfsetbuttcap%
\pgfsetroundjoin%
\definecolor{currentfill}{rgb}{0.277941,0.056324,0.381191}%
\pgfsetfillcolor{currentfill}%
\pgfsetfillopacity{0.700000}%
\pgfsetlinewidth{0.000000pt}%
\definecolor{currentstroke}{rgb}{0.000000,0.000000,0.000000}%
\pgfsetstrokecolor{currentstroke}%
\pgfsetdash{}{0pt}%
\pgfpathmoveto{\pgfqpoint{3.331082in}{2.040979in}}%
\pgfpathlineto{\pgfqpoint{3.344721in}{2.032756in}}%
\pgfpathlineto{\pgfqpoint{3.358363in}{2.024658in}}%
\pgfpathlineto{\pgfqpoint{3.372007in}{2.016683in}}%
\pgfpathlineto{\pgfqpoint{3.385654in}{2.008830in}}%
\pgfpathlineto{\pgfqpoint{3.377398in}{2.004833in}}%
\pgfpathlineto{\pgfqpoint{3.369133in}{2.000995in}}%
\pgfpathlineto{\pgfqpoint{3.360857in}{1.997319in}}%
\pgfpathlineto{\pgfqpoint{3.352572in}{1.993810in}}%
\pgfpathlineto{\pgfqpoint{3.338900in}{2.002058in}}%
\pgfpathlineto{\pgfqpoint{3.325231in}{2.010428in}}%
\pgfpathlineto{\pgfqpoint{3.311563in}{2.018922in}}%
\pgfpathlineto{\pgfqpoint{3.297898in}{2.027539in}}%
\pgfpathlineto{\pgfqpoint{3.306209in}{2.030646in}}%
\pgfpathlineto{\pgfqpoint{3.314510in}{2.033924in}}%
\pgfpathlineto{\pgfqpoint{3.322801in}{2.037369in}}%
\pgfpathlineto{\pgfqpoint{3.331082in}{2.040979in}}%
\pgfpathclose%
\pgfusepath{fill}%
\end{pgfscope}%
\begin{pgfscope}%
\pgfpathrectangle{\pgfqpoint{1.150000in}{0.150000in}}{\pgfqpoint{5.700000in}{5.700000in}}%
\pgfusepath{clip}%
\pgfsetbuttcap%
\pgfsetroundjoin%
\definecolor{currentfill}{rgb}{0.273809,0.031497,0.358853}%
\pgfsetfillcolor{currentfill}%
\pgfsetfillopacity{0.700000}%
\pgfsetlinewidth{0.000000pt}%
\definecolor{currentstroke}{rgb}{0.000000,0.000000,0.000000}%
\pgfsetstrokecolor{currentstroke}%
\pgfsetdash{}{0pt}%
\pgfpathmoveto{\pgfqpoint{4.126412in}{2.002376in}}%
\pgfpathlineto{\pgfqpoint{4.140176in}{2.001095in}}%
\pgfpathlineto{\pgfqpoint{4.153948in}{1.999919in}}%
\pgfpathlineto{\pgfqpoint{4.167728in}{1.998848in}}%
\pgfpathlineto{\pgfqpoint{4.181516in}{1.997881in}}%
\pgfpathlineto{\pgfqpoint{4.173604in}{1.988070in}}%
\pgfpathlineto{\pgfqpoint{4.165685in}{1.978291in}}%
\pgfpathlineto{\pgfqpoint{4.157762in}{1.968545in}}%
\pgfpathlineto{\pgfqpoint{4.149833in}{1.958837in}}%
\pgfpathlineto{\pgfqpoint{4.136036in}{1.960086in}}%
\pgfpathlineto{\pgfqpoint{4.122246in}{1.961440in}}%
\pgfpathlineto{\pgfqpoint{4.108464in}{1.962899in}}%
\pgfpathlineto{\pgfqpoint{4.094690in}{1.964462in}}%
\pgfpathlineto{\pgfqpoint{4.102629in}{1.973881in}}%
\pgfpathlineto{\pgfqpoint{4.110562in}{1.983341in}}%
\pgfpathlineto{\pgfqpoint{4.118490in}{1.992841in}}%
\pgfpathlineto{\pgfqpoint{4.126412in}{2.002376in}}%
\pgfpathclose%
\pgfusepath{fill}%
\end{pgfscope}%
\begin{pgfscope}%
\pgfpathrectangle{\pgfqpoint{1.150000in}{0.150000in}}{\pgfqpoint{5.700000in}{5.700000in}}%
\pgfusepath{clip}%
\pgfsetbuttcap%
\pgfsetroundjoin%
\definecolor{currentfill}{rgb}{0.151918,0.500685,0.557587}%
\pgfsetfillcolor{currentfill}%
\pgfsetfillopacity{0.700000}%
\pgfsetlinewidth{0.000000pt}%
\definecolor{currentstroke}{rgb}{0.000000,0.000000,0.000000}%
\pgfsetstrokecolor{currentstroke}%
\pgfsetdash{}{0pt}%
\pgfpathmoveto{\pgfqpoint{5.750447in}{3.038231in}}%
\pgfpathlineto{\pgfqpoint{5.764937in}{3.046035in}}%
\pgfpathlineto{\pgfqpoint{5.779444in}{3.053938in}}%
\pgfpathlineto{\pgfqpoint{5.793966in}{3.061941in}}%
\pgfpathlineto{\pgfqpoint{5.808504in}{3.070045in}}%
\pgfpathlineto{\pgfqpoint{5.801238in}{3.063606in}}%
\pgfpathlineto{\pgfqpoint{5.793963in}{3.057062in}}%
\pgfpathlineto{\pgfqpoint{5.786679in}{3.050409in}}%
\pgfpathlineto{\pgfqpoint{5.779386in}{3.043648in}}%
\pgfpathlineto{\pgfqpoint{5.764835in}{3.035453in}}%
\pgfpathlineto{\pgfqpoint{5.750300in}{3.027359in}}%
\pgfpathlineto{\pgfqpoint{5.735781in}{3.019364in}}%
\pgfpathlineto{\pgfqpoint{5.721279in}{3.011470in}}%
\pgfpathlineto{\pgfqpoint{5.728584in}{3.018315in}}%
\pgfpathlineto{\pgfqpoint{5.735880in}{3.025056in}}%
\pgfpathlineto{\pgfqpoint{5.743168in}{3.031695in}}%
\pgfpathlineto{\pgfqpoint{5.750447in}{3.038231in}}%
\pgfpathclose%
\pgfusepath{fill}%
\end{pgfscope}%
\begin{pgfscope}%
\pgfpathrectangle{\pgfqpoint{1.150000in}{0.150000in}}{\pgfqpoint{5.700000in}{5.700000in}}%
\pgfusepath{clip}%
\pgfsetbuttcap%
\pgfsetroundjoin%
\definecolor{currentfill}{rgb}{0.267004,0.004874,0.329415}%
\pgfsetfillcolor{currentfill}%
\pgfsetfillopacity{0.700000}%
\pgfsetlinewidth{0.000000pt}%
\definecolor{currentstroke}{rgb}{0.000000,0.000000,0.000000}%
\pgfsetstrokecolor{currentstroke}%
\pgfsetdash{}{0pt}%
\pgfpathmoveto{\pgfqpoint{3.669469in}{1.948326in}}%
\pgfpathlineto{\pgfqpoint{3.683136in}{1.943201in}}%
\pgfpathlineto{\pgfqpoint{3.696808in}{1.938189in}}%
\pgfpathlineto{\pgfqpoint{3.710484in}{1.933289in}}%
\pgfpathlineto{\pgfqpoint{3.724166in}{1.928501in}}%
\pgfpathlineto{\pgfqpoint{3.716079in}{1.921622in}}%
\pgfpathlineto{\pgfqpoint{3.707985in}{1.914850in}}%
\pgfpathlineto{\pgfqpoint{3.699884in}{1.908189in}}%
\pgfpathlineto{\pgfqpoint{3.691776in}{1.901640in}}%
\pgfpathlineto{\pgfqpoint{3.678076in}{1.906783in}}%
\pgfpathlineto{\pgfqpoint{3.664382in}{1.912038in}}%
\pgfpathlineto{\pgfqpoint{3.650693in}{1.917406in}}%
\pgfpathlineto{\pgfqpoint{3.637007in}{1.922886in}}%
\pgfpathlineto{\pgfqpoint{3.645134in}{1.929073in}}%
\pgfpathlineto{\pgfqpoint{3.653253in}{1.935377in}}%
\pgfpathlineto{\pgfqpoint{3.661365in}{1.941796in}}%
\pgfpathlineto{\pgfqpoint{3.669469in}{1.948326in}}%
\pgfpathclose%
\pgfusepath{fill}%
\end{pgfscope}%
\begin{pgfscope}%
\pgfpathrectangle{\pgfqpoint{1.150000in}{0.150000in}}{\pgfqpoint{5.700000in}{5.700000in}}%
\pgfusepath{clip}%
\pgfsetbuttcap%
\pgfsetroundjoin%
\definecolor{currentfill}{rgb}{0.266580,0.228262,0.514349}%
\pgfsetfillcolor{currentfill}%
\pgfsetfillopacity{0.700000}%
\pgfsetlinewidth{0.000000pt}%
\definecolor{currentstroke}{rgb}{0.000000,0.000000,0.000000}%
\pgfsetstrokecolor{currentstroke}%
\pgfsetdash{}{0pt}%
\pgfpathmoveto{\pgfqpoint{4.820289in}{2.375183in}}%
\pgfpathlineto{\pgfqpoint{4.834316in}{2.378792in}}%
\pgfpathlineto{\pgfqpoint{4.848355in}{2.382501in}}%
\pgfpathlineto{\pgfqpoint{4.862406in}{2.386311in}}%
\pgfpathlineto{\pgfqpoint{4.876468in}{2.390221in}}%
\pgfpathlineto{\pgfqpoint{4.868772in}{2.379441in}}%
\pgfpathlineto{\pgfqpoint{4.861069in}{2.368597in}}%
\pgfpathlineto{\pgfqpoint{4.853361in}{2.357692in}}%
\pgfpathlineto{\pgfqpoint{4.845647in}{2.346724in}}%
\pgfpathlineto{\pgfqpoint{4.831580in}{2.342953in}}%
\pgfpathlineto{\pgfqpoint{4.817525in}{2.339282in}}%
\pgfpathlineto{\pgfqpoint{4.803481in}{2.335712in}}%
\pgfpathlineto{\pgfqpoint{4.789450in}{2.332242in}}%
\pgfpathlineto{\pgfqpoint{4.797168in}{2.343063in}}%
\pgfpathlineto{\pgfqpoint{4.804881in}{2.353828in}}%
\pgfpathlineto{\pgfqpoint{4.812588in}{2.364535in}}%
\pgfpathlineto{\pgfqpoint{4.820289in}{2.375183in}}%
\pgfpathclose%
\pgfusepath{fill}%
\end{pgfscope}%
\begin{pgfscope}%
\pgfpathrectangle{\pgfqpoint{1.150000in}{0.150000in}}{\pgfqpoint{5.700000in}{5.700000in}}%
\pgfusepath{clip}%
\pgfsetbuttcap%
\pgfsetroundjoin%
\definecolor{currentfill}{rgb}{0.269944,0.014625,0.341379}%
\pgfsetfillcolor{currentfill}%
\pgfsetfillopacity{0.700000}%
\pgfsetlinewidth{0.000000pt}%
\definecolor{currentstroke}{rgb}{0.000000,0.000000,0.000000}%
\pgfsetstrokecolor{currentstroke}%
\pgfsetdash{}{0pt}%
\pgfpathmoveto{\pgfqpoint{3.527679in}{1.970853in}}%
\pgfpathlineto{\pgfqpoint{3.541331in}{1.964452in}}%
\pgfpathlineto{\pgfqpoint{3.554987in}{1.958168in}}%
\pgfpathlineto{\pgfqpoint{3.568647in}{1.952000in}}%
\pgfpathlineto{\pgfqpoint{3.582311in}{1.945947in}}%
\pgfpathlineto{\pgfqpoint{3.574156in}{1.940249in}}%
\pgfpathlineto{\pgfqpoint{3.565994in}{1.934681in}}%
\pgfpathlineto{\pgfqpoint{3.557824in}{1.929246in}}%
\pgfpathlineto{\pgfqpoint{3.549645in}{1.923948in}}%
\pgfpathlineto{\pgfqpoint{3.535960in}{1.930375in}}%
\pgfpathlineto{\pgfqpoint{3.522279in}{1.936917in}}%
\pgfpathlineto{\pgfqpoint{3.508602in}{1.943576in}}%
\pgfpathlineto{\pgfqpoint{3.494929in}{1.950352in}}%
\pgfpathlineto{\pgfqpoint{3.503129in}{1.955268in}}%
\pgfpathlineto{\pgfqpoint{3.511321in}{1.960326in}}%
\pgfpathlineto{\pgfqpoint{3.519504in}{1.965523in}}%
\pgfpathlineto{\pgfqpoint{3.527679in}{1.970853in}}%
\pgfpathclose%
\pgfusepath{fill}%
\end{pgfscope}%
\begin{pgfscope}%
\pgfpathrectangle{\pgfqpoint{1.150000in}{0.150000in}}{\pgfqpoint{5.700000in}{5.700000in}}%
\pgfusepath{clip}%
\pgfsetbuttcap%
\pgfsetroundjoin%
\definecolor{currentfill}{rgb}{0.204903,0.375746,0.553533}%
\pgfsetfillcolor{currentfill}%
\pgfsetfillopacity{0.700000}%
\pgfsetlinewidth{0.000000pt}%
\definecolor{currentstroke}{rgb}{0.000000,0.000000,0.000000}%
\pgfsetstrokecolor{currentstroke}%
\pgfsetdash{}{0pt}%
\pgfpathmoveto{\pgfqpoint{5.285563in}{2.711157in}}%
\pgfpathlineto{\pgfqpoint{5.299814in}{2.717212in}}%
\pgfpathlineto{\pgfqpoint{5.314079in}{2.723366in}}%
\pgfpathlineto{\pgfqpoint{5.328358in}{2.729621in}}%
\pgfpathlineto{\pgfqpoint{5.342651in}{2.735976in}}%
\pgfpathlineto{\pgfqpoint{5.335138in}{2.726826in}}%
\pgfpathlineto{\pgfqpoint{5.327618in}{2.717577in}}%
\pgfpathlineto{\pgfqpoint{5.320091in}{2.708229in}}%
\pgfpathlineto{\pgfqpoint{5.312556in}{2.698782in}}%
\pgfpathlineto{\pgfqpoint{5.298256in}{2.692453in}}%
\pgfpathlineto{\pgfqpoint{5.283970in}{2.686225in}}%
\pgfpathlineto{\pgfqpoint{5.269699in}{2.680097in}}%
\pgfpathlineto{\pgfqpoint{5.255441in}{2.674068in}}%
\pgfpathlineto{\pgfqpoint{5.262982in}{2.683482in}}%
\pgfpathlineto{\pgfqpoint{5.270516in}{2.692801in}}%
\pgfpathlineto{\pgfqpoint{5.278043in}{2.702026in}}%
\pgfpathlineto{\pgfqpoint{5.285563in}{2.711157in}}%
\pgfpathclose%
\pgfusepath{fill}%
\end{pgfscope}%
\begin{pgfscope}%
\pgfpathrectangle{\pgfqpoint{1.150000in}{0.150000in}}{\pgfqpoint{5.700000in}{5.700000in}}%
\pgfusepath{clip}%
\pgfsetbuttcap%
\pgfsetroundjoin%
\definecolor{currentfill}{rgb}{0.267004,0.004874,0.329415}%
\pgfsetfillcolor{currentfill}%
\pgfsetfillopacity{0.700000}%
\pgfsetlinewidth{0.000000pt}%
\definecolor{currentstroke}{rgb}{0.000000,0.000000,0.000000}%
\pgfsetstrokecolor{currentstroke}%
\pgfsetdash{}{0pt}%
\pgfpathmoveto{\pgfqpoint{3.811156in}{1.940319in}}%
\pgfpathlineto{\pgfqpoint{3.824849in}{1.936421in}}%
\pgfpathlineto{\pgfqpoint{3.838547in}{1.932634in}}%
\pgfpathlineto{\pgfqpoint{3.852251in}{1.928955in}}%
\pgfpathlineto{\pgfqpoint{3.865961in}{1.925386in}}%
\pgfpathlineto{\pgfqpoint{3.857933in}{1.917455in}}%
\pgfpathlineto{\pgfqpoint{3.849898in}{1.909609in}}%
\pgfpathlineto{\pgfqpoint{3.841856in}{1.901850in}}%
\pgfpathlineto{\pgfqpoint{3.833809in}{1.894182in}}%
\pgfpathlineto{\pgfqpoint{3.820084in}{1.898088in}}%
\pgfpathlineto{\pgfqpoint{3.806365in}{1.902103in}}%
\pgfpathlineto{\pgfqpoint{3.792651in}{1.906227in}}%
\pgfpathlineto{\pgfqpoint{3.778944in}{1.910461in}}%
\pgfpathlineto{\pgfqpoint{3.787007in}{1.917786in}}%
\pgfpathlineto{\pgfqpoint{3.795063in}{1.925206in}}%
\pgfpathlineto{\pgfqpoint{3.803113in}{1.932718in}}%
\pgfpathlineto{\pgfqpoint{3.811156in}{1.940319in}}%
\pgfpathclose%
\pgfusepath{fill}%
\end{pgfscope}%
\begin{pgfscope}%
\pgfpathrectangle{\pgfqpoint{1.150000in}{0.150000in}}{\pgfqpoint{5.700000in}{5.700000in}}%
\pgfusepath{clip}%
\pgfsetbuttcap%
\pgfsetroundjoin%
\definecolor{currentfill}{rgb}{0.143343,0.522773,0.556295}%
\pgfsetfillcolor{currentfill}%
\pgfsetfillopacity{0.700000}%
\pgfsetlinewidth{0.000000pt}%
\definecolor{currentstroke}{rgb}{0.000000,0.000000,0.000000}%
\pgfsetstrokecolor{currentstroke}%
\pgfsetdash{}{0pt}%
\pgfpathmoveto{\pgfqpoint{5.837478in}{3.094759in}}%
\pgfpathlineto{\pgfqpoint{5.852019in}{3.102852in}}%
\pgfpathlineto{\pgfqpoint{5.866575in}{3.111044in}}%
\pgfpathlineto{\pgfqpoint{5.881148in}{3.119337in}}%
\pgfpathlineto{\pgfqpoint{5.895738in}{3.127729in}}%
\pgfpathlineto{\pgfqpoint{5.888522in}{3.121821in}}%
\pgfpathlineto{\pgfqpoint{5.881297in}{3.115808in}}%
\pgfpathlineto{\pgfqpoint{5.874062in}{3.109687in}}%
\pgfpathlineto{\pgfqpoint{5.866819in}{3.103458in}}%
\pgfpathlineto{\pgfqpoint{5.852216in}{3.094955in}}%
\pgfpathlineto{\pgfqpoint{5.837629in}{3.086551in}}%
\pgfpathlineto{\pgfqpoint{5.823058in}{3.078248in}}%
\pgfpathlineto{\pgfqpoint{5.808504in}{3.070045in}}%
\pgfpathlineto{\pgfqpoint{5.815761in}{3.076378in}}%
\pgfpathlineto{\pgfqpoint{5.823009in}{3.082607in}}%
\pgfpathlineto{\pgfqpoint{5.830248in}{3.088734in}}%
\pgfpathlineto{\pgfqpoint{5.837478in}{3.094759in}}%
\pgfpathclose%
\pgfusepath{fill}%
\end{pgfscope}%
\begin{pgfscope}%
\pgfpathrectangle{\pgfqpoint{1.150000in}{0.150000in}}{\pgfqpoint{5.700000in}{5.700000in}}%
\pgfusepath{clip}%
\pgfsetbuttcap%
\pgfsetroundjoin%
\definecolor{currentfill}{rgb}{0.271305,0.019942,0.347269}%
\pgfsetfillcolor{currentfill}%
\pgfsetfillopacity{0.700000}%
\pgfsetlinewidth{0.000000pt}%
\definecolor{currentstroke}{rgb}{0.000000,0.000000,0.000000}%
\pgfsetstrokecolor{currentstroke}%
\pgfsetdash{}{0pt}%
\pgfpathmoveto{\pgfqpoint{4.039668in}{1.971768in}}%
\pgfpathlineto{\pgfqpoint{4.053412in}{1.969783in}}%
\pgfpathlineto{\pgfqpoint{4.067164in}{1.967903in}}%
\pgfpathlineto{\pgfqpoint{4.080923in}{1.966130in}}%
\pgfpathlineto{\pgfqpoint{4.094690in}{1.964462in}}%
\pgfpathlineto{\pgfqpoint{4.086746in}{1.955087in}}%
\pgfpathlineto{\pgfqpoint{4.078796in}{1.945758in}}%
\pgfpathlineto{\pgfqpoint{4.070841in}{1.936479in}}%
\pgfpathlineto{\pgfqpoint{4.062880in}{1.927253in}}%
\pgfpathlineto{\pgfqpoint{4.049103in}{1.929221in}}%
\pgfpathlineto{\pgfqpoint{4.035332in}{1.931296in}}%
\pgfpathlineto{\pgfqpoint{4.021569in}{1.933475in}}%
\pgfpathlineto{\pgfqpoint{4.007814in}{1.935760in}}%
\pgfpathlineto{\pgfqpoint{4.015786in}{1.944680in}}%
\pgfpathlineto{\pgfqpoint{4.023752in}{1.953656in}}%
\pgfpathlineto{\pgfqpoint{4.031713in}{1.962686in}}%
\pgfpathlineto{\pgfqpoint{4.039668in}{1.971768in}}%
\pgfpathclose%
\pgfusepath{fill}%
\end{pgfscope}%
\begin{pgfscope}%
\pgfpathrectangle{\pgfqpoint{1.150000in}{0.150000in}}{\pgfqpoint{5.700000in}{5.700000in}}%
\pgfusepath{clip}%
\pgfsetbuttcap%
\pgfsetroundjoin%
\definecolor{currentfill}{rgb}{0.283091,0.110553,0.431554}%
\pgfsetfillcolor{currentfill}%
\pgfsetfillopacity{0.700000}%
\pgfsetlinewidth{0.000000pt}%
\definecolor{currentstroke}{rgb}{0.000000,0.000000,0.000000}%
\pgfsetstrokecolor{currentstroke}%
\pgfsetdash{}{0pt}%
\pgfpathmoveto{\pgfqpoint{3.134027in}{2.140883in}}%
\pgfpathlineto{\pgfqpoint{3.147677in}{2.130721in}}%
\pgfpathlineto{\pgfqpoint{3.161328in}{2.120692in}}%
\pgfpathlineto{\pgfqpoint{3.174980in}{2.110795in}}%
\pgfpathlineto{\pgfqpoint{3.188633in}{2.101030in}}%
\pgfpathlineto{\pgfqpoint{3.180255in}{2.098916in}}%
\pgfpathlineto{\pgfqpoint{3.171865in}{2.096991in}}%
\pgfpathlineto{\pgfqpoint{3.163464in}{2.095260in}}%
\pgfpathlineto{\pgfqpoint{3.155051in}{2.093726in}}%
\pgfpathlineto{\pgfqpoint{3.141369in}{2.103909in}}%
\pgfpathlineto{\pgfqpoint{3.127687in}{2.114223in}}%
\pgfpathlineto{\pgfqpoint{3.114005in}{2.124670in}}%
\pgfpathlineto{\pgfqpoint{3.100324in}{2.135250in}}%
\pgfpathlineto{\pgfqpoint{3.108768in}{2.136358in}}%
\pgfpathlineto{\pgfqpoint{3.117200in}{2.137669in}}%
\pgfpathlineto{\pgfqpoint{3.125619in}{2.139179in}}%
\pgfpathlineto{\pgfqpoint{3.134027in}{2.140883in}}%
\pgfpathclose%
\pgfusepath{fill}%
\end{pgfscope}%
\begin{pgfscope}%
\pgfpathrectangle{\pgfqpoint{1.150000in}{0.150000in}}{\pgfqpoint{5.700000in}{5.700000in}}%
\pgfusepath{clip}%
\pgfsetbuttcap%
\pgfsetroundjoin%
\definecolor{currentfill}{rgb}{0.257322,0.256130,0.526563}%
\pgfsetfillcolor{currentfill}%
\pgfsetfillopacity{0.700000}%
\pgfsetlinewidth{0.000000pt}%
\definecolor{currentstroke}{rgb}{0.000000,0.000000,0.000000}%
\pgfsetstrokecolor{currentstroke}%
\pgfsetdash{}{0pt}%
\pgfpathmoveto{\pgfqpoint{4.907197in}{2.432686in}}%
\pgfpathlineto{\pgfqpoint{4.921267in}{2.436817in}}%
\pgfpathlineto{\pgfqpoint{4.935350in}{2.441049in}}%
\pgfpathlineto{\pgfqpoint{4.949444in}{2.445380in}}%
\pgfpathlineto{\pgfqpoint{4.963552in}{2.449812in}}%
\pgfpathlineto{\pgfqpoint{4.955883in}{2.439183in}}%
\pgfpathlineto{\pgfqpoint{4.948208in}{2.428481in}}%
\pgfpathlineto{\pgfqpoint{4.940528in}{2.417708in}}%
\pgfpathlineto{\pgfqpoint{4.932841in}{2.406865in}}%
\pgfpathlineto{\pgfqpoint{4.918729in}{2.402554in}}%
\pgfpathlineto{\pgfqpoint{4.904630in}{2.398343in}}%
\pgfpathlineto{\pgfqpoint{4.890543in}{2.394232in}}%
\pgfpathlineto{\pgfqpoint{4.876468in}{2.390221in}}%
\pgfpathlineto{\pgfqpoint{4.884159in}{2.400937in}}%
\pgfpathlineto{\pgfqpoint{4.891844in}{2.411586in}}%
\pgfpathlineto{\pgfqpoint{4.899524in}{2.422170in}}%
\pgfpathlineto{\pgfqpoint{4.907197in}{2.432686in}}%
\pgfpathclose%
\pgfusepath{fill}%
\end{pgfscope}%
\begin{pgfscope}%
\pgfpathrectangle{\pgfqpoint{1.150000in}{0.150000in}}{\pgfqpoint{5.700000in}{5.700000in}}%
\pgfusepath{clip}%
\pgfsetbuttcap%
\pgfsetroundjoin%
\definecolor{currentfill}{rgb}{0.135066,0.544853,0.554029}%
\pgfsetfillcolor{currentfill}%
\pgfsetfillopacity{0.700000}%
\pgfsetlinewidth{0.000000pt}%
\definecolor{currentstroke}{rgb}{0.000000,0.000000,0.000000}%
\pgfsetstrokecolor{currentstroke}%
\pgfsetdash{}{0pt}%
\pgfpathmoveto{\pgfqpoint{5.924508in}{3.150333in}}%
\pgfpathlineto{\pgfqpoint{5.939099in}{3.158694in}}%
\pgfpathlineto{\pgfqpoint{5.953706in}{3.167156in}}%
\pgfpathlineto{\pgfqpoint{5.968330in}{3.175717in}}%
\pgfpathlineto{\pgfqpoint{5.982970in}{3.184378in}}%
\pgfpathlineto{\pgfqpoint{5.975807in}{3.179016in}}%
\pgfpathlineto{\pgfqpoint{5.968634in}{3.173550in}}%
\pgfpathlineto{\pgfqpoint{5.961452in}{3.167978in}}%
\pgfpathlineto{\pgfqpoint{5.954260in}{3.162299in}}%
\pgfpathlineto{\pgfqpoint{5.939604in}{3.153506in}}%
\pgfpathlineto{\pgfqpoint{5.924966in}{3.144814in}}%
\pgfpathlineto{\pgfqpoint{5.910343in}{3.136221in}}%
\pgfpathlineto{\pgfqpoint{5.895738in}{3.127729in}}%
\pgfpathlineto{\pgfqpoint{5.902944in}{3.133532in}}%
\pgfpathlineto{\pgfqpoint{5.910141in}{3.139233in}}%
\pgfpathlineto{\pgfqpoint{5.917329in}{3.144833in}}%
\pgfpathlineto{\pgfqpoint{5.924508in}{3.150333in}}%
\pgfpathclose%
\pgfusepath{fill}%
\end{pgfscope}%
\begin{pgfscope}%
\pgfpathrectangle{\pgfqpoint{1.150000in}{0.150000in}}{\pgfqpoint{5.700000in}{5.700000in}}%
\pgfusepath{clip}%
\pgfsetbuttcap%
\pgfsetroundjoin%
\definecolor{currentfill}{rgb}{0.194100,0.399323,0.555565}%
\pgfsetfillcolor{currentfill}%
\pgfsetfillopacity{0.700000}%
\pgfsetlinewidth{0.000000pt}%
\definecolor{currentstroke}{rgb}{0.000000,0.000000,0.000000}%
\pgfsetstrokecolor{currentstroke}%
\pgfsetdash{}{0pt}%
\pgfpathmoveto{\pgfqpoint{5.372627in}{2.771589in}}%
\pgfpathlineto{\pgfqpoint{5.386928in}{2.778051in}}%
\pgfpathlineto{\pgfqpoint{5.401242in}{2.784612in}}%
\pgfpathlineto{\pgfqpoint{5.415571in}{2.791274in}}%
\pgfpathlineto{\pgfqpoint{5.429915in}{2.798035in}}%
\pgfpathlineto{\pgfqpoint{5.422440in}{2.789280in}}%
\pgfpathlineto{\pgfqpoint{5.414957in}{2.780421in}}%
\pgfpathlineto{\pgfqpoint{5.407467in}{2.771460in}}%
\pgfpathlineto{\pgfqpoint{5.399968in}{2.762394in}}%
\pgfpathlineto{\pgfqpoint{5.385617in}{2.755640in}}%
\pgfpathlineto{\pgfqpoint{5.371281in}{2.748985in}}%
\pgfpathlineto{\pgfqpoint{5.356959in}{2.742430in}}%
\pgfpathlineto{\pgfqpoint{5.342651in}{2.735976in}}%
\pgfpathlineto{\pgfqpoint{5.350156in}{2.745027in}}%
\pgfpathlineto{\pgfqpoint{5.357654in}{2.753979in}}%
\pgfpathlineto{\pgfqpoint{5.365145in}{2.762833in}}%
\pgfpathlineto{\pgfqpoint{5.372627in}{2.771589in}}%
\pgfpathclose%
\pgfusepath{fill}%
\end{pgfscope}%
\begin{pgfscope}%
\pgfpathrectangle{\pgfqpoint{1.150000in}{0.150000in}}{\pgfqpoint{5.700000in}{5.700000in}}%
\pgfusepath{clip}%
\pgfsetbuttcap%
\pgfsetroundjoin%
\definecolor{currentfill}{rgb}{0.225863,0.330805,0.547314}%
\pgfsetfillcolor{currentfill}%
\pgfsetfillopacity{0.700000}%
\pgfsetlinewidth{0.000000pt}%
\definecolor{currentstroke}{rgb}{0.000000,0.000000,0.000000}%
\pgfsetstrokecolor{currentstroke}%
\pgfsetdash{}{0pt}%
\pgfpathmoveto{\pgfqpoint{2.606919in}{2.612893in}}%
\pgfpathlineto{\pgfqpoint{2.620678in}{2.596846in}}%
\pgfpathlineto{\pgfqpoint{2.634432in}{2.580972in}}%
\pgfpathlineto{\pgfqpoint{2.648182in}{2.565271in}}%
\pgfpathlineto{\pgfqpoint{2.661927in}{2.549740in}}%
\pgfpathlineto{\pgfqpoint{2.653174in}{2.552340in}}%
\pgfpathlineto{\pgfqpoint{2.644405in}{2.555193in}}%
\pgfpathlineto{\pgfqpoint{2.635619in}{2.558303in}}%
\pgfpathlineto{\pgfqpoint{2.626815in}{2.561675in}}%
\pgfpathlineto{\pgfqpoint{2.613027in}{2.577661in}}%
\pgfpathlineto{\pgfqpoint{2.599234in}{2.593818in}}%
\pgfpathlineto{\pgfqpoint{2.585436in}{2.610148in}}%
\pgfpathlineto{\pgfqpoint{2.571633in}{2.626652in}}%
\pgfpathlineto{\pgfqpoint{2.580481in}{2.622816in}}%
\pgfpathlineto{\pgfqpoint{2.589311in}{2.619247in}}%
\pgfpathlineto{\pgfqpoint{2.598124in}{2.615941in}}%
\pgfpathlineto{\pgfqpoint{2.606919in}{2.612893in}}%
\pgfpathclose%
\pgfusepath{fill}%
\end{pgfscope}%
\begin{pgfscope}%
\pgfpathrectangle{\pgfqpoint{1.150000in}{0.150000in}}{\pgfqpoint{5.700000in}{5.700000in}}%
\pgfusepath{clip}%
\pgfsetbuttcap%
\pgfsetroundjoin%
\definecolor{currentfill}{rgb}{0.237441,0.305202,0.541921}%
\pgfsetfillcolor{currentfill}%
\pgfsetfillopacity{0.700000}%
\pgfsetlinewidth{0.000000pt}%
\definecolor{currentstroke}{rgb}{0.000000,0.000000,0.000000}%
\pgfsetstrokecolor{currentstroke}%
\pgfsetdash{}{0pt}%
\pgfpathmoveto{\pgfqpoint{2.661927in}{2.549740in}}%
\pgfpathlineto{\pgfqpoint{2.675667in}{2.534378in}}%
\pgfpathlineto{\pgfqpoint{2.689404in}{2.519185in}}%
\pgfpathlineto{\pgfqpoint{2.703136in}{2.504157in}}%
\pgfpathlineto{\pgfqpoint{2.716865in}{2.489295in}}%
\pgfpathlineto{\pgfqpoint{2.708154in}{2.491450in}}%
\pgfpathlineto{\pgfqpoint{2.699427in}{2.493852in}}%
\pgfpathlineto{\pgfqpoint{2.690683in}{2.496506in}}%
\pgfpathlineto{\pgfqpoint{2.681923in}{2.499417in}}%
\pgfpathlineto{\pgfqpoint{2.668152in}{2.514732in}}%
\pgfpathlineto{\pgfqpoint{2.654378in}{2.530212in}}%
\pgfpathlineto{\pgfqpoint{2.640599in}{2.545860in}}%
\pgfpathlineto{\pgfqpoint{2.626815in}{2.561675in}}%
\pgfpathlineto{\pgfqpoint{2.635619in}{2.558303in}}%
\pgfpathlineto{\pgfqpoint{2.644405in}{2.555193in}}%
\pgfpathlineto{\pgfqpoint{2.653174in}{2.552340in}}%
\pgfpathlineto{\pgfqpoint{2.661927in}{2.549740in}}%
\pgfpathclose%
\pgfusepath{fill}%
\end{pgfscope}%
\begin{pgfscope}%
\pgfpathrectangle{\pgfqpoint{1.150000in}{0.150000in}}{\pgfqpoint{5.700000in}{5.700000in}}%
\pgfusepath{clip}%
\pgfsetbuttcap%
\pgfsetroundjoin%
\definecolor{currentfill}{rgb}{0.214298,0.355619,0.551184}%
\pgfsetfillcolor{currentfill}%
\pgfsetfillopacity{0.700000}%
\pgfsetlinewidth{0.000000pt}%
\definecolor{currentstroke}{rgb}{0.000000,0.000000,0.000000}%
\pgfsetstrokecolor{currentstroke}%
\pgfsetdash{}{0pt}%
\pgfpathmoveto{\pgfqpoint{2.551832in}{2.678848in}}%
\pgfpathlineto{\pgfqpoint{2.565611in}{2.662091in}}%
\pgfpathlineto{\pgfqpoint{2.579386in}{2.645514in}}%
\pgfpathlineto{\pgfqpoint{2.593155in}{2.629115in}}%
\pgfpathlineto{\pgfqpoint{2.606919in}{2.612893in}}%
\pgfpathlineto{\pgfqpoint{2.598124in}{2.615941in}}%
\pgfpathlineto{\pgfqpoint{2.589311in}{2.619247in}}%
\pgfpathlineto{\pgfqpoint{2.580481in}{2.622816in}}%
\pgfpathlineto{\pgfqpoint{2.571633in}{2.626652in}}%
\pgfpathlineto{\pgfqpoint{2.557824in}{2.643331in}}%
\pgfpathlineto{\pgfqpoint{2.544010in}{2.660189in}}%
\pgfpathlineto{\pgfqpoint{2.530191in}{2.677225in}}%
\pgfpathlineto{\pgfqpoint{2.516365in}{2.694441in}}%
\pgfpathlineto{\pgfqpoint{2.525259in}{2.690138in}}%
\pgfpathlineto{\pgfqpoint{2.534135in}{2.686108in}}%
\pgfpathlineto{\pgfqpoint{2.542992in}{2.682346in}}%
\pgfpathlineto{\pgfqpoint{2.551832in}{2.678848in}}%
\pgfpathclose%
\pgfusepath{fill}%
\end{pgfscope}%
\begin{pgfscope}%
\pgfpathrectangle{\pgfqpoint{1.150000in}{0.150000in}}{\pgfqpoint{5.700000in}{5.700000in}}%
\pgfusepath{clip}%
\pgfsetbuttcap%
\pgfsetroundjoin%
\definecolor{currentfill}{rgb}{0.123463,0.581687,0.547445}%
\pgfsetfillcolor{currentfill}%
\pgfsetfillopacity{0.700000}%
\pgfsetlinewidth{0.000000pt}%
\definecolor{currentstroke}{rgb}{0.000000,0.000000,0.000000}%
\pgfsetstrokecolor{currentstroke}%
\pgfsetdash{}{0pt}%
\pgfpathmoveto{\pgfqpoint{6.098528in}{3.258096in}}%
\pgfpathlineto{\pgfqpoint{6.113218in}{3.266935in}}%
\pgfpathlineto{\pgfqpoint{6.127925in}{3.275874in}}%
\pgfpathlineto{\pgfqpoint{6.142649in}{3.284913in}}%
\pgfpathlineto{\pgfqpoint{6.135593in}{3.280631in}}%
\pgfpathlineto{\pgfqpoint{6.128527in}{3.276251in}}%
\pgfpathlineto{\pgfqpoint{6.121452in}{3.271772in}}%
\pgfpathlineto{\pgfqpoint{6.114366in}{3.267191in}}%
\pgfpathlineto{\pgfqpoint{6.099624in}{3.257980in}}%
\pgfpathlineto{\pgfqpoint{6.084899in}{3.248870in}}%
\pgfpathlineto{\pgfqpoint{6.070191in}{3.239859in}}%
\pgfpathlineto{\pgfqpoint{6.077289in}{3.244563in}}%
\pgfpathlineto{\pgfqpoint{6.084378in}{3.249169in}}%
\pgfpathlineto{\pgfqpoint{6.091458in}{3.253679in}}%
\pgfpathlineto{\pgfqpoint{6.098528in}{3.258096in}}%
\pgfpathclose%
\pgfusepath{fill}%
\end{pgfscope}%
\begin{pgfscope}%
\pgfpathrectangle{\pgfqpoint{1.150000in}{0.150000in}}{\pgfqpoint{5.700000in}{5.700000in}}%
\pgfusepath{clip}%
\pgfsetbuttcap%
\pgfsetroundjoin%
\definecolor{currentfill}{rgb}{0.248629,0.278775,0.534556}%
\pgfsetfillcolor{currentfill}%
\pgfsetfillopacity{0.700000}%
\pgfsetlinewidth{0.000000pt}%
\definecolor{currentstroke}{rgb}{0.000000,0.000000,0.000000}%
\pgfsetstrokecolor{currentstroke}%
\pgfsetdash{}{0pt}%
\pgfpathmoveto{\pgfqpoint{2.716865in}{2.489295in}}%
\pgfpathlineto{\pgfqpoint{2.730590in}{2.474596in}}%
\pgfpathlineto{\pgfqpoint{2.744312in}{2.460059in}}%
\pgfpathlineto{\pgfqpoint{2.758030in}{2.445684in}}%
\pgfpathlineto{\pgfqpoint{2.771745in}{2.431468in}}%
\pgfpathlineto{\pgfqpoint{2.763074in}{2.433180in}}%
\pgfpathlineto{\pgfqpoint{2.754388in}{2.435135in}}%
\pgfpathlineto{\pgfqpoint{2.745685in}{2.437336in}}%
\pgfpathlineto{\pgfqpoint{2.736967in}{2.439788in}}%
\pgfpathlineto{\pgfqpoint{2.723211in}{2.454453in}}%
\pgfpathlineto{\pgfqpoint{2.709452in}{2.469279in}}%
\pgfpathlineto{\pgfqpoint{2.695689in}{2.484267in}}%
\pgfpathlineto{\pgfqpoint{2.681923in}{2.499417in}}%
\pgfpathlineto{\pgfqpoint{2.690683in}{2.496506in}}%
\pgfpathlineto{\pgfqpoint{2.699427in}{2.493852in}}%
\pgfpathlineto{\pgfqpoint{2.708154in}{2.491450in}}%
\pgfpathlineto{\pgfqpoint{2.716865in}{2.489295in}}%
\pgfpathclose%
\pgfusepath{fill}%
\end{pgfscope}%
\begin{pgfscope}%
\pgfpathrectangle{\pgfqpoint{1.150000in}{0.150000in}}{\pgfqpoint{5.700000in}{5.700000in}}%
\pgfusepath{clip}%
\pgfsetbuttcap%
\pgfsetroundjoin%
\definecolor{currentfill}{rgb}{0.276022,0.044167,0.370164}%
\pgfsetfillcolor{currentfill}%
\pgfsetfillopacity{0.700000}%
\pgfsetlinewidth{0.000000pt}%
\definecolor{currentstroke}{rgb}{0.000000,0.000000,0.000000}%
\pgfsetstrokecolor{currentstroke}%
\pgfsetdash{}{0pt}%
\pgfpathmoveto{\pgfqpoint{3.385654in}{2.008830in}}%
\pgfpathlineto{\pgfqpoint{3.399303in}{2.001100in}}%
\pgfpathlineto{\pgfqpoint{3.412955in}{1.993490in}}%
\pgfpathlineto{\pgfqpoint{3.426610in}{1.986002in}}%
\pgfpathlineto{\pgfqpoint{3.440267in}{1.978634in}}%
\pgfpathlineto{\pgfqpoint{3.432035in}{1.974250in}}%
\pgfpathlineto{\pgfqpoint{3.423794in}{1.970019in}}%
\pgfpathlineto{\pgfqpoint{3.415543in}{1.965947in}}%
\pgfpathlineto{\pgfqpoint{3.407283in}{1.962036in}}%
\pgfpathlineto{\pgfqpoint{3.393601in}{1.969799in}}%
\pgfpathlineto{\pgfqpoint{3.379922in}{1.977681in}}%
\pgfpathlineto{\pgfqpoint{3.366246in}{1.985685in}}%
\pgfpathlineto{\pgfqpoint{3.352572in}{1.993810in}}%
\pgfpathlineto{\pgfqpoint{3.360857in}{1.997319in}}%
\pgfpathlineto{\pgfqpoint{3.369133in}{2.000995in}}%
\pgfpathlineto{\pgfqpoint{3.377398in}{2.004833in}}%
\pgfpathlineto{\pgfqpoint{3.385654in}{2.008830in}}%
\pgfpathclose%
\pgfusepath{fill}%
\end{pgfscope}%
\begin{pgfscope}%
\pgfpathrectangle{\pgfqpoint{1.150000in}{0.150000in}}{\pgfqpoint{5.700000in}{5.700000in}}%
\pgfusepath{clip}%
\pgfsetbuttcap%
\pgfsetroundjoin%
\definecolor{currentfill}{rgb}{0.201239,0.383670,0.554294}%
\pgfsetfillcolor{currentfill}%
\pgfsetfillopacity{0.700000}%
\pgfsetlinewidth{0.000000pt}%
\definecolor{currentstroke}{rgb}{0.000000,0.000000,0.000000}%
\pgfsetstrokecolor{currentstroke}%
\pgfsetdash{}{0pt}%
\pgfpathmoveto{\pgfqpoint{2.496653in}{2.747705in}}%
\pgfpathlineto{\pgfqpoint{2.510457in}{2.730213in}}%
\pgfpathlineto{\pgfqpoint{2.524254in}{2.712907in}}%
\pgfpathlineto{\pgfqpoint{2.538046in}{2.695786in}}%
\pgfpathlineto{\pgfqpoint{2.551832in}{2.678848in}}%
\pgfpathlineto{\pgfqpoint{2.542992in}{2.682346in}}%
\pgfpathlineto{\pgfqpoint{2.534135in}{2.686108in}}%
\pgfpathlineto{\pgfqpoint{2.525259in}{2.690138in}}%
\pgfpathlineto{\pgfqpoint{2.516365in}{2.694441in}}%
\pgfpathlineto{\pgfqpoint{2.502533in}{2.711840in}}%
\pgfpathlineto{\pgfqpoint{2.488695in}{2.729423in}}%
\pgfpathlineto{\pgfqpoint{2.474851in}{2.747191in}}%
\pgfpathlineto{\pgfqpoint{2.461000in}{2.765146in}}%
\pgfpathlineto{\pgfqpoint{2.469942in}{2.760373in}}%
\pgfpathlineto{\pgfqpoint{2.478864in}{2.755878in}}%
\pgfpathlineto{\pgfqpoint{2.487768in}{2.751657in}}%
\pgfpathlineto{\pgfqpoint{2.496653in}{2.747705in}}%
\pgfpathclose%
\pgfusepath{fill}%
\end{pgfscope}%
\begin{pgfscope}%
\pgfpathrectangle{\pgfqpoint{1.150000in}{0.150000in}}{\pgfqpoint{5.700000in}{5.700000in}}%
\pgfusepath{clip}%
\pgfsetbuttcap%
\pgfsetroundjoin%
\definecolor{currentfill}{rgb}{0.128729,0.563265,0.551229}%
\pgfsetfillcolor{currentfill}%
\pgfsetfillopacity{0.700000}%
\pgfsetlinewidth{0.000000pt}%
\definecolor{currentstroke}{rgb}{0.000000,0.000000,0.000000}%
\pgfsetstrokecolor{currentstroke}%
\pgfsetdash{}{0pt}%
\pgfpathmoveto{\pgfqpoint{6.011528in}{3.204818in}}%
\pgfpathlineto{\pgfqpoint{6.026169in}{3.213428in}}%
\pgfpathlineto{\pgfqpoint{6.040826in}{3.222139in}}%
\pgfpathlineto{\pgfqpoint{6.055500in}{3.230949in}}%
\pgfpathlineto{\pgfqpoint{6.070191in}{3.239859in}}%
\pgfpathlineto{\pgfqpoint{6.063082in}{3.235055in}}%
\pgfpathlineto{\pgfqpoint{6.055964in}{3.230149in}}%
\pgfpathlineto{\pgfqpoint{6.048837in}{3.225140in}}%
\pgfpathlineto{\pgfqpoint{6.041699in}{3.220025in}}%
\pgfpathlineto{\pgfqpoint{6.026991in}{3.210963in}}%
\pgfpathlineto{\pgfqpoint{6.012301in}{3.202001in}}%
\pgfpathlineto{\pgfqpoint{5.997627in}{3.193139in}}%
\pgfpathlineto{\pgfqpoint{5.982970in}{3.184378in}}%
\pgfpathlineto{\pgfqpoint{5.990123in}{3.189637in}}%
\pgfpathlineto{\pgfqpoint{5.997268in}{3.194796in}}%
\pgfpathlineto{\pgfqpoint{6.004403in}{3.199855in}}%
\pgfpathlineto{\pgfqpoint{6.011528in}{3.204818in}}%
\pgfpathclose%
\pgfusepath{fill}%
\end{pgfscope}%
\begin{pgfscope}%
\pgfpathrectangle{\pgfqpoint{1.150000in}{0.150000in}}{\pgfqpoint{5.700000in}{5.700000in}}%
\pgfusepath{clip}%
\pgfsetbuttcap%
\pgfsetroundjoin%
\definecolor{currentfill}{rgb}{0.244972,0.287675,0.537260}%
\pgfsetfillcolor{currentfill}%
\pgfsetfillopacity{0.700000}%
\pgfsetlinewidth{0.000000pt}%
\definecolor{currentstroke}{rgb}{0.000000,0.000000,0.000000}%
\pgfsetstrokecolor{currentstroke}%
\pgfsetdash{}{0pt}%
\pgfpathmoveto{\pgfqpoint{4.994166in}{2.491593in}}%
\pgfpathlineto{\pgfqpoint{5.008280in}{2.496228in}}%
\pgfpathlineto{\pgfqpoint{5.022408in}{2.500962in}}%
\pgfpathlineto{\pgfqpoint{5.036549in}{2.505797in}}%
\pgfpathlineto{\pgfqpoint{5.050702in}{2.510732in}}%
\pgfpathlineto{\pgfqpoint{5.043062in}{2.500304in}}%
\pgfpathlineto{\pgfqpoint{5.035417in}{2.489796in}}%
\pgfpathlineto{\pgfqpoint{5.027765in}{2.479208in}}%
\pgfpathlineto{\pgfqpoint{5.020107in}{2.468543in}}%
\pgfpathlineto{\pgfqpoint{5.005949in}{2.463710in}}%
\pgfpathlineto{\pgfqpoint{4.991804in}{2.458977in}}%
\pgfpathlineto{\pgfqpoint{4.977671in}{2.454345in}}%
\pgfpathlineto{\pgfqpoint{4.963552in}{2.449812in}}%
\pgfpathlineto{\pgfqpoint{4.971214in}{2.460369in}}%
\pgfpathlineto{\pgfqpoint{4.978871in}{2.470852in}}%
\pgfpathlineto{\pgfqpoint{4.986521in}{2.481260in}}%
\pgfpathlineto{\pgfqpoint{4.994166in}{2.491593in}}%
\pgfpathclose%
\pgfusepath{fill}%
\end{pgfscope}%
\begin{pgfscope}%
\pgfpathrectangle{\pgfqpoint{1.150000in}{0.150000in}}{\pgfqpoint{5.700000in}{5.700000in}}%
\pgfusepath{clip}%
\pgfsetbuttcap%
\pgfsetroundjoin%
\definecolor{currentfill}{rgb}{0.258965,0.251537,0.524736}%
\pgfsetfillcolor{currentfill}%
\pgfsetfillopacity{0.700000}%
\pgfsetlinewidth{0.000000pt}%
\definecolor{currentstroke}{rgb}{0.000000,0.000000,0.000000}%
\pgfsetstrokecolor{currentstroke}%
\pgfsetdash{}{0pt}%
\pgfpathmoveto{\pgfqpoint{2.771745in}{2.431468in}}%
\pgfpathlineto{\pgfqpoint{2.785457in}{2.417411in}}%
\pgfpathlineto{\pgfqpoint{2.799166in}{2.403511in}}%
\pgfpathlineto{\pgfqpoint{2.812873in}{2.389767in}}%
\pgfpathlineto{\pgfqpoint{2.826576in}{2.376178in}}%
\pgfpathlineto{\pgfqpoint{2.817944in}{2.377450in}}%
\pgfpathlineto{\pgfqpoint{2.809297in}{2.378959in}}%
\pgfpathlineto{\pgfqpoint{2.800635in}{2.380709in}}%
\pgfpathlineto{\pgfqpoint{2.791957in}{2.382705in}}%
\pgfpathlineto{\pgfqpoint{2.778214in}{2.396741in}}%
\pgfpathlineto{\pgfqpoint{2.764468in}{2.410933in}}%
\pgfpathlineto{\pgfqpoint{2.750719in}{2.425282in}}%
\pgfpathlineto{\pgfqpoint{2.736967in}{2.439788in}}%
\pgfpathlineto{\pgfqpoint{2.745685in}{2.437336in}}%
\pgfpathlineto{\pgfqpoint{2.754388in}{2.435135in}}%
\pgfpathlineto{\pgfqpoint{2.763074in}{2.433180in}}%
\pgfpathlineto{\pgfqpoint{2.771745in}{2.431468in}}%
\pgfpathclose%
\pgfusepath{fill}%
\end{pgfscope}%
\begin{pgfscope}%
\pgfpathrectangle{\pgfqpoint{1.150000in}{0.150000in}}{\pgfqpoint{5.700000in}{5.700000in}}%
\pgfusepath{clip}%
\pgfsetbuttcap%
\pgfsetroundjoin%
\definecolor{currentfill}{rgb}{0.268510,0.009605,0.335427}%
\pgfsetfillcolor{currentfill}%
\pgfsetfillopacity{0.700000}%
\pgfsetlinewidth{0.000000pt}%
\definecolor{currentstroke}{rgb}{0.000000,0.000000,0.000000}%
\pgfsetstrokecolor{currentstroke}%
\pgfsetdash{}{0pt}%
\pgfpathmoveto{\pgfqpoint{3.952859in}{1.945966in}}%
\pgfpathlineto{\pgfqpoint{3.966588in}{1.943254in}}%
\pgfpathlineto{\pgfqpoint{3.980323in}{1.940650in}}%
\pgfpathlineto{\pgfqpoint{3.994065in}{1.938152in}}%
\pgfpathlineto{\pgfqpoint{4.007814in}{1.935760in}}%
\pgfpathlineto{\pgfqpoint{3.999836in}{1.926901in}}%
\pgfpathlineto{\pgfqpoint{3.991852in}{1.918104in}}%
\pgfpathlineto{\pgfqpoint{3.983862in}{1.909374in}}%
\pgfpathlineto{\pgfqpoint{3.975867in}{1.900712in}}%
\pgfpathlineto{\pgfqpoint{3.962106in}{1.903422in}}%
\pgfpathlineto{\pgfqpoint{3.948351in}{1.906238in}}%
\pgfpathlineto{\pgfqpoint{3.934604in}{1.909161in}}%
\pgfpathlineto{\pgfqpoint{3.920862in}{1.912191in}}%
\pgfpathlineto{\pgfqpoint{3.928871in}{1.920527in}}%
\pgfpathlineto{\pgfqpoint{3.936873in}{1.928938in}}%
\pgfpathlineto{\pgfqpoint{3.944869in}{1.937418in}}%
\pgfpathlineto{\pgfqpoint{3.952859in}{1.945966in}}%
\pgfpathclose%
\pgfusepath{fill}%
\end{pgfscope}%
\begin{pgfscope}%
\pgfpathrectangle{\pgfqpoint{1.150000in}{0.150000in}}{\pgfqpoint{5.700000in}{5.700000in}}%
\pgfusepath{clip}%
\pgfsetbuttcap%
\pgfsetroundjoin%
\definecolor{currentfill}{rgb}{0.188923,0.410910,0.556326}%
\pgfsetfillcolor{currentfill}%
\pgfsetfillopacity{0.700000}%
\pgfsetlinewidth{0.000000pt}%
\definecolor{currentstroke}{rgb}{0.000000,0.000000,0.000000}%
\pgfsetstrokecolor{currentstroke}%
\pgfsetdash{}{0pt}%
\pgfpathmoveto{\pgfqpoint{2.441370in}{2.819573in}}%
\pgfpathlineto{\pgfqpoint{2.455201in}{2.801318in}}%
\pgfpathlineto{\pgfqpoint{2.469025in}{2.783256in}}%
\pgfpathlineto{\pgfqpoint{2.482842in}{2.765385in}}%
\pgfpathlineto{\pgfqpoint{2.496653in}{2.747705in}}%
\pgfpathlineto{\pgfqpoint{2.487768in}{2.751657in}}%
\pgfpathlineto{\pgfqpoint{2.478864in}{2.755878in}}%
\pgfpathlineto{\pgfqpoint{2.469942in}{2.760373in}}%
\pgfpathlineto{\pgfqpoint{2.461000in}{2.765146in}}%
\pgfpathlineto{\pgfqpoint{2.447142in}{2.783290in}}%
\pgfpathlineto{\pgfqpoint{2.433278in}{2.801625in}}%
\pgfpathlineto{\pgfqpoint{2.419406in}{2.820153in}}%
\pgfpathlineto{\pgfqpoint{2.405526in}{2.838875in}}%
\pgfpathlineto{\pgfqpoint{2.414517in}{2.833628in}}%
\pgfpathlineto{\pgfqpoint{2.423487in}{2.828665in}}%
\pgfpathlineto{\pgfqpoint{2.432438in}{2.823982in}}%
\pgfpathlineto{\pgfqpoint{2.441370in}{2.819573in}}%
\pgfpathclose%
\pgfusepath{fill}%
\end{pgfscope}%
\begin{pgfscope}%
\pgfpathrectangle{\pgfqpoint{1.150000in}{0.150000in}}{\pgfqpoint{5.700000in}{5.700000in}}%
\pgfusepath{clip}%
\pgfsetbuttcap%
\pgfsetroundjoin%
\definecolor{currentfill}{rgb}{0.182256,0.426184,0.557120}%
\pgfsetfillcolor{currentfill}%
\pgfsetfillopacity{0.700000}%
\pgfsetlinewidth{0.000000pt}%
\definecolor{currentstroke}{rgb}{0.000000,0.000000,0.000000}%
\pgfsetstrokecolor{currentstroke}%
\pgfsetdash{}{0pt}%
\pgfpathmoveto{\pgfqpoint{5.459738in}{2.832035in}}%
\pgfpathlineto{\pgfqpoint{5.474088in}{2.838884in}}%
\pgfpathlineto{\pgfqpoint{5.488453in}{2.845833in}}%
\pgfpathlineto{\pgfqpoint{5.502832in}{2.852882in}}%
\pgfpathlineto{\pgfqpoint{5.517227in}{2.860031in}}%
\pgfpathlineto{\pgfqpoint{5.509792in}{2.851704in}}%
\pgfpathlineto{\pgfqpoint{5.502348in}{2.843270in}}%
\pgfpathlineto{\pgfqpoint{5.494897in}{2.834730in}}%
\pgfpathlineto{\pgfqpoint{5.487438in}{2.826082in}}%
\pgfpathlineto{\pgfqpoint{5.473035in}{2.818920in}}%
\pgfpathlineto{\pgfqpoint{5.458647in}{2.811859in}}%
\pgfpathlineto{\pgfqpoint{5.444273in}{2.804897in}}%
\pgfpathlineto{\pgfqpoint{5.429915in}{2.798035in}}%
\pgfpathlineto{\pgfqpoint{5.437382in}{2.806688in}}%
\pgfpathlineto{\pgfqpoint{5.444842in}{2.815239in}}%
\pgfpathlineto{\pgfqpoint{5.452294in}{2.823687in}}%
\pgfpathlineto{\pgfqpoint{5.459738in}{2.832035in}}%
\pgfpathclose%
\pgfusepath{fill}%
\end{pgfscope}%
\begin{pgfscope}%
\pgfpathrectangle{\pgfqpoint{1.150000in}{0.150000in}}{\pgfqpoint{5.700000in}{5.700000in}}%
\pgfusepath{clip}%
\pgfsetbuttcap%
\pgfsetroundjoin%
\definecolor{currentfill}{rgb}{0.282327,0.094955,0.417331}%
\pgfsetfillcolor{currentfill}%
\pgfsetfillopacity{0.700000}%
\pgfsetlinewidth{0.000000pt}%
\definecolor{currentstroke}{rgb}{0.000000,0.000000,0.000000}%
\pgfsetstrokecolor{currentstroke}%
\pgfsetdash{}{0pt}%
\pgfpathmoveto{\pgfqpoint{3.188633in}{2.101030in}}%
\pgfpathlineto{\pgfqpoint{3.202286in}{2.091394in}}%
\pgfpathlineto{\pgfqpoint{3.215941in}{2.081889in}}%
\pgfpathlineto{\pgfqpoint{3.229597in}{2.072513in}}%
\pgfpathlineto{\pgfqpoint{3.243254in}{2.063264in}}%
\pgfpathlineto{\pgfqpoint{3.234904in}{2.060741in}}%
\pgfpathlineto{\pgfqpoint{3.226544in}{2.058403in}}%
\pgfpathlineto{\pgfqpoint{3.218173in}{2.056252in}}%
\pgfpathlineto{\pgfqpoint{3.209790in}{2.054295in}}%
\pgfpathlineto{\pgfqpoint{3.196104in}{2.063959in}}%
\pgfpathlineto{\pgfqpoint{3.182419in}{2.073752in}}%
\pgfpathlineto{\pgfqpoint{3.168735in}{2.083674in}}%
\pgfpathlineto{\pgfqpoint{3.155051in}{2.093726in}}%
\pgfpathlineto{\pgfqpoint{3.163464in}{2.095260in}}%
\pgfpathlineto{\pgfqpoint{3.171865in}{2.096991in}}%
\pgfpathlineto{\pgfqpoint{3.180255in}{2.098916in}}%
\pgfpathlineto{\pgfqpoint{3.188633in}{2.101030in}}%
\pgfpathclose%
\pgfusepath{fill}%
\end{pgfscope}%
\begin{pgfscope}%
\pgfpathrectangle{\pgfqpoint{1.150000in}{0.150000in}}{\pgfqpoint{5.700000in}{5.700000in}}%
\pgfusepath{clip}%
\pgfsetbuttcap%
\pgfsetroundjoin%
\definecolor{currentfill}{rgb}{0.266580,0.228262,0.514349}%
\pgfsetfillcolor{currentfill}%
\pgfsetfillopacity{0.700000}%
\pgfsetlinewidth{0.000000pt}%
\definecolor{currentstroke}{rgb}{0.000000,0.000000,0.000000}%
\pgfsetstrokecolor{currentstroke}%
\pgfsetdash{}{0pt}%
\pgfpathmoveto{\pgfqpoint{2.826576in}{2.376178in}}%
\pgfpathlineto{\pgfqpoint{2.840278in}{2.362743in}}%
\pgfpathlineto{\pgfqpoint{2.853977in}{2.349460in}}%
\pgfpathlineto{\pgfqpoint{2.867674in}{2.336329in}}%
\pgfpathlineto{\pgfqpoint{2.881369in}{2.323348in}}%
\pgfpathlineto{\pgfqpoint{2.872774in}{2.324181in}}%
\pgfpathlineto{\pgfqpoint{2.864165in}{2.325247in}}%
\pgfpathlineto{\pgfqpoint{2.855541in}{2.326548in}}%
\pgfpathlineto{\pgfqpoint{2.846902in}{2.328090in}}%
\pgfpathlineto{\pgfqpoint{2.833170in}{2.341516in}}%
\pgfpathlineto{\pgfqpoint{2.819434in}{2.355094in}}%
\pgfpathlineto{\pgfqpoint{2.805697in}{2.368823in}}%
\pgfpathlineto{\pgfqpoint{2.791957in}{2.382705in}}%
\pgfpathlineto{\pgfqpoint{2.800635in}{2.380709in}}%
\pgfpathlineto{\pgfqpoint{2.809297in}{2.378959in}}%
\pgfpathlineto{\pgfqpoint{2.817944in}{2.377450in}}%
\pgfpathlineto{\pgfqpoint{2.826576in}{2.376178in}}%
\pgfpathclose%
\pgfusepath{fill}%
\end{pgfscope}%
\begin{pgfscope}%
\pgfpathrectangle{\pgfqpoint{1.150000in}{0.150000in}}{\pgfqpoint{5.700000in}{5.700000in}}%
\pgfusepath{clip}%
\pgfsetbuttcap%
\pgfsetroundjoin%
\definecolor{currentfill}{rgb}{0.283091,0.110553,0.431554}%
\pgfsetfillcolor{currentfill}%
\pgfsetfillopacity{0.700000}%
\pgfsetlinewidth{0.000000pt}%
\definecolor{currentstroke}{rgb}{0.000000,0.000000,0.000000}%
\pgfsetstrokecolor{currentstroke}%
\pgfsetdash{}{0pt}%
\pgfpathmoveto{\pgfqpoint{4.441905in}{2.122542in}}%
\pgfpathlineto{\pgfqpoint{4.455786in}{2.123655in}}%
\pgfpathlineto{\pgfqpoint{4.469677in}{2.124871in}}%
\pgfpathlineto{\pgfqpoint{4.483578in}{2.126188in}}%
\pgfpathlineto{\pgfqpoint{4.497488in}{2.127607in}}%
\pgfpathlineto{\pgfqpoint{4.489666in}{2.116715in}}%
\pgfpathlineto{\pgfqpoint{4.481838in}{2.105808in}}%
\pgfpathlineto{\pgfqpoint{4.474006in}{2.094890in}}%
\pgfpathlineto{\pgfqpoint{4.466168in}{2.083961in}}%
\pgfpathlineto{\pgfqpoint{4.452252in}{2.082771in}}%
\pgfpathlineto{\pgfqpoint{4.438345in}{2.081684in}}%
\pgfpathlineto{\pgfqpoint{4.424448in}{2.080698in}}%
\pgfpathlineto{\pgfqpoint{4.410561in}{2.079813in}}%
\pgfpathlineto{\pgfqpoint{4.418404in}{2.090506in}}%
\pgfpathlineto{\pgfqpoint{4.426243in}{2.101193in}}%
\pgfpathlineto{\pgfqpoint{4.434077in}{2.111872in}}%
\pgfpathlineto{\pgfqpoint{4.441905in}{2.122542in}}%
\pgfpathclose%
\pgfusepath{fill}%
\end{pgfscope}%
\begin{pgfscope}%
\pgfpathrectangle{\pgfqpoint{1.150000in}{0.150000in}}{\pgfqpoint{5.700000in}{5.700000in}}%
\pgfusepath{clip}%
\pgfsetbuttcap%
\pgfsetroundjoin%
\definecolor{currentfill}{rgb}{0.281924,0.089666,0.412415}%
\pgfsetfillcolor{currentfill}%
\pgfsetfillopacity{0.700000}%
\pgfsetlinewidth{0.000000pt}%
\definecolor{currentstroke}{rgb}{0.000000,0.000000,0.000000}%
\pgfsetstrokecolor{currentstroke}%
\pgfsetdash{}{0pt}%
\pgfpathmoveto{\pgfqpoint{4.355106in}{2.077298in}}%
\pgfpathlineto{\pgfqpoint{4.368956in}{2.077773in}}%
\pgfpathlineto{\pgfqpoint{4.382815in}{2.078351in}}%
\pgfpathlineto{\pgfqpoint{4.396683in}{2.079031in}}%
\pgfpathlineto{\pgfqpoint{4.410561in}{2.079813in}}%
\pgfpathlineto{\pgfqpoint{4.402712in}{2.069118in}}%
\pgfpathlineto{\pgfqpoint{4.394858in}{2.058421in}}%
\pgfpathlineto{\pgfqpoint{4.387000in}{2.047725in}}%
\pgfpathlineto{\pgfqpoint{4.379136in}{2.037032in}}%
\pgfpathlineto{\pgfqpoint{4.365251in}{2.036497in}}%
\pgfpathlineto{\pgfqpoint{4.351376in}{2.036065in}}%
\pgfpathlineto{\pgfqpoint{4.337510in}{2.035734in}}%
\pgfpathlineto{\pgfqpoint{4.323654in}{2.035506in}}%
\pgfpathlineto{\pgfqpoint{4.331524in}{2.045945in}}%
\pgfpathlineto{\pgfqpoint{4.339390in}{2.056391in}}%
\pgfpathlineto{\pgfqpoint{4.347250in}{2.066843in}}%
\pgfpathlineto{\pgfqpoint{4.355106in}{2.077298in}}%
\pgfpathclose%
\pgfusepath{fill}%
\end{pgfscope}%
\begin{pgfscope}%
\pgfpathrectangle{\pgfqpoint{1.150000in}{0.150000in}}{\pgfqpoint{5.700000in}{5.700000in}}%
\pgfusepath{clip}%
\pgfsetbuttcap%
\pgfsetroundjoin%
\definecolor{currentfill}{rgb}{0.267004,0.004874,0.329415}%
\pgfsetfillcolor{currentfill}%
\pgfsetfillopacity{0.700000}%
\pgfsetlinewidth{0.000000pt}%
\definecolor{currentstroke}{rgb}{0.000000,0.000000,0.000000}%
\pgfsetstrokecolor{currentstroke}%
\pgfsetdash{}{0pt}%
\pgfpathmoveto{\pgfqpoint{3.724166in}{1.928501in}}%
\pgfpathlineto{\pgfqpoint{3.737852in}{1.923824in}}%
\pgfpathlineto{\pgfqpoint{3.751544in}{1.919259in}}%
\pgfpathlineto{\pgfqpoint{3.765241in}{1.914805in}}%
\pgfpathlineto{\pgfqpoint{3.778944in}{1.910461in}}%
\pgfpathlineto{\pgfqpoint{3.770873in}{1.903235in}}%
\pgfpathlineto{\pgfqpoint{3.762796in}{1.896110in}}%
\pgfpathlineto{\pgfqpoint{3.754712in}{1.889091in}}%
\pgfpathlineto{\pgfqpoint{3.746621in}{1.882180in}}%
\pgfpathlineto{\pgfqpoint{3.732902in}{1.886879in}}%
\pgfpathlineto{\pgfqpoint{3.719188in}{1.891688in}}%
\pgfpathlineto{\pgfqpoint{3.705479in}{1.896609in}}%
\pgfpathlineto{\pgfqpoint{3.691776in}{1.901640in}}%
\pgfpathlineto{\pgfqpoint{3.699884in}{1.908189in}}%
\pgfpathlineto{\pgfqpoint{3.707985in}{1.914850in}}%
\pgfpathlineto{\pgfqpoint{3.716079in}{1.921622in}}%
\pgfpathlineto{\pgfqpoint{3.724166in}{1.928501in}}%
\pgfpathclose%
\pgfusepath{fill}%
\end{pgfscope}%
\begin{pgfscope}%
\pgfpathrectangle{\pgfqpoint{1.150000in}{0.150000in}}{\pgfqpoint{5.700000in}{5.700000in}}%
\pgfusepath{clip}%
\pgfsetbuttcap%
\pgfsetroundjoin%
\definecolor{currentfill}{rgb}{0.268510,0.009605,0.335427}%
\pgfsetfillcolor{currentfill}%
\pgfsetfillopacity{0.700000}%
\pgfsetlinewidth{0.000000pt}%
\definecolor{currentstroke}{rgb}{0.000000,0.000000,0.000000}%
\pgfsetstrokecolor{currentstroke}%
\pgfsetdash{}{0pt}%
\pgfpathmoveto{\pgfqpoint{3.582311in}{1.945947in}}%
\pgfpathlineto{\pgfqpoint{3.595978in}{1.940010in}}%
\pgfpathlineto{\pgfqpoint{3.609650in}{1.934188in}}%
\pgfpathlineto{\pgfqpoint{3.623327in}{1.928480in}}%
\pgfpathlineto{\pgfqpoint{3.637007in}{1.922886in}}%
\pgfpathlineto{\pgfqpoint{3.628873in}{1.916821in}}%
\pgfpathlineto{\pgfqpoint{3.620730in}{1.910881in}}%
\pgfpathlineto{\pgfqpoint{3.612580in}{1.905070in}}%
\pgfpathlineto{\pgfqpoint{3.604422in}{1.899391in}}%
\pgfpathlineto{\pgfqpoint{3.590721in}{1.905359in}}%
\pgfpathlineto{\pgfqpoint{3.577025in}{1.911441in}}%
\pgfpathlineto{\pgfqpoint{3.563333in}{1.917637in}}%
\pgfpathlineto{\pgfqpoint{3.549645in}{1.923948in}}%
\pgfpathlineto{\pgfqpoint{3.557824in}{1.929246in}}%
\pgfpathlineto{\pgfqpoint{3.565994in}{1.934681in}}%
\pgfpathlineto{\pgfqpoint{3.574156in}{1.940249in}}%
\pgfpathlineto{\pgfqpoint{3.582311in}{1.945947in}}%
\pgfpathclose%
\pgfusepath{fill}%
\end{pgfscope}%
\begin{pgfscope}%
\pgfpathrectangle{\pgfqpoint{1.150000in}{0.150000in}}{\pgfqpoint{5.700000in}{5.700000in}}%
\pgfusepath{clip}%
\pgfsetbuttcap%
\pgfsetroundjoin%
\definecolor{currentfill}{rgb}{0.233603,0.313828,0.543914}%
\pgfsetfillcolor{currentfill}%
\pgfsetfillopacity{0.700000}%
\pgfsetlinewidth{0.000000pt}%
\definecolor{currentstroke}{rgb}{0.000000,0.000000,0.000000}%
\pgfsetstrokecolor{currentstroke}%
\pgfsetdash{}{0pt}%
\pgfpathmoveto{\pgfqpoint{5.081196in}{2.551632in}}%
\pgfpathlineto{\pgfqpoint{5.095357in}{2.556751in}}%
\pgfpathlineto{\pgfqpoint{5.109531in}{2.561969in}}%
\pgfpathlineto{\pgfqpoint{5.123719in}{2.567287in}}%
\pgfpathlineto{\pgfqpoint{5.137920in}{2.572706in}}%
\pgfpathlineto{\pgfqpoint{5.130311in}{2.562527in}}%
\pgfpathlineto{\pgfqpoint{5.122696in}{2.552262in}}%
\pgfpathlineto{\pgfqpoint{5.115074in}{2.541910in}}%
\pgfpathlineto{\pgfqpoint{5.107446in}{2.531472in}}%
\pgfpathlineto{\pgfqpoint{5.093240in}{2.526137in}}%
\pgfpathlineto{\pgfqpoint{5.079047in}{2.520902in}}%
\pgfpathlineto{\pgfqpoint{5.064868in}{2.515767in}}%
\pgfpathlineto{\pgfqpoint{5.050702in}{2.510732in}}%
\pgfpathlineto{\pgfqpoint{5.058335in}{2.521079in}}%
\pgfpathlineto{\pgfqpoint{5.065962in}{2.531346in}}%
\pgfpathlineto{\pgfqpoint{5.073582in}{2.541530in}}%
\pgfpathlineto{\pgfqpoint{5.081196in}{2.551632in}}%
\pgfpathclose%
\pgfusepath{fill}%
\end{pgfscope}%
\begin{pgfscope}%
\pgfpathrectangle{\pgfqpoint{1.150000in}{0.150000in}}{\pgfqpoint{5.700000in}{5.700000in}}%
\pgfusepath{clip}%
\pgfsetbuttcap%
\pgfsetroundjoin%
\definecolor{currentfill}{rgb}{0.282884,0.135920,0.453427}%
\pgfsetfillcolor{currentfill}%
\pgfsetfillopacity{0.700000}%
\pgfsetlinewidth{0.000000pt}%
\definecolor{currentstroke}{rgb}{0.000000,0.000000,0.000000}%
\pgfsetstrokecolor{currentstroke}%
\pgfsetdash{}{0pt}%
\pgfpathmoveto{\pgfqpoint{4.528728in}{2.170995in}}%
\pgfpathlineto{\pgfqpoint{4.542643in}{2.172727in}}%
\pgfpathlineto{\pgfqpoint{4.556568in}{2.174561in}}%
\pgfpathlineto{\pgfqpoint{4.570504in}{2.176496in}}%
\pgfpathlineto{\pgfqpoint{4.584450in}{2.178532in}}%
\pgfpathlineto{\pgfqpoint{4.576653in}{2.167511in}}%
\pgfpathlineto{\pgfqpoint{4.568851in}{2.156463in}}%
\pgfpathlineto{\pgfqpoint{4.561044in}{2.145391in}}%
\pgfpathlineto{\pgfqpoint{4.553232in}{2.134297in}}%
\pgfpathlineto{\pgfqpoint{4.539281in}{2.132473in}}%
\pgfpathlineto{\pgfqpoint{4.525340in}{2.130749in}}%
\pgfpathlineto{\pgfqpoint{4.511409in}{2.129128in}}%
\pgfpathlineto{\pgfqpoint{4.497488in}{2.127607in}}%
\pgfpathlineto{\pgfqpoint{4.505306in}{2.138483in}}%
\pgfpathlineto{\pgfqpoint{4.513118in}{2.149341in}}%
\pgfpathlineto{\pgfqpoint{4.520926in}{2.160179in}}%
\pgfpathlineto{\pgfqpoint{4.528728in}{2.170995in}}%
\pgfpathclose%
\pgfusepath{fill}%
\end{pgfscope}%
\begin{pgfscope}%
\pgfpathrectangle{\pgfqpoint{1.150000in}{0.150000in}}{\pgfqpoint{5.700000in}{5.700000in}}%
\pgfusepath{clip}%
\pgfsetbuttcap%
\pgfsetroundjoin%
\definecolor{currentfill}{rgb}{0.175841,0.441290,0.557685}%
\pgfsetfillcolor{currentfill}%
\pgfsetfillopacity{0.700000}%
\pgfsetlinewidth{0.000000pt}%
\definecolor{currentstroke}{rgb}{0.000000,0.000000,0.000000}%
\pgfsetstrokecolor{currentstroke}%
\pgfsetdash{}{0pt}%
\pgfpathmoveto{\pgfqpoint{2.385972in}{2.894567in}}%
\pgfpathlineto{\pgfqpoint{2.399833in}{2.875519in}}%
\pgfpathlineto{\pgfqpoint{2.413687in}{2.856672in}}%
\pgfpathlineto{\pgfqpoint{2.427532in}{2.838024in}}%
\pgfpathlineto{\pgfqpoint{2.441370in}{2.819573in}}%
\pgfpathlineto{\pgfqpoint{2.432438in}{2.823982in}}%
\pgfpathlineto{\pgfqpoint{2.423487in}{2.828665in}}%
\pgfpathlineto{\pgfqpoint{2.414517in}{2.833628in}}%
\pgfpathlineto{\pgfqpoint{2.405526in}{2.838875in}}%
\pgfpathlineto{\pgfqpoint{2.391639in}{2.857793in}}%
\pgfpathlineto{\pgfqpoint{2.377744in}{2.876909in}}%
\pgfpathlineto{\pgfqpoint{2.363841in}{2.896225in}}%
\pgfpathlineto{\pgfqpoint{2.349930in}{2.915744in}}%
\pgfpathlineto{\pgfqpoint{2.358971in}{2.910020in}}%
\pgfpathlineto{\pgfqpoint{2.367991in}{2.904585in}}%
\pgfpathlineto{\pgfqpoint{2.376992in}{2.899436in}}%
\pgfpathlineto{\pgfqpoint{2.385972in}{2.894567in}}%
\pgfpathclose%
\pgfusepath{fill}%
\end{pgfscope}%
\begin{pgfscope}%
\pgfpathrectangle{\pgfqpoint{1.150000in}{0.150000in}}{\pgfqpoint{5.700000in}{5.700000in}}%
\pgfusepath{clip}%
\pgfsetbuttcap%
\pgfsetroundjoin%
\definecolor{currentfill}{rgb}{0.278791,0.062145,0.386592}%
\pgfsetfillcolor{currentfill}%
\pgfsetfillopacity{0.700000}%
\pgfsetlinewidth{0.000000pt}%
\definecolor{currentstroke}{rgb}{0.000000,0.000000,0.000000}%
\pgfsetstrokecolor{currentstroke}%
\pgfsetdash{}{0pt}%
\pgfpathmoveto{\pgfqpoint{4.268315in}{2.035621in}}%
\pgfpathlineto{\pgfqpoint{4.282137in}{2.035438in}}%
\pgfpathlineto{\pgfqpoint{4.295967in}{2.035358in}}%
\pgfpathlineto{\pgfqpoint{4.309806in}{2.035381in}}%
\pgfpathlineto{\pgfqpoint{4.323654in}{2.035506in}}%
\pgfpathlineto{\pgfqpoint{4.315778in}{2.025078in}}%
\pgfpathlineto{\pgfqpoint{4.307897in}{2.014662in}}%
\pgfpathlineto{\pgfqpoint{4.300011in}{2.004261in}}%
\pgfpathlineto{\pgfqpoint{4.292120in}{1.993877in}}%
\pgfpathlineto{\pgfqpoint{4.278265in}{1.994017in}}%
\pgfpathlineto{\pgfqpoint{4.264418in}{1.994260in}}%
\pgfpathlineto{\pgfqpoint{4.250580in}{1.994605in}}%
\pgfpathlineto{\pgfqpoint{4.236751in}{1.995053in}}%
\pgfpathlineto{\pgfqpoint{4.244649in}{2.005165in}}%
\pgfpathlineto{\pgfqpoint{4.252543in}{2.015298in}}%
\pgfpathlineto{\pgfqpoint{4.260432in}{2.025451in}}%
\pgfpathlineto{\pgfqpoint{4.268315in}{2.035621in}}%
\pgfpathclose%
\pgfusepath{fill}%
\end{pgfscope}%
\begin{pgfscope}%
\pgfpathrectangle{\pgfqpoint{1.150000in}{0.150000in}}{\pgfqpoint{5.700000in}{5.700000in}}%
\pgfusepath{clip}%
\pgfsetbuttcap%
\pgfsetroundjoin%
\definecolor{currentfill}{rgb}{0.171176,0.452530,0.557965}%
\pgfsetfillcolor{currentfill}%
\pgfsetfillopacity{0.700000}%
\pgfsetlinewidth{0.000000pt}%
\definecolor{currentstroke}{rgb}{0.000000,0.000000,0.000000}%
\pgfsetstrokecolor{currentstroke}%
\pgfsetdash{}{0pt}%
\pgfpathmoveto{\pgfqpoint{5.546887in}{2.892290in}}%
\pgfpathlineto{\pgfqpoint{5.561288in}{2.899507in}}%
\pgfpathlineto{\pgfqpoint{5.575704in}{2.906824in}}%
\pgfpathlineto{\pgfqpoint{5.590135in}{2.914242in}}%
\pgfpathlineto{\pgfqpoint{5.604581in}{2.921759in}}%
\pgfpathlineto{\pgfqpoint{5.597188in}{2.913890in}}%
\pgfpathlineto{\pgfqpoint{5.589786in}{2.905912in}}%
\pgfpathlineto{\pgfqpoint{5.582376in}{2.897825in}}%
\pgfpathlineto{\pgfqpoint{5.574958in}{2.889629in}}%
\pgfpathlineto{\pgfqpoint{5.560502in}{2.882079in}}%
\pgfpathlineto{\pgfqpoint{5.546062in}{2.874630in}}%
\pgfpathlineto{\pgfqpoint{5.531637in}{2.867280in}}%
\pgfpathlineto{\pgfqpoint{5.517227in}{2.860031in}}%
\pgfpathlineto{\pgfqpoint{5.524654in}{2.868253in}}%
\pgfpathlineto{\pgfqpoint{5.532074in}{2.876370in}}%
\pgfpathlineto{\pgfqpoint{5.539485in}{2.884382in}}%
\pgfpathlineto{\pgfqpoint{5.546887in}{2.892290in}}%
\pgfpathclose%
\pgfusepath{fill}%
\end{pgfscope}%
\begin{pgfscope}%
\pgfpathrectangle{\pgfqpoint{1.150000in}{0.150000in}}{\pgfqpoint{5.700000in}{5.700000in}}%
\pgfusepath{clip}%
\pgfsetbuttcap%
\pgfsetroundjoin%
\definecolor{currentfill}{rgb}{0.280868,0.160771,0.472899}%
\pgfsetfillcolor{currentfill}%
\pgfsetfillopacity{0.700000}%
\pgfsetlinewidth{0.000000pt}%
\definecolor{currentstroke}{rgb}{0.000000,0.000000,0.000000}%
\pgfsetstrokecolor{currentstroke}%
\pgfsetdash{}{0pt}%
\pgfpathmoveto{\pgfqpoint{4.615587in}{2.222316in}}%
\pgfpathlineto{\pgfqpoint{4.629538in}{2.224646in}}%
\pgfpathlineto{\pgfqpoint{4.643501in}{2.227078in}}%
\pgfpathlineto{\pgfqpoint{4.657474in}{2.229611in}}%
\pgfpathlineto{\pgfqpoint{4.671458in}{2.232245in}}%
\pgfpathlineto{\pgfqpoint{4.663687in}{2.221160in}}%
\pgfpathlineto{\pgfqpoint{4.655910in}{2.210038in}}%
\pgfpathlineto{\pgfqpoint{4.648128in}{2.198879in}}%
\pgfpathlineto{\pgfqpoint{4.640341in}{2.187686in}}%
\pgfpathlineto{\pgfqpoint{4.626352in}{2.185246in}}%
\pgfpathlineto{\pgfqpoint{4.612374in}{2.182907in}}%
\pgfpathlineto{\pgfqpoint{4.598407in}{2.180669in}}%
\pgfpathlineto{\pgfqpoint{4.584450in}{2.178532in}}%
\pgfpathlineto{\pgfqpoint{4.592242in}{2.189524in}}%
\pgfpathlineto{\pgfqpoint{4.600029in}{2.200487in}}%
\pgfpathlineto{\pgfqpoint{4.607810in}{2.211418in}}%
\pgfpathlineto{\pgfqpoint{4.615587in}{2.222316in}}%
\pgfpathclose%
\pgfusepath{fill}%
\end{pgfscope}%
\begin{pgfscope}%
\pgfpathrectangle{\pgfqpoint{1.150000in}{0.150000in}}{\pgfqpoint{5.700000in}{5.700000in}}%
\pgfusepath{clip}%
\pgfsetbuttcap%
\pgfsetroundjoin%
\definecolor{currentfill}{rgb}{0.271828,0.209303,0.504434}%
\pgfsetfillcolor{currentfill}%
\pgfsetfillopacity{0.700000}%
\pgfsetlinewidth{0.000000pt}%
\definecolor{currentstroke}{rgb}{0.000000,0.000000,0.000000}%
\pgfsetstrokecolor{currentstroke}%
\pgfsetdash{}{0pt}%
\pgfpathmoveto{\pgfqpoint{2.881369in}{2.323348in}}%
\pgfpathlineto{\pgfqpoint{2.895062in}{2.310516in}}%
\pgfpathlineto{\pgfqpoint{2.908753in}{2.297832in}}%
\pgfpathlineto{\pgfqpoint{2.922443in}{2.285295in}}%
\pgfpathlineto{\pgfqpoint{2.936131in}{2.272905in}}%
\pgfpathlineto{\pgfqpoint{2.927573in}{2.273302in}}%
\pgfpathlineto{\pgfqpoint{2.919001in}{2.273926in}}%
\pgfpathlineto{\pgfqpoint{2.910414in}{2.274780in}}%
\pgfpathlineto{\pgfqpoint{2.901813in}{2.275871in}}%
\pgfpathlineto{\pgfqpoint{2.888088in}{2.288705in}}%
\pgfpathlineto{\pgfqpoint{2.874362in}{2.301685in}}%
\pgfpathlineto{\pgfqpoint{2.860633in}{2.314813in}}%
\pgfpathlineto{\pgfqpoint{2.846902in}{2.328090in}}%
\pgfpathlineto{\pgfqpoint{2.855541in}{2.326548in}}%
\pgfpathlineto{\pgfqpoint{2.864165in}{2.325247in}}%
\pgfpathlineto{\pgfqpoint{2.872774in}{2.324181in}}%
\pgfpathlineto{\pgfqpoint{2.881369in}{2.323348in}}%
\pgfpathclose%
\pgfusepath{fill}%
\end{pgfscope}%
\begin{pgfscope}%
\pgfpathrectangle{\pgfqpoint{1.150000in}{0.150000in}}{\pgfqpoint{5.700000in}{5.700000in}}%
\pgfusepath{clip}%
\pgfsetbuttcap%
\pgfsetroundjoin%
\definecolor{currentfill}{rgb}{0.276022,0.044167,0.370164}%
\pgfsetfillcolor{currentfill}%
\pgfsetfillopacity{0.700000}%
\pgfsetlinewidth{0.000000pt}%
\definecolor{currentstroke}{rgb}{0.000000,0.000000,0.000000}%
\pgfsetstrokecolor{currentstroke}%
\pgfsetdash{}{0pt}%
\pgfpathmoveto{\pgfqpoint{4.181516in}{1.997881in}}%
\pgfpathlineto{\pgfqpoint{4.195313in}{1.997019in}}%
\pgfpathlineto{\pgfqpoint{4.209117in}{1.996260in}}%
\pgfpathlineto{\pgfqpoint{4.222930in}{1.995605in}}%
\pgfpathlineto{\pgfqpoint{4.236751in}{1.995053in}}%
\pgfpathlineto{\pgfqpoint{4.228846in}{1.984966in}}%
\pgfpathlineto{\pgfqpoint{4.220937in}{1.974906in}}%
\pgfpathlineto{\pgfqpoint{4.213023in}{1.964875in}}%
\pgfpathlineto{\pgfqpoint{4.205103in}{1.954877in}}%
\pgfpathlineto{\pgfqpoint{4.191273in}{1.955712in}}%
\pgfpathlineto{\pgfqpoint{4.177452in}{1.956650in}}%
\pgfpathlineto{\pgfqpoint{4.163638in}{1.957691in}}%
\pgfpathlineto{\pgfqpoint{4.149833in}{1.958837in}}%
\pgfpathlineto{\pgfqpoint{4.157762in}{1.968545in}}%
\pgfpathlineto{\pgfqpoint{4.165685in}{1.978291in}}%
\pgfpathlineto{\pgfqpoint{4.173604in}{1.988070in}}%
\pgfpathlineto{\pgfqpoint{4.181516in}{1.997881in}}%
\pgfpathclose%
\pgfusepath{fill}%
\end{pgfscope}%
\begin{pgfscope}%
\pgfpathrectangle{\pgfqpoint{1.150000in}{0.150000in}}{\pgfqpoint{5.700000in}{5.700000in}}%
\pgfusepath{clip}%
\pgfsetbuttcap%
\pgfsetroundjoin%
\definecolor{currentfill}{rgb}{0.267004,0.004874,0.329415}%
\pgfsetfillcolor{currentfill}%
\pgfsetfillopacity{0.700000}%
\pgfsetlinewidth{0.000000pt}%
\definecolor{currentstroke}{rgb}{0.000000,0.000000,0.000000}%
\pgfsetstrokecolor{currentstroke}%
\pgfsetdash{}{0pt}%
\pgfpathmoveto{\pgfqpoint{3.865961in}{1.925386in}}%
\pgfpathlineto{\pgfqpoint{3.879677in}{1.921925in}}%
\pgfpathlineto{\pgfqpoint{3.893399in}{1.918573in}}%
\pgfpathlineto{\pgfqpoint{3.907128in}{1.915328in}}%
\pgfpathlineto{\pgfqpoint{3.920862in}{1.912191in}}%
\pgfpathlineto{\pgfqpoint{3.912848in}{1.903931in}}%
\pgfpathlineto{\pgfqpoint{3.904827in}{1.895750in}}%
\pgfpathlineto{\pgfqpoint{3.896800in}{1.887652in}}%
\pgfpathlineto{\pgfqpoint{3.888767in}{1.879640in}}%
\pgfpathlineto{\pgfqpoint{3.875018in}{1.883114in}}%
\pgfpathlineto{\pgfqpoint{3.861276in}{1.886695in}}%
\pgfpathlineto{\pgfqpoint{3.847539in}{1.890385in}}%
\pgfpathlineto{\pgfqpoint{3.833809in}{1.894182in}}%
\pgfpathlineto{\pgfqpoint{3.841856in}{1.901850in}}%
\pgfpathlineto{\pgfqpoint{3.849898in}{1.909609in}}%
\pgfpathlineto{\pgfqpoint{3.857933in}{1.917455in}}%
\pgfpathlineto{\pgfqpoint{3.865961in}{1.925386in}}%
\pgfpathclose%
\pgfusepath{fill}%
\end{pgfscope}%
\begin{pgfscope}%
\pgfpathrectangle{\pgfqpoint{1.150000in}{0.150000in}}{\pgfqpoint{5.700000in}{5.700000in}}%
\pgfusepath{clip}%
\pgfsetbuttcap%
\pgfsetroundjoin%
\definecolor{currentfill}{rgb}{0.276194,0.190074,0.493001}%
\pgfsetfillcolor{currentfill}%
\pgfsetfillopacity{0.700000}%
\pgfsetlinewidth{0.000000pt}%
\definecolor{currentstroke}{rgb}{0.000000,0.000000,0.000000}%
\pgfsetstrokecolor{currentstroke}%
\pgfsetdash{}{0pt}%
\pgfpathmoveto{\pgfqpoint{4.702491in}{2.276171in}}%
\pgfpathlineto{\pgfqpoint{4.716482in}{2.279081in}}%
\pgfpathlineto{\pgfqpoint{4.730484in}{2.282092in}}%
\pgfpathlineto{\pgfqpoint{4.744497in}{2.285203in}}%
\pgfpathlineto{\pgfqpoint{4.758521in}{2.288415in}}%
\pgfpathlineto{\pgfqpoint{4.750776in}{2.277329in}}%
\pgfpathlineto{\pgfqpoint{4.743024in}{2.266195in}}%
\pgfpathlineto{\pgfqpoint{4.735268in}{2.255013in}}%
\pgfpathlineto{\pgfqpoint{4.727506in}{2.243786in}}%
\pgfpathlineto{\pgfqpoint{4.713477in}{2.240750in}}%
\pgfpathlineto{\pgfqpoint{4.699460in}{2.237814in}}%
\pgfpathlineto{\pgfqpoint{4.685453in}{2.234979in}}%
\pgfpathlineto{\pgfqpoint{4.671458in}{2.232245in}}%
\pgfpathlineto{\pgfqpoint{4.679224in}{2.243289in}}%
\pgfpathlineto{\pgfqpoint{4.686985in}{2.254293in}}%
\pgfpathlineto{\pgfqpoint{4.694741in}{2.265254in}}%
\pgfpathlineto{\pgfqpoint{4.702491in}{2.276171in}}%
\pgfpathclose%
\pgfusepath{fill}%
\end{pgfscope}%
\begin{pgfscope}%
\pgfpathrectangle{\pgfqpoint{1.150000in}{0.150000in}}{\pgfqpoint{5.700000in}{5.700000in}}%
\pgfusepath{clip}%
\pgfsetbuttcap%
\pgfsetroundjoin%
\definecolor{currentfill}{rgb}{0.273809,0.031497,0.358853}%
\pgfsetfillcolor{currentfill}%
\pgfsetfillopacity{0.700000}%
\pgfsetlinewidth{0.000000pt}%
\definecolor{currentstroke}{rgb}{0.000000,0.000000,0.000000}%
\pgfsetstrokecolor{currentstroke}%
\pgfsetdash{}{0pt}%
\pgfpathmoveto{\pgfqpoint{3.440267in}{1.978634in}}%
\pgfpathlineto{\pgfqpoint{3.453928in}{1.971385in}}%
\pgfpathlineto{\pgfqpoint{3.467592in}{1.964256in}}%
\pgfpathlineto{\pgfqpoint{3.481259in}{1.957245in}}%
\pgfpathlineto{\pgfqpoint{3.494929in}{1.950352in}}%
\pgfpathlineto{\pgfqpoint{3.486720in}{1.945581in}}%
\pgfpathlineto{\pgfqpoint{3.478501in}{1.940959in}}%
\pgfpathlineto{\pgfqpoint{3.470274in}{1.936491in}}%
\pgfpathlineto{\pgfqpoint{3.462038in}{1.932179in}}%
\pgfpathlineto{\pgfqpoint{3.448345in}{1.939466in}}%
\pgfpathlineto{\pgfqpoint{3.434655in}{1.946871in}}%
\pgfpathlineto{\pgfqpoint{3.420967in}{1.954394in}}%
\pgfpathlineto{\pgfqpoint{3.407283in}{1.962036in}}%
\pgfpathlineto{\pgfqpoint{3.415543in}{1.965947in}}%
\pgfpathlineto{\pgfqpoint{3.423794in}{1.970019in}}%
\pgfpathlineto{\pgfqpoint{3.432035in}{1.974250in}}%
\pgfpathlineto{\pgfqpoint{3.440267in}{1.978634in}}%
\pgfpathclose%
\pgfusepath{fill}%
\end{pgfscope}%
\begin{pgfscope}%
\pgfpathrectangle{\pgfqpoint{1.150000in}{0.150000in}}{\pgfqpoint{5.700000in}{5.700000in}}%
\pgfusepath{clip}%
\pgfsetbuttcap%
\pgfsetroundjoin%
\definecolor{currentfill}{rgb}{0.220057,0.343307,0.549413}%
\pgfsetfillcolor{currentfill}%
\pgfsetfillopacity{0.700000}%
\pgfsetlinewidth{0.000000pt}%
\definecolor{currentstroke}{rgb}{0.000000,0.000000,0.000000}%
\pgfsetstrokecolor{currentstroke}%
\pgfsetdash{}{0pt}%
\pgfpathmoveto{\pgfqpoint{5.168288in}{2.612541in}}%
\pgfpathlineto{\pgfqpoint{5.182497in}{2.618124in}}%
\pgfpathlineto{\pgfqpoint{5.196719in}{2.623807in}}%
\pgfpathlineto{\pgfqpoint{5.210956in}{2.629590in}}%
\pgfpathlineto{\pgfqpoint{5.225205in}{2.635473in}}%
\pgfpathlineto{\pgfqpoint{5.217629in}{2.625589in}}%
\pgfpathlineto{\pgfqpoint{5.210046in}{2.615612in}}%
\pgfpathlineto{\pgfqpoint{5.202456in}{2.605542in}}%
\pgfpathlineto{\pgfqpoint{5.194859in}{2.595380in}}%
\pgfpathlineto{\pgfqpoint{5.180604in}{2.589561in}}%
\pgfpathlineto{\pgfqpoint{5.166362in}{2.583843in}}%
\pgfpathlineto{\pgfqpoint{5.152134in}{2.578224in}}%
\pgfpathlineto{\pgfqpoint{5.137920in}{2.572706in}}%
\pgfpathlineto{\pgfqpoint{5.145522in}{2.582797in}}%
\pgfpathlineto{\pgfqpoint{5.153117in}{2.592800in}}%
\pgfpathlineto{\pgfqpoint{5.160706in}{2.602715in}}%
\pgfpathlineto{\pgfqpoint{5.168288in}{2.612541in}}%
\pgfpathclose%
\pgfusepath{fill}%
\end{pgfscope}%
\begin{pgfscope}%
\pgfpathrectangle{\pgfqpoint{1.150000in}{0.150000in}}{\pgfqpoint{5.700000in}{5.700000in}}%
\pgfusepath{clip}%
\pgfsetbuttcap%
\pgfsetroundjoin%
\definecolor{currentfill}{rgb}{0.281446,0.084320,0.407414}%
\pgfsetfillcolor{currentfill}%
\pgfsetfillopacity{0.700000}%
\pgfsetlinewidth{0.000000pt}%
\definecolor{currentstroke}{rgb}{0.000000,0.000000,0.000000}%
\pgfsetstrokecolor{currentstroke}%
\pgfsetdash{}{0pt}%
\pgfpathmoveto{\pgfqpoint{3.243254in}{2.063264in}}%
\pgfpathlineto{\pgfqpoint{3.256912in}{2.054144in}}%
\pgfpathlineto{\pgfqpoint{3.270572in}{2.045150in}}%
\pgfpathlineto{\pgfqpoint{3.284234in}{2.036282in}}%
\pgfpathlineto{\pgfqpoint{3.297898in}{2.027539in}}%
\pgfpathlineto{\pgfqpoint{3.289576in}{2.024608in}}%
\pgfpathlineto{\pgfqpoint{3.281244in}{2.021856in}}%
\pgfpathlineto{\pgfqpoint{3.272901in}{2.019288in}}%
\pgfpathlineto{\pgfqpoint{3.264546in}{2.016908in}}%
\pgfpathlineto{\pgfqpoint{3.250855in}{2.026066in}}%
\pgfpathlineto{\pgfqpoint{3.237165in}{2.035349in}}%
\pgfpathlineto{\pgfqpoint{3.223477in}{2.044758in}}%
\pgfpathlineto{\pgfqpoint{3.209790in}{2.054295in}}%
\pgfpathlineto{\pgfqpoint{3.218173in}{2.056252in}}%
\pgfpathlineto{\pgfqpoint{3.226544in}{2.058403in}}%
\pgfpathlineto{\pgfqpoint{3.234904in}{2.060741in}}%
\pgfpathlineto{\pgfqpoint{3.243254in}{2.063264in}}%
\pgfpathclose%
\pgfusepath{fill}%
\end{pgfscope}%
\begin{pgfscope}%
\pgfpathrectangle{\pgfqpoint{1.150000in}{0.150000in}}{\pgfqpoint{5.700000in}{5.700000in}}%
\pgfusepath{clip}%
\pgfsetbuttcap%
\pgfsetroundjoin%
\definecolor{currentfill}{rgb}{0.277134,0.185228,0.489898}%
\pgfsetfillcolor{currentfill}%
\pgfsetfillopacity{0.700000}%
\pgfsetlinewidth{0.000000pt}%
\definecolor{currentstroke}{rgb}{0.000000,0.000000,0.000000}%
\pgfsetstrokecolor{currentstroke}%
\pgfsetdash{}{0pt}%
\pgfpathmoveto{\pgfqpoint{2.936131in}{2.272905in}}%
\pgfpathlineto{\pgfqpoint{2.949818in}{2.260659in}}%
\pgfpathlineto{\pgfqpoint{2.963503in}{2.248557in}}%
\pgfpathlineto{\pgfqpoint{2.977188in}{2.236599in}}%
\pgfpathlineto{\pgfqpoint{2.990872in}{2.224782in}}%
\pgfpathlineto{\pgfqpoint{2.982349in}{2.224745in}}%
\pgfpathlineto{\pgfqpoint{2.973812in}{2.224929in}}%
\pgfpathlineto{\pgfqpoint{2.965262in}{2.225339in}}%
\pgfpathlineto{\pgfqpoint{2.956698in}{2.225979in}}%
\pgfpathlineto{\pgfqpoint{2.942979in}{2.238237in}}%
\pgfpathlineto{\pgfqpoint{2.929258in}{2.250638in}}%
\pgfpathlineto{\pgfqpoint{2.915536in}{2.263182in}}%
\pgfpathlineto{\pgfqpoint{2.901813in}{2.275871in}}%
\pgfpathlineto{\pgfqpoint{2.910414in}{2.274780in}}%
\pgfpathlineto{\pgfqpoint{2.919001in}{2.273926in}}%
\pgfpathlineto{\pgfqpoint{2.927573in}{2.273302in}}%
\pgfpathlineto{\pgfqpoint{2.936131in}{2.272905in}}%
\pgfpathclose%
\pgfusepath{fill}%
\end{pgfscope}%
\begin{pgfscope}%
\pgfpathrectangle{\pgfqpoint{1.150000in}{0.150000in}}{\pgfqpoint{5.700000in}{5.700000in}}%
\pgfusepath{clip}%
\pgfsetbuttcap%
\pgfsetroundjoin%
\definecolor{currentfill}{rgb}{0.162142,0.474838,0.558140}%
\pgfsetfillcolor{currentfill}%
\pgfsetfillopacity{0.700000}%
\pgfsetlinewidth{0.000000pt}%
\definecolor{currentstroke}{rgb}{0.000000,0.000000,0.000000}%
\pgfsetstrokecolor{currentstroke}%
\pgfsetdash{}{0pt}%
\pgfpathmoveto{\pgfqpoint{5.634070in}{2.952162in}}%
\pgfpathlineto{\pgfqpoint{5.648522in}{2.959728in}}%
\pgfpathlineto{\pgfqpoint{5.662989in}{2.967393in}}%
\pgfpathlineto{\pgfqpoint{5.677472in}{2.975159in}}%
\pgfpathlineto{\pgfqpoint{5.691970in}{2.983025in}}%
\pgfpathlineto{\pgfqpoint{5.684621in}{2.975642in}}%
\pgfpathlineto{\pgfqpoint{5.677264in}{2.968149in}}%
\pgfpathlineto{\pgfqpoint{5.669897in}{2.960545in}}%
\pgfpathlineto{\pgfqpoint{5.662522in}{2.952829in}}%
\pgfpathlineto{\pgfqpoint{5.648014in}{2.944911in}}%
\pgfpathlineto{\pgfqpoint{5.633521in}{2.937093in}}%
\pgfpathlineto{\pgfqpoint{5.619043in}{2.929376in}}%
\pgfpathlineto{\pgfqpoint{5.604581in}{2.921759in}}%
\pgfpathlineto{\pgfqpoint{5.611966in}{2.929520in}}%
\pgfpathlineto{\pgfqpoint{5.619343in}{2.937173in}}%
\pgfpathlineto{\pgfqpoint{5.626711in}{2.944720in}}%
\pgfpathlineto{\pgfqpoint{5.634070in}{2.952162in}}%
\pgfpathclose%
\pgfusepath{fill}%
\end{pgfscope}%
\begin{pgfscope}%
\pgfpathrectangle{\pgfqpoint{1.150000in}{0.150000in}}{\pgfqpoint{5.700000in}{5.700000in}}%
\pgfusepath{clip}%
\pgfsetbuttcap%
\pgfsetroundjoin%
\definecolor{currentfill}{rgb}{0.272594,0.025563,0.353093}%
\pgfsetfillcolor{currentfill}%
\pgfsetfillopacity{0.700000}%
\pgfsetlinewidth{0.000000pt}%
\definecolor{currentstroke}{rgb}{0.000000,0.000000,0.000000}%
\pgfsetstrokecolor{currentstroke}%
\pgfsetdash{}{0pt}%
\pgfpathmoveto{\pgfqpoint{4.094690in}{1.964462in}}%
\pgfpathlineto{\pgfqpoint{4.108464in}{1.962899in}}%
\pgfpathlineto{\pgfqpoint{4.122246in}{1.961440in}}%
\pgfpathlineto{\pgfqpoint{4.136036in}{1.960086in}}%
\pgfpathlineto{\pgfqpoint{4.149833in}{1.958837in}}%
\pgfpathlineto{\pgfqpoint{4.141899in}{1.949168in}}%
\pgfpathlineto{\pgfqpoint{4.133959in}{1.939541in}}%
\pgfpathlineto{\pgfqpoint{4.126014in}{1.929959in}}%
\pgfpathlineto{\pgfqpoint{4.118064in}{1.920425in}}%
\pgfpathlineto{\pgfqpoint{4.104257in}{1.921975in}}%
\pgfpathlineto{\pgfqpoint{4.090457in}{1.923630in}}%
\pgfpathlineto{\pgfqpoint{4.076665in}{1.925389in}}%
\pgfpathlineto{\pgfqpoint{4.062880in}{1.927253in}}%
\pgfpathlineto{\pgfqpoint{4.070841in}{1.936479in}}%
\pgfpathlineto{\pgfqpoint{4.078796in}{1.945758in}}%
\pgfpathlineto{\pgfqpoint{4.086746in}{1.955087in}}%
\pgfpathlineto{\pgfqpoint{4.094690in}{1.964462in}}%
\pgfpathclose%
\pgfusepath{fill}%
\end{pgfscope}%
\begin{pgfscope}%
\pgfpathrectangle{\pgfqpoint{1.150000in}{0.150000in}}{\pgfqpoint{5.700000in}{5.700000in}}%
\pgfusepath{clip}%
\pgfsetbuttcap%
\pgfsetroundjoin%
\definecolor{currentfill}{rgb}{0.269308,0.218818,0.509577}%
\pgfsetfillcolor{currentfill}%
\pgfsetfillopacity{0.700000}%
\pgfsetlinewidth{0.000000pt}%
\definecolor{currentstroke}{rgb}{0.000000,0.000000,0.000000}%
\pgfsetstrokecolor{currentstroke}%
\pgfsetdash{}{0pt}%
\pgfpathmoveto{\pgfqpoint{4.789450in}{2.332242in}}%
\pgfpathlineto{\pgfqpoint{4.803481in}{2.335712in}}%
\pgfpathlineto{\pgfqpoint{4.817525in}{2.339282in}}%
\pgfpathlineto{\pgfqpoint{4.831580in}{2.342953in}}%
\pgfpathlineto{\pgfqpoint{4.845647in}{2.346724in}}%
\pgfpathlineto{\pgfqpoint{4.837928in}{2.335697in}}%
\pgfpathlineto{\pgfqpoint{4.830202in}{2.324611in}}%
\pgfpathlineto{\pgfqpoint{4.822472in}{2.313467in}}%
\pgfpathlineto{\pgfqpoint{4.814736in}{2.302267in}}%
\pgfpathlineto{\pgfqpoint{4.800664in}{2.298654in}}%
\pgfpathlineto{\pgfqpoint{4.786605in}{2.295140in}}%
\pgfpathlineto{\pgfqpoint{4.772557in}{2.291727in}}%
\pgfpathlineto{\pgfqpoint{4.758521in}{2.288415in}}%
\pgfpathlineto{\pgfqpoint{4.766262in}{2.299450in}}%
\pgfpathlineto{\pgfqpoint{4.773996in}{2.310434in}}%
\pgfpathlineto{\pgfqpoint{4.781726in}{2.321365in}}%
\pgfpathlineto{\pgfqpoint{4.789450in}{2.332242in}}%
\pgfpathclose%
\pgfusepath{fill}%
\end{pgfscope}%
\begin{pgfscope}%
\pgfpathrectangle{\pgfqpoint{1.150000in}{0.150000in}}{\pgfqpoint{5.700000in}{5.700000in}}%
\pgfusepath{clip}%
\pgfsetbuttcap%
\pgfsetroundjoin%
\definecolor{currentfill}{rgb}{0.208623,0.367752,0.552675}%
\pgfsetfillcolor{currentfill}%
\pgfsetfillopacity{0.700000}%
\pgfsetlinewidth{0.000000pt}%
\definecolor{currentstroke}{rgb}{0.000000,0.000000,0.000000}%
\pgfsetstrokecolor{currentstroke}%
\pgfsetdash{}{0pt}%
\pgfpathmoveto{\pgfqpoint{5.255441in}{2.674068in}}%
\pgfpathlineto{\pgfqpoint{5.269699in}{2.680097in}}%
\pgfpathlineto{\pgfqpoint{5.283970in}{2.686225in}}%
\pgfpathlineto{\pgfqpoint{5.298256in}{2.692453in}}%
\pgfpathlineto{\pgfqpoint{5.312556in}{2.698782in}}%
\pgfpathlineto{\pgfqpoint{5.305014in}{2.689236in}}%
\pgfpathlineto{\pgfqpoint{5.297465in}{2.679591in}}%
\pgfpathlineto{\pgfqpoint{5.289908in}{2.669847in}}%
\pgfpathlineto{\pgfqpoint{5.282344in}{2.660005in}}%
\pgfpathlineto{\pgfqpoint{5.268038in}{2.653722in}}%
\pgfpathlineto{\pgfqpoint{5.253747in}{2.647539in}}%
\pgfpathlineto{\pgfqpoint{5.239469in}{2.641456in}}%
\pgfpathlineto{\pgfqpoint{5.225205in}{2.635473in}}%
\pgfpathlineto{\pgfqpoint{5.232775in}{2.645262in}}%
\pgfpathlineto{\pgfqpoint{5.240337in}{2.654958in}}%
\pgfpathlineto{\pgfqpoint{5.247893in}{2.664560in}}%
\pgfpathlineto{\pgfqpoint{5.255441in}{2.674068in}}%
\pgfpathclose%
\pgfusepath{fill}%
\end{pgfscope}%
\begin{pgfscope}%
\pgfpathrectangle{\pgfqpoint{1.150000in}{0.150000in}}{\pgfqpoint{5.700000in}{5.700000in}}%
\pgfusepath{clip}%
\pgfsetbuttcap%
\pgfsetroundjoin%
\definecolor{currentfill}{rgb}{0.151918,0.500685,0.557587}%
\pgfsetfillcolor{currentfill}%
\pgfsetfillopacity{0.700000}%
\pgfsetlinewidth{0.000000pt}%
\definecolor{currentstroke}{rgb}{0.000000,0.000000,0.000000}%
\pgfsetstrokecolor{currentstroke}%
\pgfsetdash{}{0pt}%
\pgfpathmoveto{\pgfqpoint{5.721279in}{3.011470in}}%
\pgfpathlineto{\pgfqpoint{5.735781in}{3.019364in}}%
\pgfpathlineto{\pgfqpoint{5.750300in}{3.027359in}}%
\pgfpathlineto{\pgfqpoint{5.764835in}{3.035453in}}%
\pgfpathlineto{\pgfqpoint{5.779386in}{3.043648in}}%
\pgfpathlineto{\pgfqpoint{5.772084in}{3.036776in}}%
\pgfpathlineto{\pgfqpoint{5.764772in}{3.029793in}}%
\pgfpathlineto{\pgfqpoint{5.757452in}{3.022698in}}%
\pgfpathlineto{\pgfqpoint{5.750123in}{3.015489in}}%
\pgfpathlineto{\pgfqpoint{5.735561in}{3.007223in}}%
\pgfpathlineto{\pgfqpoint{5.721015in}{2.999056in}}%
\pgfpathlineto{\pgfqpoint{5.706485in}{2.990990in}}%
\pgfpathlineto{\pgfqpoint{5.691970in}{2.983025in}}%
\pgfpathlineto{\pgfqpoint{5.699310in}{2.990298in}}%
\pgfpathlineto{\pgfqpoint{5.706642in}{2.997462in}}%
\pgfpathlineto{\pgfqpoint{5.713965in}{3.004519in}}%
\pgfpathlineto{\pgfqpoint{5.721279in}{3.011470in}}%
\pgfpathclose%
\pgfusepath{fill}%
\end{pgfscope}%
\begin{pgfscope}%
\pgfpathrectangle{\pgfqpoint{1.150000in}{0.150000in}}{\pgfqpoint{5.700000in}{5.700000in}}%
\pgfusepath{clip}%
\pgfsetbuttcap%
\pgfsetroundjoin%
\definecolor{currentfill}{rgb}{0.280255,0.165693,0.476498}%
\pgfsetfillcolor{currentfill}%
\pgfsetfillopacity{0.700000}%
\pgfsetlinewidth{0.000000pt}%
\definecolor{currentstroke}{rgb}{0.000000,0.000000,0.000000}%
\pgfsetstrokecolor{currentstroke}%
\pgfsetdash{}{0pt}%
\pgfpathmoveto{\pgfqpoint{2.990872in}{2.224782in}}%
\pgfpathlineto{\pgfqpoint{3.004555in}{2.213106in}}%
\pgfpathlineto{\pgfqpoint{3.018237in}{2.201571in}}%
\pgfpathlineto{\pgfqpoint{3.031919in}{2.190175in}}%
\pgfpathlineto{\pgfqpoint{3.045600in}{2.178917in}}%
\pgfpathlineto{\pgfqpoint{3.037111in}{2.178446in}}%
\pgfpathlineto{\pgfqpoint{3.028609in}{2.178192in}}%
\pgfpathlineto{\pgfqpoint{3.020094in}{2.178159in}}%
\pgfpathlineto{\pgfqpoint{3.011565in}{2.178351in}}%
\pgfpathlineto{\pgfqpoint{2.997849in}{2.190049in}}%
\pgfpathlineto{\pgfqpoint{2.984133in}{2.201886in}}%
\pgfpathlineto{\pgfqpoint{2.970416in}{2.213862in}}%
\pgfpathlineto{\pgfqpoint{2.956698in}{2.225979in}}%
\pgfpathlineto{\pgfqpoint{2.965262in}{2.225339in}}%
\pgfpathlineto{\pgfqpoint{2.973812in}{2.224929in}}%
\pgfpathlineto{\pgfqpoint{2.982349in}{2.224745in}}%
\pgfpathlineto{\pgfqpoint{2.990872in}{2.224782in}}%
\pgfpathclose%
\pgfusepath{fill}%
\end{pgfscope}%
\begin{pgfscope}%
\pgfpathrectangle{\pgfqpoint{1.150000in}{0.150000in}}{\pgfqpoint{5.700000in}{5.700000in}}%
\pgfusepath{clip}%
\pgfsetbuttcap%
\pgfsetroundjoin%
\definecolor{currentfill}{rgb}{0.260571,0.246922,0.522828}%
\pgfsetfillcolor{currentfill}%
\pgfsetfillopacity{0.700000}%
\pgfsetlinewidth{0.000000pt}%
\definecolor{currentstroke}{rgb}{0.000000,0.000000,0.000000}%
\pgfsetstrokecolor{currentstroke}%
\pgfsetdash{}{0pt}%
\pgfpathmoveto{\pgfqpoint{4.876468in}{2.390221in}}%
\pgfpathlineto{\pgfqpoint{4.890543in}{2.394232in}}%
\pgfpathlineto{\pgfqpoint{4.904630in}{2.398343in}}%
\pgfpathlineto{\pgfqpoint{4.918729in}{2.402554in}}%
\pgfpathlineto{\pgfqpoint{4.932841in}{2.406865in}}%
\pgfpathlineto{\pgfqpoint{4.925148in}{2.395953in}}%
\pgfpathlineto{\pgfqpoint{4.917450in}{2.384973in}}%
\pgfpathlineto{\pgfqpoint{4.909746in}{2.373925in}}%
\pgfpathlineto{\pgfqpoint{4.902036in}{2.362812in}}%
\pgfpathlineto{\pgfqpoint{4.887921in}{2.358639in}}%
\pgfpathlineto{\pgfqpoint{4.873817in}{2.354568in}}%
\pgfpathlineto{\pgfqpoint{4.859726in}{2.350596in}}%
\pgfpathlineto{\pgfqpoint{4.845647in}{2.346724in}}%
\pgfpathlineto{\pgfqpoint{4.853361in}{2.357692in}}%
\pgfpathlineto{\pgfqpoint{4.861069in}{2.368597in}}%
\pgfpathlineto{\pgfqpoint{4.868772in}{2.379441in}}%
\pgfpathlineto{\pgfqpoint{4.876468in}{2.390221in}}%
\pgfpathclose%
\pgfusepath{fill}%
\end{pgfscope}%
\begin{pgfscope}%
\pgfpathrectangle{\pgfqpoint{1.150000in}{0.150000in}}{\pgfqpoint{5.700000in}{5.700000in}}%
\pgfusepath{clip}%
\pgfsetbuttcap%
\pgfsetroundjoin%
\definecolor{currentfill}{rgb}{0.268510,0.009605,0.335427}%
\pgfsetfillcolor{currentfill}%
\pgfsetfillopacity{0.700000}%
\pgfsetlinewidth{0.000000pt}%
\definecolor{currentstroke}{rgb}{0.000000,0.000000,0.000000}%
\pgfsetstrokecolor{currentstroke}%
\pgfsetdash{}{0pt}%
\pgfpathmoveto{\pgfqpoint{3.637007in}{1.922886in}}%
\pgfpathlineto{\pgfqpoint{3.650693in}{1.917406in}}%
\pgfpathlineto{\pgfqpoint{3.664382in}{1.912038in}}%
\pgfpathlineto{\pgfqpoint{3.678076in}{1.906783in}}%
\pgfpathlineto{\pgfqpoint{3.691776in}{1.901640in}}%
\pgfpathlineto{\pgfqpoint{3.683660in}{1.895208in}}%
\pgfpathlineto{\pgfqpoint{3.675536in}{1.888897in}}%
\pgfpathlineto{\pgfqpoint{3.667405in}{1.882710in}}%
\pgfpathlineto{\pgfqpoint{3.659266in}{1.876650in}}%
\pgfpathlineto{\pgfqpoint{3.645548in}{1.882167in}}%
\pgfpathlineto{\pgfqpoint{3.631835in}{1.887796in}}%
\pgfpathlineto{\pgfqpoint{3.618126in}{1.893537in}}%
\pgfpathlineto{\pgfqpoint{3.604422in}{1.899391in}}%
\pgfpathlineto{\pgfqpoint{3.612580in}{1.905070in}}%
\pgfpathlineto{\pgfqpoint{3.620730in}{1.910881in}}%
\pgfpathlineto{\pgfqpoint{3.628873in}{1.916821in}}%
\pgfpathlineto{\pgfqpoint{3.637007in}{1.922886in}}%
\pgfpathclose%
\pgfusepath{fill}%
\end{pgfscope}%
\begin{pgfscope}%
\pgfpathrectangle{\pgfqpoint{1.150000in}{0.150000in}}{\pgfqpoint{5.700000in}{5.700000in}}%
\pgfusepath{clip}%
\pgfsetbuttcap%
\pgfsetroundjoin%
\definecolor{currentfill}{rgb}{0.269944,0.014625,0.341379}%
\pgfsetfillcolor{currentfill}%
\pgfsetfillopacity{0.700000}%
\pgfsetlinewidth{0.000000pt}%
\definecolor{currentstroke}{rgb}{0.000000,0.000000,0.000000}%
\pgfsetstrokecolor{currentstroke}%
\pgfsetdash{}{0pt}%
\pgfpathmoveto{\pgfqpoint{4.007814in}{1.935760in}}%
\pgfpathlineto{\pgfqpoint{4.021569in}{1.933475in}}%
\pgfpathlineto{\pgfqpoint{4.035332in}{1.931296in}}%
\pgfpathlineto{\pgfqpoint{4.049103in}{1.929221in}}%
\pgfpathlineto{\pgfqpoint{4.062880in}{1.927253in}}%
\pgfpathlineto{\pgfqpoint{4.054913in}{1.918081in}}%
\pgfpathlineto{\pgfqpoint{4.046941in}{1.908968in}}%
\pgfpathlineto{\pgfqpoint{4.038963in}{1.899917in}}%
\pgfpathlineto{\pgfqpoint{4.030980in}{1.890929in}}%
\pgfpathlineto{\pgfqpoint{4.017191in}{1.893217in}}%
\pgfpathlineto{\pgfqpoint{4.003409in}{1.895610in}}%
\pgfpathlineto{\pgfqpoint{3.989635in}{1.898108in}}%
\pgfpathlineto{\pgfqpoint{3.975867in}{1.900712in}}%
\pgfpathlineto{\pgfqpoint{3.983862in}{1.909374in}}%
\pgfpathlineto{\pgfqpoint{3.991852in}{1.918104in}}%
\pgfpathlineto{\pgfqpoint{3.999836in}{1.926901in}}%
\pgfpathlineto{\pgfqpoint{4.007814in}{1.935760in}}%
\pgfpathclose%
\pgfusepath{fill}%
\end{pgfscope}%
\begin{pgfscope}%
\pgfpathrectangle{\pgfqpoint{1.150000in}{0.150000in}}{\pgfqpoint{5.700000in}{5.700000in}}%
\pgfusepath{clip}%
\pgfsetbuttcap%
\pgfsetroundjoin%
\definecolor{currentfill}{rgb}{0.267004,0.004874,0.329415}%
\pgfsetfillcolor{currentfill}%
\pgfsetfillopacity{0.700000}%
\pgfsetlinewidth{0.000000pt}%
\definecolor{currentstroke}{rgb}{0.000000,0.000000,0.000000}%
\pgfsetstrokecolor{currentstroke}%
\pgfsetdash{}{0pt}%
\pgfpathmoveto{\pgfqpoint{3.778944in}{1.910461in}}%
\pgfpathlineto{\pgfqpoint{3.792651in}{1.906227in}}%
\pgfpathlineto{\pgfqpoint{3.806365in}{1.902103in}}%
\pgfpathlineto{\pgfqpoint{3.820084in}{1.898088in}}%
\pgfpathlineto{\pgfqpoint{3.833809in}{1.894182in}}%
\pgfpathlineto{\pgfqpoint{3.825754in}{1.886608in}}%
\pgfpathlineto{\pgfqpoint{3.817693in}{1.879130in}}%
\pgfpathlineto{\pgfqpoint{3.809625in}{1.871754in}}%
\pgfpathlineto{\pgfqpoint{3.801550in}{1.864482in}}%
\pgfpathlineto{\pgfqpoint{3.787810in}{1.868743in}}%
\pgfpathlineto{\pgfqpoint{3.774075in}{1.873112in}}%
\pgfpathlineto{\pgfqpoint{3.760345in}{1.877591in}}%
\pgfpathlineto{\pgfqpoint{3.746621in}{1.882180in}}%
\pgfpathlineto{\pgfqpoint{3.754712in}{1.889091in}}%
\pgfpathlineto{\pgfqpoint{3.762796in}{1.896110in}}%
\pgfpathlineto{\pgfqpoint{3.770873in}{1.903235in}}%
\pgfpathlineto{\pgfqpoint{3.778944in}{1.910461in}}%
\pgfpathclose%
\pgfusepath{fill}%
\end{pgfscope}%
\begin{pgfscope}%
\pgfpathrectangle{\pgfqpoint{1.150000in}{0.150000in}}{\pgfqpoint{5.700000in}{5.700000in}}%
\pgfusepath{clip}%
\pgfsetbuttcap%
\pgfsetroundjoin%
\definecolor{currentfill}{rgb}{0.279566,0.067836,0.391917}%
\pgfsetfillcolor{currentfill}%
\pgfsetfillopacity{0.700000}%
\pgfsetlinewidth{0.000000pt}%
\definecolor{currentstroke}{rgb}{0.000000,0.000000,0.000000}%
\pgfsetstrokecolor{currentstroke}%
\pgfsetdash{}{0pt}%
\pgfpathmoveto{\pgfqpoint{3.297898in}{2.027539in}}%
\pgfpathlineto{\pgfqpoint{3.311563in}{2.018922in}}%
\pgfpathlineto{\pgfqpoint{3.325231in}{2.010428in}}%
\pgfpathlineto{\pgfqpoint{3.338900in}{2.002058in}}%
\pgfpathlineto{\pgfqpoint{3.352572in}{1.993810in}}%
\pgfpathlineto{\pgfqpoint{3.344277in}{1.990472in}}%
\pgfpathlineto{\pgfqpoint{3.335971in}{1.987308in}}%
\pgfpathlineto{\pgfqpoint{3.327655in}{1.984323in}}%
\pgfpathlineto{\pgfqpoint{3.319329in}{1.981522in}}%
\pgfpathlineto{\pgfqpoint{3.305630in}{1.990183in}}%
\pgfpathlineto{\pgfqpoint{3.291934in}{1.998968in}}%
\pgfpathlineto{\pgfqpoint{3.278239in}{2.007876in}}%
\pgfpathlineto{\pgfqpoint{3.264546in}{2.016908in}}%
\pgfpathlineto{\pgfqpoint{3.272901in}{2.019288in}}%
\pgfpathlineto{\pgfqpoint{3.281244in}{2.021856in}}%
\pgfpathlineto{\pgfqpoint{3.289576in}{2.024608in}}%
\pgfpathlineto{\pgfqpoint{3.297898in}{2.027539in}}%
\pgfpathclose%
\pgfusepath{fill}%
\end{pgfscope}%
\begin{pgfscope}%
\pgfpathrectangle{\pgfqpoint{1.150000in}{0.150000in}}{\pgfqpoint{5.700000in}{5.700000in}}%
\pgfusepath{clip}%
\pgfsetbuttcap%
\pgfsetroundjoin%
\definecolor{currentfill}{rgb}{0.143343,0.522773,0.556295}%
\pgfsetfillcolor{currentfill}%
\pgfsetfillopacity{0.700000}%
\pgfsetlinewidth{0.000000pt}%
\definecolor{currentstroke}{rgb}{0.000000,0.000000,0.000000}%
\pgfsetstrokecolor{currentstroke}%
\pgfsetdash{}{0pt}%
\pgfpathmoveto{\pgfqpoint{5.808504in}{3.070045in}}%
\pgfpathlineto{\pgfqpoint{5.823058in}{3.078248in}}%
\pgfpathlineto{\pgfqpoint{5.837629in}{3.086551in}}%
\pgfpathlineto{\pgfqpoint{5.852216in}{3.094955in}}%
\pgfpathlineto{\pgfqpoint{5.866819in}{3.103458in}}%
\pgfpathlineto{\pgfqpoint{5.859566in}{3.097119in}}%
\pgfpathlineto{\pgfqpoint{5.852304in}{3.090669in}}%
\pgfpathlineto{\pgfqpoint{5.845032in}{3.084105in}}%
\pgfpathlineto{\pgfqpoint{5.837751in}{3.077429in}}%
\pgfpathlineto{\pgfqpoint{5.823135in}{3.068833in}}%
\pgfpathlineto{\pgfqpoint{5.808536in}{3.060337in}}%
\pgfpathlineto{\pgfqpoint{5.793953in}{3.051942in}}%
\pgfpathlineto{\pgfqpoint{5.779386in}{3.043648in}}%
\pgfpathlineto{\pgfqpoint{5.786679in}{3.050409in}}%
\pgfpathlineto{\pgfqpoint{5.793963in}{3.057062in}}%
\pgfpathlineto{\pgfqpoint{5.801238in}{3.063606in}}%
\pgfpathlineto{\pgfqpoint{5.808504in}{3.070045in}}%
\pgfpathclose%
\pgfusepath{fill}%
\end{pgfscope}%
\begin{pgfscope}%
\pgfpathrectangle{\pgfqpoint{1.150000in}{0.150000in}}{\pgfqpoint{5.700000in}{5.700000in}}%
\pgfusepath{clip}%
\pgfsetbuttcap%
\pgfsetroundjoin%
\definecolor{currentfill}{rgb}{0.195860,0.395433,0.555276}%
\pgfsetfillcolor{currentfill}%
\pgfsetfillopacity{0.700000}%
\pgfsetlinewidth{0.000000pt}%
\definecolor{currentstroke}{rgb}{0.000000,0.000000,0.000000}%
\pgfsetstrokecolor{currentstroke}%
\pgfsetdash{}{0pt}%
\pgfpathmoveto{\pgfqpoint{5.342651in}{2.735976in}}%
\pgfpathlineto{\pgfqpoint{5.356959in}{2.742430in}}%
\pgfpathlineto{\pgfqpoint{5.371281in}{2.748985in}}%
\pgfpathlineto{\pgfqpoint{5.385617in}{2.755640in}}%
\pgfpathlineto{\pgfqpoint{5.399968in}{2.762394in}}%
\pgfpathlineto{\pgfqpoint{5.392462in}{2.753225in}}%
\pgfpathlineto{\pgfqpoint{5.384949in}{2.743953in}}%
\pgfpathlineto{\pgfqpoint{5.377428in}{2.734576in}}%
\pgfpathlineto{\pgfqpoint{5.369899in}{2.725096in}}%
\pgfpathlineto{\pgfqpoint{5.355542in}{2.718367in}}%
\pgfpathlineto{\pgfqpoint{5.341199in}{2.711739in}}%
\pgfpathlineto{\pgfqpoint{5.326870in}{2.705210in}}%
\pgfpathlineto{\pgfqpoint{5.312556in}{2.698782in}}%
\pgfpathlineto{\pgfqpoint{5.320091in}{2.708229in}}%
\pgfpathlineto{\pgfqpoint{5.327618in}{2.717577in}}%
\pgfpathlineto{\pgfqpoint{5.335138in}{2.726826in}}%
\pgfpathlineto{\pgfqpoint{5.342651in}{2.735976in}}%
\pgfpathclose%
\pgfusepath{fill}%
\end{pgfscope}%
\begin{pgfscope}%
\pgfpathrectangle{\pgfqpoint{1.150000in}{0.150000in}}{\pgfqpoint{5.700000in}{5.700000in}}%
\pgfusepath{clip}%
\pgfsetbuttcap%
\pgfsetroundjoin%
\definecolor{currentfill}{rgb}{0.272594,0.025563,0.353093}%
\pgfsetfillcolor{currentfill}%
\pgfsetfillopacity{0.700000}%
\pgfsetlinewidth{0.000000pt}%
\definecolor{currentstroke}{rgb}{0.000000,0.000000,0.000000}%
\pgfsetstrokecolor{currentstroke}%
\pgfsetdash{}{0pt}%
\pgfpathmoveto{\pgfqpoint{3.494929in}{1.950352in}}%
\pgfpathlineto{\pgfqpoint{3.508602in}{1.943576in}}%
\pgfpathlineto{\pgfqpoint{3.522279in}{1.936917in}}%
\pgfpathlineto{\pgfqpoint{3.535960in}{1.930375in}}%
\pgfpathlineto{\pgfqpoint{3.549645in}{1.923948in}}%
\pgfpathlineto{\pgfqpoint{3.541457in}{1.918791in}}%
\pgfpathlineto{\pgfqpoint{3.533261in}{1.913779in}}%
\pgfpathlineto{\pgfqpoint{3.525057in}{1.908915in}}%
\pgfpathlineto{\pgfqpoint{3.516843in}{1.904203in}}%
\pgfpathlineto{\pgfqpoint{3.503137in}{1.911023in}}%
\pgfpathlineto{\pgfqpoint{3.489434in}{1.917958in}}%
\pgfpathlineto{\pgfqpoint{3.475734in}{1.925010in}}%
\pgfpathlineto{\pgfqpoint{3.462038in}{1.932179in}}%
\pgfpathlineto{\pgfqpoint{3.470274in}{1.936491in}}%
\pgfpathlineto{\pgfqpoint{3.478501in}{1.940959in}}%
\pgfpathlineto{\pgfqpoint{3.486720in}{1.945581in}}%
\pgfpathlineto{\pgfqpoint{3.494929in}{1.950352in}}%
\pgfpathclose%
\pgfusepath{fill}%
\end{pgfscope}%
\begin{pgfscope}%
\pgfpathrectangle{\pgfqpoint{1.150000in}{0.150000in}}{\pgfqpoint{5.700000in}{5.700000in}}%
\pgfusepath{clip}%
\pgfsetbuttcap%
\pgfsetroundjoin%
\definecolor{currentfill}{rgb}{0.248629,0.278775,0.534556}%
\pgfsetfillcolor{currentfill}%
\pgfsetfillopacity{0.700000}%
\pgfsetlinewidth{0.000000pt}%
\definecolor{currentstroke}{rgb}{0.000000,0.000000,0.000000}%
\pgfsetstrokecolor{currentstroke}%
\pgfsetdash{}{0pt}%
\pgfpathmoveto{\pgfqpoint{4.963552in}{2.449812in}}%
\pgfpathlineto{\pgfqpoint{4.977671in}{2.454345in}}%
\pgfpathlineto{\pgfqpoint{4.991804in}{2.458977in}}%
\pgfpathlineto{\pgfqpoint{5.005949in}{2.463710in}}%
\pgfpathlineto{\pgfqpoint{5.020107in}{2.468543in}}%
\pgfpathlineto{\pgfqpoint{5.012442in}{2.457799in}}%
\pgfpathlineto{\pgfqpoint{5.004772in}{2.446979in}}%
\pgfpathlineto{\pgfqpoint{4.997095in}{2.436083in}}%
\pgfpathlineto{\pgfqpoint{4.989413in}{2.425112in}}%
\pgfpathlineto{\pgfqpoint{4.975251in}{2.420400in}}%
\pgfpathlineto{\pgfqpoint{4.961102in}{2.415789in}}%
\pgfpathlineto{\pgfqpoint{4.946965in}{2.411277in}}%
\pgfpathlineto{\pgfqpoint{4.932841in}{2.406865in}}%
\pgfpathlineto{\pgfqpoint{4.940528in}{2.417708in}}%
\pgfpathlineto{\pgfqpoint{4.948208in}{2.428481in}}%
\pgfpathlineto{\pgfqpoint{4.955883in}{2.439183in}}%
\pgfpathlineto{\pgfqpoint{4.963552in}{2.449812in}}%
\pgfpathclose%
\pgfusepath{fill}%
\end{pgfscope}%
\begin{pgfscope}%
\pgfpathrectangle{\pgfqpoint{1.150000in}{0.150000in}}{\pgfqpoint{5.700000in}{5.700000in}}%
\pgfusepath{clip}%
\pgfsetbuttcap%
\pgfsetroundjoin%
\definecolor{currentfill}{rgb}{0.282290,0.145912,0.461510}%
\pgfsetfillcolor{currentfill}%
\pgfsetfillopacity{0.700000}%
\pgfsetlinewidth{0.000000pt}%
\definecolor{currentstroke}{rgb}{0.000000,0.000000,0.000000}%
\pgfsetstrokecolor{currentstroke}%
\pgfsetdash{}{0pt}%
\pgfpathmoveto{\pgfqpoint{3.045600in}{2.178917in}}%
\pgfpathlineto{\pgfqpoint{3.059281in}{2.167796in}}%
\pgfpathlineto{\pgfqpoint{3.072962in}{2.156812in}}%
\pgfpathlineto{\pgfqpoint{3.086643in}{2.145963in}}%
\pgfpathlineto{\pgfqpoint{3.100324in}{2.135250in}}%
\pgfpathlineto{\pgfqpoint{3.091868in}{2.134348in}}%
\pgfpathlineto{\pgfqpoint{3.083399in}{2.133657in}}%
\pgfpathlineto{\pgfqpoint{3.074917in}{2.133183in}}%
\pgfpathlineto{\pgfqpoint{3.066422in}{2.132928in}}%
\pgfpathlineto{\pgfqpoint{3.052708in}{2.144080in}}%
\pgfpathlineto{\pgfqpoint{3.038994in}{2.155367in}}%
\pgfpathlineto{\pgfqpoint{3.025280in}{2.166791in}}%
\pgfpathlineto{\pgfqpoint{3.011565in}{2.178351in}}%
\pgfpathlineto{\pgfqpoint{3.020094in}{2.178159in}}%
\pgfpathlineto{\pgfqpoint{3.028609in}{2.178192in}}%
\pgfpathlineto{\pgfqpoint{3.037111in}{2.178446in}}%
\pgfpathlineto{\pgfqpoint{3.045600in}{2.178917in}}%
\pgfpathclose%
\pgfusepath{fill}%
\end{pgfscope}%
\begin{pgfscope}%
\pgfpathrectangle{\pgfqpoint{1.150000in}{0.150000in}}{\pgfqpoint{5.700000in}{5.700000in}}%
\pgfusepath{clip}%
\pgfsetbuttcap%
\pgfsetroundjoin%
\definecolor{currentfill}{rgb}{0.135066,0.544853,0.554029}%
\pgfsetfillcolor{currentfill}%
\pgfsetfillopacity{0.700000}%
\pgfsetlinewidth{0.000000pt}%
\definecolor{currentstroke}{rgb}{0.000000,0.000000,0.000000}%
\pgfsetstrokecolor{currentstroke}%
\pgfsetdash{}{0pt}%
\pgfpathmoveto{\pgfqpoint{5.895738in}{3.127729in}}%
\pgfpathlineto{\pgfqpoint{5.910343in}{3.136221in}}%
\pgfpathlineto{\pgfqpoint{5.924966in}{3.144814in}}%
\pgfpathlineto{\pgfqpoint{5.939604in}{3.153506in}}%
\pgfpathlineto{\pgfqpoint{5.954260in}{3.162299in}}%
\pgfpathlineto{\pgfqpoint{5.947059in}{3.156511in}}%
\pgfpathlineto{\pgfqpoint{5.939848in}{3.150612in}}%
\pgfpathlineto{\pgfqpoint{5.932628in}{3.144601in}}%
\pgfpathlineto{\pgfqpoint{5.925398in}{3.138476in}}%
\pgfpathlineto{\pgfqpoint{5.910728in}{3.129571in}}%
\pgfpathlineto{\pgfqpoint{5.896075in}{3.120766in}}%
\pgfpathlineto{\pgfqpoint{5.881439in}{3.112062in}}%
\pgfpathlineto{\pgfqpoint{5.866819in}{3.103458in}}%
\pgfpathlineto{\pgfqpoint{5.874062in}{3.109687in}}%
\pgfpathlineto{\pgfqpoint{5.881297in}{3.115808in}}%
\pgfpathlineto{\pgfqpoint{5.888522in}{3.121821in}}%
\pgfpathlineto{\pgfqpoint{5.895738in}{3.127729in}}%
\pgfpathclose%
\pgfusepath{fill}%
\end{pgfscope}%
\begin{pgfscope}%
\pgfpathrectangle{\pgfqpoint{1.150000in}{0.150000in}}{\pgfqpoint{5.700000in}{5.700000in}}%
\pgfusepath{clip}%
\pgfsetbuttcap%
\pgfsetroundjoin%
\definecolor{currentfill}{rgb}{0.267004,0.004874,0.329415}%
\pgfsetfillcolor{currentfill}%
\pgfsetfillopacity{0.700000}%
\pgfsetlinewidth{0.000000pt}%
\definecolor{currentstroke}{rgb}{0.000000,0.000000,0.000000}%
\pgfsetstrokecolor{currentstroke}%
\pgfsetdash{}{0pt}%
\pgfpathmoveto{\pgfqpoint{3.920862in}{1.912191in}}%
\pgfpathlineto{\pgfqpoint{3.934604in}{1.909161in}}%
\pgfpathlineto{\pgfqpoint{3.948351in}{1.906238in}}%
\pgfpathlineto{\pgfqpoint{3.962106in}{1.903422in}}%
\pgfpathlineto{\pgfqpoint{3.975867in}{1.900712in}}%
\pgfpathlineto{\pgfqpoint{3.967865in}{1.892122in}}%
\pgfpathlineto{\pgfqpoint{3.959858in}{1.883607in}}%
\pgfpathlineto{\pgfqpoint{3.951844in}{1.875170in}}%
\pgfpathlineto{\pgfqpoint{3.943825in}{1.866814in}}%
\pgfpathlineto{\pgfqpoint{3.930051in}{1.869861in}}%
\pgfpathlineto{\pgfqpoint{3.916283in}{1.873014in}}%
\pgfpathlineto{\pgfqpoint{3.902522in}{1.876274in}}%
\pgfpathlineto{\pgfqpoint{3.888767in}{1.879640in}}%
\pgfpathlineto{\pgfqpoint{3.896800in}{1.887652in}}%
\pgfpathlineto{\pgfqpoint{3.904827in}{1.895750in}}%
\pgfpathlineto{\pgfqpoint{3.912848in}{1.903931in}}%
\pgfpathlineto{\pgfqpoint{3.920862in}{1.912191in}}%
\pgfpathclose%
\pgfusepath{fill}%
\end{pgfscope}%
\begin{pgfscope}%
\pgfpathrectangle{\pgfqpoint{1.150000in}{0.150000in}}{\pgfqpoint{5.700000in}{5.700000in}}%
\pgfusepath{clip}%
\pgfsetbuttcap%
\pgfsetroundjoin%
\definecolor{currentfill}{rgb}{0.282656,0.100196,0.422160}%
\pgfsetfillcolor{currentfill}%
\pgfsetfillopacity{0.700000}%
\pgfsetlinewidth{0.000000pt}%
\definecolor{currentstroke}{rgb}{0.000000,0.000000,0.000000}%
\pgfsetstrokecolor{currentstroke}%
\pgfsetdash{}{0pt}%
\pgfpathmoveto{\pgfqpoint{4.410561in}{2.079813in}}%
\pgfpathlineto{\pgfqpoint{4.424448in}{2.080698in}}%
\pgfpathlineto{\pgfqpoint{4.438345in}{2.081684in}}%
\pgfpathlineto{\pgfqpoint{4.452252in}{2.082771in}}%
\pgfpathlineto{\pgfqpoint{4.466168in}{2.083961in}}%
\pgfpathlineto{\pgfqpoint{4.458325in}{2.073024in}}%
\pgfpathlineto{\pgfqpoint{4.450478in}{2.062081in}}%
\pgfpathlineto{\pgfqpoint{4.442625in}{2.051135in}}%
\pgfpathlineto{\pgfqpoint{4.434768in}{2.040188in}}%
\pgfpathlineto{\pgfqpoint{4.420846in}{2.039247in}}%
\pgfpathlineto{\pgfqpoint{4.406933in}{2.038407in}}%
\pgfpathlineto{\pgfqpoint{4.393030in}{2.037669in}}%
\pgfpathlineto{\pgfqpoint{4.379136in}{2.037032in}}%
\pgfpathlineto{\pgfqpoint{4.387000in}{2.047725in}}%
\pgfpathlineto{\pgfqpoint{4.394858in}{2.058421in}}%
\pgfpathlineto{\pgfqpoint{4.402712in}{2.069118in}}%
\pgfpathlineto{\pgfqpoint{4.410561in}{2.079813in}}%
\pgfpathclose%
\pgfusepath{fill}%
\end{pgfscope}%
\begin{pgfscope}%
\pgfpathrectangle{\pgfqpoint{1.150000in}{0.150000in}}{\pgfqpoint{5.700000in}{5.700000in}}%
\pgfusepath{clip}%
\pgfsetbuttcap%
\pgfsetroundjoin%
\definecolor{currentfill}{rgb}{0.183898,0.422383,0.556944}%
\pgfsetfillcolor{currentfill}%
\pgfsetfillopacity{0.700000}%
\pgfsetlinewidth{0.000000pt}%
\definecolor{currentstroke}{rgb}{0.000000,0.000000,0.000000}%
\pgfsetstrokecolor{currentstroke}%
\pgfsetdash{}{0pt}%
\pgfpathmoveto{\pgfqpoint{5.429915in}{2.798035in}}%
\pgfpathlineto{\pgfqpoint{5.444273in}{2.804897in}}%
\pgfpathlineto{\pgfqpoint{5.458647in}{2.811859in}}%
\pgfpathlineto{\pgfqpoint{5.473035in}{2.818920in}}%
\pgfpathlineto{\pgfqpoint{5.487438in}{2.826082in}}%
\pgfpathlineto{\pgfqpoint{5.479970in}{2.817327in}}%
\pgfpathlineto{\pgfqpoint{5.472495in}{2.808464in}}%
\pgfpathlineto{\pgfqpoint{5.465011in}{2.799493in}}%
\pgfpathlineto{\pgfqpoint{5.457520in}{2.790414in}}%
\pgfpathlineto{\pgfqpoint{5.443110in}{2.783259in}}%
\pgfpathlineto{\pgfqpoint{5.428714in}{2.776204in}}%
\pgfpathlineto{\pgfqpoint{5.414334in}{2.769249in}}%
\pgfpathlineto{\pgfqpoint{5.399968in}{2.762394in}}%
\pgfpathlineto{\pgfqpoint{5.407467in}{2.771460in}}%
\pgfpathlineto{\pgfqpoint{5.414957in}{2.780421in}}%
\pgfpathlineto{\pgfqpoint{5.422440in}{2.789280in}}%
\pgfpathlineto{\pgfqpoint{5.429915in}{2.798035in}}%
\pgfpathclose%
\pgfusepath{fill}%
\end{pgfscope}%
\begin{pgfscope}%
\pgfpathrectangle{\pgfqpoint{1.150000in}{0.150000in}}{\pgfqpoint{5.700000in}{5.700000in}}%
\pgfusepath{clip}%
\pgfsetbuttcap%
\pgfsetroundjoin%
\definecolor{currentfill}{rgb}{0.280894,0.078907,0.402329}%
\pgfsetfillcolor{currentfill}%
\pgfsetfillopacity{0.700000}%
\pgfsetlinewidth{0.000000pt}%
\definecolor{currentstroke}{rgb}{0.000000,0.000000,0.000000}%
\pgfsetstrokecolor{currentstroke}%
\pgfsetdash{}{0pt}%
\pgfpathmoveto{\pgfqpoint{4.323654in}{2.035506in}}%
\pgfpathlineto{\pgfqpoint{4.337510in}{2.035734in}}%
\pgfpathlineto{\pgfqpoint{4.351376in}{2.036065in}}%
\pgfpathlineto{\pgfqpoint{4.365251in}{2.036497in}}%
\pgfpathlineto{\pgfqpoint{4.379136in}{2.037032in}}%
\pgfpathlineto{\pgfqpoint{4.371267in}{2.026345in}}%
\pgfpathlineto{\pgfqpoint{4.363393in}{2.015666in}}%
\pgfpathlineto{\pgfqpoint{4.355515in}{2.004997in}}%
\pgfpathlineto{\pgfqpoint{4.347631in}{1.994341in}}%
\pgfpathlineto{\pgfqpoint{4.333740in}{1.994072in}}%
\pgfpathlineto{\pgfqpoint{4.319857in}{1.993905in}}%
\pgfpathlineto{\pgfqpoint{4.305984in}{1.993840in}}%
\pgfpathlineto{\pgfqpoint{4.292120in}{1.993877in}}%
\pgfpathlineto{\pgfqpoint{4.300011in}{2.004261in}}%
\pgfpathlineto{\pgfqpoint{4.307897in}{2.014662in}}%
\pgfpathlineto{\pgfqpoint{4.315778in}{2.025078in}}%
\pgfpathlineto{\pgfqpoint{4.323654in}{2.035506in}}%
\pgfpathclose%
\pgfusepath{fill}%
\end{pgfscope}%
\begin{pgfscope}%
\pgfpathrectangle{\pgfqpoint{1.150000in}{0.150000in}}{\pgfqpoint{5.700000in}{5.700000in}}%
\pgfusepath{clip}%
\pgfsetbuttcap%
\pgfsetroundjoin%
\definecolor{currentfill}{rgb}{0.283187,0.125848,0.444960}%
\pgfsetfillcolor{currentfill}%
\pgfsetfillopacity{0.700000}%
\pgfsetlinewidth{0.000000pt}%
\definecolor{currentstroke}{rgb}{0.000000,0.000000,0.000000}%
\pgfsetstrokecolor{currentstroke}%
\pgfsetdash{}{0pt}%
\pgfpathmoveto{\pgfqpoint{4.497488in}{2.127607in}}%
\pgfpathlineto{\pgfqpoint{4.511409in}{2.129128in}}%
\pgfpathlineto{\pgfqpoint{4.525340in}{2.130749in}}%
\pgfpathlineto{\pgfqpoint{4.539281in}{2.132473in}}%
\pgfpathlineto{\pgfqpoint{4.553232in}{2.134297in}}%
\pgfpathlineto{\pgfqpoint{4.545415in}{2.123181in}}%
\pgfpathlineto{\pgfqpoint{4.537593in}{2.112047in}}%
\pgfpathlineto{\pgfqpoint{4.529766in}{2.100896in}}%
\pgfpathlineto{\pgfqpoint{4.521933in}{2.089730in}}%
\pgfpathlineto{\pgfqpoint{4.507977in}{2.088136in}}%
\pgfpathlineto{\pgfqpoint{4.494031in}{2.086643in}}%
\pgfpathlineto{\pgfqpoint{4.480094in}{2.085251in}}%
\pgfpathlineto{\pgfqpoint{4.466168in}{2.083961in}}%
\pgfpathlineto{\pgfqpoint{4.474006in}{2.094890in}}%
\pgfpathlineto{\pgfqpoint{4.481838in}{2.105808in}}%
\pgfpathlineto{\pgfqpoint{4.489666in}{2.116715in}}%
\pgfpathlineto{\pgfqpoint{4.497488in}{2.127607in}}%
\pgfpathclose%
\pgfusepath{fill}%
\end{pgfscope}%
\begin{pgfscope}%
\pgfpathrectangle{\pgfqpoint{1.150000in}{0.150000in}}{\pgfqpoint{5.700000in}{5.700000in}}%
\pgfusepath{clip}%
\pgfsetbuttcap%
\pgfsetroundjoin%
\definecolor{currentfill}{rgb}{0.237441,0.305202,0.541921}%
\pgfsetfillcolor{currentfill}%
\pgfsetfillopacity{0.700000}%
\pgfsetlinewidth{0.000000pt}%
\definecolor{currentstroke}{rgb}{0.000000,0.000000,0.000000}%
\pgfsetstrokecolor{currentstroke}%
\pgfsetdash{}{0pt}%
\pgfpathmoveto{\pgfqpoint{5.050702in}{2.510732in}}%
\pgfpathlineto{\pgfqpoint{5.064868in}{2.515767in}}%
\pgfpathlineto{\pgfqpoint{5.079047in}{2.520902in}}%
\pgfpathlineto{\pgfqpoint{5.093240in}{2.526137in}}%
\pgfpathlineto{\pgfqpoint{5.107446in}{2.531472in}}%
\pgfpathlineto{\pgfqpoint{5.099811in}{2.520949in}}%
\pgfpathlineto{\pgfqpoint{5.092170in}{2.510341in}}%
\pgfpathlineto{\pgfqpoint{5.084522in}{2.499649in}}%
\pgfpathlineto{\pgfqpoint{5.076868in}{2.488874in}}%
\pgfpathlineto{\pgfqpoint{5.062658in}{2.483641in}}%
\pgfpathlineto{\pgfqpoint{5.048461in}{2.478508in}}%
\pgfpathlineto{\pgfqpoint{5.034277in}{2.473475in}}%
\pgfpathlineto{\pgfqpoint{5.020107in}{2.468543in}}%
\pgfpathlineto{\pgfqpoint{5.027765in}{2.479208in}}%
\pgfpathlineto{\pgfqpoint{5.035417in}{2.489796in}}%
\pgfpathlineto{\pgfqpoint{5.043062in}{2.500304in}}%
\pgfpathlineto{\pgfqpoint{5.050702in}{2.510732in}}%
\pgfpathclose%
\pgfusepath{fill}%
\end{pgfscope}%
\begin{pgfscope}%
\pgfpathrectangle{\pgfqpoint{1.150000in}{0.150000in}}{\pgfqpoint{5.700000in}{5.700000in}}%
\pgfusepath{clip}%
\pgfsetbuttcap%
\pgfsetroundjoin%
\definecolor{currentfill}{rgb}{0.122606,0.585371,0.546557}%
\pgfsetfillcolor{currentfill}%
\pgfsetfillopacity{0.700000}%
\pgfsetlinewidth{0.000000pt}%
\definecolor{currentstroke}{rgb}{0.000000,0.000000,0.000000}%
\pgfsetstrokecolor{currentstroke}%
\pgfsetdash{}{0pt}%
\pgfpathmoveto{\pgfqpoint{6.070191in}{3.239859in}}%
\pgfpathlineto{\pgfqpoint{6.084899in}{3.248870in}}%
\pgfpathlineto{\pgfqpoint{6.099624in}{3.257980in}}%
\pgfpathlineto{\pgfqpoint{6.114366in}{3.267191in}}%
\pgfpathlineto{\pgfqpoint{6.107271in}{3.262507in}}%
\pgfpathlineto{\pgfqpoint{6.100166in}{3.257717in}}%
\pgfpathlineto{\pgfqpoint{6.093051in}{3.252820in}}%
\pgfpathlineto{\pgfqpoint{6.085926in}{3.247814in}}%
\pgfpathlineto{\pgfqpoint{6.071167in}{3.238450in}}%
\pgfpathlineto{\pgfqpoint{6.056424in}{3.229187in}}%
\pgfpathlineto{\pgfqpoint{6.041699in}{3.220025in}}%
\pgfpathlineto{\pgfqpoint{6.048837in}{3.225140in}}%
\pgfpathlineto{\pgfqpoint{6.055964in}{3.230149in}}%
\pgfpathlineto{\pgfqpoint{6.063082in}{3.235055in}}%
\pgfpathlineto{\pgfqpoint{6.070191in}{3.239859in}}%
\pgfpathclose%
\pgfusepath{fill}%
\end{pgfscope}%
\begin{pgfscope}%
\pgfpathrectangle{\pgfqpoint{1.150000in}{0.150000in}}{\pgfqpoint{5.700000in}{5.700000in}}%
\pgfusepath{clip}%
\pgfsetbuttcap%
\pgfsetroundjoin%
\definecolor{currentfill}{rgb}{0.127568,0.566949,0.550556}%
\pgfsetfillcolor{currentfill}%
\pgfsetfillopacity{0.700000}%
\pgfsetlinewidth{0.000000pt}%
\definecolor{currentstroke}{rgb}{0.000000,0.000000,0.000000}%
\pgfsetstrokecolor{currentstroke}%
\pgfsetdash{}{0pt}%
\pgfpathmoveto{\pgfqpoint{5.982970in}{3.184378in}}%
\pgfpathlineto{\pgfqpoint{5.997627in}{3.193139in}}%
\pgfpathlineto{\pgfqpoint{6.012301in}{3.202001in}}%
\pgfpathlineto{\pgfqpoint{6.026991in}{3.210963in}}%
\pgfpathlineto{\pgfqpoint{6.041699in}{3.220025in}}%
\pgfpathlineto{\pgfqpoint{6.034552in}{3.214802in}}%
\pgfpathlineto{\pgfqpoint{6.027395in}{3.209471in}}%
\pgfpathlineto{\pgfqpoint{6.020228in}{3.204029in}}%
\pgfpathlineto{\pgfqpoint{6.013052in}{3.198475in}}%
\pgfpathlineto{\pgfqpoint{5.998328in}{3.189280in}}%
\pgfpathlineto{\pgfqpoint{5.983622in}{3.180186in}}%
\pgfpathlineto{\pgfqpoint{5.968933in}{3.171192in}}%
\pgfpathlineto{\pgfqpoint{5.954260in}{3.162299in}}%
\pgfpathlineto{\pgfqpoint{5.961452in}{3.167978in}}%
\pgfpathlineto{\pgfqpoint{5.968634in}{3.173550in}}%
\pgfpathlineto{\pgfqpoint{5.975807in}{3.179016in}}%
\pgfpathlineto{\pgfqpoint{5.982970in}{3.184378in}}%
\pgfpathclose%
\pgfusepath{fill}%
\end{pgfscope}%
\begin{pgfscope}%
\pgfpathrectangle{\pgfqpoint{1.150000in}{0.150000in}}{\pgfqpoint{5.700000in}{5.700000in}}%
\pgfusepath{clip}%
\pgfsetbuttcap%
\pgfsetroundjoin%
\definecolor{currentfill}{rgb}{0.277941,0.056324,0.381191}%
\pgfsetfillcolor{currentfill}%
\pgfsetfillopacity{0.700000}%
\pgfsetlinewidth{0.000000pt}%
\definecolor{currentstroke}{rgb}{0.000000,0.000000,0.000000}%
\pgfsetstrokecolor{currentstroke}%
\pgfsetdash{}{0pt}%
\pgfpathmoveto{\pgfqpoint{4.236751in}{1.995053in}}%
\pgfpathlineto{\pgfqpoint{4.250580in}{1.994605in}}%
\pgfpathlineto{\pgfqpoint{4.264418in}{1.994260in}}%
\pgfpathlineto{\pgfqpoint{4.278265in}{1.994017in}}%
\pgfpathlineto{\pgfqpoint{4.292120in}{1.993877in}}%
\pgfpathlineto{\pgfqpoint{4.284224in}{1.983514in}}%
\pgfpathlineto{\pgfqpoint{4.276323in}{1.973172in}}%
\pgfpathlineto{\pgfqpoint{4.268417in}{1.962856in}}%
\pgfpathlineto{\pgfqpoint{4.260505in}{1.952567in}}%
\pgfpathlineto{\pgfqpoint{4.246642in}{1.952991in}}%
\pgfpathlineto{\pgfqpoint{4.232787in}{1.953517in}}%
\pgfpathlineto{\pgfqpoint{4.218941in}{1.954145in}}%
\pgfpathlineto{\pgfqpoint{4.205103in}{1.954877in}}%
\pgfpathlineto{\pgfqpoint{4.213023in}{1.964875in}}%
\pgfpathlineto{\pgfqpoint{4.220937in}{1.974906in}}%
\pgfpathlineto{\pgfqpoint{4.228846in}{1.984966in}}%
\pgfpathlineto{\pgfqpoint{4.236751in}{1.995053in}}%
\pgfpathclose%
\pgfusepath{fill}%
\end{pgfscope}%
\begin{pgfscope}%
\pgfpathrectangle{\pgfqpoint{1.150000in}{0.150000in}}{\pgfqpoint{5.700000in}{5.700000in}}%
\pgfusepath{clip}%
\pgfsetbuttcap%
\pgfsetroundjoin%
\definecolor{currentfill}{rgb}{0.281887,0.150881,0.465405}%
\pgfsetfillcolor{currentfill}%
\pgfsetfillopacity{0.700000}%
\pgfsetlinewidth{0.000000pt}%
\definecolor{currentstroke}{rgb}{0.000000,0.000000,0.000000}%
\pgfsetstrokecolor{currentstroke}%
\pgfsetdash{}{0pt}%
\pgfpathmoveto{\pgfqpoint{4.584450in}{2.178532in}}%
\pgfpathlineto{\pgfqpoint{4.598407in}{2.180669in}}%
\pgfpathlineto{\pgfqpoint{4.612374in}{2.182907in}}%
\pgfpathlineto{\pgfqpoint{4.626352in}{2.185246in}}%
\pgfpathlineto{\pgfqpoint{4.640341in}{2.187686in}}%
\pgfpathlineto{\pgfqpoint{4.632549in}{2.176459in}}%
\pgfpathlineto{\pgfqpoint{4.624752in}{2.165202in}}%
\pgfpathlineto{\pgfqpoint{4.616950in}{2.153916in}}%
\pgfpathlineto{\pgfqpoint{4.609142in}{2.142602in}}%
\pgfpathlineto{\pgfqpoint{4.595149in}{2.140374in}}%
\pgfpathlineto{\pgfqpoint{4.581166in}{2.138248in}}%
\pgfpathlineto{\pgfqpoint{4.567194in}{2.136222in}}%
\pgfpathlineto{\pgfqpoint{4.553232in}{2.134297in}}%
\pgfpathlineto{\pgfqpoint{4.561044in}{2.145391in}}%
\pgfpathlineto{\pgfqpoint{4.568851in}{2.156463in}}%
\pgfpathlineto{\pgfqpoint{4.576653in}{2.167511in}}%
\pgfpathlineto{\pgfqpoint{4.584450in}{2.178532in}}%
\pgfpathclose%
\pgfusepath{fill}%
\end{pgfscope}%
\begin{pgfscope}%
\pgfpathrectangle{\pgfqpoint{1.150000in}{0.150000in}}{\pgfqpoint{5.700000in}{5.700000in}}%
\pgfusepath{clip}%
\pgfsetbuttcap%
\pgfsetroundjoin%
\definecolor{currentfill}{rgb}{0.283072,0.130895,0.449241}%
\pgfsetfillcolor{currentfill}%
\pgfsetfillopacity{0.700000}%
\pgfsetlinewidth{0.000000pt}%
\definecolor{currentstroke}{rgb}{0.000000,0.000000,0.000000}%
\pgfsetstrokecolor{currentstroke}%
\pgfsetdash{}{0pt}%
\pgfpathmoveto{\pgfqpoint{3.100324in}{2.135250in}}%
\pgfpathlineto{\pgfqpoint{3.114005in}{2.124670in}}%
\pgfpathlineto{\pgfqpoint{3.127687in}{2.114223in}}%
\pgfpathlineto{\pgfqpoint{3.141369in}{2.103909in}}%
\pgfpathlineto{\pgfqpoint{3.155051in}{2.093726in}}%
\pgfpathlineto{\pgfqpoint{3.146627in}{2.092394in}}%
\pgfpathlineto{\pgfqpoint{3.138190in}{2.091269in}}%
\pgfpathlineto{\pgfqpoint{3.129740in}{2.090354in}}%
\pgfpathlineto{\pgfqpoint{3.121278in}{2.089654in}}%
\pgfpathlineto{\pgfqpoint{3.107564in}{2.100274in}}%
\pgfpathlineto{\pgfqpoint{3.093850in}{2.111026in}}%
\pgfpathlineto{\pgfqpoint{3.080136in}{2.121910in}}%
\pgfpathlineto{\pgfqpoint{3.066422in}{2.132928in}}%
\pgfpathlineto{\pgfqpoint{3.074917in}{2.133183in}}%
\pgfpathlineto{\pgfqpoint{3.083399in}{2.133657in}}%
\pgfpathlineto{\pgfqpoint{3.091868in}{2.134348in}}%
\pgfpathlineto{\pgfqpoint{3.100324in}{2.135250in}}%
\pgfpathclose%
\pgfusepath{fill}%
\end{pgfscope}%
\begin{pgfscope}%
\pgfpathrectangle{\pgfqpoint{1.150000in}{0.150000in}}{\pgfqpoint{5.700000in}{5.700000in}}%
\pgfusepath{clip}%
\pgfsetbuttcap%
\pgfsetroundjoin%
\definecolor{currentfill}{rgb}{0.277941,0.056324,0.381191}%
\pgfsetfillcolor{currentfill}%
\pgfsetfillopacity{0.700000}%
\pgfsetlinewidth{0.000000pt}%
\definecolor{currentstroke}{rgb}{0.000000,0.000000,0.000000}%
\pgfsetstrokecolor{currentstroke}%
\pgfsetdash{}{0pt}%
\pgfpathmoveto{\pgfqpoint{3.352572in}{1.993810in}}%
\pgfpathlineto{\pgfqpoint{3.366246in}{1.985685in}}%
\pgfpathlineto{\pgfqpoint{3.379922in}{1.977681in}}%
\pgfpathlineto{\pgfqpoint{3.393601in}{1.969799in}}%
\pgfpathlineto{\pgfqpoint{3.407283in}{1.962036in}}%
\pgfpathlineto{\pgfqpoint{3.399013in}{1.958292in}}%
\pgfpathlineto{\pgfqpoint{3.390733in}{1.954717in}}%
\pgfpathlineto{\pgfqpoint{3.382444in}{1.951316in}}%
\pgfpathlineto{\pgfqpoint{3.374144in}{1.948093in}}%
\pgfpathlineto{\pgfqpoint{3.360436in}{1.956269in}}%
\pgfpathlineto{\pgfqpoint{3.346732in}{1.964565in}}%
\pgfpathlineto{\pgfqpoint{3.333029in}{1.972982in}}%
\pgfpathlineto{\pgfqpoint{3.319329in}{1.981522in}}%
\pgfpathlineto{\pgfqpoint{3.327655in}{1.984323in}}%
\pgfpathlineto{\pgfqpoint{3.335971in}{1.987308in}}%
\pgfpathlineto{\pgfqpoint{3.344277in}{1.990472in}}%
\pgfpathlineto{\pgfqpoint{3.352572in}{1.993810in}}%
\pgfpathclose%
\pgfusepath{fill}%
\end{pgfscope}%
\begin{pgfscope}%
\pgfpathrectangle{\pgfqpoint{1.150000in}{0.150000in}}{\pgfqpoint{5.700000in}{5.700000in}}%
\pgfusepath{clip}%
\pgfsetbuttcap%
\pgfsetroundjoin%
\definecolor{currentfill}{rgb}{0.278012,0.180367,0.486697}%
\pgfsetfillcolor{currentfill}%
\pgfsetfillopacity{0.700000}%
\pgfsetlinewidth{0.000000pt}%
\definecolor{currentstroke}{rgb}{0.000000,0.000000,0.000000}%
\pgfsetstrokecolor{currentstroke}%
\pgfsetdash{}{0pt}%
\pgfpathmoveto{\pgfqpoint{4.671458in}{2.232245in}}%
\pgfpathlineto{\pgfqpoint{4.685453in}{2.234979in}}%
\pgfpathlineto{\pgfqpoint{4.699460in}{2.237814in}}%
\pgfpathlineto{\pgfqpoint{4.713477in}{2.240750in}}%
\pgfpathlineto{\pgfqpoint{4.727506in}{2.243786in}}%
\pgfpathlineto{\pgfqpoint{4.719739in}{2.232514in}}%
\pgfpathlineto{\pgfqpoint{4.711967in}{2.221200in}}%
\pgfpathlineto{\pgfqpoint{4.704189in}{2.209844in}}%
\pgfpathlineto{\pgfqpoint{4.696406in}{2.198450in}}%
\pgfpathlineto{\pgfqpoint{4.682374in}{2.195608in}}%
\pgfpathlineto{\pgfqpoint{4.668352in}{2.192867in}}%
\pgfpathlineto{\pgfqpoint{4.654341in}{2.190226in}}%
\pgfpathlineto{\pgfqpoint{4.640341in}{2.187686in}}%
\pgfpathlineto{\pgfqpoint{4.648128in}{2.198879in}}%
\pgfpathlineto{\pgfqpoint{4.655910in}{2.210038in}}%
\pgfpathlineto{\pgfqpoint{4.663687in}{2.221160in}}%
\pgfpathlineto{\pgfqpoint{4.671458in}{2.232245in}}%
\pgfpathclose%
\pgfusepath{fill}%
\end{pgfscope}%
\begin{pgfscope}%
\pgfpathrectangle{\pgfqpoint{1.150000in}{0.150000in}}{\pgfqpoint{5.700000in}{5.700000in}}%
\pgfusepath{clip}%
\pgfsetbuttcap%
\pgfsetroundjoin%
\definecolor{currentfill}{rgb}{0.274952,0.037752,0.364543}%
\pgfsetfillcolor{currentfill}%
\pgfsetfillopacity{0.700000}%
\pgfsetlinewidth{0.000000pt}%
\definecolor{currentstroke}{rgb}{0.000000,0.000000,0.000000}%
\pgfsetstrokecolor{currentstroke}%
\pgfsetdash{}{0pt}%
\pgfpathmoveto{\pgfqpoint{4.149833in}{1.958837in}}%
\pgfpathlineto{\pgfqpoint{4.163638in}{1.957691in}}%
\pgfpathlineto{\pgfqpoint{4.177452in}{1.956650in}}%
\pgfpathlineto{\pgfqpoint{4.191273in}{1.955712in}}%
\pgfpathlineto{\pgfqpoint{4.205103in}{1.954877in}}%
\pgfpathlineto{\pgfqpoint{4.197178in}{1.944913in}}%
\pgfpathlineto{\pgfqpoint{4.189248in}{1.934988in}}%
\pgfpathlineto{\pgfqpoint{4.181312in}{1.925102in}}%
\pgfpathlineto{\pgfqpoint{4.173372in}{1.915260in}}%
\pgfpathlineto{\pgfqpoint{4.159533in}{1.916396in}}%
\pgfpathlineto{\pgfqpoint{4.145702in}{1.917635in}}%
\pgfpathlineto{\pgfqpoint{4.131879in}{1.918978in}}%
\pgfpathlineto{\pgfqpoint{4.118064in}{1.920425in}}%
\pgfpathlineto{\pgfqpoint{4.126014in}{1.929959in}}%
\pgfpathlineto{\pgfqpoint{4.133959in}{1.939541in}}%
\pgfpathlineto{\pgfqpoint{4.141899in}{1.949168in}}%
\pgfpathlineto{\pgfqpoint{4.149833in}{1.958837in}}%
\pgfpathclose%
\pgfusepath{fill}%
\end{pgfscope}%
\begin{pgfscope}%
\pgfpathrectangle{\pgfqpoint{1.150000in}{0.150000in}}{\pgfqpoint{5.700000in}{5.700000in}}%
\pgfusepath{clip}%
\pgfsetbuttcap%
\pgfsetroundjoin%
\definecolor{currentfill}{rgb}{0.172719,0.448791,0.557885}%
\pgfsetfillcolor{currentfill}%
\pgfsetfillopacity{0.700000}%
\pgfsetlinewidth{0.000000pt}%
\definecolor{currentstroke}{rgb}{0.000000,0.000000,0.000000}%
\pgfsetstrokecolor{currentstroke}%
\pgfsetdash{}{0pt}%
\pgfpathmoveto{\pgfqpoint{5.517227in}{2.860031in}}%
\pgfpathlineto{\pgfqpoint{5.531637in}{2.867280in}}%
\pgfpathlineto{\pgfqpoint{5.546062in}{2.874630in}}%
\pgfpathlineto{\pgfqpoint{5.560502in}{2.882079in}}%
\pgfpathlineto{\pgfqpoint{5.574958in}{2.889629in}}%
\pgfpathlineto{\pgfqpoint{5.567531in}{2.881321in}}%
\pgfpathlineto{\pgfqpoint{5.560096in}{2.872903in}}%
\pgfpathlineto{\pgfqpoint{5.552653in}{2.864373in}}%
\pgfpathlineto{\pgfqpoint{5.545201in}{2.855731in}}%
\pgfpathlineto{\pgfqpoint{5.530737in}{2.848168in}}%
\pgfpathlineto{\pgfqpoint{5.516289in}{2.840706in}}%
\pgfpathlineto{\pgfqpoint{5.501856in}{2.833344in}}%
\pgfpathlineto{\pgfqpoint{5.487438in}{2.826082in}}%
\pgfpathlineto{\pgfqpoint{5.494897in}{2.834730in}}%
\pgfpathlineto{\pgfqpoint{5.502348in}{2.843270in}}%
\pgfpathlineto{\pgfqpoint{5.509792in}{2.851704in}}%
\pgfpathlineto{\pgfqpoint{5.517227in}{2.860031in}}%
\pgfpathclose%
\pgfusepath{fill}%
\end{pgfscope}%
\begin{pgfscope}%
\pgfpathrectangle{\pgfqpoint{1.150000in}{0.150000in}}{\pgfqpoint{5.700000in}{5.700000in}}%
\pgfusepath{clip}%
\pgfsetbuttcap%
\pgfsetroundjoin%
\definecolor{currentfill}{rgb}{0.267004,0.004874,0.329415}%
\pgfsetfillcolor{currentfill}%
\pgfsetfillopacity{0.700000}%
\pgfsetlinewidth{0.000000pt}%
\definecolor{currentstroke}{rgb}{0.000000,0.000000,0.000000}%
\pgfsetstrokecolor{currentstroke}%
\pgfsetdash{}{0pt}%
\pgfpathmoveto{\pgfqpoint{3.691776in}{1.901640in}}%
\pgfpathlineto{\pgfqpoint{3.705479in}{1.896609in}}%
\pgfpathlineto{\pgfqpoint{3.719188in}{1.891688in}}%
\pgfpathlineto{\pgfqpoint{3.732902in}{1.886879in}}%
\pgfpathlineto{\pgfqpoint{3.746621in}{1.882180in}}%
\pgfpathlineto{\pgfqpoint{3.738523in}{1.875382in}}%
\pgfpathlineto{\pgfqpoint{3.730417in}{1.868699in}}%
\pgfpathlineto{\pgfqpoint{3.722304in}{1.862136in}}%
\pgfpathlineto{\pgfqpoint{3.714184in}{1.855696in}}%
\pgfpathlineto{\pgfqpoint{3.700447in}{1.860769in}}%
\pgfpathlineto{\pgfqpoint{3.686715in}{1.865951in}}%
\pgfpathlineto{\pgfqpoint{3.672988in}{1.871245in}}%
\pgfpathlineto{\pgfqpoint{3.659266in}{1.876650in}}%
\pgfpathlineto{\pgfqpoint{3.667405in}{1.882710in}}%
\pgfpathlineto{\pgfqpoint{3.675536in}{1.888897in}}%
\pgfpathlineto{\pgfqpoint{3.683660in}{1.895208in}}%
\pgfpathlineto{\pgfqpoint{3.691776in}{1.901640in}}%
\pgfpathclose%
\pgfusepath{fill}%
\end{pgfscope}%
\begin{pgfscope}%
\pgfpathrectangle{\pgfqpoint{1.150000in}{0.150000in}}{\pgfqpoint{5.700000in}{5.700000in}}%
\pgfusepath{clip}%
\pgfsetbuttcap%
\pgfsetroundjoin%
\definecolor{currentfill}{rgb}{0.223925,0.334994,0.548053}%
\pgfsetfillcolor{currentfill}%
\pgfsetfillopacity{0.700000}%
\pgfsetlinewidth{0.000000pt}%
\definecolor{currentstroke}{rgb}{0.000000,0.000000,0.000000}%
\pgfsetstrokecolor{currentstroke}%
\pgfsetdash{}{0pt}%
\pgfpathmoveto{\pgfqpoint{5.137920in}{2.572706in}}%
\pgfpathlineto{\pgfqpoint{5.152134in}{2.578224in}}%
\pgfpathlineto{\pgfqpoint{5.166362in}{2.583843in}}%
\pgfpathlineto{\pgfqpoint{5.180604in}{2.589561in}}%
\pgfpathlineto{\pgfqpoint{5.194859in}{2.595380in}}%
\pgfpathlineto{\pgfqpoint{5.187255in}{2.585125in}}%
\pgfpathlineto{\pgfqpoint{5.179644in}{2.574779in}}%
\pgfpathlineto{\pgfqpoint{5.172027in}{2.564341in}}%
\pgfpathlineto{\pgfqpoint{5.164403in}{2.553812in}}%
\pgfpathlineto{\pgfqpoint{5.150143in}{2.548077in}}%
\pgfpathlineto{\pgfqpoint{5.135897in}{2.542442in}}%
\pgfpathlineto{\pgfqpoint{5.121665in}{2.536907in}}%
\pgfpathlineto{\pgfqpoint{5.107446in}{2.531472in}}%
\pgfpathlineto{\pgfqpoint{5.115074in}{2.541910in}}%
\pgfpathlineto{\pgfqpoint{5.122696in}{2.552262in}}%
\pgfpathlineto{\pgfqpoint{5.130311in}{2.562527in}}%
\pgfpathlineto{\pgfqpoint{5.137920in}{2.572706in}}%
\pgfpathclose%
\pgfusepath{fill}%
\end{pgfscope}%
\begin{pgfscope}%
\pgfpathrectangle{\pgfqpoint{1.150000in}{0.150000in}}{\pgfqpoint{5.700000in}{5.700000in}}%
\pgfusepath{clip}%
\pgfsetbuttcap%
\pgfsetroundjoin%
\definecolor{currentfill}{rgb}{0.269944,0.014625,0.341379}%
\pgfsetfillcolor{currentfill}%
\pgfsetfillopacity{0.700000}%
\pgfsetlinewidth{0.000000pt}%
\definecolor{currentstroke}{rgb}{0.000000,0.000000,0.000000}%
\pgfsetstrokecolor{currentstroke}%
\pgfsetdash{}{0pt}%
\pgfpathmoveto{\pgfqpoint{3.549645in}{1.923948in}}%
\pgfpathlineto{\pgfqpoint{3.563333in}{1.917637in}}%
\pgfpathlineto{\pgfqpoint{3.577025in}{1.911441in}}%
\pgfpathlineto{\pgfqpoint{3.590721in}{1.905359in}}%
\pgfpathlineto{\pgfqpoint{3.604422in}{1.899391in}}%
\pgfpathlineto{\pgfqpoint{3.596255in}{1.893849in}}%
\pgfpathlineto{\pgfqpoint{3.588080in}{1.888446in}}%
\pgfpathlineto{\pgfqpoint{3.579897in}{1.883186in}}%
\pgfpathlineto{\pgfqpoint{3.571705in}{1.878075in}}%
\pgfpathlineto{\pgfqpoint{3.557984in}{1.884435in}}%
\pgfpathlineto{\pgfqpoint{3.544267in}{1.890910in}}%
\pgfpathlineto{\pgfqpoint{3.530553in}{1.897499in}}%
\pgfpathlineto{\pgfqpoint{3.516843in}{1.904203in}}%
\pgfpathlineto{\pgfqpoint{3.525057in}{1.908915in}}%
\pgfpathlineto{\pgfqpoint{3.533261in}{1.913779in}}%
\pgfpathlineto{\pgfqpoint{3.541457in}{1.918791in}}%
\pgfpathlineto{\pgfqpoint{3.549645in}{1.923948in}}%
\pgfpathclose%
\pgfusepath{fill}%
\end{pgfscope}%
\begin{pgfscope}%
\pgfpathrectangle{\pgfqpoint{1.150000in}{0.150000in}}{\pgfqpoint{5.700000in}{5.700000in}}%
\pgfusepath{clip}%
\pgfsetbuttcap%
\pgfsetroundjoin%
\definecolor{currentfill}{rgb}{0.271828,0.209303,0.504434}%
\pgfsetfillcolor{currentfill}%
\pgfsetfillopacity{0.700000}%
\pgfsetlinewidth{0.000000pt}%
\definecolor{currentstroke}{rgb}{0.000000,0.000000,0.000000}%
\pgfsetstrokecolor{currentstroke}%
\pgfsetdash{}{0pt}%
\pgfpathmoveto{\pgfqpoint{4.758521in}{2.288415in}}%
\pgfpathlineto{\pgfqpoint{4.772557in}{2.291727in}}%
\pgfpathlineto{\pgfqpoint{4.786605in}{2.295140in}}%
\pgfpathlineto{\pgfqpoint{4.800664in}{2.298654in}}%
\pgfpathlineto{\pgfqpoint{4.814736in}{2.302267in}}%
\pgfpathlineto{\pgfqpoint{4.806994in}{2.291012in}}%
\pgfpathlineto{\pgfqpoint{4.799247in}{2.279704in}}%
\pgfpathlineto{\pgfqpoint{4.791494in}{2.268344in}}%
\pgfpathlineto{\pgfqpoint{4.783736in}{2.256933in}}%
\pgfpathlineto{\pgfqpoint{4.769661in}{2.253496in}}%
\pgfpathlineto{\pgfqpoint{4.755598in}{2.250159in}}%
\pgfpathlineto{\pgfqpoint{4.741546in}{2.246922in}}%
\pgfpathlineto{\pgfqpoint{4.727506in}{2.243786in}}%
\pgfpathlineto{\pgfqpoint{4.735268in}{2.255013in}}%
\pgfpathlineto{\pgfqpoint{4.743024in}{2.266195in}}%
\pgfpathlineto{\pgfqpoint{4.750776in}{2.277329in}}%
\pgfpathlineto{\pgfqpoint{4.758521in}{2.288415in}}%
\pgfpathclose%
\pgfusepath{fill}%
\end{pgfscope}%
\begin{pgfscope}%
\pgfpathrectangle{\pgfqpoint{1.150000in}{0.150000in}}{\pgfqpoint{5.700000in}{5.700000in}}%
\pgfusepath{clip}%
\pgfsetbuttcap%
\pgfsetroundjoin%
\definecolor{currentfill}{rgb}{0.267004,0.004874,0.329415}%
\pgfsetfillcolor{currentfill}%
\pgfsetfillopacity{0.700000}%
\pgfsetlinewidth{0.000000pt}%
\definecolor{currentstroke}{rgb}{0.000000,0.000000,0.000000}%
\pgfsetstrokecolor{currentstroke}%
\pgfsetdash{}{0pt}%
\pgfpathmoveto{\pgfqpoint{3.833809in}{1.894182in}}%
\pgfpathlineto{\pgfqpoint{3.847539in}{1.890385in}}%
\pgfpathlineto{\pgfqpoint{3.861276in}{1.886695in}}%
\pgfpathlineto{\pgfqpoint{3.875018in}{1.883114in}}%
\pgfpathlineto{\pgfqpoint{3.888767in}{1.879640in}}%
\pgfpathlineto{\pgfqpoint{3.880727in}{1.871718in}}%
\pgfpathlineto{\pgfqpoint{3.872681in}{1.863888in}}%
\pgfpathlineto{\pgfqpoint{3.864629in}{1.856154in}}%
\pgfpathlineto{\pgfqpoint{3.856569in}{1.848520in}}%
\pgfpathlineto{\pgfqpoint{3.842806in}{1.852349in}}%
\pgfpathlineto{\pgfqpoint{3.829048in}{1.856285in}}%
\pgfpathlineto{\pgfqpoint{3.815297in}{1.860329in}}%
\pgfpathlineto{\pgfqpoint{3.801550in}{1.864482in}}%
\pgfpathlineto{\pgfqpoint{3.809625in}{1.871754in}}%
\pgfpathlineto{\pgfqpoint{3.817693in}{1.879130in}}%
\pgfpathlineto{\pgfqpoint{3.825754in}{1.886608in}}%
\pgfpathlineto{\pgfqpoint{3.833809in}{1.894182in}}%
\pgfpathclose%
\pgfusepath{fill}%
\end{pgfscope}%
\begin{pgfscope}%
\pgfpathrectangle{\pgfqpoint{1.150000in}{0.150000in}}{\pgfqpoint{5.700000in}{5.700000in}}%
\pgfusepath{clip}%
\pgfsetbuttcap%
\pgfsetroundjoin%
\definecolor{currentfill}{rgb}{0.283091,0.110553,0.431554}%
\pgfsetfillcolor{currentfill}%
\pgfsetfillopacity{0.700000}%
\pgfsetlinewidth{0.000000pt}%
\definecolor{currentstroke}{rgb}{0.000000,0.000000,0.000000}%
\pgfsetstrokecolor{currentstroke}%
\pgfsetdash{}{0pt}%
\pgfpathmoveto{\pgfqpoint{3.155051in}{2.093726in}}%
\pgfpathlineto{\pgfqpoint{3.168735in}{2.083674in}}%
\pgfpathlineto{\pgfqpoint{3.182419in}{2.073752in}}%
\pgfpathlineto{\pgfqpoint{3.196104in}{2.063959in}}%
\pgfpathlineto{\pgfqpoint{3.209790in}{2.054295in}}%
\pgfpathlineto{\pgfqpoint{3.201395in}{2.052534in}}%
\pgfpathlineto{\pgfqpoint{3.192989in}{2.050975in}}%
\pgfpathlineto{\pgfqpoint{3.184571in}{2.049621in}}%
\pgfpathlineto{\pgfqpoint{3.176141in}{2.048478in}}%
\pgfpathlineto{\pgfqpoint{3.162424in}{2.058578in}}%
\pgfpathlineto{\pgfqpoint{3.148708in}{2.068807in}}%
\pgfpathlineto{\pgfqpoint{3.134993in}{2.079165in}}%
\pgfpathlineto{\pgfqpoint{3.121278in}{2.089654in}}%
\pgfpathlineto{\pgfqpoint{3.129740in}{2.090354in}}%
\pgfpathlineto{\pgfqpoint{3.138190in}{2.091269in}}%
\pgfpathlineto{\pgfqpoint{3.146627in}{2.092394in}}%
\pgfpathlineto{\pgfqpoint{3.155051in}{2.093726in}}%
\pgfpathclose%
\pgfusepath{fill}%
\end{pgfscope}%
\begin{pgfscope}%
\pgfpathrectangle{\pgfqpoint{1.150000in}{0.150000in}}{\pgfqpoint{5.700000in}{5.700000in}}%
\pgfusepath{clip}%
\pgfsetbuttcap%
\pgfsetroundjoin%
\definecolor{currentfill}{rgb}{0.271305,0.019942,0.347269}%
\pgfsetfillcolor{currentfill}%
\pgfsetfillopacity{0.700000}%
\pgfsetlinewidth{0.000000pt}%
\definecolor{currentstroke}{rgb}{0.000000,0.000000,0.000000}%
\pgfsetstrokecolor{currentstroke}%
\pgfsetdash{}{0pt}%
\pgfpathmoveto{\pgfqpoint{4.062880in}{1.927253in}}%
\pgfpathlineto{\pgfqpoint{4.076665in}{1.925389in}}%
\pgfpathlineto{\pgfqpoint{4.090457in}{1.923630in}}%
\pgfpathlineto{\pgfqpoint{4.104257in}{1.921975in}}%
\pgfpathlineto{\pgfqpoint{4.118064in}{1.920425in}}%
\pgfpathlineto{\pgfqpoint{4.110108in}{1.910941in}}%
\pgfpathlineto{\pgfqpoint{4.102147in}{1.901512in}}%
\pgfpathlineto{\pgfqpoint{4.094180in}{1.892139in}}%
\pgfpathlineto{\pgfqpoint{4.086207in}{1.882825in}}%
\pgfpathlineto{\pgfqpoint{4.072389in}{1.884695in}}%
\pgfpathlineto{\pgfqpoint{4.058579in}{1.886668in}}%
\pgfpathlineto{\pgfqpoint{4.044776in}{1.888747in}}%
\pgfpathlineto{\pgfqpoint{4.030980in}{1.890929in}}%
\pgfpathlineto{\pgfqpoint{4.038963in}{1.899917in}}%
\pgfpathlineto{\pgfqpoint{4.046941in}{1.908968in}}%
\pgfpathlineto{\pgfqpoint{4.054913in}{1.918081in}}%
\pgfpathlineto{\pgfqpoint{4.062880in}{1.927253in}}%
\pgfpathclose%
\pgfusepath{fill}%
\end{pgfscope}%
\begin{pgfscope}%
\pgfpathrectangle{\pgfqpoint{1.150000in}{0.150000in}}{\pgfqpoint{5.700000in}{5.700000in}}%
\pgfusepath{clip}%
\pgfsetbuttcap%
\pgfsetroundjoin%
\definecolor{currentfill}{rgb}{0.162142,0.474838,0.558140}%
\pgfsetfillcolor{currentfill}%
\pgfsetfillopacity{0.700000}%
\pgfsetlinewidth{0.000000pt}%
\definecolor{currentstroke}{rgb}{0.000000,0.000000,0.000000}%
\pgfsetstrokecolor{currentstroke}%
\pgfsetdash{}{0pt}%
\pgfpathmoveto{\pgfqpoint{5.604581in}{2.921759in}}%
\pgfpathlineto{\pgfqpoint{5.619043in}{2.929376in}}%
\pgfpathlineto{\pgfqpoint{5.633521in}{2.937093in}}%
\pgfpathlineto{\pgfqpoint{5.648014in}{2.944911in}}%
\pgfpathlineto{\pgfqpoint{5.662522in}{2.952829in}}%
\pgfpathlineto{\pgfqpoint{5.655139in}{2.945000in}}%
\pgfpathlineto{\pgfqpoint{5.647746in}{2.937057in}}%
\pgfpathlineto{\pgfqpoint{5.640345in}{2.929001in}}%
\pgfpathlineto{\pgfqpoint{5.632935in}{2.920829in}}%
\pgfpathlineto{\pgfqpoint{5.618418in}{2.912879in}}%
\pgfpathlineto{\pgfqpoint{5.603915in}{2.905028in}}%
\pgfpathlineto{\pgfqpoint{5.589429in}{2.897278in}}%
\pgfpathlineto{\pgfqpoint{5.574958in}{2.889629in}}%
\pgfpathlineto{\pgfqpoint{5.582376in}{2.897825in}}%
\pgfpathlineto{\pgfqpoint{5.589786in}{2.905912in}}%
\pgfpathlineto{\pgfqpoint{5.597188in}{2.913890in}}%
\pgfpathlineto{\pgfqpoint{5.604581in}{2.921759in}}%
\pgfpathclose%
\pgfusepath{fill}%
\end{pgfscope}%
\begin{pgfscope}%
\pgfpathrectangle{\pgfqpoint{1.150000in}{0.150000in}}{\pgfqpoint{5.700000in}{5.700000in}}%
\pgfusepath{clip}%
\pgfsetbuttcap%
\pgfsetroundjoin%
\definecolor{currentfill}{rgb}{0.227802,0.326594,0.546532}%
\pgfsetfillcolor{currentfill}%
\pgfsetfillopacity{0.700000}%
\pgfsetlinewidth{0.000000pt}%
\definecolor{currentstroke}{rgb}{0.000000,0.000000,0.000000}%
\pgfsetstrokecolor{currentstroke}%
\pgfsetdash{}{0pt}%
\pgfpathmoveto{\pgfqpoint{2.626815in}{2.561675in}}%
\pgfpathlineto{\pgfqpoint{2.640599in}{2.545860in}}%
\pgfpathlineto{\pgfqpoint{2.654378in}{2.530212in}}%
\pgfpathlineto{\pgfqpoint{2.668152in}{2.514732in}}%
\pgfpathlineto{\pgfqpoint{2.681923in}{2.499417in}}%
\pgfpathlineto{\pgfqpoint{2.673145in}{2.502589in}}%
\pgfpathlineto{\pgfqpoint{2.664351in}{2.506027in}}%
\pgfpathlineto{\pgfqpoint{2.655539in}{2.509736in}}%
\pgfpathlineto{\pgfqpoint{2.646709in}{2.513721in}}%
\pgfpathlineto{\pgfqpoint{2.632894in}{2.529510in}}%
\pgfpathlineto{\pgfqpoint{2.619075in}{2.545465in}}%
\pgfpathlineto{\pgfqpoint{2.605251in}{2.561587in}}%
\pgfpathlineto{\pgfqpoint{2.591422in}{2.577879in}}%
\pgfpathlineto{\pgfqpoint{2.600298in}{2.573411in}}%
\pgfpathlineto{\pgfqpoint{2.609155in}{2.569224in}}%
\pgfpathlineto{\pgfqpoint{2.617994in}{2.565314in}}%
\pgfpathlineto{\pgfqpoint{2.626815in}{2.561675in}}%
\pgfpathclose%
\pgfusepath{fill}%
\end{pgfscope}%
\begin{pgfscope}%
\pgfpathrectangle{\pgfqpoint{1.150000in}{0.150000in}}{\pgfqpoint{5.700000in}{5.700000in}}%
\pgfusepath{clip}%
\pgfsetbuttcap%
\pgfsetroundjoin%
\definecolor{currentfill}{rgb}{0.216210,0.351535,0.550627}%
\pgfsetfillcolor{currentfill}%
\pgfsetfillopacity{0.700000}%
\pgfsetlinewidth{0.000000pt}%
\definecolor{currentstroke}{rgb}{0.000000,0.000000,0.000000}%
\pgfsetstrokecolor{currentstroke}%
\pgfsetdash{}{0pt}%
\pgfpathmoveto{\pgfqpoint{2.571633in}{2.626652in}}%
\pgfpathlineto{\pgfqpoint{2.585436in}{2.610148in}}%
\pgfpathlineto{\pgfqpoint{2.599234in}{2.593818in}}%
\pgfpathlineto{\pgfqpoint{2.613027in}{2.577661in}}%
\pgfpathlineto{\pgfqpoint{2.626815in}{2.561675in}}%
\pgfpathlineto{\pgfqpoint{2.617994in}{2.565314in}}%
\pgfpathlineto{\pgfqpoint{2.609155in}{2.569224in}}%
\pgfpathlineto{\pgfqpoint{2.600298in}{2.573411in}}%
\pgfpathlineto{\pgfqpoint{2.591422in}{2.577879in}}%
\pgfpathlineto{\pgfqpoint{2.577589in}{2.594341in}}%
\pgfpathlineto{\pgfqpoint{2.563750in}{2.610976in}}%
\pgfpathlineto{\pgfqpoint{2.549905in}{2.627783in}}%
\pgfpathlineto{\pgfqpoint{2.536056in}{2.644766in}}%
\pgfpathlineto{\pgfqpoint{2.544978in}{2.639812in}}%
\pgfpathlineto{\pgfqpoint{2.553882in}{2.635145in}}%
\pgfpathlineto{\pgfqpoint{2.562766in}{2.630760in}}%
\pgfpathlineto{\pgfqpoint{2.571633in}{2.626652in}}%
\pgfpathclose%
\pgfusepath{fill}%
\end{pgfscope}%
\begin{pgfscope}%
\pgfpathrectangle{\pgfqpoint{1.150000in}{0.150000in}}{\pgfqpoint{5.700000in}{5.700000in}}%
\pgfusepath{clip}%
\pgfsetbuttcap%
\pgfsetroundjoin%
\definecolor{currentfill}{rgb}{0.239346,0.300855,0.540844}%
\pgfsetfillcolor{currentfill}%
\pgfsetfillopacity{0.700000}%
\pgfsetlinewidth{0.000000pt}%
\definecolor{currentstroke}{rgb}{0.000000,0.000000,0.000000}%
\pgfsetstrokecolor{currentstroke}%
\pgfsetdash{}{0pt}%
\pgfpathmoveto{\pgfqpoint{2.681923in}{2.499417in}}%
\pgfpathlineto{\pgfqpoint{2.695689in}{2.484267in}}%
\pgfpathlineto{\pgfqpoint{2.709452in}{2.469279in}}%
\pgfpathlineto{\pgfqpoint{2.723211in}{2.454453in}}%
\pgfpathlineto{\pgfqpoint{2.736967in}{2.439788in}}%
\pgfpathlineto{\pgfqpoint{2.728232in}{2.442496in}}%
\pgfpathlineto{\pgfqpoint{2.719480in}{2.445465in}}%
\pgfpathlineto{\pgfqpoint{2.710711in}{2.448699in}}%
\pgfpathlineto{\pgfqpoint{2.701926in}{2.452203in}}%
\pgfpathlineto{\pgfqpoint{2.688127in}{2.467340in}}%
\pgfpathlineto{\pgfqpoint{2.674325in}{2.482638in}}%
\pgfpathlineto{\pgfqpoint{2.660519in}{2.498098in}}%
\pgfpathlineto{\pgfqpoint{2.646709in}{2.513721in}}%
\pgfpathlineto{\pgfqpoint{2.655539in}{2.509736in}}%
\pgfpathlineto{\pgfqpoint{2.664351in}{2.506027in}}%
\pgfpathlineto{\pgfqpoint{2.673145in}{2.502589in}}%
\pgfpathlineto{\pgfqpoint{2.681923in}{2.499417in}}%
\pgfpathclose%
\pgfusepath{fill}%
\end{pgfscope}%
\begin{pgfscope}%
\pgfpathrectangle{\pgfqpoint{1.150000in}{0.150000in}}{\pgfqpoint{5.700000in}{5.700000in}}%
\pgfusepath{clip}%
\pgfsetbuttcap%
\pgfsetroundjoin%
\definecolor{currentfill}{rgb}{0.210503,0.363727,0.552206}%
\pgfsetfillcolor{currentfill}%
\pgfsetfillopacity{0.700000}%
\pgfsetlinewidth{0.000000pt}%
\definecolor{currentstroke}{rgb}{0.000000,0.000000,0.000000}%
\pgfsetstrokecolor{currentstroke}%
\pgfsetdash{}{0pt}%
\pgfpathmoveto{\pgfqpoint{5.225205in}{2.635473in}}%
\pgfpathlineto{\pgfqpoint{5.239469in}{2.641456in}}%
\pgfpathlineto{\pgfqpoint{5.253747in}{2.647539in}}%
\pgfpathlineto{\pgfqpoint{5.268038in}{2.653722in}}%
\pgfpathlineto{\pgfqpoint{5.282344in}{2.660005in}}%
\pgfpathlineto{\pgfqpoint{5.274773in}{2.650064in}}%
\pgfpathlineto{\pgfqpoint{5.267195in}{2.640025in}}%
\pgfpathlineto{\pgfqpoint{5.259610in}{2.629889in}}%
\pgfpathlineto{\pgfqpoint{5.252018in}{2.619654in}}%
\pgfpathlineto{\pgfqpoint{5.237707in}{2.613436in}}%
\pgfpathlineto{\pgfqpoint{5.223410in}{2.607317in}}%
\pgfpathlineto{\pgfqpoint{5.209128in}{2.601298in}}%
\pgfpathlineto{\pgfqpoint{5.194859in}{2.595380in}}%
\pgfpathlineto{\pgfqpoint{5.202456in}{2.605542in}}%
\pgfpathlineto{\pgfqpoint{5.210046in}{2.615612in}}%
\pgfpathlineto{\pgfqpoint{5.217629in}{2.625589in}}%
\pgfpathlineto{\pgfqpoint{5.225205in}{2.635473in}}%
\pgfpathclose%
\pgfusepath{fill}%
\end{pgfscope}%
\begin{pgfscope}%
\pgfpathrectangle{\pgfqpoint{1.150000in}{0.150000in}}{\pgfqpoint{5.700000in}{5.700000in}}%
\pgfusepath{clip}%
\pgfsetbuttcap%
\pgfsetroundjoin%
\definecolor{currentfill}{rgb}{0.263663,0.237631,0.518762}%
\pgfsetfillcolor{currentfill}%
\pgfsetfillopacity{0.700000}%
\pgfsetlinewidth{0.000000pt}%
\definecolor{currentstroke}{rgb}{0.000000,0.000000,0.000000}%
\pgfsetstrokecolor{currentstroke}%
\pgfsetdash{}{0pt}%
\pgfpathmoveto{\pgfqpoint{4.845647in}{2.346724in}}%
\pgfpathlineto{\pgfqpoint{4.859726in}{2.350596in}}%
\pgfpathlineto{\pgfqpoint{4.873817in}{2.354568in}}%
\pgfpathlineto{\pgfqpoint{4.887921in}{2.358639in}}%
\pgfpathlineto{\pgfqpoint{4.902036in}{2.362812in}}%
\pgfpathlineto{\pgfqpoint{4.894321in}{2.351633in}}%
\pgfpathlineto{\pgfqpoint{4.886599in}{2.340391in}}%
\pgfpathlineto{\pgfqpoint{4.878872in}{2.329087in}}%
\pgfpathlineto{\pgfqpoint{4.871140in}{2.317722in}}%
\pgfpathlineto{\pgfqpoint{4.857021in}{2.313709in}}%
\pgfpathlineto{\pgfqpoint{4.842914in}{2.309795in}}%
\pgfpathlineto{\pgfqpoint{4.828819in}{2.305981in}}%
\pgfpathlineto{\pgfqpoint{4.814736in}{2.302267in}}%
\pgfpathlineto{\pgfqpoint{4.822472in}{2.313467in}}%
\pgfpathlineto{\pgfqpoint{4.830202in}{2.324611in}}%
\pgfpathlineto{\pgfqpoint{4.837928in}{2.335697in}}%
\pgfpathlineto{\pgfqpoint{4.845647in}{2.346724in}}%
\pgfpathclose%
\pgfusepath{fill}%
\end{pgfscope}%
\begin{pgfscope}%
\pgfpathrectangle{\pgfqpoint{1.150000in}{0.150000in}}{\pgfqpoint{5.700000in}{5.700000in}}%
\pgfusepath{clip}%
\pgfsetbuttcap%
\pgfsetroundjoin%
\definecolor{currentfill}{rgb}{0.203063,0.379716,0.553925}%
\pgfsetfillcolor{currentfill}%
\pgfsetfillopacity{0.700000}%
\pgfsetlinewidth{0.000000pt}%
\definecolor{currentstroke}{rgb}{0.000000,0.000000,0.000000}%
\pgfsetstrokecolor{currentstroke}%
\pgfsetdash{}{0pt}%
\pgfpathmoveto{\pgfqpoint{2.516365in}{2.694441in}}%
\pgfpathlineto{\pgfqpoint{2.530191in}{2.677225in}}%
\pgfpathlineto{\pgfqpoint{2.544010in}{2.660189in}}%
\pgfpathlineto{\pgfqpoint{2.557824in}{2.643331in}}%
\pgfpathlineto{\pgfqpoint{2.571633in}{2.626652in}}%
\pgfpathlineto{\pgfqpoint{2.562766in}{2.630760in}}%
\pgfpathlineto{\pgfqpoint{2.553882in}{2.635145in}}%
\pgfpathlineto{\pgfqpoint{2.544978in}{2.639812in}}%
\pgfpathlineto{\pgfqpoint{2.536056in}{2.644766in}}%
\pgfpathlineto{\pgfqpoint{2.522200in}{2.661926in}}%
\pgfpathlineto{\pgfqpoint{2.508339in}{2.679264in}}%
\pgfpathlineto{\pgfqpoint{2.494471in}{2.696781in}}%
\pgfpathlineto{\pgfqpoint{2.480598in}{2.714480in}}%
\pgfpathlineto{\pgfqpoint{2.489569in}{2.709036in}}%
\pgfpathlineto{\pgfqpoint{2.498520in}{2.703886in}}%
\pgfpathlineto{\pgfqpoint{2.507452in}{2.699022in}}%
\pgfpathlineto{\pgfqpoint{2.516365in}{2.694441in}}%
\pgfpathclose%
\pgfusepath{fill}%
\end{pgfscope}%
\begin{pgfscope}%
\pgfpathrectangle{\pgfqpoint{1.150000in}{0.150000in}}{\pgfqpoint{5.700000in}{5.700000in}}%
\pgfusepath{clip}%
\pgfsetbuttcap%
\pgfsetroundjoin%
\definecolor{currentfill}{rgb}{0.250425,0.274290,0.533103}%
\pgfsetfillcolor{currentfill}%
\pgfsetfillopacity{0.700000}%
\pgfsetlinewidth{0.000000pt}%
\definecolor{currentstroke}{rgb}{0.000000,0.000000,0.000000}%
\pgfsetstrokecolor{currentstroke}%
\pgfsetdash{}{0pt}%
\pgfpathmoveto{\pgfqpoint{2.736967in}{2.439788in}}%
\pgfpathlineto{\pgfqpoint{2.750719in}{2.425282in}}%
\pgfpathlineto{\pgfqpoint{2.764468in}{2.410933in}}%
\pgfpathlineto{\pgfqpoint{2.778214in}{2.396741in}}%
\pgfpathlineto{\pgfqpoint{2.791957in}{2.382705in}}%
\pgfpathlineto{\pgfqpoint{2.783263in}{2.384952in}}%
\pgfpathlineto{\pgfqpoint{2.774553in}{2.387453in}}%
\pgfpathlineto{\pgfqpoint{2.765826in}{2.390215in}}%
\pgfpathlineto{\pgfqpoint{2.757083in}{2.393242in}}%
\pgfpathlineto{\pgfqpoint{2.743299in}{2.407747in}}%
\pgfpathlineto{\pgfqpoint{2.729511in}{2.422408in}}%
\pgfpathlineto{\pgfqpoint{2.715720in}{2.437227in}}%
\pgfpathlineto{\pgfqpoint{2.701926in}{2.452203in}}%
\pgfpathlineto{\pgfqpoint{2.710711in}{2.448699in}}%
\pgfpathlineto{\pgfqpoint{2.719480in}{2.445465in}}%
\pgfpathlineto{\pgfqpoint{2.728232in}{2.442496in}}%
\pgfpathlineto{\pgfqpoint{2.736967in}{2.439788in}}%
\pgfpathclose%
\pgfusepath{fill}%
\end{pgfscope}%
\begin{pgfscope}%
\pgfpathrectangle{\pgfqpoint{1.150000in}{0.150000in}}{\pgfqpoint{5.700000in}{5.700000in}}%
\pgfusepath{clip}%
\pgfsetbuttcap%
\pgfsetroundjoin%
\definecolor{currentfill}{rgb}{0.276022,0.044167,0.370164}%
\pgfsetfillcolor{currentfill}%
\pgfsetfillopacity{0.700000}%
\pgfsetlinewidth{0.000000pt}%
\definecolor{currentstroke}{rgb}{0.000000,0.000000,0.000000}%
\pgfsetstrokecolor{currentstroke}%
\pgfsetdash{}{0pt}%
\pgfpathmoveto{\pgfqpoint{3.407283in}{1.962036in}}%
\pgfpathlineto{\pgfqpoint{3.420967in}{1.954394in}}%
\pgfpathlineto{\pgfqpoint{3.434655in}{1.946871in}}%
\pgfpathlineto{\pgfqpoint{3.448345in}{1.939466in}}%
\pgfpathlineto{\pgfqpoint{3.462038in}{1.932179in}}%
\pgfpathlineto{\pgfqpoint{3.453792in}{1.928029in}}%
\pgfpathlineto{\pgfqpoint{3.445537in}{1.924044in}}%
\pgfpathlineto{\pgfqpoint{3.437272in}{1.920227in}}%
\pgfpathlineto{\pgfqpoint{3.428998in}{1.916584in}}%
\pgfpathlineto{\pgfqpoint{3.415280in}{1.924284in}}%
\pgfpathlineto{\pgfqpoint{3.401565in}{1.932101in}}%
\pgfpathlineto{\pgfqpoint{3.387853in}{1.940037in}}%
\pgfpathlineto{\pgfqpoint{3.374144in}{1.948093in}}%
\pgfpathlineto{\pgfqpoint{3.382444in}{1.951316in}}%
\pgfpathlineto{\pgfqpoint{3.390733in}{1.954717in}}%
\pgfpathlineto{\pgfqpoint{3.399013in}{1.958292in}}%
\pgfpathlineto{\pgfqpoint{3.407283in}{1.962036in}}%
\pgfpathclose%
\pgfusepath{fill}%
\end{pgfscope}%
\begin{pgfscope}%
\pgfpathrectangle{\pgfqpoint{1.150000in}{0.150000in}}{\pgfqpoint{5.700000in}{5.700000in}}%
\pgfusepath{clip}%
\pgfsetbuttcap%
\pgfsetroundjoin%
\definecolor{currentfill}{rgb}{0.190631,0.407061,0.556089}%
\pgfsetfillcolor{currentfill}%
\pgfsetfillopacity{0.700000}%
\pgfsetlinewidth{0.000000pt}%
\definecolor{currentstroke}{rgb}{0.000000,0.000000,0.000000}%
\pgfsetstrokecolor{currentstroke}%
\pgfsetdash{}{0pt}%
\pgfpathmoveto{\pgfqpoint{2.461000in}{2.765146in}}%
\pgfpathlineto{\pgfqpoint{2.474851in}{2.747191in}}%
\pgfpathlineto{\pgfqpoint{2.488695in}{2.729423in}}%
\pgfpathlineto{\pgfqpoint{2.502533in}{2.711840in}}%
\pgfpathlineto{\pgfqpoint{2.516365in}{2.694441in}}%
\pgfpathlineto{\pgfqpoint{2.507452in}{2.699022in}}%
\pgfpathlineto{\pgfqpoint{2.498520in}{2.703886in}}%
\pgfpathlineto{\pgfqpoint{2.489569in}{2.709036in}}%
\pgfpathlineto{\pgfqpoint{2.480598in}{2.714480in}}%
\pgfpathlineto{\pgfqpoint{2.466718in}{2.732362in}}%
\pgfpathlineto{\pgfqpoint{2.452831in}{2.750428in}}%
\pgfpathlineto{\pgfqpoint{2.438937in}{2.768681in}}%
\pgfpathlineto{\pgfqpoint{2.425037in}{2.787122in}}%
\pgfpathlineto{\pgfqpoint{2.434058in}{2.781185in}}%
\pgfpathlineto{\pgfqpoint{2.443059in}{2.775547in}}%
\pgfpathlineto{\pgfqpoint{2.452039in}{2.770203in}}%
\pgfpathlineto{\pgfqpoint{2.461000in}{2.765146in}}%
\pgfpathclose%
\pgfusepath{fill}%
\end{pgfscope}%
\begin{pgfscope}%
\pgfpathrectangle{\pgfqpoint{1.150000in}{0.150000in}}{\pgfqpoint{5.700000in}{5.700000in}}%
\pgfusepath{clip}%
\pgfsetbuttcap%
\pgfsetroundjoin%
\definecolor{currentfill}{rgb}{0.258965,0.251537,0.524736}%
\pgfsetfillcolor{currentfill}%
\pgfsetfillopacity{0.700000}%
\pgfsetlinewidth{0.000000pt}%
\definecolor{currentstroke}{rgb}{0.000000,0.000000,0.000000}%
\pgfsetstrokecolor{currentstroke}%
\pgfsetdash{}{0pt}%
\pgfpathmoveto{\pgfqpoint{2.791957in}{2.382705in}}%
\pgfpathlineto{\pgfqpoint{2.805697in}{2.368823in}}%
\pgfpathlineto{\pgfqpoint{2.819434in}{2.355094in}}%
\pgfpathlineto{\pgfqpoint{2.833170in}{2.341516in}}%
\pgfpathlineto{\pgfqpoint{2.846902in}{2.328090in}}%
\pgfpathlineto{\pgfqpoint{2.838248in}{2.329877in}}%
\pgfpathlineto{\pgfqpoint{2.829578in}{2.331915in}}%
\pgfpathlineto{\pgfqpoint{2.820893in}{2.334206in}}%
\pgfpathlineto{\pgfqpoint{2.812191in}{2.336758in}}%
\pgfpathlineto{\pgfqpoint{2.798418in}{2.350651in}}%
\pgfpathlineto{\pgfqpoint{2.784643in}{2.364695in}}%
\pgfpathlineto{\pgfqpoint{2.770865in}{2.378892in}}%
\pgfpathlineto{\pgfqpoint{2.757083in}{2.393242in}}%
\pgfpathlineto{\pgfqpoint{2.765826in}{2.390215in}}%
\pgfpathlineto{\pgfqpoint{2.774553in}{2.387453in}}%
\pgfpathlineto{\pgfqpoint{2.783263in}{2.384952in}}%
\pgfpathlineto{\pgfqpoint{2.791957in}{2.382705in}}%
\pgfpathclose%
\pgfusepath{fill}%
\end{pgfscope}%
\begin{pgfscope}%
\pgfpathrectangle{\pgfqpoint{1.150000in}{0.150000in}}{\pgfqpoint{5.700000in}{5.700000in}}%
\pgfusepath{clip}%
\pgfsetbuttcap%
\pgfsetroundjoin%
\definecolor{currentfill}{rgb}{0.153364,0.497000,0.557724}%
\pgfsetfillcolor{currentfill}%
\pgfsetfillopacity{0.700000}%
\pgfsetlinewidth{0.000000pt}%
\definecolor{currentstroke}{rgb}{0.000000,0.000000,0.000000}%
\pgfsetstrokecolor{currentstroke}%
\pgfsetdash{}{0pt}%
\pgfpathmoveto{\pgfqpoint{5.691970in}{2.983025in}}%
\pgfpathlineto{\pgfqpoint{5.706485in}{2.990990in}}%
\pgfpathlineto{\pgfqpoint{5.721015in}{2.999056in}}%
\pgfpathlineto{\pgfqpoint{5.735561in}{3.007223in}}%
\pgfpathlineto{\pgfqpoint{5.750123in}{3.015489in}}%
\pgfpathlineto{\pgfqpoint{5.742785in}{3.008167in}}%
\pgfpathlineto{\pgfqpoint{5.735438in}{3.000729in}}%
\pgfpathlineto{\pgfqpoint{5.728081in}{2.993175in}}%
\pgfpathlineto{\pgfqpoint{5.720716in}{2.985504in}}%
\pgfpathlineto{\pgfqpoint{5.706144in}{2.977184in}}%
\pgfpathlineto{\pgfqpoint{5.691587in}{2.968965in}}%
\pgfpathlineto{\pgfqpoint{5.677047in}{2.960847in}}%
\pgfpathlineto{\pgfqpoint{5.662522in}{2.952829in}}%
\pgfpathlineto{\pgfqpoint{5.669897in}{2.960545in}}%
\pgfpathlineto{\pgfqpoint{5.677264in}{2.968149in}}%
\pgfpathlineto{\pgfqpoint{5.684621in}{2.975642in}}%
\pgfpathlineto{\pgfqpoint{5.691970in}{2.983025in}}%
\pgfpathclose%
\pgfusepath{fill}%
\end{pgfscope}%
\begin{pgfscope}%
\pgfpathrectangle{\pgfqpoint{1.150000in}{0.150000in}}{\pgfqpoint{5.700000in}{5.700000in}}%
\pgfusepath{clip}%
\pgfsetbuttcap%
\pgfsetroundjoin%
\definecolor{currentfill}{rgb}{0.268510,0.009605,0.335427}%
\pgfsetfillcolor{currentfill}%
\pgfsetfillopacity{0.700000}%
\pgfsetlinewidth{0.000000pt}%
\definecolor{currentstroke}{rgb}{0.000000,0.000000,0.000000}%
\pgfsetstrokecolor{currentstroke}%
\pgfsetdash{}{0pt}%
\pgfpathmoveto{\pgfqpoint{3.975867in}{1.900712in}}%
\pgfpathlineto{\pgfqpoint{3.989635in}{1.898108in}}%
\pgfpathlineto{\pgfqpoint{4.003409in}{1.895610in}}%
\pgfpathlineto{\pgfqpoint{4.017191in}{1.893217in}}%
\pgfpathlineto{\pgfqpoint{4.030980in}{1.890929in}}%
\pgfpathlineto{\pgfqpoint{4.022991in}{1.882009in}}%
\pgfpathlineto{\pgfqpoint{4.014995in}{1.873159in}}%
\pgfpathlineto{\pgfqpoint{4.006994in}{1.864383in}}%
\pgfpathlineto{\pgfqpoint{3.998987in}{1.855684in}}%
\pgfpathlineto{\pgfqpoint{3.985187in}{1.858309in}}%
\pgfpathlineto{\pgfqpoint{3.971393in}{1.861039in}}%
\pgfpathlineto{\pgfqpoint{3.957605in}{1.863874in}}%
\pgfpathlineto{\pgfqpoint{3.943825in}{1.866814in}}%
\pgfpathlineto{\pgfqpoint{3.951844in}{1.875170in}}%
\pgfpathlineto{\pgfqpoint{3.959858in}{1.883607in}}%
\pgfpathlineto{\pgfqpoint{3.967865in}{1.892122in}}%
\pgfpathlineto{\pgfqpoint{3.975867in}{1.900712in}}%
\pgfpathclose%
\pgfusepath{fill}%
\end{pgfscope}%
\begin{pgfscope}%
\pgfpathrectangle{\pgfqpoint{1.150000in}{0.150000in}}{\pgfqpoint{5.700000in}{5.700000in}}%
\pgfusepath{clip}%
\pgfsetbuttcap%
\pgfsetroundjoin%
\definecolor{currentfill}{rgb}{0.252194,0.269783,0.531579}%
\pgfsetfillcolor{currentfill}%
\pgfsetfillopacity{0.700000}%
\pgfsetlinewidth{0.000000pt}%
\definecolor{currentstroke}{rgb}{0.000000,0.000000,0.000000}%
\pgfsetstrokecolor{currentstroke}%
\pgfsetdash{}{0pt}%
\pgfpathmoveto{\pgfqpoint{4.932841in}{2.406865in}}%
\pgfpathlineto{\pgfqpoint{4.946965in}{2.411277in}}%
\pgfpathlineto{\pgfqpoint{4.961102in}{2.415789in}}%
\pgfpathlineto{\pgfqpoint{4.975251in}{2.420400in}}%
\pgfpathlineto{\pgfqpoint{4.989413in}{2.425112in}}%
\pgfpathlineto{\pgfqpoint{4.981724in}{2.414067in}}%
\pgfpathlineto{\pgfqpoint{4.974029in}{2.402950in}}%
\pgfpathlineto{\pgfqpoint{4.966329in}{2.391760in}}%
\pgfpathlineto{\pgfqpoint{4.958623in}{2.380500in}}%
\pgfpathlineto{\pgfqpoint{4.944457in}{2.375928in}}%
\pgfpathlineto{\pgfqpoint{4.930304in}{2.371456in}}%
\pgfpathlineto{\pgfqpoint{4.916164in}{2.367084in}}%
\pgfpathlineto{\pgfqpoint{4.902036in}{2.362812in}}%
\pgfpathlineto{\pgfqpoint{4.909746in}{2.373925in}}%
\pgfpathlineto{\pgfqpoint{4.917450in}{2.384973in}}%
\pgfpathlineto{\pgfqpoint{4.925148in}{2.395953in}}%
\pgfpathlineto{\pgfqpoint{4.932841in}{2.406865in}}%
\pgfpathclose%
\pgfusepath{fill}%
\end{pgfscope}%
\begin{pgfscope}%
\pgfpathrectangle{\pgfqpoint{1.150000in}{0.150000in}}{\pgfqpoint{5.700000in}{5.700000in}}%
\pgfusepath{clip}%
\pgfsetbuttcap%
\pgfsetroundjoin%
\definecolor{currentfill}{rgb}{0.177423,0.437527,0.557565}%
\pgfsetfillcolor{currentfill}%
\pgfsetfillopacity{0.700000}%
\pgfsetlinewidth{0.000000pt}%
\definecolor{currentstroke}{rgb}{0.000000,0.000000,0.000000}%
\pgfsetstrokecolor{currentstroke}%
\pgfsetdash{}{0pt}%
\pgfpathmoveto{\pgfqpoint{2.405526in}{2.838875in}}%
\pgfpathlineto{\pgfqpoint{2.419406in}{2.820153in}}%
\pgfpathlineto{\pgfqpoint{2.433278in}{2.801625in}}%
\pgfpathlineto{\pgfqpoint{2.447142in}{2.783290in}}%
\pgfpathlineto{\pgfqpoint{2.461000in}{2.765146in}}%
\pgfpathlineto{\pgfqpoint{2.452039in}{2.770203in}}%
\pgfpathlineto{\pgfqpoint{2.443059in}{2.775547in}}%
\pgfpathlineto{\pgfqpoint{2.434058in}{2.781185in}}%
\pgfpathlineto{\pgfqpoint{2.425037in}{2.787122in}}%
\pgfpathlineto{\pgfqpoint{2.411129in}{2.805752in}}%
\pgfpathlineto{\pgfqpoint{2.397214in}{2.824575in}}%
\pgfpathlineto{\pgfqpoint{2.383291in}{2.843591in}}%
\pgfpathlineto{\pgfqpoint{2.369361in}{2.862802in}}%
\pgfpathlineto{\pgfqpoint{2.378433in}{2.856369in}}%
\pgfpathlineto{\pgfqpoint{2.387485in}{2.850240in}}%
\pgfpathlineto{\pgfqpoint{2.396516in}{2.844411in}}%
\pgfpathlineto{\pgfqpoint{2.405526in}{2.838875in}}%
\pgfpathclose%
\pgfusepath{fill}%
\end{pgfscope}%
\begin{pgfscope}%
\pgfpathrectangle{\pgfqpoint{1.150000in}{0.150000in}}{\pgfqpoint{5.700000in}{5.700000in}}%
\pgfusepath{clip}%
\pgfsetbuttcap%
\pgfsetroundjoin%
\definecolor{currentfill}{rgb}{0.197636,0.391528,0.554969}%
\pgfsetfillcolor{currentfill}%
\pgfsetfillopacity{0.700000}%
\pgfsetlinewidth{0.000000pt}%
\definecolor{currentstroke}{rgb}{0.000000,0.000000,0.000000}%
\pgfsetstrokecolor{currentstroke}%
\pgfsetdash{}{0pt}%
\pgfpathmoveto{\pgfqpoint{5.312556in}{2.698782in}}%
\pgfpathlineto{\pgfqpoint{5.326870in}{2.705210in}}%
\pgfpathlineto{\pgfqpoint{5.341199in}{2.711739in}}%
\pgfpathlineto{\pgfqpoint{5.355542in}{2.718367in}}%
\pgfpathlineto{\pgfqpoint{5.369899in}{2.725096in}}%
\pgfpathlineto{\pgfqpoint{5.362363in}{2.715512in}}%
\pgfpathlineto{\pgfqpoint{5.354819in}{2.705824in}}%
\pgfpathlineto{\pgfqpoint{5.347268in}{2.696033in}}%
\pgfpathlineto{\pgfqpoint{5.339710in}{2.686138in}}%
\pgfpathlineto{\pgfqpoint{5.325347in}{2.679455in}}%
\pgfpathlineto{\pgfqpoint{5.310998in}{2.672871in}}%
\pgfpathlineto{\pgfqpoint{5.296664in}{2.666388in}}%
\pgfpathlineto{\pgfqpoint{5.282344in}{2.660005in}}%
\pgfpathlineto{\pgfqpoint{5.289908in}{2.669847in}}%
\pgfpathlineto{\pgfqpoint{5.297465in}{2.679591in}}%
\pgfpathlineto{\pgfqpoint{5.305014in}{2.689236in}}%
\pgfpathlineto{\pgfqpoint{5.312556in}{2.698782in}}%
\pgfpathclose%
\pgfusepath{fill}%
\end{pgfscope}%
\begin{pgfscope}%
\pgfpathrectangle{\pgfqpoint{1.150000in}{0.150000in}}{\pgfqpoint{5.700000in}{5.700000in}}%
\pgfusepath{clip}%
\pgfsetbuttcap%
\pgfsetroundjoin%
\definecolor{currentfill}{rgb}{0.282327,0.094955,0.417331}%
\pgfsetfillcolor{currentfill}%
\pgfsetfillopacity{0.700000}%
\pgfsetlinewidth{0.000000pt}%
\definecolor{currentstroke}{rgb}{0.000000,0.000000,0.000000}%
\pgfsetstrokecolor{currentstroke}%
\pgfsetdash{}{0pt}%
\pgfpathmoveto{\pgfqpoint{3.209790in}{2.054295in}}%
\pgfpathlineto{\pgfqpoint{3.223477in}{2.044758in}}%
\pgfpathlineto{\pgfqpoint{3.237165in}{2.035349in}}%
\pgfpathlineto{\pgfqpoint{3.250855in}{2.026066in}}%
\pgfpathlineto{\pgfqpoint{3.264546in}{2.016908in}}%
\pgfpathlineto{\pgfqpoint{3.256181in}{2.014720in}}%
\pgfpathlineto{\pgfqpoint{3.247805in}{2.012728in}}%
\pgfpathlineto{\pgfqpoint{3.239417in}{2.010937in}}%
\pgfpathlineto{\pgfqpoint{3.231017in}{2.009351in}}%
\pgfpathlineto{\pgfqpoint{3.217296in}{2.018943in}}%
\pgfpathlineto{\pgfqpoint{3.203577in}{2.028661in}}%
\pgfpathlineto{\pgfqpoint{3.189858in}{2.038506in}}%
\pgfpathlineto{\pgfqpoint{3.176141in}{2.048478in}}%
\pgfpathlineto{\pgfqpoint{3.184571in}{2.049621in}}%
\pgfpathlineto{\pgfqpoint{3.192989in}{2.050975in}}%
\pgfpathlineto{\pgfqpoint{3.201395in}{2.052534in}}%
\pgfpathlineto{\pgfqpoint{3.209790in}{2.054295in}}%
\pgfpathclose%
\pgfusepath{fill}%
\end{pgfscope}%
\begin{pgfscope}%
\pgfpathrectangle{\pgfqpoint{1.150000in}{0.150000in}}{\pgfqpoint{5.700000in}{5.700000in}}%
\pgfusepath{clip}%
\pgfsetbuttcap%
\pgfsetroundjoin%
\definecolor{currentfill}{rgb}{0.266580,0.228262,0.514349}%
\pgfsetfillcolor{currentfill}%
\pgfsetfillopacity{0.700000}%
\pgfsetlinewidth{0.000000pt}%
\definecolor{currentstroke}{rgb}{0.000000,0.000000,0.000000}%
\pgfsetstrokecolor{currentstroke}%
\pgfsetdash{}{0pt}%
\pgfpathmoveto{\pgfqpoint{2.846902in}{2.328090in}}%
\pgfpathlineto{\pgfqpoint{2.860633in}{2.314813in}}%
\pgfpathlineto{\pgfqpoint{2.874362in}{2.301685in}}%
\pgfpathlineto{\pgfqpoint{2.888088in}{2.288705in}}%
\pgfpathlineto{\pgfqpoint{2.901813in}{2.275871in}}%
\pgfpathlineto{\pgfqpoint{2.893197in}{2.277201in}}%
\pgfpathlineto{\pgfqpoint{2.884567in}{2.278776in}}%
\pgfpathlineto{\pgfqpoint{2.875921in}{2.280600in}}%
\pgfpathlineto{\pgfqpoint{2.867259in}{2.282678in}}%
\pgfpathlineto{\pgfqpoint{2.853495in}{2.295977in}}%
\pgfpathlineto{\pgfqpoint{2.839730in}{2.309422in}}%
\pgfpathlineto{\pgfqpoint{2.825962in}{2.323015in}}%
\pgfpathlineto{\pgfqpoint{2.812191in}{2.336758in}}%
\pgfpathlineto{\pgfqpoint{2.820893in}{2.334206in}}%
\pgfpathlineto{\pgfqpoint{2.829578in}{2.331915in}}%
\pgfpathlineto{\pgfqpoint{2.838248in}{2.329877in}}%
\pgfpathlineto{\pgfqpoint{2.846902in}{2.328090in}}%
\pgfpathclose%
\pgfusepath{fill}%
\end{pgfscope}%
\begin{pgfscope}%
\pgfpathrectangle{\pgfqpoint{1.150000in}{0.150000in}}{\pgfqpoint{5.700000in}{5.700000in}}%
\pgfusepath{clip}%
\pgfsetbuttcap%
\pgfsetroundjoin%
\definecolor{currentfill}{rgb}{0.267004,0.004874,0.329415}%
\pgfsetfillcolor{currentfill}%
\pgfsetfillopacity{0.700000}%
\pgfsetlinewidth{0.000000pt}%
\definecolor{currentstroke}{rgb}{0.000000,0.000000,0.000000}%
\pgfsetstrokecolor{currentstroke}%
\pgfsetdash{}{0pt}%
\pgfpathmoveto{\pgfqpoint{3.746621in}{1.882180in}}%
\pgfpathlineto{\pgfqpoint{3.760345in}{1.877591in}}%
\pgfpathlineto{\pgfqpoint{3.774075in}{1.873112in}}%
\pgfpathlineto{\pgfqpoint{3.787810in}{1.868743in}}%
\pgfpathlineto{\pgfqpoint{3.801550in}{1.864482in}}%
\pgfpathlineto{\pgfqpoint{3.793469in}{1.857317in}}%
\pgfpathlineto{\pgfqpoint{3.785380in}{1.850263in}}%
\pgfpathlineto{\pgfqpoint{3.777284in}{1.843324in}}%
\pgfpathlineto{\pgfqpoint{3.769182in}{1.836504in}}%
\pgfpathlineto{\pgfqpoint{3.755424in}{1.841138in}}%
\pgfpathlineto{\pgfqpoint{3.741672in}{1.845881in}}%
\pgfpathlineto{\pgfqpoint{3.727926in}{1.850734in}}%
\pgfpathlineto{\pgfqpoint{3.714184in}{1.855696in}}%
\pgfpathlineto{\pgfqpoint{3.722304in}{1.862136in}}%
\pgfpathlineto{\pgfqpoint{3.730417in}{1.868699in}}%
\pgfpathlineto{\pgfqpoint{3.738523in}{1.875382in}}%
\pgfpathlineto{\pgfqpoint{3.746621in}{1.882180in}}%
\pgfpathclose%
\pgfusepath{fill}%
\end{pgfscope}%
\begin{pgfscope}%
\pgfpathrectangle{\pgfqpoint{1.150000in}{0.150000in}}{\pgfqpoint{5.700000in}{5.700000in}}%
\pgfusepath{clip}%
\pgfsetbuttcap%
\pgfsetroundjoin%
\definecolor{currentfill}{rgb}{0.268510,0.009605,0.335427}%
\pgfsetfillcolor{currentfill}%
\pgfsetfillopacity{0.700000}%
\pgfsetlinewidth{0.000000pt}%
\definecolor{currentstroke}{rgb}{0.000000,0.000000,0.000000}%
\pgfsetstrokecolor{currentstroke}%
\pgfsetdash{}{0pt}%
\pgfpathmoveto{\pgfqpoint{3.604422in}{1.899391in}}%
\pgfpathlineto{\pgfqpoint{3.618126in}{1.893537in}}%
\pgfpathlineto{\pgfqpoint{3.631835in}{1.887796in}}%
\pgfpathlineto{\pgfqpoint{3.645548in}{1.882167in}}%
\pgfpathlineto{\pgfqpoint{3.659266in}{1.876650in}}%
\pgfpathlineto{\pgfqpoint{3.651119in}{1.870722in}}%
\pgfpathlineto{\pgfqpoint{3.642965in}{1.864929in}}%
\pgfpathlineto{\pgfqpoint{3.634802in}{1.859275in}}%
\pgfpathlineto{\pgfqpoint{3.626631in}{1.853764in}}%
\pgfpathlineto{\pgfqpoint{3.612893in}{1.859673in}}%
\pgfpathlineto{\pgfqpoint{3.599160in}{1.865694in}}%
\pgfpathlineto{\pgfqpoint{3.585431in}{1.871828in}}%
\pgfpathlineto{\pgfqpoint{3.571705in}{1.878075in}}%
\pgfpathlineto{\pgfqpoint{3.579897in}{1.883186in}}%
\pgfpathlineto{\pgfqpoint{3.588080in}{1.888446in}}%
\pgfpathlineto{\pgfqpoint{3.596255in}{1.893849in}}%
\pgfpathlineto{\pgfqpoint{3.604422in}{1.899391in}}%
\pgfpathclose%
\pgfusepath{fill}%
\end{pgfscope}%
\begin{pgfscope}%
\pgfpathrectangle{\pgfqpoint{1.150000in}{0.150000in}}{\pgfqpoint{5.700000in}{5.700000in}}%
\pgfusepath{clip}%
\pgfsetbuttcap%
\pgfsetroundjoin%
\definecolor{currentfill}{rgb}{0.143343,0.522773,0.556295}%
\pgfsetfillcolor{currentfill}%
\pgfsetfillopacity{0.700000}%
\pgfsetlinewidth{0.000000pt}%
\definecolor{currentstroke}{rgb}{0.000000,0.000000,0.000000}%
\pgfsetstrokecolor{currentstroke}%
\pgfsetdash{}{0pt}%
\pgfpathmoveto{\pgfqpoint{5.779386in}{3.043648in}}%
\pgfpathlineto{\pgfqpoint{5.793953in}{3.051942in}}%
\pgfpathlineto{\pgfqpoint{5.808536in}{3.060337in}}%
\pgfpathlineto{\pgfqpoint{5.823135in}{3.068833in}}%
\pgfpathlineto{\pgfqpoint{5.837751in}{3.077429in}}%
\pgfpathlineto{\pgfqpoint{5.830461in}{3.070637in}}%
\pgfpathlineto{\pgfqpoint{5.823162in}{3.063729in}}%
\pgfpathlineto{\pgfqpoint{5.815853in}{3.056704in}}%
\pgfpathlineto{\pgfqpoint{5.808535in}{3.049561in}}%
\pgfpathlineto{\pgfqpoint{5.793907in}{3.040892in}}%
\pgfpathlineto{\pgfqpoint{5.779296in}{3.032324in}}%
\pgfpathlineto{\pgfqpoint{5.764701in}{3.023856in}}%
\pgfpathlineto{\pgfqpoint{5.750123in}{3.015489in}}%
\pgfpathlineto{\pgfqpoint{5.757452in}{3.022698in}}%
\pgfpathlineto{\pgfqpoint{5.764772in}{3.029793in}}%
\pgfpathlineto{\pgfqpoint{5.772084in}{3.036776in}}%
\pgfpathlineto{\pgfqpoint{5.779386in}{3.043648in}}%
\pgfpathclose%
\pgfusepath{fill}%
\end{pgfscope}%
\begin{pgfscope}%
\pgfpathrectangle{\pgfqpoint{1.150000in}{0.150000in}}{\pgfqpoint{5.700000in}{5.700000in}}%
\pgfusepath{clip}%
\pgfsetbuttcap%
\pgfsetroundjoin%
\definecolor{currentfill}{rgb}{0.165117,0.467423,0.558141}%
\pgfsetfillcolor{currentfill}%
\pgfsetfillopacity{0.700000}%
\pgfsetlinewidth{0.000000pt}%
\definecolor{currentstroke}{rgb}{0.000000,0.000000,0.000000}%
\pgfsetstrokecolor{currentstroke}%
\pgfsetdash{}{0pt}%
\pgfpathmoveto{\pgfqpoint{2.349930in}{2.915744in}}%
\pgfpathlineto{\pgfqpoint{2.363841in}{2.896225in}}%
\pgfpathlineto{\pgfqpoint{2.377744in}{2.876909in}}%
\pgfpathlineto{\pgfqpoint{2.391639in}{2.857793in}}%
\pgfpathlineto{\pgfqpoint{2.405526in}{2.838875in}}%
\pgfpathlineto{\pgfqpoint{2.396516in}{2.844411in}}%
\pgfpathlineto{\pgfqpoint{2.387485in}{2.850240in}}%
\pgfpathlineto{\pgfqpoint{2.378433in}{2.856369in}}%
\pgfpathlineto{\pgfqpoint{2.369361in}{2.862802in}}%
\pgfpathlineto{\pgfqpoint{2.355422in}{2.882210in}}%
\pgfpathlineto{\pgfqpoint{2.341476in}{2.901818in}}%
\pgfpathlineto{\pgfqpoint{2.327520in}{2.921626in}}%
\pgfpathlineto{\pgfqpoint{2.313556in}{2.941638in}}%
\pgfpathlineto{\pgfqpoint{2.322682in}{2.934705in}}%
\pgfpathlineto{\pgfqpoint{2.331786in}{2.928081in}}%
\pgfpathlineto{\pgfqpoint{2.340869in}{2.921763in}}%
\pgfpathlineto{\pgfqpoint{2.349930in}{2.915744in}}%
\pgfpathclose%
\pgfusepath{fill}%
\end{pgfscope}%
\begin{pgfscope}%
\pgfpathrectangle{\pgfqpoint{1.150000in}{0.150000in}}{\pgfqpoint{5.700000in}{5.700000in}}%
\pgfusepath{clip}%
\pgfsetbuttcap%
\pgfsetroundjoin%
\definecolor{currentfill}{rgb}{0.281924,0.089666,0.412415}%
\pgfsetfillcolor{currentfill}%
\pgfsetfillopacity{0.700000}%
\pgfsetlinewidth{0.000000pt}%
\definecolor{currentstroke}{rgb}{0.000000,0.000000,0.000000}%
\pgfsetstrokecolor{currentstroke}%
\pgfsetdash{}{0pt}%
\pgfpathmoveto{\pgfqpoint{4.379136in}{2.037032in}}%
\pgfpathlineto{\pgfqpoint{4.393030in}{2.037669in}}%
\pgfpathlineto{\pgfqpoint{4.406933in}{2.038407in}}%
\pgfpathlineto{\pgfqpoint{4.420846in}{2.039247in}}%
\pgfpathlineto{\pgfqpoint{4.434768in}{2.040188in}}%
\pgfpathlineto{\pgfqpoint{4.426905in}{2.029241in}}%
\pgfpathlineto{\pgfqpoint{4.419038in}{2.018298in}}%
\pgfpathlineto{\pgfqpoint{4.411166in}{2.007361in}}%
\pgfpathlineto{\pgfqpoint{4.403288in}{1.996432in}}%
\pgfpathlineto{\pgfqpoint{4.389360in}{1.995757in}}%
\pgfpathlineto{\pgfqpoint{4.375441in}{1.995184in}}%
\pgfpathlineto{\pgfqpoint{4.361531in}{1.994711in}}%
\pgfpathlineto{\pgfqpoint{4.347631in}{1.994341in}}%
\pgfpathlineto{\pgfqpoint{4.355515in}{2.004997in}}%
\pgfpathlineto{\pgfqpoint{4.363393in}{2.015666in}}%
\pgfpathlineto{\pgfqpoint{4.371267in}{2.026345in}}%
\pgfpathlineto{\pgfqpoint{4.379136in}{2.037032in}}%
\pgfpathclose%
\pgfusepath{fill}%
\end{pgfscope}%
\begin{pgfscope}%
\pgfpathrectangle{\pgfqpoint{1.150000in}{0.150000in}}{\pgfqpoint{5.700000in}{5.700000in}}%
\pgfusepath{clip}%
\pgfsetbuttcap%
\pgfsetroundjoin%
\definecolor{currentfill}{rgb}{0.283197,0.115680,0.436115}%
\pgfsetfillcolor{currentfill}%
\pgfsetfillopacity{0.700000}%
\pgfsetlinewidth{0.000000pt}%
\definecolor{currentstroke}{rgb}{0.000000,0.000000,0.000000}%
\pgfsetstrokecolor{currentstroke}%
\pgfsetdash{}{0pt}%
\pgfpathmoveto{\pgfqpoint{4.466168in}{2.083961in}}%
\pgfpathlineto{\pgfqpoint{4.480094in}{2.085251in}}%
\pgfpathlineto{\pgfqpoint{4.494031in}{2.086643in}}%
\pgfpathlineto{\pgfqpoint{4.507977in}{2.088136in}}%
\pgfpathlineto{\pgfqpoint{4.521933in}{2.089730in}}%
\pgfpathlineto{\pgfqpoint{4.514096in}{2.078552in}}%
\pgfpathlineto{\pgfqpoint{4.506254in}{2.067363in}}%
\pgfpathlineto{\pgfqpoint{4.498407in}{2.056166in}}%
\pgfpathlineto{\pgfqpoint{4.490555in}{2.044964in}}%
\pgfpathlineto{\pgfqpoint{4.476593in}{2.043618in}}%
\pgfpathlineto{\pgfqpoint{4.462642in}{2.042374in}}%
\pgfpathlineto{\pgfqpoint{4.448700in}{2.041230in}}%
\pgfpathlineto{\pgfqpoint{4.434768in}{2.040188in}}%
\pgfpathlineto{\pgfqpoint{4.442625in}{2.051135in}}%
\pgfpathlineto{\pgfqpoint{4.450478in}{2.062081in}}%
\pgfpathlineto{\pgfqpoint{4.458325in}{2.073024in}}%
\pgfpathlineto{\pgfqpoint{4.466168in}{2.083961in}}%
\pgfpathclose%
\pgfusepath{fill}%
\end{pgfscope}%
\begin{pgfscope}%
\pgfpathrectangle{\pgfqpoint{1.150000in}{0.150000in}}{\pgfqpoint{5.700000in}{5.700000in}}%
\pgfusepath{clip}%
\pgfsetbuttcap%
\pgfsetroundjoin%
\definecolor{currentfill}{rgb}{0.273006,0.204520,0.501721}%
\pgfsetfillcolor{currentfill}%
\pgfsetfillopacity{0.700000}%
\pgfsetlinewidth{0.000000pt}%
\definecolor{currentstroke}{rgb}{0.000000,0.000000,0.000000}%
\pgfsetstrokecolor{currentstroke}%
\pgfsetdash{}{0pt}%
\pgfpathmoveto{\pgfqpoint{2.901813in}{2.275871in}}%
\pgfpathlineto{\pgfqpoint{2.915536in}{2.263182in}}%
\pgfpathlineto{\pgfqpoint{2.929258in}{2.250638in}}%
\pgfpathlineto{\pgfqpoint{2.942979in}{2.238237in}}%
\pgfpathlineto{\pgfqpoint{2.956698in}{2.225979in}}%
\pgfpathlineto{\pgfqpoint{2.948119in}{2.226854in}}%
\pgfpathlineto{\pgfqpoint{2.939526in}{2.227968in}}%
\pgfpathlineto{\pgfqpoint{2.930918in}{2.229327in}}%
\pgfpathlineto{\pgfqpoint{2.922295in}{2.230934in}}%
\pgfpathlineto{\pgfqpoint{2.908539in}{2.243655in}}%
\pgfpathlineto{\pgfqpoint{2.894781in}{2.256518in}}%
\pgfpathlineto{\pgfqpoint{2.881021in}{2.269526in}}%
\pgfpathlineto{\pgfqpoint{2.867259in}{2.282678in}}%
\pgfpathlineto{\pgfqpoint{2.875921in}{2.280600in}}%
\pgfpathlineto{\pgfqpoint{2.884567in}{2.278776in}}%
\pgfpathlineto{\pgfqpoint{2.893197in}{2.277201in}}%
\pgfpathlineto{\pgfqpoint{2.901813in}{2.275871in}}%
\pgfpathclose%
\pgfusepath{fill}%
\end{pgfscope}%
\begin{pgfscope}%
\pgfpathrectangle{\pgfqpoint{1.150000in}{0.150000in}}{\pgfqpoint{5.700000in}{5.700000in}}%
\pgfusepath{clip}%
\pgfsetbuttcap%
\pgfsetroundjoin%
\definecolor{currentfill}{rgb}{0.241237,0.296485,0.539709}%
\pgfsetfillcolor{currentfill}%
\pgfsetfillopacity{0.700000}%
\pgfsetlinewidth{0.000000pt}%
\definecolor{currentstroke}{rgb}{0.000000,0.000000,0.000000}%
\pgfsetstrokecolor{currentstroke}%
\pgfsetdash{}{0pt}%
\pgfpathmoveto{\pgfqpoint{5.020107in}{2.468543in}}%
\pgfpathlineto{\pgfqpoint{5.034277in}{2.473475in}}%
\pgfpathlineto{\pgfqpoint{5.048461in}{2.478508in}}%
\pgfpathlineto{\pgfqpoint{5.062658in}{2.483641in}}%
\pgfpathlineto{\pgfqpoint{5.076868in}{2.488874in}}%
\pgfpathlineto{\pgfqpoint{5.069207in}{2.478017in}}%
\pgfpathlineto{\pgfqpoint{5.061541in}{2.467078in}}%
\pgfpathlineto{\pgfqpoint{5.053868in}{2.456058in}}%
\pgfpathlineto{\pgfqpoint{5.046189in}{2.444959in}}%
\pgfpathlineto{\pgfqpoint{5.031975in}{2.439848in}}%
\pgfpathlineto{\pgfqpoint{5.017775in}{2.434836in}}%
\pgfpathlineto{\pgfqpoint{5.003587in}{2.429924in}}%
\pgfpathlineto{\pgfqpoint{4.989413in}{2.425112in}}%
\pgfpathlineto{\pgfqpoint{4.997095in}{2.436083in}}%
\pgfpathlineto{\pgfqpoint{5.004772in}{2.446979in}}%
\pgfpathlineto{\pgfqpoint{5.012442in}{2.457799in}}%
\pgfpathlineto{\pgfqpoint{5.020107in}{2.468543in}}%
\pgfpathclose%
\pgfusepath{fill}%
\end{pgfscope}%
\begin{pgfscope}%
\pgfpathrectangle{\pgfqpoint{1.150000in}{0.150000in}}{\pgfqpoint{5.700000in}{5.700000in}}%
\pgfusepath{clip}%
\pgfsetbuttcap%
\pgfsetroundjoin%
\definecolor{currentfill}{rgb}{0.279566,0.067836,0.391917}%
\pgfsetfillcolor{currentfill}%
\pgfsetfillopacity{0.700000}%
\pgfsetlinewidth{0.000000pt}%
\definecolor{currentstroke}{rgb}{0.000000,0.000000,0.000000}%
\pgfsetstrokecolor{currentstroke}%
\pgfsetdash{}{0pt}%
\pgfpathmoveto{\pgfqpoint{4.292120in}{1.993877in}}%
\pgfpathlineto{\pgfqpoint{4.305984in}{1.993840in}}%
\pgfpathlineto{\pgfqpoint{4.319857in}{1.993905in}}%
\pgfpathlineto{\pgfqpoint{4.333740in}{1.994072in}}%
\pgfpathlineto{\pgfqpoint{4.347631in}{1.994341in}}%
\pgfpathlineto{\pgfqpoint{4.339742in}{1.983700in}}%
\pgfpathlineto{\pgfqpoint{4.331848in}{1.973077in}}%
\pgfpathlineto{\pgfqpoint{4.323949in}{1.962475in}}%
\pgfpathlineto{\pgfqpoint{4.316046in}{1.951896in}}%
\pgfpathlineto{\pgfqpoint{4.302147in}{1.951911in}}%
\pgfpathlineto{\pgfqpoint{4.288258in}{1.952028in}}%
\pgfpathlineto{\pgfqpoint{4.274377in}{1.952246in}}%
\pgfpathlineto{\pgfqpoint{4.260505in}{1.952567in}}%
\pgfpathlineto{\pgfqpoint{4.268417in}{1.962856in}}%
\pgfpathlineto{\pgfqpoint{4.276323in}{1.973172in}}%
\pgfpathlineto{\pgfqpoint{4.284224in}{1.983514in}}%
\pgfpathlineto{\pgfqpoint{4.292120in}{1.993877in}}%
\pgfpathclose%
\pgfusepath{fill}%
\end{pgfscope}%
\begin{pgfscope}%
\pgfpathrectangle{\pgfqpoint{1.150000in}{0.150000in}}{\pgfqpoint{5.700000in}{5.700000in}}%
\pgfusepath{clip}%
\pgfsetbuttcap%
\pgfsetroundjoin%
\definecolor{currentfill}{rgb}{0.282623,0.140926,0.457517}%
\pgfsetfillcolor{currentfill}%
\pgfsetfillopacity{0.700000}%
\pgfsetlinewidth{0.000000pt}%
\definecolor{currentstroke}{rgb}{0.000000,0.000000,0.000000}%
\pgfsetstrokecolor{currentstroke}%
\pgfsetdash{}{0pt}%
\pgfpathmoveto{\pgfqpoint{4.553232in}{2.134297in}}%
\pgfpathlineto{\pgfqpoint{4.567194in}{2.136222in}}%
\pgfpathlineto{\pgfqpoint{4.581166in}{2.138248in}}%
\pgfpathlineto{\pgfqpoint{4.595149in}{2.140374in}}%
\pgfpathlineto{\pgfqpoint{4.609142in}{2.142602in}}%
\pgfpathlineto{\pgfqpoint{4.601330in}{2.131262in}}%
\pgfpathlineto{\pgfqpoint{4.593512in}{2.119900in}}%
\pgfpathlineto{\pgfqpoint{4.585690in}{2.108516in}}%
\pgfpathlineto{\pgfqpoint{4.577862in}{2.097113in}}%
\pgfpathlineto{\pgfqpoint{4.563864in}{2.095116in}}%
\pgfpathlineto{\pgfqpoint{4.549877in}{2.093220in}}%
\pgfpathlineto{\pgfqpoint{4.535900in}{2.091425in}}%
\pgfpathlineto{\pgfqpoint{4.521933in}{2.089730in}}%
\pgfpathlineto{\pgfqpoint{4.529766in}{2.100896in}}%
\pgfpathlineto{\pgfqpoint{4.537593in}{2.112047in}}%
\pgfpathlineto{\pgfqpoint{4.545415in}{2.123181in}}%
\pgfpathlineto{\pgfqpoint{4.553232in}{2.134297in}}%
\pgfpathclose%
\pgfusepath{fill}%
\end{pgfscope}%
\begin{pgfscope}%
\pgfpathrectangle{\pgfqpoint{1.150000in}{0.150000in}}{\pgfqpoint{5.700000in}{5.700000in}}%
\pgfusepath{clip}%
\pgfsetbuttcap%
\pgfsetroundjoin%
\definecolor{currentfill}{rgb}{0.185556,0.418570,0.556753}%
\pgfsetfillcolor{currentfill}%
\pgfsetfillopacity{0.700000}%
\pgfsetlinewidth{0.000000pt}%
\definecolor{currentstroke}{rgb}{0.000000,0.000000,0.000000}%
\pgfsetstrokecolor{currentstroke}%
\pgfsetdash{}{0pt}%
\pgfpathmoveto{\pgfqpoint{5.399968in}{2.762394in}}%
\pgfpathlineto{\pgfqpoint{5.414334in}{2.769249in}}%
\pgfpathlineto{\pgfqpoint{5.428714in}{2.776204in}}%
\pgfpathlineto{\pgfqpoint{5.443110in}{2.783259in}}%
\pgfpathlineto{\pgfqpoint{5.457520in}{2.790414in}}%
\pgfpathlineto{\pgfqpoint{5.450021in}{2.781227in}}%
\pgfpathlineto{\pgfqpoint{5.442514in}{2.771930in}}%
\pgfpathlineto{\pgfqpoint{5.434999in}{2.762526in}}%
\pgfpathlineto{\pgfqpoint{5.427476in}{2.753012in}}%
\pgfpathlineto{\pgfqpoint{5.413060in}{2.745883in}}%
\pgfpathlineto{\pgfqpoint{5.398658in}{2.738854in}}%
\pgfpathlineto{\pgfqpoint{5.384271in}{2.731925in}}%
\pgfpathlineto{\pgfqpoint{5.369899in}{2.725096in}}%
\pgfpathlineto{\pgfqpoint{5.377428in}{2.734576in}}%
\pgfpathlineto{\pgfqpoint{5.384949in}{2.743953in}}%
\pgfpathlineto{\pgfqpoint{5.392462in}{2.753225in}}%
\pgfpathlineto{\pgfqpoint{5.399968in}{2.762394in}}%
\pgfpathclose%
\pgfusepath{fill}%
\end{pgfscope}%
\begin{pgfscope}%
\pgfpathrectangle{\pgfqpoint{1.150000in}{0.150000in}}{\pgfqpoint{5.700000in}{5.700000in}}%
\pgfusepath{clip}%
\pgfsetbuttcap%
\pgfsetroundjoin%
\definecolor{currentfill}{rgb}{0.267004,0.004874,0.329415}%
\pgfsetfillcolor{currentfill}%
\pgfsetfillopacity{0.700000}%
\pgfsetlinewidth{0.000000pt}%
\definecolor{currentstroke}{rgb}{0.000000,0.000000,0.000000}%
\pgfsetstrokecolor{currentstroke}%
\pgfsetdash{}{0pt}%
\pgfpathmoveto{\pgfqpoint{3.888767in}{1.879640in}}%
\pgfpathlineto{\pgfqpoint{3.902522in}{1.876274in}}%
\pgfpathlineto{\pgfqpoint{3.916283in}{1.873014in}}%
\pgfpathlineto{\pgfqpoint{3.930051in}{1.869861in}}%
\pgfpathlineto{\pgfqpoint{3.943825in}{1.866814in}}%
\pgfpathlineto{\pgfqpoint{3.935799in}{1.858544in}}%
\pgfpathlineto{\pgfqpoint{3.927767in}{1.850361in}}%
\pgfpathlineto{\pgfqpoint{3.919728in}{1.842270in}}%
\pgfpathlineto{\pgfqpoint{3.911684in}{1.834274in}}%
\pgfpathlineto{\pgfqpoint{3.897896in}{1.837676in}}%
\pgfpathlineto{\pgfqpoint{3.884114in}{1.841184in}}%
\pgfpathlineto{\pgfqpoint{3.870339in}{1.844799in}}%
\pgfpathlineto{\pgfqpoint{3.856569in}{1.848520in}}%
\pgfpathlineto{\pgfqpoint{3.864629in}{1.856154in}}%
\pgfpathlineto{\pgfqpoint{3.872681in}{1.863888in}}%
\pgfpathlineto{\pgfqpoint{3.880727in}{1.871718in}}%
\pgfpathlineto{\pgfqpoint{3.888767in}{1.879640in}}%
\pgfpathclose%
\pgfusepath{fill}%
\end{pgfscope}%
\begin{pgfscope}%
\pgfpathrectangle{\pgfqpoint{1.150000in}{0.150000in}}{\pgfqpoint{5.700000in}{5.700000in}}%
\pgfusepath{clip}%
\pgfsetbuttcap%
\pgfsetroundjoin%
\definecolor{currentfill}{rgb}{0.276022,0.044167,0.370164}%
\pgfsetfillcolor{currentfill}%
\pgfsetfillopacity{0.700000}%
\pgfsetlinewidth{0.000000pt}%
\definecolor{currentstroke}{rgb}{0.000000,0.000000,0.000000}%
\pgfsetstrokecolor{currentstroke}%
\pgfsetdash{}{0pt}%
\pgfpathmoveto{\pgfqpoint{4.205103in}{1.954877in}}%
\pgfpathlineto{\pgfqpoint{4.218941in}{1.954145in}}%
\pgfpathlineto{\pgfqpoint{4.232787in}{1.953517in}}%
\pgfpathlineto{\pgfqpoint{4.246642in}{1.952991in}}%
\pgfpathlineto{\pgfqpoint{4.260505in}{1.952567in}}%
\pgfpathlineto{\pgfqpoint{4.252589in}{1.942309in}}%
\pgfpathlineto{\pgfqpoint{4.244667in}{1.932084in}}%
\pgfpathlineto{\pgfqpoint{4.236740in}{1.921895in}}%
\pgfpathlineto{\pgfqpoint{4.228808in}{1.911745in}}%
\pgfpathlineto{\pgfqpoint{4.214937in}{1.912470in}}%
\pgfpathlineto{\pgfqpoint{4.201073in}{1.913297in}}%
\pgfpathlineto{\pgfqpoint{4.187218in}{1.914227in}}%
\pgfpathlineto{\pgfqpoint{4.173372in}{1.915260in}}%
\pgfpathlineto{\pgfqpoint{4.181312in}{1.925102in}}%
\pgfpathlineto{\pgfqpoint{4.189248in}{1.934988in}}%
\pgfpathlineto{\pgfqpoint{4.197178in}{1.944913in}}%
\pgfpathlineto{\pgfqpoint{4.205103in}{1.954877in}}%
\pgfpathclose%
\pgfusepath{fill}%
\end{pgfscope}%
\begin{pgfscope}%
\pgfpathrectangle{\pgfqpoint{1.150000in}{0.150000in}}{\pgfqpoint{5.700000in}{5.700000in}}%
\pgfusepath{clip}%
\pgfsetbuttcap%
\pgfsetroundjoin%
\definecolor{currentfill}{rgb}{0.273809,0.031497,0.358853}%
\pgfsetfillcolor{currentfill}%
\pgfsetfillopacity{0.700000}%
\pgfsetlinewidth{0.000000pt}%
\definecolor{currentstroke}{rgb}{0.000000,0.000000,0.000000}%
\pgfsetstrokecolor{currentstroke}%
\pgfsetdash{}{0pt}%
\pgfpathmoveto{\pgfqpoint{3.462038in}{1.932179in}}%
\pgfpathlineto{\pgfqpoint{3.475734in}{1.925010in}}%
\pgfpathlineto{\pgfqpoint{3.489434in}{1.917958in}}%
\pgfpathlineto{\pgfqpoint{3.503137in}{1.911023in}}%
\pgfpathlineto{\pgfqpoint{3.516843in}{1.904203in}}%
\pgfpathlineto{\pgfqpoint{3.508621in}{1.899648in}}%
\pgfpathlineto{\pgfqpoint{3.500389in}{1.895252in}}%
\pgfpathlineto{\pgfqpoint{3.492149in}{1.891021in}}%
\pgfpathlineto{\pgfqpoint{3.483898in}{1.886959in}}%
\pgfpathlineto{\pgfqpoint{3.470169in}{1.894191in}}%
\pgfpathlineto{\pgfqpoint{3.456442in}{1.901538in}}%
\pgfpathlineto{\pgfqpoint{3.442718in}{1.909003in}}%
\pgfpathlineto{\pgfqpoint{3.428998in}{1.916584in}}%
\pgfpathlineto{\pgfqpoint{3.437272in}{1.920227in}}%
\pgfpathlineto{\pgfqpoint{3.445537in}{1.924044in}}%
\pgfpathlineto{\pgfqpoint{3.453792in}{1.928029in}}%
\pgfpathlineto{\pgfqpoint{3.462038in}{1.932179in}}%
\pgfpathclose%
\pgfusepath{fill}%
\end{pgfscope}%
\begin{pgfscope}%
\pgfpathrectangle{\pgfqpoint{1.150000in}{0.150000in}}{\pgfqpoint{5.700000in}{5.700000in}}%
\pgfusepath{clip}%
\pgfsetbuttcap%
\pgfsetroundjoin%
\definecolor{currentfill}{rgb}{0.135066,0.544853,0.554029}%
\pgfsetfillcolor{currentfill}%
\pgfsetfillopacity{0.700000}%
\pgfsetlinewidth{0.000000pt}%
\definecolor{currentstroke}{rgb}{0.000000,0.000000,0.000000}%
\pgfsetstrokecolor{currentstroke}%
\pgfsetdash{}{0pt}%
\pgfpathmoveto{\pgfqpoint{5.866819in}{3.103458in}}%
\pgfpathlineto{\pgfqpoint{5.881439in}{3.112062in}}%
\pgfpathlineto{\pgfqpoint{5.896075in}{3.120766in}}%
\pgfpathlineto{\pgfqpoint{5.910728in}{3.129571in}}%
\pgfpathlineto{\pgfqpoint{5.925398in}{3.138476in}}%
\pgfpathlineto{\pgfqpoint{5.918158in}{3.132237in}}%
\pgfpathlineto{\pgfqpoint{5.910909in}{3.125882in}}%
\pgfpathlineto{\pgfqpoint{5.903650in}{3.119409in}}%
\pgfpathlineto{\pgfqpoint{5.896381in}{3.112818in}}%
\pgfpathlineto{\pgfqpoint{5.881699in}{3.103819in}}%
\pgfpathlineto{\pgfqpoint{5.867033in}{3.094922in}}%
\pgfpathlineto{\pgfqpoint{5.852384in}{3.086125in}}%
\pgfpathlineto{\pgfqpoint{5.837751in}{3.077429in}}%
\pgfpathlineto{\pgfqpoint{5.845032in}{3.084105in}}%
\pgfpathlineto{\pgfqpoint{5.852304in}{3.090669in}}%
\pgfpathlineto{\pgfqpoint{5.859566in}{3.097119in}}%
\pgfpathlineto{\pgfqpoint{5.866819in}{3.103458in}}%
\pgfpathclose%
\pgfusepath{fill}%
\end{pgfscope}%
\begin{pgfscope}%
\pgfpathrectangle{\pgfqpoint{1.150000in}{0.150000in}}{\pgfqpoint{5.700000in}{5.700000in}}%
\pgfusepath{clip}%
\pgfsetbuttcap%
\pgfsetroundjoin%
\definecolor{currentfill}{rgb}{0.279574,0.170599,0.479997}%
\pgfsetfillcolor{currentfill}%
\pgfsetfillopacity{0.700000}%
\pgfsetlinewidth{0.000000pt}%
\definecolor{currentstroke}{rgb}{0.000000,0.000000,0.000000}%
\pgfsetstrokecolor{currentstroke}%
\pgfsetdash{}{0pt}%
\pgfpathmoveto{\pgfqpoint{4.640341in}{2.187686in}}%
\pgfpathlineto{\pgfqpoint{4.654341in}{2.190226in}}%
\pgfpathlineto{\pgfqpoint{4.668352in}{2.192867in}}%
\pgfpathlineto{\pgfqpoint{4.682374in}{2.195608in}}%
\pgfpathlineto{\pgfqpoint{4.696406in}{2.198450in}}%
\pgfpathlineto{\pgfqpoint{4.688619in}{2.187018in}}%
\pgfpathlineto{\pgfqpoint{4.680825in}{2.175550in}}%
\pgfpathlineto{\pgfqpoint{4.673027in}{2.164048in}}%
\pgfpathlineto{\pgfqpoint{4.665224in}{2.152515in}}%
\pgfpathlineto{\pgfqpoint{4.651187in}{2.149886in}}%
\pgfpathlineto{\pgfqpoint{4.637161in}{2.147358in}}%
\pgfpathlineto{\pgfqpoint{4.623146in}{2.144930in}}%
\pgfpathlineto{\pgfqpoint{4.609142in}{2.142602in}}%
\pgfpathlineto{\pgfqpoint{4.616950in}{2.153916in}}%
\pgfpathlineto{\pgfqpoint{4.624752in}{2.165202in}}%
\pgfpathlineto{\pgfqpoint{4.632549in}{2.176459in}}%
\pgfpathlineto{\pgfqpoint{4.640341in}{2.187686in}}%
\pgfpathclose%
\pgfusepath{fill}%
\end{pgfscope}%
\begin{pgfscope}%
\pgfpathrectangle{\pgfqpoint{1.150000in}{0.150000in}}{\pgfqpoint{5.700000in}{5.700000in}}%
\pgfusepath{clip}%
\pgfsetbuttcap%
\pgfsetroundjoin%
\definecolor{currentfill}{rgb}{0.281446,0.084320,0.407414}%
\pgfsetfillcolor{currentfill}%
\pgfsetfillopacity{0.700000}%
\pgfsetlinewidth{0.000000pt}%
\definecolor{currentstroke}{rgb}{0.000000,0.000000,0.000000}%
\pgfsetstrokecolor{currentstroke}%
\pgfsetdash{}{0pt}%
\pgfpathmoveto{\pgfqpoint{3.264546in}{2.016908in}}%
\pgfpathlineto{\pgfqpoint{3.278239in}{2.007876in}}%
\pgfpathlineto{\pgfqpoint{3.291934in}{1.998968in}}%
\pgfpathlineto{\pgfqpoint{3.305630in}{1.990183in}}%
\pgfpathlineto{\pgfqpoint{3.319329in}{1.981522in}}%
\pgfpathlineto{\pgfqpoint{3.310992in}{1.978907in}}%
\pgfpathlineto{\pgfqpoint{3.302644in}{1.976483in}}%
\pgfpathlineto{\pgfqpoint{3.294285in}{1.974255in}}%
\pgfpathlineto{\pgfqpoint{3.285914in}{1.972228in}}%
\pgfpathlineto{\pgfqpoint{3.272188in}{1.981323in}}%
\pgfpathlineto{\pgfqpoint{3.258463in}{1.990541in}}%
\pgfpathlineto{\pgfqpoint{3.244739in}{1.999884in}}%
\pgfpathlineto{\pgfqpoint{3.231017in}{2.009351in}}%
\pgfpathlineto{\pgfqpoint{3.239417in}{2.010937in}}%
\pgfpathlineto{\pgfqpoint{3.247805in}{2.012728in}}%
\pgfpathlineto{\pgfqpoint{3.256181in}{2.014720in}}%
\pgfpathlineto{\pgfqpoint{3.264546in}{2.016908in}}%
\pgfpathclose%
\pgfusepath{fill}%
\end{pgfscope}%
\begin{pgfscope}%
\pgfpathrectangle{\pgfqpoint{1.150000in}{0.150000in}}{\pgfqpoint{5.700000in}{5.700000in}}%
\pgfusepath{clip}%
\pgfsetbuttcap%
\pgfsetroundjoin%
\definecolor{currentfill}{rgb}{0.277134,0.185228,0.489898}%
\pgfsetfillcolor{currentfill}%
\pgfsetfillopacity{0.700000}%
\pgfsetlinewidth{0.000000pt}%
\definecolor{currentstroke}{rgb}{0.000000,0.000000,0.000000}%
\pgfsetstrokecolor{currentstroke}%
\pgfsetdash{}{0pt}%
\pgfpathmoveto{\pgfqpoint{2.956698in}{2.225979in}}%
\pgfpathlineto{\pgfqpoint{2.970416in}{2.213862in}}%
\pgfpathlineto{\pgfqpoint{2.984133in}{2.201886in}}%
\pgfpathlineto{\pgfqpoint{2.997849in}{2.190049in}}%
\pgfpathlineto{\pgfqpoint{3.011565in}{2.178351in}}%
\pgfpathlineto{\pgfqpoint{3.003022in}{2.178772in}}%
\pgfpathlineto{\pgfqpoint{2.994465in}{2.179428in}}%
\pgfpathlineto{\pgfqpoint{2.985894in}{2.180323in}}%
\pgfpathlineto{\pgfqpoint{2.977309in}{2.181462in}}%
\pgfpathlineto{\pgfqpoint{2.963557in}{2.193620in}}%
\pgfpathlineto{\pgfqpoint{2.949805in}{2.205918in}}%
\pgfpathlineto{\pgfqpoint{2.936051in}{2.218356in}}%
\pgfpathlineto{\pgfqpoint{2.922295in}{2.230934in}}%
\pgfpathlineto{\pgfqpoint{2.930918in}{2.229327in}}%
\pgfpathlineto{\pgfqpoint{2.939526in}{2.227968in}}%
\pgfpathlineto{\pgfqpoint{2.948119in}{2.226854in}}%
\pgfpathlineto{\pgfqpoint{2.956698in}{2.225979in}}%
\pgfpathclose%
\pgfusepath{fill}%
\end{pgfscope}%
\begin{pgfscope}%
\pgfpathrectangle{\pgfqpoint{1.150000in}{0.150000in}}{\pgfqpoint{5.700000in}{5.700000in}}%
\pgfusepath{clip}%
\pgfsetbuttcap%
\pgfsetroundjoin%
\definecolor{currentfill}{rgb}{0.227802,0.326594,0.546532}%
\pgfsetfillcolor{currentfill}%
\pgfsetfillopacity{0.700000}%
\pgfsetlinewidth{0.000000pt}%
\definecolor{currentstroke}{rgb}{0.000000,0.000000,0.000000}%
\pgfsetstrokecolor{currentstroke}%
\pgfsetdash{}{0pt}%
\pgfpathmoveto{\pgfqpoint{5.107446in}{2.531472in}}%
\pgfpathlineto{\pgfqpoint{5.121665in}{2.536907in}}%
\pgfpathlineto{\pgfqpoint{5.135897in}{2.542442in}}%
\pgfpathlineto{\pgfqpoint{5.150143in}{2.548077in}}%
\pgfpathlineto{\pgfqpoint{5.164403in}{2.553812in}}%
\pgfpathlineto{\pgfqpoint{5.156772in}{2.543194in}}%
\pgfpathlineto{\pgfqpoint{5.149135in}{2.532486in}}%
\pgfpathlineto{\pgfqpoint{5.141491in}{2.521690in}}%
\pgfpathlineto{\pgfqpoint{5.133841in}{2.510805in}}%
\pgfpathlineto{\pgfqpoint{5.119578in}{2.505172in}}%
\pgfpathlineto{\pgfqpoint{5.105328in}{2.499640in}}%
\pgfpathlineto{\pgfqpoint{5.091091in}{2.494207in}}%
\pgfpathlineto{\pgfqpoint{5.076868in}{2.488874in}}%
\pgfpathlineto{\pgfqpoint{5.084522in}{2.499649in}}%
\pgfpathlineto{\pgfqpoint{5.092170in}{2.510341in}}%
\pgfpathlineto{\pgfqpoint{5.099811in}{2.520949in}}%
\pgfpathlineto{\pgfqpoint{5.107446in}{2.531472in}}%
\pgfpathclose%
\pgfusepath{fill}%
\end{pgfscope}%
\begin{pgfscope}%
\pgfpathrectangle{\pgfqpoint{1.150000in}{0.150000in}}{\pgfqpoint{5.700000in}{5.700000in}}%
\pgfusepath{clip}%
\pgfsetbuttcap%
\pgfsetroundjoin%
\definecolor{currentfill}{rgb}{0.272594,0.025563,0.353093}%
\pgfsetfillcolor{currentfill}%
\pgfsetfillopacity{0.700000}%
\pgfsetlinewidth{0.000000pt}%
\definecolor{currentstroke}{rgb}{0.000000,0.000000,0.000000}%
\pgfsetstrokecolor{currentstroke}%
\pgfsetdash{}{0pt}%
\pgfpathmoveto{\pgfqpoint{4.118064in}{1.920425in}}%
\pgfpathlineto{\pgfqpoint{4.131879in}{1.918978in}}%
\pgfpathlineto{\pgfqpoint{4.145702in}{1.917635in}}%
\pgfpathlineto{\pgfqpoint{4.159533in}{1.916396in}}%
\pgfpathlineto{\pgfqpoint{4.173372in}{1.915260in}}%
\pgfpathlineto{\pgfqpoint{4.165425in}{1.905464in}}%
\pgfpathlineto{\pgfqpoint{4.157474in}{1.895718in}}%
\pgfpathlineto{\pgfqpoint{4.149517in}{1.886023in}}%
\pgfpathlineto{\pgfqpoint{4.141555in}{1.876384in}}%
\pgfpathlineto{\pgfqpoint{4.127706in}{1.877839in}}%
\pgfpathlineto{\pgfqpoint{4.113866in}{1.879398in}}%
\pgfpathlineto{\pgfqpoint{4.100033in}{1.881060in}}%
\pgfpathlineto{\pgfqpoint{4.086207in}{1.882825in}}%
\pgfpathlineto{\pgfqpoint{4.094180in}{1.892139in}}%
\pgfpathlineto{\pgfqpoint{4.102147in}{1.901512in}}%
\pgfpathlineto{\pgfqpoint{4.110108in}{1.910941in}}%
\pgfpathlineto{\pgfqpoint{4.118064in}{1.920425in}}%
\pgfpathclose%
\pgfusepath{fill}%
\end{pgfscope}%
\begin{pgfscope}%
\pgfpathrectangle{\pgfqpoint{1.150000in}{0.150000in}}{\pgfqpoint{5.700000in}{5.700000in}}%
\pgfusepath{clip}%
\pgfsetbuttcap%
\pgfsetroundjoin%
\definecolor{currentfill}{rgb}{0.274128,0.199721,0.498911}%
\pgfsetfillcolor{currentfill}%
\pgfsetfillopacity{0.700000}%
\pgfsetlinewidth{0.000000pt}%
\definecolor{currentstroke}{rgb}{0.000000,0.000000,0.000000}%
\pgfsetstrokecolor{currentstroke}%
\pgfsetdash{}{0pt}%
\pgfpathmoveto{\pgfqpoint{4.727506in}{2.243786in}}%
\pgfpathlineto{\pgfqpoint{4.741546in}{2.246922in}}%
\pgfpathlineto{\pgfqpoint{4.755598in}{2.250159in}}%
\pgfpathlineto{\pgfqpoint{4.769661in}{2.253496in}}%
\pgfpathlineto{\pgfqpoint{4.783736in}{2.256933in}}%
\pgfpathlineto{\pgfqpoint{4.775973in}{2.245474in}}%
\pgfpathlineto{\pgfqpoint{4.768204in}{2.233967in}}%
\pgfpathlineto{\pgfqpoint{4.760431in}{2.222414in}}%
\pgfpathlineto{\pgfqpoint{4.752651in}{2.210818in}}%
\pgfpathlineto{\pgfqpoint{4.738573in}{2.207576in}}%
\pgfpathlineto{\pgfqpoint{4.724506in}{2.204434in}}%
\pgfpathlineto{\pgfqpoint{4.710451in}{2.201392in}}%
\pgfpathlineto{\pgfqpoint{4.696406in}{2.198450in}}%
\pgfpathlineto{\pgfqpoint{4.704189in}{2.209844in}}%
\pgfpathlineto{\pgfqpoint{4.711967in}{2.221200in}}%
\pgfpathlineto{\pgfqpoint{4.719739in}{2.232514in}}%
\pgfpathlineto{\pgfqpoint{4.727506in}{2.243786in}}%
\pgfpathclose%
\pgfusepath{fill}%
\end{pgfscope}%
\begin{pgfscope}%
\pgfpathrectangle{\pgfqpoint{1.150000in}{0.150000in}}{\pgfqpoint{5.700000in}{5.700000in}}%
\pgfusepath{clip}%
\pgfsetbuttcap%
\pgfsetroundjoin%
\definecolor{currentfill}{rgb}{0.174274,0.445044,0.557792}%
\pgfsetfillcolor{currentfill}%
\pgfsetfillopacity{0.700000}%
\pgfsetlinewidth{0.000000pt}%
\definecolor{currentstroke}{rgb}{0.000000,0.000000,0.000000}%
\pgfsetstrokecolor{currentstroke}%
\pgfsetdash{}{0pt}%
\pgfpathmoveto{\pgfqpoint{5.487438in}{2.826082in}}%
\pgfpathlineto{\pgfqpoint{5.501856in}{2.833344in}}%
\pgfpathlineto{\pgfqpoint{5.516289in}{2.840706in}}%
\pgfpathlineto{\pgfqpoint{5.530737in}{2.848168in}}%
\pgfpathlineto{\pgfqpoint{5.545201in}{2.855731in}}%
\pgfpathlineto{\pgfqpoint{5.537741in}{2.846977in}}%
\pgfpathlineto{\pgfqpoint{5.530273in}{2.838110in}}%
\pgfpathlineto{\pgfqpoint{5.522796in}{2.829130in}}%
\pgfpathlineto{\pgfqpoint{5.515312in}{2.820037in}}%
\pgfpathlineto{\pgfqpoint{5.500841in}{2.812481in}}%
\pgfpathlineto{\pgfqpoint{5.486386in}{2.805025in}}%
\pgfpathlineto{\pgfqpoint{5.471945in}{2.797670in}}%
\pgfpathlineto{\pgfqpoint{5.457520in}{2.790414in}}%
\pgfpathlineto{\pgfqpoint{5.465011in}{2.799493in}}%
\pgfpathlineto{\pgfqpoint{5.472495in}{2.808464in}}%
\pgfpathlineto{\pgfqpoint{5.479970in}{2.817327in}}%
\pgfpathlineto{\pgfqpoint{5.487438in}{2.826082in}}%
\pgfpathclose%
\pgfusepath{fill}%
\end{pgfscope}%
\begin{pgfscope}%
\pgfpathrectangle{\pgfqpoint{1.150000in}{0.150000in}}{\pgfqpoint{5.700000in}{5.700000in}}%
\pgfusepath{clip}%
\pgfsetbuttcap%
\pgfsetroundjoin%
\definecolor{currentfill}{rgb}{0.121831,0.589055,0.545623}%
\pgfsetfillcolor{currentfill}%
\pgfsetfillopacity{0.700000}%
\pgfsetlinewidth{0.000000pt}%
\definecolor{currentstroke}{rgb}{0.000000,0.000000,0.000000}%
\pgfsetstrokecolor{currentstroke}%
\pgfsetdash{}{0pt}%
\pgfpathmoveto{\pgfqpoint{6.041699in}{3.220025in}}%
\pgfpathlineto{\pgfqpoint{6.056424in}{3.229187in}}%
\pgfpathlineto{\pgfqpoint{6.071167in}{3.238450in}}%
\pgfpathlineto{\pgfqpoint{6.085926in}{3.247814in}}%
\pgfpathlineto{\pgfqpoint{6.078791in}{3.242697in}}%
\pgfpathlineto{\pgfqpoint{6.071646in}{3.237468in}}%
\pgfpathlineto{\pgfqpoint{6.064491in}{3.232124in}}%
\pgfpathlineto{\pgfqpoint{6.057326in}{3.226664in}}%
\pgfpathlineto{\pgfqpoint{6.042551in}{3.217167in}}%
\pgfpathlineto{\pgfqpoint{6.027793in}{3.207770in}}%
\pgfpathlineto{\pgfqpoint{6.013052in}{3.198475in}}%
\pgfpathlineto{\pgfqpoint{6.020228in}{3.204029in}}%
\pgfpathlineto{\pgfqpoint{6.027395in}{3.209471in}}%
\pgfpathlineto{\pgfqpoint{6.034552in}{3.214802in}}%
\pgfpathlineto{\pgfqpoint{6.041699in}{3.220025in}}%
\pgfpathclose%
\pgfusepath{fill}%
\end{pgfscope}%
\begin{pgfscope}%
\pgfpathrectangle{\pgfqpoint{1.150000in}{0.150000in}}{\pgfqpoint{5.700000in}{5.700000in}}%
\pgfusepath{clip}%
\pgfsetbuttcap%
\pgfsetroundjoin%
\definecolor{currentfill}{rgb}{0.126453,0.570633,0.549841}%
\pgfsetfillcolor{currentfill}%
\pgfsetfillopacity{0.700000}%
\pgfsetlinewidth{0.000000pt}%
\definecolor{currentstroke}{rgb}{0.000000,0.000000,0.000000}%
\pgfsetstrokecolor{currentstroke}%
\pgfsetdash{}{0pt}%
\pgfpathmoveto{\pgfqpoint{5.954260in}{3.162299in}}%
\pgfpathlineto{\pgfqpoint{5.968933in}{3.171192in}}%
\pgfpathlineto{\pgfqpoint{5.983622in}{3.180186in}}%
\pgfpathlineto{\pgfqpoint{5.998328in}{3.189280in}}%
\pgfpathlineto{\pgfqpoint{6.013052in}{3.198475in}}%
\pgfpathlineto{\pgfqpoint{6.005865in}{3.192807in}}%
\pgfpathlineto{\pgfqpoint{5.998669in}{3.187023in}}%
\pgfpathlineto{\pgfqpoint{5.991463in}{3.181123in}}%
\pgfpathlineto{\pgfqpoint{5.984246in}{3.175104in}}%
\pgfpathlineto{\pgfqpoint{5.969509in}{3.165796in}}%
\pgfpathlineto{\pgfqpoint{5.954788in}{3.156589in}}%
\pgfpathlineto{\pgfqpoint{5.940084in}{3.147482in}}%
\pgfpathlineto{\pgfqpoint{5.925398in}{3.138476in}}%
\pgfpathlineto{\pgfqpoint{5.932628in}{3.144601in}}%
\pgfpathlineto{\pgfqpoint{5.939848in}{3.150612in}}%
\pgfpathlineto{\pgfqpoint{5.947059in}{3.156511in}}%
\pgfpathlineto{\pgfqpoint{5.954260in}{3.162299in}}%
\pgfpathclose%
\pgfusepath{fill}%
\end{pgfscope}%
\begin{pgfscope}%
\pgfpathrectangle{\pgfqpoint{1.150000in}{0.150000in}}{\pgfqpoint{5.700000in}{5.700000in}}%
\pgfusepath{clip}%
\pgfsetbuttcap%
\pgfsetroundjoin%
\definecolor{currentfill}{rgb}{0.280255,0.165693,0.476498}%
\pgfsetfillcolor{currentfill}%
\pgfsetfillopacity{0.700000}%
\pgfsetlinewidth{0.000000pt}%
\definecolor{currentstroke}{rgb}{0.000000,0.000000,0.000000}%
\pgfsetstrokecolor{currentstroke}%
\pgfsetdash{}{0pt}%
\pgfpathmoveto{\pgfqpoint{3.011565in}{2.178351in}}%
\pgfpathlineto{\pgfqpoint{3.025280in}{2.166791in}}%
\pgfpathlineto{\pgfqpoint{3.038994in}{2.155367in}}%
\pgfpathlineto{\pgfqpoint{3.052708in}{2.144080in}}%
\pgfpathlineto{\pgfqpoint{3.066422in}{2.132928in}}%
\pgfpathlineto{\pgfqpoint{3.057914in}{2.132898in}}%
\pgfpathlineto{\pgfqpoint{3.049393in}{2.133097in}}%
\pgfpathlineto{\pgfqpoint{3.040857in}{2.133530in}}%
\pgfpathlineto{\pgfqpoint{3.032308in}{2.134201in}}%
\pgfpathlineto{\pgfqpoint{3.018559in}{2.145812in}}%
\pgfpathlineto{\pgfqpoint{3.004810in}{2.157558in}}%
\pgfpathlineto{\pgfqpoint{2.991060in}{2.169441in}}%
\pgfpathlineto{\pgfqpoint{2.977309in}{2.181462in}}%
\pgfpathlineto{\pgfqpoint{2.985894in}{2.180323in}}%
\pgfpathlineto{\pgfqpoint{2.994465in}{2.179428in}}%
\pgfpathlineto{\pgfqpoint{3.003022in}{2.178772in}}%
\pgfpathlineto{\pgfqpoint{3.011565in}{2.178351in}}%
\pgfpathclose%
\pgfusepath{fill}%
\end{pgfscope}%
\begin{pgfscope}%
\pgfpathrectangle{\pgfqpoint{1.150000in}{0.150000in}}{\pgfqpoint{5.700000in}{5.700000in}}%
\pgfusepath{clip}%
\pgfsetbuttcap%
\pgfsetroundjoin%
\definecolor{currentfill}{rgb}{0.266580,0.228262,0.514349}%
\pgfsetfillcolor{currentfill}%
\pgfsetfillopacity{0.700000}%
\pgfsetlinewidth{0.000000pt}%
\definecolor{currentstroke}{rgb}{0.000000,0.000000,0.000000}%
\pgfsetstrokecolor{currentstroke}%
\pgfsetdash{}{0pt}%
\pgfpathmoveto{\pgfqpoint{4.814736in}{2.302267in}}%
\pgfpathlineto{\pgfqpoint{4.828819in}{2.305981in}}%
\pgfpathlineto{\pgfqpoint{4.842914in}{2.309795in}}%
\pgfpathlineto{\pgfqpoint{4.857021in}{2.313709in}}%
\pgfpathlineto{\pgfqpoint{4.871140in}{2.317722in}}%
\pgfpathlineto{\pgfqpoint{4.863402in}{2.306298in}}%
\pgfpathlineto{\pgfqpoint{4.855658in}{2.294815in}}%
\pgfpathlineto{\pgfqpoint{4.847909in}{2.283276in}}%
\pgfpathlineto{\pgfqpoint{4.840154in}{2.271682in}}%
\pgfpathlineto{\pgfqpoint{4.826032in}{2.267845in}}%
\pgfpathlineto{\pgfqpoint{4.811921in}{2.264108in}}%
\pgfpathlineto{\pgfqpoint{4.797823in}{2.260470in}}%
\pgfpathlineto{\pgfqpoint{4.783736in}{2.256933in}}%
\pgfpathlineto{\pgfqpoint{4.791494in}{2.268344in}}%
\pgfpathlineto{\pgfqpoint{4.799247in}{2.279704in}}%
\pgfpathlineto{\pgfqpoint{4.806994in}{2.291012in}}%
\pgfpathlineto{\pgfqpoint{4.814736in}{2.302267in}}%
\pgfpathclose%
\pgfusepath{fill}%
\end{pgfscope}%
\begin{pgfscope}%
\pgfpathrectangle{\pgfqpoint{1.150000in}{0.150000in}}{\pgfqpoint{5.700000in}{5.700000in}}%
\pgfusepath{clip}%
\pgfsetbuttcap%
\pgfsetroundjoin%
\definecolor{currentfill}{rgb}{0.268510,0.009605,0.335427}%
\pgfsetfillcolor{currentfill}%
\pgfsetfillopacity{0.700000}%
\pgfsetlinewidth{0.000000pt}%
\definecolor{currentstroke}{rgb}{0.000000,0.000000,0.000000}%
\pgfsetstrokecolor{currentstroke}%
\pgfsetdash{}{0pt}%
\pgfpathmoveto{\pgfqpoint{3.659266in}{1.876650in}}%
\pgfpathlineto{\pgfqpoint{3.672988in}{1.871245in}}%
\pgfpathlineto{\pgfqpoint{3.686715in}{1.865951in}}%
\pgfpathlineto{\pgfqpoint{3.700447in}{1.860769in}}%
\pgfpathlineto{\pgfqpoint{3.714184in}{1.855696in}}%
\pgfpathlineto{\pgfqpoint{3.706056in}{1.849383in}}%
\pgfpathlineto{\pgfqpoint{3.697920in}{1.843200in}}%
\pgfpathlineto{\pgfqpoint{3.689777in}{1.837152in}}%
\pgfpathlineto{\pgfqpoint{3.681626in}{1.831241in}}%
\pgfpathlineto{\pgfqpoint{3.667870in}{1.836706in}}%
\pgfpathlineto{\pgfqpoint{3.654119in}{1.842281in}}%
\pgfpathlineto{\pgfqpoint{3.640373in}{1.847967in}}%
\pgfpathlineto{\pgfqpoint{3.626631in}{1.853764in}}%
\pgfpathlineto{\pgfqpoint{3.634802in}{1.859275in}}%
\pgfpathlineto{\pgfqpoint{3.642965in}{1.864929in}}%
\pgfpathlineto{\pgfqpoint{3.651119in}{1.870722in}}%
\pgfpathlineto{\pgfqpoint{3.659266in}{1.876650in}}%
\pgfpathclose%
\pgfusepath{fill}%
\end{pgfscope}%
\begin{pgfscope}%
\pgfpathrectangle{\pgfqpoint{1.150000in}{0.150000in}}{\pgfqpoint{5.700000in}{5.700000in}}%
\pgfusepath{clip}%
\pgfsetbuttcap%
\pgfsetroundjoin%
\definecolor{currentfill}{rgb}{0.214298,0.355619,0.551184}%
\pgfsetfillcolor{currentfill}%
\pgfsetfillopacity{0.700000}%
\pgfsetlinewidth{0.000000pt}%
\definecolor{currentstroke}{rgb}{0.000000,0.000000,0.000000}%
\pgfsetstrokecolor{currentstroke}%
\pgfsetdash{}{0pt}%
\pgfpathmoveto{\pgfqpoint{5.194859in}{2.595380in}}%
\pgfpathlineto{\pgfqpoint{5.209128in}{2.601298in}}%
\pgfpathlineto{\pgfqpoint{5.223410in}{2.607317in}}%
\pgfpathlineto{\pgfqpoint{5.237707in}{2.613436in}}%
\pgfpathlineto{\pgfqpoint{5.252018in}{2.619654in}}%
\pgfpathlineto{\pgfqpoint{5.244419in}{2.609323in}}%
\pgfpathlineto{\pgfqpoint{5.236812in}{2.598896in}}%
\pgfpathlineto{\pgfqpoint{5.229199in}{2.588372in}}%
\pgfpathlineto{\pgfqpoint{5.221580in}{2.577753in}}%
\pgfpathlineto{\pgfqpoint{5.207265in}{2.571618in}}%
\pgfpathlineto{\pgfqpoint{5.192964in}{2.565583in}}%
\pgfpathlineto{\pgfqpoint{5.178676in}{2.559648in}}%
\pgfpathlineto{\pgfqpoint{5.164403in}{2.553812in}}%
\pgfpathlineto{\pgfqpoint{5.172027in}{2.564341in}}%
\pgfpathlineto{\pgfqpoint{5.179644in}{2.574779in}}%
\pgfpathlineto{\pgfqpoint{5.187255in}{2.585125in}}%
\pgfpathlineto{\pgfqpoint{5.194859in}{2.595380in}}%
\pgfpathclose%
\pgfusepath{fill}%
\end{pgfscope}%
\begin{pgfscope}%
\pgfpathrectangle{\pgfqpoint{1.150000in}{0.150000in}}{\pgfqpoint{5.700000in}{5.700000in}}%
\pgfusepath{clip}%
\pgfsetbuttcap%
\pgfsetroundjoin%
\definecolor{currentfill}{rgb}{0.269944,0.014625,0.341379}%
\pgfsetfillcolor{currentfill}%
\pgfsetfillopacity{0.700000}%
\pgfsetlinewidth{0.000000pt}%
\definecolor{currentstroke}{rgb}{0.000000,0.000000,0.000000}%
\pgfsetstrokecolor{currentstroke}%
\pgfsetdash{}{0pt}%
\pgfpathmoveto{\pgfqpoint{4.030980in}{1.890929in}}%
\pgfpathlineto{\pgfqpoint{4.044776in}{1.888747in}}%
\pgfpathlineto{\pgfqpoint{4.058579in}{1.886668in}}%
\pgfpathlineto{\pgfqpoint{4.072389in}{1.884695in}}%
\pgfpathlineto{\pgfqpoint{4.086207in}{1.882825in}}%
\pgfpathlineto{\pgfqpoint{4.078229in}{1.873575in}}%
\pgfpathlineto{\pgfqpoint{4.070246in}{1.864390in}}%
\pgfpathlineto{\pgfqpoint{4.062256in}{1.855275in}}%
\pgfpathlineto{\pgfqpoint{4.054261in}{1.846232in}}%
\pgfpathlineto{\pgfqpoint{4.040432in}{1.848438in}}%
\pgfpathlineto{\pgfqpoint{4.026610in}{1.850749in}}%
\pgfpathlineto{\pgfqpoint{4.012795in}{1.853164in}}%
\pgfpathlineto{\pgfqpoint{3.998987in}{1.855684in}}%
\pgfpathlineto{\pgfqpoint{4.006994in}{1.864383in}}%
\pgfpathlineto{\pgfqpoint{4.014995in}{1.873159in}}%
\pgfpathlineto{\pgfqpoint{4.022991in}{1.882009in}}%
\pgfpathlineto{\pgfqpoint{4.030980in}{1.890929in}}%
\pgfpathclose%
\pgfusepath{fill}%
\end{pgfscope}%
\begin{pgfscope}%
\pgfpathrectangle{\pgfqpoint{1.150000in}{0.150000in}}{\pgfqpoint{5.700000in}{5.700000in}}%
\pgfusepath{clip}%
\pgfsetbuttcap%
\pgfsetroundjoin%
\definecolor{currentfill}{rgb}{0.267004,0.004874,0.329415}%
\pgfsetfillcolor{currentfill}%
\pgfsetfillopacity{0.700000}%
\pgfsetlinewidth{0.000000pt}%
\definecolor{currentstroke}{rgb}{0.000000,0.000000,0.000000}%
\pgfsetstrokecolor{currentstroke}%
\pgfsetdash{}{0pt}%
\pgfpathmoveto{\pgfqpoint{3.801550in}{1.864482in}}%
\pgfpathlineto{\pgfqpoint{3.815297in}{1.860329in}}%
\pgfpathlineto{\pgfqpoint{3.829048in}{1.856285in}}%
\pgfpathlineto{\pgfqpoint{3.842806in}{1.852349in}}%
\pgfpathlineto{\pgfqpoint{3.856569in}{1.848520in}}%
\pgfpathlineto{\pgfqpoint{3.848503in}{1.840989in}}%
\pgfpathlineto{\pgfqpoint{3.840431in}{1.833564in}}%
\pgfpathlineto{\pgfqpoint{3.832351in}{1.826250in}}%
\pgfpathlineto{\pgfqpoint{3.824265in}{1.819049in}}%
\pgfpathlineto{\pgfqpoint{3.810486in}{1.823251in}}%
\pgfpathlineto{\pgfqpoint{3.796712in}{1.827561in}}%
\pgfpathlineto{\pgfqpoint{3.782944in}{1.831978in}}%
\pgfpathlineto{\pgfqpoint{3.769182in}{1.836504in}}%
\pgfpathlineto{\pgfqpoint{3.777284in}{1.843324in}}%
\pgfpathlineto{\pgfqpoint{3.785380in}{1.850263in}}%
\pgfpathlineto{\pgfqpoint{3.793469in}{1.857317in}}%
\pgfpathlineto{\pgfqpoint{3.801550in}{1.864482in}}%
\pgfpathclose%
\pgfusepath{fill}%
\end{pgfscope}%
\begin{pgfscope}%
\pgfpathrectangle{\pgfqpoint{1.150000in}{0.150000in}}{\pgfqpoint{5.700000in}{5.700000in}}%
\pgfusepath{clip}%
\pgfsetbuttcap%
\pgfsetroundjoin%
\definecolor{currentfill}{rgb}{0.163625,0.471133,0.558148}%
\pgfsetfillcolor{currentfill}%
\pgfsetfillopacity{0.700000}%
\pgfsetlinewidth{0.000000pt}%
\definecolor{currentstroke}{rgb}{0.000000,0.000000,0.000000}%
\pgfsetstrokecolor{currentstroke}%
\pgfsetdash{}{0pt}%
\pgfpathmoveto{\pgfqpoint{5.574958in}{2.889629in}}%
\pgfpathlineto{\pgfqpoint{5.589429in}{2.897278in}}%
\pgfpathlineto{\pgfqpoint{5.603915in}{2.905028in}}%
\pgfpathlineto{\pgfqpoint{5.618418in}{2.912879in}}%
\pgfpathlineto{\pgfqpoint{5.632935in}{2.920829in}}%
\pgfpathlineto{\pgfqpoint{5.625517in}{2.912542in}}%
\pgfpathlineto{\pgfqpoint{5.618090in}{2.904140in}}%
\pgfpathlineto{\pgfqpoint{5.610655in}{2.895620in}}%
\pgfpathlineto{\pgfqpoint{5.603211in}{2.886985in}}%
\pgfpathlineto{\pgfqpoint{5.588685in}{2.879021in}}%
\pgfpathlineto{\pgfqpoint{5.574175in}{2.871157in}}%
\pgfpathlineto{\pgfqpoint{5.559680in}{2.863394in}}%
\pgfpathlineto{\pgfqpoint{5.545201in}{2.855731in}}%
\pgfpathlineto{\pgfqpoint{5.552653in}{2.864373in}}%
\pgfpathlineto{\pgfqpoint{5.560096in}{2.872903in}}%
\pgfpathlineto{\pgfqpoint{5.567531in}{2.881321in}}%
\pgfpathlineto{\pgfqpoint{5.574958in}{2.889629in}}%
\pgfpathclose%
\pgfusepath{fill}%
\end{pgfscope}%
\begin{pgfscope}%
\pgfpathrectangle{\pgfqpoint{1.150000in}{0.150000in}}{\pgfqpoint{5.700000in}{5.700000in}}%
\pgfusepath{clip}%
\pgfsetbuttcap%
\pgfsetroundjoin%
\definecolor{currentfill}{rgb}{0.279566,0.067836,0.391917}%
\pgfsetfillcolor{currentfill}%
\pgfsetfillopacity{0.700000}%
\pgfsetlinewidth{0.000000pt}%
\definecolor{currentstroke}{rgb}{0.000000,0.000000,0.000000}%
\pgfsetstrokecolor{currentstroke}%
\pgfsetdash{}{0pt}%
\pgfpathmoveto{\pgfqpoint{3.319329in}{1.981522in}}%
\pgfpathlineto{\pgfqpoint{3.333029in}{1.972982in}}%
\pgfpathlineto{\pgfqpoint{3.346732in}{1.964565in}}%
\pgfpathlineto{\pgfqpoint{3.360436in}{1.956269in}}%
\pgfpathlineto{\pgfqpoint{3.374144in}{1.948093in}}%
\pgfpathlineto{\pgfqpoint{3.365833in}{1.945052in}}%
\pgfpathlineto{\pgfqpoint{3.357513in}{1.942198in}}%
\pgfpathlineto{\pgfqpoint{3.349182in}{1.939535in}}%
\pgfpathlineto{\pgfqpoint{3.340840in}{1.937067in}}%
\pgfpathlineto{\pgfqpoint{3.327105in}{1.945676in}}%
\pgfpathlineto{\pgfqpoint{3.313373in}{1.954405in}}%
\pgfpathlineto{\pgfqpoint{3.299643in}{1.963255in}}%
\pgfpathlineto{\pgfqpoint{3.285914in}{1.972228in}}%
\pgfpathlineto{\pgfqpoint{3.294285in}{1.974255in}}%
\pgfpathlineto{\pgfqpoint{3.302644in}{1.976483in}}%
\pgfpathlineto{\pgfqpoint{3.310992in}{1.978907in}}%
\pgfpathlineto{\pgfqpoint{3.319329in}{1.981522in}}%
\pgfpathclose%
\pgfusepath{fill}%
\end{pgfscope}%
\begin{pgfscope}%
\pgfpathrectangle{\pgfqpoint{1.150000in}{0.150000in}}{\pgfqpoint{5.700000in}{5.700000in}}%
\pgfusepath{clip}%
\pgfsetbuttcap%
\pgfsetroundjoin%
\definecolor{currentfill}{rgb}{0.272594,0.025563,0.353093}%
\pgfsetfillcolor{currentfill}%
\pgfsetfillopacity{0.700000}%
\pgfsetlinewidth{0.000000pt}%
\definecolor{currentstroke}{rgb}{0.000000,0.000000,0.000000}%
\pgfsetstrokecolor{currentstroke}%
\pgfsetdash{}{0pt}%
\pgfpathmoveto{\pgfqpoint{3.516843in}{1.904203in}}%
\pgfpathlineto{\pgfqpoint{3.530553in}{1.897499in}}%
\pgfpathlineto{\pgfqpoint{3.544267in}{1.890910in}}%
\pgfpathlineto{\pgfqpoint{3.557984in}{1.884435in}}%
\pgfpathlineto{\pgfqpoint{3.571705in}{1.878075in}}%
\pgfpathlineto{\pgfqpoint{3.563505in}{1.873115in}}%
\pgfpathlineto{\pgfqpoint{3.555296in}{1.868310in}}%
\pgfpathlineto{\pgfqpoint{3.547078in}{1.863665in}}%
\pgfpathlineto{\pgfqpoint{3.538851in}{1.859184in}}%
\pgfpathlineto{\pgfqpoint{3.525108in}{1.865956in}}%
\pgfpathlineto{\pgfqpoint{3.511368in}{1.872842in}}%
\pgfpathlineto{\pgfqpoint{3.497631in}{1.879843in}}%
\pgfpathlineto{\pgfqpoint{3.483898in}{1.886959in}}%
\pgfpathlineto{\pgfqpoint{3.492149in}{1.891021in}}%
\pgfpathlineto{\pgfqpoint{3.500389in}{1.895252in}}%
\pgfpathlineto{\pgfqpoint{3.508621in}{1.899648in}}%
\pgfpathlineto{\pgfqpoint{3.516843in}{1.904203in}}%
\pgfpathclose%
\pgfusepath{fill}%
\end{pgfscope}%
\begin{pgfscope}%
\pgfpathrectangle{\pgfqpoint{1.150000in}{0.150000in}}{\pgfqpoint{5.700000in}{5.700000in}}%
\pgfusepath{clip}%
\pgfsetbuttcap%
\pgfsetroundjoin%
\definecolor{currentfill}{rgb}{0.282290,0.145912,0.461510}%
\pgfsetfillcolor{currentfill}%
\pgfsetfillopacity{0.700000}%
\pgfsetlinewidth{0.000000pt}%
\definecolor{currentstroke}{rgb}{0.000000,0.000000,0.000000}%
\pgfsetstrokecolor{currentstroke}%
\pgfsetdash{}{0pt}%
\pgfpathmoveto{\pgfqpoint{3.066422in}{2.132928in}}%
\pgfpathlineto{\pgfqpoint{3.080136in}{2.121910in}}%
\pgfpathlineto{\pgfqpoint{3.093850in}{2.111026in}}%
\pgfpathlineto{\pgfqpoint{3.107564in}{2.100274in}}%
\pgfpathlineto{\pgfqpoint{3.121278in}{2.089654in}}%
\pgfpathlineto{\pgfqpoint{3.112804in}{2.089174in}}%
\pgfpathlineto{\pgfqpoint{3.104316in}{2.088917in}}%
\pgfpathlineto{\pgfqpoint{3.095815in}{2.088890in}}%
\pgfpathlineto{\pgfqpoint{3.087301in}{2.089096in}}%
\pgfpathlineto{\pgfqpoint{3.073553in}{2.100173in}}%
\pgfpathlineto{\pgfqpoint{3.059805in}{2.111382in}}%
\pgfpathlineto{\pgfqpoint{3.046057in}{2.122725in}}%
\pgfpathlineto{\pgfqpoint{3.032308in}{2.134201in}}%
\pgfpathlineto{\pgfqpoint{3.040857in}{2.133530in}}%
\pgfpathlineto{\pgfqpoint{3.049393in}{2.133097in}}%
\pgfpathlineto{\pgfqpoint{3.057914in}{2.132898in}}%
\pgfpathlineto{\pgfqpoint{3.066422in}{2.132928in}}%
\pgfpathclose%
\pgfusepath{fill}%
\end{pgfscope}%
\begin{pgfscope}%
\pgfpathrectangle{\pgfqpoint{1.150000in}{0.150000in}}{\pgfqpoint{5.700000in}{5.700000in}}%
\pgfusepath{clip}%
\pgfsetbuttcap%
\pgfsetroundjoin%
\definecolor{currentfill}{rgb}{0.255645,0.260703,0.528312}%
\pgfsetfillcolor{currentfill}%
\pgfsetfillopacity{0.700000}%
\pgfsetlinewidth{0.000000pt}%
\definecolor{currentstroke}{rgb}{0.000000,0.000000,0.000000}%
\pgfsetstrokecolor{currentstroke}%
\pgfsetdash{}{0pt}%
\pgfpathmoveto{\pgfqpoint{4.902036in}{2.362812in}}%
\pgfpathlineto{\pgfqpoint{4.916164in}{2.367084in}}%
\pgfpathlineto{\pgfqpoint{4.930304in}{2.371456in}}%
\pgfpathlineto{\pgfqpoint{4.944457in}{2.375928in}}%
\pgfpathlineto{\pgfqpoint{4.958623in}{2.380500in}}%
\pgfpathlineto{\pgfqpoint{4.950910in}{2.369170in}}%
\pgfpathlineto{\pgfqpoint{4.943192in}{2.357773in}}%
\pgfpathlineto{\pgfqpoint{4.935469in}{2.346308in}}%
\pgfpathlineto{\pgfqpoint{4.927739in}{2.334777in}}%
\pgfpathlineto{\pgfqpoint{4.913571in}{2.330364in}}%
\pgfpathlineto{\pgfqpoint{4.899415in}{2.326050in}}%
\pgfpathlineto{\pgfqpoint{4.885271in}{2.321836in}}%
\pgfpathlineto{\pgfqpoint{4.871140in}{2.317722in}}%
\pgfpathlineto{\pgfqpoint{4.878872in}{2.329087in}}%
\pgfpathlineto{\pgfqpoint{4.886599in}{2.340391in}}%
\pgfpathlineto{\pgfqpoint{4.894321in}{2.351633in}}%
\pgfpathlineto{\pgfqpoint{4.902036in}{2.362812in}}%
\pgfpathclose%
\pgfusepath{fill}%
\end{pgfscope}%
\begin{pgfscope}%
\pgfpathrectangle{\pgfqpoint{1.150000in}{0.150000in}}{\pgfqpoint{5.700000in}{5.700000in}}%
\pgfusepath{clip}%
\pgfsetbuttcap%
\pgfsetroundjoin%
\definecolor{currentfill}{rgb}{0.201239,0.383670,0.554294}%
\pgfsetfillcolor{currentfill}%
\pgfsetfillopacity{0.700000}%
\pgfsetlinewidth{0.000000pt}%
\definecolor{currentstroke}{rgb}{0.000000,0.000000,0.000000}%
\pgfsetstrokecolor{currentstroke}%
\pgfsetdash{}{0pt}%
\pgfpathmoveto{\pgfqpoint{5.282344in}{2.660005in}}%
\pgfpathlineto{\pgfqpoint{5.296664in}{2.666388in}}%
\pgfpathlineto{\pgfqpoint{5.310998in}{2.672871in}}%
\pgfpathlineto{\pgfqpoint{5.325347in}{2.679455in}}%
\pgfpathlineto{\pgfqpoint{5.339710in}{2.686138in}}%
\pgfpathlineto{\pgfqpoint{5.332144in}{2.676140in}}%
\pgfpathlineto{\pgfqpoint{5.324571in}{2.666039in}}%
\pgfpathlineto{\pgfqpoint{5.316991in}{2.655836in}}%
\pgfpathlineto{\pgfqpoint{5.309403in}{2.645530in}}%
\pgfpathlineto{\pgfqpoint{5.295035in}{2.638911in}}%
\pgfpathlineto{\pgfqpoint{5.280682in}{2.632392in}}%
\pgfpathlineto{\pgfqpoint{5.266343in}{2.625973in}}%
\pgfpathlineto{\pgfqpoint{5.252018in}{2.619654in}}%
\pgfpathlineto{\pgfqpoint{5.259610in}{2.629889in}}%
\pgfpathlineto{\pgfqpoint{5.267195in}{2.640025in}}%
\pgfpathlineto{\pgfqpoint{5.274773in}{2.650064in}}%
\pgfpathlineto{\pgfqpoint{5.282344in}{2.660005in}}%
\pgfpathclose%
\pgfusepath{fill}%
\end{pgfscope}%
\begin{pgfscope}%
\pgfpathrectangle{\pgfqpoint{1.150000in}{0.150000in}}{\pgfqpoint{5.700000in}{5.700000in}}%
\pgfusepath{clip}%
\pgfsetbuttcap%
\pgfsetroundjoin%
\definecolor{currentfill}{rgb}{0.153364,0.497000,0.557724}%
\pgfsetfillcolor{currentfill}%
\pgfsetfillopacity{0.700000}%
\pgfsetlinewidth{0.000000pt}%
\definecolor{currentstroke}{rgb}{0.000000,0.000000,0.000000}%
\pgfsetstrokecolor{currentstroke}%
\pgfsetdash{}{0pt}%
\pgfpathmoveto{\pgfqpoint{5.662522in}{2.952829in}}%
\pgfpathlineto{\pgfqpoint{5.677047in}{2.960847in}}%
\pgfpathlineto{\pgfqpoint{5.691587in}{2.968965in}}%
\pgfpathlineto{\pgfqpoint{5.706144in}{2.977184in}}%
\pgfpathlineto{\pgfqpoint{5.720716in}{2.985504in}}%
\pgfpathlineto{\pgfqpoint{5.713342in}{2.977715in}}%
\pgfpathlineto{\pgfqpoint{5.705959in}{2.969808in}}%
\pgfpathlineto{\pgfqpoint{5.698567in}{2.961782in}}%
\pgfpathlineto{\pgfqpoint{5.691166in}{2.953637in}}%
\pgfpathlineto{\pgfqpoint{5.676584in}{2.945284in}}%
\pgfpathlineto{\pgfqpoint{5.662019in}{2.937032in}}%
\pgfpathlineto{\pgfqpoint{5.647469in}{2.928880in}}%
\pgfpathlineto{\pgfqpoint{5.632935in}{2.920829in}}%
\pgfpathlineto{\pgfqpoint{5.640345in}{2.929001in}}%
\pgfpathlineto{\pgfqpoint{5.647746in}{2.937057in}}%
\pgfpathlineto{\pgfqpoint{5.655139in}{2.945000in}}%
\pgfpathlineto{\pgfqpoint{5.662522in}{2.952829in}}%
\pgfpathclose%
\pgfusepath{fill}%
\end{pgfscope}%
\begin{pgfscope}%
\pgfpathrectangle{\pgfqpoint{1.150000in}{0.150000in}}{\pgfqpoint{5.700000in}{5.700000in}}%
\pgfusepath{clip}%
\pgfsetbuttcap%
\pgfsetroundjoin%
\definecolor{currentfill}{rgb}{0.268510,0.009605,0.335427}%
\pgfsetfillcolor{currentfill}%
\pgfsetfillopacity{0.700000}%
\pgfsetlinewidth{0.000000pt}%
\definecolor{currentstroke}{rgb}{0.000000,0.000000,0.000000}%
\pgfsetstrokecolor{currentstroke}%
\pgfsetdash{}{0pt}%
\pgfpathmoveto{\pgfqpoint{3.943825in}{1.866814in}}%
\pgfpathlineto{\pgfqpoint{3.957605in}{1.863874in}}%
\pgfpathlineto{\pgfqpoint{3.971393in}{1.861039in}}%
\pgfpathlineto{\pgfqpoint{3.985187in}{1.858309in}}%
\pgfpathlineto{\pgfqpoint{3.998987in}{1.855684in}}%
\pgfpathlineto{\pgfqpoint{3.990975in}{1.847065in}}%
\pgfpathlineto{\pgfqpoint{3.982956in}{1.838530in}}%
\pgfpathlineto{\pgfqpoint{3.974931in}{1.830081in}}%
\pgfpathlineto{\pgfqpoint{3.966900in}{1.821723in}}%
\pgfpathlineto{\pgfqpoint{3.953086in}{1.824703in}}%
\pgfpathlineto{\pgfqpoint{3.939279in}{1.827788in}}%
\pgfpathlineto{\pgfqpoint{3.925478in}{1.830978in}}%
\pgfpathlineto{\pgfqpoint{3.911684in}{1.834274in}}%
\pgfpathlineto{\pgfqpoint{3.919728in}{1.842270in}}%
\pgfpathlineto{\pgfqpoint{3.927767in}{1.850361in}}%
\pgfpathlineto{\pgfqpoint{3.935799in}{1.858544in}}%
\pgfpathlineto{\pgfqpoint{3.943825in}{1.866814in}}%
\pgfpathclose%
\pgfusepath{fill}%
\end{pgfscope}%
\begin{pgfscope}%
\pgfpathrectangle{\pgfqpoint{1.150000in}{0.150000in}}{\pgfqpoint{5.700000in}{5.700000in}}%
\pgfusepath{clip}%
\pgfsetbuttcap%
\pgfsetroundjoin%
\definecolor{currentfill}{rgb}{0.282910,0.105393,0.426902}%
\pgfsetfillcolor{currentfill}%
\pgfsetfillopacity{0.700000}%
\pgfsetlinewidth{0.000000pt}%
\definecolor{currentstroke}{rgb}{0.000000,0.000000,0.000000}%
\pgfsetstrokecolor{currentstroke}%
\pgfsetdash{}{0pt}%
\pgfpathmoveto{\pgfqpoint{4.434768in}{2.040188in}}%
\pgfpathlineto{\pgfqpoint{4.448700in}{2.041230in}}%
\pgfpathlineto{\pgfqpoint{4.462642in}{2.042374in}}%
\pgfpathlineto{\pgfqpoint{4.476593in}{2.043618in}}%
\pgfpathlineto{\pgfqpoint{4.490555in}{2.044964in}}%
\pgfpathlineto{\pgfqpoint{4.482698in}{2.033757in}}%
\pgfpathlineto{\pgfqpoint{4.474836in}{2.022550in}}%
\pgfpathlineto{\pgfqpoint{4.466970in}{2.011344in}}%
\pgfpathlineto{\pgfqpoint{4.459098in}{2.000141in}}%
\pgfpathlineto{\pgfqpoint{4.445131in}{1.999063in}}%
\pgfpathlineto{\pgfqpoint{4.431174in}{1.998085in}}%
\pgfpathlineto{\pgfqpoint{4.417226in}{1.997208in}}%
\pgfpathlineto{\pgfqpoint{4.403288in}{1.996432in}}%
\pgfpathlineto{\pgfqpoint{4.411166in}{2.007361in}}%
\pgfpathlineto{\pgfqpoint{4.419038in}{2.018298in}}%
\pgfpathlineto{\pgfqpoint{4.426905in}{2.029241in}}%
\pgfpathlineto{\pgfqpoint{4.434768in}{2.040188in}}%
\pgfpathclose%
\pgfusepath{fill}%
\end{pgfscope}%
\begin{pgfscope}%
\pgfpathrectangle{\pgfqpoint{1.150000in}{0.150000in}}{\pgfqpoint{5.700000in}{5.700000in}}%
\pgfusepath{clip}%
\pgfsetbuttcap%
\pgfsetroundjoin%
\definecolor{currentfill}{rgb}{0.280894,0.078907,0.402329}%
\pgfsetfillcolor{currentfill}%
\pgfsetfillopacity{0.700000}%
\pgfsetlinewidth{0.000000pt}%
\definecolor{currentstroke}{rgb}{0.000000,0.000000,0.000000}%
\pgfsetstrokecolor{currentstroke}%
\pgfsetdash{}{0pt}%
\pgfpathmoveto{\pgfqpoint{4.347631in}{1.994341in}}%
\pgfpathlineto{\pgfqpoint{4.361531in}{1.994711in}}%
\pgfpathlineto{\pgfqpoint{4.375441in}{1.995184in}}%
\pgfpathlineto{\pgfqpoint{4.389360in}{1.995757in}}%
\pgfpathlineto{\pgfqpoint{4.403288in}{1.996432in}}%
\pgfpathlineto{\pgfqpoint{4.395406in}{1.985514in}}%
\pgfpathlineto{\pgfqpoint{4.387519in}{1.974609in}}%
\pgfpathlineto{\pgfqpoint{4.379627in}{1.963720in}}%
\pgfpathlineto{\pgfqpoint{4.371730in}{1.952850in}}%
\pgfpathlineto{\pgfqpoint{4.357795in}{1.952460in}}%
\pgfpathlineto{\pgfqpoint{4.343869in}{1.952170in}}%
\pgfpathlineto{\pgfqpoint{4.329953in}{1.951982in}}%
\pgfpathlineto{\pgfqpoint{4.316046in}{1.951896in}}%
\pgfpathlineto{\pgfqpoint{4.323949in}{1.962475in}}%
\pgfpathlineto{\pgfqpoint{4.331848in}{1.973077in}}%
\pgfpathlineto{\pgfqpoint{4.339742in}{1.983700in}}%
\pgfpathlineto{\pgfqpoint{4.347631in}{1.994341in}}%
\pgfpathclose%
\pgfusepath{fill}%
\end{pgfscope}%
\begin{pgfscope}%
\pgfpathrectangle{\pgfqpoint{1.150000in}{0.150000in}}{\pgfqpoint{5.700000in}{5.700000in}}%
\pgfusepath{clip}%
\pgfsetbuttcap%
\pgfsetroundjoin%
\definecolor{currentfill}{rgb}{0.244972,0.287675,0.537260}%
\pgfsetfillcolor{currentfill}%
\pgfsetfillopacity{0.700000}%
\pgfsetlinewidth{0.000000pt}%
\definecolor{currentstroke}{rgb}{0.000000,0.000000,0.000000}%
\pgfsetstrokecolor{currentstroke}%
\pgfsetdash{}{0pt}%
\pgfpathmoveto{\pgfqpoint{4.989413in}{2.425112in}}%
\pgfpathlineto{\pgfqpoint{5.003587in}{2.429924in}}%
\pgfpathlineto{\pgfqpoint{5.017775in}{2.434836in}}%
\pgfpathlineto{\pgfqpoint{5.031975in}{2.439848in}}%
\pgfpathlineto{\pgfqpoint{5.046189in}{2.444959in}}%
\pgfpathlineto{\pgfqpoint{5.038504in}{2.433782in}}%
\pgfpathlineto{\pgfqpoint{5.030813in}{2.422526in}}%
\pgfpathlineto{\pgfqpoint{5.023115in}{2.411194in}}%
\pgfpathlineto{\pgfqpoint{5.015412in}{2.399787in}}%
\pgfpathlineto{\pgfqpoint{5.001195in}{2.394816in}}%
\pgfpathlineto{\pgfqpoint{4.986992in}{2.389944in}}%
\pgfpathlineto{\pgfqpoint{4.972801in}{2.385172in}}%
\pgfpathlineto{\pgfqpoint{4.958623in}{2.380500in}}%
\pgfpathlineto{\pgfqpoint{4.966329in}{2.391760in}}%
\pgfpathlineto{\pgfqpoint{4.974029in}{2.402950in}}%
\pgfpathlineto{\pgfqpoint{4.981724in}{2.414067in}}%
\pgfpathlineto{\pgfqpoint{4.989413in}{2.425112in}}%
\pgfpathclose%
\pgfusepath{fill}%
\end{pgfscope}%
\begin{pgfscope}%
\pgfpathrectangle{\pgfqpoint{1.150000in}{0.150000in}}{\pgfqpoint{5.700000in}{5.700000in}}%
\pgfusepath{clip}%
\pgfsetbuttcap%
\pgfsetroundjoin%
\definecolor{currentfill}{rgb}{0.283072,0.130895,0.449241}%
\pgfsetfillcolor{currentfill}%
\pgfsetfillopacity{0.700000}%
\pgfsetlinewidth{0.000000pt}%
\definecolor{currentstroke}{rgb}{0.000000,0.000000,0.000000}%
\pgfsetstrokecolor{currentstroke}%
\pgfsetdash{}{0pt}%
\pgfpathmoveto{\pgfqpoint{4.521933in}{2.089730in}}%
\pgfpathlineto{\pgfqpoint{4.535900in}{2.091425in}}%
\pgfpathlineto{\pgfqpoint{4.549877in}{2.093220in}}%
\pgfpathlineto{\pgfqpoint{4.563864in}{2.095116in}}%
\pgfpathlineto{\pgfqpoint{4.577862in}{2.097113in}}%
\pgfpathlineto{\pgfqpoint{4.570030in}{2.085692in}}%
\pgfpathlineto{\pgfqpoint{4.562193in}{2.074257in}}%
\pgfpathlineto{\pgfqpoint{4.554350in}{2.062809in}}%
\pgfpathlineto{\pgfqpoint{4.546503in}{2.051351in}}%
\pgfpathlineto{\pgfqpoint{4.532501in}{2.049603in}}%
\pgfpathlineto{\pgfqpoint{4.518509in}{2.047956in}}%
\pgfpathlineto{\pgfqpoint{4.504527in}{2.046410in}}%
\pgfpathlineto{\pgfqpoint{4.490555in}{2.044964in}}%
\pgfpathlineto{\pgfqpoint{4.498407in}{2.056166in}}%
\pgfpathlineto{\pgfqpoint{4.506254in}{2.067363in}}%
\pgfpathlineto{\pgfqpoint{4.514096in}{2.078552in}}%
\pgfpathlineto{\pgfqpoint{4.521933in}{2.089730in}}%
\pgfpathclose%
\pgfusepath{fill}%
\end{pgfscope}%
\begin{pgfscope}%
\pgfpathrectangle{\pgfqpoint{1.150000in}{0.150000in}}{\pgfqpoint{5.700000in}{5.700000in}}%
\pgfusepath{clip}%
\pgfsetbuttcap%
\pgfsetroundjoin%
\definecolor{currentfill}{rgb}{0.283072,0.130895,0.449241}%
\pgfsetfillcolor{currentfill}%
\pgfsetfillopacity{0.700000}%
\pgfsetlinewidth{0.000000pt}%
\definecolor{currentstroke}{rgb}{0.000000,0.000000,0.000000}%
\pgfsetstrokecolor{currentstroke}%
\pgfsetdash{}{0pt}%
\pgfpathmoveto{\pgfqpoint{3.121278in}{2.089654in}}%
\pgfpathlineto{\pgfqpoint{3.134993in}{2.079165in}}%
\pgfpathlineto{\pgfqpoint{3.148708in}{2.068807in}}%
\pgfpathlineto{\pgfqpoint{3.162424in}{2.058578in}}%
\pgfpathlineto{\pgfqpoint{3.176141in}{2.048478in}}%
\pgfpathlineto{\pgfqpoint{3.167699in}{2.047549in}}%
\pgfpathlineto{\pgfqpoint{3.159244in}{2.046838in}}%
\pgfpathlineto{\pgfqpoint{3.150776in}{2.046352in}}%
\pgfpathlineto{\pgfqpoint{3.142296in}{2.046094in}}%
\pgfpathlineto{\pgfqpoint{3.128546in}{2.056650in}}%
\pgfpathlineto{\pgfqpoint{3.114798in}{2.067335in}}%
\pgfpathlineto{\pgfqpoint{3.101049in}{2.078150in}}%
\pgfpathlineto{\pgfqpoint{3.087301in}{2.089096in}}%
\pgfpathlineto{\pgfqpoint{3.095815in}{2.088890in}}%
\pgfpathlineto{\pgfqpoint{3.104316in}{2.088917in}}%
\pgfpathlineto{\pgfqpoint{3.112804in}{2.089174in}}%
\pgfpathlineto{\pgfqpoint{3.121278in}{2.089654in}}%
\pgfpathclose%
\pgfusepath{fill}%
\end{pgfscope}%
\begin{pgfscope}%
\pgfpathrectangle{\pgfqpoint{1.150000in}{0.150000in}}{\pgfqpoint{5.700000in}{5.700000in}}%
\pgfusepath{clip}%
\pgfsetbuttcap%
\pgfsetroundjoin%
\definecolor{currentfill}{rgb}{0.277941,0.056324,0.381191}%
\pgfsetfillcolor{currentfill}%
\pgfsetfillopacity{0.700000}%
\pgfsetlinewidth{0.000000pt}%
\definecolor{currentstroke}{rgb}{0.000000,0.000000,0.000000}%
\pgfsetstrokecolor{currentstroke}%
\pgfsetdash{}{0pt}%
\pgfpathmoveto{\pgfqpoint{4.260505in}{1.952567in}}%
\pgfpathlineto{\pgfqpoint{4.274377in}{1.952246in}}%
\pgfpathlineto{\pgfqpoint{4.288258in}{1.952028in}}%
\pgfpathlineto{\pgfqpoint{4.302147in}{1.951911in}}%
\pgfpathlineto{\pgfqpoint{4.316046in}{1.951896in}}%
\pgfpathlineto{\pgfqpoint{4.308137in}{1.941342in}}%
\pgfpathlineto{\pgfqpoint{4.300223in}{1.930818in}}%
\pgfpathlineto{\pgfqpoint{4.292304in}{1.920324in}}%
\pgfpathlineto{\pgfqpoint{4.284380in}{1.909865in}}%
\pgfpathlineto{\pgfqpoint{4.270474in}{1.910182in}}%
\pgfpathlineto{\pgfqpoint{4.256577in}{1.910601in}}%
\pgfpathlineto{\pgfqpoint{4.242688in}{1.911122in}}%
\pgfpathlineto{\pgfqpoint{4.228808in}{1.911745in}}%
\pgfpathlineto{\pgfqpoint{4.236740in}{1.921895in}}%
\pgfpathlineto{\pgfqpoint{4.244667in}{1.932084in}}%
\pgfpathlineto{\pgfqpoint{4.252589in}{1.942309in}}%
\pgfpathlineto{\pgfqpoint{4.260505in}{1.952567in}}%
\pgfpathclose%
\pgfusepath{fill}%
\end{pgfscope}%
\begin{pgfscope}%
\pgfpathrectangle{\pgfqpoint{1.150000in}{0.150000in}}{\pgfqpoint{5.700000in}{5.700000in}}%
\pgfusepath{clip}%
\pgfsetbuttcap%
\pgfsetroundjoin%
\definecolor{currentfill}{rgb}{0.277941,0.056324,0.381191}%
\pgfsetfillcolor{currentfill}%
\pgfsetfillopacity{0.700000}%
\pgfsetlinewidth{0.000000pt}%
\definecolor{currentstroke}{rgb}{0.000000,0.000000,0.000000}%
\pgfsetstrokecolor{currentstroke}%
\pgfsetdash{}{0pt}%
\pgfpathmoveto{\pgfqpoint{3.374144in}{1.948093in}}%
\pgfpathlineto{\pgfqpoint{3.387853in}{1.940037in}}%
\pgfpathlineto{\pgfqpoint{3.401565in}{1.932101in}}%
\pgfpathlineto{\pgfqpoint{3.415280in}{1.924284in}}%
\pgfpathlineto{\pgfqpoint{3.428998in}{1.916584in}}%
\pgfpathlineto{\pgfqpoint{3.420714in}{1.913119in}}%
\pgfpathlineto{\pgfqpoint{3.412419in}{1.909835in}}%
\pgfpathlineto{\pgfqpoint{3.404115in}{1.906737in}}%
\pgfpathlineto{\pgfqpoint{3.395800in}{1.903829in}}%
\pgfpathlineto{\pgfqpoint{3.382056in}{1.911961in}}%
\pgfpathlineto{\pgfqpoint{3.368315in}{1.920210in}}%
\pgfpathlineto{\pgfqpoint{3.354576in}{1.928579in}}%
\pgfpathlineto{\pgfqpoint{3.340840in}{1.937067in}}%
\pgfpathlineto{\pgfqpoint{3.349182in}{1.939535in}}%
\pgfpathlineto{\pgfqpoint{3.357513in}{1.942198in}}%
\pgfpathlineto{\pgfqpoint{3.365833in}{1.945052in}}%
\pgfpathlineto{\pgfqpoint{3.374144in}{1.948093in}}%
\pgfpathclose%
\pgfusepath{fill}%
\end{pgfscope}%
\begin{pgfscope}%
\pgfpathrectangle{\pgfqpoint{1.150000in}{0.150000in}}{\pgfqpoint{5.700000in}{5.700000in}}%
\pgfusepath{clip}%
\pgfsetbuttcap%
\pgfsetroundjoin%
\definecolor{currentfill}{rgb}{0.187231,0.414746,0.556547}%
\pgfsetfillcolor{currentfill}%
\pgfsetfillopacity{0.700000}%
\pgfsetlinewidth{0.000000pt}%
\definecolor{currentstroke}{rgb}{0.000000,0.000000,0.000000}%
\pgfsetstrokecolor{currentstroke}%
\pgfsetdash{}{0pt}%
\pgfpathmoveto{\pgfqpoint{5.369899in}{2.725096in}}%
\pgfpathlineto{\pgfqpoint{5.384271in}{2.731925in}}%
\pgfpathlineto{\pgfqpoint{5.398658in}{2.738854in}}%
\pgfpathlineto{\pgfqpoint{5.413060in}{2.745883in}}%
\pgfpathlineto{\pgfqpoint{5.427476in}{2.753012in}}%
\pgfpathlineto{\pgfqpoint{5.419946in}{2.743390in}}%
\pgfpathlineto{\pgfqpoint{5.412408in}{2.733660in}}%
\pgfpathlineto{\pgfqpoint{5.404862in}{2.723821in}}%
\pgfpathlineto{\pgfqpoint{5.397309in}{2.713874in}}%
\pgfpathlineto{\pgfqpoint{5.382887in}{2.706789in}}%
\pgfpathlineto{\pgfqpoint{5.368480in}{2.699805in}}%
\pgfpathlineto{\pgfqpoint{5.354088in}{2.692922in}}%
\pgfpathlineto{\pgfqpoint{5.339710in}{2.686138in}}%
\pgfpathlineto{\pgfqpoint{5.347268in}{2.696033in}}%
\pgfpathlineto{\pgfqpoint{5.354819in}{2.705824in}}%
\pgfpathlineto{\pgfqpoint{5.362363in}{2.715512in}}%
\pgfpathlineto{\pgfqpoint{5.369899in}{2.725096in}}%
\pgfpathclose%
\pgfusepath{fill}%
\end{pgfscope}%
\begin{pgfscope}%
\pgfpathrectangle{\pgfqpoint{1.150000in}{0.150000in}}{\pgfqpoint{5.700000in}{5.700000in}}%
\pgfusepath{clip}%
\pgfsetbuttcap%
\pgfsetroundjoin%
\definecolor{currentfill}{rgb}{0.143343,0.522773,0.556295}%
\pgfsetfillcolor{currentfill}%
\pgfsetfillopacity{0.700000}%
\pgfsetlinewidth{0.000000pt}%
\definecolor{currentstroke}{rgb}{0.000000,0.000000,0.000000}%
\pgfsetstrokecolor{currentstroke}%
\pgfsetdash{}{0pt}%
\pgfpathmoveto{\pgfqpoint{5.750123in}{3.015489in}}%
\pgfpathlineto{\pgfqpoint{5.764701in}{3.023856in}}%
\pgfpathlineto{\pgfqpoint{5.779296in}{3.032324in}}%
\pgfpathlineto{\pgfqpoint{5.793907in}{3.040892in}}%
\pgfpathlineto{\pgfqpoint{5.808535in}{3.049561in}}%
\pgfpathlineto{\pgfqpoint{5.801207in}{3.042298in}}%
\pgfpathlineto{\pgfqpoint{5.793871in}{3.034916in}}%
\pgfpathlineto{\pgfqpoint{5.786525in}{3.027413in}}%
\pgfpathlineto{\pgfqpoint{5.779169in}{3.019788in}}%
\pgfpathlineto{\pgfqpoint{5.764531in}{3.011066in}}%
\pgfpathlineto{\pgfqpoint{5.749910in}{3.002444in}}%
\pgfpathlineto{\pgfqpoint{5.735305in}{2.993924in}}%
\pgfpathlineto{\pgfqpoint{5.720716in}{2.985504in}}%
\pgfpathlineto{\pgfqpoint{5.728081in}{2.993175in}}%
\pgfpathlineto{\pgfqpoint{5.735438in}{3.000729in}}%
\pgfpathlineto{\pgfqpoint{5.742785in}{3.008167in}}%
\pgfpathlineto{\pgfqpoint{5.750123in}{3.015489in}}%
\pgfpathclose%
\pgfusepath{fill}%
\end{pgfscope}%
\begin{pgfscope}%
\pgfpathrectangle{\pgfqpoint{1.150000in}{0.150000in}}{\pgfqpoint{5.700000in}{5.700000in}}%
\pgfusepath{clip}%
\pgfsetbuttcap%
\pgfsetroundjoin%
\definecolor{currentfill}{rgb}{0.280868,0.160771,0.472899}%
\pgfsetfillcolor{currentfill}%
\pgfsetfillopacity{0.700000}%
\pgfsetlinewidth{0.000000pt}%
\definecolor{currentstroke}{rgb}{0.000000,0.000000,0.000000}%
\pgfsetstrokecolor{currentstroke}%
\pgfsetdash{}{0pt}%
\pgfpathmoveto{\pgfqpoint{4.609142in}{2.142602in}}%
\pgfpathlineto{\pgfqpoint{4.623146in}{2.144930in}}%
\pgfpathlineto{\pgfqpoint{4.637161in}{2.147358in}}%
\pgfpathlineto{\pgfqpoint{4.651187in}{2.149886in}}%
\pgfpathlineto{\pgfqpoint{4.665224in}{2.152515in}}%
\pgfpathlineto{\pgfqpoint{4.657416in}{2.140952in}}%
\pgfpathlineto{\pgfqpoint{4.649602in}{2.129360in}}%
\pgfpathlineto{\pgfqpoint{4.641784in}{2.117743in}}%
\pgfpathlineto{\pgfqpoint{4.633961in}{2.106101in}}%
\pgfpathlineto{\pgfqpoint{4.619920in}{2.103704in}}%
\pgfpathlineto{\pgfqpoint{4.605890in}{2.101407in}}%
\pgfpathlineto{\pgfqpoint{4.591871in}{2.099210in}}%
\pgfpathlineto{\pgfqpoint{4.577862in}{2.097113in}}%
\pgfpathlineto{\pgfqpoint{4.585690in}{2.108516in}}%
\pgfpathlineto{\pgfqpoint{4.593512in}{2.119900in}}%
\pgfpathlineto{\pgfqpoint{4.601330in}{2.131262in}}%
\pgfpathlineto{\pgfqpoint{4.609142in}{2.142602in}}%
\pgfpathclose%
\pgfusepath{fill}%
\end{pgfscope}%
\begin{pgfscope}%
\pgfpathrectangle{\pgfqpoint{1.150000in}{0.150000in}}{\pgfqpoint{5.700000in}{5.700000in}}%
\pgfusepath{clip}%
\pgfsetbuttcap%
\pgfsetroundjoin%
\definecolor{currentfill}{rgb}{0.267004,0.004874,0.329415}%
\pgfsetfillcolor{currentfill}%
\pgfsetfillopacity{0.700000}%
\pgfsetlinewidth{0.000000pt}%
\definecolor{currentstroke}{rgb}{0.000000,0.000000,0.000000}%
\pgfsetstrokecolor{currentstroke}%
\pgfsetdash{}{0pt}%
\pgfpathmoveto{\pgfqpoint{3.714184in}{1.855696in}}%
\pgfpathlineto{\pgfqpoint{3.727926in}{1.850734in}}%
\pgfpathlineto{\pgfqpoint{3.741672in}{1.845881in}}%
\pgfpathlineto{\pgfqpoint{3.755424in}{1.841138in}}%
\pgfpathlineto{\pgfqpoint{3.769182in}{1.836504in}}%
\pgfpathlineto{\pgfqpoint{3.761071in}{1.829806in}}%
\pgfpathlineto{\pgfqpoint{3.752954in}{1.823233in}}%
\pgfpathlineto{\pgfqpoint{3.744829in}{1.816790in}}%
\pgfpathlineto{\pgfqpoint{3.736697in}{1.810481in}}%
\pgfpathlineto{\pgfqpoint{3.722922in}{1.815507in}}%
\pgfpathlineto{\pgfqpoint{3.709151in}{1.820642in}}%
\pgfpathlineto{\pgfqpoint{3.695386in}{1.825887in}}%
\pgfpathlineto{\pgfqpoint{3.681626in}{1.831241in}}%
\pgfpathlineto{\pgfqpoint{3.689777in}{1.837152in}}%
\pgfpathlineto{\pgfqpoint{3.697920in}{1.843200in}}%
\pgfpathlineto{\pgfqpoint{3.706056in}{1.849383in}}%
\pgfpathlineto{\pgfqpoint{3.714184in}{1.855696in}}%
\pgfpathclose%
\pgfusepath{fill}%
\end{pgfscope}%
\begin{pgfscope}%
\pgfpathrectangle{\pgfqpoint{1.150000in}{0.150000in}}{\pgfqpoint{5.700000in}{5.700000in}}%
\pgfusepath{clip}%
\pgfsetbuttcap%
\pgfsetroundjoin%
\definecolor{currentfill}{rgb}{0.274952,0.037752,0.364543}%
\pgfsetfillcolor{currentfill}%
\pgfsetfillopacity{0.700000}%
\pgfsetlinewidth{0.000000pt}%
\definecolor{currentstroke}{rgb}{0.000000,0.000000,0.000000}%
\pgfsetstrokecolor{currentstroke}%
\pgfsetdash{}{0pt}%
\pgfpathmoveto{\pgfqpoint{4.173372in}{1.915260in}}%
\pgfpathlineto{\pgfqpoint{4.187218in}{1.914227in}}%
\pgfpathlineto{\pgfqpoint{4.201073in}{1.913297in}}%
\pgfpathlineto{\pgfqpoint{4.214937in}{1.912470in}}%
\pgfpathlineto{\pgfqpoint{4.228808in}{1.911745in}}%
\pgfpathlineto{\pgfqpoint{4.220871in}{1.901636in}}%
\pgfpathlineto{\pgfqpoint{4.212929in}{1.891572in}}%
\pgfpathlineto{\pgfqpoint{4.204981in}{1.881555in}}%
\pgfpathlineto{\pgfqpoint{4.197028in}{1.871589in}}%
\pgfpathlineto{\pgfqpoint{4.183148in}{1.872634in}}%
\pgfpathlineto{\pgfqpoint{4.169276in}{1.873781in}}%
\pgfpathlineto{\pgfqpoint{4.155411in}{1.875031in}}%
\pgfpathlineto{\pgfqpoint{4.141555in}{1.876384in}}%
\pgfpathlineto{\pgfqpoint{4.149517in}{1.886023in}}%
\pgfpathlineto{\pgfqpoint{4.157474in}{1.895718in}}%
\pgfpathlineto{\pgfqpoint{4.165425in}{1.905464in}}%
\pgfpathlineto{\pgfqpoint{4.173372in}{1.915260in}}%
\pgfpathclose%
\pgfusepath{fill}%
\end{pgfscope}%
\begin{pgfscope}%
\pgfpathrectangle{\pgfqpoint{1.150000in}{0.150000in}}{\pgfqpoint{5.700000in}{5.700000in}}%
\pgfusepath{clip}%
\pgfsetbuttcap%
\pgfsetroundjoin%
\definecolor{currentfill}{rgb}{0.271305,0.019942,0.347269}%
\pgfsetfillcolor{currentfill}%
\pgfsetfillopacity{0.700000}%
\pgfsetlinewidth{0.000000pt}%
\definecolor{currentstroke}{rgb}{0.000000,0.000000,0.000000}%
\pgfsetstrokecolor{currentstroke}%
\pgfsetdash{}{0pt}%
\pgfpathmoveto{\pgfqpoint{3.571705in}{1.878075in}}%
\pgfpathlineto{\pgfqpoint{3.585431in}{1.871828in}}%
\pgfpathlineto{\pgfqpoint{3.599160in}{1.865694in}}%
\pgfpathlineto{\pgfqpoint{3.612893in}{1.859673in}}%
\pgfpathlineto{\pgfqpoint{3.626631in}{1.853764in}}%
\pgfpathlineto{\pgfqpoint{3.618452in}{1.848399in}}%
\pgfpathlineto{\pgfqpoint{3.610264in}{1.843186in}}%
\pgfpathlineto{\pgfqpoint{3.602068in}{1.838127in}}%
\pgfpathlineto{\pgfqpoint{3.593863in}{1.833227in}}%
\pgfpathlineto{\pgfqpoint{3.580105in}{1.839548in}}%
\pgfpathlineto{\pgfqpoint{3.566350in}{1.845980in}}%
\pgfpathlineto{\pgfqpoint{3.552599in}{1.852525in}}%
\pgfpathlineto{\pgfqpoint{3.538851in}{1.859184in}}%
\pgfpathlineto{\pgfqpoint{3.547078in}{1.863665in}}%
\pgfpathlineto{\pgfqpoint{3.555296in}{1.868310in}}%
\pgfpathlineto{\pgfqpoint{3.563505in}{1.873115in}}%
\pgfpathlineto{\pgfqpoint{3.571705in}{1.878075in}}%
\pgfpathclose%
\pgfusepath{fill}%
\end{pgfscope}%
\begin{pgfscope}%
\pgfpathrectangle{\pgfqpoint{1.150000in}{0.150000in}}{\pgfqpoint{5.700000in}{5.700000in}}%
\pgfusepath{clip}%
\pgfsetbuttcap%
\pgfsetroundjoin%
\definecolor{currentfill}{rgb}{0.231674,0.318106,0.544834}%
\pgfsetfillcolor{currentfill}%
\pgfsetfillopacity{0.700000}%
\pgfsetlinewidth{0.000000pt}%
\definecolor{currentstroke}{rgb}{0.000000,0.000000,0.000000}%
\pgfsetstrokecolor{currentstroke}%
\pgfsetdash{}{0pt}%
\pgfpathmoveto{\pgfqpoint{5.076868in}{2.488874in}}%
\pgfpathlineto{\pgfqpoint{5.091091in}{2.494207in}}%
\pgfpathlineto{\pgfqpoint{5.105328in}{2.499640in}}%
\pgfpathlineto{\pgfqpoint{5.119578in}{2.505172in}}%
\pgfpathlineto{\pgfqpoint{5.133841in}{2.510805in}}%
\pgfpathlineto{\pgfqpoint{5.126184in}{2.499833in}}%
\pgfpathlineto{\pgfqpoint{5.118521in}{2.488776in}}%
\pgfpathlineto{\pgfqpoint{5.110852in}{2.477632in}}%
\pgfpathlineto{\pgfqpoint{5.103176in}{2.466405in}}%
\pgfpathlineto{\pgfqpoint{5.088909in}{2.460894in}}%
\pgfpathlineto{\pgfqpoint{5.074656in}{2.455482in}}%
\pgfpathlineto{\pgfqpoint{5.060416in}{2.450171in}}%
\pgfpathlineto{\pgfqpoint{5.046189in}{2.444959in}}%
\pgfpathlineto{\pgfqpoint{5.053868in}{2.456058in}}%
\pgfpathlineto{\pgfqpoint{5.061541in}{2.467078in}}%
\pgfpathlineto{\pgfqpoint{5.069207in}{2.478017in}}%
\pgfpathlineto{\pgfqpoint{5.076868in}{2.488874in}}%
\pgfpathclose%
\pgfusepath{fill}%
\end{pgfscope}%
\begin{pgfscope}%
\pgfpathrectangle{\pgfqpoint{1.150000in}{0.150000in}}{\pgfqpoint{5.700000in}{5.700000in}}%
\pgfusepath{clip}%
\pgfsetbuttcap%
\pgfsetroundjoin%
\definecolor{currentfill}{rgb}{0.277134,0.185228,0.489898}%
\pgfsetfillcolor{currentfill}%
\pgfsetfillopacity{0.700000}%
\pgfsetlinewidth{0.000000pt}%
\definecolor{currentstroke}{rgb}{0.000000,0.000000,0.000000}%
\pgfsetstrokecolor{currentstroke}%
\pgfsetdash{}{0pt}%
\pgfpathmoveto{\pgfqpoint{4.696406in}{2.198450in}}%
\pgfpathlineto{\pgfqpoint{4.710451in}{2.201392in}}%
\pgfpathlineto{\pgfqpoint{4.724506in}{2.204434in}}%
\pgfpathlineto{\pgfqpoint{4.738573in}{2.207576in}}%
\pgfpathlineto{\pgfqpoint{4.752651in}{2.210818in}}%
\pgfpathlineto{\pgfqpoint{4.744867in}{2.199180in}}%
\pgfpathlineto{\pgfqpoint{4.737078in}{2.187501in}}%
\pgfpathlineto{\pgfqpoint{4.729283in}{2.175784in}}%
\pgfpathlineto{\pgfqpoint{4.721483in}{2.164030in}}%
\pgfpathlineto{\pgfqpoint{4.707401in}{2.161002in}}%
\pgfpathlineto{\pgfqpoint{4.693331in}{2.158073in}}%
\pgfpathlineto{\pgfqpoint{4.679272in}{2.155244in}}%
\pgfpathlineto{\pgfqpoint{4.665224in}{2.152515in}}%
\pgfpathlineto{\pgfqpoint{4.673027in}{2.164048in}}%
\pgfpathlineto{\pgfqpoint{4.680825in}{2.175550in}}%
\pgfpathlineto{\pgfqpoint{4.688619in}{2.187018in}}%
\pgfpathlineto{\pgfqpoint{4.696406in}{2.198450in}}%
\pgfpathclose%
\pgfusepath{fill}%
\end{pgfscope}%
\begin{pgfscope}%
\pgfpathrectangle{\pgfqpoint{1.150000in}{0.150000in}}{\pgfqpoint{5.700000in}{5.700000in}}%
\pgfusepath{clip}%
\pgfsetbuttcap%
\pgfsetroundjoin%
\definecolor{currentfill}{rgb}{0.267004,0.004874,0.329415}%
\pgfsetfillcolor{currentfill}%
\pgfsetfillopacity{0.700000}%
\pgfsetlinewidth{0.000000pt}%
\definecolor{currentstroke}{rgb}{0.000000,0.000000,0.000000}%
\pgfsetstrokecolor{currentstroke}%
\pgfsetdash{}{0pt}%
\pgfpathmoveto{\pgfqpoint{3.856569in}{1.848520in}}%
\pgfpathlineto{\pgfqpoint{3.870339in}{1.844799in}}%
\pgfpathlineto{\pgfqpoint{3.884114in}{1.841184in}}%
\pgfpathlineto{\pgfqpoint{3.897896in}{1.837676in}}%
\pgfpathlineto{\pgfqpoint{3.911684in}{1.834274in}}%
\pgfpathlineto{\pgfqpoint{3.903633in}{1.826376in}}%
\pgfpathlineto{\pgfqpoint{3.895575in}{1.818581in}}%
\pgfpathlineto{\pgfqpoint{3.887511in}{1.810890in}}%
\pgfpathlineto{\pgfqpoint{3.879440in}{1.803309in}}%
\pgfpathlineto{\pgfqpoint{3.865637in}{1.807084in}}%
\pgfpathlineto{\pgfqpoint{3.851841in}{1.810966in}}%
\pgfpathlineto{\pgfqpoint{3.838050in}{1.814954in}}%
\pgfpathlineto{\pgfqpoint{3.824265in}{1.819049in}}%
\pgfpathlineto{\pgfqpoint{3.832351in}{1.826250in}}%
\pgfpathlineto{\pgfqpoint{3.840431in}{1.833564in}}%
\pgfpathlineto{\pgfqpoint{3.848503in}{1.840989in}}%
\pgfpathlineto{\pgfqpoint{3.856569in}{1.848520in}}%
\pgfpathclose%
\pgfusepath{fill}%
\end{pgfscope}%
\begin{pgfscope}%
\pgfpathrectangle{\pgfqpoint{1.150000in}{0.150000in}}{\pgfqpoint{5.700000in}{5.700000in}}%
\pgfusepath{clip}%
\pgfsetbuttcap%
\pgfsetroundjoin%
\definecolor{currentfill}{rgb}{0.133743,0.548535,0.553541}%
\pgfsetfillcolor{currentfill}%
\pgfsetfillopacity{0.700000}%
\pgfsetlinewidth{0.000000pt}%
\definecolor{currentstroke}{rgb}{0.000000,0.000000,0.000000}%
\pgfsetstrokecolor{currentstroke}%
\pgfsetdash{}{0pt}%
\pgfpathmoveto{\pgfqpoint{5.837751in}{3.077429in}}%
\pgfpathlineto{\pgfqpoint{5.852384in}{3.086125in}}%
\pgfpathlineto{\pgfqpoint{5.867033in}{3.094922in}}%
\pgfpathlineto{\pgfqpoint{5.881699in}{3.103819in}}%
\pgfpathlineto{\pgfqpoint{5.896381in}{3.112818in}}%
\pgfpathlineto{\pgfqpoint{5.889103in}{3.106107in}}%
\pgfpathlineto{\pgfqpoint{5.881816in}{3.099274in}}%
\pgfpathlineto{\pgfqpoint{5.874518in}{3.092320in}}%
\pgfpathlineto{\pgfqpoint{5.867211in}{3.085243in}}%
\pgfpathlineto{\pgfqpoint{5.852517in}{3.076171in}}%
\pgfpathlineto{\pgfqpoint{5.837839in}{3.067200in}}%
\pgfpathlineto{\pgfqpoint{5.823179in}{3.058330in}}%
\pgfpathlineto{\pgfqpoint{5.808535in}{3.049561in}}%
\pgfpathlineto{\pgfqpoint{5.815853in}{3.056704in}}%
\pgfpathlineto{\pgfqpoint{5.823162in}{3.063729in}}%
\pgfpathlineto{\pgfqpoint{5.830461in}{3.070637in}}%
\pgfpathlineto{\pgfqpoint{5.837751in}{3.077429in}}%
\pgfpathclose%
\pgfusepath{fill}%
\end{pgfscope}%
\begin{pgfscope}%
\pgfpathrectangle{\pgfqpoint{1.150000in}{0.150000in}}{\pgfqpoint{5.700000in}{5.700000in}}%
\pgfusepath{clip}%
\pgfsetbuttcap%
\pgfsetroundjoin%
\definecolor{currentfill}{rgb}{0.216210,0.351535,0.550627}%
\pgfsetfillcolor{currentfill}%
\pgfsetfillopacity{0.700000}%
\pgfsetlinewidth{0.000000pt}%
\definecolor{currentstroke}{rgb}{0.000000,0.000000,0.000000}%
\pgfsetstrokecolor{currentstroke}%
\pgfsetdash{}{0pt}%
\pgfpathmoveto{\pgfqpoint{2.591422in}{2.577879in}}%
\pgfpathlineto{\pgfqpoint{2.605251in}{2.561587in}}%
\pgfpathlineto{\pgfqpoint{2.619075in}{2.545465in}}%
\pgfpathlineto{\pgfqpoint{2.632894in}{2.529510in}}%
\pgfpathlineto{\pgfqpoint{2.646709in}{2.513721in}}%
\pgfpathlineto{\pgfqpoint{2.637861in}{2.517987in}}%
\pgfpathlineto{\pgfqpoint{2.628994in}{2.522538in}}%
\pgfpathlineto{\pgfqpoint{2.620109in}{2.527379in}}%
\pgfpathlineto{\pgfqpoint{2.611205in}{2.532517in}}%
\pgfpathlineto{\pgfqpoint{2.597344in}{2.548802in}}%
\pgfpathlineto{\pgfqpoint{2.583477in}{2.565253in}}%
\pgfpathlineto{\pgfqpoint{2.569606in}{2.581873in}}%
\pgfpathlineto{\pgfqpoint{2.555730in}{2.598663in}}%
\pgfpathlineto{\pgfqpoint{2.564682in}{2.593020in}}%
\pgfpathlineto{\pgfqpoint{2.573615in}{2.587678in}}%
\pgfpathlineto{\pgfqpoint{2.582528in}{2.582633in}}%
\pgfpathlineto{\pgfqpoint{2.591422in}{2.577879in}}%
\pgfpathclose%
\pgfusepath{fill}%
\end{pgfscope}%
\begin{pgfscope}%
\pgfpathrectangle{\pgfqpoint{1.150000in}{0.150000in}}{\pgfqpoint{5.700000in}{5.700000in}}%
\pgfusepath{clip}%
\pgfsetbuttcap%
\pgfsetroundjoin%
\definecolor{currentfill}{rgb}{0.283197,0.115680,0.436115}%
\pgfsetfillcolor{currentfill}%
\pgfsetfillopacity{0.700000}%
\pgfsetlinewidth{0.000000pt}%
\definecolor{currentstroke}{rgb}{0.000000,0.000000,0.000000}%
\pgfsetstrokecolor{currentstroke}%
\pgfsetdash{}{0pt}%
\pgfpathmoveto{\pgfqpoint{3.176141in}{2.048478in}}%
\pgfpathlineto{\pgfqpoint{3.189858in}{2.038506in}}%
\pgfpathlineto{\pgfqpoint{3.203577in}{2.028661in}}%
\pgfpathlineto{\pgfqpoint{3.217296in}{2.018943in}}%
\pgfpathlineto{\pgfqpoint{3.231017in}{2.009351in}}%
\pgfpathlineto{\pgfqpoint{3.222606in}{2.007974in}}%
\pgfpathlineto{\pgfqpoint{3.214182in}{2.006811in}}%
\pgfpathlineto{\pgfqpoint{3.205747in}{2.005867in}}%
\pgfpathlineto{\pgfqpoint{3.197299in}{2.005147in}}%
\pgfpathlineto{\pgfqpoint{3.183547in}{2.015193in}}%
\pgfpathlineto{\pgfqpoint{3.169796in}{2.025366in}}%
\pgfpathlineto{\pgfqpoint{3.156045in}{2.035666in}}%
\pgfpathlineto{\pgfqpoint{3.142296in}{2.046094in}}%
\pgfpathlineto{\pgfqpoint{3.150776in}{2.046352in}}%
\pgfpathlineto{\pgfqpoint{3.159244in}{2.046838in}}%
\pgfpathlineto{\pgfqpoint{3.167699in}{2.047549in}}%
\pgfpathlineto{\pgfqpoint{3.176141in}{2.048478in}}%
\pgfpathclose%
\pgfusepath{fill}%
\end{pgfscope}%
\begin{pgfscope}%
\pgfpathrectangle{\pgfqpoint{1.150000in}{0.150000in}}{\pgfqpoint{5.700000in}{5.700000in}}%
\pgfusepath{clip}%
\pgfsetbuttcap%
\pgfsetroundjoin%
\definecolor{currentfill}{rgb}{0.227802,0.326594,0.546532}%
\pgfsetfillcolor{currentfill}%
\pgfsetfillopacity{0.700000}%
\pgfsetlinewidth{0.000000pt}%
\definecolor{currentstroke}{rgb}{0.000000,0.000000,0.000000}%
\pgfsetstrokecolor{currentstroke}%
\pgfsetdash{}{0pt}%
\pgfpathmoveto{\pgfqpoint{2.646709in}{2.513721in}}%
\pgfpathlineto{\pgfqpoint{2.660519in}{2.498098in}}%
\pgfpathlineto{\pgfqpoint{2.674325in}{2.482638in}}%
\pgfpathlineto{\pgfqpoint{2.688127in}{2.467340in}}%
\pgfpathlineto{\pgfqpoint{2.701926in}{2.452203in}}%
\pgfpathlineto{\pgfqpoint{2.693122in}{2.455983in}}%
\pgfpathlineto{\pgfqpoint{2.684301in}{2.460043in}}%
\pgfpathlineto{\pgfqpoint{2.675462in}{2.464388in}}%
\pgfpathlineto{\pgfqpoint{2.666605in}{2.469023in}}%
\pgfpathlineto{\pgfqpoint{2.652761in}{2.484653in}}%
\pgfpathlineto{\pgfqpoint{2.638913in}{2.500444in}}%
\pgfpathlineto{\pgfqpoint{2.625061in}{2.516399in}}%
\pgfpathlineto{\pgfqpoint{2.611205in}{2.532517in}}%
\pgfpathlineto{\pgfqpoint{2.620109in}{2.527379in}}%
\pgfpathlineto{\pgfqpoint{2.628994in}{2.522538in}}%
\pgfpathlineto{\pgfqpoint{2.637861in}{2.517987in}}%
\pgfpathlineto{\pgfqpoint{2.646709in}{2.513721in}}%
\pgfpathclose%
\pgfusepath{fill}%
\end{pgfscope}%
\begin{pgfscope}%
\pgfpathrectangle{\pgfqpoint{1.150000in}{0.150000in}}{\pgfqpoint{5.700000in}{5.700000in}}%
\pgfusepath{clip}%
\pgfsetbuttcap%
\pgfsetroundjoin%
\definecolor{currentfill}{rgb}{0.271305,0.019942,0.347269}%
\pgfsetfillcolor{currentfill}%
\pgfsetfillopacity{0.700000}%
\pgfsetlinewidth{0.000000pt}%
\definecolor{currentstroke}{rgb}{0.000000,0.000000,0.000000}%
\pgfsetstrokecolor{currentstroke}%
\pgfsetdash{}{0pt}%
\pgfpathmoveto{\pgfqpoint{4.086207in}{1.882825in}}%
\pgfpathlineto{\pgfqpoint{4.100033in}{1.881060in}}%
\pgfpathlineto{\pgfqpoint{4.113866in}{1.879398in}}%
\pgfpathlineto{\pgfqpoint{4.127706in}{1.877839in}}%
\pgfpathlineto{\pgfqpoint{4.141555in}{1.876384in}}%
\pgfpathlineto{\pgfqpoint{4.133587in}{1.866802in}}%
\pgfpathlineto{\pgfqpoint{4.125614in}{1.857283in}}%
\pgfpathlineto{\pgfqpoint{4.117636in}{1.847827in}}%
\pgfpathlineto{\pgfqpoint{4.109651in}{1.838440in}}%
\pgfpathlineto{\pgfqpoint{4.095793in}{1.840233in}}%
\pgfpathlineto{\pgfqpoint{4.081941in}{1.842129in}}%
\pgfpathlineto{\pgfqpoint{4.068098in}{1.844129in}}%
\pgfpathlineto{\pgfqpoint{4.054261in}{1.846232in}}%
\pgfpathlineto{\pgfqpoint{4.062256in}{1.855275in}}%
\pgfpathlineto{\pgfqpoint{4.070246in}{1.864390in}}%
\pgfpathlineto{\pgfqpoint{4.078229in}{1.873575in}}%
\pgfpathlineto{\pgfqpoint{4.086207in}{1.882825in}}%
\pgfpathclose%
\pgfusepath{fill}%
\end{pgfscope}%
\begin{pgfscope}%
\pgfpathrectangle{\pgfqpoint{1.150000in}{0.150000in}}{\pgfqpoint{5.700000in}{5.700000in}}%
\pgfusepath{clip}%
\pgfsetbuttcap%
\pgfsetroundjoin%
\definecolor{currentfill}{rgb}{0.203063,0.379716,0.553925}%
\pgfsetfillcolor{currentfill}%
\pgfsetfillopacity{0.700000}%
\pgfsetlinewidth{0.000000pt}%
\definecolor{currentstroke}{rgb}{0.000000,0.000000,0.000000}%
\pgfsetstrokecolor{currentstroke}%
\pgfsetdash{}{0pt}%
\pgfpathmoveto{\pgfqpoint{2.536056in}{2.644766in}}%
\pgfpathlineto{\pgfqpoint{2.549905in}{2.627783in}}%
\pgfpathlineto{\pgfqpoint{2.563750in}{2.610976in}}%
\pgfpathlineto{\pgfqpoint{2.577589in}{2.594341in}}%
\pgfpathlineto{\pgfqpoint{2.591422in}{2.577879in}}%
\pgfpathlineto{\pgfqpoint{2.582528in}{2.582633in}}%
\pgfpathlineto{\pgfqpoint{2.573615in}{2.587678in}}%
\pgfpathlineto{\pgfqpoint{2.564682in}{2.593020in}}%
\pgfpathlineto{\pgfqpoint{2.555730in}{2.598663in}}%
\pgfpathlineto{\pgfqpoint{2.541848in}{2.615625in}}%
\pgfpathlineto{\pgfqpoint{2.527961in}{2.632759in}}%
\pgfpathlineto{\pgfqpoint{2.514068in}{2.650068in}}%
\pgfpathlineto{\pgfqpoint{2.500169in}{2.667552in}}%
\pgfpathlineto{\pgfqpoint{2.509171in}{2.661400in}}%
\pgfpathlineto{\pgfqpoint{2.518152in}{2.655555in}}%
\pgfpathlineto{\pgfqpoint{2.527114in}{2.650012in}}%
\pgfpathlineto{\pgfqpoint{2.536056in}{2.644766in}}%
\pgfpathclose%
\pgfusepath{fill}%
\end{pgfscope}%
\begin{pgfscope}%
\pgfpathrectangle{\pgfqpoint{1.150000in}{0.150000in}}{\pgfqpoint{5.700000in}{5.700000in}}%
\pgfusepath{clip}%
\pgfsetbuttcap%
\pgfsetroundjoin%
\definecolor{currentfill}{rgb}{0.239346,0.300855,0.540844}%
\pgfsetfillcolor{currentfill}%
\pgfsetfillopacity{0.700000}%
\pgfsetlinewidth{0.000000pt}%
\definecolor{currentstroke}{rgb}{0.000000,0.000000,0.000000}%
\pgfsetstrokecolor{currentstroke}%
\pgfsetdash{}{0pt}%
\pgfpathmoveto{\pgfqpoint{2.701926in}{2.452203in}}%
\pgfpathlineto{\pgfqpoint{2.715720in}{2.437227in}}%
\pgfpathlineto{\pgfqpoint{2.729511in}{2.422408in}}%
\pgfpathlineto{\pgfqpoint{2.743299in}{2.407747in}}%
\pgfpathlineto{\pgfqpoint{2.757083in}{2.393242in}}%
\pgfpathlineto{\pgfqpoint{2.748323in}{2.396538in}}%
\pgfpathlineto{\pgfqpoint{2.739546in}{2.400109in}}%
\pgfpathlineto{\pgfqpoint{2.730752in}{2.403960in}}%
\pgfpathlineto{\pgfqpoint{2.721939in}{2.408096in}}%
\pgfpathlineto{\pgfqpoint{2.708111in}{2.423092in}}%
\pgfpathlineto{\pgfqpoint{2.694280in}{2.438244in}}%
\pgfpathlineto{\pgfqpoint{2.680444in}{2.453554in}}%
\pgfpathlineto{\pgfqpoint{2.666605in}{2.469023in}}%
\pgfpathlineto{\pgfqpoint{2.675462in}{2.464388in}}%
\pgfpathlineto{\pgfqpoint{2.684301in}{2.460043in}}%
\pgfpathlineto{\pgfqpoint{2.693122in}{2.455983in}}%
\pgfpathlineto{\pgfqpoint{2.701926in}{2.452203in}}%
\pgfpathclose%
\pgfusepath{fill}%
\end{pgfscope}%
\begin{pgfscope}%
\pgfpathrectangle{\pgfqpoint{1.150000in}{0.150000in}}{\pgfqpoint{5.700000in}{5.700000in}}%
\pgfusepath{clip}%
\pgfsetbuttcap%
\pgfsetroundjoin%
\definecolor{currentfill}{rgb}{0.175841,0.441290,0.557685}%
\pgfsetfillcolor{currentfill}%
\pgfsetfillopacity{0.700000}%
\pgfsetlinewidth{0.000000pt}%
\definecolor{currentstroke}{rgb}{0.000000,0.000000,0.000000}%
\pgfsetstrokecolor{currentstroke}%
\pgfsetdash{}{0pt}%
\pgfpathmoveto{\pgfqpoint{5.457520in}{2.790414in}}%
\pgfpathlineto{\pgfqpoint{5.471945in}{2.797670in}}%
\pgfpathlineto{\pgfqpoint{5.486386in}{2.805025in}}%
\pgfpathlineto{\pgfqpoint{5.500841in}{2.812481in}}%
\pgfpathlineto{\pgfqpoint{5.515312in}{2.820037in}}%
\pgfpathlineto{\pgfqpoint{5.507819in}{2.810831in}}%
\pgfpathlineto{\pgfqpoint{5.500319in}{2.801512in}}%
\pgfpathlineto{\pgfqpoint{5.492810in}{2.792079in}}%
\pgfpathlineto{\pgfqpoint{5.485293in}{2.782533in}}%
\pgfpathlineto{\pgfqpoint{5.470816in}{2.775002in}}%
\pgfpathlineto{\pgfqpoint{5.456354in}{2.767572in}}%
\pgfpathlineto{\pgfqpoint{5.441908in}{2.760242in}}%
\pgfpathlineto{\pgfqpoint{5.427476in}{2.753012in}}%
\pgfpathlineto{\pgfqpoint{5.434999in}{2.762526in}}%
\pgfpathlineto{\pgfqpoint{5.442514in}{2.771930in}}%
\pgfpathlineto{\pgfqpoint{5.450021in}{2.781227in}}%
\pgfpathlineto{\pgfqpoint{5.457520in}{2.790414in}}%
\pgfpathclose%
\pgfusepath{fill}%
\end{pgfscope}%
\begin{pgfscope}%
\pgfpathrectangle{\pgfqpoint{1.150000in}{0.150000in}}{\pgfqpoint{5.700000in}{5.700000in}}%
\pgfusepath{clip}%
\pgfsetbuttcap%
\pgfsetroundjoin%
\definecolor{currentfill}{rgb}{0.269308,0.218818,0.509577}%
\pgfsetfillcolor{currentfill}%
\pgfsetfillopacity{0.700000}%
\pgfsetlinewidth{0.000000pt}%
\definecolor{currentstroke}{rgb}{0.000000,0.000000,0.000000}%
\pgfsetstrokecolor{currentstroke}%
\pgfsetdash{}{0pt}%
\pgfpathmoveto{\pgfqpoint{4.783736in}{2.256933in}}%
\pgfpathlineto{\pgfqpoint{4.797823in}{2.260470in}}%
\pgfpathlineto{\pgfqpoint{4.811921in}{2.264108in}}%
\pgfpathlineto{\pgfqpoint{4.826032in}{2.267845in}}%
\pgfpathlineto{\pgfqpoint{4.840154in}{2.271682in}}%
\pgfpathlineto{\pgfqpoint{4.832394in}{2.260034in}}%
\pgfpathlineto{\pgfqpoint{4.824629in}{2.248334in}}%
\pgfpathlineto{\pgfqpoint{4.816858in}{2.236584in}}%
\pgfpathlineto{\pgfqpoint{4.809082in}{2.224786in}}%
\pgfpathlineto{\pgfqpoint{4.794957in}{2.221144in}}%
\pgfpathlineto{\pgfqpoint{4.780843in}{2.217602in}}%
\pgfpathlineto{\pgfqpoint{4.766741in}{2.214160in}}%
\pgfpathlineto{\pgfqpoint{4.752651in}{2.210818in}}%
\pgfpathlineto{\pgfqpoint{4.760431in}{2.222414in}}%
\pgfpathlineto{\pgfqpoint{4.768204in}{2.233967in}}%
\pgfpathlineto{\pgfqpoint{4.775973in}{2.245474in}}%
\pgfpathlineto{\pgfqpoint{4.783736in}{2.256933in}}%
\pgfpathclose%
\pgfusepath{fill}%
\end{pgfscope}%
\begin{pgfscope}%
\pgfpathrectangle{\pgfqpoint{1.150000in}{0.150000in}}{\pgfqpoint{5.700000in}{5.700000in}}%
\pgfusepath{clip}%
\pgfsetbuttcap%
\pgfsetroundjoin%
\definecolor{currentfill}{rgb}{0.190631,0.407061,0.556089}%
\pgfsetfillcolor{currentfill}%
\pgfsetfillopacity{0.700000}%
\pgfsetlinewidth{0.000000pt}%
\definecolor{currentstroke}{rgb}{0.000000,0.000000,0.000000}%
\pgfsetstrokecolor{currentstroke}%
\pgfsetdash{}{0pt}%
\pgfpathmoveto{\pgfqpoint{2.480598in}{2.714480in}}%
\pgfpathlineto{\pgfqpoint{2.494471in}{2.696781in}}%
\pgfpathlineto{\pgfqpoint{2.508339in}{2.679264in}}%
\pgfpathlineto{\pgfqpoint{2.522200in}{2.661926in}}%
\pgfpathlineto{\pgfqpoint{2.536056in}{2.644766in}}%
\pgfpathlineto{\pgfqpoint{2.527114in}{2.650012in}}%
\pgfpathlineto{\pgfqpoint{2.518152in}{2.655555in}}%
\pgfpathlineto{\pgfqpoint{2.509171in}{2.661400in}}%
\pgfpathlineto{\pgfqpoint{2.500169in}{2.667552in}}%
\pgfpathlineto{\pgfqpoint{2.486264in}{2.685214in}}%
\pgfpathlineto{\pgfqpoint{2.472352in}{2.703055in}}%
\pgfpathlineto{\pgfqpoint{2.458435in}{2.721076in}}%
\pgfpathlineto{\pgfqpoint{2.444510in}{2.739280in}}%
\pgfpathlineto{\pgfqpoint{2.453563in}{2.732616in}}%
\pgfpathlineto{\pgfqpoint{2.462595in}{2.726264in}}%
\pgfpathlineto{\pgfqpoint{2.471607in}{2.720221in}}%
\pgfpathlineto{\pgfqpoint{2.480598in}{2.714480in}}%
\pgfpathclose%
\pgfusepath{fill}%
\end{pgfscope}%
\begin{pgfscope}%
\pgfpathrectangle{\pgfqpoint{1.150000in}{0.150000in}}{\pgfqpoint{5.700000in}{5.700000in}}%
\pgfusepath{clip}%
\pgfsetbuttcap%
\pgfsetroundjoin%
\definecolor{currentfill}{rgb}{0.248629,0.278775,0.534556}%
\pgfsetfillcolor{currentfill}%
\pgfsetfillopacity{0.700000}%
\pgfsetlinewidth{0.000000pt}%
\definecolor{currentstroke}{rgb}{0.000000,0.000000,0.000000}%
\pgfsetstrokecolor{currentstroke}%
\pgfsetdash{}{0pt}%
\pgfpathmoveto{\pgfqpoint{2.757083in}{2.393242in}}%
\pgfpathlineto{\pgfqpoint{2.770865in}{2.378892in}}%
\pgfpathlineto{\pgfqpoint{2.784643in}{2.364695in}}%
\pgfpathlineto{\pgfqpoint{2.798418in}{2.350651in}}%
\pgfpathlineto{\pgfqpoint{2.812191in}{2.336758in}}%
\pgfpathlineto{\pgfqpoint{2.803473in}{2.339574in}}%
\pgfpathlineto{\pgfqpoint{2.794739in}{2.342659in}}%
\pgfpathlineto{\pgfqpoint{2.785987in}{2.346018in}}%
\pgfpathlineto{\pgfqpoint{2.777219in}{2.349657in}}%
\pgfpathlineto{\pgfqpoint{2.763404in}{2.364038in}}%
\pgfpathlineto{\pgfqpoint{2.749585in}{2.378570in}}%
\pgfpathlineto{\pgfqpoint{2.735764in}{2.393256in}}%
\pgfpathlineto{\pgfqpoint{2.721939in}{2.408096in}}%
\pgfpathlineto{\pgfqpoint{2.730752in}{2.403960in}}%
\pgfpathlineto{\pgfqpoint{2.739546in}{2.400109in}}%
\pgfpathlineto{\pgfqpoint{2.748323in}{2.396538in}}%
\pgfpathlineto{\pgfqpoint{2.757083in}{2.393242in}}%
\pgfpathclose%
\pgfusepath{fill}%
\end{pgfscope}%
\begin{pgfscope}%
\pgfpathrectangle{\pgfqpoint{1.150000in}{0.150000in}}{\pgfqpoint{5.700000in}{5.700000in}}%
\pgfusepath{clip}%
\pgfsetbuttcap%
\pgfsetroundjoin%
\definecolor{currentfill}{rgb}{0.218130,0.347432,0.550038}%
\pgfsetfillcolor{currentfill}%
\pgfsetfillopacity{0.700000}%
\pgfsetlinewidth{0.000000pt}%
\definecolor{currentstroke}{rgb}{0.000000,0.000000,0.000000}%
\pgfsetstrokecolor{currentstroke}%
\pgfsetdash{}{0pt}%
\pgfpathmoveto{\pgfqpoint{5.164403in}{2.553812in}}%
\pgfpathlineto{\pgfqpoint{5.178676in}{2.559648in}}%
\pgfpathlineto{\pgfqpoint{5.192964in}{2.565583in}}%
\pgfpathlineto{\pgfqpoint{5.207265in}{2.571618in}}%
\pgfpathlineto{\pgfqpoint{5.221580in}{2.577753in}}%
\pgfpathlineto{\pgfqpoint{5.213953in}{2.567039in}}%
\pgfpathlineto{\pgfqpoint{5.206319in}{2.556231in}}%
\pgfpathlineto{\pgfqpoint{5.198679in}{2.545329in}}%
\pgfpathlineto{\pgfqpoint{5.191032in}{2.534335in}}%
\pgfpathlineto{\pgfqpoint{5.176714in}{2.528303in}}%
\pgfpathlineto{\pgfqpoint{5.162409in}{2.522370in}}%
\pgfpathlineto{\pgfqpoint{5.148118in}{2.516538in}}%
\pgfpathlineto{\pgfqpoint{5.133841in}{2.510805in}}%
\pgfpathlineto{\pgfqpoint{5.141491in}{2.521690in}}%
\pgfpathlineto{\pgfqpoint{5.149135in}{2.532486in}}%
\pgfpathlineto{\pgfqpoint{5.156772in}{2.543194in}}%
\pgfpathlineto{\pgfqpoint{5.164403in}{2.553812in}}%
\pgfpathclose%
\pgfusepath{fill}%
\end{pgfscope}%
\begin{pgfscope}%
\pgfpathrectangle{\pgfqpoint{1.150000in}{0.150000in}}{\pgfqpoint{5.700000in}{5.700000in}}%
\pgfusepath{clip}%
\pgfsetbuttcap%
\pgfsetroundjoin%
\definecolor{currentfill}{rgb}{0.276022,0.044167,0.370164}%
\pgfsetfillcolor{currentfill}%
\pgfsetfillopacity{0.700000}%
\pgfsetlinewidth{0.000000pt}%
\definecolor{currentstroke}{rgb}{0.000000,0.000000,0.000000}%
\pgfsetstrokecolor{currentstroke}%
\pgfsetdash{}{0pt}%
\pgfpathmoveto{\pgfqpoint{3.428998in}{1.916584in}}%
\pgfpathlineto{\pgfqpoint{3.442718in}{1.909003in}}%
\pgfpathlineto{\pgfqpoint{3.456442in}{1.901538in}}%
\pgfpathlineto{\pgfqpoint{3.470169in}{1.894191in}}%
\pgfpathlineto{\pgfqpoint{3.483898in}{1.886959in}}%
\pgfpathlineto{\pgfqpoint{3.475639in}{1.883069in}}%
\pgfpathlineto{\pgfqpoint{3.467370in}{1.879356in}}%
\pgfpathlineto{\pgfqpoint{3.459090in}{1.875824in}}%
\pgfpathlineto{\pgfqpoint{3.450802in}{1.872478in}}%
\pgfpathlineto{\pgfqpoint{3.437047in}{1.880141in}}%
\pgfpathlineto{\pgfqpoint{3.423295in}{1.887920in}}%
\pgfpathlineto{\pgfqpoint{3.409546in}{1.895816in}}%
\pgfpathlineto{\pgfqpoint{3.395800in}{1.903829in}}%
\pgfpathlineto{\pgfqpoint{3.404115in}{1.906737in}}%
\pgfpathlineto{\pgfqpoint{3.412419in}{1.909835in}}%
\pgfpathlineto{\pgfqpoint{3.420714in}{1.913119in}}%
\pgfpathlineto{\pgfqpoint{3.428998in}{1.916584in}}%
\pgfpathclose%
\pgfusepath{fill}%
\end{pgfscope}%
\begin{pgfscope}%
\pgfpathrectangle{\pgfqpoint{1.150000in}{0.150000in}}{\pgfqpoint{5.700000in}{5.700000in}}%
\pgfusepath{clip}%
\pgfsetbuttcap%
\pgfsetroundjoin%
\definecolor{currentfill}{rgb}{0.258965,0.251537,0.524736}%
\pgfsetfillcolor{currentfill}%
\pgfsetfillopacity{0.700000}%
\pgfsetlinewidth{0.000000pt}%
\definecolor{currentstroke}{rgb}{0.000000,0.000000,0.000000}%
\pgfsetstrokecolor{currentstroke}%
\pgfsetdash{}{0pt}%
\pgfpathmoveto{\pgfqpoint{2.812191in}{2.336758in}}%
\pgfpathlineto{\pgfqpoint{2.825962in}{2.323015in}}%
\pgfpathlineto{\pgfqpoint{2.839730in}{2.309422in}}%
\pgfpathlineto{\pgfqpoint{2.853495in}{2.295977in}}%
\pgfpathlineto{\pgfqpoint{2.867259in}{2.282678in}}%
\pgfpathlineto{\pgfqpoint{2.858582in}{2.285016in}}%
\pgfpathlineto{\pgfqpoint{2.849888in}{2.287617in}}%
\pgfpathlineto{\pgfqpoint{2.841179in}{2.290487in}}%
\pgfpathlineto{\pgfqpoint{2.832453in}{2.293632in}}%
\pgfpathlineto{\pgfqpoint{2.818648in}{2.307415in}}%
\pgfpathlineto{\pgfqpoint{2.804841in}{2.321347in}}%
\pgfpathlineto{\pgfqpoint{2.791031in}{2.335427in}}%
\pgfpathlineto{\pgfqpoint{2.777219in}{2.349657in}}%
\pgfpathlineto{\pgfqpoint{2.785987in}{2.346018in}}%
\pgfpathlineto{\pgfqpoint{2.794739in}{2.342659in}}%
\pgfpathlineto{\pgfqpoint{2.803473in}{2.339574in}}%
\pgfpathlineto{\pgfqpoint{2.812191in}{2.336758in}}%
\pgfpathclose%
\pgfusepath{fill}%
\end{pgfscope}%
\begin{pgfscope}%
\pgfpathrectangle{\pgfqpoint{1.150000in}{0.150000in}}{\pgfqpoint{5.700000in}{5.700000in}}%
\pgfusepath{clip}%
\pgfsetbuttcap%
\pgfsetroundjoin%
\definecolor{currentfill}{rgb}{0.126453,0.570633,0.549841}%
\pgfsetfillcolor{currentfill}%
\pgfsetfillopacity{0.700000}%
\pgfsetlinewidth{0.000000pt}%
\definecolor{currentstroke}{rgb}{0.000000,0.000000,0.000000}%
\pgfsetstrokecolor{currentstroke}%
\pgfsetdash{}{0pt}%
\pgfpathmoveto{\pgfqpoint{5.925398in}{3.138476in}}%
\pgfpathlineto{\pgfqpoint{5.940084in}{3.147482in}}%
\pgfpathlineto{\pgfqpoint{5.954788in}{3.156589in}}%
\pgfpathlineto{\pgfqpoint{5.969509in}{3.165796in}}%
\pgfpathlineto{\pgfqpoint{5.984246in}{3.175104in}}%
\pgfpathlineto{\pgfqpoint{5.977020in}{3.168966in}}%
\pgfpathlineto{\pgfqpoint{5.969784in}{3.162707in}}%
\pgfpathlineto{\pgfqpoint{5.962538in}{3.156325in}}%
\pgfpathlineto{\pgfqpoint{5.955282in}{3.149821in}}%
\pgfpathlineto{\pgfqpoint{5.940531in}{3.140418in}}%
\pgfpathlineto{\pgfqpoint{5.925798in}{3.131117in}}%
\pgfpathlineto{\pgfqpoint{5.911081in}{3.121917in}}%
\pgfpathlineto{\pgfqpoint{5.896381in}{3.112818in}}%
\pgfpathlineto{\pgfqpoint{5.903650in}{3.119409in}}%
\pgfpathlineto{\pgfqpoint{5.910909in}{3.125882in}}%
\pgfpathlineto{\pgfqpoint{5.918158in}{3.132237in}}%
\pgfpathlineto{\pgfqpoint{5.925398in}{3.138476in}}%
\pgfpathclose%
\pgfusepath{fill}%
\end{pgfscope}%
\begin{pgfscope}%
\pgfpathrectangle{\pgfqpoint{1.150000in}{0.150000in}}{\pgfqpoint{5.700000in}{5.700000in}}%
\pgfusepath{clip}%
\pgfsetbuttcap%
\pgfsetroundjoin%
\definecolor{currentfill}{rgb}{0.179019,0.433756,0.557430}%
\pgfsetfillcolor{currentfill}%
\pgfsetfillopacity{0.700000}%
\pgfsetlinewidth{0.000000pt}%
\definecolor{currentstroke}{rgb}{0.000000,0.000000,0.000000}%
\pgfsetstrokecolor{currentstroke}%
\pgfsetdash{}{0pt}%
\pgfpathmoveto{\pgfqpoint{2.425037in}{2.787122in}}%
\pgfpathlineto{\pgfqpoint{2.438937in}{2.768681in}}%
\pgfpathlineto{\pgfqpoint{2.452831in}{2.750428in}}%
\pgfpathlineto{\pgfqpoint{2.466718in}{2.732362in}}%
\pgfpathlineto{\pgfqpoint{2.480598in}{2.714480in}}%
\pgfpathlineto{\pgfqpoint{2.471607in}{2.720221in}}%
\pgfpathlineto{\pgfqpoint{2.462595in}{2.726264in}}%
\pgfpathlineto{\pgfqpoint{2.453563in}{2.732616in}}%
\pgfpathlineto{\pgfqpoint{2.444510in}{2.739280in}}%
\pgfpathlineto{\pgfqpoint{2.430579in}{2.757668in}}%
\pgfpathlineto{\pgfqpoint{2.416641in}{2.776241in}}%
\pgfpathlineto{\pgfqpoint{2.402696in}{2.795001in}}%
\pgfpathlineto{\pgfqpoint{2.388743in}{2.813951in}}%
\pgfpathlineto{\pgfqpoint{2.397849in}{2.806770in}}%
\pgfpathlineto{\pgfqpoint{2.406933in}{2.799909in}}%
\pgfpathlineto{\pgfqpoint{2.415995in}{2.793361in}}%
\pgfpathlineto{\pgfqpoint{2.425037in}{2.787122in}}%
\pgfpathclose%
\pgfusepath{fill}%
\end{pgfscope}%
\begin{pgfscope}%
\pgfpathrectangle{\pgfqpoint{1.150000in}{0.150000in}}{\pgfqpoint{5.700000in}{5.700000in}}%
\pgfusepath{clip}%
\pgfsetbuttcap%
\pgfsetroundjoin%
\definecolor{currentfill}{rgb}{0.121148,0.592739,0.544641}%
\pgfsetfillcolor{currentfill}%
\pgfsetfillopacity{0.700000}%
\pgfsetlinewidth{0.000000pt}%
\definecolor{currentstroke}{rgb}{0.000000,0.000000,0.000000}%
\pgfsetstrokecolor{currentstroke}%
\pgfsetdash{}{0pt}%
\pgfpathmoveto{\pgfqpoint{6.013052in}{3.198475in}}%
\pgfpathlineto{\pgfqpoint{6.027793in}{3.207770in}}%
\pgfpathlineto{\pgfqpoint{6.042551in}{3.217167in}}%
\pgfpathlineto{\pgfqpoint{6.057326in}{3.226664in}}%
\pgfpathlineto{\pgfqpoint{6.050151in}{3.221087in}}%
\pgfpathlineto{\pgfqpoint{6.042965in}{3.215391in}}%
\pgfpathlineto{\pgfqpoint{6.035770in}{3.209574in}}%
\pgfpathlineto{\pgfqpoint{6.028564in}{3.203636in}}%
\pgfpathlineto{\pgfqpoint{6.013774in}{3.194024in}}%
\pgfpathlineto{\pgfqpoint{5.999002in}{3.184514in}}%
\pgfpathlineto{\pgfqpoint{5.984246in}{3.175104in}}%
\pgfpathlineto{\pgfqpoint{5.991463in}{3.181123in}}%
\pgfpathlineto{\pgfqpoint{5.998669in}{3.187023in}}%
\pgfpathlineto{\pgfqpoint{6.005865in}{3.192807in}}%
\pgfpathlineto{\pgfqpoint{6.013052in}{3.198475in}}%
\pgfpathclose%
\pgfusepath{fill}%
\end{pgfscope}%
\begin{pgfscope}%
\pgfpathrectangle{\pgfqpoint{1.150000in}{0.150000in}}{\pgfqpoint{5.700000in}{5.700000in}}%
\pgfusepath{clip}%
\pgfsetbuttcap%
\pgfsetroundjoin%
\definecolor{currentfill}{rgb}{0.268510,0.009605,0.335427}%
\pgfsetfillcolor{currentfill}%
\pgfsetfillopacity{0.700000}%
\pgfsetlinewidth{0.000000pt}%
\definecolor{currentstroke}{rgb}{0.000000,0.000000,0.000000}%
\pgfsetstrokecolor{currentstroke}%
\pgfsetdash{}{0pt}%
\pgfpathmoveto{\pgfqpoint{3.998987in}{1.855684in}}%
\pgfpathlineto{\pgfqpoint{4.012795in}{1.853164in}}%
\pgfpathlineto{\pgfqpoint{4.026610in}{1.850749in}}%
\pgfpathlineto{\pgfqpoint{4.040432in}{1.848438in}}%
\pgfpathlineto{\pgfqpoint{4.054261in}{1.846232in}}%
\pgfpathlineto{\pgfqpoint{4.046260in}{1.837264in}}%
\pgfpathlineto{\pgfqpoint{4.038254in}{1.828375in}}%
\pgfpathlineto{\pgfqpoint{4.030241in}{1.819569in}}%
\pgfpathlineto{\pgfqpoint{4.022223in}{1.810849in}}%
\pgfpathlineto{\pgfqpoint{4.008381in}{1.813411in}}%
\pgfpathlineto{\pgfqpoint{3.994547in}{1.816077in}}%
\pgfpathlineto{\pgfqpoint{3.980720in}{1.818848in}}%
\pgfpathlineto{\pgfqpoint{3.966900in}{1.821723in}}%
\pgfpathlineto{\pgfqpoint{3.974931in}{1.830081in}}%
\pgfpathlineto{\pgfqpoint{3.982956in}{1.838530in}}%
\pgfpathlineto{\pgfqpoint{3.990975in}{1.847065in}}%
\pgfpathlineto{\pgfqpoint{3.998987in}{1.855684in}}%
\pgfpathclose%
\pgfusepath{fill}%
\end{pgfscope}%
\begin{pgfscope}%
\pgfpathrectangle{\pgfqpoint{1.150000in}{0.150000in}}{\pgfqpoint{5.700000in}{5.700000in}}%
\pgfusepath{clip}%
\pgfsetbuttcap%
\pgfsetroundjoin%
\definecolor{currentfill}{rgb}{0.260571,0.246922,0.522828}%
\pgfsetfillcolor{currentfill}%
\pgfsetfillopacity{0.700000}%
\pgfsetlinewidth{0.000000pt}%
\definecolor{currentstroke}{rgb}{0.000000,0.000000,0.000000}%
\pgfsetstrokecolor{currentstroke}%
\pgfsetdash{}{0pt}%
\pgfpathmoveto{\pgfqpoint{4.871140in}{2.317722in}}%
\pgfpathlineto{\pgfqpoint{4.885271in}{2.321836in}}%
\pgfpathlineto{\pgfqpoint{4.899415in}{2.326050in}}%
\pgfpathlineto{\pgfqpoint{4.913571in}{2.330364in}}%
\pgfpathlineto{\pgfqpoint{4.927739in}{2.334777in}}%
\pgfpathlineto{\pgfqpoint{4.920004in}{2.323183in}}%
\pgfpathlineto{\pgfqpoint{4.912263in}{2.311525in}}%
\pgfpathlineto{\pgfqpoint{4.904517in}{2.299807in}}%
\pgfpathlineto{\pgfqpoint{4.896765in}{2.288028in}}%
\pgfpathlineto{\pgfqpoint{4.882594in}{2.283792in}}%
\pgfpathlineto{\pgfqpoint{4.868435in}{2.279656in}}%
\pgfpathlineto{\pgfqpoint{4.854289in}{2.275619in}}%
\pgfpathlineto{\pgfqpoint{4.840154in}{2.271682in}}%
\pgfpathlineto{\pgfqpoint{4.847909in}{2.283276in}}%
\pgfpathlineto{\pgfqpoint{4.855658in}{2.294815in}}%
\pgfpathlineto{\pgfqpoint{4.863402in}{2.306298in}}%
\pgfpathlineto{\pgfqpoint{4.871140in}{2.317722in}}%
\pgfpathclose%
\pgfusepath{fill}%
\end{pgfscope}%
\begin{pgfscope}%
\pgfpathrectangle{\pgfqpoint{1.150000in}{0.150000in}}{\pgfqpoint{5.700000in}{5.700000in}}%
\pgfusepath{clip}%
\pgfsetbuttcap%
\pgfsetroundjoin%
\definecolor{currentfill}{rgb}{0.165117,0.467423,0.558141}%
\pgfsetfillcolor{currentfill}%
\pgfsetfillopacity{0.700000}%
\pgfsetlinewidth{0.000000pt}%
\definecolor{currentstroke}{rgb}{0.000000,0.000000,0.000000}%
\pgfsetstrokecolor{currentstroke}%
\pgfsetdash{}{0pt}%
\pgfpathmoveto{\pgfqpoint{5.545201in}{2.855731in}}%
\pgfpathlineto{\pgfqpoint{5.559680in}{2.863394in}}%
\pgfpathlineto{\pgfqpoint{5.574175in}{2.871157in}}%
\pgfpathlineto{\pgfqpoint{5.588685in}{2.879021in}}%
\pgfpathlineto{\pgfqpoint{5.603211in}{2.886985in}}%
\pgfpathlineto{\pgfqpoint{5.595758in}{2.878232in}}%
\pgfpathlineto{\pgfqpoint{5.588297in}{2.869361in}}%
\pgfpathlineto{\pgfqpoint{5.580828in}{2.860373in}}%
\pgfpathlineto{\pgfqpoint{5.573350in}{2.851267in}}%
\pgfpathlineto{\pgfqpoint{5.558817in}{2.843309in}}%
\pgfpathlineto{\pgfqpoint{5.544300in}{2.835451in}}%
\pgfpathlineto{\pgfqpoint{5.529798in}{2.827694in}}%
\pgfpathlineto{\pgfqpoint{5.515312in}{2.820037in}}%
\pgfpathlineto{\pgfqpoint{5.522796in}{2.829130in}}%
\pgfpathlineto{\pgfqpoint{5.530273in}{2.838110in}}%
\pgfpathlineto{\pgfqpoint{5.537741in}{2.846977in}}%
\pgfpathlineto{\pgfqpoint{5.545201in}{2.855731in}}%
\pgfpathclose%
\pgfusepath{fill}%
\end{pgfscope}%
\begin{pgfscope}%
\pgfpathrectangle{\pgfqpoint{1.150000in}{0.150000in}}{\pgfqpoint{5.700000in}{5.700000in}}%
\pgfusepath{clip}%
\pgfsetbuttcap%
\pgfsetroundjoin%
\definecolor{currentfill}{rgb}{0.282656,0.100196,0.422160}%
\pgfsetfillcolor{currentfill}%
\pgfsetfillopacity{0.700000}%
\pgfsetlinewidth{0.000000pt}%
\definecolor{currentstroke}{rgb}{0.000000,0.000000,0.000000}%
\pgfsetstrokecolor{currentstroke}%
\pgfsetdash{}{0pt}%
\pgfpathmoveto{\pgfqpoint{3.231017in}{2.009351in}}%
\pgfpathlineto{\pgfqpoint{3.244739in}{1.999884in}}%
\pgfpathlineto{\pgfqpoint{3.258463in}{1.990541in}}%
\pgfpathlineto{\pgfqpoint{3.272188in}{1.981323in}}%
\pgfpathlineto{\pgfqpoint{3.285914in}{1.972228in}}%
\pgfpathlineto{\pgfqpoint{3.277533in}{1.970405in}}%
\pgfpathlineto{\pgfqpoint{3.269140in}{1.968791in}}%
\pgfpathlineto{\pgfqpoint{3.260735in}{1.967390in}}%
\pgfpathlineto{\pgfqpoint{3.252318in}{1.966208in}}%
\pgfpathlineto{\pgfqpoint{3.238561in}{1.975757in}}%
\pgfpathlineto{\pgfqpoint{3.224806in}{1.985429in}}%
\pgfpathlineto{\pgfqpoint{3.211052in}{1.995225in}}%
\pgfpathlineto{\pgfqpoint{3.197299in}{2.005147in}}%
\pgfpathlineto{\pgfqpoint{3.205747in}{2.005867in}}%
\pgfpathlineto{\pgfqpoint{3.214182in}{2.006811in}}%
\pgfpathlineto{\pgfqpoint{3.222606in}{2.007974in}}%
\pgfpathlineto{\pgfqpoint{3.231017in}{2.009351in}}%
\pgfpathclose%
\pgfusepath{fill}%
\end{pgfscope}%
\begin{pgfscope}%
\pgfpathrectangle{\pgfqpoint{1.150000in}{0.150000in}}{\pgfqpoint{5.700000in}{5.700000in}}%
\pgfusepath{clip}%
\pgfsetbuttcap%
\pgfsetroundjoin%
\definecolor{currentfill}{rgb}{0.266580,0.228262,0.514349}%
\pgfsetfillcolor{currentfill}%
\pgfsetfillopacity{0.700000}%
\pgfsetlinewidth{0.000000pt}%
\definecolor{currentstroke}{rgb}{0.000000,0.000000,0.000000}%
\pgfsetstrokecolor{currentstroke}%
\pgfsetdash{}{0pt}%
\pgfpathmoveto{\pgfqpoint{2.867259in}{2.282678in}}%
\pgfpathlineto{\pgfqpoint{2.881021in}{2.269526in}}%
\pgfpathlineto{\pgfqpoint{2.894781in}{2.256518in}}%
\pgfpathlineto{\pgfqpoint{2.908539in}{2.243655in}}%
\pgfpathlineto{\pgfqpoint{2.922295in}{2.230934in}}%
\pgfpathlineto{\pgfqpoint{2.913657in}{2.232795in}}%
\pgfpathlineto{\pgfqpoint{2.905004in}{2.234915in}}%
\pgfpathlineto{\pgfqpoint{2.896335in}{2.237299in}}%
\pgfpathlineto{\pgfqpoint{2.887650in}{2.239951in}}%
\pgfpathlineto{\pgfqpoint{2.873854in}{2.253155in}}%
\pgfpathlineto{\pgfqpoint{2.860055in}{2.266502in}}%
\pgfpathlineto{\pgfqpoint{2.846255in}{2.279994in}}%
\pgfpathlineto{\pgfqpoint{2.832453in}{2.293632in}}%
\pgfpathlineto{\pgfqpoint{2.841179in}{2.290487in}}%
\pgfpathlineto{\pgfqpoint{2.849888in}{2.287617in}}%
\pgfpathlineto{\pgfqpoint{2.858582in}{2.285016in}}%
\pgfpathlineto{\pgfqpoint{2.867259in}{2.282678in}}%
\pgfpathclose%
\pgfusepath{fill}%
\end{pgfscope}%
\begin{pgfscope}%
\pgfpathrectangle{\pgfqpoint{1.150000in}{0.150000in}}{\pgfqpoint{5.700000in}{5.700000in}}%
\pgfusepath{clip}%
\pgfsetbuttcap%
\pgfsetroundjoin%
\definecolor{currentfill}{rgb}{0.166617,0.463708,0.558119}%
\pgfsetfillcolor{currentfill}%
\pgfsetfillopacity{0.700000}%
\pgfsetlinewidth{0.000000pt}%
\definecolor{currentstroke}{rgb}{0.000000,0.000000,0.000000}%
\pgfsetstrokecolor{currentstroke}%
\pgfsetdash{}{0pt}%
\pgfpathmoveto{\pgfqpoint{2.369361in}{2.862802in}}%
\pgfpathlineto{\pgfqpoint{2.383291in}{2.843591in}}%
\pgfpathlineto{\pgfqpoint{2.397214in}{2.824575in}}%
\pgfpathlineto{\pgfqpoint{2.411129in}{2.805752in}}%
\pgfpathlineto{\pgfqpoint{2.425037in}{2.787122in}}%
\pgfpathlineto{\pgfqpoint{2.415995in}{2.793361in}}%
\pgfpathlineto{\pgfqpoint{2.406933in}{2.799909in}}%
\pgfpathlineto{\pgfqpoint{2.397849in}{2.806770in}}%
\pgfpathlineto{\pgfqpoint{2.388743in}{2.813951in}}%
\pgfpathlineto{\pgfqpoint{2.374782in}{2.833091in}}%
\pgfpathlineto{\pgfqpoint{2.360814in}{2.852424in}}%
\pgfpathlineto{\pgfqpoint{2.346838in}{2.871952in}}%
\pgfpathlineto{\pgfqpoint{2.332854in}{2.891676in}}%
\pgfpathlineto{\pgfqpoint{2.342014in}{2.883975in}}%
\pgfpathlineto{\pgfqpoint{2.351151in}{2.876600in}}%
\pgfpathlineto{\pgfqpoint{2.360267in}{2.869544in}}%
\pgfpathlineto{\pgfqpoint{2.369361in}{2.862802in}}%
\pgfpathclose%
\pgfusepath{fill}%
\end{pgfscope}%
\begin{pgfscope}%
\pgfpathrectangle{\pgfqpoint{1.150000in}{0.150000in}}{\pgfqpoint{5.700000in}{5.700000in}}%
\pgfusepath{clip}%
\pgfsetbuttcap%
\pgfsetroundjoin%
\definecolor{currentfill}{rgb}{0.267004,0.004874,0.329415}%
\pgfsetfillcolor{currentfill}%
\pgfsetfillopacity{0.700000}%
\pgfsetlinewidth{0.000000pt}%
\definecolor{currentstroke}{rgb}{0.000000,0.000000,0.000000}%
\pgfsetstrokecolor{currentstroke}%
\pgfsetdash{}{0pt}%
\pgfpathmoveto{\pgfqpoint{3.769182in}{1.836504in}}%
\pgfpathlineto{\pgfqpoint{3.782944in}{1.831978in}}%
\pgfpathlineto{\pgfqpoint{3.796712in}{1.827561in}}%
\pgfpathlineto{\pgfqpoint{3.810486in}{1.823251in}}%
\pgfpathlineto{\pgfqpoint{3.824265in}{1.819049in}}%
\pgfpathlineto{\pgfqpoint{3.816172in}{1.811966in}}%
\pgfpathlineto{\pgfqpoint{3.808071in}{1.805004in}}%
\pgfpathlineto{\pgfqpoint{3.799964in}{1.798167in}}%
\pgfpathlineto{\pgfqpoint{3.791849in}{1.791458in}}%
\pgfpathlineto{\pgfqpoint{3.778053in}{1.796052in}}%
\pgfpathlineto{\pgfqpoint{3.764262in}{1.800754in}}%
\pgfpathlineto{\pgfqpoint{3.750477in}{1.805563in}}%
\pgfpathlineto{\pgfqpoint{3.736697in}{1.810481in}}%
\pgfpathlineto{\pgfqpoint{3.744829in}{1.816790in}}%
\pgfpathlineto{\pgfqpoint{3.752954in}{1.823233in}}%
\pgfpathlineto{\pgfqpoint{3.761071in}{1.829806in}}%
\pgfpathlineto{\pgfqpoint{3.769182in}{1.836504in}}%
\pgfpathclose%
\pgfusepath{fill}%
\end{pgfscope}%
\begin{pgfscope}%
\pgfpathrectangle{\pgfqpoint{1.150000in}{0.150000in}}{\pgfqpoint{5.700000in}{5.700000in}}%
\pgfusepath{clip}%
\pgfsetbuttcap%
\pgfsetroundjoin%
\definecolor{currentfill}{rgb}{0.269944,0.014625,0.341379}%
\pgfsetfillcolor{currentfill}%
\pgfsetfillopacity{0.700000}%
\pgfsetlinewidth{0.000000pt}%
\definecolor{currentstroke}{rgb}{0.000000,0.000000,0.000000}%
\pgfsetstrokecolor{currentstroke}%
\pgfsetdash{}{0pt}%
\pgfpathmoveto{\pgfqpoint{3.626631in}{1.853764in}}%
\pgfpathlineto{\pgfqpoint{3.640373in}{1.847967in}}%
\pgfpathlineto{\pgfqpoint{3.654119in}{1.842281in}}%
\pgfpathlineto{\pgfqpoint{3.667870in}{1.836706in}}%
\pgfpathlineto{\pgfqpoint{3.681626in}{1.831241in}}%
\pgfpathlineto{\pgfqpoint{3.673467in}{1.825473in}}%
\pgfpathlineto{\pgfqpoint{3.665300in}{1.819851in}}%
\pgfpathlineto{\pgfqpoint{3.657124in}{1.814379in}}%
\pgfpathlineto{\pgfqpoint{3.648941in}{1.809061in}}%
\pgfpathlineto{\pgfqpoint{3.635165in}{1.814936in}}%
\pgfpathlineto{\pgfqpoint{3.621394in}{1.820922in}}%
\pgfpathlineto{\pgfqpoint{3.607626in}{1.827019in}}%
\pgfpathlineto{\pgfqpoint{3.593863in}{1.833227in}}%
\pgfpathlineto{\pgfqpoint{3.602068in}{1.838127in}}%
\pgfpathlineto{\pgfqpoint{3.610264in}{1.843186in}}%
\pgfpathlineto{\pgfqpoint{3.618452in}{1.848399in}}%
\pgfpathlineto{\pgfqpoint{3.626631in}{1.853764in}}%
\pgfpathclose%
\pgfusepath{fill}%
\end{pgfscope}%
\begin{pgfscope}%
\pgfpathrectangle{\pgfqpoint{1.150000in}{0.150000in}}{\pgfqpoint{5.700000in}{5.700000in}}%
\pgfusepath{clip}%
\pgfsetbuttcap%
\pgfsetroundjoin%
\definecolor{currentfill}{rgb}{0.203063,0.379716,0.553925}%
\pgfsetfillcolor{currentfill}%
\pgfsetfillopacity{0.700000}%
\pgfsetlinewidth{0.000000pt}%
\definecolor{currentstroke}{rgb}{0.000000,0.000000,0.000000}%
\pgfsetstrokecolor{currentstroke}%
\pgfsetdash{}{0pt}%
\pgfpathmoveto{\pgfqpoint{5.252018in}{2.619654in}}%
\pgfpathlineto{\pgfqpoint{5.266343in}{2.625973in}}%
\pgfpathlineto{\pgfqpoint{5.280682in}{2.632392in}}%
\pgfpathlineto{\pgfqpoint{5.295035in}{2.638911in}}%
\pgfpathlineto{\pgfqpoint{5.309403in}{2.645530in}}%
\pgfpathlineto{\pgfqpoint{5.301808in}{2.635122in}}%
\pgfpathlineto{\pgfqpoint{5.294206in}{2.624614in}}%
\pgfpathlineto{\pgfqpoint{5.286597in}{2.614004in}}%
\pgfpathlineto{\pgfqpoint{5.278981in}{2.603294in}}%
\pgfpathlineto{\pgfqpoint{5.264609in}{2.596759in}}%
\pgfpathlineto{\pgfqpoint{5.250252in}{2.590323in}}%
\pgfpathlineto{\pgfqpoint{5.235909in}{2.583988in}}%
\pgfpathlineto{\pgfqpoint{5.221580in}{2.577753in}}%
\pgfpathlineto{\pgfqpoint{5.229199in}{2.588372in}}%
\pgfpathlineto{\pgfqpoint{5.236812in}{2.598896in}}%
\pgfpathlineto{\pgfqpoint{5.244419in}{2.609323in}}%
\pgfpathlineto{\pgfqpoint{5.252018in}{2.619654in}}%
\pgfpathclose%
\pgfusepath{fill}%
\end{pgfscope}%
\begin{pgfscope}%
\pgfpathrectangle{\pgfqpoint{1.150000in}{0.150000in}}{\pgfqpoint{5.700000in}{5.700000in}}%
\pgfusepath{clip}%
\pgfsetbuttcap%
\pgfsetroundjoin%
\definecolor{currentfill}{rgb}{0.271828,0.209303,0.504434}%
\pgfsetfillcolor{currentfill}%
\pgfsetfillopacity{0.700000}%
\pgfsetlinewidth{0.000000pt}%
\definecolor{currentstroke}{rgb}{0.000000,0.000000,0.000000}%
\pgfsetstrokecolor{currentstroke}%
\pgfsetdash{}{0pt}%
\pgfpathmoveto{\pgfqpoint{2.922295in}{2.230934in}}%
\pgfpathlineto{\pgfqpoint{2.936051in}{2.218356in}}%
\pgfpathlineto{\pgfqpoint{2.949805in}{2.205918in}}%
\pgfpathlineto{\pgfqpoint{2.963557in}{2.193620in}}%
\pgfpathlineto{\pgfqpoint{2.977309in}{2.181462in}}%
\pgfpathlineto{\pgfqpoint{2.968709in}{2.182849in}}%
\pgfpathlineto{\pgfqpoint{2.960094in}{2.184489in}}%
\pgfpathlineto{\pgfqpoint{2.951464in}{2.186388in}}%
\pgfpathlineto{\pgfqpoint{2.942819in}{2.188550in}}%
\pgfpathlineto{\pgfqpoint{2.929029in}{2.201190in}}%
\pgfpathlineto{\pgfqpoint{2.915237in}{2.213969in}}%
\pgfpathlineto{\pgfqpoint{2.901444in}{2.226889in}}%
\pgfpathlineto{\pgfqpoint{2.887650in}{2.239951in}}%
\pgfpathlineto{\pgfqpoint{2.896335in}{2.237299in}}%
\pgfpathlineto{\pgfqpoint{2.905004in}{2.234915in}}%
\pgfpathlineto{\pgfqpoint{2.913657in}{2.232795in}}%
\pgfpathlineto{\pgfqpoint{2.922295in}{2.230934in}}%
\pgfpathclose%
\pgfusepath{fill}%
\end{pgfscope}%
\begin{pgfscope}%
\pgfpathrectangle{\pgfqpoint{1.150000in}{0.150000in}}{\pgfqpoint{5.700000in}{5.700000in}}%
\pgfusepath{clip}%
\pgfsetbuttcap%
\pgfsetroundjoin%
\definecolor{currentfill}{rgb}{0.248629,0.278775,0.534556}%
\pgfsetfillcolor{currentfill}%
\pgfsetfillopacity{0.700000}%
\pgfsetlinewidth{0.000000pt}%
\definecolor{currentstroke}{rgb}{0.000000,0.000000,0.000000}%
\pgfsetstrokecolor{currentstroke}%
\pgfsetdash{}{0pt}%
\pgfpathmoveto{\pgfqpoint{4.958623in}{2.380500in}}%
\pgfpathlineto{\pgfqpoint{4.972801in}{2.385172in}}%
\pgfpathlineto{\pgfqpoint{4.986992in}{2.389944in}}%
\pgfpathlineto{\pgfqpoint{5.001195in}{2.394816in}}%
\pgfpathlineto{\pgfqpoint{5.015412in}{2.399787in}}%
\pgfpathlineto{\pgfqpoint{5.007703in}{2.388306in}}%
\pgfpathlineto{\pgfqpoint{4.999988in}{2.376751in}}%
\pgfpathlineto{\pgfqpoint{4.992267in}{2.365125in}}%
\pgfpathlineto{\pgfqpoint{4.984540in}{2.353429in}}%
\pgfpathlineto{\pgfqpoint{4.970321in}{2.348617in}}%
\pgfpathlineto{\pgfqpoint{4.956114in}{2.343904in}}%
\pgfpathlineto{\pgfqpoint{4.941920in}{2.339291in}}%
\pgfpathlineto{\pgfqpoint{4.927739in}{2.334777in}}%
\pgfpathlineto{\pgfqpoint{4.935469in}{2.346308in}}%
\pgfpathlineto{\pgfqpoint{4.943192in}{2.357773in}}%
\pgfpathlineto{\pgfqpoint{4.950910in}{2.369170in}}%
\pgfpathlineto{\pgfqpoint{4.958623in}{2.380500in}}%
\pgfpathclose%
\pgfusepath{fill}%
\end{pgfscope}%
\begin{pgfscope}%
\pgfpathrectangle{\pgfqpoint{1.150000in}{0.150000in}}{\pgfqpoint{5.700000in}{5.700000in}}%
\pgfusepath{clip}%
\pgfsetbuttcap%
\pgfsetroundjoin%
\definecolor{currentfill}{rgb}{0.282327,0.094955,0.417331}%
\pgfsetfillcolor{currentfill}%
\pgfsetfillopacity{0.700000}%
\pgfsetlinewidth{0.000000pt}%
\definecolor{currentstroke}{rgb}{0.000000,0.000000,0.000000}%
\pgfsetstrokecolor{currentstroke}%
\pgfsetdash{}{0pt}%
\pgfpathmoveto{\pgfqpoint{4.403288in}{1.996432in}}%
\pgfpathlineto{\pgfqpoint{4.417226in}{1.997208in}}%
\pgfpathlineto{\pgfqpoint{4.431174in}{1.998085in}}%
\pgfpathlineto{\pgfqpoint{4.445131in}{1.999063in}}%
\pgfpathlineto{\pgfqpoint{4.459098in}{2.000141in}}%
\pgfpathlineto{\pgfqpoint{4.451222in}{1.988945in}}%
\pgfpathlineto{\pgfqpoint{4.443340in}{1.977758in}}%
\pgfpathlineto{\pgfqpoint{4.435454in}{1.966582in}}%
\pgfpathlineto{\pgfqpoint{4.427563in}{1.955420in}}%
\pgfpathlineto{\pgfqpoint{4.413591in}{1.954626in}}%
\pgfpathlineto{\pgfqpoint{4.399628in}{1.953933in}}%
\pgfpathlineto{\pgfqpoint{4.385674in}{1.953341in}}%
\pgfpathlineto{\pgfqpoint{4.371730in}{1.952850in}}%
\pgfpathlineto{\pgfqpoint{4.379627in}{1.963720in}}%
\pgfpathlineto{\pgfqpoint{4.387519in}{1.974609in}}%
\pgfpathlineto{\pgfqpoint{4.395406in}{1.985514in}}%
\pgfpathlineto{\pgfqpoint{4.403288in}{1.996432in}}%
\pgfpathclose%
\pgfusepath{fill}%
\end{pgfscope}%
\begin{pgfscope}%
\pgfpathrectangle{\pgfqpoint{1.150000in}{0.150000in}}{\pgfqpoint{5.700000in}{5.700000in}}%
\pgfusepath{clip}%
\pgfsetbuttcap%
\pgfsetroundjoin%
\definecolor{currentfill}{rgb}{0.280267,0.073417,0.397163}%
\pgfsetfillcolor{currentfill}%
\pgfsetfillopacity{0.700000}%
\pgfsetlinewidth{0.000000pt}%
\definecolor{currentstroke}{rgb}{0.000000,0.000000,0.000000}%
\pgfsetstrokecolor{currentstroke}%
\pgfsetdash{}{0pt}%
\pgfpathmoveto{\pgfqpoint{4.316046in}{1.951896in}}%
\pgfpathlineto{\pgfqpoint{4.329953in}{1.951982in}}%
\pgfpathlineto{\pgfqpoint{4.343869in}{1.952170in}}%
\pgfpathlineto{\pgfqpoint{4.357795in}{1.952460in}}%
\pgfpathlineto{\pgfqpoint{4.371730in}{1.952850in}}%
\pgfpathlineto{\pgfqpoint{4.363828in}{1.942001in}}%
\pgfpathlineto{\pgfqpoint{4.355921in}{1.931176in}}%
\pgfpathlineto{\pgfqpoint{4.348009in}{1.920378in}}%
\pgfpathlineto{\pgfqpoint{4.340092in}{1.909609in}}%
\pgfpathlineto{\pgfqpoint{4.326151in}{1.909521in}}%
\pgfpathlineto{\pgfqpoint{4.312218in}{1.909535in}}%
\pgfpathlineto{\pgfqpoint{4.298295in}{1.909649in}}%
\pgfpathlineto{\pgfqpoint{4.284380in}{1.909865in}}%
\pgfpathlineto{\pgfqpoint{4.292304in}{1.920324in}}%
\pgfpathlineto{\pgfqpoint{4.300223in}{1.930818in}}%
\pgfpathlineto{\pgfqpoint{4.308137in}{1.941342in}}%
\pgfpathlineto{\pgfqpoint{4.316046in}{1.951896in}}%
\pgfpathclose%
\pgfusepath{fill}%
\end{pgfscope}%
\begin{pgfscope}%
\pgfpathrectangle{\pgfqpoint{1.150000in}{0.150000in}}{\pgfqpoint{5.700000in}{5.700000in}}%
\pgfusepath{clip}%
\pgfsetbuttcap%
\pgfsetroundjoin%
\definecolor{currentfill}{rgb}{0.154815,0.493313,0.557840}%
\pgfsetfillcolor{currentfill}%
\pgfsetfillopacity{0.700000}%
\pgfsetlinewidth{0.000000pt}%
\definecolor{currentstroke}{rgb}{0.000000,0.000000,0.000000}%
\pgfsetstrokecolor{currentstroke}%
\pgfsetdash{}{0pt}%
\pgfpathmoveto{\pgfqpoint{2.313556in}{2.941638in}}%
\pgfpathlineto{\pgfqpoint{2.327520in}{2.921626in}}%
\pgfpathlineto{\pgfqpoint{2.341476in}{2.901818in}}%
\pgfpathlineto{\pgfqpoint{2.355422in}{2.882210in}}%
\pgfpathlineto{\pgfqpoint{2.369361in}{2.862802in}}%
\pgfpathlineto{\pgfqpoint{2.360267in}{2.869544in}}%
\pgfpathlineto{\pgfqpoint{2.351151in}{2.876600in}}%
\pgfpathlineto{\pgfqpoint{2.342014in}{2.883975in}}%
\pgfpathlineto{\pgfqpoint{2.332854in}{2.891676in}}%
\pgfpathlineto{\pgfqpoint{2.318861in}{2.911598in}}%
\pgfpathlineto{\pgfqpoint{2.304859in}{2.931720in}}%
\pgfpathlineto{\pgfqpoint{2.290849in}{2.952044in}}%
\pgfpathlineto{\pgfqpoint{2.276830in}{2.972573in}}%
\pgfpathlineto{\pgfqpoint{2.286046in}{2.964348in}}%
\pgfpathlineto{\pgfqpoint{2.295239in}{2.956454in}}%
\pgfpathlineto{\pgfqpoint{2.304409in}{2.948886in}}%
\pgfpathlineto{\pgfqpoint{2.313556in}{2.941638in}}%
\pgfpathclose%
\pgfusepath{fill}%
\end{pgfscope}%
\begin{pgfscope}%
\pgfpathrectangle{\pgfqpoint{1.150000in}{0.150000in}}{\pgfqpoint{5.700000in}{5.700000in}}%
\pgfusepath{clip}%
\pgfsetbuttcap%
\pgfsetroundjoin%
\definecolor{currentfill}{rgb}{0.283229,0.120777,0.440584}%
\pgfsetfillcolor{currentfill}%
\pgfsetfillopacity{0.700000}%
\pgfsetlinewidth{0.000000pt}%
\definecolor{currentstroke}{rgb}{0.000000,0.000000,0.000000}%
\pgfsetstrokecolor{currentstroke}%
\pgfsetdash{}{0pt}%
\pgfpathmoveto{\pgfqpoint{4.490555in}{2.044964in}}%
\pgfpathlineto{\pgfqpoint{4.504527in}{2.046410in}}%
\pgfpathlineto{\pgfqpoint{4.518509in}{2.047956in}}%
\pgfpathlineto{\pgfqpoint{4.532501in}{2.049603in}}%
\pgfpathlineto{\pgfqpoint{4.546503in}{2.051351in}}%
\pgfpathlineto{\pgfqpoint{4.538651in}{2.039884in}}%
\pgfpathlineto{\pgfqpoint{4.530794in}{2.028412in}}%
\pgfpathlineto{\pgfqpoint{4.522933in}{2.016936in}}%
\pgfpathlineto{\pgfqpoint{4.515066in}{2.005460in}}%
\pgfpathlineto{\pgfqpoint{4.501059in}{2.003980in}}%
\pgfpathlineto{\pgfqpoint{4.487062in}{2.002600in}}%
\pgfpathlineto{\pgfqpoint{4.473075in}{2.001320in}}%
\pgfpathlineto{\pgfqpoint{4.459098in}{2.000141in}}%
\pgfpathlineto{\pgfqpoint{4.466970in}{2.011344in}}%
\pgfpathlineto{\pgfqpoint{4.474836in}{2.022550in}}%
\pgfpathlineto{\pgfqpoint{4.482698in}{2.033757in}}%
\pgfpathlineto{\pgfqpoint{4.490555in}{2.044964in}}%
\pgfpathclose%
\pgfusepath{fill}%
\end{pgfscope}%
\begin{pgfscope}%
\pgfpathrectangle{\pgfqpoint{1.150000in}{0.150000in}}{\pgfqpoint{5.700000in}{5.700000in}}%
\pgfusepath{clip}%
\pgfsetbuttcap%
\pgfsetroundjoin%
\definecolor{currentfill}{rgb}{0.154815,0.493313,0.557840}%
\pgfsetfillcolor{currentfill}%
\pgfsetfillopacity{0.700000}%
\pgfsetlinewidth{0.000000pt}%
\definecolor{currentstroke}{rgb}{0.000000,0.000000,0.000000}%
\pgfsetstrokecolor{currentstroke}%
\pgfsetdash{}{0pt}%
\pgfpathmoveto{\pgfqpoint{5.632935in}{2.920829in}}%
\pgfpathlineto{\pgfqpoint{5.647469in}{2.928880in}}%
\pgfpathlineto{\pgfqpoint{5.662019in}{2.937032in}}%
\pgfpathlineto{\pgfqpoint{5.676584in}{2.945284in}}%
\pgfpathlineto{\pgfqpoint{5.691166in}{2.953637in}}%
\pgfpathlineto{\pgfqpoint{5.683756in}{2.945371in}}%
\pgfpathlineto{\pgfqpoint{5.676338in}{2.936984in}}%
\pgfpathlineto{\pgfqpoint{5.668910in}{2.928477in}}%
\pgfpathlineto{\pgfqpoint{5.661474in}{2.919847in}}%
\pgfpathlineto{\pgfqpoint{5.646884in}{2.911480in}}%
\pgfpathlineto{\pgfqpoint{5.632310in}{2.903215in}}%
\pgfpathlineto{\pgfqpoint{5.617753in}{2.895049in}}%
\pgfpathlineto{\pgfqpoint{5.603211in}{2.886985in}}%
\pgfpathlineto{\pgfqpoint{5.610655in}{2.895620in}}%
\pgfpathlineto{\pgfqpoint{5.618090in}{2.904140in}}%
\pgfpathlineto{\pgfqpoint{5.625517in}{2.912542in}}%
\pgfpathlineto{\pgfqpoint{5.632935in}{2.920829in}}%
\pgfpathclose%
\pgfusepath{fill}%
\end{pgfscope}%
\begin{pgfscope}%
\pgfpathrectangle{\pgfqpoint{1.150000in}{0.150000in}}{\pgfqpoint{5.700000in}{5.700000in}}%
\pgfusepath{clip}%
\pgfsetbuttcap%
\pgfsetroundjoin%
\definecolor{currentfill}{rgb}{0.267004,0.004874,0.329415}%
\pgfsetfillcolor{currentfill}%
\pgfsetfillopacity{0.700000}%
\pgfsetlinewidth{0.000000pt}%
\definecolor{currentstroke}{rgb}{0.000000,0.000000,0.000000}%
\pgfsetstrokecolor{currentstroke}%
\pgfsetdash{}{0pt}%
\pgfpathmoveto{\pgfqpoint{3.911684in}{1.834274in}}%
\pgfpathlineto{\pgfqpoint{3.925478in}{1.830978in}}%
\pgfpathlineto{\pgfqpoint{3.939279in}{1.827788in}}%
\pgfpathlineto{\pgfqpoint{3.953086in}{1.824703in}}%
\pgfpathlineto{\pgfqpoint{3.966900in}{1.821723in}}%
\pgfpathlineto{\pgfqpoint{3.958862in}{1.813459in}}%
\pgfpathlineto{\pgfqpoint{3.950819in}{1.805292in}}%
\pgfpathlineto{\pgfqpoint{3.942769in}{1.797225in}}%
\pgfpathlineto{\pgfqpoint{3.934713in}{1.789264in}}%
\pgfpathlineto{\pgfqpoint{3.920885in}{1.792617in}}%
\pgfpathlineto{\pgfqpoint{3.907064in}{1.796076in}}%
\pgfpathlineto{\pgfqpoint{3.893249in}{1.799640in}}%
\pgfpathlineto{\pgfqpoint{3.879440in}{1.803309in}}%
\pgfpathlineto{\pgfqpoint{3.887511in}{1.810890in}}%
\pgfpathlineto{\pgfqpoint{3.895575in}{1.818581in}}%
\pgfpathlineto{\pgfqpoint{3.903633in}{1.826376in}}%
\pgfpathlineto{\pgfqpoint{3.911684in}{1.834274in}}%
\pgfpathclose%
\pgfusepath{fill}%
\end{pgfscope}%
\begin{pgfscope}%
\pgfpathrectangle{\pgfqpoint{1.150000in}{0.150000in}}{\pgfqpoint{5.700000in}{5.700000in}}%
\pgfusepath{clip}%
\pgfsetbuttcap%
\pgfsetroundjoin%
\definecolor{currentfill}{rgb}{0.273809,0.031497,0.358853}%
\pgfsetfillcolor{currentfill}%
\pgfsetfillopacity{0.700000}%
\pgfsetlinewidth{0.000000pt}%
\definecolor{currentstroke}{rgb}{0.000000,0.000000,0.000000}%
\pgfsetstrokecolor{currentstroke}%
\pgfsetdash{}{0pt}%
\pgfpathmoveto{\pgfqpoint{3.483898in}{1.886959in}}%
\pgfpathlineto{\pgfqpoint{3.497631in}{1.879843in}}%
\pgfpathlineto{\pgfqpoint{3.511368in}{1.872842in}}%
\pgfpathlineto{\pgfqpoint{3.525108in}{1.865956in}}%
\pgfpathlineto{\pgfqpoint{3.538851in}{1.859184in}}%
\pgfpathlineto{\pgfqpoint{3.530615in}{1.854870in}}%
\pgfpathlineto{\pgfqpoint{3.522370in}{1.850729in}}%
\pgfpathlineto{\pgfqpoint{3.514116in}{1.846764in}}%
\pgfpathlineto{\pgfqpoint{3.505851in}{1.842979in}}%
\pgfpathlineto{\pgfqpoint{3.492084in}{1.850182in}}%
\pgfpathlineto{\pgfqpoint{3.478320in}{1.857499in}}%
\pgfpathlineto{\pgfqpoint{3.464559in}{1.864931in}}%
\pgfpathlineto{\pgfqpoint{3.450802in}{1.872478in}}%
\pgfpathlineto{\pgfqpoint{3.459090in}{1.875824in}}%
\pgfpathlineto{\pgfqpoint{3.467370in}{1.879356in}}%
\pgfpathlineto{\pgfqpoint{3.475639in}{1.883069in}}%
\pgfpathlineto{\pgfqpoint{3.483898in}{1.886959in}}%
\pgfpathclose%
\pgfusepath{fill}%
\end{pgfscope}%
\begin{pgfscope}%
\pgfpathrectangle{\pgfqpoint{1.150000in}{0.150000in}}{\pgfqpoint{5.700000in}{5.700000in}}%
\pgfusepath{clip}%
\pgfsetbuttcap%
\pgfsetroundjoin%
\definecolor{currentfill}{rgb}{0.277018,0.050344,0.375715}%
\pgfsetfillcolor{currentfill}%
\pgfsetfillopacity{0.700000}%
\pgfsetlinewidth{0.000000pt}%
\definecolor{currentstroke}{rgb}{0.000000,0.000000,0.000000}%
\pgfsetstrokecolor{currentstroke}%
\pgfsetdash{}{0pt}%
\pgfpathmoveto{\pgfqpoint{4.228808in}{1.911745in}}%
\pgfpathlineto{\pgfqpoint{4.242688in}{1.911122in}}%
\pgfpathlineto{\pgfqpoint{4.256577in}{1.910601in}}%
\pgfpathlineto{\pgfqpoint{4.270474in}{1.910182in}}%
\pgfpathlineto{\pgfqpoint{4.284380in}{1.909865in}}%
\pgfpathlineto{\pgfqpoint{4.276451in}{1.899443in}}%
\pgfpathlineto{\pgfqpoint{4.268517in}{1.889061in}}%
\pgfpathlineto{\pgfqpoint{4.260578in}{1.878722in}}%
\pgfpathlineto{\pgfqpoint{4.252633in}{1.868429in}}%
\pgfpathlineto{\pgfqpoint{4.238720in}{1.869067in}}%
\pgfpathlineto{\pgfqpoint{4.224814in}{1.869806in}}%
\pgfpathlineto{\pgfqpoint{4.210917in}{1.870647in}}%
\pgfpathlineto{\pgfqpoint{4.197028in}{1.871589in}}%
\pgfpathlineto{\pgfqpoint{4.204981in}{1.881555in}}%
\pgfpathlineto{\pgfqpoint{4.212929in}{1.891572in}}%
\pgfpathlineto{\pgfqpoint{4.220871in}{1.901636in}}%
\pgfpathlineto{\pgfqpoint{4.228808in}{1.911745in}}%
\pgfpathclose%
\pgfusepath{fill}%
\end{pgfscope}%
\begin{pgfscope}%
\pgfpathrectangle{\pgfqpoint{1.150000in}{0.150000in}}{\pgfqpoint{5.700000in}{5.700000in}}%
\pgfusepath{clip}%
\pgfsetbuttcap%
\pgfsetroundjoin%
\definecolor{currentfill}{rgb}{0.282290,0.145912,0.461510}%
\pgfsetfillcolor{currentfill}%
\pgfsetfillopacity{0.700000}%
\pgfsetlinewidth{0.000000pt}%
\definecolor{currentstroke}{rgb}{0.000000,0.000000,0.000000}%
\pgfsetstrokecolor{currentstroke}%
\pgfsetdash{}{0pt}%
\pgfpathmoveto{\pgfqpoint{4.577862in}{2.097113in}}%
\pgfpathlineto{\pgfqpoint{4.591871in}{2.099210in}}%
\pgfpathlineto{\pgfqpoint{4.605890in}{2.101407in}}%
\pgfpathlineto{\pgfqpoint{4.619920in}{2.103704in}}%
\pgfpathlineto{\pgfqpoint{4.633961in}{2.106101in}}%
\pgfpathlineto{\pgfqpoint{4.626132in}{2.094439in}}%
\pgfpathlineto{\pgfqpoint{4.618299in}{2.082756in}}%
\pgfpathlineto{\pgfqpoint{4.610461in}{2.071056in}}%
\pgfpathlineto{\pgfqpoint{4.602618in}{2.059341in}}%
\pgfpathlineto{\pgfqpoint{4.588573in}{2.057194in}}%
\pgfpathlineto{\pgfqpoint{4.574539in}{2.055146in}}%
\pgfpathlineto{\pgfqpoint{4.560516in}{2.053198in}}%
\pgfpathlineto{\pgfqpoint{4.546503in}{2.051351in}}%
\pgfpathlineto{\pgfqpoint{4.554350in}{2.062809in}}%
\pgfpathlineto{\pgfqpoint{4.562193in}{2.074257in}}%
\pgfpathlineto{\pgfqpoint{4.570030in}{2.085692in}}%
\pgfpathlineto{\pgfqpoint{4.577862in}{2.097113in}}%
\pgfpathclose%
\pgfusepath{fill}%
\end{pgfscope}%
\begin{pgfscope}%
\pgfpathrectangle{\pgfqpoint{1.150000in}{0.150000in}}{\pgfqpoint{5.700000in}{5.700000in}}%
\pgfusepath{clip}%
\pgfsetbuttcap%
\pgfsetroundjoin%
\definecolor{currentfill}{rgb}{0.281446,0.084320,0.407414}%
\pgfsetfillcolor{currentfill}%
\pgfsetfillopacity{0.700000}%
\pgfsetlinewidth{0.000000pt}%
\definecolor{currentstroke}{rgb}{0.000000,0.000000,0.000000}%
\pgfsetstrokecolor{currentstroke}%
\pgfsetdash{}{0pt}%
\pgfpathmoveto{\pgfqpoint{3.285914in}{1.972228in}}%
\pgfpathlineto{\pgfqpoint{3.299643in}{1.963255in}}%
\pgfpathlineto{\pgfqpoint{3.313373in}{1.954405in}}%
\pgfpathlineto{\pgfqpoint{3.327105in}{1.945676in}}%
\pgfpathlineto{\pgfqpoint{3.340840in}{1.937067in}}%
\pgfpathlineto{\pgfqpoint{3.332487in}{1.934799in}}%
\pgfpathlineto{\pgfqpoint{3.324123in}{1.932734in}}%
\pgfpathlineto{\pgfqpoint{3.315747in}{1.930879in}}%
\pgfpathlineto{\pgfqpoint{3.307361in}{1.929236in}}%
\pgfpathlineto{\pgfqpoint{3.293597in}{1.938297in}}%
\pgfpathlineto{\pgfqpoint{3.279836in}{1.947479in}}%
\pgfpathlineto{\pgfqpoint{3.266076in}{1.956782in}}%
\pgfpathlineto{\pgfqpoint{3.252318in}{1.966208in}}%
\pgfpathlineto{\pgfqpoint{3.260735in}{1.967390in}}%
\pgfpathlineto{\pgfqpoint{3.269140in}{1.968791in}}%
\pgfpathlineto{\pgfqpoint{3.277533in}{1.970405in}}%
\pgfpathlineto{\pgfqpoint{3.285914in}{1.972228in}}%
\pgfpathclose%
\pgfusepath{fill}%
\end{pgfscope}%
\begin{pgfscope}%
\pgfpathrectangle{\pgfqpoint{1.150000in}{0.150000in}}{\pgfqpoint{5.700000in}{5.700000in}}%
\pgfusepath{clip}%
\pgfsetbuttcap%
\pgfsetroundjoin%
\definecolor{currentfill}{rgb}{0.277134,0.185228,0.489898}%
\pgfsetfillcolor{currentfill}%
\pgfsetfillopacity{0.700000}%
\pgfsetlinewidth{0.000000pt}%
\definecolor{currentstroke}{rgb}{0.000000,0.000000,0.000000}%
\pgfsetstrokecolor{currentstroke}%
\pgfsetdash{}{0pt}%
\pgfpathmoveto{\pgfqpoint{2.977309in}{2.181462in}}%
\pgfpathlineto{\pgfqpoint{2.991060in}{2.169441in}}%
\pgfpathlineto{\pgfqpoint{3.004810in}{2.157558in}}%
\pgfpathlineto{\pgfqpoint{3.018559in}{2.145812in}}%
\pgfpathlineto{\pgfqpoint{3.032308in}{2.134201in}}%
\pgfpathlineto{\pgfqpoint{3.023745in}{2.135116in}}%
\pgfpathlineto{\pgfqpoint{3.015168in}{2.136278in}}%
\pgfpathlineto{\pgfqpoint{3.006575in}{2.137694in}}%
\pgfpathlineto{\pgfqpoint{2.997968in}{2.139368in}}%
\pgfpathlineto{\pgfqpoint{2.984182in}{2.151459in}}%
\pgfpathlineto{\pgfqpoint{2.970396in}{2.163685in}}%
\pgfpathlineto{\pgfqpoint{2.956608in}{2.176049in}}%
\pgfpathlineto{\pgfqpoint{2.942819in}{2.188550in}}%
\pgfpathlineto{\pgfqpoint{2.951464in}{2.186388in}}%
\pgfpathlineto{\pgfqpoint{2.960094in}{2.184489in}}%
\pgfpathlineto{\pgfqpoint{2.968709in}{2.182849in}}%
\pgfpathlineto{\pgfqpoint{2.977309in}{2.181462in}}%
\pgfpathclose%
\pgfusepath{fill}%
\end{pgfscope}%
\begin{pgfscope}%
\pgfpathrectangle{\pgfqpoint{1.150000in}{0.150000in}}{\pgfqpoint{5.700000in}{5.700000in}}%
\pgfusepath{clip}%
\pgfsetbuttcap%
\pgfsetroundjoin%
\definecolor{currentfill}{rgb}{0.190631,0.407061,0.556089}%
\pgfsetfillcolor{currentfill}%
\pgfsetfillopacity{0.700000}%
\pgfsetlinewidth{0.000000pt}%
\definecolor{currentstroke}{rgb}{0.000000,0.000000,0.000000}%
\pgfsetstrokecolor{currentstroke}%
\pgfsetdash{}{0pt}%
\pgfpathmoveto{\pgfqpoint{5.339710in}{2.686138in}}%
\pgfpathlineto{\pgfqpoint{5.354088in}{2.692922in}}%
\pgfpathlineto{\pgfqpoint{5.368480in}{2.699805in}}%
\pgfpathlineto{\pgfqpoint{5.382887in}{2.706789in}}%
\pgfpathlineto{\pgfqpoint{5.397309in}{2.713874in}}%
\pgfpathlineto{\pgfqpoint{5.389748in}{2.703818in}}%
\pgfpathlineto{\pgfqpoint{5.382180in}{2.693655in}}%
\pgfpathlineto{\pgfqpoint{5.374604in}{2.683385in}}%
\pgfpathlineto{\pgfqpoint{5.367020in}{2.673008in}}%
\pgfpathlineto{\pgfqpoint{5.352594in}{2.665988in}}%
\pgfpathlineto{\pgfqpoint{5.338182in}{2.659068in}}%
\pgfpathlineto{\pgfqpoint{5.323785in}{2.652249in}}%
\pgfpathlineto{\pgfqpoint{5.309403in}{2.645530in}}%
\pgfpathlineto{\pgfqpoint{5.316991in}{2.655836in}}%
\pgfpathlineto{\pgfqpoint{5.324571in}{2.666039in}}%
\pgfpathlineto{\pgfqpoint{5.332144in}{2.676140in}}%
\pgfpathlineto{\pgfqpoint{5.339710in}{2.686138in}}%
\pgfpathclose%
\pgfusepath{fill}%
\end{pgfscope}%
\begin{pgfscope}%
\pgfpathrectangle{\pgfqpoint{1.150000in}{0.150000in}}{\pgfqpoint{5.700000in}{5.700000in}}%
\pgfusepath{clip}%
\pgfsetbuttcap%
\pgfsetroundjoin%
\definecolor{currentfill}{rgb}{0.278826,0.175490,0.483397}%
\pgfsetfillcolor{currentfill}%
\pgfsetfillopacity{0.700000}%
\pgfsetlinewidth{0.000000pt}%
\definecolor{currentstroke}{rgb}{0.000000,0.000000,0.000000}%
\pgfsetstrokecolor{currentstroke}%
\pgfsetdash{}{0pt}%
\pgfpathmoveto{\pgfqpoint{4.665224in}{2.152515in}}%
\pgfpathlineto{\pgfqpoint{4.679272in}{2.155244in}}%
\pgfpathlineto{\pgfqpoint{4.693331in}{2.158073in}}%
\pgfpathlineto{\pgfqpoint{4.707401in}{2.161002in}}%
\pgfpathlineto{\pgfqpoint{4.721483in}{2.164030in}}%
\pgfpathlineto{\pgfqpoint{4.713678in}{2.152242in}}%
\pgfpathlineto{\pgfqpoint{4.705868in}{2.140421in}}%
\pgfpathlineto{\pgfqpoint{4.698054in}{2.128570in}}%
\pgfpathlineto{\pgfqpoint{4.690234in}{2.116690in}}%
\pgfpathlineto{\pgfqpoint{4.676149in}{2.113893in}}%
\pgfpathlineto{\pgfqpoint{4.662075in}{2.111196in}}%
\pgfpathlineto{\pgfqpoint{4.648012in}{2.108599in}}%
\pgfpathlineto{\pgfqpoint{4.633961in}{2.106101in}}%
\pgfpathlineto{\pgfqpoint{4.641784in}{2.117743in}}%
\pgfpathlineto{\pgfqpoint{4.649602in}{2.129360in}}%
\pgfpathlineto{\pgfqpoint{4.657416in}{2.140952in}}%
\pgfpathlineto{\pgfqpoint{4.665224in}{2.152515in}}%
\pgfpathclose%
\pgfusepath{fill}%
\end{pgfscope}%
\begin{pgfscope}%
\pgfpathrectangle{\pgfqpoint{1.150000in}{0.150000in}}{\pgfqpoint{5.700000in}{5.700000in}}%
\pgfusepath{clip}%
\pgfsetbuttcap%
\pgfsetroundjoin%
\definecolor{currentfill}{rgb}{0.235526,0.309527,0.542944}%
\pgfsetfillcolor{currentfill}%
\pgfsetfillopacity{0.700000}%
\pgfsetlinewidth{0.000000pt}%
\definecolor{currentstroke}{rgb}{0.000000,0.000000,0.000000}%
\pgfsetstrokecolor{currentstroke}%
\pgfsetdash{}{0pt}%
\pgfpathmoveto{\pgfqpoint{5.046189in}{2.444959in}}%
\pgfpathlineto{\pgfqpoint{5.060416in}{2.450171in}}%
\pgfpathlineto{\pgfqpoint{5.074656in}{2.455482in}}%
\pgfpathlineto{\pgfqpoint{5.088909in}{2.460894in}}%
\pgfpathlineto{\pgfqpoint{5.103176in}{2.466405in}}%
\pgfpathlineto{\pgfqpoint{5.095494in}{2.455094in}}%
\pgfpathlineto{\pgfqpoint{5.087806in}{2.443701in}}%
\pgfpathlineto{\pgfqpoint{5.080111in}{2.432226in}}%
\pgfpathlineto{\pgfqpoint{5.072410in}{2.420671in}}%
\pgfpathlineto{\pgfqpoint{5.058141in}{2.415301in}}%
\pgfpathlineto{\pgfqpoint{5.043885in}{2.410030in}}%
\pgfpathlineto{\pgfqpoint{5.029642in}{2.404858in}}%
\pgfpathlineto{\pgfqpoint{5.015412in}{2.399787in}}%
\pgfpathlineto{\pgfqpoint{5.023115in}{2.411194in}}%
\pgfpathlineto{\pgfqpoint{5.030813in}{2.422526in}}%
\pgfpathlineto{\pgfqpoint{5.038504in}{2.433782in}}%
\pgfpathlineto{\pgfqpoint{5.046189in}{2.444959in}}%
\pgfpathclose%
\pgfusepath{fill}%
\end{pgfscope}%
\begin{pgfscope}%
\pgfpathrectangle{\pgfqpoint{1.150000in}{0.150000in}}{\pgfqpoint{5.700000in}{5.700000in}}%
\pgfusepath{clip}%
\pgfsetbuttcap%
\pgfsetroundjoin%
\definecolor{currentfill}{rgb}{0.273809,0.031497,0.358853}%
\pgfsetfillcolor{currentfill}%
\pgfsetfillopacity{0.700000}%
\pgfsetlinewidth{0.000000pt}%
\definecolor{currentstroke}{rgb}{0.000000,0.000000,0.000000}%
\pgfsetstrokecolor{currentstroke}%
\pgfsetdash{}{0pt}%
\pgfpathmoveto{\pgfqpoint{4.141555in}{1.876384in}}%
\pgfpathlineto{\pgfqpoint{4.155411in}{1.875031in}}%
\pgfpathlineto{\pgfqpoint{4.169276in}{1.873781in}}%
\pgfpathlineto{\pgfqpoint{4.183148in}{1.872634in}}%
\pgfpathlineto{\pgfqpoint{4.197028in}{1.871589in}}%
\pgfpathlineto{\pgfqpoint{4.189070in}{1.861677in}}%
\pgfpathlineto{\pgfqpoint{4.181107in}{1.851822in}}%
\pgfpathlineto{\pgfqpoint{4.173138in}{1.842026in}}%
\pgfpathlineto{\pgfqpoint{4.165164in}{1.832294in}}%
\pgfpathlineto{\pgfqpoint{4.151274in}{1.833677in}}%
\pgfpathlineto{\pgfqpoint{4.137392in}{1.835162in}}%
\pgfpathlineto{\pgfqpoint{4.123518in}{1.836750in}}%
\pgfpathlineto{\pgfqpoint{4.109651in}{1.838440in}}%
\pgfpathlineto{\pgfqpoint{4.117636in}{1.847827in}}%
\pgfpathlineto{\pgfqpoint{4.125614in}{1.857283in}}%
\pgfpathlineto{\pgfqpoint{4.133587in}{1.866802in}}%
\pgfpathlineto{\pgfqpoint{4.141555in}{1.876384in}}%
\pgfpathclose%
\pgfusepath{fill}%
\end{pgfscope}%
\begin{pgfscope}%
\pgfpathrectangle{\pgfqpoint{1.150000in}{0.150000in}}{\pgfqpoint{5.700000in}{5.700000in}}%
\pgfusepath{clip}%
\pgfsetbuttcap%
\pgfsetroundjoin%
\definecolor{currentfill}{rgb}{0.144759,0.519093,0.556572}%
\pgfsetfillcolor{currentfill}%
\pgfsetfillopacity{0.700000}%
\pgfsetlinewidth{0.000000pt}%
\definecolor{currentstroke}{rgb}{0.000000,0.000000,0.000000}%
\pgfsetstrokecolor{currentstroke}%
\pgfsetdash{}{0pt}%
\pgfpathmoveto{\pgfqpoint{5.720716in}{2.985504in}}%
\pgfpathlineto{\pgfqpoint{5.735305in}{2.993924in}}%
\pgfpathlineto{\pgfqpoint{5.749910in}{3.002444in}}%
\pgfpathlineto{\pgfqpoint{5.764531in}{3.011066in}}%
\pgfpathlineto{\pgfqpoint{5.779169in}{3.019788in}}%
\pgfpathlineto{\pgfqpoint{5.771805in}{3.012040in}}%
\pgfpathlineto{\pgfqpoint{5.764431in}{3.004169in}}%
\pgfpathlineto{\pgfqpoint{5.757048in}{2.996175in}}%
\pgfpathlineto{\pgfqpoint{5.749656in}{2.988056in}}%
\pgfpathlineto{\pgfqpoint{5.735009in}{2.979300in}}%
\pgfpathlineto{\pgfqpoint{5.720378in}{2.970644in}}%
\pgfpathlineto{\pgfqpoint{5.705764in}{2.962090in}}%
\pgfpathlineto{\pgfqpoint{5.691166in}{2.953637in}}%
\pgfpathlineto{\pgfqpoint{5.698567in}{2.961782in}}%
\pgfpathlineto{\pgfqpoint{5.705959in}{2.969808in}}%
\pgfpathlineto{\pgfqpoint{5.713342in}{2.977715in}}%
\pgfpathlineto{\pgfqpoint{5.720716in}{2.985504in}}%
\pgfpathclose%
\pgfusepath{fill}%
\end{pgfscope}%
\begin{pgfscope}%
\pgfpathrectangle{\pgfqpoint{1.150000in}{0.150000in}}{\pgfqpoint{5.700000in}{5.700000in}}%
\pgfusepath{clip}%
\pgfsetbuttcap%
\pgfsetroundjoin%
\definecolor{currentfill}{rgb}{0.280255,0.165693,0.476498}%
\pgfsetfillcolor{currentfill}%
\pgfsetfillopacity{0.700000}%
\pgfsetlinewidth{0.000000pt}%
\definecolor{currentstroke}{rgb}{0.000000,0.000000,0.000000}%
\pgfsetstrokecolor{currentstroke}%
\pgfsetdash{}{0pt}%
\pgfpathmoveto{\pgfqpoint{3.032308in}{2.134201in}}%
\pgfpathlineto{\pgfqpoint{3.046057in}{2.122725in}}%
\pgfpathlineto{\pgfqpoint{3.059805in}{2.111382in}}%
\pgfpathlineto{\pgfqpoint{3.073553in}{2.100173in}}%
\pgfpathlineto{\pgfqpoint{3.087301in}{2.089096in}}%
\pgfpathlineto{\pgfqpoint{3.078773in}{2.089540in}}%
\pgfpathlineto{\pgfqpoint{3.070232in}{2.090227in}}%
\pgfpathlineto{\pgfqpoint{3.061676in}{2.091162in}}%
\pgfpathlineto{\pgfqpoint{3.053106in}{2.092349in}}%
\pgfpathlineto{\pgfqpoint{3.039322in}{2.103904in}}%
\pgfpathlineto{\pgfqpoint{3.025538in}{2.115592in}}%
\pgfpathlineto{\pgfqpoint{3.011754in}{2.127413in}}%
\pgfpathlineto{\pgfqpoint{2.997968in}{2.139368in}}%
\pgfpathlineto{\pgfqpoint{3.006575in}{2.137694in}}%
\pgfpathlineto{\pgfqpoint{3.015168in}{2.136278in}}%
\pgfpathlineto{\pgfqpoint{3.023745in}{2.135116in}}%
\pgfpathlineto{\pgfqpoint{3.032308in}{2.134201in}}%
\pgfpathclose%
\pgfusepath{fill}%
\end{pgfscope}%
\begin{pgfscope}%
\pgfpathrectangle{\pgfqpoint{1.150000in}{0.150000in}}{\pgfqpoint{5.700000in}{5.700000in}}%
\pgfusepath{clip}%
\pgfsetbuttcap%
\pgfsetroundjoin%
\definecolor{currentfill}{rgb}{0.273006,0.204520,0.501721}%
\pgfsetfillcolor{currentfill}%
\pgfsetfillopacity{0.700000}%
\pgfsetlinewidth{0.000000pt}%
\definecolor{currentstroke}{rgb}{0.000000,0.000000,0.000000}%
\pgfsetstrokecolor{currentstroke}%
\pgfsetdash{}{0pt}%
\pgfpathmoveto{\pgfqpoint{4.752651in}{2.210818in}}%
\pgfpathlineto{\pgfqpoint{4.766741in}{2.214160in}}%
\pgfpathlineto{\pgfqpoint{4.780843in}{2.217602in}}%
\pgfpathlineto{\pgfqpoint{4.794957in}{2.221144in}}%
\pgfpathlineto{\pgfqpoint{4.809082in}{2.224786in}}%
\pgfpathlineto{\pgfqpoint{4.801301in}{2.212941in}}%
\pgfpathlineto{\pgfqpoint{4.793514in}{2.201051in}}%
\pgfpathlineto{\pgfqpoint{4.785722in}{2.189117in}}%
\pgfpathlineto{\pgfqpoint{4.777925in}{2.177143in}}%
\pgfpathlineto{\pgfqpoint{4.763797in}{2.173715in}}%
\pgfpathlineto{\pgfqpoint{4.749681in}{2.170387in}}%
\pgfpathlineto{\pgfqpoint{4.735576in}{2.167159in}}%
\pgfpathlineto{\pgfqpoint{4.721483in}{2.164030in}}%
\pgfpathlineto{\pgfqpoint{4.729283in}{2.175784in}}%
\pgfpathlineto{\pgfqpoint{4.737078in}{2.187501in}}%
\pgfpathlineto{\pgfqpoint{4.744867in}{2.199180in}}%
\pgfpathlineto{\pgfqpoint{4.752651in}{2.210818in}}%
\pgfpathclose%
\pgfusepath{fill}%
\end{pgfscope}%
\begin{pgfscope}%
\pgfpathrectangle{\pgfqpoint{1.150000in}{0.150000in}}{\pgfqpoint{5.700000in}{5.700000in}}%
\pgfusepath{clip}%
\pgfsetbuttcap%
\pgfsetroundjoin%
\definecolor{currentfill}{rgb}{0.268510,0.009605,0.335427}%
\pgfsetfillcolor{currentfill}%
\pgfsetfillopacity{0.700000}%
\pgfsetlinewidth{0.000000pt}%
\definecolor{currentstroke}{rgb}{0.000000,0.000000,0.000000}%
\pgfsetstrokecolor{currentstroke}%
\pgfsetdash{}{0pt}%
\pgfpathmoveto{\pgfqpoint{3.681626in}{1.831241in}}%
\pgfpathlineto{\pgfqpoint{3.695386in}{1.825887in}}%
\pgfpathlineto{\pgfqpoint{3.709151in}{1.820642in}}%
\pgfpathlineto{\pgfqpoint{3.722922in}{1.815507in}}%
\pgfpathlineto{\pgfqpoint{3.736697in}{1.810481in}}%
\pgfpathlineto{\pgfqpoint{3.728557in}{1.804309in}}%
\pgfpathlineto{\pgfqpoint{3.720409in}{1.798279in}}%
\pgfpathlineto{\pgfqpoint{3.712253in}{1.792393in}}%
\pgfpathlineto{\pgfqpoint{3.704089in}{1.786658in}}%
\pgfpathlineto{\pgfqpoint{3.690295in}{1.792095in}}%
\pgfpathlineto{\pgfqpoint{3.676506in}{1.797641in}}%
\pgfpathlineto{\pgfqpoint{3.662721in}{1.803296in}}%
\pgfpathlineto{\pgfqpoint{3.648941in}{1.809061in}}%
\pgfpathlineto{\pgfqpoint{3.657124in}{1.814379in}}%
\pgfpathlineto{\pgfqpoint{3.665300in}{1.819851in}}%
\pgfpathlineto{\pgfqpoint{3.673467in}{1.825473in}}%
\pgfpathlineto{\pgfqpoint{3.681626in}{1.831241in}}%
\pgfpathclose%
\pgfusepath{fill}%
\end{pgfscope}%
\begin{pgfscope}%
\pgfpathrectangle{\pgfqpoint{1.150000in}{0.150000in}}{\pgfqpoint{5.700000in}{5.700000in}}%
\pgfusepath{clip}%
\pgfsetbuttcap%
\pgfsetroundjoin%
\definecolor{currentfill}{rgb}{0.177423,0.437527,0.557565}%
\pgfsetfillcolor{currentfill}%
\pgfsetfillopacity{0.700000}%
\pgfsetlinewidth{0.000000pt}%
\definecolor{currentstroke}{rgb}{0.000000,0.000000,0.000000}%
\pgfsetstrokecolor{currentstroke}%
\pgfsetdash{}{0pt}%
\pgfpathmoveto{\pgfqpoint{5.427476in}{2.753012in}}%
\pgfpathlineto{\pgfqpoint{5.441908in}{2.760242in}}%
\pgfpathlineto{\pgfqpoint{5.456354in}{2.767572in}}%
\pgfpathlineto{\pgfqpoint{5.470816in}{2.775002in}}%
\pgfpathlineto{\pgfqpoint{5.485293in}{2.782533in}}%
\pgfpathlineto{\pgfqpoint{5.477768in}{2.772873in}}%
\pgfpathlineto{\pgfqpoint{5.470236in}{2.763099in}}%
\pgfpathlineto{\pgfqpoint{5.462695in}{2.753213in}}%
\pgfpathlineto{\pgfqpoint{5.455147in}{2.743213in}}%
\pgfpathlineto{\pgfqpoint{5.440665in}{2.735728in}}%
\pgfpathlineto{\pgfqpoint{5.426198in}{2.728343in}}%
\pgfpathlineto{\pgfqpoint{5.411746in}{2.721058in}}%
\pgfpathlineto{\pgfqpoint{5.397309in}{2.713874in}}%
\pgfpathlineto{\pgfqpoint{5.404862in}{2.723821in}}%
\pgfpathlineto{\pgfqpoint{5.412408in}{2.733660in}}%
\pgfpathlineto{\pgfqpoint{5.419946in}{2.743390in}}%
\pgfpathlineto{\pgfqpoint{5.427476in}{2.753012in}}%
\pgfpathclose%
\pgfusepath{fill}%
\end{pgfscope}%
\begin{pgfscope}%
\pgfpathrectangle{\pgfqpoint{1.150000in}{0.150000in}}{\pgfqpoint{5.700000in}{5.700000in}}%
\pgfusepath{clip}%
\pgfsetbuttcap%
\pgfsetroundjoin%
\definecolor{currentfill}{rgb}{0.269944,0.014625,0.341379}%
\pgfsetfillcolor{currentfill}%
\pgfsetfillopacity{0.700000}%
\pgfsetlinewidth{0.000000pt}%
\definecolor{currentstroke}{rgb}{0.000000,0.000000,0.000000}%
\pgfsetstrokecolor{currentstroke}%
\pgfsetdash{}{0pt}%
\pgfpathmoveto{\pgfqpoint{4.054261in}{1.846232in}}%
\pgfpathlineto{\pgfqpoint{4.068098in}{1.844129in}}%
\pgfpathlineto{\pgfqpoint{4.081941in}{1.842129in}}%
\pgfpathlineto{\pgfqpoint{4.095793in}{1.840233in}}%
\pgfpathlineto{\pgfqpoint{4.109651in}{1.838440in}}%
\pgfpathlineto{\pgfqpoint{4.101662in}{1.829123in}}%
\pgfpathlineto{\pgfqpoint{4.093666in}{1.819881in}}%
\pgfpathlineto{\pgfqpoint{4.085665in}{1.810717in}}%
\pgfpathlineto{\pgfqpoint{4.077659in}{1.801634in}}%
\pgfpathlineto{\pgfqpoint{4.063789in}{1.803783in}}%
\pgfpathlineto{\pgfqpoint{4.049926in}{1.806035in}}%
\pgfpathlineto{\pgfqpoint{4.036071in}{1.808390in}}%
\pgfpathlineto{\pgfqpoint{4.022223in}{1.810849in}}%
\pgfpathlineto{\pgfqpoint{4.030241in}{1.819569in}}%
\pgfpathlineto{\pgfqpoint{4.038254in}{1.828375in}}%
\pgfpathlineto{\pgfqpoint{4.046260in}{1.837264in}}%
\pgfpathlineto{\pgfqpoint{4.054261in}{1.846232in}}%
\pgfpathclose%
\pgfusepath{fill}%
\end{pgfscope}%
\begin{pgfscope}%
\pgfpathrectangle{\pgfqpoint{1.150000in}{0.150000in}}{\pgfqpoint{5.700000in}{5.700000in}}%
\pgfusepath{clip}%
\pgfsetbuttcap%
\pgfsetroundjoin%
\definecolor{currentfill}{rgb}{0.267004,0.004874,0.329415}%
\pgfsetfillcolor{currentfill}%
\pgfsetfillopacity{0.700000}%
\pgfsetlinewidth{0.000000pt}%
\definecolor{currentstroke}{rgb}{0.000000,0.000000,0.000000}%
\pgfsetstrokecolor{currentstroke}%
\pgfsetdash{}{0pt}%
\pgfpathmoveto{\pgfqpoint{3.824265in}{1.819049in}}%
\pgfpathlineto{\pgfqpoint{3.838050in}{1.814954in}}%
\pgfpathlineto{\pgfqpoint{3.851841in}{1.810966in}}%
\pgfpathlineto{\pgfqpoint{3.865637in}{1.807084in}}%
\pgfpathlineto{\pgfqpoint{3.879440in}{1.803309in}}%
\pgfpathlineto{\pgfqpoint{3.871362in}{1.795841in}}%
\pgfpathlineto{\pgfqpoint{3.863278in}{1.788489in}}%
\pgfpathlineto{\pgfqpoint{3.855187in}{1.781258in}}%
\pgfpathlineto{\pgfqpoint{3.847089in}{1.774151in}}%
\pgfpathlineto{\pgfqpoint{3.833270in}{1.778318in}}%
\pgfpathlineto{\pgfqpoint{3.819458in}{1.782591in}}%
\pgfpathlineto{\pgfqpoint{3.805650in}{1.786971in}}%
\pgfpathlineto{\pgfqpoint{3.791849in}{1.791458in}}%
\pgfpathlineto{\pgfqpoint{3.799964in}{1.798167in}}%
\pgfpathlineto{\pgfqpoint{3.808071in}{1.805004in}}%
\pgfpathlineto{\pgfqpoint{3.816172in}{1.811966in}}%
\pgfpathlineto{\pgfqpoint{3.824265in}{1.819049in}}%
\pgfpathclose%
\pgfusepath{fill}%
\end{pgfscope}%
\begin{pgfscope}%
\pgfpathrectangle{\pgfqpoint{1.150000in}{0.150000in}}{\pgfqpoint{5.700000in}{5.700000in}}%
\pgfusepath{clip}%
\pgfsetbuttcap%
\pgfsetroundjoin%
\definecolor{currentfill}{rgb}{0.221989,0.339161,0.548752}%
\pgfsetfillcolor{currentfill}%
\pgfsetfillopacity{0.700000}%
\pgfsetlinewidth{0.000000pt}%
\definecolor{currentstroke}{rgb}{0.000000,0.000000,0.000000}%
\pgfsetstrokecolor{currentstroke}%
\pgfsetdash{}{0pt}%
\pgfpathmoveto{\pgfqpoint{5.133841in}{2.510805in}}%
\pgfpathlineto{\pgfqpoint{5.148118in}{2.516538in}}%
\pgfpathlineto{\pgfqpoint{5.162409in}{2.522370in}}%
\pgfpathlineto{\pgfqpoint{5.176714in}{2.528303in}}%
\pgfpathlineto{\pgfqpoint{5.191032in}{2.534335in}}%
\pgfpathlineto{\pgfqpoint{5.183379in}{2.523249in}}%
\pgfpathlineto{\pgfqpoint{5.175719in}{2.512072in}}%
\pgfpathlineto{\pgfqpoint{5.168052in}{2.500805in}}%
\pgfpathlineto{\pgfqpoint{5.160379in}{2.489448in}}%
\pgfpathlineto{\pgfqpoint{5.146058in}{2.483538in}}%
\pgfpathlineto{\pgfqpoint{5.131750in}{2.477727in}}%
\pgfpathlineto{\pgfqpoint{5.117456in}{2.472016in}}%
\pgfpathlineto{\pgfqpoint{5.103176in}{2.466405in}}%
\pgfpathlineto{\pgfqpoint{5.110852in}{2.477632in}}%
\pgfpathlineto{\pgfqpoint{5.118521in}{2.488776in}}%
\pgfpathlineto{\pgfqpoint{5.126184in}{2.499833in}}%
\pgfpathlineto{\pgfqpoint{5.133841in}{2.510805in}}%
\pgfpathclose%
\pgfusepath{fill}%
\end{pgfscope}%
\begin{pgfscope}%
\pgfpathrectangle{\pgfqpoint{1.150000in}{0.150000in}}{\pgfqpoint{5.700000in}{5.700000in}}%
\pgfusepath{clip}%
\pgfsetbuttcap%
\pgfsetroundjoin%
\definecolor{currentfill}{rgb}{0.135066,0.544853,0.554029}%
\pgfsetfillcolor{currentfill}%
\pgfsetfillopacity{0.700000}%
\pgfsetlinewidth{0.000000pt}%
\definecolor{currentstroke}{rgb}{0.000000,0.000000,0.000000}%
\pgfsetstrokecolor{currentstroke}%
\pgfsetdash{}{0pt}%
\pgfpathmoveto{\pgfqpoint{5.808535in}{3.049561in}}%
\pgfpathlineto{\pgfqpoint{5.823179in}{3.058330in}}%
\pgfpathlineto{\pgfqpoint{5.837839in}{3.067200in}}%
\pgfpathlineto{\pgfqpoint{5.852517in}{3.076171in}}%
\pgfpathlineto{\pgfqpoint{5.867211in}{3.085243in}}%
\pgfpathlineto{\pgfqpoint{5.859895in}{3.078042in}}%
\pgfpathlineto{\pgfqpoint{5.852568in}{3.070716in}}%
\pgfpathlineto{\pgfqpoint{5.845233in}{3.063264in}}%
\pgfpathlineto{\pgfqpoint{5.837887in}{3.055686in}}%
\pgfpathlineto{\pgfqpoint{5.823183in}{3.046560in}}%
\pgfpathlineto{\pgfqpoint{5.808495in}{3.037535in}}%
\pgfpathlineto{\pgfqpoint{5.793824in}{3.028611in}}%
\pgfpathlineto{\pgfqpoint{5.779169in}{3.019788in}}%
\pgfpathlineto{\pgfqpoint{5.786525in}{3.027413in}}%
\pgfpathlineto{\pgfqpoint{5.793871in}{3.034916in}}%
\pgfpathlineto{\pgfqpoint{5.801207in}{3.042298in}}%
\pgfpathlineto{\pgfqpoint{5.808535in}{3.049561in}}%
\pgfpathclose%
\pgfusepath{fill}%
\end{pgfscope}%
\begin{pgfscope}%
\pgfpathrectangle{\pgfqpoint{1.150000in}{0.150000in}}{\pgfqpoint{5.700000in}{5.700000in}}%
\pgfusepath{clip}%
\pgfsetbuttcap%
\pgfsetroundjoin%
\definecolor{currentfill}{rgb}{0.280267,0.073417,0.397163}%
\pgfsetfillcolor{currentfill}%
\pgfsetfillopacity{0.700000}%
\pgfsetlinewidth{0.000000pt}%
\definecolor{currentstroke}{rgb}{0.000000,0.000000,0.000000}%
\pgfsetstrokecolor{currentstroke}%
\pgfsetdash{}{0pt}%
\pgfpathmoveto{\pgfqpoint{3.340840in}{1.937067in}}%
\pgfpathlineto{\pgfqpoint{3.354576in}{1.928579in}}%
\pgfpathlineto{\pgfqpoint{3.368315in}{1.920210in}}%
\pgfpathlineto{\pgfqpoint{3.382056in}{1.911961in}}%
\pgfpathlineto{\pgfqpoint{3.395800in}{1.903829in}}%
\pgfpathlineto{\pgfqpoint{3.387474in}{1.901116in}}%
\pgfpathlineto{\pgfqpoint{3.379138in}{1.898603in}}%
\pgfpathlineto{\pgfqpoint{3.370791in}{1.896293in}}%
\pgfpathlineto{\pgfqpoint{3.362433in}{1.894191in}}%
\pgfpathlineto{\pgfqpoint{3.348662in}{1.902774in}}%
\pgfpathlineto{\pgfqpoint{3.334893in}{1.911475in}}%
\pgfpathlineto{\pgfqpoint{3.321126in}{1.920296in}}%
\pgfpathlineto{\pgfqpoint{3.307361in}{1.929236in}}%
\pgfpathlineto{\pgfqpoint{3.315747in}{1.930879in}}%
\pgfpathlineto{\pgfqpoint{3.324123in}{1.932734in}}%
\pgfpathlineto{\pgfqpoint{3.332487in}{1.934799in}}%
\pgfpathlineto{\pgfqpoint{3.340840in}{1.937067in}}%
\pgfpathclose%
\pgfusepath{fill}%
\end{pgfscope}%
\begin{pgfscope}%
\pgfpathrectangle{\pgfqpoint{1.150000in}{0.150000in}}{\pgfqpoint{5.700000in}{5.700000in}}%
\pgfusepath{clip}%
\pgfsetbuttcap%
\pgfsetroundjoin%
\definecolor{currentfill}{rgb}{0.272594,0.025563,0.353093}%
\pgfsetfillcolor{currentfill}%
\pgfsetfillopacity{0.700000}%
\pgfsetlinewidth{0.000000pt}%
\definecolor{currentstroke}{rgb}{0.000000,0.000000,0.000000}%
\pgfsetstrokecolor{currentstroke}%
\pgfsetdash{}{0pt}%
\pgfpathmoveto{\pgfqpoint{3.538851in}{1.859184in}}%
\pgfpathlineto{\pgfqpoint{3.552599in}{1.852525in}}%
\pgfpathlineto{\pgfqpoint{3.566350in}{1.845980in}}%
\pgfpathlineto{\pgfqpoint{3.580105in}{1.839548in}}%
\pgfpathlineto{\pgfqpoint{3.593863in}{1.833227in}}%
\pgfpathlineto{\pgfqpoint{3.585650in}{1.828491in}}%
\pgfpathlineto{\pgfqpoint{3.577428in}{1.823921in}}%
\pgfpathlineto{\pgfqpoint{3.569196in}{1.819523in}}%
\pgfpathlineto{\pgfqpoint{3.560956in}{1.815301in}}%
\pgfpathlineto{\pgfqpoint{3.547174in}{1.822052in}}%
\pgfpathlineto{\pgfqpoint{3.533396in}{1.828915in}}%
\pgfpathlineto{\pgfqpoint{3.519622in}{1.835890in}}%
\pgfpathlineto{\pgfqpoint{3.505851in}{1.842979in}}%
\pgfpathlineto{\pgfqpoint{3.514116in}{1.846764in}}%
\pgfpathlineto{\pgfqpoint{3.522370in}{1.850729in}}%
\pgfpathlineto{\pgfqpoint{3.530615in}{1.854870in}}%
\pgfpathlineto{\pgfqpoint{3.538851in}{1.859184in}}%
\pgfpathclose%
\pgfusepath{fill}%
\end{pgfscope}%
\begin{pgfscope}%
\pgfpathrectangle{\pgfqpoint{1.150000in}{0.150000in}}{\pgfqpoint{5.700000in}{5.700000in}}%
\pgfusepath{clip}%
\pgfsetbuttcap%
\pgfsetroundjoin%
\definecolor{currentfill}{rgb}{0.263663,0.237631,0.518762}%
\pgfsetfillcolor{currentfill}%
\pgfsetfillopacity{0.700000}%
\pgfsetlinewidth{0.000000pt}%
\definecolor{currentstroke}{rgb}{0.000000,0.000000,0.000000}%
\pgfsetstrokecolor{currentstroke}%
\pgfsetdash{}{0pt}%
\pgfpathmoveto{\pgfqpoint{4.840154in}{2.271682in}}%
\pgfpathlineto{\pgfqpoint{4.854289in}{2.275619in}}%
\pgfpathlineto{\pgfqpoint{4.868435in}{2.279656in}}%
\pgfpathlineto{\pgfqpoint{4.882594in}{2.283792in}}%
\pgfpathlineto{\pgfqpoint{4.896765in}{2.288028in}}%
\pgfpathlineto{\pgfqpoint{4.889008in}{2.276192in}}%
\pgfpathlineto{\pgfqpoint{4.881245in}{2.264299in}}%
\pgfpathlineto{\pgfqpoint{4.873477in}{2.252351in}}%
\pgfpathlineto{\pgfqpoint{4.865704in}{2.240349in}}%
\pgfpathlineto{\pgfqpoint{4.851530in}{2.236309in}}%
\pgfpathlineto{\pgfqpoint{4.837368in}{2.232368in}}%
\pgfpathlineto{\pgfqpoint{4.823219in}{2.228527in}}%
\pgfpathlineto{\pgfqpoint{4.809082in}{2.224786in}}%
\pgfpathlineto{\pgfqpoint{4.816858in}{2.236584in}}%
\pgfpathlineto{\pgfqpoint{4.824629in}{2.248334in}}%
\pgfpathlineto{\pgfqpoint{4.832394in}{2.260034in}}%
\pgfpathlineto{\pgfqpoint{4.840154in}{2.271682in}}%
\pgfpathclose%
\pgfusepath{fill}%
\end{pgfscope}%
\begin{pgfscope}%
\pgfpathrectangle{\pgfqpoint{1.150000in}{0.150000in}}{\pgfqpoint{5.700000in}{5.700000in}}%
\pgfusepath{clip}%
\pgfsetbuttcap%
\pgfsetroundjoin%
\definecolor{currentfill}{rgb}{0.281887,0.150881,0.465405}%
\pgfsetfillcolor{currentfill}%
\pgfsetfillopacity{0.700000}%
\pgfsetlinewidth{0.000000pt}%
\definecolor{currentstroke}{rgb}{0.000000,0.000000,0.000000}%
\pgfsetstrokecolor{currentstroke}%
\pgfsetdash{}{0pt}%
\pgfpathmoveto{\pgfqpoint{3.087301in}{2.089096in}}%
\pgfpathlineto{\pgfqpoint{3.101049in}{2.078150in}}%
\pgfpathlineto{\pgfqpoint{3.114798in}{2.067335in}}%
\pgfpathlineto{\pgfqpoint{3.128546in}{2.056650in}}%
\pgfpathlineto{\pgfqpoint{3.142296in}{2.046094in}}%
\pgfpathlineto{\pgfqpoint{3.133802in}{2.046069in}}%
\pgfpathlineto{\pgfqpoint{3.125295in}{2.046282in}}%
\pgfpathlineto{\pgfqpoint{3.116775in}{2.046737in}}%
\pgfpathlineto{\pgfqpoint{3.108240in}{2.047440in}}%
\pgfpathlineto{\pgfqpoint{3.094457in}{2.058472in}}%
\pgfpathlineto{\pgfqpoint{3.080673in}{2.069634in}}%
\pgfpathlineto{\pgfqpoint{3.066890in}{2.080926in}}%
\pgfpathlineto{\pgfqpoint{3.053106in}{2.092349in}}%
\pgfpathlineto{\pgfqpoint{3.061676in}{2.091162in}}%
\pgfpathlineto{\pgfqpoint{3.070232in}{2.090227in}}%
\pgfpathlineto{\pgfqpoint{3.078773in}{2.089540in}}%
\pgfpathlineto{\pgfqpoint{3.087301in}{2.089096in}}%
\pgfpathclose%
\pgfusepath{fill}%
\end{pgfscope}%
\begin{pgfscope}%
\pgfpathrectangle{\pgfqpoint{1.150000in}{0.150000in}}{\pgfqpoint{5.700000in}{5.700000in}}%
\pgfusepath{clip}%
\pgfsetbuttcap%
\pgfsetroundjoin%
\definecolor{currentfill}{rgb}{0.166617,0.463708,0.558119}%
\pgfsetfillcolor{currentfill}%
\pgfsetfillopacity{0.700000}%
\pgfsetlinewidth{0.000000pt}%
\definecolor{currentstroke}{rgb}{0.000000,0.000000,0.000000}%
\pgfsetstrokecolor{currentstroke}%
\pgfsetdash{}{0pt}%
\pgfpathmoveto{\pgfqpoint{5.515312in}{2.820037in}}%
\pgfpathlineto{\pgfqpoint{5.529798in}{2.827694in}}%
\pgfpathlineto{\pgfqpoint{5.544300in}{2.835451in}}%
\pgfpathlineto{\pgfqpoint{5.558817in}{2.843309in}}%
\pgfpathlineto{\pgfqpoint{5.573350in}{2.851267in}}%
\pgfpathlineto{\pgfqpoint{5.565864in}{2.842042in}}%
\pgfpathlineto{\pgfqpoint{5.558370in}{2.832700in}}%
\pgfpathlineto{\pgfqpoint{5.550867in}{2.823239in}}%
\pgfpathlineto{\pgfqpoint{5.543356in}{2.813660in}}%
\pgfpathlineto{\pgfqpoint{5.528817in}{2.805727in}}%
\pgfpathlineto{\pgfqpoint{5.514293in}{2.797895in}}%
\pgfpathlineto{\pgfqpoint{5.499785in}{2.790164in}}%
\pgfpathlineto{\pgfqpoint{5.485293in}{2.782533in}}%
\pgfpathlineto{\pgfqpoint{5.492810in}{2.792079in}}%
\pgfpathlineto{\pgfqpoint{5.500319in}{2.801512in}}%
\pgfpathlineto{\pgfqpoint{5.507819in}{2.810831in}}%
\pgfpathlineto{\pgfqpoint{5.515312in}{2.820037in}}%
\pgfpathclose%
\pgfusepath{fill}%
\end{pgfscope}%
\begin{pgfscope}%
\pgfpathrectangle{\pgfqpoint{1.150000in}{0.150000in}}{\pgfqpoint{5.700000in}{5.700000in}}%
\pgfusepath{clip}%
\pgfsetbuttcap%
\pgfsetroundjoin%
\definecolor{currentfill}{rgb}{0.268510,0.009605,0.335427}%
\pgfsetfillcolor{currentfill}%
\pgfsetfillopacity{0.700000}%
\pgfsetlinewidth{0.000000pt}%
\definecolor{currentstroke}{rgb}{0.000000,0.000000,0.000000}%
\pgfsetstrokecolor{currentstroke}%
\pgfsetdash{}{0pt}%
\pgfpathmoveto{\pgfqpoint{3.966900in}{1.821723in}}%
\pgfpathlineto{\pgfqpoint{3.980720in}{1.818848in}}%
\pgfpathlineto{\pgfqpoint{3.994547in}{1.816077in}}%
\pgfpathlineto{\pgfqpoint{4.008381in}{1.813411in}}%
\pgfpathlineto{\pgfqpoint{4.022223in}{1.810849in}}%
\pgfpathlineto{\pgfqpoint{4.014198in}{1.802217in}}%
\pgfpathlineto{\pgfqpoint{4.006168in}{1.793679in}}%
\pgfpathlineto{\pgfqpoint{3.998131in}{1.785237in}}%
\pgfpathlineto{\pgfqpoint{3.990089in}{1.776894in}}%
\pgfpathlineto{\pgfqpoint{3.976235in}{1.779831in}}%
\pgfpathlineto{\pgfqpoint{3.962387in}{1.782871in}}%
\pgfpathlineto{\pgfqpoint{3.948547in}{1.786015in}}%
\pgfpathlineto{\pgfqpoint{3.934713in}{1.789264in}}%
\pgfpathlineto{\pgfqpoint{3.942769in}{1.797225in}}%
\pgfpathlineto{\pgfqpoint{3.950819in}{1.805292in}}%
\pgfpathlineto{\pgfqpoint{3.958862in}{1.813459in}}%
\pgfpathlineto{\pgfqpoint{3.966900in}{1.821723in}}%
\pgfpathclose%
\pgfusepath{fill}%
\end{pgfscope}%
\begin{pgfscope}%
\pgfpathrectangle{\pgfqpoint{1.150000in}{0.150000in}}{\pgfqpoint{5.700000in}{5.700000in}}%
\pgfusepath{clip}%
\pgfsetbuttcap%
\pgfsetroundjoin%
\definecolor{currentfill}{rgb}{0.126453,0.570633,0.549841}%
\pgfsetfillcolor{currentfill}%
\pgfsetfillopacity{0.700000}%
\pgfsetlinewidth{0.000000pt}%
\definecolor{currentstroke}{rgb}{0.000000,0.000000,0.000000}%
\pgfsetstrokecolor{currentstroke}%
\pgfsetdash{}{0pt}%
\pgfpathmoveto{\pgfqpoint{5.896381in}{3.112818in}}%
\pgfpathlineto{\pgfqpoint{5.911081in}{3.121917in}}%
\pgfpathlineto{\pgfqpoint{5.925798in}{3.131117in}}%
\pgfpathlineto{\pgfqpoint{5.940531in}{3.140418in}}%
\pgfpathlineto{\pgfqpoint{5.955282in}{3.149821in}}%
\pgfpathlineto{\pgfqpoint{5.948016in}{3.143191in}}%
\pgfpathlineto{\pgfqpoint{5.940741in}{3.136435in}}%
\pgfpathlineto{\pgfqpoint{5.933455in}{3.129553in}}%
\pgfpathlineto{\pgfqpoint{5.926159in}{3.122543in}}%
\pgfpathlineto{\pgfqpoint{5.911396in}{3.113066in}}%
\pgfpathlineto{\pgfqpoint{5.896651in}{3.103691in}}%
\pgfpathlineto{\pgfqpoint{5.881923in}{3.094416in}}%
\pgfpathlineto{\pgfqpoint{5.867211in}{3.085243in}}%
\pgfpathlineto{\pgfqpoint{5.874518in}{3.092320in}}%
\pgfpathlineto{\pgfqpoint{5.881816in}{3.099274in}}%
\pgfpathlineto{\pgfqpoint{5.889103in}{3.106107in}}%
\pgfpathlineto{\pgfqpoint{5.896381in}{3.112818in}}%
\pgfpathclose%
\pgfusepath{fill}%
\end{pgfscope}%
\begin{pgfscope}%
\pgfpathrectangle{\pgfqpoint{1.150000in}{0.150000in}}{\pgfqpoint{5.700000in}{5.700000in}}%
\pgfusepath{clip}%
\pgfsetbuttcap%
\pgfsetroundjoin%
\definecolor{currentfill}{rgb}{0.206756,0.371758,0.553117}%
\pgfsetfillcolor{currentfill}%
\pgfsetfillopacity{0.700000}%
\pgfsetlinewidth{0.000000pt}%
\definecolor{currentstroke}{rgb}{0.000000,0.000000,0.000000}%
\pgfsetstrokecolor{currentstroke}%
\pgfsetdash{}{0pt}%
\pgfpathmoveto{\pgfqpoint{5.221580in}{2.577753in}}%
\pgfpathlineto{\pgfqpoint{5.235909in}{2.583988in}}%
\pgfpathlineto{\pgfqpoint{5.250252in}{2.590323in}}%
\pgfpathlineto{\pgfqpoint{5.264609in}{2.596759in}}%
\pgfpathlineto{\pgfqpoint{5.278981in}{2.603294in}}%
\pgfpathlineto{\pgfqpoint{5.271358in}{2.592484in}}%
\pgfpathlineto{\pgfqpoint{5.263728in}{2.581576in}}%
\pgfpathlineto{\pgfqpoint{5.256091in}{2.570569in}}%
\pgfpathlineto{\pgfqpoint{5.248447in}{2.559465in}}%
\pgfpathlineto{\pgfqpoint{5.234072in}{2.553032in}}%
\pgfpathlineto{\pgfqpoint{5.219711in}{2.546700in}}%
\pgfpathlineto{\pgfqpoint{5.205365in}{2.540468in}}%
\pgfpathlineto{\pgfqpoint{5.191032in}{2.534335in}}%
\pgfpathlineto{\pgfqpoint{5.198679in}{2.545329in}}%
\pgfpathlineto{\pgfqpoint{5.206319in}{2.556231in}}%
\pgfpathlineto{\pgfqpoint{5.213953in}{2.567039in}}%
\pgfpathlineto{\pgfqpoint{5.221580in}{2.577753in}}%
\pgfpathclose%
\pgfusepath{fill}%
\end{pgfscope}%
\begin{pgfscope}%
\pgfpathrectangle{\pgfqpoint{1.150000in}{0.150000in}}{\pgfqpoint{5.700000in}{5.700000in}}%
\pgfusepath{clip}%
\pgfsetbuttcap%
\pgfsetroundjoin%
\definecolor{currentfill}{rgb}{0.121148,0.592739,0.544641}%
\pgfsetfillcolor{currentfill}%
\pgfsetfillopacity{0.700000}%
\pgfsetlinewidth{0.000000pt}%
\definecolor{currentstroke}{rgb}{0.000000,0.000000,0.000000}%
\pgfsetstrokecolor{currentstroke}%
\pgfsetdash{}{0pt}%
\pgfpathmoveto{\pgfqpoint{5.984246in}{3.175104in}}%
\pgfpathlineto{\pgfqpoint{5.999002in}{3.184514in}}%
\pgfpathlineto{\pgfqpoint{6.013774in}{3.194024in}}%
\pgfpathlineto{\pgfqpoint{6.028564in}{3.203636in}}%
\pgfpathlineto{\pgfqpoint{6.021348in}{3.197574in}}%
\pgfpathlineto{\pgfqpoint{6.014122in}{3.191387in}}%
\pgfpathlineto{\pgfqpoint{6.006886in}{3.185075in}}%
\pgfpathlineto{\pgfqpoint{5.999639in}{3.178635in}}%
\pgfpathlineto{\pgfqpoint{5.984836in}{3.168929in}}%
\pgfpathlineto{\pgfqpoint{5.970051in}{3.159324in}}%
\pgfpathlineto{\pgfqpoint{5.955282in}{3.149821in}}%
\pgfpathlineto{\pgfqpoint{5.962538in}{3.156325in}}%
\pgfpathlineto{\pgfqpoint{5.969784in}{3.162707in}}%
\pgfpathlineto{\pgfqpoint{5.977020in}{3.168966in}}%
\pgfpathlineto{\pgfqpoint{5.984246in}{3.175104in}}%
\pgfpathclose%
\pgfusepath{fill}%
\end{pgfscope}%
\begin{pgfscope}%
\pgfpathrectangle{\pgfqpoint{1.150000in}{0.150000in}}{\pgfqpoint{5.700000in}{5.700000in}}%
\pgfusepath{clip}%
\pgfsetbuttcap%
\pgfsetroundjoin%
\definecolor{currentfill}{rgb}{0.252194,0.269783,0.531579}%
\pgfsetfillcolor{currentfill}%
\pgfsetfillopacity{0.700000}%
\pgfsetlinewidth{0.000000pt}%
\definecolor{currentstroke}{rgb}{0.000000,0.000000,0.000000}%
\pgfsetstrokecolor{currentstroke}%
\pgfsetdash{}{0pt}%
\pgfpathmoveto{\pgfqpoint{4.927739in}{2.334777in}}%
\pgfpathlineto{\pgfqpoint{4.941920in}{2.339291in}}%
\pgfpathlineto{\pgfqpoint{4.956114in}{2.343904in}}%
\pgfpathlineto{\pgfqpoint{4.970321in}{2.348617in}}%
\pgfpathlineto{\pgfqpoint{4.984540in}{2.353429in}}%
\pgfpathlineto{\pgfqpoint{4.976807in}{2.341664in}}%
\pgfpathlineto{\pgfqpoint{4.969069in}{2.329831in}}%
\pgfpathlineto{\pgfqpoint{4.961325in}{2.317933in}}%
\pgfpathlineto{\pgfqpoint{4.953576in}{2.305970in}}%
\pgfpathlineto{\pgfqpoint{4.939354in}{2.301335in}}%
\pgfpathlineto{\pgfqpoint{4.925145in}{2.296800in}}%
\pgfpathlineto{\pgfqpoint{4.910949in}{2.292364in}}%
\pgfpathlineto{\pgfqpoint{4.896765in}{2.288028in}}%
\pgfpathlineto{\pgfqpoint{4.904517in}{2.299807in}}%
\pgfpathlineto{\pgfqpoint{4.912263in}{2.311525in}}%
\pgfpathlineto{\pgfqpoint{4.920004in}{2.323183in}}%
\pgfpathlineto{\pgfqpoint{4.927739in}{2.334777in}}%
\pgfpathclose%
\pgfusepath{fill}%
\end{pgfscope}%
\begin{pgfscope}%
\pgfpathrectangle{\pgfqpoint{1.150000in}{0.150000in}}{\pgfqpoint{5.700000in}{5.700000in}}%
\pgfusepath{clip}%
\pgfsetbuttcap%
\pgfsetroundjoin%
\definecolor{currentfill}{rgb}{0.281446,0.084320,0.407414}%
\pgfsetfillcolor{currentfill}%
\pgfsetfillopacity{0.700000}%
\pgfsetlinewidth{0.000000pt}%
\definecolor{currentstroke}{rgb}{0.000000,0.000000,0.000000}%
\pgfsetstrokecolor{currentstroke}%
\pgfsetdash{}{0pt}%
\pgfpathmoveto{\pgfqpoint{4.371730in}{1.952850in}}%
\pgfpathlineto{\pgfqpoint{4.385674in}{1.953341in}}%
\pgfpathlineto{\pgfqpoint{4.399628in}{1.953933in}}%
\pgfpathlineto{\pgfqpoint{4.413591in}{1.954626in}}%
\pgfpathlineto{\pgfqpoint{4.427563in}{1.955420in}}%
\pgfpathlineto{\pgfqpoint{4.419667in}{1.944274in}}%
\pgfpathlineto{\pgfqpoint{4.411767in}{1.933149in}}%
\pgfpathlineto{\pgfqpoint{4.403861in}{1.922045in}}%
\pgfpathlineto{\pgfqpoint{4.395951in}{1.910967in}}%
\pgfpathlineto{\pgfqpoint{4.381972in}{1.910477in}}%
\pgfpathlineto{\pgfqpoint{4.368003in}{1.910087in}}%
\pgfpathlineto{\pgfqpoint{4.354043in}{1.909798in}}%
\pgfpathlineto{\pgfqpoint{4.340092in}{1.909609in}}%
\pgfpathlineto{\pgfqpoint{4.348009in}{1.920378in}}%
\pgfpathlineto{\pgfqpoint{4.355921in}{1.931176in}}%
\pgfpathlineto{\pgfqpoint{4.363828in}{1.942001in}}%
\pgfpathlineto{\pgfqpoint{4.371730in}{1.952850in}}%
\pgfpathclose%
\pgfusepath{fill}%
\end{pgfscope}%
\begin{pgfscope}%
\pgfpathrectangle{\pgfqpoint{1.150000in}{0.150000in}}{\pgfqpoint{5.700000in}{5.700000in}}%
\pgfusepath{clip}%
\pgfsetbuttcap%
\pgfsetroundjoin%
\definecolor{currentfill}{rgb}{0.283091,0.110553,0.431554}%
\pgfsetfillcolor{currentfill}%
\pgfsetfillopacity{0.700000}%
\pgfsetlinewidth{0.000000pt}%
\definecolor{currentstroke}{rgb}{0.000000,0.000000,0.000000}%
\pgfsetstrokecolor{currentstroke}%
\pgfsetdash{}{0pt}%
\pgfpathmoveto{\pgfqpoint{4.459098in}{2.000141in}}%
\pgfpathlineto{\pgfqpoint{4.473075in}{2.001320in}}%
\pgfpathlineto{\pgfqpoint{4.487062in}{2.002600in}}%
\pgfpathlineto{\pgfqpoint{4.501059in}{2.003980in}}%
\pgfpathlineto{\pgfqpoint{4.515066in}{2.005460in}}%
\pgfpathlineto{\pgfqpoint{4.507195in}{1.993985in}}%
\pgfpathlineto{\pgfqpoint{4.499319in}{1.982514in}}%
\pgfpathlineto{\pgfqpoint{4.491438in}{1.971050in}}%
\pgfpathlineto{\pgfqpoint{4.483552in}{1.959596in}}%
\pgfpathlineto{\pgfqpoint{4.469540in}{1.958402in}}%
\pgfpathlineto{\pgfqpoint{4.455538in}{1.957308in}}%
\pgfpathlineto{\pgfqpoint{4.441546in}{1.956313in}}%
\pgfpathlineto{\pgfqpoint{4.427563in}{1.955420in}}%
\pgfpathlineto{\pgfqpoint{4.435454in}{1.966582in}}%
\pgfpathlineto{\pgfqpoint{4.443340in}{1.977758in}}%
\pgfpathlineto{\pgfqpoint{4.451222in}{1.988945in}}%
\pgfpathlineto{\pgfqpoint{4.459098in}{2.000141in}}%
\pgfpathclose%
\pgfusepath{fill}%
\end{pgfscope}%
\begin{pgfscope}%
\pgfpathrectangle{\pgfqpoint{1.150000in}{0.150000in}}{\pgfqpoint{5.700000in}{5.700000in}}%
\pgfusepath{clip}%
\pgfsetbuttcap%
\pgfsetroundjoin%
\definecolor{currentfill}{rgb}{0.283072,0.130895,0.449241}%
\pgfsetfillcolor{currentfill}%
\pgfsetfillopacity{0.700000}%
\pgfsetlinewidth{0.000000pt}%
\definecolor{currentstroke}{rgb}{0.000000,0.000000,0.000000}%
\pgfsetstrokecolor{currentstroke}%
\pgfsetdash{}{0pt}%
\pgfpathmoveto{\pgfqpoint{3.142296in}{2.046094in}}%
\pgfpathlineto{\pgfqpoint{3.156045in}{2.035666in}}%
\pgfpathlineto{\pgfqpoint{3.169796in}{2.025366in}}%
\pgfpathlineto{\pgfqpoint{3.183547in}{2.015193in}}%
\pgfpathlineto{\pgfqpoint{3.197299in}{2.005147in}}%
\pgfpathlineto{\pgfqpoint{3.188838in}{2.004654in}}%
\pgfpathlineto{\pgfqpoint{3.180365in}{2.004394in}}%
\pgfpathlineto{\pgfqpoint{3.171878in}{2.004371in}}%
\pgfpathlineto{\pgfqpoint{3.163378in}{2.004591in}}%
\pgfpathlineto{\pgfqpoint{3.149593in}{2.015113in}}%
\pgfpathlineto{\pgfqpoint{3.135808in}{2.025761in}}%
\pgfpathlineto{\pgfqpoint{3.122024in}{2.036536in}}%
\pgfpathlineto{\pgfqpoint{3.108240in}{2.047440in}}%
\pgfpathlineto{\pgfqpoint{3.116775in}{2.046737in}}%
\pgfpathlineto{\pgfqpoint{3.125295in}{2.046282in}}%
\pgfpathlineto{\pgfqpoint{3.133802in}{2.046069in}}%
\pgfpathlineto{\pgfqpoint{3.142296in}{2.046094in}}%
\pgfpathclose%
\pgfusepath{fill}%
\end{pgfscope}%
\begin{pgfscope}%
\pgfpathrectangle{\pgfqpoint{1.150000in}{0.150000in}}{\pgfqpoint{5.700000in}{5.700000in}}%
\pgfusepath{clip}%
\pgfsetbuttcap%
\pgfsetroundjoin%
\definecolor{currentfill}{rgb}{0.278791,0.062145,0.386592}%
\pgfsetfillcolor{currentfill}%
\pgfsetfillopacity{0.700000}%
\pgfsetlinewidth{0.000000pt}%
\definecolor{currentstroke}{rgb}{0.000000,0.000000,0.000000}%
\pgfsetstrokecolor{currentstroke}%
\pgfsetdash{}{0pt}%
\pgfpathmoveto{\pgfqpoint{4.284380in}{1.909865in}}%
\pgfpathlineto{\pgfqpoint{4.298295in}{1.909649in}}%
\pgfpathlineto{\pgfqpoint{4.312218in}{1.909535in}}%
\pgfpathlineto{\pgfqpoint{4.326151in}{1.909521in}}%
\pgfpathlineto{\pgfqpoint{4.340092in}{1.909609in}}%
\pgfpathlineto{\pgfqpoint{4.332171in}{1.898873in}}%
\pgfpathlineto{\pgfqpoint{4.324244in}{1.888173in}}%
\pgfpathlineto{\pgfqpoint{4.316312in}{1.877511in}}%
\pgfpathlineto{\pgfqpoint{4.308376in}{1.866890in}}%
\pgfpathlineto{\pgfqpoint{4.294427in}{1.867123in}}%
\pgfpathlineto{\pgfqpoint{4.280487in}{1.867458in}}%
\pgfpathlineto{\pgfqpoint{4.266556in}{1.867893in}}%
\pgfpathlineto{\pgfqpoint{4.252633in}{1.868429in}}%
\pgfpathlineto{\pgfqpoint{4.260578in}{1.878722in}}%
\pgfpathlineto{\pgfqpoint{4.268517in}{1.889061in}}%
\pgfpathlineto{\pgfqpoint{4.276451in}{1.899443in}}%
\pgfpathlineto{\pgfqpoint{4.284380in}{1.909865in}}%
\pgfpathclose%
\pgfusepath{fill}%
\end{pgfscope}%
\begin{pgfscope}%
\pgfpathrectangle{\pgfqpoint{1.150000in}{0.150000in}}{\pgfqpoint{5.700000in}{5.700000in}}%
\pgfusepath{clip}%
\pgfsetbuttcap%
\pgfsetroundjoin%
\definecolor{currentfill}{rgb}{0.277941,0.056324,0.381191}%
\pgfsetfillcolor{currentfill}%
\pgfsetfillopacity{0.700000}%
\pgfsetlinewidth{0.000000pt}%
\definecolor{currentstroke}{rgb}{0.000000,0.000000,0.000000}%
\pgfsetstrokecolor{currentstroke}%
\pgfsetdash{}{0pt}%
\pgfpathmoveto{\pgfqpoint{3.395800in}{1.903829in}}%
\pgfpathlineto{\pgfqpoint{3.409546in}{1.895816in}}%
\pgfpathlineto{\pgfqpoint{3.423295in}{1.887920in}}%
\pgfpathlineto{\pgfqpoint{3.437047in}{1.880141in}}%
\pgfpathlineto{\pgfqpoint{3.450802in}{1.872478in}}%
\pgfpathlineto{\pgfqpoint{3.442502in}{1.869321in}}%
\pgfpathlineto{\pgfqpoint{3.434193in}{1.866359in}}%
\pgfpathlineto{\pgfqpoint{3.425873in}{1.863596in}}%
\pgfpathlineto{\pgfqpoint{3.417543in}{1.861036in}}%
\pgfpathlineto{\pgfqpoint{3.403762in}{1.869149in}}%
\pgfpathlineto{\pgfqpoint{3.389983in}{1.877379in}}%
\pgfpathlineto{\pgfqpoint{3.376207in}{1.885727in}}%
\pgfpathlineto{\pgfqpoint{3.362433in}{1.894191in}}%
\pgfpathlineto{\pgfqpoint{3.370791in}{1.896293in}}%
\pgfpathlineto{\pgfqpoint{3.379138in}{1.898603in}}%
\pgfpathlineto{\pgfqpoint{3.387474in}{1.901116in}}%
\pgfpathlineto{\pgfqpoint{3.395800in}{1.903829in}}%
\pgfpathclose%
\pgfusepath{fill}%
\end{pgfscope}%
\begin{pgfscope}%
\pgfpathrectangle{\pgfqpoint{1.150000in}{0.150000in}}{\pgfqpoint{5.700000in}{5.700000in}}%
\pgfusepath{clip}%
\pgfsetbuttcap%
\pgfsetroundjoin%
\definecolor{currentfill}{rgb}{0.282884,0.135920,0.453427}%
\pgfsetfillcolor{currentfill}%
\pgfsetfillopacity{0.700000}%
\pgfsetlinewidth{0.000000pt}%
\definecolor{currentstroke}{rgb}{0.000000,0.000000,0.000000}%
\pgfsetstrokecolor{currentstroke}%
\pgfsetdash{}{0pt}%
\pgfpathmoveto{\pgfqpoint{4.546503in}{2.051351in}}%
\pgfpathlineto{\pgfqpoint{4.560516in}{2.053198in}}%
\pgfpathlineto{\pgfqpoint{4.574539in}{2.055146in}}%
\pgfpathlineto{\pgfqpoint{4.588573in}{2.057194in}}%
\pgfpathlineto{\pgfqpoint{4.602618in}{2.059341in}}%
\pgfpathlineto{\pgfqpoint{4.594770in}{2.047614in}}%
\pgfpathlineto{\pgfqpoint{4.586917in}{2.035876in}}%
\pgfpathlineto{\pgfqpoint{4.579060in}{2.024130in}}%
\pgfpathlineto{\pgfqpoint{4.571198in}{2.012379in}}%
\pgfpathlineto{\pgfqpoint{4.557149in}{2.010499in}}%
\pgfpathlineto{\pgfqpoint{4.543111in}{2.008720in}}%
\pgfpathlineto{\pgfqpoint{4.529083in}{2.007040in}}%
\pgfpathlineto{\pgfqpoint{4.515066in}{2.005460in}}%
\pgfpathlineto{\pgfqpoint{4.522933in}{2.016936in}}%
\pgfpathlineto{\pgfqpoint{4.530794in}{2.028412in}}%
\pgfpathlineto{\pgfqpoint{4.538651in}{2.039884in}}%
\pgfpathlineto{\pgfqpoint{4.546503in}{2.051351in}}%
\pgfpathclose%
\pgfusepath{fill}%
\end{pgfscope}%
\begin{pgfscope}%
\pgfpathrectangle{\pgfqpoint{1.150000in}{0.150000in}}{\pgfqpoint{5.700000in}{5.700000in}}%
\pgfusepath{clip}%
\pgfsetbuttcap%
\pgfsetroundjoin%
\definecolor{currentfill}{rgb}{0.267004,0.004874,0.329415}%
\pgfsetfillcolor{currentfill}%
\pgfsetfillopacity{0.700000}%
\pgfsetlinewidth{0.000000pt}%
\definecolor{currentstroke}{rgb}{0.000000,0.000000,0.000000}%
\pgfsetstrokecolor{currentstroke}%
\pgfsetdash{}{0pt}%
\pgfpathmoveto{\pgfqpoint{3.736697in}{1.810481in}}%
\pgfpathlineto{\pgfqpoint{3.750477in}{1.805563in}}%
\pgfpathlineto{\pgfqpoint{3.764262in}{1.800754in}}%
\pgfpathlineto{\pgfqpoint{3.778053in}{1.796052in}}%
\pgfpathlineto{\pgfqpoint{3.791849in}{1.791458in}}%
\pgfpathlineto{\pgfqpoint{3.783727in}{1.784883in}}%
\pgfpathlineto{\pgfqpoint{3.775597in}{1.778444in}}%
\pgfpathlineto{\pgfqpoint{3.767460in}{1.772146in}}%
\pgfpathlineto{\pgfqpoint{3.759316in}{1.765993in}}%
\pgfpathlineto{\pgfqpoint{3.745501in}{1.770997in}}%
\pgfpathlineto{\pgfqpoint{3.731692in}{1.776110in}}%
\pgfpathlineto{\pgfqpoint{3.717889in}{1.781330in}}%
\pgfpathlineto{\pgfqpoint{3.704089in}{1.786658in}}%
\pgfpathlineto{\pgfqpoint{3.712253in}{1.792393in}}%
\pgfpathlineto{\pgfqpoint{3.720409in}{1.798279in}}%
\pgfpathlineto{\pgfqpoint{3.728557in}{1.804309in}}%
\pgfpathlineto{\pgfqpoint{3.736697in}{1.810481in}}%
\pgfpathclose%
\pgfusepath{fill}%
\end{pgfscope}%
\begin{pgfscope}%
\pgfpathrectangle{\pgfqpoint{1.150000in}{0.150000in}}{\pgfqpoint{5.700000in}{5.700000in}}%
\pgfusepath{clip}%
\pgfsetbuttcap%
\pgfsetroundjoin%
\definecolor{currentfill}{rgb}{0.156270,0.489624,0.557936}%
\pgfsetfillcolor{currentfill}%
\pgfsetfillopacity{0.700000}%
\pgfsetlinewidth{0.000000pt}%
\definecolor{currentstroke}{rgb}{0.000000,0.000000,0.000000}%
\pgfsetstrokecolor{currentstroke}%
\pgfsetdash{}{0pt}%
\pgfpathmoveto{\pgfqpoint{5.603211in}{2.886985in}}%
\pgfpathlineto{\pgfqpoint{5.617753in}{2.895049in}}%
\pgfpathlineto{\pgfqpoint{5.632310in}{2.903215in}}%
\pgfpathlineto{\pgfqpoint{5.646884in}{2.911480in}}%
\pgfpathlineto{\pgfqpoint{5.661474in}{2.919847in}}%
\pgfpathlineto{\pgfqpoint{5.654029in}{2.911096in}}%
\pgfpathlineto{\pgfqpoint{5.646575in}{2.902222in}}%
\pgfpathlineto{\pgfqpoint{5.639112in}{2.893225in}}%
\pgfpathlineto{\pgfqpoint{5.631641in}{2.884106in}}%
\pgfpathlineto{\pgfqpoint{5.617044in}{2.875745in}}%
\pgfpathlineto{\pgfqpoint{5.602464in}{2.867485in}}%
\pgfpathlineto{\pgfqpoint{5.587899in}{2.859325in}}%
\pgfpathlineto{\pgfqpoint{5.573350in}{2.851267in}}%
\pgfpathlineto{\pgfqpoint{5.580828in}{2.860373in}}%
\pgfpathlineto{\pgfqpoint{5.588297in}{2.869361in}}%
\pgfpathlineto{\pgfqpoint{5.595758in}{2.878232in}}%
\pgfpathlineto{\pgfqpoint{5.603211in}{2.886985in}}%
\pgfpathclose%
\pgfusepath{fill}%
\end{pgfscope}%
\begin{pgfscope}%
\pgfpathrectangle{\pgfqpoint{1.150000in}{0.150000in}}{\pgfqpoint{5.700000in}{5.700000in}}%
\pgfusepath{clip}%
\pgfsetbuttcap%
\pgfsetroundjoin%
\definecolor{currentfill}{rgb}{0.274952,0.037752,0.364543}%
\pgfsetfillcolor{currentfill}%
\pgfsetfillopacity{0.700000}%
\pgfsetlinewidth{0.000000pt}%
\definecolor{currentstroke}{rgb}{0.000000,0.000000,0.000000}%
\pgfsetstrokecolor{currentstroke}%
\pgfsetdash{}{0pt}%
\pgfpathmoveto{\pgfqpoint{4.197028in}{1.871589in}}%
\pgfpathlineto{\pgfqpoint{4.210917in}{1.870647in}}%
\pgfpathlineto{\pgfqpoint{4.224814in}{1.869806in}}%
\pgfpathlineto{\pgfqpoint{4.238720in}{1.869067in}}%
\pgfpathlineto{\pgfqpoint{4.252633in}{1.868429in}}%
\pgfpathlineto{\pgfqpoint{4.244684in}{1.858185in}}%
\pgfpathlineto{\pgfqpoint{4.236729in}{1.847994in}}%
\pgfpathlineto{\pgfqpoint{4.228770in}{1.837857in}}%
\pgfpathlineto{\pgfqpoint{4.220805in}{1.827780in}}%
\pgfpathlineto{\pgfqpoint{4.206882in}{1.828756in}}%
\pgfpathlineto{\pgfqpoint{4.192968in}{1.829834in}}%
\pgfpathlineto{\pgfqpoint{4.179062in}{1.831013in}}%
\pgfpathlineto{\pgfqpoint{4.165164in}{1.832294in}}%
\pgfpathlineto{\pgfqpoint{4.173138in}{1.842026in}}%
\pgfpathlineto{\pgfqpoint{4.181107in}{1.851822in}}%
\pgfpathlineto{\pgfqpoint{4.189070in}{1.861677in}}%
\pgfpathlineto{\pgfqpoint{4.197028in}{1.871589in}}%
\pgfpathclose%
\pgfusepath{fill}%
\end{pgfscope}%
\begin{pgfscope}%
\pgfpathrectangle{\pgfqpoint{1.150000in}{0.150000in}}{\pgfqpoint{5.700000in}{5.700000in}}%
\pgfusepath{clip}%
\pgfsetbuttcap%
\pgfsetroundjoin%
\definecolor{currentfill}{rgb}{0.280255,0.165693,0.476498}%
\pgfsetfillcolor{currentfill}%
\pgfsetfillopacity{0.700000}%
\pgfsetlinewidth{0.000000pt}%
\definecolor{currentstroke}{rgb}{0.000000,0.000000,0.000000}%
\pgfsetstrokecolor{currentstroke}%
\pgfsetdash{}{0pt}%
\pgfpathmoveto{\pgfqpoint{4.633961in}{2.106101in}}%
\pgfpathlineto{\pgfqpoint{4.648012in}{2.108599in}}%
\pgfpathlineto{\pgfqpoint{4.662075in}{2.111196in}}%
\pgfpathlineto{\pgfqpoint{4.676149in}{2.113893in}}%
\pgfpathlineto{\pgfqpoint{4.690234in}{2.116690in}}%
\pgfpathlineto{\pgfqpoint{4.682409in}{2.104784in}}%
\pgfpathlineto{\pgfqpoint{4.674579in}{2.092853in}}%
\pgfpathlineto{\pgfqpoint{4.666744in}{2.080901in}}%
\pgfpathlineto{\pgfqpoint{4.658905in}{2.068929in}}%
\pgfpathlineto{\pgfqpoint{4.644817in}{2.066383in}}%
\pgfpathlineto{\pgfqpoint{4.630740in}{2.063936in}}%
\pgfpathlineto{\pgfqpoint{4.616673in}{2.061589in}}%
\pgfpathlineto{\pgfqpoint{4.602618in}{2.059341in}}%
\pgfpathlineto{\pgfqpoint{4.610461in}{2.071056in}}%
\pgfpathlineto{\pgfqpoint{4.618299in}{2.082756in}}%
\pgfpathlineto{\pgfqpoint{4.626132in}{2.094439in}}%
\pgfpathlineto{\pgfqpoint{4.633961in}{2.106101in}}%
\pgfpathclose%
\pgfusepath{fill}%
\end{pgfscope}%
\begin{pgfscope}%
\pgfpathrectangle{\pgfqpoint{1.150000in}{0.150000in}}{\pgfqpoint{5.700000in}{5.700000in}}%
\pgfusepath{clip}%
\pgfsetbuttcap%
\pgfsetroundjoin%
\definecolor{currentfill}{rgb}{0.194100,0.399323,0.555565}%
\pgfsetfillcolor{currentfill}%
\pgfsetfillopacity{0.700000}%
\pgfsetlinewidth{0.000000pt}%
\definecolor{currentstroke}{rgb}{0.000000,0.000000,0.000000}%
\pgfsetstrokecolor{currentstroke}%
\pgfsetdash{}{0pt}%
\pgfpathmoveto{\pgfqpoint{5.309403in}{2.645530in}}%
\pgfpathlineto{\pgfqpoint{5.323785in}{2.652249in}}%
\pgfpathlineto{\pgfqpoint{5.338182in}{2.659068in}}%
\pgfpathlineto{\pgfqpoint{5.352594in}{2.665988in}}%
\pgfpathlineto{\pgfqpoint{5.367020in}{2.673008in}}%
\pgfpathlineto{\pgfqpoint{5.359430in}{2.662523in}}%
\pgfpathlineto{\pgfqpoint{5.351832in}{2.651933in}}%
\pgfpathlineto{\pgfqpoint{5.344226in}{2.641237in}}%
\pgfpathlineto{\pgfqpoint{5.336614in}{2.630436in}}%
\pgfpathlineto{\pgfqpoint{5.322183in}{2.623501in}}%
\pgfpathlineto{\pgfqpoint{5.307768in}{2.616665in}}%
\pgfpathlineto{\pgfqpoint{5.293367in}{2.609929in}}%
\pgfpathlineto{\pgfqpoint{5.278981in}{2.603294in}}%
\pgfpathlineto{\pgfqpoint{5.286597in}{2.614004in}}%
\pgfpathlineto{\pgfqpoint{5.294206in}{2.624614in}}%
\pgfpathlineto{\pgfqpoint{5.301808in}{2.635122in}}%
\pgfpathlineto{\pgfqpoint{5.309403in}{2.645530in}}%
\pgfpathclose%
\pgfusepath{fill}%
\end{pgfscope}%
\begin{pgfscope}%
\pgfpathrectangle{\pgfqpoint{1.150000in}{0.150000in}}{\pgfqpoint{5.700000in}{5.700000in}}%
\pgfusepath{clip}%
\pgfsetbuttcap%
\pgfsetroundjoin%
\definecolor{currentfill}{rgb}{0.271305,0.019942,0.347269}%
\pgfsetfillcolor{currentfill}%
\pgfsetfillopacity{0.700000}%
\pgfsetlinewidth{0.000000pt}%
\definecolor{currentstroke}{rgb}{0.000000,0.000000,0.000000}%
\pgfsetstrokecolor{currentstroke}%
\pgfsetdash{}{0pt}%
\pgfpathmoveto{\pgfqpoint{3.593863in}{1.833227in}}%
\pgfpathlineto{\pgfqpoint{3.607626in}{1.827019in}}%
\pgfpathlineto{\pgfqpoint{3.621394in}{1.820922in}}%
\pgfpathlineto{\pgfqpoint{3.635165in}{1.814936in}}%
\pgfpathlineto{\pgfqpoint{3.648941in}{1.809061in}}%
\pgfpathlineto{\pgfqpoint{3.640749in}{1.803902in}}%
\pgfpathlineto{\pgfqpoint{3.632548in}{1.798905in}}%
\pgfpathlineto{\pgfqpoint{3.624339in}{1.794074in}}%
\pgfpathlineto{\pgfqpoint{3.616121in}{1.789415in}}%
\pgfpathlineto{\pgfqpoint{3.602324in}{1.795720in}}%
\pgfpathlineto{\pgfqpoint{3.588530in}{1.802136in}}%
\pgfpathlineto{\pgfqpoint{3.574741in}{1.808663in}}%
\pgfpathlineto{\pgfqpoint{3.560956in}{1.815301in}}%
\pgfpathlineto{\pgfqpoint{3.569196in}{1.819523in}}%
\pgfpathlineto{\pgfqpoint{3.577428in}{1.823921in}}%
\pgfpathlineto{\pgfqpoint{3.585650in}{1.828491in}}%
\pgfpathlineto{\pgfqpoint{3.593863in}{1.833227in}}%
\pgfpathclose%
\pgfusepath{fill}%
\end{pgfscope}%
\begin{pgfscope}%
\pgfpathrectangle{\pgfqpoint{1.150000in}{0.150000in}}{\pgfqpoint{5.700000in}{5.700000in}}%
\pgfusepath{clip}%
\pgfsetbuttcap%
\pgfsetroundjoin%
\definecolor{currentfill}{rgb}{0.239346,0.300855,0.540844}%
\pgfsetfillcolor{currentfill}%
\pgfsetfillopacity{0.700000}%
\pgfsetlinewidth{0.000000pt}%
\definecolor{currentstroke}{rgb}{0.000000,0.000000,0.000000}%
\pgfsetstrokecolor{currentstroke}%
\pgfsetdash{}{0pt}%
\pgfpathmoveto{\pgfqpoint{5.015412in}{2.399787in}}%
\pgfpathlineto{\pgfqpoint{5.029642in}{2.404858in}}%
\pgfpathlineto{\pgfqpoint{5.043885in}{2.410030in}}%
\pgfpathlineto{\pgfqpoint{5.058141in}{2.415301in}}%
\pgfpathlineto{\pgfqpoint{5.072410in}{2.420671in}}%
\pgfpathlineto{\pgfqpoint{5.064704in}{2.409038in}}%
\pgfpathlineto{\pgfqpoint{5.056991in}{2.397327in}}%
\pgfpathlineto{\pgfqpoint{5.049272in}{2.385539in}}%
\pgfpathlineto{\pgfqpoint{5.041548in}{2.373676in}}%
\pgfpathlineto{\pgfqpoint{5.027276in}{2.368465in}}%
\pgfpathlineto{\pgfqpoint{5.013018in}{2.363353in}}%
\pgfpathlineto{\pgfqpoint{4.998772in}{2.358342in}}%
\pgfpathlineto{\pgfqpoint{4.984540in}{2.353429in}}%
\pgfpathlineto{\pgfqpoint{4.992267in}{2.365125in}}%
\pgfpathlineto{\pgfqpoint{4.999988in}{2.376751in}}%
\pgfpathlineto{\pgfqpoint{5.007703in}{2.388306in}}%
\pgfpathlineto{\pgfqpoint{5.015412in}{2.399787in}}%
\pgfpathclose%
\pgfusepath{fill}%
\end{pgfscope}%
\begin{pgfscope}%
\pgfpathrectangle{\pgfqpoint{1.150000in}{0.150000in}}{\pgfqpoint{5.700000in}{5.700000in}}%
\pgfusepath{clip}%
\pgfsetbuttcap%
\pgfsetroundjoin%
\definecolor{currentfill}{rgb}{0.267004,0.004874,0.329415}%
\pgfsetfillcolor{currentfill}%
\pgfsetfillopacity{0.700000}%
\pgfsetlinewidth{0.000000pt}%
\definecolor{currentstroke}{rgb}{0.000000,0.000000,0.000000}%
\pgfsetstrokecolor{currentstroke}%
\pgfsetdash{}{0pt}%
\pgfpathmoveto{\pgfqpoint{3.879440in}{1.803309in}}%
\pgfpathlineto{\pgfqpoint{3.893249in}{1.799640in}}%
\pgfpathlineto{\pgfqpoint{3.907064in}{1.796076in}}%
\pgfpathlineto{\pgfqpoint{3.920885in}{1.792617in}}%
\pgfpathlineto{\pgfqpoint{3.934713in}{1.789264in}}%
\pgfpathlineto{\pgfqpoint{3.926650in}{1.781411in}}%
\pgfpathlineto{\pgfqpoint{3.918581in}{1.773669in}}%
\pgfpathlineto{\pgfqpoint{3.910505in}{1.766044in}}%
\pgfpathlineto{\pgfqpoint{3.902422in}{1.758538in}}%
\pgfpathlineto{\pgfqpoint{3.888580in}{1.762283in}}%
\pgfpathlineto{\pgfqpoint{3.874744in}{1.766134in}}%
\pgfpathlineto{\pgfqpoint{3.860913in}{1.770089in}}%
\pgfpathlineto{\pgfqpoint{3.847089in}{1.774151in}}%
\pgfpathlineto{\pgfqpoint{3.855187in}{1.781258in}}%
\pgfpathlineto{\pgfqpoint{3.863278in}{1.788489in}}%
\pgfpathlineto{\pgfqpoint{3.871362in}{1.795841in}}%
\pgfpathlineto{\pgfqpoint{3.879440in}{1.803309in}}%
\pgfpathclose%
\pgfusepath{fill}%
\end{pgfscope}%
\begin{pgfscope}%
\pgfpathrectangle{\pgfqpoint{1.150000in}{0.150000in}}{\pgfqpoint{5.700000in}{5.700000in}}%
\pgfusepath{clip}%
\pgfsetbuttcap%
\pgfsetroundjoin%
\definecolor{currentfill}{rgb}{0.216210,0.351535,0.550627}%
\pgfsetfillcolor{currentfill}%
\pgfsetfillopacity{0.700000}%
\pgfsetlinewidth{0.000000pt}%
\definecolor{currentstroke}{rgb}{0.000000,0.000000,0.000000}%
\pgfsetstrokecolor{currentstroke}%
\pgfsetdash{}{0pt}%
\pgfpathmoveto{\pgfqpoint{2.611205in}{2.532517in}}%
\pgfpathlineto{\pgfqpoint{2.625061in}{2.516399in}}%
\pgfpathlineto{\pgfqpoint{2.638913in}{2.500444in}}%
\pgfpathlineto{\pgfqpoint{2.652761in}{2.484653in}}%
\pgfpathlineto{\pgfqpoint{2.666605in}{2.469023in}}%
\pgfpathlineto{\pgfqpoint{2.657728in}{2.473954in}}%
\pgfpathlineto{\pgfqpoint{2.648833in}{2.479186in}}%
\pgfpathlineto{\pgfqpoint{2.639919in}{2.484724in}}%
\pgfpathlineto{\pgfqpoint{2.630985in}{2.490573in}}%
\pgfpathlineto{\pgfqpoint{2.617094in}{2.506718in}}%
\pgfpathlineto{\pgfqpoint{2.603198in}{2.523026in}}%
\pgfpathlineto{\pgfqpoint{2.589297in}{2.539497in}}%
\pgfpathlineto{\pgfqpoint{2.575392in}{2.556134in}}%
\pgfpathlineto{\pgfqpoint{2.584375in}{2.549759in}}%
\pgfpathlineto{\pgfqpoint{2.593338in}{2.543702in}}%
\pgfpathlineto{\pgfqpoint{2.602281in}{2.537956in}}%
\pgfpathlineto{\pgfqpoint{2.611205in}{2.532517in}}%
\pgfpathclose%
\pgfusepath{fill}%
\end{pgfscope}%
\begin{pgfscope}%
\pgfpathrectangle{\pgfqpoint{1.150000in}{0.150000in}}{\pgfqpoint{5.700000in}{5.700000in}}%
\pgfusepath{clip}%
\pgfsetbuttcap%
\pgfsetroundjoin%
\definecolor{currentfill}{rgb}{0.227802,0.326594,0.546532}%
\pgfsetfillcolor{currentfill}%
\pgfsetfillopacity{0.700000}%
\pgfsetlinewidth{0.000000pt}%
\definecolor{currentstroke}{rgb}{0.000000,0.000000,0.000000}%
\pgfsetstrokecolor{currentstroke}%
\pgfsetdash{}{0pt}%
\pgfpathmoveto{\pgfqpoint{2.666605in}{2.469023in}}%
\pgfpathlineto{\pgfqpoint{2.680444in}{2.453554in}}%
\pgfpathlineto{\pgfqpoint{2.694280in}{2.438244in}}%
\pgfpathlineto{\pgfqpoint{2.708111in}{2.423092in}}%
\pgfpathlineto{\pgfqpoint{2.721939in}{2.408096in}}%
\pgfpathlineto{\pgfqpoint{2.713109in}{2.412522in}}%
\pgfpathlineto{\pgfqpoint{2.704260in}{2.417243in}}%
\pgfpathlineto{\pgfqpoint{2.695393in}{2.422264in}}%
\pgfpathlineto{\pgfqpoint{2.686507in}{2.427592in}}%
\pgfpathlineto{\pgfqpoint{2.672632in}{2.443100in}}%
\pgfpathlineto{\pgfqpoint{2.658754in}{2.458765in}}%
\pgfpathlineto{\pgfqpoint{2.644872in}{2.474589in}}%
\pgfpathlineto{\pgfqpoint{2.630985in}{2.490573in}}%
\pgfpathlineto{\pgfqpoint{2.639919in}{2.484724in}}%
\pgfpathlineto{\pgfqpoint{2.648833in}{2.479186in}}%
\pgfpathlineto{\pgfqpoint{2.657728in}{2.473954in}}%
\pgfpathlineto{\pgfqpoint{2.666605in}{2.469023in}}%
\pgfpathclose%
\pgfusepath{fill}%
\end{pgfscope}%
\begin{pgfscope}%
\pgfpathrectangle{\pgfqpoint{1.150000in}{0.150000in}}{\pgfqpoint{5.700000in}{5.700000in}}%
\pgfusepath{clip}%
\pgfsetbuttcap%
\pgfsetroundjoin%
\definecolor{currentfill}{rgb}{0.272594,0.025563,0.353093}%
\pgfsetfillcolor{currentfill}%
\pgfsetfillopacity{0.700000}%
\pgfsetlinewidth{0.000000pt}%
\definecolor{currentstroke}{rgb}{0.000000,0.000000,0.000000}%
\pgfsetstrokecolor{currentstroke}%
\pgfsetdash{}{0pt}%
\pgfpathmoveto{\pgfqpoint{4.109651in}{1.838440in}}%
\pgfpathlineto{\pgfqpoint{4.123518in}{1.836750in}}%
\pgfpathlineto{\pgfqpoint{4.137392in}{1.835162in}}%
\pgfpathlineto{\pgfqpoint{4.151274in}{1.833677in}}%
\pgfpathlineto{\pgfqpoint{4.165164in}{1.832294in}}%
\pgfpathlineto{\pgfqpoint{4.157184in}{1.822628in}}%
\pgfpathlineto{\pgfqpoint{4.149200in}{1.813032in}}%
\pgfpathlineto{\pgfqpoint{4.141209in}{1.803510in}}%
\pgfpathlineto{\pgfqpoint{4.133213in}{1.794064in}}%
\pgfpathlineto{\pgfqpoint{4.119313in}{1.795803in}}%
\pgfpathlineto{\pgfqpoint{4.105421in}{1.797644in}}%
\pgfpathlineto{\pgfqpoint{4.091536in}{1.799588in}}%
\pgfpathlineto{\pgfqpoint{4.077659in}{1.801634in}}%
\pgfpathlineto{\pgfqpoint{4.085665in}{1.810717in}}%
\pgfpathlineto{\pgfqpoint{4.093666in}{1.819881in}}%
\pgfpathlineto{\pgfqpoint{4.101662in}{1.829123in}}%
\pgfpathlineto{\pgfqpoint{4.109651in}{1.838440in}}%
\pgfpathclose%
\pgfusepath{fill}%
\end{pgfscope}%
\begin{pgfscope}%
\pgfpathrectangle{\pgfqpoint{1.150000in}{0.150000in}}{\pgfqpoint{5.700000in}{5.700000in}}%
\pgfusepath{clip}%
\pgfsetbuttcap%
\pgfsetroundjoin%
\definecolor{currentfill}{rgb}{0.203063,0.379716,0.553925}%
\pgfsetfillcolor{currentfill}%
\pgfsetfillopacity{0.700000}%
\pgfsetlinewidth{0.000000pt}%
\definecolor{currentstroke}{rgb}{0.000000,0.000000,0.000000}%
\pgfsetstrokecolor{currentstroke}%
\pgfsetdash{}{0pt}%
\pgfpathmoveto{\pgfqpoint{2.555730in}{2.598663in}}%
\pgfpathlineto{\pgfqpoint{2.569606in}{2.581873in}}%
\pgfpathlineto{\pgfqpoint{2.583477in}{2.565253in}}%
\pgfpathlineto{\pgfqpoint{2.597344in}{2.548802in}}%
\pgfpathlineto{\pgfqpoint{2.611205in}{2.532517in}}%
\pgfpathlineto{\pgfqpoint{2.602281in}{2.537956in}}%
\pgfpathlineto{\pgfqpoint{2.593338in}{2.543702in}}%
\pgfpathlineto{\pgfqpoint{2.584375in}{2.549759in}}%
\pgfpathlineto{\pgfqpoint{2.575392in}{2.556134in}}%
\pgfpathlineto{\pgfqpoint{2.561481in}{2.572937in}}%
\pgfpathlineto{\pgfqpoint{2.547566in}{2.589908in}}%
\pgfpathlineto{\pgfqpoint{2.533644in}{2.607048in}}%
\pgfpathlineto{\pgfqpoint{2.519718in}{2.624359in}}%
\pgfpathlineto{\pgfqpoint{2.528752in}{2.617456in}}%
\pgfpathlineto{\pgfqpoint{2.537765in}{2.610876in}}%
\pgfpathlineto{\pgfqpoint{2.546758in}{2.604614in}}%
\pgfpathlineto{\pgfqpoint{2.555730in}{2.598663in}}%
\pgfpathclose%
\pgfusepath{fill}%
\end{pgfscope}%
\begin{pgfscope}%
\pgfpathrectangle{\pgfqpoint{1.150000in}{0.150000in}}{\pgfqpoint{5.700000in}{5.700000in}}%
\pgfusepath{clip}%
\pgfsetbuttcap%
\pgfsetroundjoin%
\definecolor{currentfill}{rgb}{0.283197,0.115680,0.436115}%
\pgfsetfillcolor{currentfill}%
\pgfsetfillopacity{0.700000}%
\pgfsetlinewidth{0.000000pt}%
\definecolor{currentstroke}{rgb}{0.000000,0.000000,0.000000}%
\pgfsetstrokecolor{currentstroke}%
\pgfsetdash{}{0pt}%
\pgfpathmoveto{\pgfqpoint{3.197299in}{2.005147in}}%
\pgfpathlineto{\pgfqpoint{3.211052in}{1.995225in}}%
\pgfpathlineto{\pgfqpoint{3.224806in}{1.985429in}}%
\pgfpathlineto{\pgfqpoint{3.238561in}{1.975757in}}%
\pgfpathlineto{\pgfqpoint{3.252318in}{1.966208in}}%
\pgfpathlineto{\pgfqpoint{3.243889in}{1.965249in}}%
\pgfpathlineto{\pgfqpoint{3.235448in}{1.964517in}}%
\pgfpathlineto{\pgfqpoint{3.226994in}{1.964018in}}%
\pgfpathlineto{\pgfqpoint{3.218528in}{1.963757in}}%
\pgfpathlineto{\pgfqpoint{3.204739in}{1.973779in}}%
\pgfpathlineto{\pgfqpoint{3.190951in}{1.983925in}}%
\pgfpathlineto{\pgfqpoint{3.177164in}{1.994195in}}%
\pgfpathlineto{\pgfqpoint{3.163378in}{2.004591in}}%
\pgfpathlineto{\pgfqpoint{3.171878in}{2.004371in}}%
\pgfpathlineto{\pgfqpoint{3.180365in}{2.004394in}}%
\pgfpathlineto{\pgfqpoint{3.188838in}{2.004654in}}%
\pgfpathlineto{\pgfqpoint{3.197299in}{2.005147in}}%
\pgfpathclose%
\pgfusepath{fill}%
\end{pgfscope}%
\begin{pgfscope}%
\pgfpathrectangle{\pgfqpoint{1.150000in}{0.150000in}}{\pgfqpoint{5.700000in}{5.700000in}}%
\pgfusepath{clip}%
\pgfsetbuttcap%
\pgfsetroundjoin%
\definecolor{currentfill}{rgb}{0.275191,0.194905,0.496005}%
\pgfsetfillcolor{currentfill}%
\pgfsetfillopacity{0.700000}%
\pgfsetlinewidth{0.000000pt}%
\definecolor{currentstroke}{rgb}{0.000000,0.000000,0.000000}%
\pgfsetstrokecolor{currentstroke}%
\pgfsetdash{}{0pt}%
\pgfpathmoveto{\pgfqpoint{4.721483in}{2.164030in}}%
\pgfpathlineto{\pgfqpoint{4.735576in}{2.167159in}}%
\pgfpathlineto{\pgfqpoint{4.749681in}{2.170387in}}%
\pgfpathlineto{\pgfqpoint{4.763797in}{2.173715in}}%
\pgfpathlineto{\pgfqpoint{4.777925in}{2.177143in}}%
\pgfpathlineto{\pgfqpoint{4.770124in}{2.165129in}}%
\pgfpathlineto{\pgfqpoint{4.762316in}{2.153078in}}%
\pgfpathlineto{\pgfqpoint{4.754504in}{2.140992in}}%
\pgfpathlineto{\pgfqpoint{4.746687in}{2.128873in}}%
\pgfpathlineto{\pgfqpoint{4.732557in}{2.125678in}}%
\pgfpathlineto{\pgfqpoint{4.718438in}{2.122582in}}%
\pgfpathlineto{\pgfqpoint{4.704330in}{2.119586in}}%
\pgfpathlineto{\pgfqpoint{4.690234in}{2.116690in}}%
\pgfpathlineto{\pgfqpoint{4.698054in}{2.128570in}}%
\pgfpathlineto{\pgfqpoint{4.705868in}{2.140421in}}%
\pgfpathlineto{\pgfqpoint{4.713678in}{2.152242in}}%
\pgfpathlineto{\pgfqpoint{4.721483in}{2.164030in}}%
\pgfpathclose%
\pgfusepath{fill}%
\end{pgfscope}%
\begin{pgfscope}%
\pgfpathrectangle{\pgfqpoint{1.150000in}{0.150000in}}{\pgfqpoint{5.700000in}{5.700000in}}%
\pgfusepath{clip}%
\pgfsetbuttcap%
\pgfsetroundjoin%
\definecolor{currentfill}{rgb}{0.239346,0.300855,0.540844}%
\pgfsetfillcolor{currentfill}%
\pgfsetfillopacity{0.700000}%
\pgfsetlinewidth{0.000000pt}%
\definecolor{currentstroke}{rgb}{0.000000,0.000000,0.000000}%
\pgfsetstrokecolor{currentstroke}%
\pgfsetdash{}{0pt}%
\pgfpathmoveto{\pgfqpoint{2.721939in}{2.408096in}}%
\pgfpathlineto{\pgfqpoint{2.735764in}{2.393256in}}%
\pgfpathlineto{\pgfqpoint{2.749585in}{2.378570in}}%
\pgfpathlineto{\pgfqpoint{2.763404in}{2.364038in}}%
\pgfpathlineto{\pgfqpoint{2.777219in}{2.349657in}}%
\pgfpathlineto{\pgfqpoint{2.768433in}{2.353580in}}%
\pgfpathlineto{\pgfqpoint{2.759629in}{2.357793in}}%
\pgfpathlineto{\pgfqpoint{2.750808in}{2.362301in}}%
\pgfpathlineto{\pgfqpoint{2.741968in}{2.367109in}}%
\pgfpathlineto{\pgfqpoint{2.728108in}{2.382000in}}%
\pgfpathlineto{\pgfqpoint{2.714244in}{2.397043in}}%
\pgfpathlineto{\pgfqpoint{2.700377in}{2.412240in}}%
\pgfpathlineto{\pgfqpoint{2.686507in}{2.427592in}}%
\pgfpathlineto{\pgfqpoint{2.695393in}{2.422264in}}%
\pgfpathlineto{\pgfqpoint{2.704260in}{2.417243in}}%
\pgfpathlineto{\pgfqpoint{2.713109in}{2.412522in}}%
\pgfpathlineto{\pgfqpoint{2.721939in}{2.408096in}}%
\pgfpathclose%
\pgfusepath{fill}%
\end{pgfscope}%
\begin{pgfscope}%
\pgfpathrectangle{\pgfqpoint{1.150000in}{0.150000in}}{\pgfqpoint{5.700000in}{5.700000in}}%
\pgfusepath{clip}%
\pgfsetbuttcap%
\pgfsetroundjoin%
\definecolor{currentfill}{rgb}{0.190631,0.407061,0.556089}%
\pgfsetfillcolor{currentfill}%
\pgfsetfillopacity{0.700000}%
\pgfsetlinewidth{0.000000pt}%
\definecolor{currentstroke}{rgb}{0.000000,0.000000,0.000000}%
\pgfsetstrokecolor{currentstroke}%
\pgfsetdash{}{0pt}%
\pgfpathmoveto{\pgfqpoint{2.500169in}{2.667552in}}%
\pgfpathlineto{\pgfqpoint{2.514068in}{2.650068in}}%
\pgfpathlineto{\pgfqpoint{2.527961in}{2.632759in}}%
\pgfpathlineto{\pgfqpoint{2.541848in}{2.615625in}}%
\pgfpathlineto{\pgfqpoint{2.555730in}{2.598663in}}%
\pgfpathlineto{\pgfqpoint{2.546758in}{2.604614in}}%
\pgfpathlineto{\pgfqpoint{2.537765in}{2.610876in}}%
\pgfpathlineto{\pgfqpoint{2.528752in}{2.617456in}}%
\pgfpathlineto{\pgfqpoint{2.519718in}{2.624359in}}%
\pgfpathlineto{\pgfqpoint{2.505785in}{2.641843in}}%
\pgfpathlineto{\pgfqpoint{2.491847in}{2.659500in}}%
\pgfpathlineto{\pgfqpoint{2.477902in}{2.677332in}}%
\pgfpathlineto{\pgfqpoint{2.463951in}{2.695341in}}%
\pgfpathlineto{\pgfqpoint{2.473038in}{2.687906in}}%
\pgfpathlineto{\pgfqpoint{2.482103in}{2.680800in}}%
\pgfpathlineto{\pgfqpoint{2.491146in}{2.674017in}}%
\pgfpathlineto{\pgfqpoint{2.500169in}{2.667552in}}%
\pgfpathclose%
\pgfusepath{fill}%
\end{pgfscope}%
\begin{pgfscope}%
\pgfpathrectangle{\pgfqpoint{1.150000in}{0.150000in}}{\pgfqpoint{5.700000in}{5.700000in}}%
\pgfusepath{clip}%
\pgfsetbuttcap%
\pgfsetroundjoin%
\definecolor{currentfill}{rgb}{0.144759,0.519093,0.556572}%
\pgfsetfillcolor{currentfill}%
\pgfsetfillopacity{0.700000}%
\pgfsetlinewidth{0.000000pt}%
\definecolor{currentstroke}{rgb}{0.000000,0.000000,0.000000}%
\pgfsetstrokecolor{currentstroke}%
\pgfsetdash{}{0pt}%
\pgfpathmoveto{\pgfqpoint{5.691166in}{2.953637in}}%
\pgfpathlineto{\pgfqpoint{5.705764in}{2.962090in}}%
\pgfpathlineto{\pgfqpoint{5.720378in}{2.970644in}}%
\pgfpathlineto{\pgfqpoint{5.735009in}{2.979300in}}%
\pgfpathlineto{\pgfqpoint{5.749656in}{2.988056in}}%
\pgfpathlineto{\pgfqpoint{5.742254in}{2.979811in}}%
\pgfpathlineto{\pgfqpoint{5.734844in}{2.971441in}}%
\pgfpathlineto{\pgfqpoint{5.727424in}{2.962945in}}%
\pgfpathlineto{\pgfqpoint{5.719995in}{2.954322in}}%
\pgfpathlineto{\pgfqpoint{5.705340in}{2.945552in}}%
\pgfpathlineto{\pgfqpoint{5.690702in}{2.936883in}}%
\pgfpathlineto{\pgfqpoint{5.676080in}{2.928314in}}%
\pgfpathlineto{\pgfqpoint{5.661474in}{2.919847in}}%
\pgfpathlineto{\pgfqpoint{5.668910in}{2.928477in}}%
\pgfpathlineto{\pgfqpoint{5.676338in}{2.936984in}}%
\pgfpathlineto{\pgfqpoint{5.683756in}{2.945371in}}%
\pgfpathlineto{\pgfqpoint{5.691166in}{2.953637in}}%
\pgfpathclose%
\pgfusepath{fill}%
\end{pgfscope}%
\begin{pgfscope}%
\pgfpathrectangle{\pgfqpoint{1.150000in}{0.150000in}}{\pgfqpoint{5.700000in}{5.700000in}}%
\pgfusepath{clip}%
\pgfsetbuttcap%
\pgfsetroundjoin%
\definecolor{currentfill}{rgb}{0.248629,0.278775,0.534556}%
\pgfsetfillcolor{currentfill}%
\pgfsetfillopacity{0.700000}%
\pgfsetlinewidth{0.000000pt}%
\definecolor{currentstroke}{rgb}{0.000000,0.000000,0.000000}%
\pgfsetstrokecolor{currentstroke}%
\pgfsetdash{}{0pt}%
\pgfpathmoveto{\pgfqpoint{2.777219in}{2.349657in}}%
\pgfpathlineto{\pgfqpoint{2.791031in}{2.335427in}}%
\pgfpathlineto{\pgfqpoint{2.804841in}{2.321347in}}%
\pgfpathlineto{\pgfqpoint{2.818648in}{2.307415in}}%
\pgfpathlineto{\pgfqpoint{2.832453in}{2.293632in}}%
\pgfpathlineto{\pgfqpoint{2.823710in}{2.297055in}}%
\pgfpathlineto{\pgfqpoint{2.814950in}{2.300762in}}%
\pgfpathlineto{\pgfqpoint{2.806172in}{2.304759in}}%
\pgfpathlineto{\pgfqpoint{2.797377in}{2.309051in}}%
\pgfpathlineto{\pgfqpoint{2.783529in}{2.323342in}}%
\pgfpathlineto{\pgfqpoint{2.769679in}{2.337782in}}%
\pgfpathlineto{\pgfqpoint{2.755825in}{2.352370in}}%
\pgfpathlineto{\pgfqpoint{2.741968in}{2.367109in}}%
\pgfpathlineto{\pgfqpoint{2.750808in}{2.362301in}}%
\pgfpathlineto{\pgfqpoint{2.759629in}{2.357793in}}%
\pgfpathlineto{\pgfqpoint{2.768433in}{2.353580in}}%
\pgfpathlineto{\pgfqpoint{2.777219in}{2.349657in}}%
\pgfpathclose%
\pgfusepath{fill}%
\end{pgfscope}%
\begin{pgfscope}%
\pgfpathrectangle{\pgfqpoint{1.150000in}{0.150000in}}{\pgfqpoint{5.700000in}{5.700000in}}%
\pgfusepath{clip}%
\pgfsetbuttcap%
\pgfsetroundjoin%
\definecolor{currentfill}{rgb}{0.225863,0.330805,0.547314}%
\pgfsetfillcolor{currentfill}%
\pgfsetfillopacity{0.700000}%
\pgfsetlinewidth{0.000000pt}%
\definecolor{currentstroke}{rgb}{0.000000,0.000000,0.000000}%
\pgfsetstrokecolor{currentstroke}%
\pgfsetdash{}{0pt}%
\pgfpathmoveto{\pgfqpoint{5.103176in}{2.466405in}}%
\pgfpathlineto{\pgfqpoint{5.117456in}{2.472016in}}%
\pgfpathlineto{\pgfqpoint{5.131750in}{2.477727in}}%
\pgfpathlineto{\pgfqpoint{5.146058in}{2.483538in}}%
\pgfpathlineto{\pgfqpoint{5.160379in}{2.489448in}}%
\pgfpathlineto{\pgfqpoint{5.152700in}{2.478004in}}%
\pgfpathlineto{\pgfqpoint{5.145014in}{2.466472in}}%
\pgfpathlineto{\pgfqpoint{5.137322in}{2.454854in}}%
\pgfpathlineto{\pgfqpoint{5.129623in}{2.443152in}}%
\pgfpathlineto{\pgfqpoint{5.115300in}{2.437382in}}%
\pgfpathlineto{\pgfqpoint{5.100990in}{2.431712in}}%
\pgfpathlineto{\pgfqpoint{5.086693in}{2.426142in}}%
\pgfpathlineto{\pgfqpoint{5.072410in}{2.420671in}}%
\pgfpathlineto{\pgfqpoint{5.080111in}{2.432226in}}%
\pgfpathlineto{\pgfqpoint{5.087806in}{2.443701in}}%
\pgfpathlineto{\pgfqpoint{5.095494in}{2.455094in}}%
\pgfpathlineto{\pgfqpoint{5.103176in}{2.466405in}}%
\pgfpathclose%
\pgfusepath{fill}%
\end{pgfscope}%
\begin{pgfscope}%
\pgfpathrectangle{\pgfqpoint{1.150000in}{0.150000in}}{\pgfqpoint{5.700000in}{5.700000in}}%
\pgfusepath{clip}%
\pgfsetbuttcap%
\pgfsetroundjoin%
\definecolor{currentfill}{rgb}{0.179019,0.433756,0.557430}%
\pgfsetfillcolor{currentfill}%
\pgfsetfillopacity{0.700000}%
\pgfsetlinewidth{0.000000pt}%
\definecolor{currentstroke}{rgb}{0.000000,0.000000,0.000000}%
\pgfsetstrokecolor{currentstroke}%
\pgfsetdash{}{0pt}%
\pgfpathmoveto{\pgfqpoint{2.444510in}{2.739280in}}%
\pgfpathlineto{\pgfqpoint{2.458435in}{2.721076in}}%
\pgfpathlineto{\pgfqpoint{2.472352in}{2.703055in}}%
\pgfpathlineto{\pgfqpoint{2.486264in}{2.685214in}}%
\pgfpathlineto{\pgfqpoint{2.500169in}{2.667552in}}%
\pgfpathlineto{\pgfqpoint{2.491146in}{2.674017in}}%
\pgfpathlineto{\pgfqpoint{2.482103in}{2.680800in}}%
\pgfpathlineto{\pgfqpoint{2.473038in}{2.687906in}}%
\pgfpathlineto{\pgfqpoint{2.463951in}{2.695341in}}%
\pgfpathlineto{\pgfqpoint{2.449994in}{2.713528in}}%
\pgfpathlineto{\pgfqpoint{2.436030in}{2.731896in}}%
\pgfpathlineto{\pgfqpoint{2.422059in}{2.750445in}}%
\pgfpathlineto{\pgfqpoint{2.408081in}{2.769177in}}%
\pgfpathlineto{\pgfqpoint{2.417222in}{2.761206in}}%
\pgfpathlineto{\pgfqpoint{2.426340in}{2.753570in}}%
\pgfpathlineto{\pgfqpoint{2.435436in}{2.746263in}}%
\pgfpathlineto{\pgfqpoint{2.444510in}{2.739280in}}%
\pgfpathclose%
\pgfusepath{fill}%
\end{pgfscope}%
\begin{pgfscope}%
\pgfpathrectangle{\pgfqpoint{1.150000in}{0.150000in}}{\pgfqpoint{5.700000in}{5.700000in}}%
\pgfusepath{clip}%
\pgfsetbuttcap%
\pgfsetroundjoin%
\definecolor{currentfill}{rgb}{0.180629,0.429975,0.557282}%
\pgfsetfillcolor{currentfill}%
\pgfsetfillopacity{0.700000}%
\pgfsetlinewidth{0.000000pt}%
\definecolor{currentstroke}{rgb}{0.000000,0.000000,0.000000}%
\pgfsetstrokecolor{currentstroke}%
\pgfsetdash{}{0pt}%
\pgfpathmoveto{\pgfqpoint{5.397309in}{2.713874in}}%
\pgfpathlineto{\pgfqpoint{5.411746in}{2.721058in}}%
\pgfpathlineto{\pgfqpoint{5.426198in}{2.728343in}}%
\pgfpathlineto{\pgfqpoint{5.440665in}{2.735728in}}%
\pgfpathlineto{\pgfqpoint{5.455147in}{2.743213in}}%
\pgfpathlineto{\pgfqpoint{5.447591in}{2.733101in}}%
\pgfpathlineto{\pgfqpoint{5.440027in}{2.722876in}}%
\pgfpathlineto{\pgfqpoint{5.432455in}{2.712538in}}%
\pgfpathlineto{\pgfqpoint{5.424876in}{2.702089in}}%
\pgfpathlineto{\pgfqpoint{5.410389in}{2.694668in}}%
\pgfpathlineto{\pgfqpoint{5.395918in}{2.687348in}}%
\pgfpathlineto{\pgfqpoint{5.381462in}{2.680127in}}%
\pgfpathlineto{\pgfqpoint{5.367020in}{2.673008in}}%
\pgfpathlineto{\pgfqpoint{5.374604in}{2.683385in}}%
\pgfpathlineto{\pgfqpoint{5.382180in}{2.693655in}}%
\pgfpathlineto{\pgfqpoint{5.389748in}{2.703818in}}%
\pgfpathlineto{\pgfqpoint{5.397309in}{2.713874in}}%
\pgfpathclose%
\pgfusepath{fill}%
\end{pgfscope}%
\begin{pgfscope}%
\pgfpathrectangle{\pgfqpoint{1.150000in}{0.150000in}}{\pgfqpoint{5.700000in}{5.700000in}}%
\pgfusepath{clip}%
\pgfsetbuttcap%
\pgfsetroundjoin%
\definecolor{currentfill}{rgb}{0.277018,0.050344,0.375715}%
\pgfsetfillcolor{currentfill}%
\pgfsetfillopacity{0.700000}%
\pgfsetlinewidth{0.000000pt}%
\definecolor{currentstroke}{rgb}{0.000000,0.000000,0.000000}%
\pgfsetstrokecolor{currentstroke}%
\pgfsetdash{}{0pt}%
\pgfpathmoveto{\pgfqpoint{3.450802in}{1.872478in}}%
\pgfpathlineto{\pgfqpoint{3.464559in}{1.864931in}}%
\pgfpathlineto{\pgfqpoint{3.478320in}{1.857499in}}%
\pgfpathlineto{\pgfqpoint{3.492084in}{1.850182in}}%
\pgfpathlineto{\pgfqpoint{3.505851in}{1.842979in}}%
\pgfpathlineto{\pgfqpoint{3.497577in}{1.839380in}}%
\pgfpathlineto{\pgfqpoint{3.489294in}{1.835970in}}%
\pgfpathlineto{\pgfqpoint{3.481000in}{1.832754in}}%
\pgfpathlineto{\pgfqpoint{3.472696in}{1.829736in}}%
\pgfpathlineto{\pgfqpoint{3.458903in}{1.837389in}}%
\pgfpathlineto{\pgfqpoint{3.445114in}{1.845156in}}%
\pgfpathlineto{\pgfqpoint{3.431327in}{1.853038in}}%
\pgfpathlineto{\pgfqpoint{3.417543in}{1.861036in}}%
\pgfpathlineto{\pgfqpoint{3.425873in}{1.863596in}}%
\pgfpathlineto{\pgfqpoint{3.434193in}{1.866359in}}%
\pgfpathlineto{\pgfqpoint{3.442502in}{1.869321in}}%
\pgfpathlineto{\pgfqpoint{3.450802in}{1.872478in}}%
\pgfpathclose%
\pgfusepath{fill}%
\end{pgfscope}%
\begin{pgfscope}%
\pgfpathrectangle{\pgfqpoint{1.150000in}{0.150000in}}{\pgfqpoint{5.700000in}{5.700000in}}%
\pgfusepath{clip}%
\pgfsetbuttcap%
\pgfsetroundjoin%
\definecolor{currentfill}{rgb}{0.257322,0.256130,0.526563}%
\pgfsetfillcolor{currentfill}%
\pgfsetfillopacity{0.700000}%
\pgfsetlinewidth{0.000000pt}%
\definecolor{currentstroke}{rgb}{0.000000,0.000000,0.000000}%
\pgfsetstrokecolor{currentstroke}%
\pgfsetdash{}{0pt}%
\pgfpathmoveto{\pgfqpoint{2.832453in}{2.293632in}}%
\pgfpathlineto{\pgfqpoint{2.846255in}{2.279994in}}%
\pgfpathlineto{\pgfqpoint{2.860055in}{2.266502in}}%
\pgfpathlineto{\pgfqpoint{2.873854in}{2.253155in}}%
\pgfpathlineto{\pgfqpoint{2.887650in}{2.239951in}}%
\pgfpathlineto{\pgfqpoint{2.878949in}{2.242876in}}%
\pgfpathlineto{\pgfqpoint{2.870231in}{2.246081in}}%
\pgfpathlineto{\pgfqpoint{2.861496in}{2.249569in}}%
\pgfpathlineto{\pgfqpoint{2.852745in}{2.253347in}}%
\pgfpathlineto{\pgfqpoint{2.838906in}{2.267056in}}%
\pgfpathlineto{\pgfqpoint{2.825066in}{2.280909in}}%
\pgfpathlineto{\pgfqpoint{2.811223in}{2.294907in}}%
\pgfpathlineto{\pgfqpoint{2.797377in}{2.309051in}}%
\pgfpathlineto{\pgfqpoint{2.806172in}{2.304759in}}%
\pgfpathlineto{\pgfqpoint{2.814950in}{2.300762in}}%
\pgfpathlineto{\pgfqpoint{2.823710in}{2.297055in}}%
\pgfpathlineto{\pgfqpoint{2.832453in}{2.293632in}}%
\pgfpathclose%
\pgfusepath{fill}%
\end{pgfscope}%
\begin{pgfscope}%
\pgfpathrectangle{\pgfqpoint{1.150000in}{0.150000in}}{\pgfqpoint{5.700000in}{5.700000in}}%
\pgfusepath{clip}%
\pgfsetbuttcap%
\pgfsetroundjoin%
\definecolor{currentfill}{rgb}{0.267968,0.223549,0.512008}%
\pgfsetfillcolor{currentfill}%
\pgfsetfillopacity{0.700000}%
\pgfsetlinewidth{0.000000pt}%
\definecolor{currentstroke}{rgb}{0.000000,0.000000,0.000000}%
\pgfsetstrokecolor{currentstroke}%
\pgfsetdash{}{0pt}%
\pgfpathmoveto{\pgfqpoint{4.809082in}{2.224786in}}%
\pgfpathlineto{\pgfqpoint{4.823219in}{2.228527in}}%
\pgfpathlineto{\pgfqpoint{4.837368in}{2.232368in}}%
\pgfpathlineto{\pgfqpoint{4.851530in}{2.236309in}}%
\pgfpathlineto{\pgfqpoint{4.865704in}{2.240349in}}%
\pgfpathlineto{\pgfqpoint{4.857925in}{2.228297in}}%
\pgfpathlineto{\pgfqpoint{4.850141in}{2.216194in}}%
\pgfpathlineto{\pgfqpoint{4.842351in}{2.204044in}}%
\pgfpathlineto{\pgfqpoint{4.834557in}{2.191848in}}%
\pgfpathlineto{\pgfqpoint{4.820381in}{2.188023in}}%
\pgfpathlineto{\pgfqpoint{4.806217in}{2.184297in}}%
\pgfpathlineto{\pgfqpoint{4.792065in}{2.180670in}}%
\pgfpathlineto{\pgfqpoint{4.777925in}{2.177143in}}%
\pgfpathlineto{\pgfqpoint{4.785722in}{2.189117in}}%
\pgfpathlineto{\pgfqpoint{4.793514in}{2.201051in}}%
\pgfpathlineto{\pgfqpoint{4.801301in}{2.212941in}}%
\pgfpathlineto{\pgfqpoint{4.809082in}{2.224786in}}%
\pgfpathclose%
\pgfusepath{fill}%
\end{pgfscope}%
\begin{pgfscope}%
\pgfpathrectangle{\pgfqpoint{1.150000in}{0.150000in}}{\pgfqpoint{5.700000in}{5.700000in}}%
\pgfusepath{clip}%
\pgfsetbuttcap%
\pgfsetroundjoin%
\definecolor{currentfill}{rgb}{0.269944,0.014625,0.341379}%
\pgfsetfillcolor{currentfill}%
\pgfsetfillopacity{0.700000}%
\pgfsetlinewidth{0.000000pt}%
\definecolor{currentstroke}{rgb}{0.000000,0.000000,0.000000}%
\pgfsetstrokecolor{currentstroke}%
\pgfsetdash{}{0pt}%
\pgfpathmoveto{\pgfqpoint{4.022223in}{1.810849in}}%
\pgfpathlineto{\pgfqpoint{4.036071in}{1.808390in}}%
\pgfpathlineto{\pgfqpoint{4.049926in}{1.806035in}}%
\pgfpathlineto{\pgfqpoint{4.063789in}{1.803783in}}%
\pgfpathlineto{\pgfqpoint{4.077659in}{1.801634in}}%
\pgfpathlineto{\pgfqpoint{4.069646in}{1.792636in}}%
\pgfpathlineto{\pgfqpoint{4.061628in}{1.783725in}}%
\pgfpathlineto{\pgfqpoint{4.053604in}{1.774907in}}%
\pgfpathlineto{\pgfqpoint{4.045574in}{1.766184in}}%
\pgfpathlineto{\pgfqpoint{4.031692in}{1.768707in}}%
\pgfpathlineto{\pgfqpoint{4.017817in}{1.771333in}}%
\pgfpathlineto{\pgfqpoint{4.003949in}{1.774062in}}%
\pgfpathlineto{\pgfqpoint{3.990089in}{1.776894in}}%
\pgfpathlineto{\pgfqpoint{3.998131in}{1.785237in}}%
\pgfpathlineto{\pgfqpoint{4.006168in}{1.793679in}}%
\pgfpathlineto{\pgfqpoint{4.014198in}{1.802217in}}%
\pgfpathlineto{\pgfqpoint{4.022223in}{1.810849in}}%
\pgfpathclose%
\pgfusepath{fill}%
\end{pgfscope}%
\begin{pgfscope}%
\pgfpathrectangle{\pgfqpoint{1.150000in}{0.150000in}}{\pgfqpoint{5.700000in}{5.700000in}}%
\pgfusepath{clip}%
\pgfsetbuttcap%
\pgfsetroundjoin%
\definecolor{currentfill}{rgb}{0.168126,0.459988,0.558082}%
\pgfsetfillcolor{currentfill}%
\pgfsetfillopacity{0.700000}%
\pgfsetlinewidth{0.000000pt}%
\definecolor{currentstroke}{rgb}{0.000000,0.000000,0.000000}%
\pgfsetstrokecolor{currentstroke}%
\pgfsetdash{}{0pt}%
\pgfpathmoveto{\pgfqpoint{2.388743in}{2.813951in}}%
\pgfpathlineto{\pgfqpoint{2.402696in}{2.795001in}}%
\pgfpathlineto{\pgfqpoint{2.416641in}{2.776241in}}%
\pgfpathlineto{\pgfqpoint{2.430579in}{2.757668in}}%
\pgfpathlineto{\pgfqpoint{2.444510in}{2.739280in}}%
\pgfpathlineto{\pgfqpoint{2.435436in}{2.746263in}}%
\pgfpathlineto{\pgfqpoint{2.426340in}{2.753570in}}%
\pgfpathlineto{\pgfqpoint{2.417222in}{2.761206in}}%
\pgfpathlineto{\pgfqpoint{2.408081in}{2.769177in}}%
\pgfpathlineto{\pgfqpoint{2.394096in}{2.788094in}}%
\pgfpathlineto{\pgfqpoint{2.380104in}{2.807197in}}%
\pgfpathlineto{\pgfqpoint{2.366104in}{2.826489in}}%
\pgfpathlineto{\pgfqpoint{2.352096in}{2.845971in}}%
\pgfpathlineto{\pgfqpoint{2.361292in}{2.837460in}}%
\pgfpathlineto{\pgfqpoint{2.370465in}{2.829290in}}%
\pgfpathlineto{\pgfqpoint{2.379615in}{2.821456in}}%
\pgfpathlineto{\pgfqpoint{2.388743in}{2.813951in}}%
\pgfpathclose%
\pgfusepath{fill}%
\end{pgfscope}%
\begin{pgfscope}%
\pgfpathrectangle{\pgfqpoint{1.150000in}{0.150000in}}{\pgfqpoint{5.700000in}{5.700000in}}%
\pgfusepath{clip}%
\pgfsetbuttcap%
\pgfsetroundjoin%
\definecolor{currentfill}{rgb}{0.282656,0.100196,0.422160}%
\pgfsetfillcolor{currentfill}%
\pgfsetfillopacity{0.700000}%
\pgfsetlinewidth{0.000000pt}%
\definecolor{currentstroke}{rgb}{0.000000,0.000000,0.000000}%
\pgfsetstrokecolor{currentstroke}%
\pgfsetdash{}{0pt}%
\pgfpathmoveto{\pgfqpoint{3.252318in}{1.966208in}}%
\pgfpathlineto{\pgfqpoint{3.266076in}{1.956782in}}%
\pgfpathlineto{\pgfqpoint{3.279836in}{1.947479in}}%
\pgfpathlineto{\pgfqpoint{3.293597in}{1.938297in}}%
\pgfpathlineto{\pgfqpoint{3.307361in}{1.929236in}}%
\pgfpathlineto{\pgfqpoint{3.298962in}{1.927812in}}%
\pgfpathlineto{\pgfqpoint{3.290552in}{1.926610in}}%
\pgfpathlineto{\pgfqpoint{3.282130in}{1.925636in}}%
\pgfpathlineto{\pgfqpoint{3.273695in}{1.924893in}}%
\pgfpathlineto{\pgfqpoint{3.259901in}{1.934427in}}%
\pgfpathlineto{\pgfqpoint{3.246109in}{1.944081in}}%
\pgfpathlineto{\pgfqpoint{3.232317in}{1.953858in}}%
\pgfpathlineto{\pgfqpoint{3.218528in}{1.963757in}}%
\pgfpathlineto{\pgfqpoint{3.226994in}{1.964018in}}%
\pgfpathlineto{\pgfqpoint{3.235448in}{1.964517in}}%
\pgfpathlineto{\pgfqpoint{3.243889in}{1.965249in}}%
\pgfpathlineto{\pgfqpoint{3.252318in}{1.966208in}}%
\pgfpathclose%
\pgfusepath{fill}%
\end{pgfscope}%
\begin{pgfscope}%
\pgfpathrectangle{\pgfqpoint{1.150000in}{0.150000in}}{\pgfqpoint{5.700000in}{5.700000in}}%
\pgfusepath{clip}%
\pgfsetbuttcap%
\pgfsetroundjoin%
\definecolor{currentfill}{rgb}{0.265145,0.232956,0.516599}%
\pgfsetfillcolor{currentfill}%
\pgfsetfillopacity{0.700000}%
\pgfsetlinewidth{0.000000pt}%
\definecolor{currentstroke}{rgb}{0.000000,0.000000,0.000000}%
\pgfsetstrokecolor{currentstroke}%
\pgfsetdash{}{0pt}%
\pgfpathmoveto{\pgfqpoint{2.887650in}{2.239951in}}%
\pgfpathlineto{\pgfqpoint{2.901444in}{2.226889in}}%
\pgfpathlineto{\pgfqpoint{2.915237in}{2.213969in}}%
\pgfpathlineto{\pgfqpoint{2.929029in}{2.201190in}}%
\pgfpathlineto{\pgfqpoint{2.942819in}{2.188550in}}%
\pgfpathlineto{\pgfqpoint{2.934158in}{2.190980in}}%
\pgfpathlineto{\pgfqpoint{2.925481in}{2.193684in}}%
\pgfpathlineto{\pgfqpoint{2.916788in}{2.196666in}}%
\pgfpathlineto{\pgfqpoint{2.908078in}{2.199932in}}%
\pgfpathlineto{\pgfqpoint{2.894248in}{2.213075in}}%
\pgfpathlineto{\pgfqpoint{2.880415in}{2.226358in}}%
\pgfpathlineto{\pgfqpoint{2.866581in}{2.239781in}}%
\pgfpathlineto{\pgfqpoint{2.852745in}{2.253347in}}%
\pgfpathlineto{\pgfqpoint{2.861496in}{2.249569in}}%
\pgfpathlineto{\pgfqpoint{2.870231in}{2.246081in}}%
\pgfpathlineto{\pgfqpoint{2.878949in}{2.242876in}}%
\pgfpathlineto{\pgfqpoint{2.887650in}{2.239951in}}%
\pgfpathclose%
\pgfusepath{fill}%
\end{pgfscope}%
\begin{pgfscope}%
\pgfpathrectangle{\pgfqpoint{1.150000in}{0.150000in}}{\pgfqpoint{5.700000in}{5.700000in}}%
\pgfusepath{clip}%
\pgfsetbuttcap%
\pgfsetroundjoin%
\definecolor{currentfill}{rgb}{0.135066,0.544853,0.554029}%
\pgfsetfillcolor{currentfill}%
\pgfsetfillopacity{0.700000}%
\pgfsetlinewidth{0.000000pt}%
\definecolor{currentstroke}{rgb}{0.000000,0.000000,0.000000}%
\pgfsetstrokecolor{currentstroke}%
\pgfsetdash{}{0pt}%
\pgfpathmoveto{\pgfqpoint{5.779169in}{3.019788in}}%
\pgfpathlineto{\pgfqpoint{5.793824in}{3.028611in}}%
\pgfpathlineto{\pgfqpoint{5.808495in}{3.037535in}}%
\pgfpathlineto{\pgfqpoint{5.823183in}{3.046560in}}%
\pgfpathlineto{\pgfqpoint{5.837887in}{3.055686in}}%
\pgfpathlineto{\pgfqpoint{5.830532in}{3.047980in}}%
\pgfpathlineto{\pgfqpoint{5.823168in}{3.040146in}}%
\pgfpathlineto{\pgfqpoint{5.815794in}{3.032183in}}%
\pgfpathlineto{\pgfqpoint{5.808410in}{3.024091in}}%
\pgfpathlineto{\pgfqpoint{5.793697in}{3.014930in}}%
\pgfpathlineto{\pgfqpoint{5.779000in}{3.005871in}}%
\pgfpathlineto{\pgfqpoint{5.764319in}{2.996913in}}%
\pgfpathlineto{\pgfqpoint{5.749656in}{2.988056in}}%
\pgfpathlineto{\pgfqpoint{5.757048in}{2.996175in}}%
\pgfpathlineto{\pgfqpoint{5.764431in}{3.004169in}}%
\pgfpathlineto{\pgfqpoint{5.771805in}{3.012040in}}%
\pgfpathlineto{\pgfqpoint{5.779169in}{3.019788in}}%
\pgfpathclose%
\pgfusepath{fill}%
\end{pgfscope}%
\begin{pgfscope}%
\pgfpathrectangle{\pgfqpoint{1.150000in}{0.150000in}}{\pgfqpoint{5.700000in}{5.700000in}}%
\pgfusepath{clip}%
\pgfsetbuttcap%
\pgfsetroundjoin%
\definecolor{currentfill}{rgb}{0.267004,0.004874,0.329415}%
\pgfsetfillcolor{currentfill}%
\pgfsetfillopacity{0.700000}%
\pgfsetlinewidth{0.000000pt}%
\definecolor{currentstroke}{rgb}{0.000000,0.000000,0.000000}%
\pgfsetstrokecolor{currentstroke}%
\pgfsetdash{}{0pt}%
\pgfpathmoveto{\pgfqpoint{3.791849in}{1.791458in}}%
\pgfpathlineto{\pgfqpoint{3.805650in}{1.786971in}}%
\pgfpathlineto{\pgfqpoint{3.819458in}{1.782591in}}%
\pgfpathlineto{\pgfqpoint{3.833270in}{1.778318in}}%
\pgfpathlineto{\pgfqpoint{3.847089in}{1.774151in}}%
\pgfpathlineto{\pgfqpoint{3.838984in}{1.767172in}}%
\pgfpathlineto{\pgfqpoint{3.830872in}{1.760325in}}%
\pgfpathlineto{\pgfqpoint{3.822752in}{1.753614in}}%
\pgfpathlineto{\pgfqpoint{3.814625in}{1.747043in}}%
\pgfpathlineto{\pgfqpoint{3.800790in}{1.751621in}}%
\pgfpathlineto{\pgfqpoint{3.786960in}{1.756305in}}%
\pgfpathlineto{\pgfqpoint{3.773135in}{1.761096in}}%
\pgfpathlineto{\pgfqpoint{3.759316in}{1.765993in}}%
\pgfpathlineto{\pgfqpoint{3.767460in}{1.772146in}}%
\pgfpathlineto{\pgfqpoint{3.775597in}{1.778444in}}%
\pgfpathlineto{\pgfqpoint{3.783727in}{1.784883in}}%
\pgfpathlineto{\pgfqpoint{3.791849in}{1.791458in}}%
\pgfpathclose%
\pgfusepath{fill}%
\end{pgfscope}%
\begin{pgfscope}%
\pgfpathrectangle{\pgfqpoint{1.150000in}{0.150000in}}{\pgfqpoint{5.700000in}{5.700000in}}%
\pgfusepath{clip}%
\pgfsetbuttcap%
\pgfsetroundjoin%
\definecolor{currentfill}{rgb}{0.269944,0.014625,0.341379}%
\pgfsetfillcolor{currentfill}%
\pgfsetfillopacity{0.700000}%
\pgfsetlinewidth{0.000000pt}%
\definecolor{currentstroke}{rgb}{0.000000,0.000000,0.000000}%
\pgfsetstrokecolor{currentstroke}%
\pgfsetdash{}{0pt}%
\pgfpathmoveto{\pgfqpoint{3.648941in}{1.809061in}}%
\pgfpathlineto{\pgfqpoint{3.662721in}{1.803296in}}%
\pgfpathlineto{\pgfqpoint{3.676506in}{1.797641in}}%
\pgfpathlineto{\pgfqpoint{3.690295in}{1.792095in}}%
\pgfpathlineto{\pgfqpoint{3.704089in}{1.786658in}}%
\pgfpathlineto{\pgfqpoint{3.695918in}{1.781076in}}%
\pgfpathlineto{\pgfqpoint{3.687738in}{1.775652in}}%
\pgfpathlineto{\pgfqpoint{3.679550in}{1.770389in}}%
\pgfpathlineto{\pgfqpoint{3.671353in}{1.765293in}}%
\pgfpathlineto{\pgfqpoint{3.657539in}{1.771160in}}%
\pgfpathlineto{\pgfqpoint{3.643728in}{1.777135in}}%
\pgfpathlineto{\pgfqpoint{3.629923in}{1.783220in}}%
\pgfpathlineto{\pgfqpoint{3.616121in}{1.789415in}}%
\pgfpathlineto{\pgfqpoint{3.624339in}{1.794074in}}%
\pgfpathlineto{\pgfqpoint{3.632548in}{1.798905in}}%
\pgfpathlineto{\pgfqpoint{3.640749in}{1.803902in}}%
\pgfpathlineto{\pgfqpoint{3.648941in}{1.809061in}}%
\pgfpathclose%
\pgfusepath{fill}%
\end{pgfscope}%
\begin{pgfscope}%
\pgfpathrectangle{\pgfqpoint{1.150000in}{0.150000in}}{\pgfqpoint{5.700000in}{5.700000in}}%
\pgfusepath{clip}%
\pgfsetbuttcap%
\pgfsetroundjoin%
\definecolor{currentfill}{rgb}{0.210503,0.363727,0.552206}%
\pgfsetfillcolor{currentfill}%
\pgfsetfillopacity{0.700000}%
\pgfsetlinewidth{0.000000pt}%
\definecolor{currentstroke}{rgb}{0.000000,0.000000,0.000000}%
\pgfsetstrokecolor{currentstroke}%
\pgfsetdash{}{0pt}%
\pgfpathmoveto{\pgfqpoint{5.191032in}{2.534335in}}%
\pgfpathlineto{\pgfqpoint{5.205365in}{2.540468in}}%
\pgfpathlineto{\pgfqpoint{5.219711in}{2.546700in}}%
\pgfpathlineto{\pgfqpoint{5.234072in}{2.553032in}}%
\pgfpathlineto{\pgfqpoint{5.248447in}{2.559465in}}%
\pgfpathlineto{\pgfqpoint{5.240797in}{2.548264in}}%
\pgfpathlineto{\pgfqpoint{5.233139in}{2.536967in}}%
\pgfpathlineto{\pgfqpoint{5.225475in}{2.525575in}}%
\pgfpathlineto{\pgfqpoint{5.217805in}{2.514090in}}%
\pgfpathlineto{\pgfqpoint{5.203427in}{2.507780in}}%
\pgfpathlineto{\pgfqpoint{5.189064in}{2.501569in}}%
\pgfpathlineto{\pgfqpoint{5.174715in}{2.495459in}}%
\pgfpathlineto{\pgfqpoint{5.160379in}{2.489448in}}%
\pgfpathlineto{\pgfqpoint{5.168052in}{2.500805in}}%
\pgfpathlineto{\pgfqpoint{5.175719in}{2.512072in}}%
\pgfpathlineto{\pgfqpoint{5.183379in}{2.523249in}}%
\pgfpathlineto{\pgfqpoint{5.191032in}{2.534335in}}%
\pgfpathclose%
\pgfusepath{fill}%
\end{pgfscope}%
\begin{pgfscope}%
\pgfpathrectangle{\pgfqpoint{1.150000in}{0.150000in}}{\pgfqpoint{5.700000in}{5.700000in}}%
\pgfusepath{clip}%
\pgfsetbuttcap%
\pgfsetroundjoin%
\definecolor{currentfill}{rgb}{0.257322,0.256130,0.526563}%
\pgfsetfillcolor{currentfill}%
\pgfsetfillopacity{0.700000}%
\pgfsetlinewidth{0.000000pt}%
\definecolor{currentstroke}{rgb}{0.000000,0.000000,0.000000}%
\pgfsetstrokecolor{currentstroke}%
\pgfsetdash{}{0pt}%
\pgfpathmoveto{\pgfqpoint{4.896765in}{2.288028in}}%
\pgfpathlineto{\pgfqpoint{4.910949in}{2.292364in}}%
\pgfpathlineto{\pgfqpoint{4.925145in}{2.296800in}}%
\pgfpathlineto{\pgfqpoint{4.939354in}{2.301335in}}%
\pgfpathlineto{\pgfqpoint{4.953576in}{2.305970in}}%
\pgfpathlineto{\pgfqpoint{4.945821in}{2.293944in}}%
\pgfpathlineto{\pgfqpoint{4.938060in}{2.281857in}}%
\pgfpathlineto{\pgfqpoint{4.930294in}{2.269710in}}%
\pgfpathlineto{\pgfqpoint{4.922522in}{2.257505in}}%
\pgfpathlineto{\pgfqpoint{4.908299in}{2.253067in}}%
\pgfpathlineto{\pgfqpoint{4.894088in}{2.248728in}}%
\pgfpathlineto{\pgfqpoint{4.879890in}{2.244489in}}%
\pgfpathlineto{\pgfqpoint{4.865704in}{2.240349in}}%
\pgfpathlineto{\pgfqpoint{4.873477in}{2.252351in}}%
\pgfpathlineto{\pgfqpoint{4.881245in}{2.264299in}}%
\pgfpathlineto{\pgfqpoint{4.889008in}{2.276192in}}%
\pgfpathlineto{\pgfqpoint{4.896765in}{2.288028in}}%
\pgfpathclose%
\pgfusepath{fill}%
\end{pgfscope}%
\begin{pgfscope}%
\pgfpathrectangle{\pgfqpoint{1.150000in}{0.150000in}}{\pgfqpoint{5.700000in}{5.700000in}}%
\pgfusepath{clip}%
\pgfsetbuttcap%
\pgfsetroundjoin%
\definecolor{currentfill}{rgb}{0.156270,0.489624,0.557936}%
\pgfsetfillcolor{currentfill}%
\pgfsetfillopacity{0.700000}%
\pgfsetlinewidth{0.000000pt}%
\definecolor{currentstroke}{rgb}{0.000000,0.000000,0.000000}%
\pgfsetstrokecolor{currentstroke}%
\pgfsetdash{}{0pt}%
\pgfpathmoveto{\pgfqpoint{2.332854in}{2.891676in}}%
\pgfpathlineto{\pgfqpoint{2.346838in}{2.871952in}}%
\pgfpathlineto{\pgfqpoint{2.360814in}{2.852424in}}%
\pgfpathlineto{\pgfqpoint{2.374782in}{2.833091in}}%
\pgfpathlineto{\pgfqpoint{2.388743in}{2.813951in}}%
\pgfpathlineto{\pgfqpoint{2.379615in}{2.821456in}}%
\pgfpathlineto{\pgfqpoint{2.370465in}{2.829290in}}%
\pgfpathlineto{\pgfqpoint{2.361292in}{2.837460in}}%
\pgfpathlineto{\pgfqpoint{2.352096in}{2.845971in}}%
\pgfpathlineto{\pgfqpoint{2.338080in}{2.865645in}}%
\pgfpathlineto{\pgfqpoint{2.324056in}{2.885512in}}%
\pgfpathlineto{\pgfqpoint{2.310023in}{2.905575in}}%
\pgfpathlineto{\pgfqpoint{2.295982in}{2.925836in}}%
\pgfpathlineto{\pgfqpoint{2.305235in}{2.916781in}}%
\pgfpathlineto{\pgfqpoint{2.314465in}{2.908073in}}%
\pgfpathlineto{\pgfqpoint{2.323671in}{2.899706in}}%
\pgfpathlineto{\pgfqpoint{2.332854in}{2.891676in}}%
\pgfpathclose%
\pgfusepath{fill}%
\end{pgfscope}%
\begin{pgfscope}%
\pgfpathrectangle{\pgfqpoint{1.150000in}{0.150000in}}{\pgfqpoint{5.700000in}{5.700000in}}%
\pgfusepath{clip}%
\pgfsetbuttcap%
\pgfsetroundjoin%
\definecolor{currentfill}{rgb}{0.168126,0.459988,0.558082}%
\pgfsetfillcolor{currentfill}%
\pgfsetfillopacity{0.700000}%
\pgfsetlinewidth{0.000000pt}%
\definecolor{currentstroke}{rgb}{0.000000,0.000000,0.000000}%
\pgfsetstrokecolor{currentstroke}%
\pgfsetdash{}{0pt}%
\pgfpathmoveto{\pgfqpoint{5.485293in}{2.782533in}}%
\pgfpathlineto{\pgfqpoint{5.499785in}{2.790164in}}%
\pgfpathlineto{\pgfqpoint{5.514293in}{2.797895in}}%
\pgfpathlineto{\pgfqpoint{5.528817in}{2.805727in}}%
\pgfpathlineto{\pgfqpoint{5.543356in}{2.813660in}}%
\pgfpathlineto{\pgfqpoint{5.535836in}{2.803962in}}%
\pgfpathlineto{\pgfqpoint{5.528309in}{2.794146in}}%
\pgfpathlineto{\pgfqpoint{5.520773in}{2.784212in}}%
\pgfpathlineto{\pgfqpoint{5.513230in}{2.774160in}}%
\pgfpathlineto{\pgfqpoint{5.498686in}{2.766273in}}%
\pgfpathlineto{\pgfqpoint{5.484157in}{2.758486in}}%
\pgfpathlineto{\pgfqpoint{5.469644in}{2.750799in}}%
\pgfpathlineto{\pgfqpoint{5.455147in}{2.743213in}}%
\pgfpathlineto{\pgfqpoint{5.462695in}{2.753213in}}%
\pgfpathlineto{\pgfqpoint{5.470236in}{2.763099in}}%
\pgfpathlineto{\pgfqpoint{5.477768in}{2.772873in}}%
\pgfpathlineto{\pgfqpoint{5.485293in}{2.782533in}}%
\pgfpathclose%
\pgfusepath{fill}%
\end{pgfscope}%
\begin{pgfscope}%
\pgfpathrectangle{\pgfqpoint{1.150000in}{0.150000in}}{\pgfqpoint{5.700000in}{5.700000in}}%
\pgfusepath{clip}%
\pgfsetbuttcap%
\pgfsetroundjoin%
\definecolor{currentfill}{rgb}{0.271828,0.209303,0.504434}%
\pgfsetfillcolor{currentfill}%
\pgfsetfillopacity{0.700000}%
\pgfsetlinewidth{0.000000pt}%
\definecolor{currentstroke}{rgb}{0.000000,0.000000,0.000000}%
\pgfsetstrokecolor{currentstroke}%
\pgfsetdash{}{0pt}%
\pgfpathmoveto{\pgfqpoint{2.942819in}{2.188550in}}%
\pgfpathlineto{\pgfqpoint{2.956608in}{2.176049in}}%
\pgfpathlineto{\pgfqpoint{2.970396in}{2.163685in}}%
\pgfpathlineto{\pgfqpoint{2.984182in}{2.151459in}}%
\pgfpathlineto{\pgfqpoint{2.997968in}{2.139368in}}%
\pgfpathlineto{\pgfqpoint{2.989346in}{2.141305in}}%
\pgfpathlineto{\pgfqpoint{2.980709in}{2.143510in}}%
\pgfpathlineto{\pgfqpoint{2.972056in}{2.145988in}}%
\pgfpathlineto{\pgfqpoint{2.963387in}{2.148745in}}%
\pgfpathlineto{\pgfqpoint{2.949562in}{2.161336in}}%
\pgfpathlineto{\pgfqpoint{2.935735in}{2.174064in}}%
\pgfpathlineto{\pgfqpoint{2.921908in}{2.186929in}}%
\pgfpathlineto{\pgfqpoint{2.908078in}{2.199932in}}%
\pgfpathlineto{\pgfqpoint{2.916788in}{2.196666in}}%
\pgfpathlineto{\pgfqpoint{2.925481in}{2.193684in}}%
\pgfpathlineto{\pgfqpoint{2.934158in}{2.190980in}}%
\pgfpathlineto{\pgfqpoint{2.942819in}{2.188550in}}%
\pgfpathclose%
\pgfusepath{fill}%
\end{pgfscope}%
\begin{pgfscope}%
\pgfpathrectangle{\pgfqpoint{1.150000in}{0.150000in}}{\pgfqpoint{5.700000in}{5.700000in}}%
\pgfusepath{clip}%
\pgfsetbuttcap%
\pgfsetroundjoin%
\definecolor{currentfill}{rgb}{0.282656,0.100196,0.422160}%
\pgfsetfillcolor{currentfill}%
\pgfsetfillopacity{0.700000}%
\pgfsetlinewidth{0.000000pt}%
\definecolor{currentstroke}{rgb}{0.000000,0.000000,0.000000}%
\pgfsetstrokecolor{currentstroke}%
\pgfsetdash{}{0pt}%
\pgfpathmoveto{\pgfqpoint{4.427563in}{1.955420in}}%
\pgfpathlineto{\pgfqpoint{4.441546in}{1.956313in}}%
\pgfpathlineto{\pgfqpoint{4.455538in}{1.957308in}}%
\pgfpathlineto{\pgfqpoint{4.469540in}{1.958402in}}%
\pgfpathlineto{\pgfqpoint{4.483552in}{1.959596in}}%
\pgfpathlineto{\pgfqpoint{4.475661in}{1.948154in}}%
\pgfpathlineto{\pgfqpoint{4.467766in}{1.936726in}}%
\pgfpathlineto{\pgfqpoint{4.459866in}{1.925317in}}%
\pgfpathlineto{\pgfqpoint{4.451961in}{1.913928in}}%
\pgfpathlineto{\pgfqpoint{4.437944in}{1.913038in}}%
\pgfpathlineto{\pgfqpoint{4.423937in}{1.912247in}}%
\pgfpathlineto{\pgfqpoint{4.409939in}{1.911557in}}%
\pgfpathlineto{\pgfqpoint{4.395951in}{1.910967in}}%
\pgfpathlineto{\pgfqpoint{4.403861in}{1.922045in}}%
\pgfpathlineto{\pgfqpoint{4.411767in}{1.933149in}}%
\pgfpathlineto{\pgfqpoint{4.419667in}{1.944274in}}%
\pgfpathlineto{\pgfqpoint{4.427563in}{1.955420in}}%
\pgfpathclose%
\pgfusepath{fill}%
\end{pgfscope}%
\begin{pgfscope}%
\pgfpathrectangle{\pgfqpoint{1.150000in}{0.150000in}}{\pgfqpoint{5.700000in}{5.700000in}}%
\pgfusepath{clip}%
\pgfsetbuttcap%
\pgfsetroundjoin%
\definecolor{currentfill}{rgb}{0.280267,0.073417,0.397163}%
\pgfsetfillcolor{currentfill}%
\pgfsetfillopacity{0.700000}%
\pgfsetlinewidth{0.000000pt}%
\definecolor{currentstroke}{rgb}{0.000000,0.000000,0.000000}%
\pgfsetstrokecolor{currentstroke}%
\pgfsetdash{}{0pt}%
\pgfpathmoveto{\pgfqpoint{4.340092in}{1.909609in}}%
\pgfpathlineto{\pgfqpoint{4.354043in}{1.909798in}}%
\pgfpathlineto{\pgfqpoint{4.368003in}{1.910087in}}%
\pgfpathlineto{\pgfqpoint{4.381972in}{1.910477in}}%
\pgfpathlineto{\pgfqpoint{4.395951in}{1.910967in}}%
\pgfpathlineto{\pgfqpoint{4.388036in}{1.899916in}}%
\pgfpathlineto{\pgfqpoint{4.380115in}{1.888897in}}%
\pgfpathlineto{\pgfqpoint{4.372191in}{1.877911in}}%
\pgfpathlineto{\pgfqpoint{4.364261in}{1.866963in}}%
\pgfpathlineto{\pgfqpoint{4.350276in}{1.866794in}}%
\pgfpathlineto{\pgfqpoint{4.336300in}{1.866726in}}%
\pgfpathlineto{\pgfqpoint{4.322333in}{1.866758in}}%
\pgfpathlineto{\pgfqpoint{4.308376in}{1.866890in}}%
\pgfpathlineto{\pgfqpoint{4.316312in}{1.877511in}}%
\pgfpathlineto{\pgfqpoint{4.324244in}{1.888173in}}%
\pgfpathlineto{\pgfqpoint{4.332171in}{1.898873in}}%
\pgfpathlineto{\pgfqpoint{4.340092in}{1.909609in}}%
\pgfpathclose%
\pgfusepath{fill}%
\end{pgfscope}%
\begin{pgfscope}%
\pgfpathrectangle{\pgfqpoint{1.150000in}{0.150000in}}{\pgfqpoint{5.700000in}{5.700000in}}%
\pgfusepath{clip}%
\pgfsetbuttcap%
\pgfsetroundjoin%
\definecolor{currentfill}{rgb}{0.267004,0.004874,0.329415}%
\pgfsetfillcolor{currentfill}%
\pgfsetfillopacity{0.700000}%
\pgfsetlinewidth{0.000000pt}%
\definecolor{currentstroke}{rgb}{0.000000,0.000000,0.000000}%
\pgfsetstrokecolor{currentstroke}%
\pgfsetdash{}{0pt}%
\pgfpathmoveto{\pgfqpoint{3.934713in}{1.789264in}}%
\pgfpathlineto{\pgfqpoint{3.948547in}{1.786015in}}%
\pgfpathlineto{\pgfqpoint{3.962387in}{1.782871in}}%
\pgfpathlineto{\pgfqpoint{3.976235in}{1.779831in}}%
\pgfpathlineto{\pgfqpoint{3.990089in}{1.776894in}}%
\pgfpathlineto{\pgfqpoint{3.982040in}{1.768656in}}%
\pgfpathlineto{\pgfqpoint{3.973985in}{1.760525in}}%
\pgfpathlineto{\pgfqpoint{3.965923in}{1.752505in}}%
\pgfpathlineto{\pgfqpoint{3.957855in}{1.744600in}}%
\pgfpathlineto{\pgfqpoint{3.943987in}{1.747928in}}%
\pgfpathlineto{\pgfqpoint{3.930126in}{1.751360in}}%
\pgfpathlineto{\pgfqpoint{3.916271in}{1.754897in}}%
\pgfpathlineto{\pgfqpoint{3.902422in}{1.758538in}}%
\pgfpathlineto{\pgfqpoint{3.910505in}{1.766044in}}%
\pgfpathlineto{\pgfqpoint{3.918581in}{1.773669in}}%
\pgfpathlineto{\pgfqpoint{3.926650in}{1.781411in}}%
\pgfpathlineto{\pgfqpoint{3.934713in}{1.789264in}}%
\pgfpathclose%
\pgfusepath{fill}%
\end{pgfscope}%
\begin{pgfscope}%
\pgfpathrectangle{\pgfqpoint{1.150000in}{0.150000in}}{\pgfqpoint{5.700000in}{5.700000in}}%
\pgfusepath{clip}%
\pgfsetbuttcap%
\pgfsetroundjoin%
\definecolor{currentfill}{rgb}{0.283187,0.125848,0.444960}%
\pgfsetfillcolor{currentfill}%
\pgfsetfillopacity{0.700000}%
\pgfsetlinewidth{0.000000pt}%
\definecolor{currentstroke}{rgb}{0.000000,0.000000,0.000000}%
\pgfsetstrokecolor{currentstroke}%
\pgfsetdash{}{0pt}%
\pgfpathmoveto{\pgfqpoint{4.515066in}{2.005460in}}%
\pgfpathlineto{\pgfqpoint{4.529083in}{2.007040in}}%
\pgfpathlineto{\pgfqpoint{4.543111in}{2.008720in}}%
\pgfpathlineto{\pgfqpoint{4.557149in}{2.010499in}}%
\pgfpathlineto{\pgfqpoint{4.571198in}{2.012379in}}%
\pgfpathlineto{\pgfqpoint{4.563331in}{2.000625in}}%
\pgfpathlineto{\pgfqpoint{4.555459in}{1.988870in}}%
\pgfpathlineto{\pgfqpoint{4.547582in}{1.977118in}}%
\pgfpathlineto{\pgfqpoint{4.539701in}{1.965371in}}%
\pgfpathlineto{\pgfqpoint{4.525648in}{1.963777in}}%
\pgfpathlineto{\pgfqpoint{4.511606in}{1.962284in}}%
\pgfpathlineto{\pgfqpoint{4.497574in}{1.960890in}}%
\pgfpathlineto{\pgfqpoint{4.483552in}{1.959596in}}%
\pgfpathlineto{\pgfqpoint{4.491438in}{1.971050in}}%
\pgfpathlineto{\pgfqpoint{4.499319in}{1.982514in}}%
\pgfpathlineto{\pgfqpoint{4.507195in}{1.993985in}}%
\pgfpathlineto{\pgfqpoint{4.515066in}{2.005460in}}%
\pgfpathclose%
\pgfusepath{fill}%
\end{pgfscope}%
\begin{pgfscope}%
\pgfpathrectangle{\pgfqpoint{1.150000in}{0.150000in}}{\pgfqpoint{5.700000in}{5.700000in}}%
\pgfusepath{clip}%
\pgfsetbuttcap%
\pgfsetroundjoin%
\definecolor{currentfill}{rgb}{0.126453,0.570633,0.549841}%
\pgfsetfillcolor{currentfill}%
\pgfsetfillopacity{0.700000}%
\pgfsetlinewidth{0.000000pt}%
\definecolor{currentstroke}{rgb}{0.000000,0.000000,0.000000}%
\pgfsetstrokecolor{currentstroke}%
\pgfsetdash{}{0pt}%
\pgfpathmoveto{\pgfqpoint{5.867211in}{3.085243in}}%
\pgfpathlineto{\pgfqpoint{5.881923in}{3.094416in}}%
\pgfpathlineto{\pgfqpoint{5.896651in}{3.103691in}}%
\pgfpathlineto{\pgfqpoint{5.911396in}{3.113066in}}%
\pgfpathlineto{\pgfqpoint{5.926159in}{3.122543in}}%
\pgfpathlineto{\pgfqpoint{5.918853in}{3.115404in}}%
\pgfpathlineto{\pgfqpoint{5.911538in}{3.108135in}}%
\pgfpathlineto{\pgfqpoint{5.904212in}{3.100735in}}%
\pgfpathlineto{\pgfqpoint{5.896877in}{3.093204in}}%
\pgfpathlineto{\pgfqpoint{5.882104in}{3.083672in}}%
\pgfpathlineto{\pgfqpoint{5.867348in}{3.074242in}}%
\pgfpathlineto{\pgfqpoint{5.852609in}{3.064913in}}%
\pgfpathlineto{\pgfqpoint{5.837887in}{3.055686in}}%
\pgfpathlineto{\pgfqpoint{5.845233in}{3.063264in}}%
\pgfpathlineto{\pgfqpoint{5.852568in}{3.070716in}}%
\pgfpathlineto{\pgfqpoint{5.859895in}{3.078042in}}%
\pgfpathlineto{\pgfqpoint{5.867211in}{3.085243in}}%
\pgfpathclose%
\pgfusepath{fill}%
\end{pgfscope}%
\begin{pgfscope}%
\pgfpathrectangle{\pgfqpoint{1.150000in}{0.150000in}}{\pgfqpoint{5.700000in}{5.700000in}}%
\pgfusepath{clip}%
\pgfsetbuttcap%
\pgfsetroundjoin%
\definecolor{currentfill}{rgb}{0.277018,0.050344,0.375715}%
\pgfsetfillcolor{currentfill}%
\pgfsetfillopacity{0.700000}%
\pgfsetlinewidth{0.000000pt}%
\definecolor{currentstroke}{rgb}{0.000000,0.000000,0.000000}%
\pgfsetstrokecolor{currentstroke}%
\pgfsetdash{}{0pt}%
\pgfpathmoveto{\pgfqpoint{4.252633in}{1.868429in}}%
\pgfpathlineto{\pgfqpoint{4.266556in}{1.867893in}}%
\pgfpathlineto{\pgfqpoint{4.280487in}{1.867458in}}%
\pgfpathlineto{\pgfqpoint{4.294427in}{1.867123in}}%
\pgfpathlineto{\pgfqpoint{4.308376in}{1.866890in}}%
\pgfpathlineto{\pgfqpoint{4.300434in}{1.856314in}}%
\pgfpathlineto{\pgfqpoint{4.292487in}{1.845786in}}%
\pgfpathlineto{\pgfqpoint{4.284536in}{1.835309in}}%
\pgfpathlineto{\pgfqpoint{4.276579in}{1.824886in}}%
\pgfpathlineto{\pgfqpoint{4.262623in}{1.825458in}}%
\pgfpathlineto{\pgfqpoint{4.248675in}{1.826131in}}%
\pgfpathlineto{\pgfqpoint{4.234735in}{1.826905in}}%
\pgfpathlineto{\pgfqpoint{4.220805in}{1.827780in}}%
\pgfpathlineto{\pgfqpoint{4.228770in}{1.837857in}}%
\pgfpathlineto{\pgfqpoint{4.236729in}{1.847994in}}%
\pgfpathlineto{\pgfqpoint{4.244684in}{1.858185in}}%
\pgfpathlineto{\pgfqpoint{4.252633in}{1.868429in}}%
\pgfpathclose%
\pgfusepath{fill}%
\end{pgfscope}%
\begin{pgfscope}%
\pgfpathrectangle{\pgfqpoint{1.150000in}{0.150000in}}{\pgfqpoint{5.700000in}{5.700000in}}%
\pgfusepath{clip}%
\pgfsetbuttcap%
\pgfsetroundjoin%
\definecolor{currentfill}{rgb}{0.274952,0.037752,0.364543}%
\pgfsetfillcolor{currentfill}%
\pgfsetfillopacity{0.700000}%
\pgfsetlinewidth{0.000000pt}%
\definecolor{currentstroke}{rgb}{0.000000,0.000000,0.000000}%
\pgfsetstrokecolor{currentstroke}%
\pgfsetdash{}{0pt}%
\pgfpathmoveto{\pgfqpoint{3.505851in}{1.842979in}}%
\pgfpathlineto{\pgfqpoint{3.519622in}{1.835890in}}%
\pgfpathlineto{\pgfqpoint{3.533396in}{1.828915in}}%
\pgfpathlineto{\pgfqpoint{3.547174in}{1.822052in}}%
\pgfpathlineto{\pgfqpoint{3.560956in}{1.815301in}}%
\pgfpathlineto{\pgfqpoint{3.552706in}{1.811260in}}%
\pgfpathlineto{\pgfqpoint{3.544447in}{1.807402in}}%
\pgfpathlineto{\pgfqpoint{3.536178in}{1.803734in}}%
\pgfpathlineto{\pgfqpoint{3.527899in}{1.800260in}}%
\pgfpathlineto{\pgfqpoint{3.514093in}{1.807459in}}%
\pgfpathlineto{\pgfqpoint{3.500291in}{1.814772in}}%
\pgfpathlineto{\pgfqpoint{3.486492in}{1.822197in}}%
\pgfpathlineto{\pgfqpoint{3.472696in}{1.829736in}}%
\pgfpathlineto{\pgfqpoint{3.481000in}{1.832754in}}%
\pgfpathlineto{\pgfqpoint{3.489294in}{1.835970in}}%
\pgfpathlineto{\pgfqpoint{3.497577in}{1.839380in}}%
\pgfpathlineto{\pgfqpoint{3.505851in}{1.842979in}}%
\pgfpathclose%
\pgfusepath{fill}%
\end{pgfscope}%
\begin{pgfscope}%
\pgfpathrectangle{\pgfqpoint{1.150000in}{0.150000in}}{\pgfqpoint{5.700000in}{5.700000in}}%
\pgfusepath{clip}%
\pgfsetbuttcap%
\pgfsetroundjoin%
\definecolor{currentfill}{rgb}{0.143343,0.522773,0.556295}%
\pgfsetfillcolor{currentfill}%
\pgfsetfillopacity{0.700000}%
\pgfsetlinewidth{0.000000pt}%
\definecolor{currentstroke}{rgb}{0.000000,0.000000,0.000000}%
\pgfsetstrokecolor{currentstroke}%
\pgfsetdash{}{0pt}%
\pgfpathmoveto{\pgfqpoint{2.276830in}{2.972573in}}%
\pgfpathlineto{\pgfqpoint{2.290849in}{2.952044in}}%
\pgfpathlineto{\pgfqpoint{2.304859in}{2.931720in}}%
\pgfpathlineto{\pgfqpoint{2.318861in}{2.911598in}}%
\pgfpathlineto{\pgfqpoint{2.332854in}{2.891676in}}%
\pgfpathlineto{\pgfqpoint{2.323671in}{2.899706in}}%
\pgfpathlineto{\pgfqpoint{2.314465in}{2.908073in}}%
\pgfpathlineto{\pgfqpoint{2.305235in}{2.916781in}}%
\pgfpathlineto{\pgfqpoint{2.295982in}{2.925836in}}%
\pgfpathlineto{\pgfqpoint{2.281932in}{2.946296in}}%
\pgfpathlineto{\pgfqpoint{2.267873in}{2.966957in}}%
\pgfpathlineto{\pgfqpoint{2.253805in}{2.987821in}}%
\pgfpathlineto{\pgfqpoint{2.239727in}{3.008891in}}%
\pgfpathlineto{\pgfqpoint{2.249039in}{2.999287in}}%
\pgfpathlineto{\pgfqpoint{2.258327in}{2.990037in}}%
\pgfpathlineto{\pgfqpoint{2.267590in}{2.981134in}}%
\pgfpathlineto{\pgfqpoint{2.276830in}{2.972573in}}%
\pgfpathclose%
\pgfusepath{fill}%
\end{pgfscope}%
\begin{pgfscope}%
\pgfpathrectangle{\pgfqpoint{1.150000in}{0.150000in}}{\pgfqpoint{5.700000in}{5.700000in}}%
\pgfusepath{clip}%
\pgfsetbuttcap%
\pgfsetroundjoin%
\definecolor{currentfill}{rgb}{0.281887,0.150881,0.465405}%
\pgfsetfillcolor{currentfill}%
\pgfsetfillopacity{0.700000}%
\pgfsetlinewidth{0.000000pt}%
\definecolor{currentstroke}{rgb}{0.000000,0.000000,0.000000}%
\pgfsetstrokecolor{currentstroke}%
\pgfsetdash{}{0pt}%
\pgfpathmoveto{\pgfqpoint{4.602618in}{2.059341in}}%
\pgfpathlineto{\pgfqpoint{4.616673in}{2.061589in}}%
\pgfpathlineto{\pgfqpoint{4.630740in}{2.063936in}}%
\pgfpathlineto{\pgfqpoint{4.644817in}{2.066383in}}%
\pgfpathlineto{\pgfqpoint{4.658905in}{2.068929in}}%
\pgfpathlineto{\pgfqpoint{4.651060in}{2.056940in}}%
\pgfpathlineto{\pgfqpoint{4.643211in}{2.044936in}}%
\pgfpathlineto{\pgfqpoint{4.635357in}{2.032919in}}%
\pgfpathlineto{\pgfqpoint{4.627499in}{2.020893in}}%
\pgfpathlineto{\pgfqpoint{4.613407in}{2.018615in}}%
\pgfpathlineto{\pgfqpoint{4.599327in}{2.016437in}}%
\pgfpathlineto{\pgfqpoint{4.585257in}{2.014358in}}%
\pgfpathlineto{\pgfqpoint{4.571198in}{2.012379in}}%
\pgfpathlineto{\pgfqpoint{4.579060in}{2.024130in}}%
\pgfpathlineto{\pgfqpoint{4.586917in}{2.035876in}}%
\pgfpathlineto{\pgfqpoint{4.594770in}{2.047614in}}%
\pgfpathlineto{\pgfqpoint{4.602618in}{2.059341in}}%
\pgfpathclose%
\pgfusepath{fill}%
\end{pgfscope}%
\begin{pgfscope}%
\pgfpathrectangle{\pgfqpoint{1.150000in}{0.150000in}}{\pgfqpoint{5.700000in}{5.700000in}}%
\pgfusepath{clip}%
\pgfsetbuttcap%
\pgfsetroundjoin%
\definecolor{currentfill}{rgb}{0.281924,0.089666,0.412415}%
\pgfsetfillcolor{currentfill}%
\pgfsetfillopacity{0.700000}%
\pgfsetlinewidth{0.000000pt}%
\definecolor{currentstroke}{rgb}{0.000000,0.000000,0.000000}%
\pgfsetstrokecolor{currentstroke}%
\pgfsetdash{}{0pt}%
\pgfpathmoveto{\pgfqpoint{3.307361in}{1.929236in}}%
\pgfpathlineto{\pgfqpoint{3.321126in}{1.920296in}}%
\pgfpathlineto{\pgfqpoint{3.334893in}{1.911475in}}%
\pgfpathlineto{\pgfqpoint{3.348662in}{1.902774in}}%
\pgfpathlineto{\pgfqpoint{3.362433in}{1.894191in}}%
\pgfpathlineto{\pgfqpoint{3.354064in}{1.892302in}}%
\pgfpathlineto{\pgfqpoint{3.345684in}{1.890631in}}%
\pgfpathlineto{\pgfqpoint{3.337292in}{1.889183in}}%
\pgfpathlineto{\pgfqpoint{3.328888in}{1.887961in}}%
\pgfpathlineto{\pgfqpoint{3.315087in}{1.897015in}}%
\pgfpathlineto{\pgfqpoint{3.301288in}{1.906188in}}%
\pgfpathlineto{\pgfqpoint{3.287491in}{1.915481in}}%
\pgfpathlineto{\pgfqpoint{3.273695in}{1.924893in}}%
\pgfpathlineto{\pgfqpoint{3.282130in}{1.925636in}}%
\pgfpathlineto{\pgfqpoint{3.290552in}{1.926610in}}%
\pgfpathlineto{\pgfqpoint{3.298962in}{1.927812in}}%
\pgfpathlineto{\pgfqpoint{3.307361in}{1.929236in}}%
\pgfpathclose%
\pgfusepath{fill}%
\end{pgfscope}%
\begin{pgfscope}%
\pgfpathrectangle{\pgfqpoint{1.150000in}{0.150000in}}{\pgfqpoint{5.700000in}{5.700000in}}%
\pgfusepath{clip}%
\pgfsetbuttcap%
\pgfsetroundjoin%
\definecolor{currentfill}{rgb}{0.243113,0.292092,0.538516}%
\pgfsetfillcolor{currentfill}%
\pgfsetfillopacity{0.700000}%
\pgfsetlinewidth{0.000000pt}%
\definecolor{currentstroke}{rgb}{0.000000,0.000000,0.000000}%
\pgfsetstrokecolor{currentstroke}%
\pgfsetdash{}{0pt}%
\pgfpathmoveto{\pgfqpoint{4.984540in}{2.353429in}}%
\pgfpathlineto{\pgfqpoint{4.998772in}{2.358342in}}%
\pgfpathlineto{\pgfqpoint{5.013018in}{2.363353in}}%
\pgfpathlineto{\pgfqpoint{5.027276in}{2.368465in}}%
\pgfpathlineto{\pgfqpoint{5.041548in}{2.373676in}}%
\pgfpathlineto{\pgfqpoint{5.033817in}{2.361740in}}%
\pgfpathlineto{\pgfqpoint{5.026081in}{2.349732in}}%
\pgfpathlineto{\pgfqpoint{5.018339in}{2.337652in}}%
\pgfpathlineto{\pgfqpoint{5.010591in}{2.325504in}}%
\pgfpathlineto{\pgfqpoint{4.996318in}{2.320471in}}%
\pgfpathlineto{\pgfqpoint{4.982057in}{2.315538in}}%
\pgfpathlineto{\pgfqpoint{4.967810in}{2.310704in}}%
\pgfpathlineto{\pgfqpoint{4.953576in}{2.305970in}}%
\pgfpathlineto{\pgfqpoint{4.961325in}{2.317933in}}%
\pgfpathlineto{\pgfqpoint{4.969069in}{2.329831in}}%
\pgfpathlineto{\pgfqpoint{4.976807in}{2.341664in}}%
\pgfpathlineto{\pgfqpoint{4.984540in}{2.353429in}}%
\pgfpathclose%
\pgfusepath{fill}%
\end{pgfscope}%
\begin{pgfscope}%
\pgfpathrectangle{\pgfqpoint{1.150000in}{0.150000in}}{\pgfqpoint{5.700000in}{5.700000in}}%
\pgfusepath{clip}%
\pgfsetbuttcap%
\pgfsetroundjoin%
\definecolor{currentfill}{rgb}{0.121148,0.592739,0.544641}%
\pgfsetfillcolor{currentfill}%
\pgfsetfillopacity{0.700000}%
\pgfsetlinewidth{0.000000pt}%
\definecolor{currentstroke}{rgb}{0.000000,0.000000,0.000000}%
\pgfsetstrokecolor{currentstroke}%
\pgfsetdash{}{0pt}%
\pgfpathmoveto{\pgfqpoint{5.955282in}{3.149821in}}%
\pgfpathlineto{\pgfqpoint{5.970051in}{3.159324in}}%
\pgfpathlineto{\pgfqpoint{5.984836in}{3.168929in}}%
\pgfpathlineto{\pgfqpoint{5.999639in}{3.178635in}}%
\pgfpathlineto{\pgfqpoint{5.992383in}{3.172067in}}%
\pgfpathlineto{\pgfqpoint{5.985116in}{3.165370in}}%
\pgfpathlineto{\pgfqpoint{5.977839in}{3.158542in}}%
\pgfpathlineto{\pgfqpoint{5.970551in}{3.151582in}}%
\pgfpathlineto{\pgfqpoint{5.955736in}{3.141801in}}%
\pgfpathlineto{\pgfqpoint{5.940939in}{3.132121in}}%
\pgfpathlineto{\pgfqpoint{5.926159in}{3.122543in}}%
\pgfpathlineto{\pgfqpoint{5.933455in}{3.129553in}}%
\pgfpathlineto{\pgfqpoint{5.940741in}{3.136435in}}%
\pgfpathlineto{\pgfqpoint{5.948016in}{3.143191in}}%
\pgfpathlineto{\pgfqpoint{5.955282in}{3.149821in}}%
\pgfpathclose%
\pgfusepath{fill}%
\end{pgfscope}%
\begin{pgfscope}%
\pgfpathrectangle{\pgfqpoint{1.150000in}{0.150000in}}{\pgfqpoint{5.700000in}{5.700000in}}%
\pgfusepath{clip}%
\pgfsetbuttcap%
\pgfsetroundjoin%
\definecolor{currentfill}{rgb}{0.276194,0.190074,0.493001}%
\pgfsetfillcolor{currentfill}%
\pgfsetfillopacity{0.700000}%
\pgfsetlinewidth{0.000000pt}%
\definecolor{currentstroke}{rgb}{0.000000,0.000000,0.000000}%
\pgfsetstrokecolor{currentstroke}%
\pgfsetdash{}{0pt}%
\pgfpathmoveto{\pgfqpoint{2.997968in}{2.139368in}}%
\pgfpathlineto{\pgfqpoint{3.011754in}{2.127413in}}%
\pgfpathlineto{\pgfqpoint{3.025538in}{2.115592in}}%
\pgfpathlineto{\pgfqpoint{3.039322in}{2.103904in}}%
\pgfpathlineto{\pgfqpoint{3.053106in}{2.092349in}}%
\pgfpathlineto{\pgfqpoint{3.044522in}{2.093794in}}%
\pgfpathlineto{\pgfqpoint{3.035923in}{2.095503in}}%
\pgfpathlineto{\pgfqpoint{3.027308in}{2.097479in}}%
\pgfpathlineto{\pgfqpoint{3.018679in}{2.099728in}}%
\pgfpathlineto{\pgfqpoint{3.004857in}{2.111782in}}%
\pgfpathlineto{\pgfqpoint{2.991034in}{2.123969in}}%
\pgfpathlineto{\pgfqpoint{2.977211in}{2.136289in}}%
\pgfpathlineto{\pgfqpoint{2.963387in}{2.148745in}}%
\pgfpathlineto{\pgfqpoint{2.972056in}{2.145988in}}%
\pgfpathlineto{\pgfqpoint{2.980709in}{2.143510in}}%
\pgfpathlineto{\pgfqpoint{2.989346in}{2.141305in}}%
\pgfpathlineto{\pgfqpoint{2.997968in}{2.139368in}}%
\pgfpathclose%
\pgfusepath{fill}%
\end{pgfscope}%
\begin{pgfscope}%
\pgfpathrectangle{\pgfqpoint{1.150000in}{0.150000in}}{\pgfqpoint{5.700000in}{5.700000in}}%
\pgfusepath{clip}%
\pgfsetbuttcap%
\pgfsetroundjoin%
\definecolor{currentfill}{rgb}{0.197636,0.391528,0.554969}%
\pgfsetfillcolor{currentfill}%
\pgfsetfillopacity{0.700000}%
\pgfsetlinewidth{0.000000pt}%
\definecolor{currentstroke}{rgb}{0.000000,0.000000,0.000000}%
\pgfsetstrokecolor{currentstroke}%
\pgfsetdash{}{0pt}%
\pgfpathmoveto{\pgfqpoint{5.278981in}{2.603294in}}%
\pgfpathlineto{\pgfqpoint{5.293367in}{2.609929in}}%
\pgfpathlineto{\pgfqpoint{5.307768in}{2.616665in}}%
\pgfpathlineto{\pgfqpoint{5.322183in}{2.623501in}}%
\pgfpathlineto{\pgfqpoint{5.336614in}{2.630436in}}%
\pgfpathlineto{\pgfqpoint{5.328994in}{2.619531in}}%
\pgfpathlineto{\pgfqpoint{5.321367in}{2.608522in}}%
\pgfpathlineto{\pgfqpoint{5.313733in}{2.597410in}}%
\pgfpathlineto{\pgfqpoint{5.306092in}{2.586195in}}%
\pgfpathlineto{\pgfqpoint{5.291659in}{2.579362in}}%
\pgfpathlineto{\pgfqpoint{5.277240in}{2.572630in}}%
\pgfpathlineto{\pgfqpoint{5.262837in}{2.565997in}}%
\pgfpathlineto{\pgfqpoint{5.248447in}{2.559465in}}%
\pgfpathlineto{\pgfqpoint{5.256091in}{2.570569in}}%
\pgfpathlineto{\pgfqpoint{5.263728in}{2.581576in}}%
\pgfpathlineto{\pgfqpoint{5.271358in}{2.592484in}}%
\pgfpathlineto{\pgfqpoint{5.278981in}{2.603294in}}%
\pgfpathclose%
\pgfusepath{fill}%
\end{pgfscope}%
\begin{pgfscope}%
\pgfpathrectangle{\pgfqpoint{1.150000in}{0.150000in}}{\pgfqpoint{5.700000in}{5.700000in}}%
\pgfusepath{clip}%
\pgfsetbuttcap%
\pgfsetroundjoin%
\definecolor{currentfill}{rgb}{0.273809,0.031497,0.358853}%
\pgfsetfillcolor{currentfill}%
\pgfsetfillopacity{0.700000}%
\pgfsetlinewidth{0.000000pt}%
\definecolor{currentstroke}{rgb}{0.000000,0.000000,0.000000}%
\pgfsetstrokecolor{currentstroke}%
\pgfsetdash{}{0pt}%
\pgfpathmoveto{\pgfqpoint{4.165164in}{1.832294in}}%
\pgfpathlineto{\pgfqpoint{4.179062in}{1.831013in}}%
\pgfpathlineto{\pgfqpoint{4.192968in}{1.829834in}}%
\pgfpathlineto{\pgfqpoint{4.206882in}{1.828756in}}%
\pgfpathlineto{\pgfqpoint{4.220805in}{1.827780in}}%
\pgfpathlineto{\pgfqpoint{4.212834in}{1.817765in}}%
\pgfpathlineto{\pgfqpoint{4.204859in}{1.807814in}}%
\pgfpathlineto{\pgfqpoint{4.196878in}{1.797933in}}%
\pgfpathlineto{\pgfqpoint{4.188892in}{1.788123in}}%
\pgfpathlineto{\pgfqpoint{4.174960in}{1.789456in}}%
\pgfpathlineto{\pgfqpoint{4.161037in}{1.790890in}}%
\pgfpathlineto{\pgfqpoint{4.147121in}{1.792426in}}%
\pgfpathlineto{\pgfqpoint{4.133213in}{1.794064in}}%
\pgfpathlineto{\pgfqpoint{4.141209in}{1.803510in}}%
\pgfpathlineto{\pgfqpoint{4.149200in}{1.813032in}}%
\pgfpathlineto{\pgfqpoint{4.157184in}{1.822628in}}%
\pgfpathlineto{\pgfqpoint{4.165164in}{1.832294in}}%
\pgfpathclose%
\pgfusepath{fill}%
\end{pgfscope}%
\begin{pgfscope}%
\pgfpathrectangle{\pgfqpoint{1.150000in}{0.150000in}}{\pgfqpoint{5.700000in}{5.700000in}}%
\pgfusepath{clip}%
\pgfsetbuttcap%
\pgfsetroundjoin%
\definecolor{currentfill}{rgb}{0.157729,0.485932,0.558013}%
\pgfsetfillcolor{currentfill}%
\pgfsetfillopacity{0.700000}%
\pgfsetlinewidth{0.000000pt}%
\definecolor{currentstroke}{rgb}{0.000000,0.000000,0.000000}%
\pgfsetstrokecolor{currentstroke}%
\pgfsetdash{}{0pt}%
\pgfpathmoveto{\pgfqpoint{5.573350in}{2.851267in}}%
\pgfpathlineto{\pgfqpoint{5.587899in}{2.859325in}}%
\pgfpathlineto{\pgfqpoint{5.602464in}{2.867485in}}%
\pgfpathlineto{\pgfqpoint{5.617044in}{2.875745in}}%
\pgfpathlineto{\pgfqpoint{5.631641in}{2.884106in}}%
\pgfpathlineto{\pgfqpoint{5.624161in}{2.874864in}}%
\pgfpathlineto{\pgfqpoint{5.616673in}{2.865498in}}%
\pgfpathlineto{\pgfqpoint{5.609176in}{2.856009in}}%
\pgfpathlineto{\pgfqpoint{5.601670in}{2.846397in}}%
\pgfpathlineto{\pgfqpoint{5.587068in}{2.838062in}}%
\pgfpathlineto{\pgfqpoint{5.572481in}{2.829827in}}%
\pgfpathlineto{\pgfqpoint{5.557910in}{2.821693in}}%
\pgfpathlineto{\pgfqpoint{5.543356in}{2.813660in}}%
\pgfpathlineto{\pgfqpoint{5.550867in}{2.823239in}}%
\pgfpathlineto{\pgfqpoint{5.558370in}{2.832700in}}%
\pgfpathlineto{\pgfqpoint{5.565864in}{2.842042in}}%
\pgfpathlineto{\pgfqpoint{5.573350in}{2.851267in}}%
\pgfpathclose%
\pgfusepath{fill}%
\end{pgfscope}%
\begin{pgfscope}%
\pgfpathrectangle{\pgfqpoint{1.150000in}{0.150000in}}{\pgfqpoint{5.700000in}{5.700000in}}%
\pgfusepath{clip}%
\pgfsetbuttcap%
\pgfsetroundjoin%
\definecolor{currentfill}{rgb}{0.278012,0.180367,0.486697}%
\pgfsetfillcolor{currentfill}%
\pgfsetfillopacity{0.700000}%
\pgfsetlinewidth{0.000000pt}%
\definecolor{currentstroke}{rgb}{0.000000,0.000000,0.000000}%
\pgfsetstrokecolor{currentstroke}%
\pgfsetdash{}{0pt}%
\pgfpathmoveto{\pgfqpoint{4.690234in}{2.116690in}}%
\pgfpathlineto{\pgfqpoint{4.704330in}{2.119586in}}%
\pgfpathlineto{\pgfqpoint{4.718438in}{2.122582in}}%
\pgfpathlineto{\pgfqpoint{4.732557in}{2.125678in}}%
\pgfpathlineto{\pgfqpoint{4.746687in}{2.128873in}}%
\pgfpathlineto{\pgfqpoint{4.738865in}{2.116723in}}%
\pgfpathlineto{\pgfqpoint{4.731038in}{2.104544in}}%
\pgfpathlineto{\pgfqpoint{4.723206in}{2.092339in}}%
\pgfpathlineto{\pgfqpoint{4.715370in}{2.080109in}}%
\pgfpathlineto{\pgfqpoint{4.701236in}{2.077165in}}%
\pgfpathlineto{\pgfqpoint{4.687115in}{2.074320in}}%
\pgfpathlineto{\pgfqpoint{4.673004in}{2.071575in}}%
\pgfpathlineto{\pgfqpoint{4.658905in}{2.068929in}}%
\pgfpathlineto{\pgfqpoint{4.666744in}{2.080901in}}%
\pgfpathlineto{\pgfqpoint{4.674579in}{2.092853in}}%
\pgfpathlineto{\pgfqpoint{4.682409in}{2.104784in}}%
\pgfpathlineto{\pgfqpoint{4.690234in}{2.116690in}}%
\pgfpathclose%
\pgfusepath{fill}%
\end{pgfscope}%
\begin{pgfscope}%
\pgfpathrectangle{\pgfqpoint{1.150000in}{0.150000in}}{\pgfqpoint{5.700000in}{5.700000in}}%
\pgfusepath{clip}%
\pgfsetbuttcap%
\pgfsetroundjoin%
\definecolor{currentfill}{rgb}{0.279574,0.170599,0.479997}%
\pgfsetfillcolor{currentfill}%
\pgfsetfillopacity{0.700000}%
\pgfsetlinewidth{0.000000pt}%
\definecolor{currentstroke}{rgb}{0.000000,0.000000,0.000000}%
\pgfsetstrokecolor{currentstroke}%
\pgfsetdash{}{0pt}%
\pgfpathmoveto{\pgfqpoint{3.053106in}{2.092349in}}%
\pgfpathlineto{\pgfqpoint{3.066890in}{2.080926in}}%
\pgfpathlineto{\pgfqpoint{3.080673in}{2.069634in}}%
\pgfpathlineto{\pgfqpoint{3.094457in}{2.058472in}}%
\pgfpathlineto{\pgfqpoint{3.108240in}{2.047440in}}%
\pgfpathlineto{\pgfqpoint{3.099692in}{2.048395in}}%
\pgfpathlineto{\pgfqpoint{3.091130in}{2.049609in}}%
\pgfpathlineto{\pgfqpoint{3.082553in}{2.051084in}}%
\pgfpathlineto{\pgfqpoint{3.073961in}{2.052828in}}%
\pgfpathlineto{\pgfqpoint{3.060141in}{2.064358in}}%
\pgfpathlineto{\pgfqpoint{3.046321in}{2.076017in}}%
\pgfpathlineto{\pgfqpoint{3.032500in}{2.087807in}}%
\pgfpathlineto{\pgfqpoint{3.018679in}{2.099728in}}%
\pgfpathlineto{\pgfqpoint{3.027308in}{2.097479in}}%
\pgfpathlineto{\pgfqpoint{3.035923in}{2.095503in}}%
\pgfpathlineto{\pgfqpoint{3.044522in}{2.093794in}}%
\pgfpathlineto{\pgfqpoint{3.053106in}{2.092349in}}%
\pgfpathclose%
\pgfusepath{fill}%
\end{pgfscope}%
\begin{pgfscope}%
\pgfpathrectangle{\pgfqpoint{1.150000in}{0.150000in}}{\pgfqpoint{5.700000in}{5.700000in}}%
\pgfusepath{clip}%
\pgfsetbuttcap%
\pgfsetroundjoin%
\definecolor{currentfill}{rgb}{0.268510,0.009605,0.335427}%
\pgfsetfillcolor{currentfill}%
\pgfsetfillopacity{0.700000}%
\pgfsetlinewidth{0.000000pt}%
\definecolor{currentstroke}{rgb}{0.000000,0.000000,0.000000}%
\pgfsetstrokecolor{currentstroke}%
\pgfsetdash{}{0pt}%
\pgfpathmoveto{\pgfqpoint{3.704089in}{1.786658in}}%
\pgfpathlineto{\pgfqpoint{3.717889in}{1.781330in}}%
\pgfpathlineto{\pgfqpoint{3.731692in}{1.776110in}}%
\pgfpathlineto{\pgfqpoint{3.745501in}{1.770997in}}%
\pgfpathlineto{\pgfqpoint{3.759316in}{1.765993in}}%
\pgfpathlineto{\pgfqpoint{3.751163in}{1.759989in}}%
\pgfpathlineto{\pgfqpoint{3.743003in}{1.754137in}}%
\pgfpathlineto{\pgfqpoint{3.734835in}{1.748443in}}%
\pgfpathlineto{\pgfqpoint{3.726659in}{1.742911in}}%
\pgfpathlineto{\pgfqpoint{3.712825in}{1.748345in}}%
\pgfpathlineto{\pgfqpoint{3.698996in}{1.753886in}}%
\pgfpathlineto{\pgfqpoint{3.685172in}{1.759536in}}%
\pgfpathlineto{\pgfqpoint{3.671353in}{1.765293in}}%
\pgfpathlineto{\pgfqpoint{3.679550in}{1.770389in}}%
\pgfpathlineto{\pgfqpoint{3.687738in}{1.775652in}}%
\pgfpathlineto{\pgfqpoint{3.695918in}{1.781076in}}%
\pgfpathlineto{\pgfqpoint{3.704089in}{1.786658in}}%
\pgfpathclose%
\pgfusepath{fill}%
\end{pgfscope}%
\begin{pgfscope}%
\pgfpathrectangle{\pgfqpoint{1.150000in}{0.150000in}}{\pgfqpoint{5.700000in}{5.700000in}}%
\pgfusepath{clip}%
\pgfsetbuttcap%
\pgfsetroundjoin%
\definecolor{currentfill}{rgb}{0.271305,0.019942,0.347269}%
\pgfsetfillcolor{currentfill}%
\pgfsetfillopacity{0.700000}%
\pgfsetlinewidth{0.000000pt}%
\definecolor{currentstroke}{rgb}{0.000000,0.000000,0.000000}%
\pgfsetstrokecolor{currentstroke}%
\pgfsetdash{}{0pt}%
\pgfpathmoveto{\pgfqpoint{4.077659in}{1.801634in}}%
\pgfpathlineto{\pgfqpoint{4.091536in}{1.799588in}}%
\pgfpathlineto{\pgfqpoint{4.105421in}{1.797644in}}%
\pgfpathlineto{\pgfqpoint{4.119313in}{1.795803in}}%
\pgfpathlineto{\pgfqpoint{4.133213in}{1.794064in}}%
\pgfpathlineto{\pgfqpoint{4.125212in}{1.784698in}}%
\pgfpathlineto{\pgfqpoint{4.117205in}{1.775415in}}%
\pgfpathlineto{\pgfqpoint{4.109192in}{1.766220in}}%
\pgfpathlineto{\pgfqpoint{4.101173in}{1.757116in}}%
\pgfpathlineto{\pgfqpoint{4.087262in}{1.759229in}}%
\pgfpathlineto{\pgfqpoint{4.073359in}{1.761445in}}%
\pgfpathlineto{\pgfqpoint{4.059463in}{1.763763in}}%
\pgfpathlineto{\pgfqpoint{4.045574in}{1.766184in}}%
\pgfpathlineto{\pgfqpoint{4.053604in}{1.774907in}}%
\pgfpathlineto{\pgfqpoint{4.061628in}{1.783725in}}%
\pgfpathlineto{\pgfqpoint{4.069646in}{1.792636in}}%
\pgfpathlineto{\pgfqpoint{4.077659in}{1.801634in}}%
\pgfpathclose%
\pgfusepath{fill}%
\end{pgfscope}%
\begin{pgfscope}%
\pgfpathrectangle{\pgfqpoint{1.150000in}{0.150000in}}{\pgfqpoint{5.700000in}{5.700000in}}%
\pgfusepath{clip}%
\pgfsetbuttcap%
\pgfsetroundjoin%
\definecolor{currentfill}{rgb}{0.229739,0.322361,0.545706}%
\pgfsetfillcolor{currentfill}%
\pgfsetfillopacity{0.700000}%
\pgfsetlinewidth{0.000000pt}%
\definecolor{currentstroke}{rgb}{0.000000,0.000000,0.000000}%
\pgfsetstrokecolor{currentstroke}%
\pgfsetdash{}{0pt}%
\pgfpathmoveto{\pgfqpoint{5.072410in}{2.420671in}}%
\pgfpathlineto{\pgfqpoint{5.086693in}{2.426142in}}%
\pgfpathlineto{\pgfqpoint{5.100990in}{2.431712in}}%
\pgfpathlineto{\pgfqpoint{5.115300in}{2.437382in}}%
\pgfpathlineto{\pgfqpoint{5.129623in}{2.443152in}}%
\pgfpathlineto{\pgfqpoint{5.121919in}{2.431366in}}%
\pgfpathlineto{\pgfqpoint{5.114208in}{2.419497in}}%
\pgfpathlineto{\pgfqpoint{5.106491in}{2.407547in}}%
\pgfpathlineto{\pgfqpoint{5.098768in}{2.395518in}}%
\pgfpathlineto{\pgfqpoint{5.084443in}{2.389908in}}%
\pgfpathlineto{\pgfqpoint{5.070131in}{2.384398in}}%
\pgfpathlineto{\pgfqpoint{5.055833in}{2.378987in}}%
\pgfpathlineto{\pgfqpoint{5.041548in}{2.373676in}}%
\pgfpathlineto{\pgfqpoint{5.049272in}{2.385539in}}%
\pgfpathlineto{\pgfqpoint{5.056991in}{2.397327in}}%
\pgfpathlineto{\pgfqpoint{5.064704in}{2.409038in}}%
\pgfpathlineto{\pgfqpoint{5.072410in}{2.420671in}}%
\pgfpathclose%
\pgfusepath{fill}%
\end{pgfscope}%
\begin{pgfscope}%
\pgfpathrectangle{\pgfqpoint{1.150000in}{0.150000in}}{\pgfqpoint{5.700000in}{5.700000in}}%
\pgfusepath{clip}%
\pgfsetbuttcap%
\pgfsetroundjoin%
\definecolor{currentfill}{rgb}{0.267004,0.004874,0.329415}%
\pgfsetfillcolor{currentfill}%
\pgfsetfillopacity{0.700000}%
\pgfsetlinewidth{0.000000pt}%
\definecolor{currentstroke}{rgb}{0.000000,0.000000,0.000000}%
\pgfsetstrokecolor{currentstroke}%
\pgfsetdash{}{0pt}%
\pgfpathmoveto{\pgfqpoint{3.847089in}{1.774151in}}%
\pgfpathlineto{\pgfqpoint{3.860913in}{1.770089in}}%
\pgfpathlineto{\pgfqpoint{3.874744in}{1.766134in}}%
\pgfpathlineto{\pgfqpoint{3.888580in}{1.762283in}}%
\pgfpathlineto{\pgfqpoint{3.902422in}{1.758538in}}%
\pgfpathlineto{\pgfqpoint{3.894333in}{1.751155in}}%
\pgfpathlineto{\pgfqpoint{3.886237in}{1.743900in}}%
\pgfpathlineto{\pgfqpoint{3.878134in}{1.736776in}}%
\pgfpathlineto{\pgfqpoint{3.870024in}{1.729788in}}%
\pgfpathlineto{\pgfqpoint{3.856166in}{1.733944in}}%
\pgfpathlineto{\pgfqpoint{3.842313in}{1.738205in}}%
\pgfpathlineto{\pgfqpoint{3.828466in}{1.742571in}}%
\pgfpathlineto{\pgfqpoint{3.814625in}{1.747043in}}%
\pgfpathlineto{\pgfqpoint{3.822752in}{1.753614in}}%
\pgfpathlineto{\pgfqpoint{3.830872in}{1.760325in}}%
\pgfpathlineto{\pgfqpoint{3.838984in}{1.767172in}}%
\pgfpathlineto{\pgfqpoint{3.847089in}{1.774151in}}%
\pgfpathclose%
\pgfusepath{fill}%
\end{pgfscope}%
\begin{pgfscope}%
\pgfpathrectangle{\pgfqpoint{1.150000in}{0.150000in}}{\pgfqpoint{5.700000in}{5.700000in}}%
\pgfusepath{clip}%
\pgfsetbuttcap%
\pgfsetroundjoin%
\definecolor{currentfill}{rgb}{0.270595,0.214069,0.507052}%
\pgfsetfillcolor{currentfill}%
\pgfsetfillopacity{0.700000}%
\pgfsetlinewidth{0.000000pt}%
\definecolor{currentstroke}{rgb}{0.000000,0.000000,0.000000}%
\pgfsetstrokecolor{currentstroke}%
\pgfsetdash{}{0pt}%
\pgfpathmoveto{\pgfqpoint{4.777925in}{2.177143in}}%
\pgfpathlineto{\pgfqpoint{4.792065in}{2.180670in}}%
\pgfpathlineto{\pgfqpoint{4.806217in}{2.184297in}}%
\pgfpathlineto{\pgfqpoint{4.820381in}{2.188023in}}%
\pgfpathlineto{\pgfqpoint{4.834557in}{2.191848in}}%
\pgfpathlineto{\pgfqpoint{4.826757in}{2.179609in}}%
\pgfpathlineto{\pgfqpoint{4.818952in}{2.167327in}}%
\pgfpathlineto{\pgfqpoint{4.811142in}{2.155005in}}%
\pgfpathlineto{\pgfqpoint{4.803327in}{2.142646in}}%
\pgfpathlineto{\pgfqpoint{4.789149in}{2.139054in}}%
\pgfpathlineto{\pgfqpoint{4.774984in}{2.135561in}}%
\pgfpathlineto{\pgfqpoint{4.760830in}{2.132167in}}%
\pgfpathlineto{\pgfqpoint{4.746687in}{2.128873in}}%
\pgfpathlineto{\pgfqpoint{4.754504in}{2.140992in}}%
\pgfpathlineto{\pgfqpoint{4.762316in}{2.153078in}}%
\pgfpathlineto{\pgfqpoint{4.770124in}{2.165129in}}%
\pgfpathlineto{\pgfqpoint{4.777925in}{2.177143in}}%
\pgfpathclose%
\pgfusepath{fill}%
\end{pgfscope}%
\begin{pgfscope}%
\pgfpathrectangle{\pgfqpoint{1.150000in}{0.150000in}}{\pgfqpoint{5.700000in}{5.700000in}}%
\pgfusepath{clip}%
\pgfsetbuttcap%
\pgfsetroundjoin%
\definecolor{currentfill}{rgb}{0.183898,0.422383,0.556944}%
\pgfsetfillcolor{currentfill}%
\pgfsetfillopacity{0.700000}%
\pgfsetlinewidth{0.000000pt}%
\definecolor{currentstroke}{rgb}{0.000000,0.000000,0.000000}%
\pgfsetstrokecolor{currentstroke}%
\pgfsetdash{}{0pt}%
\pgfpathmoveto{\pgfqpoint{5.367020in}{2.673008in}}%
\pgfpathlineto{\pgfqpoint{5.381462in}{2.680127in}}%
\pgfpathlineto{\pgfqpoint{5.395918in}{2.687348in}}%
\pgfpathlineto{\pgfqpoint{5.410389in}{2.694668in}}%
\pgfpathlineto{\pgfqpoint{5.424876in}{2.702089in}}%
\pgfpathlineto{\pgfqpoint{5.417289in}{2.691528in}}%
\pgfpathlineto{\pgfqpoint{5.409694in}{2.680857in}}%
\pgfpathlineto{\pgfqpoint{5.402092in}{2.670075in}}%
\pgfpathlineto{\pgfqpoint{5.394483in}{2.659182in}}%
\pgfpathlineto{\pgfqpoint{5.379993in}{2.651845in}}%
\pgfpathlineto{\pgfqpoint{5.365518in}{2.644609in}}%
\pgfpathlineto{\pgfqpoint{5.351058in}{2.637472in}}%
\pgfpathlineto{\pgfqpoint{5.336614in}{2.630436in}}%
\pgfpathlineto{\pgfqpoint{5.344226in}{2.641237in}}%
\pgfpathlineto{\pgfqpoint{5.351832in}{2.651933in}}%
\pgfpathlineto{\pgfqpoint{5.359430in}{2.662523in}}%
\pgfpathlineto{\pgfqpoint{5.367020in}{2.673008in}}%
\pgfpathclose%
\pgfusepath{fill}%
\end{pgfscope}%
\begin{pgfscope}%
\pgfpathrectangle{\pgfqpoint{1.150000in}{0.150000in}}{\pgfqpoint{5.700000in}{5.700000in}}%
\pgfusepath{clip}%
\pgfsetbuttcap%
\pgfsetroundjoin%
\definecolor{currentfill}{rgb}{0.146180,0.515413,0.556823}%
\pgfsetfillcolor{currentfill}%
\pgfsetfillopacity{0.700000}%
\pgfsetlinewidth{0.000000pt}%
\definecolor{currentstroke}{rgb}{0.000000,0.000000,0.000000}%
\pgfsetstrokecolor{currentstroke}%
\pgfsetdash{}{0pt}%
\pgfpathmoveto{\pgfqpoint{5.661474in}{2.919847in}}%
\pgfpathlineto{\pgfqpoint{5.676080in}{2.928314in}}%
\pgfpathlineto{\pgfqpoint{5.690702in}{2.936883in}}%
\pgfpathlineto{\pgfqpoint{5.705340in}{2.945552in}}%
\pgfpathlineto{\pgfqpoint{5.719995in}{2.954322in}}%
\pgfpathlineto{\pgfqpoint{5.712558in}{2.945573in}}%
\pgfpathlineto{\pgfqpoint{5.705111in}{2.936696in}}%
\pgfpathlineto{\pgfqpoint{5.697655in}{2.927691in}}%
\pgfpathlineto{\pgfqpoint{5.690190in}{2.918559in}}%
\pgfpathlineto{\pgfqpoint{5.675528in}{2.909794in}}%
\pgfpathlineto{\pgfqpoint{5.660883in}{2.901131in}}%
\pgfpathlineto{\pgfqpoint{5.646254in}{2.892568in}}%
\pgfpathlineto{\pgfqpoint{5.631641in}{2.884106in}}%
\pgfpathlineto{\pgfqpoint{5.639112in}{2.893225in}}%
\pgfpathlineto{\pgfqpoint{5.646575in}{2.902222in}}%
\pgfpathlineto{\pgfqpoint{5.654029in}{2.911096in}}%
\pgfpathlineto{\pgfqpoint{5.661474in}{2.919847in}}%
\pgfpathclose%
\pgfusepath{fill}%
\end{pgfscope}%
\begin{pgfscope}%
\pgfpathrectangle{\pgfqpoint{1.150000in}{0.150000in}}{\pgfqpoint{5.700000in}{5.700000in}}%
\pgfusepath{clip}%
\pgfsetbuttcap%
\pgfsetroundjoin%
\definecolor{currentfill}{rgb}{0.280267,0.073417,0.397163}%
\pgfsetfillcolor{currentfill}%
\pgfsetfillopacity{0.700000}%
\pgfsetlinewidth{0.000000pt}%
\definecolor{currentstroke}{rgb}{0.000000,0.000000,0.000000}%
\pgfsetstrokecolor{currentstroke}%
\pgfsetdash{}{0pt}%
\pgfpathmoveto{\pgfqpoint{3.362433in}{1.894191in}}%
\pgfpathlineto{\pgfqpoint{3.376207in}{1.885727in}}%
\pgfpathlineto{\pgfqpoint{3.389983in}{1.877379in}}%
\pgfpathlineto{\pgfqpoint{3.403762in}{1.869149in}}%
\pgfpathlineto{\pgfqpoint{3.417543in}{1.861036in}}%
\pgfpathlineto{\pgfqpoint{3.409202in}{1.858684in}}%
\pgfpathlineto{\pgfqpoint{3.400850in}{1.856545in}}%
\pgfpathlineto{\pgfqpoint{3.392487in}{1.854623in}}%
\pgfpathlineto{\pgfqpoint{3.384112in}{1.852923in}}%
\pgfpathlineto{\pgfqpoint{3.370303in}{1.861507in}}%
\pgfpathlineto{\pgfqpoint{3.356496in}{1.870208in}}%
\pgfpathlineto{\pgfqpoint{3.342691in}{1.879026in}}%
\pgfpathlineto{\pgfqpoint{3.328888in}{1.887961in}}%
\pgfpathlineto{\pgfqpoint{3.337292in}{1.889183in}}%
\pgfpathlineto{\pgfqpoint{3.345684in}{1.890631in}}%
\pgfpathlineto{\pgfqpoint{3.354064in}{1.892302in}}%
\pgfpathlineto{\pgfqpoint{3.362433in}{1.894191in}}%
\pgfpathclose%
\pgfusepath{fill}%
\end{pgfscope}%
\begin{pgfscope}%
\pgfpathrectangle{\pgfqpoint{1.150000in}{0.150000in}}{\pgfqpoint{5.700000in}{5.700000in}}%
\pgfusepath{clip}%
\pgfsetbuttcap%
\pgfsetroundjoin%
\definecolor{currentfill}{rgb}{0.272594,0.025563,0.353093}%
\pgfsetfillcolor{currentfill}%
\pgfsetfillopacity{0.700000}%
\pgfsetlinewidth{0.000000pt}%
\definecolor{currentstroke}{rgb}{0.000000,0.000000,0.000000}%
\pgfsetstrokecolor{currentstroke}%
\pgfsetdash{}{0pt}%
\pgfpathmoveto{\pgfqpoint{3.560956in}{1.815301in}}%
\pgfpathlineto{\pgfqpoint{3.574741in}{1.808663in}}%
\pgfpathlineto{\pgfqpoint{3.588530in}{1.802136in}}%
\pgfpathlineto{\pgfqpoint{3.602324in}{1.795720in}}%
\pgfpathlineto{\pgfqpoint{3.616121in}{1.789415in}}%
\pgfpathlineto{\pgfqpoint{3.607894in}{1.784931in}}%
\pgfpathlineto{\pgfqpoint{3.599658in}{1.780628in}}%
\pgfpathlineto{\pgfqpoint{3.591413in}{1.776508in}}%
\pgfpathlineto{\pgfqpoint{3.583159in}{1.772577in}}%
\pgfpathlineto{\pgfqpoint{3.569338in}{1.779331in}}%
\pgfpathlineto{\pgfqpoint{3.555521in}{1.786196in}}%
\pgfpathlineto{\pgfqpoint{3.541709in}{1.793172in}}%
\pgfpathlineto{\pgfqpoint{3.527899in}{1.800260in}}%
\pgfpathlineto{\pgfqpoint{3.536178in}{1.803734in}}%
\pgfpathlineto{\pgfqpoint{3.544447in}{1.807402in}}%
\pgfpathlineto{\pgfqpoint{3.552706in}{1.811260in}}%
\pgfpathlineto{\pgfqpoint{3.560956in}{1.815301in}}%
\pgfpathclose%
\pgfusepath{fill}%
\end{pgfscope}%
\begin{pgfscope}%
\pgfpathrectangle{\pgfqpoint{1.150000in}{0.150000in}}{\pgfqpoint{5.700000in}{5.700000in}}%
\pgfusepath{clip}%
\pgfsetbuttcap%
\pgfsetroundjoin%
\definecolor{currentfill}{rgb}{0.281887,0.150881,0.465405}%
\pgfsetfillcolor{currentfill}%
\pgfsetfillopacity{0.700000}%
\pgfsetlinewidth{0.000000pt}%
\definecolor{currentstroke}{rgb}{0.000000,0.000000,0.000000}%
\pgfsetstrokecolor{currentstroke}%
\pgfsetdash{}{0pt}%
\pgfpathmoveto{\pgfqpoint{3.108240in}{2.047440in}}%
\pgfpathlineto{\pgfqpoint{3.122024in}{2.036536in}}%
\pgfpathlineto{\pgfqpoint{3.135808in}{2.025761in}}%
\pgfpathlineto{\pgfqpoint{3.149593in}{2.015113in}}%
\pgfpathlineto{\pgfqpoint{3.163378in}{2.004591in}}%
\pgfpathlineto{\pgfqpoint{3.154865in}{2.005058in}}%
\pgfpathlineto{\pgfqpoint{3.146338in}{2.005778in}}%
\pgfpathlineto{\pgfqpoint{3.137798in}{2.006755in}}%
\pgfpathlineto{\pgfqpoint{3.129243in}{2.007995in}}%
\pgfpathlineto{\pgfqpoint{3.115422in}{2.019012in}}%
\pgfpathlineto{\pgfqpoint{3.101602in}{2.030156in}}%
\pgfpathlineto{\pgfqpoint{3.087781in}{2.041428in}}%
\pgfpathlineto{\pgfqpoint{3.073961in}{2.052828in}}%
\pgfpathlineto{\pgfqpoint{3.082553in}{2.051084in}}%
\pgfpathlineto{\pgfqpoint{3.091130in}{2.049609in}}%
\pgfpathlineto{\pgfqpoint{3.099692in}{2.048395in}}%
\pgfpathlineto{\pgfqpoint{3.108240in}{2.047440in}}%
\pgfpathclose%
\pgfusepath{fill}%
\end{pgfscope}%
\begin{pgfscope}%
\pgfpathrectangle{\pgfqpoint{1.150000in}{0.150000in}}{\pgfqpoint{5.700000in}{5.700000in}}%
\pgfusepath{clip}%
\pgfsetbuttcap%
\pgfsetroundjoin%
\definecolor{currentfill}{rgb}{0.260571,0.246922,0.522828}%
\pgfsetfillcolor{currentfill}%
\pgfsetfillopacity{0.700000}%
\pgfsetlinewidth{0.000000pt}%
\definecolor{currentstroke}{rgb}{0.000000,0.000000,0.000000}%
\pgfsetstrokecolor{currentstroke}%
\pgfsetdash{}{0pt}%
\pgfpathmoveto{\pgfqpoint{4.865704in}{2.240349in}}%
\pgfpathlineto{\pgfqpoint{4.879890in}{2.244489in}}%
\pgfpathlineto{\pgfqpoint{4.894088in}{2.248728in}}%
\pgfpathlineto{\pgfqpoint{4.908299in}{2.253067in}}%
\pgfpathlineto{\pgfqpoint{4.922522in}{2.257505in}}%
\pgfpathlineto{\pgfqpoint{4.914745in}{2.245244in}}%
\pgfpathlineto{\pgfqpoint{4.906963in}{2.232929in}}%
\pgfpathlineto{\pgfqpoint{4.899175in}{2.220562in}}%
\pgfpathlineto{\pgfqpoint{4.891382in}{2.208144in}}%
\pgfpathlineto{\pgfqpoint{4.877157in}{2.203921in}}%
\pgfpathlineto{\pgfqpoint{4.862945in}{2.199797in}}%
\pgfpathlineto{\pgfqpoint{4.848745in}{2.195773in}}%
\pgfpathlineto{\pgfqpoint{4.834557in}{2.191848in}}%
\pgfpathlineto{\pgfqpoint{4.842351in}{2.204044in}}%
\pgfpathlineto{\pgfqpoint{4.850141in}{2.216194in}}%
\pgfpathlineto{\pgfqpoint{4.857925in}{2.228297in}}%
\pgfpathlineto{\pgfqpoint{4.865704in}{2.240349in}}%
\pgfpathclose%
\pgfusepath{fill}%
\end{pgfscope}%
\begin{pgfscope}%
\pgfpathrectangle{\pgfqpoint{1.150000in}{0.150000in}}{\pgfqpoint{5.700000in}{5.700000in}}%
\pgfusepath{clip}%
\pgfsetbuttcap%
\pgfsetroundjoin%
\definecolor{currentfill}{rgb}{0.268510,0.009605,0.335427}%
\pgfsetfillcolor{currentfill}%
\pgfsetfillopacity{0.700000}%
\pgfsetlinewidth{0.000000pt}%
\definecolor{currentstroke}{rgb}{0.000000,0.000000,0.000000}%
\pgfsetstrokecolor{currentstroke}%
\pgfsetdash{}{0pt}%
\pgfpathmoveto{\pgfqpoint{3.990089in}{1.776894in}}%
\pgfpathlineto{\pgfqpoint{4.003949in}{1.774062in}}%
\pgfpathlineto{\pgfqpoint{4.017817in}{1.771333in}}%
\pgfpathlineto{\pgfqpoint{4.031692in}{1.768707in}}%
\pgfpathlineto{\pgfqpoint{4.045574in}{1.766184in}}%
\pgfpathlineto{\pgfqpoint{4.037538in}{1.757560in}}%
\pgfpathlineto{\pgfqpoint{4.029496in}{1.749038in}}%
\pgfpathlineto{\pgfqpoint{4.021447in}{1.740624in}}%
\pgfpathlineto{\pgfqpoint{4.013393in}{1.732319in}}%
\pgfpathlineto{\pgfqpoint{3.999499in}{1.735235in}}%
\pgfpathlineto{\pgfqpoint{3.985611in}{1.738253in}}%
\pgfpathlineto{\pgfqpoint{3.971730in}{1.741375in}}%
\pgfpathlineto{\pgfqpoint{3.957855in}{1.744600in}}%
\pgfpathlineto{\pgfqpoint{3.965923in}{1.752505in}}%
\pgfpathlineto{\pgfqpoint{3.973985in}{1.760525in}}%
\pgfpathlineto{\pgfqpoint{3.982040in}{1.768656in}}%
\pgfpathlineto{\pgfqpoint{3.990089in}{1.776894in}}%
\pgfpathclose%
\pgfusepath{fill}%
\end{pgfscope}%
\begin{pgfscope}%
\pgfpathrectangle{\pgfqpoint{1.150000in}{0.150000in}}{\pgfqpoint{5.700000in}{5.700000in}}%
\pgfusepath{clip}%
\pgfsetbuttcap%
\pgfsetroundjoin%
\definecolor{currentfill}{rgb}{0.214298,0.355619,0.551184}%
\pgfsetfillcolor{currentfill}%
\pgfsetfillopacity{0.700000}%
\pgfsetlinewidth{0.000000pt}%
\definecolor{currentstroke}{rgb}{0.000000,0.000000,0.000000}%
\pgfsetstrokecolor{currentstroke}%
\pgfsetdash{}{0pt}%
\pgfpathmoveto{\pgfqpoint{5.160379in}{2.489448in}}%
\pgfpathlineto{\pgfqpoint{5.174715in}{2.495459in}}%
\pgfpathlineto{\pgfqpoint{5.189064in}{2.501569in}}%
\pgfpathlineto{\pgfqpoint{5.203427in}{2.507780in}}%
\pgfpathlineto{\pgfqpoint{5.217805in}{2.514090in}}%
\pgfpathlineto{\pgfqpoint{5.210127in}{2.502511in}}%
\pgfpathlineto{\pgfqpoint{5.202444in}{2.490841in}}%
\pgfpathlineto{\pgfqpoint{5.194753in}{2.479079in}}%
\pgfpathlineto{\pgfqpoint{5.187057in}{2.467229in}}%
\pgfpathlineto{\pgfqpoint{5.172677in}{2.461060in}}%
\pgfpathlineto{\pgfqpoint{5.158312in}{2.454991in}}%
\pgfpathlineto{\pgfqpoint{5.143961in}{2.449021in}}%
\pgfpathlineto{\pgfqpoint{5.129623in}{2.443152in}}%
\pgfpathlineto{\pgfqpoint{5.137322in}{2.454854in}}%
\pgfpathlineto{\pgfqpoint{5.145014in}{2.466472in}}%
\pgfpathlineto{\pgfqpoint{5.152700in}{2.478004in}}%
\pgfpathlineto{\pgfqpoint{5.160379in}{2.489448in}}%
\pgfpathclose%
\pgfusepath{fill}%
\end{pgfscope}%
\begin{pgfscope}%
\pgfpathrectangle{\pgfqpoint{1.150000in}{0.150000in}}{\pgfqpoint{5.700000in}{5.700000in}}%
\pgfusepath{clip}%
\pgfsetbuttcap%
\pgfsetroundjoin%
\definecolor{currentfill}{rgb}{0.136408,0.541173,0.554483}%
\pgfsetfillcolor{currentfill}%
\pgfsetfillopacity{0.700000}%
\pgfsetlinewidth{0.000000pt}%
\definecolor{currentstroke}{rgb}{0.000000,0.000000,0.000000}%
\pgfsetstrokecolor{currentstroke}%
\pgfsetdash{}{0pt}%
\pgfpathmoveto{\pgfqpoint{5.749656in}{2.988056in}}%
\pgfpathlineto{\pgfqpoint{5.764319in}{2.996913in}}%
\pgfpathlineto{\pgfqpoint{5.779000in}{3.005871in}}%
\pgfpathlineto{\pgfqpoint{5.793697in}{3.014930in}}%
\pgfpathlineto{\pgfqpoint{5.808410in}{3.024091in}}%
\pgfpathlineto{\pgfqpoint{5.801018in}{3.015868in}}%
\pgfpathlineto{\pgfqpoint{5.793615in}{3.007515in}}%
\pgfpathlineto{\pgfqpoint{5.786203in}{2.999031in}}%
\pgfpathlineto{\pgfqpoint{5.778782in}{2.990416in}}%
\pgfpathlineto{\pgfqpoint{5.764060in}{2.981240in}}%
\pgfpathlineto{\pgfqpoint{5.749355in}{2.972167in}}%
\pgfpathlineto{\pgfqpoint{5.734667in}{2.963194in}}%
\pgfpathlineto{\pgfqpoint{5.719995in}{2.954322in}}%
\pgfpathlineto{\pgfqpoint{5.727424in}{2.962945in}}%
\pgfpathlineto{\pgfqpoint{5.734844in}{2.971441in}}%
\pgfpathlineto{\pgfqpoint{5.742254in}{2.979811in}}%
\pgfpathlineto{\pgfqpoint{5.749656in}{2.988056in}}%
\pgfpathclose%
\pgfusepath{fill}%
\end{pgfscope}%
\begin{pgfscope}%
\pgfpathrectangle{\pgfqpoint{1.150000in}{0.150000in}}{\pgfqpoint{5.700000in}{5.700000in}}%
\pgfusepath{clip}%
\pgfsetbuttcap%
\pgfsetroundjoin%
\definecolor{currentfill}{rgb}{0.171176,0.452530,0.557965}%
\pgfsetfillcolor{currentfill}%
\pgfsetfillopacity{0.700000}%
\pgfsetlinewidth{0.000000pt}%
\definecolor{currentstroke}{rgb}{0.000000,0.000000,0.000000}%
\pgfsetstrokecolor{currentstroke}%
\pgfsetdash{}{0pt}%
\pgfpathmoveto{\pgfqpoint{5.455147in}{2.743213in}}%
\pgfpathlineto{\pgfqpoint{5.469644in}{2.750799in}}%
\pgfpathlineto{\pgfqpoint{5.484157in}{2.758486in}}%
\pgfpathlineto{\pgfqpoint{5.498686in}{2.766273in}}%
\pgfpathlineto{\pgfqpoint{5.513230in}{2.774160in}}%
\pgfpathlineto{\pgfqpoint{5.505678in}{2.763991in}}%
\pgfpathlineto{\pgfqpoint{5.498118in}{2.753703in}}%
\pgfpathlineto{\pgfqpoint{5.490551in}{2.743299in}}%
\pgfpathlineto{\pgfqpoint{5.482975in}{2.732778in}}%
\pgfpathlineto{\pgfqpoint{5.468427in}{2.724955in}}%
\pgfpathlineto{\pgfqpoint{5.453895in}{2.717232in}}%
\pgfpathlineto{\pgfqpoint{5.439377in}{2.709611in}}%
\pgfpathlineto{\pgfqpoint{5.424876in}{2.702089in}}%
\pgfpathlineto{\pgfqpoint{5.432455in}{2.712538in}}%
\pgfpathlineto{\pgfqpoint{5.440027in}{2.722876in}}%
\pgfpathlineto{\pgfqpoint{5.447591in}{2.733101in}}%
\pgfpathlineto{\pgfqpoint{5.455147in}{2.743213in}}%
\pgfpathclose%
\pgfusepath{fill}%
\end{pgfscope}%
\begin{pgfscope}%
\pgfpathrectangle{\pgfqpoint{1.150000in}{0.150000in}}{\pgfqpoint{5.700000in}{5.700000in}}%
\pgfusepath{clip}%
\pgfsetbuttcap%
\pgfsetroundjoin%
\definecolor{currentfill}{rgb}{0.281924,0.089666,0.412415}%
\pgfsetfillcolor{currentfill}%
\pgfsetfillopacity{0.700000}%
\pgfsetlinewidth{0.000000pt}%
\definecolor{currentstroke}{rgb}{0.000000,0.000000,0.000000}%
\pgfsetstrokecolor{currentstroke}%
\pgfsetdash{}{0pt}%
\pgfpathmoveto{\pgfqpoint{4.395951in}{1.910967in}}%
\pgfpathlineto{\pgfqpoint{4.409939in}{1.911557in}}%
\pgfpathlineto{\pgfqpoint{4.423937in}{1.912247in}}%
\pgfpathlineto{\pgfqpoint{4.437944in}{1.913038in}}%
\pgfpathlineto{\pgfqpoint{4.451961in}{1.913928in}}%
\pgfpathlineto{\pgfqpoint{4.444051in}{1.902562in}}%
\pgfpathlineto{\pgfqpoint{4.436137in}{1.891223in}}%
\pgfpathlineto{\pgfqpoint{4.428218in}{1.879913in}}%
\pgfpathlineto{\pgfqpoint{4.420294in}{1.868636in}}%
\pgfpathlineto{\pgfqpoint{4.406272in}{1.868068in}}%
\pgfpathlineto{\pgfqpoint{4.392259in}{1.867600in}}%
\pgfpathlineto{\pgfqpoint{4.378255in}{1.867231in}}%
\pgfpathlineto{\pgfqpoint{4.364261in}{1.866963in}}%
\pgfpathlineto{\pgfqpoint{4.372191in}{1.877911in}}%
\pgfpathlineto{\pgfqpoint{4.380115in}{1.888897in}}%
\pgfpathlineto{\pgfqpoint{4.388036in}{1.899916in}}%
\pgfpathlineto{\pgfqpoint{4.395951in}{1.910967in}}%
\pgfpathclose%
\pgfusepath{fill}%
\end{pgfscope}%
\begin{pgfscope}%
\pgfpathrectangle{\pgfqpoint{1.150000in}{0.150000in}}{\pgfqpoint{5.700000in}{5.700000in}}%
\pgfusepath{clip}%
\pgfsetbuttcap%
\pgfsetroundjoin%
\definecolor{currentfill}{rgb}{0.283197,0.115680,0.436115}%
\pgfsetfillcolor{currentfill}%
\pgfsetfillopacity{0.700000}%
\pgfsetlinewidth{0.000000pt}%
\definecolor{currentstroke}{rgb}{0.000000,0.000000,0.000000}%
\pgfsetstrokecolor{currentstroke}%
\pgfsetdash{}{0pt}%
\pgfpathmoveto{\pgfqpoint{4.483552in}{1.959596in}}%
\pgfpathlineto{\pgfqpoint{4.497574in}{1.960890in}}%
\pgfpathlineto{\pgfqpoint{4.511606in}{1.962284in}}%
\pgfpathlineto{\pgfqpoint{4.525648in}{1.963777in}}%
\pgfpathlineto{\pgfqpoint{4.539701in}{1.965371in}}%
\pgfpathlineto{\pgfqpoint{4.531815in}{1.953631in}}%
\pgfpathlineto{\pgfqpoint{4.523924in}{1.941901in}}%
\pgfpathlineto{\pgfqpoint{4.516029in}{1.930185in}}%
\pgfpathlineto{\pgfqpoint{4.508129in}{1.918485in}}%
\pgfpathlineto{\pgfqpoint{4.494072in}{1.917196in}}%
\pgfpathlineto{\pgfqpoint{4.480025in}{1.916007in}}%
\pgfpathlineto{\pgfqpoint{4.465988in}{1.914918in}}%
\pgfpathlineto{\pgfqpoint{4.451961in}{1.913928in}}%
\pgfpathlineto{\pgfqpoint{4.459866in}{1.925317in}}%
\pgfpathlineto{\pgfqpoint{4.467766in}{1.936726in}}%
\pgfpathlineto{\pgfqpoint{4.475661in}{1.948154in}}%
\pgfpathlineto{\pgfqpoint{4.483552in}{1.959596in}}%
\pgfpathclose%
\pgfusepath{fill}%
\end{pgfscope}%
\begin{pgfscope}%
\pgfpathrectangle{\pgfqpoint{1.150000in}{0.150000in}}{\pgfqpoint{5.700000in}{5.700000in}}%
\pgfusepath{clip}%
\pgfsetbuttcap%
\pgfsetroundjoin%
\definecolor{currentfill}{rgb}{0.278791,0.062145,0.386592}%
\pgfsetfillcolor{currentfill}%
\pgfsetfillopacity{0.700000}%
\pgfsetlinewidth{0.000000pt}%
\definecolor{currentstroke}{rgb}{0.000000,0.000000,0.000000}%
\pgfsetstrokecolor{currentstroke}%
\pgfsetdash{}{0pt}%
\pgfpathmoveto{\pgfqpoint{4.308376in}{1.866890in}}%
\pgfpathlineto{\pgfqpoint{4.322333in}{1.866758in}}%
\pgfpathlineto{\pgfqpoint{4.336300in}{1.866726in}}%
\pgfpathlineto{\pgfqpoint{4.350276in}{1.866794in}}%
\pgfpathlineto{\pgfqpoint{4.364261in}{1.866963in}}%
\pgfpathlineto{\pgfqpoint{4.356326in}{1.856054in}}%
\pgfpathlineto{\pgfqpoint{4.348386in}{1.845189in}}%
\pgfpathlineto{\pgfqpoint{4.340442in}{1.834370in}}%
\pgfpathlineto{\pgfqpoint{4.332493in}{1.823600in}}%
\pgfpathlineto{\pgfqpoint{4.318501in}{1.823771in}}%
\pgfpathlineto{\pgfqpoint{4.304518in}{1.824043in}}%
\pgfpathlineto{\pgfqpoint{4.290544in}{1.824414in}}%
\pgfpathlineto{\pgfqpoint{4.276579in}{1.824886in}}%
\pgfpathlineto{\pgfqpoint{4.284536in}{1.835309in}}%
\pgfpathlineto{\pgfqpoint{4.292487in}{1.845786in}}%
\pgfpathlineto{\pgfqpoint{4.300434in}{1.856314in}}%
\pgfpathlineto{\pgfqpoint{4.308376in}{1.866890in}}%
\pgfpathclose%
\pgfusepath{fill}%
\end{pgfscope}%
\begin{pgfscope}%
\pgfpathrectangle{\pgfqpoint{1.150000in}{0.150000in}}{\pgfqpoint{5.700000in}{5.700000in}}%
\pgfusepath{clip}%
\pgfsetbuttcap%
\pgfsetroundjoin%
\definecolor{currentfill}{rgb}{0.282884,0.135920,0.453427}%
\pgfsetfillcolor{currentfill}%
\pgfsetfillopacity{0.700000}%
\pgfsetlinewidth{0.000000pt}%
\definecolor{currentstroke}{rgb}{0.000000,0.000000,0.000000}%
\pgfsetstrokecolor{currentstroke}%
\pgfsetdash{}{0pt}%
\pgfpathmoveto{\pgfqpoint{3.163378in}{2.004591in}}%
\pgfpathlineto{\pgfqpoint{3.177164in}{1.994195in}}%
\pgfpathlineto{\pgfqpoint{3.190951in}{1.983925in}}%
\pgfpathlineto{\pgfqpoint{3.204739in}{1.973779in}}%
\pgfpathlineto{\pgfqpoint{3.218528in}{1.963757in}}%
\pgfpathlineto{\pgfqpoint{3.210048in}{1.963737in}}%
\pgfpathlineto{\pgfqpoint{3.201555in}{1.963965in}}%
\pgfpathlineto{\pgfqpoint{3.193049in}{1.964445in}}%
\pgfpathlineto{\pgfqpoint{3.184530in}{1.965183in}}%
\pgfpathlineto{\pgfqpoint{3.170707in}{1.975699in}}%
\pgfpathlineto{\pgfqpoint{3.156885in}{1.986339in}}%
\pgfpathlineto{\pgfqpoint{3.143064in}{1.997104in}}%
\pgfpathlineto{\pgfqpoint{3.129243in}{2.007995in}}%
\pgfpathlineto{\pgfqpoint{3.137798in}{2.006755in}}%
\pgfpathlineto{\pgfqpoint{3.146338in}{2.005778in}}%
\pgfpathlineto{\pgfqpoint{3.154865in}{2.005058in}}%
\pgfpathlineto{\pgfqpoint{3.163378in}{2.004591in}}%
\pgfpathclose%
\pgfusepath{fill}%
\end{pgfscope}%
\begin{pgfscope}%
\pgfpathrectangle{\pgfqpoint{1.150000in}{0.150000in}}{\pgfqpoint{5.700000in}{5.700000in}}%
\pgfusepath{clip}%
\pgfsetbuttcap%
\pgfsetroundjoin%
\definecolor{currentfill}{rgb}{0.278791,0.062145,0.386592}%
\pgfsetfillcolor{currentfill}%
\pgfsetfillopacity{0.700000}%
\pgfsetlinewidth{0.000000pt}%
\definecolor{currentstroke}{rgb}{0.000000,0.000000,0.000000}%
\pgfsetstrokecolor{currentstroke}%
\pgfsetdash{}{0pt}%
\pgfpathmoveto{\pgfqpoint{3.417543in}{1.861036in}}%
\pgfpathlineto{\pgfqpoint{3.431327in}{1.853038in}}%
\pgfpathlineto{\pgfqpoint{3.445114in}{1.845156in}}%
\pgfpathlineto{\pgfqpoint{3.458903in}{1.837389in}}%
\pgfpathlineto{\pgfqpoint{3.472696in}{1.829736in}}%
\pgfpathlineto{\pgfqpoint{3.464382in}{1.826922in}}%
\pgfpathlineto{\pgfqpoint{3.456057in}{1.824315in}}%
\pgfpathlineto{\pgfqpoint{3.447722in}{1.821921in}}%
\pgfpathlineto{\pgfqpoint{3.439376in}{1.819744in}}%
\pgfpathlineto{\pgfqpoint{3.425556in}{1.827866in}}%
\pgfpathlineto{\pgfqpoint{3.411739in}{1.836103in}}%
\pgfpathlineto{\pgfqpoint{3.397924in}{1.844455in}}%
\pgfpathlineto{\pgfqpoint{3.384112in}{1.852923in}}%
\pgfpathlineto{\pgfqpoint{3.392487in}{1.854623in}}%
\pgfpathlineto{\pgfqpoint{3.400850in}{1.856545in}}%
\pgfpathlineto{\pgfqpoint{3.409202in}{1.858684in}}%
\pgfpathlineto{\pgfqpoint{3.417543in}{1.861036in}}%
\pgfpathclose%
\pgfusepath{fill}%
\end{pgfscope}%
\begin{pgfscope}%
\pgfpathrectangle{\pgfqpoint{1.150000in}{0.150000in}}{\pgfqpoint{5.700000in}{5.700000in}}%
\pgfusepath{clip}%
\pgfsetbuttcap%
\pgfsetroundjoin%
\definecolor{currentfill}{rgb}{0.268510,0.009605,0.335427}%
\pgfsetfillcolor{currentfill}%
\pgfsetfillopacity{0.700000}%
\pgfsetlinewidth{0.000000pt}%
\definecolor{currentstroke}{rgb}{0.000000,0.000000,0.000000}%
\pgfsetstrokecolor{currentstroke}%
\pgfsetdash{}{0pt}%
\pgfpathmoveto{\pgfqpoint{3.759316in}{1.765993in}}%
\pgfpathlineto{\pgfqpoint{3.773135in}{1.761096in}}%
\pgfpathlineto{\pgfqpoint{3.786960in}{1.756305in}}%
\pgfpathlineto{\pgfqpoint{3.800790in}{1.751621in}}%
\pgfpathlineto{\pgfqpoint{3.814625in}{1.747043in}}%
\pgfpathlineto{\pgfqpoint{3.806491in}{1.740617in}}%
\pgfpathlineto{\pgfqpoint{3.798349in}{1.734339in}}%
\pgfpathlineto{\pgfqpoint{3.790200in}{1.728213in}}%
\pgfpathlineto{\pgfqpoint{3.782043in}{1.722245in}}%
\pgfpathlineto{\pgfqpoint{3.768189in}{1.727251in}}%
\pgfpathlineto{\pgfqpoint{3.754340in}{1.732365in}}%
\pgfpathlineto{\pgfqpoint{3.740497in}{1.737584in}}%
\pgfpathlineto{\pgfqpoint{3.726659in}{1.742911in}}%
\pgfpathlineto{\pgfqpoint{3.734835in}{1.748443in}}%
\pgfpathlineto{\pgfqpoint{3.743003in}{1.754137in}}%
\pgfpathlineto{\pgfqpoint{3.751163in}{1.759989in}}%
\pgfpathlineto{\pgfqpoint{3.759316in}{1.765993in}}%
\pgfpathclose%
\pgfusepath{fill}%
\end{pgfscope}%
\begin{pgfscope}%
\pgfpathrectangle{\pgfqpoint{1.150000in}{0.150000in}}{\pgfqpoint{5.700000in}{5.700000in}}%
\pgfusepath{clip}%
\pgfsetbuttcap%
\pgfsetroundjoin%
\definecolor{currentfill}{rgb}{0.248629,0.278775,0.534556}%
\pgfsetfillcolor{currentfill}%
\pgfsetfillopacity{0.700000}%
\pgfsetlinewidth{0.000000pt}%
\definecolor{currentstroke}{rgb}{0.000000,0.000000,0.000000}%
\pgfsetstrokecolor{currentstroke}%
\pgfsetdash{}{0pt}%
\pgfpathmoveto{\pgfqpoint{4.953576in}{2.305970in}}%
\pgfpathlineto{\pgfqpoint{4.967810in}{2.310704in}}%
\pgfpathlineto{\pgfqpoint{4.982057in}{2.315538in}}%
\pgfpathlineto{\pgfqpoint{4.996318in}{2.320471in}}%
\pgfpathlineto{\pgfqpoint{5.010591in}{2.325504in}}%
\pgfpathlineto{\pgfqpoint{5.002837in}{2.313288in}}%
\pgfpathlineto{\pgfqpoint{4.995078in}{2.301006in}}%
\pgfpathlineto{\pgfqpoint{4.987314in}{2.288660in}}%
\pgfpathlineto{\pgfqpoint{4.979543in}{2.276251in}}%
\pgfpathlineto{\pgfqpoint{4.965269in}{2.271416in}}%
\pgfpathlineto{\pgfqpoint{4.951007in}{2.266679in}}%
\pgfpathlineto{\pgfqpoint{4.936758in}{2.262043in}}%
\pgfpathlineto{\pgfqpoint{4.922522in}{2.257505in}}%
\pgfpathlineto{\pgfqpoint{4.930294in}{2.269710in}}%
\pgfpathlineto{\pgfqpoint{4.938060in}{2.281857in}}%
\pgfpathlineto{\pgfqpoint{4.945821in}{2.293944in}}%
\pgfpathlineto{\pgfqpoint{4.953576in}{2.305970in}}%
\pgfpathclose%
\pgfusepath{fill}%
\end{pgfscope}%
\begin{pgfscope}%
\pgfpathrectangle{\pgfqpoint{1.150000in}{0.150000in}}{\pgfqpoint{5.700000in}{5.700000in}}%
\pgfusepath{clip}%
\pgfsetbuttcap%
\pgfsetroundjoin%
\definecolor{currentfill}{rgb}{0.282623,0.140926,0.457517}%
\pgfsetfillcolor{currentfill}%
\pgfsetfillopacity{0.700000}%
\pgfsetlinewidth{0.000000pt}%
\definecolor{currentstroke}{rgb}{0.000000,0.000000,0.000000}%
\pgfsetstrokecolor{currentstroke}%
\pgfsetdash{}{0pt}%
\pgfpathmoveto{\pgfqpoint{4.571198in}{2.012379in}}%
\pgfpathlineto{\pgfqpoint{4.585257in}{2.014358in}}%
\pgfpathlineto{\pgfqpoint{4.599327in}{2.016437in}}%
\pgfpathlineto{\pgfqpoint{4.613407in}{2.018615in}}%
\pgfpathlineto{\pgfqpoint{4.627499in}{2.020893in}}%
\pgfpathlineto{\pgfqpoint{4.619635in}{2.008858in}}%
\pgfpathlineto{\pgfqpoint{4.611767in}{1.996819in}}%
\pgfpathlineto{\pgfqpoint{4.603894in}{1.984778in}}%
\pgfpathlineto{\pgfqpoint{4.596016in}{1.972737in}}%
\pgfpathlineto{\pgfqpoint{4.581922in}{1.970746in}}%
\pgfpathlineto{\pgfqpoint{4.567837in}{1.968855in}}%
\pgfpathlineto{\pgfqpoint{4.553764in}{1.967063in}}%
\pgfpathlineto{\pgfqpoint{4.539701in}{1.965371in}}%
\pgfpathlineto{\pgfqpoint{4.547582in}{1.977118in}}%
\pgfpathlineto{\pgfqpoint{4.555459in}{1.988870in}}%
\pgfpathlineto{\pgfqpoint{4.563331in}{2.000625in}}%
\pgfpathlineto{\pgfqpoint{4.571198in}{2.012379in}}%
\pgfpathclose%
\pgfusepath{fill}%
\end{pgfscope}%
\begin{pgfscope}%
\pgfpathrectangle{\pgfqpoint{1.150000in}{0.150000in}}{\pgfqpoint{5.700000in}{5.700000in}}%
\pgfusepath{clip}%
\pgfsetbuttcap%
\pgfsetroundjoin%
\definecolor{currentfill}{rgb}{0.276022,0.044167,0.370164}%
\pgfsetfillcolor{currentfill}%
\pgfsetfillopacity{0.700000}%
\pgfsetlinewidth{0.000000pt}%
\definecolor{currentstroke}{rgb}{0.000000,0.000000,0.000000}%
\pgfsetstrokecolor{currentstroke}%
\pgfsetdash{}{0pt}%
\pgfpathmoveto{\pgfqpoint{4.220805in}{1.827780in}}%
\pgfpathlineto{\pgfqpoint{4.234735in}{1.826905in}}%
\pgfpathlineto{\pgfqpoint{4.248675in}{1.826131in}}%
\pgfpathlineto{\pgfqpoint{4.262623in}{1.825458in}}%
\pgfpathlineto{\pgfqpoint{4.276579in}{1.824886in}}%
\pgfpathlineto{\pgfqpoint{4.268617in}{1.814520in}}%
\pgfpathlineto{\pgfqpoint{4.260650in}{1.804215in}}%
\pgfpathlineto{\pgfqpoint{4.252678in}{1.793974in}}%
\pgfpathlineto{\pgfqpoint{4.244701in}{1.783801in}}%
\pgfpathlineto{\pgfqpoint{4.230736in}{1.784730in}}%
\pgfpathlineto{\pgfqpoint{4.216780in}{1.785761in}}%
\pgfpathlineto{\pgfqpoint{4.202832in}{1.786892in}}%
\pgfpathlineto{\pgfqpoint{4.188892in}{1.788123in}}%
\pgfpathlineto{\pgfqpoint{4.196878in}{1.797933in}}%
\pgfpathlineto{\pgfqpoint{4.204859in}{1.807814in}}%
\pgfpathlineto{\pgfqpoint{4.212834in}{1.817765in}}%
\pgfpathlineto{\pgfqpoint{4.220805in}{1.827780in}}%
\pgfpathclose%
\pgfusepath{fill}%
\end{pgfscope}%
\begin{pgfscope}%
\pgfpathrectangle{\pgfqpoint{1.150000in}{0.150000in}}{\pgfqpoint{5.700000in}{5.700000in}}%
\pgfusepath{clip}%
\pgfsetbuttcap%
\pgfsetroundjoin%
\definecolor{currentfill}{rgb}{0.201239,0.383670,0.554294}%
\pgfsetfillcolor{currentfill}%
\pgfsetfillopacity{0.700000}%
\pgfsetlinewidth{0.000000pt}%
\definecolor{currentstroke}{rgb}{0.000000,0.000000,0.000000}%
\pgfsetstrokecolor{currentstroke}%
\pgfsetdash{}{0pt}%
\pgfpathmoveto{\pgfqpoint{5.248447in}{2.559465in}}%
\pgfpathlineto{\pgfqpoint{5.262837in}{2.565997in}}%
\pgfpathlineto{\pgfqpoint{5.277240in}{2.572630in}}%
\pgfpathlineto{\pgfqpoint{5.291659in}{2.579362in}}%
\pgfpathlineto{\pgfqpoint{5.306092in}{2.586195in}}%
\pgfpathlineto{\pgfqpoint{5.298444in}{2.574879in}}%
\pgfpathlineto{\pgfqpoint{5.290789in}{2.563462in}}%
\pgfpathlineto{\pgfqpoint{5.283127in}{2.551946in}}%
\pgfpathlineto{\pgfqpoint{5.275458in}{2.540330in}}%
\pgfpathlineto{\pgfqpoint{5.261023in}{2.533620in}}%
\pgfpathlineto{\pgfqpoint{5.246603in}{2.527010in}}%
\pgfpathlineto{\pgfqpoint{5.232196in}{2.520500in}}%
\pgfpathlineto{\pgfqpoint{5.217805in}{2.514090in}}%
\pgfpathlineto{\pgfqpoint{5.225475in}{2.525575in}}%
\pgfpathlineto{\pgfqpoint{5.233139in}{2.536967in}}%
\pgfpathlineto{\pgfqpoint{5.240797in}{2.548264in}}%
\pgfpathlineto{\pgfqpoint{5.248447in}{2.559465in}}%
\pgfpathclose%
\pgfusepath{fill}%
\end{pgfscope}%
\begin{pgfscope}%
\pgfpathrectangle{\pgfqpoint{1.150000in}{0.150000in}}{\pgfqpoint{5.700000in}{5.700000in}}%
\pgfusepath{clip}%
\pgfsetbuttcap%
\pgfsetroundjoin%
\definecolor{currentfill}{rgb}{0.271305,0.019942,0.347269}%
\pgfsetfillcolor{currentfill}%
\pgfsetfillopacity{0.700000}%
\pgfsetlinewidth{0.000000pt}%
\definecolor{currentstroke}{rgb}{0.000000,0.000000,0.000000}%
\pgfsetstrokecolor{currentstroke}%
\pgfsetdash{}{0pt}%
\pgfpathmoveto{\pgfqpoint{3.616121in}{1.789415in}}%
\pgfpathlineto{\pgfqpoint{3.629923in}{1.783220in}}%
\pgfpathlineto{\pgfqpoint{3.643728in}{1.777135in}}%
\pgfpathlineto{\pgfqpoint{3.657539in}{1.771160in}}%
\pgfpathlineto{\pgfqpoint{3.671353in}{1.765293in}}%
\pgfpathlineto{\pgfqpoint{3.663148in}{1.760368in}}%
\pgfpathlineto{\pgfqpoint{3.654934in}{1.755618in}}%
\pgfpathlineto{\pgfqpoint{3.646712in}{1.751047in}}%
\pgfpathlineto{\pgfqpoint{3.638480in}{1.746661in}}%
\pgfpathlineto{\pgfqpoint{3.624644in}{1.752976in}}%
\pgfpathlineto{\pgfqpoint{3.610811in}{1.759400in}}%
\pgfpathlineto{\pgfqpoint{3.596983in}{1.765933in}}%
\pgfpathlineto{\pgfqpoint{3.583159in}{1.772577in}}%
\pgfpathlineto{\pgfqpoint{3.591413in}{1.776508in}}%
\pgfpathlineto{\pgfqpoint{3.599658in}{1.780628in}}%
\pgfpathlineto{\pgfqpoint{3.607894in}{1.784931in}}%
\pgfpathlineto{\pgfqpoint{3.616121in}{1.789415in}}%
\pgfpathclose%
\pgfusepath{fill}%
\end{pgfscope}%
\begin{pgfscope}%
\pgfpathrectangle{\pgfqpoint{1.150000in}{0.150000in}}{\pgfqpoint{5.700000in}{5.700000in}}%
\pgfusepath{clip}%
\pgfsetbuttcap%
\pgfsetroundjoin%
\definecolor{currentfill}{rgb}{0.267004,0.004874,0.329415}%
\pgfsetfillcolor{currentfill}%
\pgfsetfillopacity{0.700000}%
\pgfsetlinewidth{0.000000pt}%
\definecolor{currentstroke}{rgb}{0.000000,0.000000,0.000000}%
\pgfsetstrokecolor{currentstroke}%
\pgfsetdash{}{0pt}%
\pgfpathmoveto{\pgfqpoint{3.902422in}{1.758538in}}%
\pgfpathlineto{\pgfqpoint{3.916271in}{1.754897in}}%
\pgfpathlineto{\pgfqpoint{3.930126in}{1.751360in}}%
\pgfpathlineto{\pgfqpoint{3.943987in}{1.747928in}}%
\pgfpathlineto{\pgfqpoint{3.957855in}{1.744600in}}%
\pgfpathlineto{\pgfqpoint{3.949781in}{1.736813in}}%
\pgfpathlineto{\pgfqpoint{3.941700in}{1.729150in}}%
\pgfpathlineto{\pgfqpoint{3.933613in}{1.721613in}}%
\pgfpathlineto{\pgfqpoint{3.925518in}{1.714208in}}%
\pgfpathlineto{\pgfqpoint{3.911636in}{1.717947in}}%
\pgfpathlineto{\pgfqpoint{3.897759in}{1.721790in}}%
\pgfpathlineto{\pgfqpoint{3.883889in}{1.725737in}}%
\pgfpathlineto{\pgfqpoint{3.870024in}{1.729788in}}%
\pgfpathlineto{\pgfqpoint{3.878134in}{1.736776in}}%
\pgfpathlineto{\pgfqpoint{3.886237in}{1.743900in}}%
\pgfpathlineto{\pgfqpoint{3.894333in}{1.751155in}}%
\pgfpathlineto{\pgfqpoint{3.902422in}{1.758538in}}%
\pgfpathclose%
\pgfusepath{fill}%
\end{pgfscope}%
\begin{pgfscope}%
\pgfpathrectangle{\pgfqpoint{1.150000in}{0.150000in}}{\pgfqpoint{5.700000in}{5.700000in}}%
\pgfusepath{clip}%
\pgfsetbuttcap%
\pgfsetroundjoin%
\definecolor{currentfill}{rgb}{0.126453,0.570633,0.549841}%
\pgfsetfillcolor{currentfill}%
\pgfsetfillopacity{0.700000}%
\pgfsetlinewidth{0.000000pt}%
\definecolor{currentstroke}{rgb}{0.000000,0.000000,0.000000}%
\pgfsetstrokecolor{currentstroke}%
\pgfsetdash{}{0pt}%
\pgfpathmoveto{\pgfqpoint{5.837887in}{3.055686in}}%
\pgfpathlineto{\pgfqpoint{5.852609in}{3.064913in}}%
\pgfpathlineto{\pgfqpoint{5.867348in}{3.074242in}}%
\pgfpathlineto{\pgfqpoint{5.882104in}{3.083672in}}%
\pgfpathlineto{\pgfqpoint{5.896877in}{3.093204in}}%
\pgfpathlineto{\pgfqpoint{5.889532in}{3.085540in}}%
\pgfpathlineto{\pgfqpoint{5.882176in}{3.077743in}}%
\pgfpathlineto{\pgfqpoint{5.874811in}{3.069813in}}%
\pgfpathlineto{\pgfqpoint{5.867437in}{3.061748in}}%
\pgfpathlineto{\pgfqpoint{5.852654in}{3.052181in}}%
\pgfpathlineto{\pgfqpoint{5.837889in}{3.042716in}}%
\pgfpathlineto{\pgfqpoint{5.823141in}{3.033353in}}%
\pgfpathlineto{\pgfqpoint{5.808410in}{3.024091in}}%
\pgfpathlineto{\pgfqpoint{5.815794in}{3.032183in}}%
\pgfpathlineto{\pgfqpoint{5.823168in}{3.040146in}}%
\pgfpathlineto{\pgfqpoint{5.830532in}{3.047980in}}%
\pgfpathlineto{\pgfqpoint{5.837887in}{3.055686in}}%
\pgfpathclose%
\pgfusepath{fill}%
\end{pgfscope}%
\begin{pgfscope}%
\pgfpathrectangle{\pgfqpoint{1.150000in}{0.150000in}}{\pgfqpoint{5.700000in}{5.700000in}}%
\pgfusepath{clip}%
\pgfsetbuttcap%
\pgfsetroundjoin%
\definecolor{currentfill}{rgb}{0.279574,0.170599,0.479997}%
\pgfsetfillcolor{currentfill}%
\pgfsetfillopacity{0.700000}%
\pgfsetlinewidth{0.000000pt}%
\definecolor{currentstroke}{rgb}{0.000000,0.000000,0.000000}%
\pgfsetstrokecolor{currentstroke}%
\pgfsetdash{}{0pt}%
\pgfpathmoveto{\pgfqpoint{4.658905in}{2.068929in}}%
\pgfpathlineto{\pgfqpoint{4.673004in}{2.071575in}}%
\pgfpathlineto{\pgfqpoint{4.687115in}{2.074320in}}%
\pgfpathlineto{\pgfqpoint{4.701236in}{2.077165in}}%
\pgfpathlineto{\pgfqpoint{4.715370in}{2.080109in}}%
\pgfpathlineto{\pgfqpoint{4.707528in}{2.067857in}}%
\pgfpathlineto{\pgfqpoint{4.699682in}{2.055586in}}%
\pgfpathlineto{\pgfqpoint{4.691830in}{2.043298in}}%
\pgfpathlineto{\pgfqpoint{4.683974in}{2.030995in}}%
\pgfpathlineto{\pgfqpoint{4.669839in}{2.028321in}}%
\pgfpathlineto{\pgfqpoint{4.655714in}{2.025746in}}%
\pgfpathlineto{\pgfqpoint{4.641601in}{2.023270in}}%
\pgfpathlineto{\pgfqpoint{4.627499in}{2.020893in}}%
\pgfpathlineto{\pgfqpoint{4.635357in}{2.032919in}}%
\pgfpathlineto{\pgfqpoint{4.643211in}{2.044936in}}%
\pgfpathlineto{\pgfqpoint{4.651060in}{2.056940in}}%
\pgfpathlineto{\pgfqpoint{4.658905in}{2.068929in}}%
\pgfpathclose%
\pgfusepath{fill}%
\end{pgfscope}%
\begin{pgfscope}%
\pgfpathrectangle{\pgfqpoint{1.150000in}{0.150000in}}{\pgfqpoint{5.700000in}{5.700000in}}%
\pgfusepath{clip}%
\pgfsetbuttcap%
\pgfsetroundjoin%
\definecolor{currentfill}{rgb}{0.159194,0.482237,0.558073}%
\pgfsetfillcolor{currentfill}%
\pgfsetfillopacity{0.700000}%
\pgfsetlinewidth{0.000000pt}%
\definecolor{currentstroke}{rgb}{0.000000,0.000000,0.000000}%
\pgfsetstrokecolor{currentstroke}%
\pgfsetdash{}{0pt}%
\pgfpathmoveto{\pgfqpoint{5.543356in}{2.813660in}}%
\pgfpathlineto{\pgfqpoint{5.557910in}{2.821693in}}%
\pgfpathlineto{\pgfqpoint{5.572481in}{2.829827in}}%
\pgfpathlineto{\pgfqpoint{5.587068in}{2.838062in}}%
\pgfpathlineto{\pgfqpoint{5.601670in}{2.846397in}}%
\pgfpathlineto{\pgfqpoint{5.594156in}{2.836662in}}%
\pgfpathlineto{\pgfqpoint{5.586634in}{2.826804in}}%
\pgfpathlineto{\pgfqpoint{5.579103in}{2.816822in}}%
\pgfpathlineto{\pgfqpoint{5.571564in}{2.806718in}}%
\pgfpathlineto{\pgfqpoint{5.556956in}{2.798427in}}%
\pgfpathlineto{\pgfqpoint{5.542365in}{2.790237in}}%
\pgfpathlineto{\pgfqpoint{5.527789in}{2.782148in}}%
\pgfpathlineto{\pgfqpoint{5.513230in}{2.774160in}}%
\pgfpathlineto{\pgfqpoint{5.520773in}{2.784212in}}%
\pgfpathlineto{\pgfqpoint{5.528309in}{2.794146in}}%
\pgfpathlineto{\pgfqpoint{5.535836in}{2.803962in}}%
\pgfpathlineto{\pgfqpoint{5.543356in}{2.813660in}}%
\pgfpathclose%
\pgfusepath{fill}%
\end{pgfscope}%
\begin{pgfscope}%
\pgfpathrectangle{\pgfqpoint{1.150000in}{0.150000in}}{\pgfqpoint{5.700000in}{5.700000in}}%
\pgfusepath{clip}%
\pgfsetbuttcap%
\pgfsetroundjoin%
\definecolor{currentfill}{rgb}{0.272594,0.025563,0.353093}%
\pgfsetfillcolor{currentfill}%
\pgfsetfillopacity{0.700000}%
\pgfsetlinewidth{0.000000pt}%
\definecolor{currentstroke}{rgb}{0.000000,0.000000,0.000000}%
\pgfsetstrokecolor{currentstroke}%
\pgfsetdash{}{0pt}%
\pgfpathmoveto{\pgfqpoint{4.133213in}{1.794064in}}%
\pgfpathlineto{\pgfqpoint{4.147121in}{1.792426in}}%
\pgfpathlineto{\pgfqpoint{4.161037in}{1.790890in}}%
\pgfpathlineto{\pgfqpoint{4.174960in}{1.789456in}}%
\pgfpathlineto{\pgfqpoint{4.188892in}{1.788123in}}%
\pgfpathlineto{\pgfqpoint{4.180901in}{1.778389in}}%
\pgfpathlineto{\pgfqpoint{4.172904in}{1.768735in}}%
\pgfpathlineto{\pgfqpoint{4.164902in}{1.759162in}}%
\pgfpathlineto{\pgfqpoint{4.156894in}{1.749676in}}%
\pgfpathlineto{\pgfqpoint{4.142952in}{1.751384in}}%
\pgfpathlineto{\pgfqpoint{4.129018in}{1.753193in}}%
\pgfpathlineto{\pgfqpoint{4.115092in}{1.755104in}}%
\pgfpathlineto{\pgfqpoint{4.101173in}{1.757116in}}%
\pgfpathlineto{\pgfqpoint{4.109192in}{1.766220in}}%
\pgfpathlineto{\pgfqpoint{4.117205in}{1.775415in}}%
\pgfpathlineto{\pgfqpoint{4.125212in}{1.784698in}}%
\pgfpathlineto{\pgfqpoint{4.133213in}{1.794064in}}%
\pgfpathclose%
\pgfusepath{fill}%
\end{pgfscope}%
\begin{pgfscope}%
\pgfpathrectangle{\pgfqpoint{1.150000in}{0.150000in}}{\pgfqpoint{5.700000in}{5.700000in}}%
\pgfusepath{clip}%
\pgfsetbuttcap%
\pgfsetroundjoin%
\definecolor{currentfill}{rgb}{0.214298,0.355619,0.551184}%
\pgfsetfillcolor{currentfill}%
\pgfsetfillopacity{0.700000}%
\pgfsetlinewidth{0.000000pt}%
\definecolor{currentstroke}{rgb}{0.000000,0.000000,0.000000}%
\pgfsetstrokecolor{currentstroke}%
\pgfsetdash{}{0pt}%
\pgfpathmoveto{\pgfqpoint{2.630985in}{2.490573in}}%
\pgfpathlineto{\pgfqpoint{2.644872in}{2.474589in}}%
\pgfpathlineto{\pgfqpoint{2.658754in}{2.458765in}}%
\pgfpathlineto{\pgfqpoint{2.672632in}{2.443100in}}%
\pgfpathlineto{\pgfqpoint{2.686507in}{2.427592in}}%
\pgfpathlineto{\pgfqpoint{2.677601in}{2.433231in}}%
\pgfpathlineto{\pgfqpoint{2.668676in}{2.439186in}}%
\pgfpathlineto{\pgfqpoint{2.659731in}{2.445465in}}%
\pgfpathlineto{\pgfqpoint{2.650766in}{2.452071in}}%
\pgfpathlineto{\pgfqpoint{2.636843in}{2.468115in}}%
\pgfpathlineto{\pgfqpoint{2.622915in}{2.484316in}}%
\pgfpathlineto{\pgfqpoint{2.608983in}{2.500677in}}%
\pgfpathlineto{\pgfqpoint{2.595046in}{2.517198in}}%
\pgfpathlineto{\pgfqpoint{2.604062in}{2.510047in}}%
\pgfpathlineto{\pgfqpoint{2.613057in}{2.503229in}}%
\pgfpathlineto{\pgfqpoint{2.622031in}{2.496740in}}%
\pgfpathlineto{\pgfqpoint{2.630985in}{2.490573in}}%
\pgfpathclose%
\pgfusepath{fill}%
\end{pgfscope}%
\begin{pgfscope}%
\pgfpathrectangle{\pgfqpoint{1.150000in}{0.150000in}}{\pgfqpoint{5.700000in}{5.700000in}}%
\pgfusepath{clip}%
\pgfsetbuttcap%
\pgfsetroundjoin%
\definecolor{currentfill}{rgb}{0.203063,0.379716,0.553925}%
\pgfsetfillcolor{currentfill}%
\pgfsetfillopacity{0.700000}%
\pgfsetlinewidth{0.000000pt}%
\definecolor{currentstroke}{rgb}{0.000000,0.000000,0.000000}%
\pgfsetstrokecolor{currentstroke}%
\pgfsetdash{}{0pt}%
\pgfpathmoveto{\pgfqpoint{2.575392in}{2.556134in}}%
\pgfpathlineto{\pgfqpoint{2.589297in}{2.539497in}}%
\pgfpathlineto{\pgfqpoint{2.603198in}{2.523026in}}%
\pgfpathlineto{\pgfqpoint{2.617094in}{2.506718in}}%
\pgfpathlineto{\pgfqpoint{2.630985in}{2.490573in}}%
\pgfpathlineto{\pgfqpoint{2.622031in}{2.496740in}}%
\pgfpathlineto{\pgfqpoint{2.613057in}{2.503229in}}%
\pgfpathlineto{\pgfqpoint{2.604062in}{2.510047in}}%
\pgfpathlineto{\pgfqpoint{2.595046in}{2.517198in}}%
\pgfpathlineto{\pgfqpoint{2.581104in}{2.533882in}}%
\pgfpathlineto{\pgfqpoint{2.567158in}{2.550729in}}%
\pgfpathlineto{\pgfqpoint{2.553206in}{2.567740in}}%
\pgfpathlineto{\pgfqpoint{2.539249in}{2.584918in}}%
\pgfpathlineto{\pgfqpoint{2.548317in}{2.577218in}}%
\pgfpathlineto{\pgfqpoint{2.557363in}{2.569858in}}%
\pgfpathlineto{\pgfqpoint{2.566388in}{2.562832in}}%
\pgfpathlineto{\pgfqpoint{2.575392in}{2.556134in}}%
\pgfpathclose%
\pgfusepath{fill}%
\end{pgfscope}%
\begin{pgfscope}%
\pgfpathrectangle{\pgfqpoint{1.150000in}{0.150000in}}{\pgfqpoint{5.700000in}{5.700000in}}%
\pgfusepath{clip}%
\pgfsetbuttcap%
\pgfsetroundjoin%
\definecolor{currentfill}{rgb}{0.225863,0.330805,0.547314}%
\pgfsetfillcolor{currentfill}%
\pgfsetfillopacity{0.700000}%
\pgfsetlinewidth{0.000000pt}%
\definecolor{currentstroke}{rgb}{0.000000,0.000000,0.000000}%
\pgfsetstrokecolor{currentstroke}%
\pgfsetdash{}{0pt}%
\pgfpathmoveto{\pgfqpoint{2.686507in}{2.427592in}}%
\pgfpathlineto{\pgfqpoint{2.700377in}{2.412240in}}%
\pgfpathlineto{\pgfqpoint{2.714244in}{2.397043in}}%
\pgfpathlineto{\pgfqpoint{2.728108in}{2.382000in}}%
\pgfpathlineto{\pgfqpoint{2.741968in}{2.367109in}}%
\pgfpathlineto{\pgfqpoint{2.733109in}{2.372223in}}%
\pgfpathlineto{\pgfqpoint{2.724232in}{2.377649in}}%
\pgfpathlineto{\pgfqpoint{2.715335in}{2.383391in}}%
\pgfpathlineto{\pgfqpoint{2.706419in}{2.389455in}}%
\pgfpathlineto{\pgfqpoint{2.692512in}{2.404878in}}%
\pgfpathlineto{\pgfqpoint{2.678600in}{2.420454in}}%
\pgfpathlineto{\pgfqpoint{2.664685in}{2.436185in}}%
\pgfpathlineto{\pgfqpoint{2.650766in}{2.452071in}}%
\pgfpathlineto{\pgfqpoint{2.659731in}{2.445465in}}%
\pgfpathlineto{\pgfqpoint{2.668676in}{2.439186in}}%
\pgfpathlineto{\pgfqpoint{2.677601in}{2.433231in}}%
\pgfpathlineto{\pgfqpoint{2.686507in}{2.427592in}}%
\pgfpathclose%
\pgfusepath{fill}%
\end{pgfscope}%
\begin{pgfscope}%
\pgfpathrectangle{\pgfqpoint{1.150000in}{0.150000in}}{\pgfqpoint{5.700000in}{5.700000in}}%
\pgfusepath{clip}%
\pgfsetbuttcap%
\pgfsetroundjoin%
\definecolor{currentfill}{rgb}{0.121148,0.592739,0.544641}%
\pgfsetfillcolor{currentfill}%
\pgfsetfillopacity{0.700000}%
\pgfsetlinewidth{0.000000pt}%
\definecolor{currentstroke}{rgb}{0.000000,0.000000,0.000000}%
\pgfsetstrokecolor{currentstroke}%
\pgfsetdash{}{0pt}%
\pgfpathmoveto{\pgfqpoint{5.926159in}{3.122543in}}%
\pgfpathlineto{\pgfqpoint{5.940939in}{3.132121in}}%
\pgfpathlineto{\pgfqpoint{5.955736in}{3.141801in}}%
\pgfpathlineto{\pgfqpoint{5.970551in}{3.151582in}}%
\pgfpathlineto{\pgfqpoint{5.963254in}{3.144490in}}%
\pgfpathlineto{\pgfqpoint{5.955946in}{3.137264in}}%
\pgfpathlineto{\pgfqpoint{5.948629in}{3.129904in}}%
\pgfpathlineto{\pgfqpoint{5.941301in}{3.122409in}}%
\pgfpathlineto{\pgfqpoint{5.926475in}{3.112572in}}%
\pgfpathlineto{\pgfqpoint{5.911667in}{3.102837in}}%
\pgfpathlineto{\pgfqpoint{5.896877in}{3.093204in}}%
\pgfpathlineto{\pgfqpoint{5.904212in}{3.100735in}}%
\pgfpathlineto{\pgfqpoint{5.911538in}{3.108135in}}%
\pgfpathlineto{\pgfqpoint{5.918853in}{3.115404in}}%
\pgfpathlineto{\pgfqpoint{5.926159in}{3.122543in}}%
\pgfpathclose%
\pgfusepath{fill}%
\end{pgfscope}%
\begin{pgfscope}%
\pgfpathrectangle{\pgfqpoint{1.150000in}{0.150000in}}{\pgfqpoint{5.700000in}{5.700000in}}%
\pgfusepath{clip}%
\pgfsetbuttcap%
\pgfsetroundjoin%
\definecolor{currentfill}{rgb}{0.283229,0.120777,0.440584}%
\pgfsetfillcolor{currentfill}%
\pgfsetfillopacity{0.700000}%
\pgfsetlinewidth{0.000000pt}%
\definecolor{currentstroke}{rgb}{0.000000,0.000000,0.000000}%
\pgfsetstrokecolor{currentstroke}%
\pgfsetdash{}{0pt}%
\pgfpathmoveto{\pgfqpoint{3.218528in}{1.963757in}}%
\pgfpathlineto{\pgfqpoint{3.232317in}{1.953858in}}%
\pgfpathlineto{\pgfqpoint{3.246109in}{1.944081in}}%
\pgfpathlineto{\pgfqpoint{3.259901in}{1.934427in}}%
\pgfpathlineto{\pgfqpoint{3.273695in}{1.924893in}}%
\pgfpathlineto{\pgfqpoint{3.265248in}{1.924389in}}%
\pgfpathlineto{\pgfqpoint{3.256788in}{1.924126in}}%
\pgfpathlineto{\pgfqpoint{3.248316in}{1.924110in}}%
\pgfpathlineto{\pgfqpoint{3.239830in}{1.924347in}}%
\pgfpathlineto{\pgfqpoint{3.226003in}{1.934373in}}%
\pgfpathlineto{\pgfqpoint{3.212178in}{1.944520in}}%
\pgfpathlineto{\pgfqpoint{3.198353in}{1.954790in}}%
\pgfpathlineto{\pgfqpoint{3.184530in}{1.965183in}}%
\pgfpathlineto{\pgfqpoint{3.193049in}{1.964445in}}%
\pgfpathlineto{\pgfqpoint{3.201555in}{1.963965in}}%
\pgfpathlineto{\pgfqpoint{3.210048in}{1.963737in}}%
\pgfpathlineto{\pgfqpoint{3.218528in}{1.963757in}}%
\pgfpathclose%
\pgfusepath{fill}%
\end{pgfscope}%
\begin{pgfscope}%
\pgfpathrectangle{\pgfqpoint{1.150000in}{0.150000in}}{\pgfqpoint{5.700000in}{5.700000in}}%
\pgfusepath{clip}%
\pgfsetbuttcap%
\pgfsetroundjoin%
\definecolor{currentfill}{rgb}{0.190631,0.407061,0.556089}%
\pgfsetfillcolor{currentfill}%
\pgfsetfillopacity{0.700000}%
\pgfsetlinewidth{0.000000pt}%
\definecolor{currentstroke}{rgb}{0.000000,0.000000,0.000000}%
\pgfsetstrokecolor{currentstroke}%
\pgfsetdash{}{0pt}%
\pgfpathmoveto{\pgfqpoint{2.519718in}{2.624359in}}%
\pgfpathlineto{\pgfqpoint{2.533644in}{2.607048in}}%
\pgfpathlineto{\pgfqpoint{2.547566in}{2.589908in}}%
\pgfpathlineto{\pgfqpoint{2.561481in}{2.572937in}}%
\pgfpathlineto{\pgfqpoint{2.575392in}{2.556134in}}%
\pgfpathlineto{\pgfqpoint{2.566388in}{2.562832in}}%
\pgfpathlineto{\pgfqpoint{2.557363in}{2.569858in}}%
\pgfpathlineto{\pgfqpoint{2.548317in}{2.577218in}}%
\pgfpathlineto{\pgfqpoint{2.539249in}{2.584918in}}%
\pgfpathlineto{\pgfqpoint{2.525286in}{2.602263in}}%
\pgfpathlineto{\pgfqpoint{2.511318in}{2.619776in}}%
\pgfpathlineto{\pgfqpoint{2.497344in}{2.637460in}}%
\pgfpathlineto{\pgfqpoint{2.483364in}{2.655316in}}%
\pgfpathlineto{\pgfqpoint{2.492486in}{2.647064in}}%
\pgfpathlineto{\pgfqpoint{2.501585in}{2.639158in}}%
\pgfpathlineto{\pgfqpoint{2.510662in}{2.631591in}}%
\pgfpathlineto{\pgfqpoint{2.519718in}{2.624359in}}%
\pgfpathclose%
\pgfusepath{fill}%
\end{pgfscope}%
\begin{pgfscope}%
\pgfpathrectangle{\pgfqpoint{1.150000in}{0.150000in}}{\pgfqpoint{5.700000in}{5.700000in}}%
\pgfusepath{clip}%
\pgfsetbuttcap%
\pgfsetroundjoin%
\definecolor{currentfill}{rgb}{0.237441,0.305202,0.541921}%
\pgfsetfillcolor{currentfill}%
\pgfsetfillopacity{0.700000}%
\pgfsetlinewidth{0.000000pt}%
\definecolor{currentstroke}{rgb}{0.000000,0.000000,0.000000}%
\pgfsetstrokecolor{currentstroke}%
\pgfsetdash{}{0pt}%
\pgfpathmoveto{\pgfqpoint{2.741968in}{2.367109in}}%
\pgfpathlineto{\pgfqpoint{2.755825in}{2.352370in}}%
\pgfpathlineto{\pgfqpoint{2.769679in}{2.337782in}}%
\pgfpathlineto{\pgfqpoint{2.783529in}{2.323342in}}%
\pgfpathlineto{\pgfqpoint{2.797377in}{2.309051in}}%
\pgfpathlineto{\pgfqpoint{2.788564in}{2.313643in}}%
\pgfpathlineto{\pgfqpoint{2.779733in}{2.318541in}}%
\pgfpathlineto{\pgfqpoint{2.770883in}{2.323750in}}%
\pgfpathlineto{\pgfqpoint{2.762015in}{2.329275in}}%
\pgfpathlineto{\pgfqpoint{2.748121in}{2.344096in}}%
\pgfpathlineto{\pgfqpoint{2.734223in}{2.359065in}}%
\pgfpathlineto{\pgfqpoint{2.720323in}{2.374185in}}%
\pgfpathlineto{\pgfqpoint{2.706419in}{2.389455in}}%
\pgfpathlineto{\pgfqpoint{2.715335in}{2.383391in}}%
\pgfpathlineto{\pgfqpoint{2.724232in}{2.377649in}}%
\pgfpathlineto{\pgfqpoint{2.733109in}{2.372223in}}%
\pgfpathlineto{\pgfqpoint{2.741968in}{2.367109in}}%
\pgfpathclose%
\pgfusepath{fill}%
\end{pgfscope}%
\begin{pgfscope}%
\pgfpathrectangle{\pgfqpoint{1.150000in}{0.150000in}}{\pgfqpoint{5.700000in}{5.700000in}}%
\pgfusepath{clip}%
\pgfsetbuttcap%
\pgfsetroundjoin%
\definecolor{currentfill}{rgb}{0.235526,0.309527,0.542944}%
\pgfsetfillcolor{currentfill}%
\pgfsetfillopacity{0.700000}%
\pgfsetlinewidth{0.000000pt}%
\definecolor{currentstroke}{rgb}{0.000000,0.000000,0.000000}%
\pgfsetstrokecolor{currentstroke}%
\pgfsetdash{}{0pt}%
\pgfpathmoveto{\pgfqpoint{5.041548in}{2.373676in}}%
\pgfpathlineto{\pgfqpoint{5.055833in}{2.378987in}}%
\pgfpathlineto{\pgfqpoint{5.070131in}{2.384398in}}%
\pgfpathlineto{\pgfqpoint{5.084443in}{2.389908in}}%
\pgfpathlineto{\pgfqpoint{5.098768in}{2.395518in}}%
\pgfpathlineto{\pgfqpoint{5.091039in}{2.383410in}}%
\pgfpathlineto{\pgfqpoint{5.083304in}{2.371225in}}%
\pgfpathlineto{\pgfqpoint{5.075564in}{2.358965in}}%
\pgfpathlineto{\pgfqpoint{5.067817in}{2.346630in}}%
\pgfpathlineto{\pgfqpoint{5.053490in}{2.341200in}}%
\pgfpathlineto{\pgfqpoint{5.039177in}{2.335868in}}%
\pgfpathlineto{\pgfqpoint{5.024877in}{2.330636in}}%
\pgfpathlineto{\pgfqpoint{5.010591in}{2.325504in}}%
\pgfpathlineto{\pgfqpoint{5.018339in}{2.337652in}}%
\pgfpathlineto{\pgfqpoint{5.026081in}{2.349732in}}%
\pgfpathlineto{\pgfqpoint{5.033817in}{2.361740in}}%
\pgfpathlineto{\pgfqpoint{5.041548in}{2.373676in}}%
\pgfpathclose%
\pgfusepath{fill}%
\end{pgfscope}%
\begin{pgfscope}%
\pgfpathrectangle{\pgfqpoint{1.150000in}{0.150000in}}{\pgfqpoint{5.700000in}{5.700000in}}%
\pgfusepath{clip}%
\pgfsetbuttcap%
\pgfsetroundjoin%
\definecolor{currentfill}{rgb}{0.274128,0.199721,0.498911}%
\pgfsetfillcolor{currentfill}%
\pgfsetfillopacity{0.700000}%
\pgfsetlinewidth{0.000000pt}%
\definecolor{currentstroke}{rgb}{0.000000,0.000000,0.000000}%
\pgfsetstrokecolor{currentstroke}%
\pgfsetdash{}{0pt}%
\pgfpathmoveto{\pgfqpoint{4.746687in}{2.128873in}}%
\pgfpathlineto{\pgfqpoint{4.760830in}{2.132167in}}%
\pgfpathlineto{\pgfqpoint{4.774984in}{2.135561in}}%
\pgfpathlineto{\pgfqpoint{4.789149in}{2.139054in}}%
\pgfpathlineto{\pgfqpoint{4.803327in}{2.142646in}}%
\pgfpathlineto{\pgfqpoint{4.795507in}{2.130251in}}%
\pgfpathlineto{\pgfqpoint{4.787682in}{2.117823in}}%
\pgfpathlineto{\pgfqpoint{4.779853in}{2.105364in}}%
\pgfpathlineto{\pgfqpoint{4.772018in}{2.092876in}}%
\pgfpathlineto{\pgfqpoint{4.757838in}{2.089535in}}%
\pgfpathlineto{\pgfqpoint{4.743670in}{2.086294in}}%
\pgfpathlineto{\pgfqpoint{4.729514in}{2.083152in}}%
\pgfpathlineto{\pgfqpoint{4.715370in}{2.080109in}}%
\pgfpathlineto{\pgfqpoint{4.723206in}{2.092339in}}%
\pgfpathlineto{\pgfqpoint{4.731038in}{2.104544in}}%
\pgfpathlineto{\pgfqpoint{4.738865in}{2.116723in}}%
\pgfpathlineto{\pgfqpoint{4.746687in}{2.128873in}}%
\pgfpathclose%
\pgfusepath{fill}%
\end{pgfscope}%
\begin{pgfscope}%
\pgfpathrectangle{\pgfqpoint{1.150000in}{0.150000in}}{\pgfqpoint{5.700000in}{5.700000in}}%
\pgfusepath{clip}%
\pgfsetbuttcap%
\pgfsetroundjoin%
\definecolor{currentfill}{rgb}{0.179019,0.433756,0.557430}%
\pgfsetfillcolor{currentfill}%
\pgfsetfillopacity{0.700000}%
\pgfsetlinewidth{0.000000pt}%
\definecolor{currentstroke}{rgb}{0.000000,0.000000,0.000000}%
\pgfsetstrokecolor{currentstroke}%
\pgfsetdash{}{0pt}%
\pgfpathmoveto{\pgfqpoint{2.463951in}{2.695341in}}%
\pgfpathlineto{\pgfqpoint{2.477902in}{2.677332in}}%
\pgfpathlineto{\pgfqpoint{2.491847in}{2.659500in}}%
\pgfpathlineto{\pgfqpoint{2.505785in}{2.641843in}}%
\pgfpathlineto{\pgfqpoint{2.519718in}{2.624359in}}%
\pgfpathlineto{\pgfqpoint{2.510662in}{2.631591in}}%
\pgfpathlineto{\pgfqpoint{2.501585in}{2.639158in}}%
\pgfpathlineto{\pgfqpoint{2.492486in}{2.647064in}}%
\pgfpathlineto{\pgfqpoint{2.483364in}{2.655316in}}%
\pgfpathlineto{\pgfqpoint{2.469378in}{2.673345in}}%
\pgfpathlineto{\pgfqpoint{2.455386in}{2.691548in}}%
\pgfpathlineto{\pgfqpoint{2.441387in}{2.709928in}}%
\pgfpathlineto{\pgfqpoint{2.427381in}{2.728485in}}%
\pgfpathlineto{\pgfqpoint{2.436558in}{2.719677in}}%
\pgfpathlineto{\pgfqpoint{2.445712in}{2.711221in}}%
\pgfpathlineto{\pgfqpoint{2.454843in}{2.703111in}}%
\pgfpathlineto{\pgfqpoint{2.463951in}{2.695341in}}%
\pgfpathclose%
\pgfusepath{fill}%
\end{pgfscope}%
\begin{pgfscope}%
\pgfpathrectangle{\pgfqpoint{1.150000in}{0.150000in}}{\pgfqpoint{5.700000in}{5.700000in}}%
\pgfusepath{clip}%
\pgfsetbuttcap%
\pgfsetroundjoin%
\definecolor{currentfill}{rgb}{0.246811,0.283237,0.535941}%
\pgfsetfillcolor{currentfill}%
\pgfsetfillopacity{0.700000}%
\pgfsetlinewidth{0.000000pt}%
\definecolor{currentstroke}{rgb}{0.000000,0.000000,0.000000}%
\pgfsetstrokecolor{currentstroke}%
\pgfsetdash{}{0pt}%
\pgfpathmoveto{\pgfqpoint{2.797377in}{2.309051in}}%
\pgfpathlineto{\pgfqpoint{2.811223in}{2.294907in}}%
\pgfpathlineto{\pgfqpoint{2.825066in}{2.280909in}}%
\pgfpathlineto{\pgfqpoint{2.838906in}{2.267056in}}%
\pgfpathlineto{\pgfqpoint{2.852745in}{2.253347in}}%
\pgfpathlineto{\pgfqpoint{2.843976in}{2.257420in}}%
\pgfpathlineto{\pgfqpoint{2.835189in}{2.261792in}}%
\pgfpathlineto{\pgfqpoint{2.826385in}{2.266471in}}%
\pgfpathlineto{\pgfqpoint{2.817562in}{2.271460in}}%
\pgfpathlineto{\pgfqpoint{2.803679in}{2.285696in}}%
\pgfpathlineto{\pgfqpoint{2.789794in}{2.300076in}}%
\pgfpathlineto{\pgfqpoint{2.775906in}{2.314602in}}%
\pgfpathlineto{\pgfqpoint{2.762015in}{2.329275in}}%
\pgfpathlineto{\pgfqpoint{2.770883in}{2.323750in}}%
\pgfpathlineto{\pgfqpoint{2.779733in}{2.318541in}}%
\pgfpathlineto{\pgfqpoint{2.788564in}{2.313643in}}%
\pgfpathlineto{\pgfqpoint{2.797377in}{2.309051in}}%
\pgfpathclose%
\pgfusepath{fill}%
\end{pgfscope}%
\begin{pgfscope}%
\pgfpathrectangle{\pgfqpoint{1.150000in}{0.150000in}}{\pgfqpoint{5.700000in}{5.700000in}}%
\pgfusepath{clip}%
\pgfsetbuttcap%
\pgfsetroundjoin%
\definecolor{currentfill}{rgb}{0.277018,0.050344,0.375715}%
\pgfsetfillcolor{currentfill}%
\pgfsetfillopacity{0.700000}%
\pgfsetlinewidth{0.000000pt}%
\definecolor{currentstroke}{rgb}{0.000000,0.000000,0.000000}%
\pgfsetstrokecolor{currentstroke}%
\pgfsetdash{}{0pt}%
\pgfpathmoveto{\pgfqpoint{3.472696in}{1.829736in}}%
\pgfpathlineto{\pgfqpoint{3.486492in}{1.822197in}}%
\pgfpathlineto{\pgfqpoint{3.500291in}{1.814772in}}%
\pgfpathlineto{\pgfqpoint{3.514093in}{1.807459in}}%
\pgfpathlineto{\pgfqpoint{3.527899in}{1.800260in}}%
\pgfpathlineto{\pgfqpoint{3.519611in}{1.796983in}}%
\pgfpathlineto{\pgfqpoint{3.511312in}{1.793910in}}%
\pgfpathlineto{\pgfqpoint{3.503003in}{1.791044in}}%
\pgfpathlineto{\pgfqpoint{3.494684in}{1.788391in}}%
\pgfpathlineto{\pgfqpoint{3.480853in}{1.796060in}}%
\pgfpathlineto{\pgfqpoint{3.467024in}{1.803841in}}%
\pgfpathlineto{\pgfqpoint{3.453198in}{1.811736in}}%
\pgfpathlineto{\pgfqpoint{3.439376in}{1.819744in}}%
\pgfpathlineto{\pgfqpoint{3.447722in}{1.821921in}}%
\pgfpathlineto{\pgfqpoint{3.456057in}{1.824315in}}%
\pgfpathlineto{\pgfqpoint{3.464382in}{1.826922in}}%
\pgfpathlineto{\pgfqpoint{3.472696in}{1.829736in}}%
\pgfpathclose%
\pgfusepath{fill}%
\end{pgfscope}%
\begin{pgfscope}%
\pgfpathrectangle{\pgfqpoint{1.150000in}{0.150000in}}{\pgfqpoint{5.700000in}{5.700000in}}%
\pgfusepath{clip}%
\pgfsetbuttcap%
\pgfsetroundjoin%
\definecolor{currentfill}{rgb}{0.187231,0.414746,0.556547}%
\pgfsetfillcolor{currentfill}%
\pgfsetfillopacity{0.700000}%
\pgfsetlinewidth{0.000000pt}%
\definecolor{currentstroke}{rgb}{0.000000,0.000000,0.000000}%
\pgfsetstrokecolor{currentstroke}%
\pgfsetdash{}{0pt}%
\pgfpathmoveto{\pgfqpoint{5.336614in}{2.630436in}}%
\pgfpathlineto{\pgfqpoint{5.351058in}{2.637472in}}%
\pgfpathlineto{\pgfqpoint{5.365518in}{2.644609in}}%
\pgfpathlineto{\pgfqpoint{5.379993in}{2.651845in}}%
\pgfpathlineto{\pgfqpoint{5.394483in}{2.659182in}}%
\pgfpathlineto{\pgfqpoint{5.386866in}{2.648181in}}%
\pgfpathlineto{\pgfqpoint{5.379242in}{2.637071in}}%
\pgfpathlineto{\pgfqpoint{5.371610in}{2.625853in}}%
\pgfpathlineto{\pgfqpoint{5.363972in}{2.614527in}}%
\pgfpathlineto{\pgfqpoint{5.349479in}{2.607294in}}%
\pgfpathlineto{\pgfqpoint{5.335002in}{2.600161in}}%
\pgfpathlineto{\pgfqpoint{5.320539in}{2.593128in}}%
\pgfpathlineto{\pgfqpoint{5.306092in}{2.586195in}}%
\pgfpathlineto{\pgfqpoint{5.313733in}{2.597410in}}%
\pgfpathlineto{\pgfqpoint{5.321367in}{2.608522in}}%
\pgfpathlineto{\pgfqpoint{5.328994in}{2.619531in}}%
\pgfpathlineto{\pgfqpoint{5.336614in}{2.630436in}}%
\pgfpathclose%
\pgfusepath{fill}%
\end{pgfscope}%
\begin{pgfscope}%
\pgfpathrectangle{\pgfqpoint{1.150000in}{0.150000in}}{\pgfqpoint{5.700000in}{5.700000in}}%
\pgfusepath{clip}%
\pgfsetbuttcap%
\pgfsetroundjoin%
\definecolor{currentfill}{rgb}{0.269944,0.014625,0.341379}%
\pgfsetfillcolor{currentfill}%
\pgfsetfillopacity{0.700000}%
\pgfsetlinewidth{0.000000pt}%
\definecolor{currentstroke}{rgb}{0.000000,0.000000,0.000000}%
\pgfsetstrokecolor{currentstroke}%
\pgfsetdash{}{0pt}%
\pgfpathmoveto{\pgfqpoint{4.045574in}{1.766184in}}%
\pgfpathlineto{\pgfqpoint{4.059463in}{1.763763in}}%
\pgfpathlineto{\pgfqpoint{4.073359in}{1.761445in}}%
\pgfpathlineto{\pgfqpoint{4.087262in}{1.759229in}}%
\pgfpathlineto{\pgfqpoint{4.101173in}{1.757116in}}%
\pgfpathlineto{\pgfqpoint{4.093149in}{1.748106in}}%
\pgfpathlineto{\pgfqpoint{4.085119in}{1.739194in}}%
\pgfpathlineto{\pgfqpoint{4.077084in}{1.730384in}}%
\pgfpathlineto{\pgfqpoint{4.069042in}{1.721680in}}%
\pgfpathlineto{\pgfqpoint{4.055119in}{1.724187in}}%
\pgfpathlineto{\pgfqpoint{4.041203in}{1.726796in}}%
\pgfpathlineto{\pgfqpoint{4.027295in}{1.729506in}}%
\pgfpathlineto{\pgfqpoint{4.013393in}{1.732319in}}%
\pgfpathlineto{\pgfqpoint{4.021447in}{1.740624in}}%
\pgfpathlineto{\pgfqpoint{4.029496in}{1.749038in}}%
\pgfpathlineto{\pgfqpoint{4.037538in}{1.757560in}}%
\pgfpathlineto{\pgfqpoint{4.045574in}{1.766184in}}%
\pgfpathclose%
\pgfusepath{fill}%
\end{pgfscope}%
\begin{pgfscope}%
\pgfpathrectangle{\pgfqpoint{1.150000in}{0.150000in}}{\pgfqpoint{5.700000in}{5.700000in}}%
\pgfusepath{clip}%
\pgfsetbuttcap%
\pgfsetroundjoin%
\definecolor{currentfill}{rgb}{0.255645,0.260703,0.528312}%
\pgfsetfillcolor{currentfill}%
\pgfsetfillopacity{0.700000}%
\pgfsetlinewidth{0.000000pt}%
\definecolor{currentstroke}{rgb}{0.000000,0.000000,0.000000}%
\pgfsetstrokecolor{currentstroke}%
\pgfsetdash{}{0pt}%
\pgfpathmoveto{\pgfqpoint{2.852745in}{2.253347in}}%
\pgfpathlineto{\pgfqpoint{2.866581in}{2.239781in}}%
\pgfpathlineto{\pgfqpoint{2.880415in}{2.226358in}}%
\pgfpathlineto{\pgfqpoint{2.894248in}{2.213075in}}%
\pgfpathlineto{\pgfqpoint{2.908078in}{2.199932in}}%
\pgfpathlineto{\pgfqpoint{2.899352in}{2.203488in}}%
\pgfpathlineto{\pgfqpoint{2.890609in}{2.207338in}}%
\pgfpathlineto{\pgfqpoint{2.881848in}{2.211488in}}%
\pgfpathlineto{\pgfqpoint{2.873070in}{2.215944in}}%
\pgfpathlineto{\pgfqpoint{2.859196in}{2.229611in}}%
\pgfpathlineto{\pgfqpoint{2.845321in}{2.243418in}}%
\pgfpathlineto{\pgfqpoint{2.831443in}{2.257368in}}%
\pgfpathlineto{\pgfqpoint{2.817562in}{2.271460in}}%
\pgfpathlineto{\pgfqpoint{2.826385in}{2.266471in}}%
\pgfpathlineto{\pgfqpoint{2.835189in}{2.261792in}}%
\pgfpathlineto{\pgfqpoint{2.843976in}{2.257420in}}%
\pgfpathlineto{\pgfqpoint{2.852745in}{2.253347in}}%
\pgfpathclose%
\pgfusepath{fill}%
\end{pgfscope}%
\begin{pgfscope}%
\pgfpathrectangle{\pgfqpoint{1.150000in}{0.150000in}}{\pgfqpoint{5.700000in}{5.700000in}}%
\pgfusepath{clip}%
\pgfsetbuttcap%
\pgfsetroundjoin%
\definecolor{currentfill}{rgb}{0.166617,0.463708,0.558119}%
\pgfsetfillcolor{currentfill}%
\pgfsetfillopacity{0.700000}%
\pgfsetlinewidth{0.000000pt}%
\definecolor{currentstroke}{rgb}{0.000000,0.000000,0.000000}%
\pgfsetstrokecolor{currentstroke}%
\pgfsetdash{}{0pt}%
\pgfpathmoveto{\pgfqpoint{2.408081in}{2.769177in}}%
\pgfpathlineto{\pgfqpoint{2.422059in}{2.750445in}}%
\pgfpathlineto{\pgfqpoint{2.436030in}{2.731896in}}%
\pgfpathlineto{\pgfqpoint{2.449994in}{2.713528in}}%
\pgfpathlineto{\pgfqpoint{2.463951in}{2.695341in}}%
\pgfpathlineto{\pgfqpoint{2.454843in}{2.703111in}}%
\pgfpathlineto{\pgfqpoint{2.445712in}{2.711221in}}%
\pgfpathlineto{\pgfqpoint{2.436558in}{2.719677in}}%
\pgfpathlineto{\pgfqpoint{2.427381in}{2.728485in}}%
\pgfpathlineto{\pgfqpoint{2.413369in}{2.747222in}}%
\pgfpathlineto{\pgfqpoint{2.399349in}{2.766140in}}%
\pgfpathlineto{\pgfqpoint{2.385322in}{2.785240in}}%
\pgfpathlineto{\pgfqpoint{2.371288in}{2.804524in}}%
\pgfpathlineto{\pgfqpoint{2.380522in}{2.795156in}}%
\pgfpathlineto{\pgfqpoint{2.389732in}{2.786146in}}%
\pgfpathlineto{\pgfqpoint{2.398918in}{2.777488in}}%
\pgfpathlineto{\pgfqpoint{2.408081in}{2.769177in}}%
\pgfpathclose%
\pgfusepath{fill}%
\end{pgfscope}%
\begin{pgfscope}%
\pgfpathrectangle{\pgfqpoint{1.150000in}{0.150000in}}{\pgfqpoint{5.700000in}{5.700000in}}%
\pgfusepath{clip}%
\pgfsetbuttcap%
\pgfsetroundjoin%
\definecolor{currentfill}{rgb}{0.147607,0.511733,0.557049}%
\pgfsetfillcolor{currentfill}%
\pgfsetfillopacity{0.700000}%
\pgfsetlinewidth{0.000000pt}%
\definecolor{currentstroke}{rgb}{0.000000,0.000000,0.000000}%
\pgfsetstrokecolor{currentstroke}%
\pgfsetdash{}{0pt}%
\pgfpathmoveto{\pgfqpoint{5.631641in}{2.884106in}}%
\pgfpathlineto{\pgfqpoint{5.646254in}{2.892568in}}%
\pgfpathlineto{\pgfqpoint{5.660883in}{2.901131in}}%
\pgfpathlineto{\pgfqpoint{5.675528in}{2.909794in}}%
\pgfpathlineto{\pgfqpoint{5.690190in}{2.918559in}}%
\pgfpathlineto{\pgfqpoint{5.682717in}{2.909299in}}%
\pgfpathlineto{\pgfqpoint{5.675234in}{2.899911in}}%
\pgfpathlineto{\pgfqpoint{5.667743in}{2.890394in}}%
\pgfpathlineto{\pgfqpoint{5.660242in}{2.880750in}}%
\pgfpathlineto{\pgfqpoint{5.645575in}{2.872010in}}%
\pgfpathlineto{\pgfqpoint{5.630924in}{2.863371in}}%
\pgfpathlineto{\pgfqpoint{5.616289in}{2.854834in}}%
\pgfpathlineto{\pgfqpoint{5.601670in}{2.846397in}}%
\pgfpathlineto{\pgfqpoint{5.609176in}{2.856009in}}%
\pgfpathlineto{\pgfqpoint{5.616673in}{2.865498in}}%
\pgfpathlineto{\pgfqpoint{5.624161in}{2.874864in}}%
\pgfpathlineto{\pgfqpoint{5.631641in}{2.884106in}}%
\pgfpathclose%
\pgfusepath{fill}%
\end{pgfscope}%
\begin{pgfscope}%
\pgfpathrectangle{\pgfqpoint{1.150000in}{0.150000in}}{\pgfqpoint{5.700000in}{5.700000in}}%
\pgfusepath{clip}%
\pgfsetbuttcap%
\pgfsetroundjoin%
\definecolor{currentfill}{rgb}{0.265145,0.232956,0.516599}%
\pgfsetfillcolor{currentfill}%
\pgfsetfillopacity{0.700000}%
\pgfsetlinewidth{0.000000pt}%
\definecolor{currentstroke}{rgb}{0.000000,0.000000,0.000000}%
\pgfsetstrokecolor{currentstroke}%
\pgfsetdash{}{0pt}%
\pgfpathmoveto{\pgfqpoint{4.834557in}{2.191848in}}%
\pgfpathlineto{\pgfqpoint{4.848745in}{2.195773in}}%
\pgfpathlineto{\pgfqpoint{4.862945in}{2.199797in}}%
\pgfpathlineto{\pgfqpoint{4.877157in}{2.203921in}}%
\pgfpathlineto{\pgfqpoint{4.891382in}{2.208144in}}%
\pgfpathlineto{\pgfqpoint{4.883584in}{2.195677in}}%
\pgfpathlineto{\pgfqpoint{4.875781in}{2.183164in}}%
\pgfpathlineto{\pgfqpoint{4.867973in}{2.170606in}}%
\pgfpathlineto{\pgfqpoint{4.860159in}{2.158006in}}%
\pgfpathlineto{\pgfqpoint{4.845933in}{2.154018in}}%
\pgfpathlineto{\pgfqpoint{4.831719in}{2.150128in}}%
\pgfpathlineto{\pgfqpoint{4.817517in}{2.146338in}}%
\pgfpathlineto{\pgfqpoint{4.803327in}{2.142646in}}%
\pgfpathlineto{\pgfqpoint{4.811142in}{2.155005in}}%
\pgfpathlineto{\pgfqpoint{4.818952in}{2.167327in}}%
\pgfpathlineto{\pgfqpoint{4.826757in}{2.179609in}}%
\pgfpathlineto{\pgfqpoint{4.834557in}{2.191848in}}%
\pgfpathclose%
\pgfusepath{fill}%
\end{pgfscope}%
\begin{pgfscope}%
\pgfpathrectangle{\pgfqpoint{1.150000in}{0.150000in}}{\pgfqpoint{5.700000in}{5.700000in}}%
\pgfusepath{clip}%
\pgfsetbuttcap%
\pgfsetroundjoin%
\definecolor{currentfill}{rgb}{0.282910,0.105393,0.426902}%
\pgfsetfillcolor{currentfill}%
\pgfsetfillopacity{0.700000}%
\pgfsetlinewidth{0.000000pt}%
\definecolor{currentstroke}{rgb}{0.000000,0.000000,0.000000}%
\pgfsetstrokecolor{currentstroke}%
\pgfsetdash{}{0pt}%
\pgfpathmoveto{\pgfqpoint{3.273695in}{1.924893in}}%
\pgfpathlineto{\pgfqpoint{3.287491in}{1.915481in}}%
\pgfpathlineto{\pgfqpoint{3.301288in}{1.906188in}}%
\pgfpathlineto{\pgfqpoint{3.315087in}{1.897015in}}%
\pgfpathlineto{\pgfqpoint{3.328888in}{1.887961in}}%
\pgfpathlineto{\pgfqpoint{3.320472in}{1.886972in}}%
\pgfpathlineto{\pgfqpoint{3.312044in}{1.886220in}}%
\pgfpathlineto{\pgfqpoint{3.303604in}{1.885710in}}%
\pgfpathlineto{\pgfqpoint{3.295151in}{1.885447in}}%
\pgfpathlineto{\pgfqpoint{3.281318in}{1.894993in}}%
\pgfpathlineto{\pgfqpoint{3.267487in}{1.904658in}}%
\pgfpathlineto{\pgfqpoint{3.253658in}{1.914442in}}%
\pgfpathlineto{\pgfqpoint{3.239830in}{1.924347in}}%
\pgfpathlineto{\pgfqpoint{3.248316in}{1.924110in}}%
\pgfpathlineto{\pgfqpoint{3.256788in}{1.924126in}}%
\pgfpathlineto{\pgfqpoint{3.265248in}{1.924389in}}%
\pgfpathlineto{\pgfqpoint{3.273695in}{1.924893in}}%
\pgfpathclose%
\pgfusepath{fill}%
\end{pgfscope}%
\begin{pgfscope}%
\pgfpathrectangle{\pgfqpoint{1.150000in}{0.150000in}}{\pgfqpoint{5.700000in}{5.700000in}}%
\pgfusepath{clip}%
\pgfsetbuttcap%
\pgfsetroundjoin%
\definecolor{currentfill}{rgb}{0.268510,0.009605,0.335427}%
\pgfsetfillcolor{currentfill}%
\pgfsetfillopacity{0.700000}%
\pgfsetlinewidth{0.000000pt}%
\definecolor{currentstroke}{rgb}{0.000000,0.000000,0.000000}%
\pgfsetstrokecolor{currentstroke}%
\pgfsetdash{}{0pt}%
\pgfpathmoveto{\pgfqpoint{3.814625in}{1.747043in}}%
\pgfpathlineto{\pgfqpoint{3.828466in}{1.742571in}}%
\pgfpathlineto{\pgfqpoint{3.842313in}{1.738205in}}%
\pgfpathlineto{\pgfqpoint{3.856166in}{1.733944in}}%
\pgfpathlineto{\pgfqpoint{3.870024in}{1.729788in}}%
\pgfpathlineto{\pgfqpoint{3.861907in}{1.722939in}}%
\pgfpathlineto{\pgfqpoint{3.853783in}{1.716234in}}%
\pgfpathlineto{\pgfqpoint{3.845651in}{1.709677in}}%
\pgfpathlineto{\pgfqpoint{3.837512in}{1.703273in}}%
\pgfpathlineto{\pgfqpoint{3.823637in}{1.707858in}}%
\pgfpathlineto{\pgfqpoint{3.809767in}{1.712548in}}%
\pgfpathlineto{\pgfqpoint{3.795902in}{1.717343in}}%
\pgfpathlineto{\pgfqpoint{3.782043in}{1.722245in}}%
\pgfpathlineto{\pgfqpoint{3.790200in}{1.728213in}}%
\pgfpathlineto{\pgfqpoint{3.798349in}{1.734339in}}%
\pgfpathlineto{\pgfqpoint{3.806491in}{1.740617in}}%
\pgfpathlineto{\pgfqpoint{3.814625in}{1.747043in}}%
\pgfpathclose%
\pgfusepath{fill}%
\end{pgfscope}%
\begin{pgfscope}%
\pgfpathrectangle{\pgfqpoint{1.150000in}{0.150000in}}{\pgfqpoint{5.700000in}{5.700000in}}%
\pgfusepath{clip}%
\pgfsetbuttcap%
\pgfsetroundjoin%
\definecolor{currentfill}{rgb}{0.220057,0.343307,0.549413}%
\pgfsetfillcolor{currentfill}%
\pgfsetfillopacity{0.700000}%
\pgfsetlinewidth{0.000000pt}%
\definecolor{currentstroke}{rgb}{0.000000,0.000000,0.000000}%
\pgfsetstrokecolor{currentstroke}%
\pgfsetdash{}{0pt}%
\pgfpathmoveto{\pgfqpoint{5.129623in}{2.443152in}}%
\pgfpathlineto{\pgfqpoint{5.143961in}{2.449021in}}%
\pgfpathlineto{\pgfqpoint{5.158312in}{2.454991in}}%
\pgfpathlineto{\pgfqpoint{5.172677in}{2.461060in}}%
\pgfpathlineto{\pgfqpoint{5.187057in}{2.467229in}}%
\pgfpathlineto{\pgfqpoint{5.179354in}{2.455289in}}%
\pgfpathlineto{\pgfqpoint{5.171644in}{2.443263in}}%
\pgfpathlineto{\pgfqpoint{5.163929in}{2.431150in}}%
\pgfpathlineto{\pgfqpoint{5.156207in}{2.418953in}}%
\pgfpathlineto{\pgfqpoint{5.141827in}{2.412945in}}%
\pgfpathlineto{\pgfqpoint{5.127460in}{2.407036in}}%
\pgfpathlineto{\pgfqpoint{5.113107in}{2.401227in}}%
\pgfpathlineto{\pgfqpoint{5.098768in}{2.395518in}}%
\pgfpathlineto{\pgfqpoint{5.106491in}{2.407547in}}%
\pgfpathlineto{\pgfqpoint{5.114208in}{2.419497in}}%
\pgfpathlineto{\pgfqpoint{5.121919in}{2.431366in}}%
\pgfpathlineto{\pgfqpoint{5.129623in}{2.443152in}}%
\pgfpathclose%
\pgfusepath{fill}%
\end{pgfscope}%
\begin{pgfscope}%
\pgfpathrectangle{\pgfqpoint{1.150000in}{0.150000in}}{\pgfqpoint{5.700000in}{5.700000in}}%
\pgfusepath{clip}%
\pgfsetbuttcap%
\pgfsetroundjoin%
\definecolor{currentfill}{rgb}{0.269944,0.014625,0.341379}%
\pgfsetfillcolor{currentfill}%
\pgfsetfillopacity{0.700000}%
\pgfsetlinewidth{0.000000pt}%
\definecolor{currentstroke}{rgb}{0.000000,0.000000,0.000000}%
\pgfsetstrokecolor{currentstroke}%
\pgfsetdash{}{0pt}%
\pgfpathmoveto{\pgfqpoint{3.671353in}{1.765293in}}%
\pgfpathlineto{\pgfqpoint{3.685172in}{1.759536in}}%
\pgfpathlineto{\pgfqpoint{3.698996in}{1.753886in}}%
\pgfpathlineto{\pgfqpoint{3.712825in}{1.748345in}}%
\pgfpathlineto{\pgfqpoint{3.726659in}{1.742911in}}%
\pgfpathlineto{\pgfqpoint{3.718474in}{1.737545in}}%
\pgfpathlineto{\pgfqpoint{3.710281in}{1.732348in}}%
\pgfpathlineto{\pgfqpoint{3.702080in}{1.727327in}}%
\pgfpathlineto{\pgfqpoint{3.693871in}{1.722485in}}%
\pgfpathlineto{\pgfqpoint{3.680016in}{1.728367in}}%
\pgfpathlineto{\pgfqpoint{3.666166in}{1.734357in}}%
\pgfpathlineto{\pgfqpoint{3.652321in}{1.740454in}}%
\pgfpathlineto{\pgfqpoint{3.638480in}{1.746661in}}%
\pgfpathlineto{\pgfqpoint{3.646712in}{1.751047in}}%
\pgfpathlineto{\pgfqpoint{3.654934in}{1.755618in}}%
\pgfpathlineto{\pgfqpoint{3.663148in}{1.760368in}}%
\pgfpathlineto{\pgfqpoint{3.671353in}{1.765293in}}%
\pgfpathclose%
\pgfusepath{fill}%
\end{pgfscope}%
\begin{pgfscope}%
\pgfpathrectangle{\pgfqpoint{1.150000in}{0.150000in}}{\pgfqpoint{5.700000in}{5.700000in}}%
\pgfusepath{clip}%
\pgfsetbuttcap%
\pgfsetroundjoin%
\definecolor{currentfill}{rgb}{0.263663,0.237631,0.518762}%
\pgfsetfillcolor{currentfill}%
\pgfsetfillopacity{0.700000}%
\pgfsetlinewidth{0.000000pt}%
\definecolor{currentstroke}{rgb}{0.000000,0.000000,0.000000}%
\pgfsetstrokecolor{currentstroke}%
\pgfsetdash{}{0pt}%
\pgfpathmoveto{\pgfqpoint{2.908078in}{2.199932in}}%
\pgfpathlineto{\pgfqpoint{2.921908in}{2.186929in}}%
\pgfpathlineto{\pgfqpoint{2.935735in}{2.174064in}}%
\pgfpathlineto{\pgfqpoint{2.949562in}{2.161336in}}%
\pgfpathlineto{\pgfqpoint{2.963387in}{2.148745in}}%
\pgfpathlineto{\pgfqpoint{2.954702in}{2.151786in}}%
\pgfpathlineto{\pgfqpoint{2.946001in}{2.155115in}}%
\pgfpathlineto{\pgfqpoint{2.937283in}{2.158740in}}%
\pgfpathlineto{\pgfqpoint{2.928548in}{2.162664in}}%
\pgfpathlineto{\pgfqpoint{2.914681in}{2.175777in}}%
\pgfpathlineto{\pgfqpoint{2.900812in}{2.189028in}}%
\pgfpathlineto{\pgfqpoint{2.886942in}{2.202416in}}%
\pgfpathlineto{\pgfqpoint{2.873070in}{2.215944in}}%
\pgfpathlineto{\pgfqpoint{2.881848in}{2.211488in}}%
\pgfpathlineto{\pgfqpoint{2.890609in}{2.207338in}}%
\pgfpathlineto{\pgfqpoint{2.899352in}{2.203488in}}%
\pgfpathlineto{\pgfqpoint{2.908078in}{2.199932in}}%
\pgfpathclose%
\pgfusepath{fill}%
\end{pgfscope}%
\begin{pgfscope}%
\pgfpathrectangle{\pgfqpoint{1.150000in}{0.150000in}}{\pgfqpoint{5.700000in}{5.700000in}}%
\pgfusepath{clip}%
\pgfsetbuttcap%
\pgfsetroundjoin%
\definecolor{currentfill}{rgb}{0.154815,0.493313,0.557840}%
\pgfsetfillcolor{currentfill}%
\pgfsetfillopacity{0.700000}%
\pgfsetlinewidth{0.000000pt}%
\definecolor{currentstroke}{rgb}{0.000000,0.000000,0.000000}%
\pgfsetstrokecolor{currentstroke}%
\pgfsetdash{}{0pt}%
\pgfpathmoveto{\pgfqpoint{2.352096in}{2.845971in}}%
\pgfpathlineto{\pgfqpoint{2.366104in}{2.826489in}}%
\pgfpathlineto{\pgfqpoint{2.380104in}{2.807197in}}%
\pgfpathlineto{\pgfqpoint{2.394096in}{2.788094in}}%
\pgfpathlineto{\pgfqpoint{2.408081in}{2.769177in}}%
\pgfpathlineto{\pgfqpoint{2.398918in}{2.777488in}}%
\pgfpathlineto{\pgfqpoint{2.389732in}{2.786146in}}%
\pgfpathlineto{\pgfqpoint{2.380522in}{2.795156in}}%
\pgfpathlineto{\pgfqpoint{2.371288in}{2.804524in}}%
\pgfpathlineto{\pgfqpoint{2.357246in}{2.823995in}}%
\pgfpathlineto{\pgfqpoint{2.343196in}{2.843653in}}%
\pgfpathlineto{\pgfqpoint{2.329139in}{2.863500in}}%
\pgfpathlineto{\pgfqpoint{2.315072in}{2.883539in}}%
\pgfpathlineto{\pgfqpoint{2.324365in}{2.873607in}}%
\pgfpathlineto{\pgfqpoint{2.333633in}{2.864038in}}%
\pgfpathlineto{\pgfqpoint{2.342876in}{2.854829in}}%
\pgfpathlineto{\pgfqpoint{2.352096in}{2.845971in}}%
\pgfpathclose%
\pgfusepath{fill}%
\end{pgfscope}%
\begin{pgfscope}%
\pgfpathrectangle{\pgfqpoint{1.150000in}{0.150000in}}{\pgfqpoint{5.700000in}{5.700000in}}%
\pgfusepath{clip}%
\pgfsetbuttcap%
\pgfsetroundjoin%
\definecolor{currentfill}{rgb}{0.174274,0.445044,0.557792}%
\pgfsetfillcolor{currentfill}%
\pgfsetfillopacity{0.700000}%
\pgfsetlinewidth{0.000000pt}%
\definecolor{currentstroke}{rgb}{0.000000,0.000000,0.000000}%
\pgfsetstrokecolor{currentstroke}%
\pgfsetdash{}{0pt}%
\pgfpathmoveto{\pgfqpoint{5.424876in}{2.702089in}}%
\pgfpathlineto{\pgfqpoint{5.439377in}{2.709611in}}%
\pgfpathlineto{\pgfqpoint{5.453895in}{2.717232in}}%
\pgfpathlineto{\pgfqpoint{5.468427in}{2.724955in}}%
\pgfpathlineto{\pgfqpoint{5.482975in}{2.732778in}}%
\pgfpathlineto{\pgfqpoint{5.475392in}{2.722140in}}%
\pgfpathlineto{\pgfqpoint{5.467801in}{2.711387in}}%
\pgfpathlineto{\pgfqpoint{5.460202in}{2.700518in}}%
\pgfpathlineto{\pgfqpoint{5.452595in}{2.689535in}}%
\pgfpathlineto{\pgfqpoint{5.438044in}{2.681796in}}%
\pgfpathlineto{\pgfqpoint{5.423508in}{2.674157in}}%
\pgfpathlineto{\pgfqpoint{5.408988in}{2.666620in}}%
\pgfpathlineto{\pgfqpoint{5.394483in}{2.659182in}}%
\pgfpathlineto{\pgfqpoint{5.402092in}{2.670075in}}%
\pgfpathlineto{\pgfqpoint{5.409694in}{2.680857in}}%
\pgfpathlineto{\pgfqpoint{5.417289in}{2.691528in}}%
\pgfpathlineto{\pgfqpoint{5.424876in}{2.702089in}}%
\pgfpathclose%
\pgfusepath{fill}%
\end{pgfscope}%
\begin{pgfscope}%
\pgfpathrectangle{\pgfqpoint{1.150000in}{0.150000in}}{\pgfqpoint{5.700000in}{5.700000in}}%
\pgfusepath{clip}%
\pgfsetbuttcap%
\pgfsetroundjoin%
\definecolor{currentfill}{rgb}{0.280894,0.078907,0.402329}%
\pgfsetfillcolor{currentfill}%
\pgfsetfillopacity{0.700000}%
\pgfsetlinewidth{0.000000pt}%
\definecolor{currentstroke}{rgb}{0.000000,0.000000,0.000000}%
\pgfsetstrokecolor{currentstroke}%
\pgfsetdash{}{0pt}%
\pgfpathmoveto{\pgfqpoint{4.364261in}{1.866963in}}%
\pgfpathlineto{\pgfqpoint{4.378255in}{1.867231in}}%
\pgfpathlineto{\pgfqpoint{4.392259in}{1.867600in}}%
\pgfpathlineto{\pgfqpoint{4.406272in}{1.868068in}}%
\pgfpathlineto{\pgfqpoint{4.420294in}{1.868636in}}%
\pgfpathlineto{\pgfqpoint{4.412366in}{1.857394in}}%
\pgfpathlineto{\pgfqpoint{4.404432in}{1.846191in}}%
\pgfpathlineto{\pgfqpoint{4.396494in}{1.835030in}}%
\pgfpathlineto{\pgfqpoint{4.388551in}{1.823913in}}%
\pgfpathlineto{\pgfqpoint{4.374523in}{1.823685in}}%
\pgfpathlineto{\pgfqpoint{4.360503in}{1.823557in}}%
\pgfpathlineto{\pgfqpoint{4.346493in}{1.823529in}}%
\pgfpathlineto{\pgfqpoint{4.332493in}{1.823600in}}%
\pgfpathlineto{\pgfqpoint{4.340442in}{1.834370in}}%
\pgfpathlineto{\pgfqpoint{4.348386in}{1.845189in}}%
\pgfpathlineto{\pgfqpoint{4.356326in}{1.856054in}}%
\pgfpathlineto{\pgfqpoint{4.364261in}{1.866963in}}%
\pgfpathclose%
\pgfusepath{fill}%
\end{pgfscope}%
\begin{pgfscope}%
\pgfpathrectangle{\pgfqpoint{1.150000in}{0.150000in}}{\pgfqpoint{5.700000in}{5.700000in}}%
\pgfusepath{clip}%
\pgfsetbuttcap%
\pgfsetroundjoin%
\definecolor{currentfill}{rgb}{0.282656,0.100196,0.422160}%
\pgfsetfillcolor{currentfill}%
\pgfsetfillopacity{0.700000}%
\pgfsetlinewidth{0.000000pt}%
\definecolor{currentstroke}{rgb}{0.000000,0.000000,0.000000}%
\pgfsetstrokecolor{currentstroke}%
\pgfsetdash{}{0pt}%
\pgfpathmoveto{\pgfqpoint{4.451961in}{1.913928in}}%
\pgfpathlineto{\pgfqpoint{4.465988in}{1.914918in}}%
\pgfpathlineto{\pgfqpoint{4.480025in}{1.916007in}}%
\pgfpathlineto{\pgfqpoint{4.494072in}{1.917196in}}%
\pgfpathlineto{\pgfqpoint{4.508129in}{1.918485in}}%
\pgfpathlineto{\pgfqpoint{4.500224in}{1.906803in}}%
\pgfpathlineto{\pgfqpoint{4.492315in}{1.895144in}}%
\pgfpathlineto{\pgfqpoint{4.484400in}{1.883509in}}%
\pgfpathlineto{\pgfqpoint{4.476482in}{1.871902in}}%
\pgfpathlineto{\pgfqpoint{4.462420in}{1.870937in}}%
\pgfpathlineto{\pgfqpoint{4.448368in}{1.870071in}}%
\pgfpathlineto{\pgfqpoint{4.434326in}{1.869304in}}%
\pgfpathlineto{\pgfqpoint{4.420294in}{1.868636in}}%
\pgfpathlineto{\pgfqpoint{4.428218in}{1.879913in}}%
\pgfpathlineto{\pgfqpoint{4.436137in}{1.891223in}}%
\pgfpathlineto{\pgfqpoint{4.444051in}{1.902562in}}%
\pgfpathlineto{\pgfqpoint{4.451961in}{1.913928in}}%
\pgfpathclose%
\pgfusepath{fill}%
\end{pgfscope}%
\begin{pgfscope}%
\pgfpathrectangle{\pgfqpoint{1.150000in}{0.150000in}}{\pgfqpoint{5.700000in}{5.700000in}}%
\pgfusepath{clip}%
\pgfsetbuttcap%
\pgfsetroundjoin%
\definecolor{currentfill}{rgb}{0.268510,0.009605,0.335427}%
\pgfsetfillcolor{currentfill}%
\pgfsetfillopacity{0.700000}%
\pgfsetlinewidth{0.000000pt}%
\definecolor{currentstroke}{rgb}{0.000000,0.000000,0.000000}%
\pgfsetstrokecolor{currentstroke}%
\pgfsetdash{}{0pt}%
\pgfpathmoveto{\pgfqpoint{3.957855in}{1.744600in}}%
\pgfpathlineto{\pgfqpoint{3.971730in}{1.741375in}}%
\pgfpathlineto{\pgfqpoint{3.985611in}{1.738253in}}%
\pgfpathlineto{\pgfqpoint{3.999499in}{1.735235in}}%
\pgfpathlineto{\pgfqpoint{4.013393in}{1.732319in}}%
\pgfpathlineto{\pgfqpoint{4.005333in}{1.724129in}}%
\pgfpathlineto{\pgfqpoint{3.997266in}{1.716057in}}%
\pgfpathlineto{\pgfqpoint{3.989193in}{1.708107in}}%
\pgfpathlineto{\pgfqpoint{3.981113in}{1.700284in}}%
\pgfpathlineto{\pgfqpoint{3.967205in}{1.703610in}}%
\pgfpathlineto{\pgfqpoint{3.953303in}{1.707040in}}%
\pgfpathlineto{\pgfqpoint{3.939407in}{1.710572in}}%
\pgfpathlineto{\pgfqpoint{3.925518in}{1.714208in}}%
\pgfpathlineto{\pgfqpoint{3.933613in}{1.721613in}}%
\pgfpathlineto{\pgfqpoint{3.941700in}{1.729150in}}%
\pgfpathlineto{\pgfqpoint{3.949781in}{1.736813in}}%
\pgfpathlineto{\pgfqpoint{3.957855in}{1.744600in}}%
\pgfpathclose%
\pgfusepath{fill}%
\end{pgfscope}%
\begin{pgfscope}%
\pgfpathrectangle{\pgfqpoint{1.150000in}{0.150000in}}{\pgfqpoint{5.700000in}{5.700000in}}%
\pgfusepath{clip}%
\pgfsetbuttcap%
\pgfsetroundjoin%
\definecolor{currentfill}{rgb}{0.270595,0.214069,0.507052}%
\pgfsetfillcolor{currentfill}%
\pgfsetfillopacity{0.700000}%
\pgfsetlinewidth{0.000000pt}%
\definecolor{currentstroke}{rgb}{0.000000,0.000000,0.000000}%
\pgfsetstrokecolor{currentstroke}%
\pgfsetdash{}{0pt}%
\pgfpathmoveto{\pgfqpoint{2.963387in}{2.148745in}}%
\pgfpathlineto{\pgfqpoint{2.977211in}{2.136289in}}%
\pgfpathlineto{\pgfqpoint{2.991034in}{2.123969in}}%
\pgfpathlineto{\pgfqpoint{3.004857in}{2.111782in}}%
\pgfpathlineto{\pgfqpoint{3.018679in}{2.099728in}}%
\pgfpathlineto{\pgfqpoint{3.010034in}{2.102256in}}%
\pgfpathlineto{\pgfqpoint{3.001373in}{2.105067in}}%
\pgfpathlineto{\pgfqpoint{2.992696in}{2.108168in}}%
\pgfpathlineto{\pgfqpoint{2.984003in}{2.111563in}}%
\pgfpathlineto{\pgfqpoint{2.970141in}{2.124137in}}%
\pgfpathlineto{\pgfqpoint{2.956277in}{2.136845in}}%
\pgfpathlineto{\pgfqpoint{2.942413in}{2.149687in}}%
\pgfpathlineto{\pgfqpoint{2.928548in}{2.162664in}}%
\pgfpathlineto{\pgfqpoint{2.937283in}{2.158740in}}%
\pgfpathlineto{\pgfqpoint{2.946001in}{2.155115in}}%
\pgfpathlineto{\pgfqpoint{2.954702in}{2.151786in}}%
\pgfpathlineto{\pgfqpoint{2.963387in}{2.148745in}}%
\pgfpathclose%
\pgfusepath{fill}%
\end{pgfscope}%
\begin{pgfscope}%
\pgfpathrectangle{\pgfqpoint{1.150000in}{0.150000in}}{\pgfqpoint{5.700000in}{5.700000in}}%
\pgfusepath{clip}%
\pgfsetbuttcap%
\pgfsetroundjoin%
\definecolor{currentfill}{rgb}{0.253935,0.265254,0.529983}%
\pgfsetfillcolor{currentfill}%
\pgfsetfillopacity{0.700000}%
\pgfsetlinewidth{0.000000pt}%
\definecolor{currentstroke}{rgb}{0.000000,0.000000,0.000000}%
\pgfsetstrokecolor{currentstroke}%
\pgfsetdash{}{0pt}%
\pgfpathmoveto{\pgfqpoint{4.922522in}{2.257505in}}%
\pgfpathlineto{\pgfqpoint{4.936758in}{2.262043in}}%
\pgfpathlineto{\pgfqpoint{4.951007in}{2.266679in}}%
\pgfpathlineto{\pgfqpoint{4.965269in}{2.271416in}}%
\pgfpathlineto{\pgfqpoint{4.979543in}{2.276251in}}%
\pgfpathlineto{\pgfqpoint{4.971768in}{2.263782in}}%
\pgfpathlineto{\pgfqpoint{4.963987in}{2.251253in}}%
\pgfpathlineto{\pgfqpoint{4.956200in}{2.238667in}}%
\pgfpathlineto{\pgfqpoint{4.948408in}{2.226027in}}%
\pgfpathlineto{\pgfqpoint{4.934133in}{2.221407in}}%
\pgfpathlineto{\pgfqpoint{4.919870in}{2.216887in}}%
\pgfpathlineto{\pgfqpoint{4.905620in}{2.212466in}}%
\pgfpathlineto{\pgfqpoint{4.891382in}{2.208144in}}%
\pgfpathlineto{\pgfqpoint{4.899175in}{2.220562in}}%
\pgfpathlineto{\pgfqpoint{4.906963in}{2.232929in}}%
\pgfpathlineto{\pgfqpoint{4.914745in}{2.245244in}}%
\pgfpathlineto{\pgfqpoint{4.922522in}{2.257505in}}%
\pgfpathclose%
\pgfusepath{fill}%
\end{pgfscope}%
\begin{pgfscope}%
\pgfpathrectangle{\pgfqpoint{1.150000in}{0.150000in}}{\pgfqpoint{5.700000in}{5.700000in}}%
\pgfusepath{clip}%
\pgfsetbuttcap%
\pgfsetroundjoin%
\definecolor{currentfill}{rgb}{0.143343,0.522773,0.556295}%
\pgfsetfillcolor{currentfill}%
\pgfsetfillopacity{0.700000}%
\pgfsetlinewidth{0.000000pt}%
\definecolor{currentstroke}{rgb}{0.000000,0.000000,0.000000}%
\pgfsetstrokecolor{currentstroke}%
\pgfsetdash{}{0pt}%
\pgfpathmoveto{\pgfqpoint{2.295982in}{2.925836in}}%
\pgfpathlineto{\pgfqpoint{2.310023in}{2.905575in}}%
\pgfpathlineto{\pgfqpoint{2.324056in}{2.885512in}}%
\pgfpathlineto{\pgfqpoint{2.338080in}{2.865645in}}%
\pgfpathlineto{\pgfqpoint{2.352096in}{2.845971in}}%
\pgfpathlineto{\pgfqpoint{2.342876in}{2.854829in}}%
\pgfpathlineto{\pgfqpoint{2.333633in}{2.864038in}}%
\pgfpathlineto{\pgfqpoint{2.324365in}{2.873607in}}%
\pgfpathlineto{\pgfqpoint{2.315072in}{2.883539in}}%
\pgfpathlineto{\pgfqpoint{2.300998in}{2.903770in}}%
\pgfpathlineto{\pgfqpoint{2.286915in}{2.924197in}}%
\pgfpathlineto{\pgfqpoint{2.272823in}{2.944820in}}%
\pgfpathlineto{\pgfqpoint{2.258722in}{2.965642in}}%
\pgfpathlineto{\pgfqpoint{2.268075in}{2.955141in}}%
\pgfpathlineto{\pgfqpoint{2.277402in}{2.945010in}}%
\pgfpathlineto{\pgfqpoint{2.286704in}{2.935244in}}%
\pgfpathlineto{\pgfqpoint{2.295982in}{2.925836in}}%
\pgfpathclose%
\pgfusepath{fill}%
\end{pgfscope}%
\begin{pgfscope}%
\pgfpathrectangle{\pgfqpoint{1.150000in}{0.150000in}}{\pgfqpoint{5.700000in}{5.700000in}}%
\pgfusepath{clip}%
\pgfsetbuttcap%
\pgfsetroundjoin%
\definecolor{currentfill}{rgb}{0.136408,0.541173,0.554483}%
\pgfsetfillcolor{currentfill}%
\pgfsetfillopacity{0.700000}%
\pgfsetlinewidth{0.000000pt}%
\definecolor{currentstroke}{rgb}{0.000000,0.000000,0.000000}%
\pgfsetstrokecolor{currentstroke}%
\pgfsetdash{}{0pt}%
\pgfpathmoveto{\pgfqpoint{5.719995in}{2.954322in}}%
\pgfpathlineto{\pgfqpoint{5.734667in}{2.963194in}}%
\pgfpathlineto{\pgfqpoint{5.749355in}{2.972167in}}%
\pgfpathlineto{\pgfqpoint{5.764060in}{2.981240in}}%
\pgfpathlineto{\pgfqpoint{5.778782in}{2.990416in}}%
\pgfpathlineto{\pgfqpoint{5.771352in}{2.981668in}}%
\pgfpathlineto{\pgfqpoint{5.763912in}{2.972789in}}%
\pgfpathlineto{\pgfqpoint{5.756462in}{2.963777in}}%
\pgfpathlineto{\pgfqpoint{5.749004in}{2.954632in}}%
\pgfpathlineto{\pgfqpoint{5.734275in}{2.945462in}}%
\pgfpathlineto{\pgfqpoint{5.719564in}{2.936393in}}%
\pgfpathlineto{\pgfqpoint{5.704869in}{2.927425in}}%
\pgfpathlineto{\pgfqpoint{5.690190in}{2.918559in}}%
\pgfpathlineto{\pgfqpoint{5.697655in}{2.927691in}}%
\pgfpathlineto{\pgfqpoint{5.705111in}{2.936696in}}%
\pgfpathlineto{\pgfqpoint{5.712558in}{2.945573in}}%
\pgfpathlineto{\pgfqpoint{5.719995in}{2.954322in}}%
\pgfpathclose%
\pgfusepath{fill}%
\end{pgfscope}%
\begin{pgfscope}%
\pgfpathrectangle{\pgfqpoint{1.150000in}{0.150000in}}{\pgfqpoint{5.700000in}{5.700000in}}%
\pgfusepath{clip}%
\pgfsetbuttcap%
\pgfsetroundjoin%
\definecolor{currentfill}{rgb}{0.277941,0.056324,0.381191}%
\pgfsetfillcolor{currentfill}%
\pgfsetfillopacity{0.700000}%
\pgfsetlinewidth{0.000000pt}%
\definecolor{currentstroke}{rgb}{0.000000,0.000000,0.000000}%
\pgfsetstrokecolor{currentstroke}%
\pgfsetdash{}{0pt}%
\pgfpathmoveto{\pgfqpoint{4.276579in}{1.824886in}}%
\pgfpathlineto{\pgfqpoint{4.290544in}{1.824414in}}%
\pgfpathlineto{\pgfqpoint{4.304518in}{1.824043in}}%
\pgfpathlineto{\pgfqpoint{4.318501in}{1.823771in}}%
\pgfpathlineto{\pgfqpoint{4.332493in}{1.823600in}}%
\pgfpathlineto{\pgfqpoint{4.324538in}{1.812884in}}%
\pgfpathlineto{\pgfqpoint{4.316579in}{1.802223in}}%
\pgfpathlineto{\pgfqpoint{4.308615in}{1.791622in}}%
\pgfpathlineto{\pgfqpoint{4.300646in}{1.781084in}}%
\pgfpathlineto{\pgfqpoint{4.286647in}{1.781613in}}%
\pgfpathlineto{\pgfqpoint{4.272656in}{1.782242in}}%
\pgfpathlineto{\pgfqpoint{4.258674in}{1.782972in}}%
\pgfpathlineto{\pgfqpoint{4.244701in}{1.783801in}}%
\pgfpathlineto{\pgfqpoint{4.252678in}{1.793974in}}%
\pgfpathlineto{\pgfqpoint{4.260650in}{1.804215in}}%
\pgfpathlineto{\pgfqpoint{4.268617in}{1.814520in}}%
\pgfpathlineto{\pgfqpoint{4.276579in}{1.824886in}}%
\pgfpathclose%
\pgfusepath{fill}%
\end{pgfscope}%
\begin{pgfscope}%
\pgfpathrectangle{\pgfqpoint{1.150000in}{0.150000in}}{\pgfqpoint{5.700000in}{5.700000in}}%
\pgfusepath{clip}%
\pgfsetbuttcap%
\pgfsetroundjoin%
\definecolor{currentfill}{rgb}{0.283072,0.130895,0.449241}%
\pgfsetfillcolor{currentfill}%
\pgfsetfillopacity{0.700000}%
\pgfsetlinewidth{0.000000pt}%
\definecolor{currentstroke}{rgb}{0.000000,0.000000,0.000000}%
\pgfsetstrokecolor{currentstroke}%
\pgfsetdash{}{0pt}%
\pgfpathmoveto{\pgfqpoint{4.539701in}{1.965371in}}%
\pgfpathlineto{\pgfqpoint{4.553764in}{1.967063in}}%
\pgfpathlineto{\pgfqpoint{4.567837in}{1.968855in}}%
\pgfpathlineto{\pgfqpoint{4.581922in}{1.970746in}}%
\pgfpathlineto{\pgfqpoint{4.596016in}{1.972737in}}%
\pgfpathlineto{\pgfqpoint{4.588134in}{1.960699in}}%
\pgfpathlineto{\pgfqpoint{4.580247in}{1.948666in}}%
\pgfpathlineto{\pgfqpoint{4.572356in}{1.936643in}}%
\pgfpathlineto{\pgfqpoint{4.564459in}{1.924630in}}%
\pgfpathlineto{\pgfqpoint{4.550361in}{1.922945in}}%
\pgfpathlineto{\pgfqpoint{4.536273in}{1.921359in}}%
\pgfpathlineto{\pgfqpoint{4.522196in}{1.919873in}}%
\pgfpathlineto{\pgfqpoint{4.508129in}{1.918485in}}%
\pgfpathlineto{\pgfqpoint{4.516029in}{1.930185in}}%
\pgfpathlineto{\pgfqpoint{4.523924in}{1.941901in}}%
\pgfpathlineto{\pgfqpoint{4.531815in}{1.953631in}}%
\pgfpathlineto{\pgfqpoint{4.539701in}{1.965371in}}%
\pgfpathclose%
\pgfusepath{fill}%
\end{pgfscope}%
\begin{pgfscope}%
\pgfpathrectangle{\pgfqpoint{1.150000in}{0.150000in}}{\pgfqpoint{5.700000in}{5.700000in}}%
\pgfusepath{clip}%
\pgfsetbuttcap%
\pgfsetroundjoin%
\definecolor{currentfill}{rgb}{0.276022,0.044167,0.370164}%
\pgfsetfillcolor{currentfill}%
\pgfsetfillopacity{0.700000}%
\pgfsetlinewidth{0.000000pt}%
\definecolor{currentstroke}{rgb}{0.000000,0.000000,0.000000}%
\pgfsetstrokecolor{currentstroke}%
\pgfsetdash{}{0pt}%
\pgfpathmoveto{\pgfqpoint{3.527899in}{1.800260in}}%
\pgfpathlineto{\pgfqpoint{3.541709in}{1.793172in}}%
\pgfpathlineto{\pgfqpoint{3.555521in}{1.786196in}}%
\pgfpathlineto{\pgfqpoint{3.569338in}{1.779331in}}%
\pgfpathlineto{\pgfqpoint{3.583159in}{1.772577in}}%
\pgfpathlineto{\pgfqpoint{3.574895in}{1.768840in}}%
\pgfpathlineto{\pgfqpoint{3.566621in}{1.765300in}}%
\pgfpathlineto{\pgfqpoint{3.558338in}{1.761964in}}%
\pgfpathlineto{\pgfqpoint{3.550044in}{1.758834in}}%
\pgfpathlineto{\pgfqpoint{3.536199in}{1.766057in}}%
\pgfpathlineto{\pgfqpoint{3.522357in}{1.773390in}}%
\pgfpathlineto{\pgfqpoint{3.508519in}{1.780835in}}%
\pgfpathlineto{\pgfqpoint{3.494684in}{1.788391in}}%
\pgfpathlineto{\pgfqpoint{3.503003in}{1.791044in}}%
\pgfpathlineto{\pgfqpoint{3.511312in}{1.793910in}}%
\pgfpathlineto{\pgfqpoint{3.519611in}{1.796983in}}%
\pgfpathlineto{\pgfqpoint{3.527899in}{1.800260in}}%
\pgfpathclose%
\pgfusepath{fill}%
\end{pgfscope}%
\begin{pgfscope}%
\pgfpathrectangle{\pgfqpoint{1.150000in}{0.150000in}}{\pgfqpoint{5.700000in}{5.700000in}}%
\pgfusepath{clip}%
\pgfsetbuttcap%
\pgfsetroundjoin%
\definecolor{currentfill}{rgb}{0.204903,0.375746,0.553533}%
\pgfsetfillcolor{currentfill}%
\pgfsetfillopacity{0.700000}%
\pgfsetlinewidth{0.000000pt}%
\definecolor{currentstroke}{rgb}{0.000000,0.000000,0.000000}%
\pgfsetstrokecolor{currentstroke}%
\pgfsetdash{}{0pt}%
\pgfpathmoveto{\pgfqpoint{5.217805in}{2.514090in}}%
\pgfpathlineto{\pgfqpoint{5.232196in}{2.520500in}}%
\pgfpathlineto{\pgfqpoint{5.246603in}{2.527010in}}%
\pgfpathlineto{\pgfqpoint{5.261023in}{2.533620in}}%
\pgfpathlineto{\pgfqpoint{5.275458in}{2.540330in}}%
\pgfpathlineto{\pgfqpoint{5.267783in}{2.528617in}}%
\pgfpathlineto{\pgfqpoint{5.260101in}{2.516808in}}%
\pgfpathlineto{\pgfqpoint{5.252412in}{2.504902in}}%
\pgfpathlineto{\pgfqpoint{5.244717in}{2.492903in}}%
\pgfpathlineto{\pgfqpoint{5.230280in}{2.486334in}}%
\pgfpathlineto{\pgfqpoint{5.215858in}{2.479866in}}%
\pgfpathlineto{\pgfqpoint{5.201450in}{2.473497in}}%
\pgfpathlineto{\pgfqpoint{5.187057in}{2.467229in}}%
\pgfpathlineto{\pgfqpoint{5.194753in}{2.479079in}}%
\pgfpathlineto{\pgfqpoint{5.202444in}{2.490841in}}%
\pgfpathlineto{\pgfqpoint{5.210127in}{2.502511in}}%
\pgfpathlineto{\pgfqpoint{5.217805in}{2.514090in}}%
\pgfpathclose%
\pgfusepath{fill}%
\end{pgfscope}%
\begin{pgfscope}%
\pgfpathrectangle{\pgfqpoint{1.150000in}{0.150000in}}{\pgfqpoint{5.700000in}{5.700000in}}%
\pgfusepath{clip}%
\pgfsetbuttcap%
\pgfsetroundjoin%
\definecolor{currentfill}{rgb}{0.281924,0.089666,0.412415}%
\pgfsetfillcolor{currentfill}%
\pgfsetfillopacity{0.700000}%
\pgfsetlinewidth{0.000000pt}%
\definecolor{currentstroke}{rgb}{0.000000,0.000000,0.000000}%
\pgfsetstrokecolor{currentstroke}%
\pgfsetdash{}{0pt}%
\pgfpathmoveto{\pgfqpoint{3.328888in}{1.887961in}}%
\pgfpathlineto{\pgfqpoint{3.342691in}{1.879026in}}%
\pgfpathlineto{\pgfqpoint{3.356496in}{1.870208in}}%
\pgfpathlineto{\pgfqpoint{3.370303in}{1.861507in}}%
\pgfpathlineto{\pgfqpoint{3.384112in}{1.852923in}}%
\pgfpathlineto{\pgfqpoint{3.375727in}{1.851451in}}%
\pgfpathlineto{\pgfqpoint{3.367329in}{1.850210in}}%
\pgfpathlineto{\pgfqpoint{3.358920in}{1.849207in}}%
\pgfpathlineto{\pgfqpoint{3.350498in}{1.848446in}}%
\pgfpathlineto{\pgfqpoint{3.336658in}{1.857520in}}%
\pgfpathlineto{\pgfqpoint{3.322821in}{1.866712in}}%
\pgfpathlineto{\pgfqpoint{3.308985in}{1.876020in}}%
\pgfpathlineto{\pgfqpoint{3.295151in}{1.885447in}}%
\pgfpathlineto{\pgfqpoint{3.303604in}{1.885710in}}%
\pgfpathlineto{\pgfqpoint{3.312044in}{1.886220in}}%
\pgfpathlineto{\pgfqpoint{3.320472in}{1.886972in}}%
\pgfpathlineto{\pgfqpoint{3.328888in}{1.887961in}}%
\pgfpathclose%
\pgfusepath{fill}%
\end{pgfscope}%
\begin{pgfscope}%
\pgfpathrectangle{\pgfqpoint{1.150000in}{0.150000in}}{\pgfqpoint{5.700000in}{5.700000in}}%
\pgfusepath{clip}%
\pgfsetbuttcap%
\pgfsetroundjoin%
\definecolor{currentfill}{rgb}{0.281412,0.155834,0.469201}%
\pgfsetfillcolor{currentfill}%
\pgfsetfillopacity{0.700000}%
\pgfsetlinewidth{0.000000pt}%
\definecolor{currentstroke}{rgb}{0.000000,0.000000,0.000000}%
\pgfsetstrokecolor{currentstroke}%
\pgfsetdash{}{0pt}%
\pgfpathmoveto{\pgfqpoint{4.627499in}{2.020893in}}%
\pgfpathlineto{\pgfqpoint{4.641601in}{2.023270in}}%
\pgfpathlineto{\pgfqpoint{4.655714in}{2.025746in}}%
\pgfpathlineto{\pgfqpoint{4.669839in}{2.028321in}}%
\pgfpathlineto{\pgfqpoint{4.683974in}{2.030995in}}%
\pgfpathlineto{\pgfqpoint{4.676114in}{2.018680in}}%
\pgfpathlineto{\pgfqpoint{4.668248in}{2.006356in}}%
\pgfpathlineto{\pgfqpoint{4.660378in}{1.994024in}}%
\pgfpathlineto{\pgfqpoint{4.652504in}{1.981689in}}%
\pgfpathlineto{\pgfqpoint{4.638365in}{1.979302in}}%
\pgfpathlineto{\pgfqpoint{4.624238in}{1.977015in}}%
\pgfpathlineto{\pgfqpoint{4.610122in}{1.974826in}}%
\pgfpathlineto{\pgfqpoint{4.596016in}{1.972737in}}%
\pgfpathlineto{\pgfqpoint{4.603894in}{1.984778in}}%
\pgfpathlineto{\pgfqpoint{4.611767in}{1.996819in}}%
\pgfpathlineto{\pgfqpoint{4.619635in}{2.008858in}}%
\pgfpathlineto{\pgfqpoint{4.627499in}{2.020893in}}%
\pgfpathclose%
\pgfusepath{fill}%
\end{pgfscope}%
\begin{pgfscope}%
\pgfpathrectangle{\pgfqpoint{1.150000in}{0.150000in}}{\pgfqpoint{5.700000in}{5.700000in}}%
\pgfusepath{clip}%
\pgfsetbuttcap%
\pgfsetroundjoin%
\definecolor{currentfill}{rgb}{0.274952,0.037752,0.364543}%
\pgfsetfillcolor{currentfill}%
\pgfsetfillopacity{0.700000}%
\pgfsetlinewidth{0.000000pt}%
\definecolor{currentstroke}{rgb}{0.000000,0.000000,0.000000}%
\pgfsetstrokecolor{currentstroke}%
\pgfsetdash{}{0pt}%
\pgfpathmoveto{\pgfqpoint{4.188892in}{1.788123in}}%
\pgfpathlineto{\pgfqpoint{4.202832in}{1.786892in}}%
\pgfpathlineto{\pgfqpoint{4.216780in}{1.785761in}}%
\pgfpathlineto{\pgfqpoint{4.230736in}{1.784730in}}%
\pgfpathlineto{\pgfqpoint{4.244701in}{1.783801in}}%
\pgfpathlineto{\pgfqpoint{4.236719in}{1.773698in}}%
\pgfpathlineto{\pgfqpoint{4.228731in}{1.763670in}}%
\pgfpathlineto{\pgfqpoint{4.220738in}{1.753721in}}%
\pgfpathlineto{\pgfqpoint{4.212740in}{1.743852in}}%
\pgfpathlineto{\pgfqpoint{4.198767in}{1.745158in}}%
\pgfpathlineto{\pgfqpoint{4.184801in}{1.746563in}}%
\pgfpathlineto{\pgfqpoint{4.170843in}{1.748069in}}%
\pgfpathlineto{\pgfqpoint{4.156894in}{1.749676in}}%
\pgfpathlineto{\pgfqpoint{4.164902in}{1.759162in}}%
\pgfpathlineto{\pgfqpoint{4.172904in}{1.768735in}}%
\pgfpathlineto{\pgfqpoint{4.180901in}{1.778389in}}%
\pgfpathlineto{\pgfqpoint{4.188892in}{1.788123in}}%
\pgfpathclose%
\pgfusepath{fill}%
\end{pgfscope}%
\begin{pgfscope}%
\pgfpathrectangle{\pgfqpoint{1.150000in}{0.150000in}}{\pgfqpoint{5.700000in}{5.700000in}}%
\pgfusepath{clip}%
\pgfsetbuttcap%
\pgfsetroundjoin%
\definecolor{currentfill}{rgb}{0.275191,0.194905,0.496005}%
\pgfsetfillcolor{currentfill}%
\pgfsetfillopacity{0.700000}%
\pgfsetlinewidth{0.000000pt}%
\definecolor{currentstroke}{rgb}{0.000000,0.000000,0.000000}%
\pgfsetstrokecolor{currentstroke}%
\pgfsetdash{}{0pt}%
\pgfpathmoveto{\pgfqpoint{3.018679in}{2.099728in}}%
\pgfpathlineto{\pgfqpoint{3.032500in}{2.087807in}}%
\pgfpathlineto{\pgfqpoint{3.046321in}{2.076017in}}%
\pgfpathlineto{\pgfqpoint{3.060141in}{2.064358in}}%
\pgfpathlineto{\pgfqpoint{3.073961in}{2.052828in}}%
\pgfpathlineto{\pgfqpoint{3.065355in}{2.054845in}}%
\pgfpathlineto{\pgfqpoint{3.056733in}{2.057140in}}%
\pgfpathlineto{\pgfqpoint{3.048096in}{2.059719in}}%
\pgfpathlineto{\pgfqpoint{3.039443in}{2.062588in}}%
\pgfpathlineto{\pgfqpoint{3.025584in}{2.074635in}}%
\pgfpathlineto{\pgfqpoint{3.011724in}{2.086813in}}%
\pgfpathlineto{\pgfqpoint{2.997864in}{2.099122in}}%
\pgfpathlineto{\pgfqpoint{2.984003in}{2.111563in}}%
\pgfpathlineto{\pgfqpoint{2.992696in}{2.108168in}}%
\pgfpathlineto{\pgfqpoint{3.001373in}{2.105067in}}%
\pgfpathlineto{\pgfqpoint{3.010034in}{2.102256in}}%
\pgfpathlineto{\pgfqpoint{3.018679in}{2.099728in}}%
\pgfpathclose%
\pgfusepath{fill}%
\end{pgfscope}%
\begin{pgfscope}%
\pgfpathrectangle{\pgfqpoint{1.150000in}{0.150000in}}{\pgfqpoint{5.700000in}{5.700000in}}%
\pgfusepath{clip}%
\pgfsetbuttcap%
\pgfsetroundjoin%
\definecolor{currentfill}{rgb}{0.132444,0.552216,0.553018}%
\pgfsetfillcolor{currentfill}%
\pgfsetfillopacity{0.700000}%
\pgfsetlinewidth{0.000000pt}%
\definecolor{currentstroke}{rgb}{0.000000,0.000000,0.000000}%
\pgfsetstrokecolor{currentstroke}%
\pgfsetdash{}{0pt}%
\pgfpathmoveto{\pgfqpoint{2.239727in}{3.008891in}}%
\pgfpathlineto{\pgfqpoint{2.253805in}{2.987821in}}%
\pgfpathlineto{\pgfqpoint{2.267873in}{2.966957in}}%
\pgfpathlineto{\pgfqpoint{2.281932in}{2.946296in}}%
\pgfpathlineto{\pgfqpoint{2.295982in}{2.925836in}}%
\pgfpathlineto{\pgfqpoint{2.286704in}{2.935244in}}%
\pgfpathlineto{\pgfqpoint{2.277402in}{2.945010in}}%
\pgfpathlineto{\pgfqpoint{2.268075in}{2.955141in}}%
\pgfpathlineto{\pgfqpoint{2.258722in}{2.965642in}}%
\pgfpathlineto{\pgfqpoint{2.244612in}{2.986664in}}%
\pgfpathlineto{\pgfqpoint{2.230492in}{3.007889in}}%
\pgfpathlineto{\pgfqpoint{2.216363in}{3.029318in}}%
\pgfpathlineto{\pgfqpoint{2.202223in}{3.050954in}}%
\pgfpathlineto{\pgfqpoint{2.211638in}{3.039879in}}%
\pgfpathlineto{\pgfqpoint{2.221027in}{3.029181in}}%
\pgfpathlineto{\pgfqpoint{2.230390in}{3.018853in}}%
\pgfpathlineto{\pgfqpoint{2.239727in}{3.008891in}}%
\pgfpathclose%
\pgfusepath{fill}%
\end{pgfscope}%
\begin{pgfscope}%
\pgfpathrectangle{\pgfqpoint{1.150000in}{0.150000in}}{\pgfqpoint{5.700000in}{5.700000in}}%
\pgfusepath{clip}%
\pgfsetbuttcap%
\pgfsetroundjoin%
\definecolor{currentfill}{rgb}{0.239346,0.300855,0.540844}%
\pgfsetfillcolor{currentfill}%
\pgfsetfillopacity{0.700000}%
\pgfsetlinewidth{0.000000pt}%
\definecolor{currentstroke}{rgb}{0.000000,0.000000,0.000000}%
\pgfsetstrokecolor{currentstroke}%
\pgfsetdash{}{0pt}%
\pgfpathmoveto{\pgfqpoint{5.010591in}{2.325504in}}%
\pgfpathlineto{\pgfqpoint{5.024877in}{2.330636in}}%
\pgfpathlineto{\pgfqpoint{5.039177in}{2.335868in}}%
\pgfpathlineto{\pgfqpoint{5.053490in}{2.341200in}}%
\pgfpathlineto{\pgfqpoint{5.067817in}{2.346630in}}%
\pgfpathlineto{\pgfqpoint{5.060064in}{2.334224in}}%
\pgfpathlineto{\pgfqpoint{5.052306in}{2.321747in}}%
\pgfpathlineto{\pgfqpoint{5.044543in}{2.309200in}}%
\pgfpathlineto{\pgfqpoint{5.036773in}{2.296587in}}%
\pgfpathlineto{\pgfqpoint{5.022446in}{2.291354in}}%
\pgfpathlineto{\pgfqpoint{5.008132in}{2.286220in}}%
\pgfpathlineto{\pgfqpoint{4.993831in}{2.281186in}}%
\pgfpathlineto{\pgfqpoint{4.979543in}{2.276251in}}%
\pgfpathlineto{\pgfqpoint{4.987314in}{2.288660in}}%
\pgfpathlineto{\pgfqpoint{4.995078in}{2.301006in}}%
\pgfpathlineto{\pgfqpoint{5.002837in}{2.313288in}}%
\pgfpathlineto{\pgfqpoint{5.010591in}{2.325504in}}%
\pgfpathclose%
\pgfusepath{fill}%
\end{pgfscope}%
\begin{pgfscope}%
\pgfpathrectangle{\pgfqpoint{1.150000in}{0.150000in}}{\pgfqpoint{5.700000in}{5.700000in}}%
\pgfusepath{clip}%
\pgfsetbuttcap%
\pgfsetroundjoin%
\definecolor{currentfill}{rgb}{0.162142,0.474838,0.558140}%
\pgfsetfillcolor{currentfill}%
\pgfsetfillopacity{0.700000}%
\pgfsetlinewidth{0.000000pt}%
\definecolor{currentstroke}{rgb}{0.000000,0.000000,0.000000}%
\pgfsetstrokecolor{currentstroke}%
\pgfsetdash{}{0pt}%
\pgfpathmoveto{\pgfqpoint{5.513230in}{2.774160in}}%
\pgfpathlineto{\pgfqpoint{5.527789in}{2.782148in}}%
\pgfpathlineto{\pgfqpoint{5.542365in}{2.790237in}}%
\pgfpathlineto{\pgfqpoint{5.556956in}{2.798427in}}%
\pgfpathlineto{\pgfqpoint{5.571564in}{2.806718in}}%
\pgfpathlineto{\pgfqpoint{5.564016in}{2.796491in}}%
\pgfpathlineto{\pgfqpoint{5.556460in}{2.786141in}}%
\pgfpathlineto{\pgfqpoint{5.548897in}{2.775670in}}%
\pgfpathlineto{\pgfqpoint{5.541325in}{2.765077in}}%
\pgfpathlineto{\pgfqpoint{5.526713in}{2.756851in}}%
\pgfpathlineto{\pgfqpoint{5.512118in}{2.748726in}}%
\pgfpathlineto{\pgfqpoint{5.497539in}{2.740701in}}%
\pgfpathlineto{\pgfqpoint{5.482975in}{2.732778in}}%
\pgfpathlineto{\pgfqpoint{5.490551in}{2.743299in}}%
\pgfpathlineto{\pgfqpoint{5.498118in}{2.753703in}}%
\pgfpathlineto{\pgfqpoint{5.505678in}{2.763991in}}%
\pgfpathlineto{\pgfqpoint{5.513230in}{2.774160in}}%
\pgfpathclose%
\pgfusepath{fill}%
\end{pgfscope}%
\begin{pgfscope}%
\pgfpathrectangle{\pgfqpoint{1.150000in}{0.150000in}}{\pgfqpoint{5.700000in}{5.700000in}}%
\pgfusepath{clip}%
\pgfsetbuttcap%
\pgfsetroundjoin%
\definecolor{currentfill}{rgb}{0.276194,0.190074,0.493001}%
\pgfsetfillcolor{currentfill}%
\pgfsetfillopacity{0.700000}%
\pgfsetlinewidth{0.000000pt}%
\definecolor{currentstroke}{rgb}{0.000000,0.000000,0.000000}%
\pgfsetstrokecolor{currentstroke}%
\pgfsetdash{}{0pt}%
\pgfpathmoveto{\pgfqpoint{4.715370in}{2.080109in}}%
\pgfpathlineto{\pgfqpoint{4.729514in}{2.083152in}}%
\pgfpathlineto{\pgfqpoint{4.743670in}{2.086294in}}%
\pgfpathlineto{\pgfqpoint{4.757838in}{2.089535in}}%
\pgfpathlineto{\pgfqpoint{4.772018in}{2.092876in}}%
\pgfpathlineto{\pgfqpoint{4.764178in}{2.080361in}}%
\pgfpathlineto{\pgfqpoint{4.756334in}{2.067822in}}%
\pgfpathlineto{\pgfqpoint{4.748485in}{2.055262in}}%
\pgfpathlineto{\pgfqpoint{4.740631in}{2.042682in}}%
\pgfpathlineto{\pgfqpoint{4.726449in}{2.039612in}}%
\pgfpathlineto{\pgfqpoint{4.712280in}{2.036641in}}%
\pgfpathlineto{\pgfqpoint{4.698121in}{2.033768in}}%
\pgfpathlineto{\pgfqpoint{4.683974in}{2.030995in}}%
\pgfpathlineto{\pgfqpoint{4.691830in}{2.043298in}}%
\pgfpathlineto{\pgfqpoint{4.699682in}{2.055586in}}%
\pgfpathlineto{\pgfqpoint{4.707528in}{2.067857in}}%
\pgfpathlineto{\pgfqpoint{4.715370in}{2.080109in}}%
\pgfpathclose%
\pgfusepath{fill}%
\end{pgfscope}%
\begin{pgfscope}%
\pgfpathrectangle{\pgfqpoint{1.150000in}{0.150000in}}{\pgfqpoint{5.700000in}{5.700000in}}%
\pgfusepath{clip}%
\pgfsetbuttcap%
\pgfsetroundjoin%
\definecolor{currentfill}{rgb}{0.127568,0.566949,0.550556}%
\pgfsetfillcolor{currentfill}%
\pgfsetfillopacity{0.700000}%
\pgfsetlinewidth{0.000000pt}%
\definecolor{currentstroke}{rgb}{0.000000,0.000000,0.000000}%
\pgfsetstrokecolor{currentstroke}%
\pgfsetdash{}{0pt}%
\pgfpathmoveto{\pgfqpoint{5.808410in}{3.024091in}}%
\pgfpathlineto{\pgfqpoint{5.823141in}{3.033353in}}%
\pgfpathlineto{\pgfqpoint{5.837889in}{3.042716in}}%
\pgfpathlineto{\pgfqpoint{5.852654in}{3.052181in}}%
\pgfpathlineto{\pgfqpoint{5.867437in}{3.061748in}}%
\pgfpathlineto{\pgfqpoint{5.860052in}{3.053548in}}%
\pgfpathlineto{\pgfqpoint{5.852658in}{3.045212in}}%
\pgfpathlineto{\pgfqpoint{5.845254in}{3.036741in}}%
\pgfpathlineto{\pgfqpoint{5.837840in}{3.028133in}}%
\pgfpathlineto{\pgfqpoint{5.823050in}{3.018551in}}%
\pgfpathlineto{\pgfqpoint{5.808277in}{3.009071in}}%
\pgfpathlineto{\pgfqpoint{5.793521in}{2.999693in}}%
\pgfpathlineto{\pgfqpoint{5.778782in}{2.990416in}}%
\pgfpathlineto{\pgfqpoint{5.786203in}{2.999031in}}%
\pgfpathlineto{\pgfqpoint{5.793615in}{3.007515in}}%
\pgfpathlineto{\pgfqpoint{5.801018in}{3.015868in}}%
\pgfpathlineto{\pgfqpoint{5.808410in}{3.024091in}}%
\pgfpathclose%
\pgfusepath{fill}%
\end{pgfscope}%
\begin{pgfscope}%
\pgfpathrectangle{\pgfqpoint{1.150000in}{0.150000in}}{\pgfqpoint{5.700000in}{5.700000in}}%
\pgfusepath{clip}%
\pgfsetbuttcap%
\pgfsetroundjoin%
\definecolor{currentfill}{rgb}{0.269944,0.014625,0.341379}%
\pgfsetfillcolor{currentfill}%
\pgfsetfillopacity{0.700000}%
\pgfsetlinewidth{0.000000pt}%
\definecolor{currentstroke}{rgb}{0.000000,0.000000,0.000000}%
\pgfsetstrokecolor{currentstroke}%
\pgfsetdash{}{0pt}%
\pgfpathmoveto{\pgfqpoint{3.726659in}{1.742911in}}%
\pgfpathlineto{\pgfqpoint{3.740497in}{1.737584in}}%
\pgfpathlineto{\pgfqpoint{3.754340in}{1.732365in}}%
\pgfpathlineto{\pgfqpoint{3.768189in}{1.727251in}}%
\pgfpathlineto{\pgfqpoint{3.782043in}{1.722245in}}%
\pgfpathlineto{\pgfqpoint{3.773878in}{1.716437in}}%
\pgfpathlineto{\pgfqpoint{3.765705in}{1.710795in}}%
\pgfpathlineto{\pgfqpoint{3.757524in}{1.705323in}}%
\pgfpathlineto{\pgfqpoint{3.749335in}{1.700026in}}%
\pgfpathlineto{\pgfqpoint{3.735462in}{1.705481in}}%
\pgfpathlineto{\pgfqpoint{3.721593in}{1.711042in}}%
\pgfpathlineto{\pgfqpoint{3.707729in}{1.716710in}}%
\pgfpathlineto{\pgfqpoint{3.693871in}{1.722485in}}%
\pgfpathlineto{\pgfqpoint{3.702080in}{1.727327in}}%
\pgfpathlineto{\pgfqpoint{3.710281in}{1.732348in}}%
\pgfpathlineto{\pgfqpoint{3.718474in}{1.737545in}}%
\pgfpathlineto{\pgfqpoint{3.726659in}{1.742911in}}%
\pgfpathclose%
\pgfusepath{fill}%
\end{pgfscope}%
\begin{pgfscope}%
\pgfpathrectangle{\pgfqpoint{1.150000in}{0.150000in}}{\pgfqpoint{5.700000in}{5.700000in}}%
\pgfusepath{clip}%
\pgfsetbuttcap%
\pgfsetroundjoin%
\definecolor{currentfill}{rgb}{0.271305,0.019942,0.347269}%
\pgfsetfillcolor{currentfill}%
\pgfsetfillopacity{0.700000}%
\pgfsetlinewidth{0.000000pt}%
\definecolor{currentstroke}{rgb}{0.000000,0.000000,0.000000}%
\pgfsetstrokecolor{currentstroke}%
\pgfsetdash{}{0pt}%
\pgfpathmoveto{\pgfqpoint{4.101173in}{1.757116in}}%
\pgfpathlineto{\pgfqpoint{4.115092in}{1.755104in}}%
\pgfpathlineto{\pgfqpoint{4.129018in}{1.753193in}}%
\pgfpathlineto{\pgfqpoint{4.142952in}{1.751384in}}%
\pgfpathlineto{\pgfqpoint{4.156894in}{1.749676in}}%
\pgfpathlineto{\pgfqpoint{4.148881in}{1.740280in}}%
\pgfpathlineto{\pgfqpoint{4.140862in}{1.730978in}}%
\pgfpathlineto{\pgfqpoint{4.132837in}{1.721773in}}%
\pgfpathlineto{\pgfqpoint{4.124807in}{1.712669in}}%
\pgfpathlineto{\pgfqpoint{4.110855in}{1.714770in}}%
\pgfpathlineto{\pgfqpoint{4.096910in}{1.716972in}}%
\pgfpathlineto{\pgfqpoint{4.082972in}{1.719276in}}%
\pgfpathlineto{\pgfqpoint{4.069042in}{1.721680in}}%
\pgfpathlineto{\pgfqpoint{4.077084in}{1.730384in}}%
\pgfpathlineto{\pgfqpoint{4.085119in}{1.739194in}}%
\pgfpathlineto{\pgfqpoint{4.093149in}{1.748106in}}%
\pgfpathlineto{\pgfqpoint{4.101173in}{1.757116in}}%
\pgfpathclose%
\pgfusepath{fill}%
\end{pgfscope}%
\begin{pgfscope}%
\pgfpathrectangle{\pgfqpoint{1.150000in}{0.150000in}}{\pgfqpoint{5.700000in}{5.700000in}}%
\pgfusepath{clip}%
\pgfsetbuttcap%
\pgfsetroundjoin%
\definecolor{currentfill}{rgb}{0.268510,0.009605,0.335427}%
\pgfsetfillcolor{currentfill}%
\pgfsetfillopacity{0.700000}%
\pgfsetlinewidth{0.000000pt}%
\definecolor{currentstroke}{rgb}{0.000000,0.000000,0.000000}%
\pgfsetstrokecolor{currentstroke}%
\pgfsetdash{}{0pt}%
\pgfpathmoveto{\pgfqpoint{3.870024in}{1.729788in}}%
\pgfpathlineto{\pgfqpoint{3.883889in}{1.725737in}}%
\pgfpathlineto{\pgfqpoint{3.897759in}{1.721790in}}%
\pgfpathlineto{\pgfqpoint{3.911636in}{1.717947in}}%
\pgfpathlineto{\pgfqpoint{3.925518in}{1.714208in}}%
\pgfpathlineto{\pgfqpoint{3.917417in}{1.706937in}}%
\pgfpathlineto{\pgfqpoint{3.909309in}{1.699805in}}%
\pgfpathlineto{\pgfqpoint{3.901194in}{1.692816in}}%
\pgfpathlineto{\pgfqpoint{3.893073in}{1.685975in}}%
\pgfpathlineto{\pgfqpoint{3.879174in}{1.690144in}}%
\pgfpathlineto{\pgfqpoint{3.865281in}{1.694416in}}%
\pgfpathlineto{\pgfqpoint{3.851394in}{1.698792in}}%
\pgfpathlineto{\pgfqpoint{3.837512in}{1.703273in}}%
\pgfpathlineto{\pgfqpoint{3.845651in}{1.709677in}}%
\pgfpathlineto{\pgfqpoint{3.853783in}{1.716234in}}%
\pgfpathlineto{\pgfqpoint{3.861907in}{1.722939in}}%
\pgfpathlineto{\pgfqpoint{3.870024in}{1.729788in}}%
\pgfpathclose%
\pgfusepath{fill}%
\end{pgfscope}%
\begin{pgfscope}%
\pgfpathrectangle{\pgfqpoint{1.150000in}{0.150000in}}{\pgfqpoint{5.700000in}{5.700000in}}%
\pgfusepath{clip}%
\pgfsetbuttcap%
\pgfsetroundjoin%
\definecolor{currentfill}{rgb}{0.278826,0.175490,0.483397}%
\pgfsetfillcolor{currentfill}%
\pgfsetfillopacity{0.700000}%
\pgfsetlinewidth{0.000000pt}%
\definecolor{currentstroke}{rgb}{0.000000,0.000000,0.000000}%
\pgfsetstrokecolor{currentstroke}%
\pgfsetdash{}{0pt}%
\pgfpathmoveto{\pgfqpoint{3.073961in}{2.052828in}}%
\pgfpathlineto{\pgfqpoint{3.087781in}{2.041428in}}%
\pgfpathlineto{\pgfqpoint{3.101602in}{2.030156in}}%
\pgfpathlineto{\pgfqpoint{3.115422in}{2.019012in}}%
\pgfpathlineto{\pgfqpoint{3.129243in}{2.007995in}}%
\pgfpathlineto{\pgfqpoint{3.120673in}{2.009503in}}%
\pgfpathlineto{\pgfqpoint{3.112089in}{2.011284in}}%
\pgfpathlineto{\pgfqpoint{3.103490in}{2.013343in}}%
\pgfpathlineto{\pgfqpoint{3.094876in}{2.015687in}}%
\pgfpathlineto{\pgfqpoint{3.081018in}{2.027220in}}%
\pgfpathlineto{\pgfqpoint{3.067160in}{2.038881in}}%
\pgfpathlineto{\pgfqpoint{3.053302in}{2.050670in}}%
\pgfpathlineto{\pgfqpoint{3.039443in}{2.062588in}}%
\pgfpathlineto{\pgfqpoint{3.048096in}{2.059719in}}%
\pgfpathlineto{\pgfqpoint{3.056733in}{2.057140in}}%
\pgfpathlineto{\pgfqpoint{3.065355in}{2.054845in}}%
\pgfpathlineto{\pgfqpoint{3.073961in}{2.052828in}}%
\pgfpathclose%
\pgfusepath{fill}%
\end{pgfscope}%
\begin{pgfscope}%
\pgfpathrectangle{\pgfqpoint{1.150000in}{0.150000in}}{\pgfqpoint{5.700000in}{5.700000in}}%
\pgfusepath{clip}%
\pgfsetbuttcap%
\pgfsetroundjoin%
\definecolor{currentfill}{rgb}{0.190631,0.407061,0.556089}%
\pgfsetfillcolor{currentfill}%
\pgfsetfillopacity{0.700000}%
\pgfsetlinewidth{0.000000pt}%
\definecolor{currentstroke}{rgb}{0.000000,0.000000,0.000000}%
\pgfsetstrokecolor{currentstroke}%
\pgfsetdash{}{0pt}%
\pgfpathmoveto{\pgfqpoint{5.306092in}{2.586195in}}%
\pgfpathlineto{\pgfqpoint{5.320539in}{2.593128in}}%
\pgfpathlineto{\pgfqpoint{5.335002in}{2.600161in}}%
\pgfpathlineto{\pgfqpoint{5.349479in}{2.607294in}}%
\pgfpathlineto{\pgfqpoint{5.363972in}{2.614527in}}%
\pgfpathlineto{\pgfqpoint{5.356326in}{2.603096in}}%
\pgfpathlineto{\pgfqpoint{5.348673in}{2.591559in}}%
\pgfpathlineto{\pgfqpoint{5.341013in}{2.579917in}}%
\pgfpathlineto{\pgfqpoint{5.333346in}{2.568172in}}%
\pgfpathlineto{\pgfqpoint{5.318852in}{2.561061in}}%
\pgfpathlineto{\pgfqpoint{5.304372in}{2.554051in}}%
\pgfpathlineto{\pgfqpoint{5.289908in}{2.547141in}}%
\pgfpathlineto{\pgfqpoint{5.275458in}{2.540330in}}%
\pgfpathlineto{\pgfqpoint{5.283127in}{2.551946in}}%
\pgfpathlineto{\pgfqpoint{5.290789in}{2.563462in}}%
\pgfpathlineto{\pgfqpoint{5.298444in}{2.574879in}}%
\pgfpathlineto{\pgfqpoint{5.306092in}{2.586195in}}%
\pgfpathclose%
\pgfusepath{fill}%
\end{pgfscope}%
\begin{pgfscope}%
\pgfpathrectangle{\pgfqpoint{1.150000in}{0.150000in}}{\pgfqpoint{5.700000in}{5.700000in}}%
\pgfusepath{clip}%
\pgfsetbuttcap%
\pgfsetroundjoin%
\definecolor{currentfill}{rgb}{0.121148,0.592739,0.544641}%
\pgfsetfillcolor{currentfill}%
\pgfsetfillopacity{0.700000}%
\pgfsetlinewidth{0.000000pt}%
\definecolor{currentstroke}{rgb}{0.000000,0.000000,0.000000}%
\pgfsetstrokecolor{currentstroke}%
\pgfsetdash{}{0pt}%
\pgfpathmoveto{\pgfqpoint{5.896877in}{3.093204in}}%
\pgfpathlineto{\pgfqpoint{5.911667in}{3.102837in}}%
\pgfpathlineto{\pgfqpoint{5.926475in}{3.112572in}}%
\pgfpathlineto{\pgfqpoint{5.941301in}{3.122409in}}%
\pgfpathlineto{\pgfqpoint{5.933963in}{3.114777in}}%
\pgfpathlineto{\pgfqpoint{5.926615in}{3.107009in}}%
\pgfpathlineto{\pgfqpoint{5.919256in}{3.099103in}}%
\pgfpathlineto{\pgfqpoint{5.911888in}{3.091059in}}%
\pgfpathlineto{\pgfqpoint{5.897053in}{3.081187in}}%
\pgfpathlineto{\pgfqpoint{5.882236in}{3.071416in}}%
\pgfpathlineto{\pgfqpoint{5.867437in}{3.061748in}}%
\pgfpathlineto{\pgfqpoint{5.874811in}{3.069813in}}%
\pgfpathlineto{\pgfqpoint{5.882176in}{3.077743in}}%
\pgfpathlineto{\pgfqpoint{5.889532in}{3.085540in}}%
\pgfpathlineto{\pgfqpoint{5.896877in}{3.093204in}}%
\pgfpathclose%
\pgfusepath{fill}%
\end{pgfscope}%
\begin{pgfscope}%
\pgfpathrectangle{\pgfqpoint{1.150000in}{0.150000in}}{\pgfqpoint{5.700000in}{5.700000in}}%
\pgfusepath{clip}%
\pgfsetbuttcap%
\pgfsetroundjoin%
\definecolor{currentfill}{rgb}{0.280894,0.078907,0.402329}%
\pgfsetfillcolor{currentfill}%
\pgfsetfillopacity{0.700000}%
\pgfsetlinewidth{0.000000pt}%
\definecolor{currentstroke}{rgb}{0.000000,0.000000,0.000000}%
\pgfsetstrokecolor{currentstroke}%
\pgfsetdash{}{0pt}%
\pgfpathmoveto{\pgfqpoint{3.384112in}{1.852923in}}%
\pgfpathlineto{\pgfqpoint{3.397924in}{1.844455in}}%
\pgfpathlineto{\pgfqpoint{3.411739in}{1.836103in}}%
\pgfpathlineto{\pgfqpoint{3.425556in}{1.827866in}}%
\pgfpathlineto{\pgfqpoint{3.439376in}{1.819744in}}%
\pgfpathlineto{\pgfqpoint{3.431019in}{1.817789in}}%
\pgfpathlineto{\pgfqpoint{3.422650in}{1.816061in}}%
\pgfpathlineto{\pgfqpoint{3.414271in}{1.814566in}}%
\pgfpathlineto{\pgfqpoint{3.405879in}{1.813307in}}%
\pgfpathlineto{\pgfqpoint{3.392031in}{1.821919in}}%
\pgfpathlineto{\pgfqpoint{3.378184in}{1.830646in}}%
\pgfpathlineto{\pgfqpoint{3.364340in}{1.839488in}}%
\pgfpathlineto{\pgfqpoint{3.350498in}{1.848446in}}%
\pgfpathlineto{\pgfqpoint{3.358920in}{1.849207in}}%
\pgfpathlineto{\pgfqpoint{3.367329in}{1.850210in}}%
\pgfpathlineto{\pgfqpoint{3.375727in}{1.851451in}}%
\pgfpathlineto{\pgfqpoint{3.384112in}{1.852923in}}%
\pgfpathclose%
\pgfusepath{fill}%
\end{pgfscope}%
\begin{pgfscope}%
\pgfpathrectangle{\pgfqpoint{1.150000in}{0.150000in}}{\pgfqpoint{5.700000in}{5.700000in}}%
\pgfusepath{clip}%
\pgfsetbuttcap%
\pgfsetroundjoin%
\definecolor{currentfill}{rgb}{0.273809,0.031497,0.358853}%
\pgfsetfillcolor{currentfill}%
\pgfsetfillopacity{0.700000}%
\pgfsetlinewidth{0.000000pt}%
\definecolor{currentstroke}{rgb}{0.000000,0.000000,0.000000}%
\pgfsetstrokecolor{currentstroke}%
\pgfsetdash{}{0pt}%
\pgfpathmoveto{\pgfqpoint{3.583159in}{1.772577in}}%
\pgfpathlineto{\pgfqpoint{3.596983in}{1.765933in}}%
\pgfpathlineto{\pgfqpoint{3.610811in}{1.759400in}}%
\pgfpathlineto{\pgfqpoint{3.624644in}{1.752976in}}%
\pgfpathlineto{\pgfqpoint{3.638480in}{1.746661in}}%
\pgfpathlineto{\pgfqpoint{3.630240in}{1.742462in}}%
\pgfpathlineto{\pgfqpoint{3.621990in}{1.738458in}}%
\pgfpathlineto{\pgfqpoint{3.613731in}{1.734651in}}%
\pgfpathlineto{\pgfqpoint{3.605462in}{1.731046in}}%
\pgfpathlineto{\pgfqpoint{3.591602in}{1.737829in}}%
\pgfpathlineto{\pgfqpoint{3.577746in}{1.744721in}}%
\pgfpathlineto{\pgfqpoint{3.563893in}{1.751723in}}%
\pgfpathlineto{\pgfqpoint{3.550044in}{1.758834in}}%
\pgfpathlineto{\pgfqpoint{3.558338in}{1.761964in}}%
\pgfpathlineto{\pgfqpoint{3.566621in}{1.765300in}}%
\pgfpathlineto{\pgfqpoint{3.574895in}{1.768840in}}%
\pgfpathlineto{\pgfqpoint{3.583159in}{1.772577in}}%
\pgfpathclose%
\pgfusepath{fill}%
\end{pgfscope}%
\begin{pgfscope}%
\pgfpathrectangle{\pgfqpoint{1.150000in}{0.150000in}}{\pgfqpoint{5.700000in}{5.700000in}}%
\pgfusepath{clip}%
\pgfsetbuttcap%
\pgfsetroundjoin%
\definecolor{currentfill}{rgb}{0.269308,0.218818,0.509577}%
\pgfsetfillcolor{currentfill}%
\pgfsetfillopacity{0.700000}%
\pgfsetlinewidth{0.000000pt}%
\definecolor{currentstroke}{rgb}{0.000000,0.000000,0.000000}%
\pgfsetstrokecolor{currentstroke}%
\pgfsetdash{}{0pt}%
\pgfpathmoveto{\pgfqpoint{4.803327in}{2.142646in}}%
\pgfpathlineto{\pgfqpoint{4.817517in}{2.146338in}}%
\pgfpathlineto{\pgfqpoint{4.831719in}{2.150128in}}%
\pgfpathlineto{\pgfqpoint{4.845933in}{2.154018in}}%
\pgfpathlineto{\pgfqpoint{4.860159in}{2.158006in}}%
\pgfpathlineto{\pgfqpoint{4.852341in}{2.145366in}}%
\pgfpathlineto{\pgfqpoint{4.844517in}{2.132688in}}%
\pgfpathlineto{\pgfqpoint{4.836689in}{2.119974in}}%
\pgfpathlineto{\pgfqpoint{4.828855in}{2.107227in}}%
\pgfpathlineto{\pgfqpoint{4.814628in}{2.103490in}}%
\pgfpathlineto{\pgfqpoint{4.800413in}{2.099853in}}%
\pgfpathlineto{\pgfqpoint{4.786209in}{2.096315in}}%
\pgfpathlineto{\pgfqpoint{4.772018in}{2.092876in}}%
\pgfpathlineto{\pgfqpoint{4.779853in}{2.105364in}}%
\pgfpathlineto{\pgfqpoint{4.787682in}{2.117823in}}%
\pgfpathlineto{\pgfqpoint{4.795507in}{2.130251in}}%
\pgfpathlineto{\pgfqpoint{4.803327in}{2.142646in}}%
\pgfpathclose%
\pgfusepath{fill}%
\end{pgfscope}%
\begin{pgfscope}%
\pgfpathrectangle{\pgfqpoint{1.150000in}{0.150000in}}{\pgfqpoint{5.700000in}{5.700000in}}%
\pgfusepath{clip}%
\pgfsetbuttcap%
\pgfsetroundjoin%
\definecolor{currentfill}{rgb}{0.223925,0.334994,0.548053}%
\pgfsetfillcolor{currentfill}%
\pgfsetfillopacity{0.700000}%
\pgfsetlinewidth{0.000000pt}%
\definecolor{currentstroke}{rgb}{0.000000,0.000000,0.000000}%
\pgfsetstrokecolor{currentstroke}%
\pgfsetdash{}{0pt}%
\pgfpathmoveto{\pgfqpoint{5.098768in}{2.395518in}}%
\pgfpathlineto{\pgfqpoint{5.113107in}{2.401227in}}%
\pgfpathlineto{\pgfqpoint{5.127460in}{2.407036in}}%
\pgfpathlineto{\pgfqpoint{5.141827in}{2.412945in}}%
\pgfpathlineto{\pgfqpoint{5.156207in}{2.418953in}}%
\pgfpathlineto{\pgfqpoint{5.148479in}{2.406673in}}%
\pgfpathlineto{\pgfqpoint{5.140745in}{2.394311in}}%
\pgfpathlineto{\pgfqpoint{5.133005in}{2.381869in}}%
\pgfpathlineto{\pgfqpoint{5.125259in}{2.369349in}}%
\pgfpathlineto{\pgfqpoint{5.110878in}{2.363520in}}%
\pgfpathlineto{\pgfqpoint{5.096511in}{2.357791in}}%
\pgfpathlineto{\pgfqpoint{5.082157in}{2.352161in}}%
\pgfpathlineto{\pgfqpoint{5.067817in}{2.346630in}}%
\pgfpathlineto{\pgfqpoint{5.075564in}{2.358965in}}%
\pgfpathlineto{\pgfqpoint{5.083304in}{2.371225in}}%
\pgfpathlineto{\pgfqpoint{5.091039in}{2.383410in}}%
\pgfpathlineto{\pgfqpoint{5.098768in}{2.395518in}}%
\pgfpathclose%
\pgfusepath{fill}%
\end{pgfscope}%
\begin{pgfscope}%
\pgfpathrectangle{\pgfqpoint{1.150000in}{0.150000in}}{\pgfqpoint{5.700000in}{5.700000in}}%
\pgfusepath{clip}%
\pgfsetbuttcap%
\pgfsetroundjoin%
\definecolor{currentfill}{rgb}{0.150476,0.504369,0.557430}%
\pgfsetfillcolor{currentfill}%
\pgfsetfillopacity{0.700000}%
\pgfsetlinewidth{0.000000pt}%
\definecolor{currentstroke}{rgb}{0.000000,0.000000,0.000000}%
\pgfsetstrokecolor{currentstroke}%
\pgfsetdash{}{0pt}%
\pgfpathmoveto{\pgfqpoint{5.601670in}{2.846397in}}%
\pgfpathlineto{\pgfqpoint{5.616289in}{2.854834in}}%
\pgfpathlineto{\pgfqpoint{5.630924in}{2.863371in}}%
\pgfpathlineto{\pgfqpoint{5.645575in}{2.872010in}}%
\pgfpathlineto{\pgfqpoint{5.660242in}{2.880750in}}%
\pgfpathlineto{\pgfqpoint{5.652733in}{2.870977in}}%
\pgfpathlineto{\pgfqpoint{5.645216in}{2.861076in}}%
\pgfpathlineto{\pgfqpoint{5.637690in}{2.851047in}}%
\pgfpathlineto{\pgfqpoint{5.630155in}{2.840891in}}%
\pgfpathlineto{\pgfqpoint{5.615482in}{2.832196in}}%
\pgfpathlineto{\pgfqpoint{5.600827in}{2.823602in}}%
\pgfpathlineto{\pgfqpoint{5.586187in}{2.815109in}}%
\pgfpathlineto{\pgfqpoint{5.571564in}{2.806718in}}%
\pgfpathlineto{\pgfqpoint{5.579103in}{2.816822in}}%
\pgfpathlineto{\pgfqpoint{5.586634in}{2.826804in}}%
\pgfpathlineto{\pgfqpoint{5.594156in}{2.836662in}}%
\pgfpathlineto{\pgfqpoint{5.601670in}{2.846397in}}%
\pgfpathclose%
\pgfusepath{fill}%
\end{pgfscope}%
\begin{pgfscope}%
\pgfpathrectangle{\pgfqpoint{1.150000in}{0.150000in}}{\pgfqpoint{5.700000in}{5.700000in}}%
\pgfusepath{clip}%
\pgfsetbuttcap%
\pgfsetroundjoin%
\definecolor{currentfill}{rgb}{0.280868,0.160771,0.472899}%
\pgfsetfillcolor{currentfill}%
\pgfsetfillopacity{0.700000}%
\pgfsetlinewidth{0.000000pt}%
\definecolor{currentstroke}{rgb}{0.000000,0.000000,0.000000}%
\pgfsetstrokecolor{currentstroke}%
\pgfsetdash{}{0pt}%
\pgfpathmoveto{\pgfqpoint{3.129243in}{2.007995in}}%
\pgfpathlineto{\pgfqpoint{3.143064in}{1.997104in}}%
\pgfpathlineto{\pgfqpoint{3.156885in}{1.986339in}}%
\pgfpathlineto{\pgfqpoint{3.170707in}{1.975699in}}%
\pgfpathlineto{\pgfqpoint{3.184530in}{1.965183in}}%
\pgfpathlineto{\pgfqpoint{3.175996in}{1.966183in}}%
\pgfpathlineto{\pgfqpoint{3.167449in}{1.967451in}}%
\pgfpathlineto{\pgfqpoint{3.158887in}{1.968992in}}%
\pgfpathlineto{\pgfqpoint{3.150311in}{1.970812in}}%
\pgfpathlineto{\pgfqpoint{3.136452in}{1.981843in}}%
\pgfpathlineto{\pgfqpoint{3.122593in}{1.992999in}}%
\pgfpathlineto{\pgfqpoint{3.108735in}{2.004280in}}%
\pgfpathlineto{\pgfqpoint{3.094876in}{2.015687in}}%
\pgfpathlineto{\pgfqpoint{3.103490in}{2.013343in}}%
\pgfpathlineto{\pgfqpoint{3.112089in}{2.011284in}}%
\pgfpathlineto{\pgfqpoint{3.120673in}{2.009503in}}%
\pgfpathlineto{\pgfqpoint{3.129243in}{2.007995in}}%
\pgfpathclose%
\pgfusepath{fill}%
\end{pgfscope}%
\begin{pgfscope}%
\pgfpathrectangle{\pgfqpoint{1.150000in}{0.150000in}}{\pgfqpoint{5.700000in}{5.700000in}}%
\pgfusepath{clip}%
\pgfsetbuttcap%
\pgfsetroundjoin%
\definecolor{currentfill}{rgb}{0.268510,0.009605,0.335427}%
\pgfsetfillcolor{currentfill}%
\pgfsetfillopacity{0.700000}%
\pgfsetlinewidth{0.000000pt}%
\definecolor{currentstroke}{rgb}{0.000000,0.000000,0.000000}%
\pgfsetstrokecolor{currentstroke}%
\pgfsetdash{}{0pt}%
\pgfpathmoveto{\pgfqpoint{4.013393in}{1.732319in}}%
\pgfpathlineto{\pgfqpoint{4.027295in}{1.729506in}}%
\pgfpathlineto{\pgfqpoint{4.041203in}{1.726796in}}%
\pgfpathlineto{\pgfqpoint{4.055119in}{1.724187in}}%
\pgfpathlineto{\pgfqpoint{4.069042in}{1.721680in}}%
\pgfpathlineto{\pgfqpoint{4.060994in}{1.713086in}}%
\pgfpathlineto{\pgfqpoint{4.052941in}{1.704606in}}%
\pgfpathlineto{\pgfqpoint{4.044881in}{1.696242in}}%
\pgfpathlineto{\pgfqpoint{4.036815in}{1.688001in}}%
\pgfpathlineto{\pgfqpoint{4.022879in}{1.690919in}}%
\pgfpathlineto{\pgfqpoint{4.008951in}{1.693938in}}%
\pgfpathlineto{\pgfqpoint{3.995029in}{1.697060in}}%
\pgfpathlineto{\pgfqpoint{3.981113in}{1.700284in}}%
\pgfpathlineto{\pgfqpoint{3.989193in}{1.708107in}}%
\pgfpathlineto{\pgfqpoint{3.997266in}{1.716057in}}%
\pgfpathlineto{\pgfqpoint{4.005333in}{1.724129in}}%
\pgfpathlineto{\pgfqpoint{4.013393in}{1.732319in}}%
\pgfpathclose%
\pgfusepath{fill}%
\end{pgfscope}%
\begin{pgfscope}%
\pgfpathrectangle{\pgfqpoint{1.150000in}{0.150000in}}{\pgfqpoint{5.700000in}{5.700000in}}%
\pgfusepath{clip}%
\pgfsetbuttcap%
\pgfsetroundjoin%
\definecolor{currentfill}{rgb}{0.257322,0.256130,0.526563}%
\pgfsetfillcolor{currentfill}%
\pgfsetfillopacity{0.700000}%
\pgfsetlinewidth{0.000000pt}%
\definecolor{currentstroke}{rgb}{0.000000,0.000000,0.000000}%
\pgfsetstrokecolor{currentstroke}%
\pgfsetdash{}{0pt}%
\pgfpathmoveto{\pgfqpoint{4.891382in}{2.208144in}}%
\pgfpathlineto{\pgfqpoint{4.905620in}{2.212466in}}%
\pgfpathlineto{\pgfqpoint{4.919870in}{2.216887in}}%
\pgfpathlineto{\pgfqpoint{4.934133in}{2.221407in}}%
\pgfpathlineto{\pgfqpoint{4.948408in}{2.226027in}}%
\pgfpathlineto{\pgfqpoint{4.940611in}{2.213333in}}%
\pgfpathlineto{\pgfqpoint{4.932809in}{2.200587in}}%
\pgfpathlineto{\pgfqpoint{4.925001in}{2.187793in}}%
\pgfpathlineto{\pgfqpoint{4.917189in}{2.174951in}}%
\pgfpathlineto{\pgfqpoint{4.902912in}{2.170567in}}%
\pgfpathlineto{\pgfqpoint{4.888649in}{2.166281in}}%
\pgfpathlineto{\pgfqpoint{4.874398in}{2.162094in}}%
\pgfpathlineto{\pgfqpoint{4.860159in}{2.158006in}}%
\pgfpathlineto{\pgfqpoint{4.867973in}{2.170606in}}%
\pgfpathlineto{\pgfqpoint{4.875781in}{2.183164in}}%
\pgfpathlineto{\pgfqpoint{4.883584in}{2.195677in}}%
\pgfpathlineto{\pgfqpoint{4.891382in}{2.208144in}}%
\pgfpathclose%
\pgfusepath{fill}%
\end{pgfscope}%
\begin{pgfscope}%
\pgfpathrectangle{\pgfqpoint{1.150000in}{0.150000in}}{\pgfqpoint{5.700000in}{5.700000in}}%
\pgfusepath{clip}%
\pgfsetbuttcap%
\pgfsetroundjoin%
\definecolor{currentfill}{rgb}{0.177423,0.437527,0.557565}%
\pgfsetfillcolor{currentfill}%
\pgfsetfillopacity{0.700000}%
\pgfsetlinewidth{0.000000pt}%
\definecolor{currentstroke}{rgb}{0.000000,0.000000,0.000000}%
\pgfsetstrokecolor{currentstroke}%
\pgfsetdash{}{0pt}%
\pgfpathmoveto{\pgfqpoint{5.394483in}{2.659182in}}%
\pgfpathlineto{\pgfqpoint{5.408988in}{2.666620in}}%
\pgfpathlineto{\pgfqpoint{5.423508in}{2.674157in}}%
\pgfpathlineto{\pgfqpoint{5.438044in}{2.681796in}}%
\pgfpathlineto{\pgfqpoint{5.452595in}{2.689535in}}%
\pgfpathlineto{\pgfqpoint{5.444981in}{2.678437in}}%
\pgfpathlineto{\pgfqpoint{5.437359in}{2.667225in}}%
\pgfpathlineto{\pgfqpoint{5.429730in}{2.655901in}}%
\pgfpathlineto{\pgfqpoint{5.422093in}{2.644465in}}%
\pgfpathlineto{\pgfqpoint{5.407540in}{2.636830in}}%
\pgfpathlineto{\pgfqpoint{5.393002in}{2.629295in}}%
\pgfpathlineto{\pgfqpoint{5.378479in}{2.621861in}}%
\pgfpathlineto{\pgfqpoint{5.363972in}{2.614527in}}%
\pgfpathlineto{\pgfqpoint{5.371610in}{2.625853in}}%
\pgfpathlineto{\pgfqpoint{5.379242in}{2.637071in}}%
\pgfpathlineto{\pgfqpoint{5.386866in}{2.648181in}}%
\pgfpathlineto{\pgfqpoint{5.394483in}{2.659182in}}%
\pgfpathclose%
\pgfusepath{fill}%
\end{pgfscope}%
\begin{pgfscope}%
\pgfpathrectangle{\pgfqpoint{1.150000in}{0.150000in}}{\pgfqpoint{5.700000in}{5.700000in}}%
\pgfusepath{clip}%
\pgfsetbuttcap%
\pgfsetroundjoin%
\definecolor{currentfill}{rgb}{0.281924,0.089666,0.412415}%
\pgfsetfillcolor{currentfill}%
\pgfsetfillopacity{0.700000}%
\pgfsetlinewidth{0.000000pt}%
\definecolor{currentstroke}{rgb}{0.000000,0.000000,0.000000}%
\pgfsetstrokecolor{currentstroke}%
\pgfsetdash{}{0pt}%
\pgfpathmoveto{\pgfqpoint{4.420294in}{1.868636in}}%
\pgfpathlineto{\pgfqpoint{4.434326in}{1.869304in}}%
\pgfpathlineto{\pgfqpoint{4.448368in}{1.870071in}}%
\pgfpathlineto{\pgfqpoint{4.462420in}{1.870937in}}%
\pgfpathlineto{\pgfqpoint{4.476482in}{1.871902in}}%
\pgfpathlineto{\pgfqpoint{4.468558in}{1.860327in}}%
\pgfpathlineto{\pgfqpoint{4.460630in}{1.848785in}}%
\pgfpathlineto{\pgfqpoint{4.452698in}{1.837280in}}%
\pgfpathlineto{\pgfqpoint{4.444760in}{1.825816in}}%
\pgfpathlineto{\pgfqpoint{4.430694in}{1.825192in}}%
\pgfpathlineto{\pgfqpoint{4.416637in}{1.824666in}}%
\pgfpathlineto{\pgfqpoint{4.402589in}{1.824240in}}%
\pgfpathlineto{\pgfqpoint{4.388551in}{1.823913in}}%
\pgfpathlineto{\pgfqpoint{4.396494in}{1.835030in}}%
\pgfpathlineto{\pgfqpoint{4.404432in}{1.846191in}}%
\pgfpathlineto{\pgfqpoint{4.412366in}{1.857394in}}%
\pgfpathlineto{\pgfqpoint{4.420294in}{1.868636in}}%
\pgfpathclose%
\pgfusepath{fill}%
\end{pgfscope}%
\begin{pgfscope}%
\pgfpathrectangle{\pgfqpoint{1.150000in}{0.150000in}}{\pgfqpoint{5.700000in}{5.700000in}}%
\pgfusepath{clip}%
\pgfsetbuttcap%
\pgfsetroundjoin%
\definecolor{currentfill}{rgb}{0.279566,0.067836,0.391917}%
\pgfsetfillcolor{currentfill}%
\pgfsetfillopacity{0.700000}%
\pgfsetlinewidth{0.000000pt}%
\definecolor{currentstroke}{rgb}{0.000000,0.000000,0.000000}%
\pgfsetstrokecolor{currentstroke}%
\pgfsetdash{}{0pt}%
\pgfpathmoveto{\pgfqpoint{4.332493in}{1.823600in}}%
\pgfpathlineto{\pgfqpoint{4.346493in}{1.823529in}}%
\pgfpathlineto{\pgfqpoint{4.360503in}{1.823557in}}%
\pgfpathlineto{\pgfqpoint{4.374523in}{1.823685in}}%
\pgfpathlineto{\pgfqpoint{4.388551in}{1.823913in}}%
\pgfpathlineto{\pgfqpoint{4.380604in}{1.812845in}}%
\pgfpathlineto{\pgfqpoint{4.372651in}{1.801829in}}%
\pgfpathlineto{\pgfqpoint{4.364694in}{1.790867in}}%
\pgfpathlineto{\pgfqpoint{4.356732in}{1.779964in}}%
\pgfpathlineto{\pgfqpoint{4.342697in}{1.780095in}}%
\pgfpathlineto{\pgfqpoint{4.328671in}{1.780325in}}%
\pgfpathlineto{\pgfqpoint{4.314654in}{1.780655in}}%
\pgfpathlineto{\pgfqpoint{4.300646in}{1.781084in}}%
\pgfpathlineto{\pgfqpoint{4.308615in}{1.791622in}}%
\pgfpathlineto{\pgfqpoint{4.316579in}{1.802223in}}%
\pgfpathlineto{\pgfqpoint{4.324538in}{1.812884in}}%
\pgfpathlineto{\pgfqpoint{4.332493in}{1.823600in}}%
\pgfpathclose%
\pgfusepath{fill}%
\end{pgfscope}%
\begin{pgfscope}%
\pgfpathrectangle{\pgfqpoint{1.150000in}{0.150000in}}{\pgfqpoint{5.700000in}{5.700000in}}%
\pgfusepath{clip}%
\pgfsetbuttcap%
\pgfsetroundjoin%
\definecolor{currentfill}{rgb}{0.283197,0.115680,0.436115}%
\pgfsetfillcolor{currentfill}%
\pgfsetfillopacity{0.700000}%
\pgfsetlinewidth{0.000000pt}%
\definecolor{currentstroke}{rgb}{0.000000,0.000000,0.000000}%
\pgfsetstrokecolor{currentstroke}%
\pgfsetdash{}{0pt}%
\pgfpathmoveto{\pgfqpoint{4.508129in}{1.918485in}}%
\pgfpathlineto{\pgfqpoint{4.522196in}{1.919873in}}%
\pgfpathlineto{\pgfqpoint{4.536273in}{1.921359in}}%
\pgfpathlineto{\pgfqpoint{4.550361in}{1.922945in}}%
\pgfpathlineto{\pgfqpoint{4.564459in}{1.924630in}}%
\pgfpathlineto{\pgfqpoint{4.556559in}{1.912632in}}%
\pgfpathlineto{\pgfqpoint{4.548653in}{1.900652in}}%
\pgfpathlineto{\pgfqpoint{4.540743in}{1.888691in}}%
\pgfpathlineto{\pgfqpoint{4.532829in}{1.876754in}}%
\pgfpathlineto{\pgfqpoint{4.518727in}{1.875393in}}%
\pgfpathlineto{\pgfqpoint{4.504635in}{1.874131in}}%
\pgfpathlineto{\pgfqpoint{4.490553in}{1.872967in}}%
\pgfpathlineto{\pgfqpoint{4.476482in}{1.871902in}}%
\pgfpathlineto{\pgfqpoint{4.484400in}{1.883509in}}%
\pgfpathlineto{\pgfqpoint{4.492315in}{1.895144in}}%
\pgfpathlineto{\pgfqpoint{4.500224in}{1.906803in}}%
\pgfpathlineto{\pgfqpoint{4.508129in}{1.918485in}}%
\pgfpathclose%
\pgfusepath{fill}%
\end{pgfscope}%
\begin{pgfscope}%
\pgfpathrectangle{\pgfqpoint{1.150000in}{0.150000in}}{\pgfqpoint{5.700000in}{5.700000in}}%
\pgfusepath{clip}%
\pgfsetbuttcap%
\pgfsetroundjoin%
\definecolor{currentfill}{rgb}{0.268510,0.009605,0.335427}%
\pgfsetfillcolor{currentfill}%
\pgfsetfillopacity{0.700000}%
\pgfsetlinewidth{0.000000pt}%
\definecolor{currentstroke}{rgb}{0.000000,0.000000,0.000000}%
\pgfsetstrokecolor{currentstroke}%
\pgfsetdash{}{0pt}%
\pgfpathmoveto{\pgfqpoint{3.782043in}{1.722245in}}%
\pgfpathlineto{\pgfqpoint{3.795902in}{1.717343in}}%
\pgfpathlineto{\pgfqpoint{3.809767in}{1.712548in}}%
\pgfpathlineto{\pgfqpoint{3.823637in}{1.707858in}}%
\pgfpathlineto{\pgfqpoint{3.837512in}{1.703273in}}%
\pgfpathlineto{\pgfqpoint{3.829366in}{1.697024in}}%
\pgfpathlineto{\pgfqpoint{3.821212in}{1.690937in}}%
\pgfpathlineto{\pgfqpoint{3.813050in}{1.685014in}}%
\pgfpathlineto{\pgfqpoint{3.804881in}{1.679262in}}%
\pgfpathlineto{\pgfqpoint{3.790986in}{1.684295in}}%
\pgfpathlineto{\pgfqpoint{3.777098in}{1.689433in}}%
\pgfpathlineto{\pgfqpoint{3.763214in}{1.694677in}}%
\pgfpathlineto{\pgfqpoint{3.749335in}{1.700026in}}%
\pgfpathlineto{\pgfqpoint{3.757524in}{1.705323in}}%
\pgfpathlineto{\pgfqpoint{3.765705in}{1.710795in}}%
\pgfpathlineto{\pgfqpoint{3.773878in}{1.716437in}}%
\pgfpathlineto{\pgfqpoint{3.782043in}{1.722245in}}%
\pgfpathclose%
\pgfusepath{fill}%
\end{pgfscope}%
\begin{pgfscope}%
\pgfpathrectangle{\pgfqpoint{1.150000in}{0.150000in}}{\pgfqpoint{5.700000in}{5.700000in}}%
\pgfusepath{clip}%
\pgfsetbuttcap%
\pgfsetroundjoin%
\definecolor{currentfill}{rgb}{0.279566,0.067836,0.391917}%
\pgfsetfillcolor{currentfill}%
\pgfsetfillopacity{0.700000}%
\pgfsetlinewidth{0.000000pt}%
\definecolor{currentstroke}{rgb}{0.000000,0.000000,0.000000}%
\pgfsetstrokecolor{currentstroke}%
\pgfsetdash{}{0pt}%
\pgfpathmoveto{\pgfqpoint{3.439376in}{1.819744in}}%
\pgfpathlineto{\pgfqpoint{3.453198in}{1.811736in}}%
\pgfpathlineto{\pgfqpoint{3.467024in}{1.803841in}}%
\pgfpathlineto{\pgfqpoint{3.480853in}{1.796060in}}%
\pgfpathlineto{\pgfqpoint{3.494684in}{1.788391in}}%
\pgfpathlineto{\pgfqpoint{3.486355in}{1.785955in}}%
\pgfpathlineto{\pgfqpoint{3.478014in}{1.783741in}}%
\pgfpathlineto{\pgfqpoint{3.469663in}{1.781754in}}%
\pgfpathlineto{\pgfqpoint{3.461301in}{1.779999in}}%
\pgfpathlineto{\pgfqpoint{3.447441in}{1.788156in}}%
\pgfpathlineto{\pgfqpoint{3.433585in}{1.796426in}}%
\pgfpathlineto{\pgfqpoint{3.419731in}{1.804810in}}%
\pgfpathlineto{\pgfqpoint{3.405879in}{1.813307in}}%
\pgfpathlineto{\pgfqpoint{3.414271in}{1.814566in}}%
\pgfpathlineto{\pgfqpoint{3.422650in}{1.816061in}}%
\pgfpathlineto{\pgfqpoint{3.431019in}{1.817789in}}%
\pgfpathlineto{\pgfqpoint{3.439376in}{1.819744in}}%
\pgfpathclose%
\pgfusepath{fill}%
\end{pgfscope}%
\begin{pgfscope}%
\pgfpathrectangle{\pgfqpoint{1.150000in}{0.150000in}}{\pgfqpoint{5.700000in}{5.700000in}}%
\pgfusepath{clip}%
\pgfsetbuttcap%
\pgfsetroundjoin%
\definecolor{currentfill}{rgb}{0.208623,0.367752,0.552675}%
\pgfsetfillcolor{currentfill}%
\pgfsetfillopacity{0.700000}%
\pgfsetlinewidth{0.000000pt}%
\definecolor{currentstroke}{rgb}{0.000000,0.000000,0.000000}%
\pgfsetstrokecolor{currentstroke}%
\pgfsetdash{}{0pt}%
\pgfpathmoveto{\pgfqpoint{5.187057in}{2.467229in}}%
\pgfpathlineto{\pgfqpoint{5.201450in}{2.473497in}}%
\pgfpathlineto{\pgfqpoint{5.215858in}{2.479866in}}%
\pgfpathlineto{\pgfqpoint{5.230280in}{2.486334in}}%
\pgfpathlineto{\pgfqpoint{5.244717in}{2.492903in}}%
\pgfpathlineto{\pgfqpoint{5.237015in}{2.480810in}}%
\pgfpathlineto{\pgfqpoint{5.229306in}{2.468625in}}%
\pgfpathlineto{\pgfqpoint{5.221592in}{2.456349in}}%
\pgfpathlineto{\pgfqpoint{5.213871in}{2.443984in}}%
\pgfpathlineto{\pgfqpoint{5.199433in}{2.437577in}}%
\pgfpathlineto{\pgfqpoint{5.185010in}{2.431269in}}%
\pgfpathlineto{\pgfqpoint{5.170602in}{2.425061in}}%
\pgfpathlineto{\pgfqpoint{5.156207in}{2.418953in}}%
\pgfpathlineto{\pgfqpoint{5.163929in}{2.431150in}}%
\pgfpathlineto{\pgfqpoint{5.171644in}{2.443263in}}%
\pgfpathlineto{\pgfqpoint{5.179354in}{2.455289in}}%
\pgfpathlineto{\pgfqpoint{5.187057in}{2.467229in}}%
\pgfpathclose%
\pgfusepath{fill}%
\end{pgfscope}%
\begin{pgfscope}%
\pgfpathrectangle{\pgfqpoint{1.150000in}{0.150000in}}{\pgfqpoint{5.700000in}{5.700000in}}%
\pgfusepath{clip}%
\pgfsetbuttcap%
\pgfsetroundjoin%
\definecolor{currentfill}{rgb}{0.282623,0.140926,0.457517}%
\pgfsetfillcolor{currentfill}%
\pgfsetfillopacity{0.700000}%
\pgfsetlinewidth{0.000000pt}%
\definecolor{currentstroke}{rgb}{0.000000,0.000000,0.000000}%
\pgfsetstrokecolor{currentstroke}%
\pgfsetdash{}{0pt}%
\pgfpathmoveto{\pgfqpoint{3.184530in}{1.965183in}}%
\pgfpathlineto{\pgfqpoint{3.198353in}{1.954790in}}%
\pgfpathlineto{\pgfqpoint{3.212178in}{1.944520in}}%
\pgfpathlineto{\pgfqpoint{3.226003in}{1.934373in}}%
\pgfpathlineto{\pgfqpoint{3.239830in}{1.924347in}}%
\pgfpathlineto{\pgfqpoint{3.231331in}{1.924841in}}%
\pgfpathlineto{\pgfqpoint{3.222819in}{1.925598in}}%
\pgfpathlineto{\pgfqpoint{3.214293in}{1.926623in}}%
\pgfpathlineto{\pgfqpoint{3.205752in}{1.927921in}}%
\pgfpathlineto{\pgfqpoint{3.191891in}{1.938460in}}%
\pgfpathlineto{\pgfqpoint{3.178030in}{1.949121in}}%
\pgfpathlineto{\pgfqpoint{3.164170in}{1.959905in}}%
\pgfpathlineto{\pgfqpoint{3.150311in}{1.970812in}}%
\pgfpathlineto{\pgfqpoint{3.158887in}{1.968992in}}%
\pgfpathlineto{\pgfqpoint{3.167449in}{1.967451in}}%
\pgfpathlineto{\pgfqpoint{3.175996in}{1.966183in}}%
\pgfpathlineto{\pgfqpoint{3.184530in}{1.965183in}}%
\pgfpathclose%
\pgfusepath{fill}%
\end{pgfscope}%
\begin{pgfscope}%
\pgfpathrectangle{\pgfqpoint{1.150000in}{0.150000in}}{\pgfqpoint{5.700000in}{5.700000in}}%
\pgfusepath{clip}%
\pgfsetbuttcap%
\pgfsetroundjoin%
\definecolor{currentfill}{rgb}{0.276022,0.044167,0.370164}%
\pgfsetfillcolor{currentfill}%
\pgfsetfillopacity{0.700000}%
\pgfsetlinewidth{0.000000pt}%
\definecolor{currentstroke}{rgb}{0.000000,0.000000,0.000000}%
\pgfsetstrokecolor{currentstroke}%
\pgfsetdash{}{0pt}%
\pgfpathmoveto{\pgfqpoint{4.244701in}{1.783801in}}%
\pgfpathlineto{\pgfqpoint{4.258674in}{1.782972in}}%
\pgfpathlineto{\pgfqpoint{4.272656in}{1.782242in}}%
\pgfpathlineto{\pgfqpoint{4.286647in}{1.781613in}}%
\pgfpathlineto{\pgfqpoint{4.300646in}{1.781084in}}%
\pgfpathlineto{\pgfqpoint{4.292672in}{1.770613in}}%
\pgfpathlineto{\pgfqpoint{4.284692in}{1.760211in}}%
\pgfpathlineto{\pgfqpoint{4.276708in}{1.749883in}}%
\pgfpathlineto{\pgfqpoint{4.268719in}{1.739633in}}%
\pgfpathlineto{\pgfqpoint{4.254712in}{1.740538in}}%
\pgfpathlineto{\pgfqpoint{4.240713in}{1.741543in}}%
\pgfpathlineto{\pgfqpoint{4.226722in}{1.742648in}}%
\pgfpathlineto{\pgfqpoint{4.212740in}{1.743852in}}%
\pgfpathlineto{\pgfqpoint{4.220738in}{1.753721in}}%
\pgfpathlineto{\pgfqpoint{4.228731in}{1.763670in}}%
\pgfpathlineto{\pgfqpoint{4.236719in}{1.773698in}}%
\pgfpathlineto{\pgfqpoint{4.244701in}{1.783801in}}%
\pgfpathclose%
\pgfusepath{fill}%
\end{pgfscope}%
\begin{pgfscope}%
\pgfpathrectangle{\pgfqpoint{1.150000in}{0.150000in}}{\pgfqpoint{5.700000in}{5.700000in}}%
\pgfusepath{clip}%
\pgfsetbuttcap%
\pgfsetroundjoin%
\definecolor{currentfill}{rgb}{0.282290,0.145912,0.461510}%
\pgfsetfillcolor{currentfill}%
\pgfsetfillopacity{0.700000}%
\pgfsetlinewidth{0.000000pt}%
\definecolor{currentstroke}{rgb}{0.000000,0.000000,0.000000}%
\pgfsetstrokecolor{currentstroke}%
\pgfsetdash{}{0pt}%
\pgfpathmoveto{\pgfqpoint{4.596016in}{1.972737in}}%
\pgfpathlineto{\pgfqpoint{4.610122in}{1.974826in}}%
\pgfpathlineto{\pgfqpoint{4.624238in}{1.977015in}}%
\pgfpathlineto{\pgfqpoint{4.638365in}{1.979302in}}%
\pgfpathlineto{\pgfqpoint{4.652504in}{1.981689in}}%
\pgfpathlineto{\pgfqpoint{4.644624in}{1.969351in}}%
\pgfpathlineto{\pgfqpoint{4.636740in}{1.957015in}}%
\pgfpathlineto{\pgfqpoint{4.628852in}{1.944683in}}%
\pgfpathlineto{\pgfqpoint{4.620959in}{1.932358in}}%
\pgfpathlineto{\pgfqpoint{4.606818in}{1.930278in}}%
\pgfpathlineto{\pgfqpoint{4.592688in}{1.928297in}}%
\pgfpathlineto{\pgfqpoint{4.578568in}{1.926414in}}%
\pgfpathlineto{\pgfqpoint{4.564459in}{1.924630in}}%
\pgfpathlineto{\pgfqpoint{4.572356in}{1.936643in}}%
\pgfpathlineto{\pgfqpoint{4.580247in}{1.948666in}}%
\pgfpathlineto{\pgfqpoint{4.588134in}{1.960699in}}%
\pgfpathlineto{\pgfqpoint{4.596016in}{1.972737in}}%
\pgfpathclose%
\pgfusepath{fill}%
\end{pgfscope}%
\begin{pgfscope}%
\pgfpathrectangle{\pgfqpoint{1.150000in}{0.150000in}}{\pgfqpoint{5.700000in}{5.700000in}}%
\pgfusepath{clip}%
\pgfsetbuttcap%
\pgfsetroundjoin%
\definecolor{currentfill}{rgb}{0.139147,0.533812,0.555298}%
\pgfsetfillcolor{currentfill}%
\pgfsetfillopacity{0.700000}%
\pgfsetlinewidth{0.000000pt}%
\definecolor{currentstroke}{rgb}{0.000000,0.000000,0.000000}%
\pgfsetstrokecolor{currentstroke}%
\pgfsetdash{}{0pt}%
\pgfpathmoveto{\pgfqpoint{5.690190in}{2.918559in}}%
\pgfpathlineto{\pgfqpoint{5.704869in}{2.927425in}}%
\pgfpathlineto{\pgfqpoint{5.719564in}{2.936393in}}%
\pgfpathlineto{\pgfqpoint{5.734275in}{2.945462in}}%
\pgfpathlineto{\pgfqpoint{5.749004in}{2.954632in}}%
\pgfpathlineto{\pgfqpoint{5.741536in}{2.945354in}}%
\pgfpathlineto{\pgfqpoint{5.734060in}{2.935943in}}%
\pgfpathlineto{\pgfqpoint{5.726574in}{2.926399in}}%
\pgfpathlineto{\pgfqpoint{5.719079in}{2.916722in}}%
\pgfpathlineto{\pgfqpoint{5.704345in}{2.907577in}}%
\pgfpathlineto{\pgfqpoint{5.689627in}{2.898533in}}%
\pgfpathlineto{\pgfqpoint{5.674926in}{2.889591in}}%
\pgfpathlineto{\pgfqpoint{5.660242in}{2.880750in}}%
\pgfpathlineto{\pgfqpoint{5.667743in}{2.890394in}}%
\pgfpathlineto{\pgfqpoint{5.675234in}{2.899911in}}%
\pgfpathlineto{\pgfqpoint{5.682717in}{2.909299in}}%
\pgfpathlineto{\pgfqpoint{5.690190in}{2.918559in}}%
\pgfpathclose%
\pgfusepath{fill}%
\end{pgfscope}%
\begin{pgfscope}%
\pgfpathrectangle{\pgfqpoint{1.150000in}{0.150000in}}{\pgfqpoint{5.700000in}{5.700000in}}%
\pgfusepath{clip}%
\pgfsetbuttcap%
\pgfsetroundjoin%
\definecolor{currentfill}{rgb}{0.272594,0.025563,0.353093}%
\pgfsetfillcolor{currentfill}%
\pgfsetfillopacity{0.700000}%
\pgfsetlinewidth{0.000000pt}%
\definecolor{currentstroke}{rgb}{0.000000,0.000000,0.000000}%
\pgfsetstrokecolor{currentstroke}%
\pgfsetdash{}{0pt}%
\pgfpathmoveto{\pgfqpoint{3.638480in}{1.746661in}}%
\pgfpathlineto{\pgfqpoint{3.652321in}{1.740454in}}%
\pgfpathlineto{\pgfqpoint{3.666166in}{1.734357in}}%
\pgfpathlineto{\pgfqpoint{3.680016in}{1.728367in}}%
\pgfpathlineto{\pgfqpoint{3.693871in}{1.722485in}}%
\pgfpathlineto{\pgfqpoint{3.685652in}{1.717826in}}%
\pgfpathlineto{\pgfqpoint{3.677425in}{1.713356in}}%
\pgfpathlineto{\pgfqpoint{3.669189in}{1.709079in}}%
\pgfpathlineto{\pgfqpoint{3.660943in}{1.705000in}}%
\pgfpathlineto{\pgfqpoint{3.647067in}{1.711350in}}%
\pgfpathlineto{\pgfqpoint{3.633194in}{1.717807in}}%
\pgfpathlineto{\pgfqpoint{3.619326in}{1.724372in}}%
\pgfpathlineto{\pgfqpoint{3.605462in}{1.731046in}}%
\pgfpathlineto{\pgfqpoint{3.613731in}{1.734651in}}%
\pgfpathlineto{\pgfqpoint{3.621990in}{1.738458in}}%
\pgfpathlineto{\pgfqpoint{3.630240in}{1.742462in}}%
\pgfpathlineto{\pgfqpoint{3.638480in}{1.746661in}}%
\pgfpathclose%
\pgfusepath{fill}%
\end{pgfscope}%
\begin{pgfscope}%
\pgfpathrectangle{\pgfqpoint{1.150000in}{0.150000in}}{\pgfqpoint{5.700000in}{5.700000in}}%
\pgfusepath{clip}%
\pgfsetbuttcap%
\pgfsetroundjoin%
\definecolor{currentfill}{rgb}{0.268510,0.009605,0.335427}%
\pgfsetfillcolor{currentfill}%
\pgfsetfillopacity{0.700000}%
\pgfsetlinewidth{0.000000pt}%
\definecolor{currentstroke}{rgb}{0.000000,0.000000,0.000000}%
\pgfsetstrokecolor{currentstroke}%
\pgfsetdash{}{0pt}%
\pgfpathmoveto{\pgfqpoint{3.925518in}{1.714208in}}%
\pgfpathlineto{\pgfqpoint{3.939407in}{1.710572in}}%
\pgfpathlineto{\pgfqpoint{3.953303in}{1.707040in}}%
\pgfpathlineto{\pgfqpoint{3.967205in}{1.703610in}}%
\pgfpathlineto{\pgfqpoint{3.981113in}{1.700284in}}%
\pgfpathlineto{\pgfqpoint{3.973027in}{1.692591in}}%
\pgfpathlineto{\pgfqpoint{3.964935in}{1.685032in}}%
\pgfpathlineto{\pgfqpoint{3.956835in}{1.677612in}}%
\pgfpathlineto{\pgfqpoint{3.948729in}{1.670334in}}%
\pgfpathlineto{\pgfqpoint{3.934806in}{1.674090in}}%
\pgfpathlineto{\pgfqpoint{3.920889in}{1.677949in}}%
\pgfpathlineto{\pgfqpoint{3.906977in}{1.681910in}}%
\pgfpathlineto{\pgfqpoint{3.893073in}{1.685975in}}%
\pgfpathlineto{\pgfqpoint{3.901194in}{1.692816in}}%
\pgfpathlineto{\pgfqpoint{3.909309in}{1.699805in}}%
\pgfpathlineto{\pgfqpoint{3.917417in}{1.706937in}}%
\pgfpathlineto{\pgfqpoint{3.925518in}{1.714208in}}%
\pgfpathclose%
\pgfusepath{fill}%
\end{pgfscope}%
\begin{pgfscope}%
\pgfpathrectangle{\pgfqpoint{1.150000in}{0.150000in}}{\pgfqpoint{5.700000in}{5.700000in}}%
\pgfusepath{clip}%
\pgfsetbuttcap%
\pgfsetroundjoin%
\definecolor{currentfill}{rgb}{0.244972,0.287675,0.537260}%
\pgfsetfillcolor{currentfill}%
\pgfsetfillopacity{0.700000}%
\pgfsetlinewidth{0.000000pt}%
\definecolor{currentstroke}{rgb}{0.000000,0.000000,0.000000}%
\pgfsetstrokecolor{currentstroke}%
\pgfsetdash{}{0pt}%
\pgfpathmoveto{\pgfqpoint{4.979543in}{2.276251in}}%
\pgfpathlineto{\pgfqpoint{4.993831in}{2.281186in}}%
\pgfpathlineto{\pgfqpoint{5.008132in}{2.286220in}}%
\pgfpathlineto{\pgfqpoint{5.022446in}{2.291354in}}%
\pgfpathlineto{\pgfqpoint{5.036773in}{2.296587in}}%
\pgfpathlineto{\pgfqpoint{5.028998in}{2.283908in}}%
\pgfpathlineto{\pgfqpoint{5.021218in}{2.271165in}}%
\pgfpathlineto{\pgfqpoint{5.013432in}{2.258361in}}%
\pgfpathlineto{\pgfqpoint{5.005640in}{2.245496in}}%
\pgfpathlineto{\pgfqpoint{4.991313in}{2.240480in}}%
\pgfpathlineto{\pgfqpoint{4.976998in}{2.235563in}}%
\pgfpathlineto{\pgfqpoint{4.962697in}{2.230745in}}%
\pgfpathlineto{\pgfqpoint{4.948408in}{2.226027in}}%
\pgfpathlineto{\pgfqpoint{4.956200in}{2.238667in}}%
\pgfpathlineto{\pgfqpoint{4.963987in}{2.251253in}}%
\pgfpathlineto{\pgfqpoint{4.971768in}{2.263782in}}%
\pgfpathlineto{\pgfqpoint{4.979543in}{2.276251in}}%
\pgfpathclose%
\pgfusepath{fill}%
\end{pgfscope}%
\begin{pgfscope}%
\pgfpathrectangle{\pgfqpoint{1.150000in}{0.150000in}}{\pgfqpoint{5.700000in}{5.700000in}}%
\pgfusepath{clip}%
\pgfsetbuttcap%
\pgfsetroundjoin%
\definecolor{currentfill}{rgb}{0.272594,0.025563,0.353093}%
\pgfsetfillcolor{currentfill}%
\pgfsetfillopacity{0.700000}%
\pgfsetlinewidth{0.000000pt}%
\definecolor{currentstroke}{rgb}{0.000000,0.000000,0.000000}%
\pgfsetstrokecolor{currentstroke}%
\pgfsetdash{}{0pt}%
\pgfpathmoveto{\pgfqpoint{4.156894in}{1.749676in}}%
\pgfpathlineto{\pgfqpoint{4.170843in}{1.748069in}}%
\pgfpathlineto{\pgfqpoint{4.184801in}{1.746563in}}%
\pgfpathlineto{\pgfqpoint{4.198767in}{1.745158in}}%
\pgfpathlineto{\pgfqpoint{4.212740in}{1.743852in}}%
\pgfpathlineto{\pgfqpoint{4.204737in}{1.734070in}}%
\pgfpathlineto{\pgfqpoint{4.196728in}{1.724376in}}%
\pgfpathlineto{\pgfqpoint{4.188714in}{1.714775in}}%
\pgfpathlineto{\pgfqpoint{4.180695in}{1.705271in}}%
\pgfpathlineto{\pgfqpoint{4.166711in}{1.706970in}}%
\pgfpathlineto{\pgfqpoint{4.152735in}{1.708769in}}%
\pgfpathlineto{\pgfqpoint{4.138767in}{1.710668in}}%
\pgfpathlineto{\pgfqpoint{4.124807in}{1.712669in}}%
\pgfpathlineto{\pgfqpoint{4.132837in}{1.721773in}}%
\pgfpathlineto{\pgfqpoint{4.140862in}{1.730978in}}%
\pgfpathlineto{\pgfqpoint{4.148881in}{1.740280in}}%
\pgfpathlineto{\pgfqpoint{4.156894in}{1.749676in}}%
\pgfpathclose%
\pgfusepath{fill}%
\end{pgfscope}%
\begin{pgfscope}%
\pgfpathrectangle{\pgfqpoint{1.150000in}{0.150000in}}{\pgfqpoint{5.700000in}{5.700000in}}%
\pgfusepath{clip}%
\pgfsetbuttcap%
\pgfsetroundjoin%
\definecolor{currentfill}{rgb}{0.278826,0.175490,0.483397}%
\pgfsetfillcolor{currentfill}%
\pgfsetfillopacity{0.700000}%
\pgfsetlinewidth{0.000000pt}%
\definecolor{currentstroke}{rgb}{0.000000,0.000000,0.000000}%
\pgfsetstrokecolor{currentstroke}%
\pgfsetdash{}{0pt}%
\pgfpathmoveto{\pgfqpoint{4.683974in}{2.030995in}}%
\pgfpathlineto{\pgfqpoint{4.698121in}{2.033768in}}%
\pgfpathlineto{\pgfqpoint{4.712280in}{2.036641in}}%
\pgfpathlineto{\pgfqpoint{4.726449in}{2.039612in}}%
\pgfpathlineto{\pgfqpoint{4.740631in}{2.042682in}}%
\pgfpathlineto{\pgfqpoint{4.732772in}{2.030085in}}%
\pgfpathlineto{\pgfqpoint{4.724909in}{2.017475in}}%
\pgfpathlineto{\pgfqpoint{4.717041in}{2.004852in}}%
\pgfpathlineto{\pgfqpoint{4.709168in}{1.992221in}}%
\pgfpathlineto{\pgfqpoint{4.694985in}{1.989440in}}%
\pgfpathlineto{\pgfqpoint{4.680813in}{1.986758in}}%
\pgfpathlineto{\pgfqpoint{4.666653in}{1.984174in}}%
\pgfpathlineto{\pgfqpoint{4.652504in}{1.981689in}}%
\pgfpathlineto{\pgfqpoint{4.660378in}{1.994024in}}%
\pgfpathlineto{\pgfqpoint{4.668248in}{2.006356in}}%
\pgfpathlineto{\pgfqpoint{4.676114in}{2.018680in}}%
\pgfpathlineto{\pgfqpoint{4.683974in}{2.030995in}}%
\pgfpathclose%
\pgfusepath{fill}%
\end{pgfscope}%
\begin{pgfscope}%
\pgfpathrectangle{\pgfqpoint{1.150000in}{0.150000in}}{\pgfqpoint{5.700000in}{5.700000in}}%
\pgfusepath{clip}%
\pgfsetbuttcap%
\pgfsetroundjoin%
\definecolor{currentfill}{rgb}{0.165117,0.467423,0.558141}%
\pgfsetfillcolor{currentfill}%
\pgfsetfillopacity{0.700000}%
\pgfsetlinewidth{0.000000pt}%
\definecolor{currentstroke}{rgb}{0.000000,0.000000,0.000000}%
\pgfsetstrokecolor{currentstroke}%
\pgfsetdash{}{0pt}%
\pgfpathmoveto{\pgfqpoint{5.482975in}{2.732778in}}%
\pgfpathlineto{\pgfqpoint{5.497539in}{2.740701in}}%
\pgfpathlineto{\pgfqpoint{5.512118in}{2.748726in}}%
\pgfpathlineto{\pgfqpoint{5.526713in}{2.756851in}}%
\pgfpathlineto{\pgfqpoint{5.541325in}{2.765077in}}%
\pgfpathlineto{\pgfqpoint{5.533744in}{2.754362in}}%
\pgfpathlineto{\pgfqpoint{5.526156in}{2.743527in}}%
\pgfpathlineto{\pgfqpoint{5.518560in}{2.732572in}}%
\pgfpathlineto{\pgfqpoint{5.510956in}{2.721496in}}%
\pgfpathlineto{\pgfqpoint{5.496342in}{2.713355in}}%
\pgfpathlineto{\pgfqpoint{5.481744in}{2.705314in}}%
\pgfpathlineto{\pgfqpoint{5.467162in}{2.697374in}}%
\pgfpathlineto{\pgfqpoint{5.452595in}{2.689535in}}%
\pgfpathlineto{\pgfqpoint{5.460202in}{2.700518in}}%
\pgfpathlineto{\pgfqpoint{5.467801in}{2.711387in}}%
\pgfpathlineto{\pgfqpoint{5.475392in}{2.722140in}}%
\pgfpathlineto{\pgfqpoint{5.482975in}{2.732778in}}%
\pgfpathclose%
\pgfusepath{fill}%
\end{pgfscope}%
\begin{pgfscope}%
\pgfpathrectangle{\pgfqpoint{1.150000in}{0.150000in}}{\pgfqpoint{5.700000in}{5.700000in}}%
\pgfusepath{clip}%
\pgfsetbuttcap%
\pgfsetroundjoin%
\definecolor{currentfill}{rgb}{0.212395,0.359683,0.551710}%
\pgfsetfillcolor{currentfill}%
\pgfsetfillopacity{0.700000}%
\pgfsetlinewidth{0.000000pt}%
\definecolor{currentstroke}{rgb}{0.000000,0.000000,0.000000}%
\pgfsetstrokecolor{currentstroke}%
\pgfsetdash{}{0pt}%
\pgfpathmoveto{\pgfqpoint{2.650766in}{2.452071in}}%
\pgfpathlineto{\pgfqpoint{2.664685in}{2.436185in}}%
\pgfpathlineto{\pgfqpoint{2.678600in}{2.420454in}}%
\pgfpathlineto{\pgfqpoint{2.692512in}{2.404878in}}%
\pgfpathlineto{\pgfqpoint{2.706419in}{2.389455in}}%
\pgfpathlineto{\pgfqpoint{2.697483in}{2.395848in}}%
\pgfpathlineto{\pgfqpoint{2.688527in}{2.402575in}}%
\pgfpathlineto{\pgfqpoint{2.679550in}{2.409641in}}%
\pgfpathlineto{\pgfqpoint{2.670552in}{2.417054in}}%
\pgfpathlineto{\pgfqpoint{2.656594in}{2.433032in}}%
\pgfpathlineto{\pgfqpoint{2.642633in}{2.449164in}}%
\pgfpathlineto{\pgfqpoint{2.628666in}{2.465452in}}%
\pgfpathlineto{\pgfqpoint{2.614696in}{2.481896in}}%
\pgfpathlineto{\pgfqpoint{2.623746in}{2.473918in}}%
\pgfpathlineto{\pgfqpoint{2.632774in}{2.466292in}}%
\pgfpathlineto{\pgfqpoint{2.641780in}{2.459012in}}%
\pgfpathlineto{\pgfqpoint{2.650766in}{2.452071in}}%
\pgfpathclose%
\pgfusepath{fill}%
\end{pgfscope}%
\begin{pgfscope}%
\pgfpathrectangle{\pgfqpoint{1.150000in}{0.150000in}}{\pgfqpoint{5.700000in}{5.700000in}}%
\pgfusepath{clip}%
\pgfsetbuttcap%
\pgfsetroundjoin%
\definecolor{currentfill}{rgb}{0.201239,0.383670,0.554294}%
\pgfsetfillcolor{currentfill}%
\pgfsetfillopacity{0.700000}%
\pgfsetlinewidth{0.000000pt}%
\definecolor{currentstroke}{rgb}{0.000000,0.000000,0.000000}%
\pgfsetstrokecolor{currentstroke}%
\pgfsetdash{}{0pt}%
\pgfpathmoveto{\pgfqpoint{2.595046in}{2.517198in}}%
\pgfpathlineto{\pgfqpoint{2.608983in}{2.500677in}}%
\pgfpathlineto{\pgfqpoint{2.622915in}{2.484316in}}%
\pgfpathlineto{\pgfqpoint{2.636843in}{2.468115in}}%
\pgfpathlineto{\pgfqpoint{2.650766in}{2.452071in}}%
\pgfpathlineto{\pgfqpoint{2.641780in}{2.459012in}}%
\pgfpathlineto{\pgfqpoint{2.632774in}{2.466292in}}%
\pgfpathlineto{\pgfqpoint{2.623746in}{2.473918in}}%
\pgfpathlineto{\pgfqpoint{2.614696in}{2.481896in}}%
\pgfpathlineto{\pgfqpoint{2.600721in}{2.498497in}}%
\pgfpathlineto{\pgfqpoint{2.586741in}{2.515258in}}%
\pgfpathlineto{\pgfqpoint{2.572756in}{2.532179in}}%
\pgfpathlineto{\pgfqpoint{2.558767in}{2.549262in}}%
\pgfpathlineto{\pgfqpoint{2.567870in}{2.540716in}}%
\pgfpathlineto{\pgfqpoint{2.576950in}{2.532527in}}%
\pgfpathlineto{\pgfqpoint{2.586009in}{2.524690in}}%
\pgfpathlineto{\pgfqpoint{2.595046in}{2.517198in}}%
\pgfpathclose%
\pgfusepath{fill}%
\end{pgfscope}%
\begin{pgfscope}%
\pgfpathrectangle{\pgfqpoint{1.150000in}{0.150000in}}{\pgfqpoint{5.700000in}{5.700000in}}%
\pgfusepath{clip}%
\pgfsetbuttcap%
\pgfsetroundjoin%
\definecolor{currentfill}{rgb}{0.223925,0.334994,0.548053}%
\pgfsetfillcolor{currentfill}%
\pgfsetfillopacity{0.700000}%
\pgfsetlinewidth{0.000000pt}%
\definecolor{currentstroke}{rgb}{0.000000,0.000000,0.000000}%
\pgfsetstrokecolor{currentstroke}%
\pgfsetdash{}{0pt}%
\pgfpathmoveto{\pgfqpoint{2.706419in}{2.389455in}}%
\pgfpathlineto{\pgfqpoint{2.720323in}{2.374185in}}%
\pgfpathlineto{\pgfqpoint{2.734223in}{2.359065in}}%
\pgfpathlineto{\pgfqpoint{2.748121in}{2.344096in}}%
\pgfpathlineto{\pgfqpoint{2.762015in}{2.329275in}}%
\pgfpathlineto{\pgfqpoint{2.753127in}{2.335123in}}%
\pgfpathlineto{\pgfqpoint{2.744219in}{2.341300in}}%
\pgfpathlineto{\pgfqpoint{2.735292in}{2.347810in}}%
\pgfpathlineto{\pgfqpoint{2.726345in}{2.354660in}}%
\pgfpathlineto{\pgfqpoint{2.712402in}{2.370033in}}%
\pgfpathlineto{\pgfqpoint{2.698456in}{2.385555in}}%
\pgfpathlineto{\pgfqpoint{2.684506in}{2.401229in}}%
\pgfpathlineto{\pgfqpoint{2.670552in}{2.417054in}}%
\pgfpathlineto{\pgfqpoint{2.679550in}{2.409641in}}%
\pgfpathlineto{\pgfqpoint{2.688527in}{2.402575in}}%
\pgfpathlineto{\pgfqpoint{2.697483in}{2.395848in}}%
\pgfpathlineto{\pgfqpoint{2.706419in}{2.389455in}}%
\pgfpathclose%
\pgfusepath{fill}%
\end{pgfscope}%
\begin{pgfscope}%
\pgfpathrectangle{\pgfqpoint{1.150000in}{0.150000in}}{\pgfqpoint{5.700000in}{5.700000in}}%
\pgfusepath{clip}%
\pgfsetbuttcap%
\pgfsetroundjoin%
\definecolor{currentfill}{rgb}{0.283187,0.125848,0.444960}%
\pgfsetfillcolor{currentfill}%
\pgfsetfillopacity{0.700000}%
\pgfsetlinewidth{0.000000pt}%
\definecolor{currentstroke}{rgb}{0.000000,0.000000,0.000000}%
\pgfsetstrokecolor{currentstroke}%
\pgfsetdash{}{0pt}%
\pgfpathmoveto{\pgfqpoint{3.239830in}{1.924347in}}%
\pgfpathlineto{\pgfqpoint{3.253658in}{1.914442in}}%
\pgfpathlineto{\pgfqpoint{3.267487in}{1.904658in}}%
\pgfpathlineto{\pgfqpoint{3.281318in}{1.894993in}}%
\pgfpathlineto{\pgfqpoint{3.295151in}{1.885447in}}%
\pgfpathlineto{\pgfqpoint{3.286685in}{1.885437in}}%
\pgfpathlineto{\pgfqpoint{3.278206in}{1.885684in}}%
\pgfpathlineto{\pgfqpoint{3.269714in}{1.886194in}}%
\pgfpathlineto{\pgfqpoint{3.261209in}{1.886972in}}%
\pgfpathlineto{\pgfqpoint{3.247343in}{1.897029in}}%
\pgfpathlineto{\pgfqpoint{3.233478in}{1.907206in}}%
\pgfpathlineto{\pgfqpoint{3.219615in}{1.917503in}}%
\pgfpathlineto{\pgfqpoint{3.205752in}{1.927921in}}%
\pgfpathlineto{\pgfqpoint{3.214293in}{1.926623in}}%
\pgfpathlineto{\pgfqpoint{3.222819in}{1.925598in}}%
\pgfpathlineto{\pgfqpoint{3.231331in}{1.924841in}}%
\pgfpathlineto{\pgfqpoint{3.239830in}{1.924347in}}%
\pgfpathclose%
\pgfusepath{fill}%
\end{pgfscope}%
\begin{pgfscope}%
\pgfpathrectangle{\pgfqpoint{1.150000in}{0.150000in}}{\pgfqpoint{5.700000in}{5.700000in}}%
\pgfusepath{clip}%
\pgfsetbuttcap%
\pgfsetroundjoin%
\definecolor{currentfill}{rgb}{0.188923,0.410910,0.556326}%
\pgfsetfillcolor{currentfill}%
\pgfsetfillopacity{0.700000}%
\pgfsetlinewidth{0.000000pt}%
\definecolor{currentstroke}{rgb}{0.000000,0.000000,0.000000}%
\pgfsetstrokecolor{currentstroke}%
\pgfsetdash{}{0pt}%
\pgfpathmoveto{\pgfqpoint{2.539249in}{2.584918in}}%
\pgfpathlineto{\pgfqpoint{2.553206in}{2.567740in}}%
\pgfpathlineto{\pgfqpoint{2.567158in}{2.550729in}}%
\pgfpathlineto{\pgfqpoint{2.581104in}{2.533882in}}%
\pgfpathlineto{\pgfqpoint{2.595046in}{2.517198in}}%
\pgfpathlineto{\pgfqpoint{2.586009in}{2.524690in}}%
\pgfpathlineto{\pgfqpoint{2.576950in}{2.532527in}}%
\pgfpathlineto{\pgfqpoint{2.567870in}{2.540716in}}%
\pgfpathlineto{\pgfqpoint{2.558767in}{2.549262in}}%
\pgfpathlineto{\pgfqpoint{2.544772in}{2.566508in}}%
\pgfpathlineto{\pgfqpoint{2.530771in}{2.583917in}}%
\pgfpathlineto{\pgfqpoint{2.516765in}{2.601493in}}%
\pgfpathlineto{\pgfqpoint{2.502754in}{2.619235in}}%
\pgfpathlineto{\pgfqpoint{2.511912in}{2.610116in}}%
\pgfpathlineto{\pgfqpoint{2.521047in}{2.601361in}}%
\pgfpathlineto{\pgfqpoint{2.530159in}{2.592964in}}%
\pgfpathlineto{\pgfqpoint{2.539249in}{2.584918in}}%
\pgfpathclose%
\pgfusepath{fill}%
\end{pgfscope}%
\begin{pgfscope}%
\pgfpathrectangle{\pgfqpoint{1.150000in}{0.150000in}}{\pgfqpoint{5.700000in}{5.700000in}}%
\pgfusepath{clip}%
\pgfsetbuttcap%
\pgfsetroundjoin%
\definecolor{currentfill}{rgb}{0.235526,0.309527,0.542944}%
\pgfsetfillcolor{currentfill}%
\pgfsetfillopacity{0.700000}%
\pgfsetlinewidth{0.000000pt}%
\definecolor{currentstroke}{rgb}{0.000000,0.000000,0.000000}%
\pgfsetstrokecolor{currentstroke}%
\pgfsetdash{}{0pt}%
\pgfpathmoveto{\pgfqpoint{2.762015in}{2.329275in}}%
\pgfpathlineto{\pgfqpoint{2.775906in}{2.314602in}}%
\pgfpathlineto{\pgfqpoint{2.789794in}{2.300076in}}%
\pgfpathlineto{\pgfqpoint{2.803679in}{2.285696in}}%
\pgfpathlineto{\pgfqpoint{2.817562in}{2.271460in}}%
\pgfpathlineto{\pgfqpoint{2.808721in}{2.276766in}}%
\pgfpathlineto{\pgfqpoint{2.799861in}{2.282395in}}%
\pgfpathlineto{\pgfqpoint{2.790982in}{2.288352in}}%
\pgfpathlineto{\pgfqpoint{2.782083in}{2.294643in}}%
\pgfpathlineto{\pgfqpoint{2.768153in}{2.309428in}}%
\pgfpathlineto{\pgfqpoint{2.754220in}{2.324359in}}%
\pgfpathlineto{\pgfqpoint{2.740284in}{2.339436in}}%
\pgfpathlineto{\pgfqpoint{2.726345in}{2.354660in}}%
\pgfpathlineto{\pgfqpoint{2.735292in}{2.347810in}}%
\pgfpathlineto{\pgfqpoint{2.744219in}{2.341300in}}%
\pgfpathlineto{\pgfqpoint{2.753127in}{2.335123in}}%
\pgfpathlineto{\pgfqpoint{2.762015in}{2.329275in}}%
\pgfpathclose%
\pgfusepath{fill}%
\end{pgfscope}%
\begin{pgfscope}%
\pgfpathrectangle{\pgfqpoint{1.150000in}{0.150000in}}{\pgfqpoint{5.700000in}{5.700000in}}%
\pgfusepath{clip}%
\pgfsetbuttcap%
\pgfsetroundjoin%
\definecolor{currentfill}{rgb}{0.194100,0.399323,0.555565}%
\pgfsetfillcolor{currentfill}%
\pgfsetfillopacity{0.700000}%
\pgfsetlinewidth{0.000000pt}%
\definecolor{currentstroke}{rgb}{0.000000,0.000000,0.000000}%
\pgfsetstrokecolor{currentstroke}%
\pgfsetdash{}{0pt}%
\pgfpathmoveto{\pgfqpoint{5.275458in}{2.540330in}}%
\pgfpathlineto{\pgfqpoint{5.289908in}{2.547141in}}%
\pgfpathlineto{\pgfqpoint{5.304372in}{2.554051in}}%
\pgfpathlineto{\pgfqpoint{5.318852in}{2.561061in}}%
\pgfpathlineto{\pgfqpoint{5.333346in}{2.568172in}}%
\pgfpathlineto{\pgfqpoint{5.325672in}{2.556324in}}%
\pgfpathlineto{\pgfqpoint{5.317991in}{2.544375in}}%
\pgfpathlineto{\pgfqpoint{5.310303in}{2.532325in}}%
\pgfpathlineto{\pgfqpoint{5.302609in}{2.520176in}}%
\pgfpathlineto{\pgfqpoint{5.288113in}{2.513207in}}%
\pgfpathlineto{\pgfqpoint{5.273633in}{2.506339in}}%
\pgfpathlineto{\pgfqpoint{5.259168in}{2.499571in}}%
\pgfpathlineto{\pgfqpoint{5.244717in}{2.492903in}}%
\pgfpathlineto{\pgfqpoint{5.252412in}{2.504902in}}%
\pgfpathlineto{\pgfqpoint{5.260101in}{2.516808in}}%
\pgfpathlineto{\pgfqpoint{5.267783in}{2.528617in}}%
\pgfpathlineto{\pgfqpoint{5.275458in}{2.540330in}}%
\pgfpathclose%
\pgfusepath{fill}%
\end{pgfscope}%
\begin{pgfscope}%
\pgfpathrectangle{\pgfqpoint{1.150000in}{0.150000in}}{\pgfqpoint{5.700000in}{5.700000in}}%
\pgfusepath{clip}%
\pgfsetbuttcap%
\pgfsetroundjoin%
\definecolor{currentfill}{rgb}{0.128729,0.563265,0.551229}%
\pgfsetfillcolor{currentfill}%
\pgfsetfillopacity{0.700000}%
\pgfsetlinewidth{0.000000pt}%
\definecolor{currentstroke}{rgb}{0.000000,0.000000,0.000000}%
\pgfsetstrokecolor{currentstroke}%
\pgfsetdash{}{0pt}%
\pgfpathmoveto{\pgfqpoint{5.778782in}{2.990416in}}%
\pgfpathlineto{\pgfqpoint{5.793521in}{2.999693in}}%
\pgfpathlineto{\pgfqpoint{5.808277in}{3.009071in}}%
\pgfpathlineto{\pgfqpoint{5.823050in}{3.018551in}}%
\pgfpathlineto{\pgfqpoint{5.837840in}{3.028133in}}%
\pgfpathlineto{\pgfqpoint{5.830417in}{3.019388in}}%
\pgfpathlineto{\pgfqpoint{5.822984in}{3.010506in}}%
\pgfpathlineto{\pgfqpoint{5.815541in}{3.001486in}}%
\pgfpathlineto{\pgfqpoint{5.808089in}{2.992329in}}%
\pgfpathlineto{\pgfqpoint{5.793292in}{2.982752in}}%
\pgfpathlineto{\pgfqpoint{5.778512in}{2.973277in}}%
\pgfpathlineto{\pgfqpoint{5.763750in}{2.963904in}}%
\pgfpathlineto{\pgfqpoint{5.749004in}{2.954632in}}%
\pgfpathlineto{\pgfqpoint{5.756462in}{2.963777in}}%
\pgfpathlineto{\pgfqpoint{5.763912in}{2.972789in}}%
\pgfpathlineto{\pgfqpoint{5.771352in}{2.981668in}}%
\pgfpathlineto{\pgfqpoint{5.778782in}{2.990416in}}%
\pgfpathclose%
\pgfusepath{fill}%
\end{pgfscope}%
\begin{pgfscope}%
\pgfpathrectangle{\pgfqpoint{1.150000in}{0.150000in}}{\pgfqpoint{5.700000in}{5.700000in}}%
\pgfusepath{clip}%
\pgfsetbuttcap%
\pgfsetroundjoin%
\definecolor{currentfill}{rgb}{0.177423,0.437527,0.557565}%
\pgfsetfillcolor{currentfill}%
\pgfsetfillopacity{0.700000}%
\pgfsetlinewidth{0.000000pt}%
\definecolor{currentstroke}{rgb}{0.000000,0.000000,0.000000}%
\pgfsetstrokecolor{currentstroke}%
\pgfsetdash{}{0pt}%
\pgfpathmoveto{\pgfqpoint{2.483364in}{2.655316in}}%
\pgfpathlineto{\pgfqpoint{2.497344in}{2.637460in}}%
\pgfpathlineto{\pgfqpoint{2.511318in}{2.619776in}}%
\pgfpathlineto{\pgfqpoint{2.525286in}{2.602263in}}%
\pgfpathlineto{\pgfqpoint{2.539249in}{2.584918in}}%
\pgfpathlineto{\pgfqpoint{2.530159in}{2.592964in}}%
\pgfpathlineto{\pgfqpoint{2.521047in}{2.601361in}}%
\pgfpathlineto{\pgfqpoint{2.511912in}{2.610116in}}%
\pgfpathlineto{\pgfqpoint{2.502754in}{2.619235in}}%
\pgfpathlineto{\pgfqpoint{2.488736in}{2.637145in}}%
\pgfpathlineto{\pgfqpoint{2.474713in}{2.655226in}}%
\pgfpathlineto{\pgfqpoint{2.460683in}{2.673477in}}%
\pgfpathlineto{\pgfqpoint{2.446647in}{2.691902in}}%
\pgfpathlineto{\pgfqpoint{2.455862in}{2.682207in}}%
\pgfpathlineto{\pgfqpoint{2.465053in}{2.672881in}}%
\pgfpathlineto{\pgfqpoint{2.474220in}{2.663920in}}%
\pgfpathlineto{\pgfqpoint{2.483364in}{2.655316in}}%
\pgfpathclose%
\pgfusepath{fill}%
\end{pgfscope}%
\begin{pgfscope}%
\pgfpathrectangle{\pgfqpoint{1.150000in}{0.150000in}}{\pgfqpoint{5.700000in}{5.700000in}}%
\pgfusepath{clip}%
\pgfsetbuttcap%
\pgfsetroundjoin%
\definecolor{currentfill}{rgb}{0.271828,0.209303,0.504434}%
\pgfsetfillcolor{currentfill}%
\pgfsetfillopacity{0.700000}%
\pgfsetlinewidth{0.000000pt}%
\definecolor{currentstroke}{rgb}{0.000000,0.000000,0.000000}%
\pgfsetstrokecolor{currentstroke}%
\pgfsetdash{}{0pt}%
\pgfpathmoveto{\pgfqpoint{4.772018in}{2.092876in}}%
\pgfpathlineto{\pgfqpoint{4.786209in}{2.096315in}}%
\pgfpathlineto{\pgfqpoint{4.800413in}{2.099853in}}%
\pgfpathlineto{\pgfqpoint{4.814628in}{2.103490in}}%
\pgfpathlineto{\pgfqpoint{4.828855in}{2.107227in}}%
\pgfpathlineto{\pgfqpoint{4.821017in}{2.094448in}}%
\pgfpathlineto{\pgfqpoint{4.813174in}{2.081640in}}%
\pgfpathlineto{\pgfqpoint{4.805326in}{2.068807in}}%
\pgfpathlineto{\pgfqpoint{4.797474in}{2.055949in}}%
\pgfpathlineto{\pgfqpoint{4.783245in}{2.052484in}}%
\pgfpathlineto{\pgfqpoint{4.769029in}{2.049118in}}%
\pgfpathlineto{\pgfqpoint{4.754824in}{2.045851in}}%
\pgfpathlineto{\pgfqpoint{4.740631in}{2.042682in}}%
\pgfpathlineto{\pgfqpoint{4.748485in}{2.055262in}}%
\pgfpathlineto{\pgfqpoint{4.756334in}{2.067822in}}%
\pgfpathlineto{\pgfqpoint{4.764178in}{2.080361in}}%
\pgfpathlineto{\pgfqpoint{4.772018in}{2.092876in}}%
\pgfpathclose%
\pgfusepath{fill}%
\end{pgfscope}%
\begin{pgfscope}%
\pgfpathrectangle{\pgfqpoint{1.150000in}{0.150000in}}{\pgfqpoint{5.700000in}{5.700000in}}%
\pgfusepath{clip}%
\pgfsetbuttcap%
\pgfsetroundjoin%
\definecolor{currentfill}{rgb}{0.277941,0.056324,0.381191}%
\pgfsetfillcolor{currentfill}%
\pgfsetfillopacity{0.700000}%
\pgfsetlinewidth{0.000000pt}%
\definecolor{currentstroke}{rgb}{0.000000,0.000000,0.000000}%
\pgfsetstrokecolor{currentstroke}%
\pgfsetdash{}{0pt}%
\pgfpathmoveto{\pgfqpoint{3.494684in}{1.788391in}}%
\pgfpathlineto{\pgfqpoint{3.508519in}{1.780835in}}%
\pgfpathlineto{\pgfqpoint{3.522357in}{1.773390in}}%
\pgfpathlineto{\pgfqpoint{3.536199in}{1.766057in}}%
\pgfpathlineto{\pgfqpoint{3.550044in}{1.758834in}}%
\pgfpathlineto{\pgfqpoint{3.541741in}{1.755917in}}%
\pgfpathlineto{\pgfqpoint{3.533427in}{1.753218in}}%
\pgfpathlineto{\pgfqpoint{3.525103in}{1.750740in}}%
\pgfpathlineto{\pgfqpoint{3.516768in}{1.748490in}}%
\pgfpathlineto{\pgfqpoint{3.502897in}{1.756200in}}%
\pgfpathlineto{\pgfqpoint{3.489028in}{1.764021in}}%
\pgfpathlineto{\pgfqpoint{3.475163in}{1.771954in}}%
\pgfpathlineto{\pgfqpoint{3.461301in}{1.779999in}}%
\pgfpathlineto{\pgfqpoint{3.469663in}{1.781754in}}%
\pgfpathlineto{\pgfqpoint{3.478014in}{1.783741in}}%
\pgfpathlineto{\pgfqpoint{3.486355in}{1.785955in}}%
\pgfpathlineto{\pgfqpoint{3.494684in}{1.788391in}}%
\pgfpathclose%
\pgfusepath{fill}%
\end{pgfscope}%
\begin{pgfscope}%
\pgfpathrectangle{\pgfqpoint{1.150000in}{0.150000in}}{\pgfqpoint{5.700000in}{5.700000in}}%
\pgfusepath{clip}%
\pgfsetbuttcap%
\pgfsetroundjoin%
\definecolor{currentfill}{rgb}{0.269944,0.014625,0.341379}%
\pgfsetfillcolor{currentfill}%
\pgfsetfillopacity{0.700000}%
\pgfsetlinewidth{0.000000pt}%
\definecolor{currentstroke}{rgb}{0.000000,0.000000,0.000000}%
\pgfsetstrokecolor{currentstroke}%
\pgfsetdash{}{0pt}%
\pgfpathmoveto{\pgfqpoint{4.069042in}{1.721680in}}%
\pgfpathlineto{\pgfqpoint{4.082972in}{1.719276in}}%
\pgfpathlineto{\pgfqpoint{4.096910in}{1.716972in}}%
\pgfpathlineto{\pgfqpoint{4.110855in}{1.714770in}}%
\pgfpathlineto{\pgfqpoint{4.124807in}{1.712669in}}%
\pgfpathlineto{\pgfqpoint{4.116771in}{1.703670in}}%
\pgfpathlineto{\pgfqpoint{4.108730in}{1.694780in}}%
\pgfpathlineto{\pgfqpoint{4.100682in}{1.686003in}}%
\pgfpathlineto{\pgfqpoint{4.092629in}{1.677343in}}%
\pgfpathlineto{\pgfqpoint{4.078665in}{1.679856in}}%
\pgfpathlineto{\pgfqpoint{4.064708in}{1.682470in}}%
\pgfpathlineto{\pgfqpoint{4.050758in}{1.685185in}}%
\pgfpathlineto{\pgfqpoint{4.036815in}{1.688001in}}%
\pgfpathlineto{\pgfqpoint{4.044881in}{1.696242in}}%
\pgfpathlineto{\pgfqpoint{4.052941in}{1.704606in}}%
\pgfpathlineto{\pgfqpoint{4.060994in}{1.713086in}}%
\pgfpathlineto{\pgfqpoint{4.069042in}{1.721680in}}%
\pgfpathclose%
\pgfusepath{fill}%
\end{pgfscope}%
\begin{pgfscope}%
\pgfpathrectangle{\pgfqpoint{1.150000in}{0.150000in}}{\pgfqpoint{5.700000in}{5.700000in}}%
\pgfusepath{clip}%
\pgfsetbuttcap%
\pgfsetroundjoin%
\definecolor{currentfill}{rgb}{0.244972,0.287675,0.537260}%
\pgfsetfillcolor{currentfill}%
\pgfsetfillopacity{0.700000}%
\pgfsetlinewidth{0.000000pt}%
\definecolor{currentstroke}{rgb}{0.000000,0.000000,0.000000}%
\pgfsetstrokecolor{currentstroke}%
\pgfsetdash{}{0pt}%
\pgfpathmoveto{\pgfqpoint{2.817562in}{2.271460in}}%
\pgfpathlineto{\pgfqpoint{2.831443in}{2.257368in}}%
\pgfpathlineto{\pgfqpoint{2.845321in}{2.243418in}}%
\pgfpathlineto{\pgfqpoint{2.859196in}{2.229611in}}%
\pgfpathlineto{\pgfqpoint{2.873070in}{2.215944in}}%
\pgfpathlineto{\pgfqpoint{2.864274in}{2.220710in}}%
\pgfpathlineto{\pgfqpoint{2.855460in}{2.225794in}}%
\pgfpathlineto{\pgfqpoint{2.846628in}{2.231201in}}%
\pgfpathlineto{\pgfqpoint{2.837776in}{2.236936in}}%
\pgfpathlineto{\pgfqpoint{2.823857in}{2.251149in}}%
\pgfpathlineto{\pgfqpoint{2.809935in}{2.265505in}}%
\pgfpathlineto{\pgfqpoint{2.796010in}{2.280002in}}%
\pgfpathlineto{\pgfqpoint{2.782083in}{2.294643in}}%
\pgfpathlineto{\pgfqpoint{2.790982in}{2.288352in}}%
\pgfpathlineto{\pgfqpoint{2.799861in}{2.282395in}}%
\pgfpathlineto{\pgfqpoint{2.808721in}{2.276766in}}%
\pgfpathlineto{\pgfqpoint{2.817562in}{2.271460in}}%
\pgfpathclose%
\pgfusepath{fill}%
\end{pgfscope}%
\begin{pgfscope}%
\pgfpathrectangle{\pgfqpoint{1.150000in}{0.150000in}}{\pgfqpoint{5.700000in}{5.700000in}}%
\pgfusepath{clip}%
\pgfsetbuttcap%
\pgfsetroundjoin%
\definecolor{currentfill}{rgb}{0.229739,0.322361,0.545706}%
\pgfsetfillcolor{currentfill}%
\pgfsetfillopacity{0.700000}%
\pgfsetlinewidth{0.000000pt}%
\definecolor{currentstroke}{rgb}{0.000000,0.000000,0.000000}%
\pgfsetstrokecolor{currentstroke}%
\pgfsetdash{}{0pt}%
\pgfpathmoveto{\pgfqpoint{5.067817in}{2.346630in}}%
\pgfpathlineto{\pgfqpoint{5.082157in}{2.352161in}}%
\pgfpathlineto{\pgfqpoint{5.096511in}{2.357791in}}%
\pgfpathlineto{\pgfqpoint{5.110878in}{2.363520in}}%
\pgfpathlineto{\pgfqpoint{5.125259in}{2.369349in}}%
\pgfpathlineto{\pgfqpoint{5.117507in}{2.356751in}}%
\pgfpathlineto{\pgfqpoint{5.109750in}{2.344078in}}%
\pgfpathlineto{\pgfqpoint{5.101986in}{2.331331in}}%
\pgfpathlineto{\pgfqpoint{5.094217in}{2.318512in}}%
\pgfpathlineto{\pgfqpoint{5.079836in}{2.312881in}}%
\pgfpathlineto{\pgfqpoint{5.065468in}{2.307351in}}%
\pgfpathlineto{\pgfqpoint{5.051114in}{2.301919in}}%
\pgfpathlineto{\pgfqpoint{5.036773in}{2.296587in}}%
\pgfpathlineto{\pgfqpoint{5.044543in}{2.309200in}}%
\pgfpathlineto{\pgfqpoint{5.052306in}{2.321747in}}%
\pgfpathlineto{\pgfqpoint{5.060064in}{2.334224in}}%
\pgfpathlineto{\pgfqpoint{5.067817in}{2.346630in}}%
\pgfpathclose%
\pgfusepath{fill}%
\end{pgfscope}%
\begin{pgfscope}%
\pgfpathrectangle{\pgfqpoint{1.150000in}{0.150000in}}{\pgfqpoint{5.700000in}{5.700000in}}%
\pgfusepath{clip}%
\pgfsetbuttcap%
\pgfsetroundjoin%
\definecolor{currentfill}{rgb}{0.166617,0.463708,0.558119}%
\pgfsetfillcolor{currentfill}%
\pgfsetfillopacity{0.700000}%
\pgfsetlinewidth{0.000000pt}%
\definecolor{currentstroke}{rgb}{0.000000,0.000000,0.000000}%
\pgfsetstrokecolor{currentstroke}%
\pgfsetdash{}{0pt}%
\pgfpathmoveto{\pgfqpoint{2.427381in}{2.728485in}}%
\pgfpathlineto{\pgfqpoint{2.441387in}{2.709928in}}%
\pgfpathlineto{\pgfqpoint{2.455386in}{2.691548in}}%
\pgfpathlineto{\pgfqpoint{2.469378in}{2.673345in}}%
\pgfpathlineto{\pgfqpoint{2.483364in}{2.655316in}}%
\pgfpathlineto{\pgfqpoint{2.474220in}{2.663920in}}%
\pgfpathlineto{\pgfqpoint{2.465053in}{2.672881in}}%
\pgfpathlineto{\pgfqpoint{2.455862in}{2.682207in}}%
\pgfpathlineto{\pgfqpoint{2.446647in}{2.691902in}}%
\pgfpathlineto{\pgfqpoint{2.432604in}{2.710500in}}%
\pgfpathlineto{\pgfqpoint{2.418555in}{2.729274in}}%
\pgfpathlineto{\pgfqpoint{2.404498in}{2.748225in}}%
\pgfpathlineto{\pgfqpoint{2.390435in}{2.767355in}}%
\pgfpathlineto{\pgfqpoint{2.399708in}{2.757080in}}%
\pgfpathlineto{\pgfqpoint{2.408957in}{2.747181in}}%
\pgfpathlineto{\pgfqpoint{2.418181in}{2.737651in}}%
\pgfpathlineto{\pgfqpoint{2.427381in}{2.728485in}}%
\pgfpathclose%
\pgfusepath{fill}%
\end{pgfscope}%
\begin{pgfscope}%
\pgfpathrectangle{\pgfqpoint{1.150000in}{0.150000in}}{\pgfqpoint{5.700000in}{5.700000in}}%
\pgfusepath{clip}%
\pgfsetbuttcap%
\pgfsetroundjoin%
\definecolor{currentfill}{rgb}{0.253935,0.265254,0.529983}%
\pgfsetfillcolor{currentfill}%
\pgfsetfillopacity{0.700000}%
\pgfsetlinewidth{0.000000pt}%
\definecolor{currentstroke}{rgb}{0.000000,0.000000,0.000000}%
\pgfsetstrokecolor{currentstroke}%
\pgfsetdash{}{0pt}%
\pgfpathmoveto{\pgfqpoint{2.873070in}{2.215944in}}%
\pgfpathlineto{\pgfqpoint{2.886942in}{2.202416in}}%
\pgfpathlineto{\pgfqpoint{2.900812in}{2.189028in}}%
\pgfpathlineto{\pgfqpoint{2.914681in}{2.175777in}}%
\pgfpathlineto{\pgfqpoint{2.928548in}{2.162664in}}%
\pgfpathlineto{\pgfqpoint{2.919796in}{2.166894in}}%
\pgfpathlineto{\pgfqpoint{2.911026in}{2.171435in}}%
\pgfpathlineto{\pgfqpoint{2.902238in}{2.176294in}}%
\pgfpathlineto{\pgfqpoint{2.893433in}{2.181475in}}%
\pgfpathlineto{\pgfqpoint{2.879521in}{2.195133in}}%
\pgfpathlineto{\pgfqpoint{2.865608in}{2.208928in}}%
\pgfpathlineto{\pgfqpoint{2.851693in}{2.222862in}}%
\pgfpathlineto{\pgfqpoint{2.837776in}{2.236936in}}%
\pgfpathlineto{\pgfqpoint{2.846628in}{2.231201in}}%
\pgfpathlineto{\pgfqpoint{2.855460in}{2.225794in}}%
\pgfpathlineto{\pgfqpoint{2.864274in}{2.220710in}}%
\pgfpathlineto{\pgfqpoint{2.873070in}{2.215944in}}%
\pgfpathclose%
\pgfusepath{fill}%
\end{pgfscope}%
\begin{pgfscope}%
\pgfpathrectangle{\pgfqpoint{1.150000in}{0.150000in}}{\pgfqpoint{5.700000in}{5.700000in}}%
\pgfusepath{clip}%
\pgfsetbuttcap%
\pgfsetroundjoin%
\definecolor{currentfill}{rgb}{0.121831,0.589055,0.545623}%
\pgfsetfillcolor{currentfill}%
\pgfsetfillopacity{0.700000}%
\pgfsetlinewidth{0.000000pt}%
\definecolor{currentstroke}{rgb}{0.000000,0.000000,0.000000}%
\pgfsetstrokecolor{currentstroke}%
\pgfsetdash{}{0pt}%
\pgfpathmoveto{\pgfqpoint{5.867437in}{3.061748in}}%
\pgfpathlineto{\pgfqpoint{5.882236in}{3.071416in}}%
\pgfpathlineto{\pgfqpoint{5.897053in}{3.081187in}}%
\pgfpathlineto{\pgfqpoint{5.911888in}{3.091059in}}%
\pgfpathlineto{\pgfqpoint{5.904510in}{3.082876in}}%
\pgfpathlineto{\pgfqpoint{5.897122in}{3.074554in}}%
\pgfpathlineto{\pgfqpoint{5.889723in}{3.066092in}}%
\pgfpathlineto{\pgfqpoint{5.882315in}{3.057490in}}%
\pgfpathlineto{\pgfqpoint{5.867473in}{3.047602in}}%
\pgfpathlineto{\pgfqpoint{5.852648in}{3.037817in}}%
\pgfpathlineto{\pgfqpoint{5.837840in}{3.028133in}}%
\pgfpathlineto{\pgfqpoint{5.845254in}{3.036741in}}%
\pgfpathlineto{\pgfqpoint{5.852658in}{3.045212in}}%
\pgfpathlineto{\pgfqpoint{5.860052in}{3.053548in}}%
\pgfpathlineto{\pgfqpoint{5.867437in}{3.061748in}}%
\pgfpathclose%
\pgfusepath{fill}%
\end{pgfscope}%
\begin{pgfscope}%
\pgfpathrectangle{\pgfqpoint{1.150000in}{0.150000in}}{\pgfqpoint{5.700000in}{5.700000in}}%
\pgfusepath{clip}%
\pgfsetbuttcap%
\pgfsetroundjoin%
\definecolor{currentfill}{rgb}{0.268510,0.009605,0.335427}%
\pgfsetfillcolor{currentfill}%
\pgfsetfillopacity{0.700000}%
\pgfsetlinewidth{0.000000pt}%
\definecolor{currentstroke}{rgb}{0.000000,0.000000,0.000000}%
\pgfsetstrokecolor{currentstroke}%
\pgfsetdash{}{0pt}%
\pgfpathmoveto{\pgfqpoint{3.837512in}{1.703273in}}%
\pgfpathlineto{\pgfqpoint{3.851394in}{1.698792in}}%
\pgfpathlineto{\pgfqpoint{3.865281in}{1.694416in}}%
\pgfpathlineto{\pgfqpoint{3.879174in}{1.690144in}}%
\pgfpathlineto{\pgfqpoint{3.893073in}{1.685975in}}%
\pgfpathlineto{\pgfqpoint{3.884943in}{1.679286in}}%
\pgfpathlineto{\pgfqpoint{3.876807in}{1.672753in}}%
\pgfpathlineto{\pgfqpoint{3.868663in}{1.666380in}}%
\pgfpathlineto{\pgfqpoint{3.860512in}{1.660173in}}%
\pgfpathlineto{\pgfqpoint{3.846596in}{1.664789in}}%
\pgfpathlineto{\pgfqpoint{3.832685in}{1.669509in}}%
\pgfpathlineto{\pgfqpoint{3.818780in}{1.674333in}}%
\pgfpathlineto{\pgfqpoint{3.804881in}{1.679262in}}%
\pgfpathlineto{\pgfqpoint{3.813050in}{1.685014in}}%
\pgfpathlineto{\pgfqpoint{3.821212in}{1.690937in}}%
\pgfpathlineto{\pgfqpoint{3.829366in}{1.697024in}}%
\pgfpathlineto{\pgfqpoint{3.837512in}{1.703273in}}%
\pgfpathclose%
\pgfusepath{fill}%
\end{pgfscope}%
\begin{pgfscope}%
\pgfpathrectangle{\pgfqpoint{1.150000in}{0.150000in}}{\pgfqpoint{5.700000in}{5.700000in}}%
\pgfusepath{clip}%
\pgfsetbuttcap%
\pgfsetroundjoin%
\definecolor{currentfill}{rgb}{0.151918,0.500685,0.557587}%
\pgfsetfillcolor{currentfill}%
\pgfsetfillopacity{0.700000}%
\pgfsetlinewidth{0.000000pt}%
\definecolor{currentstroke}{rgb}{0.000000,0.000000,0.000000}%
\pgfsetstrokecolor{currentstroke}%
\pgfsetdash{}{0pt}%
\pgfpathmoveto{\pgfqpoint{5.571564in}{2.806718in}}%
\pgfpathlineto{\pgfqpoint{5.586187in}{2.815109in}}%
\pgfpathlineto{\pgfqpoint{5.600827in}{2.823602in}}%
\pgfpathlineto{\pgfqpoint{5.615482in}{2.832196in}}%
\pgfpathlineto{\pgfqpoint{5.630155in}{2.840891in}}%
\pgfpathlineto{\pgfqpoint{5.622611in}{2.830606in}}%
\pgfpathlineto{\pgfqpoint{5.615059in}{2.820195in}}%
\pgfpathlineto{\pgfqpoint{5.607499in}{2.809656in}}%
\pgfpathlineto{\pgfqpoint{5.599930in}{2.798991in}}%
\pgfpathlineto{\pgfqpoint{5.585254in}{2.790361in}}%
\pgfpathlineto{\pgfqpoint{5.570595in}{2.781832in}}%
\pgfpathlineto{\pgfqpoint{5.555952in}{2.773404in}}%
\pgfpathlineto{\pgfqpoint{5.541325in}{2.765077in}}%
\pgfpathlineto{\pgfqpoint{5.548897in}{2.775670in}}%
\pgfpathlineto{\pgfqpoint{5.556460in}{2.786141in}}%
\pgfpathlineto{\pgfqpoint{5.564016in}{2.796491in}}%
\pgfpathlineto{\pgfqpoint{5.571564in}{2.806718in}}%
\pgfpathclose%
\pgfusepath{fill}%
\end{pgfscope}%
\begin{pgfscope}%
\pgfpathrectangle{\pgfqpoint{1.150000in}{0.150000in}}{\pgfqpoint{5.700000in}{5.700000in}}%
\pgfusepath{clip}%
\pgfsetbuttcap%
\pgfsetroundjoin%
\definecolor{currentfill}{rgb}{0.271305,0.019942,0.347269}%
\pgfsetfillcolor{currentfill}%
\pgfsetfillopacity{0.700000}%
\pgfsetlinewidth{0.000000pt}%
\definecolor{currentstroke}{rgb}{0.000000,0.000000,0.000000}%
\pgfsetstrokecolor{currentstroke}%
\pgfsetdash{}{0pt}%
\pgfpathmoveto{\pgfqpoint{3.693871in}{1.722485in}}%
\pgfpathlineto{\pgfqpoint{3.707729in}{1.716710in}}%
\pgfpathlineto{\pgfqpoint{3.721593in}{1.711042in}}%
\pgfpathlineto{\pgfqpoint{3.735462in}{1.705481in}}%
\pgfpathlineto{\pgfqpoint{3.749335in}{1.700026in}}%
\pgfpathlineto{\pgfqpoint{3.741138in}{1.694907in}}%
\pgfpathlineto{\pgfqpoint{3.732932in}{1.689973in}}%
\pgfpathlineto{\pgfqpoint{3.724718in}{1.685226in}}%
\pgfpathlineto{\pgfqpoint{3.716495in}{1.680672in}}%
\pgfpathlineto{\pgfqpoint{3.702600in}{1.686594in}}%
\pgfpathlineto{\pgfqpoint{3.688710in}{1.692623in}}%
\pgfpathlineto{\pgfqpoint{3.674824in}{1.698758in}}%
\pgfpathlineto{\pgfqpoint{3.660943in}{1.705000in}}%
\pgfpathlineto{\pgfqpoint{3.669189in}{1.709079in}}%
\pgfpathlineto{\pgfqpoint{3.677425in}{1.713356in}}%
\pgfpathlineto{\pgfqpoint{3.685652in}{1.717826in}}%
\pgfpathlineto{\pgfqpoint{3.693871in}{1.722485in}}%
\pgfpathclose%
\pgfusepath{fill}%
\end{pgfscope}%
\begin{pgfscope}%
\pgfpathrectangle{\pgfqpoint{1.150000in}{0.150000in}}{\pgfqpoint{5.700000in}{5.700000in}}%
\pgfusepath{clip}%
\pgfsetbuttcap%
\pgfsetroundjoin%
\definecolor{currentfill}{rgb}{0.154815,0.493313,0.557840}%
\pgfsetfillcolor{currentfill}%
\pgfsetfillopacity{0.700000}%
\pgfsetlinewidth{0.000000pt}%
\definecolor{currentstroke}{rgb}{0.000000,0.000000,0.000000}%
\pgfsetstrokecolor{currentstroke}%
\pgfsetdash{}{0pt}%
\pgfpathmoveto{\pgfqpoint{2.371288in}{2.804524in}}%
\pgfpathlineto{\pgfqpoint{2.385322in}{2.785240in}}%
\pgfpathlineto{\pgfqpoint{2.399349in}{2.766140in}}%
\pgfpathlineto{\pgfqpoint{2.413369in}{2.747222in}}%
\pgfpathlineto{\pgfqpoint{2.427381in}{2.728485in}}%
\pgfpathlineto{\pgfqpoint{2.418181in}{2.737651in}}%
\pgfpathlineto{\pgfqpoint{2.408957in}{2.747181in}}%
\pgfpathlineto{\pgfqpoint{2.399708in}{2.757080in}}%
\pgfpathlineto{\pgfqpoint{2.390435in}{2.767355in}}%
\pgfpathlineto{\pgfqpoint{2.376364in}{2.786666in}}%
\pgfpathlineto{\pgfqpoint{2.362286in}{2.806158in}}%
\pgfpathlineto{\pgfqpoint{2.348200in}{2.825835in}}%
\pgfpathlineto{\pgfqpoint{2.334106in}{2.845696in}}%
\pgfpathlineto{\pgfqpoint{2.343439in}{2.834836in}}%
\pgfpathlineto{\pgfqpoint{2.352747in}{2.824358in}}%
\pgfpathlineto{\pgfqpoint{2.362030in}{2.814256in}}%
\pgfpathlineto{\pgfqpoint{2.371288in}{2.804524in}}%
\pgfpathclose%
\pgfusepath{fill}%
\end{pgfscope}%
\begin{pgfscope}%
\pgfpathrectangle{\pgfqpoint{1.150000in}{0.150000in}}{\pgfqpoint{5.700000in}{5.700000in}}%
\pgfusepath{clip}%
\pgfsetbuttcap%
\pgfsetroundjoin%
\definecolor{currentfill}{rgb}{0.283091,0.110553,0.431554}%
\pgfsetfillcolor{currentfill}%
\pgfsetfillopacity{0.700000}%
\pgfsetlinewidth{0.000000pt}%
\definecolor{currentstroke}{rgb}{0.000000,0.000000,0.000000}%
\pgfsetstrokecolor{currentstroke}%
\pgfsetdash{}{0pt}%
\pgfpathmoveto{\pgfqpoint{3.295151in}{1.885447in}}%
\pgfpathlineto{\pgfqpoint{3.308985in}{1.876020in}}%
\pgfpathlineto{\pgfqpoint{3.322821in}{1.866712in}}%
\pgfpathlineto{\pgfqpoint{3.336658in}{1.857520in}}%
\pgfpathlineto{\pgfqpoint{3.350498in}{1.848446in}}%
\pgfpathlineto{\pgfqpoint{3.342064in}{1.847932in}}%
\pgfpathlineto{\pgfqpoint{3.333618in}{1.847670in}}%
\pgfpathlineto{\pgfqpoint{3.325159in}{1.847666in}}%
\pgfpathlineto{\pgfqpoint{3.316688in}{1.847925in}}%
\pgfpathlineto{\pgfqpoint{3.302816in}{1.857511in}}%
\pgfpathlineto{\pgfqpoint{3.288945in}{1.867213in}}%
\pgfpathlineto{\pgfqpoint{3.275076in}{1.877033in}}%
\pgfpathlineto{\pgfqpoint{3.261209in}{1.886972in}}%
\pgfpathlineto{\pgfqpoint{3.269714in}{1.886194in}}%
\pgfpathlineto{\pgfqpoint{3.278206in}{1.885684in}}%
\pgfpathlineto{\pgfqpoint{3.286685in}{1.885437in}}%
\pgfpathlineto{\pgfqpoint{3.295151in}{1.885447in}}%
\pgfpathclose%
\pgfusepath{fill}%
\end{pgfscope}%
\begin{pgfscope}%
\pgfpathrectangle{\pgfqpoint{1.150000in}{0.150000in}}{\pgfqpoint{5.700000in}{5.700000in}}%
\pgfusepath{clip}%
\pgfsetbuttcap%
\pgfsetroundjoin%
\definecolor{currentfill}{rgb}{0.262138,0.242286,0.520837}%
\pgfsetfillcolor{currentfill}%
\pgfsetfillopacity{0.700000}%
\pgfsetlinewidth{0.000000pt}%
\definecolor{currentstroke}{rgb}{0.000000,0.000000,0.000000}%
\pgfsetstrokecolor{currentstroke}%
\pgfsetdash{}{0pt}%
\pgfpathmoveto{\pgfqpoint{4.860159in}{2.158006in}}%
\pgfpathlineto{\pgfqpoint{4.874398in}{2.162094in}}%
\pgfpathlineto{\pgfqpoint{4.888649in}{2.166281in}}%
\pgfpathlineto{\pgfqpoint{4.902912in}{2.170567in}}%
\pgfpathlineto{\pgfqpoint{4.917189in}{2.174951in}}%
\pgfpathlineto{\pgfqpoint{4.909371in}{2.162065in}}%
\pgfpathlineto{\pgfqpoint{4.901548in}{2.149136in}}%
\pgfpathlineto{\pgfqpoint{4.893720in}{2.136166in}}%
\pgfpathlineto{\pgfqpoint{4.885888in}{2.123159in}}%
\pgfpathlineto{\pgfqpoint{4.871611in}{2.119027in}}%
\pgfpathlineto{\pgfqpoint{4.857347in}{2.114995in}}%
\pgfpathlineto{\pgfqpoint{4.843095in}{2.111061in}}%
\pgfpathlineto{\pgfqpoint{4.828855in}{2.107227in}}%
\pgfpathlineto{\pgfqpoint{4.836689in}{2.119974in}}%
\pgfpathlineto{\pgfqpoint{4.844517in}{2.132688in}}%
\pgfpathlineto{\pgfqpoint{4.852341in}{2.145366in}}%
\pgfpathlineto{\pgfqpoint{4.860159in}{2.158006in}}%
\pgfpathclose%
\pgfusepath{fill}%
\end{pgfscope}%
\begin{pgfscope}%
\pgfpathrectangle{\pgfqpoint{1.150000in}{0.150000in}}{\pgfqpoint{5.700000in}{5.700000in}}%
\pgfusepath{clip}%
\pgfsetbuttcap%
\pgfsetroundjoin%
\definecolor{currentfill}{rgb}{0.262138,0.242286,0.520837}%
\pgfsetfillcolor{currentfill}%
\pgfsetfillopacity{0.700000}%
\pgfsetlinewidth{0.000000pt}%
\definecolor{currentstroke}{rgb}{0.000000,0.000000,0.000000}%
\pgfsetstrokecolor{currentstroke}%
\pgfsetdash{}{0pt}%
\pgfpathmoveto{\pgfqpoint{2.928548in}{2.162664in}}%
\pgfpathlineto{\pgfqpoint{2.942413in}{2.149687in}}%
\pgfpathlineto{\pgfqpoint{2.956277in}{2.136845in}}%
\pgfpathlineto{\pgfqpoint{2.970141in}{2.124137in}}%
\pgfpathlineto{\pgfqpoint{2.984003in}{2.111563in}}%
\pgfpathlineto{\pgfqpoint{2.975293in}{2.115259in}}%
\pgfpathlineto{\pgfqpoint{2.966566in}{2.119260in}}%
\pgfpathlineto{\pgfqpoint{2.957822in}{2.123573in}}%
\pgfpathlineto{\pgfqpoint{2.949061in}{2.128203in}}%
\pgfpathlineto{\pgfqpoint{2.935156in}{2.141319in}}%
\pgfpathlineto{\pgfqpoint{2.921250in}{2.154569in}}%
\pgfpathlineto{\pgfqpoint{2.907342in}{2.167954in}}%
\pgfpathlineto{\pgfqpoint{2.893433in}{2.181475in}}%
\pgfpathlineto{\pgfqpoint{2.902238in}{2.176294in}}%
\pgfpathlineto{\pgfqpoint{2.911026in}{2.171435in}}%
\pgfpathlineto{\pgfqpoint{2.919796in}{2.166894in}}%
\pgfpathlineto{\pgfqpoint{2.928548in}{2.162664in}}%
\pgfpathclose%
\pgfusepath{fill}%
\end{pgfscope}%
\begin{pgfscope}%
\pgfpathrectangle{\pgfqpoint{1.150000in}{0.150000in}}{\pgfqpoint{5.700000in}{5.700000in}}%
\pgfusepath{clip}%
\pgfsetbuttcap%
\pgfsetroundjoin%
\definecolor{currentfill}{rgb}{0.180629,0.429975,0.557282}%
\pgfsetfillcolor{currentfill}%
\pgfsetfillopacity{0.700000}%
\pgfsetlinewidth{0.000000pt}%
\definecolor{currentstroke}{rgb}{0.000000,0.000000,0.000000}%
\pgfsetstrokecolor{currentstroke}%
\pgfsetdash{}{0pt}%
\pgfpathmoveto{\pgfqpoint{5.363972in}{2.614527in}}%
\pgfpathlineto{\pgfqpoint{5.378479in}{2.621861in}}%
\pgfpathlineto{\pgfqpoint{5.393002in}{2.629295in}}%
\pgfpathlineto{\pgfqpoint{5.407540in}{2.636830in}}%
\pgfpathlineto{\pgfqpoint{5.422093in}{2.644465in}}%
\pgfpathlineto{\pgfqpoint{5.414449in}{2.632918in}}%
\pgfpathlineto{\pgfqpoint{5.406798in}{2.621260in}}%
\pgfpathlineto{\pgfqpoint{5.399139in}{2.609493in}}%
\pgfpathlineto{\pgfqpoint{5.391473in}{2.597617in}}%
\pgfpathlineto{\pgfqpoint{5.376918in}{2.590105in}}%
\pgfpathlineto{\pgfqpoint{5.362379in}{2.582694in}}%
\pgfpathlineto{\pgfqpoint{5.347855in}{2.575383in}}%
\pgfpathlineto{\pgfqpoint{5.333346in}{2.568172in}}%
\pgfpathlineto{\pgfqpoint{5.341013in}{2.579917in}}%
\pgfpathlineto{\pgfqpoint{5.348673in}{2.591559in}}%
\pgfpathlineto{\pgfqpoint{5.356326in}{2.603096in}}%
\pgfpathlineto{\pgfqpoint{5.363972in}{2.614527in}}%
\pgfpathclose%
\pgfusepath{fill}%
\end{pgfscope}%
\begin{pgfscope}%
\pgfpathrectangle{\pgfqpoint{1.150000in}{0.150000in}}{\pgfqpoint{5.700000in}{5.700000in}}%
\pgfusepath{clip}%
\pgfsetbuttcap%
\pgfsetroundjoin%
\definecolor{currentfill}{rgb}{0.280894,0.078907,0.402329}%
\pgfsetfillcolor{currentfill}%
\pgfsetfillopacity{0.700000}%
\pgfsetlinewidth{0.000000pt}%
\definecolor{currentstroke}{rgb}{0.000000,0.000000,0.000000}%
\pgfsetstrokecolor{currentstroke}%
\pgfsetdash{}{0pt}%
\pgfpathmoveto{\pgfqpoint{4.388551in}{1.823913in}}%
\pgfpathlineto{\pgfqpoint{4.402589in}{1.824240in}}%
\pgfpathlineto{\pgfqpoint{4.416637in}{1.824666in}}%
\pgfpathlineto{\pgfqpoint{4.430694in}{1.825192in}}%
\pgfpathlineto{\pgfqpoint{4.444760in}{1.825816in}}%
\pgfpathlineto{\pgfqpoint{4.436818in}{1.814396in}}%
\pgfpathlineto{\pgfqpoint{4.428872in}{1.803023in}}%
\pgfpathlineto{\pgfqpoint{4.420920in}{1.791700in}}%
\pgfpathlineto{\pgfqpoint{4.412964in}{1.780431in}}%
\pgfpathlineto{\pgfqpoint{4.398892in}{1.780166in}}%
\pgfpathlineto{\pgfqpoint{4.384829in}{1.780000in}}%
\pgfpathlineto{\pgfqpoint{4.370776in}{1.779932in}}%
\pgfpathlineto{\pgfqpoint{4.356732in}{1.779964in}}%
\pgfpathlineto{\pgfqpoint{4.364694in}{1.790867in}}%
\pgfpathlineto{\pgfqpoint{4.372651in}{1.801829in}}%
\pgfpathlineto{\pgfqpoint{4.380604in}{1.812845in}}%
\pgfpathlineto{\pgfqpoint{4.388551in}{1.823913in}}%
\pgfpathclose%
\pgfusepath{fill}%
\end{pgfscope}%
\begin{pgfscope}%
\pgfpathrectangle{\pgfqpoint{1.150000in}{0.150000in}}{\pgfqpoint{5.700000in}{5.700000in}}%
\pgfusepath{clip}%
\pgfsetbuttcap%
\pgfsetroundjoin%
\definecolor{currentfill}{rgb}{0.268510,0.009605,0.335427}%
\pgfsetfillcolor{currentfill}%
\pgfsetfillopacity{0.700000}%
\pgfsetlinewidth{0.000000pt}%
\definecolor{currentstroke}{rgb}{0.000000,0.000000,0.000000}%
\pgfsetstrokecolor{currentstroke}%
\pgfsetdash{}{0pt}%
\pgfpathmoveto{\pgfqpoint{3.981113in}{1.700284in}}%
\pgfpathlineto{\pgfqpoint{3.995029in}{1.697060in}}%
\pgfpathlineto{\pgfqpoint{4.008951in}{1.693938in}}%
\pgfpathlineto{\pgfqpoint{4.022879in}{1.690919in}}%
\pgfpathlineto{\pgfqpoint{4.036815in}{1.688001in}}%
\pgfpathlineto{\pgfqpoint{4.028743in}{1.679885in}}%
\pgfpathlineto{\pgfqpoint{4.020665in}{1.671899in}}%
\pgfpathlineto{\pgfqpoint{4.012580in}{1.664047in}}%
\pgfpathlineto{\pgfqpoint{4.004488in}{1.656333in}}%
\pgfpathlineto{\pgfqpoint{3.990539in}{1.659680in}}%
\pgfpathlineto{\pgfqpoint{3.976596in}{1.663130in}}%
\pgfpathlineto{\pgfqpoint{3.962659in}{1.666681in}}%
\pgfpathlineto{\pgfqpoint{3.948729in}{1.670334in}}%
\pgfpathlineto{\pgfqpoint{3.956835in}{1.677612in}}%
\pgfpathlineto{\pgfqpoint{3.964935in}{1.685032in}}%
\pgfpathlineto{\pgfqpoint{3.973027in}{1.692591in}}%
\pgfpathlineto{\pgfqpoint{3.981113in}{1.700284in}}%
\pgfpathclose%
\pgfusepath{fill}%
\end{pgfscope}%
\begin{pgfscope}%
\pgfpathrectangle{\pgfqpoint{1.150000in}{0.150000in}}{\pgfqpoint{5.700000in}{5.700000in}}%
\pgfusepath{clip}%
\pgfsetbuttcap%
\pgfsetroundjoin%
\definecolor{currentfill}{rgb}{0.282910,0.105393,0.426902}%
\pgfsetfillcolor{currentfill}%
\pgfsetfillopacity{0.700000}%
\pgfsetlinewidth{0.000000pt}%
\definecolor{currentstroke}{rgb}{0.000000,0.000000,0.000000}%
\pgfsetstrokecolor{currentstroke}%
\pgfsetdash{}{0pt}%
\pgfpathmoveto{\pgfqpoint{4.476482in}{1.871902in}}%
\pgfpathlineto{\pgfqpoint{4.490553in}{1.872967in}}%
\pgfpathlineto{\pgfqpoint{4.504635in}{1.874131in}}%
\pgfpathlineto{\pgfqpoint{4.518727in}{1.875393in}}%
\pgfpathlineto{\pgfqpoint{4.532829in}{1.876754in}}%
\pgfpathlineto{\pgfqpoint{4.524910in}{1.864844in}}%
\pgfpathlineto{\pgfqpoint{4.516986in}{1.852962in}}%
\pgfpathlineto{\pgfqpoint{4.509058in}{1.841114in}}%
\pgfpathlineto{\pgfqpoint{4.501126in}{1.829301in}}%
\pgfpathlineto{\pgfqpoint{4.487019in}{1.828282in}}%
\pgfpathlineto{\pgfqpoint{4.472923in}{1.827361in}}%
\pgfpathlineto{\pgfqpoint{4.458837in}{1.826539in}}%
\pgfpathlineto{\pgfqpoint{4.444760in}{1.825816in}}%
\pgfpathlineto{\pgfqpoint{4.452698in}{1.837280in}}%
\pgfpathlineto{\pgfqpoint{4.460630in}{1.848785in}}%
\pgfpathlineto{\pgfqpoint{4.468558in}{1.860327in}}%
\pgfpathlineto{\pgfqpoint{4.476482in}{1.871902in}}%
\pgfpathclose%
\pgfusepath{fill}%
\end{pgfscope}%
\begin{pgfscope}%
\pgfpathrectangle{\pgfqpoint{1.150000in}{0.150000in}}{\pgfqpoint{5.700000in}{5.700000in}}%
\pgfusepath{clip}%
\pgfsetbuttcap%
\pgfsetroundjoin%
\definecolor{currentfill}{rgb}{0.214298,0.355619,0.551184}%
\pgfsetfillcolor{currentfill}%
\pgfsetfillopacity{0.700000}%
\pgfsetlinewidth{0.000000pt}%
\definecolor{currentstroke}{rgb}{0.000000,0.000000,0.000000}%
\pgfsetstrokecolor{currentstroke}%
\pgfsetdash{}{0pt}%
\pgfpathmoveto{\pgfqpoint{5.156207in}{2.418953in}}%
\pgfpathlineto{\pgfqpoint{5.170602in}{2.425061in}}%
\pgfpathlineto{\pgfqpoint{5.185010in}{2.431269in}}%
\pgfpathlineto{\pgfqpoint{5.199433in}{2.437577in}}%
\pgfpathlineto{\pgfqpoint{5.213871in}{2.443984in}}%
\pgfpathlineto{\pgfqpoint{5.206143in}{2.431531in}}%
\pgfpathlineto{\pgfqpoint{5.198410in}{2.418992in}}%
\pgfpathlineto{\pgfqpoint{5.190670in}{2.406367in}}%
\pgfpathlineto{\pgfqpoint{5.182924in}{2.393659in}}%
\pgfpathlineto{\pgfqpoint{5.168487in}{2.387432in}}%
\pgfpathlineto{\pgfqpoint{5.154064in}{2.381305in}}%
\pgfpathlineto{\pgfqpoint{5.139654in}{2.375277in}}%
\pgfpathlineto{\pgfqpoint{5.125259in}{2.369349in}}%
\pgfpathlineto{\pgfqpoint{5.133005in}{2.381869in}}%
\pgfpathlineto{\pgfqpoint{5.140745in}{2.394311in}}%
\pgfpathlineto{\pgfqpoint{5.148479in}{2.406673in}}%
\pgfpathlineto{\pgfqpoint{5.156207in}{2.418953in}}%
\pgfpathclose%
\pgfusepath{fill}%
\end{pgfscope}%
\begin{pgfscope}%
\pgfpathrectangle{\pgfqpoint{1.150000in}{0.150000in}}{\pgfqpoint{5.700000in}{5.700000in}}%
\pgfusepath{clip}%
\pgfsetbuttcap%
\pgfsetroundjoin%
\definecolor{currentfill}{rgb}{0.277941,0.056324,0.381191}%
\pgfsetfillcolor{currentfill}%
\pgfsetfillopacity{0.700000}%
\pgfsetlinewidth{0.000000pt}%
\definecolor{currentstroke}{rgb}{0.000000,0.000000,0.000000}%
\pgfsetstrokecolor{currentstroke}%
\pgfsetdash{}{0pt}%
\pgfpathmoveto{\pgfqpoint{4.300646in}{1.781084in}}%
\pgfpathlineto{\pgfqpoint{4.314654in}{1.780655in}}%
\pgfpathlineto{\pgfqpoint{4.328671in}{1.780325in}}%
\pgfpathlineto{\pgfqpoint{4.342697in}{1.780095in}}%
\pgfpathlineto{\pgfqpoint{4.356732in}{1.779964in}}%
\pgfpathlineto{\pgfqpoint{4.348765in}{1.769123in}}%
\pgfpathlineto{\pgfqpoint{4.340793in}{1.758347in}}%
\pgfpathlineto{\pgfqpoint{4.332816in}{1.747640in}}%
\pgfpathlineto{\pgfqpoint{4.324835in}{1.737006in}}%
\pgfpathlineto{\pgfqpoint{4.310792in}{1.737514in}}%
\pgfpathlineto{\pgfqpoint{4.296759in}{1.738121in}}%
\pgfpathlineto{\pgfqpoint{4.282735in}{1.738827in}}%
\pgfpathlineto{\pgfqpoint{4.268719in}{1.739633in}}%
\pgfpathlineto{\pgfqpoint{4.276708in}{1.749883in}}%
\pgfpathlineto{\pgfqpoint{4.284692in}{1.760211in}}%
\pgfpathlineto{\pgfqpoint{4.292672in}{1.770613in}}%
\pgfpathlineto{\pgfqpoint{4.300646in}{1.781084in}}%
\pgfpathclose%
\pgfusepath{fill}%
\end{pgfscope}%
\begin{pgfscope}%
\pgfpathrectangle{\pgfqpoint{1.150000in}{0.150000in}}{\pgfqpoint{5.700000in}{5.700000in}}%
\pgfusepath{clip}%
\pgfsetbuttcap%
\pgfsetroundjoin%
\definecolor{currentfill}{rgb}{0.143343,0.522773,0.556295}%
\pgfsetfillcolor{currentfill}%
\pgfsetfillopacity{0.700000}%
\pgfsetlinewidth{0.000000pt}%
\definecolor{currentstroke}{rgb}{0.000000,0.000000,0.000000}%
\pgfsetstrokecolor{currentstroke}%
\pgfsetdash{}{0pt}%
\pgfpathmoveto{\pgfqpoint{2.315072in}{2.883539in}}%
\pgfpathlineto{\pgfqpoint{2.329139in}{2.863500in}}%
\pgfpathlineto{\pgfqpoint{2.343196in}{2.843653in}}%
\pgfpathlineto{\pgfqpoint{2.357246in}{2.823995in}}%
\pgfpathlineto{\pgfqpoint{2.371288in}{2.804524in}}%
\pgfpathlineto{\pgfqpoint{2.362030in}{2.814256in}}%
\pgfpathlineto{\pgfqpoint{2.352747in}{2.824358in}}%
\pgfpathlineto{\pgfqpoint{2.343439in}{2.834836in}}%
\pgfpathlineto{\pgfqpoint{2.334106in}{2.845696in}}%
\pgfpathlineto{\pgfqpoint{2.320004in}{2.865745in}}%
\pgfpathlineto{\pgfqpoint{2.305894in}{2.885982in}}%
\pgfpathlineto{\pgfqpoint{2.291775in}{2.906410in}}%
\pgfpathlineto{\pgfqpoint{2.277648in}{2.927031in}}%
\pgfpathlineto{\pgfqpoint{2.287043in}{2.915581in}}%
\pgfpathlineto{\pgfqpoint{2.296412in}{2.904520in}}%
\pgfpathlineto{\pgfqpoint{2.305755in}{2.893841in}}%
\pgfpathlineto{\pgfqpoint{2.315072in}{2.883539in}}%
\pgfpathclose%
\pgfusepath{fill}%
\end{pgfscope}%
\begin{pgfscope}%
\pgfpathrectangle{\pgfqpoint{1.150000in}{0.150000in}}{\pgfqpoint{5.700000in}{5.700000in}}%
\pgfusepath{clip}%
\pgfsetbuttcap%
\pgfsetroundjoin%
\definecolor{currentfill}{rgb}{0.283072,0.130895,0.449241}%
\pgfsetfillcolor{currentfill}%
\pgfsetfillopacity{0.700000}%
\pgfsetlinewidth{0.000000pt}%
\definecolor{currentstroke}{rgb}{0.000000,0.000000,0.000000}%
\pgfsetstrokecolor{currentstroke}%
\pgfsetdash{}{0pt}%
\pgfpathmoveto{\pgfqpoint{4.564459in}{1.924630in}}%
\pgfpathlineto{\pgfqpoint{4.578568in}{1.926414in}}%
\pgfpathlineto{\pgfqpoint{4.592688in}{1.928297in}}%
\pgfpathlineto{\pgfqpoint{4.606818in}{1.930278in}}%
\pgfpathlineto{\pgfqpoint{4.620959in}{1.932358in}}%
\pgfpathlineto{\pgfqpoint{4.613061in}{1.920043in}}%
\pgfpathlineto{\pgfqpoint{4.605159in}{1.907740in}}%
\pgfpathlineto{\pgfqpoint{4.597253in}{1.895453in}}%
\pgfpathlineto{\pgfqpoint{4.589341in}{1.883185in}}%
\pgfpathlineto{\pgfqpoint{4.575198in}{1.881430in}}%
\pgfpathlineto{\pgfqpoint{4.561064in}{1.879773in}}%
\pgfpathlineto{\pgfqpoint{4.546941in}{1.878214in}}%
\pgfpathlineto{\pgfqpoint{4.532829in}{1.876754in}}%
\pgfpathlineto{\pgfqpoint{4.540743in}{1.888691in}}%
\pgfpathlineto{\pgfqpoint{4.548653in}{1.900652in}}%
\pgfpathlineto{\pgfqpoint{4.556559in}{1.912632in}}%
\pgfpathlineto{\pgfqpoint{4.564459in}{1.924630in}}%
\pgfpathclose%
\pgfusepath{fill}%
\end{pgfscope}%
\begin{pgfscope}%
\pgfpathrectangle{\pgfqpoint{1.150000in}{0.150000in}}{\pgfqpoint{5.700000in}{5.700000in}}%
\pgfusepath{clip}%
\pgfsetbuttcap%
\pgfsetroundjoin%
\definecolor{currentfill}{rgb}{0.267968,0.223549,0.512008}%
\pgfsetfillcolor{currentfill}%
\pgfsetfillopacity{0.700000}%
\pgfsetlinewidth{0.000000pt}%
\definecolor{currentstroke}{rgb}{0.000000,0.000000,0.000000}%
\pgfsetstrokecolor{currentstroke}%
\pgfsetdash{}{0pt}%
\pgfpathmoveto{\pgfqpoint{2.984003in}{2.111563in}}%
\pgfpathlineto{\pgfqpoint{2.997864in}{2.099122in}}%
\pgfpathlineto{\pgfqpoint{3.011724in}{2.086813in}}%
\pgfpathlineto{\pgfqpoint{3.025584in}{2.074635in}}%
\pgfpathlineto{\pgfqpoint{3.039443in}{2.062588in}}%
\pgfpathlineto{\pgfqpoint{3.030774in}{2.065751in}}%
\pgfpathlineto{\pgfqpoint{3.022089in}{2.069214in}}%
\pgfpathlineto{\pgfqpoint{3.013387in}{2.072984in}}%
\pgfpathlineto{\pgfqpoint{3.004669in}{2.077065in}}%
\pgfpathlineto{\pgfqpoint{2.990768in}{2.089652in}}%
\pgfpathlineto{\pgfqpoint{2.976867in}{2.102371in}}%
\pgfpathlineto{\pgfqpoint{2.962964in}{2.115221in}}%
\pgfpathlineto{\pgfqpoint{2.949061in}{2.128203in}}%
\pgfpathlineto{\pgfqpoint{2.957822in}{2.123573in}}%
\pgfpathlineto{\pgfqpoint{2.966566in}{2.119260in}}%
\pgfpathlineto{\pgfqpoint{2.975293in}{2.115259in}}%
\pgfpathlineto{\pgfqpoint{2.984003in}{2.111563in}}%
\pgfpathclose%
\pgfusepath{fill}%
\end{pgfscope}%
\begin{pgfscope}%
\pgfpathrectangle{\pgfqpoint{1.150000in}{0.150000in}}{\pgfqpoint{5.700000in}{5.700000in}}%
\pgfusepath{clip}%
\pgfsetbuttcap%
\pgfsetroundjoin%
\definecolor{currentfill}{rgb}{0.276022,0.044167,0.370164}%
\pgfsetfillcolor{currentfill}%
\pgfsetfillopacity{0.700000}%
\pgfsetlinewidth{0.000000pt}%
\definecolor{currentstroke}{rgb}{0.000000,0.000000,0.000000}%
\pgfsetstrokecolor{currentstroke}%
\pgfsetdash{}{0pt}%
\pgfpathmoveto{\pgfqpoint{3.550044in}{1.758834in}}%
\pgfpathlineto{\pgfqpoint{3.563893in}{1.751723in}}%
\pgfpathlineto{\pgfqpoint{3.577746in}{1.744721in}}%
\pgfpathlineto{\pgfqpoint{3.591602in}{1.737829in}}%
\pgfpathlineto{\pgfqpoint{3.605462in}{1.731046in}}%
\pgfpathlineto{\pgfqpoint{3.597184in}{1.727649in}}%
\pgfpathlineto{\pgfqpoint{3.588895in}{1.724464in}}%
\pgfpathlineto{\pgfqpoint{3.580597in}{1.721497in}}%
\pgfpathlineto{\pgfqpoint{3.572289in}{1.718752in}}%
\pgfpathlineto{\pgfqpoint{3.558403in}{1.726022in}}%
\pgfpathlineto{\pgfqpoint{3.544522in}{1.733401in}}%
\pgfpathlineto{\pgfqpoint{3.530643in}{1.740890in}}%
\pgfpathlineto{\pgfqpoint{3.516768in}{1.748490in}}%
\pgfpathlineto{\pgfqpoint{3.525103in}{1.750740in}}%
\pgfpathlineto{\pgfqpoint{3.533427in}{1.753218in}}%
\pgfpathlineto{\pgfqpoint{3.541741in}{1.755917in}}%
\pgfpathlineto{\pgfqpoint{3.550044in}{1.758834in}}%
\pgfpathclose%
\pgfusepath{fill}%
\end{pgfscope}%
\begin{pgfscope}%
\pgfpathrectangle{\pgfqpoint{1.150000in}{0.150000in}}{\pgfqpoint{5.700000in}{5.700000in}}%
\pgfusepath{clip}%
\pgfsetbuttcap%
\pgfsetroundjoin%
\definecolor{currentfill}{rgb}{0.250425,0.274290,0.533103}%
\pgfsetfillcolor{currentfill}%
\pgfsetfillopacity{0.700000}%
\pgfsetlinewidth{0.000000pt}%
\definecolor{currentstroke}{rgb}{0.000000,0.000000,0.000000}%
\pgfsetstrokecolor{currentstroke}%
\pgfsetdash{}{0pt}%
\pgfpathmoveto{\pgfqpoint{4.948408in}{2.226027in}}%
\pgfpathlineto{\pgfqpoint{4.962697in}{2.230745in}}%
\pgfpathlineto{\pgfqpoint{4.976998in}{2.235563in}}%
\pgfpathlineto{\pgfqpoint{4.991313in}{2.240480in}}%
\pgfpathlineto{\pgfqpoint{5.005640in}{2.245496in}}%
\pgfpathlineto{\pgfqpoint{4.997843in}{2.232574in}}%
\pgfpathlineto{\pgfqpoint{4.990041in}{2.219595in}}%
\pgfpathlineto{\pgfqpoint{4.982234in}{2.206563in}}%
\pgfpathlineto{\pgfqpoint{4.974422in}{2.193479in}}%
\pgfpathlineto{\pgfqpoint{4.960094in}{2.188699in}}%
\pgfpathlineto{\pgfqpoint{4.945779in}{2.184017in}}%
\pgfpathlineto{\pgfqpoint{4.931478in}{2.179435in}}%
\pgfpathlineto{\pgfqpoint{4.917189in}{2.174951in}}%
\pgfpathlineto{\pgfqpoint{4.925001in}{2.187793in}}%
\pgfpathlineto{\pgfqpoint{4.932809in}{2.200587in}}%
\pgfpathlineto{\pgfqpoint{4.940611in}{2.213333in}}%
\pgfpathlineto{\pgfqpoint{4.948408in}{2.226027in}}%
\pgfpathclose%
\pgfusepath{fill}%
\end{pgfscope}%
\begin{pgfscope}%
\pgfpathrectangle{\pgfqpoint{1.150000in}{0.150000in}}{\pgfqpoint{5.700000in}{5.700000in}}%
\pgfusepath{clip}%
\pgfsetbuttcap%
\pgfsetroundjoin%
\definecolor{currentfill}{rgb}{0.274952,0.037752,0.364543}%
\pgfsetfillcolor{currentfill}%
\pgfsetfillopacity{0.700000}%
\pgfsetlinewidth{0.000000pt}%
\definecolor{currentstroke}{rgb}{0.000000,0.000000,0.000000}%
\pgfsetstrokecolor{currentstroke}%
\pgfsetdash{}{0pt}%
\pgfpathmoveto{\pgfqpoint{4.212740in}{1.743852in}}%
\pgfpathlineto{\pgfqpoint{4.226722in}{1.742648in}}%
\pgfpathlineto{\pgfqpoint{4.240713in}{1.741543in}}%
\pgfpathlineto{\pgfqpoint{4.254712in}{1.740538in}}%
\pgfpathlineto{\pgfqpoint{4.268719in}{1.739633in}}%
\pgfpathlineto{\pgfqpoint{4.260724in}{1.729463in}}%
\pgfpathlineto{\pgfqpoint{4.252725in}{1.719377in}}%
\pgfpathlineto{\pgfqpoint{4.244720in}{1.709380in}}%
\pgfpathlineto{\pgfqpoint{4.236710in}{1.699475in}}%
\pgfpathlineto{\pgfqpoint{4.222694in}{1.700774in}}%
\pgfpathlineto{\pgfqpoint{4.208686in}{1.702173in}}%
\pgfpathlineto{\pgfqpoint{4.194686in}{1.703672in}}%
\pgfpathlineto{\pgfqpoint{4.180695in}{1.705271in}}%
\pgfpathlineto{\pgfqpoint{4.188714in}{1.714775in}}%
\pgfpathlineto{\pgfqpoint{4.196728in}{1.724376in}}%
\pgfpathlineto{\pgfqpoint{4.204737in}{1.734070in}}%
\pgfpathlineto{\pgfqpoint{4.212740in}{1.743852in}}%
\pgfpathclose%
\pgfusepath{fill}%
\end{pgfscope}%
\begin{pgfscope}%
\pgfpathrectangle{\pgfqpoint{1.150000in}{0.150000in}}{\pgfqpoint{5.700000in}{5.700000in}}%
\pgfusepath{clip}%
\pgfsetbuttcap%
\pgfsetroundjoin%
\definecolor{currentfill}{rgb}{0.140536,0.530132,0.555659}%
\pgfsetfillcolor{currentfill}%
\pgfsetfillopacity{0.700000}%
\pgfsetlinewidth{0.000000pt}%
\definecolor{currentstroke}{rgb}{0.000000,0.000000,0.000000}%
\pgfsetstrokecolor{currentstroke}%
\pgfsetdash{}{0pt}%
\pgfpathmoveto{\pgfqpoint{5.660242in}{2.880750in}}%
\pgfpathlineto{\pgfqpoint{5.674926in}{2.889591in}}%
\pgfpathlineto{\pgfqpoint{5.689627in}{2.898533in}}%
\pgfpathlineto{\pgfqpoint{5.704345in}{2.907577in}}%
\pgfpathlineto{\pgfqpoint{5.719079in}{2.916722in}}%
\pgfpathlineto{\pgfqpoint{5.711575in}{2.906912in}}%
\pgfpathlineto{\pgfqpoint{5.704062in}{2.896969in}}%
\pgfpathlineto{\pgfqpoint{5.696540in}{2.886892in}}%
\pgfpathlineto{\pgfqpoint{5.689009in}{2.876683in}}%
\pgfpathlineto{\pgfqpoint{5.674270in}{2.867583in}}%
\pgfpathlineto{\pgfqpoint{5.659549in}{2.858584in}}%
\pgfpathlineto{\pgfqpoint{5.644843in}{2.849687in}}%
\pgfpathlineto{\pgfqpoint{5.630155in}{2.840891in}}%
\pgfpathlineto{\pgfqpoint{5.637690in}{2.851047in}}%
\pgfpathlineto{\pgfqpoint{5.645216in}{2.861076in}}%
\pgfpathlineto{\pgfqpoint{5.652733in}{2.870977in}}%
\pgfpathlineto{\pgfqpoint{5.660242in}{2.880750in}}%
\pgfpathclose%
\pgfusepath{fill}%
\end{pgfscope}%
\begin{pgfscope}%
\pgfpathrectangle{\pgfqpoint{1.150000in}{0.150000in}}{\pgfqpoint{5.700000in}{5.700000in}}%
\pgfusepath{clip}%
\pgfsetbuttcap%
\pgfsetroundjoin%
\definecolor{currentfill}{rgb}{0.280868,0.160771,0.472899}%
\pgfsetfillcolor{currentfill}%
\pgfsetfillopacity{0.700000}%
\pgfsetlinewidth{0.000000pt}%
\definecolor{currentstroke}{rgb}{0.000000,0.000000,0.000000}%
\pgfsetstrokecolor{currentstroke}%
\pgfsetdash{}{0pt}%
\pgfpathmoveto{\pgfqpoint{4.652504in}{1.981689in}}%
\pgfpathlineto{\pgfqpoint{4.666653in}{1.984174in}}%
\pgfpathlineto{\pgfqpoint{4.680813in}{1.986758in}}%
\pgfpathlineto{\pgfqpoint{4.694985in}{1.989440in}}%
\pgfpathlineto{\pgfqpoint{4.709168in}{1.992221in}}%
\pgfpathlineto{\pgfqpoint{4.701291in}{1.979584in}}%
\pgfpathlineto{\pgfqpoint{4.693410in}{1.966944in}}%
\pgfpathlineto{\pgfqpoint{4.685523in}{1.954303in}}%
\pgfpathlineto{\pgfqpoint{4.677633in}{1.941664in}}%
\pgfpathlineto{\pgfqpoint{4.663448in}{1.939190in}}%
\pgfpathlineto{\pgfqpoint{4.649274in}{1.936814in}}%
\pgfpathlineto{\pgfqpoint{4.635111in}{1.934537in}}%
\pgfpathlineto{\pgfqpoint{4.620959in}{1.932358in}}%
\pgfpathlineto{\pgfqpoint{4.628852in}{1.944683in}}%
\pgfpathlineto{\pgfqpoint{4.636740in}{1.957015in}}%
\pgfpathlineto{\pgfqpoint{4.644624in}{1.969351in}}%
\pgfpathlineto{\pgfqpoint{4.652504in}{1.981689in}}%
\pgfpathclose%
\pgfusepath{fill}%
\end{pgfscope}%
\begin{pgfscope}%
\pgfpathrectangle{\pgfqpoint{1.150000in}{0.150000in}}{\pgfqpoint{5.700000in}{5.700000in}}%
\pgfusepath{clip}%
\pgfsetbuttcap%
\pgfsetroundjoin%
\definecolor{currentfill}{rgb}{0.282656,0.100196,0.422160}%
\pgfsetfillcolor{currentfill}%
\pgfsetfillopacity{0.700000}%
\pgfsetlinewidth{0.000000pt}%
\definecolor{currentstroke}{rgb}{0.000000,0.000000,0.000000}%
\pgfsetstrokecolor{currentstroke}%
\pgfsetdash{}{0pt}%
\pgfpathmoveto{\pgfqpoint{3.350498in}{1.848446in}}%
\pgfpathlineto{\pgfqpoint{3.364340in}{1.839488in}}%
\pgfpathlineto{\pgfqpoint{3.378184in}{1.830646in}}%
\pgfpathlineto{\pgfqpoint{3.392031in}{1.821919in}}%
\pgfpathlineto{\pgfqpoint{3.405879in}{1.813307in}}%
\pgfpathlineto{\pgfqpoint{3.397476in}{1.812291in}}%
\pgfpathlineto{\pgfqpoint{3.389061in}{1.811522in}}%
\pgfpathlineto{\pgfqpoint{3.380634in}{1.811006in}}%
\pgfpathlineto{\pgfqpoint{3.372195in}{1.810747in}}%
\pgfpathlineto{\pgfqpoint{3.358315in}{1.819868in}}%
\pgfpathlineto{\pgfqpoint{3.344437in}{1.829105in}}%
\pgfpathlineto{\pgfqpoint{3.330561in}{1.838457in}}%
\pgfpathlineto{\pgfqpoint{3.316688in}{1.847925in}}%
\pgfpathlineto{\pgfqpoint{3.325159in}{1.847666in}}%
\pgfpathlineto{\pgfqpoint{3.333618in}{1.847670in}}%
\pgfpathlineto{\pgfqpoint{3.342064in}{1.847932in}}%
\pgfpathlineto{\pgfqpoint{3.350498in}{1.848446in}}%
\pgfpathclose%
\pgfusepath{fill}%
\end{pgfscope}%
\begin{pgfscope}%
\pgfpathrectangle{\pgfqpoint{1.150000in}{0.150000in}}{\pgfqpoint{5.700000in}{5.700000in}}%
\pgfusepath{clip}%
\pgfsetbuttcap%
\pgfsetroundjoin%
\definecolor{currentfill}{rgb}{0.131172,0.555899,0.552459}%
\pgfsetfillcolor{currentfill}%
\pgfsetfillopacity{0.700000}%
\pgfsetlinewidth{0.000000pt}%
\definecolor{currentstroke}{rgb}{0.000000,0.000000,0.000000}%
\pgfsetstrokecolor{currentstroke}%
\pgfsetdash{}{0pt}%
\pgfpathmoveto{\pgfqpoint{2.258722in}{2.965642in}}%
\pgfpathlineto{\pgfqpoint{2.272823in}{2.944820in}}%
\pgfpathlineto{\pgfqpoint{2.286915in}{2.924197in}}%
\pgfpathlineto{\pgfqpoint{2.300998in}{2.903770in}}%
\pgfpathlineto{\pgfqpoint{2.315072in}{2.883539in}}%
\pgfpathlineto{\pgfqpoint{2.305755in}{2.893841in}}%
\pgfpathlineto{\pgfqpoint{2.296412in}{2.904520in}}%
\pgfpathlineto{\pgfqpoint{2.287043in}{2.915581in}}%
\pgfpathlineto{\pgfqpoint{2.277648in}{2.927031in}}%
\pgfpathlineto{\pgfqpoint{2.263512in}{2.947845in}}%
\pgfpathlineto{\pgfqpoint{2.249366in}{2.968856in}}%
\pgfpathlineto{\pgfqpoint{2.235212in}{2.990065in}}%
\pgfpathlineto{\pgfqpoint{2.221048in}{3.011473in}}%
\pgfpathlineto{\pgfqpoint{2.230506in}{2.999429in}}%
\pgfpathlineto{\pgfqpoint{2.239938in}{2.987780in}}%
\pgfpathlineto{\pgfqpoint{2.249343in}{2.976520in}}%
\pgfpathlineto{\pgfqpoint{2.258722in}{2.965642in}}%
\pgfpathclose%
\pgfusepath{fill}%
\end{pgfscope}%
\begin{pgfscope}%
\pgfpathrectangle{\pgfqpoint{1.150000in}{0.150000in}}{\pgfqpoint{5.700000in}{5.700000in}}%
\pgfusepath{clip}%
\pgfsetbuttcap%
\pgfsetroundjoin%
\definecolor{currentfill}{rgb}{0.273006,0.204520,0.501721}%
\pgfsetfillcolor{currentfill}%
\pgfsetfillopacity{0.700000}%
\pgfsetlinewidth{0.000000pt}%
\definecolor{currentstroke}{rgb}{0.000000,0.000000,0.000000}%
\pgfsetstrokecolor{currentstroke}%
\pgfsetdash{}{0pt}%
\pgfpathmoveto{\pgfqpoint{3.039443in}{2.062588in}}%
\pgfpathlineto{\pgfqpoint{3.053302in}{2.050670in}}%
\pgfpathlineto{\pgfqpoint{3.067160in}{2.038881in}}%
\pgfpathlineto{\pgfqpoint{3.081018in}{2.027220in}}%
\pgfpathlineto{\pgfqpoint{3.094876in}{2.015687in}}%
\pgfpathlineto{\pgfqpoint{3.086247in}{2.018319in}}%
\pgfpathlineto{\pgfqpoint{3.077602in}{2.021247in}}%
\pgfpathlineto{\pgfqpoint{3.068941in}{2.024475in}}%
\pgfpathlineto{\pgfqpoint{3.060264in}{2.028009in}}%
\pgfpathlineto{\pgfqpoint{3.046366in}{2.040081in}}%
\pgfpathlineto{\pgfqpoint{3.032467in}{2.052280in}}%
\pgfpathlineto{\pgfqpoint{3.018568in}{2.064608in}}%
\pgfpathlineto{\pgfqpoint{3.004669in}{2.077065in}}%
\pgfpathlineto{\pgfqpoint{3.013387in}{2.072984in}}%
\pgfpathlineto{\pgfqpoint{3.022089in}{2.069214in}}%
\pgfpathlineto{\pgfqpoint{3.030774in}{2.065751in}}%
\pgfpathlineto{\pgfqpoint{3.039443in}{2.062588in}}%
\pgfpathclose%
\pgfusepath{fill}%
\end{pgfscope}%
\begin{pgfscope}%
\pgfpathrectangle{\pgfqpoint{1.150000in}{0.150000in}}{\pgfqpoint{5.700000in}{5.700000in}}%
\pgfusepath{clip}%
\pgfsetbuttcap%
\pgfsetroundjoin%
\definecolor{currentfill}{rgb}{0.168126,0.459988,0.558082}%
\pgfsetfillcolor{currentfill}%
\pgfsetfillopacity{0.700000}%
\pgfsetlinewidth{0.000000pt}%
\definecolor{currentstroke}{rgb}{0.000000,0.000000,0.000000}%
\pgfsetstrokecolor{currentstroke}%
\pgfsetdash{}{0pt}%
\pgfpathmoveto{\pgfqpoint{5.452595in}{2.689535in}}%
\pgfpathlineto{\pgfqpoint{5.467162in}{2.697374in}}%
\pgfpathlineto{\pgfqpoint{5.481744in}{2.705314in}}%
\pgfpathlineto{\pgfqpoint{5.496342in}{2.713355in}}%
\pgfpathlineto{\pgfqpoint{5.510956in}{2.721496in}}%
\pgfpathlineto{\pgfqpoint{5.503344in}{2.710302in}}%
\pgfpathlineto{\pgfqpoint{5.495725in}{2.698989in}}%
\pgfpathlineto{\pgfqpoint{5.488097in}{2.687559in}}%
\pgfpathlineto{\pgfqpoint{5.480462in}{2.676011in}}%
\pgfpathlineto{\pgfqpoint{5.465846in}{2.667974in}}%
\pgfpathlineto{\pgfqpoint{5.451246in}{2.660037in}}%
\pgfpathlineto{\pgfqpoint{5.436662in}{2.652201in}}%
\pgfpathlineto{\pgfqpoint{5.422093in}{2.644465in}}%
\pgfpathlineto{\pgfqpoint{5.429730in}{2.655901in}}%
\pgfpathlineto{\pgfqpoint{5.437359in}{2.667225in}}%
\pgfpathlineto{\pgfqpoint{5.444981in}{2.678437in}}%
\pgfpathlineto{\pgfqpoint{5.452595in}{2.689535in}}%
\pgfpathclose%
\pgfusepath{fill}%
\end{pgfscope}%
\begin{pgfscope}%
\pgfpathrectangle{\pgfqpoint{1.150000in}{0.150000in}}{\pgfqpoint{5.700000in}{5.700000in}}%
\pgfusepath{clip}%
\pgfsetbuttcap%
\pgfsetroundjoin%
\definecolor{currentfill}{rgb}{0.271305,0.019942,0.347269}%
\pgfsetfillcolor{currentfill}%
\pgfsetfillopacity{0.700000}%
\pgfsetlinewidth{0.000000pt}%
\definecolor{currentstroke}{rgb}{0.000000,0.000000,0.000000}%
\pgfsetstrokecolor{currentstroke}%
\pgfsetdash{}{0pt}%
\pgfpathmoveto{\pgfqpoint{4.124807in}{1.712669in}}%
\pgfpathlineto{\pgfqpoint{4.138767in}{1.710668in}}%
\pgfpathlineto{\pgfqpoint{4.152735in}{1.708769in}}%
\pgfpathlineto{\pgfqpoint{4.166711in}{1.706970in}}%
\pgfpathlineto{\pgfqpoint{4.180695in}{1.705271in}}%
\pgfpathlineto{\pgfqpoint{4.172670in}{1.695867in}}%
\pgfpathlineto{\pgfqpoint{4.164639in}{1.686568in}}%
\pgfpathlineto{\pgfqpoint{4.156603in}{1.677377in}}%
\pgfpathlineto{\pgfqpoint{4.148561in}{1.668298in}}%
\pgfpathlineto{\pgfqpoint{4.134567in}{1.670409in}}%
\pgfpathlineto{\pgfqpoint{4.120580in}{1.672620in}}%
\pgfpathlineto{\pgfqpoint{4.106601in}{1.674932in}}%
\pgfpathlineto{\pgfqpoint{4.092629in}{1.677343in}}%
\pgfpathlineto{\pgfqpoint{4.100682in}{1.686003in}}%
\pgfpathlineto{\pgfqpoint{4.108730in}{1.694780in}}%
\pgfpathlineto{\pgfqpoint{4.116771in}{1.703670in}}%
\pgfpathlineto{\pgfqpoint{4.124807in}{1.712669in}}%
\pgfpathclose%
\pgfusepath{fill}%
\end{pgfscope}%
\begin{pgfscope}%
\pgfpathrectangle{\pgfqpoint{1.150000in}{0.150000in}}{\pgfqpoint{5.700000in}{5.700000in}}%
\pgfusepath{clip}%
\pgfsetbuttcap%
\pgfsetroundjoin%
\definecolor{currentfill}{rgb}{0.269944,0.014625,0.341379}%
\pgfsetfillcolor{currentfill}%
\pgfsetfillopacity{0.700000}%
\pgfsetlinewidth{0.000000pt}%
\definecolor{currentstroke}{rgb}{0.000000,0.000000,0.000000}%
\pgfsetstrokecolor{currentstroke}%
\pgfsetdash{}{0pt}%
\pgfpathmoveto{\pgfqpoint{3.749335in}{1.700026in}}%
\pgfpathlineto{\pgfqpoint{3.763214in}{1.694677in}}%
\pgfpathlineto{\pgfqpoint{3.777098in}{1.689433in}}%
\pgfpathlineto{\pgfqpoint{3.790986in}{1.684295in}}%
\pgfpathlineto{\pgfqpoint{3.804881in}{1.679262in}}%
\pgfpathlineto{\pgfqpoint{3.796703in}{1.673684in}}%
\pgfpathlineto{\pgfqpoint{3.788518in}{1.668284in}}%
\pgfpathlineto{\pgfqpoint{3.780324in}{1.663068in}}%
\pgfpathlineto{\pgfqpoint{3.772122in}{1.658041in}}%
\pgfpathlineto{\pgfqpoint{3.758208in}{1.663541in}}%
\pgfpathlineto{\pgfqpoint{3.744298in}{1.669146in}}%
\pgfpathlineto{\pgfqpoint{3.730394in}{1.674856in}}%
\pgfpathlineto{\pgfqpoint{3.716495in}{1.680672in}}%
\pgfpathlineto{\pgfqpoint{3.724718in}{1.685226in}}%
\pgfpathlineto{\pgfqpoint{3.732932in}{1.689973in}}%
\pgfpathlineto{\pgfqpoint{3.741138in}{1.694907in}}%
\pgfpathlineto{\pgfqpoint{3.749335in}{1.700026in}}%
\pgfpathclose%
\pgfusepath{fill}%
\end{pgfscope}%
\begin{pgfscope}%
\pgfpathrectangle{\pgfqpoint{1.150000in}{0.150000in}}{\pgfqpoint{5.700000in}{5.700000in}}%
\pgfusepath{clip}%
\pgfsetbuttcap%
\pgfsetroundjoin%
\definecolor{currentfill}{rgb}{0.199430,0.387607,0.554642}%
\pgfsetfillcolor{currentfill}%
\pgfsetfillopacity{0.700000}%
\pgfsetlinewidth{0.000000pt}%
\definecolor{currentstroke}{rgb}{0.000000,0.000000,0.000000}%
\pgfsetstrokecolor{currentstroke}%
\pgfsetdash{}{0pt}%
\pgfpathmoveto{\pgfqpoint{5.244717in}{2.492903in}}%
\pgfpathlineto{\pgfqpoint{5.259168in}{2.499571in}}%
\pgfpathlineto{\pgfqpoint{5.273633in}{2.506339in}}%
\pgfpathlineto{\pgfqpoint{5.288113in}{2.513207in}}%
\pgfpathlineto{\pgfqpoint{5.302609in}{2.520176in}}%
\pgfpathlineto{\pgfqpoint{5.294907in}{2.507929in}}%
\pgfpathlineto{\pgfqpoint{5.287200in}{2.495585in}}%
\pgfpathlineto{\pgfqpoint{5.279485in}{2.483145in}}%
\pgfpathlineto{\pgfqpoint{5.271764in}{2.470612in}}%
\pgfpathlineto{\pgfqpoint{5.257269in}{2.463805in}}%
\pgfpathlineto{\pgfqpoint{5.242788in}{2.457098in}}%
\pgfpathlineto{\pgfqpoint{5.228322in}{2.450491in}}%
\pgfpathlineto{\pgfqpoint{5.213871in}{2.443984in}}%
\pgfpathlineto{\pgfqpoint{5.221592in}{2.456349in}}%
\pgfpathlineto{\pgfqpoint{5.229306in}{2.468625in}}%
\pgfpathlineto{\pgfqpoint{5.237015in}{2.480810in}}%
\pgfpathlineto{\pgfqpoint{5.244717in}{2.492903in}}%
\pgfpathclose%
\pgfusepath{fill}%
\end{pgfscope}%
\begin{pgfscope}%
\pgfpathrectangle{\pgfqpoint{1.150000in}{0.150000in}}{\pgfqpoint{5.700000in}{5.700000in}}%
\pgfusepath{clip}%
\pgfsetbuttcap%
\pgfsetroundjoin%
\definecolor{currentfill}{rgb}{0.275191,0.194905,0.496005}%
\pgfsetfillcolor{currentfill}%
\pgfsetfillopacity{0.700000}%
\pgfsetlinewidth{0.000000pt}%
\definecolor{currentstroke}{rgb}{0.000000,0.000000,0.000000}%
\pgfsetstrokecolor{currentstroke}%
\pgfsetdash{}{0pt}%
\pgfpathmoveto{\pgfqpoint{4.740631in}{2.042682in}}%
\pgfpathlineto{\pgfqpoint{4.754824in}{2.045851in}}%
\pgfpathlineto{\pgfqpoint{4.769029in}{2.049118in}}%
\pgfpathlineto{\pgfqpoint{4.783245in}{2.052484in}}%
\pgfpathlineto{\pgfqpoint{4.797474in}{2.055949in}}%
\pgfpathlineto{\pgfqpoint{4.789616in}{2.043070in}}%
\pgfpathlineto{\pgfqpoint{4.781754in}{2.030172in}}%
\pgfpathlineto{\pgfqpoint{4.773888in}{2.017258in}}%
\pgfpathlineto{\pgfqpoint{4.766017in}{2.004331in}}%
\pgfpathlineto{\pgfqpoint{4.751787in}{2.001156in}}%
\pgfpathlineto{\pgfqpoint{4.737569in}{1.998079in}}%
\pgfpathlineto{\pgfqpoint{4.723363in}{1.995101in}}%
\pgfpathlineto{\pgfqpoint{4.709168in}{1.992221in}}%
\pgfpathlineto{\pgfqpoint{4.717041in}{2.004852in}}%
\pgfpathlineto{\pgfqpoint{4.724909in}{2.017475in}}%
\pgfpathlineto{\pgfqpoint{4.732772in}{2.030085in}}%
\pgfpathlineto{\pgfqpoint{4.740631in}{2.042682in}}%
\pgfpathclose%
\pgfusepath{fill}%
\end{pgfscope}%
\begin{pgfscope}%
\pgfpathrectangle{\pgfqpoint{1.150000in}{0.150000in}}{\pgfqpoint{5.700000in}{5.700000in}}%
\pgfusepath{clip}%
\pgfsetbuttcap%
\pgfsetroundjoin%
\definecolor{currentfill}{rgb}{0.268510,0.009605,0.335427}%
\pgfsetfillcolor{currentfill}%
\pgfsetfillopacity{0.700000}%
\pgfsetlinewidth{0.000000pt}%
\definecolor{currentstroke}{rgb}{0.000000,0.000000,0.000000}%
\pgfsetstrokecolor{currentstroke}%
\pgfsetdash{}{0pt}%
\pgfpathmoveto{\pgfqpoint{3.893073in}{1.685975in}}%
\pgfpathlineto{\pgfqpoint{3.906977in}{1.681910in}}%
\pgfpathlineto{\pgfqpoint{3.920889in}{1.677949in}}%
\pgfpathlineto{\pgfqpoint{3.934806in}{1.674090in}}%
\pgfpathlineto{\pgfqpoint{3.948729in}{1.670334in}}%
\pgfpathlineto{\pgfqpoint{3.940616in}{1.663204in}}%
\pgfpathlineto{\pgfqpoint{3.932497in}{1.656225in}}%
\pgfpathlineto{\pgfqpoint{3.924370in}{1.649402in}}%
\pgfpathlineto{\pgfqpoint{3.916236in}{1.642740in}}%
\pgfpathlineto{\pgfqpoint{3.902296in}{1.646944in}}%
\pgfpathlineto{\pgfqpoint{3.888362in}{1.651250in}}%
\pgfpathlineto{\pgfqpoint{3.874434in}{1.655660in}}%
\pgfpathlineto{\pgfqpoint{3.860512in}{1.660173in}}%
\pgfpathlineto{\pgfqpoint{3.868663in}{1.666380in}}%
\pgfpathlineto{\pgfqpoint{3.876807in}{1.672753in}}%
\pgfpathlineto{\pgfqpoint{3.884943in}{1.679286in}}%
\pgfpathlineto{\pgfqpoint{3.893073in}{1.685975in}}%
\pgfpathclose%
\pgfusepath{fill}%
\end{pgfscope}%
\begin{pgfscope}%
\pgfpathrectangle{\pgfqpoint{1.150000in}{0.150000in}}{\pgfqpoint{5.700000in}{5.700000in}}%
\pgfusepath{clip}%
\pgfsetbuttcap%
\pgfsetroundjoin%
\definecolor{currentfill}{rgb}{0.235526,0.309527,0.542944}%
\pgfsetfillcolor{currentfill}%
\pgfsetfillopacity{0.700000}%
\pgfsetlinewidth{0.000000pt}%
\definecolor{currentstroke}{rgb}{0.000000,0.000000,0.000000}%
\pgfsetstrokecolor{currentstroke}%
\pgfsetdash{}{0pt}%
\pgfpathmoveto{\pgfqpoint{5.036773in}{2.296587in}}%
\pgfpathlineto{\pgfqpoint{5.051114in}{2.301919in}}%
\pgfpathlineto{\pgfqpoint{5.065468in}{2.307351in}}%
\pgfpathlineto{\pgfqpoint{5.079836in}{2.312881in}}%
\pgfpathlineto{\pgfqpoint{5.094217in}{2.318512in}}%
\pgfpathlineto{\pgfqpoint{5.086442in}{2.305622in}}%
\pgfpathlineto{\pgfqpoint{5.078662in}{2.292665in}}%
\pgfpathlineto{\pgfqpoint{5.070876in}{2.279640in}}%
\pgfpathlineto{\pgfqpoint{5.063084in}{2.266551in}}%
\pgfpathlineto{\pgfqpoint{5.048703in}{2.261139in}}%
\pgfpathlineto{\pgfqpoint{5.034335in}{2.255825in}}%
\pgfpathlineto{\pgfqpoint{5.019981in}{2.250611in}}%
\pgfpathlineto{\pgfqpoint{5.005640in}{2.245496in}}%
\pgfpathlineto{\pgfqpoint{5.013432in}{2.258361in}}%
\pgfpathlineto{\pgfqpoint{5.021218in}{2.271165in}}%
\pgfpathlineto{\pgfqpoint{5.028998in}{2.283908in}}%
\pgfpathlineto{\pgfqpoint{5.036773in}{2.296587in}}%
\pgfpathclose%
\pgfusepath{fill}%
\end{pgfscope}%
\begin{pgfscope}%
\pgfpathrectangle{\pgfqpoint{1.150000in}{0.150000in}}{\pgfqpoint{5.700000in}{5.700000in}}%
\pgfusepath{clip}%
\pgfsetbuttcap%
\pgfsetroundjoin%
\definecolor{currentfill}{rgb}{0.277134,0.185228,0.489898}%
\pgfsetfillcolor{currentfill}%
\pgfsetfillopacity{0.700000}%
\pgfsetlinewidth{0.000000pt}%
\definecolor{currentstroke}{rgb}{0.000000,0.000000,0.000000}%
\pgfsetstrokecolor{currentstroke}%
\pgfsetdash{}{0pt}%
\pgfpathmoveto{\pgfqpoint{3.094876in}{2.015687in}}%
\pgfpathlineto{\pgfqpoint{3.108735in}{2.004280in}}%
\pgfpathlineto{\pgfqpoint{3.122593in}{1.992999in}}%
\pgfpathlineto{\pgfqpoint{3.136452in}{1.981843in}}%
\pgfpathlineto{\pgfqpoint{3.150311in}{1.970812in}}%
\pgfpathlineto{\pgfqpoint{3.141719in}{1.972916in}}%
\pgfpathlineto{\pgfqpoint{3.133113in}{1.975310in}}%
\pgfpathlineto{\pgfqpoint{3.124491in}{1.977999in}}%
\pgfpathlineto{\pgfqpoint{3.115854in}{1.980988in}}%
\pgfpathlineto{\pgfqpoint{3.101957in}{1.992555in}}%
\pgfpathlineto{\pgfqpoint{3.088059in}{2.004247in}}%
\pgfpathlineto{\pgfqpoint{3.074162in}{2.016065in}}%
\pgfpathlineto{\pgfqpoint{3.060264in}{2.028009in}}%
\pgfpathlineto{\pgfqpoint{3.068941in}{2.024475in}}%
\pgfpathlineto{\pgfqpoint{3.077602in}{2.021247in}}%
\pgfpathlineto{\pgfqpoint{3.086247in}{2.018319in}}%
\pgfpathlineto{\pgfqpoint{3.094876in}{2.015687in}}%
\pgfpathclose%
\pgfusepath{fill}%
\end{pgfscope}%
\begin{pgfscope}%
\pgfpathrectangle{\pgfqpoint{1.150000in}{0.150000in}}{\pgfqpoint{5.700000in}{5.700000in}}%
\pgfusepath{clip}%
\pgfsetbuttcap%
\pgfsetroundjoin%
\definecolor{currentfill}{rgb}{0.121831,0.589055,0.545623}%
\pgfsetfillcolor{currentfill}%
\pgfsetfillopacity{0.700000}%
\pgfsetlinewidth{0.000000pt}%
\definecolor{currentstroke}{rgb}{0.000000,0.000000,0.000000}%
\pgfsetstrokecolor{currentstroke}%
\pgfsetdash{}{0pt}%
\pgfpathmoveto{\pgfqpoint{2.202223in}{3.050954in}}%
\pgfpathlineto{\pgfqpoint{2.216363in}{3.029318in}}%
\pgfpathlineto{\pgfqpoint{2.230492in}{3.007889in}}%
\pgfpathlineto{\pgfqpoint{2.244612in}{2.986664in}}%
\pgfpathlineto{\pgfqpoint{2.258722in}{2.965642in}}%
\pgfpathlineto{\pgfqpoint{2.249343in}{2.976520in}}%
\pgfpathlineto{\pgfqpoint{2.239938in}{2.987780in}}%
\pgfpathlineto{\pgfqpoint{2.230506in}{2.999429in}}%
\pgfpathlineto{\pgfqpoint{2.221048in}{3.011473in}}%
\pgfpathlineto{\pgfqpoint{2.206874in}{3.033084in}}%
\pgfpathlineto{\pgfqpoint{2.192690in}{3.054898in}}%
\pgfpathlineto{\pgfqpoint{2.178496in}{3.076918in}}%
\pgfpathlineto{\pgfqpoint{2.164292in}{3.099146in}}%
\pgfpathlineto{\pgfqpoint{2.173816in}{3.086502in}}%
\pgfpathlineto{\pgfqpoint{2.183313in}{3.074259in}}%
\pgfpathlineto{\pgfqpoint{2.192781in}{3.062412in}}%
\pgfpathlineto{\pgfqpoint{2.202223in}{3.050954in}}%
\pgfpathclose%
\pgfusepath{fill}%
\end{pgfscope}%
\begin{pgfscope}%
\pgfpathrectangle{\pgfqpoint{1.150000in}{0.150000in}}{\pgfqpoint{5.700000in}{5.700000in}}%
\pgfusepath{clip}%
\pgfsetbuttcap%
\pgfsetroundjoin%
\definecolor{currentfill}{rgb}{0.129933,0.559582,0.551864}%
\pgfsetfillcolor{currentfill}%
\pgfsetfillopacity{0.700000}%
\pgfsetlinewidth{0.000000pt}%
\definecolor{currentstroke}{rgb}{0.000000,0.000000,0.000000}%
\pgfsetstrokecolor{currentstroke}%
\pgfsetdash{}{0pt}%
\pgfpathmoveto{\pgfqpoint{5.749004in}{2.954632in}}%
\pgfpathlineto{\pgfqpoint{5.763750in}{2.963904in}}%
\pgfpathlineto{\pgfqpoint{5.778512in}{2.973277in}}%
\pgfpathlineto{\pgfqpoint{5.793292in}{2.982752in}}%
\pgfpathlineto{\pgfqpoint{5.808089in}{2.992329in}}%
\pgfpathlineto{\pgfqpoint{5.800627in}{2.983034in}}%
\pgfpathlineto{\pgfqpoint{5.793156in}{2.973601in}}%
\pgfpathlineto{\pgfqpoint{5.785675in}{2.964030in}}%
\pgfpathlineto{\pgfqpoint{5.778185in}{2.954320in}}%
\pgfpathlineto{\pgfqpoint{5.763383in}{2.944768in}}%
\pgfpathlineto{\pgfqpoint{5.748598in}{2.935318in}}%
\pgfpathlineto{\pgfqpoint{5.733830in}{2.925969in}}%
\pgfpathlineto{\pgfqpoint{5.719079in}{2.916722in}}%
\pgfpathlineto{\pgfqpoint{5.726574in}{2.926399in}}%
\pgfpathlineto{\pgfqpoint{5.734060in}{2.935943in}}%
\pgfpathlineto{\pgfqpoint{5.741536in}{2.945354in}}%
\pgfpathlineto{\pgfqpoint{5.749004in}{2.954632in}}%
\pgfpathclose%
\pgfusepath{fill}%
\end{pgfscope}%
\begin{pgfscope}%
\pgfpathrectangle{\pgfqpoint{1.150000in}{0.150000in}}{\pgfqpoint{5.700000in}{5.700000in}}%
\pgfusepath{clip}%
\pgfsetbuttcap%
\pgfsetroundjoin%
\definecolor{currentfill}{rgb}{0.274952,0.037752,0.364543}%
\pgfsetfillcolor{currentfill}%
\pgfsetfillopacity{0.700000}%
\pgfsetlinewidth{0.000000pt}%
\definecolor{currentstroke}{rgb}{0.000000,0.000000,0.000000}%
\pgfsetstrokecolor{currentstroke}%
\pgfsetdash{}{0pt}%
\pgfpathmoveto{\pgfqpoint{3.605462in}{1.731046in}}%
\pgfpathlineto{\pgfqpoint{3.619326in}{1.724372in}}%
\pgfpathlineto{\pgfqpoint{3.633194in}{1.717807in}}%
\pgfpathlineto{\pgfqpoint{3.647067in}{1.711350in}}%
\pgfpathlineto{\pgfqpoint{3.660943in}{1.705000in}}%
\pgfpathlineto{\pgfqpoint{3.652689in}{1.701124in}}%
\pgfpathlineto{\pgfqpoint{3.644425in}{1.697454in}}%
\pgfpathlineto{\pgfqpoint{3.636151in}{1.693997in}}%
\pgfpathlineto{\pgfqpoint{3.627868in}{1.690758in}}%
\pgfpathlineto{\pgfqpoint{3.613968in}{1.697594in}}%
\pgfpathlineto{\pgfqpoint{3.600071in}{1.704538in}}%
\pgfpathlineto{\pgfqpoint{3.586178in}{1.711591in}}%
\pgfpathlineto{\pgfqpoint{3.572289in}{1.718752in}}%
\pgfpathlineto{\pgfqpoint{3.580597in}{1.721497in}}%
\pgfpathlineto{\pgfqpoint{3.588895in}{1.724464in}}%
\pgfpathlineto{\pgfqpoint{3.597184in}{1.727649in}}%
\pgfpathlineto{\pgfqpoint{3.605462in}{1.731046in}}%
\pgfpathclose%
\pgfusepath{fill}%
\end{pgfscope}%
\begin{pgfscope}%
\pgfpathrectangle{\pgfqpoint{1.150000in}{0.150000in}}{\pgfqpoint{5.700000in}{5.700000in}}%
\pgfusepath{clip}%
\pgfsetbuttcap%
\pgfsetroundjoin%
\definecolor{currentfill}{rgb}{0.281446,0.084320,0.407414}%
\pgfsetfillcolor{currentfill}%
\pgfsetfillopacity{0.700000}%
\pgfsetlinewidth{0.000000pt}%
\definecolor{currentstroke}{rgb}{0.000000,0.000000,0.000000}%
\pgfsetstrokecolor{currentstroke}%
\pgfsetdash{}{0pt}%
\pgfpathmoveto{\pgfqpoint{3.405879in}{1.813307in}}%
\pgfpathlineto{\pgfqpoint{3.419731in}{1.804810in}}%
\pgfpathlineto{\pgfqpoint{3.433585in}{1.796426in}}%
\pgfpathlineto{\pgfqpoint{3.447441in}{1.788156in}}%
\pgfpathlineto{\pgfqpoint{3.461301in}{1.779999in}}%
\pgfpathlineto{\pgfqpoint{3.452927in}{1.778481in}}%
\pgfpathlineto{\pgfqpoint{3.444542in}{1.777206in}}%
\pgfpathlineto{\pgfqpoint{3.436145in}{1.776178in}}%
\pgfpathlineto{\pgfqpoint{3.427736in}{1.775402in}}%
\pgfpathlineto{\pgfqpoint{3.413847in}{1.784068in}}%
\pgfpathlineto{\pgfqpoint{3.399961in}{1.792847in}}%
\pgfpathlineto{\pgfqpoint{3.386076in}{1.801740in}}%
\pgfpathlineto{\pgfqpoint{3.372195in}{1.810747in}}%
\pgfpathlineto{\pgfqpoint{3.380634in}{1.811006in}}%
\pgfpathlineto{\pgfqpoint{3.389061in}{1.811522in}}%
\pgfpathlineto{\pgfqpoint{3.397476in}{1.812291in}}%
\pgfpathlineto{\pgfqpoint{3.405879in}{1.813307in}}%
\pgfpathclose%
\pgfusepath{fill}%
\end{pgfscope}%
\begin{pgfscope}%
\pgfpathrectangle{\pgfqpoint{1.150000in}{0.150000in}}{\pgfqpoint{5.700000in}{5.700000in}}%
\pgfusepath{clip}%
\pgfsetbuttcap%
\pgfsetroundjoin%
\definecolor{currentfill}{rgb}{0.266580,0.228262,0.514349}%
\pgfsetfillcolor{currentfill}%
\pgfsetfillopacity{0.700000}%
\pgfsetlinewidth{0.000000pt}%
\definecolor{currentstroke}{rgb}{0.000000,0.000000,0.000000}%
\pgfsetstrokecolor{currentstroke}%
\pgfsetdash{}{0pt}%
\pgfpathmoveto{\pgfqpoint{4.828855in}{2.107227in}}%
\pgfpathlineto{\pgfqpoint{4.843095in}{2.111061in}}%
\pgfpathlineto{\pgfqpoint{4.857347in}{2.114995in}}%
\pgfpathlineto{\pgfqpoint{4.871611in}{2.119027in}}%
\pgfpathlineto{\pgfqpoint{4.885888in}{2.123159in}}%
\pgfpathlineto{\pgfqpoint{4.878050in}{2.110115in}}%
\pgfpathlineto{\pgfqpoint{4.870208in}{2.097038in}}%
\pgfpathlineto{\pgfqpoint{4.862360in}{2.083930in}}%
\pgfpathlineto{\pgfqpoint{4.854508in}{2.070794in}}%
\pgfpathlineto{\pgfqpoint{4.840231in}{2.066935in}}%
\pgfpathlineto{\pgfqpoint{4.825967in}{2.063174in}}%
\pgfpathlineto{\pgfqpoint{4.811714in}{2.059512in}}%
\pgfpathlineto{\pgfqpoint{4.797474in}{2.055949in}}%
\pgfpathlineto{\pgfqpoint{4.805326in}{2.068807in}}%
\pgfpathlineto{\pgfqpoint{4.813174in}{2.081640in}}%
\pgfpathlineto{\pgfqpoint{4.821017in}{2.094448in}}%
\pgfpathlineto{\pgfqpoint{4.828855in}{2.107227in}}%
\pgfpathclose%
\pgfusepath{fill}%
\end{pgfscope}%
\begin{pgfscope}%
\pgfpathrectangle{\pgfqpoint{1.150000in}{0.150000in}}{\pgfqpoint{5.700000in}{5.700000in}}%
\pgfusepath{clip}%
\pgfsetbuttcap%
\pgfsetroundjoin%
\definecolor{currentfill}{rgb}{0.269944,0.014625,0.341379}%
\pgfsetfillcolor{currentfill}%
\pgfsetfillopacity{0.700000}%
\pgfsetlinewidth{0.000000pt}%
\definecolor{currentstroke}{rgb}{0.000000,0.000000,0.000000}%
\pgfsetstrokecolor{currentstroke}%
\pgfsetdash{}{0pt}%
\pgfpathmoveto{\pgfqpoint{4.036815in}{1.688001in}}%
\pgfpathlineto{\pgfqpoint{4.050758in}{1.685185in}}%
\pgfpathlineto{\pgfqpoint{4.064708in}{1.682470in}}%
\pgfpathlineto{\pgfqpoint{4.078665in}{1.679856in}}%
\pgfpathlineto{\pgfqpoint{4.092629in}{1.677343in}}%
\pgfpathlineto{\pgfqpoint{4.084570in}{1.668805in}}%
\pgfpathlineto{\pgfqpoint{4.076505in}{1.660391in}}%
\pgfpathlineto{\pgfqpoint{4.068433in}{1.652107in}}%
\pgfpathlineto{\pgfqpoint{4.060356in}{1.643956in}}%
\pgfpathlineto{\pgfqpoint{4.046378in}{1.646899in}}%
\pgfpathlineto{\pgfqpoint{4.032408in}{1.649942in}}%
\pgfpathlineto{\pgfqpoint{4.018445in}{1.653087in}}%
\pgfpathlineto{\pgfqpoint{4.004488in}{1.656333in}}%
\pgfpathlineto{\pgfqpoint{4.012580in}{1.664047in}}%
\pgfpathlineto{\pgfqpoint{4.020665in}{1.671899in}}%
\pgfpathlineto{\pgfqpoint{4.028743in}{1.679885in}}%
\pgfpathlineto{\pgfqpoint{4.036815in}{1.688001in}}%
\pgfpathclose%
\pgfusepath{fill}%
\end{pgfscope}%
\begin{pgfscope}%
\pgfpathrectangle{\pgfqpoint{1.150000in}{0.150000in}}{\pgfqpoint{5.700000in}{5.700000in}}%
\pgfusepath{clip}%
\pgfsetbuttcap%
\pgfsetroundjoin%
\definecolor{currentfill}{rgb}{0.154815,0.493313,0.557840}%
\pgfsetfillcolor{currentfill}%
\pgfsetfillopacity{0.700000}%
\pgfsetlinewidth{0.000000pt}%
\definecolor{currentstroke}{rgb}{0.000000,0.000000,0.000000}%
\pgfsetstrokecolor{currentstroke}%
\pgfsetdash{}{0pt}%
\pgfpathmoveto{\pgfqpoint{5.541325in}{2.765077in}}%
\pgfpathlineto{\pgfqpoint{5.555952in}{2.773404in}}%
\pgfpathlineto{\pgfqpoint{5.570595in}{2.781832in}}%
\pgfpathlineto{\pgfqpoint{5.585254in}{2.790361in}}%
\pgfpathlineto{\pgfqpoint{5.599930in}{2.798991in}}%
\pgfpathlineto{\pgfqpoint{5.592353in}{2.788199in}}%
\pgfpathlineto{\pgfqpoint{5.584768in}{2.777282in}}%
\pgfpathlineto{\pgfqpoint{5.577174in}{2.766239in}}%
\pgfpathlineto{\pgfqpoint{5.569572in}{2.755072in}}%
\pgfpathlineto{\pgfqpoint{5.554894in}{2.746527in}}%
\pgfpathlineto{\pgfqpoint{5.540232in}{2.738082in}}%
\pgfpathlineto{\pgfqpoint{5.525586in}{2.729739in}}%
\pgfpathlineto{\pgfqpoint{5.510956in}{2.721496in}}%
\pgfpathlineto{\pgfqpoint{5.518560in}{2.732572in}}%
\pgfpathlineto{\pgfqpoint{5.526156in}{2.743527in}}%
\pgfpathlineto{\pgfqpoint{5.533744in}{2.754362in}}%
\pgfpathlineto{\pgfqpoint{5.541325in}{2.765077in}}%
\pgfpathclose%
\pgfusepath{fill}%
\end{pgfscope}%
\begin{pgfscope}%
\pgfpathrectangle{\pgfqpoint{1.150000in}{0.150000in}}{\pgfqpoint{5.700000in}{5.700000in}}%
\pgfusepath{clip}%
\pgfsetbuttcap%
\pgfsetroundjoin%
\definecolor{currentfill}{rgb}{0.121831,0.589055,0.545623}%
\pgfsetfillcolor{currentfill}%
\pgfsetfillopacity{0.700000}%
\pgfsetlinewidth{0.000000pt}%
\definecolor{currentstroke}{rgb}{0.000000,0.000000,0.000000}%
\pgfsetstrokecolor{currentstroke}%
\pgfsetdash{}{0pt}%
\pgfpathmoveto{\pgfqpoint{5.837840in}{3.028133in}}%
\pgfpathlineto{\pgfqpoint{5.852648in}{3.037817in}}%
\pgfpathlineto{\pgfqpoint{5.867473in}{3.047602in}}%
\pgfpathlineto{\pgfqpoint{5.882315in}{3.057490in}}%
\pgfpathlineto{\pgfqpoint{5.874897in}{3.048747in}}%
\pgfpathlineto{\pgfqpoint{5.867469in}{3.039864in}}%
\pgfpathlineto{\pgfqpoint{5.860031in}{3.030839in}}%
\pgfpathlineto{\pgfqpoint{5.852584in}{3.021673in}}%
\pgfpathlineto{\pgfqpoint{5.837735in}{3.011789in}}%
\pgfpathlineto{\pgfqpoint{5.822903in}{3.002008in}}%
\pgfpathlineto{\pgfqpoint{5.808089in}{2.992329in}}%
\pgfpathlineto{\pgfqpoint{5.815541in}{3.001486in}}%
\pgfpathlineto{\pgfqpoint{5.822984in}{3.010506in}}%
\pgfpathlineto{\pgfqpoint{5.830417in}{3.019388in}}%
\pgfpathlineto{\pgfqpoint{5.837840in}{3.028133in}}%
\pgfpathclose%
\pgfusepath{fill}%
\end{pgfscope}%
\begin{pgfscope}%
\pgfpathrectangle{\pgfqpoint{1.150000in}{0.150000in}}{\pgfqpoint{5.700000in}{5.700000in}}%
\pgfusepath{clip}%
\pgfsetbuttcap%
\pgfsetroundjoin%
\definecolor{currentfill}{rgb}{0.280255,0.165693,0.476498}%
\pgfsetfillcolor{currentfill}%
\pgfsetfillopacity{0.700000}%
\pgfsetlinewidth{0.000000pt}%
\definecolor{currentstroke}{rgb}{0.000000,0.000000,0.000000}%
\pgfsetstrokecolor{currentstroke}%
\pgfsetdash{}{0pt}%
\pgfpathmoveto{\pgfqpoint{3.150311in}{1.970812in}}%
\pgfpathlineto{\pgfqpoint{3.164170in}{1.959905in}}%
\pgfpathlineto{\pgfqpoint{3.178030in}{1.949121in}}%
\pgfpathlineto{\pgfqpoint{3.191891in}{1.938460in}}%
\pgfpathlineto{\pgfqpoint{3.205752in}{1.927921in}}%
\pgfpathlineto{\pgfqpoint{3.197198in}{1.929498in}}%
\pgfpathlineto{\pgfqpoint{3.188629in}{1.931359in}}%
\pgfpathlineto{\pgfqpoint{3.180045in}{1.933510in}}%
\pgfpathlineto{\pgfqpoint{3.171447in}{1.935957in}}%
\pgfpathlineto{\pgfqpoint{3.157548in}{1.947031in}}%
\pgfpathlineto{\pgfqpoint{3.143650in}{1.958226in}}%
\pgfpathlineto{\pgfqpoint{3.129752in}{1.969545in}}%
\pgfpathlineto{\pgfqpoint{3.115854in}{1.980988in}}%
\pgfpathlineto{\pgfqpoint{3.124491in}{1.977999in}}%
\pgfpathlineto{\pgfqpoint{3.133113in}{1.975310in}}%
\pgfpathlineto{\pgfqpoint{3.141719in}{1.972916in}}%
\pgfpathlineto{\pgfqpoint{3.150311in}{1.970812in}}%
\pgfpathclose%
\pgfusepath{fill}%
\end{pgfscope}%
\begin{pgfscope}%
\pgfpathrectangle{\pgfqpoint{1.150000in}{0.150000in}}{\pgfqpoint{5.700000in}{5.700000in}}%
\pgfusepath{clip}%
\pgfsetbuttcap%
\pgfsetroundjoin%
\definecolor{currentfill}{rgb}{0.183898,0.422383,0.556944}%
\pgfsetfillcolor{currentfill}%
\pgfsetfillopacity{0.700000}%
\pgfsetlinewidth{0.000000pt}%
\definecolor{currentstroke}{rgb}{0.000000,0.000000,0.000000}%
\pgfsetstrokecolor{currentstroke}%
\pgfsetdash{}{0pt}%
\pgfpathmoveto{\pgfqpoint{5.333346in}{2.568172in}}%
\pgfpathlineto{\pgfqpoint{5.347855in}{2.575383in}}%
\pgfpathlineto{\pgfqpoint{5.362379in}{2.582694in}}%
\pgfpathlineto{\pgfqpoint{5.376918in}{2.590105in}}%
\pgfpathlineto{\pgfqpoint{5.391473in}{2.597617in}}%
\pgfpathlineto{\pgfqpoint{5.383800in}{2.585634in}}%
\pgfpathlineto{\pgfqpoint{5.376120in}{2.573544in}}%
\pgfpathlineto{\pgfqpoint{5.368433in}{2.561350in}}%
\pgfpathlineto{\pgfqpoint{5.360739in}{2.549051in}}%
\pgfpathlineto{\pgfqpoint{5.346184in}{2.541682in}}%
\pgfpathlineto{\pgfqpoint{5.331644in}{2.534413in}}%
\pgfpathlineto{\pgfqpoint{5.317119in}{2.527244in}}%
\pgfpathlineto{\pgfqpoint{5.302609in}{2.520176in}}%
\pgfpathlineto{\pgfqpoint{5.310303in}{2.532325in}}%
\pgfpathlineto{\pgfqpoint{5.317991in}{2.544375in}}%
\pgfpathlineto{\pgfqpoint{5.325672in}{2.556324in}}%
\pgfpathlineto{\pgfqpoint{5.333346in}{2.568172in}}%
\pgfpathclose%
\pgfusepath{fill}%
\end{pgfscope}%
\begin{pgfscope}%
\pgfpathrectangle{\pgfqpoint{1.150000in}{0.150000in}}{\pgfqpoint{5.700000in}{5.700000in}}%
\pgfusepath{clip}%
\pgfsetbuttcap%
\pgfsetroundjoin%
\definecolor{currentfill}{rgb}{0.220057,0.343307,0.549413}%
\pgfsetfillcolor{currentfill}%
\pgfsetfillopacity{0.700000}%
\pgfsetlinewidth{0.000000pt}%
\definecolor{currentstroke}{rgb}{0.000000,0.000000,0.000000}%
\pgfsetstrokecolor{currentstroke}%
\pgfsetdash{}{0pt}%
\pgfpathmoveto{\pgfqpoint{5.125259in}{2.369349in}}%
\pgfpathlineto{\pgfqpoint{5.139654in}{2.375277in}}%
\pgfpathlineto{\pgfqpoint{5.154064in}{2.381305in}}%
\pgfpathlineto{\pgfqpoint{5.168487in}{2.387432in}}%
\pgfpathlineto{\pgfqpoint{5.182924in}{2.393659in}}%
\pgfpathlineto{\pgfqpoint{5.175172in}{2.380870in}}%
\pgfpathlineto{\pgfqpoint{5.167414in}{2.368000in}}%
\pgfpathlineto{\pgfqpoint{5.159651in}{2.355051in}}%
\pgfpathlineto{\pgfqpoint{5.151881in}{2.342026in}}%
\pgfpathlineto{\pgfqpoint{5.137444in}{2.335998in}}%
\pgfpathlineto{\pgfqpoint{5.123021in}{2.330070in}}%
\pgfpathlineto{\pgfqpoint{5.108612in}{2.324241in}}%
\pgfpathlineto{\pgfqpoint{5.094217in}{2.318512in}}%
\pgfpathlineto{\pgfqpoint{5.101986in}{2.331331in}}%
\pgfpathlineto{\pgfqpoint{5.109750in}{2.344078in}}%
\pgfpathlineto{\pgfqpoint{5.117507in}{2.356751in}}%
\pgfpathlineto{\pgfqpoint{5.125259in}{2.369349in}}%
\pgfpathclose%
\pgfusepath{fill}%
\end{pgfscope}%
\begin{pgfscope}%
\pgfpathrectangle{\pgfqpoint{1.150000in}{0.150000in}}{\pgfqpoint{5.700000in}{5.700000in}}%
\pgfusepath{clip}%
\pgfsetbuttcap%
\pgfsetroundjoin%
\definecolor{currentfill}{rgb}{0.282327,0.094955,0.417331}%
\pgfsetfillcolor{currentfill}%
\pgfsetfillopacity{0.700000}%
\pgfsetlinewidth{0.000000pt}%
\definecolor{currentstroke}{rgb}{0.000000,0.000000,0.000000}%
\pgfsetstrokecolor{currentstroke}%
\pgfsetdash{}{0pt}%
\pgfpathmoveto{\pgfqpoint{4.444760in}{1.825816in}}%
\pgfpathlineto{\pgfqpoint{4.458837in}{1.826539in}}%
\pgfpathlineto{\pgfqpoint{4.472923in}{1.827361in}}%
\pgfpathlineto{\pgfqpoint{4.487019in}{1.828282in}}%
\pgfpathlineto{\pgfqpoint{4.501126in}{1.829301in}}%
\pgfpathlineto{\pgfqpoint{4.493188in}{1.817528in}}%
\pgfpathlineto{\pgfqpoint{4.485247in}{1.805797in}}%
\pgfpathlineto{\pgfqpoint{4.477300in}{1.794112in}}%
\pgfpathlineto{\pgfqpoint{4.469349in}{1.782477in}}%
\pgfpathlineto{\pgfqpoint{4.455239in}{1.781818in}}%
\pgfpathlineto{\pgfqpoint{4.441137in}{1.781257in}}%
\pgfpathlineto{\pgfqpoint{4.427046in}{1.780795in}}%
\pgfpathlineto{\pgfqpoint{4.412964in}{1.780431in}}%
\pgfpathlineto{\pgfqpoint{4.420920in}{1.791700in}}%
\pgfpathlineto{\pgfqpoint{4.428872in}{1.803023in}}%
\pgfpathlineto{\pgfqpoint{4.436818in}{1.814396in}}%
\pgfpathlineto{\pgfqpoint{4.444760in}{1.825816in}}%
\pgfpathclose%
\pgfusepath{fill}%
\end{pgfscope}%
\begin{pgfscope}%
\pgfpathrectangle{\pgfqpoint{1.150000in}{0.150000in}}{\pgfqpoint{5.700000in}{5.700000in}}%
\pgfusepath{clip}%
\pgfsetbuttcap%
\pgfsetroundjoin%
\definecolor{currentfill}{rgb}{0.279566,0.067836,0.391917}%
\pgfsetfillcolor{currentfill}%
\pgfsetfillopacity{0.700000}%
\pgfsetlinewidth{0.000000pt}%
\definecolor{currentstroke}{rgb}{0.000000,0.000000,0.000000}%
\pgfsetstrokecolor{currentstroke}%
\pgfsetdash{}{0pt}%
\pgfpathmoveto{\pgfqpoint{4.356732in}{1.779964in}}%
\pgfpathlineto{\pgfqpoint{4.370776in}{1.779932in}}%
\pgfpathlineto{\pgfqpoint{4.384829in}{1.780000in}}%
\pgfpathlineto{\pgfqpoint{4.398892in}{1.780166in}}%
\pgfpathlineto{\pgfqpoint{4.412964in}{1.780431in}}%
\pgfpathlineto{\pgfqpoint{4.405004in}{1.769219in}}%
\pgfpathlineto{\pgfqpoint{4.397038in}{1.758069in}}%
\pgfpathlineto{\pgfqpoint{4.389068in}{1.746982in}}%
\pgfpathlineto{\pgfqpoint{4.381093in}{1.735964in}}%
\pgfpathlineto{\pgfqpoint{4.367015in}{1.736077in}}%
\pgfpathlineto{\pgfqpoint{4.352946in}{1.736288in}}%
\pgfpathlineto{\pgfqpoint{4.338886in}{1.736598in}}%
\pgfpathlineto{\pgfqpoint{4.324835in}{1.737006in}}%
\pgfpathlineto{\pgfqpoint{4.332816in}{1.747640in}}%
\pgfpathlineto{\pgfqpoint{4.340793in}{1.758347in}}%
\pgfpathlineto{\pgfqpoint{4.348765in}{1.769123in}}%
\pgfpathlineto{\pgfqpoint{4.356732in}{1.779964in}}%
\pgfpathclose%
\pgfusepath{fill}%
\end{pgfscope}%
\begin{pgfscope}%
\pgfpathrectangle{\pgfqpoint{1.150000in}{0.150000in}}{\pgfqpoint{5.700000in}{5.700000in}}%
\pgfusepath{clip}%
\pgfsetbuttcap%
\pgfsetroundjoin%
\definecolor{currentfill}{rgb}{0.283229,0.120777,0.440584}%
\pgfsetfillcolor{currentfill}%
\pgfsetfillopacity{0.700000}%
\pgfsetlinewidth{0.000000pt}%
\definecolor{currentstroke}{rgb}{0.000000,0.000000,0.000000}%
\pgfsetstrokecolor{currentstroke}%
\pgfsetdash{}{0pt}%
\pgfpathmoveto{\pgfqpoint{4.532829in}{1.876754in}}%
\pgfpathlineto{\pgfqpoint{4.546941in}{1.878214in}}%
\pgfpathlineto{\pgfqpoint{4.561064in}{1.879773in}}%
\pgfpathlineto{\pgfqpoint{4.575198in}{1.881430in}}%
\pgfpathlineto{\pgfqpoint{4.589341in}{1.883185in}}%
\pgfpathlineto{\pgfqpoint{4.581426in}{1.870939in}}%
\pgfpathlineto{\pgfqpoint{4.573506in}{1.858717in}}%
\pgfpathlineto{\pgfqpoint{4.565581in}{1.846524in}}%
\pgfpathlineto{\pgfqpoint{4.557653in}{1.834362in}}%
\pgfpathlineto{\pgfqpoint{4.543505in}{1.832950in}}%
\pgfpathlineto{\pgfqpoint{4.529368in}{1.831635in}}%
\pgfpathlineto{\pgfqpoint{4.515242in}{1.830419in}}%
\pgfpathlineto{\pgfqpoint{4.501126in}{1.829301in}}%
\pgfpathlineto{\pgfqpoint{4.509058in}{1.841114in}}%
\pgfpathlineto{\pgfqpoint{4.516986in}{1.852962in}}%
\pgfpathlineto{\pgfqpoint{4.524910in}{1.864844in}}%
\pgfpathlineto{\pgfqpoint{4.532829in}{1.876754in}}%
\pgfpathclose%
\pgfusepath{fill}%
\end{pgfscope}%
\begin{pgfscope}%
\pgfpathrectangle{\pgfqpoint{1.150000in}{0.150000in}}{\pgfqpoint{5.700000in}{5.700000in}}%
\pgfusepath{clip}%
\pgfsetbuttcap%
\pgfsetroundjoin%
\definecolor{currentfill}{rgb}{0.255645,0.260703,0.528312}%
\pgfsetfillcolor{currentfill}%
\pgfsetfillopacity{0.700000}%
\pgfsetlinewidth{0.000000pt}%
\definecolor{currentstroke}{rgb}{0.000000,0.000000,0.000000}%
\pgfsetstrokecolor{currentstroke}%
\pgfsetdash{}{0pt}%
\pgfpathmoveto{\pgfqpoint{4.917189in}{2.174951in}}%
\pgfpathlineto{\pgfqpoint{4.931478in}{2.179435in}}%
\pgfpathlineto{\pgfqpoint{4.945779in}{2.184017in}}%
\pgfpathlineto{\pgfqpoint{4.960094in}{2.188699in}}%
\pgfpathlineto{\pgfqpoint{4.974422in}{2.193479in}}%
\pgfpathlineto{\pgfqpoint{4.966604in}{2.180346in}}%
\pgfpathlineto{\pgfqpoint{4.958781in}{2.167165in}}%
\pgfpathlineto{\pgfqpoint{4.950954in}{2.153939in}}%
\pgfpathlineto{\pgfqpoint{4.943121in}{2.140670in}}%
\pgfpathlineto{\pgfqpoint{4.928793in}{2.136144in}}%
\pgfpathlineto{\pgfqpoint{4.914479in}{2.131717in}}%
\pgfpathlineto{\pgfqpoint{4.900177in}{2.127388in}}%
\pgfpathlineto{\pgfqpoint{4.885888in}{2.123159in}}%
\pgfpathlineto{\pgfqpoint{4.893720in}{2.136166in}}%
\pgfpathlineto{\pgfqpoint{4.901548in}{2.149136in}}%
\pgfpathlineto{\pgfqpoint{4.909371in}{2.162065in}}%
\pgfpathlineto{\pgfqpoint{4.917189in}{2.174951in}}%
\pgfpathclose%
\pgfusepath{fill}%
\end{pgfscope}%
\begin{pgfscope}%
\pgfpathrectangle{\pgfqpoint{1.150000in}{0.150000in}}{\pgfqpoint{5.700000in}{5.700000in}}%
\pgfusepath{clip}%
\pgfsetbuttcap%
\pgfsetroundjoin%
\definecolor{currentfill}{rgb}{0.277018,0.050344,0.375715}%
\pgfsetfillcolor{currentfill}%
\pgfsetfillopacity{0.700000}%
\pgfsetlinewidth{0.000000pt}%
\definecolor{currentstroke}{rgb}{0.000000,0.000000,0.000000}%
\pgfsetstrokecolor{currentstroke}%
\pgfsetdash{}{0pt}%
\pgfpathmoveto{\pgfqpoint{4.268719in}{1.739633in}}%
\pgfpathlineto{\pgfqpoint{4.282735in}{1.738827in}}%
\pgfpathlineto{\pgfqpoint{4.296759in}{1.738121in}}%
\pgfpathlineto{\pgfqpoint{4.310792in}{1.737514in}}%
\pgfpathlineto{\pgfqpoint{4.324835in}{1.737006in}}%
\pgfpathlineto{\pgfqpoint{4.316848in}{1.726448in}}%
\pgfpathlineto{\pgfqpoint{4.308857in}{1.715970in}}%
\pgfpathlineto{\pgfqpoint{4.300860in}{1.705576in}}%
\pgfpathlineto{\pgfqpoint{4.292859in}{1.695269in}}%
\pgfpathlineto{\pgfqpoint{4.278809in}{1.696172in}}%
\pgfpathlineto{\pgfqpoint{4.264767in}{1.697174in}}%
\pgfpathlineto{\pgfqpoint{4.250734in}{1.698274in}}%
\pgfpathlineto{\pgfqpoint{4.236710in}{1.699475in}}%
\pgfpathlineto{\pgfqpoint{4.244720in}{1.709380in}}%
\pgfpathlineto{\pgfqpoint{4.252725in}{1.719377in}}%
\pgfpathlineto{\pgfqpoint{4.260724in}{1.729463in}}%
\pgfpathlineto{\pgfqpoint{4.268719in}{1.739633in}}%
\pgfpathclose%
\pgfusepath{fill}%
\end{pgfscope}%
\begin{pgfscope}%
\pgfpathrectangle{\pgfqpoint{1.150000in}{0.150000in}}{\pgfqpoint{5.700000in}{5.700000in}}%
\pgfusepath{clip}%
\pgfsetbuttcap%
\pgfsetroundjoin%
\definecolor{currentfill}{rgb}{0.269944,0.014625,0.341379}%
\pgfsetfillcolor{currentfill}%
\pgfsetfillopacity{0.700000}%
\pgfsetlinewidth{0.000000pt}%
\definecolor{currentstroke}{rgb}{0.000000,0.000000,0.000000}%
\pgfsetstrokecolor{currentstroke}%
\pgfsetdash{}{0pt}%
\pgfpathmoveto{\pgfqpoint{3.804881in}{1.679262in}}%
\pgfpathlineto{\pgfqpoint{3.818780in}{1.674333in}}%
\pgfpathlineto{\pgfqpoint{3.832685in}{1.669509in}}%
\pgfpathlineto{\pgfqpoint{3.846596in}{1.664789in}}%
\pgfpathlineto{\pgfqpoint{3.860512in}{1.660173in}}%
\pgfpathlineto{\pgfqpoint{3.852353in}{1.654135in}}%
\pgfpathlineto{\pgfqpoint{3.844187in}{1.648271in}}%
\pgfpathlineto{\pgfqpoint{3.836013in}{1.642586in}}%
\pgfpathlineto{\pgfqpoint{3.827830in}{1.637084in}}%
\pgfpathlineto{\pgfqpoint{3.813895in}{1.642167in}}%
\pgfpathlineto{\pgfqpoint{3.799966in}{1.647354in}}%
\pgfpathlineto{\pgfqpoint{3.786041in}{1.652645in}}%
\pgfpathlineto{\pgfqpoint{3.772122in}{1.658041in}}%
\pgfpathlineto{\pgfqpoint{3.780324in}{1.663068in}}%
\pgfpathlineto{\pgfqpoint{3.788518in}{1.668284in}}%
\pgfpathlineto{\pgfqpoint{3.796703in}{1.673684in}}%
\pgfpathlineto{\pgfqpoint{3.804881in}{1.679262in}}%
\pgfpathclose%
\pgfusepath{fill}%
\end{pgfscope}%
\begin{pgfscope}%
\pgfpathrectangle{\pgfqpoint{1.150000in}{0.150000in}}{\pgfqpoint{5.700000in}{5.700000in}}%
\pgfusepath{clip}%
\pgfsetbuttcap%
\pgfsetroundjoin%
\definecolor{currentfill}{rgb}{0.280267,0.073417,0.397163}%
\pgfsetfillcolor{currentfill}%
\pgfsetfillopacity{0.700000}%
\pgfsetlinewidth{0.000000pt}%
\definecolor{currentstroke}{rgb}{0.000000,0.000000,0.000000}%
\pgfsetstrokecolor{currentstroke}%
\pgfsetdash{}{0pt}%
\pgfpathmoveto{\pgfqpoint{3.461301in}{1.779999in}}%
\pgfpathlineto{\pgfqpoint{3.475163in}{1.771954in}}%
\pgfpathlineto{\pgfqpoint{3.489028in}{1.764021in}}%
\pgfpathlineto{\pgfqpoint{3.502897in}{1.756200in}}%
\pgfpathlineto{\pgfqpoint{3.516768in}{1.748490in}}%
\pgfpathlineto{\pgfqpoint{3.508423in}{1.746471in}}%
\pgfpathlineto{\pgfqpoint{3.500066in}{1.744690in}}%
\pgfpathlineto{\pgfqpoint{3.491699in}{1.743152in}}%
\pgfpathlineto{\pgfqpoint{3.483320in}{1.741861in}}%
\pgfpathlineto{\pgfqpoint{3.469420in}{1.750079in}}%
\pgfpathlineto{\pgfqpoint{3.455522in}{1.758408in}}%
\pgfpathlineto{\pgfqpoint{3.441628in}{1.766849in}}%
\pgfpathlineto{\pgfqpoint{3.427736in}{1.775402in}}%
\pgfpathlineto{\pgfqpoint{3.436145in}{1.776178in}}%
\pgfpathlineto{\pgfqpoint{3.444542in}{1.777206in}}%
\pgfpathlineto{\pgfqpoint{3.452927in}{1.778481in}}%
\pgfpathlineto{\pgfqpoint{3.461301in}{1.779999in}}%
\pgfpathclose%
\pgfusepath{fill}%
\end{pgfscope}%
\begin{pgfscope}%
\pgfpathrectangle{\pgfqpoint{1.150000in}{0.150000in}}{\pgfqpoint{5.700000in}{5.700000in}}%
\pgfusepath{clip}%
\pgfsetbuttcap%
\pgfsetroundjoin%
\definecolor{currentfill}{rgb}{0.281887,0.150881,0.465405}%
\pgfsetfillcolor{currentfill}%
\pgfsetfillopacity{0.700000}%
\pgfsetlinewidth{0.000000pt}%
\definecolor{currentstroke}{rgb}{0.000000,0.000000,0.000000}%
\pgfsetstrokecolor{currentstroke}%
\pgfsetdash{}{0pt}%
\pgfpathmoveto{\pgfqpoint{4.620959in}{1.932358in}}%
\pgfpathlineto{\pgfqpoint{4.635111in}{1.934537in}}%
\pgfpathlineto{\pgfqpoint{4.649274in}{1.936814in}}%
\pgfpathlineto{\pgfqpoint{4.663448in}{1.939190in}}%
\pgfpathlineto{\pgfqpoint{4.677633in}{1.941664in}}%
\pgfpathlineto{\pgfqpoint{4.669737in}{1.929030in}}%
\pgfpathlineto{\pgfqpoint{4.661838in}{1.916405in}}%
\pgfpathlineto{\pgfqpoint{4.653934in}{1.903790in}}%
\pgfpathlineto{\pgfqpoint{4.646025in}{1.891190in}}%
\pgfpathlineto{\pgfqpoint{4.631838in}{1.889042in}}%
\pgfpathlineto{\pgfqpoint{4.617662in}{1.886991in}}%
\pgfpathlineto{\pgfqpoint{4.603496in}{1.885039in}}%
\pgfpathlineto{\pgfqpoint{4.589341in}{1.883185in}}%
\pgfpathlineto{\pgfqpoint{4.597253in}{1.895453in}}%
\pgfpathlineto{\pgfqpoint{4.605159in}{1.907740in}}%
\pgfpathlineto{\pgfqpoint{4.613061in}{1.920043in}}%
\pgfpathlineto{\pgfqpoint{4.620959in}{1.932358in}}%
\pgfpathclose%
\pgfusepath{fill}%
\end{pgfscope}%
\begin{pgfscope}%
\pgfpathrectangle{\pgfqpoint{1.150000in}{0.150000in}}{\pgfqpoint{5.700000in}{5.700000in}}%
\pgfusepath{clip}%
\pgfsetbuttcap%
\pgfsetroundjoin%
\definecolor{currentfill}{rgb}{0.281887,0.150881,0.465405}%
\pgfsetfillcolor{currentfill}%
\pgfsetfillopacity{0.700000}%
\pgfsetlinewidth{0.000000pt}%
\definecolor{currentstroke}{rgb}{0.000000,0.000000,0.000000}%
\pgfsetstrokecolor{currentstroke}%
\pgfsetdash{}{0pt}%
\pgfpathmoveto{\pgfqpoint{3.205752in}{1.927921in}}%
\pgfpathlineto{\pgfqpoint{3.219615in}{1.917503in}}%
\pgfpathlineto{\pgfqpoint{3.233478in}{1.907206in}}%
\pgfpathlineto{\pgfqpoint{3.247343in}{1.897029in}}%
\pgfpathlineto{\pgfqpoint{3.261209in}{1.886972in}}%
\pgfpathlineto{\pgfqpoint{3.252690in}{1.888023in}}%
\pgfpathlineto{\pgfqpoint{3.244157in}{1.889354in}}%
\pgfpathlineto{\pgfqpoint{3.235610in}{1.890969in}}%
\pgfpathlineto{\pgfqpoint{3.227049in}{1.892874in}}%
\pgfpathlineto{\pgfqpoint{3.213147in}{1.903464in}}%
\pgfpathlineto{\pgfqpoint{3.199246in}{1.914174in}}%
\pgfpathlineto{\pgfqpoint{3.185346in}{1.925005in}}%
\pgfpathlineto{\pgfqpoint{3.171447in}{1.935957in}}%
\pgfpathlineto{\pgfqpoint{3.180045in}{1.933510in}}%
\pgfpathlineto{\pgfqpoint{3.188629in}{1.931359in}}%
\pgfpathlineto{\pgfqpoint{3.197198in}{1.929498in}}%
\pgfpathlineto{\pgfqpoint{3.205752in}{1.927921in}}%
\pgfpathclose%
\pgfusepath{fill}%
\end{pgfscope}%
\begin{pgfscope}%
\pgfpathrectangle{\pgfqpoint{1.150000in}{0.150000in}}{\pgfqpoint{5.700000in}{5.700000in}}%
\pgfusepath{clip}%
\pgfsetbuttcap%
\pgfsetroundjoin%
\definecolor{currentfill}{rgb}{0.268510,0.009605,0.335427}%
\pgfsetfillcolor{currentfill}%
\pgfsetfillopacity{0.700000}%
\pgfsetlinewidth{0.000000pt}%
\definecolor{currentstroke}{rgb}{0.000000,0.000000,0.000000}%
\pgfsetstrokecolor{currentstroke}%
\pgfsetdash{}{0pt}%
\pgfpathmoveto{\pgfqpoint{3.948729in}{1.670334in}}%
\pgfpathlineto{\pgfqpoint{3.962659in}{1.666681in}}%
\pgfpathlineto{\pgfqpoint{3.976596in}{1.663130in}}%
\pgfpathlineto{\pgfqpoint{3.990539in}{1.659680in}}%
\pgfpathlineto{\pgfqpoint{4.004488in}{1.656333in}}%
\pgfpathlineto{\pgfqpoint{3.996391in}{1.648762in}}%
\pgfpathlineto{\pgfqpoint{3.988286in}{1.641337in}}%
\pgfpathlineto{\pgfqpoint{3.980175in}{1.634064in}}%
\pgfpathlineto{\pgfqpoint{3.972057in}{1.626946in}}%
\pgfpathlineto{\pgfqpoint{3.958092in}{1.630742in}}%
\pgfpathlineto{\pgfqpoint{3.944134in}{1.634639in}}%
\pgfpathlineto{\pgfqpoint{3.930182in}{1.638638in}}%
\pgfpathlineto{\pgfqpoint{3.916236in}{1.642740in}}%
\pgfpathlineto{\pgfqpoint{3.924370in}{1.649402in}}%
\pgfpathlineto{\pgfqpoint{3.932497in}{1.656225in}}%
\pgfpathlineto{\pgfqpoint{3.940616in}{1.663204in}}%
\pgfpathlineto{\pgfqpoint{3.948729in}{1.670334in}}%
\pgfpathclose%
\pgfusepath{fill}%
\end{pgfscope}%
\begin{pgfscope}%
\pgfpathrectangle{\pgfqpoint{1.150000in}{0.150000in}}{\pgfqpoint{5.700000in}{5.700000in}}%
\pgfusepath{clip}%
\pgfsetbuttcap%
\pgfsetroundjoin%
\definecolor{currentfill}{rgb}{0.273809,0.031497,0.358853}%
\pgfsetfillcolor{currentfill}%
\pgfsetfillopacity{0.700000}%
\pgfsetlinewidth{0.000000pt}%
\definecolor{currentstroke}{rgb}{0.000000,0.000000,0.000000}%
\pgfsetstrokecolor{currentstroke}%
\pgfsetdash{}{0pt}%
\pgfpathmoveto{\pgfqpoint{3.660943in}{1.705000in}}%
\pgfpathlineto{\pgfqpoint{3.674824in}{1.698758in}}%
\pgfpathlineto{\pgfqpoint{3.688710in}{1.692623in}}%
\pgfpathlineto{\pgfqpoint{3.702600in}{1.686594in}}%
\pgfpathlineto{\pgfqpoint{3.716495in}{1.680672in}}%
\pgfpathlineto{\pgfqpoint{3.708263in}{1.676317in}}%
\pgfpathlineto{\pgfqpoint{3.700022in}{1.672163in}}%
\pgfpathlineto{\pgfqpoint{3.691772in}{1.668217in}}%
\pgfpathlineto{\pgfqpoint{3.683513in}{1.664484in}}%
\pgfpathlineto{\pgfqpoint{3.669595in}{1.670893in}}%
\pgfpathlineto{\pgfqpoint{3.655682in}{1.677407in}}%
\pgfpathlineto{\pgfqpoint{3.641773in}{1.684029in}}%
\pgfpathlineto{\pgfqpoint{3.627868in}{1.690758in}}%
\pgfpathlineto{\pgfqpoint{3.636151in}{1.693997in}}%
\pgfpathlineto{\pgfqpoint{3.644425in}{1.697454in}}%
\pgfpathlineto{\pgfqpoint{3.652689in}{1.701124in}}%
\pgfpathlineto{\pgfqpoint{3.660943in}{1.705000in}}%
\pgfpathclose%
\pgfusepath{fill}%
\end{pgfscope}%
\begin{pgfscope}%
\pgfpathrectangle{\pgfqpoint{1.150000in}{0.150000in}}{\pgfqpoint{5.700000in}{5.700000in}}%
\pgfusepath{clip}%
\pgfsetbuttcap%
\pgfsetroundjoin%
\definecolor{currentfill}{rgb}{0.143343,0.522773,0.556295}%
\pgfsetfillcolor{currentfill}%
\pgfsetfillopacity{0.700000}%
\pgfsetlinewidth{0.000000pt}%
\definecolor{currentstroke}{rgb}{0.000000,0.000000,0.000000}%
\pgfsetstrokecolor{currentstroke}%
\pgfsetdash{}{0pt}%
\pgfpathmoveto{\pgfqpoint{5.630155in}{2.840891in}}%
\pgfpathlineto{\pgfqpoint{5.644843in}{2.849687in}}%
\pgfpathlineto{\pgfqpoint{5.659549in}{2.858584in}}%
\pgfpathlineto{\pgfqpoint{5.674270in}{2.867583in}}%
\pgfpathlineto{\pgfqpoint{5.689009in}{2.876683in}}%
\pgfpathlineto{\pgfqpoint{5.681469in}{2.866342in}}%
\pgfpathlineto{\pgfqpoint{5.673921in}{2.855868in}}%
\pgfpathlineto{\pgfqpoint{5.666364in}{2.845262in}}%
\pgfpathlineto{\pgfqpoint{5.658798in}{2.834524in}}%
\pgfpathlineto{\pgfqpoint{5.644056in}{2.825489in}}%
\pgfpathlineto{\pgfqpoint{5.629331in}{2.816555in}}%
\pgfpathlineto{\pgfqpoint{5.614622in}{2.807722in}}%
\pgfpathlineto{\pgfqpoint{5.599930in}{2.798991in}}%
\pgfpathlineto{\pgfqpoint{5.607499in}{2.809656in}}%
\pgfpathlineto{\pgfqpoint{5.615059in}{2.820195in}}%
\pgfpathlineto{\pgfqpoint{5.622611in}{2.830606in}}%
\pgfpathlineto{\pgfqpoint{5.630155in}{2.840891in}}%
\pgfpathclose%
\pgfusepath{fill}%
\end{pgfscope}%
\begin{pgfscope}%
\pgfpathrectangle{\pgfqpoint{1.150000in}{0.150000in}}{\pgfqpoint{5.700000in}{5.700000in}}%
\pgfusepath{clip}%
\pgfsetbuttcap%
\pgfsetroundjoin%
\definecolor{currentfill}{rgb}{0.273809,0.031497,0.358853}%
\pgfsetfillcolor{currentfill}%
\pgfsetfillopacity{0.700000}%
\pgfsetlinewidth{0.000000pt}%
\definecolor{currentstroke}{rgb}{0.000000,0.000000,0.000000}%
\pgfsetstrokecolor{currentstroke}%
\pgfsetdash{}{0pt}%
\pgfpathmoveto{\pgfqpoint{4.180695in}{1.705271in}}%
\pgfpathlineto{\pgfqpoint{4.194686in}{1.703672in}}%
\pgfpathlineto{\pgfqpoint{4.208686in}{1.702173in}}%
\pgfpathlineto{\pgfqpoint{4.222694in}{1.700774in}}%
\pgfpathlineto{\pgfqpoint{4.236710in}{1.699475in}}%
\pgfpathlineto{\pgfqpoint{4.228695in}{1.689665in}}%
\pgfpathlineto{\pgfqpoint{4.220674in}{1.679956in}}%
\pgfpathlineto{\pgfqpoint{4.212648in}{1.670350in}}%
\pgfpathlineto{\pgfqpoint{4.204617in}{1.660853in}}%
\pgfpathlineto{\pgfqpoint{4.190591in}{1.662565in}}%
\pgfpathlineto{\pgfqpoint{4.176573in}{1.664376in}}%
\pgfpathlineto{\pgfqpoint{4.162563in}{1.666287in}}%
\pgfpathlineto{\pgfqpoint{4.148561in}{1.668298in}}%
\pgfpathlineto{\pgfqpoint{4.156603in}{1.677377in}}%
\pgfpathlineto{\pgfqpoint{4.164639in}{1.686568in}}%
\pgfpathlineto{\pgfqpoint{4.172670in}{1.695867in}}%
\pgfpathlineto{\pgfqpoint{4.180695in}{1.705271in}}%
\pgfpathclose%
\pgfusepath{fill}%
\end{pgfscope}%
\begin{pgfscope}%
\pgfpathrectangle{\pgfqpoint{1.150000in}{0.150000in}}{\pgfqpoint{5.700000in}{5.700000in}}%
\pgfusepath{clip}%
\pgfsetbuttcap%
\pgfsetroundjoin%
\definecolor{currentfill}{rgb}{0.171176,0.452530,0.557965}%
\pgfsetfillcolor{currentfill}%
\pgfsetfillopacity{0.700000}%
\pgfsetlinewidth{0.000000pt}%
\definecolor{currentstroke}{rgb}{0.000000,0.000000,0.000000}%
\pgfsetstrokecolor{currentstroke}%
\pgfsetdash{}{0pt}%
\pgfpathmoveto{\pgfqpoint{5.422093in}{2.644465in}}%
\pgfpathlineto{\pgfqpoint{5.436662in}{2.652201in}}%
\pgfpathlineto{\pgfqpoint{5.451246in}{2.660037in}}%
\pgfpathlineto{\pgfqpoint{5.465846in}{2.667974in}}%
\pgfpathlineto{\pgfqpoint{5.480462in}{2.676011in}}%
\pgfpathlineto{\pgfqpoint{5.472820in}{2.664348in}}%
\pgfpathlineto{\pgfqpoint{5.465169in}{2.652569in}}%
\pgfpathlineto{\pgfqpoint{5.457512in}{2.640676in}}%
\pgfpathlineto{\pgfqpoint{5.449847in}{2.628669in}}%
\pgfpathlineto{\pgfqpoint{5.435230in}{2.620755in}}%
\pgfpathlineto{\pgfqpoint{5.420629in}{2.612942in}}%
\pgfpathlineto{\pgfqpoint{5.406043in}{2.605229in}}%
\pgfpathlineto{\pgfqpoint{5.391473in}{2.597617in}}%
\pgfpathlineto{\pgfqpoint{5.399139in}{2.609493in}}%
\pgfpathlineto{\pgfqpoint{5.406798in}{2.621260in}}%
\pgfpathlineto{\pgfqpoint{5.414449in}{2.632918in}}%
\pgfpathlineto{\pgfqpoint{5.422093in}{2.644465in}}%
\pgfpathclose%
\pgfusepath{fill}%
\end{pgfscope}%
\begin{pgfscope}%
\pgfpathrectangle{\pgfqpoint{1.150000in}{0.150000in}}{\pgfqpoint{5.700000in}{5.700000in}}%
\pgfusepath{clip}%
\pgfsetbuttcap%
\pgfsetroundjoin%
\definecolor{currentfill}{rgb}{0.204903,0.375746,0.553533}%
\pgfsetfillcolor{currentfill}%
\pgfsetfillopacity{0.700000}%
\pgfsetlinewidth{0.000000pt}%
\definecolor{currentstroke}{rgb}{0.000000,0.000000,0.000000}%
\pgfsetstrokecolor{currentstroke}%
\pgfsetdash{}{0pt}%
\pgfpathmoveto{\pgfqpoint{5.213871in}{2.443984in}}%
\pgfpathlineto{\pgfqpoint{5.228322in}{2.450491in}}%
\pgfpathlineto{\pgfqpoint{5.242788in}{2.457098in}}%
\pgfpathlineto{\pgfqpoint{5.257269in}{2.463805in}}%
\pgfpathlineto{\pgfqpoint{5.271764in}{2.470612in}}%
\pgfpathlineto{\pgfqpoint{5.264037in}{2.457985in}}%
\pgfpathlineto{\pgfqpoint{5.256303in}{2.445267in}}%
\pgfpathlineto{\pgfqpoint{5.248564in}{2.432460in}}%
\pgfpathlineto{\pgfqpoint{5.240817in}{2.419564in}}%
\pgfpathlineto{\pgfqpoint{5.226322in}{2.412938in}}%
\pgfpathlineto{\pgfqpoint{5.211842in}{2.406412in}}%
\pgfpathlineto{\pgfqpoint{5.197376in}{2.399986in}}%
\pgfpathlineto{\pgfqpoint{5.182924in}{2.393659in}}%
\pgfpathlineto{\pgfqpoint{5.190670in}{2.406367in}}%
\pgfpathlineto{\pgfqpoint{5.198410in}{2.418992in}}%
\pgfpathlineto{\pgfqpoint{5.206143in}{2.431531in}}%
\pgfpathlineto{\pgfqpoint{5.213871in}{2.443984in}}%
\pgfpathclose%
\pgfusepath{fill}%
\end{pgfscope}%
\begin{pgfscope}%
\pgfpathrectangle{\pgfqpoint{1.150000in}{0.150000in}}{\pgfqpoint{5.700000in}{5.700000in}}%
\pgfusepath{clip}%
\pgfsetbuttcap%
\pgfsetroundjoin%
\definecolor{currentfill}{rgb}{0.278012,0.180367,0.486697}%
\pgfsetfillcolor{currentfill}%
\pgfsetfillopacity{0.700000}%
\pgfsetlinewidth{0.000000pt}%
\definecolor{currentstroke}{rgb}{0.000000,0.000000,0.000000}%
\pgfsetstrokecolor{currentstroke}%
\pgfsetdash{}{0pt}%
\pgfpathmoveto{\pgfqpoint{4.709168in}{1.992221in}}%
\pgfpathlineto{\pgfqpoint{4.723363in}{1.995101in}}%
\pgfpathlineto{\pgfqpoint{4.737569in}{1.998079in}}%
\pgfpathlineto{\pgfqpoint{4.751787in}{2.001156in}}%
\pgfpathlineto{\pgfqpoint{4.766017in}{2.004331in}}%
\pgfpathlineto{\pgfqpoint{4.758141in}{1.991393in}}%
\pgfpathlineto{\pgfqpoint{4.750260in}{1.978447in}}%
\pgfpathlineto{\pgfqpoint{4.742376in}{1.965496in}}%
\pgfpathlineto{\pgfqpoint{4.734486in}{1.952542in}}%
\pgfpathlineto{\pgfqpoint{4.720256in}{1.949675in}}%
\pgfpathlineto{\pgfqpoint{4.706037in}{1.946907in}}%
\pgfpathlineto{\pgfqpoint{4.691829in}{1.944236in}}%
\pgfpathlineto{\pgfqpoint{4.677633in}{1.941664in}}%
\pgfpathlineto{\pgfqpoint{4.685523in}{1.954303in}}%
\pgfpathlineto{\pgfqpoint{4.693410in}{1.966944in}}%
\pgfpathlineto{\pgfqpoint{4.701291in}{1.979584in}}%
\pgfpathlineto{\pgfqpoint{4.709168in}{1.992221in}}%
\pgfpathclose%
\pgfusepath{fill}%
\end{pgfscope}%
\begin{pgfscope}%
\pgfpathrectangle{\pgfqpoint{1.150000in}{0.150000in}}{\pgfqpoint{5.700000in}{5.700000in}}%
\pgfusepath{clip}%
\pgfsetbuttcap%
\pgfsetroundjoin%
\definecolor{currentfill}{rgb}{0.241237,0.296485,0.539709}%
\pgfsetfillcolor{currentfill}%
\pgfsetfillopacity{0.700000}%
\pgfsetlinewidth{0.000000pt}%
\definecolor{currentstroke}{rgb}{0.000000,0.000000,0.000000}%
\pgfsetstrokecolor{currentstroke}%
\pgfsetdash{}{0pt}%
\pgfpathmoveto{\pgfqpoint{5.005640in}{2.245496in}}%
\pgfpathlineto{\pgfqpoint{5.019981in}{2.250611in}}%
\pgfpathlineto{\pgfqpoint{5.034335in}{2.255825in}}%
\pgfpathlineto{\pgfqpoint{5.048703in}{2.261139in}}%
\pgfpathlineto{\pgfqpoint{5.063084in}{2.266551in}}%
\pgfpathlineto{\pgfqpoint{5.055287in}{2.253400in}}%
\pgfpathlineto{\pgfqpoint{5.047485in}{2.240188in}}%
\pgfpathlineto{\pgfqpoint{5.039677in}{2.226917in}}%
\pgfpathlineto{\pgfqpoint{5.031864in}{2.213590in}}%
\pgfpathlineto{\pgfqpoint{5.017483in}{2.208414in}}%
\pgfpathlineto{\pgfqpoint{5.003116in}{2.203337in}}%
\pgfpathlineto{\pgfqpoint{4.988762in}{2.198359in}}%
\pgfpathlineto{\pgfqpoint{4.974422in}{2.193479in}}%
\pgfpathlineto{\pgfqpoint{4.982234in}{2.206563in}}%
\pgfpathlineto{\pgfqpoint{4.990041in}{2.219595in}}%
\pgfpathlineto{\pgfqpoint{4.997843in}{2.232574in}}%
\pgfpathlineto{\pgfqpoint{5.005640in}{2.245496in}}%
\pgfpathclose%
\pgfusepath{fill}%
\end{pgfscope}%
\begin{pgfscope}%
\pgfpathrectangle{\pgfqpoint{1.150000in}{0.150000in}}{\pgfqpoint{5.700000in}{5.700000in}}%
\pgfusepath{clip}%
\pgfsetbuttcap%
\pgfsetroundjoin%
\definecolor{currentfill}{rgb}{0.208623,0.367752,0.552675}%
\pgfsetfillcolor{currentfill}%
\pgfsetfillopacity{0.700000}%
\pgfsetlinewidth{0.000000pt}%
\definecolor{currentstroke}{rgb}{0.000000,0.000000,0.000000}%
\pgfsetstrokecolor{currentstroke}%
\pgfsetdash{}{0pt}%
\pgfpathmoveto{\pgfqpoint{2.670552in}{2.417054in}}%
\pgfpathlineto{\pgfqpoint{2.684506in}{2.401229in}}%
\pgfpathlineto{\pgfqpoint{2.698456in}{2.385555in}}%
\pgfpathlineto{\pgfqpoint{2.712402in}{2.370033in}}%
\pgfpathlineto{\pgfqpoint{2.726345in}{2.354660in}}%
\pgfpathlineto{\pgfqpoint{2.717377in}{2.361856in}}%
\pgfpathlineto{\pgfqpoint{2.708388in}{2.369404in}}%
\pgfpathlineto{\pgfqpoint{2.699378in}{2.377311in}}%
\pgfpathlineto{\pgfqpoint{2.690346in}{2.385581in}}%
\pgfpathlineto{\pgfqpoint{2.676352in}{2.401530in}}%
\pgfpathlineto{\pgfqpoint{2.662354in}{2.417629in}}%
\pgfpathlineto{\pgfqpoint{2.648351in}{2.433879in}}%
\pgfpathlineto{\pgfqpoint{2.634345in}{2.450283in}}%
\pgfpathlineto{\pgfqpoint{2.643430in}{2.441426in}}%
\pgfpathlineto{\pgfqpoint{2.652492in}{2.432940in}}%
\pgfpathlineto{\pgfqpoint{2.661533in}{2.424818in}}%
\pgfpathlineto{\pgfqpoint{2.670552in}{2.417054in}}%
\pgfpathclose%
\pgfusepath{fill}%
\end{pgfscope}%
\begin{pgfscope}%
\pgfpathrectangle{\pgfqpoint{1.150000in}{0.150000in}}{\pgfqpoint{5.700000in}{5.700000in}}%
\pgfusepath{clip}%
\pgfsetbuttcap%
\pgfsetroundjoin%
\definecolor{currentfill}{rgb}{0.197636,0.391528,0.554969}%
\pgfsetfillcolor{currentfill}%
\pgfsetfillopacity{0.700000}%
\pgfsetlinewidth{0.000000pt}%
\definecolor{currentstroke}{rgb}{0.000000,0.000000,0.000000}%
\pgfsetstrokecolor{currentstroke}%
\pgfsetdash{}{0pt}%
\pgfpathmoveto{\pgfqpoint{2.614696in}{2.481896in}}%
\pgfpathlineto{\pgfqpoint{2.628666in}{2.465452in}}%
\pgfpathlineto{\pgfqpoint{2.642633in}{2.449164in}}%
\pgfpathlineto{\pgfqpoint{2.656594in}{2.433032in}}%
\pgfpathlineto{\pgfqpoint{2.670552in}{2.417054in}}%
\pgfpathlineto{\pgfqpoint{2.661533in}{2.424818in}}%
\pgfpathlineto{\pgfqpoint{2.652492in}{2.432940in}}%
\pgfpathlineto{\pgfqpoint{2.643430in}{2.441426in}}%
\pgfpathlineto{\pgfqpoint{2.634345in}{2.450283in}}%
\pgfpathlineto{\pgfqpoint{2.620334in}{2.466840in}}%
\pgfpathlineto{\pgfqpoint{2.606319in}{2.483552in}}%
\pgfpathlineto{\pgfqpoint{2.592299in}{2.500420in}}%
\pgfpathlineto{\pgfqpoint{2.578274in}{2.517445in}}%
\pgfpathlineto{\pgfqpoint{2.587414in}{2.508000in}}%
\pgfpathlineto{\pgfqpoint{2.596530in}{2.498930in}}%
\pgfpathlineto{\pgfqpoint{2.605624in}{2.490231in}}%
\pgfpathlineto{\pgfqpoint{2.614696in}{2.481896in}}%
\pgfpathclose%
\pgfusepath{fill}%
\end{pgfscope}%
\begin{pgfscope}%
\pgfpathrectangle{\pgfqpoint{1.150000in}{0.150000in}}{\pgfqpoint{5.700000in}{5.700000in}}%
\pgfusepath{clip}%
\pgfsetbuttcap%
\pgfsetroundjoin%
\definecolor{currentfill}{rgb}{0.283072,0.130895,0.449241}%
\pgfsetfillcolor{currentfill}%
\pgfsetfillopacity{0.700000}%
\pgfsetlinewidth{0.000000pt}%
\definecolor{currentstroke}{rgb}{0.000000,0.000000,0.000000}%
\pgfsetstrokecolor{currentstroke}%
\pgfsetdash{}{0pt}%
\pgfpathmoveto{\pgfqpoint{3.261209in}{1.886972in}}%
\pgfpathlineto{\pgfqpoint{3.275076in}{1.877033in}}%
\pgfpathlineto{\pgfqpoint{3.288945in}{1.867213in}}%
\pgfpathlineto{\pgfqpoint{3.302816in}{1.857511in}}%
\pgfpathlineto{\pgfqpoint{3.316688in}{1.847925in}}%
\pgfpathlineto{\pgfqpoint{3.308203in}{1.848453in}}%
\pgfpathlineto{\pgfqpoint{3.299705in}{1.849254in}}%
\pgfpathlineto{\pgfqpoint{3.291193in}{1.850335in}}%
\pgfpathlineto{\pgfqpoint{3.282667in}{1.851700in}}%
\pgfpathlineto{\pgfqpoint{3.268761in}{1.861816in}}%
\pgfpathlineto{\pgfqpoint{3.254856in}{1.872051in}}%
\pgfpathlineto{\pgfqpoint{3.240952in}{1.882403in}}%
\pgfpathlineto{\pgfqpoint{3.227049in}{1.892874in}}%
\pgfpathlineto{\pgfqpoint{3.235610in}{1.890969in}}%
\pgfpathlineto{\pgfqpoint{3.244157in}{1.889354in}}%
\pgfpathlineto{\pgfqpoint{3.252690in}{1.888023in}}%
\pgfpathlineto{\pgfqpoint{3.261209in}{1.886972in}}%
\pgfpathclose%
\pgfusepath{fill}%
\end{pgfscope}%
\begin{pgfscope}%
\pgfpathrectangle{\pgfqpoint{1.150000in}{0.150000in}}{\pgfqpoint{5.700000in}{5.700000in}}%
\pgfusepath{clip}%
\pgfsetbuttcap%
\pgfsetroundjoin%
\definecolor{currentfill}{rgb}{0.220057,0.343307,0.549413}%
\pgfsetfillcolor{currentfill}%
\pgfsetfillopacity{0.700000}%
\pgfsetlinewidth{0.000000pt}%
\definecolor{currentstroke}{rgb}{0.000000,0.000000,0.000000}%
\pgfsetstrokecolor{currentstroke}%
\pgfsetdash{}{0pt}%
\pgfpathmoveto{\pgfqpoint{2.726345in}{2.354660in}}%
\pgfpathlineto{\pgfqpoint{2.740284in}{2.339436in}}%
\pgfpathlineto{\pgfqpoint{2.754220in}{2.324359in}}%
\pgfpathlineto{\pgfqpoint{2.768153in}{2.309428in}}%
\pgfpathlineto{\pgfqpoint{2.782083in}{2.294643in}}%
\pgfpathlineto{\pgfqpoint{2.773165in}{2.301274in}}%
\pgfpathlineto{\pgfqpoint{2.764226in}{2.308252in}}%
\pgfpathlineto{\pgfqpoint{2.755267in}{2.315581in}}%
\pgfpathlineto{\pgfqpoint{2.746287in}{2.323270in}}%
\pgfpathlineto{\pgfqpoint{2.732307in}{2.338627in}}%
\pgfpathlineto{\pgfqpoint{2.718323in}{2.354131in}}%
\pgfpathlineto{\pgfqpoint{2.704337in}{2.369782in}}%
\pgfpathlineto{\pgfqpoint{2.690346in}{2.385581in}}%
\pgfpathlineto{\pgfqpoint{2.699378in}{2.377311in}}%
\pgfpathlineto{\pgfqpoint{2.708388in}{2.369404in}}%
\pgfpathlineto{\pgfqpoint{2.717377in}{2.361856in}}%
\pgfpathlineto{\pgfqpoint{2.726345in}{2.354660in}}%
\pgfpathclose%
\pgfusepath{fill}%
\end{pgfscope}%
\begin{pgfscope}%
\pgfpathrectangle{\pgfqpoint{1.150000in}{0.150000in}}{\pgfqpoint{5.700000in}{5.700000in}}%
\pgfusepath{clip}%
\pgfsetbuttcap%
\pgfsetroundjoin%
\definecolor{currentfill}{rgb}{0.185556,0.418570,0.556753}%
\pgfsetfillcolor{currentfill}%
\pgfsetfillopacity{0.700000}%
\pgfsetlinewidth{0.000000pt}%
\definecolor{currentstroke}{rgb}{0.000000,0.000000,0.000000}%
\pgfsetstrokecolor{currentstroke}%
\pgfsetdash{}{0pt}%
\pgfpathmoveto{\pgfqpoint{2.558767in}{2.549262in}}%
\pgfpathlineto{\pgfqpoint{2.572756in}{2.532179in}}%
\pgfpathlineto{\pgfqpoint{2.586741in}{2.515258in}}%
\pgfpathlineto{\pgfqpoint{2.600721in}{2.498497in}}%
\pgfpathlineto{\pgfqpoint{2.614696in}{2.481896in}}%
\pgfpathlineto{\pgfqpoint{2.605624in}{2.490231in}}%
\pgfpathlineto{\pgfqpoint{2.596530in}{2.498930in}}%
\pgfpathlineto{\pgfqpoint{2.587414in}{2.508000in}}%
\pgfpathlineto{\pgfqpoint{2.578274in}{2.517445in}}%
\pgfpathlineto{\pgfqpoint{2.564244in}{2.534630in}}%
\pgfpathlineto{\pgfqpoint{2.550209in}{2.551974in}}%
\pgfpathlineto{\pgfqpoint{2.536169in}{2.569479in}}%
\pgfpathlineto{\pgfqpoint{2.522123in}{2.587147in}}%
\pgfpathlineto{\pgfqpoint{2.531319in}{2.577109in}}%
\pgfpathlineto{\pgfqpoint{2.540492in}{2.567452in}}%
\pgfpathlineto{\pgfqpoint{2.549641in}{2.558172in}}%
\pgfpathlineto{\pgfqpoint{2.558767in}{2.549262in}}%
\pgfpathclose%
\pgfusepath{fill}%
\end{pgfscope}%
\begin{pgfscope}%
\pgfpathrectangle{\pgfqpoint{1.150000in}{0.150000in}}{\pgfqpoint{5.700000in}{5.700000in}}%
\pgfusepath{clip}%
\pgfsetbuttcap%
\pgfsetroundjoin%
\definecolor{currentfill}{rgb}{0.271305,0.019942,0.347269}%
\pgfsetfillcolor{currentfill}%
\pgfsetfillopacity{0.700000}%
\pgfsetlinewidth{0.000000pt}%
\definecolor{currentstroke}{rgb}{0.000000,0.000000,0.000000}%
\pgfsetstrokecolor{currentstroke}%
\pgfsetdash{}{0pt}%
\pgfpathmoveto{\pgfqpoint{4.092629in}{1.677343in}}%
\pgfpathlineto{\pgfqpoint{4.106601in}{1.674932in}}%
\pgfpathlineto{\pgfqpoint{4.120580in}{1.672620in}}%
\pgfpathlineto{\pgfqpoint{4.134567in}{1.670409in}}%
\pgfpathlineto{\pgfqpoint{4.148561in}{1.668298in}}%
\pgfpathlineto{\pgfqpoint{4.140514in}{1.659336in}}%
\pgfpathlineto{\pgfqpoint{4.132460in}{1.650495in}}%
\pgfpathlineto{\pgfqpoint{4.124401in}{1.641778in}}%
\pgfpathlineto{\pgfqpoint{4.116337in}{1.633190in}}%
\pgfpathlineto{\pgfqpoint{4.102330in}{1.635731in}}%
\pgfpathlineto{\pgfqpoint{4.088332in}{1.638373in}}%
\pgfpathlineto{\pgfqpoint{4.074340in}{1.641114in}}%
\pgfpathlineto{\pgfqpoint{4.060356in}{1.643956in}}%
\pgfpathlineto{\pgfqpoint{4.068433in}{1.652107in}}%
\pgfpathlineto{\pgfqpoint{4.076505in}{1.660391in}}%
\pgfpathlineto{\pgfqpoint{4.084570in}{1.668805in}}%
\pgfpathlineto{\pgfqpoint{4.092629in}{1.677343in}}%
\pgfpathclose%
\pgfusepath{fill}%
\end{pgfscope}%
\begin{pgfscope}%
\pgfpathrectangle{\pgfqpoint{1.150000in}{0.150000in}}{\pgfqpoint{5.700000in}{5.700000in}}%
\pgfusepath{clip}%
\pgfsetbuttcap%
\pgfsetroundjoin%
\definecolor{currentfill}{rgb}{0.231674,0.318106,0.544834}%
\pgfsetfillcolor{currentfill}%
\pgfsetfillopacity{0.700000}%
\pgfsetlinewidth{0.000000pt}%
\definecolor{currentstroke}{rgb}{0.000000,0.000000,0.000000}%
\pgfsetstrokecolor{currentstroke}%
\pgfsetdash{}{0pt}%
\pgfpathmoveto{\pgfqpoint{2.782083in}{2.294643in}}%
\pgfpathlineto{\pgfqpoint{2.796010in}{2.280002in}}%
\pgfpathlineto{\pgfqpoint{2.809935in}{2.265505in}}%
\pgfpathlineto{\pgfqpoint{2.823857in}{2.251149in}}%
\pgfpathlineto{\pgfqpoint{2.837776in}{2.236936in}}%
\pgfpathlineto{\pgfqpoint{2.828906in}{2.243005in}}%
\pgfpathlineto{\pgfqpoint{2.820016in}{2.249415in}}%
\pgfpathlineto{\pgfqpoint{2.811106in}{2.256171in}}%
\pgfpathlineto{\pgfqpoint{2.802176in}{2.263279in}}%
\pgfpathlineto{\pgfqpoint{2.788208in}{2.278063in}}%
\pgfpathlineto{\pgfqpoint{2.774237in}{2.292988in}}%
\pgfpathlineto{\pgfqpoint{2.760263in}{2.308057in}}%
\pgfpathlineto{\pgfqpoint{2.746287in}{2.323270in}}%
\pgfpathlineto{\pgfqpoint{2.755267in}{2.315581in}}%
\pgfpathlineto{\pgfqpoint{2.764226in}{2.308252in}}%
\pgfpathlineto{\pgfqpoint{2.773165in}{2.301274in}}%
\pgfpathlineto{\pgfqpoint{2.782083in}{2.294643in}}%
\pgfpathclose%
\pgfusepath{fill}%
\end{pgfscope}%
\begin{pgfscope}%
\pgfpathrectangle{\pgfqpoint{1.150000in}{0.150000in}}{\pgfqpoint{5.700000in}{5.700000in}}%
\pgfusepath{clip}%
\pgfsetbuttcap%
\pgfsetroundjoin%
\definecolor{currentfill}{rgb}{0.270595,0.214069,0.507052}%
\pgfsetfillcolor{currentfill}%
\pgfsetfillopacity{0.700000}%
\pgfsetlinewidth{0.000000pt}%
\definecolor{currentstroke}{rgb}{0.000000,0.000000,0.000000}%
\pgfsetstrokecolor{currentstroke}%
\pgfsetdash{}{0pt}%
\pgfpathmoveto{\pgfqpoint{4.797474in}{2.055949in}}%
\pgfpathlineto{\pgfqpoint{4.811714in}{2.059512in}}%
\pgfpathlineto{\pgfqpoint{4.825967in}{2.063174in}}%
\pgfpathlineto{\pgfqpoint{4.840231in}{2.066935in}}%
\pgfpathlineto{\pgfqpoint{4.854508in}{2.070794in}}%
\pgfpathlineto{\pgfqpoint{4.846652in}{2.057632in}}%
\pgfpathlineto{\pgfqpoint{4.838790in}{2.044446in}}%
\pgfpathlineto{\pgfqpoint{4.830924in}{2.031239in}}%
\pgfpathlineto{\pgfqpoint{4.823054in}{2.018015in}}%
\pgfpathlineto{\pgfqpoint{4.808776in}{2.014447in}}%
\pgfpathlineto{\pgfqpoint{4.794511in}{2.010976in}}%
\pgfpathlineto{\pgfqpoint{4.780258in}{2.007605in}}%
\pgfpathlineto{\pgfqpoint{4.766017in}{2.004331in}}%
\pgfpathlineto{\pgfqpoint{4.773888in}{2.017258in}}%
\pgfpathlineto{\pgfqpoint{4.781754in}{2.030172in}}%
\pgfpathlineto{\pgfqpoint{4.789616in}{2.043070in}}%
\pgfpathlineto{\pgfqpoint{4.797474in}{2.055949in}}%
\pgfpathclose%
\pgfusepath{fill}%
\end{pgfscope}%
\begin{pgfscope}%
\pgfpathrectangle{\pgfqpoint{1.150000in}{0.150000in}}{\pgfqpoint{5.700000in}{5.700000in}}%
\pgfusepath{clip}%
\pgfsetbuttcap%
\pgfsetroundjoin%
\definecolor{currentfill}{rgb}{0.174274,0.445044,0.557792}%
\pgfsetfillcolor{currentfill}%
\pgfsetfillopacity{0.700000}%
\pgfsetlinewidth{0.000000pt}%
\definecolor{currentstroke}{rgb}{0.000000,0.000000,0.000000}%
\pgfsetstrokecolor{currentstroke}%
\pgfsetdash{}{0pt}%
\pgfpathmoveto{\pgfqpoint{2.502754in}{2.619235in}}%
\pgfpathlineto{\pgfqpoint{2.516765in}{2.601493in}}%
\pgfpathlineto{\pgfqpoint{2.530771in}{2.583917in}}%
\pgfpathlineto{\pgfqpoint{2.544772in}{2.566508in}}%
\pgfpathlineto{\pgfqpoint{2.558767in}{2.549262in}}%
\pgfpathlineto{\pgfqpoint{2.549641in}{2.558172in}}%
\pgfpathlineto{\pgfqpoint{2.540492in}{2.567452in}}%
\pgfpathlineto{\pgfqpoint{2.531319in}{2.577109in}}%
\pgfpathlineto{\pgfqpoint{2.522123in}{2.587147in}}%
\pgfpathlineto{\pgfqpoint{2.508072in}{2.604979in}}%
\pgfpathlineto{\pgfqpoint{2.494014in}{2.622976in}}%
\pgfpathlineto{\pgfqpoint{2.479951in}{2.641140in}}%
\pgfpathlineto{\pgfqpoint{2.465882in}{2.659471in}}%
\pgfpathlineto{\pgfqpoint{2.475137in}{2.648835in}}%
\pgfpathlineto{\pgfqpoint{2.484367in}{2.638588in}}%
\pgfpathlineto{\pgfqpoint{2.493572in}{2.628724in}}%
\pgfpathlineto{\pgfqpoint{2.502754in}{2.619235in}}%
\pgfpathclose%
\pgfusepath{fill}%
\end{pgfscope}%
\begin{pgfscope}%
\pgfpathrectangle{\pgfqpoint{1.150000in}{0.150000in}}{\pgfqpoint{5.700000in}{5.700000in}}%
\pgfusepath{clip}%
\pgfsetbuttcap%
\pgfsetroundjoin%
\definecolor{currentfill}{rgb}{0.278791,0.062145,0.386592}%
\pgfsetfillcolor{currentfill}%
\pgfsetfillopacity{0.700000}%
\pgfsetlinewidth{0.000000pt}%
\definecolor{currentstroke}{rgb}{0.000000,0.000000,0.000000}%
\pgfsetstrokecolor{currentstroke}%
\pgfsetdash{}{0pt}%
\pgfpathmoveto{\pgfqpoint{3.516768in}{1.748490in}}%
\pgfpathlineto{\pgfqpoint{3.530643in}{1.740890in}}%
\pgfpathlineto{\pgfqpoint{3.544522in}{1.733401in}}%
\pgfpathlineto{\pgfqpoint{3.558403in}{1.726022in}}%
\pgfpathlineto{\pgfqpoint{3.572289in}{1.718752in}}%
\pgfpathlineto{\pgfqpoint{3.563970in}{1.716233in}}%
\pgfpathlineto{\pgfqpoint{3.555641in}{1.713948in}}%
\pgfpathlineto{\pgfqpoint{3.547301in}{1.711899in}}%
\pgfpathlineto{\pgfqpoint{3.538951in}{1.710094in}}%
\pgfpathlineto{\pgfqpoint{3.525038in}{1.717871in}}%
\pgfpathlineto{\pgfqpoint{3.511129in}{1.725757in}}%
\pgfpathlineto{\pgfqpoint{3.497223in}{1.733754in}}%
\pgfpathlineto{\pgfqpoint{3.483320in}{1.741861in}}%
\pgfpathlineto{\pgfqpoint{3.491699in}{1.743152in}}%
\pgfpathlineto{\pgfqpoint{3.500066in}{1.744690in}}%
\pgfpathlineto{\pgfqpoint{3.508423in}{1.746471in}}%
\pgfpathlineto{\pgfqpoint{3.516768in}{1.748490in}}%
\pgfpathclose%
\pgfusepath{fill}%
\end{pgfscope}%
\begin{pgfscope}%
\pgfpathrectangle{\pgfqpoint{1.150000in}{0.150000in}}{\pgfqpoint{5.700000in}{5.700000in}}%
\pgfusepath{clip}%
\pgfsetbuttcap%
\pgfsetroundjoin%
\definecolor{currentfill}{rgb}{0.131172,0.555899,0.552459}%
\pgfsetfillcolor{currentfill}%
\pgfsetfillopacity{0.700000}%
\pgfsetlinewidth{0.000000pt}%
\definecolor{currentstroke}{rgb}{0.000000,0.000000,0.000000}%
\pgfsetstrokecolor{currentstroke}%
\pgfsetdash{}{0pt}%
\pgfpathmoveto{\pgfqpoint{5.719079in}{2.916722in}}%
\pgfpathlineto{\pgfqpoint{5.733830in}{2.925969in}}%
\pgfpathlineto{\pgfqpoint{5.748598in}{2.935318in}}%
\pgfpathlineto{\pgfqpoint{5.763383in}{2.944768in}}%
\pgfpathlineto{\pgfqpoint{5.778185in}{2.954320in}}%
\pgfpathlineto{\pgfqpoint{5.770686in}{2.944473in}}%
\pgfpathlineto{\pgfqpoint{5.763178in}{2.934487in}}%
\pgfpathlineto{\pgfqpoint{5.755660in}{2.924363in}}%
\pgfpathlineto{\pgfqpoint{5.748133in}{2.914102in}}%
\pgfpathlineto{\pgfqpoint{5.733326in}{2.904595in}}%
\pgfpathlineto{\pgfqpoint{5.718537in}{2.895189in}}%
\pgfpathlineto{\pgfqpoint{5.703764in}{2.885885in}}%
\pgfpathlineto{\pgfqpoint{5.689009in}{2.876683in}}%
\pgfpathlineto{\pgfqpoint{5.696540in}{2.886892in}}%
\pgfpathlineto{\pgfqpoint{5.704062in}{2.896969in}}%
\pgfpathlineto{\pgfqpoint{5.711575in}{2.906912in}}%
\pgfpathlineto{\pgfqpoint{5.719079in}{2.916722in}}%
\pgfpathclose%
\pgfusepath{fill}%
\end{pgfscope}%
\begin{pgfscope}%
\pgfpathrectangle{\pgfqpoint{1.150000in}{0.150000in}}{\pgfqpoint{5.700000in}{5.700000in}}%
\pgfusepath{clip}%
\pgfsetbuttcap%
\pgfsetroundjoin%
\definecolor{currentfill}{rgb}{0.241237,0.296485,0.539709}%
\pgfsetfillcolor{currentfill}%
\pgfsetfillopacity{0.700000}%
\pgfsetlinewidth{0.000000pt}%
\definecolor{currentstroke}{rgb}{0.000000,0.000000,0.000000}%
\pgfsetstrokecolor{currentstroke}%
\pgfsetdash{}{0pt}%
\pgfpathmoveto{\pgfqpoint{2.837776in}{2.236936in}}%
\pgfpathlineto{\pgfqpoint{2.851693in}{2.222862in}}%
\pgfpathlineto{\pgfqpoint{2.865608in}{2.208928in}}%
\pgfpathlineto{\pgfqpoint{2.879521in}{2.195133in}}%
\pgfpathlineto{\pgfqpoint{2.893433in}{2.181475in}}%
\pgfpathlineto{\pgfqpoint{2.884608in}{2.186985in}}%
\pgfpathlineto{\pgfqpoint{2.875766in}{2.192830in}}%
\pgfpathlineto{\pgfqpoint{2.866904in}{2.199015in}}%
\pgfpathlineto{\pgfqpoint{2.858022in}{2.205547in}}%
\pgfpathlineto{\pgfqpoint{2.844064in}{2.219772in}}%
\pgfpathlineto{\pgfqpoint{2.830104in}{2.234135in}}%
\pgfpathlineto{\pgfqpoint{2.816141in}{2.248637in}}%
\pgfpathlineto{\pgfqpoint{2.802176in}{2.263279in}}%
\pgfpathlineto{\pgfqpoint{2.811106in}{2.256171in}}%
\pgfpathlineto{\pgfqpoint{2.820016in}{2.249415in}}%
\pgfpathlineto{\pgfqpoint{2.828906in}{2.243005in}}%
\pgfpathlineto{\pgfqpoint{2.837776in}{2.236936in}}%
\pgfpathclose%
\pgfusepath{fill}%
\end{pgfscope}%
\begin{pgfscope}%
\pgfpathrectangle{\pgfqpoint{1.150000in}{0.150000in}}{\pgfqpoint{5.700000in}{5.700000in}}%
\pgfusepath{clip}%
\pgfsetbuttcap%
\pgfsetroundjoin%
\definecolor{currentfill}{rgb}{0.163625,0.471133,0.558148}%
\pgfsetfillcolor{currentfill}%
\pgfsetfillopacity{0.700000}%
\pgfsetlinewidth{0.000000pt}%
\definecolor{currentstroke}{rgb}{0.000000,0.000000,0.000000}%
\pgfsetstrokecolor{currentstroke}%
\pgfsetdash{}{0pt}%
\pgfpathmoveto{\pgfqpoint{2.446647in}{2.691902in}}%
\pgfpathlineto{\pgfqpoint{2.460683in}{2.673477in}}%
\pgfpathlineto{\pgfqpoint{2.474713in}{2.655226in}}%
\pgfpathlineto{\pgfqpoint{2.488736in}{2.637145in}}%
\pgfpathlineto{\pgfqpoint{2.502754in}{2.619235in}}%
\pgfpathlineto{\pgfqpoint{2.493572in}{2.628724in}}%
\pgfpathlineto{\pgfqpoint{2.484367in}{2.638588in}}%
\pgfpathlineto{\pgfqpoint{2.475137in}{2.648835in}}%
\pgfpathlineto{\pgfqpoint{2.465882in}{2.659471in}}%
\pgfpathlineto{\pgfqpoint{2.451807in}{2.677972in}}%
\pgfpathlineto{\pgfqpoint{2.437725in}{2.696644in}}%
\pgfpathlineto{\pgfqpoint{2.423636in}{2.715487in}}%
\pgfpathlineto{\pgfqpoint{2.409540in}{2.734505in}}%
\pgfpathlineto{\pgfqpoint{2.418855in}{2.723268in}}%
\pgfpathlineto{\pgfqpoint{2.428144in}{2.712426in}}%
\pgfpathlineto{\pgfqpoint{2.437408in}{2.701973in}}%
\pgfpathlineto{\pgfqpoint{2.446647in}{2.691902in}}%
\pgfpathclose%
\pgfusepath{fill}%
\end{pgfscope}%
\begin{pgfscope}%
\pgfpathrectangle{\pgfqpoint{1.150000in}{0.150000in}}{\pgfqpoint{5.700000in}{5.700000in}}%
\pgfusepath{clip}%
\pgfsetbuttcap%
\pgfsetroundjoin%
\definecolor{currentfill}{rgb}{0.269944,0.014625,0.341379}%
\pgfsetfillcolor{currentfill}%
\pgfsetfillopacity{0.700000}%
\pgfsetlinewidth{0.000000pt}%
\definecolor{currentstroke}{rgb}{0.000000,0.000000,0.000000}%
\pgfsetstrokecolor{currentstroke}%
\pgfsetdash{}{0pt}%
\pgfpathmoveto{\pgfqpoint{3.860512in}{1.660173in}}%
\pgfpathlineto{\pgfqpoint{3.874434in}{1.655660in}}%
\pgfpathlineto{\pgfqpoint{3.888362in}{1.651250in}}%
\pgfpathlineto{\pgfqpoint{3.902296in}{1.646944in}}%
\pgfpathlineto{\pgfqpoint{3.916236in}{1.642740in}}%
\pgfpathlineto{\pgfqpoint{3.908095in}{1.636242in}}%
\pgfpathlineto{\pgfqpoint{3.899946in}{1.629914in}}%
\pgfpathlineto{\pgfqpoint{3.891790in}{1.623760in}}%
\pgfpathlineto{\pgfqpoint{3.883627in}{1.617784in}}%
\pgfpathlineto{\pgfqpoint{3.869669in}{1.622455in}}%
\pgfpathlineto{\pgfqpoint{3.855717in}{1.627228in}}%
\pgfpathlineto{\pgfqpoint{3.841771in}{1.632105in}}%
\pgfpathlineto{\pgfqpoint{3.827830in}{1.637084in}}%
\pgfpathlineto{\pgfqpoint{3.836013in}{1.642586in}}%
\pgfpathlineto{\pgfqpoint{3.844187in}{1.648271in}}%
\pgfpathlineto{\pgfqpoint{3.852353in}{1.654135in}}%
\pgfpathlineto{\pgfqpoint{3.860512in}{1.660173in}}%
\pgfpathclose%
\pgfusepath{fill}%
\end{pgfscope}%
\begin{pgfscope}%
\pgfpathrectangle{\pgfqpoint{1.150000in}{0.150000in}}{\pgfqpoint{5.700000in}{5.700000in}}%
\pgfusepath{clip}%
\pgfsetbuttcap%
\pgfsetroundjoin%
\definecolor{currentfill}{rgb}{0.157729,0.485932,0.558013}%
\pgfsetfillcolor{currentfill}%
\pgfsetfillopacity{0.700000}%
\pgfsetlinewidth{0.000000pt}%
\definecolor{currentstroke}{rgb}{0.000000,0.000000,0.000000}%
\pgfsetstrokecolor{currentstroke}%
\pgfsetdash{}{0pt}%
\pgfpathmoveto{\pgfqpoint{5.510956in}{2.721496in}}%
\pgfpathlineto{\pgfqpoint{5.525586in}{2.729739in}}%
\pgfpathlineto{\pgfqpoint{5.540232in}{2.738082in}}%
\pgfpathlineto{\pgfqpoint{5.554894in}{2.746527in}}%
\pgfpathlineto{\pgfqpoint{5.569572in}{2.755072in}}%
\pgfpathlineto{\pgfqpoint{5.561962in}{2.743781in}}%
\pgfpathlineto{\pgfqpoint{5.554345in}{2.732367in}}%
\pgfpathlineto{\pgfqpoint{5.546719in}{2.720829in}}%
\pgfpathlineto{\pgfqpoint{5.539085in}{2.709170in}}%
\pgfpathlineto{\pgfqpoint{5.524405in}{2.700729in}}%
\pgfpathlineto{\pgfqpoint{5.509742in}{2.692389in}}%
\pgfpathlineto{\pgfqpoint{5.495094in}{2.684150in}}%
\pgfpathlineto{\pgfqpoint{5.480462in}{2.676011in}}%
\pgfpathlineto{\pgfqpoint{5.488097in}{2.687559in}}%
\pgfpathlineto{\pgfqpoint{5.495725in}{2.698989in}}%
\pgfpathlineto{\pgfqpoint{5.503344in}{2.710302in}}%
\pgfpathlineto{\pgfqpoint{5.510956in}{2.721496in}}%
\pgfpathclose%
\pgfusepath{fill}%
\end{pgfscope}%
\begin{pgfscope}%
\pgfpathrectangle{\pgfqpoint{1.150000in}{0.150000in}}{\pgfqpoint{5.700000in}{5.700000in}}%
\pgfusepath{clip}%
\pgfsetbuttcap%
\pgfsetroundjoin%
\definecolor{currentfill}{rgb}{0.188923,0.410910,0.556326}%
\pgfsetfillcolor{currentfill}%
\pgfsetfillopacity{0.700000}%
\pgfsetlinewidth{0.000000pt}%
\definecolor{currentstroke}{rgb}{0.000000,0.000000,0.000000}%
\pgfsetstrokecolor{currentstroke}%
\pgfsetdash{}{0pt}%
\pgfpathmoveto{\pgfqpoint{5.302609in}{2.520176in}}%
\pgfpathlineto{\pgfqpoint{5.317119in}{2.527244in}}%
\pgfpathlineto{\pgfqpoint{5.331644in}{2.534413in}}%
\pgfpathlineto{\pgfqpoint{5.346184in}{2.541682in}}%
\pgfpathlineto{\pgfqpoint{5.360739in}{2.549051in}}%
\pgfpathlineto{\pgfqpoint{5.353038in}{2.536649in}}%
\pgfpathlineto{\pgfqpoint{5.345330in}{2.524145in}}%
\pgfpathlineto{\pgfqpoint{5.337616in}{2.511541in}}%
\pgfpathlineto{\pgfqpoint{5.329895in}{2.498838in}}%
\pgfpathlineto{\pgfqpoint{5.315340in}{2.491631in}}%
\pgfpathlineto{\pgfqpoint{5.300800in}{2.484525in}}%
\pgfpathlineto{\pgfqpoint{5.286275in}{2.477518in}}%
\pgfpathlineto{\pgfqpoint{5.271764in}{2.470612in}}%
\pgfpathlineto{\pgfqpoint{5.279485in}{2.483145in}}%
\pgfpathlineto{\pgfqpoint{5.287200in}{2.495585in}}%
\pgfpathlineto{\pgfqpoint{5.294907in}{2.507929in}}%
\pgfpathlineto{\pgfqpoint{5.302609in}{2.520176in}}%
\pgfpathclose%
\pgfusepath{fill}%
\end{pgfscope}%
\begin{pgfscope}%
\pgfpathrectangle{\pgfqpoint{1.150000in}{0.150000in}}{\pgfqpoint{5.700000in}{5.700000in}}%
\pgfusepath{clip}%
\pgfsetbuttcap%
\pgfsetroundjoin%
\definecolor{currentfill}{rgb}{0.225863,0.330805,0.547314}%
\pgfsetfillcolor{currentfill}%
\pgfsetfillopacity{0.700000}%
\pgfsetlinewidth{0.000000pt}%
\definecolor{currentstroke}{rgb}{0.000000,0.000000,0.000000}%
\pgfsetstrokecolor{currentstroke}%
\pgfsetdash{}{0pt}%
\pgfpathmoveto{\pgfqpoint{5.094217in}{2.318512in}}%
\pgfpathlineto{\pgfqpoint{5.108612in}{2.324241in}}%
\pgfpathlineto{\pgfqpoint{5.123021in}{2.330070in}}%
\pgfpathlineto{\pgfqpoint{5.137444in}{2.335998in}}%
\pgfpathlineto{\pgfqpoint{5.151881in}{2.342026in}}%
\pgfpathlineto{\pgfqpoint{5.144106in}{2.328926in}}%
\pgfpathlineto{\pgfqpoint{5.136325in}{2.315752in}}%
\pgfpathlineto{\pgfqpoint{5.128538in}{2.302507in}}%
\pgfpathlineto{\pgfqpoint{5.120746in}{2.289193in}}%
\pgfpathlineto{\pgfqpoint{5.106309in}{2.283384in}}%
\pgfpathlineto{\pgfqpoint{5.091887in}{2.277674in}}%
\pgfpathlineto{\pgfqpoint{5.077479in}{2.272063in}}%
\pgfpathlineto{\pgfqpoint{5.063084in}{2.266551in}}%
\pgfpathlineto{\pgfqpoint{5.070876in}{2.279640in}}%
\pgfpathlineto{\pgfqpoint{5.078662in}{2.292665in}}%
\pgfpathlineto{\pgfqpoint{5.086442in}{2.305622in}}%
\pgfpathlineto{\pgfqpoint{5.094217in}{2.318512in}}%
\pgfpathclose%
\pgfusepath{fill}%
\end{pgfscope}%
\begin{pgfscope}%
\pgfpathrectangle{\pgfqpoint{1.150000in}{0.150000in}}{\pgfqpoint{5.700000in}{5.700000in}}%
\pgfusepath{clip}%
\pgfsetbuttcap%
\pgfsetroundjoin%
\definecolor{currentfill}{rgb}{0.272594,0.025563,0.353093}%
\pgfsetfillcolor{currentfill}%
\pgfsetfillopacity{0.700000}%
\pgfsetlinewidth{0.000000pt}%
\definecolor{currentstroke}{rgb}{0.000000,0.000000,0.000000}%
\pgfsetstrokecolor{currentstroke}%
\pgfsetdash{}{0pt}%
\pgfpathmoveto{\pgfqpoint{3.716495in}{1.680672in}}%
\pgfpathlineto{\pgfqpoint{3.730394in}{1.674856in}}%
\pgfpathlineto{\pgfqpoint{3.744298in}{1.669146in}}%
\pgfpathlineto{\pgfqpoint{3.758208in}{1.663541in}}%
\pgfpathlineto{\pgfqpoint{3.772122in}{1.658041in}}%
\pgfpathlineto{\pgfqpoint{3.763911in}{1.653206in}}%
\pgfpathlineto{\pgfqpoint{3.755692in}{1.648569in}}%
\pgfpathlineto{\pgfqpoint{3.747465in}{1.644134in}}%
\pgfpathlineto{\pgfqpoint{3.739228in}{1.639908in}}%
\pgfpathlineto{\pgfqpoint{3.725292in}{1.645894in}}%
\pgfpathlineto{\pgfqpoint{3.711361in}{1.651985in}}%
\pgfpathlineto{\pgfqpoint{3.697435in}{1.658182in}}%
\pgfpathlineto{\pgfqpoint{3.683513in}{1.664484in}}%
\pgfpathlineto{\pgfqpoint{3.691772in}{1.668217in}}%
\pgfpathlineto{\pgfqpoint{3.700022in}{1.672163in}}%
\pgfpathlineto{\pgfqpoint{3.708263in}{1.676317in}}%
\pgfpathlineto{\pgfqpoint{3.716495in}{1.680672in}}%
\pgfpathclose%
\pgfusepath{fill}%
\end{pgfscope}%
\begin{pgfscope}%
\pgfpathrectangle{\pgfqpoint{1.150000in}{0.150000in}}{\pgfqpoint{5.700000in}{5.700000in}}%
\pgfusepath{clip}%
\pgfsetbuttcap%
\pgfsetroundjoin%
\definecolor{currentfill}{rgb}{0.250425,0.274290,0.533103}%
\pgfsetfillcolor{currentfill}%
\pgfsetfillopacity{0.700000}%
\pgfsetlinewidth{0.000000pt}%
\definecolor{currentstroke}{rgb}{0.000000,0.000000,0.000000}%
\pgfsetstrokecolor{currentstroke}%
\pgfsetdash{}{0pt}%
\pgfpathmoveto{\pgfqpoint{2.893433in}{2.181475in}}%
\pgfpathlineto{\pgfqpoint{2.907342in}{2.167954in}}%
\pgfpathlineto{\pgfqpoint{2.921250in}{2.154569in}}%
\pgfpathlineto{\pgfqpoint{2.935156in}{2.141319in}}%
\pgfpathlineto{\pgfqpoint{2.949061in}{2.128203in}}%
\pgfpathlineto{\pgfqpoint{2.940282in}{2.133156in}}%
\pgfpathlineto{\pgfqpoint{2.931484in}{2.138439in}}%
\pgfpathlineto{\pgfqpoint{2.922669in}{2.144056in}}%
\pgfpathlineto{\pgfqpoint{2.913834in}{2.150014in}}%
\pgfpathlineto{\pgfqpoint{2.899884in}{2.163694in}}%
\pgfpathlineto{\pgfqpoint{2.885932in}{2.177509in}}%
\pgfpathlineto{\pgfqpoint{2.871978in}{2.191460in}}%
\pgfpathlineto{\pgfqpoint{2.858022in}{2.205547in}}%
\pgfpathlineto{\pgfqpoint{2.866904in}{2.199015in}}%
\pgfpathlineto{\pgfqpoint{2.875766in}{2.192830in}}%
\pgfpathlineto{\pgfqpoint{2.884608in}{2.186985in}}%
\pgfpathlineto{\pgfqpoint{2.893433in}{2.181475in}}%
\pgfpathclose%
\pgfusepath{fill}%
\end{pgfscope}%
\begin{pgfscope}%
\pgfpathrectangle{\pgfqpoint{1.150000in}{0.150000in}}{\pgfqpoint{5.700000in}{5.700000in}}%
\pgfusepath{clip}%
\pgfsetbuttcap%
\pgfsetroundjoin%
\definecolor{currentfill}{rgb}{0.153364,0.497000,0.557724}%
\pgfsetfillcolor{currentfill}%
\pgfsetfillopacity{0.700000}%
\pgfsetlinewidth{0.000000pt}%
\definecolor{currentstroke}{rgb}{0.000000,0.000000,0.000000}%
\pgfsetstrokecolor{currentstroke}%
\pgfsetdash{}{0pt}%
\pgfpathmoveto{\pgfqpoint{2.390435in}{2.767355in}}%
\pgfpathlineto{\pgfqpoint{2.404498in}{2.748225in}}%
\pgfpathlineto{\pgfqpoint{2.418555in}{2.729274in}}%
\pgfpathlineto{\pgfqpoint{2.432604in}{2.710500in}}%
\pgfpathlineto{\pgfqpoint{2.446647in}{2.691902in}}%
\pgfpathlineto{\pgfqpoint{2.437408in}{2.701973in}}%
\pgfpathlineto{\pgfqpoint{2.428144in}{2.712426in}}%
\pgfpathlineto{\pgfqpoint{2.418855in}{2.723268in}}%
\pgfpathlineto{\pgfqpoint{2.409540in}{2.734505in}}%
\pgfpathlineto{\pgfqpoint{2.395438in}{2.753698in}}%
\pgfpathlineto{\pgfqpoint{2.381328in}{2.773068in}}%
\pgfpathlineto{\pgfqpoint{2.367211in}{2.792616in}}%
\pgfpathlineto{\pgfqpoint{2.353086in}{2.812344in}}%
\pgfpathlineto{\pgfqpoint{2.362463in}{2.800501in}}%
\pgfpathlineto{\pgfqpoint{2.371813in}{2.789059in}}%
\pgfpathlineto{\pgfqpoint{2.381136in}{2.778013in}}%
\pgfpathlineto{\pgfqpoint{2.390435in}{2.767355in}}%
\pgfpathclose%
\pgfusepath{fill}%
\end{pgfscope}%
\begin{pgfscope}%
\pgfpathrectangle{\pgfqpoint{1.150000in}{0.150000in}}{\pgfqpoint{5.700000in}{5.700000in}}%
\pgfusepath{clip}%
\pgfsetbuttcap%
\pgfsetroundjoin%
\definecolor{currentfill}{rgb}{0.283229,0.120777,0.440584}%
\pgfsetfillcolor{currentfill}%
\pgfsetfillopacity{0.700000}%
\pgfsetlinewidth{0.000000pt}%
\definecolor{currentstroke}{rgb}{0.000000,0.000000,0.000000}%
\pgfsetstrokecolor{currentstroke}%
\pgfsetdash{}{0pt}%
\pgfpathmoveto{\pgfqpoint{3.316688in}{1.847925in}}%
\pgfpathlineto{\pgfqpoint{3.330561in}{1.838457in}}%
\pgfpathlineto{\pgfqpoint{3.344437in}{1.829105in}}%
\pgfpathlineto{\pgfqpoint{3.358315in}{1.819868in}}%
\pgfpathlineto{\pgfqpoint{3.372195in}{1.810747in}}%
\pgfpathlineto{\pgfqpoint{3.363742in}{1.810751in}}%
\pgfpathlineto{\pgfqpoint{3.355278in}{1.811025in}}%
\pgfpathlineto{\pgfqpoint{3.346800in}{1.811572in}}%
\pgfpathlineto{\pgfqpoint{3.338308in}{1.812399in}}%
\pgfpathlineto{\pgfqpoint{3.324396in}{1.822050in}}%
\pgfpathlineto{\pgfqpoint{3.310484in}{1.831817in}}%
\pgfpathlineto{\pgfqpoint{3.296575in}{1.841700in}}%
\pgfpathlineto{\pgfqpoint{3.282667in}{1.851700in}}%
\pgfpathlineto{\pgfqpoint{3.291193in}{1.850335in}}%
\pgfpathlineto{\pgfqpoint{3.299705in}{1.849254in}}%
\pgfpathlineto{\pgfqpoint{3.308203in}{1.848453in}}%
\pgfpathlineto{\pgfqpoint{3.316688in}{1.847925in}}%
\pgfpathclose%
\pgfusepath{fill}%
\end{pgfscope}%
\begin{pgfscope}%
\pgfpathrectangle{\pgfqpoint{1.150000in}{0.150000in}}{\pgfqpoint{5.700000in}{5.700000in}}%
\pgfusepath{clip}%
\pgfsetbuttcap%
\pgfsetroundjoin%
\definecolor{currentfill}{rgb}{0.260571,0.246922,0.522828}%
\pgfsetfillcolor{currentfill}%
\pgfsetfillopacity{0.700000}%
\pgfsetlinewidth{0.000000pt}%
\definecolor{currentstroke}{rgb}{0.000000,0.000000,0.000000}%
\pgfsetstrokecolor{currentstroke}%
\pgfsetdash{}{0pt}%
\pgfpathmoveto{\pgfqpoint{4.885888in}{2.123159in}}%
\pgfpathlineto{\pgfqpoint{4.900177in}{2.127388in}}%
\pgfpathlineto{\pgfqpoint{4.914479in}{2.131717in}}%
\pgfpathlineto{\pgfqpoint{4.928793in}{2.136144in}}%
\pgfpathlineto{\pgfqpoint{4.943121in}{2.140670in}}%
\pgfpathlineto{\pgfqpoint{4.935283in}{2.127361in}}%
\pgfpathlineto{\pgfqpoint{4.927441in}{2.114014in}}%
\pgfpathlineto{\pgfqpoint{4.919593in}{2.100631in}}%
\pgfpathlineto{\pgfqpoint{4.911741in}{2.087215in}}%
\pgfpathlineto{\pgfqpoint{4.897414in}{2.082962in}}%
\pgfpathlineto{\pgfqpoint{4.883100in}{2.078807in}}%
\pgfpathlineto{\pgfqpoint{4.868798in}{2.074751in}}%
\pgfpathlineto{\pgfqpoint{4.854508in}{2.070794in}}%
\pgfpathlineto{\pgfqpoint{4.862360in}{2.083930in}}%
\pgfpathlineto{\pgfqpoint{4.870208in}{2.097038in}}%
\pgfpathlineto{\pgfqpoint{4.878050in}{2.110115in}}%
\pgfpathlineto{\pgfqpoint{4.885888in}{2.123159in}}%
\pgfpathclose%
\pgfusepath{fill}%
\end{pgfscope}%
\begin{pgfscope}%
\pgfpathrectangle{\pgfqpoint{1.150000in}{0.150000in}}{\pgfqpoint{5.700000in}{5.700000in}}%
\pgfusepath{clip}%
\pgfsetbuttcap%
\pgfsetroundjoin%
\definecolor{currentfill}{rgb}{0.123463,0.581687,0.547445}%
\pgfsetfillcolor{currentfill}%
\pgfsetfillopacity{0.700000}%
\pgfsetlinewidth{0.000000pt}%
\definecolor{currentstroke}{rgb}{0.000000,0.000000,0.000000}%
\pgfsetstrokecolor{currentstroke}%
\pgfsetdash{}{0pt}%
\pgfpathmoveto{\pgfqpoint{5.808089in}{2.992329in}}%
\pgfpathlineto{\pgfqpoint{5.822903in}{3.002008in}}%
\pgfpathlineto{\pgfqpoint{5.837735in}{3.011789in}}%
\pgfpathlineto{\pgfqpoint{5.852584in}{3.021673in}}%
\pgfpathlineto{\pgfqpoint{5.845127in}{3.012365in}}%
\pgfpathlineto{\pgfqpoint{5.837660in}{3.002915in}}%
\pgfpathlineto{\pgfqpoint{5.830183in}{2.993323in}}%
\pgfpathlineto{\pgfqpoint{5.822697in}{2.983589in}}%
\pgfpathlineto{\pgfqpoint{5.807842in}{2.973731in}}%
\pgfpathlineto{\pgfqpoint{5.793005in}{2.963974in}}%
\pgfpathlineto{\pgfqpoint{5.778185in}{2.954320in}}%
\pgfpathlineto{\pgfqpoint{5.785675in}{2.964030in}}%
\pgfpathlineto{\pgfqpoint{5.793156in}{2.973601in}}%
\pgfpathlineto{\pgfqpoint{5.800627in}{2.983034in}}%
\pgfpathlineto{\pgfqpoint{5.808089in}{2.992329in}}%
\pgfpathclose%
\pgfusepath{fill}%
\end{pgfscope}%
\begin{pgfscope}%
\pgfpathrectangle{\pgfqpoint{1.150000in}{0.150000in}}{\pgfqpoint{5.700000in}{5.700000in}}%
\pgfusepath{clip}%
\pgfsetbuttcap%
\pgfsetroundjoin%
\definecolor{currentfill}{rgb}{0.281446,0.084320,0.407414}%
\pgfsetfillcolor{currentfill}%
\pgfsetfillopacity{0.700000}%
\pgfsetlinewidth{0.000000pt}%
\definecolor{currentstroke}{rgb}{0.000000,0.000000,0.000000}%
\pgfsetstrokecolor{currentstroke}%
\pgfsetdash{}{0pt}%
\pgfpathmoveto{\pgfqpoint{4.412964in}{1.780431in}}%
\pgfpathlineto{\pgfqpoint{4.427046in}{1.780795in}}%
\pgfpathlineto{\pgfqpoint{4.441137in}{1.781257in}}%
\pgfpathlineto{\pgfqpoint{4.455239in}{1.781818in}}%
\pgfpathlineto{\pgfqpoint{4.469349in}{1.782477in}}%
\pgfpathlineto{\pgfqpoint{4.461394in}{1.770894in}}%
\pgfpathlineto{\pgfqpoint{4.453434in}{1.759368in}}%
\pgfpathlineto{\pgfqpoint{4.445470in}{1.747901in}}%
\pgfpathlineto{\pgfqpoint{4.437501in}{1.736498in}}%
\pgfpathlineto{\pgfqpoint{4.423385in}{1.736217in}}%
\pgfpathlineto{\pgfqpoint{4.409278in}{1.736035in}}%
\pgfpathlineto{\pgfqpoint{4.395181in}{1.735950in}}%
\pgfpathlineto{\pgfqpoint{4.381093in}{1.735964in}}%
\pgfpathlineto{\pgfqpoint{4.389068in}{1.746982in}}%
\pgfpathlineto{\pgfqpoint{4.397038in}{1.758069in}}%
\pgfpathlineto{\pgfqpoint{4.405004in}{1.769219in}}%
\pgfpathlineto{\pgfqpoint{4.412964in}{1.780431in}}%
\pgfpathclose%
\pgfusepath{fill}%
\end{pgfscope}%
\begin{pgfscope}%
\pgfpathrectangle{\pgfqpoint{1.150000in}{0.150000in}}{\pgfqpoint{5.700000in}{5.700000in}}%
\pgfusepath{clip}%
\pgfsetbuttcap%
\pgfsetroundjoin%
\definecolor{currentfill}{rgb}{0.269944,0.014625,0.341379}%
\pgfsetfillcolor{currentfill}%
\pgfsetfillopacity{0.700000}%
\pgfsetlinewidth{0.000000pt}%
\definecolor{currentstroke}{rgb}{0.000000,0.000000,0.000000}%
\pgfsetstrokecolor{currentstroke}%
\pgfsetdash{}{0pt}%
\pgfpathmoveto{\pgfqpoint{4.004488in}{1.656333in}}%
\pgfpathlineto{\pgfqpoint{4.018445in}{1.653087in}}%
\pgfpathlineto{\pgfqpoint{4.032408in}{1.649942in}}%
\pgfpathlineto{\pgfqpoint{4.046378in}{1.646899in}}%
\pgfpathlineto{\pgfqpoint{4.060356in}{1.643956in}}%
\pgfpathlineto{\pgfqpoint{4.052272in}{1.635944in}}%
\pgfpathlineto{\pgfqpoint{4.044182in}{1.628073in}}%
\pgfpathlineto{\pgfqpoint{4.036085in}{1.620349in}}%
\pgfpathlineto{\pgfqpoint{4.027982in}{1.612776in}}%
\pgfpathlineto{\pgfqpoint{4.013991in}{1.616167in}}%
\pgfpathlineto{\pgfqpoint{4.000007in}{1.619659in}}%
\pgfpathlineto{\pgfqpoint{3.986029in}{1.623252in}}%
\pgfpathlineto{\pgfqpoint{3.972057in}{1.626946in}}%
\pgfpathlineto{\pgfqpoint{3.980175in}{1.634064in}}%
\pgfpathlineto{\pgfqpoint{3.988286in}{1.641337in}}%
\pgfpathlineto{\pgfqpoint{3.996391in}{1.648762in}}%
\pgfpathlineto{\pgfqpoint{4.004488in}{1.656333in}}%
\pgfpathclose%
\pgfusepath{fill}%
\end{pgfscope}%
\begin{pgfscope}%
\pgfpathrectangle{\pgfqpoint{1.150000in}{0.150000in}}{\pgfqpoint{5.700000in}{5.700000in}}%
\pgfusepath{clip}%
\pgfsetbuttcap%
\pgfsetroundjoin%
\definecolor{currentfill}{rgb}{0.283091,0.110553,0.431554}%
\pgfsetfillcolor{currentfill}%
\pgfsetfillopacity{0.700000}%
\pgfsetlinewidth{0.000000pt}%
\definecolor{currentstroke}{rgb}{0.000000,0.000000,0.000000}%
\pgfsetstrokecolor{currentstroke}%
\pgfsetdash{}{0pt}%
\pgfpathmoveto{\pgfqpoint{4.501126in}{1.829301in}}%
\pgfpathlineto{\pgfqpoint{4.515242in}{1.830419in}}%
\pgfpathlineto{\pgfqpoint{4.529368in}{1.831635in}}%
\pgfpathlineto{\pgfqpoint{4.543505in}{1.832950in}}%
\pgfpathlineto{\pgfqpoint{4.557653in}{1.834362in}}%
\pgfpathlineto{\pgfqpoint{4.549719in}{1.822235in}}%
\pgfpathlineto{\pgfqpoint{4.541781in}{1.810146in}}%
\pgfpathlineto{\pgfqpoint{4.533839in}{1.798098in}}%
\pgfpathlineto{\pgfqpoint{4.525893in}{1.786095in}}%
\pgfpathlineto{\pgfqpoint{4.511742in}{1.785044in}}%
\pgfpathlineto{\pgfqpoint{4.497601in}{1.784090in}}%
\pgfpathlineto{\pgfqpoint{4.483470in}{1.783234in}}%
\pgfpathlineto{\pgfqpoint{4.469349in}{1.782477in}}%
\pgfpathlineto{\pgfqpoint{4.477300in}{1.794112in}}%
\pgfpathlineto{\pgfqpoint{4.485247in}{1.805797in}}%
\pgfpathlineto{\pgfqpoint{4.493188in}{1.817528in}}%
\pgfpathlineto{\pgfqpoint{4.501126in}{1.829301in}}%
\pgfpathclose%
\pgfusepath{fill}%
\end{pgfscope}%
\begin{pgfscope}%
\pgfpathrectangle{\pgfqpoint{1.150000in}{0.150000in}}{\pgfqpoint{5.700000in}{5.700000in}}%
\pgfusepath{clip}%
\pgfsetbuttcap%
\pgfsetroundjoin%
\definecolor{currentfill}{rgb}{0.278791,0.062145,0.386592}%
\pgfsetfillcolor{currentfill}%
\pgfsetfillopacity{0.700000}%
\pgfsetlinewidth{0.000000pt}%
\definecolor{currentstroke}{rgb}{0.000000,0.000000,0.000000}%
\pgfsetstrokecolor{currentstroke}%
\pgfsetdash{}{0pt}%
\pgfpathmoveto{\pgfqpoint{4.324835in}{1.737006in}}%
\pgfpathlineto{\pgfqpoint{4.338886in}{1.736598in}}%
\pgfpathlineto{\pgfqpoint{4.352946in}{1.736288in}}%
\pgfpathlineto{\pgfqpoint{4.367015in}{1.736077in}}%
\pgfpathlineto{\pgfqpoint{4.381093in}{1.735964in}}%
\pgfpathlineto{\pgfqpoint{4.373114in}{1.725018in}}%
\pgfpathlineto{\pgfqpoint{4.365129in}{1.714147in}}%
\pgfpathlineto{\pgfqpoint{4.357140in}{1.703355in}}%
\pgfpathlineto{\pgfqpoint{4.349146in}{1.692646in}}%
\pgfpathlineto{\pgfqpoint{4.335061in}{1.693154in}}%
\pgfpathlineto{\pgfqpoint{4.320985in}{1.693760in}}%
\pgfpathlineto{\pgfqpoint{4.306917in}{1.694466in}}%
\pgfpathlineto{\pgfqpoint{4.292859in}{1.695269in}}%
\pgfpathlineto{\pgfqpoint{4.300860in}{1.705576in}}%
\pgfpathlineto{\pgfqpoint{4.308857in}{1.715970in}}%
\pgfpathlineto{\pgfqpoint{4.316848in}{1.726448in}}%
\pgfpathlineto{\pgfqpoint{4.324835in}{1.737006in}}%
\pgfpathclose%
\pgfusepath{fill}%
\end{pgfscope}%
\begin{pgfscope}%
\pgfpathrectangle{\pgfqpoint{1.150000in}{0.150000in}}{\pgfqpoint{5.700000in}{5.700000in}}%
\pgfusepath{clip}%
\pgfsetbuttcap%
\pgfsetroundjoin%
\definecolor{currentfill}{rgb}{0.258965,0.251537,0.524736}%
\pgfsetfillcolor{currentfill}%
\pgfsetfillopacity{0.700000}%
\pgfsetlinewidth{0.000000pt}%
\definecolor{currentstroke}{rgb}{0.000000,0.000000,0.000000}%
\pgfsetstrokecolor{currentstroke}%
\pgfsetdash{}{0pt}%
\pgfpathmoveto{\pgfqpoint{2.949061in}{2.128203in}}%
\pgfpathlineto{\pgfqpoint{2.962964in}{2.115221in}}%
\pgfpathlineto{\pgfqpoint{2.976867in}{2.102371in}}%
\pgfpathlineto{\pgfqpoint{2.990768in}{2.089652in}}%
\pgfpathlineto{\pgfqpoint{3.004669in}{2.077065in}}%
\pgfpathlineto{\pgfqpoint{2.995933in}{2.081464in}}%
\pgfpathlineto{\pgfqpoint{2.987180in}{2.086186in}}%
\pgfpathlineto{\pgfqpoint{2.978409in}{2.091237in}}%
\pgfpathlineto{\pgfqpoint{2.969620in}{2.096624in}}%
\pgfpathlineto{\pgfqpoint{2.955676in}{2.109774in}}%
\pgfpathlineto{\pgfqpoint{2.941730in}{2.123055in}}%
\pgfpathlineto{\pgfqpoint{2.927783in}{2.136468in}}%
\pgfpathlineto{\pgfqpoint{2.913834in}{2.150014in}}%
\pgfpathlineto{\pgfqpoint{2.922669in}{2.144056in}}%
\pgfpathlineto{\pgfqpoint{2.931484in}{2.138439in}}%
\pgfpathlineto{\pgfqpoint{2.940282in}{2.133156in}}%
\pgfpathlineto{\pgfqpoint{2.949061in}{2.128203in}}%
\pgfpathclose%
\pgfusepath{fill}%
\end{pgfscope}%
\begin{pgfscope}%
\pgfpathrectangle{\pgfqpoint{1.150000in}{0.150000in}}{\pgfqpoint{5.700000in}{5.700000in}}%
\pgfusepath{clip}%
\pgfsetbuttcap%
\pgfsetroundjoin%
\definecolor{currentfill}{rgb}{0.282884,0.135920,0.453427}%
\pgfsetfillcolor{currentfill}%
\pgfsetfillopacity{0.700000}%
\pgfsetlinewidth{0.000000pt}%
\definecolor{currentstroke}{rgb}{0.000000,0.000000,0.000000}%
\pgfsetstrokecolor{currentstroke}%
\pgfsetdash{}{0pt}%
\pgfpathmoveto{\pgfqpoint{4.589341in}{1.883185in}}%
\pgfpathlineto{\pgfqpoint{4.603496in}{1.885039in}}%
\pgfpathlineto{\pgfqpoint{4.617662in}{1.886991in}}%
\pgfpathlineto{\pgfqpoint{4.631838in}{1.889042in}}%
\pgfpathlineto{\pgfqpoint{4.646025in}{1.891190in}}%
\pgfpathlineto{\pgfqpoint{4.638112in}{1.878607in}}%
\pgfpathlineto{\pgfqpoint{4.630195in}{1.866045in}}%
\pgfpathlineto{\pgfqpoint{4.622273in}{1.853506in}}%
\pgfpathlineto{\pgfqpoint{4.614347in}{1.840994in}}%
\pgfpathlineto{\pgfqpoint{4.600157in}{1.839189in}}%
\pgfpathlineto{\pgfqpoint{4.585978in}{1.837482in}}%
\pgfpathlineto{\pgfqpoint{4.571810in}{1.835873in}}%
\pgfpathlineto{\pgfqpoint{4.557653in}{1.834362in}}%
\pgfpathlineto{\pgfqpoint{4.565581in}{1.846524in}}%
\pgfpathlineto{\pgfqpoint{4.573506in}{1.858717in}}%
\pgfpathlineto{\pgfqpoint{4.581426in}{1.870939in}}%
\pgfpathlineto{\pgfqpoint{4.589341in}{1.883185in}}%
\pgfpathclose%
\pgfusepath{fill}%
\end{pgfscope}%
\begin{pgfscope}%
\pgfpathrectangle{\pgfqpoint{1.150000in}{0.150000in}}{\pgfqpoint{5.700000in}{5.700000in}}%
\pgfusepath{clip}%
\pgfsetbuttcap%
\pgfsetroundjoin%
\definecolor{currentfill}{rgb}{0.141935,0.526453,0.555991}%
\pgfsetfillcolor{currentfill}%
\pgfsetfillopacity{0.700000}%
\pgfsetlinewidth{0.000000pt}%
\definecolor{currentstroke}{rgb}{0.000000,0.000000,0.000000}%
\pgfsetstrokecolor{currentstroke}%
\pgfsetdash{}{0pt}%
\pgfpathmoveto{\pgfqpoint{2.334106in}{2.845696in}}%
\pgfpathlineto{\pgfqpoint{2.348200in}{2.825835in}}%
\pgfpathlineto{\pgfqpoint{2.362286in}{2.806158in}}%
\pgfpathlineto{\pgfqpoint{2.376364in}{2.786666in}}%
\pgfpathlineto{\pgfqpoint{2.390435in}{2.767355in}}%
\pgfpathlineto{\pgfqpoint{2.381136in}{2.778013in}}%
\pgfpathlineto{\pgfqpoint{2.371813in}{2.789059in}}%
\pgfpathlineto{\pgfqpoint{2.362463in}{2.800501in}}%
\pgfpathlineto{\pgfqpoint{2.353086in}{2.812344in}}%
\pgfpathlineto{\pgfqpoint{2.338954in}{2.832253in}}%
\pgfpathlineto{\pgfqpoint{2.324814in}{2.852346in}}%
\pgfpathlineto{\pgfqpoint{2.310665in}{2.872623in}}%
\pgfpathlineto{\pgfqpoint{2.296509in}{2.893088in}}%
\pgfpathlineto{\pgfqpoint{2.305948in}{2.880634in}}%
\pgfpathlineto{\pgfqpoint{2.315361in}{2.868589in}}%
\pgfpathlineto{\pgfqpoint{2.324747in}{2.856945in}}%
\pgfpathlineto{\pgfqpoint{2.334106in}{2.845696in}}%
\pgfpathclose%
\pgfusepath{fill}%
\end{pgfscope}%
\begin{pgfscope}%
\pgfpathrectangle{\pgfqpoint{1.150000in}{0.150000in}}{\pgfqpoint{5.700000in}{5.700000in}}%
\pgfusepath{clip}%
\pgfsetbuttcap%
\pgfsetroundjoin%
\definecolor{currentfill}{rgb}{0.274952,0.037752,0.364543}%
\pgfsetfillcolor{currentfill}%
\pgfsetfillopacity{0.700000}%
\pgfsetlinewidth{0.000000pt}%
\definecolor{currentstroke}{rgb}{0.000000,0.000000,0.000000}%
\pgfsetstrokecolor{currentstroke}%
\pgfsetdash{}{0pt}%
\pgfpathmoveto{\pgfqpoint{4.236710in}{1.699475in}}%
\pgfpathlineto{\pgfqpoint{4.250734in}{1.698274in}}%
\pgfpathlineto{\pgfqpoint{4.264767in}{1.697174in}}%
\pgfpathlineto{\pgfqpoint{4.278809in}{1.696172in}}%
\pgfpathlineto{\pgfqpoint{4.292859in}{1.695269in}}%
\pgfpathlineto{\pgfqpoint{4.284852in}{1.685054in}}%
\pgfpathlineto{\pgfqpoint{4.276840in}{1.674934in}}%
\pgfpathlineto{\pgfqpoint{4.268824in}{1.664913in}}%
\pgfpathlineto{\pgfqpoint{4.260802in}{1.654996in}}%
\pgfpathlineto{\pgfqpoint{4.246743in}{1.656312in}}%
\pgfpathlineto{\pgfqpoint{4.232693in}{1.657726in}}%
\pgfpathlineto{\pgfqpoint{4.218651in}{1.659240in}}%
\pgfpathlineto{\pgfqpoint{4.204617in}{1.660853in}}%
\pgfpathlineto{\pgfqpoint{4.212648in}{1.670350in}}%
\pgfpathlineto{\pgfqpoint{4.220674in}{1.679956in}}%
\pgfpathlineto{\pgfqpoint{4.228695in}{1.689665in}}%
\pgfpathlineto{\pgfqpoint{4.236710in}{1.699475in}}%
\pgfpathclose%
\pgfusepath{fill}%
\end{pgfscope}%
\begin{pgfscope}%
\pgfpathrectangle{\pgfqpoint{1.150000in}{0.150000in}}{\pgfqpoint{5.700000in}{5.700000in}}%
\pgfusepath{clip}%
\pgfsetbuttcap%
\pgfsetroundjoin%
\definecolor{currentfill}{rgb}{0.277018,0.050344,0.375715}%
\pgfsetfillcolor{currentfill}%
\pgfsetfillopacity{0.700000}%
\pgfsetlinewidth{0.000000pt}%
\definecolor{currentstroke}{rgb}{0.000000,0.000000,0.000000}%
\pgfsetstrokecolor{currentstroke}%
\pgfsetdash{}{0pt}%
\pgfpathmoveto{\pgfqpoint{3.572289in}{1.718752in}}%
\pgfpathlineto{\pgfqpoint{3.586178in}{1.711591in}}%
\pgfpathlineto{\pgfqpoint{3.600071in}{1.704538in}}%
\pgfpathlineto{\pgfqpoint{3.613968in}{1.697594in}}%
\pgfpathlineto{\pgfqpoint{3.627868in}{1.690758in}}%
\pgfpathlineto{\pgfqpoint{3.619575in}{1.687740in}}%
\pgfpathlineto{\pgfqpoint{3.611272in}{1.684951in}}%
\pgfpathlineto{\pgfqpoint{3.602959in}{1.682393in}}%
\pgfpathlineto{\pgfqpoint{3.594635in}{1.680073in}}%
\pgfpathlineto{\pgfqpoint{3.580709in}{1.687416in}}%
\pgfpathlineto{\pgfqpoint{3.566786in}{1.694867in}}%
\pgfpathlineto{\pgfqpoint{3.552866in}{1.702426in}}%
\pgfpathlineto{\pgfqpoint{3.538951in}{1.710094in}}%
\pgfpathlineto{\pgfqpoint{3.547301in}{1.711899in}}%
\pgfpathlineto{\pgfqpoint{3.555641in}{1.713948in}}%
\pgfpathlineto{\pgfqpoint{3.563970in}{1.716233in}}%
\pgfpathlineto{\pgfqpoint{3.572289in}{1.718752in}}%
\pgfpathclose%
\pgfusepath{fill}%
\end{pgfscope}%
\begin{pgfscope}%
\pgfpathrectangle{\pgfqpoint{1.150000in}{0.150000in}}{\pgfqpoint{5.700000in}{5.700000in}}%
\pgfusepath{clip}%
\pgfsetbuttcap%
\pgfsetroundjoin%
\definecolor{currentfill}{rgb}{0.265145,0.232956,0.516599}%
\pgfsetfillcolor{currentfill}%
\pgfsetfillopacity{0.700000}%
\pgfsetlinewidth{0.000000pt}%
\definecolor{currentstroke}{rgb}{0.000000,0.000000,0.000000}%
\pgfsetstrokecolor{currentstroke}%
\pgfsetdash{}{0pt}%
\pgfpathmoveto{\pgfqpoint{3.004669in}{2.077065in}}%
\pgfpathlineto{\pgfqpoint{3.018568in}{2.064608in}}%
\pgfpathlineto{\pgfqpoint{3.032467in}{2.052280in}}%
\pgfpathlineto{\pgfqpoint{3.046366in}{2.040081in}}%
\pgfpathlineto{\pgfqpoint{3.060264in}{2.028009in}}%
\pgfpathlineto{\pgfqpoint{3.051570in}{2.031856in}}%
\pgfpathlineto{\pgfqpoint{3.042860in}{2.036020in}}%
\pgfpathlineto{\pgfqpoint{3.034132in}{2.040508in}}%
\pgfpathlineto{\pgfqpoint{3.025388in}{2.045326in}}%
\pgfpathlineto{\pgfqpoint{3.011447in}{2.057957in}}%
\pgfpathlineto{\pgfqpoint{2.997506in}{2.070717in}}%
\pgfpathlineto{\pgfqpoint{2.983563in}{2.083606in}}%
\pgfpathlineto{\pgfqpoint{2.969620in}{2.096624in}}%
\pgfpathlineto{\pgfqpoint{2.978409in}{2.091237in}}%
\pgfpathlineto{\pgfqpoint{2.987180in}{2.086186in}}%
\pgfpathlineto{\pgfqpoint{2.995933in}{2.081464in}}%
\pgfpathlineto{\pgfqpoint{3.004669in}{2.077065in}}%
\pgfpathclose%
\pgfusepath{fill}%
\end{pgfscope}%
\begin{pgfscope}%
\pgfpathrectangle{\pgfqpoint{1.150000in}{0.150000in}}{\pgfqpoint{5.700000in}{5.700000in}}%
\pgfusepath{clip}%
\pgfsetbuttcap%
\pgfsetroundjoin%
\definecolor{currentfill}{rgb}{0.210503,0.363727,0.552206}%
\pgfsetfillcolor{currentfill}%
\pgfsetfillopacity{0.700000}%
\pgfsetlinewidth{0.000000pt}%
\definecolor{currentstroke}{rgb}{0.000000,0.000000,0.000000}%
\pgfsetstrokecolor{currentstroke}%
\pgfsetdash{}{0pt}%
\pgfpathmoveto{\pgfqpoint{5.182924in}{2.393659in}}%
\pgfpathlineto{\pgfqpoint{5.197376in}{2.399986in}}%
\pgfpathlineto{\pgfqpoint{5.211842in}{2.406412in}}%
\pgfpathlineto{\pgfqpoint{5.226322in}{2.412938in}}%
\pgfpathlineto{\pgfqpoint{5.240817in}{2.419564in}}%
\pgfpathlineto{\pgfqpoint{5.233065in}{2.406582in}}%
\pgfpathlineto{\pgfqpoint{5.225306in}{2.393514in}}%
\pgfpathlineto{\pgfqpoint{5.217542in}{2.380364in}}%
\pgfpathlineto{\pgfqpoint{5.209771in}{2.367131in}}%
\pgfpathlineto{\pgfqpoint{5.195277in}{2.360706in}}%
\pgfpathlineto{\pgfqpoint{5.180798in}{2.354380in}}%
\pgfpathlineto{\pgfqpoint{5.166332in}{2.348153in}}%
\pgfpathlineto{\pgfqpoint{5.151881in}{2.342026in}}%
\pgfpathlineto{\pgfqpoint{5.159651in}{2.355051in}}%
\pgfpathlineto{\pgfqpoint{5.167414in}{2.368000in}}%
\pgfpathlineto{\pgfqpoint{5.175172in}{2.380870in}}%
\pgfpathlineto{\pgfqpoint{5.182924in}{2.393659in}}%
\pgfpathclose%
\pgfusepath{fill}%
\end{pgfscope}%
\begin{pgfscope}%
\pgfpathrectangle{\pgfqpoint{1.150000in}{0.150000in}}{\pgfqpoint{5.700000in}{5.700000in}}%
\pgfusepath{clip}%
\pgfsetbuttcap%
\pgfsetroundjoin%
\definecolor{currentfill}{rgb}{0.146180,0.515413,0.556823}%
\pgfsetfillcolor{currentfill}%
\pgfsetfillopacity{0.700000}%
\pgfsetlinewidth{0.000000pt}%
\definecolor{currentstroke}{rgb}{0.000000,0.000000,0.000000}%
\pgfsetstrokecolor{currentstroke}%
\pgfsetdash{}{0pt}%
\pgfpathmoveto{\pgfqpoint{5.599930in}{2.798991in}}%
\pgfpathlineto{\pgfqpoint{5.614622in}{2.807722in}}%
\pgfpathlineto{\pgfqpoint{5.629331in}{2.816555in}}%
\pgfpathlineto{\pgfqpoint{5.644056in}{2.825489in}}%
\pgfpathlineto{\pgfqpoint{5.658798in}{2.834524in}}%
\pgfpathlineto{\pgfqpoint{5.651224in}{2.823656in}}%
\pgfpathlineto{\pgfqpoint{5.643641in}{2.812656in}}%
\pgfpathlineto{\pgfqpoint{5.636049in}{2.801527in}}%
\pgfpathlineto{\pgfqpoint{5.628450in}{2.790267in}}%
\pgfpathlineto{\pgfqpoint{5.613706in}{2.781316in}}%
\pgfpathlineto{\pgfqpoint{5.598978in}{2.772467in}}%
\pgfpathlineto{\pgfqpoint{5.584267in}{2.763719in}}%
\pgfpathlineto{\pgfqpoint{5.569572in}{2.755072in}}%
\pgfpathlineto{\pgfqpoint{5.577174in}{2.766239in}}%
\pgfpathlineto{\pgfqpoint{5.584768in}{2.777282in}}%
\pgfpathlineto{\pgfqpoint{5.592353in}{2.788199in}}%
\pgfpathlineto{\pgfqpoint{5.599930in}{2.798991in}}%
\pgfpathclose%
\pgfusepath{fill}%
\end{pgfscope}%
\begin{pgfscope}%
\pgfpathrectangle{\pgfqpoint{1.150000in}{0.150000in}}{\pgfqpoint{5.700000in}{5.700000in}}%
\pgfusepath{clip}%
\pgfsetbuttcap%
\pgfsetroundjoin%
\definecolor{currentfill}{rgb}{0.280255,0.165693,0.476498}%
\pgfsetfillcolor{currentfill}%
\pgfsetfillopacity{0.700000}%
\pgfsetlinewidth{0.000000pt}%
\definecolor{currentstroke}{rgb}{0.000000,0.000000,0.000000}%
\pgfsetstrokecolor{currentstroke}%
\pgfsetdash{}{0pt}%
\pgfpathmoveto{\pgfqpoint{4.677633in}{1.941664in}}%
\pgfpathlineto{\pgfqpoint{4.691829in}{1.944236in}}%
\pgfpathlineto{\pgfqpoint{4.706037in}{1.946907in}}%
\pgfpathlineto{\pgfqpoint{4.720256in}{1.949675in}}%
\pgfpathlineto{\pgfqpoint{4.734486in}{1.952542in}}%
\pgfpathlineto{\pgfqpoint{4.726593in}{1.939590in}}%
\pgfpathlineto{\pgfqpoint{4.718694in}{1.926640in}}%
\pgfpathlineto{\pgfqpoint{4.710792in}{1.913698in}}%
\pgfpathlineto{\pgfqpoint{4.702885in}{1.900764in}}%
\pgfpathlineto{\pgfqpoint{4.688653in}{1.898224in}}%
\pgfpathlineto{\pgfqpoint{4.674433in}{1.895781in}}%
\pgfpathlineto{\pgfqpoint{4.660223in}{1.893437in}}%
\pgfpathlineto{\pgfqpoint{4.646025in}{1.891190in}}%
\pgfpathlineto{\pgfqpoint{4.653934in}{1.903790in}}%
\pgfpathlineto{\pgfqpoint{4.661838in}{1.916405in}}%
\pgfpathlineto{\pgfqpoint{4.669737in}{1.929030in}}%
\pgfpathlineto{\pgfqpoint{4.677633in}{1.941664in}}%
\pgfpathclose%
\pgfusepath{fill}%
\end{pgfscope}%
\begin{pgfscope}%
\pgfpathrectangle{\pgfqpoint{1.150000in}{0.150000in}}{\pgfqpoint{5.700000in}{5.700000in}}%
\pgfusepath{clip}%
\pgfsetbuttcap%
\pgfsetroundjoin%
\definecolor{currentfill}{rgb}{0.174274,0.445044,0.557792}%
\pgfsetfillcolor{currentfill}%
\pgfsetfillopacity{0.700000}%
\pgfsetlinewidth{0.000000pt}%
\definecolor{currentstroke}{rgb}{0.000000,0.000000,0.000000}%
\pgfsetstrokecolor{currentstroke}%
\pgfsetdash{}{0pt}%
\pgfpathmoveto{\pgfqpoint{5.391473in}{2.597617in}}%
\pgfpathlineto{\pgfqpoint{5.406043in}{2.605229in}}%
\pgfpathlineto{\pgfqpoint{5.420629in}{2.612942in}}%
\pgfpathlineto{\pgfqpoint{5.435230in}{2.620755in}}%
\pgfpathlineto{\pgfqpoint{5.449847in}{2.628669in}}%
\pgfpathlineto{\pgfqpoint{5.442174in}{2.616550in}}%
\pgfpathlineto{\pgfqpoint{5.434495in}{2.604320in}}%
\pgfpathlineto{\pgfqpoint{5.426808in}{2.591980in}}%
\pgfpathlineto{\pgfqpoint{5.419114in}{2.579530in}}%
\pgfpathlineto{\pgfqpoint{5.404497in}{2.571760in}}%
\pgfpathlineto{\pgfqpoint{5.389895in}{2.564090in}}%
\pgfpathlineto{\pgfqpoint{5.375309in}{2.556520in}}%
\pgfpathlineto{\pgfqpoint{5.360739in}{2.549051in}}%
\pgfpathlineto{\pgfqpoint{5.368433in}{2.561350in}}%
\pgfpathlineto{\pgfqpoint{5.376120in}{2.573544in}}%
\pgfpathlineto{\pgfqpoint{5.383800in}{2.585634in}}%
\pgfpathlineto{\pgfqpoint{5.391473in}{2.597617in}}%
\pgfpathclose%
\pgfusepath{fill}%
\end{pgfscope}%
\begin{pgfscope}%
\pgfpathrectangle{\pgfqpoint{1.150000in}{0.150000in}}{\pgfqpoint{5.700000in}{5.700000in}}%
\pgfusepath{clip}%
\pgfsetbuttcap%
\pgfsetroundjoin%
\definecolor{currentfill}{rgb}{0.246811,0.283237,0.535941}%
\pgfsetfillcolor{currentfill}%
\pgfsetfillopacity{0.700000}%
\pgfsetlinewidth{0.000000pt}%
\definecolor{currentstroke}{rgb}{0.000000,0.000000,0.000000}%
\pgfsetstrokecolor{currentstroke}%
\pgfsetdash{}{0pt}%
\pgfpathmoveto{\pgfqpoint{4.974422in}{2.193479in}}%
\pgfpathlineto{\pgfqpoint{4.988762in}{2.198359in}}%
\pgfpathlineto{\pgfqpoint{5.003116in}{2.203337in}}%
\pgfpathlineto{\pgfqpoint{5.017483in}{2.208414in}}%
\pgfpathlineto{\pgfqpoint{5.031864in}{2.213590in}}%
\pgfpathlineto{\pgfqpoint{5.024046in}{2.200209in}}%
\pgfpathlineto{\pgfqpoint{5.016222in}{2.186775in}}%
\pgfpathlineto{\pgfqpoint{5.008394in}{2.173292in}}%
\pgfpathlineto{\pgfqpoint{5.000560in}{2.159761in}}%
\pgfpathlineto{\pgfqpoint{4.986181in}{2.154840in}}%
\pgfpathlineto{\pgfqpoint{4.971814in}{2.150018in}}%
\pgfpathlineto{\pgfqpoint{4.957461in}{2.145295in}}%
\pgfpathlineto{\pgfqpoint{4.943121in}{2.140670in}}%
\pgfpathlineto{\pgfqpoint{4.950954in}{2.153939in}}%
\pgfpathlineto{\pgfqpoint{4.958781in}{2.167165in}}%
\pgfpathlineto{\pgfqpoint{4.966604in}{2.180346in}}%
\pgfpathlineto{\pgfqpoint{4.974422in}{2.193479in}}%
\pgfpathclose%
\pgfusepath{fill}%
\end{pgfscope}%
\begin{pgfscope}%
\pgfpathrectangle{\pgfqpoint{1.150000in}{0.150000in}}{\pgfqpoint{5.700000in}{5.700000in}}%
\pgfusepath{clip}%
\pgfsetbuttcap%
\pgfsetroundjoin%
\definecolor{currentfill}{rgb}{0.282910,0.105393,0.426902}%
\pgfsetfillcolor{currentfill}%
\pgfsetfillopacity{0.700000}%
\pgfsetlinewidth{0.000000pt}%
\definecolor{currentstroke}{rgb}{0.000000,0.000000,0.000000}%
\pgfsetstrokecolor{currentstroke}%
\pgfsetdash{}{0pt}%
\pgfpathmoveto{\pgfqpoint{3.372195in}{1.810747in}}%
\pgfpathlineto{\pgfqpoint{3.386076in}{1.801740in}}%
\pgfpathlineto{\pgfqpoint{3.399961in}{1.792847in}}%
\pgfpathlineto{\pgfqpoint{3.413847in}{1.784068in}}%
\pgfpathlineto{\pgfqpoint{3.427736in}{1.775402in}}%
\pgfpathlineto{\pgfqpoint{3.419316in}{1.774885in}}%
\pgfpathlineto{\pgfqpoint{3.410883in}{1.774632in}}%
\pgfpathlineto{\pgfqpoint{3.402437in}{1.774647in}}%
\pgfpathlineto{\pgfqpoint{3.393979in}{1.774937in}}%
\pgfpathlineto{\pgfqpoint{3.380058in}{1.784131in}}%
\pgfpathlineto{\pgfqpoint{3.366140in}{1.793440in}}%
\pgfpathlineto{\pgfqpoint{3.352223in}{1.802862in}}%
\pgfpathlineto{\pgfqpoint{3.338308in}{1.812399in}}%
\pgfpathlineto{\pgfqpoint{3.346800in}{1.811572in}}%
\pgfpathlineto{\pgfqpoint{3.355278in}{1.811025in}}%
\pgfpathlineto{\pgfqpoint{3.363742in}{1.810751in}}%
\pgfpathlineto{\pgfqpoint{3.372195in}{1.810747in}}%
\pgfpathclose%
\pgfusepath{fill}%
\end{pgfscope}%
\begin{pgfscope}%
\pgfpathrectangle{\pgfqpoint{1.150000in}{0.150000in}}{\pgfqpoint{5.700000in}{5.700000in}}%
\pgfusepath{clip}%
\pgfsetbuttcap%
\pgfsetroundjoin%
\definecolor{currentfill}{rgb}{0.129933,0.559582,0.551864}%
\pgfsetfillcolor{currentfill}%
\pgfsetfillopacity{0.700000}%
\pgfsetlinewidth{0.000000pt}%
\definecolor{currentstroke}{rgb}{0.000000,0.000000,0.000000}%
\pgfsetstrokecolor{currentstroke}%
\pgfsetdash{}{0pt}%
\pgfpathmoveto{\pgfqpoint{2.277648in}{2.927031in}}%
\pgfpathlineto{\pgfqpoint{2.291775in}{2.906410in}}%
\pgfpathlineto{\pgfqpoint{2.305894in}{2.885982in}}%
\pgfpathlineto{\pgfqpoint{2.320004in}{2.865745in}}%
\pgfpathlineto{\pgfqpoint{2.334106in}{2.845696in}}%
\pgfpathlineto{\pgfqpoint{2.324747in}{2.856945in}}%
\pgfpathlineto{\pgfqpoint{2.315361in}{2.868589in}}%
\pgfpathlineto{\pgfqpoint{2.305948in}{2.880634in}}%
\pgfpathlineto{\pgfqpoint{2.296509in}{2.893088in}}%
\pgfpathlineto{\pgfqpoint{2.282343in}{2.913740in}}%
\pgfpathlineto{\pgfqpoint{2.268169in}{2.934583in}}%
\pgfpathlineto{\pgfqpoint{2.253987in}{2.955617in}}%
\pgfpathlineto{\pgfqpoint{2.239794in}{2.976845in}}%
\pgfpathlineto{\pgfqpoint{2.249300in}{2.963776in}}%
\pgfpathlineto{\pgfqpoint{2.258776in}{2.951122in}}%
\pgfpathlineto{\pgfqpoint{2.268226in}{2.938876in}}%
\pgfpathlineto{\pgfqpoint{2.277648in}{2.927031in}}%
\pgfpathclose%
\pgfusepath{fill}%
\end{pgfscope}%
\begin{pgfscope}%
\pgfpathrectangle{\pgfqpoint{1.150000in}{0.150000in}}{\pgfqpoint{5.700000in}{5.700000in}}%
\pgfusepath{clip}%
\pgfsetbuttcap%
\pgfsetroundjoin%
\definecolor{currentfill}{rgb}{0.272594,0.025563,0.353093}%
\pgfsetfillcolor{currentfill}%
\pgfsetfillopacity{0.700000}%
\pgfsetlinewidth{0.000000pt}%
\definecolor{currentstroke}{rgb}{0.000000,0.000000,0.000000}%
\pgfsetstrokecolor{currentstroke}%
\pgfsetdash{}{0pt}%
\pgfpathmoveto{\pgfqpoint{4.148561in}{1.668298in}}%
\pgfpathlineto{\pgfqpoint{4.162563in}{1.666287in}}%
\pgfpathlineto{\pgfqpoint{4.176573in}{1.664376in}}%
\pgfpathlineto{\pgfqpoint{4.190591in}{1.662565in}}%
\pgfpathlineto{\pgfqpoint{4.204617in}{1.660853in}}%
\pgfpathlineto{\pgfqpoint{4.196580in}{1.651467in}}%
\pgfpathlineto{\pgfqpoint{4.188538in}{1.642197in}}%
\pgfpathlineto{\pgfqpoint{4.180490in}{1.633047in}}%
\pgfpathlineto{\pgfqpoint{4.172437in}{1.624022in}}%
\pgfpathlineto{\pgfqpoint{4.158400in}{1.626165in}}%
\pgfpathlineto{\pgfqpoint{4.144371in}{1.628407in}}%
\pgfpathlineto{\pgfqpoint{4.130350in}{1.630749in}}%
\pgfpathlineto{\pgfqpoint{4.116337in}{1.633190in}}%
\pgfpathlineto{\pgfqpoint{4.124401in}{1.641778in}}%
\pgfpathlineto{\pgfqpoint{4.132460in}{1.650495in}}%
\pgfpathlineto{\pgfqpoint{4.140514in}{1.659336in}}%
\pgfpathlineto{\pgfqpoint{4.148561in}{1.668298in}}%
\pgfpathclose%
\pgfusepath{fill}%
\end{pgfscope}%
\begin{pgfscope}%
\pgfpathrectangle{\pgfqpoint{1.150000in}{0.150000in}}{\pgfqpoint{5.700000in}{5.700000in}}%
\pgfusepath{clip}%
\pgfsetbuttcap%
\pgfsetroundjoin%
\definecolor{currentfill}{rgb}{0.271828,0.209303,0.504434}%
\pgfsetfillcolor{currentfill}%
\pgfsetfillopacity{0.700000}%
\pgfsetlinewidth{0.000000pt}%
\definecolor{currentstroke}{rgb}{0.000000,0.000000,0.000000}%
\pgfsetstrokecolor{currentstroke}%
\pgfsetdash{}{0pt}%
\pgfpathmoveto{\pgfqpoint{3.060264in}{2.028009in}}%
\pgfpathlineto{\pgfqpoint{3.074162in}{2.016065in}}%
\pgfpathlineto{\pgfqpoint{3.088059in}{2.004247in}}%
\pgfpathlineto{\pgfqpoint{3.101957in}{1.992555in}}%
\pgfpathlineto{\pgfqpoint{3.115854in}{1.980988in}}%
\pgfpathlineto{\pgfqpoint{3.107201in}{1.984284in}}%
\pgfpathlineto{\pgfqpoint{3.098532in}{1.987893in}}%
\pgfpathlineto{\pgfqpoint{3.089846in}{1.991820in}}%
\pgfpathlineto{\pgfqpoint{3.081144in}{1.996071in}}%
\pgfpathlineto{\pgfqpoint{3.067206in}{2.008195in}}%
\pgfpathlineto{\pgfqpoint{3.053267in}{2.020446in}}%
\pgfpathlineto{\pgfqpoint{3.039327in}{2.032822in}}%
\pgfpathlineto{\pgfqpoint{3.025388in}{2.045326in}}%
\pgfpathlineto{\pgfqpoint{3.034132in}{2.040508in}}%
\pgfpathlineto{\pgfqpoint{3.042860in}{2.036020in}}%
\pgfpathlineto{\pgfqpoint{3.051570in}{2.031856in}}%
\pgfpathlineto{\pgfqpoint{3.060264in}{2.028009in}}%
\pgfpathclose%
\pgfusepath{fill}%
\end{pgfscope}%
\begin{pgfscope}%
\pgfpathrectangle{\pgfqpoint{1.150000in}{0.150000in}}{\pgfqpoint{5.700000in}{5.700000in}}%
\pgfusepath{clip}%
\pgfsetbuttcap%
\pgfsetroundjoin%
\definecolor{currentfill}{rgb}{0.271305,0.019942,0.347269}%
\pgfsetfillcolor{currentfill}%
\pgfsetfillopacity{0.700000}%
\pgfsetlinewidth{0.000000pt}%
\definecolor{currentstroke}{rgb}{0.000000,0.000000,0.000000}%
\pgfsetstrokecolor{currentstroke}%
\pgfsetdash{}{0pt}%
\pgfpathmoveto{\pgfqpoint{3.772122in}{1.658041in}}%
\pgfpathlineto{\pgfqpoint{3.786041in}{1.652645in}}%
\pgfpathlineto{\pgfqpoint{3.799966in}{1.647354in}}%
\pgfpathlineto{\pgfqpoint{3.813895in}{1.642167in}}%
\pgfpathlineto{\pgfqpoint{3.827830in}{1.637084in}}%
\pgfpathlineto{\pgfqpoint{3.819640in}{1.631771in}}%
\pgfpathlineto{\pgfqpoint{3.811442in}{1.626650in}}%
\pgfpathlineto{\pgfqpoint{3.803235in}{1.621727in}}%
\pgfpathlineto{\pgfqpoint{3.795020in}{1.617008in}}%
\pgfpathlineto{\pgfqpoint{3.781064in}{1.622576in}}%
\pgfpathlineto{\pgfqpoint{3.767114in}{1.628249in}}%
\pgfpathlineto{\pgfqpoint{3.753169in}{1.634026in}}%
\pgfpathlineto{\pgfqpoint{3.739228in}{1.639908in}}%
\pgfpathlineto{\pgfqpoint{3.747465in}{1.644134in}}%
\pgfpathlineto{\pgfqpoint{3.755692in}{1.648569in}}%
\pgfpathlineto{\pgfqpoint{3.763911in}{1.653206in}}%
\pgfpathlineto{\pgfqpoint{3.772122in}{1.658041in}}%
\pgfpathclose%
\pgfusepath{fill}%
\end{pgfscope}%
\begin{pgfscope}%
\pgfpathrectangle{\pgfqpoint{1.150000in}{0.150000in}}{\pgfqpoint{5.700000in}{5.700000in}}%
\pgfusepath{clip}%
\pgfsetbuttcap%
\pgfsetroundjoin%
\definecolor{currentfill}{rgb}{0.269944,0.014625,0.341379}%
\pgfsetfillcolor{currentfill}%
\pgfsetfillopacity{0.700000}%
\pgfsetlinewidth{0.000000pt}%
\definecolor{currentstroke}{rgb}{0.000000,0.000000,0.000000}%
\pgfsetstrokecolor{currentstroke}%
\pgfsetdash{}{0pt}%
\pgfpathmoveto{\pgfqpoint{3.916236in}{1.642740in}}%
\pgfpathlineto{\pgfqpoint{3.930182in}{1.638638in}}%
\pgfpathlineto{\pgfqpoint{3.944134in}{1.634639in}}%
\pgfpathlineto{\pgfqpoint{3.958092in}{1.630742in}}%
\pgfpathlineto{\pgfqpoint{3.972057in}{1.626946in}}%
\pgfpathlineto{\pgfqpoint{3.963933in}{1.619989in}}%
\pgfpathlineto{\pgfqpoint{3.955801in}{1.613196in}}%
\pgfpathlineto{\pgfqpoint{3.947662in}{1.606572in}}%
\pgfpathlineto{\pgfqpoint{3.939516in}{1.600123in}}%
\pgfpathlineto{\pgfqpoint{3.925535in}{1.604385in}}%
\pgfpathlineto{\pgfqpoint{3.911559in}{1.608750in}}%
\pgfpathlineto{\pgfqpoint{3.897590in}{1.613216in}}%
\pgfpathlineto{\pgfqpoint{3.883627in}{1.617784in}}%
\pgfpathlineto{\pgfqpoint{3.891790in}{1.623760in}}%
\pgfpathlineto{\pgfqpoint{3.899946in}{1.629914in}}%
\pgfpathlineto{\pgfqpoint{3.908095in}{1.636242in}}%
\pgfpathlineto{\pgfqpoint{3.916236in}{1.642740in}}%
\pgfpathclose%
\pgfusepath{fill}%
\end{pgfscope}%
\begin{pgfscope}%
\pgfpathrectangle{\pgfqpoint{1.150000in}{0.150000in}}{\pgfqpoint{5.700000in}{5.700000in}}%
\pgfusepath{clip}%
\pgfsetbuttcap%
\pgfsetroundjoin%
\definecolor{currentfill}{rgb}{0.274128,0.199721,0.498911}%
\pgfsetfillcolor{currentfill}%
\pgfsetfillopacity{0.700000}%
\pgfsetlinewidth{0.000000pt}%
\definecolor{currentstroke}{rgb}{0.000000,0.000000,0.000000}%
\pgfsetstrokecolor{currentstroke}%
\pgfsetdash{}{0pt}%
\pgfpathmoveto{\pgfqpoint{4.766017in}{2.004331in}}%
\pgfpathlineto{\pgfqpoint{4.780258in}{2.007605in}}%
\pgfpathlineto{\pgfqpoint{4.794511in}{2.010976in}}%
\pgfpathlineto{\pgfqpoint{4.808776in}{2.014447in}}%
\pgfpathlineto{\pgfqpoint{4.823054in}{2.018015in}}%
\pgfpathlineto{\pgfqpoint{4.815178in}{2.004775in}}%
\pgfpathlineto{\pgfqpoint{4.807299in}{1.991523in}}%
\pgfpathlineto{\pgfqpoint{4.799414in}{1.978260in}}%
\pgfpathlineto{\pgfqpoint{4.791526in}{1.964991in}}%
\pgfpathlineto{\pgfqpoint{4.777248in}{1.961732in}}%
\pgfpathlineto{\pgfqpoint{4.762983in}{1.958571in}}%
\pgfpathlineto{\pgfqpoint{4.748729in}{1.955508in}}%
\pgfpathlineto{\pgfqpoint{4.734486in}{1.952542in}}%
\pgfpathlineto{\pgfqpoint{4.742376in}{1.965496in}}%
\pgfpathlineto{\pgfqpoint{4.750260in}{1.978447in}}%
\pgfpathlineto{\pgfqpoint{4.758141in}{1.991393in}}%
\pgfpathlineto{\pgfqpoint{4.766017in}{2.004331in}}%
\pgfpathclose%
\pgfusepath{fill}%
\end{pgfscope}%
\begin{pgfscope}%
\pgfpathrectangle{\pgfqpoint{1.150000in}{0.150000in}}{\pgfqpoint{5.700000in}{5.700000in}}%
\pgfusepath{clip}%
\pgfsetbuttcap%
\pgfsetroundjoin%
\definecolor{currentfill}{rgb}{0.121148,0.592739,0.544641}%
\pgfsetfillcolor{currentfill}%
\pgfsetfillopacity{0.700000}%
\pgfsetlinewidth{0.000000pt}%
\definecolor{currentstroke}{rgb}{0.000000,0.000000,0.000000}%
\pgfsetstrokecolor{currentstroke}%
\pgfsetdash{}{0pt}%
\pgfpathmoveto{\pgfqpoint{2.221048in}{3.011473in}}%
\pgfpathlineto{\pgfqpoint{2.235212in}{2.990065in}}%
\pgfpathlineto{\pgfqpoint{2.249366in}{2.968856in}}%
\pgfpathlineto{\pgfqpoint{2.263512in}{2.947845in}}%
\pgfpathlineto{\pgfqpoint{2.277648in}{2.927031in}}%
\pgfpathlineto{\pgfqpoint{2.268226in}{2.938876in}}%
\pgfpathlineto{\pgfqpoint{2.258776in}{2.951122in}}%
\pgfpathlineto{\pgfqpoint{2.249300in}{2.963776in}}%
\pgfpathlineto{\pgfqpoint{2.239794in}{2.976845in}}%
\pgfpathlineto{\pgfqpoint{2.225593in}{2.998269in}}%
\pgfpathlineto{\pgfqpoint{2.211382in}{3.019890in}}%
\pgfpathlineto{\pgfqpoint{2.197161in}{3.041710in}}%
\pgfpathlineto{\pgfqpoint{2.182931in}{3.063732in}}%
\pgfpathlineto{\pgfqpoint{2.192503in}{3.050042in}}%
\pgfpathlineto{\pgfqpoint{2.202046in}{3.036773in}}%
\pgfpathlineto{\pgfqpoint{2.211561in}{3.023919in}}%
\pgfpathlineto{\pgfqpoint{2.221048in}{3.011473in}}%
\pgfpathclose%
\pgfusepath{fill}%
\end{pgfscope}%
\begin{pgfscope}%
\pgfpathrectangle{\pgfqpoint{1.150000in}{0.150000in}}{\pgfqpoint{5.700000in}{5.700000in}}%
\pgfusepath{clip}%
\pgfsetbuttcap%
\pgfsetroundjoin%
\definecolor{currentfill}{rgb}{0.231674,0.318106,0.544834}%
\pgfsetfillcolor{currentfill}%
\pgfsetfillopacity{0.700000}%
\pgfsetlinewidth{0.000000pt}%
\definecolor{currentstroke}{rgb}{0.000000,0.000000,0.000000}%
\pgfsetstrokecolor{currentstroke}%
\pgfsetdash{}{0pt}%
\pgfpathmoveto{\pgfqpoint{5.063084in}{2.266551in}}%
\pgfpathlineto{\pgfqpoint{5.077479in}{2.272063in}}%
\pgfpathlineto{\pgfqpoint{5.091887in}{2.277674in}}%
\pgfpathlineto{\pgfqpoint{5.106309in}{2.283384in}}%
\pgfpathlineto{\pgfqpoint{5.120746in}{2.289193in}}%
\pgfpathlineto{\pgfqpoint{5.112948in}{2.275812in}}%
\pgfpathlineto{\pgfqpoint{5.105144in}{2.262365in}}%
\pgfpathlineto{\pgfqpoint{5.097335in}{2.248855in}}%
\pgfpathlineto{\pgfqpoint{5.089521in}{2.235283in}}%
\pgfpathlineto{\pgfqpoint{5.075086in}{2.229712in}}%
\pgfpathlineto{\pgfqpoint{5.060665in}{2.224239in}}%
\pgfpathlineto{\pgfqpoint{5.046258in}{2.218865in}}%
\pgfpathlineto{\pgfqpoint{5.031864in}{2.213590in}}%
\pgfpathlineto{\pgfqpoint{5.039677in}{2.226917in}}%
\pgfpathlineto{\pgfqpoint{5.047485in}{2.240188in}}%
\pgfpathlineto{\pgfqpoint{5.055287in}{2.253400in}}%
\pgfpathlineto{\pgfqpoint{5.063084in}{2.266551in}}%
\pgfpathclose%
\pgfusepath{fill}%
\end{pgfscope}%
\begin{pgfscope}%
\pgfpathrectangle{\pgfqpoint{1.150000in}{0.150000in}}{\pgfqpoint{5.700000in}{5.700000in}}%
\pgfusepath{clip}%
\pgfsetbuttcap%
\pgfsetroundjoin%
\definecolor{currentfill}{rgb}{0.194100,0.399323,0.555565}%
\pgfsetfillcolor{currentfill}%
\pgfsetfillopacity{0.700000}%
\pgfsetlinewidth{0.000000pt}%
\definecolor{currentstroke}{rgb}{0.000000,0.000000,0.000000}%
\pgfsetstrokecolor{currentstroke}%
\pgfsetdash{}{0pt}%
\pgfpathmoveto{\pgfqpoint{5.271764in}{2.470612in}}%
\pgfpathlineto{\pgfqpoint{5.286275in}{2.477518in}}%
\pgfpathlineto{\pgfqpoint{5.300800in}{2.484525in}}%
\pgfpathlineto{\pgfqpoint{5.315340in}{2.491631in}}%
\pgfpathlineto{\pgfqpoint{5.329895in}{2.498838in}}%
\pgfpathlineto{\pgfqpoint{5.322167in}{2.486038in}}%
\pgfpathlineto{\pgfqpoint{5.314433in}{2.473141in}}%
\pgfpathlineto{\pgfqpoint{5.306692in}{2.460150in}}%
\pgfpathlineto{\pgfqpoint{5.298945in}{2.447065in}}%
\pgfpathlineto{\pgfqpoint{5.284391in}{2.440040in}}%
\pgfpathlineto{\pgfqpoint{5.269851in}{2.433115in}}%
\pgfpathlineto{\pgfqpoint{5.255327in}{2.426290in}}%
\pgfpathlineto{\pgfqpoint{5.240817in}{2.419564in}}%
\pgfpathlineto{\pgfqpoint{5.248564in}{2.432460in}}%
\pgfpathlineto{\pgfqpoint{5.256303in}{2.445267in}}%
\pgfpathlineto{\pgfqpoint{5.264037in}{2.457985in}}%
\pgfpathlineto{\pgfqpoint{5.271764in}{2.470612in}}%
\pgfpathclose%
\pgfusepath{fill}%
\end{pgfscope}%
\begin{pgfscope}%
\pgfpathrectangle{\pgfqpoint{1.150000in}{0.150000in}}{\pgfqpoint{5.700000in}{5.700000in}}%
\pgfusepath{clip}%
\pgfsetbuttcap%
\pgfsetroundjoin%
\definecolor{currentfill}{rgb}{0.276022,0.044167,0.370164}%
\pgfsetfillcolor{currentfill}%
\pgfsetfillopacity{0.700000}%
\pgfsetlinewidth{0.000000pt}%
\definecolor{currentstroke}{rgb}{0.000000,0.000000,0.000000}%
\pgfsetstrokecolor{currentstroke}%
\pgfsetdash{}{0pt}%
\pgfpathmoveto{\pgfqpoint{3.627868in}{1.690758in}}%
\pgfpathlineto{\pgfqpoint{3.641773in}{1.684029in}}%
\pgfpathlineto{\pgfqpoint{3.655682in}{1.677407in}}%
\pgfpathlineto{\pgfqpoint{3.669595in}{1.670893in}}%
\pgfpathlineto{\pgfqpoint{3.683513in}{1.664484in}}%
\pgfpathlineto{\pgfqpoint{3.675244in}{1.660968in}}%
\pgfpathlineto{\pgfqpoint{3.666966in}{1.657675in}}%
\pgfpathlineto{\pgfqpoint{3.658678in}{1.654609in}}%
\pgfpathlineto{\pgfqpoint{3.650380in}{1.651775in}}%
\pgfpathlineto{\pgfqpoint{3.636438in}{1.658690in}}%
\pgfpathlineto{\pgfqpoint{3.622500in}{1.665711in}}%
\pgfpathlineto{\pgfqpoint{3.608566in}{1.672838in}}%
\pgfpathlineto{\pgfqpoint{3.594635in}{1.680073in}}%
\pgfpathlineto{\pgfqpoint{3.602959in}{1.682393in}}%
\pgfpathlineto{\pgfqpoint{3.611272in}{1.684951in}}%
\pgfpathlineto{\pgfqpoint{3.619575in}{1.687740in}}%
\pgfpathlineto{\pgfqpoint{3.627868in}{1.690758in}}%
\pgfpathclose%
\pgfusepath{fill}%
\end{pgfscope}%
\begin{pgfscope}%
\pgfpathrectangle{\pgfqpoint{1.150000in}{0.150000in}}{\pgfqpoint{5.700000in}{5.700000in}}%
\pgfusepath{clip}%
\pgfsetbuttcap%
\pgfsetroundjoin%
\definecolor{currentfill}{rgb}{0.276194,0.190074,0.493001}%
\pgfsetfillcolor{currentfill}%
\pgfsetfillopacity{0.700000}%
\pgfsetlinewidth{0.000000pt}%
\definecolor{currentstroke}{rgb}{0.000000,0.000000,0.000000}%
\pgfsetstrokecolor{currentstroke}%
\pgfsetdash{}{0pt}%
\pgfpathmoveto{\pgfqpoint{3.115854in}{1.980988in}}%
\pgfpathlineto{\pgfqpoint{3.129752in}{1.969545in}}%
\pgfpathlineto{\pgfqpoint{3.143650in}{1.958226in}}%
\pgfpathlineto{\pgfqpoint{3.157548in}{1.947031in}}%
\pgfpathlineto{\pgfqpoint{3.171447in}{1.935957in}}%
\pgfpathlineto{\pgfqpoint{3.162833in}{1.938705in}}%
\pgfpathlineto{\pgfqpoint{3.154204in}{1.941759in}}%
\pgfpathlineto{\pgfqpoint{3.145558in}{1.945127in}}%
\pgfpathlineto{\pgfqpoint{3.136897in}{1.948813in}}%
\pgfpathlineto{\pgfqpoint{3.122959in}{1.960442in}}%
\pgfpathlineto{\pgfqpoint{3.109021in}{1.972195in}}%
\pgfpathlineto{\pgfqpoint{3.095082in}{1.984071in}}%
\pgfpathlineto{\pgfqpoint{3.081144in}{1.996071in}}%
\pgfpathlineto{\pgfqpoint{3.089846in}{1.991820in}}%
\pgfpathlineto{\pgfqpoint{3.098532in}{1.987893in}}%
\pgfpathlineto{\pgfqpoint{3.107201in}{1.984284in}}%
\pgfpathlineto{\pgfqpoint{3.115854in}{1.980988in}}%
\pgfpathclose%
\pgfusepath{fill}%
\end{pgfscope}%
\begin{pgfscope}%
\pgfpathrectangle{\pgfqpoint{1.150000in}{0.150000in}}{\pgfqpoint{5.700000in}{5.700000in}}%
\pgfusepath{clip}%
\pgfsetbuttcap%
\pgfsetroundjoin%
\definecolor{currentfill}{rgb}{0.133743,0.548535,0.553541}%
\pgfsetfillcolor{currentfill}%
\pgfsetfillopacity{0.700000}%
\pgfsetlinewidth{0.000000pt}%
\definecolor{currentstroke}{rgb}{0.000000,0.000000,0.000000}%
\pgfsetstrokecolor{currentstroke}%
\pgfsetdash{}{0pt}%
\pgfpathmoveto{\pgfqpoint{5.689009in}{2.876683in}}%
\pgfpathlineto{\pgfqpoint{5.703764in}{2.885885in}}%
\pgfpathlineto{\pgfqpoint{5.718537in}{2.895189in}}%
\pgfpathlineto{\pgfqpoint{5.733326in}{2.904595in}}%
\pgfpathlineto{\pgfqpoint{5.748133in}{2.914102in}}%
\pgfpathlineto{\pgfqpoint{5.740597in}{2.903703in}}%
\pgfpathlineto{\pgfqpoint{5.733052in}{2.893167in}}%
\pgfpathlineto{\pgfqpoint{5.725498in}{2.882493in}}%
\pgfpathlineto{\pgfqpoint{5.717935in}{2.871683in}}%
\pgfpathlineto{\pgfqpoint{5.703125in}{2.862241in}}%
\pgfpathlineto{\pgfqpoint{5.688332in}{2.852900in}}%
\pgfpathlineto{\pgfqpoint{5.673557in}{2.843661in}}%
\pgfpathlineto{\pgfqpoint{5.658798in}{2.834524in}}%
\pgfpathlineto{\pgfqpoint{5.666364in}{2.845262in}}%
\pgfpathlineto{\pgfqpoint{5.673921in}{2.855868in}}%
\pgfpathlineto{\pgfqpoint{5.681469in}{2.866342in}}%
\pgfpathlineto{\pgfqpoint{5.689009in}{2.876683in}}%
\pgfpathclose%
\pgfusepath{fill}%
\end{pgfscope}%
\begin{pgfscope}%
\pgfpathrectangle{\pgfqpoint{1.150000in}{0.150000in}}{\pgfqpoint{5.700000in}{5.700000in}}%
\pgfusepath{clip}%
\pgfsetbuttcap%
\pgfsetroundjoin%
\definecolor{currentfill}{rgb}{0.269944,0.014625,0.341379}%
\pgfsetfillcolor{currentfill}%
\pgfsetfillopacity{0.700000}%
\pgfsetlinewidth{0.000000pt}%
\definecolor{currentstroke}{rgb}{0.000000,0.000000,0.000000}%
\pgfsetstrokecolor{currentstroke}%
\pgfsetdash{}{0pt}%
\pgfpathmoveto{\pgfqpoint{4.060356in}{1.643956in}}%
\pgfpathlineto{\pgfqpoint{4.074340in}{1.641114in}}%
\pgfpathlineto{\pgfqpoint{4.088332in}{1.638373in}}%
\pgfpathlineto{\pgfqpoint{4.102330in}{1.635731in}}%
\pgfpathlineto{\pgfqpoint{4.116337in}{1.633190in}}%
\pgfpathlineto{\pgfqpoint{4.108266in}{1.624736in}}%
\pgfpathlineto{\pgfqpoint{4.100189in}{1.616419in}}%
\pgfpathlineto{\pgfqpoint{4.092106in}{1.608244in}}%
\pgfpathlineto{\pgfqpoint{4.084017in}{1.600215in}}%
\pgfpathlineto{\pgfqpoint{4.069998in}{1.603205in}}%
\pgfpathlineto{\pgfqpoint{4.055985in}{1.606295in}}%
\pgfpathlineto{\pgfqpoint{4.041980in}{1.609485in}}%
\pgfpathlineto{\pgfqpoint{4.027982in}{1.612776in}}%
\pgfpathlineto{\pgfqpoint{4.036085in}{1.620349in}}%
\pgfpathlineto{\pgfqpoint{4.044182in}{1.628073in}}%
\pgfpathlineto{\pgfqpoint{4.052272in}{1.635944in}}%
\pgfpathlineto{\pgfqpoint{4.060356in}{1.643956in}}%
\pgfpathclose%
\pgfusepath{fill}%
\end{pgfscope}%
\begin{pgfscope}%
\pgfpathrectangle{\pgfqpoint{1.150000in}{0.150000in}}{\pgfqpoint{5.700000in}{5.700000in}}%
\pgfusepath{clip}%
\pgfsetbuttcap%
\pgfsetroundjoin%
\definecolor{currentfill}{rgb}{0.162142,0.474838,0.558140}%
\pgfsetfillcolor{currentfill}%
\pgfsetfillopacity{0.700000}%
\pgfsetlinewidth{0.000000pt}%
\definecolor{currentstroke}{rgb}{0.000000,0.000000,0.000000}%
\pgfsetstrokecolor{currentstroke}%
\pgfsetdash{}{0pt}%
\pgfpathmoveto{\pgfqpoint{5.480462in}{2.676011in}}%
\pgfpathlineto{\pgfqpoint{5.495094in}{2.684150in}}%
\pgfpathlineto{\pgfqpoint{5.509742in}{2.692389in}}%
\pgfpathlineto{\pgfqpoint{5.524405in}{2.700729in}}%
\pgfpathlineto{\pgfqpoint{5.539085in}{2.709170in}}%
\pgfpathlineto{\pgfqpoint{5.531443in}{2.697390in}}%
\pgfpathlineto{\pgfqpoint{5.523794in}{2.685490in}}%
\pgfpathlineto{\pgfqpoint{5.516137in}{2.673470in}}%
\pgfpathlineto{\pgfqpoint{5.508472in}{2.661332in}}%
\pgfpathlineto{\pgfqpoint{5.493792in}{2.653015in}}%
\pgfpathlineto{\pgfqpoint{5.479128in}{2.644799in}}%
\pgfpathlineto{\pgfqpoint{5.464479in}{2.636684in}}%
\pgfpathlineto{\pgfqpoint{5.449847in}{2.628669in}}%
\pgfpathlineto{\pgfqpoint{5.457512in}{2.640676in}}%
\pgfpathlineto{\pgfqpoint{5.465169in}{2.652569in}}%
\pgfpathlineto{\pgfqpoint{5.472820in}{2.664348in}}%
\pgfpathlineto{\pgfqpoint{5.480462in}{2.676011in}}%
\pgfpathclose%
\pgfusepath{fill}%
\end{pgfscope}%
\begin{pgfscope}%
\pgfpathrectangle{\pgfqpoint{1.150000in}{0.150000in}}{\pgfqpoint{5.700000in}{5.700000in}}%
\pgfusepath{clip}%
\pgfsetbuttcap%
\pgfsetroundjoin%
\definecolor{currentfill}{rgb}{0.282327,0.094955,0.417331}%
\pgfsetfillcolor{currentfill}%
\pgfsetfillopacity{0.700000}%
\pgfsetlinewidth{0.000000pt}%
\definecolor{currentstroke}{rgb}{0.000000,0.000000,0.000000}%
\pgfsetstrokecolor{currentstroke}%
\pgfsetdash{}{0pt}%
\pgfpathmoveto{\pgfqpoint{3.427736in}{1.775402in}}%
\pgfpathlineto{\pgfqpoint{3.441628in}{1.766849in}}%
\pgfpathlineto{\pgfqpoint{3.455522in}{1.758408in}}%
\pgfpathlineto{\pgfqpoint{3.469420in}{1.750079in}}%
\pgfpathlineto{\pgfqpoint{3.483320in}{1.741861in}}%
\pgfpathlineto{\pgfqpoint{3.474929in}{1.740823in}}%
\pgfpathlineto{\pgfqpoint{3.466527in}{1.740044in}}%
\pgfpathlineto{\pgfqpoint{3.458113in}{1.739528in}}%
\pgfpathlineto{\pgfqpoint{3.449686in}{1.739282in}}%
\pgfpathlineto{\pgfqpoint{3.435756in}{1.748028in}}%
\pgfpathlineto{\pgfqpoint{3.421828in}{1.756885in}}%
\pgfpathlineto{\pgfqpoint{3.407902in}{1.765855in}}%
\pgfpathlineto{\pgfqpoint{3.393979in}{1.774937in}}%
\pgfpathlineto{\pgfqpoint{3.402437in}{1.774647in}}%
\pgfpathlineto{\pgfqpoint{3.410883in}{1.774632in}}%
\pgfpathlineto{\pgfqpoint{3.419316in}{1.774885in}}%
\pgfpathlineto{\pgfqpoint{3.427736in}{1.775402in}}%
\pgfpathclose%
\pgfusepath{fill}%
\end{pgfscope}%
\begin{pgfscope}%
\pgfpathrectangle{\pgfqpoint{1.150000in}{0.150000in}}{\pgfqpoint{5.700000in}{5.700000in}}%
\pgfusepath{clip}%
\pgfsetbuttcap%
\pgfsetroundjoin%
\definecolor{currentfill}{rgb}{0.265145,0.232956,0.516599}%
\pgfsetfillcolor{currentfill}%
\pgfsetfillopacity{0.700000}%
\pgfsetlinewidth{0.000000pt}%
\definecolor{currentstroke}{rgb}{0.000000,0.000000,0.000000}%
\pgfsetstrokecolor{currentstroke}%
\pgfsetdash{}{0pt}%
\pgfpathmoveto{\pgfqpoint{4.854508in}{2.070794in}}%
\pgfpathlineto{\pgfqpoint{4.868798in}{2.074751in}}%
\pgfpathlineto{\pgfqpoint{4.883100in}{2.078807in}}%
\pgfpathlineto{\pgfqpoint{4.897414in}{2.082962in}}%
\pgfpathlineto{\pgfqpoint{4.911741in}{2.087215in}}%
\pgfpathlineto{\pgfqpoint{4.903884in}{2.073768in}}%
\pgfpathlineto{\pgfqpoint{4.896022in}{2.060294in}}%
\pgfpathlineto{\pgfqpoint{4.888156in}{2.046793in}}%
\pgfpathlineto{\pgfqpoint{4.880285in}{2.033271in}}%
\pgfpathlineto{\pgfqpoint{4.865959in}{2.029310in}}%
\pgfpathlineto{\pgfqpoint{4.851645in}{2.025446in}}%
\pgfpathlineto{\pgfqpoint{4.837343in}{2.021682in}}%
\pgfpathlineto{\pgfqpoint{4.823054in}{2.018015in}}%
\pgfpathlineto{\pgfqpoint{4.830924in}{2.031239in}}%
\pgfpathlineto{\pgfqpoint{4.838790in}{2.044446in}}%
\pgfpathlineto{\pgfqpoint{4.846652in}{2.057632in}}%
\pgfpathlineto{\pgfqpoint{4.854508in}{2.070794in}}%
\pgfpathclose%
\pgfusepath{fill}%
\end{pgfscope}%
\begin{pgfscope}%
\pgfpathrectangle{\pgfqpoint{1.150000in}{0.150000in}}{\pgfqpoint{5.700000in}{5.700000in}}%
\pgfusepath{clip}%
\pgfsetbuttcap%
\pgfsetroundjoin%
\definecolor{currentfill}{rgb}{0.282327,0.094955,0.417331}%
\pgfsetfillcolor{currentfill}%
\pgfsetfillopacity{0.700000}%
\pgfsetlinewidth{0.000000pt}%
\definecolor{currentstroke}{rgb}{0.000000,0.000000,0.000000}%
\pgfsetstrokecolor{currentstroke}%
\pgfsetdash{}{0pt}%
\pgfpathmoveto{\pgfqpoint{4.469349in}{1.782477in}}%
\pgfpathlineto{\pgfqpoint{4.483470in}{1.783234in}}%
\pgfpathlineto{\pgfqpoint{4.497601in}{1.784090in}}%
\pgfpathlineto{\pgfqpoint{4.511742in}{1.785044in}}%
\pgfpathlineto{\pgfqpoint{4.525893in}{1.786095in}}%
\pgfpathlineto{\pgfqpoint{4.517942in}{1.774141in}}%
\pgfpathlineto{\pgfqpoint{4.509986in}{1.762238in}}%
\pgfpathlineto{\pgfqpoint{4.502026in}{1.750390in}}%
\pgfpathlineto{\pgfqpoint{4.494062in}{1.738601in}}%
\pgfpathlineto{\pgfqpoint{4.479907in}{1.737929in}}%
\pgfpathlineto{\pgfqpoint{4.465762in}{1.737354in}}%
\pgfpathlineto{\pgfqpoint{4.451626in}{1.736877in}}%
\pgfpathlineto{\pgfqpoint{4.437501in}{1.736498in}}%
\pgfpathlineto{\pgfqpoint{4.445470in}{1.747901in}}%
\pgfpathlineto{\pgfqpoint{4.453434in}{1.759368in}}%
\pgfpathlineto{\pgfqpoint{4.461394in}{1.770894in}}%
\pgfpathlineto{\pgfqpoint{4.469349in}{1.782477in}}%
\pgfpathclose%
\pgfusepath{fill}%
\end{pgfscope}%
\begin{pgfscope}%
\pgfpathrectangle{\pgfqpoint{1.150000in}{0.150000in}}{\pgfqpoint{5.700000in}{5.700000in}}%
\pgfusepath{clip}%
\pgfsetbuttcap%
\pgfsetroundjoin%
\definecolor{currentfill}{rgb}{0.120638,0.625828,0.533488}%
\pgfsetfillcolor{currentfill}%
\pgfsetfillopacity{0.700000}%
\pgfsetlinewidth{0.000000pt}%
\definecolor{currentstroke}{rgb}{0.000000,0.000000,0.000000}%
\pgfsetstrokecolor{currentstroke}%
\pgfsetdash{}{0pt}%
\pgfpathmoveto{\pgfqpoint{2.164292in}{3.099146in}}%
\pgfpathlineto{\pgfqpoint{2.178496in}{3.076918in}}%
\pgfpathlineto{\pgfqpoint{2.192690in}{3.054898in}}%
\pgfpathlineto{\pgfqpoint{2.206874in}{3.033084in}}%
\pgfpathlineto{\pgfqpoint{2.221048in}{3.011473in}}%
\pgfpathlineto{\pgfqpoint{2.211561in}{3.023919in}}%
\pgfpathlineto{\pgfqpoint{2.202046in}{3.036773in}}%
\pgfpathlineto{\pgfqpoint{2.192503in}{3.050042in}}%
\pgfpathlineto{\pgfqpoint{2.182931in}{3.063732in}}%
\pgfpathlineto{\pgfqpoint{2.168690in}{3.085956in}}%
\pgfpathlineto{\pgfqpoint{2.154439in}{3.108386in}}%
\pgfpathlineto{\pgfqpoint{2.140177in}{3.131024in}}%
\pgfpathlineto{\pgfqpoint{2.125904in}{3.153870in}}%
\pgfpathlineto{\pgfqpoint{2.135545in}{3.139554in}}%
\pgfpathlineto{\pgfqpoint{2.145157in}{3.125665in}}%
\pgfpathlineto{\pgfqpoint{2.154739in}{3.112198in}}%
\pgfpathlineto{\pgfqpoint{2.164292in}{3.099146in}}%
\pgfpathclose%
\pgfusepath{fill}%
\end{pgfscope}%
\begin{pgfscope}%
\pgfpathrectangle{\pgfqpoint{1.150000in}{0.150000in}}{\pgfqpoint{5.700000in}{5.700000in}}%
\pgfusepath{clip}%
\pgfsetbuttcap%
\pgfsetroundjoin%
\definecolor{currentfill}{rgb}{0.280267,0.073417,0.397163}%
\pgfsetfillcolor{currentfill}%
\pgfsetfillopacity{0.700000}%
\pgfsetlinewidth{0.000000pt}%
\definecolor{currentstroke}{rgb}{0.000000,0.000000,0.000000}%
\pgfsetstrokecolor{currentstroke}%
\pgfsetdash{}{0pt}%
\pgfpathmoveto{\pgfqpoint{4.381093in}{1.735964in}}%
\pgfpathlineto{\pgfqpoint{4.395181in}{1.735950in}}%
\pgfpathlineto{\pgfqpoint{4.409278in}{1.736035in}}%
\pgfpathlineto{\pgfqpoint{4.423385in}{1.736217in}}%
\pgfpathlineto{\pgfqpoint{4.437501in}{1.736498in}}%
\pgfpathlineto{\pgfqpoint{4.429527in}{1.725162in}}%
\pgfpathlineto{\pgfqpoint{4.421549in}{1.713897in}}%
\pgfpathlineto{\pgfqpoint{4.413566in}{1.702707in}}%
\pgfpathlineto{\pgfqpoint{4.405578in}{1.691595in}}%
\pgfpathlineto{\pgfqpoint{4.391456in}{1.691711in}}%
\pgfpathlineto{\pgfqpoint{4.377344in}{1.691924in}}%
\pgfpathlineto{\pgfqpoint{4.363240in}{1.692236in}}%
\pgfpathlineto{\pgfqpoint{4.349146in}{1.692646in}}%
\pgfpathlineto{\pgfqpoint{4.357140in}{1.703355in}}%
\pgfpathlineto{\pgfqpoint{4.365129in}{1.714147in}}%
\pgfpathlineto{\pgfqpoint{4.373114in}{1.725018in}}%
\pgfpathlineto{\pgfqpoint{4.381093in}{1.735964in}}%
\pgfpathclose%
\pgfusepath{fill}%
\end{pgfscope}%
\begin{pgfscope}%
\pgfpathrectangle{\pgfqpoint{1.150000in}{0.150000in}}{\pgfqpoint{5.700000in}{5.700000in}}%
\pgfusepath{clip}%
\pgfsetbuttcap%
\pgfsetroundjoin%
\definecolor{currentfill}{rgb}{0.278826,0.175490,0.483397}%
\pgfsetfillcolor{currentfill}%
\pgfsetfillopacity{0.700000}%
\pgfsetlinewidth{0.000000pt}%
\definecolor{currentstroke}{rgb}{0.000000,0.000000,0.000000}%
\pgfsetstrokecolor{currentstroke}%
\pgfsetdash{}{0pt}%
\pgfpathmoveto{\pgfqpoint{3.171447in}{1.935957in}}%
\pgfpathlineto{\pgfqpoint{3.185346in}{1.925005in}}%
\pgfpathlineto{\pgfqpoint{3.199246in}{1.914174in}}%
\pgfpathlineto{\pgfqpoint{3.213147in}{1.903464in}}%
\pgfpathlineto{\pgfqpoint{3.227049in}{1.892874in}}%
\pgfpathlineto{\pgfqpoint{3.218473in}{1.895075in}}%
\pgfpathlineto{\pgfqpoint{3.209882in}{1.897577in}}%
\pgfpathlineto{\pgfqpoint{3.201276in}{1.900387in}}%
\pgfpathlineto{\pgfqpoint{3.192654in}{1.903511in}}%
\pgfpathlineto{\pgfqpoint{3.178714in}{1.914655in}}%
\pgfpathlineto{\pgfqpoint{3.164775in}{1.925920in}}%
\pgfpathlineto{\pgfqpoint{3.150836in}{1.937306in}}%
\pgfpathlineto{\pgfqpoint{3.136897in}{1.948813in}}%
\pgfpathlineto{\pgfqpoint{3.145558in}{1.945127in}}%
\pgfpathlineto{\pgfqpoint{3.154204in}{1.941759in}}%
\pgfpathlineto{\pgfqpoint{3.162833in}{1.938705in}}%
\pgfpathlineto{\pgfqpoint{3.171447in}{1.935957in}}%
\pgfpathclose%
\pgfusepath{fill}%
\end{pgfscope}%
\begin{pgfscope}%
\pgfpathrectangle{\pgfqpoint{1.150000in}{0.150000in}}{\pgfqpoint{5.700000in}{5.700000in}}%
\pgfusepath{clip}%
\pgfsetbuttcap%
\pgfsetroundjoin%
\definecolor{currentfill}{rgb}{0.124395,0.578002,0.548287}%
\pgfsetfillcolor{currentfill}%
\pgfsetfillopacity{0.700000}%
\pgfsetlinewidth{0.000000pt}%
\definecolor{currentstroke}{rgb}{0.000000,0.000000,0.000000}%
\pgfsetstrokecolor{currentstroke}%
\pgfsetdash{}{0pt}%
\pgfpathmoveto{\pgfqpoint{5.778185in}{2.954320in}}%
\pgfpathlineto{\pgfqpoint{5.793005in}{2.963974in}}%
\pgfpathlineto{\pgfqpoint{5.807842in}{2.973731in}}%
\pgfpathlineto{\pgfqpoint{5.822697in}{2.983589in}}%
\pgfpathlineto{\pgfqpoint{5.815201in}{2.973714in}}%
\pgfpathlineto{\pgfqpoint{5.807696in}{2.963697in}}%
\pgfpathlineto{\pgfqpoint{5.800181in}{2.953537in}}%
\pgfpathlineto{\pgfqpoint{5.792657in}{2.943237in}}%
\pgfpathlineto{\pgfqpoint{5.777798in}{2.933423in}}%
\pgfpathlineto{\pgfqpoint{5.762957in}{2.923711in}}%
\pgfpathlineto{\pgfqpoint{5.748133in}{2.914102in}}%
\pgfpathlineto{\pgfqpoint{5.755660in}{2.924363in}}%
\pgfpathlineto{\pgfqpoint{5.763178in}{2.934487in}}%
\pgfpathlineto{\pgfqpoint{5.770686in}{2.944473in}}%
\pgfpathlineto{\pgfqpoint{5.778185in}{2.954320in}}%
\pgfpathclose%
\pgfusepath{fill}%
\end{pgfscope}%
\begin{pgfscope}%
\pgfpathrectangle{\pgfqpoint{1.150000in}{0.150000in}}{\pgfqpoint{5.700000in}{5.700000in}}%
\pgfusepath{clip}%
\pgfsetbuttcap%
\pgfsetroundjoin%
\definecolor{currentfill}{rgb}{0.283229,0.120777,0.440584}%
\pgfsetfillcolor{currentfill}%
\pgfsetfillopacity{0.700000}%
\pgfsetlinewidth{0.000000pt}%
\definecolor{currentstroke}{rgb}{0.000000,0.000000,0.000000}%
\pgfsetstrokecolor{currentstroke}%
\pgfsetdash{}{0pt}%
\pgfpathmoveto{\pgfqpoint{4.557653in}{1.834362in}}%
\pgfpathlineto{\pgfqpoint{4.571810in}{1.835873in}}%
\pgfpathlineto{\pgfqpoint{4.585978in}{1.837482in}}%
\pgfpathlineto{\pgfqpoint{4.600157in}{1.839189in}}%
\pgfpathlineto{\pgfqpoint{4.614347in}{1.840994in}}%
\pgfpathlineto{\pgfqpoint{4.606417in}{1.828512in}}%
\pgfpathlineto{\pgfqpoint{4.598482in}{1.816063in}}%
\pgfpathlineto{\pgfqpoint{4.590543in}{1.803652in}}%
\pgfpathlineto{\pgfqpoint{4.582600in}{1.791280in}}%
\pgfpathlineto{\pgfqpoint{4.568407in}{1.789837in}}%
\pgfpathlineto{\pgfqpoint{4.554225in}{1.788492in}}%
\pgfpathlineto{\pgfqpoint{4.540054in}{1.787245in}}%
\pgfpathlineto{\pgfqpoint{4.525893in}{1.786095in}}%
\pgfpathlineto{\pgfqpoint{4.533839in}{1.798098in}}%
\pgfpathlineto{\pgfqpoint{4.541781in}{1.810146in}}%
\pgfpathlineto{\pgfqpoint{4.549719in}{1.822235in}}%
\pgfpathlineto{\pgfqpoint{4.557653in}{1.834362in}}%
\pgfpathclose%
\pgfusepath{fill}%
\end{pgfscope}%
\begin{pgfscope}%
\pgfpathrectangle{\pgfqpoint{1.150000in}{0.150000in}}{\pgfqpoint{5.700000in}{5.700000in}}%
\pgfusepath{clip}%
\pgfsetbuttcap%
\pgfsetroundjoin%
\definecolor{currentfill}{rgb}{0.277018,0.050344,0.375715}%
\pgfsetfillcolor{currentfill}%
\pgfsetfillopacity{0.700000}%
\pgfsetlinewidth{0.000000pt}%
\definecolor{currentstroke}{rgb}{0.000000,0.000000,0.000000}%
\pgfsetstrokecolor{currentstroke}%
\pgfsetdash{}{0pt}%
\pgfpathmoveto{\pgfqpoint{4.292859in}{1.695269in}}%
\pgfpathlineto{\pgfqpoint{4.306917in}{1.694466in}}%
\pgfpathlineto{\pgfqpoint{4.320985in}{1.693760in}}%
\pgfpathlineto{\pgfqpoint{4.335061in}{1.693154in}}%
\pgfpathlineto{\pgfqpoint{4.349146in}{1.692646in}}%
\pgfpathlineto{\pgfqpoint{4.341147in}{1.682024in}}%
\pgfpathlineto{\pgfqpoint{4.333144in}{1.671492in}}%
\pgfpathlineto{\pgfqpoint{4.325135in}{1.661056in}}%
\pgfpathlineto{\pgfqpoint{4.317121in}{1.650718in}}%
\pgfpathlineto{\pgfqpoint{4.303029in}{1.651640in}}%
\pgfpathlineto{\pgfqpoint{4.288944in}{1.652660in}}%
\pgfpathlineto{\pgfqpoint{4.274869in}{1.653779in}}%
\pgfpathlineto{\pgfqpoint{4.260802in}{1.654996in}}%
\pgfpathlineto{\pgfqpoint{4.268824in}{1.664913in}}%
\pgfpathlineto{\pgfqpoint{4.276840in}{1.674934in}}%
\pgfpathlineto{\pgfqpoint{4.284852in}{1.685054in}}%
\pgfpathlineto{\pgfqpoint{4.292859in}{1.695269in}}%
\pgfpathclose%
\pgfusepath{fill}%
\end{pgfscope}%
\begin{pgfscope}%
\pgfpathrectangle{\pgfqpoint{1.150000in}{0.150000in}}{\pgfqpoint{5.700000in}{5.700000in}}%
\pgfusepath{clip}%
\pgfsetbuttcap%
\pgfsetroundjoin%
\definecolor{currentfill}{rgb}{0.271305,0.019942,0.347269}%
\pgfsetfillcolor{currentfill}%
\pgfsetfillopacity{0.700000}%
\pgfsetlinewidth{0.000000pt}%
\definecolor{currentstroke}{rgb}{0.000000,0.000000,0.000000}%
\pgfsetstrokecolor{currentstroke}%
\pgfsetdash{}{0pt}%
\pgfpathmoveto{\pgfqpoint{3.827830in}{1.637084in}}%
\pgfpathlineto{\pgfqpoint{3.841771in}{1.632105in}}%
\pgfpathlineto{\pgfqpoint{3.855717in}{1.627228in}}%
\pgfpathlineto{\pgfqpoint{3.869669in}{1.622455in}}%
\pgfpathlineto{\pgfqpoint{3.883627in}{1.617784in}}%
\pgfpathlineto{\pgfqpoint{3.875455in}{1.611992in}}%
\pgfpathlineto{\pgfqpoint{3.867276in}{1.606388in}}%
\pgfpathlineto{\pgfqpoint{3.859090in}{1.600977in}}%
\pgfpathlineto{\pgfqpoint{3.850895in}{1.595764in}}%
\pgfpathlineto{\pgfqpoint{3.836918in}{1.600921in}}%
\pgfpathlineto{\pgfqpoint{3.822947in}{1.606180in}}%
\pgfpathlineto{\pgfqpoint{3.808981in}{1.611542in}}%
\pgfpathlineto{\pgfqpoint{3.795020in}{1.617008in}}%
\pgfpathlineto{\pgfqpoint{3.803235in}{1.621727in}}%
\pgfpathlineto{\pgfqpoint{3.811442in}{1.626650in}}%
\pgfpathlineto{\pgfqpoint{3.819640in}{1.631771in}}%
\pgfpathlineto{\pgfqpoint{3.827830in}{1.637084in}}%
\pgfpathclose%
\pgfusepath{fill}%
\end{pgfscope}%
\begin{pgfscope}%
\pgfpathrectangle{\pgfqpoint{1.150000in}{0.150000in}}{\pgfqpoint{5.700000in}{5.700000in}}%
\pgfusepath{clip}%
\pgfsetbuttcap%
\pgfsetroundjoin%
\definecolor{currentfill}{rgb}{0.216210,0.351535,0.550627}%
\pgfsetfillcolor{currentfill}%
\pgfsetfillopacity{0.700000}%
\pgfsetlinewidth{0.000000pt}%
\definecolor{currentstroke}{rgb}{0.000000,0.000000,0.000000}%
\pgfsetstrokecolor{currentstroke}%
\pgfsetdash{}{0pt}%
\pgfpathmoveto{\pgfqpoint{5.151881in}{2.342026in}}%
\pgfpathlineto{\pgfqpoint{5.166332in}{2.348153in}}%
\pgfpathlineto{\pgfqpoint{5.180798in}{2.354380in}}%
\pgfpathlineto{\pgfqpoint{5.195277in}{2.360706in}}%
\pgfpathlineto{\pgfqpoint{5.209771in}{2.367131in}}%
\pgfpathlineto{\pgfqpoint{5.201995in}{2.353819in}}%
\pgfpathlineto{\pgfqpoint{5.194213in}{2.340429in}}%
\pgfpathlineto{\pgfqpoint{5.186424in}{2.326963in}}%
\pgfpathlineto{\pgfqpoint{5.178631in}{2.313423in}}%
\pgfpathlineto{\pgfqpoint{5.164138in}{2.307216in}}%
\pgfpathlineto{\pgfqpoint{5.149660in}{2.301109in}}%
\pgfpathlineto{\pgfqpoint{5.135196in}{2.295102in}}%
\pgfpathlineto{\pgfqpoint{5.120746in}{2.289193in}}%
\pgfpathlineto{\pgfqpoint{5.128538in}{2.302507in}}%
\pgfpathlineto{\pgfqpoint{5.136325in}{2.315752in}}%
\pgfpathlineto{\pgfqpoint{5.144106in}{2.328926in}}%
\pgfpathlineto{\pgfqpoint{5.151881in}{2.342026in}}%
\pgfpathclose%
\pgfusepath{fill}%
\end{pgfscope}%
\begin{pgfscope}%
\pgfpathrectangle{\pgfqpoint{1.150000in}{0.150000in}}{\pgfqpoint{5.700000in}{5.700000in}}%
\pgfusepath{clip}%
\pgfsetbuttcap%
\pgfsetroundjoin%
\definecolor{currentfill}{rgb}{0.252194,0.269783,0.531579}%
\pgfsetfillcolor{currentfill}%
\pgfsetfillopacity{0.700000}%
\pgfsetlinewidth{0.000000pt}%
\definecolor{currentstroke}{rgb}{0.000000,0.000000,0.000000}%
\pgfsetstrokecolor{currentstroke}%
\pgfsetdash{}{0pt}%
\pgfpathmoveto{\pgfqpoint{4.943121in}{2.140670in}}%
\pgfpathlineto{\pgfqpoint{4.957461in}{2.145295in}}%
\pgfpathlineto{\pgfqpoint{4.971814in}{2.150018in}}%
\pgfpathlineto{\pgfqpoint{4.986181in}{2.154840in}}%
\pgfpathlineto{\pgfqpoint{5.000560in}{2.159761in}}%
\pgfpathlineto{\pgfqpoint{4.992722in}{2.146185in}}%
\pgfpathlineto{\pgfqpoint{4.984879in}{2.132566in}}%
\pgfpathlineto{\pgfqpoint{4.977030in}{2.118907in}}%
\pgfpathlineto{\pgfqpoint{4.969177in}{2.105211in}}%
\pgfpathlineto{\pgfqpoint{4.954799in}{2.100564in}}%
\pgfpathlineto{\pgfqpoint{4.940433in}{2.096016in}}%
\pgfpathlineto{\pgfqpoint{4.926081in}{2.091566in}}%
\pgfpathlineto{\pgfqpoint{4.911741in}{2.087215in}}%
\pgfpathlineto{\pgfqpoint{4.919593in}{2.100631in}}%
\pgfpathlineto{\pgfqpoint{4.927441in}{2.114014in}}%
\pgfpathlineto{\pgfqpoint{4.935283in}{2.127361in}}%
\pgfpathlineto{\pgfqpoint{4.943121in}{2.140670in}}%
\pgfpathclose%
\pgfusepath{fill}%
\end{pgfscope}%
\begin{pgfscope}%
\pgfpathrectangle{\pgfqpoint{1.150000in}{0.150000in}}{\pgfqpoint{5.700000in}{5.700000in}}%
\pgfusepath{clip}%
\pgfsetbuttcap%
\pgfsetroundjoin%
\definecolor{currentfill}{rgb}{0.281887,0.150881,0.465405}%
\pgfsetfillcolor{currentfill}%
\pgfsetfillopacity{0.700000}%
\pgfsetlinewidth{0.000000pt}%
\definecolor{currentstroke}{rgb}{0.000000,0.000000,0.000000}%
\pgfsetstrokecolor{currentstroke}%
\pgfsetdash{}{0pt}%
\pgfpathmoveto{\pgfqpoint{4.646025in}{1.891190in}}%
\pgfpathlineto{\pgfqpoint{4.660223in}{1.893437in}}%
\pgfpathlineto{\pgfqpoint{4.674433in}{1.895781in}}%
\pgfpathlineto{\pgfqpoint{4.688653in}{1.898224in}}%
\pgfpathlineto{\pgfqpoint{4.702885in}{1.900764in}}%
\pgfpathlineto{\pgfqpoint{4.694974in}{1.887844in}}%
\pgfpathlineto{\pgfqpoint{4.687058in}{1.874939in}}%
\pgfpathlineto{\pgfqpoint{4.679138in}{1.862054in}}%
\pgfpathlineto{\pgfqpoint{4.671214in}{1.849191in}}%
\pgfpathlineto{\pgfqpoint{4.656981in}{1.846995in}}%
\pgfpathlineto{\pgfqpoint{4.642759in}{1.844897in}}%
\pgfpathlineto{\pgfqpoint{4.628547in}{1.842896in}}%
\pgfpathlineto{\pgfqpoint{4.614347in}{1.840994in}}%
\pgfpathlineto{\pgfqpoint{4.622273in}{1.853506in}}%
\pgfpathlineto{\pgfqpoint{4.630195in}{1.866045in}}%
\pgfpathlineto{\pgfqpoint{4.638112in}{1.878607in}}%
\pgfpathlineto{\pgfqpoint{4.646025in}{1.891190in}}%
\pgfpathclose%
\pgfusepath{fill}%
\end{pgfscope}%
\begin{pgfscope}%
\pgfpathrectangle{\pgfqpoint{1.150000in}{0.150000in}}{\pgfqpoint{5.700000in}{5.700000in}}%
\pgfusepath{clip}%
\pgfsetbuttcap%
\pgfsetroundjoin%
\definecolor{currentfill}{rgb}{0.179019,0.433756,0.557430}%
\pgfsetfillcolor{currentfill}%
\pgfsetfillopacity{0.700000}%
\pgfsetlinewidth{0.000000pt}%
\definecolor{currentstroke}{rgb}{0.000000,0.000000,0.000000}%
\pgfsetstrokecolor{currentstroke}%
\pgfsetdash{}{0pt}%
\pgfpathmoveto{\pgfqpoint{5.360739in}{2.549051in}}%
\pgfpathlineto{\pgfqpoint{5.375309in}{2.556520in}}%
\pgfpathlineto{\pgfqpoint{5.389895in}{2.564090in}}%
\pgfpathlineto{\pgfqpoint{5.404497in}{2.571760in}}%
\pgfpathlineto{\pgfqpoint{5.419114in}{2.579530in}}%
\pgfpathlineto{\pgfqpoint{5.411412in}{2.566973in}}%
\pgfpathlineto{\pgfqpoint{5.403704in}{2.554309in}}%
\pgfpathlineto{\pgfqpoint{5.395989in}{2.541540in}}%
\pgfpathlineto{\pgfqpoint{5.388267in}{2.528667in}}%
\pgfpathlineto{\pgfqpoint{5.373651in}{2.521060in}}%
\pgfpathlineto{\pgfqpoint{5.359050in}{2.513552in}}%
\pgfpathlineto{\pgfqpoint{5.344465in}{2.506145in}}%
\pgfpathlineto{\pgfqpoint{5.329895in}{2.498838in}}%
\pgfpathlineto{\pgfqpoint{5.337616in}{2.511541in}}%
\pgfpathlineto{\pgfqpoint{5.345330in}{2.524145in}}%
\pgfpathlineto{\pgfqpoint{5.353038in}{2.536649in}}%
\pgfpathlineto{\pgfqpoint{5.360739in}{2.549051in}}%
\pgfpathclose%
\pgfusepath{fill}%
\end{pgfscope}%
\begin{pgfscope}%
\pgfpathrectangle{\pgfqpoint{1.150000in}{0.150000in}}{\pgfqpoint{5.700000in}{5.700000in}}%
\pgfusepath{clip}%
\pgfsetbuttcap%
\pgfsetroundjoin%
\definecolor{currentfill}{rgb}{0.269944,0.014625,0.341379}%
\pgfsetfillcolor{currentfill}%
\pgfsetfillopacity{0.700000}%
\pgfsetlinewidth{0.000000pt}%
\definecolor{currentstroke}{rgb}{0.000000,0.000000,0.000000}%
\pgfsetstrokecolor{currentstroke}%
\pgfsetdash{}{0pt}%
\pgfpathmoveto{\pgfqpoint{3.972057in}{1.626946in}}%
\pgfpathlineto{\pgfqpoint{3.986029in}{1.623252in}}%
\pgfpathlineto{\pgfqpoint{4.000007in}{1.619659in}}%
\pgfpathlineto{\pgfqpoint{4.013991in}{1.616167in}}%
\pgfpathlineto{\pgfqpoint{4.027982in}{1.612776in}}%
\pgfpathlineto{\pgfqpoint{4.019873in}{1.605359in}}%
\pgfpathlineto{\pgfqpoint{4.011757in}{1.598101in}}%
\pgfpathlineto{\pgfqpoint{4.003634in}{1.591008in}}%
\pgfpathlineto{\pgfqpoint{3.995504in}{1.584085in}}%
\pgfpathlineto{\pgfqpoint{3.981497in}{1.587943in}}%
\pgfpathlineto{\pgfqpoint{3.967497in}{1.591902in}}%
\pgfpathlineto{\pgfqpoint{3.953503in}{1.595962in}}%
\pgfpathlineto{\pgfqpoint{3.939516in}{1.600123in}}%
\pgfpathlineto{\pgfqpoint{3.947662in}{1.606572in}}%
\pgfpathlineto{\pgfqpoint{3.955801in}{1.613196in}}%
\pgfpathlineto{\pgfqpoint{3.963933in}{1.619989in}}%
\pgfpathlineto{\pgfqpoint{3.972057in}{1.626946in}}%
\pgfpathclose%
\pgfusepath{fill}%
\end{pgfscope}%
\begin{pgfscope}%
\pgfpathrectangle{\pgfqpoint{1.150000in}{0.150000in}}{\pgfqpoint{5.700000in}{5.700000in}}%
\pgfusepath{clip}%
\pgfsetbuttcap%
\pgfsetroundjoin%
\definecolor{currentfill}{rgb}{0.280894,0.078907,0.402329}%
\pgfsetfillcolor{currentfill}%
\pgfsetfillopacity{0.700000}%
\pgfsetlinewidth{0.000000pt}%
\definecolor{currentstroke}{rgb}{0.000000,0.000000,0.000000}%
\pgfsetstrokecolor{currentstroke}%
\pgfsetdash{}{0pt}%
\pgfpathmoveto{\pgfqpoint{3.483320in}{1.741861in}}%
\pgfpathlineto{\pgfqpoint{3.497223in}{1.733754in}}%
\pgfpathlineto{\pgfqpoint{3.511129in}{1.725757in}}%
\pgfpathlineto{\pgfqpoint{3.525038in}{1.717871in}}%
\pgfpathlineto{\pgfqpoint{3.538951in}{1.710094in}}%
\pgfpathlineto{\pgfqpoint{3.530589in}{1.708536in}}%
\pgfpathlineto{\pgfqpoint{3.522216in}{1.707231in}}%
\pgfpathlineto{\pgfqpoint{3.513831in}{1.706186in}}%
\pgfpathlineto{\pgfqpoint{3.505435in}{1.705404in}}%
\pgfpathlineto{\pgfqpoint{3.491494in}{1.713709in}}%
\pgfpathlineto{\pgfqpoint{3.477555in}{1.722123in}}%
\pgfpathlineto{\pgfqpoint{3.463619in}{1.730647in}}%
\pgfpathlineto{\pgfqpoint{3.449686in}{1.739282in}}%
\pgfpathlineto{\pgfqpoint{3.458113in}{1.739528in}}%
\pgfpathlineto{\pgfqpoint{3.466527in}{1.740044in}}%
\pgfpathlineto{\pgfqpoint{3.474929in}{1.740823in}}%
\pgfpathlineto{\pgfqpoint{3.483320in}{1.741861in}}%
\pgfpathclose%
\pgfusepath{fill}%
\end{pgfscope}%
\begin{pgfscope}%
\pgfpathrectangle{\pgfqpoint{1.150000in}{0.150000in}}{\pgfqpoint{5.700000in}{5.700000in}}%
\pgfusepath{clip}%
\pgfsetbuttcap%
\pgfsetroundjoin%
\definecolor{currentfill}{rgb}{0.149039,0.508051,0.557250}%
\pgfsetfillcolor{currentfill}%
\pgfsetfillopacity{0.700000}%
\pgfsetlinewidth{0.000000pt}%
\definecolor{currentstroke}{rgb}{0.000000,0.000000,0.000000}%
\pgfsetstrokecolor{currentstroke}%
\pgfsetdash{}{0pt}%
\pgfpathmoveto{\pgfqpoint{5.569572in}{2.755072in}}%
\pgfpathlineto{\pgfqpoint{5.584267in}{2.763719in}}%
\pgfpathlineto{\pgfqpoint{5.598978in}{2.772467in}}%
\pgfpathlineto{\pgfqpoint{5.613706in}{2.781316in}}%
\pgfpathlineto{\pgfqpoint{5.628450in}{2.790267in}}%
\pgfpathlineto{\pgfqpoint{5.620842in}{2.778879in}}%
\pgfpathlineto{\pgfqpoint{5.613225in}{2.767362in}}%
\pgfpathlineto{\pgfqpoint{5.605600in}{2.755718in}}%
\pgfpathlineto{\pgfqpoint{5.597968in}{2.743947in}}%
\pgfpathlineto{\pgfqpoint{5.583222in}{2.735101in}}%
\pgfpathlineto{\pgfqpoint{5.568493in}{2.726356in}}%
\pgfpathlineto{\pgfqpoint{5.553781in}{2.717713in}}%
\pgfpathlineto{\pgfqpoint{5.539085in}{2.709170in}}%
\pgfpathlineto{\pgfqpoint{5.546719in}{2.720829in}}%
\pgfpathlineto{\pgfqpoint{5.554345in}{2.732367in}}%
\pgfpathlineto{\pgfqpoint{5.561962in}{2.743781in}}%
\pgfpathlineto{\pgfqpoint{5.569572in}{2.755072in}}%
\pgfpathclose%
\pgfusepath{fill}%
\end{pgfscope}%
\begin{pgfscope}%
\pgfpathrectangle{\pgfqpoint{1.150000in}{0.150000in}}{\pgfqpoint{5.700000in}{5.700000in}}%
\pgfusepath{clip}%
\pgfsetbuttcap%
\pgfsetroundjoin%
\definecolor{currentfill}{rgb}{0.273809,0.031497,0.358853}%
\pgfsetfillcolor{currentfill}%
\pgfsetfillopacity{0.700000}%
\pgfsetlinewidth{0.000000pt}%
\definecolor{currentstroke}{rgb}{0.000000,0.000000,0.000000}%
\pgfsetstrokecolor{currentstroke}%
\pgfsetdash{}{0pt}%
\pgfpathmoveto{\pgfqpoint{4.204617in}{1.660853in}}%
\pgfpathlineto{\pgfqpoint{4.218651in}{1.659240in}}%
\pgfpathlineto{\pgfqpoint{4.232693in}{1.657726in}}%
\pgfpathlineto{\pgfqpoint{4.246743in}{1.656312in}}%
\pgfpathlineto{\pgfqpoint{4.260802in}{1.654996in}}%
\pgfpathlineto{\pgfqpoint{4.252775in}{1.645186in}}%
\pgfpathlineto{\pgfqpoint{4.244742in}{1.635487in}}%
\pgfpathlineto{\pgfqpoint{4.236705in}{1.625904in}}%
\pgfpathlineto{\pgfqpoint{4.228662in}{1.616440in}}%
\pgfpathlineto{\pgfqpoint{4.214593in}{1.618188in}}%
\pgfpathlineto{\pgfqpoint{4.200533in}{1.620033in}}%
\pgfpathlineto{\pgfqpoint{4.186481in}{1.621978in}}%
\pgfpathlineto{\pgfqpoint{4.172437in}{1.624022in}}%
\pgfpathlineto{\pgfqpoint{4.180490in}{1.633047in}}%
\pgfpathlineto{\pgfqpoint{4.188538in}{1.642197in}}%
\pgfpathlineto{\pgfqpoint{4.196580in}{1.651467in}}%
\pgfpathlineto{\pgfqpoint{4.204617in}{1.660853in}}%
\pgfpathclose%
\pgfusepath{fill}%
\end{pgfscope}%
\begin{pgfscope}%
\pgfpathrectangle{\pgfqpoint{1.150000in}{0.150000in}}{\pgfqpoint{5.700000in}{5.700000in}}%
\pgfusepath{clip}%
\pgfsetbuttcap%
\pgfsetroundjoin%
\definecolor{currentfill}{rgb}{0.274952,0.037752,0.364543}%
\pgfsetfillcolor{currentfill}%
\pgfsetfillopacity{0.700000}%
\pgfsetlinewidth{0.000000pt}%
\definecolor{currentstroke}{rgb}{0.000000,0.000000,0.000000}%
\pgfsetstrokecolor{currentstroke}%
\pgfsetdash{}{0pt}%
\pgfpathmoveto{\pgfqpoint{3.683513in}{1.664484in}}%
\pgfpathlineto{\pgfqpoint{3.697435in}{1.658182in}}%
\pgfpathlineto{\pgfqpoint{3.711361in}{1.651985in}}%
\pgfpathlineto{\pgfqpoint{3.725292in}{1.645894in}}%
\pgfpathlineto{\pgfqpoint{3.739228in}{1.639908in}}%
\pgfpathlineto{\pgfqpoint{3.730982in}{1.635893in}}%
\pgfpathlineto{\pgfqpoint{3.722728in}{1.632097in}}%
\pgfpathlineto{\pgfqpoint{3.714464in}{1.628523in}}%
\pgfpathlineto{\pgfqpoint{3.706190in}{1.625177in}}%
\pgfpathlineto{\pgfqpoint{3.692231in}{1.631668in}}%
\pgfpathlineto{\pgfqpoint{3.678276in}{1.638265in}}%
\pgfpathlineto{\pgfqpoint{3.664326in}{1.644967in}}%
\pgfpathlineto{\pgfqpoint{3.650380in}{1.651775in}}%
\pgfpathlineto{\pgfqpoint{3.658678in}{1.654609in}}%
\pgfpathlineto{\pgfqpoint{3.666966in}{1.657675in}}%
\pgfpathlineto{\pgfqpoint{3.675244in}{1.660968in}}%
\pgfpathlineto{\pgfqpoint{3.683513in}{1.664484in}}%
\pgfpathclose%
\pgfusepath{fill}%
\end{pgfscope}%
\begin{pgfscope}%
\pgfpathrectangle{\pgfqpoint{1.150000in}{0.150000in}}{\pgfqpoint{5.700000in}{5.700000in}}%
\pgfusepath{clip}%
\pgfsetbuttcap%
\pgfsetroundjoin%
\definecolor{currentfill}{rgb}{0.281412,0.155834,0.469201}%
\pgfsetfillcolor{currentfill}%
\pgfsetfillopacity{0.700000}%
\pgfsetlinewidth{0.000000pt}%
\definecolor{currentstroke}{rgb}{0.000000,0.000000,0.000000}%
\pgfsetstrokecolor{currentstroke}%
\pgfsetdash{}{0pt}%
\pgfpathmoveto{\pgfqpoint{3.227049in}{1.892874in}}%
\pgfpathlineto{\pgfqpoint{3.240952in}{1.882403in}}%
\pgfpathlineto{\pgfqpoint{3.254856in}{1.872051in}}%
\pgfpathlineto{\pgfqpoint{3.268761in}{1.861816in}}%
\pgfpathlineto{\pgfqpoint{3.282667in}{1.851700in}}%
\pgfpathlineto{\pgfqpoint{3.274127in}{1.853356in}}%
\pgfpathlineto{\pgfqpoint{3.265573in}{1.855308in}}%
\pgfpathlineto{\pgfqpoint{3.257005in}{1.857562in}}%
\pgfpathlineto{\pgfqpoint{3.248421in}{1.860124in}}%
\pgfpathlineto{\pgfqpoint{3.234478in}{1.870793in}}%
\pgfpathlineto{\pgfqpoint{3.220536in}{1.881580in}}%
\pgfpathlineto{\pgfqpoint{3.206595in}{1.892486in}}%
\pgfpathlineto{\pgfqpoint{3.192654in}{1.903511in}}%
\pgfpathlineto{\pgfqpoint{3.201276in}{1.900387in}}%
\pgfpathlineto{\pgfqpoint{3.209882in}{1.897577in}}%
\pgfpathlineto{\pgfqpoint{3.218473in}{1.895075in}}%
\pgfpathlineto{\pgfqpoint{3.227049in}{1.892874in}}%
\pgfpathclose%
\pgfusepath{fill}%
\end{pgfscope}%
\begin{pgfscope}%
\pgfpathrectangle{\pgfqpoint{1.150000in}{0.150000in}}{\pgfqpoint{5.700000in}{5.700000in}}%
\pgfusepath{clip}%
\pgfsetbuttcap%
\pgfsetroundjoin%
\definecolor{currentfill}{rgb}{0.277134,0.185228,0.489898}%
\pgfsetfillcolor{currentfill}%
\pgfsetfillopacity{0.700000}%
\pgfsetlinewidth{0.000000pt}%
\definecolor{currentstroke}{rgb}{0.000000,0.000000,0.000000}%
\pgfsetstrokecolor{currentstroke}%
\pgfsetdash{}{0pt}%
\pgfpathmoveto{\pgfqpoint{4.734486in}{1.952542in}}%
\pgfpathlineto{\pgfqpoint{4.748729in}{1.955508in}}%
\pgfpathlineto{\pgfqpoint{4.762983in}{1.958571in}}%
\pgfpathlineto{\pgfqpoint{4.777248in}{1.961732in}}%
\pgfpathlineto{\pgfqpoint{4.791526in}{1.964991in}}%
\pgfpathlineto{\pgfqpoint{4.783633in}{1.951718in}}%
\pgfpathlineto{\pgfqpoint{4.775735in}{1.938444in}}%
\pgfpathlineto{\pgfqpoint{4.767834in}{1.925172in}}%
\pgfpathlineto{\pgfqpoint{4.759927in}{1.911905in}}%
\pgfpathlineto{\pgfqpoint{4.745649in}{1.908973in}}%
\pgfpathlineto{\pgfqpoint{4.731383in}{1.906139in}}%
\pgfpathlineto{\pgfqpoint{4.717128in}{1.903403in}}%
\pgfpathlineto{\pgfqpoint{4.702885in}{1.900764in}}%
\pgfpathlineto{\pgfqpoint{4.710792in}{1.913698in}}%
\pgfpathlineto{\pgfqpoint{4.718694in}{1.926640in}}%
\pgfpathlineto{\pgfqpoint{4.726593in}{1.939590in}}%
\pgfpathlineto{\pgfqpoint{4.734486in}{1.952542in}}%
\pgfpathclose%
\pgfusepath{fill}%
\end{pgfscope}%
\begin{pgfscope}%
\pgfpathrectangle{\pgfqpoint{1.150000in}{0.150000in}}{\pgfqpoint{5.700000in}{5.700000in}}%
\pgfusepath{clip}%
\pgfsetbuttcap%
\pgfsetroundjoin%
\definecolor{currentfill}{rgb}{0.237441,0.305202,0.541921}%
\pgfsetfillcolor{currentfill}%
\pgfsetfillopacity{0.700000}%
\pgfsetlinewidth{0.000000pt}%
\definecolor{currentstroke}{rgb}{0.000000,0.000000,0.000000}%
\pgfsetstrokecolor{currentstroke}%
\pgfsetdash{}{0pt}%
\pgfpathmoveto{\pgfqpoint{5.031864in}{2.213590in}}%
\pgfpathlineto{\pgfqpoint{5.046258in}{2.218865in}}%
\pgfpathlineto{\pgfqpoint{5.060665in}{2.224239in}}%
\pgfpathlineto{\pgfqpoint{5.075086in}{2.229712in}}%
\pgfpathlineto{\pgfqpoint{5.089521in}{2.235283in}}%
\pgfpathlineto{\pgfqpoint{5.081702in}{2.221653in}}%
\pgfpathlineto{\pgfqpoint{5.073877in}{2.207966in}}%
\pgfpathlineto{\pgfqpoint{5.066047in}{2.194224in}}%
\pgfpathlineto{\pgfqpoint{5.058212in}{2.180431in}}%
\pgfpathlineto{\pgfqpoint{5.043779in}{2.175115in}}%
\pgfpathlineto{\pgfqpoint{5.029360in}{2.169898in}}%
\pgfpathlineto{\pgfqpoint{5.014953in}{2.164780in}}%
\pgfpathlineto{\pgfqpoint{5.000560in}{2.159761in}}%
\pgfpathlineto{\pgfqpoint{5.008394in}{2.173292in}}%
\pgfpathlineto{\pgfqpoint{5.016222in}{2.186775in}}%
\pgfpathlineto{\pgfqpoint{5.024046in}{2.200209in}}%
\pgfpathlineto{\pgfqpoint{5.031864in}{2.213590in}}%
\pgfpathclose%
\pgfusepath{fill}%
\end{pgfscope}%
\begin{pgfscope}%
\pgfpathrectangle{\pgfqpoint{1.150000in}{0.150000in}}{\pgfqpoint{5.700000in}{5.700000in}}%
\pgfusepath{clip}%
\pgfsetbuttcap%
\pgfsetroundjoin%
\definecolor{currentfill}{rgb}{0.271305,0.019942,0.347269}%
\pgfsetfillcolor{currentfill}%
\pgfsetfillopacity{0.700000}%
\pgfsetlinewidth{0.000000pt}%
\definecolor{currentstroke}{rgb}{0.000000,0.000000,0.000000}%
\pgfsetstrokecolor{currentstroke}%
\pgfsetdash{}{0pt}%
\pgfpathmoveto{\pgfqpoint{4.116337in}{1.633190in}}%
\pgfpathlineto{\pgfqpoint{4.130350in}{1.630749in}}%
\pgfpathlineto{\pgfqpoint{4.144371in}{1.628407in}}%
\pgfpathlineto{\pgfqpoint{4.158400in}{1.626165in}}%
\pgfpathlineto{\pgfqpoint{4.172437in}{1.624022in}}%
\pgfpathlineto{\pgfqpoint{4.164378in}{1.615125in}}%
\pgfpathlineto{\pgfqpoint{4.156313in}{1.606362in}}%
\pgfpathlineto{\pgfqpoint{4.148242in}{1.597735in}}%
\pgfpathlineto{\pgfqpoint{4.140166in}{1.589251in}}%
\pgfpathlineto{\pgfqpoint{4.126117in}{1.591843in}}%
\pgfpathlineto{\pgfqpoint{4.112076in}{1.594534in}}%
\pgfpathlineto{\pgfqpoint{4.098043in}{1.597325in}}%
\pgfpathlineto{\pgfqpoint{4.084017in}{1.600215in}}%
\pgfpathlineto{\pgfqpoint{4.092106in}{1.608244in}}%
\pgfpathlineto{\pgfqpoint{4.100189in}{1.616419in}}%
\pgfpathlineto{\pgfqpoint{4.108266in}{1.624736in}}%
\pgfpathlineto{\pgfqpoint{4.116337in}{1.633190in}}%
\pgfpathclose%
\pgfusepath{fill}%
\end{pgfscope}%
\begin{pgfscope}%
\pgfpathrectangle{\pgfqpoint{1.150000in}{0.150000in}}{\pgfqpoint{5.700000in}{5.700000in}}%
\pgfusepath{clip}%
\pgfsetbuttcap%
\pgfsetroundjoin%
\definecolor{currentfill}{rgb}{0.199430,0.387607,0.554642}%
\pgfsetfillcolor{currentfill}%
\pgfsetfillopacity{0.700000}%
\pgfsetlinewidth{0.000000pt}%
\definecolor{currentstroke}{rgb}{0.000000,0.000000,0.000000}%
\pgfsetstrokecolor{currentstroke}%
\pgfsetdash{}{0pt}%
\pgfpathmoveto{\pgfqpoint{5.240817in}{2.419564in}}%
\pgfpathlineto{\pgfqpoint{5.255327in}{2.426290in}}%
\pgfpathlineto{\pgfqpoint{5.269851in}{2.433115in}}%
\pgfpathlineto{\pgfqpoint{5.284391in}{2.440040in}}%
\pgfpathlineto{\pgfqpoint{5.298945in}{2.447065in}}%
\pgfpathlineto{\pgfqpoint{5.291191in}{2.433889in}}%
\pgfpathlineto{\pgfqpoint{5.283431in}{2.420624in}}%
\pgfpathlineto{\pgfqpoint{5.275666in}{2.407270in}}%
\pgfpathlineto{\pgfqpoint{5.267893in}{2.393830in}}%
\pgfpathlineto{\pgfqpoint{5.253341in}{2.387006in}}%
\pgfpathlineto{\pgfqpoint{5.238803in}{2.380281in}}%
\pgfpathlineto{\pgfqpoint{5.224280in}{2.373656in}}%
\pgfpathlineto{\pgfqpoint{5.209771in}{2.367131in}}%
\pgfpathlineto{\pgfqpoint{5.217542in}{2.380364in}}%
\pgfpathlineto{\pgfqpoint{5.225306in}{2.393514in}}%
\pgfpathlineto{\pgfqpoint{5.233065in}{2.406582in}}%
\pgfpathlineto{\pgfqpoint{5.240817in}{2.419564in}}%
\pgfpathclose%
\pgfusepath{fill}%
\end{pgfscope}%
\begin{pgfscope}%
\pgfpathrectangle{\pgfqpoint{1.150000in}{0.150000in}}{\pgfqpoint{5.700000in}{5.700000in}}%
\pgfusepath{clip}%
\pgfsetbuttcap%
\pgfsetroundjoin%
\definecolor{currentfill}{rgb}{0.282623,0.140926,0.457517}%
\pgfsetfillcolor{currentfill}%
\pgfsetfillopacity{0.700000}%
\pgfsetlinewidth{0.000000pt}%
\definecolor{currentstroke}{rgb}{0.000000,0.000000,0.000000}%
\pgfsetstrokecolor{currentstroke}%
\pgfsetdash{}{0pt}%
\pgfpathmoveto{\pgfqpoint{3.282667in}{1.851700in}}%
\pgfpathlineto{\pgfqpoint{3.296575in}{1.841700in}}%
\pgfpathlineto{\pgfqpoint{3.310484in}{1.831817in}}%
\pgfpathlineto{\pgfqpoint{3.324396in}{1.822050in}}%
\pgfpathlineto{\pgfqpoint{3.338308in}{1.812399in}}%
\pgfpathlineto{\pgfqpoint{3.329804in}{1.813511in}}%
\pgfpathlineto{\pgfqpoint{3.321285in}{1.814914in}}%
\pgfpathlineto{\pgfqpoint{3.312753in}{1.816614in}}%
\pgfpathlineto{\pgfqpoint{3.304206in}{1.818616in}}%
\pgfpathlineto{\pgfqpoint{3.290258in}{1.828819in}}%
\pgfpathlineto{\pgfqpoint{3.276311in}{1.839138in}}%
\pgfpathlineto{\pgfqpoint{3.262366in}{1.849572in}}%
\pgfpathlineto{\pgfqpoint{3.248421in}{1.860124in}}%
\pgfpathlineto{\pgfqpoint{3.257005in}{1.857562in}}%
\pgfpathlineto{\pgfqpoint{3.265573in}{1.855308in}}%
\pgfpathlineto{\pgfqpoint{3.274127in}{1.853356in}}%
\pgfpathlineto{\pgfqpoint{3.282667in}{1.851700in}}%
\pgfpathclose%
\pgfusepath{fill}%
\end{pgfscope}%
\begin{pgfscope}%
\pgfpathrectangle{\pgfqpoint{1.150000in}{0.150000in}}{\pgfqpoint{5.700000in}{5.700000in}}%
\pgfusepath{clip}%
\pgfsetbuttcap%
\pgfsetroundjoin%
\definecolor{currentfill}{rgb}{0.269308,0.218818,0.509577}%
\pgfsetfillcolor{currentfill}%
\pgfsetfillopacity{0.700000}%
\pgfsetlinewidth{0.000000pt}%
\definecolor{currentstroke}{rgb}{0.000000,0.000000,0.000000}%
\pgfsetstrokecolor{currentstroke}%
\pgfsetdash{}{0pt}%
\pgfpathmoveto{\pgfqpoint{4.823054in}{2.018015in}}%
\pgfpathlineto{\pgfqpoint{4.837343in}{2.021682in}}%
\pgfpathlineto{\pgfqpoint{4.851645in}{2.025446in}}%
\pgfpathlineto{\pgfqpoint{4.865959in}{2.029310in}}%
\pgfpathlineto{\pgfqpoint{4.880285in}{2.033271in}}%
\pgfpathlineto{\pgfqpoint{4.872410in}{2.019728in}}%
\pgfpathlineto{\pgfqpoint{4.864530in}{2.006168in}}%
\pgfpathlineto{\pgfqpoint{4.856646in}{1.992594in}}%
\pgfpathlineto{\pgfqpoint{4.848757in}{1.979008in}}%
\pgfpathlineto{\pgfqpoint{4.834431in}{1.975357in}}%
\pgfpathlineto{\pgfqpoint{4.820117in}{1.971804in}}%
\pgfpathlineto{\pgfqpoint{4.805815in}{1.968349in}}%
\pgfpathlineto{\pgfqpoint{4.791526in}{1.964991in}}%
\pgfpathlineto{\pgfqpoint{4.799414in}{1.978260in}}%
\pgfpathlineto{\pgfqpoint{4.807299in}{1.991523in}}%
\pgfpathlineto{\pgfqpoint{4.815178in}{2.004775in}}%
\pgfpathlineto{\pgfqpoint{4.823054in}{2.018015in}}%
\pgfpathclose%
\pgfusepath{fill}%
\end{pgfscope}%
\begin{pgfscope}%
\pgfpathrectangle{\pgfqpoint{1.150000in}{0.150000in}}{\pgfqpoint{5.700000in}{5.700000in}}%
\pgfusepath{clip}%
\pgfsetbuttcap%
\pgfsetroundjoin%
\definecolor{currentfill}{rgb}{0.165117,0.467423,0.558141}%
\pgfsetfillcolor{currentfill}%
\pgfsetfillopacity{0.700000}%
\pgfsetlinewidth{0.000000pt}%
\definecolor{currentstroke}{rgb}{0.000000,0.000000,0.000000}%
\pgfsetstrokecolor{currentstroke}%
\pgfsetdash{}{0pt}%
\pgfpathmoveto{\pgfqpoint{5.449847in}{2.628669in}}%
\pgfpathlineto{\pgfqpoint{5.464479in}{2.636684in}}%
\pgfpathlineto{\pgfqpoint{5.479128in}{2.644799in}}%
\pgfpathlineto{\pgfqpoint{5.493792in}{2.653015in}}%
\pgfpathlineto{\pgfqpoint{5.508472in}{2.661332in}}%
\pgfpathlineto{\pgfqpoint{5.500800in}{2.649077in}}%
\pgfpathlineto{\pgfqpoint{5.493120in}{2.636706in}}%
\pgfpathlineto{\pgfqpoint{5.485433in}{2.624219in}}%
\pgfpathlineto{\pgfqpoint{5.477739in}{2.611619in}}%
\pgfpathlineto{\pgfqpoint{5.463059in}{2.603445in}}%
\pgfpathlineto{\pgfqpoint{5.448394in}{2.595373in}}%
\pgfpathlineto{\pgfqpoint{5.433746in}{2.587401in}}%
\pgfpathlineto{\pgfqpoint{5.419114in}{2.579530in}}%
\pgfpathlineto{\pgfqpoint{5.426808in}{2.591980in}}%
\pgfpathlineto{\pgfqpoint{5.434495in}{2.604320in}}%
\pgfpathlineto{\pgfqpoint{5.442174in}{2.616550in}}%
\pgfpathlineto{\pgfqpoint{5.449847in}{2.628669in}}%
\pgfpathclose%
\pgfusepath{fill}%
\end{pgfscope}%
\begin{pgfscope}%
\pgfpathrectangle{\pgfqpoint{1.150000in}{0.150000in}}{\pgfqpoint{5.700000in}{5.700000in}}%
\pgfusepath{clip}%
\pgfsetbuttcap%
\pgfsetroundjoin%
\definecolor{currentfill}{rgb}{0.194100,0.399323,0.555565}%
\pgfsetfillcolor{currentfill}%
\pgfsetfillopacity{0.700000}%
\pgfsetlinewidth{0.000000pt}%
\definecolor{currentstroke}{rgb}{0.000000,0.000000,0.000000}%
\pgfsetstrokecolor{currentstroke}%
\pgfsetdash{}{0pt}%
\pgfpathmoveto{\pgfqpoint{2.634345in}{2.450283in}}%
\pgfpathlineto{\pgfqpoint{2.648351in}{2.433879in}}%
\pgfpathlineto{\pgfqpoint{2.662354in}{2.417629in}}%
\pgfpathlineto{\pgfqpoint{2.676352in}{2.401530in}}%
\pgfpathlineto{\pgfqpoint{2.690346in}{2.385581in}}%
\pgfpathlineto{\pgfqpoint{2.681292in}{2.394223in}}%
\pgfpathlineto{\pgfqpoint{2.672216in}{2.403241in}}%
\pgfpathlineto{\pgfqpoint{2.663118in}{2.412643in}}%
\pgfpathlineto{\pgfqpoint{2.653996in}{2.422435in}}%
\pgfpathlineto{\pgfqpoint{2.639947in}{2.438983in}}%
\pgfpathlineto{\pgfqpoint{2.625894in}{2.455683in}}%
\pgfpathlineto{\pgfqpoint{2.611836in}{2.472535in}}%
\pgfpathlineto{\pgfqpoint{2.597774in}{2.489540in}}%
\pgfpathlineto{\pgfqpoint{2.606952in}{2.479138in}}%
\pgfpathlineto{\pgfqpoint{2.616106in}{2.469132in}}%
\pgfpathlineto{\pgfqpoint{2.625237in}{2.459516in}}%
\pgfpathlineto{\pgfqpoint{2.634345in}{2.450283in}}%
\pgfpathclose%
\pgfusepath{fill}%
\end{pgfscope}%
\begin{pgfscope}%
\pgfpathrectangle{\pgfqpoint{1.150000in}{0.150000in}}{\pgfqpoint{5.700000in}{5.700000in}}%
\pgfusepath{clip}%
\pgfsetbuttcap%
\pgfsetroundjoin%
\definecolor{currentfill}{rgb}{0.204903,0.375746,0.553533}%
\pgfsetfillcolor{currentfill}%
\pgfsetfillopacity{0.700000}%
\pgfsetlinewidth{0.000000pt}%
\definecolor{currentstroke}{rgb}{0.000000,0.000000,0.000000}%
\pgfsetstrokecolor{currentstroke}%
\pgfsetdash{}{0pt}%
\pgfpathmoveto{\pgfqpoint{2.690346in}{2.385581in}}%
\pgfpathlineto{\pgfqpoint{2.704337in}{2.369782in}}%
\pgfpathlineto{\pgfqpoint{2.718323in}{2.354131in}}%
\pgfpathlineto{\pgfqpoint{2.732307in}{2.338627in}}%
\pgfpathlineto{\pgfqpoint{2.746287in}{2.323270in}}%
\pgfpathlineto{\pgfqpoint{2.737285in}{2.331322in}}%
\pgfpathlineto{\pgfqpoint{2.728263in}{2.339746in}}%
\pgfpathlineto{\pgfqpoint{2.719218in}{2.348548in}}%
\pgfpathlineto{\pgfqpoint{2.710151in}{2.357733in}}%
\pgfpathlineto{\pgfqpoint{2.696118in}{2.373687in}}%
\pgfpathlineto{\pgfqpoint{2.682081in}{2.389788in}}%
\pgfpathlineto{\pgfqpoint{2.668040in}{2.406037in}}%
\pgfpathlineto{\pgfqpoint{2.653996in}{2.422435in}}%
\pgfpathlineto{\pgfqpoint{2.663118in}{2.412643in}}%
\pgfpathlineto{\pgfqpoint{2.672216in}{2.403241in}}%
\pgfpathlineto{\pgfqpoint{2.681292in}{2.394223in}}%
\pgfpathlineto{\pgfqpoint{2.690346in}{2.385581in}}%
\pgfpathclose%
\pgfusepath{fill}%
\end{pgfscope}%
\begin{pgfscope}%
\pgfpathrectangle{\pgfqpoint{1.150000in}{0.150000in}}{\pgfqpoint{5.700000in}{5.700000in}}%
\pgfusepath{clip}%
\pgfsetbuttcap%
\pgfsetroundjoin%
\definecolor{currentfill}{rgb}{0.182256,0.426184,0.557120}%
\pgfsetfillcolor{currentfill}%
\pgfsetfillopacity{0.700000}%
\pgfsetlinewidth{0.000000pt}%
\definecolor{currentstroke}{rgb}{0.000000,0.000000,0.000000}%
\pgfsetstrokecolor{currentstroke}%
\pgfsetdash{}{0pt}%
\pgfpathmoveto{\pgfqpoint{2.578274in}{2.517445in}}%
\pgfpathlineto{\pgfqpoint{2.592299in}{2.500420in}}%
\pgfpathlineto{\pgfqpoint{2.606319in}{2.483552in}}%
\pgfpathlineto{\pgfqpoint{2.620334in}{2.466840in}}%
\pgfpathlineto{\pgfqpoint{2.634345in}{2.450283in}}%
\pgfpathlineto{\pgfqpoint{2.625237in}{2.459516in}}%
\pgfpathlineto{\pgfqpoint{2.616106in}{2.469132in}}%
\pgfpathlineto{\pgfqpoint{2.606952in}{2.479138in}}%
\pgfpathlineto{\pgfqpoint{2.597774in}{2.489540in}}%
\pgfpathlineto{\pgfqpoint{2.583707in}{2.506701in}}%
\pgfpathlineto{\pgfqpoint{2.569635in}{2.524017in}}%
\pgfpathlineto{\pgfqpoint{2.555558in}{2.541490in}}%
\pgfpathlineto{\pgfqpoint{2.541475in}{2.559122in}}%
\pgfpathlineto{\pgfqpoint{2.550712in}{2.548106in}}%
\pgfpathlineto{\pgfqpoint{2.559923in}{2.537492in}}%
\pgfpathlineto{\pgfqpoint{2.569111in}{2.527274in}}%
\pgfpathlineto{\pgfqpoint{2.578274in}{2.517445in}}%
\pgfpathclose%
\pgfusepath{fill}%
\end{pgfscope}%
\begin{pgfscope}%
\pgfpathrectangle{\pgfqpoint{1.150000in}{0.150000in}}{\pgfqpoint{5.700000in}{5.700000in}}%
\pgfusepath{clip}%
\pgfsetbuttcap%
\pgfsetroundjoin%
\definecolor{currentfill}{rgb}{0.279566,0.067836,0.391917}%
\pgfsetfillcolor{currentfill}%
\pgfsetfillopacity{0.700000}%
\pgfsetlinewidth{0.000000pt}%
\definecolor{currentstroke}{rgb}{0.000000,0.000000,0.000000}%
\pgfsetstrokecolor{currentstroke}%
\pgfsetdash{}{0pt}%
\pgfpathmoveto{\pgfqpoint{3.538951in}{1.710094in}}%
\pgfpathlineto{\pgfqpoint{3.552866in}{1.702426in}}%
\pgfpathlineto{\pgfqpoint{3.566786in}{1.694867in}}%
\pgfpathlineto{\pgfqpoint{3.580709in}{1.687416in}}%
\pgfpathlineto{\pgfqpoint{3.594635in}{1.680073in}}%
\pgfpathlineto{\pgfqpoint{3.586301in}{1.677997in}}%
\pgfpathlineto{\pgfqpoint{3.577956in}{1.676168in}}%
\pgfpathlineto{\pgfqpoint{3.569600in}{1.674593in}}%
\pgfpathlineto{\pgfqpoint{3.561233in}{1.673278in}}%
\pgfpathlineto{\pgfqpoint{3.547279in}{1.681147in}}%
\pgfpathlineto{\pgfqpoint{3.533328in}{1.689124in}}%
\pgfpathlineto{\pgfqpoint{3.519380in}{1.697210in}}%
\pgfpathlineto{\pgfqpoint{3.505435in}{1.705404in}}%
\pgfpathlineto{\pgfqpoint{3.513831in}{1.706186in}}%
\pgfpathlineto{\pgfqpoint{3.522216in}{1.707231in}}%
\pgfpathlineto{\pgfqpoint{3.530589in}{1.708536in}}%
\pgfpathlineto{\pgfqpoint{3.538951in}{1.710094in}}%
\pgfpathclose%
\pgfusepath{fill}%
\end{pgfscope}%
\begin{pgfscope}%
\pgfpathrectangle{\pgfqpoint{1.150000in}{0.150000in}}{\pgfqpoint{5.700000in}{5.700000in}}%
\pgfusepath{clip}%
\pgfsetbuttcap%
\pgfsetroundjoin%
\definecolor{currentfill}{rgb}{0.216210,0.351535,0.550627}%
\pgfsetfillcolor{currentfill}%
\pgfsetfillopacity{0.700000}%
\pgfsetlinewidth{0.000000pt}%
\definecolor{currentstroke}{rgb}{0.000000,0.000000,0.000000}%
\pgfsetstrokecolor{currentstroke}%
\pgfsetdash{}{0pt}%
\pgfpathmoveto{\pgfqpoint{2.746287in}{2.323270in}}%
\pgfpathlineto{\pgfqpoint{2.760263in}{2.308057in}}%
\pgfpathlineto{\pgfqpoint{2.774237in}{2.292988in}}%
\pgfpathlineto{\pgfqpoint{2.788208in}{2.278063in}}%
\pgfpathlineto{\pgfqpoint{2.802176in}{2.263279in}}%
\pgfpathlineto{\pgfqpoint{2.793225in}{2.270747in}}%
\pgfpathlineto{\pgfqpoint{2.784254in}{2.278579in}}%
\pgfpathlineto{\pgfqpoint{2.775262in}{2.286783in}}%
\pgfpathlineto{\pgfqpoint{2.766248in}{2.295365in}}%
\pgfpathlineto{\pgfqpoint{2.752228in}{2.310742in}}%
\pgfpathlineto{\pgfqpoint{2.738206in}{2.326261in}}%
\pgfpathlineto{\pgfqpoint{2.724180in}{2.341925in}}%
\pgfpathlineto{\pgfqpoint{2.710151in}{2.357733in}}%
\pgfpathlineto{\pgfqpoint{2.719218in}{2.348548in}}%
\pgfpathlineto{\pgfqpoint{2.728263in}{2.339746in}}%
\pgfpathlineto{\pgfqpoint{2.737285in}{2.331322in}}%
\pgfpathlineto{\pgfqpoint{2.746287in}{2.323270in}}%
\pgfpathclose%
\pgfusepath{fill}%
\end{pgfscope}%
\begin{pgfscope}%
\pgfpathrectangle{\pgfqpoint{1.150000in}{0.150000in}}{\pgfqpoint{5.700000in}{5.700000in}}%
\pgfusepath{clip}%
\pgfsetbuttcap%
\pgfsetroundjoin%
\definecolor{currentfill}{rgb}{0.136408,0.541173,0.554483}%
\pgfsetfillcolor{currentfill}%
\pgfsetfillopacity{0.700000}%
\pgfsetlinewidth{0.000000pt}%
\definecolor{currentstroke}{rgb}{0.000000,0.000000,0.000000}%
\pgfsetstrokecolor{currentstroke}%
\pgfsetdash{}{0pt}%
\pgfpathmoveto{\pgfqpoint{5.658798in}{2.834524in}}%
\pgfpathlineto{\pgfqpoint{5.673557in}{2.843661in}}%
\pgfpathlineto{\pgfqpoint{5.688332in}{2.852900in}}%
\pgfpathlineto{\pgfqpoint{5.703125in}{2.862241in}}%
\pgfpathlineto{\pgfqpoint{5.717935in}{2.871683in}}%
\pgfpathlineto{\pgfqpoint{5.710363in}{2.860737in}}%
\pgfpathlineto{\pgfqpoint{5.702782in}{2.849656in}}%
\pgfpathlineto{\pgfqpoint{5.695193in}{2.838438in}}%
\pgfpathlineto{\pgfqpoint{5.687595in}{2.827087in}}%
\pgfpathlineto{\pgfqpoint{5.672783in}{2.817729in}}%
\pgfpathlineto{\pgfqpoint{5.657988in}{2.808474in}}%
\pgfpathlineto{\pgfqpoint{5.643211in}{2.799320in}}%
\pgfpathlineto{\pgfqpoint{5.628450in}{2.790267in}}%
\pgfpathlineto{\pgfqpoint{5.636049in}{2.801527in}}%
\pgfpathlineto{\pgfqpoint{5.643641in}{2.812656in}}%
\pgfpathlineto{\pgfqpoint{5.651224in}{2.823656in}}%
\pgfpathlineto{\pgfqpoint{5.658798in}{2.834524in}}%
\pgfpathclose%
\pgfusepath{fill}%
\end{pgfscope}%
\begin{pgfscope}%
\pgfpathrectangle{\pgfqpoint{1.150000in}{0.150000in}}{\pgfqpoint{5.700000in}{5.700000in}}%
\pgfusepath{clip}%
\pgfsetbuttcap%
\pgfsetroundjoin%
\definecolor{currentfill}{rgb}{0.227802,0.326594,0.546532}%
\pgfsetfillcolor{currentfill}%
\pgfsetfillopacity{0.700000}%
\pgfsetlinewidth{0.000000pt}%
\definecolor{currentstroke}{rgb}{0.000000,0.000000,0.000000}%
\pgfsetstrokecolor{currentstroke}%
\pgfsetdash{}{0pt}%
\pgfpathmoveto{\pgfqpoint{2.802176in}{2.263279in}}%
\pgfpathlineto{\pgfqpoint{2.816141in}{2.248637in}}%
\pgfpathlineto{\pgfqpoint{2.830104in}{2.234135in}}%
\pgfpathlineto{\pgfqpoint{2.844064in}{2.219772in}}%
\pgfpathlineto{\pgfqpoint{2.858022in}{2.205547in}}%
\pgfpathlineto{\pgfqpoint{2.849121in}{2.212432in}}%
\pgfpathlineto{\pgfqpoint{2.840200in}{2.219677in}}%
\pgfpathlineto{\pgfqpoint{2.831258in}{2.227286in}}%
\pgfpathlineto{\pgfqpoint{2.822296in}{2.235268in}}%
\pgfpathlineto{\pgfqpoint{2.808288in}{2.250083in}}%
\pgfpathlineto{\pgfqpoint{2.794277in}{2.265037in}}%
\pgfpathlineto{\pgfqpoint{2.780264in}{2.280131in}}%
\pgfpathlineto{\pgfqpoint{2.766248in}{2.295365in}}%
\pgfpathlineto{\pgfqpoint{2.775262in}{2.286783in}}%
\pgfpathlineto{\pgfqpoint{2.784254in}{2.278579in}}%
\pgfpathlineto{\pgfqpoint{2.793225in}{2.270747in}}%
\pgfpathlineto{\pgfqpoint{2.802176in}{2.263279in}}%
\pgfpathclose%
\pgfusepath{fill}%
\end{pgfscope}%
\begin{pgfscope}%
\pgfpathrectangle{\pgfqpoint{1.150000in}{0.150000in}}{\pgfqpoint{5.700000in}{5.700000in}}%
\pgfusepath{clip}%
\pgfsetbuttcap%
\pgfsetroundjoin%
\definecolor{currentfill}{rgb}{0.269944,0.014625,0.341379}%
\pgfsetfillcolor{currentfill}%
\pgfsetfillopacity{0.700000}%
\pgfsetlinewidth{0.000000pt}%
\definecolor{currentstroke}{rgb}{0.000000,0.000000,0.000000}%
\pgfsetstrokecolor{currentstroke}%
\pgfsetdash{}{0pt}%
\pgfpathmoveto{\pgfqpoint{3.883627in}{1.617784in}}%
\pgfpathlineto{\pgfqpoint{3.897590in}{1.613216in}}%
\pgfpathlineto{\pgfqpoint{3.911559in}{1.608750in}}%
\pgfpathlineto{\pgfqpoint{3.925535in}{1.604385in}}%
\pgfpathlineto{\pgfqpoint{3.939516in}{1.600123in}}%
\pgfpathlineto{\pgfqpoint{3.931363in}{1.593852in}}%
\pgfpathlineto{\pgfqpoint{3.923202in}{1.587765in}}%
\pgfpathlineto{\pgfqpoint{3.915034in}{1.581866in}}%
\pgfpathlineto{\pgfqpoint{3.906858in}{1.576160in}}%
\pgfpathlineto{\pgfqpoint{3.892858in}{1.580909in}}%
\pgfpathlineto{\pgfqpoint{3.878865in}{1.585759in}}%
\pgfpathlineto{\pgfqpoint{3.864877in}{1.590710in}}%
\pgfpathlineto{\pgfqpoint{3.850895in}{1.595764in}}%
\pgfpathlineto{\pgfqpoint{3.859090in}{1.600977in}}%
\pgfpathlineto{\pgfqpoint{3.867276in}{1.606388in}}%
\pgfpathlineto{\pgfqpoint{3.875455in}{1.611992in}}%
\pgfpathlineto{\pgfqpoint{3.883627in}{1.617784in}}%
\pgfpathclose%
\pgfusepath{fill}%
\end{pgfscope}%
\begin{pgfscope}%
\pgfpathrectangle{\pgfqpoint{1.150000in}{0.150000in}}{\pgfqpoint{5.700000in}{5.700000in}}%
\pgfusepath{clip}%
\pgfsetbuttcap%
\pgfsetroundjoin%
\definecolor{currentfill}{rgb}{0.171176,0.452530,0.557965}%
\pgfsetfillcolor{currentfill}%
\pgfsetfillopacity{0.700000}%
\pgfsetlinewidth{0.000000pt}%
\definecolor{currentstroke}{rgb}{0.000000,0.000000,0.000000}%
\pgfsetstrokecolor{currentstroke}%
\pgfsetdash{}{0pt}%
\pgfpathmoveto{\pgfqpoint{2.522123in}{2.587147in}}%
\pgfpathlineto{\pgfqpoint{2.536169in}{2.569479in}}%
\pgfpathlineto{\pgfqpoint{2.550209in}{2.551974in}}%
\pgfpathlineto{\pgfqpoint{2.564244in}{2.534630in}}%
\pgfpathlineto{\pgfqpoint{2.578274in}{2.517445in}}%
\pgfpathlineto{\pgfqpoint{2.569111in}{2.527274in}}%
\pgfpathlineto{\pgfqpoint{2.559923in}{2.537492in}}%
\pgfpathlineto{\pgfqpoint{2.550712in}{2.548106in}}%
\pgfpathlineto{\pgfqpoint{2.541475in}{2.559122in}}%
\pgfpathlineto{\pgfqpoint{2.527387in}{2.576914in}}%
\pgfpathlineto{\pgfqpoint{2.513294in}{2.594866in}}%
\pgfpathlineto{\pgfqpoint{2.499195in}{2.612981in}}%
\pgfpathlineto{\pgfqpoint{2.485090in}{2.631259in}}%
\pgfpathlineto{\pgfqpoint{2.494386in}{2.619624in}}%
\pgfpathlineto{\pgfqpoint{2.503657in}{2.608399in}}%
\pgfpathlineto{\pgfqpoint{2.512902in}{2.597575in}}%
\pgfpathlineto{\pgfqpoint{2.522123in}{2.587147in}}%
\pgfpathclose%
\pgfusepath{fill}%
\end{pgfscope}%
\begin{pgfscope}%
\pgfpathrectangle{\pgfqpoint{1.150000in}{0.150000in}}{\pgfqpoint{5.700000in}{5.700000in}}%
\pgfusepath{clip}%
\pgfsetbuttcap%
\pgfsetroundjoin%
\definecolor{currentfill}{rgb}{0.273809,0.031497,0.358853}%
\pgfsetfillcolor{currentfill}%
\pgfsetfillopacity{0.700000}%
\pgfsetlinewidth{0.000000pt}%
\definecolor{currentstroke}{rgb}{0.000000,0.000000,0.000000}%
\pgfsetstrokecolor{currentstroke}%
\pgfsetdash{}{0pt}%
\pgfpathmoveto{\pgfqpoint{3.739228in}{1.639908in}}%
\pgfpathlineto{\pgfqpoint{3.753169in}{1.634026in}}%
\pgfpathlineto{\pgfqpoint{3.767114in}{1.628249in}}%
\pgfpathlineto{\pgfqpoint{3.781064in}{1.622576in}}%
\pgfpathlineto{\pgfqpoint{3.795020in}{1.617008in}}%
\pgfpathlineto{\pgfqpoint{3.786796in}{1.612495in}}%
\pgfpathlineto{\pgfqpoint{3.778564in}{1.608196in}}%
\pgfpathlineto{\pgfqpoint{3.770323in}{1.604114in}}%
\pgfpathlineto{\pgfqpoint{3.762072in}{1.600256in}}%
\pgfpathlineto{\pgfqpoint{3.748095in}{1.606330in}}%
\pgfpathlineto{\pgfqpoint{3.734122in}{1.612508in}}%
\pgfpathlineto{\pgfqpoint{3.720154in}{1.618790in}}%
\pgfpathlineto{\pgfqpoint{3.706190in}{1.625177in}}%
\pgfpathlineto{\pgfqpoint{3.714464in}{1.628523in}}%
\pgfpathlineto{\pgfqpoint{3.722728in}{1.632097in}}%
\pgfpathlineto{\pgfqpoint{3.730982in}{1.635893in}}%
\pgfpathlineto{\pgfqpoint{3.739228in}{1.639908in}}%
\pgfpathclose%
\pgfusepath{fill}%
\end{pgfscope}%
\begin{pgfscope}%
\pgfpathrectangle{\pgfqpoint{1.150000in}{0.150000in}}{\pgfqpoint{5.700000in}{5.700000in}}%
\pgfusepath{clip}%
\pgfsetbuttcap%
\pgfsetroundjoin%
\definecolor{currentfill}{rgb}{0.237441,0.305202,0.541921}%
\pgfsetfillcolor{currentfill}%
\pgfsetfillopacity{0.700000}%
\pgfsetlinewidth{0.000000pt}%
\definecolor{currentstroke}{rgb}{0.000000,0.000000,0.000000}%
\pgfsetstrokecolor{currentstroke}%
\pgfsetdash{}{0pt}%
\pgfpathmoveto{\pgfqpoint{2.858022in}{2.205547in}}%
\pgfpathlineto{\pgfqpoint{2.871978in}{2.191460in}}%
\pgfpathlineto{\pgfqpoint{2.885932in}{2.177509in}}%
\pgfpathlineto{\pgfqpoint{2.899884in}{2.163694in}}%
\pgfpathlineto{\pgfqpoint{2.913834in}{2.150014in}}%
\pgfpathlineto{\pgfqpoint{2.904981in}{2.156319in}}%
\pgfpathlineto{\pgfqpoint{2.896108in}{2.162978in}}%
\pgfpathlineto{\pgfqpoint{2.887216in}{2.169997in}}%
\pgfpathlineto{\pgfqpoint{2.878303in}{2.177381in}}%
\pgfpathlineto{\pgfqpoint{2.864305in}{2.191649in}}%
\pgfpathlineto{\pgfqpoint{2.850304in}{2.206052in}}%
\pgfpathlineto{\pgfqpoint{2.836301in}{2.220591in}}%
\pgfpathlineto{\pgfqpoint{2.822296in}{2.235268in}}%
\pgfpathlineto{\pgfqpoint{2.831258in}{2.227286in}}%
\pgfpathlineto{\pgfqpoint{2.840200in}{2.219677in}}%
\pgfpathlineto{\pgfqpoint{2.849121in}{2.212432in}}%
\pgfpathlineto{\pgfqpoint{2.858022in}{2.205547in}}%
\pgfpathclose%
\pgfusepath{fill}%
\end{pgfscope}%
\begin{pgfscope}%
\pgfpathrectangle{\pgfqpoint{1.150000in}{0.150000in}}{\pgfqpoint{5.700000in}{5.700000in}}%
\pgfusepath{clip}%
\pgfsetbuttcap%
\pgfsetroundjoin%
\definecolor{currentfill}{rgb}{0.160665,0.478540,0.558115}%
\pgfsetfillcolor{currentfill}%
\pgfsetfillopacity{0.700000}%
\pgfsetlinewidth{0.000000pt}%
\definecolor{currentstroke}{rgb}{0.000000,0.000000,0.000000}%
\pgfsetstrokecolor{currentstroke}%
\pgfsetdash{}{0pt}%
\pgfpathmoveto{\pgfqpoint{2.465882in}{2.659471in}}%
\pgfpathlineto{\pgfqpoint{2.479951in}{2.641140in}}%
\pgfpathlineto{\pgfqpoint{2.494014in}{2.622976in}}%
\pgfpathlineto{\pgfqpoint{2.508072in}{2.604979in}}%
\pgfpathlineto{\pgfqpoint{2.522123in}{2.587147in}}%
\pgfpathlineto{\pgfqpoint{2.512902in}{2.597575in}}%
\pgfpathlineto{\pgfqpoint{2.503657in}{2.608399in}}%
\pgfpathlineto{\pgfqpoint{2.494386in}{2.619624in}}%
\pgfpathlineto{\pgfqpoint{2.485090in}{2.631259in}}%
\pgfpathlineto{\pgfqpoint{2.470979in}{2.649702in}}%
\pgfpathlineto{\pgfqpoint{2.456862in}{2.668311in}}%
\pgfpathlineto{\pgfqpoint{2.442738in}{2.687088in}}%
\pgfpathlineto{\pgfqpoint{2.428608in}{2.706034in}}%
\pgfpathlineto{\pgfqpoint{2.437966in}{2.693777in}}%
\pgfpathlineto{\pgfqpoint{2.447297in}{2.681935in}}%
\pgfpathlineto{\pgfqpoint{2.456602in}{2.670502in}}%
\pgfpathlineto{\pgfqpoint{2.465882in}{2.659471in}}%
\pgfpathclose%
\pgfusepath{fill}%
\end{pgfscope}%
\begin{pgfscope}%
\pgfpathrectangle{\pgfqpoint{1.150000in}{0.150000in}}{\pgfqpoint{5.700000in}{5.700000in}}%
\pgfusepath{clip}%
\pgfsetbuttcap%
\pgfsetroundjoin%
\definecolor{currentfill}{rgb}{0.281446,0.084320,0.407414}%
\pgfsetfillcolor{currentfill}%
\pgfsetfillopacity{0.700000}%
\pgfsetlinewidth{0.000000pt}%
\definecolor{currentstroke}{rgb}{0.000000,0.000000,0.000000}%
\pgfsetstrokecolor{currentstroke}%
\pgfsetdash{}{0pt}%
\pgfpathmoveto{\pgfqpoint{4.437501in}{1.736498in}}%
\pgfpathlineto{\pgfqpoint{4.451626in}{1.736877in}}%
\pgfpathlineto{\pgfqpoint{4.465762in}{1.737354in}}%
\pgfpathlineto{\pgfqpoint{4.479907in}{1.737929in}}%
\pgfpathlineto{\pgfqpoint{4.494062in}{1.738601in}}%
\pgfpathlineto{\pgfqpoint{4.486093in}{1.726875in}}%
\pgfpathlineto{\pgfqpoint{4.478120in}{1.715216in}}%
\pgfpathlineto{\pgfqpoint{4.470143in}{1.703626in}}%
\pgfpathlineto{\pgfqpoint{4.462161in}{1.692110in}}%
\pgfpathlineto{\pgfqpoint{4.448001in}{1.691835in}}%
\pgfpathlineto{\pgfqpoint{4.433850in}{1.691657in}}%
\pgfpathlineto{\pgfqpoint{4.419710in}{1.691577in}}%
\pgfpathlineto{\pgfqpoint{4.405578in}{1.691595in}}%
\pgfpathlineto{\pgfqpoint{4.413566in}{1.702707in}}%
\pgfpathlineto{\pgfqpoint{4.421549in}{1.713897in}}%
\pgfpathlineto{\pgfqpoint{4.429527in}{1.725162in}}%
\pgfpathlineto{\pgfqpoint{4.437501in}{1.736498in}}%
\pgfpathclose%
\pgfusepath{fill}%
\end{pgfscope}%
\begin{pgfscope}%
\pgfpathrectangle{\pgfqpoint{1.150000in}{0.150000in}}{\pgfqpoint{5.700000in}{5.700000in}}%
\pgfusepath{clip}%
\pgfsetbuttcap%
\pgfsetroundjoin%
\definecolor{currentfill}{rgb}{0.269944,0.014625,0.341379}%
\pgfsetfillcolor{currentfill}%
\pgfsetfillopacity{0.700000}%
\pgfsetlinewidth{0.000000pt}%
\definecolor{currentstroke}{rgb}{0.000000,0.000000,0.000000}%
\pgfsetstrokecolor{currentstroke}%
\pgfsetdash{}{0pt}%
\pgfpathmoveto{\pgfqpoint{4.027982in}{1.612776in}}%
\pgfpathlineto{\pgfqpoint{4.041980in}{1.609485in}}%
\pgfpathlineto{\pgfqpoint{4.055985in}{1.606295in}}%
\pgfpathlineto{\pgfqpoint{4.069998in}{1.603205in}}%
\pgfpathlineto{\pgfqpoint{4.084017in}{1.600215in}}%
\pgfpathlineto{\pgfqpoint{4.075921in}{1.592337in}}%
\pgfpathlineto{\pgfqpoint{4.067819in}{1.584615in}}%
\pgfpathlineto{\pgfqpoint{4.059711in}{1.577053in}}%
\pgfpathlineto{\pgfqpoint{4.051597in}{1.569655in}}%
\pgfpathlineto{\pgfqpoint{4.037563in}{1.573113in}}%
\pgfpathlineto{\pgfqpoint{4.023537in}{1.576670in}}%
\pgfpathlineto{\pgfqpoint{4.009517in}{1.580327in}}%
\pgfpathlineto{\pgfqpoint{3.995504in}{1.584085in}}%
\pgfpathlineto{\pgfqpoint{4.003634in}{1.591008in}}%
\pgfpathlineto{\pgfqpoint{4.011757in}{1.598101in}}%
\pgfpathlineto{\pgfqpoint{4.019873in}{1.605359in}}%
\pgfpathlineto{\pgfqpoint{4.027982in}{1.612776in}}%
\pgfpathclose%
\pgfusepath{fill}%
\end{pgfscope}%
\begin{pgfscope}%
\pgfpathrectangle{\pgfqpoint{1.150000in}{0.150000in}}{\pgfqpoint{5.700000in}{5.700000in}}%
\pgfusepath{clip}%
\pgfsetbuttcap%
\pgfsetroundjoin%
\definecolor{currentfill}{rgb}{0.283091,0.110553,0.431554}%
\pgfsetfillcolor{currentfill}%
\pgfsetfillopacity{0.700000}%
\pgfsetlinewidth{0.000000pt}%
\definecolor{currentstroke}{rgb}{0.000000,0.000000,0.000000}%
\pgfsetstrokecolor{currentstroke}%
\pgfsetdash{}{0pt}%
\pgfpathmoveto{\pgfqpoint{4.525893in}{1.786095in}}%
\pgfpathlineto{\pgfqpoint{4.540054in}{1.787245in}}%
\pgfpathlineto{\pgfqpoint{4.554225in}{1.788492in}}%
\pgfpathlineto{\pgfqpoint{4.568407in}{1.789837in}}%
\pgfpathlineto{\pgfqpoint{4.582600in}{1.791280in}}%
\pgfpathlineto{\pgfqpoint{4.574652in}{1.778953in}}%
\pgfpathlineto{\pgfqpoint{4.566700in}{1.766672in}}%
\pgfpathlineto{\pgfqpoint{4.558744in}{1.754442in}}%
\pgfpathlineto{\pgfqpoint{4.550783in}{1.742267in}}%
\pgfpathlineto{\pgfqpoint{4.536588in}{1.741205in}}%
\pgfpathlineto{\pgfqpoint{4.522402in}{1.740239in}}%
\pgfpathlineto{\pgfqpoint{4.508227in}{1.739371in}}%
\pgfpathlineto{\pgfqpoint{4.494062in}{1.738601in}}%
\pgfpathlineto{\pgfqpoint{4.502026in}{1.750390in}}%
\pgfpathlineto{\pgfqpoint{4.509986in}{1.762238in}}%
\pgfpathlineto{\pgfqpoint{4.517942in}{1.774141in}}%
\pgfpathlineto{\pgfqpoint{4.525893in}{1.786095in}}%
\pgfpathclose%
\pgfusepath{fill}%
\end{pgfscope}%
\begin{pgfscope}%
\pgfpathrectangle{\pgfqpoint{1.150000in}{0.150000in}}{\pgfqpoint{5.700000in}{5.700000in}}%
\pgfusepath{clip}%
\pgfsetbuttcap%
\pgfsetroundjoin%
\definecolor{currentfill}{rgb}{0.221989,0.339161,0.548752}%
\pgfsetfillcolor{currentfill}%
\pgfsetfillopacity{0.700000}%
\pgfsetlinewidth{0.000000pt}%
\definecolor{currentstroke}{rgb}{0.000000,0.000000,0.000000}%
\pgfsetstrokecolor{currentstroke}%
\pgfsetdash{}{0pt}%
\pgfpathmoveto{\pgfqpoint{5.120746in}{2.289193in}}%
\pgfpathlineto{\pgfqpoint{5.135196in}{2.295102in}}%
\pgfpathlineto{\pgfqpoint{5.149660in}{2.301109in}}%
\pgfpathlineto{\pgfqpoint{5.164138in}{2.307216in}}%
\pgfpathlineto{\pgfqpoint{5.178631in}{2.313423in}}%
\pgfpathlineto{\pgfqpoint{5.170831in}{2.299810in}}%
\pgfpathlineto{\pgfqpoint{5.163026in}{2.286128in}}%
\pgfpathlineto{\pgfqpoint{5.155216in}{2.272377in}}%
\pgfpathlineto{\pgfqpoint{5.147400in}{2.258561in}}%
\pgfpathlineto{\pgfqpoint{5.132909in}{2.252593in}}%
\pgfpathlineto{\pgfqpoint{5.118432in}{2.246724in}}%
\pgfpathlineto{\pgfqpoint{5.103970in}{2.240954in}}%
\pgfpathlineto{\pgfqpoint{5.089521in}{2.235283in}}%
\pgfpathlineto{\pgfqpoint{5.097335in}{2.248855in}}%
\pgfpathlineto{\pgfqpoint{5.105144in}{2.262365in}}%
\pgfpathlineto{\pgfqpoint{5.112948in}{2.275812in}}%
\pgfpathlineto{\pgfqpoint{5.120746in}{2.289193in}}%
\pgfpathclose%
\pgfusepath{fill}%
\end{pgfscope}%
\begin{pgfscope}%
\pgfpathrectangle{\pgfqpoint{1.150000in}{0.150000in}}{\pgfqpoint{5.700000in}{5.700000in}}%
\pgfusepath{clip}%
\pgfsetbuttcap%
\pgfsetroundjoin%
\definecolor{currentfill}{rgb}{0.278791,0.062145,0.386592}%
\pgfsetfillcolor{currentfill}%
\pgfsetfillopacity{0.700000}%
\pgfsetlinewidth{0.000000pt}%
\definecolor{currentstroke}{rgb}{0.000000,0.000000,0.000000}%
\pgfsetstrokecolor{currentstroke}%
\pgfsetdash{}{0pt}%
\pgfpathmoveto{\pgfqpoint{4.349146in}{1.692646in}}%
\pgfpathlineto{\pgfqpoint{4.363240in}{1.692236in}}%
\pgfpathlineto{\pgfqpoint{4.377344in}{1.691924in}}%
\pgfpathlineto{\pgfqpoint{4.391456in}{1.691711in}}%
\pgfpathlineto{\pgfqpoint{4.405578in}{1.691595in}}%
\pgfpathlineto{\pgfqpoint{4.397586in}{1.680566in}}%
\pgfpathlineto{\pgfqpoint{4.389589in}{1.669622in}}%
\pgfpathlineto{\pgfqpoint{4.381588in}{1.658769in}}%
\pgfpathlineto{\pgfqpoint{4.373582in}{1.648010in}}%
\pgfpathlineto{\pgfqpoint{4.359453in}{1.648540in}}%
\pgfpathlineto{\pgfqpoint{4.345334in}{1.649168in}}%
\pgfpathlineto{\pgfqpoint{4.331223in}{1.649894in}}%
\pgfpathlineto{\pgfqpoint{4.317121in}{1.650718in}}%
\pgfpathlineto{\pgfqpoint{4.325135in}{1.661056in}}%
\pgfpathlineto{\pgfqpoint{4.333144in}{1.671492in}}%
\pgfpathlineto{\pgfqpoint{4.341147in}{1.682024in}}%
\pgfpathlineto{\pgfqpoint{4.349146in}{1.692646in}}%
\pgfpathclose%
\pgfusepath{fill}%
\end{pgfscope}%
\begin{pgfscope}%
\pgfpathrectangle{\pgfqpoint{1.150000in}{0.150000in}}{\pgfqpoint{5.700000in}{5.700000in}}%
\pgfusepath{clip}%
\pgfsetbuttcap%
\pgfsetroundjoin%
\definecolor{currentfill}{rgb}{0.258965,0.251537,0.524736}%
\pgfsetfillcolor{currentfill}%
\pgfsetfillopacity{0.700000}%
\pgfsetlinewidth{0.000000pt}%
\definecolor{currentstroke}{rgb}{0.000000,0.000000,0.000000}%
\pgfsetstrokecolor{currentstroke}%
\pgfsetdash{}{0pt}%
\pgfpathmoveto{\pgfqpoint{4.911741in}{2.087215in}}%
\pgfpathlineto{\pgfqpoint{4.926081in}{2.091566in}}%
\pgfpathlineto{\pgfqpoint{4.940433in}{2.096016in}}%
\pgfpathlineto{\pgfqpoint{4.954799in}{2.100564in}}%
\pgfpathlineto{\pgfqpoint{4.969177in}{2.105211in}}%
\pgfpathlineto{\pgfqpoint{4.961319in}{2.091479in}}%
\pgfpathlineto{\pgfqpoint{4.953457in}{2.077714in}}%
\pgfpathlineto{\pgfqpoint{4.945589in}{2.063919in}}%
\pgfpathlineto{\pgfqpoint{4.937718in}{2.050097in}}%
\pgfpathlineto{\pgfqpoint{4.923340in}{2.045744in}}%
\pgfpathlineto{\pgfqpoint{4.908976in}{2.041488in}}%
\pgfpathlineto{\pgfqpoint{4.894624in}{2.037330in}}%
\pgfpathlineto{\pgfqpoint{4.880285in}{2.033271in}}%
\pgfpathlineto{\pgfqpoint{4.888156in}{2.046793in}}%
\pgfpathlineto{\pgfqpoint{4.896022in}{2.060294in}}%
\pgfpathlineto{\pgfqpoint{4.903884in}{2.073768in}}%
\pgfpathlineto{\pgfqpoint{4.911741in}{2.087215in}}%
\pgfpathclose%
\pgfusepath{fill}%
\end{pgfscope}%
\begin{pgfscope}%
\pgfpathrectangle{\pgfqpoint{1.150000in}{0.150000in}}{\pgfqpoint{5.700000in}{5.700000in}}%
\pgfusepath{clip}%
\pgfsetbuttcap%
\pgfsetroundjoin%
\definecolor{currentfill}{rgb}{0.246811,0.283237,0.535941}%
\pgfsetfillcolor{currentfill}%
\pgfsetfillopacity{0.700000}%
\pgfsetlinewidth{0.000000pt}%
\definecolor{currentstroke}{rgb}{0.000000,0.000000,0.000000}%
\pgfsetstrokecolor{currentstroke}%
\pgfsetdash{}{0pt}%
\pgfpathmoveto{\pgfqpoint{2.913834in}{2.150014in}}%
\pgfpathlineto{\pgfqpoint{2.927783in}{2.136468in}}%
\pgfpathlineto{\pgfqpoint{2.941730in}{2.123055in}}%
\pgfpathlineto{\pgfqpoint{2.955676in}{2.109774in}}%
\pgfpathlineto{\pgfqpoint{2.969620in}{2.096624in}}%
\pgfpathlineto{\pgfqpoint{2.960813in}{2.102353in}}%
\pgfpathlineto{\pgfqpoint{2.951987in}{2.108429in}}%
\pgfpathlineto{\pgfqpoint{2.943142in}{2.114858in}}%
\pgfpathlineto{\pgfqpoint{2.934278in}{2.121649in}}%
\pgfpathlineto{\pgfqpoint{2.920287in}{2.135383in}}%
\pgfpathlineto{\pgfqpoint{2.906294in}{2.149249in}}%
\pgfpathlineto{\pgfqpoint{2.892300in}{2.163248in}}%
\pgfpathlineto{\pgfqpoint{2.878303in}{2.177381in}}%
\pgfpathlineto{\pgfqpoint{2.887216in}{2.169997in}}%
\pgfpathlineto{\pgfqpoint{2.896108in}{2.162978in}}%
\pgfpathlineto{\pgfqpoint{2.904981in}{2.156319in}}%
\pgfpathlineto{\pgfqpoint{2.913834in}{2.150014in}}%
\pgfpathclose%
\pgfusepath{fill}%
\end{pgfscope}%
\begin{pgfscope}%
\pgfpathrectangle{\pgfqpoint{1.150000in}{0.150000in}}{\pgfqpoint{5.700000in}{5.700000in}}%
\pgfusepath{clip}%
\pgfsetbuttcap%
\pgfsetroundjoin%
\definecolor{currentfill}{rgb}{0.126453,0.570633,0.549841}%
\pgfsetfillcolor{currentfill}%
\pgfsetfillopacity{0.700000}%
\pgfsetlinewidth{0.000000pt}%
\definecolor{currentstroke}{rgb}{0.000000,0.000000,0.000000}%
\pgfsetstrokecolor{currentstroke}%
\pgfsetdash{}{0pt}%
\pgfpathmoveto{\pgfqpoint{5.748133in}{2.914102in}}%
\pgfpathlineto{\pgfqpoint{5.762957in}{2.923711in}}%
\pgfpathlineto{\pgfqpoint{5.777798in}{2.933423in}}%
\pgfpathlineto{\pgfqpoint{5.792657in}{2.943237in}}%
\pgfpathlineto{\pgfqpoint{5.785123in}{2.932795in}}%
\pgfpathlineto{\pgfqpoint{5.777580in}{2.922212in}}%
\pgfpathlineto{\pgfqpoint{5.770029in}{2.911488in}}%
\pgfpathlineto{\pgfqpoint{5.762467in}{2.900623in}}%
\pgfpathlineto{\pgfqpoint{5.747606in}{2.890875in}}%
\pgfpathlineto{\pgfqpoint{5.732762in}{2.881228in}}%
\pgfpathlineto{\pgfqpoint{5.717935in}{2.871683in}}%
\pgfpathlineto{\pgfqpoint{5.725498in}{2.882493in}}%
\pgfpathlineto{\pgfqpoint{5.733052in}{2.893167in}}%
\pgfpathlineto{\pgfqpoint{5.740597in}{2.903703in}}%
\pgfpathlineto{\pgfqpoint{5.748133in}{2.914102in}}%
\pgfpathclose%
\pgfusepath{fill}%
\end{pgfscope}%
\begin{pgfscope}%
\pgfpathrectangle{\pgfqpoint{1.150000in}{0.150000in}}{\pgfqpoint{5.700000in}{5.700000in}}%
\pgfusepath{clip}%
\pgfsetbuttcap%
\pgfsetroundjoin%
\definecolor{currentfill}{rgb}{0.283187,0.125848,0.444960}%
\pgfsetfillcolor{currentfill}%
\pgfsetfillopacity{0.700000}%
\pgfsetlinewidth{0.000000pt}%
\definecolor{currentstroke}{rgb}{0.000000,0.000000,0.000000}%
\pgfsetstrokecolor{currentstroke}%
\pgfsetdash{}{0pt}%
\pgfpathmoveto{\pgfqpoint{3.338308in}{1.812399in}}%
\pgfpathlineto{\pgfqpoint{3.352223in}{1.802862in}}%
\pgfpathlineto{\pgfqpoint{3.366140in}{1.793440in}}%
\pgfpathlineto{\pgfqpoint{3.380058in}{1.784131in}}%
\pgfpathlineto{\pgfqpoint{3.393979in}{1.774937in}}%
\pgfpathlineto{\pgfqpoint{3.385508in}{1.775506in}}%
\pgfpathlineto{\pgfqpoint{3.377024in}{1.776362in}}%
\pgfpathlineto{\pgfqpoint{3.368526in}{1.777509in}}%
\pgfpathlineto{\pgfqpoint{3.360015in}{1.778953in}}%
\pgfpathlineto{\pgfqpoint{3.346060in}{1.788698in}}%
\pgfpathlineto{\pgfqpoint{3.332107in}{1.798556in}}%
\pgfpathlineto{\pgfqpoint{3.318156in}{1.808529in}}%
\pgfpathlineto{\pgfqpoint{3.304206in}{1.818616in}}%
\pgfpathlineto{\pgfqpoint{3.312753in}{1.816614in}}%
\pgfpathlineto{\pgfqpoint{3.321285in}{1.814914in}}%
\pgfpathlineto{\pgfqpoint{3.329804in}{1.813511in}}%
\pgfpathlineto{\pgfqpoint{3.338308in}{1.812399in}}%
\pgfpathclose%
\pgfusepath{fill}%
\end{pgfscope}%
\begin{pgfscope}%
\pgfpathrectangle{\pgfqpoint{1.150000in}{0.150000in}}{\pgfqpoint{5.700000in}{5.700000in}}%
\pgfusepath{clip}%
\pgfsetbuttcap%
\pgfsetroundjoin%
\definecolor{currentfill}{rgb}{0.150476,0.504369,0.557430}%
\pgfsetfillcolor{currentfill}%
\pgfsetfillopacity{0.700000}%
\pgfsetlinewidth{0.000000pt}%
\definecolor{currentstroke}{rgb}{0.000000,0.000000,0.000000}%
\pgfsetstrokecolor{currentstroke}%
\pgfsetdash{}{0pt}%
\pgfpathmoveto{\pgfqpoint{2.409540in}{2.734505in}}%
\pgfpathlineto{\pgfqpoint{2.423636in}{2.715487in}}%
\pgfpathlineto{\pgfqpoint{2.437725in}{2.696644in}}%
\pgfpathlineto{\pgfqpoint{2.451807in}{2.677972in}}%
\pgfpathlineto{\pgfqpoint{2.465882in}{2.659471in}}%
\pgfpathlineto{\pgfqpoint{2.456602in}{2.670502in}}%
\pgfpathlineto{\pgfqpoint{2.447297in}{2.681935in}}%
\pgfpathlineto{\pgfqpoint{2.437966in}{2.693777in}}%
\pgfpathlineto{\pgfqpoint{2.428608in}{2.706034in}}%
\pgfpathlineto{\pgfqpoint{2.414471in}{2.725150in}}%
\pgfpathlineto{\pgfqpoint{2.400327in}{2.744439in}}%
\pgfpathlineto{\pgfqpoint{2.386176in}{2.763900in}}%
\pgfpathlineto{\pgfqpoint{2.372018in}{2.783537in}}%
\pgfpathlineto{\pgfqpoint{2.381439in}{2.770653in}}%
\pgfpathlineto{\pgfqpoint{2.390833in}{2.758191in}}%
\pgfpathlineto{\pgfqpoint{2.400200in}{2.746144in}}%
\pgfpathlineto{\pgfqpoint{2.409540in}{2.734505in}}%
\pgfpathclose%
\pgfusepath{fill}%
\end{pgfscope}%
\begin{pgfscope}%
\pgfpathrectangle{\pgfqpoint{1.150000in}{0.150000in}}{\pgfqpoint{5.700000in}{5.700000in}}%
\pgfusepath{clip}%
\pgfsetbuttcap%
\pgfsetroundjoin%
\definecolor{currentfill}{rgb}{0.183898,0.422383,0.556944}%
\pgfsetfillcolor{currentfill}%
\pgfsetfillopacity{0.700000}%
\pgfsetlinewidth{0.000000pt}%
\definecolor{currentstroke}{rgb}{0.000000,0.000000,0.000000}%
\pgfsetstrokecolor{currentstroke}%
\pgfsetdash{}{0pt}%
\pgfpathmoveto{\pgfqpoint{5.329895in}{2.498838in}}%
\pgfpathlineto{\pgfqpoint{5.344465in}{2.506145in}}%
\pgfpathlineto{\pgfqpoint{5.359050in}{2.513552in}}%
\pgfpathlineto{\pgfqpoint{5.373651in}{2.521060in}}%
\pgfpathlineto{\pgfqpoint{5.388267in}{2.528667in}}%
\pgfpathlineto{\pgfqpoint{5.380539in}{2.515692in}}%
\pgfpathlineto{\pgfqpoint{5.372803in}{2.502615in}}%
\pgfpathlineto{\pgfqpoint{5.365061in}{2.489440in}}%
\pgfpathlineto{\pgfqpoint{5.357312in}{2.476166in}}%
\pgfpathlineto{\pgfqpoint{5.342698in}{2.468741in}}%
\pgfpathlineto{\pgfqpoint{5.328098in}{2.461415in}}%
\pgfpathlineto{\pgfqpoint{5.313514in}{2.454190in}}%
\pgfpathlineto{\pgfqpoint{5.298945in}{2.447065in}}%
\pgfpathlineto{\pgfqpoint{5.306692in}{2.460150in}}%
\pgfpathlineto{\pgfqpoint{5.314433in}{2.473141in}}%
\pgfpathlineto{\pgfqpoint{5.322167in}{2.486038in}}%
\pgfpathlineto{\pgfqpoint{5.329895in}{2.498838in}}%
\pgfpathclose%
\pgfusepath{fill}%
\end{pgfscope}%
\begin{pgfscope}%
\pgfpathrectangle{\pgfqpoint{1.150000in}{0.150000in}}{\pgfqpoint{5.700000in}{5.700000in}}%
\pgfusepath{clip}%
\pgfsetbuttcap%
\pgfsetroundjoin%
\definecolor{currentfill}{rgb}{0.282623,0.140926,0.457517}%
\pgfsetfillcolor{currentfill}%
\pgfsetfillopacity{0.700000}%
\pgfsetlinewidth{0.000000pt}%
\definecolor{currentstroke}{rgb}{0.000000,0.000000,0.000000}%
\pgfsetstrokecolor{currentstroke}%
\pgfsetdash{}{0pt}%
\pgfpathmoveto{\pgfqpoint{4.614347in}{1.840994in}}%
\pgfpathlineto{\pgfqpoint{4.628547in}{1.842896in}}%
\pgfpathlineto{\pgfqpoint{4.642759in}{1.844897in}}%
\pgfpathlineto{\pgfqpoint{4.656981in}{1.846995in}}%
\pgfpathlineto{\pgfqpoint{4.671214in}{1.849191in}}%
\pgfpathlineto{\pgfqpoint{4.663286in}{1.836353in}}%
\pgfpathlineto{\pgfqpoint{4.655354in}{1.823544in}}%
\pgfpathlineto{\pgfqpoint{4.647417in}{1.810768in}}%
\pgfpathlineto{\pgfqpoint{4.639476in}{1.798027in}}%
\pgfpathlineto{\pgfqpoint{4.625241in}{1.796194in}}%
\pgfpathlineto{\pgfqpoint{4.611016in}{1.794459in}}%
\pgfpathlineto{\pgfqpoint{4.596803in}{1.792821in}}%
\pgfpathlineto{\pgfqpoint{4.582600in}{1.791280in}}%
\pgfpathlineto{\pgfqpoint{4.590543in}{1.803652in}}%
\pgfpathlineto{\pgfqpoint{4.598482in}{1.816063in}}%
\pgfpathlineto{\pgfqpoint{4.606417in}{1.828512in}}%
\pgfpathlineto{\pgfqpoint{4.614347in}{1.840994in}}%
\pgfpathclose%
\pgfusepath{fill}%
\end{pgfscope}%
\begin{pgfscope}%
\pgfpathrectangle{\pgfqpoint{1.150000in}{0.150000in}}{\pgfqpoint{5.700000in}{5.700000in}}%
\pgfusepath{clip}%
\pgfsetbuttcap%
\pgfsetroundjoin%
\definecolor{currentfill}{rgb}{0.276022,0.044167,0.370164}%
\pgfsetfillcolor{currentfill}%
\pgfsetfillopacity{0.700000}%
\pgfsetlinewidth{0.000000pt}%
\definecolor{currentstroke}{rgb}{0.000000,0.000000,0.000000}%
\pgfsetstrokecolor{currentstroke}%
\pgfsetdash{}{0pt}%
\pgfpathmoveto{\pgfqpoint{4.260802in}{1.654996in}}%
\pgfpathlineto{\pgfqpoint{4.274869in}{1.653779in}}%
\pgfpathlineto{\pgfqpoint{4.288944in}{1.652660in}}%
\pgfpathlineto{\pgfqpoint{4.303029in}{1.651640in}}%
\pgfpathlineto{\pgfqpoint{4.317121in}{1.650718in}}%
\pgfpathlineto{\pgfqpoint{4.309103in}{1.640483in}}%
\pgfpathlineto{\pgfqpoint{4.301079in}{1.630354in}}%
\pgfpathlineto{\pgfqpoint{4.293051in}{1.620337in}}%
\pgfpathlineto{\pgfqpoint{4.285017in}{1.610435in}}%
\pgfpathlineto{\pgfqpoint{4.270916in}{1.611789in}}%
\pgfpathlineto{\pgfqpoint{4.256823in}{1.613241in}}%
\pgfpathlineto{\pgfqpoint{4.242738in}{1.614792in}}%
\pgfpathlineto{\pgfqpoint{4.228662in}{1.616440in}}%
\pgfpathlineto{\pgfqpoint{4.236705in}{1.625904in}}%
\pgfpathlineto{\pgfqpoint{4.244742in}{1.635487in}}%
\pgfpathlineto{\pgfqpoint{4.252775in}{1.645186in}}%
\pgfpathlineto{\pgfqpoint{4.260802in}{1.654996in}}%
\pgfpathclose%
\pgfusepath{fill}%
\end{pgfscope}%
\begin{pgfscope}%
\pgfpathrectangle{\pgfqpoint{1.150000in}{0.150000in}}{\pgfqpoint{5.700000in}{5.700000in}}%
\pgfusepath{clip}%
\pgfsetbuttcap%
\pgfsetroundjoin%
\definecolor{currentfill}{rgb}{0.151918,0.500685,0.557587}%
\pgfsetfillcolor{currentfill}%
\pgfsetfillopacity{0.700000}%
\pgfsetlinewidth{0.000000pt}%
\definecolor{currentstroke}{rgb}{0.000000,0.000000,0.000000}%
\pgfsetstrokecolor{currentstroke}%
\pgfsetdash{}{0pt}%
\pgfpathmoveto{\pgfqpoint{5.539085in}{2.709170in}}%
\pgfpathlineto{\pgfqpoint{5.553781in}{2.717713in}}%
\pgfpathlineto{\pgfqpoint{5.568493in}{2.726356in}}%
\pgfpathlineto{\pgfqpoint{5.583222in}{2.735101in}}%
\pgfpathlineto{\pgfqpoint{5.597968in}{2.743947in}}%
\pgfpathlineto{\pgfqpoint{5.590327in}{2.732050in}}%
\pgfpathlineto{\pgfqpoint{5.582678in}{2.720028in}}%
\pgfpathlineto{\pgfqpoint{5.575021in}{2.707881in}}%
\pgfpathlineto{\pgfqpoint{5.567356in}{2.695611in}}%
\pgfpathlineto{\pgfqpoint{5.552611in}{2.686890in}}%
\pgfpathlineto{\pgfqpoint{5.537882in}{2.678269in}}%
\pgfpathlineto{\pgfqpoint{5.523169in}{2.669750in}}%
\pgfpathlineto{\pgfqpoint{5.508472in}{2.661332in}}%
\pgfpathlineto{\pgfqpoint{5.516137in}{2.673470in}}%
\pgfpathlineto{\pgfqpoint{5.523794in}{2.685490in}}%
\pgfpathlineto{\pgfqpoint{5.531443in}{2.697390in}}%
\pgfpathlineto{\pgfqpoint{5.539085in}{2.709170in}}%
\pgfpathclose%
\pgfusepath{fill}%
\end{pgfscope}%
\begin{pgfscope}%
\pgfpathrectangle{\pgfqpoint{1.150000in}{0.150000in}}{\pgfqpoint{5.700000in}{5.700000in}}%
\pgfusepath{clip}%
\pgfsetbuttcap%
\pgfsetroundjoin%
\definecolor{currentfill}{rgb}{0.255645,0.260703,0.528312}%
\pgfsetfillcolor{currentfill}%
\pgfsetfillopacity{0.700000}%
\pgfsetlinewidth{0.000000pt}%
\definecolor{currentstroke}{rgb}{0.000000,0.000000,0.000000}%
\pgfsetstrokecolor{currentstroke}%
\pgfsetdash{}{0pt}%
\pgfpathmoveto{\pgfqpoint{2.969620in}{2.096624in}}%
\pgfpathlineto{\pgfqpoint{2.983563in}{2.083606in}}%
\pgfpathlineto{\pgfqpoint{2.997506in}{2.070717in}}%
\pgfpathlineto{\pgfqpoint{3.011447in}{2.057957in}}%
\pgfpathlineto{\pgfqpoint{3.025388in}{2.045326in}}%
\pgfpathlineto{\pgfqpoint{3.016625in}{2.050480in}}%
\pgfpathlineto{\pgfqpoint{3.007845in}{2.055975in}}%
\pgfpathlineto{\pgfqpoint{2.999046in}{2.061819in}}%
\pgfpathlineto{\pgfqpoint{2.990228in}{2.068017in}}%
\pgfpathlineto{\pgfqpoint{2.976243in}{2.081231in}}%
\pgfpathlineto{\pgfqpoint{2.962256in}{2.094573in}}%
\pgfpathlineto{\pgfqpoint{2.948268in}{2.108046in}}%
\pgfpathlineto{\pgfqpoint{2.934278in}{2.121649in}}%
\pgfpathlineto{\pgfqpoint{2.943142in}{2.114858in}}%
\pgfpathlineto{\pgfqpoint{2.951987in}{2.108429in}}%
\pgfpathlineto{\pgfqpoint{2.960813in}{2.102353in}}%
\pgfpathlineto{\pgfqpoint{2.969620in}{2.096624in}}%
\pgfpathclose%
\pgfusepath{fill}%
\end{pgfscope}%
\begin{pgfscope}%
\pgfpathrectangle{\pgfqpoint{1.150000in}{0.150000in}}{\pgfqpoint{5.700000in}{5.700000in}}%
\pgfusepath{clip}%
\pgfsetbuttcap%
\pgfsetroundjoin%
\definecolor{currentfill}{rgb}{0.278791,0.062145,0.386592}%
\pgfsetfillcolor{currentfill}%
\pgfsetfillopacity{0.700000}%
\pgfsetlinewidth{0.000000pt}%
\definecolor{currentstroke}{rgb}{0.000000,0.000000,0.000000}%
\pgfsetstrokecolor{currentstroke}%
\pgfsetdash{}{0pt}%
\pgfpathmoveto{\pgfqpoint{3.594635in}{1.680073in}}%
\pgfpathlineto{\pgfqpoint{3.608566in}{1.672838in}}%
\pgfpathlineto{\pgfqpoint{3.622500in}{1.665711in}}%
\pgfpathlineto{\pgfqpoint{3.636438in}{1.658690in}}%
\pgfpathlineto{\pgfqpoint{3.650380in}{1.651775in}}%
\pgfpathlineto{\pgfqpoint{3.642072in}{1.649180in}}%
\pgfpathlineto{\pgfqpoint{3.633754in}{1.646828in}}%
\pgfpathlineto{\pgfqpoint{3.625425in}{1.644725in}}%
\pgfpathlineto{\pgfqpoint{3.617086in}{1.642876in}}%
\pgfpathlineto{\pgfqpoint{3.603117in}{1.650316in}}%
\pgfpathlineto{\pgfqpoint{3.589152in}{1.657863in}}%
\pgfpathlineto{\pgfqpoint{3.575191in}{1.665517in}}%
\pgfpathlineto{\pgfqpoint{3.561233in}{1.673278in}}%
\pgfpathlineto{\pgfqpoint{3.569600in}{1.674593in}}%
\pgfpathlineto{\pgfqpoint{3.577956in}{1.676168in}}%
\pgfpathlineto{\pgfqpoint{3.586301in}{1.677997in}}%
\pgfpathlineto{\pgfqpoint{3.594635in}{1.680073in}}%
\pgfpathclose%
\pgfusepath{fill}%
\end{pgfscope}%
\begin{pgfscope}%
\pgfpathrectangle{\pgfqpoint{1.150000in}{0.150000in}}{\pgfqpoint{5.700000in}{5.700000in}}%
\pgfusepath{clip}%
\pgfsetbuttcap%
\pgfsetroundjoin%
\definecolor{currentfill}{rgb}{0.139147,0.533812,0.555298}%
\pgfsetfillcolor{currentfill}%
\pgfsetfillopacity{0.700000}%
\pgfsetlinewidth{0.000000pt}%
\definecolor{currentstroke}{rgb}{0.000000,0.000000,0.000000}%
\pgfsetstrokecolor{currentstroke}%
\pgfsetdash{}{0pt}%
\pgfpathmoveto{\pgfqpoint{2.353086in}{2.812344in}}%
\pgfpathlineto{\pgfqpoint{2.367211in}{2.792616in}}%
\pgfpathlineto{\pgfqpoint{2.381328in}{2.773068in}}%
\pgfpathlineto{\pgfqpoint{2.395438in}{2.753698in}}%
\pgfpathlineto{\pgfqpoint{2.409540in}{2.734505in}}%
\pgfpathlineto{\pgfqpoint{2.400200in}{2.746144in}}%
\pgfpathlineto{\pgfqpoint{2.390833in}{2.758191in}}%
\pgfpathlineto{\pgfqpoint{2.381439in}{2.770653in}}%
\pgfpathlineto{\pgfqpoint{2.372018in}{2.783537in}}%
\pgfpathlineto{\pgfqpoint{2.357852in}{2.803350in}}%
\pgfpathlineto{\pgfqpoint{2.343679in}{2.823341in}}%
\pgfpathlineto{\pgfqpoint{2.329498in}{2.843512in}}%
\pgfpathlineto{\pgfqpoint{2.315309in}{2.863864in}}%
\pgfpathlineto{\pgfqpoint{2.324795in}{2.850348in}}%
\pgfpathlineto{\pgfqpoint{2.334253in}{2.837260in}}%
\pgfpathlineto{\pgfqpoint{2.343683in}{2.824595in}}%
\pgfpathlineto{\pgfqpoint{2.353086in}{2.812344in}}%
\pgfpathclose%
\pgfusepath{fill}%
\end{pgfscope}%
\begin{pgfscope}%
\pgfpathrectangle{\pgfqpoint{1.150000in}{0.150000in}}{\pgfqpoint{5.700000in}{5.700000in}}%
\pgfusepath{clip}%
\pgfsetbuttcap%
\pgfsetroundjoin%
\definecolor{currentfill}{rgb}{0.279574,0.170599,0.479997}%
\pgfsetfillcolor{currentfill}%
\pgfsetfillopacity{0.700000}%
\pgfsetlinewidth{0.000000pt}%
\definecolor{currentstroke}{rgb}{0.000000,0.000000,0.000000}%
\pgfsetstrokecolor{currentstroke}%
\pgfsetdash{}{0pt}%
\pgfpathmoveto{\pgfqpoint{4.702885in}{1.900764in}}%
\pgfpathlineto{\pgfqpoint{4.717128in}{1.903403in}}%
\pgfpathlineto{\pgfqpoint{4.731383in}{1.906139in}}%
\pgfpathlineto{\pgfqpoint{4.745649in}{1.908973in}}%
\pgfpathlineto{\pgfqpoint{4.759927in}{1.911905in}}%
\pgfpathlineto{\pgfqpoint{4.752017in}{1.898646in}}%
\pgfpathlineto{\pgfqpoint{4.744102in}{1.885398in}}%
\pgfpathlineto{\pgfqpoint{4.736184in}{1.872165in}}%
\pgfpathlineto{\pgfqpoint{4.728261in}{1.858950in}}%
\pgfpathlineto{\pgfqpoint{4.713982in}{1.856364in}}%
\pgfpathlineto{\pgfqpoint{4.699715in}{1.853875in}}%
\pgfpathlineto{\pgfqpoint{4.685459in}{1.851484in}}%
\pgfpathlineto{\pgfqpoint{4.671214in}{1.849191in}}%
\pgfpathlineto{\pgfqpoint{4.679138in}{1.862054in}}%
\pgfpathlineto{\pgfqpoint{4.687058in}{1.874939in}}%
\pgfpathlineto{\pgfqpoint{4.694974in}{1.887844in}}%
\pgfpathlineto{\pgfqpoint{4.702885in}{1.900764in}}%
\pgfpathclose%
\pgfusepath{fill}%
\end{pgfscope}%
\begin{pgfscope}%
\pgfpathrectangle{\pgfqpoint{1.150000in}{0.150000in}}{\pgfqpoint{5.700000in}{5.700000in}}%
\pgfusepath{clip}%
\pgfsetbuttcap%
\pgfsetroundjoin%
\definecolor{currentfill}{rgb}{0.272594,0.025563,0.353093}%
\pgfsetfillcolor{currentfill}%
\pgfsetfillopacity{0.700000}%
\pgfsetlinewidth{0.000000pt}%
\definecolor{currentstroke}{rgb}{0.000000,0.000000,0.000000}%
\pgfsetstrokecolor{currentstroke}%
\pgfsetdash{}{0pt}%
\pgfpathmoveto{\pgfqpoint{4.172437in}{1.624022in}}%
\pgfpathlineto{\pgfqpoint{4.186481in}{1.621978in}}%
\pgfpathlineto{\pgfqpoint{4.200533in}{1.620033in}}%
\pgfpathlineto{\pgfqpoint{4.214593in}{1.618188in}}%
\pgfpathlineto{\pgfqpoint{4.228662in}{1.616440in}}%
\pgfpathlineto{\pgfqpoint{4.220613in}{1.607101in}}%
\pgfpathlineto{\pgfqpoint{4.212559in}{1.597890in}}%
\pgfpathlineto{\pgfqpoint{4.204500in}{1.588812in}}%
\pgfpathlineto{\pgfqpoint{4.196435in}{1.579871in}}%
\pgfpathlineto{\pgfqpoint{4.182356in}{1.582068in}}%
\pgfpathlineto{\pgfqpoint{4.168285in}{1.584363in}}%
\pgfpathlineto{\pgfqpoint{4.154221in}{1.586758in}}%
\pgfpathlineto{\pgfqpoint{4.140166in}{1.589251in}}%
\pgfpathlineto{\pgfqpoint{4.148242in}{1.597735in}}%
\pgfpathlineto{\pgfqpoint{4.156313in}{1.606362in}}%
\pgfpathlineto{\pgfqpoint{4.164378in}{1.615125in}}%
\pgfpathlineto{\pgfqpoint{4.172437in}{1.624022in}}%
\pgfpathclose%
\pgfusepath{fill}%
\end{pgfscope}%
\begin{pgfscope}%
\pgfpathrectangle{\pgfqpoint{1.150000in}{0.150000in}}{\pgfqpoint{5.700000in}{5.700000in}}%
\pgfusepath{clip}%
\pgfsetbuttcap%
\pgfsetroundjoin%
\definecolor{currentfill}{rgb}{0.262138,0.242286,0.520837}%
\pgfsetfillcolor{currentfill}%
\pgfsetfillopacity{0.700000}%
\pgfsetlinewidth{0.000000pt}%
\definecolor{currentstroke}{rgb}{0.000000,0.000000,0.000000}%
\pgfsetstrokecolor{currentstroke}%
\pgfsetdash{}{0pt}%
\pgfpathmoveto{\pgfqpoint{3.025388in}{2.045326in}}%
\pgfpathlineto{\pgfqpoint{3.039327in}{2.032822in}}%
\pgfpathlineto{\pgfqpoint{3.053267in}{2.020446in}}%
\pgfpathlineto{\pgfqpoint{3.067206in}{2.008195in}}%
\pgfpathlineto{\pgfqpoint{3.081144in}{1.996071in}}%
\pgfpathlineto{\pgfqpoint{3.072425in}{2.000652in}}%
\pgfpathlineto{\pgfqpoint{3.063688in}{2.005569in}}%
\pgfpathlineto{\pgfqpoint{3.054934in}{2.010830in}}%
\pgfpathlineto{\pgfqpoint{3.046162in}{2.016438in}}%
\pgfpathlineto{\pgfqpoint{3.032180in}{2.029143in}}%
\pgfpathlineto{\pgfqpoint{3.018197in}{2.041974in}}%
\pgfpathlineto{\pgfqpoint{3.004213in}{2.054932in}}%
\pgfpathlineto{\pgfqpoint{2.990228in}{2.068017in}}%
\pgfpathlineto{\pgfqpoint{2.999046in}{2.061819in}}%
\pgfpathlineto{\pgfqpoint{3.007845in}{2.055975in}}%
\pgfpathlineto{\pgfqpoint{3.016625in}{2.050480in}}%
\pgfpathlineto{\pgfqpoint{3.025388in}{2.045326in}}%
\pgfpathclose%
\pgfusepath{fill}%
\end{pgfscope}%
\begin{pgfscope}%
\pgfpathrectangle{\pgfqpoint{1.150000in}{0.150000in}}{\pgfqpoint{5.700000in}{5.700000in}}%
\pgfusepath{clip}%
\pgfsetbuttcap%
\pgfsetroundjoin%
\definecolor{currentfill}{rgb}{0.244972,0.287675,0.537260}%
\pgfsetfillcolor{currentfill}%
\pgfsetfillopacity{0.700000}%
\pgfsetlinewidth{0.000000pt}%
\definecolor{currentstroke}{rgb}{0.000000,0.000000,0.000000}%
\pgfsetstrokecolor{currentstroke}%
\pgfsetdash{}{0pt}%
\pgfpathmoveto{\pgfqpoint{5.000560in}{2.159761in}}%
\pgfpathlineto{\pgfqpoint{5.014953in}{2.164780in}}%
\pgfpathlineto{\pgfqpoint{5.029360in}{2.169898in}}%
\pgfpathlineto{\pgfqpoint{5.043779in}{2.175115in}}%
\pgfpathlineto{\pgfqpoint{5.058212in}{2.180431in}}%
\pgfpathlineto{\pgfqpoint{5.050372in}{2.166587in}}%
\pgfpathlineto{\pgfqpoint{5.042527in}{2.152696in}}%
\pgfpathlineto{\pgfqpoint{5.034678in}{2.138760in}}%
\pgfpathlineto{\pgfqpoint{5.026823in}{2.124781in}}%
\pgfpathlineto{\pgfqpoint{5.012391in}{2.119741in}}%
\pgfpathlineto{\pgfqpoint{4.997973in}{2.114799in}}%
\pgfpathlineto{\pgfqpoint{4.983569in}{2.109956in}}%
\pgfpathlineto{\pgfqpoint{4.969177in}{2.105211in}}%
\pgfpathlineto{\pgfqpoint{4.977030in}{2.118907in}}%
\pgfpathlineto{\pgfqpoint{4.984879in}{2.132566in}}%
\pgfpathlineto{\pgfqpoint{4.992722in}{2.146185in}}%
\pgfpathlineto{\pgfqpoint{5.000560in}{2.159761in}}%
\pgfpathclose%
\pgfusepath{fill}%
\end{pgfscope}%
\begin{pgfscope}%
\pgfpathrectangle{\pgfqpoint{1.150000in}{0.150000in}}{\pgfqpoint{5.700000in}{5.700000in}}%
\pgfusepath{clip}%
\pgfsetbuttcap%
\pgfsetroundjoin%
\definecolor{currentfill}{rgb}{0.283197,0.115680,0.436115}%
\pgfsetfillcolor{currentfill}%
\pgfsetfillopacity{0.700000}%
\pgfsetlinewidth{0.000000pt}%
\definecolor{currentstroke}{rgb}{0.000000,0.000000,0.000000}%
\pgfsetstrokecolor{currentstroke}%
\pgfsetdash{}{0pt}%
\pgfpathmoveto{\pgfqpoint{3.393979in}{1.774937in}}%
\pgfpathlineto{\pgfqpoint{3.407902in}{1.765855in}}%
\pgfpathlineto{\pgfqpoint{3.421828in}{1.756885in}}%
\pgfpathlineto{\pgfqpoint{3.435756in}{1.748028in}}%
\pgfpathlineto{\pgfqpoint{3.449686in}{1.739282in}}%
\pgfpathlineto{\pgfqpoint{3.441247in}{1.739310in}}%
\pgfpathlineto{\pgfqpoint{3.432796in}{1.739619in}}%
\pgfpathlineto{\pgfqpoint{3.424332in}{1.740214in}}%
\pgfpathlineto{\pgfqpoint{3.415855in}{1.741102in}}%
\pgfpathlineto{\pgfqpoint{3.401892in}{1.750396in}}%
\pgfpathlineto{\pgfqpoint{3.387931in}{1.759803in}}%
\pgfpathlineto{\pgfqpoint{3.373972in}{1.769321in}}%
\pgfpathlineto{\pgfqpoint{3.360015in}{1.778953in}}%
\pgfpathlineto{\pgfqpoint{3.368526in}{1.777509in}}%
\pgfpathlineto{\pgfqpoint{3.377024in}{1.776362in}}%
\pgfpathlineto{\pgfqpoint{3.385508in}{1.775506in}}%
\pgfpathlineto{\pgfqpoint{3.393979in}{1.774937in}}%
\pgfpathclose%
\pgfusepath{fill}%
\end{pgfscope}%
\begin{pgfscope}%
\pgfpathrectangle{\pgfqpoint{1.150000in}{0.150000in}}{\pgfqpoint{5.700000in}{5.700000in}}%
\pgfusepath{clip}%
\pgfsetbuttcap%
\pgfsetroundjoin%
\definecolor{currentfill}{rgb}{0.204903,0.375746,0.553533}%
\pgfsetfillcolor{currentfill}%
\pgfsetfillopacity{0.700000}%
\pgfsetlinewidth{0.000000pt}%
\definecolor{currentstroke}{rgb}{0.000000,0.000000,0.000000}%
\pgfsetstrokecolor{currentstroke}%
\pgfsetdash{}{0pt}%
\pgfpathmoveto{\pgfqpoint{5.209771in}{2.367131in}}%
\pgfpathlineto{\pgfqpoint{5.224280in}{2.373656in}}%
\pgfpathlineto{\pgfqpoint{5.238803in}{2.380281in}}%
\pgfpathlineto{\pgfqpoint{5.253341in}{2.387006in}}%
\pgfpathlineto{\pgfqpoint{5.267893in}{2.393830in}}%
\pgfpathlineto{\pgfqpoint{5.260115in}{2.380305in}}%
\pgfpathlineto{\pgfqpoint{5.252331in}{2.366697in}}%
\pgfpathlineto{\pgfqpoint{5.244541in}{2.353009in}}%
\pgfpathlineto{\pgfqpoint{5.236745in}{2.339242in}}%
\pgfpathlineto{\pgfqpoint{5.222195in}{2.332638in}}%
\pgfpathlineto{\pgfqpoint{5.207659in}{2.326133in}}%
\pgfpathlineto{\pgfqpoint{5.193138in}{2.319728in}}%
\pgfpathlineto{\pgfqpoint{5.178631in}{2.313423in}}%
\pgfpathlineto{\pgfqpoint{5.186424in}{2.326963in}}%
\pgfpathlineto{\pgfqpoint{5.194213in}{2.340429in}}%
\pgfpathlineto{\pgfqpoint{5.201995in}{2.353819in}}%
\pgfpathlineto{\pgfqpoint{5.209771in}{2.367131in}}%
\pgfpathclose%
\pgfusepath{fill}%
\end{pgfscope}%
\begin{pgfscope}%
\pgfpathrectangle{\pgfqpoint{1.150000in}{0.150000in}}{\pgfqpoint{5.700000in}{5.700000in}}%
\pgfusepath{clip}%
\pgfsetbuttcap%
\pgfsetroundjoin%
\definecolor{currentfill}{rgb}{0.127568,0.566949,0.550556}%
\pgfsetfillcolor{currentfill}%
\pgfsetfillopacity{0.700000}%
\pgfsetlinewidth{0.000000pt}%
\definecolor{currentstroke}{rgb}{0.000000,0.000000,0.000000}%
\pgfsetstrokecolor{currentstroke}%
\pgfsetdash{}{0pt}%
\pgfpathmoveto{\pgfqpoint{2.296509in}{2.893088in}}%
\pgfpathlineto{\pgfqpoint{2.310665in}{2.872623in}}%
\pgfpathlineto{\pgfqpoint{2.324814in}{2.852346in}}%
\pgfpathlineto{\pgfqpoint{2.338954in}{2.832253in}}%
\pgfpathlineto{\pgfqpoint{2.353086in}{2.812344in}}%
\pgfpathlineto{\pgfqpoint{2.343683in}{2.824595in}}%
\pgfpathlineto{\pgfqpoint{2.334253in}{2.837260in}}%
\pgfpathlineto{\pgfqpoint{2.324795in}{2.850348in}}%
\pgfpathlineto{\pgfqpoint{2.315309in}{2.863864in}}%
\pgfpathlineto{\pgfqpoint{2.301111in}{2.884398in}}%
\pgfpathlineto{\pgfqpoint{2.286906in}{2.905117in}}%
\pgfpathlineto{\pgfqpoint{2.272691in}{2.926023in}}%
\pgfpathlineto{\pgfqpoint{2.258468in}{2.947116in}}%
\pgfpathlineto{\pgfqpoint{2.268021in}{2.932963in}}%
\pgfpathlineto{\pgfqpoint{2.277546in}{2.919245in}}%
\pgfpathlineto{\pgfqpoint{2.287041in}{2.905956in}}%
\pgfpathlineto{\pgfqpoint{2.296509in}{2.893088in}}%
\pgfpathclose%
\pgfusepath{fill}%
\end{pgfscope}%
\begin{pgfscope}%
\pgfpathrectangle{\pgfqpoint{1.150000in}{0.150000in}}{\pgfqpoint{5.700000in}{5.700000in}}%
\pgfusepath{clip}%
\pgfsetbuttcap%
\pgfsetroundjoin%
\definecolor{currentfill}{rgb}{0.269944,0.014625,0.341379}%
\pgfsetfillcolor{currentfill}%
\pgfsetfillopacity{0.700000}%
\pgfsetlinewidth{0.000000pt}%
\definecolor{currentstroke}{rgb}{0.000000,0.000000,0.000000}%
\pgfsetstrokecolor{currentstroke}%
\pgfsetdash{}{0pt}%
\pgfpathmoveto{\pgfqpoint{3.939516in}{1.600123in}}%
\pgfpathlineto{\pgfqpoint{3.953503in}{1.595962in}}%
\pgfpathlineto{\pgfqpoint{3.967497in}{1.591902in}}%
\pgfpathlineto{\pgfqpoint{3.981497in}{1.587943in}}%
\pgfpathlineto{\pgfqpoint{3.995504in}{1.584085in}}%
\pgfpathlineto{\pgfqpoint{3.987367in}{1.577335in}}%
\pgfpathlineto{\pgfqpoint{3.979224in}{1.570765in}}%
\pgfpathlineto{\pgfqpoint{3.971073in}{1.564378in}}%
\pgfpathlineto{\pgfqpoint{3.962915in}{1.558179in}}%
\pgfpathlineto{\pgfqpoint{3.948891in}{1.562523in}}%
\pgfpathlineto{\pgfqpoint{3.934874in}{1.566968in}}%
\pgfpathlineto{\pgfqpoint{3.920863in}{1.571513in}}%
\pgfpathlineto{\pgfqpoint{3.906858in}{1.576160in}}%
\pgfpathlineto{\pgfqpoint{3.915034in}{1.581866in}}%
\pgfpathlineto{\pgfqpoint{3.923202in}{1.587765in}}%
\pgfpathlineto{\pgfqpoint{3.931363in}{1.593852in}}%
\pgfpathlineto{\pgfqpoint{3.939516in}{1.600123in}}%
\pgfpathclose%
\pgfusepath{fill}%
\end{pgfscope}%
\begin{pgfscope}%
\pgfpathrectangle{\pgfqpoint{1.150000in}{0.150000in}}{\pgfqpoint{5.700000in}{5.700000in}}%
\pgfusepath{clip}%
\pgfsetbuttcap%
\pgfsetroundjoin%
\definecolor{currentfill}{rgb}{0.272594,0.025563,0.353093}%
\pgfsetfillcolor{currentfill}%
\pgfsetfillopacity{0.700000}%
\pgfsetlinewidth{0.000000pt}%
\definecolor{currentstroke}{rgb}{0.000000,0.000000,0.000000}%
\pgfsetstrokecolor{currentstroke}%
\pgfsetdash{}{0pt}%
\pgfpathmoveto{\pgfqpoint{3.795020in}{1.617008in}}%
\pgfpathlineto{\pgfqpoint{3.808981in}{1.611542in}}%
\pgfpathlineto{\pgfqpoint{3.822947in}{1.606180in}}%
\pgfpathlineto{\pgfqpoint{3.836918in}{1.600921in}}%
\pgfpathlineto{\pgfqpoint{3.850895in}{1.595764in}}%
\pgfpathlineto{\pgfqpoint{3.842692in}{1.590754in}}%
\pgfpathlineto{\pgfqpoint{3.834480in}{1.585952in}}%
\pgfpathlineto{\pgfqpoint{3.826260in}{1.581363in}}%
\pgfpathlineto{\pgfqpoint{3.818032in}{1.576992in}}%
\pgfpathlineto{\pgfqpoint{3.804035in}{1.582654in}}%
\pgfpathlineto{\pgfqpoint{3.790042in}{1.588418in}}%
\pgfpathlineto{\pgfqpoint{3.776055in}{1.594285in}}%
\pgfpathlineto{\pgfqpoint{3.762072in}{1.600256in}}%
\pgfpathlineto{\pgfqpoint{3.770323in}{1.604114in}}%
\pgfpathlineto{\pgfqpoint{3.778564in}{1.608196in}}%
\pgfpathlineto{\pgfqpoint{3.786796in}{1.612495in}}%
\pgfpathlineto{\pgfqpoint{3.795020in}{1.617008in}}%
\pgfpathclose%
\pgfusepath{fill}%
\end{pgfscope}%
\begin{pgfscope}%
\pgfpathrectangle{\pgfqpoint{1.150000in}{0.150000in}}{\pgfqpoint{5.700000in}{5.700000in}}%
\pgfusepath{clip}%
\pgfsetbuttcap%
\pgfsetroundjoin%
\definecolor{currentfill}{rgb}{0.273006,0.204520,0.501721}%
\pgfsetfillcolor{currentfill}%
\pgfsetfillopacity{0.700000}%
\pgfsetlinewidth{0.000000pt}%
\definecolor{currentstroke}{rgb}{0.000000,0.000000,0.000000}%
\pgfsetstrokecolor{currentstroke}%
\pgfsetdash{}{0pt}%
\pgfpathmoveto{\pgfqpoint{4.791526in}{1.964991in}}%
\pgfpathlineto{\pgfqpoint{4.805815in}{1.968349in}}%
\pgfpathlineto{\pgfqpoint{4.820117in}{1.971804in}}%
\pgfpathlineto{\pgfqpoint{4.834431in}{1.975357in}}%
\pgfpathlineto{\pgfqpoint{4.848757in}{1.979008in}}%
\pgfpathlineto{\pgfqpoint{4.840863in}{1.965414in}}%
\pgfpathlineto{\pgfqpoint{4.832966in}{1.951814in}}%
\pgfpathlineto{\pgfqpoint{4.825064in}{1.938211in}}%
\pgfpathlineto{\pgfqpoint{4.817158in}{1.924609in}}%
\pgfpathlineto{\pgfqpoint{4.802832in}{1.921287in}}%
\pgfpathlineto{\pgfqpoint{4.788519in}{1.918062in}}%
\pgfpathlineto{\pgfqpoint{4.774217in}{1.914934in}}%
\pgfpathlineto{\pgfqpoint{4.759927in}{1.911905in}}%
\pgfpathlineto{\pgfqpoint{4.767834in}{1.925172in}}%
\pgfpathlineto{\pgfqpoint{4.775735in}{1.938444in}}%
\pgfpathlineto{\pgfqpoint{4.783633in}{1.951718in}}%
\pgfpathlineto{\pgfqpoint{4.791526in}{1.964991in}}%
\pgfpathclose%
\pgfusepath{fill}%
\end{pgfscope}%
\begin{pgfscope}%
\pgfpathrectangle{\pgfqpoint{1.150000in}{0.150000in}}{\pgfqpoint{5.700000in}{5.700000in}}%
\pgfusepath{clip}%
\pgfsetbuttcap%
\pgfsetroundjoin%
\definecolor{currentfill}{rgb}{0.169646,0.456262,0.558030}%
\pgfsetfillcolor{currentfill}%
\pgfsetfillopacity{0.700000}%
\pgfsetlinewidth{0.000000pt}%
\definecolor{currentstroke}{rgb}{0.000000,0.000000,0.000000}%
\pgfsetstrokecolor{currentstroke}%
\pgfsetdash{}{0pt}%
\pgfpathmoveto{\pgfqpoint{5.419114in}{2.579530in}}%
\pgfpathlineto{\pgfqpoint{5.433746in}{2.587401in}}%
\pgfpathlineto{\pgfqpoint{5.448394in}{2.595373in}}%
\pgfpathlineto{\pgfqpoint{5.463059in}{2.603445in}}%
\pgfpathlineto{\pgfqpoint{5.477739in}{2.611619in}}%
\pgfpathlineto{\pgfqpoint{5.470037in}{2.598905in}}%
\pgfpathlineto{\pgfqpoint{5.462328in}{2.586081in}}%
\pgfpathlineto{\pgfqpoint{5.454612in}{2.573146in}}%
\pgfpathlineto{\pgfqpoint{5.446889in}{2.560102in}}%
\pgfpathlineto{\pgfqpoint{5.432210in}{2.552093in}}%
\pgfpathlineto{\pgfqpoint{5.417546in}{2.544184in}}%
\pgfpathlineto{\pgfqpoint{5.402899in}{2.536375in}}%
\pgfpathlineto{\pgfqpoint{5.388267in}{2.528667in}}%
\pgfpathlineto{\pgfqpoint{5.395989in}{2.541540in}}%
\pgfpathlineto{\pgfqpoint{5.403704in}{2.554309in}}%
\pgfpathlineto{\pgfqpoint{5.411412in}{2.566973in}}%
\pgfpathlineto{\pgfqpoint{5.419114in}{2.579530in}}%
\pgfpathclose%
\pgfusepath{fill}%
\end{pgfscope}%
\begin{pgfscope}%
\pgfpathrectangle{\pgfqpoint{1.150000in}{0.150000in}}{\pgfqpoint{5.700000in}{5.700000in}}%
\pgfusepath{clip}%
\pgfsetbuttcap%
\pgfsetroundjoin%
\definecolor{currentfill}{rgb}{0.269308,0.218818,0.509577}%
\pgfsetfillcolor{currentfill}%
\pgfsetfillopacity{0.700000}%
\pgfsetlinewidth{0.000000pt}%
\definecolor{currentstroke}{rgb}{0.000000,0.000000,0.000000}%
\pgfsetstrokecolor{currentstroke}%
\pgfsetdash{}{0pt}%
\pgfpathmoveto{\pgfqpoint{3.081144in}{1.996071in}}%
\pgfpathlineto{\pgfqpoint{3.095082in}{1.984071in}}%
\pgfpathlineto{\pgfqpoint{3.109021in}{1.972195in}}%
\pgfpathlineto{\pgfqpoint{3.122959in}{1.960442in}}%
\pgfpathlineto{\pgfqpoint{3.136897in}{1.948813in}}%
\pgfpathlineto{\pgfqpoint{3.128220in}{1.952824in}}%
\pgfpathlineto{\pgfqpoint{3.119526in}{1.957166in}}%
\pgfpathlineto{\pgfqpoint{3.110814in}{1.961844in}}%
\pgfpathlineto{\pgfqpoint{3.102086in}{1.966866in}}%
\pgfpathlineto{\pgfqpoint{3.088105in}{1.979073in}}%
\pgfpathlineto{\pgfqpoint{3.074125in}{1.991404in}}%
\pgfpathlineto{\pgfqpoint{3.060144in}{2.003859in}}%
\pgfpathlineto{\pgfqpoint{3.046162in}{2.016438in}}%
\pgfpathlineto{\pgfqpoint{3.054934in}{2.010830in}}%
\pgfpathlineto{\pgfqpoint{3.063688in}{2.005569in}}%
\pgfpathlineto{\pgfqpoint{3.072425in}{2.000652in}}%
\pgfpathlineto{\pgfqpoint{3.081144in}{1.996071in}}%
\pgfpathclose%
\pgfusepath{fill}%
\end{pgfscope}%
\begin{pgfscope}%
\pgfpathrectangle{\pgfqpoint{1.150000in}{0.150000in}}{\pgfqpoint{5.700000in}{5.700000in}}%
\pgfusepath{clip}%
\pgfsetbuttcap%
\pgfsetroundjoin%
\definecolor{currentfill}{rgb}{0.139147,0.533812,0.555298}%
\pgfsetfillcolor{currentfill}%
\pgfsetfillopacity{0.700000}%
\pgfsetlinewidth{0.000000pt}%
\definecolor{currentstroke}{rgb}{0.000000,0.000000,0.000000}%
\pgfsetstrokecolor{currentstroke}%
\pgfsetdash{}{0pt}%
\pgfpathmoveto{\pgfqpoint{5.628450in}{2.790267in}}%
\pgfpathlineto{\pgfqpoint{5.643211in}{2.799320in}}%
\pgfpathlineto{\pgfqpoint{5.657988in}{2.808474in}}%
\pgfpathlineto{\pgfqpoint{5.672783in}{2.817729in}}%
\pgfpathlineto{\pgfqpoint{5.687595in}{2.827087in}}%
\pgfpathlineto{\pgfqpoint{5.679988in}{2.815601in}}%
\pgfpathlineto{\pgfqpoint{5.672373in}{2.803982in}}%
\pgfpathlineto{\pgfqpoint{5.664749in}{2.792230in}}%
\pgfpathlineto{\pgfqpoint{5.657117in}{2.780347in}}%
\pgfpathlineto{\pgfqpoint{5.642304in}{2.771094in}}%
\pgfpathlineto{\pgfqpoint{5.627508in}{2.761944in}}%
\pgfpathlineto{\pgfqpoint{5.612730in}{2.752895in}}%
\pgfpathlineto{\pgfqpoint{5.597968in}{2.743947in}}%
\pgfpathlineto{\pgfqpoint{5.605600in}{2.755718in}}%
\pgfpathlineto{\pgfqpoint{5.613225in}{2.767362in}}%
\pgfpathlineto{\pgfqpoint{5.620842in}{2.778879in}}%
\pgfpathlineto{\pgfqpoint{5.628450in}{2.790267in}}%
\pgfpathclose%
\pgfusepath{fill}%
\end{pgfscope}%
\begin{pgfscope}%
\pgfpathrectangle{\pgfqpoint{1.150000in}{0.150000in}}{\pgfqpoint{5.700000in}{5.700000in}}%
\pgfusepath{clip}%
\pgfsetbuttcap%
\pgfsetroundjoin%
\definecolor{currentfill}{rgb}{0.120565,0.596422,0.543611}%
\pgfsetfillcolor{currentfill}%
\pgfsetfillopacity{0.700000}%
\pgfsetlinewidth{0.000000pt}%
\definecolor{currentstroke}{rgb}{0.000000,0.000000,0.000000}%
\pgfsetstrokecolor{currentstroke}%
\pgfsetdash{}{0pt}%
\pgfpathmoveto{\pgfqpoint{2.239794in}{2.976845in}}%
\pgfpathlineto{\pgfqpoint{2.253987in}{2.955617in}}%
\pgfpathlineto{\pgfqpoint{2.268169in}{2.934583in}}%
\pgfpathlineto{\pgfqpoint{2.282343in}{2.913740in}}%
\pgfpathlineto{\pgfqpoint{2.296509in}{2.893088in}}%
\pgfpathlineto{\pgfqpoint{2.287041in}{2.905956in}}%
\pgfpathlineto{\pgfqpoint{2.277546in}{2.919245in}}%
\pgfpathlineto{\pgfqpoint{2.268021in}{2.932963in}}%
\pgfpathlineto{\pgfqpoint{2.258468in}{2.947116in}}%
\pgfpathlineto{\pgfqpoint{2.244236in}{2.968398in}}%
\pgfpathlineto{\pgfqpoint{2.229995in}{2.989873in}}%
\pgfpathlineto{\pgfqpoint{2.215744in}{3.011540in}}%
\pgfpathlineto{\pgfqpoint{2.201484in}{3.033403in}}%
\pgfpathlineto{\pgfqpoint{2.211106in}{3.018607in}}%
\pgfpathlineto{\pgfqpoint{2.220698in}{3.004254in}}%
\pgfpathlineto{\pgfqpoint{2.230261in}{2.990335in}}%
\pgfpathlineto{\pgfqpoint{2.239794in}{2.976845in}}%
\pgfpathclose%
\pgfusepath{fill}%
\end{pgfscope}%
\begin{pgfscope}%
\pgfpathrectangle{\pgfqpoint{1.150000in}{0.150000in}}{\pgfqpoint{5.700000in}{5.700000in}}%
\pgfusepath{clip}%
\pgfsetbuttcap%
\pgfsetroundjoin%
\definecolor{currentfill}{rgb}{0.271305,0.019942,0.347269}%
\pgfsetfillcolor{currentfill}%
\pgfsetfillopacity{0.700000}%
\pgfsetlinewidth{0.000000pt}%
\definecolor{currentstroke}{rgb}{0.000000,0.000000,0.000000}%
\pgfsetstrokecolor{currentstroke}%
\pgfsetdash{}{0pt}%
\pgfpathmoveto{\pgfqpoint{4.084017in}{1.600215in}}%
\pgfpathlineto{\pgfqpoint{4.098043in}{1.597325in}}%
\pgfpathlineto{\pgfqpoint{4.112076in}{1.594534in}}%
\pgfpathlineto{\pgfqpoint{4.126117in}{1.591843in}}%
\pgfpathlineto{\pgfqpoint{4.140166in}{1.589251in}}%
\pgfpathlineto{\pgfqpoint{4.132083in}{1.580912in}}%
\pgfpathlineto{\pgfqpoint{4.123995in}{1.572725in}}%
\pgfpathlineto{\pgfqpoint{4.115900in}{1.564693in}}%
\pgfpathlineto{\pgfqpoint{4.107800in}{1.556821in}}%
\pgfpathlineto{\pgfqpoint{4.093738in}{1.559880in}}%
\pgfpathlineto{\pgfqpoint{4.079684in}{1.563039in}}%
\pgfpathlineto{\pgfqpoint{4.065637in}{1.566297in}}%
\pgfpathlineto{\pgfqpoint{4.051597in}{1.569655in}}%
\pgfpathlineto{\pgfqpoint{4.059711in}{1.577053in}}%
\pgfpathlineto{\pgfqpoint{4.067819in}{1.584615in}}%
\pgfpathlineto{\pgfqpoint{4.075921in}{1.592337in}}%
\pgfpathlineto{\pgfqpoint{4.084017in}{1.600215in}}%
\pgfpathclose%
\pgfusepath{fill}%
\end{pgfscope}%
\begin{pgfscope}%
\pgfpathrectangle{\pgfqpoint{1.150000in}{0.150000in}}{\pgfqpoint{5.700000in}{5.700000in}}%
\pgfusepath{clip}%
\pgfsetbuttcap%
\pgfsetroundjoin%
\definecolor{currentfill}{rgb}{0.277018,0.050344,0.375715}%
\pgfsetfillcolor{currentfill}%
\pgfsetfillopacity{0.700000}%
\pgfsetlinewidth{0.000000pt}%
\definecolor{currentstroke}{rgb}{0.000000,0.000000,0.000000}%
\pgfsetstrokecolor{currentstroke}%
\pgfsetdash{}{0pt}%
\pgfpathmoveto{\pgfqpoint{3.650380in}{1.651775in}}%
\pgfpathlineto{\pgfqpoint{3.664326in}{1.644967in}}%
\pgfpathlineto{\pgfqpoint{3.678276in}{1.638265in}}%
\pgfpathlineto{\pgfqpoint{3.692231in}{1.631668in}}%
\pgfpathlineto{\pgfqpoint{3.706190in}{1.625177in}}%
\pgfpathlineto{\pgfqpoint{3.697907in}{1.622064in}}%
\pgfpathlineto{\pgfqpoint{3.689614in}{1.619189in}}%
\pgfpathlineto{\pgfqpoint{3.681312in}{1.616558in}}%
\pgfpathlineto{\pgfqpoint{3.672999in}{1.614176in}}%
\pgfpathlineto{\pgfqpoint{3.659014in}{1.621193in}}%
\pgfpathlineto{\pgfqpoint{3.645034in}{1.628315in}}%
\pgfpathlineto{\pgfqpoint{3.631058in}{1.635543in}}%
\pgfpathlineto{\pgfqpoint{3.617086in}{1.642876in}}%
\pgfpathlineto{\pgfqpoint{3.625425in}{1.644725in}}%
\pgfpathlineto{\pgfqpoint{3.633754in}{1.646828in}}%
\pgfpathlineto{\pgfqpoint{3.642072in}{1.649180in}}%
\pgfpathlineto{\pgfqpoint{3.650380in}{1.651775in}}%
\pgfpathclose%
\pgfusepath{fill}%
\end{pgfscope}%
\begin{pgfscope}%
\pgfpathrectangle{\pgfqpoint{1.150000in}{0.150000in}}{\pgfqpoint{5.700000in}{5.700000in}}%
\pgfusepath{clip}%
\pgfsetbuttcap%
\pgfsetroundjoin%
\definecolor{currentfill}{rgb}{0.227802,0.326594,0.546532}%
\pgfsetfillcolor{currentfill}%
\pgfsetfillopacity{0.700000}%
\pgfsetlinewidth{0.000000pt}%
\definecolor{currentstroke}{rgb}{0.000000,0.000000,0.000000}%
\pgfsetstrokecolor{currentstroke}%
\pgfsetdash{}{0pt}%
\pgfpathmoveto{\pgfqpoint{5.089521in}{2.235283in}}%
\pgfpathlineto{\pgfqpoint{5.103970in}{2.240954in}}%
\pgfpathlineto{\pgfqpoint{5.118432in}{2.246724in}}%
\pgfpathlineto{\pgfqpoint{5.132909in}{2.252593in}}%
\pgfpathlineto{\pgfqpoint{5.147400in}{2.258561in}}%
\pgfpathlineto{\pgfqpoint{5.139578in}{2.244681in}}%
\pgfpathlineto{\pgfqpoint{5.131752in}{2.230739in}}%
\pgfpathlineto{\pgfqpoint{5.123920in}{2.216738in}}%
\pgfpathlineto{\pgfqpoint{5.116083in}{2.202680in}}%
\pgfpathlineto{\pgfqpoint{5.101594in}{2.196970in}}%
\pgfpathlineto{\pgfqpoint{5.087120in}{2.191358in}}%
\pgfpathlineto{\pgfqpoint{5.072659in}{2.185845in}}%
\pgfpathlineto{\pgfqpoint{5.058212in}{2.180431in}}%
\pgfpathlineto{\pgfqpoint{5.066047in}{2.194224in}}%
\pgfpathlineto{\pgfqpoint{5.073877in}{2.207966in}}%
\pgfpathlineto{\pgfqpoint{5.081702in}{2.221653in}}%
\pgfpathlineto{\pgfqpoint{5.089521in}{2.235283in}}%
\pgfpathclose%
\pgfusepath{fill}%
\end{pgfscope}%
\begin{pgfscope}%
\pgfpathrectangle{\pgfqpoint{1.150000in}{0.150000in}}{\pgfqpoint{5.700000in}{5.700000in}}%
\pgfusepath{clip}%
\pgfsetbuttcap%
\pgfsetroundjoin%
\definecolor{currentfill}{rgb}{0.263663,0.237631,0.518762}%
\pgfsetfillcolor{currentfill}%
\pgfsetfillopacity{0.700000}%
\pgfsetlinewidth{0.000000pt}%
\definecolor{currentstroke}{rgb}{0.000000,0.000000,0.000000}%
\pgfsetstrokecolor{currentstroke}%
\pgfsetdash{}{0pt}%
\pgfpathmoveto{\pgfqpoint{4.880285in}{2.033271in}}%
\pgfpathlineto{\pgfqpoint{4.894624in}{2.037330in}}%
\pgfpathlineto{\pgfqpoint{4.908976in}{2.041488in}}%
\pgfpathlineto{\pgfqpoint{4.923340in}{2.045744in}}%
\pgfpathlineto{\pgfqpoint{4.937718in}{2.050097in}}%
\pgfpathlineto{\pgfqpoint{4.929841in}{2.036251in}}%
\pgfpathlineto{\pgfqpoint{4.921960in}{2.022382in}}%
\pgfpathlineto{\pgfqpoint{4.914075in}{2.008495in}}%
\pgfpathlineto{\pgfqpoint{4.906185in}{1.994591in}}%
\pgfpathlineto{\pgfqpoint{4.891809in}{1.990549in}}%
\pgfpathlineto{\pgfqpoint{4.877446in}{1.986604in}}%
\pgfpathlineto{\pgfqpoint{4.863095in}{1.982757in}}%
\pgfpathlineto{\pgfqpoint{4.848757in}{1.979008in}}%
\pgfpathlineto{\pgfqpoint{4.856646in}{1.992594in}}%
\pgfpathlineto{\pgfqpoint{4.864530in}{2.006168in}}%
\pgfpathlineto{\pgfqpoint{4.872410in}{2.019728in}}%
\pgfpathlineto{\pgfqpoint{4.880285in}{2.033271in}}%
\pgfpathclose%
\pgfusepath{fill}%
\end{pgfscope}%
\begin{pgfscope}%
\pgfpathrectangle{\pgfqpoint{1.150000in}{0.150000in}}{\pgfqpoint{5.700000in}{5.700000in}}%
\pgfusepath{clip}%
\pgfsetbuttcap%
\pgfsetroundjoin%
\definecolor{currentfill}{rgb}{0.273006,0.204520,0.501721}%
\pgfsetfillcolor{currentfill}%
\pgfsetfillopacity{0.700000}%
\pgfsetlinewidth{0.000000pt}%
\definecolor{currentstroke}{rgb}{0.000000,0.000000,0.000000}%
\pgfsetstrokecolor{currentstroke}%
\pgfsetdash{}{0pt}%
\pgfpathmoveto{\pgfqpoint{3.136897in}{1.948813in}}%
\pgfpathlineto{\pgfqpoint{3.150836in}{1.937306in}}%
\pgfpathlineto{\pgfqpoint{3.164775in}{1.925920in}}%
\pgfpathlineto{\pgfqpoint{3.178714in}{1.914655in}}%
\pgfpathlineto{\pgfqpoint{3.192654in}{1.903511in}}%
\pgfpathlineto{\pgfqpoint{3.184017in}{1.906953in}}%
\pgfpathlineto{\pgfqpoint{3.175364in}{1.910721in}}%
\pgfpathlineto{\pgfqpoint{3.166694in}{1.914820in}}%
\pgfpathlineto{\pgfqpoint{3.158008in}{1.919257in}}%
\pgfpathlineto{\pgfqpoint{3.144027in}{1.930977in}}%
\pgfpathlineto{\pgfqpoint{3.130047in}{1.942818in}}%
\pgfpathlineto{\pgfqpoint{3.116066in}{1.954781in}}%
\pgfpathlineto{\pgfqpoint{3.102086in}{1.966866in}}%
\pgfpathlineto{\pgfqpoint{3.110814in}{1.961844in}}%
\pgfpathlineto{\pgfqpoint{3.119526in}{1.957166in}}%
\pgfpathlineto{\pgfqpoint{3.128220in}{1.952824in}}%
\pgfpathlineto{\pgfqpoint{3.136897in}{1.948813in}}%
\pgfpathclose%
\pgfusepath{fill}%
\end{pgfscope}%
\begin{pgfscope}%
\pgfpathrectangle{\pgfqpoint{1.150000in}{0.150000in}}{\pgfqpoint{5.700000in}{5.700000in}}%
\pgfusepath{clip}%
\pgfsetbuttcap%
\pgfsetroundjoin%
\definecolor{currentfill}{rgb}{0.282656,0.100196,0.422160}%
\pgfsetfillcolor{currentfill}%
\pgfsetfillopacity{0.700000}%
\pgfsetlinewidth{0.000000pt}%
\definecolor{currentstroke}{rgb}{0.000000,0.000000,0.000000}%
\pgfsetstrokecolor{currentstroke}%
\pgfsetdash{}{0pt}%
\pgfpathmoveto{\pgfqpoint{3.449686in}{1.739282in}}%
\pgfpathlineto{\pgfqpoint{3.463619in}{1.730647in}}%
\pgfpathlineto{\pgfqpoint{3.477555in}{1.722123in}}%
\pgfpathlineto{\pgfqpoint{3.491494in}{1.713709in}}%
\pgfpathlineto{\pgfqpoint{3.505435in}{1.705404in}}%
\pgfpathlineto{\pgfqpoint{3.497028in}{1.704893in}}%
\pgfpathlineto{\pgfqpoint{3.488608in}{1.704656in}}%
\pgfpathlineto{\pgfqpoint{3.480176in}{1.704701in}}%
\pgfpathlineto{\pgfqpoint{3.471731in}{1.705033in}}%
\pgfpathlineto{\pgfqpoint{3.457758in}{1.713884in}}%
\pgfpathlineto{\pgfqpoint{3.443788in}{1.722846in}}%
\pgfpathlineto{\pgfqpoint{3.429820in}{1.731918in}}%
\pgfpathlineto{\pgfqpoint{3.415855in}{1.741102in}}%
\pgfpathlineto{\pgfqpoint{3.424332in}{1.740214in}}%
\pgfpathlineto{\pgfqpoint{3.432796in}{1.739619in}}%
\pgfpathlineto{\pgfqpoint{3.441247in}{1.739310in}}%
\pgfpathlineto{\pgfqpoint{3.449686in}{1.739282in}}%
\pgfpathclose%
\pgfusepath{fill}%
\end{pgfscope}%
\begin{pgfscope}%
\pgfpathrectangle{\pgfqpoint{1.150000in}{0.150000in}}{\pgfqpoint{5.700000in}{5.700000in}}%
\pgfusepath{clip}%
\pgfsetbuttcap%
\pgfsetroundjoin%
\definecolor{currentfill}{rgb}{0.282656,0.100196,0.422160}%
\pgfsetfillcolor{currentfill}%
\pgfsetfillopacity{0.700000}%
\pgfsetlinewidth{0.000000pt}%
\definecolor{currentstroke}{rgb}{0.000000,0.000000,0.000000}%
\pgfsetstrokecolor{currentstroke}%
\pgfsetdash{}{0pt}%
\pgfpathmoveto{\pgfqpoint{4.494062in}{1.738601in}}%
\pgfpathlineto{\pgfqpoint{4.508227in}{1.739371in}}%
\pgfpathlineto{\pgfqpoint{4.522402in}{1.740239in}}%
\pgfpathlineto{\pgfqpoint{4.536588in}{1.741205in}}%
\pgfpathlineto{\pgfqpoint{4.550783in}{1.742267in}}%
\pgfpathlineto{\pgfqpoint{4.542819in}{1.730150in}}%
\pgfpathlineto{\pgfqpoint{4.534850in}{1.718095in}}%
\pgfpathlineto{\pgfqpoint{4.526876in}{1.706105in}}%
\pgfpathlineto{\pgfqpoint{4.518899in}{1.694185in}}%
\pgfpathlineto{\pgfqpoint{4.504699in}{1.693520in}}%
\pgfpathlineto{\pgfqpoint{4.490510in}{1.692953in}}%
\pgfpathlineto{\pgfqpoint{4.476330in}{1.692483in}}%
\pgfpathlineto{\pgfqpoint{4.462161in}{1.692110in}}%
\pgfpathlineto{\pgfqpoint{4.470143in}{1.703626in}}%
\pgfpathlineto{\pgfqpoint{4.478120in}{1.715216in}}%
\pgfpathlineto{\pgfqpoint{4.486093in}{1.726875in}}%
\pgfpathlineto{\pgfqpoint{4.494062in}{1.738601in}}%
\pgfpathclose%
\pgfusepath{fill}%
\end{pgfscope}%
\begin{pgfscope}%
\pgfpathrectangle{\pgfqpoint{1.150000in}{0.150000in}}{\pgfqpoint{5.700000in}{5.700000in}}%
\pgfusepath{clip}%
\pgfsetbuttcap%
\pgfsetroundjoin%
\definecolor{currentfill}{rgb}{0.280267,0.073417,0.397163}%
\pgfsetfillcolor{currentfill}%
\pgfsetfillopacity{0.700000}%
\pgfsetlinewidth{0.000000pt}%
\definecolor{currentstroke}{rgb}{0.000000,0.000000,0.000000}%
\pgfsetstrokecolor{currentstroke}%
\pgfsetdash{}{0pt}%
\pgfpathmoveto{\pgfqpoint{4.405578in}{1.691595in}}%
\pgfpathlineto{\pgfqpoint{4.419710in}{1.691577in}}%
\pgfpathlineto{\pgfqpoint{4.433850in}{1.691657in}}%
\pgfpathlineto{\pgfqpoint{4.448001in}{1.691835in}}%
\pgfpathlineto{\pgfqpoint{4.462161in}{1.692110in}}%
\pgfpathlineto{\pgfqpoint{4.454174in}{1.680672in}}%
\pgfpathlineto{\pgfqpoint{4.446183in}{1.669316in}}%
\pgfpathlineto{\pgfqpoint{4.438188in}{1.658046in}}%
\pgfpathlineto{\pgfqpoint{4.430188in}{1.646865in}}%
\pgfpathlineto{\pgfqpoint{4.416022in}{1.647005in}}%
\pgfpathlineto{\pgfqpoint{4.401866in}{1.647242in}}%
\pgfpathlineto{\pgfqpoint{4.387719in}{1.647577in}}%
\pgfpathlineto{\pgfqpoint{4.373582in}{1.648010in}}%
\pgfpathlineto{\pgfqpoint{4.381588in}{1.658769in}}%
\pgfpathlineto{\pgfqpoint{4.389589in}{1.669622in}}%
\pgfpathlineto{\pgfqpoint{4.397586in}{1.680566in}}%
\pgfpathlineto{\pgfqpoint{4.405578in}{1.691595in}}%
\pgfpathclose%
\pgfusepath{fill}%
\end{pgfscope}%
\begin{pgfscope}%
\pgfpathrectangle{\pgfqpoint{1.150000in}{0.150000in}}{\pgfqpoint{5.700000in}{5.700000in}}%
\pgfusepath{clip}%
\pgfsetbuttcap%
\pgfsetroundjoin%
\definecolor{currentfill}{rgb}{0.188923,0.410910,0.556326}%
\pgfsetfillcolor{currentfill}%
\pgfsetfillopacity{0.700000}%
\pgfsetlinewidth{0.000000pt}%
\definecolor{currentstroke}{rgb}{0.000000,0.000000,0.000000}%
\pgfsetstrokecolor{currentstroke}%
\pgfsetdash{}{0pt}%
\pgfpathmoveto{\pgfqpoint{5.298945in}{2.447065in}}%
\pgfpathlineto{\pgfqpoint{5.313514in}{2.454190in}}%
\pgfpathlineto{\pgfqpoint{5.328098in}{2.461415in}}%
\pgfpathlineto{\pgfqpoint{5.342698in}{2.468741in}}%
\pgfpathlineto{\pgfqpoint{5.357312in}{2.476166in}}%
\pgfpathlineto{\pgfqpoint{5.349557in}{2.462796in}}%
\pgfpathlineto{\pgfqpoint{5.341796in}{2.449331in}}%
\pgfpathlineto{\pgfqpoint{5.334028in}{2.435773in}}%
\pgfpathlineto{\pgfqpoint{5.326254in}{2.422124in}}%
\pgfpathlineto{\pgfqpoint{5.311641in}{2.414901in}}%
\pgfpathlineto{\pgfqpoint{5.297043in}{2.407777in}}%
\pgfpathlineto{\pgfqpoint{5.282461in}{2.400754in}}%
\pgfpathlineto{\pgfqpoint{5.267893in}{2.393830in}}%
\pgfpathlineto{\pgfqpoint{5.275666in}{2.407270in}}%
\pgfpathlineto{\pgfqpoint{5.283431in}{2.420624in}}%
\pgfpathlineto{\pgfqpoint{5.291191in}{2.433889in}}%
\pgfpathlineto{\pgfqpoint{5.298945in}{2.447065in}}%
\pgfpathclose%
\pgfusepath{fill}%
\end{pgfscope}%
\begin{pgfscope}%
\pgfpathrectangle{\pgfqpoint{1.150000in}{0.150000in}}{\pgfqpoint{5.700000in}{5.700000in}}%
\pgfusepath{clip}%
\pgfsetbuttcap%
\pgfsetroundjoin%
\definecolor{currentfill}{rgb}{0.283187,0.125848,0.444960}%
\pgfsetfillcolor{currentfill}%
\pgfsetfillopacity{0.700000}%
\pgfsetlinewidth{0.000000pt}%
\definecolor{currentstroke}{rgb}{0.000000,0.000000,0.000000}%
\pgfsetstrokecolor{currentstroke}%
\pgfsetdash{}{0pt}%
\pgfpathmoveto{\pgfqpoint{4.582600in}{1.791280in}}%
\pgfpathlineto{\pgfqpoint{4.596803in}{1.792821in}}%
\pgfpathlineto{\pgfqpoint{4.611016in}{1.794459in}}%
\pgfpathlineto{\pgfqpoint{4.625241in}{1.796194in}}%
\pgfpathlineto{\pgfqpoint{4.639476in}{1.798027in}}%
\pgfpathlineto{\pgfqpoint{4.631531in}{1.785325in}}%
\pgfpathlineto{\pgfqpoint{4.623581in}{1.772666in}}%
\pgfpathlineto{\pgfqpoint{4.615628in}{1.760054in}}%
\pgfpathlineto{\pgfqpoint{4.607670in}{1.747491in}}%
\pgfpathlineto{\pgfqpoint{4.593433in}{1.746039in}}%
\pgfpathlineto{\pgfqpoint{4.579206in}{1.744685in}}%
\pgfpathlineto{\pgfqpoint{4.564989in}{1.743427in}}%
\pgfpathlineto{\pgfqpoint{4.550783in}{1.742267in}}%
\pgfpathlineto{\pgfqpoint{4.558744in}{1.754442in}}%
\pgfpathlineto{\pgfqpoint{4.566700in}{1.766672in}}%
\pgfpathlineto{\pgfqpoint{4.574652in}{1.778953in}}%
\pgfpathlineto{\pgfqpoint{4.582600in}{1.791280in}}%
\pgfpathclose%
\pgfusepath{fill}%
\end{pgfscope}%
\begin{pgfscope}%
\pgfpathrectangle{\pgfqpoint{1.150000in}{0.150000in}}{\pgfqpoint{5.700000in}{5.700000in}}%
\pgfusepath{clip}%
\pgfsetbuttcap%
\pgfsetroundjoin%
\definecolor{currentfill}{rgb}{0.277018,0.050344,0.375715}%
\pgfsetfillcolor{currentfill}%
\pgfsetfillopacity{0.700000}%
\pgfsetlinewidth{0.000000pt}%
\definecolor{currentstroke}{rgb}{0.000000,0.000000,0.000000}%
\pgfsetstrokecolor{currentstroke}%
\pgfsetdash{}{0pt}%
\pgfpathmoveto{\pgfqpoint{4.317121in}{1.650718in}}%
\pgfpathlineto{\pgfqpoint{4.331223in}{1.649894in}}%
\pgfpathlineto{\pgfqpoint{4.345334in}{1.649168in}}%
\pgfpathlineto{\pgfqpoint{4.359453in}{1.648540in}}%
\pgfpathlineto{\pgfqpoint{4.373582in}{1.648010in}}%
\pgfpathlineto{\pgfqpoint{4.365571in}{1.637349in}}%
\pgfpathlineto{\pgfqpoint{4.357555in}{1.626791in}}%
\pgfpathlineto{\pgfqpoint{4.349534in}{1.616339in}}%
\pgfpathlineto{\pgfqpoint{4.341509in}{1.605997in}}%
\pgfpathlineto{\pgfqpoint{4.327373in}{1.606960in}}%
\pgfpathlineto{\pgfqpoint{4.313246in}{1.608021in}}%
\pgfpathlineto{\pgfqpoint{4.299127in}{1.609179in}}%
\pgfpathlineto{\pgfqpoint{4.285017in}{1.610435in}}%
\pgfpathlineto{\pgfqpoint{4.293051in}{1.620337in}}%
\pgfpathlineto{\pgfqpoint{4.301079in}{1.630354in}}%
\pgfpathlineto{\pgfqpoint{4.309103in}{1.640483in}}%
\pgfpathlineto{\pgfqpoint{4.317121in}{1.650718in}}%
\pgfpathclose%
\pgfusepath{fill}%
\end{pgfscope}%
\begin{pgfscope}%
\pgfpathrectangle{\pgfqpoint{1.150000in}{0.150000in}}{\pgfqpoint{5.700000in}{5.700000in}}%
\pgfusepath{clip}%
\pgfsetbuttcap%
\pgfsetroundjoin%
\definecolor{currentfill}{rgb}{0.128729,0.563265,0.551229}%
\pgfsetfillcolor{currentfill}%
\pgfsetfillopacity{0.700000}%
\pgfsetlinewidth{0.000000pt}%
\definecolor{currentstroke}{rgb}{0.000000,0.000000,0.000000}%
\pgfsetstrokecolor{currentstroke}%
\pgfsetdash{}{0pt}%
\pgfpathmoveto{\pgfqpoint{5.717935in}{2.871683in}}%
\pgfpathlineto{\pgfqpoint{5.732762in}{2.881228in}}%
\pgfpathlineto{\pgfqpoint{5.747606in}{2.890875in}}%
\pgfpathlineto{\pgfqpoint{5.762467in}{2.900623in}}%
\pgfpathlineto{\pgfqpoint{5.754897in}{2.889619in}}%
\pgfpathlineto{\pgfqpoint{5.747318in}{2.878475in}}%
\pgfpathlineto{\pgfqpoint{5.739730in}{2.867193in}}%
\pgfpathlineto{\pgfqpoint{5.732133in}{2.855771in}}%
\pgfpathlineto{\pgfqpoint{5.717269in}{2.846108in}}%
\pgfpathlineto{\pgfqpoint{5.702423in}{2.836546in}}%
\pgfpathlineto{\pgfqpoint{5.687595in}{2.827087in}}%
\pgfpathlineto{\pgfqpoint{5.695193in}{2.838438in}}%
\pgfpathlineto{\pgfqpoint{5.702782in}{2.849656in}}%
\pgfpathlineto{\pgfqpoint{5.710363in}{2.860737in}}%
\pgfpathlineto{\pgfqpoint{5.717935in}{2.871683in}}%
\pgfpathclose%
\pgfusepath{fill}%
\end{pgfscope}%
\begin{pgfscope}%
\pgfpathrectangle{\pgfqpoint{1.150000in}{0.150000in}}{\pgfqpoint{5.700000in}{5.700000in}}%
\pgfusepath{clip}%
\pgfsetbuttcap%
\pgfsetroundjoin%
\definecolor{currentfill}{rgb}{0.121380,0.629492,0.531973}%
\pgfsetfillcolor{currentfill}%
\pgfsetfillopacity{0.700000}%
\pgfsetlinewidth{0.000000pt}%
\definecolor{currentstroke}{rgb}{0.000000,0.000000,0.000000}%
\pgfsetstrokecolor{currentstroke}%
\pgfsetdash{}{0pt}%
\pgfpathmoveto{\pgfqpoint{2.182931in}{3.063732in}}%
\pgfpathlineto{\pgfqpoint{2.197161in}{3.041710in}}%
\pgfpathlineto{\pgfqpoint{2.211382in}{3.019890in}}%
\pgfpathlineto{\pgfqpoint{2.225593in}{2.998269in}}%
\pgfpathlineto{\pgfqpoint{2.239794in}{2.976845in}}%
\pgfpathlineto{\pgfqpoint{2.230261in}{2.990335in}}%
\pgfpathlineto{\pgfqpoint{2.220698in}{3.004254in}}%
\pgfpathlineto{\pgfqpoint{2.211106in}{3.018607in}}%
\pgfpathlineto{\pgfqpoint{2.201484in}{3.033403in}}%
\pgfpathlineto{\pgfqpoint{2.187213in}{3.055462in}}%
\pgfpathlineto{\pgfqpoint{2.172933in}{3.077720in}}%
\pgfpathlineto{\pgfqpoint{2.158643in}{3.100179in}}%
\pgfpathlineto{\pgfqpoint{2.144342in}{3.122841in}}%
\pgfpathlineto{\pgfqpoint{2.154035in}{3.107397in}}%
\pgfpathlineto{\pgfqpoint{2.163697in}{3.092402in}}%
\pgfpathlineto{\pgfqpoint{2.173329in}{3.077850in}}%
\pgfpathlineto{\pgfqpoint{2.182931in}{3.063732in}}%
\pgfpathclose%
\pgfusepath{fill}%
\end{pgfscope}%
\begin{pgfscope}%
\pgfpathrectangle{\pgfqpoint{1.150000in}{0.150000in}}{\pgfqpoint{5.700000in}{5.700000in}}%
\pgfusepath{clip}%
\pgfsetbuttcap%
\pgfsetroundjoin%
\definecolor{currentfill}{rgb}{0.156270,0.489624,0.557936}%
\pgfsetfillcolor{currentfill}%
\pgfsetfillopacity{0.700000}%
\pgfsetlinewidth{0.000000pt}%
\definecolor{currentstroke}{rgb}{0.000000,0.000000,0.000000}%
\pgfsetstrokecolor{currentstroke}%
\pgfsetdash{}{0pt}%
\pgfpathmoveto{\pgfqpoint{5.508472in}{2.661332in}}%
\pgfpathlineto{\pgfqpoint{5.523169in}{2.669750in}}%
\pgfpathlineto{\pgfqpoint{5.537882in}{2.678269in}}%
\pgfpathlineto{\pgfqpoint{5.552611in}{2.686890in}}%
\pgfpathlineto{\pgfqpoint{5.567356in}{2.695611in}}%
\pgfpathlineto{\pgfqpoint{5.559684in}{2.683219in}}%
\pgfpathlineto{\pgfqpoint{5.552004in}{2.670706in}}%
\pgfpathlineto{\pgfqpoint{5.544316in}{2.658073in}}%
\pgfpathlineto{\pgfqpoint{5.536620in}{2.645321in}}%
\pgfpathlineto{\pgfqpoint{5.521876in}{2.636744in}}%
\pgfpathlineto{\pgfqpoint{5.507147in}{2.628268in}}%
\pgfpathlineto{\pgfqpoint{5.492435in}{2.619893in}}%
\pgfpathlineto{\pgfqpoint{5.477739in}{2.611619in}}%
\pgfpathlineto{\pgfqpoint{5.485433in}{2.624219in}}%
\pgfpathlineto{\pgfqpoint{5.493120in}{2.636706in}}%
\pgfpathlineto{\pgfqpoint{5.500800in}{2.649077in}}%
\pgfpathlineto{\pgfqpoint{5.508472in}{2.661332in}}%
\pgfpathclose%
\pgfusepath{fill}%
\end{pgfscope}%
\begin{pgfscope}%
\pgfpathrectangle{\pgfqpoint{1.150000in}{0.150000in}}{\pgfqpoint{5.700000in}{5.700000in}}%
\pgfusepath{clip}%
\pgfsetbuttcap%
\pgfsetroundjoin%
\definecolor{currentfill}{rgb}{0.272594,0.025563,0.353093}%
\pgfsetfillcolor{currentfill}%
\pgfsetfillopacity{0.700000}%
\pgfsetlinewidth{0.000000pt}%
\definecolor{currentstroke}{rgb}{0.000000,0.000000,0.000000}%
\pgfsetstrokecolor{currentstroke}%
\pgfsetdash{}{0pt}%
\pgfpathmoveto{\pgfqpoint{3.850895in}{1.595764in}}%
\pgfpathlineto{\pgfqpoint{3.864877in}{1.590710in}}%
\pgfpathlineto{\pgfqpoint{3.878865in}{1.585759in}}%
\pgfpathlineto{\pgfqpoint{3.892858in}{1.580909in}}%
\pgfpathlineto{\pgfqpoint{3.906858in}{1.576160in}}%
\pgfpathlineto{\pgfqpoint{3.898674in}{1.570653in}}%
\pgfpathlineto{\pgfqpoint{3.890482in}{1.565348in}}%
\pgfpathlineto{\pgfqpoint{3.882283in}{1.560252in}}%
\pgfpathlineto{\pgfqpoint{3.874075in}{1.555369in}}%
\pgfpathlineto{\pgfqpoint{3.860056in}{1.560622in}}%
\pgfpathlineto{\pgfqpoint{3.846043in}{1.565977in}}%
\pgfpathlineto{\pgfqpoint{3.832035in}{1.571434in}}%
\pgfpathlineto{\pgfqpoint{3.818032in}{1.576992in}}%
\pgfpathlineto{\pgfqpoint{3.826260in}{1.581363in}}%
\pgfpathlineto{\pgfqpoint{3.834480in}{1.585952in}}%
\pgfpathlineto{\pgfqpoint{3.842692in}{1.590754in}}%
\pgfpathlineto{\pgfqpoint{3.850895in}{1.595764in}}%
\pgfpathclose%
\pgfusepath{fill}%
\end{pgfscope}%
\begin{pgfscope}%
\pgfpathrectangle{\pgfqpoint{1.150000in}{0.150000in}}{\pgfqpoint{5.700000in}{5.700000in}}%
\pgfusepath{clip}%
\pgfsetbuttcap%
\pgfsetroundjoin%
\definecolor{currentfill}{rgb}{0.277134,0.185228,0.489898}%
\pgfsetfillcolor{currentfill}%
\pgfsetfillopacity{0.700000}%
\pgfsetlinewidth{0.000000pt}%
\definecolor{currentstroke}{rgb}{0.000000,0.000000,0.000000}%
\pgfsetstrokecolor{currentstroke}%
\pgfsetdash{}{0pt}%
\pgfpathmoveto{\pgfqpoint{3.192654in}{1.903511in}}%
\pgfpathlineto{\pgfqpoint{3.206595in}{1.892486in}}%
\pgfpathlineto{\pgfqpoint{3.220536in}{1.881580in}}%
\pgfpathlineto{\pgfqpoint{3.234478in}{1.870793in}}%
\pgfpathlineto{\pgfqpoint{3.248421in}{1.860124in}}%
\pgfpathlineto{\pgfqpoint{3.239823in}{1.863000in}}%
\pgfpathlineto{\pgfqpoint{3.231209in}{1.866196in}}%
\pgfpathlineto{\pgfqpoint{3.222580in}{1.869717in}}%
\pgfpathlineto{\pgfqpoint{3.213935in}{1.873571in}}%
\pgfpathlineto{\pgfqpoint{3.199952in}{1.884814in}}%
\pgfpathlineto{\pgfqpoint{3.185970in}{1.896176in}}%
\pgfpathlineto{\pgfqpoint{3.171989in}{1.907656in}}%
\pgfpathlineto{\pgfqpoint{3.158008in}{1.919257in}}%
\pgfpathlineto{\pgfqpoint{3.166694in}{1.914820in}}%
\pgfpathlineto{\pgfqpoint{3.175364in}{1.910721in}}%
\pgfpathlineto{\pgfqpoint{3.184017in}{1.906953in}}%
\pgfpathlineto{\pgfqpoint{3.192654in}{1.903511in}}%
\pgfpathclose%
\pgfusepath{fill}%
\end{pgfscope}%
\begin{pgfscope}%
\pgfpathrectangle{\pgfqpoint{1.150000in}{0.150000in}}{\pgfqpoint{5.700000in}{5.700000in}}%
\pgfusepath{clip}%
\pgfsetbuttcap%
\pgfsetroundjoin%
\definecolor{currentfill}{rgb}{0.281412,0.155834,0.469201}%
\pgfsetfillcolor{currentfill}%
\pgfsetfillopacity{0.700000}%
\pgfsetlinewidth{0.000000pt}%
\definecolor{currentstroke}{rgb}{0.000000,0.000000,0.000000}%
\pgfsetstrokecolor{currentstroke}%
\pgfsetdash{}{0pt}%
\pgfpathmoveto{\pgfqpoint{4.671214in}{1.849191in}}%
\pgfpathlineto{\pgfqpoint{4.685459in}{1.851484in}}%
\pgfpathlineto{\pgfqpoint{4.699715in}{1.853875in}}%
\pgfpathlineto{\pgfqpoint{4.713982in}{1.856364in}}%
\pgfpathlineto{\pgfqpoint{4.728261in}{1.858950in}}%
\pgfpathlineto{\pgfqpoint{4.720333in}{1.845755in}}%
\pgfpathlineto{\pgfqpoint{4.712402in}{1.832585in}}%
\pgfpathlineto{\pgfqpoint{4.704466in}{1.819443in}}%
\pgfpathlineto{\pgfqpoint{4.696527in}{1.806331in}}%
\pgfpathlineto{\pgfqpoint{4.682247in}{1.804109in}}%
\pgfpathlineto{\pgfqpoint{4.667979in}{1.801985in}}%
\pgfpathlineto{\pgfqpoint{4.653722in}{1.799957in}}%
\pgfpathlineto{\pgfqpoint{4.639476in}{1.798027in}}%
\pgfpathlineto{\pgfqpoint{4.647417in}{1.810768in}}%
\pgfpathlineto{\pgfqpoint{4.655354in}{1.823544in}}%
\pgfpathlineto{\pgfqpoint{4.663286in}{1.836353in}}%
\pgfpathlineto{\pgfqpoint{4.671214in}{1.849191in}}%
\pgfpathclose%
\pgfusepath{fill}%
\end{pgfscope}%
\begin{pgfscope}%
\pgfpathrectangle{\pgfqpoint{1.150000in}{0.150000in}}{\pgfqpoint{5.700000in}{5.700000in}}%
\pgfusepath{clip}%
\pgfsetbuttcap%
\pgfsetroundjoin%
\definecolor{currentfill}{rgb}{0.271305,0.019942,0.347269}%
\pgfsetfillcolor{currentfill}%
\pgfsetfillopacity{0.700000}%
\pgfsetlinewidth{0.000000pt}%
\definecolor{currentstroke}{rgb}{0.000000,0.000000,0.000000}%
\pgfsetstrokecolor{currentstroke}%
\pgfsetdash{}{0pt}%
\pgfpathmoveto{\pgfqpoint{3.995504in}{1.584085in}}%
\pgfpathlineto{\pgfqpoint{4.009517in}{1.580327in}}%
\pgfpathlineto{\pgfqpoint{4.023537in}{1.576670in}}%
\pgfpathlineto{\pgfqpoint{4.037563in}{1.573113in}}%
\pgfpathlineto{\pgfqpoint{4.051597in}{1.569655in}}%
\pgfpathlineto{\pgfqpoint{4.043476in}{1.562427in}}%
\pgfpathlineto{\pgfqpoint{4.035348in}{1.555373in}}%
\pgfpathlineto{\pgfqpoint{4.027213in}{1.548497in}}%
\pgfpathlineto{\pgfqpoint{4.019071in}{1.541806in}}%
\pgfpathlineto{\pgfqpoint{4.005023in}{1.545749in}}%
\pgfpathlineto{\pgfqpoint{3.990980in}{1.549792in}}%
\pgfpathlineto{\pgfqpoint{3.976944in}{1.553935in}}%
\pgfpathlineto{\pgfqpoint{3.962915in}{1.558179in}}%
\pgfpathlineto{\pgfqpoint{3.971073in}{1.564378in}}%
\pgfpathlineto{\pgfqpoint{3.979224in}{1.570765in}}%
\pgfpathlineto{\pgfqpoint{3.987367in}{1.577335in}}%
\pgfpathlineto{\pgfqpoint{3.995504in}{1.584085in}}%
\pgfpathclose%
\pgfusepath{fill}%
\end{pgfscope}%
\begin{pgfscope}%
\pgfpathrectangle{\pgfqpoint{1.150000in}{0.150000in}}{\pgfqpoint{5.700000in}{5.700000in}}%
\pgfusepath{clip}%
\pgfsetbuttcap%
\pgfsetroundjoin%
\definecolor{currentfill}{rgb}{0.274952,0.037752,0.364543}%
\pgfsetfillcolor{currentfill}%
\pgfsetfillopacity{0.700000}%
\pgfsetlinewidth{0.000000pt}%
\definecolor{currentstroke}{rgb}{0.000000,0.000000,0.000000}%
\pgfsetstrokecolor{currentstroke}%
\pgfsetdash{}{0pt}%
\pgfpathmoveto{\pgfqpoint{4.228662in}{1.616440in}}%
\pgfpathlineto{\pgfqpoint{4.242738in}{1.614792in}}%
\pgfpathlineto{\pgfqpoint{4.256823in}{1.613241in}}%
\pgfpathlineto{\pgfqpoint{4.270916in}{1.611789in}}%
\pgfpathlineto{\pgfqpoint{4.285017in}{1.610435in}}%
\pgfpathlineto{\pgfqpoint{4.276978in}{1.600653in}}%
\pgfpathlineto{\pgfqpoint{4.268934in}{1.590994in}}%
\pgfpathlineto{\pgfqpoint{4.260885in}{1.581463in}}%
\pgfpathlineto{\pgfqpoint{4.252830in}{1.572065in}}%
\pgfpathlineto{\pgfqpoint{4.238719in}{1.573870in}}%
\pgfpathlineto{\pgfqpoint{4.224617in}{1.575772in}}%
\pgfpathlineto{\pgfqpoint{4.210522in}{1.577772in}}%
\pgfpathlineto{\pgfqpoint{4.196435in}{1.579871in}}%
\pgfpathlineto{\pgfqpoint{4.204500in}{1.588812in}}%
\pgfpathlineto{\pgfqpoint{4.212559in}{1.597890in}}%
\pgfpathlineto{\pgfqpoint{4.220613in}{1.607101in}}%
\pgfpathlineto{\pgfqpoint{4.228662in}{1.616440in}}%
\pgfpathclose%
\pgfusepath{fill}%
\end{pgfscope}%
\begin{pgfscope}%
\pgfpathrectangle{\pgfqpoint{1.150000in}{0.150000in}}{\pgfqpoint{5.700000in}{5.700000in}}%
\pgfusepath{clip}%
\pgfsetbuttcap%
\pgfsetroundjoin%
\definecolor{currentfill}{rgb}{0.250425,0.274290,0.533103}%
\pgfsetfillcolor{currentfill}%
\pgfsetfillopacity{0.700000}%
\pgfsetlinewidth{0.000000pt}%
\definecolor{currentstroke}{rgb}{0.000000,0.000000,0.000000}%
\pgfsetstrokecolor{currentstroke}%
\pgfsetdash{}{0pt}%
\pgfpathmoveto{\pgfqpoint{4.969177in}{2.105211in}}%
\pgfpathlineto{\pgfqpoint{4.983569in}{2.109956in}}%
\pgfpathlineto{\pgfqpoint{4.997973in}{2.114799in}}%
\pgfpathlineto{\pgfqpoint{5.012391in}{2.119741in}}%
\pgfpathlineto{\pgfqpoint{5.026823in}{2.124781in}}%
\pgfpathlineto{\pgfqpoint{5.018963in}{2.110763in}}%
\pgfpathlineto{\pgfqpoint{5.011099in}{2.096707in}}%
\pgfpathlineto{\pgfqpoint{5.003230in}{2.082617in}}%
\pgfpathlineto{\pgfqpoint{4.995356in}{2.068495in}}%
\pgfpathlineto{\pgfqpoint{4.980927in}{2.063748in}}%
\pgfpathlineto{\pgfqpoint{4.966511in}{2.059100in}}%
\pgfpathlineto{\pgfqpoint{4.952108in}{2.054549in}}%
\pgfpathlineto{\pgfqpoint{4.937718in}{2.050097in}}%
\pgfpathlineto{\pgfqpoint{4.945589in}{2.063919in}}%
\pgfpathlineto{\pgfqpoint{4.953457in}{2.077714in}}%
\pgfpathlineto{\pgfqpoint{4.961319in}{2.091479in}}%
\pgfpathlineto{\pgfqpoint{4.969177in}{2.105211in}}%
\pgfpathclose%
\pgfusepath{fill}%
\end{pgfscope}%
\begin{pgfscope}%
\pgfpathrectangle{\pgfqpoint{1.150000in}{0.150000in}}{\pgfqpoint{5.700000in}{5.700000in}}%
\pgfusepath{clip}%
\pgfsetbuttcap%
\pgfsetroundjoin%
\definecolor{currentfill}{rgb}{0.210503,0.363727,0.552206}%
\pgfsetfillcolor{currentfill}%
\pgfsetfillopacity{0.700000}%
\pgfsetlinewidth{0.000000pt}%
\definecolor{currentstroke}{rgb}{0.000000,0.000000,0.000000}%
\pgfsetstrokecolor{currentstroke}%
\pgfsetdash{}{0pt}%
\pgfpathmoveto{\pgfqpoint{5.178631in}{2.313423in}}%
\pgfpathlineto{\pgfqpoint{5.193138in}{2.319728in}}%
\pgfpathlineto{\pgfqpoint{5.207659in}{2.326133in}}%
\pgfpathlineto{\pgfqpoint{5.222195in}{2.332638in}}%
\pgfpathlineto{\pgfqpoint{5.236745in}{2.339242in}}%
\pgfpathlineto{\pgfqpoint{5.228944in}{2.325397in}}%
\pgfpathlineto{\pgfqpoint{5.221136in}{2.311478in}}%
\pgfpathlineto{\pgfqpoint{5.213324in}{2.297486in}}%
\pgfpathlineto{\pgfqpoint{5.205505in}{2.283424in}}%
\pgfpathlineto{\pgfqpoint{5.190957in}{2.277059in}}%
\pgfpathlineto{\pgfqpoint{5.176424in}{2.270794in}}%
\pgfpathlineto{\pgfqpoint{5.161905in}{2.264628in}}%
\pgfpathlineto{\pgfqpoint{5.147400in}{2.258561in}}%
\pgfpathlineto{\pgfqpoint{5.155216in}{2.272377in}}%
\pgfpathlineto{\pgfqpoint{5.163026in}{2.286128in}}%
\pgfpathlineto{\pgfqpoint{5.170831in}{2.299810in}}%
\pgfpathlineto{\pgfqpoint{5.178631in}{2.313423in}}%
\pgfpathclose%
\pgfusepath{fill}%
\end{pgfscope}%
\begin{pgfscope}%
\pgfpathrectangle{\pgfqpoint{1.150000in}{0.150000in}}{\pgfqpoint{5.700000in}{5.700000in}}%
\pgfusepath{clip}%
\pgfsetbuttcap%
\pgfsetroundjoin%
\definecolor{currentfill}{rgb}{0.281924,0.089666,0.412415}%
\pgfsetfillcolor{currentfill}%
\pgfsetfillopacity{0.700000}%
\pgfsetlinewidth{0.000000pt}%
\definecolor{currentstroke}{rgb}{0.000000,0.000000,0.000000}%
\pgfsetstrokecolor{currentstroke}%
\pgfsetdash{}{0pt}%
\pgfpathmoveto{\pgfqpoint{3.505435in}{1.705404in}}%
\pgfpathlineto{\pgfqpoint{3.519380in}{1.697210in}}%
\pgfpathlineto{\pgfqpoint{3.533328in}{1.689124in}}%
\pgfpathlineto{\pgfqpoint{3.547279in}{1.681147in}}%
\pgfpathlineto{\pgfqpoint{3.561233in}{1.673278in}}%
\pgfpathlineto{\pgfqpoint{3.552855in}{1.672227in}}%
\pgfpathlineto{\pgfqpoint{3.544465in}{1.671446in}}%
\pgfpathlineto{\pgfqpoint{3.536064in}{1.670941in}}%
\pgfpathlineto{\pgfqpoint{3.527650in}{1.670718in}}%
\pgfpathlineto{\pgfqpoint{3.513666in}{1.679134in}}%
\pgfpathlineto{\pgfqpoint{3.499685in}{1.687658in}}%
\pgfpathlineto{\pgfqpoint{3.485706in}{1.696291in}}%
\pgfpathlineto{\pgfqpoint{3.471731in}{1.705033in}}%
\pgfpathlineto{\pgfqpoint{3.480176in}{1.704701in}}%
\pgfpathlineto{\pgfqpoint{3.488608in}{1.704656in}}%
\pgfpathlineto{\pgfqpoint{3.497028in}{1.704893in}}%
\pgfpathlineto{\pgfqpoint{3.505435in}{1.705404in}}%
\pgfpathclose%
\pgfusepath{fill}%
\end{pgfscope}%
\begin{pgfscope}%
\pgfpathrectangle{\pgfqpoint{1.150000in}{0.150000in}}{\pgfqpoint{5.700000in}{5.700000in}}%
\pgfusepath{clip}%
\pgfsetbuttcap%
\pgfsetroundjoin%
\definecolor{currentfill}{rgb}{0.276022,0.044167,0.370164}%
\pgfsetfillcolor{currentfill}%
\pgfsetfillopacity{0.700000}%
\pgfsetlinewidth{0.000000pt}%
\definecolor{currentstroke}{rgb}{0.000000,0.000000,0.000000}%
\pgfsetstrokecolor{currentstroke}%
\pgfsetdash{}{0pt}%
\pgfpathmoveto{\pgfqpoint{3.706190in}{1.625177in}}%
\pgfpathlineto{\pgfqpoint{3.720154in}{1.618790in}}%
\pgfpathlineto{\pgfqpoint{3.734122in}{1.612508in}}%
\pgfpathlineto{\pgfqpoint{3.748095in}{1.606330in}}%
\pgfpathlineto{\pgfqpoint{3.762072in}{1.600256in}}%
\pgfpathlineto{\pgfqpoint{3.753813in}{1.596625in}}%
\pgfpathlineto{\pgfqpoint{3.745544in}{1.593228in}}%
\pgfpathlineto{\pgfqpoint{3.737266in}{1.590070in}}%
\pgfpathlineto{\pgfqpoint{3.728978in}{1.587156in}}%
\pgfpathlineto{\pgfqpoint{3.714977in}{1.593755in}}%
\pgfpathlineto{\pgfqpoint{3.700980in}{1.600457in}}%
\pgfpathlineto{\pgfqpoint{3.686987in}{1.607265in}}%
\pgfpathlineto{\pgfqpoint{3.672999in}{1.614176in}}%
\pgfpathlineto{\pgfqpoint{3.681312in}{1.616558in}}%
\pgfpathlineto{\pgfqpoint{3.689614in}{1.619189in}}%
\pgfpathlineto{\pgfqpoint{3.697907in}{1.622064in}}%
\pgfpathlineto{\pgfqpoint{3.706190in}{1.625177in}}%
\pgfpathclose%
\pgfusepath{fill}%
\end{pgfscope}%
\begin{pgfscope}%
\pgfpathrectangle{\pgfqpoint{1.150000in}{0.150000in}}{\pgfqpoint{5.700000in}{5.700000in}}%
\pgfusepath{clip}%
\pgfsetbuttcap%
\pgfsetroundjoin%
\definecolor{currentfill}{rgb}{0.276194,0.190074,0.493001}%
\pgfsetfillcolor{currentfill}%
\pgfsetfillopacity{0.700000}%
\pgfsetlinewidth{0.000000pt}%
\definecolor{currentstroke}{rgb}{0.000000,0.000000,0.000000}%
\pgfsetstrokecolor{currentstroke}%
\pgfsetdash{}{0pt}%
\pgfpathmoveto{\pgfqpoint{4.759927in}{1.911905in}}%
\pgfpathlineto{\pgfqpoint{4.774217in}{1.914934in}}%
\pgfpathlineto{\pgfqpoint{4.788519in}{1.918062in}}%
\pgfpathlineto{\pgfqpoint{4.802832in}{1.921287in}}%
\pgfpathlineto{\pgfqpoint{4.817158in}{1.924609in}}%
\pgfpathlineto{\pgfqpoint{4.809247in}{1.911011in}}%
\pgfpathlineto{\pgfqpoint{4.801333in}{1.897419in}}%
\pgfpathlineto{\pgfqpoint{4.793414in}{1.883837in}}%
\pgfpathlineto{\pgfqpoint{4.785491in}{1.870268in}}%
\pgfpathlineto{\pgfqpoint{4.771166in}{1.867292in}}%
\pgfpathlineto{\pgfqpoint{4.756852in}{1.864414in}}%
\pgfpathlineto{\pgfqpoint{4.742551in}{1.861633in}}%
\pgfpathlineto{\pgfqpoint{4.728261in}{1.858950in}}%
\pgfpathlineto{\pgfqpoint{4.736184in}{1.872165in}}%
\pgfpathlineto{\pgfqpoint{4.744102in}{1.885398in}}%
\pgfpathlineto{\pgfqpoint{4.752017in}{1.898646in}}%
\pgfpathlineto{\pgfqpoint{4.759927in}{1.911905in}}%
\pgfpathclose%
\pgfusepath{fill}%
\end{pgfscope}%
\begin{pgfscope}%
\pgfpathrectangle{\pgfqpoint{1.150000in}{0.150000in}}{\pgfqpoint{5.700000in}{5.700000in}}%
\pgfusepath{clip}%
\pgfsetbuttcap%
\pgfsetroundjoin%
\definecolor{currentfill}{rgb}{0.140210,0.665859,0.513427}%
\pgfsetfillcolor{currentfill}%
\pgfsetfillopacity{0.700000}%
\pgfsetlinewidth{0.000000pt}%
\definecolor{currentstroke}{rgb}{0.000000,0.000000,0.000000}%
\pgfsetstrokecolor{currentstroke}%
\pgfsetdash{}{0pt}%
\pgfpathmoveto{\pgfqpoint{2.125904in}{3.153870in}}%
\pgfpathlineto{\pgfqpoint{2.140177in}{3.131024in}}%
\pgfpathlineto{\pgfqpoint{2.154439in}{3.108386in}}%
\pgfpathlineto{\pgfqpoint{2.168690in}{3.085956in}}%
\pgfpathlineto{\pgfqpoint{2.182931in}{3.063732in}}%
\pgfpathlineto{\pgfqpoint{2.173329in}{3.077850in}}%
\pgfpathlineto{\pgfqpoint{2.163697in}{3.092402in}}%
\pgfpathlineto{\pgfqpoint{2.154035in}{3.107397in}}%
\pgfpathlineto{\pgfqpoint{2.144342in}{3.122841in}}%
\pgfpathlineto{\pgfqpoint{2.130030in}{3.145707in}}%
\pgfpathlineto{\pgfqpoint{2.115708in}{3.168780in}}%
\pgfpathlineto{\pgfqpoint{2.101374in}{3.192062in}}%
\pgfpathlineto{\pgfqpoint{2.087029in}{3.215555in}}%
\pgfpathlineto{\pgfqpoint{2.096795in}{3.199457in}}%
\pgfpathlineto{\pgfqpoint{2.106529in}{3.183815in}}%
\pgfpathlineto{\pgfqpoint{2.116232in}{3.168622in}}%
\pgfpathlineto{\pgfqpoint{2.125904in}{3.153870in}}%
\pgfpathclose%
\pgfusepath{fill}%
\end{pgfscope}%
\begin{pgfscope}%
\pgfpathrectangle{\pgfqpoint{1.150000in}{0.150000in}}{\pgfqpoint{5.700000in}{5.700000in}}%
\pgfusepath{clip}%
\pgfsetbuttcap%
\pgfsetroundjoin%
\definecolor{currentfill}{rgb}{0.174274,0.445044,0.557792}%
\pgfsetfillcolor{currentfill}%
\pgfsetfillopacity{0.700000}%
\pgfsetlinewidth{0.000000pt}%
\definecolor{currentstroke}{rgb}{0.000000,0.000000,0.000000}%
\pgfsetstrokecolor{currentstroke}%
\pgfsetdash{}{0pt}%
\pgfpathmoveto{\pgfqpoint{5.388267in}{2.528667in}}%
\pgfpathlineto{\pgfqpoint{5.402899in}{2.536375in}}%
\pgfpathlineto{\pgfqpoint{5.417546in}{2.544184in}}%
\pgfpathlineto{\pgfqpoint{5.432210in}{2.552093in}}%
\pgfpathlineto{\pgfqpoint{5.446889in}{2.560102in}}%
\pgfpathlineto{\pgfqpoint{5.439158in}{2.546951in}}%
\pgfpathlineto{\pgfqpoint{5.431421in}{2.533695in}}%
\pgfpathlineto{\pgfqpoint{5.423677in}{2.520333in}}%
\pgfpathlineto{\pgfqpoint{5.415927in}{2.506869in}}%
\pgfpathlineto{\pgfqpoint{5.401250in}{2.499043in}}%
\pgfpathlineto{\pgfqpoint{5.386588in}{2.491317in}}%
\pgfpathlineto{\pgfqpoint{5.371943in}{2.483691in}}%
\pgfpathlineto{\pgfqpoint{5.357312in}{2.476166in}}%
\pgfpathlineto{\pgfqpoint{5.365061in}{2.489440in}}%
\pgfpathlineto{\pgfqpoint{5.372803in}{2.502615in}}%
\pgfpathlineto{\pgfqpoint{5.380539in}{2.515692in}}%
\pgfpathlineto{\pgfqpoint{5.388267in}{2.528667in}}%
\pgfpathclose%
\pgfusepath{fill}%
\end{pgfscope}%
\begin{pgfscope}%
\pgfpathrectangle{\pgfqpoint{1.150000in}{0.150000in}}{\pgfqpoint{5.700000in}{5.700000in}}%
\pgfusepath{clip}%
\pgfsetbuttcap%
\pgfsetroundjoin%
\definecolor{currentfill}{rgb}{0.280255,0.165693,0.476498}%
\pgfsetfillcolor{currentfill}%
\pgfsetfillopacity{0.700000}%
\pgfsetlinewidth{0.000000pt}%
\definecolor{currentstroke}{rgb}{0.000000,0.000000,0.000000}%
\pgfsetstrokecolor{currentstroke}%
\pgfsetdash{}{0pt}%
\pgfpathmoveto{\pgfqpoint{3.248421in}{1.860124in}}%
\pgfpathlineto{\pgfqpoint{3.262366in}{1.849572in}}%
\pgfpathlineto{\pgfqpoint{3.276311in}{1.839138in}}%
\pgfpathlineto{\pgfqpoint{3.290258in}{1.828819in}}%
\pgfpathlineto{\pgfqpoint{3.304206in}{1.818616in}}%
\pgfpathlineto{\pgfqpoint{3.295645in}{1.820927in}}%
\pgfpathlineto{\pgfqpoint{3.287070in}{1.823552in}}%
\pgfpathlineto{\pgfqpoint{3.278479in}{1.826498in}}%
\pgfpathlineto{\pgfqpoint{3.269873in}{1.829770in}}%
\pgfpathlineto{\pgfqpoint{3.255887in}{1.840546in}}%
\pgfpathlineto{\pgfqpoint{3.241902in}{1.851437in}}%
\pgfpathlineto{\pgfqpoint{3.227918in}{1.862445in}}%
\pgfpathlineto{\pgfqpoint{3.213935in}{1.873571in}}%
\pgfpathlineto{\pgfqpoint{3.222580in}{1.869717in}}%
\pgfpathlineto{\pgfqpoint{3.231209in}{1.866196in}}%
\pgfpathlineto{\pgfqpoint{3.239823in}{1.863000in}}%
\pgfpathlineto{\pgfqpoint{3.248421in}{1.860124in}}%
\pgfpathclose%
\pgfusepath{fill}%
\end{pgfscope}%
\begin{pgfscope}%
\pgfpathrectangle{\pgfqpoint{1.150000in}{0.150000in}}{\pgfqpoint{5.700000in}{5.700000in}}%
\pgfusepath{clip}%
\pgfsetbuttcap%
\pgfsetroundjoin%
\definecolor{currentfill}{rgb}{0.272594,0.025563,0.353093}%
\pgfsetfillcolor{currentfill}%
\pgfsetfillopacity{0.700000}%
\pgfsetlinewidth{0.000000pt}%
\definecolor{currentstroke}{rgb}{0.000000,0.000000,0.000000}%
\pgfsetstrokecolor{currentstroke}%
\pgfsetdash{}{0pt}%
\pgfpathmoveto{\pgfqpoint{4.140166in}{1.589251in}}%
\pgfpathlineto{\pgfqpoint{4.154221in}{1.586758in}}%
\pgfpathlineto{\pgfqpoint{4.168285in}{1.584363in}}%
\pgfpathlineto{\pgfqpoint{4.182356in}{1.582068in}}%
\pgfpathlineto{\pgfqpoint{4.196435in}{1.579871in}}%
\pgfpathlineto{\pgfqpoint{4.188364in}{1.571072in}}%
\pgfpathlineto{\pgfqpoint{4.180288in}{1.562418in}}%
\pgfpathlineto{\pgfqpoint{4.172206in}{1.553916in}}%
\pgfpathlineto{\pgfqpoint{4.164118in}{1.545569in}}%
\pgfpathlineto{\pgfqpoint{4.150027in}{1.548234in}}%
\pgfpathlineto{\pgfqpoint{4.135944in}{1.550998in}}%
\pgfpathlineto{\pgfqpoint{4.121868in}{1.553860in}}%
\pgfpathlineto{\pgfqpoint{4.107800in}{1.556821in}}%
\pgfpathlineto{\pgfqpoint{4.115900in}{1.564693in}}%
\pgfpathlineto{\pgfqpoint{4.123995in}{1.572725in}}%
\pgfpathlineto{\pgfqpoint{4.132083in}{1.580912in}}%
\pgfpathlineto{\pgfqpoint{4.140166in}{1.589251in}}%
\pgfpathclose%
\pgfusepath{fill}%
\end{pgfscope}%
\begin{pgfscope}%
\pgfpathrectangle{\pgfqpoint{1.150000in}{0.150000in}}{\pgfqpoint{5.700000in}{5.700000in}}%
\pgfusepath{clip}%
\pgfsetbuttcap%
\pgfsetroundjoin%
\definecolor{currentfill}{rgb}{0.141935,0.526453,0.555991}%
\pgfsetfillcolor{currentfill}%
\pgfsetfillopacity{0.700000}%
\pgfsetlinewidth{0.000000pt}%
\definecolor{currentstroke}{rgb}{0.000000,0.000000,0.000000}%
\pgfsetstrokecolor{currentstroke}%
\pgfsetdash{}{0pt}%
\pgfpathmoveto{\pgfqpoint{5.597968in}{2.743947in}}%
\pgfpathlineto{\pgfqpoint{5.612730in}{2.752895in}}%
\pgfpathlineto{\pgfqpoint{5.627508in}{2.761944in}}%
\pgfpathlineto{\pgfqpoint{5.642304in}{2.771094in}}%
\pgfpathlineto{\pgfqpoint{5.657117in}{2.780347in}}%
\pgfpathlineto{\pgfqpoint{5.649476in}{2.768332in}}%
\pgfpathlineto{\pgfqpoint{5.641827in}{2.756188in}}%
\pgfpathlineto{\pgfqpoint{5.634170in}{2.743914in}}%
\pgfpathlineto{\pgfqpoint{5.626505in}{2.731512in}}%
\pgfpathlineto{\pgfqpoint{5.611693in}{2.722384in}}%
\pgfpathlineto{\pgfqpoint{5.596897in}{2.713358in}}%
\pgfpathlineto{\pgfqpoint{5.582119in}{2.704434in}}%
\pgfpathlineto{\pgfqpoint{5.567356in}{2.695611in}}%
\pgfpathlineto{\pgfqpoint{5.575021in}{2.707881in}}%
\pgfpathlineto{\pgfqpoint{5.582678in}{2.720028in}}%
\pgfpathlineto{\pgfqpoint{5.590327in}{2.732050in}}%
\pgfpathlineto{\pgfqpoint{5.597968in}{2.743947in}}%
\pgfpathclose%
\pgfusepath{fill}%
\end{pgfscope}%
\begin{pgfscope}%
\pgfpathrectangle{\pgfqpoint{1.150000in}{0.150000in}}{\pgfqpoint{5.700000in}{5.700000in}}%
\pgfusepath{clip}%
\pgfsetbuttcap%
\pgfsetroundjoin%
\definecolor{currentfill}{rgb}{0.235526,0.309527,0.542944}%
\pgfsetfillcolor{currentfill}%
\pgfsetfillopacity{0.700000}%
\pgfsetlinewidth{0.000000pt}%
\definecolor{currentstroke}{rgb}{0.000000,0.000000,0.000000}%
\pgfsetstrokecolor{currentstroke}%
\pgfsetdash{}{0pt}%
\pgfpathmoveto{\pgfqpoint{5.058212in}{2.180431in}}%
\pgfpathlineto{\pgfqpoint{5.072659in}{2.185845in}}%
\pgfpathlineto{\pgfqpoint{5.087120in}{2.191358in}}%
\pgfpathlineto{\pgfqpoint{5.101594in}{2.196970in}}%
\pgfpathlineto{\pgfqpoint{5.116083in}{2.202680in}}%
\pgfpathlineto{\pgfqpoint{5.108240in}{2.188568in}}%
\pgfpathlineto{\pgfqpoint{5.100393in}{2.174403in}}%
\pgfpathlineto{\pgfqpoint{5.092541in}{2.160189in}}%
\pgfpathlineto{\pgfqpoint{5.084683in}{2.145928in}}%
\pgfpathlineto{\pgfqpoint{5.070198in}{2.140493in}}%
\pgfpathlineto{\pgfqpoint{5.055726in}{2.135158in}}%
\pgfpathlineto{\pgfqpoint{5.041268in}{2.129920in}}%
\pgfpathlineto{\pgfqpoint{5.026823in}{2.124781in}}%
\pgfpathlineto{\pgfqpoint{5.034678in}{2.138760in}}%
\pgfpathlineto{\pgfqpoint{5.042527in}{2.152696in}}%
\pgfpathlineto{\pgfqpoint{5.050372in}{2.166587in}}%
\pgfpathlineto{\pgfqpoint{5.058212in}{2.180431in}}%
\pgfpathclose%
\pgfusepath{fill}%
\end{pgfscope}%
\begin{pgfscope}%
\pgfpathrectangle{\pgfqpoint{1.150000in}{0.150000in}}{\pgfqpoint{5.700000in}{5.700000in}}%
\pgfusepath{clip}%
\pgfsetbuttcap%
\pgfsetroundjoin%
\definecolor{currentfill}{rgb}{0.267968,0.223549,0.512008}%
\pgfsetfillcolor{currentfill}%
\pgfsetfillopacity{0.700000}%
\pgfsetlinewidth{0.000000pt}%
\definecolor{currentstroke}{rgb}{0.000000,0.000000,0.000000}%
\pgfsetstrokecolor{currentstroke}%
\pgfsetdash{}{0pt}%
\pgfpathmoveto{\pgfqpoint{4.848757in}{1.979008in}}%
\pgfpathlineto{\pgfqpoint{4.863095in}{1.982757in}}%
\pgfpathlineto{\pgfqpoint{4.877446in}{1.986604in}}%
\pgfpathlineto{\pgfqpoint{4.891809in}{1.990549in}}%
\pgfpathlineto{\pgfqpoint{4.906185in}{1.994591in}}%
\pgfpathlineto{\pgfqpoint{4.898290in}{1.980675in}}%
\pgfpathlineto{\pgfqpoint{4.890392in}{1.966748in}}%
\pgfpathlineto{\pgfqpoint{4.882489in}{1.952814in}}%
\pgfpathlineto{\pgfqpoint{4.874582in}{1.938875in}}%
\pgfpathlineto{\pgfqpoint{4.860207in}{1.935162in}}%
\pgfpathlineto{\pgfqpoint{4.845845in}{1.931547in}}%
\pgfpathlineto{\pgfqpoint{4.831495in}{1.928029in}}%
\pgfpathlineto{\pgfqpoint{4.817158in}{1.924609in}}%
\pgfpathlineto{\pgfqpoint{4.825064in}{1.938211in}}%
\pgfpathlineto{\pgfqpoint{4.832966in}{1.951814in}}%
\pgfpathlineto{\pgfqpoint{4.840863in}{1.965414in}}%
\pgfpathlineto{\pgfqpoint{4.848757in}{1.979008in}}%
\pgfpathclose%
\pgfusepath{fill}%
\end{pgfscope}%
\begin{pgfscope}%
\pgfpathrectangle{\pgfqpoint{1.150000in}{0.150000in}}{\pgfqpoint{5.700000in}{5.700000in}}%
\pgfusepath{clip}%
\pgfsetbuttcap%
\pgfsetroundjoin%
\definecolor{currentfill}{rgb}{0.194100,0.399323,0.555565}%
\pgfsetfillcolor{currentfill}%
\pgfsetfillopacity{0.700000}%
\pgfsetlinewidth{0.000000pt}%
\definecolor{currentstroke}{rgb}{0.000000,0.000000,0.000000}%
\pgfsetstrokecolor{currentstroke}%
\pgfsetdash{}{0pt}%
\pgfpathmoveto{\pgfqpoint{5.267893in}{2.393830in}}%
\pgfpathlineto{\pgfqpoint{5.282461in}{2.400754in}}%
\pgfpathlineto{\pgfqpoint{5.297043in}{2.407777in}}%
\pgfpathlineto{\pgfqpoint{5.311641in}{2.414901in}}%
\pgfpathlineto{\pgfqpoint{5.326254in}{2.422124in}}%
\pgfpathlineto{\pgfqpoint{5.318473in}{2.408386in}}%
\pgfpathlineto{\pgfqpoint{5.310687in}{2.394560in}}%
\pgfpathlineto{\pgfqpoint{5.302894in}{2.380648in}}%
\pgfpathlineto{\pgfqpoint{5.295096in}{2.366653in}}%
\pgfpathlineto{\pgfqpoint{5.280486in}{2.359651in}}%
\pgfpathlineto{\pgfqpoint{5.265891in}{2.352748in}}%
\pgfpathlineto{\pgfqpoint{5.251311in}{2.345945in}}%
\pgfpathlineto{\pgfqpoint{5.236745in}{2.339242in}}%
\pgfpathlineto{\pgfqpoint{5.244541in}{2.353009in}}%
\pgfpathlineto{\pgfqpoint{5.252331in}{2.366697in}}%
\pgfpathlineto{\pgfqpoint{5.260115in}{2.380305in}}%
\pgfpathlineto{\pgfqpoint{5.267893in}{2.393830in}}%
\pgfpathclose%
\pgfusepath{fill}%
\end{pgfscope}%
\begin{pgfscope}%
\pgfpathrectangle{\pgfqpoint{1.150000in}{0.150000in}}{\pgfqpoint{5.700000in}{5.700000in}}%
\pgfusepath{clip}%
\pgfsetbuttcap%
\pgfsetroundjoin%
\definecolor{currentfill}{rgb}{0.271305,0.019942,0.347269}%
\pgfsetfillcolor{currentfill}%
\pgfsetfillopacity{0.700000}%
\pgfsetlinewidth{0.000000pt}%
\definecolor{currentstroke}{rgb}{0.000000,0.000000,0.000000}%
\pgfsetstrokecolor{currentstroke}%
\pgfsetdash{}{0pt}%
\pgfpathmoveto{\pgfqpoint{3.906858in}{1.576160in}}%
\pgfpathlineto{\pgfqpoint{3.920863in}{1.571513in}}%
\pgfpathlineto{\pgfqpoint{3.934874in}{1.566968in}}%
\pgfpathlineto{\pgfqpoint{3.948891in}{1.562523in}}%
\pgfpathlineto{\pgfqpoint{3.962915in}{1.558179in}}%
\pgfpathlineto{\pgfqpoint{3.954749in}{1.552174in}}%
\pgfpathlineto{\pgfqpoint{3.946576in}{1.546367in}}%
\pgfpathlineto{\pgfqpoint{3.938396in}{1.540764in}}%
\pgfpathlineto{\pgfqpoint{3.930207in}{1.535369in}}%
\pgfpathlineto{\pgfqpoint{3.916166in}{1.540218in}}%
\pgfpathlineto{\pgfqpoint{3.902130in}{1.545167in}}%
\pgfpathlineto{\pgfqpoint{3.888100in}{1.550217in}}%
\pgfpathlineto{\pgfqpoint{3.874075in}{1.555369in}}%
\pgfpathlineto{\pgfqpoint{3.882283in}{1.560252in}}%
\pgfpathlineto{\pgfqpoint{3.890482in}{1.565348in}}%
\pgfpathlineto{\pgfqpoint{3.898674in}{1.570653in}}%
\pgfpathlineto{\pgfqpoint{3.906858in}{1.576160in}}%
\pgfpathclose%
\pgfusepath{fill}%
\end{pgfscope}%
\begin{pgfscope}%
\pgfpathrectangle{\pgfqpoint{1.150000in}{0.150000in}}{\pgfqpoint{5.700000in}{5.700000in}}%
\pgfusepath{clip}%
\pgfsetbuttcap%
\pgfsetroundjoin%
\definecolor{currentfill}{rgb}{0.280894,0.078907,0.402329}%
\pgfsetfillcolor{currentfill}%
\pgfsetfillopacity{0.700000}%
\pgfsetlinewidth{0.000000pt}%
\definecolor{currentstroke}{rgb}{0.000000,0.000000,0.000000}%
\pgfsetstrokecolor{currentstroke}%
\pgfsetdash{}{0pt}%
\pgfpathmoveto{\pgfqpoint{3.561233in}{1.673278in}}%
\pgfpathlineto{\pgfqpoint{3.575191in}{1.665517in}}%
\pgfpathlineto{\pgfqpoint{3.589152in}{1.657863in}}%
\pgfpathlineto{\pgfqpoint{3.603117in}{1.650316in}}%
\pgfpathlineto{\pgfqpoint{3.617086in}{1.642876in}}%
\pgfpathlineto{\pgfqpoint{3.608736in}{1.641287in}}%
\pgfpathlineto{\pgfqpoint{3.600375in}{1.639963in}}%
\pgfpathlineto{\pgfqpoint{3.592002in}{1.638909in}}%
\pgfpathlineto{\pgfqpoint{3.583619in}{1.638132in}}%
\pgfpathlineto{\pgfqpoint{3.569622in}{1.646118in}}%
\pgfpathlineto{\pgfqpoint{3.555628in}{1.654211in}}%
\pgfpathlineto{\pgfqpoint{3.541638in}{1.662411in}}%
\pgfpathlineto{\pgfqpoint{3.527650in}{1.670718in}}%
\pgfpathlineto{\pgfqpoint{3.536064in}{1.670941in}}%
\pgfpathlineto{\pgfqpoint{3.544465in}{1.671446in}}%
\pgfpathlineto{\pgfqpoint{3.552855in}{1.672227in}}%
\pgfpathlineto{\pgfqpoint{3.561233in}{1.673278in}}%
\pgfpathclose%
\pgfusepath{fill}%
\end{pgfscope}%
\begin{pgfscope}%
\pgfpathrectangle{\pgfqpoint{1.150000in}{0.150000in}}{\pgfqpoint{5.700000in}{5.700000in}}%
\pgfusepath{clip}%
\pgfsetbuttcap%
\pgfsetroundjoin%
\definecolor{currentfill}{rgb}{0.281887,0.150881,0.465405}%
\pgfsetfillcolor{currentfill}%
\pgfsetfillopacity{0.700000}%
\pgfsetlinewidth{0.000000pt}%
\definecolor{currentstroke}{rgb}{0.000000,0.000000,0.000000}%
\pgfsetstrokecolor{currentstroke}%
\pgfsetdash{}{0pt}%
\pgfpathmoveto{\pgfqpoint{3.304206in}{1.818616in}}%
\pgfpathlineto{\pgfqpoint{3.318156in}{1.808529in}}%
\pgfpathlineto{\pgfqpoint{3.332107in}{1.798556in}}%
\pgfpathlineto{\pgfqpoint{3.346060in}{1.788698in}}%
\pgfpathlineto{\pgfqpoint{3.360015in}{1.778953in}}%
\pgfpathlineto{\pgfqpoint{3.351490in}{1.780700in}}%
\pgfpathlineto{\pgfqpoint{3.342951in}{1.782756in}}%
\pgfpathlineto{\pgfqpoint{3.334397in}{1.785128in}}%
\pgfpathlineto{\pgfqpoint{3.325829in}{1.787820in}}%
\pgfpathlineto{\pgfqpoint{3.311838in}{1.798136in}}%
\pgfpathlineto{\pgfqpoint{3.297848in}{1.808566in}}%
\pgfpathlineto{\pgfqpoint{3.283860in}{1.819111in}}%
\pgfpathlineto{\pgfqpoint{3.269873in}{1.829770in}}%
\pgfpathlineto{\pgfqpoint{3.278479in}{1.826498in}}%
\pgfpathlineto{\pgfqpoint{3.287070in}{1.823552in}}%
\pgfpathlineto{\pgfqpoint{3.295645in}{1.820927in}}%
\pgfpathlineto{\pgfqpoint{3.304206in}{1.818616in}}%
\pgfpathclose%
\pgfusepath{fill}%
\end{pgfscope}%
\begin{pgfscope}%
\pgfpathrectangle{\pgfqpoint{1.150000in}{0.150000in}}{\pgfqpoint{5.700000in}{5.700000in}}%
\pgfusepath{clip}%
\pgfsetbuttcap%
\pgfsetroundjoin%
\definecolor{currentfill}{rgb}{0.188923,0.410910,0.556326}%
\pgfsetfillcolor{currentfill}%
\pgfsetfillopacity{0.700000}%
\pgfsetlinewidth{0.000000pt}%
\definecolor{currentstroke}{rgb}{0.000000,0.000000,0.000000}%
\pgfsetstrokecolor{currentstroke}%
\pgfsetdash{}{0pt}%
\pgfpathmoveto{\pgfqpoint{2.653996in}{2.422435in}}%
\pgfpathlineto{\pgfqpoint{2.668040in}{2.406037in}}%
\pgfpathlineto{\pgfqpoint{2.682081in}{2.389788in}}%
\pgfpathlineto{\pgfqpoint{2.696118in}{2.373687in}}%
\pgfpathlineto{\pgfqpoint{2.710151in}{2.357733in}}%
\pgfpathlineto{\pgfqpoint{2.701061in}{2.367309in}}%
\pgfpathlineto{\pgfqpoint{2.691948in}{2.377282in}}%
\pgfpathlineto{\pgfqpoint{2.682812in}{2.387659in}}%
\pgfpathlineto{\pgfqpoint{2.673651in}{2.398447in}}%
\pgfpathlineto{\pgfqpoint{2.659562in}{2.415022in}}%
\pgfpathlineto{\pgfqpoint{2.645469in}{2.431744in}}%
\pgfpathlineto{\pgfqpoint{2.631371in}{2.448616in}}%
\pgfpathlineto{\pgfqpoint{2.617269in}{2.465638in}}%
\pgfpathlineto{\pgfqpoint{2.626488in}{2.454218in}}%
\pgfpathlineto{\pgfqpoint{2.635681in}{2.443216in}}%
\pgfpathlineto{\pgfqpoint{2.644850in}{2.432624in}}%
\pgfpathlineto{\pgfqpoint{2.653996in}{2.422435in}}%
\pgfpathclose%
\pgfusepath{fill}%
\end{pgfscope}%
\begin{pgfscope}%
\pgfpathrectangle{\pgfqpoint{1.150000in}{0.150000in}}{\pgfqpoint{5.700000in}{5.700000in}}%
\pgfusepath{clip}%
\pgfsetbuttcap%
\pgfsetroundjoin%
\definecolor{currentfill}{rgb}{0.199430,0.387607,0.554642}%
\pgfsetfillcolor{currentfill}%
\pgfsetfillopacity{0.700000}%
\pgfsetlinewidth{0.000000pt}%
\definecolor{currentstroke}{rgb}{0.000000,0.000000,0.000000}%
\pgfsetstrokecolor{currentstroke}%
\pgfsetdash{}{0pt}%
\pgfpathmoveto{\pgfqpoint{2.710151in}{2.357733in}}%
\pgfpathlineto{\pgfqpoint{2.724180in}{2.341925in}}%
\pgfpathlineto{\pgfqpoint{2.738206in}{2.326261in}}%
\pgfpathlineto{\pgfqpoint{2.752228in}{2.310742in}}%
\pgfpathlineto{\pgfqpoint{2.766248in}{2.295365in}}%
\pgfpathlineto{\pgfqpoint{2.757212in}{2.304332in}}%
\pgfpathlineto{\pgfqpoint{2.748154in}{2.313689in}}%
\pgfpathlineto{\pgfqpoint{2.739073in}{2.323445in}}%
\pgfpathlineto{\pgfqpoint{2.729969in}{2.333605in}}%
\pgfpathlineto{\pgfqpoint{2.715895in}{2.349599in}}%
\pgfpathlineto{\pgfqpoint{2.701818in}{2.365736in}}%
\pgfpathlineto{\pgfqpoint{2.687736in}{2.382019in}}%
\pgfpathlineto{\pgfqpoint{2.673651in}{2.398447in}}%
\pgfpathlineto{\pgfqpoint{2.682812in}{2.387659in}}%
\pgfpathlineto{\pgfqpoint{2.691948in}{2.377282in}}%
\pgfpathlineto{\pgfqpoint{2.701061in}{2.367309in}}%
\pgfpathlineto{\pgfqpoint{2.710151in}{2.357733in}}%
\pgfpathclose%
\pgfusepath{fill}%
\end{pgfscope}%
\begin{pgfscope}%
\pgfpathrectangle{\pgfqpoint{1.150000in}{0.150000in}}{\pgfqpoint{5.700000in}{5.700000in}}%
\pgfusepath{clip}%
\pgfsetbuttcap%
\pgfsetroundjoin%
\definecolor{currentfill}{rgb}{0.281924,0.089666,0.412415}%
\pgfsetfillcolor{currentfill}%
\pgfsetfillopacity{0.700000}%
\pgfsetlinewidth{0.000000pt}%
\definecolor{currentstroke}{rgb}{0.000000,0.000000,0.000000}%
\pgfsetstrokecolor{currentstroke}%
\pgfsetdash{}{0pt}%
\pgfpathmoveto{\pgfqpoint{4.462161in}{1.692110in}}%
\pgfpathlineto{\pgfqpoint{4.476330in}{1.692483in}}%
\pgfpathlineto{\pgfqpoint{4.490510in}{1.692953in}}%
\pgfpathlineto{\pgfqpoint{4.504699in}{1.693520in}}%
\pgfpathlineto{\pgfqpoint{4.518899in}{1.694185in}}%
\pgfpathlineto{\pgfqpoint{4.510917in}{1.682338in}}%
\pgfpathlineto{\pgfqpoint{4.502931in}{1.670568in}}%
\pgfpathlineto{\pgfqpoint{4.494940in}{1.658879in}}%
\pgfpathlineto{\pgfqpoint{4.486945in}{1.647275in}}%
\pgfpathlineto{\pgfqpoint{4.472741in}{1.647027in}}%
\pgfpathlineto{\pgfqpoint{4.458547in}{1.646876in}}%
\pgfpathlineto{\pgfqpoint{4.444363in}{1.646821in}}%
\pgfpathlineto{\pgfqpoint{4.430188in}{1.646865in}}%
\pgfpathlineto{\pgfqpoint{4.438188in}{1.658046in}}%
\pgfpathlineto{\pgfqpoint{4.446183in}{1.669316in}}%
\pgfpathlineto{\pgfqpoint{4.454174in}{1.680672in}}%
\pgfpathlineto{\pgfqpoint{4.462161in}{1.692110in}}%
\pgfpathclose%
\pgfusepath{fill}%
\end{pgfscope}%
\begin{pgfscope}%
\pgfpathrectangle{\pgfqpoint{1.150000in}{0.150000in}}{\pgfqpoint{5.700000in}{5.700000in}}%
\pgfusepath{clip}%
\pgfsetbuttcap%
\pgfsetroundjoin%
\definecolor{currentfill}{rgb}{0.274952,0.037752,0.364543}%
\pgfsetfillcolor{currentfill}%
\pgfsetfillopacity{0.700000}%
\pgfsetlinewidth{0.000000pt}%
\definecolor{currentstroke}{rgb}{0.000000,0.000000,0.000000}%
\pgfsetstrokecolor{currentstroke}%
\pgfsetdash{}{0pt}%
\pgfpathmoveto{\pgfqpoint{3.762072in}{1.600256in}}%
\pgfpathlineto{\pgfqpoint{3.776055in}{1.594285in}}%
\pgfpathlineto{\pgfqpoint{3.790042in}{1.588418in}}%
\pgfpathlineto{\pgfqpoint{3.804035in}{1.582654in}}%
\pgfpathlineto{\pgfqpoint{3.818032in}{1.576992in}}%
\pgfpathlineto{\pgfqpoint{3.809795in}{1.572845in}}%
\pgfpathlineto{\pgfqpoint{3.801549in}{1.568926in}}%
\pgfpathlineto{\pgfqpoint{3.793294in}{1.565241in}}%
\pgfpathlineto{\pgfqpoint{3.785030in}{1.561795in}}%
\pgfpathlineto{\pgfqpoint{3.771010in}{1.567981in}}%
\pgfpathlineto{\pgfqpoint{3.756995in}{1.574269in}}%
\pgfpathlineto{\pgfqpoint{3.742984in}{1.580661in}}%
\pgfpathlineto{\pgfqpoint{3.728978in}{1.587156in}}%
\pgfpathlineto{\pgfqpoint{3.737266in}{1.590070in}}%
\pgfpathlineto{\pgfqpoint{3.745544in}{1.593228in}}%
\pgfpathlineto{\pgfqpoint{3.753813in}{1.596625in}}%
\pgfpathlineto{\pgfqpoint{3.762072in}{1.600256in}}%
\pgfpathclose%
\pgfusepath{fill}%
\end{pgfscope}%
\begin{pgfscope}%
\pgfpathrectangle{\pgfqpoint{1.150000in}{0.150000in}}{\pgfqpoint{5.700000in}{5.700000in}}%
\pgfusepath{clip}%
\pgfsetbuttcap%
\pgfsetroundjoin%
\definecolor{currentfill}{rgb}{0.179019,0.433756,0.557430}%
\pgfsetfillcolor{currentfill}%
\pgfsetfillopacity{0.700000}%
\pgfsetlinewidth{0.000000pt}%
\definecolor{currentstroke}{rgb}{0.000000,0.000000,0.000000}%
\pgfsetstrokecolor{currentstroke}%
\pgfsetdash{}{0pt}%
\pgfpathmoveto{\pgfqpoint{2.597774in}{2.489540in}}%
\pgfpathlineto{\pgfqpoint{2.611836in}{2.472535in}}%
\pgfpathlineto{\pgfqpoint{2.625894in}{2.455683in}}%
\pgfpathlineto{\pgfqpoint{2.639947in}{2.438983in}}%
\pgfpathlineto{\pgfqpoint{2.653996in}{2.422435in}}%
\pgfpathlineto{\pgfqpoint{2.644850in}{2.432624in}}%
\pgfpathlineto{\pgfqpoint{2.635681in}{2.443216in}}%
\pgfpathlineto{\pgfqpoint{2.626488in}{2.454218in}}%
\pgfpathlineto{\pgfqpoint{2.617269in}{2.465638in}}%
\pgfpathlineto{\pgfqpoint{2.603163in}{2.482810in}}%
\pgfpathlineto{\pgfqpoint{2.589051in}{2.500135in}}%
\pgfpathlineto{\pgfqpoint{2.574935in}{2.517614in}}%
\pgfpathlineto{\pgfqpoint{2.560813in}{2.535247in}}%
\pgfpathlineto{\pgfqpoint{2.570091in}{2.523192in}}%
\pgfpathlineto{\pgfqpoint{2.579344in}{2.511560in}}%
\pgfpathlineto{\pgfqpoint{2.588571in}{2.500345in}}%
\pgfpathlineto{\pgfqpoint{2.597774in}{2.489540in}}%
\pgfpathclose%
\pgfusepath{fill}%
\end{pgfscope}%
\begin{pgfscope}%
\pgfpathrectangle{\pgfqpoint{1.150000in}{0.150000in}}{\pgfqpoint{5.700000in}{5.700000in}}%
\pgfusepath{clip}%
\pgfsetbuttcap%
\pgfsetroundjoin%
\definecolor{currentfill}{rgb}{0.210503,0.363727,0.552206}%
\pgfsetfillcolor{currentfill}%
\pgfsetfillopacity{0.700000}%
\pgfsetlinewidth{0.000000pt}%
\definecolor{currentstroke}{rgb}{0.000000,0.000000,0.000000}%
\pgfsetstrokecolor{currentstroke}%
\pgfsetdash{}{0pt}%
\pgfpathmoveto{\pgfqpoint{2.766248in}{2.295365in}}%
\pgfpathlineto{\pgfqpoint{2.780264in}{2.280131in}}%
\pgfpathlineto{\pgfqpoint{2.794277in}{2.265037in}}%
\pgfpathlineto{\pgfqpoint{2.808288in}{2.250083in}}%
\pgfpathlineto{\pgfqpoint{2.822296in}{2.235268in}}%
\pgfpathlineto{\pgfqpoint{2.813312in}{2.243628in}}%
\pgfpathlineto{\pgfqpoint{2.804307in}{2.252374in}}%
\pgfpathlineto{\pgfqpoint{2.795280in}{2.261511in}}%
\pgfpathlineto{\pgfqpoint{2.786231in}{2.271046in}}%
\pgfpathlineto{\pgfqpoint{2.772170in}{2.286475in}}%
\pgfpathlineto{\pgfqpoint{2.758106in}{2.302044in}}%
\pgfpathlineto{\pgfqpoint{2.744039in}{2.317753in}}%
\pgfpathlineto{\pgfqpoint{2.729969in}{2.333605in}}%
\pgfpathlineto{\pgfqpoint{2.739073in}{2.323445in}}%
\pgfpathlineto{\pgfqpoint{2.748154in}{2.313689in}}%
\pgfpathlineto{\pgfqpoint{2.757212in}{2.304332in}}%
\pgfpathlineto{\pgfqpoint{2.766248in}{2.295365in}}%
\pgfpathclose%
\pgfusepath{fill}%
\end{pgfscope}%
\begin{pgfscope}%
\pgfpathrectangle{\pgfqpoint{1.150000in}{0.150000in}}{\pgfqpoint{5.700000in}{5.700000in}}%
\pgfusepath{clip}%
\pgfsetbuttcap%
\pgfsetroundjoin%
\definecolor{currentfill}{rgb}{0.283091,0.110553,0.431554}%
\pgfsetfillcolor{currentfill}%
\pgfsetfillopacity{0.700000}%
\pgfsetlinewidth{0.000000pt}%
\definecolor{currentstroke}{rgb}{0.000000,0.000000,0.000000}%
\pgfsetstrokecolor{currentstroke}%
\pgfsetdash{}{0pt}%
\pgfpathmoveto{\pgfqpoint{4.550783in}{1.742267in}}%
\pgfpathlineto{\pgfqpoint{4.564989in}{1.743427in}}%
\pgfpathlineto{\pgfqpoint{4.579206in}{1.744685in}}%
\pgfpathlineto{\pgfqpoint{4.593433in}{1.746039in}}%
\pgfpathlineto{\pgfqpoint{4.607670in}{1.747491in}}%
\pgfpathlineto{\pgfqpoint{4.599708in}{1.734982in}}%
\pgfpathlineto{\pgfqpoint{4.591742in}{1.722530in}}%
\pgfpathlineto{\pgfqpoint{4.583772in}{1.710139in}}%
\pgfpathlineto{\pgfqpoint{4.575798in}{1.697813in}}%
\pgfpathlineto{\pgfqpoint{4.561558in}{1.696760in}}%
\pgfpathlineto{\pgfqpoint{4.547328in}{1.695805in}}%
\pgfpathlineto{\pgfqpoint{4.533108in}{1.694946in}}%
\pgfpathlineto{\pgfqpoint{4.518899in}{1.694185in}}%
\pgfpathlineto{\pgfqpoint{4.526876in}{1.706105in}}%
\pgfpathlineto{\pgfqpoint{4.534850in}{1.718095in}}%
\pgfpathlineto{\pgfqpoint{4.542819in}{1.730150in}}%
\pgfpathlineto{\pgfqpoint{4.550783in}{1.742267in}}%
\pgfpathclose%
\pgfusepath{fill}%
\end{pgfscope}%
\begin{pgfscope}%
\pgfpathrectangle{\pgfqpoint{1.150000in}{0.150000in}}{\pgfqpoint{5.700000in}{5.700000in}}%
\pgfusepath{clip}%
\pgfsetbuttcap%
\pgfsetroundjoin%
\definecolor{currentfill}{rgb}{0.278791,0.062145,0.386592}%
\pgfsetfillcolor{currentfill}%
\pgfsetfillopacity{0.700000}%
\pgfsetlinewidth{0.000000pt}%
\definecolor{currentstroke}{rgb}{0.000000,0.000000,0.000000}%
\pgfsetstrokecolor{currentstroke}%
\pgfsetdash{}{0pt}%
\pgfpathmoveto{\pgfqpoint{4.373582in}{1.648010in}}%
\pgfpathlineto{\pgfqpoint{4.387719in}{1.647577in}}%
\pgfpathlineto{\pgfqpoint{4.401866in}{1.647242in}}%
\pgfpathlineto{\pgfqpoint{4.416022in}{1.647005in}}%
\pgfpathlineto{\pgfqpoint{4.430188in}{1.646865in}}%
\pgfpathlineto{\pgfqpoint{4.422183in}{1.635777in}}%
\pgfpathlineto{\pgfqpoint{4.414174in}{1.624788in}}%
\pgfpathlineto{\pgfqpoint{4.406160in}{1.613900in}}%
\pgfpathlineto{\pgfqpoint{4.398142in}{1.603119in}}%
\pgfpathlineto{\pgfqpoint{4.383970in}{1.603692in}}%
\pgfpathlineto{\pgfqpoint{4.369807in}{1.604363in}}%
\pgfpathlineto{\pgfqpoint{4.355653in}{1.605132in}}%
\pgfpathlineto{\pgfqpoint{4.341509in}{1.605997in}}%
\pgfpathlineto{\pgfqpoint{4.349534in}{1.616339in}}%
\pgfpathlineto{\pgfqpoint{4.357555in}{1.626791in}}%
\pgfpathlineto{\pgfqpoint{4.365571in}{1.637349in}}%
\pgfpathlineto{\pgfqpoint{4.373582in}{1.648010in}}%
\pgfpathclose%
\pgfusepath{fill}%
\end{pgfscope}%
\begin{pgfscope}%
\pgfpathrectangle{\pgfqpoint{1.150000in}{0.150000in}}{\pgfqpoint{5.700000in}{5.700000in}}%
\pgfusepath{clip}%
\pgfsetbuttcap%
\pgfsetroundjoin%
\definecolor{currentfill}{rgb}{0.271305,0.019942,0.347269}%
\pgfsetfillcolor{currentfill}%
\pgfsetfillopacity{0.700000}%
\pgfsetlinewidth{0.000000pt}%
\definecolor{currentstroke}{rgb}{0.000000,0.000000,0.000000}%
\pgfsetstrokecolor{currentstroke}%
\pgfsetdash{}{0pt}%
\pgfpathmoveto{\pgfqpoint{4.051597in}{1.569655in}}%
\pgfpathlineto{\pgfqpoint{4.065637in}{1.566297in}}%
\pgfpathlineto{\pgfqpoint{4.079684in}{1.563039in}}%
\pgfpathlineto{\pgfqpoint{4.093738in}{1.559880in}}%
\pgfpathlineto{\pgfqpoint{4.107800in}{1.556821in}}%
\pgfpathlineto{\pgfqpoint{4.099693in}{1.549113in}}%
\pgfpathlineto{\pgfqpoint{4.091579in}{1.541575in}}%
\pgfpathlineto{\pgfqpoint{4.083460in}{1.534211in}}%
\pgfpathlineto{\pgfqpoint{4.075333in}{1.527026in}}%
\pgfpathlineto{\pgfqpoint{4.061258in}{1.530572in}}%
\pgfpathlineto{\pgfqpoint{4.047189in}{1.534217in}}%
\pgfpathlineto{\pgfqpoint{4.033127in}{1.537962in}}%
\pgfpathlineto{\pgfqpoint{4.019071in}{1.541806in}}%
\pgfpathlineto{\pgfqpoint{4.027213in}{1.548497in}}%
\pgfpathlineto{\pgfqpoint{4.035348in}{1.555373in}}%
\pgfpathlineto{\pgfqpoint{4.043476in}{1.562427in}}%
\pgfpathlineto{\pgfqpoint{4.051597in}{1.569655in}}%
\pgfpathclose%
\pgfusepath{fill}%
\end{pgfscope}%
\begin{pgfscope}%
\pgfpathrectangle{\pgfqpoint{1.150000in}{0.150000in}}{\pgfqpoint{5.700000in}{5.700000in}}%
\pgfusepath{clip}%
\pgfsetbuttcap%
\pgfsetroundjoin%
\definecolor{currentfill}{rgb}{0.168126,0.459988,0.558082}%
\pgfsetfillcolor{currentfill}%
\pgfsetfillopacity{0.700000}%
\pgfsetlinewidth{0.000000pt}%
\definecolor{currentstroke}{rgb}{0.000000,0.000000,0.000000}%
\pgfsetstrokecolor{currentstroke}%
\pgfsetdash{}{0pt}%
\pgfpathmoveto{\pgfqpoint{2.541475in}{2.559122in}}%
\pgfpathlineto{\pgfqpoint{2.555558in}{2.541490in}}%
\pgfpathlineto{\pgfqpoint{2.569635in}{2.524017in}}%
\pgfpathlineto{\pgfqpoint{2.583707in}{2.506701in}}%
\pgfpathlineto{\pgfqpoint{2.597774in}{2.489540in}}%
\pgfpathlineto{\pgfqpoint{2.588571in}{2.500345in}}%
\pgfpathlineto{\pgfqpoint{2.579344in}{2.511560in}}%
\pgfpathlineto{\pgfqpoint{2.570091in}{2.523192in}}%
\pgfpathlineto{\pgfqpoint{2.560813in}{2.535247in}}%
\pgfpathlineto{\pgfqpoint{2.546687in}{2.553036in}}%
\pgfpathlineto{\pgfqpoint{2.532555in}{2.570982in}}%
\pgfpathlineto{\pgfqpoint{2.518417in}{2.589085in}}%
\pgfpathlineto{\pgfqpoint{2.504274in}{2.607349in}}%
\pgfpathlineto{\pgfqpoint{2.513613in}{2.594654in}}%
\pgfpathlineto{\pgfqpoint{2.522927in}{2.582389in}}%
\pgfpathlineto{\pgfqpoint{2.532214in}{2.570548in}}%
\pgfpathlineto{\pgfqpoint{2.541475in}{2.559122in}}%
\pgfpathclose%
\pgfusepath{fill}%
\end{pgfscope}%
\begin{pgfscope}%
\pgfpathrectangle{\pgfqpoint{1.150000in}{0.150000in}}{\pgfqpoint{5.700000in}{5.700000in}}%
\pgfusepath{clip}%
\pgfsetbuttcap%
\pgfsetroundjoin%
\definecolor{currentfill}{rgb}{0.131172,0.555899,0.552459}%
\pgfsetfillcolor{currentfill}%
\pgfsetfillopacity{0.700000}%
\pgfsetlinewidth{0.000000pt}%
\definecolor{currentstroke}{rgb}{0.000000,0.000000,0.000000}%
\pgfsetstrokecolor{currentstroke}%
\pgfsetdash{}{0pt}%
\pgfpathmoveto{\pgfqpoint{5.687595in}{2.827087in}}%
\pgfpathlineto{\pgfqpoint{5.702423in}{2.836546in}}%
\pgfpathlineto{\pgfqpoint{5.717269in}{2.846108in}}%
\pgfpathlineto{\pgfqpoint{5.732133in}{2.855771in}}%
\pgfpathlineto{\pgfqpoint{5.724527in}{2.844212in}}%
\pgfpathlineto{\pgfqpoint{5.716912in}{2.832516in}}%
\pgfpathlineto{\pgfqpoint{5.709289in}{2.820684in}}%
\pgfpathlineto{\pgfqpoint{5.701657in}{2.808716in}}%
\pgfpathlineto{\pgfqpoint{5.686793in}{2.799157in}}%
\pgfpathlineto{\pgfqpoint{5.671946in}{2.789701in}}%
\pgfpathlineto{\pgfqpoint{5.657117in}{2.780347in}}%
\pgfpathlineto{\pgfqpoint{5.664749in}{2.792230in}}%
\pgfpathlineto{\pgfqpoint{5.672373in}{2.803982in}}%
\pgfpathlineto{\pgfqpoint{5.679988in}{2.815601in}}%
\pgfpathlineto{\pgfqpoint{5.687595in}{2.827087in}}%
\pgfpathclose%
\pgfusepath{fill}%
\end{pgfscope}%
\begin{pgfscope}%
\pgfpathrectangle{\pgfqpoint{1.150000in}{0.150000in}}{\pgfqpoint{5.700000in}{5.700000in}}%
\pgfusepath{clip}%
\pgfsetbuttcap%
\pgfsetroundjoin%
\definecolor{currentfill}{rgb}{0.221989,0.339161,0.548752}%
\pgfsetfillcolor{currentfill}%
\pgfsetfillopacity{0.700000}%
\pgfsetlinewidth{0.000000pt}%
\definecolor{currentstroke}{rgb}{0.000000,0.000000,0.000000}%
\pgfsetstrokecolor{currentstroke}%
\pgfsetdash{}{0pt}%
\pgfpathmoveto{\pgfqpoint{2.822296in}{2.235268in}}%
\pgfpathlineto{\pgfqpoint{2.836301in}{2.220591in}}%
\pgfpathlineto{\pgfqpoint{2.850304in}{2.206052in}}%
\pgfpathlineto{\pgfqpoint{2.864305in}{2.191649in}}%
\pgfpathlineto{\pgfqpoint{2.878303in}{2.177381in}}%
\pgfpathlineto{\pgfqpoint{2.869370in}{2.185138in}}%
\pgfpathlineto{\pgfqpoint{2.860416in}{2.193275in}}%
\pgfpathlineto{\pgfqpoint{2.851442in}{2.201796in}}%
\pgfpathlineto{\pgfqpoint{2.842445in}{2.210711in}}%
\pgfpathlineto{\pgfqpoint{2.828396in}{2.225589in}}%
\pgfpathlineto{\pgfqpoint{2.814343in}{2.240604in}}%
\pgfpathlineto{\pgfqpoint{2.800288in}{2.255756in}}%
\pgfpathlineto{\pgfqpoint{2.786231in}{2.271046in}}%
\pgfpathlineto{\pgfqpoint{2.795280in}{2.261511in}}%
\pgfpathlineto{\pgfqpoint{2.804307in}{2.252374in}}%
\pgfpathlineto{\pgfqpoint{2.813312in}{2.243628in}}%
\pgfpathlineto{\pgfqpoint{2.822296in}{2.235268in}}%
\pgfpathclose%
\pgfusepath{fill}%
\end{pgfscope}%
\begin{pgfscope}%
\pgfpathrectangle{\pgfqpoint{1.150000in}{0.150000in}}{\pgfqpoint{5.700000in}{5.700000in}}%
\pgfusepath{clip}%
\pgfsetbuttcap%
\pgfsetroundjoin%
\definecolor{currentfill}{rgb}{0.160665,0.478540,0.558115}%
\pgfsetfillcolor{currentfill}%
\pgfsetfillopacity{0.700000}%
\pgfsetlinewidth{0.000000pt}%
\definecolor{currentstroke}{rgb}{0.000000,0.000000,0.000000}%
\pgfsetstrokecolor{currentstroke}%
\pgfsetdash{}{0pt}%
\pgfpathmoveto{\pgfqpoint{5.477739in}{2.611619in}}%
\pgfpathlineto{\pgfqpoint{5.492435in}{2.619893in}}%
\pgfpathlineto{\pgfqpoint{5.507147in}{2.628268in}}%
\pgfpathlineto{\pgfqpoint{5.521876in}{2.636744in}}%
\pgfpathlineto{\pgfqpoint{5.536620in}{2.645321in}}%
\pgfpathlineto{\pgfqpoint{5.528917in}{2.632451in}}%
\pgfpathlineto{\pgfqpoint{5.521207in}{2.619465in}}%
\pgfpathlineto{\pgfqpoint{5.513489in}{2.606363in}}%
\pgfpathlineto{\pgfqpoint{5.505764in}{2.593148in}}%
\pgfpathlineto{\pgfqpoint{5.491021in}{2.584736in}}%
\pgfpathlineto{\pgfqpoint{5.476294in}{2.576424in}}%
\pgfpathlineto{\pgfqpoint{5.461583in}{2.568213in}}%
\pgfpathlineto{\pgfqpoint{5.446889in}{2.560102in}}%
\pgfpathlineto{\pgfqpoint{5.454612in}{2.573146in}}%
\pgfpathlineto{\pgfqpoint{5.462328in}{2.586081in}}%
\pgfpathlineto{\pgfqpoint{5.470037in}{2.598905in}}%
\pgfpathlineto{\pgfqpoint{5.477739in}{2.611619in}}%
\pgfpathclose%
\pgfusepath{fill}%
\end{pgfscope}%
\begin{pgfscope}%
\pgfpathrectangle{\pgfqpoint{1.150000in}{0.150000in}}{\pgfqpoint{5.700000in}{5.700000in}}%
\pgfusepath{clip}%
\pgfsetbuttcap%
\pgfsetroundjoin%
\definecolor{currentfill}{rgb}{0.255645,0.260703,0.528312}%
\pgfsetfillcolor{currentfill}%
\pgfsetfillopacity{0.700000}%
\pgfsetlinewidth{0.000000pt}%
\definecolor{currentstroke}{rgb}{0.000000,0.000000,0.000000}%
\pgfsetstrokecolor{currentstroke}%
\pgfsetdash{}{0pt}%
\pgfpathmoveto{\pgfqpoint{4.937718in}{2.050097in}}%
\pgfpathlineto{\pgfqpoint{4.952108in}{2.054549in}}%
\pgfpathlineto{\pgfqpoint{4.966511in}{2.059100in}}%
\pgfpathlineto{\pgfqpoint{4.980927in}{2.063748in}}%
\pgfpathlineto{\pgfqpoint{4.995356in}{2.068495in}}%
\pgfpathlineto{\pgfqpoint{4.987478in}{2.054343in}}%
\pgfpathlineto{\pgfqpoint{4.979595in}{2.040165in}}%
\pgfpathlineto{\pgfqpoint{4.971707in}{2.025963in}}%
\pgfpathlineto{\pgfqpoint{4.963816in}{2.011741in}}%
\pgfpathlineto{\pgfqpoint{4.949389in}{2.007307in}}%
\pgfpathlineto{\pgfqpoint{4.934975in}{2.002970in}}%
\pgfpathlineto{\pgfqpoint{4.920573in}{1.998732in}}%
\pgfpathlineto{\pgfqpoint{4.906185in}{1.994591in}}%
\pgfpathlineto{\pgfqpoint{4.914075in}{2.008495in}}%
\pgfpathlineto{\pgfqpoint{4.921960in}{2.022382in}}%
\pgfpathlineto{\pgfqpoint{4.929841in}{2.036251in}}%
\pgfpathlineto{\pgfqpoint{4.937718in}{2.050097in}}%
\pgfpathclose%
\pgfusepath{fill}%
\end{pgfscope}%
\begin{pgfscope}%
\pgfpathrectangle{\pgfqpoint{1.150000in}{0.150000in}}{\pgfqpoint{5.700000in}{5.700000in}}%
\pgfusepath{clip}%
\pgfsetbuttcap%
\pgfsetroundjoin%
\definecolor{currentfill}{rgb}{0.282623,0.140926,0.457517}%
\pgfsetfillcolor{currentfill}%
\pgfsetfillopacity{0.700000}%
\pgfsetlinewidth{0.000000pt}%
\definecolor{currentstroke}{rgb}{0.000000,0.000000,0.000000}%
\pgfsetstrokecolor{currentstroke}%
\pgfsetdash{}{0pt}%
\pgfpathmoveto{\pgfqpoint{4.639476in}{1.798027in}}%
\pgfpathlineto{\pgfqpoint{4.653722in}{1.799957in}}%
\pgfpathlineto{\pgfqpoint{4.667979in}{1.801985in}}%
\pgfpathlineto{\pgfqpoint{4.682247in}{1.804109in}}%
\pgfpathlineto{\pgfqpoint{4.696527in}{1.806331in}}%
\pgfpathlineto{\pgfqpoint{4.688583in}{1.793255in}}%
\pgfpathlineto{\pgfqpoint{4.680636in}{1.780216in}}%
\pgfpathlineto{\pgfqpoint{4.672684in}{1.767220in}}%
\pgfpathlineto{\pgfqpoint{4.664728in}{1.754268in}}%
\pgfpathlineto{\pgfqpoint{4.650447in}{1.752429in}}%
\pgfpathlineto{\pgfqpoint{4.636177in}{1.750686in}}%
\pgfpathlineto{\pgfqpoint{4.621918in}{1.749040in}}%
\pgfpathlineto{\pgfqpoint{4.607670in}{1.747491in}}%
\pgfpathlineto{\pgfqpoint{4.615628in}{1.760054in}}%
\pgfpathlineto{\pgfqpoint{4.623581in}{1.772666in}}%
\pgfpathlineto{\pgfqpoint{4.631531in}{1.785325in}}%
\pgfpathlineto{\pgfqpoint{4.639476in}{1.798027in}}%
\pgfpathclose%
\pgfusepath{fill}%
\end{pgfscope}%
\begin{pgfscope}%
\pgfpathrectangle{\pgfqpoint{1.150000in}{0.150000in}}{\pgfqpoint{5.700000in}{5.700000in}}%
\pgfusepath{clip}%
\pgfsetbuttcap%
\pgfsetroundjoin%
\definecolor{currentfill}{rgb}{0.157729,0.485932,0.558013}%
\pgfsetfillcolor{currentfill}%
\pgfsetfillopacity{0.700000}%
\pgfsetlinewidth{0.000000pt}%
\definecolor{currentstroke}{rgb}{0.000000,0.000000,0.000000}%
\pgfsetstrokecolor{currentstroke}%
\pgfsetdash{}{0pt}%
\pgfpathmoveto{\pgfqpoint{2.485090in}{2.631259in}}%
\pgfpathlineto{\pgfqpoint{2.499195in}{2.612981in}}%
\pgfpathlineto{\pgfqpoint{2.513294in}{2.594866in}}%
\pgfpathlineto{\pgfqpoint{2.527387in}{2.576914in}}%
\pgfpathlineto{\pgfqpoint{2.541475in}{2.559122in}}%
\pgfpathlineto{\pgfqpoint{2.532214in}{2.570548in}}%
\pgfpathlineto{\pgfqpoint{2.522927in}{2.582389in}}%
\pgfpathlineto{\pgfqpoint{2.513613in}{2.594654in}}%
\pgfpathlineto{\pgfqpoint{2.504274in}{2.607349in}}%
\pgfpathlineto{\pgfqpoint{2.490125in}{2.625773in}}%
\pgfpathlineto{\pgfqpoint{2.475970in}{2.644359in}}%
\pgfpathlineto{\pgfqpoint{2.461808in}{2.663109in}}%
\pgfpathlineto{\pgfqpoint{2.447641in}{2.682023in}}%
\pgfpathlineto{\pgfqpoint{2.457044in}{2.668684in}}%
\pgfpathlineto{\pgfqpoint{2.466419in}{2.655782in}}%
\pgfpathlineto{\pgfqpoint{2.475768in}{2.643309in}}%
\pgfpathlineto{\pgfqpoint{2.485090in}{2.631259in}}%
\pgfpathclose%
\pgfusepath{fill}%
\end{pgfscope}%
\begin{pgfscope}%
\pgfpathrectangle{\pgfqpoint{1.150000in}{0.150000in}}{\pgfqpoint{5.700000in}{5.700000in}}%
\pgfusepath{clip}%
\pgfsetbuttcap%
\pgfsetroundjoin%
\definecolor{currentfill}{rgb}{0.276022,0.044167,0.370164}%
\pgfsetfillcolor{currentfill}%
\pgfsetfillopacity{0.700000}%
\pgfsetlinewidth{0.000000pt}%
\definecolor{currentstroke}{rgb}{0.000000,0.000000,0.000000}%
\pgfsetstrokecolor{currentstroke}%
\pgfsetdash{}{0pt}%
\pgfpathmoveto{\pgfqpoint{4.285017in}{1.610435in}}%
\pgfpathlineto{\pgfqpoint{4.299127in}{1.609179in}}%
\pgfpathlineto{\pgfqpoint{4.313246in}{1.608021in}}%
\pgfpathlineto{\pgfqpoint{4.327373in}{1.606960in}}%
\pgfpathlineto{\pgfqpoint{4.341509in}{1.605997in}}%
\pgfpathlineto{\pgfqpoint{4.333478in}{1.595771in}}%
\pgfpathlineto{\pgfqpoint{4.325443in}{1.585664in}}%
\pgfpathlineto{\pgfqpoint{4.317403in}{1.575680in}}%
\pgfpathlineto{\pgfqpoint{4.309357in}{1.565824in}}%
\pgfpathlineto{\pgfqpoint{4.295213in}{1.567238in}}%
\pgfpathlineto{\pgfqpoint{4.281077in}{1.568750in}}%
\pgfpathlineto{\pgfqpoint{4.266950in}{1.570359in}}%
\pgfpathlineto{\pgfqpoint{4.252830in}{1.572065in}}%
\pgfpathlineto{\pgfqpoint{4.260885in}{1.581463in}}%
\pgfpathlineto{\pgfqpoint{4.268934in}{1.590994in}}%
\pgfpathlineto{\pgfqpoint{4.276978in}{1.600653in}}%
\pgfpathlineto{\pgfqpoint{4.285017in}{1.610435in}}%
\pgfpathclose%
\pgfusepath{fill}%
\end{pgfscope}%
\begin{pgfscope}%
\pgfpathrectangle{\pgfqpoint{1.150000in}{0.150000in}}{\pgfqpoint{5.700000in}{5.700000in}}%
\pgfusepath{clip}%
\pgfsetbuttcap%
\pgfsetroundjoin%
\definecolor{currentfill}{rgb}{0.231674,0.318106,0.544834}%
\pgfsetfillcolor{currentfill}%
\pgfsetfillopacity{0.700000}%
\pgfsetlinewidth{0.000000pt}%
\definecolor{currentstroke}{rgb}{0.000000,0.000000,0.000000}%
\pgfsetstrokecolor{currentstroke}%
\pgfsetdash{}{0pt}%
\pgfpathmoveto{\pgfqpoint{2.878303in}{2.177381in}}%
\pgfpathlineto{\pgfqpoint{2.892300in}{2.163248in}}%
\pgfpathlineto{\pgfqpoint{2.906294in}{2.149249in}}%
\pgfpathlineto{\pgfqpoint{2.920287in}{2.135383in}}%
\pgfpathlineto{\pgfqpoint{2.934278in}{2.121649in}}%
\pgfpathlineto{\pgfqpoint{2.925394in}{2.128805in}}%
\pgfpathlineto{\pgfqpoint{2.916490in}{2.136335in}}%
\pgfpathlineto{\pgfqpoint{2.907566in}{2.144245in}}%
\pgfpathlineto{\pgfqpoint{2.898621in}{2.152541in}}%
\pgfpathlineto{\pgfqpoint{2.884580in}{2.166884in}}%
\pgfpathlineto{\pgfqpoint{2.870537in}{2.181359in}}%
\pgfpathlineto{\pgfqpoint{2.856493in}{2.195968in}}%
\pgfpathlineto{\pgfqpoint{2.842445in}{2.210711in}}%
\pgfpathlineto{\pgfqpoint{2.851442in}{2.201796in}}%
\pgfpathlineto{\pgfqpoint{2.860416in}{2.193275in}}%
\pgfpathlineto{\pgfqpoint{2.869370in}{2.185138in}}%
\pgfpathlineto{\pgfqpoint{2.878303in}{2.177381in}}%
\pgfpathclose%
\pgfusepath{fill}%
\end{pgfscope}%
\begin{pgfscope}%
\pgfpathrectangle{\pgfqpoint{1.150000in}{0.150000in}}{\pgfqpoint{5.700000in}{5.700000in}}%
\pgfusepath{clip}%
\pgfsetbuttcap%
\pgfsetroundjoin%
\definecolor{currentfill}{rgb}{0.218130,0.347432,0.550038}%
\pgfsetfillcolor{currentfill}%
\pgfsetfillopacity{0.700000}%
\pgfsetlinewidth{0.000000pt}%
\definecolor{currentstroke}{rgb}{0.000000,0.000000,0.000000}%
\pgfsetstrokecolor{currentstroke}%
\pgfsetdash{}{0pt}%
\pgfpathmoveto{\pgfqpoint{5.147400in}{2.258561in}}%
\pgfpathlineto{\pgfqpoint{5.161905in}{2.264628in}}%
\pgfpathlineto{\pgfqpoint{5.176424in}{2.270794in}}%
\pgfpathlineto{\pgfqpoint{5.190957in}{2.277059in}}%
\pgfpathlineto{\pgfqpoint{5.205505in}{2.283424in}}%
\pgfpathlineto{\pgfqpoint{5.197681in}{2.269293in}}%
\pgfpathlineto{\pgfqpoint{5.189852in}{2.255095in}}%
\pgfpathlineto{\pgfqpoint{5.182017in}{2.240834in}}%
\pgfpathlineto{\pgfqpoint{5.174177in}{2.226511in}}%
\pgfpathlineto{\pgfqpoint{5.159632in}{2.220405in}}%
\pgfpathlineto{\pgfqpoint{5.145101in}{2.214398in}}%
\pgfpathlineto{\pgfqpoint{5.130585in}{2.208489in}}%
\pgfpathlineto{\pgfqpoint{5.116083in}{2.202680in}}%
\pgfpathlineto{\pgfqpoint{5.123920in}{2.216738in}}%
\pgfpathlineto{\pgfqpoint{5.131752in}{2.230739in}}%
\pgfpathlineto{\pgfqpoint{5.139578in}{2.244681in}}%
\pgfpathlineto{\pgfqpoint{5.147400in}{2.258561in}}%
\pgfpathclose%
\pgfusepath{fill}%
\end{pgfscope}%
\begin{pgfscope}%
\pgfpathrectangle{\pgfqpoint{1.150000in}{0.150000in}}{\pgfqpoint{5.700000in}{5.700000in}}%
\pgfusepath{clip}%
\pgfsetbuttcap%
\pgfsetroundjoin%
\definecolor{currentfill}{rgb}{0.282884,0.135920,0.453427}%
\pgfsetfillcolor{currentfill}%
\pgfsetfillopacity{0.700000}%
\pgfsetlinewidth{0.000000pt}%
\definecolor{currentstroke}{rgb}{0.000000,0.000000,0.000000}%
\pgfsetstrokecolor{currentstroke}%
\pgfsetdash{}{0pt}%
\pgfpathmoveto{\pgfqpoint{3.360015in}{1.778953in}}%
\pgfpathlineto{\pgfqpoint{3.373972in}{1.769321in}}%
\pgfpathlineto{\pgfqpoint{3.387931in}{1.759803in}}%
\pgfpathlineto{\pgfqpoint{3.401892in}{1.750396in}}%
\pgfpathlineto{\pgfqpoint{3.415855in}{1.741102in}}%
\pgfpathlineto{\pgfqpoint{3.407364in}{1.742287in}}%
\pgfpathlineto{\pgfqpoint{3.398860in}{1.743776in}}%
\pgfpathlineto{\pgfqpoint{3.390342in}{1.745574in}}%
\pgfpathlineto{\pgfqpoint{3.381811in}{1.747689in}}%
\pgfpathlineto{\pgfqpoint{3.367813in}{1.757553in}}%
\pgfpathlineto{\pgfqpoint{3.353816in}{1.767529in}}%
\pgfpathlineto{\pgfqpoint{3.339822in}{1.777618in}}%
\pgfpathlineto{\pgfqpoint{3.325829in}{1.787820in}}%
\pgfpathlineto{\pgfqpoint{3.334397in}{1.785128in}}%
\pgfpathlineto{\pgfqpoint{3.342951in}{1.782756in}}%
\pgfpathlineto{\pgfqpoint{3.351490in}{1.780700in}}%
\pgfpathlineto{\pgfqpoint{3.360015in}{1.778953in}}%
\pgfpathclose%
\pgfusepath{fill}%
\end{pgfscope}%
\begin{pgfscope}%
\pgfpathrectangle{\pgfqpoint{1.150000in}{0.150000in}}{\pgfqpoint{5.700000in}{5.700000in}}%
\pgfusepath{clip}%
\pgfsetbuttcap%
\pgfsetroundjoin%
\definecolor{currentfill}{rgb}{0.241237,0.296485,0.539709}%
\pgfsetfillcolor{currentfill}%
\pgfsetfillopacity{0.700000}%
\pgfsetlinewidth{0.000000pt}%
\definecolor{currentstroke}{rgb}{0.000000,0.000000,0.000000}%
\pgfsetstrokecolor{currentstroke}%
\pgfsetdash{}{0pt}%
\pgfpathmoveto{\pgfqpoint{2.934278in}{2.121649in}}%
\pgfpathlineto{\pgfqpoint{2.948268in}{2.108046in}}%
\pgfpathlineto{\pgfqpoint{2.962256in}{2.094573in}}%
\pgfpathlineto{\pgfqpoint{2.976243in}{2.081231in}}%
\pgfpathlineto{\pgfqpoint{2.990228in}{2.068017in}}%
\pgfpathlineto{\pgfqpoint{2.981392in}{2.074577in}}%
\pgfpathlineto{\pgfqpoint{2.972536in}{2.081503in}}%
\pgfpathlineto{\pgfqpoint{2.963661in}{2.088803in}}%
\pgfpathlineto{\pgfqpoint{2.954766in}{2.096484in}}%
\pgfpathlineto{\pgfqpoint{2.940732in}{2.110303in}}%
\pgfpathlineto{\pgfqpoint{2.926697in}{2.124252in}}%
\pgfpathlineto{\pgfqpoint{2.912660in}{2.138331in}}%
\pgfpathlineto{\pgfqpoint{2.898621in}{2.152541in}}%
\pgfpathlineto{\pgfqpoint{2.907566in}{2.144245in}}%
\pgfpathlineto{\pgfqpoint{2.916490in}{2.136335in}}%
\pgfpathlineto{\pgfqpoint{2.925394in}{2.128805in}}%
\pgfpathlineto{\pgfqpoint{2.934278in}{2.121649in}}%
\pgfpathclose%
\pgfusepath{fill}%
\end{pgfscope}%
\begin{pgfscope}%
\pgfpathrectangle{\pgfqpoint{1.150000in}{0.150000in}}{\pgfqpoint{5.700000in}{5.700000in}}%
\pgfusepath{clip}%
\pgfsetbuttcap%
\pgfsetroundjoin%
\definecolor{currentfill}{rgb}{0.146180,0.515413,0.556823}%
\pgfsetfillcolor{currentfill}%
\pgfsetfillopacity{0.700000}%
\pgfsetlinewidth{0.000000pt}%
\definecolor{currentstroke}{rgb}{0.000000,0.000000,0.000000}%
\pgfsetstrokecolor{currentstroke}%
\pgfsetdash{}{0pt}%
\pgfpathmoveto{\pgfqpoint{2.428608in}{2.706034in}}%
\pgfpathlineto{\pgfqpoint{2.442738in}{2.687088in}}%
\pgfpathlineto{\pgfqpoint{2.456862in}{2.668311in}}%
\pgfpathlineto{\pgfqpoint{2.470979in}{2.649702in}}%
\pgfpathlineto{\pgfqpoint{2.485090in}{2.631259in}}%
\pgfpathlineto{\pgfqpoint{2.475768in}{2.643309in}}%
\pgfpathlineto{\pgfqpoint{2.466419in}{2.655782in}}%
\pgfpathlineto{\pgfqpoint{2.457044in}{2.668684in}}%
\pgfpathlineto{\pgfqpoint{2.447641in}{2.682023in}}%
\pgfpathlineto{\pgfqpoint{2.433467in}{2.701103in}}%
\pgfpathlineto{\pgfqpoint{2.419286in}{2.720350in}}%
\pgfpathlineto{\pgfqpoint{2.405098in}{2.739767in}}%
\pgfpathlineto{\pgfqpoint{2.390904in}{2.759353in}}%
\pgfpathlineto{\pgfqpoint{2.400372in}{2.745366in}}%
\pgfpathlineto{\pgfqpoint{2.409811in}{2.731821in}}%
\pgfpathlineto{\pgfqpoint{2.419223in}{2.718713in}}%
\pgfpathlineto{\pgfqpoint{2.428608in}{2.706034in}}%
\pgfpathclose%
\pgfusepath{fill}%
\end{pgfscope}%
\begin{pgfscope}%
\pgfpathrectangle{\pgfqpoint{1.150000in}{0.150000in}}{\pgfqpoint{5.700000in}{5.700000in}}%
\pgfusepath{clip}%
\pgfsetbuttcap%
\pgfsetroundjoin%
\definecolor{currentfill}{rgb}{0.278826,0.175490,0.483397}%
\pgfsetfillcolor{currentfill}%
\pgfsetfillopacity{0.700000}%
\pgfsetlinewidth{0.000000pt}%
\definecolor{currentstroke}{rgb}{0.000000,0.000000,0.000000}%
\pgfsetstrokecolor{currentstroke}%
\pgfsetdash{}{0pt}%
\pgfpathmoveto{\pgfqpoint{4.728261in}{1.858950in}}%
\pgfpathlineto{\pgfqpoint{4.742551in}{1.861633in}}%
\pgfpathlineto{\pgfqpoint{4.756852in}{1.864414in}}%
\pgfpathlineto{\pgfqpoint{4.771166in}{1.867292in}}%
\pgfpathlineto{\pgfqpoint{4.785491in}{1.870268in}}%
\pgfpathlineto{\pgfqpoint{4.777564in}{1.856716in}}%
\pgfpathlineto{\pgfqpoint{4.769633in}{1.843183in}}%
\pgfpathlineto{\pgfqpoint{4.761698in}{1.829674in}}%
\pgfpathlineto{\pgfqpoint{4.753759in}{1.816191in}}%
\pgfpathlineto{\pgfqpoint{4.739433in}{1.813580in}}%
\pgfpathlineto{\pgfqpoint{4.725120in}{1.811067in}}%
\pgfpathlineto{\pgfqpoint{4.710818in}{1.808651in}}%
\pgfpathlineto{\pgfqpoint{4.696527in}{1.806331in}}%
\pgfpathlineto{\pgfqpoint{4.704466in}{1.819443in}}%
\pgfpathlineto{\pgfqpoint{4.712402in}{1.832585in}}%
\pgfpathlineto{\pgfqpoint{4.720333in}{1.845755in}}%
\pgfpathlineto{\pgfqpoint{4.728261in}{1.858950in}}%
\pgfpathclose%
\pgfusepath{fill}%
\end{pgfscope}%
\begin{pgfscope}%
\pgfpathrectangle{\pgfqpoint{1.150000in}{0.150000in}}{\pgfqpoint{5.700000in}{5.700000in}}%
\pgfusepath{clip}%
\pgfsetbuttcap%
\pgfsetroundjoin%
\definecolor{currentfill}{rgb}{0.279566,0.067836,0.391917}%
\pgfsetfillcolor{currentfill}%
\pgfsetfillopacity{0.700000}%
\pgfsetlinewidth{0.000000pt}%
\definecolor{currentstroke}{rgb}{0.000000,0.000000,0.000000}%
\pgfsetstrokecolor{currentstroke}%
\pgfsetdash{}{0pt}%
\pgfpathmoveto{\pgfqpoint{3.617086in}{1.642876in}}%
\pgfpathlineto{\pgfqpoint{3.631058in}{1.635543in}}%
\pgfpathlineto{\pgfqpoint{3.645034in}{1.628315in}}%
\pgfpathlineto{\pgfqpoint{3.659014in}{1.621193in}}%
\pgfpathlineto{\pgfqpoint{3.672999in}{1.614176in}}%
\pgfpathlineto{\pgfqpoint{3.664676in}{1.612049in}}%
\pgfpathlineto{\pgfqpoint{3.656342in}{1.610182in}}%
\pgfpathlineto{\pgfqpoint{3.647998in}{1.608581in}}%
\pgfpathlineto{\pgfqpoint{3.639643in}{1.607251in}}%
\pgfpathlineto{\pgfqpoint{3.625631in}{1.614813in}}%
\pgfpathlineto{\pgfqpoint{3.611623in}{1.622480in}}%
\pgfpathlineto{\pgfqpoint{3.597619in}{1.630253in}}%
\pgfpathlineto{\pgfqpoint{3.583619in}{1.638132in}}%
\pgfpathlineto{\pgfqpoint{3.592002in}{1.638909in}}%
\pgfpathlineto{\pgfqpoint{3.600375in}{1.639963in}}%
\pgfpathlineto{\pgfqpoint{3.608736in}{1.641287in}}%
\pgfpathlineto{\pgfqpoint{3.617086in}{1.642876in}}%
\pgfpathclose%
\pgfusepath{fill}%
\end{pgfscope}%
\begin{pgfscope}%
\pgfpathrectangle{\pgfqpoint{1.150000in}{0.150000in}}{\pgfqpoint{5.700000in}{5.700000in}}%
\pgfusepath{clip}%
\pgfsetbuttcap%
\pgfsetroundjoin%
\definecolor{currentfill}{rgb}{0.179019,0.433756,0.557430}%
\pgfsetfillcolor{currentfill}%
\pgfsetfillopacity{0.700000}%
\pgfsetlinewidth{0.000000pt}%
\definecolor{currentstroke}{rgb}{0.000000,0.000000,0.000000}%
\pgfsetstrokecolor{currentstroke}%
\pgfsetdash{}{0pt}%
\pgfpathmoveto{\pgfqpoint{5.357312in}{2.476166in}}%
\pgfpathlineto{\pgfqpoint{5.371943in}{2.483691in}}%
\pgfpathlineto{\pgfqpoint{5.386588in}{2.491317in}}%
\pgfpathlineto{\pgfqpoint{5.401250in}{2.499043in}}%
\pgfpathlineto{\pgfqpoint{5.415927in}{2.506869in}}%
\pgfpathlineto{\pgfqpoint{5.408169in}{2.493304in}}%
\pgfpathlineto{\pgfqpoint{5.400405in}{2.479640in}}%
\pgfpathlineto{\pgfqpoint{5.392635in}{2.465877in}}%
\pgfpathlineto{\pgfqpoint{5.384858in}{2.452018in}}%
\pgfpathlineto{\pgfqpoint{5.370184in}{2.444395in}}%
\pgfpathlineto{\pgfqpoint{5.355525in}{2.436871in}}%
\pgfpathlineto{\pgfqpoint{5.340882in}{2.429448in}}%
\pgfpathlineto{\pgfqpoint{5.326254in}{2.422124in}}%
\pgfpathlineto{\pgfqpoint{5.334028in}{2.435773in}}%
\pgfpathlineto{\pgfqpoint{5.341796in}{2.449331in}}%
\pgfpathlineto{\pgfqpoint{5.349557in}{2.462796in}}%
\pgfpathlineto{\pgfqpoint{5.357312in}{2.476166in}}%
\pgfpathclose%
\pgfusepath{fill}%
\end{pgfscope}%
\begin{pgfscope}%
\pgfpathrectangle{\pgfqpoint{1.150000in}{0.150000in}}{\pgfqpoint{5.700000in}{5.700000in}}%
\pgfusepath{clip}%
\pgfsetbuttcap%
\pgfsetroundjoin%
\definecolor{currentfill}{rgb}{0.273809,0.031497,0.358853}%
\pgfsetfillcolor{currentfill}%
\pgfsetfillopacity{0.700000}%
\pgfsetlinewidth{0.000000pt}%
\definecolor{currentstroke}{rgb}{0.000000,0.000000,0.000000}%
\pgfsetstrokecolor{currentstroke}%
\pgfsetdash{}{0pt}%
\pgfpathmoveto{\pgfqpoint{4.196435in}{1.579871in}}%
\pgfpathlineto{\pgfqpoint{4.210522in}{1.577772in}}%
\pgfpathlineto{\pgfqpoint{4.224617in}{1.575772in}}%
\pgfpathlineto{\pgfqpoint{4.238719in}{1.573870in}}%
\pgfpathlineto{\pgfqpoint{4.252830in}{1.572065in}}%
\pgfpathlineto{\pgfqpoint{4.244770in}{1.562804in}}%
\pgfpathlineto{\pgfqpoint{4.236705in}{1.553685in}}%
\pgfpathlineto{\pgfqpoint{4.228634in}{1.544712in}}%
\pgfpathlineto{\pgfqpoint{4.220558in}{1.535889in}}%
\pgfpathlineto{\pgfqpoint{4.206436in}{1.538162in}}%
\pgfpathlineto{\pgfqpoint{4.192323in}{1.540533in}}%
\pgfpathlineto{\pgfqpoint{4.178216in}{1.543002in}}%
\pgfpathlineto{\pgfqpoint{4.164118in}{1.545569in}}%
\pgfpathlineto{\pgfqpoint{4.172206in}{1.553916in}}%
\pgfpathlineto{\pgfqpoint{4.180288in}{1.562418in}}%
\pgfpathlineto{\pgfqpoint{4.188364in}{1.571072in}}%
\pgfpathlineto{\pgfqpoint{4.196435in}{1.579871in}}%
\pgfpathclose%
\pgfusepath{fill}%
\end{pgfscope}%
\begin{pgfscope}%
\pgfpathrectangle{\pgfqpoint{1.150000in}{0.150000in}}{\pgfqpoint{5.700000in}{5.700000in}}%
\pgfusepath{clip}%
\pgfsetbuttcap%
\pgfsetroundjoin%
\definecolor{currentfill}{rgb}{0.250425,0.274290,0.533103}%
\pgfsetfillcolor{currentfill}%
\pgfsetfillopacity{0.700000}%
\pgfsetlinewidth{0.000000pt}%
\definecolor{currentstroke}{rgb}{0.000000,0.000000,0.000000}%
\pgfsetstrokecolor{currentstroke}%
\pgfsetdash{}{0pt}%
\pgfpathmoveto{\pgfqpoint{2.990228in}{2.068017in}}%
\pgfpathlineto{\pgfqpoint{3.004213in}{2.054932in}}%
\pgfpathlineto{\pgfqpoint{3.018197in}{2.041974in}}%
\pgfpathlineto{\pgfqpoint{3.032180in}{2.029143in}}%
\pgfpathlineto{\pgfqpoint{3.046162in}{2.016438in}}%
\pgfpathlineto{\pgfqpoint{3.037372in}{2.022403in}}%
\pgfpathlineto{\pgfqpoint{3.028563in}{2.028728in}}%
\pgfpathlineto{\pgfqpoint{3.019735in}{2.035422in}}%
\pgfpathlineto{\pgfqpoint{3.010888in}{2.042490in}}%
\pgfpathlineto{\pgfqpoint{2.996859in}{2.055798in}}%
\pgfpathlineto{\pgfqpoint{2.982829in}{2.069232in}}%
\pgfpathlineto{\pgfqpoint{2.968798in}{2.082794in}}%
\pgfpathlineto{\pgfqpoint{2.954766in}{2.096484in}}%
\pgfpathlineto{\pgfqpoint{2.963661in}{2.088803in}}%
\pgfpathlineto{\pgfqpoint{2.972536in}{2.081503in}}%
\pgfpathlineto{\pgfqpoint{2.981392in}{2.074577in}}%
\pgfpathlineto{\pgfqpoint{2.990228in}{2.068017in}}%
\pgfpathclose%
\pgfusepath{fill}%
\end{pgfscope}%
\begin{pgfscope}%
\pgfpathrectangle{\pgfqpoint{1.150000in}{0.150000in}}{\pgfqpoint{5.700000in}{5.700000in}}%
\pgfusepath{clip}%
\pgfsetbuttcap%
\pgfsetroundjoin%
\definecolor{currentfill}{rgb}{0.135066,0.544853,0.554029}%
\pgfsetfillcolor{currentfill}%
\pgfsetfillopacity{0.700000}%
\pgfsetlinewidth{0.000000pt}%
\definecolor{currentstroke}{rgb}{0.000000,0.000000,0.000000}%
\pgfsetstrokecolor{currentstroke}%
\pgfsetdash{}{0pt}%
\pgfpathmoveto{\pgfqpoint{2.372018in}{2.783537in}}%
\pgfpathlineto{\pgfqpoint{2.386176in}{2.763900in}}%
\pgfpathlineto{\pgfqpoint{2.400327in}{2.744439in}}%
\pgfpathlineto{\pgfqpoint{2.414471in}{2.725150in}}%
\pgfpathlineto{\pgfqpoint{2.428608in}{2.706034in}}%
\pgfpathlineto{\pgfqpoint{2.419223in}{2.718713in}}%
\pgfpathlineto{\pgfqpoint{2.409811in}{2.731821in}}%
\pgfpathlineto{\pgfqpoint{2.400372in}{2.745366in}}%
\pgfpathlineto{\pgfqpoint{2.390904in}{2.759353in}}%
\pgfpathlineto{\pgfqpoint{2.376702in}{2.779111in}}%
\pgfpathlineto{\pgfqpoint{2.362492in}{2.799043in}}%
\pgfpathlineto{\pgfqpoint{2.348276in}{2.819149in}}%
\pgfpathlineto{\pgfqpoint{2.334051in}{2.839431in}}%
\pgfpathlineto{\pgfqpoint{2.343586in}{2.824789in}}%
\pgfpathlineto{\pgfqpoint{2.353092in}{2.810598in}}%
\pgfpathlineto{\pgfqpoint{2.362569in}{2.796850in}}%
\pgfpathlineto{\pgfqpoint{2.372018in}{2.783537in}}%
\pgfpathclose%
\pgfusepath{fill}%
\end{pgfscope}%
\begin{pgfscope}%
\pgfpathrectangle{\pgfqpoint{1.150000in}{0.150000in}}{\pgfqpoint{5.700000in}{5.700000in}}%
\pgfusepath{clip}%
\pgfsetbuttcap%
\pgfsetroundjoin%
\definecolor{currentfill}{rgb}{0.241237,0.296485,0.539709}%
\pgfsetfillcolor{currentfill}%
\pgfsetfillopacity{0.700000}%
\pgfsetlinewidth{0.000000pt}%
\definecolor{currentstroke}{rgb}{0.000000,0.000000,0.000000}%
\pgfsetstrokecolor{currentstroke}%
\pgfsetdash{}{0pt}%
\pgfpathmoveto{\pgfqpoint{5.026823in}{2.124781in}}%
\pgfpathlineto{\pgfqpoint{5.041268in}{2.129920in}}%
\pgfpathlineto{\pgfqpoint{5.055726in}{2.135158in}}%
\pgfpathlineto{\pgfqpoint{5.070198in}{2.140493in}}%
\pgfpathlineto{\pgfqpoint{5.084683in}{2.145928in}}%
\pgfpathlineto{\pgfqpoint{5.076821in}{2.131622in}}%
\pgfpathlineto{\pgfqpoint{5.068954in}{2.117274in}}%
\pgfpathlineto{\pgfqpoint{5.061083in}{2.102887in}}%
\pgfpathlineto{\pgfqpoint{5.053206in}{2.088463in}}%
\pgfpathlineto{\pgfqpoint{5.038724in}{2.083323in}}%
\pgfpathlineto{\pgfqpoint{5.024254in}{2.078282in}}%
\pgfpathlineto{\pgfqpoint{5.009798in}{2.073339in}}%
\pgfpathlineto{\pgfqpoint{4.995356in}{2.068495in}}%
\pgfpathlineto{\pgfqpoint{5.003230in}{2.082617in}}%
\pgfpathlineto{\pgfqpoint{5.011099in}{2.096707in}}%
\pgfpathlineto{\pgfqpoint{5.018963in}{2.110763in}}%
\pgfpathlineto{\pgfqpoint{5.026823in}{2.124781in}}%
\pgfpathclose%
\pgfusepath{fill}%
\end{pgfscope}%
\begin{pgfscope}%
\pgfpathrectangle{\pgfqpoint{1.150000in}{0.150000in}}{\pgfqpoint{5.700000in}{5.700000in}}%
\pgfusepath{clip}%
\pgfsetbuttcap%
\pgfsetroundjoin%
\definecolor{currentfill}{rgb}{0.271305,0.019942,0.347269}%
\pgfsetfillcolor{currentfill}%
\pgfsetfillopacity{0.700000}%
\pgfsetlinewidth{0.000000pt}%
\definecolor{currentstroke}{rgb}{0.000000,0.000000,0.000000}%
\pgfsetstrokecolor{currentstroke}%
\pgfsetdash{}{0pt}%
\pgfpathmoveto{\pgfqpoint{3.962915in}{1.558179in}}%
\pgfpathlineto{\pgfqpoint{3.976944in}{1.553935in}}%
\pgfpathlineto{\pgfqpoint{3.990980in}{1.549792in}}%
\pgfpathlineto{\pgfqpoint{4.005023in}{1.545749in}}%
\pgfpathlineto{\pgfqpoint{4.019071in}{1.541806in}}%
\pgfpathlineto{\pgfqpoint{4.010923in}{1.535303in}}%
\pgfpathlineto{\pgfqpoint{4.002767in}{1.528993in}}%
\pgfpathlineto{\pgfqpoint{3.994604in}{1.522883in}}%
\pgfpathlineto{\pgfqpoint{3.986434in}{1.516976in}}%
\pgfpathlineto{\pgfqpoint{3.972368in}{1.521424in}}%
\pgfpathlineto{\pgfqpoint{3.958309in}{1.525972in}}%
\pgfpathlineto{\pgfqpoint{3.944255in}{1.530620in}}%
\pgfpathlineto{\pgfqpoint{3.930207in}{1.535369in}}%
\pgfpathlineto{\pgfqpoint{3.938396in}{1.540764in}}%
\pgfpathlineto{\pgfqpoint{3.946576in}{1.546367in}}%
\pgfpathlineto{\pgfqpoint{3.954749in}{1.552174in}}%
\pgfpathlineto{\pgfqpoint{3.962915in}{1.558179in}}%
\pgfpathclose%
\pgfusepath{fill}%
\end{pgfscope}%
\begin{pgfscope}%
\pgfpathrectangle{\pgfqpoint{1.150000in}{0.150000in}}{\pgfqpoint{5.700000in}{5.700000in}}%
\pgfusepath{clip}%
\pgfsetbuttcap%
\pgfsetroundjoin%
\definecolor{currentfill}{rgb}{0.273809,0.031497,0.358853}%
\pgfsetfillcolor{currentfill}%
\pgfsetfillopacity{0.700000}%
\pgfsetlinewidth{0.000000pt}%
\definecolor{currentstroke}{rgb}{0.000000,0.000000,0.000000}%
\pgfsetstrokecolor{currentstroke}%
\pgfsetdash{}{0pt}%
\pgfpathmoveto{\pgfqpoint{3.818032in}{1.576992in}}%
\pgfpathlineto{\pgfqpoint{3.832035in}{1.571434in}}%
\pgfpathlineto{\pgfqpoint{3.846043in}{1.565977in}}%
\pgfpathlineto{\pgfqpoint{3.860056in}{1.560622in}}%
\pgfpathlineto{\pgfqpoint{3.874075in}{1.555369in}}%
\pgfpathlineto{\pgfqpoint{3.865859in}{1.550705in}}%
\pgfpathlineto{\pgfqpoint{3.857635in}{1.546264in}}%
\pgfpathlineto{\pgfqpoint{3.849402in}{1.542052in}}%
\pgfpathlineto{\pgfqpoint{3.841160in}{1.538075in}}%
\pgfpathlineto{\pgfqpoint{3.827120in}{1.543852in}}%
\pgfpathlineto{\pgfqpoint{3.813085in}{1.549731in}}%
\pgfpathlineto{\pgfqpoint{3.799055in}{1.555712in}}%
\pgfpathlineto{\pgfqpoint{3.785030in}{1.561795in}}%
\pgfpathlineto{\pgfqpoint{3.793294in}{1.565241in}}%
\pgfpathlineto{\pgfqpoint{3.801549in}{1.568926in}}%
\pgfpathlineto{\pgfqpoint{3.809795in}{1.572845in}}%
\pgfpathlineto{\pgfqpoint{3.818032in}{1.576992in}}%
\pgfpathclose%
\pgfusepath{fill}%
\end{pgfscope}%
\begin{pgfscope}%
\pgfpathrectangle{\pgfqpoint{1.150000in}{0.150000in}}{\pgfqpoint{5.700000in}{5.700000in}}%
\pgfusepath{clip}%
\pgfsetbuttcap%
\pgfsetroundjoin%
\definecolor{currentfill}{rgb}{0.271828,0.209303,0.504434}%
\pgfsetfillcolor{currentfill}%
\pgfsetfillopacity{0.700000}%
\pgfsetlinewidth{0.000000pt}%
\definecolor{currentstroke}{rgb}{0.000000,0.000000,0.000000}%
\pgfsetstrokecolor{currentstroke}%
\pgfsetdash{}{0pt}%
\pgfpathmoveto{\pgfqpoint{4.817158in}{1.924609in}}%
\pgfpathlineto{\pgfqpoint{4.831495in}{1.928029in}}%
\pgfpathlineto{\pgfqpoint{4.845845in}{1.931547in}}%
\pgfpathlineto{\pgfqpoint{4.860207in}{1.935162in}}%
\pgfpathlineto{\pgfqpoint{4.874582in}{1.938875in}}%
\pgfpathlineto{\pgfqpoint{4.866671in}{1.924936in}}%
\pgfpathlineto{\pgfqpoint{4.858755in}{1.910999in}}%
\pgfpathlineto{\pgfqpoint{4.850835in}{1.897067in}}%
\pgfpathlineto{\pgfqpoint{4.842912in}{1.883144in}}%
\pgfpathlineto{\pgfqpoint{4.828538in}{1.879779in}}%
\pgfpathlineto{\pgfqpoint{4.814177in}{1.876511in}}%
\pgfpathlineto{\pgfqpoint{4.799828in}{1.873341in}}%
\pgfpathlineto{\pgfqpoint{4.785491in}{1.870268in}}%
\pgfpathlineto{\pgfqpoint{4.793414in}{1.883837in}}%
\pgfpathlineto{\pgfqpoint{4.801333in}{1.897419in}}%
\pgfpathlineto{\pgfqpoint{4.809247in}{1.911011in}}%
\pgfpathlineto{\pgfqpoint{4.817158in}{1.924609in}}%
\pgfpathclose%
\pgfusepath{fill}%
\end{pgfscope}%
\begin{pgfscope}%
\pgfpathrectangle{\pgfqpoint{1.150000in}{0.150000in}}{\pgfqpoint{5.700000in}{5.700000in}}%
\pgfusepath{clip}%
\pgfsetbuttcap%
\pgfsetroundjoin%
\definecolor{currentfill}{rgb}{0.146180,0.515413,0.556823}%
\pgfsetfillcolor{currentfill}%
\pgfsetfillopacity{0.700000}%
\pgfsetlinewidth{0.000000pt}%
\definecolor{currentstroke}{rgb}{0.000000,0.000000,0.000000}%
\pgfsetstrokecolor{currentstroke}%
\pgfsetdash{}{0pt}%
\pgfpathmoveto{\pgfqpoint{5.567356in}{2.695611in}}%
\pgfpathlineto{\pgfqpoint{5.582119in}{2.704434in}}%
\pgfpathlineto{\pgfqpoint{5.596897in}{2.713358in}}%
\pgfpathlineto{\pgfqpoint{5.611693in}{2.722384in}}%
\pgfpathlineto{\pgfqpoint{5.626505in}{2.731512in}}%
\pgfpathlineto{\pgfqpoint{5.618832in}{2.718982in}}%
\pgfpathlineto{\pgfqpoint{5.611151in}{2.706327in}}%
\pgfpathlineto{\pgfqpoint{5.603462in}{2.693546in}}%
\pgfpathlineto{\pgfqpoint{5.595765in}{2.680642in}}%
\pgfpathlineto{\pgfqpoint{5.580954in}{2.671659in}}%
\pgfpathlineto{\pgfqpoint{5.566159in}{2.662778in}}%
\pgfpathlineto{\pgfqpoint{5.551382in}{2.653999in}}%
\pgfpathlineto{\pgfqpoint{5.536620in}{2.645321in}}%
\pgfpathlineto{\pgfqpoint{5.544316in}{2.658073in}}%
\pgfpathlineto{\pgfqpoint{5.552004in}{2.670706in}}%
\pgfpathlineto{\pgfqpoint{5.559684in}{2.683219in}}%
\pgfpathlineto{\pgfqpoint{5.567356in}{2.695611in}}%
\pgfpathclose%
\pgfusepath{fill}%
\end{pgfscope}%
\begin{pgfscope}%
\pgfpathrectangle{\pgfqpoint{1.150000in}{0.150000in}}{\pgfqpoint{5.700000in}{5.700000in}}%
\pgfusepath{clip}%
\pgfsetbuttcap%
\pgfsetroundjoin%
\definecolor{currentfill}{rgb}{0.283187,0.125848,0.444960}%
\pgfsetfillcolor{currentfill}%
\pgfsetfillopacity{0.700000}%
\pgfsetlinewidth{0.000000pt}%
\definecolor{currentstroke}{rgb}{0.000000,0.000000,0.000000}%
\pgfsetstrokecolor{currentstroke}%
\pgfsetdash{}{0pt}%
\pgfpathmoveto{\pgfqpoint{3.415855in}{1.741102in}}%
\pgfpathlineto{\pgfqpoint{3.429820in}{1.731918in}}%
\pgfpathlineto{\pgfqpoint{3.443788in}{1.722846in}}%
\pgfpathlineto{\pgfqpoint{3.457758in}{1.713884in}}%
\pgfpathlineto{\pgfqpoint{3.471731in}{1.705033in}}%
\pgfpathlineto{\pgfqpoint{3.463274in}{1.705657in}}%
\pgfpathlineto{\pgfqpoint{3.454803in}{1.706580in}}%
\pgfpathlineto{\pgfqpoint{3.446320in}{1.707807in}}%
\pgfpathlineto{\pgfqpoint{3.437823in}{1.709344in}}%
\pgfpathlineto{\pgfqpoint{3.423817in}{1.718764in}}%
\pgfpathlineto{\pgfqpoint{3.409813in}{1.728295in}}%
\pgfpathlineto{\pgfqpoint{3.395811in}{1.737936in}}%
\pgfpathlineto{\pgfqpoint{3.381811in}{1.747689in}}%
\pgfpathlineto{\pgfqpoint{3.390342in}{1.745574in}}%
\pgfpathlineto{\pgfqpoint{3.398860in}{1.743776in}}%
\pgfpathlineto{\pgfqpoint{3.407364in}{1.742287in}}%
\pgfpathlineto{\pgfqpoint{3.415855in}{1.741102in}}%
\pgfpathclose%
\pgfusepath{fill}%
\end{pgfscope}%
\begin{pgfscope}%
\pgfpathrectangle{\pgfqpoint{1.150000in}{0.150000in}}{\pgfqpoint{5.700000in}{5.700000in}}%
\pgfusepath{clip}%
\pgfsetbuttcap%
\pgfsetroundjoin%
\definecolor{currentfill}{rgb}{0.258965,0.251537,0.524736}%
\pgfsetfillcolor{currentfill}%
\pgfsetfillopacity{0.700000}%
\pgfsetlinewidth{0.000000pt}%
\definecolor{currentstroke}{rgb}{0.000000,0.000000,0.000000}%
\pgfsetstrokecolor{currentstroke}%
\pgfsetdash{}{0pt}%
\pgfpathmoveto{\pgfqpoint{3.046162in}{2.016438in}}%
\pgfpathlineto{\pgfqpoint{3.060144in}{2.003859in}}%
\pgfpathlineto{\pgfqpoint{3.074125in}{1.991404in}}%
\pgfpathlineto{\pgfqpoint{3.088105in}{1.979073in}}%
\pgfpathlineto{\pgfqpoint{3.102086in}{1.966866in}}%
\pgfpathlineto{\pgfqpoint{3.093340in}{1.972237in}}%
\pgfpathlineto{\pgfqpoint{3.084576in}{1.977964in}}%
\pgfpathlineto{\pgfqpoint{3.075795in}{1.984053in}}%
\pgfpathlineto{\pgfqpoint{3.066994in}{1.990512in}}%
\pgfpathlineto{\pgfqpoint{3.052969in}{2.003320in}}%
\pgfpathlineto{\pgfqpoint{3.038943in}{2.016252in}}%
\pgfpathlineto{\pgfqpoint{3.024916in}{2.029308in}}%
\pgfpathlineto{\pgfqpoint{3.010888in}{2.042490in}}%
\pgfpathlineto{\pgfqpoint{3.019735in}{2.035422in}}%
\pgfpathlineto{\pgfqpoint{3.028563in}{2.028728in}}%
\pgfpathlineto{\pgfqpoint{3.037372in}{2.022403in}}%
\pgfpathlineto{\pgfqpoint{3.046162in}{2.016438in}}%
\pgfpathclose%
\pgfusepath{fill}%
\end{pgfscope}%
\begin{pgfscope}%
\pgfpathrectangle{\pgfqpoint{1.150000in}{0.150000in}}{\pgfqpoint{5.700000in}{5.700000in}}%
\pgfusepath{clip}%
\pgfsetbuttcap%
\pgfsetroundjoin%
\definecolor{currentfill}{rgb}{0.201239,0.383670,0.554294}%
\pgfsetfillcolor{currentfill}%
\pgfsetfillopacity{0.700000}%
\pgfsetlinewidth{0.000000pt}%
\definecolor{currentstroke}{rgb}{0.000000,0.000000,0.000000}%
\pgfsetstrokecolor{currentstroke}%
\pgfsetdash{}{0pt}%
\pgfpathmoveto{\pgfqpoint{5.236745in}{2.339242in}}%
\pgfpathlineto{\pgfqpoint{5.251311in}{2.345945in}}%
\pgfpathlineto{\pgfqpoint{5.265891in}{2.352748in}}%
\pgfpathlineto{\pgfqpoint{5.280486in}{2.359651in}}%
\pgfpathlineto{\pgfqpoint{5.295096in}{2.366653in}}%
\pgfpathlineto{\pgfqpoint{5.287291in}{2.352576in}}%
\pgfpathlineto{\pgfqpoint{5.279481in}{2.338419in}}%
\pgfpathlineto{\pgfqpoint{5.271665in}{2.324185in}}%
\pgfpathlineto{\pgfqpoint{5.263843in}{2.309875in}}%
\pgfpathlineto{\pgfqpoint{5.249237in}{2.303113in}}%
\pgfpathlineto{\pgfqpoint{5.234645in}{2.296451in}}%
\pgfpathlineto{\pgfqpoint{5.220068in}{2.289887in}}%
\pgfpathlineto{\pgfqpoint{5.205505in}{2.283424in}}%
\pgfpathlineto{\pgfqpoint{5.213324in}{2.297486in}}%
\pgfpathlineto{\pgfqpoint{5.221136in}{2.311478in}}%
\pgfpathlineto{\pgfqpoint{5.228944in}{2.325397in}}%
\pgfpathlineto{\pgfqpoint{5.236745in}{2.339242in}}%
\pgfpathclose%
\pgfusepath{fill}%
\end{pgfscope}%
\begin{pgfscope}%
\pgfpathrectangle{\pgfqpoint{1.150000in}{0.150000in}}{\pgfqpoint{5.700000in}{5.700000in}}%
\pgfusepath{clip}%
\pgfsetbuttcap%
\pgfsetroundjoin%
\definecolor{currentfill}{rgb}{0.125394,0.574318,0.549086}%
\pgfsetfillcolor{currentfill}%
\pgfsetfillopacity{0.700000}%
\pgfsetlinewidth{0.000000pt}%
\definecolor{currentstroke}{rgb}{0.000000,0.000000,0.000000}%
\pgfsetstrokecolor{currentstroke}%
\pgfsetdash{}{0pt}%
\pgfpathmoveto{\pgfqpoint{2.315309in}{2.863864in}}%
\pgfpathlineto{\pgfqpoint{2.329498in}{2.843512in}}%
\pgfpathlineto{\pgfqpoint{2.343679in}{2.823341in}}%
\pgfpathlineto{\pgfqpoint{2.357852in}{2.803350in}}%
\pgfpathlineto{\pgfqpoint{2.372018in}{2.783537in}}%
\pgfpathlineto{\pgfqpoint{2.362569in}{2.796850in}}%
\pgfpathlineto{\pgfqpoint{2.353092in}{2.810598in}}%
\pgfpathlineto{\pgfqpoint{2.343586in}{2.824789in}}%
\pgfpathlineto{\pgfqpoint{2.334051in}{2.839431in}}%
\pgfpathlineto{\pgfqpoint{2.319819in}{2.859891in}}%
\pgfpathlineto{\pgfqpoint{2.305578in}{2.880530in}}%
\pgfpathlineto{\pgfqpoint{2.291329in}{2.901350in}}%
\pgfpathlineto{\pgfqpoint{2.277072in}{2.922352in}}%
\pgfpathlineto{\pgfqpoint{2.286676in}{2.907052in}}%
\pgfpathlineto{\pgfqpoint{2.296250in}{2.892209in}}%
\pgfpathlineto{\pgfqpoint{2.305794in}{2.877815in}}%
\pgfpathlineto{\pgfqpoint{2.315309in}{2.863864in}}%
\pgfpathclose%
\pgfusepath{fill}%
\end{pgfscope}%
\begin{pgfscope}%
\pgfpathrectangle{\pgfqpoint{1.150000in}{0.150000in}}{\pgfqpoint{5.700000in}{5.700000in}}%
\pgfusepath{clip}%
\pgfsetbuttcap%
\pgfsetroundjoin%
\definecolor{currentfill}{rgb}{0.272594,0.025563,0.353093}%
\pgfsetfillcolor{currentfill}%
\pgfsetfillopacity{0.700000}%
\pgfsetlinewidth{0.000000pt}%
\definecolor{currentstroke}{rgb}{0.000000,0.000000,0.000000}%
\pgfsetstrokecolor{currentstroke}%
\pgfsetdash{}{0pt}%
\pgfpathmoveto{\pgfqpoint{4.107800in}{1.556821in}}%
\pgfpathlineto{\pgfqpoint{4.121868in}{1.553860in}}%
\pgfpathlineto{\pgfqpoint{4.135944in}{1.550998in}}%
\pgfpathlineto{\pgfqpoint{4.150027in}{1.548234in}}%
\pgfpathlineto{\pgfqpoint{4.164118in}{1.545569in}}%
\pgfpathlineto{\pgfqpoint{4.156024in}{1.537382in}}%
\pgfpathlineto{\pgfqpoint{4.147924in}{1.529359in}}%
\pgfpathlineto{\pgfqpoint{4.139818in}{1.521507in}}%
\pgfpathlineto{\pgfqpoint{4.131706in}{1.513828in}}%
\pgfpathlineto{\pgfqpoint{4.117602in}{1.516980in}}%
\pgfpathlineto{\pgfqpoint{4.103506in}{1.520230in}}%
\pgfpathlineto{\pgfqpoint{4.089416in}{1.523579in}}%
\pgfpathlineto{\pgfqpoint{4.075333in}{1.527026in}}%
\pgfpathlineto{\pgfqpoint{4.083460in}{1.534211in}}%
\pgfpathlineto{\pgfqpoint{4.091579in}{1.541575in}}%
\pgfpathlineto{\pgfqpoint{4.099693in}{1.549113in}}%
\pgfpathlineto{\pgfqpoint{4.107800in}{1.556821in}}%
\pgfpathclose%
\pgfusepath{fill}%
\end{pgfscope}%
\begin{pgfscope}%
\pgfpathrectangle{\pgfqpoint{1.150000in}{0.150000in}}{\pgfqpoint{5.700000in}{5.700000in}}%
\pgfusepath{clip}%
\pgfsetbuttcap%
\pgfsetroundjoin%
\definecolor{currentfill}{rgb}{0.265145,0.232956,0.516599}%
\pgfsetfillcolor{currentfill}%
\pgfsetfillopacity{0.700000}%
\pgfsetlinewidth{0.000000pt}%
\definecolor{currentstroke}{rgb}{0.000000,0.000000,0.000000}%
\pgfsetstrokecolor{currentstroke}%
\pgfsetdash{}{0pt}%
\pgfpathmoveto{\pgfqpoint{3.102086in}{1.966866in}}%
\pgfpathlineto{\pgfqpoint{3.116066in}{1.954781in}}%
\pgfpathlineto{\pgfqpoint{3.130047in}{1.942818in}}%
\pgfpathlineto{\pgfqpoint{3.144027in}{1.930977in}}%
\pgfpathlineto{\pgfqpoint{3.158008in}{1.919257in}}%
\pgfpathlineto{\pgfqpoint{3.149305in}{1.924037in}}%
\pgfpathlineto{\pgfqpoint{3.140585in}{1.929168in}}%
\pgfpathlineto{\pgfqpoint{3.131847in}{1.934655in}}%
\pgfpathlineto{\pgfqpoint{3.123092in}{1.940506in}}%
\pgfpathlineto{\pgfqpoint{3.109068in}{1.952824in}}%
\pgfpathlineto{\pgfqpoint{3.095044in}{1.965265in}}%
\pgfpathlineto{\pgfqpoint{3.081019in}{1.977827in}}%
\pgfpathlineto{\pgfqpoint{3.066994in}{1.990512in}}%
\pgfpathlineto{\pgfqpoint{3.075795in}{1.984053in}}%
\pgfpathlineto{\pgfqpoint{3.084576in}{1.977964in}}%
\pgfpathlineto{\pgfqpoint{3.093340in}{1.972237in}}%
\pgfpathlineto{\pgfqpoint{3.102086in}{1.966866in}}%
\pgfpathclose%
\pgfusepath{fill}%
\end{pgfscope}%
\begin{pgfscope}%
\pgfpathrectangle{\pgfqpoint{1.150000in}{0.150000in}}{\pgfqpoint{5.700000in}{5.700000in}}%
\pgfusepath{clip}%
\pgfsetbuttcap%
\pgfsetroundjoin%
\definecolor{currentfill}{rgb}{0.278791,0.062145,0.386592}%
\pgfsetfillcolor{currentfill}%
\pgfsetfillopacity{0.700000}%
\pgfsetlinewidth{0.000000pt}%
\definecolor{currentstroke}{rgb}{0.000000,0.000000,0.000000}%
\pgfsetstrokecolor{currentstroke}%
\pgfsetdash{}{0pt}%
\pgfpathmoveto{\pgfqpoint{3.672999in}{1.614176in}}%
\pgfpathlineto{\pgfqpoint{3.686987in}{1.607265in}}%
\pgfpathlineto{\pgfqpoint{3.700980in}{1.600457in}}%
\pgfpathlineto{\pgfqpoint{3.714977in}{1.593755in}}%
\pgfpathlineto{\pgfqpoint{3.728978in}{1.587156in}}%
\pgfpathlineto{\pgfqpoint{3.720681in}{1.584492in}}%
\pgfpathlineto{\pgfqpoint{3.712373in}{1.582082in}}%
\pgfpathlineto{\pgfqpoint{3.704055in}{1.579934in}}%
\pgfpathlineto{\pgfqpoint{3.695727in}{1.578052in}}%
\pgfpathlineto{\pgfqpoint{3.681700in}{1.585195in}}%
\pgfpathlineto{\pgfqpoint{3.667677in}{1.592443in}}%
\pgfpathlineto{\pgfqpoint{3.653658in}{1.599794in}}%
\pgfpathlineto{\pgfqpoint{3.639643in}{1.607251in}}%
\pgfpathlineto{\pgfqpoint{3.647998in}{1.608581in}}%
\pgfpathlineto{\pgfqpoint{3.656342in}{1.610182in}}%
\pgfpathlineto{\pgfqpoint{3.664676in}{1.612049in}}%
\pgfpathlineto{\pgfqpoint{3.672999in}{1.614176in}}%
\pgfpathclose%
\pgfusepath{fill}%
\end{pgfscope}%
\begin{pgfscope}%
\pgfpathrectangle{\pgfqpoint{1.150000in}{0.150000in}}{\pgfqpoint{5.700000in}{5.700000in}}%
\pgfusepath{clip}%
\pgfsetbuttcap%
\pgfsetroundjoin%
\definecolor{currentfill}{rgb}{0.165117,0.467423,0.558141}%
\pgfsetfillcolor{currentfill}%
\pgfsetfillopacity{0.700000}%
\pgfsetlinewidth{0.000000pt}%
\definecolor{currentstroke}{rgb}{0.000000,0.000000,0.000000}%
\pgfsetstrokecolor{currentstroke}%
\pgfsetdash{}{0pt}%
\pgfpathmoveto{\pgfqpoint{5.446889in}{2.560102in}}%
\pgfpathlineto{\pgfqpoint{5.461583in}{2.568213in}}%
\pgfpathlineto{\pgfqpoint{5.476294in}{2.576424in}}%
\pgfpathlineto{\pgfqpoint{5.491021in}{2.584736in}}%
\pgfpathlineto{\pgfqpoint{5.505764in}{2.593148in}}%
\pgfpathlineto{\pgfqpoint{5.498032in}{2.579821in}}%
\pgfpathlineto{\pgfqpoint{5.490293in}{2.566383in}}%
\pgfpathlineto{\pgfqpoint{5.482547in}{2.552836in}}%
\pgfpathlineto{\pgfqpoint{5.474793in}{2.539181in}}%
\pgfpathlineto{\pgfqpoint{5.460053in}{2.530952in}}%
\pgfpathlineto{\pgfqpoint{5.445328in}{2.522824in}}%
\pgfpathlineto{\pgfqpoint{5.430619in}{2.514796in}}%
\pgfpathlineto{\pgfqpoint{5.415927in}{2.506869in}}%
\pgfpathlineto{\pgfqpoint{5.423677in}{2.520333in}}%
\pgfpathlineto{\pgfqpoint{5.431421in}{2.533695in}}%
\pgfpathlineto{\pgfqpoint{5.439158in}{2.546951in}}%
\pgfpathlineto{\pgfqpoint{5.446889in}{2.560102in}}%
\pgfpathclose%
\pgfusepath{fill}%
\end{pgfscope}%
\begin{pgfscope}%
\pgfpathrectangle{\pgfqpoint{1.150000in}{0.150000in}}{\pgfqpoint{5.700000in}{5.700000in}}%
\pgfusepath{clip}%
\pgfsetbuttcap%
\pgfsetroundjoin%
\definecolor{currentfill}{rgb}{0.262138,0.242286,0.520837}%
\pgfsetfillcolor{currentfill}%
\pgfsetfillopacity{0.700000}%
\pgfsetlinewidth{0.000000pt}%
\definecolor{currentstroke}{rgb}{0.000000,0.000000,0.000000}%
\pgfsetstrokecolor{currentstroke}%
\pgfsetdash{}{0pt}%
\pgfpathmoveto{\pgfqpoint{4.906185in}{1.994591in}}%
\pgfpathlineto{\pgfqpoint{4.920573in}{1.998732in}}%
\pgfpathlineto{\pgfqpoint{4.934975in}{2.002970in}}%
\pgfpathlineto{\pgfqpoint{4.949389in}{2.007307in}}%
\pgfpathlineto{\pgfqpoint{4.963816in}{2.011741in}}%
\pgfpathlineto{\pgfqpoint{4.955920in}{1.997500in}}%
\pgfpathlineto{\pgfqpoint{4.948019in}{1.983245in}}%
\pgfpathlineto{\pgfqpoint{4.940114in}{1.968978in}}%
\pgfpathlineto{\pgfqpoint{4.932205in}{1.954703in}}%
\pgfpathlineto{\pgfqpoint{4.917781in}{1.950599in}}%
\pgfpathlineto{\pgfqpoint{4.903368in}{1.946594in}}%
\pgfpathlineto{\pgfqpoint{4.888969in}{1.942686in}}%
\pgfpathlineto{\pgfqpoint{4.874582in}{1.938875in}}%
\pgfpathlineto{\pgfqpoint{4.882489in}{1.952814in}}%
\pgfpathlineto{\pgfqpoint{4.890392in}{1.966748in}}%
\pgfpathlineto{\pgfqpoint{4.898290in}{1.980675in}}%
\pgfpathlineto{\pgfqpoint{4.906185in}{1.994591in}}%
\pgfpathclose%
\pgfusepath{fill}%
\end{pgfscope}%
\begin{pgfscope}%
\pgfpathrectangle{\pgfqpoint{1.150000in}{0.150000in}}{\pgfqpoint{5.700000in}{5.700000in}}%
\pgfusepath{clip}%
\pgfsetbuttcap%
\pgfsetroundjoin%
\definecolor{currentfill}{rgb}{0.282656,0.100196,0.422160}%
\pgfsetfillcolor{currentfill}%
\pgfsetfillopacity{0.700000}%
\pgfsetlinewidth{0.000000pt}%
\definecolor{currentstroke}{rgb}{0.000000,0.000000,0.000000}%
\pgfsetstrokecolor{currentstroke}%
\pgfsetdash{}{0pt}%
\pgfpathmoveto{\pgfqpoint{4.518899in}{1.694185in}}%
\pgfpathlineto{\pgfqpoint{4.533108in}{1.694946in}}%
\pgfpathlineto{\pgfqpoint{4.547328in}{1.695805in}}%
\pgfpathlineto{\pgfqpoint{4.561558in}{1.696760in}}%
\pgfpathlineto{\pgfqpoint{4.575798in}{1.697813in}}%
\pgfpathlineto{\pgfqpoint{4.567820in}{1.685556in}}%
\pgfpathlineto{\pgfqpoint{4.559837in}{1.673371in}}%
\pgfpathlineto{\pgfqpoint{4.551851in}{1.661263in}}%
\pgfpathlineto{\pgfqpoint{4.543860in}{1.649235in}}%
\pgfpathlineto{\pgfqpoint{4.529616in}{1.648600in}}%
\pgfpathlineto{\pgfqpoint{4.515383in}{1.648062in}}%
\pgfpathlineto{\pgfqpoint{4.501159in}{1.647620in}}%
\pgfpathlineto{\pgfqpoint{4.486945in}{1.647275in}}%
\pgfpathlineto{\pgfqpoint{4.494940in}{1.658879in}}%
\pgfpathlineto{\pgfqpoint{4.502931in}{1.670568in}}%
\pgfpathlineto{\pgfqpoint{4.510917in}{1.682338in}}%
\pgfpathlineto{\pgfqpoint{4.518899in}{1.694185in}}%
\pgfpathclose%
\pgfusepath{fill}%
\end{pgfscope}%
\begin{pgfscope}%
\pgfpathrectangle{\pgfqpoint{1.150000in}{0.150000in}}{\pgfqpoint{5.700000in}{5.700000in}}%
\pgfusepath{clip}%
\pgfsetbuttcap%
\pgfsetroundjoin%
\definecolor{currentfill}{rgb}{0.280894,0.078907,0.402329}%
\pgfsetfillcolor{currentfill}%
\pgfsetfillopacity{0.700000}%
\pgfsetlinewidth{0.000000pt}%
\definecolor{currentstroke}{rgb}{0.000000,0.000000,0.000000}%
\pgfsetstrokecolor{currentstroke}%
\pgfsetdash{}{0pt}%
\pgfpathmoveto{\pgfqpoint{4.430188in}{1.646865in}}%
\pgfpathlineto{\pgfqpoint{4.444363in}{1.646821in}}%
\pgfpathlineto{\pgfqpoint{4.458547in}{1.646876in}}%
\pgfpathlineto{\pgfqpoint{4.472741in}{1.647027in}}%
\pgfpathlineto{\pgfqpoint{4.486945in}{1.647275in}}%
\pgfpathlineto{\pgfqpoint{4.478946in}{1.635760in}}%
\pgfpathlineto{\pgfqpoint{4.470942in}{1.624339in}}%
\pgfpathlineto{\pgfqpoint{4.462934in}{1.613015in}}%
\pgfpathlineto{\pgfqpoint{4.454922in}{1.601792in}}%
\pgfpathlineto{\pgfqpoint{4.440713in}{1.601979in}}%
\pgfpathlineto{\pgfqpoint{4.426513in}{1.602262in}}%
\pgfpathlineto{\pgfqpoint{4.412323in}{1.602642in}}%
\pgfpathlineto{\pgfqpoint{4.398142in}{1.603119in}}%
\pgfpathlineto{\pgfqpoint{4.406160in}{1.613900in}}%
\pgfpathlineto{\pgfqpoint{4.414174in}{1.624788in}}%
\pgfpathlineto{\pgfqpoint{4.422183in}{1.635777in}}%
\pgfpathlineto{\pgfqpoint{4.430188in}{1.646865in}}%
\pgfpathclose%
\pgfusepath{fill}%
\end{pgfscope}%
\begin{pgfscope}%
\pgfpathrectangle{\pgfqpoint{1.150000in}{0.150000in}}{\pgfqpoint{5.700000in}{5.700000in}}%
\pgfusepath{clip}%
\pgfsetbuttcap%
\pgfsetroundjoin%
\definecolor{currentfill}{rgb}{0.119738,0.603785,0.541400}%
\pgfsetfillcolor{currentfill}%
\pgfsetfillopacity{0.700000}%
\pgfsetlinewidth{0.000000pt}%
\definecolor{currentstroke}{rgb}{0.000000,0.000000,0.000000}%
\pgfsetstrokecolor{currentstroke}%
\pgfsetdash{}{0pt}%
\pgfpathmoveto{\pgfqpoint{2.258468in}{2.947116in}}%
\pgfpathlineto{\pgfqpoint{2.272691in}{2.926023in}}%
\pgfpathlineto{\pgfqpoint{2.286906in}{2.905117in}}%
\pgfpathlineto{\pgfqpoint{2.301111in}{2.884398in}}%
\pgfpathlineto{\pgfqpoint{2.315309in}{2.863864in}}%
\pgfpathlineto{\pgfqpoint{2.305794in}{2.877815in}}%
\pgfpathlineto{\pgfqpoint{2.296250in}{2.892209in}}%
\pgfpathlineto{\pgfqpoint{2.286676in}{2.907052in}}%
\pgfpathlineto{\pgfqpoint{2.277072in}{2.922352in}}%
\pgfpathlineto{\pgfqpoint{2.262806in}{2.943539in}}%
\pgfpathlineto{\pgfqpoint{2.248531in}{2.964911in}}%
\pgfpathlineto{\pgfqpoint{2.234247in}{2.986471in}}%
\pgfpathlineto{\pgfqpoint{2.219954in}{3.008221in}}%
\pgfpathlineto{\pgfqpoint{2.229629in}{2.992256in}}%
\pgfpathlineto{\pgfqpoint{2.239272in}{2.976755in}}%
\pgfpathlineto{\pgfqpoint{2.248885in}{2.961711in}}%
\pgfpathlineto{\pgfqpoint{2.258468in}{2.947116in}}%
\pgfpathclose%
\pgfusepath{fill}%
\end{pgfscope}%
\begin{pgfscope}%
\pgfpathrectangle{\pgfqpoint{1.150000in}{0.150000in}}{\pgfqpoint{5.700000in}{5.700000in}}%
\pgfusepath{clip}%
\pgfsetbuttcap%
\pgfsetroundjoin%
\definecolor{currentfill}{rgb}{0.223925,0.334994,0.548053}%
\pgfsetfillcolor{currentfill}%
\pgfsetfillopacity{0.700000}%
\pgfsetlinewidth{0.000000pt}%
\definecolor{currentstroke}{rgb}{0.000000,0.000000,0.000000}%
\pgfsetstrokecolor{currentstroke}%
\pgfsetdash{}{0pt}%
\pgfpathmoveto{\pgfqpoint{5.116083in}{2.202680in}}%
\pgfpathlineto{\pgfqpoint{5.130585in}{2.208489in}}%
\pgfpathlineto{\pgfqpoint{5.145101in}{2.214398in}}%
\pgfpathlineto{\pgfqpoint{5.159632in}{2.220405in}}%
\pgfpathlineto{\pgfqpoint{5.174177in}{2.226511in}}%
\pgfpathlineto{\pgfqpoint{5.166332in}{2.212129in}}%
\pgfpathlineto{\pgfqpoint{5.158481in}{2.197690in}}%
\pgfpathlineto{\pgfqpoint{5.150626in}{2.183196in}}%
\pgfpathlineto{\pgfqpoint{5.142765in}{2.168651in}}%
\pgfpathlineto{\pgfqpoint{5.128224in}{2.162822in}}%
\pgfpathlineto{\pgfqpoint{5.113696in}{2.157092in}}%
\pgfpathlineto{\pgfqpoint{5.099183in}{2.151461in}}%
\pgfpathlineto{\pgfqpoint{5.084683in}{2.145928in}}%
\pgfpathlineto{\pgfqpoint{5.092541in}{2.160189in}}%
\pgfpathlineto{\pgfqpoint{5.100393in}{2.174403in}}%
\pgfpathlineto{\pgfqpoint{5.108240in}{2.188568in}}%
\pgfpathlineto{\pgfqpoint{5.116083in}{2.202680in}}%
\pgfpathclose%
\pgfusepath{fill}%
\end{pgfscope}%
\begin{pgfscope}%
\pgfpathrectangle{\pgfqpoint{1.150000in}{0.150000in}}{\pgfqpoint{5.700000in}{5.700000in}}%
\pgfusepath{clip}%
\pgfsetbuttcap%
\pgfsetroundjoin%
\definecolor{currentfill}{rgb}{0.133743,0.548535,0.553541}%
\pgfsetfillcolor{currentfill}%
\pgfsetfillopacity{0.700000}%
\pgfsetlinewidth{0.000000pt}%
\definecolor{currentstroke}{rgb}{0.000000,0.000000,0.000000}%
\pgfsetstrokecolor{currentstroke}%
\pgfsetdash{}{0pt}%
\pgfpathmoveto{\pgfqpoint{5.657117in}{2.780347in}}%
\pgfpathlineto{\pgfqpoint{5.671946in}{2.789701in}}%
\pgfpathlineto{\pgfqpoint{5.686793in}{2.799157in}}%
\pgfpathlineto{\pgfqpoint{5.701657in}{2.808716in}}%
\pgfpathlineto{\pgfqpoint{5.694016in}{2.796613in}}%
\pgfpathlineto{\pgfqpoint{5.686367in}{2.784376in}}%
\pgfpathlineto{\pgfqpoint{5.678710in}{2.772006in}}%
\pgfpathlineto{\pgfqpoint{5.671044in}{2.759505in}}%
\pgfpathlineto{\pgfqpoint{5.656181in}{2.750072in}}%
\pgfpathlineto{\pgfqpoint{5.641334in}{2.740741in}}%
\pgfpathlineto{\pgfqpoint{5.626505in}{2.731512in}}%
\pgfpathlineto{\pgfqpoint{5.634170in}{2.743914in}}%
\pgfpathlineto{\pgfqpoint{5.641827in}{2.756188in}}%
\pgfpathlineto{\pgfqpoint{5.649476in}{2.768332in}}%
\pgfpathlineto{\pgfqpoint{5.657117in}{2.780347in}}%
\pgfpathclose%
\pgfusepath{fill}%
\end{pgfscope}%
\begin{pgfscope}%
\pgfpathrectangle{\pgfqpoint{1.150000in}{0.150000in}}{\pgfqpoint{5.700000in}{5.700000in}}%
\pgfusepath{clip}%
\pgfsetbuttcap%
\pgfsetroundjoin%
\definecolor{currentfill}{rgb}{0.283187,0.125848,0.444960}%
\pgfsetfillcolor{currentfill}%
\pgfsetfillopacity{0.700000}%
\pgfsetlinewidth{0.000000pt}%
\definecolor{currentstroke}{rgb}{0.000000,0.000000,0.000000}%
\pgfsetstrokecolor{currentstroke}%
\pgfsetdash{}{0pt}%
\pgfpathmoveto{\pgfqpoint{4.607670in}{1.747491in}}%
\pgfpathlineto{\pgfqpoint{4.621918in}{1.749040in}}%
\pgfpathlineto{\pgfqpoint{4.636177in}{1.750686in}}%
\pgfpathlineto{\pgfqpoint{4.650447in}{1.752429in}}%
\pgfpathlineto{\pgfqpoint{4.664728in}{1.754268in}}%
\pgfpathlineto{\pgfqpoint{4.656768in}{1.741366in}}%
\pgfpathlineto{\pgfqpoint{4.648804in}{1.728516in}}%
\pgfpathlineto{\pgfqpoint{4.640836in}{1.715723in}}%
\pgfpathlineto{\pgfqpoint{4.632865in}{1.702991in}}%
\pgfpathlineto{\pgfqpoint{4.618582in}{1.701551in}}%
\pgfpathlineto{\pgfqpoint{4.604310in}{1.700208in}}%
\pgfpathlineto{\pgfqpoint{4.590049in}{1.698962in}}%
\pgfpathlineto{\pgfqpoint{4.575798in}{1.697813in}}%
\pgfpathlineto{\pgfqpoint{4.583772in}{1.710139in}}%
\pgfpathlineto{\pgfqpoint{4.591742in}{1.722530in}}%
\pgfpathlineto{\pgfqpoint{4.599708in}{1.734982in}}%
\pgfpathlineto{\pgfqpoint{4.607670in}{1.747491in}}%
\pgfpathclose%
\pgfusepath{fill}%
\end{pgfscope}%
\begin{pgfscope}%
\pgfpathrectangle{\pgfqpoint{1.150000in}{0.150000in}}{\pgfqpoint{5.700000in}{5.700000in}}%
\pgfusepath{clip}%
\pgfsetbuttcap%
\pgfsetroundjoin%
\definecolor{currentfill}{rgb}{0.277941,0.056324,0.381191}%
\pgfsetfillcolor{currentfill}%
\pgfsetfillopacity{0.700000}%
\pgfsetlinewidth{0.000000pt}%
\definecolor{currentstroke}{rgb}{0.000000,0.000000,0.000000}%
\pgfsetstrokecolor{currentstroke}%
\pgfsetdash{}{0pt}%
\pgfpathmoveto{\pgfqpoint{4.341509in}{1.605997in}}%
\pgfpathlineto{\pgfqpoint{4.355653in}{1.605132in}}%
\pgfpathlineto{\pgfqpoint{4.369807in}{1.604363in}}%
\pgfpathlineto{\pgfqpoint{4.383970in}{1.603692in}}%
\pgfpathlineto{\pgfqpoint{4.398142in}{1.603119in}}%
\pgfpathlineto{\pgfqpoint{4.390119in}{1.592447in}}%
\pgfpathlineto{\pgfqpoint{4.382091in}{1.581891in}}%
\pgfpathlineto{\pgfqpoint{4.374059in}{1.571454in}}%
\pgfpathlineto{\pgfqpoint{4.366021in}{1.561140in}}%
\pgfpathlineto{\pgfqpoint{4.351842in}{1.562165in}}%
\pgfpathlineto{\pgfqpoint{4.337672in}{1.563288in}}%
\pgfpathlineto{\pgfqpoint{4.323510in}{1.564508in}}%
\pgfpathlineto{\pgfqpoint{4.309357in}{1.565824in}}%
\pgfpathlineto{\pgfqpoint{4.317403in}{1.575680in}}%
\pgfpathlineto{\pgfqpoint{4.325443in}{1.585664in}}%
\pgfpathlineto{\pgfqpoint{4.333478in}{1.595771in}}%
\pgfpathlineto{\pgfqpoint{4.341509in}{1.605997in}}%
\pgfpathclose%
\pgfusepath{fill}%
\end{pgfscope}%
\begin{pgfscope}%
\pgfpathrectangle{\pgfqpoint{1.150000in}{0.150000in}}{\pgfqpoint{5.700000in}{5.700000in}}%
\pgfusepath{clip}%
\pgfsetbuttcap%
\pgfsetroundjoin%
\definecolor{currentfill}{rgb}{0.283091,0.110553,0.431554}%
\pgfsetfillcolor{currentfill}%
\pgfsetfillopacity{0.700000}%
\pgfsetlinewidth{0.000000pt}%
\definecolor{currentstroke}{rgb}{0.000000,0.000000,0.000000}%
\pgfsetstrokecolor{currentstroke}%
\pgfsetdash{}{0pt}%
\pgfpathmoveto{\pgfqpoint{3.471731in}{1.705033in}}%
\pgfpathlineto{\pgfqpoint{3.485706in}{1.696291in}}%
\pgfpathlineto{\pgfqpoint{3.499685in}{1.687658in}}%
\pgfpathlineto{\pgfqpoint{3.513666in}{1.679134in}}%
\pgfpathlineto{\pgfqpoint{3.527650in}{1.670718in}}%
\pgfpathlineto{\pgfqpoint{3.519225in}{1.670783in}}%
\pgfpathlineto{\pgfqpoint{3.510787in}{1.671140in}}%
\pgfpathlineto{\pgfqpoint{3.502336in}{1.671797in}}%
\pgfpathlineto{\pgfqpoint{3.493873in}{1.672759in}}%
\pgfpathlineto{\pgfqpoint{3.479857in}{1.681742in}}%
\pgfpathlineto{\pgfqpoint{3.465843in}{1.690833in}}%
\pgfpathlineto{\pgfqpoint{3.451832in}{1.700034in}}%
\pgfpathlineto{\pgfqpoint{3.437823in}{1.709344in}}%
\pgfpathlineto{\pgfqpoint{3.446320in}{1.707807in}}%
\pgfpathlineto{\pgfqpoint{3.454803in}{1.706580in}}%
\pgfpathlineto{\pgfqpoint{3.463274in}{1.705657in}}%
\pgfpathlineto{\pgfqpoint{3.471731in}{1.705033in}}%
\pgfpathclose%
\pgfusepath{fill}%
\end{pgfscope}%
\begin{pgfscope}%
\pgfpathrectangle{\pgfqpoint{1.150000in}{0.150000in}}{\pgfqpoint{5.700000in}{5.700000in}}%
\pgfusepath{clip}%
\pgfsetbuttcap%
\pgfsetroundjoin%
\definecolor{currentfill}{rgb}{0.270595,0.214069,0.507052}%
\pgfsetfillcolor{currentfill}%
\pgfsetfillopacity{0.700000}%
\pgfsetlinewidth{0.000000pt}%
\definecolor{currentstroke}{rgb}{0.000000,0.000000,0.000000}%
\pgfsetstrokecolor{currentstroke}%
\pgfsetdash{}{0pt}%
\pgfpathmoveto{\pgfqpoint{3.158008in}{1.919257in}}%
\pgfpathlineto{\pgfqpoint{3.171989in}{1.907656in}}%
\pgfpathlineto{\pgfqpoint{3.185970in}{1.896176in}}%
\pgfpathlineto{\pgfqpoint{3.199952in}{1.884814in}}%
\pgfpathlineto{\pgfqpoint{3.213935in}{1.873571in}}%
\pgfpathlineto{\pgfqpoint{3.205273in}{1.877762in}}%
\pgfpathlineto{\pgfqpoint{3.196595in}{1.882299in}}%
\pgfpathlineto{\pgfqpoint{3.187900in}{1.887186in}}%
\pgfpathlineto{\pgfqpoint{3.179189in}{1.892431in}}%
\pgfpathlineto{\pgfqpoint{3.165164in}{1.904271in}}%
\pgfpathlineto{\pgfqpoint{3.151140in}{1.916229in}}%
\pgfpathlineto{\pgfqpoint{3.137116in}{1.928307in}}%
\pgfpathlineto{\pgfqpoint{3.123092in}{1.940506in}}%
\pgfpathlineto{\pgfqpoint{3.131847in}{1.934655in}}%
\pgfpathlineto{\pgfqpoint{3.140585in}{1.929168in}}%
\pgfpathlineto{\pgfqpoint{3.149305in}{1.924037in}}%
\pgfpathlineto{\pgfqpoint{3.158008in}{1.919257in}}%
\pgfpathclose%
\pgfusepath{fill}%
\end{pgfscope}%
\begin{pgfscope}%
\pgfpathrectangle{\pgfqpoint{1.150000in}{0.150000in}}{\pgfqpoint{5.700000in}{5.700000in}}%
\pgfusepath{clip}%
\pgfsetbuttcap%
\pgfsetroundjoin%
\definecolor{currentfill}{rgb}{0.280868,0.160771,0.472899}%
\pgfsetfillcolor{currentfill}%
\pgfsetfillopacity{0.700000}%
\pgfsetlinewidth{0.000000pt}%
\definecolor{currentstroke}{rgb}{0.000000,0.000000,0.000000}%
\pgfsetstrokecolor{currentstroke}%
\pgfsetdash{}{0pt}%
\pgfpathmoveto{\pgfqpoint{4.696527in}{1.806331in}}%
\pgfpathlineto{\pgfqpoint{4.710818in}{1.808651in}}%
\pgfpathlineto{\pgfqpoint{4.725120in}{1.811067in}}%
\pgfpathlineto{\pgfqpoint{4.739433in}{1.813580in}}%
\pgfpathlineto{\pgfqpoint{4.753759in}{1.816191in}}%
\pgfpathlineto{\pgfqpoint{4.745816in}{1.802738in}}%
\pgfpathlineto{\pgfqpoint{4.737868in}{1.789319in}}%
\pgfpathlineto{\pgfqpoint{4.729917in}{1.775937in}}%
\pgfpathlineto{\pgfqpoint{4.721962in}{1.762596in}}%
\pgfpathlineto{\pgfqpoint{4.707637in}{1.760369in}}%
\pgfpathlineto{\pgfqpoint{4.693323in}{1.758238in}}%
\pgfpathlineto{\pgfqpoint{4.679020in}{1.756205in}}%
\pgfpathlineto{\pgfqpoint{4.664728in}{1.754268in}}%
\pgfpathlineto{\pgfqpoint{4.672684in}{1.767220in}}%
\pgfpathlineto{\pgfqpoint{4.680636in}{1.780216in}}%
\pgfpathlineto{\pgfqpoint{4.688583in}{1.793255in}}%
\pgfpathlineto{\pgfqpoint{4.696527in}{1.806331in}}%
\pgfpathclose%
\pgfusepath{fill}%
\end{pgfscope}%
\begin{pgfscope}%
\pgfpathrectangle{\pgfqpoint{1.150000in}{0.150000in}}{\pgfqpoint{5.700000in}{5.700000in}}%
\pgfusepath{clip}%
\pgfsetbuttcap%
\pgfsetroundjoin%
\definecolor{currentfill}{rgb}{0.183898,0.422383,0.556944}%
\pgfsetfillcolor{currentfill}%
\pgfsetfillopacity{0.700000}%
\pgfsetlinewidth{0.000000pt}%
\definecolor{currentstroke}{rgb}{0.000000,0.000000,0.000000}%
\pgfsetstrokecolor{currentstroke}%
\pgfsetdash{}{0pt}%
\pgfpathmoveto{\pgfqpoint{5.326254in}{2.422124in}}%
\pgfpathlineto{\pgfqpoint{5.340882in}{2.429448in}}%
\pgfpathlineto{\pgfqpoint{5.355525in}{2.436871in}}%
\pgfpathlineto{\pgfqpoint{5.370184in}{2.444395in}}%
\pgfpathlineto{\pgfqpoint{5.384858in}{2.452018in}}%
\pgfpathlineto{\pgfqpoint{5.377075in}{2.438065in}}%
\pgfpathlineto{\pgfqpoint{5.369285in}{2.424020in}}%
\pgfpathlineto{\pgfqpoint{5.361490in}{2.409884in}}%
\pgfpathlineto{\pgfqpoint{5.353688in}{2.395660in}}%
\pgfpathlineto{\pgfqpoint{5.339017in}{2.388258in}}%
\pgfpathlineto{\pgfqpoint{5.324361in}{2.380957in}}%
\pgfpathlineto{\pgfqpoint{5.309721in}{2.373755in}}%
\pgfpathlineto{\pgfqpoint{5.295096in}{2.366653in}}%
\pgfpathlineto{\pgfqpoint{5.302894in}{2.380648in}}%
\pgfpathlineto{\pgfqpoint{5.310687in}{2.394560in}}%
\pgfpathlineto{\pgfqpoint{5.318473in}{2.408386in}}%
\pgfpathlineto{\pgfqpoint{5.326254in}{2.422124in}}%
\pgfpathclose%
\pgfusepath{fill}%
\end{pgfscope}%
\begin{pgfscope}%
\pgfpathrectangle{\pgfqpoint{1.150000in}{0.150000in}}{\pgfqpoint{5.700000in}{5.700000in}}%
\pgfusepath{clip}%
\pgfsetbuttcap%
\pgfsetroundjoin%
\definecolor{currentfill}{rgb}{0.273809,0.031497,0.358853}%
\pgfsetfillcolor{currentfill}%
\pgfsetfillopacity{0.700000}%
\pgfsetlinewidth{0.000000pt}%
\definecolor{currentstroke}{rgb}{0.000000,0.000000,0.000000}%
\pgfsetstrokecolor{currentstroke}%
\pgfsetdash{}{0pt}%
\pgfpathmoveto{\pgfqpoint{3.874075in}{1.555369in}}%
\pgfpathlineto{\pgfqpoint{3.888100in}{1.550217in}}%
\pgfpathlineto{\pgfqpoint{3.902130in}{1.545167in}}%
\pgfpathlineto{\pgfqpoint{3.916166in}{1.540218in}}%
\pgfpathlineto{\pgfqpoint{3.930207in}{1.535369in}}%
\pgfpathlineto{\pgfqpoint{3.922011in}{1.530188in}}%
\pgfpathlineto{\pgfqpoint{3.913807in}{1.525225in}}%
\pgfpathlineto{\pgfqpoint{3.905595in}{1.520487in}}%
\pgfpathlineto{\pgfqpoint{3.897374in}{1.515979in}}%
\pgfpathlineto{\pgfqpoint{3.883313in}{1.521351in}}%
\pgfpathlineto{\pgfqpoint{3.869257in}{1.526825in}}%
\pgfpathlineto{\pgfqpoint{3.855206in}{1.532399in}}%
\pgfpathlineto{\pgfqpoint{3.841160in}{1.538075in}}%
\pgfpathlineto{\pgfqpoint{3.849402in}{1.542052in}}%
\pgfpathlineto{\pgfqpoint{3.857635in}{1.546264in}}%
\pgfpathlineto{\pgfqpoint{3.865859in}{1.550705in}}%
\pgfpathlineto{\pgfqpoint{3.874075in}{1.555369in}}%
\pgfpathclose%
\pgfusepath{fill}%
\end{pgfscope}%
\begin{pgfscope}%
\pgfpathrectangle{\pgfqpoint{1.150000in}{0.150000in}}{\pgfqpoint{5.700000in}{5.700000in}}%
\pgfusepath{clip}%
\pgfsetbuttcap%
\pgfsetroundjoin%
\definecolor{currentfill}{rgb}{0.274952,0.037752,0.364543}%
\pgfsetfillcolor{currentfill}%
\pgfsetfillopacity{0.700000}%
\pgfsetlinewidth{0.000000pt}%
\definecolor{currentstroke}{rgb}{0.000000,0.000000,0.000000}%
\pgfsetstrokecolor{currentstroke}%
\pgfsetdash{}{0pt}%
\pgfpathmoveto{\pgfqpoint{4.252830in}{1.572065in}}%
\pgfpathlineto{\pgfqpoint{4.266950in}{1.570359in}}%
\pgfpathlineto{\pgfqpoint{4.281077in}{1.568750in}}%
\pgfpathlineto{\pgfqpoint{4.295213in}{1.567238in}}%
\pgfpathlineto{\pgfqpoint{4.309357in}{1.565824in}}%
\pgfpathlineto{\pgfqpoint{4.301307in}{1.556101in}}%
\pgfpathlineto{\pgfqpoint{4.293251in}{1.546515in}}%
\pgfpathlineto{\pgfqpoint{4.285191in}{1.537071in}}%
\pgfpathlineto{\pgfqpoint{4.277125in}{1.527772in}}%
\pgfpathlineto{\pgfqpoint{4.262971in}{1.529655in}}%
\pgfpathlineto{\pgfqpoint{4.248825in}{1.531636in}}%
\pgfpathlineto{\pgfqpoint{4.234688in}{1.533714in}}%
\pgfpathlineto{\pgfqpoint{4.220558in}{1.535889in}}%
\pgfpathlineto{\pgfqpoint{4.228634in}{1.544712in}}%
\pgfpathlineto{\pgfqpoint{4.236705in}{1.553685in}}%
\pgfpathlineto{\pgfqpoint{4.244770in}{1.562804in}}%
\pgfpathlineto{\pgfqpoint{4.252830in}{1.572065in}}%
\pgfpathclose%
\pgfusepath{fill}%
\end{pgfscope}%
\begin{pgfscope}%
\pgfpathrectangle{\pgfqpoint{1.150000in}{0.150000in}}{\pgfqpoint{5.700000in}{5.700000in}}%
\pgfusepath{clip}%
\pgfsetbuttcap%
\pgfsetroundjoin%
\definecolor{currentfill}{rgb}{0.123444,0.636809,0.528763}%
\pgfsetfillcolor{currentfill}%
\pgfsetfillopacity{0.700000}%
\pgfsetlinewidth{0.000000pt}%
\definecolor{currentstroke}{rgb}{0.000000,0.000000,0.000000}%
\pgfsetstrokecolor{currentstroke}%
\pgfsetdash{}{0pt}%
\pgfpathmoveto{\pgfqpoint{2.201484in}{3.033403in}}%
\pgfpathlineto{\pgfqpoint{2.215744in}{3.011540in}}%
\pgfpathlineto{\pgfqpoint{2.229995in}{2.989873in}}%
\pgfpathlineto{\pgfqpoint{2.244236in}{2.968398in}}%
\pgfpathlineto{\pgfqpoint{2.258468in}{2.947116in}}%
\pgfpathlineto{\pgfqpoint{2.248885in}{2.961711in}}%
\pgfpathlineto{\pgfqpoint{2.239272in}{2.976755in}}%
\pgfpathlineto{\pgfqpoint{2.229629in}{2.992256in}}%
\pgfpathlineto{\pgfqpoint{2.219954in}{3.008221in}}%
\pgfpathlineto{\pgfqpoint{2.205651in}{3.030161in}}%
\pgfpathlineto{\pgfqpoint{2.191339in}{3.052295in}}%
\pgfpathlineto{\pgfqpoint{2.177016in}{3.074623in}}%
\pgfpathlineto{\pgfqpoint{2.162684in}{3.097147in}}%
\pgfpathlineto{\pgfqpoint{2.172431in}{3.080512in}}%
\pgfpathlineto{\pgfqpoint{2.182147in}{3.064348in}}%
\pgfpathlineto{\pgfqpoint{2.191831in}{3.048647in}}%
\pgfpathlineto{\pgfqpoint{2.201484in}{3.033403in}}%
\pgfpathclose%
\pgfusepath{fill}%
\end{pgfscope}%
\begin{pgfscope}%
\pgfpathrectangle{\pgfqpoint{1.150000in}{0.150000in}}{\pgfqpoint{5.700000in}{5.700000in}}%
\pgfusepath{clip}%
\pgfsetbuttcap%
\pgfsetroundjoin%
\definecolor{currentfill}{rgb}{0.271305,0.019942,0.347269}%
\pgfsetfillcolor{currentfill}%
\pgfsetfillopacity{0.700000}%
\pgfsetlinewidth{0.000000pt}%
\definecolor{currentstroke}{rgb}{0.000000,0.000000,0.000000}%
\pgfsetstrokecolor{currentstroke}%
\pgfsetdash{}{0pt}%
\pgfpathmoveto{\pgfqpoint{4.019071in}{1.541806in}}%
\pgfpathlineto{\pgfqpoint{4.033127in}{1.537962in}}%
\pgfpathlineto{\pgfqpoint{4.047189in}{1.534217in}}%
\pgfpathlineto{\pgfqpoint{4.061258in}{1.530572in}}%
\pgfpathlineto{\pgfqpoint{4.075333in}{1.527026in}}%
\pgfpathlineto{\pgfqpoint{4.067201in}{1.520025in}}%
\pgfpathlineto{\pgfqpoint{4.059061in}{1.513213in}}%
\pgfpathlineto{\pgfqpoint{4.050915in}{1.506595in}}%
\pgfpathlineto{\pgfqpoint{4.042761in}{1.500176in}}%
\pgfpathlineto{\pgfqpoint{4.028670in}{1.504228in}}%
\pgfpathlineto{\pgfqpoint{4.014585in}{1.508378in}}%
\pgfpathlineto{\pgfqpoint{4.000506in}{1.512627in}}%
\pgfpathlineto{\pgfqpoint{3.986434in}{1.516976in}}%
\pgfpathlineto{\pgfqpoint{3.994604in}{1.522883in}}%
\pgfpathlineto{\pgfqpoint{4.002767in}{1.528993in}}%
\pgfpathlineto{\pgfqpoint{4.010923in}{1.535303in}}%
\pgfpathlineto{\pgfqpoint{4.019071in}{1.541806in}}%
\pgfpathclose%
\pgfusepath{fill}%
\end{pgfscope}%
\begin{pgfscope}%
\pgfpathrectangle{\pgfqpoint{1.150000in}{0.150000in}}{\pgfqpoint{5.700000in}{5.700000in}}%
\pgfusepath{clip}%
\pgfsetbuttcap%
\pgfsetroundjoin%
\definecolor{currentfill}{rgb}{0.248629,0.278775,0.534556}%
\pgfsetfillcolor{currentfill}%
\pgfsetfillopacity{0.700000}%
\pgfsetlinewidth{0.000000pt}%
\definecolor{currentstroke}{rgb}{0.000000,0.000000,0.000000}%
\pgfsetstrokecolor{currentstroke}%
\pgfsetdash{}{0pt}%
\pgfpathmoveto{\pgfqpoint{4.995356in}{2.068495in}}%
\pgfpathlineto{\pgfqpoint{5.009798in}{2.073339in}}%
\pgfpathlineto{\pgfqpoint{5.024254in}{2.078282in}}%
\pgfpathlineto{\pgfqpoint{5.038724in}{2.083323in}}%
\pgfpathlineto{\pgfqpoint{5.053206in}{2.088463in}}%
\pgfpathlineto{\pgfqpoint{5.045326in}{2.074005in}}%
\pgfpathlineto{\pgfqpoint{5.037440in}{2.059516in}}%
\pgfpathlineto{\pgfqpoint{5.029550in}{2.044999in}}%
\pgfpathlineto{\pgfqpoint{5.021656in}{2.030456in}}%
\pgfpathlineto{\pgfqpoint{5.007176in}{2.025630in}}%
\pgfpathlineto{\pgfqpoint{4.992709in}{2.020903in}}%
\pgfpathlineto{\pgfqpoint{4.978256in}{2.016273in}}%
\pgfpathlineto{\pgfqpoint{4.963816in}{2.011741in}}%
\pgfpathlineto{\pgfqpoint{4.971707in}{2.025963in}}%
\pgfpathlineto{\pgfqpoint{4.979595in}{2.040165in}}%
\pgfpathlineto{\pgfqpoint{4.987478in}{2.054343in}}%
\pgfpathlineto{\pgfqpoint{4.995356in}{2.068495in}}%
\pgfpathclose%
\pgfusepath{fill}%
\end{pgfscope}%
\begin{pgfscope}%
\pgfpathrectangle{\pgfqpoint{1.150000in}{0.150000in}}{\pgfqpoint{5.700000in}{5.700000in}}%
\pgfusepath{clip}%
\pgfsetbuttcap%
\pgfsetroundjoin%
\definecolor{currentfill}{rgb}{0.275191,0.194905,0.496005}%
\pgfsetfillcolor{currentfill}%
\pgfsetfillopacity{0.700000}%
\pgfsetlinewidth{0.000000pt}%
\definecolor{currentstroke}{rgb}{0.000000,0.000000,0.000000}%
\pgfsetstrokecolor{currentstroke}%
\pgfsetdash{}{0pt}%
\pgfpathmoveto{\pgfqpoint{3.213935in}{1.873571in}}%
\pgfpathlineto{\pgfqpoint{3.227918in}{1.862445in}}%
\pgfpathlineto{\pgfqpoint{3.241902in}{1.851437in}}%
\pgfpathlineto{\pgfqpoint{3.255887in}{1.840546in}}%
\pgfpathlineto{\pgfqpoint{3.269873in}{1.829770in}}%
\pgfpathlineto{\pgfqpoint{3.261251in}{1.833375in}}%
\pgfpathlineto{\pgfqpoint{3.252614in}{1.837319in}}%
\pgfpathlineto{\pgfqpoint{3.243960in}{1.841609in}}%
\pgfpathlineto{\pgfqpoint{3.235291in}{1.846250in}}%
\pgfpathlineto{\pgfqpoint{3.221264in}{1.857620in}}%
\pgfpathlineto{\pgfqpoint{3.207238in}{1.869106in}}%
\pgfpathlineto{\pgfqpoint{3.193213in}{1.880710in}}%
\pgfpathlineto{\pgfqpoint{3.179189in}{1.892431in}}%
\pgfpathlineto{\pgfqpoint{3.187900in}{1.887186in}}%
\pgfpathlineto{\pgfqpoint{3.196595in}{1.882299in}}%
\pgfpathlineto{\pgfqpoint{3.205273in}{1.877762in}}%
\pgfpathlineto{\pgfqpoint{3.213935in}{1.873571in}}%
\pgfpathclose%
\pgfusepath{fill}%
\end{pgfscope}%
\begin{pgfscope}%
\pgfpathrectangle{\pgfqpoint{1.150000in}{0.150000in}}{\pgfqpoint{5.700000in}{5.700000in}}%
\pgfusepath{clip}%
\pgfsetbuttcap%
\pgfsetroundjoin%
\definecolor{currentfill}{rgb}{0.276194,0.190074,0.493001}%
\pgfsetfillcolor{currentfill}%
\pgfsetfillopacity{0.700000}%
\pgfsetlinewidth{0.000000pt}%
\definecolor{currentstroke}{rgb}{0.000000,0.000000,0.000000}%
\pgfsetstrokecolor{currentstroke}%
\pgfsetdash{}{0pt}%
\pgfpathmoveto{\pgfqpoint{4.785491in}{1.870268in}}%
\pgfpathlineto{\pgfqpoint{4.799828in}{1.873341in}}%
\pgfpathlineto{\pgfqpoint{4.814177in}{1.876511in}}%
\pgfpathlineto{\pgfqpoint{4.828538in}{1.879779in}}%
\pgfpathlineto{\pgfqpoint{4.842912in}{1.883144in}}%
\pgfpathlineto{\pgfqpoint{4.834984in}{1.869232in}}%
\pgfpathlineto{\pgfqpoint{4.827052in}{1.855336in}}%
\pgfpathlineto{\pgfqpoint{4.819116in}{1.841458in}}%
\pgfpathlineto{\pgfqpoint{4.811177in}{1.827603in}}%
\pgfpathlineto{\pgfqpoint{4.796804in}{1.824604in}}%
\pgfpathlineto{\pgfqpoint{4.782444in}{1.821703in}}%
\pgfpathlineto{\pgfqpoint{4.768095in}{1.818898in}}%
\pgfpathlineto{\pgfqpoint{4.753759in}{1.816191in}}%
\pgfpathlineto{\pgfqpoint{4.761698in}{1.829674in}}%
\pgfpathlineto{\pgfqpoint{4.769633in}{1.843183in}}%
\pgfpathlineto{\pgfqpoint{4.777564in}{1.856716in}}%
\pgfpathlineto{\pgfqpoint{4.785491in}{1.870268in}}%
\pgfpathclose%
\pgfusepath{fill}%
\end{pgfscope}%
\begin{pgfscope}%
\pgfpathrectangle{\pgfqpoint{1.150000in}{0.150000in}}{\pgfqpoint{5.700000in}{5.700000in}}%
\pgfusepath{clip}%
\pgfsetbuttcap%
\pgfsetroundjoin%
\definecolor{currentfill}{rgb}{0.277941,0.056324,0.381191}%
\pgfsetfillcolor{currentfill}%
\pgfsetfillopacity{0.700000}%
\pgfsetlinewidth{0.000000pt}%
\definecolor{currentstroke}{rgb}{0.000000,0.000000,0.000000}%
\pgfsetstrokecolor{currentstroke}%
\pgfsetdash{}{0pt}%
\pgfpathmoveto{\pgfqpoint{3.728978in}{1.587156in}}%
\pgfpathlineto{\pgfqpoint{3.742984in}{1.580661in}}%
\pgfpathlineto{\pgfqpoint{3.756995in}{1.574269in}}%
\pgfpathlineto{\pgfqpoint{3.771010in}{1.567981in}}%
\pgfpathlineto{\pgfqpoint{3.785030in}{1.561795in}}%
\pgfpathlineto{\pgfqpoint{3.776757in}{1.558594in}}%
\pgfpathlineto{\pgfqpoint{3.768474in}{1.555643in}}%
\pgfpathlineto{\pgfqpoint{3.760182in}{1.552948in}}%
\pgfpathlineto{\pgfqpoint{3.751879in}{1.550514in}}%
\pgfpathlineto{\pgfqpoint{3.737835in}{1.557244in}}%
\pgfpathlineto{\pgfqpoint{3.723795in}{1.564077in}}%
\pgfpathlineto{\pgfqpoint{3.709759in}{1.571013in}}%
\pgfpathlineto{\pgfqpoint{3.695727in}{1.578052in}}%
\pgfpathlineto{\pgfqpoint{3.704055in}{1.579934in}}%
\pgfpathlineto{\pgfqpoint{3.712373in}{1.582082in}}%
\pgfpathlineto{\pgfqpoint{3.720681in}{1.584492in}}%
\pgfpathlineto{\pgfqpoint{3.728978in}{1.587156in}}%
\pgfpathclose%
\pgfusepath{fill}%
\end{pgfscope}%
\begin{pgfscope}%
\pgfpathrectangle{\pgfqpoint{1.150000in}{0.150000in}}{\pgfqpoint{5.700000in}{5.700000in}}%
\pgfusepath{clip}%
\pgfsetbuttcap%
\pgfsetroundjoin%
\definecolor{currentfill}{rgb}{0.206756,0.371758,0.553117}%
\pgfsetfillcolor{currentfill}%
\pgfsetfillopacity{0.700000}%
\pgfsetlinewidth{0.000000pt}%
\definecolor{currentstroke}{rgb}{0.000000,0.000000,0.000000}%
\pgfsetstrokecolor{currentstroke}%
\pgfsetdash{}{0pt}%
\pgfpathmoveto{\pgfqpoint{5.205505in}{2.283424in}}%
\pgfpathlineto{\pgfqpoint{5.220068in}{2.289887in}}%
\pgfpathlineto{\pgfqpoint{5.234645in}{2.296451in}}%
\pgfpathlineto{\pgfqpoint{5.249237in}{2.303113in}}%
\pgfpathlineto{\pgfqpoint{5.263843in}{2.309875in}}%
\pgfpathlineto{\pgfqpoint{5.256016in}{2.295492in}}%
\pgfpathlineto{\pgfqpoint{5.248183in}{2.281038in}}%
\pgfpathlineto{\pgfqpoint{5.240345in}{2.266515in}}%
\pgfpathlineto{\pgfqpoint{5.232501in}{2.251926in}}%
\pgfpathlineto{\pgfqpoint{5.217898in}{2.245423in}}%
\pgfpathlineto{\pgfqpoint{5.203310in}{2.239020in}}%
\pgfpathlineto{\pgfqpoint{5.188736in}{2.232716in}}%
\pgfpathlineto{\pgfqpoint{5.174177in}{2.226511in}}%
\pgfpathlineto{\pgfqpoint{5.182017in}{2.240834in}}%
\pgfpathlineto{\pgfqpoint{5.189852in}{2.255095in}}%
\pgfpathlineto{\pgfqpoint{5.197681in}{2.269293in}}%
\pgfpathlineto{\pgfqpoint{5.205505in}{2.283424in}}%
\pgfpathclose%
\pgfusepath{fill}%
\end{pgfscope}%
\begin{pgfscope}%
\pgfpathrectangle{\pgfqpoint{1.150000in}{0.150000in}}{\pgfqpoint{5.700000in}{5.700000in}}%
\pgfusepath{clip}%
\pgfsetbuttcap%
\pgfsetroundjoin%
\definecolor{currentfill}{rgb}{0.150476,0.504369,0.557430}%
\pgfsetfillcolor{currentfill}%
\pgfsetfillopacity{0.700000}%
\pgfsetlinewidth{0.000000pt}%
\definecolor{currentstroke}{rgb}{0.000000,0.000000,0.000000}%
\pgfsetstrokecolor{currentstroke}%
\pgfsetdash{}{0pt}%
\pgfpathmoveto{\pgfqpoint{5.536620in}{2.645321in}}%
\pgfpathlineto{\pgfqpoint{5.551382in}{2.653999in}}%
\pgfpathlineto{\pgfqpoint{5.566159in}{2.662778in}}%
\pgfpathlineto{\pgfqpoint{5.580954in}{2.671659in}}%
\pgfpathlineto{\pgfqpoint{5.595765in}{2.680642in}}%
\pgfpathlineto{\pgfqpoint{5.588060in}{2.667615in}}%
\pgfpathlineto{\pgfqpoint{5.580348in}{2.654466in}}%
\pgfpathlineto{\pgfqpoint{5.572628in}{2.641198in}}%
\pgfpathlineto{\pgfqpoint{5.564901in}{2.627811in}}%
\pgfpathlineto{\pgfqpoint{5.550092in}{2.618993in}}%
\pgfpathlineto{\pgfqpoint{5.535300in}{2.610277in}}%
\pgfpathlineto{\pgfqpoint{5.520524in}{2.601662in}}%
\pgfpathlineto{\pgfqpoint{5.505764in}{2.593148in}}%
\pgfpathlineto{\pgfqpoint{5.513489in}{2.606363in}}%
\pgfpathlineto{\pgfqpoint{5.521207in}{2.619465in}}%
\pgfpathlineto{\pgfqpoint{5.528917in}{2.632451in}}%
\pgfpathlineto{\pgfqpoint{5.536620in}{2.645321in}}%
\pgfpathclose%
\pgfusepath{fill}%
\end{pgfscope}%
\begin{pgfscope}%
\pgfpathrectangle{\pgfqpoint{1.150000in}{0.150000in}}{\pgfqpoint{5.700000in}{5.700000in}}%
\pgfusepath{clip}%
\pgfsetbuttcap%
\pgfsetroundjoin%
\definecolor{currentfill}{rgb}{0.282656,0.100196,0.422160}%
\pgfsetfillcolor{currentfill}%
\pgfsetfillopacity{0.700000}%
\pgfsetlinewidth{0.000000pt}%
\definecolor{currentstroke}{rgb}{0.000000,0.000000,0.000000}%
\pgfsetstrokecolor{currentstroke}%
\pgfsetdash{}{0pt}%
\pgfpathmoveto{\pgfqpoint{3.527650in}{1.670718in}}%
\pgfpathlineto{\pgfqpoint{3.541638in}{1.662411in}}%
\pgfpathlineto{\pgfqpoint{3.555628in}{1.654211in}}%
\pgfpathlineto{\pgfqpoint{3.569622in}{1.646118in}}%
\pgfpathlineto{\pgfqpoint{3.583619in}{1.638132in}}%
\pgfpathlineto{\pgfqpoint{3.575224in}{1.637638in}}%
\pgfpathlineto{\pgfqpoint{3.566817in}{1.637431in}}%
\pgfpathlineto{\pgfqpoint{3.558398in}{1.637519in}}%
\pgfpathlineto{\pgfqpoint{3.549966in}{1.637906in}}%
\pgfpathlineto{\pgfqpoint{3.535939in}{1.646458in}}%
\pgfpathlineto{\pgfqpoint{3.521914in}{1.655118in}}%
\pgfpathlineto{\pgfqpoint{3.507892in}{1.663884in}}%
\pgfpathlineto{\pgfqpoint{3.493873in}{1.672759in}}%
\pgfpathlineto{\pgfqpoint{3.502336in}{1.671797in}}%
\pgfpathlineto{\pgfqpoint{3.510787in}{1.671140in}}%
\pgfpathlineto{\pgfqpoint{3.519225in}{1.670783in}}%
\pgfpathlineto{\pgfqpoint{3.527650in}{1.670718in}}%
\pgfpathclose%
\pgfusepath{fill}%
\end{pgfscope}%
\begin{pgfscope}%
\pgfpathrectangle{\pgfqpoint{1.150000in}{0.150000in}}{\pgfqpoint{5.700000in}{5.700000in}}%
\pgfusepath{clip}%
\pgfsetbuttcap%
\pgfsetroundjoin%
\definecolor{currentfill}{rgb}{0.272594,0.025563,0.353093}%
\pgfsetfillcolor{currentfill}%
\pgfsetfillopacity{0.700000}%
\pgfsetlinewidth{0.000000pt}%
\definecolor{currentstroke}{rgb}{0.000000,0.000000,0.000000}%
\pgfsetstrokecolor{currentstroke}%
\pgfsetdash{}{0pt}%
\pgfpathmoveto{\pgfqpoint{4.164118in}{1.545569in}}%
\pgfpathlineto{\pgfqpoint{4.178216in}{1.543002in}}%
\pgfpathlineto{\pgfqpoint{4.192323in}{1.540533in}}%
\pgfpathlineto{\pgfqpoint{4.206436in}{1.538162in}}%
\pgfpathlineto{\pgfqpoint{4.220558in}{1.535889in}}%
\pgfpathlineto{\pgfqpoint{4.212476in}{1.527222in}}%
\pgfpathlineto{\pgfqpoint{4.204388in}{1.518715in}}%
\pgfpathlineto{\pgfqpoint{4.196295in}{1.510373in}}%
\pgfpathlineto{\pgfqpoint{4.188196in}{1.502201in}}%
\pgfpathlineto{\pgfqpoint{4.174062in}{1.504961in}}%
\pgfpathlineto{\pgfqpoint{4.159936in}{1.507819in}}%
\pgfpathlineto{\pgfqpoint{4.145818in}{1.510775in}}%
\pgfpathlineto{\pgfqpoint{4.131706in}{1.513828in}}%
\pgfpathlineto{\pgfqpoint{4.139818in}{1.521507in}}%
\pgfpathlineto{\pgfqpoint{4.147924in}{1.529359in}}%
\pgfpathlineto{\pgfqpoint{4.156024in}{1.537382in}}%
\pgfpathlineto{\pgfqpoint{4.164118in}{1.545569in}}%
\pgfpathclose%
\pgfusepath{fill}%
\end{pgfscope}%
\begin{pgfscope}%
\pgfpathrectangle{\pgfqpoint{1.150000in}{0.150000in}}{\pgfqpoint{5.700000in}{5.700000in}}%
\pgfusepath{clip}%
\pgfsetbuttcap%
\pgfsetroundjoin%
\definecolor{currentfill}{rgb}{0.146616,0.673050,0.508936}%
\pgfsetfillcolor{currentfill}%
\pgfsetfillopacity{0.700000}%
\pgfsetlinewidth{0.000000pt}%
\definecolor{currentstroke}{rgb}{0.000000,0.000000,0.000000}%
\pgfsetstrokecolor{currentstroke}%
\pgfsetdash{}{0pt}%
\pgfpathmoveto{\pgfqpoint{2.144342in}{3.122841in}}%
\pgfpathlineto{\pgfqpoint{2.158643in}{3.100179in}}%
\pgfpathlineto{\pgfqpoint{2.172933in}{3.077720in}}%
\pgfpathlineto{\pgfqpoint{2.187213in}{3.055462in}}%
\pgfpathlineto{\pgfqpoint{2.201484in}{3.033403in}}%
\pgfpathlineto{\pgfqpoint{2.191831in}{3.048647in}}%
\pgfpathlineto{\pgfqpoint{2.182147in}{3.064348in}}%
\pgfpathlineto{\pgfqpoint{2.172431in}{3.080512in}}%
\pgfpathlineto{\pgfqpoint{2.162684in}{3.097147in}}%
\pgfpathlineto{\pgfqpoint{2.148341in}{3.119870in}}%
\pgfpathlineto{\pgfqpoint{2.133988in}{3.142793in}}%
\pgfpathlineto{\pgfqpoint{2.119624in}{3.165919in}}%
\pgfpathlineto{\pgfqpoint{2.105249in}{3.189249in}}%
\pgfpathlineto{\pgfqpoint{2.115071in}{3.171937in}}%
\pgfpathlineto{\pgfqpoint{2.124860in}{3.155104in}}%
\pgfpathlineto{\pgfqpoint{2.134617in}{3.138741in}}%
\pgfpathlineto{\pgfqpoint{2.144342in}{3.122841in}}%
\pgfpathclose%
\pgfusepath{fill}%
\end{pgfscope}%
\begin{pgfscope}%
\pgfpathrectangle{\pgfqpoint{1.150000in}{0.150000in}}{\pgfqpoint{5.700000in}{5.700000in}}%
\pgfusepath{clip}%
\pgfsetbuttcap%
\pgfsetroundjoin%
\definecolor{currentfill}{rgb}{0.278012,0.180367,0.486697}%
\pgfsetfillcolor{currentfill}%
\pgfsetfillopacity{0.700000}%
\pgfsetlinewidth{0.000000pt}%
\definecolor{currentstroke}{rgb}{0.000000,0.000000,0.000000}%
\pgfsetstrokecolor{currentstroke}%
\pgfsetdash{}{0pt}%
\pgfpathmoveto{\pgfqpoint{3.269873in}{1.829770in}}%
\pgfpathlineto{\pgfqpoint{3.283860in}{1.819111in}}%
\pgfpathlineto{\pgfqpoint{3.297848in}{1.808566in}}%
\pgfpathlineto{\pgfqpoint{3.311838in}{1.798136in}}%
\pgfpathlineto{\pgfqpoint{3.325829in}{1.787820in}}%
\pgfpathlineto{\pgfqpoint{3.317246in}{1.790840in}}%
\pgfpathlineto{\pgfqpoint{3.308648in}{1.794194in}}%
\pgfpathlineto{\pgfqpoint{3.300034in}{1.797887in}}%
\pgfpathlineto{\pgfqpoint{3.291405in}{1.801926in}}%
\pgfpathlineto{\pgfqpoint{3.277375in}{1.812835in}}%
\pgfpathlineto{\pgfqpoint{3.263346in}{1.823858in}}%
\pgfpathlineto{\pgfqpoint{3.249318in}{1.834996in}}%
\pgfpathlineto{\pgfqpoint{3.235291in}{1.846250in}}%
\pgfpathlineto{\pgfqpoint{3.243960in}{1.841609in}}%
\pgfpathlineto{\pgfqpoint{3.252614in}{1.837319in}}%
\pgfpathlineto{\pgfqpoint{3.261251in}{1.833375in}}%
\pgfpathlineto{\pgfqpoint{3.269873in}{1.829770in}}%
\pgfpathclose%
\pgfusepath{fill}%
\end{pgfscope}%
\begin{pgfscope}%
\pgfpathrectangle{\pgfqpoint{1.150000in}{0.150000in}}{\pgfqpoint{5.700000in}{5.700000in}}%
\pgfusepath{clip}%
\pgfsetbuttcap%
\pgfsetroundjoin%
\definecolor{currentfill}{rgb}{0.266580,0.228262,0.514349}%
\pgfsetfillcolor{currentfill}%
\pgfsetfillopacity{0.700000}%
\pgfsetlinewidth{0.000000pt}%
\definecolor{currentstroke}{rgb}{0.000000,0.000000,0.000000}%
\pgfsetstrokecolor{currentstroke}%
\pgfsetdash{}{0pt}%
\pgfpathmoveto{\pgfqpoint{4.874582in}{1.938875in}}%
\pgfpathlineto{\pgfqpoint{4.888969in}{1.942686in}}%
\pgfpathlineto{\pgfqpoint{4.903368in}{1.946594in}}%
\pgfpathlineto{\pgfqpoint{4.917781in}{1.950599in}}%
\pgfpathlineto{\pgfqpoint{4.932205in}{1.954703in}}%
\pgfpathlineto{\pgfqpoint{4.924292in}{1.940421in}}%
\pgfpathlineto{\pgfqpoint{4.916375in}{1.926138in}}%
\pgfpathlineto{\pgfqpoint{4.908454in}{1.911854in}}%
\pgfpathlineto{\pgfqpoint{4.900528in}{1.897575in}}%
\pgfpathlineto{\pgfqpoint{4.886105in}{1.893822in}}%
\pgfpathlineto{\pgfqpoint{4.871695in}{1.890165in}}%
\pgfpathlineto{\pgfqpoint{4.857297in}{1.886606in}}%
\pgfpathlineto{\pgfqpoint{4.842912in}{1.883144in}}%
\pgfpathlineto{\pgfqpoint{4.850835in}{1.897067in}}%
\pgfpathlineto{\pgfqpoint{4.858755in}{1.910999in}}%
\pgfpathlineto{\pgfqpoint{4.866671in}{1.924936in}}%
\pgfpathlineto{\pgfqpoint{4.874582in}{1.938875in}}%
\pgfpathclose%
\pgfusepath{fill}%
\end{pgfscope}%
\begin{pgfscope}%
\pgfpathrectangle{\pgfqpoint{1.150000in}{0.150000in}}{\pgfqpoint{5.700000in}{5.700000in}}%
\pgfusepath{clip}%
\pgfsetbuttcap%
\pgfsetroundjoin%
\definecolor{currentfill}{rgb}{0.169646,0.456262,0.558030}%
\pgfsetfillcolor{currentfill}%
\pgfsetfillopacity{0.700000}%
\pgfsetlinewidth{0.000000pt}%
\definecolor{currentstroke}{rgb}{0.000000,0.000000,0.000000}%
\pgfsetstrokecolor{currentstroke}%
\pgfsetdash{}{0pt}%
\pgfpathmoveto{\pgfqpoint{5.415927in}{2.506869in}}%
\pgfpathlineto{\pgfqpoint{5.430619in}{2.514796in}}%
\pgfpathlineto{\pgfqpoint{5.445328in}{2.522824in}}%
\pgfpathlineto{\pgfqpoint{5.460053in}{2.530952in}}%
\pgfpathlineto{\pgfqpoint{5.474793in}{2.539181in}}%
\pgfpathlineto{\pgfqpoint{5.467033in}{2.525420in}}%
\pgfpathlineto{\pgfqpoint{5.459267in}{2.511554in}}%
\pgfpathlineto{\pgfqpoint{5.451493in}{2.497586in}}%
\pgfpathlineto{\pgfqpoint{5.443713in}{2.483517in}}%
\pgfpathlineto{\pgfqpoint{5.428975in}{2.475491in}}%
\pgfpathlineto{\pgfqpoint{5.414254in}{2.467567in}}%
\pgfpathlineto{\pgfqpoint{5.399548in}{2.459742in}}%
\pgfpathlineto{\pgfqpoint{5.384858in}{2.452018in}}%
\pgfpathlineto{\pgfqpoint{5.392635in}{2.465877in}}%
\pgfpathlineto{\pgfqpoint{5.400405in}{2.479640in}}%
\pgfpathlineto{\pgfqpoint{5.408169in}{2.493304in}}%
\pgfpathlineto{\pgfqpoint{5.415927in}{2.506869in}}%
\pgfpathclose%
\pgfusepath{fill}%
\end{pgfscope}%
\begin{pgfscope}%
\pgfpathrectangle{\pgfqpoint{1.150000in}{0.150000in}}{\pgfqpoint{5.700000in}{5.700000in}}%
\pgfusepath{clip}%
\pgfsetbuttcap%
\pgfsetroundjoin%
\definecolor{currentfill}{rgb}{0.231674,0.318106,0.544834}%
\pgfsetfillcolor{currentfill}%
\pgfsetfillopacity{0.700000}%
\pgfsetlinewidth{0.000000pt}%
\definecolor{currentstroke}{rgb}{0.000000,0.000000,0.000000}%
\pgfsetstrokecolor{currentstroke}%
\pgfsetdash{}{0pt}%
\pgfpathmoveto{\pgfqpoint{5.084683in}{2.145928in}}%
\pgfpathlineto{\pgfqpoint{5.099183in}{2.151461in}}%
\pgfpathlineto{\pgfqpoint{5.113696in}{2.157092in}}%
\pgfpathlineto{\pgfqpoint{5.128224in}{2.162822in}}%
\pgfpathlineto{\pgfqpoint{5.142765in}{2.168651in}}%
\pgfpathlineto{\pgfqpoint{5.134900in}{2.154056in}}%
\pgfpathlineto{\pgfqpoint{5.127030in}{2.139415in}}%
\pgfpathlineto{\pgfqpoint{5.119155in}{2.124730in}}%
\pgfpathlineto{\pgfqpoint{5.111275in}{2.110003in}}%
\pgfpathlineto{\pgfqpoint{5.096737in}{2.104470in}}%
\pgfpathlineto{\pgfqpoint{5.082213in}{2.099036in}}%
\pgfpathlineto{\pgfqpoint{5.067703in}{2.093700in}}%
\pgfpathlineto{\pgfqpoint{5.053206in}{2.088463in}}%
\pgfpathlineto{\pgfqpoint{5.061083in}{2.102887in}}%
\pgfpathlineto{\pgfqpoint{5.068954in}{2.117274in}}%
\pgfpathlineto{\pgfqpoint{5.076821in}{2.131622in}}%
\pgfpathlineto{\pgfqpoint{5.084683in}{2.145928in}}%
\pgfpathclose%
\pgfusepath{fill}%
\end{pgfscope}%
\begin{pgfscope}%
\pgfpathrectangle{\pgfqpoint{1.150000in}{0.150000in}}{\pgfqpoint{5.700000in}{5.700000in}}%
\pgfusepath{clip}%
\pgfsetbuttcap%
\pgfsetroundjoin%
\definecolor{currentfill}{rgb}{0.137770,0.537492,0.554906}%
\pgfsetfillcolor{currentfill}%
\pgfsetfillopacity{0.700000}%
\pgfsetlinewidth{0.000000pt}%
\definecolor{currentstroke}{rgb}{0.000000,0.000000,0.000000}%
\pgfsetstrokecolor{currentstroke}%
\pgfsetdash{}{0pt}%
\pgfpathmoveto{\pgfqpoint{5.626505in}{2.731512in}}%
\pgfpathlineto{\pgfqpoint{5.641334in}{2.740741in}}%
\pgfpathlineto{\pgfqpoint{5.656181in}{2.750072in}}%
\pgfpathlineto{\pgfqpoint{5.671044in}{2.759505in}}%
\pgfpathlineto{\pgfqpoint{5.663370in}{2.746872in}}%
\pgfpathlineto{\pgfqpoint{5.655688in}{2.734109in}}%
\pgfpathlineto{\pgfqpoint{5.647998in}{2.721218in}}%
\pgfpathlineto{\pgfqpoint{5.640300in}{2.708198in}}%
\pgfpathlineto{\pgfqpoint{5.625438in}{2.698911in}}%
\pgfpathlineto{\pgfqpoint{5.610593in}{2.689725in}}%
\pgfpathlineto{\pgfqpoint{5.595765in}{2.680642in}}%
\pgfpathlineto{\pgfqpoint{5.603462in}{2.693546in}}%
\pgfpathlineto{\pgfqpoint{5.611151in}{2.706327in}}%
\pgfpathlineto{\pgfqpoint{5.618832in}{2.718982in}}%
\pgfpathlineto{\pgfqpoint{5.626505in}{2.731512in}}%
\pgfpathclose%
\pgfusepath{fill}%
\end{pgfscope}%
\begin{pgfscope}%
\pgfpathrectangle{\pgfqpoint{1.150000in}{0.150000in}}{\pgfqpoint{5.700000in}{5.700000in}}%
\pgfusepath{clip}%
\pgfsetbuttcap%
\pgfsetroundjoin%
\definecolor{currentfill}{rgb}{0.272594,0.025563,0.353093}%
\pgfsetfillcolor{currentfill}%
\pgfsetfillopacity{0.700000}%
\pgfsetlinewidth{0.000000pt}%
\definecolor{currentstroke}{rgb}{0.000000,0.000000,0.000000}%
\pgfsetstrokecolor{currentstroke}%
\pgfsetdash{}{0pt}%
\pgfpathmoveto{\pgfqpoint{3.930207in}{1.535369in}}%
\pgfpathlineto{\pgfqpoint{3.944255in}{1.530620in}}%
\pgfpathlineto{\pgfqpoint{3.958309in}{1.525972in}}%
\pgfpathlineto{\pgfqpoint{3.972368in}{1.521424in}}%
\pgfpathlineto{\pgfqpoint{3.986434in}{1.516976in}}%
\pgfpathlineto{\pgfqpoint{3.978257in}{1.511278in}}%
\pgfpathlineto{\pgfqpoint{3.970071in}{1.505794in}}%
\pgfpathlineto{\pgfqpoint{3.961879in}{1.500530in}}%
\pgfpathlineto{\pgfqpoint{3.953678in}{1.495490in}}%
\pgfpathlineto{\pgfqpoint{3.939593in}{1.500462in}}%
\pgfpathlineto{\pgfqpoint{3.925515in}{1.505534in}}%
\pgfpathlineto{\pgfqpoint{3.911442in}{1.510706in}}%
\pgfpathlineto{\pgfqpoint{3.897374in}{1.515979in}}%
\pgfpathlineto{\pgfqpoint{3.905595in}{1.520487in}}%
\pgfpathlineto{\pgfqpoint{3.913807in}{1.525225in}}%
\pgfpathlineto{\pgfqpoint{3.922011in}{1.530188in}}%
\pgfpathlineto{\pgfqpoint{3.930207in}{1.535369in}}%
\pgfpathclose%
\pgfusepath{fill}%
\end{pgfscope}%
\begin{pgfscope}%
\pgfpathrectangle{\pgfqpoint{1.150000in}{0.150000in}}{\pgfqpoint{5.700000in}{5.700000in}}%
\pgfusepath{clip}%
\pgfsetbuttcap%
\pgfsetroundjoin%
\definecolor{currentfill}{rgb}{0.281924,0.089666,0.412415}%
\pgfsetfillcolor{currentfill}%
\pgfsetfillopacity{0.700000}%
\pgfsetlinewidth{0.000000pt}%
\definecolor{currentstroke}{rgb}{0.000000,0.000000,0.000000}%
\pgfsetstrokecolor{currentstroke}%
\pgfsetdash{}{0pt}%
\pgfpathmoveto{\pgfqpoint{4.486945in}{1.647275in}}%
\pgfpathlineto{\pgfqpoint{4.501159in}{1.647620in}}%
\pgfpathlineto{\pgfqpoint{4.515383in}{1.648062in}}%
\pgfpathlineto{\pgfqpoint{4.529616in}{1.648600in}}%
\pgfpathlineto{\pgfqpoint{4.543860in}{1.649235in}}%
\pgfpathlineto{\pgfqpoint{4.535865in}{1.637292in}}%
\pgfpathlineto{\pgfqpoint{4.527866in}{1.625438in}}%
\pgfpathlineto{\pgfqpoint{4.519863in}{1.613676in}}%
\pgfpathlineto{\pgfqpoint{4.511855in}{1.602012in}}%
\pgfpathlineto{\pgfqpoint{4.497607in}{1.601813in}}%
\pgfpathlineto{\pgfqpoint{4.483369in}{1.601709in}}%
\pgfpathlineto{\pgfqpoint{4.469141in}{1.601702in}}%
\pgfpathlineto{\pgfqpoint{4.454922in}{1.601792in}}%
\pgfpathlineto{\pgfqpoint{4.462934in}{1.613015in}}%
\pgfpathlineto{\pgfqpoint{4.470942in}{1.624339in}}%
\pgfpathlineto{\pgfqpoint{4.478946in}{1.635760in}}%
\pgfpathlineto{\pgfqpoint{4.486945in}{1.647275in}}%
\pgfpathclose%
\pgfusepath{fill}%
\end{pgfscope}%
\begin{pgfscope}%
\pgfpathrectangle{\pgfqpoint{1.150000in}{0.150000in}}{\pgfqpoint{5.700000in}{5.700000in}}%
\pgfusepath{clip}%
\pgfsetbuttcap%
\pgfsetroundjoin%
\definecolor{currentfill}{rgb}{0.283197,0.115680,0.436115}%
\pgfsetfillcolor{currentfill}%
\pgfsetfillopacity{0.700000}%
\pgfsetlinewidth{0.000000pt}%
\definecolor{currentstroke}{rgb}{0.000000,0.000000,0.000000}%
\pgfsetstrokecolor{currentstroke}%
\pgfsetdash{}{0pt}%
\pgfpathmoveto{\pgfqpoint{4.575798in}{1.697813in}}%
\pgfpathlineto{\pgfqpoint{4.590049in}{1.698962in}}%
\pgfpathlineto{\pgfqpoint{4.604310in}{1.700208in}}%
\pgfpathlineto{\pgfqpoint{4.618582in}{1.701551in}}%
\pgfpathlineto{\pgfqpoint{4.632865in}{1.702991in}}%
\pgfpathlineto{\pgfqpoint{4.624889in}{1.690322in}}%
\pgfpathlineto{\pgfqpoint{4.616909in}{1.677721in}}%
\pgfpathlineto{\pgfqpoint{4.608925in}{1.665193in}}%
\pgfpathlineto{\pgfqpoint{4.600937in}{1.652741in}}%
\pgfpathlineto{\pgfqpoint{4.586652in}{1.651720in}}%
\pgfpathlineto{\pgfqpoint{4.572378in}{1.650795in}}%
\pgfpathlineto{\pgfqpoint{4.558114in}{1.649967in}}%
\pgfpathlineto{\pgfqpoint{4.543860in}{1.649235in}}%
\pgfpathlineto{\pgfqpoint{4.551851in}{1.661263in}}%
\pgfpathlineto{\pgfqpoint{4.559837in}{1.673371in}}%
\pgfpathlineto{\pgfqpoint{4.567820in}{1.685556in}}%
\pgfpathlineto{\pgfqpoint{4.575798in}{1.697813in}}%
\pgfpathclose%
\pgfusepath{fill}%
\end{pgfscope}%
\begin{pgfscope}%
\pgfpathrectangle{\pgfqpoint{1.150000in}{0.150000in}}{\pgfqpoint{5.700000in}{5.700000in}}%
\pgfusepath{clip}%
\pgfsetbuttcap%
\pgfsetroundjoin%
\definecolor{currentfill}{rgb}{0.279566,0.067836,0.391917}%
\pgfsetfillcolor{currentfill}%
\pgfsetfillopacity{0.700000}%
\pgfsetlinewidth{0.000000pt}%
\definecolor{currentstroke}{rgb}{0.000000,0.000000,0.000000}%
\pgfsetstrokecolor{currentstroke}%
\pgfsetdash{}{0pt}%
\pgfpathmoveto{\pgfqpoint{4.398142in}{1.603119in}}%
\pgfpathlineto{\pgfqpoint{4.412323in}{1.602642in}}%
\pgfpathlineto{\pgfqpoint{4.426513in}{1.602262in}}%
\pgfpathlineto{\pgfqpoint{4.440713in}{1.601979in}}%
\pgfpathlineto{\pgfqpoint{4.454922in}{1.601792in}}%
\pgfpathlineto{\pgfqpoint{4.446905in}{1.590676in}}%
\pgfpathlineto{\pgfqpoint{4.438884in}{1.579669in}}%
\pgfpathlineto{\pgfqpoint{4.430858in}{1.568777in}}%
\pgfpathlineto{\pgfqpoint{4.422828in}{1.558004in}}%
\pgfpathlineto{\pgfqpoint{4.408613in}{1.558643in}}%
\pgfpathlineto{\pgfqpoint{4.394406in}{1.559379in}}%
\pgfpathlineto{\pgfqpoint{4.380209in}{1.560211in}}%
\pgfpathlineto{\pgfqpoint{4.366021in}{1.561140in}}%
\pgfpathlineto{\pgfqpoint{4.374059in}{1.571454in}}%
\pgfpathlineto{\pgfqpoint{4.382091in}{1.581891in}}%
\pgfpathlineto{\pgfqpoint{4.390119in}{1.592447in}}%
\pgfpathlineto{\pgfqpoint{4.398142in}{1.603119in}}%
\pgfpathclose%
\pgfusepath{fill}%
\end{pgfscope}%
\begin{pgfscope}%
\pgfpathrectangle{\pgfqpoint{1.150000in}{0.150000in}}{\pgfqpoint{5.700000in}{5.700000in}}%
\pgfusepath{clip}%
\pgfsetbuttcap%
\pgfsetroundjoin%
\definecolor{currentfill}{rgb}{0.281924,0.089666,0.412415}%
\pgfsetfillcolor{currentfill}%
\pgfsetfillopacity{0.700000}%
\pgfsetlinewidth{0.000000pt}%
\definecolor{currentstroke}{rgb}{0.000000,0.000000,0.000000}%
\pgfsetstrokecolor{currentstroke}%
\pgfsetdash{}{0pt}%
\pgfpathmoveto{\pgfqpoint{3.583619in}{1.638132in}}%
\pgfpathlineto{\pgfqpoint{3.597619in}{1.630253in}}%
\pgfpathlineto{\pgfqpoint{3.611623in}{1.622480in}}%
\pgfpathlineto{\pgfqpoint{3.625631in}{1.614813in}}%
\pgfpathlineto{\pgfqpoint{3.639643in}{1.607251in}}%
\pgfpathlineto{\pgfqpoint{3.631276in}{1.606198in}}%
\pgfpathlineto{\pgfqpoint{3.622899in}{1.605429in}}%
\pgfpathlineto{\pgfqpoint{3.614510in}{1.604948in}}%
\pgfpathlineto{\pgfqpoint{3.606110in}{1.604762in}}%
\pgfpathlineto{\pgfqpoint{3.592069in}{1.612889in}}%
\pgfpathlineto{\pgfqpoint{3.578031in}{1.621122in}}%
\pgfpathlineto{\pgfqpoint{3.563997in}{1.629461in}}%
\pgfpathlineto{\pgfqpoint{3.549966in}{1.637906in}}%
\pgfpathlineto{\pgfqpoint{3.558398in}{1.637519in}}%
\pgfpathlineto{\pgfqpoint{3.566817in}{1.637431in}}%
\pgfpathlineto{\pgfqpoint{3.575224in}{1.637638in}}%
\pgfpathlineto{\pgfqpoint{3.583619in}{1.638132in}}%
\pgfpathclose%
\pgfusepath{fill}%
\end{pgfscope}%
\begin{pgfscope}%
\pgfpathrectangle{\pgfqpoint{1.150000in}{0.150000in}}{\pgfqpoint{5.700000in}{5.700000in}}%
\pgfusepath{clip}%
\pgfsetbuttcap%
\pgfsetroundjoin%
\definecolor{currentfill}{rgb}{0.185783,0.704891,0.485273}%
\pgfsetfillcolor{currentfill}%
\pgfsetfillopacity{0.700000}%
\pgfsetlinewidth{0.000000pt}%
\definecolor{currentstroke}{rgb}{0.000000,0.000000,0.000000}%
\pgfsetstrokecolor{currentstroke}%
\pgfsetdash{}{0pt}%
\pgfpathmoveto{\pgfqpoint{2.087029in}{3.215555in}}%
\pgfpathlineto{\pgfqpoint{2.101374in}{3.192062in}}%
\pgfpathlineto{\pgfqpoint{2.115708in}{3.168780in}}%
\pgfpathlineto{\pgfqpoint{2.130030in}{3.145707in}}%
\pgfpathlineto{\pgfqpoint{2.144342in}{3.122841in}}%
\pgfpathlineto{\pgfqpoint{2.134617in}{3.138741in}}%
\pgfpathlineto{\pgfqpoint{2.124860in}{3.155104in}}%
\pgfpathlineto{\pgfqpoint{2.115071in}{3.171937in}}%
\pgfpathlineto{\pgfqpoint{2.105249in}{3.189249in}}%
\pgfpathlineto{\pgfqpoint{2.090862in}{3.212785in}}%
\pgfpathlineto{\pgfqpoint{2.076465in}{3.236530in}}%
\pgfpathlineto{\pgfqpoint{2.062055in}{3.260485in}}%
\pgfpathlineto{\pgfqpoint{2.047634in}{3.284653in}}%
\pgfpathlineto{\pgfqpoint{2.057533in}{3.266658in}}%
\pgfpathlineto{\pgfqpoint{2.067399in}{3.249148in}}%
\pgfpathlineto{\pgfqpoint{2.077230in}{3.232117in}}%
\pgfpathlineto{\pgfqpoint{2.087029in}{3.215555in}}%
\pgfpathclose%
\pgfusepath{fill}%
\end{pgfscope}%
\begin{pgfscope}%
\pgfpathrectangle{\pgfqpoint{1.150000in}{0.150000in}}{\pgfqpoint{5.700000in}{5.700000in}}%
\pgfusepath{clip}%
\pgfsetbuttcap%
\pgfsetroundjoin%
\definecolor{currentfill}{rgb}{0.272594,0.025563,0.353093}%
\pgfsetfillcolor{currentfill}%
\pgfsetfillopacity{0.700000}%
\pgfsetlinewidth{0.000000pt}%
\definecolor{currentstroke}{rgb}{0.000000,0.000000,0.000000}%
\pgfsetstrokecolor{currentstroke}%
\pgfsetdash{}{0pt}%
\pgfpathmoveto{\pgfqpoint{4.075333in}{1.527026in}}%
\pgfpathlineto{\pgfqpoint{4.089416in}{1.523579in}}%
\pgfpathlineto{\pgfqpoint{4.103506in}{1.520230in}}%
\pgfpathlineto{\pgfqpoint{4.117602in}{1.516980in}}%
\pgfpathlineto{\pgfqpoint{4.131706in}{1.513828in}}%
\pgfpathlineto{\pgfqpoint{4.123588in}{1.506329in}}%
\pgfpathlineto{\pgfqpoint{4.115463in}{1.499015in}}%
\pgfpathlineto{\pgfqpoint{4.107332in}{1.491889in}}%
\pgfpathlineto{\pgfqpoint{4.099194in}{1.484958in}}%
\pgfpathlineto{\pgfqpoint{4.085076in}{1.488615in}}%
\pgfpathlineto{\pgfqpoint{4.070964in}{1.492370in}}%
\pgfpathlineto{\pgfqpoint{4.056859in}{1.496224in}}%
\pgfpathlineto{\pgfqpoint{4.042761in}{1.500176in}}%
\pgfpathlineto{\pgfqpoint{4.050915in}{1.506595in}}%
\pgfpathlineto{\pgfqpoint{4.059061in}{1.513213in}}%
\pgfpathlineto{\pgfqpoint{4.067201in}{1.520025in}}%
\pgfpathlineto{\pgfqpoint{4.075333in}{1.527026in}}%
\pgfpathclose%
\pgfusepath{fill}%
\end{pgfscope}%
\begin{pgfscope}%
\pgfpathrectangle{\pgfqpoint{1.150000in}{0.150000in}}{\pgfqpoint{5.700000in}{5.700000in}}%
\pgfusepath{clip}%
\pgfsetbuttcap%
\pgfsetroundjoin%
\definecolor{currentfill}{rgb}{0.190631,0.407061,0.556089}%
\pgfsetfillcolor{currentfill}%
\pgfsetfillopacity{0.700000}%
\pgfsetlinewidth{0.000000pt}%
\definecolor{currentstroke}{rgb}{0.000000,0.000000,0.000000}%
\pgfsetstrokecolor{currentstroke}%
\pgfsetdash{}{0pt}%
\pgfpathmoveto{\pgfqpoint{5.295096in}{2.366653in}}%
\pgfpathlineto{\pgfqpoint{5.309721in}{2.373755in}}%
\pgfpathlineto{\pgfqpoint{5.324361in}{2.380957in}}%
\pgfpathlineto{\pgfqpoint{5.339017in}{2.388258in}}%
\pgfpathlineto{\pgfqpoint{5.353688in}{2.395660in}}%
\pgfpathlineto{\pgfqpoint{5.345880in}{2.381349in}}%
\pgfpathlineto{\pgfqpoint{5.338066in}{2.366954in}}%
\pgfpathlineto{\pgfqpoint{5.330246in}{2.352476in}}%
\pgfpathlineto{\pgfqpoint{5.322421in}{2.337918in}}%
\pgfpathlineto{\pgfqpoint{5.307754in}{2.330758in}}%
\pgfpathlineto{\pgfqpoint{5.293102in}{2.323697in}}%
\pgfpathlineto{\pgfqpoint{5.278465in}{2.316736in}}%
\pgfpathlineto{\pgfqpoint{5.263843in}{2.309875in}}%
\pgfpathlineto{\pgfqpoint{5.271665in}{2.324185in}}%
\pgfpathlineto{\pgfqpoint{5.279481in}{2.338419in}}%
\pgfpathlineto{\pgfqpoint{5.287291in}{2.352576in}}%
\pgfpathlineto{\pgfqpoint{5.295096in}{2.366653in}}%
\pgfpathclose%
\pgfusepath{fill}%
\end{pgfscope}%
\begin{pgfscope}%
\pgfpathrectangle{\pgfqpoint{1.150000in}{0.150000in}}{\pgfqpoint{5.700000in}{5.700000in}}%
\pgfusepath{clip}%
\pgfsetbuttcap%
\pgfsetroundjoin%
\definecolor{currentfill}{rgb}{0.276022,0.044167,0.370164}%
\pgfsetfillcolor{currentfill}%
\pgfsetfillopacity{0.700000}%
\pgfsetlinewidth{0.000000pt}%
\definecolor{currentstroke}{rgb}{0.000000,0.000000,0.000000}%
\pgfsetstrokecolor{currentstroke}%
\pgfsetdash{}{0pt}%
\pgfpathmoveto{\pgfqpoint{3.785030in}{1.561795in}}%
\pgfpathlineto{\pgfqpoint{3.799055in}{1.555712in}}%
\pgfpathlineto{\pgfqpoint{3.813085in}{1.549731in}}%
\pgfpathlineto{\pgfqpoint{3.827120in}{1.543852in}}%
\pgfpathlineto{\pgfqpoint{3.841160in}{1.538075in}}%
\pgfpathlineto{\pgfqpoint{3.832910in}{1.534338in}}%
\pgfpathlineto{\pgfqpoint{3.824651in}{1.530845in}}%
\pgfpathlineto{\pgfqpoint{3.816382in}{1.527604in}}%
\pgfpathlineto{\pgfqpoint{3.808104in}{1.524619in}}%
\pgfpathlineto{\pgfqpoint{3.794041in}{1.530940in}}%
\pgfpathlineto{\pgfqpoint{3.779982in}{1.537363in}}%
\pgfpathlineto{\pgfqpoint{3.765928in}{1.543887in}}%
\pgfpathlineto{\pgfqpoint{3.751879in}{1.550514in}}%
\pgfpathlineto{\pgfqpoint{3.760182in}{1.552948in}}%
\pgfpathlineto{\pgfqpoint{3.768474in}{1.555643in}}%
\pgfpathlineto{\pgfqpoint{3.776757in}{1.558594in}}%
\pgfpathlineto{\pgfqpoint{3.785030in}{1.561795in}}%
\pgfpathclose%
\pgfusepath{fill}%
\end{pgfscope}%
\begin{pgfscope}%
\pgfpathrectangle{\pgfqpoint{1.150000in}{0.150000in}}{\pgfqpoint{5.700000in}{5.700000in}}%
\pgfusepath{clip}%
\pgfsetbuttcap%
\pgfsetroundjoin%
\definecolor{currentfill}{rgb}{0.280255,0.165693,0.476498}%
\pgfsetfillcolor{currentfill}%
\pgfsetfillopacity{0.700000}%
\pgfsetlinewidth{0.000000pt}%
\definecolor{currentstroke}{rgb}{0.000000,0.000000,0.000000}%
\pgfsetstrokecolor{currentstroke}%
\pgfsetdash{}{0pt}%
\pgfpathmoveto{\pgfqpoint{3.325829in}{1.787820in}}%
\pgfpathlineto{\pgfqpoint{3.339822in}{1.777618in}}%
\pgfpathlineto{\pgfqpoint{3.353816in}{1.767529in}}%
\pgfpathlineto{\pgfqpoint{3.367813in}{1.757553in}}%
\pgfpathlineto{\pgfqpoint{3.381811in}{1.747689in}}%
\pgfpathlineto{\pgfqpoint{3.373265in}{1.750125in}}%
\pgfpathlineto{\pgfqpoint{3.364704in}{1.752889in}}%
\pgfpathlineto{\pgfqpoint{3.356129in}{1.755988in}}%
\pgfpathlineto{\pgfqpoint{3.347538in}{1.759427in}}%
\pgfpathlineto{\pgfqpoint{3.333503in}{1.769883in}}%
\pgfpathlineto{\pgfqpoint{3.319469in}{1.780451in}}%
\pgfpathlineto{\pgfqpoint{3.305436in}{1.791132in}}%
\pgfpathlineto{\pgfqpoint{3.291405in}{1.801926in}}%
\pgfpathlineto{\pgfqpoint{3.300034in}{1.797887in}}%
\pgfpathlineto{\pgfqpoint{3.308648in}{1.794194in}}%
\pgfpathlineto{\pgfqpoint{3.317246in}{1.790840in}}%
\pgfpathlineto{\pgfqpoint{3.325829in}{1.787820in}}%
\pgfpathclose%
\pgfusepath{fill}%
\end{pgfscope}%
\begin{pgfscope}%
\pgfpathrectangle{\pgfqpoint{1.150000in}{0.150000in}}{\pgfqpoint{5.700000in}{5.700000in}}%
\pgfusepath{clip}%
\pgfsetbuttcap%
\pgfsetroundjoin%
\definecolor{currentfill}{rgb}{0.282290,0.145912,0.461510}%
\pgfsetfillcolor{currentfill}%
\pgfsetfillopacity{0.700000}%
\pgfsetlinewidth{0.000000pt}%
\definecolor{currentstroke}{rgb}{0.000000,0.000000,0.000000}%
\pgfsetstrokecolor{currentstroke}%
\pgfsetdash{}{0pt}%
\pgfpathmoveto{\pgfqpoint{4.664728in}{1.754268in}}%
\pgfpathlineto{\pgfqpoint{4.679020in}{1.756205in}}%
\pgfpathlineto{\pgfqpoint{4.693323in}{1.758238in}}%
\pgfpathlineto{\pgfqpoint{4.707637in}{1.760369in}}%
\pgfpathlineto{\pgfqpoint{4.721962in}{1.762596in}}%
\pgfpathlineto{\pgfqpoint{4.714003in}{1.749299in}}%
\pgfpathlineto{\pgfqpoint{4.706041in}{1.736051in}}%
\pgfpathlineto{\pgfqpoint{4.698074in}{1.722855in}}%
\pgfpathlineto{\pgfqpoint{4.690104in}{1.709714in}}%
\pgfpathlineto{\pgfqpoint{4.675777in}{1.707888in}}%
\pgfpathlineto{\pgfqpoint{4.661462in}{1.706159in}}%
\pgfpathlineto{\pgfqpoint{4.647158in}{1.704527in}}%
\pgfpathlineto{\pgfqpoint{4.632865in}{1.702991in}}%
\pgfpathlineto{\pgfqpoint{4.640836in}{1.715723in}}%
\pgfpathlineto{\pgfqpoint{4.648804in}{1.728516in}}%
\pgfpathlineto{\pgfqpoint{4.656768in}{1.741366in}}%
\pgfpathlineto{\pgfqpoint{4.664728in}{1.754268in}}%
\pgfpathclose%
\pgfusepath{fill}%
\end{pgfscope}%
\begin{pgfscope}%
\pgfpathrectangle{\pgfqpoint{1.150000in}{0.150000in}}{\pgfqpoint{5.700000in}{5.700000in}}%
\pgfusepath{clip}%
\pgfsetbuttcap%
\pgfsetroundjoin%
\definecolor{currentfill}{rgb}{0.277018,0.050344,0.375715}%
\pgfsetfillcolor{currentfill}%
\pgfsetfillopacity{0.700000}%
\pgfsetlinewidth{0.000000pt}%
\definecolor{currentstroke}{rgb}{0.000000,0.000000,0.000000}%
\pgfsetstrokecolor{currentstroke}%
\pgfsetdash{}{0pt}%
\pgfpathmoveto{\pgfqpoint{4.309357in}{1.565824in}}%
\pgfpathlineto{\pgfqpoint{4.323510in}{1.564508in}}%
\pgfpathlineto{\pgfqpoint{4.337672in}{1.563288in}}%
\pgfpathlineto{\pgfqpoint{4.351842in}{1.562165in}}%
\pgfpathlineto{\pgfqpoint{4.366021in}{1.561140in}}%
\pgfpathlineto{\pgfqpoint{4.357979in}{1.550954in}}%
\pgfpathlineto{\pgfqpoint{4.349932in}{1.540900in}}%
\pgfpathlineto{\pgfqpoint{4.341881in}{1.530984in}}%
\pgfpathlineto{\pgfqpoint{4.333824in}{1.521209in}}%
\pgfpathlineto{\pgfqpoint{4.319636in}{1.522704in}}%
\pgfpathlineto{\pgfqpoint{4.305457in}{1.524297in}}%
\pgfpathlineto{\pgfqpoint{4.291287in}{1.525986in}}%
\pgfpathlineto{\pgfqpoint{4.277125in}{1.527772in}}%
\pgfpathlineto{\pgfqpoint{4.285191in}{1.537071in}}%
\pgfpathlineto{\pgfqpoint{4.293251in}{1.546515in}}%
\pgfpathlineto{\pgfqpoint{4.301307in}{1.556101in}}%
\pgfpathlineto{\pgfqpoint{4.309357in}{1.565824in}}%
\pgfpathclose%
\pgfusepath{fill}%
\end{pgfscope}%
\begin{pgfscope}%
\pgfpathrectangle{\pgfqpoint{1.150000in}{0.150000in}}{\pgfqpoint{5.700000in}{5.700000in}}%
\pgfusepath{clip}%
\pgfsetbuttcap%
\pgfsetroundjoin%
\definecolor{currentfill}{rgb}{0.253935,0.265254,0.529983}%
\pgfsetfillcolor{currentfill}%
\pgfsetfillopacity{0.700000}%
\pgfsetlinewidth{0.000000pt}%
\definecolor{currentstroke}{rgb}{0.000000,0.000000,0.000000}%
\pgfsetstrokecolor{currentstroke}%
\pgfsetdash{}{0pt}%
\pgfpathmoveto{\pgfqpoint{4.963816in}{2.011741in}}%
\pgfpathlineto{\pgfqpoint{4.978256in}{2.016273in}}%
\pgfpathlineto{\pgfqpoint{4.992709in}{2.020903in}}%
\pgfpathlineto{\pgfqpoint{5.007176in}{2.025630in}}%
\pgfpathlineto{\pgfqpoint{5.021656in}{2.030456in}}%
\pgfpathlineto{\pgfqpoint{5.013757in}{2.015891in}}%
\pgfpathlineto{\pgfqpoint{5.005853in}{2.001306in}}%
\pgfpathlineto{\pgfqpoint{4.997946in}{1.986705in}}%
\pgfpathlineto{\pgfqpoint{4.990034in}{1.972091in}}%
\pgfpathlineto{\pgfqpoint{4.975557in}{1.967598in}}%
\pgfpathlineto{\pgfqpoint{4.961094in}{1.963202in}}%
\pgfpathlineto{\pgfqpoint{4.946643in}{1.958903in}}%
\pgfpathlineto{\pgfqpoint{4.932205in}{1.954703in}}%
\pgfpathlineto{\pgfqpoint{4.940114in}{1.968978in}}%
\pgfpathlineto{\pgfqpoint{4.948019in}{1.983245in}}%
\pgfpathlineto{\pgfqpoint{4.955920in}{1.997500in}}%
\pgfpathlineto{\pgfqpoint{4.963816in}{2.011741in}}%
\pgfpathclose%
\pgfusepath{fill}%
\end{pgfscope}%
\begin{pgfscope}%
\pgfpathrectangle{\pgfqpoint{1.150000in}{0.150000in}}{\pgfqpoint{5.700000in}{5.700000in}}%
\pgfusepath{clip}%
\pgfsetbuttcap%
\pgfsetroundjoin%
\definecolor{currentfill}{rgb}{0.183898,0.422383,0.556944}%
\pgfsetfillcolor{currentfill}%
\pgfsetfillopacity{0.700000}%
\pgfsetlinewidth{0.000000pt}%
\definecolor{currentstroke}{rgb}{0.000000,0.000000,0.000000}%
\pgfsetstrokecolor{currentstroke}%
\pgfsetdash{}{0pt}%
\pgfpathmoveto{\pgfqpoint{2.673651in}{2.398447in}}%
\pgfpathlineto{\pgfqpoint{2.687736in}{2.382019in}}%
\pgfpathlineto{\pgfqpoint{2.701818in}{2.365736in}}%
\pgfpathlineto{\pgfqpoint{2.715895in}{2.349599in}}%
\pgfpathlineto{\pgfqpoint{2.729969in}{2.333605in}}%
\pgfpathlineto{\pgfqpoint{2.720841in}{2.344176in}}%
\pgfpathlineto{\pgfqpoint{2.711690in}{2.355166in}}%
\pgfpathlineto{\pgfqpoint{2.702514in}{2.366582in}}%
\pgfpathlineto{\pgfqpoint{2.693314in}{2.378430in}}%
\pgfpathlineto{\pgfqpoint{2.679182in}{2.395067in}}%
\pgfpathlineto{\pgfqpoint{2.665046in}{2.411848in}}%
\pgfpathlineto{\pgfqpoint{2.650907in}{2.428774in}}%
\pgfpathlineto{\pgfqpoint{2.636762in}{2.445848in}}%
\pgfpathlineto{\pgfqpoint{2.646023in}{2.433346in}}%
\pgfpathlineto{\pgfqpoint{2.655257in}{2.421283in}}%
\pgfpathlineto{\pgfqpoint{2.664467in}{2.409652in}}%
\pgfpathlineto{\pgfqpoint{2.673651in}{2.398447in}}%
\pgfpathclose%
\pgfusepath{fill}%
\end{pgfscope}%
\begin{pgfscope}%
\pgfpathrectangle{\pgfqpoint{1.150000in}{0.150000in}}{\pgfqpoint{5.700000in}{5.700000in}}%
\pgfusepath{clip}%
\pgfsetbuttcap%
\pgfsetroundjoin%
\definecolor{currentfill}{rgb}{0.194100,0.399323,0.555565}%
\pgfsetfillcolor{currentfill}%
\pgfsetfillopacity{0.700000}%
\pgfsetlinewidth{0.000000pt}%
\definecolor{currentstroke}{rgb}{0.000000,0.000000,0.000000}%
\pgfsetstrokecolor{currentstroke}%
\pgfsetdash{}{0pt}%
\pgfpathmoveto{\pgfqpoint{2.729969in}{2.333605in}}%
\pgfpathlineto{\pgfqpoint{2.744039in}{2.317753in}}%
\pgfpathlineto{\pgfqpoint{2.758106in}{2.302044in}}%
\pgfpathlineto{\pgfqpoint{2.772170in}{2.286475in}}%
\pgfpathlineto{\pgfqpoint{2.786231in}{2.271046in}}%
\pgfpathlineto{\pgfqpoint{2.777159in}{2.280987in}}%
\pgfpathlineto{\pgfqpoint{2.768063in}{2.291340in}}%
\pgfpathlineto{\pgfqpoint{2.758945in}{2.302113in}}%
\pgfpathlineto{\pgfqpoint{2.749802in}{2.313312in}}%
\pgfpathlineto{\pgfqpoint{2.735686in}{2.329379in}}%
\pgfpathlineto{\pgfqpoint{2.721565in}{2.345588in}}%
\pgfpathlineto{\pgfqpoint{2.707441in}{2.361938in}}%
\pgfpathlineto{\pgfqpoint{2.693314in}{2.378430in}}%
\pgfpathlineto{\pgfqpoint{2.702514in}{2.366582in}}%
\pgfpathlineto{\pgfqpoint{2.711690in}{2.355166in}}%
\pgfpathlineto{\pgfqpoint{2.720841in}{2.344176in}}%
\pgfpathlineto{\pgfqpoint{2.729969in}{2.333605in}}%
\pgfpathclose%
\pgfusepath{fill}%
\end{pgfscope}%
\begin{pgfscope}%
\pgfpathrectangle{\pgfqpoint{1.150000in}{0.150000in}}{\pgfqpoint{5.700000in}{5.700000in}}%
\pgfusepath{clip}%
\pgfsetbuttcap%
\pgfsetroundjoin%
\definecolor{currentfill}{rgb}{0.174274,0.445044,0.557792}%
\pgfsetfillcolor{currentfill}%
\pgfsetfillopacity{0.700000}%
\pgfsetlinewidth{0.000000pt}%
\definecolor{currentstroke}{rgb}{0.000000,0.000000,0.000000}%
\pgfsetstrokecolor{currentstroke}%
\pgfsetdash{}{0pt}%
\pgfpathmoveto{\pgfqpoint{2.617269in}{2.465638in}}%
\pgfpathlineto{\pgfqpoint{2.631371in}{2.448616in}}%
\pgfpathlineto{\pgfqpoint{2.645469in}{2.431744in}}%
\pgfpathlineto{\pgfqpoint{2.659562in}{2.415022in}}%
\pgfpathlineto{\pgfqpoint{2.673651in}{2.398447in}}%
\pgfpathlineto{\pgfqpoint{2.664467in}{2.409652in}}%
\pgfpathlineto{\pgfqpoint{2.655257in}{2.421283in}}%
\pgfpathlineto{\pgfqpoint{2.646023in}{2.433346in}}%
\pgfpathlineto{\pgfqpoint{2.636762in}{2.445848in}}%
\pgfpathlineto{\pgfqpoint{2.622614in}{2.463069in}}%
\pgfpathlineto{\pgfqpoint{2.608461in}{2.480438in}}%
\pgfpathlineto{\pgfqpoint{2.594303in}{2.497958in}}%
\pgfpathlineto{\pgfqpoint{2.580140in}{2.515629in}}%
\pgfpathlineto{\pgfqpoint{2.589461in}{2.502470in}}%
\pgfpathlineto{\pgfqpoint{2.598757in}{2.489756in}}%
\pgfpathlineto{\pgfqpoint{2.608026in}{2.477481in}}%
\pgfpathlineto{\pgfqpoint{2.617269in}{2.465638in}}%
\pgfpathclose%
\pgfusepath{fill}%
\end{pgfscope}%
\begin{pgfscope}%
\pgfpathrectangle{\pgfqpoint{1.150000in}{0.150000in}}{\pgfqpoint{5.700000in}{5.700000in}}%
\pgfusepath{clip}%
\pgfsetbuttcap%
\pgfsetroundjoin%
\definecolor{currentfill}{rgb}{0.204903,0.375746,0.553533}%
\pgfsetfillcolor{currentfill}%
\pgfsetfillopacity{0.700000}%
\pgfsetlinewidth{0.000000pt}%
\definecolor{currentstroke}{rgb}{0.000000,0.000000,0.000000}%
\pgfsetstrokecolor{currentstroke}%
\pgfsetdash{}{0pt}%
\pgfpathmoveto{\pgfqpoint{2.786231in}{2.271046in}}%
\pgfpathlineto{\pgfqpoint{2.800288in}{2.255756in}}%
\pgfpathlineto{\pgfqpoint{2.814343in}{2.240604in}}%
\pgfpathlineto{\pgfqpoint{2.828396in}{2.225589in}}%
\pgfpathlineto{\pgfqpoint{2.842445in}{2.210711in}}%
\pgfpathlineto{\pgfqpoint{2.833427in}{2.220024in}}%
\pgfpathlineto{\pgfqpoint{2.824387in}{2.229744in}}%
\pgfpathlineto{\pgfqpoint{2.815323in}{2.239877in}}%
\pgfpathlineto{\pgfqpoint{2.806237in}{2.250430in}}%
\pgfpathlineto{\pgfqpoint{2.792133in}{2.265944in}}%
\pgfpathlineto{\pgfqpoint{2.778026in}{2.281595in}}%
\pgfpathlineto{\pgfqpoint{2.763916in}{2.297384in}}%
\pgfpathlineto{\pgfqpoint{2.749802in}{2.313312in}}%
\pgfpathlineto{\pgfqpoint{2.758945in}{2.302113in}}%
\pgfpathlineto{\pgfqpoint{2.768063in}{2.291340in}}%
\pgfpathlineto{\pgfqpoint{2.777159in}{2.280987in}}%
\pgfpathlineto{\pgfqpoint{2.786231in}{2.271046in}}%
\pgfpathclose%
\pgfusepath{fill}%
\end{pgfscope}%
\begin{pgfscope}%
\pgfpathrectangle{\pgfqpoint{1.150000in}{0.150000in}}{\pgfqpoint{5.700000in}{5.700000in}}%
\pgfusepath{clip}%
\pgfsetbuttcap%
\pgfsetroundjoin%
\definecolor{currentfill}{rgb}{0.214298,0.355619,0.551184}%
\pgfsetfillcolor{currentfill}%
\pgfsetfillopacity{0.700000}%
\pgfsetlinewidth{0.000000pt}%
\definecolor{currentstroke}{rgb}{0.000000,0.000000,0.000000}%
\pgfsetstrokecolor{currentstroke}%
\pgfsetdash{}{0pt}%
\pgfpathmoveto{\pgfqpoint{5.174177in}{2.226511in}}%
\pgfpathlineto{\pgfqpoint{5.188736in}{2.232716in}}%
\pgfpathlineto{\pgfqpoint{5.203310in}{2.239020in}}%
\pgfpathlineto{\pgfqpoint{5.217898in}{2.245423in}}%
\pgfpathlineto{\pgfqpoint{5.232501in}{2.251926in}}%
\pgfpathlineto{\pgfqpoint{5.224652in}{2.237273in}}%
\pgfpathlineto{\pgfqpoint{5.216798in}{2.222558in}}%
\pgfpathlineto{\pgfqpoint{5.208939in}{2.207784in}}%
\pgfpathlineto{\pgfqpoint{5.201074in}{2.192953in}}%
\pgfpathlineto{\pgfqpoint{5.186476in}{2.186729in}}%
\pgfpathlineto{\pgfqpoint{5.171891in}{2.180604in}}%
\pgfpathlineto{\pgfqpoint{5.157321in}{2.174578in}}%
\pgfpathlineto{\pgfqpoint{5.142765in}{2.168651in}}%
\pgfpathlineto{\pgfqpoint{5.150626in}{2.183196in}}%
\pgfpathlineto{\pgfqpoint{5.158481in}{2.197690in}}%
\pgfpathlineto{\pgfqpoint{5.166332in}{2.212129in}}%
\pgfpathlineto{\pgfqpoint{5.174177in}{2.226511in}}%
\pgfpathclose%
\pgfusepath{fill}%
\end{pgfscope}%
\begin{pgfscope}%
\pgfpathrectangle{\pgfqpoint{1.150000in}{0.150000in}}{\pgfqpoint{5.700000in}{5.700000in}}%
\pgfusepath{clip}%
\pgfsetbuttcap%
\pgfsetroundjoin%
\definecolor{currentfill}{rgb}{0.278826,0.175490,0.483397}%
\pgfsetfillcolor{currentfill}%
\pgfsetfillopacity{0.700000}%
\pgfsetlinewidth{0.000000pt}%
\definecolor{currentstroke}{rgb}{0.000000,0.000000,0.000000}%
\pgfsetstrokecolor{currentstroke}%
\pgfsetdash{}{0pt}%
\pgfpathmoveto{\pgfqpoint{4.753759in}{1.816191in}}%
\pgfpathlineto{\pgfqpoint{4.768095in}{1.818898in}}%
\pgfpathlineto{\pgfqpoint{4.782444in}{1.821703in}}%
\pgfpathlineto{\pgfqpoint{4.796804in}{1.824604in}}%
\pgfpathlineto{\pgfqpoint{4.811177in}{1.827603in}}%
\pgfpathlineto{\pgfqpoint{4.803233in}{1.813773in}}%
\pgfpathlineto{\pgfqpoint{4.795286in}{1.799972in}}%
\pgfpathlineto{\pgfqpoint{4.787334in}{1.786203in}}%
\pgfpathlineto{\pgfqpoint{4.779379in}{1.772471in}}%
\pgfpathlineto{\pgfqpoint{4.765008in}{1.769858in}}%
\pgfpathlineto{\pgfqpoint{4.750648in}{1.767340in}}%
\pgfpathlineto{\pgfqpoint{4.736299in}{1.764920in}}%
\pgfpathlineto{\pgfqpoint{4.721962in}{1.762596in}}%
\pgfpathlineto{\pgfqpoint{4.729917in}{1.775937in}}%
\pgfpathlineto{\pgfqpoint{4.737868in}{1.789319in}}%
\pgfpathlineto{\pgfqpoint{4.745816in}{1.802738in}}%
\pgfpathlineto{\pgfqpoint{4.753759in}{1.816191in}}%
\pgfpathclose%
\pgfusepath{fill}%
\end{pgfscope}%
\begin{pgfscope}%
\pgfpathrectangle{\pgfqpoint{1.150000in}{0.150000in}}{\pgfqpoint{5.700000in}{5.700000in}}%
\pgfusepath{clip}%
\pgfsetbuttcap%
\pgfsetroundjoin%
\definecolor{currentfill}{rgb}{0.163625,0.471133,0.558148}%
\pgfsetfillcolor{currentfill}%
\pgfsetfillopacity{0.700000}%
\pgfsetlinewidth{0.000000pt}%
\definecolor{currentstroke}{rgb}{0.000000,0.000000,0.000000}%
\pgfsetstrokecolor{currentstroke}%
\pgfsetdash{}{0pt}%
\pgfpathmoveto{\pgfqpoint{2.560813in}{2.535247in}}%
\pgfpathlineto{\pgfqpoint{2.574935in}{2.517614in}}%
\pgfpathlineto{\pgfqpoint{2.589051in}{2.500135in}}%
\pgfpathlineto{\pgfqpoint{2.603163in}{2.482810in}}%
\pgfpathlineto{\pgfqpoint{2.617269in}{2.465638in}}%
\pgfpathlineto{\pgfqpoint{2.608026in}{2.477481in}}%
\pgfpathlineto{\pgfqpoint{2.598757in}{2.489756in}}%
\pgfpathlineto{\pgfqpoint{2.589461in}{2.502470in}}%
\pgfpathlineto{\pgfqpoint{2.580140in}{2.515629in}}%
\pgfpathlineto{\pgfqpoint{2.565972in}{2.533452in}}%
\pgfpathlineto{\pgfqpoint{2.551798in}{2.551428in}}%
\pgfpathlineto{\pgfqpoint{2.537620in}{2.569559in}}%
\pgfpathlineto{\pgfqpoint{2.523436in}{2.587846in}}%
\pgfpathlineto{\pgfqpoint{2.532821in}{2.574025in}}%
\pgfpathlineto{\pgfqpoint{2.542179in}{2.560656in}}%
\pgfpathlineto{\pgfqpoint{2.551509in}{2.547733in}}%
\pgfpathlineto{\pgfqpoint{2.560813in}{2.535247in}}%
\pgfpathclose%
\pgfusepath{fill}%
\end{pgfscope}%
\begin{pgfscope}%
\pgfpathrectangle{\pgfqpoint{1.150000in}{0.150000in}}{\pgfqpoint{5.700000in}{5.700000in}}%
\pgfusepath{clip}%
\pgfsetbuttcap%
\pgfsetroundjoin%
\definecolor{currentfill}{rgb}{0.216210,0.351535,0.550627}%
\pgfsetfillcolor{currentfill}%
\pgfsetfillopacity{0.700000}%
\pgfsetlinewidth{0.000000pt}%
\definecolor{currentstroke}{rgb}{0.000000,0.000000,0.000000}%
\pgfsetstrokecolor{currentstroke}%
\pgfsetdash{}{0pt}%
\pgfpathmoveto{\pgfqpoint{2.842445in}{2.210711in}}%
\pgfpathlineto{\pgfqpoint{2.856493in}{2.195968in}}%
\pgfpathlineto{\pgfqpoint{2.870537in}{2.181359in}}%
\pgfpathlineto{\pgfqpoint{2.884580in}{2.166884in}}%
\pgfpathlineto{\pgfqpoint{2.898621in}{2.152541in}}%
\pgfpathlineto{\pgfqpoint{2.889655in}{2.161231in}}%
\pgfpathlineto{\pgfqpoint{2.880667in}{2.170320in}}%
\pgfpathlineto{\pgfqpoint{2.871658in}{2.179816in}}%
\pgfpathlineto{\pgfqpoint{2.862627in}{2.189726in}}%
\pgfpathlineto{\pgfqpoint{2.848533in}{2.204701in}}%
\pgfpathlineto{\pgfqpoint{2.834437in}{2.219809in}}%
\pgfpathlineto{\pgfqpoint{2.820339in}{2.235052in}}%
\pgfpathlineto{\pgfqpoint{2.806237in}{2.250430in}}%
\pgfpathlineto{\pgfqpoint{2.815323in}{2.239877in}}%
\pgfpathlineto{\pgfqpoint{2.824387in}{2.229744in}}%
\pgfpathlineto{\pgfqpoint{2.833427in}{2.220024in}}%
\pgfpathlineto{\pgfqpoint{2.842445in}{2.210711in}}%
\pgfpathclose%
\pgfusepath{fill}%
\end{pgfscope}%
\begin{pgfscope}%
\pgfpathrectangle{\pgfqpoint{1.150000in}{0.150000in}}{\pgfqpoint{5.700000in}{5.700000in}}%
\pgfusepath{clip}%
\pgfsetbuttcap%
\pgfsetroundjoin%
\definecolor{currentfill}{rgb}{0.154815,0.493313,0.557840}%
\pgfsetfillcolor{currentfill}%
\pgfsetfillopacity{0.700000}%
\pgfsetlinewidth{0.000000pt}%
\definecolor{currentstroke}{rgb}{0.000000,0.000000,0.000000}%
\pgfsetstrokecolor{currentstroke}%
\pgfsetdash{}{0pt}%
\pgfpathmoveto{\pgfqpoint{5.505764in}{2.593148in}}%
\pgfpathlineto{\pgfqpoint{5.520524in}{2.601662in}}%
\pgfpathlineto{\pgfqpoint{5.535300in}{2.610277in}}%
\pgfpathlineto{\pgfqpoint{5.550092in}{2.618993in}}%
\pgfpathlineto{\pgfqpoint{5.564901in}{2.627811in}}%
\pgfpathlineto{\pgfqpoint{5.557167in}{2.614306in}}%
\pgfpathlineto{\pgfqpoint{5.549425in}{2.600686in}}%
\pgfpathlineto{\pgfqpoint{5.541675in}{2.586952in}}%
\pgfpathlineto{\pgfqpoint{5.533919in}{2.573105in}}%
\pgfpathlineto{\pgfqpoint{5.519113in}{2.564472in}}%
\pgfpathlineto{\pgfqpoint{5.504324in}{2.555941in}}%
\pgfpathlineto{\pgfqpoint{5.489550in}{2.547510in}}%
\pgfpathlineto{\pgfqpoint{5.474793in}{2.539181in}}%
\pgfpathlineto{\pgfqpoint{5.482547in}{2.552836in}}%
\pgfpathlineto{\pgfqpoint{5.490293in}{2.566383in}}%
\pgfpathlineto{\pgfqpoint{5.498032in}{2.579821in}}%
\pgfpathlineto{\pgfqpoint{5.505764in}{2.593148in}}%
\pgfpathclose%
\pgfusepath{fill}%
\end{pgfscope}%
\begin{pgfscope}%
\pgfpathrectangle{\pgfqpoint{1.150000in}{0.150000in}}{\pgfqpoint{5.700000in}{5.700000in}}%
\pgfusepath{clip}%
\pgfsetbuttcap%
\pgfsetroundjoin%
\definecolor{currentfill}{rgb}{0.273809,0.031497,0.358853}%
\pgfsetfillcolor{currentfill}%
\pgfsetfillopacity{0.700000}%
\pgfsetlinewidth{0.000000pt}%
\definecolor{currentstroke}{rgb}{0.000000,0.000000,0.000000}%
\pgfsetstrokecolor{currentstroke}%
\pgfsetdash{}{0pt}%
\pgfpathmoveto{\pgfqpoint{4.220558in}{1.535889in}}%
\pgfpathlineto{\pgfqpoint{4.234688in}{1.533714in}}%
\pgfpathlineto{\pgfqpoint{4.248825in}{1.531636in}}%
\pgfpathlineto{\pgfqpoint{4.262971in}{1.529655in}}%
\pgfpathlineto{\pgfqpoint{4.277125in}{1.527772in}}%
\pgfpathlineto{\pgfqpoint{4.269053in}{1.518624in}}%
\pgfpathlineto{\pgfqpoint{4.260977in}{1.509632in}}%
\pgfpathlineto{\pgfqpoint{4.252895in}{1.500801in}}%
\pgfpathlineto{\pgfqpoint{4.244807in}{1.492134in}}%
\pgfpathlineto{\pgfqpoint{4.230643in}{1.494505in}}%
\pgfpathlineto{\pgfqpoint{4.216486in}{1.496973in}}%
\pgfpathlineto{\pgfqpoint{4.202337in}{1.499538in}}%
\pgfpathlineto{\pgfqpoint{4.188196in}{1.502201in}}%
\pgfpathlineto{\pgfqpoint{4.196295in}{1.510373in}}%
\pgfpathlineto{\pgfqpoint{4.204388in}{1.518715in}}%
\pgfpathlineto{\pgfqpoint{4.212476in}{1.527222in}}%
\pgfpathlineto{\pgfqpoint{4.220558in}{1.535889in}}%
\pgfpathclose%
\pgfusepath{fill}%
\end{pgfscope}%
\begin{pgfscope}%
\pgfpathrectangle{\pgfqpoint{1.150000in}{0.150000in}}{\pgfqpoint{5.700000in}{5.700000in}}%
\pgfusepath{clip}%
\pgfsetbuttcap%
\pgfsetroundjoin%
\definecolor{currentfill}{rgb}{0.153364,0.497000,0.557724}%
\pgfsetfillcolor{currentfill}%
\pgfsetfillopacity{0.700000}%
\pgfsetlinewidth{0.000000pt}%
\definecolor{currentstroke}{rgb}{0.000000,0.000000,0.000000}%
\pgfsetstrokecolor{currentstroke}%
\pgfsetdash{}{0pt}%
\pgfpathmoveto{\pgfqpoint{2.504274in}{2.607349in}}%
\pgfpathlineto{\pgfqpoint{2.518417in}{2.589085in}}%
\pgfpathlineto{\pgfqpoint{2.532555in}{2.570982in}}%
\pgfpathlineto{\pgfqpoint{2.546687in}{2.553036in}}%
\pgfpathlineto{\pgfqpoint{2.560813in}{2.535247in}}%
\pgfpathlineto{\pgfqpoint{2.551509in}{2.547733in}}%
\pgfpathlineto{\pgfqpoint{2.542179in}{2.560656in}}%
\pgfpathlineto{\pgfqpoint{2.532821in}{2.574025in}}%
\pgfpathlineto{\pgfqpoint{2.523436in}{2.587846in}}%
\pgfpathlineto{\pgfqpoint{2.509246in}{2.606289in}}%
\pgfpathlineto{\pgfqpoint{2.495051in}{2.624890in}}%
\pgfpathlineto{\pgfqpoint{2.480849in}{2.643651in}}%
\pgfpathlineto{\pgfqpoint{2.466641in}{2.662572in}}%
\pgfpathlineto{\pgfqpoint{2.476092in}{2.648085in}}%
\pgfpathlineto{\pgfqpoint{2.485513in}{2.634057in}}%
\pgfpathlineto{\pgfqpoint{2.494907in}{2.620481in}}%
\pgfpathlineto{\pgfqpoint{2.504274in}{2.607349in}}%
\pgfpathclose%
\pgfusepath{fill}%
\end{pgfscope}%
\begin{pgfscope}%
\pgfpathrectangle{\pgfqpoint{1.150000in}{0.150000in}}{\pgfqpoint{5.700000in}{5.700000in}}%
\pgfusepath{clip}%
\pgfsetbuttcap%
\pgfsetroundjoin%
\definecolor{currentfill}{rgb}{0.281887,0.150881,0.465405}%
\pgfsetfillcolor{currentfill}%
\pgfsetfillopacity{0.700000}%
\pgfsetlinewidth{0.000000pt}%
\definecolor{currentstroke}{rgb}{0.000000,0.000000,0.000000}%
\pgfsetstrokecolor{currentstroke}%
\pgfsetdash{}{0pt}%
\pgfpathmoveto{\pgfqpoint{3.381811in}{1.747689in}}%
\pgfpathlineto{\pgfqpoint{3.395811in}{1.737936in}}%
\pgfpathlineto{\pgfqpoint{3.409813in}{1.728295in}}%
\pgfpathlineto{\pgfqpoint{3.423817in}{1.718764in}}%
\pgfpathlineto{\pgfqpoint{3.437823in}{1.709344in}}%
\pgfpathlineto{\pgfqpoint{3.429313in}{1.711198in}}%
\pgfpathlineto{\pgfqpoint{3.420788in}{1.713375in}}%
\pgfpathlineto{\pgfqpoint{3.412250in}{1.715881in}}%
\pgfpathlineto{\pgfqpoint{3.403697in}{1.718722in}}%
\pgfpathlineto{\pgfqpoint{3.389655in}{1.728732in}}%
\pgfpathlineto{\pgfqpoint{3.375614in}{1.738852in}}%
\pgfpathlineto{\pgfqpoint{3.361575in}{1.749084in}}%
\pgfpathlineto{\pgfqpoint{3.347538in}{1.759427in}}%
\pgfpathlineto{\pgfqpoint{3.356129in}{1.755988in}}%
\pgfpathlineto{\pgfqpoint{3.364704in}{1.752889in}}%
\pgfpathlineto{\pgfqpoint{3.373265in}{1.750125in}}%
\pgfpathlineto{\pgfqpoint{3.381811in}{1.747689in}}%
\pgfpathclose%
\pgfusepath{fill}%
\end{pgfscope}%
\begin{pgfscope}%
\pgfpathrectangle{\pgfqpoint{1.150000in}{0.150000in}}{\pgfqpoint{5.700000in}{5.700000in}}%
\pgfusepath{clip}%
\pgfsetbuttcap%
\pgfsetroundjoin%
\definecolor{currentfill}{rgb}{0.225863,0.330805,0.547314}%
\pgfsetfillcolor{currentfill}%
\pgfsetfillopacity{0.700000}%
\pgfsetlinewidth{0.000000pt}%
\definecolor{currentstroke}{rgb}{0.000000,0.000000,0.000000}%
\pgfsetstrokecolor{currentstroke}%
\pgfsetdash{}{0pt}%
\pgfpathmoveto{\pgfqpoint{2.898621in}{2.152541in}}%
\pgfpathlineto{\pgfqpoint{2.912660in}{2.138331in}}%
\pgfpathlineto{\pgfqpoint{2.926697in}{2.124252in}}%
\pgfpathlineto{\pgfqpoint{2.940732in}{2.110303in}}%
\pgfpathlineto{\pgfqpoint{2.954766in}{2.096484in}}%
\pgfpathlineto{\pgfqpoint{2.945851in}{2.104552in}}%
\pgfpathlineto{\pgfqpoint{2.936914in}{2.113014in}}%
\pgfpathlineto{\pgfqpoint{2.927957in}{2.121877in}}%
\pgfpathlineto{\pgfqpoint{2.918978in}{2.131148in}}%
\pgfpathlineto{\pgfqpoint{2.904894in}{2.145596in}}%
\pgfpathlineto{\pgfqpoint{2.890807in}{2.160175in}}%
\pgfpathlineto{\pgfqpoint{2.876718in}{2.174885in}}%
\pgfpathlineto{\pgfqpoint{2.862627in}{2.189726in}}%
\pgfpathlineto{\pgfqpoint{2.871658in}{2.179816in}}%
\pgfpathlineto{\pgfqpoint{2.880667in}{2.170320in}}%
\pgfpathlineto{\pgfqpoint{2.889655in}{2.161231in}}%
\pgfpathlineto{\pgfqpoint{2.898621in}{2.152541in}}%
\pgfpathclose%
\pgfusepath{fill}%
\end{pgfscope}%
\begin{pgfscope}%
\pgfpathrectangle{\pgfqpoint{1.150000in}{0.150000in}}{\pgfqpoint{5.700000in}{5.700000in}}%
\pgfusepath{clip}%
\pgfsetbuttcap%
\pgfsetroundjoin%
\definecolor{currentfill}{rgb}{0.280894,0.078907,0.402329}%
\pgfsetfillcolor{currentfill}%
\pgfsetfillopacity{0.700000}%
\pgfsetlinewidth{0.000000pt}%
\definecolor{currentstroke}{rgb}{0.000000,0.000000,0.000000}%
\pgfsetstrokecolor{currentstroke}%
\pgfsetdash{}{0pt}%
\pgfpathmoveto{\pgfqpoint{3.639643in}{1.607251in}}%
\pgfpathlineto{\pgfqpoint{3.653658in}{1.599794in}}%
\pgfpathlineto{\pgfqpoint{3.667677in}{1.592443in}}%
\pgfpathlineto{\pgfqpoint{3.681700in}{1.585195in}}%
\pgfpathlineto{\pgfqpoint{3.695727in}{1.578052in}}%
\pgfpathlineto{\pgfqpoint{3.687389in}{1.576442in}}%
\pgfpathlineto{\pgfqpoint{3.679040in}{1.575110in}}%
\pgfpathlineto{\pgfqpoint{3.670679in}{1.574062in}}%
\pgfpathlineto{\pgfqpoint{3.662308in}{1.573303in}}%
\pgfpathlineto{\pgfqpoint{3.648253in}{1.581011in}}%
\pgfpathlineto{\pgfqpoint{3.634202in}{1.588823in}}%
\pgfpathlineto{\pgfqpoint{3.620154in}{1.596740in}}%
\pgfpathlineto{\pgfqpoint{3.606110in}{1.604762in}}%
\pgfpathlineto{\pgfqpoint{3.614510in}{1.604948in}}%
\pgfpathlineto{\pgfqpoint{3.622899in}{1.605429in}}%
\pgfpathlineto{\pgfqpoint{3.631276in}{1.606198in}}%
\pgfpathlineto{\pgfqpoint{3.639643in}{1.607251in}}%
\pgfpathclose%
\pgfusepath{fill}%
\end{pgfscope}%
\begin{pgfscope}%
\pgfpathrectangle{\pgfqpoint{1.150000in}{0.150000in}}{\pgfqpoint{5.700000in}{5.700000in}}%
\pgfusepath{clip}%
\pgfsetbuttcap%
\pgfsetroundjoin%
\definecolor{currentfill}{rgb}{0.235526,0.309527,0.542944}%
\pgfsetfillcolor{currentfill}%
\pgfsetfillopacity{0.700000}%
\pgfsetlinewidth{0.000000pt}%
\definecolor{currentstroke}{rgb}{0.000000,0.000000,0.000000}%
\pgfsetstrokecolor{currentstroke}%
\pgfsetdash{}{0pt}%
\pgfpathmoveto{\pgfqpoint{2.954766in}{2.096484in}}%
\pgfpathlineto{\pgfqpoint{2.968798in}{2.082794in}}%
\pgfpathlineto{\pgfqpoint{2.982829in}{2.069232in}}%
\pgfpathlineto{\pgfqpoint{2.996859in}{2.055798in}}%
\pgfpathlineto{\pgfqpoint{3.010888in}{2.042490in}}%
\pgfpathlineto{\pgfqpoint{3.002021in}{2.049940in}}%
\pgfpathlineto{\pgfqpoint{2.993135in}{2.057777in}}%
\pgfpathlineto{\pgfqpoint{2.984228in}{2.066009in}}%
\pgfpathlineto{\pgfqpoint{2.975301in}{2.074644in}}%
\pgfpathlineto{\pgfqpoint{2.961223in}{2.088578in}}%
\pgfpathlineto{\pgfqpoint{2.947143in}{2.102640in}}%
\pgfpathlineto{\pgfqpoint{2.933061in}{2.116829in}}%
\pgfpathlineto{\pgfqpoint{2.918978in}{2.131148in}}%
\pgfpathlineto{\pgfqpoint{2.927957in}{2.121877in}}%
\pgfpathlineto{\pgfqpoint{2.936914in}{2.113014in}}%
\pgfpathlineto{\pgfqpoint{2.945851in}{2.104552in}}%
\pgfpathlineto{\pgfqpoint{2.954766in}{2.096484in}}%
\pgfpathclose%
\pgfusepath{fill}%
\end{pgfscope}%
\begin{pgfscope}%
\pgfpathrectangle{\pgfqpoint{1.150000in}{0.150000in}}{\pgfqpoint{5.700000in}{5.700000in}}%
\pgfusepath{clip}%
\pgfsetbuttcap%
\pgfsetroundjoin%
\definecolor{currentfill}{rgb}{0.141935,0.526453,0.555991}%
\pgfsetfillcolor{currentfill}%
\pgfsetfillopacity{0.700000}%
\pgfsetlinewidth{0.000000pt}%
\definecolor{currentstroke}{rgb}{0.000000,0.000000,0.000000}%
\pgfsetstrokecolor{currentstroke}%
\pgfsetdash{}{0pt}%
\pgfpathmoveto{\pgfqpoint{2.447641in}{2.682023in}}%
\pgfpathlineto{\pgfqpoint{2.461808in}{2.663109in}}%
\pgfpathlineto{\pgfqpoint{2.475970in}{2.644359in}}%
\pgfpathlineto{\pgfqpoint{2.490125in}{2.625773in}}%
\pgfpathlineto{\pgfqpoint{2.504274in}{2.607349in}}%
\pgfpathlineto{\pgfqpoint{2.494907in}{2.620481in}}%
\pgfpathlineto{\pgfqpoint{2.485513in}{2.634057in}}%
\pgfpathlineto{\pgfqpoint{2.476092in}{2.648085in}}%
\pgfpathlineto{\pgfqpoint{2.466641in}{2.662572in}}%
\pgfpathlineto{\pgfqpoint{2.452427in}{2.681655in}}%
\pgfpathlineto{\pgfqpoint{2.438207in}{2.700901in}}%
\pgfpathlineto{\pgfqpoint{2.423980in}{2.720312in}}%
\pgfpathlineto{\pgfqpoint{2.409746in}{2.739889in}}%
\pgfpathlineto{\pgfqpoint{2.419263in}{2.724731in}}%
\pgfpathlineto{\pgfqpoint{2.428751in}{2.710039in}}%
\pgfpathlineto{\pgfqpoint{2.438210in}{2.695805in}}%
\pgfpathlineto{\pgfqpoint{2.447641in}{2.682023in}}%
\pgfpathclose%
\pgfusepath{fill}%
\end{pgfscope}%
\begin{pgfscope}%
\pgfpathrectangle{\pgfqpoint{1.150000in}{0.150000in}}{\pgfqpoint{5.700000in}{5.700000in}}%
\pgfusepath{clip}%
\pgfsetbuttcap%
\pgfsetroundjoin%
\definecolor{currentfill}{rgb}{0.272594,0.025563,0.353093}%
\pgfsetfillcolor{currentfill}%
\pgfsetfillopacity{0.700000}%
\pgfsetlinewidth{0.000000pt}%
\definecolor{currentstroke}{rgb}{0.000000,0.000000,0.000000}%
\pgfsetstrokecolor{currentstroke}%
\pgfsetdash{}{0pt}%
\pgfpathmoveto{\pgfqpoint{3.986434in}{1.516976in}}%
\pgfpathlineto{\pgfqpoint{4.000506in}{1.512627in}}%
\pgfpathlineto{\pgfqpoint{4.014585in}{1.508378in}}%
\pgfpathlineto{\pgfqpoint{4.028670in}{1.504228in}}%
\pgfpathlineto{\pgfqpoint{4.042761in}{1.500176in}}%
\pgfpathlineto{\pgfqpoint{4.034601in}{1.493962in}}%
\pgfpathlineto{\pgfqpoint{4.026433in}{1.487956in}}%
\pgfpathlineto{\pgfqpoint{4.018259in}{1.482166in}}%
\pgfpathlineto{\pgfqpoint{4.010077in}{1.476595in}}%
\pgfpathlineto{\pgfqpoint{3.995968in}{1.481170in}}%
\pgfpathlineto{\pgfqpoint{3.981865in}{1.485844in}}%
\pgfpathlineto{\pgfqpoint{3.967768in}{1.490617in}}%
\pgfpathlineto{\pgfqpoint{3.953678in}{1.495490in}}%
\pgfpathlineto{\pgfqpoint{3.961879in}{1.500530in}}%
\pgfpathlineto{\pgfqpoint{3.970071in}{1.505794in}}%
\pgfpathlineto{\pgfqpoint{3.978257in}{1.511278in}}%
\pgfpathlineto{\pgfqpoint{3.986434in}{1.516976in}}%
\pgfpathclose%
\pgfusepath{fill}%
\end{pgfscope}%
\begin{pgfscope}%
\pgfpathrectangle{\pgfqpoint{1.150000in}{0.150000in}}{\pgfqpoint{5.700000in}{5.700000in}}%
\pgfusepath{clip}%
\pgfsetbuttcap%
\pgfsetroundjoin%
\definecolor{currentfill}{rgb}{0.239346,0.300855,0.540844}%
\pgfsetfillcolor{currentfill}%
\pgfsetfillopacity{0.700000}%
\pgfsetlinewidth{0.000000pt}%
\definecolor{currentstroke}{rgb}{0.000000,0.000000,0.000000}%
\pgfsetstrokecolor{currentstroke}%
\pgfsetdash{}{0pt}%
\pgfpathmoveto{\pgfqpoint{5.053206in}{2.088463in}}%
\pgfpathlineto{\pgfqpoint{5.067703in}{2.093700in}}%
\pgfpathlineto{\pgfqpoint{5.082213in}{2.099036in}}%
\pgfpathlineto{\pgfqpoint{5.096737in}{2.104470in}}%
\pgfpathlineto{\pgfqpoint{5.111275in}{2.110003in}}%
\pgfpathlineto{\pgfqpoint{5.103391in}{2.095238in}}%
\pgfpathlineto{\pgfqpoint{5.095502in}{2.080436in}}%
\pgfpathlineto{\pgfqpoint{5.087608in}{2.065603in}}%
\pgfpathlineto{\pgfqpoint{5.079710in}{2.050738in}}%
\pgfpathlineto{\pgfqpoint{5.065176in}{2.045521in}}%
\pgfpathlineto{\pgfqpoint{5.050656in}{2.040401in}}%
\pgfpathlineto{\pgfqpoint{5.036149in}{2.035380in}}%
\pgfpathlineto{\pgfqpoint{5.021656in}{2.030456in}}%
\pgfpathlineto{\pgfqpoint{5.029550in}{2.044999in}}%
\pgfpathlineto{\pgfqpoint{5.037440in}{2.059516in}}%
\pgfpathlineto{\pgfqpoint{5.045326in}{2.074005in}}%
\pgfpathlineto{\pgfqpoint{5.053206in}{2.088463in}}%
\pgfpathclose%
\pgfusepath{fill}%
\end{pgfscope}%
\begin{pgfscope}%
\pgfpathrectangle{\pgfqpoint{1.150000in}{0.150000in}}{\pgfqpoint{5.700000in}{5.700000in}}%
\pgfusepath{clip}%
\pgfsetbuttcap%
\pgfsetroundjoin%
\definecolor{currentfill}{rgb}{0.271828,0.209303,0.504434}%
\pgfsetfillcolor{currentfill}%
\pgfsetfillopacity{0.700000}%
\pgfsetlinewidth{0.000000pt}%
\definecolor{currentstroke}{rgb}{0.000000,0.000000,0.000000}%
\pgfsetstrokecolor{currentstroke}%
\pgfsetdash{}{0pt}%
\pgfpathmoveto{\pgfqpoint{4.842912in}{1.883144in}}%
\pgfpathlineto{\pgfqpoint{4.857297in}{1.886606in}}%
\pgfpathlineto{\pgfqpoint{4.871695in}{1.890165in}}%
\pgfpathlineto{\pgfqpoint{4.886105in}{1.893822in}}%
\pgfpathlineto{\pgfqpoint{4.900528in}{1.897575in}}%
\pgfpathlineto{\pgfqpoint{4.892599in}{1.883304in}}%
\pgfpathlineto{\pgfqpoint{4.884665in}{1.869043in}}%
\pgfpathlineto{\pgfqpoint{4.876728in}{1.854796in}}%
\pgfpathlineto{\pgfqpoint{4.868787in}{1.840566in}}%
\pgfpathlineto{\pgfqpoint{4.854366in}{1.837180in}}%
\pgfpathlineto{\pgfqpoint{4.839957in}{1.833890in}}%
\pgfpathlineto{\pgfqpoint{4.825561in}{1.830698in}}%
\pgfpathlineto{\pgfqpoint{4.811177in}{1.827603in}}%
\pgfpathlineto{\pgfqpoint{4.819116in}{1.841458in}}%
\pgfpathlineto{\pgfqpoint{4.827052in}{1.855336in}}%
\pgfpathlineto{\pgfqpoint{4.834984in}{1.869232in}}%
\pgfpathlineto{\pgfqpoint{4.842912in}{1.883144in}}%
\pgfpathclose%
\pgfusepath{fill}%
\end{pgfscope}%
\begin{pgfscope}%
\pgfpathrectangle{\pgfqpoint{1.150000in}{0.150000in}}{\pgfqpoint{5.700000in}{5.700000in}}%
\pgfusepath{clip}%
\pgfsetbuttcap%
\pgfsetroundjoin%
\definecolor{currentfill}{rgb}{0.174274,0.445044,0.557792}%
\pgfsetfillcolor{currentfill}%
\pgfsetfillopacity{0.700000}%
\pgfsetlinewidth{0.000000pt}%
\definecolor{currentstroke}{rgb}{0.000000,0.000000,0.000000}%
\pgfsetstrokecolor{currentstroke}%
\pgfsetdash{}{0pt}%
\pgfpathmoveto{\pgfqpoint{5.384858in}{2.452018in}}%
\pgfpathlineto{\pgfqpoint{5.399548in}{2.459742in}}%
\pgfpathlineto{\pgfqpoint{5.414254in}{2.467567in}}%
\pgfpathlineto{\pgfqpoint{5.428975in}{2.475491in}}%
\pgfpathlineto{\pgfqpoint{5.443713in}{2.483517in}}%
\pgfpathlineto{\pgfqpoint{5.435926in}{2.469348in}}%
\pgfpathlineto{\pgfqpoint{5.428133in}{2.455082in}}%
\pgfpathlineto{\pgfqpoint{5.420333in}{2.440721in}}%
\pgfpathlineto{\pgfqpoint{5.412527in}{2.426267in}}%
\pgfpathlineto{\pgfqpoint{5.397794in}{2.418465in}}%
\pgfpathlineto{\pgfqpoint{5.383076in}{2.410763in}}%
\pgfpathlineto{\pgfqpoint{5.368374in}{2.403161in}}%
\pgfpathlineto{\pgfqpoint{5.353688in}{2.395660in}}%
\pgfpathlineto{\pgfqpoint{5.361490in}{2.409884in}}%
\pgfpathlineto{\pgfqpoint{5.369285in}{2.424020in}}%
\pgfpathlineto{\pgfqpoint{5.377075in}{2.438065in}}%
\pgfpathlineto{\pgfqpoint{5.384858in}{2.452018in}}%
\pgfpathclose%
\pgfusepath{fill}%
\end{pgfscope}%
\begin{pgfscope}%
\pgfpathrectangle{\pgfqpoint{1.150000in}{0.150000in}}{\pgfqpoint{5.700000in}{5.700000in}}%
\pgfusepath{clip}%
\pgfsetbuttcap%
\pgfsetroundjoin%
\definecolor{currentfill}{rgb}{0.276022,0.044167,0.370164}%
\pgfsetfillcolor{currentfill}%
\pgfsetfillopacity{0.700000}%
\pgfsetlinewidth{0.000000pt}%
\definecolor{currentstroke}{rgb}{0.000000,0.000000,0.000000}%
\pgfsetstrokecolor{currentstroke}%
\pgfsetdash{}{0pt}%
\pgfpathmoveto{\pgfqpoint{3.841160in}{1.538075in}}%
\pgfpathlineto{\pgfqpoint{3.855206in}{1.532399in}}%
\pgfpathlineto{\pgfqpoint{3.869257in}{1.526825in}}%
\pgfpathlineto{\pgfqpoint{3.883313in}{1.521351in}}%
\pgfpathlineto{\pgfqpoint{3.897374in}{1.515979in}}%
\pgfpathlineto{\pgfqpoint{3.889146in}{1.511705in}}%
\pgfpathlineto{\pgfqpoint{3.880908in}{1.507672in}}%
\pgfpathlineto{\pgfqpoint{3.872662in}{1.503884in}}%
\pgfpathlineto{\pgfqpoint{3.864407in}{1.500349in}}%
\pgfpathlineto{\pgfqpoint{3.850324in}{1.506265in}}%
\pgfpathlineto{\pgfqpoint{3.836245in}{1.512282in}}%
\pgfpathlineto{\pgfqpoint{3.822172in}{1.518400in}}%
\pgfpathlineto{\pgfqpoint{3.808104in}{1.524619in}}%
\pgfpathlineto{\pgfqpoint{3.816382in}{1.527604in}}%
\pgfpathlineto{\pgfqpoint{3.824651in}{1.530845in}}%
\pgfpathlineto{\pgfqpoint{3.832910in}{1.534338in}}%
\pgfpathlineto{\pgfqpoint{3.841160in}{1.538075in}}%
\pgfpathclose%
\pgfusepath{fill}%
\end{pgfscope}%
\begin{pgfscope}%
\pgfpathrectangle{\pgfqpoint{1.150000in}{0.150000in}}{\pgfqpoint{5.700000in}{5.700000in}}%
\pgfusepath{clip}%
\pgfsetbuttcap%
\pgfsetroundjoin%
\definecolor{currentfill}{rgb}{0.244972,0.287675,0.537260}%
\pgfsetfillcolor{currentfill}%
\pgfsetfillopacity{0.700000}%
\pgfsetlinewidth{0.000000pt}%
\definecolor{currentstroke}{rgb}{0.000000,0.000000,0.000000}%
\pgfsetstrokecolor{currentstroke}%
\pgfsetdash{}{0pt}%
\pgfpathmoveto{\pgfqpoint{3.010888in}{2.042490in}}%
\pgfpathlineto{\pgfqpoint{3.024916in}{2.029308in}}%
\pgfpathlineto{\pgfqpoint{3.038943in}{2.016252in}}%
\pgfpathlineto{\pgfqpoint{3.052969in}{2.003320in}}%
\pgfpathlineto{\pgfqpoint{3.066994in}{1.990512in}}%
\pgfpathlineto{\pgfqpoint{3.058175in}{1.997345in}}%
\pgfpathlineto{\pgfqpoint{3.049337in}{2.004561in}}%
\pgfpathlineto{\pgfqpoint{3.040479in}{2.012166in}}%
\pgfpathlineto{\pgfqpoint{3.031601in}{2.020166in}}%
\pgfpathlineto{\pgfqpoint{3.017528in}{2.033598in}}%
\pgfpathlineto{\pgfqpoint{3.003453in}{2.047154in}}%
\pgfpathlineto{\pgfqpoint{2.989378in}{2.060836in}}%
\pgfpathlineto{\pgfqpoint{2.975301in}{2.074644in}}%
\pgfpathlineto{\pgfqpoint{2.984228in}{2.066009in}}%
\pgfpathlineto{\pgfqpoint{2.993135in}{2.057777in}}%
\pgfpathlineto{\pgfqpoint{3.002021in}{2.049940in}}%
\pgfpathlineto{\pgfqpoint{3.010888in}{2.042490in}}%
\pgfpathclose%
\pgfusepath{fill}%
\end{pgfscope}%
\begin{pgfscope}%
\pgfpathrectangle{\pgfqpoint{1.150000in}{0.150000in}}{\pgfqpoint{5.700000in}{5.700000in}}%
\pgfusepath{clip}%
\pgfsetbuttcap%
\pgfsetroundjoin%
\definecolor{currentfill}{rgb}{0.131172,0.555899,0.552459}%
\pgfsetfillcolor{currentfill}%
\pgfsetfillopacity{0.700000}%
\pgfsetlinewidth{0.000000pt}%
\definecolor{currentstroke}{rgb}{0.000000,0.000000,0.000000}%
\pgfsetstrokecolor{currentstroke}%
\pgfsetdash{}{0pt}%
\pgfpathmoveto{\pgfqpoint{2.390904in}{2.759353in}}%
\pgfpathlineto{\pgfqpoint{2.405098in}{2.739767in}}%
\pgfpathlineto{\pgfqpoint{2.419286in}{2.720350in}}%
\pgfpathlineto{\pgfqpoint{2.433467in}{2.701103in}}%
\pgfpathlineto{\pgfqpoint{2.447641in}{2.682023in}}%
\pgfpathlineto{\pgfqpoint{2.438210in}{2.695805in}}%
\pgfpathlineto{\pgfqpoint{2.428751in}{2.710039in}}%
\pgfpathlineto{\pgfqpoint{2.419263in}{2.724731in}}%
\pgfpathlineto{\pgfqpoint{2.409746in}{2.739889in}}%
\pgfpathlineto{\pgfqpoint{2.395505in}{2.759633in}}%
\pgfpathlineto{\pgfqpoint{2.381257in}{2.779545in}}%
\pgfpathlineto{\pgfqpoint{2.367002in}{2.799628in}}%
\pgfpathlineto{\pgfqpoint{2.352739in}{2.819882in}}%
\pgfpathlineto{\pgfqpoint{2.362325in}{2.804048in}}%
\pgfpathlineto{\pgfqpoint{2.371881in}{2.788687in}}%
\pgfpathlineto{\pgfqpoint{2.381407in}{2.773791in}}%
\pgfpathlineto{\pgfqpoint{2.390904in}{2.759353in}}%
\pgfpathclose%
\pgfusepath{fill}%
\end{pgfscope}%
\begin{pgfscope}%
\pgfpathrectangle{\pgfqpoint{1.150000in}{0.150000in}}{\pgfqpoint{5.700000in}{5.700000in}}%
\pgfusepath{clip}%
\pgfsetbuttcap%
\pgfsetroundjoin%
\definecolor{currentfill}{rgb}{0.141935,0.526453,0.555991}%
\pgfsetfillcolor{currentfill}%
\pgfsetfillopacity{0.700000}%
\pgfsetlinewidth{0.000000pt}%
\definecolor{currentstroke}{rgb}{0.000000,0.000000,0.000000}%
\pgfsetstrokecolor{currentstroke}%
\pgfsetdash{}{0pt}%
\pgfpathmoveto{\pgfqpoint{5.595765in}{2.680642in}}%
\pgfpathlineto{\pgfqpoint{5.610593in}{2.689725in}}%
\pgfpathlineto{\pgfqpoint{5.625438in}{2.698911in}}%
\pgfpathlineto{\pgfqpoint{5.640300in}{2.708198in}}%
\pgfpathlineto{\pgfqpoint{5.632594in}{2.695053in}}%
\pgfpathlineto{\pgfqpoint{5.624880in}{2.681782in}}%
\pgfpathlineto{\pgfqpoint{5.617158in}{2.668388in}}%
\pgfpathlineto{\pgfqpoint{5.609429in}{2.654871in}}%
\pgfpathlineto{\pgfqpoint{5.594570in}{2.645750in}}%
\pgfpathlineto{\pgfqpoint{5.579727in}{2.636729in}}%
\pgfpathlineto{\pgfqpoint{5.564901in}{2.627811in}}%
\pgfpathlineto{\pgfqpoint{5.572628in}{2.641198in}}%
\pgfpathlineto{\pgfqpoint{5.580348in}{2.654466in}}%
\pgfpathlineto{\pgfqpoint{5.588060in}{2.667615in}}%
\pgfpathlineto{\pgfqpoint{5.595765in}{2.680642in}}%
\pgfpathclose%
\pgfusepath{fill}%
\end{pgfscope}%
\begin{pgfscope}%
\pgfpathrectangle{\pgfqpoint{1.150000in}{0.150000in}}{\pgfqpoint{5.700000in}{5.700000in}}%
\pgfusepath{clip}%
\pgfsetbuttcap%
\pgfsetroundjoin%
\definecolor{currentfill}{rgb}{0.272594,0.025563,0.353093}%
\pgfsetfillcolor{currentfill}%
\pgfsetfillopacity{0.700000}%
\pgfsetlinewidth{0.000000pt}%
\definecolor{currentstroke}{rgb}{0.000000,0.000000,0.000000}%
\pgfsetstrokecolor{currentstroke}%
\pgfsetdash{}{0pt}%
\pgfpathmoveto{\pgfqpoint{4.131706in}{1.513828in}}%
\pgfpathlineto{\pgfqpoint{4.145818in}{1.510775in}}%
\pgfpathlineto{\pgfqpoint{4.159936in}{1.507819in}}%
\pgfpathlineto{\pgfqpoint{4.174062in}{1.504961in}}%
\pgfpathlineto{\pgfqpoint{4.188196in}{1.502201in}}%
\pgfpathlineto{\pgfqpoint{4.180091in}{1.494203in}}%
\pgfpathlineto{\pgfqpoint{4.171979in}{1.486385in}}%
\pgfpathlineto{\pgfqpoint{4.163862in}{1.478752in}}%
\pgfpathlineto{\pgfqpoint{4.155739in}{1.471308in}}%
\pgfpathlineto{\pgfqpoint{4.141592in}{1.474574in}}%
\pgfpathlineto{\pgfqpoint{4.127452in}{1.477938in}}%
\pgfpathlineto{\pgfqpoint{4.113320in}{1.481399in}}%
\pgfpathlineto{\pgfqpoint{4.099194in}{1.484958in}}%
\pgfpathlineto{\pgfqpoint{4.107332in}{1.491889in}}%
\pgfpathlineto{\pgfqpoint{4.115463in}{1.499015in}}%
\pgfpathlineto{\pgfqpoint{4.123588in}{1.506329in}}%
\pgfpathlineto{\pgfqpoint{4.131706in}{1.513828in}}%
\pgfpathclose%
\pgfusepath{fill}%
\end{pgfscope}%
\begin{pgfscope}%
\pgfpathrectangle{\pgfqpoint{1.150000in}{0.150000in}}{\pgfqpoint{5.700000in}{5.700000in}}%
\pgfusepath{clip}%
\pgfsetbuttcap%
\pgfsetroundjoin%
\definecolor{currentfill}{rgb}{0.282884,0.135920,0.453427}%
\pgfsetfillcolor{currentfill}%
\pgfsetfillopacity{0.700000}%
\pgfsetlinewidth{0.000000pt}%
\definecolor{currentstroke}{rgb}{0.000000,0.000000,0.000000}%
\pgfsetstrokecolor{currentstroke}%
\pgfsetdash{}{0pt}%
\pgfpathmoveto{\pgfqpoint{3.437823in}{1.709344in}}%
\pgfpathlineto{\pgfqpoint{3.451832in}{1.700034in}}%
\pgfpathlineto{\pgfqpoint{3.465843in}{1.690833in}}%
\pgfpathlineto{\pgfqpoint{3.479857in}{1.681742in}}%
\pgfpathlineto{\pgfqpoint{3.493873in}{1.672759in}}%
\pgfpathlineto{\pgfqpoint{3.485397in}{1.674032in}}%
\pgfpathlineto{\pgfqpoint{3.476907in}{1.675623in}}%
\pgfpathlineto{\pgfqpoint{3.468404in}{1.677537in}}%
\pgfpathlineto{\pgfqpoint{3.459887in}{1.679781in}}%
\pgfpathlineto{\pgfqpoint{3.445836in}{1.689352in}}%
\pgfpathlineto{\pgfqpoint{3.431788in}{1.699033in}}%
\pgfpathlineto{\pgfqpoint{3.417741in}{1.708822in}}%
\pgfpathlineto{\pgfqpoint{3.403697in}{1.718722in}}%
\pgfpathlineto{\pgfqpoint{3.412250in}{1.715881in}}%
\pgfpathlineto{\pgfqpoint{3.420788in}{1.713375in}}%
\pgfpathlineto{\pgfqpoint{3.429313in}{1.711198in}}%
\pgfpathlineto{\pgfqpoint{3.437823in}{1.709344in}}%
\pgfpathclose%
\pgfusepath{fill}%
\end{pgfscope}%
\begin{pgfscope}%
\pgfpathrectangle{\pgfqpoint{1.150000in}{0.150000in}}{\pgfqpoint{5.700000in}{5.700000in}}%
\pgfusepath{clip}%
\pgfsetbuttcap%
\pgfsetroundjoin%
\definecolor{currentfill}{rgb}{0.197636,0.391528,0.554969}%
\pgfsetfillcolor{currentfill}%
\pgfsetfillopacity{0.700000}%
\pgfsetlinewidth{0.000000pt}%
\definecolor{currentstroke}{rgb}{0.000000,0.000000,0.000000}%
\pgfsetstrokecolor{currentstroke}%
\pgfsetdash{}{0pt}%
\pgfpathmoveto{\pgfqpoint{5.263843in}{2.309875in}}%
\pgfpathlineto{\pgfqpoint{5.278465in}{2.316736in}}%
\pgfpathlineto{\pgfqpoint{5.293102in}{2.323697in}}%
\pgfpathlineto{\pgfqpoint{5.307754in}{2.330758in}}%
\pgfpathlineto{\pgfqpoint{5.322421in}{2.337918in}}%
\pgfpathlineto{\pgfqpoint{5.314589in}{2.323281in}}%
\pgfpathlineto{\pgfqpoint{5.306752in}{2.308570in}}%
\pgfpathlineto{\pgfqpoint{5.298910in}{2.293784in}}%
\pgfpathlineto{\pgfqpoint{5.291062in}{2.278927in}}%
\pgfpathlineto{\pgfqpoint{5.276399in}{2.272028in}}%
\pgfpathlineto{\pgfqpoint{5.261752in}{2.265228in}}%
\pgfpathlineto{\pgfqpoint{5.247119in}{2.258527in}}%
\pgfpathlineto{\pgfqpoint{5.232501in}{2.251926in}}%
\pgfpathlineto{\pgfqpoint{5.240345in}{2.266515in}}%
\pgfpathlineto{\pgfqpoint{5.248183in}{2.281038in}}%
\pgfpathlineto{\pgfqpoint{5.256016in}{2.295492in}}%
\pgfpathlineto{\pgfqpoint{5.263843in}{2.309875in}}%
\pgfpathclose%
\pgfusepath{fill}%
\end{pgfscope}%
\begin{pgfscope}%
\pgfpathrectangle{\pgfqpoint{1.150000in}{0.150000in}}{\pgfqpoint{5.700000in}{5.700000in}}%
\pgfusepath{clip}%
\pgfsetbuttcap%
\pgfsetroundjoin%
\definecolor{currentfill}{rgb}{0.253935,0.265254,0.529983}%
\pgfsetfillcolor{currentfill}%
\pgfsetfillopacity{0.700000}%
\pgfsetlinewidth{0.000000pt}%
\definecolor{currentstroke}{rgb}{0.000000,0.000000,0.000000}%
\pgfsetstrokecolor{currentstroke}%
\pgfsetdash{}{0pt}%
\pgfpathmoveto{\pgfqpoint{3.066994in}{1.990512in}}%
\pgfpathlineto{\pgfqpoint{3.081019in}{1.977827in}}%
\pgfpathlineto{\pgfqpoint{3.095044in}{1.965265in}}%
\pgfpathlineto{\pgfqpoint{3.109068in}{1.952824in}}%
\pgfpathlineto{\pgfqpoint{3.123092in}{1.940506in}}%
\pgfpathlineto{\pgfqpoint{3.114319in}{1.946726in}}%
\pgfpathlineto{\pgfqpoint{3.105527in}{1.953322in}}%
\pgfpathlineto{\pgfqpoint{3.096716in}{1.960301in}}%
\pgfpathlineto{\pgfqpoint{3.087886in}{1.967671in}}%
\pgfpathlineto{\pgfqpoint{3.073816in}{1.980611in}}%
\pgfpathlineto{\pgfqpoint{3.059745in}{1.993673in}}%
\pgfpathlineto{\pgfqpoint{3.045673in}{2.006858in}}%
\pgfpathlineto{\pgfqpoint{3.031601in}{2.020166in}}%
\pgfpathlineto{\pgfqpoint{3.040479in}{2.012166in}}%
\pgfpathlineto{\pgfqpoint{3.049337in}{2.004561in}}%
\pgfpathlineto{\pgfqpoint{3.058175in}{1.997345in}}%
\pgfpathlineto{\pgfqpoint{3.066994in}{1.990512in}}%
\pgfpathclose%
\pgfusepath{fill}%
\end{pgfscope}%
\begin{pgfscope}%
\pgfpathrectangle{\pgfqpoint{1.150000in}{0.150000in}}{\pgfqpoint{5.700000in}{5.700000in}}%
\pgfusepath{clip}%
\pgfsetbuttcap%
\pgfsetroundjoin%
\definecolor{currentfill}{rgb}{0.282910,0.105393,0.426902}%
\pgfsetfillcolor{currentfill}%
\pgfsetfillopacity{0.700000}%
\pgfsetlinewidth{0.000000pt}%
\definecolor{currentstroke}{rgb}{0.000000,0.000000,0.000000}%
\pgfsetstrokecolor{currentstroke}%
\pgfsetdash{}{0pt}%
\pgfpathmoveto{\pgfqpoint{4.543860in}{1.649235in}}%
\pgfpathlineto{\pgfqpoint{4.558114in}{1.649967in}}%
\pgfpathlineto{\pgfqpoint{4.572378in}{1.650795in}}%
\pgfpathlineto{\pgfqpoint{4.586652in}{1.651720in}}%
\pgfpathlineto{\pgfqpoint{4.600937in}{1.652741in}}%
\pgfpathlineto{\pgfqpoint{4.592946in}{1.640368in}}%
\pgfpathlineto{\pgfqpoint{4.584950in}{1.628080in}}%
\pgfpathlineto{\pgfqpoint{4.576950in}{1.615880in}}%
\pgfpathlineto{\pgfqpoint{4.568946in}{1.603773in}}%
\pgfpathlineto{\pgfqpoint{4.554658in}{1.603189in}}%
\pgfpathlineto{\pgfqpoint{4.540381in}{1.602700in}}%
\pgfpathlineto{\pgfqpoint{4.526113in}{1.602308in}}%
\pgfpathlineto{\pgfqpoint{4.511855in}{1.602012in}}%
\pgfpathlineto{\pgfqpoint{4.519863in}{1.613676in}}%
\pgfpathlineto{\pgfqpoint{4.527866in}{1.625438in}}%
\pgfpathlineto{\pgfqpoint{4.535865in}{1.637292in}}%
\pgfpathlineto{\pgfqpoint{4.543860in}{1.649235in}}%
\pgfpathclose%
\pgfusepath{fill}%
\end{pgfscope}%
\begin{pgfscope}%
\pgfpathrectangle{\pgfqpoint{1.150000in}{0.150000in}}{\pgfqpoint{5.700000in}{5.700000in}}%
\pgfusepath{clip}%
\pgfsetbuttcap%
\pgfsetroundjoin%
\definecolor{currentfill}{rgb}{0.280894,0.078907,0.402329}%
\pgfsetfillcolor{currentfill}%
\pgfsetfillopacity{0.700000}%
\pgfsetlinewidth{0.000000pt}%
\definecolor{currentstroke}{rgb}{0.000000,0.000000,0.000000}%
\pgfsetstrokecolor{currentstroke}%
\pgfsetdash{}{0pt}%
\pgfpathmoveto{\pgfqpoint{4.454922in}{1.601792in}}%
\pgfpathlineto{\pgfqpoint{4.469141in}{1.601702in}}%
\pgfpathlineto{\pgfqpoint{4.483369in}{1.601709in}}%
\pgfpathlineto{\pgfqpoint{4.497607in}{1.601813in}}%
\pgfpathlineto{\pgfqpoint{4.511855in}{1.602012in}}%
\pgfpathlineto{\pgfqpoint{4.503843in}{1.590449in}}%
\pgfpathlineto{\pgfqpoint{4.495827in}{1.578992in}}%
\pgfpathlineto{\pgfqpoint{4.487807in}{1.567644in}}%
\pgfpathlineto{\pgfqpoint{4.479783in}{1.556411in}}%
\pgfpathlineto{\pgfqpoint{4.465530in}{1.556665in}}%
\pgfpathlineto{\pgfqpoint{4.451286in}{1.557015in}}%
\pgfpathlineto{\pgfqpoint{4.437053in}{1.557462in}}%
\pgfpathlineto{\pgfqpoint{4.422828in}{1.558004in}}%
\pgfpathlineto{\pgfqpoint{4.430858in}{1.568777in}}%
\pgfpathlineto{\pgfqpoint{4.438884in}{1.579669in}}%
\pgfpathlineto{\pgfqpoint{4.446905in}{1.590676in}}%
\pgfpathlineto{\pgfqpoint{4.454922in}{1.601792in}}%
\pgfpathclose%
\pgfusepath{fill}%
\end{pgfscope}%
\begin{pgfscope}%
\pgfpathrectangle{\pgfqpoint{1.150000in}{0.150000in}}{\pgfqpoint{5.700000in}{5.700000in}}%
\pgfusepath{clip}%
\pgfsetbuttcap%
\pgfsetroundjoin%
\definecolor{currentfill}{rgb}{0.122606,0.585371,0.546557}%
\pgfsetfillcolor{currentfill}%
\pgfsetfillopacity{0.700000}%
\pgfsetlinewidth{0.000000pt}%
\definecolor{currentstroke}{rgb}{0.000000,0.000000,0.000000}%
\pgfsetstrokecolor{currentstroke}%
\pgfsetdash{}{0pt}%
\pgfpathmoveto{\pgfqpoint{2.334051in}{2.839431in}}%
\pgfpathlineto{\pgfqpoint{2.348276in}{2.819149in}}%
\pgfpathlineto{\pgfqpoint{2.362492in}{2.799043in}}%
\pgfpathlineto{\pgfqpoint{2.376702in}{2.779111in}}%
\pgfpathlineto{\pgfqpoint{2.390904in}{2.759353in}}%
\pgfpathlineto{\pgfqpoint{2.381407in}{2.773791in}}%
\pgfpathlineto{\pgfqpoint{2.371881in}{2.788687in}}%
\pgfpathlineto{\pgfqpoint{2.362325in}{2.804048in}}%
\pgfpathlineto{\pgfqpoint{2.352739in}{2.819882in}}%
\pgfpathlineto{\pgfqpoint{2.338468in}{2.840309in}}%
\pgfpathlineto{\pgfqpoint{2.324190in}{2.860911in}}%
\pgfpathlineto{\pgfqpoint{2.309904in}{2.881688in}}%
\pgfpathlineto{\pgfqpoint{2.295609in}{2.902643in}}%
\pgfpathlineto{\pgfqpoint{2.305266in}{2.886128in}}%
\pgfpathlineto{\pgfqpoint{2.314892in}{2.870093in}}%
\pgfpathlineto{\pgfqpoint{2.324486in}{2.854529in}}%
\pgfpathlineto{\pgfqpoint{2.334051in}{2.839431in}}%
\pgfpathclose%
\pgfusepath{fill}%
\end{pgfscope}%
\begin{pgfscope}%
\pgfpathrectangle{\pgfqpoint{1.150000in}{0.150000in}}{\pgfqpoint{5.700000in}{5.700000in}}%
\pgfusepath{clip}%
\pgfsetbuttcap%
\pgfsetroundjoin%
\definecolor{currentfill}{rgb}{0.260571,0.246922,0.522828}%
\pgfsetfillcolor{currentfill}%
\pgfsetfillopacity{0.700000}%
\pgfsetlinewidth{0.000000pt}%
\definecolor{currentstroke}{rgb}{0.000000,0.000000,0.000000}%
\pgfsetstrokecolor{currentstroke}%
\pgfsetdash{}{0pt}%
\pgfpathmoveto{\pgfqpoint{4.932205in}{1.954703in}}%
\pgfpathlineto{\pgfqpoint{4.946643in}{1.958903in}}%
\pgfpathlineto{\pgfqpoint{4.961094in}{1.963202in}}%
\pgfpathlineto{\pgfqpoint{4.975557in}{1.967598in}}%
\pgfpathlineto{\pgfqpoint{4.990034in}{1.972091in}}%
\pgfpathlineto{\pgfqpoint{4.982118in}{1.957467in}}%
\pgfpathlineto{\pgfqpoint{4.974198in}{1.942835in}}%
\pgfpathlineto{\pgfqpoint{4.966274in}{1.928199in}}%
\pgfpathlineto{\pgfqpoint{4.958346in}{1.913563in}}%
\pgfpathlineto{\pgfqpoint{4.943873in}{1.909420in}}%
\pgfpathlineto{\pgfqpoint{4.929412in}{1.905375in}}%
\pgfpathlineto{\pgfqpoint{4.914964in}{1.901427in}}%
\pgfpathlineto{\pgfqpoint{4.900528in}{1.897575in}}%
\pgfpathlineto{\pgfqpoint{4.908454in}{1.911854in}}%
\pgfpathlineto{\pgfqpoint{4.916375in}{1.926138in}}%
\pgfpathlineto{\pgfqpoint{4.924292in}{1.940421in}}%
\pgfpathlineto{\pgfqpoint{4.932205in}{1.954703in}}%
\pgfpathclose%
\pgfusepath{fill}%
\end{pgfscope}%
\begin{pgfscope}%
\pgfpathrectangle{\pgfqpoint{1.150000in}{0.150000in}}{\pgfqpoint{5.700000in}{5.700000in}}%
\pgfusepath{clip}%
\pgfsetbuttcap%
\pgfsetroundjoin%
\definecolor{currentfill}{rgb}{0.283072,0.130895,0.449241}%
\pgfsetfillcolor{currentfill}%
\pgfsetfillopacity{0.700000}%
\pgfsetlinewidth{0.000000pt}%
\definecolor{currentstroke}{rgb}{0.000000,0.000000,0.000000}%
\pgfsetstrokecolor{currentstroke}%
\pgfsetdash{}{0pt}%
\pgfpathmoveto{\pgfqpoint{4.632865in}{1.702991in}}%
\pgfpathlineto{\pgfqpoint{4.647158in}{1.704527in}}%
\pgfpathlineto{\pgfqpoint{4.661462in}{1.706159in}}%
\pgfpathlineto{\pgfqpoint{4.675777in}{1.707888in}}%
\pgfpathlineto{\pgfqpoint{4.690104in}{1.709714in}}%
\pgfpathlineto{\pgfqpoint{4.682129in}{1.696633in}}%
\pgfpathlineto{\pgfqpoint{4.674151in}{1.683616in}}%
\pgfpathlineto{\pgfqpoint{4.666169in}{1.670666in}}%
\pgfpathlineto{\pgfqpoint{4.658183in}{1.657788in}}%
\pgfpathlineto{\pgfqpoint{4.643856in}{1.656381in}}%
\pgfpathlineto{\pgfqpoint{4.629539in}{1.655072in}}%
\pgfpathlineto{\pgfqpoint{4.615233in}{1.653858in}}%
\pgfpathlineto{\pgfqpoint{4.600937in}{1.652741in}}%
\pgfpathlineto{\pgfqpoint{4.608925in}{1.665193in}}%
\pgfpathlineto{\pgfqpoint{4.616909in}{1.677721in}}%
\pgfpathlineto{\pgfqpoint{4.624889in}{1.690322in}}%
\pgfpathlineto{\pgfqpoint{4.632865in}{1.702991in}}%
\pgfpathclose%
\pgfusepath{fill}%
\end{pgfscope}%
\begin{pgfscope}%
\pgfpathrectangle{\pgfqpoint{1.150000in}{0.150000in}}{\pgfqpoint{5.700000in}{5.700000in}}%
\pgfusepath{clip}%
\pgfsetbuttcap%
\pgfsetroundjoin%
\definecolor{currentfill}{rgb}{0.279566,0.067836,0.391917}%
\pgfsetfillcolor{currentfill}%
\pgfsetfillopacity{0.700000}%
\pgfsetlinewidth{0.000000pt}%
\definecolor{currentstroke}{rgb}{0.000000,0.000000,0.000000}%
\pgfsetstrokecolor{currentstroke}%
\pgfsetdash{}{0pt}%
\pgfpathmoveto{\pgfqpoint{3.695727in}{1.578052in}}%
\pgfpathlineto{\pgfqpoint{3.709759in}{1.571013in}}%
\pgfpathlineto{\pgfqpoint{3.723795in}{1.564077in}}%
\pgfpathlineto{\pgfqpoint{3.737835in}{1.557244in}}%
\pgfpathlineto{\pgfqpoint{3.751879in}{1.550514in}}%
\pgfpathlineto{\pgfqpoint{3.743567in}{1.548348in}}%
\pgfpathlineto{\pgfqpoint{3.735245in}{1.546454in}}%
\pgfpathlineto{\pgfqpoint{3.726912in}{1.544839in}}%
\pgfpathlineto{\pgfqpoint{3.718568in}{1.543508in}}%
\pgfpathlineto{\pgfqpoint{3.704497in}{1.550802in}}%
\pgfpathlineto{\pgfqpoint{3.690430in}{1.558199in}}%
\pgfpathlineto{\pgfqpoint{3.676367in}{1.565699in}}%
\pgfpathlineto{\pgfqpoint{3.662308in}{1.573303in}}%
\pgfpathlineto{\pgfqpoint{3.670679in}{1.574062in}}%
\pgfpathlineto{\pgfqpoint{3.679040in}{1.575110in}}%
\pgfpathlineto{\pgfqpoint{3.687389in}{1.576442in}}%
\pgfpathlineto{\pgfqpoint{3.695727in}{1.578052in}}%
\pgfpathclose%
\pgfusepath{fill}%
\end{pgfscope}%
\begin{pgfscope}%
\pgfpathrectangle{\pgfqpoint{1.150000in}{0.150000in}}{\pgfqpoint{5.700000in}{5.700000in}}%
\pgfusepath{clip}%
\pgfsetbuttcap%
\pgfsetroundjoin%
\definecolor{currentfill}{rgb}{0.277941,0.056324,0.381191}%
\pgfsetfillcolor{currentfill}%
\pgfsetfillopacity{0.700000}%
\pgfsetlinewidth{0.000000pt}%
\definecolor{currentstroke}{rgb}{0.000000,0.000000,0.000000}%
\pgfsetstrokecolor{currentstroke}%
\pgfsetdash{}{0pt}%
\pgfpathmoveto{\pgfqpoint{4.366021in}{1.561140in}}%
\pgfpathlineto{\pgfqpoint{4.380209in}{1.560211in}}%
\pgfpathlineto{\pgfqpoint{4.394406in}{1.559379in}}%
\pgfpathlineto{\pgfqpoint{4.408613in}{1.558643in}}%
\pgfpathlineto{\pgfqpoint{4.422828in}{1.558004in}}%
\pgfpathlineto{\pgfqpoint{4.414793in}{1.547355in}}%
\pgfpathlineto{\pgfqpoint{4.406754in}{1.536833in}}%
\pgfpathlineto{\pgfqpoint{4.398710in}{1.526443in}}%
\pgfpathlineto{\pgfqpoint{4.390661in}{1.516191in}}%
\pgfpathlineto{\pgfqpoint{4.376438in}{1.517301in}}%
\pgfpathlineto{\pgfqpoint{4.362225in}{1.518507in}}%
\pgfpathlineto{\pgfqpoint{4.348020in}{1.519809in}}%
\pgfpathlineto{\pgfqpoint{4.333824in}{1.521209in}}%
\pgfpathlineto{\pgfqpoint{4.341881in}{1.530984in}}%
\pgfpathlineto{\pgfqpoint{4.349932in}{1.540900in}}%
\pgfpathlineto{\pgfqpoint{4.357979in}{1.550954in}}%
\pgfpathlineto{\pgfqpoint{4.366021in}{1.561140in}}%
\pgfpathclose%
\pgfusepath{fill}%
\end{pgfscope}%
\begin{pgfscope}%
\pgfpathrectangle{\pgfqpoint{1.150000in}{0.150000in}}{\pgfqpoint{5.700000in}{5.700000in}}%
\pgfusepath{clip}%
\pgfsetbuttcap%
\pgfsetroundjoin%
\definecolor{currentfill}{rgb}{0.221989,0.339161,0.548752}%
\pgfsetfillcolor{currentfill}%
\pgfsetfillopacity{0.700000}%
\pgfsetlinewidth{0.000000pt}%
\definecolor{currentstroke}{rgb}{0.000000,0.000000,0.000000}%
\pgfsetstrokecolor{currentstroke}%
\pgfsetdash{}{0pt}%
\pgfpathmoveto{\pgfqpoint{5.142765in}{2.168651in}}%
\pgfpathlineto{\pgfqpoint{5.157321in}{2.174578in}}%
\pgfpathlineto{\pgfqpoint{5.171891in}{2.180604in}}%
\pgfpathlineto{\pgfqpoint{5.186476in}{2.186729in}}%
\pgfpathlineto{\pgfqpoint{5.201074in}{2.192953in}}%
\pgfpathlineto{\pgfqpoint{5.193205in}{2.178069in}}%
\pgfpathlineto{\pgfqpoint{5.185331in}{2.163133in}}%
\pgfpathlineto{\pgfqpoint{5.177451in}{2.148148in}}%
\pgfpathlineto{\pgfqpoint{5.169567in}{2.133117in}}%
\pgfpathlineto{\pgfqpoint{5.154973in}{2.127191in}}%
\pgfpathlineto{\pgfqpoint{5.140393in}{2.121363in}}%
\pgfpathlineto{\pgfqpoint{5.125827in}{2.115634in}}%
\pgfpathlineto{\pgfqpoint{5.111275in}{2.110003in}}%
\pgfpathlineto{\pgfqpoint{5.119155in}{2.124730in}}%
\pgfpathlineto{\pgfqpoint{5.127030in}{2.139415in}}%
\pgfpathlineto{\pgfqpoint{5.134900in}{2.154056in}}%
\pgfpathlineto{\pgfqpoint{5.142765in}{2.168651in}}%
\pgfpathclose%
\pgfusepath{fill}%
\end{pgfscope}%
\begin{pgfscope}%
\pgfpathrectangle{\pgfqpoint{1.150000in}{0.150000in}}{\pgfqpoint{5.700000in}{5.700000in}}%
\pgfusepath{clip}%
\pgfsetbuttcap%
\pgfsetroundjoin%
\definecolor{currentfill}{rgb}{0.260571,0.246922,0.522828}%
\pgfsetfillcolor{currentfill}%
\pgfsetfillopacity{0.700000}%
\pgfsetlinewidth{0.000000pt}%
\definecolor{currentstroke}{rgb}{0.000000,0.000000,0.000000}%
\pgfsetstrokecolor{currentstroke}%
\pgfsetdash{}{0pt}%
\pgfpathmoveto{\pgfqpoint{3.123092in}{1.940506in}}%
\pgfpathlineto{\pgfqpoint{3.137116in}{1.928307in}}%
\pgfpathlineto{\pgfqpoint{3.151140in}{1.916229in}}%
\pgfpathlineto{\pgfqpoint{3.165164in}{1.904271in}}%
\pgfpathlineto{\pgfqpoint{3.179189in}{1.892431in}}%
\pgfpathlineto{\pgfqpoint{3.170459in}{1.898040in}}%
\pgfpathlineto{\pgfqpoint{3.161712in}{1.904019in}}%
\pgfpathlineto{\pgfqpoint{3.152948in}{1.910376in}}%
\pgfpathlineto{\pgfqpoint{3.144164in}{1.917116in}}%
\pgfpathlineto{\pgfqpoint{3.130095in}{1.929575in}}%
\pgfpathlineto{\pgfqpoint{3.116026in}{1.942153in}}%
\pgfpathlineto{\pgfqpoint{3.101956in}{1.954852in}}%
\pgfpathlineto{\pgfqpoint{3.087886in}{1.967671in}}%
\pgfpathlineto{\pgfqpoint{3.096716in}{1.960301in}}%
\pgfpathlineto{\pgfqpoint{3.105527in}{1.953322in}}%
\pgfpathlineto{\pgfqpoint{3.114319in}{1.946726in}}%
\pgfpathlineto{\pgfqpoint{3.123092in}{1.940506in}}%
\pgfpathclose%
\pgfusepath{fill}%
\end{pgfscope}%
\begin{pgfscope}%
\pgfpathrectangle{\pgfqpoint{1.150000in}{0.150000in}}{\pgfqpoint{5.700000in}{5.700000in}}%
\pgfusepath{clip}%
\pgfsetbuttcap%
\pgfsetroundjoin%
\definecolor{currentfill}{rgb}{0.280868,0.160771,0.472899}%
\pgfsetfillcolor{currentfill}%
\pgfsetfillopacity{0.700000}%
\pgfsetlinewidth{0.000000pt}%
\definecolor{currentstroke}{rgb}{0.000000,0.000000,0.000000}%
\pgfsetstrokecolor{currentstroke}%
\pgfsetdash{}{0pt}%
\pgfpathmoveto{\pgfqpoint{4.721962in}{1.762596in}}%
\pgfpathlineto{\pgfqpoint{4.736299in}{1.764920in}}%
\pgfpathlineto{\pgfqpoint{4.750648in}{1.767340in}}%
\pgfpathlineto{\pgfqpoint{4.765008in}{1.769858in}}%
\pgfpathlineto{\pgfqpoint{4.779379in}{1.772471in}}%
\pgfpathlineto{\pgfqpoint{4.771420in}{1.758779in}}%
\pgfpathlineto{\pgfqpoint{4.763457in}{1.745131in}}%
\pgfpathlineto{\pgfqpoint{4.755491in}{1.731530in}}%
\pgfpathlineto{\pgfqpoint{4.747521in}{1.717980in}}%
\pgfpathlineto{\pgfqpoint{4.733149in}{1.715769in}}%
\pgfpathlineto{\pgfqpoint{4.718790in}{1.713654in}}%
\pgfpathlineto{\pgfqpoint{4.704441in}{1.711636in}}%
\pgfpathlineto{\pgfqpoint{4.690104in}{1.709714in}}%
\pgfpathlineto{\pgfqpoint{4.698074in}{1.722855in}}%
\pgfpathlineto{\pgfqpoint{4.706041in}{1.736051in}}%
\pgfpathlineto{\pgfqpoint{4.714003in}{1.749299in}}%
\pgfpathlineto{\pgfqpoint{4.721962in}{1.762596in}}%
\pgfpathclose%
\pgfusepath{fill}%
\end{pgfscope}%
\begin{pgfscope}%
\pgfpathrectangle{\pgfqpoint{1.150000in}{0.150000in}}{\pgfqpoint{5.700000in}{5.700000in}}%
\pgfusepath{clip}%
\pgfsetbuttcap%
\pgfsetroundjoin%
\definecolor{currentfill}{rgb}{0.160665,0.478540,0.558115}%
\pgfsetfillcolor{currentfill}%
\pgfsetfillopacity{0.700000}%
\pgfsetlinewidth{0.000000pt}%
\definecolor{currentstroke}{rgb}{0.000000,0.000000,0.000000}%
\pgfsetstrokecolor{currentstroke}%
\pgfsetdash{}{0pt}%
\pgfpathmoveto{\pgfqpoint{5.474793in}{2.539181in}}%
\pgfpathlineto{\pgfqpoint{5.489550in}{2.547510in}}%
\pgfpathlineto{\pgfqpoint{5.504324in}{2.555941in}}%
\pgfpathlineto{\pgfqpoint{5.519113in}{2.564472in}}%
\pgfpathlineto{\pgfqpoint{5.533919in}{2.573105in}}%
\pgfpathlineto{\pgfqpoint{5.526156in}{2.559147in}}%
\pgfpathlineto{\pgfqpoint{5.518385in}{2.545080in}}%
\pgfpathlineto{\pgfqpoint{5.510608in}{2.530905in}}%
\pgfpathlineto{\pgfqpoint{5.502824in}{2.516624in}}%
\pgfpathlineto{\pgfqpoint{5.488022in}{2.508196in}}%
\pgfpathlineto{\pgfqpoint{5.473236in}{2.499869in}}%
\pgfpathlineto{\pgfqpoint{5.458466in}{2.491642in}}%
\pgfpathlineto{\pgfqpoint{5.443713in}{2.483517in}}%
\pgfpathlineto{\pgfqpoint{5.451493in}{2.497586in}}%
\pgfpathlineto{\pgfqpoint{5.459267in}{2.511554in}}%
\pgfpathlineto{\pgfqpoint{5.467033in}{2.525420in}}%
\pgfpathlineto{\pgfqpoint{5.474793in}{2.539181in}}%
\pgfpathclose%
\pgfusepath{fill}%
\end{pgfscope}%
\begin{pgfscope}%
\pgfpathrectangle{\pgfqpoint{1.150000in}{0.150000in}}{\pgfqpoint{5.700000in}{5.700000in}}%
\pgfusepath{clip}%
\pgfsetbuttcap%
\pgfsetroundjoin%
\definecolor{currentfill}{rgb}{0.119483,0.614817,0.537692}%
\pgfsetfillcolor{currentfill}%
\pgfsetfillopacity{0.700000}%
\pgfsetlinewidth{0.000000pt}%
\definecolor{currentstroke}{rgb}{0.000000,0.000000,0.000000}%
\pgfsetstrokecolor{currentstroke}%
\pgfsetdash{}{0pt}%
\pgfpathmoveto{\pgfqpoint{2.277072in}{2.922352in}}%
\pgfpathlineto{\pgfqpoint{2.291329in}{2.901350in}}%
\pgfpathlineto{\pgfqpoint{2.305578in}{2.880530in}}%
\pgfpathlineto{\pgfqpoint{2.319819in}{2.859891in}}%
\pgfpathlineto{\pgfqpoint{2.334051in}{2.839431in}}%
\pgfpathlineto{\pgfqpoint{2.324486in}{2.854529in}}%
\pgfpathlineto{\pgfqpoint{2.314892in}{2.870093in}}%
\pgfpathlineto{\pgfqpoint{2.305266in}{2.886128in}}%
\pgfpathlineto{\pgfqpoint{2.295609in}{2.902643in}}%
\pgfpathlineto{\pgfqpoint{2.281306in}{2.923778in}}%
\pgfpathlineto{\pgfqpoint{2.266994in}{2.945092in}}%
\pgfpathlineto{\pgfqpoint{2.252674in}{2.966590in}}%
\pgfpathlineto{\pgfqpoint{2.238345in}{2.988271in}}%
\pgfpathlineto{\pgfqpoint{2.248074in}{2.971068in}}%
\pgfpathlineto{\pgfqpoint{2.257772in}{2.954353in}}%
\pgfpathlineto{\pgfqpoint{2.267437in}{2.938117in}}%
\pgfpathlineto{\pgfqpoint{2.277072in}{2.922352in}}%
\pgfpathclose%
\pgfusepath{fill}%
\end{pgfscope}%
\begin{pgfscope}%
\pgfpathrectangle{\pgfqpoint{1.150000in}{0.150000in}}{\pgfqpoint{5.700000in}{5.700000in}}%
\pgfusepath{clip}%
\pgfsetbuttcap%
\pgfsetroundjoin%
\definecolor{currentfill}{rgb}{0.283229,0.120777,0.440584}%
\pgfsetfillcolor{currentfill}%
\pgfsetfillopacity{0.700000}%
\pgfsetlinewidth{0.000000pt}%
\definecolor{currentstroke}{rgb}{0.000000,0.000000,0.000000}%
\pgfsetstrokecolor{currentstroke}%
\pgfsetdash{}{0pt}%
\pgfpathmoveto{\pgfqpoint{3.493873in}{1.672759in}}%
\pgfpathlineto{\pgfqpoint{3.507892in}{1.663884in}}%
\pgfpathlineto{\pgfqpoint{3.521914in}{1.655118in}}%
\pgfpathlineto{\pgfqpoint{3.535939in}{1.646458in}}%
\pgfpathlineto{\pgfqpoint{3.549966in}{1.637906in}}%
\pgfpathlineto{\pgfqpoint{3.541523in}{1.638600in}}%
\pgfpathlineto{\pgfqpoint{3.533067in}{1.639605in}}%
\pgfpathlineto{\pgfqpoint{3.524597in}{1.640929in}}%
\pgfpathlineto{\pgfqpoint{3.516115in}{1.642578in}}%
\pgfpathlineto{\pgfqpoint{3.502054in}{1.651717in}}%
\pgfpathlineto{\pgfqpoint{3.487996in}{1.660964in}}%
\pgfpathlineto{\pgfqpoint{3.473940in}{1.670318in}}%
\pgfpathlineto{\pgfqpoint{3.459887in}{1.679781in}}%
\pgfpathlineto{\pgfqpoint{3.468404in}{1.677537in}}%
\pgfpathlineto{\pgfqpoint{3.476907in}{1.675623in}}%
\pgfpathlineto{\pgfqpoint{3.485397in}{1.674032in}}%
\pgfpathlineto{\pgfqpoint{3.493873in}{1.672759in}}%
\pgfpathclose%
\pgfusepath{fill}%
\end{pgfscope}%
\begin{pgfscope}%
\pgfpathrectangle{\pgfqpoint{1.150000in}{0.150000in}}{\pgfqpoint{5.700000in}{5.700000in}}%
\pgfusepath{clip}%
\pgfsetbuttcap%
\pgfsetroundjoin%
\definecolor{currentfill}{rgb}{0.276022,0.044167,0.370164}%
\pgfsetfillcolor{currentfill}%
\pgfsetfillopacity{0.700000}%
\pgfsetlinewidth{0.000000pt}%
\definecolor{currentstroke}{rgb}{0.000000,0.000000,0.000000}%
\pgfsetstrokecolor{currentstroke}%
\pgfsetdash{}{0pt}%
\pgfpathmoveto{\pgfqpoint{4.277125in}{1.527772in}}%
\pgfpathlineto{\pgfqpoint{4.291287in}{1.525986in}}%
\pgfpathlineto{\pgfqpoint{4.305457in}{1.524297in}}%
\pgfpathlineto{\pgfqpoint{4.319636in}{1.522704in}}%
\pgfpathlineto{\pgfqpoint{4.333824in}{1.521209in}}%
\pgfpathlineto{\pgfqpoint{4.325762in}{1.511580in}}%
\pgfpathlineto{\pgfqpoint{4.317695in}{1.502102in}}%
\pgfpathlineto{\pgfqpoint{4.309623in}{1.492780in}}%
\pgfpathlineto{\pgfqpoint{4.301546in}{1.483618in}}%
\pgfpathlineto{\pgfqpoint{4.287349in}{1.485602in}}%
\pgfpathlineto{\pgfqpoint{4.273160in}{1.487683in}}%
\pgfpathlineto{\pgfqpoint{4.258980in}{1.489860in}}%
\pgfpathlineto{\pgfqpoint{4.244807in}{1.492134in}}%
\pgfpathlineto{\pgfqpoint{4.252895in}{1.500801in}}%
\pgfpathlineto{\pgfqpoint{4.260977in}{1.509632in}}%
\pgfpathlineto{\pgfqpoint{4.269053in}{1.518624in}}%
\pgfpathlineto{\pgfqpoint{4.277125in}{1.527772in}}%
\pgfpathclose%
\pgfusepath{fill}%
\end{pgfscope}%
\begin{pgfscope}%
\pgfpathrectangle{\pgfqpoint{1.150000in}{0.150000in}}{\pgfqpoint{5.700000in}{5.700000in}}%
\pgfusepath{clip}%
\pgfsetbuttcap%
\pgfsetroundjoin%
\definecolor{currentfill}{rgb}{0.274952,0.037752,0.364543}%
\pgfsetfillcolor{currentfill}%
\pgfsetfillopacity{0.700000}%
\pgfsetlinewidth{0.000000pt}%
\definecolor{currentstroke}{rgb}{0.000000,0.000000,0.000000}%
\pgfsetstrokecolor{currentstroke}%
\pgfsetdash{}{0pt}%
\pgfpathmoveto{\pgfqpoint{3.897374in}{1.515979in}}%
\pgfpathlineto{\pgfqpoint{3.911442in}{1.510706in}}%
\pgfpathlineto{\pgfqpoint{3.925515in}{1.505534in}}%
\pgfpathlineto{\pgfqpoint{3.939593in}{1.500462in}}%
\pgfpathlineto{\pgfqpoint{3.953678in}{1.495490in}}%
\pgfpathlineto{\pgfqpoint{3.945469in}{1.490680in}}%
\pgfpathlineto{\pgfqpoint{3.937253in}{1.486106in}}%
\pgfpathlineto{\pgfqpoint{3.929028in}{1.481773in}}%
\pgfpathlineto{\pgfqpoint{3.920795in}{1.477687in}}%
\pgfpathlineto{\pgfqpoint{3.906689in}{1.483202in}}%
\pgfpathlineto{\pgfqpoint{3.892590in}{1.488818in}}%
\pgfpathlineto{\pgfqpoint{3.878496in}{1.494533in}}%
\pgfpathlineto{\pgfqpoint{3.864407in}{1.500349in}}%
\pgfpathlineto{\pgfqpoint{3.872662in}{1.503884in}}%
\pgfpathlineto{\pgfqpoint{3.880908in}{1.507672in}}%
\pgfpathlineto{\pgfqpoint{3.889146in}{1.511705in}}%
\pgfpathlineto{\pgfqpoint{3.897374in}{1.515979in}}%
\pgfpathclose%
\pgfusepath{fill}%
\end{pgfscope}%
\begin{pgfscope}%
\pgfpathrectangle{\pgfqpoint{1.150000in}{0.150000in}}{\pgfqpoint{5.700000in}{5.700000in}}%
\pgfusepath{clip}%
\pgfsetbuttcap%
\pgfsetroundjoin%
\definecolor{currentfill}{rgb}{0.272594,0.025563,0.353093}%
\pgfsetfillcolor{currentfill}%
\pgfsetfillopacity{0.700000}%
\pgfsetlinewidth{0.000000pt}%
\definecolor{currentstroke}{rgb}{0.000000,0.000000,0.000000}%
\pgfsetstrokecolor{currentstroke}%
\pgfsetdash{}{0pt}%
\pgfpathmoveto{\pgfqpoint{4.042761in}{1.500176in}}%
\pgfpathlineto{\pgfqpoint{4.056859in}{1.496224in}}%
\pgfpathlineto{\pgfqpoint{4.070964in}{1.492370in}}%
\pgfpathlineto{\pgfqpoint{4.085076in}{1.488615in}}%
\pgfpathlineto{\pgfqpoint{4.099194in}{1.484958in}}%
\pgfpathlineto{\pgfqpoint{4.091050in}{1.478226in}}%
\pgfpathlineto{\pgfqpoint{4.082899in}{1.471699in}}%
\pgfpathlineto{\pgfqpoint{4.074741in}{1.465382in}}%
\pgfpathlineto{\pgfqpoint{4.066576in}{1.459280in}}%
\pgfpathlineto{\pgfqpoint{4.052441in}{1.463461in}}%
\pgfpathlineto{\pgfqpoint{4.038313in}{1.467740in}}%
\pgfpathlineto{\pgfqpoint{4.024192in}{1.472118in}}%
\pgfpathlineto{\pgfqpoint{4.010077in}{1.476595in}}%
\pgfpathlineto{\pgfqpoint{4.018259in}{1.482166in}}%
\pgfpathlineto{\pgfqpoint{4.026433in}{1.487956in}}%
\pgfpathlineto{\pgfqpoint{4.034601in}{1.493962in}}%
\pgfpathlineto{\pgfqpoint{4.042761in}{1.500176in}}%
\pgfpathclose%
\pgfusepath{fill}%
\end{pgfscope}%
\begin{pgfscope}%
\pgfpathrectangle{\pgfqpoint{1.150000in}{0.150000in}}{\pgfqpoint{5.700000in}{5.700000in}}%
\pgfusepath{clip}%
\pgfsetbuttcap%
\pgfsetroundjoin%
\definecolor{currentfill}{rgb}{0.266580,0.228262,0.514349}%
\pgfsetfillcolor{currentfill}%
\pgfsetfillopacity{0.700000}%
\pgfsetlinewidth{0.000000pt}%
\definecolor{currentstroke}{rgb}{0.000000,0.000000,0.000000}%
\pgfsetstrokecolor{currentstroke}%
\pgfsetdash{}{0pt}%
\pgfpathmoveto{\pgfqpoint{3.179189in}{1.892431in}}%
\pgfpathlineto{\pgfqpoint{3.193213in}{1.880710in}}%
\pgfpathlineto{\pgfqpoint{3.207238in}{1.869106in}}%
\pgfpathlineto{\pgfqpoint{3.221264in}{1.857620in}}%
\pgfpathlineto{\pgfqpoint{3.235291in}{1.846250in}}%
\pgfpathlineto{\pgfqpoint{3.226604in}{1.851249in}}%
\pgfpathlineto{\pgfqpoint{3.217901in}{1.856614in}}%
\pgfpathlineto{\pgfqpoint{3.209180in}{1.862350in}}%
\pgfpathlineto{\pgfqpoint{3.200441in}{1.868464in}}%
\pgfpathlineto{\pgfqpoint{3.186372in}{1.880450in}}%
\pgfpathlineto{\pgfqpoint{3.172303in}{1.892554in}}%
\pgfpathlineto{\pgfqpoint{3.158233in}{1.904776in}}%
\pgfpathlineto{\pgfqpoint{3.144164in}{1.917116in}}%
\pgfpathlineto{\pgfqpoint{3.152948in}{1.910376in}}%
\pgfpathlineto{\pgfqpoint{3.161712in}{1.904019in}}%
\pgfpathlineto{\pgfqpoint{3.170459in}{1.898040in}}%
\pgfpathlineto{\pgfqpoint{3.179189in}{1.892431in}}%
\pgfpathclose%
\pgfusepath{fill}%
\end{pgfscope}%
\begin{pgfscope}%
\pgfpathrectangle{\pgfqpoint{1.150000in}{0.150000in}}{\pgfqpoint{5.700000in}{5.700000in}}%
\pgfusepath{clip}%
\pgfsetbuttcap%
\pgfsetroundjoin%
\definecolor{currentfill}{rgb}{0.246811,0.283237,0.535941}%
\pgfsetfillcolor{currentfill}%
\pgfsetfillopacity{0.700000}%
\pgfsetlinewidth{0.000000pt}%
\definecolor{currentstroke}{rgb}{0.000000,0.000000,0.000000}%
\pgfsetstrokecolor{currentstroke}%
\pgfsetdash{}{0pt}%
\pgfpathmoveto{\pgfqpoint{5.021656in}{2.030456in}}%
\pgfpathlineto{\pgfqpoint{5.036149in}{2.035380in}}%
\pgfpathlineto{\pgfqpoint{5.050656in}{2.040401in}}%
\pgfpathlineto{\pgfqpoint{5.065176in}{2.045521in}}%
\pgfpathlineto{\pgfqpoint{5.079710in}{2.050738in}}%
\pgfpathlineto{\pgfqpoint{5.071807in}{2.035847in}}%
\pgfpathlineto{\pgfqpoint{5.063901in}{2.020932in}}%
\pgfpathlineto{\pgfqpoint{5.055990in}{2.005996in}}%
\pgfpathlineto{\pgfqpoint{5.048074in}{1.991042in}}%
\pgfpathlineto{\pgfqpoint{5.033544in}{1.986157in}}%
\pgfpathlineto{\pgfqpoint{5.019027in}{1.981371in}}%
\pgfpathlineto{\pgfqpoint{5.004524in}{1.976682in}}%
\pgfpathlineto{\pgfqpoint{4.990034in}{1.972091in}}%
\pgfpathlineto{\pgfqpoint{4.997946in}{1.986705in}}%
\pgfpathlineto{\pgfqpoint{5.005853in}{2.001306in}}%
\pgfpathlineto{\pgfqpoint{5.013757in}{2.015891in}}%
\pgfpathlineto{\pgfqpoint{5.021656in}{2.030456in}}%
\pgfpathclose%
\pgfusepath{fill}%
\end{pgfscope}%
\begin{pgfscope}%
\pgfpathrectangle{\pgfqpoint{1.150000in}{0.150000in}}{\pgfqpoint{5.700000in}{5.700000in}}%
\pgfusepath{clip}%
\pgfsetbuttcap%
\pgfsetroundjoin%
\definecolor{currentfill}{rgb}{0.180629,0.429975,0.557282}%
\pgfsetfillcolor{currentfill}%
\pgfsetfillopacity{0.700000}%
\pgfsetlinewidth{0.000000pt}%
\definecolor{currentstroke}{rgb}{0.000000,0.000000,0.000000}%
\pgfsetstrokecolor{currentstroke}%
\pgfsetdash{}{0pt}%
\pgfpathmoveto{\pgfqpoint{5.353688in}{2.395660in}}%
\pgfpathlineto{\pgfqpoint{5.368374in}{2.403161in}}%
\pgfpathlineto{\pgfqpoint{5.383076in}{2.410763in}}%
\pgfpathlineto{\pgfqpoint{5.397794in}{2.418465in}}%
\pgfpathlineto{\pgfqpoint{5.412527in}{2.426267in}}%
\pgfpathlineto{\pgfqpoint{5.404715in}{2.411721in}}%
\pgfpathlineto{\pgfqpoint{5.396897in}{2.397086in}}%
\pgfpathlineto{\pgfqpoint{5.389073in}{2.382363in}}%
\pgfpathlineto{\pgfqpoint{5.381243in}{2.367556in}}%
\pgfpathlineto{\pgfqpoint{5.366514in}{2.359996in}}%
\pgfpathlineto{\pgfqpoint{5.351801in}{2.352537in}}%
\pgfpathlineto{\pgfqpoint{5.337103in}{2.345177in}}%
\pgfpathlineto{\pgfqpoint{5.322421in}{2.337918in}}%
\pgfpathlineto{\pgfqpoint{5.330246in}{2.352476in}}%
\pgfpathlineto{\pgfqpoint{5.338066in}{2.366954in}}%
\pgfpathlineto{\pgfqpoint{5.345880in}{2.381349in}}%
\pgfpathlineto{\pgfqpoint{5.353688in}{2.395660in}}%
\pgfpathclose%
\pgfusepath{fill}%
\end{pgfscope}%
\begin{pgfscope}%
\pgfpathrectangle{\pgfqpoint{1.150000in}{0.150000in}}{\pgfqpoint{5.700000in}{5.700000in}}%
\pgfusepath{clip}%
\pgfsetbuttcap%
\pgfsetroundjoin%
\definecolor{currentfill}{rgb}{0.275191,0.194905,0.496005}%
\pgfsetfillcolor{currentfill}%
\pgfsetfillopacity{0.700000}%
\pgfsetlinewidth{0.000000pt}%
\definecolor{currentstroke}{rgb}{0.000000,0.000000,0.000000}%
\pgfsetstrokecolor{currentstroke}%
\pgfsetdash{}{0pt}%
\pgfpathmoveto{\pgfqpoint{4.811177in}{1.827603in}}%
\pgfpathlineto{\pgfqpoint{4.825561in}{1.830698in}}%
\pgfpathlineto{\pgfqpoint{4.839957in}{1.833890in}}%
\pgfpathlineto{\pgfqpoint{4.854366in}{1.837180in}}%
\pgfpathlineto{\pgfqpoint{4.868787in}{1.840566in}}%
\pgfpathlineto{\pgfqpoint{4.860842in}{1.826357in}}%
\pgfpathlineto{\pgfqpoint{4.852893in}{1.812173in}}%
\pgfpathlineto{\pgfqpoint{4.844940in}{1.798017in}}%
\pgfpathlineto{\pgfqpoint{4.836984in}{1.783893in}}%
\pgfpathlineto{\pgfqpoint{4.822565in}{1.780893in}}%
\pgfpathlineto{\pgfqpoint{4.808158in}{1.777989in}}%
\pgfpathlineto{\pgfqpoint{4.793763in}{1.775182in}}%
\pgfpathlineto{\pgfqpoint{4.779379in}{1.772471in}}%
\pgfpathlineto{\pgfqpoint{4.787334in}{1.786203in}}%
\pgfpathlineto{\pgfqpoint{4.795286in}{1.799972in}}%
\pgfpathlineto{\pgfqpoint{4.803233in}{1.813773in}}%
\pgfpathlineto{\pgfqpoint{4.811177in}{1.827603in}}%
\pgfpathclose%
\pgfusepath{fill}%
\end{pgfscope}%
\begin{pgfscope}%
\pgfpathrectangle{\pgfqpoint{1.150000in}{0.150000in}}{\pgfqpoint{5.700000in}{5.700000in}}%
\pgfusepath{clip}%
\pgfsetbuttcap%
\pgfsetroundjoin%
\definecolor{currentfill}{rgb}{0.128087,0.647749,0.523491}%
\pgfsetfillcolor{currentfill}%
\pgfsetfillopacity{0.700000}%
\pgfsetlinewidth{0.000000pt}%
\definecolor{currentstroke}{rgb}{0.000000,0.000000,0.000000}%
\pgfsetstrokecolor{currentstroke}%
\pgfsetdash{}{0pt}%
\pgfpathmoveto{\pgfqpoint{2.219954in}{3.008221in}}%
\pgfpathlineto{\pgfqpoint{2.234247in}{2.986471in}}%
\pgfpathlineto{\pgfqpoint{2.248531in}{2.964911in}}%
\pgfpathlineto{\pgfqpoint{2.262806in}{2.943539in}}%
\pgfpathlineto{\pgfqpoint{2.277072in}{2.922352in}}%
\pgfpathlineto{\pgfqpoint{2.267437in}{2.938117in}}%
\pgfpathlineto{\pgfqpoint{2.257772in}{2.954353in}}%
\pgfpathlineto{\pgfqpoint{2.248074in}{2.971068in}}%
\pgfpathlineto{\pgfqpoint{2.238345in}{2.988271in}}%
\pgfpathlineto{\pgfqpoint{2.224006in}{3.010137in}}%
\pgfpathlineto{\pgfqpoint{2.209658in}{3.032191in}}%
\pgfpathlineto{\pgfqpoint{2.195301in}{3.054434in}}%
\pgfpathlineto{\pgfqpoint{2.180933in}{3.076868in}}%
\pgfpathlineto{\pgfqpoint{2.190738in}{3.058973in}}%
\pgfpathlineto{\pgfqpoint{2.200509in}{3.041572in}}%
\pgfpathlineto{\pgfqpoint{2.210247in}{3.024657in}}%
\pgfpathlineto{\pgfqpoint{2.219954in}{3.008221in}}%
\pgfpathclose%
\pgfusepath{fill}%
\end{pgfscope}%
\begin{pgfscope}%
\pgfpathrectangle{\pgfqpoint{1.150000in}{0.150000in}}{\pgfqpoint{5.700000in}{5.700000in}}%
\pgfusepath{clip}%
\pgfsetbuttcap%
\pgfsetroundjoin%
\definecolor{currentfill}{rgb}{0.278791,0.062145,0.386592}%
\pgfsetfillcolor{currentfill}%
\pgfsetfillopacity{0.700000}%
\pgfsetlinewidth{0.000000pt}%
\definecolor{currentstroke}{rgb}{0.000000,0.000000,0.000000}%
\pgfsetstrokecolor{currentstroke}%
\pgfsetdash{}{0pt}%
\pgfpathmoveto{\pgfqpoint{3.751879in}{1.550514in}}%
\pgfpathlineto{\pgfqpoint{3.765928in}{1.543887in}}%
\pgfpathlineto{\pgfqpoint{3.779982in}{1.537363in}}%
\pgfpathlineto{\pgfqpoint{3.794041in}{1.530940in}}%
\pgfpathlineto{\pgfqpoint{3.808104in}{1.524619in}}%
\pgfpathlineto{\pgfqpoint{3.799817in}{1.521896in}}%
\pgfpathlineto{\pgfqpoint{3.791519in}{1.519441in}}%
\pgfpathlineto{\pgfqpoint{3.783213in}{1.517260in}}%
\pgfpathlineto{\pgfqpoint{3.774896in}{1.515358in}}%
\pgfpathlineto{\pgfqpoint{3.760807in}{1.522242in}}%
\pgfpathlineto{\pgfqpoint{3.746723in}{1.529229in}}%
\pgfpathlineto{\pgfqpoint{3.732644in}{1.536317in}}%
\pgfpathlineto{\pgfqpoint{3.718568in}{1.543508in}}%
\pgfpathlineto{\pgfqpoint{3.726912in}{1.544839in}}%
\pgfpathlineto{\pgfqpoint{3.735245in}{1.546454in}}%
\pgfpathlineto{\pgfqpoint{3.743567in}{1.548348in}}%
\pgfpathlineto{\pgfqpoint{3.751879in}{1.550514in}}%
\pgfpathclose%
\pgfusepath{fill}%
\end{pgfscope}%
\begin{pgfscope}%
\pgfpathrectangle{\pgfqpoint{1.150000in}{0.150000in}}{\pgfqpoint{5.700000in}{5.700000in}}%
\pgfusepath{clip}%
\pgfsetbuttcap%
\pgfsetroundjoin%
\definecolor{currentfill}{rgb}{0.273809,0.031497,0.358853}%
\pgfsetfillcolor{currentfill}%
\pgfsetfillopacity{0.700000}%
\pgfsetlinewidth{0.000000pt}%
\definecolor{currentstroke}{rgb}{0.000000,0.000000,0.000000}%
\pgfsetstrokecolor{currentstroke}%
\pgfsetdash{}{0pt}%
\pgfpathmoveto{\pgfqpoint{4.188196in}{1.502201in}}%
\pgfpathlineto{\pgfqpoint{4.202337in}{1.499538in}}%
\pgfpathlineto{\pgfqpoint{4.216486in}{1.496973in}}%
\pgfpathlineto{\pgfqpoint{4.230643in}{1.494505in}}%
\pgfpathlineto{\pgfqpoint{4.244807in}{1.492134in}}%
\pgfpathlineto{\pgfqpoint{4.236714in}{1.483637in}}%
\pgfpathlineto{\pgfqpoint{4.228615in}{1.475316in}}%
\pgfpathlineto{\pgfqpoint{4.220511in}{1.467174in}}%
\pgfpathlineto{\pgfqpoint{4.212400in}{1.459217in}}%
\pgfpathlineto{\pgfqpoint{4.198224in}{1.462094in}}%
\pgfpathlineto{\pgfqpoint{4.184055in}{1.465068in}}%
\pgfpathlineto{\pgfqpoint{4.169893in}{1.468140in}}%
\pgfpathlineto{\pgfqpoint{4.155739in}{1.471308in}}%
\pgfpathlineto{\pgfqpoint{4.163862in}{1.478752in}}%
\pgfpathlineto{\pgfqpoint{4.171979in}{1.486385in}}%
\pgfpathlineto{\pgfqpoint{4.180091in}{1.494203in}}%
\pgfpathlineto{\pgfqpoint{4.188196in}{1.502201in}}%
\pgfpathclose%
\pgfusepath{fill}%
\end{pgfscope}%
\begin{pgfscope}%
\pgfpathrectangle{\pgfqpoint{1.150000in}{0.150000in}}{\pgfqpoint{5.700000in}{5.700000in}}%
\pgfusepath{clip}%
\pgfsetbuttcap%
\pgfsetroundjoin%
\definecolor{currentfill}{rgb}{0.204903,0.375746,0.553533}%
\pgfsetfillcolor{currentfill}%
\pgfsetfillopacity{0.700000}%
\pgfsetlinewidth{0.000000pt}%
\definecolor{currentstroke}{rgb}{0.000000,0.000000,0.000000}%
\pgfsetstrokecolor{currentstroke}%
\pgfsetdash{}{0pt}%
\pgfpathmoveto{\pgfqpoint{5.232501in}{2.251926in}}%
\pgfpathlineto{\pgfqpoint{5.247119in}{2.258527in}}%
\pgfpathlineto{\pgfqpoint{5.261752in}{2.265228in}}%
\pgfpathlineto{\pgfqpoint{5.276399in}{2.272028in}}%
\pgfpathlineto{\pgfqpoint{5.291062in}{2.278927in}}%
\pgfpathlineto{\pgfqpoint{5.283208in}{2.264002in}}%
\pgfpathlineto{\pgfqpoint{5.275350in}{2.249010in}}%
\pgfpathlineto{\pgfqpoint{5.267486in}{2.233954in}}%
\pgfpathlineto{\pgfqpoint{5.259617in}{2.218837in}}%
\pgfpathlineto{\pgfqpoint{5.244959in}{2.212218in}}%
\pgfpathlineto{\pgfqpoint{5.230316in}{2.205697in}}%
\pgfpathlineto{\pgfqpoint{5.215688in}{2.199276in}}%
\pgfpathlineto{\pgfqpoint{5.201074in}{2.192953in}}%
\pgfpathlineto{\pgfqpoint{5.208939in}{2.207784in}}%
\pgfpathlineto{\pgfqpoint{5.216798in}{2.222558in}}%
\pgfpathlineto{\pgfqpoint{5.224652in}{2.237273in}}%
\pgfpathlineto{\pgfqpoint{5.232501in}{2.251926in}}%
\pgfpathclose%
\pgfusepath{fill}%
\end{pgfscope}%
\begin{pgfscope}%
\pgfpathrectangle{\pgfqpoint{1.150000in}{0.150000in}}{\pgfqpoint{5.700000in}{5.700000in}}%
\pgfusepath{clip}%
\pgfsetbuttcap%
\pgfsetroundjoin%
\definecolor{currentfill}{rgb}{0.146180,0.515413,0.556823}%
\pgfsetfillcolor{currentfill}%
\pgfsetfillopacity{0.700000}%
\pgfsetlinewidth{0.000000pt}%
\definecolor{currentstroke}{rgb}{0.000000,0.000000,0.000000}%
\pgfsetstrokecolor{currentstroke}%
\pgfsetdash{}{0pt}%
\pgfpathmoveto{\pgfqpoint{5.564901in}{2.627811in}}%
\pgfpathlineto{\pgfqpoint{5.579727in}{2.636729in}}%
\pgfpathlineto{\pgfqpoint{5.594570in}{2.645750in}}%
\pgfpathlineto{\pgfqpoint{5.609429in}{2.654871in}}%
\pgfpathlineto{\pgfqpoint{5.601692in}{2.641234in}}%
\pgfpathlineto{\pgfqpoint{5.593948in}{2.627477in}}%
\pgfpathlineto{\pgfqpoint{5.586196in}{2.613601in}}%
\pgfpathlineto{\pgfqpoint{5.578437in}{2.599610in}}%
\pgfpathlineto{\pgfqpoint{5.563581in}{2.590674in}}%
\pgfpathlineto{\pgfqpoint{5.548742in}{2.581839in}}%
\pgfpathlineto{\pgfqpoint{5.533919in}{2.573105in}}%
\pgfpathlineto{\pgfqpoint{5.541675in}{2.586952in}}%
\pgfpathlineto{\pgfqpoint{5.549425in}{2.600686in}}%
\pgfpathlineto{\pgfqpoint{5.557167in}{2.614306in}}%
\pgfpathlineto{\pgfqpoint{5.564901in}{2.627811in}}%
\pgfpathclose%
\pgfusepath{fill}%
\end{pgfscope}%
\begin{pgfscope}%
\pgfpathrectangle{\pgfqpoint{1.150000in}{0.150000in}}{\pgfqpoint{5.700000in}{5.700000in}}%
\pgfusepath{clip}%
\pgfsetbuttcap%
\pgfsetroundjoin%
\definecolor{currentfill}{rgb}{0.271828,0.209303,0.504434}%
\pgfsetfillcolor{currentfill}%
\pgfsetfillopacity{0.700000}%
\pgfsetlinewidth{0.000000pt}%
\definecolor{currentstroke}{rgb}{0.000000,0.000000,0.000000}%
\pgfsetstrokecolor{currentstroke}%
\pgfsetdash{}{0pt}%
\pgfpathmoveto{\pgfqpoint{3.235291in}{1.846250in}}%
\pgfpathlineto{\pgfqpoint{3.249318in}{1.834996in}}%
\pgfpathlineto{\pgfqpoint{3.263346in}{1.823858in}}%
\pgfpathlineto{\pgfqpoint{3.277375in}{1.812835in}}%
\pgfpathlineto{\pgfqpoint{3.291405in}{1.801926in}}%
\pgfpathlineto{\pgfqpoint{3.282760in}{1.806319in}}%
\pgfpathlineto{\pgfqpoint{3.274098in}{1.811070in}}%
\pgfpathlineto{\pgfqpoint{3.265420in}{1.816187in}}%
\pgfpathlineto{\pgfqpoint{3.256725in}{1.821677in}}%
\pgfpathlineto{\pgfqpoint{3.242653in}{1.833201in}}%
\pgfpathlineto{\pgfqpoint{3.228582in}{1.844839in}}%
\pgfpathlineto{\pgfqpoint{3.214511in}{1.856593in}}%
\pgfpathlineto{\pgfqpoint{3.200441in}{1.868464in}}%
\pgfpathlineto{\pgfqpoint{3.209180in}{1.862350in}}%
\pgfpathlineto{\pgfqpoint{3.217901in}{1.856614in}}%
\pgfpathlineto{\pgfqpoint{3.226604in}{1.851249in}}%
\pgfpathlineto{\pgfqpoint{3.235291in}{1.846250in}}%
\pgfpathclose%
\pgfusepath{fill}%
\end{pgfscope}%
\begin{pgfscope}%
\pgfpathrectangle{\pgfqpoint{1.150000in}{0.150000in}}{\pgfqpoint{5.700000in}{5.700000in}}%
\pgfusepath{clip}%
\pgfsetbuttcap%
\pgfsetroundjoin%
\definecolor{currentfill}{rgb}{0.283091,0.110553,0.431554}%
\pgfsetfillcolor{currentfill}%
\pgfsetfillopacity{0.700000}%
\pgfsetlinewidth{0.000000pt}%
\definecolor{currentstroke}{rgb}{0.000000,0.000000,0.000000}%
\pgfsetstrokecolor{currentstroke}%
\pgfsetdash{}{0pt}%
\pgfpathmoveto{\pgfqpoint{3.549966in}{1.637906in}}%
\pgfpathlineto{\pgfqpoint{3.563997in}{1.629461in}}%
\pgfpathlineto{\pgfqpoint{3.578031in}{1.621122in}}%
\pgfpathlineto{\pgfqpoint{3.592069in}{1.612889in}}%
\pgfpathlineto{\pgfqpoint{3.606110in}{1.604762in}}%
\pgfpathlineto{\pgfqpoint{3.597697in}{1.604877in}}%
\pgfpathlineto{\pgfqpoint{3.589273in}{1.605298in}}%
\pgfpathlineto{\pgfqpoint{3.580836in}{1.606033in}}%
\pgfpathlineto{\pgfqpoint{3.572387in}{1.607087in}}%
\pgfpathlineto{\pgfqpoint{3.558314in}{1.615800in}}%
\pgfpathlineto{\pgfqpoint{3.544245in}{1.624620in}}%
\pgfpathlineto{\pgfqpoint{3.530179in}{1.633545in}}%
\pgfpathlineto{\pgfqpoint{3.516115in}{1.642578in}}%
\pgfpathlineto{\pgfqpoint{3.524597in}{1.640929in}}%
\pgfpathlineto{\pgfqpoint{3.533067in}{1.639605in}}%
\pgfpathlineto{\pgfqpoint{3.541523in}{1.638600in}}%
\pgfpathlineto{\pgfqpoint{3.549966in}{1.637906in}}%
\pgfpathclose%
\pgfusepath{fill}%
\end{pgfscope}%
\begin{pgfscope}%
\pgfpathrectangle{\pgfqpoint{1.150000in}{0.150000in}}{\pgfqpoint{5.700000in}{5.700000in}}%
\pgfusepath{clip}%
\pgfsetbuttcap%
\pgfsetroundjoin%
\definecolor{currentfill}{rgb}{0.266580,0.228262,0.514349}%
\pgfsetfillcolor{currentfill}%
\pgfsetfillopacity{0.700000}%
\pgfsetlinewidth{0.000000pt}%
\definecolor{currentstroke}{rgb}{0.000000,0.000000,0.000000}%
\pgfsetstrokecolor{currentstroke}%
\pgfsetdash{}{0pt}%
\pgfpathmoveto{\pgfqpoint{4.900528in}{1.897575in}}%
\pgfpathlineto{\pgfqpoint{4.914964in}{1.901427in}}%
\pgfpathlineto{\pgfqpoint{4.929412in}{1.905375in}}%
\pgfpathlineto{\pgfqpoint{4.943873in}{1.909420in}}%
\pgfpathlineto{\pgfqpoint{4.958346in}{1.913563in}}%
\pgfpathlineto{\pgfqpoint{4.950414in}{1.898930in}}%
\pgfpathlineto{\pgfqpoint{4.942478in}{1.884303in}}%
\pgfpathlineto{\pgfqpoint{4.934538in}{1.869685in}}%
\pgfpathlineto{\pgfqpoint{4.926595in}{1.855080in}}%
\pgfpathlineto{\pgfqpoint{4.912124in}{1.851306in}}%
\pgfpathlineto{\pgfqpoint{4.897666in}{1.847629in}}%
\pgfpathlineto{\pgfqpoint{4.883220in}{1.844049in}}%
\pgfpathlineto{\pgfqpoint{4.868787in}{1.840566in}}%
\pgfpathlineto{\pgfqpoint{4.876728in}{1.854796in}}%
\pgfpathlineto{\pgfqpoint{4.884665in}{1.869043in}}%
\pgfpathlineto{\pgfqpoint{4.892599in}{1.883304in}}%
\pgfpathlineto{\pgfqpoint{4.900528in}{1.897575in}}%
\pgfpathclose%
\pgfusepath{fill}%
\end{pgfscope}%
\begin{pgfscope}%
\pgfpathrectangle{\pgfqpoint{1.150000in}{0.150000in}}{\pgfqpoint{5.700000in}{5.700000in}}%
\pgfusepath{clip}%
\pgfsetbuttcap%
\pgfsetroundjoin%
\definecolor{currentfill}{rgb}{0.153894,0.680203,0.504172}%
\pgfsetfillcolor{currentfill}%
\pgfsetfillopacity{0.700000}%
\pgfsetlinewidth{0.000000pt}%
\definecolor{currentstroke}{rgb}{0.000000,0.000000,0.000000}%
\pgfsetstrokecolor{currentstroke}%
\pgfsetdash{}{0pt}%
\pgfpathmoveto{\pgfqpoint{2.162684in}{3.097147in}}%
\pgfpathlineto{\pgfqpoint{2.177016in}{3.074623in}}%
\pgfpathlineto{\pgfqpoint{2.191339in}{3.052295in}}%
\pgfpathlineto{\pgfqpoint{2.205651in}{3.030161in}}%
\pgfpathlineto{\pgfqpoint{2.219954in}{3.008221in}}%
\pgfpathlineto{\pgfqpoint{2.210247in}{3.024657in}}%
\pgfpathlineto{\pgfqpoint{2.200509in}{3.041572in}}%
\pgfpathlineto{\pgfqpoint{2.190738in}{3.058973in}}%
\pgfpathlineto{\pgfqpoint{2.180933in}{3.076868in}}%
\pgfpathlineto{\pgfqpoint{2.166556in}{3.099495in}}%
\pgfpathlineto{\pgfqpoint{2.152168in}{3.122316in}}%
\pgfpathlineto{\pgfqpoint{2.137771in}{3.145333in}}%
\pgfpathlineto{\pgfqpoint{2.123362in}{3.168548in}}%
\pgfpathlineto{\pgfqpoint{2.133243in}{3.149954in}}%
\pgfpathlineto{\pgfqpoint{2.143090in}{3.131860in}}%
\pgfpathlineto{\pgfqpoint{2.152904in}{3.114261in}}%
\pgfpathlineto{\pgfqpoint{2.162684in}{3.097147in}}%
\pgfpathclose%
\pgfusepath{fill}%
\end{pgfscope}%
\begin{pgfscope}%
\pgfpathrectangle{\pgfqpoint{1.150000in}{0.150000in}}{\pgfqpoint{5.700000in}{5.700000in}}%
\pgfusepath{clip}%
\pgfsetbuttcap%
\pgfsetroundjoin%
\definecolor{currentfill}{rgb}{0.229739,0.322361,0.545706}%
\pgfsetfillcolor{currentfill}%
\pgfsetfillopacity{0.700000}%
\pgfsetlinewidth{0.000000pt}%
\definecolor{currentstroke}{rgb}{0.000000,0.000000,0.000000}%
\pgfsetstrokecolor{currentstroke}%
\pgfsetdash{}{0pt}%
\pgfpathmoveto{\pgfqpoint{5.111275in}{2.110003in}}%
\pgfpathlineto{\pgfqpoint{5.125827in}{2.115634in}}%
\pgfpathlineto{\pgfqpoint{5.140393in}{2.121363in}}%
\pgfpathlineto{\pgfqpoint{5.154973in}{2.127191in}}%
\pgfpathlineto{\pgfqpoint{5.169567in}{2.133117in}}%
\pgfpathlineto{\pgfqpoint{5.161679in}{2.118043in}}%
\pgfpathlineto{\pgfqpoint{5.153785in}{2.102929in}}%
\pgfpathlineto{\pgfqpoint{5.145887in}{2.087777in}}%
\pgfpathlineto{\pgfqpoint{5.137985in}{2.072590in}}%
\pgfpathlineto{\pgfqpoint{5.123395in}{2.066980in}}%
\pgfpathlineto{\pgfqpoint{5.108819in}{2.061468in}}%
\pgfpathlineto{\pgfqpoint{5.094258in}{2.056054in}}%
\pgfpathlineto{\pgfqpoint{5.079710in}{2.050738in}}%
\pgfpathlineto{\pgfqpoint{5.087608in}{2.065603in}}%
\pgfpathlineto{\pgfqpoint{5.095502in}{2.080436in}}%
\pgfpathlineto{\pgfqpoint{5.103391in}{2.095238in}}%
\pgfpathlineto{\pgfqpoint{5.111275in}{2.110003in}}%
\pgfpathclose%
\pgfusepath{fill}%
\end{pgfscope}%
\begin{pgfscope}%
\pgfpathrectangle{\pgfqpoint{1.150000in}{0.150000in}}{\pgfqpoint{5.700000in}{5.700000in}}%
\pgfusepath{clip}%
\pgfsetbuttcap%
\pgfsetroundjoin%
\definecolor{currentfill}{rgb}{0.281924,0.089666,0.412415}%
\pgfsetfillcolor{currentfill}%
\pgfsetfillopacity{0.700000}%
\pgfsetlinewidth{0.000000pt}%
\definecolor{currentstroke}{rgb}{0.000000,0.000000,0.000000}%
\pgfsetstrokecolor{currentstroke}%
\pgfsetdash{}{0pt}%
\pgfpathmoveto{\pgfqpoint{4.511855in}{1.602012in}}%
\pgfpathlineto{\pgfqpoint{4.526113in}{1.602308in}}%
\pgfpathlineto{\pgfqpoint{4.540381in}{1.602700in}}%
\pgfpathlineto{\pgfqpoint{4.554658in}{1.603189in}}%
\pgfpathlineto{\pgfqpoint{4.568946in}{1.603773in}}%
\pgfpathlineto{\pgfqpoint{4.560939in}{1.591763in}}%
\pgfpathlineto{\pgfqpoint{4.552927in}{1.579854in}}%
\pgfpathlineto{\pgfqpoint{4.544911in}{1.568050in}}%
\pgfpathlineto{\pgfqpoint{4.536891in}{1.556356in}}%
\pgfpathlineto{\pgfqpoint{4.522599in}{1.556226in}}%
\pgfpathlineto{\pgfqpoint{4.508317in}{1.556192in}}%
\pgfpathlineto{\pgfqpoint{4.494045in}{1.556254in}}%
\pgfpathlineto{\pgfqpoint{4.479783in}{1.556411in}}%
\pgfpathlineto{\pgfqpoint{4.487807in}{1.567644in}}%
\pgfpathlineto{\pgfqpoint{4.495827in}{1.578992in}}%
\pgfpathlineto{\pgfqpoint{4.503843in}{1.590449in}}%
\pgfpathlineto{\pgfqpoint{4.511855in}{1.602012in}}%
\pgfpathclose%
\pgfusepath{fill}%
\end{pgfscope}%
\begin{pgfscope}%
\pgfpathrectangle{\pgfqpoint{1.150000in}{0.150000in}}{\pgfqpoint{5.700000in}{5.700000in}}%
\pgfusepath{clip}%
\pgfsetbuttcap%
\pgfsetroundjoin%
\definecolor{currentfill}{rgb}{0.283197,0.115680,0.436115}%
\pgfsetfillcolor{currentfill}%
\pgfsetfillopacity{0.700000}%
\pgfsetlinewidth{0.000000pt}%
\definecolor{currentstroke}{rgb}{0.000000,0.000000,0.000000}%
\pgfsetstrokecolor{currentstroke}%
\pgfsetdash{}{0pt}%
\pgfpathmoveto{\pgfqpoint{4.600937in}{1.652741in}}%
\pgfpathlineto{\pgfqpoint{4.615233in}{1.653858in}}%
\pgfpathlineto{\pgfqpoint{4.629539in}{1.655072in}}%
\pgfpathlineto{\pgfqpoint{4.643856in}{1.656381in}}%
\pgfpathlineto{\pgfqpoint{4.658183in}{1.657788in}}%
\pgfpathlineto{\pgfqpoint{4.650194in}{1.644985in}}%
\pgfpathlineto{\pgfqpoint{4.642200in}{1.632262in}}%
\pgfpathlineto{\pgfqpoint{4.634203in}{1.619623in}}%
\pgfpathlineto{\pgfqpoint{4.626202in}{1.607071in}}%
\pgfpathlineto{\pgfqpoint{4.611872in}{1.606103in}}%
\pgfpathlineto{\pgfqpoint{4.597553in}{1.605230in}}%
\pgfpathlineto{\pgfqpoint{4.583245in}{1.604454in}}%
\pgfpathlineto{\pgfqpoint{4.568946in}{1.603773in}}%
\pgfpathlineto{\pgfqpoint{4.576950in}{1.615880in}}%
\pgfpathlineto{\pgfqpoint{4.584950in}{1.628080in}}%
\pgfpathlineto{\pgfqpoint{4.592946in}{1.640368in}}%
\pgfpathlineto{\pgfqpoint{4.600937in}{1.652741in}}%
\pgfpathclose%
\pgfusepath{fill}%
\end{pgfscope}%
\begin{pgfscope}%
\pgfpathrectangle{\pgfqpoint{1.150000in}{0.150000in}}{\pgfqpoint{5.700000in}{5.700000in}}%
\pgfusepath{clip}%
\pgfsetbuttcap%
\pgfsetroundjoin%
\definecolor{currentfill}{rgb}{0.279566,0.067836,0.391917}%
\pgfsetfillcolor{currentfill}%
\pgfsetfillopacity{0.700000}%
\pgfsetlinewidth{0.000000pt}%
\definecolor{currentstroke}{rgb}{0.000000,0.000000,0.000000}%
\pgfsetstrokecolor{currentstroke}%
\pgfsetdash{}{0pt}%
\pgfpathmoveto{\pgfqpoint{4.422828in}{1.558004in}}%
\pgfpathlineto{\pgfqpoint{4.437053in}{1.557462in}}%
\pgfpathlineto{\pgfqpoint{4.451286in}{1.557015in}}%
\pgfpathlineto{\pgfqpoint{4.465530in}{1.556665in}}%
\pgfpathlineto{\pgfqpoint{4.479783in}{1.556411in}}%
\pgfpathlineto{\pgfqpoint{4.471754in}{1.545297in}}%
\pgfpathlineto{\pgfqpoint{4.463721in}{1.534306in}}%
\pgfpathlineto{\pgfqpoint{4.455683in}{1.523443in}}%
\pgfpathlineto{\pgfqpoint{4.447641in}{1.512713in}}%
\pgfpathlineto{\pgfqpoint{4.433382in}{1.513438in}}%
\pgfpathlineto{\pgfqpoint{4.419133in}{1.514260in}}%
\pgfpathlineto{\pgfqpoint{4.404892in}{1.515177in}}%
\pgfpathlineto{\pgfqpoint{4.390661in}{1.516191in}}%
\pgfpathlineto{\pgfqpoint{4.398710in}{1.526443in}}%
\pgfpathlineto{\pgfqpoint{4.406754in}{1.536833in}}%
\pgfpathlineto{\pgfqpoint{4.414793in}{1.547355in}}%
\pgfpathlineto{\pgfqpoint{4.422828in}{1.558004in}}%
\pgfpathclose%
\pgfusepath{fill}%
\end{pgfscope}%
\begin{pgfscope}%
\pgfpathrectangle{\pgfqpoint{1.150000in}{0.150000in}}{\pgfqpoint{5.700000in}{5.700000in}}%
\pgfusepath{clip}%
\pgfsetbuttcap%
\pgfsetroundjoin%
\definecolor{currentfill}{rgb}{0.275191,0.194905,0.496005}%
\pgfsetfillcolor{currentfill}%
\pgfsetfillopacity{0.700000}%
\pgfsetlinewidth{0.000000pt}%
\definecolor{currentstroke}{rgb}{0.000000,0.000000,0.000000}%
\pgfsetstrokecolor{currentstroke}%
\pgfsetdash{}{0pt}%
\pgfpathmoveto{\pgfqpoint{3.291405in}{1.801926in}}%
\pgfpathlineto{\pgfqpoint{3.305436in}{1.791132in}}%
\pgfpathlineto{\pgfqpoint{3.319469in}{1.780451in}}%
\pgfpathlineto{\pgfqpoint{3.333503in}{1.769883in}}%
\pgfpathlineto{\pgfqpoint{3.347538in}{1.759427in}}%
\pgfpathlineto{\pgfqpoint{3.338933in}{1.763214in}}%
\pgfpathlineto{\pgfqpoint{3.330311in}{1.767354in}}%
\pgfpathlineto{\pgfqpoint{3.321674in}{1.771855in}}%
\pgfpathlineto{\pgfqpoint{3.313021in}{1.776722in}}%
\pgfpathlineto{\pgfqpoint{3.298945in}{1.787791in}}%
\pgfpathlineto{\pgfqpoint{3.284871in}{1.798973in}}%
\pgfpathlineto{\pgfqpoint{3.270797in}{1.810268in}}%
\pgfpathlineto{\pgfqpoint{3.256725in}{1.821677in}}%
\pgfpathlineto{\pgfqpoint{3.265420in}{1.816187in}}%
\pgfpathlineto{\pgfqpoint{3.274098in}{1.811070in}}%
\pgfpathlineto{\pgfqpoint{3.282760in}{1.806319in}}%
\pgfpathlineto{\pgfqpoint{3.291405in}{1.801926in}}%
\pgfpathclose%
\pgfusepath{fill}%
\end{pgfscope}%
\begin{pgfscope}%
\pgfpathrectangle{\pgfqpoint{1.150000in}{0.150000in}}{\pgfqpoint{5.700000in}{5.700000in}}%
\pgfusepath{clip}%
\pgfsetbuttcap%
\pgfsetroundjoin%
\definecolor{currentfill}{rgb}{0.274952,0.037752,0.364543}%
\pgfsetfillcolor{currentfill}%
\pgfsetfillopacity{0.700000}%
\pgfsetlinewidth{0.000000pt}%
\definecolor{currentstroke}{rgb}{0.000000,0.000000,0.000000}%
\pgfsetstrokecolor{currentstroke}%
\pgfsetdash{}{0pt}%
\pgfpathmoveto{\pgfqpoint{3.953678in}{1.495490in}}%
\pgfpathlineto{\pgfqpoint{3.967768in}{1.490617in}}%
\pgfpathlineto{\pgfqpoint{3.981865in}{1.485844in}}%
\pgfpathlineto{\pgfqpoint{3.995968in}{1.481170in}}%
\pgfpathlineto{\pgfqpoint{4.010077in}{1.476595in}}%
\pgfpathlineto{\pgfqpoint{4.001887in}{1.471249in}}%
\pgfpathlineto{\pgfqpoint{3.993690in}{1.466134in}}%
\pgfpathlineto{\pgfqpoint{3.985485in}{1.461256in}}%
\pgfpathlineto{\pgfqpoint{3.977272in}{1.456619in}}%
\pgfpathlineto{\pgfqpoint{3.963144in}{1.461737in}}%
\pgfpathlineto{\pgfqpoint{3.949022in}{1.466954in}}%
\pgfpathlineto{\pgfqpoint{3.934905in}{1.472271in}}%
\pgfpathlineto{\pgfqpoint{3.920795in}{1.477687in}}%
\pgfpathlineto{\pgfqpoint{3.929028in}{1.481773in}}%
\pgfpathlineto{\pgfqpoint{3.937253in}{1.486106in}}%
\pgfpathlineto{\pgfqpoint{3.945469in}{1.490680in}}%
\pgfpathlineto{\pgfqpoint{3.953678in}{1.495490in}}%
\pgfpathclose%
\pgfusepath{fill}%
\end{pgfscope}%
\begin{pgfscope}%
\pgfpathrectangle{\pgfqpoint{1.150000in}{0.150000in}}{\pgfqpoint{5.700000in}{5.700000in}}%
\pgfusepath{clip}%
\pgfsetbuttcap%
\pgfsetroundjoin%
\definecolor{currentfill}{rgb}{0.165117,0.467423,0.558141}%
\pgfsetfillcolor{currentfill}%
\pgfsetfillopacity{0.700000}%
\pgfsetlinewidth{0.000000pt}%
\definecolor{currentstroke}{rgb}{0.000000,0.000000,0.000000}%
\pgfsetstrokecolor{currentstroke}%
\pgfsetdash{}{0pt}%
\pgfpathmoveto{\pgfqpoint{5.443713in}{2.483517in}}%
\pgfpathlineto{\pgfqpoint{5.458466in}{2.491642in}}%
\pgfpathlineto{\pgfqpoint{5.473236in}{2.499869in}}%
\pgfpathlineto{\pgfqpoint{5.488022in}{2.508196in}}%
\pgfpathlineto{\pgfqpoint{5.502824in}{2.516624in}}%
\pgfpathlineto{\pgfqpoint{5.495033in}{2.502239in}}%
\pgfpathlineto{\pgfqpoint{5.487236in}{2.487752in}}%
\pgfpathlineto{\pgfqpoint{5.479432in}{2.473164in}}%
\pgfpathlineto{\pgfqpoint{5.471621in}{2.458479in}}%
\pgfpathlineto{\pgfqpoint{5.456824in}{2.450275in}}%
\pgfpathlineto{\pgfqpoint{5.442042in}{2.442172in}}%
\pgfpathlineto{\pgfqpoint{5.427277in}{2.434169in}}%
\pgfpathlineto{\pgfqpoint{5.412527in}{2.426267in}}%
\pgfpathlineto{\pgfqpoint{5.420333in}{2.440721in}}%
\pgfpathlineto{\pgfqpoint{5.428133in}{2.455082in}}%
\pgfpathlineto{\pgfqpoint{5.435926in}{2.469348in}}%
\pgfpathlineto{\pgfqpoint{5.443713in}{2.483517in}}%
\pgfpathclose%
\pgfusepath{fill}%
\end{pgfscope}%
\begin{pgfscope}%
\pgfpathrectangle{\pgfqpoint{1.150000in}{0.150000in}}{\pgfqpoint{5.700000in}{5.700000in}}%
\pgfusepath{clip}%
\pgfsetbuttcap%
\pgfsetroundjoin%
\definecolor{currentfill}{rgb}{0.273809,0.031497,0.358853}%
\pgfsetfillcolor{currentfill}%
\pgfsetfillopacity{0.700000}%
\pgfsetlinewidth{0.000000pt}%
\definecolor{currentstroke}{rgb}{0.000000,0.000000,0.000000}%
\pgfsetstrokecolor{currentstroke}%
\pgfsetdash{}{0pt}%
\pgfpathmoveto{\pgfqpoint{4.099194in}{1.484958in}}%
\pgfpathlineto{\pgfqpoint{4.113320in}{1.481399in}}%
\pgfpathlineto{\pgfqpoint{4.127452in}{1.477938in}}%
\pgfpathlineto{\pgfqpoint{4.141592in}{1.474574in}}%
\pgfpathlineto{\pgfqpoint{4.155739in}{1.471308in}}%
\pgfpathlineto{\pgfqpoint{4.147609in}{1.464059in}}%
\pgfpathlineto{\pgfqpoint{4.139473in}{1.457010in}}%
\pgfpathlineto{\pgfqpoint{4.131331in}{1.450166in}}%
\pgfpathlineto{\pgfqpoint{4.123182in}{1.443532in}}%
\pgfpathlineto{\pgfqpoint{4.109020in}{1.447323in}}%
\pgfpathlineto{\pgfqpoint{4.094865in}{1.451211in}}%
\pgfpathlineto{\pgfqpoint{4.080717in}{1.455196in}}%
\pgfpathlineto{\pgfqpoint{4.066576in}{1.459280in}}%
\pgfpathlineto{\pgfqpoint{4.074741in}{1.465382in}}%
\pgfpathlineto{\pgfqpoint{4.082899in}{1.471699in}}%
\pgfpathlineto{\pgfqpoint{4.091050in}{1.478226in}}%
\pgfpathlineto{\pgfqpoint{4.099194in}{1.484958in}}%
\pgfpathclose%
\pgfusepath{fill}%
\end{pgfscope}%
\begin{pgfscope}%
\pgfpathrectangle{\pgfqpoint{1.150000in}{0.150000in}}{\pgfqpoint{5.700000in}{5.700000in}}%
\pgfusepath{clip}%
\pgfsetbuttcap%
\pgfsetroundjoin%
\definecolor{currentfill}{rgb}{0.282290,0.145912,0.461510}%
\pgfsetfillcolor{currentfill}%
\pgfsetfillopacity{0.700000}%
\pgfsetlinewidth{0.000000pt}%
\definecolor{currentstroke}{rgb}{0.000000,0.000000,0.000000}%
\pgfsetstrokecolor{currentstroke}%
\pgfsetdash{}{0pt}%
\pgfpathmoveto{\pgfqpoint{4.690104in}{1.709714in}}%
\pgfpathlineto{\pgfqpoint{4.704441in}{1.711636in}}%
\pgfpathlineto{\pgfqpoint{4.718790in}{1.713654in}}%
\pgfpathlineto{\pgfqpoint{4.733149in}{1.715769in}}%
\pgfpathlineto{\pgfqpoint{4.747521in}{1.717980in}}%
\pgfpathlineto{\pgfqpoint{4.739547in}{1.704486in}}%
\pgfpathlineto{\pgfqpoint{4.731569in}{1.691051in}}%
\pgfpathlineto{\pgfqpoint{4.723588in}{1.677678in}}%
\pgfpathlineto{\pgfqpoint{4.715603in}{1.664373in}}%
\pgfpathlineto{\pgfqpoint{4.701231in}{1.662582in}}%
\pgfpathlineto{\pgfqpoint{4.686871in}{1.660888in}}%
\pgfpathlineto{\pgfqpoint{4.672522in}{1.659290in}}%
\pgfpathlineto{\pgfqpoint{4.658183in}{1.657788in}}%
\pgfpathlineto{\pgfqpoint{4.666169in}{1.670666in}}%
\pgfpathlineto{\pgfqpoint{4.674151in}{1.683616in}}%
\pgfpathlineto{\pgfqpoint{4.682129in}{1.696633in}}%
\pgfpathlineto{\pgfqpoint{4.690104in}{1.709714in}}%
\pgfpathclose%
\pgfusepath{fill}%
\end{pgfscope}%
\begin{pgfscope}%
\pgfpathrectangle{\pgfqpoint{1.150000in}{0.150000in}}{\pgfqpoint{5.700000in}{5.700000in}}%
\pgfusepath{clip}%
\pgfsetbuttcap%
\pgfsetroundjoin%
\definecolor{currentfill}{rgb}{0.277018,0.050344,0.375715}%
\pgfsetfillcolor{currentfill}%
\pgfsetfillopacity{0.700000}%
\pgfsetlinewidth{0.000000pt}%
\definecolor{currentstroke}{rgb}{0.000000,0.000000,0.000000}%
\pgfsetstrokecolor{currentstroke}%
\pgfsetdash{}{0pt}%
\pgfpathmoveto{\pgfqpoint{4.333824in}{1.521209in}}%
\pgfpathlineto{\pgfqpoint{4.348020in}{1.519809in}}%
\pgfpathlineto{\pgfqpoint{4.362225in}{1.518507in}}%
\pgfpathlineto{\pgfqpoint{4.376438in}{1.517301in}}%
\pgfpathlineto{\pgfqpoint{4.390661in}{1.516191in}}%
\pgfpathlineto{\pgfqpoint{4.382607in}{1.506080in}}%
\pgfpathlineto{\pgfqpoint{4.374549in}{1.496116in}}%
\pgfpathlineto{\pgfqpoint{4.366486in}{1.486303in}}%
\pgfpathlineto{\pgfqpoint{4.358418in}{1.476645in}}%
\pgfpathlineto{\pgfqpoint{4.344187in}{1.478244in}}%
\pgfpathlineto{\pgfqpoint{4.329965in}{1.479939in}}%
\pgfpathlineto{\pgfqpoint{4.315751in}{1.481730in}}%
\pgfpathlineto{\pgfqpoint{4.301546in}{1.483618in}}%
\pgfpathlineto{\pgfqpoint{4.309623in}{1.492780in}}%
\pgfpathlineto{\pgfqpoint{4.317695in}{1.502102in}}%
\pgfpathlineto{\pgfqpoint{4.325762in}{1.511580in}}%
\pgfpathlineto{\pgfqpoint{4.333824in}{1.521209in}}%
\pgfpathclose%
\pgfusepath{fill}%
\end{pgfscope}%
\begin{pgfscope}%
\pgfpathrectangle{\pgfqpoint{1.150000in}{0.150000in}}{\pgfqpoint{5.700000in}{5.700000in}}%
\pgfusepath{clip}%
\pgfsetbuttcap%
\pgfsetroundjoin%
\definecolor{currentfill}{rgb}{0.282656,0.100196,0.422160}%
\pgfsetfillcolor{currentfill}%
\pgfsetfillopacity{0.700000}%
\pgfsetlinewidth{0.000000pt}%
\definecolor{currentstroke}{rgb}{0.000000,0.000000,0.000000}%
\pgfsetstrokecolor{currentstroke}%
\pgfsetdash{}{0pt}%
\pgfpathmoveto{\pgfqpoint{3.606110in}{1.604762in}}%
\pgfpathlineto{\pgfqpoint{3.620154in}{1.596740in}}%
\pgfpathlineto{\pgfqpoint{3.634202in}{1.588823in}}%
\pgfpathlineto{\pgfqpoint{3.648253in}{1.581011in}}%
\pgfpathlineto{\pgfqpoint{3.662308in}{1.573303in}}%
\pgfpathlineto{\pgfqpoint{3.653926in}{1.572840in}}%
\pgfpathlineto{\pgfqpoint{3.645532in}{1.572678in}}%
\pgfpathlineto{\pgfqpoint{3.637126in}{1.572825in}}%
\pgfpathlineto{\pgfqpoint{3.628708in}{1.573285in}}%
\pgfpathlineto{\pgfqpoint{3.614623in}{1.581578in}}%
\pgfpathlineto{\pgfqpoint{3.600541in}{1.589976in}}%
\pgfpathlineto{\pgfqpoint{3.586462in}{1.598479in}}%
\pgfpathlineto{\pgfqpoint{3.572387in}{1.607087in}}%
\pgfpathlineto{\pgfqpoint{3.580836in}{1.606033in}}%
\pgfpathlineto{\pgfqpoint{3.589273in}{1.605298in}}%
\pgfpathlineto{\pgfqpoint{3.597697in}{1.604877in}}%
\pgfpathlineto{\pgfqpoint{3.606110in}{1.604762in}}%
\pgfpathclose%
\pgfusepath{fill}%
\end{pgfscope}%
\begin{pgfscope}%
\pgfpathrectangle{\pgfqpoint{1.150000in}{0.150000in}}{\pgfqpoint{5.700000in}{5.700000in}}%
\pgfusepath{clip}%
\pgfsetbuttcap%
\pgfsetroundjoin%
\definecolor{currentfill}{rgb}{0.277941,0.056324,0.381191}%
\pgfsetfillcolor{currentfill}%
\pgfsetfillopacity{0.700000}%
\pgfsetlinewidth{0.000000pt}%
\definecolor{currentstroke}{rgb}{0.000000,0.000000,0.000000}%
\pgfsetstrokecolor{currentstroke}%
\pgfsetdash{}{0pt}%
\pgfpathmoveto{\pgfqpoint{3.808104in}{1.524619in}}%
\pgfpathlineto{\pgfqpoint{3.822172in}{1.518400in}}%
\pgfpathlineto{\pgfqpoint{3.836245in}{1.512282in}}%
\pgfpathlineto{\pgfqpoint{3.850324in}{1.506265in}}%
\pgfpathlineto{\pgfqpoint{3.864407in}{1.500349in}}%
\pgfpathlineto{\pgfqpoint{3.856143in}{1.497070in}}%
\pgfpathlineto{\pgfqpoint{3.847870in}{1.494054in}}%
\pgfpathlineto{\pgfqpoint{3.839588in}{1.491307in}}%
\pgfpathlineto{\pgfqpoint{3.831296in}{1.488834in}}%
\pgfpathlineto{\pgfqpoint{3.817189in}{1.495314in}}%
\pgfpathlineto{\pgfqpoint{3.803086in}{1.501894in}}%
\pgfpathlineto{\pgfqpoint{3.788989in}{1.508575in}}%
\pgfpathlineto{\pgfqpoint{3.774896in}{1.515358in}}%
\pgfpathlineto{\pgfqpoint{3.783213in}{1.517260in}}%
\pgfpathlineto{\pgfqpoint{3.791519in}{1.519441in}}%
\pgfpathlineto{\pgfqpoint{3.799817in}{1.521896in}}%
\pgfpathlineto{\pgfqpoint{3.808104in}{1.524619in}}%
\pgfpathclose%
\pgfusepath{fill}%
\end{pgfscope}%
\begin{pgfscope}%
\pgfpathrectangle{\pgfqpoint{1.150000in}{0.150000in}}{\pgfqpoint{5.700000in}{5.700000in}}%
\pgfusepath{clip}%
\pgfsetbuttcap%
\pgfsetroundjoin%
\definecolor{currentfill}{rgb}{0.252194,0.269783,0.531579}%
\pgfsetfillcolor{currentfill}%
\pgfsetfillopacity{0.700000}%
\pgfsetlinewidth{0.000000pt}%
\definecolor{currentstroke}{rgb}{0.000000,0.000000,0.000000}%
\pgfsetstrokecolor{currentstroke}%
\pgfsetdash{}{0pt}%
\pgfpathmoveto{\pgfqpoint{4.990034in}{1.972091in}}%
\pgfpathlineto{\pgfqpoint{5.004524in}{1.976682in}}%
\pgfpathlineto{\pgfqpoint{5.019027in}{1.981371in}}%
\pgfpathlineto{\pgfqpoint{5.033544in}{1.986157in}}%
\pgfpathlineto{\pgfqpoint{5.048074in}{1.991042in}}%
\pgfpathlineto{\pgfqpoint{5.040155in}{1.976072in}}%
\pgfpathlineto{\pgfqpoint{5.032231in}{1.961092in}}%
\pgfpathlineto{\pgfqpoint{5.024303in}{1.946102in}}%
\pgfpathlineto{\pgfqpoint{5.016372in}{1.931108in}}%
\pgfpathlineto{\pgfqpoint{5.001846in}{1.926576in}}%
\pgfpathlineto{\pgfqpoint{4.987333in}{1.922141in}}%
\pgfpathlineto{\pgfqpoint{4.972833in}{1.917803in}}%
\pgfpathlineto{\pgfqpoint{4.958346in}{1.913563in}}%
\pgfpathlineto{\pgfqpoint{4.966274in}{1.928199in}}%
\pgfpathlineto{\pgfqpoint{4.974198in}{1.942835in}}%
\pgfpathlineto{\pgfqpoint{4.982118in}{1.957467in}}%
\pgfpathlineto{\pgfqpoint{4.990034in}{1.972091in}}%
\pgfpathclose%
\pgfusepath{fill}%
\end{pgfscope}%
\begin{pgfscope}%
\pgfpathrectangle{\pgfqpoint{1.150000in}{0.150000in}}{\pgfqpoint{5.700000in}{5.700000in}}%
\pgfusepath{clip}%
\pgfsetbuttcap%
\pgfsetroundjoin%
\definecolor{currentfill}{rgb}{0.187231,0.414746,0.556547}%
\pgfsetfillcolor{currentfill}%
\pgfsetfillopacity{0.700000}%
\pgfsetlinewidth{0.000000pt}%
\definecolor{currentstroke}{rgb}{0.000000,0.000000,0.000000}%
\pgfsetstrokecolor{currentstroke}%
\pgfsetdash{}{0pt}%
\pgfpathmoveto{\pgfqpoint{5.322421in}{2.337918in}}%
\pgfpathlineto{\pgfqpoint{5.337103in}{2.345177in}}%
\pgfpathlineto{\pgfqpoint{5.351801in}{2.352537in}}%
\pgfpathlineto{\pgfqpoint{5.366514in}{2.359996in}}%
\pgfpathlineto{\pgfqpoint{5.381243in}{2.367556in}}%
\pgfpathlineto{\pgfqpoint{5.373407in}{2.352665in}}%
\pgfpathlineto{\pgfqpoint{5.365565in}{2.337694in}}%
\pgfpathlineto{\pgfqpoint{5.357718in}{2.322645in}}%
\pgfpathlineto{\pgfqpoint{5.349865in}{2.307520in}}%
\pgfpathlineto{\pgfqpoint{5.335141in}{2.300222in}}%
\pgfpathlineto{\pgfqpoint{5.320433in}{2.293024in}}%
\pgfpathlineto{\pgfqpoint{5.305740in}{2.285926in}}%
\pgfpathlineto{\pgfqpoint{5.291062in}{2.278927in}}%
\pgfpathlineto{\pgfqpoint{5.298910in}{2.293784in}}%
\pgfpathlineto{\pgfqpoint{5.306752in}{2.308570in}}%
\pgfpathlineto{\pgfqpoint{5.314589in}{2.323281in}}%
\pgfpathlineto{\pgfqpoint{5.322421in}{2.337918in}}%
\pgfpathclose%
\pgfusepath{fill}%
\end{pgfscope}%
\begin{pgfscope}%
\pgfpathrectangle{\pgfqpoint{1.150000in}{0.150000in}}{\pgfqpoint{5.700000in}{5.700000in}}%
\pgfusepath{clip}%
\pgfsetbuttcap%
\pgfsetroundjoin%
\definecolor{currentfill}{rgb}{0.202219,0.715272,0.476084}%
\pgfsetfillcolor{currentfill}%
\pgfsetfillopacity{0.700000}%
\pgfsetlinewidth{0.000000pt}%
\definecolor{currentstroke}{rgb}{0.000000,0.000000,0.000000}%
\pgfsetstrokecolor{currentstroke}%
\pgfsetdash{}{0pt}%
\pgfpathmoveto{\pgfqpoint{2.105249in}{3.189249in}}%
\pgfpathlineto{\pgfqpoint{2.119624in}{3.165919in}}%
\pgfpathlineto{\pgfqpoint{2.133988in}{3.142793in}}%
\pgfpathlineto{\pgfqpoint{2.148341in}{3.119870in}}%
\pgfpathlineto{\pgfqpoint{2.162684in}{3.097147in}}%
\pgfpathlineto{\pgfqpoint{2.152904in}{3.114261in}}%
\pgfpathlineto{\pgfqpoint{2.143090in}{3.131860in}}%
\pgfpathlineto{\pgfqpoint{2.133243in}{3.149954in}}%
\pgfpathlineto{\pgfqpoint{2.123362in}{3.168548in}}%
\pgfpathlineto{\pgfqpoint{2.108943in}{3.191964in}}%
\pgfpathlineto{\pgfqpoint{2.094512in}{3.215581in}}%
\pgfpathlineto{\pgfqpoint{2.080070in}{3.239403in}}%
\pgfpathlineto{\pgfqpoint{2.065617in}{3.263430in}}%
\pgfpathlineto{\pgfqpoint{2.075577in}{3.244129in}}%
\pgfpathlineto{\pgfqpoint{2.085502in}{3.225338in}}%
\pgfpathlineto{\pgfqpoint{2.095393in}{3.207047in}}%
\pgfpathlineto{\pgfqpoint{2.105249in}{3.189249in}}%
\pgfpathclose%
\pgfusepath{fill}%
\end{pgfscope}%
\begin{pgfscope}%
\pgfpathrectangle{\pgfqpoint{1.150000in}{0.150000in}}{\pgfqpoint{5.700000in}{5.700000in}}%
\pgfusepath{clip}%
\pgfsetbuttcap%
\pgfsetroundjoin%
\definecolor{currentfill}{rgb}{0.278826,0.175490,0.483397}%
\pgfsetfillcolor{currentfill}%
\pgfsetfillopacity{0.700000}%
\pgfsetlinewidth{0.000000pt}%
\definecolor{currentstroke}{rgb}{0.000000,0.000000,0.000000}%
\pgfsetstrokecolor{currentstroke}%
\pgfsetdash{}{0pt}%
\pgfpathmoveto{\pgfqpoint{3.347538in}{1.759427in}}%
\pgfpathlineto{\pgfqpoint{3.361575in}{1.749084in}}%
\pgfpathlineto{\pgfqpoint{3.375614in}{1.738852in}}%
\pgfpathlineto{\pgfqpoint{3.389655in}{1.728732in}}%
\pgfpathlineto{\pgfqpoint{3.403697in}{1.718722in}}%
\pgfpathlineto{\pgfqpoint{3.395129in}{1.721905in}}%
\pgfpathlineto{\pgfqpoint{3.386547in}{1.725435in}}%
\pgfpathlineto{\pgfqpoint{3.377949in}{1.729321in}}%
\pgfpathlineto{\pgfqpoint{3.369336in}{1.733568in}}%
\pgfpathlineto{\pgfqpoint{3.355255in}{1.744189in}}%
\pgfpathlineto{\pgfqpoint{3.341176in}{1.754922in}}%
\pgfpathlineto{\pgfqpoint{3.327098in}{1.765766in}}%
\pgfpathlineto{\pgfqpoint{3.313021in}{1.776722in}}%
\pgfpathlineto{\pgfqpoint{3.321674in}{1.771855in}}%
\pgfpathlineto{\pgfqpoint{3.330311in}{1.767354in}}%
\pgfpathlineto{\pgfqpoint{3.338933in}{1.763214in}}%
\pgfpathlineto{\pgfqpoint{3.347538in}{1.759427in}}%
\pgfpathclose%
\pgfusepath{fill}%
\end{pgfscope}%
\begin{pgfscope}%
\pgfpathrectangle{\pgfqpoint{1.150000in}{0.150000in}}{\pgfqpoint{5.700000in}{5.700000in}}%
\pgfusepath{clip}%
\pgfsetbuttcap%
\pgfsetroundjoin%
\definecolor{currentfill}{rgb}{0.278012,0.180367,0.486697}%
\pgfsetfillcolor{currentfill}%
\pgfsetfillopacity{0.700000}%
\pgfsetlinewidth{0.000000pt}%
\definecolor{currentstroke}{rgb}{0.000000,0.000000,0.000000}%
\pgfsetstrokecolor{currentstroke}%
\pgfsetdash{}{0pt}%
\pgfpathmoveto{\pgfqpoint{4.779379in}{1.772471in}}%
\pgfpathlineto{\pgfqpoint{4.793763in}{1.775182in}}%
\pgfpathlineto{\pgfqpoint{4.808158in}{1.777989in}}%
\pgfpathlineto{\pgfqpoint{4.822565in}{1.780893in}}%
\pgfpathlineto{\pgfqpoint{4.836984in}{1.783893in}}%
\pgfpathlineto{\pgfqpoint{4.829024in}{1.769804in}}%
\pgfpathlineto{\pgfqpoint{4.821060in}{1.755755in}}%
\pgfpathlineto{\pgfqpoint{4.813093in}{1.741748in}}%
\pgfpathlineto{\pgfqpoint{4.805122in}{1.727788in}}%
\pgfpathlineto{\pgfqpoint{4.790704in}{1.725192in}}%
\pgfpathlineto{\pgfqpoint{4.776298in}{1.722692in}}%
\pgfpathlineto{\pgfqpoint{4.761904in}{1.720288in}}%
\pgfpathlineto{\pgfqpoint{4.747521in}{1.717980in}}%
\pgfpathlineto{\pgfqpoint{4.755491in}{1.731530in}}%
\pgfpathlineto{\pgfqpoint{4.763457in}{1.745131in}}%
\pgfpathlineto{\pgfqpoint{4.771420in}{1.758779in}}%
\pgfpathlineto{\pgfqpoint{4.779379in}{1.772471in}}%
\pgfpathclose%
\pgfusepath{fill}%
\end{pgfscope}%
\begin{pgfscope}%
\pgfpathrectangle{\pgfqpoint{1.150000in}{0.150000in}}{\pgfqpoint{5.700000in}{5.700000in}}%
\pgfusepath{clip}%
\pgfsetbuttcap%
\pgfsetroundjoin%
\definecolor{currentfill}{rgb}{0.274952,0.037752,0.364543}%
\pgfsetfillcolor{currentfill}%
\pgfsetfillopacity{0.700000}%
\pgfsetlinewidth{0.000000pt}%
\definecolor{currentstroke}{rgb}{0.000000,0.000000,0.000000}%
\pgfsetstrokecolor{currentstroke}%
\pgfsetdash{}{0pt}%
\pgfpathmoveto{\pgfqpoint{4.244807in}{1.492134in}}%
\pgfpathlineto{\pgfqpoint{4.258980in}{1.489860in}}%
\pgfpathlineto{\pgfqpoint{4.273160in}{1.487683in}}%
\pgfpathlineto{\pgfqpoint{4.287349in}{1.485602in}}%
\pgfpathlineto{\pgfqpoint{4.301546in}{1.483618in}}%
\pgfpathlineto{\pgfqpoint{4.293464in}{1.474622in}}%
\pgfpathlineto{\pgfqpoint{4.285376in}{1.465796in}}%
\pgfpathlineto{\pgfqpoint{4.277283in}{1.457145in}}%
\pgfpathlineto{\pgfqpoint{4.269184in}{1.448675in}}%
\pgfpathlineto{\pgfqpoint{4.254977in}{1.451166in}}%
\pgfpathlineto{\pgfqpoint{4.240777in}{1.453753in}}%
\pgfpathlineto{\pgfqpoint{4.226585in}{1.456437in}}%
\pgfpathlineto{\pgfqpoint{4.212400in}{1.459217in}}%
\pgfpathlineto{\pgfqpoint{4.220511in}{1.467174in}}%
\pgfpathlineto{\pgfqpoint{4.228615in}{1.475316in}}%
\pgfpathlineto{\pgfqpoint{4.236714in}{1.483637in}}%
\pgfpathlineto{\pgfqpoint{4.244807in}{1.492134in}}%
\pgfpathclose%
\pgfusepath{fill}%
\end{pgfscope}%
\begin{pgfscope}%
\pgfpathrectangle{\pgfqpoint{1.150000in}{0.150000in}}{\pgfqpoint{5.700000in}{5.700000in}}%
\pgfusepath{clip}%
\pgfsetbuttcap%
\pgfsetroundjoin%
\definecolor{currentfill}{rgb}{0.210503,0.363727,0.552206}%
\pgfsetfillcolor{currentfill}%
\pgfsetfillopacity{0.700000}%
\pgfsetlinewidth{0.000000pt}%
\definecolor{currentstroke}{rgb}{0.000000,0.000000,0.000000}%
\pgfsetstrokecolor{currentstroke}%
\pgfsetdash{}{0pt}%
\pgfpathmoveto{\pgfqpoint{5.201074in}{2.192953in}}%
\pgfpathlineto{\pgfqpoint{5.215688in}{2.199276in}}%
\pgfpathlineto{\pgfqpoint{5.230316in}{2.205697in}}%
\pgfpathlineto{\pgfqpoint{5.244959in}{2.212218in}}%
\pgfpathlineto{\pgfqpoint{5.259617in}{2.218837in}}%
\pgfpathlineto{\pgfqpoint{5.251742in}{2.203662in}}%
\pgfpathlineto{\pgfqpoint{5.243863in}{2.188430in}}%
\pgfpathlineto{\pgfqpoint{5.235979in}{2.173144in}}%
\pgfpathlineto{\pgfqpoint{5.228090in}{2.157808in}}%
\pgfpathlineto{\pgfqpoint{5.213437in}{2.151487in}}%
\pgfpathlineto{\pgfqpoint{5.198800in}{2.145265in}}%
\pgfpathlineto{\pgfqpoint{5.184176in}{2.139142in}}%
\pgfpathlineto{\pgfqpoint{5.169567in}{2.133117in}}%
\pgfpathlineto{\pgfqpoint{5.177451in}{2.148148in}}%
\pgfpathlineto{\pgfqpoint{5.185331in}{2.163133in}}%
\pgfpathlineto{\pgfqpoint{5.193205in}{2.178069in}}%
\pgfpathlineto{\pgfqpoint{5.201074in}{2.192953in}}%
\pgfpathclose%
\pgfusepath{fill}%
\end{pgfscope}%
\begin{pgfscope}%
\pgfpathrectangle{\pgfqpoint{1.150000in}{0.150000in}}{\pgfqpoint{5.700000in}{5.700000in}}%
\pgfusepath{clip}%
\pgfsetbuttcap%
\pgfsetroundjoin%
\definecolor{currentfill}{rgb}{0.151918,0.500685,0.557587}%
\pgfsetfillcolor{currentfill}%
\pgfsetfillopacity{0.700000}%
\pgfsetlinewidth{0.000000pt}%
\definecolor{currentstroke}{rgb}{0.000000,0.000000,0.000000}%
\pgfsetstrokecolor{currentstroke}%
\pgfsetdash{}{0pt}%
\pgfpathmoveto{\pgfqpoint{5.533919in}{2.573105in}}%
\pgfpathlineto{\pgfqpoint{5.548742in}{2.581839in}}%
\pgfpathlineto{\pgfqpoint{5.563581in}{2.590674in}}%
\pgfpathlineto{\pgfqpoint{5.578437in}{2.599610in}}%
\pgfpathlineto{\pgfqpoint{5.570671in}{2.585504in}}%
\pgfpathlineto{\pgfqpoint{5.562897in}{2.571284in}}%
\pgfpathlineto{\pgfqpoint{5.555117in}{2.556954in}}%
\pgfpathlineto{\pgfqpoint{5.547329in}{2.542513in}}%
\pgfpathlineto{\pgfqpoint{5.532478in}{2.533782in}}%
\pgfpathlineto{\pgfqpoint{5.517643in}{2.525153in}}%
\pgfpathlineto{\pgfqpoint{5.502824in}{2.516624in}}%
\pgfpathlineto{\pgfqpoint{5.510608in}{2.530905in}}%
\pgfpathlineto{\pgfqpoint{5.518385in}{2.545080in}}%
\pgfpathlineto{\pgfqpoint{5.526156in}{2.559147in}}%
\pgfpathlineto{\pgfqpoint{5.533919in}{2.573105in}}%
\pgfpathclose%
\pgfusepath{fill}%
\end{pgfscope}%
\begin{pgfscope}%
\pgfpathrectangle{\pgfqpoint{1.150000in}{0.150000in}}{\pgfqpoint{5.700000in}{5.700000in}}%
\pgfusepath{clip}%
\pgfsetbuttcap%
\pgfsetroundjoin%
\definecolor{currentfill}{rgb}{0.177423,0.437527,0.557565}%
\pgfsetfillcolor{currentfill}%
\pgfsetfillopacity{0.700000}%
\pgfsetlinewidth{0.000000pt}%
\definecolor{currentstroke}{rgb}{0.000000,0.000000,0.000000}%
\pgfsetstrokecolor{currentstroke}%
\pgfsetdash{}{0pt}%
\pgfpathmoveto{\pgfqpoint{2.693314in}{2.378430in}}%
\pgfpathlineto{\pgfqpoint{2.707441in}{2.361938in}}%
\pgfpathlineto{\pgfqpoint{2.721565in}{2.345588in}}%
\pgfpathlineto{\pgfqpoint{2.735686in}{2.329379in}}%
\pgfpathlineto{\pgfqpoint{2.749802in}{2.313312in}}%
\pgfpathlineto{\pgfqpoint{2.740636in}{2.324944in}}%
\pgfpathlineto{\pgfqpoint{2.731444in}{2.337018in}}%
\pgfpathlineto{\pgfqpoint{2.722227in}{2.349539in}}%
\pgfpathlineto{\pgfqpoint{2.712985in}{2.362517in}}%
\pgfpathlineto{\pgfqpoint{2.698809in}{2.379249in}}%
\pgfpathlineto{\pgfqpoint{2.684628in}{2.396123in}}%
\pgfpathlineto{\pgfqpoint{2.670444in}{2.413139in}}%
\pgfpathlineto{\pgfqpoint{2.656256in}{2.430299in}}%
\pgfpathlineto{\pgfqpoint{2.665559in}{2.416646in}}%
\pgfpathlineto{\pgfqpoint{2.674837in}{2.403455in}}%
\pgfpathlineto{\pgfqpoint{2.684088in}{2.390719in}}%
\pgfpathlineto{\pgfqpoint{2.693314in}{2.378430in}}%
\pgfpathclose%
\pgfusepath{fill}%
\end{pgfscope}%
\begin{pgfscope}%
\pgfpathrectangle{\pgfqpoint{1.150000in}{0.150000in}}{\pgfqpoint{5.700000in}{5.700000in}}%
\pgfusepath{clip}%
\pgfsetbuttcap%
\pgfsetroundjoin%
\definecolor{currentfill}{rgb}{0.188923,0.410910,0.556326}%
\pgfsetfillcolor{currentfill}%
\pgfsetfillopacity{0.700000}%
\pgfsetlinewidth{0.000000pt}%
\definecolor{currentstroke}{rgb}{0.000000,0.000000,0.000000}%
\pgfsetstrokecolor{currentstroke}%
\pgfsetdash{}{0pt}%
\pgfpathmoveto{\pgfqpoint{2.749802in}{2.313312in}}%
\pgfpathlineto{\pgfqpoint{2.763916in}{2.297384in}}%
\pgfpathlineto{\pgfqpoint{2.778026in}{2.281595in}}%
\pgfpathlineto{\pgfqpoint{2.792133in}{2.265944in}}%
\pgfpathlineto{\pgfqpoint{2.806237in}{2.250430in}}%
\pgfpathlineto{\pgfqpoint{2.797128in}{2.261410in}}%
\pgfpathlineto{\pgfqpoint{2.787994in}{2.272825in}}%
\pgfpathlineto{\pgfqpoint{2.778836in}{2.284681in}}%
\pgfpathlineto{\pgfqpoint{2.769654in}{2.296987in}}%
\pgfpathlineto{\pgfqpoint{2.755492in}{2.313162in}}%
\pgfpathlineto{\pgfqpoint{2.741326in}{2.329474in}}%
\pgfpathlineto{\pgfqpoint{2.727158in}{2.345926in}}%
\pgfpathlineto{\pgfqpoint{2.712985in}{2.362517in}}%
\pgfpathlineto{\pgfqpoint{2.722227in}{2.349539in}}%
\pgfpathlineto{\pgfqpoint{2.731444in}{2.337018in}}%
\pgfpathlineto{\pgfqpoint{2.740636in}{2.324944in}}%
\pgfpathlineto{\pgfqpoint{2.749802in}{2.313312in}}%
\pgfpathclose%
\pgfusepath{fill}%
\end{pgfscope}%
\begin{pgfscope}%
\pgfpathrectangle{\pgfqpoint{1.150000in}{0.150000in}}{\pgfqpoint{5.700000in}{5.700000in}}%
\pgfusepath{clip}%
\pgfsetbuttcap%
\pgfsetroundjoin%
\definecolor{currentfill}{rgb}{0.199430,0.387607,0.554642}%
\pgfsetfillcolor{currentfill}%
\pgfsetfillopacity{0.700000}%
\pgfsetlinewidth{0.000000pt}%
\definecolor{currentstroke}{rgb}{0.000000,0.000000,0.000000}%
\pgfsetstrokecolor{currentstroke}%
\pgfsetdash{}{0pt}%
\pgfpathmoveto{\pgfqpoint{2.806237in}{2.250430in}}%
\pgfpathlineto{\pgfqpoint{2.820339in}{2.235052in}}%
\pgfpathlineto{\pgfqpoint{2.834437in}{2.219809in}}%
\pgfpathlineto{\pgfqpoint{2.848533in}{2.204701in}}%
\pgfpathlineto{\pgfqpoint{2.862627in}{2.189726in}}%
\pgfpathlineto{\pgfqpoint{2.853572in}{2.200058in}}%
\pgfpathlineto{\pgfqpoint{2.844495in}{2.210817in}}%
\pgfpathlineto{\pgfqpoint{2.835394in}{2.222012in}}%
\pgfpathlineto{\pgfqpoint{2.826270in}{2.233650in}}%
\pgfpathlineto{\pgfqpoint{2.812120in}{2.249282in}}%
\pgfpathlineto{\pgfqpoint{2.797968in}{2.265048in}}%
\pgfpathlineto{\pgfqpoint{2.783812in}{2.280950in}}%
\pgfpathlineto{\pgfqpoint{2.769654in}{2.296987in}}%
\pgfpathlineto{\pgfqpoint{2.778836in}{2.284681in}}%
\pgfpathlineto{\pgfqpoint{2.787994in}{2.272825in}}%
\pgfpathlineto{\pgfqpoint{2.797128in}{2.261410in}}%
\pgfpathlineto{\pgfqpoint{2.806237in}{2.250430in}}%
\pgfpathclose%
\pgfusepath{fill}%
\end{pgfscope}%
\begin{pgfscope}%
\pgfpathrectangle{\pgfqpoint{1.150000in}{0.150000in}}{\pgfqpoint{5.700000in}{5.700000in}}%
\pgfusepath{clip}%
\pgfsetbuttcap%
\pgfsetroundjoin%
\definecolor{currentfill}{rgb}{0.168126,0.459988,0.558082}%
\pgfsetfillcolor{currentfill}%
\pgfsetfillopacity{0.700000}%
\pgfsetlinewidth{0.000000pt}%
\definecolor{currentstroke}{rgb}{0.000000,0.000000,0.000000}%
\pgfsetstrokecolor{currentstroke}%
\pgfsetdash{}{0pt}%
\pgfpathmoveto{\pgfqpoint{2.636762in}{2.445848in}}%
\pgfpathlineto{\pgfqpoint{2.650907in}{2.428774in}}%
\pgfpathlineto{\pgfqpoint{2.665046in}{2.411848in}}%
\pgfpathlineto{\pgfqpoint{2.679182in}{2.395067in}}%
\pgfpathlineto{\pgfqpoint{2.693314in}{2.378430in}}%
\pgfpathlineto{\pgfqpoint{2.684088in}{2.390719in}}%
\pgfpathlineto{\pgfqpoint{2.674837in}{2.403455in}}%
\pgfpathlineto{\pgfqpoint{2.665559in}{2.416646in}}%
\pgfpathlineto{\pgfqpoint{2.656256in}{2.430299in}}%
\pgfpathlineto{\pgfqpoint{2.642063in}{2.447603in}}%
\pgfpathlineto{\pgfqpoint{2.627865in}{2.465054in}}%
\pgfpathlineto{\pgfqpoint{2.613663in}{2.482650in}}%
\pgfpathlineto{\pgfqpoint{2.599456in}{2.500395in}}%
\pgfpathlineto{\pgfqpoint{2.608824in}{2.486062in}}%
\pgfpathlineto{\pgfqpoint{2.618163in}{2.472198in}}%
\pgfpathlineto{\pgfqpoint{2.627476in}{2.458796in}}%
\pgfpathlineto{\pgfqpoint{2.636762in}{2.445848in}}%
\pgfpathclose%
\pgfusepath{fill}%
\end{pgfscope}%
\begin{pgfscope}%
\pgfpathrectangle{\pgfqpoint{1.150000in}{0.150000in}}{\pgfqpoint{5.700000in}{5.700000in}}%
\pgfusepath{clip}%
\pgfsetbuttcap%
\pgfsetroundjoin%
\definecolor{currentfill}{rgb}{0.270595,0.214069,0.507052}%
\pgfsetfillcolor{currentfill}%
\pgfsetfillopacity{0.700000}%
\pgfsetlinewidth{0.000000pt}%
\definecolor{currentstroke}{rgb}{0.000000,0.000000,0.000000}%
\pgfsetstrokecolor{currentstroke}%
\pgfsetdash{}{0pt}%
\pgfpathmoveto{\pgfqpoint{4.868787in}{1.840566in}}%
\pgfpathlineto{\pgfqpoint{4.883220in}{1.844049in}}%
\pgfpathlineto{\pgfqpoint{4.897666in}{1.847629in}}%
\pgfpathlineto{\pgfqpoint{4.912124in}{1.851306in}}%
\pgfpathlineto{\pgfqpoint{4.926595in}{1.855080in}}%
\pgfpathlineto{\pgfqpoint{4.918647in}{1.840492in}}%
\pgfpathlineto{\pgfqpoint{4.910696in}{1.825923in}}%
\pgfpathlineto{\pgfqpoint{4.902741in}{1.811378in}}%
\pgfpathlineto{\pgfqpoint{4.894783in}{1.796860in}}%
\pgfpathlineto{\pgfqpoint{4.880315in}{1.793474in}}%
\pgfpathlineto{\pgfqpoint{4.865859in}{1.790184in}}%
\pgfpathlineto{\pgfqpoint{4.851415in}{1.786990in}}%
\pgfpathlineto{\pgfqpoint{4.836984in}{1.783893in}}%
\pgfpathlineto{\pgfqpoint{4.844940in}{1.798017in}}%
\pgfpathlineto{\pgfqpoint{4.852893in}{1.812173in}}%
\pgfpathlineto{\pgfqpoint{4.860842in}{1.826357in}}%
\pgfpathlineto{\pgfqpoint{4.868787in}{1.840566in}}%
\pgfpathclose%
\pgfusepath{fill}%
\end{pgfscope}%
\begin{pgfscope}%
\pgfpathrectangle{\pgfqpoint{1.150000in}{0.150000in}}{\pgfqpoint{5.700000in}{5.700000in}}%
\pgfusepath{clip}%
\pgfsetbuttcap%
\pgfsetroundjoin%
\definecolor{currentfill}{rgb}{0.273809,0.031497,0.358853}%
\pgfsetfillcolor{currentfill}%
\pgfsetfillopacity{0.700000}%
\pgfsetlinewidth{0.000000pt}%
\definecolor{currentstroke}{rgb}{0.000000,0.000000,0.000000}%
\pgfsetstrokecolor{currentstroke}%
\pgfsetdash{}{0pt}%
\pgfpathmoveto{\pgfqpoint{4.010077in}{1.476595in}}%
\pgfpathlineto{\pgfqpoint{4.024192in}{1.472118in}}%
\pgfpathlineto{\pgfqpoint{4.038313in}{1.467740in}}%
\pgfpathlineto{\pgfqpoint{4.052441in}{1.463461in}}%
\pgfpathlineto{\pgfqpoint{4.066576in}{1.459280in}}%
\pgfpathlineto{\pgfqpoint{4.058404in}{1.453398in}}%
\pgfpathlineto{\pgfqpoint{4.050225in}{1.447742in}}%
\pgfpathlineto{\pgfqpoint{4.042038in}{1.442318in}}%
\pgfpathlineto{\pgfqpoint{4.033844in}{1.437131in}}%
\pgfpathlineto{\pgfqpoint{4.019692in}{1.441855in}}%
\pgfpathlineto{\pgfqpoint{4.005546in}{1.446678in}}%
\pgfpathlineto{\pgfqpoint{3.991406in}{1.451599in}}%
\pgfpathlineto{\pgfqpoint{3.977272in}{1.456619in}}%
\pgfpathlineto{\pgfqpoint{3.985485in}{1.461256in}}%
\pgfpathlineto{\pgfqpoint{3.993690in}{1.466134in}}%
\pgfpathlineto{\pgfqpoint{4.001887in}{1.471249in}}%
\pgfpathlineto{\pgfqpoint{4.010077in}{1.476595in}}%
\pgfpathclose%
\pgfusepath{fill}%
\end{pgfscope}%
\begin{pgfscope}%
\pgfpathrectangle{\pgfqpoint{1.150000in}{0.150000in}}{\pgfqpoint{5.700000in}{5.700000in}}%
\pgfusepath{clip}%
\pgfsetbuttcap%
\pgfsetroundjoin%
\definecolor{currentfill}{rgb}{0.281924,0.089666,0.412415}%
\pgfsetfillcolor{currentfill}%
\pgfsetfillopacity{0.700000}%
\pgfsetlinewidth{0.000000pt}%
\definecolor{currentstroke}{rgb}{0.000000,0.000000,0.000000}%
\pgfsetstrokecolor{currentstroke}%
\pgfsetdash{}{0pt}%
\pgfpathmoveto{\pgfqpoint{3.662308in}{1.573303in}}%
\pgfpathlineto{\pgfqpoint{3.676367in}{1.565699in}}%
\pgfpathlineto{\pgfqpoint{3.690430in}{1.558199in}}%
\pgfpathlineto{\pgfqpoint{3.704497in}{1.550802in}}%
\pgfpathlineto{\pgfqpoint{3.718568in}{1.543508in}}%
\pgfpathlineto{\pgfqpoint{3.710214in}{1.542468in}}%
\pgfpathlineto{\pgfqpoint{3.701849in}{1.541724in}}%
\pgfpathlineto{\pgfqpoint{3.693473in}{1.541283in}}%
\pgfpathlineto{\pgfqpoint{3.685085in}{1.541151in}}%
\pgfpathlineto{\pgfqpoint{3.670985in}{1.549030in}}%
\pgfpathlineto{\pgfqpoint{3.656889in}{1.557011in}}%
\pgfpathlineto{\pgfqpoint{3.642797in}{1.565096in}}%
\pgfpathlineto{\pgfqpoint{3.628708in}{1.573285in}}%
\pgfpathlineto{\pgfqpoint{3.637126in}{1.572825in}}%
\pgfpathlineto{\pgfqpoint{3.645532in}{1.572678in}}%
\pgfpathlineto{\pgfqpoint{3.653926in}{1.572840in}}%
\pgfpathlineto{\pgfqpoint{3.662308in}{1.573303in}}%
\pgfpathclose%
\pgfusepath{fill}%
\end{pgfscope}%
\begin{pgfscope}%
\pgfpathrectangle{\pgfqpoint{1.150000in}{0.150000in}}{\pgfqpoint{5.700000in}{5.700000in}}%
\pgfusepath{clip}%
\pgfsetbuttcap%
\pgfsetroundjoin%
\definecolor{currentfill}{rgb}{0.280868,0.160771,0.472899}%
\pgfsetfillcolor{currentfill}%
\pgfsetfillopacity{0.700000}%
\pgfsetlinewidth{0.000000pt}%
\definecolor{currentstroke}{rgb}{0.000000,0.000000,0.000000}%
\pgfsetstrokecolor{currentstroke}%
\pgfsetdash{}{0pt}%
\pgfpathmoveto{\pgfqpoint{3.403697in}{1.718722in}}%
\pgfpathlineto{\pgfqpoint{3.417741in}{1.708822in}}%
\pgfpathlineto{\pgfqpoint{3.431788in}{1.699033in}}%
\pgfpathlineto{\pgfqpoint{3.445836in}{1.689352in}}%
\pgfpathlineto{\pgfqpoint{3.459887in}{1.679781in}}%
\pgfpathlineto{\pgfqpoint{3.451356in}{1.682361in}}%
\pgfpathlineto{\pgfqpoint{3.442811in}{1.685284in}}%
\pgfpathlineto{\pgfqpoint{3.434251in}{1.688556in}}%
\pgfpathlineto{\pgfqpoint{3.425677in}{1.692184in}}%
\pgfpathlineto{\pgfqpoint{3.411589in}{1.702366in}}%
\pgfpathlineto{\pgfqpoint{3.397503in}{1.712656in}}%
\pgfpathlineto{\pgfqpoint{3.383419in}{1.723057in}}%
\pgfpathlineto{\pgfqpoint{3.369336in}{1.733568in}}%
\pgfpathlineto{\pgfqpoint{3.377949in}{1.729321in}}%
\pgfpathlineto{\pgfqpoint{3.386547in}{1.725435in}}%
\pgfpathlineto{\pgfqpoint{3.395129in}{1.721905in}}%
\pgfpathlineto{\pgfqpoint{3.403697in}{1.718722in}}%
\pgfpathclose%
\pgfusepath{fill}%
\end{pgfscope}%
\begin{pgfscope}%
\pgfpathrectangle{\pgfqpoint{1.150000in}{0.150000in}}{\pgfqpoint{5.700000in}{5.700000in}}%
\pgfusepath{clip}%
\pgfsetbuttcap%
\pgfsetroundjoin%
\definecolor{currentfill}{rgb}{0.237441,0.305202,0.541921}%
\pgfsetfillcolor{currentfill}%
\pgfsetfillopacity{0.700000}%
\pgfsetlinewidth{0.000000pt}%
\definecolor{currentstroke}{rgb}{0.000000,0.000000,0.000000}%
\pgfsetstrokecolor{currentstroke}%
\pgfsetdash{}{0pt}%
\pgfpathmoveto{\pgfqpoint{5.079710in}{2.050738in}}%
\pgfpathlineto{\pgfqpoint{5.094258in}{2.056054in}}%
\pgfpathlineto{\pgfqpoint{5.108819in}{2.061468in}}%
\pgfpathlineto{\pgfqpoint{5.123395in}{2.066980in}}%
\pgfpathlineto{\pgfqpoint{5.137985in}{2.072590in}}%
\pgfpathlineto{\pgfqpoint{5.130078in}{2.057371in}}%
\pgfpathlineto{\pgfqpoint{5.122167in}{2.042124in}}%
\pgfpathlineto{\pgfqpoint{5.114251in}{2.026851in}}%
\pgfpathlineto{\pgfqpoint{5.106331in}{2.011555in}}%
\pgfpathlineto{\pgfqpoint{5.091746in}{2.006280in}}%
\pgfpathlineto{\pgfqpoint{5.077175in}{2.001103in}}%
\pgfpathlineto{\pgfqpoint{5.062618in}{1.996023in}}%
\pgfpathlineto{\pgfqpoint{5.048074in}{1.991042in}}%
\pgfpathlineto{\pgfqpoint{5.055990in}{2.005996in}}%
\pgfpathlineto{\pgfqpoint{5.063901in}{2.020932in}}%
\pgfpathlineto{\pgfqpoint{5.071807in}{2.035847in}}%
\pgfpathlineto{\pgfqpoint{5.079710in}{2.050738in}}%
\pgfpathclose%
\pgfusepath{fill}%
\end{pgfscope}%
\begin{pgfscope}%
\pgfpathrectangle{\pgfqpoint{1.150000in}{0.150000in}}{\pgfqpoint{5.700000in}{5.700000in}}%
\pgfusepath{clip}%
\pgfsetbuttcap%
\pgfsetroundjoin%
\definecolor{currentfill}{rgb}{0.266941,0.748751,0.440573}%
\pgfsetfillcolor{currentfill}%
\pgfsetfillopacity{0.700000}%
\pgfsetlinewidth{0.000000pt}%
\definecolor{currentstroke}{rgb}{0.000000,0.000000,0.000000}%
\pgfsetstrokecolor{currentstroke}%
\pgfsetdash{}{0pt}%
\pgfpathmoveto{\pgfqpoint{2.047634in}{3.284653in}}%
\pgfpathlineto{\pgfqpoint{2.062055in}{3.260485in}}%
\pgfpathlineto{\pgfqpoint{2.076465in}{3.236530in}}%
\pgfpathlineto{\pgfqpoint{2.090862in}{3.212785in}}%
\pgfpathlineto{\pgfqpoint{2.105249in}{3.189249in}}%
\pgfpathlineto{\pgfqpoint{2.095393in}{3.207047in}}%
\pgfpathlineto{\pgfqpoint{2.085502in}{3.225338in}}%
\pgfpathlineto{\pgfqpoint{2.075577in}{3.244129in}}%
\pgfpathlineto{\pgfqpoint{2.065617in}{3.263430in}}%
\pgfpathlineto{\pgfqpoint{2.051152in}{3.287666in}}%
\pgfpathlineto{\pgfqpoint{2.036675in}{3.312112in}}%
\pgfpathlineto{\pgfqpoint{2.022186in}{3.336770in}}%
\pgfpathlineto{\pgfqpoint{2.007684in}{3.361642in}}%
\pgfpathlineto{\pgfqpoint{2.017726in}{3.341628in}}%
\pgfpathlineto{\pgfqpoint{2.027731in}{3.322130in}}%
\pgfpathlineto{\pgfqpoint{2.037700in}{3.303141in}}%
\pgfpathlineto{\pgfqpoint{2.047634in}{3.284653in}}%
\pgfpathclose%
\pgfusepath{fill}%
\end{pgfscope}%
\begin{pgfscope}%
\pgfpathrectangle{\pgfqpoint{1.150000in}{0.150000in}}{\pgfqpoint{5.700000in}{5.700000in}}%
\pgfusepath{clip}%
\pgfsetbuttcap%
\pgfsetroundjoin%
\definecolor{currentfill}{rgb}{0.208623,0.367752,0.552675}%
\pgfsetfillcolor{currentfill}%
\pgfsetfillopacity{0.700000}%
\pgfsetlinewidth{0.000000pt}%
\definecolor{currentstroke}{rgb}{0.000000,0.000000,0.000000}%
\pgfsetstrokecolor{currentstroke}%
\pgfsetdash{}{0pt}%
\pgfpathmoveto{\pgfqpoint{2.862627in}{2.189726in}}%
\pgfpathlineto{\pgfqpoint{2.876718in}{2.174885in}}%
\pgfpathlineto{\pgfqpoint{2.890807in}{2.160175in}}%
\pgfpathlineto{\pgfqpoint{2.904894in}{2.145596in}}%
\pgfpathlineto{\pgfqpoint{2.918978in}{2.131148in}}%
\pgfpathlineto{\pgfqpoint{2.909978in}{2.140834in}}%
\pgfpathlineto{\pgfqpoint{2.900955in}{2.150941in}}%
\pgfpathlineto{\pgfqpoint{2.891910in}{2.161478in}}%
\pgfpathlineto{\pgfqpoint{2.882842in}{2.172452in}}%
\pgfpathlineto{\pgfqpoint{2.868703in}{2.187554in}}%
\pgfpathlineto{\pgfqpoint{2.854561in}{2.202787in}}%
\pgfpathlineto{\pgfqpoint{2.840417in}{2.218152in}}%
\pgfpathlineto{\pgfqpoint{2.826270in}{2.233650in}}%
\pgfpathlineto{\pgfqpoint{2.835394in}{2.222012in}}%
\pgfpathlineto{\pgfqpoint{2.844495in}{2.210817in}}%
\pgfpathlineto{\pgfqpoint{2.853572in}{2.200058in}}%
\pgfpathlineto{\pgfqpoint{2.862627in}{2.189726in}}%
\pgfpathclose%
\pgfusepath{fill}%
\end{pgfscope}%
\begin{pgfscope}%
\pgfpathrectangle{\pgfqpoint{1.150000in}{0.150000in}}{\pgfqpoint{5.700000in}{5.700000in}}%
\pgfusepath{clip}%
\pgfsetbuttcap%
\pgfsetroundjoin%
\definecolor{currentfill}{rgb}{0.157729,0.485932,0.558013}%
\pgfsetfillcolor{currentfill}%
\pgfsetfillopacity{0.700000}%
\pgfsetlinewidth{0.000000pt}%
\definecolor{currentstroke}{rgb}{0.000000,0.000000,0.000000}%
\pgfsetstrokecolor{currentstroke}%
\pgfsetdash{}{0pt}%
\pgfpathmoveto{\pgfqpoint{2.580140in}{2.515629in}}%
\pgfpathlineto{\pgfqpoint{2.594303in}{2.497958in}}%
\pgfpathlineto{\pgfqpoint{2.608461in}{2.480438in}}%
\pgfpathlineto{\pgfqpoint{2.622614in}{2.463069in}}%
\pgfpathlineto{\pgfqpoint{2.636762in}{2.445848in}}%
\pgfpathlineto{\pgfqpoint{2.627476in}{2.458796in}}%
\pgfpathlineto{\pgfqpoint{2.618163in}{2.472198in}}%
\pgfpathlineto{\pgfqpoint{2.608824in}{2.486062in}}%
\pgfpathlineto{\pgfqpoint{2.599456in}{2.500395in}}%
\pgfpathlineto{\pgfqpoint{2.585245in}{2.518288in}}%
\pgfpathlineto{\pgfqpoint{2.571028in}{2.536332in}}%
\pgfpathlineto{\pgfqpoint{2.556806in}{2.554526in}}%
\pgfpathlineto{\pgfqpoint{2.542578in}{2.572873in}}%
\pgfpathlineto{\pgfqpoint{2.552011in}{2.557856in}}%
\pgfpathlineto{\pgfqpoint{2.561415in}{2.543314in}}%
\pgfpathlineto{\pgfqpoint{2.570791in}{2.529241in}}%
\pgfpathlineto{\pgfqpoint{2.580140in}{2.515629in}}%
\pgfpathclose%
\pgfusepath{fill}%
\end{pgfscope}%
\begin{pgfscope}%
\pgfpathrectangle{\pgfqpoint{1.150000in}{0.150000in}}{\pgfqpoint{5.700000in}{5.700000in}}%
\pgfusepath{clip}%
\pgfsetbuttcap%
\pgfsetroundjoin%
\definecolor{currentfill}{rgb}{0.220057,0.343307,0.549413}%
\pgfsetfillcolor{currentfill}%
\pgfsetfillopacity{0.700000}%
\pgfsetlinewidth{0.000000pt}%
\definecolor{currentstroke}{rgb}{0.000000,0.000000,0.000000}%
\pgfsetstrokecolor{currentstroke}%
\pgfsetdash{}{0pt}%
\pgfpathmoveto{\pgfqpoint{2.918978in}{2.131148in}}%
\pgfpathlineto{\pgfqpoint{2.933061in}{2.116829in}}%
\pgfpathlineto{\pgfqpoint{2.947143in}{2.102640in}}%
\pgfpathlineto{\pgfqpoint{2.961223in}{2.088578in}}%
\pgfpathlineto{\pgfqpoint{2.975301in}{2.074644in}}%
\pgfpathlineto{\pgfqpoint{2.966352in}{2.083687in}}%
\pgfpathlineto{\pgfqpoint{2.957383in}{2.093146in}}%
\pgfpathlineto{\pgfqpoint{2.948391in}{2.103028in}}%
\pgfpathlineto{\pgfqpoint{2.939377in}{2.113340in}}%
\pgfpathlineto{\pgfqpoint{2.925246in}{2.127925in}}%
\pgfpathlineto{\pgfqpoint{2.911114in}{2.142638in}}%
\pgfpathlineto{\pgfqpoint{2.896979in}{2.157480in}}%
\pgfpathlineto{\pgfqpoint{2.882842in}{2.172452in}}%
\pgfpathlineto{\pgfqpoint{2.891910in}{2.161478in}}%
\pgfpathlineto{\pgfqpoint{2.900955in}{2.150941in}}%
\pgfpathlineto{\pgfqpoint{2.909978in}{2.140834in}}%
\pgfpathlineto{\pgfqpoint{2.918978in}{2.131148in}}%
\pgfpathclose%
\pgfusepath{fill}%
\end{pgfscope}%
\begin{pgfscope}%
\pgfpathrectangle{\pgfqpoint{1.150000in}{0.150000in}}{\pgfqpoint{5.700000in}{5.700000in}}%
\pgfusepath{clip}%
\pgfsetbuttcap%
\pgfsetroundjoin%
\definecolor{currentfill}{rgb}{0.277018,0.050344,0.375715}%
\pgfsetfillcolor{currentfill}%
\pgfsetfillopacity{0.700000}%
\pgfsetlinewidth{0.000000pt}%
\definecolor{currentstroke}{rgb}{0.000000,0.000000,0.000000}%
\pgfsetstrokecolor{currentstroke}%
\pgfsetdash{}{0pt}%
\pgfpathmoveto{\pgfqpoint{3.864407in}{1.500349in}}%
\pgfpathlineto{\pgfqpoint{3.878496in}{1.494533in}}%
\pgfpathlineto{\pgfqpoint{3.892590in}{1.488818in}}%
\pgfpathlineto{\pgfqpoint{3.906689in}{1.483202in}}%
\pgfpathlineto{\pgfqpoint{3.920795in}{1.477687in}}%
\pgfpathlineto{\pgfqpoint{3.912553in}{1.473853in}}%
\pgfpathlineto{\pgfqpoint{3.904302in}{1.470276in}}%
\pgfpathlineto{\pgfqpoint{3.896043in}{1.466964in}}%
\pgfpathlineto{\pgfqpoint{3.887775in}{1.463921in}}%
\pgfpathlineto{\pgfqpoint{3.873647in}{1.469999in}}%
\pgfpathlineto{\pgfqpoint{3.859525in}{1.476177in}}%
\pgfpathlineto{\pgfqpoint{3.845408in}{1.482456in}}%
\pgfpathlineto{\pgfqpoint{3.831296in}{1.488834in}}%
\pgfpathlineto{\pgfqpoint{3.839588in}{1.491307in}}%
\pgfpathlineto{\pgfqpoint{3.847870in}{1.494054in}}%
\pgfpathlineto{\pgfqpoint{3.856143in}{1.497070in}}%
\pgfpathlineto{\pgfqpoint{3.864407in}{1.500349in}}%
\pgfpathclose%
\pgfusepath{fill}%
\end{pgfscope}%
\begin{pgfscope}%
\pgfpathrectangle{\pgfqpoint{1.150000in}{0.150000in}}{\pgfqpoint{5.700000in}{5.700000in}}%
\pgfusepath{clip}%
\pgfsetbuttcap%
\pgfsetroundjoin%
\definecolor{currentfill}{rgb}{0.147607,0.511733,0.557049}%
\pgfsetfillcolor{currentfill}%
\pgfsetfillopacity{0.700000}%
\pgfsetlinewidth{0.000000pt}%
\definecolor{currentstroke}{rgb}{0.000000,0.000000,0.000000}%
\pgfsetstrokecolor{currentstroke}%
\pgfsetdash{}{0pt}%
\pgfpathmoveto{\pgfqpoint{2.523436in}{2.587846in}}%
\pgfpathlineto{\pgfqpoint{2.537620in}{2.569559in}}%
\pgfpathlineto{\pgfqpoint{2.551798in}{2.551428in}}%
\pgfpathlineto{\pgfqpoint{2.565972in}{2.533452in}}%
\pgfpathlineto{\pgfqpoint{2.580140in}{2.515629in}}%
\pgfpathlineto{\pgfqpoint{2.570791in}{2.529241in}}%
\pgfpathlineto{\pgfqpoint{2.561415in}{2.543314in}}%
\pgfpathlineto{\pgfqpoint{2.552011in}{2.557856in}}%
\pgfpathlineto{\pgfqpoint{2.542578in}{2.572873in}}%
\pgfpathlineto{\pgfqpoint{2.528345in}{2.591372in}}%
\pgfpathlineto{\pgfqpoint{2.514107in}{2.610027in}}%
\pgfpathlineto{\pgfqpoint{2.499863in}{2.628836in}}%
\pgfpathlineto{\pgfqpoint{2.485612in}{2.647803in}}%
\pgfpathlineto{\pgfqpoint{2.495112in}{2.632097in}}%
\pgfpathlineto{\pgfqpoint{2.504582in}{2.616874in}}%
\pgfpathlineto{\pgfqpoint{2.514023in}{2.602126in}}%
\pgfpathlineto{\pgfqpoint{2.523436in}{2.587846in}}%
\pgfpathclose%
\pgfusepath{fill}%
\end{pgfscope}%
\begin{pgfscope}%
\pgfpathrectangle{\pgfqpoint{1.150000in}{0.150000in}}{\pgfqpoint{5.700000in}{5.700000in}}%
\pgfusepath{clip}%
\pgfsetbuttcap%
\pgfsetroundjoin%
\definecolor{currentfill}{rgb}{0.273809,0.031497,0.358853}%
\pgfsetfillcolor{currentfill}%
\pgfsetfillopacity{0.700000}%
\pgfsetlinewidth{0.000000pt}%
\definecolor{currentstroke}{rgb}{0.000000,0.000000,0.000000}%
\pgfsetstrokecolor{currentstroke}%
\pgfsetdash{}{0pt}%
\pgfpathmoveto{\pgfqpoint{4.155739in}{1.471308in}}%
\pgfpathlineto{\pgfqpoint{4.169893in}{1.468140in}}%
\pgfpathlineto{\pgfqpoint{4.184055in}{1.465068in}}%
\pgfpathlineto{\pgfqpoint{4.198224in}{1.462094in}}%
\pgfpathlineto{\pgfqpoint{4.212400in}{1.459217in}}%
\pgfpathlineto{\pgfqpoint{4.204284in}{1.451450in}}%
\pgfpathlineto{\pgfqpoint{4.196162in}{1.443879in}}%
\pgfpathlineto{\pgfqpoint{4.188034in}{1.436508in}}%
\pgfpathlineto{\pgfqpoint{4.179899in}{1.429343in}}%
\pgfpathlineto{\pgfqpoint{4.165709in}{1.432745in}}%
\pgfpathlineto{\pgfqpoint{4.151526in}{1.436244in}}%
\pgfpathlineto{\pgfqpoint{4.137350in}{1.439839in}}%
\pgfpathlineto{\pgfqpoint{4.123182in}{1.443532in}}%
\pgfpathlineto{\pgfqpoint{4.131331in}{1.450166in}}%
\pgfpathlineto{\pgfqpoint{4.139473in}{1.457010in}}%
\pgfpathlineto{\pgfqpoint{4.147609in}{1.464059in}}%
\pgfpathlineto{\pgfqpoint{4.155739in}{1.471308in}}%
\pgfpathclose%
\pgfusepath{fill}%
\end{pgfscope}%
\begin{pgfscope}%
\pgfpathrectangle{\pgfqpoint{1.150000in}{0.150000in}}{\pgfqpoint{5.700000in}{5.700000in}}%
\pgfusepath{clip}%
\pgfsetbuttcap%
\pgfsetroundjoin%
\definecolor{currentfill}{rgb}{0.171176,0.452530,0.557965}%
\pgfsetfillcolor{currentfill}%
\pgfsetfillopacity{0.700000}%
\pgfsetlinewidth{0.000000pt}%
\definecolor{currentstroke}{rgb}{0.000000,0.000000,0.000000}%
\pgfsetstrokecolor{currentstroke}%
\pgfsetdash{}{0pt}%
\pgfpathmoveto{\pgfqpoint{5.412527in}{2.426267in}}%
\pgfpathlineto{\pgfqpoint{5.427277in}{2.434169in}}%
\pgfpathlineto{\pgfqpoint{5.442042in}{2.442172in}}%
\pgfpathlineto{\pgfqpoint{5.456824in}{2.450275in}}%
\pgfpathlineto{\pgfqpoint{5.471621in}{2.458479in}}%
\pgfpathlineto{\pgfqpoint{5.463804in}{2.443697in}}%
\pgfpathlineto{\pgfqpoint{5.455981in}{2.428821in}}%
\pgfpathlineto{\pgfqpoint{5.448152in}{2.413853in}}%
\pgfpathlineto{\pgfqpoint{5.440317in}{2.398794in}}%
\pgfpathlineto{\pgfqpoint{5.425524in}{2.390834in}}%
\pgfpathlineto{\pgfqpoint{5.410748in}{2.382975in}}%
\pgfpathlineto{\pgfqpoint{5.395988in}{2.375215in}}%
\pgfpathlineto{\pgfqpoint{5.381243in}{2.367556in}}%
\pgfpathlineto{\pgfqpoint{5.389073in}{2.382363in}}%
\pgfpathlineto{\pgfqpoint{5.396897in}{2.397086in}}%
\pgfpathlineto{\pgfqpoint{5.404715in}{2.411721in}}%
\pgfpathlineto{\pgfqpoint{5.412527in}{2.426267in}}%
\pgfpathclose%
\pgfusepath{fill}%
\end{pgfscope}%
\begin{pgfscope}%
\pgfpathrectangle{\pgfqpoint{1.150000in}{0.150000in}}{\pgfqpoint{5.700000in}{5.700000in}}%
\pgfusepath{clip}%
\pgfsetbuttcap%
\pgfsetroundjoin%
\definecolor{currentfill}{rgb}{0.282910,0.105393,0.426902}%
\pgfsetfillcolor{currentfill}%
\pgfsetfillopacity{0.700000}%
\pgfsetlinewidth{0.000000pt}%
\definecolor{currentstroke}{rgb}{0.000000,0.000000,0.000000}%
\pgfsetstrokecolor{currentstroke}%
\pgfsetdash{}{0pt}%
\pgfpathmoveto{\pgfqpoint{4.568946in}{1.603773in}}%
\pgfpathlineto{\pgfqpoint{4.583245in}{1.604454in}}%
\pgfpathlineto{\pgfqpoint{4.597553in}{1.605230in}}%
\pgfpathlineto{\pgfqpoint{4.611872in}{1.606103in}}%
\pgfpathlineto{\pgfqpoint{4.626202in}{1.607071in}}%
\pgfpathlineto{\pgfqpoint{4.618197in}{1.594612in}}%
\pgfpathlineto{\pgfqpoint{4.610188in}{1.582250in}}%
\pgfpathlineto{\pgfqpoint{4.602176in}{1.569989in}}%
\pgfpathlineto{\pgfqpoint{4.594159in}{1.557833in}}%
\pgfpathlineto{\pgfqpoint{4.579827in}{1.557320in}}%
\pgfpathlineto{\pgfqpoint{4.565505in}{1.556903in}}%
\pgfpathlineto{\pgfqpoint{4.551193in}{1.556581in}}%
\pgfpathlineto{\pgfqpoint{4.536891in}{1.556356in}}%
\pgfpathlineto{\pgfqpoint{4.544911in}{1.568050in}}%
\pgfpathlineto{\pgfqpoint{4.552927in}{1.579854in}}%
\pgfpathlineto{\pgfqpoint{4.560939in}{1.591763in}}%
\pgfpathlineto{\pgfqpoint{4.568946in}{1.603773in}}%
\pgfpathclose%
\pgfusepath{fill}%
\end{pgfscope}%
\begin{pgfscope}%
\pgfpathrectangle{\pgfqpoint{1.150000in}{0.150000in}}{\pgfqpoint{5.700000in}{5.700000in}}%
\pgfusepath{clip}%
\pgfsetbuttcap%
\pgfsetroundjoin%
\definecolor{currentfill}{rgb}{0.229739,0.322361,0.545706}%
\pgfsetfillcolor{currentfill}%
\pgfsetfillopacity{0.700000}%
\pgfsetlinewidth{0.000000pt}%
\definecolor{currentstroke}{rgb}{0.000000,0.000000,0.000000}%
\pgfsetstrokecolor{currentstroke}%
\pgfsetdash{}{0pt}%
\pgfpathmoveto{\pgfqpoint{2.975301in}{2.074644in}}%
\pgfpathlineto{\pgfqpoint{2.989378in}{2.060836in}}%
\pgfpathlineto{\pgfqpoint{3.003453in}{2.047154in}}%
\pgfpathlineto{\pgfqpoint{3.017528in}{2.033598in}}%
\pgfpathlineto{\pgfqpoint{3.031601in}{2.020166in}}%
\pgfpathlineto{\pgfqpoint{3.022703in}{2.028569in}}%
\pgfpathlineto{\pgfqpoint{3.013784in}{2.037382in}}%
\pgfpathlineto{\pgfqpoint{3.004845in}{2.046612in}}%
\pgfpathlineto{\pgfqpoint{2.995884in}{2.056267in}}%
\pgfpathlineto{\pgfqpoint{2.981760in}{2.070346in}}%
\pgfpathlineto{\pgfqpoint{2.967634in}{2.084551in}}%
\pgfpathlineto{\pgfqpoint{2.953506in}{2.098882in}}%
\pgfpathlineto{\pgfqpoint{2.939377in}{2.113340in}}%
\pgfpathlineto{\pgfqpoint{2.948391in}{2.103028in}}%
\pgfpathlineto{\pgfqpoint{2.957383in}{2.093146in}}%
\pgfpathlineto{\pgfqpoint{2.966352in}{2.083687in}}%
\pgfpathlineto{\pgfqpoint{2.975301in}{2.074644in}}%
\pgfpathclose%
\pgfusepath{fill}%
\end{pgfscope}%
\begin{pgfscope}%
\pgfpathrectangle{\pgfqpoint{1.150000in}{0.150000in}}{\pgfqpoint{5.700000in}{5.700000in}}%
\pgfusepath{clip}%
\pgfsetbuttcap%
\pgfsetroundjoin%
\definecolor{currentfill}{rgb}{0.280894,0.078907,0.402329}%
\pgfsetfillcolor{currentfill}%
\pgfsetfillopacity{0.700000}%
\pgfsetlinewidth{0.000000pt}%
\definecolor{currentstroke}{rgb}{0.000000,0.000000,0.000000}%
\pgfsetstrokecolor{currentstroke}%
\pgfsetdash{}{0pt}%
\pgfpathmoveto{\pgfqpoint{4.479783in}{1.556411in}}%
\pgfpathlineto{\pgfqpoint{4.494045in}{1.556254in}}%
\pgfpathlineto{\pgfqpoint{4.508317in}{1.556192in}}%
\pgfpathlineto{\pgfqpoint{4.522599in}{1.556226in}}%
\pgfpathlineto{\pgfqpoint{4.536891in}{1.556356in}}%
\pgfpathlineto{\pgfqpoint{4.528867in}{1.544776in}}%
\pgfpathlineto{\pgfqpoint{4.520839in}{1.533315in}}%
\pgfpathlineto{\pgfqpoint{4.512807in}{1.521977in}}%
\pgfpathlineto{\pgfqpoint{4.504771in}{1.510768in}}%
\pgfpathlineto{\pgfqpoint{4.490474in}{1.511110in}}%
\pgfpathlineto{\pgfqpoint{4.476187in}{1.511549in}}%
\pgfpathlineto{\pgfqpoint{4.461910in}{1.512083in}}%
\pgfpathlineto{\pgfqpoint{4.447641in}{1.512713in}}%
\pgfpathlineto{\pgfqpoint{4.455683in}{1.523443in}}%
\pgfpathlineto{\pgfqpoint{4.463721in}{1.534306in}}%
\pgfpathlineto{\pgfqpoint{4.471754in}{1.545297in}}%
\pgfpathlineto{\pgfqpoint{4.479783in}{1.556411in}}%
\pgfpathclose%
\pgfusepath{fill}%
\end{pgfscope}%
\begin{pgfscope}%
\pgfpathrectangle{\pgfqpoint{1.150000in}{0.150000in}}{\pgfqpoint{5.700000in}{5.700000in}}%
\pgfusepath{clip}%
\pgfsetbuttcap%
\pgfsetroundjoin%
\definecolor{currentfill}{rgb}{0.136408,0.541173,0.554483}%
\pgfsetfillcolor{currentfill}%
\pgfsetfillopacity{0.700000}%
\pgfsetlinewidth{0.000000pt}%
\definecolor{currentstroke}{rgb}{0.000000,0.000000,0.000000}%
\pgfsetstrokecolor{currentstroke}%
\pgfsetdash{}{0pt}%
\pgfpathmoveto{\pgfqpoint{2.466641in}{2.662572in}}%
\pgfpathlineto{\pgfqpoint{2.480849in}{2.643651in}}%
\pgfpathlineto{\pgfqpoint{2.495051in}{2.624890in}}%
\pgfpathlineto{\pgfqpoint{2.509246in}{2.606289in}}%
\pgfpathlineto{\pgfqpoint{2.523436in}{2.587846in}}%
\pgfpathlineto{\pgfqpoint{2.514023in}{2.602126in}}%
\pgfpathlineto{\pgfqpoint{2.504582in}{2.616874in}}%
\pgfpathlineto{\pgfqpoint{2.495112in}{2.632097in}}%
\pgfpathlineto{\pgfqpoint{2.485612in}{2.647803in}}%
\pgfpathlineto{\pgfqpoint{2.471356in}{2.666928in}}%
\pgfpathlineto{\pgfqpoint{2.457093in}{2.686212in}}%
\pgfpathlineto{\pgfqpoint{2.442824in}{2.705656in}}%
\pgfpathlineto{\pgfqpoint{2.428548in}{2.725262in}}%
\pgfpathlineto{\pgfqpoint{2.438116in}{2.708863in}}%
\pgfpathlineto{\pgfqpoint{2.447654in}{2.692953in}}%
\pgfpathlineto{\pgfqpoint{2.457162in}{2.677526in}}%
\pgfpathlineto{\pgfqpoint{2.466641in}{2.662572in}}%
\pgfpathclose%
\pgfusepath{fill}%
\end{pgfscope}%
\begin{pgfscope}%
\pgfpathrectangle{\pgfqpoint{1.150000in}{0.150000in}}{\pgfqpoint{5.700000in}{5.700000in}}%
\pgfusepath{clip}%
\pgfsetbuttcap%
\pgfsetroundjoin%
\definecolor{currentfill}{rgb}{0.283072,0.130895,0.449241}%
\pgfsetfillcolor{currentfill}%
\pgfsetfillopacity{0.700000}%
\pgfsetlinewidth{0.000000pt}%
\definecolor{currentstroke}{rgb}{0.000000,0.000000,0.000000}%
\pgfsetstrokecolor{currentstroke}%
\pgfsetdash{}{0pt}%
\pgfpathmoveto{\pgfqpoint{4.658183in}{1.657788in}}%
\pgfpathlineto{\pgfqpoint{4.672522in}{1.659290in}}%
\pgfpathlineto{\pgfqpoint{4.686871in}{1.660888in}}%
\pgfpathlineto{\pgfqpoint{4.701231in}{1.662582in}}%
\pgfpathlineto{\pgfqpoint{4.715603in}{1.664373in}}%
\pgfpathlineto{\pgfqpoint{4.707614in}{1.651138in}}%
\pgfpathlineto{\pgfqpoint{4.699622in}{1.637979in}}%
\pgfpathlineto{\pgfqpoint{4.691626in}{1.624899in}}%
\pgfpathlineto{\pgfqpoint{4.683627in}{1.611903in}}%
\pgfpathlineto{\pgfqpoint{4.669254in}{1.610551in}}%
\pgfpathlineto{\pgfqpoint{4.654893in}{1.609296in}}%
\pgfpathlineto{\pgfqpoint{4.640542in}{1.608136in}}%
\pgfpathlineto{\pgfqpoint{4.626202in}{1.607071in}}%
\pgfpathlineto{\pgfqpoint{4.634203in}{1.619623in}}%
\pgfpathlineto{\pgfqpoint{4.642200in}{1.632262in}}%
\pgfpathlineto{\pgfqpoint{4.650194in}{1.644985in}}%
\pgfpathlineto{\pgfqpoint{4.658183in}{1.657788in}}%
\pgfpathclose%
\pgfusepath{fill}%
\end{pgfscope}%
\begin{pgfscope}%
\pgfpathrectangle{\pgfqpoint{1.150000in}{0.150000in}}{\pgfqpoint{5.700000in}{5.700000in}}%
\pgfusepath{clip}%
\pgfsetbuttcap%
\pgfsetroundjoin%
\definecolor{currentfill}{rgb}{0.258965,0.251537,0.524736}%
\pgfsetfillcolor{currentfill}%
\pgfsetfillopacity{0.700000}%
\pgfsetlinewidth{0.000000pt}%
\definecolor{currentstroke}{rgb}{0.000000,0.000000,0.000000}%
\pgfsetstrokecolor{currentstroke}%
\pgfsetdash{}{0pt}%
\pgfpathmoveto{\pgfqpoint{4.958346in}{1.913563in}}%
\pgfpathlineto{\pgfqpoint{4.972833in}{1.917803in}}%
\pgfpathlineto{\pgfqpoint{4.987333in}{1.922141in}}%
\pgfpathlineto{\pgfqpoint{5.001846in}{1.926576in}}%
\pgfpathlineto{\pgfqpoint{5.016372in}{1.931108in}}%
\pgfpathlineto{\pgfqpoint{5.008436in}{1.916111in}}%
\pgfpathlineto{\pgfqpoint{5.000497in}{1.901117in}}%
\pgfpathlineto{\pgfqpoint{4.992553in}{1.886127in}}%
\pgfpathlineto{\pgfqpoint{4.984606in}{1.871145in}}%
\pgfpathlineto{\pgfqpoint{4.970084in}{1.866983in}}%
\pgfpathlineto{\pgfqpoint{4.955575in}{1.862919in}}%
\pgfpathlineto{\pgfqpoint{4.941078in}{1.858951in}}%
\pgfpathlineto{\pgfqpoint{4.926595in}{1.855080in}}%
\pgfpathlineto{\pgfqpoint{4.934538in}{1.869685in}}%
\pgfpathlineto{\pgfqpoint{4.942478in}{1.884303in}}%
\pgfpathlineto{\pgfqpoint{4.950414in}{1.898930in}}%
\pgfpathlineto{\pgfqpoint{4.958346in}{1.913563in}}%
\pgfpathclose%
\pgfusepath{fill}%
\end{pgfscope}%
\begin{pgfscope}%
\pgfpathrectangle{\pgfqpoint{1.150000in}{0.150000in}}{\pgfqpoint{5.700000in}{5.700000in}}%
\pgfusepath{clip}%
\pgfsetbuttcap%
\pgfsetroundjoin%
\definecolor{currentfill}{rgb}{0.194100,0.399323,0.555565}%
\pgfsetfillcolor{currentfill}%
\pgfsetfillopacity{0.700000}%
\pgfsetlinewidth{0.000000pt}%
\definecolor{currentstroke}{rgb}{0.000000,0.000000,0.000000}%
\pgfsetstrokecolor{currentstroke}%
\pgfsetdash{}{0pt}%
\pgfpathmoveto{\pgfqpoint{5.291062in}{2.278927in}}%
\pgfpathlineto{\pgfqpoint{5.305740in}{2.285926in}}%
\pgfpathlineto{\pgfqpoint{5.320433in}{2.293024in}}%
\pgfpathlineto{\pgfqpoint{5.335141in}{2.300222in}}%
\pgfpathlineto{\pgfqpoint{5.349865in}{2.307520in}}%
\pgfpathlineto{\pgfqpoint{5.342006in}{2.292321in}}%
\pgfpathlineto{\pgfqpoint{5.334142in}{2.277051in}}%
\pgfpathlineto{\pgfqpoint{5.326273in}{2.261712in}}%
\pgfpathlineto{\pgfqpoint{5.318398in}{2.246307in}}%
\pgfpathlineto{\pgfqpoint{5.303680in}{2.239291in}}%
\pgfpathlineto{\pgfqpoint{5.288977in}{2.232374in}}%
\pgfpathlineto{\pgfqpoint{5.274289in}{2.225556in}}%
\pgfpathlineto{\pgfqpoint{5.259617in}{2.218837in}}%
\pgfpathlineto{\pgfqpoint{5.267486in}{2.233954in}}%
\pgfpathlineto{\pgfqpoint{5.275350in}{2.249010in}}%
\pgfpathlineto{\pgfqpoint{5.283208in}{2.264002in}}%
\pgfpathlineto{\pgfqpoint{5.291062in}{2.278927in}}%
\pgfpathclose%
\pgfusepath{fill}%
\end{pgfscope}%
\begin{pgfscope}%
\pgfpathrectangle{\pgfqpoint{1.150000in}{0.150000in}}{\pgfqpoint{5.700000in}{5.700000in}}%
\pgfusepath{clip}%
\pgfsetbuttcap%
\pgfsetroundjoin%
\definecolor{currentfill}{rgb}{0.278791,0.062145,0.386592}%
\pgfsetfillcolor{currentfill}%
\pgfsetfillopacity{0.700000}%
\pgfsetlinewidth{0.000000pt}%
\definecolor{currentstroke}{rgb}{0.000000,0.000000,0.000000}%
\pgfsetstrokecolor{currentstroke}%
\pgfsetdash{}{0pt}%
\pgfpathmoveto{\pgfqpoint{4.390661in}{1.516191in}}%
\pgfpathlineto{\pgfqpoint{4.404892in}{1.515177in}}%
\pgfpathlineto{\pgfqpoint{4.419133in}{1.514260in}}%
\pgfpathlineto{\pgfqpoint{4.433382in}{1.513438in}}%
\pgfpathlineto{\pgfqpoint{4.447641in}{1.512713in}}%
\pgfpathlineto{\pgfqpoint{4.439595in}{1.502119in}}%
\pgfpathlineto{\pgfqpoint{4.431544in}{1.491667in}}%
\pgfpathlineto{\pgfqpoint{4.423489in}{1.481362in}}%
\pgfpathlineto{\pgfqpoint{4.415429in}{1.471209in}}%
\pgfpathlineto{\pgfqpoint{4.401163in}{1.472424in}}%
\pgfpathlineto{\pgfqpoint{4.386906in}{1.473735in}}%
\pgfpathlineto{\pgfqpoint{4.372658in}{1.475142in}}%
\pgfpathlineto{\pgfqpoint{4.358418in}{1.476645in}}%
\pgfpathlineto{\pgfqpoint{4.366486in}{1.486303in}}%
\pgfpathlineto{\pgfqpoint{4.374549in}{1.496116in}}%
\pgfpathlineto{\pgfqpoint{4.382607in}{1.506080in}}%
\pgfpathlineto{\pgfqpoint{4.390661in}{1.516191in}}%
\pgfpathclose%
\pgfusepath{fill}%
\end{pgfscope}%
\begin{pgfscope}%
\pgfpathrectangle{\pgfqpoint{1.150000in}{0.150000in}}{\pgfqpoint{5.700000in}{5.700000in}}%
\pgfusepath{clip}%
\pgfsetbuttcap%
\pgfsetroundjoin%
\definecolor{currentfill}{rgb}{0.282290,0.145912,0.461510}%
\pgfsetfillcolor{currentfill}%
\pgfsetfillopacity{0.700000}%
\pgfsetlinewidth{0.000000pt}%
\definecolor{currentstroke}{rgb}{0.000000,0.000000,0.000000}%
\pgfsetstrokecolor{currentstroke}%
\pgfsetdash{}{0pt}%
\pgfpathmoveto{\pgfqpoint{3.459887in}{1.679781in}}%
\pgfpathlineto{\pgfqpoint{3.473940in}{1.670318in}}%
\pgfpathlineto{\pgfqpoint{3.487996in}{1.660964in}}%
\pgfpathlineto{\pgfqpoint{3.502054in}{1.651717in}}%
\pgfpathlineto{\pgfqpoint{3.516115in}{1.642578in}}%
\pgfpathlineto{\pgfqpoint{3.507619in}{1.644557in}}%
\pgfpathlineto{\pgfqpoint{3.499110in}{1.646873in}}%
\pgfpathlineto{\pgfqpoint{3.490587in}{1.649533in}}%
\pgfpathlineto{\pgfqpoint{3.482049in}{1.652544in}}%
\pgfpathlineto{\pgfqpoint{3.467953in}{1.662292in}}%
\pgfpathlineto{\pgfqpoint{3.453859in}{1.672148in}}%
\pgfpathlineto{\pgfqpoint{3.439767in}{1.682112in}}%
\pgfpathlineto{\pgfqpoint{3.425677in}{1.692184in}}%
\pgfpathlineto{\pgfqpoint{3.434251in}{1.688556in}}%
\pgfpathlineto{\pgfqpoint{3.442811in}{1.685284in}}%
\pgfpathlineto{\pgfqpoint{3.451356in}{1.682361in}}%
\pgfpathlineto{\pgfqpoint{3.459887in}{1.679781in}}%
\pgfpathclose%
\pgfusepath{fill}%
\end{pgfscope}%
\begin{pgfscope}%
\pgfpathrectangle{\pgfqpoint{1.150000in}{0.150000in}}{\pgfqpoint{5.700000in}{5.700000in}}%
\pgfusepath{clip}%
\pgfsetbuttcap%
\pgfsetroundjoin%
\definecolor{currentfill}{rgb}{0.239346,0.300855,0.540844}%
\pgfsetfillcolor{currentfill}%
\pgfsetfillopacity{0.700000}%
\pgfsetlinewidth{0.000000pt}%
\definecolor{currentstroke}{rgb}{0.000000,0.000000,0.000000}%
\pgfsetstrokecolor{currentstroke}%
\pgfsetdash{}{0pt}%
\pgfpathmoveto{\pgfqpoint{3.031601in}{2.020166in}}%
\pgfpathlineto{\pgfqpoint{3.045673in}{2.006858in}}%
\pgfpathlineto{\pgfqpoint{3.059745in}{1.993673in}}%
\pgfpathlineto{\pgfqpoint{3.073816in}{1.980611in}}%
\pgfpathlineto{\pgfqpoint{3.087886in}{1.967671in}}%
\pgfpathlineto{\pgfqpoint{3.079037in}{1.975437in}}%
\pgfpathlineto{\pgfqpoint{3.070168in}{1.983607in}}%
\pgfpathlineto{\pgfqpoint{3.061279in}{1.992188in}}%
\pgfpathlineto{\pgfqpoint{3.052370in}{2.001187in}}%
\pgfpathlineto{\pgfqpoint{3.038250in}{2.014772in}}%
\pgfpathlineto{\pgfqpoint{3.024129in}{2.028480in}}%
\pgfpathlineto{\pgfqpoint{3.010007in}{2.042311in}}%
\pgfpathlineto{\pgfqpoint{2.995884in}{2.056267in}}%
\pgfpathlineto{\pgfqpoint{3.004845in}{2.046612in}}%
\pgfpathlineto{\pgfqpoint{3.013784in}{2.037382in}}%
\pgfpathlineto{\pgfqpoint{3.022703in}{2.028569in}}%
\pgfpathlineto{\pgfqpoint{3.031601in}{2.020166in}}%
\pgfpathclose%
\pgfusepath{fill}%
\end{pgfscope}%
\begin{pgfscope}%
\pgfpathrectangle{\pgfqpoint{1.150000in}{0.150000in}}{\pgfqpoint{5.700000in}{5.700000in}}%
\pgfusepath{clip}%
\pgfsetbuttcap%
\pgfsetroundjoin%
\definecolor{currentfill}{rgb}{0.127568,0.566949,0.550556}%
\pgfsetfillcolor{currentfill}%
\pgfsetfillopacity{0.700000}%
\pgfsetlinewidth{0.000000pt}%
\definecolor{currentstroke}{rgb}{0.000000,0.000000,0.000000}%
\pgfsetstrokecolor{currentstroke}%
\pgfsetdash{}{0pt}%
\pgfpathmoveto{\pgfqpoint{2.409746in}{2.739889in}}%
\pgfpathlineto{\pgfqpoint{2.423980in}{2.720312in}}%
\pgfpathlineto{\pgfqpoint{2.438207in}{2.700901in}}%
\pgfpathlineto{\pgfqpoint{2.452427in}{2.681655in}}%
\pgfpathlineto{\pgfqpoint{2.466641in}{2.662572in}}%
\pgfpathlineto{\pgfqpoint{2.457162in}{2.677526in}}%
\pgfpathlineto{\pgfqpoint{2.447654in}{2.692953in}}%
\pgfpathlineto{\pgfqpoint{2.438116in}{2.708863in}}%
\pgfpathlineto{\pgfqpoint{2.428548in}{2.725262in}}%
\pgfpathlineto{\pgfqpoint{2.414265in}{2.745032in}}%
\pgfpathlineto{\pgfqpoint{2.399975in}{2.764966in}}%
\pgfpathlineto{\pgfqpoint{2.385679in}{2.785066in}}%
\pgfpathlineto{\pgfqpoint{2.371375in}{2.805333in}}%
\pgfpathlineto{\pgfqpoint{2.381014in}{2.788234in}}%
\pgfpathlineto{\pgfqpoint{2.390622in}{2.771633in}}%
\pgfpathlineto{\pgfqpoint{2.400199in}{2.755520in}}%
\pgfpathlineto{\pgfqpoint{2.409746in}{2.739889in}}%
\pgfpathclose%
\pgfusepath{fill}%
\end{pgfscope}%
\begin{pgfscope}%
\pgfpathrectangle{\pgfqpoint{1.150000in}{0.150000in}}{\pgfqpoint{5.700000in}{5.700000in}}%
\pgfusepath{clip}%
\pgfsetbuttcap%
\pgfsetroundjoin%
\definecolor{currentfill}{rgb}{0.280868,0.160771,0.472899}%
\pgfsetfillcolor{currentfill}%
\pgfsetfillopacity{0.700000}%
\pgfsetlinewidth{0.000000pt}%
\definecolor{currentstroke}{rgb}{0.000000,0.000000,0.000000}%
\pgfsetstrokecolor{currentstroke}%
\pgfsetdash{}{0pt}%
\pgfpathmoveto{\pgfqpoint{4.747521in}{1.717980in}}%
\pgfpathlineto{\pgfqpoint{4.761904in}{1.720288in}}%
\pgfpathlineto{\pgfqpoint{4.776298in}{1.722692in}}%
\pgfpathlineto{\pgfqpoint{4.790704in}{1.725192in}}%
\pgfpathlineto{\pgfqpoint{4.805122in}{1.727788in}}%
\pgfpathlineto{\pgfqpoint{4.797147in}{1.713879in}}%
\pgfpathlineto{\pgfqpoint{4.789169in}{1.700024in}}%
\pgfpathlineto{\pgfqpoint{4.781187in}{1.686228in}}%
\pgfpathlineto{\pgfqpoint{4.773202in}{1.672494in}}%
\pgfpathlineto{\pgfqpoint{4.758785in}{1.670320in}}%
\pgfpathlineto{\pgfqpoint{4.744380in}{1.668241in}}%
\pgfpathlineto{\pgfqpoint{4.729986in}{1.666259in}}%
\pgfpathlineto{\pgfqpoint{4.715603in}{1.664373in}}%
\pgfpathlineto{\pgfqpoint{4.723588in}{1.677678in}}%
\pgfpathlineto{\pgfqpoint{4.731569in}{1.691051in}}%
\pgfpathlineto{\pgfqpoint{4.739547in}{1.704486in}}%
\pgfpathlineto{\pgfqpoint{4.747521in}{1.717980in}}%
\pgfpathclose%
\pgfusepath{fill}%
\end{pgfscope}%
\begin{pgfscope}%
\pgfpathrectangle{\pgfqpoint{1.150000in}{0.150000in}}{\pgfqpoint{5.700000in}{5.700000in}}%
\pgfusepath{clip}%
\pgfsetbuttcap%
\pgfsetroundjoin%
\definecolor{currentfill}{rgb}{0.280894,0.078907,0.402329}%
\pgfsetfillcolor{currentfill}%
\pgfsetfillopacity{0.700000}%
\pgfsetlinewidth{0.000000pt}%
\definecolor{currentstroke}{rgb}{0.000000,0.000000,0.000000}%
\pgfsetstrokecolor{currentstroke}%
\pgfsetdash{}{0pt}%
\pgfpathmoveto{\pgfqpoint{3.718568in}{1.543508in}}%
\pgfpathlineto{\pgfqpoint{3.732644in}{1.536317in}}%
\pgfpathlineto{\pgfqpoint{3.746723in}{1.529229in}}%
\pgfpathlineto{\pgfqpoint{3.760807in}{1.522242in}}%
\pgfpathlineto{\pgfqpoint{3.774896in}{1.515358in}}%
\pgfpathlineto{\pgfqpoint{3.766568in}{1.513742in}}%
\pgfpathlineto{\pgfqpoint{3.758231in}{1.512416in}}%
\pgfpathlineto{\pgfqpoint{3.749883in}{1.511389in}}%
\pgfpathlineto{\pgfqpoint{3.741524in}{1.510665in}}%
\pgfpathlineto{\pgfqpoint{3.727408in}{1.518133in}}%
\pgfpathlineto{\pgfqpoint{3.713297in}{1.525703in}}%
\pgfpathlineto{\pgfqpoint{3.699189in}{1.533376in}}%
\pgfpathlineto{\pgfqpoint{3.685085in}{1.541151in}}%
\pgfpathlineto{\pgfqpoint{3.693473in}{1.541283in}}%
\pgfpathlineto{\pgfqpoint{3.701849in}{1.541724in}}%
\pgfpathlineto{\pgfqpoint{3.710214in}{1.542468in}}%
\pgfpathlineto{\pgfqpoint{3.718568in}{1.543508in}}%
\pgfpathclose%
\pgfusepath{fill}%
\end{pgfscope}%
\begin{pgfscope}%
\pgfpathrectangle{\pgfqpoint{1.150000in}{0.150000in}}{\pgfqpoint{5.700000in}{5.700000in}}%
\pgfusepath{clip}%
\pgfsetbuttcap%
\pgfsetroundjoin%
\definecolor{currentfill}{rgb}{0.218130,0.347432,0.550038}%
\pgfsetfillcolor{currentfill}%
\pgfsetfillopacity{0.700000}%
\pgfsetlinewidth{0.000000pt}%
\definecolor{currentstroke}{rgb}{0.000000,0.000000,0.000000}%
\pgfsetstrokecolor{currentstroke}%
\pgfsetdash{}{0pt}%
\pgfpathmoveto{\pgfqpoint{5.169567in}{2.133117in}}%
\pgfpathlineto{\pgfqpoint{5.184176in}{2.139142in}}%
\pgfpathlineto{\pgfqpoint{5.198800in}{2.145265in}}%
\pgfpathlineto{\pgfqpoint{5.213437in}{2.151487in}}%
\pgfpathlineto{\pgfqpoint{5.228090in}{2.157808in}}%
\pgfpathlineto{\pgfqpoint{5.220196in}{2.142424in}}%
\pgfpathlineto{\pgfqpoint{5.212297in}{2.126995in}}%
\pgfpathlineto{\pgfqpoint{5.204394in}{2.111523in}}%
\pgfpathlineto{\pgfqpoint{5.196486in}{2.096012in}}%
\pgfpathlineto{\pgfqpoint{5.181840in}{2.090009in}}%
\pgfpathlineto{\pgfqpoint{5.167207in}{2.084105in}}%
\pgfpathlineto{\pgfqpoint{5.152589in}{2.078298in}}%
\pgfpathlineto{\pgfqpoint{5.137985in}{2.072590in}}%
\pgfpathlineto{\pgfqpoint{5.145887in}{2.087777in}}%
\pgfpathlineto{\pgfqpoint{5.153785in}{2.102929in}}%
\pgfpathlineto{\pgfqpoint{5.161679in}{2.118043in}}%
\pgfpathlineto{\pgfqpoint{5.169567in}{2.133117in}}%
\pgfpathclose%
\pgfusepath{fill}%
\end{pgfscope}%
\begin{pgfscope}%
\pgfpathrectangle{\pgfqpoint{1.150000in}{0.150000in}}{\pgfqpoint{5.700000in}{5.700000in}}%
\pgfusepath{clip}%
\pgfsetbuttcap%
\pgfsetroundjoin%
\definecolor{currentfill}{rgb}{0.246811,0.283237,0.535941}%
\pgfsetfillcolor{currentfill}%
\pgfsetfillopacity{0.700000}%
\pgfsetlinewidth{0.000000pt}%
\definecolor{currentstroke}{rgb}{0.000000,0.000000,0.000000}%
\pgfsetstrokecolor{currentstroke}%
\pgfsetdash{}{0pt}%
\pgfpathmoveto{\pgfqpoint{3.087886in}{1.967671in}}%
\pgfpathlineto{\pgfqpoint{3.101956in}{1.954852in}}%
\pgfpathlineto{\pgfqpoint{3.116026in}{1.942153in}}%
\pgfpathlineto{\pgfqpoint{3.130095in}{1.929575in}}%
\pgfpathlineto{\pgfqpoint{3.144164in}{1.917116in}}%
\pgfpathlineto{\pgfqpoint{3.135362in}{1.924248in}}%
\pgfpathlineto{\pgfqpoint{3.126541in}{1.931777in}}%
\pgfpathlineto{\pgfqpoint{3.117701in}{1.939711in}}%
\pgfpathlineto{\pgfqpoint{3.108841in}{1.948058in}}%
\pgfpathlineto{\pgfqpoint{3.094724in}{1.961159in}}%
\pgfpathlineto{\pgfqpoint{3.080607in}{1.974381in}}%
\pgfpathlineto{\pgfqpoint{3.066489in}{1.987723in}}%
\pgfpathlineto{\pgfqpoint{3.052370in}{2.001187in}}%
\pgfpathlineto{\pgfqpoint{3.061279in}{1.992188in}}%
\pgfpathlineto{\pgfqpoint{3.070168in}{1.983607in}}%
\pgfpathlineto{\pgfqpoint{3.079037in}{1.975437in}}%
\pgfpathlineto{\pgfqpoint{3.087886in}{1.967671in}}%
\pgfpathclose%
\pgfusepath{fill}%
\end{pgfscope}%
\begin{pgfscope}%
\pgfpathrectangle{\pgfqpoint{1.150000in}{0.150000in}}{\pgfqpoint{5.700000in}{5.700000in}}%
\pgfusepath{clip}%
\pgfsetbuttcap%
\pgfsetroundjoin%
\definecolor{currentfill}{rgb}{0.276022,0.044167,0.370164}%
\pgfsetfillcolor{currentfill}%
\pgfsetfillopacity{0.700000}%
\pgfsetlinewidth{0.000000pt}%
\definecolor{currentstroke}{rgb}{0.000000,0.000000,0.000000}%
\pgfsetstrokecolor{currentstroke}%
\pgfsetdash{}{0pt}%
\pgfpathmoveto{\pgfqpoint{4.301546in}{1.483618in}}%
\pgfpathlineto{\pgfqpoint{4.315751in}{1.481730in}}%
\pgfpathlineto{\pgfqpoint{4.329965in}{1.479939in}}%
\pgfpathlineto{\pgfqpoint{4.344187in}{1.478244in}}%
\pgfpathlineto{\pgfqpoint{4.358418in}{1.476645in}}%
\pgfpathlineto{\pgfqpoint{4.350345in}{1.467149in}}%
\pgfpathlineto{\pgfqpoint{4.342267in}{1.457818in}}%
\pgfpathlineto{\pgfqpoint{4.334184in}{1.448658in}}%
\pgfpathlineto{\pgfqpoint{4.326096in}{1.439674in}}%
\pgfpathlineto{\pgfqpoint{4.311856in}{1.441780in}}%
\pgfpathlineto{\pgfqpoint{4.297624in}{1.443982in}}%
\pgfpathlineto{\pgfqpoint{4.283400in}{1.446281in}}%
\pgfpathlineto{\pgfqpoint{4.269184in}{1.448675in}}%
\pgfpathlineto{\pgfqpoint{4.277283in}{1.457145in}}%
\pgfpathlineto{\pgfqpoint{4.285376in}{1.465796in}}%
\pgfpathlineto{\pgfqpoint{4.293464in}{1.474622in}}%
\pgfpathlineto{\pgfqpoint{4.301546in}{1.483618in}}%
\pgfpathclose%
\pgfusepath{fill}%
\end{pgfscope}%
\begin{pgfscope}%
\pgfpathrectangle{\pgfqpoint{1.150000in}{0.150000in}}{\pgfqpoint{5.700000in}{5.700000in}}%
\pgfusepath{clip}%
\pgfsetbuttcap%
\pgfsetroundjoin%
\definecolor{currentfill}{rgb}{0.156270,0.489624,0.557936}%
\pgfsetfillcolor{currentfill}%
\pgfsetfillopacity{0.700000}%
\pgfsetlinewidth{0.000000pt}%
\definecolor{currentstroke}{rgb}{0.000000,0.000000,0.000000}%
\pgfsetstrokecolor{currentstroke}%
\pgfsetdash{}{0pt}%
\pgfpathmoveto{\pgfqpoint{5.502824in}{2.516624in}}%
\pgfpathlineto{\pgfqpoint{5.517643in}{2.525153in}}%
\pgfpathlineto{\pgfqpoint{5.532478in}{2.533782in}}%
\pgfpathlineto{\pgfqpoint{5.547329in}{2.542513in}}%
\pgfpathlineto{\pgfqpoint{5.539535in}{2.527965in}}%
\pgfpathlineto{\pgfqpoint{5.531734in}{2.513312in}}%
\pgfpathlineto{\pgfqpoint{5.523926in}{2.498554in}}%
\pgfpathlineto{\pgfqpoint{5.516112in}{2.483694in}}%
\pgfpathlineto{\pgfqpoint{5.501265in}{2.475188in}}%
\pgfpathlineto{\pgfqpoint{5.486435in}{2.466783in}}%
\pgfpathlineto{\pgfqpoint{5.471621in}{2.458479in}}%
\pgfpathlineto{\pgfqpoint{5.479432in}{2.473164in}}%
\pgfpathlineto{\pgfqpoint{5.487236in}{2.487752in}}%
\pgfpathlineto{\pgfqpoint{5.495033in}{2.502239in}}%
\pgfpathlineto{\pgfqpoint{5.502824in}{2.516624in}}%
\pgfpathclose%
\pgfusepath{fill}%
\end{pgfscope}%
\begin{pgfscope}%
\pgfpathrectangle{\pgfqpoint{1.150000in}{0.150000in}}{\pgfqpoint{5.700000in}{5.700000in}}%
\pgfusepath{clip}%
\pgfsetbuttcap%
\pgfsetroundjoin%
\definecolor{currentfill}{rgb}{0.120565,0.596422,0.543611}%
\pgfsetfillcolor{currentfill}%
\pgfsetfillopacity{0.700000}%
\pgfsetlinewidth{0.000000pt}%
\definecolor{currentstroke}{rgb}{0.000000,0.000000,0.000000}%
\pgfsetstrokecolor{currentstroke}%
\pgfsetdash{}{0pt}%
\pgfpathmoveto{\pgfqpoint{2.352739in}{2.819882in}}%
\pgfpathlineto{\pgfqpoint{2.367002in}{2.799628in}}%
\pgfpathlineto{\pgfqpoint{2.381257in}{2.779545in}}%
\pgfpathlineto{\pgfqpoint{2.395505in}{2.759633in}}%
\pgfpathlineto{\pgfqpoint{2.409746in}{2.739889in}}%
\pgfpathlineto{\pgfqpoint{2.400199in}{2.755520in}}%
\pgfpathlineto{\pgfqpoint{2.390622in}{2.771633in}}%
\pgfpathlineto{\pgfqpoint{2.381014in}{2.788234in}}%
\pgfpathlineto{\pgfqpoint{2.371375in}{2.805333in}}%
\pgfpathlineto{\pgfqpoint{2.357063in}{2.825768in}}%
\pgfpathlineto{\pgfqpoint{2.342744in}{2.846374in}}%
\pgfpathlineto{\pgfqpoint{2.328417in}{2.867150in}}%
\pgfpathlineto{\pgfqpoint{2.314082in}{2.888100in}}%
\pgfpathlineto{\pgfqpoint{2.323794in}{2.870298in}}%
\pgfpathlineto{\pgfqpoint{2.333474in}{2.852999in}}%
\pgfpathlineto{\pgfqpoint{2.343122in}{2.836197in}}%
\pgfpathlineto{\pgfqpoint{2.352739in}{2.819882in}}%
\pgfpathclose%
\pgfusepath{fill}%
\end{pgfscope}%
\begin{pgfscope}%
\pgfpathrectangle{\pgfqpoint{1.150000in}{0.150000in}}{\pgfqpoint{5.700000in}{5.700000in}}%
\pgfusepath{clip}%
\pgfsetbuttcap%
\pgfsetroundjoin%
\definecolor{currentfill}{rgb}{0.273809,0.031497,0.358853}%
\pgfsetfillcolor{currentfill}%
\pgfsetfillopacity{0.700000}%
\pgfsetlinewidth{0.000000pt}%
\definecolor{currentstroke}{rgb}{0.000000,0.000000,0.000000}%
\pgfsetstrokecolor{currentstroke}%
\pgfsetdash{}{0pt}%
\pgfpathmoveto{\pgfqpoint{4.066576in}{1.459280in}}%
\pgfpathlineto{\pgfqpoint{4.080717in}{1.455196in}}%
\pgfpathlineto{\pgfqpoint{4.094865in}{1.451211in}}%
\pgfpathlineto{\pgfqpoint{4.109020in}{1.447323in}}%
\pgfpathlineto{\pgfqpoint{4.123182in}{1.443532in}}%
\pgfpathlineto{\pgfqpoint{4.115026in}{1.437115in}}%
\pgfpathlineto{\pgfqpoint{4.106864in}{1.430918in}}%
\pgfpathlineto{\pgfqpoint{4.098694in}{1.424948in}}%
\pgfpathlineto{\pgfqpoint{4.090518in}{1.419210in}}%
\pgfpathlineto{\pgfqpoint{4.076340in}{1.423544in}}%
\pgfpathlineto{\pgfqpoint{4.062168in}{1.427975in}}%
\pgfpathlineto{\pgfqpoint{4.048003in}{1.432504in}}%
\pgfpathlineto{\pgfqpoint{4.033844in}{1.437131in}}%
\pgfpathlineto{\pgfqpoint{4.042038in}{1.442318in}}%
\pgfpathlineto{\pgfqpoint{4.050225in}{1.447742in}}%
\pgfpathlineto{\pgfqpoint{4.058404in}{1.453398in}}%
\pgfpathlineto{\pgfqpoint{4.066576in}{1.459280in}}%
\pgfpathclose%
\pgfusepath{fill}%
\end{pgfscope}%
\begin{pgfscope}%
\pgfpathrectangle{\pgfqpoint{1.150000in}{0.150000in}}{\pgfqpoint{5.700000in}{5.700000in}}%
\pgfusepath{clip}%
\pgfsetbuttcap%
\pgfsetroundjoin%
\definecolor{currentfill}{rgb}{0.276022,0.044167,0.370164}%
\pgfsetfillcolor{currentfill}%
\pgfsetfillopacity{0.700000}%
\pgfsetlinewidth{0.000000pt}%
\definecolor{currentstroke}{rgb}{0.000000,0.000000,0.000000}%
\pgfsetstrokecolor{currentstroke}%
\pgfsetdash{}{0pt}%
\pgfpathmoveto{\pgfqpoint{3.920795in}{1.477687in}}%
\pgfpathlineto{\pgfqpoint{3.934905in}{1.472271in}}%
\pgfpathlineto{\pgfqpoint{3.949022in}{1.466954in}}%
\pgfpathlineto{\pgfqpoint{3.963144in}{1.461737in}}%
\pgfpathlineto{\pgfqpoint{3.977272in}{1.456619in}}%
\pgfpathlineto{\pgfqpoint{3.969051in}{1.452229in}}%
\pgfpathlineto{\pgfqpoint{3.960822in}{1.448093in}}%
\pgfpathlineto{\pgfqpoint{3.952584in}{1.444215in}}%
\pgfpathlineto{\pgfqpoint{3.944338in}{1.440602in}}%
\pgfpathlineto{\pgfqpoint{3.930189in}{1.446283in}}%
\pgfpathlineto{\pgfqpoint{3.916046in}{1.452063in}}%
\pgfpathlineto{\pgfqpoint{3.901908in}{1.457942in}}%
\pgfpathlineto{\pgfqpoint{3.887775in}{1.463921in}}%
\pgfpathlineto{\pgfqpoint{3.896043in}{1.466964in}}%
\pgfpathlineto{\pgfqpoint{3.904302in}{1.470276in}}%
\pgfpathlineto{\pgfqpoint{3.912553in}{1.473853in}}%
\pgfpathlineto{\pgfqpoint{3.920795in}{1.477687in}}%
\pgfpathclose%
\pgfusepath{fill}%
\end{pgfscope}%
\begin{pgfscope}%
\pgfpathrectangle{\pgfqpoint{1.150000in}{0.150000in}}{\pgfqpoint{5.700000in}{5.700000in}}%
\pgfusepath{clip}%
\pgfsetbuttcap%
\pgfsetroundjoin%
\definecolor{currentfill}{rgb}{0.275191,0.194905,0.496005}%
\pgfsetfillcolor{currentfill}%
\pgfsetfillopacity{0.700000}%
\pgfsetlinewidth{0.000000pt}%
\definecolor{currentstroke}{rgb}{0.000000,0.000000,0.000000}%
\pgfsetstrokecolor{currentstroke}%
\pgfsetdash{}{0pt}%
\pgfpathmoveto{\pgfqpoint{4.836984in}{1.783893in}}%
\pgfpathlineto{\pgfqpoint{4.851415in}{1.786990in}}%
\pgfpathlineto{\pgfqpoint{4.865859in}{1.790184in}}%
\pgfpathlineto{\pgfqpoint{4.880315in}{1.793474in}}%
\pgfpathlineto{\pgfqpoint{4.894783in}{1.796860in}}%
\pgfpathlineto{\pgfqpoint{4.886821in}{1.782374in}}%
\pgfpathlineto{\pgfqpoint{4.878855in}{1.767921in}}%
\pgfpathlineto{\pgfqpoint{4.870886in}{1.753507in}}%
\pgfpathlineto{\pgfqpoint{4.862913in}{1.739136in}}%
\pgfpathlineto{\pgfqpoint{4.848447in}{1.736155in}}%
\pgfpathlineto{\pgfqpoint{4.833993in}{1.733270in}}%
\pgfpathlineto{\pgfqpoint{4.819552in}{1.730481in}}%
\pgfpathlineto{\pgfqpoint{4.805122in}{1.727788in}}%
\pgfpathlineto{\pgfqpoint{4.813093in}{1.741748in}}%
\pgfpathlineto{\pgfqpoint{4.821060in}{1.755755in}}%
\pgfpathlineto{\pgfqpoint{4.829024in}{1.769804in}}%
\pgfpathlineto{\pgfqpoint{4.836984in}{1.783893in}}%
\pgfpathclose%
\pgfusepath{fill}%
\end{pgfscope}%
\begin{pgfscope}%
\pgfpathrectangle{\pgfqpoint{1.150000in}{0.150000in}}{\pgfqpoint{5.700000in}{5.700000in}}%
\pgfusepath{clip}%
\pgfsetbuttcap%
\pgfsetroundjoin%
\definecolor{currentfill}{rgb}{0.244972,0.287675,0.537260}%
\pgfsetfillcolor{currentfill}%
\pgfsetfillopacity{0.700000}%
\pgfsetlinewidth{0.000000pt}%
\definecolor{currentstroke}{rgb}{0.000000,0.000000,0.000000}%
\pgfsetstrokecolor{currentstroke}%
\pgfsetdash{}{0pt}%
\pgfpathmoveto{\pgfqpoint{5.048074in}{1.991042in}}%
\pgfpathlineto{\pgfqpoint{5.062618in}{1.996023in}}%
\pgfpathlineto{\pgfqpoint{5.077175in}{2.001103in}}%
\pgfpathlineto{\pgfqpoint{5.091746in}{2.006280in}}%
\pgfpathlineto{\pgfqpoint{5.106331in}{2.011555in}}%
\pgfpathlineto{\pgfqpoint{5.098407in}{1.996240in}}%
\pgfpathlineto{\pgfqpoint{5.090479in}{1.980909in}}%
\pgfpathlineto{\pgfqpoint{5.082547in}{1.965564in}}%
\pgfpathlineto{\pgfqpoint{5.074611in}{1.950210in}}%
\pgfpathlineto{\pgfqpoint{5.060031in}{1.945288in}}%
\pgfpathlineto{\pgfqpoint{5.045464in}{1.940464in}}%
\pgfpathlineto{\pgfqpoint{5.030911in}{1.935737in}}%
\pgfpathlineto{\pgfqpoint{5.016372in}{1.931108in}}%
\pgfpathlineto{\pgfqpoint{5.024303in}{1.946102in}}%
\pgfpathlineto{\pgfqpoint{5.032231in}{1.961092in}}%
\pgfpathlineto{\pgfqpoint{5.040155in}{1.976072in}}%
\pgfpathlineto{\pgfqpoint{5.048074in}{1.991042in}}%
\pgfpathclose%
\pgfusepath{fill}%
\end{pgfscope}%
\begin{pgfscope}%
\pgfpathrectangle{\pgfqpoint{1.150000in}{0.150000in}}{\pgfqpoint{5.700000in}{5.700000in}}%
\pgfusepath{clip}%
\pgfsetbuttcap%
\pgfsetroundjoin%
\definecolor{currentfill}{rgb}{0.255645,0.260703,0.528312}%
\pgfsetfillcolor{currentfill}%
\pgfsetfillopacity{0.700000}%
\pgfsetlinewidth{0.000000pt}%
\definecolor{currentstroke}{rgb}{0.000000,0.000000,0.000000}%
\pgfsetstrokecolor{currentstroke}%
\pgfsetdash{}{0pt}%
\pgfpathmoveto{\pgfqpoint{3.144164in}{1.917116in}}%
\pgfpathlineto{\pgfqpoint{3.158233in}{1.904776in}}%
\pgfpathlineto{\pgfqpoint{3.172303in}{1.892554in}}%
\pgfpathlineto{\pgfqpoint{3.186372in}{1.880450in}}%
\pgfpathlineto{\pgfqpoint{3.200441in}{1.868464in}}%
\pgfpathlineto{\pgfqpoint{3.191685in}{1.874963in}}%
\pgfpathlineto{\pgfqpoint{3.182911in}{1.881854in}}%
\pgfpathlineto{\pgfqpoint{3.174117in}{1.889144in}}%
\pgfpathlineto{\pgfqpoint{3.165305in}{1.896841in}}%
\pgfpathlineto{\pgfqpoint{3.151189in}{1.909468in}}%
\pgfpathlineto{\pgfqpoint{3.137074in}{1.922213in}}%
\pgfpathlineto{\pgfqpoint{3.122957in}{1.935076in}}%
\pgfpathlineto{\pgfqpoint{3.108841in}{1.948058in}}%
\pgfpathlineto{\pgfqpoint{3.117701in}{1.939711in}}%
\pgfpathlineto{\pgfqpoint{3.126541in}{1.931777in}}%
\pgfpathlineto{\pgfqpoint{3.135362in}{1.924248in}}%
\pgfpathlineto{\pgfqpoint{3.144164in}{1.917116in}}%
\pgfpathclose%
\pgfusepath{fill}%
\end{pgfscope}%
\begin{pgfscope}%
\pgfpathrectangle{\pgfqpoint{1.150000in}{0.150000in}}{\pgfqpoint{5.700000in}{5.700000in}}%
\pgfusepath{clip}%
\pgfsetbuttcap%
\pgfsetroundjoin%
\definecolor{currentfill}{rgb}{0.282884,0.135920,0.453427}%
\pgfsetfillcolor{currentfill}%
\pgfsetfillopacity{0.700000}%
\pgfsetlinewidth{0.000000pt}%
\definecolor{currentstroke}{rgb}{0.000000,0.000000,0.000000}%
\pgfsetstrokecolor{currentstroke}%
\pgfsetdash{}{0pt}%
\pgfpathmoveto{\pgfqpoint{3.516115in}{1.642578in}}%
\pgfpathlineto{\pgfqpoint{3.530179in}{1.633545in}}%
\pgfpathlineto{\pgfqpoint{3.544245in}{1.624620in}}%
\pgfpathlineto{\pgfqpoint{3.558314in}{1.615800in}}%
\pgfpathlineto{\pgfqpoint{3.572387in}{1.607087in}}%
\pgfpathlineto{\pgfqpoint{3.563925in}{1.608466in}}%
\pgfpathlineto{\pgfqpoint{3.555449in}{1.610177in}}%
\pgfpathlineto{\pgfqpoint{3.546961in}{1.612227in}}%
\pgfpathlineto{\pgfqpoint{3.538459in}{1.614622in}}%
\pgfpathlineto{\pgfqpoint{3.524353in}{1.623942in}}%
\pgfpathlineto{\pgfqpoint{3.510249in}{1.633370in}}%
\pgfpathlineto{\pgfqpoint{3.496148in}{1.642903in}}%
\pgfpathlineto{\pgfqpoint{3.482049in}{1.652544in}}%
\pgfpathlineto{\pgfqpoint{3.490587in}{1.649533in}}%
\pgfpathlineto{\pgfqpoint{3.499110in}{1.646873in}}%
\pgfpathlineto{\pgfqpoint{3.507619in}{1.644557in}}%
\pgfpathlineto{\pgfqpoint{3.516115in}{1.642578in}}%
\pgfpathclose%
\pgfusepath{fill}%
\end{pgfscope}%
\begin{pgfscope}%
\pgfpathrectangle{\pgfqpoint{1.150000in}{0.150000in}}{\pgfqpoint{5.700000in}{5.700000in}}%
\pgfusepath{clip}%
\pgfsetbuttcap%
\pgfsetroundjoin%
\definecolor{currentfill}{rgb}{0.121380,0.629492,0.531973}%
\pgfsetfillcolor{currentfill}%
\pgfsetfillopacity{0.700000}%
\pgfsetlinewidth{0.000000pt}%
\definecolor{currentstroke}{rgb}{0.000000,0.000000,0.000000}%
\pgfsetstrokecolor{currentstroke}%
\pgfsetdash{}{0pt}%
\pgfpathmoveto{\pgfqpoint{2.295609in}{2.902643in}}%
\pgfpathlineto{\pgfqpoint{2.309904in}{2.881688in}}%
\pgfpathlineto{\pgfqpoint{2.324190in}{2.860911in}}%
\pgfpathlineto{\pgfqpoint{2.338468in}{2.840309in}}%
\pgfpathlineto{\pgfqpoint{2.352739in}{2.819882in}}%
\pgfpathlineto{\pgfqpoint{2.343122in}{2.836197in}}%
\pgfpathlineto{\pgfqpoint{2.333474in}{2.852999in}}%
\pgfpathlineto{\pgfqpoint{2.323794in}{2.870298in}}%
\pgfpathlineto{\pgfqpoint{2.314082in}{2.888100in}}%
\pgfpathlineto{\pgfqpoint{2.299739in}{2.909224in}}%
\pgfpathlineto{\pgfqpoint{2.285387in}{2.930524in}}%
\pgfpathlineto{\pgfqpoint{2.271027in}{2.952002in}}%
\pgfpathlineto{\pgfqpoint{2.256659in}{2.973658in}}%
\pgfpathlineto{\pgfqpoint{2.266446in}{2.955146in}}%
\pgfpathlineto{\pgfqpoint{2.276200in}{2.937144in}}%
\pgfpathlineto{\pgfqpoint{2.285921in}{2.919646in}}%
\pgfpathlineto{\pgfqpoint{2.295609in}{2.902643in}}%
\pgfpathclose%
\pgfusepath{fill}%
\end{pgfscope}%
\begin{pgfscope}%
\pgfpathrectangle{\pgfqpoint{1.150000in}{0.150000in}}{\pgfqpoint{5.700000in}{5.700000in}}%
\pgfusepath{clip}%
\pgfsetbuttcap%
\pgfsetroundjoin%
\definecolor{currentfill}{rgb}{0.177423,0.437527,0.557565}%
\pgfsetfillcolor{currentfill}%
\pgfsetfillopacity{0.700000}%
\pgfsetlinewidth{0.000000pt}%
\definecolor{currentstroke}{rgb}{0.000000,0.000000,0.000000}%
\pgfsetstrokecolor{currentstroke}%
\pgfsetdash{}{0pt}%
\pgfpathmoveto{\pgfqpoint{5.381243in}{2.367556in}}%
\pgfpathlineto{\pgfqpoint{5.395988in}{2.375215in}}%
\pgfpathlineto{\pgfqpoint{5.410748in}{2.382975in}}%
\pgfpathlineto{\pgfqpoint{5.425524in}{2.390834in}}%
\pgfpathlineto{\pgfqpoint{5.440317in}{2.398794in}}%
\pgfpathlineto{\pgfqpoint{5.432475in}{2.383649in}}%
\pgfpathlineto{\pgfqpoint{5.424628in}{2.368417in}}%
\pgfpathlineto{\pgfqpoint{5.416775in}{2.353103in}}%
\pgfpathlineto{\pgfqpoint{5.408916in}{2.337707in}}%
\pgfpathlineto{\pgfqpoint{5.394130in}{2.330011in}}%
\pgfpathlineto{\pgfqpoint{5.379359in}{2.322414in}}%
\pgfpathlineto{\pgfqpoint{5.364604in}{2.314917in}}%
\pgfpathlineto{\pgfqpoint{5.349865in}{2.307520in}}%
\pgfpathlineto{\pgfqpoint{5.357718in}{2.322645in}}%
\pgfpathlineto{\pgfqpoint{5.365565in}{2.337694in}}%
\pgfpathlineto{\pgfqpoint{5.373407in}{2.352665in}}%
\pgfpathlineto{\pgfqpoint{5.381243in}{2.367556in}}%
\pgfpathclose%
\pgfusepath{fill}%
\end{pgfscope}%
\begin{pgfscope}%
\pgfpathrectangle{\pgfqpoint{1.150000in}{0.150000in}}{\pgfqpoint{5.700000in}{5.700000in}}%
\pgfusepath{clip}%
\pgfsetbuttcap%
\pgfsetroundjoin%
\definecolor{currentfill}{rgb}{0.274952,0.037752,0.364543}%
\pgfsetfillcolor{currentfill}%
\pgfsetfillopacity{0.700000}%
\pgfsetlinewidth{0.000000pt}%
\definecolor{currentstroke}{rgb}{0.000000,0.000000,0.000000}%
\pgfsetstrokecolor{currentstroke}%
\pgfsetdash{}{0pt}%
\pgfpathmoveto{\pgfqpoint{4.212400in}{1.459217in}}%
\pgfpathlineto{\pgfqpoint{4.226585in}{1.456437in}}%
\pgfpathlineto{\pgfqpoint{4.240777in}{1.453753in}}%
\pgfpathlineto{\pgfqpoint{4.254977in}{1.451166in}}%
\pgfpathlineto{\pgfqpoint{4.269184in}{1.448675in}}%
\pgfpathlineto{\pgfqpoint{4.261080in}{1.440390in}}%
\pgfpathlineto{\pgfqpoint{4.252971in}{1.432296in}}%
\pgfpathlineto{\pgfqpoint{4.244855in}{1.424397in}}%
\pgfpathlineto{\pgfqpoint{4.236734in}{1.416700in}}%
\pgfpathlineto{\pgfqpoint{4.222514in}{1.419716in}}%
\pgfpathlineto{\pgfqpoint{4.208302in}{1.422828in}}%
\pgfpathlineto{\pgfqpoint{4.194097in}{1.426037in}}%
\pgfpathlineto{\pgfqpoint{4.179899in}{1.429343in}}%
\pgfpathlineto{\pgfqpoint{4.188034in}{1.436508in}}%
\pgfpathlineto{\pgfqpoint{4.196162in}{1.443879in}}%
\pgfpathlineto{\pgfqpoint{4.204284in}{1.451450in}}%
\pgfpathlineto{\pgfqpoint{4.212400in}{1.459217in}}%
\pgfpathclose%
\pgfusepath{fill}%
\end{pgfscope}%
\begin{pgfscope}%
\pgfpathrectangle{\pgfqpoint{1.150000in}{0.150000in}}{\pgfqpoint{5.700000in}{5.700000in}}%
\pgfusepath{clip}%
\pgfsetbuttcap%
\pgfsetroundjoin%
\definecolor{currentfill}{rgb}{0.262138,0.242286,0.520837}%
\pgfsetfillcolor{currentfill}%
\pgfsetfillopacity{0.700000}%
\pgfsetlinewidth{0.000000pt}%
\definecolor{currentstroke}{rgb}{0.000000,0.000000,0.000000}%
\pgfsetstrokecolor{currentstroke}%
\pgfsetdash{}{0pt}%
\pgfpathmoveto{\pgfqpoint{3.200441in}{1.868464in}}%
\pgfpathlineto{\pgfqpoint{3.214511in}{1.856593in}}%
\pgfpathlineto{\pgfqpoint{3.228582in}{1.844839in}}%
\pgfpathlineto{\pgfqpoint{3.242653in}{1.833201in}}%
\pgfpathlineto{\pgfqpoint{3.256725in}{1.821677in}}%
\pgfpathlineto{\pgfqpoint{3.248012in}{1.827546in}}%
\pgfpathlineto{\pgfqpoint{3.239283in}{1.833802in}}%
\pgfpathlineto{\pgfqpoint{3.230535in}{1.840450in}}%
\pgfpathlineto{\pgfqpoint{3.221769in}{1.847499in}}%
\pgfpathlineto{\pgfqpoint{3.207653in}{1.859660in}}%
\pgfpathlineto{\pgfqpoint{3.193537in}{1.871938in}}%
\pgfpathlineto{\pgfqpoint{3.179421in}{1.884331in}}%
\pgfpathlineto{\pgfqpoint{3.165305in}{1.896841in}}%
\pgfpathlineto{\pgfqpoint{3.174117in}{1.889144in}}%
\pgfpathlineto{\pgfqpoint{3.182911in}{1.881854in}}%
\pgfpathlineto{\pgfqpoint{3.191685in}{1.874963in}}%
\pgfpathlineto{\pgfqpoint{3.200441in}{1.868464in}}%
\pgfpathclose%
\pgfusepath{fill}%
\end{pgfscope}%
\begin{pgfscope}%
\pgfpathrectangle{\pgfqpoint{1.150000in}{0.150000in}}{\pgfqpoint{5.700000in}{5.700000in}}%
\pgfusepath{clip}%
\pgfsetbuttcap%
\pgfsetroundjoin%
\definecolor{currentfill}{rgb}{0.280267,0.073417,0.397163}%
\pgfsetfillcolor{currentfill}%
\pgfsetfillopacity{0.700000}%
\pgfsetlinewidth{0.000000pt}%
\definecolor{currentstroke}{rgb}{0.000000,0.000000,0.000000}%
\pgfsetstrokecolor{currentstroke}%
\pgfsetdash{}{0pt}%
\pgfpathmoveto{\pgfqpoint{3.774896in}{1.515358in}}%
\pgfpathlineto{\pgfqpoint{3.788989in}{1.508575in}}%
\pgfpathlineto{\pgfqpoint{3.803086in}{1.501894in}}%
\pgfpathlineto{\pgfqpoint{3.817189in}{1.495314in}}%
\pgfpathlineto{\pgfqpoint{3.831296in}{1.488834in}}%
\pgfpathlineto{\pgfqpoint{3.822994in}{1.486642in}}%
\pgfpathlineto{\pgfqpoint{3.814683in}{1.484736in}}%
\pgfpathlineto{\pgfqpoint{3.806361in}{1.483123in}}%
\pgfpathlineto{\pgfqpoint{3.798030in}{1.481808in}}%
\pgfpathlineto{\pgfqpoint{3.783897in}{1.488870in}}%
\pgfpathlineto{\pgfqpoint{3.769768in}{1.496034in}}%
\pgfpathlineto{\pgfqpoint{3.755644in}{1.503299in}}%
\pgfpathlineto{\pgfqpoint{3.741524in}{1.510665in}}%
\pgfpathlineto{\pgfqpoint{3.749883in}{1.511389in}}%
\pgfpathlineto{\pgfqpoint{3.758231in}{1.512416in}}%
\pgfpathlineto{\pgfqpoint{3.766568in}{1.513742in}}%
\pgfpathlineto{\pgfqpoint{3.774896in}{1.515358in}}%
\pgfpathclose%
\pgfusepath{fill}%
\end{pgfscope}%
\begin{pgfscope}%
\pgfpathrectangle{\pgfqpoint{1.150000in}{0.150000in}}{\pgfqpoint{5.700000in}{5.700000in}}%
\pgfusepath{clip}%
\pgfsetbuttcap%
\pgfsetroundjoin%
\definecolor{currentfill}{rgb}{0.201239,0.383670,0.554294}%
\pgfsetfillcolor{currentfill}%
\pgfsetfillopacity{0.700000}%
\pgfsetlinewidth{0.000000pt}%
\definecolor{currentstroke}{rgb}{0.000000,0.000000,0.000000}%
\pgfsetstrokecolor{currentstroke}%
\pgfsetdash{}{0pt}%
\pgfpathmoveto{\pgfqpoint{5.259617in}{2.218837in}}%
\pgfpathlineto{\pgfqpoint{5.274289in}{2.225556in}}%
\pgfpathlineto{\pgfqpoint{5.288977in}{2.232374in}}%
\pgfpathlineto{\pgfqpoint{5.303680in}{2.239291in}}%
\pgfpathlineto{\pgfqpoint{5.318398in}{2.246307in}}%
\pgfpathlineto{\pgfqpoint{5.310518in}{2.230839in}}%
\pgfpathlineto{\pgfqpoint{5.302633in}{2.215310in}}%
\pgfpathlineto{\pgfqpoint{5.294743in}{2.199722in}}%
\pgfpathlineto{\pgfqpoint{5.286848in}{2.184080in}}%
\pgfpathlineto{\pgfqpoint{5.272136in}{2.177363in}}%
\pgfpathlineto{\pgfqpoint{5.257439in}{2.170746in}}%
\pgfpathlineto{\pgfqpoint{5.242757in}{2.164228in}}%
\pgfpathlineto{\pgfqpoint{5.228090in}{2.157808in}}%
\pgfpathlineto{\pgfqpoint{5.235979in}{2.173144in}}%
\pgfpathlineto{\pgfqpoint{5.243863in}{2.188430in}}%
\pgfpathlineto{\pgfqpoint{5.251742in}{2.203662in}}%
\pgfpathlineto{\pgfqpoint{5.259617in}{2.218837in}}%
\pgfpathclose%
\pgfusepath{fill}%
\end{pgfscope}%
\begin{pgfscope}%
\pgfpathrectangle{\pgfqpoint{1.150000in}{0.150000in}}{\pgfqpoint{5.700000in}{5.700000in}}%
\pgfusepath{clip}%
\pgfsetbuttcap%
\pgfsetroundjoin%
\definecolor{currentfill}{rgb}{0.265145,0.232956,0.516599}%
\pgfsetfillcolor{currentfill}%
\pgfsetfillopacity{0.700000}%
\pgfsetlinewidth{0.000000pt}%
\definecolor{currentstroke}{rgb}{0.000000,0.000000,0.000000}%
\pgfsetstrokecolor{currentstroke}%
\pgfsetdash{}{0pt}%
\pgfpathmoveto{\pgfqpoint{4.926595in}{1.855080in}}%
\pgfpathlineto{\pgfqpoint{4.941078in}{1.858951in}}%
\pgfpathlineto{\pgfqpoint{4.955575in}{1.862919in}}%
\pgfpathlineto{\pgfqpoint{4.970084in}{1.866983in}}%
\pgfpathlineto{\pgfqpoint{4.984606in}{1.871145in}}%
\pgfpathlineto{\pgfqpoint{4.976655in}{1.856175in}}%
\pgfpathlineto{\pgfqpoint{4.968701in}{1.841221in}}%
\pgfpathlineto{\pgfqpoint{4.960743in}{1.826286in}}%
\pgfpathlineto{\pgfqpoint{4.952781in}{1.811373in}}%
\pgfpathlineto{\pgfqpoint{4.938262in}{1.807600in}}%
\pgfpathlineto{\pgfqpoint{4.923757in}{1.803923in}}%
\pgfpathlineto{\pgfqpoint{4.909263in}{1.800344in}}%
\pgfpathlineto{\pgfqpoint{4.894783in}{1.796860in}}%
\pgfpathlineto{\pgfqpoint{4.902741in}{1.811378in}}%
\pgfpathlineto{\pgfqpoint{4.910696in}{1.825923in}}%
\pgfpathlineto{\pgfqpoint{4.918647in}{1.840492in}}%
\pgfpathlineto{\pgfqpoint{4.926595in}{1.855080in}}%
\pgfpathclose%
\pgfusepath{fill}%
\end{pgfscope}%
\begin{pgfscope}%
\pgfpathrectangle{\pgfqpoint{1.150000in}{0.150000in}}{\pgfqpoint{5.700000in}{5.700000in}}%
\pgfusepath{clip}%
\pgfsetbuttcap%
\pgfsetroundjoin%
\definecolor{currentfill}{rgb}{0.137339,0.662252,0.515571}%
\pgfsetfillcolor{currentfill}%
\pgfsetfillopacity{0.700000}%
\pgfsetlinewidth{0.000000pt}%
\definecolor{currentstroke}{rgb}{0.000000,0.000000,0.000000}%
\pgfsetstrokecolor{currentstroke}%
\pgfsetdash{}{0pt}%
\pgfpathmoveto{\pgfqpoint{2.238345in}{2.988271in}}%
\pgfpathlineto{\pgfqpoint{2.252674in}{2.966590in}}%
\pgfpathlineto{\pgfqpoint{2.266994in}{2.945092in}}%
\pgfpathlineto{\pgfqpoint{2.281306in}{2.923778in}}%
\pgfpathlineto{\pgfqpoint{2.295609in}{2.902643in}}%
\pgfpathlineto{\pgfqpoint{2.285921in}{2.919646in}}%
\pgfpathlineto{\pgfqpoint{2.276200in}{2.937144in}}%
\pgfpathlineto{\pgfqpoint{2.266446in}{2.955146in}}%
\pgfpathlineto{\pgfqpoint{2.256659in}{2.973658in}}%
\pgfpathlineto{\pgfqpoint{2.242281in}{2.995495in}}%
\pgfpathlineto{\pgfqpoint{2.227894in}{3.017515in}}%
\pgfpathlineto{\pgfqpoint{2.213498in}{3.039718in}}%
\pgfpathlineto{\pgfqpoint{2.199093in}{3.062106in}}%
\pgfpathlineto{\pgfqpoint{2.208957in}{3.042877in}}%
\pgfpathlineto{\pgfqpoint{2.218786in}{3.024167in}}%
\pgfpathlineto{\pgfqpoint{2.228582in}{3.005968in}}%
\pgfpathlineto{\pgfqpoint{2.238345in}{2.988271in}}%
\pgfpathclose%
\pgfusepath{fill}%
\end{pgfscope}%
\begin{pgfscope}%
\pgfpathrectangle{\pgfqpoint{1.150000in}{0.150000in}}{\pgfqpoint{5.700000in}{5.700000in}}%
\pgfusepath{clip}%
\pgfsetbuttcap%
\pgfsetroundjoin%
\definecolor{currentfill}{rgb}{0.282327,0.094955,0.417331}%
\pgfsetfillcolor{currentfill}%
\pgfsetfillopacity{0.700000}%
\pgfsetlinewidth{0.000000pt}%
\definecolor{currentstroke}{rgb}{0.000000,0.000000,0.000000}%
\pgfsetstrokecolor{currentstroke}%
\pgfsetdash{}{0pt}%
\pgfpathmoveto{\pgfqpoint{4.536891in}{1.556356in}}%
\pgfpathlineto{\pgfqpoint{4.551193in}{1.556581in}}%
\pgfpathlineto{\pgfqpoint{4.565505in}{1.556903in}}%
\pgfpathlineto{\pgfqpoint{4.579827in}{1.557320in}}%
\pgfpathlineto{\pgfqpoint{4.594159in}{1.557833in}}%
\pgfpathlineto{\pgfqpoint{4.586139in}{1.545787in}}%
\pgfpathlineto{\pgfqpoint{4.578115in}{1.533855in}}%
\pgfpathlineto{\pgfqpoint{4.570087in}{1.522041in}}%
\pgfpathlineto{\pgfqpoint{4.562055in}{1.510352in}}%
\pgfpathlineto{\pgfqpoint{4.547719in}{1.510313in}}%
\pgfpathlineto{\pgfqpoint{4.533393in}{1.510369in}}%
\pgfpathlineto{\pgfqpoint{4.519077in}{1.510521in}}%
\pgfpathlineto{\pgfqpoint{4.504771in}{1.510768in}}%
\pgfpathlineto{\pgfqpoint{4.512807in}{1.521977in}}%
\pgfpathlineto{\pgfqpoint{4.520839in}{1.533315in}}%
\pgfpathlineto{\pgfqpoint{4.528867in}{1.544776in}}%
\pgfpathlineto{\pgfqpoint{4.536891in}{1.556356in}}%
\pgfpathclose%
\pgfusepath{fill}%
\end{pgfscope}%
\begin{pgfscope}%
\pgfpathrectangle{\pgfqpoint{1.150000in}{0.150000in}}{\pgfqpoint{5.700000in}{5.700000in}}%
\pgfusepath{clip}%
\pgfsetbuttcap%
\pgfsetroundjoin%
\definecolor{currentfill}{rgb}{0.283229,0.120777,0.440584}%
\pgfsetfillcolor{currentfill}%
\pgfsetfillopacity{0.700000}%
\pgfsetlinewidth{0.000000pt}%
\definecolor{currentstroke}{rgb}{0.000000,0.000000,0.000000}%
\pgfsetstrokecolor{currentstroke}%
\pgfsetdash{}{0pt}%
\pgfpathmoveto{\pgfqpoint{3.572387in}{1.607087in}}%
\pgfpathlineto{\pgfqpoint{3.586462in}{1.598479in}}%
\pgfpathlineto{\pgfqpoint{3.600541in}{1.589976in}}%
\pgfpathlineto{\pgfqpoint{3.614623in}{1.581578in}}%
\pgfpathlineto{\pgfqpoint{3.628708in}{1.573285in}}%
\pgfpathlineto{\pgfqpoint{3.620278in}{1.574066in}}%
\pgfpathlineto{\pgfqpoint{3.611836in}{1.575173in}}%
\pgfpathlineto{\pgfqpoint{3.603381in}{1.576613in}}%
\pgfpathlineto{\pgfqpoint{3.594913in}{1.578393in}}%
\pgfpathlineto{\pgfqpoint{3.580795in}{1.587293in}}%
\pgfpathlineto{\pgfqpoint{3.566680in}{1.596297in}}%
\pgfpathlineto{\pgfqpoint{3.552568in}{1.605406in}}%
\pgfpathlineto{\pgfqpoint{3.538459in}{1.614622in}}%
\pgfpathlineto{\pgfqpoint{3.546961in}{1.612227in}}%
\pgfpathlineto{\pgfqpoint{3.555449in}{1.610177in}}%
\pgfpathlineto{\pgfqpoint{3.563925in}{1.608466in}}%
\pgfpathlineto{\pgfqpoint{3.572387in}{1.607087in}}%
\pgfpathclose%
\pgfusepath{fill}%
\end{pgfscope}%
\begin{pgfscope}%
\pgfpathrectangle{\pgfqpoint{1.150000in}{0.150000in}}{\pgfqpoint{5.700000in}{5.700000in}}%
\pgfusepath{clip}%
\pgfsetbuttcap%
\pgfsetroundjoin%
\definecolor{currentfill}{rgb}{0.283229,0.120777,0.440584}%
\pgfsetfillcolor{currentfill}%
\pgfsetfillopacity{0.700000}%
\pgfsetlinewidth{0.000000pt}%
\definecolor{currentstroke}{rgb}{0.000000,0.000000,0.000000}%
\pgfsetstrokecolor{currentstroke}%
\pgfsetdash{}{0pt}%
\pgfpathmoveto{\pgfqpoint{4.626202in}{1.607071in}}%
\pgfpathlineto{\pgfqpoint{4.640542in}{1.608136in}}%
\pgfpathlineto{\pgfqpoint{4.654893in}{1.609296in}}%
\pgfpathlineto{\pgfqpoint{4.669254in}{1.610551in}}%
\pgfpathlineto{\pgfqpoint{4.683627in}{1.611903in}}%
\pgfpathlineto{\pgfqpoint{4.675623in}{1.598994in}}%
\pgfpathlineto{\pgfqpoint{4.667617in}{1.586178in}}%
\pgfpathlineto{\pgfqpoint{4.659606in}{1.573458in}}%
\pgfpathlineto{\pgfqpoint{4.651592in}{1.560839in}}%
\pgfpathlineto{\pgfqpoint{4.637218in}{1.559944in}}%
\pgfpathlineto{\pgfqpoint{4.622855in}{1.559145in}}%
\pgfpathlineto{\pgfqpoint{4.608502in}{1.558441in}}%
\pgfpathlineto{\pgfqpoint{4.594159in}{1.557833in}}%
\pgfpathlineto{\pgfqpoint{4.602176in}{1.569989in}}%
\pgfpathlineto{\pgfqpoint{4.610188in}{1.582250in}}%
\pgfpathlineto{\pgfqpoint{4.618197in}{1.594612in}}%
\pgfpathlineto{\pgfqpoint{4.626202in}{1.607071in}}%
\pgfpathclose%
\pgfusepath{fill}%
\end{pgfscope}%
\begin{pgfscope}%
\pgfpathrectangle{\pgfqpoint{1.150000in}{0.150000in}}{\pgfqpoint{5.700000in}{5.700000in}}%
\pgfusepath{clip}%
\pgfsetbuttcap%
\pgfsetroundjoin%
\definecolor{currentfill}{rgb}{0.280267,0.073417,0.397163}%
\pgfsetfillcolor{currentfill}%
\pgfsetfillopacity{0.700000}%
\pgfsetlinewidth{0.000000pt}%
\definecolor{currentstroke}{rgb}{0.000000,0.000000,0.000000}%
\pgfsetstrokecolor{currentstroke}%
\pgfsetdash{}{0pt}%
\pgfpathmoveto{\pgfqpoint{4.447641in}{1.512713in}}%
\pgfpathlineto{\pgfqpoint{4.461910in}{1.512083in}}%
\pgfpathlineto{\pgfqpoint{4.476187in}{1.511549in}}%
\pgfpathlineto{\pgfqpoint{4.490474in}{1.511110in}}%
\pgfpathlineto{\pgfqpoint{4.504771in}{1.510768in}}%
\pgfpathlineto{\pgfqpoint{4.496730in}{1.499691in}}%
\pgfpathlineto{\pgfqpoint{4.488686in}{1.488751in}}%
\pgfpathlineto{\pgfqpoint{4.480637in}{1.477953in}}%
\pgfpathlineto{\pgfqpoint{4.472584in}{1.467302in}}%
\pgfpathlineto{\pgfqpoint{4.458281in}{1.468136in}}%
\pgfpathlineto{\pgfqpoint{4.443988in}{1.469064in}}%
\pgfpathlineto{\pgfqpoint{4.429704in}{1.470089in}}%
\pgfpathlineto{\pgfqpoint{4.415429in}{1.471209in}}%
\pgfpathlineto{\pgfqpoint{4.423489in}{1.481362in}}%
\pgfpathlineto{\pgfqpoint{4.431544in}{1.491667in}}%
\pgfpathlineto{\pgfqpoint{4.439595in}{1.502119in}}%
\pgfpathlineto{\pgfqpoint{4.447641in}{1.512713in}}%
\pgfpathclose%
\pgfusepath{fill}%
\end{pgfscope}%
\begin{pgfscope}%
\pgfpathrectangle{\pgfqpoint{1.150000in}{0.150000in}}{\pgfqpoint{5.700000in}{5.700000in}}%
\pgfusepath{clip}%
\pgfsetbuttcap%
\pgfsetroundjoin%
\definecolor{currentfill}{rgb}{0.227802,0.326594,0.546532}%
\pgfsetfillcolor{currentfill}%
\pgfsetfillopacity{0.700000}%
\pgfsetlinewidth{0.000000pt}%
\definecolor{currentstroke}{rgb}{0.000000,0.000000,0.000000}%
\pgfsetstrokecolor{currentstroke}%
\pgfsetdash{}{0pt}%
\pgfpathmoveto{\pgfqpoint{5.137985in}{2.072590in}}%
\pgfpathlineto{\pgfqpoint{5.152589in}{2.078298in}}%
\pgfpathlineto{\pgfqpoint{5.167207in}{2.084105in}}%
\pgfpathlineto{\pgfqpoint{5.181840in}{2.090009in}}%
\pgfpathlineto{\pgfqpoint{5.196486in}{2.096012in}}%
\pgfpathlineto{\pgfqpoint{5.188574in}{2.080465in}}%
\pgfpathlineto{\pgfqpoint{5.180658in}{2.064884in}}%
\pgfpathlineto{\pgfqpoint{5.172737in}{2.049273in}}%
\pgfpathlineto{\pgfqpoint{5.164811in}{2.033635in}}%
\pgfpathlineto{\pgfqpoint{5.150170in}{2.027968in}}%
\pgfpathlineto{\pgfqpoint{5.135543in}{2.022399in}}%
\pgfpathlineto{\pgfqpoint{5.120930in}{2.016928in}}%
\pgfpathlineto{\pgfqpoint{5.106331in}{2.011555in}}%
\pgfpathlineto{\pgfqpoint{5.114251in}{2.026851in}}%
\pgfpathlineto{\pgfqpoint{5.122167in}{2.042124in}}%
\pgfpathlineto{\pgfqpoint{5.130078in}{2.057371in}}%
\pgfpathlineto{\pgfqpoint{5.137985in}{2.072590in}}%
\pgfpathclose%
\pgfusepath{fill}%
\end{pgfscope}%
\begin{pgfscope}%
\pgfpathrectangle{\pgfqpoint{1.150000in}{0.150000in}}{\pgfqpoint{5.700000in}{5.700000in}}%
\pgfusepath{clip}%
\pgfsetbuttcap%
\pgfsetroundjoin%
\definecolor{currentfill}{rgb}{0.267968,0.223549,0.512008}%
\pgfsetfillcolor{currentfill}%
\pgfsetfillopacity{0.700000}%
\pgfsetlinewidth{0.000000pt}%
\definecolor{currentstroke}{rgb}{0.000000,0.000000,0.000000}%
\pgfsetstrokecolor{currentstroke}%
\pgfsetdash{}{0pt}%
\pgfpathmoveto{\pgfqpoint{3.256725in}{1.821677in}}%
\pgfpathlineto{\pgfqpoint{3.270797in}{1.810268in}}%
\pgfpathlineto{\pgfqpoint{3.284871in}{1.798973in}}%
\pgfpathlineto{\pgfqpoint{3.298945in}{1.787791in}}%
\pgfpathlineto{\pgfqpoint{3.313021in}{1.776722in}}%
\pgfpathlineto{\pgfqpoint{3.304351in}{1.781963in}}%
\pgfpathlineto{\pgfqpoint{3.295664in}{1.787585in}}%
\pgfpathlineto{\pgfqpoint{3.286960in}{1.793594in}}%
\pgfpathlineto{\pgfqpoint{3.278239in}{1.799997in}}%
\pgfpathlineto{\pgfqpoint{3.264121in}{1.811702in}}%
\pgfpathlineto{\pgfqpoint{3.250003in}{1.823520in}}%
\pgfpathlineto{\pgfqpoint{3.235886in}{1.835452in}}%
\pgfpathlineto{\pgfqpoint{3.221769in}{1.847499in}}%
\pgfpathlineto{\pgfqpoint{3.230535in}{1.840450in}}%
\pgfpathlineto{\pgfqpoint{3.239283in}{1.833802in}}%
\pgfpathlineto{\pgfqpoint{3.248012in}{1.827546in}}%
\pgfpathlineto{\pgfqpoint{3.256725in}{1.821677in}}%
\pgfpathclose%
\pgfusepath{fill}%
\end{pgfscope}%
\begin{pgfscope}%
\pgfpathrectangle{\pgfqpoint{1.150000in}{0.150000in}}{\pgfqpoint{5.700000in}{5.700000in}}%
\pgfusepath{clip}%
\pgfsetbuttcap%
\pgfsetroundjoin%
\definecolor{currentfill}{rgb}{0.282290,0.145912,0.461510}%
\pgfsetfillcolor{currentfill}%
\pgfsetfillopacity{0.700000}%
\pgfsetlinewidth{0.000000pt}%
\definecolor{currentstroke}{rgb}{0.000000,0.000000,0.000000}%
\pgfsetstrokecolor{currentstroke}%
\pgfsetdash{}{0pt}%
\pgfpathmoveto{\pgfqpoint{4.715603in}{1.664373in}}%
\pgfpathlineto{\pgfqpoint{4.729986in}{1.666259in}}%
\pgfpathlineto{\pgfqpoint{4.744380in}{1.668241in}}%
\pgfpathlineto{\pgfqpoint{4.758785in}{1.670320in}}%
\pgfpathlineto{\pgfqpoint{4.773202in}{1.672494in}}%
\pgfpathlineto{\pgfqpoint{4.765214in}{1.658827in}}%
\pgfpathlineto{\pgfqpoint{4.757221in}{1.645230in}}%
\pgfpathlineto{\pgfqpoint{4.749226in}{1.631708in}}%
\pgfpathlineto{\pgfqpoint{4.741226in}{1.618266in}}%
\pgfpathlineto{\pgfqpoint{4.726810in}{1.616532in}}%
\pgfpathlineto{\pgfqpoint{4.712404in}{1.614893in}}%
\pgfpathlineto{\pgfqpoint{4.698010in}{1.613350in}}%
\pgfpathlineto{\pgfqpoint{4.683627in}{1.611903in}}%
\pgfpathlineto{\pgfqpoint{4.691626in}{1.624899in}}%
\pgfpathlineto{\pgfqpoint{4.699622in}{1.637979in}}%
\pgfpathlineto{\pgfqpoint{4.707614in}{1.651138in}}%
\pgfpathlineto{\pgfqpoint{4.715603in}{1.664373in}}%
\pgfpathclose%
\pgfusepath{fill}%
\end{pgfscope}%
\begin{pgfscope}%
\pgfpathrectangle{\pgfqpoint{1.150000in}{0.150000in}}{\pgfqpoint{5.700000in}{5.700000in}}%
\pgfusepath{clip}%
\pgfsetbuttcap%
\pgfsetroundjoin%
\definecolor{currentfill}{rgb}{0.276022,0.044167,0.370164}%
\pgfsetfillcolor{currentfill}%
\pgfsetfillopacity{0.700000}%
\pgfsetlinewidth{0.000000pt}%
\definecolor{currentstroke}{rgb}{0.000000,0.000000,0.000000}%
\pgfsetstrokecolor{currentstroke}%
\pgfsetdash{}{0pt}%
\pgfpathmoveto{\pgfqpoint{3.977272in}{1.456619in}}%
\pgfpathlineto{\pgfqpoint{3.991406in}{1.451599in}}%
\pgfpathlineto{\pgfqpoint{4.005546in}{1.446678in}}%
\pgfpathlineto{\pgfqpoint{4.019692in}{1.441855in}}%
\pgfpathlineto{\pgfqpoint{4.033844in}{1.437131in}}%
\pgfpathlineto{\pgfqpoint{4.025643in}{1.432186in}}%
\pgfpathlineto{\pgfqpoint{4.017433in}{1.427489in}}%
\pgfpathlineto{\pgfqpoint{4.009216in}{1.423046in}}%
\pgfpathlineto{\pgfqpoint{4.000991in}{1.418863in}}%
\pgfpathlineto{\pgfqpoint{3.986819in}{1.424150in}}%
\pgfpathlineto{\pgfqpoint{3.972653in}{1.429536in}}%
\pgfpathlineto{\pgfqpoint{3.958493in}{1.435019in}}%
\pgfpathlineto{\pgfqpoint{3.944338in}{1.440602in}}%
\pgfpathlineto{\pgfqpoint{3.952584in}{1.444215in}}%
\pgfpathlineto{\pgfqpoint{3.960822in}{1.448093in}}%
\pgfpathlineto{\pgfqpoint{3.969051in}{1.452229in}}%
\pgfpathlineto{\pgfqpoint{3.977272in}{1.456619in}}%
\pgfpathclose%
\pgfusepath{fill}%
\end{pgfscope}%
\begin{pgfscope}%
\pgfpathrectangle{\pgfqpoint{1.150000in}{0.150000in}}{\pgfqpoint{5.700000in}{5.700000in}}%
\pgfusepath{clip}%
\pgfsetbuttcap%
\pgfsetroundjoin%
\definecolor{currentfill}{rgb}{0.162142,0.474838,0.558140}%
\pgfsetfillcolor{currentfill}%
\pgfsetfillopacity{0.700000}%
\pgfsetlinewidth{0.000000pt}%
\definecolor{currentstroke}{rgb}{0.000000,0.000000,0.000000}%
\pgfsetstrokecolor{currentstroke}%
\pgfsetdash{}{0pt}%
\pgfpathmoveto{\pgfqpoint{5.471621in}{2.458479in}}%
\pgfpathlineto{\pgfqpoint{5.486435in}{2.466783in}}%
\pgfpathlineto{\pgfqpoint{5.501265in}{2.475188in}}%
\pgfpathlineto{\pgfqpoint{5.516112in}{2.483694in}}%
\pgfpathlineto{\pgfqpoint{5.508291in}{2.468734in}}%
\pgfpathlineto{\pgfqpoint{5.500464in}{2.453677in}}%
\pgfpathlineto{\pgfqpoint{5.492630in}{2.438524in}}%
\pgfpathlineto{\pgfqpoint{5.484791in}{2.423277in}}%
\pgfpathlineto{\pgfqpoint{5.469950in}{2.415015in}}%
\pgfpathlineto{\pgfqpoint{5.455125in}{2.406855in}}%
\pgfpathlineto{\pgfqpoint{5.440317in}{2.398794in}}%
\pgfpathlineto{\pgfqpoint{5.448152in}{2.413853in}}%
\pgfpathlineto{\pgfqpoint{5.455981in}{2.428821in}}%
\pgfpathlineto{\pgfqpoint{5.463804in}{2.443697in}}%
\pgfpathlineto{\pgfqpoint{5.471621in}{2.458479in}}%
\pgfpathclose%
\pgfusepath{fill}%
\end{pgfscope}%
\begin{pgfscope}%
\pgfpathrectangle{\pgfqpoint{1.150000in}{0.150000in}}{\pgfqpoint{5.700000in}{5.700000in}}%
\pgfusepath{clip}%
\pgfsetbuttcap%
\pgfsetroundjoin%
\definecolor{currentfill}{rgb}{0.274952,0.037752,0.364543}%
\pgfsetfillcolor{currentfill}%
\pgfsetfillopacity{0.700000}%
\pgfsetlinewidth{0.000000pt}%
\definecolor{currentstroke}{rgb}{0.000000,0.000000,0.000000}%
\pgfsetstrokecolor{currentstroke}%
\pgfsetdash{}{0pt}%
\pgfpathmoveto{\pgfqpoint{4.123182in}{1.443532in}}%
\pgfpathlineto{\pgfqpoint{4.137350in}{1.439839in}}%
\pgfpathlineto{\pgfqpoint{4.151526in}{1.436244in}}%
\pgfpathlineto{\pgfqpoint{4.165709in}{1.432745in}}%
\pgfpathlineto{\pgfqpoint{4.179899in}{1.429343in}}%
\pgfpathlineto{\pgfqpoint{4.171759in}{1.422388in}}%
\pgfpathlineto{\pgfqpoint{4.163611in}{1.415650in}}%
\pgfpathlineto{\pgfqpoint{4.155458in}{1.409134in}}%
\pgfpathlineto{\pgfqpoint{4.147298in}{1.402846in}}%
\pgfpathlineto{\pgfqpoint{4.133093in}{1.406792in}}%
\pgfpathlineto{\pgfqpoint{4.118894in}{1.410834in}}%
\pgfpathlineto{\pgfqpoint{4.104703in}{1.414974in}}%
\pgfpathlineto{\pgfqpoint{4.090518in}{1.419210in}}%
\pgfpathlineto{\pgfqpoint{4.098694in}{1.424948in}}%
\pgfpathlineto{\pgfqpoint{4.106864in}{1.430918in}}%
\pgfpathlineto{\pgfqpoint{4.115026in}{1.437115in}}%
\pgfpathlineto{\pgfqpoint{4.123182in}{1.443532in}}%
\pgfpathclose%
\pgfusepath{fill}%
\end{pgfscope}%
\begin{pgfscope}%
\pgfpathrectangle{\pgfqpoint{1.150000in}{0.150000in}}{\pgfqpoint{5.700000in}{5.700000in}}%
\pgfusepath{clip}%
\pgfsetbuttcap%
\pgfsetroundjoin%
\definecolor{currentfill}{rgb}{0.277941,0.056324,0.381191}%
\pgfsetfillcolor{currentfill}%
\pgfsetfillopacity{0.700000}%
\pgfsetlinewidth{0.000000pt}%
\definecolor{currentstroke}{rgb}{0.000000,0.000000,0.000000}%
\pgfsetstrokecolor{currentstroke}%
\pgfsetdash{}{0pt}%
\pgfpathmoveto{\pgfqpoint{4.358418in}{1.476645in}}%
\pgfpathlineto{\pgfqpoint{4.372658in}{1.475142in}}%
\pgfpathlineto{\pgfqpoint{4.386906in}{1.473735in}}%
\pgfpathlineto{\pgfqpoint{4.401163in}{1.472424in}}%
\pgfpathlineto{\pgfqpoint{4.415429in}{1.471209in}}%
\pgfpathlineto{\pgfqpoint{4.407364in}{1.461211in}}%
\pgfpathlineto{\pgfqpoint{4.399295in}{1.451375in}}%
\pgfpathlineto{\pgfqpoint{4.391221in}{1.441705in}}%
\pgfpathlineto{\pgfqpoint{4.383142in}{1.432206in}}%
\pgfpathlineto{\pgfqpoint{4.368868in}{1.433929in}}%
\pgfpathlineto{\pgfqpoint{4.354602in}{1.435748in}}%
\pgfpathlineto{\pgfqpoint{4.340345in}{1.437663in}}%
\pgfpathlineto{\pgfqpoint{4.326096in}{1.439674in}}%
\pgfpathlineto{\pgfqpoint{4.334184in}{1.448658in}}%
\pgfpathlineto{\pgfqpoint{4.342267in}{1.457818in}}%
\pgfpathlineto{\pgfqpoint{4.350345in}{1.467149in}}%
\pgfpathlineto{\pgfqpoint{4.358418in}{1.476645in}}%
\pgfpathclose%
\pgfusepath{fill}%
\end{pgfscope}%
\begin{pgfscope}%
\pgfpathrectangle{\pgfqpoint{1.150000in}{0.150000in}}{\pgfqpoint{5.700000in}{5.700000in}}%
\pgfusepath{clip}%
\pgfsetbuttcap%
\pgfsetroundjoin%
\definecolor{currentfill}{rgb}{0.170948,0.694384,0.493803}%
\pgfsetfillcolor{currentfill}%
\pgfsetfillopacity{0.700000}%
\pgfsetlinewidth{0.000000pt}%
\definecolor{currentstroke}{rgb}{0.000000,0.000000,0.000000}%
\pgfsetstrokecolor{currentstroke}%
\pgfsetdash{}{0pt}%
\pgfpathmoveto{\pgfqpoint{2.180933in}{3.076868in}}%
\pgfpathlineto{\pgfqpoint{2.195301in}{3.054434in}}%
\pgfpathlineto{\pgfqpoint{2.209658in}{3.032191in}}%
\pgfpathlineto{\pgfqpoint{2.224006in}{3.010137in}}%
\pgfpathlineto{\pgfqpoint{2.238345in}{2.988271in}}%
\pgfpathlineto{\pgfqpoint{2.228582in}{3.005968in}}%
\pgfpathlineto{\pgfqpoint{2.218786in}{3.024167in}}%
\pgfpathlineto{\pgfqpoint{2.208957in}{3.042877in}}%
\pgfpathlineto{\pgfqpoint{2.199093in}{3.062106in}}%
\pgfpathlineto{\pgfqpoint{2.184678in}{3.084682in}}%
\pgfpathlineto{\pgfqpoint{2.170252in}{3.107447in}}%
\pgfpathlineto{\pgfqpoint{2.155817in}{3.130402in}}%
\pgfpathlineto{\pgfqpoint{2.141372in}{3.153550in}}%
\pgfpathlineto{\pgfqpoint{2.151315in}{3.133598in}}%
\pgfpathlineto{\pgfqpoint{2.161222in}{3.114173in}}%
\pgfpathlineto{\pgfqpoint{2.171095in}{3.095266in}}%
\pgfpathlineto{\pgfqpoint{2.180933in}{3.076868in}}%
\pgfpathclose%
\pgfusepath{fill}%
\end{pgfscope}%
\begin{pgfscope}%
\pgfpathrectangle{\pgfqpoint{1.150000in}{0.150000in}}{\pgfqpoint{5.700000in}{5.700000in}}%
\pgfusepath{clip}%
\pgfsetbuttcap%
\pgfsetroundjoin%
\definecolor{currentfill}{rgb}{0.252194,0.269783,0.531579}%
\pgfsetfillcolor{currentfill}%
\pgfsetfillopacity{0.700000}%
\pgfsetlinewidth{0.000000pt}%
\definecolor{currentstroke}{rgb}{0.000000,0.000000,0.000000}%
\pgfsetstrokecolor{currentstroke}%
\pgfsetdash{}{0pt}%
\pgfpathmoveto{\pgfqpoint{5.016372in}{1.931108in}}%
\pgfpathlineto{\pgfqpoint{5.030911in}{1.935737in}}%
\pgfpathlineto{\pgfqpoint{5.045464in}{1.940464in}}%
\pgfpathlineto{\pgfqpoint{5.060031in}{1.945288in}}%
\pgfpathlineto{\pgfqpoint{5.074611in}{1.950210in}}%
\pgfpathlineto{\pgfqpoint{5.066671in}{1.934849in}}%
\pgfpathlineto{\pgfqpoint{5.058727in}{1.919485in}}%
\pgfpathlineto{\pgfqpoint{5.050779in}{1.904122in}}%
\pgfpathlineto{\pgfqpoint{5.042827in}{1.888762in}}%
\pgfpathlineto{\pgfqpoint{5.028252in}{1.884212in}}%
\pgfpathlineto{\pgfqpoint{5.013690in}{1.879759in}}%
\pgfpathlineto{\pgfqpoint{4.999142in}{1.875404in}}%
\pgfpathlineto{\pgfqpoint{4.984606in}{1.871145in}}%
\pgfpathlineto{\pgfqpoint{4.992553in}{1.886127in}}%
\pgfpathlineto{\pgfqpoint{5.000497in}{1.901117in}}%
\pgfpathlineto{\pgfqpoint{5.008436in}{1.916111in}}%
\pgfpathlineto{\pgfqpoint{5.016372in}{1.931108in}}%
\pgfpathclose%
\pgfusepath{fill}%
\end{pgfscope}%
\begin{pgfscope}%
\pgfpathrectangle{\pgfqpoint{1.150000in}{0.150000in}}{\pgfqpoint{5.700000in}{5.700000in}}%
\pgfusepath{clip}%
\pgfsetbuttcap%
\pgfsetroundjoin%
\definecolor{currentfill}{rgb}{0.271828,0.209303,0.504434}%
\pgfsetfillcolor{currentfill}%
\pgfsetfillopacity{0.700000}%
\pgfsetlinewidth{0.000000pt}%
\definecolor{currentstroke}{rgb}{0.000000,0.000000,0.000000}%
\pgfsetstrokecolor{currentstroke}%
\pgfsetdash{}{0pt}%
\pgfpathmoveto{\pgfqpoint{3.313021in}{1.776722in}}%
\pgfpathlineto{\pgfqpoint{3.327098in}{1.765766in}}%
\pgfpathlineto{\pgfqpoint{3.341176in}{1.754922in}}%
\pgfpathlineto{\pgfqpoint{3.355255in}{1.744189in}}%
\pgfpathlineto{\pgfqpoint{3.369336in}{1.733568in}}%
\pgfpathlineto{\pgfqpoint{3.360707in}{1.738183in}}%
\pgfpathlineto{\pgfqpoint{3.352062in}{1.743172in}}%
\pgfpathlineto{\pgfqpoint{3.343400in}{1.748544in}}%
\pgfpathlineto{\pgfqpoint{3.334722in}{1.754304in}}%
\pgfpathlineto{\pgfqpoint{3.320600in}{1.765559in}}%
\pgfpathlineto{\pgfqpoint{3.306479in}{1.776926in}}%
\pgfpathlineto{\pgfqpoint{3.292359in}{1.788405in}}%
\pgfpathlineto{\pgfqpoint{3.278239in}{1.799997in}}%
\pgfpathlineto{\pgfqpoint{3.286960in}{1.793594in}}%
\pgfpathlineto{\pgfqpoint{3.295664in}{1.787585in}}%
\pgfpathlineto{\pgfqpoint{3.304351in}{1.781963in}}%
\pgfpathlineto{\pgfqpoint{3.313021in}{1.776722in}}%
\pgfpathclose%
\pgfusepath{fill}%
\end{pgfscope}%
\begin{pgfscope}%
\pgfpathrectangle{\pgfqpoint{1.150000in}{0.150000in}}{\pgfqpoint{5.700000in}{5.700000in}}%
\pgfusepath{clip}%
\pgfsetbuttcap%
\pgfsetroundjoin%
\definecolor{currentfill}{rgb}{0.278012,0.180367,0.486697}%
\pgfsetfillcolor{currentfill}%
\pgfsetfillopacity{0.700000}%
\pgfsetlinewidth{0.000000pt}%
\definecolor{currentstroke}{rgb}{0.000000,0.000000,0.000000}%
\pgfsetstrokecolor{currentstroke}%
\pgfsetdash{}{0pt}%
\pgfpathmoveto{\pgfqpoint{4.805122in}{1.727788in}}%
\pgfpathlineto{\pgfqpoint{4.819552in}{1.730481in}}%
\pgfpathlineto{\pgfqpoint{4.833993in}{1.733270in}}%
\pgfpathlineto{\pgfqpoint{4.848447in}{1.736155in}}%
\pgfpathlineto{\pgfqpoint{4.862913in}{1.739136in}}%
\pgfpathlineto{\pgfqpoint{4.854936in}{1.724810in}}%
\pgfpathlineto{\pgfqpoint{4.846957in}{1.710535in}}%
\pgfpathlineto{\pgfqpoint{4.838974in}{1.696313in}}%
\pgfpathlineto{\pgfqpoint{4.830987in}{1.682150in}}%
\pgfpathlineto{\pgfqpoint{4.816523in}{1.679592in}}%
\pgfpathlineto{\pgfqpoint{4.802071in}{1.677130in}}%
\pgfpathlineto{\pgfqpoint{4.787631in}{1.674764in}}%
\pgfpathlineto{\pgfqpoint{4.773202in}{1.672494in}}%
\pgfpathlineto{\pgfqpoint{4.781187in}{1.686228in}}%
\pgfpathlineto{\pgfqpoint{4.789169in}{1.700024in}}%
\pgfpathlineto{\pgfqpoint{4.797147in}{1.713879in}}%
\pgfpathlineto{\pgfqpoint{4.805122in}{1.727788in}}%
\pgfpathclose%
\pgfusepath{fill}%
\end{pgfscope}%
\begin{pgfscope}%
\pgfpathrectangle{\pgfqpoint{1.150000in}{0.150000in}}{\pgfqpoint{5.700000in}{5.700000in}}%
\pgfusepath{clip}%
\pgfsetbuttcap%
\pgfsetroundjoin%
\definecolor{currentfill}{rgb}{0.279566,0.067836,0.391917}%
\pgfsetfillcolor{currentfill}%
\pgfsetfillopacity{0.700000}%
\pgfsetlinewidth{0.000000pt}%
\definecolor{currentstroke}{rgb}{0.000000,0.000000,0.000000}%
\pgfsetstrokecolor{currentstroke}%
\pgfsetdash{}{0pt}%
\pgfpathmoveto{\pgfqpoint{3.831296in}{1.488834in}}%
\pgfpathlineto{\pgfqpoint{3.845408in}{1.482456in}}%
\pgfpathlineto{\pgfqpoint{3.859525in}{1.476177in}}%
\pgfpathlineto{\pgfqpoint{3.873647in}{1.469999in}}%
\pgfpathlineto{\pgfqpoint{3.887775in}{1.463921in}}%
\pgfpathlineto{\pgfqpoint{3.879497in}{1.461153in}}%
\pgfpathlineto{\pgfqpoint{3.871211in}{1.458667in}}%
\pgfpathlineto{\pgfqpoint{3.862915in}{1.456468in}}%
\pgfpathlineto{\pgfqpoint{3.854609in}{1.454562in}}%
\pgfpathlineto{\pgfqpoint{3.840457in}{1.461223in}}%
\pgfpathlineto{\pgfqpoint{3.826310in}{1.467984in}}%
\pgfpathlineto{\pgfqpoint{3.812167in}{1.474846in}}%
\pgfpathlineto{\pgfqpoint{3.798030in}{1.481808in}}%
\pgfpathlineto{\pgfqpoint{3.806361in}{1.483123in}}%
\pgfpathlineto{\pgfqpoint{3.814683in}{1.484736in}}%
\pgfpathlineto{\pgfqpoint{3.822994in}{1.486642in}}%
\pgfpathlineto{\pgfqpoint{3.831296in}{1.488834in}}%
\pgfpathclose%
\pgfusepath{fill}%
\end{pgfscope}%
\begin{pgfscope}%
\pgfpathrectangle{\pgfqpoint{1.150000in}{0.150000in}}{\pgfqpoint{5.700000in}{5.700000in}}%
\pgfusepath{clip}%
\pgfsetbuttcap%
\pgfsetroundjoin%
\definecolor{currentfill}{rgb}{0.283091,0.110553,0.431554}%
\pgfsetfillcolor{currentfill}%
\pgfsetfillopacity{0.700000}%
\pgfsetlinewidth{0.000000pt}%
\definecolor{currentstroke}{rgb}{0.000000,0.000000,0.000000}%
\pgfsetstrokecolor{currentstroke}%
\pgfsetdash{}{0pt}%
\pgfpathmoveto{\pgfqpoint{3.628708in}{1.573285in}}%
\pgfpathlineto{\pgfqpoint{3.642797in}{1.565096in}}%
\pgfpathlineto{\pgfqpoint{3.656889in}{1.557011in}}%
\pgfpathlineto{\pgfqpoint{3.670985in}{1.549030in}}%
\pgfpathlineto{\pgfqpoint{3.685085in}{1.541151in}}%
\pgfpathlineto{\pgfqpoint{3.676686in}{1.541334in}}%
\pgfpathlineto{\pgfqpoint{3.668275in}{1.541838in}}%
\pgfpathlineto{\pgfqpoint{3.659852in}{1.542670in}}%
\pgfpathlineto{\pgfqpoint{3.651417in}{1.543836in}}%
\pgfpathlineto{\pgfqpoint{3.637286in}{1.552320in}}%
\pgfpathlineto{\pgfqpoint{3.623159in}{1.560907in}}%
\pgfpathlineto{\pgfqpoint{3.609034in}{1.569598in}}%
\pgfpathlineto{\pgfqpoint{3.594913in}{1.578393in}}%
\pgfpathlineto{\pgfqpoint{3.603381in}{1.576613in}}%
\pgfpathlineto{\pgfqpoint{3.611836in}{1.575173in}}%
\pgfpathlineto{\pgfqpoint{3.620278in}{1.574066in}}%
\pgfpathlineto{\pgfqpoint{3.628708in}{1.573285in}}%
\pgfpathclose%
\pgfusepath{fill}%
\end{pgfscope}%
\begin{pgfscope}%
\pgfpathrectangle{\pgfqpoint{1.150000in}{0.150000in}}{\pgfqpoint{5.700000in}{5.700000in}}%
\pgfusepath{clip}%
\pgfsetbuttcap%
\pgfsetroundjoin%
\definecolor{currentfill}{rgb}{0.183898,0.422383,0.556944}%
\pgfsetfillcolor{currentfill}%
\pgfsetfillopacity{0.700000}%
\pgfsetlinewidth{0.000000pt}%
\definecolor{currentstroke}{rgb}{0.000000,0.000000,0.000000}%
\pgfsetstrokecolor{currentstroke}%
\pgfsetdash{}{0pt}%
\pgfpathmoveto{\pgfqpoint{5.349865in}{2.307520in}}%
\pgfpathlineto{\pgfqpoint{5.364604in}{2.314917in}}%
\pgfpathlineto{\pgfqpoint{5.379359in}{2.322414in}}%
\pgfpathlineto{\pgfqpoint{5.394130in}{2.330011in}}%
\pgfpathlineto{\pgfqpoint{5.408916in}{2.337707in}}%
\pgfpathlineto{\pgfqpoint{5.401051in}{2.322234in}}%
\pgfpathlineto{\pgfqpoint{5.393181in}{2.306684in}}%
\pgfpathlineto{\pgfqpoint{5.385306in}{2.291061in}}%
\pgfpathlineto{\pgfqpoint{5.377425in}{2.275367in}}%
\pgfpathlineto{\pgfqpoint{5.362645in}{2.267953in}}%
\pgfpathlineto{\pgfqpoint{5.347880in}{2.260638in}}%
\pgfpathlineto{\pgfqpoint{5.333131in}{2.253423in}}%
\pgfpathlineto{\pgfqpoint{5.318398in}{2.246307in}}%
\pgfpathlineto{\pgfqpoint{5.326273in}{2.261712in}}%
\pgfpathlineto{\pgfqpoint{5.334142in}{2.277051in}}%
\pgfpathlineto{\pgfqpoint{5.342006in}{2.292321in}}%
\pgfpathlineto{\pgfqpoint{5.349865in}{2.307520in}}%
\pgfpathclose%
\pgfusepath{fill}%
\end{pgfscope}%
\begin{pgfscope}%
\pgfpathrectangle{\pgfqpoint{1.150000in}{0.150000in}}{\pgfqpoint{5.700000in}{5.700000in}}%
\pgfusepath{clip}%
\pgfsetbuttcap%
\pgfsetroundjoin%
\definecolor{currentfill}{rgb}{0.276022,0.044167,0.370164}%
\pgfsetfillcolor{currentfill}%
\pgfsetfillopacity{0.700000}%
\pgfsetlinewidth{0.000000pt}%
\definecolor{currentstroke}{rgb}{0.000000,0.000000,0.000000}%
\pgfsetstrokecolor{currentstroke}%
\pgfsetdash{}{0pt}%
\pgfpathmoveto{\pgfqpoint{4.269184in}{1.448675in}}%
\pgfpathlineto{\pgfqpoint{4.283400in}{1.446281in}}%
\pgfpathlineto{\pgfqpoint{4.297624in}{1.443982in}}%
\pgfpathlineto{\pgfqpoint{4.311856in}{1.441780in}}%
\pgfpathlineto{\pgfqpoint{4.326096in}{1.439674in}}%
\pgfpathlineto{\pgfqpoint{4.318003in}{1.430870in}}%
\pgfpathlineto{\pgfqpoint{4.309905in}{1.422253in}}%
\pgfpathlineto{\pgfqpoint{4.301801in}{1.413826in}}%
\pgfpathlineto{\pgfqpoint{4.293692in}{1.405596in}}%
\pgfpathlineto{\pgfqpoint{4.279441in}{1.408228in}}%
\pgfpathlineto{\pgfqpoint{4.265197in}{1.410956in}}%
\pgfpathlineto{\pgfqpoint{4.250962in}{1.413780in}}%
\pgfpathlineto{\pgfqpoint{4.236734in}{1.416700in}}%
\pgfpathlineto{\pgfqpoint{4.244855in}{1.424397in}}%
\pgfpathlineto{\pgfqpoint{4.252971in}{1.432296in}}%
\pgfpathlineto{\pgfqpoint{4.261080in}{1.440390in}}%
\pgfpathlineto{\pgfqpoint{4.269184in}{1.448675in}}%
\pgfpathclose%
\pgfusepath{fill}%
\end{pgfscope}%
\begin{pgfscope}%
\pgfpathrectangle{\pgfqpoint{1.150000in}{0.150000in}}{\pgfqpoint{5.700000in}{5.700000in}}%
\pgfusepath{clip}%
\pgfsetbuttcap%
\pgfsetroundjoin%
\definecolor{currentfill}{rgb}{0.208623,0.367752,0.552675}%
\pgfsetfillcolor{currentfill}%
\pgfsetfillopacity{0.700000}%
\pgfsetlinewidth{0.000000pt}%
\definecolor{currentstroke}{rgb}{0.000000,0.000000,0.000000}%
\pgfsetstrokecolor{currentstroke}%
\pgfsetdash{}{0pt}%
\pgfpathmoveto{\pgfqpoint{5.228090in}{2.157808in}}%
\pgfpathlineto{\pgfqpoint{5.242757in}{2.164228in}}%
\pgfpathlineto{\pgfqpoint{5.257439in}{2.170746in}}%
\pgfpathlineto{\pgfqpoint{5.272136in}{2.177363in}}%
\pgfpathlineto{\pgfqpoint{5.286848in}{2.184080in}}%
\pgfpathlineto{\pgfqpoint{5.278948in}{2.168384in}}%
\pgfpathlineto{\pgfqpoint{5.271044in}{2.152639in}}%
\pgfpathlineto{\pgfqpoint{5.263135in}{2.136846in}}%
\pgfpathlineto{\pgfqpoint{5.255221in}{2.121010in}}%
\pgfpathlineto{\pgfqpoint{5.240515in}{2.114612in}}%
\pgfpathlineto{\pgfqpoint{5.225824in}{2.108314in}}%
\pgfpathlineto{\pgfqpoint{5.211148in}{2.102114in}}%
\pgfpathlineto{\pgfqpoint{5.196486in}{2.096012in}}%
\pgfpathlineto{\pgfqpoint{5.204394in}{2.111523in}}%
\pgfpathlineto{\pgfqpoint{5.212297in}{2.126995in}}%
\pgfpathlineto{\pgfqpoint{5.220196in}{2.142424in}}%
\pgfpathlineto{\pgfqpoint{5.228090in}{2.157808in}}%
\pgfpathclose%
\pgfusepath{fill}%
\end{pgfscope}%
\begin{pgfscope}%
\pgfpathrectangle{\pgfqpoint{1.150000in}{0.150000in}}{\pgfqpoint{5.700000in}{5.700000in}}%
\pgfusepath{clip}%
\pgfsetbuttcap%
\pgfsetroundjoin%
\definecolor{currentfill}{rgb}{0.220124,0.725509,0.466226}%
\pgfsetfillcolor{currentfill}%
\pgfsetfillopacity{0.700000}%
\pgfsetlinewidth{0.000000pt}%
\definecolor{currentstroke}{rgb}{0.000000,0.000000,0.000000}%
\pgfsetstrokecolor{currentstroke}%
\pgfsetdash{}{0pt}%
\pgfpathmoveto{\pgfqpoint{2.123362in}{3.168548in}}%
\pgfpathlineto{\pgfqpoint{2.137771in}{3.145333in}}%
\pgfpathlineto{\pgfqpoint{2.152168in}{3.122316in}}%
\pgfpathlineto{\pgfqpoint{2.166556in}{3.099495in}}%
\pgfpathlineto{\pgfqpoint{2.180933in}{3.076868in}}%
\pgfpathlineto{\pgfqpoint{2.171095in}{3.095266in}}%
\pgfpathlineto{\pgfqpoint{2.161222in}{3.114173in}}%
\pgfpathlineto{\pgfqpoint{2.151315in}{3.133598in}}%
\pgfpathlineto{\pgfqpoint{2.141372in}{3.153550in}}%
\pgfpathlineto{\pgfqpoint{2.126915in}{3.176892in}}%
\pgfpathlineto{\pgfqpoint{2.112449in}{3.200430in}}%
\pgfpathlineto{\pgfqpoint{2.097971in}{3.224167in}}%
\pgfpathlineto{\pgfqpoint{2.083482in}{3.248103in}}%
\pgfpathlineto{\pgfqpoint{2.093506in}{3.227422in}}%
\pgfpathlineto{\pgfqpoint{2.103494in}{3.207274in}}%
\pgfpathlineto{\pgfqpoint{2.113446in}{3.187653in}}%
\pgfpathlineto{\pgfqpoint{2.123362in}{3.168548in}}%
\pgfpathclose%
\pgfusepath{fill}%
\end{pgfscope}%
\begin{pgfscope}%
\pgfpathrectangle{\pgfqpoint{1.150000in}{0.150000in}}{\pgfqpoint{5.700000in}{5.700000in}}%
\pgfusepath{clip}%
\pgfsetbuttcap%
\pgfsetroundjoin%
\definecolor{currentfill}{rgb}{0.276194,0.190074,0.493001}%
\pgfsetfillcolor{currentfill}%
\pgfsetfillopacity{0.700000}%
\pgfsetlinewidth{0.000000pt}%
\definecolor{currentstroke}{rgb}{0.000000,0.000000,0.000000}%
\pgfsetstrokecolor{currentstroke}%
\pgfsetdash{}{0pt}%
\pgfpathmoveto{\pgfqpoint{3.369336in}{1.733568in}}%
\pgfpathlineto{\pgfqpoint{3.383419in}{1.723057in}}%
\pgfpathlineto{\pgfqpoint{3.397503in}{1.712656in}}%
\pgfpathlineto{\pgfqpoint{3.411589in}{1.702366in}}%
\pgfpathlineto{\pgfqpoint{3.425677in}{1.692184in}}%
\pgfpathlineto{\pgfqpoint{3.417087in}{1.696175in}}%
\pgfpathlineto{\pgfqpoint{3.408482in}{1.700534in}}%
\pgfpathlineto{\pgfqpoint{3.399861in}{1.705270in}}%
\pgfpathlineto{\pgfqpoint{3.391225in}{1.710388in}}%
\pgfpathlineto{\pgfqpoint{3.377097in}{1.721202in}}%
\pgfpathlineto{\pgfqpoint{3.362971in}{1.732125in}}%
\pgfpathlineto{\pgfqpoint{3.348846in}{1.743159in}}%
\pgfpathlineto{\pgfqpoint{3.334722in}{1.754304in}}%
\pgfpathlineto{\pgfqpoint{3.343400in}{1.748544in}}%
\pgfpathlineto{\pgfqpoint{3.352062in}{1.743172in}}%
\pgfpathlineto{\pgfqpoint{3.360707in}{1.738183in}}%
\pgfpathlineto{\pgfqpoint{3.369336in}{1.733568in}}%
\pgfpathclose%
\pgfusepath{fill}%
\end{pgfscope}%
\begin{pgfscope}%
\pgfpathrectangle{\pgfqpoint{1.150000in}{0.150000in}}{\pgfqpoint{5.700000in}{5.700000in}}%
\pgfusepath{clip}%
\pgfsetbuttcap%
\pgfsetroundjoin%
\definecolor{currentfill}{rgb}{0.270595,0.214069,0.507052}%
\pgfsetfillcolor{currentfill}%
\pgfsetfillopacity{0.700000}%
\pgfsetlinewidth{0.000000pt}%
\definecolor{currentstroke}{rgb}{0.000000,0.000000,0.000000}%
\pgfsetstrokecolor{currentstroke}%
\pgfsetdash{}{0pt}%
\pgfpathmoveto{\pgfqpoint{4.894783in}{1.796860in}}%
\pgfpathlineto{\pgfqpoint{4.909263in}{1.800344in}}%
\pgfpathlineto{\pgfqpoint{4.923757in}{1.803923in}}%
\pgfpathlineto{\pgfqpoint{4.938262in}{1.807600in}}%
\pgfpathlineto{\pgfqpoint{4.952781in}{1.811373in}}%
\pgfpathlineto{\pgfqpoint{4.944816in}{1.796487in}}%
\pgfpathlineto{\pgfqpoint{4.936847in}{1.781630in}}%
\pgfpathlineto{\pgfqpoint{4.928875in}{1.766808in}}%
\pgfpathlineto{\pgfqpoint{4.920899in}{1.752023in}}%
\pgfpathlineto{\pgfqpoint{4.906384in}{1.748657in}}%
\pgfpathlineto{\pgfqpoint{4.891881in}{1.745387in}}%
\pgfpathlineto{\pgfqpoint{4.877391in}{1.742213in}}%
\pgfpathlineto{\pgfqpoint{4.862913in}{1.739136in}}%
\pgfpathlineto{\pgfqpoint{4.870886in}{1.753507in}}%
\pgfpathlineto{\pgfqpoint{4.878855in}{1.767921in}}%
\pgfpathlineto{\pgfqpoint{4.886821in}{1.782374in}}%
\pgfpathlineto{\pgfqpoint{4.894783in}{1.796860in}}%
\pgfpathclose%
\pgfusepath{fill}%
\end{pgfscope}%
\begin{pgfscope}%
\pgfpathrectangle{\pgfqpoint{1.150000in}{0.150000in}}{\pgfqpoint{5.700000in}{5.700000in}}%
\pgfusepath{clip}%
\pgfsetbuttcap%
\pgfsetroundjoin%
\definecolor{currentfill}{rgb}{0.235526,0.309527,0.542944}%
\pgfsetfillcolor{currentfill}%
\pgfsetfillopacity{0.700000}%
\pgfsetlinewidth{0.000000pt}%
\definecolor{currentstroke}{rgb}{0.000000,0.000000,0.000000}%
\pgfsetstrokecolor{currentstroke}%
\pgfsetdash{}{0pt}%
\pgfpathmoveto{\pgfqpoint{5.106331in}{2.011555in}}%
\pgfpathlineto{\pgfqpoint{5.120930in}{2.016928in}}%
\pgfpathlineto{\pgfqpoint{5.135543in}{2.022399in}}%
\pgfpathlineto{\pgfqpoint{5.150170in}{2.027968in}}%
\pgfpathlineto{\pgfqpoint{5.164811in}{2.033635in}}%
\pgfpathlineto{\pgfqpoint{5.156882in}{2.017972in}}%
\pgfpathlineto{\pgfqpoint{5.148948in}{2.002289in}}%
\pgfpathlineto{\pgfqpoint{5.141011in}{1.986587in}}%
\pgfpathlineto{\pgfqpoint{5.133069in}{1.970872in}}%
\pgfpathlineto{\pgfqpoint{5.118433in}{1.965560in}}%
\pgfpathlineto{\pgfqpoint{5.103812in}{1.960346in}}%
\pgfpathlineto{\pgfqpoint{5.089204in}{1.955229in}}%
\pgfpathlineto{\pgfqpoint{5.074611in}{1.950210in}}%
\pgfpathlineto{\pgfqpoint{5.082547in}{1.965564in}}%
\pgfpathlineto{\pgfqpoint{5.090479in}{1.980909in}}%
\pgfpathlineto{\pgfqpoint{5.098407in}{1.996240in}}%
\pgfpathlineto{\pgfqpoint{5.106331in}{2.011555in}}%
\pgfpathclose%
\pgfusepath{fill}%
\end{pgfscope}%
\begin{pgfscope}%
\pgfpathrectangle{\pgfqpoint{1.150000in}{0.150000in}}{\pgfqpoint{5.700000in}{5.700000in}}%
\pgfusepath{clip}%
\pgfsetbuttcap%
\pgfsetroundjoin%
\definecolor{currentfill}{rgb}{0.276022,0.044167,0.370164}%
\pgfsetfillcolor{currentfill}%
\pgfsetfillopacity{0.700000}%
\pgfsetlinewidth{0.000000pt}%
\definecolor{currentstroke}{rgb}{0.000000,0.000000,0.000000}%
\pgfsetstrokecolor{currentstroke}%
\pgfsetdash{}{0pt}%
\pgfpathmoveto{\pgfqpoint{4.033844in}{1.437131in}}%
\pgfpathlineto{\pgfqpoint{4.048003in}{1.432504in}}%
\pgfpathlineto{\pgfqpoint{4.062168in}{1.427975in}}%
\pgfpathlineto{\pgfqpoint{4.076340in}{1.423544in}}%
\pgfpathlineto{\pgfqpoint{4.090518in}{1.419210in}}%
\pgfpathlineto{\pgfqpoint{4.082334in}{1.413710in}}%
\pgfpathlineto{\pgfqpoint{4.074144in}{1.408453in}}%
\pgfpathlineto{\pgfqpoint{4.065945in}{1.403445in}}%
\pgfpathlineto{\pgfqpoint{4.057740in}{1.398692in}}%
\pgfpathlineto{\pgfqpoint{4.043543in}{1.403589in}}%
\pgfpathlineto{\pgfqpoint{4.029353in}{1.408583in}}%
\pgfpathlineto{\pgfqpoint{4.015169in}{1.413674in}}%
\pgfpathlineto{\pgfqpoint{4.000991in}{1.418863in}}%
\pgfpathlineto{\pgfqpoint{4.009216in}{1.423046in}}%
\pgfpathlineto{\pgfqpoint{4.017433in}{1.427489in}}%
\pgfpathlineto{\pgfqpoint{4.025643in}{1.432186in}}%
\pgfpathlineto{\pgfqpoint{4.033844in}{1.437131in}}%
\pgfpathclose%
\pgfusepath{fill}%
\end{pgfscope}%
\begin{pgfscope}%
\pgfpathrectangle{\pgfqpoint{1.150000in}{0.150000in}}{\pgfqpoint{5.700000in}{5.700000in}}%
\pgfusepath{clip}%
\pgfsetbuttcap%
\pgfsetroundjoin%
\definecolor{currentfill}{rgb}{0.168126,0.459988,0.558082}%
\pgfsetfillcolor{currentfill}%
\pgfsetfillopacity{0.700000}%
\pgfsetlinewidth{0.000000pt}%
\definecolor{currentstroke}{rgb}{0.000000,0.000000,0.000000}%
\pgfsetstrokecolor{currentstroke}%
\pgfsetdash{}{0pt}%
\pgfpathmoveto{\pgfqpoint{5.440317in}{2.398794in}}%
\pgfpathlineto{\pgfqpoint{5.455125in}{2.406855in}}%
\pgfpathlineto{\pgfqpoint{5.469950in}{2.415015in}}%
\pgfpathlineto{\pgfqpoint{5.484791in}{2.423277in}}%
\pgfpathlineto{\pgfqpoint{5.476945in}{2.407938in}}%
\pgfpathlineto{\pgfqpoint{5.469093in}{2.392511in}}%
\pgfpathlineto{\pgfqpoint{5.461235in}{2.376997in}}%
\pgfpathlineto{\pgfqpoint{5.453371in}{2.361398in}}%
\pgfpathlineto{\pgfqpoint{5.438537in}{2.353401in}}%
\pgfpathlineto{\pgfqpoint{5.423718in}{2.345504in}}%
\pgfpathlineto{\pgfqpoint{5.408916in}{2.337707in}}%
\pgfpathlineto{\pgfqpoint{5.416775in}{2.353103in}}%
\pgfpathlineto{\pgfqpoint{5.424628in}{2.368417in}}%
\pgfpathlineto{\pgfqpoint{5.432475in}{2.383649in}}%
\pgfpathlineto{\pgfqpoint{5.440317in}{2.398794in}}%
\pgfpathclose%
\pgfusepath{fill}%
\end{pgfscope}%
\begin{pgfscope}%
\pgfpathrectangle{\pgfqpoint{1.150000in}{0.150000in}}{\pgfqpoint{5.700000in}{5.700000in}}%
\pgfusepath{clip}%
\pgfsetbuttcap%
\pgfsetroundjoin%
\definecolor{currentfill}{rgb}{0.282656,0.100196,0.422160}%
\pgfsetfillcolor{currentfill}%
\pgfsetfillopacity{0.700000}%
\pgfsetlinewidth{0.000000pt}%
\definecolor{currentstroke}{rgb}{0.000000,0.000000,0.000000}%
\pgfsetstrokecolor{currentstroke}%
\pgfsetdash{}{0pt}%
\pgfpathmoveto{\pgfqpoint{3.685085in}{1.541151in}}%
\pgfpathlineto{\pgfqpoint{3.699189in}{1.533376in}}%
\pgfpathlineto{\pgfqpoint{3.713297in}{1.525703in}}%
\pgfpathlineto{\pgfqpoint{3.727408in}{1.518133in}}%
\pgfpathlineto{\pgfqpoint{3.741524in}{1.510665in}}%
\pgfpathlineto{\pgfqpoint{3.733154in}{1.510251in}}%
\pgfpathlineto{\pgfqpoint{3.724773in}{1.510153in}}%
\pgfpathlineto{\pgfqpoint{3.716381in}{1.510377in}}%
\pgfpathlineto{\pgfqpoint{3.707976in}{1.510931in}}%
\pgfpathlineto{\pgfqpoint{3.693831in}{1.519003in}}%
\pgfpathlineto{\pgfqpoint{3.679689in}{1.527178in}}%
\pgfpathlineto{\pgfqpoint{3.665551in}{1.535456in}}%
\pgfpathlineto{\pgfqpoint{3.651417in}{1.543836in}}%
\pgfpathlineto{\pgfqpoint{3.659852in}{1.542670in}}%
\pgfpathlineto{\pgfqpoint{3.668275in}{1.541838in}}%
\pgfpathlineto{\pgfqpoint{3.676686in}{1.541334in}}%
\pgfpathlineto{\pgfqpoint{3.685085in}{1.541151in}}%
\pgfpathclose%
\pgfusepath{fill}%
\end{pgfscope}%
\begin{pgfscope}%
\pgfpathrectangle{\pgfqpoint{1.150000in}{0.150000in}}{\pgfqpoint{5.700000in}{5.700000in}}%
\pgfusepath{clip}%
\pgfsetbuttcap%
\pgfsetroundjoin%
\definecolor{currentfill}{rgb}{0.282910,0.105393,0.426902}%
\pgfsetfillcolor{currentfill}%
\pgfsetfillopacity{0.700000}%
\pgfsetlinewidth{0.000000pt}%
\definecolor{currentstroke}{rgb}{0.000000,0.000000,0.000000}%
\pgfsetstrokecolor{currentstroke}%
\pgfsetdash{}{0pt}%
\pgfpathmoveto{\pgfqpoint{4.594159in}{1.557833in}}%
\pgfpathlineto{\pgfqpoint{4.608502in}{1.558441in}}%
\pgfpathlineto{\pgfqpoint{4.622855in}{1.559145in}}%
\pgfpathlineto{\pgfqpoint{4.637218in}{1.559944in}}%
\pgfpathlineto{\pgfqpoint{4.651592in}{1.560839in}}%
\pgfpathlineto{\pgfqpoint{4.643574in}{1.548325in}}%
\pgfpathlineto{\pgfqpoint{4.635553in}{1.535921in}}%
\pgfpathlineto{\pgfqpoint{4.627528in}{1.523631in}}%
\pgfpathlineto{\pgfqpoint{4.619499in}{1.511460in}}%
\pgfpathlineto{\pgfqpoint{4.605123in}{1.511040in}}%
\pgfpathlineto{\pgfqpoint{4.590757in}{1.510715in}}%
\pgfpathlineto{\pgfqpoint{4.576401in}{1.510486in}}%
\pgfpathlineto{\pgfqpoint{4.562055in}{1.510352in}}%
\pgfpathlineto{\pgfqpoint{4.570087in}{1.522041in}}%
\pgfpathlineto{\pgfqpoint{4.578115in}{1.533855in}}%
\pgfpathlineto{\pgfqpoint{4.586139in}{1.545787in}}%
\pgfpathlineto{\pgfqpoint{4.594159in}{1.557833in}}%
\pgfpathclose%
\pgfusepath{fill}%
\end{pgfscope}%
\begin{pgfscope}%
\pgfpathrectangle{\pgfqpoint{1.150000in}{0.150000in}}{\pgfqpoint{5.700000in}{5.700000in}}%
\pgfusepath{clip}%
\pgfsetbuttcap%
\pgfsetroundjoin%
\definecolor{currentfill}{rgb}{0.274952,0.037752,0.364543}%
\pgfsetfillcolor{currentfill}%
\pgfsetfillopacity{0.700000}%
\pgfsetlinewidth{0.000000pt}%
\definecolor{currentstroke}{rgb}{0.000000,0.000000,0.000000}%
\pgfsetstrokecolor{currentstroke}%
\pgfsetdash{}{0pt}%
\pgfpathmoveto{\pgfqpoint{4.179899in}{1.429343in}}%
\pgfpathlineto{\pgfqpoint{4.194097in}{1.426037in}}%
\pgfpathlineto{\pgfqpoint{4.208302in}{1.422828in}}%
\pgfpathlineto{\pgfqpoint{4.222514in}{1.419716in}}%
\pgfpathlineto{\pgfqpoint{4.236734in}{1.416700in}}%
\pgfpathlineto{\pgfqpoint{4.228607in}{1.409209in}}%
\pgfpathlineto{\pgfqpoint{4.220474in}{1.401930in}}%
\pgfpathlineto{\pgfqpoint{4.212335in}{1.394867in}}%
\pgfpathlineto{\pgfqpoint{4.204190in}{1.388028in}}%
\pgfpathlineto{\pgfqpoint{4.189956in}{1.391588in}}%
\pgfpathlineto{\pgfqpoint{4.175729in}{1.395244in}}%
\pgfpathlineto{\pgfqpoint{4.161510in}{1.398997in}}%
\pgfpathlineto{\pgfqpoint{4.147298in}{1.402846in}}%
\pgfpathlineto{\pgfqpoint{4.155458in}{1.409134in}}%
\pgfpathlineto{\pgfqpoint{4.163611in}{1.415650in}}%
\pgfpathlineto{\pgfqpoint{4.171759in}{1.422388in}}%
\pgfpathlineto{\pgfqpoint{4.179899in}{1.429343in}}%
\pgfpathclose%
\pgfusepath{fill}%
\end{pgfscope}%
\begin{pgfscope}%
\pgfpathrectangle{\pgfqpoint{1.150000in}{0.150000in}}{\pgfqpoint{5.700000in}{5.700000in}}%
\pgfusepath{clip}%
\pgfsetbuttcap%
\pgfsetroundjoin%
\definecolor{currentfill}{rgb}{0.281446,0.084320,0.407414}%
\pgfsetfillcolor{currentfill}%
\pgfsetfillopacity{0.700000}%
\pgfsetlinewidth{0.000000pt}%
\definecolor{currentstroke}{rgb}{0.000000,0.000000,0.000000}%
\pgfsetstrokecolor{currentstroke}%
\pgfsetdash{}{0pt}%
\pgfpathmoveto{\pgfqpoint{4.504771in}{1.510768in}}%
\pgfpathlineto{\pgfqpoint{4.519077in}{1.510521in}}%
\pgfpathlineto{\pgfqpoint{4.533393in}{1.510369in}}%
\pgfpathlineto{\pgfqpoint{4.547719in}{1.510313in}}%
\pgfpathlineto{\pgfqpoint{4.562055in}{1.510352in}}%
\pgfpathlineto{\pgfqpoint{4.554019in}{1.498790in}}%
\pgfpathlineto{\pgfqpoint{4.545979in}{1.487361in}}%
\pgfpathlineto{\pgfqpoint{4.537936in}{1.476069in}}%
\pgfpathlineto{\pgfqpoint{4.529888in}{1.464920in}}%
\pgfpathlineto{\pgfqpoint{4.515548in}{1.465373in}}%
\pgfpathlineto{\pgfqpoint{4.501217in}{1.465921in}}%
\pgfpathlineto{\pgfqpoint{4.486895in}{1.466564in}}%
\pgfpathlineto{\pgfqpoint{4.472584in}{1.467302in}}%
\pgfpathlineto{\pgfqpoint{4.480637in}{1.477953in}}%
\pgfpathlineto{\pgfqpoint{4.488686in}{1.488751in}}%
\pgfpathlineto{\pgfqpoint{4.496730in}{1.499691in}}%
\pgfpathlineto{\pgfqpoint{4.504771in}{1.510768in}}%
\pgfpathclose%
\pgfusepath{fill}%
\end{pgfscope}%
\begin{pgfscope}%
\pgfpathrectangle{\pgfqpoint{1.150000in}{0.150000in}}{\pgfqpoint{5.700000in}{5.700000in}}%
\pgfusepath{clip}%
\pgfsetbuttcap%
\pgfsetroundjoin%
\definecolor{currentfill}{rgb}{0.278791,0.062145,0.386592}%
\pgfsetfillcolor{currentfill}%
\pgfsetfillopacity{0.700000}%
\pgfsetlinewidth{0.000000pt}%
\definecolor{currentstroke}{rgb}{0.000000,0.000000,0.000000}%
\pgfsetstrokecolor{currentstroke}%
\pgfsetdash{}{0pt}%
\pgfpathmoveto{\pgfqpoint{3.887775in}{1.463921in}}%
\pgfpathlineto{\pgfqpoint{3.901908in}{1.457942in}}%
\pgfpathlineto{\pgfqpoint{3.916046in}{1.452063in}}%
\pgfpathlineto{\pgfqpoint{3.930189in}{1.446283in}}%
\pgfpathlineto{\pgfqpoint{3.944338in}{1.440602in}}%
\pgfpathlineto{\pgfqpoint{3.936083in}{1.437259in}}%
\pgfpathlineto{\pgfqpoint{3.927820in}{1.434193in}}%
\pgfpathlineto{\pgfqpoint{3.919548in}{1.431409in}}%
\pgfpathlineto{\pgfqpoint{3.911266in}{1.428913in}}%
\pgfpathlineto{\pgfqpoint{3.897094in}{1.435176in}}%
\pgfpathlineto{\pgfqpoint{3.882927in}{1.441539in}}%
\pgfpathlineto{\pgfqpoint{3.868765in}{1.448001in}}%
\pgfpathlineto{\pgfqpoint{3.854609in}{1.454562in}}%
\pgfpathlineto{\pgfqpoint{3.862915in}{1.456468in}}%
\pgfpathlineto{\pgfqpoint{3.871211in}{1.458667in}}%
\pgfpathlineto{\pgfqpoint{3.879497in}{1.461153in}}%
\pgfpathlineto{\pgfqpoint{3.887775in}{1.463921in}}%
\pgfpathclose%
\pgfusepath{fill}%
\end{pgfscope}%
\begin{pgfscope}%
\pgfpathrectangle{\pgfqpoint{1.150000in}{0.150000in}}{\pgfqpoint{5.700000in}{5.700000in}}%
\pgfusepath{clip}%
\pgfsetbuttcap%
\pgfsetroundjoin%
\definecolor{currentfill}{rgb}{0.278826,0.175490,0.483397}%
\pgfsetfillcolor{currentfill}%
\pgfsetfillopacity{0.700000}%
\pgfsetlinewidth{0.000000pt}%
\definecolor{currentstroke}{rgb}{0.000000,0.000000,0.000000}%
\pgfsetstrokecolor{currentstroke}%
\pgfsetdash{}{0pt}%
\pgfpathmoveto{\pgfqpoint{3.425677in}{1.692184in}}%
\pgfpathlineto{\pgfqpoint{3.439767in}{1.682112in}}%
\pgfpathlineto{\pgfqpoint{3.453859in}{1.672148in}}%
\pgfpathlineto{\pgfqpoint{3.467953in}{1.662292in}}%
\pgfpathlineto{\pgfqpoint{3.482049in}{1.652544in}}%
\pgfpathlineto{\pgfqpoint{3.473497in}{1.655911in}}%
\pgfpathlineto{\pgfqpoint{3.464931in}{1.659642in}}%
\pgfpathlineto{\pgfqpoint{3.456349in}{1.663744in}}%
\pgfpathlineto{\pgfqpoint{3.447752in}{1.668223in}}%
\pgfpathlineto{\pgfqpoint{3.433618in}{1.678601in}}%
\pgfpathlineto{\pgfqpoint{3.419485in}{1.689088in}}%
\pgfpathlineto{\pgfqpoint{3.405354in}{1.699684in}}%
\pgfpathlineto{\pgfqpoint{3.391225in}{1.710388in}}%
\pgfpathlineto{\pgfqpoint{3.399861in}{1.705270in}}%
\pgfpathlineto{\pgfqpoint{3.408482in}{1.700534in}}%
\pgfpathlineto{\pgfqpoint{3.417087in}{1.696175in}}%
\pgfpathlineto{\pgfqpoint{3.425677in}{1.692184in}}%
\pgfpathclose%
\pgfusepath{fill}%
\end{pgfscope}%
\begin{pgfscope}%
\pgfpathrectangle{\pgfqpoint{1.150000in}{0.150000in}}{\pgfqpoint{5.700000in}{5.700000in}}%
\pgfusepath{clip}%
\pgfsetbuttcap%
\pgfsetroundjoin%
\definecolor{currentfill}{rgb}{0.282884,0.135920,0.453427}%
\pgfsetfillcolor{currentfill}%
\pgfsetfillopacity{0.700000}%
\pgfsetlinewidth{0.000000pt}%
\definecolor{currentstroke}{rgb}{0.000000,0.000000,0.000000}%
\pgfsetstrokecolor{currentstroke}%
\pgfsetdash{}{0pt}%
\pgfpathmoveto{\pgfqpoint{4.683627in}{1.611903in}}%
\pgfpathlineto{\pgfqpoint{4.698010in}{1.613350in}}%
\pgfpathlineto{\pgfqpoint{4.712404in}{1.614893in}}%
\pgfpathlineto{\pgfqpoint{4.726810in}{1.616532in}}%
\pgfpathlineto{\pgfqpoint{4.741226in}{1.618266in}}%
\pgfpathlineto{\pgfqpoint{4.733224in}{1.604906in}}%
\pgfpathlineto{\pgfqpoint{4.725218in}{1.591635in}}%
\pgfpathlineto{\pgfqpoint{4.717208in}{1.578455in}}%
\pgfpathlineto{\pgfqpoint{4.709195in}{1.565371in}}%
\pgfpathlineto{\pgfqpoint{4.694778in}{1.564095in}}%
\pgfpathlineto{\pgfqpoint{4.680372in}{1.562914in}}%
\pgfpathlineto{\pgfqpoint{4.665977in}{1.561829in}}%
\pgfpathlineto{\pgfqpoint{4.651592in}{1.560839in}}%
\pgfpathlineto{\pgfqpoint{4.659606in}{1.573458in}}%
\pgfpathlineto{\pgfqpoint{4.667617in}{1.586178in}}%
\pgfpathlineto{\pgfqpoint{4.675623in}{1.598994in}}%
\pgfpathlineto{\pgfqpoint{4.683627in}{1.611903in}}%
\pgfpathclose%
\pgfusepath{fill}%
\end{pgfscope}%
\begin{pgfscope}%
\pgfpathrectangle{\pgfqpoint{1.150000in}{0.150000in}}{\pgfqpoint{5.700000in}{5.700000in}}%
\pgfusepath{clip}%
\pgfsetbuttcap%
\pgfsetroundjoin%
\definecolor{currentfill}{rgb}{0.288921,0.758394,0.428426}%
\pgfsetfillcolor{currentfill}%
\pgfsetfillopacity{0.700000}%
\pgfsetlinewidth{0.000000pt}%
\definecolor{currentstroke}{rgb}{0.000000,0.000000,0.000000}%
\pgfsetstrokecolor{currentstroke}%
\pgfsetdash{}{0pt}%
\pgfpathmoveto{\pgfqpoint{2.065617in}{3.263430in}}%
\pgfpathlineto{\pgfqpoint{2.080070in}{3.239403in}}%
\pgfpathlineto{\pgfqpoint{2.094512in}{3.215581in}}%
\pgfpathlineto{\pgfqpoint{2.108943in}{3.191964in}}%
\pgfpathlineto{\pgfqpoint{2.123362in}{3.168548in}}%
\pgfpathlineto{\pgfqpoint{2.113446in}{3.187653in}}%
\pgfpathlineto{\pgfqpoint{2.103494in}{3.207274in}}%
\pgfpathlineto{\pgfqpoint{2.093506in}{3.227422in}}%
\pgfpathlineto{\pgfqpoint{2.083482in}{3.248103in}}%
\pgfpathlineto{\pgfqpoint{2.068982in}{3.272241in}}%
\pgfpathlineto{\pgfqpoint{2.054470in}{3.296583in}}%
\pgfpathlineto{\pgfqpoint{2.039946in}{3.321130in}}%
\pgfpathlineto{\pgfqpoint{2.025410in}{3.345886in}}%
\pgfpathlineto{\pgfqpoint{2.035518in}{3.324468in}}%
\pgfpathlineto{\pgfqpoint{2.045588in}{3.303591in}}%
\pgfpathlineto{\pgfqpoint{2.055621in}{3.283248in}}%
\pgfpathlineto{\pgfqpoint{2.065617in}{3.263430in}}%
\pgfpathclose%
\pgfusepath{fill}%
\end{pgfscope}%
\begin{pgfscope}%
\pgfpathrectangle{\pgfqpoint{1.150000in}{0.150000in}}{\pgfqpoint{5.700000in}{5.700000in}}%
\pgfusepath{clip}%
\pgfsetbuttcap%
\pgfsetroundjoin%
\definecolor{currentfill}{rgb}{0.182256,0.426184,0.557120}%
\pgfsetfillcolor{currentfill}%
\pgfsetfillopacity{0.700000}%
\pgfsetlinewidth{0.000000pt}%
\definecolor{currentstroke}{rgb}{0.000000,0.000000,0.000000}%
\pgfsetstrokecolor{currentstroke}%
\pgfsetdash{}{0pt}%
\pgfpathmoveto{\pgfqpoint{2.769654in}{2.296987in}}%
\pgfpathlineto{\pgfqpoint{2.783812in}{2.280950in}}%
\pgfpathlineto{\pgfqpoint{2.797968in}{2.265048in}}%
\pgfpathlineto{\pgfqpoint{2.812120in}{2.249282in}}%
\pgfpathlineto{\pgfqpoint{2.826270in}{2.233650in}}%
\pgfpathlineto{\pgfqpoint{2.817121in}{2.245739in}}%
\pgfpathlineto{\pgfqpoint{2.807948in}{2.258285in}}%
\pgfpathlineto{\pgfqpoint{2.798749in}{2.271298in}}%
\pgfpathlineto{\pgfqpoint{2.789525in}{2.284783in}}%
\pgfpathlineto{\pgfqpoint{2.775315in}{2.301098in}}%
\pgfpathlineto{\pgfqpoint{2.761103in}{2.317548in}}%
\pgfpathlineto{\pgfqpoint{2.746887in}{2.334134in}}%
\pgfpathlineto{\pgfqpoint{2.732667in}{2.350857in}}%
\pgfpathlineto{\pgfqpoint{2.741953in}{2.336677in}}%
\pgfpathlineto{\pgfqpoint{2.751213in}{2.322978in}}%
\pgfpathlineto{\pgfqpoint{2.760446in}{2.309750in}}%
\pgfpathlineto{\pgfqpoint{2.769654in}{2.296987in}}%
\pgfpathclose%
\pgfusepath{fill}%
\end{pgfscope}%
\begin{pgfscope}%
\pgfpathrectangle{\pgfqpoint{1.150000in}{0.150000in}}{\pgfqpoint{5.700000in}{5.700000in}}%
\pgfusepath{clip}%
\pgfsetbuttcap%
\pgfsetroundjoin%
\definecolor{currentfill}{rgb}{0.171176,0.452530,0.557965}%
\pgfsetfillcolor{currentfill}%
\pgfsetfillopacity{0.700000}%
\pgfsetlinewidth{0.000000pt}%
\definecolor{currentstroke}{rgb}{0.000000,0.000000,0.000000}%
\pgfsetstrokecolor{currentstroke}%
\pgfsetdash{}{0pt}%
\pgfpathmoveto{\pgfqpoint{2.712985in}{2.362517in}}%
\pgfpathlineto{\pgfqpoint{2.727158in}{2.345926in}}%
\pgfpathlineto{\pgfqpoint{2.741326in}{2.329474in}}%
\pgfpathlineto{\pgfqpoint{2.755492in}{2.313162in}}%
\pgfpathlineto{\pgfqpoint{2.769654in}{2.296987in}}%
\pgfpathlineto{\pgfqpoint{2.760446in}{2.309750in}}%
\pgfpathlineto{\pgfqpoint{2.751213in}{2.322978in}}%
\pgfpathlineto{\pgfqpoint{2.741953in}{2.336677in}}%
\pgfpathlineto{\pgfqpoint{2.732667in}{2.350857in}}%
\pgfpathlineto{\pgfqpoint{2.718444in}{2.367719in}}%
\pgfpathlineto{\pgfqpoint{2.704217in}{2.384719in}}%
\pgfpathlineto{\pgfqpoint{2.689986in}{2.401859in}}%
\pgfpathlineto{\pgfqpoint{2.675750in}{2.419140in}}%
\pgfpathlineto{\pgfqpoint{2.685100in}{2.404262in}}%
\pgfpathlineto{\pgfqpoint{2.694422in}{2.389871in}}%
\pgfpathlineto{\pgfqpoint{2.703717in}{2.375958in}}%
\pgfpathlineto{\pgfqpoint{2.712985in}{2.362517in}}%
\pgfpathclose%
\pgfusepath{fill}%
\end{pgfscope}%
\begin{pgfscope}%
\pgfpathrectangle{\pgfqpoint{1.150000in}{0.150000in}}{\pgfqpoint{5.700000in}{5.700000in}}%
\pgfusepath{clip}%
\pgfsetbuttcap%
\pgfsetroundjoin%
\definecolor{currentfill}{rgb}{0.258965,0.251537,0.524736}%
\pgfsetfillcolor{currentfill}%
\pgfsetfillopacity{0.700000}%
\pgfsetlinewidth{0.000000pt}%
\definecolor{currentstroke}{rgb}{0.000000,0.000000,0.000000}%
\pgfsetstrokecolor{currentstroke}%
\pgfsetdash{}{0pt}%
\pgfpathmoveto{\pgfqpoint{4.984606in}{1.871145in}}%
\pgfpathlineto{\pgfqpoint{4.999142in}{1.875404in}}%
\pgfpathlineto{\pgfqpoint{5.013690in}{1.879759in}}%
\pgfpathlineto{\pgfqpoint{5.028252in}{1.884212in}}%
\pgfpathlineto{\pgfqpoint{5.042827in}{1.888762in}}%
\pgfpathlineto{\pgfqpoint{5.034872in}{1.873409in}}%
\pgfpathlineto{\pgfqpoint{5.026913in}{1.858068in}}%
\pgfpathlineto{\pgfqpoint{5.018951in}{1.842741in}}%
\pgfpathlineto{\pgfqpoint{5.010985in}{1.827431in}}%
\pgfpathlineto{\pgfqpoint{4.996414in}{1.823272in}}%
\pgfpathlineto{\pgfqpoint{4.981857in}{1.819209in}}%
\pgfpathlineto{\pgfqpoint{4.967312in}{1.815243in}}%
\pgfpathlineto{\pgfqpoint{4.952781in}{1.811373in}}%
\pgfpathlineto{\pgfqpoint{4.960743in}{1.826286in}}%
\pgfpathlineto{\pgfqpoint{4.968701in}{1.841221in}}%
\pgfpathlineto{\pgfqpoint{4.976655in}{1.856175in}}%
\pgfpathlineto{\pgfqpoint{4.984606in}{1.871145in}}%
\pgfpathclose%
\pgfusepath{fill}%
\end{pgfscope}%
\begin{pgfscope}%
\pgfpathrectangle{\pgfqpoint{1.150000in}{0.150000in}}{\pgfqpoint{5.700000in}{5.700000in}}%
\pgfusepath{clip}%
\pgfsetbuttcap%
\pgfsetroundjoin%
\definecolor{currentfill}{rgb}{0.278791,0.062145,0.386592}%
\pgfsetfillcolor{currentfill}%
\pgfsetfillopacity{0.700000}%
\pgfsetlinewidth{0.000000pt}%
\definecolor{currentstroke}{rgb}{0.000000,0.000000,0.000000}%
\pgfsetstrokecolor{currentstroke}%
\pgfsetdash{}{0pt}%
\pgfpathmoveto{\pgfqpoint{4.415429in}{1.471209in}}%
\pgfpathlineto{\pgfqpoint{4.429704in}{1.470089in}}%
\pgfpathlineto{\pgfqpoint{4.443988in}{1.469064in}}%
\pgfpathlineto{\pgfqpoint{4.458281in}{1.468136in}}%
\pgfpathlineto{\pgfqpoint{4.472584in}{1.467302in}}%
\pgfpathlineto{\pgfqpoint{4.464526in}{1.456803in}}%
\pgfpathlineto{\pgfqpoint{4.456464in}{1.446460in}}%
\pgfpathlineto{\pgfqpoint{4.448398in}{1.436279in}}%
\pgfpathlineto{\pgfqpoint{4.440327in}{1.426264in}}%
\pgfpathlineto{\pgfqpoint{4.426017in}{1.427607in}}%
\pgfpathlineto{\pgfqpoint{4.411717in}{1.429044in}}%
\pgfpathlineto{\pgfqpoint{4.397425in}{1.430577in}}%
\pgfpathlineto{\pgfqpoint{4.383142in}{1.432206in}}%
\pgfpathlineto{\pgfqpoint{4.391221in}{1.441705in}}%
\pgfpathlineto{\pgfqpoint{4.399295in}{1.451375in}}%
\pgfpathlineto{\pgfqpoint{4.407364in}{1.461211in}}%
\pgfpathlineto{\pgfqpoint{4.415429in}{1.471209in}}%
\pgfpathclose%
\pgfusepath{fill}%
\end{pgfscope}%
\begin{pgfscope}%
\pgfpathrectangle{\pgfqpoint{1.150000in}{0.150000in}}{\pgfqpoint{5.700000in}{5.700000in}}%
\pgfusepath{clip}%
\pgfsetbuttcap%
\pgfsetroundjoin%
\definecolor{currentfill}{rgb}{0.192357,0.403199,0.555836}%
\pgfsetfillcolor{currentfill}%
\pgfsetfillopacity{0.700000}%
\pgfsetlinewidth{0.000000pt}%
\definecolor{currentstroke}{rgb}{0.000000,0.000000,0.000000}%
\pgfsetstrokecolor{currentstroke}%
\pgfsetdash{}{0pt}%
\pgfpathmoveto{\pgfqpoint{2.826270in}{2.233650in}}%
\pgfpathlineto{\pgfqpoint{2.840417in}{2.218152in}}%
\pgfpathlineto{\pgfqpoint{2.854561in}{2.202787in}}%
\pgfpathlineto{\pgfqpoint{2.868703in}{2.187554in}}%
\pgfpathlineto{\pgfqpoint{2.882842in}{2.172452in}}%
\pgfpathlineto{\pgfqpoint{2.873750in}{2.183869in}}%
\pgfpathlineto{\pgfqpoint{2.864635in}{2.195739in}}%
\pgfpathlineto{\pgfqpoint{2.855495in}{2.208067in}}%
\pgfpathlineto{\pgfqpoint{2.846331in}{2.220862in}}%
\pgfpathlineto{\pgfqpoint{2.832133in}{2.236644in}}%
\pgfpathlineto{\pgfqpoint{2.817934in}{2.252557in}}%
\pgfpathlineto{\pgfqpoint{2.803731in}{2.268603in}}%
\pgfpathlineto{\pgfqpoint{2.789525in}{2.284783in}}%
\pgfpathlineto{\pgfqpoint{2.798749in}{2.271298in}}%
\pgfpathlineto{\pgfqpoint{2.807948in}{2.258285in}}%
\pgfpathlineto{\pgfqpoint{2.817121in}{2.245739in}}%
\pgfpathlineto{\pgfqpoint{2.826270in}{2.233650in}}%
\pgfpathclose%
\pgfusepath{fill}%
\end{pgfscope}%
\begin{pgfscope}%
\pgfpathrectangle{\pgfqpoint{1.150000in}{0.150000in}}{\pgfqpoint{5.700000in}{5.700000in}}%
\pgfusepath{clip}%
\pgfsetbuttcap%
\pgfsetroundjoin%
\definecolor{currentfill}{rgb}{0.162142,0.474838,0.558140}%
\pgfsetfillcolor{currentfill}%
\pgfsetfillopacity{0.700000}%
\pgfsetlinewidth{0.000000pt}%
\definecolor{currentstroke}{rgb}{0.000000,0.000000,0.000000}%
\pgfsetstrokecolor{currentstroke}%
\pgfsetdash{}{0pt}%
\pgfpathmoveto{\pgfqpoint{2.656256in}{2.430299in}}%
\pgfpathlineto{\pgfqpoint{2.670444in}{2.413139in}}%
\pgfpathlineto{\pgfqpoint{2.684628in}{2.396123in}}%
\pgfpathlineto{\pgfqpoint{2.698809in}{2.379249in}}%
\pgfpathlineto{\pgfqpoint{2.712985in}{2.362517in}}%
\pgfpathlineto{\pgfqpoint{2.703717in}{2.375958in}}%
\pgfpathlineto{\pgfqpoint{2.694422in}{2.389871in}}%
\pgfpathlineto{\pgfqpoint{2.685100in}{2.404262in}}%
\pgfpathlineto{\pgfqpoint{2.675750in}{2.419140in}}%
\pgfpathlineto{\pgfqpoint{2.661511in}{2.436563in}}%
\pgfpathlineto{\pgfqpoint{2.647267in}{2.454128in}}%
\pgfpathlineto{\pgfqpoint{2.633019in}{2.471837in}}%
\pgfpathlineto{\pgfqpoint{2.618765in}{2.489691in}}%
\pgfpathlineto{\pgfqpoint{2.628180in}{2.474110in}}%
\pgfpathlineto{\pgfqpoint{2.637566in}{2.459023in}}%
\pgfpathlineto{\pgfqpoint{2.646925in}{2.444422in}}%
\pgfpathlineto{\pgfqpoint{2.656256in}{2.430299in}}%
\pgfpathclose%
\pgfusepath{fill}%
\end{pgfscope}%
\begin{pgfscope}%
\pgfpathrectangle{\pgfqpoint{1.150000in}{0.150000in}}{\pgfqpoint{5.700000in}{5.700000in}}%
\pgfusepath{clip}%
\pgfsetbuttcap%
\pgfsetroundjoin%
\definecolor{currentfill}{rgb}{0.201239,0.383670,0.554294}%
\pgfsetfillcolor{currentfill}%
\pgfsetfillopacity{0.700000}%
\pgfsetlinewidth{0.000000pt}%
\definecolor{currentstroke}{rgb}{0.000000,0.000000,0.000000}%
\pgfsetstrokecolor{currentstroke}%
\pgfsetdash{}{0pt}%
\pgfpathmoveto{\pgfqpoint{2.882842in}{2.172452in}}%
\pgfpathlineto{\pgfqpoint{2.896979in}{2.157480in}}%
\pgfpathlineto{\pgfqpoint{2.911114in}{2.142638in}}%
\pgfpathlineto{\pgfqpoint{2.925246in}{2.127925in}}%
\pgfpathlineto{\pgfqpoint{2.939377in}{2.113340in}}%
\pgfpathlineto{\pgfqpoint{2.930341in}{2.124090in}}%
\pgfpathlineto{\pgfqpoint{2.921282in}{2.135285in}}%
\pgfpathlineto{\pgfqpoint{2.912199in}{2.146933in}}%
\pgfpathlineto{\pgfqpoint{2.903093in}{2.159042in}}%
\pgfpathlineto{\pgfqpoint{2.888906in}{2.174303in}}%
\pgfpathlineto{\pgfqpoint{2.874717in}{2.189693in}}%
\pgfpathlineto{\pgfqpoint{2.860525in}{2.205212in}}%
\pgfpathlineto{\pgfqpoint{2.846331in}{2.220862in}}%
\pgfpathlineto{\pgfqpoint{2.855495in}{2.208067in}}%
\pgfpathlineto{\pgfqpoint{2.864635in}{2.195739in}}%
\pgfpathlineto{\pgfqpoint{2.873750in}{2.183869in}}%
\pgfpathlineto{\pgfqpoint{2.882842in}{2.172452in}}%
\pgfpathclose%
\pgfusepath{fill}%
\end{pgfscope}%
\begin{pgfscope}%
\pgfpathrectangle{\pgfqpoint{1.150000in}{0.150000in}}{\pgfqpoint{5.700000in}{5.700000in}}%
\pgfusepath{clip}%
\pgfsetbuttcap%
\pgfsetroundjoin%
\definecolor{currentfill}{rgb}{0.151918,0.500685,0.557587}%
\pgfsetfillcolor{currentfill}%
\pgfsetfillopacity{0.700000}%
\pgfsetlinewidth{0.000000pt}%
\definecolor{currentstroke}{rgb}{0.000000,0.000000,0.000000}%
\pgfsetstrokecolor{currentstroke}%
\pgfsetdash{}{0pt}%
\pgfpathmoveto{\pgfqpoint{2.599456in}{2.500395in}}%
\pgfpathlineto{\pgfqpoint{2.613663in}{2.482650in}}%
\pgfpathlineto{\pgfqpoint{2.627865in}{2.465054in}}%
\pgfpathlineto{\pgfqpoint{2.642063in}{2.447603in}}%
\pgfpathlineto{\pgfqpoint{2.656256in}{2.430299in}}%
\pgfpathlineto{\pgfqpoint{2.646925in}{2.444422in}}%
\pgfpathlineto{\pgfqpoint{2.637566in}{2.459023in}}%
\pgfpathlineto{\pgfqpoint{2.628180in}{2.474110in}}%
\pgfpathlineto{\pgfqpoint{2.618765in}{2.489691in}}%
\pgfpathlineto{\pgfqpoint{2.604507in}{2.507691in}}%
\pgfpathlineto{\pgfqpoint{2.590244in}{2.525837in}}%
\pgfpathlineto{\pgfqpoint{2.575976in}{2.544131in}}%
\pgfpathlineto{\pgfqpoint{2.561703in}{2.562574in}}%
\pgfpathlineto{\pgfqpoint{2.571185in}{2.546286in}}%
\pgfpathlineto{\pgfqpoint{2.580638in}{2.530499in}}%
\pgfpathlineto{\pgfqpoint{2.590061in}{2.515205in}}%
\pgfpathlineto{\pgfqpoint{2.599456in}{2.500395in}}%
\pgfpathclose%
\pgfusepath{fill}%
\end{pgfscope}%
\begin{pgfscope}%
\pgfpathrectangle{\pgfqpoint{1.150000in}{0.150000in}}{\pgfqpoint{5.700000in}{5.700000in}}%
\pgfusepath{clip}%
\pgfsetbuttcap%
\pgfsetroundjoin%
\definecolor{currentfill}{rgb}{0.190631,0.407061,0.556089}%
\pgfsetfillcolor{currentfill}%
\pgfsetfillopacity{0.700000}%
\pgfsetlinewidth{0.000000pt}%
\definecolor{currentstroke}{rgb}{0.000000,0.000000,0.000000}%
\pgfsetstrokecolor{currentstroke}%
\pgfsetdash{}{0pt}%
\pgfpathmoveto{\pgfqpoint{5.318398in}{2.246307in}}%
\pgfpathlineto{\pgfqpoint{5.333131in}{2.253423in}}%
\pgfpathlineto{\pgfqpoint{5.347880in}{2.260638in}}%
\pgfpathlineto{\pgfqpoint{5.362645in}{2.267953in}}%
\pgfpathlineto{\pgfqpoint{5.377425in}{2.275367in}}%
\pgfpathlineto{\pgfqpoint{5.369538in}{2.259605in}}%
\pgfpathlineto{\pgfqpoint{5.361647in}{2.243777in}}%
\pgfpathlineto{\pgfqpoint{5.353750in}{2.227886in}}%
\pgfpathlineto{\pgfqpoint{5.345849in}{2.211935in}}%
\pgfpathlineto{\pgfqpoint{5.331076in}{2.204822in}}%
\pgfpathlineto{\pgfqpoint{5.316318in}{2.197809in}}%
\pgfpathlineto{\pgfqpoint{5.301575in}{2.190895in}}%
\pgfpathlineto{\pgfqpoint{5.286848in}{2.184080in}}%
\pgfpathlineto{\pgfqpoint{5.294743in}{2.199722in}}%
\pgfpathlineto{\pgfqpoint{5.302633in}{2.215310in}}%
\pgfpathlineto{\pgfqpoint{5.310518in}{2.230839in}}%
\pgfpathlineto{\pgfqpoint{5.318398in}{2.246307in}}%
\pgfpathclose%
\pgfusepath{fill}%
\end{pgfscope}%
\begin{pgfscope}%
\pgfpathrectangle{\pgfqpoint{1.150000in}{0.150000in}}{\pgfqpoint{5.700000in}{5.700000in}}%
\pgfusepath{clip}%
\pgfsetbuttcap%
\pgfsetroundjoin%
\definecolor{currentfill}{rgb}{0.280255,0.165693,0.476498}%
\pgfsetfillcolor{currentfill}%
\pgfsetfillopacity{0.700000}%
\pgfsetlinewidth{0.000000pt}%
\definecolor{currentstroke}{rgb}{0.000000,0.000000,0.000000}%
\pgfsetstrokecolor{currentstroke}%
\pgfsetdash{}{0pt}%
\pgfpathmoveto{\pgfqpoint{4.773202in}{1.672494in}}%
\pgfpathlineto{\pgfqpoint{4.787631in}{1.674764in}}%
\pgfpathlineto{\pgfqpoint{4.802071in}{1.677130in}}%
\pgfpathlineto{\pgfqpoint{4.816523in}{1.679592in}}%
\pgfpathlineto{\pgfqpoint{4.830987in}{1.682150in}}%
\pgfpathlineto{\pgfqpoint{4.822997in}{1.668048in}}%
\pgfpathlineto{\pgfqpoint{4.815004in}{1.654013in}}%
\pgfpathlineto{\pgfqpoint{4.807007in}{1.640048in}}%
\pgfpathlineto{\pgfqpoint{4.799007in}{1.626158in}}%
\pgfpathlineto{\pgfqpoint{4.784545in}{1.624041in}}%
\pgfpathlineto{\pgfqpoint{4.770094in}{1.622021in}}%
\pgfpathlineto{\pgfqpoint{4.755655in}{1.620095in}}%
\pgfpathlineto{\pgfqpoint{4.741226in}{1.618266in}}%
\pgfpathlineto{\pgfqpoint{4.749226in}{1.631708in}}%
\pgfpathlineto{\pgfqpoint{4.757221in}{1.645230in}}%
\pgfpathlineto{\pgfqpoint{4.765214in}{1.658827in}}%
\pgfpathlineto{\pgfqpoint{4.773202in}{1.672494in}}%
\pgfpathclose%
\pgfusepath{fill}%
\end{pgfscope}%
\begin{pgfscope}%
\pgfpathrectangle{\pgfqpoint{1.150000in}{0.150000in}}{\pgfqpoint{5.700000in}{5.700000in}}%
\pgfusepath{clip}%
\pgfsetbuttcap%
\pgfsetroundjoin%
\definecolor{currentfill}{rgb}{0.212395,0.359683,0.551710}%
\pgfsetfillcolor{currentfill}%
\pgfsetfillopacity{0.700000}%
\pgfsetlinewidth{0.000000pt}%
\definecolor{currentstroke}{rgb}{0.000000,0.000000,0.000000}%
\pgfsetstrokecolor{currentstroke}%
\pgfsetdash{}{0pt}%
\pgfpathmoveto{\pgfqpoint{2.939377in}{2.113340in}}%
\pgfpathlineto{\pgfqpoint{2.953506in}{2.098882in}}%
\pgfpathlineto{\pgfqpoint{2.967634in}{2.084551in}}%
\pgfpathlineto{\pgfqpoint{2.981760in}{2.070346in}}%
\pgfpathlineto{\pgfqpoint{2.995884in}{2.056267in}}%
\pgfpathlineto{\pgfqpoint{2.986902in}{2.066352in}}%
\pgfpathlineto{\pgfqpoint{2.977897in}{2.076877in}}%
\pgfpathlineto{\pgfqpoint{2.968870in}{2.087848in}}%
\pgfpathlineto{\pgfqpoint{2.959820in}{2.099273in}}%
\pgfpathlineto{\pgfqpoint{2.945641in}{2.114025in}}%
\pgfpathlineto{\pgfqpoint{2.931460in}{2.128904in}}%
\pgfpathlineto{\pgfqpoint{2.917278in}{2.143909in}}%
\pgfpathlineto{\pgfqpoint{2.903093in}{2.159042in}}%
\pgfpathlineto{\pgfqpoint{2.912199in}{2.146933in}}%
\pgfpathlineto{\pgfqpoint{2.921282in}{2.135285in}}%
\pgfpathlineto{\pgfqpoint{2.930341in}{2.124090in}}%
\pgfpathlineto{\pgfqpoint{2.939377in}{2.113340in}}%
\pgfpathclose%
\pgfusepath{fill}%
\end{pgfscope}%
\begin{pgfscope}%
\pgfpathrectangle{\pgfqpoint{1.150000in}{0.150000in}}{\pgfqpoint{5.700000in}{5.700000in}}%
\pgfusepath{clip}%
\pgfsetbuttcap%
\pgfsetroundjoin%
\definecolor{currentfill}{rgb}{0.141935,0.526453,0.555991}%
\pgfsetfillcolor{currentfill}%
\pgfsetfillopacity{0.700000}%
\pgfsetlinewidth{0.000000pt}%
\definecolor{currentstroke}{rgb}{0.000000,0.000000,0.000000}%
\pgfsetstrokecolor{currentstroke}%
\pgfsetdash{}{0pt}%
\pgfpathmoveto{\pgfqpoint{2.542578in}{2.572873in}}%
\pgfpathlineto{\pgfqpoint{2.556806in}{2.554526in}}%
\pgfpathlineto{\pgfqpoint{2.571028in}{2.536332in}}%
\pgfpathlineto{\pgfqpoint{2.585245in}{2.518288in}}%
\pgfpathlineto{\pgfqpoint{2.599456in}{2.500395in}}%
\pgfpathlineto{\pgfqpoint{2.590061in}{2.515205in}}%
\pgfpathlineto{\pgfqpoint{2.580638in}{2.530499in}}%
\pgfpathlineto{\pgfqpoint{2.571185in}{2.546286in}}%
\pgfpathlineto{\pgfqpoint{2.561703in}{2.562574in}}%
\pgfpathlineto{\pgfqpoint{2.547425in}{2.581167in}}%
\pgfpathlineto{\pgfqpoint{2.533140in}{2.599911in}}%
\pgfpathlineto{\pgfqpoint{2.518850in}{2.618807in}}%
\pgfpathlineto{\pgfqpoint{2.504555in}{2.637857in}}%
\pgfpathlineto{\pgfqpoint{2.514106in}{2.620857in}}%
\pgfpathlineto{\pgfqpoint{2.523626in}{2.604366in}}%
\pgfpathlineto{\pgfqpoint{2.533117in}{2.588373in}}%
\pgfpathlineto{\pgfqpoint{2.542578in}{2.572873in}}%
\pgfpathclose%
\pgfusepath{fill}%
\end{pgfscope}%
\begin{pgfscope}%
\pgfpathrectangle{\pgfqpoint{1.150000in}{0.150000in}}{\pgfqpoint{5.700000in}{5.700000in}}%
\pgfusepath{clip}%
\pgfsetbuttcap%
\pgfsetroundjoin%
\definecolor{currentfill}{rgb}{0.216210,0.351535,0.550627}%
\pgfsetfillcolor{currentfill}%
\pgfsetfillopacity{0.700000}%
\pgfsetlinewidth{0.000000pt}%
\definecolor{currentstroke}{rgb}{0.000000,0.000000,0.000000}%
\pgfsetstrokecolor{currentstroke}%
\pgfsetdash{}{0pt}%
\pgfpathmoveto{\pgfqpoint{5.196486in}{2.096012in}}%
\pgfpathlineto{\pgfqpoint{5.211148in}{2.102114in}}%
\pgfpathlineto{\pgfqpoint{5.225824in}{2.108314in}}%
\pgfpathlineto{\pgfqpoint{5.240515in}{2.114612in}}%
\pgfpathlineto{\pgfqpoint{5.255221in}{2.121010in}}%
\pgfpathlineto{\pgfqpoint{5.247302in}{2.105132in}}%
\pgfpathlineto{\pgfqpoint{5.239379in}{2.089216in}}%
\pgfpathlineto{\pgfqpoint{5.231452in}{2.073265in}}%
\pgfpathlineto{\pgfqpoint{5.223520in}{2.057283in}}%
\pgfpathlineto{\pgfqpoint{5.208821in}{2.051223in}}%
\pgfpathlineto{\pgfqpoint{5.194137in}{2.045262in}}%
\pgfpathlineto{\pgfqpoint{5.179467in}{2.039399in}}%
\pgfpathlineto{\pgfqpoint{5.164811in}{2.033635in}}%
\pgfpathlineto{\pgfqpoint{5.172737in}{2.049273in}}%
\pgfpathlineto{\pgfqpoint{5.180658in}{2.064884in}}%
\pgfpathlineto{\pgfqpoint{5.188574in}{2.080465in}}%
\pgfpathlineto{\pgfqpoint{5.196486in}{2.096012in}}%
\pgfpathclose%
\pgfusepath{fill}%
\end{pgfscope}%
\begin{pgfscope}%
\pgfpathrectangle{\pgfqpoint{1.150000in}{0.150000in}}{\pgfqpoint{5.700000in}{5.700000in}}%
\pgfusepath{clip}%
\pgfsetbuttcap%
\pgfsetroundjoin%
\definecolor{currentfill}{rgb}{0.277018,0.050344,0.375715}%
\pgfsetfillcolor{currentfill}%
\pgfsetfillopacity{0.700000}%
\pgfsetlinewidth{0.000000pt}%
\definecolor{currentstroke}{rgb}{0.000000,0.000000,0.000000}%
\pgfsetstrokecolor{currentstroke}%
\pgfsetdash{}{0pt}%
\pgfpathmoveto{\pgfqpoint{4.326096in}{1.439674in}}%
\pgfpathlineto{\pgfqpoint{4.340345in}{1.437663in}}%
\pgfpathlineto{\pgfqpoint{4.354602in}{1.435748in}}%
\pgfpathlineto{\pgfqpoint{4.368868in}{1.433929in}}%
\pgfpathlineto{\pgfqpoint{4.383142in}{1.432206in}}%
\pgfpathlineto{\pgfqpoint{4.375059in}{1.422883in}}%
\pgfpathlineto{\pgfqpoint{4.366970in}{1.413741in}}%
\pgfpathlineto{\pgfqpoint{4.358876in}{1.404786in}}%
\pgfpathlineto{\pgfqpoint{4.350778in}{1.396023in}}%
\pgfpathlineto{\pgfqpoint{4.336494in}{1.398273in}}%
\pgfpathlineto{\pgfqpoint{4.322219in}{1.400618in}}%
\pgfpathlineto{\pgfqpoint{4.307951in}{1.403059in}}%
\pgfpathlineto{\pgfqpoint{4.293692in}{1.405596in}}%
\pgfpathlineto{\pgfqpoint{4.301801in}{1.413826in}}%
\pgfpathlineto{\pgfqpoint{4.309905in}{1.422253in}}%
\pgfpathlineto{\pgfqpoint{4.318003in}{1.430870in}}%
\pgfpathlineto{\pgfqpoint{4.326096in}{1.439674in}}%
\pgfpathclose%
\pgfusepath{fill}%
\end{pgfscope}%
\begin{pgfscope}%
\pgfpathrectangle{\pgfqpoint{1.150000in}{0.150000in}}{\pgfqpoint{5.700000in}{5.700000in}}%
\pgfusepath{clip}%
\pgfsetbuttcap%
\pgfsetroundjoin%
\definecolor{currentfill}{rgb}{0.221989,0.339161,0.548752}%
\pgfsetfillcolor{currentfill}%
\pgfsetfillopacity{0.700000}%
\pgfsetlinewidth{0.000000pt}%
\definecolor{currentstroke}{rgb}{0.000000,0.000000,0.000000}%
\pgfsetstrokecolor{currentstroke}%
\pgfsetdash{}{0pt}%
\pgfpathmoveto{\pgfqpoint{2.995884in}{2.056267in}}%
\pgfpathlineto{\pgfqpoint{3.010007in}{2.042311in}}%
\pgfpathlineto{\pgfqpoint{3.024129in}{2.028480in}}%
\pgfpathlineto{\pgfqpoint{3.038250in}{2.014772in}}%
\pgfpathlineto{\pgfqpoint{3.052370in}{2.001187in}}%
\pgfpathlineto{\pgfqpoint{3.043439in}{2.010611in}}%
\pgfpathlineto{\pgfqpoint{3.034487in}{2.020468in}}%
\pgfpathlineto{\pgfqpoint{3.025514in}{2.030765in}}%
\pgfpathlineto{\pgfqpoint{3.016519in}{2.041510in}}%
\pgfpathlineto{\pgfqpoint{3.002346in}{2.055765in}}%
\pgfpathlineto{\pgfqpoint{2.988172in}{2.070143in}}%
\pgfpathlineto{\pgfqpoint{2.973997in}{2.084646in}}%
\pgfpathlineto{\pgfqpoint{2.959820in}{2.099273in}}%
\pgfpathlineto{\pgfqpoint{2.968870in}{2.087848in}}%
\pgfpathlineto{\pgfqpoint{2.977897in}{2.076877in}}%
\pgfpathlineto{\pgfqpoint{2.986902in}{2.066352in}}%
\pgfpathlineto{\pgfqpoint{2.995884in}{2.056267in}}%
\pgfpathclose%
\pgfusepath{fill}%
\end{pgfscope}%
\begin{pgfscope}%
\pgfpathrectangle{\pgfqpoint{1.150000in}{0.150000in}}{\pgfqpoint{5.700000in}{5.700000in}}%
\pgfusepath{clip}%
\pgfsetbuttcap%
\pgfsetroundjoin%
\definecolor{currentfill}{rgb}{0.280868,0.160771,0.472899}%
\pgfsetfillcolor{currentfill}%
\pgfsetfillopacity{0.700000}%
\pgfsetlinewidth{0.000000pt}%
\definecolor{currentstroke}{rgb}{0.000000,0.000000,0.000000}%
\pgfsetstrokecolor{currentstroke}%
\pgfsetdash{}{0pt}%
\pgfpathmoveto{\pgfqpoint{3.482049in}{1.652544in}}%
\pgfpathlineto{\pgfqpoint{3.496148in}{1.642903in}}%
\pgfpathlineto{\pgfqpoint{3.510249in}{1.633370in}}%
\pgfpathlineto{\pgfqpoint{3.524353in}{1.623942in}}%
\pgfpathlineto{\pgfqpoint{3.538459in}{1.614622in}}%
\pgfpathlineto{\pgfqpoint{3.529944in}{1.617367in}}%
\pgfpathlineto{\pgfqpoint{3.521414in}{1.620471in}}%
\pgfpathlineto{\pgfqpoint{3.512870in}{1.623940in}}%
\pgfpathlineto{\pgfqpoint{3.504312in}{1.627781in}}%
\pgfpathlineto{\pgfqpoint{3.490169in}{1.637731in}}%
\pgfpathlineto{\pgfqpoint{3.476028in}{1.647788in}}%
\pgfpathlineto{\pgfqpoint{3.461889in}{1.657952in}}%
\pgfpathlineto{\pgfqpoint{3.447752in}{1.668223in}}%
\pgfpathlineto{\pgfqpoint{3.456349in}{1.663744in}}%
\pgfpathlineto{\pgfqpoint{3.464931in}{1.659642in}}%
\pgfpathlineto{\pgfqpoint{3.473497in}{1.655911in}}%
\pgfpathlineto{\pgfqpoint{3.482049in}{1.652544in}}%
\pgfpathclose%
\pgfusepath{fill}%
\end{pgfscope}%
\begin{pgfscope}%
\pgfpathrectangle{\pgfqpoint{1.150000in}{0.150000in}}{\pgfqpoint{5.700000in}{5.700000in}}%
\pgfusepath{clip}%
\pgfsetbuttcap%
\pgfsetroundjoin%
\definecolor{currentfill}{rgb}{0.131172,0.555899,0.552459}%
\pgfsetfillcolor{currentfill}%
\pgfsetfillopacity{0.700000}%
\pgfsetlinewidth{0.000000pt}%
\definecolor{currentstroke}{rgb}{0.000000,0.000000,0.000000}%
\pgfsetstrokecolor{currentstroke}%
\pgfsetdash{}{0pt}%
\pgfpathmoveto{\pgfqpoint{2.485612in}{2.647803in}}%
\pgfpathlineto{\pgfqpoint{2.499863in}{2.628836in}}%
\pgfpathlineto{\pgfqpoint{2.514107in}{2.610027in}}%
\pgfpathlineto{\pgfqpoint{2.528345in}{2.591372in}}%
\pgfpathlineto{\pgfqpoint{2.542578in}{2.572873in}}%
\pgfpathlineto{\pgfqpoint{2.533117in}{2.588373in}}%
\pgfpathlineto{\pgfqpoint{2.523626in}{2.604366in}}%
\pgfpathlineto{\pgfqpoint{2.514106in}{2.620857in}}%
\pgfpathlineto{\pgfqpoint{2.504555in}{2.637857in}}%
\pgfpathlineto{\pgfqpoint{2.490253in}{2.657061in}}%
\pgfpathlineto{\pgfqpoint{2.475945in}{2.676421in}}%
\pgfpathlineto{\pgfqpoint{2.461631in}{2.695937in}}%
\pgfpathlineto{\pgfqpoint{2.447310in}{2.715612in}}%
\pgfpathlineto{\pgfqpoint{2.456932in}{2.697896in}}%
\pgfpathlineto{\pgfqpoint{2.466523in}{2.680694in}}%
\pgfpathlineto{\pgfqpoint{2.476083in}{2.663999in}}%
\pgfpathlineto{\pgfqpoint{2.485612in}{2.647803in}}%
\pgfpathclose%
\pgfusepath{fill}%
\end{pgfscope}%
\begin{pgfscope}%
\pgfpathrectangle{\pgfqpoint{1.150000in}{0.150000in}}{\pgfqpoint{5.700000in}{5.700000in}}%
\pgfusepath{clip}%
\pgfsetbuttcap%
\pgfsetroundjoin%
\definecolor{currentfill}{rgb}{0.282327,0.094955,0.417331}%
\pgfsetfillcolor{currentfill}%
\pgfsetfillopacity{0.700000}%
\pgfsetlinewidth{0.000000pt}%
\definecolor{currentstroke}{rgb}{0.000000,0.000000,0.000000}%
\pgfsetstrokecolor{currentstroke}%
\pgfsetdash{}{0pt}%
\pgfpathmoveto{\pgfqpoint{3.741524in}{1.510665in}}%
\pgfpathlineto{\pgfqpoint{3.755644in}{1.503299in}}%
\pgfpathlineto{\pgfqpoint{3.769768in}{1.496034in}}%
\pgfpathlineto{\pgfqpoint{3.783897in}{1.488870in}}%
\pgfpathlineto{\pgfqpoint{3.798030in}{1.481808in}}%
\pgfpathlineto{\pgfqpoint{3.789688in}{1.480797in}}%
\pgfpathlineto{\pgfqpoint{3.781335in}{1.480098in}}%
\pgfpathlineto{\pgfqpoint{3.772972in}{1.479716in}}%
\pgfpathlineto{\pgfqpoint{3.764597in}{1.479657in}}%
\pgfpathlineto{\pgfqpoint{3.750436in}{1.487324in}}%
\pgfpathlineto{\pgfqpoint{3.736279in}{1.495091in}}%
\pgfpathlineto{\pgfqpoint{3.722126in}{1.502960in}}%
\pgfpathlineto{\pgfqpoint{3.707976in}{1.510931in}}%
\pgfpathlineto{\pgfqpoint{3.716381in}{1.510377in}}%
\pgfpathlineto{\pgfqpoint{3.724773in}{1.510153in}}%
\pgfpathlineto{\pgfqpoint{3.733154in}{1.510251in}}%
\pgfpathlineto{\pgfqpoint{3.741524in}{1.510665in}}%
\pgfpathclose%
\pgfusepath{fill}%
\end{pgfscope}%
\begin{pgfscope}%
\pgfpathrectangle{\pgfqpoint{1.150000in}{0.150000in}}{\pgfqpoint{5.700000in}{5.700000in}}%
\pgfusepath{clip}%
\pgfsetbuttcap%
\pgfsetroundjoin%
\definecolor{currentfill}{rgb}{0.274128,0.199721,0.498911}%
\pgfsetfillcolor{currentfill}%
\pgfsetfillopacity{0.700000}%
\pgfsetlinewidth{0.000000pt}%
\definecolor{currentstroke}{rgb}{0.000000,0.000000,0.000000}%
\pgfsetstrokecolor{currentstroke}%
\pgfsetdash{}{0pt}%
\pgfpathmoveto{\pgfqpoint{4.862913in}{1.739136in}}%
\pgfpathlineto{\pgfqpoint{4.877391in}{1.742213in}}%
\pgfpathlineto{\pgfqpoint{4.891881in}{1.745387in}}%
\pgfpathlineto{\pgfqpoint{4.906384in}{1.748657in}}%
\pgfpathlineto{\pgfqpoint{4.920899in}{1.752023in}}%
\pgfpathlineto{\pgfqpoint{4.912920in}{1.737280in}}%
\pgfpathlineto{\pgfqpoint{4.904937in}{1.722583in}}%
\pgfpathlineto{\pgfqpoint{4.896952in}{1.707934in}}%
\pgfpathlineto{\pgfqpoint{4.888963in}{1.693340in}}%
\pgfpathlineto{\pgfqpoint{4.874451in}{1.690399in}}%
\pgfpathlineto{\pgfqpoint{4.859951in}{1.687553in}}%
\pgfpathlineto{\pgfqpoint{4.845463in}{1.684804in}}%
\pgfpathlineto{\pgfqpoint{4.830987in}{1.682150in}}%
\pgfpathlineto{\pgfqpoint{4.838974in}{1.696313in}}%
\pgfpathlineto{\pgfqpoint{4.846957in}{1.710535in}}%
\pgfpathlineto{\pgfqpoint{4.854936in}{1.724810in}}%
\pgfpathlineto{\pgfqpoint{4.862913in}{1.739136in}}%
\pgfpathclose%
\pgfusepath{fill}%
\end{pgfscope}%
\begin{pgfscope}%
\pgfpathrectangle{\pgfqpoint{1.150000in}{0.150000in}}{\pgfqpoint{5.700000in}{5.700000in}}%
\pgfusepath{clip}%
\pgfsetbuttcap%
\pgfsetroundjoin%
\definecolor{currentfill}{rgb}{0.231674,0.318106,0.544834}%
\pgfsetfillcolor{currentfill}%
\pgfsetfillopacity{0.700000}%
\pgfsetlinewidth{0.000000pt}%
\definecolor{currentstroke}{rgb}{0.000000,0.000000,0.000000}%
\pgfsetstrokecolor{currentstroke}%
\pgfsetdash{}{0pt}%
\pgfpathmoveto{\pgfqpoint{3.052370in}{2.001187in}}%
\pgfpathlineto{\pgfqpoint{3.066489in}{1.987723in}}%
\pgfpathlineto{\pgfqpoint{3.080607in}{1.974381in}}%
\pgfpathlineto{\pgfqpoint{3.094724in}{1.961159in}}%
\pgfpathlineto{\pgfqpoint{3.108841in}{1.948058in}}%
\pgfpathlineto{\pgfqpoint{3.099961in}{1.956823in}}%
\pgfpathlineto{\pgfqpoint{3.091060in}{1.966016in}}%
\pgfpathlineto{\pgfqpoint{3.082139in}{1.975642in}}%
\pgfpathlineto{\pgfqpoint{3.073196in}{1.985710in}}%
\pgfpathlineto{\pgfqpoint{3.059028in}{1.999478in}}%
\pgfpathlineto{\pgfqpoint{3.044860in}{2.013367in}}%
\pgfpathlineto{\pgfqpoint{3.030690in}{2.027377in}}%
\pgfpathlineto{\pgfqpoint{3.016519in}{2.041510in}}%
\pgfpathlineto{\pgfqpoint{3.025514in}{2.030765in}}%
\pgfpathlineto{\pgfqpoint{3.034487in}{2.020468in}}%
\pgfpathlineto{\pgfqpoint{3.043439in}{2.010611in}}%
\pgfpathlineto{\pgfqpoint{3.052370in}{2.001187in}}%
\pgfpathclose%
\pgfusepath{fill}%
\end{pgfscope}%
\begin{pgfscope}%
\pgfpathrectangle{\pgfqpoint{1.150000in}{0.150000in}}{\pgfqpoint{5.700000in}{5.700000in}}%
\pgfusepath{clip}%
\pgfsetbuttcap%
\pgfsetroundjoin%
\definecolor{currentfill}{rgb}{0.377779,0.791781,0.377939}%
\pgfsetfillcolor{currentfill}%
\pgfsetfillopacity{0.700000}%
\pgfsetlinewidth{0.000000pt}%
\definecolor{currentstroke}{rgb}{0.000000,0.000000,0.000000}%
\pgfsetstrokecolor{currentstroke}%
\pgfsetdash{}{0pt}%
\pgfpathmoveto{\pgfqpoint{2.007684in}{3.361642in}}%
\pgfpathlineto{\pgfqpoint{2.022186in}{3.336770in}}%
\pgfpathlineto{\pgfqpoint{2.036675in}{3.312112in}}%
\pgfpathlineto{\pgfqpoint{2.051152in}{3.287666in}}%
\pgfpathlineto{\pgfqpoint{2.065617in}{3.263430in}}%
\pgfpathlineto{\pgfqpoint{2.055621in}{3.283248in}}%
\pgfpathlineto{\pgfqpoint{2.045588in}{3.303591in}}%
\pgfpathlineto{\pgfqpoint{2.035518in}{3.324468in}}%
\pgfpathlineto{\pgfqpoint{2.025410in}{3.345886in}}%
\pgfpathlineto{\pgfqpoint{2.010862in}{3.370851in}}%
\pgfpathlineto{\pgfqpoint{1.996301in}{3.396029in}}%
\pgfpathlineto{\pgfqpoint{1.981728in}{3.421421in}}%
\pgfpathlineto{\pgfqpoint{1.967141in}{3.447029in}}%
\pgfpathlineto{\pgfqpoint{1.977335in}{3.424866in}}%
\pgfpathlineto{\pgfqpoint{1.987489in}{3.403253in}}%
\pgfpathlineto{\pgfqpoint{1.997605in}{3.382181in}}%
\pgfpathlineto{\pgfqpoint{2.007684in}{3.361642in}}%
\pgfpathclose%
\pgfusepath{fill}%
\end{pgfscope}%
\begin{pgfscope}%
\pgfpathrectangle{\pgfqpoint{1.150000in}{0.150000in}}{\pgfqpoint{5.700000in}{5.700000in}}%
\pgfusepath{clip}%
\pgfsetbuttcap%
\pgfsetroundjoin%
\definecolor{currentfill}{rgb}{0.276022,0.044167,0.370164}%
\pgfsetfillcolor{currentfill}%
\pgfsetfillopacity{0.700000}%
\pgfsetlinewidth{0.000000pt}%
\definecolor{currentstroke}{rgb}{0.000000,0.000000,0.000000}%
\pgfsetstrokecolor{currentstroke}%
\pgfsetdash{}{0pt}%
\pgfpathmoveto{\pgfqpoint{4.090518in}{1.419210in}}%
\pgfpathlineto{\pgfqpoint{4.104703in}{1.414974in}}%
\pgfpathlineto{\pgfqpoint{4.118894in}{1.410834in}}%
\pgfpathlineto{\pgfqpoint{4.133093in}{1.406792in}}%
\pgfpathlineto{\pgfqpoint{4.147298in}{1.402846in}}%
\pgfpathlineto{\pgfqpoint{4.139131in}{1.396790in}}%
\pgfpathlineto{\pgfqpoint{4.130957in}{1.390973in}}%
\pgfpathlineto{\pgfqpoint{4.122777in}{1.385400in}}%
\pgfpathlineto{\pgfqpoint{4.114589in}{1.380077in}}%
\pgfpathlineto{\pgfqpoint{4.100367in}{1.384586in}}%
\pgfpathlineto{\pgfqpoint{4.086152in}{1.389191in}}%
\pgfpathlineto{\pgfqpoint{4.071942in}{1.393893in}}%
\pgfpathlineto{\pgfqpoint{4.057740in}{1.398692in}}%
\pgfpathlineto{\pgfqpoint{4.065945in}{1.403445in}}%
\pgfpathlineto{\pgfqpoint{4.074144in}{1.408453in}}%
\pgfpathlineto{\pgfqpoint{4.082334in}{1.413710in}}%
\pgfpathlineto{\pgfqpoint{4.090518in}{1.419210in}}%
\pgfpathclose%
\pgfusepath{fill}%
\end{pgfscope}%
\begin{pgfscope}%
\pgfpathrectangle{\pgfqpoint{1.150000in}{0.150000in}}{\pgfqpoint{5.700000in}{5.700000in}}%
\pgfusepath{clip}%
\pgfsetbuttcap%
\pgfsetroundjoin%
\definecolor{currentfill}{rgb}{0.243113,0.292092,0.538516}%
\pgfsetfillcolor{currentfill}%
\pgfsetfillopacity{0.700000}%
\pgfsetlinewidth{0.000000pt}%
\definecolor{currentstroke}{rgb}{0.000000,0.000000,0.000000}%
\pgfsetstrokecolor{currentstroke}%
\pgfsetdash{}{0pt}%
\pgfpathmoveto{\pgfqpoint{5.074611in}{1.950210in}}%
\pgfpathlineto{\pgfqpoint{5.089204in}{1.955229in}}%
\pgfpathlineto{\pgfqpoint{5.103812in}{1.960346in}}%
\pgfpathlineto{\pgfqpoint{5.118433in}{1.965560in}}%
\pgfpathlineto{\pgfqpoint{5.133069in}{1.970872in}}%
\pgfpathlineto{\pgfqpoint{5.125123in}{1.955145in}}%
\pgfpathlineto{\pgfqpoint{5.117174in}{1.939411in}}%
\pgfpathlineto{\pgfqpoint{5.109221in}{1.923672in}}%
\pgfpathlineto{\pgfqpoint{5.101264in}{1.907932in}}%
\pgfpathlineto{\pgfqpoint{5.086634in}{1.902994in}}%
\pgfpathlineto{\pgfqpoint{5.072018in}{1.898153in}}%
\pgfpathlineto{\pgfqpoint{5.057416in}{1.893409in}}%
\pgfpathlineto{\pgfqpoint{5.042827in}{1.888762in}}%
\pgfpathlineto{\pgfqpoint{5.050779in}{1.904122in}}%
\pgfpathlineto{\pgfqpoint{5.058727in}{1.919485in}}%
\pgfpathlineto{\pgfqpoint{5.066671in}{1.934849in}}%
\pgfpathlineto{\pgfqpoint{5.074611in}{1.950210in}}%
\pgfpathclose%
\pgfusepath{fill}%
\end{pgfscope}%
\begin{pgfscope}%
\pgfpathrectangle{\pgfqpoint{1.150000in}{0.150000in}}{\pgfqpoint{5.700000in}{5.700000in}}%
\pgfusepath{clip}%
\pgfsetbuttcap%
\pgfsetroundjoin%
\definecolor{currentfill}{rgb}{0.122606,0.585371,0.546557}%
\pgfsetfillcolor{currentfill}%
\pgfsetfillopacity{0.700000}%
\pgfsetlinewidth{0.000000pt}%
\definecolor{currentstroke}{rgb}{0.000000,0.000000,0.000000}%
\pgfsetstrokecolor{currentstroke}%
\pgfsetdash{}{0pt}%
\pgfpathmoveto{\pgfqpoint{2.428548in}{2.725262in}}%
\pgfpathlineto{\pgfqpoint{2.442824in}{2.705656in}}%
\pgfpathlineto{\pgfqpoint{2.457093in}{2.686212in}}%
\pgfpathlineto{\pgfqpoint{2.471356in}{2.666928in}}%
\pgfpathlineto{\pgfqpoint{2.485612in}{2.647803in}}%
\pgfpathlineto{\pgfqpoint{2.476083in}{2.663999in}}%
\pgfpathlineto{\pgfqpoint{2.466523in}{2.680694in}}%
\pgfpathlineto{\pgfqpoint{2.456932in}{2.697896in}}%
\pgfpathlineto{\pgfqpoint{2.447310in}{2.715612in}}%
\pgfpathlineto{\pgfqpoint{2.432983in}{2.735446in}}%
\pgfpathlineto{\pgfqpoint{2.418649in}{2.755441in}}%
\pgfpathlineto{\pgfqpoint{2.404308in}{2.775597in}}%
\pgfpathlineto{\pgfqpoint{2.389960in}{2.795917in}}%
\pgfpathlineto{\pgfqpoint{2.399655in}{2.777479in}}%
\pgfpathlineto{\pgfqpoint{2.409318in}{2.759562in}}%
\pgfpathlineto{\pgfqpoint{2.418948in}{2.742159in}}%
\pgfpathlineto{\pgfqpoint{2.428548in}{2.725262in}}%
\pgfpathclose%
\pgfusepath{fill}%
\end{pgfscope}%
\begin{pgfscope}%
\pgfpathrectangle{\pgfqpoint{1.150000in}{0.150000in}}{\pgfqpoint{5.700000in}{5.700000in}}%
\pgfusepath{clip}%
\pgfsetbuttcap%
\pgfsetroundjoin%
\definecolor{currentfill}{rgb}{0.277941,0.056324,0.381191}%
\pgfsetfillcolor{currentfill}%
\pgfsetfillopacity{0.700000}%
\pgfsetlinewidth{0.000000pt}%
\definecolor{currentstroke}{rgb}{0.000000,0.000000,0.000000}%
\pgfsetstrokecolor{currentstroke}%
\pgfsetdash{}{0pt}%
\pgfpathmoveto{\pgfqpoint{3.944338in}{1.440602in}}%
\pgfpathlineto{\pgfqpoint{3.958493in}{1.435019in}}%
\pgfpathlineto{\pgfqpoint{3.972653in}{1.429536in}}%
\pgfpathlineto{\pgfqpoint{3.986819in}{1.424150in}}%
\pgfpathlineto{\pgfqpoint{4.000991in}{1.418863in}}%
\pgfpathlineto{\pgfqpoint{3.992758in}{1.414946in}}%
\pgfpathlineto{\pgfqpoint{3.984516in}{1.411299in}}%
\pgfpathlineto{\pgfqpoint{3.976266in}{1.407930in}}%
\pgfpathlineto{\pgfqpoint{3.968008in}{1.404845in}}%
\pgfpathlineto{\pgfqpoint{3.953814in}{1.410715in}}%
\pgfpathlineto{\pgfqpoint{3.939626in}{1.416682in}}%
\pgfpathlineto{\pgfqpoint{3.925443in}{1.422748in}}%
\pgfpathlineto{\pgfqpoint{3.911266in}{1.428913in}}%
\pgfpathlineto{\pgfqpoint{3.919548in}{1.431409in}}%
\pgfpathlineto{\pgfqpoint{3.927820in}{1.434193in}}%
\pgfpathlineto{\pgfqpoint{3.936083in}{1.437259in}}%
\pgfpathlineto{\pgfqpoint{3.944338in}{1.440602in}}%
\pgfpathclose%
\pgfusepath{fill}%
\end{pgfscope}%
\begin{pgfscope}%
\pgfpathrectangle{\pgfqpoint{1.150000in}{0.150000in}}{\pgfqpoint{5.700000in}{5.700000in}}%
\pgfusepath{clip}%
\pgfsetbuttcap%
\pgfsetroundjoin%
\definecolor{currentfill}{rgb}{0.175841,0.441290,0.557685}%
\pgfsetfillcolor{currentfill}%
\pgfsetfillopacity{0.700000}%
\pgfsetlinewidth{0.000000pt}%
\definecolor{currentstroke}{rgb}{0.000000,0.000000,0.000000}%
\pgfsetstrokecolor{currentstroke}%
\pgfsetdash{}{0pt}%
\pgfpathmoveto{\pgfqpoint{5.408916in}{2.337707in}}%
\pgfpathlineto{\pgfqpoint{5.423718in}{2.345504in}}%
\pgfpathlineto{\pgfqpoint{5.438537in}{2.353401in}}%
\pgfpathlineto{\pgfqpoint{5.453371in}{2.361398in}}%
\pgfpathlineto{\pgfqpoint{5.445502in}{2.345718in}}%
\pgfpathlineto{\pgfqpoint{5.437627in}{2.329957in}}%
\pgfpathlineto{\pgfqpoint{5.429746in}{2.314120in}}%
\pgfpathlineto{\pgfqpoint{5.421860in}{2.298208in}}%
\pgfpathlineto{\pgfqpoint{5.407032in}{2.290495in}}%
\pgfpathlineto{\pgfqpoint{5.392221in}{2.282881in}}%
\pgfpathlineto{\pgfqpoint{5.377425in}{2.275367in}}%
\pgfpathlineto{\pgfqpoint{5.385306in}{2.291061in}}%
\pgfpathlineto{\pgfqpoint{5.393181in}{2.306684in}}%
\pgfpathlineto{\pgfqpoint{5.401051in}{2.322234in}}%
\pgfpathlineto{\pgfqpoint{5.408916in}{2.337707in}}%
\pgfpathclose%
\pgfusepath{fill}%
\end{pgfscope}%
\begin{pgfscope}%
\pgfpathrectangle{\pgfqpoint{1.150000in}{0.150000in}}{\pgfqpoint{5.700000in}{5.700000in}}%
\pgfusepath{clip}%
\pgfsetbuttcap%
\pgfsetroundjoin%
\definecolor{currentfill}{rgb}{0.239346,0.300855,0.540844}%
\pgfsetfillcolor{currentfill}%
\pgfsetfillopacity{0.700000}%
\pgfsetlinewidth{0.000000pt}%
\definecolor{currentstroke}{rgb}{0.000000,0.000000,0.000000}%
\pgfsetstrokecolor{currentstroke}%
\pgfsetdash{}{0pt}%
\pgfpathmoveto{\pgfqpoint{3.108841in}{1.948058in}}%
\pgfpathlineto{\pgfqpoint{3.122957in}{1.935076in}}%
\pgfpathlineto{\pgfqpoint{3.137074in}{1.922213in}}%
\pgfpathlineto{\pgfqpoint{3.151189in}{1.909468in}}%
\pgfpathlineto{\pgfqpoint{3.165305in}{1.896841in}}%
\pgfpathlineto{\pgfqpoint{3.156474in}{1.904951in}}%
\pgfpathlineto{\pgfqpoint{3.147623in}{1.913481in}}%
\pgfpathlineto{\pgfqpoint{3.138752in}{1.922439in}}%
\pgfpathlineto{\pgfqpoint{3.129860in}{1.931832in}}%
\pgfpathlineto{\pgfqpoint{3.115695in}{1.945123in}}%
\pgfpathlineto{\pgfqpoint{3.101530in}{1.958533in}}%
\pgfpathlineto{\pgfqpoint{3.087363in}{1.972062in}}%
\pgfpathlineto{\pgfqpoint{3.073196in}{1.985710in}}%
\pgfpathlineto{\pgfqpoint{3.082139in}{1.975642in}}%
\pgfpathlineto{\pgfqpoint{3.091060in}{1.966016in}}%
\pgfpathlineto{\pgfqpoint{3.099961in}{1.956823in}}%
\pgfpathlineto{\pgfqpoint{3.108841in}{1.948058in}}%
\pgfpathclose%
\pgfusepath{fill}%
\end{pgfscope}%
\begin{pgfscope}%
\pgfpathrectangle{\pgfqpoint{1.150000in}{0.150000in}}{\pgfqpoint{5.700000in}{5.700000in}}%
\pgfusepath{clip}%
\pgfsetbuttcap%
\pgfsetroundjoin%
\definecolor{currentfill}{rgb}{0.276022,0.044167,0.370164}%
\pgfsetfillcolor{currentfill}%
\pgfsetfillopacity{0.700000}%
\pgfsetlinewidth{0.000000pt}%
\definecolor{currentstroke}{rgb}{0.000000,0.000000,0.000000}%
\pgfsetstrokecolor{currentstroke}%
\pgfsetdash{}{0pt}%
\pgfpathmoveto{\pgfqpoint{4.236734in}{1.416700in}}%
\pgfpathlineto{\pgfqpoint{4.250962in}{1.413780in}}%
\pgfpathlineto{\pgfqpoint{4.265197in}{1.410956in}}%
\pgfpathlineto{\pgfqpoint{4.279441in}{1.408228in}}%
\pgfpathlineto{\pgfqpoint{4.293692in}{1.405596in}}%
\pgfpathlineto{\pgfqpoint{4.285577in}{1.397568in}}%
\pgfpathlineto{\pgfqpoint{4.277457in}{1.389746in}}%
\pgfpathlineto{\pgfqpoint{4.269331in}{1.382137in}}%
\pgfpathlineto{\pgfqpoint{4.261199in}{1.374746in}}%
\pgfpathlineto{\pgfqpoint{4.246936in}{1.377923in}}%
\pgfpathlineto{\pgfqpoint{4.232680in}{1.381195in}}%
\pgfpathlineto{\pgfqpoint{4.218431in}{1.384564in}}%
\pgfpathlineto{\pgfqpoint{4.204190in}{1.388028in}}%
\pgfpathlineto{\pgfqpoint{4.212335in}{1.394867in}}%
\pgfpathlineto{\pgfqpoint{4.220474in}{1.401930in}}%
\pgfpathlineto{\pgfqpoint{4.228607in}{1.409209in}}%
\pgfpathlineto{\pgfqpoint{4.236734in}{1.416700in}}%
\pgfpathclose%
\pgfusepath{fill}%
\end{pgfscope}%
\begin{pgfscope}%
\pgfpathrectangle{\pgfqpoint{1.150000in}{0.150000in}}{\pgfqpoint{5.700000in}{5.700000in}}%
\pgfusepath{clip}%
\pgfsetbuttcap%
\pgfsetroundjoin%
\definecolor{currentfill}{rgb}{0.281887,0.150881,0.465405}%
\pgfsetfillcolor{currentfill}%
\pgfsetfillopacity{0.700000}%
\pgfsetlinewidth{0.000000pt}%
\definecolor{currentstroke}{rgb}{0.000000,0.000000,0.000000}%
\pgfsetstrokecolor{currentstroke}%
\pgfsetdash{}{0pt}%
\pgfpathmoveto{\pgfqpoint{3.538459in}{1.614622in}}%
\pgfpathlineto{\pgfqpoint{3.552568in}{1.605406in}}%
\pgfpathlineto{\pgfqpoint{3.566680in}{1.596297in}}%
\pgfpathlineto{\pgfqpoint{3.580795in}{1.587293in}}%
\pgfpathlineto{\pgfqpoint{3.594913in}{1.578393in}}%
\pgfpathlineto{\pgfqpoint{3.586432in}{1.580519in}}%
\pgfpathlineto{\pgfqpoint{3.577938in}{1.582997in}}%
\pgfpathlineto{\pgfqpoint{3.569430in}{1.585835in}}%
\pgfpathlineto{\pgfqpoint{3.560909in}{1.589039in}}%
\pgfpathlineto{\pgfqpoint{3.546756in}{1.598566in}}%
\pgfpathlineto{\pgfqpoint{3.532605in}{1.608199in}}%
\pgfpathlineto{\pgfqpoint{3.518457in}{1.617937in}}%
\pgfpathlineto{\pgfqpoint{3.504312in}{1.627781in}}%
\pgfpathlineto{\pgfqpoint{3.512870in}{1.623940in}}%
\pgfpathlineto{\pgfqpoint{3.521414in}{1.620471in}}%
\pgfpathlineto{\pgfqpoint{3.529944in}{1.617367in}}%
\pgfpathlineto{\pgfqpoint{3.538459in}{1.614622in}}%
\pgfpathclose%
\pgfusepath{fill}%
\end{pgfscope}%
\begin{pgfscope}%
\pgfpathrectangle{\pgfqpoint{1.150000in}{0.150000in}}{\pgfqpoint{5.700000in}{5.700000in}}%
\pgfusepath{clip}%
\pgfsetbuttcap%
\pgfsetroundjoin%
\definecolor{currentfill}{rgb}{0.119483,0.614817,0.537692}%
\pgfsetfillcolor{currentfill}%
\pgfsetfillopacity{0.700000}%
\pgfsetlinewidth{0.000000pt}%
\definecolor{currentstroke}{rgb}{0.000000,0.000000,0.000000}%
\pgfsetstrokecolor{currentstroke}%
\pgfsetdash{}{0pt}%
\pgfpathmoveto{\pgfqpoint{2.371375in}{2.805333in}}%
\pgfpathlineto{\pgfqpoint{2.385679in}{2.785066in}}%
\pgfpathlineto{\pgfqpoint{2.399975in}{2.764966in}}%
\pgfpathlineto{\pgfqpoint{2.414265in}{2.745032in}}%
\pgfpathlineto{\pgfqpoint{2.428548in}{2.725262in}}%
\pgfpathlineto{\pgfqpoint{2.418948in}{2.742159in}}%
\pgfpathlineto{\pgfqpoint{2.409318in}{2.759562in}}%
\pgfpathlineto{\pgfqpoint{2.399655in}{2.777479in}}%
\pgfpathlineto{\pgfqpoint{2.389960in}{2.795917in}}%
\pgfpathlineto{\pgfqpoint{2.375604in}{2.816401in}}%
\pgfpathlineto{\pgfqpoint{2.361241in}{2.837052in}}%
\pgfpathlineto{\pgfqpoint{2.346871in}{2.857869in}}%
\pgfpathlineto{\pgfqpoint{2.332493in}{2.878855in}}%
\pgfpathlineto{\pgfqpoint{2.342263in}{2.859688in}}%
\pgfpathlineto{\pgfqpoint{2.352000in}{2.841051in}}%
\pgfpathlineto{\pgfqpoint{2.361703in}{2.822936in}}%
\pgfpathlineto{\pgfqpoint{2.371375in}{2.805333in}}%
\pgfpathclose%
\pgfusepath{fill}%
\end{pgfscope}%
\begin{pgfscope}%
\pgfpathrectangle{\pgfqpoint{1.150000in}{0.150000in}}{\pgfqpoint{5.700000in}{5.700000in}}%
\pgfusepath{clip}%
\pgfsetbuttcap%
\pgfsetroundjoin%
\definecolor{currentfill}{rgb}{0.263663,0.237631,0.518762}%
\pgfsetfillcolor{currentfill}%
\pgfsetfillopacity{0.700000}%
\pgfsetlinewidth{0.000000pt}%
\definecolor{currentstroke}{rgb}{0.000000,0.000000,0.000000}%
\pgfsetstrokecolor{currentstroke}%
\pgfsetdash{}{0pt}%
\pgfpathmoveto{\pgfqpoint{4.952781in}{1.811373in}}%
\pgfpathlineto{\pgfqpoint{4.967312in}{1.815243in}}%
\pgfpathlineto{\pgfqpoint{4.981857in}{1.819209in}}%
\pgfpathlineto{\pgfqpoint{4.996414in}{1.823272in}}%
\pgfpathlineto{\pgfqpoint{5.010985in}{1.827431in}}%
\pgfpathlineto{\pgfqpoint{5.003015in}{1.812144in}}%
\pgfpathlineto{\pgfqpoint{4.995042in}{1.796882in}}%
\pgfpathlineto{\pgfqpoint{4.987066in}{1.781650in}}%
\pgfpathlineto{\pgfqpoint{4.979086in}{1.766451in}}%
\pgfpathlineto{\pgfqpoint{4.964520in}{1.762699in}}%
\pgfpathlineto{\pgfqpoint{4.949967in}{1.759044in}}%
\pgfpathlineto{\pgfqpoint{4.935427in}{1.755486in}}%
\pgfpathlineto{\pgfqpoint{4.920899in}{1.752023in}}%
\pgfpathlineto{\pgfqpoint{4.928875in}{1.766808in}}%
\pgfpathlineto{\pgfqpoint{4.936847in}{1.781630in}}%
\pgfpathlineto{\pgfqpoint{4.944816in}{1.796487in}}%
\pgfpathlineto{\pgfqpoint{4.952781in}{1.811373in}}%
\pgfpathclose%
\pgfusepath{fill}%
\end{pgfscope}%
\begin{pgfscope}%
\pgfpathrectangle{\pgfqpoint{1.150000in}{0.150000in}}{\pgfqpoint{5.700000in}{5.700000in}}%
\pgfusepath{clip}%
\pgfsetbuttcap%
\pgfsetroundjoin%
\definecolor{currentfill}{rgb}{0.248629,0.278775,0.534556}%
\pgfsetfillcolor{currentfill}%
\pgfsetfillopacity{0.700000}%
\pgfsetlinewidth{0.000000pt}%
\definecolor{currentstroke}{rgb}{0.000000,0.000000,0.000000}%
\pgfsetstrokecolor{currentstroke}%
\pgfsetdash{}{0pt}%
\pgfpathmoveto{\pgfqpoint{3.165305in}{1.896841in}}%
\pgfpathlineto{\pgfqpoint{3.179421in}{1.884331in}}%
\pgfpathlineto{\pgfqpoint{3.193537in}{1.871938in}}%
\pgfpathlineto{\pgfqpoint{3.207653in}{1.859660in}}%
\pgfpathlineto{\pgfqpoint{3.221769in}{1.847499in}}%
\pgfpathlineto{\pgfqpoint{3.212985in}{1.854955in}}%
\pgfpathlineto{\pgfqpoint{3.204181in}{1.862826in}}%
\pgfpathlineto{\pgfqpoint{3.195359in}{1.871118in}}%
\pgfpathlineto{\pgfqpoint{3.186517in}{1.879840in}}%
\pgfpathlineto{\pgfqpoint{3.172353in}{1.892663in}}%
\pgfpathlineto{\pgfqpoint{3.158189in}{1.905603in}}%
\pgfpathlineto{\pgfqpoint{3.144025in}{1.918659in}}%
\pgfpathlineto{\pgfqpoint{3.129860in}{1.931832in}}%
\pgfpathlineto{\pgfqpoint{3.138752in}{1.922439in}}%
\pgfpathlineto{\pgfqpoint{3.147623in}{1.913481in}}%
\pgfpathlineto{\pgfqpoint{3.156474in}{1.904951in}}%
\pgfpathlineto{\pgfqpoint{3.165305in}{1.896841in}}%
\pgfpathclose%
\pgfusepath{fill}%
\end{pgfscope}%
\begin{pgfscope}%
\pgfpathrectangle{\pgfqpoint{1.150000in}{0.150000in}}{\pgfqpoint{5.700000in}{5.700000in}}%
\pgfusepath{clip}%
\pgfsetbuttcap%
\pgfsetroundjoin%
\definecolor{currentfill}{rgb}{0.282327,0.094955,0.417331}%
\pgfsetfillcolor{currentfill}%
\pgfsetfillopacity{0.700000}%
\pgfsetlinewidth{0.000000pt}%
\definecolor{currentstroke}{rgb}{0.000000,0.000000,0.000000}%
\pgfsetstrokecolor{currentstroke}%
\pgfsetdash{}{0pt}%
\pgfpathmoveto{\pgfqpoint{4.562055in}{1.510352in}}%
\pgfpathlineto{\pgfqpoint{4.576401in}{1.510486in}}%
\pgfpathlineto{\pgfqpoint{4.590757in}{1.510715in}}%
\pgfpathlineto{\pgfqpoint{4.605123in}{1.511040in}}%
\pgfpathlineto{\pgfqpoint{4.619499in}{1.511460in}}%
\pgfpathlineto{\pgfqpoint{4.611467in}{1.499412in}}%
\pgfpathlineto{\pgfqpoint{4.603431in}{1.487493in}}%
\pgfpathlineto{\pgfqpoint{4.595391in}{1.475707in}}%
\pgfpathlineto{\pgfqpoint{4.587348in}{1.464059in}}%
\pgfpathlineto{\pgfqpoint{4.572968in}{1.464132in}}%
\pgfpathlineto{\pgfqpoint{4.558598in}{1.464300in}}%
\pgfpathlineto{\pgfqpoint{4.544238in}{1.464562in}}%
\pgfpathlineto{\pgfqpoint{4.529888in}{1.464920in}}%
\pgfpathlineto{\pgfqpoint{4.537936in}{1.476069in}}%
\pgfpathlineto{\pgfqpoint{4.545979in}{1.487361in}}%
\pgfpathlineto{\pgfqpoint{4.554019in}{1.498790in}}%
\pgfpathlineto{\pgfqpoint{4.562055in}{1.510352in}}%
\pgfpathclose%
\pgfusepath{fill}%
\end{pgfscope}%
\begin{pgfscope}%
\pgfpathrectangle{\pgfqpoint{1.150000in}{0.150000in}}{\pgfqpoint{5.700000in}{5.700000in}}%
\pgfusepath{clip}%
\pgfsetbuttcap%
\pgfsetroundjoin%
\definecolor{currentfill}{rgb}{0.197636,0.391528,0.554969}%
\pgfsetfillcolor{currentfill}%
\pgfsetfillopacity{0.700000}%
\pgfsetlinewidth{0.000000pt}%
\definecolor{currentstroke}{rgb}{0.000000,0.000000,0.000000}%
\pgfsetstrokecolor{currentstroke}%
\pgfsetdash{}{0pt}%
\pgfpathmoveto{\pgfqpoint{5.286848in}{2.184080in}}%
\pgfpathlineto{\pgfqpoint{5.301575in}{2.190895in}}%
\pgfpathlineto{\pgfqpoint{5.316318in}{2.197809in}}%
\pgfpathlineto{\pgfqpoint{5.331076in}{2.204822in}}%
\pgfpathlineto{\pgfqpoint{5.345849in}{2.211935in}}%
\pgfpathlineto{\pgfqpoint{5.337942in}{2.195927in}}%
\pgfpathlineto{\pgfqpoint{5.330031in}{2.179863in}}%
\pgfpathlineto{\pgfqpoint{5.322114in}{2.163749in}}%
\pgfpathlineto{\pgfqpoint{5.314193in}{2.147585in}}%
\pgfpathlineto{\pgfqpoint{5.299428in}{2.140793in}}%
\pgfpathlineto{\pgfqpoint{5.284677in}{2.134100in}}%
\pgfpathlineto{\pgfqpoint{5.269941in}{2.127505in}}%
\pgfpathlineto{\pgfqpoint{5.255221in}{2.121010in}}%
\pgfpathlineto{\pgfqpoint{5.263135in}{2.136846in}}%
\pgfpathlineto{\pgfqpoint{5.271044in}{2.152639in}}%
\pgfpathlineto{\pgfqpoint{5.278948in}{2.168384in}}%
\pgfpathlineto{\pgfqpoint{5.286848in}{2.184080in}}%
\pgfpathclose%
\pgfusepath{fill}%
\end{pgfscope}%
\begin{pgfscope}%
\pgfpathrectangle{\pgfqpoint{1.150000in}{0.150000in}}{\pgfqpoint{5.700000in}{5.700000in}}%
\pgfusepath{clip}%
\pgfsetbuttcap%
\pgfsetroundjoin%
\definecolor{currentfill}{rgb}{0.283229,0.120777,0.440584}%
\pgfsetfillcolor{currentfill}%
\pgfsetfillopacity{0.700000}%
\pgfsetlinewidth{0.000000pt}%
\definecolor{currentstroke}{rgb}{0.000000,0.000000,0.000000}%
\pgfsetstrokecolor{currentstroke}%
\pgfsetdash{}{0pt}%
\pgfpathmoveto{\pgfqpoint{4.651592in}{1.560839in}}%
\pgfpathlineto{\pgfqpoint{4.665977in}{1.561829in}}%
\pgfpathlineto{\pgfqpoint{4.680372in}{1.562914in}}%
\pgfpathlineto{\pgfqpoint{4.694778in}{1.564095in}}%
\pgfpathlineto{\pgfqpoint{4.709195in}{1.565371in}}%
\pgfpathlineto{\pgfqpoint{4.701179in}{1.552388in}}%
\pgfpathlineto{\pgfqpoint{4.693159in}{1.539511in}}%
\pgfpathlineto{\pgfqpoint{4.685136in}{1.526743in}}%
\pgfpathlineto{\pgfqpoint{4.677109in}{1.514090in}}%
\pgfpathlineto{\pgfqpoint{4.662691in}{1.513290in}}%
\pgfpathlineto{\pgfqpoint{4.648283in}{1.512585in}}%
\pgfpathlineto{\pgfqpoint{4.633886in}{1.511975in}}%
\pgfpathlineto{\pgfqpoint{4.619499in}{1.511460in}}%
\pgfpathlineto{\pgfqpoint{4.627528in}{1.523631in}}%
\pgfpathlineto{\pgfqpoint{4.635553in}{1.535921in}}%
\pgfpathlineto{\pgfqpoint{4.643574in}{1.548325in}}%
\pgfpathlineto{\pgfqpoint{4.651592in}{1.560839in}}%
\pgfpathclose%
\pgfusepath{fill}%
\end{pgfscope}%
\begin{pgfscope}%
\pgfpathrectangle{\pgfqpoint{1.150000in}{0.150000in}}{\pgfqpoint{5.700000in}{5.700000in}}%
\pgfusepath{clip}%
\pgfsetbuttcap%
\pgfsetroundjoin%
\definecolor{currentfill}{rgb}{0.281446,0.084320,0.407414}%
\pgfsetfillcolor{currentfill}%
\pgfsetfillopacity{0.700000}%
\pgfsetlinewidth{0.000000pt}%
\definecolor{currentstroke}{rgb}{0.000000,0.000000,0.000000}%
\pgfsetstrokecolor{currentstroke}%
\pgfsetdash{}{0pt}%
\pgfpathmoveto{\pgfqpoint{3.798030in}{1.481808in}}%
\pgfpathlineto{\pgfqpoint{3.812167in}{1.474846in}}%
\pgfpathlineto{\pgfqpoint{3.826310in}{1.467984in}}%
\pgfpathlineto{\pgfqpoint{3.840457in}{1.461223in}}%
\pgfpathlineto{\pgfqpoint{3.854609in}{1.454562in}}%
\pgfpathlineto{\pgfqpoint{3.846293in}{1.452956in}}%
\pgfpathlineto{\pgfqpoint{3.837967in}{1.451656in}}%
\pgfpathlineto{\pgfqpoint{3.829631in}{1.450668in}}%
\pgfpathlineto{\pgfqpoint{3.821285in}{1.449998in}}%
\pgfpathlineto{\pgfqpoint{3.807106in}{1.457263in}}%
\pgfpathlineto{\pgfqpoint{3.792932in}{1.464627in}}%
\pgfpathlineto{\pgfqpoint{3.778763in}{1.472092in}}%
\pgfpathlineto{\pgfqpoint{3.764597in}{1.479657in}}%
\pgfpathlineto{\pgfqpoint{3.772972in}{1.479716in}}%
\pgfpathlineto{\pgfqpoint{3.781335in}{1.480098in}}%
\pgfpathlineto{\pgfqpoint{3.789688in}{1.480797in}}%
\pgfpathlineto{\pgfqpoint{3.798030in}{1.481808in}}%
\pgfpathclose%
\pgfusepath{fill}%
\end{pgfscope}%
\begin{pgfscope}%
\pgfpathrectangle{\pgfqpoint{1.150000in}{0.150000in}}{\pgfqpoint{5.700000in}{5.700000in}}%
\pgfusepath{clip}%
\pgfsetbuttcap%
\pgfsetroundjoin%
\definecolor{currentfill}{rgb}{0.280267,0.073417,0.397163}%
\pgfsetfillcolor{currentfill}%
\pgfsetfillopacity{0.700000}%
\pgfsetlinewidth{0.000000pt}%
\definecolor{currentstroke}{rgb}{0.000000,0.000000,0.000000}%
\pgfsetstrokecolor{currentstroke}%
\pgfsetdash{}{0pt}%
\pgfpathmoveto{\pgfqpoint{4.472584in}{1.467302in}}%
\pgfpathlineto{\pgfqpoint{4.486895in}{1.466564in}}%
\pgfpathlineto{\pgfqpoint{4.501217in}{1.465921in}}%
\pgfpathlineto{\pgfqpoint{4.515548in}{1.465373in}}%
\pgfpathlineto{\pgfqpoint{4.529888in}{1.464920in}}%
\pgfpathlineto{\pgfqpoint{4.521836in}{1.453918in}}%
\pgfpathlineto{\pgfqpoint{4.513780in}{1.443068in}}%
\pgfpathlineto{\pgfqpoint{4.505720in}{1.432375in}}%
\pgfpathlineto{\pgfqpoint{4.497656in}{1.421844in}}%
\pgfpathlineto{\pgfqpoint{4.483310in}{1.422807in}}%
\pgfpathlineto{\pgfqpoint{4.468973in}{1.423864in}}%
\pgfpathlineto{\pgfqpoint{4.454645in}{1.425017in}}%
\pgfpathlineto{\pgfqpoint{4.440327in}{1.426264in}}%
\pgfpathlineto{\pgfqpoint{4.448398in}{1.436279in}}%
\pgfpathlineto{\pgfqpoint{4.456464in}{1.446460in}}%
\pgfpathlineto{\pgfqpoint{4.464526in}{1.456803in}}%
\pgfpathlineto{\pgfqpoint{4.472584in}{1.467302in}}%
\pgfpathclose%
\pgfusepath{fill}%
\end{pgfscope}%
\begin{pgfscope}%
\pgfpathrectangle{\pgfqpoint{1.150000in}{0.150000in}}{\pgfqpoint{5.700000in}{5.700000in}}%
\pgfusepath{clip}%
\pgfsetbuttcap%
\pgfsetroundjoin%
\definecolor{currentfill}{rgb}{0.126326,0.644107,0.525311}%
\pgfsetfillcolor{currentfill}%
\pgfsetfillopacity{0.700000}%
\pgfsetlinewidth{0.000000pt}%
\definecolor{currentstroke}{rgb}{0.000000,0.000000,0.000000}%
\pgfsetstrokecolor{currentstroke}%
\pgfsetdash{}{0pt}%
\pgfpathmoveto{\pgfqpoint{2.314082in}{2.888100in}}%
\pgfpathlineto{\pgfqpoint{2.328417in}{2.867150in}}%
\pgfpathlineto{\pgfqpoint{2.342744in}{2.846374in}}%
\pgfpathlineto{\pgfqpoint{2.357063in}{2.825768in}}%
\pgfpathlineto{\pgfqpoint{2.371375in}{2.805333in}}%
\pgfpathlineto{\pgfqpoint{2.361703in}{2.822936in}}%
\pgfpathlineto{\pgfqpoint{2.352000in}{2.841051in}}%
\pgfpathlineto{\pgfqpoint{2.342263in}{2.859688in}}%
\pgfpathlineto{\pgfqpoint{2.332493in}{2.878855in}}%
\pgfpathlineto{\pgfqpoint{2.318106in}{2.900011in}}%
\pgfpathlineto{\pgfqpoint{2.303712in}{2.921338in}}%
\pgfpathlineto{\pgfqpoint{2.289309in}{2.942838in}}%
\pgfpathlineto{\pgfqpoint{2.274898in}{2.964513in}}%
\pgfpathlineto{\pgfqpoint{2.284746in}{2.944613in}}%
\pgfpathlineto{\pgfqpoint{2.294558in}{2.925250in}}%
\pgfpathlineto{\pgfqpoint{2.304337in}{2.906415in}}%
\pgfpathlineto{\pgfqpoint{2.314082in}{2.888100in}}%
\pgfpathclose%
\pgfusepath{fill}%
\end{pgfscope}%
\begin{pgfscope}%
\pgfpathrectangle{\pgfqpoint{1.150000in}{0.150000in}}{\pgfqpoint{5.700000in}{5.700000in}}%
\pgfusepath{clip}%
\pgfsetbuttcap%
\pgfsetroundjoin%
\definecolor{currentfill}{rgb}{0.225863,0.330805,0.547314}%
\pgfsetfillcolor{currentfill}%
\pgfsetfillopacity{0.700000}%
\pgfsetlinewidth{0.000000pt}%
\definecolor{currentstroke}{rgb}{0.000000,0.000000,0.000000}%
\pgfsetstrokecolor{currentstroke}%
\pgfsetdash{}{0pt}%
\pgfpathmoveto{\pgfqpoint{5.164811in}{2.033635in}}%
\pgfpathlineto{\pgfqpoint{5.179467in}{2.039399in}}%
\pgfpathlineto{\pgfqpoint{5.194137in}{2.045262in}}%
\pgfpathlineto{\pgfqpoint{5.208821in}{2.051223in}}%
\pgfpathlineto{\pgfqpoint{5.223520in}{2.057283in}}%
\pgfpathlineto{\pgfqpoint{5.215584in}{2.041271in}}%
\pgfpathlineto{\pgfqpoint{5.207644in}{2.025234in}}%
\pgfpathlineto{\pgfqpoint{5.199700in}{2.009174in}}%
\pgfpathlineto{\pgfqpoint{5.191752in}{1.993096in}}%
\pgfpathlineto{\pgfqpoint{5.177060in}{1.987393in}}%
\pgfpathlineto{\pgfqpoint{5.162382in}{1.981788in}}%
\pgfpathlineto{\pgfqpoint{5.147718in}{1.976281in}}%
\pgfpathlineto{\pgfqpoint{5.133069in}{1.970872in}}%
\pgfpathlineto{\pgfqpoint{5.141011in}{1.986587in}}%
\pgfpathlineto{\pgfqpoint{5.148948in}{2.002289in}}%
\pgfpathlineto{\pgfqpoint{5.156882in}{2.017972in}}%
\pgfpathlineto{\pgfqpoint{5.164811in}{2.033635in}}%
\pgfpathclose%
\pgfusepath{fill}%
\end{pgfscope}%
\begin{pgfscope}%
\pgfpathrectangle{\pgfqpoint{1.150000in}{0.150000in}}{\pgfqpoint{5.700000in}{5.700000in}}%
\pgfusepath{clip}%
\pgfsetbuttcap%
\pgfsetroundjoin%
\definecolor{currentfill}{rgb}{0.281887,0.150881,0.465405}%
\pgfsetfillcolor{currentfill}%
\pgfsetfillopacity{0.700000}%
\pgfsetlinewidth{0.000000pt}%
\definecolor{currentstroke}{rgb}{0.000000,0.000000,0.000000}%
\pgfsetstrokecolor{currentstroke}%
\pgfsetdash{}{0pt}%
\pgfpathmoveto{\pgfqpoint{4.741226in}{1.618266in}}%
\pgfpathlineto{\pgfqpoint{4.755655in}{1.620095in}}%
\pgfpathlineto{\pgfqpoint{4.770094in}{1.622021in}}%
\pgfpathlineto{\pgfqpoint{4.784545in}{1.624041in}}%
\pgfpathlineto{\pgfqpoint{4.799007in}{1.626158in}}%
\pgfpathlineto{\pgfqpoint{4.791004in}{1.612346in}}%
\pgfpathlineto{\pgfqpoint{4.782998in}{1.598618in}}%
\pgfpathlineto{\pgfqpoint{4.774988in}{1.584977in}}%
\pgfpathlineto{\pgfqpoint{4.766975in}{1.571427in}}%
\pgfpathlineto{\pgfqpoint{4.752513in}{1.569770in}}%
\pgfpathlineto{\pgfqpoint{4.738063in}{1.568209in}}%
\pgfpathlineto{\pgfqpoint{4.723624in}{1.566742in}}%
\pgfpathlineto{\pgfqpoint{4.709195in}{1.565371in}}%
\pgfpathlineto{\pgfqpoint{4.717208in}{1.578455in}}%
\pgfpathlineto{\pgfqpoint{4.725218in}{1.591635in}}%
\pgfpathlineto{\pgfqpoint{4.733224in}{1.604906in}}%
\pgfpathlineto{\pgfqpoint{4.741226in}{1.618266in}}%
\pgfpathclose%
\pgfusepath{fill}%
\end{pgfscope}%
\begin{pgfscope}%
\pgfpathrectangle{\pgfqpoint{1.150000in}{0.150000in}}{\pgfqpoint{5.700000in}{5.700000in}}%
\pgfusepath{clip}%
\pgfsetbuttcap%
\pgfsetroundjoin%
\definecolor{currentfill}{rgb}{0.255645,0.260703,0.528312}%
\pgfsetfillcolor{currentfill}%
\pgfsetfillopacity{0.700000}%
\pgfsetlinewidth{0.000000pt}%
\definecolor{currentstroke}{rgb}{0.000000,0.000000,0.000000}%
\pgfsetstrokecolor{currentstroke}%
\pgfsetdash{}{0pt}%
\pgfpathmoveto{\pgfqpoint{3.221769in}{1.847499in}}%
\pgfpathlineto{\pgfqpoint{3.235886in}{1.835452in}}%
\pgfpathlineto{\pgfqpoint{3.250003in}{1.823520in}}%
\pgfpathlineto{\pgfqpoint{3.264121in}{1.811702in}}%
\pgfpathlineto{\pgfqpoint{3.278239in}{1.799997in}}%
\pgfpathlineto{\pgfqpoint{3.269500in}{1.806802in}}%
\pgfpathlineto{\pgfqpoint{3.260743in}{1.814015in}}%
\pgfpathlineto{\pgfqpoint{3.251968in}{1.821645in}}%
\pgfpathlineto{\pgfqpoint{3.243174in}{1.829697in}}%
\pgfpathlineto{\pgfqpoint{3.229009in}{1.842061in}}%
\pgfpathlineto{\pgfqpoint{3.214845in}{1.854539in}}%
\pgfpathlineto{\pgfqpoint{3.200681in}{1.867132in}}%
\pgfpathlineto{\pgfqpoint{3.186517in}{1.879840in}}%
\pgfpathlineto{\pgfqpoint{3.195359in}{1.871118in}}%
\pgfpathlineto{\pgfqpoint{3.204181in}{1.862826in}}%
\pgfpathlineto{\pgfqpoint{3.212985in}{1.854955in}}%
\pgfpathlineto{\pgfqpoint{3.221769in}{1.847499in}}%
\pgfpathclose%
\pgfusepath{fill}%
\end{pgfscope}%
\begin{pgfscope}%
\pgfpathrectangle{\pgfqpoint{1.150000in}{0.150000in}}{\pgfqpoint{5.700000in}{5.700000in}}%
\pgfusepath{clip}%
\pgfsetbuttcap%
\pgfsetroundjoin%
\definecolor{currentfill}{rgb}{0.277941,0.056324,0.381191}%
\pgfsetfillcolor{currentfill}%
\pgfsetfillopacity{0.700000}%
\pgfsetlinewidth{0.000000pt}%
\definecolor{currentstroke}{rgb}{0.000000,0.000000,0.000000}%
\pgfsetstrokecolor{currentstroke}%
\pgfsetdash{}{0pt}%
\pgfpathmoveto{\pgfqpoint{4.383142in}{1.432206in}}%
\pgfpathlineto{\pgfqpoint{4.397425in}{1.430577in}}%
\pgfpathlineto{\pgfqpoint{4.411717in}{1.429044in}}%
\pgfpathlineto{\pgfqpoint{4.426017in}{1.427607in}}%
\pgfpathlineto{\pgfqpoint{4.440327in}{1.426264in}}%
\pgfpathlineto{\pgfqpoint{4.432252in}{1.416421in}}%
\pgfpathlineto{\pgfqpoint{4.424172in}{1.406755in}}%
\pgfpathlineto{\pgfqpoint{4.416087in}{1.397271in}}%
\pgfpathlineto{\pgfqpoint{4.407998in}{1.387973in}}%
\pgfpathlineto{\pgfqpoint{4.393680in}{1.389843in}}%
\pgfpathlineto{\pgfqpoint{4.379371in}{1.391808in}}%
\pgfpathlineto{\pgfqpoint{4.365070in}{1.393868in}}%
\pgfpathlineto{\pgfqpoint{4.350778in}{1.396023in}}%
\pgfpathlineto{\pgfqpoint{4.358876in}{1.404786in}}%
\pgfpathlineto{\pgfqpoint{4.366970in}{1.413741in}}%
\pgfpathlineto{\pgfqpoint{4.375059in}{1.422883in}}%
\pgfpathlineto{\pgfqpoint{4.383142in}{1.432206in}}%
\pgfpathclose%
\pgfusepath{fill}%
\end{pgfscope}%
\begin{pgfscope}%
\pgfpathrectangle{\pgfqpoint{1.150000in}{0.150000in}}{\pgfqpoint{5.700000in}{5.700000in}}%
\pgfusepath{clip}%
\pgfsetbuttcap%
\pgfsetroundjoin%
\definecolor{currentfill}{rgb}{0.282884,0.135920,0.453427}%
\pgfsetfillcolor{currentfill}%
\pgfsetfillopacity{0.700000}%
\pgfsetlinewidth{0.000000pt}%
\definecolor{currentstroke}{rgb}{0.000000,0.000000,0.000000}%
\pgfsetstrokecolor{currentstroke}%
\pgfsetdash{}{0pt}%
\pgfpathmoveto{\pgfqpoint{3.594913in}{1.578393in}}%
\pgfpathlineto{\pgfqpoint{3.609034in}{1.569598in}}%
\pgfpathlineto{\pgfqpoint{3.623159in}{1.560907in}}%
\pgfpathlineto{\pgfqpoint{3.637286in}{1.552320in}}%
\pgfpathlineto{\pgfqpoint{3.651417in}{1.543836in}}%
\pgfpathlineto{\pgfqpoint{3.642969in}{1.545343in}}%
\pgfpathlineto{\pgfqpoint{3.634509in}{1.547197in}}%
\pgfpathlineto{\pgfqpoint{3.626036in}{1.549405in}}%
\pgfpathlineto{\pgfqpoint{3.617549in}{1.551973in}}%
\pgfpathlineto{\pgfqpoint{3.603385in}{1.561084in}}%
\pgfpathlineto{\pgfqpoint{3.589223in}{1.570298in}}%
\pgfpathlineto{\pgfqpoint{3.575065in}{1.579616in}}%
\pgfpathlineto{\pgfqpoint{3.560909in}{1.589039in}}%
\pgfpathlineto{\pgfqpoint{3.569430in}{1.585835in}}%
\pgfpathlineto{\pgfqpoint{3.577938in}{1.582997in}}%
\pgfpathlineto{\pgfqpoint{3.586432in}{1.580519in}}%
\pgfpathlineto{\pgfqpoint{3.594913in}{1.578393in}}%
\pgfpathclose%
\pgfusepath{fill}%
\end{pgfscope}%
\begin{pgfscope}%
\pgfpathrectangle{\pgfqpoint{1.150000in}{0.150000in}}{\pgfqpoint{5.700000in}{5.700000in}}%
\pgfusepath{clip}%
\pgfsetbuttcap%
\pgfsetroundjoin%
\definecolor{currentfill}{rgb}{0.276022,0.044167,0.370164}%
\pgfsetfillcolor{currentfill}%
\pgfsetfillopacity{0.700000}%
\pgfsetlinewidth{0.000000pt}%
\definecolor{currentstroke}{rgb}{0.000000,0.000000,0.000000}%
\pgfsetstrokecolor{currentstroke}%
\pgfsetdash{}{0pt}%
\pgfpathmoveto{\pgfqpoint{4.147298in}{1.402846in}}%
\pgfpathlineto{\pgfqpoint{4.161510in}{1.398997in}}%
\pgfpathlineto{\pgfqpoint{4.175729in}{1.395244in}}%
\pgfpathlineto{\pgfqpoint{4.189956in}{1.391588in}}%
\pgfpathlineto{\pgfqpoint{4.204190in}{1.388028in}}%
\pgfpathlineto{\pgfqpoint{4.196038in}{1.381416in}}%
\pgfpathlineto{\pgfqpoint{4.187880in}{1.375039in}}%
\pgfpathlineto{\pgfqpoint{4.179716in}{1.368900in}}%
\pgfpathlineto{\pgfqpoint{4.171545in}{1.363007in}}%
\pgfpathlineto{\pgfqpoint{4.157296in}{1.367130in}}%
\pgfpathlineto{\pgfqpoint{4.143054in}{1.371350in}}%
\pgfpathlineto{\pgfqpoint{4.128818in}{1.375665in}}%
\pgfpathlineto{\pgfqpoint{4.114589in}{1.380077in}}%
\pgfpathlineto{\pgfqpoint{4.122777in}{1.385400in}}%
\pgfpathlineto{\pgfqpoint{4.130957in}{1.390973in}}%
\pgfpathlineto{\pgfqpoint{4.139131in}{1.396790in}}%
\pgfpathlineto{\pgfqpoint{4.147298in}{1.402846in}}%
\pgfpathclose%
\pgfusepath{fill}%
\end{pgfscope}%
\begin{pgfscope}%
\pgfpathrectangle{\pgfqpoint{1.150000in}{0.150000in}}{\pgfqpoint{5.700000in}{5.700000in}}%
\pgfusepath{clip}%
\pgfsetbuttcap%
\pgfsetroundjoin%
\definecolor{currentfill}{rgb}{0.277941,0.056324,0.381191}%
\pgfsetfillcolor{currentfill}%
\pgfsetfillopacity{0.700000}%
\pgfsetlinewidth{0.000000pt}%
\definecolor{currentstroke}{rgb}{0.000000,0.000000,0.000000}%
\pgfsetstrokecolor{currentstroke}%
\pgfsetdash{}{0pt}%
\pgfpathmoveto{\pgfqpoint{4.000991in}{1.418863in}}%
\pgfpathlineto{\pgfqpoint{4.015169in}{1.413674in}}%
\pgfpathlineto{\pgfqpoint{4.029353in}{1.408583in}}%
\pgfpathlineto{\pgfqpoint{4.043543in}{1.403589in}}%
\pgfpathlineto{\pgfqpoint{4.057740in}{1.398692in}}%
\pgfpathlineto{\pgfqpoint{4.049526in}{1.394200in}}%
\pgfpathlineto{\pgfqpoint{4.041305in}{1.389974in}}%
\pgfpathlineto{\pgfqpoint{4.033076in}{1.386020in}}%
\pgfpathlineto{\pgfqpoint{4.024839in}{1.382346in}}%
\pgfpathlineto{\pgfqpoint{4.010623in}{1.387824in}}%
\pgfpathlineto{\pgfqpoint{3.996412in}{1.393400in}}%
\pgfpathlineto{\pgfqpoint{3.982207in}{1.399074in}}%
\pgfpathlineto{\pgfqpoint{3.968008in}{1.404845in}}%
\pgfpathlineto{\pgfqpoint{3.976266in}{1.407930in}}%
\pgfpathlineto{\pgfqpoint{3.984516in}{1.411299in}}%
\pgfpathlineto{\pgfqpoint{3.992758in}{1.414946in}}%
\pgfpathlineto{\pgfqpoint{4.000991in}{1.418863in}}%
\pgfpathclose%
\pgfusepath{fill}%
\end{pgfscope}%
\begin{pgfscope}%
\pgfpathrectangle{\pgfqpoint{1.150000in}{0.150000in}}{\pgfqpoint{5.700000in}{5.700000in}}%
\pgfusepath{clip}%
\pgfsetbuttcap%
\pgfsetroundjoin%
\definecolor{currentfill}{rgb}{0.250425,0.274290,0.533103}%
\pgfsetfillcolor{currentfill}%
\pgfsetfillopacity{0.700000}%
\pgfsetlinewidth{0.000000pt}%
\definecolor{currentstroke}{rgb}{0.000000,0.000000,0.000000}%
\pgfsetstrokecolor{currentstroke}%
\pgfsetdash{}{0pt}%
\pgfpathmoveto{\pgfqpoint{5.042827in}{1.888762in}}%
\pgfpathlineto{\pgfqpoint{5.057416in}{1.893409in}}%
\pgfpathlineto{\pgfqpoint{5.072018in}{1.898153in}}%
\pgfpathlineto{\pgfqpoint{5.086634in}{1.902994in}}%
\pgfpathlineto{\pgfqpoint{5.101264in}{1.907932in}}%
\pgfpathlineto{\pgfqpoint{5.093303in}{1.892196in}}%
\pgfpathlineto{\pgfqpoint{5.085339in}{1.876465in}}%
\pgfpathlineto{\pgfqpoint{5.077371in}{1.860745in}}%
\pgfpathlineto{\pgfqpoint{5.069400in}{1.845038in}}%
\pgfpathlineto{\pgfqpoint{5.054776in}{1.840491in}}%
\pgfpathlineto{\pgfqpoint{5.040165in}{1.836041in}}%
\pgfpathlineto{\pgfqpoint{5.025568in}{1.831688in}}%
\pgfpathlineto{\pgfqpoint{5.010985in}{1.827431in}}%
\pgfpathlineto{\pgfqpoint{5.018951in}{1.842741in}}%
\pgfpathlineto{\pgfqpoint{5.026913in}{1.858068in}}%
\pgfpathlineto{\pgfqpoint{5.034872in}{1.873409in}}%
\pgfpathlineto{\pgfqpoint{5.042827in}{1.888762in}}%
\pgfpathclose%
\pgfusepath{fill}%
\end{pgfscope}%
\begin{pgfscope}%
\pgfpathrectangle{\pgfqpoint{1.150000in}{0.150000in}}{\pgfqpoint{5.700000in}{5.700000in}}%
\pgfusepath{clip}%
\pgfsetbuttcap%
\pgfsetroundjoin%
\definecolor{currentfill}{rgb}{0.278012,0.180367,0.486697}%
\pgfsetfillcolor{currentfill}%
\pgfsetfillopacity{0.700000}%
\pgfsetlinewidth{0.000000pt}%
\definecolor{currentstroke}{rgb}{0.000000,0.000000,0.000000}%
\pgfsetstrokecolor{currentstroke}%
\pgfsetdash{}{0pt}%
\pgfpathmoveto{\pgfqpoint{4.830987in}{1.682150in}}%
\pgfpathlineto{\pgfqpoint{4.845463in}{1.684804in}}%
\pgfpathlineto{\pgfqpoint{4.859951in}{1.687553in}}%
\pgfpathlineto{\pgfqpoint{4.874451in}{1.690399in}}%
\pgfpathlineto{\pgfqpoint{4.888963in}{1.693340in}}%
\pgfpathlineto{\pgfqpoint{4.880971in}{1.678803in}}%
\pgfpathlineto{\pgfqpoint{4.872975in}{1.664327in}}%
\pgfpathlineto{\pgfqpoint{4.864976in}{1.649918in}}%
\pgfpathlineto{\pgfqpoint{4.856975in}{1.635578in}}%
\pgfpathlineto{\pgfqpoint{4.842465in}{1.633080in}}%
\pgfpathlineto{\pgfqpoint{4.827967in}{1.630677in}}%
\pgfpathlineto{\pgfqpoint{4.813481in}{1.628370in}}%
\pgfpathlineto{\pgfqpoint{4.799007in}{1.626158in}}%
\pgfpathlineto{\pgfqpoint{4.807007in}{1.640048in}}%
\pgfpathlineto{\pgfqpoint{4.815004in}{1.654013in}}%
\pgfpathlineto{\pgfqpoint{4.822997in}{1.668048in}}%
\pgfpathlineto{\pgfqpoint{4.830987in}{1.682150in}}%
\pgfpathclose%
\pgfusepath{fill}%
\end{pgfscope}%
\begin{pgfscope}%
\pgfpathrectangle{\pgfqpoint{1.150000in}{0.150000in}}{\pgfqpoint{5.700000in}{5.700000in}}%
\pgfusepath{clip}%
\pgfsetbuttcap%
\pgfsetroundjoin%
\definecolor{currentfill}{rgb}{0.150148,0.676631,0.506589}%
\pgfsetfillcolor{currentfill}%
\pgfsetfillopacity{0.700000}%
\pgfsetlinewidth{0.000000pt}%
\definecolor{currentstroke}{rgb}{0.000000,0.000000,0.000000}%
\pgfsetstrokecolor{currentstroke}%
\pgfsetdash{}{0pt}%
\pgfpathmoveto{\pgfqpoint{2.256659in}{2.973658in}}%
\pgfpathlineto{\pgfqpoint{2.271027in}{2.952002in}}%
\pgfpathlineto{\pgfqpoint{2.285387in}{2.930524in}}%
\pgfpathlineto{\pgfqpoint{2.299739in}{2.909224in}}%
\pgfpathlineto{\pgfqpoint{2.314082in}{2.888100in}}%
\pgfpathlineto{\pgfqpoint{2.304337in}{2.906415in}}%
\pgfpathlineto{\pgfqpoint{2.294558in}{2.925250in}}%
\pgfpathlineto{\pgfqpoint{2.284746in}{2.944613in}}%
\pgfpathlineto{\pgfqpoint{2.274898in}{2.964513in}}%
\pgfpathlineto{\pgfqpoint{2.260478in}{2.986363in}}%
\pgfpathlineto{\pgfqpoint{2.246050in}{3.008391in}}%
\pgfpathlineto{\pgfqpoint{2.231612in}{3.030598in}}%
\pgfpathlineto{\pgfqpoint{2.217165in}{3.052986in}}%
\pgfpathlineto{\pgfqpoint{2.227091in}{3.032346in}}%
\pgfpathlineto{\pgfqpoint{2.236982in}{3.012250in}}%
\pgfpathlineto{\pgfqpoint{2.246838in}{2.992690in}}%
\pgfpathlineto{\pgfqpoint{2.256659in}{2.973658in}}%
\pgfpathclose%
\pgfusepath{fill}%
\end{pgfscope}%
\begin{pgfscope}%
\pgfpathrectangle{\pgfqpoint{1.150000in}{0.150000in}}{\pgfqpoint{5.700000in}{5.700000in}}%
\pgfusepath{clip}%
\pgfsetbuttcap%
\pgfsetroundjoin%
\definecolor{currentfill}{rgb}{0.262138,0.242286,0.520837}%
\pgfsetfillcolor{currentfill}%
\pgfsetfillopacity{0.700000}%
\pgfsetlinewidth{0.000000pt}%
\definecolor{currentstroke}{rgb}{0.000000,0.000000,0.000000}%
\pgfsetstrokecolor{currentstroke}%
\pgfsetdash{}{0pt}%
\pgfpathmoveto{\pgfqpoint{3.278239in}{1.799997in}}%
\pgfpathlineto{\pgfqpoint{3.292359in}{1.788405in}}%
\pgfpathlineto{\pgfqpoint{3.306479in}{1.776926in}}%
\pgfpathlineto{\pgfqpoint{3.320600in}{1.765559in}}%
\pgfpathlineto{\pgfqpoint{3.334722in}{1.754304in}}%
\pgfpathlineto{\pgfqpoint{3.326027in}{1.760459in}}%
\pgfpathlineto{\pgfqpoint{3.317315in}{1.767018in}}%
\pgfpathlineto{\pgfqpoint{3.308585in}{1.773986in}}%
\pgfpathlineto{\pgfqpoint{3.299837in}{1.781372in}}%
\pgfpathlineto{\pgfqpoint{3.285670in}{1.793284in}}%
\pgfpathlineto{\pgfqpoint{3.271504in}{1.805309in}}%
\pgfpathlineto{\pgfqpoint{3.257339in}{1.817447in}}%
\pgfpathlineto{\pgfqpoint{3.243174in}{1.829697in}}%
\pgfpathlineto{\pgfqpoint{3.251968in}{1.821645in}}%
\pgfpathlineto{\pgfqpoint{3.260743in}{1.814015in}}%
\pgfpathlineto{\pgfqpoint{3.269500in}{1.806802in}}%
\pgfpathlineto{\pgfqpoint{3.278239in}{1.799997in}}%
\pgfpathclose%
\pgfusepath{fill}%
\end{pgfscope}%
\begin{pgfscope}%
\pgfpathrectangle{\pgfqpoint{1.150000in}{0.150000in}}{\pgfqpoint{5.700000in}{5.700000in}}%
\pgfusepath{clip}%
\pgfsetbuttcap%
\pgfsetroundjoin%
\definecolor{currentfill}{rgb}{0.182256,0.426184,0.557120}%
\pgfsetfillcolor{currentfill}%
\pgfsetfillopacity{0.700000}%
\pgfsetlinewidth{0.000000pt}%
\definecolor{currentstroke}{rgb}{0.000000,0.000000,0.000000}%
\pgfsetstrokecolor{currentstroke}%
\pgfsetdash{}{0pt}%
\pgfpathmoveto{\pgfqpoint{5.377425in}{2.275367in}}%
\pgfpathlineto{\pgfqpoint{5.392221in}{2.282881in}}%
\pgfpathlineto{\pgfqpoint{5.407032in}{2.290495in}}%
\pgfpathlineto{\pgfqpoint{5.421860in}{2.298208in}}%
\pgfpathlineto{\pgfqpoint{5.413968in}{2.282225in}}%
\pgfpathlineto{\pgfqpoint{5.406071in}{2.266172in}}%
\pgfpathlineto{\pgfqpoint{5.398169in}{2.250052in}}%
\pgfpathlineto{\pgfqpoint{5.390262in}{2.233869in}}%
\pgfpathlineto{\pgfqpoint{5.375442in}{2.226458in}}%
\pgfpathlineto{\pgfqpoint{5.360637in}{2.219147in}}%
\pgfpathlineto{\pgfqpoint{5.345849in}{2.211935in}}%
\pgfpathlineto{\pgfqpoint{5.353750in}{2.227886in}}%
\pgfpathlineto{\pgfqpoint{5.361647in}{2.243777in}}%
\pgfpathlineto{\pgfqpoint{5.369538in}{2.259605in}}%
\pgfpathlineto{\pgfqpoint{5.377425in}{2.275367in}}%
\pgfpathclose%
\pgfusepath{fill}%
\end{pgfscope}%
\begin{pgfscope}%
\pgfpathrectangle{\pgfqpoint{1.150000in}{0.150000in}}{\pgfqpoint{5.700000in}{5.700000in}}%
\pgfusepath{clip}%
\pgfsetbuttcap%
\pgfsetroundjoin%
\definecolor{currentfill}{rgb}{0.277018,0.050344,0.375715}%
\pgfsetfillcolor{currentfill}%
\pgfsetfillopacity{0.700000}%
\pgfsetlinewidth{0.000000pt}%
\definecolor{currentstroke}{rgb}{0.000000,0.000000,0.000000}%
\pgfsetstrokecolor{currentstroke}%
\pgfsetdash{}{0pt}%
\pgfpathmoveto{\pgfqpoint{4.293692in}{1.405596in}}%
\pgfpathlineto{\pgfqpoint{4.307951in}{1.403059in}}%
\pgfpathlineto{\pgfqpoint{4.322219in}{1.400618in}}%
\pgfpathlineto{\pgfqpoint{4.336494in}{1.398273in}}%
\pgfpathlineto{\pgfqpoint{4.350778in}{1.396023in}}%
\pgfpathlineto{\pgfqpoint{4.342674in}{1.387456in}}%
\pgfpathlineto{\pgfqpoint{4.334565in}{1.379093in}}%
\pgfpathlineto{\pgfqpoint{4.326451in}{1.370936in}}%
\pgfpathlineto{\pgfqpoint{4.318332in}{1.362994in}}%
\pgfpathlineto{\pgfqpoint{4.304037in}{1.365789in}}%
\pgfpathlineto{\pgfqpoint{4.289750in}{1.368679in}}%
\pgfpathlineto{\pgfqpoint{4.275471in}{1.371665in}}%
\pgfpathlineto{\pgfqpoint{4.261199in}{1.374746in}}%
\pgfpathlineto{\pgfqpoint{4.269331in}{1.382137in}}%
\pgfpathlineto{\pgfqpoint{4.277457in}{1.389746in}}%
\pgfpathlineto{\pgfqpoint{4.285577in}{1.397568in}}%
\pgfpathlineto{\pgfqpoint{4.293692in}{1.405596in}}%
\pgfpathclose%
\pgfusepath{fill}%
\end{pgfscope}%
\begin{pgfscope}%
\pgfpathrectangle{\pgfqpoint{1.150000in}{0.150000in}}{\pgfqpoint{5.700000in}{5.700000in}}%
\pgfusepath{clip}%
\pgfsetbuttcap%
\pgfsetroundjoin%
\definecolor{currentfill}{rgb}{0.280894,0.078907,0.402329}%
\pgfsetfillcolor{currentfill}%
\pgfsetfillopacity{0.700000}%
\pgfsetlinewidth{0.000000pt}%
\definecolor{currentstroke}{rgb}{0.000000,0.000000,0.000000}%
\pgfsetstrokecolor{currentstroke}%
\pgfsetdash{}{0pt}%
\pgfpathmoveto{\pgfqpoint{3.854609in}{1.454562in}}%
\pgfpathlineto{\pgfqpoint{3.868765in}{1.448001in}}%
\pgfpathlineto{\pgfqpoint{3.882927in}{1.441539in}}%
\pgfpathlineto{\pgfqpoint{3.897094in}{1.435176in}}%
\pgfpathlineto{\pgfqpoint{3.911266in}{1.428913in}}%
\pgfpathlineto{\pgfqpoint{3.902975in}{1.426712in}}%
\pgfpathlineto{\pgfqpoint{3.894675in}{1.424811in}}%
\pgfpathlineto{\pgfqpoint{3.886365in}{1.423218in}}%
\pgfpathlineto{\pgfqpoint{3.878045in}{1.421938in}}%
\pgfpathlineto{\pgfqpoint{3.863848in}{1.428804in}}%
\pgfpathlineto{\pgfqpoint{3.849656in}{1.435769in}}%
\pgfpathlineto{\pgfqpoint{3.835468in}{1.442834in}}%
\pgfpathlineto{\pgfqpoint{3.821285in}{1.449998in}}%
\pgfpathlineto{\pgfqpoint{3.829631in}{1.450668in}}%
\pgfpathlineto{\pgfqpoint{3.837967in}{1.451656in}}%
\pgfpathlineto{\pgfqpoint{3.846293in}{1.452956in}}%
\pgfpathlineto{\pgfqpoint{3.854609in}{1.454562in}}%
\pgfpathclose%
\pgfusepath{fill}%
\end{pgfscope}%
\begin{pgfscope}%
\pgfpathrectangle{\pgfqpoint{1.150000in}{0.150000in}}{\pgfqpoint{5.700000in}{5.700000in}}%
\pgfusepath{clip}%
\pgfsetbuttcap%
\pgfsetroundjoin%
\definecolor{currentfill}{rgb}{0.269308,0.218818,0.509577}%
\pgfsetfillcolor{currentfill}%
\pgfsetfillopacity{0.700000}%
\pgfsetlinewidth{0.000000pt}%
\definecolor{currentstroke}{rgb}{0.000000,0.000000,0.000000}%
\pgfsetstrokecolor{currentstroke}%
\pgfsetdash{}{0pt}%
\pgfpathmoveto{\pgfqpoint{4.920899in}{1.752023in}}%
\pgfpathlineto{\pgfqpoint{4.935427in}{1.755486in}}%
\pgfpathlineto{\pgfqpoint{4.949967in}{1.759044in}}%
\pgfpathlineto{\pgfqpoint{4.964520in}{1.762699in}}%
\pgfpathlineto{\pgfqpoint{4.979086in}{1.766451in}}%
\pgfpathlineto{\pgfqpoint{4.971103in}{1.751289in}}%
\pgfpathlineto{\pgfqpoint{4.963117in}{1.736167in}}%
\pgfpathlineto{\pgfqpoint{4.955128in}{1.721091in}}%
\pgfpathlineto{\pgfqpoint{4.947135in}{1.706064in}}%
\pgfpathlineto{\pgfqpoint{4.932573in}{1.702739in}}%
\pgfpathlineto{\pgfqpoint{4.918024in}{1.699510in}}%
\pgfpathlineto{\pgfqpoint{4.903487in}{1.696377in}}%
\pgfpathlineto{\pgfqpoint{4.888963in}{1.693340in}}%
\pgfpathlineto{\pgfqpoint{4.896952in}{1.707934in}}%
\pgfpathlineto{\pgfqpoint{4.904937in}{1.722583in}}%
\pgfpathlineto{\pgfqpoint{4.912920in}{1.737280in}}%
\pgfpathlineto{\pgfqpoint{4.920899in}{1.752023in}}%
\pgfpathclose%
\pgfusepath{fill}%
\end{pgfscope}%
\begin{pgfscope}%
\pgfpathrectangle{\pgfqpoint{1.150000in}{0.150000in}}{\pgfqpoint{5.700000in}{5.700000in}}%
\pgfusepath{clip}%
\pgfsetbuttcap%
\pgfsetroundjoin%
\definecolor{currentfill}{rgb}{0.191090,0.708366,0.482284}%
\pgfsetfillcolor{currentfill}%
\pgfsetfillopacity{0.700000}%
\pgfsetlinewidth{0.000000pt}%
\definecolor{currentstroke}{rgb}{0.000000,0.000000,0.000000}%
\pgfsetstrokecolor{currentstroke}%
\pgfsetdash{}{0pt}%
\pgfpathmoveto{\pgfqpoint{2.199093in}{3.062106in}}%
\pgfpathlineto{\pgfqpoint{2.213498in}{3.039718in}}%
\pgfpathlineto{\pgfqpoint{2.227894in}{3.017515in}}%
\pgfpathlineto{\pgfqpoint{2.242281in}{2.995495in}}%
\pgfpathlineto{\pgfqpoint{2.256659in}{2.973658in}}%
\pgfpathlineto{\pgfqpoint{2.246838in}{2.992690in}}%
\pgfpathlineto{\pgfqpoint{2.236982in}{3.012250in}}%
\pgfpathlineto{\pgfqpoint{2.227091in}{3.032346in}}%
\pgfpathlineto{\pgfqpoint{2.217165in}{3.052986in}}%
\pgfpathlineto{\pgfqpoint{2.202708in}{3.075556in}}%
\pgfpathlineto{\pgfqpoint{2.188242in}{3.098309in}}%
\pgfpathlineto{\pgfqpoint{2.173766in}{3.121248in}}%
\pgfpathlineto{\pgfqpoint{2.159280in}{3.144375in}}%
\pgfpathlineto{\pgfqpoint{2.169288in}{3.122988in}}%
\pgfpathlineto{\pgfqpoint{2.179259in}{3.102153in}}%
\pgfpathlineto{\pgfqpoint{2.189194in}{3.081862in}}%
\pgfpathlineto{\pgfqpoint{2.199093in}{3.062106in}}%
\pgfpathclose%
\pgfusepath{fill}%
\end{pgfscope}%
\begin{pgfscope}%
\pgfpathrectangle{\pgfqpoint{1.150000in}{0.150000in}}{\pgfqpoint{5.700000in}{5.700000in}}%
\pgfusepath{clip}%
\pgfsetbuttcap%
\pgfsetroundjoin%
\definecolor{currentfill}{rgb}{0.206756,0.371758,0.553117}%
\pgfsetfillcolor{currentfill}%
\pgfsetfillopacity{0.700000}%
\pgfsetlinewidth{0.000000pt}%
\definecolor{currentstroke}{rgb}{0.000000,0.000000,0.000000}%
\pgfsetstrokecolor{currentstroke}%
\pgfsetdash{}{0pt}%
\pgfpathmoveto{\pgfqpoint{5.255221in}{2.121010in}}%
\pgfpathlineto{\pgfqpoint{5.269941in}{2.127505in}}%
\pgfpathlineto{\pgfqpoint{5.284677in}{2.134100in}}%
\pgfpathlineto{\pgfqpoint{5.299428in}{2.140793in}}%
\pgfpathlineto{\pgfqpoint{5.314193in}{2.147585in}}%
\pgfpathlineto{\pgfqpoint{5.306268in}{2.131375in}}%
\pgfpathlineto{\pgfqpoint{5.298338in}{2.115123in}}%
\pgfpathlineto{\pgfqpoint{5.290403in}{2.098831in}}%
\pgfpathlineto{\pgfqpoint{5.282464in}{2.082503in}}%
\pgfpathlineto{\pgfqpoint{5.267706in}{2.076050in}}%
\pgfpathlineto{\pgfqpoint{5.252963in}{2.069696in}}%
\pgfpathlineto{\pgfqpoint{5.238234in}{2.063440in}}%
\pgfpathlineto{\pgfqpoint{5.223520in}{2.057283in}}%
\pgfpathlineto{\pgfqpoint{5.231452in}{2.073265in}}%
\pgfpathlineto{\pgfqpoint{5.239379in}{2.089216in}}%
\pgfpathlineto{\pgfqpoint{5.247302in}{2.105132in}}%
\pgfpathlineto{\pgfqpoint{5.255221in}{2.121010in}}%
\pgfpathclose%
\pgfusepath{fill}%
\end{pgfscope}%
\begin{pgfscope}%
\pgfpathrectangle{\pgfqpoint{1.150000in}{0.150000in}}{\pgfqpoint{5.700000in}{5.700000in}}%
\pgfusepath{clip}%
\pgfsetbuttcap%
\pgfsetroundjoin%
\definecolor{currentfill}{rgb}{0.267968,0.223549,0.512008}%
\pgfsetfillcolor{currentfill}%
\pgfsetfillopacity{0.700000}%
\pgfsetlinewidth{0.000000pt}%
\definecolor{currentstroke}{rgb}{0.000000,0.000000,0.000000}%
\pgfsetstrokecolor{currentstroke}%
\pgfsetdash{}{0pt}%
\pgfpathmoveto{\pgfqpoint{3.334722in}{1.754304in}}%
\pgfpathlineto{\pgfqpoint{3.348846in}{1.743159in}}%
\pgfpathlineto{\pgfqpoint{3.362971in}{1.732125in}}%
\pgfpathlineto{\pgfqpoint{3.377097in}{1.721202in}}%
\pgfpathlineto{\pgfqpoint{3.391225in}{1.710388in}}%
\pgfpathlineto{\pgfqpoint{3.382572in}{1.715896in}}%
\pgfpathlineto{\pgfqpoint{3.373902in}{1.721802in}}%
\pgfpathlineto{\pgfqpoint{3.365216in}{1.728111in}}%
\pgfpathlineto{\pgfqpoint{3.356512in}{1.734832in}}%
\pgfpathlineto{\pgfqpoint{3.342342in}{1.746301in}}%
\pgfpathlineto{\pgfqpoint{3.328172in}{1.757880in}}%
\pgfpathlineto{\pgfqpoint{3.314004in}{1.769570in}}%
\pgfpathlineto{\pgfqpoint{3.299837in}{1.781372in}}%
\pgfpathlineto{\pgfqpoint{3.308585in}{1.773986in}}%
\pgfpathlineto{\pgfqpoint{3.317315in}{1.767018in}}%
\pgfpathlineto{\pgfqpoint{3.326027in}{1.760459in}}%
\pgfpathlineto{\pgfqpoint{3.334722in}{1.754304in}}%
\pgfpathclose%
\pgfusepath{fill}%
\end{pgfscope}%
\begin{pgfscope}%
\pgfpathrectangle{\pgfqpoint{1.150000in}{0.150000in}}{\pgfqpoint{5.700000in}{5.700000in}}%
\pgfusepath{clip}%
\pgfsetbuttcap%
\pgfsetroundjoin%
\definecolor{currentfill}{rgb}{0.283187,0.125848,0.444960}%
\pgfsetfillcolor{currentfill}%
\pgfsetfillopacity{0.700000}%
\pgfsetlinewidth{0.000000pt}%
\definecolor{currentstroke}{rgb}{0.000000,0.000000,0.000000}%
\pgfsetstrokecolor{currentstroke}%
\pgfsetdash{}{0pt}%
\pgfpathmoveto{\pgfqpoint{3.651417in}{1.543836in}}%
\pgfpathlineto{\pgfqpoint{3.665551in}{1.535456in}}%
\pgfpathlineto{\pgfqpoint{3.679689in}{1.527178in}}%
\pgfpathlineto{\pgfqpoint{3.693831in}{1.519003in}}%
\pgfpathlineto{\pgfqpoint{3.707976in}{1.510931in}}%
\pgfpathlineto{\pgfqpoint{3.699560in}{1.511819in}}%
\pgfpathlineto{\pgfqpoint{3.691132in}{1.513050in}}%
\pgfpathlineto{\pgfqpoint{3.682692in}{1.514629in}}%
\pgfpathlineto{\pgfqpoint{3.674239in}{1.516564in}}%
\pgfpathlineto{\pgfqpoint{3.660062in}{1.525262in}}%
\pgfpathlineto{\pgfqpoint{3.645888in}{1.534063in}}%
\pgfpathlineto{\pgfqpoint{3.631717in}{1.542967in}}%
\pgfpathlineto{\pgfqpoint{3.617549in}{1.551973in}}%
\pgfpathlineto{\pgfqpoint{3.626036in}{1.549405in}}%
\pgfpathlineto{\pgfqpoint{3.634509in}{1.547197in}}%
\pgfpathlineto{\pgfqpoint{3.642969in}{1.545343in}}%
\pgfpathlineto{\pgfqpoint{3.651417in}{1.543836in}}%
\pgfpathclose%
\pgfusepath{fill}%
\end{pgfscope}%
\begin{pgfscope}%
\pgfpathrectangle{\pgfqpoint{1.150000in}{0.150000in}}{\pgfqpoint{5.700000in}{5.700000in}}%
\pgfusepath{clip}%
\pgfsetbuttcap%
\pgfsetroundjoin%
\definecolor{currentfill}{rgb}{0.233603,0.313828,0.543914}%
\pgfsetfillcolor{currentfill}%
\pgfsetfillopacity{0.700000}%
\pgfsetlinewidth{0.000000pt}%
\definecolor{currentstroke}{rgb}{0.000000,0.000000,0.000000}%
\pgfsetstrokecolor{currentstroke}%
\pgfsetdash{}{0pt}%
\pgfpathmoveto{\pgfqpoint{5.133069in}{1.970872in}}%
\pgfpathlineto{\pgfqpoint{5.147718in}{1.976281in}}%
\pgfpathlineto{\pgfqpoint{5.162382in}{1.981788in}}%
\pgfpathlineto{\pgfqpoint{5.177060in}{1.987393in}}%
\pgfpathlineto{\pgfqpoint{5.191752in}{1.993096in}}%
\pgfpathlineto{\pgfqpoint{5.183800in}{1.977002in}}%
\pgfpathlineto{\pgfqpoint{5.175844in}{1.960895in}}%
\pgfpathlineto{\pgfqpoint{5.167885in}{1.944780in}}%
\pgfpathlineto{\pgfqpoint{5.159921in}{1.928659in}}%
\pgfpathlineto{\pgfqpoint{5.145236in}{1.923331in}}%
\pgfpathlineto{\pgfqpoint{5.130564in}{1.918101in}}%
\pgfpathlineto{\pgfqpoint{5.115907in}{1.912968in}}%
\pgfpathlineto{\pgfqpoint{5.101264in}{1.907932in}}%
\pgfpathlineto{\pgfqpoint{5.109221in}{1.923672in}}%
\pgfpathlineto{\pgfqpoint{5.117174in}{1.939411in}}%
\pgfpathlineto{\pgfqpoint{5.125123in}{1.955145in}}%
\pgfpathlineto{\pgfqpoint{5.133069in}{1.970872in}}%
\pgfpathclose%
\pgfusepath{fill}%
\end{pgfscope}%
\begin{pgfscope}%
\pgfpathrectangle{\pgfqpoint{1.150000in}{0.150000in}}{\pgfqpoint{5.700000in}{5.700000in}}%
\pgfusepath{clip}%
\pgfsetbuttcap%
\pgfsetroundjoin%
\definecolor{currentfill}{rgb}{0.282910,0.105393,0.426902}%
\pgfsetfillcolor{currentfill}%
\pgfsetfillopacity{0.700000}%
\pgfsetlinewidth{0.000000pt}%
\definecolor{currentstroke}{rgb}{0.000000,0.000000,0.000000}%
\pgfsetstrokecolor{currentstroke}%
\pgfsetdash{}{0pt}%
\pgfpathmoveto{\pgfqpoint{4.619499in}{1.511460in}}%
\pgfpathlineto{\pgfqpoint{4.633886in}{1.511975in}}%
\pgfpathlineto{\pgfqpoint{4.648283in}{1.512585in}}%
\pgfpathlineto{\pgfqpoint{4.662691in}{1.513290in}}%
\pgfpathlineto{\pgfqpoint{4.677109in}{1.514090in}}%
\pgfpathlineto{\pgfqpoint{4.669079in}{1.501555in}}%
\pgfpathlineto{\pgfqpoint{4.661046in}{1.489145in}}%
\pgfpathlineto{\pgfqpoint{4.653009in}{1.476863in}}%
\pgfpathlineto{\pgfqpoint{4.644968in}{1.464714in}}%
\pgfpathlineto{\pgfqpoint{4.630548in}{1.464408in}}%
\pgfpathlineto{\pgfqpoint{4.616137in}{1.464197in}}%
\pgfpathlineto{\pgfqpoint{4.601738in}{1.464080in}}%
\pgfpathlineto{\pgfqpoint{4.587348in}{1.464059in}}%
\pgfpathlineto{\pgfqpoint{4.595391in}{1.475707in}}%
\pgfpathlineto{\pgfqpoint{4.603431in}{1.487493in}}%
\pgfpathlineto{\pgfqpoint{4.611467in}{1.499412in}}%
\pgfpathlineto{\pgfqpoint{4.619499in}{1.511460in}}%
\pgfpathclose%
\pgfusepath{fill}%
\end{pgfscope}%
\begin{pgfscope}%
\pgfpathrectangle{\pgfqpoint{1.150000in}{0.150000in}}{\pgfqpoint{5.700000in}{5.700000in}}%
\pgfusepath{clip}%
\pgfsetbuttcap%
\pgfsetroundjoin%
\definecolor{currentfill}{rgb}{0.277018,0.050344,0.375715}%
\pgfsetfillcolor{currentfill}%
\pgfsetfillopacity{0.700000}%
\pgfsetlinewidth{0.000000pt}%
\definecolor{currentstroke}{rgb}{0.000000,0.000000,0.000000}%
\pgfsetstrokecolor{currentstroke}%
\pgfsetdash{}{0pt}%
\pgfpathmoveto{\pgfqpoint{4.057740in}{1.398692in}}%
\pgfpathlineto{\pgfqpoint{4.071942in}{1.393893in}}%
\pgfpathlineto{\pgfqpoint{4.086152in}{1.389191in}}%
\pgfpathlineto{\pgfqpoint{4.100367in}{1.384586in}}%
\pgfpathlineto{\pgfqpoint{4.114589in}{1.380077in}}%
\pgfpathlineto{\pgfqpoint{4.106394in}{1.375010in}}%
\pgfpathlineto{\pgfqpoint{4.098192in}{1.370204in}}%
\pgfpathlineto{\pgfqpoint{4.089983in}{1.365666in}}%
\pgfpathlineto{\pgfqpoint{4.081766in}{1.361402in}}%
\pgfpathlineto{\pgfqpoint{4.067525in}{1.366493in}}%
\pgfpathlineto{\pgfqpoint{4.053290in}{1.371680in}}%
\pgfpathlineto{\pgfqpoint{4.039062in}{1.376964in}}%
\pgfpathlineto{\pgfqpoint{4.024839in}{1.382346in}}%
\pgfpathlineto{\pgfqpoint{4.033076in}{1.386020in}}%
\pgfpathlineto{\pgfqpoint{4.041305in}{1.389974in}}%
\pgfpathlineto{\pgfqpoint{4.049526in}{1.394200in}}%
\pgfpathlineto{\pgfqpoint{4.057740in}{1.398692in}}%
\pgfpathclose%
\pgfusepath{fill}%
\end{pgfscope}%
\begin{pgfscope}%
\pgfpathrectangle{\pgfqpoint{1.150000in}{0.150000in}}{\pgfqpoint{5.700000in}{5.700000in}}%
\pgfusepath{clip}%
\pgfsetbuttcap%
\pgfsetroundjoin%
\definecolor{currentfill}{rgb}{0.281446,0.084320,0.407414}%
\pgfsetfillcolor{currentfill}%
\pgfsetfillopacity{0.700000}%
\pgfsetlinewidth{0.000000pt}%
\definecolor{currentstroke}{rgb}{0.000000,0.000000,0.000000}%
\pgfsetstrokecolor{currentstroke}%
\pgfsetdash{}{0pt}%
\pgfpathmoveto{\pgfqpoint{4.529888in}{1.464920in}}%
\pgfpathlineto{\pgfqpoint{4.544238in}{1.464562in}}%
\pgfpathlineto{\pgfqpoint{4.558598in}{1.464300in}}%
\pgfpathlineto{\pgfqpoint{4.572968in}{1.464132in}}%
\pgfpathlineto{\pgfqpoint{4.587348in}{1.464059in}}%
\pgfpathlineto{\pgfqpoint{4.579300in}{1.452553in}}%
\pgfpathlineto{\pgfqpoint{4.571249in}{1.441194in}}%
\pgfpathlineto{\pgfqpoint{4.563194in}{1.429989in}}%
\pgfpathlineto{\pgfqpoint{4.555136in}{1.418940in}}%
\pgfpathlineto{\pgfqpoint{4.540752in}{1.419524in}}%
\pgfpathlineto{\pgfqpoint{4.526377in}{1.420203in}}%
\pgfpathlineto{\pgfqpoint{4.512012in}{1.420976in}}%
\pgfpathlineto{\pgfqpoint{4.497656in}{1.421844in}}%
\pgfpathlineto{\pgfqpoint{4.505720in}{1.432375in}}%
\pgfpathlineto{\pgfqpoint{4.513780in}{1.443068in}}%
\pgfpathlineto{\pgfqpoint{4.521836in}{1.453918in}}%
\pgfpathlineto{\pgfqpoint{4.529888in}{1.464920in}}%
\pgfpathclose%
\pgfusepath{fill}%
\end{pgfscope}%
\begin{pgfscope}%
\pgfpathrectangle{\pgfqpoint{1.150000in}{0.150000in}}{\pgfqpoint{5.700000in}{5.700000in}}%
\pgfusepath{clip}%
\pgfsetbuttcap%
\pgfsetroundjoin%
\definecolor{currentfill}{rgb}{0.271828,0.209303,0.504434}%
\pgfsetfillcolor{currentfill}%
\pgfsetfillopacity{0.700000}%
\pgfsetlinewidth{0.000000pt}%
\definecolor{currentstroke}{rgb}{0.000000,0.000000,0.000000}%
\pgfsetstrokecolor{currentstroke}%
\pgfsetdash{}{0pt}%
\pgfpathmoveto{\pgfqpoint{3.391225in}{1.710388in}}%
\pgfpathlineto{\pgfqpoint{3.405354in}{1.699684in}}%
\pgfpathlineto{\pgfqpoint{3.419485in}{1.689088in}}%
\pgfpathlineto{\pgfqpoint{3.433618in}{1.678601in}}%
\pgfpathlineto{\pgfqpoint{3.447752in}{1.668223in}}%
\pgfpathlineto{\pgfqpoint{3.439140in}{1.673086in}}%
\pgfpathlineto{\pgfqpoint{3.430512in}{1.678340in}}%
\pgfpathlineto{\pgfqpoint{3.421868in}{1.683992in}}%
\pgfpathlineto{\pgfqpoint{3.413207in}{1.690051in}}%
\pgfpathlineto{\pgfqpoint{3.399031in}{1.701083in}}%
\pgfpathlineto{\pgfqpoint{3.384857in}{1.712223in}}%
\pgfpathlineto{\pgfqpoint{3.370684in}{1.723473in}}%
\pgfpathlineto{\pgfqpoint{3.356512in}{1.734832in}}%
\pgfpathlineto{\pgfqpoint{3.365216in}{1.728111in}}%
\pgfpathlineto{\pgfqpoint{3.373902in}{1.721802in}}%
\pgfpathlineto{\pgfqpoint{3.382572in}{1.715896in}}%
\pgfpathlineto{\pgfqpoint{3.391225in}{1.710388in}}%
\pgfpathclose%
\pgfusepath{fill}%
\end{pgfscope}%
\begin{pgfscope}%
\pgfpathrectangle{\pgfqpoint{1.150000in}{0.150000in}}{\pgfqpoint{5.700000in}{5.700000in}}%
\pgfusepath{clip}%
\pgfsetbuttcap%
\pgfsetroundjoin%
\definecolor{currentfill}{rgb}{0.282884,0.135920,0.453427}%
\pgfsetfillcolor{currentfill}%
\pgfsetfillopacity{0.700000}%
\pgfsetlinewidth{0.000000pt}%
\definecolor{currentstroke}{rgb}{0.000000,0.000000,0.000000}%
\pgfsetstrokecolor{currentstroke}%
\pgfsetdash{}{0pt}%
\pgfpathmoveto{\pgfqpoint{4.709195in}{1.565371in}}%
\pgfpathlineto{\pgfqpoint{4.723624in}{1.566742in}}%
\pgfpathlineto{\pgfqpoint{4.738063in}{1.568209in}}%
\pgfpathlineto{\pgfqpoint{4.752513in}{1.569770in}}%
\pgfpathlineto{\pgfqpoint{4.766975in}{1.571427in}}%
\pgfpathlineto{\pgfqpoint{4.758959in}{1.557974in}}%
\pgfpathlineto{\pgfqpoint{4.750939in}{1.544622in}}%
\pgfpathlineto{\pgfqpoint{4.742917in}{1.531375in}}%
\pgfpathlineto{\pgfqpoint{4.734891in}{1.518238in}}%
\pgfpathlineto{\pgfqpoint{4.720429in}{1.517059in}}%
\pgfpathlineto{\pgfqpoint{4.705978in}{1.515974in}}%
\pgfpathlineto{\pgfqpoint{4.691538in}{1.514984in}}%
\pgfpathlineto{\pgfqpoint{4.677109in}{1.514090in}}%
\pgfpathlineto{\pgfqpoint{4.685136in}{1.526743in}}%
\pgfpathlineto{\pgfqpoint{4.693159in}{1.539511in}}%
\pgfpathlineto{\pgfqpoint{4.701179in}{1.552388in}}%
\pgfpathlineto{\pgfqpoint{4.709195in}{1.565371in}}%
\pgfpathclose%
\pgfusepath{fill}%
\end{pgfscope}%
\begin{pgfscope}%
\pgfpathrectangle{\pgfqpoint{1.150000in}{0.150000in}}{\pgfqpoint{5.700000in}{5.700000in}}%
\pgfusepath{clip}%
\pgfsetbuttcap%
\pgfsetroundjoin%
\definecolor{currentfill}{rgb}{0.276022,0.044167,0.370164}%
\pgfsetfillcolor{currentfill}%
\pgfsetfillopacity{0.700000}%
\pgfsetlinewidth{0.000000pt}%
\definecolor{currentstroke}{rgb}{0.000000,0.000000,0.000000}%
\pgfsetstrokecolor{currentstroke}%
\pgfsetdash{}{0pt}%
\pgfpathmoveto{\pgfqpoint{4.204190in}{1.388028in}}%
\pgfpathlineto{\pgfqpoint{4.218431in}{1.384564in}}%
\pgfpathlineto{\pgfqpoint{4.232680in}{1.381195in}}%
\pgfpathlineto{\pgfqpoint{4.246936in}{1.377923in}}%
\pgfpathlineto{\pgfqpoint{4.261199in}{1.374746in}}%
\pgfpathlineto{\pgfqpoint{4.253062in}{1.367579in}}%
\pgfpathlineto{\pgfqpoint{4.244918in}{1.360640in}}%
\pgfpathlineto{\pgfqpoint{4.236769in}{1.353937in}}%
\pgfpathlineto{\pgfqpoint{4.228613in}{1.347473in}}%
\pgfpathlineto{\pgfqpoint{4.214335in}{1.351213in}}%
\pgfpathlineto{\pgfqpoint{4.200065in}{1.355049in}}%
\pgfpathlineto{\pgfqpoint{4.185802in}{1.358980in}}%
\pgfpathlineto{\pgfqpoint{4.171545in}{1.363007in}}%
\pgfpathlineto{\pgfqpoint{4.179716in}{1.368900in}}%
\pgfpathlineto{\pgfqpoint{4.187880in}{1.375039in}}%
\pgfpathlineto{\pgfqpoint{4.196038in}{1.381416in}}%
\pgfpathlineto{\pgfqpoint{4.204190in}{1.388028in}}%
\pgfpathclose%
\pgfusepath{fill}%
\end{pgfscope}%
\begin{pgfscope}%
\pgfpathrectangle{\pgfqpoint{1.150000in}{0.150000in}}{\pgfqpoint{5.700000in}{5.700000in}}%
\pgfusepath{clip}%
\pgfsetbuttcap%
\pgfsetroundjoin%
\definecolor{currentfill}{rgb}{0.252899,0.742211,0.448284}%
\pgfsetfillcolor{currentfill}%
\pgfsetfillopacity{0.700000}%
\pgfsetlinewidth{0.000000pt}%
\definecolor{currentstroke}{rgb}{0.000000,0.000000,0.000000}%
\pgfsetstrokecolor{currentstroke}%
\pgfsetdash{}{0pt}%
\pgfpathmoveto{\pgfqpoint{2.141372in}{3.153550in}}%
\pgfpathlineto{\pgfqpoint{2.155817in}{3.130402in}}%
\pgfpathlineto{\pgfqpoint{2.170252in}{3.107447in}}%
\pgfpathlineto{\pgfqpoint{2.184678in}{3.084682in}}%
\pgfpathlineto{\pgfqpoint{2.199093in}{3.062106in}}%
\pgfpathlineto{\pgfqpoint{2.189194in}{3.081862in}}%
\pgfpathlineto{\pgfqpoint{2.179259in}{3.102153in}}%
\pgfpathlineto{\pgfqpoint{2.169288in}{3.122988in}}%
\pgfpathlineto{\pgfqpoint{2.159280in}{3.144375in}}%
\pgfpathlineto{\pgfqpoint{2.144784in}{3.167690in}}%
\pgfpathlineto{\pgfqpoint{2.130277in}{3.191196in}}%
\pgfpathlineto{\pgfqpoint{2.115759in}{3.214894in}}%
\pgfpathlineto{\pgfqpoint{2.101231in}{3.238786in}}%
\pgfpathlineto{\pgfqpoint{2.111323in}{3.216646in}}%
\pgfpathlineto{\pgfqpoint{2.121376in}{3.195065in}}%
\pgfpathlineto{\pgfqpoint{2.131392in}{3.174036in}}%
\pgfpathlineto{\pgfqpoint{2.141372in}{3.153550in}}%
\pgfpathclose%
\pgfusepath{fill}%
\end{pgfscope}%
\begin{pgfscope}%
\pgfpathrectangle{\pgfqpoint{1.150000in}{0.150000in}}{\pgfqpoint{5.700000in}{5.700000in}}%
\pgfusepath{clip}%
\pgfsetbuttcap%
\pgfsetroundjoin%
\definecolor{currentfill}{rgb}{0.257322,0.256130,0.526563}%
\pgfsetfillcolor{currentfill}%
\pgfsetfillopacity{0.700000}%
\pgfsetlinewidth{0.000000pt}%
\definecolor{currentstroke}{rgb}{0.000000,0.000000,0.000000}%
\pgfsetstrokecolor{currentstroke}%
\pgfsetdash{}{0pt}%
\pgfpathmoveto{\pgfqpoint{5.010985in}{1.827431in}}%
\pgfpathlineto{\pgfqpoint{5.025568in}{1.831688in}}%
\pgfpathlineto{\pgfqpoint{5.040165in}{1.836041in}}%
\pgfpathlineto{\pgfqpoint{5.054776in}{1.840491in}}%
\pgfpathlineto{\pgfqpoint{5.069400in}{1.845038in}}%
\pgfpathlineto{\pgfqpoint{5.061425in}{1.829348in}}%
\pgfpathlineto{\pgfqpoint{5.053447in}{1.813679in}}%
\pgfpathlineto{\pgfqpoint{5.045465in}{1.798035in}}%
\pgfpathlineto{\pgfqpoint{5.037481in}{1.782419in}}%
\pgfpathlineto{\pgfqpoint{5.022862in}{1.778283in}}%
\pgfpathlineto{\pgfqpoint{5.008257in}{1.774242in}}%
\pgfpathlineto{\pgfqpoint{4.993665in}{1.770298in}}%
\pgfpathlineto{\pgfqpoint{4.979086in}{1.766451in}}%
\pgfpathlineto{\pgfqpoint{4.987066in}{1.781650in}}%
\pgfpathlineto{\pgfqpoint{4.995042in}{1.796882in}}%
\pgfpathlineto{\pgfqpoint{5.003015in}{1.812144in}}%
\pgfpathlineto{\pgfqpoint{5.010985in}{1.827431in}}%
\pgfpathclose%
\pgfusepath{fill}%
\end{pgfscope}%
\begin{pgfscope}%
\pgfpathrectangle{\pgfqpoint{1.150000in}{0.150000in}}{\pgfqpoint{5.700000in}{5.700000in}}%
\pgfusepath{clip}%
\pgfsetbuttcap%
\pgfsetroundjoin%
\definecolor{currentfill}{rgb}{0.279566,0.067836,0.391917}%
\pgfsetfillcolor{currentfill}%
\pgfsetfillopacity{0.700000}%
\pgfsetlinewidth{0.000000pt}%
\definecolor{currentstroke}{rgb}{0.000000,0.000000,0.000000}%
\pgfsetstrokecolor{currentstroke}%
\pgfsetdash{}{0pt}%
\pgfpathmoveto{\pgfqpoint{4.440327in}{1.426264in}}%
\pgfpathlineto{\pgfqpoint{4.454645in}{1.425017in}}%
\pgfpathlineto{\pgfqpoint{4.468973in}{1.423864in}}%
\pgfpathlineto{\pgfqpoint{4.483310in}{1.422807in}}%
\pgfpathlineto{\pgfqpoint{4.497656in}{1.421844in}}%
\pgfpathlineto{\pgfqpoint{4.489588in}{1.411480in}}%
\pgfpathlineto{\pgfqpoint{4.481515in}{1.401288in}}%
\pgfpathlineto{\pgfqpoint{4.473438in}{1.391274in}}%
\pgfpathlineto{\pgfqpoint{4.465357in}{1.381442in}}%
\pgfpathlineto{\pgfqpoint{4.451004in}{1.382933in}}%
\pgfpathlineto{\pgfqpoint{4.436660in}{1.384518in}}%
\pgfpathlineto{\pgfqpoint{4.422324in}{1.386198in}}%
\pgfpathlineto{\pgfqpoint{4.407998in}{1.387973in}}%
\pgfpathlineto{\pgfqpoint{4.416087in}{1.397271in}}%
\pgfpathlineto{\pgfqpoint{4.424172in}{1.406755in}}%
\pgfpathlineto{\pgfqpoint{4.432252in}{1.416421in}}%
\pgfpathlineto{\pgfqpoint{4.440327in}{1.426264in}}%
\pgfpathclose%
\pgfusepath{fill}%
\end{pgfscope}%
\begin{pgfscope}%
\pgfpathrectangle{\pgfqpoint{1.150000in}{0.150000in}}{\pgfqpoint{5.700000in}{5.700000in}}%
\pgfusepath{clip}%
\pgfsetbuttcap%
\pgfsetroundjoin%
\definecolor{currentfill}{rgb}{0.280267,0.073417,0.397163}%
\pgfsetfillcolor{currentfill}%
\pgfsetfillopacity{0.700000}%
\pgfsetlinewidth{0.000000pt}%
\definecolor{currentstroke}{rgb}{0.000000,0.000000,0.000000}%
\pgfsetstrokecolor{currentstroke}%
\pgfsetdash{}{0pt}%
\pgfpathmoveto{\pgfqpoint{3.911266in}{1.428913in}}%
\pgfpathlineto{\pgfqpoint{3.925443in}{1.422748in}}%
\pgfpathlineto{\pgfqpoint{3.939626in}{1.416682in}}%
\pgfpathlineto{\pgfqpoint{3.953814in}{1.410715in}}%
\pgfpathlineto{\pgfqpoint{3.968008in}{1.404845in}}%
\pgfpathlineto{\pgfqpoint{3.959740in}{1.402049in}}%
\pgfpathlineto{\pgfqpoint{3.951464in}{1.399549in}}%
\pgfpathlineto{\pgfqpoint{3.943179in}{1.397351in}}%
\pgfpathlineto{\pgfqpoint{3.934884in}{1.395461in}}%
\pgfpathlineto{\pgfqpoint{3.920667in}{1.401932in}}%
\pgfpathlineto{\pgfqpoint{3.906454in}{1.408502in}}%
\pgfpathlineto{\pgfqpoint{3.892247in}{1.415171in}}%
\pgfpathlineto{\pgfqpoint{3.878045in}{1.421938in}}%
\pgfpathlineto{\pgfqpoint{3.886365in}{1.423218in}}%
\pgfpathlineto{\pgfqpoint{3.894675in}{1.424811in}}%
\pgfpathlineto{\pgfqpoint{3.902975in}{1.426712in}}%
\pgfpathlineto{\pgfqpoint{3.911266in}{1.428913in}}%
\pgfpathclose%
\pgfusepath{fill}%
\end{pgfscope}%
\begin{pgfscope}%
\pgfpathrectangle{\pgfqpoint{1.150000in}{0.150000in}}{\pgfqpoint{5.700000in}{5.700000in}}%
\pgfusepath{clip}%
\pgfsetbuttcap%
\pgfsetroundjoin%
\definecolor{currentfill}{rgb}{0.188923,0.410910,0.556326}%
\pgfsetfillcolor{currentfill}%
\pgfsetfillopacity{0.700000}%
\pgfsetlinewidth{0.000000pt}%
\definecolor{currentstroke}{rgb}{0.000000,0.000000,0.000000}%
\pgfsetstrokecolor{currentstroke}%
\pgfsetdash{}{0pt}%
\pgfpathmoveto{\pgfqpoint{5.345849in}{2.211935in}}%
\pgfpathlineto{\pgfqpoint{5.360637in}{2.219147in}}%
\pgfpathlineto{\pgfqpoint{5.375442in}{2.226458in}}%
\pgfpathlineto{\pgfqpoint{5.390262in}{2.233869in}}%
\pgfpathlineto{\pgfqpoint{5.382349in}{2.217625in}}%
\pgfpathlineto{\pgfqpoint{5.374432in}{2.201323in}}%
\pgfpathlineto{\pgfqpoint{5.366510in}{2.184965in}}%
\pgfpathlineto{\pgfqpoint{5.358583in}{2.168555in}}%
\pgfpathlineto{\pgfqpoint{5.343771in}{2.161466in}}%
\pgfpathlineto{\pgfqpoint{5.328975in}{2.154476in}}%
\pgfpathlineto{\pgfqpoint{5.314193in}{2.147585in}}%
\pgfpathlineto{\pgfqpoint{5.322114in}{2.163749in}}%
\pgfpathlineto{\pgfqpoint{5.330031in}{2.179863in}}%
\pgfpathlineto{\pgfqpoint{5.337942in}{2.195927in}}%
\pgfpathlineto{\pgfqpoint{5.345849in}{2.211935in}}%
\pgfpathclose%
\pgfusepath{fill}%
\end{pgfscope}%
\begin{pgfscope}%
\pgfpathrectangle{\pgfqpoint{1.150000in}{0.150000in}}{\pgfqpoint{5.700000in}{5.700000in}}%
\pgfusepath{clip}%
\pgfsetbuttcap%
\pgfsetroundjoin%
\definecolor{currentfill}{rgb}{0.283197,0.115680,0.436115}%
\pgfsetfillcolor{currentfill}%
\pgfsetfillopacity{0.700000}%
\pgfsetlinewidth{0.000000pt}%
\definecolor{currentstroke}{rgb}{0.000000,0.000000,0.000000}%
\pgfsetstrokecolor{currentstroke}%
\pgfsetdash{}{0pt}%
\pgfpathmoveto{\pgfqpoint{3.707976in}{1.510931in}}%
\pgfpathlineto{\pgfqpoint{3.722126in}{1.502960in}}%
\pgfpathlineto{\pgfqpoint{3.736279in}{1.495091in}}%
\pgfpathlineto{\pgfqpoint{3.750436in}{1.487324in}}%
\pgfpathlineto{\pgfqpoint{3.764597in}{1.479657in}}%
\pgfpathlineto{\pgfqpoint{3.756211in}{1.479929in}}%
\pgfpathlineto{\pgfqpoint{3.747814in}{1.480537in}}%
\pgfpathlineto{\pgfqpoint{3.739405in}{1.481489in}}%
\pgfpathlineto{\pgfqpoint{3.730985in}{1.482790in}}%
\pgfpathlineto{\pgfqpoint{3.716793in}{1.491081in}}%
\pgfpathlineto{\pgfqpoint{3.702605in}{1.499474in}}%
\pgfpathlineto{\pgfqpoint{3.688420in}{1.507968in}}%
\pgfpathlineto{\pgfqpoint{3.674239in}{1.516564in}}%
\pgfpathlineto{\pgfqpoint{3.682692in}{1.514629in}}%
\pgfpathlineto{\pgfqpoint{3.691132in}{1.513050in}}%
\pgfpathlineto{\pgfqpoint{3.699560in}{1.511819in}}%
\pgfpathlineto{\pgfqpoint{3.707976in}{1.510931in}}%
\pgfpathclose%
\pgfusepath{fill}%
\end{pgfscope}%
\begin{pgfscope}%
\pgfpathrectangle{\pgfqpoint{1.150000in}{0.150000in}}{\pgfqpoint{5.700000in}{5.700000in}}%
\pgfusepath{clip}%
\pgfsetbuttcap%
\pgfsetroundjoin%
\definecolor{currentfill}{rgb}{0.280255,0.165693,0.476498}%
\pgfsetfillcolor{currentfill}%
\pgfsetfillopacity{0.700000}%
\pgfsetlinewidth{0.000000pt}%
\definecolor{currentstroke}{rgb}{0.000000,0.000000,0.000000}%
\pgfsetstrokecolor{currentstroke}%
\pgfsetdash{}{0pt}%
\pgfpathmoveto{\pgfqpoint{4.799007in}{1.626158in}}%
\pgfpathlineto{\pgfqpoint{4.813481in}{1.628370in}}%
\pgfpathlineto{\pgfqpoint{4.827967in}{1.630677in}}%
\pgfpathlineto{\pgfqpoint{4.842465in}{1.633080in}}%
\pgfpathlineto{\pgfqpoint{4.856975in}{1.635578in}}%
\pgfpathlineto{\pgfqpoint{4.848970in}{1.621313in}}%
\pgfpathlineto{\pgfqpoint{4.840962in}{1.607126in}}%
\pgfpathlineto{\pgfqpoint{4.832951in}{1.593023in}}%
\pgfpathlineto{\pgfqpoint{4.824936in}{1.579006in}}%
\pgfpathlineto{\pgfqpoint{4.810429in}{1.576969in}}%
\pgfpathlineto{\pgfqpoint{4.795933in}{1.575027in}}%
\pgfpathlineto{\pgfqpoint{4.781448in}{1.573179in}}%
\pgfpathlineto{\pgfqpoint{4.766975in}{1.571427in}}%
\pgfpathlineto{\pgfqpoint{4.774988in}{1.584977in}}%
\pgfpathlineto{\pgfqpoint{4.782998in}{1.598618in}}%
\pgfpathlineto{\pgfqpoint{4.791004in}{1.612346in}}%
\pgfpathlineto{\pgfqpoint{4.799007in}{1.626158in}}%
\pgfpathclose%
\pgfusepath{fill}%
\end{pgfscope}%
\begin{pgfscope}%
\pgfpathrectangle{\pgfqpoint{1.150000in}{0.150000in}}{\pgfqpoint{5.700000in}{5.700000in}}%
\pgfusepath{clip}%
\pgfsetbuttcap%
\pgfsetroundjoin%
\definecolor{currentfill}{rgb}{0.275191,0.194905,0.496005}%
\pgfsetfillcolor{currentfill}%
\pgfsetfillopacity{0.700000}%
\pgfsetlinewidth{0.000000pt}%
\definecolor{currentstroke}{rgb}{0.000000,0.000000,0.000000}%
\pgfsetstrokecolor{currentstroke}%
\pgfsetdash{}{0pt}%
\pgfpathmoveto{\pgfqpoint{3.447752in}{1.668223in}}%
\pgfpathlineto{\pgfqpoint{3.461889in}{1.657952in}}%
\pgfpathlineto{\pgfqpoint{3.476028in}{1.647788in}}%
\pgfpathlineto{\pgfqpoint{3.490169in}{1.637731in}}%
\pgfpathlineto{\pgfqpoint{3.504312in}{1.627781in}}%
\pgfpathlineto{\pgfqpoint{3.495738in}{1.632000in}}%
\pgfpathlineto{\pgfqpoint{3.487150in}{1.636605in}}%
\pgfpathlineto{\pgfqpoint{3.478546in}{1.641603in}}%
\pgfpathlineto{\pgfqpoint{3.469927in}{1.647000in}}%
\pgfpathlineto{\pgfqpoint{3.455744in}{1.657602in}}%
\pgfpathlineto{\pgfqpoint{3.441563in}{1.668311in}}%
\pgfpathlineto{\pgfqpoint{3.427384in}{1.679127in}}%
\pgfpathlineto{\pgfqpoint{3.413207in}{1.690051in}}%
\pgfpathlineto{\pgfqpoint{3.421868in}{1.683992in}}%
\pgfpathlineto{\pgfqpoint{3.430512in}{1.678340in}}%
\pgfpathlineto{\pgfqpoint{3.439140in}{1.673086in}}%
\pgfpathlineto{\pgfqpoint{3.447752in}{1.668223in}}%
\pgfpathclose%
\pgfusepath{fill}%
\end{pgfscope}%
\begin{pgfscope}%
\pgfpathrectangle{\pgfqpoint{1.150000in}{0.150000in}}{\pgfqpoint{5.700000in}{5.700000in}}%
\pgfusepath{clip}%
\pgfsetbuttcap%
\pgfsetroundjoin%
\definecolor{currentfill}{rgb}{0.277941,0.056324,0.381191}%
\pgfsetfillcolor{currentfill}%
\pgfsetfillopacity{0.700000}%
\pgfsetlinewidth{0.000000pt}%
\definecolor{currentstroke}{rgb}{0.000000,0.000000,0.000000}%
\pgfsetstrokecolor{currentstroke}%
\pgfsetdash{}{0pt}%
\pgfpathmoveto{\pgfqpoint{4.350778in}{1.396023in}}%
\pgfpathlineto{\pgfqpoint{4.365070in}{1.393868in}}%
\pgfpathlineto{\pgfqpoint{4.379371in}{1.391808in}}%
\pgfpathlineto{\pgfqpoint{4.393680in}{1.389843in}}%
\pgfpathlineto{\pgfqpoint{4.407998in}{1.387973in}}%
\pgfpathlineto{\pgfqpoint{4.399904in}{1.378869in}}%
\pgfpathlineto{\pgfqpoint{4.391805in}{1.369961in}}%
\pgfpathlineto{\pgfqpoint{4.383701in}{1.361258in}}%
\pgfpathlineto{\pgfqpoint{4.375593in}{1.352762in}}%
\pgfpathlineto{\pgfqpoint{4.361265in}{1.355178in}}%
\pgfpathlineto{\pgfqpoint{4.346946in}{1.357688in}}%
\pgfpathlineto{\pgfqpoint{4.332635in}{1.360293in}}%
\pgfpathlineto{\pgfqpoint{4.318332in}{1.362994in}}%
\pgfpathlineto{\pgfqpoint{4.326451in}{1.370936in}}%
\pgfpathlineto{\pgfqpoint{4.334565in}{1.379093in}}%
\pgfpathlineto{\pgfqpoint{4.342674in}{1.387456in}}%
\pgfpathlineto{\pgfqpoint{4.350778in}{1.396023in}}%
\pgfpathclose%
\pgfusepath{fill}%
\end{pgfscope}%
\begin{pgfscope}%
\pgfpathrectangle{\pgfqpoint{1.150000in}{0.150000in}}{\pgfqpoint{5.700000in}{5.700000in}}%
\pgfusepath{clip}%
\pgfsetbuttcap%
\pgfsetroundjoin%
\definecolor{currentfill}{rgb}{0.214298,0.355619,0.551184}%
\pgfsetfillcolor{currentfill}%
\pgfsetfillopacity{0.700000}%
\pgfsetlinewidth{0.000000pt}%
\definecolor{currentstroke}{rgb}{0.000000,0.000000,0.000000}%
\pgfsetstrokecolor{currentstroke}%
\pgfsetdash{}{0pt}%
\pgfpathmoveto{\pgfqpoint{5.223520in}{2.057283in}}%
\pgfpathlineto{\pgfqpoint{5.238234in}{2.063440in}}%
\pgfpathlineto{\pgfqpoint{5.252963in}{2.069696in}}%
\pgfpathlineto{\pgfqpoint{5.267706in}{2.076050in}}%
\pgfpathlineto{\pgfqpoint{5.282464in}{2.082503in}}%
\pgfpathlineto{\pgfqpoint{5.274521in}{2.066140in}}%
\pgfpathlineto{\pgfqpoint{5.266574in}{2.049748in}}%
\pgfpathlineto{\pgfqpoint{5.258622in}{2.033329in}}%
\pgfpathlineto{\pgfqpoint{5.250667in}{2.016886in}}%
\pgfpathlineto{\pgfqpoint{5.235916in}{2.010791in}}%
\pgfpathlineto{\pgfqpoint{5.221180in}{2.004795in}}%
\pgfpathlineto{\pgfqpoint{5.206459in}{1.998897in}}%
\pgfpathlineto{\pgfqpoint{5.191752in}{1.993096in}}%
\pgfpathlineto{\pgfqpoint{5.199700in}{2.009174in}}%
\pgfpathlineto{\pgfqpoint{5.207644in}{2.025234in}}%
\pgfpathlineto{\pgfqpoint{5.215584in}{2.041271in}}%
\pgfpathlineto{\pgfqpoint{5.223520in}{2.057283in}}%
\pgfpathclose%
\pgfusepath{fill}%
\end{pgfscope}%
\begin{pgfscope}%
\pgfpathrectangle{\pgfqpoint{1.150000in}{0.150000in}}{\pgfqpoint{5.700000in}{5.700000in}}%
\pgfusepath{clip}%
\pgfsetbuttcap%
\pgfsetroundjoin%
\definecolor{currentfill}{rgb}{0.319809,0.770914,0.411152}%
\pgfsetfillcolor{currentfill}%
\pgfsetfillopacity{0.700000}%
\pgfsetlinewidth{0.000000pt}%
\definecolor{currentstroke}{rgb}{0.000000,0.000000,0.000000}%
\pgfsetstrokecolor{currentstroke}%
\pgfsetdash{}{0pt}%
\pgfpathmoveto{\pgfqpoint{2.083482in}{3.248103in}}%
\pgfpathlineto{\pgfqpoint{2.097971in}{3.224167in}}%
\pgfpathlineto{\pgfqpoint{2.112449in}{3.200430in}}%
\pgfpathlineto{\pgfqpoint{2.126915in}{3.176892in}}%
\pgfpathlineto{\pgfqpoint{2.141372in}{3.153550in}}%
\pgfpathlineto{\pgfqpoint{2.131392in}{3.174036in}}%
\pgfpathlineto{\pgfqpoint{2.121376in}{3.195065in}}%
\pgfpathlineto{\pgfqpoint{2.111323in}{3.216646in}}%
\pgfpathlineto{\pgfqpoint{2.101231in}{3.238786in}}%
\pgfpathlineto{\pgfqpoint{2.086692in}{3.262875in}}%
\pgfpathlineto{\pgfqpoint{2.072141in}{3.287161in}}%
\pgfpathlineto{\pgfqpoint{2.057579in}{3.311648in}}%
\pgfpathlineto{\pgfqpoint{2.043005in}{3.336336in}}%
\pgfpathlineto{\pgfqpoint{2.053183in}{3.313434in}}%
\pgfpathlineto{\pgfqpoint{2.063321in}{3.291101in}}%
\pgfpathlineto{\pgfqpoint{2.073420in}{3.269326in}}%
\pgfpathlineto{\pgfqpoint{2.083482in}{3.248103in}}%
\pgfpathclose%
\pgfusepath{fill}%
\end{pgfscope}%
\begin{pgfscope}%
\pgfpathrectangle{\pgfqpoint{1.150000in}{0.150000in}}{\pgfqpoint{5.700000in}{5.700000in}}%
\pgfusepath{clip}%
\pgfsetbuttcap%
\pgfsetroundjoin%
\definecolor{currentfill}{rgb}{0.274128,0.199721,0.498911}%
\pgfsetfillcolor{currentfill}%
\pgfsetfillopacity{0.700000}%
\pgfsetlinewidth{0.000000pt}%
\definecolor{currentstroke}{rgb}{0.000000,0.000000,0.000000}%
\pgfsetstrokecolor{currentstroke}%
\pgfsetdash{}{0pt}%
\pgfpathmoveto{\pgfqpoint{4.888963in}{1.693340in}}%
\pgfpathlineto{\pgfqpoint{4.903487in}{1.696377in}}%
\pgfpathlineto{\pgfqpoint{4.918024in}{1.699510in}}%
\pgfpathlineto{\pgfqpoint{4.932573in}{1.702739in}}%
\pgfpathlineto{\pgfqpoint{4.947135in}{1.706064in}}%
\pgfpathlineto{\pgfqpoint{4.939140in}{1.691090in}}%
\pgfpathlineto{\pgfqpoint{4.931141in}{1.676173in}}%
\pgfpathlineto{\pgfqpoint{4.923139in}{1.661317in}}%
\pgfpathlineto{\pgfqpoint{4.915134in}{1.646527in}}%
\pgfpathlineto{\pgfqpoint{4.900576in}{1.643646in}}%
\pgfpathlineto{\pgfqpoint{4.886030in}{1.640861in}}%
\pgfpathlineto{\pgfqpoint{4.871496in}{1.638172in}}%
\pgfpathlineto{\pgfqpoint{4.856975in}{1.635578in}}%
\pgfpathlineto{\pgfqpoint{4.864976in}{1.649918in}}%
\pgfpathlineto{\pgfqpoint{4.872975in}{1.664327in}}%
\pgfpathlineto{\pgfqpoint{4.880971in}{1.678803in}}%
\pgfpathlineto{\pgfqpoint{4.888963in}{1.693340in}}%
\pgfpathclose%
\pgfusepath{fill}%
\end{pgfscope}%
\begin{pgfscope}%
\pgfpathrectangle{\pgfqpoint{1.150000in}{0.150000in}}{\pgfqpoint{5.700000in}{5.700000in}}%
\pgfusepath{clip}%
\pgfsetbuttcap%
\pgfsetroundjoin%
\definecolor{currentfill}{rgb}{0.174274,0.445044,0.557792}%
\pgfsetfillcolor{currentfill}%
\pgfsetfillopacity{0.700000}%
\pgfsetlinewidth{0.000000pt}%
\definecolor{currentstroke}{rgb}{0.000000,0.000000,0.000000}%
\pgfsetstrokecolor{currentstroke}%
\pgfsetdash{}{0pt}%
\pgfpathmoveto{\pgfqpoint{2.789525in}{2.284783in}}%
\pgfpathlineto{\pgfqpoint{2.803731in}{2.268603in}}%
\pgfpathlineto{\pgfqpoint{2.817934in}{2.252557in}}%
\pgfpathlineto{\pgfqpoint{2.832133in}{2.236644in}}%
\pgfpathlineto{\pgfqpoint{2.846331in}{2.220862in}}%
\pgfpathlineto{\pgfqpoint{2.837141in}{2.234132in}}%
\pgfpathlineto{\pgfqpoint{2.827926in}{2.247885in}}%
\pgfpathlineto{\pgfqpoint{2.818685in}{2.262128in}}%
\pgfpathlineto{\pgfqpoint{2.809417in}{2.276870in}}%
\pgfpathlineto{\pgfqpoint{2.795158in}{2.293357in}}%
\pgfpathlineto{\pgfqpoint{2.780896in}{2.309977in}}%
\pgfpathlineto{\pgfqpoint{2.766631in}{2.326731in}}%
\pgfpathlineto{\pgfqpoint{2.752362in}{2.343620in}}%
\pgfpathlineto{\pgfqpoint{2.761694in}{2.328160in}}%
\pgfpathlineto{\pgfqpoint{2.770997in}{2.313207in}}%
\pgfpathlineto{\pgfqpoint{2.780274in}{2.298750in}}%
\pgfpathlineto{\pgfqpoint{2.789525in}{2.284783in}}%
\pgfpathclose%
\pgfusepath{fill}%
\end{pgfscope}%
\begin{pgfscope}%
\pgfpathrectangle{\pgfqpoint{1.150000in}{0.150000in}}{\pgfqpoint{5.700000in}{5.700000in}}%
\pgfusepath{clip}%
\pgfsetbuttcap%
\pgfsetroundjoin%
\definecolor{currentfill}{rgb}{0.163625,0.471133,0.558148}%
\pgfsetfillcolor{currentfill}%
\pgfsetfillopacity{0.700000}%
\pgfsetlinewidth{0.000000pt}%
\definecolor{currentstroke}{rgb}{0.000000,0.000000,0.000000}%
\pgfsetstrokecolor{currentstroke}%
\pgfsetdash{}{0pt}%
\pgfpathmoveto{\pgfqpoint{2.732667in}{2.350857in}}%
\pgfpathlineto{\pgfqpoint{2.746887in}{2.334134in}}%
\pgfpathlineto{\pgfqpoint{2.761103in}{2.317548in}}%
\pgfpathlineto{\pgfqpoint{2.775315in}{2.301098in}}%
\pgfpathlineto{\pgfqpoint{2.789525in}{2.284783in}}%
\pgfpathlineto{\pgfqpoint{2.780274in}{2.298750in}}%
\pgfpathlineto{\pgfqpoint{2.770997in}{2.313207in}}%
\pgfpathlineto{\pgfqpoint{2.761694in}{2.328160in}}%
\pgfpathlineto{\pgfqpoint{2.752362in}{2.343620in}}%
\pgfpathlineto{\pgfqpoint{2.738090in}{2.360644in}}%
\pgfpathlineto{\pgfqpoint{2.723813in}{2.377805in}}%
\pgfpathlineto{\pgfqpoint{2.709533in}{2.395103in}}%
\pgfpathlineto{\pgfqpoint{2.695249in}{2.412539in}}%
\pgfpathlineto{\pgfqpoint{2.704646in}{2.396358in}}%
\pgfpathlineto{\pgfqpoint{2.714014in}{2.380689in}}%
\pgfpathlineto{\pgfqpoint{2.723354in}{2.365525in}}%
\pgfpathlineto{\pgfqpoint{2.732667in}{2.350857in}}%
\pgfpathclose%
\pgfusepath{fill}%
\end{pgfscope}%
\begin{pgfscope}%
\pgfpathrectangle{\pgfqpoint{1.150000in}{0.150000in}}{\pgfqpoint{5.700000in}{5.700000in}}%
\pgfusepath{clip}%
\pgfsetbuttcap%
\pgfsetroundjoin%
\definecolor{currentfill}{rgb}{0.241237,0.296485,0.539709}%
\pgfsetfillcolor{currentfill}%
\pgfsetfillopacity{0.700000}%
\pgfsetlinewidth{0.000000pt}%
\definecolor{currentstroke}{rgb}{0.000000,0.000000,0.000000}%
\pgfsetstrokecolor{currentstroke}%
\pgfsetdash{}{0pt}%
\pgfpathmoveto{\pgfqpoint{5.101264in}{1.907932in}}%
\pgfpathlineto{\pgfqpoint{5.115907in}{1.912968in}}%
\pgfpathlineto{\pgfqpoint{5.130564in}{1.918101in}}%
\pgfpathlineto{\pgfqpoint{5.145236in}{1.923331in}}%
\pgfpathlineto{\pgfqpoint{5.159921in}{1.928659in}}%
\pgfpathlineto{\pgfqpoint{5.151954in}{1.912536in}}%
\pgfpathlineto{\pgfqpoint{5.143984in}{1.896415in}}%
\pgfpathlineto{\pgfqpoint{5.136009in}{1.880300in}}%
\pgfpathlineto{\pgfqpoint{5.128032in}{1.864193in}}%
\pgfpathlineto{\pgfqpoint{5.113353in}{1.859259in}}%
\pgfpathlineto{\pgfqpoint{5.098688in}{1.854421in}}%
\pgfpathlineto{\pgfqpoint{5.084037in}{1.849681in}}%
\pgfpathlineto{\pgfqpoint{5.069400in}{1.845038in}}%
\pgfpathlineto{\pgfqpoint{5.077371in}{1.860745in}}%
\pgfpathlineto{\pgfqpoint{5.085339in}{1.876465in}}%
\pgfpathlineto{\pgfqpoint{5.093303in}{1.892196in}}%
\pgfpathlineto{\pgfqpoint{5.101264in}{1.907932in}}%
\pgfpathclose%
\pgfusepath{fill}%
\end{pgfscope}%
\begin{pgfscope}%
\pgfpathrectangle{\pgfqpoint{1.150000in}{0.150000in}}{\pgfqpoint{5.700000in}{5.700000in}}%
\pgfusepath{clip}%
\pgfsetbuttcap%
\pgfsetroundjoin%
\definecolor{currentfill}{rgb}{0.183898,0.422383,0.556944}%
\pgfsetfillcolor{currentfill}%
\pgfsetfillopacity{0.700000}%
\pgfsetlinewidth{0.000000pt}%
\definecolor{currentstroke}{rgb}{0.000000,0.000000,0.000000}%
\pgfsetstrokecolor{currentstroke}%
\pgfsetdash{}{0pt}%
\pgfpathmoveto{\pgfqpoint{2.846331in}{2.220862in}}%
\pgfpathlineto{\pgfqpoint{2.860525in}{2.205212in}}%
\pgfpathlineto{\pgfqpoint{2.874717in}{2.189693in}}%
\pgfpathlineto{\pgfqpoint{2.888906in}{2.174303in}}%
\pgfpathlineto{\pgfqpoint{2.903093in}{2.159042in}}%
\pgfpathlineto{\pgfqpoint{2.893962in}{2.171618in}}%
\pgfpathlineto{\pgfqpoint{2.884807in}{2.184671in}}%
\pgfpathlineto{\pgfqpoint{2.875627in}{2.198208in}}%
\pgfpathlineto{\pgfqpoint{2.866421in}{2.212237in}}%
\pgfpathlineto{\pgfqpoint{2.852174in}{2.228200in}}%
\pgfpathlineto{\pgfqpoint{2.837925in}{2.244292in}}%
\pgfpathlineto{\pgfqpoint{2.823672in}{2.260516in}}%
\pgfpathlineto{\pgfqpoint{2.809417in}{2.276870in}}%
\pgfpathlineto{\pgfqpoint{2.818685in}{2.262128in}}%
\pgfpathlineto{\pgfqpoint{2.827926in}{2.247885in}}%
\pgfpathlineto{\pgfqpoint{2.837141in}{2.234132in}}%
\pgfpathlineto{\pgfqpoint{2.846331in}{2.220862in}}%
\pgfpathclose%
\pgfusepath{fill}%
\end{pgfscope}%
\begin{pgfscope}%
\pgfpathrectangle{\pgfqpoint{1.150000in}{0.150000in}}{\pgfqpoint{5.700000in}{5.700000in}}%
\pgfusepath{clip}%
\pgfsetbuttcap%
\pgfsetroundjoin%
\definecolor{currentfill}{rgb}{0.154815,0.493313,0.557840}%
\pgfsetfillcolor{currentfill}%
\pgfsetfillopacity{0.700000}%
\pgfsetlinewidth{0.000000pt}%
\definecolor{currentstroke}{rgb}{0.000000,0.000000,0.000000}%
\pgfsetstrokecolor{currentstroke}%
\pgfsetdash{}{0pt}%
\pgfpathmoveto{\pgfqpoint{2.675750in}{2.419140in}}%
\pgfpathlineto{\pgfqpoint{2.689986in}{2.401859in}}%
\pgfpathlineto{\pgfqpoint{2.704217in}{2.384719in}}%
\pgfpathlineto{\pgfqpoint{2.718444in}{2.367719in}}%
\pgfpathlineto{\pgfqpoint{2.732667in}{2.350857in}}%
\pgfpathlineto{\pgfqpoint{2.723354in}{2.365525in}}%
\pgfpathlineto{\pgfqpoint{2.714014in}{2.380689in}}%
\pgfpathlineto{\pgfqpoint{2.704646in}{2.396358in}}%
\pgfpathlineto{\pgfqpoint{2.695249in}{2.412539in}}%
\pgfpathlineto{\pgfqpoint{2.680960in}{2.430114in}}%
\pgfpathlineto{\pgfqpoint{2.666667in}{2.447829in}}%
\pgfpathlineto{\pgfqpoint{2.652370in}{2.465685in}}%
\pgfpathlineto{\pgfqpoint{2.638068in}{2.483683in}}%
\pgfpathlineto{\pgfqpoint{2.647532in}{2.466776in}}%
\pgfpathlineto{\pgfqpoint{2.656967in}{2.450389in}}%
\pgfpathlineto{\pgfqpoint{2.666373in}{2.434513in}}%
\pgfpathlineto{\pgfqpoint{2.675750in}{2.419140in}}%
\pgfpathclose%
\pgfusepath{fill}%
\end{pgfscope}%
\begin{pgfscope}%
\pgfpathrectangle{\pgfqpoint{1.150000in}{0.150000in}}{\pgfqpoint{5.700000in}{5.700000in}}%
\pgfusepath{clip}%
\pgfsetbuttcap%
\pgfsetroundjoin%
\definecolor{currentfill}{rgb}{0.194100,0.399323,0.555565}%
\pgfsetfillcolor{currentfill}%
\pgfsetfillopacity{0.700000}%
\pgfsetlinewidth{0.000000pt}%
\definecolor{currentstroke}{rgb}{0.000000,0.000000,0.000000}%
\pgfsetstrokecolor{currentstroke}%
\pgfsetdash{}{0pt}%
\pgfpathmoveto{\pgfqpoint{2.903093in}{2.159042in}}%
\pgfpathlineto{\pgfqpoint{2.917278in}{2.143909in}}%
\pgfpathlineto{\pgfqpoint{2.931460in}{2.128904in}}%
\pgfpathlineto{\pgfqpoint{2.945641in}{2.114025in}}%
\pgfpathlineto{\pgfqpoint{2.959820in}{2.099273in}}%
\pgfpathlineto{\pgfqpoint{2.950747in}{2.111160in}}%
\pgfpathlineto{\pgfqpoint{2.941649in}{2.123516in}}%
\pgfpathlineto{\pgfqpoint{2.932528in}{2.136350in}}%
\pgfpathlineto{\pgfqpoint{2.923382in}{2.149669in}}%
\pgfpathlineto{\pgfqpoint{2.909145in}{2.165120in}}%
\pgfpathlineto{\pgfqpoint{2.894906in}{2.180698in}}%
\pgfpathlineto{\pgfqpoint{2.880665in}{2.196403in}}%
\pgfpathlineto{\pgfqpoint{2.866421in}{2.212237in}}%
\pgfpathlineto{\pgfqpoint{2.875627in}{2.198208in}}%
\pgfpathlineto{\pgfqpoint{2.884807in}{2.184671in}}%
\pgfpathlineto{\pgfqpoint{2.893962in}{2.171618in}}%
\pgfpathlineto{\pgfqpoint{2.903093in}{2.159042in}}%
\pgfpathclose%
\pgfusepath{fill}%
\end{pgfscope}%
\begin{pgfscope}%
\pgfpathrectangle{\pgfqpoint{1.150000in}{0.150000in}}{\pgfqpoint{5.700000in}{5.700000in}}%
\pgfusepath{clip}%
\pgfsetbuttcap%
\pgfsetroundjoin%
\definecolor{currentfill}{rgb}{0.277018,0.050344,0.375715}%
\pgfsetfillcolor{currentfill}%
\pgfsetfillopacity{0.700000}%
\pgfsetlinewidth{0.000000pt}%
\definecolor{currentstroke}{rgb}{0.000000,0.000000,0.000000}%
\pgfsetstrokecolor{currentstroke}%
\pgfsetdash{}{0pt}%
\pgfpathmoveto{\pgfqpoint{4.114589in}{1.380077in}}%
\pgfpathlineto{\pgfqpoint{4.128818in}{1.375665in}}%
\pgfpathlineto{\pgfqpoint{4.143054in}{1.371350in}}%
\pgfpathlineto{\pgfqpoint{4.157296in}{1.367130in}}%
\pgfpathlineto{\pgfqpoint{4.171545in}{1.363007in}}%
\pgfpathlineto{\pgfqpoint{4.163368in}{1.357365in}}%
\pgfpathlineto{\pgfqpoint{4.155183in}{1.351980in}}%
\pgfpathlineto{\pgfqpoint{4.146992in}{1.346857in}}%
\pgfpathlineto{\pgfqpoint{4.138793in}{1.342003in}}%
\pgfpathlineto{\pgfqpoint{4.124527in}{1.346709in}}%
\pgfpathlineto{\pgfqpoint{4.110267in}{1.351510in}}%
\pgfpathlineto{\pgfqpoint{4.096013in}{1.356408in}}%
\pgfpathlineto{\pgfqpoint{4.081766in}{1.361402in}}%
\pgfpathlineto{\pgfqpoint{4.089983in}{1.365666in}}%
\pgfpathlineto{\pgfqpoint{4.098192in}{1.370204in}}%
\pgfpathlineto{\pgfqpoint{4.106394in}{1.375010in}}%
\pgfpathlineto{\pgfqpoint{4.114589in}{1.380077in}}%
\pgfpathclose%
\pgfusepath{fill}%
\end{pgfscope}%
\begin{pgfscope}%
\pgfpathrectangle{\pgfqpoint{1.150000in}{0.150000in}}{\pgfqpoint{5.700000in}{5.700000in}}%
\pgfusepath{clip}%
\pgfsetbuttcap%
\pgfsetroundjoin%
\definecolor{currentfill}{rgb}{0.144759,0.519093,0.556572}%
\pgfsetfillcolor{currentfill}%
\pgfsetfillopacity{0.700000}%
\pgfsetlinewidth{0.000000pt}%
\definecolor{currentstroke}{rgb}{0.000000,0.000000,0.000000}%
\pgfsetstrokecolor{currentstroke}%
\pgfsetdash{}{0pt}%
\pgfpathmoveto{\pgfqpoint{2.618765in}{2.489691in}}%
\pgfpathlineto{\pgfqpoint{2.633019in}{2.471837in}}%
\pgfpathlineto{\pgfqpoint{2.647267in}{2.454128in}}%
\pgfpathlineto{\pgfqpoint{2.661511in}{2.436563in}}%
\pgfpathlineto{\pgfqpoint{2.675750in}{2.419140in}}%
\pgfpathlineto{\pgfqpoint{2.666373in}{2.434513in}}%
\pgfpathlineto{\pgfqpoint{2.656967in}{2.450389in}}%
\pgfpathlineto{\pgfqpoint{2.647532in}{2.466776in}}%
\pgfpathlineto{\pgfqpoint{2.638068in}{2.483683in}}%
\pgfpathlineto{\pgfqpoint{2.623762in}{2.501824in}}%
\pgfpathlineto{\pgfqpoint{2.609450in}{2.520109in}}%
\pgfpathlineto{\pgfqpoint{2.595134in}{2.538538in}}%
\pgfpathlineto{\pgfqpoint{2.580812in}{2.557114in}}%
\pgfpathlineto{\pgfqpoint{2.590345in}{2.539476in}}%
\pgfpathlineto{\pgfqpoint{2.599849in}{2.522366in}}%
\pgfpathlineto{\pgfqpoint{2.609322in}{2.505773in}}%
\pgfpathlineto{\pgfqpoint{2.618765in}{2.489691in}}%
\pgfpathclose%
\pgfusepath{fill}%
\end{pgfscope}%
\begin{pgfscope}%
\pgfpathrectangle{\pgfqpoint{1.150000in}{0.150000in}}{\pgfqpoint{5.700000in}{5.700000in}}%
\pgfusepath{clip}%
\pgfsetbuttcap%
\pgfsetroundjoin%
\definecolor{currentfill}{rgb}{0.282910,0.105393,0.426902}%
\pgfsetfillcolor{currentfill}%
\pgfsetfillopacity{0.700000}%
\pgfsetlinewidth{0.000000pt}%
\definecolor{currentstroke}{rgb}{0.000000,0.000000,0.000000}%
\pgfsetstrokecolor{currentstroke}%
\pgfsetdash{}{0pt}%
\pgfpathmoveto{\pgfqpoint{3.764597in}{1.479657in}}%
\pgfpathlineto{\pgfqpoint{3.778763in}{1.472092in}}%
\pgfpathlineto{\pgfqpoint{3.792932in}{1.464627in}}%
\pgfpathlineto{\pgfqpoint{3.807106in}{1.457263in}}%
\pgfpathlineto{\pgfqpoint{3.821285in}{1.449998in}}%
\pgfpathlineto{\pgfqpoint{3.812928in}{1.449654in}}%
\pgfpathlineto{\pgfqpoint{3.804560in}{1.449640in}}%
\pgfpathlineto{\pgfqpoint{3.796181in}{1.449965in}}%
\pgfpathlineto{\pgfqpoint{3.787791in}{1.450634in}}%
\pgfpathlineto{\pgfqpoint{3.773583in}{1.458523in}}%
\pgfpathlineto{\pgfqpoint{3.759380in}{1.466511in}}%
\pgfpathlineto{\pgfqpoint{3.745180in}{1.474600in}}%
\pgfpathlineto{\pgfqpoint{3.730985in}{1.482790in}}%
\pgfpathlineto{\pgfqpoint{3.739405in}{1.481489in}}%
\pgfpathlineto{\pgfqpoint{3.747814in}{1.480537in}}%
\pgfpathlineto{\pgfqpoint{3.756211in}{1.479929in}}%
\pgfpathlineto{\pgfqpoint{3.764597in}{1.479657in}}%
\pgfpathclose%
\pgfusepath{fill}%
\end{pgfscope}%
\begin{pgfscope}%
\pgfpathrectangle{\pgfqpoint{1.150000in}{0.150000in}}{\pgfqpoint{5.700000in}{5.700000in}}%
\pgfusepath{clip}%
\pgfsetbuttcap%
\pgfsetroundjoin%
\definecolor{currentfill}{rgb}{0.203063,0.379716,0.553925}%
\pgfsetfillcolor{currentfill}%
\pgfsetfillopacity{0.700000}%
\pgfsetlinewidth{0.000000pt}%
\definecolor{currentstroke}{rgb}{0.000000,0.000000,0.000000}%
\pgfsetstrokecolor{currentstroke}%
\pgfsetdash{}{0pt}%
\pgfpathmoveto{\pgfqpoint{2.959820in}{2.099273in}}%
\pgfpathlineto{\pgfqpoint{2.973997in}{2.084646in}}%
\pgfpathlineto{\pgfqpoint{2.988172in}{2.070143in}}%
\pgfpathlineto{\pgfqpoint{3.002346in}{2.055765in}}%
\pgfpathlineto{\pgfqpoint{3.016519in}{2.041510in}}%
\pgfpathlineto{\pgfqpoint{3.007501in}{2.052710in}}%
\pgfpathlineto{\pgfqpoint{2.998460in}{2.064374in}}%
\pgfpathlineto{\pgfqpoint{2.989396in}{2.076508in}}%
\pgfpathlineto{\pgfqpoint{2.980308in}{2.089121in}}%
\pgfpathlineto{\pgfqpoint{2.966079in}{2.104071in}}%
\pgfpathlineto{\pgfqpoint{2.951849in}{2.119146in}}%
\pgfpathlineto{\pgfqpoint{2.937617in}{2.134345in}}%
\pgfpathlineto{\pgfqpoint{2.923382in}{2.149669in}}%
\pgfpathlineto{\pgfqpoint{2.932528in}{2.136350in}}%
\pgfpathlineto{\pgfqpoint{2.941649in}{2.123516in}}%
\pgfpathlineto{\pgfqpoint{2.950747in}{2.111160in}}%
\pgfpathlineto{\pgfqpoint{2.959820in}{2.099273in}}%
\pgfpathclose%
\pgfusepath{fill}%
\end{pgfscope}%
\begin{pgfscope}%
\pgfpathrectangle{\pgfqpoint{1.150000in}{0.150000in}}{\pgfqpoint{5.700000in}{5.700000in}}%
\pgfusepath{clip}%
\pgfsetbuttcap%
\pgfsetroundjoin%
\definecolor{currentfill}{rgb}{0.278012,0.180367,0.486697}%
\pgfsetfillcolor{currentfill}%
\pgfsetfillopacity{0.700000}%
\pgfsetlinewidth{0.000000pt}%
\definecolor{currentstroke}{rgb}{0.000000,0.000000,0.000000}%
\pgfsetstrokecolor{currentstroke}%
\pgfsetdash{}{0pt}%
\pgfpathmoveto{\pgfqpoint{3.504312in}{1.627781in}}%
\pgfpathlineto{\pgfqpoint{3.518457in}{1.617937in}}%
\pgfpathlineto{\pgfqpoint{3.532605in}{1.608199in}}%
\pgfpathlineto{\pgfqpoint{3.546756in}{1.598566in}}%
\pgfpathlineto{\pgfqpoint{3.560909in}{1.589039in}}%
\pgfpathlineto{\pgfqpoint{3.552373in}{1.592616in}}%
\pgfpathlineto{\pgfqpoint{3.543823in}{1.596573in}}%
\pgfpathlineto{\pgfqpoint{3.535258in}{1.600917in}}%
\pgfpathlineto{\pgfqpoint{3.526678in}{1.605655in}}%
\pgfpathlineto{\pgfqpoint{3.512487in}{1.615833in}}%
\pgfpathlineto{\pgfqpoint{3.498298in}{1.626116in}}%
\pgfpathlineto{\pgfqpoint{3.484111in}{1.636505in}}%
\pgfpathlineto{\pgfqpoint{3.469927in}{1.647000in}}%
\pgfpathlineto{\pgfqpoint{3.478546in}{1.641603in}}%
\pgfpathlineto{\pgfqpoint{3.487150in}{1.636605in}}%
\pgfpathlineto{\pgfqpoint{3.495738in}{1.632000in}}%
\pgfpathlineto{\pgfqpoint{3.504312in}{1.627781in}}%
\pgfpathclose%
\pgfusepath{fill}%
\end{pgfscope}%
\begin{pgfscope}%
\pgfpathrectangle{\pgfqpoint{1.150000in}{0.150000in}}{\pgfqpoint{5.700000in}{5.700000in}}%
\pgfusepath{clip}%
\pgfsetbuttcap%
\pgfsetroundjoin%
\definecolor{currentfill}{rgb}{0.279566,0.067836,0.391917}%
\pgfsetfillcolor{currentfill}%
\pgfsetfillopacity{0.700000}%
\pgfsetlinewidth{0.000000pt}%
\definecolor{currentstroke}{rgb}{0.000000,0.000000,0.000000}%
\pgfsetstrokecolor{currentstroke}%
\pgfsetdash{}{0pt}%
\pgfpathmoveto{\pgfqpoint{3.968008in}{1.404845in}}%
\pgfpathlineto{\pgfqpoint{3.982207in}{1.399074in}}%
\pgfpathlineto{\pgfqpoint{3.996412in}{1.393400in}}%
\pgfpathlineto{\pgfqpoint{4.010623in}{1.387824in}}%
\pgfpathlineto{\pgfqpoint{4.024839in}{1.382346in}}%
\pgfpathlineto{\pgfqpoint{4.016594in}{1.378955in}}%
\pgfpathlineto{\pgfqpoint{4.008340in}{1.375855in}}%
\pgfpathlineto{\pgfqpoint{4.000078in}{1.373053in}}%
\pgfpathlineto{\pgfqpoint{3.991807in}{1.370553in}}%
\pgfpathlineto{\pgfqpoint{3.977568in}{1.376633in}}%
\pgfpathlineto{\pgfqpoint{3.963334in}{1.382811in}}%
\pgfpathlineto{\pgfqpoint{3.949106in}{1.389087in}}%
\pgfpathlineto{\pgfqpoint{3.934884in}{1.395461in}}%
\pgfpathlineto{\pgfqpoint{3.943179in}{1.397351in}}%
\pgfpathlineto{\pgfqpoint{3.951464in}{1.399549in}}%
\pgfpathlineto{\pgfqpoint{3.959740in}{1.402049in}}%
\pgfpathlineto{\pgfqpoint{3.968008in}{1.404845in}}%
\pgfpathclose%
\pgfusepath{fill}%
\end{pgfscope}%
\begin{pgfscope}%
\pgfpathrectangle{\pgfqpoint{1.150000in}{0.150000in}}{\pgfqpoint{5.700000in}{5.700000in}}%
\pgfusepath{clip}%
\pgfsetbuttcap%
\pgfsetroundjoin%
\definecolor{currentfill}{rgb}{0.135066,0.544853,0.554029}%
\pgfsetfillcolor{currentfill}%
\pgfsetfillopacity{0.700000}%
\pgfsetlinewidth{0.000000pt}%
\definecolor{currentstroke}{rgb}{0.000000,0.000000,0.000000}%
\pgfsetstrokecolor{currentstroke}%
\pgfsetdash{}{0pt}%
\pgfpathmoveto{\pgfqpoint{2.561703in}{2.562574in}}%
\pgfpathlineto{\pgfqpoint{2.575976in}{2.544131in}}%
\pgfpathlineto{\pgfqpoint{2.590244in}{2.525837in}}%
\pgfpathlineto{\pgfqpoint{2.604507in}{2.507691in}}%
\pgfpathlineto{\pgfqpoint{2.618765in}{2.489691in}}%
\pgfpathlineto{\pgfqpoint{2.609322in}{2.505773in}}%
\pgfpathlineto{\pgfqpoint{2.599849in}{2.522366in}}%
\pgfpathlineto{\pgfqpoint{2.590345in}{2.539476in}}%
\pgfpathlineto{\pgfqpoint{2.580812in}{2.557114in}}%
\pgfpathlineto{\pgfqpoint{2.566485in}{2.575836in}}%
\pgfpathlineto{\pgfqpoint{2.552152in}{2.594707in}}%
\pgfpathlineto{\pgfqpoint{2.537814in}{2.613726in}}%
\pgfpathlineto{\pgfqpoint{2.523471in}{2.632896in}}%
\pgfpathlineto{\pgfqpoint{2.533076in}{2.614523in}}%
\pgfpathlineto{\pgfqpoint{2.542649in}{2.596684in}}%
\pgfpathlineto{\pgfqpoint{2.552191in}{2.579370in}}%
\pgfpathlineto{\pgfqpoint{2.561703in}{2.562574in}}%
\pgfpathclose%
\pgfusepath{fill}%
\end{pgfscope}%
\begin{pgfscope}%
\pgfpathrectangle{\pgfqpoint{1.150000in}{0.150000in}}{\pgfqpoint{5.700000in}{5.700000in}}%
\pgfusepath{clip}%
\pgfsetbuttcap%
\pgfsetroundjoin%
\definecolor{currentfill}{rgb}{0.277018,0.050344,0.375715}%
\pgfsetfillcolor{currentfill}%
\pgfsetfillopacity{0.700000}%
\pgfsetlinewidth{0.000000pt}%
\definecolor{currentstroke}{rgb}{0.000000,0.000000,0.000000}%
\pgfsetstrokecolor{currentstroke}%
\pgfsetdash{}{0pt}%
\pgfpathmoveto{\pgfqpoint{4.261199in}{1.374746in}}%
\pgfpathlineto{\pgfqpoint{4.275471in}{1.371665in}}%
\pgfpathlineto{\pgfqpoint{4.289750in}{1.368679in}}%
\pgfpathlineto{\pgfqpoint{4.304037in}{1.365789in}}%
\pgfpathlineto{\pgfqpoint{4.318332in}{1.362994in}}%
\pgfpathlineto{\pgfqpoint{4.310207in}{1.355270in}}%
\pgfpathlineto{\pgfqpoint{4.302076in}{1.347770in}}%
\pgfpathlineto{\pgfqpoint{4.293940in}{1.340500in}}%
\pgfpathlineto{\pgfqpoint{4.285799in}{1.333466in}}%
\pgfpathlineto{\pgfqpoint{4.271491in}{1.336825in}}%
\pgfpathlineto{\pgfqpoint{4.257191in}{1.340279in}}%
\pgfpathlineto{\pgfqpoint{4.242898in}{1.343828in}}%
\pgfpathlineto{\pgfqpoint{4.228613in}{1.347473in}}%
\pgfpathlineto{\pgfqpoint{4.236769in}{1.353937in}}%
\pgfpathlineto{\pgfqpoint{4.244918in}{1.360640in}}%
\pgfpathlineto{\pgfqpoint{4.253062in}{1.367579in}}%
\pgfpathlineto{\pgfqpoint{4.261199in}{1.374746in}}%
\pgfpathclose%
\pgfusepath{fill}%
\end{pgfscope}%
\begin{pgfscope}%
\pgfpathrectangle{\pgfqpoint{1.150000in}{0.150000in}}{\pgfqpoint{5.700000in}{5.700000in}}%
\pgfusepath{clip}%
\pgfsetbuttcap%
\pgfsetroundjoin%
\definecolor{currentfill}{rgb}{0.212395,0.359683,0.551710}%
\pgfsetfillcolor{currentfill}%
\pgfsetfillopacity{0.700000}%
\pgfsetlinewidth{0.000000pt}%
\definecolor{currentstroke}{rgb}{0.000000,0.000000,0.000000}%
\pgfsetstrokecolor{currentstroke}%
\pgfsetdash{}{0pt}%
\pgfpathmoveto{\pgfqpoint{3.016519in}{2.041510in}}%
\pgfpathlineto{\pgfqpoint{3.030690in}{2.027377in}}%
\pgfpathlineto{\pgfqpoint{3.044860in}{2.013367in}}%
\pgfpathlineto{\pgfqpoint{3.059028in}{1.999478in}}%
\pgfpathlineto{\pgfqpoint{3.073196in}{1.985710in}}%
\pgfpathlineto{\pgfqpoint{3.064232in}{1.996227in}}%
\pgfpathlineto{\pgfqpoint{3.055246in}{2.007200in}}%
\pgfpathlineto{\pgfqpoint{3.046237in}{2.018638in}}%
\pgfpathlineto{\pgfqpoint{3.037206in}{2.030549in}}%
\pgfpathlineto{\pgfqpoint{3.022984in}{2.045009in}}%
\pgfpathlineto{\pgfqpoint{3.008760in}{2.059591in}}%
\pgfpathlineto{\pgfqpoint{2.994535in}{2.074295in}}%
\pgfpathlineto{\pgfqpoint{2.980308in}{2.089121in}}%
\pgfpathlineto{\pgfqpoint{2.989396in}{2.076508in}}%
\pgfpathlineto{\pgfqpoint{2.998460in}{2.064374in}}%
\pgfpathlineto{\pgfqpoint{3.007501in}{2.052710in}}%
\pgfpathlineto{\pgfqpoint{3.016519in}{2.041510in}}%
\pgfpathclose%
\pgfusepath{fill}%
\end{pgfscope}%
\begin{pgfscope}%
\pgfpathrectangle{\pgfqpoint{1.150000in}{0.150000in}}{\pgfqpoint{5.700000in}{5.700000in}}%
\pgfusepath{clip}%
\pgfsetbuttcap%
\pgfsetroundjoin%
\definecolor{currentfill}{rgb}{0.263663,0.237631,0.518762}%
\pgfsetfillcolor{currentfill}%
\pgfsetfillopacity{0.700000}%
\pgfsetlinewidth{0.000000pt}%
\definecolor{currentstroke}{rgb}{0.000000,0.000000,0.000000}%
\pgfsetstrokecolor{currentstroke}%
\pgfsetdash{}{0pt}%
\pgfpathmoveto{\pgfqpoint{4.979086in}{1.766451in}}%
\pgfpathlineto{\pgfqpoint{4.993665in}{1.770298in}}%
\pgfpathlineto{\pgfqpoint{5.008257in}{1.774242in}}%
\pgfpathlineto{\pgfqpoint{5.022862in}{1.778283in}}%
\pgfpathlineto{\pgfqpoint{5.037481in}{1.782419in}}%
\pgfpathlineto{\pgfqpoint{5.029493in}{1.766837in}}%
\pgfpathlineto{\pgfqpoint{5.021502in}{1.751290in}}%
\pgfpathlineto{\pgfqpoint{5.013508in}{1.735784in}}%
\pgfpathlineto{\pgfqpoint{5.005511in}{1.720323in}}%
\pgfpathlineto{\pgfqpoint{4.990897in}{1.716614in}}%
\pgfpathlineto{\pgfqpoint{4.976297in}{1.713001in}}%
\pgfpathlineto{\pgfqpoint{4.961710in}{1.709485in}}%
\pgfpathlineto{\pgfqpoint{4.947135in}{1.706064in}}%
\pgfpathlineto{\pgfqpoint{4.955128in}{1.721091in}}%
\pgfpathlineto{\pgfqpoint{4.963117in}{1.736167in}}%
\pgfpathlineto{\pgfqpoint{4.971103in}{1.751289in}}%
\pgfpathlineto{\pgfqpoint{4.979086in}{1.766451in}}%
\pgfpathclose%
\pgfusepath{fill}%
\end{pgfscope}%
\begin{pgfscope}%
\pgfpathrectangle{\pgfqpoint{1.150000in}{0.150000in}}{\pgfqpoint{5.700000in}{5.700000in}}%
\pgfusepath{clip}%
\pgfsetbuttcap%
\pgfsetroundjoin%
\definecolor{currentfill}{rgb}{0.197636,0.391528,0.554969}%
\pgfsetfillcolor{currentfill}%
\pgfsetfillopacity{0.700000}%
\pgfsetlinewidth{0.000000pt}%
\definecolor{currentstroke}{rgb}{0.000000,0.000000,0.000000}%
\pgfsetstrokecolor{currentstroke}%
\pgfsetdash{}{0pt}%
\pgfpathmoveto{\pgfqpoint{5.314193in}{2.147585in}}%
\pgfpathlineto{\pgfqpoint{5.328975in}{2.154476in}}%
\pgfpathlineto{\pgfqpoint{5.343771in}{2.161466in}}%
\pgfpathlineto{\pgfqpoint{5.358583in}{2.168555in}}%
\pgfpathlineto{\pgfqpoint{5.350652in}{2.152096in}}%
\pgfpathlineto{\pgfqpoint{5.342716in}{2.135590in}}%
\pgfpathlineto{\pgfqpoint{5.334775in}{2.119041in}}%
\pgfpathlineto{\pgfqpoint{5.326830in}{2.102452in}}%
\pgfpathlineto{\pgfqpoint{5.312026in}{2.095703in}}%
\pgfpathlineto{\pgfqpoint{5.297238in}{2.089054in}}%
\pgfpathlineto{\pgfqpoint{5.282464in}{2.082503in}}%
\pgfpathlineto{\pgfqpoint{5.290403in}{2.098831in}}%
\pgfpathlineto{\pgfqpoint{5.298338in}{2.115123in}}%
\pgfpathlineto{\pgfqpoint{5.306268in}{2.131375in}}%
\pgfpathlineto{\pgfqpoint{5.314193in}{2.147585in}}%
\pgfpathclose%
\pgfusepath{fill}%
\end{pgfscope}%
\begin{pgfscope}%
\pgfpathrectangle{\pgfqpoint{1.150000in}{0.150000in}}{\pgfqpoint{5.700000in}{5.700000in}}%
\pgfusepath{clip}%
\pgfsetbuttcap%
\pgfsetroundjoin%
\definecolor{currentfill}{rgb}{0.282327,0.094955,0.417331}%
\pgfsetfillcolor{currentfill}%
\pgfsetfillopacity{0.700000}%
\pgfsetlinewidth{0.000000pt}%
\definecolor{currentstroke}{rgb}{0.000000,0.000000,0.000000}%
\pgfsetstrokecolor{currentstroke}%
\pgfsetdash{}{0pt}%
\pgfpathmoveto{\pgfqpoint{4.587348in}{1.464059in}}%
\pgfpathlineto{\pgfqpoint{4.601738in}{1.464080in}}%
\pgfpathlineto{\pgfqpoint{4.616137in}{1.464197in}}%
\pgfpathlineto{\pgfqpoint{4.630548in}{1.464408in}}%
\pgfpathlineto{\pgfqpoint{4.644968in}{1.464714in}}%
\pgfpathlineto{\pgfqpoint{4.636924in}{1.452703in}}%
\pgfpathlineto{\pgfqpoint{4.628877in}{1.440835in}}%
\pgfpathlineto{\pgfqpoint{4.620826in}{1.429116in}}%
\pgfpathlineto{\pgfqpoint{4.612771in}{1.417549in}}%
\pgfpathlineto{\pgfqpoint{4.598347in}{1.417755in}}%
\pgfpathlineto{\pgfqpoint{4.583934in}{1.418056in}}%
\pgfpathlineto{\pgfqpoint{4.569530in}{1.418451in}}%
\pgfpathlineto{\pgfqpoint{4.555136in}{1.418940in}}%
\pgfpathlineto{\pgfqpoint{4.563194in}{1.429989in}}%
\pgfpathlineto{\pgfqpoint{4.571249in}{1.441194in}}%
\pgfpathlineto{\pgfqpoint{4.579300in}{1.452553in}}%
\pgfpathlineto{\pgfqpoint{4.587348in}{1.464059in}}%
\pgfpathclose%
\pgfusepath{fill}%
\end{pgfscope}%
\begin{pgfscope}%
\pgfpathrectangle{\pgfqpoint{1.150000in}{0.150000in}}{\pgfqpoint{5.700000in}{5.700000in}}%
\pgfusepath{clip}%
\pgfsetbuttcap%
\pgfsetroundjoin%
\definecolor{currentfill}{rgb}{0.412913,0.803041,0.357269}%
\pgfsetfillcolor{currentfill}%
\pgfsetfillopacity{0.700000}%
\pgfsetlinewidth{0.000000pt}%
\definecolor{currentstroke}{rgb}{0.000000,0.000000,0.000000}%
\pgfsetstrokecolor{currentstroke}%
\pgfsetdash{}{0pt}%
\pgfpathmoveto{\pgfqpoint{2.025410in}{3.345886in}}%
\pgfpathlineto{\pgfqpoint{2.039946in}{3.321130in}}%
\pgfpathlineto{\pgfqpoint{2.054470in}{3.296583in}}%
\pgfpathlineto{\pgfqpoint{2.068982in}{3.272241in}}%
\pgfpathlineto{\pgfqpoint{2.083482in}{3.248103in}}%
\pgfpathlineto{\pgfqpoint{2.073420in}{3.269326in}}%
\pgfpathlineto{\pgfqpoint{2.063321in}{3.291101in}}%
\pgfpathlineto{\pgfqpoint{2.053183in}{3.313434in}}%
\pgfpathlineto{\pgfqpoint{2.043005in}{3.336336in}}%
\pgfpathlineto{\pgfqpoint{2.028419in}{3.361228in}}%
\pgfpathlineto{\pgfqpoint{2.013821in}{3.386325in}}%
\pgfpathlineto{\pgfqpoint{1.999211in}{3.411631in}}%
\pgfpathlineto{\pgfqpoint{1.984588in}{3.437146in}}%
\pgfpathlineto{\pgfqpoint{1.994853in}{3.413475in}}%
\pgfpathlineto{\pgfqpoint{2.005078in}{3.390381in}}%
\pgfpathlineto{\pgfqpoint{2.015264in}{3.367854in}}%
\pgfpathlineto{\pgfqpoint{2.025410in}{3.345886in}}%
\pgfpathclose%
\pgfusepath{fill}%
\end{pgfscope}%
\begin{pgfscope}%
\pgfpathrectangle{\pgfqpoint{1.150000in}{0.150000in}}{\pgfqpoint{5.700000in}{5.700000in}}%
\pgfusepath{clip}%
\pgfsetbuttcap%
\pgfsetroundjoin%
\definecolor{currentfill}{rgb}{0.125394,0.574318,0.549086}%
\pgfsetfillcolor{currentfill}%
\pgfsetfillopacity{0.700000}%
\pgfsetlinewidth{0.000000pt}%
\definecolor{currentstroke}{rgb}{0.000000,0.000000,0.000000}%
\pgfsetstrokecolor{currentstroke}%
\pgfsetdash{}{0pt}%
\pgfpathmoveto{\pgfqpoint{2.504555in}{2.637857in}}%
\pgfpathlineto{\pgfqpoint{2.518850in}{2.618807in}}%
\pgfpathlineto{\pgfqpoint{2.533140in}{2.599911in}}%
\pgfpathlineto{\pgfqpoint{2.547425in}{2.581167in}}%
\pgfpathlineto{\pgfqpoint{2.561703in}{2.562574in}}%
\pgfpathlineto{\pgfqpoint{2.552191in}{2.579370in}}%
\pgfpathlineto{\pgfqpoint{2.542649in}{2.596684in}}%
\pgfpathlineto{\pgfqpoint{2.533076in}{2.614523in}}%
\pgfpathlineto{\pgfqpoint{2.523471in}{2.632896in}}%
\pgfpathlineto{\pgfqpoint{2.509121in}{2.652217in}}%
\pgfpathlineto{\pgfqpoint{2.494765in}{2.671690in}}%
\pgfpathlineto{\pgfqpoint{2.480404in}{2.691316in}}%
\pgfpathlineto{\pgfqpoint{2.466035in}{2.711098in}}%
\pgfpathlineto{\pgfqpoint{2.475714in}{2.691984in}}%
\pgfpathlineto{\pgfqpoint{2.485359in}{2.673412in}}%
\pgfpathlineto{\pgfqpoint{2.494973in}{2.655372in}}%
\pgfpathlineto{\pgfqpoint{2.504555in}{2.637857in}}%
\pgfpathclose%
\pgfusepath{fill}%
\end{pgfscope}%
\begin{pgfscope}%
\pgfpathrectangle{\pgfqpoint{1.150000in}{0.150000in}}{\pgfqpoint{5.700000in}{5.700000in}}%
\pgfusepath{clip}%
\pgfsetbuttcap%
\pgfsetroundjoin%
\definecolor{currentfill}{rgb}{0.283229,0.120777,0.440584}%
\pgfsetfillcolor{currentfill}%
\pgfsetfillopacity{0.700000}%
\pgfsetlinewidth{0.000000pt}%
\definecolor{currentstroke}{rgb}{0.000000,0.000000,0.000000}%
\pgfsetstrokecolor{currentstroke}%
\pgfsetdash{}{0pt}%
\pgfpathmoveto{\pgfqpoint{4.677109in}{1.514090in}}%
\pgfpathlineto{\pgfqpoint{4.691538in}{1.514984in}}%
\pgfpathlineto{\pgfqpoint{4.705978in}{1.515974in}}%
\pgfpathlineto{\pgfqpoint{4.720429in}{1.517059in}}%
\pgfpathlineto{\pgfqpoint{4.734891in}{1.518238in}}%
\pgfpathlineto{\pgfqpoint{4.726862in}{1.505216in}}%
\pgfpathlineto{\pgfqpoint{4.718829in}{1.492313in}}%
\pgfpathlineto{\pgfqpoint{4.710794in}{1.479534in}}%
\pgfpathlineto{\pgfqpoint{4.702755in}{1.466883in}}%
\pgfpathlineto{\pgfqpoint{4.688292in}{1.466199in}}%
\pgfpathlineto{\pgfqpoint{4.673840in}{1.465609in}}%
\pgfpathlineto{\pgfqpoint{4.659399in}{1.465114in}}%
\pgfpathlineto{\pgfqpoint{4.644968in}{1.464714in}}%
\pgfpathlineto{\pgfqpoint{4.653009in}{1.476863in}}%
\pgfpathlineto{\pgfqpoint{4.661046in}{1.489145in}}%
\pgfpathlineto{\pgfqpoint{4.669079in}{1.501555in}}%
\pgfpathlineto{\pgfqpoint{4.677109in}{1.514090in}}%
\pgfpathclose%
\pgfusepath{fill}%
\end{pgfscope}%
\begin{pgfscope}%
\pgfpathrectangle{\pgfqpoint{1.150000in}{0.150000in}}{\pgfqpoint{5.700000in}{5.700000in}}%
\pgfusepath{clip}%
\pgfsetbuttcap%
\pgfsetroundjoin%
\definecolor{currentfill}{rgb}{0.221989,0.339161,0.548752}%
\pgfsetfillcolor{currentfill}%
\pgfsetfillopacity{0.700000}%
\pgfsetlinewidth{0.000000pt}%
\definecolor{currentstroke}{rgb}{0.000000,0.000000,0.000000}%
\pgfsetstrokecolor{currentstroke}%
\pgfsetdash{}{0pt}%
\pgfpathmoveto{\pgfqpoint{3.073196in}{1.985710in}}%
\pgfpathlineto{\pgfqpoint{3.087363in}{1.972062in}}%
\pgfpathlineto{\pgfqpoint{3.101530in}{1.958533in}}%
\pgfpathlineto{\pgfqpoint{3.115695in}{1.945123in}}%
\pgfpathlineto{\pgfqpoint{3.129860in}{1.931832in}}%
\pgfpathlineto{\pgfqpoint{3.120948in}{1.941668in}}%
\pgfpathlineto{\pgfqpoint{3.112015in}{1.951955in}}%
\pgfpathlineto{\pgfqpoint{3.103060in}{1.962700in}}%
\pgfpathlineto{\pgfqpoint{3.094083in}{1.973911in}}%
\pgfpathlineto{\pgfqpoint{3.079865in}{1.987892in}}%
\pgfpathlineto{\pgfqpoint{3.065647in}{2.001991in}}%
\pgfpathlineto{\pgfqpoint{3.051427in}{2.016210in}}%
\pgfpathlineto{\pgfqpoint{3.037206in}{2.030549in}}%
\pgfpathlineto{\pgfqpoint{3.046237in}{2.018638in}}%
\pgfpathlineto{\pgfqpoint{3.055246in}{2.007200in}}%
\pgfpathlineto{\pgfqpoint{3.064232in}{1.996227in}}%
\pgfpathlineto{\pgfqpoint{3.073196in}{1.985710in}}%
\pgfpathclose%
\pgfusepath{fill}%
\end{pgfscope}%
\begin{pgfscope}%
\pgfpathrectangle{\pgfqpoint{1.150000in}{0.150000in}}{\pgfqpoint{5.700000in}{5.700000in}}%
\pgfusepath{clip}%
\pgfsetbuttcap%
\pgfsetroundjoin%
\definecolor{currentfill}{rgb}{0.280894,0.078907,0.402329}%
\pgfsetfillcolor{currentfill}%
\pgfsetfillopacity{0.700000}%
\pgfsetlinewidth{0.000000pt}%
\definecolor{currentstroke}{rgb}{0.000000,0.000000,0.000000}%
\pgfsetstrokecolor{currentstroke}%
\pgfsetdash{}{0pt}%
\pgfpathmoveto{\pgfqpoint{4.497656in}{1.421844in}}%
\pgfpathlineto{\pgfqpoint{4.512012in}{1.420976in}}%
\pgfpathlineto{\pgfqpoint{4.526377in}{1.420203in}}%
\pgfpathlineto{\pgfqpoint{4.540752in}{1.419524in}}%
\pgfpathlineto{\pgfqpoint{4.555136in}{1.418940in}}%
\pgfpathlineto{\pgfqpoint{4.547073in}{1.408054in}}%
\pgfpathlineto{\pgfqpoint{4.539007in}{1.397336in}}%
\pgfpathlineto{\pgfqpoint{4.530936in}{1.386790in}}%
\pgfpathlineto{\pgfqpoint{4.522862in}{1.376423in}}%
\pgfpathlineto{\pgfqpoint{4.508472in}{1.377536in}}%
\pgfpathlineto{\pgfqpoint{4.494091in}{1.378744in}}%
\pgfpathlineto{\pgfqpoint{4.479720in}{1.380045in}}%
\pgfpathlineto{\pgfqpoint{4.465357in}{1.381442in}}%
\pgfpathlineto{\pgfqpoint{4.473438in}{1.391274in}}%
\pgfpathlineto{\pgfqpoint{4.481515in}{1.401288in}}%
\pgfpathlineto{\pgfqpoint{4.489588in}{1.411480in}}%
\pgfpathlineto{\pgfqpoint{4.497656in}{1.421844in}}%
\pgfpathclose%
\pgfusepath{fill}%
\end{pgfscope}%
\begin{pgfscope}%
\pgfpathrectangle{\pgfqpoint{1.150000in}{0.150000in}}{\pgfqpoint{5.700000in}{5.700000in}}%
\pgfusepath{clip}%
\pgfsetbuttcap%
\pgfsetroundjoin%
\definecolor{currentfill}{rgb}{0.281887,0.150881,0.465405}%
\pgfsetfillcolor{currentfill}%
\pgfsetfillopacity{0.700000}%
\pgfsetlinewidth{0.000000pt}%
\definecolor{currentstroke}{rgb}{0.000000,0.000000,0.000000}%
\pgfsetstrokecolor{currentstroke}%
\pgfsetdash{}{0pt}%
\pgfpathmoveto{\pgfqpoint{4.766975in}{1.571427in}}%
\pgfpathlineto{\pgfqpoint{4.781448in}{1.573179in}}%
\pgfpathlineto{\pgfqpoint{4.795933in}{1.575027in}}%
\pgfpathlineto{\pgfqpoint{4.810429in}{1.576969in}}%
\pgfpathlineto{\pgfqpoint{4.824936in}{1.579006in}}%
\pgfpathlineto{\pgfqpoint{4.816919in}{1.565082in}}%
\pgfpathlineto{\pgfqpoint{4.808899in}{1.551254in}}%
\pgfpathlineto{\pgfqpoint{4.800876in}{1.537526in}}%
\pgfpathlineto{\pgfqpoint{4.792849in}{1.523904in}}%
\pgfpathlineto{\pgfqpoint{4.778343in}{1.522346in}}%
\pgfpathlineto{\pgfqpoint{4.763848in}{1.520882in}}%
\pgfpathlineto{\pgfqpoint{4.749364in}{1.519513in}}%
\pgfpathlineto{\pgfqpoint{4.734891in}{1.518238in}}%
\pgfpathlineto{\pgfqpoint{4.742917in}{1.531375in}}%
\pgfpathlineto{\pgfqpoint{4.750939in}{1.544622in}}%
\pgfpathlineto{\pgfqpoint{4.758959in}{1.557974in}}%
\pgfpathlineto{\pgfqpoint{4.766975in}{1.571427in}}%
\pgfpathclose%
\pgfusepath{fill}%
\end{pgfscope}%
\begin{pgfscope}%
\pgfpathrectangle{\pgfqpoint{1.150000in}{0.150000in}}{\pgfqpoint{5.700000in}{5.700000in}}%
\pgfusepath{clip}%
\pgfsetbuttcap%
\pgfsetroundjoin%
\definecolor{currentfill}{rgb}{0.119738,0.603785,0.541400}%
\pgfsetfillcolor{currentfill}%
\pgfsetfillopacity{0.700000}%
\pgfsetlinewidth{0.000000pt}%
\definecolor{currentstroke}{rgb}{0.000000,0.000000,0.000000}%
\pgfsetstrokecolor{currentstroke}%
\pgfsetdash{}{0pt}%
\pgfpathmoveto{\pgfqpoint{2.447310in}{2.715612in}}%
\pgfpathlineto{\pgfqpoint{2.461631in}{2.695937in}}%
\pgfpathlineto{\pgfqpoint{2.475945in}{2.676421in}}%
\pgfpathlineto{\pgfqpoint{2.490253in}{2.657061in}}%
\pgfpathlineto{\pgfqpoint{2.504555in}{2.637857in}}%
\pgfpathlineto{\pgfqpoint{2.494973in}{2.655372in}}%
\pgfpathlineto{\pgfqpoint{2.485359in}{2.673412in}}%
\pgfpathlineto{\pgfqpoint{2.475714in}{2.691984in}}%
\pgfpathlineto{\pgfqpoint{2.466035in}{2.711098in}}%
\pgfpathlineto{\pgfqpoint{2.451661in}{2.731035in}}%
\pgfpathlineto{\pgfqpoint{2.437279in}{2.751129in}}%
\pgfpathlineto{\pgfqpoint{2.422891in}{2.771382in}}%
\pgfpathlineto{\pgfqpoint{2.408496in}{2.791793in}}%
\pgfpathlineto{\pgfqpoint{2.418250in}{2.771934in}}%
\pgfpathlineto{\pgfqpoint{2.427969in}{2.752623in}}%
\pgfpathlineto{\pgfqpoint{2.437656in}{2.733852in}}%
\pgfpathlineto{\pgfqpoint{2.447310in}{2.715612in}}%
\pgfpathclose%
\pgfusepath{fill}%
\end{pgfscope}%
\begin{pgfscope}%
\pgfpathrectangle{\pgfqpoint{1.150000in}{0.150000in}}{\pgfqpoint{5.700000in}{5.700000in}}%
\pgfusepath{clip}%
\pgfsetbuttcap%
\pgfsetroundjoin%
\definecolor{currentfill}{rgb}{0.280255,0.165693,0.476498}%
\pgfsetfillcolor{currentfill}%
\pgfsetfillopacity{0.700000}%
\pgfsetlinewidth{0.000000pt}%
\definecolor{currentstroke}{rgb}{0.000000,0.000000,0.000000}%
\pgfsetstrokecolor{currentstroke}%
\pgfsetdash{}{0pt}%
\pgfpathmoveto{\pgfqpoint{3.560909in}{1.589039in}}%
\pgfpathlineto{\pgfqpoint{3.575065in}{1.579616in}}%
\pgfpathlineto{\pgfqpoint{3.589223in}{1.570298in}}%
\pgfpathlineto{\pgfqpoint{3.603385in}{1.561084in}}%
\pgfpathlineto{\pgfqpoint{3.617549in}{1.551973in}}%
\pgfpathlineto{\pgfqpoint{3.609049in}{1.554910in}}%
\pgfpathlineto{\pgfqpoint{3.600536in}{1.558220in}}%
\pgfpathlineto{\pgfqpoint{3.592008in}{1.561912in}}%
\pgfpathlineto{\pgfqpoint{3.583466in}{1.565993in}}%
\pgfpathlineto{\pgfqpoint{3.569265in}{1.575752in}}%
\pgfpathlineto{\pgfqpoint{3.555067in}{1.585615in}}%
\pgfpathlineto{\pgfqpoint{3.540871in}{1.595583in}}%
\pgfpathlineto{\pgfqpoint{3.526678in}{1.605655in}}%
\pgfpathlineto{\pgfqpoint{3.535258in}{1.600917in}}%
\pgfpathlineto{\pgfqpoint{3.543823in}{1.596573in}}%
\pgfpathlineto{\pgfqpoint{3.552373in}{1.592616in}}%
\pgfpathlineto{\pgfqpoint{3.560909in}{1.589039in}}%
\pgfpathclose%
\pgfusepath{fill}%
\end{pgfscope}%
\begin{pgfscope}%
\pgfpathrectangle{\pgfqpoint{1.150000in}{0.150000in}}{\pgfqpoint{5.700000in}{5.700000in}}%
\pgfusepath{clip}%
\pgfsetbuttcap%
\pgfsetroundjoin%
\definecolor{currentfill}{rgb}{0.223925,0.334994,0.548053}%
\pgfsetfillcolor{currentfill}%
\pgfsetfillopacity{0.700000}%
\pgfsetlinewidth{0.000000pt}%
\definecolor{currentstroke}{rgb}{0.000000,0.000000,0.000000}%
\pgfsetstrokecolor{currentstroke}%
\pgfsetdash{}{0pt}%
\pgfpathmoveto{\pgfqpoint{5.191752in}{1.993096in}}%
\pgfpathlineto{\pgfqpoint{5.206459in}{1.998897in}}%
\pgfpathlineto{\pgfqpoint{5.221180in}{2.004795in}}%
\pgfpathlineto{\pgfqpoint{5.235916in}{2.010791in}}%
\pgfpathlineto{\pgfqpoint{5.250667in}{2.016886in}}%
\pgfpathlineto{\pgfqpoint{5.242707in}{2.000422in}}%
\pgfpathlineto{\pgfqpoint{5.234744in}{1.983942in}}%
\pgfpathlineto{\pgfqpoint{5.226777in}{1.967448in}}%
\pgfpathlineto{\pgfqpoint{5.218806in}{1.950945in}}%
\pgfpathlineto{\pgfqpoint{5.204063in}{1.945227in}}%
\pgfpathlineto{\pgfqpoint{5.189335in}{1.939607in}}%
\pgfpathlineto{\pgfqpoint{5.174621in}{1.934084in}}%
\pgfpathlineto{\pgfqpoint{5.159921in}{1.928659in}}%
\pgfpathlineto{\pgfqpoint{5.167885in}{1.944780in}}%
\pgfpathlineto{\pgfqpoint{5.175844in}{1.960895in}}%
\pgfpathlineto{\pgfqpoint{5.183800in}{1.977002in}}%
\pgfpathlineto{\pgfqpoint{5.191752in}{1.993096in}}%
\pgfpathclose%
\pgfusepath{fill}%
\end{pgfscope}%
\begin{pgfscope}%
\pgfpathrectangle{\pgfqpoint{1.150000in}{0.150000in}}{\pgfqpoint{5.700000in}{5.700000in}}%
\pgfusepath{clip}%
\pgfsetbuttcap%
\pgfsetroundjoin%
\definecolor{currentfill}{rgb}{0.231674,0.318106,0.544834}%
\pgfsetfillcolor{currentfill}%
\pgfsetfillopacity{0.700000}%
\pgfsetlinewidth{0.000000pt}%
\definecolor{currentstroke}{rgb}{0.000000,0.000000,0.000000}%
\pgfsetstrokecolor{currentstroke}%
\pgfsetdash{}{0pt}%
\pgfpathmoveto{\pgfqpoint{3.129860in}{1.931832in}}%
\pgfpathlineto{\pgfqpoint{3.144025in}{1.918659in}}%
\pgfpathlineto{\pgfqpoint{3.158189in}{1.905603in}}%
\pgfpathlineto{\pgfqpoint{3.172353in}{1.892663in}}%
\pgfpathlineto{\pgfqpoint{3.186517in}{1.879840in}}%
\pgfpathlineto{\pgfqpoint{3.177655in}{1.888998in}}%
\pgfpathlineto{\pgfqpoint{3.168773in}{1.898601in}}%
\pgfpathlineto{\pgfqpoint{3.159870in}{1.908656in}}%
\pgfpathlineto{\pgfqpoint{3.150946in}{1.919170in}}%
\pgfpathlineto{\pgfqpoint{3.136731in}{1.932680in}}%
\pgfpathlineto{\pgfqpoint{3.122516in}{1.946306in}}%
\pgfpathlineto{\pgfqpoint{3.108300in}{1.960050in}}%
\pgfpathlineto{\pgfqpoint{3.094083in}{1.973911in}}%
\pgfpathlineto{\pgfqpoint{3.103060in}{1.962700in}}%
\pgfpathlineto{\pgfqpoint{3.112015in}{1.951955in}}%
\pgfpathlineto{\pgfqpoint{3.120948in}{1.941668in}}%
\pgfpathlineto{\pgfqpoint{3.129860in}{1.931832in}}%
\pgfpathclose%
\pgfusepath{fill}%
\end{pgfscope}%
\begin{pgfscope}%
\pgfpathrectangle{\pgfqpoint{1.150000in}{0.150000in}}{\pgfqpoint{5.700000in}{5.700000in}}%
\pgfusepath{clip}%
\pgfsetbuttcap%
\pgfsetroundjoin%
\definecolor{currentfill}{rgb}{0.282327,0.094955,0.417331}%
\pgfsetfillcolor{currentfill}%
\pgfsetfillopacity{0.700000}%
\pgfsetlinewidth{0.000000pt}%
\definecolor{currentstroke}{rgb}{0.000000,0.000000,0.000000}%
\pgfsetstrokecolor{currentstroke}%
\pgfsetdash{}{0pt}%
\pgfpathmoveto{\pgfqpoint{3.821285in}{1.449998in}}%
\pgfpathlineto{\pgfqpoint{3.835468in}{1.442834in}}%
\pgfpathlineto{\pgfqpoint{3.849656in}{1.435769in}}%
\pgfpathlineto{\pgfqpoint{3.863848in}{1.428804in}}%
\pgfpathlineto{\pgfqpoint{3.878045in}{1.421938in}}%
\pgfpathlineto{\pgfqpoint{3.869715in}{1.420978in}}%
\pgfpathlineto{\pgfqpoint{3.861375in}{1.420343in}}%
\pgfpathlineto{\pgfqpoint{3.853025in}{1.420042in}}%
\pgfpathlineto{\pgfqpoint{3.844664in}{1.420080in}}%
\pgfpathlineto{\pgfqpoint{3.830439in}{1.427569in}}%
\pgfpathlineto{\pgfqpoint{3.816218in}{1.435158in}}%
\pgfpathlineto{\pgfqpoint{3.802002in}{1.442846in}}%
\pgfpathlineto{\pgfqpoint{3.787791in}{1.450634in}}%
\pgfpathlineto{\pgfqpoint{3.796181in}{1.449965in}}%
\pgfpathlineto{\pgfqpoint{3.804560in}{1.449640in}}%
\pgfpathlineto{\pgfqpoint{3.812928in}{1.449654in}}%
\pgfpathlineto{\pgfqpoint{3.821285in}{1.449998in}}%
\pgfpathclose%
\pgfusepath{fill}%
\end{pgfscope}%
\begin{pgfscope}%
\pgfpathrectangle{\pgfqpoint{1.150000in}{0.150000in}}{\pgfqpoint{5.700000in}{5.700000in}}%
\pgfusepath{clip}%
\pgfsetbuttcap%
\pgfsetroundjoin%
\definecolor{currentfill}{rgb}{0.278791,0.062145,0.386592}%
\pgfsetfillcolor{currentfill}%
\pgfsetfillopacity{0.700000}%
\pgfsetlinewidth{0.000000pt}%
\definecolor{currentstroke}{rgb}{0.000000,0.000000,0.000000}%
\pgfsetstrokecolor{currentstroke}%
\pgfsetdash{}{0pt}%
\pgfpathmoveto{\pgfqpoint{4.407998in}{1.387973in}}%
\pgfpathlineto{\pgfqpoint{4.422324in}{1.386198in}}%
\pgfpathlineto{\pgfqpoint{4.436660in}{1.384518in}}%
\pgfpathlineto{\pgfqpoint{4.451004in}{1.382933in}}%
\pgfpathlineto{\pgfqpoint{4.465357in}{1.381442in}}%
\pgfpathlineto{\pgfqpoint{4.457272in}{1.371798in}}%
\pgfpathlineto{\pgfqpoint{4.449182in}{1.362347in}}%
\pgfpathlineto{\pgfqpoint{4.441087in}{1.353094in}}%
\pgfpathlineto{\pgfqpoint{4.432988in}{1.344046in}}%
\pgfpathlineto{\pgfqpoint{4.418626in}{1.346083in}}%
\pgfpathlineto{\pgfqpoint{4.404273in}{1.348215in}}%
\pgfpathlineto{\pgfqpoint{4.389929in}{1.350441in}}%
\pgfpathlineto{\pgfqpoint{4.375593in}{1.352762in}}%
\pgfpathlineto{\pgfqpoint{4.383701in}{1.361258in}}%
\pgfpathlineto{\pgfqpoint{4.391805in}{1.369961in}}%
\pgfpathlineto{\pgfqpoint{4.399904in}{1.378869in}}%
\pgfpathlineto{\pgfqpoint{4.407998in}{1.387973in}}%
\pgfpathclose%
\pgfusepath{fill}%
\end{pgfscope}%
\begin{pgfscope}%
\pgfpathrectangle{\pgfqpoint{1.150000in}{0.150000in}}{\pgfqpoint{5.700000in}{5.700000in}}%
\pgfusepath{clip}%
\pgfsetbuttcap%
\pgfsetroundjoin%
\definecolor{currentfill}{rgb}{0.278012,0.180367,0.486697}%
\pgfsetfillcolor{currentfill}%
\pgfsetfillopacity{0.700000}%
\pgfsetlinewidth{0.000000pt}%
\definecolor{currentstroke}{rgb}{0.000000,0.000000,0.000000}%
\pgfsetstrokecolor{currentstroke}%
\pgfsetdash{}{0pt}%
\pgfpathmoveto{\pgfqpoint{4.856975in}{1.635578in}}%
\pgfpathlineto{\pgfqpoint{4.871496in}{1.638172in}}%
\pgfpathlineto{\pgfqpoint{4.886030in}{1.640861in}}%
\pgfpathlineto{\pgfqpoint{4.900576in}{1.643646in}}%
\pgfpathlineto{\pgfqpoint{4.915134in}{1.646527in}}%
\pgfpathlineto{\pgfqpoint{4.907126in}{1.631806in}}%
\pgfpathlineto{\pgfqpoint{4.899116in}{1.617160in}}%
\pgfpathlineto{\pgfqpoint{4.891102in}{1.602592in}}%
\pgfpathlineto{\pgfqpoint{4.883085in}{1.588108in}}%
\pgfpathlineto{\pgfqpoint{4.868530in}{1.585690in}}%
\pgfpathlineto{\pgfqpoint{4.853987in}{1.583367in}}%
\pgfpathlineto{\pgfqpoint{4.839456in}{1.581139in}}%
\pgfpathlineto{\pgfqpoint{4.824936in}{1.579006in}}%
\pgfpathlineto{\pgfqpoint{4.832951in}{1.593023in}}%
\pgfpathlineto{\pgfqpoint{4.840962in}{1.607126in}}%
\pgfpathlineto{\pgfqpoint{4.848970in}{1.621313in}}%
\pgfpathlineto{\pgfqpoint{4.856975in}{1.635578in}}%
\pgfpathclose%
\pgfusepath{fill}%
\end{pgfscope}%
\begin{pgfscope}%
\pgfpathrectangle{\pgfqpoint{1.150000in}{0.150000in}}{\pgfqpoint{5.700000in}{5.700000in}}%
\pgfusepath{clip}%
\pgfsetbuttcap%
\pgfsetroundjoin%
\definecolor{currentfill}{rgb}{0.250425,0.274290,0.533103}%
\pgfsetfillcolor{currentfill}%
\pgfsetfillopacity{0.700000}%
\pgfsetlinewidth{0.000000pt}%
\definecolor{currentstroke}{rgb}{0.000000,0.000000,0.000000}%
\pgfsetstrokecolor{currentstroke}%
\pgfsetdash{}{0pt}%
\pgfpathmoveto{\pgfqpoint{5.069400in}{1.845038in}}%
\pgfpathlineto{\pgfqpoint{5.084037in}{1.849681in}}%
\pgfpathlineto{\pgfqpoint{5.098688in}{1.854421in}}%
\pgfpathlineto{\pgfqpoint{5.113353in}{1.859259in}}%
\pgfpathlineto{\pgfqpoint{5.128032in}{1.864193in}}%
\pgfpathlineto{\pgfqpoint{5.120051in}{1.848099in}}%
\pgfpathlineto{\pgfqpoint{5.112067in}{1.832021in}}%
\pgfpathlineto{\pgfqpoint{5.104079in}{1.815964in}}%
\pgfpathlineto{\pgfqpoint{5.096088in}{1.799931in}}%
\pgfpathlineto{\pgfqpoint{5.081416in}{1.795408in}}%
\pgfpathlineto{\pgfqpoint{5.066758in}{1.790982in}}%
\pgfpathlineto{\pgfqpoint{5.052112in}{1.786653in}}%
\pgfpathlineto{\pgfqpoint{5.037481in}{1.782419in}}%
\pgfpathlineto{\pgfqpoint{5.045465in}{1.798035in}}%
\pgfpathlineto{\pgfqpoint{5.053447in}{1.813679in}}%
\pgfpathlineto{\pgfqpoint{5.061425in}{1.829348in}}%
\pgfpathlineto{\pgfqpoint{5.069400in}{1.845038in}}%
\pgfpathclose%
\pgfusepath{fill}%
\end{pgfscope}%
\begin{pgfscope}%
\pgfpathrectangle{\pgfqpoint{1.150000in}{0.150000in}}{\pgfqpoint{5.700000in}{5.700000in}}%
\pgfusepath{clip}%
\pgfsetbuttcap%
\pgfsetroundjoin%
\definecolor{currentfill}{rgb}{0.122312,0.633153,0.530398}%
\pgfsetfillcolor{currentfill}%
\pgfsetfillopacity{0.700000}%
\pgfsetlinewidth{0.000000pt}%
\definecolor{currentstroke}{rgb}{0.000000,0.000000,0.000000}%
\pgfsetstrokecolor{currentstroke}%
\pgfsetdash{}{0pt}%
\pgfpathmoveto{\pgfqpoint{2.389960in}{2.795917in}}%
\pgfpathlineto{\pgfqpoint{2.404308in}{2.775597in}}%
\pgfpathlineto{\pgfqpoint{2.418649in}{2.755441in}}%
\pgfpathlineto{\pgfqpoint{2.432983in}{2.735446in}}%
\pgfpathlineto{\pgfqpoint{2.447310in}{2.715612in}}%
\pgfpathlineto{\pgfqpoint{2.437656in}{2.733852in}}%
\pgfpathlineto{\pgfqpoint{2.427969in}{2.752623in}}%
\pgfpathlineto{\pgfqpoint{2.418250in}{2.771934in}}%
\pgfpathlineto{\pgfqpoint{2.408496in}{2.791793in}}%
\pgfpathlineto{\pgfqpoint{2.394094in}{2.812366in}}%
\pgfpathlineto{\pgfqpoint{2.379684in}{2.833101in}}%
\pgfpathlineto{\pgfqpoint{2.365267in}{2.853999in}}%
\pgfpathlineto{\pgfqpoint{2.350843in}{2.875061in}}%
\pgfpathlineto{\pgfqpoint{2.360674in}{2.854450in}}%
\pgfpathlineto{\pgfqpoint{2.370470in}{2.834394in}}%
\pgfpathlineto{\pgfqpoint{2.380232in}{2.814886in}}%
\pgfpathlineto{\pgfqpoint{2.389960in}{2.795917in}}%
\pgfpathclose%
\pgfusepath{fill}%
\end{pgfscope}%
\begin{pgfscope}%
\pgfpathrectangle{\pgfqpoint{1.150000in}{0.150000in}}{\pgfqpoint{5.700000in}{5.700000in}}%
\pgfusepath{clip}%
\pgfsetbuttcap%
\pgfsetroundjoin%
\definecolor{currentfill}{rgb}{0.241237,0.296485,0.539709}%
\pgfsetfillcolor{currentfill}%
\pgfsetfillopacity{0.700000}%
\pgfsetlinewidth{0.000000pt}%
\definecolor{currentstroke}{rgb}{0.000000,0.000000,0.000000}%
\pgfsetstrokecolor{currentstroke}%
\pgfsetdash{}{0pt}%
\pgfpathmoveto{\pgfqpoint{3.186517in}{1.879840in}}%
\pgfpathlineto{\pgfqpoint{3.200681in}{1.867132in}}%
\pgfpathlineto{\pgfqpoint{3.214845in}{1.854539in}}%
\pgfpathlineto{\pgfqpoint{3.229009in}{1.842061in}}%
\pgfpathlineto{\pgfqpoint{3.243174in}{1.829697in}}%
\pgfpathlineto{\pgfqpoint{3.234360in}{1.838180in}}%
\pgfpathlineto{\pgfqpoint{3.225528in}{1.847102in}}%
\pgfpathlineto{\pgfqpoint{3.216675in}{1.856469in}}%
\pgfpathlineto{\pgfqpoint{3.207802in}{1.866290in}}%
\pgfpathlineto{\pgfqpoint{3.193588in}{1.879337in}}%
\pgfpathlineto{\pgfqpoint{3.179375in}{1.892500in}}%
\pgfpathlineto{\pgfqpoint{3.165160in}{1.905777in}}%
\pgfpathlineto{\pgfqpoint{3.150946in}{1.919170in}}%
\pgfpathlineto{\pgfqpoint{3.159870in}{1.908656in}}%
\pgfpathlineto{\pgfqpoint{3.168773in}{1.898601in}}%
\pgfpathlineto{\pgfqpoint{3.177655in}{1.888998in}}%
\pgfpathlineto{\pgfqpoint{3.186517in}{1.879840in}}%
\pgfpathclose%
\pgfusepath{fill}%
\end{pgfscope}%
\begin{pgfscope}%
\pgfpathrectangle{\pgfqpoint{1.150000in}{0.150000in}}{\pgfqpoint{5.700000in}{5.700000in}}%
\pgfusepath{clip}%
\pgfsetbuttcap%
\pgfsetroundjoin%
\definecolor{currentfill}{rgb}{0.277018,0.050344,0.375715}%
\pgfsetfillcolor{currentfill}%
\pgfsetfillopacity{0.700000}%
\pgfsetlinewidth{0.000000pt}%
\definecolor{currentstroke}{rgb}{0.000000,0.000000,0.000000}%
\pgfsetstrokecolor{currentstroke}%
\pgfsetdash{}{0pt}%
\pgfpathmoveto{\pgfqpoint{4.171545in}{1.363007in}}%
\pgfpathlineto{\pgfqpoint{4.185802in}{1.358980in}}%
\pgfpathlineto{\pgfqpoint{4.200065in}{1.355049in}}%
\pgfpathlineto{\pgfqpoint{4.214335in}{1.351213in}}%
\pgfpathlineto{\pgfqpoint{4.228613in}{1.347473in}}%
\pgfpathlineto{\pgfqpoint{4.220451in}{1.341256in}}%
\pgfpathlineto{\pgfqpoint{4.212283in}{1.335291in}}%
\pgfpathlineto{\pgfqpoint{4.204108in}{1.329583in}}%
\pgfpathlineto{\pgfqpoint{4.195927in}{1.324140in}}%
\pgfpathlineto{\pgfqpoint{4.181633in}{1.328462in}}%
\pgfpathlineto{\pgfqpoint{4.167346in}{1.332880in}}%
\pgfpathlineto{\pgfqpoint{4.153066in}{1.337394in}}%
\pgfpathlineto{\pgfqpoint{4.138793in}{1.342003in}}%
\pgfpathlineto{\pgfqpoint{4.146992in}{1.346857in}}%
\pgfpathlineto{\pgfqpoint{4.155183in}{1.351980in}}%
\pgfpathlineto{\pgfqpoint{4.163368in}{1.357365in}}%
\pgfpathlineto{\pgfqpoint{4.171545in}{1.363007in}}%
\pgfpathclose%
\pgfusepath{fill}%
\end{pgfscope}%
\begin{pgfscope}%
\pgfpathrectangle{\pgfqpoint{1.150000in}{0.150000in}}{\pgfqpoint{5.700000in}{5.700000in}}%
\pgfusepath{clip}%
\pgfsetbuttcap%
\pgfsetroundjoin%
\definecolor{currentfill}{rgb}{0.278791,0.062145,0.386592}%
\pgfsetfillcolor{currentfill}%
\pgfsetfillopacity{0.700000}%
\pgfsetlinewidth{0.000000pt}%
\definecolor{currentstroke}{rgb}{0.000000,0.000000,0.000000}%
\pgfsetstrokecolor{currentstroke}%
\pgfsetdash{}{0pt}%
\pgfpathmoveto{\pgfqpoint{4.024839in}{1.382346in}}%
\pgfpathlineto{\pgfqpoint{4.039062in}{1.376964in}}%
\pgfpathlineto{\pgfqpoint{4.053290in}{1.371680in}}%
\pgfpathlineto{\pgfqpoint{4.067525in}{1.366493in}}%
\pgfpathlineto{\pgfqpoint{4.081766in}{1.361402in}}%
\pgfpathlineto{\pgfqpoint{4.073541in}{1.357417in}}%
\pgfpathlineto{\pgfqpoint{4.065308in}{1.353718in}}%
\pgfpathlineto{\pgfqpoint{4.057067in}{1.350311in}}%
\pgfpathlineto{\pgfqpoint{4.048818in}{1.347202in}}%
\pgfpathlineto{\pgfqpoint{4.034557in}{1.352894in}}%
\pgfpathlineto{\pgfqpoint{4.020301in}{1.358684in}}%
\pgfpathlineto{\pgfqpoint{4.006051in}{1.364570in}}%
\pgfpathlineto{\pgfqpoint{3.991807in}{1.370553in}}%
\pgfpathlineto{\pgfqpoint{4.000078in}{1.373053in}}%
\pgfpathlineto{\pgfqpoint{4.008340in}{1.375855in}}%
\pgfpathlineto{\pgfqpoint{4.016594in}{1.378955in}}%
\pgfpathlineto{\pgfqpoint{4.024839in}{1.382346in}}%
\pgfpathclose%
\pgfusepath{fill}%
\end{pgfscope}%
\begin{pgfscope}%
\pgfpathrectangle{\pgfqpoint{1.150000in}{0.150000in}}{\pgfqpoint{5.700000in}{5.700000in}}%
\pgfusepath{clip}%
\pgfsetbuttcap%
\pgfsetroundjoin%
\definecolor{currentfill}{rgb}{0.506271,0.828786,0.300362}%
\pgfsetfillcolor{currentfill}%
\pgfsetfillopacity{0.700000}%
\pgfsetlinewidth{0.000000pt}%
\definecolor{currentstroke}{rgb}{0.000000,0.000000,0.000000}%
\pgfsetstrokecolor{currentstroke}%
\pgfsetdash{}{0pt}%
\pgfpathmoveto{\pgfqpoint{1.967141in}{3.447029in}}%
\pgfpathlineto{\pgfqpoint{1.981728in}{3.421421in}}%
\pgfpathlineto{\pgfqpoint{1.996301in}{3.396029in}}%
\pgfpathlineto{\pgfqpoint{2.010862in}{3.370851in}}%
\pgfpathlineto{\pgfqpoint{2.025410in}{3.345886in}}%
\pgfpathlineto{\pgfqpoint{2.015264in}{3.367854in}}%
\pgfpathlineto{\pgfqpoint{2.005078in}{3.390381in}}%
\pgfpathlineto{\pgfqpoint{1.994853in}{3.413475in}}%
\pgfpathlineto{\pgfqpoint{1.984588in}{3.437146in}}%
\pgfpathlineto{\pgfqpoint{1.969952in}{3.462873in}}%
\pgfpathlineto{\pgfqpoint{1.955303in}{3.488814in}}%
\pgfpathlineto{\pgfqpoint{1.940640in}{3.514972in}}%
\pgfpathlineto{\pgfqpoint{1.925964in}{3.541348in}}%
\pgfpathlineto{\pgfqpoint{1.936320in}{3.516900in}}%
\pgfpathlineto{\pgfqpoint{1.946634in}{3.493037in}}%
\pgfpathlineto{\pgfqpoint{1.956908in}{3.469749in}}%
\pgfpathlineto{\pgfqpoint{1.967141in}{3.447029in}}%
\pgfpathclose%
\pgfusepath{fill}%
\end{pgfscope}%
\begin{pgfscope}%
\pgfpathrectangle{\pgfqpoint{1.150000in}{0.150000in}}{\pgfqpoint{5.700000in}{5.700000in}}%
\pgfusepath{clip}%
\pgfsetbuttcap%
\pgfsetroundjoin%
\definecolor{currentfill}{rgb}{0.281887,0.150881,0.465405}%
\pgfsetfillcolor{currentfill}%
\pgfsetfillopacity{0.700000}%
\pgfsetlinewidth{0.000000pt}%
\definecolor{currentstroke}{rgb}{0.000000,0.000000,0.000000}%
\pgfsetstrokecolor{currentstroke}%
\pgfsetdash{}{0pt}%
\pgfpathmoveto{\pgfqpoint{3.617549in}{1.551973in}}%
\pgfpathlineto{\pgfqpoint{3.631717in}{1.542967in}}%
\pgfpathlineto{\pgfqpoint{3.645888in}{1.534063in}}%
\pgfpathlineto{\pgfqpoint{3.660062in}{1.525262in}}%
\pgfpathlineto{\pgfqpoint{3.674239in}{1.516564in}}%
\pgfpathlineto{\pgfqpoint{3.665774in}{1.518860in}}%
\pgfpathlineto{\pgfqpoint{3.657295in}{1.521526in}}%
\pgfpathlineto{\pgfqpoint{3.648803in}{1.524567in}}%
\pgfpathlineto{\pgfqpoint{3.640298in}{1.527992in}}%
\pgfpathlineto{\pgfqpoint{3.626085in}{1.537338in}}%
\pgfpathlineto{\pgfqpoint{3.611876in}{1.546786in}}%
\pgfpathlineto{\pgfqpoint{3.597670in}{1.556338in}}%
\pgfpathlineto{\pgfqpoint{3.583466in}{1.565993in}}%
\pgfpathlineto{\pgfqpoint{3.592008in}{1.561912in}}%
\pgfpathlineto{\pgfqpoint{3.600536in}{1.558220in}}%
\pgfpathlineto{\pgfqpoint{3.609049in}{1.554910in}}%
\pgfpathlineto{\pgfqpoint{3.617549in}{1.551973in}}%
\pgfpathclose%
\pgfusepath{fill}%
\end{pgfscope}%
\begin{pgfscope}%
\pgfpathrectangle{\pgfqpoint{1.150000in}{0.150000in}}{\pgfqpoint{5.700000in}{5.700000in}}%
\pgfusepath{clip}%
\pgfsetbuttcap%
\pgfsetroundjoin%
\definecolor{currentfill}{rgb}{0.248629,0.278775,0.534556}%
\pgfsetfillcolor{currentfill}%
\pgfsetfillopacity{0.700000}%
\pgfsetlinewidth{0.000000pt}%
\definecolor{currentstroke}{rgb}{0.000000,0.000000,0.000000}%
\pgfsetstrokecolor{currentstroke}%
\pgfsetdash{}{0pt}%
\pgfpathmoveto{\pgfqpoint{3.243174in}{1.829697in}}%
\pgfpathlineto{\pgfqpoint{3.257339in}{1.817447in}}%
\pgfpathlineto{\pgfqpoint{3.271504in}{1.805309in}}%
\pgfpathlineto{\pgfqpoint{3.285670in}{1.793284in}}%
\pgfpathlineto{\pgfqpoint{3.299837in}{1.781372in}}%
\pgfpathlineto{\pgfqpoint{3.291070in}{1.789182in}}%
\pgfpathlineto{\pgfqpoint{3.282285in}{1.797424in}}%
\pgfpathlineto{\pgfqpoint{3.273481in}{1.806107in}}%
\pgfpathlineto{\pgfqpoint{3.264658in}{1.815236in}}%
\pgfpathlineto{\pgfqpoint{3.250444in}{1.827830in}}%
\pgfpathlineto{\pgfqpoint{3.236230in}{1.840537in}}%
\pgfpathlineto{\pgfqpoint{3.222016in}{1.853356in}}%
\pgfpathlineto{\pgfqpoint{3.207802in}{1.866290in}}%
\pgfpathlineto{\pgfqpoint{3.216675in}{1.856469in}}%
\pgfpathlineto{\pgfqpoint{3.225528in}{1.847102in}}%
\pgfpathlineto{\pgfqpoint{3.234360in}{1.838180in}}%
\pgfpathlineto{\pgfqpoint{3.243174in}{1.829697in}}%
\pgfpathclose%
\pgfusepath{fill}%
\end{pgfscope}%
\begin{pgfscope}%
\pgfpathrectangle{\pgfqpoint{1.150000in}{0.150000in}}{\pgfqpoint{5.700000in}{5.700000in}}%
\pgfusepath{clip}%
\pgfsetbuttcap%
\pgfsetroundjoin%
\definecolor{currentfill}{rgb}{0.137339,0.662252,0.515571}%
\pgfsetfillcolor{currentfill}%
\pgfsetfillopacity{0.700000}%
\pgfsetlinewidth{0.000000pt}%
\definecolor{currentstroke}{rgb}{0.000000,0.000000,0.000000}%
\pgfsetstrokecolor{currentstroke}%
\pgfsetdash{}{0pt}%
\pgfpathmoveto{\pgfqpoint{2.332493in}{2.878855in}}%
\pgfpathlineto{\pgfqpoint{2.346871in}{2.857869in}}%
\pgfpathlineto{\pgfqpoint{2.361241in}{2.837052in}}%
\pgfpathlineto{\pgfqpoint{2.375604in}{2.816401in}}%
\pgfpathlineto{\pgfqpoint{2.389960in}{2.795917in}}%
\pgfpathlineto{\pgfqpoint{2.380232in}{2.814886in}}%
\pgfpathlineto{\pgfqpoint{2.370470in}{2.834394in}}%
\pgfpathlineto{\pgfqpoint{2.360674in}{2.854450in}}%
\pgfpathlineto{\pgfqpoint{2.350843in}{2.875061in}}%
\pgfpathlineto{\pgfqpoint{2.336410in}{2.896290in}}%
\pgfpathlineto{\pgfqpoint{2.321970in}{2.917686in}}%
\pgfpathlineto{\pgfqpoint{2.307521in}{2.939251in}}%
\pgfpathlineto{\pgfqpoint{2.293065in}{2.960985in}}%
\pgfpathlineto{\pgfqpoint{2.302975in}{2.939616in}}%
\pgfpathlineto{\pgfqpoint{2.312849in}{2.918810in}}%
\pgfpathlineto{\pgfqpoint{2.322688in}{2.898559in}}%
\pgfpathlineto{\pgfqpoint{2.332493in}{2.878855in}}%
\pgfpathclose%
\pgfusepath{fill}%
\end{pgfscope}%
\begin{pgfscope}%
\pgfpathrectangle{\pgfqpoint{1.150000in}{0.150000in}}{\pgfqpoint{5.700000in}{5.700000in}}%
\pgfusepath{clip}%
\pgfsetbuttcap%
\pgfsetroundjoin%
\definecolor{currentfill}{rgb}{0.204903,0.375746,0.553533}%
\pgfsetfillcolor{currentfill}%
\pgfsetfillopacity{0.700000}%
\pgfsetlinewidth{0.000000pt}%
\definecolor{currentstroke}{rgb}{0.000000,0.000000,0.000000}%
\pgfsetstrokecolor{currentstroke}%
\pgfsetdash{}{0pt}%
\pgfpathmoveto{\pgfqpoint{5.282464in}{2.082503in}}%
\pgfpathlineto{\pgfqpoint{5.297238in}{2.089054in}}%
\pgfpathlineto{\pgfqpoint{5.312026in}{2.095703in}}%
\pgfpathlineto{\pgfqpoint{5.326830in}{2.102452in}}%
\pgfpathlineto{\pgfqpoint{5.318881in}{2.085826in}}%
\pgfpathlineto{\pgfqpoint{5.310928in}{2.069166in}}%
\pgfpathlineto{\pgfqpoint{5.302970in}{2.052476in}}%
\pgfpathlineto{\pgfqpoint{5.295008in}{2.035758in}}%
\pgfpathlineto{\pgfqpoint{5.280213in}{2.029369in}}%
\pgfpathlineto{\pgfqpoint{5.265432in}{2.023078in}}%
\pgfpathlineto{\pgfqpoint{5.250667in}{2.016886in}}%
\pgfpathlineto{\pgfqpoint{5.258622in}{2.033329in}}%
\pgfpathlineto{\pgfqpoint{5.266574in}{2.049748in}}%
\pgfpathlineto{\pgfqpoint{5.274521in}{2.066140in}}%
\pgfpathlineto{\pgfqpoint{5.282464in}{2.082503in}}%
\pgfpathclose%
\pgfusepath{fill}%
\end{pgfscope}%
\begin{pgfscope}%
\pgfpathrectangle{\pgfqpoint{1.150000in}{0.150000in}}{\pgfqpoint{5.700000in}{5.700000in}}%
\pgfusepath{clip}%
\pgfsetbuttcap%
\pgfsetroundjoin%
\definecolor{currentfill}{rgb}{0.277941,0.056324,0.381191}%
\pgfsetfillcolor{currentfill}%
\pgfsetfillopacity{0.700000}%
\pgfsetlinewidth{0.000000pt}%
\definecolor{currentstroke}{rgb}{0.000000,0.000000,0.000000}%
\pgfsetstrokecolor{currentstroke}%
\pgfsetdash{}{0pt}%
\pgfpathmoveto{\pgfqpoint{4.318332in}{1.362994in}}%
\pgfpathlineto{\pgfqpoint{4.332635in}{1.360293in}}%
\pgfpathlineto{\pgfqpoint{4.346946in}{1.357688in}}%
\pgfpathlineto{\pgfqpoint{4.361265in}{1.355178in}}%
\pgfpathlineto{\pgfqpoint{4.375593in}{1.352762in}}%
\pgfpathlineto{\pgfqpoint{4.367479in}{1.344481in}}%
\pgfpathlineto{\pgfqpoint{4.359360in}{1.336420in}}%
\pgfpathlineto{\pgfqpoint{4.351236in}{1.328583in}}%
\pgfpathlineto{\pgfqpoint{4.343107in}{1.320978in}}%
\pgfpathlineto{\pgfqpoint{4.328768in}{1.323958in}}%
\pgfpathlineto{\pgfqpoint{4.314437in}{1.327032in}}%
\pgfpathlineto{\pgfqpoint{4.300114in}{1.330202in}}%
\pgfpathlineto{\pgfqpoint{4.285799in}{1.333466in}}%
\pgfpathlineto{\pgfqpoint{4.293940in}{1.340500in}}%
\pgfpathlineto{\pgfqpoint{4.302076in}{1.347770in}}%
\pgfpathlineto{\pgfqpoint{4.310207in}{1.355270in}}%
\pgfpathlineto{\pgfqpoint{4.318332in}{1.362994in}}%
\pgfpathclose%
\pgfusepath{fill}%
\end{pgfscope}%
\begin{pgfscope}%
\pgfpathrectangle{\pgfqpoint{1.150000in}{0.150000in}}{\pgfqpoint{5.700000in}{5.700000in}}%
\pgfusepath{clip}%
\pgfsetbuttcap%
\pgfsetroundjoin%
\definecolor{currentfill}{rgb}{0.269308,0.218818,0.509577}%
\pgfsetfillcolor{currentfill}%
\pgfsetfillopacity{0.700000}%
\pgfsetlinewidth{0.000000pt}%
\definecolor{currentstroke}{rgb}{0.000000,0.000000,0.000000}%
\pgfsetstrokecolor{currentstroke}%
\pgfsetdash{}{0pt}%
\pgfpathmoveto{\pgfqpoint{4.947135in}{1.706064in}}%
\pgfpathlineto{\pgfqpoint{4.961710in}{1.709485in}}%
\pgfpathlineto{\pgfqpoint{4.976297in}{1.713001in}}%
\pgfpathlineto{\pgfqpoint{4.990897in}{1.716614in}}%
\pgfpathlineto{\pgfqpoint{5.005511in}{1.720323in}}%
\pgfpathlineto{\pgfqpoint{4.997510in}{1.704910in}}%
\pgfpathlineto{\pgfqpoint{4.989507in}{1.689550in}}%
\pgfpathlineto{\pgfqpoint{4.981501in}{1.674246in}}%
\pgfpathlineto{\pgfqpoint{4.973492in}{1.659004in}}%
\pgfpathlineto{\pgfqpoint{4.958884in}{1.655741in}}%
\pgfpathlineto{\pgfqpoint{4.944288in}{1.652574in}}%
\pgfpathlineto{\pgfqpoint{4.929705in}{1.649503in}}%
\pgfpathlineto{\pgfqpoint{4.915134in}{1.646527in}}%
\pgfpathlineto{\pgfqpoint{4.923139in}{1.661317in}}%
\pgfpathlineto{\pgfqpoint{4.931141in}{1.676173in}}%
\pgfpathlineto{\pgfqpoint{4.939140in}{1.691090in}}%
\pgfpathlineto{\pgfqpoint{4.947135in}{1.706064in}}%
\pgfpathclose%
\pgfusepath{fill}%
\end{pgfscope}%
\begin{pgfscope}%
\pgfpathrectangle{\pgfqpoint{1.150000in}{0.150000in}}{\pgfqpoint{5.700000in}{5.700000in}}%
\pgfusepath{clip}%
\pgfsetbuttcap%
\pgfsetroundjoin%
\definecolor{currentfill}{rgb}{0.281924,0.089666,0.412415}%
\pgfsetfillcolor{currentfill}%
\pgfsetfillopacity{0.700000}%
\pgfsetlinewidth{0.000000pt}%
\definecolor{currentstroke}{rgb}{0.000000,0.000000,0.000000}%
\pgfsetstrokecolor{currentstroke}%
\pgfsetdash{}{0pt}%
\pgfpathmoveto{\pgfqpoint{3.878045in}{1.421938in}}%
\pgfpathlineto{\pgfqpoint{3.892247in}{1.415171in}}%
\pgfpathlineto{\pgfqpoint{3.906454in}{1.408502in}}%
\pgfpathlineto{\pgfqpoint{3.920667in}{1.401932in}}%
\pgfpathlineto{\pgfqpoint{3.934884in}{1.395461in}}%
\pgfpathlineto{\pgfqpoint{3.926580in}{1.393885in}}%
\pgfpathlineto{\pgfqpoint{3.918266in}{1.392631in}}%
\pgfpathlineto{\pgfqpoint{3.909942in}{1.391704in}}%
\pgfpathlineto{\pgfqpoint{3.901609in}{1.391111in}}%
\pgfpathlineto{\pgfqpoint{3.887365in}{1.398206in}}%
\pgfpathlineto{\pgfqpoint{3.873127in}{1.405398in}}%
\pgfpathlineto{\pgfqpoint{3.858893in}{1.412690in}}%
\pgfpathlineto{\pgfqpoint{3.844664in}{1.420080in}}%
\pgfpathlineto{\pgfqpoint{3.853025in}{1.420042in}}%
\pgfpathlineto{\pgfqpoint{3.861375in}{1.420343in}}%
\pgfpathlineto{\pgfqpoint{3.869715in}{1.420978in}}%
\pgfpathlineto{\pgfqpoint{3.878045in}{1.421938in}}%
\pgfpathclose%
\pgfusepath{fill}%
\end{pgfscope}%
\begin{pgfscope}%
\pgfpathrectangle{\pgfqpoint{1.150000in}{0.150000in}}{\pgfqpoint{5.700000in}{5.700000in}}%
\pgfusepath{clip}%
\pgfsetbuttcap%
\pgfsetroundjoin%
\definecolor{currentfill}{rgb}{0.255645,0.260703,0.528312}%
\pgfsetfillcolor{currentfill}%
\pgfsetfillopacity{0.700000}%
\pgfsetlinewidth{0.000000pt}%
\definecolor{currentstroke}{rgb}{0.000000,0.000000,0.000000}%
\pgfsetstrokecolor{currentstroke}%
\pgfsetdash{}{0pt}%
\pgfpathmoveto{\pgfqpoint{3.299837in}{1.781372in}}%
\pgfpathlineto{\pgfqpoint{3.314004in}{1.769570in}}%
\pgfpathlineto{\pgfqpoint{3.328172in}{1.757880in}}%
\pgfpathlineto{\pgfqpoint{3.342342in}{1.746301in}}%
\pgfpathlineto{\pgfqpoint{3.356512in}{1.734832in}}%
\pgfpathlineto{\pgfqpoint{3.347791in}{1.741972in}}%
\pgfpathlineto{\pgfqpoint{3.339052in}{1.749538in}}%
\pgfpathlineto{\pgfqpoint{3.330295in}{1.757537in}}%
\pgfpathlineto{\pgfqpoint{3.321519in}{1.765978in}}%
\pgfpathlineto{\pgfqpoint{3.307303in}{1.778126in}}%
\pgfpathlineto{\pgfqpoint{3.293087in}{1.790385in}}%
\pgfpathlineto{\pgfqpoint{3.278872in}{1.802754in}}%
\pgfpathlineto{\pgfqpoint{3.264658in}{1.815236in}}%
\pgfpathlineto{\pgfqpoint{3.273481in}{1.806107in}}%
\pgfpathlineto{\pgfqpoint{3.282285in}{1.797424in}}%
\pgfpathlineto{\pgfqpoint{3.291070in}{1.789182in}}%
\pgfpathlineto{\pgfqpoint{3.299837in}{1.781372in}}%
\pgfpathclose%
\pgfusepath{fill}%
\end{pgfscope}%
\begin{pgfscope}%
\pgfpathrectangle{\pgfqpoint{1.150000in}{0.150000in}}{\pgfqpoint{5.700000in}{5.700000in}}%
\pgfusepath{clip}%
\pgfsetbuttcap%
\pgfsetroundjoin%
\definecolor{currentfill}{rgb}{0.231674,0.318106,0.544834}%
\pgfsetfillcolor{currentfill}%
\pgfsetfillopacity{0.700000}%
\pgfsetlinewidth{0.000000pt}%
\definecolor{currentstroke}{rgb}{0.000000,0.000000,0.000000}%
\pgfsetstrokecolor{currentstroke}%
\pgfsetdash{}{0pt}%
\pgfpathmoveto{\pgfqpoint{5.159921in}{1.928659in}}%
\pgfpathlineto{\pgfqpoint{5.174621in}{1.934084in}}%
\pgfpathlineto{\pgfqpoint{5.189335in}{1.939607in}}%
\pgfpathlineto{\pgfqpoint{5.204063in}{1.945227in}}%
\pgfpathlineto{\pgfqpoint{5.218806in}{1.950945in}}%
\pgfpathlineto{\pgfqpoint{5.210832in}{1.934434in}}%
\pgfpathlineto{\pgfqpoint{5.202854in}{1.917921in}}%
\pgfpathlineto{\pgfqpoint{5.194872in}{1.901409in}}%
\pgfpathlineto{\pgfqpoint{5.186887in}{1.884901in}}%
\pgfpathlineto{\pgfqpoint{5.172152in}{1.879578in}}%
\pgfpathlineto{\pgfqpoint{5.157431in}{1.874353in}}%
\pgfpathlineto{\pgfqpoint{5.142725in}{1.869224in}}%
\pgfpathlineto{\pgfqpoint{5.128032in}{1.864193in}}%
\pgfpathlineto{\pgfqpoint{5.136009in}{1.880300in}}%
\pgfpathlineto{\pgfqpoint{5.143984in}{1.896415in}}%
\pgfpathlineto{\pgfqpoint{5.151954in}{1.912536in}}%
\pgfpathlineto{\pgfqpoint{5.159921in}{1.928659in}}%
\pgfpathclose%
\pgfusepath{fill}%
\end{pgfscope}%
\begin{pgfscope}%
\pgfpathrectangle{\pgfqpoint{1.150000in}{0.150000in}}{\pgfqpoint{5.700000in}{5.700000in}}%
\pgfusepath{clip}%
\pgfsetbuttcap%
\pgfsetroundjoin%
\definecolor{currentfill}{rgb}{0.170948,0.694384,0.493803}%
\pgfsetfillcolor{currentfill}%
\pgfsetfillopacity{0.700000}%
\pgfsetlinewidth{0.000000pt}%
\definecolor{currentstroke}{rgb}{0.000000,0.000000,0.000000}%
\pgfsetstrokecolor{currentstroke}%
\pgfsetdash{}{0pt}%
\pgfpathmoveto{\pgfqpoint{2.274898in}{2.964513in}}%
\pgfpathlineto{\pgfqpoint{2.289309in}{2.942838in}}%
\pgfpathlineto{\pgfqpoint{2.303712in}{2.921338in}}%
\pgfpathlineto{\pgfqpoint{2.318106in}{2.900011in}}%
\pgfpathlineto{\pgfqpoint{2.332493in}{2.878855in}}%
\pgfpathlineto{\pgfqpoint{2.322688in}{2.898559in}}%
\pgfpathlineto{\pgfqpoint{2.312849in}{2.918810in}}%
\pgfpathlineto{\pgfqpoint{2.302975in}{2.939616in}}%
\pgfpathlineto{\pgfqpoint{2.293065in}{2.960985in}}%
\pgfpathlineto{\pgfqpoint{2.278599in}{2.982892in}}%
\pgfpathlineto{\pgfqpoint{2.264125in}{3.004971in}}%
\pgfpathlineto{\pgfqpoint{2.249642in}{3.027225in}}%
\pgfpathlineto{\pgfqpoint{2.235150in}{3.049655in}}%
\pgfpathlineto{\pgfqpoint{2.245142in}{3.027521in}}%
\pgfpathlineto{\pgfqpoint{2.255097in}{3.005958in}}%
\pgfpathlineto{\pgfqpoint{2.265016in}{2.984959in}}%
\pgfpathlineto{\pgfqpoint{2.274898in}{2.964513in}}%
\pgfpathclose%
\pgfusepath{fill}%
\end{pgfscope}%
\begin{pgfscope}%
\pgfpathrectangle{\pgfqpoint{1.150000in}{0.150000in}}{\pgfqpoint{5.700000in}{5.700000in}}%
\pgfusepath{clip}%
\pgfsetbuttcap%
\pgfsetroundjoin%
\definecolor{currentfill}{rgb}{0.283091,0.110553,0.431554}%
\pgfsetfillcolor{currentfill}%
\pgfsetfillopacity{0.700000}%
\pgfsetlinewidth{0.000000pt}%
\definecolor{currentstroke}{rgb}{0.000000,0.000000,0.000000}%
\pgfsetstrokecolor{currentstroke}%
\pgfsetdash{}{0pt}%
\pgfpathmoveto{\pgfqpoint{4.644968in}{1.464714in}}%
\pgfpathlineto{\pgfqpoint{4.659399in}{1.465114in}}%
\pgfpathlineto{\pgfqpoint{4.673840in}{1.465609in}}%
\pgfpathlineto{\pgfqpoint{4.688292in}{1.466199in}}%
\pgfpathlineto{\pgfqpoint{4.702755in}{1.466883in}}%
\pgfpathlineto{\pgfqpoint{4.694713in}{1.454366in}}%
\pgfpathlineto{\pgfqpoint{4.686668in}{1.441988in}}%
\pgfpathlineto{\pgfqpoint{4.678619in}{1.429753in}}%
\pgfpathlineto{\pgfqpoint{4.670567in}{1.417667in}}%
\pgfpathlineto{\pgfqpoint{4.656103in}{1.417496in}}%
\pgfpathlineto{\pgfqpoint{4.641649in}{1.417419in}}%
\pgfpathlineto{\pgfqpoint{4.627205in}{1.417437in}}%
\pgfpathlineto{\pgfqpoint{4.612771in}{1.417549in}}%
\pgfpathlineto{\pgfqpoint{4.620826in}{1.429116in}}%
\pgfpathlineto{\pgfqpoint{4.628877in}{1.440835in}}%
\pgfpathlineto{\pgfqpoint{4.636924in}{1.452703in}}%
\pgfpathlineto{\pgfqpoint{4.644968in}{1.464714in}}%
\pgfpathclose%
\pgfusepath{fill}%
\end{pgfscope}%
\begin{pgfscope}%
\pgfpathrectangle{\pgfqpoint{1.150000in}{0.150000in}}{\pgfqpoint{5.700000in}{5.700000in}}%
\pgfusepath{clip}%
\pgfsetbuttcap%
\pgfsetroundjoin%
\definecolor{currentfill}{rgb}{0.281924,0.089666,0.412415}%
\pgfsetfillcolor{currentfill}%
\pgfsetfillopacity{0.700000}%
\pgfsetlinewidth{0.000000pt}%
\definecolor{currentstroke}{rgb}{0.000000,0.000000,0.000000}%
\pgfsetstrokecolor{currentstroke}%
\pgfsetdash{}{0pt}%
\pgfpathmoveto{\pgfqpoint{4.555136in}{1.418940in}}%
\pgfpathlineto{\pgfqpoint{4.569530in}{1.418451in}}%
\pgfpathlineto{\pgfqpoint{4.583934in}{1.418056in}}%
\pgfpathlineto{\pgfqpoint{4.598347in}{1.417755in}}%
\pgfpathlineto{\pgfqpoint{4.612771in}{1.417549in}}%
\pgfpathlineto{\pgfqpoint{4.604713in}{1.406140in}}%
\pgfpathlineto{\pgfqpoint{4.596651in}{1.394894in}}%
\pgfpathlineto{\pgfqpoint{4.588585in}{1.383817in}}%
\pgfpathlineto{\pgfqpoint{4.580516in}{1.372913in}}%
\pgfpathlineto{\pgfqpoint{4.566088in}{1.373649in}}%
\pgfpathlineto{\pgfqpoint{4.551670in}{1.374479in}}%
\pgfpathlineto{\pgfqpoint{4.537261in}{1.375404in}}%
\pgfpathlineto{\pgfqpoint{4.522862in}{1.376423in}}%
\pgfpathlineto{\pgfqpoint{4.530936in}{1.386790in}}%
\pgfpathlineto{\pgfqpoint{4.539007in}{1.397336in}}%
\pgfpathlineto{\pgfqpoint{4.547073in}{1.408054in}}%
\pgfpathlineto{\pgfqpoint{4.555136in}{1.418940in}}%
\pgfpathclose%
\pgfusepath{fill}%
\end{pgfscope}%
\begin{pgfscope}%
\pgfpathrectangle{\pgfqpoint{1.150000in}{0.150000in}}{\pgfqpoint{5.700000in}{5.700000in}}%
\pgfusepath{clip}%
\pgfsetbuttcap%
\pgfsetroundjoin%
\definecolor{currentfill}{rgb}{0.282884,0.135920,0.453427}%
\pgfsetfillcolor{currentfill}%
\pgfsetfillopacity{0.700000}%
\pgfsetlinewidth{0.000000pt}%
\definecolor{currentstroke}{rgb}{0.000000,0.000000,0.000000}%
\pgfsetstrokecolor{currentstroke}%
\pgfsetdash{}{0pt}%
\pgfpathmoveto{\pgfqpoint{4.734891in}{1.518238in}}%
\pgfpathlineto{\pgfqpoint{4.749364in}{1.519513in}}%
\pgfpathlineto{\pgfqpoint{4.763848in}{1.520882in}}%
\pgfpathlineto{\pgfqpoint{4.778343in}{1.522346in}}%
\pgfpathlineto{\pgfqpoint{4.792849in}{1.523904in}}%
\pgfpathlineto{\pgfqpoint{4.784820in}{1.510392in}}%
\pgfpathlineto{\pgfqpoint{4.776788in}{1.496995in}}%
\pgfpathlineto{\pgfqpoint{4.768752in}{1.483718in}}%
\pgfpathlineto{\pgfqpoint{4.760714in}{1.470564in}}%
\pgfpathlineto{\pgfqpoint{4.746208in}{1.469502in}}%
\pgfpathlineto{\pgfqpoint{4.731713in}{1.468535in}}%
\pgfpathlineto{\pgfqpoint{4.717228in}{1.467662in}}%
\pgfpathlineto{\pgfqpoint{4.702755in}{1.466883in}}%
\pgfpathlineto{\pgfqpoint{4.710794in}{1.479534in}}%
\pgfpathlineto{\pgfqpoint{4.718829in}{1.492313in}}%
\pgfpathlineto{\pgfqpoint{4.726862in}{1.505216in}}%
\pgfpathlineto{\pgfqpoint{4.734891in}{1.518238in}}%
\pgfpathclose%
\pgfusepath{fill}%
\end{pgfscope}%
\begin{pgfscope}%
\pgfpathrectangle{\pgfqpoint{1.150000in}{0.150000in}}{\pgfqpoint{5.700000in}{5.700000in}}%
\pgfusepath{clip}%
\pgfsetbuttcap%
\pgfsetroundjoin%
\definecolor{currentfill}{rgb}{0.282623,0.140926,0.457517}%
\pgfsetfillcolor{currentfill}%
\pgfsetfillopacity{0.700000}%
\pgfsetlinewidth{0.000000pt}%
\definecolor{currentstroke}{rgb}{0.000000,0.000000,0.000000}%
\pgfsetstrokecolor{currentstroke}%
\pgfsetdash{}{0pt}%
\pgfpathmoveto{\pgfqpoint{3.674239in}{1.516564in}}%
\pgfpathlineto{\pgfqpoint{3.688420in}{1.507968in}}%
\pgfpathlineto{\pgfqpoint{3.702605in}{1.499474in}}%
\pgfpathlineto{\pgfqpoint{3.716793in}{1.491081in}}%
\pgfpathlineto{\pgfqpoint{3.730985in}{1.482790in}}%
\pgfpathlineto{\pgfqpoint{3.722552in}{1.484448in}}%
\pgfpathlineto{\pgfqpoint{3.714106in}{1.486470in}}%
\pgfpathlineto{\pgfqpoint{3.705649in}{1.488862in}}%
\pgfpathlineto{\pgfqpoint{3.697178in}{1.491632in}}%
\pgfpathlineto{\pgfqpoint{3.682953in}{1.500569in}}%
\pgfpathlineto{\pgfqpoint{3.668731in}{1.509608in}}%
\pgfpathlineto{\pgfqpoint{3.654513in}{1.518749in}}%
\pgfpathlineto{\pgfqpoint{3.640298in}{1.527992in}}%
\pgfpathlineto{\pgfqpoint{3.648803in}{1.524567in}}%
\pgfpathlineto{\pgfqpoint{3.657295in}{1.521526in}}%
\pgfpathlineto{\pgfqpoint{3.665774in}{1.518860in}}%
\pgfpathlineto{\pgfqpoint{3.674239in}{1.516564in}}%
\pgfpathclose%
\pgfusepath{fill}%
\end{pgfscope}%
\begin{pgfscope}%
\pgfpathrectangle{\pgfqpoint{1.150000in}{0.150000in}}{\pgfqpoint{5.700000in}{5.700000in}}%
\pgfusepath{clip}%
\pgfsetbuttcap%
\pgfsetroundjoin%
\definecolor{currentfill}{rgb}{0.257322,0.256130,0.526563}%
\pgfsetfillcolor{currentfill}%
\pgfsetfillopacity{0.700000}%
\pgfsetlinewidth{0.000000pt}%
\definecolor{currentstroke}{rgb}{0.000000,0.000000,0.000000}%
\pgfsetstrokecolor{currentstroke}%
\pgfsetdash{}{0pt}%
\pgfpathmoveto{\pgfqpoint{5.037481in}{1.782419in}}%
\pgfpathlineto{\pgfqpoint{5.052112in}{1.786653in}}%
\pgfpathlineto{\pgfqpoint{5.066758in}{1.790982in}}%
\pgfpathlineto{\pgfqpoint{5.081416in}{1.795408in}}%
\pgfpathlineto{\pgfqpoint{5.096088in}{1.799931in}}%
\pgfpathlineto{\pgfqpoint{5.088094in}{1.783926in}}%
\pgfpathlineto{\pgfqpoint{5.080097in}{1.767953in}}%
\pgfpathlineto{\pgfqpoint{5.072097in}{1.752015in}}%
\pgfpathlineto{\pgfqpoint{5.064094in}{1.736118in}}%
\pgfpathlineto{\pgfqpoint{5.049428in}{1.732025in}}%
\pgfpathlineto{\pgfqpoint{5.034776in}{1.728028in}}%
\pgfpathlineto{\pgfqpoint{5.020137in}{1.724127in}}%
\pgfpathlineto{\pgfqpoint{5.005511in}{1.720323in}}%
\pgfpathlineto{\pgfqpoint{5.013508in}{1.735784in}}%
\pgfpathlineto{\pgfqpoint{5.021502in}{1.751290in}}%
\pgfpathlineto{\pgfqpoint{5.029493in}{1.766837in}}%
\pgfpathlineto{\pgfqpoint{5.037481in}{1.782419in}}%
\pgfpathclose%
\pgfusepath{fill}%
\end{pgfscope}%
\begin{pgfscope}%
\pgfpathrectangle{\pgfqpoint{1.150000in}{0.150000in}}{\pgfqpoint{5.700000in}{5.700000in}}%
\pgfusepath{clip}%
\pgfsetbuttcap%
\pgfsetroundjoin%
\definecolor{currentfill}{rgb}{0.278791,0.062145,0.386592}%
\pgfsetfillcolor{currentfill}%
\pgfsetfillopacity{0.700000}%
\pgfsetlinewidth{0.000000pt}%
\definecolor{currentstroke}{rgb}{0.000000,0.000000,0.000000}%
\pgfsetstrokecolor{currentstroke}%
\pgfsetdash{}{0pt}%
\pgfpathmoveto{\pgfqpoint{4.081766in}{1.361402in}}%
\pgfpathlineto{\pgfqpoint{4.096013in}{1.356408in}}%
\pgfpathlineto{\pgfqpoint{4.110267in}{1.351510in}}%
\pgfpathlineto{\pgfqpoint{4.124527in}{1.346709in}}%
\pgfpathlineto{\pgfqpoint{4.138793in}{1.342003in}}%
\pgfpathlineto{\pgfqpoint{4.130587in}{1.337424in}}%
\pgfpathlineto{\pgfqpoint{4.122374in}{1.333126in}}%
\pgfpathlineto{\pgfqpoint{4.114154in}{1.329115in}}%
\pgfpathlineto{\pgfqpoint{4.105925in}{1.325396in}}%
\pgfpathlineto{\pgfqpoint{4.091639in}{1.330704in}}%
\pgfpathlineto{\pgfqpoint{4.077360in}{1.336107in}}%
\pgfpathlineto{\pgfqpoint{4.063086in}{1.341606in}}%
\pgfpathlineto{\pgfqpoint{4.048818in}{1.347202in}}%
\pgfpathlineto{\pgfqpoint{4.057067in}{1.350311in}}%
\pgfpathlineto{\pgfqpoint{4.065308in}{1.353718in}}%
\pgfpathlineto{\pgfqpoint{4.073541in}{1.357417in}}%
\pgfpathlineto{\pgfqpoint{4.081766in}{1.361402in}}%
\pgfpathclose%
\pgfusepath{fill}%
\end{pgfscope}%
\begin{pgfscope}%
\pgfpathrectangle{\pgfqpoint{1.150000in}{0.150000in}}{\pgfqpoint{5.700000in}{5.700000in}}%
\pgfusepath{clip}%
\pgfsetbuttcap%
\pgfsetroundjoin%
\definecolor{currentfill}{rgb}{0.262138,0.242286,0.520837}%
\pgfsetfillcolor{currentfill}%
\pgfsetfillopacity{0.700000}%
\pgfsetlinewidth{0.000000pt}%
\definecolor{currentstroke}{rgb}{0.000000,0.000000,0.000000}%
\pgfsetstrokecolor{currentstroke}%
\pgfsetdash{}{0pt}%
\pgfpathmoveto{\pgfqpoint{3.356512in}{1.734832in}}%
\pgfpathlineto{\pgfqpoint{3.370684in}{1.723473in}}%
\pgfpathlineto{\pgfqpoint{3.384857in}{1.712223in}}%
\pgfpathlineto{\pgfqpoint{3.399031in}{1.701083in}}%
\pgfpathlineto{\pgfqpoint{3.413207in}{1.690051in}}%
\pgfpathlineto{\pgfqpoint{3.404529in}{1.696522in}}%
\pgfpathlineto{\pgfqpoint{3.395835in}{1.703413in}}%
\pgfpathlineto{\pgfqpoint{3.387123in}{1.710732in}}%
\pgfpathlineto{\pgfqpoint{3.378393in}{1.718487in}}%
\pgfpathlineto{\pgfqpoint{3.364173in}{1.730196in}}%
\pgfpathlineto{\pgfqpoint{3.349954in}{1.742013in}}%
\pgfpathlineto{\pgfqpoint{3.335736in}{1.753941in}}%
\pgfpathlineto{\pgfqpoint{3.321519in}{1.765978in}}%
\pgfpathlineto{\pgfqpoint{3.330295in}{1.757537in}}%
\pgfpathlineto{\pgfqpoint{3.339052in}{1.749538in}}%
\pgfpathlineto{\pgfqpoint{3.347791in}{1.741972in}}%
\pgfpathlineto{\pgfqpoint{3.356512in}{1.734832in}}%
\pgfpathclose%
\pgfusepath{fill}%
\end{pgfscope}%
\begin{pgfscope}%
\pgfpathrectangle{\pgfqpoint{1.150000in}{0.150000in}}{\pgfqpoint{5.700000in}{5.700000in}}%
\pgfusepath{clip}%
\pgfsetbuttcap%
\pgfsetroundjoin%
\definecolor{currentfill}{rgb}{0.277018,0.050344,0.375715}%
\pgfsetfillcolor{currentfill}%
\pgfsetfillopacity{0.700000}%
\pgfsetlinewidth{0.000000pt}%
\definecolor{currentstroke}{rgb}{0.000000,0.000000,0.000000}%
\pgfsetstrokecolor{currentstroke}%
\pgfsetdash{}{0pt}%
\pgfpathmoveto{\pgfqpoint{4.228613in}{1.347473in}}%
\pgfpathlineto{\pgfqpoint{4.242898in}{1.343828in}}%
\pgfpathlineto{\pgfqpoint{4.257191in}{1.340279in}}%
\pgfpathlineto{\pgfqpoint{4.271491in}{1.336825in}}%
\pgfpathlineto{\pgfqpoint{4.285799in}{1.333466in}}%
\pgfpathlineto{\pgfqpoint{4.277651in}{1.326673in}}%
\pgfpathlineto{\pgfqpoint{4.269498in}{1.320128in}}%
\pgfpathlineto{\pgfqpoint{4.261338in}{1.313835in}}%
\pgfpathlineto{\pgfqpoint{4.253172in}{1.307802in}}%
\pgfpathlineto{\pgfqpoint{4.238850in}{1.311744in}}%
\pgfpathlineto{\pgfqpoint{4.224535in}{1.315781in}}%
\pgfpathlineto{\pgfqpoint{4.210228in}{1.319913in}}%
\pgfpathlineto{\pgfqpoint{4.195927in}{1.324140in}}%
\pgfpathlineto{\pgfqpoint{4.204108in}{1.329583in}}%
\pgfpathlineto{\pgfqpoint{4.212283in}{1.335291in}}%
\pgfpathlineto{\pgfqpoint{4.220451in}{1.341256in}}%
\pgfpathlineto{\pgfqpoint{4.228613in}{1.347473in}}%
\pgfpathclose%
\pgfusepath{fill}%
\end{pgfscope}%
\begin{pgfscope}%
\pgfpathrectangle{\pgfqpoint{1.150000in}{0.150000in}}{\pgfqpoint{5.700000in}{5.700000in}}%
\pgfusepath{clip}%
\pgfsetbuttcap%
\pgfsetroundjoin%
\definecolor{currentfill}{rgb}{0.279566,0.067836,0.391917}%
\pgfsetfillcolor{currentfill}%
\pgfsetfillopacity{0.700000}%
\pgfsetlinewidth{0.000000pt}%
\definecolor{currentstroke}{rgb}{0.000000,0.000000,0.000000}%
\pgfsetstrokecolor{currentstroke}%
\pgfsetdash{}{0pt}%
\pgfpathmoveto{\pgfqpoint{4.465357in}{1.381442in}}%
\pgfpathlineto{\pgfqpoint{4.479720in}{1.380045in}}%
\pgfpathlineto{\pgfqpoint{4.494091in}{1.378744in}}%
\pgfpathlineto{\pgfqpoint{4.508472in}{1.377536in}}%
\pgfpathlineto{\pgfqpoint{4.522862in}{1.376423in}}%
\pgfpathlineto{\pgfqpoint{4.514783in}{1.366239in}}%
\pgfpathlineto{\pgfqpoint{4.506700in}{1.356243in}}%
\pgfpathlineto{\pgfqpoint{4.498613in}{1.346441in}}%
\pgfpathlineto{\pgfqpoint{4.490522in}{1.336839in}}%
\pgfpathlineto{\pgfqpoint{4.476125in}{1.338499in}}%
\pgfpathlineto{\pgfqpoint{4.461737in}{1.340254in}}%
\pgfpathlineto{\pgfqpoint{4.447358in}{1.342103in}}%
\pgfpathlineto{\pgfqpoint{4.432988in}{1.344046in}}%
\pgfpathlineto{\pgfqpoint{4.441087in}{1.353094in}}%
\pgfpathlineto{\pgfqpoint{4.449182in}{1.362347in}}%
\pgfpathlineto{\pgfqpoint{4.457272in}{1.371798in}}%
\pgfpathlineto{\pgfqpoint{4.465357in}{1.381442in}}%
\pgfpathclose%
\pgfusepath{fill}%
\end{pgfscope}%
\begin{pgfscope}%
\pgfpathrectangle{\pgfqpoint{1.150000in}{0.150000in}}{\pgfqpoint{5.700000in}{5.700000in}}%
\pgfusepath{clip}%
\pgfsetbuttcap%
\pgfsetroundjoin%
\definecolor{currentfill}{rgb}{0.280255,0.165693,0.476498}%
\pgfsetfillcolor{currentfill}%
\pgfsetfillopacity{0.700000}%
\pgfsetlinewidth{0.000000pt}%
\definecolor{currentstroke}{rgb}{0.000000,0.000000,0.000000}%
\pgfsetstrokecolor{currentstroke}%
\pgfsetdash{}{0pt}%
\pgfpathmoveto{\pgfqpoint{4.824936in}{1.579006in}}%
\pgfpathlineto{\pgfqpoint{4.839456in}{1.581139in}}%
\pgfpathlineto{\pgfqpoint{4.853987in}{1.583367in}}%
\pgfpathlineto{\pgfqpoint{4.868530in}{1.585690in}}%
\pgfpathlineto{\pgfqpoint{4.883085in}{1.588108in}}%
\pgfpathlineto{\pgfqpoint{4.875066in}{1.573710in}}%
\pgfpathlineto{\pgfqpoint{4.867044in}{1.559404in}}%
\pgfpathlineto{\pgfqpoint{4.859018in}{1.545195in}}%
\pgfpathlineto{\pgfqpoint{4.850990in}{1.531087in}}%
\pgfpathlineto{\pgfqpoint{4.836438in}{1.529149in}}%
\pgfpathlineto{\pgfqpoint{4.821897in}{1.527306in}}%
\pgfpathlineto{\pgfqpoint{4.807367in}{1.525558in}}%
\pgfpathlineto{\pgfqpoint{4.792849in}{1.523904in}}%
\pgfpathlineto{\pgfqpoint{4.800876in}{1.537526in}}%
\pgfpathlineto{\pgfqpoint{4.808899in}{1.551254in}}%
\pgfpathlineto{\pgfqpoint{4.816919in}{1.565082in}}%
\pgfpathlineto{\pgfqpoint{4.824936in}{1.579006in}}%
\pgfpathclose%
\pgfusepath{fill}%
\end{pgfscope}%
\begin{pgfscope}%
\pgfpathrectangle{\pgfqpoint{1.150000in}{0.150000in}}{\pgfqpoint{5.700000in}{5.700000in}}%
\pgfusepath{clip}%
\pgfsetbuttcap%
\pgfsetroundjoin%
\definecolor{currentfill}{rgb}{0.220124,0.725509,0.466226}%
\pgfsetfillcolor{currentfill}%
\pgfsetfillopacity{0.700000}%
\pgfsetlinewidth{0.000000pt}%
\definecolor{currentstroke}{rgb}{0.000000,0.000000,0.000000}%
\pgfsetstrokecolor{currentstroke}%
\pgfsetdash{}{0pt}%
\pgfpathmoveto{\pgfqpoint{2.217165in}{3.052986in}}%
\pgfpathlineto{\pgfqpoint{2.231612in}{3.030598in}}%
\pgfpathlineto{\pgfqpoint{2.246050in}{3.008391in}}%
\pgfpathlineto{\pgfqpoint{2.260478in}{2.986363in}}%
\pgfpathlineto{\pgfqpoint{2.274898in}{2.964513in}}%
\pgfpathlineto{\pgfqpoint{2.265016in}{2.984959in}}%
\pgfpathlineto{\pgfqpoint{2.255097in}{3.005958in}}%
\pgfpathlineto{\pgfqpoint{2.245142in}{3.027521in}}%
\pgfpathlineto{\pgfqpoint{2.235150in}{3.049655in}}%
\pgfpathlineto{\pgfqpoint{2.220649in}{3.072263in}}%
\pgfpathlineto{\pgfqpoint{2.206139in}{3.095049in}}%
\pgfpathlineto{\pgfqpoint{2.191619in}{3.118016in}}%
\pgfpathlineto{\pgfqpoint{2.177089in}{3.141166in}}%
\pgfpathlineto{\pgfqpoint{2.187164in}{3.118260in}}%
\pgfpathlineto{\pgfqpoint{2.197202in}{3.095935in}}%
\pgfpathlineto{\pgfqpoint{2.207202in}{3.074179in}}%
\pgfpathlineto{\pgfqpoint{2.217165in}{3.052986in}}%
\pgfpathclose%
\pgfusepath{fill}%
\end{pgfscope}%
\begin{pgfscope}%
\pgfpathrectangle{\pgfqpoint{1.150000in}{0.150000in}}{\pgfqpoint{5.700000in}{5.700000in}}%
\pgfusepath{clip}%
\pgfsetbuttcap%
\pgfsetroundjoin%
\definecolor{currentfill}{rgb}{0.214298,0.355619,0.551184}%
\pgfsetfillcolor{currentfill}%
\pgfsetfillopacity{0.700000}%
\pgfsetlinewidth{0.000000pt}%
\definecolor{currentstroke}{rgb}{0.000000,0.000000,0.000000}%
\pgfsetstrokecolor{currentstroke}%
\pgfsetdash{}{0pt}%
\pgfpathmoveto{\pgfqpoint{5.250667in}{2.016886in}}%
\pgfpathlineto{\pgfqpoint{5.265432in}{2.023078in}}%
\pgfpathlineto{\pgfqpoint{5.280213in}{2.029369in}}%
\pgfpathlineto{\pgfqpoint{5.295008in}{2.035758in}}%
\pgfpathlineto{\pgfqpoint{5.287043in}{2.019017in}}%
\pgfpathlineto{\pgfqpoint{5.279073in}{2.002255in}}%
\pgfpathlineto{\pgfqpoint{5.271100in}{1.985476in}}%
\pgfpathlineto{\pgfqpoint{5.263123in}{1.968684in}}%
\pgfpathlineto{\pgfqpoint{5.248336in}{1.962673in}}%
\pgfpathlineto{\pgfqpoint{5.233564in}{1.956760in}}%
\pgfpathlineto{\pgfqpoint{5.218806in}{1.950945in}}%
\pgfpathlineto{\pgfqpoint{5.226777in}{1.967448in}}%
\pgfpathlineto{\pgfqpoint{5.234744in}{1.983942in}}%
\pgfpathlineto{\pgfqpoint{5.242707in}{2.000422in}}%
\pgfpathlineto{\pgfqpoint{5.250667in}{2.016886in}}%
\pgfpathclose%
\pgfusepath{fill}%
\end{pgfscope}%
\begin{pgfscope}%
\pgfpathrectangle{\pgfqpoint{1.150000in}{0.150000in}}{\pgfqpoint{5.700000in}{5.700000in}}%
\pgfusepath{clip}%
\pgfsetbuttcap%
\pgfsetroundjoin%
\definecolor{currentfill}{rgb}{0.281446,0.084320,0.407414}%
\pgfsetfillcolor{currentfill}%
\pgfsetfillopacity{0.700000}%
\pgfsetlinewidth{0.000000pt}%
\definecolor{currentstroke}{rgb}{0.000000,0.000000,0.000000}%
\pgfsetstrokecolor{currentstroke}%
\pgfsetdash{}{0pt}%
\pgfpathmoveto{\pgfqpoint{3.934884in}{1.395461in}}%
\pgfpathlineto{\pgfqpoint{3.949106in}{1.389087in}}%
\pgfpathlineto{\pgfqpoint{3.963334in}{1.382811in}}%
\pgfpathlineto{\pgfqpoint{3.977568in}{1.376633in}}%
\pgfpathlineto{\pgfqpoint{3.991807in}{1.370553in}}%
\pgfpathlineto{\pgfqpoint{3.983527in}{1.368363in}}%
\pgfpathlineto{\pgfqpoint{3.975237in}{1.366489in}}%
\pgfpathlineto{\pgfqpoint{3.966939in}{1.364937in}}%
\pgfpathlineto{\pgfqpoint{3.958631in}{1.363714in}}%
\pgfpathlineto{\pgfqpoint{3.944368in}{1.370417in}}%
\pgfpathlineto{\pgfqpoint{3.930110in}{1.377217in}}%
\pgfpathlineto{\pgfqpoint{3.915857in}{1.384115in}}%
\pgfpathlineto{\pgfqpoint{3.901609in}{1.391111in}}%
\pgfpathlineto{\pgfqpoint{3.909942in}{1.391704in}}%
\pgfpathlineto{\pgfqpoint{3.918266in}{1.392631in}}%
\pgfpathlineto{\pgfqpoint{3.926580in}{1.393885in}}%
\pgfpathlineto{\pgfqpoint{3.934884in}{1.395461in}}%
\pgfpathclose%
\pgfusepath{fill}%
\end{pgfscope}%
\begin{pgfscope}%
\pgfpathrectangle{\pgfqpoint{1.150000in}{0.150000in}}{\pgfqpoint{5.700000in}{5.700000in}}%
\pgfusepath{clip}%
\pgfsetbuttcap%
\pgfsetroundjoin%
\definecolor{currentfill}{rgb}{0.266580,0.228262,0.514349}%
\pgfsetfillcolor{currentfill}%
\pgfsetfillopacity{0.700000}%
\pgfsetlinewidth{0.000000pt}%
\definecolor{currentstroke}{rgb}{0.000000,0.000000,0.000000}%
\pgfsetstrokecolor{currentstroke}%
\pgfsetdash{}{0pt}%
\pgfpathmoveto{\pgfqpoint{3.413207in}{1.690051in}}%
\pgfpathlineto{\pgfqpoint{3.427384in}{1.679127in}}%
\pgfpathlineto{\pgfqpoint{3.441563in}{1.668311in}}%
\pgfpathlineto{\pgfqpoint{3.455744in}{1.657602in}}%
\pgfpathlineto{\pgfqpoint{3.469927in}{1.647000in}}%
\pgfpathlineto{\pgfqpoint{3.461291in}{1.652804in}}%
\pgfpathlineto{\pgfqpoint{3.452639in}{1.659023in}}%
\pgfpathlineto{\pgfqpoint{3.443971in}{1.665664in}}%
\pgfpathlineto{\pgfqpoint{3.435286in}{1.672734in}}%
\pgfpathlineto{\pgfqpoint{3.421060in}{1.684011in}}%
\pgfpathlineto{\pgfqpoint{3.406837in}{1.695395in}}%
\pgfpathlineto{\pgfqpoint{3.392614in}{1.706887in}}%
\pgfpathlineto{\pgfqpoint{3.378393in}{1.718487in}}%
\pgfpathlineto{\pgfqpoint{3.387123in}{1.710732in}}%
\pgfpathlineto{\pgfqpoint{3.395835in}{1.703413in}}%
\pgfpathlineto{\pgfqpoint{3.404529in}{1.696522in}}%
\pgfpathlineto{\pgfqpoint{3.413207in}{1.690051in}}%
\pgfpathclose%
\pgfusepath{fill}%
\end{pgfscope}%
\begin{pgfscope}%
\pgfpathrectangle{\pgfqpoint{1.150000in}{0.150000in}}{\pgfqpoint{5.700000in}{5.700000in}}%
\pgfusepath{clip}%
\pgfsetbuttcap%
\pgfsetroundjoin%
\definecolor{currentfill}{rgb}{0.278791,0.062145,0.386592}%
\pgfsetfillcolor{currentfill}%
\pgfsetfillopacity{0.700000}%
\pgfsetlinewidth{0.000000pt}%
\definecolor{currentstroke}{rgb}{0.000000,0.000000,0.000000}%
\pgfsetstrokecolor{currentstroke}%
\pgfsetdash{}{0pt}%
\pgfpathmoveto{\pgfqpoint{4.375593in}{1.352762in}}%
\pgfpathlineto{\pgfqpoint{4.389929in}{1.350441in}}%
\pgfpathlineto{\pgfqpoint{4.404273in}{1.348215in}}%
\pgfpathlineto{\pgfqpoint{4.418626in}{1.346083in}}%
\pgfpathlineto{\pgfqpoint{4.432988in}{1.344046in}}%
\pgfpathlineto{\pgfqpoint{4.424884in}{1.335207in}}%
\pgfpathlineto{\pgfqpoint{4.416775in}{1.326583in}}%
\pgfpathlineto{\pgfqpoint{4.408662in}{1.318180in}}%
\pgfpathlineto{\pgfqpoint{4.400544in}{1.310003in}}%
\pgfpathlineto{\pgfqpoint{4.386172in}{1.312605in}}%
\pgfpathlineto{\pgfqpoint{4.371809in}{1.315302in}}%
\pgfpathlineto{\pgfqpoint{4.357454in}{1.318093in}}%
\pgfpathlineto{\pgfqpoint{4.343107in}{1.320978in}}%
\pgfpathlineto{\pgfqpoint{4.351236in}{1.328583in}}%
\pgfpathlineto{\pgfqpoint{4.359360in}{1.336420in}}%
\pgfpathlineto{\pgfqpoint{4.367479in}{1.344481in}}%
\pgfpathlineto{\pgfqpoint{4.375593in}{1.352762in}}%
\pgfpathclose%
\pgfusepath{fill}%
\end{pgfscope}%
\begin{pgfscope}%
\pgfpathrectangle{\pgfqpoint{1.150000in}{0.150000in}}{\pgfqpoint{5.700000in}{5.700000in}}%
\pgfusepath{clip}%
\pgfsetbuttcap%
\pgfsetroundjoin%
\definecolor{currentfill}{rgb}{0.274128,0.199721,0.498911}%
\pgfsetfillcolor{currentfill}%
\pgfsetfillopacity{0.700000}%
\pgfsetlinewidth{0.000000pt}%
\definecolor{currentstroke}{rgb}{0.000000,0.000000,0.000000}%
\pgfsetstrokecolor{currentstroke}%
\pgfsetdash{}{0pt}%
\pgfpathmoveto{\pgfqpoint{4.915134in}{1.646527in}}%
\pgfpathlineto{\pgfqpoint{4.929705in}{1.649503in}}%
\pgfpathlineto{\pgfqpoint{4.944288in}{1.652574in}}%
\pgfpathlineto{\pgfqpoint{4.958884in}{1.655741in}}%
\pgfpathlineto{\pgfqpoint{4.973492in}{1.659004in}}%
\pgfpathlineto{\pgfqpoint{4.965480in}{1.643827in}}%
\pgfpathlineto{\pgfqpoint{4.957465in}{1.628719in}}%
\pgfpathlineto{\pgfqpoint{4.949448in}{1.613686in}}%
\pgfpathlineto{\pgfqpoint{4.941428in}{1.598731in}}%
\pgfpathlineto{\pgfqpoint{4.926824in}{1.595932in}}%
\pgfpathlineto{\pgfqpoint{4.912232in}{1.593229in}}%
\pgfpathlineto{\pgfqpoint{4.897653in}{1.590621in}}%
\pgfpathlineto{\pgfqpoint{4.883085in}{1.588108in}}%
\pgfpathlineto{\pgfqpoint{4.891102in}{1.602592in}}%
\pgfpathlineto{\pgfqpoint{4.899116in}{1.617160in}}%
\pgfpathlineto{\pgfqpoint{4.907126in}{1.631806in}}%
\pgfpathlineto{\pgfqpoint{4.915134in}{1.646527in}}%
\pgfpathclose%
\pgfusepath{fill}%
\end{pgfscope}%
\begin{pgfscope}%
\pgfpathrectangle{\pgfqpoint{1.150000in}{0.150000in}}{\pgfqpoint{5.700000in}{5.700000in}}%
\pgfusepath{clip}%
\pgfsetbuttcap%
\pgfsetroundjoin%
\definecolor{currentfill}{rgb}{0.283072,0.130895,0.449241}%
\pgfsetfillcolor{currentfill}%
\pgfsetfillopacity{0.700000}%
\pgfsetlinewidth{0.000000pt}%
\definecolor{currentstroke}{rgb}{0.000000,0.000000,0.000000}%
\pgfsetstrokecolor{currentstroke}%
\pgfsetdash{}{0pt}%
\pgfpathmoveto{\pgfqpoint{3.730985in}{1.482790in}}%
\pgfpathlineto{\pgfqpoint{3.745180in}{1.474600in}}%
\pgfpathlineto{\pgfqpoint{3.759380in}{1.466511in}}%
\pgfpathlineto{\pgfqpoint{3.773583in}{1.458523in}}%
\pgfpathlineto{\pgfqpoint{3.787791in}{1.450634in}}%
\pgfpathlineto{\pgfqpoint{3.779389in}{1.451655in}}%
\pgfpathlineto{\pgfqpoint{3.770976in}{1.453034in}}%
\pgfpathlineto{\pgfqpoint{3.762550in}{1.454778in}}%
\pgfpathlineto{\pgfqpoint{3.754113in}{1.456894in}}%
\pgfpathlineto{\pgfqpoint{3.739874in}{1.465427in}}%
\pgfpathlineto{\pgfqpoint{3.725638in}{1.474061in}}%
\pgfpathlineto{\pgfqpoint{3.711406in}{1.482796in}}%
\pgfpathlineto{\pgfqpoint{3.697178in}{1.491632in}}%
\pgfpathlineto{\pgfqpoint{3.705649in}{1.488862in}}%
\pgfpathlineto{\pgfqpoint{3.714106in}{1.486470in}}%
\pgfpathlineto{\pgfqpoint{3.722552in}{1.484448in}}%
\pgfpathlineto{\pgfqpoint{3.730985in}{1.482790in}}%
\pgfpathclose%
\pgfusepath{fill}%
\end{pgfscope}%
\begin{pgfscope}%
\pgfpathrectangle{\pgfqpoint{1.150000in}{0.150000in}}{\pgfqpoint{5.700000in}{5.700000in}}%
\pgfusepath{clip}%
\pgfsetbuttcap%
\pgfsetroundjoin%
\definecolor{currentfill}{rgb}{0.281477,0.755203,0.432552}%
\pgfsetfillcolor{currentfill}%
\pgfsetfillopacity{0.700000}%
\pgfsetlinewidth{0.000000pt}%
\definecolor{currentstroke}{rgb}{0.000000,0.000000,0.000000}%
\pgfsetstrokecolor{currentstroke}%
\pgfsetdash{}{0pt}%
\pgfpathmoveto{\pgfqpoint{2.159280in}{3.144375in}}%
\pgfpathlineto{\pgfqpoint{2.173766in}{3.121248in}}%
\pgfpathlineto{\pgfqpoint{2.188242in}{3.098309in}}%
\pgfpathlineto{\pgfqpoint{2.202708in}{3.075556in}}%
\pgfpathlineto{\pgfqpoint{2.217165in}{3.052986in}}%
\pgfpathlineto{\pgfqpoint{2.207202in}{3.074179in}}%
\pgfpathlineto{\pgfqpoint{2.197202in}{3.095935in}}%
\pgfpathlineto{\pgfqpoint{2.187164in}{3.118260in}}%
\pgfpathlineto{\pgfqpoint{2.177089in}{3.141166in}}%
\pgfpathlineto{\pgfqpoint{2.162549in}{3.164499in}}%
\pgfpathlineto{\pgfqpoint{2.147998in}{3.188019in}}%
\pgfpathlineto{\pgfqpoint{2.133438in}{3.211725in}}%
\pgfpathlineto{\pgfqpoint{2.118867in}{3.235620in}}%
\pgfpathlineto{\pgfqpoint{2.129028in}{3.211936in}}%
\pgfpathlineto{\pgfqpoint{2.139151in}{3.188840in}}%
\pgfpathlineto{\pgfqpoint{2.149234in}{3.166322in}}%
\pgfpathlineto{\pgfqpoint{2.159280in}{3.144375in}}%
\pgfpathclose%
\pgfusepath{fill}%
\end{pgfscope}%
\begin{pgfscope}%
\pgfpathrectangle{\pgfqpoint{1.150000in}{0.150000in}}{\pgfqpoint{5.700000in}{5.700000in}}%
\pgfusepath{clip}%
\pgfsetbuttcap%
\pgfsetroundjoin%
\definecolor{currentfill}{rgb}{0.241237,0.296485,0.539709}%
\pgfsetfillcolor{currentfill}%
\pgfsetfillopacity{0.700000}%
\pgfsetlinewidth{0.000000pt}%
\definecolor{currentstroke}{rgb}{0.000000,0.000000,0.000000}%
\pgfsetstrokecolor{currentstroke}%
\pgfsetdash{}{0pt}%
\pgfpathmoveto{\pgfqpoint{5.128032in}{1.864193in}}%
\pgfpathlineto{\pgfqpoint{5.142725in}{1.869224in}}%
\pgfpathlineto{\pgfqpoint{5.157431in}{1.874353in}}%
\pgfpathlineto{\pgfqpoint{5.172152in}{1.879578in}}%
\pgfpathlineto{\pgfqpoint{5.186887in}{1.884901in}}%
\pgfpathlineto{\pgfqpoint{5.178899in}{1.868401in}}%
\pgfpathlineto{\pgfqpoint{5.170907in}{1.851913in}}%
\pgfpathlineto{\pgfqpoint{5.162912in}{1.835441in}}%
\pgfpathlineto{\pgfqpoint{5.154914in}{1.818988in}}%
\pgfpathlineto{\pgfqpoint{5.140187in}{1.814079in}}%
\pgfpathlineto{\pgfqpoint{5.125474in}{1.809266in}}%
\pgfpathlineto{\pgfqpoint{5.110774in}{1.804550in}}%
\pgfpathlineto{\pgfqpoint{5.096088in}{1.799931in}}%
\pgfpathlineto{\pgfqpoint{5.104079in}{1.815964in}}%
\pgfpathlineto{\pgfqpoint{5.112067in}{1.832021in}}%
\pgfpathlineto{\pgfqpoint{5.120051in}{1.848099in}}%
\pgfpathlineto{\pgfqpoint{5.128032in}{1.864193in}}%
\pgfpathclose%
\pgfusepath{fill}%
\end{pgfscope}%
\begin{pgfscope}%
\pgfpathrectangle{\pgfqpoint{1.150000in}{0.150000in}}{\pgfqpoint{5.700000in}{5.700000in}}%
\pgfusepath{clip}%
\pgfsetbuttcap%
\pgfsetroundjoin%
\definecolor{currentfill}{rgb}{0.271828,0.209303,0.504434}%
\pgfsetfillcolor{currentfill}%
\pgfsetfillopacity{0.700000}%
\pgfsetlinewidth{0.000000pt}%
\definecolor{currentstroke}{rgb}{0.000000,0.000000,0.000000}%
\pgfsetstrokecolor{currentstroke}%
\pgfsetdash{}{0pt}%
\pgfpathmoveto{\pgfqpoint{3.469927in}{1.647000in}}%
\pgfpathlineto{\pgfqpoint{3.484111in}{1.636505in}}%
\pgfpathlineto{\pgfqpoint{3.498298in}{1.626116in}}%
\pgfpathlineto{\pgfqpoint{3.512487in}{1.615833in}}%
\pgfpathlineto{\pgfqpoint{3.526678in}{1.605655in}}%
\pgfpathlineto{\pgfqpoint{3.518083in}{1.610795in}}%
\pgfpathlineto{\pgfqpoint{3.509472in}{1.616343in}}%
\pgfpathlineto{\pgfqpoint{3.500845in}{1.622307in}}%
\pgfpathlineto{\pgfqpoint{3.492203in}{1.628695in}}%
\pgfpathlineto{\pgfqpoint{3.477971in}{1.639546in}}%
\pgfpathlineto{\pgfqpoint{3.463741in}{1.650502in}}%
\pgfpathlineto{\pgfqpoint{3.449512in}{1.661565in}}%
\pgfpathlineto{\pgfqpoint{3.435286in}{1.672734in}}%
\pgfpathlineto{\pgfqpoint{3.443971in}{1.665664in}}%
\pgfpathlineto{\pgfqpoint{3.452639in}{1.659023in}}%
\pgfpathlineto{\pgfqpoint{3.461291in}{1.652804in}}%
\pgfpathlineto{\pgfqpoint{3.469927in}{1.647000in}}%
\pgfpathclose%
\pgfusepath{fill}%
\end{pgfscope}%
\begin{pgfscope}%
\pgfpathrectangle{\pgfqpoint{1.150000in}{0.150000in}}{\pgfqpoint{5.700000in}{5.700000in}}%
\pgfusepath{clip}%
\pgfsetbuttcap%
\pgfsetroundjoin%
\definecolor{currentfill}{rgb}{0.278791,0.062145,0.386592}%
\pgfsetfillcolor{currentfill}%
\pgfsetfillopacity{0.700000}%
\pgfsetlinewidth{0.000000pt}%
\definecolor{currentstroke}{rgb}{0.000000,0.000000,0.000000}%
\pgfsetstrokecolor{currentstroke}%
\pgfsetdash{}{0pt}%
\pgfpathmoveto{\pgfqpoint{4.138793in}{1.342003in}}%
\pgfpathlineto{\pgfqpoint{4.153066in}{1.337394in}}%
\pgfpathlineto{\pgfqpoint{4.167346in}{1.332880in}}%
\pgfpathlineto{\pgfqpoint{4.181633in}{1.328462in}}%
\pgfpathlineto{\pgfqpoint{4.195927in}{1.324140in}}%
\pgfpathlineto{\pgfqpoint{4.187739in}{1.318966in}}%
\pgfpathlineto{\pgfqpoint{4.179544in}{1.314069in}}%
\pgfpathlineto{\pgfqpoint{4.171342in}{1.309453in}}%
\pgfpathlineto{\pgfqpoint{4.163133in}{1.305126in}}%
\pgfpathlineto{\pgfqpoint{4.148821in}{1.310050in}}%
\pgfpathlineto{\pgfqpoint{4.134516in}{1.315070in}}%
\pgfpathlineto{\pgfqpoint{4.120218in}{1.320185in}}%
\pgfpathlineto{\pgfqpoint{4.105925in}{1.325396in}}%
\pgfpathlineto{\pgfqpoint{4.114154in}{1.329115in}}%
\pgfpathlineto{\pgfqpoint{4.122374in}{1.333126in}}%
\pgfpathlineto{\pgfqpoint{4.130587in}{1.337424in}}%
\pgfpathlineto{\pgfqpoint{4.138793in}{1.342003in}}%
\pgfpathclose%
\pgfusepath{fill}%
\end{pgfscope}%
\begin{pgfscope}%
\pgfpathrectangle{\pgfqpoint{1.150000in}{0.150000in}}{\pgfqpoint{5.700000in}{5.700000in}}%
\pgfusepath{clip}%
\pgfsetbuttcap%
\pgfsetroundjoin%
\definecolor{currentfill}{rgb}{0.263663,0.237631,0.518762}%
\pgfsetfillcolor{currentfill}%
\pgfsetfillopacity{0.700000}%
\pgfsetlinewidth{0.000000pt}%
\definecolor{currentstroke}{rgb}{0.000000,0.000000,0.000000}%
\pgfsetstrokecolor{currentstroke}%
\pgfsetdash{}{0pt}%
\pgfpathmoveto{\pgfqpoint{5.005511in}{1.720323in}}%
\pgfpathlineto{\pgfqpoint{5.020137in}{1.724127in}}%
\pgfpathlineto{\pgfqpoint{5.034776in}{1.728028in}}%
\pgfpathlineto{\pgfqpoint{5.049428in}{1.732025in}}%
\pgfpathlineto{\pgfqpoint{5.064094in}{1.736118in}}%
\pgfpathlineto{\pgfqpoint{5.056088in}{1.720265in}}%
\pgfpathlineto{\pgfqpoint{5.048080in}{1.704459in}}%
\pgfpathlineto{\pgfqpoint{5.040068in}{1.688707in}}%
\pgfpathlineto{\pgfqpoint{5.032054in}{1.673011in}}%
\pgfpathlineto{\pgfqpoint{5.017394in}{1.669365in}}%
\pgfpathlineto{\pgfqpoint{5.002747in}{1.665816in}}%
\pgfpathlineto{\pgfqpoint{4.988113in}{1.662362in}}%
\pgfpathlineto{\pgfqpoint{4.973492in}{1.659004in}}%
\pgfpathlineto{\pgfqpoint{4.981501in}{1.674246in}}%
\pgfpathlineto{\pgfqpoint{4.989507in}{1.689550in}}%
\pgfpathlineto{\pgfqpoint{4.997510in}{1.704910in}}%
\pgfpathlineto{\pgfqpoint{5.005511in}{1.720323in}}%
\pgfpathclose%
\pgfusepath{fill}%
\end{pgfscope}%
\begin{pgfscope}%
\pgfpathrectangle{\pgfqpoint{1.150000in}{0.150000in}}{\pgfqpoint{5.700000in}{5.700000in}}%
\pgfusepath{clip}%
\pgfsetbuttcap%
\pgfsetroundjoin%
\definecolor{currentfill}{rgb}{0.277941,0.056324,0.381191}%
\pgfsetfillcolor{currentfill}%
\pgfsetfillopacity{0.700000}%
\pgfsetlinewidth{0.000000pt}%
\definecolor{currentstroke}{rgb}{0.000000,0.000000,0.000000}%
\pgfsetstrokecolor{currentstroke}%
\pgfsetdash{}{0pt}%
\pgfpathmoveto{\pgfqpoint{4.285799in}{1.333466in}}%
\pgfpathlineto{\pgfqpoint{4.300114in}{1.330202in}}%
\pgfpathlineto{\pgfqpoint{4.314437in}{1.327032in}}%
\pgfpathlineto{\pgfqpoint{4.328768in}{1.323958in}}%
\pgfpathlineto{\pgfqpoint{4.343107in}{1.320978in}}%
\pgfpathlineto{\pgfqpoint{4.334972in}{1.313609in}}%
\pgfpathlineto{\pgfqpoint{4.326832in}{1.306483in}}%
\pgfpathlineto{\pgfqpoint{4.318686in}{1.299606in}}%
\pgfpathlineto{\pgfqpoint{4.310535in}{1.292982in}}%
\pgfpathlineto{\pgfqpoint{4.296183in}{1.296545in}}%
\pgfpathlineto{\pgfqpoint{4.281839in}{1.300203in}}%
\pgfpathlineto{\pgfqpoint{4.267502in}{1.303955in}}%
\pgfpathlineto{\pgfqpoint{4.253172in}{1.307802in}}%
\pgfpathlineto{\pgfqpoint{4.261338in}{1.313835in}}%
\pgfpathlineto{\pgfqpoint{4.269498in}{1.320128in}}%
\pgfpathlineto{\pgfqpoint{4.277651in}{1.326673in}}%
\pgfpathlineto{\pgfqpoint{4.285799in}{1.333466in}}%
\pgfpathclose%
\pgfusepath{fill}%
\end{pgfscope}%
\begin{pgfscope}%
\pgfpathrectangle{\pgfqpoint{1.150000in}{0.150000in}}{\pgfqpoint{5.700000in}{5.700000in}}%
\pgfusepath{clip}%
\pgfsetbuttcap%
\pgfsetroundjoin%
\definecolor{currentfill}{rgb}{0.282656,0.100196,0.422160}%
\pgfsetfillcolor{currentfill}%
\pgfsetfillopacity{0.700000}%
\pgfsetlinewidth{0.000000pt}%
\definecolor{currentstroke}{rgb}{0.000000,0.000000,0.000000}%
\pgfsetstrokecolor{currentstroke}%
\pgfsetdash{}{0pt}%
\pgfpathmoveto{\pgfqpoint{4.612771in}{1.417549in}}%
\pgfpathlineto{\pgfqpoint{4.627205in}{1.417437in}}%
\pgfpathlineto{\pgfqpoint{4.641649in}{1.417419in}}%
\pgfpathlineto{\pgfqpoint{4.656103in}{1.417496in}}%
\pgfpathlineto{\pgfqpoint{4.670567in}{1.417667in}}%
\pgfpathlineto{\pgfqpoint{4.662512in}{1.405734in}}%
\pgfpathlineto{\pgfqpoint{4.654454in}{1.393960in}}%
\pgfpathlineto{\pgfqpoint{4.646392in}{1.382349in}}%
\pgfpathlineto{\pgfqpoint{4.638327in}{1.370907in}}%
\pgfpathlineto{\pgfqpoint{4.623859in}{1.371268in}}%
\pgfpathlineto{\pgfqpoint{4.609402in}{1.371722in}}%
\pgfpathlineto{\pgfqpoint{4.594954in}{1.372270in}}%
\pgfpathlineto{\pgfqpoint{4.580516in}{1.372913in}}%
\pgfpathlineto{\pgfqpoint{4.588585in}{1.383817in}}%
\pgfpathlineto{\pgfqpoint{4.596651in}{1.394894in}}%
\pgfpathlineto{\pgfqpoint{4.604713in}{1.406140in}}%
\pgfpathlineto{\pgfqpoint{4.612771in}{1.417549in}}%
\pgfpathclose%
\pgfusepath{fill}%
\end{pgfscope}%
\begin{pgfscope}%
\pgfpathrectangle{\pgfqpoint{1.150000in}{0.150000in}}{\pgfqpoint{5.700000in}{5.700000in}}%
\pgfusepath{clip}%
\pgfsetbuttcap%
\pgfsetroundjoin%
\definecolor{currentfill}{rgb}{0.283229,0.120777,0.440584}%
\pgfsetfillcolor{currentfill}%
\pgfsetfillopacity{0.700000}%
\pgfsetlinewidth{0.000000pt}%
\definecolor{currentstroke}{rgb}{0.000000,0.000000,0.000000}%
\pgfsetstrokecolor{currentstroke}%
\pgfsetdash{}{0pt}%
\pgfpathmoveto{\pgfqpoint{4.702755in}{1.466883in}}%
\pgfpathlineto{\pgfqpoint{4.717228in}{1.467662in}}%
\pgfpathlineto{\pgfqpoint{4.731713in}{1.468535in}}%
\pgfpathlineto{\pgfqpoint{4.746208in}{1.469502in}}%
\pgfpathlineto{\pgfqpoint{4.760714in}{1.470564in}}%
\pgfpathlineto{\pgfqpoint{4.752673in}{1.457540in}}%
\pgfpathlineto{\pgfqpoint{4.744628in}{1.444650in}}%
\pgfpathlineto{\pgfqpoint{4.736581in}{1.431899in}}%
\pgfpathlineto{\pgfqpoint{4.728531in}{1.419292in}}%
\pgfpathlineto{\pgfqpoint{4.714024in}{1.418744in}}%
\pgfpathlineto{\pgfqpoint{4.699528in}{1.418291in}}%
\pgfpathlineto{\pgfqpoint{4.685042in}{1.417932in}}%
\pgfpathlineto{\pgfqpoint{4.670567in}{1.417667in}}%
\pgfpathlineto{\pgfqpoint{4.678619in}{1.429753in}}%
\pgfpathlineto{\pgfqpoint{4.686668in}{1.441988in}}%
\pgfpathlineto{\pgfqpoint{4.694713in}{1.454366in}}%
\pgfpathlineto{\pgfqpoint{4.702755in}{1.466883in}}%
\pgfpathclose%
\pgfusepath{fill}%
\end{pgfscope}%
\begin{pgfscope}%
\pgfpathrectangle{\pgfqpoint{1.150000in}{0.150000in}}{\pgfqpoint{5.700000in}{5.700000in}}%
\pgfusepath{clip}%
\pgfsetbuttcap%
\pgfsetroundjoin%
\definecolor{currentfill}{rgb}{0.283229,0.120777,0.440584}%
\pgfsetfillcolor{currentfill}%
\pgfsetfillopacity{0.700000}%
\pgfsetlinewidth{0.000000pt}%
\definecolor{currentstroke}{rgb}{0.000000,0.000000,0.000000}%
\pgfsetstrokecolor{currentstroke}%
\pgfsetdash{}{0pt}%
\pgfpathmoveto{\pgfqpoint{3.787791in}{1.450634in}}%
\pgfpathlineto{\pgfqpoint{3.802002in}{1.442846in}}%
\pgfpathlineto{\pgfqpoint{3.816218in}{1.435158in}}%
\pgfpathlineto{\pgfqpoint{3.830439in}{1.427569in}}%
\pgfpathlineto{\pgfqpoint{3.844664in}{1.420080in}}%
\pgfpathlineto{\pgfqpoint{3.836292in}{1.420464in}}%
\pgfpathlineto{\pgfqpoint{3.827908in}{1.421201in}}%
\pgfpathlineto{\pgfqpoint{3.819514in}{1.422298in}}%
\pgfpathlineto{\pgfqpoint{3.811108in}{1.423761in}}%
\pgfpathlineto{\pgfqpoint{3.796853in}{1.431895in}}%
\pgfpathlineto{\pgfqpoint{3.782603in}{1.440128in}}%
\pgfpathlineto{\pgfqpoint{3.768356in}{1.448461in}}%
\pgfpathlineto{\pgfqpoint{3.754113in}{1.456894in}}%
\pgfpathlineto{\pgfqpoint{3.762550in}{1.454778in}}%
\pgfpathlineto{\pgfqpoint{3.770976in}{1.453034in}}%
\pgfpathlineto{\pgfqpoint{3.779389in}{1.451655in}}%
\pgfpathlineto{\pgfqpoint{3.787791in}{1.450634in}}%
\pgfpathclose%
\pgfusepath{fill}%
\end{pgfscope}%
\begin{pgfscope}%
\pgfpathrectangle{\pgfqpoint{1.150000in}{0.150000in}}{\pgfqpoint{5.700000in}{5.700000in}}%
\pgfusepath{clip}%
\pgfsetbuttcap%
\pgfsetroundjoin%
\definecolor{currentfill}{rgb}{0.280894,0.078907,0.402329}%
\pgfsetfillcolor{currentfill}%
\pgfsetfillopacity{0.700000}%
\pgfsetlinewidth{0.000000pt}%
\definecolor{currentstroke}{rgb}{0.000000,0.000000,0.000000}%
\pgfsetstrokecolor{currentstroke}%
\pgfsetdash{}{0pt}%
\pgfpathmoveto{\pgfqpoint{3.991807in}{1.370553in}}%
\pgfpathlineto{\pgfqpoint{4.006051in}{1.364570in}}%
\pgfpathlineto{\pgfqpoint{4.020301in}{1.358684in}}%
\pgfpathlineto{\pgfqpoint{4.034557in}{1.352894in}}%
\pgfpathlineto{\pgfqpoint{4.048818in}{1.347202in}}%
\pgfpathlineto{\pgfqpoint{4.040561in}{1.344398in}}%
\pgfpathlineto{\pgfqpoint{4.032295in}{1.341904in}}%
\pgfpathlineto{\pgfqpoint{4.024021in}{1.339728in}}%
\pgfpathlineto{\pgfqpoint{4.015738in}{1.337876in}}%
\pgfpathlineto{\pgfqpoint{4.001453in}{1.344190in}}%
\pgfpathlineto{\pgfqpoint{3.987174in}{1.350601in}}%
\pgfpathlineto{\pgfqpoint{3.972900in}{1.357109in}}%
\pgfpathlineto{\pgfqpoint{3.958631in}{1.363714in}}%
\pgfpathlineto{\pgfqpoint{3.966939in}{1.364937in}}%
\pgfpathlineto{\pgfqpoint{3.975237in}{1.366489in}}%
\pgfpathlineto{\pgfqpoint{3.983527in}{1.368363in}}%
\pgfpathlineto{\pgfqpoint{3.991807in}{1.370553in}}%
\pgfpathclose%
\pgfusepath{fill}%
\end{pgfscope}%
\begin{pgfscope}%
\pgfpathrectangle{\pgfqpoint{1.150000in}{0.150000in}}{\pgfqpoint{5.700000in}{5.700000in}}%
\pgfusepath{clip}%
\pgfsetbuttcap%
\pgfsetroundjoin%
\definecolor{currentfill}{rgb}{0.360741,0.785964,0.387814}%
\pgfsetfillcolor{currentfill}%
\pgfsetfillopacity{0.700000}%
\pgfsetlinewidth{0.000000pt}%
\definecolor{currentstroke}{rgb}{0.000000,0.000000,0.000000}%
\pgfsetstrokecolor{currentstroke}%
\pgfsetdash{}{0pt}%
\pgfpathmoveto{\pgfqpoint{2.101231in}{3.238786in}}%
\pgfpathlineto{\pgfqpoint{2.115759in}{3.214894in}}%
\pgfpathlineto{\pgfqpoint{2.130277in}{3.191196in}}%
\pgfpathlineto{\pgfqpoint{2.144784in}{3.167690in}}%
\pgfpathlineto{\pgfqpoint{2.159280in}{3.144375in}}%
\pgfpathlineto{\pgfqpoint{2.149234in}{3.166322in}}%
\pgfpathlineto{\pgfqpoint{2.139151in}{3.188840in}}%
\pgfpathlineto{\pgfqpoint{2.129028in}{3.211936in}}%
\pgfpathlineto{\pgfqpoint{2.118867in}{3.235620in}}%
\pgfpathlineto{\pgfqpoint{2.104285in}{3.259706in}}%
\pgfpathlineto{\pgfqpoint{2.089692in}{3.283985in}}%
\pgfpathlineto{\pgfqpoint{2.075088in}{3.308458in}}%
\pgfpathlineto{\pgfqpoint{2.060472in}{3.333127in}}%
\pgfpathlineto{\pgfqpoint{2.070722in}{3.308657in}}%
\pgfpathlineto{\pgfqpoint{2.080932in}{3.284783in}}%
\pgfpathlineto{\pgfqpoint{2.091101in}{3.261496in}}%
\pgfpathlineto{\pgfqpoint{2.101231in}{3.238786in}}%
\pgfpathclose%
\pgfusepath{fill}%
\end{pgfscope}%
\begin{pgfscope}%
\pgfpathrectangle{\pgfqpoint{1.150000in}{0.150000in}}{\pgfqpoint{5.700000in}{5.700000in}}%
\pgfusepath{clip}%
\pgfsetbuttcap%
\pgfsetroundjoin%
\definecolor{currentfill}{rgb}{0.223925,0.334994,0.548053}%
\pgfsetfillcolor{currentfill}%
\pgfsetfillopacity{0.700000}%
\pgfsetlinewidth{0.000000pt}%
\definecolor{currentstroke}{rgb}{0.000000,0.000000,0.000000}%
\pgfsetstrokecolor{currentstroke}%
\pgfsetdash{}{0pt}%
\pgfpathmoveto{\pgfqpoint{5.218806in}{1.950945in}}%
\pgfpathlineto{\pgfqpoint{5.233564in}{1.956760in}}%
\pgfpathlineto{\pgfqpoint{5.248336in}{1.962673in}}%
\pgfpathlineto{\pgfqpoint{5.263123in}{1.968684in}}%
\pgfpathlineto{\pgfqpoint{5.255142in}{1.951882in}}%
\pgfpathlineto{\pgfqpoint{5.247158in}{1.935074in}}%
\pgfpathlineto{\pgfqpoint{5.239170in}{1.918263in}}%
\pgfpathlineto{\pgfqpoint{5.231179in}{1.901452in}}%
\pgfpathlineto{\pgfqpoint{5.216400in}{1.895838in}}%
\pgfpathlineto{\pgfqpoint{5.201637in}{1.890321in}}%
\pgfpathlineto{\pgfqpoint{5.186887in}{1.884901in}}%
\pgfpathlineto{\pgfqpoint{5.194872in}{1.901409in}}%
\pgfpathlineto{\pgfqpoint{5.202854in}{1.917921in}}%
\pgfpathlineto{\pgfqpoint{5.210832in}{1.934434in}}%
\pgfpathlineto{\pgfqpoint{5.218806in}{1.950945in}}%
\pgfpathclose%
\pgfusepath{fill}%
\end{pgfscope}%
\begin{pgfscope}%
\pgfpathrectangle{\pgfqpoint{1.150000in}{0.150000in}}{\pgfqpoint{5.700000in}{5.700000in}}%
\pgfusepath{clip}%
\pgfsetbuttcap%
\pgfsetroundjoin%
\definecolor{currentfill}{rgb}{0.280894,0.078907,0.402329}%
\pgfsetfillcolor{currentfill}%
\pgfsetfillopacity{0.700000}%
\pgfsetlinewidth{0.000000pt}%
\definecolor{currentstroke}{rgb}{0.000000,0.000000,0.000000}%
\pgfsetstrokecolor{currentstroke}%
\pgfsetdash{}{0pt}%
\pgfpathmoveto{\pgfqpoint{4.522862in}{1.376423in}}%
\pgfpathlineto{\pgfqpoint{4.537261in}{1.375404in}}%
\pgfpathlineto{\pgfqpoint{4.551670in}{1.374479in}}%
\pgfpathlineto{\pgfqpoint{4.566088in}{1.373649in}}%
\pgfpathlineto{\pgfqpoint{4.580516in}{1.372913in}}%
\pgfpathlineto{\pgfqpoint{4.572443in}{1.362187in}}%
\pgfpathlineto{\pgfqpoint{4.564367in}{1.351646in}}%
\pgfpathlineto{\pgfqpoint{4.556286in}{1.341294in}}%
\pgfpathlineto{\pgfqpoint{4.548202in}{1.331137in}}%
\pgfpathlineto{\pgfqpoint{4.533768in}{1.332421in}}%
\pgfpathlineto{\pgfqpoint{4.519344in}{1.333800in}}%
\pgfpathlineto{\pgfqpoint{4.504928in}{1.335273in}}%
\pgfpathlineto{\pgfqpoint{4.490522in}{1.336839in}}%
\pgfpathlineto{\pgfqpoint{4.498613in}{1.346441in}}%
\pgfpathlineto{\pgfqpoint{4.506700in}{1.356243in}}%
\pgfpathlineto{\pgfqpoint{4.514783in}{1.366239in}}%
\pgfpathlineto{\pgfqpoint{4.522862in}{1.376423in}}%
\pgfpathclose%
\pgfusepath{fill}%
\end{pgfscope}%
\begin{pgfscope}%
\pgfpathrectangle{\pgfqpoint{1.150000in}{0.150000in}}{\pgfqpoint{5.700000in}{5.700000in}}%
\pgfusepath{clip}%
\pgfsetbuttcap%
\pgfsetroundjoin%
\definecolor{currentfill}{rgb}{0.281887,0.150881,0.465405}%
\pgfsetfillcolor{currentfill}%
\pgfsetfillopacity{0.700000}%
\pgfsetlinewidth{0.000000pt}%
\definecolor{currentstroke}{rgb}{0.000000,0.000000,0.000000}%
\pgfsetstrokecolor{currentstroke}%
\pgfsetdash{}{0pt}%
\pgfpathmoveto{\pgfqpoint{4.792849in}{1.523904in}}%
\pgfpathlineto{\pgfqpoint{4.807367in}{1.525558in}}%
\pgfpathlineto{\pgfqpoint{4.821897in}{1.527306in}}%
\pgfpathlineto{\pgfqpoint{4.836438in}{1.529149in}}%
\pgfpathlineto{\pgfqpoint{4.850990in}{1.531087in}}%
\pgfpathlineto{\pgfqpoint{4.842960in}{1.517084in}}%
\pgfpathlineto{\pgfqpoint{4.834926in}{1.503191in}}%
\pgfpathlineto{\pgfqpoint{4.826890in}{1.489414in}}%
\pgfpathlineto{\pgfqpoint{4.818851in}{1.475756in}}%
\pgfpathlineto{\pgfqpoint{4.804300in}{1.474317in}}%
\pgfpathlineto{\pgfqpoint{4.789760in}{1.472972in}}%
\pgfpathlineto{\pgfqpoint{4.775231in}{1.471721in}}%
\pgfpathlineto{\pgfqpoint{4.760714in}{1.470564in}}%
\pgfpathlineto{\pgfqpoint{4.768752in}{1.483718in}}%
\pgfpathlineto{\pgfqpoint{4.776788in}{1.496995in}}%
\pgfpathlineto{\pgfqpoint{4.784820in}{1.510392in}}%
\pgfpathlineto{\pgfqpoint{4.792849in}{1.523904in}}%
\pgfpathclose%
\pgfusepath{fill}%
\end{pgfscope}%
\begin{pgfscope}%
\pgfpathrectangle{\pgfqpoint{1.150000in}{0.150000in}}{\pgfqpoint{5.700000in}{5.700000in}}%
\pgfusepath{clip}%
\pgfsetbuttcap%
\pgfsetroundjoin%
\definecolor{currentfill}{rgb}{0.275191,0.194905,0.496005}%
\pgfsetfillcolor{currentfill}%
\pgfsetfillopacity{0.700000}%
\pgfsetlinewidth{0.000000pt}%
\definecolor{currentstroke}{rgb}{0.000000,0.000000,0.000000}%
\pgfsetstrokecolor{currentstroke}%
\pgfsetdash{}{0pt}%
\pgfpathmoveto{\pgfqpoint{3.526678in}{1.605655in}}%
\pgfpathlineto{\pgfqpoint{3.540871in}{1.595583in}}%
\pgfpathlineto{\pgfqpoint{3.555067in}{1.585615in}}%
\pgfpathlineto{\pgfqpoint{3.569265in}{1.575752in}}%
\pgfpathlineto{\pgfqpoint{3.583466in}{1.565993in}}%
\pgfpathlineto{\pgfqpoint{3.574910in}{1.570469in}}%
\pgfpathlineto{\pgfqpoint{3.566338in}{1.575349in}}%
\pgfpathlineto{\pgfqpoint{3.557752in}{1.580638in}}%
\pgfpathlineto{\pgfqpoint{3.549150in}{1.586345in}}%
\pgfpathlineto{\pgfqpoint{3.534910in}{1.596776in}}%
\pgfpathlineto{\pgfqpoint{3.520672in}{1.607310in}}%
\pgfpathlineto{\pgfqpoint{3.506436in}{1.617950in}}%
\pgfpathlineto{\pgfqpoint{3.492203in}{1.628695in}}%
\pgfpathlineto{\pgfqpoint{3.500845in}{1.622307in}}%
\pgfpathlineto{\pgfqpoint{3.509472in}{1.616343in}}%
\pgfpathlineto{\pgfqpoint{3.518083in}{1.610795in}}%
\pgfpathlineto{\pgfqpoint{3.526678in}{1.605655in}}%
\pgfpathclose%
\pgfusepath{fill}%
\end{pgfscope}%
\begin{pgfscope}%
\pgfpathrectangle{\pgfqpoint{1.150000in}{0.150000in}}{\pgfqpoint{5.700000in}{5.700000in}}%
\pgfusepath{clip}%
\pgfsetbuttcap%
\pgfsetroundjoin%
\definecolor{currentfill}{rgb}{0.165117,0.467423,0.558141}%
\pgfsetfillcolor{currentfill}%
\pgfsetfillopacity{0.700000}%
\pgfsetlinewidth{0.000000pt}%
\definecolor{currentstroke}{rgb}{0.000000,0.000000,0.000000}%
\pgfsetstrokecolor{currentstroke}%
\pgfsetdash{}{0pt}%
\pgfpathmoveto{\pgfqpoint{2.809417in}{2.276870in}}%
\pgfpathlineto{\pgfqpoint{2.823672in}{2.260516in}}%
\pgfpathlineto{\pgfqpoint{2.837925in}{2.244292in}}%
\pgfpathlineto{\pgfqpoint{2.852174in}{2.228200in}}%
\pgfpathlineto{\pgfqpoint{2.866421in}{2.212237in}}%
\pgfpathlineto{\pgfqpoint{2.857189in}{2.226766in}}%
\pgfpathlineto{\pgfqpoint{2.847931in}{2.241804in}}%
\pgfpathlineto{\pgfqpoint{2.838645in}{2.257359in}}%
\pgfpathlineto{\pgfqpoint{2.829332in}{2.273439in}}%
\pgfpathlineto{\pgfqpoint{2.815022in}{2.290130in}}%
\pgfpathlineto{\pgfqpoint{2.800708in}{2.306953in}}%
\pgfpathlineto{\pgfqpoint{2.786392in}{2.323907in}}%
\pgfpathlineto{\pgfqpoint{2.772071in}{2.340994in}}%
\pgfpathlineto{\pgfqpoint{2.781450in}{2.324173in}}%
\pgfpathlineto{\pgfqpoint{2.790800in}{2.307884in}}%
\pgfpathlineto{\pgfqpoint{2.800122in}{2.292120in}}%
\pgfpathlineto{\pgfqpoint{2.809417in}{2.276870in}}%
\pgfpathclose%
\pgfusepath{fill}%
\end{pgfscope}%
\begin{pgfscope}%
\pgfpathrectangle{\pgfqpoint{1.150000in}{0.150000in}}{\pgfqpoint{5.700000in}{5.700000in}}%
\pgfusepath{clip}%
\pgfsetbuttcap%
\pgfsetroundjoin%
\definecolor{currentfill}{rgb}{0.175841,0.441290,0.557685}%
\pgfsetfillcolor{currentfill}%
\pgfsetfillopacity{0.700000}%
\pgfsetlinewidth{0.000000pt}%
\definecolor{currentstroke}{rgb}{0.000000,0.000000,0.000000}%
\pgfsetstrokecolor{currentstroke}%
\pgfsetdash{}{0pt}%
\pgfpathmoveto{\pgfqpoint{2.866421in}{2.212237in}}%
\pgfpathlineto{\pgfqpoint{2.880665in}{2.196403in}}%
\pgfpathlineto{\pgfqpoint{2.894906in}{2.180698in}}%
\pgfpathlineto{\pgfqpoint{2.909145in}{2.165120in}}%
\pgfpathlineto{\pgfqpoint{2.923382in}{2.149669in}}%
\pgfpathlineto{\pgfqpoint{2.914211in}{2.163482in}}%
\pgfpathlineto{\pgfqpoint{2.905014in}{2.177797in}}%
\pgfpathlineto{\pgfqpoint{2.895792in}{2.192622in}}%
\pgfpathlineto{\pgfqpoint{2.886543in}{2.207965in}}%
\pgfpathlineto{\pgfqpoint{2.872244in}{2.224141in}}%
\pgfpathlineto{\pgfqpoint{2.857943in}{2.240445in}}%
\pgfpathlineto{\pgfqpoint{2.843639in}{2.256877in}}%
\pgfpathlineto{\pgfqpoint{2.829332in}{2.273439in}}%
\pgfpathlineto{\pgfqpoint{2.838645in}{2.257359in}}%
\pgfpathlineto{\pgfqpoint{2.847931in}{2.241804in}}%
\pgfpathlineto{\pgfqpoint{2.857189in}{2.226766in}}%
\pgfpathlineto{\pgfqpoint{2.866421in}{2.212237in}}%
\pgfpathclose%
\pgfusepath{fill}%
\end{pgfscope}%
\begin{pgfscope}%
\pgfpathrectangle{\pgfqpoint{1.150000in}{0.150000in}}{\pgfqpoint{5.700000in}{5.700000in}}%
\pgfusepath{clip}%
\pgfsetbuttcap%
\pgfsetroundjoin%
\definecolor{currentfill}{rgb}{0.156270,0.489624,0.557936}%
\pgfsetfillcolor{currentfill}%
\pgfsetfillopacity{0.700000}%
\pgfsetlinewidth{0.000000pt}%
\definecolor{currentstroke}{rgb}{0.000000,0.000000,0.000000}%
\pgfsetstrokecolor{currentstroke}%
\pgfsetdash{}{0pt}%
\pgfpathmoveto{\pgfqpoint{2.752362in}{2.343620in}}%
\pgfpathlineto{\pgfqpoint{2.766631in}{2.326731in}}%
\pgfpathlineto{\pgfqpoint{2.780896in}{2.309977in}}%
\pgfpathlineto{\pgfqpoint{2.795158in}{2.293357in}}%
\pgfpathlineto{\pgfqpoint{2.809417in}{2.276870in}}%
\pgfpathlineto{\pgfqpoint{2.800122in}{2.292120in}}%
\pgfpathlineto{\pgfqpoint{2.790800in}{2.307884in}}%
\pgfpathlineto{\pgfqpoint{2.781450in}{2.324173in}}%
\pgfpathlineto{\pgfqpoint{2.772071in}{2.340994in}}%
\pgfpathlineto{\pgfqpoint{2.757747in}{2.358214in}}%
\pgfpathlineto{\pgfqpoint{2.743419in}{2.375568in}}%
\pgfpathlineto{\pgfqpoint{2.729088in}{2.393057in}}%
\pgfpathlineto{\pgfqpoint{2.714752in}{2.410682in}}%
\pgfpathlineto{\pgfqpoint{2.724198in}{2.393116in}}%
\pgfpathlineto{\pgfqpoint{2.733615in}{2.376089in}}%
\pgfpathlineto{\pgfqpoint{2.743003in}{2.359593in}}%
\pgfpathlineto{\pgfqpoint{2.752362in}{2.343620in}}%
\pgfpathclose%
\pgfusepath{fill}%
\end{pgfscope}%
\begin{pgfscope}%
\pgfpathrectangle{\pgfqpoint{1.150000in}{0.150000in}}{\pgfqpoint{5.700000in}{5.700000in}}%
\pgfusepath{clip}%
\pgfsetbuttcap%
\pgfsetroundjoin%
\definecolor{currentfill}{rgb}{0.183898,0.422383,0.556944}%
\pgfsetfillcolor{currentfill}%
\pgfsetfillopacity{0.700000}%
\pgfsetlinewidth{0.000000pt}%
\definecolor{currentstroke}{rgb}{0.000000,0.000000,0.000000}%
\pgfsetstrokecolor{currentstroke}%
\pgfsetdash{}{0pt}%
\pgfpathmoveto{\pgfqpoint{2.923382in}{2.149669in}}%
\pgfpathlineto{\pgfqpoint{2.937617in}{2.134345in}}%
\pgfpathlineto{\pgfqpoint{2.951849in}{2.119146in}}%
\pgfpathlineto{\pgfqpoint{2.966079in}{2.104071in}}%
\pgfpathlineto{\pgfqpoint{2.980308in}{2.089121in}}%
\pgfpathlineto{\pgfqpoint{2.971196in}{2.102222in}}%
\pgfpathlineto{\pgfqpoint{2.962059in}{2.115817in}}%
\pgfpathlineto{\pgfqpoint{2.952898in}{2.129916in}}%
\pgfpathlineto{\pgfqpoint{2.943711in}{2.144526in}}%
\pgfpathlineto{\pgfqpoint{2.929422in}{2.160198in}}%
\pgfpathlineto{\pgfqpoint{2.915132in}{2.175994in}}%
\pgfpathlineto{\pgfqpoint{2.900839in}{2.191917in}}%
\pgfpathlineto{\pgfqpoint{2.886543in}{2.207965in}}%
\pgfpathlineto{\pgfqpoint{2.895792in}{2.192622in}}%
\pgfpathlineto{\pgfqpoint{2.905014in}{2.177797in}}%
\pgfpathlineto{\pgfqpoint{2.914211in}{2.163482in}}%
\pgfpathlineto{\pgfqpoint{2.923382in}{2.149669in}}%
\pgfpathclose%
\pgfusepath{fill}%
\end{pgfscope}%
\begin{pgfscope}%
\pgfpathrectangle{\pgfqpoint{1.150000in}{0.150000in}}{\pgfqpoint{5.700000in}{5.700000in}}%
\pgfusepath{clip}%
\pgfsetbuttcap%
\pgfsetroundjoin%
\definecolor{currentfill}{rgb}{0.146180,0.515413,0.556823}%
\pgfsetfillcolor{currentfill}%
\pgfsetfillopacity{0.700000}%
\pgfsetlinewidth{0.000000pt}%
\definecolor{currentstroke}{rgb}{0.000000,0.000000,0.000000}%
\pgfsetstrokecolor{currentstroke}%
\pgfsetdash{}{0pt}%
\pgfpathmoveto{\pgfqpoint{2.695249in}{2.412539in}}%
\pgfpathlineto{\pgfqpoint{2.709533in}{2.395103in}}%
\pgfpathlineto{\pgfqpoint{2.723813in}{2.377805in}}%
\pgfpathlineto{\pgfqpoint{2.738090in}{2.360644in}}%
\pgfpathlineto{\pgfqpoint{2.752362in}{2.343620in}}%
\pgfpathlineto{\pgfqpoint{2.743003in}{2.359593in}}%
\pgfpathlineto{\pgfqpoint{2.733615in}{2.376089in}}%
\pgfpathlineto{\pgfqpoint{2.724198in}{2.393116in}}%
\pgfpathlineto{\pgfqpoint{2.714752in}{2.410682in}}%
\pgfpathlineto{\pgfqpoint{2.700412in}{2.428444in}}%
\pgfpathlineto{\pgfqpoint{2.686068in}{2.446343in}}%
\pgfpathlineto{\pgfqpoint{2.671719in}{2.464381in}}%
\pgfpathlineto{\pgfqpoint{2.657366in}{2.482557in}}%
\pgfpathlineto{\pgfqpoint{2.666882in}{2.464241in}}%
\pgfpathlineto{\pgfqpoint{2.676368in}{2.446472in}}%
\pgfpathlineto{\pgfqpoint{2.685823in}{2.429240in}}%
\pgfpathlineto{\pgfqpoint{2.695249in}{2.412539in}}%
\pgfpathclose%
\pgfusepath{fill}%
\end{pgfscope}%
\begin{pgfscope}%
\pgfpathrectangle{\pgfqpoint{1.150000in}{0.150000in}}{\pgfqpoint{5.700000in}{5.700000in}}%
\pgfusepath{clip}%
\pgfsetbuttcap%
\pgfsetroundjoin%
\definecolor{currentfill}{rgb}{0.194100,0.399323,0.555565}%
\pgfsetfillcolor{currentfill}%
\pgfsetfillopacity{0.700000}%
\pgfsetlinewidth{0.000000pt}%
\definecolor{currentstroke}{rgb}{0.000000,0.000000,0.000000}%
\pgfsetstrokecolor{currentstroke}%
\pgfsetdash{}{0pt}%
\pgfpathmoveto{\pgfqpoint{2.980308in}{2.089121in}}%
\pgfpathlineto{\pgfqpoint{2.994535in}{2.074295in}}%
\pgfpathlineto{\pgfqpoint{3.008760in}{2.059591in}}%
\pgfpathlineto{\pgfqpoint{3.022984in}{2.045009in}}%
\pgfpathlineto{\pgfqpoint{3.037206in}{2.030549in}}%
\pgfpathlineto{\pgfqpoint{3.028151in}{2.042940in}}%
\pgfpathlineto{\pgfqpoint{3.019073in}{2.055820in}}%
\pgfpathlineto{\pgfqpoint{3.009970in}{2.069196in}}%
\pgfpathlineto{\pgfqpoint{3.000843in}{2.083077in}}%
\pgfpathlineto{\pgfqpoint{2.986563in}{2.098255in}}%
\pgfpathlineto{\pgfqpoint{2.972281in}{2.113556in}}%
\pgfpathlineto{\pgfqpoint{2.957997in}{2.128979in}}%
\pgfpathlineto{\pgfqpoint{2.943711in}{2.144526in}}%
\pgfpathlineto{\pgfqpoint{2.952898in}{2.129916in}}%
\pgfpathlineto{\pgfqpoint{2.962059in}{2.115817in}}%
\pgfpathlineto{\pgfqpoint{2.971196in}{2.102222in}}%
\pgfpathlineto{\pgfqpoint{2.980308in}{2.089121in}}%
\pgfpathclose%
\pgfusepath{fill}%
\end{pgfscope}%
\begin{pgfscope}%
\pgfpathrectangle{\pgfqpoint{1.150000in}{0.150000in}}{\pgfqpoint{5.700000in}{5.700000in}}%
\pgfusepath{clip}%
\pgfsetbuttcap%
\pgfsetroundjoin%
\definecolor{currentfill}{rgb}{0.136408,0.541173,0.554483}%
\pgfsetfillcolor{currentfill}%
\pgfsetfillopacity{0.700000}%
\pgfsetlinewidth{0.000000pt}%
\definecolor{currentstroke}{rgb}{0.000000,0.000000,0.000000}%
\pgfsetstrokecolor{currentstroke}%
\pgfsetdash{}{0pt}%
\pgfpathmoveto{\pgfqpoint{2.638068in}{2.483683in}}%
\pgfpathlineto{\pgfqpoint{2.652370in}{2.465685in}}%
\pgfpathlineto{\pgfqpoint{2.666667in}{2.447829in}}%
\pgfpathlineto{\pgfqpoint{2.680960in}{2.430114in}}%
\pgfpathlineto{\pgfqpoint{2.695249in}{2.412539in}}%
\pgfpathlineto{\pgfqpoint{2.685823in}{2.429240in}}%
\pgfpathlineto{\pgfqpoint{2.676368in}{2.446472in}}%
\pgfpathlineto{\pgfqpoint{2.666882in}{2.464241in}}%
\pgfpathlineto{\pgfqpoint{2.657366in}{2.482557in}}%
\pgfpathlineto{\pgfqpoint{2.643009in}{2.500875in}}%
\pgfpathlineto{\pgfqpoint{2.628646in}{2.519333in}}%
\pgfpathlineto{\pgfqpoint{2.614278in}{2.537933in}}%
\pgfpathlineto{\pgfqpoint{2.599906in}{2.556677in}}%
\pgfpathlineto{\pgfqpoint{2.609493in}{2.537606in}}%
\pgfpathlineto{\pgfqpoint{2.619049in}{2.519089in}}%
\pgfpathlineto{\pgfqpoint{2.628574in}{2.501118in}}%
\pgfpathlineto{\pgfqpoint{2.638068in}{2.483683in}}%
\pgfpathclose%
\pgfusepath{fill}%
\end{pgfscope}%
\begin{pgfscope}%
\pgfpathrectangle{\pgfqpoint{1.150000in}{0.150000in}}{\pgfqpoint{5.700000in}{5.700000in}}%
\pgfusepath{clip}%
\pgfsetbuttcap%
\pgfsetroundjoin%
\definecolor{currentfill}{rgb}{0.279566,0.067836,0.391917}%
\pgfsetfillcolor{currentfill}%
\pgfsetfillopacity{0.700000}%
\pgfsetlinewidth{0.000000pt}%
\definecolor{currentstroke}{rgb}{0.000000,0.000000,0.000000}%
\pgfsetstrokecolor{currentstroke}%
\pgfsetdash{}{0pt}%
\pgfpathmoveto{\pgfqpoint{4.432988in}{1.344046in}}%
\pgfpathlineto{\pgfqpoint{4.447358in}{1.342103in}}%
\pgfpathlineto{\pgfqpoint{4.461737in}{1.340254in}}%
\pgfpathlineto{\pgfqpoint{4.476125in}{1.338499in}}%
\pgfpathlineto{\pgfqpoint{4.490522in}{1.336839in}}%
\pgfpathlineto{\pgfqpoint{4.482427in}{1.327441in}}%
\pgfpathlineto{\pgfqpoint{4.474327in}{1.318254in}}%
\pgfpathlineto{\pgfqpoint{4.466223in}{1.309283in}}%
\pgfpathlineto{\pgfqpoint{4.458114in}{1.300534in}}%
\pgfpathlineto{\pgfqpoint{4.443709in}{1.302760in}}%
\pgfpathlineto{\pgfqpoint{4.429312in}{1.305080in}}%
\pgfpathlineto{\pgfqpoint{4.414924in}{1.307494in}}%
\pgfpathlineto{\pgfqpoint{4.400544in}{1.310003in}}%
\pgfpathlineto{\pgfqpoint{4.408662in}{1.318180in}}%
\pgfpathlineto{\pgfqpoint{4.416775in}{1.326583in}}%
\pgfpathlineto{\pgfqpoint{4.424884in}{1.335207in}}%
\pgfpathlineto{\pgfqpoint{4.432988in}{1.344046in}}%
\pgfpathclose%
\pgfusepath{fill}%
\end{pgfscope}%
\begin{pgfscope}%
\pgfpathrectangle{\pgfqpoint{1.150000in}{0.150000in}}{\pgfqpoint{5.700000in}{5.700000in}}%
\pgfusepath{clip}%
\pgfsetbuttcap%
\pgfsetroundjoin%
\definecolor{currentfill}{rgb}{0.277134,0.185228,0.489898}%
\pgfsetfillcolor{currentfill}%
\pgfsetfillopacity{0.700000}%
\pgfsetlinewidth{0.000000pt}%
\definecolor{currentstroke}{rgb}{0.000000,0.000000,0.000000}%
\pgfsetstrokecolor{currentstroke}%
\pgfsetdash{}{0pt}%
\pgfpathmoveto{\pgfqpoint{4.883085in}{1.588108in}}%
\pgfpathlineto{\pgfqpoint{4.897653in}{1.590621in}}%
\pgfpathlineto{\pgfqpoint{4.912232in}{1.593229in}}%
\pgfpathlineto{\pgfqpoint{4.926824in}{1.595932in}}%
\pgfpathlineto{\pgfqpoint{4.941428in}{1.598731in}}%
\pgfpathlineto{\pgfqpoint{4.933405in}{1.583859in}}%
\pgfpathlineto{\pgfqpoint{4.925379in}{1.569074in}}%
\pgfpathlineto{\pgfqpoint{4.917351in}{1.554382in}}%
\pgfpathlineto{\pgfqpoint{4.909320in}{1.539785in}}%
\pgfpathlineto{\pgfqpoint{4.894720in}{1.537468in}}%
\pgfpathlineto{\pgfqpoint{4.880131in}{1.535246in}}%
\pgfpathlineto{\pgfqpoint{4.865555in}{1.533119in}}%
\pgfpathlineto{\pgfqpoint{4.850990in}{1.531087in}}%
\pgfpathlineto{\pgfqpoint{4.859018in}{1.545195in}}%
\pgfpathlineto{\pgfqpoint{4.867044in}{1.559404in}}%
\pgfpathlineto{\pgfqpoint{4.875066in}{1.573710in}}%
\pgfpathlineto{\pgfqpoint{4.883085in}{1.588108in}}%
\pgfpathclose%
\pgfusepath{fill}%
\end{pgfscope}%
\begin{pgfscope}%
\pgfpathrectangle{\pgfqpoint{1.150000in}{0.150000in}}{\pgfqpoint{5.700000in}{5.700000in}}%
\pgfusepath{clip}%
\pgfsetbuttcap%
\pgfsetroundjoin%
\definecolor{currentfill}{rgb}{0.248629,0.278775,0.534556}%
\pgfsetfillcolor{currentfill}%
\pgfsetfillopacity{0.700000}%
\pgfsetlinewidth{0.000000pt}%
\definecolor{currentstroke}{rgb}{0.000000,0.000000,0.000000}%
\pgfsetstrokecolor{currentstroke}%
\pgfsetdash{}{0pt}%
\pgfpathmoveto{\pgfqpoint{5.096088in}{1.799931in}}%
\pgfpathlineto{\pgfqpoint{5.110774in}{1.804550in}}%
\pgfpathlineto{\pgfqpoint{5.125474in}{1.809266in}}%
\pgfpathlineto{\pgfqpoint{5.140187in}{1.814079in}}%
\pgfpathlineto{\pgfqpoint{5.154914in}{1.818988in}}%
\pgfpathlineto{\pgfqpoint{5.146913in}{1.802559in}}%
\pgfpathlineto{\pgfqpoint{5.138909in}{1.786157in}}%
\pgfpathlineto{\pgfqpoint{5.130902in}{1.769787in}}%
\pgfpathlineto{\pgfqpoint{5.122892in}{1.753452in}}%
\pgfpathlineto{\pgfqpoint{5.108172in}{1.748974in}}%
\pgfpathlineto{\pgfqpoint{5.093466in}{1.744592in}}%
\pgfpathlineto{\pgfqpoint{5.078773in}{1.740307in}}%
\pgfpathlineto{\pgfqpoint{5.064094in}{1.736118in}}%
\pgfpathlineto{\pgfqpoint{5.072097in}{1.752015in}}%
\pgfpathlineto{\pgfqpoint{5.080097in}{1.767953in}}%
\pgfpathlineto{\pgfqpoint{5.088094in}{1.783926in}}%
\pgfpathlineto{\pgfqpoint{5.096088in}{1.799931in}}%
\pgfpathclose%
\pgfusepath{fill}%
\end{pgfscope}%
\begin{pgfscope}%
\pgfpathrectangle{\pgfqpoint{1.150000in}{0.150000in}}{\pgfqpoint{5.700000in}{5.700000in}}%
\pgfusepath{clip}%
\pgfsetbuttcap%
\pgfsetroundjoin%
\definecolor{currentfill}{rgb}{0.203063,0.379716,0.553925}%
\pgfsetfillcolor{currentfill}%
\pgfsetfillopacity{0.700000}%
\pgfsetlinewidth{0.000000pt}%
\definecolor{currentstroke}{rgb}{0.000000,0.000000,0.000000}%
\pgfsetstrokecolor{currentstroke}%
\pgfsetdash{}{0pt}%
\pgfpathmoveto{\pgfqpoint{3.037206in}{2.030549in}}%
\pgfpathlineto{\pgfqpoint{3.051427in}{2.016210in}}%
\pgfpathlineto{\pgfqpoint{3.065647in}{2.001991in}}%
\pgfpathlineto{\pgfqpoint{3.079865in}{1.987892in}}%
\pgfpathlineto{\pgfqpoint{3.094083in}{1.973911in}}%
\pgfpathlineto{\pgfqpoint{3.085084in}{1.985596in}}%
\pgfpathlineto{\pgfqpoint{3.076062in}{1.997764in}}%
\pgfpathlineto{\pgfqpoint{3.067017in}{2.010421in}}%
\pgfpathlineto{\pgfqpoint{3.057948in}{2.023577in}}%
\pgfpathlineto{\pgfqpoint{3.043674in}{2.038271in}}%
\pgfpathlineto{\pgfqpoint{3.029399in}{2.053086in}}%
\pgfpathlineto{\pgfqpoint{3.015122in}{2.068021in}}%
\pgfpathlineto{\pgfqpoint{3.000843in}{2.083077in}}%
\pgfpathlineto{\pgfqpoint{3.009970in}{2.069196in}}%
\pgfpathlineto{\pgfqpoint{3.019073in}{2.055820in}}%
\pgfpathlineto{\pgfqpoint{3.028151in}{2.042940in}}%
\pgfpathlineto{\pgfqpoint{3.037206in}{2.030549in}}%
\pgfpathclose%
\pgfusepath{fill}%
\end{pgfscope}%
\begin{pgfscope}%
\pgfpathrectangle{\pgfqpoint{1.150000in}{0.150000in}}{\pgfqpoint{5.700000in}{5.700000in}}%
\pgfusepath{clip}%
\pgfsetbuttcap%
\pgfsetroundjoin%
\definecolor{currentfill}{rgb}{0.127568,0.566949,0.550556}%
\pgfsetfillcolor{currentfill}%
\pgfsetfillopacity{0.700000}%
\pgfsetlinewidth{0.000000pt}%
\definecolor{currentstroke}{rgb}{0.000000,0.000000,0.000000}%
\pgfsetstrokecolor{currentstroke}%
\pgfsetdash{}{0pt}%
\pgfpathmoveto{\pgfqpoint{2.580812in}{2.557114in}}%
\pgfpathlineto{\pgfqpoint{2.595134in}{2.538538in}}%
\pgfpathlineto{\pgfqpoint{2.609450in}{2.520109in}}%
\pgfpathlineto{\pgfqpoint{2.623762in}{2.501824in}}%
\pgfpathlineto{\pgfqpoint{2.638068in}{2.483683in}}%
\pgfpathlineto{\pgfqpoint{2.628574in}{2.501118in}}%
\pgfpathlineto{\pgfqpoint{2.619049in}{2.519089in}}%
\pgfpathlineto{\pgfqpoint{2.609493in}{2.537606in}}%
\pgfpathlineto{\pgfqpoint{2.599906in}{2.556677in}}%
\pgfpathlineto{\pgfqpoint{2.585528in}{2.575565in}}%
\pgfpathlineto{\pgfqpoint{2.571145in}{2.594598in}}%
\pgfpathlineto{\pgfqpoint{2.556756in}{2.613777in}}%
\pgfpathlineto{\pgfqpoint{2.542362in}{2.633103in}}%
\pgfpathlineto{\pgfqpoint{2.552023in}{2.613272in}}%
\pgfpathlineto{\pgfqpoint{2.561651in}{2.594003in}}%
\pgfpathlineto{\pgfqpoint{2.571247in}{2.575286in}}%
\pgfpathlineto{\pgfqpoint{2.580812in}{2.557114in}}%
\pgfpathclose%
\pgfusepath{fill}%
\end{pgfscope}%
\begin{pgfscope}%
\pgfpathrectangle{\pgfqpoint{1.150000in}{0.150000in}}{\pgfqpoint{5.700000in}{5.700000in}}%
\pgfusepath{clip}%
\pgfsetbuttcap%
\pgfsetroundjoin%
\definecolor{currentfill}{rgb}{0.449368,0.813768,0.335384}%
\pgfsetfillcolor{currentfill}%
\pgfsetfillopacity{0.700000}%
\pgfsetlinewidth{0.000000pt}%
\definecolor{currentstroke}{rgb}{0.000000,0.000000,0.000000}%
\pgfsetstrokecolor{currentstroke}%
\pgfsetdash{}{0pt}%
\pgfpathmoveto{\pgfqpoint{2.043005in}{3.336336in}}%
\pgfpathlineto{\pgfqpoint{2.057579in}{3.311648in}}%
\pgfpathlineto{\pgfqpoint{2.072141in}{3.287161in}}%
\pgfpathlineto{\pgfqpoint{2.086692in}{3.262875in}}%
\pgfpathlineto{\pgfqpoint{2.101231in}{3.238786in}}%
\pgfpathlineto{\pgfqpoint{2.091101in}{3.261496in}}%
\pgfpathlineto{\pgfqpoint{2.080932in}{3.284783in}}%
\pgfpathlineto{\pgfqpoint{2.070722in}{3.308657in}}%
\pgfpathlineto{\pgfqpoint{2.060472in}{3.333127in}}%
\pgfpathlineto{\pgfqpoint{2.045845in}{3.357994in}}%
\pgfpathlineto{\pgfqpoint{2.031205in}{3.383061in}}%
\pgfpathlineto{\pgfqpoint{2.016554in}{3.408330in}}%
\pgfpathlineto{\pgfqpoint{2.001891in}{3.433803in}}%
\pgfpathlineto{\pgfqpoint{2.012231in}{3.408539in}}%
\pgfpathlineto{\pgfqpoint{2.022530in}{3.383879in}}%
\pgfpathlineto{\pgfqpoint{2.032788in}{3.359814in}}%
\pgfpathlineto{\pgfqpoint{2.043005in}{3.336336in}}%
\pgfpathclose%
\pgfusepath{fill}%
\end{pgfscope}%
\begin{pgfscope}%
\pgfpathrectangle{\pgfqpoint{1.150000in}{0.150000in}}{\pgfqpoint{5.700000in}{5.700000in}}%
\pgfusepath{clip}%
\pgfsetbuttcap%
\pgfsetroundjoin%
\definecolor{currentfill}{rgb}{0.278012,0.180367,0.486697}%
\pgfsetfillcolor{currentfill}%
\pgfsetfillopacity{0.700000}%
\pgfsetlinewidth{0.000000pt}%
\definecolor{currentstroke}{rgb}{0.000000,0.000000,0.000000}%
\pgfsetstrokecolor{currentstroke}%
\pgfsetdash{}{0pt}%
\pgfpathmoveto{\pgfqpoint{3.583466in}{1.565993in}}%
\pgfpathlineto{\pgfqpoint{3.597670in}{1.556338in}}%
\pgfpathlineto{\pgfqpoint{3.611876in}{1.546786in}}%
\pgfpathlineto{\pgfqpoint{3.626085in}{1.537338in}}%
\pgfpathlineto{\pgfqpoint{3.640298in}{1.527992in}}%
\pgfpathlineto{\pgfqpoint{3.631778in}{1.531806in}}%
\pgfpathlineto{\pgfqpoint{3.623245in}{1.536018in}}%
\pgfpathlineto{\pgfqpoint{3.614697in}{1.540635in}}%
\pgfpathlineto{\pgfqpoint{3.606134in}{1.545663in}}%
\pgfpathlineto{\pgfqpoint{3.591885in}{1.555679in}}%
\pgfpathlineto{\pgfqpoint{3.577637in}{1.565797in}}%
\pgfpathlineto{\pgfqpoint{3.563393in}{1.576019in}}%
\pgfpathlineto{\pgfqpoint{3.549150in}{1.586345in}}%
\pgfpathlineto{\pgfqpoint{3.557752in}{1.580638in}}%
\pgfpathlineto{\pgfqpoint{3.566338in}{1.575349in}}%
\pgfpathlineto{\pgfqpoint{3.574910in}{1.570469in}}%
\pgfpathlineto{\pgfqpoint{3.583466in}{1.565993in}}%
\pgfpathclose%
\pgfusepath{fill}%
\end{pgfscope}%
\begin{pgfscope}%
\pgfpathrectangle{\pgfqpoint{1.150000in}{0.150000in}}{\pgfqpoint{5.700000in}{5.700000in}}%
\pgfusepath{clip}%
\pgfsetbuttcap%
\pgfsetroundjoin%
\definecolor{currentfill}{rgb}{0.283091,0.110553,0.431554}%
\pgfsetfillcolor{currentfill}%
\pgfsetfillopacity{0.700000}%
\pgfsetlinewidth{0.000000pt}%
\definecolor{currentstroke}{rgb}{0.000000,0.000000,0.000000}%
\pgfsetstrokecolor{currentstroke}%
\pgfsetdash{}{0pt}%
\pgfpathmoveto{\pgfqpoint{3.844664in}{1.420080in}}%
\pgfpathlineto{\pgfqpoint{3.858893in}{1.412690in}}%
\pgfpathlineto{\pgfqpoint{3.873127in}{1.405398in}}%
\pgfpathlineto{\pgfqpoint{3.887365in}{1.398206in}}%
\pgfpathlineto{\pgfqpoint{3.901609in}{1.391111in}}%
\pgfpathlineto{\pgfqpoint{3.893265in}{1.390860in}}%
\pgfpathlineto{\pgfqpoint{3.884910in}{1.390956in}}%
\pgfpathlineto{\pgfqpoint{3.876545in}{1.391406in}}%
\pgfpathlineto{\pgfqpoint{3.868169in}{1.392217in}}%
\pgfpathlineto{\pgfqpoint{3.853898in}{1.399955in}}%
\pgfpathlineto{\pgfqpoint{3.839630in}{1.407791in}}%
\pgfpathlineto{\pgfqpoint{3.825367in}{1.415727in}}%
\pgfpathlineto{\pgfqpoint{3.811108in}{1.423761in}}%
\pgfpathlineto{\pgfqpoint{3.819514in}{1.422298in}}%
\pgfpathlineto{\pgfqpoint{3.827908in}{1.421201in}}%
\pgfpathlineto{\pgfqpoint{3.836292in}{1.420464in}}%
\pgfpathlineto{\pgfqpoint{3.844664in}{1.420080in}}%
\pgfpathclose%
\pgfusepath{fill}%
\end{pgfscope}%
\begin{pgfscope}%
\pgfpathrectangle{\pgfqpoint{1.150000in}{0.150000in}}{\pgfqpoint{5.700000in}{5.700000in}}%
\pgfusepath{clip}%
\pgfsetbuttcap%
\pgfsetroundjoin%
\definecolor{currentfill}{rgb}{0.214298,0.355619,0.551184}%
\pgfsetfillcolor{currentfill}%
\pgfsetfillopacity{0.700000}%
\pgfsetlinewidth{0.000000pt}%
\definecolor{currentstroke}{rgb}{0.000000,0.000000,0.000000}%
\pgfsetstrokecolor{currentstroke}%
\pgfsetdash{}{0pt}%
\pgfpathmoveto{\pgfqpoint{3.094083in}{1.973911in}}%
\pgfpathlineto{\pgfqpoint{3.108300in}{1.960050in}}%
\pgfpathlineto{\pgfqpoint{3.122516in}{1.946306in}}%
\pgfpathlineto{\pgfqpoint{3.136731in}{1.932680in}}%
\pgfpathlineto{\pgfqpoint{3.150946in}{1.919170in}}%
\pgfpathlineto{\pgfqpoint{3.142001in}{1.930152in}}%
\pgfpathlineto{\pgfqpoint{3.133033in}{1.941610in}}%
\pgfpathlineto{\pgfqpoint{3.124044in}{1.953552in}}%
\pgfpathlineto{\pgfqpoint{3.115032in}{1.965985in}}%
\pgfpathlineto{\pgfqpoint{3.100762in}{1.980206in}}%
\pgfpathlineto{\pgfqpoint{3.086492in}{1.994545in}}%
\pgfpathlineto{\pgfqpoint{3.072221in}{2.009001in}}%
\pgfpathlineto{\pgfqpoint{3.057948in}{2.023577in}}%
\pgfpathlineto{\pgfqpoint{3.067017in}{2.010421in}}%
\pgfpathlineto{\pgfqpoint{3.076062in}{1.997764in}}%
\pgfpathlineto{\pgfqpoint{3.085084in}{1.985596in}}%
\pgfpathlineto{\pgfqpoint{3.094083in}{1.973911in}}%
\pgfpathclose%
\pgfusepath{fill}%
\end{pgfscope}%
\begin{pgfscope}%
\pgfpathrectangle{\pgfqpoint{1.150000in}{0.150000in}}{\pgfqpoint{5.700000in}{5.700000in}}%
\pgfusepath{clip}%
\pgfsetbuttcap%
\pgfsetroundjoin%
\definecolor{currentfill}{rgb}{0.278791,0.062145,0.386592}%
\pgfsetfillcolor{currentfill}%
\pgfsetfillopacity{0.700000}%
\pgfsetlinewidth{0.000000pt}%
\definecolor{currentstroke}{rgb}{0.000000,0.000000,0.000000}%
\pgfsetstrokecolor{currentstroke}%
\pgfsetdash{}{0pt}%
\pgfpathmoveto{\pgfqpoint{4.195927in}{1.324140in}}%
\pgfpathlineto{\pgfqpoint{4.210228in}{1.319913in}}%
\pgfpathlineto{\pgfqpoint{4.224535in}{1.315781in}}%
\pgfpathlineto{\pgfqpoint{4.238850in}{1.311744in}}%
\pgfpathlineto{\pgfqpoint{4.253172in}{1.307802in}}%
\pgfpathlineto{\pgfqpoint{4.245000in}{1.302034in}}%
\pgfpathlineto{\pgfqpoint{4.236822in}{1.296537in}}%
\pgfpathlineto{\pgfqpoint{4.228637in}{1.291317in}}%
\pgfpathlineto{\pgfqpoint{4.220446in}{1.286380in}}%
\pgfpathlineto{\pgfqpoint{4.206107in}{1.290924in}}%
\pgfpathlineto{\pgfqpoint{4.191776in}{1.295563in}}%
\pgfpathlineto{\pgfqpoint{4.177451in}{1.300297in}}%
\pgfpathlineto{\pgfqpoint{4.163133in}{1.305126in}}%
\pgfpathlineto{\pgfqpoint{4.171342in}{1.309453in}}%
\pgfpathlineto{\pgfqpoint{4.179544in}{1.314069in}}%
\pgfpathlineto{\pgfqpoint{4.187739in}{1.318966in}}%
\pgfpathlineto{\pgfqpoint{4.195927in}{1.324140in}}%
\pgfpathclose%
\pgfusepath{fill}%
\end{pgfscope}%
\begin{pgfscope}%
\pgfpathrectangle{\pgfqpoint{1.150000in}{0.150000in}}{\pgfqpoint{5.700000in}{5.700000in}}%
\pgfusepath{clip}%
\pgfsetbuttcap%
\pgfsetroundjoin%
\definecolor{currentfill}{rgb}{0.121148,0.592739,0.544641}%
\pgfsetfillcolor{currentfill}%
\pgfsetfillopacity{0.700000}%
\pgfsetlinewidth{0.000000pt}%
\definecolor{currentstroke}{rgb}{0.000000,0.000000,0.000000}%
\pgfsetstrokecolor{currentstroke}%
\pgfsetdash{}{0pt}%
\pgfpathmoveto{\pgfqpoint{2.523471in}{2.632896in}}%
\pgfpathlineto{\pgfqpoint{2.537814in}{2.613726in}}%
\pgfpathlineto{\pgfqpoint{2.552152in}{2.594707in}}%
\pgfpathlineto{\pgfqpoint{2.566485in}{2.575836in}}%
\pgfpathlineto{\pgfqpoint{2.580812in}{2.557114in}}%
\pgfpathlineto{\pgfqpoint{2.571247in}{2.575286in}}%
\pgfpathlineto{\pgfqpoint{2.561651in}{2.594003in}}%
\pgfpathlineto{\pgfqpoint{2.552023in}{2.613272in}}%
\pgfpathlineto{\pgfqpoint{2.542362in}{2.633103in}}%
\pgfpathlineto{\pgfqpoint{2.527961in}{2.652577in}}%
\pgfpathlineto{\pgfqpoint{2.513555in}{2.672201in}}%
\pgfpathlineto{\pgfqpoint{2.499143in}{2.691975in}}%
\pgfpathlineto{\pgfqpoint{2.484724in}{2.711900in}}%
\pgfpathlineto{\pgfqpoint{2.494461in}{2.691305in}}%
\pgfpathlineto{\pgfqpoint{2.504164in}{2.671278in}}%
\pgfpathlineto{\pgfqpoint{2.513834in}{2.651811in}}%
\pgfpathlineto{\pgfqpoint{2.523471in}{2.632896in}}%
\pgfpathclose%
\pgfusepath{fill}%
\end{pgfscope}%
\begin{pgfscope}%
\pgfpathrectangle{\pgfqpoint{1.150000in}{0.150000in}}{\pgfqpoint{5.700000in}{5.700000in}}%
\pgfusepath{clip}%
\pgfsetbuttcap%
\pgfsetroundjoin%
\definecolor{currentfill}{rgb}{0.280267,0.073417,0.397163}%
\pgfsetfillcolor{currentfill}%
\pgfsetfillopacity{0.700000}%
\pgfsetlinewidth{0.000000pt}%
\definecolor{currentstroke}{rgb}{0.000000,0.000000,0.000000}%
\pgfsetstrokecolor{currentstroke}%
\pgfsetdash{}{0pt}%
\pgfpathmoveto{\pgfqpoint{4.048818in}{1.347202in}}%
\pgfpathlineto{\pgfqpoint{4.063086in}{1.341606in}}%
\pgfpathlineto{\pgfqpoint{4.077360in}{1.336107in}}%
\pgfpathlineto{\pgfqpoint{4.091639in}{1.330704in}}%
\pgfpathlineto{\pgfqpoint{4.105925in}{1.325396in}}%
\pgfpathlineto{\pgfqpoint{4.097689in}{1.321978in}}%
\pgfpathlineto{\pgfqpoint{4.089445in}{1.318865in}}%
\pgfpathlineto{\pgfqpoint{4.081193in}{1.316065in}}%
\pgfpathlineto{\pgfqpoint{4.072933in}{1.313583in}}%
\pgfpathlineto{\pgfqpoint{4.058625in}{1.319512in}}%
\pgfpathlineto{\pgfqpoint{4.044324in}{1.325537in}}%
\pgfpathlineto{\pgfqpoint{4.030028in}{1.331658in}}%
\pgfpathlineto{\pgfqpoint{4.015738in}{1.337876in}}%
\pgfpathlineto{\pgfqpoint{4.024021in}{1.339728in}}%
\pgfpathlineto{\pgfqpoint{4.032295in}{1.341904in}}%
\pgfpathlineto{\pgfqpoint{4.040561in}{1.344398in}}%
\pgfpathlineto{\pgfqpoint{4.048818in}{1.347202in}}%
\pgfpathclose%
\pgfusepath{fill}%
\end{pgfscope}%
\begin{pgfscope}%
\pgfpathrectangle{\pgfqpoint{1.150000in}{0.150000in}}{\pgfqpoint{5.700000in}{5.700000in}}%
\pgfusepath{clip}%
\pgfsetbuttcap%
\pgfsetroundjoin%
\definecolor{currentfill}{rgb}{0.269308,0.218818,0.509577}%
\pgfsetfillcolor{currentfill}%
\pgfsetfillopacity{0.700000}%
\pgfsetlinewidth{0.000000pt}%
\definecolor{currentstroke}{rgb}{0.000000,0.000000,0.000000}%
\pgfsetstrokecolor{currentstroke}%
\pgfsetdash{}{0pt}%
\pgfpathmoveto{\pgfqpoint{4.973492in}{1.659004in}}%
\pgfpathlineto{\pgfqpoint{4.988113in}{1.662362in}}%
\pgfpathlineto{\pgfqpoint{5.002747in}{1.665816in}}%
\pgfpathlineto{\pgfqpoint{5.017394in}{1.669365in}}%
\pgfpathlineto{\pgfqpoint{5.032054in}{1.673011in}}%
\pgfpathlineto{\pgfqpoint{5.024036in}{1.657375in}}%
\pgfpathlineto{\pgfqpoint{5.016017in}{1.641805in}}%
\pgfpathlineto{\pgfqpoint{5.007994in}{1.626304in}}%
\pgfpathlineto{\pgfqpoint{4.999969in}{1.610878in}}%
\pgfpathlineto{\pgfqpoint{4.985315in}{1.607698in}}%
\pgfpathlineto{\pgfqpoint{4.970673in}{1.604614in}}%
\pgfpathlineto{\pgfqpoint{4.956044in}{1.601625in}}%
\pgfpathlineto{\pgfqpoint{4.941428in}{1.598731in}}%
\pgfpathlineto{\pgfqpoint{4.949448in}{1.613686in}}%
\pgfpathlineto{\pgfqpoint{4.957465in}{1.628719in}}%
\pgfpathlineto{\pgfqpoint{4.965480in}{1.643827in}}%
\pgfpathlineto{\pgfqpoint{4.973492in}{1.659004in}}%
\pgfpathclose%
\pgfusepath{fill}%
\end{pgfscope}%
\begin{pgfscope}%
\pgfpathrectangle{\pgfqpoint{1.150000in}{0.150000in}}{\pgfqpoint{5.700000in}{5.700000in}}%
\pgfusepath{clip}%
\pgfsetbuttcap%
\pgfsetroundjoin%
\definecolor{currentfill}{rgb}{0.221989,0.339161,0.548752}%
\pgfsetfillcolor{currentfill}%
\pgfsetfillopacity{0.700000}%
\pgfsetlinewidth{0.000000pt}%
\definecolor{currentstroke}{rgb}{0.000000,0.000000,0.000000}%
\pgfsetstrokecolor{currentstroke}%
\pgfsetdash{}{0pt}%
\pgfpathmoveto{\pgfqpoint{3.150946in}{1.919170in}}%
\pgfpathlineto{\pgfqpoint{3.165160in}{1.905777in}}%
\pgfpathlineto{\pgfqpoint{3.179375in}{1.892500in}}%
\pgfpathlineto{\pgfqpoint{3.193588in}{1.879337in}}%
\pgfpathlineto{\pgfqpoint{3.207802in}{1.866290in}}%
\pgfpathlineto{\pgfqpoint{3.198909in}{1.876572in}}%
\pgfpathlineto{\pgfqpoint{3.189994in}{1.887323in}}%
\pgfpathlineto{\pgfqpoint{3.181059in}{1.898552in}}%
\pgfpathlineto{\pgfqpoint{3.172101in}{1.910266in}}%
\pgfpathlineto{\pgfqpoint{3.157835in}{1.924022in}}%
\pgfpathlineto{\pgfqpoint{3.143568in}{1.937894in}}%
\pgfpathlineto{\pgfqpoint{3.129300in}{1.951881in}}%
\pgfpathlineto{\pgfqpoint{3.115032in}{1.965985in}}%
\pgfpathlineto{\pgfqpoint{3.124044in}{1.953552in}}%
\pgfpathlineto{\pgfqpoint{3.133033in}{1.941610in}}%
\pgfpathlineto{\pgfqpoint{3.142001in}{1.930152in}}%
\pgfpathlineto{\pgfqpoint{3.150946in}{1.919170in}}%
\pgfpathclose%
\pgfusepath{fill}%
\end{pgfscope}%
\begin{pgfscope}%
\pgfpathrectangle{\pgfqpoint{1.150000in}{0.150000in}}{\pgfqpoint{5.700000in}{5.700000in}}%
\pgfusepath{clip}%
\pgfsetbuttcap%
\pgfsetroundjoin%
\definecolor{currentfill}{rgb}{0.278791,0.062145,0.386592}%
\pgfsetfillcolor{currentfill}%
\pgfsetfillopacity{0.700000}%
\pgfsetlinewidth{0.000000pt}%
\definecolor{currentstroke}{rgb}{0.000000,0.000000,0.000000}%
\pgfsetstrokecolor{currentstroke}%
\pgfsetdash{}{0pt}%
\pgfpathmoveto{\pgfqpoint{4.343107in}{1.320978in}}%
\pgfpathlineto{\pgfqpoint{4.357454in}{1.318093in}}%
\pgfpathlineto{\pgfqpoint{4.371809in}{1.315302in}}%
\pgfpathlineto{\pgfqpoint{4.386172in}{1.312605in}}%
\pgfpathlineto{\pgfqpoint{4.400544in}{1.310003in}}%
\pgfpathlineto{\pgfqpoint{4.392421in}{1.302058in}}%
\pgfpathlineto{\pgfqpoint{4.384292in}{1.294350in}}%
\pgfpathlineto{\pgfqpoint{4.376159in}{1.286887in}}%
\pgfpathlineto{\pgfqpoint{4.368020in}{1.279673in}}%
\pgfpathlineto{\pgfqpoint{4.353637in}{1.282859in}}%
\pgfpathlineto{\pgfqpoint{4.339262in}{1.286139in}}%
\pgfpathlineto{\pgfqpoint{4.324895in}{1.289513in}}%
\pgfpathlineto{\pgfqpoint{4.310535in}{1.292982in}}%
\pgfpathlineto{\pgfqpoint{4.318686in}{1.299606in}}%
\pgfpathlineto{\pgfqpoint{4.326832in}{1.306483in}}%
\pgfpathlineto{\pgfqpoint{4.334972in}{1.313609in}}%
\pgfpathlineto{\pgfqpoint{4.343107in}{1.320978in}}%
\pgfpathclose%
\pgfusepath{fill}%
\end{pgfscope}%
\begin{pgfscope}%
\pgfpathrectangle{\pgfqpoint{1.150000in}{0.150000in}}{\pgfqpoint{5.700000in}{5.700000in}}%
\pgfusepath{clip}%
\pgfsetbuttcap%
\pgfsetroundjoin%
\definecolor{currentfill}{rgb}{0.231674,0.318106,0.544834}%
\pgfsetfillcolor{currentfill}%
\pgfsetfillopacity{0.700000}%
\pgfsetlinewidth{0.000000pt}%
\definecolor{currentstroke}{rgb}{0.000000,0.000000,0.000000}%
\pgfsetstrokecolor{currentstroke}%
\pgfsetdash{}{0pt}%
\pgfpathmoveto{\pgfqpoint{5.186887in}{1.884901in}}%
\pgfpathlineto{\pgfqpoint{5.201637in}{1.890321in}}%
\pgfpathlineto{\pgfqpoint{5.216400in}{1.895838in}}%
\pgfpathlineto{\pgfqpoint{5.231179in}{1.901452in}}%
\pgfpathlineto{\pgfqpoint{5.223184in}{1.884647in}}%
\pgfpathlineto{\pgfqpoint{5.215186in}{1.867850in}}%
\pgfpathlineto{\pgfqpoint{5.207185in}{1.851065in}}%
\pgfpathlineto{\pgfqpoint{5.199181in}{1.834297in}}%
\pgfpathlineto{\pgfqpoint{5.184412in}{1.829097in}}%
\pgfpathlineto{\pgfqpoint{5.169656in}{1.823994in}}%
\pgfpathlineto{\pgfqpoint{5.154914in}{1.818988in}}%
\pgfpathlineto{\pgfqpoint{5.162912in}{1.835441in}}%
\pgfpathlineto{\pgfqpoint{5.170907in}{1.851913in}}%
\pgfpathlineto{\pgfqpoint{5.178899in}{1.868401in}}%
\pgfpathlineto{\pgfqpoint{5.186887in}{1.884901in}}%
\pgfpathclose%
\pgfusepath{fill}%
\end{pgfscope}%
\begin{pgfscope}%
\pgfpathrectangle{\pgfqpoint{1.150000in}{0.150000in}}{\pgfqpoint{5.700000in}{5.700000in}}%
\pgfusepath{clip}%
\pgfsetbuttcap%
\pgfsetroundjoin%
\definecolor{currentfill}{rgb}{0.120081,0.622161,0.534946}%
\pgfsetfillcolor{currentfill}%
\pgfsetfillopacity{0.700000}%
\pgfsetlinewidth{0.000000pt}%
\definecolor{currentstroke}{rgb}{0.000000,0.000000,0.000000}%
\pgfsetstrokecolor{currentstroke}%
\pgfsetdash{}{0pt}%
\pgfpathmoveto{\pgfqpoint{2.466035in}{2.711098in}}%
\pgfpathlineto{\pgfqpoint{2.480404in}{2.691316in}}%
\pgfpathlineto{\pgfqpoint{2.494765in}{2.671690in}}%
\pgfpathlineto{\pgfqpoint{2.509121in}{2.652217in}}%
\pgfpathlineto{\pgfqpoint{2.523471in}{2.632896in}}%
\pgfpathlineto{\pgfqpoint{2.513834in}{2.651811in}}%
\pgfpathlineto{\pgfqpoint{2.504164in}{2.671278in}}%
\pgfpathlineto{\pgfqpoint{2.494461in}{2.691305in}}%
\pgfpathlineto{\pgfqpoint{2.484724in}{2.711900in}}%
\pgfpathlineto{\pgfqpoint{2.470300in}{2.731979in}}%
\pgfpathlineto{\pgfqpoint{2.455868in}{2.752210in}}%
\pgfpathlineto{\pgfqpoint{2.441430in}{2.772597in}}%
\pgfpathlineto{\pgfqpoint{2.426985in}{2.793140in}}%
\pgfpathlineto{\pgfqpoint{2.436799in}{2.771773in}}%
\pgfpathlineto{\pgfqpoint{2.446579in}{2.750984in}}%
\pgfpathlineto{\pgfqpoint{2.456324in}{2.730761in}}%
\pgfpathlineto{\pgfqpoint{2.466035in}{2.711098in}}%
\pgfpathclose%
\pgfusepath{fill}%
\end{pgfscope}%
\begin{pgfscope}%
\pgfpathrectangle{\pgfqpoint{1.150000in}{0.150000in}}{\pgfqpoint{5.700000in}{5.700000in}}%
\pgfusepath{clip}%
\pgfsetbuttcap%
\pgfsetroundjoin%
\definecolor{currentfill}{rgb}{0.231674,0.318106,0.544834}%
\pgfsetfillcolor{currentfill}%
\pgfsetfillopacity{0.700000}%
\pgfsetlinewidth{0.000000pt}%
\definecolor{currentstroke}{rgb}{0.000000,0.000000,0.000000}%
\pgfsetstrokecolor{currentstroke}%
\pgfsetdash{}{0pt}%
\pgfpathmoveto{\pgfqpoint{3.207802in}{1.866290in}}%
\pgfpathlineto{\pgfqpoint{3.222016in}{1.853356in}}%
\pgfpathlineto{\pgfqpoint{3.236230in}{1.840537in}}%
\pgfpathlineto{\pgfqpoint{3.250444in}{1.827830in}}%
\pgfpathlineto{\pgfqpoint{3.264658in}{1.815236in}}%
\pgfpathlineto{\pgfqpoint{3.255815in}{1.824821in}}%
\pgfpathlineto{\pgfqpoint{3.246951in}{1.834869in}}%
\pgfpathlineto{\pgfqpoint{3.238068in}{1.845387in}}%
\pgfpathlineto{\pgfqpoint{3.229163in}{1.856385in}}%
\pgfpathlineto{\pgfqpoint{3.214898in}{1.869685in}}%
\pgfpathlineto{\pgfqpoint{3.200632in}{1.883098in}}%
\pgfpathlineto{\pgfqpoint{3.186367in}{1.896625in}}%
\pgfpathlineto{\pgfqpoint{3.172101in}{1.910266in}}%
\pgfpathlineto{\pgfqpoint{3.181059in}{1.898552in}}%
\pgfpathlineto{\pgfqpoint{3.189994in}{1.887323in}}%
\pgfpathlineto{\pgfqpoint{3.198909in}{1.876572in}}%
\pgfpathlineto{\pgfqpoint{3.207802in}{1.866290in}}%
\pgfpathclose%
\pgfusepath{fill}%
\end{pgfscope}%
\begin{pgfscope}%
\pgfpathrectangle{\pgfqpoint{1.150000in}{0.150000in}}{\pgfqpoint{5.700000in}{5.700000in}}%
\pgfusepath{clip}%
\pgfsetbuttcap%
\pgfsetroundjoin%
\definecolor{currentfill}{rgb}{0.279574,0.170599,0.479997}%
\pgfsetfillcolor{currentfill}%
\pgfsetfillopacity{0.700000}%
\pgfsetlinewidth{0.000000pt}%
\definecolor{currentstroke}{rgb}{0.000000,0.000000,0.000000}%
\pgfsetstrokecolor{currentstroke}%
\pgfsetdash{}{0pt}%
\pgfpathmoveto{\pgfqpoint{3.640298in}{1.527992in}}%
\pgfpathlineto{\pgfqpoint{3.654513in}{1.518749in}}%
\pgfpathlineto{\pgfqpoint{3.668731in}{1.509608in}}%
\pgfpathlineto{\pgfqpoint{3.682953in}{1.500569in}}%
\pgfpathlineto{\pgfqpoint{3.697178in}{1.491632in}}%
\pgfpathlineto{\pgfqpoint{3.688694in}{1.494786in}}%
\pgfpathlineto{\pgfqpoint{3.680197in}{1.498332in}}%
\pgfpathlineto{\pgfqpoint{3.671686in}{1.502277in}}%
\pgfpathlineto{\pgfqpoint{3.663161in}{1.506628in}}%
\pgfpathlineto{\pgfqpoint{3.648900in}{1.516233in}}%
\pgfpathlineto{\pgfqpoint{3.634642in}{1.525941in}}%
\pgfpathlineto{\pgfqpoint{3.620387in}{1.535751in}}%
\pgfpathlineto{\pgfqpoint{3.606134in}{1.545663in}}%
\pgfpathlineto{\pgfqpoint{3.614697in}{1.540635in}}%
\pgfpathlineto{\pgfqpoint{3.623245in}{1.536018in}}%
\pgfpathlineto{\pgfqpoint{3.631778in}{1.531806in}}%
\pgfpathlineto{\pgfqpoint{3.640298in}{1.527992in}}%
\pgfpathclose%
\pgfusepath{fill}%
\end{pgfscope}%
\begin{pgfscope}%
\pgfpathrectangle{\pgfqpoint{1.150000in}{0.150000in}}{\pgfqpoint{5.700000in}{5.700000in}}%
\pgfusepath{clip}%
\pgfsetbuttcap%
\pgfsetroundjoin%
\definecolor{currentfill}{rgb}{0.283091,0.110553,0.431554}%
\pgfsetfillcolor{currentfill}%
\pgfsetfillopacity{0.700000}%
\pgfsetlinewidth{0.000000pt}%
\definecolor{currentstroke}{rgb}{0.000000,0.000000,0.000000}%
\pgfsetstrokecolor{currentstroke}%
\pgfsetdash{}{0pt}%
\pgfpathmoveto{\pgfqpoint{4.670567in}{1.417667in}}%
\pgfpathlineto{\pgfqpoint{4.685042in}{1.417932in}}%
\pgfpathlineto{\pgfqpoint{4.699528in}{1.418291in}}%
\pgfpathlineto{\pgfqpoint{4.714024in}{1.418744in}}%
\pgfpathlineto{\pgfqpoint{4.728531in}{1.419292in}}%
\pgfpathlineto{\pgfqpoint{4.720477in}{1.406833in}}%
\pgfpathlineto{\pgfqpoint{4.712421in}{1.394529in}}%
\pgfpathlineto{\pgfqpoint{4.704362in}{1.382384in}}%
\pgfpathlineto{\pgfqpoint{4.696299in}{1.370404in}}%
\pgfpathlineto{\pgfqpoint{4.681791in}{1.370389in}}%
\pgfpathlineto{\pgfqpoint{4.667292in}{1.370468in}}%
\pgfpathlineto{\pgfqpoint{4.652805in}{1.370641in}}%
\pgfpathlineto{\pgfqpoint{4.638327in}{1.370907in}}%
\pgfpathlineto{\pgfqpoint{4.646392in}{1.382349in}}%
\pgfpathlineto{\pgfqpoint{4.654454in}{1.393960in}}%
\pgfpathlineto{\pgfqpoint{4.662512in}{1.405734in}}%
\pgfpathlineto{\pgfqpoint{4.670567in}{1.417667in}}%
\pgfpathclose%
\pgfusepath{fill}%
\end{pgfscope}%
\begin{pgfscope}%
\pgfpathrectangle{\pgfqpoint{1.150000in}{0.150000in}}{\pgfqpoint{5.700000in}{5.700000in}}%
\pgfusepath{clip}%
\pgfsetbuttcap%
\pgfsetroundjoin%
\definecolor{currentfill}{rgb}{0.130067,0.651384,0.521608}%
\pgfsetfillcolor{currentfill}%
\pgfsetfillopacity{0.700000}%
\pgfsetlinewidth{0.000000pt}%
\definecolor{currentstroke}{rgb}{0.000000,0.000000,0.000000}%
\pgfsetstrokecolor{currentstroke}%
\pgfsetdash{}{0pt}%
\pgfpathmoveto{\pgfqpoint{2.408496in}{2.791793in}}%
\pgfpathlineto{\pgfqpoint{2.422891in}{2.771382in}}%
\pgfpathlineto{\pgfqpoint{2.437279in}{2.751129in}}%
\pgfpathlineto{\pgfqpoint{2.451661in}{2.731035in}}%
\pgfpathlineto{\pgfqpoint{2.466035in}{2.711098in}}%
\pgfpathlineto{\pgfqpoint{2.456324in}{2.730761in}}%
\pgfpathlineto{\pgfqpoint{2.446579in}{2.750984in}}%
\pgfpathlineto{\pgfqpoint{2.436799in}{2.771773in}}%
\pgfpathlineto{\pgfqpoint{2.426985in}{2.793140in}}%
\pgfpathlineto{\pgfqpoint{2.412533in}{2.813840in}}%
\pgfpathlineto{\pgfqpoint{2.398074in}{2.834699in}}%
\pgfpathlineto{\pgfqpoint{2.383607in}{2.855717in}}%
\pgfpathlineto{\pgfqpoint{2.369133in}{2.876897in}}%
\pgfpathlineto{\pgfqpoint{2.379028in}{2.854753in}}%
\pgfpathlineto{\pgfqpoint{2.388886in}{2.833194in}}%
\pgfpathlineto{\pgfqpoint{2.398708in}{2.812211in}}%
\pgfpathlineto{\pgfqpoint{2.408496in}{2.791793in}}%
\pgfpathclose%
\pgfusepath{fill}%
\end{pgfscope}%
\begin{pgfscope}%
\pgfpathrectangle{\pgfqpoint{1.150000in}{0.150000in}}{\pgfqpoint{5.700000in}{5.700000in}}%
\pgfusepath{clip}%
\pgfsetbuttcap%
\pgfsetroundjoin%
\definecolor{currentfill}{rgb}{0.555484,0.840254,0.269281}%
\pgfsetfillcolor{currentfill}%
\pgfsetfillopacity{0.700000}%
\pgfsetlinewidth{0.000000pt}%
\definecolor{currentstroke}{rgb}{0.000000,0.000000,0.000000}%
\pgfsetstrokecolor{currentstroke}%
\pgfsetdash{}{0pt}%
\pgfpathmoveto{\pgfqpoint{1.984588in}{3.437146in}}%
\pgfpathlineto{\pgfqpoint{1.999211in}{3.411631in}}%
\pgfpathlineto{\pgfqpoint{2.013821in}{3.386325in}}%
\pgfpathlineto{\pgfqpoint{2.028419in}{3.361228in}}%
\pgfpathlineto{\pgfqpoint{2.043005in}{3.336336in}}%
\pgfpathlineto{\pgfqpoint{2.032788in}{3.359814in}}%
\pgfpathlineto{\pgfqpoint{2.022530in}{3.383879in}}%
\pgfpathlineto{\pgfqpoint{2.012231in}{3.408539in}}%
\pgfpathlineto{\pgfqpoint{2.001891in}{3.433803in}}%
\pgfpathlineto{\pgfqpoint{1.987214in}{3.459481in}}%
\pgfpathlineto{\pgfqpoint{1.972525in}{3.485367in}}%
\pgfpathlineto{\pgfqpoint{1.957824in}{3.511464in}}%
\pgfpathlineto{\pgfqpoint{1.943108in}{3.537772in}}%
\pgfpathlineto{\pgfqpoint{1.953542in}{3.511705in}}%
\pgfpathlineto{\pgfqpoint{1.963933in}{3.486252in}}%
\pgfpathlineto{\pgfqpoint{1.974281in}{3.461402in}}%
\pgfpathlineto{\pgfqpoint{1.984588in}{3.437146in}}%
\pgfpathclose%
\pgfusepath{fill}%
\end{pgfscope}%
\begin{pgfscope}%
\pgfpathrectangle{\pgfqpoint{1.150000in}{0.150000in}}{\pgfqpoint{5.700000in}{5.700000in}}%
\pgfusepath{clip}%
\pgfsetbuttcap%
\pgfsetroundjoin%
\definecolor{currentfill}{rgb}{0.282884,0.135920,0.453427}%
\pgfsetfillcolor{currentfill}%
\pgfsetfillopacity{0.700000}%
\pgfsetlinewidth{0.000000pt}%
\definecolor{currentstroke}{rgb}{0.000000,0.000000,0.000000}%
\pgfsetstrokecolor{currentstroke}%
\pgfsetdash{}{0pt}%
\pgfpathmoveto{\pgfqpoint{4.760714in}{1.470564in}}%
\pgfpathlineto{\pgfqpoint{4.775231in}{1.471721in}}%
\pgfpathlineto{\pgfqpoint{4.789760in}{1.472972in}}%
\pgfpathlineto{\pgfqpoint{4.804300in}{1.474317in}}%
\pgfpathlineto{\pgfqpoint{4.818851in}{1.475756in}}%
\pgfpathlineto{\pgfqpoint{4.810809in}{1.462224in}}%
\pgfpathlineto{\pgfqpoint{4.802764in}{1.448820in}}%
\pgfpathlineto{\pgfqpoint{4.794717in}{1.435551in}}%
\pgfpathlineto{\pgfqpoint{4.786667in}{1.422422in}}%
\pgfpathlineto{\pgfqpoint{4.772116in}{1.421498in}}%
\pgfpathlineto{\pgfqpoint{4.757577in}{1.420669in}}%
\pgfpathlineto{\pgfqpoint{4.743048in}{1.419933in}}%
\pgfpathlineto{\pgfqpoint{4.728531in}{1.419292in}}%
\pgfpathlineto{\pgfqpoint{4.736581in}{1.431899in}}%
\pgfpathlineto{\pgfqpoint{4.744628in}{1.444650in}}%
\pgfpathlineto{\pgfqpoint{4.752673in}{1.457540in}}%
\pgfpathlineto{\pgfqpoint{4.760714in}{1.470564in}}%
\pgfpathclose%
\pgfusepath{fill}%
\end{pgfscope}%
\begin{pgfscope}%
\pgfpathrectangle{\pgfqpoint{1.150000in}{0.150000in}}{\pgfqpoint{5.700000in}{5.700000in}}%
\pgfusepath{clip}%
\pgfsetbuttcap%
\pgfsetroundjoin%
\definecolor{currentfill}{rgb}{0.281924,0.089666,0.412415}%
\pgfsetfillcolor{currentfill}%
\pgfsetfillopacity{0.700000}%
\pgfsetlinewidth{0.000000pt}%
\definecolor{currentstroke}{rgb}{0.000000,0.000000,0.000000}%
\pgfsetstrokecolor{currentstroke}%
\pgfsetdash{}{0pt}%
\pgfpathmoveto{\pgfqpoint{4.580516in}{1.372913in}}%
\pgfpathlineto{\pgfqpoint{4.594954in}{1.372270in}}%
\pgfpathlineto{\pgfqpoint{4.609402in}{1.371722in}}%
\pgfpathlineto{\pgfqpoint{4.623859in}{1.371268in}}%
\pgfpathlineto{\pgfqpoint{4.638327in}{1.370907in}}%
\pgfpathlineto{\pgfqpoint{4.630258in}{1.359640in}}%
\pgfpathlineto{\pgfqpoint{4.622186in}{1.348552in}}%
\pgfpathlineto{\pgfqpoint{4.614111in}{1.337648in}}%
\pgfpathlineto{\pgfqpoint{4.606032in}{1.326935in}}%
\pgfpathlineto{\pgfqpoint{4.591560in}{1.327845in}}%
\pgfpathlineto{\pgfqpoint{4.577098in}{1.328849in}}%
\pgfpathlineto{\pgfqpoint{4.562645in}{1.329946in}}%
\pgfpathlineto{\pgfqpoint{4.548202in}{1.331137in}}%
\pgfpathlineto{\pgfqpoint{4.556286in}{1.341294in}}%
\pgfpathlineto{\pgfqpoint{4.564367in}{1.351646in}}%
\pgfpathlineto{\pgfqpoint{4.572443in}{1.362187in}}%
\pgfpathlineto{\pgfqpoint{4.580516in}{1.372913in}}%
\pgfpathclose%
\pgfusepath{fill}%
\end{pgfscope}%
\begin{pgfscope}%
\pgfpathrectangle{\pgfqpoint{1.150000in}{0.150000in}}{\pgfqpoint{5.700000in}{5.700000in}}%
\pgfusepath{clip}%
\pgfsetbuttcap%
\pgfsetroundjoin%
\definecolor{currentfill}{rgb}{0.282910,0.105393,0.426902}%
\pgfsetfillcolor{currentfill}%
\pgfsetfillopacity{0.700000}%
\pgfsetlinewidth{0.000000pt}%
\definecolor{currentstroke}{rgb}{0.000000,0.000000,0.000000}%
\pgfsetstrokecolor{currentstroke}%
\pgfsetdash{}{0pt}%
\pgfpathmoveto{\pgfqpoint{3.901609in}{1.391111in}}%
\pgfpathlineto{\pgfqpoint{3.915857in}{1.384115in}}%
\pgfpathlineto{\pgfqpoint{3.930110in}{1.377217in}}%
\pgfpathlineto{\pgfqpoint{3.944368in}{1.370417in}}%
\pgfpathlineto{\pgfqpoint{3.958631in}{1.363714in}}%
\pgfpathlineto{\pgfqpoint{3.950314in}{1.362827in}}%
\pgfpathlineto{\pgfqpoint{3.941987in}{1.362283in}}%
\pgfpathlineto{\pgfqpoint{3.933650in}{1.362087in}}%
\pgfpathlineto{\pgfqpoint{3.925302in}{1.362248in}}%
\pgfpathlineto{\pgfqpoint{3.911012in}{1.369593in}}%
\pgfpathlineto{\pgfqpoint{3.896726in}{1.377036in}}%
\pgfpathlineto{\pgfqpoint{3.882446in}{1.384578in}}%
\pgfpathlineto{\pgfqpoint{3.868169in}{1.392217in}}%
\pgfpathlineto{\pgfqpoint{3.876545in}{1.391406in}}%
\pgfpathlineto{\pgfqpoint{3.884910in}{1.390956in}}%
\pgfpathlineto{\pgfqpoint{3.893265in}{1.390860in}}%
\pgfpathlineto{\pgfqpoint{3.901609in}{1.391111in}}%
\pgfpathclose%
\pgfusepath{fill}%
\end{pgfscope}%
\begin{pgfscope}%
\pgfpathrectangle{\pgfqpoint{1.150000in}{0.150000in}}{\pgfqpoint{5.700000in}{5.700000in}}%
\pgfusepath{clip}%
\pgfsetbuttcap%
\pgfsetroundjoin%
\definecolor{currentfill}{rgb}{0.239346,0.300855,0.540844}%
\pgfsetfillcolor{currentfill}%
\pgfsetfillopacity{0.700000}%
\pgfsetlinewidth{0.000000pt}%
\definecolor{currentstroke}{rgb}{0.000000,0.000000,0.000000}%
\pgfsetstrokecolor{currentstroke}%
\pgfsetdash{}{0pt}%
\pgfpathmoveto{\pgfqpoint{3.264658in}{1.815236in}}%
\pgfpathlineto{\pgfqpoint{3.278872in}{1.802754in}}%
\pgfpathlineto{\pgfqpoint{3.293087in}{1.790385in}}%
\pgfpathlineto{\pgfqpoint{3.307303in}{1.778126in}}%
\pgfpathlineto{\pgfqpoint{3.321519in}{1.765978in}}%
\pgfpathlineto{\pgfqpoint{3.312725in}{1.774868in}}%
\pgfpathlineto{\pgfqpoint{3.303911in}{1.784215in}}%
\pgfpathlineto{\pgfqpoint{3.295077in}{1.794026in}}%
\pgfpathlineto{\pgfqpoint{3.286224in}{1.804310in}}%
\pgfpathlineto{\pgfqpoint{3.271958in}{1.817161in}}%
\pgfpathlineto{\pgfqpoint{3.257693in}{1.830124in}}%
\pgfpathlineto{\pgfqpoint{3.243428in}{1.843198in}}%
\pgfpathlineto{\pgfqpoint{3.229163in}{1.856385in}}%
\pgfpathlineto{\pgfqpoint{3.238068in}{1.845387in}}%
\pgfpathlineto{\pgfqpoint{3.246951in}{1.834869in}}%
\pgfpathlineto{\pgfqpoint{3.255815in}{1.824821in}}%
\pgfpathlineto{\pgfqpoint{3.264658in}{1.815236in}}%
\pgfpathclose%
\pgfusepath{fill}%
\end{pgfscope}%
\begin{pgfscope}%
\pgfpathrectangle{\pgfqpoint{1.150000in}{0.150000in}}{\pgfqpoint{5.700000in}{5.700000in}}%
\pgfusepath{clip}%
\pgfsetbuttcap%
\pgfsetroundjoin%
\definecolor{currentfill}{rgb}{0.257322,0.256130,0.526563}%
\pgfsetfillcolor{currentfill}%
\pgfsetfillopacity{0.700000}%
\pgfsetlinewidth{0.000000pt}%
\definecolor{currentstroke}{rgb}{0.000000,0.000000,0.000000}%
\pgfsetstrokecolor{currentstroke}%
\pgfsetdash{}{0pt}%
\pgfpathmoveto{\pgfqpoint{5.064094in}{1.736118in}}%
\pgfpathlineto{\pgfqpoint{5.078773in}{1.740307in}}%
\pgfpathlineto{\pgfqpoint{5.093466in}{1.744592in}}%
\pgfpathlineto{\pgfqpoint{5.108172in}{1.748974in}}%
\pgfpathlineto{\pgfqpoint{5.122892in}{1.753452in}}%
\pgfpathlineto{\pgfqpoint{5.114880in}{1.737156in}}%
\pgfpathlineto{\pgfqpoint{5.106864in}{1.720904in}}%
\pgfpathlineto{\pgfqpoint{5.098846in}{1.704701in}}%
\pgfpathlineto{\pgfqpoint{5.090825in}{1.688549in}}%
\pgfpathlineto{\pgfqpoint{5.076112in}{1.684521in}}%
\pgfpathlineto{\pgfqpoint{5.061413in}{1.680588in}}%
\pgfpathlineto{\pgfqpoint{5.046726in}{1.676751in}}%
\pgfpathlineto{\pgfqpoint{5.032054in}{1.673011in}}%
\pgfpathlineto{\pgfqpoint{5.040068in}{1.688707in}}%
\pgfpathlineto{\pgfqpoint{5.048080in}{1.704459in}}%
\pgfpathlineto{\pgfqpoint{5.056088in}{1.720265in}}%
\pgfpathlineto{\pgfqpoint{5.064094in}{1.736118in}}%
\pgfpathclose%
\pgfusepath{fill}%
\end{pgfscope}%
\begin{pgfscope}%
\pgfpathrectangle{\pgfqpoint{1.150000in}{0.150000in}}{\pgfqpoint{5.700000in}{5.700000in}}%
\pgfusepath{clip}%
\pgfsetbuttcap%
\pgfsetroundjoin%
\definecolor{currentfill}{rgb}{0.280255,0.165693,0.476498}%
\pgfsetfillcolor{currentfill}%
\pgfsetfillopacity{0.700000}%
\pgfsetlinewidth{0.000000pt}%
\definecolor{currentstroke}{rgb}{0.000000,0.000000,0.000000}%
\pgfsetstrokecolor{currentstroke}%
\pgfsetdash{}{0pt}%
\pgfpathmoveto{\pgfqpoint{4.850990in}{1.531087in}}%
\pgfpathlineto{\pgfqpoint{4.865555in}{1.533119in}}%
\pgfpathlineto{\pgfqpoint{4.880131in}{1.535246in}}%
\pgfpathlineto{\pgfqpoint{4.894720in}{1.537468in}}%
\pgfpathlineto{\pgfqpoint{4.909320in}{1.539785in}}%
\pgfpathlineto{\pgfqpoint{4.901287in}{1.525290in}}%
\pgfpathlineto{\pgfqpoint{4.893250in}{1.510901in}}%
\pgfpathlineto{\pgfqpoint{4.885212in}{1.496622in}}%
\pgfpathlineto{\pgfqpoint{4.877171in}{1.482459in}}%
\pgfpathlineto{\pgfqpoint{4.862573in}{1.480642in}}%
\pgfpathlineto{\pgfqpoint{4.847987in}{1.478919in}}%
\pgfpathlineto{\pgfqpoint{4.833413in}{1.477290in}}%
\pgfpathlineto{\pgfqpoint{4.818851in}{1.475756in}}%
\pgfpathlineto{\pgfqpoint{4.826890in}{1.489414in}}%
\pgfpathlineto{\pgfqpoint{4.834926in}{1.503191in}}%
\pgfpathlineto{\pgfqpoint{4.842960in}{1.517084in}}%
\pgfpathlineto{\pgfqpoint{4.850990in}{1.531087in}}%
\pgfpathclose%
\pgfusepath{fill}%
\end{pgfscope}%
\begin{pgfscope}%
\pgfpathrectangle{\pgfqpoint{1.150000in}{0.150000in}}{\pgfqpoint{5.700000in}{5.700000in}}%
\pgfusepath{clip}%
\pgfsetbuttcap%
\pgfsetroundjoin%
\definecolor{currentfill}{rgb}{0.280267,0.073417,0.397163}%
\pgfsetfillcolor{currentfill}%
\pgfsetfillopacity{0.700000}%
\pgfsetlinewidth{0.000000pt}%
\definecolor{currentstroke}{rgb}{0.000000,0.000000,0.000000}%
\pgfsetstrokecolor{currentstroke}%
\pgfsetdash{}{0pt}%
\pgfpathmoveto{\pgfqpoint{4.490522in}{1.336839in}}%
\pgfpathlineto{\pgfqpoint{4.504928in}{1.335273in}}%
\pgfpathlineto{\pgfqpoint{4.519344in}{1.333800in}}%
\pgfpathlineto{\pgfqpoint{4.533768in}{1.332421in}}%
\pgfpathlineto{\pgfqpoint{4.548202in}{1.331137in}}%
\pgfpathlineto{\pgfqpoint{4.540113in}{1.321180in}}%
\pgfpathlineto{\pgfqpoint{4.532021in}{1.311429in}}%
\pgfpathlineto{\pgfqpoint{4.523925in}{1.301889in}}%
\pgfpathlineto{\pgfqpoint{4.515825in}{1.292566in}}%
\pgfpathlineto{\pgfqpoint{4.501384in}{1.294418in}}%
\pgfpathlineto{\pgfqpoint{4.486952in}{1.296363in}}%
\pgfpathlineto{\pgfqpoint{4.472529in}{1.298401in}}%
\pgfpathlineto{\pgfqpoint{4.458114in}{1.300534in}}%
\pgfpathlineto{\pgfqpoint{4.466223in}{1.309283in}}%
\pgfpathlineto{\pgfqpoint{4.474327in}{1.318254in}}%
\pgfpathlineto{\pgfqpoint{4.482427in}{1.327441in}}%
\pgfpathlineto{\pgfqpoint{4.490522in}{1.336839in}}%
\pgfpathclose%
\pgfusepath{fill}%
\end{pgfscope}%
\begin{pgfscope}%
\pgfpathrectangle{\pgfqpoint{1.150000in}{0.150000in}}{\pgfqpoint{5.700000in}{5.700000in}}%
\pgfusepath{clip}%
\pgfsetbuttcap%
\pgfsetroundjoin%
\definecolor{currentfill}{rgb}{0.157851,0.683765,0.501686}%
\pgfsetfillcolor{currentfill}%
\pgfsetfillopacity{0.700000}%
\pgfsetlinewidth{0.000000pt}%
\definecolor{currentstroke}{rgb}{0.000000,0.000000,0.000000}%
\pgfsetstrokecolor{currentstroke}%
\pgfsetdash{}{0pt}%
\pgfpathmoveto{\pgfqpoint{2.350843in}{2.875061in}}%
\pgfpathlineto{\pgfqpoint{2.365267in}{2.853999in}}%
\pgfpathlineto{\pgfqpoint{2.379684in}{2.833101in}}%
\pgfpathlineto{\pgfqpoint{2.394094in}{2.812366in}}%
\pgfpathlineto{\pgfqpoint{2.408496in}{2.791793in}}%
\pgfpathlineto{\pgfqpoint{2.398708in}{2.812211in}}%
\pgfpathlineto{\pgfqpoint{2.388886in}{2.833194in}}%
\pgfpathlineto{\pgfqpoint{2.379028in}{2.854753in}}%
\pgfpathlineto{\pgfqpoint{2.369133in}{2.876897in}}%
\pgfpathlineto{\pgfqpoint{2.354652in}{2.898238in}}%
\pgfpathlineto{\pgfqpoint{2.340162in}{2.919743in}}%
\pgfpathlineto{\pgfqpoint{2.325665in}{2.941413in}}%
\pgfpathlineto{\pgfqpoint{2.311159in}{2.963250in}}%
\pgfpathlineto{\pgfqpoint{2.321135in}{2.940323in}}%
\pgfpathlineto{\pgfqpoint{2.331074in}{2.917989in}}%
\pgfpathlineto{\pgfqpoint{2.340977in}{2.896238in}}%
\pgfpathlineto{\pgfqpoint{2.350843in}{2.875061in}}%
\pgfpathclose%
\pgfusepath{fill}%
\end{pgfscope}%
\begin{pgfscope}%
\pgfpathrectangle{\pgfqpoint{1.150000in}{0.150000in}}{\pgfqpoint{5.700000in}{5.700000in}}%
\pgfusepath{clip}%
\pgfsetbuttcap%
\pgfsetroundjoin%
\definecolor{currentfill}{rgb}{0.278791,0.062145,0.386592}%
\pgfsetfillcolor{currentfill}%
\pgfsetfillopacity{0.700000}%
\pgfsetlinewidth{0.000000pt}%
\definecolor{currentstroke}{rgb}{0.000000,0.000000,0.000000}%
\pgfsetstrokecolor{currentstroke}%
\pgfsetdash{}{0pt}%
\pgfpathmoveto{\pgfqpoint{4.253172in}{1.307802in}}%
\pgfpathlineto{\pgfqpoint{4.267502in}{1.303955in}}%
\pgfpathlineto{\pgfqpoint{4.281839in}{1.300203in}}%
\pgfpathlineto{\pgfqpoint{4.296183in}{1.296545in}}%
\pgfpathlineto{\pgfqpoint{4.310535in}{1.292982in}}%
\pgfpathlineto{\pgfqpoint{4.302378in}{1.286619in}}%
\pgfpathlineto{\pgfqpoint{4.294215in}{1.280522in}}%
\pgfpathlineto{\pgfqpoint{4.286046in}{1.274698in}}%
\pgfpathlineto{\pgfqpoint{4.277870in}{1.269152in}}%
\pgfpathlineto{\pgfqpoint{4.263503in}{1.273317in}}%
\pgfpathlineto{\pgfqpoint{4.249144in}{1.277577in}}%
\pgfpathlineto{\pgfqpoint{4.234791in}{1.281931in}}%
\pgfpathlineto{\pgfqpoint{4.220446in}{1.286380in}}%
\pgfpathlineto{\pgfqpoint{4.228637in}{1.291317in}}%
\pgfpathlineto{\pgfqpoint{4.236822in}{1.296537in}}%
\pgfpathlineto{\pgfqpoint{4.245000in}{1.302034in}}%
\pgfpathlineto{\pgfqpoint{4.253172in}{1.307802in}}%
\pgfpathclose%
\pgfusepath{fill}%
\end{pgfscope}%
\begin{pgfscope}%
\pgfpathrectangle{\pgfqpoint{1.150000in}{0.150000in}}{\pgfqpoint{5.700000in}{5.700000in}}%
\pgfusepath{clip}%
\pgfsetbuttcap%
\pgfsetroundjoin%
\definecolor{currentfill}{rgb}{0.280267,0.073417,0.397163}%
\pgfsetfillcolor{currentfill}%
\pgfsetfillopacity{0.700000}%
\pgfsetlinewidth{0.000000pt}%
\definecolor{currentstroke}{rgb}{0.000000,0.000000,0.000000}%
\pgfsetstrokecolor{currentstroke}%
\pgfsetdash{}{0pt}%
\pgfpathmoveto{\pgfqpoint{4.105925in}{1.325396in}}%
\pgfpathlineto{\pgfqpoint{4.120218in}{1.320185in}}%
\pgfpathlineto{\pgfqpoint{4.134516in}{1.315070in}}%
\pgfpathlineto{\pgfqpoint{4.148821in}{1.310050in}}%
\pgfpathlineto{\pgfqpoint{4.163133in}{1.305126in}}%
\pgfpathlineto{\pgfqpoint{4.154916in}{1.301093in}}%
\pgfpathlineto{\pgfqpoint{4.146692in}{1.297361in}}%
\pgfpathlineto{\pgfqpoint{4.138461in}{1.293937in}}%
\pgfpathlineto{\pgfqpoint{4.130222in}{1.290826in}}%
\pgfpathlineto{\pgfqpoint{4.115891in}{1.296372in}}%
\pgfpathlineto{\pgfqpoint{4.101565in}{1.302013in}}%
\pgfpathlineto{\pgfqpoint{4.087246in}{1.307750in}}%
\pgfpathlineto{\pgfqpoint{4.072933in}{1.313583in}}%
\pgfpathlineto{\pgfqpoint{4.081193in}{1.316065in}}%
\pgfpathlineto{\pgfqpoint{4.089445in}{1.318865in}}%
\pgfpathlineto{\pgfqpoint{4.097689in}{1.321978in}}%
\pgfpathlineto{\pgfqpoint{4.105925in}{1.325396in}}%
\pgfpathclose%
\pgfusepath{fill}%
\end{pgfscope}%
\begin{pgfscope}%
\pgfpathrectangle{\pgfqpoint{1.150000in}{0.150000in}}{\pgfqpoint{5.700000in}{5.700000in}}%
\pgfusepath{clip}%
\pgfsetbuttcap%
\pgfsetroundjoin%
\definecolor{currentfill}{rgb}{0.248629,0.278775,0.534556}%
\pgfsetfillcolor{currentfill}%
\pgfsetfillopacity{0.700000}%
\pgfsetlinewidth{0.000000pt}%
\definecolor{currentstroke}{rgb}{0.000000,0.000000,0.000000}%
\pgfsetstrokecolor{currentstroke}%
\pgfsetdash{}{0pt}%
\pgfpathmoveto{\pgfqpoint{3.321519in}{1.765978in}}%
\pgfpathlineto{\pgfqpoint{3.335736in}{1.753941in}}%
\pgfpathlineto{\pgfqpoint{3.349954in}{1.742013in}}%
\pgfpathlineto{\pgfqpoint{3.364173in}{1.730196in}}%
\pgfpathlineto{\pgfqpoint{3.378393in}{1.718487in}}%
\pgfpathlineto{\pgfqpoint{3.369645in}{1.726684in}}%
\pgfpathlineto{\pgfqpoint{3.360879in}{1.735332in}}%
\pgfpathlineto{\pgfqpoint{3.352094in}{1.744439in}}%
\pgfpathlineto{\pgfqpoint{3.343290in}{1.754012in}}%
\pgfpathlineto{\pgfqpoint{3.329023in}{1.766422in}}%
\pgfpathlineto{\pgfqpoint{3.314756in}{1.778941in}}%
\pgfpathlineto{\pgfqpoint{3.300490in}{1.791571in}}%
\pgfpathlineto{\pgfqpoint{3.286224in}{1.804310in}}%
\pgfpathlineto{\pgfqpoint{3.295077in}{1.794026in}}%
\pgfpathlineto{\pgfqpoint{3.303911in}{1.784215in}}%
\pgfpathlineto{\pgfqpoint{3.312725in}{1.774868in}}%
\pgfpathlineto{\pgfqpoint{3.321519in}{1.765978in}}%
\pgfpathclose%
\pgfusepath{fill}%
\end{pgfscope}%
\begin{pgfscope}%
\pgfpathrectangle{\pgfqpoint{1.150000in}{0.150000in}}{\pgfqpoint{5.700000in}{5.700000in}}%
\pgfusepath{clip}%
\pgfsetbuttcap%
\pgfsetroundjoin%
\definecolor{currentfill}{rgb}{0.281412,0.155834,0.469201}%
\pgfsetfillcolor{currentfill}%
\pgfsetfillopacity{0.700000}%
\pgfsetlinewidth{0.000000pt}%
\definecolor{currentstroke}{rgb}{0.000000,0.000000,0.000000}%
\pgfsetstrokecolor{currentstroke}%
\pgfsetdash{}{0pt}%
\pgfpathmoveto{\pgfqpoint{3.697178in}{1.491632in}}%
\pgfpathlineto{\pgfqpoint{3.711406in}{1.482796in}}%
\pgfpathlineto{\pgfqpoint{3.725638in}{1.474061in}}%
\pgfpathlineto{\pgfqpoint{3.739874in}{1.465427in}}%
\pgfpathlineto{\pgfqpoint{3.754113in}{1.456894in}}%
\pgfpathlineto{\pgfqpoint{3.745663in}{1.459389in}}%
\pgfpathlineto{\pgfqpoint{3.737200in}{1.462270in}}%
\pgfpathlineto{\pgfqpoint{3.728725in}{1.465545in}}%
\pgfpathlineto{\pgfqpoint{3.720236in}{1.469220in}}%
\pgfpathlineto{\pgfqpoint{3.705962in}{1.478420in}}%
\pgfpathlineto{\pgfqpoint{3.691692in}{1.487722in}}%
\pgfpathlineto{\pgfqpoint{3.677425in}{1.497124in}}%
\pgfpathlineto{\pgfqpoint{3.663161in}{1.506628in}}%
\pgfpathlineto{\pgfqpoint{3.671686in}{1.502277in}}%
\pgfpathlineto{\pgfqpoint{3.680197in}{1.498332in}}%
\pgfpathlineto{\pgfqpoint{3.688694in}{1.494786in}}%
\pgfpathlineto{\pgfqpoint{3.697178in}{1.491632in}}%
\pgfpathclose%
\pgfusepath{fill}%
\end{pgfscope}%
\begin{pgfscope}%
\pgfpathrectangle{\pgfqpoint{1.150000in}{0.150000in}}{\pgfqpoint{5.700000in}{5.700000in}}%
\pgfusepath{clip}%
\pgfsetbuttcap%
\pgfsetroundjoin%
\definecolor{currentfill}{rgb}{0.241237,0.296485,0.539709}%
\pgfsetfillcolor{currentfill}%
\pgfsetfillopacity{0.700000}%
\pgfsetlinewidth{0.000000pt}%
\definecolor{currentstroke}{rgb}{0.000000,0.000000,0.000000}%
\pgfsetstrokecolor{currentstroke}%
\pgfsetdash{}{0pt}%
\pgfpathmoveto{\pgfqpoint{5.154914in}{1.818988in}}%
\pgfpathlineto{\pgfqpoint{5.169656in}{1.823994in}}%
\pgfpathlineto{\pgfqpoint{5.184412in}{1.829097in}}%
\pgfpathlineto{\pgfqpoint{5.199181in}{1.834297in}}%
\pgfpathlineto{\pgfqpoint{5.191174in}{1.817548in}}%
\pgfpathlineto{\pgfqpoint{5.183164in}{1.800824in}}%
\pgfpathlineto{\pgfqpoint{5.175151in}{1.784127in}}%
\pgfpathlineto{\pgfqpoint{5.167135in}{1.767463in}}%
\pgfpathlineto{\pgfqpoint{5.152374in}{1.762696in}}%
\pgfpathlineto{\pgfqpoint{5.137626in}{1.758026in}}%
\pgfpathlineto{\pgfqpoint{5.122892in}{1.753452in}}%
\pgfpathlineto{\pgfqpoint{5.130902in}{1.769787in}}%
\pgfpathlineto{\pgfqpoint{5.138909in}{1.786157in}}%
\pgfpathlineto{\pgfqpoint{5.146913in}{1.802559in}}%
\pgfpathlineto{\pgfqpoint{5.154914in}{1.818988in}}%
\pgfpathclose%
\pgfusepath{fill}%
\end{pgfscope}%
\begin{pgfscope}%
\pgfpathrectangle{\pgfqpoint{1.150000in}{0.150000in}}{\pgfqpoint{5.700000in}{5.700000in}}%
\pgfusepath{clip}%
\pgfsetbuttcap%
\pgfsetroundjoin%
\definecolor{currentfill}{rgb}{0.274128,0.199721,0.498911}%
\pgfsetfillcolor{currentfill}%
\pgfsetfillopacity{0.700000}%
\pgfsetlinewidth{0.000000pt}%
\definecolor{currentstroke}{rgb}{0.000000,0.000000,0.000000}%
\pgfsetstrokecolor{currentstroke}%
\pgfsetdash{}{0pt}%
\pgfpathmoveto{\pgfqpoint{4.941428in}{1.598731in}}%
\pgfpathlineto{\pgfqpoint{4.956044in}{1.601625in}}%
\pgfpathlineto{\pgfqpoint{4.970673in}{1.604614in}}%
\pgfpathlineto{\pgfqpoint{4.985315in}{1.607698in}}%
\pgfpathlineto{\pgfqpoint{4.999969in}{1.610878in}}%
\pgfpathlineto{\pgfqpoint{4.991942in}{1.595530in}}%
\pgfpathlineto{\pgfqpoint{4.983911in}{1.580264in}}%
\pgfpathlineto{\pgfqpoint{4.975879in}{1.565086in}}%
\pgfpathlineto{\pgfqpoint{4.967844in}{1.550000in}}%
\pgfpathlineto{\pgfqpoint{4.953194in}{1.547304in}}%
\pgfpathlineto{\pgfqpoint{4.938557in}{1.544703in}}%
\pgfpathlineto{\pgfqpoint{4.923933in}{1.542197in}}%
\pgfpathlineto{\pgfqpoint{4.909320in}{1.539785in}}%
\pgfpathlineto{\pgfqpoint{4.917351in}{1.554382in}}%
\pgfpathlineto{\pgfqpoint{4.925379in}{1.569074in}}%
\pgfpathlineto{\pgfqpoint{4.933405in}{1.583859in}}%
\pgfpathlineto{\pgfqpoint{4.941428in}{1.598731in}}%
\pgfpathclose%
\pgfusepath{fill}%
\end{pgfscope}%
\begin{pgfscope}%
\pgfpathrectangle{\pgfqpoint{1.150000in}{0.150000in}}{\pgfqpoint{5.700000in}{5.700000in}}%
\pgfusepath{clip}%
\pgfsetbuttcap%
\pgfsetroundjoin%
\definecolor{currentfill}{rgb}{0.196571,0.711827,0.479221}%
\pgfsetfillcolor{currentfill}%
\pgfsetfillopacity{0.700000}%
\pgfsetlinewidth{0.000000pt}%
\definecolor{currentstroke}{rgb}{0.000000,0.000000,0.000000}%
\pgfsetstrokecolor{currentstroke}%
\pgfsetdash{}{0pt}%
\pgfpathmoveto{\pgfqpoint{2.293065in}{2.960985in}}%
\pgfpathlineto{\pgfqpoint{2.307521in}{2.939251in}}%
\pgfpathlineto{\pgfqpoint{2.321970in}{2.917686in}}%
\pgfpathlineto{\pgfqpoint{2.336410in}{2.896290in}}%
\pgfpathlineto{\pgfqpoint{2.350843in}{2.875061in}}%
\pgfpathlineto{\pgfqpoint{2.340977in}{2.896238in}}%
\pgfpathlineto{\pgfqpoint{2.331074in}{2.917989in}}%
\pgfpathlineto{\pgfqpoint{2.321135in}{2.940323in}}%
\pgfpathlineto{\pgfqpoint{2.311159in}{2.963250in}}%
\pgfpathlineto{\pgfqpoint{2.296645in}{2.985253in}}%
\pgfpathlineto{\pgfqpoint{2.282123in}{3.007426in}}%
\pgfpathlineto{\pgfqpoint{2.267591in}{3.029769in}}%
\pgfpathlineto{\pgfqpoint{2.253051in}{3.052284in}}%
\pgfpathlineto{\pgfqpoint{2.263112in}{3.028568in}}%
\pgfpathlineto{\pgfqpoint{2.273134in}{3.005453in}}%
\pgfpathlineto{\pgfqpoint{2.283118in}{2.982928in}}%
\pgfpathlineto{\pgfqpoint{2.293065in}{2.960985in}}%
\pgfpathclose%
\pgfusepath{fill}%
\end{pgfscope}%
\begin{pgfscope}%
\pgfpathrectangle{\pgfqpoint{1.150000in}{0.150000in}}{\pgfqpoint{5.700000in}{5.700000in}}%
\pgfusepath{clip}%
\pgfsetbuttcap%
\pgfsetroundjoin%
\definecolor{currentfill}{rgb}{0.668054,0.861999,0.196293}%
\pgfsetfillcolor{currentfill}%
\pgfsetfillopacity{0.700000}%
\pgfsetlinewidth{0.000000pt}%
\definecolor{currentstroke}{rgb}{0.000000,0.000000,0.000000}%
\pgfsetstrokecolor{currentstroke}%
\pgfsetdash{}{0pt}%
\pgfpathmoveto{\pgfqpoint{1.925964in}{3.541348in}}%
\pgfpathlineto{\pgfqpoint{1.940640in}{3.514972in}}%
\pgfpathlineto{\pgfqpoint{1.955303in}{3.488814in}}%
\pgfpathlineto{\pgfqpoint{1.969952in}{3.462873in}}%
\pgfpathlineto{\pgfqpoint{1.984588in}{3.437146in}}%
\pgfpathlineto{\pgfqpoint{1.974281in}{3.461402in}}%
\pgfpathlineto{\pgfqpoint{1.963933in}{3.486252in}}%
\pgfpathlineto{\pgfqpoint{1.953542in}{3.511705in}}%
\pgfpathlineto{\pgfqpoint{1.943108in}{3.537772in}}%
\pgfpathlineto{\pgfqpoint{1.928380in}{3.564294in}}%
\pgfpathlineto{\pgfqpoint{1.913637in}{3.591032in}}%
\pgfpathlineto{\pgfqpoint{1.898881in}{3.617989in}}%
\pgfpathlineto{\pgfqpoint{1.884110in}{3.645167in}}%
\pgfpathlineto{\pgfqpoint{1.894640in}{3.618289in}}%
\pgfpathlineto{\pgfqpoint{1.905125in}{3.592033in}}%
\pgfpathlineto{\pgfqpoint{1.915566in}{3.566389in}}%
\pgfpathlineto{\pgfqpoint{1.925964in}{3.541348in}}%
\pgfpathclose%
\pgfusepath{fill}%
\end{pgfscope}%
\begin{pgfscope}%
\pgfpathrectangle{\pgfqpoint{1.150000in}{0.150000in}}{\pgfqpoint{5.700000in}{5.700000in}}%
\pgfusepath{clip}%
\pgfsetbuttcap%
\pgfsetroundjoin%
\definecolor{currentfill}{rgb}{0.278791,0.062145,0.386592}%
\pgfsetfillcolor{currentfill}%
\pgfsetfillopacity{0.700000}%
\pgfsetlinewidth{0.000000pt}%
\definecolor{currentstroke}{rgb}{0.000000,0.000000,0.000000}%
\pgfsetstrokecolor{currentstroke}%
\pgfsetdash{}{0pt}%
\pgfpathmoveto{\pgfqpoint{4.400544in}{1.310003in}}%
\pgfpathlineto{\pgfqpoint{4.414924in}{1.307494in}}%
\pgfpathlineto{\pgfqpoint{4.429312in}{1.305080in}}%
\pgfpathlineto{\pgfqpoint{4.443709in}{1.302760in}}%
\pgfpathlineto{\pgfqpoint{4.458114in}{1.300534in}}%
\pgfpathlineto{\pgfqpoint{4.450001in}{1.292012in}}%
\pgfpathlineto{\pgfqpoint{4.441884in}{1.283723in}}%
\pgfpathlineto{\pgfqpoint{4.433761in}{1.275673in}}%
\pgfpathlineto{\pgfqpoint{4.425634in}{1.267867in}}%
\pgfpathlineto{\pgfqpoint{4.411218in}{1.270678in}}%
\pgfpathlineto{\pgfqpoint{4.396811in}{1.273582in}}%
\pgfpathlineto{\pgfqpoint{4.382411in}{1.276581in}}%
\pgfpathlineto{\pgfqpoint{4.368020in}{1.279673in}}%
\pgfpathlineto{\pgfqpoint{4.376159in}{1.286887in}}%
\pgfpathlineto{\pgfqpoint{4.384292in}{1.294350in}}%
\pgfpathlineto{\pgfqpoint{4.392421in}{1.302058in}}%
\pgfpathlineto{\pgfqpoint{4.400544in}{1.310003in}}%
\pgfpathclose%
\pgfusepath{fill}%
\end{pgfscope}%
\begin{pgfscope}%
\pgfpathrectangle{\pgfqpoint{1.150000in}{0.150000in}}{\pgfqpoint{5.700000in}{5.700000in}}%
\pgfusepath{clip}%
\pgfsetbuttcap%
\pgfsetroundjoin%
\definecolor{currentfill}{rgb}{0.253935,0.265254,0.529983}%
\pgfsetfillcolor{currentfill}%
\pgfsetfillopacity{0.700000}%
\pgfsetlinewidth{0.000000pt}%
\definecolor{currentstroke}{rgb}{0.000000,0.000000,0.000000}%
\pgfsetstrokecolor{currentstroke}%
\pgfsetdash{}{0pt}%
\pgfpathmoveto{\pgfqpoint{3.378393in}{1.718487in}}%
\pgfpathlineto{\pgfqpoint{3.392614in}{1.706887in}}%
\pgfpathlineto{\pgfqpoint{3.406837in}{1.695395in}}%
\pgfpathlineto{\pgfqpoint{3.421060in}{1.684011in}}%
\pgfpathlineto{\pgfqpoint{3.435286in}{1.672734in}}%
\pgfpathlineto{\pgfqpoint{3.426583in}{1.680241in}}%
\pgfpathlineto{\pgfqpoint{3.417863in}{1.688193in}}%
\pgfpathlineto{\pgfqpoint{3.409124in}{1.696597in}}%
\pgfpathlineto{\pgfqpoint{3.400368in}{1.705462in}}%
\pgfpathlineto{\pgfqpoint{3.386097in}{1.717437in}}%
\pgfpathlineto{\pgfqpoint{3.371827in}{1.729520in}}%
\pgfpathlineto{\pgfqpoint{3.357558in}{1.741712in}}%
\pgfpathlineto{\pgfqpoint{3.343290in}{1.754012in}}%
\pgfpathlineto{\pgfqpoint{3.352094in}{1.744439in}}%
\pgfpathlineto{\pgfqpoint{3.360879in}{1.735332in}}%
\pgfpathlineto{\pgfqpoint{3.369645in}{1.726684in}}%
\pgfpathlineto{\pgfqpoint{3.378393in}{1.718487in}}%
\pgfpathclose%
\pgfusepath{fill}%
\end{pgfscope}%
\begin{pgfscope}%
\pgfpathrectangle{\pgfqpoint{1.150000in}{0.150000in}}{\pgfqpoint{5.700000in}{5.700000in}}%
\pgfusepath{clip}%
\pgfsetbuttcap%
\pgfsetroundjoin%
\definecolor{currentfill}{rgb}{0.282327,0.094955,0.417331}%
\pgfsetfillcolor{currentfill}%
\pgfsetfillopacity{0.700000}%
\pgfsetlinewidth{0.000000pt}%
\definecolor{currentstroke}{rgb}{0.000000,0.000000,0.000000}%
\pgfsetstrokecolor{currentstroke}%
\pgfsetdash{}{0pt}%
\pgfpathmoveto{\pgfqpoint{3.958631in}{1.363714in}}%
\pgfpathlineto{\pgfqpoint{3.972900in}{1.357109in}}%
\pgfpathlineto{\pgfqpoint{3.987174in}{1.350601in}}%
\pgfpathlineto{\pgfqpoint{4.001453in}{1.344190in}}%
\pgfpathlineto{\pgfqpoint{4.015738in}{1.337876in}}%
\pgfpathlineto{\pgfqpoint{4.007445in}{1.336354in}}%
\pgfpathlineto{\pgfqpoint{3.999144in}{1.335169in}}%
\pgfpathlineto{\pgfqpoint{3.990833in}{1.334329in}}%
\pgfpathlineto{\pgfqpoint{3.982512in}{1.333839in}}%
\pgfpathlineto{\pgfqpoint{3.968202in}{1.340795in}}%
\pgfpathlineto{\pgfqpoint{3.953897in}{1.347849in}}%
\pgfpathlineto{\pgfqpoint{3.939597in}{1.355000in}}%
\pgfpathlineto{\pgfqpoint{3.925302in}{1.362248in}}%
\pgfpathlineto{\pgfqpoint{3.933650in}{1.362087in}}%
\pgfpathlineto{\pgfqpoint{3.941987in}{1.362283in}}%
\pgfpathlineto{\pgfqpoint{3.950314in}{1.362827in}}%
\pgfpathlineto{\pgfqpoint{3.958631in}{1.363714in}}%
\pgfpathclose%
\pgfusepath{fill}%
\end{pgfscope}%
\begin{pgfscope}%
\pgfpathrectangle{\pgfqpoint{1.150000in}{0.150000in}}{\pgfqpoint{5.700000in}{5.700000in}}%
\pgfusepath{clip}%
\pgfsetbuttcap%
\pgfsetroundjoin%
\definecolor{currentfill}{rgb}{0.282290,0.145912,0.461510}%
\pgfsetfillcolor{currentfill}%
\pgfsetfillopacity{0.700000}%
\pgfsetlinewidth{0.000000pt}%
\definecolor{currentstroke}{rgb}{0.000000,0.000000,0.000000}%
\pgfsetstrokecolor{currentstroke}%
\pgfsetdash{}{0pt}%
\pgfpathmoveto{\pgfqpoint{3.754113in}{1.456894in}}%
\pgfpathlineto{\pgfqpoint{3.768356in}{1.448461in}}%
\pgfpathlineto{\pgfqpoint{3.782603in}{1.440128in}}%
\pgfpathlineto{\pgfqpoint{3.796853in}{1.431895in}}%
\pgfpathlineto{\pgfqpoint{3.811108in}{1.423761in}}%
\pgfpathlineto{\pgfqpoint{3.802690in}{1.425598in}}%
\pgfpathlineto{\pgfqpoint{3.794261in}{1.427816in}}%
\pgfpathlineto{\pgfqpoint{3.785819in}{1.430421in}}%
\pgfpathlineto{\pgfqpoint{3.777364in}{1.433422in}}%
\pgfpathlineto{\pgfqpoint{3.763077in}{1.442221in}}%
\pgfpathlineto{\pgfqpoint{3.748793in}{1.451121in}}%
\pgfpathlineto{\pgfqpoint{3.734513in}{1.460120in}}%
\pgfpathlineto{\pgfqpoint{3.720236in}{1.469220in}}%
\pgfpathlineto{\pgfqpoint{3.728725in}{1.465545in}}%
\pgfpathlineto{\pgfqpoint{3.737200in}{1.462270in}}%
\pgfpathlineto{\pgfqpoint{3.745663in}{1.459389in}}%
\pgfpathlineto{\pgfqpoint{3.754113in}{1.456894in}}%
\pgfpathclose%
\pgfusepath{fill}%
\end{pgfscope}%
\begin{pgfscope}%
\pgfpathrectangle{\pgfqpoint{1.150000in}{0.150000in}}{\pgfqpoint{5.700000in}{5.700000in}}%
\pgfusepath{clip}%
\pgfsetbuttcap%
\pgfsetroundjoin%
\definecolor{currentfill}{rgb}{0.259857,0.745492,0.444467}%
\pgfsetfillcolor{currentfill}%
\pgfsetfillopacity{0.700000}%
\pgfsetlinewidth{0.000000pt}%
\definecolor{currentstroke}{rgb}{0.000000,0.000000,0.000000}%
\pgfsetstrokecolor{currentstroke}%
\pgfsetdash{}{0pt}%
\pgfpathmoveto{\pgfqpoint{2.235150in}{3.049655in}}%
\pgfpathlineto{\pgfqpoint{2.249642in}{3.027225in}}%
\pgfpathlineto{\pgfqpoint{2.264125in}{3.004971in}}%
\pgfpathlineto{\pgfqpoint{2.278599in}{2.982892in}}%
\pgfpathlineto{\pgfqpoint{2.293065in}{2.960985in}}%
\pgfpathlineto{\pgfqpoint{2.283118in}{2.982928in}}%
\pgfpathlineto{\pgfqpoint{2.273134in}{3.005453in}}%
\pgfpathlineto{\pgfqpoint{2.263112in}{3.028568in}}%
\pgfpathlineto{\pgfqpoint{2.253051in}{3.052284in}}%
\pgfpathlineto{\pgfqpoint{2.238502in}{3.074973in}}%
\pgfpathlineto{\pgfqpoint{2.223944in}{3.097836in}}%
\pgfpathlineto{\pgfqpoint{2.209376in}{3.120875in}}%
\pgfpathlineto{\pgfqpoint{2.194799in}{3.144092in}}%
\pgfpathlineto{\pgfqpoint{2.204946in}{3.119579in}}%
\pgfpathlineto{\pgfqpoint{2.215053in}{3.095675in}}%
\pgfpathlineto{\pgfqpoint{2.225121in}{3.072370in}}%
\pgfpathlineto{\pgfqpoint{2.235150in}{3.049655in}}%
\pgfpathclose%
\pgfusepath{fill}%
\end{pgfscope}%
\begin{pgfscope}%
\pgfpathrectangle{\pgfqpoint{1.150000in}{0.150000in}}{\pgfqpoint{5.700000in}{5.700000in}}%
\pgfusepath{clip}%
\pgfsetbuttcap%
\pgfsetroundjoin%
\definecolor{currentfill}{rgb}{0.263663,0.237631,0.518762}%
\pgfsetfillcolor{currentfill}%
\pgfsetfillopacity{0.700000}%
\pgfsetlinewidth{0.000000pt}%
\definecolor{currentstroke}{rgb}{0.000000,0.000000,0.000000}%
\pgfsetstrokecolor{currentstroke}%
\pgfsetdash{}{0pt}%
\pgfpathmoveto{\pgfqpoint{5.032054in}{1.673011in}}%
\pgfpathlineto{\pgfqpoint{5.046726in}{1.676751in}}%
\pgfpathlineto{\pgfqpoint{5.061413in}{1.680588in}}%
\pgfpathlineto{\pgfqpoint{5.076112in}{1.684521in}}%
\pgfpathlineto{\pgfqpoint{5.090825in}{1.688549in}}%
\pgfpathlineto{\pgfqpoint{5.082801in}{1.672454in}}%
\pgfpathlineto{\pgfqpoint{5.074775in}{1.656419in}}%
\pgfpathlineto{\pgfqpoint{5.066747in}{1.640449in}}%
\pgfpathlineto{\pgfqpoint{5.058716in}{1.624549in}}%
\pgfpathlineto{\pgfqpoint{5.044009in}{1.620988in}}%
\pgfpathlineto{\pgfqpoint{5.029316in}{1.617523in}}%
\pgfpathlineto{\pgfqpoint{5.014636in}{1.614153in}}%
\pgfpathlineto{\pgfqpoint{4.999969in}{1.610878in}}%
\pgfpathlineto{\pgfqpoint{5.007994in}{1.626304in}}%
\pgfpathlineto{\pgfqpoint{5.016017in}{1.641805in}}%
\pgfpathlineto{\pgfqpoint{5.024036in}{1.657375in}}%
\pgfpathlineto{\pgfqpoint{5.032054in}{1.673011in}}%
\pgfpathclose%
\pgfusepath{fill}%
\end{pgfscope}%
\begin{pgfscope}%
\pgfpathrectangle{\pgfqpoint{1.150000in}{0.150000in}}{\pgfqpoint{5.700000in}{5.700000in}}%
\pgfusepath{clip}%
\pgfsetbuttcap%
\pgfsetroundjoin%
\definecolor{currentfill}{rgb}{0.260571,0.246922,0.522828}%
\pgfsetfillcolor{currentfill}%
\pgfsetfillopacity{0.700000}%
\pgfsetlinewidth{0.000000pt}%
\definecolor{currentstroke}{rgb}{0.000000,0.000000,0.000000}%
\pgfsetstrokecolor{currentstroke}%
\pgfsetdash{}{0pt}%
\pgfpathmoveto{\pgfqpoint{3.435286in}{1.672734in}}%
\pgfpathlineto{\pgfqpoint{3.449512in}{1.661565in}}%
\pgfpathlineto{\pgfqpoint{3.463741in}{1.650502in}}%
\pgfpathlineto{\pgfqpoint{3.477971in}{1.639546in}}%
\pgfpathlineto{\pgfqpoint{3.492203in}{1.628695in}}%
\pgfpathlineto{\pgfqpoint{3.483543in}{1.635514in}}%
\pgfpathlineto{\pgfqpoint{3.474867in}{1.642771in}}%
\pgfpathlineto{\pgfqpoint{3.466174in}{1.650475in}}%
\pgfpathlineto{\pgfqpoint{3.457463in}{1.658634in}}%
\pgfpathlineto{\pgfqpoint{3.443188in}{1.670181in}}%
\pgfpathlineto{\pgfqpoint{3.428913in}{1.681834in}}%
\pgfpathlineto{\pgfqpoint{3.414640in}{1.693594in}}%
\pgfpathlineto{\pgfqpoint{3.400368in}{1.705462in}}%
\pgfpathlineto{\pgfqpoint{3.409124in}{1.696597in}}%
\pgfpathlineto{\pgfqpoint{3.417863in}{1.688193in}}%
\pgfpathlineto{\pgfqpoint{3.426583in}{1.680241in}}%
\pgfpathlineto{\pgfqpoint{3.435286in}{1.672734in}}%
\pgfpathclose%
\pgfusepath{fill}%
\end{pgfscope}%
\begin{pgfscope}%
\pgfpathrectangle{\pgfqpoint{1.150000in}{0.150000in}}{\pgfqpoint{5.700000in}{5.700000in}}%
\pgfusepath{clip}%
\pgfsetbuttcap%
\pgfsetroundjoin%
\definecolor{currentfill}{rgb}{0.283187,0.125848,0.444960}%
\pgfsetfillcolor{currentfill}%
\pgfsetfillopacity{0.700000}%
\pgfsetlinewidth{0.000000pt}%
\definecolor{currentstroke}{rgb}{0.000000,0.000000,0.000000}%
\pgfsetstrokecolor{currentstroke}%
\pgfsetdash{}{0pt}%
\pgfpathmoveto{\pgfqpoint{4.728531in}{1.419292in}}%
\pgfpathlineto{\pgfqpoint{4.743048in}{1.419933in}}%
\pgfpathlineto{\pgfqpoint{4.757577in}{1.420669in}}%
\pgfpathlineto{\pgfqpoint{4.772116in}{1.421498in}}%
\pgfpathlineto{\pgfqpoint{4.786667in}{1.422422in}}%
\pgfpathlineto{\pgfqpoint{4.778614in}{1.409437in}}%
\pgfpathlineto{\pgfqpoint{4.770558in}{1.396602in}}%
\pgfpathlineto{\pgfqpoint{4.762500in}{1.383921in}}%
\pgfpathlineto{\pgfqpoint{4.754438in}{1.371401in}}%
\pgfpathlineto{\pgfqpoint{4.739888in}{1.371011in}}%
\pgfpathlineto{\pgfqpoint{4.725347in}{1.370715in}}%
\pgfpathlineto{\pgfqpoint{4.710818in}{1.370513in}}%
\pgfpathlineto{\pgfqpoint{4.696299in}{1.370404in}}%
\pgfpathlineto{\pgfqpoint{4.704362in}{1.382384in}}%
\pgfpathlineto{\pgfqpoint{4.712421in}{1.394529in}}%
\pgfpathlineto{\pgfqpoint{4.720477in}{1.406833in}}%
\pgfpathlineto{\pgfqpoint{4.728531in}{1.419292in}}%
\pgfpathclose%
\pgfusepath{fill}%
\end{pgfscope}%
\begin{pgfscope}%
\pgfpathrectangle{\pgfqpoint{1.150000in}{0.150000in}}{\pgfqpoint{5.700000in}{5.700000in}}%
\pgfusepath{clip}%
\pgfsetbuttcap%
\pgfsetroundjoin%
\definecolor{currentfill}{rgb}{0.282656,0.100196,0.422160}%
\pgfsetfillcolor{currentfill}%
\pgfsetfillopacity{0.700000}%
\pgfsetlinewidth{0.000000pt}%
\definecolor{currentstroke}{rgb}{0.000000,0.000000,0.000000}%
\pgfsetstrokecolor{currentstroke}%
\pgfsetdash{}{0pt}%
\pgfpathmoveto{\pgfqpoint{4.638327in}{1.370907in}}%
\pgfpathlineto{\pgfqpoint{4.652805in}{1.370641in}}%
\pgfpathlineto{\pgfqpoint{4.667292in}{1.370468in}}%
\pgfpathlineto{\pgfqpoint{4.681791in}{1.370389in}}%
\pgfpathlineto{\pgfqpoint{4.696299in}{1.370404in}}%
\pgfpathlineto{\pgfqpoint{4.688233in}{1.358593in}}%
\pgfpathlineto{\pgfqpoint{4.680165in}{1.346957in}}%
\pgfpathlineto{\pgfqpoint{4.672093in}{1.335502in}}%
\pgfpathlineto{\pgfqpoint{4.664018in}{1.324232in}}%
\pgfpathlineto{\pgfqpoint{4.649506in}{1.324767in}}%
\pgfpathlineto{\pgfqpoint{4.635005in}{1.325397in}}%
\pgfpathlineto{\pgfqpoint{4.620513in}{1.326119in}}%
\pgfpathlineto{\pgfqpoint{4.606032in}{1.326935in}}%
\pgfpathlineto{\pgfqpoint{4.614111in}{1.337648in}}%
\pgfpathlineto{\pgfqpoint{4.622186in}{1.348552in}}%
\pgfpathlineto{\pgfqpoint{4.630258in}{1.359640in}}%
\pgfpathlineto{\pgfqpoint{4.638327in}{1.370907in}}%
\pgfpathclose%
\pgfusepath{fill}%
\end{pgfscope}%
\begin{pgfscope}%
\pgfpathrectangle{\pgfqpoint{1.150000in}{0.150000in}}{\pgfqpoint{5.700000in}{5.700000in}}%
\pgfusepath{clip}%
\pgfsetbuttcap%
\pgfsetroundjoin%
\definecolor{currentfill}{rgb}{0.279566,0.067836,0.391917}%
\pgfsetfillcolor{currentfill}%
\pgfsetfillopacity{0.700000}%
\pgfsetlinewidth{0.000000pt}%
\definecolor{currentstroke}{rgb}{0.000000,0.000000,0.000000}%
\pgfsetstrokecolor{currentstroke}%
\pgfsetdash{}{0pt}%
\pgfpathmoveto{\pgfqpoint{4.163133in}{1.305126in}}%
\pgfpathlineto{\pgfqpoint{4.177451in}{1.300297in}}%
\pgfpathlineto{\pgfqpoint{4.191776in}{1.295563in}}%
\pgfpathlineto{\pgfqpoint{4.206107in}{1.290924in}}%
\pgfpathlineto{\pgfqpoint{4.220446in}{1.286380in}}%
\pgfpathlineto{\pgfqpoint{4.212248in}{1.281733in}}%
\pgfpathlineto{\pgfqpoint{4.204043in}{1.277383in}}%
\pgfpathlineto{\pgfqpoint{4.195830in}{1.273334in}}%
\pgfpathlineto{\pgfqpoint{4.187611in}{1.269595in}}%
\pgfpathlineto{\pgfqpoint{4.173254in}{1.274760in}}%
\pgfpathlineto{\pgfqpoint{4.158904in}{1.280020in}}%
\pgfpathlineto{\pgfqpoint{4.144560in}{1.285376in}}%
\pgfpathlineto{\pgfqpoint{4.130222in}{1.290826in}}%
\pgfpathlineto{\pgfqpoint{4.138461in}{1.293937in}}%
\pgfpathlineto{\pgfqpoint{4.146692in}{1.297361in}}%
\pgfpathlineto{\pgfqpoint{4.154916in}{1.301093in}}%
\pgfpathlineto{\pgfqpoint{4.163133in}{1.305126in}}%
\pgfpathclose%
\pgfusepath{fill}%
\end{pgfscope}%
\begin{pgfscope}%
\pgfpathrectangle{\pgfqpoint{1.150000in}{0.150000in}}{\pgfqpoint{5.700000in}{5.700000in}}%
\pgfusepath{clip}%
\pgfsetbuttcap%
\pgfsetroundjoin%
\definecolor{currentfill}{rgb}{0.278791,0.062145,0.386592}%
\pgfsetfillcolor{currentfill}%
\pgfsetfillopacity{0.700000}%
\pgfsetlinewidth{0.000000pt}%
\definecolor{currentstroke}{rgb}{0.000000,0.000000,0.000000}%
\pgfsetstrokecolor{currentstroke}%
\pgfsetdash{}{0pt}%
\pgfpathmoveto{\pgfqpoint{4.310535in}{1.292982in}}%
\pgfpathlineto{\pgfqpoint{4.324895in}{1.289513in}}%
\pgfpathlineto{\pgfqpoint{4.339262in}{1.286139in}}%
\pgfpathlineto{\pgfqpoint{4.353637in}{1.282859in}}%
\pgfpathlineto{\pgfqpoint{4.368020in}{1.279673in}}%
\pgfpathlineto{\pgfqpoint{4.359876in}{1.272714in}}%
\pgfpathlineto{\pgfqpoint{4.351727in}{1.266017in}}%
\pgfpathlineto{\pgfqpoint{4.343572in}{1.259588in}}%
\pgfpathlineto{\pgfqpoint{4.335412in}{1.253432in}}%
\pgfpathlineto{\pgfqpoint{4.321015in}{1.257221in}}%
\pgfpathlineto{\pgfqpoint{4.306626in}{1.261104in}}%
\pgfpathlineto{\pgfqpoint{4.292244in}{1.265081in}}%
\pgfpathlineto{\pgfqpoint{4.277870in}{1.269152in}}%
\pgfpathlineto{\pgfqpoint{4.286046in}{1.274698in}}%
\pgfpathlineto{\pgfqpoint{4.294215in}{1.280522in}}%
\pgfpathlineto{\pgfqpoint{4.302378in}{1.286619in}}%
\pgfpathlineto{\pgfqpoint{4.310535in}{1.292982in}}%
\pgfpathclose%
\pgfusepath{fill}%
\end{pgfscope}%
\begin{pgfscope}%
\pgfpathrectangle{\pgfqpoint{1.150000in}{0.150000in}}{\pgfqpoint{5.700000in}{5.700000in}}%
\pgfusepath{clip}%
\pgfsetbuttcap%
\pgfsetroundjoin%
\definecolor{currentfill}{rgb}{0.281887,0.150881,0.465405}%
\pgfsetfillcolor{currentfill}%
\pgfsetfillopacity{0.700000}%
\pgfsetlinewidth{0.000000pt}%
\definecolor{currentstroke}{rgb}{0.000000,0.000000,0.000000}%
\pgfsetstrokecolor{currentstroke}%
\pgfsetdash{}{0pt}%
\pgfpathmoveto{\pgfqpoint{4.818851in}{1.475756in}}%
\pgfpathlineto{\pgfqpoint{4.833413in}{1.477290in}}%
\pgfpathlineto{\pgfqpoint{4.847987in}{1.478919in}}%
\pgfpathlineto{\pgfqpoint{4.862573in}{1.480642in}}%
\pgfpathlineto{\pgfqpoint{4.877171in}{1.482459in}}%
\pgfpathlineto{\pgfqpoint{4.869127in}{1.468416in}}%
\pgfpathlineto{\pgfqpoint{4.861081in}{1.454498in}}%
\pgfpathlineto{\pgfqpoint{4.853032in}{1.440710in}}%
\pgfpathlineto{\pgfqpoint{4.844981in}{1.427057in}}%
\pgfpathlineto{\pgfqpoint{4.830385in}{1.425757in}}%
\pgfpathlineto{\pgfqpoint{4.815801in}{1.424551in}}%
\pgfpathlineto{\pgfqpoint{4.801228in}{1.423440in}}%
\pgfpathlineto{\pgfqpoint{4.786667in}{1.422422in}}%
\pgfpathlineto{\pgfqpoint{4.794717in}{1.435551in}}%
\pgfpathlineto{\pgfqpoint{4.802764in}{1.448820in}}%
\pgfpathlineto{\pgfqpoint{4.810809in}{1.462224in}}%
\pgfpathlineto{\pgfqpoint{4.818851in}{1.475756in}}%
\pgfpathclose%
\pgfusepath{fill}%
\end{pgfscope}%
\begin{pgfscope}%
\pgfpathrectangle{\pgfqpoint{1.150000in}{0.150000in}}{\pgfqpoint{5.700000in}{5.700000in}}%
\pgfusepath{clip}%
\pgfsetbuttcap%
\pgfsetroundjoin%
\definecolor{currentfill}{rgb}{0.281446,0.084320,0.407414}%
\pgfsetfillcolor{currentfill}%
\pgfsetfillopacity{0.700000}%
\pgfsetlinewidth{0.000000pt}%
\definecolor{currentstroke}{rgb}{0.000000,0.000000,0.000000}%
\pgfsetstrokecolor{currentstroke}%
\pgfsetdash{}{0pt}%
\pgfpathmoveto{\pgfqpoint{4.548202in}{1.331137in}}%
\pgfpathlineto{\pgfqpoint{4.562645in}{1.329946in}}%
\pgfpathlineto{\pgfqpoint{4.577098in}{1.328849in}}%
\pgfpathlineto{\pgfqpoint{4.591560in}{1.327845in}}%
\pgfpathlineto{\pgfqpoint{4.606032in}{1.326935in}}%
\pgfpathlineto{\pgfqpoint{4.597949in}{1.316418in}}%
\pgfpathlineto{\pgfqpoint{4.589863in}{1.306102in}}%
\pgfpathlineto{\pgfqpoint{4.581773in}{1.295993in}}%
\pgfpathlineto{\pgfqpoint{4.573679in}{1.286096in}}%
\pgfpathlineto{\pgfqpoint{4.559202in}{1.287574in}}%
\pgfpathlineto{\pgfqpoint{4.544734in}{1.289144in}}%
\pgfpathlineto{\pgfqpoint{4.530274in}{1.290809in}}%
\pgfpathlineto{\pgfqpoint{4.515825in}{1.292566in}}%
\pgfpathlineto{\pgfqpoint{4.523925in}{1.301889in}}%
\pgfpathlineto{\pgfqpoint{4.532021in}{1.311429in}}%
\pgfpathlineto{\pgfqpoint{4.540113in}{1.321180in}}%
\pgfpathlineto{\pgfqpoint{4.548202in}{1.331137in}}%
\pgfpathclose%
\pgfusepath{fill}%
\end{pgfscope}%
\begin{pgfscope}%
\pgfpathrectangle{\pgfqpoint{1.150000in}{0.150000in}}{\pgfqpoint{5.700000in}{5.700000in}}%
\pgfusepath{clip}%
\pgfsetbuttcap%
\pgfsetroundjoin%
\definecolor{currentfill}{rgb}{0.250425,0.274290,0.533103}%
\pgfsetfillcolor{currentfill}%
\pgfsetfillopacity{0.700000}%
\pgfsetlinewidth{0.000000pt}%
\definecolor{currentstroke}{rgb}{0.000000,0.000000,0.000000}%
\pgfsetstrokecolor{currentstroke}%
\pgfsetdash{}{0pt}%
\pgfpathmoveto{\pgfqpoint{5.122892in}{1.753452in}}%
\pgfpathlineto{\pgfqpoint{5.137626in}{1.758026in}}%
\pgfpathlineto{\pgfqpoint{5.152374in}{1.762696in}}%
\pgfpathlineto{\pgfqpoint{5.167135in}{1.767463in}}%
\pgfpathlineto{\pgfqpoint{5.159117in}{1.750835in}}%
\pgfpathlineto{\pgfqpoint{5.151095in}{1.734247in}}%
\pgfpathlineto{\pgfqpoint{5.143071in}{1.717704in}}%
\pgfpathlineto{\pgfqpoint{5.135045in}{1.701209in}}%
\pgfpathlineto{\pgfqpoint{5.120291in}{1.696893in}}%
\pgfpathlineto{\pgfqpoint{5.105551in}{1.692673in}}%
\pgfpathlineto{\pgfqpoint{5.090825in}{1.688549in}}%
\pgfpathlineto{\pgfqpoint{5.098846in}{1.704701in}}%
\pgfpathlineto{\pgfqpoint{5.106864in}{1.720904in}}%
\pgfpathlineto{\pgfqpoint{5.114880in}{1.737156in}}%
\pgfpathlineto{\pgfqpoint{5.122892in}{1.753452in}}%
\pgfpathclose%
\pgfusepath{fill}%
\end{pgfscope}%
\begin{pgfscope}%
\pgfpathrectangle{\pgfqpoint{1.150000in}{0.150000in}}{\pgfqpoint{5.700000in}{5.700000in}}%
\pgfusepath{clip}%
\pgfsetbuttcap%
\pgfsetroundjoin%
\definecolor{currentfill}{rgb}{0.266580,0.228262,0.514349}%
\pgfsetfillcolor{currentfill}%
\pgfsetfillopacity{0.700000}%
\pgfsetlinewidth{0.000000pt}%
\definecolor{currentstroke}{rgb}{0.000000,0.000000,0.000000}%
\pgfsetstrokecolor{currentstroke}%
\pgfsetdash{}{0pt}%
\pgfpathmoveto{\pgfqpoint{3.492203in}{1.628695in}}%
\pgfpathlineto{\pgfqpoint{3.506436in}{1.617950in}}%
\pgfpathlineto{\pgfqpoint{3.520672in}{1.607310in}}%
\pgfpathlineto{\pgfqpoint{3.534910in}{1.596776in}}%
\pgfpathlineto{\pgfqpoint{3.549150in}{1.586345in}}%
\pgfpathlineto{\pgfqpoint{3.540533in}{1.592478in}}%
\pgfpathlineto{\pgfqpoint{3.531899in}{1.599043in}}%
\pgfpathlineto{\pgfqpoint{3.523249in}{1.606049in}}%
\pgfpathlineto{\pgfqpoint{3.514583in}{1.613503in}}%
\pgfpathlineto{\pgfqpoint{3.500301in}{1.624628in}}%
\pgfpathlineto{\pgfqpoint{3.486020in}{1.635858in}}%
\pgfpathlineto{\pgfqpoint{3.471741in}{1.647193in}}%
\pgfpathlineto{\pgfqpoint{3.457463in}{1.658634in}}%
\pgfpathlineto{\pgfqpoint{3.466174in}{1.650475in}}%
\pgfpathlineto{\pgfqpoint{3.474867in}{1.642771in}}%
\pgfpathlineto{\pgfqpoint{3.483543in}{1.635514in}}%
\pgfpathlineto{\pgfqpoint{3.492203in}{1.628695in}}%
\pgfpathclose%
\pgfusepath{fill}%
\end{pgfscope}%
\begin{pgfscope}%
\pgfpathrectangle{\pgfqpoint{1.150000in}{0.150000in}}{\pgfqpoint{5.700000in}{5.700000in}}%
\pgfusepath{clip}%
\pgfsetbuttcap%
\pgfsetroundjoin%
\definecolor{currentfill}{rgb}{0.327796,0.773980,0.406640}%
\pgfsetfillcolor{currentfill}%
\pgfsetfillopacity{0.700000}%
\pgfsetlinewidth{0.000000pt}%
\definecolor{currentstroke}{rgb}{0.000000,0.000000,0.000000}%
\pgfsetstrokecolor{currentstroke}%
\pgfsetdash{}{0pt}%
\pgfpathmoveto{\pgfqpoint{2.177089in}{3.141166in}}%
\pgfpathlineto{\pgfqpoint{2.191619in}{3.118016in}}%
\pgfpathlineto{\pgfqpoint{2.206139in}{3.095049in}}%
\pgfpathlineto{\pgfqpoint{2.220649in}{3.072263in}}%
\pgfpathlineto{\pgfqpoint{2.235150in}{3.049655in}}%
\pgfpathlineto{\pgfqpoint{2.225121in}{3.072370in}}%
\pgfpathlineto{\pgfqpoint{2.215053in}{3.095675in}}%
\pgfpathlineto{\pgfqpoint{2.204946in}{3.119579in}}%
\pgfpathlineto{\pgfqpoint{2.194799in}{3.144092in}}%
\pgfpathlineto{\pgfqpoint{2.180212in}{3.167488in}}%
\pgfpathlineto{\pgfqpoint{2.165615in}{3.191066in}}%
\pgfpathlineto{\pgfqpoint{2.151007in}{3.214826in}}%
\pgfpathlineto{\pgfqpoint{2.136390in}{3.238769in}}%
\pgfpathlineto{\pgfqpoint{2.146625in}{3.213453in}}%
\pgfpathlineto{\pgfqpoint{2.156820in}{3.188753in}}%
\pgfpathlineto{\pgfqpoint{2.166974in}{3.164660in}}%
\pgfpathlineto{\pgfqpoint{2.177089in}{3.141166in}}%
\pgfpathclose%
\pgfusepath{fill}%
\end{pgfscope}%
\begin{pgfscope}%
\pgfpathrectangle{\pgfqpoint{1.150000in}{0.150000in}}{\pgfqpoint{5.700000in}{5.700000in}}%
\pgfusepath{clip}%
\pgfsetbuttcap%
\pgfsetroundjoin%
\definecolor{currentfill}{rgb}{0.281924,0.089666,0.412415}%
\pgfsetfillcolor{currentfill}%
\pgfsetfillopacity{0.700000}%
\pgfsetlinewidth{0.000000pt}%
\definecolor{currentstroke}{rgb}{0.000000,0.000000,0.000000}%
\pgfsetstrokecolor{currentstroke}%
\pgfsetdash{}{0pt}%
\pgfpathmoveto{\pgfqpoint{4.015738in}{1.337876in}}%
\pgfpathlineto{\pgfqpoint{4.030028in}{1.331658in}}%
\pgfpathlineto{\pgfqpoint{4.044324in}{1.325537in}}%
\pgfpathlineto{\pgfqpoint{4.058625in}{1.319512in}}%
\pgfpathlineto{\pgfqpoint{4.072933in}{1.313583in}}%
\pgfpathlineto{\pgfqpoint{4.064664in}{1.311427in}}%
\pgfpathlineto{\pgfqpoint{4.056386in}{1.309603in}}%
\pgfpathlineto{\pgfqpoint{4.048100in}{1.308118in}}%
\pgfpathlineto{\pgfqpoint{4.039805in}{1.306978in}}%
\pgfpathlineto{\pgfqpoint{4.025473in}{1.313549in}}%
\pgfpathlineto{\pgfqpoint{4.011148in}{1.320215in}}%
\pgfpathlineto{\pgfqpoint{3.996827in}{1.326979in}}%
\pgfpathlineto{\pgfqpoint{3.982512in}{1.333839in}}%
\pgfpathlineto{\pgfqpoint{3.990833in}{1.334329in}}%
\pgfpathlineto{\pgfqpoint{3.999144in}{1.335169in}}%
\pgfpathlineto{\pgfqpoint{4.007445in}{1.336354in}}%
\pgfpathlineto{\pgfqpoint{4.015738in}{1.337876in}}%
\pgfpathclose%
\pgfusepath{fill}%
\end{pgfscope}%
\begin{pgfscope}%
\pgfpathrectangle{\pgfqpoint{1.150000in}{0.150000in}}{\pgfqpoint{5.700000in}{5.700000in}}%
\pgfusepath{clip}%
\pgfsetbuttcap%
\pgfsetroundjoin%
\definecolor{currentfill}{rgb}{0.277134,0.185228,0.489898}%
\pgfsetfillcolor{currentfill}%
\pgfsetfillopacity{0.700000}%
\pgfsetlinewidth{0.000000pt}%
\definecolor{currentstroke}{rgb}{0.000000,0.000000,0.000000}%
\pgfsetstrokecolor{currentstroke}%
\pgfsetdash{}{0pt}%
\pgfpathmoveto{\pgfqpoint{4.909320in}{1.539785in}}%
\pgfpathlineto{\pgfqpoint{4.923933in}{1.542197in}}%
\pgfpathlineto{\pgfqpoint{4.938557in}{1.544703in}}%
\pgfpathlineto{\pgfqpoint{4.953194in}{1.547304in}}%
\pgfpathlineto{\pgfqpoint{4.967844in}{1.550000in}}%
\pgfpathlineto{\pgfqpoint{4.959806in}{1.535011in}}%
\pgfpathlineto{\pgfqpoint{4.951767in}{1.520123in}}%
\pgfpathlineto{\pgfqpoint{4.943724in}{1.505342in}}%
\pgfpathlineto{\pgfqpoint{4.935680in}{1.490671in}}%
\pgfpathlineto{\pgfqpoint{4.921034in}{1.488476in}}%
\pgfpathlineto{\pgfqpoint{4.906401in}{1.486376in}}%
\pgfpathlineto{\pgfqpoint{4.891780in}{1.484370in}}%
\pgfpathlineto{\pgfqpoint{4.877171in}{1.482459in}}%
\pgfpathlineto{\pgfqpoint{4.885212in}{1.496622in}}%
\pgfpathlineto{\pgfqpoint{4.893250in}{1.510901in}}%
\pgfpathlineto{\pgfqpoint{4.901287in}{1.525290in}}%
\pgfpathlineto{\pgfqpoint{4.909320in}{1.539785in}}%
\pgfpathclose%
\pgfusepath{fill}%
\end{pgfscope}%
\begin{pgfscope}%
\pgfpathrectangle{\pgfqpoint{1.150000in}{0.150000in}}{\pgfqpoint{5.700000in}{5.700000in}}%
\pgfusepath{clip}%
\pgfsetbuttcap%
\pgfsetroundjoin%
\definecolor{currentfill}{rgb}{0.282884,0.135920,0.453427}%
\pgfsetfillcolor{currentfill}%
\pgfsetfillopacity{0.700000}%
\pgfsetlinewidth{0.000000pt}%
\definecolor{currentstroke}{rgb}{0.000000,0.000000,0.000000}%
\pgfsetstrokecolor{currentstroke}%
\pgfsetdash{}{0pt}%
\pgfpathmoveto{\pgfqpoint{3.811108in}{1.423761in}}%
\pgfpathlineto{\pgfqpoint{3.825367in}{1.415727in}}%
\pgfpathlineto{\pgfqpoint{3.839630in}{1.407791in}}%
\pgfpathlineto{\pgfqpoint{3.853898in}{1.399955in}}%
\pgfpathlineto{\pgfqpoint{3.868169in}{1.392217in}}%
\pgfpathlineto{\pgfqpoint{3.859782in}{1.393397in}}%
\pgfpathlineto{\pgfqpoint{3.851384in}{1.394952in}}%
\pgfpathlineto{\pgfqpoint{3.842974in}{1.396889in}}%
\pgfpathlineto{\pgfqpoint{3.834553in}{1.399217in}}%
\pgfpathlineto{\pgfqpoint{3.820250in}{1.407619in}}%
\pgfpathlineto{\pgfqpoint{3.805951in}{1.416121in}}%
\pgfpathlineto{\pgfqpoint{3.791656in}{1.424722in}}%
\pgfpathlineto{\pgfqpoint{3.777364in}{1.433422in}}%
\pgfpathlineto{\pgfqpoint{3.785819in}{1.430421in}}%
\pgfpathlineto{\pgfqpoint{3.794261in}{1.427816in}}%
\pgfpathlineto{\pgfqpoint{3.802690in}{1.425598in}}%
\pgfpathlineto{\pgfqpoint{3.811108in}{1.423761in}}%
\pgfpathclose%
\pgfusepath{fill}%
\end{pgfscope}%
\begin{pgfscope}%
\pgfpathrectangle{\pgfqpoint{1.150000in}{0.150000in}}{\pgfqpoint{5.700000in}{5.700000in}}%
\pgfusepath{clip}%
\pgfsetbuttcap%
\pgfsetroundjoin%
\definecolor{currentfill}{rgb}{0.280267,0.073417,0.397163}%
\pgfsetfillcolor{currentfill}%
\pgfsetfillopacity{0.700000}%
\pgfsetlinewidth{0.000000pt}%
\definecolor{currentstroke}{rgb}{0.000000,0.000000,0.000000}%
\pgfsetstrokecolor{currentstroke}%
\pgfsetdash{}{0pt}%
\pgfpathmoveto{\pgfqpoint{4.458114in}{1.300534in}}%
\pgfpathlineto{\pgfqpoint{4.472529in}{1.298401in}}%
\pgfpathlineto{\pgfqpoint{4.486952in}{1.296363in}}%
\pgfpathlineto{\pgfqpoint{4.501384in}{1.294418in}}%
\pgfpathlineto{\pgfqpoint{4.515825in}{1.292566in}}%
\pgfpathlineto{\pgfqpoint{4.507720in}{1.283467in}}%
\pgfpathlineto{\pgfqpoint{4.499611in}{1.274595in}}%
\pgfpathlineto{\pgfqpoint{4.491498in}{1.265958in}}%
\pgfpathlineto{\pgfqpoint{4.483381in}{1.257561in}}%
\pgfpathlineto{\pgfqpoint{4.468931in}{1.259997in}}%
\pgfpathlineto{\pgfqpoint{4.454490in}{1.262527in}}%
\pgfpathlineto{\pgfqpoint{4.440058in}{1.265151in}}%
\pgfpathlineto{\pgfqpoint{4.425634in}{1.267867in}}%
\pgfpathlineto{\pgfqpoint{4.433761in}{1.275673in}}%
\pgfpathlineto{\pgfqpoint{4.441884in}{1.283723in}}%
\pgfpathlineto{\pgfqpoint{4.450001in}{1.292012in}}%
\pgfpathlineto{\pgfqpoint{4.458114in}{1.300534in}}%
\pgfpathclose%
\pgfusepath{fill}%
\end{pgfscope}%
\begin{pgfscope}%
\pgfpathrectangle{\pgfqpoint{1.150000in}{0.150000in}}{\pgfqpoint{5.700000in}{5.700000in}}%
\pgfusepath{clip}%
\pgfsetbuttcap%
\pgfsetroundjoin%
\definecolor{currentfill}{rgb}{0.159194,0.482237,0.558073}%
\pgfsetfillcolor{currentfill}%
\pgfsetfillopacity{0.700000}%
\pgfsetlinewidth{0.000000pt}%
\definecolor{currentstroke}{rgb}{0.000000,0.000000,0.000000}%
\pgfsetstrokecolor{currentstroke}%
\pgfsetdash{}{0pt}%
\pgfpathmoveto{\pgfqpoint{2.829332in}{2.273439in}}%
\pgfpathlineto{\pgfqpoint{2.843639in}{2.256877in}}%
\pgfpathlineto{\pgfqpoint{2.857943in}{2.240445in}}%
\pgfpathlineto{\pgfqpoint{2.872244in}{2.224141in}}%
\pgfpathlineto{\pgfqpoint{2.886543in}{2.207965in}}%
\pgfpathlineto{\pgfqpoint{2.877267in}{2.223836in}}%
\pgfpathlineto{\pgfqpoint{2.867963in}{2.240242in}}%
\pgfpathlineto{\pgfqpoint{2.858632in}{2.257193in}}%
\pgfpathlineto{\pgfqpoint{2.844285in}{2.273931in}}%
\pgfpathlineto{\pgfqpoint{2.829934in}{2.290797in}}%
\pgfpathlineto{\pgfqpoint{2.815581in}{2.307793in}}%
\pgfpathlineto{\pgfqpoint{2.801224in}{2.324919in}}%
\pgfpathlineto{\pgfqpoint{2.810622in}{2.307210in}}%
\pgfpathlineto{\pgfqpoint{2.819991in}{2.290053in}}%
\pgfpathlineto{\pgfqpoint{2.829332in}{2.273439in}}%
\pgfpathclose%
\pgfusepath{fill}%
\end{pgfscope}%
\begin{pgfscope}%
\pgfpathrectangle{\pgfqpoint{1.150000in}{0.150000in}}{\pgfqpoint{5.700000in}{5.700000in}}%
\pgfusepath{clip}%
\pgfsetbuttcap%
\pgfsetroundjoin%
\definecolor{currentfill}{rgb}{0.168126,0.459988,0.558082}%
\pgfsetfillcolor{currentfill}%
\pgfsetfillopacity{0.700000}%
\pgfsetlinewidth{0.000000pt}%
\definecolor{currentstroke}{rgb}{0.000000,0.000000,0.000000}%
\pgfsetstrokecolor{currentstroke}%
\pgfsetdash{}{0pt}%
\pgfpathmoveto{\pgfqpoint{2.886543in}{2.207965in}}%
\pgfpathlineto{\pgfqpoint{2.900839in}{2.191917in}}%
\pgfpathlineto{\pgfqpoint{2.915132in}{2.175994in}}%
\pgfpathlineto{\pgfqpoint{2.929422in}{2.160198in}}%
\pgfpathlineto{\pgfqpoint{2.943711in}{2.144526in}}%
\pgfpathlineto{\pgfqpoint{2.934498in}{2.159657in}}%
\pgfpathlineto{\pgfqpoint{2.925258in}{2.175317in}}%
\pgfpathlineto{\pgfqpoint{2.915992in}{2.191514in}}%
\pgfpathlineto{\pgfqpoint{2.901656in}{2.207745in}}%
\pgfpathlineto{\pgfqpoint{2.887317in}{2.224101in}}%
\pgfpathlineto{\pgfqpoint{2.872976in}{2.240583in}}%
\pgfpathlineto{\pgfqpoint{2.858632in}{2.257193in}}%
\pgfpathlineto{\pgfqpoint{2.867963in}{2.240242in}}%
\pgfpathlineto{\pgfqpoint{2.877267in}{2.223836in}}%
\pgfpathlineto{\pgfqpoint{2.886543in}{2.207965in}}%
\pgfpathclose%
\pgfusepath{fill}%
\end{pgfscope}%
\begin{pgfscope}%
\pgfpathrectangle{\pgfqpoint{1.150000in}{0.150000in}}{\pgfqpoint{5.700000in}{5.700000in}}%
\pgfusepath{clip}%
\pgfsetbuttcap%
\pgfsetroundjoin%
\definecolor{currentfill}{rgb}{0.150476,0.504369,0.557430}%
\pgfsetfillcolor{currentfill}%
\pgfsetfillopacity{0.700000}%
\pgfsetlinewidth{0.000000pt}%
\definecolor{currentstroke}{rgb}{0.000000,0.000000,0.000000}%
\pgfsetstrokecolor{currentstroke}%
\pgfsetdash{}{0pt}%
\pgfpathmoveto{\pgfqpoint{2.772071in}{2.340994in}}%
\pgfpathlineto{\pgfqpoint{2.786392in}{2.323907in}}%
\pgfpathlineto{\pgfqpoint{2.800708in}{2.306953in}}%
\pgfpathlineto{\pgfqpoint{2.815022in}{2.290130in}}%
\pgfpathlineto{\pgfqpoint{2.829332in}{2.273439in}}%
\pgfpathlineto{\pgfqpoint{2.819991in}{2.290053in}}%
\pgfpathlineto{\pgfqpoint{2.810622in}{2.307210in}}%
\pgfpathlineto{\pgfqpoint{2.801224in}{2.324919in}}%
\pgfpathlineto{\pgfqpoint{2.786863in}{2.342176in}}%
\pgfpathlineto{\pgfqpoint{2.772499in}{2.359565in}}%
\pgfpathlineto{\pgfqpoint{2.758131in}{2.377086in}}%
\pgfpathlineto{\pgfqpoint{2.743759in}{2.394740in}}%
\pgfpathlineto{\pgfqpoint{2.753226in}{2.376269in}}%
\pgfpathlineto{\pgfqpoint{2.762664in}{2.358357in}}%
\pgfpathlineto{\pgfqpoint{2.772071in}{2.340994in}}%
\pgfpathclose%
\pgfusepath{fill}%
\end{pgfscope}%
\begin{pgfscope}%
\pgfpathrectangle{\pgfqpoint{1.150000in}{0.150000in}}{\pgfqpoint{5.700000in}{5.700000in}}%
\pgfusepath{clip}%
\pgfsetbuttcap%
\pgfsetroundjoin%
\definecolor{currentfill}{rgb}{0.177423,0.437527,0.557565}%
\pgfsetfillcolor{currentfill}%
\pgfsetfillopacity{0.700000}%
\pgfsetlinewidth{0.000000pt}%
\definecolor{currentstroke}{rgb}{0.000000,0.000000,0.000000}%
\pgfsetstrokecolor{currentstroke}%
\pgfsetdash{}{0pt}%
\pgfpathmoveto{\pgfqpoint{2.943711in}{2.144526in}}%
\pgfpathlineto{\pgfqpoint{2.957997in}{2.128979in}}%
\pgfpathlineto{\pgfqpoint{2.972281in}{2.113556in}}%
\pgfpathlineto{\pgfqpoint{2.986563in}{2.098255in}}%
\pgfpathlineto{\pgfqpoint{3.000843in}{2.083077in}}%
\pgfpathlineto{\pgfqpoint{2.991691in}{2.097472in}}%
\pgfpathlineto{\pgfqpoint{2.982514in}{2.112389in}}%
\pgfpathlineto{\pgfqpoint{2.973311in}{2.127837in}}%
\pgfpathlineto{\pgfqpoint{2.958984in}{2.143571in}}%
\pgfpathlineto{\pgfqpoint{2.944656in}{2.159428in}}%
\pgfpathlineto{\pgfqpoint{2.930325in}{2.175409in}}%
\pgfpathlineto{\pgfqpoint{2.915992in}{2.191514in}}%
\pgfpathlineto{\pgfqpoint{2.925258in}{2.175317in}}%
\pgfpathlineto{\pgfqpoint{2.934498in}{2.159657in}}%
\pgfpathlineto{\pgfqpoint{2.943711in}{2.144526in}}%
\pgfpathclose%
\pgfusepath{fill}%
\end{pgfscope}%
\begin{pgfscope}%
\pgfpathrectangle{\pgfqpoint{1.150000in}{0.150000in}}{\pgfqpoint{5.700000in}{5.700000in}}%
\pgfusepath{clip}%
\pgfsetbuttcap%
\pgfsetroundjoin%
\definecolor{currentfill}{rgb}{0.140536,0.530132,0.555659}%
\pgfsetfillcolor{currentfill}%
\pgfsetfillopacity{0.700000}%
\pgfsetlinewidth{0.000000pt}%
\definecolor{currentstroke}{rgb}{0.000000,0.000000,0.000000}%
\pgfsetstrokecolor{currentstroke}%
\pgfsetdash{}{0pt}%
\pgfpathmoveto{\pgfqpoint{2.714752in}{2.410682in}}%
\pgfpathlineto{\pgfqpoint{2.729088in}{2.393057in}}%
\pgfpathlineto{\pgfqpoint{2.743419in}{2.375568in}}%
\pgfpathlineto{\pgfqpoint{2.757747in}{2.358214in}}%
\pgfpathlineto{\pgfqpoint{2.772071in}{2.340994in}}%
\pgfpathlineto{\pgfqpoint{2.762664in}{2.358357in}}%
\pgfpathlineto{\pgfqpoint{2.753226in}{2.376269in}}%
\pgfpathlineto{\pgfqpoint{2.743759in}{2.394740in}}%
\pgfpathlineto{\pgfqpoint{2.729383in}{2.412529in}}%
\pgfpathlineto{\pgfqpoint{2.715003in}{2.430452in}}%
\pgfpathlineto{\pgfqpoint{2.700619in}{2.448511in}}%
\pgfpathlineto{\pgfqpoint{2.686231in}{2.466707in}}%
\pgfpathlineto{\pgfqpoint{2.695769in}{2.447469in}}%
\pgfpathlineto{\pgfqpoint{2.705276in}{2.428797in}}%
\pgfpathlineto{\pgfqpoint{2.714752in}{2.410682in}}%
\pgfpathclose%
\pgfusepath{fill}%
\end{pgfscope}%
\begin{pgfscope}%
\pgfpathrectangle{\pgfqpoint{1.150000in}{0.150000in}}{\pgfqpoint{5.700000in}{5.700000in}}%
\pgfusepath{clip}%
\pgfsetbuttcap%
\pgfsetroundjoin%
\definecolor{currentfill}{rgb}{0.270595,0.214069,0.507052}%
\pgfsetfillcolor{currentfill}%
\pgfsetfillopacity{0.700000}%
\pgfsetlinewidth{0.000000pt}%
\definecolor{currentstroke}{rgb}{0.000000,0.000000,0.000000}%
\pgfsetstrokecolor{currentstroke}%
\pgfsetdash{}{0pt}%
\pgfpathmoveto{\pgfqpoint{3.549150in}{1.586345in}}%
\pgfpathlineto{\pgfqpoint{3.563393in}{1.576019in}}%
\pgfpathlineto{\pgfqpoint{3.577637in}{1.565797in}}%
\pgfpathlineto{\pgfqpoint{3.591885in}{1.555679in}}%
\pgfpathlineto{\pgfqpoint{3.606134in}{1.545663in}}%
\pgfpathlineto{\pgfqpoint{3.597557in}{1.551111in}}%
\pgfpathlineto{\pgfqpoint{3.588964in}{1.556986in}}%
\pgfpathlineto{\pgfqpoint{3.580356in}{1.563296in}}%
\pgfpathlineto{\pgfqpoint{3.571732in}{1.570047in}}%
\pgfpathlineto{\pgfqpoint{3.557442in}{1.580756in}}%
\pgfpathlineto{\pgfqpoint{3.543153in}{1.591567in}}%
\pgfpathlineto{\pgfqpoint{3.528867in}{1.602483in}}%
\pgfpathlineto{\pgfqpoint{3.514583in}{1.613503in}}%
\pgfpathlineto{\pgfqpoint{3.523249in}{1.606049in}}%
\pgfpathlineto{\pgfqpoint{3.531899in}{1.599043in}}%
\pgfpathlineto{\pgfqpoint{3.540533in}{1.592478in}}%
\pgfpathlineto{\pgfqpoint{3.549150in}{1.586345in}}%
\pgfpathclose%
\pgfusepath{fill}%
\end{pgfscope}%
\begin{pgfscope}%
\pgfpathrectangle{\pgfqpoint{1.150000in}{0.150000in}}{\pgfqpoint{5.700000in}{5.700000in}}%
\pgfusepath{clip}%
\pgfsetbuttcap%
\pgfsetroundjoin%
\definecolor{currentfill}{rgb}{0.187231,0.414746,0.556547}%
\pgfsetfillcolor{currentfill}%
\pgfsetfillopacity{0.700000}%
\pgfsetlinewidth{0.000000pt}%
\definecolor{currentstroke}{rgb}{0.000000,0.000000,0.000000}%
\pgfsetstrokecolor{currentstroke}%
\pgfsetdash{}{0pt}%
\pgfpathmoveto{\pgfqpoint{3.000843in}{2.083077in}}%
\pgfpathlineto{\pgfqpoint{3.015122in}{2.068021in}}%
\pgfpathlineto{\pgfqpoint{3.029399in}{2.053086in}}%
\pgfpathlineto{\pgfqpoint{3.043674in}{2.038271in}}%
\pgfpathlineto{\pgfqpoint{3.057948in}{2.023577in}}%
\pgfpathlineto{\pgfqpoint{3.048855in}{2.037239in}}%
\pgfpathlineto{\pgfqpoint{3.039738in}{2.051417in}}%
\pgfpathlineto{\pgfqpoint{3.030596in}{2.066118in}}%
\pgfpathlineto{\pgfqpoint{3.016277in}{2.081367in}}%
\pgfpathlineto{\pgfqpoint{3.001957in}{2.096735in}}%
\pgfpathlineto{\pgfqpoint{2.987635in}{2.112225in}}%
\pgfpathlineto{\pgfqpoint{2.973311in}{2.127837in}}%
\pgfpathlineto{\pgfqpoint{2.982514in}{2.112389in}}%
\pgfpathlineto{\pgfqpoint{2.991691in}{2.097472in}}%
\pgfpathlineto{\pgfqpoint{3.000843in}{2.083077in}}%
\pgfpathclose%
\pgfusepath{fill}%
\end{pgfscope}%
\begin{pgfscope}%
\pgfpathrectangle{\pgfqpoint{1.150000in}{0.150000in}}{\pgfqpoint{5.700000in}{5.700000in}}%
\pgfusepath{clip}%
\pgfsetbuttcap%
\pgfsetroundjoin%
\definecolor{currentfill}{rgb}{0.131172,0.555899,0.552459}%
\pgfsetfillcolor{currentfill}%
\pgfsetfillopacity{0.700000}%
\pgfsetlinewidth{0.000000pt}%
\definecolor{currentstroke}{rgb}{0.000000,0.000000,0.000000}%
\pgfsetstrokecolor{currentstroke}%
\pgfsetdash{}{0pt}%
\pgfpathmoveto{\pgfqpoint{2.657366in}{2.482557in}}%
\pgfpathlineto{\pgfqpoint{2.671719in}{2.464381in}}%
\pgfpathlineto{\pgfqpoint{2.686068in}{2.446343in}}%
\pgfpathlineto{\pgfqpoint{2.700412in}{2.428444in}}%
\pgfpathlineto{\pgfqpoint{2.714752in}{2.410682in}}%
\pgfpathlineto{\pgfqpoint{2.705276in}{2.428797in}}%
\pgfpathlineto{\pgfqpoint{2.695769in}{2.447469in}}%
\pgfpathlineto{\pgfqpoint{2.686231in}{2.466707in}}%
\pgfpathlineto{\pgfqpoint{2.671837in}{2.485041in}}%
\pgfpathlineto{\pgfqpoint{2.657440in}{2.503513in}}%
\pgfpathlineto{\pgfqpoint{2.643037in}{2.522124in}}%
\pgfpathlineto{\pgfqpoint{2.628630in}{2.540876in}}%
\pgfpathlineto{\pgfqpoint{2.638241in}{2.520866in}}%
\pgfpathlineto{\pgfqpoint{2.647819in}{2.501429in}}%
\pgfpathlineto{\pgfqpoint{2.657366in}{2.482557in}}%
\pgfpathclose%
\pgfusepath{fill}%
\end{pgfscope}%
\begin{pgfscope}%
\pgfpathrectangle{\pgfqpoint{1.150000in}{0.150000in}}{\pgfqpoint{5.700000in}{5.700000in}}%
\pgfusepath{clip}%
\pgfsetbuttcap%
\pgfsetroundjoin%
\definecolor{currentfill}{rgb}{0.412913,0.803041,0.357269}%
\pgfsetfillcolor{currentfill}%
\pgfsetfillopacity{0.700000}%
\pgfsetlinewidth{0.000000pt}%
\definecolor{currentstroke}{rgb}{0.000000,0.000000,0.000000}%
\pgfsetstrokecolor{currentstroke}%
\pgfsetdash{}{0pt}%
\pgfpathmoveto{\pgfqpoint{2.118867in}{3.235620in}}%
\pgfpathlineto{\pgfqpoint{2.133438in}{3.211725in}}%
\pgfpathlineto{\pgfqpoint{2.147998in}{3.188019in}}%
\pgfpathlineto{\pgfqpoint{2.162549in}{3.164499in}}%
\pgfpathlineto{\pgfqpoint{2.177089in}{3.141166in}}%
\pgfpathlineto{\pgfqpoint{2.166974in}{3.164660in}}%
\pgfpathlineto{\pgfqpoint{2.156820in}{3.188753in}}%
\pgfpathlineto{\pgfqpoint{2.146625in}{3.213453in}}%
\pgfpathlineto{\pgfqpoint{2.136390in}{3.238769in}}%
\pgfpathlineto{\pgfqpoint{2.121761in}{3.262899in}}%
\pgfpathlineto{\pgfqpoint{2.107122in}{3.287216in}}%
\pgfpathlineto{\pgfqpoint{2.092472in}{3.311723in}}%
\pgfpathlineto{\pgfqpoint{2.077811in}{3.336420in}}%
\pgfpathlineto{\pgfqpoint{2.088138in}{3.310292in}}%
\pgfpathlineto{\pgfqpoint{2.098422in}{3.284789in}}%
\pgfpathlineto{\pgfqpoint{2.108665in}{3.259901in}}%
\pgfpathlineto{\pgfqpoint{2.118867in}{3.235620in}}%
\pgfpathclose%
\pgfusepath{fill}%
\end{pgfscope}%
\begin{pgfscope}%
\pgfpathrectangle{\pgfqpoint{1.150000in}{0.150000in}}{\pgfqpoint{5.700000in}{5.700000in}}%
\pgfusepath{clip}%
\pgfsetbuttcap%
\pgfsetroundjoin%
\definecolor{currentfill}{rgb}{0.269308,0.218818,0.509577}%
\pgfsetfillcolor{currentfill}%
\pgfsetfillopacity{0.700000}%
\pgfsetlinewidth{0.000000pt}%
\definecolor{currentstroke}{rgb}{0.000000,0.000000,0.000000}%
\pgfsetstrokecolor{currentstroke}%
\pgfsetdash{}{0pt}%
\pgfpathmoveto{\pgfqpoint{4.999969in}{1.610878in}}%
\pgfpathlineto{\pgfqpoint{5.014636in}{1.614153in}}%
\pgfpathlineto{\pgfqpoint{5.029316in}{1.617523in}}%
\pgfpathlineto{\pgfqpoint{5.044009in}{1.620988in}}%
\pgfpathlineto{\pgfqpoint{5.058716in}{1.624549in}}%
\pgfpathlineto{\pgfqpoint{5.050682in}{1.608723in}}%
\pgfpathlineto{\pgfqpoint{5.042646in}{1.592975in}}%
\pgfpathlineto{\pgfqpoint{5.034608in}{1.577310in}}%
\pgfpathlineto{\pgfqpoint{5.026568in}{1.561733in}}%
\pgfpathlineto{\pgfqpoint{5.011868in}{1.558658in}}%
\pgfpathlineto{\pgfqpoint{4.997180in}{1.555677in}}%
\pgfpathlineto{\pgfqpoint{4.982506in}{1.552791in}}%
\pgfpathlineto{\pgfqpoint{4.967844in}{1.550000in}}%
\pgfpathlineto{\pgfqpoint{4.975879in}{1.565086in}}%
\pgfpathlineto{\pgfqpoint{4.983911in}{1.580264in}}%
\pgfpathlineto{\pgfqpoint{4.991942in}{1.595530in}}%
\pgfpathlineto{\pgfqpoint{4.999969in}{1.610878in}}%
\pgfpathclose%
\pgfusepath{fill}%
\end{pgfscope}%
\begin{pgfscope}%
\pgfpathrectangle{\pgfqpoint{1.150000in}{0.150000in}}{\pgfqpoint{5.700000in}{5.700000in}}%
\pgfusepath{clip}%
\pgfsetbuttcap%
\pgfsetroundjoin%
\definecolor{currentfill}{rgb}{0.195860,0.395433,0.555276}%
\pgfsetfillcolor{currentfill}%
\pgfsetfillopacity{0.700000}%
\pgfsetlinewidth{0.000000pt}%
\definecolor{currentstroke}{rgb}{0.000000,0.000000,0.000000}%
\pgfsetstrokecolor{currentstroke}%
\pgfsetdash{}{0pt}%
\pgfpathmoveto{\pgfqpoint{3.057948in}{2.023577in}}%
\pgfpathlineto{\pgfqpoint{3.072221in}{2.009001in}}%
\pgfpathlineto{\pgfqpoint{3.086492in}{1.994545in}}%
\pgfpathlineto{\pgfqpoint{3.100762in}{1.980206in}}%
\pgfpathlineto{\pgfqpoint{3.115032in}{1.965985in}}%
\pgfpathlineto{\pgfqpoint{3.105996in}{1.978919in}}%
\pgfpathlineto{\pgfqpoint{3.096937in}{1.992361in}}%
\pgfpathlineto{\pgfqpoint{3.087855in}{2.006320in}}%
\pgfpathlineto{\pgfqpoint{3.073542in}{2.021092in}}%
\pgfpathlineto{\pgfqpoint{3.059228in}{2.035982in}}%
\pgfpathlineto{\pgfqpoint{3.044913in}{2.050990in}}%
\pgfpathlineto{\pgfqpoint{3.030596in}{2.066118in}}%
\pgfpathlineto{\pgfqpoint{3.039738in}{2.051417in}}%
\pgfpathlineto{\pgfqpoint{3.048855in}{2.037239in}}%
\pgfpathlineto{\pgfqpoint{3.057948in}{2.023577in}}%
\pgfpathclose%
\pgfusepath{fill}%
\end{pgfscope}%
\begin{pgfscope}%
\pgfpathrectangle{\pgfqpoint{1.150000in}{0.150000in}}{\pgfqpoint{5.700000in}{5.700000in}}%
\pgfusepath{clip}%
\pgfsetbuttcap%
\pgfsetroundjoin%
\definecolor{currentfill}{rgb}{0.279566,0.067836,0.391917}%
\pgfsetfillcolor{currentfill}%
\pgfsetfillopacity{0.700000}%
\pgfsetlinewidth{0.000000pt}%
\definecolor{currentstroke}{rgb}{0.000000,0.000000,0.000000}%
\pgfsetstrokecolor{currentstroke}%
\pgfsetdash{}{0pt}%
\pgfpathmoveto{\pgfqpoint{4.220446in}{1.286380in}}%
\pgfpathlineto{\pgfqpoint{4.234791in}{1.281931in}}%
\pgfpathlineto{\pgfqpoint{4.249144in}{1.277577in}}%
\pgfpathlineto{\pgfqpoint{4.263503in}{1.273317in}}%
\pgfpathlineto{\pgfqpoint{4.277870in}{1.269152in}}%
\pgfpathlineto{\pgfqpoint{4.269689in}{1.263891in}}%
\pgfpathlineto{\pgfqpoint{4.261501in}{1.258921in}}%
\pgfpathlineto{\pgfqpoint{4.253307in}{1.254248in}}%
\pgfpathlineto{\pgfqpoint{4.245106in}{1.249879in}}%
\pgfpathlineto{\pgfqpoint{4.230722in}{1.254667in}}%
\pgfpathlineto{\pgfqpoint{4.216345in}{1.259548in}}%
\pgfpathlineto{\pgfqpoint{4.201975in}{1.264524in}}%
\pgfpathlineto{\pgfqpoint{4.187611in}{1.269595in}}%
\pgfpathlineto{\pgfqpoint{4.195830in}{1.273334in}}%
\pgfpathlineto{\pgfqpoint{4.204043in}{1.277383in}}%
\pgfpathlineto{\pgfqpoint{4.212248in}{1.281733in}}%
\pgfpathlineto{\pgfqpoint{4.220446in}{1.286380in}}%
\pgfpathclose%
\pgfusepath{fill}%
\end{pgfscope}%
\begin{pgfscope}%
\pgfpathrectangle{\pgfqpoint{1.150000in}{0.150000in}}{\pgfqpoint{5.700000in}{5.700000in}}%
\pgfusepath{clip}%
\pgfsetbuttcap%
\pgfsetroundjoin%
\definecolor{currentfill}{rgb}{0.123463,0.581687,0.547445}%
\pgfsetfillcolor{currentfill}%
\pgfsetfillopacity{0.700000}%
\pgfsetlinewidth{0.000000pt}%
\definecolor{currentstroke}{rgb}{0.000000,0.000000,0.000000}%
\pgfsetstrokecolor{currentstroke}%
\pgfsetdash{}{0pt}%
\pgfpathmoveto{\pgfqpoint{2.599906in}{2.556677in}}%
\pgfpathlineto{\pgfqpoint{2.614278in}{2.537933in}}%
\pgfpathlineto{\pgfqpoint{2.628646in}{2.519333in}}%
\pgfpathlineto{\pgfqpoint{2.643009in}{2.500875in}}%
\pgfpathlineto{\pgfqpoint{2.657366in}{2.482557in}}%
\pgfpathlineto{\pgfqpoint{2.647819in}{2.501429in}}%
\pgfpathlineto{\pgfqpoint{2.638241in}{2.520866in}}%
\pgfpathlineto{\pgfqpoint{2.628630in}{2.540876in}}%
\pgfpathlineto{\pgfqpoint{2.614217in}{2.559769in}}%
\pgfpathlineto{\pgfqpoint{2.599799in}{2.578804in}}%
\pgfpathlineto{\pgfqpoint{2.585376in}{2.597982in}}%
\pgfpathlineto{\pgfqpoint{2.570948in}{2.617304in}}%
\pgfpathlineto{\pgfqpoint{2.580634in}{2.596517in}}%
\pgfpathlineto{\pgfqpoint{2.590286in}{2.576311in}}%
\pgfpathlineto{\pgfqpoint{2.599906in}{2.556677in}}%
\pgfpathclose%
\pgfusepath{fill}%
\end{pgfscope}%
\begin{pgfscope}%
\pgfpathrectangle{\pgfqpoint{1.150000in}{0.150000in}}{\pgfqpoint{5.700000in}{5.700000in}}%
\pgfusepath{clip}%
\pgfsetbuttcap%
\pgfsetroundjoin%
\definecolor{currentfill}{rgb}{0.204903,0.375746,0.553533}%
\pgfsetfillcolor{currentfill}%
\pgfsetfillopacity{0.700000}%
\pgfsetlinewidth{0.000000pt}%
\definecolor{currentstroke}{rgb}{0.000000,0.000000,0.000000}%
\pgfsetstrokecolor{currentstroke}%
\pgfsetdash{}{0pt}%
\pgfpathmoveto{\pgfqpoint{3.115032in}{1.965985in}}%
\pgfpathlineto{\pgfqpoint{3.129300in}{1.951881in}}%
\pgfpathlineto{\pgfqpoint{3.143568in}{1.937894in}}%
\pgfpathlineto{\pgfqpoint{3.157835in}{1.924022in}}%
\pgfpathlineto{\pgfqpoint{3.172101in}{1.910266in}}%
\pgfpathlineto{\pgfqpoint{3.163121in}{1.922474in}}%
\pgfpathlineto{\pgfqpoint{3.154119in}{1.935183in}}%
\pgfpathlineto{\pgfqpoint{3.145094in}{1.948403in}}%
\pgfpathlineto{\pgfqpoint{3.130785in}{1.962708in}}%
\pgfpathlineto{\pgfqpoint{3.116476in}{1.977129in}}%
\pgfpathlineto{\pgfqpoint{3.102166in}{1.991666in}}%
\pgfpathlineto{\pgfqpoint{3.087855in}{2.006320in}}%
\pgfpathlineto{\pgfqpoint{3.096937in}{1.992361in}}%
\pgfpathlineto{\pgfqpoint{3.105996in}{1.978919in}}%
\pgfpathlineto{\pgfqpoint{3.115032in}{1.965985in}}%
\pgfpathclose%
\pgfusepath{fill}%
\end{pgfscope}%
\begin{pgfscope}%
\pgfpathrectangle{\pgfqpoint{1.150000in}{0.150000in}}{\pgfqpoint{5.700000in}{5.700000in}}%
\pgfusepath{clip}%
\pgfsetbuttcap%
\pgfsetroundjoin%
\definecolor{currentfill}{rgb}{0.279566,0.067836,0.391917}%
\pgfsetfillcolor{currentfill}%
\pgfsetfillopacity{0.700000}%
\pgfsetlinewidth{0.000000pt}%
\definecolor{currentstroke}{rgb}{0.000000,0.000000,0.000000}%
\pgfsetstrokecolor{currentstroke}%
\pgfsetdash{}{0pt}%
\pgfpathmoveto{\pgfqpoint{4.368020in}{1.279673in}}%
\pgfpathlineto{\pgfqpoint{4.382411in}{1.276581in}}%
\pgfpathlineto{\pgfqpoint{4.396811in}{1.273582in}}%
\pgfpathlineto{\pgfqpoint{4.411218in}{1.270678in}}%
\pgfpathlineto{\pgfqpoint{4.425634in}{1.267867in}}%
\pgfpathlineto{\pgfqpoint{4.417502in}{1.260313in}}%
\pgfpathlineto{\pgfqpoint{4.409364in}{1.253016in}}%
\pgfpathlineto{\pgfqpoint{4.401222in}{1.245981in}}%
\pgfpathlineto{\pgfqpoint{4.393075in}{1.239215in}}%
\pgfpathlineto{\pgfqpoint{4.378647in}{1.242629in}}%
\pgfpathlineto{\pgfqpoint{4.364228in}{1.246136in}}%
\pgfpathlineto{\pgfqpoint{4.349816in}{1.249737in}}%
\pgfpathlineto{\pgfqpoint{4.335412in}{1.253432in}}%
\pgfpathlineto{\pgfqpoint{4.343572in}{1.259588in}}%
\pgfpathlineto{\pgfqpoint{4.351727in}{1.266017in}}%
\pgfpathlineto{\pgfqpoint{4.359876in}{1.272714in}}%
\pgfpathlineto{\pgfqpoint{4.368020in}{1.279673in}}%
\pgfpathclose%
\pgfusepath{fill}%
\end{pgfscope}%
\begin{pgfscope}%
\pgfpathrectangle{\pgfqpoint{1.150000in}{0.150000in}}{\pgfqpoint{5.700000in}{5.700000in}}%
\pgfusepath{clip}%
\pgfsetbuttcap%
\pgfsetroundjoin%
\definecolor{currentfill}{rgb}{0.283187,0.125848,0.444960}%
\pgfsetfillcolor{currentfill}%
\pgfsetfillopacity{0.700000}%
\pgfsetlinewidth{0.000000pt}%
\definecolor{currentstroke}{rgb}{0.000000,0.000000,0.000000}%
\pgfsetstrokecolor{currentstroke}%
\pgfsetdash{}{0pt}%
\pgfpathmoveto{\pgfqpoint{3.868169in}{1.392217in}}%
\pgfpathlineto{\pgfqpoint{3.882446in}{1.384578in}}%
\pgfpathlineto{\pgfqpoint{3.896726in}{1.377036in}}%
\pgfpathlineto{\pgfqpoint{3.911012in}{1.369593in}}%
\pgfpathlineto{\pgfqpoint{3.925302in}{1.362248in}}%
\pgfpathlineto{\pgfqpoint{3.916944in}{1.362771in}}%
\pgfpathlineto{\pgfqpoint{3.908576in}{1.363664in}}%
\pgfpathlineto{\pgfqpoint{3.900196in}{1.364935in}}%
\pgfpathlineto{\pgfqpoint{3.891806in}{1.366589in}}%
\pgfpathlineto{\pgfqpoint{3.877486in}{1.374599in}}%
\pgfpathlineto{\pgfqpoint{3.863171in}{1.382707in}}%
\pgfpathlineto{\pgfqpoint{3.848860in}{1.390912in}}%
\pgfpathlineto{\pgfqpoint{3.834553in}{1.399217in}}%
\pgfpathlineto{\pgfqpoint{3.842974in}{1.396889in}}%
\pgfpathlineto{\pgfqpoint{3.851384in}{1.394952in}}%
\pgfpathlineto{\pgfqpoint{3.859782in}{1.393397in}}%
\pgfpathlineto{\pgfqpoint{3.868169in}{1.392217in}}%
\pgfpathclose%
\pgfusepath{fill}%
\end{pgfscope}%
\begin{pgfscope}%
\pgfpathrectangle{\pgfqpoint{1.150000in}{0.150000in}}{\pgfqpoint{5.700000in}{5.700000in}}%
\pgfusepath{clip}%
\pgfsetbuttcap%
\pgfsetroundjoin%
\definecolor{currentfill}{rgb}{0.281446,0.084320,0.407414}%
\pgfsetfillcolor{currentfill}%
\pgfsetfillopacity{0.700000}%
\pgfsetlinewidth{0.000000pt}%
\definecolor{currentstroke}{rgb}{0.000000,0.000000,0.000000}%
\pgfsetstrokecolor{currentstroke}%
\pgfsetdash{}{0pt}%
\pgfpathmoveto{\pgfqpoint{4.072933in}{1.313583in}}%
\pgfpathlineto{\pgfqpoint{4.087246in}{1.307750in}}%
\pgfpathlineto{\pgfqpoint{4.101565in}{1.302013in}}%
\pgfpathlineto{\pgfqpoint{4.115891in}{1.296372in}}%
\pgfpathlineto{\pgfqpoint{4.130222in}{1.290826in}}%
\pgfpathlineto{\pgfqpoint{4.121975in}{1.288036in}}%
\pgfpathlineto{\pgfqpoint{4.113720in}{1.285573in}}%
\pgfpathlineto{\pgfqpoint{4.105457in}{1.283443in}}%
\pgfpathlineto{\pgfqpoint{4.097185in}{1.281654in}}%
\pgfpathlineto{\pgfqpoint{4.082831in}{1.287841in}}%
\pgfpathlineto{\pgfqpoint{4.068484in}{1.294124in}}%
\pgfpathlineto{\pgfqpoint{4.054141in}{1.300503in}}%
\pgfpathlineto{\pgfqpoint{4.039805in}{1.306978in}}%
\pgfpathlineto{\pgfqpoint{4.048100in}{1.308118in}}%
\pgfpathlineto{\pgfqpoint{4.056386in}{1.309603in}}%
\pgfpathlineto{\pgfqpoint{4.064664in}{1.311427in}}%
\pgfpathlineto{\pgfqpoint{4.072933in}{1.313583in}}%
\pgfpathclose%
\pgfusepath{fill}%
\end{pgfscope}%
\begin{pgfscope}%
\pgfpathrectangle{\pgfqpoint{1.150000in}{0.150000in}}{\pgfqpoint{5.700000in}{5.700000in}}%
\pgfusepath{clip}%
\pgfsetbuttcap%
\pgfsetroundjoin%
\definecolor{currentfill}{rgb}{0.119423,0.611141,0.538982}%
\pgfsetfillcolor{currentfill}%
\pgfsetfillopacity{0.700000}%
\pgfsetlinewidth{0.000000pt}%
\definecolor{currentstroke}{rgb}{0.000000,0.000000,0.000000}%
\pgfsetstrokecolor{currentstroke}%
\pgfsetdash{}{0pt}%
\pgfpathmoveto{\pgfqpoint{2.542362in}{2.633103in}}%
\pgfpathlineto{\pgfqpoint{2.556756in}{2.613777in}}%
\pgfpathlineto{\pgfqpoint{2.571145in}{2.594598in}}%
\pgfpathlineto{\pgfqpoint{2.585528in}{2.575565in}}%
\pgfpathlineto{\pgfqpoint{2.599906in}{2.556677in}}%
\pgfpathlineto{\pgfqpoint{2.590286in}{2.576311in}}%
\pgfpathlineto{\pgfqpoint{2.580634in}{2.596517in}}%
\pgfpathlineto{\pgfqpoint{2.570948in}{2.617304in}}%
\pgfpathlineto{\pgfqpoint{2.556514in}{2.636772in}}%
\pgfpathlineto{\pgfqpoint{2.542074in}{2.656385in}}%
\pgfpathlineto{\pgfqpoint{2.527628in}{2.676146in}}%
\pgfpathlineto{\pgfqpoint{2.513176in}{2.696055in}}%
\pgfpathlineto{\pgfqpoint{2.522939in}{2.674485in}}%
\pgfpathlineto{\pgfqpoint{2.532667in}{2.653504in}}%
\pgfpathlineto{\pgfqpoint{2.542362in}{2.633103in}}%
\pgfpathclose%
\pgfusepath{fill}%
\end{pgfscope}%
\begin{pgfscope}%
\pgfpathrectangle{\pgfqpoint{1.150000in}{0.150000in}}{\pgfqpoint{5.700000in}{5.700000in}}%
\pgfusepath{clip}%
\pgfsetbuttcap%
\pgfsetroundjoin%
\definecolor{currentfill}{rgb}{0.257322,0.256130,0.526563}%
\pgfsetfillcolor{currentfill}%
\pgfsetfillopacity{0.700000}%
\pgfsetlinewidth{0.000000pt}%
\definecolor{currentstroke}{rgb}{0.000000,0.000000,0.000000}%
\pgfsetstrokecolor{currentstroke}%
\pgfsetdash{}{0pt}%
\pgfpathmoveto{\pgfqpoint{5.090825in}{1.688549in}}%
\pgfpathlineto{\pgfqpoint{5.105551in}{1.692673in}}%
\pgfpathlineto{\pgfqpoint{5.120291in}{1.696893in}}%
\pgfpathlineto{\pgfqpoint{5.135045in}{1.701209in}}%
\pgfpathlineto{\pgfqpoint{5.127015in}{1.684768in}}%
\pgfpathlineto{\pgfqpoint{5.118984in}{1.668383in}}%
\pgfpathlineto{\pgfqpoint{5.110950in}{1.652061in}}%
\pgfpathlineto{\pgfqpoint{5.102913in}{1.635804in}}%
\pgfpathlineto{\pgfqpoint{5.088167in}{1.631957in}}%
\pgfpathlineto{\pgfqpoint{5.073435in}{1.628205in}}%
\pgfpathlineto{\pgfqpoint{5.058716in}{1.624549in}}%
\pgfpathlineto{\pgfqpoint{5.066747in}{1.640449in}}%
\pgfpathlineto{\pgfqpoint{5.074775in}{1.656419in}}%
\pgfpathlineto{\pgfqpoint{5.082801in}{1.672454in}}%
\pgfpathlineto{\pgfqpoint{5.090825in}{1.688549in}}%
\pgfpathclose%
\pgfusepath{fill}%
\end{pgfscope}%
\begin{pgfscope}%
\pgfpathrectangle{\pgfqpoint{1.150000in}{0.150000in}}{\pgfqpoint{5.700000in}{5.700000in}}%
\pgfusepath{clip}%
\pgfsetbuttcap%
\pgfsetroundjoin%
\definecolor{currentfill}{rgb}{0.274128,0.199721,0.498911}%
\pgfsetfillcolor{currentfill}%
\pgfsetfillopacity{0.700000}%
\pgfsetlinewidth{0.000000pt}%
\definecolor{currentstroke}{rgb}{0.000000,0.000000,0.000000}%
\pgfsetstrokecolor{currentstroke}%
\pgfsetdash{}{0pt}%
\pgfpathmoveto{\pgfqpoint{3.606134in}{1.545663in}}%
\pgfpathlineto{\pgfqpoint{3.620387in}{1.535751in}}%
\pgfpathlineto{\pgfqpoint{3.634642in}{1.525941in}}%
\pgfpathlineto{\pgfqpoint{3.648900in}{1.516233in}}%
\pgfpathlineto{\pgfqpoint{3.663161in}{1.506628in}}%
\pgfpathlineto{\pgfqpoint{3.654622in}{1.511393in}}%
\pgfpathlineto{\pgfqpoint{3.646069in}{1.516579in}}%
\pgfpathlineto{\pgfqpoint{3.637500in}{1.522194in}}%
\pgfpathlineto{\pgfqpoint{3.628917in}{1.528245in}}%
\pgfpathlineto{\pgfqpoint{3.614617in}{1.538542in}}%
\pgfpathlineto{\pgfqpoint{3.600320in}{1.548941in}}%
\pgfpathlineto{\pgfqpoint{3.586025in}{1.559443in}}%
\pgfpathlineto{\pgfqpoint{3.571732in}{1.570047in}}%
\pgfpathlineto{\pgfqpoint{3.580356in}{1.563296in}}%
\pgfpathlineto{\pgfqpoint{3.588964in}{1.556986in}}%
\pgfpathlineto{\pgfqpoint{3.597557in}{1.551111in}}%
\pgfpathlineto{\pgfqpoint{3.606134in}{1.545663in}}%
\pgfpathclose%
\pgfusepath{fill}%
\end{pgfscope}%
\begin{pgfscope}%
\pgfpathrectangle{\pgfqpoint{1.150000in}{0.150000in}}{\pgfqpoint{5.700000in}{5.700000in}}%
\pgfusepath{clip}%
\pgfsetbuttcap%
\pgfsetroundjoin%
\definecolor{currentfill}{rgb}{0.283091,0.110553,0.431554}%
\pgfsetfillcolor{currentfill}%
\pgfsetfillopacity{0.700000}%
\pgfsetlinewidth{0.000000pt}%
\definecolor{currentstroke}{rgb}{0.000000,0.000000,0.000000}%
\pgfsetstrokecolor{currentstroke}%
\pgfsetdash{}{0pt}%
\pgfpathmoveto{\pgfqpoint{4.696299in}{1.370404in}}%
\pgfpathlineto{\pgfqpoint{4.710818in}{1.370513in}}%
\pgfpathlineto{\pgfqpoint{4.725347in}{1.370715in}}%
\pgfpathlineto{\pgfqpoint{4.739888in}{1.371011in}}%
\pgfpathlineto{\pgfqpoint{4.754438in}{1.371401in}}%
\pgfpathlineto{\pgfqpoint{4.746374in}{1.359045in}}%
\pgfpathlineto{\pgfqpoint{4.738308in}{1.346860in}}%
\pgfpathlineto{\pgfqpoint{4.730238in}{1.334851in}}%
\pgfpathlineto{\pgfqpoint{4.722165in}{1.323023in}}%
\pgfpathlineto{\pgfqpoint{4.707613in}{1.323185in}}%
\pgfpathlineto{\pgfqpoint{4.693071in}{1.323441in}}%
\pgfpathlineto{\pgfqpoint{4.678539in}{1.323790in}}%
\pgfpathlineto{\pgfqpoint{4.664018in}{1.324232in}}%
\pgfpathlineto{\pgfqpoint{4.672093in}{1.335502in}}%
\pgfpathlineto{\pgfqpoint{4.680165in}{1.346957in}}%
\pgfpathlineto{\pgfqpoint{4.688233in}{1.358593in}}%
\pgfpathlineto{\pgfqpoint{4.696299in}{1.370404in}}%
\pgfpathclose%
\pgfusepath{fill}%
\end{pgfscope}%
\begin{pgfscope}%
\pgfpathrectangle{\pgfqpoint{1.150000in}{0.150000in}}{\pgfqpoint{5.700000in}{5.700000in}}%
\pgfusepath{clip}%
\pgfsetbuttcap%
\pgfsetroundjoin%
\definecolor{currentfill}{rgb}{0.214298,0.355619,0.551184}%
\pgfsetfillcolor{currentfill}%
\pgfsetfillopacity{0.700000}%
\pgfsetlinewidth{0.000000pt}%
\definecolor{currentstroke}{rgb}{0.000000,0.000000,0.000000}%
\pgfsetstrokecolor{currentstroke}%
\pgfsetdash{}{0pt}%
\pgfpathmoveto{\pgfqpoint{3.172101in}{1.910266in}}%
\pgfpathlineto{\pgfqpoint{3.186367in}{1.896625in}}%
\pgfpathlineto{\pgfqpoint{3.200632in}{1.883098in}}%
\pgfpathlineto{\pgfqpoint{3.214898in}{1.869685in}}%
\pgfpathlineto{\pgfqpoint{3.229163in}{1.856385in}}%
\pgfpathlineto{\pgfqpoint{3.220237in}{1.867870in}}%
\pgfpathlineto{\pgfqpoint{3.211289in}{1.879850in}}%
\pgfpathlineto{\pgfqpoint{3.202320in}{1.892335in}}%
\pgfpathlineto{\pgfqpoint{3.188014in}{1.906181in}}%
\pgfpathlineto{\pgfqpoint{3.173708in}{1.920140in}}%
\pgfpathlineto{\pgfqpoint{3.159401in}{1.934214in}}%
\pgfpathlineto{\pgfqpoint{3.145094in}{1.948403in}}%
\pgfpathlineto{\pgfqpoint{3.154119in}{1.935183in}}%
\pgfpathlineto{\pgfqpoint{3.163121in}{1.922474in}}%
\pgfpathlineto{\pgfqpoint{3.172101in}{1.910266in}}%
\pgfpathclose%
\pgfusepath{fill}%
\end{pgfscope}%
\begin{pgfscope}%
\pgfpathrectangle{\pgfqpoint{1.150000in}{0.150000in}}{\pgfqpoint{5.700000in}{5.700000in}}%
\pgfusepath{clip}%
\pgfsetbuttcap%
\pgfsetroundjoin%
\definecolor{currentfill}{rgb}{0.282884,0.135920,0.453427}%
\pgfsetfillcolor{currentfill}%
\pgfsetfillopacity{0.700000}%
\pgfsetlinewidth{0.000000pt}%
\definecolor{currentstroke}{rgb}{0.000000,0.000000,0.000000}%
\pgfsetstrokecolor{currentstroke}%
\pgfsetdash{}{0pt}%
\pgfpathmoveto{\pgfqpoint{4.786667in}{1.422422in}}%
\pgfpathlineto{\pgfqpoint{4.801228in}{1.423440in}}%
\pgfpathlineto{\pgfqpoint{4.815801in}{1.424551in}}%
\pgfpathlineto{\pgfqpoint{4.830385in}{1.425757in}}%
\pgfpathlineto{\pgfqpoint{4.844981in}{1.427057in}}%
\pgfpathlineto{\pgfqpoint{4.836927in}{1.413544in}}%
\pgfpathlineto{\pgfqpoint{4.828871in}{1.400176in}}%
\pgfpathlineto{\pgfqpoint{4.820812in}{1.386958in}}%
\pgfpathlineto{\pgfqpoint{4.812750in}{1.373896in}}%
\pgfpathlineto{\pgfqpoint{4.798156in}{1.373132in}}%
\pgfpathlineto{\pgfqpoint{4.783572in}{1.372461in}}%
\pgfpathlineto{\pgfqpoint{4.769000in}{1.371884in}}%
\pgfpathlineto{\pgfqpoint{4.754438in}{1.371401in}}%
\pgfpathlineto{\pgfqpoint{4.762500in}{1.383921in}}%
\pgfpathlineto{\pgfqpoint{4.770558in}{1.396602in}}%
\pgfpathlineto{\pgfqpoint{4.778614in}{1.409437in}}%
\pgfpathlineto{\pgfqpoint{4.786667in}{1.422422in}}%
\pgfpathclose%
\pgfusepath{fill}%
\end{pgfscope}%
\begin{pgfscope}%
\pgfpathrectangle{\pgfqpoint{1.150000in}{0.150000in}}{\pgfqpoint{5.700000in}{5.700000in}}%
\pgfusepath{clip}%
\pgfsetbuttcap%
\pgfsetroundjoin%
\definecolor{currentfill}{rgb}{0.281924,0.089666,0.412415}%
\pgfsetfillcolor{currentfill}%
\pgfsetfillopacity{0.700000}%
\pgfsetlinewidth{0.000000pt}%
\definecolor{currentstroke}{rgb}{0.000000,0.000000,0.000000}%
\pgfsetstrokecolor{currentstroke}%
\pgfsetdash{}{0pt}%
\pgfpathmoveto{\pgfqpoint{4.606032in}{1.326935in}}%
\pgfpathlineto{\pgfqpoint{4.620513in}{1.326119in}}%
\pgfpathlineto{\pgfqpoint{4.635005in}{1.325397in}}%
\pgfpathlineto{\pgfqpoint{4.649506in}{1.324767in}}%
\pgfpathlineto{\pgfqpoint{4.664018in}{1.324232in}}%
\pgfpathlineto{\pgfqpoint{4.655939in}{1.313153in}}%
\pgfpathlineto{\pgfqpoint{4.647858in}{1.302271in}}%
\pgfpathlineto{\pgfqpoint{4.639773in}{1.291592in}}%
\pgfpathlineto{\pgfqpoint{4.631685in}{1.281120in}}%
\pgfpathlineto{\pgfqpoint{4.617169in}{1.282224in}}%
\pgfpathlineto{\pgfqpoint{4.602663in}{1.283421in}}%
\pgfpathlineto{\pgfqpoint{4.588167in}{1.284712in}}%
\pgfpathlineto{\pgfqpoint{4.573679in}{1.286096in}}%
\pgfpathlineto{\pgfqpoint{4.581773in}{1.295993in}}%
\pgfpathlineto{\pgfqpoint{4.589863in}{1.306102in}}%
\pgfpathlineto{\pgfqpoint{4.597949in}{1.316418in}}%
\pgfpathlineto{\pgfqpoint{4.606032in}{1.326935in}}%
\pgfpathclose%
\pgfusepath{fill}%
\end{pgfscope}%
\begin{pgfscope}%
\pgfpathrectangle{\pgfqpoint{1.150000in}{0.150000in}}{\pgfqpoint{5.700000in}{5.700000in}}%
\pgfusepath{clip}%
\pgfsetbuttcap%
\pgfsetroundjoin%
\definecolor{currentfill}{rgb}{0.124780,0.640461,0.527068}%
\pgfsetfillcolor{currentfill}%
\pgfsetfillopacity{0.700000}%
\pgfsetlinewidth{0.000000pt}%
\definecolor{currentstroke}{rgb}{0.000000,0.000000,0.000000}%
\pgfsetstrokecolor{currentstroke}%
\pgfsetdash{}{0pt}%
\pgfpathmoveto{\pgfqpoint{2.484724in}{2.711900in}}%
\pgfpathlineto{\pgfqpoint{2.499143in}{2.691975in}}%
\pgfpathlineto{\pgfqpoint{2.513555in}{2.672201in}}%
\pgfpathlineto{\pgfqpoint{2.527961in}{2.652577in}}%
\pgfpathlineto{\pgfqpoint{2.542362in}{2.633103in}}%
\pgfpathlineto{\pgfqpoint{2.532667in}{2.653504in}}%
\pgfpathlineto{\pgfqpoint{2.522939in}{2.674485in}}%
\pgfpathlineto{\pgfqpoint{2.513176in}{2.696055in}}%
\pgfpathlineto{\pgfqpoint{2.498718in}{2.716113in}}%
\pgfpathlineto{\pgfqpoint{2.484254in}{2.736322in}}%
\pgfpathlineto{\pgfqpoint{2.469783in}{2.756682in}}%
\pgfpathlineto{\pgfqpoint{2.455305in}{2.777194in}}%
\pgfpathlineto{\pgfqpoint{2.465147in}{2.754836in}}%
\pgfpathlineto{\pgfqpoint{2.474953in}{2.733074in}}%
\pgfpathlineto{\pgfqpoint{2.484724in}{2.711900in}}%
\pgfpathclose%
\pgfusepath{fill}%
\end{pgfscope}%
\begin{pgfscope}%
\pgfpathrectangle{\pgfqpoint{1.150000in}{0.150000in}}{\pgfqpoint{5.700000in}{5.700000in}}%
\pgfusepath{clip}%
\pgfsetbuttcap%
\pgfsetroundjoin%
\definecolor{currentfill}{rgb}{0.506271,0.828786,0.300362}%
\pgfsetfillcolor{currentfill}%
\pgfsetfillopacity{0.700000}%
\pgfsetlinewidth{0.000000pt}%
\definecolor{currentstroke}{rgb}{0.000000,0.000000,0.000000}%
\pgfsetstrokecolor{currentstroke}%
\pgfsetdash{}{0pt}%
\pgfpathmoveto{\pgfqpoint{2.060472in}{3.333127in}}%
\pgfpathlineto{\pgfqpoint{2.075088in}{3.308458in}}%
\pgfpathlineto{\pgfqpoint{2.089692in}{3.283985in}}%
\pgfpathlineto{\pgfqpoint{2.104285in}{3.259706in}}%
\pgfpathlineto{\pgfqpoint{2.118867in}{3.235620in}}%
\pgfpathlineto{\pgfqpoint{2.108665in}{3.259901in}}%
\pgfpathlineto{\pgfqpoint{2.098422in}{3.284789in}}%
\pgfpathlineto{\pgfqpoint{2.088138in}{3.310292in}}%
\pgfpathlineto{\pgfqpoint{2.077811in}{3.336420in}}%
\pgfpathlineto{\pgfqpoint{2.063139in}{3.361310in}}%
\pgfpathlineto{\pgfqpoint{2.048455in}{3.386395in}}%
\pgfpathlineto{\pgfqpoint{2.033759in}{3.411676in}}%
\pgfpathlineto{\pgfqpoint{2.019051in}{3.437155in}}%
\pgfpathlineto{\pgfqpoint{2.029470in}{3.410207in}}%
\pgfpathlineto{\pgfqpoint{2.039847in}{3.383892in}}%
\pgfpathlineto{\pgfqpoint{2.050180in}{3.358202in}}%
\pgfpathlineto{\pgfqpoint{2.060472in}{3.333127in}}%
\pgfpathclose%
\pgfusepath{fill}%
\end{pgfscope}%
\begin{pgfscope}%
\pgfpathrectangle{\pgfqpoint{1.150000in}{0.150000in}}{\pgfqpoint{5.700000in}{5.700000in}}%
\pgfusepath{clip}%
\pgfsetbuttcap%
\pgfsetroundjoin%
\definecolor{currentfill}{rgb}{0.223925,0.334994,0.548053}%
\pgfsetfillcolor{currentfill}%
\pgfsetfillopacity{0.700000}%
\pgfsetlinewidth{0.000000pt}%
\definecolor{currentstroke}{rgb}{0.000000,0.000000,0.000000}%
\pgfsetstrokecolor{currentstroke}%
\pgfsetdash{}{0pt}%
\pgfpathmoveto{\pgfqpoint{3.229163in}{1.856385in}}%
\pgfpathlineto{\pgfqpoint{3.243428in}{1.843198in}}%
\pgfpathlineto{\pgfqpoint{3.257693in}{1.830124in}}%
\pgfpathlineto{\pgfqpoint{3.271958in}{1.817161in}}%
\pgfpathlineto{\pgfqpoint{3.286224in}{1.804310in}}%
\pgfpathlineto{\pgfqpoint{3.277350in}{1.815076in}}%
\pgfpathlineto{\pgfqpoint{3.268455in}{1.826330in}}%
\pgfpathlineto{\pgfqpoint{3.259539in}{1.838081in}}%
\pgfpathlineto{\pgfqpoint{3.245235in}{1.851476in}}%
\pgfpathlineto{\pgfqpoint{3.230930in}{1.864983in}}%
\pgfpathlineto{\pgfqpoint{3.216625in}{1.878602in}}%
\pgfpathlineto{\pgfqpoint{3.202320in}{1.892335in}}%
\pgfpathlineto{\pgfqpoint{3.211289in}{1.879850in}}%
\pgfpathlineto{\pgfqpoint{3.220237in}{1.867870in}}%
\pgfpathlineto{\pgfqpoint{3.229163in}{1.856385in}}%
\pgfpathclose%
\pgfusepath{fill}%
\end{pgfscope}%
\begin{pgfscope}%
\pgfpathrectangle{\pgfqpoint{1.150000in}{0.150000in}}{\pgfqpoint{5.700000in}{5.700000in}}%
\pgfusepath{clip}%
\pgfsetbuttcap%
\pgfsetroundjoin%
\definecolor{currentfill}{rgb}{0.280255,0.165693,0.476498}%
\pgfsetfillcolor{currentfill}%
\pgfsetfillopacity{0.700000}%
\pgfsetlinewidth{0.000000pt}%
\definecolor{currentstroke}{rgb}{0.000000,0.000000,0.000000}%
\pgfsetstrokecolor{currentstroke}%
\pgfsetdash{}{0pt}%
\pgfpathmoveto{\pgfqpoint{4.877171in}{1.482459in}}%
\pgfpathlineto{\pgfqpoint{4.891780in}{1.484370in}}%
\pgfpathlineto{\pgfqpoint{4.906401in}{1.486376in}}%
\pgfpathlineto{\pgfqpoint{4.921034in}{1.488476in}}%
\pgfpathlineto{\pgfqpoint{4.935680in}{1.490671in}}%
\pgfpathlineto{\pgfqpoint{4.927633in}{1.476116in}}%
\pgfpathlineto{\pgfqpoint{4.919584in}{1.461682in}}%
\pgfpathlineto{\pgfqpoint{4.911532in}{1.447373in}}%
\pgfpathlineto{\pgfqpoint{4.903478in}{1.433195in}}%
\pgfpathlineto{\pgfqpoint{4.888836in}{1.431520in}}%
\pgfpathlineto{\pgfqpoint{4.874206in}{1.429938in}}%
\pgfpathlineto{\pgfqpoint{4.859588in}{1.428450in}}%
\pgfpathlineto{\pgfqpoint{4.844981in}{1.427057in}}%
\pgfpathlineto{\pgfqpoint{4.853032in}{1.440710in}}%
\pgfpathlineto{\pgfqpoint{4.861081in}{1.454498in}}%
\pgfpathlineto{\pgfqpoint{4.869127in}{1.468416in}}%
\pgfpathlineto{\pgfqpoint{4.877171in}{1.482459in}}%
\pgfpathclose%
\pgfusepath{fill}%
\end{pgfscope}%
\begin{pgfscope}%
\pgfpathrectangle{\pgfqpoint{1.150000in}{0.150000in}}{\pgfqpoint{5.700000in}{5.700000in}}%
\pgfusepath{clip}%
\pgfsetbuttcap%
\pgfsetroundjoin%
\definecolor{currentfill}{rgb}{0.280894,0.078907,0.402329}%
\pgfsetfillcolor{currentfill}%
\pgfsetfillopacity{0.700000}%
\pgfsetlinewidth{0.000000pt}%
\definecolor{currentstroke}{rgb}{0.000000,0.000000,0.000000}%
\pgfsetstrokecolor{currentstroke}%
\pgfsetdash{}{0pt}%
\pgfpathmoveto{\pgfqpoint{4.515825in}{1.292566in}}%
\pgfpathlineto{\pgfqpoint{4.530274in}{1.290809in}}%
\pgfpathlineto{\pgfqpoint{4.544734in}{1.289144in}}%
\pgfpathlineto{\pgfqpoint{4.559202in}{1.287574in}}%
\pgfpathlineto{\pgfqpoint{4.573679in}{1.286096in}}%
\pgfpathlineto{\pgfqpoint{4.565582in}{1.276418in}}%
\pgfpathlineto{\pgfqpoint{4.557481in}{1.266963in}}%
\pgfpathlineto{\pgfqpoint{4.549376in}{1.257738in}}%
\pgfpathlineto{\pgfqpoint{4.541267in}{1.248748in}}%
\pgfpathlineto{\pgfqpoint{4.526782in}{1.250812in}}%
\pgfpathlineto{\pgfqpoint{4.512306in}{1.252968in}}%
\pgfpathlineto{\pgfqpoint{4.497839in}{1.255218in}}%
\pgfpathlineto{\pgfqpoint{4.483381in}{1.257561in}}%
\pgfpathlineto{\pgfqpoint{4.491498in}{1.265958in}}%
\pgfpathlineto{\pgfqpoint{4.499611in}{1.274595in}}%
\pgfpathlineto{\pgfqpoint{4.507720in}{1.283467in}}%
\pgfpathlineto{\pgfqpoint{4.515825in}{1.292566in}}%
\pgfpathclose%
\pgfusepath{fill}%
\end{pgfscope}%
\begin{pgfscope}%
\pgfpathrectangle{\pgfqpoint{1.150000in}{0.150000in}}{\pgfqpoint{5.700000in}{5.700000in}}%
\pgfusepath{clip}%
\pgfsetbuttcap%
\pgfsetroundjoin%
\definecolor{currentfill}{rgb}{0.143303,0.669459,0.511215}%
\pgfsetfillcolor{currentfill}%
\pgfsetfillopacity{0.700000}%
\pgfsetlinewidth{0.000000pt}%
\definecolor{currentstroke}{rgb}{0.000000,0.000000,0.000000}%
\pgfsetstrokecolor{currentstroke}%
\pgfsetdash{}{0pt}%
\pgfpathmoveto{\pgfqpoint{2.426985in}{2.793140in}}%
\pgfpathlineto{\pgfqpoint{2.441430in}{2.772597in}}%
\pgfpathlineto{\pgfqpoint{2.455868in}{2.752210in}}%
\pgfpathlineto{\pgfqpoint{2.470300in}{2.731979in}}%
\pgfpathlineto{\pgfqpoint{2.484724in}{2.711900in}}%
\pgfpathlineto{\pgfqpoint{2.474953in}{2.733074in}}%
\pgfpathlineto{\pgfqpoint{2.465147in}{2.754836in}}%
\pgfpathlineto{\pgfqpoint{2.455305in}{2.777194in}}%
\pgfpathlineto{\pgfqpoint{2.440821in}{2.797860in}}%
\pgfpathlineto{\pgfqpoint{2.426330in}{2.818681in}}%
\pgfpathlineto{\pgfqpoint{2.411831in}{2.839659in}}%
\pgfpathlineto{\pgfqpoint{2.397326in}{2.860793in}}%
\pgfpathlineto{\pgfqpoint{2.407249in}{2.837641in}}%
\pgfpathlineto{\pgfqpoint{2.417135in}{2.815093in}}%
\pgfpathlineto{\pgfqpoint{2.426985in}{2.793140in}}%
\pgfpathclose%
\pgfusepath{fill}%
\end{pgfscope}%
\begin{pgfscope}%
\pgfpathrectangle{\pgfqpoint{1.150000in}{0.150000in}}{\pgfqpoint{5.700000in}{5.700000in}}%
\pgfusepath{clip}%
\pgfsetbuttcap%
\pgfsetroundjoin%
\definecolor{currentfill}{rgb}{0.277134,0.185228,0.489898}%
\pgfsetfillcolor{currentfill}%
\pgfsetfillopacity{0.700000}%
\pgfsetlinewidth{0.000000pt}%
\definecolor{currentstroke}{rgb}{0.000000,0.000000,0.000000}%
\pgfsetstrokecolor{currentstroke}%
\pgfsetdash{}{0pt}%
\pgfpathmoveto{\pgfqpoint{3.663161in}{1.506628in}}%
\pgfpathlineto{\pgfqpoint{3.677425in}{1.497124in}}%
\pgfpathlineto{\pgfqpoint{3.691692in}{1.487722in}}%
\pgfpathlineto{\pgfqpoint{3.705962in}{1.478420in}}%
\pgfpathlineto{\pgfqpoint{3.720236in}{1.469220in}}%
\pgfpathlineto{\pgfqpoint{3.711734in}{1.473303in}}%
\pgfpathlineto{\pgfqpoint{3.703218in}{1.477802in}}%
\pgfpathlineto{\pgfqpoint{3.694688in}{1.482724in}}%
\pgfpathlineto{\pgfqpoint{3.686143in}{1.488077in}}%
\pgfpathlineto{\pgfqpoint{3.671833in}{1.497967in}}%
\pgfpathlineto{\pgfqpoint{3.657525in}{1.507958in}}%
\pgfpathlineto{\pgfqpoint{3.643220in}{1.518051in}}%
\pgfpathlineto{\pgfqpoint{3.628917in}{1.528245in}}%
\pgfpathlineto{\pgfqpoint{3.637500in}{1.522194in}}%
\pgfpathlineto{\pgfqpoint{3.646069in}{1.516579in}}%
\pgfpathlineto{\pgfqpoint{3.654622in}{1.511393in}}%
\pgfpathlineto{\pgfqpoint{3.663161in}{1.506628in}}%
\pgfpathclose%
\pgfusepath{fill}%
\end{pgfscope}%
\begin{pgfscope}%
\pgfpathrectangle{\pgfqpoint{1.150000in}{0.150000in}}{\pgfqpoint{5.700000in}{5.700000in}}%
\pgfusepath{clip}%
\pgfsetbuttcap%
\pgfsetroundjoin%
\definecolor{currentfill}{rgb}{0.233603,0.313828,0.543914}%
\pgfsetfillcolor{currentfill}%
\pgfsetfillopacity{0.700000}%
\pgfsetlinewidth{0.000000pt}%
\definecolor{currentstroke}{rgb}{0.000000,0.000000,0.000000}%
\pgfsetstrokecolor{currentstroke}%
\pgfsetdash{}{0pt}%
\pgfpathmoveto{\pgfqpoint{3.286224in}{1.804310in}}%
\pgfpathlineto{\pgfqpoint{3.300490in}{1.791571in}}%
\pgfpathlineto{\pgfqpoint{3.314756in}{1.778941in}}%
\pgfpathlineto{\pgfqpoint{3.329023in}{1.766422in}}%
\pgfpathlineto{\pgfqpoint{3.343290in}{1.754012in}}%
\pgfpathlineto{\pgfqpoint{3.334466in}{1.764060in}}%
\pgfpathlineto{\pgfqpoint{3.325623in}{1.774590in}}%
\pgfpathlineto{\pgfqpoint{3.316759in}{1.785612in}}%
\pgfpathlineto{\pgfqpoint{3.302454in}{1.798563in}}%
\pgfpathlineto{\pgfqpoint{3.288149in}{1.811625in}}%
\pgfpathlineto{\pgfqpoint{3.273844in}{1.824798in}}%
\pgfpathlineto{\pgfqpoint{3.259539in}{1.838081in}}%
\pgfpathlineto{\pgfqpoint{3.268455in}{1.826330in}}%
\pgfpathlineto{\pgfqpoint{3.277350in}{1.815076in}}%
\pgfpathlineto{\pgfqpoint{3.286224in}{1.804310in}}%
\pgfpathclose%
\pgfusepath{fill}%
\end{pgfscope}%
\begin{pgfscope}%
\pgfpathrectangle{\pgfqpoint{1.150000in}{0.150000in}}{\pgfqpoint{5.700000in}{5.700000in}}%
\pgfusepath{clip}%
\pgfsetbuttcap%
\pgfsetroundjoin%
\definecolor{currentfill}{rgb}{0.283197,0.115680,0.436115}%
\pgfsetfillcolor{currentfill}%
\pgfsetfillopacity{0.700000}%
\pgfsetlinewidth{0.000000pt}%
\definecolor{currentstroke}{rgb}{0.000000,0.000000,0.000000}%
\pgfsetstrokecolor{currentstroke}%
\pgfsetdash{}{0pt}%
\pgfpathmoveto{\pgfqpoint{3.925302in}{1.362248in}}%
\pgfpathlineto{\pgfqpoint{3.939597in}{1.355000in}}%
\pgfpathlineto{\pgfqpoint{3.953897in}{1.347849in}}%
\pgfpathlineto{\pgfqpoint{3.968202in}{1.340795in}}%
\pgfpathlineto{\pgfqpoint{3.982512in}{1.333839in}}%
\pgfpathlineto{\pgfqpoint{3.974182in}{1.333706in}}%
\pgfpathlineto{\pgfqpoint{3.965841in}{1.333939in}}%
\pgfpathlineto{\pgfqpoint{3.957491in}{1.334543in}}%
\pgfpathlineto{\pgfqpoint{3.949130in}{1.335526in}}%
\pgfpathlineto{\pgfqpoint{3.934792in}{1.343146in}}%
\pgfpathlineto{\pgfqpoint{3.920459in}{1.350863in}}%
\pgfpathlineto{\pgfqpoint{3.906130in}{1.358677in}}%
\pgfpathlineto{\pgfqpoint{3.891806in}{1.366589in}}%
\pgfpathlineto{\pgfqpoint{3.900196in}{1.364935in}}%
\pgfpathlineto{\pgfqpoint{3.908576in}{1.363664in}}%
\pgfpathlineto{\pgfqpoint{3.916944in}{1.362771in}}%
\pgfpathlineto{\pgfqpoint{3.925302in}{1.362248in}}%
\pgfpathclose%
\pgfusepath{fill}%
\end{pgfscope}%
\begin{pgfscope}%
\pgfpathrectangle{\pgfqpoint{1.150000in}{0.150000in}}{\pgfqpoint{5.700000in}{5.700000in}}%
\pgfusepath{clip}%
\pgfsetbuttcap%
\pgfsetroundjoin%
\definecolor{currentfill}{rgb}{0.279566,0.067836,0.391917}%
\pgfsetfillcolor{currentfill}%
\pgfsetfillopacity{0.700000}%
\pgfsetlinewidth{0.000000pt}%
\definecolor{currentstroke}{rgb}{0.000000,0.000000,0.000000}%
\pgfsetstrokecolor{currentstroke}%
\pgfsetdash{}{0pt}%
\pgfpathmoveto{\pgfqpoint{4.277870in}{1.269152in}}%
\pgfpathlineto{\pgfqpoint{4.292244in}{1.265081in}}%
\pgfpathlineto{\pgfqpoint{4.306626in}{1.261104in}}%
\pgfpathlineto{\pgfqpoint{4.321015in}{1.257221in}}%
\pgfpathlineto{\pgfqpoint{4.335412in}{1.253432in}}%
\pgfpathlineto{\pgfqpoint{4.327245in}{1.247557in}}%
\pgfpathlineto{\pgfqpoint{4.319073in}{1.241967in}}%
\pgfpathlineto{\pgfqpoint{4.310895in}{1.236670in}}%
\pgfpathlineto{\pgfqpoint{4.302711in}{1.231673in}}%
\pgfpathlineto{\pgfqpoint{4.288299in}{1.236083in}}%
\pgfpathlineto{\pgfqpoint{4.273894in}{1.240588in}}%
\pgfpathlineto{\pgfqpoint{4.259496in}{1.245187in}}%
\pgfpathlineto{\pgfqpoint{4.245106in}{1.249879in}}%
\pgfpathlineto{\pgfqpoint{4.253307in}{1.254248in}}%
\pgfpathlineto{\pgfqpoint{4.261501in}{1.258921in}}%
\pgfpathlineto{\pgfqpoint{4.269689in}{1.263891in}}%
\pgfpathlineto{\pgfqpoint{4.277870in}{1.269152in}}%
\pgfpathclose%
\pgfusepath{fill}%
\end{pgfscope}%
\begin{pgfscope}%
\pgfpathrectangle{\pgfqpoint{1.150000in}{0.150000in}}{\pgfqpoint{5.700000in}{5.700000in}}%
\pgfusepath{clip}%
\pgfsetbuttcap%
\pgfsetroundjoin%
\definecolor{currentfill}{rgb}{0.274128,0.199721,0.498911}%
\pgfsetfillcolor{currentfill}%
\pgfsetfillopacity{0.700000}%
\pgfsetlinewidth{0.000000pt}%
\definecolor{currentstroke}{rgb}{0.000000,0.000000,0.000000}%
\pgfsetstrokecolor{currentstroke}%
\pgfsetdash{}{0pt}%
\pgfpathmoveto{\pgfqpoint{4.967844in}{1.550000in}}%
\pgfpathlineto{\pgfqpoint{4.982506in}{1.552791in}}%
\pgfpathlineto{\pgfqpoint{4.997180in}{1.555677in}}%
\pgfpathlineto{\pgfqpoint{5.011868in}{1.558658in}}%
\pgfpathlineto{\pgfqpoint{5.026568in}{1.561733in}}%
\pgfpathlineto{\pgfqpoint{5.018525in}{1.546248in}}%
\pgfpathlineto{\pgfqpoint{5.010480in}{1.530861in}}%
\pgfpathlineto{\pgfqpoint{5.002433in}{1.515574in}}%
\pgfpathlineto{\pgfqpoint{4.994384in}{1.500395in}}%
\pgfpathlineto{\pgfqpoint{4.979689in}{1.497822in}}%
\pgfpathlineto{\pgfqpoint{4.965007in}{1.495344in}}%
\pgfpathlineto{\pgfqpoint{4.950337in}{1.492960in}}%
\pgfpathlineto{\pgfqpoint{4.935680in}{1.490671in}}%
\pgfpathlineto{\pgfqpoint{4.943724in}{1.505342in}}%
\pgfpathlineto{\pgfqpoint{4.951767in}{1.520123in}}%
\pgfpathlineto{\pgfqpoint{4.959806in}{1.535011in}}%
\pgfpathlineto{\pgfqpoint{4.967844in}{1.550000in}}%
\pgfpathclose%
\pgfusepath{fill}%
\end{pgfscope}%
\begin{pgfscope}%
\pgfpathrectangle{\pgfqpoint{1.150000in}{0.150000in}}{\pgfqpoint{5.700000in}{5.700000in}}%
\pgfusepath{clip}%
\pgfsetbuttcap%
\pgfsetroundjoin%
\definecolor{currentfill}{rgb}{0.281446,0.084320,0.407414}%
\pgfsetfillcolor{currentfill}%
\pgfsetfillopacity{0.700000}%
\pgfsetlinewidth{0.000000pt}%
\definecolor{currentstroke}{rgb}{0.000000,0.000000,0.000000}%
\pgfsetstrokecolor{currentstroke}%
\pgfsetdash{}{0pt}%
\pgfpathmoveto{\pgfqpoint{4.130222in}{1.290826in}}%
\pgfpathlineto{\pgfqpoint{4.144560in}{1.285376in}}%
\pgfpathlineto{\pgfqpoint{4.158904in}{1.280020in}}%
\pgfpathlineto{\pgfqpoint{4.173254in}{1.274760in}}%
\pgfpathlineto{\pgfqpoint{4.187611in}{1.269595in}}%
\pgfpathlineto{\pgfqpoint{4.179385in}{1.266170in}}%
\pgfpathlineto{\pgfqpoint{4.171151in}{1.263068in}}%
\pgfpathlineto{\pgfqpoint{4.162909in}{1.260294in}}%
\pgfpathlineto{\pgfqpoint{4.154659in}{1.257856in}}%
\pgfpathlineto{\pgfqpoint{4.140282in}{1.263663in}}%
\pgfpathlineto{\pgfqpoint{4.125910in}{1.269565in}}%
\pgfpathlineto{\pgfqpoint{4.111545in}{1.275562in}}%
\pgfpathlineto{\pgfqpoint{4.097185in}{1.281654in}}%
\pgfpathlineto{\pgfqpoint{4.105457in}{1.283443in}}%
\pgfpathlineto{\pgfqpoint{4.113720in}{1.285573in}}%
\pgfpathlineto{\pgfqpoint{4.121975in}{1.288036in}}%
\pgfpathlineto{\pgfqpoint{4.130222in}{1.290826in}}%
\pgfpathclose%
\pgfusepath{fill}%
\end{pgfscope}%
\begin{pgfscope}%
\pgfpathrectangle{\pgfqpoint{1.150000in}{0.150000in}}{\pgfqpoint{5.700000in}{5.700000in}}%
\pgfusepath{clip}%
\pgfsetbuttcap%
\pgfsetroundjoin%
\definecolor{currentfill}{rgb}{0.175707,0.697900,0.491033}%
\pgfsetfillcolor{currentfill}%
\pgfsetfillopacity{0.700000}%
\pgfsetlinewidth{0.000000pt}%
\definecolor{currentstroke}{rgb}{0.000000,0.000000,0.000000}%
\pgfsetstrokecolor{currentstroke}%
\pgfsetdash{}{0pt}%
\pgfpathmoveto{\pgfqpoint{2.369133in}{2.876897in}}%
\pgfpathlineto{\pgfqpoint{2.383607in}{2.855717in}}%
\pgfpathlineto{\pgfqpoint{2.398074in}{2.834699in}}%
\pgfpathlineto{\pgfqpoint{2.412533in}{2.813840in}}%
\pgfpathlineto{\pgfqpoint{2.426985in}{2.793140in}}%
\pgfpathlineto{\pgfqpoint{2.417135in}{2.815093in}}%
\pgfpathlineto{\pgfqpoint{2.407249in}{2.837641in}}%
\pgfpathlineto{\pgfqpoint{2.397326in}{2.860793in}}%
\pgfpathlineto{\pgfqpoint{2.382813in}{2.882086in}}%
\pgfpathlineto{\pgfqpoint{2.368292in}{2.903538in}}%
\pgfpathlineto{\pgfqpoint{2.353764in}{2.925152in}}%
\pgfpathlineto{\pgfqpoint{2.339228in}{2.946927in}}%
\pgfpathlineto{\pgfqpoint{2.349234in}{2.922974in}}%
\pgfpathlineto{\pgfqpoint{2.359202in}{2.899634in}}%
\pgfpathlineto{\pgfqpoint{2.369133in}{2.876897in}}%
\pgfpathclose%
\pgfusepath{fill}%
\end{pgfscope}%
\begin{pgfscope}%
\pgfpathrectangle{\pgfqpoint{1.150000in}{0.150000in}}{\pgfqpoint{5.700000in}{5.700000in}}%
\pgfusepath{clip}%
\pgfsetbuttcap%
\pgfsetroundjoin%
\definecolor{currentfill}{rgb}{0.241237,0.296485,0.539709}%
\pgfsetfillcolor{currentfill}%
\pgfsetfillopacity{0.700000}%
\pgfsetlinewidth{0.000000pt}%
\definecolor{currentstroke}{rgb}{0.000000,0.000000,0.000000}%
\pgfsetstrokecolor{currentstroke}%
\pgfsetdash{}{0pt}%
\pgfpathmoveto{\pgfqpoint{3.343290in}{1.754012in}}%
\pgfpathlineto{\pgfqpoint{3.357558in}{1.741712in}}%
\pgfpathlineto{\pgfqpoint{3.371827in}{1.729520in}}%
\pgfpathlineto{\pgfqpoint{3.386097in}{1.717437in}}%
\pgfpathlineto{\pgfqpoint{3.400368in}{1.705462in}}%
\pgfpathlineto{\pgfqpoint{3.391593in}{1.714795in}}%
\pgfpathlineto{\pgfqpoint{3.382798in}{1.724604in}}%
\pgfpathlineto{\pgfqpoint{3.373985in}{1.734899in}}%
\pgfpathlineto{\pgfqpoint{3.359677in}{1.747414in}}%
\pgfpathlineto{\pgfqpoint{3.345371in}{1.760037in}}%
\pgfpathlineto{\pgfqpoint{3.331065in}{1.772770in}}%
\pgfpathlineto{\pgfqpoint{3.316759in}{1.785612in}}%
\pgfpathlineto{\pgfqpoint{3.325623in}{1.774590in}}%
\pgfpathlineto{\pgfqpoint{3.334466in}{1.764060in}}%
\pgfpathlineto{\pgfqpoint{3.343290in}{1.754012in}}%
\pgfpathclose%
\pgfusepath{fill}%
\end{pgfscope}%
\begin{pgfscope}%
\pgfpathrectangle{\pgfqpoint{1.150000in}{0.150000in}}{\pgfqpoint{5.700000in}{5.700000in}}%
\pgfusepath{clip}%
\pgfsetbuttcap%
\pgfsetroundjoin%
\definecolor{currentfill}{rgb}{0.616293,0.852709,0.230052}%
\pgfsetfillcolor{currentfill}%
\pgfsetfillopacity{0.700000}%
\pgfsetlinewidth{0.000000pt}%
\definecolor{currentstroke}{rgb}{0.000000,0.000000,0.000000}%
\pgfsetstrokecolor{currentstroke}%
\pgfsetdash{}{0pt}%
\pgfpathmoveto{\pgfqpoint{2.001891in}{3.433803in}}%
\pgfpathlineto{\pgfqpoint{2.016554in}{3.408330in}}%
\pgfpathlineto{\pgfqpoint{2.031205in}{3.383061in}}%
\pgfpathlineto{\pgfqpoint{2.045845in}{3.357994in}}%
\pgfpathlineto{\pgfqpoint{2.060472in}{3.333127in}}%
\pgfpathlineto{\pgfqpoint{2.050180in}{3.358202in}}%
\pgfpathlineto{\pgfqpoint{2.039847in}{3.383892in}}%
\pgfpathlineto{\pgfqpoint{2.029470in}{3.410207in}}%
\pgfpathlineto{\pgfqpoint{2.019051in}{3.437155in}}%
\pgfpathlineto{\pgfqpoint{2.004330in}{3.462835in}}%
\pgfpathlineto{\pgfqpoint{1.989598in}{3.488716in}}%
\pgfpathlineto{\pgfqpoint{1.974852in}{3.514801in}}%
\pgfpathlineto{\pgfqpoint{1.960094in}{3.541093in}}%
\pgfpathlineto{\pgfqpoint{1.970609in}{3.513316in}}%
\pgfpathlineto{\pgfqpoint{1.981080in}{3.486182in}}%
\pgfpathlineto{\pgfqpoint{1.991507in}{3.459680in}}%
\pgfpathlineto{\pgfqpoint{2.001891in}{3.433803in}}%
\pgfpathclose%
\pgfusepath{fill}%
\end{pgfscope}%
\begin{pgfscope}%
\pgfpathrectangle{\pgfqpoint{1.150000in}{0.150000in}}{\pgfqpoint{5.700000in}{5.700000in}}%
\pgfusepath{clip}%
\pgfsetbuttcap%
\pgfsetroundjoin%
\definecolor{currentfill}{rgb}{0.280267,0.073417,0.397163}%
\pgfsetfillcolor{currentfill}%
\pgfsetfillopacity{0.700000}%
\pgfsetlinewidth{0.000000pt}%
\definecolor{currentstroke}{rgb}{0.000000,0.000000,0.000000}%
\pgfsetstrokecolor{currentstroke}%
\pgfsetdash{}{0pt}%
\pgfpathmoveto{\pgfqpoint{4.425634in}{1.267867in}}%
\pgfpathlineto{\pgfqpoint{4.440058in}{1.265151in}}%
\pgfpathlineto{\pgfqpoint{4.454490in}{1.262527in}}%
\pgfpathlineto{\pgfqpoint{4.468931in}{1.259997in}}%
\pgfpathlineto{\pgfqpoint{4.483381in}{1.257561in}}%
\pgfpathlineto{\pgfqpoint{4.475259in}{1.249410in}}%
\pgfpathlineto{\pgfqpoint{4.467133in}{1.241511in}}%
\pgfpathlineto{\pgfqpoint{4.459002in}{1.233871in}}%
\pgfpathlineto{\pgfqpoint{4.450866in}{1.226494in}}%
\pgfpathlineto{\pgfqpoint{4.436406in}{1.229535in}}%
\pgfpathlineto{\pgfqpoint{4.421954in}{1.232668in}}%
\pgfpathlineto{\pgfqpoint{4.407511in}{1.235895in}}%
\pgfpathlineto{\pgfqpoint{4.393075in}{1.239215in}}%
\pgfpathlineto{\pgfqpoint{4.401222in}{1.245981in}}%
\pgfpathlineto{\pgfqpoint{4.409364in}{1.253016in}}%
\pgfpathlineto{\pgfqpoint{4.417502in}{1.260313in}}%
\pgfpathlineto{\pgfqpoint{4.425634in}{1.267867in}}%
\pgfpathclose%
\pgfusepath{fill}%
\end{pgfscope}%
\begin{pgfscope}%
\pgfpathrectangle{\pgfqpoint{1.150000in}{0.150000in}}{\pgfqpoint{5.700000in}{5.700000in}}%
\pgfusepath{clip}%
\pgfsetbuttcap%
\pgfsetroundjoin%
\definecolor{currentfill}{rgb}{0.263663,0.237631,0.518762}%
\pgfsetfillcolor{currentfill}%
\pgfsetfillopacity{0.700000}%
\pgfsetlinewidth{0.000000pt}%
\definecolor{currentstroke}{rgb}{0.000000,0.000000,0.000000}%
\pgfsetstrokecolor{currentstroke}%
\pgfsetdash{}{0pt}%
\pgfpathmoveto{\pgfqpoint{5.058716in}{1.624549in}}%
\pgfpathlineto{\pgfqpoint{5.073435in}{1.628205in}}%
\pgfpathlineto{\pgfqpoint{5.088167in}{1.631957in}}%
\pgfpathlineto{\pgfqpoint{5.102913in}{1.635804in}}%
\pgfpathlineto{\pgfqpoint{5.094875in}{1.619619in}}%
\pgfpathlineto{\pgfqpoint{5.086834in}{1.603508in}}%
\pgfpathlineto{\pgfqpoint{5.078791in}{1.587477in}}%
\pgfpathlineto{\pgfqpoint{5.070745in}{1.571530in}}%
\pgfpathlineto{\pgfqpoint{5.056006in}{1.568169in}}%
\pgfpathlineto{\pgfqpoint{5.041280in}{1.564904in}}%
\pgfpathlineto{\pgfqpoint{5.026568in}{1.561733in}}%
\pgfpathlineto{\pgfqpoint{5.034608in}{1.577310in}}%
\pgfpathlineto{\pgfqpoint{5.042646in}{1.592975in}}%
\pgfpathlineto{\pgfqpoint{5.050682in}{1.608723in}}%
\pgfpathlineto{\pgfqpoint{5.058716in}{1.624549in}}%
\pgfpathclose%
\pgfusepath{fill}%
\end{pgfscope}%
\begin{pgfscope}%
\pgfpathrectangle{\pgfqpoint{1.150000in}{0.150000in}}{\pgfqpoint{5.700000in}{5.700000in}}%
\pgfusepath{clip}%
\pgfsetbuttcap%
\pgfsetroundjoin%
\definecolor{currentfill}{rgb}{0.278826,0.175490,0.483397}%
\pgfsetfillcolor{currentfill}%
\pgfsetfillopacity{0.700000}%
\pgfsetlinewidth{0.000000pt}%
\definecolor{currentstroke}{rgb}{0.000000,0.000000,0.000000}%
\pgfsetstrokecolor{currentstroke}%
\pgfsetdash{}{0pt}%
\pgfpathmoveto{\pgfqpoint{3.720236in}{1.469220in}}%
\pgfpathlineto{\pgfqpoint{3.734513in}{1.460120in}}%
\pgfpathlineto{\pgfqpoint{3.748793in}{1.451121in}}%
\pgfpathlineto{\pgfqpoint{3.763077in}{1.442221in}}%
\pgfpathlineto{\pgfqpoint{3.777364in}{1.433422in}}%
\pgfpathlineto{\pgfqpoint{3.768897in}{1.436825in}}%
\pgfpathlineto{\pgfqpoint{3.760417in}{1.440638in}}%
\pgfpathlineto{\pgfqpoint{3.751924in}{1.444868in}}%
\pgfpathlineto{\pgfqpoint{3.743417in}{1.449523in}}%
\pgfpathlineto{\pgfqpoint{3.729094in}{1.459011in}}%
\pgfpathlineto{\pgfqpoint{3.714774in}{1.468599in}}%
\pgfpathlineto{\pgfqpoint{3.700457in}{1.478288in}}%
\pgfpathlineto{\pgfqpoint{3.686143in}{1.488077in}}%
\pgfpathlineto{\pgfqpoint{3.694688in}{1.482724in}}%
\pgfpathlineto{\pgfqpoint{3.703218in}{1.477802in}}%
\pgfpathlineto{\pgfqpoint{3.711734in}{1.473303in}}%
\pgfpathlineto{\pgfqpoint{3.720236in}{1.469220in}}%
\pgfpathclose%
\pgfusepath{fill}%
\end{pgfscope}%
\begin{pgfscope}%
\pgfpathrectangle{\pgfqpoint{1.150000in}{0.150000in}}{\pgfqpoint{5.700000in}{5.700000in}}%
\pgfusepath{clip}%
\pgfsetbuttcap%
\pgfsetroundjoin%
\definecolor{currentfill}{rgb}{0.248629,0.278775,0.534556}%
\pgfsetfillcolor{currentfill}%
\pgfsetfillopacity{0.700000}%
\pgfsetlinewidth{0.000000pt}%
\definecolor{currentstroke}{rgb}{0.000000,0.000000,0.000000}%
\pgfsetstrokecolor{currentstroke}%
\pgfsetdash{}{0pt}%
\pgfpathmoveto{\pgfqpoint{3.400368in}{1.705462in}}%
\pgfpathlineto{\pgfqpoint{3.414640in}{1.693594in}}%
\pgfpathlineto{\pgfqpoint{3.428913in}{1.681834in}}%
\pgfpathlineto{\pgfqpoint{3.443188in}{1.670181in}}%
\pgfpathlineto{\pgfqpoint{3.457463in}{1.658634in}}%
\pgfpathlineto{\pgfqpoint{3.448735in}{1.667254in}}%
\pgfpathlineto{\pgfqpoint{3.439988in}{1.676345in}}%
\pgfpathlineto{\pgfqpoint{3.431223in}{1.685914in}}%
\pgfpathlineto{\pgfqpoint{3.416912in}{1.698000in}}%
\pgfpathlineto{\pgfqpoint{3.402602in}{1.710192in}}%
\pgfpathlineto{\pgfqpoint{3.388293in}{1.722491in}}%
\pgfpathlineto{\pgfqpoint{3.373985in}{1.734899in}}%
\pgfpathlineto{\pgfqpoint{3.382798in}{1.724604in}}%
\pgfpathlineto{\pgfqpoint{3.391593in}{1.714795in}}%
\pgfpathlineto{\pgfqpoint{3.400368in}{1.705462in}}%
\pgfpathclose%
\pgfusepath{fill}%
\end{pgfscope}%
\begin{pgfscope}%
\pgfpathrectangle{\pgfqpoint{1.150000in}{0.150000in}}{\pgfqpoint{5.700000in}{5.700000in}}%
\pgfusepath{clip}%
\pgfsetbuttcap%
\pgfsetroundjoin%
\definecolor{currentfill}{rgb}{0.226397,0.728888,0.462789}%
\pgfsetfillcolor{currentfill}%
\pgfsetfillopacity{0.700000}%
\pgfsetlinewidth{0.000000pt}%
\definecolor{currentstroke}{rgb}{0.000000,0.000000,0.000000}%
\pgfsetstrokecolor{currentstroke}%
\pgfsetdash{}{0pt}%
\pgfpathmoveto{\pgfqpoint{2.311159in}{2.963250in}}%
\pgfpathlineto{\pgfqpoint{2.325665in}{2.941413in}}%
\pgfpathlineto{\pgfqpoint{2.340162in}{2.919743in}}%
\pgfpathlineto{\pgfqpoint{2.354652in}{2.898238in}}%
\pgfpathlineto{\pgfqpoint{2.369133in}{2.876897in}}%
\pgfpathlineto{\pgfqpoint{2.359202in}{2.899634in}}%
\pgfpathlineto{\pgfqpoint{2.349234in}{2.922974in}}%
\pgfpathlineto{\pgfqpoint{2.339228in}{2.946927in}}%
\pgfpathlineto{\pgfqpoint{2.324683in}{2.968866in}}%
\pgfpathlineto{\pgfqpoint{2.310131in}{2.990970in}}%
\pgfpathlineto{\pgfqpoint{2.295570in}{3.013240in}}%
\pgfpathlineto{\pgfqpoint{2.281000in}{3.035678in}}%
\pgfpathlineto{\pgfqpoint{2.291092in}{3.010917in}}%
\pgfpathlineto{\pgfqpoint{2.301145in}{2.986778in}}%
\pgfpathlineto{\pgfqpoint{2.311159in}{2.963250in}}%
\pgfpathclose%
\pgfusepath{fill}%
\end{pgfscope}%
\begin{pgfscope}%
\pgfpathrectangle{\pgfqpoint{1.150000in}{0.150000in}}{\pgfqpoint{5.700000in}{5.700000in}}%
\pgfusepath{clip}%
\pgfsetbuttcap%
\pgfsetroundjoin%
\definecolor{currentfill}{rgb}{0.283091,0.110553,0.431554}%
\pgfsetfillcolor{currentfill}%
\pgfsetfillopacity{0.700000}%
\pgfsetlinewidth{0.000000pt}%
\definecolor{currentstroke}{rgb}{0.000000,0.000000,0.000000}%
\pgfsetstrokecolor{currentstroke}%
\pgfsetdash{}{0pt}%
\pgfpathmoveto{\pgfqpoint{3.982512in}{1.333839in}}%
\pgfpathlineto{\pgfqpoint{3.996827in}{1.326979in}}%
\pgfpathlineto{\pgfqpoint{4.011148in}{1.320215in}}%
\pgfpathlineto{\pgfqpoint{4.025473in}{1.313549in}}%
\pgfpathlineto{\pgfqpoint{4.039805in}{1.306978in}}%
\pgfpathlineto{\pgfqpoint{4.031500in}{1.306191in}}%
\pgfpathlineto{\pgfqpoint{4.023186in}{1.305763in}}%
\pgfpathlineto{\pgfqpoint{4.014863in}{1.305701in}}%
\pgfpathlineto{\pgfqpoint{4.006530in}{1.306013in}}%
\pgfpathlineto{\pgfqpoint{3.992172in}{1.313247in}}%
\pgfpathlineto{\pgfqpoint{3.977820in}{1.320576in}}%
\pgfpathlineto{\pgfqpoint{3.963472in}{1.328003in}}%
\pgfpathlineto{\pgfqpoint{3.949130in}{1.335526in}}%
\pgfpathlineto{\pgfqpoint{3.957491in}{1.334543in}}%
\pgfpathlineto{\pgfqpoint{3.965841in}{1.333939in}}%
\pgfpathlineto{\pgfqpoint{3.974182in}{1.333706in}}%
\pgfpathlineto{\pgfqpoint{3.982512in}{1.333839in}}%
\pgfpathclose%
\pgfusepath{fill}%
\end{pgfscope}%
\begin{pgfscope}%
\pgfpathrectangle{\pgfqpoint{1.150000in}{0.150000in}}{\pgfqpoint{5.700000in}{5.700000in}}%
\pgfusepath{clip}%
\pgfsetbuttcap%
\pgfsetroundjoin%
\definecolor{currentfill}{rgb}{0.283187,0.125848,0.444960}%
\pgfsetfillcolor{currentfill}%
\pgfsetfillopacity{0.700000}%
\pgfsetlinewidth{0.000000pt}%
\definecolor{currentstroke}{rgb}{0.000000,0.000000,0.000000}%
\pgfsetstrokecolor{currentstroke}%
\pgfsetdash{}{0pt}%
\pgfpathmoveto{\pgfqpoint{4.754438in}{1.371401in}}%
\pgfpathlineto{\pgfqpoint{4.769000in}{1.371884in}}%
\pgfpathlineto{\pgfqpoint{4.783572in}{1.372461in}}%
\pgfpathlineto{\pgfqpoint{4.798156in}{1.373132in}}%
\pgfpathlineto{\pgfqpoint{4.812750in}{1.373896in}}%
\pgfpathlineto{\pgfqpoint{4.804687in}{1.360995in}}%
\pgfpathlineto{\pgfqpoint{4.796620in}{1.348260in}}%
\pgfpathlineto{\pgfqpoint{4.788552in}{1.335696in}}%
\pgfpathlineto{\pgfqpoint{4.780480in}{1.323309in}}%
\pgfpathlineto{\pgfqpoint{4.765885in}{1.323097in}}%
\pgfpathlineto{\pgfqpoint{4.751302in}{1.322979in}}%
\pgfpathlineto{\pgfqpoint{4.736728in}{1.322955in}}%
\pgfpathlineto{\pgfqpoint{4.722165in}{1.323023in}}%
\pgfpathlineto{\pgfqpoint{4.730238in}{1.334851in}}%
\pgfpathlineto{\pgfqpoint{4.738308in}{1.346860in}}%
\pgfpathlineto{\pgfqpoint{4.746374in}{1.359045in}}%
\pgfpathlineto{\pgfqpoint{4.754438in}{1.371401in}}%
\pgfpathclose%
\pgfusepath{fill}%
\end{pgfscope}%
\begin{pgfscope}%
\pgfpathrectangle{\pgfqpoint{1.150000in}{0.150000in}}{\pgfqpoint{5.700000in}{5.700000in}}%
\pgfusepath{clip}%
\pgfsetbuttcap%
\pgfsetroundjoin%
\definecolor{currentfill}{rgb}{0.282656,0.100196,0.422160}%
\pgfsetfillcolor{currentfill}%
\pgfsetfillopacity{0.700000}%
\pgfsetlinewidth{0.000000pt}%
\definecolor{currentstroke}{rgb}{0.000000,0.000000,0.000000}%
\pgfsetstrokecolor{currentstroke}%
\pgfsetdash{}{0pt}%
\pgfpathmoveto{\pgfqpoint{4.664018in}{1.324232in}}%
\pgfpathlineto{\pgfqpoint{4.678539in}{1.323790in}}%
\pgfpathlineto{\pgfqpoint{4.693071in}{1.323441in}}%
\pgfpathlineto{\pgfqpoint{4.707613in}{1.323185in}}%
\pgfpathlineto{\pgfqpoint{4.722165in}{1.323023in}}%
\pgfpathlineto{\pgfqpoint{4.714090in}{1.311382in}}%
\pgfpathlineto{\pgfqpoint{4.706012in}{1.299933in}}%
\pgfpathlineto{\pgfqpoint{4.697930in}{1.288682in}}%
\pgfpathlineto{\pgfqpoint{4.689846in}{1.277634in}}%
\pgfpathlineto{\pgfqpoint{4.675291in}{1.278366in}}%
\pgfpathlineto{\pgfqpoint{4.660746in}{1.279191in}}%
\pgfpathlineto{\pgfqpoint{4.646210in}{1.280109in}}%
\pgfpathlineto{\pgfqpoint{4.631685in}{1.281120in}}%
\pgfpathlineto{\pgfqpoint{4.639773in}{1.291592in}}%
\pgfpathlineto{\pgfqpoint{4.647858in}{1.302271in}}%
\pgfpathlineto{\pgfqpoint{4.655939in}{1.313153in}}%
\pgfpathlineto{\pgfqpoint{4.664018in}{1.324232in}}%
\pgfpathclose%
\pgfusepath{fill}%
\end{pgfscope}%
\begin{pgfscope}%
\pgfpathrectangle{\pgfqpoint{1.150000in}{0.150000in}}{\pgfqpoint{5.700000in}{5.700000in}}%
\pgfusepath{clip}%
\pgfsetbuttcap%
\pgfsetroundjoin%
\definecolor{currentfill}{rgb}{0.281887,0.150881,0.465405}%
\pgfsetfillcolor{currentfill}%
\pgfsetfillopacity{0.700000}%
\pgfsetlinewidth{0.000000pt}%
\definecolor{currentstroke}{rgb}{0.000000,0.000000,0.000000}%
\pgfsetstrokecolor{currentstroke}%
\pgfsetdash{}{0pt}%
\pgfpathmoveto{\pgfqpoint{4.844981in}{1.427057in}}%
\pgfpathlineto{\pgfqpoint{4.859588in}{1.428450in}}%
\pgfpathlineto{\pgfqpoint{4.874206in}{1.429938in}}%
\pgfpathlineto{\pgfqpoint{4.888836in}{1.431520in}}%
\pgfpathlineto{\pgfqpoint{4.903478in}{1.433195in}}%
\pgfpathlineto{\pgfqpoint{4.895422in}{1.419153in}}%
\pgfpathlineto{\pgfqpoint{4.887364in}{1.405251in}}%
\pgfpathlineto{\pgfqpoint{4.879304in}{1.391495in}}%
\pgfpathlineto{\pgfqpoint{4.871241in}{1.377890in}}%
\pgfpathlineto{\pgfqpoint{4.856601in}{1.376751in}}%
\pgfpathlineto{\pgfqpoint{4.841973in}{1.375706in}}%
\pgfpathlineto{\pgfqpoint{4.827356in}{1.374754in}}%
\pgfpathlineto{\pgfqpoint{4.812750in}{1.373896in}}%
\pgfpathlineto{\pgfqpoint{4.820812in}{1.386958in}}%
\pgfpathlineto{\pgfqpoint{4.828871in}{1.400176in}}%
\pgfpathlineto{\pgfqpoint{4.836927in}{1.413544in}}%
\pgfpathlineto{\pgfqpoint{4.844981in}{1.427057in}}%
\pgfpathclose%
\pgfusepath{fill}%
\end{pgfscope}%
\begin{pgfscope}%
\pgfpathrectangle{\pgfqpoint{1.150000in}{0.150000in}}{\pgfqpoint{5.700000in}{5.700000in}}%
\pgfusepath{clip}%
\pgfsetbuttcap%
\pgfsetroundjoin%
\definecolor{currentfill}{rgb}{0.253935,0.265254,0.529983}%
\pgfsetfillcolor{currentfill}%
\pgfsetfillopacity{0.700000}%
\pgfsetlinewidth{0.000000pt}%
\definecolor{currentstroke}{rgb}{0.000000,0.000000,0.000000}%
\pgfsetstrokecolor{currentstroke}%
\pgfsetdash{}{0pt}%
\pgfpathmoveto{\pgfqpoint{3.457463in}{1.658634in}}%
\pgfpathlineto{\pgfqpoint{3.471741in}{1.647193in}}%
\pgfpathlineto{\pgfqpoint{3.486020in}{1.635858in}}%
\pgfpathlineto{\pgfqpoint{3.500301in}{1.624628in}}%
\pgfpathlineto{\pgfqpoint{3.514583in}{1.613503in}}%
\pgfpathlineto{\pgfqpoint{3.505899in}{1.621414in}}%
\pgfpathlineto{\pgfqpoint{3.497199in}{1.629788in}}%
\pgfpathlineto{\pgfqpoint{3.488480in}{1.638635in}}%
\pgfpathlineto{\pgfqpoint{3.474164in}{1.650296in}}%
\pgfpathlineto{\pgfqpoint{3.459849in}{1.662063in}}%
\pgfpathlineto{\pgfqpoint{3.445535in}{1.673936in}}%
\pgfpathlineto{\pgfqpoint{3.431223in}{1.685914in}}%
\pgfpathlineto{\pgfqpoint{3.439988in}{1.676345in}}%
\pgfpathlineto{\pgfqpoint{3.448735in}{1.667254in}}%
\pgfpathlineto{\pgfqpoint{3.457463in}{1.658634in}}%
\pgfpathclose%
\pgfusepath{fill}%
\end{pgfscope}%
\begin{pgfscope}%
\pgfpathrectangle{\pgfqpoint{1.150000in}{0.150000in}}{\pgfqpoint{5.700000in}{5.700000in}}%
\pgfusepath{clip}%
\pgfsetbuttcap%
\pgfsetroundjoin%
\definecolor{currentfill}{rgb}{0.730889,0.871916,0.156029}%
\pgfsetfillcolor{currentfill}%
\pgfsetfillopacity{0.700000}%
\pgfsetlinewidth{0.000000pt}%
\definecolor{currentstroke}{rgb}{0.000000,0.000000,0.000000}%
\pgfsetstrokecolor{currentstroke}%
\pgfsetdash{}{0pt}%
\pgfpathmoveto{\pgfqpoint{1.943108in}{3.537772in}}%
\pgfpathlineto{\pgfqpoint{1.957824in}{3.511464in}}%
\pgfpathlineto{\pgfqpoint{1.972525in}{3.485367in}}%
\pgfpathlineto{\pgfqpoint{1.987214in}{3.459481in}}%
\pgfpathlineto{\pgfqpoint{2.001891in}{3.433803in}}%
\pgfpathlineto{\pgfqpoint{1.991507in}{3.459680in}}%
\pgfpathlineto{\pgfqpoint{1.981080in}{3.486182in}}%
\pgfpathlineto{\pgfqpoint{1.970609in}{3.513316in}}%
\pgfpathlineto{\pgfqpoint{1.960094in}{3.541093in}}%
\pgfpathlineto{\pgfqpoint{1.945322in}{3.567592in}}%
\pgfpathlineto{\pgfqpoint{1.930537in}{3.594301in}}%
\pgfpathlineto{\pgfqpoint{1.915738in}{3.621223in}}%
\pgfpathlineto{\pgfqpoint{1.900926in}{3.648359in}}%
\pgfpathlineto{\pgfqpoint{1.911540in}{3.619744in}}%
\pgfpathlineto{\pgfqpoint{1.922108in}{3.591781in}}%
\pgfpathlineto{\pgfqpoint{1.932630in}{3.564461in}}%
\pgfpathlineto{\pgfqpoint{1.943108in}{3.537772in}}%
\pgfpathclose%
\pgfusepath{fill}%
\end{pgfscope}%
\begin{pgfscope}%
\pgfpathrectangle{\pgfqpoint{1.150000in}{0.150000in}}{\pgfqpoint{5.700000in}{5.700000in}}%
\pgfusepath{clip}%
\pgfsetbuttcap%
\pgfsetroundjoin%
\definecolor{currentfill}{rgb}{0.281446,0.084320,0.407414}%
\pgfsetfillcolor{currentfill}%
\pgfsetfillopacity{0.700000}%
\pgfsetlinewidth{0.000000pt}%
\definecolor{currentstroke}{rgb}{0.000000,0.000000,0.000000}%
\pgfsetstrokecolor{currentstroke}%
\pgfsetdash{}{0pt}%
\pgfpathmoveto{\pgfqpoint{4.573679in}{1.286096in}}%
\pgfpathlineto{\pgfqpoint{4.588167in}{1.284712in}}%
\pgfpathlineto{\pgfqpoint{4.602663in}{1.283421in}}%
\pgfpathlineto{\pgfqpoint{4.617169in}{1.282224in}}%
\pgfpathlineto{\pgfqpoint{4.631685in}{1.281120in}}%
\pgfpathlineto{\pgfqpoint{4.623593in}{1.270861in}}%
\pgfpathlineto{\pgfqpoint{4.615498in}{1.260823in}}%
\pgfpathlineto{\pgfqpoint{4.607400in}{1.251009in}}%
\pgfpathlineto{\pgfqpoint{4.599298in}{1.241426in}}%
\pgfpathlineto{\pgfqpoint{4.584776in}{1.243117in}}%
\pgfpathlineto{\pgfqpoint{4.570264in}{1.244901in}}%
\pgfpathlineto{\pgfqpoint{4.555761in}{1.246778in}}%
\pgfpathlineto{\pgfqpoint{4.541267in}{1.248748in}}%
\pgfpathlineto{\pgfqpoint{4.549376in}{1.257738in}}%
\pgfpathlineto{\pgfqpoint{4.557481in}{1.266963in}}%
\pgfpathlineto{\pgfqpoint{4.565582in}{1.276418in}}%
\pgfpathlineto{\pgfqpoint{4.573679in}{1.286096in}}%
\pgfpathclose%
\pgfusepath{fill}%
\end{pgfscope}%
\begin{pgfscope}%
\pgfpathrectangle{\pgfqpoint{1.150000in}{0.150000in}}{\pgfqpoint{5.700000in}{5.700000in}}%
\pgfusepath{clip}%
\pgfsetbuttcap%
\pgfsetroundjoin%
\definecolor{currentfill}{rgb}{0.280267,0.073417,0.397163}%
\pgfsetfillcolor{currentfill}%
\pgfsetfillopacity{0.700000}%
\pgfsetlinewidth{0.000000pt}%
\definecolor{currentstroke}{rgb}{0.000000,0.000000,0.000000}%
\pgfsetstrokecolor{currentstroke}%
\pgfsetdash{}{0pt}%
\pgfpathmoveto{\pgfqpoint{4.335412in}{1.253432in}}%
\pgfpathlineto{\pgfqpoint{4.349816in}{1.249737in}}%
\pgfpathlineto{\pgfqpoint{4.364228in}{1.246136in}}%
\pgfpathlineto{\pgfqpoint{4.378647in}{1.242629in}}%
\pgfpathlineto{\pgfqpoint{4.393075in}{1.239215in}}%
\pgfpathlineto{\pgfqpoint{4.384922in}{1.232725in}}%
\pgfpathlineto{\pgfqpoint{4.376764in}{1.226515in}}%
\pgfpathlineto{\pgfqpoint{4.368601in}{1.220594in}}%
\pgfpathlineto{\pgfqpoint{4.360432in}{1.214967in}}%
\pgfpathlineto{\pgfqpoint{4.345990in}{1.219003in}}%
\pgfpathlineto{\pgfqpoint{4.331556in}{1.223133in}}%
\pgfpathlineto{\pgfqpoint{4.317130in}{1.227356in}}%
\pgfpathlineto{\pgfqpoint{4.302711in}{1.231673in}}%
\pgfpathlineto{\pgfqpoint{4.310895in}{1.236670in}}%
\pgfpathlineto{\pgfqpoint{4.319073in}{1.241967in}}%
\pgfpathlineto{\pgfqpoint{4.327245in}{1.247557in}}%
\pgfpathlineto{\pgfqpoint{4.335412in}{1.253432in}}%
\pgfpathclose%
\pgfusepath{fill}%
\end{pgfscope}%
\begin{pgfscope}%
\pgfpathrectangle{\pgfqpoint{1.150000in}{0.150000in}}{\pgfqpoint{5.700000in}{5.700000in}}%
\pgfusepath{clip}%
\pgfsetbuttcap%
\pgfsetroundjoin%
\definecolor{currentfill}{rgb}{0.288921,0.758394,0.428426}%
\pgfsetfillcolor{currentfill}%
\pgfsetfillopacity{0.700000}%
\pgfsetlinewidth{0.000000pt}%
\definecolor{currentstroke}{rgb}{0.000000,0.000000,0.000000}%
\pgfsetstrokecolor{currentstroke}%
\pgfsetdash{}{0pt}%
\pgfpathmoveto{\pgfqpoint{2.253051in}{3.052284in}}%
\pgfpathlineto{\pgfqpoint{2.267591in}{3.029769in}}%
\pgfpathlineto{\pgfqpoint{2.282123in}{3.007426in}}%
\pgfpathlineto{\pgfqpoint{2.296645in}{2.985253in}}%
\pgfpathlineto{\pgfqpoint{2.311159in}{2.963250in}}%
\pgfpathlineto{\pgfqpoint{2.301145in}{2.986778in}}%
\pgfpathlineto{\pgfqpoint{2.291092in}{3.010917in}}%
\pgfpathlineto{\pgfqpoint{2.281000in}{3.035678in}}%
\pgfpathlineto{\pgfqpoint{2.266422in}{3.058284in}}%
\pgfpathlineto{\pgfqpoint{2.251835in}{3.081061in}}%
\pgfpathlineto{\pgfqpoint{2.237238in}{3.104009in}}%
\pgfpathlineto{\pgfqpoint{2.222633in}{3.127131in}}%
\pgfpathlineto{\pgfqpoint{2.232813in}{3.101556in}}%
\pgfpathlineto{\pgfqpoint{2.242952in}{3.076611in}}%
\pgfpathlineto{\pgfqpoint{2.253051in}{3.052284in}}%
\pgfpathclose%
\pgfusepath{fill}%
\end{pgfscope}%
\begin{pgfscope}%
\pgfpathrectangle{\pgfqpoint{1.150000in}{0.150000in}}{\pgfqpoint{5.700000in}{5.700000in}}%
\pgfusepath{clip}%
\pgfsetbuttcap%
\pgfsetroundjoin%
\definecolor{currentfill}{rgb}{0.280894,0.078907,0.402329}%
\pgfsetfillcolor{currentfill}%
\pgfsetfillopacity{0.700000}%
\pgfsetlinewidth{0.000000pt}%
\definecolor{currentstroke}{rgb}{0.000000,0.000000,0.000000}%
\pgfsetstrokecolor{currentstroke}%
\pgfsetdash{}{0pt}%
\pgfpathmoveto{\pgfqpoint{4.187611in}{1.269595in}}%
\pgfpathlineto{\pgfqpoint{4.201975in}{1.264524in}}%
\pgfpathlineto{\pgfqpoint{4.216345in}{1.259548in}}%
\pgfpathlineto{\pgfqpoint{4.230722in}{1.254667in}}%
\pgfpathlineto{\pgfqpoint{4.245106in}{1.249879in}}%
\pgfpathlineto{\pgfqpoint{4.236898in}{1.245821in}}%
\pgfpathlineto{\pgfqpoint{4.228683in}{1.242080in}}%
\pgfpathlineto{\pgfqpoint{4.220461in}{1.238662in}}%
\pgfpathlineto{\pgfqpoint{4.212233in}{1.235575in}}%
\pgfpathlineto{\pgfqpoint{4.197830in}{1.241003in}}%
\pgfpathlineto{\pgfqpoint{4.183433in}{1.246526in}}%
\pgfpathlineto{\pgfqpoint{4.169043in}{1.252144in}}%
\pgfpathlineto{\pgfqpoint{4.154659in}{1.257856in}}%
\pgfpathlineto{\pgfqpoint{4.162909in}{1.260294in}}%
\pgfpathlineto{\pgfqpoint{4.171151in}{1.263068in}}%
\pgfpathlineto{\pgfqpoint{4.179385in}{1.266170in}}%
\pgfpathlineto{\pgfqpoint{4.187611in}{1.269595in}}%
\pgfpathclose%
\pgfusepath{fill}%
\end{pgfscope}%
\begin{pgfscope}%
\pgfpathrectangle{\pgfqpoint{1.150000in}{0.150000in}}{\pgfqpoint{5.700000in}{5.700000in}}%
\pgfusepath{clip}%
\pgfsetbuttcap%
\pgfsetroundjoin%
\definecolor{currentfill}{rgb}{0.280868,0.160771,0.472899}%
\pgfsetfillcolor{currentfill}%
\pgfsetfillopacity{0.700000}%
\pgfsetlinewidth{0.000000pt}%
\definecolor{currentstroke}{rgb}{0.000000,0.000000,0.000000}%
\pgfsetstrokecolor{currentstroke}%
\pgfsetdash{}{0pt}%
\pgfpathmoveto{\pgfqpoint{3.777364in}{1.433422in}}%
\pgfpathlineto{\pgfqpoint{3.791656in}{1.424722in}}%
\pgfpathlineto{\pgfqpoint{3.805951in}{1.416121in}}%
\pgfpathlineto{\pgfqpoint{3.820250in}{1.407619in}}%
\pgfpathlineto{\pgfqpoint{3.834553in}{1.399217in}}%
\pgfpathlineto{\pgfqpoint{3.826119in}{1.401941in}}%
\pgfpathlineto{\pgfqpoint{3.817673in}{1.405069in}}%
\pgfpathlineto{\pgfqpoint{3.809215in}{1.408609in}}%
\pgfpathlineto{\pgfqpoint{3.800743in}{1.412569in}}%
\pgfpathlineto{\pgfqpoint{3.786406in}{1.421658in}}%
\pgfpathlineto{\pgfqpoint{3.772073in}{1.430847in}}%
\pgfpathlineto{\pgfqpoint{3.757743in}{1.440135in}}%
\pgfpathlineto{\pgfqpoint{3.743417in}{1.449523in}}%
\pgfpathlineto{\pgfqpoint{3.751924in}{1.444868in}}%
\pgfpathlineto{\pgfqpoint{3.760417in}{1.440638in}}%
\pgfpathlineto{\pgfqpoint{3.768897in}{1.436825in}}%
\pgfpathlineto{\pgfqpoint{3.777364in}{1.433422in}}%
\pgfpathclose%
\pgfusepath{fill}%
\end{pgfscope}%
\begin{pgfscope}%
\pgfpathrectangle{\pgfqpoint{1.150000in}{0.150000in}}{\pgfqpoint{5.700000in}{5.700000in}}%
\pgfusepath{clip}%
\pgfsetbuttcap%
\pgfsetroundjoin%
\definecolor{currentfill}{rgb}{0.277134,0.185228,0.489898}%
\pgfsetfillcolor{currentfill}%
\pgfsetfillopacity{0.700000}%
\pgfsetlinewidth{0.000000pt}%
\definecolor{currentstroke}{rgb}{0.000000,0.000000,0.000000}%
\pgfsetstrokecolor{currentstroke}%
\pgfsetdash{}{0pt}%
\pgfpathmoveto{\pgfqpoint{4.935680in}{1.490671in}}%
\pgfpathlineto{\pgfqpoint{4.950337in}{1.492960in}}%
\pgfpathlineto{\pgfqpoint{4.965007in}{1.495344in}}%
\pgfpathlineto{\pgfqpoint{4.979689in}{1.497822in}}%
\pgfpathlineto{\pgfqpoint{4.994384in}{1.500395in}}%
\pgfpathlineto{\pgfqpoint{4.986332in}{1.485326in}}%
\pgfpathlineto{\pgfqpoint{4.978279in}{1.470374in}}%
\pgfpathlineto{\pgfqpoint{4.970223in}{1.455543in}}%
\pgfpathlineto{\pgfqpoint{4.962166in}{1.440838in}}%
\pgfpathlineto{\pgfqpoint{4.947476in}{1.438787in}}%
\pgfpathlineto{\pgfqpoint{4.932798in}{1.436829in}}%
\pgfpathlineto{\pgfqpoint{4.918132in}{1.434965in}}%
\pgfpathlineto{\pgfqpoint{4.903478in}{1.433195in}}%
\pgfpathlineto{\pgfqpoint{4.911532in}{1.447373in}}%
\pgfpathlineto{\pgfqpoint{4.919584in}{1.461682in}}%
\pgfpathlineto{\pgfqpoint{4.927633in}{1.476116in}}%
\pgfpathlineto{\pgfqpoint{4.935680in}{1.490671in}}%
\pgfpathclose%
\pgfusepath{fill}%
\end{pgfscope}%
\begin{pgfscope}%
\pgfpathrectangle{\pgfqpoint{1.150000in}{0.150000in}}{\pgfqpoint{5.700000in}{5.700000in}}%
\pgfusepath{clip}%
\pgfsetbuttcap%
\pgfsetroundjoin%
\definecolor{currentfill}{rgb}{0.260571,0.246922,0.522828}%
\pgfsetfillcolor{currentfill}%
\pgfsetfillopacity{0.700000}%
\pgfsetlinewidth{0.000000pt}%
\definecolor{currentstroke}{rgb}{0.000000,0.000000,0.000000}%
\pgfsetstrokecolor{currentstroke}%
\pgfsetdash{}{0pt}%
\pgfpathmoveto{\pgfqpoint{3.514583in}{1.613503in}}%
\pgfpathlineto{\pgfqpoint{3.528867in}{1.602483in}}%
\pgfpathlineto{\pgfqpoint{3.543153in}{1.591567in}}%
\pgfpathlineto{\pgfqpoint{3.557442in}{1.580756in}}%
\pgfpathlineto{\pgfqpoint{3.571732in}{1.570047in}}%
\pgfpathlineto{\pgfqpoint{3.563092in}{1.577250in}}%
\pgfpathlineto{\pgfqpoint{3.554435in}{1.584910in}}%
\pgfpathlineto{\pgfqpoint{3.545762in}{1.593036in}}%
\pgfpathlineto{\pgfqpoint{3.531439in}{1.604279in}}%
\pgfpathlineto{\pgfqpoint{3.517118in}{1.615627in}}%
\pgfpathlineto{\pgfqpoint{3.502798in}{1.627078in}}%
\pgfpathlineto{\pgfqpoint{3.488480in}{1.638635in}}%
\pgfpathlineto{\pgfqpoint{3.497199in}{1.629788in}}%
\pgfpathlineto{\pgfqpoint{3.505899in}{1.621414in}}%
\pgfpathlineto{\pgfqpoint{3.514583in}{1.613503in}}%
\pgfpathclose%
\pgfusepath{fill}%
\end{pgfscope}%
\begin{pgfscope}%
\pgfpathrectangle{\pgfqpoint{1.150000in}{0.150000in}}{\pgfqpoint{5.700000in}{5.700000in}}%
\pgfusepath{clip}%
\pgfsetbuttcap%
\pgfsetroundjoin%
\definecolor{currentfill}{rgb}{0.269308,0.218818,0.509577}%
\pgfsetfillcolor{currentfill}%
\pgfsetfillopacity{0.700000}%
\pgfsetlinewidth{0.000000pt}%
\definecolor{currentstroke}{rgb}{0.000000,0.000000,0.000000}%
\pgfsetstrokecolor{currentstroke}%
\pgfsetdash{}{0pt}%
\pgfpathmoveto{\pgfqpoint{5.026568in}{1.561733in}}%
\pgfpathlineto{\pgfqpoint{5.041280in}{1.564904in}}%
\pgfpathlineto{\pgfqpoint{5.056006in}{1.568169in}}%
\pgfpathlineto{\pgfqpoint{5.070745in}{1.571530in}}%
\pgfpathlineto{\pgfqpoint{5.062698in}{1.555672in}}%
\pgfpathlineto{\pgfqpoint{5.054648in}{1.539908in}}%
\pgfpathlineto{\pgfqpoint{5.046597in}{1.524242in}}%
\pgfpathlineto{\pgfqpoint{5.038543in}{1.508680in}}%
\pgfpathlineto{\pgfqpoint{5.023810in}{1.505823in}}%
\pgfpathlineto{\pgfqpoint{5.009091in}{1.503062in}}%
\pgfpathlineto{\pgfqpoint{4.994384in}{1.500395in}}%
\pgfpathlineto{\pgfqpoint{5.002433in}{1.515574in}}%
\pgfpathlineto{\pgfqpoint{5.010480in}{1.530861in}}%
\pgfpathlineto{\pgfqpoint{5.018525in}{1.546248in}}%
\pgfpathlineto{\pgfqpoint{5.026568in}{1.561733in}}%
\pgfpathclose%
\pgfusepath{fill}%
\end{pgfscope}%
\begin{pgfscope}%
\pgfpathrectangle{\pgfqpoint{1.150000in}{0.150000in}}{\pgfqpoint{5.700000in}{5.700000in}}%
\pgfusepath{clip}%
\pgfsetbuttcap%
\pgfsetroundjoin%
\definecolor{currentfill}{rgb}{0.280894,0.078907,0.402329}%
\pgfsetfillcolor{currentfill}%
\pgfsetfillopacity{0.700000}%
\pgfsetlinewidth{0.000000pt}%
\definecolor{currentstroke}{rgb}{0.000000,0.000000,0.000000}%
\pgfsetstrokecolor{currentstroke}%
\pgfsetdash{}{0pt}%
\pgfpathmoveto{\pgfqpoint{4.483381in}{1.257561in}}%
\pgfpathlineto{\pgfqpoint{4.497839in}{1.255218in}}%
\pgfpathlineto{\pgfqpoint{4.512306in}{1.252968in}}%
\pgfpathlineto{\pgfqpoint{4.526782in}{1.250812in}}%
\pgfpathlineto{\pgfqpoint{4.541267in}{1.248748in}}%
\pgfpathlineto{\pgfqpoint{4.533154in}{1.240000in}}%
\pgfpathlineto{\pgfqpoint{4.525037in}{1.231500in}}%
\pgfpathlineto{\pgfqpoint{4.516916in}{1.223253in}}%
\pgfpathlineto{\pgfqpoint{4.508790in}{1.215265in}}%
\pgfpathlineto{\pgfqpoint{4.494297in}{1.217933in}}%
\pgfpathlineto{\pgfqpoint{4.479811in}{1.220694in}}%
\pgfpathlineto{\pgfqpoint{4.465334in}{1.223547in}}%
\pgfpathlineto{\pgfqpoint{4.450866in}{1.226494in}}%
\pgfpathlineto{\pgfqpoint{4.459002in}{1.233871in}}%
\pgfpathlineto{\pgfqpoint{4.467133in}{1.241511in}}%
\pgfpathlineto{\pgfqpoint{4.475259in}{1.249410in}}%
\pgfpathlineto{\pgfqpoint{4.483381in}{1.257561in}}%
\pgfpathclose%
\pgfusepath{fill}%
\end{pgfscope}%
\begin{pgfscope}%
\pgfpathrectangle{\pgfqpoint{1.150000in}{0.150000in}}{\pgfqpoint{5.700000in}{5.700000in}}%
\pgfusepath{clip}%
\pgfsetbuttcap%
\pgfsetroundjoin%
\definecolor{currentfill}{rgb}{0.282910,0.105393,0.426902}%
\pgfsetfillcolor{currentfill}%
\pgfsetfillopacity{0.700000}%
\pgfsetlinewidth{0.000000pt}%
\definecolor{currentstroke}{rgb}{0.000000,0.000000,0.000000}%
\pgfsetstrokecolor{currentstroke}%
\pgfsetdash{}{0pt}%
\pgfpathmoveto{\pgfqpoint{4.039805in}{1.306978in}}%
\pgfpathlineto{\pgfqpoint{4.054141in}{1.300503in}}%
\pgfpathlineto{\pgfqpoint{4.068484in}{1.294124in}}%
\pgfpathlineto{\pgfqpoint{4.082831in}{1.287841in}}%
\pgfpathlineto{\pgfqpoint{4.097185in}{1.281654in}}%
\pgfpathlineto{\pgfqpoint{4.088905in}{1.280212in}}%
\pgfpathlineto{\pgfqpoint{4.080616in}{1.279124in}}%
\pgfpathlineto{\pgfqpoint{4.072318in}{1.278398in}}%
\pgfpathlineto{\pgfqpoint{4.064011in}{1.278039in}}%
\pgfpathlineto{\pgfqpoint{4.049633in}{1.284889in}}%
\pgfpathlineto{\pgfqpoint{4.035260in}{1.291835in}}%
\pgfpathlineto{\pgfqpoint{4.020892in}{1.298876in}}%
\pgfpathlineto{\pgfqpoint{4.006530in}{1.306013in}}%
\pgfpathlineto{\pgfqpoint{4.014863in}{1.305701in}}%
\pgfpathlineto{\pgfqpoint{4.023186in}{1.305763in}}%
\pgfpathlineto{\pgfqpoint{4.031500in}{1.306191in}}%
\pgfpathlineto{\pgfqpoint{4.039805in}{1.306978in}}%
\pgfpathclose%
\pgfusepath{fill}%
\end{pgfscope}%
\begin{pgfscope}%
\pgfpathrectangle{\pgfqpoint{1.150000in}{0.150000in}}{\pgfqpoint{5.700000in}{5.700000in}}%
\pgfusepath{clip}%
\pgfsetbuttcap%
\pgfsetroundjoin%
\definecolor{currentfill}{rgb}{0.360741,0.785964,0.387814}%
\pgfsetfillcolor{currentfill}%
\pgfsetfillopacity{0.700000}%
\pgfsetlinewidth{0.000000pt}%
\definecolor{currentstroke}{rgb}{0.000000,0.000000,0.000000}%
\pgfsetstrokecolor{currentstroke}%
\pgfsetdash{}{0pt}%
\pgfpathmoveto{\pgfqpoint{2.194799in}{3.144092in}}%
\pgfpathlineto{\pgfqpoint{2.209376in}{3.120875in}}%
\pgfpathlineto{\pgfqpoint{2.223944in}{3.097836in}}%
\pgfpathlineto{\pgfqpoint{2.238502in}{3.074973in}}%
\pgfpathlineto{\pgfqpoint{2.253051in}{3.052284in}}%
\pgfpathlineto{\pgfqpoint{2.242952in}{3.076611in}}%
\pgfpathlineto{\pgfqpoint{2.232813in}{3.101556in}}%
\pgfpathlineto{\pgfqpoint{2.222633in}{3.127131in}}%
\pgfpathlineto{\pgfqpoint{2.208018in}{3.150427in}}%
\pgfpathlineto{\pgfqpoint{2.193393in}{3.173899in}}%
\pgfpathlineto{\pgfqpoint{2.178758in}{3.197549in}}%
\pgfpathlineto{\pgfqpoint{2.164113in}{3.221378in}}%
\pgfpathlineto{\pgfqpoint{2.174384in}{3.194982in}}%
\pgfpathlineto{\pgfqpoint{2.184612in}{3.169223in}}%
\pgfpathlineto{\pgfqpoint{2.194799in}{3.144092in}}%
\pgfpathclose%
\pgfusepath{fill}%
\end{pgfscope}%
\begin{pgfscope}%
\pgfpathrectangle{\pgfqpoint{1.150000in}{0.150000in}}{\pgfqpoint{5.700000in}{5.700000in}}%
\pgfusepath{clip}%
\pgfsetbuttcap%
\pgfsetroundjoin%
\definecolor{currentfill}{rgb}{0.281887,0.150881,0.465405}%
\pgfsetfillcolor{currentfill}%
\pgfsetfillopacity{0.700000}%
\pgfsetlinewidth{0.000000pt}%
\definecolor{currentstroke}{rgb}{0.000000,0.000000,0.000000}%
\pgfsetstrokecolor{currentstroke}%
\pgfsetdash{}{0pt}%
\pgfpathmoveto{\pgfqpoint{3.834553in}{1.399217in}}%
\pgfpathlineto{\pgfqpoint{3.848860in}{1.390912in}}%
\pgfpathlineto{\pgfqpoint{3.863171in}{1.382707in}}%
\pgfpathlineto{\pgfqpoint{3.877486in}{1.374599in}}%
\pgfpathlineto{\pgfqpoint{3.891806in}{1.366589in}}%
\pgfpathlineto{\pgfqpoint{3.883404in}{1.368635in}}%
\pgfpathlineto{\pgfqpoint{3.874991in}{1.371080in}}%
\pgfpathlineto{\pgfqpoint{3.866565in}{1.373931in}}%
\pgfpathlineto{\pgfqpoint{3.858128in}{1.377196in}}%
\pgfpathlineto{\pgfqpoint{3.843776in}{1.385892in}}%
\pgfpathlineto{\pgfqpoint{3.829428in}{1.394685in}}%
\pgfpathlineto{\pgfqpoint{3.815084in}{1.403578in}}%
\pgfpathlineto{\pgfqpoint{3.800743in}{1.412569in}}%
\pgfpathlineto{\pgfqpoint{3.809215in}{1.408609in}}%
\pgfpathlineto{\pgfqpoint{3.817673in}{1.405069in}}%
\pgfpathlineto{\pgfqpoint{3.826119in}{1.401941in}}%
\pgfpathlineto{\pgfqpoint{3.834553in}{1.399217in}}%
\pgfpathclose%
\pgfusepath{fill}%
\end{pgfscope}%
\begin{pgfscope}%
\pgfpathrectangle{\pgfqpoint{1.150000in}{0.150000in}}{\pgfqpoint{5.700000in}{5.700000in}}%
\pgfusepath{clip}%
\pgfsetbuttcap%
\pgfsetroundjoin%
\definecolor{currentfill}{rgb}{0.845561,0.887322,0.099702}%
\pgfsetfillcolor{currentfill}%
\pgfsetfillopacity{0.700000}%
\pgfsetlinewidth{0.000000pt}%
\definecolor{currentstroke}{rgb}{0.000000,0.000000,0.000000}%
\pgfsetstrokecolor{currentstroke}%
\pgfsetdash{}{0pt}%
\pgfpathmoveto{\pgfqpoint{1.884110in}{3.645167in}}%
\pgfpathlineto{\pgfqpoint{1.898881in}{3.617989in}}%
\pgfpathlineto{\pgfqpoint{1.913637in}{3.591032in}}%
\pgfpathlineto{\pgfqpoint{1.928380in}{3.564294in}}%
\pgfpathlineto{\pgfqpoint{1.943108in}{3.537772in}}%
\pgfpathlineto{\pgfqpoint{1.932630in}{3.564461in}}%
\pgfpathlineto{\pgfqpoint{1.922108in}{3.591781in}}%
\pgfpathlineto{\pgfqpoint{1.911540in}{3.619744in}}%
\pgfpathlineto{\pgfqpoint{1.900926in}{3.648359in}}%
\pgfpathlineto{\pgfqpoint{1.886099in}{3.675711in}}%
\pgfpathlineto{\pgfqpoint{1.871258in}{3.703282in}}%
\pgfpathlineto{\pgfqpoint{1.856402in}{3.731073in}}%
\pgfpathlineto{\pgfqpoint{1.841532in}{3.759088in}}%
\pgfpathlineto{\pgfqpoint{1.852247in}{3.729627in}}%
\pgfpathlineto{\pgfqpoint{1.862914in}{3.700826in}}%
\pgfpathlineto{\pgfqpoint{1.873535in}{3.672676in}}%
\pgfpathlineto{\pgfqpoint{1.884110in}{3.645167in}}%
\pgfpathclose%
\pgfusepath{fill}%
\end{pgfscope}%
\begin{pgfscope}%
\pgfpathrectangle{\pgfqpoint{1.150000in}{0.150000in}}{\pgfqpoint{5.700000in}{5.700000in}}%
\pgfusepath{clip}%
\pgfsetbuttcap%
\pgfsetroundjoin%
\definecolor{currentfill}{rgb}{0.265145,0.232956,0.516599}%
\pgfsetfillcolor{currentfill}%
\pgfsetfillopacity{0.700000}%
\pgfsetlinewidth{0.000000pt}%
\definecolor{currentstroke}{rgb}{0.000000,0.000000,0.000000}%
\pgfsetstrokecolor{currentstroke}%
\pgfsetdash{}{0pt}%
\pgfpathmoveto{\pgfqpoint{3.571732in}{1.570047in}}%
\pgfpathlineto{\pgfqpoint{3.586025in}{1.559443in}}%
\pgfpathlineto{\pgfqpoint{3.600320in}{1.548941in}}%
\pgfpathlineto{\pgfqpoint{3.614617in}{1.538542in}}%
\pgfpathlineto{\pgfqpoint{3.628917in}{1.528245in}}%
\pgfpathlineto{\pgfqpoint{3.620318in}{1.534741in}}%
\pgfpathlineto{\pgfqpoint{3.611704in}{1.541689in}}%
\pgfpathlineto{\pgfqpoint{3.603074in}{1.549097in}}%
\pgfpathlineto{\pgfqpoint{3.588743in}{1.559927in}}%
\pgfpathlineto{\pgfqpoint{3.574414in}{1.570860in}}%
\pgfpathlineto{\pgfqpoint{3.560087in}{1.581897in}}%
\pgfpathlineto{\pgfqpoint{3.545762in}{1.593036in}}%
\pgfpathlineto{\pgfqpoint{3.554435in}{1.584910in}}%
\pgfpathlineto{\pgfqpoint{3.563092in}{1.577250in}}%
\pgfpathlineto{\pgfqpoint{3.571732in}{1.570047in}}%
\pgfpathclose%
\pgfusepath{fill}%
\end{pgfscope}%
\begin{pgfscope}%
\pgfpathrectangle{\pgfqpoint{1.150000in}{0.150000in}}{\pgfqpoint{5.700000in}{5.700000in}}%
\pgfusepath{clip}%
\pgfsetbuttcap%
\pgfsetroundjoin%
\definecolor{currentfill}{rgb}{0.280894,0.078907,0.402329}%
\pgfsetfillcolor{currentfill}%
\pgfsetfillopacity{0.700000}%
\pgfsetlinewidth{0.000000pt}%
\definecolor{currentstroke}{rgb}{0.000000,0.000000,0.000000}%
\pgfsetstrokecolor{currentstroke}%
\pgfsetdash{}{0pt}%
\pgfpathmoveto{\pgfqpoint{4.245106in}{1.249879in}}%
\pgfpathlineto{\pgfqpoint{4.259496in}{1.245187in}}%
\pgfpathlineto{\pgfqpoint{4.273894in}{1.240588in}}%
\pgfpathlineto{\pgfqpoint{4.288299in}{1.236083in}}%
\pgfpathlineto{\pgfqpoint{4.302711in}{1.231673in}}%
\pgfpathlineto{\pgfqpoint{4.294520in}{1.226980in}}%
\pgfpathlineto{\pgfqpoint{4.286323in}{1.222600in}}%
\pgfpathlineto{\pgfqpoint{4.278120in}{1.218539in}}%
\pgfpathlineto{\pgfqpoint{4.269910in}{1.214802in}}%
\pgfpathlineto{\pgfqpoint{4.255481in}{1.219854in}}%
\pgfpathlineto{\pgfqpoint{4.241058in}{1.225001in}}%
\pgfpathlineto{\pgfqpoint{4.226642in}{1.230241in}}%
\pgfpathlineto{\pgfqpoint{4.212233in}{1.235575in}}%
\pgfpathlineto{\pgfqpoint{4.220461in}{1.238662in}}%
\pgfpathlineto{\pgfqpoint{4.228683in}{1.242080in}}%
\pgfpathlineto{\pgfqpoint{4.236898in}{1.245821in}}%
\pgfpathlineto{\pgfqpoint{4.245106in}{1.249879in}}%
\pgfpathclose%
\pgfusepath{fill}%
\end{pgfscope}%
\begin{pgfscope}%
\pgfpathrectangle{\pgfqpoint{1.150000in}{0.150000in}}{\pgfqpoint{5.700000in}{5.700000in}}%
\pgfusepath{clip}%
\pgfsetbuttcap%
\pgfsetroundjoin%
\definecolor{currentfill}{rgb}{0.280267,0.073417,0.397163}%
\pgfsetfillcolor{currentfill}%
\pgfsetfillopacity{0.700000}%
\pgfsetlinewidth{0.000000pt}%
\definecolor{currentstroke}{rgb}{0.000000,0.000000,0.000000}%
\pgfsetstrokecolor{currentstroke}%
\pgfsetdash{}{0pt}%
\pgfpathmoveto{\pgfqpoint{4.393075in}{1.239215in}}%
\pgfpathlineto{\pgfqpoint{4.407511in}{1.235895in}}%
\pgfpathlineto{\pgfqpoint{4.421954in}{1.232668in}}%
\pgfpathlineto{\pgfqpoint{4.436406in}{1.229535in}}%
\pgfpathlineto{\pgfqpoint{4.450866in}{1.226494in}}%
\pgfpathlineto{\pgfqpoint{4.442726in}{1.219388in}}%
\pgfpathlineto{\pgfqpoint{4.434580in}{1.212559in}}%
\pgfpathlineto{\pgfqpoint{4.426430in}{1.206013in}}%
\pgfpathlineto{\pgfqpoint{4.418275in}{1.199756in}}%
\pgfpathlineto{\pgfqpoint{4.403802in}{1.203419in}}%
\pgfpathlineto{\pgfqpoint{4.389338in}{1.207175in}}%
\pgfpathlineto{\pgfqpoint{4.374881in}{1.211024in}}%
\pgfpathlineto{\pgfqpoint{4.360432in}{1.214967in}}%
\pgfpathlineto{\pgfqpoint{4.368601in}{1.220594in}}%
\pgfpathlineto{\pgfqpoint{4.376764in}{1.226515in}}%
\pgfpathlineto{\pgfqpoint{4.384922in}{1.232725in}}%
\pgfpathlineto{\pgfqpoint{4.393075in}{1.239215in}}%
\pgfpathclose%
\pgfusepath{fill}%
\end{pgfscope}%
\begin{pgfscope}%
\pgfpathrectangle{\pgfqpoint{1.150000in}{0.150000in}}{\pgfqpoint{5.700000in}{5.700000in}}%
\pgfusepath{clip}%
\pgfsetbuttcap%
\pgfsetroundjoin%
\definecolor{currentfill}{rgb}{0.449368,0.813768,0.335384}%
\pgfsetfillcolor{currentfill}%
\pgfsetfillopacity{0.700000}%
\pgfsetlinewidth{0.000000pt}%
\definecolor{currentstroke}{rgb}{0.000000,0.000000,0.000000}%
\pgfsetstrokecolor{currentstroke}%
\pgfsetdash{}{0pt}%
\pgfpathmoveto{\pgfqpoint{2.136390in}{3.238769in}}%
\pgfpathlineto{\pgfqpoint{2.151007in}{3.214826in}}%
\pgfpathlineto{\pgfqpoint{2.165615in}{3.191066in}}%
\pgfpathlineto{\pgfqpoint{2.180212in}{3.167488in}}%
\pgfpathlineto{\pgfqpoint{2.194799in}{3.144092in}}%
\pgfpathlineto{\pgfqpoint{2.184612in}{3.169223in}}%
\pgfpathlineto{\pgfqpoint{2.174384in}{3.194982in}}%
\pgfpathlineto{\pgfqpoint{2.164113in}{3.221378in}}%
\pgfpathlineto{\pgfqpoint{2.149459in}{3.245388in}}%
\pgfpathlineto{\pgfqpoint{2.134793in}{3.269581in}}%
\pgfpathlineto{\pgfqpoint{2.120117in}{3.293957in}}%
\pgfpathlineto{\pgfqpoint{2.105431in}{3.318519in}}%
\pgfpathlineto{\pgfqpoint{2.115793in}{3.291293in}}%
\pgfpathlineto{\pgfqpoint{2.126113in}{3.264713in}}%
\pgfpathlineto{\pgfqpoint{2.136390in}{3.238769in}}%
\pgfpathclose%
\pgfusepath{fill}%
\end{pgfscope}%
\begin{pgfscope}%
\pgfpathrectangle{\pgfqpoint{1.150000in}{0.150000in}}{\pgfqpoint{5.700000in}{5.700000in}}%
\pgfusepath{clip}%
\pgfsetbuttcap%
\pgfsetroundjoin%
\definecolor{currentfill}{rgb}{0.283197,0.115680,0.436115}%
\pgfsetfillcolor{currentfill}%
\pgfsetfillopacity{0.700000}%
\pgfsetlinewidth{0.000000pt}%
\definecolor{currentstroke}{rgb}{0.000000,0.000000,0.000000}%
\pgfsetstrokecolor{currentstroke}%
\pgfsetdash{}{0pt}%
\pgfpathmoveto{\pgfqpoint{4.722165in}{1.323023in}}%
\pgfpathlineto{\pgfqpoint{4.736728in}{1.322955in}}%
\pgfpathlineto{\pgfqpoint{4.751302in}{1.322979in}}%
\pgfpathlineto{\pgfqpoint{4.765885in}{1.323097in}}%
\pgfpathlineto{\pgfqpoint{4.780480in}{1.323309in}}%
\pgfpathlineto{\pgfqpoint{4.772406in}{1.311104in}}%
\pgfpathlineto{\pgfqpoint{4.764330in}{1.299086in}}%
\pgfpathlineto{\pgfqpoint{4.756251in}{1.287262in}}%
\pgfpathlineto{\pgfqpoint{4.748169in}{1.275637in}}%
\pgfpathlineto{\pgfqpoint{4.733573in}{1.275997in}}%
\pgfpathlineto{\pgfqpoint{4.718987in}{1.276449in}}%
\pgfpathlineto{\pgfqpoint{4.704412in}{1.276995in}}%
\pgfpathlineto{\pgfqpoint{4.689846in}{1.277634in}}%
\pgfpathlineto{\pgfqpoint{4.697930in}{1.288682in}}%
\pgfpathlineto{\pgfqpoint{4.706012in}{1.299933in}}%
\pgfpathlineto{\pgfqpoint{4.714090in}{1.311382in}}%
\pgfpathlineto{\pgfqpoint{4.722165in}{1.323023in}}%
\pgfpathclose%
\pgfusepath{fill}%
\end{pgfscope}%
\begin{pgfscope}%
\pgfpathrectangle{\pgfqpoint{1.150000in}{0.150000in}}{\pgfqpoint{5.700000in}{5.700000in}}%
\pgfusepath{clip}%
\pgfsetbuttcap%
\pgfsetroundjoin%
\definecolor{currentfill}{rgb}{0.282884,0.135920,0.453427}%
\pgfsetfillcolor{currentfill}%
\pgfsetfillopacity{0.700000}%
\pgfsetlinewidth{0.000000pt}%
\definecolor{currentstroke}{rgb}{0.000000,0.000000,0.000000}%
\pgfsetstrokecolor{currentstroke}%
\pgfsetdash{}{0pt}%
\pgfpathmoveto{\pgfqpoint{4.812750in}{1.373896in}}%
\pgfpathlineto{\pgfqpoint{4.827356in}{1.374754in}}%
\pgfpathlineto{\pgfqpoint{4.841973in}{1.375706in}}%
\pgfpathlineto{\pgfqpoint{4.856601in}{1.376751in}}%
\pgfpathlineto{\pgfqpoint{4.871241in}{1.377890in}}%
\pgfpathlineto{\pgfqpoint{4.863176in}{1.364442in}}%
\pgfpathlineto{\pgfqpoint{4.855109in}{1.351155in}}%
\pgfpathlineto{\pgfqpoint{4.847040in}{1.338034in}}%
\pgfpathlineto{\pgfqpoint{4.838968in}{1.325086in}}%
\pgfpathlineto{\pgfqpoint{4.824329in}{1.324502in}}%
\pgfpathlineto{\pgfqpoint{4.809702in}{1.324011in}}%
\pgfpathlineto{\pgfqpoint{4.795086in}{1.323613in}}%
\pgfpathlineto{\pgfqpoint{4.780480in}{1.323309in}}%
\pgfpathlineto{\pgfqpoint{4.788552in}{1.335696in}}%
\pgfpathlineto{\pgfqpoint{4.796620in}{1.348260in}}%
\pgfpathlineto{\pgfqpoint{4.804687in}{1.360995in}}%
\pgfpathlineto{\pgfqpoint{4.812750in}{1.373896in}}%
\pgfpathclose%
\pgfusepath{fill}%
\end{pgfscope}%
\begin{pgfscope}%
\pgfpathrectangle{\pgfqpoint{1.150000in}{0.150000in}}{\pgfqpoint{5.700000in}{5.700000in}}%
\pgfusepath{clip}%
\pgfsetbuttcap%
\pgfsetroundjoin%
\definecolor{currentfill}{rgb}{0.282327,0.094955,0.417331}%
\pgfsetfillcolor{currentfill}%
\pgfsetfillopacity{0.700000}%
\pgfsetlinewidth{0.000000pt}%
\definecolor{currentstroke}{rgb}{0.000000,0.000000,0.000000}%
\pgfsetstrokecolor{currentstroke}%
\pgfsetdash{}{0pt}%
\pgfpathmoveto{\pgfqpoint{4.631685in}{1.281120in}}%
\pgfpathlineto{\pgfqpoint{4.646210in}{1.280109in}}%
\pgfpathlineto{\pgfqpoint{4.660746in}{1.279191in}}%
\pgfpathlineto{\pgfqpoint{4.675291in}{1.278366in}}%
\pgfpathlineto{\pgfqpoint{4.689846in}{1.277634in}}%
\pgfpathlineto{\pgfqpoint{4.681759in}{1.266795in}}%
\pgfpathlineto{\pgfqpoint{4.673669in}{1.256171in}}%
\pgfpathlineto{\pgfqpoint{4.665575in}{1.245767in}}%
\pgfpathlineto{\pgfqpoint{4.657479in}{1.235590in}}%
\pgfpathlineto{\pgfqpoint{4.642919in}{1.236910in}}%
\pgfpathlineto{\pgfqpoint{4.628369in}{1.238322in}}%
\pgfpathlineto{\pgfqpoint{4.613829in}{1.239827in}}%
\pgfpathlineto{\pgfqpoint{4.599298in}{1.241426in}}%
\pgfpathlineto{\pgfqpoint{4.607400in}{1.251009in}}%
\pgfpathlineto{\pgfqpoint{4.615498in}{1.260823in}}%
\pgfpathlineto{\pgfqpoint{4.623593in}{1.270861in}}%
\pgfpathlineto{\pgfqpoint{4.631685in}{1.281120in}}%
\pgfpathclose%
\pgfusepath{fill}%
\end{pgfscope}%
\begin{pgfscope}%
\pgfpathrectangle{\pgfqpoint{1.150000in}{0.150000in}}{\pgfqpoint{5.700000in}{5.700000in}}%
\pgfusepath{clip}%
\pgfsetbuttcap%
\pgfsetroundjoin%
\definecolor{currentfill}{rgb}{0.282656,0.100196,0.422160}%
\pgfsetfillcolor{currentfill}%
\pgfsetfillopacity{0.700000}%
\pgfsetlinewidth{0.000000pt}%
\definecolor{currentstroke}{rgb}{0.000000,0.000000,0.000000}%
\pgfsetstrokecolor{currentstroke}%
\pgfsetdash{}{0pt}%
\pgfpathmoveto{\pgfqpoint{4.097185in}{1.281654in}}%
\pgfpathlineto{\pgfqpoint{4.111545in}{1.275562in}}%
\pgfpathlineto{\pgfqpoint{4.125910in}{1.269565in}}%
\pgfpathlineto{\pgfqpoint{4.140282in}{1.263663in}}%
\pgfpathlineto{\pgfqpoint{4.154659in}{1.257856in}}%
\pgfpathlineto{\pgfqpoint{4.146402in}{1.255760in}}%
\pgfpathlineto{\pgfqpoint{4.138136in}{1.254013in}}%
\pgfpathlineto{\pgfqpoint{4.129862in}{1.252621in}}%
\pgfpathlineto{\pgfqpoint{4.121580in}{1.251593in}}%
\pgfpathlineto{\pgfqpoint{4.107179in}{1.258062in}}%
\pgfpathlineto{\pgfqpoint{4.092785in}{1.264626in}}%
\pgfpathlineto{\pgfqpoint{4.078395in}{1.271285in}}%
\pgfpathlineto{\pgfqpoint{4.064011in}{1.278039in}}%
\pgfpathlineto{\pgfqpoint{4.072318in}{1.278398in}}%
\pgfpathlineto{\pgfqpoint{4.080616in}{1.279124in}}%
\pgfpathlineto{\pgfqpoint{4.088905in}{1.280212in}}%
\pgfpathlineto{\pgfqpoint{4.097185in}{1.281654in}}%
\pgfpathclose%
\pgfusepath{fill}%
\end{pgfscope}%
\begin{pgfscope}%
\pgfpathrectangle{\pgfqpoint{1.150000in}{0.150000in}}{\pgfqpoint{5.700000in}{5.700000in}}%
\pgfusepath{clip}%
\pgfsetbuttcap%
\pgfsetroundjoin%
\definecolor{currentfill}{rgb}{0.280255,0.165693,0.476498}%
\pgfsetfillcolor{currentfill}%
\pgfsetfillopacity{0.700000}%
\pgfsetlinewidth{0.000000pt}%
\definecolor{currentstroke}{rgb}{0.000000,0.000000,0.000000}%
\pgfsetstrokecolor{currentstroke}%
\pgfsetdash{}{0pt}%
\pgfpathmoveto{\pgfqpoint{4.903478in}{1.433195in}}%
\pgfpathlineto{\pgfqpoint{4.918132in}{1.434965in}}%
\pgfpathlineto{\pgfqpoint{4.932798in}{1.436829in}}%
\pgfpathlineto{\pgfqpoint{4.947476in}{1.438787in}}%
\pgfpathlineto{\pgfqpoint{4.962166in}{1.440838in}}%
\pgfpathlineto{\pgfqpoint{4.954106in}{1.426265in}}%
\pgfpathlineto{\pgfqpoint{4.946045in}{1.411827in}}%
\pgfpathlineto{\pgfqpoint{4.937981in}{1.397532in}}%
\pgfpathlineto{\pgfqpoint{4.929916in}{1.383382in}}%
\pgfpathlineto{\pgfqpoint{4.915229in}{1.381869in}}%
\pgfpathlineto{\pgfqpoint{4.900555in}{1.380449in}}%
\pgfpathlineto{\pgfqpoint{4.885892in}{1.379123in}}%
\pgfpathlineto{\pgfqpoint{4.871241in}{1.377890in}}%
\pgfpathlineto{\pgfqpoint{4.879304in}{1.391495in}}%
\pgfpathlineto{\pgfqpoint{4.887364in}{1.405251in}}%
\pgfpathlineto{\pgfqpoint{4.895422in}{1.419153in}}%
\pgfpathlineto{\pgfqpoint{4.903478in}{1.433195in}}%
\pgfpathclose%
\pgfusepath{fill}%
\end{pgfscope}%
\begin{pgfscope}%
\pgfpathrectangle{\pgfqpoint{1.150000in}{0.150000in}}{\pgfqpoint{5.700000in}{5.700000in}}%
\pgfusepath{clip}%
\pgfsetbuttcap%
\pgfsetroundjoin%
\definecolor{currentfill}{rgb}{0.269308,0.218818,0.509577}%
\pgfsetfillcolor{currentfill}%
\pgfsetfillopacity{0.700000}%
\pgfsetlinewidth{0.000000pt}%
\definecolor{currentstroke}{rgb}{0.000000,0.000000,0.000000}%
\pgfsetstrokecolor{currentstroke}%
\pgfsetdash{}{0pt}%
\pgfpathmoveto{\pgfqpoint{3.628917in}{1.528245in}}%
\pgfpathlineto{\pgfqpoint{3.643220in}{1.518051in}}%
\pgfpathlineto{\pgfqpoint{3.657525in}{1.507958in}}%
\pgfpathlineto{\pgfqpoint{3.671833in}{1.497967in}}%
\pgfpathlineto{\pgfqpoint{3.686143in}{1.488077in}}%
\pgfpathlineto{\pgfqpoint{3.677585in}{1.493868in}}%
\pgfpathlineto{\pgfqpoint{3.669011in}{1.500105in}}%
\pgfpathlineto{\pgfqpoint{3.660422in}{1.506797in}}%
\pgfpathlineto{\pgfqpoint{3.646081in}{1.517219in}}%
\pgfpathlineto{\pgfqpoint{3.631743in}{1.527743in}}%
\pgfpathlineto{\pgfqpoint{3.617407in}{1.538369in}}%
\pgfpathlineto{\pgfqpoint{3.603074in}{1.549097in}}%
\pgfpathlineto{\pgfqpoint{3.611704in}{1.541689in}}%
\pgfpathlineto{\pgfqpoint{3.620318in}{1.534741in}}%
\pgfpathlineto{\pgfqpoint{3.628917in}{1.528245in}}%
\pgfpathclose%
\pgfusepath{fill}%
\end{pgfscope}%
\begin{pgfscope}%
\pgfpathrectangle{\pgfqpoint{1.150000in}{0.150000in}}{\pgfqpoint{5.700000in}{5.700000in}}%
\pgfusepath{clip}%
\pgfsetbuttcap%
\pgfsetroundjoin%
\definecolor{currentfill}{rgb}{0.282623,0.140926,0.457517}%
\pgfsetfillcolor{currentfill}%
\pgfsetfillopacity{0.700000}%
\pgfsetlinewidth{0.000000pt}%
\definecolor{currentstroke}{rgb}{0.000000,0.000000,0.000000}%
\pgfsetstrokecolor{currentstroke}%
\pgfsetdash{}{0pt}%
\pgfpathmoveto{\pgfqpoint{3.891806in}{1.366589in}}%
\pgfpathlineto{\pgfqpoint{3.906130in}{1.358677in}}%
\pgfpathlineto{\pgfqpoint{3.920459in}{1.350863in}}%
\pgfpathlineto{\pgfqpoint{3.934792in}{1.343146in}}%
\pgfpathlineto{\pgfqpoint{3.949130in}{1.335526in}}%
\pgfpathlineto{\pgfqpoint{3.940758in}{1.336895in}}%
\pgfpathlineto{\pgfqpoint{3.932375in}{1.338657in}}%
\pgfpathlineto{\pgfqpoint{3.923982in}{1.340820in}}%
\pgfpathlineto{\pgfqpoint{3.915577in}{1.343391in}}%
\pgfpathlineto{\pgfqpoint{3.901208in}{1.351696in}}%
\pgfpathlineto{\pgfqpoint{3.886844in}{1.360099in}}%
\pgfpathlineto{\pgfqpoint{3.872484in}{1.368598in}}%
\pgfpathlineto{\pgfqpoint{3.858128in}{1.377196in}}%
\pgfpathlineto{\pgfqpoint{3.866565in}{1.373931in}}%
\pgfpathlineto{\pgfqpoint{3.874991in}{1.371080in}}%
\pgfpathlineto{\pgfqpoint{3.883404in}{1.368635in}}%
\pgfpathlineto{\pgfqpoint{3.891806in}{1.366589in}}%
\pgfpathclose%
\pgfusepath{fill}%
\end{pgfscope}%
\begin{pgfscope}%
\pgfpathrectangle{\pgfqpoint{1.150000in}{0.150000in}}{\pgfqpoint{5.700000in}{5.700000in}}%
\pgfusepath{clip}%
\pgfsetbuttcap%
\pgfsetroundjoin%
\definecolor{currentfill}{rgb}{0.274128,0.199721,0.498911}%
\pgfsetfillcolor{currentfill}%
\pgfsetfillopacity{0.700000}%
\pgfsetlinewidth{0.000000pt}%
\definecolor{currentstroke}{rgb}{0.000000,0.000000,0.000000}%
\pgfsetstrokecolor{currentstroke}%
\pgfsetdash{}{0pt}%
\pgfpathmoveto{\pgfqpoint{4.994384in}{1.500395in}}%
\pgfpathlineto{\pgfqpoint{5.009091in}{1.503062in}}%
\pgfpathlineto{\pgfqpoint{5.023810in}{1.505823in}}%
\pgfpathlineto{\pgfqpoint{5.038543in}{1.508680in}}%
\pgfpathlineto{\pgfqpoint{5.030487in}{1.493225in}}%
\pgfpathlineto{\pgfqpoint{5.022430in}{1.477884in}}%
\pgfpathlineto{\pgfqpoint{5.014370in}{1.462660in}}%
\pgfpathlineto{\pgfqpoint{5.006309in}{1.447559in}}%
\pgfpathlineto{\pgfqpoint{4.991582in}{1.445224in}}%
\pgfpathlineto{\pgfqpoint{4.976868in}{1.442984in}}%
\pgfpathlineto{\pgfqpoint{4.962166in}{1.440838in}}%
\pgfpathlineto{\pgfqpoint{4.970223in}{1.455543in}}%
\pgfpathlineto{\pgfqpoint{4.978279in}{1.470374in}}%
\pgfpathlineto{\pgfqpoint{4.986332in}{1.485326in}}%
\pgfpathlineto{\pgfqpoint{4.994384in}{1.500395in}}%
\pgfpathclose%
\pgfusepath{fill}%
\end{pgfscope}%
\begin{pgfscope}%
\pgfpathrectangle{\pgfqpoint{1.150000in}{0.150000in}}{\pgfqpoint{5.700000in}{5.700000in}}%
\pgfusepath{clip}%
\pgfsetbuttcap%
\pgfsetroundjoin%
\definecolor{currentfill}{rgb}{0.281446,0.084320,0.407414}%
\pgfsetfillcolor{currentfill}%
\pgfsetfillopacity{0.700000}%
\pgfsetlinewidth{0.000000pt}%
\definecolor{currentstroke}{rgb}{0.000000,0.000000,0.000000}%
\pgfsetstrokecolor{currentstroke}%
\pgfsetdash{}{0pt}%
\pgfpathmoveto{\pgfqpoint{4.541267in}{1.248748in}}%
\pgfpathlineto{\pgfqpoint{4.555761in}{1.246778in}}%
\pgfpathlineto{\pgfqpoint{4.570264in}{1.244901in}}%
\pgfpathlineto{\pgfqpoint{4.584776in}{1.243117in}}%
\pgfpathlineto{\pgfqpoint{4.599298in}{1.241426in}}%
\pgfpathlineto{\pgfqpoint{4.591192in}{1.232080in}}%
\pgfpathlineto{\pgfqpoint{4.583083in}{1.222976in}}%
\pgfpathlineto{\pgfqpoint{4.574970in}{1.214122in}}%
\pgfpathlineto{\pgfqpoint{4.566853in}{1.205522in}}%
\pgfpathlineto{\pgfqpoint{4.552324in}{1.207819in}}%
\pgfpathlineto{\pgfqpoint{4.537804in}{1.210208in}}%
\pgfpathlineto{\pgfqpoint{4.523293in}{1.212690in}}%
\pgfpathlineto{\pgfqpoint{4.508790in}{1.215265in}}%
\pgfpathlineto{\pgfqpoint{4.516916in}{1.223253in}}%
\pgfpathlineto{\pgfqpoint{4.525037in}{1.231500in}}%
\pgfpathlineto{\pgfqpoint{4.533154in}{1.240000in}}%
\pgfpathlineto{\pgfqpoint{4.541267in}{1.248748in}}%
\pgfpathclose%
\pgfusepath{fill}%
\end{pgfscope}%
\begin{pgfscope}%
\pgfpathrectangle{\pgfqpoint{1.150000in}{0.150000in}}{\pgfqpoint{5.700000in}{5.700000in}}%
\pgfusepath{clip}%
\pgfsetbuttcap%
\pgfsetroundjoin%
\definecolor{currentfill}{rgb}{0.545524,0.838039,0.275626}%
\pgfsetfillcolor{currentfill}%
\pgfsetfillopacity{0.700000}%
\pgfsetlinewidth{0.000000pt}%
\definecolor{currentstroke}{rgb}{0.000000,0.000000,0.000000}%
\pgfsetstrokecolor{currentstroke}%
\pgfsetdash{}{0pt}%
\pgfpathmoveto{\pgfqpoint{2.077811in}{3.336420in}}%
\pgfpathlineto{\pgfqpoint{2.092472in}{3.311723in}}%
\pgfpathlineto{\pgfqpoint{2.107122in}{3.287216in}}%
\pgfpathlineto{\pgfqpoint{2.121761in}{3.262899in}}%
\pgfpathlineto{\pgfqpoint{2.136390in}{3.238769in}}%
\pgfpathlineto{\pgfqpoint{2.126113in}{3.264713in}}%
\pgfpathlineto{\pgfqpoint{2.115793in}{3.291293in}}%
\pgfpathlineto{\pgfqpoint{2.105431in}{3.318519in}}%
\pgfpathlineto{\pgfqpoint{2.090733in}{3.343268in}}%
\pgfpathlineto{\pgfqpoint{2.076024in}{3.368206in}}%
\pgfpathlineto{\pgfqpoint{2.061303in}{3.393335in}}%
\pgfpathlineto{\pgfqpoint{2.046571in}{3.418656in}}%
\pgfpathlineto{\pgfqpoint{2.057029in}{3.390593in}}%
\pgfpathlineto{\pgfqpoint{2.067442in}{3.363184in}}%
\pgfpathlineto{\pgfqpoint{2.077811in}{3.336420in}}%
\pgfpathclose%
\pgfusepath{fill}%
\end{pgfscope}%
\begin{pgfscope}%
\pgfpathrectangle{\pgfqpoint{1.150000in}{0.150000in}}{\pgfqpoint{5.700000in}{5.700000in}}%
\pgfusepath{clip}%
\pgfsetbuttcap%
\pgfsetroundjoin%
\definecolor{currentfill}{rgb}{0.273006,0.204520,0.501721}%
\pgfsetfillcolor{currentfill}%
\pgfsetfillopacity{0.700000}%
\pgfsetlinewidth{0.000000pt}%
\definecolor{currentstroke}{rgb}{0.000000,0.000000,0.000000}%
\pgfsetstrokecolor{currentstroke}%
\pgfsetdash{}{0pt}%
\pgfpathmoveto{\pgfqpoint{3.686143in}{1.488077in}}%
\pgfpathlineto{\pgfqpoint{3.700457in}{1.478288in}}%
\pgfpathlineto{\pgfqpoint{3.714774in}{1.468599in}}%
\pgfpathlineto{\pgfqpoint{3.729094in}{1.459011in}}%
\pgfpathlineto{\pgfqpoint{3.743417in}{1.449523in}}%
\pgfpathlineto{\pgfqpoint{3.734896in}{1.454611in}}%
\pgfpathlineto{\pgfqpoint{3.726361in}{1.460140in}}%
\pgfpathlineto{\pgfqpoint{3.717812in}{1.466117in}}%
\pgfpathlineto{\pgfqpoint{3.703460in}{1.476136in}}%
\pgfpathlineto{\pgfqpoint{3.689111in}{1.486255in}}%
\pgfpathlineto{\pgfqpoint{3.674765in}{1.496475in}}%
\pgfpathlineto{\pgfqpoint{3.660422in}{1.506797in}}%
\pgfpathlineto{\pgfqpoint{3.669011in}{1.500105in}}%
\pgfpathlineto{\pgfqpoint{3.677585in}{1.493868in}}%
\pgfpathlineto{\pgfqpoint{3.686143in}{1.488077in}}%
\pgfpathclose%
\pgfusepath{fill}%
\end{pgfscope}%
\begin{pgfscope}%
\pgfpathrectangle{\pgfqpoint{1.150000in}{0.150000in}}{\pgfqpoint{5.700000in}{5.700000in}}%
\pgfusepath{clip}%
\pgfsetbuttcap%
\pgfsetroundjoin%
\definecolor{currentfill}{rgb}{0.280894,0.078907,0.402329}%
\pgfsetfillcolor{currentfill}%
\pgfsetfillopacity{0.700000}%
\pgfsetlinewidth{0.000000pt}%
\definecolor{currentstroke}{rgb}{0.000000,0.000000,0.000000}%
\pgfsetstrokecolor{currentstroke}%
\pgfsetdash{}{0pt}%
\pgfpathmoveto{\pgfqpoint{4.302711in}{1.231673in}}%
\pgfpathlineto{\pgfqpoint{4.317130in}{1.227356in}}%
\pgfpathlineto{\pgfqpoint{4.331556in}{1.223133in}}%
\pgfpathlineto{\pgfqpoint{4.345990in}{1.219003in}}%
\pgfpathlineto{\pgfqpoint{4.360432in}{1.214967in}}%
\pgfpathlineto{\pgfqpoint{4.352257in}{1.209641in}}%
\pgfpathlineto{\pgfqpoint{4.344077in}{1.204621in}}%
\pgfpathlineto{\pgfqpoint{4.335890in}{1.199915in}}%
\pgfpathlineto{\pgfqpoint{4.327698in}{1.195530in}}%
\pgfpathlineto{\pgfqpoint{4.313240in}{1.200208in}}%
\pgfpathlineto{\pgfqpoint{4.298790in}{1.204979in}}%
\pgfpathlineto{\pgfqpoint{4.284346in}{1.209844in}}%
\pgfpathlineto{\pgfqpoint{4.269910in}{1.214802in}}%
\pgfpathlineto{\pgfqpoint{4.278120in}{1.218539in}}%
\pgfpathlineto{\pgfqpoint{4.286323in}{1.222600in}}%
\pgfpathlineto{\pgfqpoint{4.294520in}{1.226980in}}%
\pgfpathlineto{\pgfqpoint{4.302711in}{1.231673in}}%
\pgfpathclose%
\pgfusepath{fill}%
\end{pgfscope}%
\begin{pgfscope}%
\pgfpathrectangle{\pgfqpoint{1.150000in}{0.150000in}}{\pgfqpoint{5.700000in}{5.700000in}}%
\pgfusepath{clip}%
\pgfsetbuttcap%
\pgfsetroundjoin%
\definecolor{currentfill}{rgb}{0.282884,0.135920,0.453427}%
\pgfsetfillcolor{currentfill}%
\pgfsetfillopacity{0.700000}%
\pgfsetlinewidth{0.000000pt}%
\definecolor{currentstroke}{rgb}{0.000000,0.000000,0.000000}%
\pgfsetstrokecolor{currentstroke}%
\pgfsetdash{}{0pt}%
\pgfpathmoveto{\pgfqpoint{3.949130in}{1.335526in}}%
\pgfpathlineto{\pgfqpoint{3.963472in}{1.328003in}}%
\pgfpathlineto{\pgfqpoint{3.977820in}{1.320576in}}%
\pgfpathlineto{\pgfqpoint{3.992172in}{1.313247in}}%
\pgfpathlineto{\pgfqpoint{4.006530in}{1.306013in}}%
\pgfpathlineto{\pgfqpoint{3.998187in}{1.306706in}}%
\pgfpathlineto{\pgfqpoint{3.989833in}{1.307786in}}%
\pgfpathlineto{\pgfqpoint{3.981469in}{1.309262in}}%
\pgfpathlineto{\pgfqpoint{3.973095in}{1.311141in}}%
\pgfpathlineto{\pgfqpoint{3.958709in}{1.319059in}}%
\pgfpathlineto{\pgfqpoint{3.944327in}{1.327073in}}%
\pgfpathlineto{\pgfqpoint{3.929950in}{1.335184in}}%
\pgfpathlineto{\pgfqpoint{3.915577in}{1.343391in}}%
\pgfpathlineto{\pgfqpoint{3.923982in}{1.340820in}}%
\pgfpathlineto{\pgfqpoint{3.932375in}{1.338657in}}%
\pgfpathlineto{\pgfqpoint{3.940758in}{1.336895in}}%
\pgfpathlineto{\pgfqpoint{3.949130in}{1.335526in}}%
\pgfpathclose%
\pgfusepath{fill}%
\end{pgfscope}%
\begin{pgfscope}%
\pgfpathrectangle{\pgfqpoint{1.150000in}{0.150000in}}{\pgfqpoint{5.700000in}{5.700000in}}%
\pgfusepath{clip}%
\pgfsetbuttcap%
\pgfsetroundjoin%
\definecolor{currentfill}{rgb}{0.280894,0.078907,0.402329}%
\pgfsetfillcolor{currentfill}%
\pgfsetfillopacity{0.700000}%
\pgfsetlinewidth{0.000000pt}%
\definecolor{currentstroke}{rgb}{0.000000,0.000000,0.000000}%
\pgfsetstrokecolor{currentstroke}%
\pgfsetdash{}{0pt}%
\pgfpathmoveto{\pgfqpoint{4.450866in}{1.226494in}}%
\pgfpathlineto{\pgfqpoint{4.465334in}{1.223547in}}%
\pgfpathlineto{\pgfqpoint{4.479811in}{1.220694in}}%
\pgfpathlineto{\pgfqpoint{4.494297in}{1.217933in}}%
\pgfpathlineto{\pgfqpoint{4.508790in}{1.215265in}}%
\pgfpathlineto{\pgfqpoint{4.500661in}{1.207543in}}%
\pgfpathlineto{\pgfqpoint{4.492527in}{1.200093in}}%
\pgfpathlineto{\pgfqpoint{4.484388in}{1.192921in}}%
\pgfpathlineto{\pgfqpoint{4.476245in}{1.186034in}}%
\pgfpathlineto{\pgfqpoint{4.461740in}{1.189325in}}%
\pgfpathlineto{\pgfqpoint{4.447243in}{1.192709in}}%
\pgfpathlineto{\pgfqpoint{4.432755in}{1.196186in}}%
\pgfpathlineto{\pgfqpoint{4.418275in}{1.199756in}}%
\pgfpathlineto{\pgfqpoint{4.426430in}{1.206013in}}%
\pgfpathlineto{\pgfqpoint{4.434580in}{1.212559in}}%
\pgfpathlineto{\pgfqpoint{4.442726in}{1.219388in}}%
\pgfpathlineto{\pgfqpoint{4.450866in}{1.226494in}}%
\pgfpathclose%
\pgfusepath{fill}%
\end{pgfscope}%
\begin{pgfscope}%
\pgfpathrectangle{\pgfqpoint{1.150000in}{0.150000in}}{\pgfqpoint{5.700000in}{5.700000in}}%
\pgfusepath{clip}%
\pgfsetbuttcap%
\pgfsetroundjoin%
\definecolor{currentfill}{rgb}{0.282327,0.094955,0.417331}%
\pgfsetfillcolor{currentfill}%
\pgfsetfillopacity{0.700000}%
\pgfsetlinewidth{0.000000pt}%
\definecolor{currentstroke}{rgb}{0.000000,0.000000,0.000000}%
\pgfsetstrokecolor{currentstroke}%
\pgfsetdash{}{0pt}%
\pgfpathmoveto{\pgfqpoint{4.154659in}{1.257856in}}%
\pgfpathlineto{\pgfqpoint{4.169043in}{1.252144in}}%
\pgfpathlineto{\pgfqpoint{4.183433in}{1.246526in}}%
\pgfpathlineto{\pgfqpoint{4.197830in}{1.241003in}}%
\pgfpathlineto{\pgfqpoint{4.212233in}{1.235575in}}%
\pgfpathlineto{\pgfqpoint{4.203996in}{1.232825in}}%
\pgfpathlineto{\pgfqpoint{4.195752in}{1.230418in}}%
\pgfpathlineto{\pgfqpoint{4.187501in}{1.228363in}}%
\pgfpathlineto{\pgfqpoint{4.179242in}{1.226665in}}%
\pgfpathlineto{\pgfqpoint{4.164817in}{1.232756in}}%
\pgfpathlineto{\pgfqpoint{4.150399in}{1.238940in}}%
\pgfpathlineto{\pgfqpoint{4.135987in}{1.245219in}}%
\pgfpathlineto{\pgfqpoint{4.121580in}{1.251593in}}%
\pgfpathlineto{\pgfqpoint{4.129862in}{1.252621in}}%
\pgfpathlineto{\pgfqpoint{4.138136in}{1.254013in}}%
\pgfpathlineto{\pgfqpoint{4.146402in}{1.255760in}}%
\pgfpathlineto{\pgfqpoint{4.154659in}{1.257856in}}%
\pgfpathclose%
\pgfusepath{fill}%
\end{pgfscope}%
\begin{pgfscope}%
\pgfpathrectangle{\pgfqpoint{1.150000in}{0.150000in}}{\pgfqpoint{5.700000in}{5.700000in}}%
\pgfusepath{clip}%
\pgfsetbuttcap%
\pgfsetroundjoin%
\definecolor{currentfill}{rgb}{0.657642,0.860219,0.203082}%
\pgfsetfillcolor{currentfill}%
\pgfsetfillopacity{0.700000}%
\pgfsetlinewidth{0.000000pt}%
\definecolor{currentstroke}{rgb}{0.000000,0.000000,0.000000}%
\pgfsetstrokecolor{currentstroke}%
\pgfsetdash{}{0pt}%
\pgfpathmoveto{\pgfqpoint{2.019051in}{3.437155in}}%
\pgfpathlineto{\pgfqpoint{2.033759in}{3.411676in}}%
\pgfpathlineto{\pgfqpoint{2.048455in}{3.386395in}}%
\pgfpathlineto{\pgfqpoint{2.063139in}{3.361310in}}%
\pgfpathlineto{\pgfqpoint{2.077811in}{3.336420in}}%
\pgfpathlineto{\pgfqpoint{2.067442in}{3.363184in}}%
\pgfpathlineto{\pgfqpoint{2.057029in}{3.390593in}}%
\pgfpathlineto{\pgfqpoint{2.046571in}{3.418656in}}%
\pgfpathlineto{\pgfqpoint{2.031827in}{3.444172in}}%
\pgfpathlineto{\pgfqpoint{2.017071in}{3.469884in}}%
\pgfpathlineto{\pgfqpoint{2.002303in}{3.495794in}}%
\pgfpathlineto{\pgfqpoint{1.987522in}{3.521904in}}%
\pgfpathlineto{\pgfqpoint{1.998077in}{3.492994in}}%
\pgfpathlineto{\pgfqpoint{2.008586in}{3.464747in}}%
\pgfpathlineto{\pgfqpoint{2.019051in}{3.437155in}}%
\pgfpathclose%
\pgfusepath{fill}%
\end{pgfscope}%
\begin{pgfscope}%
\pgfpathrectangle{\pgfqpoint{1.150000in}{0.150000in}}{\pgfqpoint{5.700000in}{5.700000in}}%
\pgfusepath{clip}%
\pgfsetbuttcap%
\pgfsetroundjoin%
\definecolor{currentfill}{rgb}{0.276194,0.190074,0.493001}%
\pgfsetfillcolor{currentfill}%
\pgfsetfillopacity{0.700000}%
\pgfsetlinewidth{0.000000pt}%
\definecolor{currentstroke}{rgb}{0.000000,0.000000,0.000000}%
\pgfsetstrokecolor{currentstroke}%
\pgfsetdash{}{0pt}%
\pgfpathmoveto{\pgfqpoint{3.743417in}{1.449523in}}%
\pgfpathlineto{\pgfqpoint{3.757743in}{1.440135in}}%
\pgfpathlineto{\pgfqpoint{3.772073in}{1.430847in}}%
\pgfpathlineto{\pgfqpoint{3.786406in}{1.421658in}}%
\pgfpathlineto{\pgfqpoint{3.800743in}{1.412569in}}%
\pgfpathlineto{\pgfqpoint{3.792259in}{1.416955in}}%
\pgfpathlineto{\pgfqpoint{3.783761in}{1.421776in}}%
\pgfpathlineto{\pgfqpoint{3.775250in}{1.427040in}}%
\pgfpathlineto{\pgfqpoint{3.760886in}{1.436660in}}%
\pgfpathlineto{\pgfqpoint{3.746525in}{1.446379in}}%
\pgfpathlineto{\pgfqpoint{3.732167in}{1.456198in}}%
\pgfpathlineto{\pgfqpoint{3.717812in}{1.466117in}}%
\pgfpathlineto{\pgfqpoint{3.726361in}{1.460140in}}%
\pgfpathlineto{\pgfqpoint{3.734896in}{1.454611in}}%
\pgfpathlineto{\pgfqpoint{3.743417in}{1.449523in}}%
\pgfpathclose%
\pgfusepath{fill}%
\end{pgfscope}%
\begin{pgfscope}%
\pgfpathrectangle{\pgfqpoint{1.150000in}{0.150000in}}{\pgfqpoint{5.700000in}{5.700000in}}%
\pgfusepath{clip}%
\pgfsetbuttcap%
\pgfsetroundjoin%
\definecolor{currentfill}{rgb}{0.283187,0.125848,0.444960}%
\pgfsetfillcolor{currentfill}%
\pgfsetfillopacity{0.700000}%
\pgfsetlinewidth{0.000000pt}%
\definecolor{currentstroke}{rgb}{0.000000,0.000000,0.000000}%
\pgfsetstrokecolor{currentstroke}%
\pgfsetdash{}{0pt}%
\pgfpathmoveto{\pgfqpoint{4.780480in}{1.323309in}}%
\pgfpathlineto{\pgfqpoint{4.795086in}{1.323613in}}%
\pgfpathlineto{\pgfqpoint{4.809702in}{1.324011in}}%
\pgfpathlineto{\pgfqpoint{4.824329in}{1.324502in}}%
\pgfpathlineto{\pgfqpoint{4.838968in}{1.325086in}}%
\pgfpathlineto{\pgfqpoint{4.830894in}{1.312316in}}%
\pgfpathlineto{\pgfqpoint{4.822818in}{1.299729in}}%
\pgfpathlineto{\pgfqpoint{4.814740in}{1.287331in}}%
\pgfpathlineto{\pgfqpoint{4.806659in}{1.275127in}}%
\pgfpathlineto{\pgfqpoint{4.792020in}{1.275115in}}%
\pgfpathlineto{\pgfqpoint{4.777393in}{1.275196in}}%
\pgfpathlineto{\pgfqpoint{4.762776in}{1.275370in}}%
\pgfpathlineto{\pgfqpoint{4.748169in}{1.275637in}}%
\pgfpathlineto{\pgfqpoint{4.756251in}{1.287262in}}%
\pgfpathlineto{\pgfqpoint{4.764330in}{1.299086in}}%
\pgfpathlineto{\pgfqpoint{4.772406in}{1.311104in}}%
\pgfpathlineto{\pgfqpoint{4.780480in}{1.323309in}}%
\pgfpathclose%
\pgfusepath{fill}%
\end{pgfscope}%
\begin{pgfscope}%
\pgfpathrectangle{\pgfqpoint{1.150000in}{0.150000in}}{\pgfqpoint{5.700000in}{5.700000in}}%
\pgfusepath{clip}%
\pgfsetbuttcap%
\pgfsetroundjoin%
\definecolor{currentfill}{rgb}{0.282910,0.105393,0.426902}%
\pgfsetfillcolor{currentfill}%
\pgfsetfillopacity{0.700000}%
\pgfsetlinewidth{0.000000pt}%
\definecolor{currentstroke}{rgb}{0.000000,0.000000,0.000000}%
\pgfsetstrokecolor{currentstroke}%
\pgfsetdash{}{0pt}%
\pgfpathmoveto{\pgfqpoint{4.689846in}{1.277634in}}%
\pgfpathlineto{\pgfqpoint{4.704412in}{1.276995in}}%
\pgfpathlineto{\pgfqpoint{4.718987in}{1.276449in}}%
\pgfpathlineto{\pgfqpoint{4.733573in}{1.275997in}}%
\pgfpathlineto{\pgfqpoint{4.748169in}{1.275637in}}%
\pgfpathlineto{\pgfqpoint{4.740085in}{1.264216in}}%
\pgfpathlineto{\pgfqpoint{4.731998in}{1.253006in}}%
\pgfpathlineto{\pgfqpoint{4.723908in}{1.242011in}}%
\pgfpathlineto{\pgfqpoint{4.715815in}{1.231238in}}%
\pgfpathlineto{\pgfqpoint{4.701216in}{1.232187in}}%
\pgfpathlineto{\pgfqpoint{4.686627in}{1.233229in}}%
\pgfpathlineto{\pgfqpoint{4.672048in}{1.234363in}}%
\pgfpathlineto{\pgfqpoint{4.657479in}{1.235590in}}%
\pgfpathlineto{\pgfqpoint{4.665575in}{1.245767in}}%
\pgfpathlineto{\pgfqpoint{4.673669in}{1.256171in}}%
\pgfpathlineto{\pgfqpoint{4.681759in}{1.266795in}}%
\pgfpathlineto{\pgfqpoint{4.689846in}{1.277634in}}%
\pgfpathclose%
\pgfusepath{fill}%
\end{pgfscope}%
\begin{pgfscope}%
\pgfpathrectangle{\pgfqpoint{1.150000in}{0.150000in}}{\pgfqpoint{5.700000in}{5.700000in}}%
\pgfusepath{clip}%
\pgfsetbuttcap%
\pgfsetroundjoin%
\definecolor{currentfill}{rgb}{0.281887,0.150881,0.465405}%
\pgfsetfillcolor{currentfill}%
\pgfsetfillopacity{0.700000}%
\pgfsetlinewidth{0.000000pt}%
\definecolor{currentstroke}{rgb}{0.000000,0.000000,0.000000}%
\pgfsetstrokecolor{currentstroke}%
\pgfsetdash{}{0pt}%
\pgfpathmoveto{\pgfqpoint{4.871241in}{1.377890in}}%
\pgfpathlineto{\pgfqpoint{4.885892in}{1.379123in}}%
\pgfpathlineto{\pgfqpoint{4.900555in}{1.380449in}}%
\pgfpathlineto{\pgfqpoint{4.915229in}{1.381869in}}%
\pgfpathlineto{\pgfqpoint{4.929916in}{1.383382in}}%
\pgfpathlineto{\pgfqpoint{4.921848in}{1.369385in}}%
\pgfpathlineto{\pgfqpoint{4.913779in}{1.355544in}}%
\pgfpathlineto{\pgfqpoint{4.905707in}{1.341866in}}%
\pgfpathlineto{\pgfqpoint{4.897634in}{1.328356in}}%
\pgfpathlineto{\pgfqpoint{4.882951in}{1.327399in}}%
\pgfpathlineto{\pgfqpoint{4.868278in}{1.326535in}}%
\pgfpathlineto{\pgfqpoint{4.853618in}{1.325764in}}%
\pgfpathlineto{\pgfqpoint{4.838968in}{1.325086in}}%
\pgfpathlineto{\pgfqpoint{4.847040in}{1.338034in}}%
\pgfpathlineto{\pgfqpoint{4.855109in}{1.351155in}}%
\pgfpathlineto{\pgfqpoint{4.863176in}{1.364442in}}%
\pgfpathlineto{\pgfqpoint{4.871241in}{1.377890in}}%
\pgfpathclose%
\pgfusepath{fill}%
\end{pgfscope}%
\begin{pgfscope}%
\pgfpathrectangle{\pgfqpoint{1.150000in}{0.150000in}}{\pgfqpoint{5.700000in}{5.700000in}}%
\pgfusepath{clip}%
\pgfsetbuttcap%
\pgfsetroundjoin%
\definecolor{currentfill}{rgb}{0.278012,0.180367,0.486697}%
\pgfsetfillcolor{currentfill}%
\pgfsetfillopacity{0.700000}%
\pgfsetlinewidth{0.000000pt}%
\definecolor{currentstroke}{rgb}{0.000000,0.000000,0.000000}%
\pgfsetstrokecolor{currentstroke}%
\pgfsetdash{}{0pt}%
\pgfpathmoveto{\pgfqpoint{4.962166in}{1.440838in}}%
\pgfpathlineto{\pgfqpoint{4.976868in}{1.442984in}}%
\pgfpathlineto{\pgfqpoint{4.991582in}{1.445224in}}%
\pgfpathlineto{\pgfqpoint{5.006309in}{1.447559in}}%
\pgfpathlineto{\pgfqpoint{4.998246in}{1.432586in}}%
\pgfpathlineto{\pgfqpoint{4.990181in}{1.417745in}}%
\pgfpathlineto{\pgfqpoint{4.982114in}{1.403043in}}%
\pgfpathlineto{\pgfqpoint{4.974046in}{1.388485in}}%
\pgfpathlineto{\pgfqpoint{4.959324in}{1.386690in}}%
\pgfpathlineto{\pgfqpoint{4.944614in}{1.384989in}}%
\pgfpathlineto{\pgfqpoint{4.929916in}{1.383382in}}%
\pgfpathlineto{\pgfqpoint{4.937981in}{1.397532in}}%
\pgfpathlineto{\pgfqpoint{4.946045in}{1.411827in}}%
\pgfpathlineto{\pgfqpoint{4.954106in}{1.426265in}}%
\pgfpathlineto{\pgfqpoint{4.962166in}{1.440838in}}%
\pgfpathclose%
\pgfusepath{fill}%
\end{pgfscope}%
\begin{pgfscope}%
\pgfpathrectangle{\pgfqpoint{1.150000in}{0.150000in}}{\pgfqpoint{5.700000in}{5.700000in}}%
\pgfusepath{clip}%
\pgfsetbuttcap%
\pgfsetroundjoin%
\definecolor{currentfill}{rgb}{0.281924,0.089666,0.412415}%
\pgfsetfillcolor{currentfill}%
\pgfsetfillopacity{0.700000}%
\pgfsetlinewidth{0.000000pt}%
\definecolor{currentstroke}{rgb}{0.000000,0.000000,0.000000}%
\pgfsetstrokecolor{currentstroke}%
\pgfsetdash{}{0pt}%
\pgfpathmoveto{\pgfqpoint{4.599298in}{1.241426in}}%
\pgfpathlineto{\pgfqpoint{4.613829in}{1.239827in}}%
\pgfpathlineto{\pgfqpoint{4.628369in}{1.238322in}}%
\pgfpathlineto{\pgfqpoint{4.642919in}{1.236910in}}%
\pgfpathlineto{\pgfqpoint{4.657479in}{1.235590in}}%
\pgfpathlineto{\pgfqpoint{4.649379in}{1.225645in}}%
\pgfpathlineto{\pgfqpoint{4.641276in}{1.215938in}}%
\pgfpathlineto{\pgfqpoint{4.633170in}{1.206475in}}%
\pgfpathlineto{\pgfqpoint{4.625060in}{1.197263in}}%
\pgfpathlineto{\pgfqpoint{4.610495in}{1.199189in}}%
\pgfpathlineto{\pgfqpoint{4.595938in}{1.201207in}}%
\pgfpathlineto{\pgfqpoint{4.581391in}{1.203319in}}%
\pgfpathlineto{\pgfqpoint{4.566853in}{1.205522in}}%
\pgfpathlineto{\pgfqpoint{4.574970in}{1.214122in}}%
\pgfpathlineto{\pgfqpoint{4.583083in}{1.222976in}}%
\pgfpathlineto{\pgfqpoint{4.591192in}{1.232080in}}%
\pgfpathlineto{\pgfqpoint{4.599298in}{1.241426in}}%
\pgfpathclose%
\pgfusepath{fill}%
\end{pgfscope}%
\begin{pgfscope}%
\pgfpathrectangle{\pgfqpoint{1.150000in}{0.150000in}}{\pgfqpoint{5.700000in}{5.700000in}}%
\pgfusepath{clip}%
\pgfsetbuttcap%
\pgfsetroundjoin%
\definecolor{currentfill}{rgb}{0.283187,0.125848,0.444960}%
\pgfsetfillcolor{currentfill}%
\pgfsetfillopacity{0.700000}%
\pgfsetlinewidth{0.000000pt}%
\definecolor{currentstroke}{rgb}{0.000000,0.000000,0.000000}%
\pgfsetstrokecolor{currentstroke}%
\pgfsetdash{}{0pt}%
\pgfpathmoveto{\pgfqpoint{4.006530in}{1.306013in}}%
\pgfpathlineto{\pgfqpoint{4.020892in}{1.298876in}}%
\pgfpathlineto{\pgfqpoint{4.035260in}{1.291835in}}%
\pgfpathlineto{\pgfqpoint{4.049633in}{1.284889in}}%
\pgfpathlineto{\pgfqpoint{4.064011in}{1.278039in}}%
\pgfpathlineto{\pgfqpoint{4.055695in}{1.278056in}}%
\pgfpathlineto{\pgfqpoint{4.047369in}{1.278456in}}%
\pgfpathlineto{\pgfqpoint{4.039034in}{1.279246in}}%
\pgfpathlineto{\pgfqpoint{4.030688in}{1.280433in}}%
\pgfpathlineto{\pgfqpoint{4.016283in}{1.287966in}}%
\pgfpathlineto{\pgfqpoint{4.001882in}{1.295595in}}%
\pgfpathlineto{\pgfqpoint{3.987486in}{1.303320in}}%
\pgfpathlineto{\pgfqpoint{3.973095in}{1.311141in}}%
\pgfpathlineto{\pgfqpoint{3.981469in}{1.309262in}}%
\pgfpathlineto{\pgfqpoint{3.989833in}{1.307786in}}%
\pgfpathlineto{\pgfqpoint{3.998187in}{1.306706in}}%
\pgfpathlineto{\pgfqpoint{4.006530in}{1.306013in}}%
\pgfpathclose%
\pgfusepath{fill}%
\end{pgfscope}%
\begin{pgfscope}%
\pgfpathrectangle{\pgfqpoint{1.150000in}{0.150000in}}{\pgfqpoint{5.700000in}{5.700000in}}%
\pgfusepath{clip}%
\pgfsetbuttcap%
\pgfsetroundjoin%
\definecolor{currentfill}{rgb}{0.280894,0.078907,0.402329}%
\pgfsetfillcolor{currentfill}%
\pgfsetfillopacity{0.700000}%
\pgfsetlinewidth{0.000000pt}%
\definecolor{currentstroke}{rgb}{0.000000,0.000000,0.000000}%
\pgfsetstrokecolor{currentstroke}%
\pgfsetdash{}{0pt}%
\pgfpathmoveto{\pgfqpoint{4.360432in}{1.214967in}}%
\pgfpathlineto{\pgfqpoint{4.374881in}{1.211024in}}%
\pgfpathlineto{\pgfqpoint{4.389338in}{1.207175in}}%
\pgfpathlineto{\pgfqpoint{4.403802in}{1.203419in}}%
\pgfpathlineto{\pgfqpoint{4.418275in}{1.199756in}}%
\pgfpathlineto{\pgfqpoint{4.410114in}{1.193795in}}%
\pgfpathlineto{\pgfqpoint{4.401948in}{1.188136in}}%
\pgfpathlineto{\pgfqpoint{4.393777in}{1.182786in}}%
\pgfpathlineto{\pgfqpoint{4.385600in}{1.177752in}}%
\pgfpathlineto{\pgfqpoint{4.371114in}{1.182057in}}%
\pgfpathlineto{\pgfqpoint{4.356634in}{1.186455in}}%
\pgfpathlineto{\pgfqpoint{4.342162in}{1.190946in}}%
\pgfpathlineto{\pgfqpoint{4.327698in}{1.195530in}}%
\pgfpathlineto{\pgfqpoint{4.335890in}{1.199915in}}%
\pgfpathlineto{\pgfqpoint{4.344077in}{1.204621in}}%
\pgfpathlineto{\pgfqpoint{4.352257in}{1.209641in}}%
\pgfpathlineto{\pgfqpoint{4.360432in}{1.214967in}}%
\pgfpathclose%
\pgfusepath{fill}%
\end{pgfscope}%
\begin{pgfscope}%
\pgfpathrectangle{\pgfqpoint{1.150000in}{0.150000in}}{\pgfqpoint{5.700000in}{5.700000in}}%
\pgfusepath{clip}%
\pgfsetbuttcap%
\pgfsetroundjoin%
\definecolor{currentfill}{rgb}{0.278012,0.180367,0.486697}%
\pgfsetfillcolor{currentfill}%
\pgfsetfillopacity{0.700000}%
\pgfsetlinewidth{0.000000pt}%
\definecolor{currentstroke}{rgb}{0.000000,0.000000,0.000000}%
\pgfsetstrokecolor{currentstroke}%
\pgfsetdash{}{0pt}%
\pgfpathmoveto{\pgfqpoint{3.800743in}{1.412569in}}%
\pgfpathlineto{\pgfqpoint{3.815084in}{1.403578in}}%
\pgfpathlineto{\pgfqpoint{3.829428in}{1.394685in}}%
\pgfpathlineto{\pgfqpoint{3.843776in}{1.385892in}}%
\pgfpathlineto{\pgfqpoint{3.858128in}{1.377196in}}%
\pgfpathlineto{\pgfqpoint{3.849678in}{1.380882in}}%
\pgfpathlineto{\pgfqpoint{3.841216in}{1.384997in}}%
\pgfpathlineto{\pgfqpoint{3.832741in}{1.389549in}}%
\pgfpathlineto{\pgfqpoint{3.818363in}{1.398774in}}%
\pgfpathlineto{\pgfqpoint{3.803989in}{1.408097in}}%
\pgfpathlineto{\pgfqpoint{3.789618in}{1.417519in}}%
\pgfpathlineto{\pgfqpoint{3.775250in}{1.427040in}}%
\pgfpathlineto{\pgfqpoint{3.783761in}{1.421776in}}%
\pgfpathlineto{\pgfqpoint{3.792259in}{1.416955in}}%
\pgfpathlineto{\pgfqpoint{3.800743in}{1.412569in}}%
\pgfpathclose%
\pgfusepath{fill}%
\end{pgfscope}%
\begin{pgfscope}%
\pgfpathrectangle{\pgfqpoint{1.150000in}{0.150000in}}{\pgfqpoint{5.700000in}{5.700000in}}%
\pgfusepath{clip}%
\pgfsetbuttcap%
\pgfsetroundjoin%
\definecolor{currentfill}{rgb}{0.282327,0.094955,0.417331}%
\pgfsetfillcolor{currentfill}%
\pgfsetfillopacity{0.700000}%
\pgfsetlinewidth{0.000000pt}%
\definecolor{currentstroke}{rgb}{0.000000,0.000000,0.000000}%
\pgfsetstrokecolor{currentstroke}%
\pgfsetdash{}{0pt}%
\pgfpathmoveto{\pgfqpoint{4.212233in}{1.235575in}}%
\pgfpathlineto{\pgfqpoint{4.226642in}{1.230241in}}%
\pgfpathlineto{\pgfqpoint{4.241058in}{1.225001in}}%
\pgfpathlineto{\pgfqpoint{4.255481in}{1.219854in}}%
\pgfpathlineto{\pgfqpoint{4.269910in}{1.214802in}}%
\pgfpathlineto{\pgfqpoint{4.261693in}{1.211398in}}%
\pgfpathlineto{\pgfqpoint{4.253470in}{1.208333in}}%
\pgfpathlineto{\pgfqpoint{4.245239in}{1.205613in}}%
\pgfpathlineto{\pgfqpoint{4.237001in}{1.203246in}}%
\pgfpathlineto{\pgfqpoint{4.222552in}{1.208960in}}%
\pgfpathlineto{\pgfqpoint{4.208109in}{1.214768in}}%
\pgfpathlineto{\pgfqpoint{4.193672in}{1.220669in}}%
\pgfpathlineto{\pgfqpoint{4.179242in}{1.226665in}}%
\pgfpathlineto{\pgfqpoint{4.187501in}{1.228363in}}%
\pgfpathlineto{\pgfqpoint{4.195752in}{1.230418in}}%
\pgfpathlineto{\pgfqpoint{4.203996in}{1.232825in}}%
\pgfpathlineto{\pgfqpoint{4.212233in}{1.235575in}}%
\pgfpathclose%
\pgfusepath{fill}%
\end{pgfscope}%
\begin{pgfscope}%
\pgfpathrectangle{\pgfqpoint{1.150000in}{0.150000in}}{\pgfqpoint{5.700000in}{5.700000in}}%
\pgfusepath{clip}%
\pgfsetbuttcap%
\pgfsetroundjoin%
\definecolor{currentfill}{rgb}{0.772852,0.877868,0.131109}%
\pgfsetfillcolor{currentfill}%
\pgfsetfillopacity{0.700000}%
\pgfsetlinewidth{0.000000pt}%
\definecolor{currentstroke}{rgb}{0.000000,0.000000,0.000000}%
\pgfsetstrokecolor{currentstroke}%
\pgfsetdash{}{0pt}%
\pgfpathmoveto{\pgfqpoint{1.960094in}{3.541093in}}%
\pgfpathlineto{\pgfqpoint{1.974852in}{3.514801in}}%
\pgfpathlineto{\pgfqpoint{1.989598in}{3.488716in}}%
\pgfpathlineto{\pgfqpoint{2.004330in}{3.462835in}}%
\pgfpathlineto{\pgfqpoint{2.019051in}{3.437155in}}%
\pgfpathlineto{\pgfqpoint{2.008586in}{3.464747in}}%
\pgfpathlineto{\pgfqpoint{1.998077in}{3.492994in}}%
\pgfpathlineto{\pgfqpoint{1.987522in}{3.521904in}}%
\pgfpathlineto{\pgfqpoint{1.972729in}{3.548215in}}%
\pgfpathlineto{\pgfqpoint{1.957922in}{3.574731in}}%
\pgfpathlineto{\pgfqpoint{1.943103in}{3.601452in}}%
\pgfpathlineto{\pgfqpoint{1.928270in}{3.628380in}}%
\pgfpathlineto{\pgfqpoint{1.938925in}{3.598615in}}%
\pgfpathlineto{\pgfqpoint{1.949532in}{3.569522in}}%
\pgfpathlineto{\pgfqpoint{1.960094in}{3.541093in}}%
\pgfpathclose%
\pgfusepath{fill}%
\end{pgfscope}%
\begin{pgfscope}%
\pgfpathrectangle{\pgfqpoint{1.150000in}{0.150000in}}{\pgfqpoint{5.700000in}{5.700000in}}%
\pgfusepath{clip}%
\pgfsetbuttcap%
\pgfsetroundjoin%
\definecolor{currentfill}{rgb}{0.281446,0.084320,0.407414}%
\pgfsetfillcolor{currentfill}%
\pgfsetfillopacity{0.700000}%
\pgfsetlinewidth{0.000000pt}%
\definecolor{currentstroke}{rgb}{0.000000,0.000000,0.000000}%
\pgfsetstrokecolor{currentstroke}%
\pgfsetdash{}{0pt}%
\pgfpathmoveto{\pgfqpoint{4.508790in}{1.215265in}}%
\pgfpathlineto{\pgfqpoint{4.523293in}{1.212690in}}%
\pgfpathlineto{\pgfqpoint{4.537804in}{1.210208in}}%
\pgfpathlineto{\pgfqpoint{4.552324in}{1.207819in}}%
\pgfpathlineto{\pgfqpoint{4.566853in}{1.205522in}}%
\pgfpathlineto{\pgfqpoint{4.558733in}{1.197184in}}%
\pgfpathlineto{\pgfqpoint{4.550608in}{1.189112in}}%
\pgfpathlineto{\pgfqpoint{4.542480in}{1.181315in}}%
\pgfpathlineto{\pgfqpoint{4.534347in}{1.173797in}}%
\pgfpathlineto{\pgfqpoint{4.519809in}{1.176717in}}%
\pgfpathlineto{\pgfqpoint{4.505279in}{1.179730in}}%
\pgfpathlineto{\pgfqpoint{4.490758in}{1.182836in}}%
\pgfpathlineto{\pgfqpoint{4.476245in}{1.186034in}}%
\pgfpathlineto{\pgfqpoint{4.484388in}{1.192921in}}%
\pgfpathlineto{\pgfqpoint{4.492527in}{1.200093in}}%
\pgfpathlineto{\pgfqpoint{4.500661in}{1.207543in}}%
\pgfpathlineto{\pgfqpoint{4.508790in}{1.215265in}}%
\pgfpathclose%
\pgfusepath{fill}%
\end{pgfscope}%
\begin{pgfscope}%
\pgfpathrectangle{\pgfqpoint{1.150000in}{0.150000in}}{\pgfqpoint{5.700000in}{5.700000in}}%
\pgfusepath{clip}%
\pgfsetbuttcap%
\pgfsetroundjoin%
\definecolor{currentfill}{rgb}{0.283229,0.120777,0.440584}%
\pgfsetfillcolor{currentfill}%
\pgfsetfillopacity{0.700000}%
\pgfsetlinewidth{0.000000pt}%
\definecolor{currentstroke}{rgb}{0.000000,0.000000,0.000000}%
\pgfsetstrokecolor{currentstroke}%
\pgfsetdash{}{0pt}%
\pgfpathmoveto{\pgfqpoint{4.064011in}{1.278039in}}%
\pgfpathlineto{\pgfqpoint{4.078395in}{1.271285in}}%
\pgfpathlineto{\pgfqpoint{4.092785in}{1.264626in}}%
\pgfpathlineto{\pgfqpoint{4.107179in}{1.258062in}}%
\pgfpathlineto{\pgfqpoint{4.121580in}{1.251593in}}%
\pgfpathlineto{\pgfqpoint{4.113289in}{1.250935in}}%
\pgfpathlineto{\pgfqpoint{4.104989in}{1.250655in}}%
\pgfpathlineto{\pgfqpoint{4.096680in}{1.250759in}}%
\pgfpathlineto{\pgfqpoint{4.088362in}{1.251255in}}%
\pgfpathlineto{\pgfqpoint{4.073936in}{1.258406in}}%
\pgfpathlineto{\pgfqpoint{4.059515in}{1.265653in}}%
\pgfpathlineto{\pgfqpoint{4.045099in}{1.272995in}}%
\pgfpathlineto{\pgfqpoint{4.030688in}{1.280433in}}%
\pgfpathlineto{\pgfqpoint{4.039034in}{1.279246in}}%
\pgfpathlineto{\pgfqpoint{4.047369in}{1.278456in}}%
\pgfpathlineto{\pgfqpoint{4.055695in}{1.278056in}}%
\pgfpathlineto{\pgfqpoint{4.064011in}{1.278039in}}%
\pgfpathclose%
\pgfusepath{fill}%
\end{pgfscope}%
\begin{pgfscope}%
\pgfpathrectangle{\pgfqpoint{1.150000in}{0.150000in}}{\pgfqpoint{5.700000in}{5.700000in}}%
\pgfusepath{clip}%
\pgfsetbuttcap%
\pgfsetroundjoin%
\definecolor{currentfill}{rgb}{0.280255,0.165693,0.476498}%
\pgfsetfillcolor{currentfill}%
\pgfsetfillopacity{0.700000}%
\pgfsetlinewidth{0.000000pt}%
\definecolor{currentstroke}{rgb}{0.000000,0.000000,0.000000}%
\pgfsetstrokecolor{currentstroke}%
\pgfsetdash{}{0pt}%
\pgfpathmoveto{\pgfqpoint{3.858128in}{1.377196in}}%
\pgfpathlineto{\pgfqpoint{3.872484in}{1.368598in}}%
\pgfpathlineto{\pgfqpoint{3.886844in}{1.360099in}}%
\pgfpathlineto{\pgfqpoint{3.901208in}{1.351696in}}%
\pgfpathlineto{\pgfqpoint{3.915577in}{1.343391in}}%
\pgfpathlineto{\pgfqpoint{3.907160in}{1.346378in}}%
\pgfpathlineto{\pgfqpoint{3.898732in}{1.349789in}}%
\pgfpathlineto{\pgfqpoint{3.890291in}{1.353630in}}%
\pgfpathlineto{\pgfqpoint{3.875898in}{1.362463in}}%
\pgfpathlineto{\pgfqpoint{3.861509in}{1.371394in}}%
\pgfpathlineto{\pgfqpoint{3.847123in}{1.380423in}}%
\pgfpathlineto{\pgfqpoint{3.832741in}{1.389549in}}%
\pgfpathlineto{\pgfqpoint{3.841216in}{1.384997in}}%
\pgfpathlineto{\pgfqpoint{3.849678in}{1.380882in}}%
\pgfpathlineto{\pgfqpoint{3.858128in}{1.377196in}}%
\pgfpathclose%
\pgfusepath{fill}%
\end{pgfscope}%
\begin{pgfscope}%
\pgfpathrectangle{\pgfqpoint{1.150000in}{0.150000in}}{\pgfqpoint{5.700000in}{5.700000in}}%
\pgfusepath{clip}%
\pgfsetbuttcap%
\pgfsetroundjoin%
\definecolor{currentfill}{rgb}{0.280255,0.165693,0.476498}%
\pgfsetfillcolor{currentfill}%
\pgfsetfillopacity{0.700000}%
\pgfsetlinewidth{0.000000pt}%
\definecolor{currentstroke}{rgb}{0.000000,0.000000,0.000000}%
\pgfsetstrokecolor{currentstroke}%
\pgfsetdash{}{0pt}%
\pgfpathmoveto{\pgfqpoint{4.929916in}{1.383382in}}%
\pgfpathlineto{\pgfqpoint{4.944614in}{1.384989in}}%
\pgfpathlineto{\pgfqpoint{4.959324in}{1.386690in}}%
\pgfpathlineto{\pgfqpoint{4.974046in}{1.388485in}}%
\pgfpathlineto{\pgfqpoint{4.965976in}{1.374074in}}%
\pgfpathlineto{\pgfqpoint{4.957904in}{1.359818in}}%
\pgfpathlineto{\pgfqpoint{4.949830in}{1.345721in}}%
\pgfpathlineto{\pgfqpoint{4.941754in}{1.331788in}}%
\pgfpathlineto{\pgfqpoint{4.927036in}{1.330551in}}%
\pgfpathlineto{\pgfqpoint{4.912329in}{1.329407in}}%
\pgfpathlineto{\pgfqpoint{4.897634in}{1.328356in}}%
\pgfpathlineto{\pgfqpoint{4.905707in}{1.341866in}}%
\pgfpathlineto{\pgfqpoint{4.913779in}{1.355544in}}%
\pgfpathlineto{\pgfqpoint{4.921848in}{1.369385in}}%
\pgfpathlineto{\pgfqpoint{4.929916in}{1.383382in}}%
\pgfpathclose%
\pgfusepath{fill}%
\end{pgfscope}%
\begin{pgfscope}%
\pgfpathrectangle{\pgfqpoint{1.150000in}{0.150000in}}{\pgfqpoint{5.700000in}{5.700000in}}%
\pgfusepath{clip}%
\pgfsetbuttcap%
\pgfsetroundjoin%
\definecolor{currentfill}{rgb}{0.282884,0.135920,0.453427}%
\pgfsetfillcolor{currentfill}%
\pgfsetfillopacity{0.700000}%
\pgfsetlinewidth{0.000000pt}%
\definecolor{currentstroke}{rgb}{0.000000,0.000000,0.000000}%
\pgfsetstrokecolor{currentstroke}%
\pgfsetdash{}{0pt}%
\pgfpathmoveto{\pgfqpoint{4.838968in}{1.325086in}}%
\pgfpathlineto{\pgfqpoint{4.853618in}{1.325764in}}%
\pgfpathlineto{\pgfqpoint{4.868278in}{1.326535in}}%
\pgfpathlineto{\pgfqpoint{4.882951in}{1.327399in}}%
\pgfpathlineto{\pgfqpoint{4.897634in}{1.328356in}}%
\pgfpathlineto{\pgfqpoint{4.889559in}{1.315019in}}%
\pgfpathlineto{\pgfqpoint{4.881482in}{1.301861in}}%
\pgfpathlineto{\pgfqpoint{4.873403in}{1.288887in}}%
\pgfpathlineto{\pgfqpoint{4.865322in}{1.276104in}}%
\pgfpathlineto{\pgfqpoint{4.850639in}{1.275720in}}%
\pgfpathlineto{\pgfqpoint{4.835968in}{1.275429in}}%
\pgfpathlineto{\pgfqpoint{4.821308in}{1.275232in}}%
\pgfpathlineto{\pgfqpoint{4.806659in}{1.275127in}}%
\pgfpathlineto{\pgfqpoint{4.814740in}{1.287331in}}%
\pgfpathlineto{\pgfqpoint{4.822818in}{1.299729in}}%
\pgfpathlineto{\pgfqpoint{4.830894in}{1.312316in}}%
\pgfpathlineto{\pgfqpoint{4.838968in}{1.325086in}}%
\pgfpathclose%
\pgfusepath{fill}%
\end{pgfscope}%
\begin{pgfscope}%
\pgfpathrectangle{\pgfqpoint{1.150000in}{0.150000in}}{\pgfqpoint{5.700000in}{5.700000in}}%
\pgfusepath{clip}%
\pgfsetbuttcap%
\pgfsetroundjoin%
\definecolor{currentfill}{rgb}{0.283197,0.115680,0.436115}%
\pgfsetfillcolor{currentfill}%
\pgfsetfillopacity{0.700000}%
\pgfsetlinewidth{0.000000pt}%
\definecolor{currentstroke}{rgb}{0.000000,0.000000,0.000000}%
\pgfsetstrokecolor{currentstroke}%
\pgfsetdash{}{0pt}%
\pgfpathmoveto{\pgfqpoint{4.748169in}{1.275637in}}%
\pgfpathlineto{\pgfqpoint{4.762776in}{1.275370in}}%
\pgfpathlineto{\pgfqpoint{4.777393in}{1.275196in}}%
\pgfpathlineto{\pgfqpoint{4.792020in}{1.275115in}}%
\pgfpathlineto{\pgfqpoint{4.806659in}{1.275127in}}%
\pgfpathlineto{\pgfqpoint{4.798576in}{1.263123in}}%
\pgfpathlineto{\pgfqpoint{4.790491in}{1.251325in}}%
\pgfpathlineto{\pgfqpoint{4.782403in}{1.239738in}}%
\pgfpathlineto{\pgfqpoint{4.774313in}{1.228369in}}%
\pgfpathlineto{\pgfqpoint{4.759673in}{1.228947in}}%
\pgfpathlineto{\pgfqpoint{4.745044in}{1.229618in}}%
\pgfpathlineto{\pgfqpoint{4.730424in}{1.230382in}}%
\pgfpathlineto{\pgfqpoint{4.715815in}{1.231238in}}%
\pgfpathlineto{\pgfqpoint{4.723908in}{1.242011in}}%
\pgfpathlineto{\pgfqpoint{4.731998in}{1.253006in}}%
\pgfpathlineto{\pgfqpoint{4.740085in}{1.264216in}}%
\pgfpathlineto{\pgfqpoint{4.748169in}{1.275637in}}%
\pgfpathclose%
\pgfusepath{fill}%
\end{pgfscope}%
\begin{pgfscope}%
\pgfpathrectangle{\pgfqpoint{1.150000in}{0.150000in}}{\pgfqpoint{5.700000in}{5.700000in}}%
\pgfusepath{clip}%
\pgfsetbuttcap%
\pgfsetroundjoin%
\definecolor{currentfill}{rgb}{0.886271,0.892374,0.095374}%
\pgfsetfillcolor{currentfill}%
\pgfsetfillopacity{0.700000}%
\pgfsetlinewidth{0.000000pt}%
\definecolor{currentstroke}{rgb}{0.000000,0.000000,0.000000}%
\pgfsetstrokecolor{currentstroke}%
\pgfsetdash{}{0pt}%
\pgfpathmoveto{\pgfqpoint{1.900926in}{3.648359in}}%
\pgfpathlineto{\pgfqpoint{1.915738in}{3.621223in}}%
\pgfpathlineto{\pgfqpoint{1.930537in}{3.594301in}}%
\pgfpathlineto{\pgfqpoint{1.945322in}{3.567592in}}%
\pgfpathlineto{\pgfqpoint{1.960094in}{3.541093in}}%
\pgfpathlineto{\pgfqpoint{1.949532in}{3.569522in}}%
\pgfpathlineto{\pgfqpoint{1.938925in}{3.598615in}}%
\pgfpathlineto{\pgfqpoint{1.928270in}{3.628380in}}%
\pgfpathlineto{\pgfqpoint{1.913423in}{3.655519in}}%
\pgfpathlineto{\pgfqpoint{1.898562in}{3.682869in}}%
\pgfpathlineto{\pgfqpoint{1.883688in}{3.710433in}}%
\pgfpathlineto{\pgfqpoint{1.868798in}{3.738214in}}%
\pgfpathlineto{\pgfqpoint{1.879556in}{3.707583in}}%
\pgfpathlineto{\pgfqpoint{1.890265in}{3.677635in}}%
\pgfpathlineto{\pgfqpoint{1.900926in}{3.648359in}}%
\pgfpathclose%
\pgfusepath{fill}%
\end{pgfscope}%
\begin{pgfscope}%
\pgfpathrectangle{\pgfqpoint{1.150000in}{0.150000in}}{\pgfqpoint{5.700000in}{5.700000in}}%
\pgfusepath{clip}%
\pgfsetbuttcap%
\pgfsetroundjoin%
\definecolor{currentfill}{rgb}{0.281924,0.089666,0.412415}%
\pgfsetfillcolor{currentfill}%
\pgfsetfillopacity{0.700000}%
\pgfsetlinewidth{0.000000pt}%
\definecolor{currentstroke}{rgb}{0.000000,0.000000,0.000000}%
\pgfsetstrokecolor{currentstroke}%
\pgfsetdash{}{0pt}%
\pgfpathmoveto{\pgfqpoint{4.269910in}{1.214802in}}%
\pgfpathlineto{\pgfqpoint{4.284346in}{1.209844in}}%
\pgfpathlineto{\pgfqpoint{4.298790in}{1.204979in}}%
\pgfpathlineto{\pgfqpoint{4.313240in}{1.200208in}}%
\pgfpathlineto{\pgfqpoint{4.327698in}{1.195530in}}%
\pgfpathlineto{\pgfqpoint{4.319499in}{1.191472in}}%
\pgfpathlineto{\pgfqpoint{4.311294in}{1.187747in}}%
\pgfpathlineto{\pgfqpoint{4.303082in}{1.184364in}}%
\pgfpathlineto{\pgfqpoint{4.294864in}{1.181328in}}%
\pgfpathlineto{\pgfqpoint{4.280388in}{1.186667in}}%
\pgfpathlineto{\pgfqpoint{4.265919in}{1.192100in}}%
\pgfpathlineto{\pgfqpoint{4.251457in}{1.197626in}}%
\pgfpathlineto{\pgfqpoint{4.237001in}{1.203246in}}%
\pgfpathlineto{\pgfqpoint{4.245239in}{1.205613in}}%
\pgfpathlineto{\pgfqpoint{4.253470in}{1.208333in}}%
\pgfpathlineto{\pgfqpoint{4.261693in}{1.211398in}}%
\pgfpathlineto{\pgfqpoint{4.269910in}{1.214802in}}%
\pgfpathclose%
\pgfusepath{fill}%
\end{pgfscope}%
\begin{pgfscope}%
\pgfpathrectangle{\pgfqpoint{1.150000in}{0.150000in}}{\pgfqpoint{5.700000in}{5.700000in}}%
\pgfusepath{clip}%
\pgfsetbuttcap%
\pgfsetroundjoin%
\definecolor{currentfill}{rgb}{0.281446,0.084320,0.407414}%
\pgfsetfillcolor{currentfill}%
\pgfsetfillopacity{0.700000}%
\pgfsetlinewidth{0.000000pt}%
\definecolor{currentstroke}{rgb}{0.000000,0.000000,0.000000}%
\pgfsetstrokecolor{currentstroke}%
\pgfsetdash{}{0pt}%
\pgfpathmoveto{\pgfqpoint{4.418275in}{1.199756in}}%
\pgfpathlineto{\pgfqpoint{4.432755in}{1.196186in}}%
\pgfpathlineto{\pgfqpoint{4.447243in}{1.192709in}}%
\pgfpathlineto{\pgfqpoint{4.461740in}{1.189325in}}%
\pgfpathlineto{\pgfqpoint{4.476245in}{1.186034in}}%
\pgfpathlineto{\pgfqpoint{4.468097in}{1.179438in}}%
\pgfpathlineto{\pgfqpoint{4.459944in}{1.173140in}}%
\pgfpathlineto{\pgfqpoint{4.451786in}{1.167145in}}%
\pgfpathlineto{\pgfqpoint{4.443624in}{1.161461in}}%
\pgfpathlineto{\pgfqpoint{4.429106in}{1.165394in}}%
\pgfpathlineto{\pgfqpoint{4.414597in}{1.169421in}}%
\pgfpathlineto{\pgfqpoint{4.400095in}{1.173540in}}%
\pgfpathlineto{\pgfqpoint{4.385600in}{1.177752in}}%
\pgfpathlineto{\pgfqpoint{4.393777in}{1.182786in}}%
\pgfpathlineto{\pgfqpoint{4.401948in}{1.188136in}}%
\pgfpathlineto{\pgfqpoint{4.410114in}{1.193795in}}%
\pgfpathlineto{\pgfqpoint{4.418275in}{1.199756in}}%
\pgfpathclose%
\pgfusepath{fill}%
\end{pgfscope}%
\begin{pgfscope}%
\pgfpathrectangle{\pgfqpoint{1.150000in}{0.150000in}}{\pgfqpoint{5.700000in}{5.700000in}}%
\pgfusepath{clip}%
\pgfsetbuttcap%
\pgfsetroundjoin%
\definecolor{currentfill}{rgb}{0.282656,0.100196,0.422160}%
\pgfsetfillcolor{currentfill}%
\pgfsetfillopacity{0.700000}%
\pgfsetlinewidth{0.000000pt}%
\definecolor{currentstroke}{rgb}{0.000000,0.000000,0.000000}%
\pgfsetstrokecolor{currentstroke}%
\pgfsetdash{}{0pt}%
\pgfpathmoveto{\pgfqpoint{4.657479in}{1.235590in}}%
\pgfpathlineto{\pgfqpoint{4.672048in}{1.234363in}}%
\pgfpathlineto{\pgfqpoint{4.686627in}{1.233229in}}%
\pgfpathlineto{\pgfqpoint{4.701216in}{1.232187in}}%
\pgfpathlineto{\pgfqpoint{4.715815in}{1.231238in}}%
\pgfpathlineto{\pgfqpoint{4.707720in}{1.220693in}}%
\pgfpathlineto{\pgfqpoint{4.699622in}{1.210382in}}%
\pgfpathlineto{\pgfqpoint{4.691521in}{1.200310in}}%
\pgfpathlineto{\pgfqpoint{4.683417in}{1.190484in}}%
\pgfpathlineto{\pgfqpoint{4.668814in}{1.192040in}}%
\pgfpathlineto{\pgfqpoint{4.654220in}{1.193688in}}%
\pgfpathlineto{\pgfqpoint{4.639635in}{1.195429in}}%
\pgfpathlineto{\pgfqpoint{4.625060in}{1.197263in}}%
\pgfpathlineto{\pgfqpoint{4.633170in}{1.206475in}}%
\pgfpathlineto{\pgfqpoint{4.641276in}{1.215938in}}%
\pgfpathlineto{\pgfqpoint{4.649379in}{1.225645in}}%
\pgfpathlineto{\pgfqpoint{4.657479in}{1.235590in}}%
\pgfpathclose%
\pgfusepath{fill}%
\end{pgfscope}%
\begin{pgfscope}%
\pgfpathrectangle{\pgfqpoint{1.150000in}{0.150000in}}{\pgfqpoint{5.700000in}{5.700000in}}%
\pgfusepath{clip}%
\pgfsetbuttcap%
\pgfsetroundjoin%
\definecolor{currentfill}{rgb}{0.281412,0.155834,0.469201}%
\pgfsetfillcolor{currentfill}%
\pgfsetfillopacity{0.700000}%
\pgfsetlinewidth{0.000000pt}%
\definecolor{currentstroke}{rgb}{0.000000,0.000000,0.000000}%
\pgfsetstrokecolor{currentstroke}%
\pgfsetdash{}{0pt}%
\pgfpathmoveto{\pgfqpoint{3.915577in}{1.343391in}}%
\pgfpathlineto{\pgfqpoint{3.929950in}{1.335184in}}%
\pgfpathlineto{\pgfqpoint{3.944327in}{1.327073in}}%
\pgfpathlineto{\pgfqpoint{3.958709in}{1.319059in}}%
\pgfpathlineto{\pgfqpoint{3.973095in}{1.311141in}}%
\pgfpathlineto{\pgfqpoint{3.964710in}{1.313430in}}%
\pgfpathlineto{\pgfqpoint{3.956314in}{1.316137in}}%
\pgfpathlineto{\pgfqpoint{3.947906in}{1.319269in}}%
\pgfpathlineto{\pgfqpoint{3.933496in}{1.327714in}}%
\pgfpathlineto{\pgfqpoint{3.919090in}{1.336256in}}%
\pgfpathlineto{\pgfqpoint{3.904689in}{1.344894in}}%
\pgfpathlineto{\pgfqpoint{3.890291in}{1.353630in}}%
\pgfpathlineto{\pgfqpoint{3.898732in}{1.349789in}}%
\pgfpathlineto{\pgfqpoint{3.907160in}{1.346378in}}%
\pgfpathlineto{\pgfqpoint{3.915577in}{1.343391in}}%
\pgfpathclose%
\pgfusepath{fill}%
\end{pgfscope}%
\begin{pgfscope}%
\pgfpathrectangle{\pgfqpoint{1.150000in}{0.150000in}}{\pgfqpoint{5.700000in}{5.700000in}}%
\pgfusepath{clip}%
\pgfsetbuttcap%
\pgfsetroundjoin%
\definecolor{currentfill}{rgb}{0.283197,0.115680,0.436115}%
\pgfsetfillcolor{currentfill}%
\pgfsetfillopacity{0.700000}%
\pgfsetlinewidth{0.000000pt}%
\definecolor{currentstroke}{rgb}{0.000000,0.000000,0.000000}%
\pgfsetstrokecolor{currentstroke}%
\pgfsetdash{}{0pt}%
\pgfpathmoveto{\pgfqpoint{4.121580in}{1.251593in}}%
\pgfpathlineto{\pgfqpoint{4.135987in}{1.245219in}}%
\pgfpathlineto{\pgfqpoint{4.150399in}{1.238940in}}%
\pgfpathlineto{\pgfqpoint{4.164817in}{1.232756in}}%
\pgfpathlineto{\pgfqpoint{4.179242in}{1.226665in}}%
\pgfpathlineto{\pgfqpoint{4.170974in}{1.225333in}}%
\pgfpathlineto{\pgfqpoint{4.162699in}{1.224372in}}%
\pgfpathlineto{\pgfqpoint{4.154415in}{1.223791in}}%
\pgfpathlineto{\pgfqpoint{4.146122in}{1.223596in}}%
\pgfpathlineto{\pgfqpoint{4.131674in}{1.230369in}}%
\pgfpathlineto{\pgfqpoint{4.117231in}{1.237236in}}%
\pgfpathlineto{\pgfqpoint{4.102794in}{1.244198in}}%
\pgfpathlineto{\pgfqpoint{4.088362in}{1.251255in}}%
\pgfpathlineto{\pgfqpoint{4.096680in}{1.250759in}}%
\pgfpathlineto{\pgfqpoint{4.104989in}{1.250655in}}%
\pgfpathlineto{\pgfqpoint{4.113289in}{1.250935in}}%
\pgfpathlineto{\pgfqpoint{4.121580in}{1.251593in}}%
\pgfpathclose%
\pgfusepath{fill}%
\end{pgfscope}%
\begin{pgfscope}%
\pgfpathrectangle{\pgfqpoint{1.150000in}{0.150000in}}{\pgfqpoint{5.700000in}{5.700000in}}%
\pgfusepath{clip}%
\pgfsetbuttcap%
\pgfsetroundjoin%
\definecolor{currentfill}{rgb}{0.281924,0.089666,0.412415}%
\pgfsetfillcolor{currentfill}%
\pgfsetfillopacity{0.700000}%
\pgfsetlinewidth{0.000000pt}%
\definecolor{currentstroke}{rgb}{0.000000,0.000000,0.000000}%
\pgfsetstrokecolor{currentstroke}%
\pgfsetdash{}{0pt}%
\pgfpathmoveto{\pgfqpoint{4.566853in}{1.205522in}}%
\pgfpathlineto{\pgfqpoint{4.581391in}{1.203319in}}%
\pgfpathlineto{\pgfqpoint{4.595938in}{1.201207in}}%
\pgfpathlineto{\pgfqpoint{4.610495in}{1.199189in}}%
\pgfpathlineto{\pgfqpoint{4.625060in}{1.197263in}}%
\pgfpathlineto{\pgfqpoint{4.616947in}{1.188307in}}%
\pgfpathlineto{\pgfqpoint{4.608831in}{1.179613in}}%
\pgfpathlineto{\pgfqpoint{4.600711in}{1.171189in}}%
\pgfpathlineto{\pgfqpoint{4.592587in}{1.163039in}}%
\pgfpathlineto{\pgfqpoint{4.578014in}{1.165590in}}%
\pgfpathlineto{\pgfqpoint{4.563449in}{1.168233in}}%
\pgfpathlineto{\pgfqpoint{4.548894in}{1.170969in}}%
\pgfpathlineto{\pgfqpoint{4.534347in}{1.173797in}}%
\pgfpathlineto{\pgfqpoint{4.542480in}{1.181315in}}%
\pgfpathlineto{\pgfqpoint{4.550608in}{1.189112in}}%
\pgfpathlineto{\pgfqpoint{4.558733in}{1.197184in}}%
\pgfpathlineto{\pgfqpoint{4.566853in}{1.205522in}}%
\pgfpathclose%
\pgfusepath{fill}%
\end{pgfscope}%
\begin{pgfscope}%
\pgfpathrectangle{\pgfqpoint{1.150000in}{0.150000in}}{\pgfqpoint{5.700000in}{5.700000in}}%
\pgfusepath{clip}%
\pgfsetbuttcap%
\pgfsetroundjoin%
\definecolor{currentfill}{rgb}{0.993248,0.906157,0.143936}%
\pgfsetfillcolor{currentfill}%
\pgfsetfillopacity{0.700000}%
\pgfsetlinewidth{0.000000pt}%
\definecolor{currentstroke}{rgb}{0.000000,0.000000,0.000000}%
\pgfsetstrokecolor{currentstroke}%
\pgfsetdash{}{0pt}%
\pgfpathmoveto{\pgfqpoint{1.841532in}{3.759088in}}%
\pgfpathlineto{\pgfqpoint{1.856402in}{3.731073in}}%
\pgfpathlineto{\pgfqpoint{1.871258in}{3.703282in}}%
\pgfpathlineto{\pgfqpoint{1.886099in}{3.675711in}}%
\pgfpathlineto{\pgfqpoint{1.900926in}{3.648359in}}%
\pgfpathlineto{\pgfqpoint{1.890265in}{3.677635in}}%
\pgfpathlineto{\pgfqpoint{1.879556in}{3.707583in}}%
\pgfpathlineto{\pgfqpoint{1.868798in}{3.738214in}}%
\pgfpathlineto{\pgfqpoint{1.853895in}{3.766213in}}%
\pgfpathlineto{\pgfqpoint{1.838976in}{3.794432in}}%
\pgfpathlineto{\pgfqpoint{1.824042in}{3.822874in}}%
\pgfpathlineto{\pgfqpoint{1.809093in}{3.851541in}}%
\pgfpathlineto{\pgfqpoint{1.819956in}{3.820035in}}%
\pgfpathlineto{\pgfqpoint{1.830768in}{3.789221in}}%
\pgfpathlineto{\pgfqpoint{1.841532in}{3.759088in}}%
\pgfpathclose%
\pgfusepath{fill}%
\end{pgfscope}%
\begin{pgfscope}%
\pgfpathrectangle{\pgfqpoint{1.150000in}{0.150000in}}{\pgfqpoint{5.700000in}{5.700000in}}%
\pgfusepath{clip}%
\pgfsetbuttcap%
\pgfsetroundjoin%
\definecolor{currentfill}{rgb}{0.281924,0.089666,0.412415}%
\pgfsetfillcolor{currentfill}%
\pgfsetfillopacity{0.700000}%
\pgfsetlinewidth{0.000000pt}%
\definecolor{currentstroke}{rgb}{0.000000,0.000000,0.000000}%
\pgfsetstrokecolor{currentstroke}%
\pgfsetdash{}{0pt}%
\pgfpathmoveto{\pgfqpoint{4.327698in}{1.195530in}}%
\pgfpathlineto{\pgfqpoint{4.342162in}{1.190946in}}%
\pgfpathlineto{\pgfqpoint{4.356634in}{1.186455in}}%
\pgfpathlineto{\pgfqpoint{4.371114in}{1.182057in}}%
\pgfpathlineto{\pgfqpoint{4.385600in}{1.177752in}}%
\pgfpathlineto{\pgfqpoint{4.377418in}{1.173039in}}%
\pgfpathlineto{\pgfqpoint{4.369230in}{1.168656in}}%
\pgfpathlineto{\pgfqpoint{4.361035in}{1.164608in}}%
\pgfpathlineto{\pgfqpoint{4.352835in}{1.160903in}}%
\pgfpathlineto{\pgfqpoint{4.338332in}{1.165870in}}%
\pgfpathlineto{\pgfqpoint{4.323836in}{1.170929in}}%
\pgfpathlineto{\pgfqpoint{4.309346in}{1.176082in}}%
\pgfpathlineto{\pgfqpoint{4.294864in}{1.181328in}}%
\pgfpathlineto{\pgfqpoint{4.303082in}{1.184364in}}%
\pgfpathlineto{\pgfqpoint{4.311294in}{1.187747in}}%
\pgfpathlineto{\pgfqpoint{4.319499in}{1.191472in}}%
\pgfpathlineto{\pgfqpoint{4.327698in}{1.195530in}}%
\pgfpathclose%
\pgfusepath{fill}%
\end{pgfscope}%
\begin{pgfscope}%
\pgfpathrectangle{\pgfqpoint{1.150000in}{0.150000in}}{\pgfqpoint{5.700000in}{5.700000in}}%
\pgfusepath{clip}%
\pgfsetbuttcap%
\pgfsetroundjoin%
\definecolor{currentfill}{rgb}{0.282290,0.145912,0.461510}%
\pgfsetfillcolor{currentfill}%
\pgfsetfillopacity{0.700000}%
\pgfsetlinewidth{0.000000pt}%
\definecolor{currentstroke}{rgb}{0.000000,0.000000,0.000000}%
\pgfsetstrokecolor{currentstroke}%
\pgfsetdash{}{0pt}%
\pgfpathmoveto{\pgfqpoint{3.973095in}{1.311141in}}%
\pgfpathlineto{\pgfqpoint{3.987486in}{1.303320in}}%
\pgfpathlineto{\pgfqpoint{4.001882in}{1.295595in}}%
\pgfpathlineto{\pgfqpoint{4.016283in}{1.287966in}}%
\pgfpathlineto{\pgfqpoint{4.030688in}{1.280433in}}%
\pgfpathlineto{\pgfqpoint{4.022333in}{1.282024in}}%
\pgfpathlineto{\pgfqpoint{4.013967in}{1.284028in}}%
\pgfpathlineto{\pgfqpoint{4.005590in}{1.286452in}}%
\pgfpathlineto{\pgfqpoint{3.991162in}{1.294512in}}%
\pgfpathlineto{\pgfqpoint{3.976739in}{1.302668in}}%
\pgfpathlineto{\pgfqpoint{3.962320in}{1.310920in}}%
\pgfpathlineto{\pgfqpoint{3.947906in}{1.319269in}}%
\pgfpathlineto{\pgfqpoint{3.956314in}{1.316137in}}%
\pgfpathlineto{\pgfqpoint{3.964710in}{1.313430in}}%
\pgfpathlineto{\pgfqpoint{3.973095in}{1.311141in}}%
\pgfpathclose%
\pgfusepath{fill}%
\end{pgfscope}%
\begin{pgfscope}%
\pgfpathrectangle{\pgfqpoint{1.150000in}{0.150000in}}{\pgfqpoint{5.700000in}{5.700000in}}%
\pgfusepath{clip}%
\pgfsetbuttcap%
\pgfsetroundjoin%
\definecolor{currentfill}{rgb}{0.281887,0.150881,0.465405}%
\pgfsetfillcolor{currentfill}%
\pgfsetfillopacity{0.700000}%
\pgfsetlinewidth{0.000000pt}%
\definecolor{currentstroke}{rgb}{0.000000,0.000000,0.000000}%
\pgfsetstrokecolor{currentstroke}%
\pgfsetdash{}{0pt}%
\pgfpathmoveto{\pgfqpoint{4.897634in}{1.328356in}}%
\pgfpathlineto{\pgfqpoint{4.912329in}{1.329407in}}%
\pgfpathlineto{\pgfqpoint{4.927036in}{1.330551in}}%
\pgfpathlineto{\pgfqpoint{4.941754in}{1.331788in}}%
\pgfpathlineto{\pgfqpoint{4.933677in}{1.318025in}}%
\pgfpathlineto{\pgfqpoint{4.925598in}{1.304438in}}%
\pgfpathlineto{\pgfqpoint{4.917518in}{1.291031in}}%
\pgfpathlineto{\pgfqpoint{4.909435in}{1.277811in}}%
\pgfpathlineto{\pgfqpoint{4.894720in}{1.277149in}}%
\pgfpathlineto{\pgfqpoint{4.880015in}{1.276580in}}%
\pgfpathlineto{\pgfqpoint{4.865322in}{1.276104in}}%
\pgfpathlineto{\pgfqpoint{4.873403in}{1.288887in}}%
\pgfpathlineto{\pgfqpoint{4.881482in}{1.301861in}}%
\pgfpathlineto{\pgfqpoint{4.889559in}{1.315019in}}%
\pgfpathlineto{\pgfqpoint{4.897634in}{1.328356in}}%
\pgfpathclose%
\pgfusepath{fill}%
\end{pgfscope}%
\begin{pgfscope}%
\pgfpathrectangle{\pgfqpoint{1.150000in}{0.150000in}}{\pgfqpoint{5.700000in}{5.700000in}}%
\pgfusepath{clip}%
\pgfsetbuttcap%
\pgfsetroundjoin%
\definecolor{currentfill}{rgb}{0.281446,0.084320,0.407414}%
\pgfsetfillcolor{currentfill}%
\pgfsetfillopacity{0.700000}%
\pgfsetlinewidth{0.000000pt}%
\definecolor{currentstroke}{rgb}{0.000000,0.000000,0.000000}%
\pgfsetstrokecolor{currentstroke}%
\pgfsetdash{}{0pt}%
\pgfpathmoveto{\pgfqpoint{4.476245in}{1.186034in}}%
\pgfpathlineto{\pgfqpoint{4.490758in}{1.182836in}}%
\pgfpathlineto{\pgfqpoint{4.505279in}{1.179730in}}%
\pgfpathlineto{\pgfqpoint{4.519809in}{1.176717in}}%
\pgfpathlineto{\pgfqpoint{4.534347in}{1.173797in}}%
\pgfpathlineto{\pgfqpoint{4.526210in}{1.166565in}}%
\pgfpathlineto{\pgfqpoint{4.518069in}{1.159626in}}%
\pgfpathlineto{\pgfqpoint{4.509923in}{1.152986in}}%
\pgfpathlineto{\pgfqpoint{4.501773in}{1.146652in}}%
\pgfpathlineto{\pgfqpoint{4.487224in}{1.150216in}}%
\pgfpathlineto{\pgfqpoint{4.472683in}{1.153872in}}%
\pgfpathlineto{\pgfqpoint{4.458149in}{1.157620in}}%
\pgfpathlineto{\pgfqpoint{4.443624in}{1.161461in}}%
\pgfpathlineto{\pgfqpoint{4.451786in}{1.167145in}}%
\pgfpathlineto{\pgfqpoint{4.459944in}{1.173140in}}%
\pgfpathlineto{\pgfqpoint{4.468097in}{1.179438in}}%
\pgfpathlineto{\pgfqpoint{4.476245in}{1.186034in}}%
\pgfpathclose%
\pgfusepath{fill}%
\end{pgfscope}%
\begin{pgfscope}%
\pgfpathrectangle{\pgfqpoint{1.150000in}{0.150000in}}{\pgfqpoint{5.700000in}{5.700000in}}%
\pgfusepath{clip}%
\pgfsetbuttcap%
\pgfsetroundjoin%
\definecolor{currentfill}{rgb}{0.283187,0.125848,0.444960}%
\pgfsetfillcolor{currentfill}%
\pgfsetfillopacity{0.700000}%
\pgfsetlinewidth{0.000000pt}%
\definecolor{currentstroke}{rgb}{0.000000,0.000000,0.000000}%
\pgfsetstrokecolor{currentstroke}%
\pgfsetdash{}{0pt}%
\pgfpathmoveto{\pgfqpoint{4.806659in}{1.275127in}}%
\pgfpathlineto{\pgfqpoint{4.821308in}{1.275232in}}%
\pgfpathlineto{\pgfqpoint{4.835968in}{1.275429in}}%
\pgfpathlineto{\pgfqpoint{4.850639in}{1.275720in}}%
\pgfpathlineto{\pgfqpoint{4.865322in}{1.276104in}}%
\pgfpathlineto{\pgfqpoint{4.857239in}{1.263515in}}%
\pgfpathlineto{\pgfqpoint{4.849154in}{1.251128in}}%
\pgfpathlineto{\pgfqpoint{4.841067in}{1.238948in}}%
\pgfpathlineto{\pgfqpoint{4.832978in}{1.226981in}}%
\pgfpathlineto{\pgfqpoint{4.818296in}{1.227189in}}%
\pgfpathlineto{\pgfqpoint{4.803624in}{1.227490in}}%
\pgfpathlineto{\pgfqpoint{4.788963in}{1.227883in}}%
\pgfpathlineto{\pgfqpoint{4.774313in}{1.228369in}}%
\pgfpathlineto{\pgfqpoint{4.782403in}{1.239738in}}%
\pgfpathlineto{\pgfqpoint{4.790491in}{1.251325in}}%
\pgfpathlineto{\pgfqpoint{4.798576in}{1.263123in}}%
\pgfpathlineto{\pgfqpoint{4.806659in}{1.275127in}}%
\pgfpathclose%
\pgfusepath{fill}%
\end{pgfscope}%
\begin{pgfscope}%
\pgfpathrectangle{\pgfqpoint{1.150000in}{0.150000in}}{\pgfqpoint{5.700000in}{5.700000in}}%
\pgfusepath{clip}%
\pgfsetbuttcap%
\pgfsetroundjoin%
\definecolor{currentfill}{rgb}{0.283091,0.110553,0.431554}%
\pgfsetfillcolor{currentfill}%
\pgfsetfillopacity{0.700000}%
\pgfsetlinewidth{0.000000pt}%
\definecolor{currentstroke}{rgb}{0.000000,0.000000,0.000000}%
\pgfsetstrokecolor{currentstroke}%
\pgfsetdash{}{0pt}%
\pgfpathmoveto{\pgfqpoint{4.179242in}{1.226665in}}%
\pgfpathlineto{\pgfqpoint{4.193672in}{1.220669in}}%
\pgfpathlineto{\pgfqpoint{4.208109in}{1.214768in}}%
\pgfpathlineto{\pgfqpoint{4.222552in}{1.208960in}}%
\pgfpathlineto{\pgfqpoint{4.237001in}{1.203246in}}%
\pgfpathlineto{\pgfqpoint{4.228755in}{1.201239in}}%
\pgfpathlineto{\pgfqpoint{4.220502in}{1.199599in}}%
\pgfpathlineto{\pgfqpoint{4.212242in}{1.198333in}}%
\pgfpathlineto{\pgfqpoint{4.203973in}{1.197449in}}%
\pgfpathlineto{\pgfqpoint{4.189501in}{1.203844in}}%
\pgfpathlineto{\pgfqpoint{4.175036in}{1.210334in}}%
\pgfpathlineto{\pgfqpoint{4.160576in}{1.216918in}}%
\pgfpathlineto{\pgfqpoint{4.146122in}{1.223596in}}%
\pgfpathlineto{\pgfqpoint{4.154415in}{1.223791in}}%
\pgfpathlineto{\pgfqpoint{4.162699in}{1.224372in}}%
\pgfpathlineto{\pgfqpoint{4.170974in}{1.225333in}}%
\pgfpathlineto{\pgfqpoint{4.179242in}{1.226665in}}%
\pgfpathclose%
\pgfusepath{fill}%
\end{pgfscope}%
\begin{pgfscope}%
\pgfpathrectangle{\pgfqpoint{1.150000in}{0.150000in}}{\pgfqpoint{5.700000in}{5.700000in}}%
\pgfusepath{clip}%
\pgfsetbuttcap%
\pgfsetroundjoin%
\definecolor{currentfill}{rgb}{0.282910,0.105393,0.426902}%
\pgfsetfillcolor{currentfill}%
\pgfsetfillopacity{0.700000}%
\pgfsetlinewidth{0.000000pt}%
\definecolor{currentstroke}{rgb}{0.000000,0.000000,0.000000}%
\pgfsetstrokecolor{currentstroke}%
\pgfsetdash{}{0pt}%
\pgfpathmoveto{\pgfqpoint{4.715815in}{1.231238in}}%
\pgfpathlineto{\pgfqpoint{4.730424in}{1.230382in}}%
\pgfpathlineto{\pgfqpoint{4.745044in}{1.229618in}}%
\pgfpathlineto{\pgfqpoint{4.759673in}{1.228947in}}%
\pgfpathlineto{\pgfqpoint{4.774313in}{1.228369in}}%
\pgfpathlineto{\pgfqpoint{4.766221in}{1.217223in}}%
\pgfpathlineto{\pgfqpoint{4.758126in}{1.206305in}}%
\pgfpathlineto{\pgfqpoint{4.750029in}{1.195623in}}%
\pgfpathlineto{\pgfqpoint{4.741929in}{1.185182in}}%
\pgfpathlineto{\pgfqpoint{4.727286in}{1.186369in}}%
\pgfpathlineto{\pgfqpoint{4.712653in}{1.187648in}}%
\pgfpathlineto{\pgfqpoint{4.698030in}{1.189020in}}%
\pgfpathlineto{\pgfqpoint{4.683417in}{1.190484in}}%
\pgfpathlineto{\pgfqpoint{4.691521in}{1.200310in}}%
\pgfpathlineto{\pgfqpoint{4.699622in}{1.210382in}}%
\pgfpathlineto{\pgfqpoint{4.707720in}{1.220693in}}%
\pgfpathlineto{\pgfqpoint{4.715815in}{1.231238in}}%
\pgfpathclose%
\pgfusepath{fill}%
\end{pgfscope}%
\begin{pgfscope}%
\pgfpathrectangle{\pgfqpoint{1.150000in}{0.150000in}}{\pgfqpoint{5.700000in}{5.700000in}}%
\pgfusepath{clip}%
\pgfsetbuttcap%
\pgfsetroundjoin%
\definecolor{currentfill}{rgb}{0.282327,0.094955,0.417331}%
\pgfsetfillcolor{currentfill}%
\pgfsetfillopacity{0.700000}%
\pgfsetlinewidth{0.000000pt}%
\definecolor{currentstroke}{rgb}{0.000000,0.000000,0.000000}%
\pgfsetstrokecolor{currentstroke}%
\pgfsetdash{}{0pt}%
\pgfpathmoveto{\pgfqpoint{4.625060in}{1.197263in}}%
\pgfpathlineto{\pgfqpoint{4.639635in}{1.195429in}}%
\pgfpathlineto{\pgfqpoint{4.654220in}{1.193688in}}%
\pgfpathlineto{\pgfqpoint{4.668814in}{1.192040in}}%
\pgfpathlineto{\pgfqpoint{4.683417in}{1.190484in}}%
\pgfpathlineto{\pgfqpoint{4.675310in}{1.180909in}}%
\pgfpathlineto{\pgfqpoint{4.667200in}{1.171593in}}%
\pgfpathlineto{\pgfqpoint{4.659087in}{1.162541in}}%
\pgfpathlineto{\pgfqpoint{4.650971in}{1.153759in}}%
\pgfpathlineto{\pgfqpoint{4.636361in}{1.155941in}}%
\pgfpathlineto{\pgfqpoint{4.621761in}{1.158215in}}%
\pgfpathlineto{\pgfqpoint{4.607169in}{1.160581in}}%
\pgfpathlineto{\pgfqpoint{4.592587in}{1.163039in}}%
\pgfpathlineto{\pgfqpoint{4.600711in}{1.171189in}}%
\pgfpathlineto{\pgfqpoint{4.608831in}{1.179613in}}%
\pgfpathlineto{\pgfqpoint{4.616947in}{1.188307in}}%
\pgfpathlineto{\pgfqpoint{4.625060in}{1.197263in}}%
\pgfpathclose%
\pgfusepath{fill}%
\end{pgfscope}%
\begin{pgfscope}%
\pgfpathrectangle{\pgfqpoint{1.150000in}{0.150000in}}{\pgfqpoint{5.700000in}{5.700000in}}%
\pgfusepath{clip}%
\pgfsetbuttcap%
\pgfsetroundjoin%
\definecolor{currentfill}{rgb}{0.282623,0.140926,0.457517}%
\pgfsetfillcolor{currentfill}%
\pgfsetfillopacity{0.700000}%
\pgfsetlinewidth{0.000000pt}%
\definecolor{currentstroke}{rgb}{0.000000,0.000000,0.000000}%
\pgfsetstrokecolor{currentstroke}%
\pgfsetdash{}{0pt}%
\pgfpathmoveto{\pgfqpoint{4.030688in}{1.280433in}}%
\pgfpathlineto{\pgfqpoint{4.045099in}{1.272995in}}%
\pgfpathlineto{\pgfqpoint{4.059515in}{1.265653in}}%
\pgfpathlineto{\pgfqpoint{4.073936in}{1.258406in}}%
\pgfpathlineto{\pgfqpoint{4.088362in}{1.251255in}}%
\pgfpathlineto{\pgfqpoint{4.080035in}{1.252150in}}%
\pgfpathlineto{\pgfqpoint{4.071697in}{1.253452in}}%
\pgfpathlineto{\pgfqpoint{4.063350in}{1.255168in}}%
\pgfpathlineto{\pgfqpoint{4.048903in}{1.262846in}}%
\pgfpathlineto{\pgfqpoint{4.034460in}{1.270619in}}%
\pgfpathlineto{\pgfqpoint{4.020023in}{1.278488in}}%
\pgfpathlineto{\pgfqpoint{4.005590in}{1.286452in}}%
\pgfpathlineto{\pgfqpoint{4.013967in}{1.284028in}}%
\pgfpathlineto{\pgfqpoint{4.022333in}{1.282024in}}%
\pgfpathlineto{\pgfqpoint{4.030688in}{1.280433in}}%
\pgfpathclose%
\pgfusepath{fill}%
\end{pgfscope}%
\begin{pgfscope}%
\pgfpathrectangle{\pgfqpoint{1.150000in}{0.150000in}}{\pgfqpoint{5.700000in}{5.700000in}}%
\pgfusepath{clip}%
\pgfsetbuttcap%
\pgfsetroundjoin%
\definecolor{currentfill}{rgb}{0.281924,0.089666,0.412415}%
\pgfsetfillcolor{currentfill}%
\pgfsetfillopacity{0.700000}%
\pgfsetlinewidth{0.000000pt}%
\definecolor{currentstroke}{rgb}{0.000000,0.000000,0.000000}%
\pgfsetstrokecolor{currentstroke}%
\pgfsetdash{}{0pt}%
\pgfpathmoveto{\pgfqpoint{4.385600in}{1.177752in}}%
\pgfpathlineto{\pgfqpoint{4.400095in}{1.173540in}}%
\pgfpathlineto{\pgfqpoint{4.414597in}{1.169421in}}%
\pgfpathlineto{\pgfqpoint{4.429106in}{1.165394in}}%
\pgfpathlineto{\pgfqpoint{4.443624in}{1.161461in}}%
\pgfpathlineto{\pgfqpoint{4.435456in}{1.156094in}}%
\pgfpathlineto{\pgfqpoint{4.427283in}{1.151051in}}%
\pgfpathlineto{\pgfqpoint{4.419105in}{1.146339in}}%
\pgfpathlineto{\pgfqpoint{4.410921in}{1.141965in}}%
\pgfpathlineto{\pgfqpoint{4.396389in}{1.146560in}}%
\pgfpathlineto{\pgfqpoint{4.381864in}{1.151249in}}%
\pgfpathlineto{\pgfqpoint{4.367346in}{1.156029in}}%
\pgfpathlineto{\pgfqpoint{4.352835in}{1.160903in}}%
\pgfpathlineto{\pgfqpoint{4.361035in}{1.164608in}}%
\pgfpathlineto{\pgfqpoint{4.369230in}{1.168656in}}%
\pgfpathlineto{\pgfqpoint{4.377418in}{1.173039in}}%
\pgfpathlineto{\pgfqpoint{4.385600in}{1.177752in}}%
\pgfpathclose%
\pgfusepath{fill}%
\end{pgfscope}%
\begin{pgfscope}%
\pgfpathrectangle{\pgfqpoint{1.150000in}{0.150000in}}{\pgfqpoint{5.700000in}{5.700000in}}%
\pgfusepath{clip}%
\pgfsetbuttcap%
\pgfsetroundjoin%
\definecolor{currentfill}{rgb}{0.282910,0.105393,0.426902}%
\pgfsetfillcolor{currentfill}%
\pgfsetfillopacity{0.700000}%
\pgfsetlinewidth{0.000000pt}%
\definecolor{currentstroke}{rgb}{0.000000,0.000000,0.000000}%
\pgfsetstrokecolor{currentstroke}%
\pgfsetdash{}{0pt}%
\pgfpathmoveto{\pgfqpoint{4.237001in}{1.203246in}}%
\pgfpathlineto{\pgfqpoint{4.251457in}{1.197626in}}%
\pgfpathlineto{\pgfqpoint{4.265919in}{1.192100in}}%
\pgfpathlineto{\pgfqpoint{4.280388in}{1.186667in}}%
\pgfpathlineto{\pgfqpoint{4.294864in}{1.181328in}}%
\pgfpathlineto{\pgfqpoint{4.286639in}{1.178647in}}%
\pgfpathlineto{\pgfqpoint{4.278407in}{1.176327in}}%
\pgfpathlineto{\pgfqpoint{4.270167in}{1.174377in}}%
\pgfpathlineto{\pgfqpoint{4.261921in}{1.172803in}}%
\pgfpathlineto{\pgfqpoint{4.247425in}{1.178824in}}%
\pgfpathlineto{\pgfqpoint{4.232935in}{1.184938in}}%
\pgfpathlineto{\pgfqpoint{4.218451in}{1.191147in}}%
\pgfpathlineto{\pgfqpoint{4.203973in}{1.197449in}}%
\pgfpathlineto{\pgfqpoint{4.212242in}{1.198333in}}%
\pgfpathlineto{\pgfqpoint{4.220502in}{1.199599in}}%
\pgfpathlineto{\pgfqpoint{4.228755in}{1.201239in}}%
\pgfpathlineto{\pgfqpoint{4.237001in}{1.203246in}}%
\pgfpathclose%
\pgfusepath{fill}%
\end{pgfscope}%
\begin{pgfscope}%
\pgfpathrectangle{\pgfqpoint{1.150000in}{0.150000in}}{\pgfqpoint{5.700000in}{5.700000in}}%
\pgfusepath{clip}%
\pgfsetbuttcap%
\pgfsetroundjoin%
\definecolor{currentfill}{rgb}{0.281924,0.089666,0.412415}%
\pgfsetfillcolor{currentfill}%
\pgfsetfillopacity{0.700000}%
\pgfsetlinewidth{0.000000pt}%
\definecolor{currentstroke}{rgb}{0.000000,0.000000,0.000000}%
\pgfsetstrokecolor{currentstroke}%
\pgfsetdash{}{0pt}%
\pgfpathmoveto{\pgfqpoint{4.534347in}{1.173797in}}%
\pgfpathlineto{\pgfqpoint{4.548894in}{1.170969in}}%
\pgfpathlineto{\pgfqpoint{4.563449in}{1.168233in}}%
\pgfpathlineto{\pgfqpoint{4.578014in}{1.165590in}}%
\pgfpathlineto{\pgfqpoint{4.592587in}{1.163039in}}%
\pgfpathlineto{\pgfqpoint{4.584460in}{1.155172in}}%
\pgfpathlineto{\pgfqpoint{4.576329in}{1.147592in}}%
\pgfpathlineto{\pgfqpoint{4.568194in}{1.140306in}}%
\pgfpathlineto{\pgfqpoint{4.560055in}{1.133322in}}%
\pgfpathlineto{\pgfqpoint{4.545472in}{1.136516in}}%
\pgfpathlineto{\pgfqpoint{4.530897in}{1.139802in}}%
\pgfpathlineto{\pgfqpoint{4.516331in}{1.143181in}}%
\pgfpathlineto{\pgfqpoint{4.501773in}{1.146652in}}%
\pgfpathlineto{\pgfqpoint{4.509923in}{1.152986in}}%
\pgfpathlineto{\pgfqpoint{4.518069in}{1.159626in}}%
\pgfpathlineto{\pgfqpoint{4.526210in}{1.166565in}}%
\pgfpathlineto{\pgfqpoint{4.534347in}{1.173797in}}%
\pgfpathclose%
\pgfusepath{fill}%
\end{pgfscope}%
\begin{pgfscope}%
\pgfpathrectangle{\pgfqpoint{1.150000in}{0.150000in}}{\pgfqpoint{5.700000in}{5.700000in}}%
\pgfusepath{clip}%
\pgfsetbuttcap%
\pgfsetroundjoin%
\definecolor{currentfill}{rgb}{0.282884,0.135920,0.453427}%
\pgfsetfillcolor{currentfill}%
\pgfsetfillopacity{0.700000}%
\pgfsetlinewidth{0.000000pt}%
\definecolor{currentstroke}{rgb}{0.000000,0.000000,0.000000}%
\pgfsetstrokecolor{currentstroke}%
\pgfsetdash{}{0pt}%
\pgfpathmoveto{\pgfqpoint{4.865322in}{1.276104in}}%
\pgfpathlineto{\pgfqpoint{4.880015in}{1.276580in}}%
\pgfpathlineto{\pgfqpoint{4.894720in}{1.277149in}}%
\pgfpathlineto{\pgfqpoint{4.909435in}{1.277811in}}%
\pgfpathlineto{\pgfqpoint{4.901351in}{1.264784in}}%
\pgfpathlineto{\pgfqpoint{4.893266in}{1.251954in}}%
\pgfpathlineto{\pgfqpoint{4.885178in}{1.239328in}}%
\pgfpathlineto{\pgfqpoint{4.877089in}{1.226911in}}%
\pgfpathlineto{\pgfqpoint{4.862374in}{1.226842in}}%
\pgfpathlineto{\pgfqpoint{4.847671in}{1.226865in}}%
\pgfpathlineto{\pgfqpoint{4.832978in}{1.226981in}}%
\pgfpathlineto{\pgfqpoint{4.841067in}{1.238948in}}%
\pgfpathlineto{\pgfqpoint{4.849154in}{1.251128in}}%
\pgfpathlineto{\pgfqpoint{4.857239in}{1.263515in}}%
\pgfpathlineto{\pgfqpoint{4.865322in}{1.276104in}}%
\pgfpathclose%
\pgfusepath{fill}%
\end{pgfscope}%
\begin{pgfscope}%
\pgfpathrectangle{\pgfqpoint{1.150000in}{0.150000in}}{\pgfqpoint{5.700000in}{5.700000in}}%
\pgfusepath{clip}%
\pgfsetbuttcap%
\pgfsetroundjoin%
\definecolor{currentfill}{rgb}{0.283072,0.130895,0.449241}%
\pgfsetfillcolor{currentfill}%
\pgfsetfillopacity{0.700000}%
\pgfsetlinewidth{0.000000pt}%
\definecolor{currentstroke}{rgb}{0.000000,0.000000,0.000000}%
\pgfsetstrokecolor{currentstroke}%
\pgfsetdash{}{0pt}%
\pgfpathmoveto{\pgfqpoint{4.088362in}{1.251255in}}%
\pgfpathlineto{\pgfqpoint{4.102794in}{1.244198in}}%
\pgfpathlineto{\pgfqpoint{4.117231in}{1.237236in}}%
\pgfpathlineto{\pgfqpoint{4.131674in}{1.230369in}}%
\pgfpathlineto{\pgfqpoint{4.146122in}{1.223596in}}%
\pgfpathlineto{\pgfqpoint{4.137821in}{1.223796in}}%
\pgfpathlineto{\pgfqpoint{4.129510in}{1.224397in}}%
\pgfpathlineto{\pgfqpoint{4.121191in}{1.225407in}}%
\pgfpathlineto{\pgfqpoint{4.106723in}{1.232705in}}%
\pgfpathlineto{\pgfqpoint{4.092260in}{1.240098in}}%
\pgfpathlineto{\pgfqpoint{4.077803in}{1.247586in}}%
\pgfpathlineto{\pgfqpoint{4.063350in}{1.255168in}}%
\pgfpathlineto{\pgfqpoint{4.071697in}{1.253452in}}%
\pgfpathlineto{\pgfqpoint{4.080035in}{1.252150in}}%
\pgfpathlineto{\pgfqpoint{4.088362in}{1.251255in}}%
\pgfpathclose%
\pgfusepath{fill}%
\end{pgfscope}%
\begin{pgfscope}%
\pgfpathrectangle{\pgfqpoint{1.150000in}{0.150000in}}{\pgfqpoint{5.700000in}{5.700000in}}%
\pgfusepath{clip}%
\pgfsetbuttcap%
\pgfsetroundjoin%
\definecolor{currentfill}{rgb}{0.283197,0.115680,0.436115}%
\pgfsetfillcolor{currentfill}%
\pgfsetfillopacity{0.700000}%
\pgfsetlinewidth{0.000000pt}%
\definecolor{currentstroke}{rgb}{0.000000,0.000000,0.000000}%
\pgfsetstrokecolor{currentstroke}%
\pgfsetdash{}{0pt}%
\pgfpathmoveto{\pgfqpoint{4.774313in}{1.228369in}}%
\pgfpathlineto{\pgfqpoint{4.788963in}{1.227883in}}%
\pgfpathlineto{\pgfqpoint{4.803624in}{1.227490in}}%
\pgfpathlineto{\pgfqpoint{4.818296in}{1.227189in}}%
\pgfpathlineto{\pgfqpoint{4.832978in}{1.226981in}}%
\pgfpathlineto{\pgfqpoint{4.824887in}{1.215232in}}%
\pgfpathlineto{\pgfqpoint{4.816794in}{1.203708in}}%
\pgfpathlineto{\pgfqpoint{4.808699in}{1.192414in}}%
\pgfpathlineto{\pgfqpoint{4.800601in}{1.181357in}}%
\pgfpathlineto{\pgfqpoint{4.785918in}{1.182175in}}%
\pgfpathlineto{\pgfqpoint{4.771245in}{1.183085in}}%
\pgfpathlineto{\pgfqpoint{4.756582in}{1.184088in}}%
\pgfpathlineto{\pgfqpoint{4.741929in}{1.185182in}}%
\pgfpathlineto{\pgfqpoint{4.750029in}{1.195623in}}%
\pgfpathlineto{\pgfqpoint{4.758126in}{1.206305in}}%
\pgfpathlineto{\pgfqpoint{4.766221in}{1.217223in}}%
\pgfpathlineto{\pgfqpoint{4.774313in}{1.228369in}}%
\pgfpathclose%
\pgfusepath{fill}%
\end{pgfscope}%
\begin{pgfscope}%
\pgfpathrectangle{\pgfqpoint{1.150000in}{0.150000in}}{\pgfqpoint{5.700000in}{5.700000in}}%
\pgfusepath{clip}%
\pgfsetbuttcap%
\pgfsetroundjoin%
\definecolor{currentfill}{rgb}{0.282656,0.100196,0.422160}%
\pgfsetfillcolor{currentfill}%
\pgfsetfillopacity{0.700000}%
\pgfsetlinewidth{0.000000pt}%
\definecolor{currentstroke}{rgb}{0.000000,0.000000,0.000000}%
\pgfsetstrokecolor{currentstroke}%
\pgfsetdash{}{0pt}%
\pgfpathmoveto{\pgfqpoint{4.683417in}{1.190484in}}%
\pgfpathlineto{\pgfqpoint{4.698030in}{1.189020in}}%
\pgfpathlineto{\pgfqpoint{4.712653in}{1.187648in}}%
\pgfpathlineto{\pgfqpoint{4.727286in}{1.186369in}}%
\pgfpathlineto{\pgfqpoint{4.741929in}{1.185182in}}%
\pgfpathlineto{\pgfqpoint{4.733827in}{1.174988in}}%
\pgfpathlineto{\pgfqpoint{4.725722in}{1.165048in}}%
\pgfpathlineto{\pgfqpoint{4.717614in}{1.155367in}}%
\pgfpathlineto{\pgfqpoint{4.709504in}{1.145952in}}%
\pgfpathlineto{\pgfqpoint{4.694856in}{1.147766in}}%
\pgfpathlineto{\pgfqpoint{4.680218in}{1.149671in}}%
\pgfpathlineto{\pgfqpoint{4.665590in}{1.151669in}}%
\pgfpathlineto{\pgfqpoint{4.650971in}{1.153759in}}%
\pgfpathlineto{\pgfqpoint{4.659087in}{1.162541in}}%
\pgfpathlineto{\pgfqpoint{4.667200in}{1.171593in}}%
\pgfpathlineto{\pgfqpoint{4.675310in}{1.180909in}}%
\pgfpathlineto{\pgfqpoint{4.683417in}{1.190484in}}%
\pgfpathclose%
\pgfusepath{fill}%
\end{pgfscope}%
\begin{pgfscope}%
\pgfpathrectangle{\pgfqpoint{1.150000in}{0.150000in}}{\pgfqpoint{5.700000in}{5.700000in}}%
\pgfusepath{clip}%
\pgfsetbuttcap%
\pgfsetroundjoin%
\definecolor{currentfill}{rgb}{0.281924,0.089666,0.412415}%
\pgfsetfillcolor{currentfill}%
\pgfsetfillopacity{0.700000}%
\pgfsetlinewidth{0.000000pt}%
\definecolor{currentstroke}{rgb}{0.000000,0.000000,0.000000}%
\pgfsetstrokecolor{currentstroke}%
\pgfsetdash{}{0pt}%
\pgfpathmoveto{\pgfqpoint{4.443624in}{1.161461in}}%
\pgfpathlineto{\pgfqpoint{4.458149in}{1.157620in}}%
\pgfpathlineto{\pgfqpoint{4.472683in}{1.153872in}}%
\pgfpathlineto{\pgfqpoint{4.487224in}{1.150216in}}%
\pgfpathlineto{\pgfqpoint{4.501773in}{1.146652in}}%
\pgfpathlineto{\pgfqpoint{4.493619in}{1.140631in}}%
\pgfpathlineto{\pgfqpoint{4.485460in}{1.134928in}}%
\pgfpathlineto{\pgfqpoint{4.477296in}{1.129552in}}%
\pgfpathlineto{\pgfqpoint{4.469127in}{1.124508in}}%
\pgfpathlineto{\pgfqpoint{4.454564in}{1.128734in}}%
\pgfpathlineto{\pgfqpoint{4.440009in}{1.133052in}}%
\pgfpathlineto{\pgfqpoint{4.425461in}{1.137462in}}%
\pgfpathlineto{\pgfqpoint{4.410921in}{1.141965in}}%
\pgfpathlineto{\pgfqpoint{4.419105in}{1.146339in}}%
\pgfpathlineto{\pgfqpoint{4.427283in}{1.151051in}}%
\pgfpathlineto{\pgfqpoint{4.435456in}{1.156094in}}%
\pgfpathlineto{\pgfqpoint{4.443624in}{1.161461in}}%
\pgfpathclose%
\pgfusepath{fill}%
\end{pgfscope}%
\begin{pgfscope}%
\pgfpathrectangle{\pgfqpoint{1.150000in}{0.150000in}}{\pgfqpoint{5.700000in}{5.700000in}}%
\pgfusepath{clip}%
\pgfsetbuttcap%
\pgfsetroundjoin%
\definecolor{currentfill}{rgb}{0.282656,0.100196,0.422160}%
\pgfsetfillcolor{currentfill}%
\pgfsetfillopacity{0.700000}%
\pgfsetlinewidth{0.000000pt}%
\definecolor{currentstroke}{rgb}{0.000000,0.000000,0.000000}%
\pgfsetstrokecolor{currentstroke}%
\pgfsetdash{}{0pt}%
\pgfpathmoveto{\pgfqpoint{4.294864in}{1.181328in}}%
\pgfpathlineto{\pgfqpoint{4.309346in}{1.176082in}}%
\pgfpathlineto{\pgfqpoint{4.323836in}{1.170929in}}%
\pgfpathlineto{\pgfqpoint{4.338332in}{1.165870in}}%
\pgfpathlineto{\pgfqpoint{4.352835in}{1.160903in}}%
\pgfpathlineto{\pgfqpoint{4.344629in}{1.157548in}}%
\pgfpathlineto{\pgfqpoint{4.336416in}{1.154549in}}%
\pgfpathlineto{\pgfqpoint{4.328197in}{1.151914in}}%
\pgfpathlineto{\pgfqpoint{4.319971in}{1.149651in}}%
\pgfpathlineto{\pgfqpoint{4.305449in}{1.155299in}}%
\pgfpathlineto{\pgfqpoint{4.290933in}{1.161040in}}%
\pgfpathlineto{\pgfqpoint{4.276424in}{1.166875in}}%
\pgfpathlineto{\pgfqpoint{4.261921in}{1.172803in}}%
\pgfpathlineto{\pgfqpoint{4.270167in}{1.174377in}}%
\pgfpathlineto{\pgfqpoint{4.278407in}{1.176327in}}%
\pgfpathlineto{\pgfqpoint{4.286639in}{1.178647in}}%
\pgfpathlineto{\pgfqpoint{4.294864in}{1.181328in}}%
\pgfpathclose%
\pgfusepath{fill}%
\end{pgfscope}%
\begin{pgfscope}%
\pgfpathrectangle{\pgfqpoint{1.150000in}{0.150000in}}{\pgfqpoint{5.700000in}{5.700000in}}%
\pgfusepath{clip}%
\pgfsetbuttcap%
\pgfsetroundjoin%
\definecolor{currentfill}{rgb}{0.283187,0.125848,0.444960}%
\pgfsetfillcolor{currentfill}%
\pgfsetfillopacity{0.700000}%
\pgfsetlinewidth{0.000000pt}%
\definecolor{currentstroke}{rgb}{0.000000,0.000000,0.000000}%
\pgfsetstrokecolor{currentstroke}%
\pgfsetdash{}{0pt}%
\pgfpathmoveto{\pgfqpoint{4.146122in}{1.223596in}}%
\pgfpathlineto{\pgfqpoint{4.160576in}{1.216918in}}%
\pgfpathlineto{\pgfqpoint{4.175036in}{1.210334in}}%
\pgfpathlineto{\pgfqpoint{4.189501in}{1.203844in}}%
\pgfpathlineto{\pgfqpoint{4.203973in}{1.197449in}}%
\pgfpathlineto{\pgfqpoint{4.195696in}{1.196953in}}%
\pgfpathlineto{\pgfqpoint{4.187411in}{1.196853in}}%
\pgfpathlineto{\pgfqpoint{4.179118in}{1.197157in}}%
\pgfpathlineto{\pgfqpoint{4.164628in}{1.204078in}}%
\pgfpathlineto{\pgfqpoint{4.150143in}{1.211093in}}%
\pgfpathlineto{\pgfqpoint{4.135664in}{1.218203in}}%
\pgfpathlineto{\pgfqpoint{4.121191in}{1.225407in}}%
\pgfpathlineto{\pgfqpoint{4.129510in}{1.224397in}}%
\pgfpathlineto{\pgfqpoint{4.137821in}{1.223796in}}%
\pgfpathlineto{\pgfqpoint{4.146122in}{1.223596in}}%
\pgfpathclose%
\pgfusepath{fill}%
\end{pgfscope}%
\begin{pgfscope}%
\pgfpathrectangle{\pgfqpoint{1.150000in}{0.150000in}}{\pgfqpoint{5.700000in}{5.700000in}}%
\pgfusepath{clip}%
\pgfsetbuttcap%
\pgfsetroundjoin%
\definecolor{currentfill}{rgb}{0.282327,0.094955,0.417331}%
\pgfsetfillcolor{currentfill}%
\pgfsetfillopacity{0.700000}%
\pgfsetlinewidth{0.000000pt}%
\definecolor{currentstroke}{rgb}{0.000000,0.000000,0.000000}%
\pgfsetstrokecolor{currentstroke}%
\pgfsetdash{}{0pt}%
\pgfpathmoveto{\pgfqpoint{4.592587in}{1.163039in}}%
\pgfpathlineto{\pgfqpoint{4.607169in}{1.160581in}}%
\pgfpathlineto{\pgfqpoint{4.621761in}{1.158215in}}%
\pgfpathlineto{\pgfqpoint{4.636361in}{1.155941in}}%
\pgfpathlineto{\pgfqpoint{4.650971in}{1.153759in}}%
\pgfpathlineto{\pgfqpoint{4.642852in}{1.145254in}}%
\pgfpathlineto{\pgfqpoint{4.634729in}{1.137032in}}%
\pgfpathlineto{\pgfqpoint{4.626603in}{1.129100in}}%
\pgfpathlineto{\pgfqpoint{4.618473in}{1.121465in}}%
\pgfpathlineto{\pgfqpoint{4.603855in}{1.124291in}}%
\pgfpathlineto{\pgfqpoint{4.589246in}{1.127209in}}%
\pgfpathlineto{\pgfqpoint{4.574646in}{1.130219in}}%
\pgfpathlineto{\pgfqpoint{4.560055in}{1.133322in}}%
\pgfpathlineto{\pgfqpoint{4.568194in}{1.140306in}}%
\pgfpathlineto{\pgfqpoint{4.576329in}{1.147592in}}%
\pgfpathlineto{\pgfqpoint{4.584460in}{1.155172in}}%
\pgfpathlineto{\pgfqpoint{4.592587in}{1.163039in}}%
\pgfpathclose%
\pgfusepath{fill}%
\end{pgfscope}%
\begin{pgfscope}%
\pgfpathrectangle{\pgfqpoint{1.150000in}{0.150000in}}{\pgfqpoint{5.700000in}{5.700000in}}%
\pgfusepath{clip}%
\pgfsetbuttcap%
\pgfsetroundjoin%
\definecolor{currentfill}{rgb}{0.283187,0.125848,0.444960}%
\pgfsetfillcolor{currentfill}%
\pgfsetfillopacity{0.700000}%
\pgfsetlinewidth{0.000000pt}%
\definecolor{currentstroke}{rgb}{0.000000,0.000000,0.000000}%
\pgfsetstrokecolor{currentstroke}%
\pgfsetdash{}{0pt}%
\pgfpathmoveto{\pgfqpoint{4.832978in}{1.226981in}}%
\pgfpathlineto{\pgfqpoint{4.847671in}{1.226865in}}%
\pgfpathlineto{\pgfqpoint{4.862374in}{1.226842in}}%
\pgfpathlineto{\pgfqpoint{4.877089in}{1.226911in}}%
\pgfpathlineto{\pgfqpoint{4.868998in}{1.214710in}}%
\pgfpathlineto{\pgfqpoint{4.860906in}{1.202729in}}%
\pgfpathlineto{\pgfqpoint{4.852811in}{1.190976in}}%
\pgfpathlineto{\pgfqpoint{4.844715in}{1.179456in}}%
\pgfpathlineto{\pgfqpoint{4.830000in}{1.179998in}}%
\pgfpathlineto{\pgfqpoint{4.815295in}{1.180631in}}%
\pgfpathlineto{\pgfqpoint{4.800601in}{1.181357in}}%
\pgfpathlineto{\pgfqpoint{4.808699in}{1.192414in}}%
\pgfpathlineto{\pgfqpoint{4.816794in}{1.203708in}}%
\pgfpathlineto{\pgfqpoint{4.824887in}{1.215232in}}%
\pgfpathlineto{\pgfqpoint{4.832978in}{1.226981in}}%
\pgfpathclose%
\pgfusepath{fill}%
\end{pgfscope}%
\begin{pgfscope}%
\pgfpathrectangle{\pgfqpoint{1.150000in}{0.150000in}}{\pgfqpoint{5.700000in}{5.700000in}}%
\pgfusepath{clip}%
\pgfsetbuttcap%
\pgfsetroundjoin%
\definecolor{currentfill}{rgb}{0.282656,0.100196,0.422160}%
\pgfsetfillcolor{currentfill}%
\pgfsetfillopacity{0.700000}%
\pgfsetlinewidth{0.000000pt}%
\definecolor{currentstroke}{rgb}{0.000000,0.000000,0.000000}%
\pgfsetstrokecolor{currentstroke}%
\pgfsetdash{}{0pt}%
\pgfpathmoveto{\pgfqpoint{4.352835in}{1.160903in}}%
\pgfpathlineto{\pgfqpoint{4.367346in}{1.156029in}}%
\pgfpathlineto{\pgfqpoint{4.381864in}{1.151249in}}%
\pgfpathlineto{\pgfqpoint{4.396389in}{1.146560in}}%
\pgfpathlineto{\pgfqpoint{4.410921in}{1.141965in}}%
\pgfpathlineto{\pgfqpoint{4.402732in}{1.137935in}}%
\pgfpathlineto{\pgfqpoint{4.394537in}{1.134258in}}%
\pgfpathlineto{\pgfqpoint{4.386336in}{1.130939in}}%
\pgfpathlineto{\pgfqpoint{4.378129in}{1.127985in}}%
\pgfpathlineto{\pgfqpoint{4.363579in}{1.133263in}}%
\pgfpathlineto{\pgfqpoint{4.349036in}{1.138633in}}%
\pgfpathlineto{\pgfqpoint{4.334500in}{1.144095in}}%
\pgfpathlineto{\pgfqpoint{4.319971in}{1.149651in}}%
\pgfpathlineto{\pgfqpoint{4.328197in}{1.151914in}}%
\pgfpathlineto{\pgfqpoint{4.336416in}{1.154549in}}%
\pgfpathlineto{\pgfqpoint{4.344629in}{1.157548in}}%
\pgfpathlineto{\pgfqpoint{4.352835in}{1.160903in}}%
\pgfpathclose%
\pgfusepath{fill}%
\end{pgfscope}%
\begin{pgfscope}%
\pgfpathrectangle{\pgfqpoint{1.150000in}{0.150000in}}{\pgfqpoint{5.700000in}{5.700000in}}%
\pgfusepath{clip}%
\pgfsetbuttcap%
\pgfsetroundjoin%
\definecolor{currentfill}{rgb}{0.282327,0.094955,0.417331}%
\pgfsetfillcolor{currentfill}%
\pgfsetfillopacity{0.700000}%
\pgfsetlinewidth{0.000000pt}%
\definecolor{currentstroke}{rgb}{0.000000,0.000000,0.000000}%
\pgfsetstrokecolor{currentstroke}%
\pgfsetdash{}{0pt}%
\pgfpathmoveto{\pgfqpoint{4.501773in}{1.146652in}}%
\pgfpathlineto{\pgfqpoint{4.516331in}{1.143181in}}%
\pgfpathlineto{\pgfqpoint{4.530897in}{1.139802in}}%
\pgfpathlineto{\pgfqpoint{4.545472in}{1.136516in}}%
\pgfpathlineto{\pgfqpoint{4.560055in}{1.133322in}}%
\pgfpathlineto{\pgfqpoint{4.551912in}{1.126645in}}%
\pgfpathlineto{\pgfqpoint{4.543765in}{1.120282in}}%
\pgfpathlineto{\pgfqpoint{4.535613in}{1.114241in}}%
\pgfpathlineto{\pgfqpoint{4.527457in}{1.108527in}}%
\pgfpathlineto{\pgfqpoint{4.512863in}{1.112384in}}%
\pgfpathlineto{\pgfqpoint{4.498276in}{1.116333in}}%
\pgfpathlineto{\pgfqpoint{4.483697in}{1.120375in}}%
\pgfpathlineto{\pgfqpoint{4.469127in}{1.124508in}}%
\pgfpathlineto{\pgfqpoint{4.477296in}{1.129552in}}%
\pgfpathlineto{\pgfqpoint{4.485460in}{1.134928in}}%
\pgfpathlineto{\pgfqpoint{4.493619in}{1.140631in}}%
\pgfpathlineto{\pgfqpoint{4.501773in}{1.146652in}}%
\pgfpathclose%
\pgfusepath{fill}%
\end{pgfscope}%
\begin{pgfscope}%
\pgfpathrectangle{\pgfqpoint{1.150000in}{0.150000in}}{\pgfqpoint{5.700000in}{5.700000in}}%
\pgfusepath{clip}%
\pgfsetbuttcap%
\pgfsetroundjoin%
\definecolor{currentfill}{rgb}{0.283091,0.110553,0.431554}%
\pgfsetfillcolor{currentfill}%
\pgfsetfillopacity{0.700000}%
\pgfsetlinewidth{0.000000pt}%
\definecolor{currentstroke}{rgb}{0.000000,0.000000,0.000000}%
\pgfsetstrokecolor{currentstroke}%
\pgfsetdash{}{0pt}%
\pgfpathmoveto{\pgfqpoint{4.741929in}{1.185182in}}%
\pgfpathlineto{\pgfqpoint{4.756582in}{1.184088in}}%
\pgfpathlineto{\pgfqpoint{4.771245in}{1.183085in}}%
\pgfpathlineto{\pgfqpoint{4.785918in}{1.182175in}}%
\pgfpathlineto{\pgfqpoint{4.800601in}{1.181357in}}%
\pgfpathlineto{\pgfqpoint{4.792502in}{1.170543in}}%
\pgfpathlineto{\pgfqpoint{4.784401in}{1.159977in}}%
\pgfpathlineto{\pgfqpoint{4.776297in}{1.149667in}}%
\pgfpathlineto{\pgfqpoint{4.768191in}{1.139618in}}%
\pgfpathlineto{\pgfqpoint{4.753504in}{1.141063in}}%
\pgfpathlineto{\pgfqpoint{4.738828in}{1.142601in}}%
\pgfpathlineto{\pgfqpoint{4.724161in}{1.144231in}}%
\pgfpathlineto{\pgfqpoint{4.709504in}{1.145952in}}%
\pgfpathlineto{\pgfqpoint{4.717614in}{1.155367in}}%
\pgfpathlineto{\pgfqpoint{4.725722in}{1.165048in}}%
\pgfpathlineto{\pgfqpoint{4.733827in}{1.174988in}}%
\pgfpathlineto{\pgfqpoint{4.741929in}{1.185182in}}%
\pgfpathclose%
\pgfusepath{fill}%
\end{pgfscope}%
\begin{pgfscope}%
\pgfpathrectangle{\pgfqpoint{1.150000in}{0.150000in}}{\pgfqpoint{5.700000in}{5.700000in}}%
\pgfusepath{clip}%
\pgfsetbuttcap%
\pgfsetroundjoin%
\definecolor{currentfill}{rgb}{0.283229,0.120777,0.440584}%
\pgfsetfillcolor{currentfill}%
\pgfsetfillopacity{0.700000}%
\pgfsetlinewidth{0.000000pt}%
\definecolor{currentstroke}{rgb}{0.000000,0.000000,0.000000}%
\pgfsetstrokecolor{currentstroke}%
\pgfsetdash{}{0pt}%
\pgfpathmoveto{\pgfqpoint{4.203973in}{1.197449in}}%
\pgfpathlineto{\pgfqpoint{4.218451in}{1.191147in}}%
\pgfpathlineto{\pgfqpoint{4.232935in}{1.184938in}}%
\pgfpathlineto{\pgfqpoint{4.247425in}{1.178824in}}%
\pgfpathlineto{\pgfqpoint{4.261921in}{1.172803in}}%
\pgfpathlineto{\pgfqpoint{4.253667in}{1.171612in}}%
\pgfpathlineto{\pgfqpoint{4.245406in}{1.170812in}}%
\pgfpathlineto{\pgfqpoint{4.237136in}{1.170411in}}%
\pgfpathlineto{\pgfqpoint{4.222623in}{1.176957in}}%
\pgfpathlineto{\pgfqpoint{4.208115in}{1.183597in}}%
\pgfpathlineto{\pgfqpoint{4.193614in}{1.190330in}}%
\pgfpathlineto{\pgfqpoint{4.179118in}{1.197157in}}%
\pgfpathlineto{\pgfqpoint{4.187411in}{1.196853in}}%
\pgfpathlineto{\pgfqpoint{4.195696in}{1.196953in}}%
\pgfpathlineto{\pgfqpoint{4.203973in}{1.197449in}}%
\pgfpathclose%
\pgfusepath{fill}%
\end{pgfscope}%
\begin{pgfscope}%
\pgfpathrectangle{\pgfqpoint{1.150000in}{0.150000in}}{\pgfqpoint{5.700000in}{5.700000in}}%
\pgfusepath{clip}%
\pgfsetbuttcap%
\pgfsetroundjoin%
\definecolor{currentfill}{rgb}{0.282656,0.100196,0.422160}%
\pgfsetfillcolor{currentfill}%
\pgfsetfillopacity{0.700000}%
\pgfsetlinewidth{0.000000pt}%
\definecolor{currentstroke}{rgb}{0.000000,0.000000,0.000000}%
\pgfsetstrokecolor{currentstroke}%
\pgfsetdash{}{0pt}%
\pgfpathmoveto{\pgfqpoint{4.650971in}{1.153759in}}%
\pgfpathlineto{\pgfqpoint{4.665590in}{1.151669in}}%
\pgfpathlineto{\pgfqpoint{4.680218in}{1.149671in}}%
\pgfpathlineto{\pgfqpoint{4.694856in}{1.147766in}}%
\pgfpathlineto{\pgfqpoint{4.709504in}{1.145952in}}%
\pgfpathlineto{\pgfqpoint{4.701391in}{1.136809in}}%
\pgfpathlineto{\pgfqpoint{4.693275in}{1.127945in}}%
\pgfpathlineto{\pgfqpoint{4.685156in}{1.119366in}}%
\pgfpathlineto{\pgfqpoint{4.677034in}{1.111079in}}%
\pgfpathlineto{\pgfqpoint{4.662380in}{1.113537in}}%
\pgfpathlineto{\pgfqpoint{4.647735in}{1.116088in}}%
\pgfpathlineto{\pgfqpoint{4.633100in}{1.118730in}}%
\pgfpathlineto{\pgfqpoint{4.618473in}{1.121465in}}%
\pgfpathlineto{\pgfqpoint{4.626603in}{1.129100in}}%
\pgfpathlineto{\pgfqpoint{4.634729in}{1.137032in}}%
\pgfpathlineto{\pgfqpoint{4.642852in}{1.145254in}}%
\pgfpathlineto{\pgfqpoint{4.650971in}{1.153759in}}%
\pgfpathclose%
\pgfusepath{fill}%
\end{pgfscope}%
\begin{pgfscope}%
\pgfpathrectangle{\pgfqpoint{1.150000in}{0.150000in}}{\pgfqpoint{5.700000in}{5.700000in}}%
\pgfusepath{clip}%
\pgfsetbuttcap%
\pgfsetroundjoin%
\definecolor{currentfill}{rgb}{0.282656,0.100196,0.422160}%
\pgfsetfillcolor{currentfill}%
\pgfsetfillopacity{0.700000}%
\pgfsetlinewidth{0.000000pt}%
\definecolor{currentstroke}{rgb}{0.000000,0.000000,0.000000}%
\pgfsetstrokecolor{currentstroke}%
\pgfsetdash{}{0pt}%
\pgfpathmoveto{\pgfqpoint{4.410921in}{1.141965in}}%
\pgfpathlineto{\pgfqpoint{4.425461in}{1.137462in}}%
\pgfpathlineto{\pgfqpoint{4.440009in}{1.133052in}}%
\pgfpathlineto{\pgfqpoint{4.454564in}{1.128734in}}%
\pgfpathlineto{\pgfqpoint{4.469127in}{1.124508in}}%
\pgfpathlineto{\pgfqpoint{4.460953in}{1.119804in}}%
\pgfpathlineto{\pgfqpoint{4.452774in}{1.115447in}}%
\pgfpathlineto{\pgfqpoint{4.444589in}{1.111443in}}%
\pgfpathlineto{\pgfqpoint{4.436399in}{1.107801in}}%
\pgfpathlineto{\pgfqpoint{4.421821in}{1.112709in}}%
\pgfpathlineto{\pgfqpoint{4.407250in}{1.117708in}}%
\pgfpathlineto{\pgfqpoint{4.392686in}{1.122801in}}%
\pgfpathlineto{\pgfqpoint{4.378129in}{1.127985in}}%
\pgfpathlineto{\pgfqpoint{4.386336in}{1.130939in}}%
\pgfpathlineto{\pgfqpoint{4.394537in}{1.134258in}}%
\pgfpathlineto{\pgfqpoint{4.402732in}{1.137935in}}%
\pgfpathlineto{\pgfqpoint{4.410921in}{1.141965in}}%
\pgfpathclose%
\pgfusepath{fill}%
\end{pgfscope}%
\begin{pgfscope}%
\pgfpathrectangle{\pgfqpoint{1.150000in}{0.150000in}}{\pgfqpoint{5.700000in}{5.700000in}}%
\pgfusepath{clip}%
\pgfsetbuttcap%
\pgfsetroundjoin%
\definecolor{currentfill}{rgb}{0.283197,0.115680,0.436115}%
\pgfsetfillcolor{currentfill}%
\pgfsetfillopacity{0.700000}%
\pgfsetlinewidth{0.000000pt}%
\definecolor{currentstroke}{rgb}{0.000000,0.000000,0.000000}%
\pgfsetstrokecolor{currentstroke}%
\pgfsetdash{}{0pt}%
\pgfpathmoveto{\pgfqpoint{4.261921in}{1.172803in}}%
\pgfpathlineto{\pgfqpoint{4.276424in}{1.166875in}}%
\pgfpathlineto{\pgfqpoint{4.290933in}{1.161040in}}%
\pgfpathlineto{\pgfqpoint{4.305449in}{1.155299in}}%
\pgfpathlineto{\pgfqpoint{4.319971in}{1.149651in}}%
\pgfpathlineto{\pgfqpoint{4.311738in}{1.147765in}}%
\pgfpathlineto{\pgfqpoint{4.303499in}{1.146266in}}%
\pgfpathlineto{\pgfqpoint{4.295252in}{1.145159in}}%
\pgfpathlineto{\pgfqpoint{4.280714in}{1.151332in}}%
\pgfpathlineto{\pgfqpoint{4.266182in}{1.157598in}}%
\pgfpathlineto{\pgfqpoint{4.251656in}{1.163958in}}%
\pgfpathlineto{\pgfqpoint{4.237136in}{1.170411in}}%
\pgfpathlineto{\pgfqpoint{4.245406in}{1.170812in}}%
\pgfpathlineto{\pgfqpoint{4.253667in}{1.171612in}}%
\pgfpathlineto{\pgfqpoint{4.261921in}{1.172803in}}%
\pgfpathclose%
\pgfusepath{fill}%
\end{pgfscope}%
\begin{pgfscope}%
\pgfpathrectangle{\pgfqpoint{1.150000in}{0.150000in}}{\pgfqpoint{5.700000in}{5.700000in}}%
\pgfusepath{clip}%
\pgfsetbuttcap%
\pgfsetroundjoin%
\definecolor{currentfill}{rgb}{0.282327,0.094955,0.417331}%
\pgfsetfillcolor{currentfill}%
\pgfsetfillopacity{0.700000}%
\pgfsetlinewidth{0.000000pt}%
\definecolor{currentstroke}{rgb}{0.000000,0.000000,0.000000}%
\pgfsetstrokecolor{currentstroke}%
\pgfsetdash{}{0pt}%
\pgfpathmoveto{\pgfqpoint{4.560055in}{1.133322in}}%
\pgfpathlineto{\pgfqpoint{4.574646in}{1.130219in}}%
\pgfpathlineto{\pgfqpoint{4.589246in}{1.127209in}}%
\pgfpathlineto{\pgfqpoint{4.603855in}{1.124291in}}%
\pgfpathlineto{\pgfqpoint{4.618473in}{1.121465in}}%
\pgfpathlineto{\pgfqpoint{4.610340in}{1.114132in}}%
\pgfpathlineto{\pgfqpoint{4.602203in}{1.107109in}}%
\pgfpathlineto{\pgfqpoint{4.594063in}{1.100402in}}%
\pgfpathlineto{\pgfqpoint{4.585919in}{1.094018in}}%
\pgfpathlineto{\pgfqpoint{4.571291in}{1.097508in}}%
\pgfpathlineto{\pgfqpoint{4.556671in}{1.101089in}}%
\pgfpathlineto{\pgfqpoint{4.542060in}{1.104762in}}%
\pgfpathlineto{\pgfqpoint{4.527457in}{1.108527in}}%
\pgfpathlineto{\pgfqpoint{4.535613in}{1.114241in}}%
\pgfpathlineto{\pgfqpoint{4.543765in}{1.120282in}}%
\pgfpathlineto{\pgfqpoint{4.551912in}{1.126645in}}%
\pgfpathlineto{\pgfqpoint{4.560055in}{1.133322in}}%
\pgfpathclose%
\pgfusepath{fill}%
\end{pgfscope}%
\begin{pgfscope}%
\pgfpathrectangle{\pgfqpoint{1.150000in}{0.150000in}}{\pgfqpoint{5.700000in}{5.700000in}}%
\pgfusepath{clip}%
\pgfsetbuttcap%
\pgfsetroundjoin%
\definecolor{currentfill}{rgb}{0.283229,0.120777,0.440584}%
\pgfsetfillcolor{currentfill}%
\pgfsetfillopacity{0.700000}%
\pgfsetlinewidth{0.000000pt}%
\definecolor{currentstroke}{rgb}{0.000000,0.000000,0.000000}%
\pgfsetstrokecolor{currentstroke}%
\pgfsetdash{}{0pt}%
\pgfpathmoveto{\pgfqpoint{4.800601in}{1.181357in}}%
\pgfpathlineto{\pgfqpoint{4.815295in}{1.180631in}}%
\pgfpathlineto{\pgfqpoint{4.830000in}{1.179998in}}%
\pgfpathlineto{\pgfqpoint{4.844715in}{1.179456in}}%
\pgfpathlineto{\pgfqpoint{4.836617in}{1.168176in}}%
\pgfpathlineto{\pgfqpoint{4.828517in}{1.157141in}}%
\pgfpathlineto{\pgfqpoint{4.820415in}{1.146357in}}%
\pgfpathlineto{\pgfqpoint{4.812311in}{1.135832in}}%
\pgfpathlineto{\pgfqpoint{4.797594in}{1.137002in}}%
\pgfpathlineto{\pgfqpoint{4.782887in}{1.138264in}}%
\pgfpathlineto{\pgfqpoint{4.768191in}{1.139618in}}%
\pgfpathlineto{\pgfqpoint{4.776297in}{1.149667in}}%
\pgfpathlineto{\pgfqpoint{4.784401in}{1.159977in}}%
\pgfpathlineto{\pgfqpoint{4.792502in}{1.170543in}}%
\pgfpathlineto{\pgfqpoint{4.800601in}{1.181357in}}%
\pgfpathclose%
\pgfusepath{fill}%
\end{pgfscope}%
\begin{pgfscope}%
\pgfpathrectangle{\pgfqpoint{1.150000in}{0.150000in}}{\pgfqpoint{5.700000in}{5.700000in}}%
\pgfusepath{clip}%
\pgfsetbuttcap%
\pgfsetroundjoin%
\definecolor{currentfill}{rgb}{0.282910,0.105393,0.426902}%
\pgfsetfillcolor{currentfill}%
\pgfsetfillopacity{0.700000}%
\pgfsetlinewidth{0.000000pt}%
\definecolor{currentstroke}{rgb}{0.000000,0.000000,0.000000}%
\pgfsetstrokecolor{currentstroke}%
\pgfsetdash{}{0pt}%
\pgfpathmoveto{\pgfqpoint{4.709504in}{1.145952in}}%
\pgfpathlineto{\pgfqpoint{4.724161in}{1.144231in}}%
\pgfpathlineto{\pgfqpoint{4.738828in}{1.142601in}}%
\pgfpathlineto{\pgfqpoint{4.753504in}{1.141063in}}%
\pgfpathlineto{\pgfqpoint{4.768191in}{1.139618in}}%
\pgfpathlineto{\pgfqpoint{4.760082in}{1.129836in}}%
\pgfpathlineto{\pgfqpoint{4.751972in}{1.120328in}}%
\pgfpathlineto{\pgfqpoint{4.743859in}{1.111101in}}%
\pgfpathlineto{\pgfqpoint{4.735743in}{1.102161in}}%
\pgfpathlineto{\pgfqpoint{4.721052in}{1.104253in}}%
\pgfpathlineto{\pgfqpoint{4.706370in}{1.106436in}}%
\pgfpathlineto{\pgfqpoint{4.691697in}{1.108712in}}%
\pgfpathlineto{\pgfqpoint{4.677034in}{1.111079in}}%
\pgfpathlineto{\pgfqpoint{4.685156in}{1.119366in}}%
\pgfpathlineto{\pgfqpoint{4.693275in}{1.127945in}}%
\pgfpathlineto{\pgfqpoint{4.701391in}{1.136809in}}%
\pgfpathlineto{\pgfqpoint{4.709504in}{1.145952in}}%
\pgfpathclose%
\pgfusepath{fill}%
\end{pgfscope}%
\begin{pgfscope}%
\pgfpathrectangle{\pgfqpoint{1.150000in}{0.150000in}}{\pgfqpoint{5.700000in}{5.700000in}}%
\pgfusepath{clip}%
\pgfsetbuttcap%
\pgfsetroundjoin%
\definecolor{currentfill}{rgb}{0.283197,0.115680,0.436115}%
\pgfsetfillcolor{currentfill}%
\pgfsetfillopacity{0.700000}%
\pgfsetlinewidth{0.000000pt}%
\definecolor{currentstroke}{rgb}{0.000000,0.000000,0.000000}%
\pgfsetstrokecolor{currentstroke}%
\pgfsetdash{}{0pt}%
\pgfpathmoveto{\pgfqpoint{4.319971in}{1.149651in}}%
\pgfpathlineto{\pgfqpoint{4.334500in}{1.144095in}}%
\pgfpathlineto{\pgfqpoint{4.349036in}{1.138633in}}%
\pgfpathlineto{\pgfqpoint{4.363579in}{1.133263in}}%
\pgfpathlineto{\pgfqpoint{4.378129in}{1.127985in}}%
\pgfpathlineto{\pgfqpoint{4.369916in}{1.125406in}}%
\pgfpathlineto{\pgfqpoint{4.361696in}{1.123206in}}%
\pgfpathlineto{\pgfqpoint{4.353470in}{1.121395in}}%
\pgfpathlineto{\pgfqpoint{4.338906in}{1.127197in}}%
\pgfpathlineto{\pgfqpoint{4.324348in}{1.133091in}}%
\pgfpathlineto{\pgfqpoint{4.309797in}{1.139079in}}%
\pgfpathlineto{\pgfqpoint{4.295252in}{1.145159in}}%
\pgfpathlineto{\pgfqpoint{4.303499in}{1.146266in}}%
\pgfpathlineto{\pgfqpoint{4.311738in}{1.147765in}}%
\pgfpathlineto{\pgfqpoint{4.319971in}{1.149651in}}%
\pgfpathclose%
\pgfusepath{fill}%
\end{pgfscope}%
\begin{pgfscope}%
\pgfpathrectangle{\pgfqpoint{1.150000in}{0.150000in}}{\pgfqpoint{5.700000in}{5.700000in}}%
\pgfusepath{clip}%
\pgfsetbuttcap%
\pgfsetroundjoin%
\definecolor{currentfill}{rgb}{0.282656,0.100196,0.422160}%
\pgfsetfillcolor{currentfill}%
\pgfsetfillopacity{0.700000}%
\pgfsetlinewidth{0.000000pt}%
\definecolor{currentstroke}{rgb}{0.000000,0.000000,0.000000}%
\pgfsetstrokecolor{currentstroke}%
\pgfsetdash{}{0pt}%
\pgfpathmoveto{\pgfqpoint{4.469127in}{1.124508in}}%
\pgfpathlineto{\pgfqpoint{4.483697in}{1.120375in}}%
\pgfpathlineto{\pgfqpoint{4.498276in}{1.116333in}}%
\pgfpathlineto{\pgfqpoint{4.512863in}{1.112384in}}%
\pgfpathlineto{\pgfqpoint{4.527457in}{1.108527in}}%
\pgfpathlineto{\pgfqpoint{4.519297in}{1.103149in}}%
\pgfpathlineto{\pgfqpoint{4.511132in}{1.098112in}}%
\pgfpathlineto{\pgfqpoint{4.502963in}{1.093424in}}%
\pgfpathlineto{\pgfqpoint{4.494789in}{1.089092in}}%
\pgfpathlineto{\pgfqpoint{4.480180in}{1.093631in}}%
\pgfpathlineto{\pgfqpoint{4.465579in}{1.098262in}}%
\pgfpathlineto{\pgfqpoint{4.450985in}{1.102985in}}%
\pgfpathlineto{\pgfqpoint{4.436399in}{1.107801in}}%
\pgfpathlineto{\pgfqpoint{4.444589in}{1.111443in}}%
\pgfpathlineto{\pgfqpoint{4.452774in}{1.115447in}}%
\pgfpathlineto{\pgfqpoint{4.460953in}{1.119804in}}%
\pgfpathlineto{\pgfqpoint{4.469127in}{1.124508in}}%
\pgfpathclose%
\pgfusepath{fill}%
\end{pgfscope}%
\begin{pgfscope}%
\pgfpathrectangle{\pgfqpoint{1.150000in}{0.150000in}}{\pgfqpoint{5.700000in}{5.700000in}}%
\pgfusepath{clip}%
\pgfsetbuttcap%
\pgfsetroundjoin%
\definecolor{currentfill}{rgb}{0.282656,0.100196,0.422160}%
\pgfsetfillcolor{currentfill}%
\pgfsetfillopacity{0.700000}%
\pgfsetlinewidth{0.000000pt}%
\definecolor{currentstroke}{rgb}{0.000000,0.000000,0.000000}%
\pgfsetstrokecolor{currentstroke}%
\pgfsetdash{}{0pt}%
\pgfpathmoveto{\pgfqpoint{4.618473in}{1.121465in}}%
\pgfpathlineto{\pgfqpoint{4.633100in}{1.118730in}}%
\pgfpathlineto{\pgfqpoint{4.647735in}{1.116088in}}%
\pgfpathlineto{\pgfqpoint{4.662380in}{1.113537in}}%
\pgfpathlineto{\pgfqpoint{4.677034in}{1.111079in}}%
\pgfpathlineto{\pgfqpoint{4.668909in}{1.103089in}}%
\pgfpathlineto{\pgfqpoint{4.660781in}{1.095405in}}%
\pgfpathlineto{\pgfqpoint{4.652650in}{1.088032in}}%
\pgfpathlineto{\pgfqpoint{4.644516in}{1.080977in}}%
\pgfpathlineto{\pgfqpoint{4.629853in}{1.084100in}}%
\pgfpathlineto{\pgfqpoint{4.615200in}{1.087314in}}%
\pgfpathlineto{\pgfqpoint{4.600555in}{1.090620in}}%
\pgfpathlineto{\pgfqpoint{4.585919in}{1.094018in}}%
\pgfpathlineto{\pgfqpoint{4.594063in}{1.100402in}}%
\pgfpathlineto{\pgfqpoint{4.602203in}{1.107109in}}%
\pgfpathlineto{\pgfqpoint{4.610340in}{1.114132in}}%
\pgfpathlineto{\pgfqpoint{4.618473in}{1.121465in}}%
\pgfpathclose%
\pgfusepath{fill}%
\end{pgfscope}%
\begin{pgfscope}%
\pgfpathrectangle{\pgfqpoint{1.150000in}{0.150000in}}{\pgfqpoint{5.700000in}{5.700000in}}%
\pgfusepath{clip}%
\pgfsetbuttcap%
\pgfsetroundjoin%
\definecolor{currentfill}{rgb}{0.283091,0.110553,0.431554}%
\pgfsetfillcolor{currentfill}%
\pgfsetfillopacity{0.700000}%
\pgfsetlinewidth{0.000000pt}%
\definecolor{currentstroke}{rgb}{0.000000,0.000000,0.000000}%
\pgfsetstrokecolor{currentstroke}%
\pgfsetdash{}{0pt}%
\pgfpathmoveto{\pgfqpoint{4.768191in}{1.139618in}}%
\pgfpathlineto{\pgfqpoint{4.782887in}{1.138264in}}%
\pgfpathlineto{\pgfqpoint{4.797594in}{1.137002in}}%
\pgfpathlineto{\pgfqpoint{4.812311in}{1.135832in}}%
\pgfpathlineto{\pgfqpoint{4.804205in}{1.125570in}}%
\pgfpathlineto{\pgfqpoint{4.796097in}{1.115579in}}%
\pgfpathlineto{\pgfqpoint{4.787987in}{1.105865in}}%
\pgfpathlineto{\pgfqpoint{4.779875in}{1.096435in}}%
\pgfpathlineto{\pgfqpoint{4.765155in}{1.098252in}}%
\pgfpathlineto{\pgfqpoint{4.750444in}{1.100160in}}%
\pgfpathlineto{\pgfqpoint{4.735743in}{1.102161in}}%
\pgfpathlineto{\pgfqpoint{4.743859in}{1.111101in}}%
\pgfpathlineto{\pgfqpoint{4.751972in}{1.120328in}}%
\pgfpathlineto{\pgfqpoint{4.760082in}{1.129836in}}%
\pgfpathlineto{\pgfqpoint{4.768191in}{1.139618in}}%
\pgfpathclose%
\pgfusepath{fill}%
\end{pgfscope}%
\begin{pgfscope}%
\pgfpathrectangle{\pgfqpoint{1.150000in}{0.150000in}}{\pgfqpoint{5.700000in}{5.700000in}}%
\pgfusepath{clip}%
\pgfsetbuttcap%
\pgfsetroundjoin%
\definecolor{currentfill}{rgb}{0.283091,0.110553,0.431554}%
\pgfsetfillcolor{currentfill}%
\pgfsetfillopacity{0.700000}%
\pgfsetlinewidth{0.000000pt}%
\definecolor{currentstroke}{rgb}{0.000000,0.000000,0.000000}%
\pgfsetstrokecolor{currentstroke}%
\pgfsetdash{}{0pt}%
\pgfpathmoveto{\pgfqpoint{4.378129in}{1.127985in}}%
\pgfpathlineto{\pgfqpoint{4.392686in}{1.122801in}}%
\pgfpathlineto{\pgfqpoint{4.407250in}{1.117708in}}%
\pgfpathlineto{\pgfqpoint{4.421821in}{1.112709in}}%
\pgfpathlineto{\pgfqpoint{4.436399in}{1.107801in}}%
\pgfpathlineto{\pgfqpoint{4.428204in}{1.104527in}}%
\pgfpathlineto{\pgfqpoint{4.420003in}{1.101628in}}%
\pgfpathlineto{\pgfqpoint{4.411796in}{1.099112in}}%
\pgfpathlineto{\pgfqpoint{4.397204in}{1.104544in}}%
\pgfpathlineto{\pgfqpoint{4.382619in}{1.110068in}}%
\pgfpathlineto{\pgfqpoint{4.368041in}{1.115685in}}%
\pgfpathlineto{\pgfqpoint{4.353470in}{1.121395in}}%
\pgfpathlineto{\pgfqpoint{4.361696in}{1.123206in}}%
\pgfpathlineto{\pgfqpoint{4.369916in}{1.125406in}}%
\pgfpathlineto{\pgfqpoint{4.378129in}{1.127985in}}%
\pgfpathclose%
\pgfusepath{fill}%
\end{pgfscope}%
\begin{pgfscope}%
\pgfpathrectangle{\pgfqpoint{1.150000in}{0.150000in}}{\pgfqpoint{5.700000in}{5.700000in}}%
\pgfusepath{clip}%
\pgfsetbuttcap%
\pgfsetroundjoin%
\definecolor{currentfill}{rgb}{0.282656,0.100196,0.422160}%
\pgfsetfillcolor{currentfill}%
\pgfsetfillopacity{0.700000}%
\pgfsetlinewidth{0.000000pt}%
\definecolor{currentstroke}{rgb}{0.000000,0.000000,0.000000}%
\pgfsetstrokecolor{currentstroke}%
\pgfsetdash{}{0pt}%
\pgfpathmoveto{\pgfqpoint{4.527457in}{1.108527in}}%
\pgfpathlineto{\pgfqpoint{4.542060in}{1.104762in}}%
\pgfpathlineto{\pgfqpoint{4.556671in}{1.101089in}}%
\pgfpathlineto{\pgfqpoint{4.571291in}{1.097508in}}%
\pgfpathlineto{\pgfqpoint{4.585919in}{1.094018in}}%
\pgfpathlineto{\pgfqpoint{4.577770in}{1.087964in}}%
\pgfpathlineto{\pgfqpoint{4.569618in}{1.082247in}}%
\pgfpathlineto{\pgfqpoint{4.561462in}{1.076874in}}%
\pgfpathlineto{\pgfqpoint{4.553302in}{1.071852in}}%
\pgfpathlineto{\pgfqpoint{4.538661in}{1.076025in}}%
\pgfpathlineto{\pgfqpoint{4.524029in}{1.080288in}}%
\pgfpathlineto{\pgfqpoint{4.509405in}{1.084644in}}%
\pgfpathlineto{\pgfqpoint{4.494789in}{1.089092in}}%
\pgfpathlineto{\pgfqpoint{4.502963in}{1.093424in}}%
\pgfpathlineto{\pgfqpoint{4.511132in}{1.098112in}}%
\pgfpathlineto{\pgfqpoint{4.519297in}{1.103149in}}%
\pgfpathlineto{\pgfqpoint{4.527457in}{1.108527in}}%
\pgfpathclose%
\pgfusepath{fill}%
\end{pgfscope}%
\begin{pgfscope}%
\pgfpathrectangle{\pgfqpoint{1.150000in}{0.150000in}}{\pgfqpoint{5.700000in}{5.700000in}}%
\pgfusepath{clip}%
\pgfsetbuttcap%
\pgfsetroundjoin%
\definecolor{currentfill}{rgb}{0.282910,0.105393,0.426902}%
\pgfsetfillcolor{currentfill}%
\pgfsetfillopacity{0.700000}%
\pgfsetlinewidth{0.000000pt}%
\definecolor{currentstroke}{rgb}{0.000000,0.000000,0.000000}%
\pgfsetstrokecolor{currentstroke}%
\pgfsetdash{}{0pt}%
\pgfpathmoveto{\pgfqpoint{4.677034in}{1.111079in}}%
\pgfpathlineto{\pgfqpoint{4.691697in}{1.108712in}}%
\pgfpathlineto{\pgfqpoint{4.706370in}{1.106436in}}%
\pgfpathlineto{\pgfqpoint{4.721052in}{1.104253in}}%
\pgfpathlineto{\pgfqpoint{4.735743in}{1.102161in}}%
\pgfpathlineto{\pgfqpoint{4.727625in}{1.093514in}}%
\pgfpathlineto{\pgfqpoint{4.719504in}{1.085167in}}%
\pgfpathlineto{\pgfqpoint{4.711381in}{1.077127in}}%
\pgfpathlineto{\pgfqpoint{4.703255in}{1.069401in}}%
\pgfpathlineto{\pgfqpoint{4.688556in}{1.072158in}}%
\pgfpathlineto{\pgfqpoint{4.673867in}{1.075006in}}%
\pgfpathlineto{\pgfqpoint{4.659187in}{1.077946in}}%
\pgfpathlineto{\pgfqpoint{4.644516in}{1.080977in}}%
\pgfpathlineto{\pgfqpoint{4.652650in}{1.088032in}}%
\pgfpathlineto{\pgfqpoint{4.660781in}{1.095405in}}%
\pgfpathlineto{\pgfqpoint{4.668909in}{1.103089in}}%
\pgfpathlineto{\pgfqpoint{4.677034in}{1.111079in}}%
\pgfpathclose%
\pgfusepath{fill}%
\end{pgfscope}%
\begin{pgfscope}%
\pgfpathrectangle{\pgfqpoint{1.150000in}{0.150000in}}{\pgfqpoint{5.700000in}{5.700000in}}%
\pgfusepath{clip}%
\pgfsetbuttcap%
\pgfsetroundjoin%
\definecolor{currentfill}{rgb}{0.283091,0.110553,0.431554}%
\pgfsetfillcolor{currentfill}%
\pgfsetfillopacity{0.700000}%
\pgfsetlinewidth{0.000000pt}%
\definecolor{currentstroke}{rgb}{0.000000,0.000000,0.000000}%
\pgfsetstrokecolor{currentstroke}%
\pgfsetdash{}{0pt}%
\pgfpathmoveto{\pgfqpoint{4.436399in}{1.107801in}}%
\pgfpathlineto{\pgfqpoint{4.450985in}{1.102985in}}%
\pgfpathlineto{\pgfqpoint{4.465579in}{1.098262in}}%
\pgfpathlineto{\pgfqpoint{4.480180in}{1.093631in}}%
\pgfpathlineto{\pgfqpoint{4.494789in}{1.089092in}}%
\pgfpathlineto{\pgfqpoint{4.486609in}{1.085123in}}%
\pgfpathlineto{\pgfqpoint{4.478425in}{1.081524in}}%
\pgfpathlineto{\pgfqpoint{4.470236in}{1.078303in}}%
\pgfpathlineto{\pgfqpoint{4.455615in}{1.083367in}}%
\pgfpathlineto{\pgfqpoint{4.441001in}{1.088523in}}%
\pgfpathlineto{\pgfqpoint{4.426395in}{1.093771in}}%
\pgfpathlineto{\pgfqpoint{4.411796in}{1.099112in}}%
\pgfpathlineto{\pgfqpoint{4.420003in}{1.101628in}}%
\pgfpathlineto{\pgfqpoint{4.428204in}{1.104527in}}%
\pgfpathlineto{\pgfqpoint{4.436399in}{1.107801in}}%
\pgfpathclose%
\pgfusepath{fill}%
\end{pgfscope}%
\begin{pgfscope}%
\pgfpathrectangle{\pgfqpoint{1.150000in}{0.150000in}}{\pgfqpoint{5.700000in}{5.700000in}}%
\pgfusepath{clip}%
\pgfsetbuttcap%
\pgfsetroundjoin%
\definecolor{currentfill}{rgb}{0.282910,0.105393,0.426902}%
\pgfsetfillcolor{currentfill}%
\pgfsetfillopacity{0.700000}%
\pgfsetlinewidth{0.000000pt}%
\definecolor{currentstroke}{rgb}{0.000000,0.000000,0.000000}%
\pgfsetstrokecolor{currentstroke}%
\pgfsetdash{}{0pt}%
\pgfpathmoveto{\pgfqpoint{4.585919in}{1.094018in}}%
\pgfpathlineto{\pgfqpoint{4.600555in}{1.090620in}}%
\pgfpathlineto{\pgfqpoint{4.615200in}{1.087314in}}%
\pgfpathlineto{\pgfqpoint{4.629853in}{1.084100in}}%
\pgfpathlineto{\pgfqpoint{4.644516in}{1.080977in}}%
\pgfpathlineto{\pgfqpoint{4.636378in}{1.074248in}}%
\pgfpathlineto{\pgfqpoint{4.628237in}{1.067850in}}%
\pgfpathlineto{\pgfqpoint{4.620092in}{1.061792in}}%
\pgfpathlineto{\pgfqpoint{4.611944in}{1.056080in}}%
\pgfpathlineto{\pgfqpoint{4.597271in}{1.059886in}}%
\pgfpathlineto{\pgfqpoint{4.582606in}{1.063783in}}%
\pgfpathlineto{\pgfqpoint{4.567950in}{1.067772in}}%
\pgfpathlineto{\pgfqpoint{4.553302in}{1.071852in}}%
\pgfpathlineto{\pgfqpoint{4.561462in}{1.076874in}}%
\pgfpathlineto{\pgfqpoint{4.569618in}{1.082247in}}%
\pgfpathlineto{\pgfqpoint{4.577770in}{1.087964in}}%
\pgfpathlineto{\pgfqpoint{4.585919in}{1.094018in}}%
\pgfpathclose%
\pgfusepath{fill}%
\end{pgfscope}%
\begin{pgfscope}%
\pgfpathrectangle{\pgfqpoint{1.150000in}{0.150000in}}{\pgfqpoint{5.700000in}{5.700000in}}%
\pgfusepath{clip}%
\pgfsetbuttcap%
\pgfsetroundjoin%
\definecolor{currentfill}{rgb}{0.283091,0.110553,0.431554}%
\pgfsetfillcolor{currentfill}%
\pgfsetfillopacity{0.700000}%
\pgfsetlinewidth{0.000000pt}%
\definecolor{currentstroke}{rgb}{0.000000,0.000000,0.000000}%
\pgfsetstrokecolor{currentstroke}%
\pgfsetdash{}{0pt}%
\pgfpathmoveto{\pgfqpoint{4.735743in}{1.102161in}}%
\pgfpathlineto{\pgfqpoint{4.750444in}{1.100160in}}%
\pgfpathlineto{\pgfqpoint{4.765155in}{1.098252in}}%
\pgfpathlineto{\pgfqpoint{4.779875in}{1.096435in}}%
\pgfpathlineto{\pgfqpoint{4.771761in}{1.087294in}}%
\pgfpathlineto{\pgfqpoint{4.763645in}{1.078450in}}%
\pgfpathlineto{\pgfqpoint{4.755526in}{1.069910in}}%
\pgfpathlineto{\pgfqpoint{4.747405in}{1.061679in}}%
\pgfpathlineto{\pgfqpoint{4.732679in}{1.064162in}}%
\pgfpathlineto{\pgfqpoint{4.717962in}{1.066736in}}%
\pgfpathlineto{\pgfqpoint{4.703255in}{1.069401in}}%
\pgfpathlineto{\pgfqpoint{4.711381in}{1.077127in}}%
\pgfpathlineto{\pgfqpoint{4.719504in}{1.085167in}}%
\pgfpathlineto{\pgfqpoint{4.727625in}{1.093514in}}%
\pgfpathlineto{\pgfqpoint{4.735743in}{1.102161in}}%
\pgfpathclose%
\pgfusepath{fill}%
\end{pgfscope}%
\begin{pgfscope}%
\pgfpathrectangle{\pgfqpoint{1.150000in}{0.150000in}}{\pgfqpoint{5.700000in}{5.700000in}}%
\pgfusepath{clip}%
\pgfsetbuttcap%
\pgfsetroundjoin%
\definecolor{currentfill}{rgb}{0.283091,0.110553,0.431554}%
\pgfsetfillcolor{currentfill}%
\pgfsetfillopacity{0.700000}%
\pgfsetlinewidth{0.000000pt}%
\definecolor{currentstroke}{rgb}{0.000000,0.000000,0.000000}%
\pgfsetstrokecolor{currentstroke}%
\pgfsetdash{}{0pt}%
\pgfpathmoveto{\pgfqpoint{4.494789in}{1.089092in}}%
\pgfpathlineto{\pgfqpoint{4.509405in}{1.084644in}}%
\pgfpathlineto{\pgfqpoint{4.524029in}{1.080288in}}%
\pgfpathlineto{\pgfqpoint{4.538661in}{1.076025in}}%
\pgfpathlineto{\pgfqpoint{4.553302in}{1.071852in}}%
\pgfpathlineto{\pgfqpoint{4.545137in}{1.067189in}}%
\pgfpathlineto{\pgfqpoint{4.536968in}{1.062890in}}%
\pgfpathlineto{\pgfqpoint{4.528794in}{1.058965in}}%
\pgfpathlineto{\pgfqpoint{4.514143in}{1.063662in}}%
\pgfpathlineto{\pgfqpoint{4.499500in}{1.068450in}}%
\pgfpathlineto{\pgfqpoint{4.484864in}{1.073331in}}%
\pgfpathlineto{\pgfqpoint{4.470236in}{1.078303in}}%
\pgfpathlineto{\pgfqpoint{4.478425in}{1.081524in}}%
\pgfpathlineto{\pgfqpoint{4.486609in}{1.085123in}}%
\pgfpathlineto{\pgfqpoint{4.494789in}{1.089092in}}%
\pgfpathclose%
\pgfusepath{fill}%
\end{pgfscope}%
\begin{pgfscope}%
\pgfpathrectangle{\pgfqpoint{1.150000in}{0.150000in}}{\pgfqpoint{5.700000in}{5.700000in}}%
\pgfusepath{clip}%
\pgfsetbuttcap%
\pgfsetroundjoin%
\definecolor{currentfill}{rgb}{0.282910,0.105393,0.426902}%
\pgfsetfillcolor{currentfill}%
\pgfsetfillopacity{0.700000}%
\pgfsetlinewidth{0.000000pt}%
\definecolor{currentstroke}{rgb}{0.000000,0.000000,0.000000}%
\pgfsetstrokecolor{currentstroke}%
\pgfsetdash{}{0pt}%
\pgfpathmoveto{\pgfqpoint{4.644516in}{1.080977in}}%
\pgfpathlineto{\pgfqpoint{4.659187in}{1.077946in}}%
\pgfpathlineto{\pgfqpoint{4.673867in}{1.075006in}}%
\pgfpathlineto{\pgfqpoint{4.688556in}{1.072158in}}%
\pgfpathlineto{\pgfqpoint{4.703255in}{1.069401in}}%
\pgfpathlineto{\pgfqpoint{4.695126in}{1.061995in}}%
\pgfpathlineto{\pgfqpoint{4.686994in}{1.054917in}}%
\pgfpathlineto{\pgfqpoint{4.678859in}{1.048172in}}%
\pgfpathlineto{\pgfqpoint{4.670721in}{1.041769in}}%
\pgfpathlineto{\pgfqpoint{4.656014in}{1.045210in}}%
\pgfpathlineto{\pgfqpoint{4.641315in}{1.048742in}}%
\pgfpathlineto{\pgfqpoint{4.626625in}{1.052365in}}%
\pgfpathlineto{\pgfqpoint{4.611944in}{1.056080in}}%
\pgfpathlineto{\pgfqpoint{4.620092in}{1.061792in}}%
\pgfpathlineto{\pgfqpoint{4.628237in}{1.067850in}}%
\pgfpathlineto{\pgfqpoint{4.636378in}{1.074248in}}%
\pgfpathlineto{\pgfqpoint{4.644516in}{1.080977in}}%
\pgfpathclose%
\pgfusepath{fill}%
\end{pgfscope}%
\begin{pgfscope}%
\pgfpathrectangle{\pgfqpoint{1.150000in}{0.150000in}}{\pgfqpoint{5.700000in}{5.700000in}}%
\pgfusepath{clip}%
\pgfsetbuttcap%
\pgfsetroundjoin%
\definecolor{currentfill}{rgb}{0.283091,0.110553,0.431554}%
\pgfsetfillcolor{currentfill}%
\pgfsetfillopacity{0.700000}%
\pgfsetlinewidth{0.000000pt}%
\definecolor{currentstroke}{rgb}{0.000000,0.000000,0.000000}%
\pgfsetstrokecolor{currentstroke}%
\pgfsetdash{}{0pt}%
\pgfpathmoveto{\pgfqpoint{4.553302in}{1.071852in}}%
\pgfpathlineto{\pgfqpoint{4.567950in}{1.067772in}}%
\pgfpathlineto{\pgfqpoint{4.582606in}{1.063783in}}%
\pgfpathlineto{\pgfqpoint{4.597271in}{1.059886in}}%
\pgfpathlineto{\pgfqpoint{4.611944in}{1.056080in}}%
\pgfpathlineto{\pgfqpoint{4.603792in}{1.050721in}}%
\pgfpathlineto{\pgfqpoint{4.595636in}{1.045722in}}%
\pgfpathlineto{\pgfqpoint{4.587476in}{1.041091in}}%
\pgfpathlineto{\pgfqpoint{4.572794in}{1.045423in}}%
\pgfpathlineto{\pgfqpoint{4.558119in}{1.049845in}}%
\pgfpathlineto{\pgfqpoint{4.543452in}{1.054359in}}%
\pgfpathlineto{\pgfqpoint{4.528794in}{1.058965in}}%
\pgfpathlineto{\pgfqpoint{4.536968in}{1.062890in}}%
\pgfpathlineto{\pgfqpoint{4.545137in}{1.067189in}}%
\pgfpathlineto{\pgfqpoint{4.553302in}{1.071852in}}%
\pgfpathclose%
\pgfusepath{fill}%
\end{pgfscope}%
\begin{pgfscope}%
\pgfpathrectangle{\pgfqpoint{1.150000in}{0.150000in}}{\pgfqpoint{5.700000in}{5.700000in}}%
\pgfusepath{clip}%
\pgfsetbuttcap%
\pgfsetroundjoin%
\definecolor{currentfill}{rgb}{0.283091,0.110553,0.431554}%
\pgfsetfillcolor{currentfill}%
\pgfsetfillopacity{0.700000}%
\pgfsetlinewidth{0.000000pt}%
\definecolor{currentstroke}{rgb}{0.000000,0.000000,0.000000}%
\pgfsetstrokecolor{currentstroke}%
\pgfsetdash{}{0pt}%
\pgfpathmoveto{\pgfqpoint{4.703255in}{1.069401in}}%
\pgfpathlineto{\pgfqpoint{4.717962in}{1.066736in}}%
\pgfpathlineto{\pgfqpoint{4.732679in}{1.064162in}}%
\pgfpathlineto{\pgfqpoint{4.747405in}{1.061679in}}%
\pgfpathlineto{\pgfqpoint{4.739281in}{1.053766in}}%
\pgfpathlineto{\pgfqpoint{4.731155in}{1.046176in}}%
\pgfpathlineto{\pgfqpoint{4.723027in}{1.038916in}}%
\pgfpathlineto{\pgfqpoint{4.714896in}{1.031995in}}%
\pgfpathlineto{\pgfqpoint{4.700162in}{1.035162in}}%
\pgfpathlineto{\pgfqpoint{4.685437in}{1.038420in}}%
\pgfpathlineto{\pgfqpoint{4.670721in}{1.041769in}}%
\pgfpathlineto{\pgfqpoint{4.678859in}{1.048172in}}%
\pgfpathlineto{\pgfqpoint{4.686994in}{1.054917in}}%
\pgfpathlineto{\pgfqpoint{4.695126in}{1.061995in}}%
\pgfpathlineto{\pgfqpoint{4.703255in}{1.069401in}}%
\pgfpathclose%
\pgfusepath{fill}%
\end{pgfscope}%
\begin{pgfscope}%
\pgfpathrectangle{\pgfqpoint{1.150000in}{0.150000in}}{\pgfqpoint{5.700000in}{5.700000in}}%
\pgfusepath{clip}%
\pgfsetbuttcap%
\pgfsetroundjoin%
\definecolor{currentfill}{rgb}{0.283091,0.110553,0.431554}%
\pgfsetfillcolor{currentfill}%
\pgfsetfillopacity{0.700000}%
\pgfsetlinewidth{0.000000pt}%
\definecolor{currentstroke}{rgb}{0.000000,0.000000,0.000000}%
\pgfsetstrokecolor{currentstroke}%
\pgfsetdash{}{0pt}%
\pgfpathmoveto{\pgfqpoint{4.611944in}{1.056080in}}%
\pgfpathlineto{\pgfqpoint{4.626625in}{1.052365in}}%
\pgfpathlineto{\pgfqpoint{4.641315in}{1.048742in}}%
\pgfpathlineto{\pgfqpoint{4.656014in}{1.045210in}}%
\pgfpathlineto{\pgfqpoint{4.670721in}{1.041769in}}%
\pgfpathlineto{\pgfqpoint{4.662580in}{1.035715in}}%
\pgfpathlineto{\pgfqpoint{4.654436in}{1.030016in}}%
\pgfpathlineto{\pgfqpoint{4.646288in}{1.024680in}}%
\pgfpathlineto{\pgfqpoint{4.631572in}{1.028646in}}%
\pgfpathlineto{\pgfqpoint{4.616865in}{1.032703in}}%
\pgfpathlineto{\pgfqpoint{4.602167in}{1.036852in}}%
\pgfpathlineto{\pgfqpoint{4.587476in}{1.041091in}}%
\pgfpathlineto{\pgfqpoint{4.595636in}{1.045722in}}%
\pgfpathlineto{\pgfqpoint{4.603792in}{1.050721in}}%
\pgfpathlineto{\pgfqpoint{4.611944in}{1.056080in}}%
\pgfpathclose%
\pgfusepath{fill}%
\end{pgfscope}%
\begin{pgfscope}%
\pgfpathrectangle{\pgfqpoint{1.150000in}{0.150000in}}{\pgfqpoint{5.700000in}{5.700000in}}%
\pgfusepath{clip}%
\pgfsetbuttcap%
\pgfsetroundjoin%
\definecolor{currentfill}{rgb}{0.283197,0.115680,0.436115}%
\pgfsetfillcolor{currentfill}%
\pgfsetfillopacity{0.700000}%
\pgfsetlinewidth{0.000000pt}%
\definecolor{currentstroke}{rgb}{0.000000,0.000000,0.000000}%
\pgfsetstrokecolor{currentstroke}%
\pgfsetdash{}{0pt}%
\pgfpathmoveto{\pgfqpoint{4.670721in}{1.041769in}}%
\pgfpathlineto{\pgfqpoint{4.685437in}{1.038420in}}%
\pgfpathlineto{\pgfqpoint{4.700162in}{1.035162in}}%
\pgfpathlineto{\pgfqpoint{4.714896in}{1.031995in}}%
\pgfpathlineto{\pgfqpoint{4.706762in}{1.025418in}}%
\pgfpathlineto{\pgfqpoint{4.698625in}{1.019193in}}%
\pgfpathlineto{\pgfqpoint{4.690485in}{1.013328in}}%
\pgfpathlineto{\pgfqpoint{4.675744in}{1.017021in}}%
\pgfpathlineto{\pgfqpoint{4.661012in}{1.020805in}}%
\pgfpathlineto{\pgfqpoint{4.646288in}{1.024680in}}%
\pgfpathlineto{\pgfqpoint{4.654436in}{1.030016in}}%
\pgfpathlineto{\pgfqpoint{4.662580in}{1.035715in}}%
\pgfpathlineto{\pgfqpoint{4.670721in}{1.041769in}}%
\pgfpathclose%
\pgfusepath{fill}%
\end{pgfscope}%
\end{pgfpicture}%
\makeatother%
\endgroup%
}
        \caption{3D graf funkcie}
        \label{fig:newton_vpravo}
    \end{subfigure}
    
    \label{fig:newton_komplet}
\end{figure}

\noindent Vidíme, že v tomto prípade metóda dosiahla požadovanú presnosť už po 3. iterácii. To potvrdzuje vysokú efektivitu Newtonovej metódy pre hladké konvexné funkcie, kedy skočí takmer priamo do minima.



\newpage
\noindent \textbf{Počiatočný bod} $x^{[0]} = [1.5; 0.5]$
\vspace{0.5cm}

Skúsme bod, ktorý je ďalej od minima a kde funkcia rastie vplyvom členu $e^y$.

\begin{table}[h!]
    \centering
    \begin{tabular}{cccc}
        \toprule
        \textbf{Iterácia} & \textbf{Bod } $x^{[k]}$ & \textbf{Hodnota } $f(x^{[k]})$ & \textbf{Rozdiel } $|f_k - f_{k-1}|$ \\
        \midrule
        0 & $[1{,}5;\;0{,}5]$ & $4{,}934351$ & $0{,}028395$ \\
        1 & $[0{,}624720;\;1{,}165504]$ & $4{,}905957$ & $1{,}941525$ \\
        2 & $[-0{,}022409;\;0{,}598473]$ & $2{,}964432$ & $0{,}810767$ \\
        3 & $[-0{,}150650;\;0{,}109123]$ & $2{,}153665$ & $0{,}432282$ \\
        4 & $[-0{,}082210;\;-0{,}512321]$ & $1{,}721383$ & $0{,}143755$ \\
        5 & $[0{,}268582;\;-0{,}662316]$ & $1{,}577628$ & $0{,}008577$ \\
        6 & $[0{,}379564;\;-0{,}738632]$ & $1{,}569051$ & $0{,}000055$ \\
        \bottomrule
    \end{tabular}
    \caption{Priebeh NM pre $x^{[0]} = [1{,}5;\;0{,}5]$.}
\end{table}



Napriek tomu, že sme zvolili počiatočný bod v oblasti s výraznejším stúpaním funkcie, metóda našla minimum pomerne hladko. Počet iterácií sa síce zvýšil na šesť, čo je dvojnásobok oproti predchádzajúcemu prípadu, no vzhľadom na vzdialenosť bodu ide stále o veľmi rýchlu konvergenciu. Grafy potvrdzujú, že algoritmus dokázal efektívne korigovať smer a padnúť do optima aj z tejto nevýhodnejšej pozície. 


\begin{figure}[H]
    \centering
    
    % --- ĽAVÝ OBRÁZOK ---
    \begin{subfigure}[b]{0.48\textwidth}
        \centering
        \resizebox{\linewidth}{!}{%% Creator: Matplotlib, PGF backend
%%
%% To include the figure in your LaTeX document, write
%%   \input{<filename>.pgf}
%%
%% Make sure the required packages are loaded in your preamble
%%   \usepackage{pgf}
%%
%% Also ensure that all the required font packages are loaded; for instance,
%% the lmodern package is sometimes necessary when using math font.
%%   \usepackage{lmodern}
%%
%% Figures using additional raster images can only be included by \input if
%% they are in the same directory as the main LaTeX file. For loading figures
%% from other directories you can use the `import` package
%%   \usepackage{import}
%%
%% and then include the figures with
%%   \import{<path to file>}{<filename>.pgf}
%%
%% Matplotlib used the following preamble
%%   
%%   \usepackage{fontspec}
%%   \setmainfont{DejaVuSerif.ttf}[Path=\detokenize{/home/radimek/Documents/projekt_mat_prog/mat_prog_kernel/lib/python3.12/site-packages/matplotlib/mpl-data/fonts/ttf/}]
%%   \setsansfont{DejaVuSans.ttf}[Path=\detokenize{/home/radimek/Documents/projekt_mat_prog/mat_prog_kernel/lib/python3.12/site-packages/matplotlib/mpl-data/fonts/ttf/}]
%%   \setmonofont{DejaVuSansMono.ttf}[Path=\detokenize{/home/radimek/Documents/projekt_mat_prog/mat_prog_kernel/lib/python3.12/site-packages/matplotlib/mpl-data/fonts/ttf/}]
%%   \makeatletter\@ifpackageloaded{underscore}{}{\usepackage[strings]{underscore}}\makeatother
%%
\begingroup%
\makeatletter%
\begin{pgfpicture}%
\pgfpathrectangle{\pgfpointorigin}{\pgfqpoint{8.000000in}{6.000000in}}%
\pgfusepath{use as bounding box, clip}%
\begin{pgfscope}%
\pgfsetbuttcap%
\pgfsetmiterjoin%
\definecolor{currentfill}{rgb}{1.000000,1.000000,1.000000}%
\pgfsetfillcolor{currentfill}%
\pgfsetlinewidth{0.000000pt}%
\definecolor{currentstroke}{rgb}{1.000000,1.000000,1.000000}%
\pgfsetstrokecolor{currentstroke}%
\pgfsetdash{}{0pt}%
\pgfpathmoveto{\pgfqpoint{0.000000in}{0.000000in}}%
\pgfpathlineto{\pgfqpoint{8.000000in}{0.000000in}}%
\pgfpathlineto{\pgfqpoint{8.000000in}{6.000000in}}%
\pgfpathlineto{\pgfqpoint{0.000000in}{6.000000in}}%
\pgfpathlineto{\pgfqpoint{0.000000in}{0.000000in}}%
\pgfpathclose%
\pgfusepath{fill}%
\end{pgfscope}%
\begin{pgfscope}%
\pgfsetbuttcap%
\pgfsetmiterjoin%
\definecolor{currentfill}{rgb}{1.000000,1.000000,1.000000}%
\pgfsetfillcolor{currentfill}%
\pgfsetlinewidth{0.000000pt}%
\definecolor{currentstroke}{rgb}{0.000000,0.000000,0.000000}%
\pgfsetstrokecolor{currentstroke}%
\pgfsetstrokeopacity{0.000000}%
\pgfsetdash{}{0pt}%
\pgfpathmoveto{\pgfqpoint{0.766095in}{0.571603in}}%
\pgfpathlineto{\pgfqpoint{7.739560in}{0.571603in}}%
\pgfpathlineto{\pgfqpoint{7.739560in}{5.797238in}}%
\pgfpathlineto{\pgfqpoint{0.766095in}{5.797238in}}%
\pgfpathlineto{\pgfqpoint{0.766095in}{0.571603in}}%
\pgfpathclose%
\pgfusepath{fill}%
\end{pgfscope}%
\begin{pgfscope}%
\pgfpathrectangle{\pgfqpoint{0.766095in}{0.571603in}}{\pgfqpoint{6.973465in}{5.225635in}}%
\pgfusepath{clip}%
\pgfsetbuttcap%
\pgfsetroundjoin%
\definecolor{currentfill}{rgb}{1.000000,0.000000,0.000000}%
\pgfsetfillcolor{currentfill}%
\pgfsetlinewidth{1.003750pt}%
\definecolor{currentstroke}{rgb}{1.000000,0.000000,0.000000}%
\pgfsetstrokecolor{currentstroke}%
\pgfsetdash{}{0pt}%
\pgfsys@defobject{currentmarker}{\pgfqpoint{-0.041667in}{-0.041667in}}{\pgfqpoint{0.041667in}{0.041667in}}{%
\pgfpathmoveto{\pgfqpoint{0.000000in}{-0.041667in}}%
\pgfpathcurveto{\pgfqpoint{0.011050in}{-0.041667in}}{\pgfqpoint{0.021649in}{-0.037276in}}{\pgfqpoint{0.029463in}{-0.029463in}}%
\pgfpathcurveto{\pgfqpoint{0.037276in}{-0.021649in}}{\pgfqpoint{0.041667in}{-0.011050in}}{\pgfqpoint{0.041667in}{0.000000in}}%
\pgfpathcurveto{\pgfqpoint{0.041667in}{0.011050in}}{\pgfqpoint{0.037276in}{0.021649in}}{\pgfqpoint{0.029463in}{0.029463in}}%
\pgfpathcurveto{\pgfqpoint{0.021649in}{0.037276in}}{\pgfqpoint{0.011050in}{0.041667in}}{\pgfqpoint{0.000000in}{0.041667in}}%
\pgfpathcurveto{\pgfqpoint{-0.011050in}{0.041667in}}{\pgfqpoint{-0.021649in}{0.037276in}}{\pgfqpoint{-0.029463in}{0.029463in}}%
\pgfpathcurveto{\pgfqpoint{-0.037276in}{0.021649in}}{\pgfqpoint{-0.041667in}{0.011050in}}{\pgfqpoint{-0.041667in}{0.000000in}}%
\pgfpathcurveto{\pgfqpoint{-0.041667in}{-0.011050in}}{\pgfqpoint{-0.037276in}{-0.021649in}}{\pgfqpoint{-0.029463in}{-0.029463in}}%
\pgfpathcurveto{\pgfqpoint{-0.021649in}{-0.037276in}}{\pgfqpoint{-0.011050in}{-0.041667in}}{\pgfqpoint{0.000000in}{-0.041667in}}%
\pgfpathlineto{\pgfqpoint{0.000000in}{-0.041667in}}%
\pgfpathclose%
\pgfusepath{stroke,fill}%
}%
\begin{pgfscope}%
\pgfsys@transformshift{6.577316in}{3.706984in}%
\pgfsys@useobject{currentmarker}{}%
\end{pgfscope}%
\begin{pgfscope}%
\pgfsys@transformshift{4.542738in}{5.098057in}%
\pgfsys@useobject{currentmarker}{}%
\end{pgfscope}%
\begin{pgfscope}%
\pgfsys@transformshift{3.038495in}{3.912818in}%
\pgfsys@useobject{currentmarker}{}%
\end{pgfscope}%
\begin{pgfscope}%
\pgfsys@transformshift{2.740398in}{2.889953in}%
\pgfsys@useobject{currentmarker}{}%
\end{pgfscope}%
\begin{pgfscope}%
\pgfsys@transformshift{2.899488in}{1.590976in}%
\pgfsys@useobject{currentmarker}{}%
\end{pgfscope}%
\begin{pgfscope}%
\pgfsys@transformshift{3.714898in}{1.277449in}%
\pgfsys@useobject{currentmarker}{}%
\end{pgfscope}%
\begin{pgfscope}%
\pgfsys@transformshift{3.972874in}{1.117929in}%
\pgfsys@useobject{currentmarker}{}%
\end{pgfscope}%
\end{pgfscope}%
\begin{pgfscope}%
\pgfsetbuttcap%
\pgfsetroundjoin%
\definecolor{currentfill}{rgb}{0.000000,0.000000,0.000000}%
\pgfsetfillcolor{currentfill}%
\pgfsetlinewidth{0.803000pt}%
\definecolor{currentstroke}{rgb}{0.000000,0.000000,0.000000}%
\pgfsetstrokecolor{currentstroke}%
\pgfsetdash{}{0pt}%
\pgfsys@defobject{currentmarker}{\pgfqpoint{0.000000in}{-0.048611in}}{\pgfqpoint{0.000000in}{0.000000in}}{%
\pgfpathmoveto{\pgfqpoint{0.000000in}{0.000000in}}%
\pgfpathlineto{\pgfqpoint{0.000000in}{-0.048611in}}%
\pgfusepath{stroke,fill}%
}%
\begin{pgfscope}%
\pgfsys@transformshift{0.766095in}{0.571603in}%
\pgfsys@useobject{currentmarker}{}%
\end{pgfscope}%
\end{pgfscope}%
\begin{pgfscope}%
\definecolor{textcolor}{rgb}{0.000000,0.000000,0.000000}%
\pgfsetstrokecolor{textcolor}%
\pgfsetfillcolor{textcolor}%
\pgftext[x=0.766095in,y=0.474381in,,top]{\color{textcolor}\sffamily\fontsize{10.000000}{12.000000}\selectfont \ensuremath{-}1.0}%
\end{pgfscope}%
\begin{pgfscope}%
\pgfsetbuttcap%
\pgfsetroundjoin%
\definecolor{currentfill}{rgb}{0.000000,0.000000,0.000000}%
\pgfsetfillcolor{currentfill}%
\pgfsetlinewidth{0.803000pt}%
\definecolor{currentstroke}{rgb}{0.000000,0.000000,0.000000}%
\pgfsetstrokecolor{currentstroke}%
\pgfsetdash{}{0pt}%
\pgfsys@defobject{currentmarker}{\pgfqpoint{0.000000in}{-0.048611in}}{\pgfqpoint{0.000000in}{0.000000in}}{%
\pgfpathmoveto{\pgfqpoint{0.000000in}{0.000000in}}%
\pgfpathlineto{\pgfqpoint{0.000000in}{-0.048611in}}%
\pgfusepath{stroke,fill}%
}%
\begin{pgfscope}%
\pgfsys@transformshift{1.928339in}{0.571603in}%
\pgfsys@useobject{currentmarker}{}%
\end{pgfscope}%
\end{pgfscope}%
\begin{pgfscope}%
\definecolor{textcolor}{rgb}{0.000000,0.000000,0.000000}%
\pgfsetstrokecolor{textcolor}%
\pgfsetfillcolor{textcolor}%
\pgftext[x=1.928339in,y=0.474381in,,top]{\color{textcolor}\sffamily\fontsize{10.000000}{12.000000}\selectfont \ensuremath{-}0.5}%
\end{pgfscope}%
\begin{pgfscope}%
\pgfsetbuttcap%
\pgfsetroundjoin%
\definecolor{currentfill}{rgb}{0.000000,0.000000,0.000000}%
\pgfsetfillcolor{currentfill}%
\pgfsetlinewidth{0.803000pt}%
\definecolor{currentstroke}{rgb}{0.000000,0.000000,0.000000}%
\pgfsetstrokecolor{currentstroke}%
\pgfsetdash{}{0pt}%
\pgfsys@defobject{currentmarker}{\pgfqpoint{0.000000in}{-0.048611in}}{\pgfqpoint{0.000000in}{0.000000in}}{%
\pgfpathmoveto{\pgfqpoint{0.000000in}{0.000000in}}%
\pgfpathlineto{\pgfqpoint{0.000000in}{-0.048611in}}%
\pgfusepath{stroke,fill}%
}%
\begin{pgfscope}%
\pgfsys@transformshift{3.090583in}{0.571603in}%
\pgfsys@useobject{currentmarker}{}%
\end{pgfscope}%
\end{pgfscope}%
\begin{pgfscope}%
\definecolor{textcolor}{rgb}{0.000000,0.000000,0.000000}%
\pgfsetstrokecolor{textcolor}%
\pgfsetfillcolor{textcolor}%
\pgftext[x=3.090583in,y=0.474381in,,top]{\color{textcolor}\sffamily\fontsize{10.000000}{12.000000}\selectfont 0.0}%
\end{pgfscope}%
\begin{pgfscope}%
\pgfsetbuttcap%
\pgfsetroundjoin%
\definecolor{currentfill}{rgb}{0.000000,0.000000,0.000000}%
\pgfsetfillcolor{currentfill}%
\pgfsetlinewidth{0.803000pt}%
\definecolor{currentstroke}{rgb}{0.000000,0.000000,0.000000}%
\pgfsetstrokecolor{currentstroke}%
\pgfsetdash{}{0pt}%
\pgfsys@defobject{currentmarker}{\pgfqpoint{0.000000in}{-0.048611in}}{\pgfqpoint{0.000000in}{0.000000in}}{%
\pgfpathmoveto{\pgfqpoint{0.000000in}{0.000000in}}%
\pgfpathlineto{\pgfqpoint{0.000000in}{-0.048611in}}%
\pgfusepath{stroke,fill}%
}%
\begin{pgfscope}%
\pgfsys@transformshift{4.252828in}{0.571603in}%
\pgfsys@useobject{currentmarker}{}%
\end{pgfscope}%
\end{pgfscope}%
\begin{pgfscope}%
\definecolor{textcolor}{rgb}{0.000000,0.000000,0.000000}%
\pgfsetstrokecolor{textcolor}%
\pgfsetfillcolor{textcolor}%
\pgftext[x=4.252828in,y=0.474381in,,top]{\color{textcolor}\sffamily\fontsize{10.000000}{12.000000}\selectfont 0.5}%
\end{pgfscope}%
\begin{pgfscope}%
\pgfsetbuttcap%
\pgfsetroundjoin%
\definecolor{currentfill}{rgb}{0.000000,0.000000,0.000000}%
\pgfsetfillcolor{currentfill}%
\pgfsetlinewidth{0.803000pt}%
\definecolor{currentstroke}{rgb}{0.000000,0.000000,0.000000}%
\pgfsetstrokecolor{currentstroke}%
\pgfsetdash{}{0pt}%
\pgfsys@defobject{currentmarker}{\pgfqpoint{0.000000in}{-0.048611in}}{\pgfqpoint{0.000000in}{0.000000in}}{%
\pgfpathmoveto{\pgfqpoint{0.000000in}{0.000000in}}%
\pgfpathlineto{\pgfqpoint{0.000000in}{-0.048611in}}%
\pgfusepath{stroke,fill}%
}%
\begin{pgfscope}%
\pgfsys@transformshift{5.415072in}{0.571603in}%
\pgfsys@useobject{currentmarker}{}%
\end{pgfscope}%
\end{pgfscope}%
\begin{pgfscope}%
\definecolor{textcolor}{rgb}{0.000000,0.000000,0.000000}%
\pgfsetstrokecolor{textcolor}%
\pgfsetfillcolor{textcolor}%
\pgftext[x=5.415072in,y=0.474381in,,top]{\color{textcolor}\sffamily\fontsize{10.000000}{12.000000}\selectfont 1.0}%
\end{pgfscope}%
\begin{pgfscope}%
\pgfsetbuttcap%
\pgfsetroundjoin%
\definecolor{currentfill}{rgb}{0.000000,0.000000,0.000000}%
\pgfsetfillcolor{currentfill}%
\pgfsetlinewidth{0.803000pt}%
\definecolor{currentstroke}{rgb}{0.000000,0.000000,0.000000}%
\pgfsetstrokecolor{currentstroke}%
\pgfsetdash{}{0pt}%
\pgfsys@defobject{currentmarker}{\pgfqpoint{0.000000in}{-0.048611in}}{\pgfqpoint{0.000000in}{0.000000in}}{%
\pgfpathmoveto{\pgfqpoint{0.000000in}{0.000000in}}%
\pgfpathlineto{\pgfqpoint{0.000000in}{-0.048611in}}%
\pgfusepath{stroke,fill}%
}%
\begin{pgfscope}%
\pgfsys@transformshift{6.577316in}{0.571603in}%
\pgfsys@useobject{currentmarker}{}%
\end{pgfscope}%
\end{pgfscope}%
\begin{pgfscope}%
\definecolor{textcolor}{rgb}{0.000000,0.000000,0.000000}%
\pgfsetstrokecolor{textcolor}%
\pgfsetfillcolor{textcolor}%
\pgftext[x=6.577316in,y=0.474381in,,top]{\color{textcolor}\sffamily\fontsize{10.000000}{12.000000}\selectfont 1.5}%
\end{pgfscope}%
\begin{pgfscope}%
\pgfsetbuttcap%
\pgfsetroundjoin%
\definecolor{currentfill}{rgb}{0.000000,0.000000,0.000000}%
\pgfsetfillcolor{currentfill}%
\pgfsetlinewidth{0.803000pt}%
\definecolor{currentstroke}{rgb}{0.000000,0.000000,0.000000}%
\pgfsetstrokecolor{currentstroke}%
\pgfsetdash{}{0pt}%
\pgfsys@defobject{currentmarker}{\pgfqpoint{0.000000in}{-0.048611in}}{\pgfqpoint{0.000000in}{0.000000in}}{%
\pgfpathmoveto{\pgfqpoint{0.000000in}{0.000000in}}%
\pgfpathlineto{\pgfqpoint{0.000000in}{-0.048611in}}%
\pgfusepath{stroke,fill}%
}%
\begin{pgfscope}%
\pgfsys@transformshift{7.739560in}{0.571603in}%
\pgfsys@useobject{currentmarker}{}%
\end{pgfscope}%
\end{pgfscope}%
\begin{pgfscope}%
\definecolor{textcolor}{rgb}{0.000000,0.000000,0.000000}%
\pgfsetstrokecolor{textcolor}%
\pgfsetfillcolor{textcolor}%
\pgftext[x=7.739560in,y=0.474381in,,top]{\color{textcolor}\sffamily\fontsize{10.000000}{12.000000}\selectfont 2.0}%
\end{pgfscope}%
\begin{pgfscope}%
\definecolor{textcolor}{rgb}{0.000000,0.000000,0.000000}%
\pgfsetstrokecolor{textcolor}%
\pgfsetfillcolor{textcolor}%
\pgftext[x=4.252828in,y=0.284413in,,top]{\color{textcolor}\sffamily\fontsize{10.000000}{12.000000}\selectfont x}%
\end{pgfscope}%
\begin{pgfscope}%
\pgfsetbuttcap%
\pgfsetroundjoin%
\definecolor{currentfill}{rgb}{0.000000,0.000000,0.000000}%
\pgfsetfillcolor{currentfill}%
\pgfsetlinewidth{0.803000pt}%
\definecolor{currentstroke}{rgb}{0.000000,0.000000,0.000000}%
\pgfsetstrokecolor{currentstroke}%
\pgfsetdash{}{0pt}%
\pgfsys@defobject{currentmarker}{\pgfqpoint{-0.048611in}{0.000000in}}{\pgfqpoint{-0.000000in}{0.000000in}}{%
\pgfpathmoveto{\pgfqpoint{-0.000000in}{0.000000in}}%
\pgfpathlineto{\pgfqpoint{-0.048611in}{0.000000in}}%
\pgfusepath{stroke,fill}%
}%
\begin{pgfscope}%
\pgfsys@transformshift{0.766095in}{0.571603in}%
\pgfsys@useobject{currentmarker}{}%
\end{pgfscope}%
\end{pgfscope}%
\begin{pgfscope}%
\definecolor{textcolor}{rgb}{0.000000,0.000000,0.000000}%
\pgfsetstrokecolor{textcolor}%
\pgfsetfillcolor{textcolor}%
\pgftext[x=0.339968in, y=0.518842in, left, base]{\color{textcolor}\sffamily\fontsize{10.000000}{12.000000}\selectfont \ensuremath{-}1.0}%
\end{pgfscope}%
\begin{pgfscope}%
\pgfsetbuttcap%
\pgfsetroundjoin%
\definecolor{currentfill}{rgb}{0.000000,0.000000,0.000000}%
\pgfsetfillcolor{currentfill}%
\pgfsetlinewidth{0.803000pt}%
\definecolor{currentstroke}{rgb}{0.000000,0.000000,0.000000}%
\pgfsetstrokecolor{currentstroke}%
\pgfsetdash{}{0pt}%
\pgfsys@defobject{currentmarker}{\pgfqpoint{-0.048611in}{0.000000in}}{\pgfqpoint{-0.000000in}{0.000000in}}{%
\pgfpathmoveto{\pgfqpoint{-0.000000in}{0.000000in}}%
\pgfpathlineto{\pgfqpoint{-0.048611in}{0.000000in}}%
\pgfusepath{stroke,fill}%
}%
\begin{pgfscope}%
\pgfsys@transformshift{0.766095in}{1.616730in}%
\pgfsys@useobject{currentmarker}{}%
\end{pgfscope}%
\end{pgfscope}%
\begin{pgfscope}%
\definecolor{textcolor}{rgb}{0.000000,0.000000,0.000000}%
\pgfsetstrokecolor{textcolor}%
\pgfsetfillcolor{textcolor}%
\pgftext[x=0.339968in, y=1.563969in, left, base]{\color{textcolor}\sffamily\fontsize{10.000000}{12.000000}\selectfont \ensuremath{-}0.5}%
\end{pgfscope}%
\begin{pgfscope}%
\pgfsetbuttcap%
\pgfsetroundjoin%
\definecolor{currentfill}{rgb}{0.000000,0.000000,0.000000}%
\pgfsetfillcolor{currentfill}%
\pgfsetlinewidth{0.803000pt}%
\definecolor{currentstroke}{rgb}{0.000000,0.000000,0.000000}%
\pgfsetstrokecolor{currentstroke}%
\pgfsetdash{}{0pt}%
\pgfsys@defobject{currentmarker}{\pgfqpoint{-0.048611in}{0.000000in}}{\pgfqpoint{-0.000000in}{0.000000in}}{%
\pgfpathmoveto{\pgfqpoint{-0.000000in}{0.000000in}}%
\pgfpathlineto{\pgfqpoint{-0.048611in}{0.000000in}}%
\pgfusepath{stroke,fill}%
}%
\begin{pgfscope}%
\pgfsys@transformshift{0.766095in}{2.661857in}%
\pgfsys@useobject{currentmarker}{}%
\end{pgfscope}%
\end{pgfscope}%
\begin{pgfscope}%
\definecolor{textcolor}{rgb}{0.000000,0.000000,0.000000}%
\pgfsetstrokecolor{textcolor}%
\pgfsetfillcolor{textcolor}%
\pgftext[x=0.447993in, y=2.609096in, left, base]{\color{textcolor}\sffamily\fontsize{10.000000}{12.000000}\selectfont 0.0}%
\end{pgfscope}%
\begin{pgfscope}%
\pgfsetbuttcap%
\pgfsetroundjoin%
\definecolor{currentfill}{rgb}{0.000000,0.000000,0.000000}%
\pgfsetfillcolor{currentfill}%
\pgfsetlinewidth{0.803000pt}%
\definecolor{currentstroke}{rgb}{0.000000,0.000000,0.000000}%
\pgfsetstrokecolor{currentstroke}%
\pgfsetdash{}{0pt}%
\pgfsys@defobject{currentmarker}{\pgfqpoint{-0.048611in}{0.000000in}}{\pgfqpoint{-0.000000in}{0.000000in}}{%
\pgfpathmoveto{\pgfqpoint{-0.000000in}{0.000000in}}%
\pgfpathlineto{\pgfqpoint{-0.048611in}{0.000000in}}%
\pgfusepath{stroke,fill}%
}%
\begin{pgfscope}%
\pgfsys@transformshift{0.766095in}{3.706984in}%
\pgfsys@useobject{currentmarker}{}%
\end{pgfscope}%
\end{pgfscope}%
\begin{pgfscope}%
\definecolor{textcolor}{rgb}{0.000000,0.000000,0.000000}%
\pgfsetstrokecolor{textcolor}%
\pgfsetfillcolor{textcolor}%
\pgftext[x=0.447993in, y=3.654223in, left, base]{\color{textcolor}\sffamily\fontsize{10.000000}{12.000000}\selectfont 0.5}%
\end{pgfscope}%
\begin{pgfscope}%
\pgfsetbuttcap%
\pgfsetroundjoin%
\definecolor{currentfill}{rgb}{0.000000,0.000000,0.000000}%
\pgfsetfillcolor{currentfill}%
\pgfsetlinewidth{0.803000pt}%
\definecolor{currentstroke}{rgb}{0.000000,0.000000,0.000000}%
\pgfsetstrokecolor{currentstroke}%
\pgfsetdash{}{0pt}%
\pgfsys@defobject{currentmarker}{\pgfqpoint{-0.048611in}{0.000000in}}{\pgfqpoint{-0.000000in}{0.000000in}}{%
\pgfpathmoveto{\pgfqpoint{-0.000000in}{0.000000in}}%
\pgfpathlineto{\pgfqpoint{-0.048611in}{0.000000in}}%
\pgfusepath{stroke,fill}%
}%
\begin{pgfscope}%
\pgfsys@transformshift{0.766095in}{4.752111in}%
\pgfsys@useobject{currentmarker}{}%
\end{pgfscope}%
\end{pgfscope}%
\begin{pgfscope}%
\definecolor{textcolor}{rgb}{0.000000,0.000000,0.000000}%
\pgfsetstrokecolor{textcolor}%
\pgfsetfillcolor{textcolor}%
\pgftext[x=0.447993in, y=4.699350in, left, base]{\color{textcolor}\sffamily\fontsize{10.000000}{12.000000}\selectfont 1.0}%
\end{pgfscope}%
\begin{pgfscope}%
\pgfsetbuttcap%
\pgfsetroundjoin%
\definecolor{currentfill}{rgb}{0.000000,0.000000,0.000000}%
\pgfsetfillcolor{currentfill}%
\pgfsetlinewidth{0.803000pt}%
\definecolor{currentstroke}{rgb}{0.000000,0.000000,0.000000}%
\pgfsetstrokecolor{currentstroke}%
\pgfsetdash{}{0pt}%
\pgfsys@defobject{currentmarker}{\pgfqpoint{-0.048611in}{0.000000in}}{\pgfqpoint{-0.000000in}{0.000000in}}{%
\pgfpathmoveto{\pgfqpoint{-0.000000in}{0.000000in}}%
\pgfpathlineto{\pgfqpoint{-0.048611in}{0.000000in}}%
\pgfusepath{stroke,fill}%
}%
\begin{pgfscope}%
\pgfsys@transformshift{0.766095in}{5.797238in}%
\pgfsys@useobject{currentmarker}{}%
\end{pgfscope}%
\end{pgfscope}%
\begin{pgfscope}%
\definecolor{textcolor}{rgb}{0.000000,0.000000,0.000000}%
\pgfsetstrokecolor{textcolor}%
\pgfsetfillcolor{textcolor}%
\pgftext[x=0.447993in, y=5.744477in, left, base]{\color{textcolor}\sffamily\fontsize{10.000000}{12.000000}\selectfont 1.5}%
\end{pgfscope}%
\begin{pgfscope}%
\definecolor{textcolor}{rgb}{0.000000,0.000000,0.000000}%
\pgfsetstrokecolor{textcolor}%
\pgfsetfillcolor{textcolor}%
\pgftext[x=0.284413in,y=3.184421in,,bottom,rotate=90.000000]{\color{textcolor}\sffamily\fontsize{10.000000}{12.000000}\selectfont y}%
\end{pgfscope}%
\begin{pgfscope}%
\pgfpathrectangle{\pgfqpoint{0.766095in}{0.571603in}}{\pgfqpoint{6.973465in}{5.225635in}}%
\pgfusepath{clip}%
\pgfsetbuttcap%
\pgfsetroundjoin%
\pgfsetlinewidth{1.505625pt}%
\definecolor{currentstroke}{rgb}{0.273809,0.031497,0.358853}%
\pgfsetstrokecolor{currentstroke}%
\pgfsetdash{}{0pt}%
\pgfpathmoveto{\pgfqpoint{4.015742in}{0.571603in}}%
\pgfpathlineto{\pgfqpoint{3.884843in}{0.624122in}}%
\pgfpathlineto{\pgfqpoint{3.764089in}{0.676641in}}%
\pgfpathlineto{\pgfqpoint{3.651938in}{0.729160in}}%
\pgfpathlineto{\pgfqpoint{3.547844in}{0.781679in}}%
\pgfpathlineto{\pgfqpoint{3.451176in}{0.834198in}}%
\pgfpathlineto{\pgfqpoint{3.359243in}{0.888000in}}%
\pgfpathlineto{\pgfqpoint{3.278241in}{0.939236in}}%
\pgfpathlineto{\pgfqpoint{3.201262in}{0.991755in}}%
\pgfpathlineto{\pgfqpoint{3.130161in}{1.044274in}}%
\pgfpathlineto{\pgfqpoint{3.064711in}{1.096793in}}%
\pgfpathlineto{\pgfqpoint{3.004622in}{1.149312in}}%
\pgfpathlineto{\pgfqpoint{2.949975in}{1.201831in}}%
\pgfpathlineto{\pgfqpoint{2.900283in}{1.254350in}}%
\pgfpathlineto{\pgfqpoint{2.855790in}{1.306869in}}%
\pgfpathlineto{\pgfqpoint{2.833605in}{1.335302in}}%
\pgfpathlineto{\pgfqpoint{2.797935in}{1.385647in}}%
\pgfpathlineto{\pgfqpoint{2.763520in}{1.441828in}}%
\pgfpathlineto{\pgfqpoint{2.738039in}{1.490685in}}%
\pgfpathlineto{\pgfqpoint{2.725941in}{1.516944in}}%
\pgfpathlineto{\pgfqpoint{2.705717in}{1.569463in}}%
\pgfpathlineto{\pgfqpoint{2.693435in}{1.609808in}}%
\pgfpathlineto{\pgfqpoint{2.690183in}{1.621982in}}%
\pgfpathlineto{\pgfqpoint{2.684450in}{1.648242in}}%
\pgfpathlineto{\pgfqpoint{2.679957in}{1.674501in}}%
\pgfpathlineto{\pgfqpoint{2.676754in}{1.700761in}}%
\pgfpathlineto{\pgfqpoint{2.674891in}{1.727020in}}%
\pgfpathlineto{\pgfqpoint{2.674419in}{1.753280in}}%
\pgfpathlineto{\pgfqpoint{2.675391in}{1.779539in}}%
\pgfpathlineto{\pgfqpoint{2.677859in}{1.805799in}}%
\pgfpathlineto{\pgfqpoint{2.681878in}{1.832058in}}%
\pgfpathlineto{\pgfqpoint{2.687499in}{1.858318in}}%
\pgfpathlineto{\pgfqpoint{2.694859in}{1.884577in}}%
\pgfpathlineto{\pgfqpoint{2.704413in}{1.910836in}}%
\pgfpathlineto{\pgfqpoint{2.715864in}{1.937096in}}%
\pgfpathlineto{\pgfqpoint{2.729330in}{1.963355in}}%
\pgfpathlineto{\pgfqpoint{2.745868in}{1.989615in}}%
\pgfpathlineto{\pgfqpoint{2.764774in}{2.015874in}}%
\pgfpathlineto{\pgfqpoint{2.798562in}{2.053503in}}%
\pgfpathlineto{\pgfqpoint{2.833605in}{2.084711in}}%
\pgfpathlineto{\pgfqpoint{2.868647in}{2.110029in}}%
\pgfpathlineto{\pgfqpoint{2.903690in}{2.130773in}}%
\pgfpathlineto{\pgfqpoint{2.938732in}{2.147836in}}%
\pgfpathlineto{\pgfqpoint{2.973775in}{2.161491in}}%
\pgfpathlineto{\pgfqpoint{3.012563in}{2.173431in}}%
\pgfpathlineto{\pgfqpoint{3.043860in}{2.180840in}}%
\pgfpathlineto{\pgfqpoint{3.078903in}{2.186921in}}%
\pgfpathlineto{\pgfqpoint{3.113945in}{2.190900in}}%
\pgfpathlineto{\pgfqpoint{3.148988in}{2.192899in}}%
\pgfpathlineto{\pgfqpoint{3.184030in}{2.193030in}}%
\pgfpathlineto{\pgfqpoint{3.219073in}{2.191399in}}%
\pgfpathlineto{\pgfqpoint{3.254115in}{2.188102in}}%
\pgfpathlineto{\pgfqpoint{3.289158in}{2.183231in}}%
\pgfpathlineto{\pgfqpoint{3.339566in}{2.173431in}}%
\pgfpathlineto{\pgfqpoint{3.359243in}{2.169035in}}%
\pgfpathlineto{\pgfqpoint{3.394285in}{2.159809in}}%
\pgfpathlineto{\pgfqpoint{3.435689in}{2.147172in}}%
\pgfpathlineto{\pgfqpoint{3.499413in}{2.124574in}}%
\pgfpathlineto{\pgfqpoint{3.570917in}{2.094653in}}%
\pgfpathlineto{\pgfqpoint{3.639583in}{2.061954in}}%
\pgfpathlineto{\pgfqpoint{3.709668in}{2.025015in}}%
\pgfpathlineto{\pgfqpoint{3.779753in}{1.984948in}}%
\pgfpathlineto{\pgfqpoint{3.857741in}{1.937096in}}%
\pgfpathlineto{\pgfqpoint{3.938635in}{1.884577in}}%
\pgfpathlineto{\pgfqpoint{4.016139in}{1.832058in}}%
\pgfpathlineto{\pgfqpoint{4.091160in}{1.779539in}}%
\pgfpathlineto{\pgfqpoint{4.165221in}{1.726399in}}%
\pgfpathlineto{\pgfqpoint{4.306988in}{1.621982in}}%
\pgfpathlineto{\pgfqpoint{4.481437in}{1.490685in}}%
\pgfpathlineto{\pgfqpoint{4.667105in}{1.349211in}}%
\pgfpathlineto{\pgfqpoint{4.667105in}{1.349211in}}%
\pgfusepath{stroke}%
\end{pgfscope}%
\begin{pgfscope}%
\pgfpathrectangle{\pgfqpoint{0.766095in}{0.571603in}}{\pgfqpoint{6.973465in}{5.225635in}}%
\pgfusepath{clip}%
\pgfsetbuttcap%
\pgfsetroundjoin%
\pgfsetlinewidth{1.505625pt}%
\definecolor{currentstroke}{rgb}{0.273809,0.031497,0.358853}%
\pgfsetstrokecolor{currentstroke}%
\pgfsetdash{}{0pt}%
\pgfpathmoveto{\pgfqpoint{4.914423in}{1.159220in}}%
\pgfpathlineto{\pgfqpoint{4.927255in}{1.149312in}}%
\pgfpathlineto{\pgfqpoint{4.936157in}{1.142428in}}%
\pgfpathlineto{\pgfqpoint{4.961208in}{1.123052in}}%
\pgfpathlineto{\pgfqpoint{4.971200in}{1.115305in}}%
\pgfpathlineto{\pgfqpoint{4.995086in}{1.096793in}}%
\pgfpathlineto{\pgfqpoint{5.006242in}{1.088116in}}%
\pgfpathlineto{\pgfqpoint{5.028875in}{1.070533in}}%
\pgfpathlineto{\pgfqpoint{5.041285in}{1.060849in}}%
\pgfpathlineto{\pgfqpoint{5.062563in}{1.044274in}}%
\pgfpathlineto{\pgfqpoint{5.076327in}{1.033492in}}%
\pgfpathlineto{\pgfqpoint{5.096135in}{1.018014in}}%
\pgfpathlineto{\pgfqpoint{5.111370in}{1.006029in}}%
\pgfpathlineto{\pgfqpoint{5.129572in}{0.991755in}}%
\pgfpathlineto{\pgfqpoint{5.146412in}{0.978443in}}%
\pgfpathlineto{\pgfqpoint{5.162855in}{0.965495in}}%
\pgfpathlineto{\pgfqpoint{5.181455in}{0.950713in}}%
\pgfpathlineto{\pgfqpoint{5.195963in}{0.939236in}}%
\pgfpathlineto{\pgfqpoint{5.216497in}{0.922817in}}%
\pgfpathlineto{\pgfqpoint{5.228870in}{0.912976in}}%
\pgfpathlineto{\pgfqpoint{5.251540in}{0.894727in}}%
\pgfpathlineto{\pgfqpoint{5.261550in}{0.886717in}}%
\pgfpathlineto{\pgfqpoint{5.286583in}{0.866411in}}%
\pgfpathlineto{\pgfqpoint{5.293972in}{0.860458in}}%
\pgfpathlineto{\pgfqpoint{5.321625in}{0.837834in}}%
\pgfpathlineto{\pgfqpoint{5.326103in}{0.834198in}}%
\pgfpathlineto{\pgfqpoint{5.356668in}{0.808954in}}%
\pgfpathlineto{\pgfqpoint{5.357907in}{0.807939in}}%
\pgfpathlineto{\pgfqpoint{5.389254in}{0.781679in}}%
\pgfpathlineto{\pgfqpoint{5.391710in}{0.779570in}}%
\pgfpathlineto{\pgfqpoint{5.420124in}{0.755420in}}%
\pgfpathlineto{\pgfqpoint{5.426753in}{0.749642in}}%
\pgfpathlineto{\pgfqpoint{5.450509in}{0.729160in}}%
\pgfpathlineto{\pgfqpoint{5.461795in}{0.719156in}}%
\pgfpathlineto{\pgfqpoint{5.480350in}{0.702901in}}%
\pgfpathlineto{\pgfqpoint{5.496838in}{0.688007in}}%
\pgfpathlineto{\pgfqpoint{5.509581in}{0.676641in}}%
\pgfpathlineto{\pgfqpoint{5.531880in}{0.656067in}}%
\pgfpathlineto{\pgfqpoint{5.538128in}{0.650382in}}%
\pgfpathlineto{\pgfqpoint{5.565866in}{0.624122in}}%
\pgfpathlineto{\pgfqpoint{5.566923in}{0.623062in}}%
\pgfpathlineto{\pgfqpoint{5.592451in}{0.597863in}}%
\pgfpathlineto{\pgfqpoint{5.601965in}{0.587985in}}%
\pgfpathlineto{\pgfqpoint{5.618021in}{0.571603in}}%
\pgfusepath{stroke}%
\end{pgfscope}%
\begin{pgfscope}%
\pgfpathrectangle{\pgfqpoint{0.766095in}{0.571603in}}{\pgfqpoint{6.973465in}{5.225635in}}%
\pgfusepath{clip}%
\pgfsetbuttcap%
\pgfsetroundjoin%
\pgfsetlinewidth{1.505625pt}%
\definecolor{currentstroke}{rgb}{0.278791,0.062145,0.386592}%
\pgfsetstrokecolor{currentstroke}%
\pgfsetdash{}{0pt}%
\pgfpathmoveto{\pgfqpoint{3.440985in}{0.571603in}}%
\pgfpathlineto{\pgfqpoint{3.324200in}{0.627344in}}%
\pgfpathlineto{\pgfqpoint{3.219073in}{0.680666in}}%
\pgfpathlineto{\pgfqpoint{3.128742in}{0.729160in}}%
\pgfpathlineto{\pgfqpoint{3.036406in}{0.781679in}}%
\pgfpathlineto{\pgfqpoint{2.938732in}{0.841063in}}%
\pgfpathlineto{\pgfqpoint{2.867903in}{0.886717in}}%
\pgfpathlineto{\pgfqpoint{2.791407in}{0.939236in}}%
\pgfpathlineto{\pgfqpoint{2.719704in}{0.991755in}}%
\pgfpathlineto{\pgfqpoint{2.652639in}{1.044274in}}%
\pgfpathlineto{\pgfqpoint{2.588307in}{1.098348in}}%
\pgfpathlineto{\pgfqpoint{2.531999in}{1.149312in}}%
\pgfpathlineto{\pgfqpoint{2.478014in}{1.201831in}}%
\pgfpathlineto{\pgfqpoint{2.428248in}{1.254350in}}%
\pgfpathlineto{\pgfqpoint{2.382348in}{1.306869in}}%
\pgfpathlineto{\pgfqpoint{2.340383in}{1.359388in}}%
\pgfpathlineto{\pgfqpoint{2.302262in}{1.411906in}}%
\pgfpathlineto{\pgfqpoint{2.267809in}{1.464425in}}%
\pgfpathlineto{\pgfqpoint{2.236943in}{1.516944in}}%
\pgfpathlineto{\pgfqpoint{2.209761in}{1.569463in}}%
\pgfpathlineto{\pgfqpoint{2.186070in}{1.621982in}}%
\pgfpathlineto{\pgfqpoint{2.165694in}{1.674501in}}%
\pgfpathlineto{\pgfqpoint{2.148953in}{1.727020in}}%
\pgfpathlineto{\pgfqpoint{2.135418in}{1.779539in}}%
\pgfpathlineto{\pgfqpoint{2.125428in}{1.832058in}}%
\pgfpathlineto{\pgfqpoint{2.118825in}{1.884577in}}%
\pgfpathlineto{\pgfqpoint{2.115601in}{1.937096in}}%
\pgfpathlineto{\pgfqpoint{2.115817in}{1.989615in}}%
\pgfpathlineto{\pgfqpoint{2.119525in}{2.042134in}}%
\pgfpathlineto{\pgfqpoint{2.126756in}{2.094653in}}%
\pgfpathlineto{\pgfqpoint{2.137704in}{2.147172in}}%
\pgfpathlineto{\pgfqpoint{2.152552in}{2.199691in}}%
\pgfpathlineto{\pgfqpoint{2.171190in}{2.252210in}}%
\pgfpathlineto{\pgfqpoint{2.194200in}{2.304729in}}%
\pgfpathlineto{\pgfqpoint{2.207277in}{2.330988in}}%
\pgfpathlineto{\pgfqpoint{2.237882in}{2.384919in}}%
\pgfpathlineto{\pgfqpoint{2.272924in}{2.437462in}}%
\pgfpathlineto{\pgfqpoint{2.312508in}{2.488545in}}%
\pgfpathlineto{\pgfqpoint{2.343009in}{2.523509in}}%
\pgfpathlineto{\pgfqpoint{2.385815in}{2.567323in}}%
\pgfpathlineto{\pgfqpoint{2.414010in}{2.593583in}}%
\pgfpathlineto{\pgfqpoint{2.448137in}{2.622684in}}%
\pgfpathlineto{\pgfqpoint{2.483179in}{2.650040in}}%
\pgfpathlineto{\pgfqpoint{2.518222in}{2.675133in}}%
\pgfpathlineto{\pgfqpoint{2.553995in}{2.698621in}}%
\pgfpathlineto{\pgfqpoint{2.598365in}{2.724880in}}%
\pgfpathlineto{\pgfqpoint{2.647996in}{2.751140in}}%
\pgfpathlineto{\pgfqpoint{2.658392in}{2.756334in}}%
\pgfpathlineto{\pgfqpoint{2.705012in}{2.777399in}}%
\pgfpathlineto{\pgfqpoint{2.728477in}{2.787161in}}%
\pgfpathlineto{\pgfqpoint{2.772900in}{2.803659in}}%
\pgfpathlineto{\pgfqpoint{2.798562in}{2.812303in}}%
\pgfpathlineto{\pgfqpoint{2.860281in}{2.829918in}}%
\pgfpathlineto{\pgfqpoint{2.868647in}{2.832105in}}%
\pgfpathlineto{\pgfqpoint{2.903690in}{2.840007in}}%
\pgfpathlineto{\pgfqpoint{2.938732in}{2.846670in}}%
\pgfpathlineto{\pgfqpoint{2.973775in}{2.852083in}}%
\pgfpathlineto{\pgfqpoint{3.008818in}{2.856235in}}%
\pgfpathlineto{\pgfqpoint{3.043860in}{2.859075in}}%
\pgfpathlineto{\pgfqpoint{3.078903in}{2.860653in}}%
\pgfpathlineto{\pgfqpoint{3.113945in}{2.860956in}}%
\pgfpathlineto{\pgfqpoint{3.148988in}{2.859969in}}%
\pgfpathlineto{\pgfqpoint{3.198625in}{2.856177in}}%
\pgfpathlineto{\pgfqpoint{3.219073in}{2.854041in}}%
\pgfpathlineto{\pgfqpoint{3.254115in}{2.849041in}}%
\pgfpathlineto{\pgfqpoint{3.289158in}{2.842683in}}%
\pgfpathlineto{\pgfqpoint{3.343591in}{2.829918in}}%
\pgfpathlineto{\pgfqpoint{3.359243in}{2.825800in}}%
\pgfpathlineto{\pgfqpoint{3.394285in}{2.815202in}}%
\pgfpathlineto{\pgfqpoint{3.429328in}{2.803190in}}%
\pgfpathlineto{\pgfqpoint{3.493192in}{2.777399in}}%
\pgfpathlineto{\pgfqpoint{3.499413in}{2.774727in}}%
\pgfpathlineto{\pgfqpoint{3.548508in}{2.751140in}}%
\pgfpathlineto{\pgfqpoint{3.597638in}{2.724880in}}%
\pgfpathlineto{\pgfqpoint{3.604541in}{2.721059in}}%
\pgfpathlineto{\pgfqpoint{3.642236in}{2.698621in}}%
\pgfpathlineto{\pgfqpoint{3.683249in}{2.672361in}}%
\pgfpathlineto{\pgfqpoint{3.744711in}{2.629876in}}%
\pgfpathlineto{\pgfqpoint{3.814796in}{2.576755in}}%
\pgfpathlineto{\pgfqpoint{3.890232in}{2.514804in}}%
\pgfpathlineto{\pgfqpoint{3.954966in}{2.458639in}}%
\pgfpathlineto{\pgfqpoint{4.060094in}{2.362727in}}%
\pgfpathlineto{\pgfqpoint{4.122432in}{2.303966in}}%
\pgfpathlineto{\pgfqpoint{4.122432in}{2.303966in}}%
\pgfusepath{stroke}%
\end{pgfscope}%
\begin{pgfscope}%
\pgfpathrectangle{\pgfqpoint{0.766095in}{0.571603in}}{\pgfqpoint{6.973465in}{5.225635in}}%
\pgfusepath{clip}%
\pgfsetbuttcap%
\pgfsetroundjoin%
\pgfsetlinewidth{1.505625pt}%
\definecolor{currentstroke}{rgb}{0.278791,0.062145,0.386592}%
\pgfsetstrokecolor{currentstroke}%
\pgfsetdash{}{0pt}%
\pgfpathmoveto{\pgfqpoint{4.347969in}{2.088304in}}%
\pgfpathlineto{\pgfqpoint{4.368798in}{2.068393in}}%
\pgfpathlineto{\pgfqpoint{4.375477in}{2.062120in}}%
\pgfpathlineto{\pgfqpoint{4.396379in}{2.042134in}}%
\pgfpathlineto{\pgfqpoint{4.410519in}{2.028855in}}%
\pgfpathlineto{\pgfqpoint{4.424109in}{2.015874in}}%
\pgfpathlineto{\pgfqpoint{4.445562in}{1.995749in}}%
\pgfpathlineto{\pgfqpoint{4.451996in}{1.989615in}}%
\pgfpathlineto{\pgfqpoint{4.480034in}{1.963355in}}%
\pgfpathlineto{\pgfqpoint{4.480604in}{1.962829in}}%
\pgfpathlineto{\pgfqpoint{4.508078in}{1.937096in}}%
\pgfpathlineto{\pgfqpoint{4.515647in}{1.930131in}}%
\pgfpathlineto{\pgfqpoint{4.536314in}{1.910836in}}%
\pgfpathlineto{\pgfqpoint{4.550689in}{1.897649in}}%
\pgfpathlineto{\pgfqpoint{4.564747in}{1.884577in}}%
\pgfpathlineto{\pgfqpoint{4.585732in}{1.865399in}}%
\pgfpathlineto{\pgfqpoint{4.593380in}{1.858318in}}%
\pgfpathlineto{\pgfqpoint{4.620774in}{1.833390in}}%
\pgfpathlineto{\pgfqpoint{4.622220in}{1.832058in}}%
\pgfpathlineto{\pgfqpoint{4.651151in}{1.805799in}}%
\pgfpathlineto{\pgfqpoint{4.655817in}{1.801631in}}%
\pgfpathlineto{\pgfqpoint{4.680267in}{1.779539in}}%
\pgfpathlineto{\pgfqpoint{4.690859in}{1.770124in}}%
\pgfpathlineto{\pgfqpoint{4.709606in}{1.753280in}}%
\pgfpathlineto{\pgfqpoint{4.725902in}{1.738871in}}%
\pgfpathlineto{\pgfqpoint{4.739170in}{1.727020in}}%
\pgfpathlineto{\pgfqpoint{4.760944in}{1.707875in}}%
\pgfpathlineto{\pgfqpoint{4.768959in}{1.700761in}}%
\pgfpathlineto{\pgfqpoint{4.795987in}{1.677137in}}%
\pgfpathlineto{\pgfqpoint{4.798976in}{1.674501in}}%
\pgfpathlineto{\pgfqpoint{4.829176in}{1.648242in}}%
\pgfpathlineto{\pgfqpoint{4.831030in}{1.646651in}}%
\pgfpathlineto{\pgfqpoint{4.859539in}{1.621982in}}%
\pgfpathlineto{\pgfqpoint{4.866072in}{1.616409in}}%
\pgfpathlineto{\pgfqpoint{4.890137in}{1.595723in}}%
\pgfpathlineto{\pgfqpoint{4.901115in}{1.586416in}}%
\pgfpathlineto{\pgfqpoint{4.920969in}{1.569463in}}%
\pgfpathlineto{\pgfqpoint{4.936157in}{1.556669in}}%
\pgfpathlineto{\pgfqpoint{4.952036in}{1.543204in}}%
\pgfpathlineto{\pgfqpoint{4.971200in}{1.527163in}}%
\pgfpathlineto{\pgfqpoint{4.983335in}{1.516944in}}%
\pgfpathlineto{\pgfqpoint{5.006242in}{1.497895in}}%
\pgfpathlineto{\pgfqpoint{5.014865in}{1.490685in}}%
\pgfpathlineto{\pgfqpoint{5.041285in}{1.468859in}}%
\pgfpathlineto{\pgfqpoint{5.046624in}{1.464425in}}%
\pgfpathlineto{\pgfqpoint{5.076327in}{1.440049in}}%
\pgfpathlineto{\pgfqpoint{5.078612in}{1.438166in}}%
\pgfpathlineto{\pgfqpoint{5.110812in}{1.411906in}}%
\pgfpathlineto{\pgfqpoint{5.111370in}{1.411455in}}%
\pgfpathlineto{\pgfqpoint{5.143188in}{1.385647in}}%
\pgfpathlineto{\pgfqpoint{5.146412in}{1.383058in}}%
\pgfpathlineto{\pgfqpoint{5.175790in}{1.359388in}}%
\pgfpathlineto{\pgfqpoint{5.181455in}{1.354867in}}%
\pgfpathlineto{\pgfqpoint{5.208616in}{1.333128in}}%
\pgfpathlineto{\pgfqpoint{5.216497in}{1.326876in}}%
\pgfpathlineto{\pgfqpoint{5.241661in}{1.306869in}}%
\pgfpathlineto{\pgfqpoint{5.251540in}{1.299080in}}%
\pgfpathlineto{\pgfqpoint{5.274922in}{1.280609in}}%
\pgfpathlineto{\pgfqpoint{5.286583in}{1.271470in}}%
\pgfpathlineto{\pgfqpoint{5.308394in}{1.254350in}}%
\pgfpathlineto{\pgfqpoint{5.321625in}{1.244041in}}%
\pgfpathlineto{\pgfqpoint{5.342075in}{1.228090in}}%
\pgfpathlineto{\pgfqpoint{5.356668in}{1.216785in}}%
\pgfpathlineto{\pgfqpoint{5.375958in}{1.201831in}}%
\pgfpathlineto{\pgfqpoint{5.391710in}{1.189695in}}%
\pgfpathlineto{\pgfqpoint{5.410039in}{1.175571in}}%
\pgfpathlineto{\pgfqpoint{5.426753in}{1.162764in}}%
\pgfpathlineto{\pgfqpoint{5.444313in}{1.149312in}}%
\pgfpathlineto{\pgfqpoint{5.461795in}{1.135985in}}%
\pgfpathlineto{\pgfqpoint{5.478774in}{1.123052in}}%
\pgfpathlineto{\pgfqpoint{5.496838in}{1.109352in}}%
\pgfpathlineto{\pgfqpoint{5.513415in}{1.096793in}}%
\pgfpathlineto{\pgfqpoint{5.531880in}{1.082855in}}%
\pgfpathlineto{\pgfqpoint{5.548231in}{1.070533in}}%
\pgfpathlineto{\pgfqpoint{5.566923in}{1.056489in}}%
\pgfpathlineto{\pgfqpoint{5.583213in}{1.044274in}}%
\pgfpathlineto{\pgfqpoint{5.601965in}{1.030246in}}%
\pgfpathlineto{\pgfqpoint{5.618356in}{1.018014in}}%
\pgfpathlineto{\pgfqpoint{5.637008in}{1.004117in}}%
\pgfpathlineto{\pgfqpoint{5.653649in}{0.991755in}}%
\pgfpathlineto{\pgfqpoint{5.672050in}{0.978097in}}%
\pgfpathlineto{\pgfqpoint{5.689085in}{0.965495in}}%
\pgfpathlineto{\pgfqpoint{5.707093in}{0.952176in}}%
\pgfpathlineto{\pgfqpoint{5.724655in}{0.939236in}}%
\pgfpathlineto{\pgfqpoint{5.742136in}{0.926347in}}%
\pgfpathlineto{\pgfqpoint{5.760347in}{0.912976in}}%
\pgfpathlineto{\pgfqpoint{5.777178in}{0.900602in}}%
\pgfpathlineto{\pgfqpoint{5.796152in}{0.886717in}}%
\pgfpathlineto{\pgfqpoint{5.812221in}{0.874931in}}%
\pgfpathlineto{\pgfqpoint{5.832057in}{0.860458in}}%
\pgfpathlineto{\pgfqpoint{5.847263in}{0.849327in}}%
\pgfpathlineto{\pgfqpoint{5.868051in}{0.834198in}}%
\pgfpathlineto{\pgfqpoint{5.882306in}{0.823781in}}%
\pgfpathlineto{\pgfqpoint{5.904118in}{0.807939in}}%
\pgfpathlineto{\pgfqpoint{5.917348in}{0.798282in}}%
\pgfpathlineto{\pgfqpoint{5.940246in}{0.781679in}}%
\pgfpathlineto{\pgfqpoint{5.952391in}{0.772820in}}%
\pgfpathlineto{\pgfqpoint{5.976417in}{0.755420in}}%
\pgfpathlineto{\pgfqpoint{5.987433in}{0.747386in}}%
\pgfpathlineto{\pgfqpoint{6.012616in}{0.729160in}}%
\pgfpathlineto{\pgfqpoint{6.022476in}{0.721966in}}%
\pgfpathlineto{\pgfqpoint{6.048823in}{0.702901in}}%
\pgfpathlineto{\pgfqpoint{6.057518in}{0.696550in}}%
\pgfpathlineto{\pgfqpoint{6.085018in}{0.676641in}}%
\pgfpathlineto{\pgfqpoint{6.092561in}{0.671124in}}%
\pgfpathlineto{\pgfqpoint{6.121180in}{0.650382in}}%
\pgfpathlineto{\pgfqpoint{6.127603in}{0.645672in}}%
\pgfpathlineto{\pgfqpoint{6.157285in}{0.624122in}}%
\pgfpathlineto{\pgfqpoint{6.162646in}{0.620179in}}%
\pgfpathlineto{\pgfqpoint{6.193307in}{0.597863in}}%
\pgfpathlineto{\pgfqpoint{6.197689in}{0.594628in}}%
\pgfpathlineto{\pgfqpoint{6.229219in}{0.571603in}}%
\pgfusepath{stroke}%
\end{pgfscope}%
\begin{pgfscope}%
\pgfpathrectangle{\pgfqpoint{0.766095in}{0.571603in}}{\pgfqpoint{6.973465in}{5.225635in}}%
\pgfusepath{clip}%
\pgfsetbuttcap%
\pgfsetroundjoin%
\pgfsetlinewidth{1.505625pt}%
\definecolor{currentstroke}{rgb}{0.281924,0.089666,0.412415}%
\pgfsetstrokecolor{currentstroke}%
\pgfsetdash{}{0pt}%
\pgfpathmoveto{\pgfqpoint{3.053770in}{0.571603in}}%
\pgfpathlineto{\pgfqpoint{2.938732in}{0.630724in}}%
\pgfpathlineto{\pgfqpoint{2.853891in}{0.676641in}}%
\pgfpathlineto{\pgfqpoint{2.761741in}{0.729160in}}%
\pgfpathlineto{\pgfqpoint{2.658392in}{0.791948in}}%
\pgfpathlineto{\pgfqpoint{2.588307in}{0.836918in}}%
\pgfpathlineto{\pgfqpoint{2.514756in}{0.886717in}}%
\pgfpathlineto{\pgfqpoint{2.441619in}{0.939236in}}%
\pgfpathlineto{\pgfqpoint{2.372775in}{0.991755in}}%
\pgfpathlineto{\pgfqpoint{2.307967in}{1.044366in}}%
\pgfpathlineto{\pgfqpoint{2.237882in}{1.105570in}}%
\pgfpathlineto{\pgfqpoint{2.190879in}{1.149312in}}%
\pgfpathlineto{\pgfqpoint{2.132754in}{1.207264in}}%
\pgfpathlineto{\pgfqpoint{2.088794in}{1.254350in}}%
\pgfpathlineto{\pgfqpoint{2.043209in}{1.306869in}}%
\pgfpathlineto{\pgfqpoint{2.001060in}{1.359388in}}%
\pgfpathlineto{\pgfqpoint{1.962301in}{1.411906in}}%
\pgfpathlineto{\pgfqpoint{1.926874in}{1.464425in}}%
\pgfpathlineto{\pgfqpoint{1.894700in}{1.516944in}}%
\pgfpathlineto{\pgfqpoint{1.865689in}{1.569463in}}%
\pgfpathlineto{\pgfqpoint{1.839737in}{1.621982in}}%
\pgfpathlineto{\pgfqpoint{1.816723in}{1.674501in}}%
\pgfpathlineto{\pgfqpoint{1.796877in}{1.727020in}}%
\pgfpathlineto{\pgfqpoint{1.779786in}{1.779539in}}%
\pgfpathlineto{\pgfqpoint{1.765746in}{1.832058in}}%
\pgfpathlineto{\pgfqpoint{1.754444in}{1.884577in}}%
\pgfpathlineto{\pgfqpoint{1.745959in}{1.937096in}}%
\pgfpathlineto{\pgfqpoint{1.740403in}{1.989615in}}%
\pgfpathlineto{\pgfqpoint{1.737619in}{2.042134in}}%
\pgfpathlineto{\pgfqpoint{1.737622in}{2.094653in}}%
\pgfpathlineto{\pgfqpoint{1.740419in}{2.147172in}}%
\pgfpathlineto{\pgfqpoint{1.746008in}{2.199691in}}%
\pgfpathlineto{\pgfqpoint{1.754582in}{2.252210in}}%
\pgfpathlineto{\pgfqpoint{1.766036in}{2.304729in}}%
\pgfpathlineto{\pgfqpoint{1.782329in}{2.363769in}}%
\pgfpathlineto{\pgfqpoint{1.797793in}{2.409766in}}%
\pgfpathlineto{\pgfqpoint{1.818172in}{2.462285in}}%
\pgfpathlineto{\pgfqpoint{1.841983in}{2.514804in}}%
\pgfpathlineto{\pgfqpoint{1.869057in}{2.567323in}}%
\pgfpathlineto{\pgfqpoint{1.899591in}{2.619842in}}%
\pgfpathlineto{\pgfqpoint{1.933816in}{2.672361in}}%
\pgfpathlineto{\pgfqpoint{1.971971in}{2.724880in}}%
\pgfpathlineto{\pgfqpoint{2.014299in}{2.777399in}}%
\pgfpathlineto{\pgfqpoint{2.037117in}{2.803659in}}%
\pgfpathlineto{\pgfqpoint{2.086424in}{2.856177in}}%
\pgfpathlineto{\pgfqpoint{2.113015in}{2.882437in}}%
\pgfpathlineto{\pgfqpoint{2.140920in}{2.908696in}}%
\pgfpathlineto{\pgfqpoint{2.170254in}{2.934956in}}%
\pgfpathlineto{\pgfqpoint{2.202839in}{2.962569in}}%
\pgfpathlineto{\pgfqpoint{2.268347in}{3.013734in}}%
\pgfpathlineto{\pgfqpoint{2.307967in}{3.042280in}}%
\pgfpathlineto{\pgfqpoint{2.378052in}{3.088674in}}%
\pgfpathlineto{\pgfqpoint{2.427926in}{3.118772in}}%
\pgfpathlineto{\pgfqpoint{2.483179in}{3.149778in}}%
\pgfpathlineto{\pgfqpoint{2.553264in}{3.185458in}}%
\pgfpathlineto{\pgfqpoint{2.623350in}{3.217528in}}%
\pgfpathlineto{\pgfqpoint{2.693435in}{3.246149in}}%
\pgfpathlineto{\pgfqpoint{2.763520in}{3.271457in}}%
\pgfpathlineto{\pgfqpoint{2.833605in}{3.293558in}}%
\pgfpathlineto{\pgfqpoint{2.903690in}{3.312539in}}%
\pgfpathlineto{\pgfqpoint{2.975822in}{3.328848in}}%
\pgfpathlineto{\pgfqpoint{3.043860in}{3.341062in}}%
\pgfpathlineto{\pgfqpoint{3.113945in}{3.350515in}}%
\pgfpathlineto{\pgfqpoint{3.184030in}{3.356575in}}%
\pgfpathlineto{\pgfqpoint{3.254115in}{3.359018in}}%
\pgfpathlineto{\pgfqpoint{3.324200in}{3.357679in}}%
\pgfpathlineto{\pgfqpoint{3.363548in}{3.355107in}}%
\pgfpathlineto{\pgfqpoint{3.394285in}{3.352204in}}%
\pgfpathlineto{\pgfqpoint{3.429328in}{3.347776in}}%
\pgfpathlineto{\pgfqpoint{3.464371in}{3.342180in}}%
\pgfpathlineto{\pgfqpoint{3.527765in}{3.328848in}}%
\pgfpathlineto{\pgfqpoint{3.534456in}{3.327257in}}%
\pgfpathlineto{\pgfqpoint{3.569498in}{3.317683in}}%
\pgfpathlineto{\pgfqpoint{3.616273in}{3.302589in}}%
\pgfpathlineto{\pgfqpoint{3.639583in}{3.294136in}}%
\pgfpathlineto{\pgfqpoint{3.682766in}{3.276329in}}%
\pgfpathlineto{\pgfqpoint{3.709668in}{3.263981in}}%
\pgfpathlineto{\pgfqpoint{3.744711in}{3.246211in}}%
\pgfpathlineto{\pgfqpoint{3.784141in}{3.223810in}}%
\pgfpathlineto{\pgfqpoint{3.825419in}{3.197551in}}%
\pgfpathlineto{\pgfqpoint{3.862719in}{3.171291in}}%
\pgfpathlineto{\pgfqpoint{3.896886in}{3.145032in}}%
\pgfpathlineto{\pgfqpoint{3.928559in}{3.118772in}}%
\pgfpathlineto{\pgfqpoint{3.958226in}{3.092513in}}%
\pgfpathlineto{\pgfqpoint{4.012525in}{3.039994in}}%
\pgfpathlineto{\pgfqpoint{4.037843in}{3.013734in}}%
\pgfpathlineto{\pgfqpoint{4.085669in}{2.961215in}}%
\pgfpathlineto{\pgfqpoint{4.108416in}{2.934956in}}%
\pgfpathlineto{\pgfqpoint{4.165221in}{2.866628in}}%
\pgfpathlineto{\pgfqpoint{4.240719in}{2.770454in}}%
\pgfpathlineto{\pgfqpoint{4.240719in}{2.770454in}}%
\pgfusepath{stroke}%
\end{pgfscope}%
\begin{pgfscope}%
\pgfpathrectangle{\pgfqpoint{0.766095in}{0.571603in}}{\pgfqpoint{6.973465in}{5.225635in}}%
\pgfusepath{clip}%
\pgfsetbuttcap%
\pgfsetroundjoin%
\pgfsetlinewidth{1.505625pt}%
\definecolor{currentstroke}{rgb}{0.281924,0.089666,0.412415}%
\pgfsetstrokecolor{currentstroke}%
\pgfsetdash{}{0pt}%
\pgfpathmoveto{\pgfqpoint{4.429569in}{2.521687in}}%
\pgfpathlineto{\pgfqpoint{4.495801in}{2.436026in}}%
\pgfpathlineto{\pgfqpoint{4.600873in}{2.304729in}}%
\pgfpathlineto{\pgfqpoint{4.690859in}{2.197469in}}%
\pgfpathlineto{\pgfqpoint{4.760944in}{2.117148in}}%
\pgfpathlineto{\pgfqpoint{4.831030in}{2.039621in}}%
\pgfpathlineto{\pgfqpoint{4.902431in}{1.963355in}}%
\pgfpathlineto{\pgfqpoint{4.978840in}{1.884577in}}%
\pgfpathlineto{\pgfqpoint{5.076327in}{1.788132in}}%
\pgfpathlineto{\pgfqpoint{5.146412in}{1.721227in}}%
\pgfpathlineto{\pgfqpoint{5.225181in}{1.648242in}}%
\pgfpathlineto{\pgfqpoint{5.321625in}{1.561980in}}%
\pgfpathlineto{\pgfqpoint{5.403838in}{1.490685in}}%
\pgfpathlineto{\pgfqpoint{5.497557in}{1.411906in}}%
\pgfpathlineto{\pgfqpoint{5.601965in}{1.326752in}}%
\pgfpathlineto{\pgfqpoint{5.707093in}{1.243564in}}%
\pgfpathlineto{\pgfqpoint{5.829803in}{1.149312in}}%
\pgfpathlineto{\pgfqpoint{5.970946in}{1.044274in}}%
\pgfpathlineto{\pgfqpoint{6.116390in}{0.939236in}}%
\pgfpathlineto{\pgfqpoint{6.267774in}{0.832821in}}%
\pgfpathlineto{\pgfqpoint{6.457214in}{0.702901in}}%
\pgfpathlineto{\pgfqpoint{6.618199in}{0.594720in}}%
\pgfpathlineto{\pgfqpoint{6.652946in}{0.571603in}}%
\pgfpathlineto{\pgfqpoint{6.652946in}{0.571603in}}%
\pgfusepath{stroke}%
\end{pgfscope}%
\begin{pgfscope}%
\pgfpathrectangle{\pgfqpoint{0.766095in}{0.571603in}}{\pgfqpoint{6.973465in}{5.225635in}}%
\pgfusepath{clip}%
\pgfsetbuttcap%
\pgfsetroundjoin%
\pgfsetlinewidth{1.505625pt}%
\definecolor{currentstroke}{rgb}{0.283197,0.115680,0.436115}%
\pgfsetstrokecolor{currentstroke}%
\pgfsetdash{}{0pt}%
\pgfpathmoveto{\pgfqpoint{2.744085in}{0.571603in}}%
\pgfpathlineto{\pgfqpoint{2.646168in}{0.624122in}}%
\pgfpathlineto{\pgfqpoint{2.553111in}{0.676641in}}%
\pgfpathlineto{\pgfqpoint{2.448137in}{0.739578in}}%
\pgfpathlineto{\pgfqpoint{2.378052in}{0.783791in}}%
\pgfpathlineto{\pgfqpoint{2.302129in}{0.834198in}}%
\pgfpathlineto{\pgfqpoint{2.227268in}{0.886717in}}%
\pgfpathlineto{\pgfqpoint{2.156539in}{0.939236in}}%
\pgfpathlineto{\pgfqpoint{2.089813in}{0.991755in}}%
\pgfpathlineto{\pgfqpoint{2.026938in}{1.044274in}}%
\pgfpathlineto{\pgfqpoint{1.957541in}{1.106471in}}%
\pgfpathlineto{\pgfqpoint{1.912551in}{1.149312in}}%
\pgfpathlineto{\pgfqpoint{1.852414in}{1.210611in}}%
\pgfpathlineto{\pgfqpoint{1.812244in}{1.254350in}}%
\pgfpathlineto{\pgfqpoint{1.767204in}{1.306869in}}%
\pgfpathlineto{\pgfqpoint{1.725360in}{1.359388in}}%
\pgfpathlineto{\pgfqpoint{1.686662in}{1.411906in}}%
\pgfpathlineto{\pgfqpoint{1.651048in}{1.464425in}}%
\pgfpathlineto{\pgfqpoint{1.618444in}{1.516944in}}%
\pgfpathlineto{\pgfqpoint{1.588764in}{1.569463in}}%
\pgfpathlineto{\pgfqpoint{1.561913in}{1.621982in}}%
\pgfpathlineto{\pgfqpoint{1.537031in}{1.676338in}}%
\pgfpathlineto{\pgfqpoint{1.516590in}{1.727020in}}%
\pgfpathlineto{\pgfqpoint{1.497927in}{1.779539in}}%
\pgfpathlineto{\pgfqpoint{1.482018in}{1.832058in}}%
\pgfpathlineto{\pgfqpoint{1.466946in}{1.892029in}}%
\pgfpathlineto{\pgfqpoint{1.457848in}{1.937096in}}%
\pgfpathlineto{\pgfqpoint{1.449625in}{1.989615in}}%
\pgfpathlineto{\pgfqpoint{1.443887in}{2.042134in}}%
\pgfpathlineto{\pgfqpoint{1.440648in}{2.094653in}}%
\pgfpathlineto{\pgfqpoint{1.439911in}{2.147172in}}%
\pgfpathlineto{\pgfqpoint{1.441678in}{2.199691in}}%
\pgfpathlineto{\pgfqpoint{1.445943in}{2.252210in}}%
\pgfpathlineto{\pgfqpoint{1.452694in}{2.304729in}}%
\pgfpathlineto{\pgfqpoint{1.461913in}{2.357248in}}%
\pgfpathlineto{\pgfqpoint{1.473741in}{2.409766in}}%
\pgfpathlineto{\pgfqpoint{1.488170in}{2.462285in}}%
\pgfpathlineto{\pgfqpoint{1.505116in}{2.514804in}}%
\pgfpathlineto{\pgfqpoint{1.524874in}{2.567323in}}%
\pgfpathlineto{\pgfqpoint{1.547300in}{2.619842in}}%
\pgfpathlineto{\pgfqpoint{1.572464in}{2.672361in}}%
\pgfpathlineto{\pgfqpoint{1.607116in}{2.736122in}}%
\pgfpathlineto{\pgfqpoint{1.631909in}{2.777399in}}%
\pgfpathlineto{\pgfqpoint{1.666245in}{2.829918in}}%
\pgfpathlineto{\pgfqpoint{1.703854in}{2.882437in}}%
\pgfpathlineto{\pgfqpoint{1.744885in}{2.934956in}}%
\pgfpathlineto{\pgfqpoint{1.782329in}{2.979143in}}%
\pgfpathlineto{\pgfqpoint{1.817371in}{3.017921in}}%
\pgfpathlineto{\pgfqpoint{1.864262in}{3.066253in}}%
\pgfpathlineto{\pgfqpoint{1.919244in}{3.118772in}}%
\pgfpathlineto{\pgfqpoint{1.957541in}{3.152883in}}%
\pgfpathlineto{\pgfqpoint{2.010897in}{3.197551in}}%
\pgfpathlineto{\pgfqpoint{2.062669in}{3.238121in}}%
\pgfpathlineto{\pgfqpoint{2.114699in}{3.276329in}}%
\pgfpathlineto{\pgfqpoint{2.167797in}{3.313023in}}%
\pgfpathlineto{\pgfqpoint{2.237882in}{3.358194in}}%
\pgfpathlineto{\pgfqpoint{2.307967in}{3.399915in}}%
\pgfpathlineto{\pgfqpoint{2.378052in}{3.438632in}}%
\pgfpathlineto{\pgfqpoint{2.448137in}{3.474488in}}%
\pgfpathlineto{\pgfqpoint{2.528978in}{3.512664in}}%
\pgfpathlineto{\pgfqpoint{2.588882in}{3.538924in}}%
\pgfpathlineto{\pgfqpoint{2.658392in}{3.567175in}}%
\pgfpathlineto{\pgfqpoint{2.728477in}{3.593468in}}%
\pgfpathlineto{\pgfqpoint{2.798753in}{3.617702in}}%
\pgfpathlineto{\pgfqpoint{2.883278in}{3.643962in}}%
\pgfpathlineto{\pgfqpoint{2.938732in}{3.659615in}}%
\pgfpathlineto{\pgfqpoint{3.008818in}{3.677583in}}%
\pgfpathlineto{\pgfqpoint{3.093271in}{3.696481in}}%
\pgfpathlineto{\pgfqpoint{3.148988in}{3.707293in}}%
\pgfpathlineto{\pgfqpoint{3.219073in}{3.719029in}}%
\pgfpathlineto{\pgfqpoint{3.289158in}{3.728497in}}%
\pgfpathlineto{\pgfqpoint{3.359243in}{3.735647in}}%
\pgfpathlineto{\pgfqpoint{3.429328in}{3.740391in}}%
\pgfpathlineto{\pgfqpoint{3.499413in}{3.742514in}}%
\pgfpathlineto{\pgfqpoint{3.569498in}{3.741772in}}%
\pgfpathlineto{\pgfqpoint{3.639583in}{3.737885in}}%
\pgfpathlineto{\pgfqpoint{3.709668in}{3.730528in}}%
\pgfpathlineto{\pgfqpoint{3.779753in}{3.719231in}}%
\pgfpathlineto{\pgfqpoint{3.814796in}{3.711850in}}%
\pgfpathlineto{\pgfqpoint{3.874458in}{3.696481in}}%
\pgfpathlineto{\pgfqpoint{3.884881in}{3.693448in}}%
\pgfpathlineto{\pgfqpoint{3.919924in}{3.682054in}}%
\pgfpathlineto{\pgfqpoint{3.954966in}{3.669207in}}%
\pgfpathlineto{\pgfqpoint{4.012661in}{3.643962in}}%
\pgfpathlineto{\pgfqpoint{4.025051in}{3.637925in}}%
\pgfpathlineto{\pgfqpoint{4.062832in}{3.617702in}}%
\pgfpathlineto{\pgfqpoint{4.105492in}{3.591443in}}%
\pgfpathlineto{\pgfqpoint{4.142922in}{3.565183in}}%
\pgfpathlineto{\pgfqpoint{4.176268in}{3.538924in}}%
\pgfpathlineto{\pgfqpoint{4.206380in}{3.512664in}}%
\pgfpathlineto{\pgfqpoint{4.235306in}{3.484957in}}%
\pgfpathlineto{\pgfqpoint{4.270349in}{3.447661in}}%
\pgfpathlineto{\pgfqpoint{4.305391in}{3.406202in}}%
\pgfpathlineto{\pgfqpoint{4.344069in}{3.355107in}}%
\pgfpathlineto{\pgfqpoint{4.380041in}{3.302589in}}%
\pgfpathlineto{\pgfqpoint{4.413187in}{3.250070in}}%
\pgfpathlineto{\pgfqpoint{4.459068in}{3.171291in}}%
\pgfpathlineto{\pgfqpoint{4.502193in}{3.092513in}}%
\pgfpathlineto{\pgfqpoint{4.557615in}{2.987475in}}%
\pgfpathlineto{\pgfqpoint{4.640572in}{2.829918in}}%
\pgfpathlineto{\pgfqpoint{4.712878in}{2.698621in}}%
\pgfpathlineto{\pgfqpoint{4.774257in}{2.593583in}}%
\pgfpathlineto{\pgfqpoint{4.831030in}{2.502124in}}%
\pgfpathlineto{\pgfqpoint{4.873961in}{2.436026in}}%
\pgfpathlineto{\pgfqpoint{4.927727in}{2.357248in}}%
\pgfpathlineto{\pgfqpoint{4.984333in}{2.278469in}}%
\pgfpathlineto{\pgfqpoint{5.043995in}{2.199691in}}%
\pgfpathlineto{\pgfqpoint{5.106698in}{2.120912in}}%
\pgfpathlineto{\pgfqpoint{5.181455in}{2.031923in}}%
\pgfpathlineto{\pgfqpoint{5.241753in}{1.963355in}}%
\pgfpathlineto{\pgfqpoint{5.321625in}{1.876834in}}%
\pgfpathlineto{\pgfqpoint{5.391710in}{1.804202in}}%
\pgfpathlineto{\pgfqpoint{5.469299in}{1.727020in}}%
\pgfpathlineto{\pgfqpoint{5.566923in}{1.634304in}}%
\pgfpathlineto{\pgfqpoint{5.637916in}{1.569463in}}%
\pgfpathlineto{\pgfqpoint{5.727208in}{1.490685in}}%
\pgfpathlineto{\pgfqpoint{5.819922in}{1.411906in}}%
\pgfpathlineto{\pgfqpoint{5.917348in}{1.332015in}}%
\pgfpathlineto{\pgfqpoint{6.022476in}{1.248691in}}%
\pgfpathlineto{\pgfqpoint{6.127603in}{1.168067in}}%
\pgfpathlineto{\pgfqpoint{6.232731in}{1.089845in}}%
\pgfpathlineto{\pgfqpoint{6.337859in}{1.013768in}}%
\pgfpathlineto{\pgfqpoint{6.481344in}{0.912976in}}%
\pgfpathlineto{\pgfqpoint{6.489374in}{0.907442in}}%
\pgfpathlineto{\pgfqpoint{6.489374in}{0.907442in}}%
\pgfusepath{stroke}%
\end{pgfscope}%
\begin{pgfscope}%
\pgfpathrectangle{\pgfqpoint{0.766095in}{0.571603in}}{\pgfqpoint{6.973465in}{5.225635in}}%
\pgfusepath{clip}%
\pgfsetbuttcap%
\pgfsetroundjoin%
\pgfsetlinewidth{1.505625pt}%
\definecolor{currentstroke}{rgb}{0.283197,0.115680,0.436115}%
\pgfsetstrokecolor{currentstroke}%
\pgfsetdash{}{0pt}%
\pgfpathmoveto{\pgfqpoint{6.748050in}{0.733430in}}%
\pgfpathlineto{\pgfqpoint{6.754548in}{0.729160in}}%
\pgfpathlineto{\pgfqpoint{6.758369in}{0.726650in}}%
\pgfpathlineto{\pgfqpoint{6.793412in}{0.703753in}}%
\pgfpathlineto{\pgfqpoint{6.794722in}{0.702901in}}%
\pgfpathlineto{\pgfqpoint{6.828454in}{0.680946in}}%
\pgfpathlineto{\pgfqpoint{6.835111in}{0.676641in}}%
\pgfpathlineto{\pgfqpoint{6.863497in}{0.658269in}}%
\pgfpathlineto{\pgfqpoint{6.875766in}{0.650382in}}%
\pgfpathlineto{\pgfqpoint{6.898539in}{0.635720in}}%
\pgfpathlineto{\pgfqpoint{6.916682in}{0.624122in}}%
\pgfpathlineto{\pgfqpoint{6.933582in}{0.613296in}}%
\pgfpathlineto{\pgfqpoint{6.957853in}{0.597863in}}%
\pgfpathlineto{\pgfqpoint{6.968624in}{0.590994in}}%
\pgfpathlineto{\pgfqpoint{6.999271in}{0.571603in}}%
\pgfusepath{stroke}%
\end{pgfscope}%
\begin{pgfscope}%
\pgfpathrectangle{\pgfqpoint{0.766095in}{0.571603in}}{\pgfqpoint{6.973465in}{5.225635in}}%
\pgfusepath{clip}%
\pgfsetbuttcap%
\pgfsetroundjoin%
\pgfsetlinewidth{1.505625pt}%
\definecolor{currentstroke}{rgb}{0.282623,0.140926,0.457517}%
\pgfsetstrokecolor{currentstroke}%
\pgfsetdash{}{0pt}%
\pgfpathmoveto{\pgfqpoint{2.479988in}{0.571603in}}%
\pgfpathlineto{\pgfqpoint{2.378052in}{0.628471in}}%
\pgfpathlineto{\pgfqpoint{2.295832in}{0.676641in}}%
\pgfpathlineto{\pgfqpoint{2.202839in}{0.734061in}}%
\pgfpathlineto{\pgfqpoint{2.129524in}{0.781679in}}%
\pgfpathlineto{\pgfqpoint{2.052795in}{0.834198in}}%
\pgfpathlineto{\pgfqpoint{1.980098in}{0.886717in}}%
\pgfpathlineto{\pgfqpoint{1.911320in}{0.939236in}}%
\pgfpathlineto{\pgfqpoint{1.846328in}{0.991755in}}%
\pgfpathlineto{\pgfqpoint{1.782329in}{1.046679in}}%
\pgfpathlineto{\pgfqpoint{1.727368in}{1.096793in}}%
\pgfpathlineto{\pgfqpoint{1.673074in}{1.149312in}}%
\pgfpathlineto{\pgfqpoint{1.622229in}{1.201831in}}%
\pgfpathlineto{\pgfqpoint{1.572073in}{1.257195in}}%
\pgfpathlineto{\pgfqpoint{1.530098in}{1.306869in}}%
\pgfpathlineto{\pgfqpoint{1.488729in}{1.359388in}}%
\pgfpathlineto{\pgfqpoint{1.450327in}{1.411906in}}%
\pgfpathlineto{\pgfqpoint{1.414831in}{1.464425in}}%
\pgfpathlineto{\pgfqpoint{1.382167in}{1.516944in}}%
\pgfpathlineto{\pgfqpoint{1.352255in}{1.569463in}}%
\pgfpathlineto{\pgfqpoint{1.325004in}{1.621982in}}%
\pgfpathlineto{\pgfqpoint{1.300492in}{1.674501in}}%
\pgfpathlineto{\pgfqpoint{1.278531in}{1.727020in}}%
\pgfpathlineto{\pgfqpoint{1.256691in}{1.786463in}}%
\pgfpathlineto{\pgfqpoint{1.242125in}{1.832058in}}%
\pgfpathlineto{\pgfqpoint{1.227570in}{1.884577in}}%
\pgfpathlineto{\pgfqpoint{1.215460in}{1.937096in}}%
\pgfpathlineto{\pgfqpoint{1.205759in}{1.989615in}}%
\pgfpathlineto{\pgfqpoint{1.198359in}{2.042134in}}%
\pgfpathlineto{\pgfqpoint{1.193269in}{2.094653in}}%
\pgfpathlineto{\pgfqpoint{1.190492in}{2.147172in}}%
\pgfpathlineto{\pgfqpoint{1.190030in}{2.199691in}}%
\pgfpathlineto{\pgfqpoint{1.191877in}{2.252210in}}%
\pgfpathlineto{\pgfqpoint{1.196026in}{2.304729in}}%
\pgfpathlineto{\pgfqpoint{1.202465in}{2.357248in}}%
\pgfpathlineto{\pgfqpoint{1.211177in}{2.409766in}}%
\pgfpathlineto{\pgfqpoint{1.222153in}{2.462285in}}%
\pgfpathlineto{\pgfqpoint{1.235644in}{2.514804in}}%
\pgfpathlineto{\pgfqpoint{1.251384in}{2.567323in}}%
\pgfpathlineto{\pgfqpoint{1.269631in}{2.619842in}}%
\pgfpathlineto{\pgfqpoint{1.291733in}{2.675916in}}%
\pgfpathlineto{\pgfqpoint{1.313505in}{2.724880in}}%
\pgfpathlineto{\pgfqpoint{1.339292in}{2.777399in}}%
\pgfpathlineto{\pgfqpoint{1.367710in}{2.829918in}}%
\pgfpathlineto{\pgfqpoint{1.398884in}{2.882437in}}%
\pgfpathlineto{\pgfqpoint{1.432936in}{2.934956in}}%
\pgfpathlineto{\pgfqpoint{1.469989in}{2.987475in}}%
\pgfpathlineto{\pgfqpoint{1.510162in}{3.039994in}}%
\pgfpathlineto{\pgfqpoint{1.553571in}{3.092513in}}%
\pgfpathlineto{\pgfqpoint{1.600322in}{3.145032in}}%
\pgfpathlineto{\pgfqpoint{1.642158in}{3.188882in}}%
\pgfpathlineto{\pgfqpoint{1.677277in}{3.223810in}}%
\pgfpathlineto{\pgfqpoint{1.733631in}{3.276329in}}%
\pgfpathlineto{\pgfqpoint{1.782329in}{3.318925in}}%
\pgfpathlineto{\pgfqpoint{1.826007in}{3.355107in}}%
\pgfpathlineto{\pgfqpoint{1.893433in}{3.407626in}}%
\pgfpathlineto{\pgfqpoint{1.966055in}{3.460145in}}%
\pgfpathlineto{\pgfqpoint{2.044343in}{3.512664in}}%
\pgfpathlineto{\pgfqpoint{2.097711in}{3.546313in}}%
\pgfpathlineto{\pgfqpoint{2.173643in}{3.591443in}}%
\pgfpathlineto{\pgfqpoint{2.237882in}{3.627244in}}%
\pgfpathlineto{\pgfqpoint{2.319963in}{3.670221in}}%
\pgfpathlineto{\pgfqpoint{2.378052in}{3.698882in}}%
\pgfpathlineto{\pgfqpoint{2.448137in}{3.731533in}}%
\pgfpathlineto{\pgfqpoint{2.518222in}{3.762330in}}%
\pgfpathlineto{\pgfqpoint{2.588307in}{3.791351in}}%
\pgfpathlineto{\pgfqpoint{2.682700in}{3.827778in}}%
\pgfpathlineto{\pgfqpoint{2.763520in}{3.856637in}}%
\pgfpathlineto{\pgfqpoint{2.868647in}{3.891009in}}%
\pgfpathlineto{\pgfqpoint{2.938732in}{3.912114in}}%
\pgfpathlineto{\pgfqpoint{3.043860in}{3.940965in}}%
\pgfpathlineto{\pgfqpoint{3.148988in}{3.966572in}}%
\pgfpathlineto{\pgfqpoint{3.254115in}{3.988935in}}%
\pgfpathlineto{\pgfqpoint{3.359243in}{4.007929in}}%
\pgfpathlineto{\pgfqpoint{3.429328in}{4.018646in}}%
\pgfpathlineto{\pgfqpoint{3.499413in}{4.027785in}}%
\pgfpathlineto{\pgfqpoint{3.597714in}{4.037854in}}%
\pgfpathlineto{\pgfqpoint{3.639583in}{4.041058in}}%
\pgfpathlineto{\pgfqpoint{3.709668in}{4.044879in}}%
\pgfpathlineto{\pgfqpoint{3.779753in}{4.046707in}}%
\pgfpathlineto{\pgfqpoint{3.849838in}{4.046348in}}%
\pgfpathlineto{\pgfqpoint{3.919924in}{4.043578in}}%
\pgfpathlineto{\pgfqpoint{3.992689in}{4.037854in}}%
\pgfpathlineto{\pgfqpoint{4.060094in}{4.029447in}}%
\pgfpathlineto{\pgfqpoint{4.130179in}{4.017296in}}%
\pgfpathlineto{\pgfqpoint{4.165221in}{4.009728in}}%
\pgfpathlineto{\pgfqpoint{4.235306in}{3.990890in}}%
\pgfpathlineto{\pgfqpoint{4.270349in}{3.979465in}}%
\pgfpathlineto{\pgfqpoint{4.323524in}{3.959075in}}%
\pgfpathlineto{\pgfqpoint{4.340434in}{3.951742in}}%
\pgfpathlineto{\pgfqpoint{4.379904in}{3.932816in}}%
\pgfpathlineto{\pgfqpoint{4.426495in}{3.906556in}}%
\pgfpathlineto{\pgfqpoint{4.466310in}{3.880297in}}%
\pgfpathlineto{\pgfqpoint{4.500869in}{3.854037in}}%
\pgfpathlineto{\pgfqpoint{4.531286in}{3.827778in}}%
\pgfpathlineto{\pgfqpoint{4.558396in}{3.801519in}}%
\pgfpathlineto{\pgfqpoint{4.585732in}{3.771839in}}%
\pgfpathlineto{\pgfqpoint{4.620774in}{3.728126in}}%
\pgfpathlineto{\pgfqpoint{4.643019in}{3.696481in}}%
\pgfpathlineto{\pgfqpoint{4.659995in}{3.670221in}}%
\pgfpathlineto{\pgfqpoint{4.690859in}{3.616679in}}%
\pgfpathlineto{\pgfqpoint{4.716760in}{3.565183in}}%
\pgfpathlineto{\pgfqpoint{4.740451in}{3.512664in}}%
\pgfpathlineto{\pgfqpoint{4.762138in}{3.460145in}}%
\pgfpathlineto{\pgfqpoint{4.795987in}{3.369708in}}%
\pgfpathlineto{\pgfqpoint{4.837197in}{3.250070in}}%
\pgfpathlineto{\pgfqpoint{4.901115in}{3.062944in}}%
\pgfpathlineto{\pgfqpoint{4.936157in}{2.966488in}}%
\pgfpathlineto{\pgfqpoint{4.958238in}{2.908696in}}%
\pgfpathlineto{\pgfqpoint{4.990208in}{2.829918in}}%
\pgfpathlineto{\pgfqpoint{5.024477in}{2.751140in}}%
\pgfpathlineto{\pgfqpoint{5.061301in}{2.672361in}}%
\pgfpathlineto{\pgfqpoint{5.100909in}{2.593583in}}%
\pgfpathlineto{\pgfqpoint{5.146412in}{2.509711in}}%
\pgfpathlineto{\pgfqpoint{5.189152in}{2.436026in}}%
\pgfpathlineto{\pgfqpoint{5.238057in}{2.357248in}}%
\pgfpathlineto{\pgfqpoint{5.290374in}{2.278469in}}%
\pgfpathlineto{\pgfqpoint{5.346073in}{2.199691in}}%
\pgfpathlineto{\pgfqpoint{5.405317in}{2.120912in}}%
\pgfpathlineto{\pgfqpoint{5.468172in}{2.042134in}}%
\pgfpathlineto{\pgfqpoint{5.534650in}{1.963355in}}%
\pgfpathlineto{\pgfqpoint{5.604771in}{1.884577in}}%
\pgfpathlineto{\pgfqpoint{5.678557in}{1.805799in}}%
\pgfpathlineto{\pgfqpoint{5.756036in}{1.727020in}}%
\pgfpathlineto{\pgfqpoint{5.847263in}{1.638829in}}%
\pgfpathlineto{\pgfqpoint{5.922159in}{1.569463in}}%
\pgfpathlineto{\pgfqpoint{6.022476in}{1.480510in}}%
\pgfpathlineto{\pgfqpoint{6.102939in}{1.411906in}}%
\pgfpathlineto{\pgfqpoint{6.198832in}{1.333128in}}%
\pgfpathlineto{\pgfqpoint{6.302816in}{1.250816in}}%
\pgfpathlineto{\pgfqpoint{6.387805in}{1.185715in}}%
\pgfpathlineto{\pgfqpoint{6.387805in}{1.185715in}}%
\pgfusepath{stroke}%
\end{pgfscope}%
\begin{pgfscope}%
\pgfpathrectangle{\pgfqpoint{0.766095in}{0.571603in}}{\pgfqpoint{6.973465in}{5.225635in}}%
\pgfusepath{clip}%
\pgfsetbuttcap%
\pgfsetroundjoin%
\pgfsetlinewidth{1.505625pt}%
\definecolor{currentstroke}{rgb}{0.282623,0.140926,0.457517}%
\pgfsetstrokecolor{currentstroke}%
\pgfsetdash{}{0pt}%
\pgfpathmoveto{\pgfqpoint{6.639949in}{1.002280in}}%
\pgfpathlineto{\pgfqpoint{6.653242in}{0.992937in}}%
\pgfpathlineto{\pgfqpoint{6.654926in}{0.991755in}}%
\pgfpathlineto{\pgfqpoint{6.688284in}{0.968525in}}%
\pgfpathlineto{\pgfqpoint{6.692646in}{0.965495in}}%
\pgfpathlineto{\pgfqpoint{6.723327in}{0.944329in}}%
\pgfpathlineto{\pgfqpoint{6.730732in}{0.939236in}}%
\pgfpathlineto{\pgfqpoint{6.758369in}{0.920343in}}%
\pgfpathlineto{\pgfqpoint{6.769181in}{0.912976in}}%
\pgfpathlineto{\pgfqpoint{6.793412in}{0.896559in}}%
\pgfpathlineto{\pgfqpoint{6.807990in}{0.886717in}}%
\pgfpathlineto{\pgfqpoint{6.828454in}{0.872972in}}%
\pgfpathlineto{\pgfqpoint{6.847159in}{0.860458in}}%
\pgfpathlineto{\pgfqpoint{6.863497in}{0.849577in}}%
\pgfpathlineto{\pgfqpoint{6.886684in}{0.834198in}}%
\pgfpathlineto{\pgfqpoint{6.898539in}{0.826367in}}%
\pgfpathlineto{\pgfqpoint{6.926563in}{0.807939in}}%
\pgfpathlineto{\pgfqpoint{6.933582in}{0.803340in}}%
\pgfpathlineto{\pgfqpoint{6.966794in}{0.781679in}}%
\pgfpathlineto{\pgfqpoint{6.968624in}{0.780489in}}%
\pgfpathlineto{\pgfqpoint{7.003667in}{0.757781in}}%
\pgfpathlineto{\pgfqpoint{7.007327in}{0.755420in}}%
\pgfpathlineto{\pgfqpoint{7.038709in}{0.735222in}}%
\pgfpathlineto{\pgfqpoint{7.048177in}{0.729160in}}%
\pgfpathlineto{\pgfqpoint{7.073752in}{0.712821in}}%
\pgfpathlineto{\pgfqpoint{7.089367in}{0.702901in}}%
\pgfpathlineto{\pgfqpoint{7.108795in}{0.690577in}}%
\pgfpathlineto{\pgfqpoint{7.130893in}{0.676641in}}%
\pgfpathlineto{\pgfqpoint{7.143837in}{0.668487in}}%
\pgfpathlineto{\pgfqpoint{7.172753in}{0.650382in}}%
\pgfpathlineto{\pgfqpoint{7.178880in}{0.646547in}}%
\pgfpathlineto{\pgfqpoint{7.213922in}{0.624747in}}%
\pgfpathlineto{\pgfqpoint{7.214930in}{0.624122in}}%
\pgfpathlineto{\pgfqpoint{7.248965in}{0.603026in}}%
\pgfpathlineto{\pgfqpoint{7.257350in}{0.597863in}}%
\pgfpathlineto{\pgfqpoint{7.284007in}{0.581441in}}%
\pgfpathlineto{\pgfqpoint{7.300088in}{0.571603in}}%
\pgfusepath{stroke}%
\end{pgfscope}%
\begin{pgfscope}%
\pgfpathrectangle{\pgfqpoint{0.766095in}{0.571603in}}{\pgfqpoint{6.973465in}{5.225635in}}%
\pgfusepath{clip}%
\pgfsetbuttcap%
\pgfsetroundjoin%
\pgfsetlinewidth{1.505625pt}%
\definecolor{currentstroke}{rgb}{0.280255,0.165693,0.476498}%
\pgfsetstrokecolor{currentstroke}%
\pgfsetdash{}{0pt}%
\pgfpathmoveto{\pgfqpoint{2.247058in}{0.571603in}}%
\pgfpathlineto{\pgfqpoint{2.155466in}{0.624122in}}%
\pgfpathlineto{\pgfqpoint{2.062669in}{0.680127in}}%
\pgfpathlineto{\pgfqpoint{1.985364in}{0.729160in}}%
\pgfpathlineto{\pgfqpoint{1.906630in}{0.781679in}}%
\pgfpathlineto{\pgfqpoint{1.817371in}{0.844786in}}%
\pgfpathlineto{\pgfqpoint{1.747286in}{0.897292in}}%
\pgfpathlineto{\pgfqpoint{1.677201in}{0.952849in}}%
\pgfpathlineto{\pgfqpoint{1.630456in}{0.991755in}}%
\pgfpathlineto{\pgfqpoint{1.570473in}{1.044274in}}%
\pgfpathlineto{\pgfqpoint{1.501988in}{1.108452in}}%
\pgfpathlineto{\pgfqpoint{1.460802in}{1.149312in}}%
\pgfpathlineto{\pgfqpoint{1.410853in}{1.201831in}}%
\pgfpathlineto{\pgfqpoint{1.361818in}{1.256835in}}%
\pgfpathlineto{\pgfqpoint{1.320139in}{1.306869in}}%
\pgfpathlineto{\pgfqpoint{1.279277in}{1.359388in}}%
\pgfpathlineto{\pgfqpoint{1.241257in}{1.411906in}}%
\pgfpathlineto{\pgfqpoint{1.206017in}{1.464425in}}%
\pgfpathlineto{\pgfqpoint{1.173483in}{1.516944in}}%
\pgfpathlineto{\pgfqpoint{1.143578in}{1.569463in}}%
\pgfpathlineto{\pgfqpoint{1.116216in}{1.621982in}}%
\pgfpathlineto{\pgfqpoint{1.091491in}{1.674501in}}%
\pgfpathlineto{\pgfqpoint{1.069175in}{1.727020in}}%
\pgfpathlineto{\pgfqpoint{1.046435in}{1.787540in}}%
\pgfpathlineto{\pgfqpoint{1.031729in}{1.832058in}}%
\pgfpathlineto{\pgfqpoint{1.016472in}{1.884577in}}%
\pgfpathlineto{\pgfqpoint{1.003560in}{1.937096in}}%
\pgfpathlineto{\pgfqpoint{0.992911in}{1.989615in}}%
\pgfpathlineto{\pgfqpoint{0.984439in}{2.042134in}}%
\pgfpathlineto{\pgfqpoint{0.978149in}{2.094653in}}%
\pgfpathlineto{\pgfqpoint{0.974090in}{2.147172in}}%
\pgfpathlineto{\pgfqpoint{0.972209in}{2.199691in}}%
\pgfpathlineto{\pgfqpoint{0.972469in}{2.252210in}}%
\pgfpathlineto{\pgfqpoint{0.974862in}{2.304729in}}%
\pgfpathlineto{\pgfqpoint{0.979442in}{2.357248in}}%
\pgfpathlineto{\pgfqpoint{0.986208in}{2.409766in}}%
\pgfpathlineto{\pgfqpoint{0.995115in}{2.462285in}}%
\pgfpathlineto{\pgfqpoint{1.006145in}{2.514804in}}%
\pgfpathlineto{\pgfqpoint{1.019440in}{2.567323in}}%
\pgfpathlineto{\pgfqpoint{1.034966in}{2.619842in}}%
\pgfpathlineto{\pgfqpoint{1.052716in}{2.672361in}}%
\pgfpathlineto{\pgfqpoint{1.072814in}{2.724880in}}%
\pgfpathlineto{\pgfqpoint{1.095272in}{2.777399in}}%
\pgfpathlineto{\pgfqpoint{1.120066in}{2.829918in}}%
\pgfpathlineto{\pgfqpoint{1.151563in}{2.890014in}}%
\pgfpathlineto{\pgfqpoint{1.186605in}{2.950450in}}%
\pgfpathlineto{\pgfqpoint{1.221648in}{3.005628in}}%
\pgfpathlineto{\pgfqpoint{1.256691in}{3.056538in}}%
\pgfpathlineto{\pgfqpoint{1.291733in}{3.103943in}}%
\pgfpathlineto{\pgfqpoint{1.326776in}{3.148437in}}%
\pgfpathlineto{\pgfqpoint{1.368095in}{3.197551in}}%
\pgfpathlineto{\pgfqpoint{1.415415in}{3.250070in}}%
\pgfpathlineto{\pgfqpoint{1.466946in}{3.303546in}}%
\pgfpathlineto{\pgfqpoint{1.520309in}{3.355107in}}%
\pgfpathlineto{\pgfqpoint{1.578274in}{3.407626in}}%
\pgfpathlineto{\pgfqpoint{1.642158in}{3.461701in}}%
\pgfpathlineto{\pgfqpoint{1.712244in}{3.516977in}}%
\pgfpathlineto{\pgfqpoint{1.782329in}{3.568644in}}%
\pgfpathlineto{\pgfqpoint{1.853286in}{3.617702in}}%
\pgfpathlineto{\pgfqpoint{1.934591in}{3.670221in}}%
\pgfpathlineto{\pgfqpoint{2.021629in}{3.722740in}}%
\pgfpathlineto{\pgfqpoint{2.097711in}{3.765703in}}%
\pgfpathlineto{\pgfqpoint{2.167797in}{3.803213in}}%
\pgfpathlineto{\pgfqpoint{2.272924in}{3.855839in}}%
\pgfpathlineto{\pgfqpoint{2.382485in}{3.906556in}}%
\pgfpathlineto{\pgfqpoint{2.483179in}{3.949736in}}%
\pgfpathlineto{\pgfqpoint{2.571798in}{3.985335in}}%
\pgfpathlineto{\pgfqpoint{2.658392in}{4.017980in}}%
\pgfpathlineto{\pgfqpoint{2.763520in}{4.054972in}}%
\pgfpathlineto{\pgfqpoint{2.868647in}{4.089295in}}%
\pgfpathlineto{\pgfqpoint{2.959137in}{4.116632in}}%
\pgfpathlineto{\pgfqpoint{3.052463in}{4.142892in}}%
\pgfpathlineto{\pgfqpoint{3.153705in}{4.169151in}}%
\pgfpathlineto{\pgfqpoint{3.265303in}{4.195411in}}%
\pgfpathlineto{\pgfqpoint{3.359243in}{4.215354in}}%
\pgfpathlineto{\pgfqpoint{3.464371in}{4.235326in}}%
\pgfpathlineto{\pgfqpoint{3.569498in}{4.252827in}}%
\pgfpathlineto{\pgfqpoint{3.674626in}{4.267644in}}%
\pgfpathlineto{\pgfqpoint{3.779753in}{4.279711in}}%
\pgfpathlineto{\pgfqpoint{3.884881in}{4.288767in}}%
\pgfpathlineto{\pgfqpoint{3.954966in}{4.293070in}}%
\pgfpathlineto{\pgfqpoint{4.025051in}{4.295840in}}%
\pgfpathlineto{\pgfqpoint{4.095136in}{4.296933in}}%
\pgfpathlineto{\pgfqpoint{4.165221in}{4.296181in}}%
\pgfpathlineto{\pgfqpoint{4.235306in}{4.293396in}}%
\pgfpathlineto{\pgfqpoint{4.305391in}{4.288355in}}%
\pgfpathlineto{\pgfqpoint{4.375477in}{4.280802in}}%
\pgfpathlineto{\pgfqpoint{4.445562in}{4.270285in}}%
\pgfpathlineto{\pgfqpoint{4.515647in}{4.256188in}}%
\pgfpathlineto{\pgfqpoint{4.550689in}{4.247751in}}%
\pgfpathlineto{\pgfqpoint{4.620774in}{4.226936in}}%
\pgfpathlineto{\pgfqpoint{4.655817in}{4.214435in}}%
\pgfpathlineto{\pgfqpoint{4.702075in}{4.195411in}}%
\pgfpathlineto{\pgfqpoint{4.725902in}{4.184208in}}%
\pgfpathlineto{\pgfqpoint{4.760944in}{4.166012in}}%
\pgfpathlineto{\pgfqpoint{4.799474in}{4.142892in}}%
\pgfpathlineto{\pgfqpoint{4.836955in}{4.116632in}}%
\pgfpathlineto{\pgfqpoint{4.869309in}{4.090373in}}%
\pgfpathlineto{\pgfqpoint{4.901115in}{4.060408in}}%
\pgfpathlineto{\pgfqpoint{4.936157in}{4.021363in}}%
\pgfpathlineto{\pgfqpoint{4.944035in}{4.011594in}}%
\pgfpathlineto{\pgfqpoint{4.971200in}{3.974044in}}%
\pgfpathlineto{\pgfqpoint{4.980907in}{3.959075in}}%
\pgfpathlineto{\pgfqpoint{5.006242in}{3.915072in}}%
\pgfpathlineto{\pgfqpoint{5.023402in}{3.880297in}}%
\pgfpathlineto{\pgfqpoint{5.045651in}{3.827778in}}%
\pgfpathlineto{\pgfqpoint{5.064213in}{3.775259in}}%
\pgfpathlineto{\pgfqpoint{5.080123in}{3.722740in}}%
\pgfpathlineto{\pgfqpoint{5.093844in}{3.670221in}}%
\pgfpathlineto{\pgfqpoint{5.111808in}{3.591443in}}%
\pgfpathlineto{\pgfqpoint{5.117173in}{3.565304in}}%
\pgfpathlineto{\pgfqpoint{5.117173in}{3.565304in}}%
\pgfusepath{stroke}%
\end{pgfscope}%
\begin{pgfscope}%
\pgfpathrectangle{\pgfqpoint{0.766095in}{0.571603in}}{\pgfqpoint{6.973465in}{5.225635in}}%
\pgfusepath{clip}%
\pgfsetbuttcap%
\pgfsetroundjoin%
\pgfsetlinewidth{1.505625pt}%
\definecolor{currentstroke}{rgb}{0.280255,0.165693,0.476498}%
\pgfsetstrokecolor{currentstroke}%
\pgfsetdash{}{0pt}%
\pgfpathmoveto{\pgfqpoint{5.175520in}{3.257880in}}%
\pgfpathlineto{\pgfqpoint{5.193604in}{3.171291in}}%
\pgfpathlineto{\pgfqpoint{5.211847in}{3.092513in}}%
\pgfpathlineto{\pgfqpoint{5.232048in}{3.013734in}}%
\pgfpathlineto{\pgfqpoint{5.254662in}{2.934956in}}%
\pgfpathlineto{\pgfqpoint{5.279811in}{2.856177in}}%
\pgfpathlineto{\pgfqpoint{5.307785in}{2.777399in}}%
\pgfpathlineto{\pgfqpoint{5.338797in}{2.698621in}}%
\pgfpathlineto{\pgfqpoint{5.372998in}{2.619842in}}%
\pgfpathlineto{\pgfqpoint{5.410528in}{2.541064in}}%
\pgfpathlineto{\pgfqpoint{5.451518in}{2.462285in}}%
\pgfpathlineto{\pgfqpoint{5.496838in}{2.382271in}}%
\pgfpathlineto{\pgfqpoint{5.531880in}{2.324495in}}%
\pgfpathlineto{\pgfqpoint{5.566923in}{2.269714in}}%
\pgfpathlineto{\pgfqpoint{5.601965in}{2.217549in}}%
\pgfpathlineto{\pgfqpoint{5.637008in}{2.167680in}}%
\pgfpathlineto{\pgfqpoint{5.672050in}{2.119831in}}%
\pgfpathlineto{\pgfqpoint{5.711297in}{2.068393in}}%
\pgfpathlineto{\pgfqpoint{5.774656in}{1.989615in}}%
\pgfpathlineto{\pgfqpoint{5.847263in}{1.904787in}}%
\pgfpathlineto{\pgfqpoint{5.913035in}{1.832058in}}%
\pgfpathlineto{\pgfqpoint{5.988123in}{1.753280in}}%
\pgfpathlineto{\pgfqpoint{6.067076in}{1.674501in}}%
\pgfpathlineto{\pgfqpoint{6.149980in}{1.595723in}}%
\pgfpathlineto{\pgfqpoint{6.236826in}{1.516944in}}%
\pgfpathlineto{\pgfqpoint{6.337859in}{1.429436in}}%
\pgfpathlineto{\pgfqpoint{6.422082in}{1.359388in}}%
\pgfpathlineto{\pgfqpoint{6.520507in}{1.280609in}}%
\pgfpathlineto{\pgfqpoint{6.622726in}{1.201831in}}%
\pgfpathlineto{\pgfqpoint{6.728681in}{1.123052in}}%
\pgfpathlineto{\pgfqpoint{6.838326in}{1.044274in}}%
\pgfpathlineto{\pgfqpoint{6.968624in}{0.953896in}}%
\pgfpathlineto{\pgfqpoint{7.073752in}{0.883247in}}%
\pgfpathlineto{\pgfqpoint{7.213922in}{0.791814in}}%
\pgfpathlineto{\pgfqpoint{7.354679in}{0.702901in}}%
\pgfpathlineto{\pgfqpoint{7.494263in}{0.617165in}}%
\pgfpathlineto{\pgfqpoint{7.570022in}{0.571603in}}%
\pgfpathlineto{\pgfqpoint{7.570022in}{0.571603in}}%
\pgfusepath{stroke}%
\end{pgfscope}%
\begin{pgfscope}%
\pgfpathrectangle{\pgfqpoint{0.766095in}{0.571603in}}{\pgfqpoint{6.973465in}{5.225635in}}%
\pgfusepath{clip}%
\pgfsetbuttcap%
\pgfsetroundjoin%
\pgfsetlinewidth{1.505625pt}%
\definecolor{currentstroke}{rgb}{0.276194,0.190074,0.493001}%
\pgfsetstrokecolor{currentstroke}%
\pgfsetdash{}{0pt}%
\pgfpathmoveto{\pgfqpoint{2.037051in}{0.571603in}}%
\pgfpathlineto{\pgfqpoint{1.947838in}{0.624122in}}%
\pgfpathlineto{\pgfqpoint{1.852414in}{0.683341in}}%
\pgfpathlineto{\pgfqpoint{1.781985in}{0.729160in}}%
\pgfpathlineto{\pgfqpoint{1.705177in}{0.781679in}}%
\pgfpathlineto{\pgfqpoint{1.632204in}{0.834198in}}%
\pgfpathlineto{\pgfqpoint{1.562958in}{0.886717in}}%
\pgfpathlineto{\pgfqpoint{1.497318in}{0.939236in}}%
\pgfpathlineto{\pgfqpoint{1.431903in}{0.994653in}}%
\pgfpathlineto{\pgfqpoint{1.361818in}{1.057987in}}%
\pgfpathlineto{\pgfqpoint{1.321102in}{1.096793in}}%
\pgfpathlineto{\pgfqpoint{1.256691in}{1.162138in}}%
\pgfpathlineto{\pgfqpoint{1.219750in}{1.201831in}}%
\pgfpathlineto{\pgfqpoint{1.173697in}{1.254350in}}%
\pgfpathlineto{\pgfqpoint{1.130512in}{1.306869in}}%
\pgfpathlineto{\pgfqpoint{1.090144in}{1.359388in}}%
\pgfpathlineto{\pgfqpoint{1.046435in}{1.420870in}}%
\pgfpathlineto{\pgfqpoint{1.011393in}{1.474310in}}%
\pgfpathlineto{\pgfqpoint{0.976350in}{1.532397in}}%
\pgfpathlineto{\pgfqpoint{0.955595in}{1.569463in}}%
\pgfpathlineto{\pgfqpoint{0.928318in}{1.621982in}}%
\pgfpathlineto{\pgfqpoint{0.903418in}{1.674501in}}%
\pgfpathlineto{\pgfqpoint{0.880976in}{1.727020in}}%
\pgfpathlineto{\pgfqpoint{0.860818in}{1.779539in}}%
\pgfpathlineto{\pgfqpoint{0.842915in}{1.832058in}}%
\pgfpathlineto{\pgfqpoint{0.827273in}{1.884577in}}%
\pgfpathlineto{\pgfqpoint{0.813827in}{1.937096in}}%
\pgfpathlineto{\pgfqpoint{0.801138in}{1.996779in}}%
\pgfpathlineto{\pgfqpoint{0.793349in}{2.042134in}}%
\pgfpathlineto{\pgfqpoint{0.786315in}{2.094653in}}%
\pgfpathlineto{\pgfqpoint{0.781346in}{2.147172in}}%
\pgfpathlineto{\pgfqpoint{0.778441in}{2.199691in}}%
\pgfpathlineto{\pgfqpoint{0.777596in}{2.252210in}}%
\pgfpathlineto{\pgfqpoint{0.778808in}{2.304729in}}%
\pgfpathlineto{\pgfqpoint{0.782068in}{2.357248in}}%
\pgfpathlineto{\pgfqpoint{0.787367in}{2.409766in}}%
\pgfpathlineto{\pgfqpoint{0.794694in}{2.462285in}}%
\pgfpathlineto{\pgfqpoint{0.804091in}{2.514804in}}%
\pgfpathlineto{\pgfqpoint{0.815649in}{2.567323in}}%
\pgfpathlineto{\pgfqpoint{0.829225in}{2.619842in}}%
\pgfpathlineto{\pgfqpoint{0.844968in}{2.672361in}}%
\pgfpathlineto{\pgfqpoint{0.862860in}{2.724880in}}%
\pgfpathlineto{\pgfqpoint{0.882967in}{2.777399in}}%
\pgfpathlineto{\pgfqpoint{0.906265in}{2.832189in}}%
\pgfpathlineto{\pgfqpoint{0.929936in}{2.882437in}}%
\pgfpathlineto{\pgfqpoint{0.956932in}{2.934956in}}%
\pgfpathlineto{\pgfqpoint{0.986322in}{2.987475in}}%
\pgfpathlineto{\pgfqpoint{1.018202in}{3.039994in}}%
\pgfpathlineto{\pgfqpoint{1.052671in}{3.092513in}}%
\pgfpathlineto{\pgfqpoint{1.089823in}{3.145032in}}%
\pgfpathlineto{\pgfqpoint{1.129751in}{3.197551in}}%
\pgfpathlineto{\pgfqpoint{1.172542in}{3.250070in}}%
\pgfpathlineto{\pgfqpoint{1.221648in}{3.306304in}}%
\pgfpathlineto{\pgfqpoint{1.267267in}{3.355107in}}%
\pgfpathlineto{\pgfqpoint{1.326776in}{3.414679in}}%
\pgfpathlineto{\pgfqpoint{1.375222in}{3.460145in}}%
\pgfpathlineto{\pgfqpoint{1.434524in}{3.512664in}}%
\pgfpathlineto{\pgfqpoint{1.501988in}{3.568561in}}%
\pgfpathlineto{\pgfqpoint{1.572073in}{3.622917in}}%
\pgfpathlineto{\pgfqpoint{1.642158in}{3.673974in}}%
\pgfpathlineto{\pgfqpoint{1.713205in}{3.722740in}}%
\pgfpathlineto{\pgfqpoint{1.794710in}{3.775259in}}%
\pgfpathlineto{\pgfqpoint{1.881519in}{3.827778in}}%
\pgfpathlineto{\pgfqpoint{1.957541in}{3.871041in}}%
\pgfpathlineto{\pgfqpoint{2.027626in}{3.909011in}}%
\pgfpathlineto{\pgfqpoint{2.132754in}{3.962591in}}%
\pgfpathlineto{\pgfqpoint{2.237882in}{4.012621in}}%
\pgfpathlineto{\pgfqpoint{2.354149in}{4.064113in}}%
\pgfpathlineto{\pgfqpoint{2.448137in}{4.103060in}}%
\pgfpathlineto{\pgfqpoint{2.553264in}{4.144149in}}%
\pgfpathlineto{\pgfqpoint{2.658392in}{4.182528in}}%
\pgfpathlineto{\pgfqpoint{2.772699in}{4.221670in}}%
\pgfpathlineto{\pgfqpoint{2.868647in}{4.252386in}}%
\pgfpathlineto{\pgfqpoint{2.973775in}{4.283951in}}%
\pgfpathlineto{\pgfqpoint{3.078903in}{4.313458in}}%
\pgfpathlineto{\pgfqpoint{3.184030in}{4.340948in}}%
\pgfpathlineto{\pgfqpoint{3.289158in}{4.366460in}}%
\pgfpathlineto{\pgfqpoint{3.394285in}{4.390026in}}%
\pgfpathlineto{\pgfqpoint{3.499413in}{4.411673in}}%
\pgfpathlineto{\pgfqpoint{3.606420in}{4.431746in}}%
\pgfpathlineto{\pgfqpoint{3.744711in}{4.454515in}}%
\pgfpathlineto{\pgfqpoint{3.849838in}{4.469424in}}%
\pgfpathlineto{\pgfqpoint{3.972823in}{4.484265in}}%
\pgfpathlineto{\pgfqpoint{4.060094in}{4.492804in}}%
\pgfpathlineto{\pgfqpoint{4.165221in}{4.500914in}}%
\pgfpathlineto{\pgfqpoint{4.270349in}{4.506423in}}%
\pgfpathlineto{\pgfqpoint{4.375477in}{4.508978in}}%
\pgfpathlineto{\pgfqpoint{4.445562in}{4.508832in}}%
\pgfpathlineto{\pgfqpoint{4.515647in}{4.507034in}}%
\pgfpathlineto{\pgfqpoint{4.585732in}{4.503412in}}%
\pgfpathlineto{\pgfqpoint{4.655817in}{4.497766in}}%
\pgfpathlineto{\pgfqpoint{4.725902in}{4.489863in}}%
\pgfpathlineto{\pgfqpoint{4.795987in}{4.479219in}}%
\pgfpathlineto{\pgfqpoint{4.866072in}{4.465331in}}%
\pgfpathlineto{\pgfqpoint{4.901115in}{4.457094in}}%
\pgfpathlineto{\pgfqpoint{4.971200in}{4.436973in}}%
\pgfpathlineto{\pgfqpoint{5.006242in}{4.424956in}}%
\pgfpathlineto{\pgfqpoint{5.055375in}{4.405486in}}%
\pgfpathlineto{\pgfqpoint{5.076327in}{4.395976in}}%
\pgfpathlineto{\pgfqpoint{5.111370in}{4.378596in}}%
\pgfpathlineto{\pgfqpoint{5.155194in}{4.352967in}}%
\pgfpathlineto{\pgfqpoint{5.193097in}{4.326708in}}%
\pgfpathlineto{\pgfqpoint{5.225422in}{4.300449in}}%
\pgfpathlineto{\pgfqpoint{5.253276in}{4.274189in}}%
\pgfpathlineto{\pgfqpoint{5.286583in}{4.236583in}}%
\pgfpathlineto{\pgfqpoint{5.298190in}{4.221670in}}%
\pgfpathlineto{\pgfqpoint{5.321625in}{4.187315in}}%
\pgfpathlineto{\pgfqpoint{5.332516in}{4.169151in}}%
\pgfpathlineto{\pgfqpoint{5.356668in}{4.121771in}}%
\pgfpathlineto{\pgfqpoint{5.369825in}{4.090373in}}%
\pgfpathlineto{\pgfqpoint{5.387880in}{4.037854in}}%
\pgfpathlineto{\pgfqpoint{5.395300in}{4.011594in}}%
\pgfpathlineto{\pgfqpoint{5.407517in}{3.959075in}}%
\pgfpathlineto{\pgfqpoint{5.417029in}{3.906556in}}%
\pgfpathlineto{\pgfqpoint{5.426753in}{3.834214in}}%
\pgfpathlineto{\pgfqpoint{5.432535in}{3.775259in}}%
\pgfpathlineto{\pgfqpoint{5.439991in}{3.670221in}}%
\pgfpathlineto{\pgfqpoint{5.457525in}{3.381367in}}%
\pgfpathlineto{\pgfqpoint{5.467168in}{3.276329in}}%
\pgfpathlineto{\pgfqpoint{5.476566in}{3.197551in}}%
\pgfpathlineto{\pgfqpoint{5.488253in}{3.118772in}}%
\pgfpathlineto{\pgfqpoint{5.502462in}{3.039994in}}%
\pgfpathlineto{\pgfqpoint{5.519406in}{2.961215in}}%
\pgfpathlineto{\pgfqpoint{5.539365in}{2.882437in}}%
\pgfpathlineto{\pgfqpoint{5.562482in}{2.803659in}}%
\pgfpathlineto{\pgfqpoint{5.588877in}{2.724880in}}%
\pgfpathlineto{\pgfqpoint{5.618753in}{2.646102in}}%
\pgfpathlineto{\pgfqpoint{5.652208in}{2.567323in}}%
\pgfpathlineto{\pgfqpoint{5.689339in}{2.488545in}}%
\pgfpathlineto{\pgfqpoint{5.730242in}{2.409766in}}%
\pgfpathlineto{\pgfqpoint{5.775015in}{2.330988in}}%
\pgfpathlineto{\pgfqpoint{5.823624in}{2.252210in}}%
\pgfpathlineto{\pgfqpoint{5.876239in}{2.173431in}}%
\pgfpathlineto{\pgfqpoint{5.932817in}{2.094653in}}%
\pgfpathlineto{\pgfqpoint{5.993471in}{2.015874in}}%
\pgfpathlineto{\pgfqpoint{6.058188in}{1.937096in}}%
\pgfpathlineto{\pgfqpoint{6.103542in}{1.884577in}}%
\pgfpathlineto{\pgfqpoint{6.175035in}{1.805799in}}%
\pgfpathlineto{\pgfqpoint{6.250620in}{1.727020in}}%
\pgfpathlineto{\pgfqpoint{6.330317in}{1.648242in}}%
\pgfpathlineto{\pgfqpoint{6.414091in}{1.569463in}}%
\pgfpathlineto{\pgfqpoint{6.501922in}{1.490685in}}%
\pgfpathlineto{\pgfqpoint{6.593828in}{1.411906in}}%
\pgfpathlineto{\pgfqpoint{6.689801in}{1.333128in}}%
\pgfpathlineto{\pgfqpoint{6.793412in}{1.251530in}}%
\pgfpathlineto{\pgfqpoint{6.898539in}{1.171961in}}%
\pgfpathlineto{\pgfqpoint{7.003667in}{1.095268in}}%
\pgfpathlineto{\pgfqpoint{7.113272in}{1.018014in}}%
\pgfpathlineto{\pgfqpoint{7.228836in}{0.939236in}}%
\pgfpathlineto{\pgfqpoint{7.340065in}{0.865780in}}%
\pgfpathlineto{\pgfqpoint{7.340065in}{0.865780in}}%
\pgfusepath{stroke}%
\end{pgfscope}%
\begin{pgfscope}%
\pgfpathrectangle{\pgfqpoint{0.766095in}{0.571603in}}{\pgfqpoint{6.973465in}{5.225635in}}%
\pgfusepath{clip}%
\pgfsetbuttcap%
\pgfsetroundjoin%
\pgfsetlinewidth{1.505625pt}%
\definecolor{currentstroke}{rgb}{0.276194,0.190074,0.493001}%
\pgfsetstrokecolor{currentstroke}%
\pgfsetdash{}{0pt}%
\pgfpathmoveto{\pgfqpoint{7.603691in}{0.699455in}}%
\pgfpathlineto{\pgfqpoint{7.634433in}{0.680676in}}%
\pgfpathlineto{\pgfqpoint{7.641068in}{0.676641in}}%
\pgfpathlineto{\pgfqpoint{7.669475in}{0.659426in}}%
\pgfpathlineto{\pgfqpoint{7.684479in}{0.650382in}}%
\pgfpathlineto{\pgfqpoint{7.704518in}{0.638338in}}%
\pgfpathlineto{\pgfqpoint{7.728302in}{0.624122in}}%
\pgfpathlineto{\pgfqpoint{7.739560in}{0.617410in}}%
\pgfusepath{stroke}%
\end{pgfscope}%
\begin{pgfscope}%
\pgfpathrectangle{\pgfqpoint{0.766095in}{0.571603in}}{\pgfqpoint{6.973465in}{5.225635in}}%
\pgfusepath{clip}%
\pgfsetbuttcap%
\pgfsetroundjoin%
\pgfsetlinewidth{1.505625pt}%
\definecolor{currentstroke}{rgb}{0.270595,0.214069,0.507052}%
\pgfsetstrokecolor{currentstroke}%
\pgfsetdash{}{0pt}%
\pgfpathmoveto{\pgfqpoint{1.844905in}{0.571603in}}%
\pgfpathlineto{\pgfqpoint{1.817371in}{0.588003in}}%
\pgfpathlineto{\pgfqpoint{1.809501in}{0.592698in}}%
\pgfusepath{stroke}%
\end{pgfscope}%
\begin{pgfscope}%
\pgfpathrectangle{\pgfqpoint{0.766095in}{0.571603in}}{\pgfqpoint{6.973465in}{5.225635in}}%
\pgfusepath{clip}%
\pgfsetbuttcap%
\pgfsetroundjoin%
\pgfsetlinewidth{1.505625pt}%
\definecolor{currentstroke}{rgb}{0.270595,0.214069,0.507052}%
\pgfsetstrokecolor{currentstroke}%
\pgfsetdash{}{0pt}%
\pgfpathmoveto{\pgfqpoint{1.548015in}{0.762294in}}%
\pgfpathlineto{\pgfqpoint{1.537031in}{0.770056in}}%
\pgfpathlineto{\pgfqpoint{1.520643in}{0.781679in}}%
\pgfpathlineto{\pgfqpoint{1.501988in}{0.795200in}}%
\pgfpathlineto{\pgfqpoint{1.484479in}{0.807939in}}%
\pgfpathlineto{\pgfqpoint{1.466946in}{0.820975in}}%
\pgfpathlineto{\pgfqpoint{1.449235in}{0.834198in}}%
\pgfpathlineto{\pgfqpoint{1.431903in}{0.847424in}}%
\pgfpathlineto{\pgfqpoint{1.414897in}{0.860458in}}%
\pgfpathlineto{\pgfqpoint{1.396861in}{0.874588in}}%
\pgfpathlineto{\pgfqpoint{1.381450in}{0.886717in}}%
\pgfpathlineto{\pgfqpoint{1.361818in}{0.902513in}}%
\pgfpathlineto{\pgfqpoint{1.348877in}{0.912976in}}%
\pgfpathlineto{\pgfqpoint{1.326776in}{0.931248in}}%
\pgfpathlineto{\pgfqpoint{1.317162in}{0.939236in}}%
\pgfpathlineto{\pgfqpoint{1.291733in}{0.960844in}}%
\pgfpathlineto{\pgfqpoint{1.286288in}{0.965495in}}%
\pgfpathlineto{\pgfqpoint{1.256691in}{0.991356in}}%
\pgfpathlineto{\pgfqpoint{1.256236in}{0.991755in}}%
\pgfpathlineto{\pgfqpoint{1.227073in}{1.018014in}}%
\pgfpathlineto{\pgfqpoint{1.221648in}{1.023013in}}%
\pgfpathlineto{\pgfqpoint{1.198714in}{1.044274in}}%
\pgfpathlineto{\pgfqpoint{1.186605in}{1.055761in}}%
\pgfpathlineto{\pgfqpoint{1.171133in}{1.070533in}}%
\pgfpathlineto{\pgfqpoint{1.151563in}{1.089657in}}%
\pgfpathlineto{\pgfqpoint{1.144309in}{1.096793in}}%
\pgfpathlineto{\pgfqpoint{1.118248in}{1.123052in}}%
\pgfpathlineto{\pgfqpoint{1.116520in}{1.124840in}}%
\pgfpathlineto{\pgfqpoint{1.093031in}{1.149312in}}%
\pgfpathlineto{\pgfqpoint{1.081478in}{1.161641in}}%
\pgfpathlineto{\pgfqpoint{1.068518in}{1.175571in}}%
\pgfpathlineto{\pgfqpoint{1.046435in}{1.199893in}}%
\pgfpathlineto{\pgfqpoint{1.044689in}{1.201831in}}%
\pgfpathlineto{\pgfqpoint{1.021685in}{1.228090in}}%
\pgfpathlineto{\pgfqpoint{1.011393in}{1.240144in}}%
\pgfpathlineto{\pgfqpoint{0.999358in}{1.254350in}}%
\pgfpathlineto{\pgfqpoint{0.977679in}{1.280609in}}%
\pgfpathlineto{\pgfqpoint{0.976350in}{1.282268in}}%
\pgfpathlineto{\pgfqpoint{0.956810in}{1.306869in}}%
\pgfpathlineto{\pgfqpoint{0.941308in}{1.326901in}}%
\pgfpathlineto{\pgfqpoint{0.936530in}{1.333128in}}%
\pgfpathlineto{\pgfqpoint{0.916990in}{1.359388in}}%
\pgfpathlineto{\pgfqpoint{0.906265in}{1.374224in}}%
\pgfpathlineto{\pgfqpoint{0.898081in}{1.385647in}}%
\pgfpathlineto{\pgfqpoint{0.879841in}{1.411906in}}%
\pgfpathlineto{\pgfqpoint{0.871223in}{1.424710in}}%
\pgfpathlineto{\pgfqpoint{0.862250in}{1.438166in}}%
\pgfpathlineto{\pgfqpoint{0.845297in}{1.464425in}}%
\pgfpathlineto{\pgfqpoint{0.836180in}{1.479023in}}%
\pgfpathlineto{\pgfqpoint{0.828968in}{1.490685in}}%
\pgfpathlineto{\pgfqpoint{0.813288in}{1.516944in}}%
\pgfpathlineto{\pgfqpoint{0.801138in}{1.537995in}}%
\pgfpathlineto{\pgfqpoint{0.798162in}{1.543204in}}%
\pgfpathlineto{\pgfqpoint{0.783737in}{1.569463in}}%
\pgfpathlineto{\pgfqpoint{0.769816in}{1.595723in}}%
\pgfpathlineto{\pgfqpoint{0.766095in}{1.603039in}}%
\pgfusepath{stroke}%
\end{pgfscope}%
\begin{pgfscope}%
\pgfpathrectangle{\pgfqpoint{0.766095in}{0.571603in}}{\pgfqpoint{6.973465in}{5.225635in}}%
\pgfusepath{clip}%
\pgfsetbuttcap%
\pgfsetroundjoin%
\pgfsetlinewidth{1.505625pt}%
\definecolor{currentstroke}{rgb}{0.270595,0.214069,0.507052}%
\pgfsetstrokecolor{currentstroke}%
\pgfsetdash{}{0pt}%
\pgfpathmoveto{\pgfqpoint{0.766095in}{2.944414in}}%
\pgfpathlineto{\pgfqpoint{0.788505in}{2.987475in}}%
\pgfpathlineto{\pgfqpoint{0.817950in}{3.039994in}}%
\pgfpathlineto{\pgfqpoint{0.849788in}{3.092513in}}%
\pgfpathlineto{\pgfqpoint{0.884109in}{3.145032in}}%
\pgfpathlineto{\pgfqpoint{0.921001in}{3.197551in}}%
\pgfpathlineto{\pgfqpoint{0.960550in}{3.250070in}}%
\pgfpathlineto{\pgfqpoint{1.002840in}{3.302589in}}%
\pgfpathlineto{\pgfqpoint{1.047978in}{3.355107in}}%
\pgfpathlineto{\pgfqpoint{1.096237in}{3.407626in}}%
\pgfpathlineto{\pgfqpoint{1.151563in}{3.464138in}}%
\pgfpathlineto{\pgfqpoint{1.202185in}{3.512664in}}%
\pgfpathlineto{\pgfqpoint{1.260203in}{3.565183in}}%
\pgfpathlineto{\pgfqpoint{1.326776in}{3.621688in}}%
\pgfpathlineto{\pgfqpoint{1.396861in}{3.677449in}}%
\pgfpathlineto{\pgfqpoint{1.466946in}{3.729884in}}%
\pgfpathlineto{\pgfqpoint{1.537031in}{3.779371in}}%
\pgfpathlineto{\pgfqpoint{1.609536in}{3.827778in}}%
\pgfpathlineto{\pgfqpoint{1.693038in}{3.880297in}}%
\pgfpathlineto{\pgfqpoint{1.782329in}{3.933205in}}%
\pgfpathlineto{\pgfqpoint{1.887456in}{3.991426in}}%
\pgfpathlineto{\pgfqpoint{1.976609in}{4.037854in}}%
\pgfpathlineto{\pgfqpoint{2.062669in}{4.080254in}}%
\pgfpathlineto{\pgfqpoint{2.140063in}{4.116632in}}%
\pgfpathlineto{\pgfqpoint{2.258563in}{4.169151in}}%
\pgfpathlineto{\pgfqpoint{2.343009in}{4.204441in}}%
\pgfpathlineto{\pgfqpoint{2.452591in}{4.247930in}}%
\pgfpathlineto{\pgfqpoint{2.588307in}{4.298206in}}%
\pgfpathlineto{\pgfqpoint{2.693435in}{4.334639in}}%
\pgfpathlineto{\pgfqpoint{2.798562in}{4.369093in}}%
\pgfpathlineto{\pgfqpoint{2.916339in}{4.405486in}}%
\pgfpathlineto{\pgfqpoint{3.043860in}{4.442211in}}%
\pgfpathlineto{\pgfqpoint{3.148988in}{4.470607in}}%
\pgfpathlineto{\pgfqpoint{3.254115in}{4.497328in}}%
\pgfpathlineto{\pgfqpoint{3.359243in}{4.522414in}}%
\pgfpathlineto{\pgfqpoint{3.464371in}{4.545901in}}%
\pgfpathlineto{\pgfqpoint{3.604541in}{4.574729in}}%
\pgfpathlineto{\pgfqpoint{3.744711in}{4.600745in}}%
\pgfpathlineto{\pgfqpoint{3.884881in}{4.623921in}}%
\pgfpathlineto{\pgfqpoint{4.025051in}{4.644209in}}%
\pgfpathlineto{\pgfqpoint{4.165221in}{4.661323in}}%
\pgfpathlineto{\pgfqpoint{4.270349in}{4.672086in}}%
\pgfpathlineto{\pgfqpoint{4.375477in}{4.680807in}}%
\pgfpathlineto{\pgfqpoint{4.480604in}{4.687499in}}%
\pgfpathlineto{\pgfqpoint{4.585732in}{4.691906in}}%
\pgfpathlineto{\pgfqpoint{4.690859in}{4.693724in}}%
\pgfpathlineto{\pgfqpoint{4.795987in}{4.692581in}}%
\pgfpathlineto{\pgfqpoint{4.866072in}{4.689955in}}%
\pgfpathlineto{\pgfqpoint{4.936157in}{4.685655in}}%
\pgfpathlineto{\pgfqpoint{5.006242in}{4.679500in}}%
\pgfpathlineto{\pgfqpoint{5.076327in}{4.671275in}}%
\pgfpathlineto{\pgfqpoint{5.146412in}{4.660387in}}%
\pgfpathlineto{\pgfqpoint{5.216497in}{4.646585in}}%
\pgfpathlineto{\pgfqpoint{5.251540in}{4.638364in}}%
\pgfpathlineto{\pgfqpoint{5.321625in}{4.618696in}}%
\pgfpathlineto{\pgfqpoint{5.356668in}{4.606899in}}%
\pgfpathlineto{\pgfqpoint{5.402653in}{4.589303in}}%
\pgfpathlineto{\pgfqpoint{5.426753in}{4.578689in}}%
\pgfpathlineto{\pgfqpoint{5.461795in}{4.561788in}}%
\pgfpathlineto{\pgfqpoint{5.505708in}{4.536784in}}%
\pgfpathlineto{\pgfqpoint{5.544361in}{4.510524in}}%
\pgfpathlineto{\pgfqpoint{5.577003in}{4.484265in}}%
\pgfpathlineto{\pgfqpoint{5.604812in}{4.458005in}}%
\pgfpathlineto{\pgfqpoint{5.637008in}{4.421078in}}%
\pgfpathlineto{\pgfqpoint{5.648814in}{4.405486in}}%
\pgfpathlineto{\pgfqpoint{5.672050in}{4.369491in}}%
\pgfpathlineto{\pgfqpoint{5.681308in}{4.352967in}}%
\pgfpathlineto{\pgfqpoint{5.694166in}{4.326708in}}%
\pgfpathlineto{\pgfqpoint{5.707093in}{4.295421in}}%
\pgfpathlineto{\pgfqpoint{5.714578in}{4.274189in}}%
\pgfpathlineto{\pgfqpoint{5.722494in}{4.247930in}}%
\pgfpathlineto{\pgfqpoint{5.734754in}{4.195411in}}%
\pgfpathlineto{\pgfqpoint{5.743045in}{4.142892in}}%
\pgfpathlineto{\pgfqpoint{5.748127in}{4.090373in}}%
\pgfpathlineto{\pgfqpoint{5.750785in}{4.037854in}}%
\pgfpathlineto{\pgfqpoint{5.751559in}{3.985335in}}%
\pgfpathlineto{\pgfqpoint{5.750160in}{3.906556in}}%
\pgfpathlineto{\pgfqpoint{5.749142in}{3.879002in}}%
\pgfpathlineto{\pgfqpoint{5.749142in}{3.879002in}}%
\pgfusepath{stroke}%
\end{pgfscope}%
\begin{pgfscope}%
\pgfpathrectangle{\pgfqpoint{0.766095in}{0.571603in}}{\pgfqpoint{6.973465in}{5.225635in}}%
\pgfusepath{clip}%
\pgfsetbuttcap%
\pgfsetroundjoin%
\pgfsetlinewidth{1.505625pt}%
\definecolor{currentstroke}{rgb}{0.270595,0.214069,0.507052}%
\pgfsetstrokecolor{currentstroke}%
\pgfsetdash{}{0pt}%
\pgfpathmoveto{\pgfqpoint{5.732355in}{3.566485in}}%
\pgfpathlineto{\pgfqpoint{5.729502in}{3.486405in}}%
\pgfpathlineto{\pgfqpoint{5.728289in}{3.407626in}}%
\pgfpathlineto{\pgfqpoint{5.729027in}{3.328848in}}%
\pgfpathlineto{\pgfqpoint{5.732038in}{3.250070in}}%
\pgfpathlineto{\pgfqpoint{5.737609in}{3.171291in}}%
\pgfpathlineto{\pgfqpoint{5.745949in}{3.092513in}}%
\pgfpathlineto{\pgfqpoint{5.757252in}{3.013734in}}%
\pgfpathlineto{\pgfqpoint{5.771795in}{2.934956in}}%
\pgfpathlineto{\pgfqpoint{5.789638in}{2.856177in}}%
\pgfpathlineto{\pgfqpoint{5.803485in}{2.803659in}}%
\pgfpathlineto{\pgfqpoint{5.827220in}{2.724880in}}%
\pgfpathlineto{\pgfqpoint{5.847263in}{2.666539in}}%
\pgfpathlineto{\pgfqpoint{5.864633in}{2.619842in}}%
\pgfpathlineto{\pgfqpoint{5.885902in}{2.567323in}}%
\pgfpathlineto{\pgfqpoint{5.917348in}{2.496476in}}%
\pgfpathlineto{\pgfqpoint{5.933534in}{2.462285in}}%
\pgfpathlineto{\pgfqpoint{5.960017in}{2.409766in}}%
\pgfpathlineto{\pgfqpoint{5.988295in}{2.357248in}}%
\pgfpathlineto{\pgfqpoint{6.022476in}{2.297827in}}%
\pgfpathlineto{\pgfqpoint{6.057518in}{2.240662in}}%
\pgfpathlineto{\pgfqpoint{6.092561in}{2.186722in}}%
\pgfpathlineto{\pgfqpoint{6.127603in}{2.135543in}}%
\pgfpathlineto{\pgfqpoint{6.162646in}{2.086743in}}%
\pgfpathlineto{\pgfqpoint{6.197689in}{2.040009in}}%
\pgfpathlineto{\pgfqpoint{6.237126in}{1.989615in}}%
\pgfpathlineto{\pgfqpoint{6.280072in}{1.937096in}}%
\pgfpathlineto{\pgfqpoint{6.348043in}{1.858318in}}%
\pgfpathlineto{\pgfqpoint{6.420231in}{1.779539in}}%
\pgfpathlineto{\pgfqpoint{6.496649in}{1.700761in}}%
\pgfpathlineto{\pgfqpoint{6.577321in}{1.621982in}}%
\pgfpathlineto{\pgfqpoint{6.662185in}{1.543204in}}%
\pgfpathlineto{\pgfqpoint{6.758369in}{1.458348in}}%
\pgfpathlineto{\pgfqpoint{6.844542in}{1.385647in}}%
\pgfpathlineto{\pgfqpoint{6.942010in}{1.306869in}}%
\pgfpathlineto{\pgfqpoint{7.043628in}{1.228090in}}%
\pgfpathlineto{\pgfqpoint{7.149352in}{1.149312in}}%
\pgfpathlineto{\pgfqpoint{7.259150in}{1.070533in}}%
\pgfpathlineto{\pgfqpoint{7.372998in}{0.991755in}}%
\pgfpathlineto{\pgfqpoint{7.494263in}{0.910762in}}%
\pgfpathlineto{\pgfqpoint{7.634433in}{0.820387in}}%
\pgfpathlineto{\pgfqpoint{7.739560in}{0.754702in}}%
\pgfpathlineto{\pgfqpoint{7.739560in}{0.754702in}}%
\pgfusepath{stroke}%
\end{pgfscope}%
\begin{pgfscope}%
\pgfpathrectangle{\pgfqpoint{0.766095in}{0.571603in}}{\pgfqpoint{6.973465in}{5.225635in}}%
\pgfusepath{clip}%
\pgfsetbuttcap%
\pgfsetroundjoin%
\pgfsetlinewidth{1.505625pt}%
\definecolor{currentstroke}{rgb}{0.263663,0.237631,0.518762}%
\pgfsetstrokecolor{currentstroke}%
\pgfsetdash{}{0pt}%
\pgfpathmoveto{\pgfqpoint{1.667213in}{0.571603in}}%
\pgfpathlineto{\pgfqpoint{1.642158in}{0.586850in}}%
\pgfpathlineto{\pgfqpoint{1.624098in}{0.597863in}}%
\pgfpathlineto{\pgfqpoint{1.607116in}{0.608435in}}%
\pgfpathlineto{\pgfqpoint{1.581975in}{0.624122in}}%
\pgfpathlineto{\pgfqpoint{1.572073in}{0.630431in}}%
\pgfpathlineto{\pgfqpoint{1.540837in}{0.650382in}}%
\pgfpathlineto{\pgfqpoint{1.537031in}{0.652864in}}%
\pgfpathlineto{\pgfqpoint{1.501988in}{0.675792in}}%
\pgfpathlineto{\pgfqpoint{1.500696in}{0.676641in}}%
\pgfpathlineto{\pgfqpoint{1.466946in}{0.699278in}}%
\pgfpathlineto{\pgfqpoint{1.461562in}{0.702901in}}%
\pgfpathlineto{\pgfqpoint{1.431903in}{0.723279in}}%
\pgfpathlineto{\pgfqpoint{1.423372in}{0.729160in}}%
\pgfpathlineto{\pgfqpoint{1.396861in}{0.747826in}}%
\pgfpathlineto{\pgfqpoint{1.386114in}{0.755420in}}%
\pgfpathlineto{\pgfqpoint{1.361818in}{0.772954in}}%
\pgfpathlineto{\pgfqpoint{1.349776in}{0.781679in}}%
\pgfpathlineto{\pgfqpoint{1.336705in}{0.791352in}}%
\pgfusepath{stroke}%
\end{pgfscope}%
\begin{pgfscope}%
\pgfpathrectangle{\pgfqpoint{0.766095in}{0.571603in}}{\pgfqpoint{6.973465in}{5.225635in}}%
\pgfusepath{clip}%
\pgfsetbuttcap%
\pgfsetroundjoin%
\pgfsetlinewidth{1.505625pt}%
\definecolor{currentstroke}{rgb}{0.263663,0.237631,0.518762}%
\pgfsetstrokecolor{currentstroke}%
\pgfsetdash{}{0pt}%
\pgfpathmoveto{\pgfqpoint{1.094759in}{0.988018in}}%
\pgfpathlineto{\pgfqpoint{1.090574in}{0.991755in}}%
\pgfpathlineto{\pgfqpoint{1.081478in}{1.000059in}}%
\pgfpathlineto{\pgfqpoint{1.061927in}{1.018014in}}%
\pgfpathlineto{\pgfqpoint{1.046435in}{1.032567in}}%
\pgfpathlineto{\pgfqpoint{1.034050in}{1.044274in}}%
\pgfpathlineto{\pgfqpoint{1.011393in}{1.066181in}}%
\pgfpathlineto{\pgfqpoint{1.006921in}{1.070533in}}%
\pgfpathlineto{\pgfqpoint{0.980585in}{1.096793in}}%
\pgfpathlineto{\pgfqpoint{0.976350in}{1.101120in}}%
\pgfpathlineto{\pgfqpoint{0.955037in}{1.123052in}}%
\pgfpathlineto{\pgfqpoint{0.941308in}{1.137515in}}%
\pgfpathlineto{\pgfqpoint{0.930189in}{1.149312in}}%
\pgfpathlineto{\pgfqpoint{0.906265in}{1.175301in}}%
\pgfpathlineto{\pgfqpoint{0.906018in}{1.175571in}}%
\pgfpathlineto{\pgfqpoint{0.882680in}{1.201831in}}%
\pgfpathlineto{\pgfqpoint{0.871223in}{1.215040in}}%
\pgfpathlineto{\pgfqpoint{0.859990in}{1.228090in}}%
\pgfpathlineto{\pgfqpoint{0.837951in}{1.254350in}}%
\pgfpathlineto{\pgfqpoint{0.836180in}{1.256522in}}%
\pgfpathlineto{\pgfqpoint{0.816699in}{1.280609in}}%
\pgfpathlineto{\pgfqpoint{0.801138in}{1.300344in}}%
\pgfpathlineto{\pgfqpoint{0.796036in}{1.306869in}}%
\pgfpathlineto{\pgfqpoint{0.776094in}{1.333128in}}%
\pgfpathlineto{\pgfqpoint{0.766095in}{1.346669in}}%
\pgfusepath{stroke}%
\end{pgfscope}%
\begin{pgfscope}%
\pgfpathrectangle{\pgfqpoint{0.766095in}{0.571603in}}{\pgfqpoint{6.973465in}{5.225635in}}%
\pgfusepath{clip}%
\pgfsetbuttcap%
\pgfsetroundjoin%
\pgfsetlinewidth{1.505625pt}%
\definecolor{currentstroke}{rgb}{0.263663,0.237631,0.518762}%
\pgfsetstrokecolor{currentstroke}%
\pgfsetdash{}{0pt}%
\pgfpathmoveto{\pgfqpoint{0.766095in}{3.243445in}}%
\pgfpathlineto{\pgfqpoint{0.801138in}{3.290411in}}%
\pgfpathlineto{\pgfqpoint{0.836180in}{3.334808in}}%
\pgfpathlineto{\pgfqpoint{0.875039in}{3.381367in}}%
\pgfpathlineto{\pgfqpoint{0.921642in}{3.433886in}}%
\pgfpathlineto{\pgfqpoint{0.976350in}{3.491740in}}%
\pgfpathlineto{\pgfqpoint{1.023799in}{3.538924in}}%
\pgfpathlineto{\pgfqpoint{1.081478in}{3.593122in}}%
\pgfpathlineto{\pgfqpoint{1.151563in}{3.654693in}}%
\pgfpathlineto{\pgfqpoint{1.221648in}{3.712359in}}%
\pgfpathlineto{\pgfqpoint{1.268536in}{3.749000in}}%
\pgfpathlineto{\pgfqpoint{1.339189in}{3.801519in}}%
\pgfpathlineto{\pgfqpoint{1.414066in}{3.854037in}}%
\pgfpathlineto{\pgfqpoint{1.493388in}{3.906556in}}%
\pgfpathlineto{\pgfqpoint{1.577472in}{3.959075in}}%
\pgfpathlineto{\pgfqpoint{1.677201in}{4.017574in}}%
\pgfpathlineto{\pgfqpoint{1.761302in}{4.064113in}}%
\pgfpathlineto{\pgfqpoint{1.861736in}{4.116632in}}%
\pgfpathlineto{\pgfqpoint{1.968492in}{4.169151in}}%
\pgfpathlineto{\pgfqpoint{2.082087in}{4.221670in}}%
\pgfpathlineto{\pgfqpoint{2.167797in}{4.259162in}}%
\pgfpathlineto{\pgfqpoint{2.272924in}{4.302993in}}%
\pgfpathlineto{\pgfqpoint{2.400336in}{4.352967in}}%
\pgfpathlineto{\pgfqpoint{2.518222in}{4.396397in}}%
\pgfpathlineto{\pgfqpoint{2.623350in}{4.433148in}}%
\pgfpathlineto{\pgfqpoint{2.763520in}{4.479190in}}%
\pgfpathlineto{\pgfqpoint{2.868647in}{4.511788in}}%
\pgfpathlineto{\pgfqpoint{3.008818in}{4.552616in}}%
\pgfpathlineto{\pgfqpoint{3.143060in}{4.589303in}}%
\pgfpathlineto{\pgfqpoint{3.254115in}{4.617779in}}%
\pgfpathlineto{\pgfqpoint{3.394285in}{4.651427in}}%
\pgfpathlineto{\pgfqpoint{3.534456in}{4.682694in}}%
\pgfpathlineto{\pgfqpoint{3.674626in}{4.711590in}}%
\pgfpathlineto{\pgfqpoint{3.814796in}{4.738119in}}%
\pgfpathlineto{\pgfqpoint{3.954966in}{4.762275in}}%
\pgfpathlineto{\pgfqpoint{4.095136in}{4.784042in}}%
\pgfpathlineto{\pgfqpoint{4.235306in}{4.803395in}}%
\pgfpathlineto{\pgfqpoint{4.375477in}{4.820112in}}%
\pgfpathlineto{\pgfqpoint{4.480604in}{4.830883in}}%
\pgfpathlineto{\pgfqpoint{4.620774in}{4.842668in}}%
\pgfpathlineto{\pgfqpoint{4.725902in}{4.849543in}}%
\pgfpathlineto{\pgfqpoint{4.831030in}{4.854412in}}%
\pgfpathlineto{\pgfqpoint{4.936157in}{4.857116in}}%
\pgfpathlineto{\pgfqpoint{5.041285in}{4.857466in}}%
\pgfpathlineto{\pgfqpoint{5.146412in}{4.855132in}}%
\pgfpathlineto{\pgfqpoint{5.216497in}{4.851890in}}%
\pgfpathlineto{\pgfqpoint{5.286583in}{4.846926in}}%
\pgfpathlineto{\pgfqpoint{5.356668in}{4.840203in}}%
\pgfpathlineto{\pgfqpoint{5.426753in}{4.831520in}}%
\pgfpathlineto{\pgfqpoint{5.496838in}{4.820390in}}%
\pgfpathlineto{\pgfqpoint{5.566923in}{4.806337in}}%
\pgfpathlineto{\pgfqpoint{5.601965in}{4.798148in}}%
\pgfpathlineto{\pgfqpoint{5.672050in}{4.778386in}}%
\pgfpathlineto{\pgfqpoint{5.707093in}{4.766694in}}%
\pgfpathlineto{\pgfqpoint{5.758599in}{4.746860in}}%
\pgfpathlineto{\pgfqpoint{5.777178in}{4.738626in}}%
\pgfpathlineto{\pgfqpoint{5.814573in}{4.720600in}}%
\pgfpathlineto{\pgfqpoint{5.860152in}{4.694341in}}%
\pgfpathlineto{\pgfqpoint{5.898061in}{4.668081in}}%
\pgfpathlineto{\pgfqpoint{5.929852in}{4.641822in}}%
\pgfpathlineto{\pgfqpoint{5.956702in}{4.615562in}}%
\pgfpathlineto{\pgfqpoint{5.987433in}{4.578541in}}%
\pgfpathlineto{\pgfqpoint{5.998496in}{4.563043in}}%
\pgfpathlineto{\pgfqpoint{6.022476in}{4.522159in}}%
\pgfpathlineto{\pgfqpoint{6.028362in}{4.510524in}}%
\pgfpathlineto{\pgfqpoint{6.039730in}{4.484265in}}%
\pgfpathlineto{\pgfqpoint{6.049211in}{4.458005in}}%
\pgfpathlineto{\pgfqpoint{6.057518in}{4.429665in}}%
\pgfpathlineto{\pgfqpoint{6.063199in}{4.405486in}}%
\pgfpathlineto{\pgfqpoint{6.068066in}{4.379227in}}%
\pgfpathlineto{\pgfqpoint{6.074400in}{4.326708in}}%
\pgfpathlineto{\pgfqpoint{6.076971in}{4.274189in}}%
\pgfpathlineto{\pgfqpoint{6.076551in}{4.221670in}}%
\pgfpathlineto{\pgfqpoint{6.073776in}{4.169151in}}%
\pgfpathlineto{\pgfqpoint{6.069176in}{4.116632in}}%
\pgfpathlineto{\pgfqpoint{6.059797in}{4.037854in}}%
\pgfpathlineto{\pgfqpoint{6.044309in}{3.932816in}}%
\pgfpathlineto{\pgfqpoint{5.999835in}{3.643962in}}%
\pgfpathlineto{\pgfqpoint{5.987082in}{3.538924in}}%
\pgfpathlineto{\pgfqpoint{5.979627in}{3.460145in}}%
\pgfpathlineto{\pgfqpoint{5.974425in}{3.381367in}}%
\pgfpathlineto{\pgfqpoint{5.971744in}{3.302589in}}%
\pgfpathlineto{\pgfqpoint{5.971822in}{3.223810in}}%
\pgfpathlineto{\pgfqpoint{5.974878in}{3.145032in}}%
\pgfpathlineto{\pgfqpoint{5.981109in}{3.066253in}}%
\pgfpathlineto{\pgfqpoint{5.990669in}{2.987475in}}%
\pgfpathlineto{\pgfqpoint{5.998934in}{2.934956in}}%
\pgfpathlineto{\pgfqpoint{6.008809in}{2.882437in}}%
\pgfpathlineto{\pgfqpoint{6.022476in}{2.821192in}}%
\pgfpathlineto{\pgfqpoint{6.033476in}{2.777399in}}%
\pgfpathlineto{\pgfqpoint{6.048328in}{2.724880in}}%
\pgfpathlineto{\pgfqpoint{6.064899in}{2.672361in}}%
\pgfpathlineto{\pgfqpoint{6.083203in}{2.619842in}}%
\pgfpathlineto{\pgfqpoint{6.103276in}{2.567323in}}%
\pgfpathlineto{\pgfqpoint{6.127603in}{2.509322in}}%
\pgfpathlineto{\pgfqpoint{6.148809in}{2.462285in}}%
\pgfpathlineto{\pgfqpoint{6.174313in}{2.409766in}}%
\pgfpathlineto{\pgfqpoint{6.201677in}{2.357248in}}%
\pgfpathlineto{\pgfqpoint{6.232731in}{2.301578in}}%
\pgfpathlineto{\pgfqpoint{6.267774in}{2.242816in}}%
\pgfpathlineto{\pgfqpoint{6.302816in}{2.187619in}}%
\pgfpathlineto{\pgfqpoint{6.337859in}{2.135445in}}%
\pgfpathlineto{\pgfqpoint{6.372901in}{2.085854in}}%
\pgfpathlineto{\pgfqpoint{6.407944in}{2.038490in}}%
\pgfpathlineto{\pgfqpoint{6.445724in}{1.989615in}}%
\pgfpathlineto{\pgfqpoint{6.488191in}{1.937096in}}%
\pgfpathlineto{\pgfqpoint{6.555541in}{1.858318in}}%
\pgfpathlineto{\pgfqpoint{6.627219in}{1.779539in}}%
\pgfpathlineto{\pgfqpoint{6.703235in}{1.700761in}}%
\pgfpathlineto{\pgfqpoint{6.783610in}{1.621982in}}%
\pgfpathlineto{\pgfqpoint{6.868330in}{1.543204in}}%
\pgfpathlineto{\pgfqpoint{6.957347in}{1.464425in}}%
\pgfpathlineto{\pgfqpoint{7.050696in}{1.385647in}}%
\pgfpathlineto{\pgfqpoint{7.148362in}{1.306869in}}%
\pgfpathlineto{\pgfqpoint{7.250298in}{1.228090in}}%
\pgfpathlineto{\pgfqpoint{7.263564in}{1.218114in}}%
\pgfpathlineto{\pgfqpoint{7.263564in}{1.218114in}}%
\pgfusepath{stroke}%
\end{pgfscope}%
\begin{pgfscope}%
\pgfpathrectangle{\pgfqpoint{0.766095in}{0.571603in}}{\pgfqpoint{6.973465in}{5.225635in}}%
\pgfusepath{clip}%
\pgfsetbuttcap%
\pgfsetroundjoin%
\pgfsetlinewidth{1.505625pt}%
\definecolor{currentstroke}{rgb}{0.263663,0.237631,0.518762}%
\pgfsetstrokecolor{currentstroke}%
\pgfsetdash{}{0pt}%
\pgfpathmoveto{\pgfqpoint{7.516629in}{1.035976in}}%
\pgfpathlineto{\pgfqpoint{7.529305in}{1.027254in}}%
\pgfpathlineto{\pgfqpoint{7.542760in}{1.018014in}}%
\pgfpathlineto{\pgfqpoint{7.564348in}{1.003335in}}%
\pgfpathlineto{\pgfqpoint{7.581415in}{0.991755in}}%
\pgfpathlineto{\pgfqpoint{7.599390in}{0.979673in}}%
\pgfpathlineto{\pgfqpoint{7.620535in}{0.965495in}}%
\pgfpathlineto{\pgfqpoint{7.634433in}{0.956260in}}%
\pgfpathlineto{\pgfqpoint{7.660119in}{0.939236in}}%
\pgfpathlineto{\pgfqpoint{7.669475in}{0.933089in}}%
\pgfpathlineto{\pgfqpoint{7.700169in}{0.912976in}}%
\pgfpathlineto{\pgfqpoint{7.704518in}{0.910151in}}%
\pgfpathlineto{\pgfqpoint{7.739560in}{0.887438in}}%
\pgfusepath{stroke}%
\end{pgfscope}%
\begin{pgfscope}%
\pgfpathrectangle{\pgfqpoint{0.766095in}{0.571603in}}{\pgfqpoint{6.973465in}{5.225635in}}%
\pgfusepath{clip}%
\pgfsetbuttcap%
\pgfsetroundjoin%
\pgfsetlinewidth{1.505625pt}%
\definecolor{currentstroke}{rgb}{0.255645,0.260703,0.528312}%
\pgfsetstrokecolor{currentstroke}%
\pgfsetdash{}{0pt}%
\pgfpathmoveto{\pgfqpoint{1.501452in}{0.571603in}}%
\pgfpathlineto{\pgfqpoint{1.469066in}{0.591738in}}%
\pgfusepath{stroke}%
\end{pgfscope}%
\begin{pgfscope}%
\pgfpathrectangle{\pgfqpoint{0.766095in}{0.571603in}}{\pgfqpoint{6.973465in}{5.225635in}}%
\pgfusepath{clip}%
\pgfsetbuttcap%
\pgfsetroundjoin%
\pgfsetlinewidth{1.505625pt}%
\definecolor{currentstroke}{rgb}{0.255645,0.260703,0.528312}%
\pgfsetstrokecolor{currentstroke}%
\pgfsetdash{}{0pt}%
\pgfpathmoveto{\pgfqpoint{1.210944in}{0.766461in}}%
\pgfpathlineto{\pgfqpoint{1.190296in}{0.781679in}}%
\pgfpathlineto{\pgfqpoint{1.186605in}{0.784456in}}%
\pgfpathlineto{\pgfqpoint{1.155540in}{0.807939in}}%
\pgfpathlineto{\pgfqpoint{1.151563in}{0.811009in}}%
\pgfpathlineto{\pgfqpoint{1.121660in}{0.834198in}}%
\pgfpathlineto{\pgfqpoint{1.116520in}{0.838269in}}%
\pgfpathlineto{\pgfqpoint{1.088642in}{0.860458in}}%
\pgfpathlineto{\pgfqpoint{1.081478in}{0.866282in}}%
\pgfpathlineto{\pgfqpoint{1.056469in}{0.886717in}}%
\pgfpathlineto{\pgfqpoint{1.046435in}{0.895093in}}%
\pgfpathlineto{\pgfqpoint{1.025125in}{0.912976in}}%
\pgfpathlineto{\pgfqpoint{1.011393in}{0.924751in}}%
\pgfpathlineto{\pgfqpoint{0.994592in}{0.939236in}}%
\pgfpathlineto{\pgfqpoint{0.976350in}{0.955309in}}%
\pgfpathlineto{\pgfqpoint{0.964854in}{0.965495in}}%
\pgfpathlineto{\pgfqpoint{0.941308in}{0.986821in}}%
\pgfpathlineto{\pgfqpoint{0.935893in}{0.991755in}}%
\pgfpathlineto{\pgfqpoint{0.907710in}{1.018014in}}%
\pgfpathlineto{\pgfqpoint{0.906265in}{1.019393in}}%
\pgfpathlineto{\pgfqpoint{0.880357in}{1.044274in}}%
\pgfpathlineto{\pgfqpoint{0.871223in}{1.053245in}}%
\pgfpathlineto{\pgfqpoint{0.853736in}{1.070533in}}%
\pgfpathlineto{\pgfqpoint{0.836180in}{1.088286in}}%
\pgfpathlineto{\pgfqpoint{0.827825in}{1.096793in}}%
\pgfpathlineto{\pgfqpoint{0.802627in}{1.123052in}}%
\pgfpathlineto{\pgfqpoint{0.801138in}{1.124645in}}%
\pgfpathlineto{\pgfqpoint{0.778235in}{1.149312in}}%
\pgfpathlineto{\pgfqpoint{0.766095in}{1.162696in}}%
\pgfusepath{stroke}%
\end{pgfscope}%
\begin{pgfscope}%
\pgfpathrectangle{\pgfqpoint{0.766095in}{0.571603in}}{\pgfqpoint{6.973465in}{5.225635in}}%
\pgfusepath{clip}%
\pgfsetbuttcap%
\pgfsetroundjoin%
\pgfsetlinewidth{1.505625pt}%
\definecolor{currentstroke}{rgb}{0.255645,0.260703,0.528312}%
\pgfsetstrokecolor{currentstroke}%
\pgfsetdash{}{0pt}%
\pgfpathmoveto{\pgfqpoint{0.766095in}{3.462488in}}%
\pgfpathlineto{\pgfqpoint{0.812294in}{3.512664in}}%
\pgfpathlineto{\pgfqpoint{0.871223in}{3.572813in}}%
\pgfpathlineto{\pgfqpoint{0.917853in}{3.617702in}}%
\pgfpathlineto{\pgfqpoint{0.976350in}{3.671080in}}%
\pgfpathlineto{\pgfqpoint{1.046435in}{3.731059in}}%
\pgfpathlineto{\pgfqpoint{1.116520in}{3.787418in}}%
\pgfpathlineto{\pgfqpoint{1.186605in}{3.840558in}}%
\pgfpathlineto{\pgfqpoint{1.256691in}{3.890832in}}%
\pgfpathlineto{\pgfqpoint{1.326776in}{3.938554in}}%
\pgfpathlineto{\pgfqpoint{1.399011in}{3.985335in}}%
\pgfpathlineto{\pgfqpoint{1.484699in}{4.037854in}}%
\pgfpathlineto{\pgfqpoint{1.575267in}{4.090373in}}%
\pgfpathlineto{\pgfqpoint{1.677201in}{4.146049in}}%
\pgfpathlineto{\pgfqpoint{1.782329in}{4.200116in}}%
\pgfpathlineto{\pgfqpoint{1.887456in}{4.251140in}}%
\pgfpathlineto{\pgfqpoint{1.994941in}{4.300449in}}%
\pgfpathlineto{\pgfqpoint{2.116467in}{4.352967in}}%
\pgfpathlineto{\pgfqpoint{2.237882in}{4.402406in}}%
\pgfpathlineto{\pgfqpoint{2.343009in}{4.442900in}}%
\pgfpathlineto{\pgfqpoint{2.455698in}{4.484265in}}%
\pgfpathlineto{\pgfqpoint{2.588307in}{4.530176in}}%
\pgfpathlineto{\pgfqpoint{2.693435in}{4.564754in}}%
\pgfpathlineto{\pgfqpoint{2.857629in}{4.615562in}}%
\pgfpathlineto{\pgfqpoint{2.973775in}{4.649299in}}%
\pgfpathlineto{\pgfqpoint{3.113945in}{4.687838in}}%
\pgfpathlineto{\pgfqpoint{3.240240in}{4.720600in}}%
\pgfpathlineto{\pgfqpoint{3.394285in}{4.757998in}}%
\pgfpathlineto{\pgfqpoint{3.534456in}{4.789840in}}%
\pgfpathlineto{\pgfqpoint{3.674626in}{4.819619in}}%
\pgfpathlineto{\pgfqpoint{3.814796in}{4.847351in}}%
\pgfpathlineto{\pgfqpoint{3.954966in}{4.873049in}}%
\pgfpathlineto{\pgfqpoint{4.095136in}{4.896716in}}%
\pgfpathlineto{\pgfqpoint{4.235306in}{4.918351in}}%
\pgfpathlineto{\pgfqpoint{4.375477in}{4.937944in}}%
\pgfpathlineto{\pgfqpoint{4.528665in}{4.956935in}}%
\pgfpathlineto{\pgfqpoint{4.655817in}{4.970510in}}%
\pgfpathlineto{\pgfqpoint{4.795987in}{4.983387in}}%
\pgfpathlineto{\pgfqpoint{4.936157in}{4.993444in}}%
\pgfpathlineto{\pgfqpoint{5.076327in}{5.000754in}}%
\pgfpathlineto{\pgfqpoint{5.181455in}{5.004169in}}%
\pgfpathlineto{\pgfqpoint{5.286583in}{5.005563in}}%
\pgfpathlineto{\pgfqpoint{5.391710in}{5.004661in}}%
\pgfpathlineto{\pgfqpoint{5.496838in}{5.001136in}}%
\pgfpathlineto{\pgfqpoint{5.601965in}{4.994587in}}%
\pgfpathlineto{\pgfqpoint{5.672050in}{4.988303in}}%
\pgfpathlineto{\pgfqpoint{5.742136in}{4.980143in}}%
\pgfpathlineto{\pgfqpoint{5.812221in}{4.969680in}}%
\pgfpathlineto{\pgfqpoint{5.882306in}{4.956899in}}%
\pgfpathlineto{\pgfqpoint{5.952391in}{4.940633in}}%
\pgfpathlineto{\pgfqpoint{5.989697in}{4.930676in}}%
\pgfpathlineto{\pgfqpoint{6.057518in}{4.908767in}}%
\pgfpathlineto{\pgfqpoint{6.092561in}{4.895321in}}%
\pgfpathlineto{\pgfqpoint{6.132286in}{4.878157in}}%
\pgfpathlineto{\pgfqpoint{6.182785in}{4.851897in}}%
\pgfpathlineto{\pgfqpoint{6.224335in}{4.825638in}}%
\pgfpathlineto{\pgfqpoint{6.258782in}{4.799378in}}%
\pgfpathlineto{\pgfqpoint{6.287515in}{4.773119in}}%
\pgfpathlineto{\pgfqpoint{6.311599in}{4.746860in}}%
\pgfpathlineto{\pgfqpoint{6.337859in}{4.711304in}}%
\pgfpathlineto{\pgfqpoint{6.348478in}{4.694341in}}%
\pgfpathlineto{\pgfqpoint{6.362363in}{4.668081in}}%
\pgfpathlineto{\pgfqpoint{6.373807in}{4.641822in}}%
\pgfpathlineto{\pgfqpoint{6.382944in}{4.615562in}}%
\pgfpathlineto{\pgfqpoint{6.390203in}{4.589303in}}%
\pgfpathlineto{\pgfqpoint{6.395787in}{4.563043in}}%
\pgfpathlineto{\pgfqpoint{6.399874in}{4.536784in}}%
\pgfpathlineto{\pgfqpoint{6.402622in}{4.510524in}}%
\pgfpathlineto{\pgfqpoint{6.404656in}{4.458005in}}%
\pgfpathlineto{\pgfqpoint{6.402857in}{4.405486in}}%
\pgfpathlineto{\pgfqpoint{6.398013in}{4.352967in}}%
\pgfpathlineto{\pgfqpoint{6.390769in}{4.300449in}}%
\pgfpathlineto{\pgfqpoint{6.381660in}{4.247930in}}%
\pgfpathlineto{\pgfqpoint{6.365357in}{4.169151in}}%
\pgfpathlineto{\pgfqpoint{6.339084in}{4.058263in}}%
\pgfpathlineto{\pgfqpoint{6.339084in}{4.058263in}}%
\pgfusepath{stroke}%
\end{pgfscope}%
\begin{pgfscope}%
\pgfpathrectangle{\pgfqpoint{0.766095in}{0.571603in}}{\pgfqpoint{6.973465in}{5.225635in}}%
\pgfusepath{clip}%
\pgfsetbuttcap%
\pgfsetroundjoin%
\pgfsetlinewidth{1.505625pt}%
\definecolor{currentstroke}{rgb}{0.255645,0.260703,0.528312}%
\pgfsetstrokecolor{currentstroke}%
\pgfsetdash{}{0pt}%
\pgfpathmoveto{\pgfqpoint{6.264342in}{3.754455in}}%
\pgfpathlineto{\pgfqpoint{6.246101in}{3.670221in}}%
\pgfpathlineto{\pgfqpoint{6.230990in}{3.591443in}}%
\pgfpathlineto{\pgfqpoint{6.217950in}{3.512664in}}%
\pgfpathlineto{\pgfqpoint{6.207394in}{3.433886in}}%
\pgfpathlineto{\pgfqpoint{6.199548in}{3.355107in}}%
\pgfpathlineto{\pgfqpoint{6.194581in}{3.276329in}}%
\pgfpathlineto{\pgfqpoint{6.192726in}{3.197551in}}%
\pgfpathlineto{\pgfqpoint{6.193314in}{3.145032in}}%
\pgfpathlineto{\pgfqpoint{6.195417in}{3.092513in}}%
\pgfpathlineto{\pgfqpoint{6.199069in}{3.039994in}}%
\pgfpathlineto{\pgfqpoint{6.204290in}{2.987475in}}%
\pgfpathlineto{\pgfqpoint{6.211148in}{2.934956in}}%
\pgfpathlineto{\pgfqpoint{6.219687in}{2.882437in}}%
\pgfpathlineto{\pgfqpoint{6.232731in}{2.817327in}}%
\pgfpathlineto{\pgfqpoint{6.241895in}{2.777399in}}%
\pgfpathlineto{\pgfqpoint{6.255608in}{2.724880in}}%
\pgfpathlineto{\pgfqpoint{6.271131in}{2.672361in}}%
\pgfpathlineto{\pgfqpoint{6.288406in}{2.619842in}}%
\pgfpathlineto{\pgfqpoint{6.307554in}{2.567323in}}%
\pgfpathlineto{\pgfqpoint{6.328510in}{2.514804in}}%
\pgfpathlineto{\pgfqpoint{6.351335in}{2.462285in}}%
\pgfpathlineto{\pgfqpoint{6.376070in}{2.409766in}}%
\pgfpathlineto{\pgfqpoint{6.407944in}{2.347392in}}%
\pgfpathlineto{\pgfqpoint{6.431155in}{2.304729in}}%
\pgfpathlineto{\pgfqpoint{6.461575in}{2.252210in}}%
\pgfpathlineto{\pgfqpoint{6.493927in}{2.199691in}}%
\pgfpathlineto{\pgfqpoint{6.528216in}{2.147172in}}%
\pgfpathlineto{\pgfqpoint{6.564450in}{2.094653in}}%
\pgfpathlineto{\pgfqpoint{6.622486in}{2.015874in}}%
\pgfpathlineto{\pgfqpoint{6.663594in}{1.963355in}}%
\pgfpathlineto{\pgfqpoint{6.728979in}{1.884577in}}%
\pgfpathlineto{\pgfqpoint{6.793412in}{1.811747in}}%
\pgfpathlineto{\pgfqpoint{6.828454in}{1.773801in}}%
\pgfpathlineto{\pgfqpoint{6.873014in}{1.727020in}}%
\pgfpathlineto{\pgfqpoint{6.951687in}{1.648242in}}%
\pgfpathlineto{\pgfqpoint{7.034829in}{1.569463in}}%
\pgfpathlineto{\pgfqpoint{7.122357in}{1.490685in}}%
\pgfpathlineto{\pgfqpoint{7.214363in}{1.411906in}}%
\pgfpathlineto{\pgfqpoint{7.310715in}{1.333128in}}%
\pgfpathlineto{\pgfqpoint{7.411476in}{1.254350in}}%
\pgfpathlineto{\pgfqpoint{7.516615in}{1.175571in}}%
\pgfpathlineto{\pgfqpoint{7.634433in}{1.090950in}}%
\pgfpathlineto{\pgfqpoint{7.739560in}{1.018270in}}%
\pgfpathlineto{\pgfqpoint{7.739560in}{1.018270in}}%
\pgfusepath{stroke}%
\end{pgfscope}%
\begin{pgfscope}%
\pgfpathrectangle{\pgfqpoint{0.766095in}{0.571603in}}{\pgfqpoint{6.973465in}{5.225635in}}%
\pgfusepath{clip}%
\pgfsetbuttcap%
\pgfsetroundjoin%
\pgfsetlinewidth{1.505625pt}%
\definecolor{currentstroke}{rgb}{0.246811,0.283237,0.535941}%
\pgfsetstrokecolor{currentstroke}%
\pgfsetdash{}{0pt}%
\pgfpathmoveto{\pgfqpoint{1.345956in}{0.571603in}}%
\pgfpathlineto{\pgfqpoint{1.326776in}{0.583736in}}%
\pgfpathlineto{\pgfqpoint{1.304501in}{0.597863in}}%
\pgfpathlineto{\pgfqpoint{1.291733in}{0.606125in}}%
\pgfpathlineto{\pgfqpoint{1.264001in}{0.624122in}}%
\pgfpathlineto{\pgfqpoint{1.256691in}{0.628963in}}%
\pgfpathlineto{\pgfqpoint{1.224447in}{0.650382in}}%
\pgfpathlineto{\pgfqpoint{1.221648in}{0.652279in}}%
\pgfpathlineto{\pgfqpoint{1.186605in}{0.676118in}}%
\pgfpathlineto{\pgfqpoint{1.185840in}{0.676641in}}%
\pgfpathlineto{\pgfqpoint{1.151563in}{0.700536in}}%
\pgfpathlineto{\pgfqpoint{1.148183in}{0.702901in}}%
\pgfpathlineto{\pgfqpoint{1.116520in}{0.725509in}}%
\pgfpathlineto{\pgfqpoint{1.111427in}{0.729160in}}%
\pgfpathlineto{\pgfqpoint{1.081478in}{0.751072in}}%
\pgfpathlineto{\pgfqpoint{1.075559in}{0.755420in}}%
\pgfpathlineto{\pgfqpoint{1.046435in}{0.777259in}}%
\pgfpathlineto{\pgfqpoint{1.040566in}{0.781679in}}%
\pgfpathlineto{\pgfqpoint{1.027841in}{0.791463in}}%
\pgfusepath{stroke}%
\end{pgfscope}%
\begin{pgfscope}%
\pgfpathrectangle{\pgfqpoint{0.766095in}{0.571603in}}{\pgfqpoint{6.973465in}{5.225635in}}%
\pgfusepath{clip}%
\pgfsetbuttcap%
\pgfsetroundjoin%
\pgfsetlinewidth{1.505625pt}%
\definecolor{currentstroke}{rgb}{0.246811,0.283237,0.535941}%
\pgfsetstrokecolor{currentstroke}%
\pgfsetdash{}{0pt}%
\pgfpathmoveto{\pgfqpoint{0.789648in}{0.992704in}}%
\pgfpathlineto{\pgfqpoint{0.766095in}{1.015002in}}%
\pgfusepath{stroke}%
\end{pgfscope}%
\begin{pgfscope}%
\pgfpathrectangle{\pgfqpoint{0.766095in}{0.571603in}}{\pgfqpoint{6.973465in}{5.225635in}}%
\pgfusepath{clip}%
\pgfsetbuttcap%
\pgfsetroundjoin%
\pgfsetlinewidth{1.505625pt}%
\definecolor{currentstroke}{rgb}{0.246811,0.283237,0.535941}%
\pgfsetstrokecolor{currentstroke}%
\pgfsetdash{}{0pt}%
\pgfpathmoveto{\pgfqpoint{0.766095in}{3.639454in}}%
\pgfpathlineto{\pgfqpoint{0.770765in}{3.643962in}}%
\pgfpathlineto{\pgfqpoint{0.798518in}{3.670221in}}%
\pgfpathlineto{\pgfqpoint{0.801138in}{3.672651in}}%
\pgfpathlineto{\pgfqpoint{0.827122in}{3.696481in}}%
\pgfpathlineto{\pgfqpoint{0.836180in}{3.704632in}}%
\pgfpathlineto{\pgfqpoint{0.856524in}{3.722740in}}%
\pgfpathlineto{\pgfqpoint{0.871223in}{3.735581in}}%
\pgfpathlineto{\pgfqpoint{0.886750in}{3.749000in}}%
\pgfpathlineto{\pgfqpoint{0.906265in}{3.765556in}}%
\pgfpathlineto{\pgfqpoint{0.917825in}{3.775259in}}%
\pgfpathlineto{\pgfqpoint{0.941308in}{3.794612in}}%
\pgfpathlineto{\pgfqpoint{0.949776in}{3.801519in}}%
\pgfpathlineto{\pgfqpoint{0.976350in}{3.822801in}}%
\pgfpathlineto{\pgfqpoint{0.982628in}{3.827778in}}%
\pgfpathlineto{\pgfqpoint{1.011393in}{3.850173in}}%
\pgfpathlineto{\pgfqpoint{1.016407in}{3.854037in}}%
\pgfpathlineto{\pgfqpoint{1.046435in}{3.876772in}}%
\pgfpathlineto{\pgfqpoint{1.051137in}{3.880297in}}%
\pgfpathlineto{\pgfqpoint{1.081478in}{3.902641in}}%
\pgfpathlineto{\pgfqpoint{1.086845in}{3.906556in}}%
\pgfpathlineto{\pgfqpoint{1.116520in}{3.927822in}}%
\pgfpathlineto{\pgfqpoint{1.123554in}{3.932816in}}%
\pgfpathlineto{\pgfqpoint{1.151563in}{3.952352in}}%
\pgfpathlineto{\pgfqpoint{1.161290in}{3.959075in}}%
\pgfpathlineto{\pgfqpoint{1.186605in}{3.976268in}}%
\pgfpathlineto{\pgfqpoint{1.200074in}{3.985335in}}%
\pgfpathlineto{\pgfqpoint{1.221648in}{3.999605in}}%
\pgfpathlineto{\pgfqpoint{1.239932in}{4.011594in}}%
\pgfpathlineto{\pgfqpoint{1.256691in}{4.022394in}}%
\pgfpathlineto{\pgfqpoint{1.280884in}{4.037854in}}%
\pgfpathlineto{\pgfqpoint{1.291733in}{4.044667in}}%
\pgfpathlineto{\pgfqpoint{1.322953in}{4.064113in}}%
\pgfpathlineto{\pgfqpoint{1.326776in}{4.066453in}}%
\pgfpathlineto{\pgfqpoint{1.361818in}{4.087716in}}%
\pgfpathlineto{\pgfqpoint{1.366236in}{4.090373in}}%
\pgfpathlineto{\pgfqpoint{1.396861in}{4.108476in}}%
\pgfpathlineto{\pgfqpoint{1.410765in}{4.116632in}}%
\pgfpathlineto{\pgfqpoint{1.431903in}{4.128820in}}%
\pgfpathlineto{\pgfqpoint{1.456490in}{4.142892in}}%
\pgfpathlineto{\pgfqpoint{1.466946in}{4.148774in}}%
\pgfpathlineto{\pgfqpoint{1.501988in}{4.168340in}}%
\pgfpathlineto{\pgfqpoint{1.503455in}{4.169151in}}%
\pgfpathlineto{\pgfqpoint{1.537031in}{4.187404in}}%
\pgfpathlineto{\pgfqpoint{1.551861in}{4.195411in}}%
\pgfpathlineto{\pgfqpoint{1.572073in}{4.206137in}}%
\pgfpathlineto{\pgfqpoint{1.601536in}{4.221670in}}%
\pgfpathlineto{\pgfqpoint{1.607116in}{4.224562in}}%
\pgfpathlineto{\pgfqpoint{1.642158in}{4.242565in}}%
\pgfpathlineto{\pgfqpoint{1.652679in}{4.247930in}}%
\pgfpathlineto{\pgfqpoint{1.677201in}{4.260220in}}%
\pgfpathlineto{\pgfqpoint{1.705241in}{4.274189in}}%
\pgfpathlineto{\pgfqpoint{1.712244in}{4.277618in}}%
\pgfpathlineto{\pgfqpoint{1.747286in}{4.294634in}}%
\pgfpathlineto{\pgfqpoint{1.759348in}{4.300449in}}%
\pgfpathlineto{\pgfqpoint{1.782329in}{4.311337in}}%
\pgfpathlineto{\pgfqpoint{1.814944in}{4.326708in}}%
\pgfpathlineto{\pgfqpoint{1.817371in}{4.327832in}}%
\pgfpathlineto{\pgfqpoint{1.852414in}{4.343910in}}%
\pgfpathlineto{\pgfqpoint{1.872265in}{4.352967in}}%
\pgfpathlineto{\pgfqpoint{1.887456in}{4.359780in}}%
\pgfpathlineto{\pgfqpoint{1.922499in}{4.375389in}}%
\pgfpathlineto{\pgfqpoint{1.931187in}{4.379227in}}%
\pgfpathlineto{\pgfqpoint{1.957541in}{4.390670in}}%
\pgfpathlineto{\pgfqpoint{1.991814in}{4.405486in}}%
\pgfpathlineto{\pgfqpoint{1.992584in}{4.405813in}}%
\pgfpathlineto{\pgfqpoint{2.027626in}{4.420552in}}%
\pgfpathlineto{\pgfqpoint{2.054355in}{4.431746in}}%
\pgfpathlineto{\pgfqpoint{2.062669in}{4.435168in}}%
\pgfpathlineto{\pgfqpoint{2.097711in}{4.449469in}}%
\pgfpathlineto{\pgfqpoint{2.118746in}{4.458005in}}%
\pgfpathlineto{\pgfqpoint{2.132754in}{4.463592in}}%
\pgfpathlineto{\pgfqpoint{2.167797in}{4.477467in}}%
\pgfpathlineto{\pgfqpoint{2.185081in}{4.484265in}}%
\pgfpathlineto{\pgfqpoint{2.202839in}{4.491129in}}%
\pgfpathlineto{\pgfqpoint{2.237882in}{4.504586in}}%
\pgfpathlineto{\pgfqpoint{2.253457in}{4.510524in}}%
\pgfpathlineto{\pgfqpoint{2.272924in}{4.517818in}}%
\pgfpathlineto{\pgfqpoint{2.307967in}{4.530866in}}%
\pgfpathlineto{\pgfqpoint{2.323973in}{4.536784in}}%
\pgfpathlineto{\pgfqpoint{2.343009in}{4.543699in}}%
\pgfpathlineto{\pgfqpoint{2.378052in}{4.556347in}}%
\pgfpathlineto{\pgfqpoint{2.396727in}{4.563043in}}%
\pgfpathlineto{\pgfqpoint{2.413094in}{4.568809in}}%
\pgfpathlineto{\pgfqpoint{2.448137in}{4.581066in}}%
\pgfpathlineto{\pgfqpoint{2.471819in}{4.589303in}}%
\pgfpathlineto{\pgfqpoint{2.483179in}{4.593185in}}%
\pgfpathlineto{\pgfqpoint{2.518222in}{4.605057in}}%
\pgfpathlineto{\pgfqpoint{2.549349in}{4.615562in}}%
\pgfpathlineto{\pgfqpoint{2.553264in}{4.616860in}}%
\pgfpathlineto{\pgfqpoint{2.588307in}{4.628357in}}%
\pgfpathlineto{\pgfqpoint{2.623350in}{4.639818in}}%
\pgfpathlineto{\pgfqpoint{2.629539in}{4.641822in}}%
\pgfpathlineto{\pgfqpoint{2.658392in}{4.650997in}}%
\pgfpathlineto{\pgfqpoint{2.693435in}{4.662089in}}%
\pgfpathlineto{\pgfqpoint{2.712503in}{4.668081in}}%
\pgfpathlineto{\pgfqpoint{2.728477in}{4.673012in}}%
\pgfpathlineto{\pgfqpoint{2.763520in}{4.683741in}}%
\pgfpathlineto{\pgfqpoint{2.798262in}{4.694341in}}%
\pgfpathlineto{\pgfqpoint{2.798562in}{4.694431in}}%
\pgfpathlineto{\pgfqpoint{2.833605in}{4.704804in}}%
\pgfpathlineto{\pgfqpoint{2.868647in}{4.715140in}}%
\pgfpathlineto{\pgfqpoint{2.887304in}{4.720600in}}%
\pgfpathlineto{\pgfqpoint{2.903690in}{4.725309in}}%
\pgfpathlineto{\pgfqpoint{2.938732in}{4.735295in}}%
\pgfpathlineto{\pgfqpoint{2.973775in}{4.745242in}}%
\pgfpathlineto{\pgfqpoint{2.979540in}{4.746860in}}%
\pgfpathlineto{\pgfqpoint{3.008818in}{4.754927in}}%
\pgfpathlineto{\pgfqpoint{3.043860in}{4.764529in}}%
\pgfpathlineto{\pgfqpoint{3.075369in}{4.773119in}}%
\pgfpathlineto{\pgfqpoint{3.078903in}{4.774065in}}%
\pgfpathlineto{\pgfqpoint{3.113945in}{4.783328in}}%
\pgfpathlineto{\pgfqpoint{3.148988in}{4.792550in}}%
\pgfpathlineto{\pgfqpoint{3.175114in}{4.799378in}}%
\pgfpathlineto{\pgfqpoint{3.184030in}{4.801666in}}%
\pgfpathlineto{\pgfqpoint{3.219073in}{4.810553in}}%
\pgfpathlineto{\pgfqpoint{3.254115in}{4.819398in}}%
\pgfpathlineto{\pgfqpoint{3.279026in}{4.825638in}}%
\pgfpathlineto{\pgfqpoint{3.289158in}{4.828129in}}%
\pgfpathlineto{\pgfqpoint{3.324200in}{4.836643in}}%
\pgfpathlineto{\pgfqpoint{3.359243in}{4.845113in}}%
\pgfpathlineto{\pgfqpoint{3.382745in}{4.850753in}}%
\pgfusepath{stroke}%
\end{pgfscope}%
\begin{pgfscope}%
\pgfpathrectangle{\pgfqpoint{0.766095in}{0.571603in}}{\pgfqpoint{6.973465in}{5.225635in}}%
\pgfusepath{clip}%
\pgfsetbuttcap%
\pgfsetroundjoin%
\pgfsetlinewidth{1.505625pt}%
\definecolor{currentstroke}{rgb}{0.246811,0.283237,0.535941}%
\pgfsetstrokecolor{currentstroke}%
\pgfsetdash{}{0pt}%
\pgfpathmoveto{\pgfqpoint{3.686618in}{4.918322in}}%
\pgfpathlineto{\pgfqpoint{3.814796in}{4.944291in}}%
\pgfpathlineto{\pgfqpoint{3.954966in}{4.970991in}}%
\pgfpathlineto{\pgfqpoint{4.095136in}{4.995939in}}%
\pgfpathlineto{\pgfqpoint{4.235306in}{5.019147in}}%
\pgfpathlineto{\pgfqpoint{4.410519in}{5.045672in}}%
\pgfpathlineto{\pgfqpoint{4.585732in}{5.069400in}}%
\pgfpathlineto{\pgfqpoint{4.760944in}{5.090246in}}%
\pgfpathlineto{\pgfqpoint{4.936157in}{5.107879in}}%
\pgfpathlineto{\pgfqpoint{5.076327in}{5.119638in}}%
\pgfpathlineto{\pgfqpoint{5.216497in}{5.129013in}}%
\pgfpathlineto{\pgfqpoint{5.356668in}{5.135913in}}%
\pgfpathlineto{\pgfqpoint{5.461795in}{5.139233in}}%
\pgfpathlineto{\pgfqpoint{5.572415in}{5.140752in}}%
\pgfpathlineto{\pgfqpoint{5.672050in}{5.140203in}}%
\pgfpathlineto{\pgfqpoint{5.777178in}{5.137331in}}%
\pgfpathlineto{\pgfqpoint{5.882306in}{5.131780in}}%
\pgfpathlineto{\pgfqpoint{5.987433in}{5.123130in}}%
\pgfpathlineto{\pgfqpoint{6.064581in}{5.114492in}}%
\pgfpathlineto{\pgfqpoint{6.127603in}{5.105411in}}%
\pgfpathlineto{\pgfqpoint{6.197689in}{5.093243in}}%
\pgfpathlineto{\pgfqpoint{6.232731in}{5.086159in}}%
\pgfpathlineto{\pgfqpoint{6.302816in}{5.069296in}}%
\pgfpathlineto{\pgfqpoint{6.337859in}{5.059553in}}%
\pgfpathlineto{\pgfqpoint{6.409521in}{5.035714in}}%
\pgfpathlineto{\pgfqpoint{6.472030in}{5.009454in}}%
\pgfpathlineto{\pgfqpoint{6.513071in}{4.988376in}}%
\pgfpathlineto{\pgfqpoint{6.522405in}{4.983195in}}%
\pgfpathlineto{\pgfqpoint{6.563620in}{4.956935in}}%
\pgfpathlineto{\pgfqpoint{6.597661in}{4.930676in}}%
\pgfpathlineto{\pgfqpoint{6.625900in}{4.904416in}}%
\pgfpathlineto{\pgfqpoint{6.653242in}{4.873034in}}%
\pgfpathlineto{\pgfqpoint{6.668598in}{4.851897in}}%
\pgfpathlineto{\pgfqpoint{6.688284in}{4.818218in}}%
\pgfpathlineto{\pgfqpoint{6.697383in}{4.799378in}}%
\pgfpathlineto{\pgfqpoint{6.707663in}{4.773119in}}%
\pgfpathlineto{\pgfqpoint{6.715713in}{4.746860in}}%
\pgfpathlineto{\pgfqpoint{6.723327in}{4.711254in}}%
\pgfpathlineto{\pgfqpoint{6.726020in}{4.694341in}}%
\pgfpathlineto{\pgfqpoint{6.728685in}{4.668081in}}%
\pgfpathlineto{\pgfqpoint{6.729958in}{4.641822in}}%
\pgfpathlineto{\pgfqpoint{6.729987in}{4.615562in}}%
\pgfpathlineto{\pgfqpoint{6.726823in}{4.563043in}}%
\pgfpathlineto{\pgfqpoint{6.720043in}{4.510524in}}%
\pgfpathlineto{\pgfqpoint{6.710342in}{4.458005in}}%
\pgfpathlineto{\pgfqpoint{6.698387in}{4.405486in}}%
\pgfpathlineto{\pgfqpoint{6.684640in}{4.352967in}}%
\pgfpathlineto{\pgfqpoint{6.661450in}{4.274189in}}%
\pgfpathlineto{\pgfqpoint{6.627580in}{4.169151in}}%
\pgfpathlineto{\pgfqpoint{6.523344in}{3.854037in}}%
\pgfpathlineto{\pgfqpoint{6.492162in}{3.749000in}}%
\pgfpathlineto{\pgfqpoint{6.470988in}{3.670221in}}%
\pgfpathlineto{\pgfqpoint{6.452026in}{3.591443in}}%
\pgfpathlineto{\pgfqpoint{6.435583in}{3.512664in}}%
\pgfpathlineto{\pgfqpoint{6.421874in}{3.433886in}}%
\pgfpathlineto{\pgfqpoint{6.411173in}{3.355107in}}%
\pgfpathlineto{\pgfqpoint{6.403601in}{3.276329in}}%
\pgfpathlineto{\pgfqpoint{6.400388in}{3.223810in}}%
\pgfpathlineto{\pgfqpoint{6.398709in}{3.171291in}}%
\pgfpathlineto{\pgfqpoint{6.398604in}{3.118772in}}%
\pgfpathlineto{\pgfqpoint{6.400115in}{3.066253in}}%
\pgfpathlineto{\pgfqpoint{6.403280in}{3.013734in}}%
\pgfpathlineto{\pgfqpoint{6.408136in}{2.961215in}}%
\pgfpathlineto{\pgfqpoint{6.414663in}{2.908696in}}%
\pgfpathlineto{\pgfqpoint{6.422945in}{2.856177in}}%
\pgfpathlineto{\pgfqpoint{6.433021in}{2.803659in}}%
\pgfpathlineto{\pgfqpoint{6.444911in}{2.751140in}}%
\pgfpathlineto{\pgfqpoint{6.458567in}{2.698621in}}%
\pgfpathlineto{\pgfqpoint{6.478029in}{2.634027in}}%
\pgfpathlineto{\pgfqpoint{6.491490in}{2.593583in}}%
\pgfpathlineto{\pgfqpoint{6.513071in}{2.535306in}}%
\pgfpathlineto{\pgfqpoint{6.531934in}{2.488545in}}%
\pgfpathlineto{\pgfqpoint{6.555033in}{2.436026in}}%
\pgfpathlineto{\pgfqpoint{6.583156in}{2.377382in}}%
\pgfpathlineto{\pgfqpoint{6.606989in}{2.330988in}}%
\pgfpathlineto{\pgfqpoint{6.635889in}{2.278469in}}%
\pgfpathlineto{\pgfqpoint{6.666757in}{2.225950in}}%
\pgfpathlineto{\pgfqpoint{6.699598in}{2.173431in}}%
\pgfpathlineto{\pgfqpoint{6.734415in}{2.120912in}}%
\pgfpathlineto{\pgfqpoint{6.771216in}{2.068393in}}%
\pgfpathlineto{\pgfqpoint{6.828454in}{1.991879in}}%
\pgfpathlineto{\pgfqpoint{6.863497in}{1.947627in}}%
\pgfpathlineto{\pgfqpoint{6.898539in}{1.905012in}}%
\pgfpathlineto{\pgfqpoint{6.938416in}{1.858318in}}%
\pgfpathlineto{\pgfqpoint{7.003667in}{1.785757in}}%
\pgfpathlineto{\pgfqpoint{7.038709in}{1.748414in}}%
\pgfpathlineto{\pgfqpoint{7.108795in}{1.676759in}}%
\pgfpathlineto{\pgfqpoint{7.164855in}{1.621982in}}%
\pgfpathlineto{\pgfqpoint{7.249418in}{1.543204in}}%
\pgfpathlineto{\pgfqpoint{7.338426in}{1.464425in}}%
\pgfpathlineto{\pgfqpoint{7.431986in}{1.385647in}}%
\pgfpathlineto{\pgfqpoint{7.530050in}{1.306869in}}%
\pgfpathlineto{\pgfqpoint{7.634433in}{1.226694in}}%
\pgfpathlineto{\pgfqpoint{7.739560in}{1.149304in}}%
\pgfpathlineto{\pgfqpoint{7.739560in}{1.149304in}}%
\pgfusepath{stroke}%
\end{pgfscope}%
\begin{pgfscope}%
\pgfpathrectangle{\pgfqpoint{0.766095in}{0.571603in}}{\pgfqpoint{6.973465in}{5.225635in}}%
\pgfusepath{clip}%
\pgfsetbuttcap%
\pgfsetroundjoin%
\pgfsetlinewidth{1.505625pt}%
\definecolor{currentstroke}{rgb}{0.237441,0.305202,0.541921}%
\pgfsetstrokecolor{currentstroke}%
\pgfsetdash{}{0pt}%
\pgfpathmoveto{\pgfqpoint{1.199166in}{0.571603in}}%
\pgfpathlineto{\pgfqpoint{1.186605in}{0.579687in}}%
\pgfpathlineto{\pgfqpoint{1.175844in}{0.586633in}}%
\pgfusepath{stroke}%
\end{pgfscope}%
\begin{pgfscope}%
\pgfpathrectangle{\pgfqpoint{0.766095in}{0.571603in}}{\pgfqpoint{6.973465in}{5.225635in}}%
\pgfusepath{clip}%
\pgfsetbuttcap%
\pgfsetroundjoin%
\pgfsetlinewidth{1.505625pt}%
\definecolor{currentstroke}{rgb}{0.237441,0.305202,0.541921}%
\pgfsetstrokecolor{currentstroke}%
\pgfsetdash{}{0pt}%
\pgfpathmoveto{\pgfqpoint{0.920496in}{0.765420in}}%
\pgfpathlineto{\pgfqpoint{0.906265in}{0.776278in}}%
\pgfpathlineto{\pgfqpoint{0.899219in}{0.781679in}}%
\pgfpathlineto{\pgfqpoint{0.871223in}{0.803579in}}%
\pgfpathlineto{\pgfqpoint{0.865676in}{0.807939in}}%
\pgfpathlineto{\pgfqpoint{0.836180in}{0.831602in}}%
\pgfpathlineto{\pgfqpoint{0.832961in}{0.834198in}}%
\pgfpathlineto{\pgfqpoint{0.801138in}{0.860391in}}%
\pgfpathlineto{\pgfqpoint{0.801057in}{0.860458in}}%
\pgfpathlineto{\pgfqpoint{0.770002in}{0.886717in}}%
\pgfpathlineto{\pgfqpoint{0.766095in}{0.890091in}}%
\pgfusepath{stroke}%
\end{pgfscope}%
\begin{pgfscope}%
\pgfpathrectangle{\pgfqpoint{0.766095in}{0.571603in}}{\pgfqpoint{6.973465in}{5.225635in}}%
\pgfusepath{clip}%
\pgfsetbuttcap%
\pgfsetroundjoin%
\pgfsetlinewidth{1.505625pt}%
\definecolor{currentstroke}{rgb}{0.237441,0.305202,0.541921}%
\pgfsetstrokecolor{currentstroke}%
\pgfsetdash{}{0pt}%
\pgfpathmoveto{\pgfqpoint{0.766095in}{3.789341in}}%
\pgfpathlineto{\pgfqpoint{0.811702in}{3.827778in}}%
\pgfpathlineto{\pgfqpoint{0.877075in}{3.880297in}}%
\pgfpathlineto{\pgfqpoint{0.946115in}{3.932816in}}%
\pgfpathlineto{\pgfqpoint{1.019013in}{3.985335in}}%
\pgfpathlineto{\pgfqpoint{1.095953in}{4.037854in}}%
\pgfpathlineto{\pgfqpoint{1.186605in}{4.096315in}}%
\pgfpathlineto{\pgfqpoint{1.262750in}{4.142892in}}%
\pgfpathlineto{\pgfqpoint{1.361818in}{4.200276in}}%
\pgfpathlineto{\pgfqpoint{1.466946in}{4.257673in}}%
\pgfpathlineto{\pgfqpoint{1.549334in}{4.300449in}}%
\pgfpathlineto{\pgfqpoint{1.655742in}{4.352967in}}%
\pgfpathlineto{\pgfqpoint{1.768182in}{4.405486in}}%
\pgfpathlineto{\pgfqpoint{1.887456in}{4.458201in}}%
\pgfpathlineto{\pgfqpoint{2.027626in}{4.516464in}}%
\pgfpathlineto{\pgfqpoint{2.146111in}{4.563043in}}%
\pgfpathlineto{\pgfqpoint{2.287400in}{4.615562in}}%
\pgfpathlineto{\pgfqpoint{2.437423in}{4.668081in}}%
\pgfpathlineto{\pgfqpoint{2.588307in}{4.717806in}}%
\pgfpathlineto{\pgfqpoint{2.693435in}{4.750755in}}%
\pgfpathlineto{\pgfqpoint{2.856599in}{4.799378in}}%
\pgfpathlineto{\pgfqpoint{3.008818in}{4.842041in}}%
\pgfpathlineto{\pgfqpoint{3.148988in}{4.879357in}}%
\pgfpathlineto{\pgfqpoint{3.324200in}{4.923209in}}%
\pgfpathlineto{\pgfqpoint{3.466741in}{4.956935in}}%
\pgfpathlineto{\pgfqpoint{3.639583in}{4.995297in}}%
\pgfpathlineto{\pgfqpoint{3.814796in}{5.031774in}}%
\pgfpathlineto{\pgfqpoint{3.969848in}{5.061973in}}%
\pgfpathlineto{\pgfqpoint{4.130179in}{5.091154in}}%
\pgfpathlineto{\pgfqpoint{4.305391in}{5.120729in}}%
\pgfpathlineto{\pgfqpoint{4.480604in}{5.147928in}}%
\pgfpathlineto{\pgfqpoint{4.655817in}{5.172714in}}%
\pgfpathlineto{\pgfqpoint{4.831030in}{5.195038in}}%
\pgfpathlineto{\pgfqpoint{5.006242in}{5.214672in}}%
\pgfpathlineto{\pgfqpoint{5.146412in}{5.228394in}}%
\pgfpathlineto{\pgfqpoint{5.321625in}{5.242990in}}%
\pgfpathlineto{\pgfqpoint{5.461795in}{5.252264in}}%
\pgfpathlineto{\pgfqpoint{5.601965in}{5.259282in}}%
\pgfpathlineto{\pgfqpoint{5.742136in}{5.263817in}}%
\pgfpathlineto{\pgfqpoint{5.847263in}{5.265349in}}%
\pgfpathlineto{\pgfqpoint{5.952391in}{5.265055in}}%
\pgfpathlineto{\pgfqpoint{6.057518in}{5.262688in}}%
\pgfpathlineto{\pgfqpoint{6.162646in}{5.257955in}}%
\pgfpathlineto{\pgfqpoint{6.267774in}{5.250502in}}%
\pgfpathlineto{\pgfqpoint{6.337859in}{5.243710in}}%
\pgfpathlineto{\pgfqpoint{6.407944in}{5.235030in}}%
\pgfpathlineto{\pgfqpoint{6.478029in}{5.224502in}}%
\pgfpathlineto{\pgfqpoint{6.548114in}{5.211454in}}%
\pgfpathlineto{\pgfqpoint{6.627381in}{5.193271in}}%
\pgfpathlineto{\pgfqpoint{6.688284in}{5.175756in}}%
\pgfpathlineto{\pgfqpoint{6.723327in}{5.164251in}}%
\pgfpathlineto{\pgfqpoint{6.783713in}{5.140752in}}%
\pgfpathlineto{\pgfqpoint{6.828454in}{5.119497in}}%
\pgfpathlineto{\pgfqpoint{6.838169in}{5.114492in}}%
\pgfpathlineto{\pgfqpoint{6.882356in}{5.088233in}}%
\pgfpathlineto{\pgfqpoint{6.918586in}{5.061973in}}%
\pgfpathlineto{\pgfqpoint{6.948397in}{5.035714in}}%
\pgfpathlineto{\pgfqpoint{6.972971in}{5.009454in}}%
\pgfpathlineto{\pgfqpoint{6.993027in}{4.983195in}}%
\pgfpathlineto{\pgfqpoint{7.009372in}{4.956935in}}%
\pgfpathlineto{\pgfqpoint{7.022411in}{4.930676in}}%
\pgfpathlineto{\pgfqpoint{7.032700in}{4.904416in}}%
\pgfpathlineto{\pgfqpoint{7.040516in}{4.878157in}}%
\pgfpathlineto{\pgfqpoint{7.046098in}{4.851897in}}%
\pgfpathlineto{\pgfqpoint{7.049778in}{4.825638in}}%
\pgfpathlineto{\pgfqpoint{7.051759in}{4.799378in}}%
\pgfpathlineto{\pgfqpoint{7.052222in}{4.773119in}}%
\pgfpathlineto{\pgfqpoint{7.051325in}{4.746860in}}%
\pgfpathlineto{\pgfqpoint{7.046012in}{4.694341in}}%
\pgfpathlineto{\pgfqpoint{7.036771in}{4.641822in}}%
\pgfpathlineto{\pgfqpoint{7.024285in}{4.589303in}}%
\pgfpathlineto{\pgfqpoint{7.009320in}{4.536784in}}%
\pgfpathlineto{\pgfqpoint{6.992282in}{4.484265in}}%
\pgfpathlineto{\pgfqpoint{6.963905in}{4.405486in}}%
\pgfpathlineto{\pgfqpoint{6.948866in}{4.366529in}}%
\pgfpathlineto{\pgfqpoint{6.948866in}{4.366529in}}%
\pgfusepath{stroke}%
\end{pgfscope}%
\begin{pgfscope}%
\pgfpathrectangle{\pgfqpoint{0.766095in}{0.571603in}}{\pgfqpoint{6.973465in}{5.225635in}}%
\pgfusepath{clip}%
\pgfsetbuttcap%
\pgfsetroundjoin%
\pgfsetlinewidth{1.505625pt}%
\definecolor{currentstroke}{rgb}{0.237441,0.305202,0.541921}%
\pgfsetstrokecolor{currentstroke}%
\pgfsetdash{}{0pt}%
\pgfpathmoveto{\pgfqpoint{6.829985in}{4.077292in}}%
\pgfpathlineto{\pgfqpoint{6.782387in}{3.959075in}}%
\pgfpathlineto{\pgfqpoint{6.742670in}{3.854037in}}%
\pgfpathlineto{\pgfqpoint{6.715040in}{3.775259in}}%
\pgfpathlineto{\pgfqpoint{6.688284in}{3.692095in}}%
\pgfpathlineto{\pgfqpoint{6.666594in}{3.617702in}}%
\pgfpathlineto{\pgfqpoint{6.646337in}{3.538924in}}%
\pgfpathlineto{\pgfqpoint{6.628992in}{3.460145in}}%
\pgfpathlineto{\pgfqpoint{6.614807in}{3.381367in}}%
\pgfpathlineto{\pgfqpoint{6.607141in}{3.328848in}}%
\pgfpathlineto{\pgfqpoint{6.601009in}{3.276329in}}%
\pgfpathlineto{\pgfqpoint{6.596445in}{3.223810in}}%
\pgfpathlineto{\pgfqpoint{6.593486in}{3.171291in}}%
\pgfpathlineto{\pgfqpoint{6.592165in}{3.118772in}}%
\pgfpathlineto{\pgfqpoint{6.592518in}{3.066253in}}%
\pgfpathlineto{\pgfqpoint{6.594578in}{3.013734in}}%
\pgfpathlineto{\pgfqpoint{6.598378in}{2.961215in}}%
\pgfpathlineto{\pgfqpoint{6.603952in}{2.908696in}}%
\pgfpathlineto{\pgfqpoint{6.611335in}{2.856177in}}%
\pgfpathlineto{\pgfqpoint{6.620541in}{2.803659in}}%
\pgfpathlineto{\pgfqpoint{6.631551in}{2.751140in}}%
\pgfpathlineto{\pgfqpoint{6.644461in}{2.698621in}}%
\pgfpathlineto{\pgfqpoint{6.659256in}{2.646102in}}%
\pgfpathlineto{\pgfqpoint{6.675936in}{2.593583in}}%
\pgfpathlineto{\pgfqpoint{6.694563in}{2.541064in}}%
\pgfpathlineto{\pgfqpoint{6.715109in}{2.488545in}}%
\pgfpathlineto{\pgfqpoint{6.737602in}{2.436026in}}%
\pgfpathlineto{\pgfqpoint{6.762091in}{2.383507in}}%
\pgfpathlineto{\pgfqpoint{6.793412in}{2.321838in}}%
\pgfpathlineto{\pgfqpoint{6.816943in}{2.278469in}}%
\pgfpathlineto{\pgfqpoint{6.847364in}{2.225950in}}%
\pgfpathlineto{\pgfqpoint{6.879794in}{2.173431in}}%
\pgfpathlineto{\pgfqpoint{6.914237in}{2.120912in}}%
\pgfpathlineto{\pgfqpoint{6.950698in}{2.068393in}}%
\pgfpathlineto{\pgfqpoint{7.003667in}{1.996902in}}%
\pgfpathlineto{\pgfqpoint{7.038709in}{1.952135in}}%
\pgfpathlineto{\pgfqpoint{7.073752in}{1.909087in}}%
\pgfpathlineto{\pgfqpoint{7.116848in}{1.858318in}}%
\pgfpathlineto{\pgfqpoint{7.178880in}{1.789011in}}%
\pgfpathlineto{\pgfqpoint{7.213922in}{1.751440in}}%
\pgfpathlineto{\pgfqpoint{7.284007in}{1.679508in}}%
\pgfpathlineto{\pgfqpoint{7.319050in}{1.644926in}}%
\pgfpathlineto{\pgfqpoint{7.398563in}{1.569463in}}%
\pgfpathlineto{\pgfqpoint{7.486118in}{1.490685in}}%
\pgfpathlineto{\pgfqpoint{7.578247in}{1.411906in}}%
\pgfpathlineto{\pgfqpoint{7.674976in}{1.333128in}}%
\pgfpathlineto{\pgfqpoint{7.739560in}{1.282505in}}%
\pgfpathlineto{\pgfqpoint{7.739560in}{1.282505in}}%
\pgfusepath{stroke}%
\end{pgfscope}%
\begin{pgfscope}%
\pgfpathrectangle{\pgfqpoint{0.766095in}{0.571603in}}{\pgfqpoint{6.973465in}{5.225635in}}%
\pgfusepath{clip}%
\pgfsetbuttcap%
\pgfsetroundjoin%
\pgfsetlinewidth{1.505625pt}%
\definecolor{currentstroke}{rgb}{0.227802,0.326594,0.546532}%
\pgfsetstrokecolor{currentstroke}%
\pgfsetdash{}{0pt}%
\pgfpathmoveto{\pgfqpoint{1.060043in}{0.571603in}}%
\pgfpathlineto{\pgfqpoint{1.046435in}{0.580516in}}%
\pgfpathlineto{\pgfqpoint{1.034915in}{0.588085in}}%
\pgfusepath{stroke}%
\end{pgfscope}%
\begin{pgfscope}%
\pgfpathrectangle{\pgfqpoint{0.766095in}{0.571603in}}{\pgfqpoint{6.973465in}{5.225635in}}%
\pgfusepath{clip}%
\pgfsetbuttcap%
\pgfsetroundjoin%
\pgfsetlinewidth{1.505625pt}%
\definecolor{currentstroke}{rgb}{0.227802,0.326594,0.546532}%
\pgfsetstrokecolor{currentstroke}%
\pgfsetdash{}{0pt}%
\pgfpathmoveto{\pgfqpoint{0.781215in}{0.769224in}}%
\pgfpathlineto{\pgfqpoint{0.766095in}{0.780964in}}%
\pgfusepath{stroke}%
\end{pgfscope}%
\begin{pgfscope}%
\pgfpathrectangle{\pgfqpoint{0.766095in}{0.571603in}}{\pgfqpoint{6.973465in}{5.225635in}}%
\pgfusepath{clip}%
\pgfsetbuttcap%
\pgfsetroundjoin%
\pgfsetlinewidth{1.505625pt}%
\definecolor{currentstroke}{rgb}{0.227802,0.326594,0.546532}%
\pgfsetstrokecolor{currentstroke}%
\pgfsetdash{}{0pt}%
\pgfpathmoveto{\pgfqpoint{0.766095in}{3.920029in}}%
\pgfpathlineto{\pgfqpoint{0.782507in}{3.932816in}}%
\pgfpathlineto{\pgfqpoint{0.801138in}{3.947074in}}%
\pgfpathlineto{\pgfqpoint{0.816961in}{3.959075in}}%
\pgfpathlineto{\pgfqpoint{0.836180in}{3.973395in}}%
\pgfpathlineto{\pgfqpoint{0.852346in}{3.985335in}}%
\pgfpathlineto{\pgfqpoint{0.871223in}{3.999032in}}%
\pgfpathlineto{\pgfqpoint{0.888684in}{4.011594in}}%
\pgfpathlineto{\pgfqpoint{0.906265in}{4.024022in}}%
\pgfpathlineto{\pgfqpoint{0.925998in}{4.037854in}}%
\pgfpathlineto{\pgfqpoint{0.941308in}{4.048400in}}%
\pgfpathlineto{\pgfqpoint{0.964307in}{4.064113in}}%
\pgfpathlineto{\pgfqpoint{0.976350in}{4.072199in}}%
\pgfpathlineto{\pgfqpoint{1.003635in}{4.090373in}}%
\pgfpathlineto{\pgfqpoint{1.011393in}{4.095451in}}%
\pgfpathlineto{\pgfqpoint{1.044001in}{4.116632in}}%
\pgfpathlineto{\pgfqpoint{1.046435in}{4.118186in}}%
\pgfpathlineto{\pgfqpoint{1.081478in}{4.140373in}}%
\pgfpathlineto{\pgfqpoint{1.085488in}{4.142892in}}%
\pgfpathlineto{\pgfqpoint{1.116520in}{4.162049in}}%
\pgfpathlineto{\pgfqpoint{1.128109in}{4.169151in}}%
\pgfpathlineto{\pgfqpoint{1.151563in}{4.183280in}}%
\pgfpathlineto{\pgfqpoint{1.171842in}{4.195411in}}%
\pgfpathlineto{\pgfqpoint{1.186605in}{4.204091in}}%
\pgfpathlineto{\pgfqpoint{1.216705in}{4.221670in}}%
\pgfpathlineto{\pgfqpoint{1.221648in}{4.224508in}}%
\pgfpathlineto{\pgfqpoint{1.256691in}{4.244469in}}%
\pgfpathlineto{\pgfqpoint{1.262812in}{4.247930in}}%
\pgfpathlineto{\pgfqpoint{1.291733in}{4.264002in}}%
\pgfpathlineto{\pgfqpoint{1.310180in}{4.274189in}}%
\pgfpathlineto{\pgfqpoint{1.326776in}{4.283198in}}%
\pgfpathlineto{\pgfqpoint{1.358742in}{4.300449in}}%
\pgfpathlineto{\pgfqpoint{1.361818in}{4.302080in}}%
\pgfpathlineto{\pgfqpoint{1.396861in}{4.320517in}}%
\pgfpathlineto{\pgfqpoint{1.408702in}{4.326708in}}%
\pgfpathlineto{\pgfqpoint{1.431903in}{4.338632in}}%
\pgfpathlineto{\pgfqpoint{1.459951in}{4.352967in}}%
\pgfpathlineto{\pgfqpoint{1.466946in}{4.356482in}}%
\pgfpathlineto{\pgfqpoint{1.501988in}{4.373959in}}%
\pgfpathlineto{\pgfqpoint{1.512621in}{4.379227in}}%
\pgfpathlineto{\pgfqpoint{1.537031in}{4.391116in}}%
\pgfpathlineto{\pgfqpoint{1.566683in}{4.405486in}}%
\pgfpathlineto{\pgfqpoint{1.572073in}{4.408055in}}%
\pgfpathlineto{\pgfqpoint{1.607116in}{4.424621in}}%
\pgfpathlineto{\pgfqpoint{1.622275in}{4.431746in}}%
\pgfpathlineto{\pgfqpoint{1.642158in}{4.440933in}}%
\pgfpathlineto{\pgfqpoint{1.677201in}{4.457048in}}%
\pgfpathlineto{\pgfqpoint{1.679301in}{4.458005in}}%
\pgfpathlineto{\pgfqpoint{1.712244in}{4.472771in}}%
\pgfpathlineto{\pgfqpoint{1.737990in}{4.484265in}}%
\pgfpathlineto{\pgfqpoint{1.747286in}{4.488344in}}%
\pgfpathlineto{\pgfqpoint{1.782329in}{4.503614in}}%
\pgfpathlineto{\pgfqpoint{1.798276in}{4.510524in}}%
\pgfpathlineto{\pgfqpoint{1.817371in}{4.518657in}}%
\pgfpathlineto{\pgfqpoint{1.852414in}{4.533508in}}%
\pgfpathlineto{\pgfqpoint{1.860201in}{4.536784in}}%
\pgfpathlineto{\pgfqpoint{1.887456in}{4.548053in}}%
\pgfpathlineto{\pgfqpoint{1.922499in}{4.562495in}}%
\pgfpathlineto{\pgfqpoint{1.923841in}{4.563043in}}%
\pgfpathlineto{\pgfqpoint{1.957541in}{4.576574in}}%
\pgfpathlineto{\pgfqpoint{1.989327in}{4.589303in}}%
\pgfpathlineto{\pgfqpoint{1.992584in}{4.590584in}}%
\pgfpathlineto{\pgfqpoint{2.027626in}{4.604260in}}%
\pgfpathlineto{\pgfqpoint{2.056675in}{4.615562in}}%
\pgfpathlineto{\pgfqpoint{2.062669in}{4.617854in}}%
\pgfpathlineto{\pgfqpoint{2.097711in}{4.631148in}}%
\pgfpathlineto{\pgfqpoint{2.125938in}{4.641822in}}%
\pgfpathlineto{\pgfqpoint{2.132754in}{4.644355in}}%
\pgfpathlineto{\pgfqpoint{2.167797in}{4.657275in}}%
\pgfpathlineto{\pgfqpoint{2.197190in}{4.668081in}}%
\pgfpathlineto{\pgfqpoint{2.202839in}{4.670122in}}%
\pgfpathlineto{\pgfqpoint{2.237882in}{4.682678in}}%
\pgfpathlineto{\pgfqpoint{2.270507in}{4.694341in}}%
\pgfpathlineto{\pgfqpoint{2.272924in}{4.695190in}}%
\pgfpathlineto{\pgfqpoint{2.307967in}{4.707389in}}%
\pgfpathlineto{\pgfqpoint{2.343009in}{4.719566in}}%
\pgfpathlineto{\pgfqpoint{2.346013in}{4.720600in}}%
\pgfpathlineto{\pgfqpoint{2.378052in}{4.731442in}}%
\pgfpathlineto{\pgfqpoint{2.413094in}{4.743269in}}%
\pgfpathlineto{\pgfqpoint{2.423809in}{4.746860in}}%
\pgfpathlineto{\pgfqpoint{2.448137in}{4.754868in}}%
\pgfpathlineto{\pgfqpoint{2.483179in}{4.766354in}}%
\pgfpathlineto{\pgfqpoint{2.503927in}{4.773119in}}%
\pgfpathlineto{\pgfqpoint{2.518222in}{4.777698in}}%
\pgfpathlineto{\pgfqpoint{2.553264in}{4.788849in}}%
\pgfpathlineto{\pgfqpoint{2.586438in}{4.799378in}}%
\pgfpathlineto{\pgfqpoint{2.588307in}{4.799961in}}%
\pgfpathlineto{\pgfqpoint{2.623350in}{4.810784in}}%
\pgfpathlineto{\pgfqpoint{2.658392in}{4.821584in}}%
\pgfpathlineto{\pgfqpoint{2.671641in}{4.825638in}}%
\pgfpathlineto{\pgfqpoint{2.693435in}{4.832188in}}%
\pgfpathlineto{\pgfqpoint{2.728477in}{4.842666in}}%
\pgfpathlineto{\pgfqpoint{2.759452in}{4.851897in}}%
\pgfpathlineto{\pgfqpoint{2.763520in}{4.853088in}}%
\pgfpathlineto{\pgfqpoint{2.798562in}{4.863250in}}%
\pgfpathlineto{\pgfqpoint{2.833605in}{4.873388in}}%
\pgfpathlineto{\pgfqpoint{2.850197in}{4.878157in}}%
\pgfpathlineto{\pgfqpoint{2.868647in}{4.883364in}}%
\pgfpathlineto{\pgfqpoint{2.903690in}{4.893192in}}%
\pgfpathlineto{\pgfqpoint{2.938732in}{4.902995in}}%
\pgfpathlineto{\pgfqpoint{2.943861in}{4.904416in}}%
\pgfpathlineto{\pgfqpoint{2.973775in}{4.912556in}}%
\pgfpathlineto{\pgfqpoint{3.008818in}{4.922055in}}%
\pgfpathlineto{\pgfqpoint{3.040732in}{4.930676in}}%
\pgfpathlineto{\pgfqpoint{3.043860in}{4.931506in}}%
\pgfpathlineto{\pgfqpoint{3.078903in}{4.940706in}}%
\pgfpathlineto{\pgfqpoint{3.113945in}{4.949879in}}%
\pgfpathlineto{\pgfqpoint{3.141028in}{4.956935in}}%
\pgfpathlineto{\pgfqpoint{3.148988in}{4.958971in}}%
\pgfpathlineto{\pgfqpoint{3.184030in}{4.967851in}}%
\pgfpathlineto{\pgfqpoint{3.219073in}{4.976703in}}%
\pgfpathlineto{\pgfqpoint{3.244906in}{4.983195in}}%
\pgfpathlineto{\pgfqpoint{3.254115in}{4.985466in}}%
\pgfpathlineto{\pgfqpoint{3.289158in}{4.994029in}}%
\pgfpathlineto{\pgfqpoint{3.324200in}{5.002564in}}%
\pgfpathlineto{\pgfqpoint{3.352630in}{5.009454in}}%
\pgfpathlineto{\pgfqpoint{3.359243in}{5.011027in}}%
\pgfpathlineto{\pgfqpoint{3.394285in}{5.019277in}}%
\pgfpathlineto{\pgfqpoint{3.429328in}{5.027497in}}%
\pgfpathlineto{\pgfqpoint{3.464371in}{5.035688in}}%
\pgfpathlineto{\pgfqpoint{3.464484in}{5.035714in}}%
\pgfpathlineto{\pgfqpoint{3.499413in}{5.043628in}}%
\pgfpathlineto{\pgfqpoint{3.522605in}{5.048862in}}%
\pgfusepath{stroke}%
\end{pgfscope}%
\begin{pgfscope}%
\pgfpathrectangle{\pgfqpoint{0.766095in}{0.571603in}}{\pgfqpoint{6.973465in}{5.225635in}}%
\pgfusepath{clip}%
\pgfsetbuttcap%
\pgfsetroundjoin%
\pgfsetlinewidth{1.505625pt}%
\definecolor{currentstroke}{rgb}{0.227802,0.326594,0.546532}%
\pgfsetstrokecolor{currentstroke}%
\pgfsetdash{}{0pt}%
\pgfpathmoveto{\pgfqpoint{3.826999in}{5.114036in}}%
\pgfpathlineto{\pgfqpoint{3.962152in}{5.140752in}}%
\pgfpathlineto{\pgfqpoint{4.165221in}{5.178343in}}%
\pgfpathlineto{\pgfqpoint{4.340434in}{5.208525in}}%
\pgfpathlineto{\pgfqpoint{4.515647in}{5.236583in}}%
\pgfpathlineto{\pgfqpoint{4.690859in}{5.262496in}}%
\pgfpathlineto{\pgfqpoint{4.866072in}{5.286235in}}%
\pgfpathlineto{\pgfqpoint{5.041285in}{5.307761in}}%
\pgfpathlineto{\pgfqpoint{5.216497in}{5.327022in}}%
\pgfpathlineto{\pgfqpoint{5.391710in}{5.343710in}}%
\pgfpathlineto{\pgfqpoint{5.531880in}{5.355232in}}%
\pgfpathlineto{\pgfqpoint{5.707093in}{5.366977in}}%
\pgfpathlineto{\pgfqpoint{5.847263in}{5.374211in}}%
\pgfpathlineto{\pgfqpoint{5.987433in}{5.379134in}}%
\pgfpathlineto{\pgfqpoint{6.127603in}{5.381457in}}%
\pgfpathlineto{\pgfqpoint{6.232731in}{5.381322in}}%
\pgfpathlineto{\pgfqpoint{6.337859in}{5.379350in}}%
\pgfpathlineto{\pgfqpoint{6.442986in}{5.375208in}}%
\pgfpathlineto{\pgfqpoint{6.548114in}{5.368449in}}%
\pgfpathlineto{\pgfqpoint{6.653242in}{5.358781in}}%
\pgfpathlineto{\pgfqpoint{6.723327in}{5.350466in}}%
\pgfpathlineto{\pgfqpoint{6.793412in}{5.339933in}}%
\pgfpathlineto{\pgfqpoint{6.877511in}{5.324568in}}%
\pgfpathlineto{\pgfqpoint{6.933582in}{5.311788in}}%
\pgfpathlineto{\pgfqpoint{7.003667in}{5.292846in}}%
\pgfpathlineto{\pgfqpoint{7.073752in}{5.269233in}}%
\pgfpathlineto{\pgfqpoint{7.129773in}{5.245790in}}%
\pgfpathlineto{\pgfqpoint{7.143837in}{5.238925in}}%
\pgfpathlineto{\pgfqpoint{7.180713in}{5.219530in}}%
\pgfpathlineto{\pgfqpoint{7.221955in}{5.193271in}}%
\pgfpathlineto{\pgfqpoint{7.255701in}{5.167011in}}%
\pgfpathlineto{\pgfqpoint{7.284007in}{5.140011in}}%
\pgfpathlineto{\pgfqpoint{7.305751in}{5.114492in}}%
\pgfpathlineto{\pgfqpoint{7.323963in}{5.088233in}}%
\pgfpathlineto{\pgfqpoint{7.338407in}{5.061973in}}%
\pgfpathlineto{\pgfqpoint{7.349743in}{5.035714in}}%
\pgfpathlineto{\pgfqpoint{7.358256in}{5.009454in}}%
\pgfpathlineto{\pgfqpoint{7.364295in}{4.983195in}}%
\pgfpathlineto{\pgfqpoint{7.368197in}{4.956935in}}%
\pgfpathlineto{\pgfqpoint{7.370191in}{4.930676in}}%
\pgfpathlineto{\pgfqpoint{7.370480in}{4.904416in}}%
\pgfpathlineto{\pgfqpoint{7.369245in}{4.878157in}}%
\pgfpathlineto{\pgfqpoint{7.366647in}{4.851897in}}%
\pgfpathlineto{\pgfqpoint{7.357918in}{4.799378in}}%
\pgfpathlineto{\pgfqpoint{7.345159in}{4.746860in}}%
\pgfpathlineto{\pgfqpoint{7.329200in}{4.694341in}}%
\pgfpathlineto{\pgfqpoint{7.310652in}{4.641822in}}%
\pgfpathlineto{\pgfqpoint{7.284007in}{4.574898in}}%
\pgfpathlineto{\pgfqpoint{7.256114in}{4.510524in}}%
\pgfpathlineto{\pgfqpoint{7.213922in}{4.419617in}}%
\pgfpathlineto{\pgfqpoint{7.156064in}{4.300449in}}%
\pgfpathlineto{\pgfqpoint{7.053861in}{4.090373in}}%
\pgfpathlineto{\pgfqpoint{7.003667in}{3.981614in}}%
\pgfpathlineto{\pgfqpoint{6.968624in}{3.901348in}}%
\pgfpathlineto{\pgfqpoint{6.933582in}{3.815693in}}%
\pgfpathlineto{\pgfqpoint{6.908154in}{3.749000in}}%
\pgfpathlineto{\pgfqpoint{6.880561in}{3.670221in}}%
\pgfpathlineto{\pgfqpoint{6.855800in}{3.591443in}}%
\pgfpathlineto{\pgfqpoint{6.834052in}{3.512664in}}%
\pgfpathlineto{\pgfqpoint{6.815489in}{3.433886in}}%
\pgfpathlineto{\pgfqpoint{6.804991in}{3.381367in}}%
\pgfpathlineto{\pgfqpoint{6.793412in}{3.310897in}}%
\pgfpathlineto{\pgfqpoint{6.788685in}{3.276329in}}%
\pgfpathlineto{\pgfqpoint{6.782932in}{3.223810in}}%
\pgfpathlineto{\pgfqpoint{6.778852in}{3.171291in}}%
\pgfpathlineto{\pgfqpoint{6.776476in}{3.118772in}}%
\pgfpathlineto{\pgfqpoint{6.775833in}{3.066253in}}%
\pgfpathlineto{\pgfqpoint{6.776952in}{3.013734in}}%
\pgfpathlineto{\pgfqpoint{6.779863in}{2.961215in}}%
\pgfpathlineto{\pgfqpoint{6.784596in}{2.908696in}}%
\pgfpathlineto{\pgfqpoint{6.793412in}{2.841541in}}%
\pgfpathlineto{\pgfqpoint{6.799601in}{2.803659in}}%
\pgfpathlineto{\pgfqpoint{6.809903in}{2.751140in}}%
\pgfpathlineto{\pgfqpoint{6.822140in}{2.698621in}}%
\pgfpathlineto{\pgfqpoint{6.836282in}{2.646102in}}%
\pgfpathlineto{\pgfqpoint{6.852362in}{2.593583in}}%
\pgfpathlineto{\pgfqpoint{6.870414in}{2.541064in}}%
\pgfpathlineto{\pgfqpoint{6.890425in}{2.488545in}}%
\pgfpathlineto{\pgfqpoint{6.912414in}{2.436026in}}%
\pgfpathlineto{\pgfqpoint{6.936431in}{2.383507in}}%
\pgfpathlineto{\pgfqpoint{6.968624in}{2.319186in}}%
\pgfpathlineto{\pgfqpoint{6.990429in}{2.278469in}}%
\pgfpathlineto{\pgfqpoint{7.020472in}{2.225950in}}%
\pgfpathlineto{\pgfqpoint{7.052553in}{2.173431in}}%
\pgfpathlineto{\pgfqpoint{7.086676in}{2.120912in}}%
\pgfpathlineto{\pgfqpoint{7.122844in}{2.068393in}}%
\pgfpathlineto{\pgfqpoint{7.161066in}{2.015874in}}%
\pgfpathlineto{\pgfqpoint{7.213922in}{1.947599in}}%
\pgfpathlineto{\pgfqpoint{7.248965in}{1.904559in}}%
\pgfpathlineto{\pgfqpoint{7.288121in}{1.858318in}}%
\pgfpathlineto{\pgfqpoint{7.354092in}{1.784490in}}%
\pgfpathlineto{\pgfqpoint{7.389135in}{1.747006in}}%
\pgfpathlineto{\pgfqpoint{7.433753in}{1.700761in}}%
\pgfpathlineto{\pgfqpoint{7.513561in}{1.621982in}}%
\pgfpathlineto{\pgfqpoint{7.599390in}{1.542012in}}%
\pgfpathlineto{\pgfqpoint{7.687169in}{1.464425in}}%
\pgfpathlineto{\pgfqpoint{7.739560in}{1.419992in}}%
\pgfpathlineto{\pgfqpoint{7.739560in}{1.419992in}}%
\pgfusepath{stroke}%
\end{pgfscope}%
\begin{pgfscope}%
\pgfpathrectangle{\pgfqpoint{0.766095in}{0.571603in}}{\pgfqpoint{6.973465in}{5.225635in}}%
\pgfusepath{clip}%
\pgfsetbuttcap%
\pgfsetroundjoin%
\pgfsetlinewidth{1.505625pt}%
\definecolor{currentstroke}{rgb}{0.218130,0.347432,0.550038}%
\pgfsetstrokecolor{currentstroke}%
\pgfsetdash{}{0pt}%
\pgfpathmoveto{\pgfqpoint{0.927694in}{0.571603in}}%
\pgfpathlineto{\pgfqpoint{0.906265in}{0.585888in}}%
\pgfpathlineto{\pgfqpoint{0.888364in}{0.597863in}}%
\pgfpathlineto{\pgfqpoint{0.871223in}{0.609553in}}%
\pgfpathlineto{\pgfqpoint{0.849939in}{0.624122in}}%
\pgfpathlineto{\pgfqpoint{0.836180in}{0.633725in}}%
\pgfpathlineto{\pgfqpoint{0.812408in}{0.650382in}}%
\pgfpathlineto{\pgfqpoint{0.801138in}{0.658434in}}%
\pgfpathlineto{\pgfqpoint{0.775758in}{0.676641in}}%
\pgfpathlineto{\pgfqpoint{0.766095in}{0.683711in}}%
\pgfusepath{stroke}%
\end{pgfscope}%
\begin{pgfscope}%
\pgfpathrectangle{\pgfqpoint{0.766095in}{0.571603in}}{\pgfqpoint{6.973465in}{5.225635in}}%
\pgfusepath{clip}%
\pgfsetbuttcap%
\pgfsetroundjoin%
\pgfsetlinewidth{1.505625pt}%
\definecolor{currentstroke}{rgb}{0.218130,0.347432,0.550038}%
\pgfsetstrokecolor{currentstroke}%
\pgfsetdash{}{0pt}%
\pgfpathmoveto{\pgfqpoint{0.766095in}{4.036275in}}%
\pgfpathlineto{\pgfqpoint{0.805121in}{4.064113in}}%
\pgfpathlineto{\pgfqpoint{0.881778in}{4.116632in}}%
\pgfpathlineto{\pgfqpoint{0.976350in}{4.177857in}}%
\pgfpathlineto{\pgfqpoint{1.047487in}{4.221670in}}%
\pgfpathlineto{\pgfqpoint{1.151563in}{4.282379in}}%
\pgfpathlineto{\pgfqpoint{1.231459in}{4.326708in}}%
\pgfpathlineto{\pgfqpoint{1.330836in}{4.379227in}}%
\pgfpathlineto{\pgfqpoint{1.435574in}{4.431746in}}%
\pgfpathlineto{\pgfqpoint{1.545993in}{4.484265in}}%
\pgfpathlineto{\pgfqpoint{1.677201in}{4.543240in}}%
\pgfpathlineto{\pgfqpoint{1.785103in}{4.589303in}}%
\pgfpathlineto{\pgfqpoint{1.922499in}{4.644891in}}%
\pgfpathlineto{\pgfqpoint{2.062669in}{4.698494in}}%
\pgfpathlineto{\pgfqpoint{2.202839in}{4.749260in}}%
\pgfpathlineto{\pgfqpoint{2.348895in}{4.799378in}}%
\pgfpathlineto{\pgfqpoint{2.518222in}{4.854218in}}%
\pgfpathlineto{\pgfqpoint{2.693435in}{4.907632in}}%
\pgfpathlineto{\pgfqpoint{2.868647in}{4.957979in}}%
\pgfpathlineto{\pgfqpoint{3.059143in}{5.009454in}}%
\pgfpathlineto{\pgfqpoint{3.254115in}{5.058915in}}%
\pgfpathlineto{\pgfqpoint{3.394285in}{5.092593in}}%
\pgfpathlineto{\pgfqpoint{3.606117in}{5.140752in}}%
\pgfpathlineto{\pgfqpoint{3.814796in}{5.184952in}}%
\pgfpathlineto{\pgfqpoint{3.990009in}{5.219923in}}%
\pgfpathlineto{\pgfqpoint{4.200264in}{5.259070in}}%
\pgfpathlineto{\pgfqpoint{4.375477in}{5.289632in}}%
\pgfpathlineto{\pgfqpoint{4.550689in}{5.318280in}}%
\pgfpathlineto{\pgfqpoint{4.725902in}{5.345008in}}%
\pgfpathlineto{\pgfqpoint{4.901115in}{5.369801in}}%
\pgfpathlineto{\pgfqpoint{5.076327in}{5.392638in}}%
\pgfpathlineto{\pgfqpoint{5.251540in}{5.413487in}}%
\pgfpathlineto{\pgfqpoint{5.426753in}{5.432308in}}%
\pgfpathlineto{\pgfqpoint{5.601965in}{5.448812in}}%
\pgfpathlineto{\pgfqpoint{5.742136in}{5.460380in}}%
\pgfpathlineto{\pgfqpoint{5.917348in}{5.472470in}}%
\pgfpathlineto{\pgfqpoint{6.057518in}{5.480251in}}%
\pgfpathlineto{\pgfqpoint{6.197689in}{5.485940in}}%
\pgfpathlineto{\pgfqpoint{6.337859in}{5.489402in}}%
\pgfpathlineto{\pgfqpoint{6.478029in}{5.490369in}}%
\pgfpathlineto{\pgfqpoint{6.583156in}{5.489210in}}%
\pgfpathlineto{\pgfqpoint{6.688284in}{5.486203in}}%
\pgfpathlineto{\pgfqpoint{6.793412in}{5.481045in}}%
\pgfpathlineto{\pgfqpoint{6.898539in}{5.473148in}}%
\pgfpathlineto{\pgfqpoint{7.003667in}{5.462298in}}%
\pgfpathlineto{\pgfqpoint{7.073752in}{5.453029in}}%
\pgfpathlineto{\pgfqpoint{7.143837in}{5.441565in}}%
\pgfpathlineto{\pgfqpoint{7.213922in}{5.427925in}}%
\pgfpathlineto{\pgfqpoint{7.284007in}{5.410991in}}%
\pgfpathlineto{\pgfqpoint{7.319050in}{5.401313in}}%
\pgfpathlineto{\pgfqpoint{7.392258in}{5.377087in}}%
\pgfpathlineto{\pgfqpoint{7.455002in}{5.350827in}}%
\pgfpathlineto{\pgfqpoint{7.494263in}{5.330630in}}%
\pgfpathlineto{\pgfqpoint{7.505208in}{5.324568in}}%
\pgfpathlineto{\pgfqpoint{7.545880in}{5.298308in}}%
\pgfpathlineto{\pgfqpoint{7.578969in}{5.272049in}}%
\pgfpathlineto{\pgfqpoint{7.605882in}{5.245790in}}%
\pgfpathlineto{\pgfqpoint{7.627592in}{5.219530in}}%
\pgfpathlineto{\pgfqpoint{7.644912in}{5.193271in}}%
\pgfpathlineto{\pgfqpoint{7.658496in}{5.167011in}}%
\pgfpathlineto{\pgfqpoint{7.669475in}{5.138767in}}%
\pgfpathlineto{\pgfqpoint{7.676343in}{5.114492in}}%
\pgfpathlineto{\pgfqpoint{7.681323in}{5.088233in}}%
\pgfpathlineto{\pgfqpoint{7.684107in}{5.061973in}}%
\pgfpathlineto{\pgfqpoint{7.684932in}{5.035714in}}%
\pgfpathlineto{\pgfqpoint{7.684007in}{5.009454in}}%
\pgfpathlineto{\pgfqpoint{7.681517in}{4.983195in}}%
\pgfpathlineto{\pgfqpoint{7.677627in}{4.956935in}}%
\pgfpathlineto{\pgfqpoint{7.669475in}{4.918206in}}%
\pgfpathlineto{\pgfqpoint{7.658818in}{4.878157in}}%
\pgfpathlineto{\pgfqpoint{7.641485in}{4.825638in}}%
\pgfpathlineto{\pgfqpoint{7.621115in}{4.773119in}}%
\pgfpathlineto{\pgfqpoint{7.598368in}{4.720600in}}%
\pgfpathlineto{\pgfqpoint{7.560644in}{4.641822in}}%
\pgfpathlineto{\pgfqpoint{7.519850in}{4.563043in}}%
\pgfpathlineto{\pgfqpoint{7.459220in}{4.452029in}}%
\pgfpathlineto{\pgfqpoint{7.391934in}{4.331982in}}%
\pgfpathlineto{\pgfqpoint{7.391934in}{4.331982in}}%
\pgfusepath{stroke}%
\end{pgfscope}%
\begin{pgfscope}%
\pgfpathrectangle{\pgfqpoint{0.766095in}{0.571603in}}{\pgfqpoint{6.973465in}{5.225635in}}%
\pgfusepath{clip}%
\pgfsetbuttcap%
\pgfsetroundjoin%
\pgfsetlinewidth{1.505625pt}%
\definecolor{currentstroke}{rgb}{0.218130,0.347432,0.550038}%
\pgfsetstrokecolor{currentstroke}%
\pgfsetdash{}{0pt}%
\pgfpathmoveto{\pgfqpoint{7.242463in}{4.057484in}}%
\pgfpathlineto{\pgfqpoint{7.232351in}{4.037854in}}%
\pgfpathlineto{\pgfqpoint{7.219074in}{4.011594in}}%
\pgfpathlineto{\pgfqpoint{7.213922in}{4.001258in}}%
\pgfpathlineto{\pgfqpoint{7.205985in}{3.985335in}}%
\pgfpathlineto{\pgfqpoint{7.193120in}{3.959075in}}%
\pgfpathlineto{\pgfqpoint{7.180535in}{3.932816in}}%
\pgfpathlineto{\pgfqpoint{7.178880in}{3.929307in}}%
\pgfpathlineto{\pgfqpoint{7.168141in}{3.906556in}}%
\pgfpathlineto{\pgfqpoint{7.156031in}{3.880297in}}%
\pgfpathlineto{\pgfqpoint{7.144226in}{3.854037in}}%
\pgfpathlineto{\pgfqpoint{7.143837in}{3.853156in}}%
\pgfpathlineto{\pgfqpoint{7.132633in}{3.827778in}}%
\pgfpathlineto{\pgfqpoint{7.121359in}{3.801519in}}%
\pgfpathlineto{\pgfqpoint{7.110412in}{3.775259in}}%
\pgfpathlineto{\pgfqpoint{7.108795in}{3.771281in}}%
\pgfpathlineto{\pgfqpoint{7.099719in}{3.749000in}}%
\pgfpathlineto{\pgfqpoint{7.089353in}{3.722740in}}%
\pgfpathlineto{\pgfqpoint{7.079335in}{3.696481in}}%
\pgfpathlineto{\pgfqpoint{7.073752in}{3.681357in}}%
\pgfpathlineto{\pgfqpoint{7.069632in}{3.670221in}}%
\pgfpathlineto{\pgfqpoint{7.060241in}{3.643962in}}%
\pgfpathlineto{\pgfqpoint{7.051215in}{3.617702in}}%
\pgfpathlineto{\pgfqpoint{7.042558in}{3.591443in}}%
\pgfpathlineto{\pgfqpoint{7.038709in}{3.579282in}}%
\pgfpathlineto{\pgfqpoint{7.034235in}{3.565183in}}%
\pgfpathlineto{\pgfqpoint{7.026261in}{3.538924in}}%
\pgfpathlineto{\pgfqpoint{7.018671in}{3.512664in}}%
\pgfpathlineto{\pgfqpoint{7.011469in}{3.486405in}}%
\pgfpathlineto{\pgfqpoint{7.004659in}{3.460145in}}%
\pgfpathlineto{\pgfqpoint{7.003667in}{3.456099in}}%
\pgfpathlineto{\pgfqpoint{6.998200in}{3.433886in}}%
\pgfpathlineto{\pgfqpoint{6.992134in}{3.407626in}}%
\pgfpathlineto{\pgfqpoint{6.986475in}{3.381367in}}%
\pgfpathlineto{\pgfqpoint{6.981226in}{3.355107in}}%
\pgfpathlineto{\pgfqpoint{6.976390in}{3.328848in}}%
\pgfpathlineto{\pgfqpoint{6.971969in}{3.302589in}}%
\pgfpathlineto{\pgfqpoint{6.968624in}{3.280648in}}%
\pgfpathlineto{\pgfqpoint{6.967963in}{3.276329in}}%
\pgfpathlineto{\pgfqpoint{6.964355in}{3.250070in}}%
\pgfpathlineto{\pgfqpoint{6.961177in}{3.223810in}}%
\pgfpathlineto{\pgfqpoint{6.958431in}{3.197551in}}%
\pgfpathlineto{\pgfqpoint{6.956122in}{3.171291in}}%
\pgfpathlineto{\pgfqpoint{6.954251in}{3.145032in}}%
\pgfpathlineto{\pgfqpoint{6.952823in}{3.118772in}}%
\pgfpathlineto{\pgfqpoint{6.951840in}{3.092513in}}%
\pgfpathlineto{\pgfqpoint{6.951306in}{3.066253in}}%
\pgfpathlineto{\pgfqpoint{6.951223in}{3.039994in}}%
\pgfpathlineto{\pgfqpoint{6.951596in}{3.013734in}}%
\pgfpathlineto{\pgfqpoint{6.952428in}{2.987475in}}%
\pgfpathlineto{\pgfqpoint{6.953721in}{2.961215in}}%
\pgfpathlineto{\pgfqpoint{6.955480in}{2.934956in}}%
\pgfpathlineto{\pgfqpoint{6.957707in}{2.908696in}}%
\pgfpathlineto{\pgfqpoint{6.960406in}{2.882437in}}%
\pgfpathlineto{\pgfqpoint{6.963582in}{2.856177in}}%
\pgfpathlineto{\pgfqpoint{6.967236in}{2.829918in}}%
\pgfpathlineto{\pgfqpoint{6.968624in}{2.821113in}}%
\pgfpathlineto{\pgfqpoint{6.971352in}{2.803659in}}%
\pgfpathlineto{\pgfqpoint{6.975940in}{2.777399in}}%
\pgfpathlineto{\pgfqpoint{6.981015in}{2.751140in}}%
\pgfpathlineto{\pgfqpoint{6.986581in}{2.724880in}}%
\pgfpathlineto{\pgfqpoint{6.992642in}{2.698621in}}%
\pgfpathlineto{\pgfqpoint{6.999201in}{2.672361in}}%
\pgfpathlineto{\pgfqpoint{7.003667in}{2.655757in}}%
\pgfpathlineto{\pgfqpoint{7.006243in}{2.646102in}}%
\pgfpathlineto{\pgfqpoint{7.013754in}{2.619842in}}%
\pgfpathlineto{\pgfqpoint{7.021772in}{2.593583in}}%
\pgfpathlineto{\pgfqpoint{7.030303in}{2.567323in}}%
\pgfpathlineto{\pgfqpoint{7.038709in}{2.542924in}}%
\pgfpathlineto{\pgfqpoint{7.039346in}{2.541064in}}%
\pgfpathlineto{\pgfqpoint{7.048840in}{2.514804in}}%
\pgfpathlineto{\pgfqpoint{7.058857in}{2.488545in}}%
\pgfpathlineto{\pgfqpoint{7.069401in}{2.462285in}}%
\pgfpathlineto{\pgfqpoint{7.073752in}{2.451962in}}%
\pgfpathlineto{\pgfqpoint{7.080425in}{2.436026in}}%
\pgfpathlineto{\pgfqpoint{7.091949in}{2.409766in}}%
\pgfpathlineto{\pgfqpoint{7.104010in}{2.383507in}}%
\pgfpathlineto{\pgfqpoint{7.108795in}{2.373531in}}%
\pgfpathlineto{\pgfqpoint{7.116556in}{2.357248in}}%
\pgfpathlineto{\pgfqpoint{7.129610in}{2.330988in}}%
\pgfpathlineto{\pgfqpoint{7.143214in}{2.304729in}}%
\pgfpathlineto{\pgfqpoint{7.143837in}{2.303571in}}%
\pgfpathlineto{\pgfqpoint{7.157271in}{2.278469in}}%
\pgfpathlineto{\pgfqpoint{7.171881in}{2.252210in}}%
\pgfpathlineto{\pgfqpoint{7.178880in}{2.240083in}}%
\pgfpathlineto{\pgfqpoint{7.186993in}{2.225950in}}%
\pgfpathlineto{\pgfqpoint{7.202617in}{2.199691in}}%
\pgfpathlineto{\pgfqpoint{7.213922in}{2.181346in}}%
\pgfpathlineto{\pgfqpoint{7.218776in}{2.173431in}}%
\pgfpathlineto{\pgfqpoint{7.235422in}{2.147172in}}%
\pgfpathlineto{\pgfqpoint{7.248965in}{2.126517in}}%
\pgfpathlineto{\pgfqpoint{7.252622in}{2.120912in}}%
\pgfpathlineto{\pgfqpoint{7.270302in}{2.094653in}}%
\pgfpathlineto{\pgfqpoint{7.284007in}{2.074937in}}%
\pgfpathlineto{\pgfqpoint{7.288537in}{2.068393in}}%
\pgfpathlineto{\pgfqpoint{7.307260in}{2.042134in}}%
\pgfpathlineto{\pgfqpoint{7.319050in}{2.026090in}}%
\pgfpathlineto{\pgfqpoint{7.326526in}{2.015874in}}%
\pgfpathlineto{\pgfqpoint{7.346305in}{1.989615in}}%
\pgfpathlineto{\pgfqpoint{7.354092in}{1.979561in}}%
\pgfpathlineto{\pgfqpoint{7.366599in}{1.963355in}}%
\pgfpathlineto{\pgfqpoint{7.387446in}{1.937096in}}%
\pgfpathlineto{\pgfqpoint{7.389135in}{1.935022in}}%
\pgfpathlineto{\pgfqpoint{7.408765in}{1.910836in}}%
\pgfpathlineto{\pgfqpoint{7.424177in}{1.892358in}}%
\pgfpathlineto{\pgfqpoint{7.430646in}{1.884577in}}%
\pgfpathlineto{\pgfqpoint{7.453039in}{1.858318in}}%
\pgfpathlineto{\pgfqpoint{7.459220in}{1.851245in}}%
\pgfpathlineto{\pgfqpoint{7.475940in}{1.832058in}}%
\pgfpathlineto{\pgfqpoint{7.494263in}{1.811558in}}%
\pgfpathlineto{\pgfqpoint{7.499396in}{1.805799in}}%
\pgfpathlineto{\pgfqpoint{7.523356in}{1.779539in}}%
\pgfpathlineto{\pgfqpoint{7.529305in}{1.773168in}}%
\pgfpathlineto{\pgfqpoint{7.547829in}{1.753280in}}%
\pgfpathlineto{\pgfqpoint{7.564348in}{1.735957in}}%
\pgfpathlineto{\pgfqpoint{7.572849in}{1.727020in}}%
\pgfpathlineto{\pgfqpoint{7.598406in}{1.700761in}}%
\pgfpathlineto{\pgfqpoint{7.599390in}{1.699770in}}%
\pgfpathlineto{\pgfqpoint{7.624443in}{1.674501in}}%
\pgfpathlineto{\pgfqpoint{7.634433in}{1.664643in}}%
\pgfpathlineto{\pgfqpoint{7.651022in}{1.648242in}}%
\pgfpathlineto{\pgfqpoint{7.669475in}{1.630383in}}%
\pgfpathlineto{\pgfqpoint{7.678140in}{1.621982in}}%
\pgfpathlineto{\pgfqpoint{7.704518in}{1.596936in}}%
\pgfpathlineto{\pgfqpoint{7.705794in}{1.595723in}}%
\pgfpathlineto{\pgfqpoint{7.733939in}{1.569463in}}%
\pgfpathlineto{\pgfqpoint{7.739560in}{1.564321in}}%
\pgfusepath{stroke}%
\end{pgfscope}%
\begin{pgfscope}%
\pgfpathrectangle{\pgfqpoint{0.766095in}{0.571603in}}{\pgfqpoint{6.973465in}{5.225635in}}%
\pgfusepath{clip}%
\pgfsetbuttcap%
\pgfsetroundjoin%
\pgfsetlinewidth{1.505625pt}%
\definecolor{currentstroke}{rgb}{0.208623,0.367752,0.552675}%
\pgfsetstrokecolor{currentstroke}%
\pgfsetdash{}{0pt}%
\pgfpathmoveto{\pgfqpoint{0.801341in}{0.571603in}}%
\pgfpathlineto{\pgfqpoint{0.801138in}{0.571741in}}%
\pgfpathlineto{\pgfqpoint{0.766095in}{0.595521in}}%
\pgfusepath{stroke}%
\end{pgfscope}%
\begin{pgfscope}%
\pgfpathrectangle{\pgfqpoint{0.766095in}{0.571603in}}{\pgfqpoint{6.973465in}{5.225635in}}%
\pgfusepath{clip}%
\pgfsetbuttcap%
\pgfsetroundjoin%
\pgfsetlinewidth{1.505625pt}%
\definecolor{currentstroke}{rgb}{0.208623,0.367752,0.552675}%
\pgfsetstrokecolor{currentstroke}%
\pgfsetdash{}{0pt}%
\pgfpathmoveto{\pgfqpoint{0.766095in}{4.141070in}}%
\pgfpathlineto{\pgfqpoint{0.836180in}{4.187342in}}%
\pgfpathlineto{\pgfqpoint{0.906265in}{4.231645in}}%
\pgfpathlineto{\pgfqpoint{0.976350in}{4.274193in}}%
\pgfpathlineto{\pgfqpoint{1.081478in}{4.334707in}}%
\pgfpathlineto{\pgfqpoint{1.162729in}{4.379227in}}%
\pgfpathlineto{\pgfqpoint{1.263299in}{4.431746in}}%
\pgfpathlineto{\pgfqpoint{1.369204in}{4.484265in}}%
\pgfpathlineto{\pgfqpoint{1.480740in}{4.536784in}}%
\pgfpathlineto{\pgfqpoint{1.607116in}{4.593167in}}%
\pgfpathlineto{\pgfqpoint{1.721954in}{4.641822in}}%
\pgfpathlineto{\pgfqpoint{1.852414in}{4.694362in}}%
\pgfpathlineto{\pgfqpoint{1.992584in}{4.747831in}}%
\pgfpathlineto{\pgfqpoint{2.135141in}{4.799378in}}%
\pgfpathlineto{\pgfqpoint{2.307967in}{4.858321in}}%
\pgfpathlineto{\pgfqpoint{2.450528in}{4.904416in}}%
\pgfpathlineto{\pgfqpoint{2.623350in}{4.957330in}}%
\pgfpathlineto{\pgfqpoint{2.803773in}{5.009454in}}%
\pgfpathlineto{\pgfqpoint{3.008818in}{5.065146in}}%
\pgfpathlineto{\pgfqpoint{3.201944in}{5.114492in}}%
\pgfpathlineto{\pgfqpoint{3.394285in}{5.160768in}}%
\pgfpathlineto{\pgfqpoint{3.535995in}{5.193271in}}%
\pgfpathlineto{\pgfqpoint{3.744711in}{5.238530in}}%
\pgfpathlineto{\pgfqpoint{3.919924in}{5.274469in}}%
\pgfpathlineto{\pgfqpoint{4.130179in}{5.315016in}}%
\pgfpathlineto{\pgfqpoint{4.305391in}{5.346895in}}%
\pgfpathlineto{\pgfqpoint{4.481018in}{5.377087in}}%
\pgfpathlineto{\pgfqpoint{4.725902in}{5.416113in}}%
\pgfpathlineto{\pgfqpoint{4.936157in}{5.447023in}}%
\pgfpathlineto{\pgfqpoint{5.044093in}{5.461932in}}%
\pgfpathlineto{\pgfqpoint{5.044093in}{5.461932in}}%
\pgfusepath{stroke}%
\end{pgfscope}%
\begin{pgfscope}%
\pgfpathrectangle{\pgfqpoint{0.766095in}{0.571603in}}{\pgfqpoint{6.973465in}{5.225635in}}%
\pgfusepath{clip}%
\pgfsetbuttcap%
\pgfsetroundjoin%
\pgfsetlinewidth{1.505625pt}%
\definecolor{currentstroke}{rgb}{0.208623,0.367752,0.552675}%
\pgfsetstrokecolor{currentstroke}%
\pgfsetdash{}{0pt}%
\pgfpathmoveto{\pgfqpoint{5.352900in}{5.500818in}}%
\pgfpathlineto{\pgfqpoint{5.356668in}{5.501264in}}%
\pgfpathlineto{\pgfqpoint{5.391710in}{5.505355in}}%
\pgfpathlineto{\pgfqpoint{5.418074in}{5.508384in}}%
\pgfpathlineto{\pgfqpoint{5.426753in}{5.509359in}}%
\pgfpathlineto{\pgfqpoint{5.461795in}{5.513211in}}%
\pgfpathlineto{\pgfqpoint{5.496838in}{5.517005in}}%
\pgfpathlineto{\pgfqpoint{5.531880in}{5.520741in}}%
\pgfpathlineto{\pgfqpoint{5.566923in}{5.524418in}}%
\pgfpathlineto{\pgfqpoint{5.601965in}{5.528032in}}%
\pgfpathlineto{\pgfqpoint{5.637008in}{5.531584in}}%
\pgfpathlineto{\pgfqpoint{5.667791in}{5.534644in}}%
\pgfpathlineto{\pgfqpoint{5.672050in}{5.535057in}}%
\pgfpathlineto{\pgfqpoint{5.707093in}{5.538363in}}%
\pgfpathlineto{\pgfqpoint{5.742136in}{5.541602in}}%
\pgfpathlineto{\pgfqpoint{5.777178in}{5.544771in}}%
\pgfpathlineto{\pgfqpoint{5.812221in}{5.547869in}}%
\pgfpathlineto{\pgfqpoint{5.847263in}{5.550894in}}%
\pgfpathlineto{\pgfqpoint{5.882306in}{5.553844in}}%
\pgfpathlineto{\pgfqpoint{5.917348in}{5.556717in}}%
\pgfpathlineto{\pgfqpoint{5.952391in}{5.559510in}}%
\pgfpathlineto{\pgfqpoint{5.970488in}{5.560903in}}%
\pgfpathlineto{\pgfqpoint{5.987433in}{5.562175in}}%
\pgfpathlineto{\pgfqpoint{6.022476in}{5.564709in}}%
\pgfpathlineto{\pgfqpoint{6.057518in}{5.567158in}}%
\pgfpathlineto{\pgfqpoint{6.092561in}{5.569518in}}%
\pgfpathlineto{\pgfqpoint{6.127603in}{5.571788in}}%
\pgfpathlineto{\pgfqpoint{6.162646in}{5.573966in}}%
\pgfpathlineto{\pgfqpoint{6.197689in}{5.576046in}}%
\pgfpathlineto{\pgfqpoint{6.232731in}{5.578028in}}%
\pgfpathlineto{\pgfqpoint{6.267774in}{5.579907in}}%
\pgfpathlineto{\pgfqpoint{6.302816in}{5.581681in}}%
\pgfpathlineto{\pgfqpoint{6.337859in}{5.583345in}}%
\pgfpathlineto{\pgfqpoint{6.372901in}{5.584897in}}%
\pgfpathlineto{\pgfqpoint{6.407944in}{5.586331in}}%
\pgfpathlineto{\pgfqpoint{6.430218in}{5.587163in}}%
\pgfpathlineto{\pgfqpoint{6.442986in}{5.587626in}}%
\pgfpathlineto{\pgfqpoint{6.478029in}{5.588768in}}%
\pgfpathlineto{\pgfqpoint{6.513071in}{5.589785in}}%
\pgfpathlineto{\pgfqpoint{6.548114in}{5.590672in}}%
\pgfpathlineto{\pgfqpoint{6.583156in}{5.591424in}}%
\pgfpathlineto{\pgfqpoint{6.618199in}{5.592036in}}%
\pgfpathlineto{\pgfqpoint{6.653242in}{5.592504in}}%
\pgfpathlineto{\pgfqpoint{6.688284in}{5.592823in}}%
\pgfpathlineto{\pgfqpoint{6.723327in}{5.592985in}}%
\pgfpathlineto{\pgfqpoint{6.758369in}{5.592986in}}%
\pgfpathlineto{\pgfqpoint{6.793412in}{5.592818in}}%
\pgfpathlineto{\pgfqpoint{6.828454in}{5.592476in}}%
\pgfpathlineto{\pgfqpoint{6.863497in}{5.591951in}}%
\pgfpathlineto{\pgfqpoint{6.898539in}{5.591238in}}%
\pgfpathlineto{\pgfqpoint{6.933582in}{5.590327in}}%
\pgfpathlineto{\pgfqpoint{6.968624in}{5.589210in}}%
\pgfpathlineto{\pgfqpoint{7.003667in}{5.587878in}}%
\pgfpathlineto{\pgfqpoint{7.019917in}{5.587163in}}%
\pgfpathlineto{\pgfqpoint{7.038709in}{5.586282in}}%
\pgfpathlineto{\pgfqpoint{7.073752in}{5.584404in}}%
\pgfpathlineto{\pgfqpoint{7.108795in}{5.582266in}}%
\pgfpathlineto{\pgfqpoint{7.143837in}{5.579857in}}%
\pgfpathlineto{\pgfqpoint{7.178880in}{5.577163in}}%
\pgfpathlineto{\pgfqpoint{7.213922in}{5.574169in}}%
\pgfpathlineto{\pgfqpoint{7.248965in}{5.570861in}}%
\pgfpathlineto{\pgfqpoint{7.284007in}{5.567222in}}%
\pgfpathlineto{\pgfqpoint{7.319050in}{5.563235in}}%
\pgfpathlineto{\pgfqpoint{7.337971in}{5.560903in}}%
\pgfpathlineto{\pgfqpoint{7.354092in}{5.558766in}}%
\pgfpathlineto{\pgfqpoint{7.389135in}{5.553749in}}%
\pgfpathlineto{\pgfqpoint{7.424177in}{5.548291in}}%
\pgfpathlineto{\pgfqpoint{7.459220in}{5.542364in}}%
\pgfpathlineto{\pgfqpoint{7.494263in}{5.535939in}}%
\pgfpathlineto{\pgfqpoint{7.500904in}{5.534644in}}%
\pgfpathlineto{\pgfqpoint{7.529305in}{5.528625in}}%
\pgfpathlineto{\pgfqpoint{7.564348in}{5.520612in}}%
\pgfpathlineto{\pgfqpoint{7.599390in}{5.511942in}}%
\pgfpathlineto{\pgfqpoint{7.612880in}{5.508384in}}%
\pgfpathlineto{\pgfqpoint{7.634433in}{5.502152in}}%
\pgfpathlineto{\pgfqpoint{7.669475in}{5.491275in}}%
\pgfpathlineto{\pgfqpoint{7.696842in}{5.482125in}}%
\pgfpathlineto{\pgfqpoint{7.704518in}{5.479284in}}%
\pgfpathlineto{\pgfqpoint{7.739560in}{5.465490in}}%
\pgfusepath{stroke}%
\end{pgfscope}%
\begin{pgfscope}%
\pgfpathrectangle{\pgfqpoint{0.766095in}{0.571603in}}{\pgfqpoint{6.973465in}{5.225635in}}%
\pgfusepath{clip}%
\pgfsetbuttcap%
\pgfsetroundjoin%
\pgfsetlinewidth{1.505625pt}%
\definecolor{currentstroke}{rgb}{0.208623,0.367752,0.552675}%
\pgfsetstrokecolor{currentstroke}%
\pgfsetdash{}{0pt}%
\pgfpathmoveto{\pgfqpoint{7.739560in}{4.555901in}}%
\pgfpathlineto{\pgfqpoint{7.728323in}{4.537821in}}%
\pgfusepath{stroke}%
\end{pgfscope}%
\begin{pgfscope}%
\pgfpathrectangle{\pgfqpoint{0.766095in}{0.571603in}}{\pgfqpoint{6.973465in}{5.225635in}}%
\pgfusepath{clip}%
\pgfsetbuttcap%
\pgfsetroundjoin%
\pgfsetlinewidth{1.505625pt}%
\definecolor{currentstroke}{rgb}{0.208623,0.367752,0.552675}%
\pgfsetstrokecolor{currentstroke}%
\pgfsetdash{}{0pt}%
\pgfpathmoveto{\pgfqpoint{7.562732in}{4.272825in}}%
\pgfpathlineto{\pgfqpoint{7.494263in}{4.159059in}}%
\pgfpathlineto{\pgfqpoint{7.439670in}{4.064113in}}%
\pgfpathlineto{\pgfqpoint{7.389135in}{3.971096in}}%
\pgfpathlineto{\pgfqpoint{7.354092in}{3.902827in}}%
\pgfpathlineto{\pgfqpoint{7.317815in}{3.827778in}}%
\pgfpathlineto{\pgfqpoint{7.282498in}{3.749000in}}%
\pgfpathlineto{\pgfqpoint{7.250203in}{3.670221in}}%
\pgfpathlineto{\pgfqpoint{7.221093in}{3.591443in}}%
\pgfpathlineto{\pgfqpoint{7.195361in}{3.512664in}}%
\pgfpathlineto{\pgfqpoint{7.178880in}{3.455209in}}%
\pgfpathlineto{\pgfqpoint{7.166578in}{3.407626in}}%
\pgfpathlineto{\pgfqpoint{7.154637in}{3.355107in}}%
\pgfpathlineto{\pgfqpoint{7.143837in}{3.299265in}}%
\pgfpathlineto{\pgfqpoint{7.135840in}{3.250070in}}%
\pgfpathlineto{\pgfqpoint{7.129037in}{3.197551in}}%
\pgfpathlineto{\pgfqpoint{7.124024in}{3.145032in}}%
\pgfpathlineto{\pgfqpoint{7.120821in}{3.092513in}}%
\pgfpathlineto{\pgfqpoint{7.119452in}{3.039994in}}%
\pgfpathlineto{\pgfqpoint{7.119940in}{2.987475in}}%
\pgfpathlineto{\pgfqpoint{7.122308in}{2.934956in}}%
\pgfpathlineto{\pgfqpoint{7.126582in}{2.882437in}}%
\pgfpathlineto{\pgfqpoint{7.132787in}{2.829918in}}%
\pgfpathlineto{\pgfqpoint{7.143837in}{2.761690in}}%
\pgfpathlineto{\pgfqpoint{7.151045in}{2.724880in}}%
\pgfpathlineto{\pgfqpoint{7.163119in}{2.672361in}}%
\pgfpathlineto{\pgfqpoint{7.178880in}{2.614287in}}%
\pgfpathlineto{\pgfqpoint{7.193286in}{2.567323in}}%
\pgfpathlineto{\pgfqpoint{7.213922in}{2.508130in}}%
\pgfpathlineto{\pgfqpoint{7.231537in}{2.462285in}}%
\pgfpathlineto{\pgfqpoint{7.253742in}{2.409766in}}%
\pgfpathlineto{\pgfqpoint{7.284007in}{2.345015in}}%
\pgfpathlineto{\pgfqpoint{7.304278in}{2.304729in}}%
\pgfpathlineto{\pgfqpoint{7.332661in}{2.252210in}}%
\pgfpathlineto{\pgfqpoint{7.363131in}{2.199691in}}%
\pgfpathlineto{\pgfqpoint{7.395689in}{2.147172in}}%
\pgfpathlineto{\pgfqpoint{7.430337in}{2.094653in}}%
\pgfpathlineto{\pgfqpoint{7.467081in}{2.042134in}}%
\pgfpathlineto{\pgfqpoint{7.505928in}{1.989615in}}%
\pgfpathlineto{\pgfqpoint{7.564348in}{1.915575in}}%
\pgfpathlineto{\pgfqpoint{7.599390in}{1.873515in}}%
\pgfpathlineto{\pgfqpoint{7.635182in}{1.832058in}}%
\pgfpathlineto{\pgfqpoint{7.682465in}{1.779539in}}%
\pgfpathlineto{\pgfqpoint{7.739560in}{1.719146in}}%
\pgfpathlineto{\pgfqpoint{7.739560in}{1.719146in}}%
\pgfusepath{stroke}%
\end{pgfscope}%
\begin{pgfscope}%
\pgfpathrectangle{\pgfqpoint{0.766095in}{0.571603in}}{\pgfqpoint{6.973465in}{5.225635in}}%
\pgfusepath{clip}%
\pgfsetbuttcap%
\pgfsetroundjoin%
\pgfsetlinewidth{1.505625pt}%
\definecolor{currentstroke}{rgb}{0.199430,0.387607,0.554642}%
\pgfsetstrokecolor{currentstroke}%
\pgfsetdash{}{0pt}%
\pgfpathmoveto{\pgfqpoint{0.766095in}{4.236610in}}%
\pgfpathlineto{\pgfqpoint{0.783898in}{4.247930in}}%
\pgfpathlineto{\pgfqpoint{0.801138in}{4.258705in}}%
\pgfpathlineto{\pgfqpoint{0.826074in}{4.274189in}}%
\pgfpathlineto{\pgfqpoint{0.836180in}{4.280358in}}%
\pgfpathlineto{\pgfqpoint{0.869305in}{4.300449in}}%
\pgfpathlineto{\pgfqpoint{0.871223in}{4.301592in}}%
\pgfpathlineto{\pgfqpoint{0.906265in}{4.322329in}}%
\pgfpathlineto{\pgfqpoint{0.913713in}{4.326708in}}%
\pgfpathlineto{\pgfqpoint{0.941308in}{4.342658in}}%
\pgfpathlineto{\pgfqpoint{0.959251in}{4.352967in}}%
\pgfpathlineto{\pgfqpoint{0.976350in}{4.362627in}}%
\pgfpathlineto{\pgfqpoint{1.005904in}{4.379227in}}%
\pgfpathlineto{\pgfqpoint{1.011393in}{4.382258in}}%
\pgfpathlineto{\pgfqpoint{1.046435in}{4.401477in}}%
\pgfpathlineto{\pgfqpoint{1.053793in}{4.405486in}}%
\pgfpathlineto{\pgfqpoint{1.081478in}{4.420319in}}%
\pgfpathlineto{\pgfqpoint{1.102919in}{4.431746in}}%
\pgfpathlineto{\pgfqpoint{1.116520in}{4.438874in}}%
\pgfpathlineto{\pgfqpoint{1.151563in}{4.457139in}}%
\pgfpathlineto{\pgfqpoint{1.153237in}{4.458005in}}%
\pgfpathlineto{\pgfqpoint{1.186605in}{4.474978in}}%
\pgfpathlineto{\pgfqpoint{1.204951in}{4.484265in}}%
\pgfpathlineto{\pgfqpoint{1.221648in}{4.492576in}}%
\pgfpathlineto{\pgfqpoint{1.256691in}{4.509937in}}%
\pgfpathlineto{\pgfqpoint{1.257885in}{4.510524in}}%
\pgfpathlineto{\pgfqpoint{1.291733in}{4.526887in}}%
\pgfpathlineto{\pgfqpoint{1.312293in}{4.536784in}}%
\pgfpathlineto{\pgfqpoint{1.326776in}{4.543639in}}%
\pgfpathlineto{\pgfqpoint{1.361818in}{4.560141in}}%
\pgfpathlineto{\pgfqpoint{1.368024in}{4.563043in}}%
\pgfpathlineto{\pgfqpoint{1.396861in}{4.576306in}}%
\pgfpathlineto{\pgfqpoint{1.425222in}{4.589303in}}%
\pgfpathlineto{\pgfqpoint{1.431903in}{4.592313in}}%
\pgfpathlineto{\pgfqpoint{1.466946in}{4.607998in}}%
\pgfpathlineto{\pgfqpoint{1.483924in}{4.615562in}}%
\pgfpathlineto{\pgfqpoint{1.501988in}{4.623475in}}%
\pgfpathlineto{\pgfqpoint{1.537031in}{4.638758in}}%
\pgfpathlineto{\pgfqpoint{1.544102in}{4.641822in}}%
\pgfpathlineto{\pgfqpoint{1.572073in}{4.653737in}}%
\pgfpathlineto{\pgfqpoint{1.605838in}{4.668081in}}%
\pgfpathlineto{\pgfqpoint{1.607116in}{4.668615in}}%
\pgfpathlineto{\pgfqpoint{1.642158in}{4.683140in}}%
\pgfpathlineto{\pgfqpoint{1.669254in}{4.694341in}}%
\pgfpathlineto{\pgfqpoint{1.677201in}{4.697570in}}%
\pgfpathlineto{\pgfqpoint{1.712244in}{4.711721in}}%
\pgfpathlineto{\pgfqpoint{1.734311in}{4.720600in}}%
\pgfpathlineto{\pgfqpoint{1.747286in}{4.725733in}}%
\pgfpathlineto{\pgfqpoint{1.782329in}{4.739519in}}%
\pgfpathlineto{\pgfqpoint{1.801067in}{4.746860in}}%
\pgfpathlineto{\pgfqpoint{1.817371in}{4.753139in}}%
\pgfpathlineto{\pgfqpoint{1.852414in}{4.766569in}}%
\pgfpathlineto{\pgfqpoint{1.869582in}{4.773119in}}%
\pgfpathlineto{\pgfqpoint{1.887456in}{4.779823in}}%
\pgfpathlineto{\pgfqpoint{1.922499in}{4.792906in}}%
\pgfpathlineto{\pgfqpoint{1.939914in}{4.799378in}}%
\pgfpathlineto{\pgfqpoint{1.957541in}{4.805818in}}%
\pgfpathlineto{\pgfqpoint{1.992584in}{4.818562in}}%
\pgfpathlineto{\pgfqpoint{2.012122in}{4.825638in}}%
\pgfpathlineto{\pgfqpoint{2.027626in}{4.831157in}}%
\pgfpathlineto{\pgfqpoint{2.062669in}{4.843570in}}%
\pgfpathlineto{\pgfqpoint{2.086259in}{4.851897in}}%
\pgfpathlineto{\pgfqpoint{2.097711in}{4.855871in}}%
\pgfpathlineto{\pgfqpoint{2.132754in}{4.867960in}}%
\pgfpathlineto{\pgfqpoint{2.162380in}{4.878157in}}%
\pgfpathlineto{\pgfqpoint{2.167797in}{4.879989in}}%
\pgfpathlineto{\pgfqpoint{2.202839in}{4.891761in}}%
\pgfpathlineto{\pgfqpoint{2.237882in}{4.903519in}}%
\pgfpathlineto{\pgfqpoint{2.240576in}{4.904416in}}%
\pgfpathlineto{\pgfqpoint{2.272924in}{4.915003in}}%
\pgfpathlineto{\pgfqpoint{2.307967in}{4.926451in}}%
\pgfpathlineto{\pgfqpoint{2.320969in}{4.930676in}}%
\pgfpathlineto{\pgfqpoint{2.343009in}{4.937713in}}%
\pgfpathlineto{\pgfqpoint{2.378052in}{4.948858in}}%
\pgfpathlineto{\pgfqpoint{2.403526in}{4.956935in}}%
\pgfpathlineto{\pgfqpoint{2.413094in}{4.959916in}}%
\pgfpathlineto{\pgfqpoint{2.448137in}{4.970766in}}%
\pgfpathlineto{\pgfqpoint{2.483179in}{4.981601in}}%
\pgfpathlineto{\pgfqpoint{2.488370in}{4.983195in}}%
\pgfpathlineto{\pgfqpoint{2.518222in}{4.992200in}}%
\pgfpathlineto{\pgfqpoint{2.553264in}{5.002745in}}%
\pgfpathlineto{\pgfqpoint{2.575643in}{5.009454in}}%
\pgfpathlineto{\pgfqpoint{2.588307in}{5.013184in}}%
\pgfpathlineto{\pgfqpoint{2.623350in}{5.023446in}}%
\pgfpathlineto{\pgfqpoint{2.658392in}{5.033694in}}%
\pgfpathlineto{\pgfqpoint{2.665346in}{5.035714in}}%
\pgfpathlineto{\pgfqpoint{2.693435in}{5.043728in}}%
\pgfpathlineto{\pgfqpoint{2.728477in}{5.053697in}}%
\pgfpathlineto{\pgfqpoint{2.757639in}{5.061973in}}%
\pgfpathlineto{\pgfqpoint{2.763520in}{5.063612in}}%
\pgfpathlineto{\pgfqpoint{2.798562in}{5.073310in}}%
\pgfpathlineto{\pgfqpoint{2.833605in}{5.082994in}}%
\pgfpathlineto{\pgfqpoint{2.852652in}{5.088233in}}%
\pgfpathlineto{\pgfqpoint{2.868647in}{5.092554in}}%
\pgfpathlineto{\pgfqpoint{2.903690in}{5.101970in}}%
\pgfpathlineto{\pgfqpoint{2.938732in}{5.111373in}}%
\pgfpathlineto{\pgfqpoint{2.950431in}{5.114492in}}%
\pgfpathlineto{\pgfqpoint{2.973775in}{5.120605in}}%
\pgfpathlineto{\pgfqpoint{3.008818in}{5.129745in}}%
\pgfpathlineto{\pgfqpoint{3.043860in}{5.138871in}}%
\pgfpathlineto{\pgfqpoint{3.051133in}{5.140752in}}%
\pgfpathlineto{\pgfqpoint{3.078903in}{5.147802in}}%
\pgfpathlineto{\pgfqpoint{3.113945in}{5.156670in}}%
\pgfpathlineto{\pgfqpoint{3.148988in}{5.165524in}}%
\pgfpathlineto{\pgfqpoint{3.154917in}{5.167011in}}%
\pgfpathlineto{\pgfqpoint{3.184030in}{5.174178in}}%
\pgfpathlineto{\pgfqpoint{3.219073in}{5.182780in}}%
\pgfpathlineto{\pgfqpoint{3.254115in}{5.191366in}}%
\pgfpathlineto{\pgfqpoint{3.261947in}{5.193271in}}%
\pgfpathlineto{\pgfqpoint{3.289158in}{5.199768in}}%
\pgfpathlineto{\pgfqpoint{3.324200in}{5.208106in}}%
\pgfpathlineto{\pgfqpoint{3.359243in}{5.216428in}}%
\pgfpathlineto{\pgfqpoint{3.372389in}{5.219530in}}%
\pgfpathlineto{\pgfqpoint{3.394285in}{5.224602in}}%
\pgfpathlineto{\pgfqpoint{3.429328in}{5.232680in}}%
\pgfpathlineto{\pgfqpoint{3.464371in}{5.240742in}}%
\pgfpathlineto{\pgfqpoint{3.486414in}{5.245790in}}%
\pgfpathlineto{\pgfqpoint{3.499413in}{5.248711in}}%
\pgfpathlineto{\pgfqpoint{3.534456in}{5.256533in}}%
\pgfpathlineto{\pgfqpoint{3.569498in}{5.264338in}}%
\pgfpathlineto{\pgfqpoint{3.604193in}{5.272049in}}%
\pgfpathlineto{\pgfqpoint{3.604541in}{5.272125in}}%
\pgfpathlineto{\pgfqpoint{3.627515in}{5.277088in}}%
\pgfusepath{stroke}%
\end{pgfscope}%
\begin{pgfscope}%
\pgfpathrectangle{\pgfqpoint{0.766095in}{0.571603in}}{\pgfqpoint{6.973465in}{5.225635in}}%
\pgfusepath{clip}%
\pgfsetbuttcap%
\pgfsetroundjoin%
\pgfsetlinewidth{1.505625pt}%
\definecolor{currentstroke}{rgb}{0.199430,0.387607,0.554642}%
\pgfsetstrokecolor{currentstroke}%
\pgfsetdash{}{0pt}%
\pgfpathmoveto{\pgfqpoint{3.932257in}{5.340609in}}%
\pgfpathlineto{\pgfqpoint{3.954966in}{5.345154in}}%
\pgfpathlineto{\pgfqpoint{3.983430in}{5.350827in}}%
\pgfpathlineto{\pgfqpoint{3.990009in}{5.352113in}}%
\pgfpathlineto{\pgfqpoint{4.025051in}{5.358900in}}%
\pgfpathlineto{\pgfqpoint{4.060094in}{5.365667in}}%
\pgfpathlineto{\pgfqpoint{4.095136in}{5.372413in}}%
\pgfpathlineto{\pgfqpoint{4.119550in}{5.377087in}}%
\pgfpathlineto{\pgfqpoint{4.130179in}{5.379082in}}%
\pgfpathlineto{\pgfqpoint{4.165221in}{5.385604in}}%
\pgfpathlineto{\pgfqpoint{4.200264in}{5.392103in}}%
\pgfpathlineto{\pgfqpoint{4.235306in}{5.398580in}}%
\pgfpathlineto{\pgfqpoint{4.261237in}{5.403346in}}%
\pgfpathlineto{\pgfqpoint{4.270349in}{5.404988in}}%
\pgfpathlineto{\pgfqpoint{4.305391in}{5.411243in}}%
\pgfpathlineto{\pgfqpoint{4.340434in}{5.417474in}}%
\pgfpathlineto{\pgfqpoint{4.375477in}{5.423682in}}%
\pgfpathlineto{\pgfqpoint{4.409061in}{5.429606in}}%
\pgfpathlineto{\pgfqpoint{4.410519in}{5.429858in}}%
\pgfpathlineto{\pgfqpoint{4.445562in}{5.435845in}}%
\pgfpathlineto{\pgfqpoint{4.480604in}{5.441807in}}%
\pgfpathlineto{\pgfqpoint{4.515647in}{5.447743in}}%
\pgfpathlineto{\pgfqpoint{4.550689in}{5.453653in}}%
\pgfpathlineto{\pgfqpoint{4.563930in}{5.455865in}}%
\pgfpathlineto{\pgfqpoint{4.585732in}{5.459433in}}%
\pgfpathlineto{\pgfqpoint{4.620774in}{5.465124in}}%
\pgfpathlineto{\pgfqpoint{4.655817in}{5.470787in}}%
\pgfpathlineto{\pgfqpoint{4.690859in}{5.476422in}}%
\pgfpathlineto{\pgfqpoint{4.725902in}{5.482028in}}%
\pgfpathlineto{\pgfqpoint{4.726514in}{5.482125in}}%
\pgfpathlineto{\pgfqpoint{4.760944in}{5.487448in}}%
\pgfpathlineto{\pgfqpoint{4.795987in}{5.492835in}}%
\pgfpathlineto{\pgfqpoint{4.831030in}{5.498191in}}%
\pgfpathlineto{\pgfqpoint{4.866072in}{5.503517in}}%
\pgfpathlineto{\pgfqpoint{4.898309in}{5.508384in}}%
\pgfpathlineto{\pgfqpoint{4.901115in}{5.508799in}}%
\pgfpathlineto{\pgfqpoint{4.936157in}{5.513907in}}%
\pgfpathlineto{\pgfqpoint{4.971200in}{5.518982in}}%
\pgfpathlineto{\pgfqpoint{5.006242in}{5.524023in}}%
\pgfpathlineto{\pgfqpoint{5.041285in}{5.529031in}}%
\pgfpathlineto{\pgfqpoint{5.076327in}{5.534003in}}%
\pgfpathlineto{\pgfqpoint{5.080908in}{5.534644in}}%
\pgfpathlineto{\pgfqpoint{5.111370in}{5.538812in}}%
\pgfpathlineto{\pgfqpoint{5.146412in}{5.543565in}}%
\pgfpathlineto{\pgfqpoint{5.181455in}{5.548281in}}%
\pgfpathlineto{\pgfqpoint{5.216497in}{5.552958in}}%
\pgfpathlineto{\pgfqpoint{5.251540in}{5.557598in}}%
\pgfpathlineto{\pgfqpoint{5.276784in}{5.560903in}}%
\pgfpathlineto{\pgfqpoint{5.286583in}{5.562158in}}%
\pgfpathlineto{\pgfqpoint{5.321625in}{5.566578in}}%
\pgfpathlineto{\pgfqpoint{5.356668in}{5.570956in}}%
\pgfpathlineto{\pgfqpoint{5.391710in}{5.575292in}}%
\pgfpathlineto{\pgfqpoint{5.426753in}{5.579584in}}%
\pgfpathlineto{\pgfqpoint{5.461795in}{5.583833in}}%
\pgfpathlineto{\pgfqpoint{5.489600in}{5.587163in}}%
\pgfpathlineto{\pgfqpoint{5.496838in}{5.588010in}}%
\pgfpathlineto{\pgfqpoint{5.531880in}{5.592036in}}%
\pgfpathlineto{\pgfqpoint{5.566923in}{5.596016in}}%
\pgfpathlineto{\pgfqpoint{5.601965in}{5.599947in}}%
\pgfpathlineto{\pgfqpoint{5.637008in}{5.603829in}}%
\pgfpathlineto{\pgfqpoint{5.672050in}{5.607661in}}%
\pgfpathlineto{\pgfqpoint{5.707093in}{5.611441in}}%
\pgfpathlineto{\pgfqpoint{5.725806in}{5.613422in}}%
\pgfpathlineto{\pgfqpoint{5.742136in}{5.615111in}}%
\pgfpathlineto{\pgfqpoint{5.777178in}{5.618664in}}%
\pgfpathlineto{\pgfqpoint{5.812221in}{5.622161in}}%
\pgfpathlineto{\pgfqpoint{5.847263in}{5.625602in}}%
\pgfpathlineto{\pgfqpoint{5.882306in}{5.628984in}}%
\pgfpathlineto{\pgfqpoint{5.917348in}{5.632308in}}%
\pgfpathlineto{\pgfqpoint{5.952391in}{5.635570in}}%
\pgfpathlineto{\pgfqpoint{5.987433in}{5.638769in}}%
\pgfpathlineto{\pgfqpoint{5.997706in}{5.639682in}}%
\pgfpathlineto{\pgfqpoint{6.022476in}{5.641829in}}%
\pgfpathlineto{\pgfqpoint{6.057518in}{5.644792in}}%
\pgfpathlineto{\pgfqpoint{6.092561in}{5.647689in}}%
\pgfpathlineto{\pgfqpoint{6.127603in}{5.650516in}}%
\pgfpathlineto{\pgfqpoint{6.162646in}{5.653273in}}%
\pgfpathlineto{\pgfqpoint{6.197689in}{5.655957in}}%
\pgfpathlineto{\pgfqpoint{6.232731in}{5.658566in}}%
\pgfpathlineto{\pgfqpoint{6.267774in}{5.661097in}}%
\pgfpathlineto{\pgfqpoint{6.302816in}{5.663550in}}%
\pgfpathlineto{\pgfqpoint{6.337859in}{5.665920in}}%
\pgfpathlineto{\pgfqpoint{6.338181in}{5.665941in}}%
\pgfpathlineto{\pgfqpoint{6.372901in}{5.668125in}}%
\pgfpathlineto{\pgfqpoint{6.407944in}{5.670244in}}%
\pgfpathlineto{\pgfqpoint{6.442986in}{5.672275in}}%
\pgfpathlineto{\pgfqpoint{6.478029in}{5.674216in}}%
\pgfpathlineto{\pgfqpoint{6.513071in}{5.676064in}}%
\pgfpathlineto{\pgfqpoint{6.548114in}{5.677815in}}%
\pgfpathlineto{\pgfqpoint{6.583156in}{5.679468in}}%
\pgfpathlineto{\pgfqpoint{6.618199in}{5.681019in}}%
\pgfpathlineto{\pgfqpoint{6.653242in}{5.682463in}}%
\pgfpathlineto{\pgfqpoint{6.688284in}{5.683798in}}%
\pgfpathlineto{\pgfqpoint{6.723327in}{5.685021in}}%
\pgfpathlineto{\pgfqpoint{6.758369in}{5.686126in}}%
\pgfpathlineto{\pgfqpoint{6.793412in}{5.687110in}}%
\pgfpathlineto{\pgfqpoint{6.828454in}{5.687969in}}%
\pgfpathlineto{\pgfqpoint{6.863497in}{5.688698in}}%
\pgfpathlineto{\pgfqpoint{6.898539in}{5.689292in}}%
\pgfpathlineto{\pgfqpoint{6.933582in}{5.689746in}}%
\pgfpathlineto{\pgfqpoint{6.968624in}{5.690056in}}%
\pgfpathlineto{\pgfqpoint{7.003667in}{5.690215in}}%
\pgfpathlineto{\pgfqpoint{7.038709in}{5.690217in}}%
\pgfpathlineto{\pgfqpoint{7.073752in}{5.690057in}}%
\pgfpathlineto{\pgfqpoint{7.108795in}{5.689727in}}%
\pgfpathlineto{\pgfqpoint{7.143837in}{5.689222in}}%
\pgfpathlineto{\pgfqpoint{7.178880in}{5.688533in}}%
\pgfpathlineto{\pgfqpoint{7.213922in}{5.687652in}}%
\pgfpathlineto{\pgfqpoint{7.248965in}{5.686573in}}%
\pgfpathlineto{\pgfqpoint{7.284007in}{5.685285in}}%
\pgfpathlineto{\pgfqpoint{7.319050in}{5.683779in}}%
\pgfpathlineto{\pgfqpoint{7.354092in}{5.682047in}}%
\pgfpathlineto{\pgfqpoint{7.389135in}{5.680077in}}%
\pgfpathlineto{\pgfqpoint{7.424177in}{5.677859in}}%
\pgfpathlineto{\pgfqpoint{7.459220in}{5.675380in}}%
\pgfpathlineto{\pgfqpoint{7.494263in}{5.672628in}}%
\pgfpathlineto{\pgfqpoint{7.529305in}{5.669590in}}%
\pgfpathlineto{\pgfqpoint{7.564348in}{5.666251in}}%
\pgfpathlineto{\pgfqpoint{7.567364in}{5.665941in}}%
\pgfpathlineto{\pgfqpoint{7.599390in}{5.662410in}}%
\pgfpathlineto{\pgfqpoint{7.634433in}{5.658195in}}%
\pgfpathlineto{\pgfqpoint{7.669475in}{5.653603in}}%
\pgfpathlineto{\pgfqpoint{7.704518in}{5.648613in}}%
\pgfpathlineto{\pgfqpoint{7.739560in}{5.643201in}}%
\pgfusepath{stroke}%
\end{pgfscope}%
\begin{pgfscope}%
\pgfpathrectangle{\pgfqpoint{0.766095in}{0.571603in}}{\pgfqpoint{6.973465in}{5.225635in}}%
\pgfusepath{clip}%
\pgfsetbuttcap%
\pgfsetroundjoin%
\pgfsetlinewidth{1.505625pt}%
\definecolor{currentstroke}{rgb}{0.199430,0.387607,0.554642}%
\pgfsetstrokecolor{currentstroke}%
\pgfsetdash{}{0pt}%
\pgfpathmoveto{\pgfqpoint{7.739560in}{4.247425in}}%
\pgfpathlineto{\pgfqpoint{7.723394in}{4.222669in}}%
\pgfusepath{stroke}%
\end{pgfscope}%
\begin{pgfscope}%
\pgfpathrectangle{\pgfqpoint{0.766095in}{0.571603in}}{\pgfqpoint{6.973465in}{5.225635in}}%
\pgfusepath{clip}%
\pgfsetbuttcap%
\pgfsetroundjoin%
\pgfsetlinewidth{1.505625pt}%
\definecolor{currentstroke}{rgb}{0.199430,0.387607,0.554642}%
\pgfsetstrokecolor{currentstroke}%
\pgfsetdash{}{0pt}%
\pgfpathmoveto{\pgfqpoint{7.561876in}{3.955217in}}%
\pgfpathlineto{\pgfqpoint{7.549552in}{3.932816in}}%
\pgfpathlineto{\pgfqpoint{7.535455in}{3.906556in}}%
\pgfpathlineto{\pgfqpoint{7.529305in}{3.894852in}}%
\pgfpathlineto{\pgfqpoint{7.521652in}{3.880297in}}%
\pgfpathlineto{\pgfqpoint{7.508162in}{3.854037in}}%
\pgfpathlineto{\pgfqpoint{7.495039in}{3.827778in}}%
\pgfpathlineto{\pgfqpoint{7.494263in}{3.826190in}}%
\pgfpathlineto{\pgfqpoint{7.482190in}{3.801519in}}%
\pgfpathlineto{\pgfqpoint{7.469714in}{3.775259in}}%
\pgfpathlineto{\pgfqpoint{7.459220in}{3.752495in}}%
\pgfpathlineto{\pgfqpoint{7.457606in}{3.749000in}}%
\pgfpathlineto{\pgfqpoint{7.445804in}{3.722740in}}%
\pgfpathlineto{\pgfqpoint{7.434393in}{3.696481in}}%
\pgfpathlineto{\pgfqpoint{7.424177in}{3.672139in}}%
\pgfpathlineto{\pgfqpoint{7.423371in}{3.670221in}}%
\pgfpathlineto{\pgfqpoint{7.412670in}{3.643962in}}%
\pgfpathlineto{\pgfqpoint{7.402373in}{3.617702in}}%
\pgfpathlineto{\pgfqpoint{7.392483in}{3.591443in}}%
\pgfpathlineto{\pgfqpoint{7.389135in}{3.582199in}}%
\pgfpathlineto{\pgfqpoint{7.382956in}{3.565183in}}%
\pgfpathlineto{\pgfqpoint{7.373818in}{3.538924in}}%
\pgfpathlineto{\pgfqpoint{7.365099in}{3.512664in}}%
\pgfpathlineto{\pgfqpoint{7.356801in}{3.486405in}}%
\pgfpathlineto{\pgfqpoint{7.354092in}{3.477399in}}%
\pgfpathlineto{\pgfqpoint{7.348887in}{3.460145in}}%
\pgfpathlineto{\pgfqpoint{7.341381in}{3.433886in}}%
\pgfpathlineto{\pgfqpoint{7.334307in}{3.407626in}}%
\pgfpathlineto{\pgfqpoint{7.327668in}{3.381367in}}%
\pgfpathlineto{\pgfqpoint{7.321465in}{3.355107in}}%
\pgfpathlineto{\pgfqpoint{7.319050in}{3.344130in}}%
\pgfpathlineto{\pgfqpoint{7.315675in}{3.328848in}}%
\pgfpathlineto{\pgfqpoint{7.310312in}{3.302589in}}%
\pgfpathlineto{\pgfqpoint{7.305396in}{3.276329in}}%
\pgfpathlineto{\pgfqpoint{7.300929in}{3.250070in}}%
\pgfpathlineto{\pgfqpoint{7.296914in}{3.223810in}}%
\pgfpathlineto{\pgfqpoint{7.293353in}{3.197551in}}%
\pgfpathlineto{\pgfqpoint{7.290249in}{3.171291in}}%
\pgfpathlineto{\pgfqpoint{7.287603in}{3.145032in}}%
\pgfpathlineto{\pgfqpoint{7.285419in}{3.118772in}}%
\pgfpathlineto{\pgfqpoint{7.284007in}{3.097233in}}%
\pgfpathlineto{\pgfqpoint{7.283696in}{3.092513in}}%
\pgfpathlineto{\pgfqpoint{7.282433in}{3.066253in}}%
\pgfpathlineto{\pgfqpoint{7.281642in}{3.039994in}}%
\pgfpathlineto{\pgfqpoint{7.281326in}{3.013734in}}%
\pgfpathlineto{\pgfqpoint{7.281489in}{2.987475in}}%
\pgfpathlineto{\pgfqpoint{7.282131in}{2.961215in}}%
\pgfpathlineto{\pgfqpoint{7.283257in}{2.934956in}}%
\pgfpathlineto{\pgfqpoint{7.284007in}{2.922742in}}%
\pgfpathlineto{\pgfqpoint{7.284863in}{2.908696in}}%
\pgfpathlineto{\pgfqpoint{7.286948in}{2.882437in}}%
\pgfpathlineto{\pgfqpoint{7.289522in}{2.856177in}}%
\pgfpathlineto{\pgfqpoint{7.292588in}{2.829918in}}%
\pgfpathlineto{\pgfqpoint{7.296149in}{2.803659in}}%
\pgfpathlineto{\pgfqpoint{7.300207in}{2.777399in}}%
\pgfpathlineto{\pgfqpoint{7.304767in}{2.751140in}}%
\pgfpathlineto{\pgfqpoint{7.309831in}{2.724880in}}%
\pgfpathlineto{\pgfqpoint{7.315404in}{2.698621in}}%
\pgfpathlineto{\pgfqpoint{7.319050in}{2.682885in}}%
\pgfpathlineto{\pgfqpoint{7.321470in}{2.672361in}}%
\pgfpathlineto{\pgfqpoint{7.328022in}{2.646102in}}%
\pgfpathlineto{\pgfqpoint{7.335090in}{2.619842in}}%
\pgfpathlineto{\pgfqpoint{7.342677in}{2.593583in}}%
\pgfpathlineto{\pgfqpoint{7.350787in}{2.567323in}}%
\pgfpathlineto{\pgfqpoint{7.354092in}{2.557272in}}%
\pgfpathlineto{\pgfqpoint{7.359387in}{2.541064in}}%
\pgfpathlineto{\pgfqpoint{7.368490in}{2.514804in}}%
\pgfpathlineto{\pgfqpoint{7.378126in}{2.488545in}}%
\pgfpathlineto{\pgfqpoint{7.388299in}{2.462285in}}%
\pgfpathlineto{\pgfqpoint{7.389135in}{2.460235in}}%
\pgfpathlineto{\pgfqpoint{7.398942in}{2.436026in}}%
\pgfpathlineto{\pgfqpoint{7.410122in}{2.409766in}}%
\pgfpathlineto{\pgfqpoint{7.421848in}{2.383507in}}%
\pgfpathlineto{\pgfqpoint{7.424177in}{2.378518in}}%
\pgfpathlineto{\pgfqpoint{7.434055in}{2.357248in}}%
\pgfpathlineto{\pgfqpoint{7.446798in}{2.330988in}}%
\pgfpathlineto{\pgfqpoint{7.459220in}{2.306462in}}%
\pgfpathlineto{\pgfqpoint{7.460093in}{2.304729in}}%
\pgfpathlineto{\pgfqpoint{7.473860in}{2.278469in}}%
\pgfpathlineto{\pgfqpoint{7.488191in}{2.252210in}}%
\pgfpathlineto{\pgfqpoint{7.494263in}{2.241496in}}%
\pgfpathlineto{\pgfqpoint{7.503030in}{2.225950in}}%
\pgfpathlineto{\pgfqpoint{7.518398in}{2.199691in}}%
\pgfpathlineto{\pgfqpoint{7.529305in}{2.181714in}}%
\pgfpathlineto{\pgfqpoint{7.534308in}{2.173431in}}%
\pgfpathlineto{\pgfqpoint{7.550721in}{2.147172in}}%
\pgfpathlineto{\pgfqpoint{7.564348in}{2.126110in}}%
\pgfpathlineto{\pgfqpoint{7.567696in}{2.120912in}}%
\pgfpathlineto{\pgfqpoint{7.585163in}{2.094653in}}%
\pgfpathlineto{\pgfqpoint{7.599390in}{2.073954in}}%
\pgfpathlineto{\pgfqpoint{7.603197in}{2.068393in}}%
\pgfpathlineto{\pgfqpoint{7.621728in}{2.042134in}}%
\pgfpathlineto{\pgfqpoint{7.634433in}{2.024679in}}%
\pgfpathlineto{\pgfqpoint{7.640817in}{2.015874in}}%
\pgfpathlineto{\pgfqpoint{7.660424in}{1.989615in}}%
\pgfpathlineto{\pgfqpoint{7.669475in}{1.977836in}}%
\pgfpathlineto{\pgfqpoint{7.680565in}{1.963355in}}%
\pgfpathlineto{\pgfqpoint{7.701258in}{1.937096in}}%
\pgfpathlineto{\pgfqpoint{7.704518in}{1.933067in}}%
\pgfpathlineto{\pgfqpoint{7.722449in}{1.910836in}}%
\pgfpathlineto{\pgfqpoint{7.739560in}{1.890201in}}%
\pgfusepath{stroke}%
\end{pgfscope}%
\begin{pgfscope}%
\pgfpathrectangle{\pgfqpoint{0.766095in}{0.571603in}}{\pgfqpoint{6.973465in}{5.225635in}}%
\pgfusepath{clip}%
\pgfsetbuttcap%
\pgfsetroundjoin%
\pgfsetlinewidth{1.505625pt}%
\definecolor{currentstroke}{rgb}{0.190631,0.407061,0.556089}%
\pgfsetstrokecolor{currentstroke}%
\pgfsetdash{}{0pt}%
\pgfpathmoveto{\pgfqpoint{7.597422in}{3.686987in}}%
\pgfpathlineto{\pgfqpoint{7.589995in}{3.670221in}}%
\pgfpathlineto{\pgfqpoint{7.578767in}{3.643962in}}%
\pgfpathlineto{\pgfqpoint{7.567959in}{3.617702in}}%
\pgfpathlineto{\pgfqpoint{7.564348in}{3.608603in}}%
\pgfpathlineto{\pgfqpoint{7.557521in}{3.591443in}}%
\pgfpathlineto{\pgfqpoint{7.547484in}{3.565183in}}%
\pgfpathlineto{\pgfqpoint{7.537876in}{3.538924in}}%
\pgfpathlineto{\pgfqpoint{7.529305in}{3.514405in}}%
\pgfpathlineto{\pgfqpoint{7.528695in}{3.512664in}}%
\pgfpathlineto{\pgfqpoint{7.519887in}{3.486405in}}%
\pgfpathlineto{\pgfqpoint{7.511517in}{3.460145in}}%
\pgfpathlineto{\pgfqpoint{7.503589in}{3.433886in}}%
\pgfpathlineto{\pgfqpoint{7.496104in}{3.407626in}}%
\pgfpathlineto{\pgfqpoint{7.494263in}{3.400778in}}%
\pgfpathlineto{\pgfqpoint{7.489025in}{3.381367in}}%
\pgfpathlineto{\pgfqpoint{7.482383in}{3.355107in}}%
\pgfpathlineto{\pgfqpoint{7.476193in}{3.328848in}}%
\pgfpathlineto{\pgfqpoint{7.470458in}{3.302589in}}%
\pgfpathlineto{\pgfqpoint{7.465178in}{3.276329in}}%
\pgfpathlineto{\pgfqpoint{7.460356in}{3.250070in}}%
\pgfpathlineto{\pgfqpoint{7.459220in}{3.243242in}}%
\pgfpathlineto{\pgfqpoint{7.455972in}{3.223810in}}%
\pgfpathlineto{\pgfqpoint{7.452044in}{3.197551in}}%
\pgfpathlineto{\pgfqpoint{7.448584in}{3.171291in}}%
\pgfpathlineto{\pgfqpoint{7.445594in}{3.145032in}}%
\pgfpathlineto{\pgfqpoint{7.443076in}{3.118772in}}%
\pgfpathlineto{\pgfqpoint{7.441032in}{3.092513in}}%
\pgfpathlineto{\pgfqpoint{7.439464in}{3.066253in}}%
\pgfpathlineto{\pgfqpoint{7.438376in}{3.039994in}}%
\pgfpathlineto{\pgfqpoint{7.437768in}{3.013734in}}%
\pgfpathlineto{\pgfqpoint{7.437643in}{2.987475in}}%
\pgfpathlineto{\pgfqpoint{7.438005in}{2.961215in}}%
\pgfpathlineto{\pgfqpoint{7.438855in}{2.934956in}}%
\pgfpathlineto{\pgfqpoint{7.440197in}{2.908696in}}%
\pgfpathlineto{\pgfqpoint{7.442032in}{2.882437in}}%
\pgfpathlineto{\pgfqpoint{7.444363in}{2.856177in}}%
\pgfpathlineto{\pgfqpoint{7.447195in}{2.829918in}}%
\pgfpathlineto{\pgfqpoint{7.450528in}{2.803659in}}%
\pgfpathlineto{\pgfqpoint{7.454367in}{2.777399in}}%
\pgfpathlineto{\pgfqpoint{7.458715in}{2.751140in}}%
\pgfpathlineto{\pgfqpoint{7.459220in}{2.748411in}}%
\pgfpathlineto{\pgfqpoint{7.463543in}{2.724880in}}%
\pgfpathlineto{\pgfqpoint{7.468880in}{2.698621in}}%
\pgfpathlineto{\pgfqpoint{7.474732in}{2.672361in}}%
\pgfpathlineto{\pgfqpoint{7.481101in}{2.646102in}}%
\pgfpathlineto{\pgfqpoint{7.487993in}{2.619842in}}%
\pgfpathlineto{\pgfqpoint{7.494263in}{2.597645in}}%
\pgfpathlineto{\pgfqpoint{7.495402in}{2.593583in}}%
\pgfpathlineto{\pgfqpoint{7.503294in}{2.567323in}}%
\pgfpathlineto{\pgfqpoint{7.511715in}{2.541064in}}%
\pgfpathlineto{\pgfqpoint{7.520669in}{2.514804in}}%
\pgfpathlineto{\pgfqpoint{7.529305in}{2.490914in}}%
\pgfpathlineto{\pgfqpoint{7.530156in}{2.488545in}}%
\pgfpathlineto{\pgfqpoint{7.540121in}{2.462285in}}%
\pgfpathlineto{\pgfqpoint{7.550629in}{2.436026in}}%
\pgfpathlineto{\pgfqpoint{7.561684in}{2.409766in}}%
\pgfpathlineto{\pgfqpoint{7.564348in}{2.403734in}}%
\pgfpathlineto{\pgfqpoint{7.573229in}{2.383507in}}%
\pgfpathlineto{\pgfqpoint{7.585308in}{2.357248in}}%
\pgfpathlineto{\pgfqpoint{7.597946in}{2.330988in}}%
\pgfpathlineto{\pgfqpoint{7.599390in}{2.328110in}}%
\pgfpathlineto{\pgfqpoint{7.611064in}{2.304729in}}%
\pgfpathlineto{\pgfqpoint{7.624737in}{2.278469in}}%
\pgfpathlineto{\pgfqpoint{7.634433in}{2.260580in}}%
\pgfpathlineto{\pgfqpoint{7.638948in}{2.252210in}}%
\pgfpathlineto{\pgfqpoint{7.653662in}{2.225950in}}%
\pgfpathlineto{\pgfqpoint{7.668953in}{2.199691in}}%
\pgfpathlineto{\pgfqpoint{7.669475in}{2.198825in}}%
\pgfpathlineto{\pgfqpoint{7.684721in}{2.173431in}}%
\pgfpathlineto{\pgfqpoint{7.701068in}{2.147172in}}%
\pgfpathlineto{\pgfqpoint{7.704518in}{2.141812in}}%
\pgfpathlineto{\pgfqpoint{7.717913in}{2.120912in}}%
\pgfpathlineto{\pgfqpoint{7.735326in}{2.094653in}}%
\pgfpathlineto{\pgfqpoint{7.739560in}{2.088465in}}%
\pgfusepath{stroke}%
\end{pgfscope}%
\begin{pgfscope}%
\pgfpathrectangle{\pgfqpoint{0.766095in}{0.571603in}}{\pgfqpoint{6.973465in}{5.225635in}}%
\pgfusepath{clip}%
\pgfsetbuttcap%
\pgfsetroundjoin%
\pgfsetlinewidth{1.505625pt}%
\definecolor{currentstroke}{rgb}{0.190631,0.407061,0.556089}%
\pgfsetstrokecolor{currentstroke}%
\pgfsetdash{}{0pt}%
\pgfpathmoveto{\pgfqpoint{0.766095in}{4.324652in}}%
\pgfpathlineto{\pgfqpoint{0.836180in}{4.366144in}}%
\pgfpathlineto{\pgfqpoint{0.906265in}{4.406177in}}%
\pgfpathlineto{\pgfqpoint{1.011393in}{4.463365in}}%
\pgfpathlineto{\pgfqpoint{1.116520in}{4.517623in}}%
\pgfpathlineto{\pgfqpoint{1.221648in}{4.569224in}}%
\pgfpathlineto{\pgfqpoint{1.326776in}{4.618418in}}%
\pgfpathlineto{\pgfqpoint{1.438081in}{4.668081in}}%
\pgfpathlineto{\pgfqpoint{1.572073in}{4.724837in}}%
\pgfpathlineto{\pgfqpoint{1.712244in}{4.781053in}}%
\pgfpathlineto{\pgfqpoint{1.852414in}{4.834368in}}%
\pgfpathlineto{\pgfqpoint{1.992584in}{4.885045in}}%
\pgfpathlineto{\pgfqpoint{2.132754in}{4.933326in}}%
\pgfpathlineto{\pgfqpoint{2.307967in}{4.990502in}}%
\pgfpathlineto{\pgfqpoint{2.453614in}{5.035714in}}%
\pgfpathlineto{\pgfqpoint{2.631735in}{5.088233in}}%
\pgfpathlineto{\pgfqpoint{2.833605in}{5.144438in}}%
\pgfpathlineto{\pgfqpoint{3.018948in}{5.193271in}}%
\pgfpathlineto{\pgfqpoint{3.229828in}{5.245790in}}%
\pgfpathlineto{\pgfqpoint{3.453698in}{5.298308in}}%
\pgfpathlineto{\pgfqpoint{3.674626in}{5.347084in}}%
\pgfpathlineto{\pgfqpoint{3.849838in}{5.383764in}}%
\pgfpathlineto{\pgfqpoint{4.080595in}{5.429606in}}%
\pgfpathlineto{\pgfqpoint{4.117347in}{5.436583in}}%
\pgfpathlineto{\pgfqpoint{4.117347in}{5.436583in}}%
\pgfusepath{stroke}%
\end{pgfscope}%
\begin{pgfscope}%
\pgfpathrectangle{\pgfqpoint{0.766095in}{0.571603in}}{\pgfqpoint{6.973465in}{5.225635in}}%
\pgfusepath{clip}%
\pgfsetbuttcap%
\pgfsetroundjoin%
\pgfsetlinewidth{1.505625pt}%
\definecolor{currentstroke}{rgb}{0.190631,0.407061,0.556089}%
\pgfsetstrokecolor{currentstroke}%
\pgfsetdash{}{0pt}%
\pgfpathmoveto{\pgfqpoint{4.423555in}{5.492523in}}%
\pgfpathlineto{\pgfqpoint{4.445562in}{5.496364in}}%
\pgfpathlineto{\pgfqpoint{4.480604in}{5.502462in}}%
\pgfpathlineto{\pgfqpoint{4.514760in}{5.508384in}}%
\pgfpathlineto{\pgfqpoint{4.515647in}{5.508535in}}%
\pgfpathlineto{\pgfqpoint{4.550689in}{5.514424in}}%
\pgfpathlineto{\pgfqpoint{4.585732in}{5.520291in}}%
\pgfpathlineto{\pgfqpoint{4.620774in}{5.526136in}}%
\pgfpathlineto{\pgfqpoint{4.655817in}{5.531960in}}%
\pgfpathlineto{\pgfqpoint{4.672100in}{5.534644in}}%
\pgfpathlineto{\pgfqpoint{4.690859in}{5.537673in}}%
\pgfpathlineto{\pgfqpoint{4.725902in}{5.543289in}}%
\pgfpathlineto{\pgfqpoint{4.760944in}{5.548880in}}%
\pgfpathlineto{\pgfqpoint{4.795987in}{5.554448in}}%
\pgfpathlineto{\pgfqpoint{4.831030in}{5.559991in}}%
\pgfpathlineto{\pgfqpoint{4.836860in}{5.560903in}}%
\pgfpathlineto{\pgfqpoint{4.866072in}{5.565379in}}%
\pgfpathlineto{\pgfqpoint{4.901115in}{5.570715in}}%
\pgfpathlineto{\pgfqpoint{4.936157in}{5.576025in}}%
\pgfpathlineto{\pgfqpoint{4.971200in}{5.581308in}}%
\pgfpathlineto{\pgfqpoint{5.006242in}{5.586565in}}%
\pgfpathlineto{\pgfqpoint{5.010277in}{5.587163in}}%
\pgfpathlineto{\pgfqpoint{5.041285in}{5.591660in}}%
\pgfpathlineto{\pgfqpoint{5.076327in}{5.596710in}}%
\pgfpathlineto{\pgfqpoint{5.111370in}{5.601731in}}%
\pgfpathlineto{\pgfqpoint{5.146412in}{5.606723in}}%
\pgfpathlineto{\pgfqpoint{5.181455in}{5.611685in}}%
\pgfpathlineto{\pgfqpoint{5.193866in}{5.613422in}}%
\pgfpathlineto{\pgfqpoint{5.216497in}{5.616522in}}%
\pgfpathlineto{\pgfqpoint{5.251540in}{5.621278in}}%
\pgfpathlineto{\pgfqpoint{5.286583in}{5.626001in}}%
\pgfpathlineto{\pgfqpoint{5.321625in}{5.630692in}}%
\pgfpathlineto{\pgfqpoint{5.356668in}{5.635350in}}%
\pgfpathlineto{\pgfqpoint{5.389506in}{5.639682in}}%
\pgfpathlineto{\pgfqpoint{5.391710in}{5.639966in}}%
\pgfpathlineto{\pgfqpoint{5.426753in}{5.644417in}}%
\pgfpathlineto{\pgfqpoint{5.461795in}{5.648833in}}%
\pgfpathlineto{\pgfqpoint{5.496838in}{5.653213in}}%
\pgfpathlineto{\pgfqpoint{5.531880in}{5.657557in}}%
\pgfpathlineto{\pgfqpoint{5.566923in}{5.661862in}}%
\pgfpathlineto{\pgfqpoint{5.600426in}{5.665941in}}%
\pgfpathlineto{\pgfqpoint{5.601965in}{5.666124in}}%
\pgfpathlineto{\pgfqpoint{5.637008in}{5.670222in}}%
\pgfpathlineto{\pgfqpoint{5.672050in}{5.674279in}}%
\pgfpathlineto{\pgfqpoint{5.707093in}{5.678295in}}%
\pgfpathlineto{\pgfqpoint{5.742136in}{5.682269in}}%
\pgfpathlineto{\pgfqpoint{5.777178in}{5.686200in}}%
\pgfpathlineto{\pgfqpoint{5.812221in}{5.690088in}}%
\pgfpathlineto{\pgfqpoint{5.831562in}{5.692201in}}%
\pgfpathlineto{\pgfqpoint{5.847263in}{5.693876in}}%
\pgfpathlineto{\pgfqpoint{5.882306in}{5.697551in}}%
\pgfpathlineto{\pgfqpoint{5.917348in}{5.701179in}}%
\pgfpathlineto{\pgfqpoint{5.952391in}{5.704759in}}%
\pgfpathlineto{\pgfqpoint{5.987433in}{5.708290in}}%
\pgfpathlineto{\pgfqpoint{6.022476in}{5.711771in}}%
\pgfpathlineto{\pgfqpoint{6.057518in}{5.715201in}}%
\pgfpathlineto{\pgfqpoint{6.091353in}{5.718460in}}%
\pgfpathlineto{\pgfqpoint{6.092561in}{5.718574in}}%
\pgfpathlineto{\pgfqpoint{6.127603in}{5.721786in}}%
\pgfpathlineto{\pgfqpoint{6.162646in}{5.724943in}}%
\pgfpathlineto{\pgfqpoint{6.197689in}{5.728043in}}%
\pgfpathlineto{\pgfqpoint{6.232731in}{5.731086in}}%
\pgfpathlineto{\pgfqpoint{6.267774in}{5.734069in}}%
\pgfpathlineto{\pgfqpoint{6.302816in}{5.736991in}}%
\pgfpathlineto{\pgfqpoint{6.337859in}{5.739850in}}%
\pgfpathlineto{\pgfqpoint{6.372901in}{5.742645in}}%
\pgfpathlineto{\pgfqpoint{6.399608in}{5.744720in}}%
\pgfpathlineto{\pgfqpoint{6.407944in}{5.745351in}}%
\pgfpathlineto{\pgfqpoint{6.442986in}{5.747918in}}%
\pgfpathlineto{\pgfqpoint{6.478029in}{5.750417in}}%
\pgfpathlineto{\pgfqpoint{6.513071in}{5.752844in}}%
\pgfpathlineto{\pgfqpoint{6.548114in}{5.755199in}}%
\pgfpathlineto{\pgfqpoint{6.583156in}{5.757479in}}%
\pgfpathlineto{\pgfqpoint{6.618199in}{5.759681in}}%
\pgfpathlineto{\pgfqpoint{6.653242in}{5.761804in}}%
\pgfpathlineto{\pgfqpoint{6.688284in}{5.763846in}}%
\pgfpathlineto{\pgfqpoint{6.723327in}{5.765803in}}%
\pgfpathlineto{\pgfqpoint{6.758369in}{5.767672in}}%
\pgfpathlineto{\pgfqpoint{6.793412in}{5.769453in}}%
\pgfpathlineto{\pgfqpoint{6.825141in}{5.770979in}}%
\pgfpathlineto{\pgfqpoint{6.828454in}{5.771134in}}%
\pgfpathlineto{\pgfqpoint{6.863497in}{5.772665in}}%
\pgfpathlineto{\pgfqpoint{6.898539in}{5.774101in}}%
\pgfpathlineto{\pgfqpoint{6.933582in}{5.775437in}}%
\pgfpathlineto{\pgfqpoint{6.968624in}{5.776671in}}%
\pgfpathlineto{\pgfqpoint{7.003667in}{5.777799in}}%
\pgfpathlineto{\pgfqpoint{7.038709in}{5.778817in}}%
\pgfpathlineto{\pgfqpoint{7.073752in}{5.779723in}}%
\pgfpathlineto{\pgfqpoint{7.108795in}{5.780511in}}%
\pgfpathlineto{\pgfqpoint{7.143837in}{5.781179in}}%
\pgfpathlineto{\pgfqpoint{7.178880in}{5.781721in}}%
\pgfpathlineto{\pgfqpoint{7.213922in}{5.782133in}}%
\pgfpathlineto{\pgfqpoint{7.248965in}{5.782410in}}%
\pgfpathlineto{\pgfqpoint{7.284007in}{5.782547in}}%
\pgfpathlineto{\pgfqpoint{7.319050in}{5.782539in}}%
\pgfpathlineto{\pgfqpoint{7.354092in}{5.782381in}}%
\pgfpathlineto{\pgfqpoint{7.389135in}{5.782066in}}%
\pgfpathlineto{\pgfqpoint{7.424177in}{5.781588in}}%
\pgfpathlineto{\pgfqpoint{7.459220in}{5.780941in}}%
\pgfpathlineto{\pgfqpoint{7.494263in}{5.780118in}}%
\pgfpathlineto{\pgfqpoint{7.529305in}{5.779111in}}%
\pgfpathlineto{\pgfqpoint{7.564348in}{5.777913in}}%
\pgfpathlineto{\pgfqpoint{7.599390in}{5.776516in}}%
\pgfpathlineto{\pgfqpoint{7.634433in}{5.774911in}}%
\pgfpathlineto{\pgfqpoint{7.669475in}{5.773088in}}%
\pgfpathlineto{\pgfqpoint{7.704518in}{5.771039in}}%
\pgfpathlineto{\pgfqpoint{7.705446in}{5.770979in}}%
\pgfpathlineto{\pgfqpoint{7.739560in}{5.768636in}}%
\pgfusepath{stroke}%
\end{pgfscope}%
\begin{pgfscope}%
\pgfpathrectangle{\pgfqpoint{0.766095in}{0.571603in}}{\pgfqpoint{6.973465in}{5.225635in}}%
\pgfusepath{clip}%
\pgfsetbuttcap%
\pgfsetroundjoin%
\pgfsetlinewidth{1.505625pt}%
\definecolor{currentstroke}{rgb}{0.182256,0.426184,0.557120}%
\pgfsetstrokecolor{currentstroke}%
\pgfsetdash{}{0pt}%
\pgfpathmoveto{\pgfqpoint{7.739560in}{3.644895in}}%
\pgfpathlineto{\pgfqpoint{7.739144in}{3.643962in}}%
\pgfpathlineto{\pgfqpoint{7.734049in}{3.632161in}}%
\pgfusepath{stroke}%
\end{pgfscope}%
\begin{pgfscope}%
\pgfpathrectangle{\pgfqpoint{0.766095in}{0.571603in}}{\pgfqpoint{6.973465in}{5.225635in}}%
\pgfusepath{clip}%
\pgfsetbuttcap%
\pgfsetroundjoin%
\pgfsetlinewidth{1.505625pt}%
\definecolor{currentstroke}{rgb}{0.182256,0.426184,0.557120}%
\pgfsetstrokecolor{currentstroke}%
\pgfsetdash{}{0pt}%
\pgfpathmoveto{\pgfqpoint{7.633412in}{3.336391in}}%
\pgfpathlineto{\pgfqpoint{7.631528in}{3.328848in}}%
\pgfpathlineto{\pgfqpoint{7.625420in}{3.302589in}}%
\pgfpathlineto{\pgfqpoint{7.619779in}{3.276329in}}%
\pgfpathlineto{\pgfqpoint{7.614607in}{3.250070in}}%
\pgfpathlineto{\pgfqpoint{7.609905in}{3.223810in}}%
\pgfpathlineto{\pgfqpoint{7.605675in}{3.197551in}}%
\pgfpathlineto{\pgfqpoint{7.601920in}{3.171291in}}%
\pgfpathlineto{\pgfqpoint{7.599390in}{3.151049in}}%
\pgfpathlineto{\pgfqpoint{7.598635in}{3.145032in}}%
\pgfpathlineto{\pgfqpoint{7.595813in}{3.118772in}}%
\pgfpathlineto{\pgfqpoint{7.593475in}{3.092513in}}%
\pgfpathlineto{\pgfqpoint{7.591622in}{3.066253in}}%
\pgfpathlineto{\pgfqpoint{7.590256in}{3.039994in}}%
\pgfpathlineto{\pgfqpoint{7.589379in}{3.013734in}}%
\pgfpathlineto{\pgfqpoint{7.588994in}{2.987475in}}%
\pgfpathlineto{\pgfqpoint{7.589102in}{2.961215in}}%
\pgfpathlineto{\pgfqpoint{7.589707in}{2.934956in}}%
\pgfpathlineto{\pgfqpoint{7.590810in}{2.908696in}}%
\pgfpathlineto{\pgfqpoint{7.592415in}{2.882437in}}%
\pgfpathlineto{\pgfqpoint{7.594523in}{2.856177in}}%
\pgfpathlineto{\pgfqpoint{7.597137in}{2.829918in}}%
\pgfpathlineto{\pgfqpoint{7.599390in}{2.810978in}}%
\pgfpathlineto{\pgfqpoint{7.600254in}{2.803659in}}%
\pgfpathlineto{\pgfqpoint{7.603865in}{2.777399in}}%
\pgfpathlineto{\pgfqpoint{7.607986in}{2.751140in}}%
\pgfpathlineto{\pgfqpoint{7.612623in}{2.724880in}}%
\pgfpathlineto{\pgfqpoint{7.617776in}{2.698621in}}%
\pgfpathlineto{\pgfqpoint{7.623450in}{2.672361in}}%
\pgfpathlineto{\pgfqpoint{7.629648in}{2.646102in}}%
\pgfpathlineto{\pgfqpoint{7.634433in}{2.627421in}}%
\pgfpathlineto{\pgfqpoint{7.636361in}{2.619842in}}%
\pgfpathlineto{\pgfqpoint{7.643569in}{2.593583in}}%
\pgfpathlineto{\pgfqpoint{7.651307in}{2.567323in}}%
\pgfpathlineto{\pgfqpoint{7.659580in}{2.541064in}}%
\pgfpathlineto{\pgfqpoint{7.668392in}{2.514804in}}%
\pgfpathlineto{\pgfqpoint{7.669475in}{2.511762in}}%
\pgfpathlineto{\pgfqpoint{7.677691in}{2.488545in}}%
\pgfpathlineto{\pgfqpoint{7.687526in}{2.462285in}}%
\pgfpathlineto{\pgfqpoint{7.697907in}{2.436026in}}%
\pgfpathlineto{\pgfqpoint{7.704518in}{2.420142in}}%
\pgfpathlineto{\pgfqpoint{7.708812in}{2.409766in}}%
\pgfpathlineto{\pgfqpoint{7.720224in}{2.383507in}}%
\pgfpathlineto{\pgfqpoint{7.732194in}{2.357248in}}%
\pgfpathlineto{\pgfqpoint{7.739560in}{2.341800in}}%
\pgfusepath{stroke}%
\end{pgfscope}%
\begin{pgfscope}%
\pgfpathrectangle{\pgfqpoint{0.766095in}{0.571603in}}{\pgfqpoint{6.973465in}{5.225635in}}%
\pgfusepath{clip}%
\pgfsetbuttcap%
\pgfsetroundjoin%
\pgfsetlinewidth{1.505625pt}%
\definecolor{currentstroke}{rgb}{0.182256,0.426184,0.557120}%
\pgfsetstrokecolor{currentstroke}%
\pgfsetdash{}{0pt}%
\pgfpathmoveto{\pgfqpoint{0.766095in}{4.406249in}}%
\pgfpathlineto{\pgfqpoint{0.871223in}{4.465175in}}%
\pgfpathlineto{\pgfqpoint{0.976350in}{4.521031in}}%
\pgfpathlineto{\pgfqpoint{1.081478in}{4.574100in}}%
\pgfpathlineto{\pgfqpoint{1.186605in}{4.624642in}}%
\pgfpathlineto{\pgfqpoint{1.291733in}{4.672896in}}%
\pgfpathlineto{\pgfqpoint{1.400465in}{4.720600in}}%
\pgfpathlineto{\pgfqpoint{1.537031in}{4.777530in}}%
\pgfpathlineto{\pgfqpoint{1.677201in}{4.832924in}}%
\pgfpathlineto{\pgfqpoint{1.817371in}{4.885538in}}%
\pgfpathlineto{\pgfqpoint{1.957541in}{4.935623in}}%
\pgfpathlineto{\pgfqpoint{2.097711in}{4.983413in}}%
\pgfpathlineto{\pgfqpoint{2.272924in}{5.040073in}}%
\pgfpathlineto{\pgfqpoint{2.448137in}{5.093746in}}%
\pgfpathlineto{\pgfqpoint{2.623350in}{5.144693in}}%
\pgfpathlineto{\pgfqpoint{2.799002in}{5.193271in}}%
\pgfpathlineto{\pgfqpoint{3.008818in}{5.248186in}}%
\pgfpathlineto{\pgfqpoint{3.219073in}{5.300203in}}%
\pgfpathlineto{\pgfqpoint{3.435495in}{5.350827in}}%
\pgfpathlineto{\pgfqpoint{3.674626in}{5.403567in}}%
\pgfpathlineto{\pgfqpoint{3.927091in}{5.455865in}}%
\pgfpathlineto{\pgfqpoint{4.165221in}{5.502181in}}%
\pgfpathlineto{\pgfqpoint{4.340583in}{5.534644in}}%
\pgfpathlineto{\pgfqpoint{4.620774in}{5.583427in}}%
\pgfpathlineto{\pgfqpoint{4.803535in}{5.613422in}}%
\pgfpathlineto{\pgfqpoint{5.076327in}{5.655353in}}%
\pgfpathlineto{\pgfqpoint{5.167521in}{5.668734in}}%
\pgfpathlineto{\pgfqpoint{5.167521in}{5.668734in}}%
\pgfusepath{stroke}%
\end{pgfscope}%
\begin{pgfscope}%
\pgfpathrectangle{\pgfqpoint{0.766095in}{0.571603in}}{\pgfqpoint{6.973465in}{5.225635in}}%
\pgfusepath{clip}%
\pgfsetbuttcap%
\pgfsetroundjoin%
\pgfsetlinewidth{1.505625pt}%
\definecolor{currentstroke}{rgb}{0.182256,0.426184,0.557120}%
\pgfsetstrokecolor{currentstroke}%
\pgfsetdash{}{0pt}%
\pgfpathmoveto{\pgfqpoint{5.475876in}{5.711114in}}%
\pgfpathlineto{\pgfqpoint{5.496838in}{5.713875in}}%
\pgfpathlineto{\pgfqpoint{5.531874in}{5.718460in}}%
\pgfpathlineto{\pgfqpoint{5.531880in}{5.718461in}}%
\pgfpathlineto{\pgfqpoint{5.566923in}{5.722880in}}%
\pgfpathlineto{\pgfqpoint{5.601965in}{5.727269in}}%
\pgfpathlineto{\pgfqpoint{5.637008in}{5.731626in}}%
\pgfpathlineto{\pgfqpoint{5.672050in}{5.735950in}}%
\pgfpathlineto{\pgfqpoint{5.707093in}{5.740242in}}%
\pgfpathlineto{\pgfqpoint{5.742136in}{5.744501in}}%
\pgfpathlineto{\pgfqpoint{5.743966in}{5.744720in}}%
\pgfpathlineto{\pgfqpoint{5.777178in}{5.748602in}}%
\pgfpathlineto{\pgfqpoint{5.812221in}{5.752661in}}%
\pgfpathlineto{\pgfqpoint{5.847263in}{5.756685in}}%
\pgfpathlineto{\pgfqpoint{5.882306in}{5.760671in}}%
\pgfpathlineto{\pgfqpoint{5.917348in}{5.764621in}}%
\pgfpathlineto{\pgfqpoint{5.952391in}{5.768532in}}%
\pgfpathlineto{\pgfqpoint{5.974607in}{5.770979in}}%
\pgfpathlineto{\pgfqpoint{5.987433in}{5.772359in}}%
\pgfpathlineto{\pgfqpoint{6.022476in}{5.776070in}}%
\pgfpathlineto{\pgfqpoint{6.057518in}{5.779739in}}%
\pgfpathlineto{\pgfqpoint{6.092561in}{5.783367in}}%
\pgfpathlineto{\pgfqpoint{6.127603in}{5.786951in}}%
\pgfpathlineto{\pgfqpoint{6.162646in}{5.790492in}}%
\pgfpathlineto{\pgfqpoint{6.197689in}{5.793988in}}%
\pgfpathlineto{\pgfqpoint{6.230715in}{5.797238in}}%
\pgfusepath{stroke}%
\end{pgfscope}%
\begin{pgfscope}%
\pgfpathrectangle{\pgfqpoint{0.766095in}{0.571603in}}{\pgfqpoint{6.973465in}{5.225635in}}%
\pgfusepath{clip}%
\pgfsetbuttcap%
\pgfsetroundjoin%
\pgfsetlinewidth{1.505625pt}%
\definecolor{currentstroke}{rgb}{0.174274,0.445044,0.557792}%
\pgfsetstrokecolor{currentstroke}%
\pgfsetdash{}{0pt}%
\pgfpathmoveto{\pgfqpoint{7.739560in}{3.069086in}}%
\pgfpathlineto{\pgfqpoint{7.739333in}{3.066253in}}%
\pgfpathlineto{\pgfqpoint{7.737713in}{3.039994in}}%
\pgfpathlineto{\pgfqpoint{7.736590in}{3.013734in}}%
\pgfpathlineto{\pgfqpoint{7.735965in}{2.987475in}}%
\pgfpathlineto{\pgfqpoint{7.735841in}{2.961215in}}%
\pgfpathlineto{\pgfqpoint{7.736220in}{2.934956in}}%
\pgfpathlineto{\pgfqpoint{7.737104in}{2.908696in}}%
\pgfpathlineto{\pgfqpoint{7.738495in}{2.882437in}}%
\pgfpathlineto{\pgfqpoint{7.739560in}{2.867728in}}%
\pgfusepath{stroke}%
\end{pgfscope}%
\begin{pgfscope}%
\pgfpathrectangle{\pgfqpoint{0.766095in}{0.571603in}}{\pgfqpoint{6.973465in}{5.225635in}}%
\pgfusepath{clip}%
\pgfsetbuttcap%
\pgfsetroundjoin%
\pgfsetlinewidth{1.505625pt}%
\definecolor{currentstroke}{rgb}{0.174274,0.445044,0.557792}%
\pgfsetstrokecolor{currentstroke}%
\pgfsetdash{}{0pt}%
\pgfpathmoveto{\pgfqpoint{0.766095in}{4.482351in}}%
\pgfpathlineto{\pgfqpoint{0.836180in}{4.520361in}}%
\pgfpathlineto{\pgfqpoint{0.941308in}{4.575025in}}%
\pgfpathlineto{\pgfqpoint{1.046435in}{4.627034in}}%
\pgfpathlineto{\pgfqpoint{1.151563in}{4.676638in}}%
\pgfpathlineto{\pgfqpoint{1.256691in}{4.724064in}}%
\pgfpathlineto{\pgfqpoint{1.370497in}{4.773119in}}%
\pgfpathlineto{\pgfqpoint{1.501988in}{4.827100in}}%
\pgfpathlineto{\pgfqpoint{1.642158in}{4.881748in}}%
\pgfpathlineto{\pgfqpoint{1.782329in}{4.933728in}}%
\pgfpathlineto{\pgfqpoint{1.922499in}{4.983285in}}%
\pgfpathlineto{\pgfqpoint{2.097711in}{5.042007in}}%
\pgfpathlineto{\pgfqpoint{2.242504in}{5.088233in}}%
\pgfpathlineto{\pgfqpoint{2.415267in}{5.140752in}}%
\pgfpathlineto{\pgfqpoint{2.623350in}{5.200542in}}%
\pgfpathlineto{\pgfqpoint{2.798562in}{5.248343in}}%
\pgfpathlineto{\pgfqpoint{3.008818in}{5.302750in}}%
\pgfpathlineto{\pgfqpoint{3.219073in}{5.354282in}}%
\pgfpathlineto{\pgfqpoint{3.430091in}{5.403346in}}%
\pgfpathlineto{\pgfqpoint{3.674626in}{5.457045in}}%
\pgfpathlineto{\pgfqpoint{3.922687in}{5.508384in}}%
\pgfpathlineto{\pgfqpoint{4.200264in}{5.562302in}}%
\pgfpathlineto{\pgfqpoint{4.481519in}{5.613422in}}%
\pgfpathlineto{\pgfqpoint{4.642448in}{5.641109in}}%
\pgfpathlineto{\pgfqpoint{4.642448in}{5.641109in}}%
\pgfusepath{stroke}%
\end{pgfscope}%
\begin{pgfscope}%
\pgfpathrectangle{\pgfqpoint{0.766095in}{0.571603in}}{\pgfqpoint{6.973465in}{5.225635in}}%
\pgfusepath{clip}%
\pgfsetbuttcap%
\pgfsetroundjoin%
\pgfsetlinewidth{1.505625pt}%
\definecolor{currentstroke}{rgb}{0.174274,0.445044,0.557792}%
\pgfsetstrokecolor{currentstroke}%
\pgfsetdash{}{0pt}%
\pgfpathmoveto{\pgfqpoint{4.949683in}{5.691041in}}%
\pgfpathlineto{\pgfqpoint{4.957035in}{5.692201in}}%
\pgfpathlineto{\pgfqpoint{4.971200in}{5.694389in}}%
\pgfpathlineto{\pgfqpoint{5.006242in}{5.699763in}}%
\pgfpathlineto{\pgfqpoint{5.041285in}{5.705118in}}%
\pgfpathlineto{\pgfqpoint{5.076327in}{5.710455in}}%
\pgfpathlineto{\pgfqpoint{5.111370in}{5.715773in}}%
\pgfpathlineto{\pgfqpoint{5.129206in}{5.718460in}}%
\pgfpathlineto{\pgfqpoint{5.146412in}{5.720999in}}%
\pgfpathlineto{\pgfqpoint{5.181455in}{5.726130in}}%
\pgfpathlineto{\pgfqpoint{5.216497in}{5.731240in}}%
\pgfpathlineto{\pgfqpoint{5.251540in}{5.736331in}}%
\pgfpathlineto{\pgfqpoint{5.286583in}{5.741400in}}%
\pgfpathlineto{\pgfqpoint{5.309681in}{5.744720in}}%
\pgfpathlineto{\pgfqpoint{5.321625in}{5.746400in}}%
\pgfpathlineto{\pgfqpoint{5.356668in}{5.751283in}}%
\pgfpathlineto{\pgfqpoint{5.391710in}{5.756144in}}%
\pgfpathlineto{\pgfqpoint{5.426753in}{5.760982in}}%
\pgfpathlineto{\pgfqpoint{5.461795in}{5.765797in}}%
\pgfpathlineto{\pgfqpoint{5.496838in}{5.770590in}}%
\pgfpathlineto{\pgfqpoint{5.499721in}{5.770979in}}%
\pgfpathlineto{\pgfqpoint{5.531880in}{5.775230in}}%
\pgfpathlineto{\pgfqpoint{5.566923in}{5.779835in}}%
\pgfpathlineto{\pgfqpoint{5.601965in}{5.784414in}}%
\pgfpathlineto{\pgfqpoint{5.637008in}{5.788969in}}%
\pgfpathlineto{\pgfqpoint{5.672050in}{5.793497in}}%
\pgfpathlineto{\pgfqpoint{5.701216in}{5.797238in}}%
\pgfusepath{stroke}%
\end{pgfscope}%
\begin{pgfscope}%
\pgfpathrectangle{\pgfqpoint{0.766095in}{0.571603in}}{\pgfqpoint{6.973465in}{5.225635in}}%
\pgfusepath{clip}%
\pgfsetbuttcap%
\pgfsetroundjoin%
\pgfsetlinewidth{1.505625pt}%
\definecolor{currentstroke}{rgb}{0.166617,0.463708,0.558119}%
\pgfsetstrokecolor{currentstroke}%
\pgfsetdash{}{0pt}%
\pgfpathmoveto{\pgfqpoint{0.766095in}{4.553685in}}%
\pgfpathlineto{\pgfqpoint{0.783700in}{4.563043in}}%
\pgfpathlineto{\pgfqpoint{0.801138in}{4.572161in}}%
\pgfpathlineto{\pgfqpoint{0.834075in}{4.589303in}}%
\pgfpathlineto{\pgfqpoint{0.836180in}{4.590380in}}%
\pgfpathlineto{\pgfqpoint{0.871223in}{4.608196in}}%
\pgfpathlineto{\pgfqpoint{0.885781in}{4.615562in}}%
\pgfpathlineto{\pgfqpoint{0.906265in}{4.625757in}}%
\pgfpathlineto{\pgfqpoint{0.938675in}{4.641822in}}%
\pgfpathlineto{\pgfqpoint{0.941308in}{4.643105in}}%
\pgfpathlineto{\pgfqpoint{0.976350in}{4.660074in}}%
\pgfpathlineto{\pgfqpoint{0.992956in}{4.668081in}}%
\pgfpathlineto{\pgfqpoint{1.011393in}{4.676825in}}%
\pgfpathlineto{\pgfqpoint{1.046435in}{4.693378in}}%
\pgfpathlineto{\pgfqpoint{1.048486in}{4.694341in}}%
\pgfpathlineto{\pgfqpoint{1.081478in}{4.709566in}}%
\pgfpathlineto{\pgfqpoint{1.105469in}{4.720600in}}%
\pgfpathlineto{\pgfqpoint{1.116520in}{4.725600in}}%
\pgfpathlineto{\pgfqpoint{1.151563in}{4.741371in}}%
\pgfpathlineto{\pgfqpoint{1.163820in}{4.746860in}}%
\pgfpathlineto{\pgfqpoint{1.186605in}{4.756895in}}%
\pgfpathlineto{\pgfqpoint{1.221648in}{4.772281in}}%
\pgfpathlineto{\pgfqpoint{1.223569in}{4.773119in}}%
\pgfpathlineto{\pgfqpoint{1.256691in}{4.787326in}}%
\pgfpathlineto{\pgfqpoint{1.284859in}{4.799378in}}%
\pgfpathlineto{\pgfqpoint{1.291733in}{4.802271in}}%
\pgfpathlineto{\pgfqpoint{1.326776in}{4.816932in}}%
\pgfpathlineto{\pgfqpoint{1.347657in}{4.825638in}}%
\pgfpathlineto{\pgfqpoint{1.361818in}{4.831445in}}%
\pgfpathlineto{\pgfqpoint{1.396861in}{4.845749in}}%
\pgfpathlineto{\pgfqpoint{1.411991in}{4.851897in}}%
\pgfpathlineto{\pgfqpoint{1.431903in}{4.859856in}}%
\pgfpathlineto{\pgfqpoint{1.466946in}{4.873813in}}%
\pgfpathlineto{\pgfqpoint{1.477909in}{4.878157in}}%
\pgfpathlineto{\pgfqpoint{1.501988in}{4.887540in}}%
\pgfpathlineto{\pgfqpoint{1.537031in}{4.901158in}}%
\pgfpathlineto{\pgfqpoint{1.545463in}{4.904416in}}%
\pgfpathlineto{\pgfqpoint{1.572073in}{4.914530in}}%
\pgfpathlineto{\pgfqpoint{1.607116in}{4.927817in}}%
\pgfpathlineto{\pgfqpoint{1.614699in}{4.930676in}}%
\pgfpathlineto{\pgfqpoint{1.642158in}{4.940857in}}%
\pgfpathlineto{\pgfqpoint{1.677201in}{4.953822in}}%
\pgfpathlineto{\pgfqpoint{1.685663in}{4.956935in}}%
\pgfpathlineto{\pgfqpoint{1.712244in}{4.966553in}}%
\pgfpathlineto{\pgfqpoint{1.747286in}{4.979202in}}%
\pgfpathlineto{\pgfqpoint{1.758401in}{4.983195in}}%
\pgfpathlineto{\pgfqpoint{1.782329in}{4.991646in}}%
\pgfpathlineto{\pgfqpoint{1.817371in}{5.003988in}}%
\pgfpathlineto{\pgfqpoint{1.832955in}{5.009454in}}%
\pgfpathlineto{\pgfqpoint{1.852414in}{5.016165in}}%
\pgfpathlineto{\pgfqpoint{1.887456in}{5.028208in}}%
\pgfpathlineto{\pgfqpoint{1.909365in}{5.035714in}}%
\pgfpathlineto{\pgfqpoint{1.922499in}{5.040138in}}%
\pgfpathlineto{\pgfqpoint{1.957541in}{5.051888in}}%
\pgfpathlineto{\pgfqpoint{1.987670in}{5.061973in}}%
\pgfpathlineto{\pgfqpoint{1.992584in}{5.063591in}}%
\pgfpathlineto{\pgfqpoint{2.027626in}{5.075054in}}%
\pgfpathlineto{\pgfqpoint{2.062669in}{5.086509in}}%
\pgfpathlineto{\pgfqpoint{2.067973in}{5.088233in}}%
\pgfpathlineto{\pgfqpoint{2.097711in}{5.097733in}}%
\pgfpathlineto{\pgfqpoint{2.132754in}{5.108908in}}%
\pgfpathlineto{\pgfqpoint{2.150330in}{5.114492in}}%
\pgfpathlineto{\pgfqpoint{2.167797in}{5.119947in}}%
\pgfpathlineto{\pgfqpoint{2.202839in}{5.130850in}}%
\pgfpathlineto{\pgfqpoint{2.234707in}{5.140752in}}%
\pgfpathlineto{\pgfqpoint{2.237882in}{5.141721in}}%
\pgfpathlineto{\pgfqpoint{2.272924in}{5.152357in}}%
\pgfpathlineto{\pgfqpoint{2.307967in}{5.162985in}}%
\pgfpathlineto{\pgfqpoint{2.321304in}{5.167011in}}%
\pgfpathlineto{\pgfqpoint{2.343009in}{5.173452in}}%
\pgfpathlineto{\pgfqpoint{2.378052in}{5.183819in}}%
\pgfpathlineto{\pgfqpoint{2.410041in}{5.193271in}}%
\pgfpathlineto{\pgfqpoint{2.413094in}{5.194157in}}%
\pgfpathlineto{\pgfqpoint{2.448137in}{5.204269in}}%
\pgfpathlineto{\pgfqpoint{2.483179in}{5.214374in}}%
\pgfpathlineto{\pgfqpoint{2.501130in}{5.219530in}}%
\pgfpathlineto{\pgfqpoint{2.518222in}{5.224356in}}%
\pgfpathlineto{\pgfqpoint{2.553264in}{5.234211in}}%
\pgfpathlineto{\pgfqpoint{2.588307in}{5.244058in}}%
\pgfpathlineto{\pgfqpoint{2.594505in}{5.245790in}}%
\pgfpathlineto{\pgfqpoint{2.623350in}{5.253709in}}%
\pgfpathlineto{\pgfqpoint{2.658392in}{5.263313in}}%
\pgfpathlineto{\pgfqpoint{2.690312in}{5.272049in}}%
\pgfpathlineto{\pgfqpoint{2.693435in}{5.272889in}}%
\pgfpathlineto{\pgfqpoint{2.728477in}{5.282253in}}%
\pgfpathlineto{\pgfqpoint{2.763520in}{5.291611in}}%
\pgfpathlineto{\pgfqpoint{2.788665in}{5.298308in}}%
\pgfpathlineto{\pgfqpoint{2.798562in}{5.300899in}}%
\pgfpathlineto{\pgfqpoint{2.833605in}{5.310022in}}%
\pgfpathlineto{\pgfqpoint{2.868647in}{5.319139in}}%
\pgfpathlineto{\pgfqpoint{2.889585in}{5.324568in}}%
\pgfpathlineto{\pgfqpoint{2.903690in}{5.328161in}}%
\pgfpathlineto{\pgfqpoint{2.938732in}{5.337049in}}%
\pgfpathlineto{\pgfqpoint{2.973775in}{5.345929in}}%
\pgfpathlineto{\pgfqpoint{2.993174in}{5.350827in}}%
\pgfpathlineto{\pgfqpoint{3.008818in}{5.354708in}}%
\pgfpathlineto{\pgfqpoint{3.043860in}{5.363364in}}%
\pgfpathlineto{\pgfqpoint{3.078903in}{5.372012in}}%
\pgfpathlineto{\pgfqpoint{3.099533in}{5.377087in}}%
\pgfpathlineto{\pgfqpoint{3.113945in}{5.380569in}}%
\pgfpathlineto{\pgfqpoint{3.148988in}{5.388997in}}%
\pgfpathlineto{\pgfqpoint{3.184030in}{5.397419in}}%
\pgfpathlineto{\pgfqpoint{3.208761in}{5.403346in}}%
\pgfpathlineto{\pgfqpoint{3.219073in}{5.405774in}}%
\pgfpathlineto{\pgfqpoint{3.254115in}{5.413979in}}%
\pgfpathlineto{\pgfqpoint{3.289158in}{5.422177in}}%
\pgfpathlineto{\pgfqpoint{3.320955in}{5.429606in}}%
\pgfpathlineto{\pgfqpoint{3.324200in}{5.430350in}}%
\pgfpathlineto{\pgfqpoint{3.359243in}{5.438337in}}%
\pgfpathlineto{\pgfqpoint{3.394285in}{5.446316in}}%
\pgfpathlineto{\pgfqpoint{3.429328in}{5.454287in}}%
\pgfpathlineto{\pgfqpoint{3.436306in}{5.455865in}}%
\pgfpathlineto{\pgfqpoint{3.464371in}{5.462097in}}%
\pgfpathlineto{\pgfqpoint{3.499413in}{5.469860in}}%
\pgfpathlineto{\pgfqpoint{3.534456in}{5.477616in}}%
\pgfpathlineto{\pgfqpoint{3.554901in}{5.482125in}}%
\pgfpathlineto{\pgfqpoint{3.569498in}{5.485285in}}%
\pgfpathlineto{\pgfqpoint{3.604541in}{5.492837in}}%
\pgfpathlineto{\pgfqpoint{3.639583in}{5.500380in}}%
\pgfpathlineto{\pgfqpoint{3.674626in}{5.507916in}}%
\pgfpathlineto{\pgfqpoint{3.676818in}{5.508384in}}%
\pgfpathlineto{\pgfqpoint{3.709668in}{5.515270in}}%
\pgfpathlineto{\pgfqpoint{3.732434in}{5.520035in}}%
\pgfusepath{stroke}%
\end{pgfscope}%
\begin{pgfscope}%
\pgfpathrectangle{\pgfqpoint{0.766095in}{0.571603in}}{\pgfqpoint{6.973465in}{5.225635in}}%
\pgfusepath{clip}%
\pgfsetbuttcap%
\pgfsetroundjoin%
\pgfsetlinewidth{1.505625pt}%
\definecolor{currentstroke}{rgb}{0.166617,0.463708,0.558119}%
\pgfsetstrokecolor{currentstroke}%
\pgfsetdash{}{0pt}%
\pgfpathmoveto{\pgfqpoint{4.037532in}{5.581810in}}%
\pgfpathlineto{\pgfqpoint{4.060094in}{5.586251in}}%
\pgfpathlineto{\pgfqpoint{4.064756in}{5.587163in}}%
\pgfpathlineto{\pgfqpoint{4.095136in}{5.592990in}}%
\pgfpathlineto{\pgfqpoint{4.130179in}{5.599696in}}%
\pgfpathlineto{\pgfqpoint{4.165221in}{5.606392in}}%
\pgfpathlineto{\pgfqpoint{4.200264in}{5.613080in}}%
\pgfpathlineto{\pgfqpoint{4.202071in}{5.613422in}}%
\pgfpathlineto{\pgfqpoint{4.235306in}{5.619596in}}%
\pgfpathlineto{\pgfqpoint{4.270349in}{5.626093in}}%
\pgfpathlineto{\pgfqpoint{4.305391in}{5.632580in}}%
\pgfpathlineto{\pgfqpoint{4.340434in}{5.639057in}}%
\pgfpathlineto{\pgfqpoint{4.343837in}{5.639682in}}%
\pgfpathlineto{\pgfqpoint{4.375477in}{5.645374in}}%
\pgfpathlineto{\pgfqpoint{4.410519in}{5.651663in}}%
\pgfpathlineto{\pgfqpoint{4.445562in}{5.657942in}}%
\pgfpathlineto{\pgfqpoint{4.480604in}{5.664210in}}%
\pgfpathlineto{\pgfqpoint{4.490349in}{5.665941in}}%
\pgfpathlineto{\pgfqpoint{4.515647in}{5.670349in}}%
\pgfpathlineto{\pgfqpoint{4.550689in}{5.676431in}}%
\pgfpathlineto{\pgfqpoint{4.585732in}{5.682502in}}%
\pgfpathlineto{\pgfqpoint{4.620774in}{5.688561in}}%
\pgfpathlineto{\pgfqpoint{4.641922in}{5.692201in}}%
\pgfpathlineto{\pgfqpoint{4.655817in}{5.694545in}}%
\pgfpathlineto{\pgfqpoint{4.690859in}{5.700421in}}%
\pgfpathlineto{\pgfqpoint{4.725902in}{5.706284in}}%
\pgfpathlineto{\pgfqpoint{4.760944in}{5.712135in}}%
\pgfpathlineto{\pgfqpoint{4.795987in}{5.717973in}}%
\pgfpathlineto{\pgfqpoint{4.798937in}{5.718460in}}%
\pgfpathlineto{\pgfqpoint{4.831030in}{5.723655in}}%
\pgfpathlineto{\pgfqpoint{4.866072in}{5.729311in}}%
\pgfpathlineto{\pgfqpoint{4.901115in}{5.734953in}}%
\pgfpathlineto{\pgfqpoint{4.936157in}{5.740581in}}%
\pgfpathlineto{\pgfqpoint{4.962036in}{5.744720in}}%
\pgfpathlineto{\pgfqpoint{4.971200in}{5.746155in}}%
\pgfpathlineto{\pgfqpoint{5.006242in}{5.751603in}}%
\pgfpathlineto{\pgfqpoint{5.041285in}{5.757036in}}%
\pgfpathlineto{\pgfqpoint{5.076327in}{5.762454in}}%
\pgfpathlineto{\pgfqpoint{5.111370in}{5.767857in}}%
\pgfpathlineto{\pgfqpoint{5.131738in}{5.770979in}}%
\pgfpathlineto{\pgfqpoint{5.146412in}{5.773182in}}%
\pgfpathlineto{\pgfqpoint{5.181455in}{5.778406in}}%
\pgfpathlineto{\pgfqpoint{5.216497in}{5.783612in}}%
\pgfpathlineto{\pgfqpoint{5.251540in}{5.788803in}}%
\pgfpathlineto{\pgfqpoint{5.286583in}{5.793976in}}%
\pgfpathlineto{\pgfqpoint{5.308810in}{5.797238in}}%
\pgfusepath{stroke}%
\end{pgfscope}%
\begin{pgfscope}%
\pgfpathrectangle{\pgfqpoint{0.766095in}{0.571603in}}{\pgfqpoint{6.973465in}{5.225635in}}%
\pgfusepath{clip}%
\pgfsetbuttcap%
\pgfsetroundjoin%
\pgfsetlinewidth{1.505625pt}%
\definecolor{currentstroke}{rgb}{0.159194,0.482237,0.558073}%
\pgfsetstrokecolor{currentstroke}%
\pgfsetdash{}{0pt}%
\pgfpathmoveto{\pgfqpoint{0.766095in}{4.620952in}}%
\pgfpathlineto{\pgfqpoint{0.871223in}{4.673597in}}%
\pgfpathlineto{\pgfqpoint{0.976350in}{4.723829in}}%
\pgfpathlineto{\pgfqpoint{1.084317in}{4.773119in}}%
\pgfpathlineto{\pgfqpoint{1.221648in}{4.832670in}}%
\pgfpathlineto{\pgfqpoint{1.331529in}{4.878157in}}%
\pgfpathlineto{\pgfqpoint{1.466946in}{4.931657in}}%
\pgfpathlineto{\pgfqpoint{1.607116in}{4.984382in}}%
\pgfpathlineto{\pgfqpoint{1.750291in}{5.035714in}}%
\pgfpathlineto{\pgfqpoint{1.922499in}{5.094323in}}%
\pgfpathlineto{\pgfqpoint{2.097711in}{5.150864in}}%
\pgfpathlineto{\pgfqpoint{2.237882in}{5.194109in}}%
\pgfpathlineto{\pgfqpoint{2.413275in}{5.245790in}}%
\pgfpathlineto{\pgfqpoint{2.623350in}{5.304462in}}%
\pgfpathlineto{\pgfqpoint{2.798562in}{5.351088in}}%
\pgfpathlineto{\pgfqpoint{3.008818in}{5.404327in}}%
\pgfpathlineto{\pgfqpoint{3.254115in}{5.463064in}}%
\pgfpathlineto{\pgfqpoint{3.464371in}{5.510870in}}%
\pgfpathlineto{\pgfqpoint{3.709668in}{5.563785in}}%
\pgfpathlineto{\pgfqpoint{3.954966in}{5.613924in}}%
\pgfpathlineto{\pgfqpoint{4.222420in}{5.665581in}}%
\pgfpathlineto{\pgfqpoint{4.222420in}{5.665581in}}%
\pgfusepath{stroke}%
\end{pgfscope}%
\begin{pgfscope}%
\pgfpathrectangle{\pgfqpoint{0.766095in}{0.571603in}}{\pgfqpoint{6.973465in}{5.225635in}}%
\pgfusepath{clip}%
\pgfsetbuttcap%
\pgfsetroundjoin%
\pgfsetlinewidth{1.505625pt}%
\definecolor{currentstroke}{rgb}{0.159194,0.482237,0.558073}%
\pgfsetstrokecolor{currentstroke}%
\pgfsetdash{}{0pt}%
\pgfpathmoveto{\pgfqpoint{4.528694in}{5.721166in}}%
\pgfpathlineto{\pgfqpoint{4.550689in}{5.724987in}}%
\pgfpathlineto{\pgfqpoint{4.585732in}{5.731067in}}%
\pgfpathlineto{\pgfqpoint{4.620774in}{5.737137in}}%
\pgfpathlineto{\pgfqpoint{4.655817in}{5.743197in}}%
\pgfpathlineto{\pgfqpoint{4.664676in}{5.744720in}}%
\pgfpathlineto{\pgfqpoint{4.690859in}{5.749131in}}%
\pgfpathlineto{\pgfqpoint{4.725902in}{5.755014in}}%
\pgfpathlineto{\pgfqpoint{4.760944in}{5.760887in}}%
\pgfpathlineto{\pgfqpoint{4.795987in}{5.766750in}}%
\pgfpathlineto{\pgfqpoint{4.821359in}{5.770979in}}%
\pgfpathlineto{\pgfqpoint{4.831030in}{5.772559in}}%
\pgfpathlineto{\pgfqpoint{4.866072in}{5.778247in}}%
\pgfpathlineto{\pgfqpoint{4.901115in}{5.783923in}}%
\pgfpathlineto{\pgfqpoint{4.936157in}{5.789588in}}%
\pgfpathlineto{\pgfqpoint{4.971200in}{5.795242in}}%
\pgfpathlineto{\pgfqpoint{4.983652in}{5.797238in}}%
\pgfusepath{stroke}%
\end{pgfscope}%
\begin{pgfscope}%
\pgfpathrectangle{\pgfqpoint{0.766095in}{0.571603in}}{\pgfqpoint{6.973465in}{5.225635in}}%
\pgfusepath{clip}%
\pgfsetbuttcap%
\pgfsetroundjoin%
\pgfsetlinewidth{1.505625pt}%
\definecolor{currentstroke}{rgb}{0.151918,0.500685,0.557587}%
\pgfsetstrokecolor{currentstroke}%
\pgfsetdash{}{0pt}%
\pgfpathmoveto{\pgfqpoint{0.766095in}{4.684533in}}%
\pgfpathlineto{\pgfqpoint{0.785857in}{4.694341in}}%
\pgfpathlineto{\pgfqpoint{0.801138in}{4.701801in}}%
\pgfpathlineto{\pgfqpoint{0.836180in}{4.718835in}}%
\pgfpathlineto{\pgfqpoint{0.839835in}{4.720600in}}%
\pgfpathlineto{\pgfqpoint{0.871223in}{4.735514in}}%
\pgfpathlineto{\pgfqpoint{0.895184in}{4.746860in}}%
\pgfpathlineto{\pgfqpoint{0.906265in}{4.752021in}}%
\pgfpathlineto{\pgfqpoint{0.941308in}{4.768260in}}%
\pgfpathlineto{\pgfqpoint{0.951847in}{4.773119in}}%
\pgfpathlineto{\pgfqpoint{0.976350in}{4.784233in}}%
\pgfpathlineto{\pgfqpoint{1.009848in}{4.799378in}}%
\pgfpathlineto{\pgfqpoint{1.011393in}{4.800066in}}%
\pgfpathlineto{\pgfqpoint{1.046435in}{4.815550in}}%
\pgfpathlineto{\pgfqpoint{1.069334in}{4.825638in}}%
\pgfpathlineto{\pgfqpoint{1.081478in}{4.830900in}}%
\pgfpathlineto{\pgfqpoint{1.116520in}{4.846014in}}%
\pgfpathlineto{\pgfqpoint{1.130225in}{4.851897in}}%
\pgfpathlineto{\pgfqpoint{1.151563in}{4.860909in}}%
\pgfpathlineto{\pgfqpoint{1.186605in}{4.875661in}}%
\pgfpathlineto{\pgfqpoint{1.192569in}{4.878157in}}%
\pgfpathlineto{\pgfqpoint{1.221648in}{4.890129in}}%
\pgfpathlineto{\pgfqpoint{1.256419in}{4.904416in}}%
\pgfpathlineto{\pgfqpoint{1.256691in}{4.904526in}}%
\pgfpathlineto{\pgfqpoint{1.291733in}{4.918596in}}%
\pgfpathlineto{\pgfqpoint{1.321873in}{4.930676in}}%
\pgfpathlineto{\pgfqpoint{1.326776in}{4.932609in}}%
\pgfpathlineto{\pgfqpoint{1.361818in}{4.946343in}}%
\pgfpathlineto{\pgfqpoint{1.388905in}{4.956935in}}%
\pgfpathlineto{\pgfqpoint{1.396861in}{4.959996in}}%
\pgfpathlineto{\pgfqpoint{1.431903in}{4.973403in}}%
\pgfpathlineto{\pgfqpoint{1.457558in}{4.983195in}}%
\pgfpathlineto{\pgfqpoint{1.466946in}{4.986719in}}%
\pgfpathlineto{\pgfqpoint{1.501988in}{4.999808in}}%
\pgfpathlineto{\pgfqpoint{1.527875in}{5.009454in}}%
\pgfpathlineto{\pgfqpoint{1.537031in}{5.012810in}}%
\pgfpathlineto{\pgfqpoint{1.572073in}{5.025588in}}%
\pgfpathlineto{\pgfqpoint{1.599898in}{5.035714in}}%
\pgfpathlineto{\pgfqpoint{1.607116in}{5.038297in}}%
\pgfpathlineto{\pgfqpoint{1.642158in}{5.050772in}}%
\pgfpathlineto{\pgfqpoint{1.673667in}{5.061973in}}%
\pgfpathlineto{\pgfqpoint{1.677201in}{5.063209in}}%
\pgfpathlineto{\pgfqpoint{1.712244in}{5.075388in}}%
\pgfpathlineto{\pgfqpoint{1.747286in}{5.087557in}}%
\pgfpathlineto{\pgfqpoint{1.749243in}{5.088233in}}%
\pgfpathlineto{\pgfqpoint{1.782329in}{5.099463in}}%
\pgfpathlineto{\pgfqpoint{1.817371in}{5.111344in}}%
\pgfpathlineto{\pgfqpoint{1.826703in}{5.114492in}}%
\pgfpathlineto{\pgfqpoint{1.852414in}{5.123022in}}%
\pgfpathlineto{\pgfqpoint{1.887456in}{5.134622in}}%
\pgfpathlineto{\pgfqpoint{1.906037in}{5.140752in}}%
\pgfpathlineto{\pgfqpoint{1.922499in}{5.146092in}}%
\pgfpathlineto{\pgfqpoint{1.957541in}{5.157417in}}%
\pgfpathlineto{\pgfqpoint{1.987276in}{5.167011in}}%
\pgfpathlineto{\pgfqpoint{1.992584in}{5.168695in}}%
\pgfpathlineto{\pgfqpoint{2.027626in}{5.179752in}}%
\pgfpathlineto{\pgfqpoint{2.062669in}{5.190800in}}%
\pgfpathlineto{\pgfqpoint{2.070544in}{5.193271in}}%
\pgfpathlineto{\pgfqpoint{2.097711in}{5.201650in}}%
\pgfpathlineto{\pgfqpoint{2.132754in}{5.212436in}}%
\pgfpathlineto{\pgfqpoint{2.155859in}{5.219530in}}%
\pgfpathlineto{\pgfqpoint{2.167797in}{5.223134in}}%
\pgfpathlineto{\pgfqpoint{2.202839in}{5.233664in}}%
\pgfpathlineto{\pgfqpoint{2.237882in}{5.244188in}}%
\pgfpathlineto{\pgfqpoint{2.243244in}{5.245790in}}%
\pgfpathlineto{\pgfqpoint{2.272924in}{5.254505in}}%
\pgfpathlineto{\pgfqpoint{2.307967in}{5.264778in}}%
\pgfpathlineto{\pgfqpoint{2.332823in}{5.272049in}}%
\pgfpathlineto{\pgfqpoint{2.343009in}{5.274978in}}%
\pgfpathlineto{\pgfqpoint{2.378052in}{5.285007in}}%
\pgfpathlineto{\pgfqpoint{2.413094in}{5.295030in}}%
\pgfpathlineto{\pgfqpoint{2.424607in}{5.298308in}}%
\pgfpathlineto{\pgfqpoint{2.448137in}{5.304895in}}%
\pgfpathlineto{\pgfqpoint{2.483179in}{5.314679in}}%
\pgfpathlineto{\pgfqpoint{2.518222in}{5.324457in}}%
\pgfpathlineto{\pgfqpoint{2.518622in}{5.324568in}}%
\pgfpathlineto{\pgfqpoint{2.553264in}{5.334010in}}%
\pgfpathlineto{\pgfqpoint{2.588307in}{5.343554in}}%
\pgfpathlineto{\pgfqpoint{2.615062in}{5.350827in}}%
\pgfpathlineto{\pgfqpoint{2.623350in}{5.353042in}}%
\pgfpathlineto{\pgfqpoint{2.658392in}{5.362357in}}%
\pgfpathlineto{\pgfqpoint{2.693435in}{5.371668in}}%
\pgfpathlineto{\pgfqpoint{2.713893in}{5.377087in}}%
\pgfpathlineto{\pgfqpoint{2.728477in}{5.380884in}}%
\pgfpathlineto{\pgfqpoint{2.763520in}{5.389970in}}%
\pgfpathlineto{\pgfqpoint{2.798562in}{5.399051in}}%
\pgfpathlineto{\pgfqpoint{2.815196in}{5.403346in}}%
\pgfpathlineto{\pgfqpoint{2.833605in}{5.408018in}}%
\pgfpathlineto{\pgfqpoint{2.868647in}{5.416880in}}%
\pgfpathlineto{\pgfqpoint{2.903690in}{5.425737in}}%
\pgfpathlineto{\pgfqpoint{2.919057in}{5.429606in}}%
\pgfpathlineto{\pgfqpoint{2.938732in}{5.434474in}}%
\pgfpathlineto{\pgfqpoint{2.973775in}{5.443116in}}%
\pgfpathlineto{\pgfqpoint{3.008818in}{5.451753in}}%
\pgfpathlineto{\pgfqpoint{3.025560in}{5.455865in}}%
\pgfpathlineto{\pgfqpoint{3.043860in}{5.460281in}}%
\pgfpathlineto{\pgfqpoint{3.078903in}{5.468708in}}%
\pgfpathlineto{\pgfqpoint{3.113945in}{5.477131in}}%
\pgfpathlineto{\pgfqpoint{3.134786in}{5.482125in}}%
\pgfpathlineto{\pgfqpoint{3.148988in}{5.485468in}}%
\pgfpathlineto{\pgfqpoint{3.184030in}{5.493684in}}%
\pgfpathlineto{\pgfqpoint{3.219073in}{5.501895in}}%
\pgfpathlineto{\pgfqpoint{3.246814in}{5.508384in}}%
\pgfpathlineto{\pgfqpoint{3.254115in}{5.510062in}}%
\pgfpathlineto{\pgfqpoint{3.289158in}{5.518071in}}%
\pgfpathlineto{\pgfqpoint{3.312496in}{5.523402in}}%
\pgfusepath{stroke}%
\end{pgfscope}%
\begin{pgfscope}%
\pgfpathrectangle{\pgfqpoint{0.766095in}{0.571603in}}{\pgfqpoint{6.973465in}{5.225635in}}%
\pgfusepath{clip}%
\pgfsetbuttcap%
\pgfsetroundjoin%
\pgfsetlinewidth{1.505625pt}%
\definecolor{currentstroke}{rgb}{0.151918,0.500685,0.557587}%
\pgfsetstrokecolor{currentstroke}%
\pgfsetdash{}{0pt}%
\pgfpathmoveto{\pgfqpoint{3.616486in}{5.590472in}}%
\pgfpathlineto{\pgfqpoint{3.639583in}{5.595352in}}%
\pgfpathlineto{\pgfqpoint{3.674626in}{5.602751in}}%
\pgfpathlineto{\pgfqpoint{3.709668in}{5.610146in}}%
\pgfpathlineto{\pgfqpoint{3.725255in}{5.613422in}}%
\pgfpathlineto{\pgfqpoint{3.744711in}{5.617437in}}%
\pgfpathlineto{\pgfqpoint{3.779753in}{5.624643in}}%
\pgfpathlineto{\pgfqpoint{3.814796in}{5.631844in}}%
\pgfpathlineto{\pgfqpoint{3.849838in}{5.639040in}}%
\pgfpathlineto{\pgfqpoint{3.852982in}{5.639682in}}%
\pgfpathlineto{\pgfqpoint{3.884881in}{5.646072in}}%
\pgfpathlineto{\pgfqpoint{3.919924in}{5.653082in}}%
\pgfpathlineto{\pgfqpoint{3.954966in}{5.660087in}}%
\pgfpathlineto{\pgfqpoint{3.984303in}{5.665941in}}%
\pgfpathlineto{\pgfqpoint{3.990009in}{5.667059in}}%
\pgfpathlineto{\pgfqpoint{4.025051in}{5.673881in}}%
\pgfpathlineto{\pgfqpoint{4.060094in}{5.680698in}}%
\pgfpathlineto{\pgfqpoint{4.095136in}{5.687509in}}%
\pgfpathlineto{\pgfqpoint{4.119340in}{5.692201in}}%
\pgfpathlineto{\pgfqpoint{4.130179in}{5.694262in}}%
\pgfpathlineto{\pgfqpoint{4.165221in}{5.700894in}}%
\pgfpathlineto{\pgfqpoint{4.200264in}{5.707519in}}%
\pgfpathlineto{\pgfqpoint{4.235306in}{5.714138in}}%
\pgfpathlineto{\pgfqpoint{4.258259in}{5.718460in}}%
\pgfpathlineto{\pgfqpoint{4.270349in}{5.720694in}}%
\pgfpathlineto{\pgfqpoint{4.305391in}{5.727136in}}%
\pgfpathlineto{\pgfqpoint{4.340434in}{5.733571in}}%
\pgfpathlineto{\pgfqpoint{4.375477in}{5.740000in}}%
\pgfpathlineto{\pgfqpoint{4.401273in}{5.744720in}}%
\pgfpathlineto{\pgfqpoint{4.410519in}{5.746379in}}%
\pgfpathlineto{\pgfqpoint{4.445562in}{5.752633in}}%
\pgfpathlineto{\pgfqpoint{4.480604in}{5.758880in}}%
\pgfpathlineto{\pgfqpoint{4.515647in}{5.765119in}}%
\pgfpathlineto{\pgfqpoint{4.548605in}{5.770979in}}%
\pgfpathlineto{\pgfqpoint{4.550689in}{5.771343in}}%
\pgfpathlineto{\pgfqpoint{4.585732in}{5.777409in}}%
\pgfpathlineto{\pgfqpoint{4.620774in}{5.783468in}}%
\pgfpathlineto{\pgfqpoint{4.655817in}{5.789520in}}%
\pgfpathlineto{\pgfqpoint{4.690859in}{5.795564in}}%
\pgfpathlineto{\pgfqpoint{4.700626in}{5.797238in}}%
\pgfusepath{stroke}%
\end{pgfscope}%
\begin{pgfscope}%
\pgfpathrectangle{\pgfqpoint{0.766095in}{0.571603in}}{\pgfqpoint{6.973465in}{5.225635in}}%
\pgfusepath{clip}%
\pgfsetbuttcap%
\pgfsetroundjoin%
\pgfsetlinewidth{1.505625pt}%
\definecolor{currentstroke}{rgb}{0.144759,0.519093,0.556572}%
\pgfsetstrokecolor{currentstroke}%
\pgfsetdash{}{0pt}%
\pgfpathmoveto{\pgfqpoint{0.766095in}{4.744940in}}%
\pgfpathlineto{\pgfqpoint{0.770067in}{4.746860in}}%
\pgfpathlineto{\pgfqpoint{0.801138in}{4.761633in}}%
\pgfpathlineto{\pgfqpoint{0.825381in}{4.773119in}}%
\pgfpathlineto{\pgfqpoint{0.836180in}{4.778153in}}%
\pgfpathlineto{\pgfqpoint{0.871223in}{4.794404in}}%
\pgfpathlineto{\pgfqpoint{0.882002in}{4.799378in}}%
\pgfpathlineto{\pgfqpoint{0.906265in}{4.810393in}}%
\pgfpathlineto{\pgfqpoint{0.939950in}{4.825638in}}%
\pgfpathlineto{\pgfqpoint{0.941308in}{4.826243in}}%
\pgfpathlineto{\pgfqpoint{0.976350in}{4.841744in}}%
\pgfpathlineto{\pgfqpoint{0.999372in}{4.851897in}}%
\pgfpathlineto{\pgfqpoint{1.011393in}{4.857113in}}%
\pgfpathlineto{\pgfqpoint{1.046435in}{4.872245in}}%
\pgfpathlineto{\pgfqpoint{1.060187in}{4.878157in}}%
\pgfpathlineto{\pgfqpoint{1.081478in}{4.887162in}}%
\pgfpathlineto{\pgfqpoint{1.116520in}{4.901934in}}%
\pgfpathlineto{\pgfqpoint{1.122442in}{4.904416in}}%
\pgfpathlineto{\pgfqpoint{1.151563in}{4.916425in}}%
\pgfpathlineto{\pgfqpoint{1.186190in}{4.930676in}}%
\pgfpathlineto{\pgfqpoint{1.186605in}{4.930844in}}%
\pgfpathlineto{\pgfqpoint{1.221648in}{4.944939in}}%
\pgfpathlineto{\pgfqpoint{1.251526in}{4.956935in}}%
\pgfpathlineto{\pgfqpoint{1.256691in}{4.958975in}}%
\pgfpathlineto{\pgfqpoint{1.291733in}{4.972737in}}%
\pgfpathlineto{\pgfqpoint{1.318422in}{4.983195in}}%
\pgfpathlineto{\pgfqpoint{1.326776in}{4.986415in}}%
\pgfpathlineto{\pgfqpoint{1.361818in}{4.999852in}}%
\pgfpathlineto{\pgfqpoint{1.386923in}{5.009454in}}%
\pgfpathlineto{\pgfqpoint{1.396861in}{5.013194in}}%
\pgfpathlineto{\pgfqpoint{1.431903in}{5.026314in}}%
\pgfpathlineto{\pgfqpoint{1.457069in}{5.035714in}}%
\pgfpathlineto{\pgfqpoint{1.466946in}{5.039342in}}%
\pgfpathlineto{\pgfqpoint{1.501988in}{5.052154in}}%
\pgfpathlineto{\pgfqpoint{1.528902in}{5.061973in}}%
\pgfpathlineto{\pgfqpoint{1.537031in}{5.064890in}}%
\pgfpathlineto{\pgfqpoint{1.572073in}{5.077401in}}%
\pgfpathlineto{\pgfqpoint{1.602459in}{5.088233in}}%
\pgfpathlineto{\pgfqpoint{1.607116in}{5.089866in}}%
\pgfpathlineto{\pgfqpoint{1.642158in}{5.102083in}}%
\pgfpathlineto{\pgfqpoint{1.677201in}{5.114291in}}%
\pgfpathlineto{\pgfqpoint{1.677783in}{5.114492in}}%
\pgfpathlineto{\pgfqpoint{1.712244in}{5.126226in}}%
\pgfpathlineto{\pgfqpoint{1.747286in}{5.138147in}}%
\pgfpathlineto{\pgfqpoint{1.754981in}{5.140752in}}%
\pgfpathlineto{\pgfqpoint{1.782329in}{5.149855in}}%
\pgfpathlineto{\pgfqpoint{1.817371in}{5.161498in}}%
\pgfpathlineto{\pgfqpoint{1.834027in}{5.167011in}}%
\pgfpathlineto{\pgfqpoint{1.852414in}{5.172997in}}%
\pgfpathlineto{\pgfqpoint{1.887456in}{5.184367in}}%
\pgfpathlineto{\pgfqpoint{1.914951in}{5.193271in}}%
\pgfpathlineto{\pgfqpoint{1.922499in}{5.195674in}}%
\pgfpathlineto{\pgfqpoint{1.957541in}{5.206778in}}%
\pgfpathlineto{\pgfqpoint{1.992584in}{5.217874in}}%
\pgfpathlineto{\pgfqpoint{1.997843in}{5.219530in}}%
\pgfpathlineto{\pgfqpoint{2.027626in}{5.228754in}}%
\pgfpathlineto{\pgfqpoint{2.062669in}{5.239590in}}%
\pgfpathlineto{\pgfqpoint{2.082778in}{5.245790in}}%
\pgfpathlineto{\pgfqpoint{2.097711in}{5.250317in}}%
\pgfpathlineto{\pgfqpoint{2.132754in}{5.260899in}}%
\pgfpathlineto{\pgfqpoint{2.167797in}{5.271475in}}%
\pgfpathlineto{\pgfqpoint{2.169709in}{5.272049in}}%
\pgfpathlineto{\pgfqpoint{2.202839in}{5.281822in}}%
\pgfpathlineto{\pgfqpoint{2.237882in}{5.292150in}}%
\pgfpathlineto{\pgfqpoint{2.258835in}{5.298308in}}%
\pgfpathlineto{\pgfqpoint{2.272924in}{5.302380in}}%
\pgfpathlineto{\pgfqpoint{2.307967in}{5.312466in}}%
\pgfpathlineto{\pgfqpoint{2.343009in}{5.322546in}}%
\pgfpathlineto{\pgfqpoint{2.350074in}{5.324568in}}%
\pgfpathlineto{\pgfqpoint{2.378052in}{5.332440in}}%
\pgfpathlineto{\pgfqpoint{2.413094in}{5.342284in}}%
\pgfpathlineto{\pgfqpoint{2.443548in}{5.350827in}}%
\pgfpathlineto{\pgfqpoint{2.448137in}{5.352093in}}%
\pgfpathlineto{\pgfqpoint{2.483179in}{5.361705in}}%
\pgfpathlineto{\pgfqpoint{2.518222in}{5.371312in}}%
\pgfpathlineto{\pgfqpoint{2.539346in}{5.377087in}}%
\pgfpathlineto{\pgfqpoint{2.553264in}{5.380827in}}%
\pgfpathlineto{\pgfqpoint{2.588307in}{5.390208in}}%
\pgfpathlineto{\pgfqpoint{2.623350in}{5.399583in}}%
\pgfpathlineto{\pgfqpoint{2.637469in}{5.403346in}}%
\pgfpathlineto{\pgfqpoint{2.658392in}{5.408828in}}%
\pgfpathlineto{\pgfqpoint{2.693435in}{5.417982in}}%
\pgfpathlineto{\pgfqpoint{2.728477in}{5.427131in}}%
\pgfpathlineto{\pgfqpoint{2.737998in}{5.429606in}}%
\pgfpathlineto{\pgfqpoint{2.763520in}{5.436125in}}%
\pgfpathlineto{\pgfqpoint{2.798562in}{5.445058in}}%
\pgfpathlineto{\pgfqpoint{2.833605in}{5.453987in}}%
\pgfpathlineto{\pgfqpoint{2.841015in}{5.455865in}}%
\pgfpathlineto{\pgfqpoint{2.857672in}{5.460016in}}%
\pgfusepath{stroke}%
\end{pgfscope}%
\begin{pgfscope}%
\pgfpathrectangle{\pgfqpoint{0.766095in}{0.571603in}}{\pgfqpoint{6.973465in}{5.225635in}}%
\pgfusepath{clip}%
\pgfsetbuttcap%
\pgfsetroundjoin%
\pgfsetlinewidth{1.505625pt}%
\definecolor{currentstroke}{rgb}{0.144759,0.519093,0.556572}%
\pgfsetstrokecolor{currentstroke}%
\pgfsetdash{}{0pt}%
\pgfpathmoveto{\pgfqpoint{3.160231in}{5.533340in}}%
\pgfpathlineto{\pgfqpoint{3.165759in}{5.534644in}}%
\pgfpathlineto{\pgfqpoint{3.184030in}{5.538877in}}%
\pgfpathlineto{\pgfqpoint{3.219073in}{5.546970in}}%
\pgfpathlineto{\pgfqpoint{3.254115in}{5.555060in}}%
\pgfpathlineto{\pgfqpoint{3.279480in}{5.560903in}}%
\pgfpathlineto{\pgfqpoint{3.289158in}{5.563093in}}%
\pgfpathlineto{\pgfqpoint{3.324200in}{5.570987in}}%
\pgfpathlineto{\pgfqpoint{3.359243in}{5.578876in}}%
\pgfpathlineto{\pgfqpoint{3.394285in}{5.586763in}}%
\pgfpathlineto{\pgfqpoint{3.396074in}{5.587163in}}%
\pgfpathlineto{\pgfqpoint{3.429328in}{5.594469in}}%
\pgfpathlineto{\pgfqpoint{3.464371in}{5.602163in}}%
\pgfpathlineto{\pgfqpoint{3.499413in}{5.609853in}}%
\pgfpathlineto{\pgfqpoint{3.515737in}{5.613422in}}%
\pgfpathlineto{\pgfqpoint{3.534456in}{5.617442in}}%
\pgfpathlineto{\pgfqpoint{3.569498in}{5.624943in}}%
\pgfpathlineto{\pgfqpoint{3.604541in}{5.632440in}}%
\pgfpathlineto{\pgfqpoint{3.638411in}{5.639682in}}%
\pgfpathlineto{\pgfqpoint{3.639583in}{5.639928in}}%
\pgfpathlineto{\pgfqpoint{3.674626in}{5.647240in}}%
\pgfpathlineto{\pgfqpoint{3.709668in}{5.654547in}}%
\pgfpathlineto{\pgfqpoint{3.744711in}{5.661851in}}%
\pgfpathlineto{\pgfqpoint{3.764395in}{5.665941in}}%
\pgfpathlineto{\pgfqpoint{3.779753in}{5.669075in}}%
\pgfpathlineto{\pgfqpoint{3.814796in}{5.676196in}}%
\pgfpathlineto{\pgfqpoint{3.849838in}{5.683313in}}%
\pgfpathlineto{\pgfqpoint{3.884881in}{5.690427in}}%
\pgfpathlineto{\pgfqpoint{3.893663in}{5.692201in}}%
\pgfpathlineto{\pgfqpoint{3.919924in}{5.697408in}}%
\pgfpathlineto{\pgfqpoint{3.954966in}{5.704341in}}%
\pgfpathlineto{\pgfqpoint{3.990009in}{5.711271in}}%
\pgfpathlineto{\pgfqpoint{4.025051in}{5.718196in}}%
\pgfpathlineto{\pgfqpoint{4.026394in}{5.718460in}}%
\pgfpathlineto{\pgfqpoint{4.060094in}{5.724956in}}%
\pgfpathlineto{\pgfqpoint{4.095136in}{5.731704in}}%
\pgfpathlineto{\pgfqpoint{4.130179in}{5.738448in}}%
\pgfpathlineto{\pgfqpoint{4.162801in}{5.744720in}}%
\pgfpathlineto{\pgfqpoint{4.165221in}{5.745176in}}%
\pgfpathlineto{\pgfqpoint{4.200264in}{5.751746in}}%
\pgfpathlineto{\pgfqpoint{4.235306in}{5.758311in}}%
\pgfpathlineto{\pgfqpoint{4.270349in}{5.764871in}}%
\pgfpathlineto{\pgfqpoint{4.303011in}{5.770979in}}%
\pgfpathlineto{\pgfqpoint{4.305391in}{5.771416in}}%
\pgfpathlineto{\pgfqpoint{4.340434in}{5.777805in}}%
\pgfpathlineto{\pgfqpoint{4.375477in}{5.784188in}}%
\pgfpathlineto{\pgfqpoint{4.410519in}{5.790566in}}%
\pgfpathlineto{\pgfqpoint{4.445562in}{5.796939in}}%
\pgfpathlineto{\pgfqpoint{4.447217in}{5.797238in}}%
\pgfusepath{stroke}%
\end{pgfscope}%
\begin{pgfscope}%
\pgfpathrectangle{\pgfqpoint{0.766095in}{0.571603in}}{\pgfqpoint{6.973465in}{5.225635in}}%
\pgfusepath{clip}%
\pgfsetbuttcap%
\pgfsetroundjoin%
\pgfsetlinewidth{1.505625pt}%
\definecolor{currentstroke}{rgb}{0.137770,0.537492,0.554906}%
\pgfsetstrokecolor{currentstroke}%
\pgfsetdash{}{0pt}%
\pgfpathmoveto{\pgfqpoint{0.766095in}{4.802388in}}%
\pgfpathlineto{\pgfqpoint{0.801138in}{4.818638in}}%
\pgfpathlineto{\pgfqpoint{0.816297in}{4.825638in}}%
\pgfpathlineto{\pgfqpoint{0.836180in}{4.834671in}}%
\pgfpathlineto{\pgfqpoint{0.871223in}{4.850535in}}%
\pgfpathlineto{\pgfqpoint{0.874251in}{4.851897in}}%
\pgfpathlineto{\pgfqpoint{0.906265in}{4.866069in}}%
\pgfpathlineto{\pgfqpoint{0.933648in}{4.878157in}}%
\pgfpathlineto{\pgfqpoint{0.941308in}{4.881484in}}%
\pgfpathlineto{\pgfqpoint{0.976350in}{4.896621in}}%
\pgfpathlineto{\pgfqpoint{0.994463in}{4.904416in}}%
\pgfpathlineto{\pgfqpoint{1.011393in}{4.911585in}}%
\pgfpathlineto{\pgfqpoint{1.046435in}{4.926365in}}%
\pgfpathlineto{\pgfqpoint{1.056705in}{4.930676in}}%
\pgfpathlineto{\pgfqpoint{1.081478in}{4.940905in}}%
\pgfpathlineto{\pgfqpoint{1.116520in}{4.955338in}}%
\pgfpathlineto{\pgfqpoint{1.120423in}{4.956935in}}%
\pgfpathlineto{\pgfqpoint{1.151563in}{4.969479in}}%
\pgfpathlineto{\pgfqpoint{1.185671in}{4.983195in}}%
\pgfpathlineto{\pgfqpoint{1.186605in}{4.983564in}}%
\pgfpathlineto{\pgfqpoint{1.221648in}{4.997340in}}%
\pgfpathlineto{\pgfqpoint{1.252512in}{5.009454in}}%
\pgfpathlineto{\pgfqpoint{1.256691in}{5.011068in}}%
\pgfpathlineto{\pgfqpoint{1.291733in}{5.024521in}}%
\pgfpathlineto{\pgfqpoint{1.320940in}{5.035714in}}%
\pgfpathlineto{\pgfqpoint{1.326776in}{5.037914in}}%
\pgfpathlineto{\pgfqpoint{1.361818in}{5.051052in}}%
\pgfpathlineto{\pgfqpoint{1.390997in}{5.061973in}}%
\pgfpathlineto{\pgfqpoint{1.396861in}{5.064132in}}%
\pgfpathlineto{\pgfqpoint{1.431903in}{5.076964in}}%
\pgfpathlineto{\pgfqpoint{1.462721in}{5.088233in}}%
\pgfpathlineto{\pgfqpoint{1.466946in}{5.089752in}}%
\pgfpathlineto{\pgfqpoint{1.501988in}{5.102285in}}%
\pgfpathlineto{\pgfqpoint{1.536150in}{5.114492in}}%
\pgfpathlineto{\pgfqpoint{1.537031in}{5.114802in}}%
\pgfpathlineto{\pgfqpoint{1.572073in}{5.127043in}}%
\pgfpathlineto{\pgfqpoint{1.607116in}{5.139275in}}%
\pgfpathlineto{\pgfqpoint{1.611369in}{5.140752in}}%
\pgfpathlineto{\pgfqpoint{1.642158in}{5.151265in}}%
\pgfpathlineto{\pgfqpoint{1.677201in}{5.163213in}}%
\pgfpathlineto{\pgfqpoint{1.688392in}{5.167011in}}%
\pgfpathlineto{\pgfqpoint{1.712244in}{5.174975in}}%
\pgfpathlineto{\pgfqpoint{1.747286in}{5.186646in}}%
\pgfpathlineto{\pgfqpoint{1.767239in}{5.193271in}}%
\pgfpathlineto{\pgfqpoint{1.782329in}{5.198199in}}%
\pgfpathlineto{\pgfqpoint{1.817371in}{5.209599in}}%
\pgfpathlineto{\pgfqpoint{1.847939in}{5.219530in}}%
\pgfpathlineto{\pgfqpoint{1.852414in}{5.220960in}}%
\pgfpathlineto{\pgfqpoint{1.887456in}{5.232096in}}%
\pgfpathlineto{\pgfqpoint{1.922499in}{5.243225in}}%
\pgfpathlineto{\pgfqpoint{1.930613in}{5.245790in}}%
\pgfpathlineto{\pgfqpoint{1.957541in}{5.254159in}}%
\pgfpathlineto{\pgfqpoint{1.992584in}{5.265031in}}%
\pgfpathlineto{\pgfqpoint{2.015264in}{5.272049in}}%
\pgfpathlineto{\pgfqpoint{2.027626in}{5.275811in}}%
\pgfpathlineto{\pgfqpoint{2.062669in}{5.286431in}}%
\pgfpathlineto{\pgfqpoint{2.097711in}{5.297044in}}%
\pgfpathlineto{\pgfqpoint{2.101908in}{5.298308in}}%
\pgfpathlineto{\pgfqpoint{2.132754in}{5.307446in}}%
\pgfpathlineto{\pgfqpoint{2.167797in}{5.317814in}}%
\pgfpathlineto{\pgfqpoint{2.190681in}{5.324568in}}%
\pgfpathlineto{\pgfqpoint{2.202839in}{5.328097in}}%
\pgfpathlineto{\pgfqpoint{2.237882in}{5.338224in}}%
\pgfpathlineto{\pgfqpoint{2.272924in}{5.348347in}}%
\pgfpathlineto{\pgfqpoint{2.281552in}{5.350827in}}%
\pgfpathlineto{\pgfqpoint{2.307967in}{5.358295in}}%
\pgfpathlineto{\pgfqpoint{2.343009in}{5.368183in}}%
\pgfpathlineto{\pgfqpoint{2.374598in}{5.377087in}}%
\pgfpathlineto{\pgfqpoint{2.378052in}{5.378044in}}%
\pgfpathlineto{\pgfqpoint{2.413094in}{5.387703in}}%
\pgfpathlineto{\pgfqpoint{2.438001in}{5.394564in}}%
\pgfusepath{stroke}%
\end{pgfscope}%
\begin{pgfscope}%
\pgfpathrectangle{\pgfqpoint{0.766095in}{0.571603in}}{\pgfqpoint{6.973465in}{5.225635in}}%
\pgfusepath{clip}%
\pgfsetbuttcap%
\pgfsetroundjoin%
\pgfsetlinewidth{1.505625pt}%
\definecolor{currentstroke}{rgb}{0.137770,0.537492,0.554906}%
\pgfsetstrokecolor{currentstroke}%
\pgfsetdash{}{0pt}%
\pgfpathmoveto{\pgfqpoint{2.738974in}{5.474209in}}%
\pgfpathlineto{\pgfqpoint{2.763520in}{5.480502in}}%
\pgfpathlineto{\pgfqpoint{2.769881in}{5.482125in}}%
\pgfpathlineto{\pgfqpoint{2.798562in}{5.489317in}}%
\pgfpathlineto{\pgfqpoint{2.833605in}{5.498091in}}%
\pgfpathlineto{\pgfqpoint{2.868647in}{5.506862in}}%
\pgfpathlineto{\pgfqpoint{2.874761in}{5.508384in}}%
\pgfpathlineto{\pgfqpoint{2.903690in}{5.515465in}}%
\pgfpathlineto{\pgfqpoint{2.938732in}{5.524030in}}%
\pgfpathlineto{\pgfqpoint{2.973775in}{5.532592in}}%
\pgfpathlineto{\pgfqpoint{2.982210in}{5.534644in}}%
\pgfpathlineto{\pgfqpoint{3.008818in}{5.541003in}}%
\pgfpathlineto{\pgfqpoint{3.043860in}{5.549364in}}%
\pgfpathlineto{\pgfqpoint{3.078903in}{5.557721in}}%
\pgfpathlineto{\pgfqpoint{3.092295in}{5.560903in}}%
\pgfpathlineto{\pgfqpoint{3.113945in}{5.565958in}}%
\pgfpathlineto{\pgfqpoint{3.148988in}{5.574118in}}%
\pgfpathlineto{\pgfqpoint{3.184030in}{5.582275in}}%
\pgfpathlineto{\pgfqpoint{3.205082in}{5.587163in}}%
\pgfpathlineto{\pgfqpoint{3.219073in}{5.590354in}}%
\pgfpathlineto{\pgfqpoint{3.254115in}{5.598318in}}%
\pgfpathlineto{\pgfqpoint{3.289158in}{5.606279in}}%
\pgfpathlineto{\pgfqpoint{3.320630in}{5.613422in}}%
\pgfpathlineto{\pgfqpoint{3.324200in}{5.614218in}}%
\pgfpathlineto{\pgfqpoint{3.359243in}{5.621989in}}%
\pgfpathlineto{\pgfqpoint{3.394285in}{5.629757in}}%
\pgfpathlineto{\pgfqpoint{3.429328in}{5.637523in}}%
\pgfpathlineto{\pgfqpoint{3.439113in}{5.639682in}}%
\pgfpathlineto{\pgfqpoint{3.464371in}{5.645155in}}%
\pgfpathlineto{\pgfqpoint{3.499413in}{5.652734in}}%
\pgfpathlineto{\pgfqpoint{3.534456in}{5.660310in}}%
\pgfpathlineto{\pgfqpoint{3.560547in}{5.665941in}}%
\pgfpathlineto{\pgfqpoint{3.569498in}{5.667838in}}%
\pgfpathlineto{\pgfqpoint{3.604541in}{5.675232in}}%
\pgfpathlineto{\pgfqpoint{3.639583in}{5.682622in}}%
\pgfpathlineto{\pgfqpoint{3.674626in}{5.690010in}}%
\pgfpathlineto{\pgfqpoint{3.685065in}{5.692201in}}%
\pgfpathlineto{\pgfqpoint{3.709668in}{5.697272in}}%
\pgfpathlineto{\pgfqpoint{3.744711in}{5.704480in}}%
\pgfpathlineto{\pgfqpoint{3.779753in}{5.711684in}}%
\pgfpathlineto{\pgfqpoint{3.812732in}{5.718460in}}%
\pgfpathlineto{\pgfqpoint{3.814796in}{5.718876in}}%
\pgfpathlineto{\pgfqpoint{3.849838in}{5.725904in}}%
\pgfpathlineto{\pgfqpoint{3.884881in}{5.732929in}}%
\pgfpathlineto{\pgfqpoint{3.919924in}{5.739951in}}%
\pgfpathlineto{\pgfqpoint{3.943774in}{5.744720in}}%
\pgfpathlineto{\pgfqpoint{3.954966in}{5.746917in}}%
\pgfpathlineto{\pgfqpoint{3.990009in}{5.753764in}}%
\pgfpathlineto{\pgfqpoint{4.025051in}{5.760609in}}%
\pgfpathlineto{\pgfqpoint{4.060094in}{5.767451in}}%
\pgfpathlineto{\pgfqpoint{4.078225in}{5.770979in}}%
\pgfpathlineto{\pgfqpoint{4.095136in}{5.774210in}}%
\pgfpathlineto{\pgfqpoint{4.130179in}{5.780880in}}%
\pgfpathlineto{\pgfqpoint{4.165221in}{5.787546in}}%
\pgfpathlineto{\pgfqpoint{4.200264in}{5.794210in}}%
\pgfpathlineto{\pgfqpoint{4.216250in}{5.797238in}}%
\pgfusepath{stroke}%
\end{pgfscope}%
\begin{pgfscope}%
\pgfpathrectangle{\pgfqpoint{0.766095in}{0.571603in}}{\pgfqpoint{6.973465in}{5.225635in}}%
\pgfusepath{clip}%
\pgfsetbuttcap%
\pgfsetroundjoin%
\pgfsetlinewidth{1.505625pt}%
\definecolor{currentstroke}{rgb}{0.131172,0.555899,0.552459}%
\pgfsetstrokecolor{currentstroke}%
\pgfsetdash{}{0pt}%
\pgfpathmoveto{\pgfqpoint{0.766095in}{4.857215in}}%
\pgfpathlineto{\pgfqpoint{0.801138in}{4.873068in}}%
\pgfpathlineto{\pgfqpoint{0.812441in}{4.878157in}}%
\pgfpathlineto{\pgfqpoint{0.836180in}{4.888673in}}%
\pgfpathlineto{\pgfqpoint{0.871223in}{4.904152in}}%
\pgfpathlineto{\pgfqpoint{0.871824in}{4.904416in}}%
\pgfpathlineto{\pgfqpoint{0.906265in}{4.919290in}}%
\pgfpathlineto{\pgfqpoint{0.932698in}{4.930676in}}%
\pgfpathlineto{\pgfqpoint{0.941308in}{4.934325in}}%
\pgfpathlineto{\pgfqpoint{0.976350in}{4.949101in}}%
\pgfpathlineto{\pgfqpoint{0.994995in}{4.956935in}}%
\pgfpathlineto{\pgfqpoint{1.011393in}{4.963714in}}%
\pgfpathlineto{\pgfqpoint{1.046435in}{4.978144in}}%
\pgfpathlineto{\pgfqpoint{1.058755in}{4.983195in}}%
\pgfpathlineto{\pgfqpoint{1.081478in}{4.992360in}}%
\pgfpathlineto{\pgfqpoint{1.116520in}{5.006453in}}%
\pgfpathlineto{\pgfqpoint{1.124022in}{5.009454in}}%
\pgfpathlineto{\pgfqpoint{1.151563in}{5.020296in}}%
\pgfpathlineto{\pgfqpoint{1.186605in}{5.034061in}}%
\pgfpathlineto{\pgfqpoint{1.190838in}{5.035714in}}%
\pgfpathlineto{\pgfqpoint{1.221648in}{5.047554in}}%
\pgfpathlineto{\pgfqpoint{1.256691in}{5.060999in}}%
\pgfpathlineto{\pgfqpoint{1.259245in}{5.061973in}}%
\pgfpathlineto{\pgfqpoint{1.291733in}{5.074165in}}%
\pgfpathlineto{\pgfqpoint{1.326776in}{5.087299in}}%
\pgfpathlineto{\pgfqpoint{1.329283in}{5.088233in}}%
\pgfpathlineto{\pgfqpoint{1.361818in}{5.100159in}}%
\pgfpathlineto{\pgfqpoint{1.396861in}{5.112989in}}%
\pgfpathlineto{\pgfqpoint{1.400990in}{5.114492in}}%
\pgfpathlineto{\pgfqpoint{1.431903in}{5.125565in}}%
\pgfpathlineto{\pgfqpoint{1.466946in}{5.138098in}}%
\pgfpathlineto{\pgfqpoint{1.474403in}{5.140752in}}%
\pgfpathlineto{\pgfqpoint{1.501988in}{5.150409in}}%
\pgfpathlineto{\pgfqpoint{1.537031in}{5.162653in}}%
\pgfpathlineto{\pgfqpoint{1.549555in}{5.167011in}}%
\pgfpathlineto{\pgfqpoint{1.572073in}{5.174719in}}%
\pgfpathlineto{\pgfqpoint{1.607116in}{5.186681in}}%
\pgfpathlineto{\pgfqpoint{1.626480in}{5.193271in}}%
\pgfpathlineto{\pgfqpoint{1.642158in}{5.198519in}}%
\pgfpathlineto{\pgfqpoint{1.677201in}{5.210206in}}%
\pgfpathlineto{\pgfqpoint{1.705206in}{5.219530in}}%
\pgfpathlineto{\pgfqpoint{1.712244in}{5.221835in}}%
\pgfpathlineto{\pgfqpoint{1.747286in}{5.233254in}}%
\pgfpathlineto{\pgfqpoint{1.782329in}{5.244665in}}%
\pgfpathlineto{\pgfqpoint{1.785802in}{5.245790in}}%
\pgfpathlineto{\pgfqpoint{1.817371in}{5.255846in}}%
\pgfpathlineto{\pgfqpoint{1.852414in}{5.266995in}}%
\pgfpathlineto{\pgfqpoint{1.868354in}{5.272049in}}%
\pgfpathlineto{\pgfqpoint{1.887456in}{5.278006in}}%
\pgfpathlineto{\pgfqpoint{1.922499in}{5.288900in}}%
\pgfpathlineto{\pgfqpoint{1.952804in}{5.298308in}}%
\pgfpathlineto{\pgfqpoint{1.957541in}{5.299755in}}%
\pgfpathlineto{\pgfqpoint{1.992584in}{5.310399in}}%
\pgfpathlineto{\pgfqpoint{2.018509in}{5.318270in}}%
\pgfusepath{stroke}%
\end{pgfscope}%
\begin{pgfscope}%
\pgfpathrectangle{\pgfqpoint{0.766095in}{0.571603in}}{\pgfqpoint{6.973465in}{5.225635in}}%
\pgfusepath{clip}%
\pgfsetbuttcap%
\pgfsetroundjoin%
\pgfsetlinewidth{1.505625pt}%
\definecolor{currentstroke}{rgb}{0.131172,0.555899,0.552459}%
\pgfsetstrokecolor{currentstroke}%
\pgfsetdash{}{0pt}%
\pgfpathmoveto{\pgfqpoint{2.317512in}{5.405090in}}%
\pgfpathlineto{\pgfqpoint{2.343009in}{5.412145in}}%
\pgfpathlineto{\pgfqpoint{2.378052in}{5.421833in}}%
\pgfpathlineto{\pgfqpoint{2.406210in}{5.429606in}}%
\pgfpathlineto{\pgfqpoint{2.413094in}{5.431475in}}%
\pgfpathlineto{\pgfqpoint{2.448137in}{5.440940in}}%
\pgfpathlineto{\pgfqpoint{2.483179in}{5.450402in}}%
\pgfpathlineto{\pgfqpoint{2.503468in}{5.455865in}}%
\pgfpathlineto{\pgfqpoint{2.518222in}{5.459772in}}%
\pgfpathlineto{\pgfqpoint{2.553264in}{5.469016in}}%
\pgfpathlineto{\pgfqpoint{2.588307in}{5.478258in}}%
\pgfpathlineto{\pgfqpoint{2.603023in}{5.482125in}}%
\pgfpathlineto{\pgfqpoint{2.623350in}{5.487376in}}%
\pgfpathlineto{\pgfqpoint{2.658392in}{5.496405in}}%
\pgfpathlineto{\pgfqpoint{2.693435in}{5.505431in}}%
\pgfpathlineto{\pgfqpoint{2.704948in}{5.508384in}}%
\pgfpathlineto{\pgfqpoint{2.728477in}{5.514318in}}%
\pgfpathlineto{\pgfqpoint{2.763520in}{5.523136in}}%
\pgfpathlineto{\pgfqpoint{2.798562in}{5.531951in}}%
\pgfpathlineto{\pgfqpoint{2.809311in}{5.534644in}}%
\pgfpathlineto{\pgfqpoint{2.833605in}{5.540626in}}%
\pgfpathlineto{\pgfqpoint{2.868647in}{5.549238in}}%
\pgfpathlineto{\pgfqpoint{2.903690in}{5.557847in}}%
\pgfpathlineto{\pgfqpoint{2.916178in}{5.560903in}}%
\pgfpathlineto{\pgfqpoint{2.938732in}{5.566329in}}%
\pgfpathlineto{\pgfqpoint{2.973775in}{5.574738in}}%
\pgfpathlineto{\pgfqpoint{3.008818in}{5.583146in}}%
\pgfpathlineto{\pgfqpoint{3.025612in}{5.587163in}}%
\pgfpathlineto{\pgfqpoint{3.043860in}{5.591452in}}%
\pgfpathlineto{\pgfqpoint{3.078903in}{5.599664in}}%
\pgfpathlineto{\pgfqpoint{3.113945in}{5.607874in}}%
\pgfpathlineto{\pgfqpoint{3.137673in}{5.613422in}}%
\pgfpathlineto{\pgfqpoint{3.148988in}{5.616022in}}%
\pgfpathlineto{\pgfqpoint{3.184030in}{5.624041in}}%
\pgfpathlineto{\pgfqpoint{3.219073in}{5.632057in}}%
\pgfpathlineto{\pgfqpoint{3.252417in}{5.639682in}}%
\pgfpathlineto{\pgfqpoint{3.254115in}{5.640063in}}%
\pgfpathlineto{\pgfqpoint{3.289158in}{5.647892in}}%
\pgfpathlineto{\pgfqpoint{3.324200in}{5.655719in}}%
\pgfpathlineto{\pgfqpoint{3.359243in}{5.663545in}}%
\pgfpathlineto{\pgfqpoint{3.370020in}{5.665941in}}%
\pgfpathlineto{\pgfqpoint{3.394285in}{5.671243in}}%
\pgfpathlineto{\pgfqpoint{3.429328in}{5.678884in}}%
\pgfpathlineto{\pgfqpoint{3.464371in}{5.686523in}}%
\pgfpathlineto{\pgfqpoint{3.490459in}{5.692201in}}%
\pgfpathlineto{\pgfqpoint{3.499413in}{5.694115in}}%
\pgfpathlineto{\pgfqpoint{3.534456in}{5.701573in}}%
\pgfpathlineto{\pgfqpoint{3.569498in}{5.709030in}}%
\pgfpathlineto{\pgfqpoint{3.604541in}{5.716484in}}%
\pgfpathlineto{\pgfqpoint{3.613869in}{5.718460in}}%
\pgfpathlineto{\pgfqpoint{3.639583in}{5.723810in}}%
\pgfpathlineto{\pgfqpoint{3.674626in}{5.731087in}}%
\pgfpathlineto{\pgfqpoint{3.709668in}{5.738362in}}%
\pgfpathlineto{\pgfqpoint{3.740322in}{5.744720in}}%
\pgfpathlineto{\pgfqpoint{3.744711in}{5.745614in}}%
\pgfpathlineto{\pgfqpoint{3.779753in}{5.752714in}}%
\pgfpathlineto{\pgfqpoint{3.814796in}{5.759813in}}%
\pgfpathlineto{\pgfqpoint{3.849838in}{5.766910in}}%
\pgfpathlineto{\pgfqpoint{3.869986in}{5.770979in}}%
\pgfpathlineto{\pgfqpoint{3.884881in}{5.773933in}}%
\pgfpathlineto{\pgfqpoint{3.919924in}{5.780859in}}%
\pgfpathlineto{\pgfqpoint{3.954966in}{5.787782in}}%
\pgfpathlineto{\pgfqpoint{3.990009in}{5.794703in}}%
\pgfpathlineto{\pgfqpoint{4.002899in}{5.797238in}}%
\pgfusepath{stroke}%
\end{pgfscope}%
\begin{pgfscope}%
\pgfpathrectangle{\pgfqpoint{0.766095in}{0.571603in}}{\pgfqpoint{6.973465in}{5.225635in}}%
\pgfusepath{clip}%
\pgfsetbuttcap%
\pgfsetroundjoin%
\pgfsetlinewidth{1.505625pt}%
\definecolor{currentstroke}{rgb}{0.125394,0.574318,0.549086}%
\pgfsetstrokecolor{currentstroke}%
\pgfsetdash{}{0pt}%
\pgfpathmoveto{\pgfqpoint{0.766095in}{4.909693in}}%
\pgfpathlineto{\pgfqpoint{0.801138in}{4.925153in}}%
\pgfpathlineto{\pgfqpoint{0.813713in}{4.930676in}}%
\pgfpathlineto{\pgfqpoint{0.836180in}{4.940385in}}%
\pgfpathlineto{\pgfqpoint{0.871223in}{4.955484in}}%
\pgfpathlineto{\pgfqpoint{0.874612in}{4.956935in}}%
\pgfpathlineto{\pgfqpoint{0.906265in}{4.970276in}}%
\pgfpathlineto{\pgfqpoint{0.936985in}{4.983195in}}%
\pgfpathlineto{\pgfqpoint{0.941308in}{4.984984in}}%
\pgfpathlineto{\pgfqpoint{0.976350in}{4.999402in}}%
\pgfpathlineto{\pgfqpoint{1.000847in}{5.009454in}}%
\pgfpathlineto{\pgfqpoint{1.011393in}{5.013712in}}%
\pgfpathlineto{\pgfqpoint{1.046435in}{5.027795in}}%
\pgfpathlineto{\pgfqpoint{1.066204in}{5.035714in}}%
\pgfpathlineto{\pgfqpoint{1.081478in}{5.041734in}}%
\pgfpathlineto{\pgfqpoint{1.116520in}{5.055491in}}%
\pgfpathlineto{\pgfqpoint{1.133096in}{5.061973in}}%
\pgfpathlineto{\pgfqpoint{1.151563in}{5.069080in}}%
\pgfpathlineto{\pgfqpoint{1.186605in}{5.082519in}}%
\pgfpathlineto{\pgfqpoint{1.201563in}{5.088233in}}%
\pgfpathlineto{\pgfqpoint{1.221648in}{5.095782in}}%
\pgfpathlineto{\pgfqpoint{1.256691in}{5.108911in}}%
\pgfpathlineto{\pgfqpoint{1.271646in}{5.114492in}}%
\pgfpathlineto{\pgfqpoint{1.291733in}{5.121868in}}%
\pgfpathlineto{\pgfqpoint{1.326776in}{5.134695in}}%
\pgfpathlineto{\pgfqpoint{1.343380in}{5.140752in}}%
\pgfpathlineto{\pgfqpoint{1.361818in}{5.147368in}}%
\pgfpathlineto{\pgfqpoint{1.396861in}{5.159901in}}%
\pgfpathlineto{\pgfqpoint{1.416802in}{5.167011in}}%
\pgfpathlineto{\pgfqpoint{1.431903in}{5.172309in}}%
\pgfpathlineto{\pgfqpoint{1.466946in}{5.184555in}}%
\pgfpathlineto{\pgfqpoint{1.491944in}{5.193271in}}%
\pgfpathlineto{\pgfqpoint{1.501988in}{5.196716in}}%
\pgfpathlineto{\pgfqpoint{1.537031in}{5.208682in}}%
\pgfpathlineto{\pgfqpoint{1.568838in}{5.219530in}}%
\pgfpathlineto{\pgfqpoint{1.572073in}{5.220616in}}%
\pgfpathlineto{\pgfqpoint{1.599321in}{5.229708in}}%
\pgfusepath{stroke}%
\end{pgfscope}%
\begin{pgfscope}%
\pgfpathrectangle{\pgfqpoint{0.766095in}{0.571603in}}{\pgfqpoint{6.973465in}{5.225635in}}%
\pgfusepath{clip}%
\pgfsetbuttcap%
\pgfsetroundjoin%
\pgfsetlinewidth{1.505625pt}%
\definecolor{currentstroke}{rgb}{0.125394,0.574318,0.549086}%
\pgfsetstrokecolor{currentstroke}%
\pgfsetdash{}{0pt}%
\pgfpathmoveto{\pgfqpoint{1.895820in}{5.324809in}}%
\pgfpathlineto{\pgfqpoint{1.922499in}{5.332933in}}%
\pgfpathlineto{\pgfqpoint{1.957541in}{5.343585in}}%
\pgfpathlineto{\pgfqpoint{1.981422in}{5.350827in}}%
\pgfpathlineto{\pgfqpoint{1.992584in}{5.354157in}}%
\pgfpathlineto{\pgfqpoint{2.027626in}{5.364568in}}%
\pgfpathlineto{\pgfqpoint{2.062669in}{5.374973in}}%
\pgfpathlineto{\pgfqpoint{2.069822in}{5.377087in}}%
\pgfpathlineto{\pgfqpoint{2.097711in}{5.385193in}}%
\pgfpathlineto{\pgfqpoint{2.132754in}{5.395363in}}%
\pgfpathlineto{\pgfqpoint{2.160308in}{5.403346in}}%
\pgfpathlineto{\pgfqpoint{2.167797in}{5.405480in}}%
\pgfpathlineto{\pgfqpoint{2.202839in}{5.415420in}}%
\pgfpathlineto{\pgfqpoint{2.237882in}{5.425354in}}%
\pgfpathlineto{\pgfqpoint{2.252930in}{5.429606in}}%
\pgfpathlineto{\pgfqpoint{2.272924in}{5.435161in}}%
\pgfpathlineto{\pgfqpoint{2.307967in}{5.444871in}}%
\pgfpathlineto{\pgfqpoint{2.343009in}{5.454577in}}%
\pgfpathlineto{\pgfqpoint{2.347684in}{5.455865in}}%
\pgfpathlineto{\pgfqpoint{2.378052in}{5.464095in}}%
\pgfpathlineto{\pgfqpoint{2.413094in}{5.473581in}}%
\pgfpathlineto{\pgfqpoint{2.444682in}{5.482125in}}%
\pgfpathlineto{\pgfqpoint{2.448137in}{5.483044in}}%
\pgfpathlineto{\pgfqpoint{2.483179in}{5.492315in}}%
\pgfpathlineto{\pgfqpoint{2.518222in}{5.501583in}}%
\pgfpathlineto{\pgfqpoint{2.543983in}{5.508384in}}%
\pgfpathlineto{\pgfqpoint{2.553264in}{5.510794in}}%
\pgfpathlineto{\pgfqpoint{2.588307in}{5.519852in}}%
\pgfpathlineto{\pgfqpoint{2.623350in}{5.528907in}}%
\pgfpathlineto{\pgfqpoint{2.645603in}{5.534644in}}%
\pgfpathlineto{\pgfqpoint{2.658392in}{5.537886in}}%
\pgfpathlineto{\pgfqpoint{2.693435in}{5.546735in}}%
\pgfpathlineto{\pgfqpoint{2.728477in}{5.555582in}}%
\pgfpathlineto{\pgfqpoint{2.749606in}{5.560903in}}%
\pgfpathlineto{\pgfqpoint{2.763520in}{5.564348in}}%
\pgfpathlineto{\pgfqpoint{2.798562in}{5.572994in}}%
\pgfpathlineto{\pgfqpoint{2.833605in}{5.581637in}}%
\pgfpathlineto{\pgfqpoint{2.856055in}{5.587163in}}%
\pgfpathlineto{\pgfqpoint{2.868647in}{5.590209in}}%
\pgfpathlineto{\pgfqpoint{2.903690in}{5.598656in}}%
\pgfpathlineto{\pgfqpoint{2.938732in}{5.607100in}}%
\pgfpathlineto{\pgfqpoint{2.965008in}{5.613422in}}%
\pgfpathlineto{\pgfqpoint{2.973775in}{5.615495in}}%
\pgfpathlineto{\pgfqpoint{3.008818in}{5.623747in}}%
\pgfpathlineto{\pgfqpoint{3.043860in}{5.631997in}}%
\pgfpathlineto{\pgfqpoint{3.076522in}{5.639682in}}%
\pgfpathlineto{\pgfqpoint{3.078903in}{5.640232in}}%
\pgfpathlineto{\pgfqpoint{3.113945in}{5.648293in}}%
\pgfpathlineto{\pgfqpoint{3.148988in}{5.656352in}}%
\pgfpathlineto{\pgfqpoint{3.184030in}{5.664410in}}%
\pgfpathlineto{\pgfqpoint{3.190721in}{5.665941in}}%
\pgfpathlineto{\pgfqpoint{3.219073in}{5.672318in}}%
\pgfpathlineto{\pgfqpoint{3.254115in}{5.680190in}}%
\pgfpathlineto{\pgfqpoint{3.289158in}{5.688061in}}%
\pgfpathlineto{\pgfqpoint{3.307638in}{5.692201in}}%
\pgfpathlineto{\pgfqpoint{3.324200in}{5.695846in}}%
\pgfpathlineto{\pgfqpoint{3.359243in}{5.703535in}}%
\pgfpathlineto{\pgfqpoint{3.394285in}{5.711223in}}%
\pgfpathlineto{\pgfqpoint{3.427290in}{5.718460in}}%
\pgfpathlineto{\pgfqpoint{3.429328in}{5.718899in}}%
\pgfpathlineto{\pgfqpoint{3.464371in}{5.726409in}}%
\pgfpathlineto{\pgfqpoint{3.499413in}{5.733917in}}%
\pgfpathlineto{\pgfqpoint{3.534456in}{5.741424in}}%
\pgfpathlineto{\pgfqpoint{3.549890in}{5.744720in}}%
\pgfpathlineto{\pgfqpoint{3.569498in}{5.748833in}}%
\pgfpathlineto{\pgfqpoint{3.604541in}{5.756165in}}%
\pgfpathlineto{\pgfqpoint{3.639583in}{5.763495in}}%
\pgfpathlineto{\pgfqpoint{3.674626in}{5.770825in}}%
\pgfpathlineto{\pgfqpoint{3.675368in}{5.770979in}}%
\pgfpathlineto{\pgfqpoint{3.709668in}{5.777987in}}%
\pgfpathlineto{\pgfqpoint{3.744711in}{5.785144in}}%
\pgfpathlineto{\pgfqpoint{3.779753in}{5.792300in}}%
\pgfpathlineto{\pgfqpoint{3.803980in}{5.797238in}}%
\pgfusepath{stroke}%
\end{pgfscope}%
\begin{pgfscope}%
\pgfpathrectangle{\pgfqpoint{0.766095in}{0.571603in}}{\pgfqpoint{6.973465in}{5.225635in}}%
\pgfusepath{clip}%
\pgfsetbuttcap%
\pgfsetroundjoin%
\pgfsetlinewidth{1.505625pt}%
\definecolor{currentstroke}{rgb}{0.121148,0.592739,0.544641}%
\pgfsetstrokecolor{currentstroke}%
\pgfsetdash{}{0pt}%
\pgfpathmoveto{\pgfqpoint{0.766095in}{4.960032in}}%
\pgfpathlineto{\pgfqpoint{0.801138in}{4.975105in}}%
\pgfpathlineto{\pgfqpoint{0.820013in}{4.983195in}}%
\pgfpathlineto{\pgfqpoint{0.836180in}{4.990014in}}%
\pgfpathlineto{\pgfqpoint{0.871223in}{5.004736in}}%
\pgfpathlineto{\pgfqpoint{0.882505in}{5.009454in}}%
\pgfpathlineto{\pgfqpoint{0.906265in}{5.019232in}}%
\pgfpathlineto{\pgfqpoint{0.941308in}{5.033613in}}%
\pgfpathlineto{\pgfqpoint{0.946454in}{5.035714in}}%
\pgfpathlineto{\pgfqpoint{0.976350in}{5.047722in}}%
\pgfpathlineto{\pgfqpoint{1.011393in}{5.061771in}}%
\pgfpathlineto{\pgfqpoint{1.011901in}{5.061973in}}%
\pgfpathlineto{\pgfqpoint{1.046435in}{5.075515in}}%
\pgfpathlineto{\pgfqpoint{1.078917in}{5.088233in}}%
\pgfpathlineto{\pgfqpoint{1.081478in}{5.089219in}}%
\pgfpathlineto{\pgfqpoint{1.116520in}{5.102643in}}%
\pgfpathlineto{\pgfqpoint{1.147501in}{5.114492in}}%
\pgfpathlineto{\pgfqpoint{1.151563in}{5.116021in}}%
\pgfpathlineto{\pgfqpoint{1.186605in}{5.129137in}}%
\pgfpathlineto{\pgfqpoint{1.215389in}{5.139894in}}%
\pgfusepath{stroke}%
\end{pgfscope}%
\begin{pgfscope}%
\pgfpathrectangle{\pgfqpoint{0.766095in}{0.571603in}}{\pgfqpoint{6.973465in}{5.225635in}}%
\pgfusepath{clip}%
\pgfsetbuttcap%
\pgfsetroundjoin%
\pgfsetlinewidth{1.505625pt}%
\definecolor{currentstroke}{rgb}{0.121148,0.592739,0.544641}%
\pgfsetstrokecolor{currentstroke}%
\pgfsetdash{}{0pt}%
\pgfpathmoveto{\pgfqpoint{1.508996in}{5.243665in}}%
\pgfpathlineto{\pgfqpoint{1.515249in}{5.245790in}}%
\pgfpathlineto{\pgfqpoint{1.537031in}{5.253070in}}%
\pgfpathlineto{\pgfqpoint{1.572073in}{5.264750in}}%
\pgfpathlineto{\pgfqpoint{1.594027in}{5.272049in}}%
\pgfpathlineto{\pgfqpoint{1.607116in}{5.276330in}}%
\pgfpathlineto{\pgfqpoint{1.642158in}{5.287747in}}%
\pgfpathlineto{\pgfqpoint{1.674609in}{5.298308in}}%
\pgfpathlineto{\pgfqpoint{1.677201in}{5.299138in}}%
\pgfpathlineto{\pgfqpoint{1.712244in}{5.310298in}}%
\pgfpathlineto{\pgfqpoint{1.747286in}{5.321450in}}%
\pgfpathlineto{\pgfqpoint{1.757127in}{5.324568in}}%
\pgfpathlineto{\pgfqpoint{1.782329in}{5.332424in}}%
\pgfpathlineto{\pgfqpoint{1.817371in}{5.343325in}}%
\pgfpathlineto{\pgfqpoint{1.841542in}{5.350827in}}%
\pgfpathlineto{\pgfqpoint{1.852414in}{5.354147in}}%
\pgfpathlineto{\pgfqpoint{1.887456in}{5.364803in}}%
\pgfpathlineto{\pgfqpoint{1.922499in}{5.375453in}}%
\pgfpathlineto{\pgfqpoint{1.927901in}{5.377087in}}%
\pgfpathlineto{\pgfqpoint{1.957541in}{5.385905in}}%
\pgfpathlineto{\pgfqpoint{1.992584in}{5.396316in}}%
\pgfpathlineto{\pgfqpoint{2.016300in}{5.403346in}}%
\pgfpathlineto{\pgfqpoint{2.027626in}{5.406649in}}%
\pgfpathlineto{\pgfqpoint{2.062669in}{5.416827in}}%
\pgfpathlineto{\pgfqpoint{2.097711in}{5.426999in}}%
\pgfpathlineto{\pgfqpoint{2.106730in}{5.429606in}}%
\pgfpathlineto{\pgfqpoint{2.132754in}{5.437005in}}%
\pgfpathlineto{\pgfqpoint{2.167797in}{5.446949in}}%
\pgfpathlineto{\pgfqpoint{2.199245in}{5.455865in}}%
\pgfpathlineto{\pgfqpoint{2.202839in}{5.456868in}}%
\pgfpathlineto{\pgfqpoint{2.237882in}{5.466590in}}%
\pgfpathlineto{\pgfqpoint{2.272924in}{5.476307in}}%
\pgfpathlineto{\pgfqpoint{2.293953in}{5.482125in}}%
\pgfpathlineto{\pgfqpoint{2.307967in}{5.485937in}}%
\pgfpathlineto{\pgfqpoint{2.343009in}{5.495438in}}%
\pgfpathlineto{\pgfqpoint{2.378052in}{5.504935in}}%
\pgfpathlineto{\pgfqpoint{2.390827in}{5.508384in}}%
\pgfpathlineto{\pgfqpoint{2.413094in}{5.514297in}}%
\pgfpathlineto{\pgfqpoint{2.448137in}{5.523581in}}%
\pgfpathlineto{\pgfqpoint{2.483179in}{5.532863in}}%
\pgfpathlineto{\pgfqpoint{2.489934in}{5.534644in}}%
\pgfpathlineto{\pgfqpoint{2.518222in}{5.541978in}}%
\pgfpathlineto{\pgfqpoint{2.553264in}{5.551051in}}%
\pgfpathlineto{\pgfqpoint{2.588307in}{5.560122in}}%
\pgfpathlineto{\pgfqpoint{2.591339in}{5.560903in}}%
\pgfpathlineto{\pgfqpoint{2.623350in}{5.569010in}}%
\pgfpathlineto{\pgfqpoint{2.658392in}{5.577877in}}%
\pgfpathlineto{\pgfqpoint{2.693435in}{5.586743in}}%
\pgfpathlineto{\pgfqpoint{2.695103in}{5.587163in}}%
\pgfpathlineto{\pgfqpoint{2.728477in}{5.595421in}}%
\pgfpathlineto{\pgfqpoint{2.763520in}{5.604087in}}%
\pgfpathlineto{\pgfqpoint{2.798562in}{5.612752in}}%
\pgfpathlineto{\pgfqpoint{2.801285in}{5.613422in}}%
\pgfpathlineto{\pgfqpoint{2.833605in}{5.621239in}}%
\pgfpathlineto{\pgfqpoint{2.868647in}{5.629709in}}%
\pgfpathlineto{\pgfqpoint{2.903690in}{5.638177in}}%
\pgfpathlineto{\pgfqpoint{2.909942in}{5.639682in}}%
\pgfpathlineto{\pgfqpoint{2.938732in}{5.646490in}}%
\pgfpathlineto{\pgfqpoint{2.973775in}{5.654768in}}%
\pgfpathlineto{\pgfqpoint{3.008818in}{5.663045in}}%
\pgfpathlineto{\pgfqpoint{3.021126in}{5.665941in}}%
\pgfpathlineto{\pgfqpoint{3.043860in}{5.671200in}}%
\pgfpathlineto{\pgfqpoint{3.078903in}{5.679290in}}%
\pgfpathlineto{\pgfqpoint{3.113945in}{5.687378in}}%
\pgfpathlineto{\pgfqpoint{3.134884in}{5.692201in}}%
\pgfpathlineto{\pgfqpoint{3.148988in}{5.695393in}}%
\pgfpathlineto{\pgfqpoint{3.184030in}{5.703299in}}%
\pgfpathlineto{\pgfqpoint{3.219073in}{5.711203in}}%
\pgfpathlineto{\pgfqpoint{3.251262in}{5.718460in}}%
\pgfpathlineto{\pgfqpoint{3.254115in}{5.719092in}}%
\pgfpathlineto{\pgfqpoint{3.289158in}{5.726817in}}%
\pgfpathlineto{\pgfqpoint{3.324200in}{5.734541in}}%
\pgfpathlineto{\pgfqpoint{3.359243in}{5.742265in}}%
\pgfpathlineto{\pgfqpoint{3.370423in}{5.744720in}}%
\pgfpathlineto{\pgfqpoint{3.394285in}{5.749868in}}%
\pgfpathlineto{\pgfqpoint{3.429328in}{5.757416in}}%
\pgfpathlineto{\pgfqpoint{3.464371in}{5.764962in}}%
\pgfpathlineto{\pgfqpoint{3.492343in}{5.770979in}}%
\pgfpathlineto{\pgfqpoint{3.499413in}{5.772473in}}%
\pgfpathlineto{\pgfqpoint{3.534456in}{5.779847in}}%
\pgfpathlineto{\pgfqpoint{3.569498in}{5.787220in}}%
\pgfpathlineto{\pgfqpoint{3.604541in}{5.794593in}}%
\pgfpathlineto{\pgfqpoint{3.617161in}{5.797238in}}%
\pgfusepath{stroke}%
\end{pgfscope}%
\begin{pgfscope}%
\pgfpathrectangle{\pgfqpoint{0.766095in}{0.571603in}}{\pgfqpoint{6.973465in}{5.225635in}}%
\pgfusepath{clip}%
\pgfsetbuttcap%
\pgfsetroundjoin%
\pgfsetlinewidth{1.505625pt}%
\definecolor{currentstroke}{rgb}{0.119423,0.611141,0.538982}%
\pgfsetstrokecolor{currentstroke}%
\pgfsetdash{}{0pt}%
\pgfpathmoveto{\pgfqpoint{0.766095in}{5.008404in}}%
\pgfpathlineto{\pgfqpoint{0.768559in}{5.009454in}}%
\pgfpathlineto{\pgfqpoint{0.797150in}{5.021445in}}%
\pgfusepath{stroke}%
\end{pgfscope}%
\begin{pgfscope}%
\pgfpathrectangle{\pgfqpoint{0.766095in}{0.571603in}}{\pgfqpoint{6.973465in}{5.225635in}}%
\pgfusepath{clip}%
\pgfsetbuttcap%
\pgfsetroundjoin%
\pgfsetlinewidth{1.505625pt}%
\definecolor{currentstroke}{rgb}{0.119423,0.611141,0.538982}%
\pgfsetstrokecolor{currentstroke}%
\pgfsetdash{}{0pt}%
\pgfpathmoveto{\pgfqpoint{1.086459in}{5.136765in}}%
\pgfpathlineto{\pgfqpoint{1.096934in}{5.140752in}}%
\pgfpathlineto{\pgfqpoint{1.116520in}{5.148087in}}%
\pgfpathlineto{\pgfqpoint{1.151563in}{5.161169in}}%
\pgfpathlineto{\pgfqpoint{1.167269in}{5.167011in}}%
\pgfpathlineto{\pgfqpoint{1.186605in}{5.174088in}}%
\pgfpathlineto{\pgfqpoint{1.221648in}{5.186873in}}%
\pgfpathlineto{\pgfqpoint{1.239242in}{5.193271in}}%
\pgfpathlineto{\pgfqpoint{1.256691in}{5.199514in}}%
\pgfpathlineto{\pgfqpoint{1.291733in}{5.212009in}}%
\pgfpathlineto{\pgfqpoint{1.312886in}{5.219530in}}%
\pgfpathlineto{\pgfqpoint{1.326776in}{5.224390in}}%
\pgfpathlineto{\pgfqpoint{1.361818in}{5.236603in}}%
\pgfpathlineto{\pgfqpoint{1.388232in}{5.245790in}}%
\pgfpathlineto{\pgfqpoint{1.396861in}{5.248743in}}%
\pgfpathlineto{\pgfqpoint{1.431903in}{5.260680in}}%
\pgfpathlineto{\pgfqpoint{1.465310in}{5.272049in}}%
\pgfpathlineto{\pgfqpoint{1.466946in}{5.272597in}}%
\pgfpathlineto{\pgfqpoint{1.501988in}{5.284266in}}%
\pgfpathlineto{\pgfqpoint{1.537031in}{5.295926in}}%
\pgfpathlineto{\pgfqpoint{1.544225in}{5.298308in}}%
\pgfpathlineto{\pgfqpoint{1.572073in}{5.307383in}}%
\pgfpathlineto{\pgfqpoint{1.607116in}{5.318782in}}%
\pgfpathlineto{\pgfqpoint{1.624960in}{5.324568in}}%
\pgfpathlineto{\pgfqpoint{1.642158in}{5.330055in}}%
\pgfpathlineto{\pgfqpoint{1.677201in}{5.341199in}}%
\pgfpathlineto{\pgfqpoint{1.707519in}{5.350827in}}%
\pgfpathlineto{\pgfqpoint{1.712244in}{5.352304in}}%
\pgfpathlineto{\pgfqpoint{1.747286in}{5.363198in}}%
\pgfpathlineto{\pgfqpoint{1.782329in}{5.374086in}}%
\pgfpathlineto{\pgfqpoint{1.792027in}{5.377087in}}%
\pgfpathlineto{\pgfqpoint{1.817371in}{5.384801in}}%
\pgfpathlineto{\pgfqpoint{1.852414in}{5.395447in}}%
\pgfpathlineto{\pgfqpoint{1.878466in}{5.403346in}}%
\pgfpathlineto{\pgfqpoint{1.887456in}{5.406028in}}%
\pgfpathlineto{\pgfqpoint{1.922499in}{5.416436in}}%
\pgfpathlineto{\pgfqpoint{1.957541in}{5.426839in}}%
\pgfpathlineto{\pgfqpoint{1.966900in}{5.429606in}}%
\pgfpathlineto{\pgfqpoint{1.992584in}{5.437075in}}%
\pgfpathlineto{\pgfqpoint{2.027626in}{5.447247in}}%
\pgfpathlineto{\pgfqpoint{2.057355in}{5.455865in}}%
\pgfpathlineto{\pgfqpoint{2.062669in}{5.457381in}}%
\pgfpathlineto{\pgfqpoint{2.097711in}{5.467327in}}%
\pgfpathlineto{\pgfqpoint{2.132754in}{5.477268in}}%
\pgfpathlineto{\pgfqpoint{2.149925in}{5.482125in}}%
\pgfpathlineto{\pgfqpoint{2.167797in}{5.487097in}}%
\pgfpathlineto{\pgfqpoint{2.202839in}{5.496818in}}%
\pgfpathlineto{\pgfqpoint{2.237882in}{5.506535in}}%
\pgfpathlineto{\pgfqpoint{2.244580in}{5.508384in}}%
\pgfpathlineto{\pgfqpoint{2.272924in}{5.516080in}}%
\pgfpathlineto{\pgfqpoint{2.307967in}{5.525582in}}%
\pgfpathlineto{\pgfqpoint{2.341403in}{5.534644in}}%
\pgfpathlineto{\pgfqpoint{2.343009in}{5.535072in}}%
\pgfpathlineto{\pgfqpoint{2.378052in}{5.544363in}}%
\pgfpathlineto{\pgfqpoint{2.413094in}{5.553652in}}%
\pgfpathlineto{\pgfqpoint{2.440491in}{5.560903in}}%
\pgfpathlineto{\pgfqpoint{2.448137in}{5.562894in}}%
\pgfpathlineto{\pgfqpoint{2.483179in}{5.571976in}}%
\pgfpathlineto{\pgfqpoint{2.518222in}{5.581056in}}%
\pgfpathlineto{\pgfqpoint{2.541833in}{5.587163in}}%
\pgfpathlineto{\pgfqpoint{2.553264in}{5.590070in}}%
\pgfpathlineto{\pgfqpoint{2.588307in}{5.598949in}}%
\pgfpathlineto{\pgfqpoint{2.623350in}{5.607826in}}%
\pgfpathlineto{\pgfqpoint{2.645489in}{5.613422in}}%
\pgfpathlineto{\pgfqpoint{2.658392in}{5.616630in}}%
\pgfpathlineto{\pgfqpoint{2.693435in}{5.625310in}}%
\pgfpathlineto{\pgfqpoint{2.728477in}{5.633988in}}%
\pgfpathlineto{\pgfqpoint{2.751513in}{5.639682in}}%
\pgfpathlineto{\pgfqpoint{2.763520in}{5.642600in}}%
\pgfpathlineto{\pgfqpoint{2.798562in}{5.651086in}}%
\pgfpathlineto{\pgfqpoint{2.833605in}{5.659570in}}%
\pgfpathlineto{\pgfqpoint{2.859957in}{5.665941in}}%
\pgfpathlineto{\pgfqpoint{2.868647in}{5.668007in}}%
\pgfpathlineto{\pgfqpoint{2.903690in}{5.676303in}}%
\pgfpathlineto{\pgfqpoint{2.938732in}{5.684597in}}%
\pgfpathlineto{\pgfqpoint{2.970871in}{5.692201in}}%
\pgfpathlineto{\pgfqpoint{2.973775in}{5.692876in}}%
\pgfpathlineto{\pgfqpoint{3.008818in}{5.700986in}}%
\pgfpathlineto{\pgfqpoint{3.043860in}{5.709095in}}%
\pgfpathlineto{\pgfqpoint{3.078903in}{5.717204in}}%
\pgfpathlineto{\pgfqpoint{3.084356in}{5.718460in}}%
\pgfpathlineto{\pgfqpoint{3.113945in}{5.725160in}}%
\pgfpathlineto{\pgfqpoint{3.148988in}{5.733088in}}%
\pgfpathlineto{\pgfqpoint{3.184030in}{5.741015in}}%
\pgfpathlineto{\pgfqpoint{3.200455in}{5.744720in}}%
\pgfpathlineto{\pgfqpoint{3.219073in}{5.748847in}}%
\pgfpathlineto{\pgfqpoint{3.254115in}{5.756597in}}%
\pgfpathlineto{\pgfqpoint{3.289158in}{5.764346in}}%
\pgfpathlineto{\pgfqpoint{3.319175in}{5.770979in}}%
\pgfpathlineto{\pgfqpoint{3.324200in}{5.772070in}}%
\pgfpathlineto{\pgfqpoint{3.359243in}{5.779646in}}%
\pgfpathlineto{\pgfqpoint{3.394285in}{5.787221in}}%
\pgfpathlineto{\pgfqpoint{3.429328in}{5.794796in}}%
\pgfpathlineto{\pgfqpoint{3.440670in}{5.797238in}}%
\pgfusepath{stroke}%
\end{pgfscope}%
\begin{pgfscope}%
\pgfpathrectangle{\pgfqpoint{0.766095in}{0.571603in}}{\pgfqpoint{6.973465in}{5.225635in}}%
\pgfusepath{clip}%
\pgfsetbuttcap%
\pgfsetroundjoin%
\pgfsetlinewidth{1.505625pt}%
\definecolor{currentstroke}{rgb}{0.121380,0.629492,0.531973}%
\pgfsetstrokecolor{currentstroke}%
\pgfsetdash{}{0pt}%
\pgfpathmoveto{\pgfqpoint{0.766095in}{5.054907in}}%
\pgfpathlineto{\pgfqpoint{0.783078in}{5.061973in}}%
\pgfpathlineto{\pgfqpoint{0.801138in}{5.069369in}}%
\pgfpathlineto{\pgfqpoint{0.836180in}{5.083669in}}%
\pgfpathlineto{\pgfqpoint{0.847415in}{5.088233in}}%
\pgfpathlineto{\pgfqpoint{0.871223in}{5.097751in}}%
\pgfpathlineto{\pgfqpoint{0.906265in}{5.111724in}}%
\pgfpathlineto{\pgfqpoint{0.913244in}{5.114492in}}%
\pgfpathlineto{\pgfqpoint{0.941308in}{5.125449in}}%
\pgfpathlineto{\pgfqpoint{0.976350in}{5.139103in}}%
\pgfpathlineto{\pgfqpoint{0.980604in}{5.140752in}}%
\pgfpathlineto{\pgfqpoint{1.011393in}{5.152494in}}%
\pgfpathlineto{\pgfqpoint{1.046435in}{5.165838in}}%
\pgfpathlineto{\pgfqpoint{1.049533in}{5.167011in}}%
\pgfpathlineto{\pgfqpoint{1.081478in}{5.178916in}}%
\pgfpathlineto{\pgfqpoint{1.116520in}{5.191957in}}%
\pgfpathlineto{\pgfqpoint{1.120068in}{5.193271in}}%
\pgfpathlineto{\pgfqpoint{1.151563in}{5.204743in}}%
\pgfpathlineto{\pgfqpoint{1.186605in}{5.217490in}}%
\pgfpathlineto{\pgfqpoint{1.192242in}{5.219530in}}%
\pgfpathlineto{\pgfqpoint{1.221648in}{5.230003in}}%
\pgfpathlineto{\pgfqpoint{1.256691in}{5.242463in}}%
\pgfpathlineto{\pgfqpoint{1.266089in}{5.245790in}}%
\pgfpathlineto{\pgfqpoint{1.291733in}{5.254722in}}%
\pgfpathlineto{\pgfqpoint{1.326776in}{5.266902in}}%
\pgfpathlineto{\pgfqpoint{1.341639in}{5.272049in}}%
\pgfpathlineto{\pgfqpoint{1.361818in}{5.278926in}}%
\pgfpathlineto{\pgfqpoint{1.396861in}{5.290833in}}%
\pgfpathlineto{\pgfqpoint{1.418919in}{5.298308in}}%
\pgfpathlineto{\pgfqpoint{1.431903in}{5.302639in}}%
\pgfpathlineto{\pgfqpoint{1.466946in}{5.314279in}}%
\pgfpathlineto{\pgfqpoint{1.497958in}{5.324568in}}%
\pgfpathlineto{\pgfqpoint{1.501988in}{5.325884in}}%
\pgfpathlineto{\pgfqpoint{1.537031in}{5.337265in}}%
\pgfpathlineto{\pgfqpoint{1.572073in}{5.348639in}}%
\pgfpathlineto{\pgfqpoint{1.578849in}{5.350827in}}%
\pgfpathlineto{\pgfqpoint{1.607116in}{5.359812in}}%
\pgfpathlineto{\pgfqpoint{1.642158in}{5.370933in}}%
\pgfpathlineto{\pgfqpoint{1.661605in}{5.377087in}}%
\pgfpathlineto{\pgfqpoint{1.677201in}{5.381943in}}%
\pgfpathlineto{\pgfqpoint{1.712244in}{5.392817in}}%
\pgfpathlineto{\pgfqpoint{1.746201in}{5.403346in}}%
\pgfpathlineto{\pgfqpoint{1.747286in}{5.403677in}}%
\pgfpathlineto{\pgfqpoint{1.782329in}{5.414311in}}%
\pgfpathlineto{\pgfqpoint{1.817371in}{5.424938in}}%
\pgfpathlineto{\pgfqpoint{1.832815in}{5.429606in}}%
\pgfpathlineto{\pgfqpoint{1.852414in}{5.435434in}}%
\pgfpathlineto{\pgfqpoint{1.887456in}{5.445827in}}%
\pgfpathlineto{\pgfqpoint{1.921328in}{5.455865in}}%
\pgfpathlineto{\pgfqpoint{1.922499in}{5.456207in}}%
\pgfpathlineto{\pgfqpoint{1.957541in}{5.466370in}}%
\pgfpathlineto{\pgfqpoint{1.992584in}{5.476528in}}%
\pgfpathlineto{\pgfqpoint{2.011942in}{5.482125in}}%
\pgfpathlineto{\pgfqpoint{2.027626in}{5.486586in}}%
\pgfpathlineto{\pgfqpoint{2.062669in}{5.496520in}}%
\pgfpathlineto{\pgfqpoint{2.097711in}{5.506451in}}%
\pgfpathlineto{\pgfqpoint{2.104563in}{5.508384in}}%
\pgfpathlineto{\pgfqpoint{2.132754in}{5.516208in}}%
\pgfpathlineto{\pgfqpoint{2.167797in}{5.525921in}}%
\pgfpathlineto{\pgfqpoint{2.199297in}{5.534644in}}%
\pgfpathlineto{\pgfqpoint{2.202839in}{5.535609in}}%
\pgfpathlineto{\pgfqpoint{2.237882in}{5.545108in}}%
\pgfpathlineto{\pgfqpoint{2.272924in}{5.554603in}}%
\pgfpathlineto{\pgfqpoint{2.296219in}{5.560903in}}%
\pgfpathlineto{\pgfqpoint{2.307967in}{5.564028in}}%
\pgfpathlineto{\pgfqpoint{2.343009in}{5.573316in}}%
\pgfpathlineto{\pgfqpoint{2.378052in}{5.582600in}}%
\pgfpathlineto{\pgfqpoint{2.395320in}{5.587163in}}%
\pgfpathlineto{\pgfqpoint{2.413094in}{5.591782in}}%
\pgfpathlineto{\pgfqpoint{2.448137in}{5.600863in}}%
\pgfpathlineto{\pgfqpoint{2.483179in}{5.609942in}}%
\pgfpathlineto{\pgfqpoint{2.496659in}{5.613422in}}%
\pgfpathlineto{\pgfqpoint{2.518222in}{5.618898in}}%
\pgfpathlineto{\pgfqpoint{2.553264in}{5.627778in}}%
\pgfpathlineto{\pgfqpoint{2.588307in}{5.636656in}}%
\pgfpathlineto{\pgfqpoint{2.600291in}{5.639682in}}%
\pgfpathlineto{\pgfqpoint{2.623350in}{5.645406in}}%
\pgfpathlineto{\pgfqpoint{2.643729in}{5.650456in}}%
\pgfusepath{stroke}%
\end{pgfscope}%
\begin{pgfscope}%
\pgfpathrectangle{\pgfqpoint{0.766095in}{0.571603in}}{\pgfqpoint{6.973465in}{5.225635in}}%
\pgfusepath{clip}%
\pgfsetbuttcap%
\pgfsetroundjoin%
\pgfsetlinewidth{1.505625pt}%
\definecolor{currentstroke}{rgb}{0.121380,0.629492,0.531973}%
\pgfsetstrokecolor{currentstroke}%
\pgfsetdash{}{0pt}%
\pgfpathmoveto{\pgfqpoint{3.023878in}{5.741274in}}%
\pgfpathlineto{\pgfqpoint{3.038761in}{5.744720in}}%
\pgfpathlineto{\pgfqpoint{3.043860in}{5.745880in}}%
\pgfpathlineto{\pgfqpoint{3.078903in}{5.753822in}}%
\pgfpathlineto{\pgfqpoint{3.113945in}{5.761763in}}%
\pgfpathlineto{\pgfqpoint{3.148988in}{5.769703in}}%
\pgfpathlineto{\pgfqpoint{3.154641in}{5.770979in}}%
\pgfpathlineto{\pgfqpoint{3.184030in}{5.777498in}}%
\pgfpathlineto{\pgfqpoint{3.219073in}{5.785263in}}%
\pgfpathlineto{\pgfqpoint{3.254115in}{5.793029in}}%
\pgfpathlineto{\pgfqpoint{3.273157in}{5.797238in}}%
\pgfusepath{stroke}%
\end{pgfscope}%
\begin{pgfscope}%
\pgfpathrectangle{\pgfqpoint{0.766095in}{0.571603in}}{\pgfqpoint{6.973465in}{5.225635in}}%
\pgfusepath{clip}%
\pgfsetbuttcap%
\pgfsetroundjoin%
\pgfsetlinewidth{1.505625pt}%
\definecolor{currentstroke}{rgb}{0.128087,0.647749,0.523491}%
\pgfsetstrokecolor{currentstroke}%
\pgfsetdash{}{0pt}%
\pgfpathmoveto{\pgfqpoint{0.766095in}{5.099774in}}%
\pgfpathlineto{\pgfqpoint{0.801138in}{5.114014in}}%
\pgfpathlineto{\pgfqpoint{0.802320in}{5.114492in}}%
\pgfpathlineto{\pgfqpoint{0.836180in}{5.127956in}}%
\pgfpathlineto{\pgfqpoint{0.868413in}{5.140752in}}%
\pgfpathlineto{\pgfqpoint{0.871223in}{5.141850in}}%
\pgfpathlineto{\pgfqpoint{0.906265in}{5.155465in}}%
\pgfpathlineto{\pgfqpoint{0.936037in}{5.167011in}}%
\pgfpathlineto{\pgfqpoint{0.941308in}{5.169023in}}%
\pgfpathlineto{\pgfqpoint{0.976350in}{5.182330in}}%
\pgfpathlineto{\pgfqpoint{1.005216in}{5.193271in}}%
\pgfpathlineto{\pgfqpoint{1.011393in}{5.195575in}}%
\pgfpathlineto{\pgfqpoint{1.046435in}{5.208581in}}%
\pgfpathlineto{\pgfqpoint{1.075986in}{5.219530in}}%
\pgfpathlineto{\pgfqpoint{1.081478in}{5.221533in}}%
\pgfpathlineto{\pgfqpoint{1.116520in}{5.234246in}}%
\pgfpathlineto{\pgfqpoint{1.148380in}{5.245790in}}%
\pgfpathlineto{\pgfqpoint{1.151563in}{5.246925in}}%
\pgfpathlineto{\pgfqpoint{1.186605in}{5.259353in}}%
\pgfpathlineto{\pgfqpoint{1.221648in}{5.271771in}}%
\pgfpathlineto{\pgfqpoint{1.222438in}{5.272049in}}%
\pgfpathlineto{\pgfqpoint{1.256691in}{5.283927in}}%
\pgfpathlineto{\pgfqpoint{1.291733in}{5.296067in}}%
\pgfpathlineto{\pgfqpoint{1.298234in}{5.298308in}}%
\pgfpathlineto{\pgfqpoint{1.326776in}{5.307993in}}%
\pgfpathlineto{\pgfqpoint{1.361818in}{5.319863in}}%
\pgfpathlineto{\pgfqpoint{1.375761in}{5.324568in}}%
\pgfpathlineto{\pgfqpoint{1.396861in}{5.331575in}}%
\pgfpathlineto{\pgfqpoint{1.431903in}{5.343182in}}%
\pgfpathlineto{\pgfqpoint{1.455043in}{5.350827in}}%
\pgfpathlineto{\pgfqpoint{1.466946in}{5.354698in}}%
\pgfpathlineto{\pgfqpoint{1.501988in}{5.366047in}}%
\pgfpathlineto{\pgfqpoint{1.536104in}{5.377087in}}%
\pgfpathlineto{\pgfqpoint{1.537031in}{5.377382in}}%
\pgfpathlineto{\pgfqpoint{1.572073in}{5.388480in}}%
\pgfpathlineto{\pgfqpoint{1.607116in}{5.399571in}}%
\pgfpathlineto{\pgfqpoint{1.619089in}{5.403346in}}%
\pgfpathlineto{\pgfqpoint{1.642158in}{5.410503in}}%
\pgfpathlineto{\pgfqpoint{1.677201in}{5.421350in}}%
\pgfpathlineto{\pgfqpoint{1.703920in}{5.429606in}}%
\pgfpathlineto{\pgfqpoint{1.712244in}{5.432137in}}%
\pgfpathlineto{\pgfqpoint{1.747286in}{5.442745in}}%
\pgfpathlineto{\pgfqpoint{1.782329in}{5.453347in}}%
\pgfpathlineto{\pgfqpoint{1.790687in}{5.455865in}}%
\pgfpathlineto{\pgfqpoint{1.817371in}{5.463775in}}%
\pgfpathlineto{\pgfqpoint{1.852414in}{5.474145in}}%
\pgfpathlineto{\pgfqpoint{1.879423in}{5.482125in}}%
\pgfpathlineto{\pgfqpoint{1.887456in}{5.484460in}}%
\pgfpathlineto{\pgfqpoint{1.922499in}{5.494603in}}%
\pgfpathlineto{\pgfqpoint{1.957541in}{5.504741in}}%
\pgfpathlineto{\pgfqpoint{1.970178in}{5.508384in}}%
\pgfpathlineto{\pgfqpoint{1.992584in}{5.514739in}}%
\pgfpathlineto{\pgfqpoint{2.027626in}{5.524656in}}%
\pgfpathlineto{\pgfqpoint{2.049415in}{5.530819in}}%
\pgfusepath{stroke}%
\end{pgfscope}%
\begin{pgfscope}%
\pgfpathrectangle{\pgfqpoint{0.766095in}{0.571603in}}{\pgfqpoint{6.973465in}{5.225635in}}%
\pgfusepath{clip}%
\pgfsetbuttcap%
\pgfsetroundjoin%
\pgfsetlinewidth{1.505625pt}%
\definecolor{currentstroke}{rgb}{0.128087,0.647749,0.523491}%
\pgfsetstrokecolor{currentstroke}%
\pgfsetdash{}{0pt}%
\pgfpathmoveto{\pgfqpoint{2.426892in}{5.632278in}}%
\pgfpathlineto{\pgfqpoint{2.448137in}{5.637778in}}%
\pgfpathlineto{\pgfqpoint{2.455523in}{5.639682in}}%
\pgfpathlineto{\pgfqpoint{2.483179in}{5.646694in}}%
\pgfpathlineto{\pgfqpoint{2.518222in}{5.655567in}}%
\pgfpathlineto{\pgfqpoint{2.553264in}{5.664439in}}%
\pgfpathlineto{\pgfqpoint{2.559223in}{5.665941in}}%
\pgfpathlineto{\pgfqpoint{2.588307in}{5.673152in}}%
\pgfpathlineto{\pgfqpoint{2.623350in}{5.681831in}}%
\pgfpathlineto{\pgfqpoint{2.658392in}{5.690510in}}%
\pgfpathlineto{\pgfqpoint{2.665249in}{5.692201in}}%
\pgfpathlineto{\pgfqpoint{2.693435in}{5.699037in}}%
\pgfpathlineto{\pgfqpoint{2.728477in}{5.707527in}}%
\pgfpathlineto{\pgfqpoint{2.763520in}{5.716016in}}%
\pgfpathlineto{\pgfqpoint{2.773646in}{5.718460in}}%
\pgfpathlineto{\pgfqpoint{2.798562in}{5.724374in}}%
\pgfpathlineto{\pgfqpoint{2.833605in}{5.732679in}}%
\pgfpathlineto{\pgfqpoint{2.868647in}{5.740983in}}%
\pgfpathlineto{\pgfqpoint{2.884460in}{5.744720in}}%
\pgfpathlineto{\pgfqpoint{2.903690in}{5.749188in}}%
\pgfpathlineto{\pgfqpoint{2.938732in}{5.757312in}}%
\pgfpathlineto{\pgfqpoint{2.973775in}{5.765435in}}%
\pgfpathlineto{\pgfqpoint{2.997728in}{5.770979in}}%
\pgfpathlineto{\pgfqpoint{3.008818in}{5.773503in}}%
\pgfpathlineto{\pgfqpoint{3.043860in}{5.781449in}}%
\pgfpathlineto{\pgfqpoint{3.078903in}{5.789396in}}%
\pgfpathlineto{\pgfqpoint{3.113487in}{5.797238in}}%
\pgfusepath{stroke}%
\end{pgfscope}%
\begin{pgfscope}%
\pgfpathrectangle{\pgfqpoint{0.766095in}{0.571603in}}{\pgfqpoint{6.973465in}{5.225635in}}%
\pgfusepath{clip}%
\pgfsetbuttcap%
\pgfsetroundjoin%
\pgfsetlinewidth{1.505625pt}%
\definecolor{currentstroke}{rgb}{0.140210,0.665859,0.513427}%
\pgfsetstrokecolor{currentstroke}%
\pgfsetdash{}{0pt}%
\pgfpathmoveto{\pgfqpoint{0.766095in}{5.143158in}}%
\pgfpathlineto{\pgfqpoint{0.801138in}{5.157029in}}%
\pgfpathlineto{\pgfqpoint{0.826421in}{5.167011in}}%
\pgfpathlineto{\pgfqpoint{0.836180in}{5.170804in}}%
\pgfpathlineto{\pgfqpoint{0.871223in}{5.184360in}}%
\pgfpathlineto{\pgfqpoint{0.894318in}{5.193271in}}%
\pgfpathlineto{\pgfqpoint{0.906265in}{5.197808in}}%
\pgfpathlineto{\pgfqpoint{0.941308in}{5.211058in}}%
\pgfpathlineto{\pgfqpoint{0.963774in}{5.219530in}}%
\pgfpathlineto{\pgfqpoint{0.976350in}{5.224198in}}%
\pgfpathlineto{\pgfqpoint{1.011393in}{5.237151in}}%
\pgfpathlineto{\pgfqpoint{1.034824in}{5.245790in}}%
\pgfpathlineto{\pgfqpoint{1.046435in}{5.250003in}}%
\pgfpathlineto{\pgfqpoint{1.081478in}{5.262666in}}%
\pgfpathlineto{\pgfqpoint{1.107500in}{5.272049in}}%
\pgfpathlineto{\pgfqpoint{1.116520in}{5.275250in}}%
\pgfpathlineto{\pgfqpoint{1.151563in}{5.287631in}}%
\pgfpathlineto{\pgfqpoint{1.181833in}{5.298308in}}%
\pgfpathlineto{\pgfqpoint{1.186605in}{5.299966in}}%
\pgfpathlineto{\pgfqpoint{1.221648in}{5.312071in}}%
\pgfpathlineto{\pgfqpoint{1.256691in}{5.324165in}}%
\pgfpathlineto{\pgfqpoint{1.257863in}{5.324568in}}%
\pgfpathlineto{\pgfqpoint{1.291733in}{5.336010in}}%
\pgfpathlineto{\pgfqpoint{1.326776in}{5.347837in}}%
\pgfpathlineto{\pgfqpoint{1.335674in}{5.350827in}}%
\pgfpathlineto{\pgfqpoint{1.361818in}{5.359474in}}%
\pgfpathlineto{\pgfqpoint{1.396861in}{5.371039in}}%
\pgfpathlineto{\pgfqpoint{1.415240in}{5.377087in}}%
\pgfpathlineto{\pgfqpoint{1.431903in}{5.382483in}}%
\pgfpathlineto{\pgfqpoint{1.466946in}{5.393794in}}%
\pgfpathlineto{\pgfqpoint{1.496581in}{5.403346in}}%
\pgfpathlineto{\pgfqpoint{1.501988in}{5.405062in}}%
\pgfpathlineto{\pgfqpoint{1.537031in}{5.416124in}}%
\pgfpathlineto{\pgfqpoint{1.572073in}{5.427180in}}%
\pgfpathlineto{\pgfqpoint{1.579798in}{5.429606in}}%
\pgfpathlineto{\pgfqpoint{1.607116in}{5.438050in}}%
\pgfpathlineto{\pgfqpoint{1.642158in}{5.448863in}}%
\pgfpathlineto{\pgfqpoint{1.664901in}{5.455865in}}%
\pgfpathlineto{\pgfqpoint{1.677201in}{5.459592in}}%
\pgfpathlineto{\pgfqpoint{1.712244in}{5.470169in}}%
\pgfpathlineto{\pgfqpoint{1.747286in}{5.480741in}}%
\pgfpathlineto{\pgfqpoint{1.751896in}{5.482125in}}%
\pgfpathlineto{\pgfqpoint{1.782329in}{5.491116in}}%
\pgfpathlineto{\pgfqpoint{1.817371in}{5.501457in}}%
\pgfpathlineto{\pgfqpoint{1.840892in}{5.508384in}}%
\pgfpathlineto{\pgfqpoint{1.852414in}{5.511723in}}%
\pgfpathlineto{\pgfqpoint{1.887456in}{5.521839in}}%
\pgfpathlineto{\pgfqpoint{1.922499in}{5.531951in}}%
\pgfpathlineto{\pgfqpoint{1.931866in}{5.534644in}}%
\pgfpathlineto{\pgfqpoint{1.957541in}{5.541905in}}%
\pgfpathlineto{\pgfqpoint{1.992584in}{5.551797in}}%
\pgfpathlineto{\pgfqpoint{2.024864in}{5.560903in}}%
\pgfpathlineto{\pgfqpoint{2.027626in}{5.561670in}}%
\pgfpathlineto{\pgfqpoint{2.049278in}{5.567650in}}%
\pgfusepath{stroke}%
\end{pgfscope}%
\begin{pgfscope}%
\pgfpathrectangle{\pgfqpoint{0.766095in}{0.571603in}}{\pgfqpoint{6.973465in}{5.225635in}}%
\pgfusepath{clip}%
\pgfsetbuttcap%
\pgfsetroundjoin%
\pgfsetlinewidth{1.505625pt}%
\definecolor{currentstroke}{rgb}{0.140210,0.665859,0.513427}%
\pgfsetstrokecolor{currentstroke}%
\pgfsetdash{}{0pt}%
\pgfpathmoveto{\pgfqpoint{2.427003in}{5.668172in}}%
\pgfpathlineto{\pgfqpoint{2.448137in}{5.673521in}}%
\pgfpathlineto{\pgfqpoint{2.483179in}{5.682381in}}%
\pgfpathlineto{\pgfqpoint{2.518222in}{5.691241in}}%
\pgfpathlineto{\pgfqpoint{2.522036in}{5.692201in}}%
\pgfpathlineto{\pgfqpoint{2.553264in}{5.699931in}}%
\pgfpathlineto{\pgfqpoint{2.588307in}{5.708600in}}%
\pgfpathlineto{\pgfqpoint{2.623350in}{5.717268in}}%
\pgfpathlineto{\pgfqpoint{2.628190in}{5.718460in}}%
\pgfpathlineto{\pgfqpoint{2.658392in}{5.725776in}}%
\pgfpathlineto{\pgfqpoint{2.693435in}{5.734257in}}%
\pgfpathlineto{\pgfqpoint{2.728477in}{5.742738in}}%
\pgfpathlineto{\pgfqpoint{2.736694in}{5.744720in}}%
\pgfpathlineto{\pgfqpoint{2.763520in}{5.751080in}}%
\pgfpathlineto{\pgfqpoint{2.798562in}{5.759379in}}%
\pgfpathlineto{\pgfqpoint{2.833605in}{5.767677in}}%
\pgfpathlineto{\pgfqpoint{2.847589in}{5.770979in}}%
\pgfpathlineto{\pgfqpoint{2.868647in}{5.775868in}}%
\pgfpathlineto{\pgfqpoint{2.903690in}{5.783988in}}%
\pgfpathlineto{\pgfqpoint{2.938732in}{5.792108in}}%
\pgfpathlineto{\pgfqpoint{2.960912in}{5.797238in}}%
\pgfusepath{stroke}%
\end{pgfscope}%
\begin{pgfscope}%
\pgfpathrectangle{\pgfqpoint{0.766095in}{0.571603in}}{\pgfqpoint{6.973465in}{5.225635in}}%
\pgfusepath{clip}%
\pgfsetbuttcap%
\pgfsetroundjoin%
\pgfsetlinewidth{1.505625pt}%
\definecolor{currentstroke}{rgb}{0.157851,0.683765,0.501686}%
\pgfsetstrokecolor{currentstroke}%
\pgfsetdash{}{0pt}%
\pgfpathmoveto{\pgfqpoint{0.766095in}{5.185035in}}%
\pgfpathlineto{\pgfqpoint{0.787068in}{5.193271in}}%
\pgfpathlineto{\pgfqpoint{0.801138in}{5.198710in}}%
\pgfpathlineto{\pgfqpoint{0.836180in}{5.212202in}}%
\pgfpathlineto{\pgfqpoint{0.855275in}{5.219530in}}%
\pgfpathlineto{\pgfqpoint{0.871223in}{5.225555in}}%
\pgfpathlineto{\pgfqpoint{0.906265in}{5.238744in}}%
\pgfpathlineto{\pgfqpoint{0.925045in}{5.245790in}}%
\pgfpathlineto{\pgfqpoint{0.941308in}{5.251795in}}%
\pgfpathlineto{\pgfqpoint{0.976350in}{5.264690in}}%
\pgfpathlineto{\pgfqpoint{0.996410in}{5.272049in}}%
\pgfpathlineto{\pgfqpoint{1.011393in}{5.277459in}}%
\pgfpathlineto{\pgfqpoint{1.046435in}{5.290066in}}%
\pgfpathlineto{\pgfqpoint{1.069404in}{5.298308in}}%
\pgfpathlineto{\pgfqpoint{1.081478in}{5.302573in}}%
\pgfpathlineto{\pgfqpoint{1.116520in}{5.314900in}}%
\pgfpathlineto{\pgfqpoint{1.144055in}{5.324568in}}%
\pgfpathlineto{\pgfqpoint{1.151563in}{5.327162in}}%
\pgfpathlineto{\pgfqpoint{1.186605in}{5.339217in}}%
\pgfpathlineto{\pgfqpoint{1.220393in}{5.350827in}}%
\pgfpathlineto{\pgfqpoint{1.221648in}{5.351252in}}%
\pgfpathlineto{\pgfqpoint{1.256691in}{5.363040in}}%
\pgfpathlineto{\pgfqpoint{1.291733in}{5.374819in}}%
\pgfpathlineto{\pgfqpoint{1.298511in}{5.377087in}}%
\pgfpathlineto{\pgfqpoint{1.326776in}{5.386395in}}%
\pgfpathlineto{\pgfqpoint{1.361818in}{5.397915in}}%
\pgfpathlineto{\pgfqpoint{1.378395in}{5.403346in}}%
\pgfpathlineto{\pgfqpoint{1.396861in}{5.409302in}}%
\pgfpathlineto{\pgfqpoint{1.431903in}{5.420570in}}%
\pgfpathlineto{\pgfqpoint{1.460051in}{5.429606in}}%
\pgfpathlineto{\pgfqpoint{1.466946in}{5.431784in}}%
\pgfpathlineto{\pgfqpoint{1.501988in}{5.442806in}}%
\pgfpathlineto{\pgfqpoint{1.537031in}{5.453820in}}%
\pgfpathlineto{\pgfqpoint{1.543567in}{5.455865in}}%
\pgfpathlineto{\pgfqpoint{1.572073in}{5.464644in}}%
\pgfpathlineto{\pgfqpoint{1.607116in}{5.475419in}}%
\pgfpathlineto{\pgfqpoint{1.628978in}{5.482125in}}%
\pgfpathlineto{\pgfqpoint{1.642158in}{5.486104in}}%
\pgfpathlineto{\pgfqpoint{1.677201in}{5.496645in}}%
\pgfpathlineto{\pgfqpoint{1.712244in}{5.507180in}}%
\pgfpathlineto{\pgfqpoint{1.716269in}{5.508384in}}%
\pgfpathlineto{\pgfqpoint{1.747286in}{5.517517in}}%
\pgfpathlineto{\pgfqpoint{1.782329in}{5.527825in}}%
\pgfpathlineto{\pgfqpoint{1.805561in}{5.534644in}}%
\pgfpathlineto{\pgfqpoint{1.817371in}{5.538055in}}%
\pgfpathlineto{\pgfqpoint{1.852414in}{5.548140in}}%
\pgfpathlineto{\pgfqpoint{1.887456in}{5.558220in}}%
\pgfpathlineto{\pgfqpoint{1.896820in}{5.560903in}}%
\pgfpathlineto{\pgfqpoint{1.922499in}{5.568143in}}%
\pgfpathlineto{\pgfqpoint{1.957541in}{5.578006in}}%
\pgfpathlineto{\pgfqpoint{1.990096in}{5.587163in}}%
\pgfpathlineto{\pgfqpoint{1.992584in}{5.587851in}}%
\pgfpathlineto{\pgfqpoint{2.027626in}{5.597503in}}%
\pgfpathlineto{\pgfqpoint{2.049143in}{5.603426in}}%
\pgfusepath{stroke}%
\end{pgfscope}%
\begin{pgfscope}%
\pgfpathrectangle{\pgfqpoint{0.766095in}{0.571603in}}{\pgfqpoint{6.973465in}{5.225635in}}%
\pgfusepath{clip}%
\pgfsetbuttcap%
\pgfsetroundjoin%
\pgfsetlinewidth{1.505625pt}%
\definecolor{currentstroke}{rgb}{0.157851,0.683765,0.501686}%
\pgfsetstrokecolor{currentstroke}%
\pgfsetdash{}{0pt}%
\pgfpathmoveto{\pgfqpoint{2.427121in}{5.702979in}}%
\pgfpathlineto{\pgfqpoint{2.448137in}{5.708282in}}%
\pgfpathlineto{\pgfqpoint{2.483179in}{5.717122in}}%
\pgfpathlineto{\pgfqpoint{2.488503in}{5.718460in}}%
\pgfpathlineto{\pgfqpoint{2.518222in}{5.725804in}}%
\pgfpathlineto{\pgfqpoint{2.553264in}{5.734457in}}%
\pgfpathlineto{\pgfqpoint{2.588307in}{5.743109in}}%
\pgfpathlineto{\pgfqpoint{2.594858in}{5.744720in}}%
\pgfpathlineto{\pgfqpoint{2.623350in}{5.751611in}}%
\pgfpathlineto{\pgfqpoint{2.658392in}{5.760078in}}%
\pgfpathlineto{\pgfqpoint{2.693435in}{5.768545in}}%
\pgfpathlineto{\pgfqpoint{2.703543in}{5.770979in}}%
\pgfpathlineto{\pgfqpoint{2.728477in}{5.776884in}}%
\pgfpathlineto{\pgfqpoint{2.763520in}{5.785171in}}%
\pgfpathlineto{\pgfqpoint{2.798562in}{5.793457in}}%
\pgfpathlineto{\pgfqpoint{2.814594in}{5.797238in}}%
\pgfusepath{stroke}%
\end{pgfscope}%
\begin{pgfscope}%
\pgfpathrectangle{\pgfqpoint{0.766095in}{0.571603in}}{\pgfqpoint{6.973465in}{5.225635in}}%
\pgfusepath{clip}%
\pgfsetbuttcap%
\pgfsetroundjoin%
\pgfsetlinewidth{1.505625pt}%
\definecolor{currentstroke}{rgb}{0.180653,0.701402,0.488189}%
\pgfsetstrokecolor{currentstroke}%
\pgfsetdash{}{0pt}%
\pgfpathmoveto{\pgfqpoint{0.766095in}{5.225628in}}%
\pgfpathlineto{\pgfqpoint{0.801138in}{5.239051in}}%
\pgfpathlineto{\pgfqpoint{0.818788in}{5.245790in}}%
\pgfpathlineto{\pgfqpoint{0.836180in}{5.252326in}}%
\pgfpathlineto{\pgfqpoint{0.871223in}{5.265449in}}%
\pgfpathlineto{\pgfqpoint{0.888905in}{5.272049in}}%
\pgfpathlineto{\pgfqpoint{0.906265in}{5.278428in}}%
\pgfpathlineto{\pgfqpoint{0.941308in}{5.291259in}}%
\pgfpathlineto{\pgfqpoint{0.960620in}{5.298308in}}%
\pgfpathlineto{\pgfqpoint{0.976350in}{5.303961in}}%
\pgfpathlineto{\pgfqpoint{1.011393in}{5.316508in}}%
\pgfpathlineto{\pgfqpoint{1.033965in}{5.324568in}}%
\pgfpathlineto{\pgfqpoint{1.046435in}{5.328952in}}%
\pgfpathlineto{\pgfqpoint{1.081478in}{5.341221in}}%
\pgfpathlineto{\pgfqpoint{1.108968in}{5.350827in}}%
\pgfpathlineto{\pgfqpoint{1.116520in}{5.353426in}}%
\pgfpathlineto{\pgfqpoint{1.151563in}{5.365425in}}%
\pgfpathlineto{\pgfqpoint{1.185656in}{5.377087in}}%
\pgfpathlineto{\pgfqpoint{1.186605in}{5.377407in}}%
\pgfpathlineto{\pgfqpoint{1.221648in}{5.389142in}}%
\pgfpathlineto{\pgfqpoint{1.256691in}{5.400868in}}%
\pgfpathlineto{\pgfqpoint{1.264131in}{5.403346in}}%
\pgfpathlineto{\pgfqpoint{1.291733in}{5.412397in}}%
\pgfpathlineto{\pgfqpoint{1.326776in}{5.423867in}}%
\pgfpathlineto{\pgfqpoint{1.344365in}{5.429606in}}%
\pgfpathlineto{\pgfqpoint{1.361818in}{5.435211in}}%
\pgfpathlineto{\pgfqpoint{1.396861in}{5.446431in}}%
\pgfpathlineto{\pgfqpoint{1.426369in}{5.455865in}}%
\pgfpathlineto{\pgfqpoint{1.431903in}{5.457607in}}%
\pgfpathlineto{\pgfqpoint{1.466946in}{5.468583in}}%
\pgfpathlineto{\pgfqpoint{1.501988in}{5.479551in}}%
\pgfpathlineto{\pgfqpoint{1.510245in}{5.482125in}}%
\pgfpathlineto{\pgfqpoint{1.537031in}{5.490342in}}%
\pgfpathlineto{\pgfqpoint{1.572073in}{5.501074in}}%
\pgfpathlineto{\pgfqpoint{1.595996in}{5.508384in}}%
\pgfpathlineto{\pgfqpoint{1.607116in}{5.511729in}}%
\pgfpathlineto{\pgfqpoint{1.642158in}{5.522229in}}%
\pgfpathlineto{\pgfqpoint{1.677201in}{5.532723in}}%
\pgfpathlineto{\pgfqpoint{1.683644in}{5.534644in}}%
\pgfpathlineto{\pgfqpoint{1.712244in}{5.543036in}}%
\pgfpathlineto{\pgfqpoint{1.747286in}{5.553304in}}%
\pgfpathlineto{\pgfqpoint{1.773263in}{5.560903in}}%
\pgfpathlineto{\pgfqpoint{1.782329in}{5.563513in}}%
\pgfpathlineto{\pgfqpoint{1.817371in}{5.573561in}}%
\pgfpathlineto{\pgfqpoint{1.852414in}{5.583605in}}%
\pgfpathlineto{\pgfqpoint{1.864871in}{5.587163in}}%
\pgfpathlineto{\pgfqpoint{1.887456in}{5.593511in}}%
\pgfpathlineto{\pgfqpoint{1.922499in}{5.603340in}}%
\pgfpathlineto{\pgfqpoint{1.957541in}{5.613165in}}%
\pgfpathlineto{\pgfqpoint{1.958464in}{5.613422in}}%
\pgfpathlineto{\pgfqpoint{1.992584in}{5.622789in}}%
\pgfpathlineto{\pgfqpoint{2.027626in}{5.632404in}}%
\pgfpathlineto{\pgfqpoint{2.054186in}{5.639682in}}%
\pgfpathlineto{\pgfqpoint{2.062669in}{5.641969in}}%
\pgfpathlineto{\pgfqpoint{2.063040in}{5.642068in}}%
\pgfusepath{stroke}%
\end{pgfscope}%
\begin{pgfscope}%
\pgfpathrectangle{\pgfqpoint{0.766095in}{0.571603in}}{\pgfqpoint{6.973465in}{5.225635in}}%
\pgfusepath{clip}%
\pgfsetbuttcap%
\pgfsetroundjoin%
\pgfsetlinewidth{1.505625pt}%
\definecolor{currentstroke}{rgb}{0.180653,0.701402,0.488189}%
\pgfsetstrokecolor{currentstroke}%
\pgfsetdash{}{0pt}%
\pgfpathmoveto{\pgfqpoint{2.441328in}{5.740429in}}%
\pgfpathlineto{\pgfqpoint{2.448137in}{5.742142in}}%
\pgfpathlineto{\pgfqpoint{2.458414in}{5.744720in}}%
\pgfpathlineto{\pgfqpoint{2.483179in}{5.750828in}}%
\pgfpathlineto{\pgfqpoint{2.518222in}{5.759459in}}%
\pgfpathlineto{\pgfqpoint{2.553264in}{5.768089in}}%
\pgfpathlineto{\pgfqpoint{2.565036in}{5.770979in}}%
\pgfpathlineto{\pgfqpoint{2.588307in}{5.776599in}}%
\pgfpathlineto{\pgfqpoint{2.623350in}{5.785047in}}%
\pgfpathlineto{\pgfqpoint{2.658392in}{5.793494in}}%
\pgfpathlineto{\pgfqpoint{2.673967in}{5.797238in}}%
\pgfusepath{stroke}%
\end{pgfscope}%
\begin{pgfscope}%
\pgfpathrectangle{\pgfqpoint{0.766095in}{0.571603in}}{\pgfqpoint{6.973465in}{5.225635in}}%
\pgfusepath{clip}%
\pgfsetbuttcap%
\pgfsetroundjoin%
\pgfsetlinewidth{1.505625pt}%
\definecolor{currentstroke}{rgb}{0.208030,0.718701,0.472873}%
\pgfsetstrokecolor{currentstroke}%
\pgfsetdash{}{0pt}%
\pgfpathmoveto{\pgfqpoint{0.766095in}{5.264967in}}%
\pgfpathlineto{\pgfqpoint{0.784745in}{5.272049in}}%
\pgfpathlineto{\pgfqpoint{0.801138in}{5.278177in}}%
\pgfpathlineto{\pgfqpoint{0.836180in}{5.291231in}}%
\pgfpathlineto{\pgfqpoint{0.855240in}{5.298308in}}%
\pgfpathlineto{\pgfqpoint{0.871223in}{5.304151in}}%
\pgfpathlineto{\pgfqpoint{0.906265in}{5.316915in}}%
\pgfpathlineto{\pgfqpoint{0.927335in}{5.324568in}}%
\pgfpathlineto{\pgfqpoint{0.941308in}{5.329564in}}%
\pgfpathlineto{\pgfqpoint{0.976350in}{5.342046in}}%
\pgfpathlineto{\pgfqpoint{1.001061in}{5.350827in}}%
\pgfpathlineto{\pgfqpoint{1.011393in}{5.354442in}}%
\pgfpathlineto{\pgfqpoint{1.046435in}{5.366649in}}%
\pgfpathlineto{\pgfqpoint{1.076445in}{5.377087in}}%
\pgfpathlineto{\pgfqpoint{1.081478in}{5.378810in}}%
\pgfpathlineto{\pgfqpoint{1.116520in}{5.390750in}}%
\pgfpathlineto{\pgfqpoint{1.151563in}{5.402679in}}%
\pgfpathlineto{\pgfqpoint{1.153535in}{5.403346in}}%
\pgfpathlineto{\pgfqpoint{1.186605in}{5.414371in}}%
\pgfpathlineto{\pgfqpoint{1.221648in}{5.426039in}}%
\pgfpathlineto{\pgfqpoint{1.232402in}{5.429606in}}%
\pgfpathlineto{\pgfqpoint{1.256691in}{5.437535in}}%
\pgfpathlineto{\pgfqpoint{1.291733in}{5.448950in}}%
\pgfpathlineto{\pgfqpoint{1.313016in}{5.455865in}}%
\pgfpathlineto{\pgfqpoint{1.326776in}{5.460266in}}%
\pgfpathlineto{\pgfqpoint{1.361818in}{5.471433in}}%
\pgfpathlineto{\pgfqpoint{1.395398in}{5.482125in}}%
\pgfpathlineto{\pgfqpoint{1.396861in}{5.482583in}}%
\pgfpathlineto{\pgfqpoint{1.431903in}{5.493509in}}%
\pgfpathlineto{\pgfqpoint{1.466946in}{5.504428in}}%
\pgfpathlineto{\pgfqpoint{1.479689in}{5.508384in}}%
\pgfpathlineto{\pgfqpoint{1.501988in}{5.515199in}}%
\pgfpathlineto{\pgfqpoint{1.537031in}{5.525882in}}%
\pgfpathlineto{\pgfqpoint{1.565809in}{5.534644in}}%
\pgfpathlineto{\pgfqpoint{1.572073in}{5.536521in}}%
\pgfpathlineto{\pgfqpoint{1.607116in}{5.546975in}}%
\pgfpathlineto{\pgfqpoint{1.642158in}{5.557424in}}%
\pgfpathlineto{\pgfqpoint{1.653870in}{5.560903in}}%
\pgfpathlineto{\pgfqpoint{1.677201in}{5.567725in}}%
\pgfpathlineto{\pgfqpoint{1.712244in}{5.577950in}}%
\pgfpathlineto{\pgfqpoint{1.743845in}{5.587163in}}%
\pgfpathlineto{\pgfqpoint{1.747286in}{5.588150in}}%
\pgfpathlineto{\pgfqpoint{1.782329in}{5.598157in}}%
\pgfpathlineto{\pgfqpoint{1.817371in}{5.608159in}}%
\pgfpathlineto{\pgfqpoint{1.835857in}{5.613422in}}%
\pgfpathlineto{\pgfqpoint{1.852414in}{5.618061in}}%
\pgfpathlineto{\pgfqpoint{1.887456in}{5.627851in}}%
\pgfpathlineto{\pgfqpoint{1.922499in}{5.637637in}}%
\pgfpathlineto{\pgfqpoint{1.929850in}{5.639682in}}%
\pgfpathlineto{\pgfqpoint{1.957541in}{5.647261in}}%
\pgfpathlineto{\pgfqpoint{1.992584in}{5.656840in}}%
\pgfpathlineto{\pgfqpoint{2.025899in}{5.665941in}}%
\pgfpathlineto{\pgfqpoint{2.027626in}{5.666406in}}%
\pgfpathlineto{\pgfqpoint{2.048903in}{5.672098in}}%
\pgfusepath{stroke}%
\end{pgfscope}%
\begin{pgfscope}%
\pgfpathrectangle{\pgfqpoint{0.766095in}{0.571603in}}{\pgfqpoint{6.973465in}{5.225635in}}%
\pgfusepath{clip}%
\pgfsetbuttcap%
\pgfsetroundjoin%
\pgfsetlinewidth{1.505625pt}%
\definecolor{currentstroke}{rgb}{0.208030,0.718701,0.472873}%
\pgfsetstrokecolor{currentstroke}%
\pgfsetdash{}{0pt}%
\pgfpathmoveto{\pgfqpoint{2.427330in}{5.769917in}}%
\pgfpathlineto{\pgfqpoint{2.431576in}{5.770979in}}%
\pgfpathlineto{\pgfqpoint{2.448137in}{5.775056in}}%
\pgfpathlineto{\pgfqpoint{2.483179in}{5.783661in}}%
\pgfpathlineto{\pgfqpoint{2.518222in}{5.792264in}}%
\pgfpathlineto{\pgfqpoint{2.538522in}{5.797238in}}%
\pgfusepath{stroke}%
\end{pgfscope}%
\begin{pgfscope}%
\pgfpathrectangle{\pgfqpoint{0.766095in}{0.571603in}}{\pgfqpoint{6.973465in}{5.225635in}}%
\pgfusepath{clip}%
\pgfsetbuttcap%
\pgfsetroundjoin%
\pgfsetlinewidth{1.505625pt}%
\definecolor{currentstroke}{rgb}{0.239374,0.735588,0.455688}%
\pgfsetstrokecolor{currentstroke}%
\pgfsetdash{}{0pt}%
\pgfpathmoveto{\pgfqpoint{0.766095in}{5.303164in}}%
\pgfpathlineto{\pgfqpoint{0.801138in}{5.316144in}}%
\pgfpathlineto{\pgfqpoint{0.823940in}{5.324568in}}%
\pgfpathlineto{\pgfqpoint{0.836180in}{5.329020in}}%
\pgfpathlineto{\pgfqpoint{0.862063in}{5.338395in}}%
\pgfusepath{stroke}%
\end{pgfscope}%
\begin{pgfscope}%
\pgfpathrectangle{\pgfqpoint{0.766095in}{0.571603in}}{\pgfqpoint{6.973465in}{5.225635in}}%
\pgfusepath{clip}%
\pgfsetbuttcap%
\pgfsetroundjoin%
\pgfsetlinewidth{1.505625pt}%
\definecolor{currentstroke}{rgb}{0.239374,0.735588,0.455688}%
\pgfsetstrokecolor{currentstroke}%
\pgfsetdash{}{0pt}%
\pgfpathmoveto{\pgfqpoint{1.231880in}{5.465177in}}%
\pgfpathlineto{\pgfqpoint{1.256691in}{5.473218in}}%
\pgfpathlineto{\pgfqpoint{1.284224in}{5.482125in}}%
\pgfpathlineto{\pgfqpoint{1.291733in}{5.484516in}}%
\pgfpathlineto{\pgfqpoint{1.326776in}{5.495628in}}%
\pgfpathlineto{\pgfqpoint{1.361818in}{5.506731in}}%
\pgfpathlineto{\pgfqpoint{1.367060in}{5.508384in}}%
\pgfpathlineto{\pgfqpoint{1.396861in}{5.517637in}}%
\pgfpathlineto{\pgfqpoint{1.431903in}{5.528501in}}%
\pgfpathlineto{\pgfqpoint{1.451766in}{5.534644in}}%
\pgfpathlineto{\pgfqpoint{1.466946in}{5.539264in}}%
\pgfpathlineto{\pgfqpoint{1.501988in}{5.549896in}}%
\pgfpathlineto{\pgfqpoint{1.537031in}{5.560522in}}%
\pgfpathlineto{\pgfqpoint{1.538294in}{5.560903in}}%
\pgfpathlineto{\pgfqpoint{1.572073in}{5.570935in}}%
\pgfpathlineto{\pgfqpoint{1.607116in}{5.581334in}}%
\pgfpathlineto{\pgfqpoint{1.626807in}{5.587163in}}%
\pgfpathlineto{\pgfqpoint{1.642158in}{5.591635in}}%
\pgfpathlineto{\pgfqpoint{1.677201in}{5.601813in}}%
\pgfpathlineto{\pgfqpoint{1.712244in}{5.611986in}}%
\pgfpathlineto{\pgfqpoint{1.717212in}{5.613422in}}%
\pgfpathlineto{\pgfqpoint{1.747286in}{5.621978in}}%
\pgfpathlineto{\pgfqpoint{1.782329in}{5.631935in}}%
\pgfpathlineto{\pgfqpoint{1.809631in}{5.639682in}}%
\pgfpathlineto{\pgfqpoint{1.817371in}{5.641843in}}%
\pgfpathlineto{\pgfqpoint{1.852414in}{5.651590in}}%
\pgfpathlineto{\pgfqpoint{1.887456in}{5.661333in}}%
\pgfpathlineto{\pgfqpoint{1.904075in}{5.665941in}}%
\pgfpathlineto{\pgfqpoint{1.922499in}{5.670969in}}%
\pgfpathlineto{\pgfqpoint{1.957541in}{5.680507in}}%
\pgfpathlineto{\pgfqpoint{1.992584in}{5.690041in}}%
\pgfpathlineto{\pgfqpoint{2.000551in}{5.692201in}}%
\pgfpathlineto{\pgfqpoint{2.027626in}{5.699423in}}%
\pgfpathlineto{\pgfqpoint{2.062669in}{5.708757in}}%
\pgfpathlineto{\pgfqpoint{2.097711in}{5.718089in}}%
\pgfpathlineto{\pgfqpoint{2.099110in}{5.718460in}}%
\pgfpathlineto{\pgfqpoint{2.132754in}{5.727236in}}%
\pgfpathlineto{\pgfqpoint{2.167797in}{5.736373in}}%
\pgfpathlineto{\pgfqpoint{2.199830in}{5.744720in}}%
\pgfpathlineto{\pgfqpoint{2.202839in}{5.745491in}}%
\pgfpathlineto{\pgfqpoint{2.237882in}{5.754437in}}%
\pgfpathlineto{\pgfqpoint{2.272924in}{5.763381in}}%
\pgfpathlineto{\pgfqpoint{2.302720in}{5.770979in}}%
\pgfpathlineto{\pgfqpoint{2.307967in}{5.772296in}}%
\pgfpathlineto{\pgfqpoint{2.343009in}{5.781053in}}%
\pgfpathlineto{\pgfqpoint{2.378052in}{5.789809in}}%
\pgfpathlineto{\pgfqpoint{2.407808in}{5.797238in}}%
\pgfusepath{stroke}%
\end{pgfscope}%
\begin{pgfscope}%
\pgfpathrectangle{\pgfqpoint{0.766095in}{0.571603in}}{\pgfqpoint{6.973465in}{5.225635in}}%
\pgfusepath{clip}%
\pgfsetbuttcap%
\pgfsetroundjoin%
\pgfsetlinewidth{1.505625pt}%
\definecolor{currentstroke}{rgb}{0.274149,0.751988,0.436601}%
\pgfsetstrokecolor{currentstroke}%
\pgfsetdash{}{0pt}%
\pgfpathmoveto{\pgfqpoint{0.766095in}{5.340240in}}%
\pgfpathlineto{\pgfqpoint{0.794904in}{5.350827in}}%
\pgfpathlineto{\pgfqpoint{0.801138in}{5.353083in}}%
\pgfpathlineto{\pgfqpoint{0.836180in}{5.365702in}}%
\pgfpathlineto{\pgfqpoint{0.867842in}{5.377087in}}%
\pgfpathlineto{\pgfqpoint{0.871223in}{5.378284in}}%
\pgfpathlineto{\pgfqpoint{0.906265in}{5.390626in}}%
\pgfpathlineto{\pgfqpoint{0.941308in}{5.402956in}}%
\pgfpathlineto{\pgfqpoint{0.942423in}{5.403346in}}%
\pgfpathlineto{\pgfqpoint{0.976350in}{5.415037in}}%
\pgfpathlineto{\pgfqpoint{1.011393in}{5.427098in}}%
\pgfpathlineto{\pgfqpoint{1.018713in}{5.429606in}}%
\pgfpathlineto{\pgfqpoint{1.046435in}{5.438958in}}%
\pgfpathlineto{\pgfqpoint{1.081478in}{5.450757in}}%
\pgfpathlineto{\pgfqpoint{1.096703in}{5.455865in}}%
\pgfpathlineto{\pgfqpoint{1.116520in}{5.462412in}}%
\pgfpathlineto{\pgfqpoint{1.151563in}{5.473956in}}%
\pgfpathlineto{\pgfqpoint{1.176414in}{5.482125in}}%
\pgfpathlineto{\pgfqpoint{1.186605in}{5.485423in}}%
\pgfpathlineto{\pgfqpoint{1.221648in}{5.496718in}}%
\pgfpathlineto{\pgfqpoint{1.256691in}{5.508003in}}%
\pgfpathlineto{\pgfqpoint{1.257879in}{5.508384in}}%
\pgfpathlineto{\pgfqpoint{1.291733in}{5.519063in}}%
\pgfpathlineto{\pgfqpoint{1.326776in}{5.530107in}}%
\pgfpathlineto{\pgfqpoint{1.341221in}{5.534644in}}%
\pgfpathlineto{\pgfqpoint{1.361818in}{5.541013in}}%
\pgfpathlineto{\pgfqpoint{1.396861in}{5.551820in}}%
\pgfpathlineto{\pgfqpoint{1.426352in}{5.560903in}}%
\pgfpathlineto{\pgfqpoint{1.431903in}{5.562586in}}%
\pgfpathlineto{\pgfqpoint{1.454739in}{5.569479in}}%
\pgfusepath{stroke}%
\end{pgfscope}%
\begin{pgfscope}%
\pgfpathrectangle{\pgfqpoint{0.766095in}{0.571603in}}{\pgfqpoint{6.973465in}{5.225635in}}%
\pgfusepath{clip}%
\pgfsetbuttcap%
\pgfsetroundjoin%
\pgfsetlinewidth{1.505625pt}%
\definecolor{currentstroke}{rgb}{0.274149,0.751988,0.436601}%
\pgfsetstrokecolor{currentstroke}%
\pgfsetdash{}{0pt}%
\pgfpathmoveto{\pgfqpoint{1.830220in}{5.678161in}}%
\pgfpathlineto{\pgfqpoint{1.852414in}{5.684302in}}%
\pgfpathlineto{\pgfqpoint{1.880994in}{5.692201in}}%
\pgfpathlineto{\pgfqpoint{1.887456in}{5.693958in}}%
\pgfpathlineto{\pgfqpoint{1.922499in}{5.703452in}}%
\pgfpathlineto{\pgfqpoint{1.957541in}{5.712942in}}%
\pgfpathlineto{\pgfqpoint{1.977961in}{5.718460in}}%
\pgfpathlineto{\pgfqpoint{1.992584in}{5.722349in}}%
\pgfpathlineto{\pgfqpoint{2.027626in}{5.731641in}}%
\pgfpathlineto{\pgfqpoint{2.062669in}{5.740931in}}%
\pgfpathlineto{\pgfqpoint{2.077003in}{5.744720in}}%
\pgfpathlineto{\pgfqpoint{2.097711in}{5.750106in}}%
\pgfpathlineto{\pgfqpoint{2.132754in}{5.759203in}}%
\pgfpathlineto{\pgfqpoint{2.167797in}{5.768297in}}%
\pgfpathlineto{\pgfqpoint{2.178166in}{5.770979in}}%
\pgfpathlineto{\pgfqpoint{2.202839in}{5.777259in}}%
\pgfpathlineto{\pgfqpoint{2.237882in}{5.786165in}}%
\pgfpathlineto{\pgfqpoint{2.272924in}{5.795069in}}%
\pgfpathlineto{\pgfqpoint{2.281495in}{5.797238in}}%
\pgfusepath{stroke}%
\end{pgfscope}%
\begin{pgfscope}%
\pgfpathrectangle{\pgfqpoint{0.766095in}{0.571603in}}{\pgfqpoint{6.973465in}{5.225635in}}%
\pgfusepath{clip}%
\pgfsetbuttcap%
\pgfsetroundjoin%
\pgfsetlinewidth{1.505625pt}%
\definecolor{currentstroke}{rgb}{0.311925,0.767822,0.415586}%
\pgfsetstrokecolor{currentstroke}%
\pgfsetdash{}{0pt}%
\pgfpathmoveto{\pgfqpoint{0.766095in}{5.376375in}}%
\pgfpathlineto{\pgfqpoint{0.768054in}{5.377087in}}%
\pgfpathlineto{\pgfqpoint{0.801138in}{5.388931in}}%
\pgfpathlineto{\pgfqpoint{0.836180in}{5.401460in}}%
\pgfpathlineto{\pgfqpoint{0.841482in}{5.403346in}}%
\pgfpathlineto{\pgfqpoint{0.871223in}{5.413765in}}%
\pgfpathlineto{\pgfqpoint{0.906265in}{5.426021in}}%
\pgfpathlineto{\pgfqpoint{0.916559in}{5.429606in}}%
\pgfpathlineto{\pgfqpoint{0.941308in}{5.438093in}}%
\pgfpathlineto{\pgfqpoint{0.976350in}{5.450082in}}%
\pgfpathlineto{\pgfqpoint{0.993311in}{5.455865in}}%
\pgfpathlineto{\pgfqpoint{1.011393in}{5.461937in}}%
\pgfpathlineto{\pgfqpoint{1.046435in}{5.473666in}}%
\pgfpathlineto{\pgfqpoint{1.071761in}{5.482125in}}%
\pgfpathlineto{\pgfqpoint{1.081478in}{5.485320in}}%
\pgfpathlineto{\pgfqpoint{1.116520in}{5.496797in}}%
\pgfpathlineto{\pgfqpoint{1.151563in}{5.508264in}}%
\pgfpathlineto{\pgfqpoint{1.151934in}{5.508384in}}%
\pgfpathlineto{\pgfqpoint{1.186605in}{5.519496in}}%
\pgfpathlineto{\pgfqpoint{1.221648in}{5.530717in}}%
\pgfpathlineto{\pgfqpoint{1.233956in}{5.534644in}}%
\pgfpathlineto{\pgfqpoint{1.256691in}{5.541785in}}%
\pgfpathlineto{\pgfqpoint{1.291733in}{5.552766in}}%
\pgfpathlineto{\pgfqpoint{1.317750in}{5.560903in}}%
\pgfpathlineto{\pgfqpoint{1.326776in}{5.563682in}}%
\pgfpathlineto{\pgfqpoint{1.361818in}{5.574430in}}%
\pgfpathlineto{\pgfqpoint{1.396861in}{5.585170in}}%
\pgfpathlineto{\pgfqpoint{1.403391in}{5.587163in}}%
\pgfpathlineto{\pgfqpoint{1.431903in}{5.595729in}}%
\pgfpathlineto{\pgfqpoint{1.454580in}{5.602531in}}%
\pgfusepath{stroke}%
\end{pgfscope}%
\begin{pgfscope}%
\pgfpathrectangle{\pgfqpoint{0.766095in}{0.571603in}}{\pgfqpoint{6.973465in}{5.225635in}}%
\pgfusepath{clip}%
\pgfsetbuttcap%
\pgfsetroundjoin%
\pgfsetlinewidth{1.505625pt}%
\definecolor{currentstroke}{rgb}{0.311925,0.767822,0.415586}%
\pgfsetstrokecolor{currentstroke}%
\pgfsetdash{}{0pt}%
\pgfpathmoveto{\pgfqpoint{1.830361in}{5.710161in}}%
\pgfpathlineto{\pgfqpoint{1.852414in}{5.716229in}}%
\pgfpathlineto{\pgfqpoint{1.860550in}{5.718460in}}%
\pgfpathlineto{\pgfqpoint{1.887456in}{5.725720in}}%
\pgfpathlineto{\pgfqpoint{1.922499in}{5.735163in}}%
\pgfpathlineto{\pgfqpoint{1.957541in}{5.744602in}}%
\pgfpathlineto{\pgfqpoint{1.957978in}{5.744720in}}%
\pgfpathlineto{\pgfqpoint{1.992584in}{5.753851in}}%
\pgfpathlineto{\pgfqpoint{2.027626in}{5.763095in}}%
\pgfpathlineto{\pgfqpoint{2.057541in}{5.770979in}}%
\pgfpathlineto{\pgfqpoint{2.062669in}{5.772309in}}%
\pgfpathlineto{\pgfqpoint{2.097711in}{5.781362in}}%
\pgfpathlineto{\pgfqpoint{2.132754in}{5.790412in}}%
\pgfpathlineto{\pgfqpoint{2.159218in}{5.797238in}}%
\pgfusepath{stroke}%
\end{pgfscope}%
\begin{pgfscope}%
\pgfpathrectangle{\pgfqpoint{0.766095in}{0.571603in}}{\pgfqpoint{6.973465in}{5.225635in}}%
\pgfusepath{clip}%
\pgfsetbuttcap%
\pgfsetroundjoin%
\pgfsetlinewidth{1.505625pt}%
\definecolor{currentstroke}{rgb}{0.352360,0.783011,0.392636}%
\pgfsetstrokecolor{currentstroke}%
\pgfsetdash{}{0pt}%
\pgfpathmoveto{\pgfqpoint{0.766095in}{5.411445in}}%
\pgfpathlineto{\pgfqpoint{0.801138in}{5.423894in}}%
\pgfpathlineto{\pgfqpoint{0.817272in}{5.429606in}}%
\pgfpathlineto{\pgfqpoint{0.836180in}{5.436197in}}%
\pgfpathlineto{\pgfqpoint{0.871223in}{5.448376in}}%
\pgfpathlineto{\pgfqpoint{0.892833in}{5.455865in}}%
\pgfpathlineto{\pgfqpoint{0.906265in}{5.460449in}}%
\pgfpathlineto{\pgfqpoint{0.941308in}{5.472364in}}%
\pgfpathlineto{\pgfqpoint{0.970068in}{5.482125in}}%
\pgfpathlineto{\pgfqpoint{0.976350in}{5.484224in}}%
\pgfpathlineto{\pgfqpoint{1.011393in}{5.495882in}}%
\pgfpathlineto{\pgfqpoint{1.046435in}{5.507528in}}%
\pgfpathlineto{\pgfqpoint{1.049024in}{5.508384in}}%
\pgfpathlineto{\pgfqpoint{1.081478in}{5.518952in}}%
\pgfpathlineto{\pgfqpoint{1.116520in}{5.530349in}}%
\pgfpathlineto{\pgfqpoint{1.129773in}{5.534644in}}%
\pgfpathlineto{\pgfqpoint{1.151563in}{5.541596in}}%
\pgfpathlineto{\pgfqpoint{1.186605in}{5.552750in}}%
\pgfpathlineto{\pgfqpoint{1.212274in}{5.560903in}}%
\pgfpathlineto{\pgfqpoint{1.221648in}{5.563835in}}%
\pgfpathlineto{\pgfqpoint{1.256691in}{5.574751in}}%
\pgfpathlineto{\pgfqpoint{1.291733in}{5.585659in}}%
\pgfpathlineto{\pgfqpoint{1.296587in}{5.587163in}}%
\pgfpathlineto{\pgfqpoint{1.326776in}{5.596373in}}%
\pgfpathlineto{\pgfqpoint{1.361818in}{5.607050in}}%
\pgfpathlineto{\pgfqpoint{1.382784in}{5.613422in}}%
\pgfpathlineto{\pgfqpoint{1.396861in}{5.617634in}}%
\pgfpathlineto{\pgfqpoint{1.431903in}{5.628086in}}%
\pgfpathlineto{\pgfqpoint{1.464990in}{5.637948in}}%
\pgfusepath{stroke}%
\end{pgfscope}%
\begin{pgfscope}%
\pgfpathrectangle{\pgfqpoint{0.766095in}{0.571603in}}{\pgfqpoint{6.973465in}{5.225635in}}%
\pgfusepath{clip}%
\pgfsetbuttcap%
\pgfsetroundjoin%
\pgfsetlinewidth{1.505625pt}%
\definecolor{currentstroke}{rgb}{0.352360,0.783011,0.392636}%
\pgfsetstrokecolor{currentstroke}%
\pgfsetdash{}{0pt}%
\pgfpathmoveto{\pgfqpoint{1.841129in}{5.744311in}}%
\pgfpathlineto{\pgfqpoint{1.842622in}{5.744720in}}%
\pgfpathlineto{\pgfqpoint{1.852414in}{5.747353in}}%
\pgfpathlineto{\pgfqpoint{1.887456in}{5.756745in}}%
\pgfpathlineto{\pgfqpoint{1.922499in}{5.766134in}}%
\pgfpathlineto{\pgfqpoint{1.940626in}{5.770979in}}%
\pgfpathlineto{\pgfqpoint{1.957541in}{5.775429in}}%
\pgfpathlineto{\pgfqpoint{1.992584in}{5.784624in}}%
\pgfpathlineto{\pgfqpoint{2.027626in}{5.793817in}}%
\pgfpathlineto{\pgfqpoint{2.040708in}{5.797238in}}%
\pgfusepath{stroke}%
\end{pgfscope}%
\begin{pgfscope}%
\pgfpathrectangle{\pgfqpoint{0.766095in}{0.571603in}}{\pgfqpoint{6.973465in}{5.225635in}}%
\pgfusepath{clip}%
\pgfsetbuttcap%
\pgfsetroundjoin%
\pgfsetlinewidth{1.505625pt}%
\definecolor{currentstroke}{rgb}{0.395174,0.797475,0.367757}%
\pgfsetstrokecolor{currentstroke}%
\pgfsetdash{}{0pt}%
\pgfpathmoveto{\pgfqpoint{0.766095in}{5.445653in}}%
\pgfpathlineto{\pgfqpoint{0.795087in}{5.455865in}}%
\pgfpathlineto{\pgfqpoint{0.801138in}{5.457964in}}%
\pgfpathlineto{\pgfqpoint{0.836180in}{5.470063in}}%
\pgfpathlineto{\pgfqpoint{0.871152in}{5.482125in}}%
\pgfpathlineto{\pgfqpoint{0.871223in}{5.482149in}}%
\pgfpathlineto{\pgfqpoint{0.906265in}{5.493987in}}%
\pgfpathlineto{\pgfqpoint{0.941308in}{5.505812in}}%
\pgfpathlineto{\pgfqpoint{0.948963in}{5.508384in}}%
\pgfpathlineto{\pgfqpoint{0.976350in}{5.517446in}}%
\pgfpathlineto{\pgfqpoint{1.011393in}{5.529018in}}%
\pgfpathlineto{\pgfqpoint{1.028484in}{5.534644in}}%
\pgfpathlineto{\pgfqpoint{1.046435in}{5.540463in}}%
\pgfpathlineto{\pgfqpoint{1.081478in}{5.551787in}}%
\pgfpathlineto{\pgfqpoint{1.109734in}{5.560903in}}%
\pgfpathlineto{\pgfqpoint{1.116520in}{5.563059in}}%
\pgfpathlineto{\pgfqpoint{1.151563in}{5.574143in}}%
\pgfpathlineto{\pgfqpoint{1.186605in}{5.585217in}}%
\pgfpathlineto{\pgfqpoint{1.192790in}{5.587163in}}%
\pgfpathlineto{\pgfqpoint{1.221648in}{5.596104in}}%
\pgfpathlineto{\pgfqpoint{1.256691in}{5.606944in}}%
\pgfpathlineto{\pgfqpoint{1.277686in}{5.613422in}}%
\pgfpathlineto{\pgfqpoint{1.291733in}{5.617690in}}%
\pgfpathlineto{\pgfqpoint{1.326776in}{5.628301in}}%
\pgfpathlineto{\pgfqpoint{1.361818in}{5.638905in}}%
\pgfpathlineto{\pgfqpoint{1.364396in}{5.639682in}}%
\pgfpathlineto{\pgfqpoint{1.396861in}{5.649309in}}%
\pgfpathlineto{\pgfqpoint{1.431903in}{5.659690in}}%
\pgfpathlineto{\pgfqpoint{1.453052in}{5.665941in}}%
\pgfpathlineto{\pgfqpoint{1.454276in}{5.666297in}}%
\pgfusepath{stroke}%
\end{pgfscope}%
\begin{pgfscope}%
\pgfpathrectangle{\pgfqpoint{0.766095in}{0.571603in}}{\pgfqpoint{6.973465in}{5.225635in}}%
\pgfusepath{clip}%
\pgfsetbuttcap%
\pgfsetroundjoin%
\pgfsetlinewidth{1.505625pt}%
\definecolor{currentstroke}{rgb}{0.395174,0.797475,0.367757}%
\pgfsetstrokecolor{currentstroke}%
\pgfsetdash{}{0pt}%
\pgfpathmoveto{\pgfqpoint{1.830623in}{5.771914in}}%
\pgfpathlineto{\pgfqpoint{1.852414in}{5.777729in}}%
\pgfpathlineto{\pgfqpoint{1.887456in}{5.787064in}}%
\pgfpathlineto{\pgfqpoint{1.922499in}{5.796397in}}%
\pgfpathlineto{\pgfqpoint{1.925673in}{5.797238in}}%
\pgfusepath{stroke}%
\end{pgfscope}%
\begin{pgfscope}%
\pgfpathrectangle{\pgfqpoint{0.766095in}{0.571603in}}{\pgfqpoint{6.973465in}{5.225635in}}%
\pgfusepath{clip}%
\pgfsetbuttcap%
\pgfsetroundjoin%
\pgfsetlinewidth{1.505625pt}%
\definecolor{currentstroke}{rgb}{0.440137,0.811138,0.340967}%
\pgfsetstrokecolor{currentstroke}%
\pgfsetdash{}{0pt}%
\pgfpathmoveto{\pgfqpoint{0.766095in}{5.479046in}}%
\pgfpathlineto{\pgfqpoint{0.774927in}{5.482125in}}%
\pgfpathlineto{\pgfqpoint{0.801138in}{5.491124in}}%
\pgfpathlineto{\pgfqpoint{0.836180in}{5.503128in}}%
\pgfpathlineto{\pgfqpoint{0.851576in}{5.508384in}}%
\pgfpathlineto{\pgfqpoint{0.860772in}{5.511476in}}%
\pgfusepath{stroke}%
\end{pgfscope}%
\begin{pgfscope}%
\pgfpathrectangle{\pgfqpoint{0.766095in}{0.571603in}}{\pgfqpoint{6.973465in}{5.225635in}}%
\pgfusepath{clip}%
\pgfsetbuttcap%
\pgfsetroundjoin%
\pgfsetlinewidth{1.505625pt}%
\definecolor{currentstroke}{rgb}{0.440137,0.811138,0.340967}%
\pgfsetstrokecolor{currentstroke}%
\pgfsetdash{}{0pt}%
\pgfpathmoveto{\pgfqpoint{1.232944in}{5.631101in}}%
\pgfpathlineto{\pgfqpoint{1.256691in}{5.638394in}}%
\pgfpathlineto{\pgfqpoint{1.260904in}{5.639682in}}%
\pgfpathlineto{\pgfqpoint{1.291733in}{5.648963in}}%
\pgfpathlineto{\pgfqpoint{1.326776in}{5.659499in}}%
\pgfpathlineto{\pgfqpoint{1.348253in}{5.665941in}}%
\pgfpathlineto{\pgfqpoint{1.361818in}{5.669948in}}%
\pgfpathlineto{\pgfqpoint{1.396861in}{5.680264in}}%
\pgfpathlineto{\pgfqpoint{1.431903in}{5.690573in}}%
\pgfpathlineto{\pgfqpoint{1.437458in}{5.692201in}}%
\pgfpathlineto{\pgfqpoint{1.466946in}{5.700706in}}%
\pgfpathlineto{\pgfqpoint{1.501988in}{5.710801in}}%
\pgfpathlineto{\pgfqpoint{1.528616in}{5.718460in}}%
\pgfpathlineto{\pgfqpoint{1.537031in}{5.720843in}}%
\pgfpathlineto{\pgfqpoint{1.572073in}{5.730729in}}%
\pgfpathlineto{\pgfqpoint{1.607116in}{5.740610in}}%
\pgfpathlineto{\pgfqpoint{1.621736in}{5.744720in}}%
\pgfpathlineto{\pgfqpoint{1.642158in}{5.750372in}}%
\pgfpathlineto{\pgfqpoint{1.677201in}{5.760049in}}%
\pgfpathlineto{\pgfqpoint{1.712244in}{5.769721in}}%
\pgfpathlineto{\pgfqpoint{1.716819in}{5.770979in}}%
\pgfpathlineto{\pgfqpoint{1.747286in}{5.779224in}}%
\pgfpathlineto{\pgfqpoint{1.782329in}{5.788698in}}%
\pgfpathlineto{\pgfqpoint{1.813943in}{5.797238in}}%
\pgfusepath{stroke}%
\end{pgfscope}%
\begin{pgfscope}%
\pgfpathrectangle{\pgfqpoint{0.766095in}{0.571603in}}{\pgfqpoint{6.973465in}{5.225635in}}%
\pgfusepath{clip}%
\pgfsetbuttcap%
\pgfsetroundjoin%
\pgfsetlinewidth{1.505625pt}%
\definecolor{currentstroke}{rgb}{0.487026,0.823929,0.312321}%
\pgfsetstrokecolor{currentstroke}%
\pgfsetdash{}{0pt}%
\pgfpathmoveto{\pgfqpoint{0.766095in}{5.511594in}}%
\pgfpathlineto{\pgfqpoint{0.801138in}{5.523516in}}%
\pgfpathlineto{\pgfqpoint{0.833892in}{5.534644in}}%
\pgfpathlineto{\pgfqpoint{0.836180in}{5.535409in}}%
\pgfpathlineto{\pgfqpoint{0.871223in}{5.547075in}}%
\pgfpathlineto{\pgfqpoint{0.889934in}{5.553298in}}%
\pgfusepath{stroke}%
\end{pgfscope}%
\begin{pgfscope}%
\pgfpathrectangle{\pgfqpoint{0.766095in}{0.571603in}}{\pgfqpoint{6.973465in}{5.225635in}}%
\pgfusepath{clip}%
\pgfsetbuttcap%
\pgfsetroundjoin%
\pgfsetlinewidth{1.505625pt}%
\definecolor{currentstroke}{rgb}{0.487026,0.823929,0.312321}%
\pgfsetstrokecolor{currentstroke}%
\pgfsetdash{}{0pt}%
\pgfpathmoveto{\pgfqpoint{1.262751in}{5.670880in}}%
\pgfpathlineto{\pgfqpoint{1.291733in}{5.679536in}}%
\pgfpathlineto{\pgfqpoint{1.326776in}{5.689995in}}%
\pgfpathlineto{\pgfqpoint{1.334194in}{5.692201in}}%
\pgfpathlineto{\pgfqpoint{1.361818in}{5.700287in}}%
\pgfpathlineto{\pgfqpoint{1.396861in}{5.710529in}}%
\pgfpathlineto{\pgfqpoint{1.424041in}{5.718460in}}%
\pgfpathlineto{\pgfqpoint{1.431903in}{5.720719in}}%
\pgfpathlineto{\pgfqpoint{1.466946in}{5.730749in}}%
\pgfpathlineto{\pgfqpoint{1.501988in}{5.740773in}}%
\pgfpathlineto{\pgfqpoint{1.515830in}{5.744720in}}%
\pgfpathlineto{\pgfqpoint{1.537031in}{5.750672in}}%
\pgfpathlineto{\pgfqpoint{1.572073in}{5.760489in}}%
\pgfpathlineto{\pgfqpoint{1.607116in}{5.770301in}}%
\pgfpathlineto{\pgfqpoint{1.609547in}{5.770979in}}%
\pgfpathlineto{\pgfqpoint{1.642158in}{5.779930in}}%
\pgfpathlineto{\pgfqpoint{1.677201in}{5.789541in}}%
\pgfpathlineto{\pgfqpoint{1.705302in}{5.797238in}}%
\pgfusepath{stroke}%
\end{pgfscope}%
\begin{pgfscope}%
\pgfpathrectangle{\pgfqpoint{0.766095in}{0.571603in}}{\pgfqpoint{6.973465in}{5.225635in}}%
\pgfusepath{clip}%
\pgfsetbuttcap%
\pgfsetroundjoin%
\pgfsetlinewidth{1.505625pt}%
\definecolor{currentstroke}{rgb}{0.535621,0.835785,0.281908}%
\pgfsetstrokecolor{currentstroke}%
\pgfsetdash{}{0pt}%
\pgfpathmoveto{\pgfqpoint{0.766095in}{5.543347in}}%
\pgfpathlineto{\pgfqpoint{0.801138in}{5.555171in}}%
\pgfpathlineto{\pgfqpoint{0.818180in}{5.560903in}}%
\pgfpathlineto{\pgfqpoint{0.836180in}{5.566866in}}%
\pgfpathlineto{\pgfqpoint{0.871223in}{5.578437in}}%
\pgfpathlineto{\pgfqpoint{0.875919in}{5.579985in}}%
\pgfusepath{stroke}%
\end{pgfscope}%
\begin{pgfscope}%
\pgfpathrectangle{\pgfqpoint{0.766095in}{0.571603in}}{\pgfqpoint{6.973465in}{5.225635in}}%
\pgfusepath{clip}%
\pgfsetbuttcap%
\pgfsetroundjoin%
\pgfsetlinewidth{1.505625pt}%
\definecolor{currentstroke}{rgb}{0.535621,0.835785,0.281908}%
\pgfsetstrokecolor{currentstroke}%
\pgfsetdash{}{0pt}%
\pgfpathmoveto{\pgfqpoint{1.248986in}{5.696763in}}%
\pgfpathlineto{\pgfqpoint{1.256691in}{5.699052in}}%
\pgfpathlineto{\pgfqpoint{1.291733in}{5.709439in}}%
\pgfpathlineto{\pgfqpoint{1.322202in}{5.718460in}}%
\pgfpathlineto{\pgfqpoint{1.326776in}{5.719794in}}%
\pgfpathlineto{\pgfqpoint{1.361818in}{5.729966in}}%
\pgfpathlineto{\pgfqpoint{1.396861in}{5.740132in}}%
\pgfpathlineto{\pgfqpoint{1.412720in}{5.744720in}}%
\pgfpathlineto{\pgfqpoint{1.431903in}{5.750184in}}%
\pgfpathlineto{\pgfqpoint{1.466946in}{5.760140in}}%
\pgfpathlineto{\pgfqpoint{1.501988in}{5.770091in}}%
\pgfpathlineto{\pgfqpoint{1.505130in}{5.770979in}}%
\pgfpathlineto{\pgfqpoint{1.537031in}{5.779860in}}%
\pgfpathlineto{\pgfqpoint{1.572073in}{5.789607in}}%
\pgfpathlineto{\pgfqpoint{1.599551in}{5.797238in}}%
\pgfusepath{stroke}%
\end{pgfscope}%
\begin{pgfscope}%
\pgfpathrectangle{\pgfqpoint{0.766095in}{0.571603in}}{\pgfqpoint{6.973465in}{5.225635in}}%
\pgfusepath{clip}%
\pgfsetbuttcap%
\pgfsetroundjoin%
\pgfsetlinewidth{1.505625pt}%
\definecolor{currentstroke}{rgb}{0.585678,0.846661,0.249897}%
\pgfsetstrokecolor{currentstroke}%
\pgfsetdash{}{0pt}%
\pgfpathmoveto{\pgfqpoint{0.766095in}{5.574395in}}%
\pgfpathlineto{\pgfqpoint{0.801138in}{5.586120in}}%
\pgfpathlineto{\pgfqpoint{0.804270in}{5.587163in}}%
\pgfpathlineto{\pgfqpoint{0.836180in}{5.597628in}}%
\pgfpathlineto{\pgfqpoint{0.863794in}{5.606671in}}%
\pgfusepath{stroke}%
\end{pgfscope}%
\begin{pgfscope}%
\pgfpathrectangle{\pgfqpoint{0.766095in}{0.571603in}}{\pgfqpoint{6.973465in}{5.225635in}}%
\pgfusepath{clip}%
\pgfsetbuttcap%
\pgfsetroundjoin%
\pgfsetlinewidth{1.505625pt}%
\definecolor{currentstroke}{rgb}{0.585678,0.846661,0.249897}%
\pgfsetstrokecolor{currentstroke}%
\pgfsetdash{}{0pt}%
\pgfpathmoveto{\pgfqpoint{1.237114in}{5.722629in}}%
\pgfpathlineto{\pgfqpoint{1.256691in}{5.728392in}}%
\pgfpathlineto{\pgfqpoint{1.291733in}{5.738700in}}%
\pgfpathlineto{\pgfqpoint{1.312248in}{5.744720in}}%
\pgfpathlineto{\pgfqpoint{1.326776in}{5.748918in}}%
\pgfpathlineto{\pgfqpoint{1.361818in}{5.759013in}}%
\pgfpathlineto{\pgfqpoint{1.396861in}{5.769101in}}%
\pgfpathlineto{\pgfqpoint{1.403409in}{5.770979in}}%
\pgfpathlineto{\pgfqpoint{1.431903in}{5.779025in}}%
\pgfpathlineto{\pgfqpoint{1.466946in}{5.788907in}}%
\pgfpathlineto{\pgfqpoint{1.496527in}{5.797238in}}%
\pgfusepath{stroke}%
\end{pgfscope}%
\begin{pgfscope}%
\pgfpathrectangle{\pgfqpoint{0.766095in}{0.571603in}}{\pgfqpoint{6.973465in}{5.225635in}}%
\pgfusepath{clip}%
\pgfsetbuttcap%
\pgfsetroundjoin%
\pgfsetlinewidth{1.505625pt}%
\definecolor{currentstroke}{rgb}{0.636902,0.856542,0.216620}%
\pgfsetstrokecolor{currentstroke}%
\pgfsetdash{}{0pt}%
\pgfpathmoveto{\pgfqpoint{0.766095in}{5.604765in}}%
\pgfpathlineto{\pgfqpoint{0.792219in}{5.613422in}}%
\pgfpathlineto{\pgfqpoint{0.801138in}{5.616333in}}%
\pgfpathlineto{\pgfqpoint{0.836180in}{5.627725in}}%
\pgfpathlineto{\pgfqpoint{0.859998in}{5.635459in}}%
\pgfusepath{stroke}%
\end{pgfscope}%
\begin{pgfscope}%
\pgfpathrectangle{\pgfqpoint{0.766095in}{0.571603in}}{\pgfqpoint{6.973465in}{5.225635in}}%
\pgfusepath{clip}%
\pgfsetbuttcap%
\pgfsetroundjoin%
\pgfsetlinewidth{1.505625pt}%
\definecolor{currentstroke}{rgb}{0.636902,0.856542,0.216620}%
\pgfsetstrokecolor{currentstroke}%
\pgfsetdash{}{0pt}%
\pgfpathmoveto{\pgfqpoint{1.233634in}{5.750385in}}%
\pgfpathlineto{\pgfqpoint{1.256691in}{5.757118in}}%
\pgfpathlineto{\pgfqpoint{1.291733in}{5.767344in}}%
\pgfpathlineto{\pgfqpoint{1.304232in}{5.770979in}}%
\pgfpathlineto{\pgfqpoint{1.326776in}{5.777437in}}%
\pgfpathlineto{\pgfqpoint{1.361818in}{5.787453in}}%
\pgfpathlineto{\pgfqpoint{1.396080in}{5.797238in}}%
\pgfusepath{stroke}%
\end{pgfscope}%
\begin{pgfscope}%
\pgfpathrectangle{\pgfqpoint{0.766095in}{0.571603in}}{\pgfqpoint{6.973465in}{5.225635in}}%
\pgfusepath{clip}%
\pgfsetbuttcap%
\pgfsetroundjoin%
\pgfsetlinewidth{1.505625pt}%
\definecolor{currentstroke}{rgb}{0.688944,0.865448,0.182725}%
\pgfsetstrokecolor{currentstroke}%
\pgfsetdash{}{0pt}%
\pgfpathmoveto{\pgfqpoint{0.766095in}{5.634485in}}%
\pgfpathlineto{\pgfqpoint{0.781934in}{5.639682in}}%
\pgfpathlineto{\pgfqpoint{0.801138in}{5.645889in}}%
\pgfpathlineto{\pgfqpoint{0.836180in}{5.657182in}}%
\pgfpathlineto{\pgfqpoint{0.845220in}{5.660090in}}%
\pgfusepath{stroke}%
\end{pgfscope}%
\begin{pgfscope}%
\pgfpathrectangle{\pgfqpoint{0.766095in}{0.571603in}}{\pgfqpoint{6.973465in}{5.225635in}}%
\pgfusepath{clip}%
\pgfsetbuttcap%
\pgfsetroundjoin%
\pgfsetlinewidth{1.505625pt}%
\definecolor{currentstroke}{rgb}{0.688944,0.865448,0.182725}%
\pgfsetstrokecolor{currentstroke}%
\pgfsetdash{}{0pt}%
\pgfpathmoveto{\pgfqpoint{1.219058in}{5.774350in}}%
\pgfpathlineto{\pgfqpoint{1.221648in}{5.775103in}}%
\pgfpathlineto{\pgfqpoint{1.256691in}{5.785254in}}%
\pgfpathlineto{\pgfqpoint{1.291733in}{5.795397in}}%
\pgfpathlineto{\pgfqpoint{1.298121in}{5.797238in}}%
\pgfusepath{stroke}%
\end{pgfscope}%
\begin{pgfscope}%
\pgfpathrectangle{\pgfqpoint{0.766095in}{0.571603in}}{\pgfqpoint{6.973465in}{5.225635in}}%
\pgfusepath{clip}%
\pgfsetbuttcap%
\pgfsetroundjoin%
\pgfsetlinewidth{1.505625pt}%
\definecolor{currentstroke}{rgb}{0.741388,0.873449,0.149561}%
\pgfsetstrokecolor{currentstroke}%
\pgfsetdash{}{0pt}%
\pgfpathmoveto{\pgfqpoint{0.766095in}{5.663580in}}%
\pgfpathlineto{\pgfqpoint{0.773362in}{5.665941in}}%
\pgfpathlineto{\pgfqpoint{0.789855in}{5.671220in}}%
\pgfusepath{stroke}%
\end{pgfscope}%
\begin{pgfscope}%
\pgfpathrectangle{\pgfqpoint{0.766095in}{0.571603in}}{\pgfqpoint{6.973465in}{5.225635in}}%
\pgfusepath{clip}%
\pgfsetbuttcap%
\pgfsetroundjoin%
\pgfsetlinewidth{1.505625pt}%
\definecolor{currentstroke}{rgb}{0.741388,0.873449,0.149561}%
\pgfsetstrokecolor{currentstroke}%
\pgfsetdash{}{0pt}%
\pgfpathmoveto{\pgfqpoint{1.163583in}{5.785843in}}%
\pgfpathlineto{\pgfqpoint{1.186605in}{5.792594in}}%
\pgfpathlineto{\pgfqpoint{1.202488in}{5.797238in}}%
\pgfusepath{stroke}%
\end{pgfscope}%
\begin{pgfscope}%
\pgfpathrectangle{\pgfqpoint{0.766095in}{0.571603in}}{\pgfqpoint{6.973465in}{5.225635in}}%
\pgfusepath{clip}%
\pgfsetbuttcap%
\pgfsetroundjoin%
\pgfsetlinewidth{1.505625pt}%
\definecolor{currentstroke}{rgb}{0.845561,0.887322,0.099702}%
\pgfsetstrokecolor{currentstroke}%
\pgfsetdash{}{0pt}%
\pgfpathmoveto{\pgfqpoint{0.766095in}{5.719969in}}%
\pgfpathlineto{\pgfqpoint{0.801138in}{5.730976in}}%
\pgfpathlineto{\pgfqpoint{0.836180in}{5.741971in}}%
\pgfpathlineto{\pgfqpoint{0.844974in}{5.744720in}}%
\pgfpathlineto{\pgfqpoint{0.871223in}{5.752801in}}%
\pgfpathlineto{\pgfqpoint{0.906265in}{5.763568in}}%
\pgfpathlineto{\pgfqpoint{0.930436in}{5.770979in}}%
\pgfpathlineto{\pgfqpoint{0.941308in}{5.774263in}}%
\pgfpathlineto{\pgfqpoint{0.976350in}{5.784808in}}%
\pgfpathlineto{\pgfqpoint{1.011393in}{5.795344in}}%
\pgfpathlineto{\pgfqpoint{1.017721in}{5.797238in}}%
\pgfusepath{stroke}%
\end{pgfscope}%
\begin{pgfscope}%
\pgfpathrectangle{\pgfqpoint{0.766095in}{0.571603in}}{\pgfqpoint{6.973465in}{5.225635in}}%
\pgfusepath{clip}%
\pgfsetbuttcap%
\pgfsetroundjoin%
\pgfsetlinewidth{1.505625pt}%
\definecolor{currentstroke}{rgb}{0.896320,0.893616,0.096335}%
\pgfsetstrokecolor{currentstroke}%
\pgfsetdash{}{0pt}%
\pgfpathmoveto{\pgfqpoint{0.766095in}{5.747318in}}%
\pgfpathlineto{\pgfqpoint{0.801138in}{5.758225in}}%
\pgfpathlineto{\pgfqpoint{0.836180in}{5.769119in}}%
\pgfpathlineto{\pgfqpoint{0.842187in}{5.770979in}}%
\pgfpathlineto{\pgfqpoint{0.871223in}{5.779835in}}%
\pgfpathlineto{\pgfqpoint{0.906265in}{5.790505in}}%
\pgfpathlineto{\pgfqpoint{0.928432in}{5.797238in}}%
\pgfusepath{stroke}%
\end{pgfscope}%
\begin{pgfscope}%
\pgfpathrectangle{\pgfqpoint{0.766095in}{0.571603in}}{\pgfqpoint{6.973465in}{5.225635in}}%
\pgfusepath{clip}%
\pgfsetbuttcap%
\pgfsetroundjoin%
\pgfsetlinewidth{1.505625pt}%
\definecolor{currentstroke}{rgb}{0.945636,0.899815,0.112838}%
\pgfsetstrokecolor{currentstroke}%
\pgfsetdash{}{0pt}%
\pgfpathmoveto{\pgfqpoint{0.766095in}{5.774148in}}%
\pgfpathlineto{\pgfqpoint{0.801138in}{5.784954in}}%
\pgfpathlineto{\pgfqpoint{0.836180in}{5.795748in}}%
\pgfpathlineto{\pgfqpoint{0.841040in}{5.797238in}}%
\pgfusepath{stroke}%
\end{pgfscope}%
\begin{pgfscope}%
\pgfpathrectangle{\pgfqpoint{0.766095in}{0.571603in}}{\pgfqpoint{6.973465in}{5.225635in}}%
\pgfusepath{clip}%
\pgfsetrectcap%
\pgfsetroundjoin%
\pgfsetlinewidth{1.505625pt}%
\definecolor{currentstroke}{rgb}{0.000000,0.000000,0.000000}%
\pgfsetstrokecolor{currentstroke}%
\pgfsetdash{}{0pt}%
\pgfpathmoveto{\pgfqpoint{6.577316in}{3.706984in}}%
\pgfpathlineto{\pgfqpoint{4.542738in}{5.098057in}}%
\pgfpathlineto{\pgfqpoint{3.038495in}{3.912818in}}%
\pgfpathlineto{\pgfqpoint{2.740398in}{2.889953in}}%
\pgfpathlineto{\pgfqpoint{2.899488in}{1.590976in}}%
\pgfpathlineto{\pgfqpoint{3.714898in}{1.277449in}}%
\pgfpathlineto{\pgfqpoint{3.972874in}{1.117929in}}%
\pgfusepath{stroke}%
\end{pgfscope}%
\begin{pgfscope}%
\pgfsetrectcap%
\pgfsetmiterjoin%
\pgfsetlinewidth{0.803000pt}%
\definecolor{currentstroke}{rgb}{0.000000,0.000000,0.000000}%
\pgfsetstrokecolor{currentstroke}%
\pgfsetdash{}{0pt}%
\pgfpathmoveto{\pgfqpoint{0.766095in}{0.571603in}}%
\pgfpathlineto{\pgfqpoint{0.766095in}{5.797238in}}%
\pgfusepath{stroke}%
\end{pgfscope}%
\begin{pgfscope}%
\pgfsetrectcap%
\pgfsetmiterjoin%
\pgfsetlinewidth{0.803000pt}%
\definecolor{currentstroke}{rgb}{0.000000,0.000000,0.000000}%
\pgfsetstrokecolor{currentstroke}%
\pgfsetdash{}{0pt}%
\pgfpathmoveto{\pgfqpoint{7.739560in}{0.571603in}}%
\pgfpathlineto{\pgfqpoint{7.739560in}{5.797238in}}%
\pgfusepath{stroke}%
\end{pgfscope}%
\begin{pgfscope}%
\pgfsetrectcap%
\pgfsetmiterjoin%
\pgfsetlinewidth{0.803000pt}%
\definecolor{currentstroke}{rgb}{0.000000,0.000000,0.000000}%
\pgfsetstrokecolor{currentstroke}%
\pgfsetdash{}{0pt}%
\pgfpathmoveto{\pgfqpoint{0.766095in}{0.571603in}}%
\pgfpathlineto{\pgfqpoint{7.739560in}{0.571603in}}%
\pgfusepath{stroke}%
\end{pgfscope}%
\begin{pgfscope}%
\pgfsetrectcap%
\pgfsetmiterjoin%
\pgfsetlinewidth{0.803000pt}%
\definecolor{currentstroke}{rgb}{0.000000,0.000000,0.000000}%
\pgfsetstrokecolor{currentstroke}%
\pgfsetdash{}{0pt}%
\pgfpathmoveto{\pgfqpoint{0.766095in}{5.797238in}}%
\pgfpathlineto{\pgfqpoint{7.739560in}{5.797238in}}%
\pgfusepath{stroke}%
\end{pgfscope}%
\begin{pgfscope}%
\definecolor{textcolor}{rgb}{0.273809,0.031497,0.358853}%
\pgfsetstrokecolor{textcolor}%
\pgfsetfillcolor{textcolor}%
\pgftext[x=4.702045in, y=1.283617in, left, base,rotate=322.628138]{\color{textcolor}\sffamily\fontsize{8.000000}{9.600000}\selectfont 1.8}%
\end{pgfscope}%
\begin{pgfscope}%
\definecolor{textcolor}{rgb}{0.278791,0.062145,0.386592}%
\pgfsetstrokecolor{textcolor}%
\pgfsetfillcolor{textcolor}%
\pgftext[x=4.150196in, y=2.235001in, left, base,rotate=316.384801]{\color{textcolor}\sffamily\fontsize{8.000000}{9.600000}\selectfont 2.1}%
\end{pgfscope}%
\begin{pgfscope}%
\definecolor{textcolor}{rgb}{0.281924,0.089666,0.412415}%
\pgfsetstrokecolor{textcolor}%
\pgfsetfillcolor{textcolor}%
\pgftext[x=4.257295in, y=2.697860in, left, base,rotate=307.261522]{\color{textcolor}\sffamily\fontsize{8.000000}{9.600000}\selectfont 2.4}%
\end{pgfscope}%
\begin{pgfscope}%
\definecolor{textcolor}{rgb}{0.283197,0.115680,0.436115}%
\pgfsetstrokecolor{textcolor}%
\pgfsetfillcolor{textcolor}%
\pgftext[x=6.527717in, y=0.843302in, left, base,rotate=326.225659]{\color{textcolor}\sffamily\fontsize{8.000000}{9.600000}\selectfont 2.7}%
\end{pgfscope}%
\begin{pgfscope}%
\definecolor{textcolor}{rgb}{0.282623,0.140926,0.457517}%
\pgfsetstrokecolor{textcolor}%
\pgfsetfillcolor{textcolor}%
\pgftext[x=6.423513in, y=1.119805in, left, base,rotate=324.131978]{\color{textcolor}\sffamily\fontsize{8.000000}{9.600000}\selectfont 3.0}%
\end{pgfscope}%
\begin{pgfscope}%
\definecolor{textcolor}{rgb}{0.280255,0.165693,0.476498}%
\pgfsetstrokecolor{textcolor}%
\pgfsetfillcolor{textcolor}%
\pgftext[x=5.100005in, y=3.492796in, left, base,rotate=280.617184]{\color{textcolor}\sffamily\fontsize{8.000000}{9.600000}\selectfont 3.3}%
\end{pgfscope}%
\begin{pgfscope}%
\definecolor{textcolor}{rgb}{0.276194,0.190074,0.493001}%
\pgfsetstrokecolor{textcolor}%
\pgfsetfillcolor{textcolor}%
\pgftext[x=7.380160in, y=0.802667in, left, base,rotate=327.896753]{\color{textcolor}\sffamily\fontsize{8.000000}{9.600000}\selectfont 3.6}%
\end{pgfscope}%
\begin{pgfscope}%
\definecolor{textcolor}{rgb}{0.270595,0.214069,0.507052}%
\pgfsetstrokecolor{textcolor}%
\pgfsetfillcolor{textcolor}%
\pgftext[x=1.586327in, y=0.697153in, left, base,rotate=327.208004]{\color{textcolor}\sffamily\fontsize{8.000000}{9.600000}\selectfont 3.9}%
\end{pgfscope}%
\begin{pgfscope}%
\definecolor{textcolor}{rgb}{0.270595,0.214069,0.507052}%
\pgfsetstrokecolor{textcolor}%
\pgfsetfillcolor{textcolor}%
\pgftext[x=5.766362in, y=3.632778in, left, base,rotate=86.716516]{\color{textcolor}\sffamily\fontsize{8.000000}{9.600000}\selectfont 3.9}%
\end{pgfscope}%
\begin{pgfscope}%
\definecolor{textcolor}{rgb}{0.263663,0.237631,0.518762}%
\pgfsetstrokecolor{textcolor}%
\pgfsetfillcolor{textcolor}%
\pgftext[x=1.125347in, y=0.918374in, left, base,rotate=321.079533]{\color{textcolor}\sffamily\fontsize{8.000000}{9.600000}\selectfont 4.2}%
\end{pgfscope}%
\begin{pgfscope}%
\definecolor{textcolor}{rgb}{0.263663,0.237631,0.518762}%
\pgfsetstrokecolor{textcolor}%
\pgfsetfillcolor{textcolor}%
\pgftext[x=7.299445in, y=1.152177in, left, base,rotate=324.415799]{\color{textcolor}\sffamily\fontsize{8.000000}{9.600000}\selectfont 4.2}%
\end{pgfscope}%
\begin{pgfscope}%
\definecolor{textcolor}{rgb}{0.255645,0.260703,0.528312}%
\pgfsetstrokecolor{textcolor}%
\pgfsetfillcolor{textcolor}%
\pgftext[x=1.247935in, y=0.700502in, left, base,rotate=326.083584]{\color{textcolor}\sffamily\fontsize{8.000000}{9.600000}\selectfont 4.5}%
\end{pgfscope}%
\begin{pgfscope}%
\definecolor{textcolor}{rgb}{0.255645,0.260703,0.528312}%
\pgfsetstrokecolor{textcolor}%
\pgfsetfillcolor{textcolor}%
\pgftext[x=6.309223in, y=3.813408in, left, base,rotate=75.964698]{\color{textcolor}\sffamily\fontsize{8.000000}{9.600000}\selectfont 4.5}%
\end{pgfscope}%
\begin{pgfscope}%
\definecolor{textcolor}{rgb}{0.246811,0.283237,0.535941}%
\pgfsetstrokecolor{textcolor}%
\pgfsetfillcolor{textcolor}%
\pgftext[x=0.818889in, y=0.922450in, left, base,rotate=319.985225]{\color{textcolor}\sffamily\fontsize{8.000000}{9.600000}\selectfont 4.8}%
\end{pgfscope}%
\begin{pgfscope}%
\definecolor{textcolor}{rgb}{0.246811,0.283237,0.535941}%
\pgfsetstrokecolor{textcolor}%
\pgfsetfillcolor{textcolor}%
\pgftext[x=3.454827in, y=4.836525in, left, base,rotate=12.493939]{\color{textcolor}\sffamily\fontsize{8.000000}{9.600000}\selectfont 4.8}%
\end{pgfscope}%
\begin{pgfscope}%
\definecolor{textcolor}{rgb}{0.237441,0.305202,0.541921}%
\pgfsetstrokecolor{textcolor}%
\pgfsetfillcolor{textcolor}%
\pgftext[x=0.956400in, y=0.698807in, left, base,rotate=325.180238]{\color{textcolor}\sffamily\fontsize{8.000000}{9.600000}\selectfont 5.1}%
\end{pgfscope}%
\begin{pgfscope}%
\definecolor{textcolor}{rgb}{0.237441,0.305202,0.541921}%
\pgfsetstrokecolor{textcolor}%
\pgfsetfillcolor{textcolor}%
\pgftext[x=6.884341in, y=4.128324in, left, base,rotate=67.391368]{\color{textcolor}\sffamily\fontsize{8.000000}{9.600000}\selectfont 5.1}%
\end{pgfscope}%
\begin{pgfscope}%
\definecolor{textcolor}{rgb}{0.227802,0.326594,0.546532}%
\pgfsetstrokecolor{textcolor}%
\pgfsetfillcolor{textcolor}%
\pgftext[x=0.816468in, y=0.702225in, left, base,rotate=324.648301]{\color{textcolor}\sffamily\fontsize{8.000000}{9.600000}\selectfont 5.4}%
\end{pgfscope}%
\begin{pgfscope}%
\definecolor{textcolor}{rgb}{0.227802,0.326594,0.546532}%
\pgfsetstrokecolor{textcolor}%
\pgfsetfillcolor{textcolor}%
\pgftext[x=3.594565in, y=5.033950in, left, base,rotate=11.985085]{\color{textcolor}\sffamily\fontsize{8.000000}{9.600000}\selectfont 5.4}%
\end{pgfscope}%
\begin{pgfscope}%
\definecolor{textcolor}{rgb}{0.218130,0.347432,0.550038}%
\pgfsetstrokecolor{textcolor}%
\pgfsetfillcolor{textcolor}%
\pgftext[x=7.300292in, y=4.103215in, left, base,rotate=61.218624]{\color{textcolor}\sffamily\fontsize{8.000000}{9.600000}\selectfont 5.7}%
\end{pgfscope}%
\begin{pgfscope}%
\definecolor{textcolor}{rgb}{0.208623,0.367752,0.552675}%
\pgfsetstrokecolor{textcolor}%
\pgfsetfillcolor{textcolor}%
\pgftext[x=5.114547in, y=5.440726in, left, base,rotate=7.140731]{\color{textcolor}\sffamily\fontsize{8.000000}{9.600000}\selectfont 6.0}%
\end{pgfscope}%
\begin{pgfscope}%
\definecolor{textcolor}{rgb}{0.208623,0.367752,0.552675}%
\pgfsetstrokecolor{textcolor}%
\pgfsetfillcolor{textcolor}%
\pgftext[x=7.624086in, y=4.314399in, left, base,rotate=57.774471]{\color{textcolor}\sffamily\fontsize{8.000000}{9.600000}\selectfont 6.0}%
\end{pgfscope}%
\begin{pgfscope}%
\definecolor{textcolor}{rgb}{0.199430,0.387607,0.554642}%
\pgfsetstrokecolor{textcolor}%
\pgfsetfillcolor{textcolor}%
\pgftext[x=3.699471in, y=5.261531in, left, base,rotate=11.720729]{\color{textcolor}\sffamily\fontsize{8.000000}{9.600000}\selectfont 6.3}%
\end{pgfscope}%
\begin{pgfscope}%
\definecolor{textcolor}{rgb}{0.199430,0.387607,0.554642}%
\pgfsetstrokecolor{textcolor}%
\pgfsetfillcolor{textcolor}%
\pgftext[x=7.620533in, y=3.998965in, left, base,rotate=58.670370]{\color{textcolor}\sffamily\fontsize{8.000000}{9.600000}\selectfont 6.3}%
\end{pgfscope}%
\begin{pgfscope}%
\definecolor{textcolor}{rgb}{0.190631,0.407061,0.556089}%
\pgfsetstrokecolor{textcolor}%
\pgfsetfillcolor{textcolor}%
\pgftext[x=7.651615in, y=3.735265in, left, base,rotate=62.455786]{\color{textcolor}\sffamily\fontsize{8.000000}{9.600000}\selectfont 6.6}%
\end{pgfscope}%
\begin{pgfscope}%
\definecolor{textcolor}{rgb}{0.190631,0.407061,0.556089}%
\pgfsetstrokecolor{textcolor}%
\pgfsetfillcolor{textcolor}%
\pgftext[x=4.188915in, y=5.419138in, left, base,rotate=10.316214]{\color{textcolor}\sffamily\fontsize{8.000000}{9.600000}\selectfont 6.6}%
\end{pgfscope}%
\begin{pgfscope}%
\definecolor{textcolor}{rgb}{0.182256,0.426184,0.557120}%
\pgfsetstrokecolor{textcolor}%
\pgfsetfillcolor{textcolor}%
\pgftext[x=7.677813in, y=3.392887in, left, base,rotate=71.042093]{\color{textcolor}\sffamily\fontsize{8.000000}{9.600000}\selectfont 6.9}%
\end{pgfscope}%
\begin{pgfscope}%
\definecolor{textcolor}{rgb}{0.182256,0.426184,0.557120}%
\pgfsetstrokecolor{textcolor}%
\pgfsetfillcolor{textcolor}%
\pgftext[x=5.238243in, y=5.648118in, left, base,rotate=7.789317]{\color{textcolor}\sffamily\fontsize{8.000000}{9.600000}\selectfont 6.9}%
\end{pgfscope}%
\begin{pgfscope}%
\definecolor{textcolor}{rgb}{0.174274,0.445044,0.557792}%
\pgfsetstrokecolor{textcolor}%
\pgfsetfillcolor{textcolor}%
\pgftext[x=4.713657in, y=5.622206in, left, base,rotate=9.179416]{\color{textcolor}\sffamily\fontsize{8.000000}{9.600000}\selectfont 7.2}%
\end{pgfscope}%
\begin{pgfscope}%
\definecolor{textcolor}{rgb}{0.166617,0.463708,0.558119}%
\pgfsetstrokecolor{textcolor}%
\pgfsetfillcolor{textcolor}%
\pgftext[x=3.804303in, y=5.503969in, left, base,rotate=11.365855]{\color{textcolor}\sffamily\fontsize{8.000000}{9.600000}\selectfont 7.5}%
\end{pgfscope}%
\begin{pgfscope}%
\definecolor{textcolor}{rgb}{0.159194,0.482237,0.558073}%
\pgfsetstrokecolor{textcolor}%
\pgfsetfillcolor{textcolor}%
\pgftext[x=4.293978in, y=5.647957in, left, base,rotate=10.235895]{\color{textcolor}\sffamily\fontsize{8.000000}{9.600000}\selectfont 7.8}%
\end{pgfscope}%
\begin{pgfscope}%
\definecolor{textcolor}{rgb}{0.151918,0.500685,0.557587}%
\pgfsetstrokecolor{textcolor}%
\pgfsetfillcolor{textcolor}%
\pgftext[x=3.384620in, y=5.508649in, left, base,rotate=12.350386]{\color{textcolor}\sffamily\fontsize{8.000000}{9.600000}\selectfont 8.1}%
\end{pgfscope}%
\begin{pgfscope}%
\definecolor{textcolor}{rgb}{0.144759,0.519093,0.556572}%
\pgfsetstrokecolor{textcolor}%
\pgfsetfillcolor{textcolor}%
\pgftext[x=2.930098in, y=5.446749in, left, base,rotate=13.540323]{\color{textcolor}\sffamily\fontsize{8.000000}{9.600000}\selectfont 8.4}%
\end{pgfscope}%
\begin{pgfscope}%
\definecolor{textcolor}{rgb}{0.137770,0.537492,0.554906}%
\pgfsetstrokecolor{textcolor}%
\pgfsetfillcolor{textcolor}%
\pgftext[x=2.510663in, y=5.382950in, left, base,rotate=14.741173]{\color{textcolor}\sffamily\fontsize{8.000000}{9.600000}\selectfont 8.7}%
\end{pgfscope}%
\begin{pgfscope}%
\definecolor{textcolor}{rgb}{0.131172,0.555899,0.552459}%
\pgfsetstrokecolor{textcolor}%
\pgfsetfillcolor{textcolor}%
\pgftext[x=2.091406in, y=5.308479in, left, base,rotate=16.094687]{\color{textcolor}\sffamily\fontsize{8.000000}{9.600000}\selectfont 9.0}%
\end{pgfscope}%
\begin{pgfscope}%
\definecolor{textcolor}{rgb}{0.125394,0.574318,0.549086}%
\pgfsetstrokecolor{textcolor}%
\pgfsetfillcolor{textcolor}%
\pgftext[x=1.672401in, y=5.222138in, left, base,rotate=17.663615]{\color{textcolor}\sffamily\fontsize{8.000000}{9.600000}\selectfont 9.3}%
\end{pgfscope}%
\begin{pgfscope}%
\definecolor{textcolor}{rgb}{0.121148,0.592739,0.544641}%
\pgfsetstrokecolor{textcolor}%
\pgfsetfillcolor{textcolor}%
\pgftext[x=1.288626in, y=5.134638in, left, base,rotate=19.362015]{\color{textcolor}\sffamily\fontsize{8.000000}{9.600000}\selectfont 9.6}%
\end{pgfscope}%
\begin{pgfscope}%
\definecolor{textcolor}{rgb}{0.119423,0.611141,0.538982}%
\pgfsetstrokecolor{textcolor}%
\pgfsetfillcolor{textcolor}%
\pgftext[x=0.870461in, y=5.019321in, left, base,rotate=21.615220]{\color{textcolor}\sffamily\fontsize{8.000000}{9.600000}\selectfont 9.9}%
\end{pgfscope}%
\begin{pgfscope}%
\definecolor{textcolor}{rgb}{0.121380,0.629492,0.531973}%
\pgfsetstrokecolor{textcolor}%
\pgfsetfillcolor{textcolor}%
\pgftext[x=2.720324in, y=5.638324in, left, base,rotate=13.346775]{\color{textcolor}\sffamily\fontsize{8.000000}{9.600000}\selectfont 10.2}%
\end{pgfscope}%
\begin{pgfscope}%
\definecolor{textcolor}{rgb}{0.128087,0.647749,0.523491}%
\pgfsetstrokecolor{textcolor}%
\pgfsetfillcolor{textcolor}%
\pgftext[x=2.126298in, y=5.521010in, left, base,rotate=14.967894]{\color{textcolor}\sffamily\fontsize{8.000000}{9.600000}\selectfont 10.5}%
\end{pgfscope}%
\begin{pgfscope}%
\definecolor{textcolor}{rgb}{0.140210,0.665859,0.513427}%
\pgfsetstrokecolor{textcolor}%
\pgfsetfillcolor{textcolor}%
\pgftext[x=2.126128in, y=5.557641in, left, base,rotate=14.809432]{\color{textcolor}\sffamily\fontsize{8.000000}{9.600000}\selectfont 10.8}%
\end{pgfscope}%
\begin{pgfscope}%
\definecolor{textcolor}{rgb}{0.157851,0.683765,0.501686}%
\pgfsetstrokecolor{textcolor}%
\pgfsetfillcolor{textcolor}%
\pgftext[x=2.125985in, y=5.593172in, left, base,rotate=14.675619]{\color{textcolor}\sffamily\fontsize{8.000000}{9.600000}\selectfont 11.1}%
\end{pgfscope}%
\begin{pgfscope}%
\definecolor{textcolor}{rgb}{0.180653,0.701402,0.488189}%
\pgfsetstrokecolor{textcolor}%
\pgfsetfillcolor{textcolor}%
\pgftext[x=2.139852in, y=5.631560in, left, base,rotate=14.495923]{\color{textcolor}\sffamily\fontsize{8.000000}{9.600000}\selectfont 11.4}%
\end{pgfscope}%
\begin{pgfscope}%
\definecolor{textcolor}{rgb}{0.208030,0.718701,0.472873}%
\pgfsetstrokecolor{textcolor}%
\pgfsetfillcolor{textcolor}%
\pgftext[x=2.125696in, y=5.661467in, left, base,rotate=14.403108]{\color{textcolor}\sffamily\fontsize{8.000000}{9.600000}\selectfont 11.7}%
\end{pgfscope}%
\begin{pgfscope}%
\definecolor{textcolor}{rgb}{0.239374,0.735588,0.455688}%
\pgfsetstrokecolor{textcolor}%
\pgfsetfillcolor{textcolor}%
\pgftext[x=0.939233in, y=5.334458in, left, base,rotate=18.814929]{\color{textcolor}\sffamily\fontsize{8.000000}{9.600000}\selectfont 12.0}%
\end{pgfscope}%
\begin{pgfscope}%
\definecolor{textcolor}{rgb}{0.274149,0.751988,0.436601}%
\pgfsetstrokecolor{textcolor}%
\pgfsetfillcolor{textcolor}%
\pgftext[x=1.531760in, y=5.561279in, left, base,rotate=16.052212]{\color{textcolor}\sffamily\fontsize{8.000000}{9.600000}\selectfont 12.3}%
\end{pgfscope}%
\begin{pgfscope}%
\definecolor{textcolor}{rgb}{0.311925,0.767822,0.415586}%
\pgfsetstrokecolor{textcolor}%
\pgfsetfillcolor{textcolor}%
\pgftext[x=1.531584in, y=5.594089in, left, base,rotate=15.894007]{\color{textcolor}\sffamily\fontsize{8.000000}{9.600000}\selectfont 12.6}%
\end{pgfscope}%
\begin{pgfscope}%
\definecolor{textcolor}{rgb}{0.352360,0.783011,0.392636}%
\pgfsetstrokecolor{textcolor}%
\pgfsetfillcolor{textcolor}%
\pgftext[x=1.541972in, y=5.629217in, left, base,rotate=15.700752]{\color{textcolor}\sffamily\fontsize{8.000000}{9.600000}\selectfont 12.9}%
\end{pgfscope}%
\begin{pgfscope}%
\definecolor{textcolor}{rgb}{0.395174,0.797475,0.367757}%
\pgfsetstrokecolor{textcolor}%
\pgfsetfillcolor{textcolor}%
\pgftext[x=1.531247in, y=5.657388in, left, base,rotate=15.587761]{\color{textcolor}\sffamily\fontsize{8.000000}{9.600000}\selectfont 13.2}%
\end{pgfscope}%
\begin{pgfscope}%
\definecolor{textcolor}{rgb}{0.440137,0.811138,0.340967}%
\pgfsetstrokecolor{textcolor}%
\pgfsetfillcolor{textcolor}%
\pgftext[x=0.937945in, y=5.505756in, left, base,rotate=17.727303]{\color{textcolor}\sffamily\fontsize{8.000000}{9.600000}\selectfont 13.5}%
\end{pgfscope}%
\begin{pgfscope}%
\definecolor{textcolor}{rgb}{0.487026,0.823929,0.312321}%
\pgfsetstrokecolor{textcolor}%
\pgfsetfillcolor{textcolor}%
\pgftext[x=0.967060in, y=5.547174in, left, base,rotate=17.403797]{\color{textcolor}\sffamily\fontsize{8.000000}{9.600000}\selectfont 13.8}%
\end{pgfscope}%
\begin{pgfscope}%
\definecolor{textcolor}{rgb}{0.535621,0.835785,0.281908}%
\pgfsetstrokecolor{textcolor}%
\pgfsetfillcolor{textcolor}%
\pgftext[x=0.953040in, y=5.573665in, left, base,rotate=17.281728]{\color{textcolor}\sffamily\fontsize{8.000000}{9.600000}\selectfont 14.1}%
\end{pgfscope}%
\begin{pgfscope}%
\definecolor{textcolor}{rgb}{0.585678,0.846661,0.249897}%
\pgfsetstrokecolor{textcolor}%
\pgfsetfillcolor{textcolor}%
\pgftext[x=0.940906in, y=5.600163in, left, base,rotate=17.156542]{\color{textcolor}\sffamily\fontsize{8.000000}{9.600000}\selectfont 14.4}%
\end{pgfscope}%
\begin{pgfscope}%
\definecolor{textcolor}{rgb}{0.636902,0.856542,0.216620}%
\pgfsetstrokecolor{textcolor}%
\pgfsetfillcolor{textcolor}%
\pgftext[x=0.937108in, y=5.628688in, left, base,rotate=17.003150]{\color{textcolor}\sffamily\fontsize{8.000000}{9.600000}\selectfont 14.7}%
\end{pgfscope}%
\begin{pgfscope}%
\definecolor{textcolor}{rgb}{0.688944,0.865448,0.182725}%
\pgfsetstrokecolor{textcolor}%
\pgfsetfillcolor{textcolor}%
\pgftext[x=0.922317in, y=5.653176in, left, base,rotate=16.897443]{\color{textcolor}\sffamily\fontsize{8.000000}{9.600000}\selectfont 15.0}%
\end{pgfscope}%
\begin{pgfscope}%
\definecolor{textcolor}{rgb}{0.741388,0.873449,0.149561}%
\pgfsetstrokecolor{textcolor}%
\pgfsetfillcolor{textcolor}%
\pgftext[x=0.866976in, y=5.664339in, left, base,rotate=16.962219]{\color{textcolor}\sffamily\fontsize{8.000000}{9.600000}\selectfont 15.3}%
\end{pgfscope}%
\begin{pgfscope}%
\definecolor{textcolor}{rgb}{0.793760,0.880678,0.120005}%
\pgfsetstrokecolor{textcolor}%
\pgfsetfillcolor{textcolor}%
\pgftext[x=0.824752in, y=5.679303in, left, base,rotate=16.966786]{\color{textcolor}\sffamily\fontsize{8.000000}{9.600000}\selectfont 15.6}%
\end{pgfscope}%
\end{pgfpicture}%
\makeatother%
\endgroup%
}
        \caption{Pohľad zhora (Vrstevnice)}
        \label{fig:newton_vlavo}
    \end{subfigure}
    \hfill
    % --- PRAVÝ OBRÁZOK ---
    \begin{subfigure}[b]{0.48\textwidth}
        \centering
        \resizebox{\linewidth}{!}{%% Creator: Matplotlib, PGF backend
%%
%% To include the figure in your LaTeX document, write
%%   \input{<filename>.pgf}
%%
%% Make sure the required packages are loaded in your preamble
%%   \usepackage{pgf}
%%
%% Also ensure that all the required font packages are loaded; for instance,
%% the lmodern package is sometimes necessary when using math font.
%%   \usepackage{lmodern}
%%
%% Figures using additional raster images can only be included by \input if
%% they are in the same directory as the main LaTeX file. For loading figures
%% from other directories you can use the `import` package
%%   \usepackage{import}
%%
%% and then include the figures with
%%   \import{<path to file>}{<filename>.pgf}
%%
%% Matplotlib used the following preamble
%%   
%%   \usepackage{fontspec}
%%   \setmainfont{DejaVuSerif.ttf}[Path=\detokenize{/home/radimek/Documents/projekt_mat_prog/mat_prog_kernel/lib/python3.12/site-packages/matplotlib/mpl-data/fonts/ttf/}]
%%   \setsansfont{DejaVuSans.ttf}[Path=\detokenize{/home/radimek/Documents/projekt_mat_prog/mat_prog_kernel/lib/python3.12/site-packages/matplotlib/mpl-data/fonts/ttf/}]
%%   \setmonofont{DejaVuSansMono.ttf}[Path=\detokenize{/home/radimek/Documents/projekt_mat_prog/mat_prog_kernel/lib/python3.12/site-packages/matplotlib/mpl-data/fonts/ttf/}]
%%   \makeatletter\@ifpackageloaded{underscore}{}{\usepackage[strings]{underscore}}\makeatother
%%
\begingroup%
\makeatletter%
\begin{pgfpicture}%
\pgfpathrectangle{\pgfpointorigin}{\pgfqpoint{8.000000in}{6.000000in}}%
\pgfusepath{use as bounding box, clip}%
\begin{pgfscope}%
\pgfsetbuttcap%
\pgfsetmiterjoin%
\definecolor{currentfill}{rgb}{1.000000,1.000000,1.000000}%
\pgfsetfillcolor{currentfill}%
\pgfsetlinewidth{0.000000pt}%
\definecolor{currentstroke}{rgb}{1.000000,1.000000,1.000000}%
\pgfsetstrokecolor{currentstroke}%
\pgfsetdash{}{0pt}%
\pgfpathmoveto{\pgfqpoint{0.000000in}{0.000000in}}%
\pgfpathlineto{\pgfqpoint{8.000000in}{0.000000in}}%
\pgfpathlineto{\pgfqpoint{8.000000in}{6.000000in}}%
\pgfpathlineto{\pgfqpoint{0.000000in}{6.000000in}}%
\pgfpathlineto{\pgfqpoint{0.000000in}{0.000000in}}%
\pgfpathclose%
\pgfusepath{fill}%
\end{pgfscope}%
\begin{pgfscope}%
\pgfsetbuttcap%
\pgfsetmiterjoin%
\definecolor{currentfill}{rgb}{1.000000,1.000000,1.000000}%
\pgfsetfillcolor{currentfill}%
\pgfsetlinewidth{0.000000pt}%
\definecolor{currentstroke}{rgb}{0.000000,0.000000,0.000000}%
\pgfsetstrokecolor{currentstroke}%
\pgfsetstrokeopacity{0.000000}%
\pgfsetdash{}{0pt}%
\pgfpathmoveto{\pgfqpoint{1.150000in}{0.150000in}}%
\pgfpathlineto{\pgfqpoint{6.850000in}{0.150000in}}%
\pgfpathlineto{\pgfqpoint{6.850000in}{5.850000in}}%
\pgfpathlineto{\pgfqpoint{1.150000in}{5.850000in}}%
\pgfpathlineto{\pgfqpoint{1.150000in}{0.150000in}}%
\pgfpathclose%
\pgfusepath{fill}%
\end{pgfscope}%
\begin{pgfscope}%
\pgfsetbuttcap%
\pgfsetmiterjoin%
\definecolor{currentfill}{rgb}{0.950000,0.950000,0.950000}%
\pgfsetfillcolor{currentfill}%
\pgfsetfillopacity{0.500000}%
\pgfsetlinewidth{1.003750pt}%
\definecolor{currentstroke}{rgb}{0.950000,0.950000,0.950000}%
\pgfsetstrokecolor{currentstroke}%
\pgfsetstrokeopacity{0.500000}%
\pgfsetdash{}{0pt}%
\pgfpathmoveto{\pgfqpoint{1.580389in}{1.555437in}}%
\pgfpathlineto{\pgfqpoint{3.462715in}{3.133240in}}%
\pgfpathlineto{\pgfqpoint{3.436549in}{5.408715in}}%
\pgfpathlineto{\pgfqpoint{1.464144in}{3.969343in}}%
\pgfusepath{stroke,fill}%
\end{pgfscope}%
\begin{pgfscope}%
\pgfsetbuttcap%
\pgfsetmiterjoin%
\definecolor{currentfill}{rgb}{0.900000,0.900000,0.900000}%
\pgfsetfillcolor{currentfill}%
\pgfsetfillopacity{0.500000}%
\pgfsetlinewidth{1.003750pt}%
\definecolor{currentstroke}{rgb}{0.900000,0.900000,0.900000}%
\pgfsetstrokecolor{currentstroke}%
\pgfsetstrokeopacity{0.500000}%
\pgfsetdash{}{0pt}%
\pgfpathmoveto{\pgfqpoint{3.462715in}{3.133240in}}%
\pgfpathlineto{\pgfqpoint{6.483177in}{2.255311in}}%
\pgfpathlineto{\pgfqpoint{6.590967in}{4.609162in}}%
\pgfpathlineto{\pgfqpoint{3.436549in}{5.408715in}}%
\pgfusepath{stroke,fill}%
\end{pgfscope}%
\begin{pgfscope}%
\pgfsetbuttcap%
\pgfsetmiterjoin%
\definecolor{currentfill}{rgb}{0.925000,0.925000,0.925000}%
\pgfsetfillcolor{currentfill}%
\pgfsetfillopacity{0.500000}%
\pgfsetlinewidth{1.003750pt}%
\definecolor{currentstroke}{rgb}{0.925000,0.925000,0.925000}%
\pgfsetstrokecolor{currentstroke}%
\pgfsetstrokeopacity{0.500000}%
\pgfsetdash{}{0pt}%
\pgfpathmoveto{\pgfqpoint{1.580389in}{1.555437in}}%
\pgfpathlineto{\pgfqpoint{4.782226in}{0.509717in}}%
\pgfpathlineto{\pgfqpoint{6.483177in}{2.255311in}}%
\pgfpathlineto{\pgfqpoint{3.462715in}{3.133240in}}%
\pgfusepath{stroke,fill}%
\end{pgfscope}%
\begin{pgfscope}%
\pgfsetrectcap%
\pgfsetroundjoin%
\pgfsetlinewidth{0.803000pt}%
\definecolor{currentstroke}{rgb}{0.000000,0.000000,0.000000}%
\pgfsetstrokecolor{currentstroke}%
\pgfsetdash{}{0pt}%
\pgfpathmoveto{\pgfqpoint{1.580389in}{1.555437in}}%
\pgfpathlineto{\pgfqpoint{4.782226in}{0.509717in}}%
\pgfusepath{stroke}%
\end{pgfscope}%
\begin{pgfscope}%
\definecolor{textcolor}{rgb}{0.000000,0.000000,0.000000}%
\pgfsetstrokecolor{textcolor}%
\pgfsetfillcolor{textcolor}%
\pgftext[x=2.913491in,y=0.557898in,,]{\color{textcolor}\sffamily\fontsize{10.000000}{12.000000}\selectfont x}%
\end{pgfscope}%
\begin{pgfscope}%
\pgfsetbuttcap%
\pgfsetroundjoin%
\pgfsetlinewidth{0.803000pt}%
\definecolor{currentstroke}{rgb}{0.690196,0.690196,0.690196}%
\pgfsetstrokecolor{currentstroke}%
\pgfsetdash{}{0pt}%
\pgfpathmoveto{\pgfqpoint{1.774309in}{1.492103in}}%
\pgfpathlineto{\pgfqpoint{3.646411in}{3.079847in}}%
\pgfpathlineto{\pgfqpoint{3.628011in}{5.360185in}}%
\pgfusepath{stroke}%
\end{pgfscope}%
\begin{pgfscope}%
\pgfsetbuttcap%
\pgfsetroundjoin%
\pgfsetlinewidth{0.803000pt}%
\definecolor{currentstroke}{rgb}{0.690196,0.690196,0.690196}%
\pgfsetstrokecolor{currentstroke}%
\pgfsetdash{}{0pt}%
\pgfpathmoveto{\pgfqpoint{2.222368in}{1.345767in}}%
\pgfpathlineto{\pgfqpoint{4.070468in}{2.956591in}}%
\pgfpathlineto{\pgfqpoint{4.070186in}{5.248106in}}%
\pgfusepath{stroke}%
\end{pgfscope}%
\begin{pgfscope}%
\pgfsetbuttcap%
\pgfsetroundjoin%
\pgfsetlinewidth{0.803000pt}%
\definecolor{currentstroke}{rgb}{0.690196,0.690196,0.690196}%
\pgfsetstrokecolor{currentstroke}%
\pgfsetdash{}{0pt}%
\pgfpathmoveto{\pgfqpoint{2.677247in}{1.197204in}}%
\pgfpathlineto{\pgfqpoint{4.500444in}{2.831614in}}%
\pgfpathlineto{\pgfqpoint{4.518800in}{5.134396in}}%
\pgfusepath{stroke}%
\end{pgfscope}%
\begin{pgfscope}%
\pgfsetbuttcap%
\pgfsetroundjoin%
\pgfsetlinewidth{0.803000pt}%
\definecolor{currentstroke}{rgb}{0.690196,0.690196,0.690196}%
\pgfsetstrokecolor{currentstroke}%
\pgfsetdash{}{0pt}%
\pgfpathmoveto{\pgfqpoint{3.139103in}{1.046362in}}%
\pgfpathlineto{\pgfqpoint{4.936464in}{2.704880in}}%
\pgfpathlineto{\pgfqpoint{4.973994in}{5.019017in}}%
\pgfusepath{stroke}%
\end{pgfscope}%
\begin{pgfscope}%
\pgfsetbuttcap%
\pgfsetroundjoin%
\pgfsetlinewidth{0.803000pt}%
\definecolor{currentstroke}{rgb}{0.690196,0.690196,0.690196}%
\pgfsetstrokecolor{currentstroke}%
\pgfsetdash{}{0pt}%
\pgfpathmoveto{\pgfqpoint{3.608098in}{0.893188in}}%
\pgfpathlineto{\pgfqpoint{5.378655in}{2.576352in}}%
\pgfpathlineto{\pgfqpoint{5.435914in}{4.901934in}}%
\pgfusepath{stroke}%
\end{pgfscope}%
\begin{pgfscope}%
\pgfsetbuttcap%
\pgfsetroundjoin%
\pgfsetlinewidth{0.803000pt}%
\definecolor{currentstroke}{rgb}{0.690196,0.690196,0.690196}%
\pgfsetstrokecolor{currentstroke}%
\pgfsetdash{}{0pt}%
\pgfpathmoveto{\pgfqpoint{4.084398in}{0.737628in}}%
\pgfpathlineto{\pgfqpoint{5.827149in}{2.445993in}}%
\pgfpathlineto{\pgfqpoint{5.904712in}{4.783107in}}%
\pgfusepath{stroke}%
\end{pgfscope}%
\begin{pgfscope}%
\pgfsetbuttcap%
\pgfsetroundjoin%
\pgfsetlinewidth{0.803000pt}%
\definecolor{currentstroke}{rgb}{0.690196,0.690196,0.690196}%
\pgfsetstrokecolor{currentstroke}%
\pgfsetdash{}{0pt}%
\pgfpathmoveto{\pgfqpoint{4.568177in}{0.579626in}}%
\pgfpathlineto{\pgfqpoint{6.282083in}{2.313762in}}%
\pgfpathlineto{\pgfqpoint{6.380540in}{4.662499in}}%
\pgfusepath{stroke}%
\end{pgfscope}%
\begin{pgfscope}%
\pgfsetrectcap%
\pgfsetroundjoin%
\pgfsetlinewidth{0.803000pt}%
\definecolor{currentstroke}{rgb}{0.000000,0.000000,0.000000}%
\pgfsetstrokecolor{currentstroke}%
\pgfsetdash{}{0pt}%
\pgfpathmoveto{\pgfqpoint{1.790612in}{1.505929in}}%
\pgfpathlineto{\pgfqpoint{1.741635in}{1.464392in}}%
\pgfusepath{stroke}%
\end{pgfscope}%
\begin{pgfscope}%
\definecolor{textcolor}{rgb}{0.000000,0.000000,0.000000}%
\pgfsetstrokecolor{textcolor}%
\pgfsetfillcolor{textcolor}%
\pgftext[x=1.669876in,y=1.274184in,,top]{\color{textcolor}\sffamily\fontsize{10.000000}{12.000000}\selectfont \ensuremath{-}1.0}%
\end{pgfscope}%
\begin{pgfscope}%
\pgfsetrectcap%
\pgfsetroundjoin%
\pgfsetlinewidth{0.803000pt}%
\definecolor{currentstroke}{rgb}{0.000000,0.000000,0.000000}%
\pgfsetstrokecolor{currentstroke}%
\pgfsetdash{}{0pt}%
\pgfpathmoveto{\pgfqpoint{2.238471in}{1.359803in}}%
\pgfpathlineto{\pgfqpoint{2.190092in}{1.317635in}}%
\pgfusepath{stroke}%
\end{pgfscope}%
\begin{pgfscope}%
\definecolor{textcolor}{rgb}{0.000000,0.000000,0.000000}%
\pgfsetstrokecolor{textcolor}%
\pgfsetfillcolor{textcolor}%
\pgftext[x=2.118230in,y=1.125843in,,top]{\color{textcolor}\sffamily\fontsize{10.000000}{12.000000}\selectfont \ensuremath{-}0.5}%
\end{pgfscope}%
\begin{pgfscope}%
\pgfsetrectcap%
\pgfsetroundjoin%
\pgfsetlinewidth{0.803000pt}%
\definecolor{currentstroke}{rgb}{0.000000,0.000000,0.000000}%
\pgfsetstrokecolor{currentstroke}%
\pgfsetdash{}{0pt}%
\pgfpathmoveto{\pgfqpoint{2.693143in}{1.211454in}}%
\pgfpathlineto{\pgfqpoint{2.645386in}{1.168642in}}%
\pgfusepath{stroke}%
\end{pgfscope}%
\begin{pgfscope}%
\definecolor{textcolor}{rgb}{0.000000,0.000000,0.000000}%
\pgfsetstrokecolor{textcolor}%
\pgfsetfillcolor{textcolor}%
\pgftext[x=2.573426in,y=0.975238in,,top]{\color{textcolor}\sffamily\fontsize{10.000000}{12.000000}\selectfont 0.0}%
\end{pgfscope}%
\begin{pgfscope}%
\pgfsetrectcap%
\pgfsetroundjoin%
\pgfsetlinewidth{0.803000pt}%
\definecolor{currentstroke}{rgb}{0.000000,0.000000,0.000000}%
\pgfsetstrokecolor{currentstroke}%
\pgfsetdash{}{0pt}%
\pgfpathmoveto{\pgfqpoint{3.154783in}{1.060831in}}%
\pgfpathlineto{\pgfqpoint{3.107673in}{1.017359in}}%
\pgfusepath{stroke}%
\end{pgfscope}%
\begin{pgfscope}%
\definecolor{textcolor}{rgb}{0.000000,0.000000,0.000000}%
\pgfsetstrokecolor{textcolor}%
\pgfsetfillcolor{textcolor}%
\pgftext[x=3.035622in,y=0.822318in,,top]{\color{textcolor}\sffamily\fontsize{10.000000}{12.000000}\selectfont 0.5}%
\end{pgfscope}%
\begin{pgfscope}%
\pgfsetrectcap%
\pgfsetroundjoin%
\pgfsetlinewidth{0.803000pt}%
\definecolor{currentstroke}{rgb}{0.000000,0.000000,0.000000}%
\pgfsetstrokecolor{currentstroke}%
\pgfsetdash{}{0pt}%
\pgfpathmoveto{\pgfqpoint{3.623554in}{0.907881in}}%
\pgfpathlineto{\pgfqpoint{3.577116in}{0.863735in}}%
\pgfusepath{stroke}%
\end{pgfscope}%
\begin{pgfscope}%
\definecolor{textcolor}{rgb}{0.000000,0.000000,0.000000}%
\pgfsetstrokecolor{textcolor}%
\pgfsetfillcolor{textcolor}%
\pgftext[x=3.504979in,y=0.667028in,,top]{\color{textcolor}\sffamily\fontsize{10.000000}{12.000000}\selectfont 1.0}%
\end{pgfscope}%
\begin{pgfscope}%
\pgfsetrectcap%
\pgfsetroundjoin%
\pgfsetlinewidth{0.803000pt}%
\definecolor{currentstroke}{rgb}{0.000000,0.000000,0.000000}%
\pgfsetstrokecolor{currentstroke}%
\pgfsetdash{}{0pt}%
\pgfpathmoveto{\pgfqpoint{4.099622in}{0.752551in}}%
\pgfpathlineto{\pgfqpoint{4.053883in}{0.707715in}}%
\pgfusepath{stroke}%
\end{pgfscope}%
\begin{pgfscope}%
\definecolor{textcolor}{rgb}{0.000000,0.000000,0.000000}%
\pgfsetstrokecolor{textcolor}%
\pgfsetfillcolor{textcolor}%
\pgftext[x=3.981665in,y=0.509314in,,top]{\color{textcolor}\sffamily\fontsize{10.000000}{12.000000}\selectfont 1.5}%
\end{pgfscope}%
\begin{pgfscope}%
\pgfsetrectcap%
\pgfsetroundjoin%
\pgfsetlinewidth{0.803000pt}%
\definecolor{currentstroke}{rgb}{0.000000,0.000000,0.000000}%
\pgfsetstrokecolor{currentstroke}%
\pgfsetdash{}{0pt}%
\pgfpathmoveto{\pgfqpoint{4.583158in}{0.594784in}}%
\pgfpathlineto{\pgfqpoint{4.538146in}{0.549241in}}%
\pgfusepath{stroke}%
\end{pgfscope}%
\begin{pgfscope}%
\definecolor{textcolor}{rgb}{0.000000,0.000000,0.000000}%
\pgfsetstrokecolor{textcolor}%
\pgfsetfillcolor{textcolor}%
\pgftext[x=4.465855in,y=0.349116in,,top]{\color{textcolor}\sffamily\fontsize{10.000000}{12.000000}\selectfont 2.0}%
\end{pgfscope}%
\begin{pgfscope}%
\pgfsetrectcap%
\pgfsetroundjoin%
\pgfsetlinewidth{0.803000pt}%
\definecolor{currentstroke}{rgb}{0.000000,0.000000,0.000000}%
\pgfsetstrokecolor{currentstroke}%
\pgfsetdash{}{0pt}%
\pgfpathmoveto{\pgfqpoint{6.483177in}{2.255311in}}%
\pgfpathlineto{\pgfqpoint{4.782226in}{0.509717in}}%
\pgfusepath{stroke}%
\end{pgfscope}%
\begin{pgfscope}%
\definecolor{textcolor}{rgb}{0.000000,0.000000,0.000000}%
\pgfsetstrokecolor{textcolor}%
\pgfsetfillcolor{textcolor}%
\pgftext[x=6.045209in,y=1.032725in,,]{\color{textcolor}\sffamily\fontsize{10.000000}{12.000000}\selectfont y}%
\end{pgfscope}%
\begin{pgfscope}%
\pgfsetbuttcap%
\pgfsetroundjoin%
\pgfsetlinewidth{0.803000pt}%
\definecolor{currentstroke}{rgb}{0.690196,0.690196,0.690196}%
\pgfsetstrokecolor{currentstroke}%
\pgfsetdash{}{0pt}%
\pgfpathmoveto{\pgfqpoint{1.600541in}{4.068879in}}%
\pgfpathlineto{\pgfqpoint{1.710097in}{1.664161in}}%
\pgfpathlineto{\pgfqpoint{4.899919in}{0.630499in}}%
\pgfusepath{stroke}%
\end{pgfscope}%
\begin{pgfscope}%
\pgfsetbuttcap%
\pgfsetroundjoin%
\pgfsetlinewidth{0.803000pt}%
\definecolor{currentstroke}{rgb}{0.690196,0.690196,0.690196}%
\pgfsetstrokecolor{currentstroke}%
\pgfsetdash{}{0pt}%
\pgfpathmoveto{\pgfqpoint{1.966540in}{4.335969in}}%
\pgfpathlineto{\pgfqpoint{2.058485in}{1.956187in}}%
\pgfpathlineto{\pgfqpoint{5.215679in}{0.954546in}}%
\pgfusepath{stroke}%
\end{pgfscope}%
\begin{pgfscope}%
\pgfsetbuttcap%
\pgfsetroundjoin%
\pgfsetlinewidth{0.803000pt}%
\definecolor{currentstroke}{rgb}{0.690196,0.690196,0.690196}%
\pgfsetstrokecolor{currentstroke}%
\pgfsetdash{}{0pt}%
\pgfpathmoveto{\pgfqpoint{2.321081in}{4.594697in}}%
\pgfpathlineto{\pgfqpoint{2.396435in}{2.239463in}}%
\pgfpathlineto{\pgfqpoint{5.521485in}{1.268378in}}%
\pgfusepath{stroke}%
\end{pgfscope}%
\begin{pgfscope}%
\pgfsetbuttcap%
\pgfsetroundjoin%
\pgfsetlinewidth{0.803000pt}%
\definecolor{currentstroke}{rgb}{0.690196,0.690196,0.690196}%
\pgfsetstrokecolor{currentstroke}%
\pgfsetdash{}{0pt}%
\pgfpathmoveto{\pgfqpoint{2.664695in}{4.845450in}}%
\pgfpathlineto{\pgfqpoint{2.724408in}{2.514377in}}%
\pgfpathlineto{\pgfqpoint{5.817800in}{1.572471in}}%
\pgfusepath{stroke}%
\end{pgfscope}%
\begin{pgfscope}%
\pgfsetbuttcap%
\pgfsetroundjoin%
\pgfsetlinewidth{0.803000pt}%
\definecolor{currentstroke}{rgb}{0.690196,0.690196,0.690196}%
\pgfsetstrokecolor{currentstroke}%
\pgfsetdash{}{0pt}%
\pgfpathmoveto{\pgfqpoint{2.997878in}{5.088592in}}%
\pgfpathlineto{\pgfqpoint{3.042841in}{2.781293in}}%
\pgfpathlineto{\pgfqpoint{6.105061in}{1.867271in}}%
\pgfusepath{stroke}%
\end{pgfscope}%
\begin{pgfscope}%
\pgfsetbuttcap%
\pgfsetroundjoin%
\pgfsetlinewidth{0.803000pt}%
\definecolor{currentstroke}{rgb}{0.690196,0.690196,0.690196}%
\pgfsetstrokecolor{currentstroke}%
\pgfsetdash{}{0pt}%
\pgfpathmoveto{\pgfqpoint{3.321099in}{5.324464in}}%
\pgfpathlineto{\pgfqpoint{3.352144in}{3.040557in}}%
\pgfpathlineto{\pgfqpoint{6.383674in}{2.153197in}}%
\pgfusepath{stroke}%
\end{pgfscope}%
\begin{pgfscope}%
\pgfsetrectcap%
\pgfsetroundjoin%
\pgfsetlinewidth{0.803000pt}%
\definecolor{currentstroke}{rgb}{0.000000,0.000000,0.000000}%
\pgfsetstrokecolor{currentstroke}%
\pgfsetdash{}{0pt}%
\pgfpathmoveto{\pgfqpoint{4.873038in}{0.639210in}}%
\pgfpathlineto{\pgfqpoint{4.953750in}{0.613055in}}%
\pgfusepath{stroke}%
\end{pgfscope}%
\begin{pgfscope}%
\definecolor{textcolor}{rgb}{0.000000,0.000000,0.000000}%
\pgfsetstrokecolor{textcolor}%
\pgfsetfillcolor{textcolor}%
\pgftext[x=5.078779in,y=0.444104in,,top]{\color{textcolor}\sffamily\fontsize{10.000000}{12.000000}\selectfont \ensuremath{-}1.0}%
\end{pgfscope}%
\begin{pgfscope}%
\pgfsetrectcap%
\pgfsetroundjoin%
\pgfsetlinewidth{0.803000pt}%
\definecolor{currentstroke}{rgb}{0.000000,0.000000,0.000000}%
\pgfsetstrokecolor{currentstroke}%
\pgfsetdash{}{0pt}%
\pgfpathmoveto{\pgfqpoint{5.189095in}{0.962980in}}%
\pgfpathlineto{\pgfqpoint{5.268915in}{0.937657in}}%
\pgfusepath{stroke}%
\end{pgfscope}%
\begin{pgfscope}%
\definecolor{textcolor}{rgb}{0.000000,0.000000,0.000000}%
\pgfsetstrokecolor{textcolor}%
\pgfsetfillcolor{textcolor}%
\pgftext[x=5.391794in,y=0.771261in,,top]{\color{textcolor}\sffamily\fontsize{10.000000}{12.000000}\selectfont \ensuremath{-}0.5}%
\end{pgfscope}%
\begin{pgfscope}%
\pgfsetrectcap%
\pgfsetroundjoin%
\pgfsetlinewidth{0.803000pt}%
\definecolor{currentstroke}{rgb}{0.000000,0.000000,0.000000}%
\pgfsetstrokecolor{currentstroke}%
\pgfsetdash{}{0pt}%
\pgfpathmoveto{\pgfqpoint{5.495192in}{1.276549in}}%
\pgfpathlineto{\pgfqpoint{5.574136in}{1.252017in}}%
\pgfusepath{stroke}%
\end{pgfscope}%
\begin{pgfscope}%
\definecolor{textcolor}{rgb}{0.000000,0.000000,0.000000}%
\pgfsetstrokecolor{textcolor}%
\pgfsetfillcolor{textcolor}%
\pgftext[x=5.694938in,y=1.088101in,,top]{\color{textcolor}\sffamily\fontsize{10.000000}{12.000000}\selectfont 0.0}%
\end{pgfscope}%
\begin{pgfscope}%
\pgfsetrectcap%
\pgfsetroundjoin%
\pgfsetlinewidth{0.803000pt}%
\definecolor{currentstroke}{rgb}{0.000000,0.000000,0.000000}%
\pgfsetstrokecolor{currentstroke}%
\pgfsetdash{}{0pt}%
\pgfpathmoveto{\pgfqpoint{5.791794in}{1.580390in}}%
\pgfpathlineto{\pgfqpoint{5.869878in}{1.556614in}}%
\pgfusepath{stroke}%
\end{pgfscope}%
\begin{pgfscope}%
\definecolor{textcolor}{rgb}{0.000000,0.000000,0.000000}%
\pgfsetstrokecolor{textcolor}%
\pgfsetfillcolor{textcolor}%
\pgftext[x=5.988671in,y=1.395104in,,top]{\color{textcolor}\sffamily\fontsize{10.000000}{12.000000}\selectfont 0.5}%
\end{pgfscope}%
\begin{pgfscope}%
\pgfsetrectcap%
\pgfsetroundjoin%
\pgfsetlinewidth{0.803000pt}%
\definecolor{currentstroke}{rgb}{0.000000,0.000000,0.000000}%
\pgfsetstrokecolor{currentstroke}%
\pgfsetdash{}{0pt}%
\pgfpathmoveto{\pgfqpoint{6.079335in}{1.874949in}}%
\pgfpathlineto{\pgfqpoint{6.156574in}{1.851895in}}%
\pgfusepath{stroke}%
\end{pgfscope}%
\begin{pgfscope}%
\definecolor{textcolor}{rgb}{0.000000,0.000000,0.000000}%
\pgfsetstrokecolor{textcolor}%
\pgfsetfillcolor{textcolor}%
\pgftext[x=6.273424in,y=1.692723in,,top]{\color{textcolor}\sffamily\fontsize{10.000000}{12.000000}\selectfont 1.0}%
\end{pgfscope}%
\begin{pgfscope}%
\pgfsetrectcap%
\pgfsetroundjoin%
\pgfsetlinewidth{0.803000pt}%
\definecolor{currentstroke}{rgb}{0.000000,0.000000,0.000000}%
\pgfsetstrokecolor{currentstroke}%
\pgfsetdash{}{0pt}%
\pgfpathmoveto{\pgfqpoint{6.358225in}{2.160646in}}%
\pgfpathlineto{\pgfqpoint{6.434634in}{2.138280in}}%
\pgfusepath{stroke}%
\end{pgfscope}%
\begin{pgfscope}%
\definecolor{textcolor}{rgb}{0.000000,0.000000,0.000000}%
\pgfsetstrokecolor{textcolor}%
\pgfsetfillcolor{textcolor}%
\pgftext[x=6.549603in,y=1.981379in,,top]{\color{textcolor}\sffamily\fontsize{10.000000}{12.000000}\selectfont 1.5}%
\end{pgfscope}%
\begin{pgfscope}%
\pgfsetrectcap%
\pgfsetroundjoin%
\pgfsetlinewidth{0.803000pt}%
\definecolor{currentstroke}{rgb}{0.000000,0.000000,0.000000}%
\pgfsetstrokecolor{currentstroke}%
\pgfsetdash{}{0pt}%
\pgfpathmoveto{\pgfqpoint{6.483177in}{2.255311in}}%
\pgfpathlineto{\pgfqpoint{6.590967in}{4.609162in}}%
\pgfusepath{stroke}%
\end{pgfscope}%
\begin{pgfscope}%
\definecolor{textcolor}{rgb}{0.000000,0.000000,0.000000}%
\pgfsetstrokecolor{textcolor}%
\pgfsetfillcolor{textcolor}%
\pgftext[x=7.097978in,y=3.481758in,,,rotate=87.378092]{\color{textcolor}\sffamily\fontsize{10.000000}{12.000000}\selectfont f(x,y)}%
\end{pgfscope}%
\begin{pgfscope}%
\pgfsetbuttcap%
\pgfsetroundjoin%
\pgfsetlinewidth{0.803000pt}%
\definecolor{currentstroke}{rgb}{0.690196,0.690196,0.690196}%
\pgfsetstrokecolor{currentstroke}%
\pgfsetdash{}{0pt}%
\pgfpathmoveto{\pgfqpoint{6.492331in}{2.455207in}}%
\pgfpathlineto{\pgfqpoint{3.460489in}{3.326854in}}%
\pgfpathlineto{\pgfqpoint{1.570533in}{1.760117in}}%
\pgfusepath{stroke}%
\end{pgfscope}%
\begin{pgfscope}%
\pgfsetbuttcap%
\pgfsetroundjoin%
\pgfsetlinewidth{0.803000pt}%
\definecolor{currentstroke}{rgb}{0.690196,0.690196,0.690196}%
\pgfsetstrokecolor{currentstroke}%
\pgfsetdash{}{0pt}%
\pgfpathmoveto{\pgfqpoint{6.504333in}{2.717297in}}%
\pgfpathlineto{\pgfqpoint{3.457571in}{3.580602in}}%
\pgfpathlineto{\pgfqpoint{1.557605in}{2.028568in}}%
\pgfusepath{stroke}%
\end{pgfscope}%
\begin{pgfscope}%
\pgfsetbuttcap%
\pgfsetroundjoin%
\pgfsetlinewidth{0.803000pt}%
\definecolor{currentstroke}{rgb}{0.690196,0.690196,0.690196}%
\pgfsetstrokecolor{currentstroke}%
\pgfsetdash{}{0pt}%
\pgfpathmoveto{\pgfqpoint{6.516455in}{2.982005in}}%
\pgfpathlineto{\pgfqpoint{3.454626in}{3.836763in}}%
\pgfpathlineto{\pgfqpoint{1.544543in}{2.299803in}}%
\pgfusepath{stroke}%
\end{pgfscope}%
\begin{pgfscope}%
\pgfsetbuttcap%
\pgfsetroundjoin%
\pgfsetlinewidth{0.803000pt}%
\definecolor{currentstroke}{rgb}{0.690196,0.690196,0.690196}%
\pgfsetstrokecolor{currentstroke}%
\pgfsetdash{}{0pt}%
\pgfpathmoveto{\pgfqpoint{6.528698in}{3.249370in}}%
\pgfpathlineto{\pgfqpoint{3.451652in}{4.095372in}}%
\pgfpathlineto{\pgfqpoint{1.531345in}{2.573865in}}%
\pgfusepath{stroke}%
\end{pgfscope}%
\begin{pgfscope}%
\pgfsetbuttcap%
\pgfsetroundjoin%
\pgfsetlinewidth{0.803000pt}%
\definecolor{currentstroke}{rgb}{0.690196,0.690196,0.690196}%
\pgfsetstrokecolor{currentstroke}%
\pgfsetdash{}{0pt}%
\pgfpathmoveto{\pgfqpoint{6.541065in}{3.519432in}}%
\pgfpathlineto{\pgfqpoint{3.448649in}{4.356463in}}%
\pgfpathlineto{\pgfqpoint{1.518009in}{2.850798in}}%
\pgfusepath{stroke}%
\end{pgfscope}%
\begin{pgfscope}%
\pgfsetbuttcap%
\pgfsetroundjoin%
\pgfsetlinewidth{0.803000pt}%
\definecolor{currentstroke}{rgb}{0.690196,0.690196,0.690196}%
\pgfsetstrokecolor{currentstroke}%
\pgfsetdash{}{0pt}%
\pgfpathmoveto{\pgfqpoint{6.553557in}{3.792232in}}%
\pgfpathlineto{\pgfqpoint{3.445618in}{4.620074in}}%
\pgfpathlineto{\pgfqpoint{1.504533in}{3.130648in}}%
\pgfusepath{stroke}%
\end{pgfscope}%
\begin{pgfscope}%
\pgfsetbuttcap%
\pgfsetroundjoin%
\pgfsetlinewidth{0.803000pt}%
\definecolor{currentstroke}{rgb}{0.690196,0.690196,0.690196}%
\pgfsetstrokecolor{currentstroke}%
\pgfsetdash{}{0pt}%
\pgfpathmoveto{\pgfqpoint{6.566177in}{4.067813in}}%
\pgfpathlineto{\pgfqpoint{3.442557in}{4.886241in}}%
\pgfpathlineto{\pgfqpoint{1.490913in}{3.413461in}}%
\pgfusepath{stroke}%
\end{pgfscope}%
\begin{pgfscope}%
\pgfsetbuttcap%
\pgfsetroundjoin%
\pgfsetlinewidth{0.803000pt}%
\definecolor{currentstroke}{rgb}{0.690196,0.690196,0.690196}%
\pgfsetstrokecolor{currentstroke}%
\pgfsetdash{}{0pt}%
\pgfpathmoveto{\pgfqpoint{6.578926in}{4.346216in}}%
\pgfpathlineto{\pgfqpoint{3.439467in}{5.155001in}}%
\pgfpathlineto{\pgfqpoint{1.477149in}{3.699285in}}%
\pgfusepath{stroke}%
\end{pgfscope}%
\begin{pgfscope}%
\pgfsetrectcap%
\pgfsetroundjoin%
\pgfsetlinewidth{0.803000pt}%
\definecolor{currentstroke}{rgb}{0.000000,0.000000,0.000000}%
\pgfsetstrokecolor{currentstroke}%
\pgfsetdash{}{0pt}%
\pgfpathmoveto{\pgfqpoint{6.466881in}{2.462524in}}%
\pgfpathlineto{\pgfqpoint{6.543292in}{2.440556in}}%
\pgfusepath{stroke}%
\end{pgfscope}%
\begin{pgfscope}%
\definecolor{textcolor}{rgb}{0.000000,0.000000,0.000000}%
\pgfsetstrokecolor{textcolor}%
\pgfsetfillcolor{textcolor}%
\pgftext[x=6.746997in,y=2.491173in,,top]{\color{textcolor}\sffamily\fontsize{10.000000}{12.000000}\selectfont 2}%
\end{pgfscope}%
\begin{pgfscope}%
\pgfsetrectcap%
\pgfsetroundjoin%
\pgfsetlinewidth{0.803000pt}%
\definecolor{currentstroke}{rgb}{0.000000,0.000000,0.000000}%
\pgfsetstrokecolor{currentstroke}%
\pgfsetdash{}{0pt}%
\pgfpathmoveto{\pgfqpoint{6.478752in}{2.724546in}}%
\pgfpathlineto{\pgfqpoint{6.555557in}{2.702783in}}%
\pgfusepath{stroke}%
\end{pgfscope}%
\begin{pgfscope}%
\definecolor{textcolor}{rgb}{0.000000,0.000000,0.000000}%
\pgfsetstrokecolor{textcolor}%
\pgfsetfillcolor{textcolor}%
\pgftext[x=6.760239in,y=2.752927in,,top]{\color{textcolor}\sffamily\fontsize{10.000000}{12.000000}\selectfont 4}%
\end{pgfscope}%
\begin{pgfscope}%
\pgfsetrectcap%
\pgfsetroundjoin%
\pgfsetlinewidth{0.803000pt}%
\definecolor{currentstroke}{rgb}{0.000000,0.000000,0.000000}%
\pgfsetstrokecolor{currentstroke}%
\pgfsetdash{}{0pt}%
\pgfpathmoveto{\pgfqpoint{6.490741in}{2.989183in}}%
\pgfpathlineto{\pgfqpoint{6.567944in}{2.967631in}}%
\pgfusepath{stroke}%
\end{pgfscope}%
\begin{pgfscope}%
\definecolor{textcolor}{rgb}{0.000000,0.000000,0.000000}%
\pgfsetstrokecolor{textcolor}%
\pgfsetfillcolor{textcolor}%
\pgftext[x=6.773612in,y=3.017290in,,top]{\color{textcolor}\sffamily\fontsize{10.000000}{12.000000}\selectfont 6}%
\end{pgfscope}%
\begin{pgfscope}%
\pgfsetrectcap%
\pgfsetroundjoin%
\pgfsetlinewidth{0.803000pt}%
\definecolor{currentstroke}{rgb}{0.000000,0.000000,0.000000}%
\pgfsetstrokecolor{currentstroke}%
\pgfsetdash{}{0pt}%
\pgfpathmoveto{\pgfqpoint{6.502850in}{3.256476in}}%
\pgfpathlineto{\pgfqpoint{6.580456in}{3.235139in}}%
\pgfusepath{stroke}%
\end{pgfscope}%
\begin{pgfscope}%
\definecolor{textcolor}{rgb}{0.000000,0.000000,0.000000}%
\pgfsetstrokecolor{textcolor}%
\pgfsetfillcolor{textcolor}%
\pgftext[x=6.787119in,y=3.284301in,,top]{\color{textcolor}\sffamily\fontsize{10.000000}{12.000000}\selectfont 8}%
\end{pgfscope}%
\begin{pgfscope}%
\pgfsetrectcap%
\pgfsetroundjoin%
\pgfsetlinewidth{0.803000pt}%
\definecolor{currentstroke}{rgb}{0.000000,0.000000,0.000000}%
\pgfsetstrokecolor{currentstroke}%
\pgfsetdash{}{0pt}%
\pgfpathmoveto{\pgfqpoint{6.515082in}{3.526464in}}%
\pgfpathlineto{\pgfqpoint{6.593095in}{3.505349in}}%
\pgfusepath{stroke}%
\end{pgfscope}%
\begin{pgfscope}%
\definecolor{textcolor}{rgb}{0.000000,0.000000,0.000000}%
\pgfsetstrokecolor{textcolor}%
\pgfsetfillcolor{textcolor}%
\pgftext[x=6.800762in,y=3.554001in,,top]{\color{textcolor}\sffamily\fontsize{10.000000}{12.000000}\selectfont 10}%
\end{pgfscope}%
\begin{pgfscope}%
\pgfsetrectcap%
\pgfsetroundjoin%
\pgfsetlinewidth{0.803000pt}%
\definecolor{currentstroke}{rgb}{0.000000,0.000000,0.000000}%
\pgfsetstrokecolor{currentstroke}%
\pgfsetdash{}{0pt}%
\pgfpathmoveto{\pgfqpoint{6.527438in}{3.799189in}}%
\pgfpathlineto{\pgfqpoint{6.605861in}{3.778300in}}%
\pgfusepath{stroke}%
\end{pgfscope}%
\begin{pgfscope}%
\definecolor{textcolor}{rgb}{0.000000,0.000000,0.000000}%
\pgfsetstrokecolor{textcolor}%
\pgfsetfillcolor{textcolor}%
\pgftext[x=6.814544in,y=3.826430in,,top]{\color{textcolor}\sffamily\fontsize{10.000000}{12.000000}\selectfont 12}%
\end{pgfscope}%
\begin{pgfscope}%
\pgfsetrectcap%
\pgfsetroundjoin%
\pgfsetlinewidth{0.803000pt}%
\definecolor{currentstroke}{rgb}{0.000000,0.000000,0.000000}%
\pgfsetstrokecolor{currentstroke}%
\pgfsetdash{}{0pt}%
\pgfpathmoveto{\pgfqpoint{6.539919in}{4.074693in}}%
\pgfpathlineto{\pgfqpoint{6.618758in}{4.054036in}}%
\pgfusepath{stroke}%
\end{pgfscope}%
\begin{pgfscope}%
\definecolor{textcolor}{rgb}{0.000000,0.000000,0.000000}%
\pgfsetstrokecolor{textcolor}%
\pgfsetfillcolor{textcolor}%
\pgftext[x=6.828465in,y=4.101629in,,top]{\color{textcolor}\sffamily\fontsize{10.000000}{12.000000}\selectfont 14}%
\end{pgfscope}%
\begin{pgfscope}%
\pgfsetrectcap%
\pgfsetroundjoin%
\pgfsetlinewidth{0.803000pt}%
\definecolor{currentstroke}{rgb}{0.000000,0.000000,0.000000}%
\pgfsetstrokecolor{currentstroke}%
\pgfsetdash{}{0pt}%
\pgfpathmoveto{\pgfqpoint{6.552528in}{4.353017in}}%
\pgfpathlineto{\pgfqpoint{6.631787in}{4.332598in}}%
\pgfusepath{stroke}%
\end{pgfscope}%
\begin{pgfscope}%
\definecolor{textcolor}{rgb}{0.000000,0.000000,0.000000}%
\pgfsetstrokecolor{textcolor}%
\pgfsetfillcolor{textcolor}%
\pgftext[x=6.842529in,y=4.379642in,,top]{\color{textcolor}\sffamily\fontsize{10.000000}{12.000000}\selectfont 16}%
\end{pgfscope}%
\begin{pgfscope}%
\pgfpathrectangle{\pgfqpoint{1.150000in}{0.150000in}}{\pgfqpoint{5.700000in}{5.700000in}}%
\pgfusepath{clip}%
\pgfsetrectcap%
\pgfsetroundjoin%
\pgfsetlinewidth{2.007500pt}%
\definecolor{currentstroke}{rgb}{0.000000,0.000000,0.000000}%
\pgfsetstrokecolor{currentstroke}%
\pgfsetdash{}{0pt}%
\pgfpathmoveto{\pgfqpoint{5.157229in}{2.372390in}}%
\pgfpathlineto{\pgfqpoint{4.752998in}{2.972348in}}%
\pgfpathlineto{\pgfqpoint{3.826061in}{2.578215in}}%
\pgfpathlineto{\pgfqpoint{3.401354in}{2.232610in}}%
\pgfpathlineto{\pgfqpoint{3.055108in}{1.796121in}}%
\pgfpathlineto{\pgfqpoint{3.274671in}{1.586296in}}%
\pgfpathlineto{\pgfqpoint{3.325628in}{1.506449in}}%
\pgfusepath{stroke}%
\end{pgfscope}%
\begin{pgfscope}%
\pgfpathrectangle{\pgfqpoint{1.150000in}{0.150000in}}{\pgfqpoint{5.700000in}{5.700000in}}%
\pgfusepath{clip}%
\pgfsetbuttcap%
\pgfsetroundjoin%
\definecolor{currentfill}{rgb}{1.000000,0.000000,0.000000}%
\pgfsetfillcolor{currentfill}%
\pgfsetfillopacity{0.300000}%
\pgfsetlinewidth{1.003750pt}%
\definecolor{currentstroke}{rgb}{1.000000,0.000000,0.000000}%
\pgfsetstrokecolor{currentstroke}%
\pgfsetstrokeopacity{0.300000}%
\pgfsetdash{}{0pt}%
\pgfpathmoveto{\pgfqpoint{4.752998in}{2.923244in}}%
\pgfpathcurveto{\pgfqpoint{4.766021in}{2.923244in}}{\pgfqpoint{4.778512in}{2.928418in}}{\pgfqpoint{4.787721in}{2.937626in}}%
\pgfpathcurveto{\pgfqpoint{4.796929in}{2.946834in}}{\pgfqpoint{4.802103in}{2.959325in}}{\pgfqpoint{4.802103in}{2.972348in}}%
\pgfpathcurveto{\pgfqpoint{4.802103in}{2.985371in}}{\pgfqpoint{4.796929in}{2.997862in}}{\pgfqpoint{4.787721in}{3.007070in}}%
\pgfpathcurveto{\pgfqpoint{4.778512in}{3.016279in}}{\pgfqpoint{4.766021in}{3.021453in}}{\pgfqpoint{4.752998in}{3.021453in}}%
\pgfpathcurveto{\pgfqpoint{4.739976in}{3.021453in}}{\pgfqpoint{4.727485in}{3.016279in}}{\pgfqpoint{4.718276in}{3.007070in}}%
\pgfpathcurveto{\pgfqpoint{4.709068in}{2.997862in}}{\pgfqpoint{4.703894in}{2.985371in}}{\pgfqpoint{4.703894in}{2.972348in}}%
\pgfpathcurveto{\pgfqpoint{4.703894in}{2.959325in}}{\pgfqpoint{4.709068in}{2.946834in}}{\pgfqpoint{4.718276in}{2.937626in}}%
\pgfpathcurveto{\pgfqpoint{4.727485in}{2.928418in}}{\pgfqpoint{4.739976in}{2.923244in}}{\pgfqpoint{4.752998in}{2.923244in}}%
\pgfpathlineto{\pgfqpoint{4.752998in}{2.923244in}}%
\pgfpathclose%
\pgfusepath{stroke,fill}%
\end{pgfscope}%
\begin{pgfscope}%
\pgfpathrectangle{\pgfqpoint{1.150000in}{0.150000in}}{\pgfqpoint{5.700000in}{5.700000in}}%
\pgfusepath{clip}%
\pgfsetbuttcap%
\pgfsetroundjoin%
\definecolor{currentfill}{rgb}{1.000000,0.000000,0.000000}%
\pgfsetfillcolor{currentfill}%
\pgfsetfillopacity{0.342535}%
\pgfsetlinewidth{1.003750pt}%
\definecolor{currentstroke}{rgb}{1.000000,0.000000,0.000000}%
\pgfsetstrokecolor{currentstroke}%
\pgfsetstrokeopacity{0.342535}%
\pgfsetdash{}{0pt}%
\pgfpathmoveto{\pgfqpoint{3.826061in}{2.529110in}}%
\pgfpathcurveto{\pgfqpoint{3.839084in}{2.529110in}}{\pgfqpoint{3.851575in}{2.534284in}}{\pgfqpoint{3.860783in}{2.543493in}}%
\pgfpathcurveto{\pgfqpoint{3.869992in}{2.552701in}}{\pgfqpoint{3.875166in}{2.565192in}}{\pgfqpoint{3.875166in}{2.578215in}}%
\pgfpathcurveto{\pgfqpoint{3.875166in}{2.591237in}}{\pgfqpoint{3.869992in}{2.603729in}}{\pgfqpoint{3.860783in}{2.612937in}}%
\pgfpathcurveto{\pgfqpoint{3.851575in}{2.622145in}}{\pgfqpoint{3.839084in}{2.627319in}}{\pgfqpoint{3.826061in}{2.627319in}}%
\pgfpathcurveto{\pgfqpoint{3.813038in}{2.627319in}}{\pgfqpoint{3.800547in}{2.622145in}}{\pgfqpoint{3.791339in}{2.612937in}}%
\pgfpathcurveto{\pgfqpoint{3.782131in}{2.603729in}}{\pgfqpoint{3.776957in}{2.591237in}}{\pgfqpoint{3.776957in}{2.578215in}}%
\pgfpathcurveto{\pgfqpoint{3.776957in}{2.565192in}}{\pgfqpoint{3.782131in}{2.552701in}}{\pgfqpoint{3.791339in}{2.543493in}}%
\pgfpathcurveto{\pgfqpoint{3.800547in}{2.534284in}}{\pgfqpoint{3.813038in}{2.529110in}}{\pgfqpoint{3.826061in}{2.529110in}}%
\pgfpathlineto{\pgfqpoint{3.826061in}{2.529110in}}%
\pgfpathclose%
\pgfusepath{stroke,fill}%
\end{pgfscope}%
\begin{pgfscope}%
\pgfpathrectangle{\pgfqpoint{1.150000in}{0.150000in}}{\pgfqpoint{5.700000in}{5.700000in}}%
\pgfusepath{clip}%
\pgfsetbuttcap%
\pgfsetroundjoin%
\definecolor{currentfill}{rgb}{1.000000,0.000000,0.000000}%
\pgfsetfillcolor{currentfill}%
\pgfsetfillopacity{0.504777}%
\pgfsetlinewidth{1.003750pt}%
\definecolor{currentstroke}{rgb}{1.000000,0.000000,0.000000}%
\pgfsetstrokecolor{currentstroke}%
\pgfsetstrokeopacity{0.504777}%
\pgfsetdash{}{0pt}%
\pgfpathmoveto{\pgfqpoint{3.401354in}{2.183505in}}%
\pgfpathcurveto{\pgfqpoint{3.414376in}{2.183505in}}{\pgfqpoint{3.426867in}{2.188679in}}{\pgfqpoint{3.436076in}{2.197887in}}%
\pgfpathcurveto{\pgfqpoint{3.445284in}{2.207096in}}{\pgfqpoint{3.450458in}{2.219587in}}{\pgfqpoint{3.450458in}{2.232610in}}%
\pgfpathcurveto{\pgfqpoint{3.450458in}{2.245632in}}{\pgfqpoint{3.445284in}{2.258123in}}{\pgfqpoint{3.436076in}{2.267332in}}%
\pgfpathcurveto{\pgfqpoint{3.426867in}{2.276540in}}{\pgfqpoint{3.414376in}{2.281714in}}{\pgfqpoint{3.401354in}{2.281714in}}%
\pgfpathcurveto{\pgfqpoint{3.388331in}{2.281714in}}{\pgfqpoint{3.375840in}{2.276540in}}{\pgfqpoint{3.366631in}{2.267332in}}%
\pgfpathcurveto{\pgfqpoint{3.357423in}{2.258123in}}{\pgfqpoint{3.352249in}{2.245632in}}{\pgfqpoint{3.352249in}{2.232610in}}%
\pgfpathcurveto{\pgfqpoint{3.352249in}{2.219587in}}{\pgfqpoint{3.357423in}{2.207096in}}{\pgfqpoint{3.366631in}{2.197887in}}%
\pgfpathcurveto{\pgfqpoint{3.375840in}{2.188679in}}{\pgfqpoint{3.388331in}{2.183505in}}{\pgfqpoint{3.401354in}{2.183505in}}%
\pgfpathlineto{\pgfqpoint{3.401354in}{2.183505in}}%
\pgfpathclose%
\pgfusepath{stroke,fill}%
\end{pgfscope}%
\begin{pgfscope}%
\pgfpathrectangle{\pgfqpoint{1.150000in}{0.150000in}}{\pgfqpoint{5.700000in}{5.700000in}}%
\pgfusepath{clip}%
\pgfsetbuttcap%
\pgfsetroundjoin%
\definecolor{currentfill}{rgb}{1.000000,0.000000,0.000000}%
\pgfsetfillcolor{currentfill}%
\pgfsetfillopacity{0.791384}%
\pgfsetlinewidth{1.003750pt}%
\definecolor{currentstroke}{rgb}{1.000000,0.000000,0.000000}%
\pgfsetstrokecolor{currentstroke}%
\pgfsetstrokeopacity{0.791384}%
\pgfsetdash{}{0pt}%
\pgfpathmoveto{\pgfqpoint{3.055108in}{1.747016in}}%
\pgfpathcurveto{\pgfqpoint{3.068131in}{1.747016in}}{\pgfqpoint{3.080622in}{1.752190in}}{\pgfqpoint{3.089831in}{1.761398in}}%
\pgfpathcurveto{\pgfqpoint{3.099039in}{1.770607in}}{\pgfqpoint{3.104213in}{1.783098in}}{\pgfqpoint{3.104213in}{1.796121in}}%
\pgfpathcurveto{\pgfqpoint{3.104213in}{1.809143in}}{\pgfqpoint{3.099039in}{1.821634in}}{\pgfqpoint{3.089831in}{1.830843in}}%
\pgfpathcurveto{\pgfqpoint{3.080622in}{1.840051in}}{\pgfqpoint{3.068131in}{1.845225in}}{\pgfqpoint{3.055108in}{1.845225in}}%
\pgfpathcurveto{\pgfqpoint{3.042086in}{1.845225in}}{\pgfqpoint{3.029595in}{1.840051in}}{\pgfqpoint{3.020386in}{1.830843in}}%
\pgfpathcurveto{\pgfqpoint{3.011178in}{1.821634in}}{\pgfqpoint{3.006004in}{1.809143in}}{\pgfqpoint{3.006004in}{1.796121in}}%
\pgfpathcurveto{\pgfqpoint{3.006004in}{1.783098in}}{\pgfqpoint{3.011178in}{1.770607in}}{\pgfqpoint{3.020386in}{1.761398in}}%
\pgfpathcurveto{\pgfqpoint{3.029595in}{1.752190in}}{\pgfqpoint{3.042086in}{1.747016in}}{\pgfqpoint{3.055108in}{1.747016in}}%
\pgfpathlineto{\pgfqpoint{3.055108in}{1.747016in}}%
\pgfpathclose%
\pgfusepath{stroke,fill}%
\end{pgfscope}%
\begin{pgfscope}%
\pgfpathrectangle{\pgfqpoint{1.150000in}{0.150000in}}{\pgfqpoint{5.700000in}{5.700000in}}%
\pgfusepath{clip}%
\pgfsetbuttcap%
\pgfsetroundjoin%
\definecolor{currentfill}{rgb}{1.000000,0.000000,0.000000}%
\pgfsetfillcolor{currentfill}%
\pgfsetfillopacity{0.797645}%
\pgfsetlinewidth{1.003750pt}%
\definecolor{currentstroke}{rgb}{1.000000,0.000000,0.000000}%
\pgfsetstrokecolor{currentstroke}%
\pgfsetstrokeopacity{0.797645}%
\pgfsetdash{}{0pt}%
\pgfpathmoveto{\pgfqpoint{5.157229in}{2.323285in}}%
\pgfpathcurveto{\pgfqpoint{5.170251in}{2.323285in}}{\pgfqpoint{5.182742in}{2.328459in}}{\pgfqpoint{5.191951in}{2.337667in}}%
\pgfpathcurveto{\pgfqpoint{5.201159in}{2.346876in}}{\pgfqpoint{5.206333in}{2.359367in}}{\pgfqpoint{5.206333in}{2.372390in}}%
\pgfpathcurveto{\pgfqpoint{5.206333in}{2.385412in}}{\pgfqpoint{5.201159in}{2.397903in}}{\pgfqpoint{5.191951in}{2.407112in}}%
\pgfpathcurveto{\pgfqpoint{5.182742in}{2.416320in}}{\pgfqpoint{5.170251in}{2.421494in}}{\pgfqpoint{5.157229in}{2.421494in}}%
\pgfpathcurveto{\pgfqpoint{5.144206in}{2.421494in}}{\pgfqpoint{5.131715in}{2.416320in}}{\pgfqpoint{5.122506in}{2.407112in}}%
\pgfpathcurveto{\pgfqpoint{5.113298in}{2.397903in}}{\pgfqpoint{5.108124in}{2.385412in}}{\pgfqpoint{5.108124in}{2.372390in}}%
\pgfpathcurveto{\pgfqpoint{5.108124in}{2.359367in}}{\pgfqpoint{5.113298in}{2.346876in}}{\pgfqpoint{5.122506in}{2.337667in}}%
\pgfpathcurveto{\pgfqpoint{5.131715in}{2.328459in}}{\pgfqpoint{5.144206in}{2.323285in}}{\pgfqpoint{5.157229in}{2.323285in}}%
\pgfpathlineto{\pgfqpoint{5.157229in}{2.323285in}}%
\pgfpathclose%
\pgfusepath{stroke,fill}%
\end{pgfscope}%
\begin{pgfscope}%
\pgfpathrectangle{\pgfqpoint{1.150000in}{0.150000in}}{\pgfqpoint{5.700000in}{5.700000in}}%
\pgfusepath{clip}%
\pgfsetbuttcap%
\pgfsetroundjoin%
\definecolor{currentfill}{rgb}{1.000000,0.000000,0.000000}%
\pgfsetfillcolor{currentfill}%
\pgfsetfillopacity{0.937672}%
\pgfsetlinewidth{1.003750pt}%
\definecolor{currentstroke}{rgb}{1.000000,0.000000,0.000000}%
\pgfsetstrokecolor{currentstroke}%
\pgfsetstrokeopacity{0.937672}%
\pgfsetdash{}{0pt}%
\pgfpathmoveto{\pgfqpoint{3.274671in}{1.537191in}}%
\pgfpathcurveto{\pgfqpoint{3.287694in}{1.537191in}}{\pgfqpoint{3.300185in}{1.542365in}}{\pgfqpoint{3.309393in}{1.551574in}}%
\pgfpathcurveto{\pgfqpoint{3.318602in}{1.560782in}}{\pgfqpoint{3.323776in}{1.573273in}}{\pgfqpoint{3.323776in}{1.586296in}}%
\pgfpathcurveto{\pgfqpoint{3.323776in}{1.599319in}}{\pgfqpoint{3.318602in}{1.611810in}}{\pgfqpoint{3.309393in}{1.621018in}}%
\pgfpathcurveto{\pgfqpoint{3.300185in}{1.630227in}}{\pgfqpoint{3.287694in}{1.635401in}}{\pgfqpoint{3.274671in}{1.635401in}}%
\pgfpathcurveto{\pgfqpoint{3.261648in}{1.635401in}}{\pgfqpoint{3.249157in}{1.630227in}}{\pgfqpoint{3.239949in}{1.621018in}}%
\pgfpathcurveto{\pgfqpoint{3.230740in}{1.611810in}}{\pgfqpoint{3.225566in}{1.599319in}}{\pgfqpoint{3.225566in}{1.586296in}}%
\pgfpathcurveto{\pgfqpoint{3.225566in}{1.573273in}}{\pgfqpoint{3.230740in}{1.560782in}}{\pgfqpoint{3.239949in}{1.551574in}}%
\pgfpathcurveto{\pgfqpoint{3.249157in}{1.542365in}}{\pgfqpoint{3.261648in}{1.537191in}}{\pgfqpoint{3.274671in}{1.537191in}}%
\pgfpathlineto{\pgfqpoint{3.274671in}{1.537191in}}%
\pgfpathclose%
\pgfusepath{stroke,fill}%
\end{pgfscope}%
\begin{pgfscope}%
\pgfpathrectangle{\pgfqpoint{1.150000in}{0.150000in}}{\pgfqpoint{5.700000in}{5.700000in}}%
\pgfusepath{clip}%
\pgfsetbuttcap%
\pgfsetroundjoin%
\definecolor{currentfill}{rgb}{1.000000,0.000000,0.000000}%
\pgfsetfillcolor{currentfill}%
\pgfsetlinewidth{1.003750pt}%
\definecolor{currentstroke}{rgb}{1.000000,0.000000,0.000000}%
\pgfsetstrokecolor{currentstroke}%
\pgfsetdash{}{0pt}%
\pgfpathmoveto{\pgfqpoint{3.325628in}{1.457344in}}%
\pgfpathcurveto{\pgfqpoint{3.338651in}{1.457344in}}{\pgfqpoint{3.351142in}{1.462518in}}{\pgfqpoint{3.360350in}{1.471726in}}%
\pgfpathcurveto{\pgfqpoint{3.369559in}{1.480935in}}{\pgfqpoint{3.374732in}{1.493426in}}{\pgfqpoint{3.374732in}{1.506449in}}%
\pgfpathcurveto{\pgfqpoint{3.374732in}{1.519471in}}{\pgfqpoint{3.369559in}{1.531962in}}{\pgfqpoint{3.360350in}{1.541171in}}%
\pgfpathcurveto{\pgfqpoint{3.351142in}{1.550379in}}{\pgfqpoint{3.338651in}{1.555553in}}{\pgfqpoint{3.325628in}{1.555553in}}%
\pgfpathcurveto{\pgfqpoint{3.312605in}{1.555553in}}{\pgfqpoint{3.300114in}{1.550379in}}{\pgfqpoint{3.290906in}{1.541171in}}%
\pgfpathcurveto{\pgfqpoint{3.281697in}{1.531962in}}{\pgfqpoint{3.276523in}{1.519471in}}{\pgfqpoint{3.276523in}{1.506449in}}%
\pgfpathcurveto{\pgfqpoint{3.276523in}{1.493426in}}{\pgfqpoint{3.281697in}{1.480935in}}{\pgfqpoint{3.290906in}{1.471726in}}%
\pgfpathcurveto{\pgfqpoint{3.300114in}{1.462518in}}{\pgfqpoint{3.312605in}{1.457344in}}{\pgfqpoint{3.325628in}{1.457344in}}%
\pgfpathlineto{\pgfqpoint{3.325628in}{1.457344in}}%
\pgfpathclose%
\pgfusepath{stroke,fill}%
\end{pgfscope}%
\begin{pgfscope}%
\pgfpathrectangle{\pgfqpoint{1.150000in}{0.150000in}}{\pgfqpoint{5.700000in}{5.700000in}}%
\pgfusepath{clip}%
\pgfsetbuttcap%
\pgfsetroundjoin%
\definecolor{currentfill}{rgb}{0.121148,0.592739,0.544641}%
\pgfsetfillcolor{currentfill}%
\pgfsetfillopacity{0.700000}%
\pgfsetlinewidth{0.000000pt}%
\definecolor{currentstroke}{rgb}{0.000000,0.000000,0.000000}%
\pgfsetstrokecolor{currentstroke}%
\pgfsetdash{}{0pt}%
\pgfpathmoveto{\pgfqpoint{3.249301in}{3.881386in}}%
\pgfpathlineto{\pgfqpoint{3.262598in}{3.865868in}}%
\pgfpathlineto{\pgfqpoint{3.275893in}{3.850491in}}%
\pgfpathlineto{\pgfqpoint{3.289185in}{3.835254in}}%
\pgfpathlineto{\pgfqpoint{3.302476in}{3.820155in}}%
\pgfpathlineto{\pgfqpoint{3.310161in}{3.846101in}}%
\pgfpathlineto{\pgfqpoint{3.317839in}{3.872475in}}%
\pgfpathlineto{\pgfqpoint{3.325512in}{3.899284in}}%
\pgfpathlineto{\pgfqpoint{3.333178in}{3.926539in}}%
\pgfpathlineto{\pgfqpoint{3.319881in}{3.942074in}}%
\pgfpathlineto{\pgfqpoint{3.306582in}{3.957747in}}%
\pgfpathlineto{\pgfqpoint{3.293281in}{3.973561in}}%
\pgfpathlineto{\pgfqpoint{3.279977in}{3.989516in}}%
\pgfpathlineto{\pgfqpoint{3.272318in}{3.961816in}}%
\pgfpathlineto{\pgfqpoint{3.264652in}{3.934566in}}%
\pgfpathlineto{\pgfqpoint{3.256980in}{3.907759in}}%
\pgfpathlineto{\pgfqpoint{3.249301in}{3.881386in}}%
\pgfpathclose%
\pgfusepath{fill}%
\end{pgfscope}%
\begin{pgfscope}%
\pgfpathrectangle{\pgfqpoint{1.150000in}{0.150000in}}{\pgfqpoint{5.700000in}{5.700000in}}%
\pgfusepath{clip}%
\pgfsetbuttcap%
\pgfsetroundjoin%
\definecolor{currentfill}{rgb}{0.126453,0.570633,0.549841}%
\pgfsetfillcolor{currentfill}%
\pgfsetfillopacity{0.700000}%
\pgfsetlinewidth{0.000000pt}%
\definecolor{currentstroke}{rgb}{0.000000,0.000000,0.000000}%
\pgfsetstrokecolor{currentstroke}%
\pgfsetdash{}{0pt}%
\pgfpathmoveto{\pgfqpoint{3.302476in}{3.820155in}}%
\pgfpathlineto{\pgfqpoint{3.315765in}{3.805194in}}%
\pgfpathlineto{\pgfqpoint{3.329053in}{3.790368in}}%
\pgfpathlineto{\pgfqpoint{3.342338in}{3.775676in}}%
\pgfpathlineto{\pgfqpoint{3.355623in}{3.761118in}}%
\pgfpathlineto{\pgfqpoint{3.363313in}{3.786638in}}%
\pgfpathlineto{\pgfqpoint{3.370997in}{3.812579in}}%
\pgfpathlineto{\pgfqpoint{3.378675in}{3.838952in}}%
\pgfpathlineto{\pgfqpoint{3.386348in}{3.865763in}}%
\pgfpathlineto{\pgfqpoint{3.373058in}{3.880755in}}%
\pgfpathlineto{\pgfqpoint{3.359767in}{3.895881in}}%
\pgfpathlineto{\pgfqpoint{3.346473in}{3.911142in}}%
\pgfpathlineto{\pgfqpoint{3.333178in}{3.926539in}}%
\pgfpathlineto{\pgfqpoint{3.325512in}{3.899284in}}%
\pgfpathlineto{\pgfqpoint{3.317839in}{3.872475in}}%
\pgfpathlineto{\pgfqpoint{3.310161in}{3.846101in}}%
\pgfpathlineto{\pgfqpoint{3.302476in}{3.820155in}}%
\pgfpathclose%
\pgfusepath{fill}%
\end{pgfscope}%
\begin{pgfscope}%
\pgfpathrectangle{\pgfqpoint{1.150000in}{0.150000in}}{\pgfqpoint{5.700000in}{5.700000in}}%
\pgfusepath{clip}%
\pgfsetbuttcap%
\pgfsetroundjoin%
\definecolor{currentfill}{rgb}{0.131172,0.555899,0.552459}%
\pgfsetfillcolor{currentfill}%
\pgfsetfillopacity{0.700000}%
\pgfsetlinewidth{0.000000pt}%
\definecolor{currentstroke}{rgb}{0.000000,0.000000,0.000000}%
\pgfsetstrokecolor{currentstroke}%
\pgfsetdash{}{0pt}%
\pgfpathmoveto{\pgfqpoint{3.218518in}{3.780072in}}%
\pgfpathlineto{\pgfqpoint{3.231810in}{3.764969in}}%
\pgfpathlineto{\pgfqpoint{3.245100in}{3.750005in}}%
\pgfpathlineto{\pgfqpoint{3.258388in}{3.735181in}}%
\pgfpathlineto{\pgfqpoint{3.271674in}{3.720495in}}%
\pgfpathlineto{\pgfqpoint{3.279384in}{3.744807in}}%
\pgfpathlineto{\pgfqpoint{3.287088in}{3.769517in}}%
\pgfpathlineto{\pgfqpoint{3.294785in}{3.794630in}}%
\pgfpathlineto{\pgfqpoint{3.302476in}{3.820155in}}%
\pgfpathlineto{\pgfqpoint{3.289185in}{3.835254in}}%
\pgfpathlineto{\pgfqpoint{3.275893in}{3.850491in}}%
\pgfpathlineto{\pgfqpoint{3.262598in}{3.865868in}}%
\pgfpathlineto{\pgfqpoint{3.249301in}{3.881386in}}%
\pgfpathlineto{\pgfqpoint{3.241615in}{3.855438in}}%
\pgfpathlineto{\pgfqpoint{3.233923in}{3.829909in}}%
\pgfpathlineto{\pgfqpoint{3.226224in}{3.804790in}}%
\pgfpathlineto{\pgfqpoint{3.218518in}{3.780072in}}%
\pgfpathclose%
\pgfusepath{fill}%
\end{pgfscope}%
\begin{pgfscope}%
\pgfpathrectangle{\pgfqpoint{1.150000in}{0.150000in}}{\pgfqpoint{5.700000in}{5.700000in}}%
\pgfusepath{clip}%
\pgfsetbuttcap%
\pgfsetroundjoin%
\definecolor{currentfill}{rgb}{0.133743,0.548535,0.553541}%
\pgfsetfillcolor{currentfill}%
\pgfsetfillopacity{0.700000}%
\pgfsetlinewidth{0.000000pt}%
\definecolor{currentstroke}{rgb}{0.000000,0.000000,0.000000}%
\pgfsetstrokecolor{currentstroke}%
\pgfsetdash{}{0pt}%
\pgfpathmoveto{\pgfqpoint{3.355623in}{3.761118in}}%
\pgfpathlineto{\pgfqpoint{3.368906in}{3.746692in}}%
\pgfpathlineto{\pgfqpoint{3.382187in}{3.732397in}}%
\pgfpathlineto{\pgfqpoint{3.395468in}{3.718231in}}%
\pgfpathlineto{\pgfqpoint{3.408748in}{3.704195in}}%
\pgfpathlineto{\pgfqpoint{3.416442in}{3.729290in}}%
\pgfpathlineto{\pgfqpoint{3.424132in}{3.754802in}}%
\pgfpathlineto{\pgfqpoint{3.431816in}{3.780739in}}%
\pgfpathlineto{\pgfqpoint{3.439495in}{3.807108in}}%
\pgfpathlineto{\pgfqpoint{3.426210in}{3.821577in}}%
\pgfpathlineto{\pgfqpoint{3.412924in}{3.836175in}}%
\pgfpathlineto{\pgfqpoint{3.399637in}{3.850903in}}%
\pgfpathlineto{\pgfqpoint{3.386348in}{3.865763in}}%
\pgfpathlineto{\pgfqpoint{3.378675in}{3.838952in}}%
\pgfpathlineto{\pgfqpoint{3.370997in}{3.812579in}}%
\pgfpathlineto{\pgfqpoint{3.363313in}{3.786638in}}%
\pgfpathlineto{\pgfqpoint{3.355623in}{3.761118in}}%
\pgfpathclose%
\pgfusepath{fill}%
\end{pgfscope}%
\begin{pgfscope}%
\pgfpathrectangle{\pgfqpoint{1.150000in}{0.150000in}}{\pgfqpoint{5.700000in}{5.700000in}}%
\pgfusepath{clip}%
\pgfsetbuttcap%
\pgfsetroundjoin%
\definecolor{currentfill}{rgb}{0.119423,0.611141,0.538982}%
\pgfsetfillcolor{currentfill}%
\pgfsetfillopacity{0.700000}%
\pgfsetlinewidth{0.000000pt}%
\definecolor{currentstroke}{rgb}{0.000000,0.000000,0.000000}%
\pgfsetstrokecolor{currentstroke}%
\pgfsetdash{}{0pt}%
\pgfpathmoveto{\pgfqpoint{3.333178in}{3.926539in}}%
\pgfpathlineto{\pgfqpoint{3.346473in}{3.911142in}}%
\pgfpathlineto{\pgfqpoint{3.359767in}{3.895881in}}%
\pgfpathlineto{\pgfqpoint{3.373058in}{3.880755in}}%
\pgfpathlineto{\pgfqpoint{3.386348in}{3.865763in}}%
\pgfpathlineto{\pgfqpoint{3.394016in}{3.893020in}}%
\pgfpathlineto{\pgfqpoint{3.401677in}{3.920734in}}%
\pgfpathlineto{\pgfqpoint{3.409334in}{3.948912in}}%
\pgfpathlineto{\pgfqpoint{3.416985in}{3.977563in}}%
\pgfpathlineto{\pgfqpoint{3.403688in}{3.993013in}}%
\pgfpathlineto{\pgfqpoint{3.390389in}{4.008598in}}%
\pgfpathlineto{\pgfqpoint{3.377088in}{4.024318in}}%
\pgfpathlineto{\pgfqpoint{3.363785in}{4.040175in}}%
\pgfpathlineto{\pgfqpoint{3.356142in}{4.011056in}}%
\pgfpathlineto{\pgfqpoint{3.348493in}{3.982416in}}%
\pgfpathlineto{\pgfqpoint{3.340839in}{3.954247in}}%
\pgfpathlineto{\pgfqpoint{3.333178in}{3.926539in}}%
\pgfpathclose%
\pgfusepath{fill}%
\end{pgfscope}%
\begin{pgfscope}%
\pgfpathrectangle{\pgfqpoint{1.150000in}{0.150000in}}{\pgfqpoint{5.700000in}{5.700000in}}%
\pgfusepath{clip}%
\pgfsetbuttcap%
\pgfsetroundjoin%
\definecolor{currentfill}{rgb}{0.121831,0.589055,0.545623}%
\pgfsetfillcolor{currentfill}%
\pgfsetfillopacity{0.700000}%
\pgfsetlinewidth{0.000000pt}%
\definecolor{currentstroke}{rgb}{0.000000,0.000000,0.000000}%
\pgfsetstrokecolor{currentstroke}%
\pgfsetdash{}{0pt}%
\pgfpathmoveto{\pgfqpoint{3.386348in}{3.865763in}}%
\pgfpathlineto{\pgfqpoint{3.399637in}{3.850903in}}%
\pgfpathlineto{\pgfqpoint{3.412924in}{3.836175in}}%
\pgfpathlineto{\pgfqpoint{3.426210in}{3.821577in}}%
\pgfpathlineto{\pgfqpoint{3.439495in}{3.807108in}}%
\pgfpathlineto{\pgfqpoint{3.447168in}{3.833918in}}%
\pgfpathlineto{\pgfqpoint{3.454837in}{3.861178in}}%
\pgfpathlineto{\pgfqpoint{3.462500in}{3.888896in}}%
\pgfpathlineto{\pgfqpoint{3.470158in}{3.917082in}}%
\pgfpathlineto{\pgfqpoint{3.456867in}{3.932006in}}%
\pgfpathlineto{\pgfqpoint{3.443574in}{3.947061in}}%
\pgfpathlineto{\pgfqpoint{3.430280in}{3.962246in}}%
\pgfpathlineto{\pgfqpoint{3.416985in}{3.977563in}}%
\pgfpathlineto{\pgfqpoint{3.409334in}{3.948912in}}%
\pgfpathlineto{\pgfqpoint{3.401677in}{3.920734in}}%
\pgfpathlineto{\pgfqpoint{3.394016in}{3.893020in}}%
\pgfpathlineto{\pgfqpoint{3.386348in}{3.865763in}}%
\pgfpathclose%
\pgfusepath{fill}%
\end{pgfscope}%
\begin{pgfscope}%
\pgfpathrectangle{\pgfqpoint{1.150000in}{0.150000in}}{\pgfqpoint{5.700000in}{5.700000in}}%
\pgfusepath{clip}%
\pgfsetbuttcap%
\pgfsetroundjoin%
\definecolor{currentfill}{rgb}{0.122312,0.633153,0.530398}%
\pgfsetfillcolor{currentfill}%
\pgfsetfillopacity{0.700000}%
\pgfsetlinewidth{0.000000pt}%
\definecolor{currentstroke}{rgb}{0.000000,0.000000,0.000000}%
\pgfsetstrokecolor{currentstroke}%
\pgfsetdash{}{0pt}%
\pgfpathmoveto{\pgfqpoint{3.279977in}{3.989516in}}%
\pgfpathlineto{\pgfqpoint{3.293281in}{3.973561in}}%
\pgfpathlineto{\pgfqpoint{3.306582in}{3.957747in}}%
\pgfpathlineto{\pgfqpoint{3.319881in}{3.942074in}}%
\pgfpathlineto{\pgfqpoint{3.333178in}{3.926539in}}%
\pgfpathlineto{\pgfqpoint{3.340839in}{3.954247in}}%
\pgfpathlineto{\pgfqpoint{3.348493in}{3.982416in}}%
\pgfpathlineto{\pgfqpoint{3.356142in}{4.011056in}}%
\pgfpathlineto{\pgfqpoint{3.363785in}{4.040175in}}%
\pgfpathlineto{\pgfqpoint{3.350480in}{4.056170in}}%
\pgfpathlineto{\pgfqpoint{3.337173in}{4.072304in}}%
\pgfpathlineto{\pgfqpoint{3.323864in}{4.088579in}}%
\pgfpathlineto{\pgfqpoint{3.310552in}{4.104996in}}%
\pgfpathlineto{\pgfqpoint{3.302917in}{4.075407in}}%
\pgfpathlineto{\pgfqpoint{3.295277in}{4.046303in}}%
\pgfpathlineto{\pgfqpoint{3.287630in}{4.017676in}}%
\pgfpathlineto{\pgfqpoint{3.279977in}{3.989516in}}%
\pgfpathclose%
\pgfusepath{fill}%
\end{pgfscope}%
\begin{pgfscope}%
\pgfpathrectangle{\pgfqpoint{1.150000in}{0.150000in}}{\pgfqpoint{5.700000in}{5.700000in}}%
\pgfusepath{clip}%
\pgfsetbuttcap%
\pgfsetroundjoin%
\definecolor{currentfill}{rgb}{0.139147,0.533812,0.555298}%
\pgfsetfillcolor{currentfill}%
\pgfsetfillopacity{0.700000}%
\pgfsetlinewidth{0.000000pt}%
\definecolor{currentstroke}{rgb}{0.000000,0.000000,0.000000}%
\pgfsetstrokecolor{currentstroke}%
\pgfsetdash{}{0pt}%
\pgfpathmoveto{\pgfqpoint{3.271674in}{3.720495in}}%
\pgfpathlineto{\pgfqpoint{3.284958in}{3.705945in}}%
\pgfpathlineto{\pgfqpoint{3.298241in}{3.691530in}}%
\pgfpathlineto{\pgfqpoint{3.311523in}{3.677250in}}%
\pgfpathlineto{\pgfqpoint{3.324803in}{3.663103in}}%
\pgfpathlineto{\pgfqpoint{3.332517in}{3.687013in}}%
\pgfpathlineto{\pgfqpoint{3.340225in}{3.711313in}}%
\pgfpathlineto{\pgfqpoint{3.347927in}{3.736013in}}%
\pgfpathlineto{\pgfqpoint{3.355623in}{3.761118in}}%
\pgfpathlineto{\pgfqpoint{3.342338in}{3.775676in}}%
\pgfpathlineto{\pgfqpoint{3.329053in}{3.790368in}}%
\pgfpathlineto{\pgfqpoint{3.315765in}{3.805194in}}%
\pgfpathlineto{\pgfqpoint{3.302476in}{3.820155in}}%
\pgfpathlineto{\pgfqpoint{3.294785in}{3.794630in}}%
\pgfpathlineto{\pgfqpoint{3.287088in}{3.769517in}}%
\pgfpathlineto{\pgfqpoint{3.279384in}{3.744807in}}%
\pgfpathlineto{\pgfqpoint{3.271674in}{3.720495in}}%
\pgfpathclose%
\pgfusepath{fill}%
\end{pgfscope}%
\begin{pgfscope}%
\pgfpathrectangle{\pgfqpoint{1.150000in}{0.150000in}}{\pgfqpoint{5.700000in}{5.700000in}}%
\pgfusepath{clip}%
\pgfsetbuttcap%
\pgfsetroundjoin%
\definecolor{currentfill}{rgb}{0.127568,0.566949,0.550556}%
\pgfsetfillcolor{currentfill}%
\pgfsetfillopacity{0.700000}%
\pgfsetlinewidth{0.000000pt}%
\definecolor{currentstroke}{rgb}{0.000000,0.000000,0.000000}%
\pgfsetstrokecolor{currentstroke}%
\pgfsetdash{}{0pt}%
\pgfpathmoveto{\pgfqpoint{3.439495in}{3.807108in}}%
\pgfpathlineto{\pgfqpoint{3.452778in}{3.792767in}}%
\pgfpathlineto{\pgfqpoint{3.466061in}{3.778552in}}%
\pgfpathlineto{\pgfqpoint{3.479343in}{3.764464in}}%
\pgfpathlineto{\pgfqpoint{3.492624in}{3.750500in}}%
\pgfpathlineto{\pgfqpoint{3.500303in}{3.776865in}}%
\pgfpathlineto{\pgfqpoint{3.507978in}{3.803673in}}%
\pgfpathlineto{\pgfqpoint{3.515648in}{3.830934in}}%
\pgfpathlineto{\pgfqpoint{3.523313in}{3.858656in}}%
\pgfpathlineto{\pgfqpoint{3.510026in}{3.873073in}}%
\pgfpathlineto{\pgfqpoint{3.496738in}{3.887616in}}%
\pgfpathlineto{\pgfqpoint{3.483449in}{3.902285in}}%
\pgfpathlineto{\pgfqpoint{3.470158in}{3.917082in}}%
\pgfpathlineto{\pgfqpoint{3.462500in}{3.888896in}}%
\pgfpathlineto{\pgfqpoint{3.454837in}{3.861178in}}%
\pgfpathlineto{\pgfqpoint{3.447168in}{3.833918in}}%
\pgfpathlineto{\pgfqpoint{3.439495in}{3.807108in}}%
\pgfpathclose%
\pgfusepath{fill}%
\end{pgfscope}%
\begin{pgfscope}%
\pgfpathrectangle{\pgfqpoint{1.150000in}{0.150000in}}{\pgfqpoint{5.700000in}{5.700000in}}%
\pgfusepath{clip}%
\pgfsetbuttcap%
\pgfsetroundjoin%
\definecolor{currentfill}{rgb}{0.141935,0.526453,0.555991}%
\pgfsetfillcolor{currentfill}%
\pgfsetfillopacity{0.700000}%
\pgfsetlinewidth{0.000000pt}%
\definecolor{currentstroke}{rgb}{0.000000,0.000000,0.000000}%
\pgfsetstrokecolor{currentstroke}%
\pgfsetdash{}{0pt}%
\pgfpathmoveto{\pgfqpoint{3.408748in}{3.704195in}}%
\pgfpathlineto{\pgfqpoint{3.422026in}{3.690285in}}%
\pgfpathlineto{\pgfqpoint{3.435304in}{3.676502in}}%
\pgfpathlineto{\pgfqpoint{3.448581in}{3.662845in}}%
\pgfpathlineto{\pgfqpoint{3.461858in}{3.649311in}}%
\pgfpathlineto{\pgfqpoint{3.469557in}{3.673984in}}%
\pgfpathlineto{\pgfqpoint{3.477251in}{3.699068in}}%
\pgfpathlineto{\pgfqpoint{3.484940in}{3.724571in}}%
\pgfpathlineto{\pgfqpoint{3.492624in}{3.750500in}}%
\pgfpathlineto{\pgfqpoint{3.479343in}{3.764464in}}%
\pgfpathlineto{\pgfqpoint{3.466061in}{3.778552in}}%
\pgfpathlineto{\pgfqpoint{3.452778in}{3.792767in}}%
\pgfpathlineto{\pgfqpoint{3.439495in}{3.807108in}}%
\pgfpathlineto{\pgfqpoint{3.431816in}{3.780739in}}%
\pgfpathlineto{\pgfqpoint{3.424132in}{3.754802in}}%
\pgfpathlineto{\pgfqpoint{3.416442in}{3.729290in}}%
\pgfpathlineto{\pgfqpoint{3.408748in}{3.704195in}}%
\pgfpathclose%
\pgfusepath{fill}%
\end{pgfscope}%
\begin{pgfscope}%
\pgfpathrectangle{\pgfqpoint{1.150000in}{0.150000in}}{\pgfqpoint{5.700000in}{5.700000in}}%
\pgfusepath{clip}%
\pgfsetbuttcap%
\pgfsetroundjoin%
\definecolor{currentfill}{rgb}{0.147607,0.511733,0.557049}%
\pgfsetfillcolor{currentfill}%
\pgfsetfillopacity{0.700000}%
\pgfsetlinewidth{0.000000pt}%
\definecolor{currentstroke}{rgb}{0.000000,0.000000,0.000000}%
\pgfsetstrokecolor{currentstroke}%
\pgfsetdash{}{0pt}%
\pgfpathmoveto{\pgfqpoint{3.324803in}{3.663103in}}%
\pgfpathlineto{\pgfqpoint{3.338081in}{3.649087in}}%
\pgfpathlineto{\pgfqpoint{3.351359in}{3.635201in}}%
\pgfpathlineto{\pgfqpoint{3.364636in}{3.621445in}}%
\pgfpathlineto{\pgfqpoint{3.377912in}{3.607817in}}%
\pgfpathlineto{\pgfqpoint{3.385629in}{3.631327in}}%
\pgfpathlineto{\pgfqpoint{3.393341in}{3.655221in}}%
\pgfpathlineto{\pgfqpoint{3.401047in}{3.679507in}}%
\pgfpathlineto{\pgfqpoint{3.408748in}{3.704195in}}%
\pgfpathlineto{\pgfqpoint{3.395468in}{3.718231in}}%
\pgfpathlineto{\pgfqpoint{3.382187in}{3.732397in}}%
\pgfpathlineto{\pgfqpoint{3.368906in}{3.746692in}}%
\pgfpathlineto{\pgfqpoint{3.355623in}{3.761118in}}%
\pgfpathlineto{\pgfqpoint{3.347927in}{3.736013in}}%
\pgfpathlineto{\pgfqpoint{3.340225in}{3.711313in}}%
\pgfpathlineto{\pgfqpoint{3.332517in}{3.687013in}}%
\pgfpathlineto{\pgfqpoint{3.324803in}{3.663103in}}%
\pgfpathclose%
\pgfusepath{fill}%
\end{pgfscope}%
\begin{pgfscope}%
\pgfpathrectangle{\pgfqpoint{1.150000in}{0.150000in}}{\pgfqpoint{5.700000in}{5.700000in}}%
\pgfusepath{clip}%
\pgfsetbuttcap%
\pgfsetroundjoin%
\definecolor{currentfill}{rgb}{0.144759,0.519093,0.556572}%
\pgfsetfillcolor{currentfill}%
\pgfsetfillopacity{0.700000}%
\pgfsetlinewidth{0.000000pt}%
\definecolor{currentstroke}{rgb}{0.000000,0.000000,0.000000}%
\pgfsetstrokecolor{currentstroke}%
\pgfsetdash{}{0pt}%
\pgfpathmoveto{\pgfqpoint{3.187626in}{3.685077in}}%
\pgfpathlineto{\pgfqpoint{3.200914in}{3.670363in}}%
\pgfpathlineto{\pgfqpoint{3.214200in}{3.655790in}}%
\pgfpathlineto{\pgfqpoint{3.227484in}{3.641356in}}%
\pgfpathlineto{\pgfqpoint{3.240767in}{3.627059in}}%
\pgfpathlineto{\pgfqpoint{3.248504in}{3.649860in}}%
\pgfpathlineto{\pgfqpoint{3.256234in}{3.673029in}}%
\pgfpathlineto{\pgfqpoint{3.263957in}{3.696571in}}%
\pgfpathlineto{\pgfqpoint{3.271674in}{3.720495in}}%
\pgfpathlineto{\pgfqpoint{3.258388in}{3.735181in}}%
\pgfpathlineto{\pgfqpoint{3.245100in}{3.750005in}}%
\pgfpathlineto{\pgfqpoint{3.231810in}{3.764969in}}%
\pgfpathlineto{\pgfqpoint{3.218518in}{3.780072in}}%
\pgfpathlineto{\pgfqpoint{3.210806in}{3.755750in}}%
\pgfpathlineto{\pgfqpoint{3.203086in}{3.731815in}}%
\pgfpathlineto{\pgfqpoint{3.195360in}{3.708260in}}%
\pgfpathlineto{\pgfqpoint{3.187626in}{3.685077in}}%
\pgfpathclose%
\pgfusepath{fill}%
\end{pgfscope}%
\begin{pgfscope}%
\pgfpathrectangle{\pgfqpoint{1.150000in}{0.150000in}}{\pgfqpoint{5.700000in}{5.700000in}}%
\pgfusepath{clip}%
\pgfsetbuttcap%
\pgfsetroundjoin%
\definecolor{currentfill}{rgb}{0.135066,0.544853,0.554029}%
\pgfsetfillcolor{currentfill}%
\pgfsetfillopacity{0.700000}%
\pgfsetlinewidth{0.000000pt}%
\definecolor{currentstroke}{rgb}{0.000000,0.000000,0.000000}%
\pgfsetstrokecolor{currentstroke}%
\pgfsetdash{}{0pt}%
\pgfpathmoveto{\pgfqpoint{3.492624in}{3.750500in}}%
\pgfpathlineto{\pgfqpoint{3.505905in}{3.736660in}}%
\pgfpathlineto{\pgfqpoint{3.519185in}{3.722942in}}%
\pgfpathlineto{\pgfqpoint{3.532464in}{3.709346in}}%
\pgfpathlineto{\pgfqpoint{3.545743in}{3.695870in}}%
\pgfpathlineto{\pgfqpoint{3.553428in}{3.721791in}}%
\pgfpathlineto{\pgfqpoint{3.561108in}{3.748149in}}%
\pgfpathlineto{\pgfqpoint{3.568783in}{3.774954in}}%
\pgfpathlineto{\pgfqpoint{3.576455in}{3.802215in}}%
\pgfpathlineto{\pgfqpoint{3.563170in}{3.816142in}}%
\pgfpathlineto{\pgfqpoint{3.549885in}{3.830191in}}%
\pgfpathlineto{\pgfqpoint{3.536599in}{3.844362in}}%
\pgfpathlineto{\pgfqpoint{3.523313in}{3.858656in}}%
\pgfpathlineto{\pgfqpoint{3.515648in}{3.830934in}}%
\pgfpathlineto{\pgfqpoint{3.507978in}{3.803673in}}%
\pgfpathlineto{\pgfqpoint{3.500303in}{3.776865in}}%
\pgfpathlineto{\pgfqpoint{3.492624in}{3.750500in}}%
\pgfpathclose%
\pgfusepath{fill}%
\end{pgfscope}%
\begin{pgfscope}%
\pgfpathrectangle{\pgfqpoint{1.150000in}{0.150000in}}{\pgfqpoint{5.700000in}{5.700000in}}%
\pgfusepath{clip}%
\pgfsetbuttcap%
\pgfsetroundjoin%
\definecolor{currentfill}{rgb}{0.122312,0.633153,0.530398}%
\pgfsetfillcolor{currentfill}%
\pgfsetfillopacity{0.700000}%
\pgfsetlinewidth{0.000000pt}%
\definecolor{currentstroke}{rgb}{0.000000,0.000000,0.000000}%
\pgfsetstrokecolor{currentstroke}%
\pgfsetdash{}{0pt}%
\pgfpathmoveto{\pgfqpoint{3.416985in}{3.977563in}}%
\pgfpathlineto{\pgfqpoint{3.430280in}{3.962246in}}%
\pgfpathlineto{\pgfqpoint{3.443574in}{3.947061in}}%
\pgfpathlineto{\pgfqpoint{3.456867in}{3.932006in}}%
\pgfpathlineto{\pgfqpoint{3.470158in}{3.917082in}}%
\pgfpathlineto{\pgfqpoint{3.477812in}{3.945743in}}%
\pgfpathlineto{\pgfqpoint{3.485461in}{3.974889in}}%
\pgfpathlineto{\pgfqpoint{3.493106in}{4.004530in}}%
\pgfpathlineto{\pgfqpoint{3.500746in}{4.034673in}}%
\pgfpathlineto{\pgfqpoint{3.487446in}{4.050078in}}%
\pgfpathlineto{\pgfqpoint{3.474145in}{4.065614in}}%
\pgfpathlineto{\pgfqpoint{3.460843in}{4.081280in}}%
\pgfpathlineto{\pgfqpoint{3.447538in}{4.097079in}}%
\pgfpathlineto{\pgfqpoint{3.439907in}{4.066445in}}%
\pgfpathlineto{\pgfqpoint{3.432272in}{4.036320in}}%
\pgfpathlineto{\pgfqpoint{3.424631in}{4.006696in}}%
\pgfpathlineto{\pgfqpoint{3.416985in}{3.977563in}}%
\pgfpathclose%
\pgfusepath{fill}%
\end{pgfscope}%
\begin{pgfscope}%
\pgfpathrectangle{\pgfqpoint{1.150000in}{0.150000in}}{\pgfqpoint{5.700000in}{5.700000in}}%
\pgfusepath{clip}%
\pgfsetbuttcap%
\pgfsetroundjoin%
\definecolor{currentfill}{rgb}{0.119423,0.611141,0.538982}%
\pgfsetfillcolor{currentfill}%
\pgfsetfillopacity{0.700000}%
\pgfsetlinewidth{0.000000pt}%
\definecolor{currentstroke}{rgb}{0.000000,0.000000,0.000000}%
\pgfsetstrokecolor{currentstroke}%
\pgfsetdash{}{0pt}%
\pgfpathmoveto{\pgfqpoint{3.470158in}{3.917082in}}%
\pgfpathlineto{\pgfqpoint{3.483449in}{3.902285in}}%
\pgfpathlineto{\pgfqpoint{3.496738in}{3.887616in}}%
\pgfpathlineto{\pgfqpoint{3.510026in}{3.873073in}}%
\pgfpathlineto{\pgfqpoint{3.523313in}{3.858656in}}%
\pgfpathlineto{\pgfqpoint{3.530974in}{3.886847in}}%
\pgfpathlineto{\pgfqpoint{3.538630in}{3.915518in}}%
\pgfpathlineto{\pgfqpoint{3.546283in}{3.944676in}}%
\pgfpathlineto{\pgfqpoint{3.553931in}{3.974332in}}%
\pgfpathlineto{\pgfqpoint{3.540637in}{3.989228in}}%
\pgfpathlineto{\pgfqpoint{3.527341in}{4.004249in}}%
\pgfpathlineto{\pgfqpoint{3.514044in}{4.019397in}}%
\pgfpathlineto{\pgfqpoint{3.500746in}{4.034673in}}%
\pgfpathlineto{\pgfqpoint{3.493106in}{4.004530in}}%
\pgfpathlineto{\pgfqpoint{3.485461in}{3.974889in}}%
\pgfpathlineto{\pgfqpoint{3.477812in}{3.945743in}}%
\pgfpathlineto{\pgfqpoint{3.470158in}{3.917082in}}%
\pgfpathclose%
\pgfusepath{fill}%
\end{pgfscope}%
\begin{pgfscope}%
\pgfpathrectangle{\pgfqpoint{1.150000in}{0.150000in}}{\pgfqpoint{5.700000in}{5.700000in}}%
\pgfusepath{clip}%
\pgfsetbuttcap%
\pgfsetroundjoin%
\definecolor{currentfill}{rgb}{0.132268,0.655014,0.519661}%
\pgfsetfillcolor{currentfill}%
\pgfsetfillopacity{0.700000}%
\pgfsetlinewidth{0.000000pt}%
\definecolor{currentstroke}{rgb}{0.000000,0.000000,0.000000}%
\pgfsetstrokecolor{currentstroke}%
\pgfsetdash{}{0pt}%
\pgfpathmoveto{\pgfqpoint{3.363785in}{4.040175in}}%
\pgfpathlineto{\pgfqpoint{3.377088in}{4.024318in}}%
\pgfpathlineto{\pgfqpoint{3.390389in}{4.008598in}}%
\pgfpathlineto{\pgfqpoint{3.403688in}{3.993013in}}%
\pgfpathlineto{\pgfqpoint{3.416985in}{3.977563in}}%
\pgfpathlineto{\pgfqpoint{3.424631in}{4.006696in}}%
\pgfpathlineto{\pgfqpoint{3.432272in}{4.036320in}}%
\pgfpathlineto{\pgfqpoint{3.439907in}{4.066445in}}%
\pgfpathlineto{\pgfqpoint{3.447538in}{4.097079in}}%
\pgfpathlineto{\pgfqpoint{3.434232in}{4.113012in}}%
\pgfpathlineto{\pgfqpoint{3.420924in}{4.129079in}}%
\pgfpathlineto{\pgfqpoint{3.407615in}{4.145283in}}%
\pgfpathlineto{\pgfqpoint{3.394302in}{4.161624in}}%
\pgfpathlineto{\pgfqpoint{3.386681in}{4.130497in}}%
\pgfpathlineto{\pgfqpoint{3.379055in}{4.099886in}}%
\pgfpathlineto{\pgfqpoint{3.371423in}{4.069782in}}%
\pgfpathlineto{\pgfqpoint{3.363785in}{4.040175in}}%
\pgfpathclose%
\pgfusepath{fill}%
\end{pgfscope}%
\begin{pgfscope}%
\pgfpathrectangle{\pgfqpoint{1.150000in}{0.150000in}}{\pgfqpoint{5.700000in}{5.700000in}}%
\pgfusepath{clip}%
\pgfsetbuttcap%
\pgfsetroundjoin%
\definecolor{currentfill}{rgb}{0.149039,0.508051,0.557250}%
\pgfsetfillcolor{currentfill}%
\pgfsetfillopacity{0.700000}%
\pgfsetlinewidth{0.000000pt}%
\definecolor{currentstroke}{rgb}{0.000000,0.000000,0.000000}%
\pgfsetstrokecolor{currentstroke}%
\pgfsetdash{}{0pt}%
\pgfpathmoveto{\pgfqpoint{3.461858in}{3.649311in}}%
\pgfpathlineto{\pgfqpoint{3.475134in}{3.635901in}}%
\pgfpathlineto{\pgfqpoint{3.488410in}{3.622613in}}%
\pgfpathlineto{\pgfqpoint{3.501685in}{3.609446in}}%
\pgfpathlineto{\pgfqpoint{3.514960in}{3.596399in}}%
\pgfpathlineto{\pgfqpoint{3.522663in}{3.620651in}}%
\pgfpathlineto{\pgfqpoint{3.530361in}{3.645308in}}%
\pgfpathlineto{\pgfqpoint{3.538055in}{3.670379in}}%
\pgfpathlineto{\pgfqpoint{3.545743in}{3.695870in}}%
\pgfpathlineto{\pgfqpoint{3.532464in}{3.709346in}}%
\pgfpathlineto{\pgfqpoint{3.519185in}{3.722942in}}%
\pgfpathlineto{\pgfqpoint{3.505905in}{3.736660in}}%
\pgfpathlineto{\pgfqpoint{3.492624in}{3.750500in}}%
\pgfpathlineto{\pgfqpoint{3.484940in}{3.724571in}}%
\pgfpathlineto{\pgfqpoint{3.477251in}{3.699068in}}%
\pgfpathlineto{\pgfqpoint{3.469557in}{3.673984in}}%
\pgfpathlineto{\pgfqpoint{3.461858in}{3.649311in}}%
\pgfpathclose%
\pgfusepath{fill}%
\end{pgfscope}%
\begin{pgfscope}%
\pgfpathrectangle{\pgfqpoint{1.150000in}{0.150000in}}{\pgfqpoint{5.700000in}{5.700000in}}%
\pgfusepath{clip}%
\pgfsetbuttcap%
\pgfsetroundjoin%
\definecolor{currentfill}{rgb}{0.154815,0.493313,0.557840}%
\pgfsetfillcolor{currentfill}%
\pgfsetfillopacity{0.700000}%
\pgfsetlinewidth{0.000000pt}%
\definecolor{currentstroke}{rgb}{0.000000,0.000000,0.000000}%
\pgfsetstrokecolor{currentstroke}%
\pgfsetdash{}{0pt}%
\pgfpathmoveto{\pgfqpoint{3.377912in}{3.607817in}}%
\pgfpathlineto{\pgfqpoint{3.391187in}{3.594317in}}%
\pgfpathlineto{\pgfqpoint{3.404461in}{3.580942in}}%
\pgfpathlineto{\pgfqpoint{3.417735in}{3.567692in}}%
\pgfpathlineto{\pgfqpoint{3.431008in}{3.554566in}}%
\pgfpathlineto{\pgfqpoint{3.438729in}{3.577676in}}%
\pgfpathlineto{\pgfqpoint{3.446444in}{3.601165in}}%
\pgfpathlineto{\pgfqpoint{3.454154in}{3.625041in}}%
\pgfpathlineto{\pgfqpoint{3.461858in}{3.649311in}}%
\pgfpathlineto{\pgfqpoint{3.448581in}{3.662845in}}%
\pgfpathlineto{\pgfqpoint{3.435304in}{3.676502in}}%
\pgfpathlineto{\pgfqpoint{3.422026in}{3.690285in}}%
\pgfpathlineto{\pgfqpoint{3.408748in}{3.704195in}}%
\pgfpathlineto{\pgfqpoint{3.401047in}{3.679507in}}%
\pgfpathlineto{\pgfqpoint{3.393341in}{3.655221in}}%
\pgfpathlineto{\pgfqpoint{3.385629in}{3.631327in}}%
\pgfpathlineto{\pgfqpoint{3.377912in}{3.607817in}}%
\pgfpathclose%
\pgfusepath{fill}%
\end{pgfscope}%
\begin{pgfscope}%
\pgfpathrectangle{\pgfqpoint{1.150000in}{0.150000in}}{\pgfqpoint{5.700000in}{5.700000in}}%
\pgfusepath{clip}%
\pgfsetbuttcap%
\pgfsetroundjoin%
\definecolor{currentfill}{rgb}{0.153364,0.497000,0.557724}%
\pgfsetfillcolor{currentfill}%
\pgfsetfillopacity{0.700000}%
\pgfsetlinewidth{0.000000pt}%
\definecolor{currentstroke}{rgb}{0.000000,0.000000,0.000000}%
\pgfsetstrokecolor{currentstroke}%
\pgfsetdash{}{0pt}%
\pgfpathmoveto{\pgfqpoint{3.240767in}{3.627059in}}%
\pgfpathlineto{\pgfqpoint{3.254048in}{3.612898in}}%
\pgfpathlineto{\pgfqpoint{3.267328in}{3.598872in}}%
\pgfpathlineto{\pgfqpoint{3.280606in}{3.584980in}}%
\pgfpathlineto{\pgfqpoint{3.293883in}{3.571220in}}%
\pgfpathlineto{\pgfqpoint{3.301623in}{3.593642in}}%
\pgfpathlineto{\pgfqpoint{3.309356in}{3.616425in}}%
\pgfpathlineto{\pgfqpoint{3.317082in}{3.639576in}}%
\pgfpathlineto{\pgfqpoint{3.324803in}{3.663103in}}%
\pgfpathlineto{\pgfqpoint{3.311523in}{3.677250in}}%
\pgfpathlineto{\pgfqpoint{3.298241in}{3.691530in}}%
\pgfpathlineto{\pgfqpoint{3.284958in}{3.705945in}}%
\pgfpathlineto{\pgfqpoint{3.271674in}{3.720495in}}%
\pgfpathlineto{\pgfqpoint{3.263957in}{3.696571in}}%
\pgfpathlineto{\pgfqpoint{3.256234in}{3.673029in}}%
\pgfpathlineto{\pgfqpoint{3.248504in}{3.649860in}}%
\pgfpathlineto{\pgfqpoint{3.240767in}{3.627059in}}%
\pgfpathclose%
\pgfusepath{fill}%
\end{pgfscope}%
\begin{pgfscope}%
\pgfpathrectangle{\pgfqpoint{1.150000in}{0.150000in}}{\pgfqpoint{5.700000in}{5.700000in}}%
\pgfusepath{clip}%
\pgfsetbuttcap%
\pgfsetroundjoin%
\definecolor{currentfill}{rgb}{0.121831,0.589055,0.545623}%
\pgfsetfillcolor{currentfill}%
\pgfsetfillopacity{0.700000}%
\pgfsetlinewidth{0.000000pt}%
\definecolor{currentstroke}{rgb}{0.000000,0.000000,0.000000}%
\pgfsetstrokecolor{currentstroke}%
\pgfsetdash{}{0pt}%
\pgfpathmoveto{\pgfqpoint{3.523313in}{3.858656in}}%
\pgfpathlineto{\pgfqpoint{3.536599in}{3.844362in}}%
\pgfpathlineto{\pgfqpoint{3.549885in}{3.830191in}}%
\pgfpathlineto{\pgfqpoint{3.563170in}{3.816142in}}%
\pgfpathlineto{\pgfqpoint{3.576455in}{3.802215in}}%
\pgfpathlineto{\pgfqpoint{3.584122in}{3.829939in}}%
\pgfpathlineto{\pgfqpoint{3.591786in}{3.858136in}}%
\pgfpathlineto{\pgfqpoint{3.599446in}{3.886814in}}%
\pgfpathlineto{\pgfqpoint{3.607102in}{3.915984in}}%
\pgfpathlineto{\pgfqpoint{3.593810in}{3.930388in}}%
\pgfpathlineto{\pgfqpoint{3.580518in}{3.944913in}}%
\pgfpathlineto{\pgfqpoint{3.567225in}{3.959561in}}%
\pgfpathlineto{\pgfqpoint{3.553931in}{3.974332in}}%
\pgfpathlineto{\pgfqpoint{3.546283in}{3.944676in}}%
\pgfpathlineto{\pgfqpoint{3.538630in}{3.915518in}}%
\pgfpathlineto{\pgfqpoint{3.530974in}{3.886847in}}%
\pgfpathlineto{\pgfqpoint{3.523313in}{3.858656in}}%
\pgfpathclose%
\pgfusepath{fill}%
\end{pgfscope}%
\begin{pgfscope}%
\pgfpathrectangle{\pgfqpoint{1.150000in}{0.150000in}}{\pgfqpoint{5.700000in}{5.700000in}}%
\pgfusepath{clip}%
\pgfsetbuttcap%
\pgfsetroundjoin%
\definecolor{currentfill}{rgb}{0.153894,0.680203,0.504172}%
\pgfsetfillcolor{currentfill}%
\pgfsetfillopacity{0.700000}%
\pgfsetlinewidth{0.000000pt}%
\definecolor{currentstroke}{rgb}{0.000000,0.000000,0.000000}%
\pgfsetstrokecolor{currentstroke}%
\pgfsetdash{}{0pt}%
\pgfpathmoveto{\pgfqpoint{3.310552in}{4.104996in}}%
\pgfpathlineto{\pgfqpoint{3.323864in}{4.088579in}}%
\pgfpathlineto{\pgfqpoint{3.337173in}{4.072304in}}%
\pgfpathlineto{\pgfqpoint{3.350480in}{4.056170in}}%
\pgfpathlineto{\pgfqpoint{3.363785in}{4.040175in}}%
\pgfpathlineto{\pgfqpoint{3.371423in}{4.069782in}}%
\pgfpathlineto{\pgfqpoint{3.379055in}{4.099886in}}%
\pgfpathlineto{\pgfqpoint{3.386681in}{4.130497in}}%
\pgfpathlineto{\pgfqpoint{3.394302in}{4.161624in}}%
\pgfpathlineto{\pgfqpoint{3.380988in}{4.178105in}}%
\pgfpathlineto{\pgfqpoint{3.367671in}{4.194725in}}%
\pgfpathlineto{\pgfqpoint{3.354352in}{4.211486in}}%
\pgfpathlineto{\pgfqpoint{3.341030in}{4.228391in}}%
\pgfpathlineto{\pgfqpoint{3.333419in}{4.196768in}}%
\pgfpathlineto{\pgfqpoint{3.325803in}{4.165668in}}%
\pgfpathlineto{\pgfqpoint{3.318180in}{4.135080in}}%
\pgfpathlineto{\pgfqpoint{3.310552in}{4.104996in}}%
\pgfpathclose%
\pgfusepath{fill}%
\end{pgfscope}%
\begin{pgfscope}%
\pgfpathrectangle{\pgfqpoint{1.150000in}{0.150000in}}{\pgfqpoint{5.700000in}{5.700000in}}%
\pgfusepath{clip}%
\pgfsetbuttcap%
\pgfsetroundjoin%
\definecolor{currentfill}{rgb}{0.141935,0.526453,0.555991}%
\pgfsetfillcolor{currentfill}%
\pgfsetfillopacity{0.700000}%
\pgfsetlinewidth{0.000000pt}%
\definecolor{currentstroke}{rgb}{0.000000,0.000000,0.000000}%
\pgfsetstrokecolor{currentstroke}%
\pgfsetdash{}{0pt}%
\pgfpathmoveto{\pgfqpoint{3.545743in}{3.695870in}}%
\pgfpathlineto{\pgfqpoint{3.559023in}{3.682514in}}%
\pgfpathlineto{\pgfqpoint{3.572301in}{3.669276in}}%
\pgfpathlineto{\pgfqpoint{3.585580in}{3.656156in}}%
\pgfpathlineto{\pgfqpoint{3.598859in}{3.643152in}}%
\pgfpathlineto{\pgfqpoint{3.606548in}{3.668631in}}%
\pgfpathlineto{\pgfqpoint{3.614233in}{3.694541in}}%
\pgfpathlineto{\pgfqpoint{3.621913in}{3.720892in}}%
\pgfpathlineto{\pgfqpoint{3.629590in}{3.747693in}}%
\pgfpathlineto{\pgfqpoint{3.616307in}{3.761147in}}%
\pgfpathlineto{\pgfqpoint{3.603023in}{3.774718in}}%
\pgfpathlineto{\pgfqpoint{3.589739in}{3.788407in}}%
\pgfpathlineto{\pgfqpoint{3.576455in}{3.802215in}}%
\pgfpathlineto{\pgfqpoint{3.568783in}{3.774954in}}%
\pgfpathlineto{\pgfqpoint{3.561108in}{3.748149in}}%
\pgfpathlineto{\pgfqpoint{3.553428in}{3.721791in}}%
\pgfpathlineto{\pgfqpoint{3.545743in}{3.695870in}}%
\pgfpathclose%
\pgfusepath{fill}%
\end{pgfscope}%
\begin{pgfscope}%
\pgfpathrectangle{\pgfqpoint{1.150000in}{0.150000in}}{\pgfqpoint{5.700000in}{5.700000in}}%
\pgfusepath{clip}%
\pgfsetbuttcap%
\pgfsetroundjoin%
\definecolor{currentfill}{rgb}{0.160665,0.478540,0.558115}%
\pgfsetfillcolor{currentfill}%
\pgfsetfillopacity{0.700000}%
\pgfsetlinewidth{0.000000pt}%
\definecolor{currentstroke}{rgb}{0.000000,0.000000,0.000000}%
\pgfsetstrokecolor{currentstroke}%
\pgfsetdash{}{0pt}%
\pgfpathmoveto{\pgfqpoint{3.293883in}{3.571220in}}%
\pgfpathlineto{\pgfqpoint{3.307159in}{3.557592in}}%
\pgfpathlineto{\pgfqpoint{3.320434in}{3.544093in}}%
\pgfpathlineto{\pgfqpoint{3.333709in}{3.530724in}}%
\pgfpathlineto{\pgfqpoint{3.346982in}{3.517483in}}%
\pgfpathlineto{\pgfqpoint{3.354723in}{3.539526in}}%
\pgfpathlineto{\pgfqpoint{3.362459in}{3.561924in}}%
\pgfpathlineto{\pgfqpoint{3.370188in}{3.584686in}}%
\pgfpathlineto{\pgfqpoint{3.377912in}{3.607817in}}%
\pgfpathlineto{\pgfqpoint{3.364636in}{3.621445in}}%
\pgfpathlineto{\pgfqpoint{3.351359in}{3.635201in}}%
\pgfpathlineto{\pgfqpoint{3.338081in}{3.649087in}}%
\pgfpathlineto{\pgfqpoint{3.324803in}{3.663103in}}%
\pgfpathlineto{\pgfqpoint{3.317082in}{3.639576in}}%
\pgfpathlineto{\pgfqpoint{3.309356in}{3.616425in}}%
\pgfpathlineto{\pgfqpoint{3.301623in}{3.593642in}}%
\pgfpathlineto{\pgfqpoint{3.293883in}{3.571220in}}%
\pgfpathclose%
\pgfusepath{fill}%
\end{pgfscope}%
\begin{pgfscope}%
\pgfpathrectangle{\pgfqpoint{1.150000in}{0.150000in}}{\pgfqpoint{5.700000in}{5.700000in}}%
\pgfusepath{clip}%
\pgfsetbuttcap%
\pgfsetroundjoin%
\definecolor{currentfill}{rgb}{0.127568,0.566949,0.550556}%
\pgfsetfillcolor{currentfill}%
\pgfsetfillopacity{0.700000}%
\pgfsetlinewidth{0.000000pt}%
\definecolor{currentstroke}{rgb}{0.000000,0.000000,0.000000}%
\pgfsetstrokecolor{currentstroke}%
\pgfsetdash{}{0pt}%
\pgfpathmoveto{\pgfqpoint{3.576455in}{3.802215in}}%
\pgfpathlineto{\pgfqpoint{3.589739in}{3.788407in}}%
\pgfpathlineto{\pgfqpoint{3.603023in}{3.774718in}}%
\pgfpathlineto{\pgfqpoint{3.616307in}{3.761147in}}%
\pgfpathlineto{\pgfqpoint{3.629590in}{3.747693in}}%
\pgfpathlineto{\pgfqpoint{3.637264in}{3.774951in}}%
\pgfpathlineto{\pgfqpoint{3.644934in}{3.802676in}}%
\pgfpathlineto{\pgfqpoint{3.652600in}{3.830877in}}%
\pgfpathlineto{\pgfqpoint{3.660263in}{3.859564in}}%
\pgfpathlineto{\pgfqpoint{3.646973in}{3.873492in}}%
\pgfpathlineto{\pgfqpoint{3.633683in}{3.887537in}}%
\pgfpathlineto{\pgfqpoint{3.620393in}{3.901701in}}%
\pgfpathlineto{\pgfqpoint{3.607102in}{3.915984in}}%
\pgfpathlineto{\pgfqpoint{3.599446in}{3.886814in}}%
\pgfpathlineto{\pgfqpoint{3.591786in}{3.858136in}}%
\pgfpathlineto{\pgfqpoint{3.584122in}{3.829939in}}%
\pgfpathlineto{\pgfqpoint{3.576455in}{3.802215in}}%
\pgfpathclose%
\pgfusepath{fill}%
\end{pgfscope}%
\begin{pgfscope}%
\pgfpathrectangle{\pgfqpoint{1.150000in}{0.150000in}}{\pgfqpoint{5.700000in}{5.700000in}}%
\pgfusepath{clip}%
\pgfsetbuttcap%
\pgfsetroundjoin%
\definecolor{currentfill}{rgb}{0.156270,0.489624,0.557936}%
\pgfsetfillcolor{currentfill}%
\pgfsetfillopacity{0.700000}%
\pgfsetlinewidth{0.000000pt}%
\definecolor{currentstroke}{rgb}{0.000000,0.000000,0.000000}%
\pgfsetstrokecolor{currentstroke}%
\pgfsetdash{}{0pt}%
\pgfpathmoveto{\pgfqpoint{3.514960in}{3.596399in}}%
\pgfpathlineto{\pgfqpoint{3.528236in}{3.583471in}}%
\pgfpathlineto{\pgfqpoint{3.541511in}{3.570661in}}%
\pgfpathlineto{\pgfqpoint{3.554786in}{3.557968in}}%
\pgfpathlineto{\pgfqpoint{3.568062in}{3.545392in}}%
\pgfpathlineto{\pgfqpoint{3.575768in}{3.569225in}}%
\pgfpathlineto{\pgfqpoint{3.583469in}{3.593457in}}%
\pgfpathlineto{\pgfqpoint{3.591166in}{3.618097in}}%
\pgfpathlineto{\pgfqpoint{3.598859in}{3.643152in}}%
\pgfpathlineto{\pgfqpoint{3.585580in}{3.656156in}}%
\pgfpathlineto{\pgfqpoint{3.572301in}{3.669276in}}%
\pgfpathlineto{\pgfqpoint{3.559023in}{3.682514in}}%
\pgfpathlineto{\pgfqpoint{3.545743in}{3.695870in}}%
\pgfpathlineto{\pgfqpoint{3.538055in}{3.670379in}}%
\pgfpathlineto{\pgfqpoint{3.530361in}{3.645308in}}%
\pgfpathlineto{\pgfqpoint{3.522663in}{3.620651in}}%
\pgfpathlineto{\pgfqpoint{3.514960in}{3.596399in}}%
\pgfpathclose%
\pgfusepath{fill}%
\end{pgfscope}%
\begin{pgfscope}%
\pgfpathrectangle{\pgfqpoint{1.150000in}{0.150000in}}{\pgfqpoint{5.700000in}{5.700000in}}%
\pgfusepath{clip}%
\pgfsetbuttcap%
\pgfsetroundjoin%
\definecolor{currentfill}{rgb}{0.163625,0.471133,0.558148}%
\pgfsetfillcolor{currentfill}%
\pgfsetfillopacity{0.700000}%
\pgfsetlinewidth{0.000000pt}%
\definecolor{currentstroke}{rgb}{0.000000,0.000000,0.000000}%
\pgfsetstrokecolor{currentstroke}%
\pgfsetdash{}{0pt}%
\pgfpathmoveto{\pgfqpoint{3.431008in}{3.554566in}}%
\pgfpathlineto{\pgfqpoint{3.444281in}{3.541563in}}%
\pgfpathlineto{\pgfqpoint{3.457554in}{3.528682in}}%
\pgfpathlineto{\pgfqpoint{3.470827in}{3.515921in}}%
\pgfpathlineto{\pgfqpoint{3.484099in}{3.503280in}}%
\pgfpathlineto{\pgfqpoint{3.491822in}{3.525991in}}%
\pgfpathlineto{\pgfqpoint{3.499540in}{3.549077in}}%
\pgfpathlineto{\pgfqpoint{3.507253in}{3.572543in}}%
\pgfpathlineto{\pgfqpoint{3.514960in}{3.596399in}}%
\pgfpathlineto{\pgfqpoint{3.501685in}{3.609446in}}%
\pgfpathlineto{\pgfqpoint{3.488410in}{3.622613in}}%
\pgfpathlineto{\pgfqpoint{3.475134in}{3.635901in}}%
\pgfpathlineto{\pgfqpoint{3.461858in}{3.649311in}}%
\pgfpathlineto{\pgfqpoint{3.454154in}{3.625041in}}%
\pgfpathlineto{\pgfqpoint{3.446444in}{3.601165in}}%
\pgfpathlineto{\pgfqpoint{3.438729in}{3.577676in}}%
\pgfpathlineto{\pgfqpoint{3.431008in}{3.554566in}}%
\pgfpathclose%
\pgfusepath{fill}%
\end{pgfscope}%
\begin{pgfscope}%
\pgfpathrectangle{\pgfqpoint{1.150000in}{0.150000in}}{\pgfqpoint{5.700000in}{5.700000in}}%
\pgfusepath{clip}%
\pgfsetbuttcap%
\pgfsetroundjoin%
\definecolor{currentfill}{rgb}{0.157729,0.485932,0.558013}%
\pgfsetfillcolor{currentfill}%
\pgfsetfillopacity{0.700000}%
\pgfsetlinewidth{0.000000pt}%
\definecolor{currentstroke}{rgb}{0.000000,0.000000,0.000000}%
\pgfsetstrokecolor{currentstroke}%
\pgfsetdash{}{0pt}%
\pgfpathmoveto{\pgfqpoint{3.156620in}{3.595928in}}%
\pgfpathlineto{\pgfqpoint{3.169905in}{3.581583in}}%
\pgfpathlineto{\pgfqpoint{3.183189in}{3.567377in}}%
\pgfpathlineto{\pgfqpoint{3.196471in}{3.553310in}}%
\pgfpathlineto{\pgfqpoint{3.209752in}{3.539380in}}%
\pgfpathlineto{\pgfqpoint{3.217516in}{3.560784in}}%
\pgfpathlineto{\pgfqpoint{3.225273in}{3.582528in}}%
\pgfpathlineto{\pgfqpoint{3.233024in}{3.604617in}}%
\pgfpathlineto{\pgfqpoint{3.240767in}{3.627059in}}%
\pgfpathlineto{\pgfqpoint{3.227484in}{3.641356in}}%
\pgfpathlineto{\pgfqpoint{3.214200in}{3.655790in}}%
\pgfpathlineto{\pgfqpoint{3.200914in}{3.670363in}}%
\pgfpathlineto{\pgfqpoint{3.187626in}{3.685077in}}%
\pgfpathlineto{\pgfqpoint{3.179885in}{3.662259in}}%
\pgfpathlineto{\pgfqpoint{3.172137in}{3.639800in}}%
\pgfpathlineto{\pgfqpoint{3.164382in}{3.617692in}}%
\pgfpathlineto{\pgfqpoint{3.156620in}{3.595928in}}%
\pgfpathclose%
\pgfusepath{fill}%
\end{pgfscope}%
\begin{pgfscope}%
\pgfpathrectangle{\pgfqpoint{1.150000in}{0.150000in}}{\pgfqpoint{5.700000in}{5.700000in}}%
\pgfusepath{clip}%
\pgfsetbuttcap%
\pgfsetroundjoin%
\definecolor{currentfill}{rgb}{0.149039,0.508051,0.557250}%
\pgfsetfillcolor{currentfill}%
\pgfsetfillopacity{0.700000}%
\pgfsetlinewidth{0.000000pt}%
\definecolor{currentstroke}{rgb}{0.000000,0.000000,0.000000}%
\pgfsetstrokecolor{currentstroke}%
\pgfsetdash{}{0pt}%
\pgfpathmoveto{\pgfqpoint{3.598859in}{3.643152in}}%
\pgfpathlineto{\pgfqpoint{3.612138in}{3.630264in}}%
\pgfpathlineto{\pgfqpoint{3.625418in}{3.617491in}}%
\pgfpathlineto{\pgfqpoint{3.638697in}{3.604831in}}%
\pgfpathlineto{\pgfqpoint{3.651977in}{3.592285in}}%
\pgfpathlineto{\pgfqpoint{3.659670in}{3.617323in}}%
\pgfpathlineto{\pgfqpoint{3.667359in}{3.642787in}}%
\pgfpathlineto{\pgfqpoint{3.675044in}{3.668686in}}%
\pgfpathlineto{\pgfqpoint{3.682726in}{3.695028in}}%
\pgfpathlineto{\pgfqpoint{3.669442in}{3.708023in}}%
\pgfpathlineto{\pgfqpoint{3.656158in}{3.721132in}}%
\pgfpathlineto{\pgfqpoint{3.642874in}{3.734355in}}%
\pgfpathlineto{\pgfqpoint{3.629590in}{3.747693in}}%
\pgfpathlineto{\pgfqpoint{3.621913in}{3.720892in}}%
\pgfpathlineto{\pgfqpoint{3.614233in}{3.694541in}}%
\pgfpathlineto{\pgfqpoint{3.606548in}{3.668631in}}%
\pgfpathlineto{\pgfqpoint{3.598859in}{3.643152in}}%
\pgfpathclose%
\pgfusepath{fill}%
\end{pgfscope}%
\begin{pgfscope}%
\pgfpathrectangle{\pgfqpoint{1.150000in}{0.150000in}}{\pgfqpoint{5.700000in}{5.700000in}}%
\pgfusepath{clip}%
\pgfsetbuttcap%
\pgfsetroundjoin%
\definecolor{currentfill}{rgb}{0.132268,0.655014,0.519661}%
\pgfsetfillcolor{currentfill}%
\pgfsetfillopacity{0.700000}%
\pgfsetlinewidth{0.000000pt}%
\definecolor{currentstroke}{rgb}{0.000000,0.000000,0.000000}%
\pgfsetstrokecolor{currentstroke}%
\pgfsetdash{}{0pt}%
\pgfpathmoveto{\pgfqpoint{3.500746in}{4.034673in}}%
\pgfpathlineto{\pgfqpoint{3.514044in}{4.019397in}}%
\pgfpathlineto{\pgfqpoint{3.527341in}{4.004249in}}%
\pgfpathlineto{\pgfqpoint{3.540637in}{3.989228in}}%
\pgfpathlineto{\pgfqpoint{3.553931in}{3.974332in}}%
\pgfpathlineto{\pgfqpoint{3.561576in}{4.004495in}}%
\pgfpathlineto{\pgfqpoint{3.569216in}{4.035173in}}%
\pgfpathlineto{\pgfqpoint{3.576854in}{4.066378in}}%
\pgfpathlineto{\pgfqpoint{3.584487in}{4.098118in}}%
\pgfpathlineto{\pgfqpoint{3.571183in}{4.113517in}}%
\pgfpathlineto{\pgfqpoint{3.557878in}{4.129041in}}%
\pgfpathlineto{\pgfqpoint{3.544572in}{4.144693in}}%
\pgfpathlineto{\pgfqpoint{3.531264in}{4.160474in}}%
\pgfpathlineto{\pgfqpoint{3.523640in}{4.128221in}}%
\pgfpathlineto{\pgfqpoint{3.516013in}{4.096509in}}%
\pgfpathlineto{\pgfqpoint{3.508381in}{4.065330in}}%
\pgfpathlineto{\pgfqpoint{3.500746in}{4.034673in}}%
\pgfpathclose%
\pgfusepath{fill}%
\end{pgfscope}%
\begin{pgfscope}%
\pgfpathrectangle{\pgfqpoint{1.150000in}{0.150000in}}{\pgfqpoint{5.700000in}{5.700000in}}%
\pgfusepath{clip}%
\pgfsetbuttcap%
\pgfsetroundjoin%
\definecolor{currentfill}{rgb}{0.133743,0.548535,0.553541}%
\pgfsetfillcolor{currentfill}%
\pgfsetfillopacity{0.700000}%
\pgfsetlinewidth{0.000000pt}%
\definecolor{currentstroke}{rgb}{0.000000,0.000000,0.000000}%
\pgfsetstrokecolor{currentstroke}%
\pgfsetdash{}{0pt}%
\pgfpathmoveto{\pgfqpoint{3.629590in}{3.747693in}}%
\pgfpathlineto{\pgfqpoint{3.642874in}{3.734355in}}%
\pgfpathlineto{\pgfqpoint{3.656158in}{3.721132in}}%
\pgfpathlineto{\pgfqpoint{3.669442in}{3.708023in}}%
\pgfpathlineto{\pgfqpoint{3.682726in}{3.695028in}}%
\pgfpathlineto{\pgfqpoint{3.690405in}{3.721823in}}%
\pgfpathlineto{\pgfqpoint{3.698080in}{3.749078in}}%
\pgfpathlineto{\pgfqpoint{3.705752in}{3.776804in}}%
\pgfpathlineto{\pgfqpoint{3.713422in}{3.805008in}}%
\pgfpathlineto{\pgfqpoint{3.700132in}{3.818475in}}%
\pgfpathlineto{\pgfqpoint{3.686842in}{3.832057in}}%
\pgfpathlineto{\pgfqpoint{3.673553in}{3.845752in}}%
\pgfpathlineto{\pgfqpoint{3.660263in}{3.859564in}}%
\pgfpathlineto{\pgfqpoint{3.652600in}{3.830877in}}%
\pgfpathlineto{\pgfqpoint{3.644934in}{3.802676in}}%
\pgfpathlineto{\pgfqpoint{3.637264in}{3.774951in}}%
\pgfpathlineto{\pgfqpoint{3.629590in}{3.747693in}}%
\pgfpathclose%
\pgfusepath{fill}%
\end{pgfscope}%
\begin{pgfscope}%
\pgfpathrectangle{\pgfqpoint{1.150000in}{0.150000in}}{\pgfqpoint{5.700000in}{5.700000in}}%
\pgfusepath{clip}%
\pgfsetbuttcap%
\pgfsetroundjoin%
\definecolor{currentfill}{rgb}{0.153894,0.680203,0.504172}%
\pgfsetfillcolor{currentfill}%
\pgfsetfillopacity{0.700000}%
\pgfsetlinewidth{0.000000pt}%
\definecolor{currentstroke}{rgb}{0.000000,0.000000,0.000000}%
\pgfsetstrokecolor{currentstroke}%
\pgfsetdash{}{0pt}%
\pgfpathmoveto{\pgfqpoint{3.447538in}{4.097079in}}%
\pgfpathlineto{\pgfqpoint{3.460843in}{4.081280in}}%
\pgfpathlineto{\pgfqpoint{3.474145in}{4.065614in}}%
\pgfpathlineto{\pgfqpoint{3.487446in}{4.050078in}}%
\pgfpathlineto{\pgfqpoint{3.500746in}{4.034673in}}%
\pgfpathlineto{\pgfqpoint{3.508381in}{4.065330in}}%
\pgfpathlineto{\pgfqpoint{3.516013in}{4.096509in}}%
\pgfpathlineto{\pgfqpoint{3.523640in}{4.128221in}}%
\pgfpathlineto{\pgfqpoint{3.531264in}{4.160474in}}%
\pgfpathlineto{\pgfqpoint{3.517955in}{4.176384in}}%
\pgfpathlineto{\pgfqpoint{3.504644in}{4.192425in}}%
\pgfpathlineto{\pgfqpoint{3.491331in}{4.208598in}}%
\pgfpathlineto{\pgfqpoint{3.478016in}{4.224904in}}%
\pgfpathlineto{\pgfqpoint{3.470403in}{4.192135in}}%
\pgfpathlineto{\pgfqpoint{3.462786in}{4.159914in}}%
\pgfpathlineto{\pgfqpoint{3.455165in}{4.128232in}}%
\pgfpathlineto{\pgfqpoint{3.447538in}{4.097079in}}%
\pgfpathclose%
\pgfusepath{fill}%
\end{pgfscope}%
\begin{pgfscope}%
\pgfpathrectangle{\pgfqpoint{1.150000in}{0.150000in}}{\pgfqpoint{5.700000in}{5.700000in}}%
\pgfusepath{clip}%
\pgfsetbuttcap%
\pgfsetroundjoin%
\definecolor{currentfill}{rgb}{0.122312,0.633153,0.530398}%
\pgfsetfillcolor{currentfill}%
\pgfsetfillopacity{0.700000}%
\pgfsetlinewidth{0.000000pt}%
\definecolor{currentstroke}{rgb}{0.000000,0.000000,0.000000}%
\pgfsetstrokecolor{currentstroke}%
\pgfsetdash{}{0pt}%
\pgfpathmoveto{\pgfqpoint{3.553931in}{3.974332in}}%
\pgfpathlineto{\pgfqpoint{3.567225in}{3.959561in}}%
\pgfpathlineto{\pgfqpoint{3.580518in}{3.944913in}}%
\pgfpathlineto{\pgfqpoint{3.593810in}{3.930388in}}%
\pgfpathlineto{\pgfqpoint{3.607102in}{3.915984in}}%
\pgfpathlineto{\pgfqpoint{3.614754in}{3.945655in}}%
\pgfpathlineto{\pgfqpoint{3.622403in}{3.975836in}}%
\pgfpathlineto{\pgfqpoint{3.630049in}{4.006536in}}%
\pgfpathlineto{\pgfqpoint{3.637692in}{4.037765in}}%
\pgfpathlineto{\pgfqpoint{3.624392in}{4.052669in}}%
\pgfpathlineto{\pgfqpoint{3.611092in}{4.067696in}}%
\pgfpathlineto{\pgfqpoint{3.597790in}{4.082845in}}%
\pgfpathlineto{\pgfqpoint{3.584487in}{4.098118in}}%
\pgfpathlineto{\pgfqpoint{3.576854in}{4.066378in}}%
\pgfpathlineto{\pgfqpoint{3.569216in}{4.035173in}}%
\pgfpathlineto{\pgfqpoint{3.561576in}{4.004495in}}%
\pgfpathlineto{\pgfqpoint{3.553931in}{3.974332in}}%
\pgfpathclose%
\pgfusepath{fill}%
\end{pgfscope}%
\begin{pgfscope}%
\pgfpathrectangle{\pgfqpoint{1.150000in}{0.150000in}}{\pgfqpoint{5.700000in}{5.700000in}}%
\pgfusepath{clip}%
\pgfsetbuttcap%
\pgfsetroundjoin%
\definecolor{currentfill}{rgb}{0.168126,0.459988,0.558082}%
\pgfsetfillcolor{currentfill}%
\pgfsetfillopacity{0.700000}%
\pgfsetlinewidth{0.000000pt}%
\definecolor{currentstroke}{rgb}{0.000000,0.000000,0.000000}%
\pgfsetstrokecolor{currentstroke}%
\pgfsetdash{}{0pt}%
\pgfpathmoveto{\pgfqpoint{3.346982in}{3.517483in}}%
\pgfpathlineto{\pgfqpoint{3.360255in}{3.504368in}}%
\pgfpathlineto{\pgfqpoint{3.373527in}{3.491379in}}%
\pgfpathlineto{\pgfqpoint{3.386798in}{3.478514in}}%
\pgfpathlineto{\pgfqpoint{3.400070in}{3.465773in}}%
\pgfpathlineto{\pgfqpoint{3.407813in}{3.487439in}}%
\pgfpathlineto{\pgfqpoint{3.415550in}{3.509455in}}%
\pgfpathlineto{\pgfqpoint{3.423282in}{3.531829in}}%
\pgfpathlineto{\pgfqpoint{3.431008in}{3.554566in}}%
\pgfpathlineto{\pgfqpoint{3.417735in}{3.567692in}}%
\pgfpathlineto{\pgfqpoint{3.404461in}{3.580942in}}%
\pgfpathlineto{\pgfqpoint{3.391187in}{3.594317in}}%
\pgfpathlineto{\pgfqpoint{3.377912in}{3.607817in}}%
\pgfpathlineto{\pgfqpoint{3.370188in}{3.584686in}}%
\pgfpathlineto{\pgfqpoint{3.362459in}{3.561924in}}%
\pgfpathlineto{\pgfqpoint{3.354723in}{3.539526in}}%
\pgfpathlineto{\pgfqpoint{3.346982in}{3.517483in}}%
\pgfpathclose%
\pgfusepath{fill}%
\end{pgfscope}%
\begin{pgfscope}%
\pgfpathrectangle{\pgfqpoint{1.150000in}{0.150000in}}{\pgfqpoint{5.700000in}{5.700000in}}%
\pgfusepath{clip}%
\pgfsetbuttcap%
\pgfsetroundjoin%
\definecolor{currentfill}{rgb}{0.180653,0.701402,0.488189}%
\pgfsetfillcolor{currentfill}%
\pgfsetfillopacity{0.700000}%
\pgfsetlinewidth{0.000000pt}%
\definecolor{currentstroke}{rgb}{0.000000,0.000000,0.000000}%
\pgfsetstrokecolor{currentstroke}%
\pgfsetdash{}{0pt}%
\pgfpathmoveto{\pgfqpoint{3.394302in}{4.161624in}}%
\pgfpathlineto{\pgfqpoint{3.407615in}{4.145283in}}%
\pgfpathlineto{\pgfqpoint{3.420924in}{4.129079in}}%
\pgfpathlineto{\pgfqpoint{3.434232in}{4.113012in}}%
\pgfpathlineto{\pgfqpoint{3.447538in}{4.097079in}}%
\pgfpathlineto{\pgfqpoint{3.455165in}{4.128232in}}%
\pgfpathlineto{\pgfqpoint{3.462786in}{4.159914in}}%
\pgfpathlineto{\pgfqpoint{3.470403in}{4.192135in}}%
\pgfpathlineto{\pgfqpoint{3.478016in}{4.224904in}}%
\pgfpathlineto{\pgfqpoint{3.464700in}{4.241345in}}%
\pgfpathlineto{\pgfqpoint{3.451381in}{4.257921in}}%
\pgfpathlineto{\pgfqpoint{3.438060in}{4.274634in}}%
\pgfpathlineto{\pgfqpoint{3.424737in}{4.291486in}}%
\pgfpathlineto{\pgfqpoint{3.417136in}{4.258198in}}%
\pgfpathlineto{\pgfqpoint{3.409530in}{4.225465in}}%
\pgfpathlineto{\pgfqpoint{3.401919in}{4.193277in}}%
\pgfpathlineto{\pgfqpoint{3.394302in}{4.161624in}}%
\pgfpathclose%
\pgfusepath{fill}%
\end{pgfscope}%
\begin{pgfscope}%
\pgfpathrectangle{\pgfqpoint{1.150000in}{0.150000in}}{\pgfqpoint{5.700000in}{5.700000in}}%
\pgfusepath{clip}%
\pgfsetbuttcap%
\pgfsetroundjoin%
\definecolor{currentfill}{rgb}{0.119423,0.611141,0.538982}%
\pgfsetfillcolor{currentfill}%
\pgfsetfillopacity{0.700000}%
\pgfsetlinewidth{0.000000pt}%
\definecolor{currentstroke}{rgb}{0.000000,0.000000,0.000000}%
\pgfsetstrokecolor{currentstroke}%
\pgfsetdash{}{0pt}%
\pgfpathmoveto{\pgfqpoint{3.607102in}{3.915984in}}%
\pgfpathlineto{\pgfqpoint{3.620393in}{3.901701in}}%
\pgfpathlineto{\pgfqpoint{3.633683in}{3.887537in}}%
\pgfpathlineto{\pgfqpoint{3.646973in}{3.873492in}}%
\pgfpathlineto{\pgfqpoint{3.660263in}{3.859564in}}%
\pgfpathlineto{\pgfqpoint{3.667923in}{3.888745in}}%
\pgfpathlineto{\pgfqpoint{3.675580in}{3.918429in}}%
\pgfpathlineto{\pgfqpoint{3.683234in}{3.948627in}}%
\pgfpathlineto{\pgfqpoint{3.690886in}{3.979348in}}%
\pgfpathlineto{\pgfqpoint{3.677588in}{3.993775in}}%
\pgfpathlineto{\pgfqpoint{3.664290in}{4.008319in}}%
\pgfpathlineto{\pgfqpoint{3.650992in}{4.022982in}}%
\pgfpathlineto{\pgfqpoint{3.637692in}{4.037765in}}%
\pgfpathlineto{\pgfqpoint{3.630049in}{4.006536in}}%
\pgfpathlineto{\pgfqpoint{3.622403in}{3.975836in}}%
\pgfpathlineto{\pgfqpoint{3.614754in}{3.945655in}}%
\pgfpathlineto{\pgfqpoint{3.607102in}{3.915984in}}%
\pgfpathclose%
\pgfusepath{fill}%
\end{pgfscope}%
\begin{pgfscope}%
\pgfpathrectangle{\pgfqpoint{1.150000in}{0.150000in}}{\pgfqpoint{5.700000in}{5.700000in}}%
\pgfusepath{clip}%
\pgfsetbuttcap%
\pgfsetroundjoin%
\definecolor{currentfill}{rgb}{0.165117,0.467423,0.558141}%
\pgfsetfillcolor{currentfill}%
\pgfsetfillopacity{0.700000}%
\pgfsetlinewidth{0.000000pt}%
\definecolor{currentstroke}{rgb}{0.000000,0.000000,0.000000}%
\pgfsetstrokecolor{currentstroke}%
\pgfsetdash{}{0pt}%
\pgfpathmoveto{\pgfqpoint{3.209752in}{3.539380in}}%
\pgfpathlineto{\pgfqpoint{3.223031in}{3.525585in}}%
\pgfpathlineto{\pgfqpoint{3.236309in}{3.511925in}}%
\pgfpathlineto{\pgfqpoint{3.249585in}{3.498399in}}%
\pgfpathlineto{\pgfqpoint{3.262861in}{3.485004in}}%
\pgfpathlineto{\pgfqpoint{3.270626in}{3.506051in}}%
\pgfpathlineto{\pgfqpoint{3.278385in}{3.527432in}}%
\pgfpathlineto{\pgfqpoint{3.286138in}{3.549153in}}%
\pgfpathlineto{\pgfqpoint{3.293883in}{3.571220in}}%
\pgfpathlineto{\pgfqpoint{3.280606in}{3.584980in}}%
\pgfpathlineto{\pgfqpoint{3.267328in}{3.598872in}}%
\pgfpathlineto{\pgfqpoint{3.254048in}{3.612898in}}%
\pgfpathlineto{\pgfqpoint{3.240767in}{3.627059in}}%
\pgfpathlineto{\pgfqpoint{3.233024in}{3.604617in}}%
\pgfpathlineto{\pgfqpoint{3.225273in}{3.582528in}}%
\pgfpathlineto{\pgfqpoint{3.217516in}{3.560784in}}%
\pgfpathlineto{\pgfqpoint{3.209752in}{3.539380in}}%
\pgfpathclose%
\pgfusepath{fill}%
\end{pgfscope}%
\begin{pgfscope}%
\pgfpathrectangle{\pgfqpoint{1.150000in}{0.150000in}}{\pgfqpoint{5.700000in}{5.700000in}}%
\pgfusepath{clip}%
\pgfsetbuttcap%
\pgfsetroundjoin%
\definecolor{currentfill}{rgb}{0.163625,0.471133,0.558148}%
\pgfsetfillcolor{currentfill}%
\pgfsetfillopacity{0.700000}%
\pgfsetlinewidth{0.000000pt}%
\definecolor{currentstroke}{rgb}{0.000000,0.000000,0.000000}%
\pgfsetstrokecolor{currentstroke}%
\pgfsetdash{}{0pt}%
\pgfpathmoveto{\pgfqpoint{3.568062in}{3.545392in}}%
\pgfpathlineto{\pgfqpoint{3.581337in}{3.532931in}}%
\pgfpathlineto{\pgfqpoint{3.594614in}{3.520584in}}%
\pgfpathlineto{\pgfqpoint{3.607890in}{3.508351in}}%
\pgfpathlineto{\pgfqpoint{3.621167in}{3.496231in}}%
\pgfpathlineto{\pgfqpoint{3.628876in}{3.519646in}}%
\pgfpathlineto{\pgfqpoint{3.636580in}{3.543455in}}%
\pgfpathlineto{\pgfqpoint{3.644281in}{3.567665in}}%
\pgfpathlineto{\pgfqpoint{3.651977in}{3.592285in}}%
\pgfpathlineto{\pgfqpoint{3.638697in}{3.604831in}}%
\pgfpathlineto{\pgfqpoint{3.625418in}{3.617491in}}%
\pgfpathlineto{\pgfqpoint{3.612138in}{3.630264in}}%
\pgfpathlineto{\pgfqpoint{3.598859in}{3.643152in}}%
\pgfpathlineto{\pgfqpoint{3.591166in}{3.618097in}}%
\pgfpathlineto{\pgfqpoint{3.583469in}{3.593457in}}%
\pgfpathlineto{\pgfqpoint{3.575768in}{3.569225in}}%
\pgfpathlineto{\pgfqpoint{3.568062in}{3.545392in}}%
\pgfpathclose%
\pgfusepath{fill}%
\end{pgfscope}%
\begin{pgfscope}%
\pgfpathrectangle{\pgfqpoint{1.150000in}{0.150000in}}{\pgfqpoint{5.700000in}{5.700000in}}%
\pgfusepath{clip}%
\pgfsetbuttcap%
\pgfsetroundjoin%
\definecolor{currentfill}{rgb}{0.169646,0.456262,0.558030}%
\pgfsetfillcolor{currentfill}%
\pgfsetfillopacity{0.700000}%
\pgfsetlinewidth{0.000000pt}%
\definecolor{currentstroke}{rgb}{0.000000,0.000000,0.000000}%
\pgfsetstrokecolor{currentstroke}%
\pgfsetdash{}{0pt}%
\pgfpathmoveto{\pgfqpoint{3.484099in}{3.503280in}}%
\pgfpathlineto{\pgfqpoint{3.497372in}{3.490758in}}%
\pgfpathlineto{\pgfqpoint{3.510645in}{3.478353in}}%
\pgfpathlineto{\pgfqpoint{3.523918in}{3.466066in}}%
\pgfpathlineto{\pgfqpoint{3.537191in}{3.453894in}}%
\pgfpathlineto{\pgfqpoint{3.544916in}{3.476209in}}%
\pgfpathlineto{\pgfqpoint{3.552636in}{3.498892in}}%
\pgfpathlineto{\pgfqpoint{3.560351in}{3.521950in}}%
\pgfpathlineto{\pgfqpoint{3.568062in}{3.545392in}}%
\pgfpathlineto{\pgfqpoint{3.554786in}{3.557968in}}%
\pgfpathlineto{\pgfqpoint{3.541511in}{3.570661in}}%
\pgfpathlineto{\pgfqpoint{3.528236in}{3.583471in}}%
\pgfpathlineto{\pgfqpoint{3.514960in}{3.596399in}}%
\pgfpathlineto{\pgfqpoint{3.507253in}{3.572543in}}%
\pgfpathlineto{\pgfqpoint{3.499540in}{3.549077in}}%
\pgfpathlineto{\pgfqpoint{3.491822in}{3.525991in}}%
\pgfpathlineto{\pgfqpoint{3.484099in}{3.503280in}}%
\pgfpathclose%
\pgfusepath{fill}%
\end{pgfscope}%
\begin{pgfscope}%
\pgfpathrectangle{\pgfqpoint{1.150000in}{0.150000in}}{\pgfqpoint{5.700000in}{5.700000in}}%
\pgfusepath{clip}%
\pgfsetbuttcap%
\pgfsetroundjoin%
\definecolor{currentfill}{rgb}{0.121148,0.592739,0.544641}%
\pgfsetfillcolor{currentfill}%
\pgfsetfillopacity{0.700000}%
\pgfsetlinewidth{0.000000pt}%
\definecolor{currentstroke}{rgb}{0.000000,0.000000,0.000000}%
\pgfsetstrokecolor{currentstroke}%
\pgfsetdash{}{0pt}%
\pgfpathmoveto{\pgfqpoint{3.660263in}{3.859564in}}%
\pgfpathlineto{\pgfqpoint{3.673553in}{3.845752in}}%
\pgfpathlineto{\pgfqpoint{3.686842in}{3.832057in}}%
\pgfpathlineto{\pgfqpoint{3.700132in}{3.818475in}}%
\pgfpathlineto{\pgfqpoint{3.713422in}{3.805008in}}%
\pgfpathlineto{\pgfqpoint{3.721089in}{3.833701in}}%
\pgfpathlineto{\pgfqpoint{3.728753in}{3.862892in}}%
\pgfpathlineto{\pgfqpoint{3.736414in}{3.892590in}}%
\pgfpathlineto{\pgfqpoint{3.744074in}{3.922805in}}%
\pgfpathlineto{\pgfqpoint{3.730777in}{3.936769in}}%
\pgfpathlineto{\pgfqpoint{3.717480in}{3.950846in}}%
\pgfpathlineto{\pgfqpoint{3.704183in}{3.965039in}}%
\pgfpathlineto{\pgfqpoint{3.690886in}{3.979348in}}%
\pgfpathlineto{\pgfqpoint{3.683234in}{3.948627in}}%
\pgfpathlineto{\pgfqpoint{3.675580in}{3.918429in}}%
\pgfpathlineto{\pgfqpoint{3.667923in}{3.888745in}}%
\pgfpathlineto{\pgfqpoint{3.660263in}{3.859564in}}%
\pgfpathclose%
\pgfusepath{fill}%
\end{pgfscope}%
\begin{pgfscope}%
\pgfpathrectangle{\pgfqpoint{1.150000in}{0.150000in}}{\pgfqpoint{5.700000in}{5.700000in}}%
\pgfusepath{clip}%
\pgfsetbuttcap%
\pgfsetroundjoin%
\definecolor{currentfill}{rgb}{0.220124,0.725509,0.466226}%
\pgfsetfillcolor{currentfill}%
\pgfsetfillopacity{0.700000}%
\pgfsetlinewidth{0.000000pt}%
\definecolor{currentstroke}{rgb}{0.000000,0.000000,0.000000}%
\pgfsetstrokecolor{currentstroke}%
\pgfsetdash{}{0pt}%
\pgfpathmoveto{\pgfqpoint{3.341030in}{4.228391in}}%
\pgfpathlineto{\pgfqpoint{3.354352in}{4.211486in}}%
\pgfpathlineto{\pgfqpoint{3.367671in}{4.194725in}}%
\pgfpathlineto{\pgfqpoint{3.380988in}{4.178105in}}%
\pgfpathlineto{\pgfqpoint{3.394302in}{4.161624in}}%
\pgfpathlineto{\pgfqpoint{3.401919in}{4.193277in}}%
\pgfpathlineto{\pgfqpoint{3.409530in}{4.225465in}}%
\pgfpathlineto{\pgfqpoint{3.417136in}{4.258198in}}%
\pgfpathlineto{\pgfqpoint{3.424737in}{4.291486in}}%
\pgfpathlineto{\pgfqpoint{3.411412in}{4.308476in}}%
\pgfpathlineto{\pgfqpoint{3.398083in}{4.325608in}}%
\pgfpathlineto{\pgfqpoint{3.384753in}{4.342882in}}%
\pgfpathlineto{\pgfqpoint{3.371419in}{4.360299in}}%
\pgfpathlineto{\pgfqpoint{3.363830in}{4.326489in}}%
\pgfpathlineto{\pgfqpoint{3.356236in}{4.293242in}}%
\pgfpathlineto{\pgfqpoint{3.348636in}{4.260545in}}%
\pgfpathlineto{\pgfqpoint{3.341030in}{4.228391in}}%
\pgfpathclose%
\pgfusepath{fill}%
\end{pgfscope}%
\begin{pgfscope}%
\pgfpathrectangle{\pgfqpoint{1.150000in}{0.150000in}}{\pgfqpoint{5.700000in}{5.700000in}}%
\pgfusepath{clip}%
\pgfsetbuttcap%
\pgfsetroundjoin%
\definecolor{currentfill}{rgb}{0.140536,0.530132,0.555659}%
\pgfsetfillcolor{currentfill}%
\pgfsetfillopacity{0.700000}%
\pgfsetlinewidth{0.000000pt}%
\definecolor{currentstroke}{rgb}{0.000000,0.000000,0.000000}%
\pgfsetstrokecolor{currentstroke}%
\pgfsetdash{}{0pt}%
\pgfpathmoveto{\pgfqpoint{3.682726in}{3.695028in}}%
\pgfpathlineto{\pgfqpoint{3.696011in}{3.682146in}}%
\pgfpathlineto{\pgfqpoint{3.709296in}{3.669374in}}%
\pgfpathlineto{\pgfqpoint{3.722581in}{3.656714in}}%
\pgfpathlineto{\pgfqpoint{3.735867in}{3.644164in}}%
\pgfpathlineto{\pgfqpoint{3.743551in}{3.670496in}}%
\pgfpathlineto{\pgfqpoint{3.751231in}{3.697283in}}%
\pgfpathlineto{\pgfqpoint{3.758909in}{3.724534in}}%
\pgfpathlineto{\pgfqpoint{3.766584in}{3.752259in}}%
\pgfpathlineto{\pgfqpoint{3.753293in}{3.765280in}}%
\pgfpathlineto{\pgfqpoint{3.740002in}{3.778411in}}%
\pgfpathlineto{\pgfqpoint{3.726712in}{3.791654in}}%
\pgfpathlineto{\pgfqpoint{3.713422in}{3.805008in}}%
\pgfpathlineto{\pgfqpoint{3.705752in}{3.776804in}}%
\pgfpathlineto{\pgfqpoint{3.698080in}{3.749078in}}%
\pgfpathlineto{\pgfqpoint{3.690405in}{3.721823in}}%
\pgfpathlineto{\pgfqpoint{3.682726in}{3.695028in}}%
\pgfpathclose%
\pgfusepath{fill}%
\end{pgfscope}%
\begin{pgfscope}%
\pgfpathrectangle{\pgfqpoint{1.150000in}{0.150000in}}{\pgfqpoint{5.700000in}{5.700000in}}%
\pgfusepath{clip}%
\pgfsetbuttcap%
\pgfsetroundjoin%
\definecolor{currentfill}{rgb}{0.156270,0.489624,0.557936}%
\pgfsetfillcolor{currentfill}%
\pgfsetfillopacity{0.700000}%
\pgfsetlinewidth{0.000000pt}%
\definecolor{currentstroke}{rgb}{0.000000,0.000000,0.000000}%
\pgfsetstrokecolor{currentstroke}%
\pgfsetdash{}{0pt}%
\pgfpathmoveto{\pgfqpoint{3.651977in}{3.592285in}}%
\pgfpathlineto{\pgfqpoint{3.665258in}{3.579851in}}%
\pgfpathlineto{\pgfqpoint{3.678539in}{3.567527in}}%
\pgfpathlineto{\pgfqpoint{3.691821in}{3.555315in}}%
\pgfpathlineto{\pgfqpoint{3.705104in}{3.543212in}}%
\pgfpathlineto{\pgfqpoint{3.712799in}{3.567810in}}%
\pgfpathlineto{\pgfqpoint{3.720492in}{3.592829in}}%
\pgfpathlineto{\pgfqpoint{3.728181in}{3.618278in}}%
\pgfpathlineto{\pgfqpoint{3.735867in}{3.644164in}}%
\pgfpathlineto{\pgfqpoint{3.722581in}{3.656714in}}%
\pgfpathlineto{\pgfqpoint{3.709296in}{3.669374in}}%
\pgfpathlineto{\pgfqpoint{3.696011in}{3.682146in}}%
\pgfpathlineto{\pgfqpoint{3.682726in}{3.695028in}}%
\pgfpathlineto{\pgfqpoint{3.675044in}{3.668686in}}%
\pgfpathlineto{\pgfqpoint{3.667359in}{3.642787in}}%
\pgfpathlineto{\pgfqpoint{3.659670in}{3.617323in}}%
\pgfpathlineto{\pgfqpoint{3.651977in}{3.592285in}}%
\pgfpathclose%
\pgfusepath{fill}%
\end{pgfscope}%
\begin{pgfscope}%
\pgfpathrectangle{\pgfqpoint{1.150000in}{0.150000in}}{\pgfqpoint{5.700000in}{5.700000in}}%
\pgfusepath{clip}%
\pgfsetbuttcap%
\pgfsetroundjoin%
\definecolor{currentfill}{rgb}{0.175841,0.441290,0.557685}%
\pgfsetfillcolor{currentfill}%
\pgfsetfillopacity{0.700000}%
\pgfsetlinewidth{0.000000pt}%
\definecolor{currentstroke}{rgb}{0.000000,0.000000,0.000000}%
\pgfsetstrokecolor{currentstroke}%
\pgfsetdash{}{0pt}%
\pgfpathmoveto{\pgfqpoint{3.400070in}{3.465773in}}%
\pgfpathlineto{\pgfqpoint{3.413341in}{3.453155in}}%
\pgfpathlineto{\pgfqpoint{3.426612in}{3.440658in}}%
\pgfpathlineto{\pgfqpoint{3.439883in}{3.428281in}}%
\pgfpathlineto{\pgfqpoint{3.453154in}{3.416024in}}%
\pgfpathlineto{\pgfqpoint{3.460899in}{3.437314in}}%
\pgfpathlineto{\pgfqpoint{3.468637in}{3.458949in}}%
\pgfpathlineto{\pgfqpoint{3.476371in}{3.480935in}}%
\pgfpathlineto{\pgfqpoint{3.484099in}{3.503280in}}%
\pgfpathlineto{\pgfqpoint{3.470827in}{3.515921in}}%
\pgfpathlineto{\pgfqpoint{3.457554in}{3.528682in}}%
\pgfpathlineto{\pgfqpoint{3.444281in}{3.541563in}}%
\pgfpathlineto{\pgfqpoint{3.431008in}{3.554566in}}%
\pgfpathlineto{\pgfqpoint{3.423282in}{3.531829in}}%
\pgfpathlineto{\pgfqpoint{3.415550in}{3.509455in}}%
\pgfpathlineto{\pgfqpoint{3.407813in}{3.487439in}}%
\pgfpathlineto{\pgfqpoint{3.400070in}{3.465773in}}%
\pgfpathclose%
\pgfusepath{fill}%
\end{pgfscope}%
\begin{pgfscope}%
\pgfpathrectangle{\pgfqpoint{1.150000in}{0.150000in}}{\pgfqpoint{5.700000in}{5.700000in}}%
\pgfusepath{clip}%
\pgfsetbuttcap%
\pgfsetroundjoin%
\definecolor{currentfill}{rgb}{0.174274,0.445044,0.557792}%
\pgfsetfillcolor{currentfill}%
\pgfsetfillopacity{0.700000}%
\pgfsetlinewidth{0.000000pt}%
\definecolor{currentstroke}{rgb}{0.000000,0.000000,0.000000}%
\pgfsetstrokecolor{currentstroke}%
\pgfsetdash{}{0pt}%
\pgfpathmoveto{\pgfqpoint{3.262861in}{3.485004in}}%
\pgfpathlineto{\pgfqpoint{3.276135in}{3.471741in}}%
\pgfpathlineto{\pgfqpoint{3.289409in}{3.458607in}}%
\pgfpathlineto{\pgfqpoint{3.302682in}{3.445602in}}%
\pgfpathlineto{\pgfqpoint{3.315954in}{3.432725in}}%
\pgfpathlineto{\pgfqpoint{3.323720in}{3.453416in}}%
\pgfpathlineto{\pgfqpoint{3.331480in}{3.474435in}}%
\pgfpathlineto{\pgfqpoint{3.339234in}{3.495788in}}%
\pgfpathlineto{\pgfqpoint{3.346982in}{3.517483in}}%
\pgfpathlineto{\pgfqpoint{3.333709in}{3.530724in}}%
\pgfpathlineto{\pgfqpoint{3.320434in}{3.544093in}}%
\pgfpathlineto{\pgfqpoint{3.307159in}{3.557592in}}%
\pgfpathlineto{\pgfqpoint{3.293883in}{3.571220in}}%
\pgfpathlineto{\pgfqpoint{3.286138in}{3.549153in}}%
\pgfpathlineto{\pgfqpoint{3.278385in}{3.527432in}}%
\pgfpathlineto{\pgfqpoint{3.270626in}{3.506051in}}%
\pgfpathlineto{\pgfqpoint{3.262861in}{3.485004in}}%
\pgfpathclose%
\pgfusepath{fill}%
\end{pgfscope}%
\begin{pgfscope}%
\pgfpathrectangle{\pgfqpoint{1.150000in}{0.150000in}}{\pgfqpoint{5.700000in}{5.700000in}}%
\pgfusepath{clip}%
\pgfsetbuttcap%
\pgfsetroundjoin%
\definecolor{currentfill}{rgb}{0.125394,0.574318,0.549086}%
\pgfsetfillcolor{currentfill}%
\pgfsetfillopacity{0.700000}%
\pgfsetlinewidth{0.000000pt}%
\definecolor{currentstroke}{rgb}{0.000000,0.000000,0.000000}%
\pgfsetstrokecolor{currentstroke}%
\pgfsetdash{}{0pt}%
\pgfpathmoveto{\pgfqpoint{3.713422in}{3.805008in}}%
\pgfpathlineto{\pgfqpoint{3.726712in}{3.791654in}}%
\pgfpathlineto{\pgfqpoint{3.740002in}{3.778411in}}%
\pgfpathlineto{\pgfqpoint{3.753293in}{3.765280in}}%
\pgfpathlineto{\pgfqpoint{3.766584in}{3.752259in}}%
\pgfpathlineto{\pgfqpoint{3.774256in}{3.780466in}}%
\pgfpathlineto{\pgfqpoint{3.781927in}{3.809165in}}%
\pgfpathlineto{\pgfqpoint{3.789595in}{3.838365in}}%
\pgfpathlineto{\pgfqpoint{3.797262in}{3.868076in}}%
\pgfpathlineto{\pgfqpoint{3.783965in}{3.881591in}}%
\pgfpathlineto{\pgfqpoint{3.770668in}{3.895217in}}%
\pgfpathlineto{\pgfqpoint{3.757371in}{3.908955in}}%
\pgfpathlineto{\pgfqpoint{3.744074in}{3.922805in}}%
\pgfpathlineto{\pgfqpoint{3.736414in}{3.892590in}}%
\pgfpathlineto{\pgfqpoint{3.728753in}{3.862892in}}%
\pgfpathlineto{\pgfqpoint{3.721089in}{3.833701in}}%
\pgfpathlineto{\pgfqpoint{3.713422in}{3.805008in}}%
\pgfpathclose%
\pgfusepath{fill}%
\end{pgfscope}%
\begin{pgfscope}%
\pgfpathrectangle{\pgfqpoint{1.150000in}{0.150000in}}{\pgfqpoint{5.700000in}{5.700000in}}%
\pgfusepath{clip}%
\pgfsetbuttcap%
\pgfsetroundjoin%
\definecolor{currentfill}{rgb}{0.169646,0.456262,0.558030}%
\pgfsetfillcolor{currentfill}%
\pgfsetfillopacity{0.700000}%
\pgfsetlinewidth{0.000000pt}%
\definecolor{currentstroke}{rgb}{0.000000,0.000000,0.000000}%
\pgfsetstrokecolor{currentstroke}%
\pgfsetdash{}{0pt}%
\pgfpathmoveto{\pgfqpoint{3.621167in}{3.496231in}}%
\pgfpathlineto{\pgfqpoint{3.634445in}{3.484222in}}%
\pgfpathlineto{\pgfqpoint{3.647724in}{3.472324in}}%
\pgfpathlineto{\pgfqpoint{3.661004in}{3.460536in}}%
\pgfpathlineto{\pgfqpoint{3.674284in}{3.448858in}}%
\pgfpathlineto{\pgfqpoint{3.681995in}{3.471856in}}%
\pgfpathlineto{\pgfqpoint{3.689701in}{3.495243in}}%
\pgfpathlineto{\pgfqpoint{3.697404in}{3.519025in}}%
\pgfpathlineto{\pgfqpoint{3.705104in}{3.543212in}}%
\pgfpathlineto{\pgfqpoint{3.691821in}{3.555315in}}%
\pgfpathlineto{\pgfqpoint{3.678539in}{3.567527in}}%
\pgfpathlineto{\pgfqpoint{3.665258in}{3.579851in}}%
\pgfpathlineto{\pgfqpoint{3.651977in}{3.592285in}}%
\pgfpathlineto{\pgfqpoint{3.644281in}{3.567665in}}%
\pgfpathlineto{\pgfqpoint{3.636580in}{3.543455in}}%
\pgfpathlineto{\pgfqpoint{3.628876in}{3.519646in}}%
\pgfpathlineto{\pgfqpoint{3.621167in}{3.496231in}}%
\pgfpathclose%
\pgfusepath{fill}%
\end{pgfscope}%
\begin{pgfscope}%
\pgfpathrectangle{\pgfqpoint{1.150000in}{0.150000in}}{\pgfqpoint{5.700000in}{5.700000in}}%
\pgfusepath{clip}%
\pgfsetbuttcap%
\pgfsetroundjoin%
\definecolor{currentfill}{rgb}{0.177423,0.437527,0.557565}%
\pgfsetfillcolor{currentfill}%
\pgfsetfillopacity{0.700000}%
\pgfsetlinewidth{0.000000pt}%
\definecolor{currentstroke}{rgb}{0.000000,0.000000,0.000000}%
\pgfsetstrokecolor{currentstroke}%
\pgfsetdash{}{0pt}%
\pgfpathmoveto{\pgfqpoint{3.537191in}{3.453894in}}%
\pgfpathlineto{\pgfqpoint{3.550465in}{3.441838in}}%
\pgfpathlineto{\pgfqpoint{3.563739in}{3.429895in}}%
\pgfpathlineto{\pgfqpoint{3.577014in}{3.418066in}}%
\pgfpathlineto{\pgfqpoint{3.590290in}{3.406349in}}%
\pgfpathlineto{\pgfqpoint{3.598016in}{3.428268in}}%
\pgfpathlineto{\pgfqpoint{3.605737in}{3.450549in}}%
\pgfpathlineto{\pgfqpoint{3.613455in}{3.473201in}}%
\pgfpathlineto{\pgfqpoint{3.621167in}{3.496231in}}%
\pgfpathlineto{\pgfqpoint{3.607890in}{3.508351in}}%
\pgfpathlineto{\pgfqpoint{3.594614in}{3.520584in}}%
\pgfpathlineto{\pgfqpoint{3.581337in}{3.532931in}}%
\pgfpathlineto{\pgfqpoint{3.568062in}{3.545392in}}%
\pgfpathlineto{\pgfqpoint{3.560351in}{3.521950in}}%
\pgfpathlineto{\pgfqpoint{3.552636in}{3.498892in}}%
\pgfpathlineto{\pgfqpoint{3.544916in}{3.476209in}}%
\pgfpathlineto{\pgfqpoint{3.537191in}{3.453894in}}%
\pgfpathclose%
\pgfusepath{fill}%
\end{pgfscope}%
\begin{pgfscope}%
\pgfpathrectangle{\pgfqpoint{1.150000in}{0.150000in}}{\pgfqpoint{5.700000in}{5.700000in}}%
\pgfusepath{clip}%
\pgfsetbuttcap%
\pgfsetroundjoin%
\definecolor{currentfill}{rgb}{0.147607,0.511733,0.557049}%
\pgfsetfillcolor{currentfill}%
\pgfsetfillopacity{0.700000}%
\pgfsetlinewidth{0.000000pt}%
\definecolor{currentstroke}{rgb}{0.000000,0.000000,0.000000}%
\pgfsetstrokecolor{currentstroke}%
\pgfsetdash{}{0pt}%
\pgfpathmoveto{\pgfqpoint{3.735867in}{3.644164in}}%
\pgfpathlineto{\pgfqpoint{3.749154in}{3.631722in}}%
\pgfpathlineto{\pgfqpoint{3.762442in}{3.619389in}}%
\pgfpathlineto{\pgfqpoint{3.775731in}{3.607163in}}%
\pgfpathlineto{\pgfqpoint{3.789020in}{3.595044in}}%
\pgfpathlineto{\pgfqpoint{3.796707in}{3.620916in}}%
\pgfpathlineto{\pgfqpoint{3.804392in}{3.647237in}}%
\pgfpathlineto{\pgfqpoint{3.812074in}{3.674016in}}%
\pgfpathlineto{\pgfqpoint{3.819754in}{3.701262in}}%
\pgfpathlineto{\pgfqpoint{3.806460in}{3.713850in}}%
\pgfpathlineto{\pgfqpoint{3.793167in}{3.726545in}}%
\pgfpathlineto{\pgfqpoint{3.779875in}{3.739348in}}%
\pgfpathlineto{\pgfqpoint{3.766584in}{3.752259in}}%
\pgfpathlineto{\pgfqpoint{3.758909in}{3.724534in}}%
\pgfpathlineto{\pgfqpoint{3.751231in}{3.697283in}}%
\pgfpathlineto{\pgfqpoint{3.743551in}{3.670496in}}%
\pgfpathlineto{\pgfqpoint{3.735867in}{3.644164in}}%
\pgfpathclose%
\pgfusepath{fill}%
\end{pgfscope}%
\begin{pgfscope}%
\pgfpathrectangle{\pgfqpoint{1.150000in}{0.150000in}}{\pgfqpoint{5.700000in}{5.700000in}}%
\pgfusepath{clip}%
\pgfsetbuttcap%
\pgfsetroundjoin%
\definecolor{currentfill}{rgb}{0.169646,0.456262,0.558030}%
\pgfsetfillcolor{currentfill}%
\pgfsetfillopacity{0.700000}%
\pgfsetlinewidth{0.000000pt}%
\definecolor{currentstroke}{rgb}{0.000000,0.000000,0.000000}%
\pgfsetstrokecolor{currentstroke}%
\pgfsetdash{}{0pt}%
\pgfpathmoveto{\pgfqpoint{3.125497in}{3.512186in}}%
\pgfpathlineto{\pgfqpoint{3.138782in}{3.498186in}}%
\pgfpathlineto{\pgfqpoint{3.152065in}{3.484325in}}%
\pgfpathlineto{\pgfqpoint{3.165346in}{3.470603in}}%
\pgfpathlineto{\pgfqpoint{3.178625in}{3.457017in}}%
\pgfpathlineto{\pgfqpoint{3.186418in}{3.477132in}}%
\pgfpathlineto{\pgfqpoint{3.194203in}{3.497560in}}%
\pgfpathlineto{\pgfqpoint{3.201981in}{3.518307in}}%
\pgfpathlineto{\pgfqpoint{3.209752in}{3.539380in}}%
\pgfpathlineto{\pgfqpoint{3.196471in}{3.553310in}}%
\pgfpathlineto{\pgfqpoint{3.183189in}{3.567377in}}%
\pgfpathlineto{\pgfqpoint{3.169905in}{3.581583in}}%
\pgfpathlineto{\pgfqpoint{3.156620in}{3.595928in}}%
\pgfpathlineto{\pgfqpoint{3.148850in}{3.574502in}}%
\pgfpathlineto{\pgfqpoint{3.141073in}{3.553408in}}%
\pgfpathlineto{\pgfqpoint{3.133289in}{3.532638in}}%
\pgfpathlineto{\pgfqpoint{3.125497in}{3.512186in}}%
\pgfpathclose%
\pgfusepath{fill}%
\end{pgfscope}%
\begin{pgfscope}%
\pgfpathrectangle{\pgfqpoint{1.150000in}{0.150000in}}{\pgfqpoint{5.700000in}{5.700000in}}%
\pgfusepath{clip}%
\pgfsetbuttcap%
\pgfsetroundjoin%
\definecolor{currentfill}{rgb}{0.182256,0.426184,0.557120}%
\pgfsetfillcolor{currentfill}%
\pgfsetfillopacity{0.700000}%
\pgfsetlinewidth{0.000000pt}%
\definecolor{currentstroke}{rgb}{0.000000,0.000000,0.000000}%
\pgfsetstrokecolor{currentstroke}%
\pgfsetdash{}{0pt}%
\pgfpathmoveto{\pgfqpoint{3.315954in}{3.432725in}}%
\pgfpathlineto{\pgfqpoint{3.329225in}{3.419974in}}%
\pgfpathlineto{\pgfqpoint{3.342497in}{3.407349in}}%
\pgfpathlineto{\pgfqpoint{3.355767in}{3.394847in}}%
\pgfpathlineto{\pgfqpoint{3.369038in}{3.382469in}}%
\pgfpathlineto{\pgfqpoint{3.376805in}{3.402805in}}%
\pgfpathlineto{\pgfqpoint{3.384566in}{3.423463in}}%
\pgfpathlineto{\pgfqpoint{3.392321in}{3.444450in}}%
\pgfpathlineto{\pgfqpoint{3.400070in}{3.465773in}}%
\pgfpathlineto{\pgfqpoint{3.386798in}{3.478514in}}%
\pgfpathlineto{\pgfqpoint{3.373527in}{3.491379in}}%
\pgfpathlineto{\pgfqpoint{3.360255in}{3.504368in}}%
\pgfpathlineto{\pgfqpoint{3.346982in}{3.517483in}}%
\pgfpathlineto{\pgfqpoint{3.339234in}{3.495788in}}%
\pgfpathlineto{\pgfqpoint{3.331480in}{3.474435in}}%
\pgfpathlineto{\pgfqpoint{3.323720in}{3.453416in}}%
\pgfpathlineto{\pgfqpoint{3.315954in}{3.432725in}}%
\pgfpathclose%
\pgfusepath{fill}%
\end{pgfscope}%
\begin{pgfscope}%
\pgfpathrectangle{\pgfqpoint{1.150000in}{0.150000in}}{\pgfqpoint{5.700000in}{5.700000in}}%
\pgfusepath{clip}%
\pgfsetbuttcap%
\pgfsetroundjoin%
\definecolor{currentfill}{rgb}{0.157851,0.683765,0.501686}%
\pgfsetfillcolor{currentfill}%
\pgfsetfillopacity{0.700000}%
\pgfsetlinewidth{0.000000pt}%
\definecolor{currentstroke}{rgb}{0.000000,0.000000,0.000000}%
\pgfsetstrokecolor{currentstroke}%
\pgfsetdash{}{0pt}%
\pgfpathmoveto{\pgfqpoint{3.584487in}{4.098118in}}%
\pgfpathlineto{\pgfqpoint{3.597790in}{4.082845in}}%
\pgfpathlineto{\pgfqpoint{3.611092in}{4.067696in}}%
\pgfpathlineto{\pgfqpoint{3.624392in}{4.052669in}}%
\pgfpathlineto{\pgfqpoint{3.637692in}{4.037765in}}%
\pgfpathlineto{\pgfqpoint{3.645332in}{4.069534in}}%
\pgfpathlineto{\pgfqpoint{3.652969in}{4.101852in}}%
\pgfpathlineto{\pgfqpoint{3.660604in}{4.134730in}}%
\pgfpathlineto{\pgfqpoint{3.668236in}{4.168178in}}%
\pgfpathlineto{\pgfqpoint{3.654926in}{4.183608in}}%
\pgfpathlineto{\pgfqpoint{3.641615in}{4.199160in}}%
\pgfpathlineto{\pgfqpoint{3.628303in}{4.214837in}}%
\pgfpathlineto{\pgfqpoint{3.614989in}{4.230638in}}%
\pgfpathlineto{\pgfqpoint{3.607368in}{4.196653in}}%
\pgfpathlineto{\pgfqpoint{3.599744in}{4.163246in}}%
\pgfpathlineto{\pgfqpoint{3.592117in}{4.130404in}}%
\pgfpathlineto{\pgfqpoint{3.584487in}{4.098118in}}%
\pgfpathclose%
\pgfusepath{fill}%
\end{pgfscope}%
\begin{pgfscope}%
\pgfpathrectangle{\pgfqpoint{1.150000in}{0.150000in}}{\pgfqpoint{5.700000in}{5.700000in}}%
\pgfusepath{clip}%
\pgfsetbuttcap%
\pgfsetroundjoin%
\definecolor{currentfill}{rgb}{0.185783,0.704891,0.485273}%
\pgfsetfillcolor{currentfill}%
\pgfsetfillopacity{0.700000}%
\pgfsetlinewidth{0.000000pt}%
\definecolor{currentstroke}{rgb}{0.000000,0.000000,0.000000}%
\pgfsetstrokecolor{currentstroke}%
\pgfsetdash{}{0pt}%
\pgfpathmoveto{\pgfqpoint{3.531264in}{4.160474in}}%
\pgfpathlineto{\pgfqpoint{3.544572in}{4.144693in}}%
\pgfpathlineto{\pgfqpoint{3.557878in}{4.129041in}}%
\pgfpathlineto{\pgfqpoint{3.571183in}{4.113517in}}%
\pgfpathlineto{\pgfqpoint{3.584487in}{4.098118in}}%
\pgfpathlineto{\pgfqpoint{3.592117in}{4.130404in}}%
\pgfpathlineto{\pgfqpoint{3.599744in}{4.163246in}}%
\pgfpathlineto{\pgfqpoint{3.607368in}{4.196653in}}%
\pgfpathlineto{\pgfqpoint{3.614989in}{4.230638in}}%
\pgfpathlineto{\pgfqpoint{3.601674in}{4.246564in}}%
\pgfpathlineto{\pgfqpoint{3.588358in}{4.262618in}}%
\pgfpathlineto{\pgfqpoint{3.575041in}{4.278800in}}%
\pgfpathlineto{\pgfqpoint{3.561721in}{4.295111in}}%
\pgfpathlineto{\pgfqpoint{3.554112in}{4.260587in}}%
\pgfpathlineto{\pgfqpoint{3.546500in}{4.226647in}}%
\pgfpathlineto{\pgfqpoint{3.538884in}{4.193279in}}%
\pgfpathlineto{\pgfqpoint{3.531264in}{4.160474in}}%
\pgfpathclose%
\pgfusepath{fill}%
\end{pgfscope}%
\begin{pgfscope}%
\pgfpathrectangle{\pgfqpoint{1.150000in}{0.150000in}}{\pgfqpoint{5.700000in}{5.700000in}}%
\pgfusepath{clip}%
\pgfsetbuttcap%
\pgfsetroundjoin%
\definecolor{currentfill}{rgb}{0.137339,0.662252,0.515571}%
\pgfsetfillcolor{currentfill}%
\pgfsetfillopacity{0.700000}%
\pgfsetlinewidth{0.000000pt}%
\definecolor{currentstroke}{rgb}{0.000000,0.000000,0.000000}%
\pgfsetstrokecolor{currentstroke}%
\pgfsetdash{}{0pt}%
\pgfpathmoveto{\pgfqpoint{3.637692in}{4.037765in}}%
\pgfpathlineto{\pgfqpoint{3.650992in}{4.022982in}}%
\pgfpathlineto{\pgfqpoint{3.664290in}{4.008319in}}%
\pgfpathlineto{\pgfqpoint{3.677588in}{3.993775in}}%
\pgfpathlineto{\pgfqpoint{3.690886in}{3.979348in}}%
\pgfpathlineto{\pgfqpoint{3.698535in}{4.010603in}}%
\pgfpathlineto{\pgfqpoint{3.706181in}{4.042400in}}%
\pgfpathlineto{\pgfqpoint{3.713826in}{4.074751in}}%
\pgfpathlineto{\pgfqpoint{3.721468in}{4.107665in}}%
\pgfpathlineto{\pgfqpoint{3.708161in}{4.122614in}}%
\pgfpathlineto{\pgfqpoint{3.694854in}{4.137682in}}%
\pgfpathlineto{\pgfqpoint{3.681545in}{4.152870in}}%
\pgfpathlineto{\pgfqpoint{3.668236in}{4.168178in}}%
\pgfpathlineto{\pgfqpoint{3.660604in}{4.134730in}}%
\pgfpathlineto{\pgfqpoint{3.652969in}{4.101852in}}%
\pgfpathlineto{\pgfqpoint{3.645332in}{4.069534in}}%
\pgfpathlineto{\pgfqpoint{3.637692in}{4.037765in}}%
\pgfpathclose%
\pgfusepath{fill}%
\end{pgfscope}%
\begin{pgfscope}%
\pgfpathrectangle{\pgfqpoint{1.150000in}{0.150000in}}{\pgfqpoint{5.700000in}{5.700000in}}%
\pgfusepath{clip}%
\pgfsetbuttcap%
\pgfsetroundjoin%
\definecolor{currentfill}{rgb}{0.162142,0.474838,0.558140}%
\pgfsetfillcolor{currentfill}%
\pgfsetfillopacity{0.700000}%
\pgfsetlinewidth{0.000000pt}%
\definecolor{currentstroke}{rgb}{0.000000,0.000000,0.000000}%
\pgfsetstrokecolor{currentstroke}%
\pgfsetdash{}{0pt}%
\pgfpathmoveto{\pgfqpoint{3.705104in}{3.543212in}}%
\pgfpathlineto{\pgfqpoint{3.718387in}{3.531217in}}%
\pgfpathlineto{\pgfqpoint{3.731672in}{3.519331in}}%
\pgfpathlineto{\pgfqpoint{3.744957in}{3.507551in}}%
\pgfpathlineto{\pgfqpoint{3.758244in}{3.495878in}}%
\pgfpathlineto{\pgfqpoint{3.765942in}{3.520039in}}%
\pgfpathlineto{\pgfqpoint{3.773638in}{3.544615in}}%
\pgfpathlineto{\pgfqpoint{3.781330in}{3.569614in}}%
\pgfpathlineto{\pgfqpoint{3.789020in}{3.595044in}}%
\pgfpathlineto{\pgfqpoint{3.775731in}{3.607163in}}%
\pgfpathlineto{\pgfqpoint{3.762442in}{3.619389in}}%
\pgfpathlineto{\pgfqpoint{3.749154in}{3.631722in}}%
\pgfpathlineto{\pgfqpoint{3.735867in}{3.644164in}}%
\pgfpathlineto{\pgfqpoint{3.728181in}{3.618278in}}%
\pgfpathlineto{\pgfqpoint{3.720492in}{3.592829in}}%
\pgfpathlineto{\pgfqpoint{3.712799in}{3.567810in}}%
\pgfpathlineto{\pgfqpoint{3.705104in}{3.543212in}}%
\pgfpathclose%
\pgfusepath{fill}%
\end{pgfscope}%
\begin{pgfscope}%
\pgfpathrectangle{\pgfqpoint{1.150000in}{0.150000in}}{\pgfqpoint{5.700000in}{5.700000in}}%
\pgfusepath{clip}%
\pgfsetbuttcap%
\pgfsetroundjoin%
\definecolor{currentfill}{rgb}{0.183898,0.422383,0.556944}%
\pgfsetfillcolor{currentfill}%
\pgfsetfillopacity{0.700000}%
\pgfsetlinewidth{0.000000pt}%
\definecolor{currentstroke}{rgb}{0.000000,0.000000,0.000000}%
\pgfsetstrokecolor{currentstroke}%
\pgfsetdash{}{0pt}%
\pgfpathmoveto{\pgfqpoint{3.453154in}{3.416024in}}%
\pgfpathlineto{\pgfqpoint{3.466425in}{3.403885in}}%
\pgfpathlineto{\pgfqpoint{3.479697in}{3.391864in}}%
\pgfpathlineto{\pgfqpoint{3.492969in}{3.379960in}}%
\pgfpathlineto{\pgfqpoint{3.506241in}{3.368171in}}%
\pgfpathlineto{\pgfqpoint{3.513986in}{3.389086in}}%
\pgfpathlineto{\pgfqpoint{3.521726in}{3.410340in}}%
\pgfpathlineto{\pgfqpoint{3.529461in}{3.431941in}}%
\pgfpathlineto{\pgfqpoint{3.537191in}{3.453894in}}%
\pgfpathlineto{\pgfqpoint{3.523918in}{3.466066in}}%
\pgfpathlineto{\pgfqpoint{3.510645in}{3.478353in}}%
\pgfpathlineto{\pgfqpoint{3.497372in}{3.490758in}}%
\pgfpathlineto{\pgfqpoint{3.484099in}{3.503280in}}%
\pgfpathlineto{\pgfqpoint{3.476371in}{3.480935in}}%
\pgfpathlineto{\pgfqpoint{3.468637in}{3.458949in}}%
\pgfpathlineto{\pgfqpoint{3.460899in}{3.437314in}}%
\pgfpathlineto{\pgfqpoint{3.453154in}{3.416024in}}%
\pgfpathclose%
\pgfusepath{fill}%
\end{pgfscope}%
\begin{pgfscope}%
\pgfpathrectangle{\pgfqpoint{1.150000in}{0.150000in}}{\pgfqpoint{5.700000in}{5.700000in}}%
\pgfusepath{clip}%
\pgfsetbuttcap%
\pgfsetroundjoin%
\definecolor{currentfill}{rgb}{0.132444,0.552216,0.553018}%
\pgfsetfillcolor{currentfill}%
\pgfsetfillopacity{0.700000}%
\pgfsetlinewidth{0.000000pt}%
\definecolor{currentstroke}{rgb}{0.000000,0.000000,0.000000}%
\pgfsetstrokecolor{currentstroke}%
\pgfsetdash{}{0pt}%
\pgfpathmoveto{\pgfqpoint{3.766584in}{3.752259in}}%
\pgfpathlineto{\pgfqpoint{3.779875in}{3.739348in}}%
\pgfpathlineto{\pgfqpoint{3.793167in}{3.726545in}}%
\pgfpathlineto{\pgfqpoint{3.806460in}{3.713850in}}%
\pgfpathlineto{\pgfqpoint{3.819754in}{3.701262in}}%
\pgfpathlineto{\pgfqpoint{3.827432in}{3.728985in}}%
\pgfpathlineto{\pgfqpoint{3.835108in}{3.757194in}}%
\pgfpathlineto{\pgfqpoint{3.842783in}{3.785898in}}%
\pgfpathlineto{\pgfqpoint{3.850456in}{3.815106in}}%
\pgfpathlineto{\pgfqpoint{3.837157in}{3.828187in}}%
\pgfpathlineto{\pgfqpoint{3.823858in}{3.841375in}}%
\pgfpathlineto{\pgfqpoint{3.810560in}{3.854671in}}%
\pgfpathlineto{\pgfqpoint{3.797262in}{3.868076in}}%
\pgfpathlineto{\pgfqpoint{3.789595in}{3.838365in}}%
\pgfpathlineto{\pgfqpoint{3.781927in}{3.809165in}}%
\pgfpathlineto{\pgfqpoint{3.774256in}{3.780466in}}%
\pgfpathlineto{\pgfqpoint{3.766584in}{3.752259in}}%
\pgfpathclose%
\pgfusepath{fill}%
\end{pgfscope}%
\begin{pgfscope}%
\pgfpathrectangle{\pgfqpoint{1.150000in}{0.150000in}}{\pgfqpoint{5.700000in}{5.700000in}}%
\pgfusepath{clip}%
\pgfsetbuttcap%
\pgfsetroundjoin%
\definecolor{currentfill}{rgb}{0.226397,0.728888,0.462789}%
\pgfsetfillcolor{currentfill}%
\pgfsetfillopacity{0.700000}%
\pgfsetlinewidth{0.000000pt}%
\definecolor{currentstroke}{rgb}{0.000000,0.000000,0.000000}%
\pgfsetstrokecolor{currentstroke}%
\pgfsetdash{}{0pt}%
\pgfpathmoveto{\pgfqpoint{3.478016in}{4.224904in}}%
\pgfpathlineto{\pgfqpoint{3.491331in}{4.208598in}}%
\pgfpathlineto{\pgfqpoint{3.504644in}{4.192425in}}%
\pgfpathlineto{\pgfqpoint{3.517955in}{4.176384in}}%
\pgfpathlineto{\pgfqpoint{3.531264in}{4.160474in}}%
\pgfpathlineto{\pgfqpoint{3.538884in}{4.193279in}}%
\pgfpathlineto{\pgfqpoint{3.546500in}{4.226647in}}%
\pgfpathlineto{\pgfqpoint{3.554112in}{4.260587in}}%
\pgfpathlineto{\pgfqpoint{3.561721in}{4.295111in}}%
\pgfpathlineto{\pgfqpoint{3.548400in}{4.311552in}}%
\pgfpathlineto{\pgfqpoint{3.535078in}{4.328125in}}%
\pgfpathlineto{\pgfqpoint{3.521753in}{4.344830in}}%
\pgfpathlineto{\pgfqpoint{3.508426in}{4.361670in}}%
\pgfpathlineto{\pgfqpoint{3.500829in}{4.326604in}}%
\pgfpathlineto{\pgfqpoint{3.493229in}{4.292129in}}%
\pgfpathlineto{\pgfqpoint{3.485625in}{4.258232in}}%
\pgfpathlineto{\pgfqpoint{3.478016in}{4.224904in}}%
\pgfpathclose%
\pgfusepath{fill}%
\end{pgfscope}%
\begin{pgfscope}%
\pgfpathrectangle{\pgfqpoint{1.150000in}{0.150000in}}{\pgfqpoint{5.700000in}{5.700000in}}%
\pgfusepath{clip}%
\pgfsetbuttcap%
\pgfsetroundjoin%
\definecolor{currentfill}{rgb}{0.124780,0.640461,0.527068}%
\pgfsetfillcolor{currentfill}%
\pgfsetfillopacity{0.700000}%
\pgfsetlinewidth{0.000000pt}%
\definecolor{currentstroke}{rgb}{0.000000,0.000000,0.000000}%
\pgfsetstrokecolor{currentstroke}%
\pgfsetdash{}{0pt}%
\pgfpathmoveto{\pgfqpoint{3.690886in}{3.979348in}}%
\pgfpathlineto{\pgfqpoint{3.704183in}{3.965039in}}%
\pgfpathlineto{\pgfqpoint{3.717480in}{3.950846in}}%
\pgfpathlineto{\pgfqpoint{3.730777in}{3.936769in}}%
\pgfpathlineto{\pgfqpoint{3.744074in}{3.922805in}}%
\pgfpathlineto{\pgfqpoint{3.751731in}{3.953547in}}%
\pgfpathlineto{\pgfqpoint{3.759386in}{3.984826in}}%
\pgfpathlineto{\pgfqpoint{3.767040in}{4.016651in}}%
\pgfpathlineto{\pgfqpoint{3.774692in}{4.049034in}}%
\pgfpathlineto{\pgfqpoint{3.761387in}{4.063518in}}%
\pgfpathlineto{\pgfqpoint{3.748081in}{4.078118in}}%
\pgfpathlineto{\pgfqpoint{3.734775in}{4.092833in}}%
\pgfpathlineto{\pgfqpoint{3.721468in}{4.107665in}}%
\pgfpathlineto{\pgfqpoint{3.713826in}{4.074751in}}%
\pgfpathlineto{\pgfqpoint{3.706181in}{4.042400in}}%
\pgfpathlineto{\pgfqpoint{3.698535in}{4.010603in}}%
\pgfpathlineto{\pgfqpoint{3.690886in}{3.979348in}}%
\pgfpathclose%
\pgfusepath{fill}%
\end{pgfscope}%
\begin{pgfscope}%
\pgfpathrectangle{\pgfqpoint{1.150000in}{0.150000in}}{\pgfqpoint{5.700000in}{5.700000in}}%
\pgfusepath{clip}%
\pgfsetbuttcap%
\pgfsetroundjoin%
\definecolor{currentfill}{rgb}{0.177423,0.437527,0.557565}%
\pgfsetfillcolor{currentfill}%
\pgfsetfillopacity{0.700000}%
\pgfsetlinewidth{0.000000pt}%
\definecolor{currentstroke}{rgb}{0.000000,0.000000,0.000000}%
\pgfsetstrokecolor{currentstroke}%
\pgfsetdash{}{0pt}%
\pgfpathmoveto{\pgfqpoint{3.178625in}{3.457017in}}%
\pgfpathlineto{\pgfqpoint{3.191904in}{3.443567in}}%
\pgfpathlineto{\pgfqpoint{3.205181in}{3.430251in}}%
\pgfpathlineto{\pgfqpoint{3.218457in}{3.417068in}}%
\pgfpathlineto{\pgfqpoint{3.231732in}{3.404017in}}%
\pgfpathlineto{\pgfqpoint{3.239524in}{3.423797in}}%
\pgfpathlineto{\pgfqpoint{3.247310in}{3.443883in}}%
\pgfpathlineto{\pgfqpoint{3.255089in}{3.464284in}}%
\pgfpathlineto{\pgfqpoint{3.262861in}{3.485004in}}%
\pgfpathlineto{\pgfqpoint{3.249585in}{3.498399in}}%
\pgfpathlineto{\pgfqpoint{3.236309in}{3.511925in}}%
\pgfpathlineto{\pgfqpoint{3.223031in}{3.525585in}}%
\pgfpathlineto{\pgfqpoint{3.209752in}{3.539380in}}%
\pgfpathlineto{\pgfqpoint{3.201981in}{3.518307in}}%
\pgfpathlineto{\pgfqpoint{3.194203in}{3.497560in}}%
\pgfpathlineto{\pgfqpoint{3.186418in}{3.477132in}}%
\pgfpathlineto{\pgfqpoint{3.178625in}{3.457017in}}%
\pgfpathclose%
\pgfusepath{fill}%
\end{pgfscope}%
\begin{pgfscope}%
\pgfpathrectangle{\pgfqpoint{1.150000in}{0.150000in}}{\pgfqpoint{5.700000in}{5.700000in}}%
\pgfusepath{clip}%
\pgfsetbuttcap%
\pgfsetroundjoin%
\definecolor{currentfill}{rgb}{0.119699,0.618490,0.536347}%
\pgfsetfillcolor{currentfill}%
\pgfsetfillopacity{0.700000}%
\pgfsetlinewidth{0.000000pt}%
\definecolor{currentstroke}{rgb}{0.000000,0.000000,0.000000}%
\pgfsetstrokecolor{currentstroke}%
\pgfsetdash{}{0pt}%
\pgfpathmoveto{\pgfqpoint{3.744074in}{3.922805in}}%
\pgfpathlineto{\pgfqpoint{3.757371in}{3.908955in}}%
\pgfpathlineto{\pgfqpoint{3.770668in}{3.895217in}}%
\pgfpathlineto{\pgfqpoint{3.783965in}{3.881591in}}%
\pgfpathlineto{\pgfqpoint{3.797262in}{3.868076in}}%
\pgfpathlineto{\pgfqpoint{3.804927in}{3.898308in}}%
\pgfpathlineto{\pgfqpoint{3.812590in}{3.929070in}}%
\pgfpathlineto{\pgfqpoint{3.820252in}{3.960373in}}%
\pgfpathlineto{\pgfqpoint{3.827913in}{3.992227in}}%
\pgfpathlineto{\pgfqpoint{3.814608in}{4.006261in}}%
\pgfpathlineto{\pgfqpoint{3.801302in}{4.020406in}}%
\pgfpathlineto{\pgfqpoint{3.787997in}{4.034663in}}%
\pgfpathlineto{\pgfqpoint{3.774692in}{4.049034in}}%
\pgfpathlineto{\pgfqpoint{3.767040in}{4.016651in}}%
\pgfpathlineto{\pgfqpoint{3.759386in}{3.984826in}}%
\pgfpathlineto{\pgfqpoint{3.751731in}{3.953547in}}%
\pgfpathlineto{\pgfqpoint{3.744074in}{3.922805in}}%
\pgfpathclose%
\pgfusepath{fill}%
\end{pgfscope}%
\begin{pgfscope}%
\pgfpathrectangle{\pgfqpoint{1.150000in}{0.150000in}}{\pgfqpoint{5.700000in}{5.700000in}}%
\pgfusepath{clip}%
\pgfsetbuttcap%
\pgfsetroundjoin%
\definecolor{currentfill}{rgb}{0.274149,0.751988,0.436601}%
\pgfsetfillcolor{currentfill}%
\pgfsetfillopacity{0.700000}%
\pgfsetlinewidth{0.000000pt}%
\definecolor{currentstroke}{rgb}{0.000000,0.000000,0.000000}%
\pgfsetstrokecolor{currentstroke}%
\pgfsetdash{}{0pt}%
\pgfpathmoveto{\pgfqpoint{3.424737in}{4.291486in}}%
\pgfpathlineto{\pgfqpoint{3.438060in}{4.274634in}}%
\pgfpathlineto{\pgfqpoint{3.451381in}{4.257921in}}%
\pgfpathlineto{\pgfqpoint{3.464700in}{4.241345in}}%
\pgfpathlineto{\pgfqpoint{3.478016in}{4.224904in}}%
\pgfpathlineto{\pgfqpoint{3.485625in}{4.258232in}}%
\pgfpathlineto{\pgfqpoint{3.493229in}{4.292129in}}%
\pgfpathlineto{\pgfqpoint{3.500829in}{4.326604in}}%
\pgfpathlineto{\pgfqpoint{3.508426in}{4.361670in}}%
\pgfpathlineto{\pgfqpoint{3.495097in}{4.378644in}}%
\pgfpathlineto{\pgfqpoint{3.481766in}{4.395756in}}%
\pgfpathlineto{\pgfqpoint{3.468432in}{4.413004in}}%
\pgfpathlineto{\pgfqpoint{3.455096in}{4.430392in}}%
\pgfpathlineto{\pgfqpoint{3.447513in}{4.394781in}}%
\pgfpathlineto{\pgfqpoint{3.439926in}{4.359767in}}%
\pgfpathlineto{\pgfqpoint{3.432334in}{4.325339in}}%
\pgfpathlineto{\pgfqpoint{3.424737in}{4.291486in}}%
\pgfpathclose%
\pgfusepath{fill}%
\end{pgfscope}%
\begin{pgfscope}%
\pgfpathrectangle{\pgfqpoint{1.150000in}{0.150000in}}{\pgfqpoint{5.700000in}{5.700000in}}%
\pgfusepath{clip}%
\pgfsetbuttcap%
\pgfsetroundjoin%
\definecolor{currentfill}{rgb}{0.154815,0.493313,0.557840}%
\pgfsetfillcolor{currentfill}%
\pgfsetfillopacity{0.700000}%
\pgfsetlinewidth{0.000000pt}%
\definecolor{currentstroke}{rgb}{0.000000,0.000000,0.000000}%
\pgfsetstrokecolor{currentstroke}%
\pgfsetdash{}{0pt}%
\pgfpathmoveto{\pgfqpoint{3.789020in}{3.595044in}}%
\pgfpathlineto{\pgfqpoint{3.802311in}{3.583031in}}%
\pgfpathlineto{\pgfqpoint{3.815602in}{3.571123in}}%
\pgfpathlineto{\pgfqpoint{3.828895in}{3.559320in}}%
\pgfpathlineto{\pgfqpoint{3.842190in}{3.547620in}}%
\pgfpathlineto{\pgfqpoint{3.849880in}{3.573032in}}%
\pgfpathlineto{\pgfqpoint{3.857568in}{3.598888in}}%
\pgfpathlineto{\pgfqpoint{3.865254in}{3.625196in}}%
\pgfpathlineto{\pgfqpoint{3.872939in}{3.651966in}}%
\pgfpathlineto{\pgfqpoint{3.859641in}{3.664133in}}%
\pgfpathlineto{\pgfqpoint{3.846344in}{3.676404in}}%
\pgfpathlineto{\pgfqpoint{3.833049in}{3.688780in}}%
\pgfpathlineto{\pgfqpoint{3.819754in}{3.701262in}}%
\pgfpathlineto{\pgfqpoint{3.812074in}{3.674016in}}%
\pgfpathlineto{\pgfqpoint{3.804392in}{3.647237in}}%
\pgfpathlineto{\pgfqpoint{3.796707in}{3.620916in}}%
\pgfpathlineto{\pgfqpoint{3.789020in}{3.595044in}}%
\pgfpathclose%
\pgfusepath{fill}%
\end{pgfscope}%
\begin{pgfscope}%
\pgfpathrectangle{\pgfqpoint{1.150000in}{0.150000in}}{\pgfqpoint{5.700000in}{5.700000in}}%
\pgfusepath{clip}%
\pgfsetbuttcap%
\pgfsetroundjoin%
\definecolor{currentfill}{rgb}{0.177423,0.437527,0.557565}%
\pgfsetfillcolor{currentfill}%
\pgfsetfillopacity{0.700000}%
\pgfsetlinewidth{0.000000pt}%
\definecolor{currentstroke}{rgb}{0.000000,0.000000,0.000000}%
\pgfsetstrokecolor{currentstroke}%
\pgfsetdash{}{0pt}%
\pgfpathmoveto{\pgfqpoint{3.674284in}{3.448858in}}%
\pgfpathlineto{\pgfqpoint{3.687566in}{3.437288in}}%
\pgfpathlineto{\pgfqpoint{3.700848in}{3.425825in}}%
\pgfpathlineto{\pgfqpoint{3.714132in}{3.414469in}}%
\pgfpathlineto{\pgfqpoint{3.727417in}{3.403220in}}%
\pgfpathlineto{\pgfqpoint{3.735129in}{3.425803in}}%
\pgfpathlineto{\pgfqpoint{3.742837in}{3.448768in}}%
\pgfpathlineto{\pgfqpoint{3.750542in}{3.472124in}}%
\pgfpathlineto{\pgfqpoint{3.758244in}{3.495878in}}%
\pgfpathlineto{\pgfqpoint{3.744957in}{3.507551in}}%
\pgfpathlineto{\pgfqpoint{3.731672in}{3.519331in}}%
\pgfpathlineto{\pgfqpoint{3.718387in}{3.531217in}}%
\pgfpathlineto{\pgfqpoint{3.705104in}{3.543212in}}%
\pgfpathlineto{\pgfqpoint{3.697404in}{3.519025in}}%
\pgfpathlineto{\pgfqpoint{3.689701in}{3.495243in}}%
\pgfpathlineto{\pgfqpoint{3.681995in}{3.471856in}}%
\pgfpathlineto{\pgfqpoint{3.674284in}{3.448858in}}%
\pgfpathclose%
\pgfusepath{fill}%
\end{pgfscope}%
\begin{pgfscope}%
\pgfpathrectangle{\pgfqpoint{1.150000in}{0.150000in}}{\pgfqpoint{5.700000in}{5.700000in}}%
\pgfusepath{clip}%
\pgfsetbuttcap%
\pgfsetroundjoin%
\definecolor{currentfill}{rgb}{0.183898,0.422383,0.556944}%
\pgfsetfillcolor{currentfill}%
\pgfsetfillopacity{0.700000}%
\pgfsetlinewidth{0.000000pt}%
\definecolor{currentstroke}{rgb}{0.000000,0.000000,0.000000}%
\pgfsetstrokecolor{currentstroke}%
\pgfsetdash{}{0pt}%
\pgfpathmoveto{\pgfqpoint{3.590290in}{3.406349in}}%
\pgfpathlineto{\pgfqpoint{3.603566in}{3.394744in}}%
\pgfpathlineto{\pgfqpoint{3.616843in}{3.383249in}}%
\pgfpathlineto{\pgfqpoint{3.630122in}{3.371864in}}%
\pgfpathlineto{\pgfqpoint{3.643401in}{3.360588in}}%
\pgfpathlineto{\pgfqpoint{3.651128in}{3.382112in}}%
\pgfpathlineto{\pgfqpoint{3.658851in}{3.403993in}}%
\pgfpathlineto{\pgfqpoint{3.666570in}{3.426239in}}%
\pgfpathlineto{\pgfqpoint{3.674284in}{3.448858in}}%
\pgfpathlineto{\pgfqpoint{3.661004in}{3.460536in}}%
\pgfpathlineto{\pgfqpoint{3.647724in}{3.472324in}}%
\pgfpathlineto{\pgfqpoint{3.634445in}{3.484222in}}%
\pgfpathlineto{\pgfqpoint{3.621167in}{3.496231in}}%
\pgfpathlineto{\pgfqpoint{3.613455in}{3.473201in}}%
\pgfpathlineto{\pgfqpoint{3.605737in}{3.450549in}}%
\pgfpathlineto{\pgfqpoint{3.598016in}{3.428268in}}%
\pgfpathlineto{\pgfqpoint{3.590290in}{3.406349in}}%
\pgfpathclose%
\pgfusepath{fill}%
\end{pgfscope}%
\begin{pgfscope}%
\pgfpathrectangle{\pgfqpoint{1.150000in}{0.150000in}}{\pgfqpoint{5.700000in}{5.700000in}}%
\pgfusepath{clip}%
\pgfsetbuttcap%
\pgfsetroundjoin%
\definecolor{currentfill}{rgb}{0.188923,0.410910,0.556326}%
\pgfsetfillcolor{currentfill}%
\pgfsetfillopacity{0.700000}%
\pgfsetlinewidth{0.000000pt}%
\definecolor{currentstroke}{rgb}{0.000000,0.000000,0.000000}%
\pgfsetstrokecolor{currentstroke}%
\pgfsetdash{}{0pt}%
\pgfpathmoveto{\pgfqpoint{3.369038in}{3.382469in}}%
\pgfpathlineto{\pgfqpoint{3.382308in}{3.370214in}}%
\pgfpathlineto{\pgfqpoint{3.395579in}{3.358079in}}%
\pgfpathlineto{\pgfqpoint{3.408849in}{3.346065in}}%
\pgfpathlineto{\pgfqpoint{3.422120in}{3.334169in}}%
\pgfpathlineto{\pgfqpoint{3.429887in}{3.354151in}}%
\pgfpathlineto{\pgfqpoint{3.437649in}{3.374449in}}%
\pgfpathlineto{\pgfqpoint{3.445404in}{3.395071in}}%
\pgfpathlineto{\pgfqpoint{3.453154in}{3.416024in}}%
\pgfpathlineto{\pgfqpoint{3.439883in}{3.428281in}}%
\pgfpathlineto{\pgfqpoint{3.426612in}{3.440658in}}%
\pgfpathlineto{\pgfqpoint{3.413341in}{3.453155in}}%
\pgfpathlineto{\pgfqpoint{3.400070in}{3.465773in}}%
\pgfpathlineto{\pgfqpoint{3.392321in}{3.444450in}}%
\pgfpathlineto{\pgfqpoint{3.384566in}{3.423463in}}%
\pgfpathlineto{\pgfqpoint{3.376805in}{3.402805in}}%
\pgfpathlineto{\pgfqpoint{3.369038in}{3.382469in}}%
\pgfpathclose%
\pgfusepath{fill}%
\end{pgfscope}%
\begin{pgfscope}%
\pgfpathrectangle{\pgfqpoint{1.150000in}{0.150000in}}{\pgfqpoint{5.700000in}{5.700000in}}%
\pgfusepath{clip}%
\pgfsetbuttcap%
\pgfsetroundjoin%
\definecolor{currentfill}{rgb}{0.137770,0.537492,0.554906}%
\pgfsetfillcolor{currentfill}%
\pgfsetfillopacity{0.700000}%
\pgfsetlinewidth{0.000000pt}%
\definecolor{currentstroke}{rgb}{0.000000,0.000000,0.000000}%
\pgfsetstrokecolor{currentstroke}%
\pgfsetdash{}{0pt}%
\pgfpathmoveto{\pgfqpoint{3.819754in}{3.701262in}}%
\pgfpathlineto{\pgfqpoint{3.833049in}{3.688780in}}%
\pgfpathlineto{\pgfqpoint{3.846344in}{3.676404in}}%
\pgfpathlineto{\pgfqpoint{3.859641in}{3.664133in}}%
\pgfpathlineto{\pgfqpoint{3.872939in}{3.651966in}}%
\pgfpathlineto{\pgfqpoint{3.880621in}{3.679206in}}%
\pgfpathlineto{\pgfqpoint{3.888302in}{3.706926in}}%
\pgfpathlineto{\pgfqpoint{3.895982in}{3.735136in}}%
\pgfpathlineto{\pgfqpoint{3.903661in}{3.763844in}}%
\pgfpathlineto{\pgfqpoint{3.890358in}{3.776503in}}%
\pgfpathlineto{\pgfqpoint{3.877057in}{3.789265in}}%
\pgfpathlineto{\pgfqpoint{3.863756in}{3.802133in}}%
\pgfpathlineto{\pgfqpoint{3.850456in}{3.815106in}}%
\pgfpathlineto{\pgfqpoint{3.842783in}{3.785898in}}%
\pgfpathlineto{\pgfqpoint{3.835108in}{3.757194in}}%
\pgfpathlineto{\pgfqpoint{3.827432in}{3.728985in}}%
\pgfpathlineto{\pgfqpoint{3.819754in}{3.701262in}}%
\pgfpathclose%
\pgfusepath{fill}%
\end{pgfscope}%
\begin{pgfscope}%
\pgfpathrectangle{\pgfqpoint{1.150000in}{0.150000in}}{\pgfqpoint{5.700000in}{5.700000in}}%
\pgfusepath{clip}%
\pgfsetbuttcap%
\pgfsetroundjoin%
\definecolor{currentfill}{rgb}{0.185556,0.418570,0.556753}%
\pgfsetfillcolor{currentfill}%
\pgfsetfillopacity{0.700000}%
\pgfsetlinewidth{0.000000pt}%
\definecolor{currentstroke}{rgb}{0.000000,0.000000,0.000000}%
\pgfsetstrokecolor{currentstroke}%
\pgfsetdash{}{0pt}%
\pgfpathmoveto{\pgfqpoint{3.231732in}{3.404017in}}%
\pgfpathlineto{\pgfqpoint{3.245006in}{3.391097in}}%
\pgfpathlineto{\pgfqpoint{3.258279in}{3.378306in}}%
\pgfpathlineto{\pgfqpoint{3.271552in}{3.365644in}}%
\pgfpathlineto{\pgfqpoint{3.284824in}{3.353109in}}%
\pgfpathlineto{\pgfqpoint{3.292616in}{3.372554in}}%
\pgfpathlineto{\pgfqpoint{3.300402in}{3.392300in}}%
\pgfpathlineto{\pgfqpoint{3.308181in}{3.412356in}}%
\pgfpathlineto{\pgfqpoint{3.315954in}{3.432725in}}%
\pgfpathlineto{\pgfqpoint{3.302682in}{3.445602in}}%
\pgfpathlineto{\pgfqpoint{3.289409in}{3.458607in}}%
\pgfpathlineto{\pgfqpoint{3.276135in}{3.471741in}}%
\pgfpathlineto{\pgfqpoint{3.262861in}{3.485004in}}%
\pgfpathlineto{\pgfqpoint{3.255089in}{3.464284in}}%
\pgfpathlineto{\pgfqpoint{3.247310in}{3.443883in}}%
\pgfpathlineto{\pgfqpoint{3.239524in}{3.423797in}}%
\pgfpathlineto{\pgfqpoint{3.231732in}{3.404017in}}%
\pgfpathclose%
\pgfusepath{fill}%
\end{pgfscope}%
\begin{pgfscope}%
\pgfpathrectangle{\pgfqpoint{1.150000in}{0.150000in}}{\pgfqpoint{5.700000in}{5.700000in}}%
\pgfusepath{clip}%
\pgfsetbuttcap%
\pgfsetroundjoin%
\definecolor{currentfill}{rgb}{0.120092,0.600104,0.542530}%
\pgfsetfillcolor{currentfill}%
\pgfsetfillopacity{0.700000}%
\pgfsetlinewidth{0.000000pt}%
\definecolor{currentstroke}{rgb}{0.000000,0.000000,0.000000}%
\pgfsetstrokecolor{currentstroke}%
\pgfsetdash{}{0pt}%
\pgfpathmoveto{\pgfqpoint{3.797262in}{3.868076in}}%
\pgfpathlineto{\pgfqpoint{3.810560in}{3.854671in}}%
\pgfpathlineto{\pgfqpoint{3.823858in}{3.841375in}}%
\pgfpathlineto{\pgfqpoint{3.837157in}{3.828187in}}%
\pgfpathlineto{\pgfqpoint{3.850456in}{3.815106in}}%
\pgfpathlineto{\pgfqpoint{3.858128in}{3.844830in}}%
\pgfpathlineto{\pgfqpoint{3.865798in}{3.875078in}}%
\pgfpathlineto{\pgfqpoint{3.873468in}{3.905860in}}%
\pgfpathlineto{\pgfqpoint{3.881136in}{3.937187in}}%
\pgfpathlineto{\pgfqpoint{3.867830in}{3.950784in}}%
\pgfpathlineto{\pgfqpoint{3.854524in}{3.964489in}}%
\pgfpathlineto{\pgfqpoint{3.841218in}{3.978303in}}%
\pgfpathlineto{\pgfqpoint{3.827913in}{3.992227in}}%
\pgfpathlineto{\pgfqpoint{3.820252in}{3.960373in}}%
\pgfpathlineto{\pgfqpoint{3.812590in}{3.929070in}}%
\pgfpathlineto{\pgfqpoint{3.804927in}{3.898308in}}%
\pgfpathlineto{\pgfqpoint{3.797262in}{3.868076in}}%
\pgfpathclose%
\pgfusepath{fill}%
\end{pgfscope}%
\begin{pgfscope}%
\pgfpathrectangle{\pgfqpoint{1.150000in}{0.150000in}}{\pgfqpoint{5.700000in}{5.700000in}}%
\pgfusepath{clip}%
\pgfsetbuttcap%
\pgfsetroundjoin%
\definecolor{currentfill}{rgb}{0.169646,0.456262,0.558030}%
\pgfsetfillcolor{currentfill}%
\pgfsetfillopacity{0.700000}%
\pgfsetlinewidth{0.000000pt}%
\definecolor{currentstroke}{rgb}{0.000000,0.000000,0.000000}%
\pgfsetstrokecolor{currentstroke}%
\pgfsetdash{}{0pt}%
\pgfpathmoveto{\pgfqpoint{3.758244in}{3.495878in}}%
\pgfpathlineto{\pgfqpoint{3.771532in}{3.484311in}}%
\pgfpathlineto{\pgfqpoint{3.784821in}{3.472848in}}%
\pgfpathlineto{\pgfqpoint{3.798112in}{3.461490in}}%
\pgfpathlineto{\pgfqpoint{3.811404in}{3.450234in}}%
\pgfpathlineto{\pgfqpoint{3.819104in}{3.473958in}}%
\pgfpathlineto{\pgfqpoint{3.826802in}{3.498092in}}%
\pgfpathlineto{\pgfqpoint{3.834497in}{3.522643in}}%
\pgfpathlineto{\pgfqpoint{3.842190in}{3.547620in}}%
\pgfpathlineto{\pgfqpoint{3.828895in}{3.559320in}}%
\pgfpathlineto{\pgfqpoint{3.815602in}{3.571123in}}%
\pgfpathlineto{\pgfqpoint{3.802311in}{3.583031in}}%
\pgfpathlineto{\pgfqpoint{3.789020in}{3.595044in}}%
\pgfpathlineto{\pgfqpoint{3.781330in}{3.569614in}}%
\pgfpathlineto{\pgfqpoint{3.773638in}{3.544615in}}%
\pgfpathlineto{\pgfqpoint{3.765942in}{3.520039in}}%
\pgfpathlineto{\pgfqpoint{3.758244in}{3.495878in}}%
\pgfpathclose%
\pgfusepath{fill}%
\end{pgfscope}%
\begin{pgfscope}%
\pgfpathrectangle{\pgfqpoint{1.150000in}{0.150000in}}{\pgfqpoint{5.700000in}{5.700000in}}%
\pgfusepath{clip}%
\pgfsetbuttcap%
\pgfsetroundjoin%
\definecolor{currentfill}{rgb}{0.190631,0.407061,0.556089}%
\pgfsetfillcolor{currentfill}%
\pgfsetfillopacity{0.700000}%
\pgfsetlinewidth{0.000000pt}%
\definecolor{currentstroke}{rgb}{0.000000,0.000000,0.000000}%
\pgfsetstrokecolor{currentstroke}%
\pgfsetdash{}{0pt}%
\pgfpathmoveto{\pgfqpoint{3.506241in}{3.368171in}}%
\pgfpathlineto{\pgfqpoint{3.519514in}{3.356497in}}%
\pgfpathlineto{\pgfqpoint{3.532788in}{3.344936in}}%
\pgfpathlineto{\pgfqpoint{3.546062in}{3.333489in}}%
\pgfpathlineto{\pgfqpoint{3.559337in}{3.322154in}}%
\pgfpathlineto{\pgfqpoint{3.567082in}{3.342695in}}%
\pgfpathlineto{\pgfqpoint{3.574823in}{3.363570in}}%
\pgfpathlineto{\pgfqpoint{3.582559in}{3.384786in}}%
\pgfpathlineto{\pgfqpoint{3.590290in}{3.406349in}}%
\pgfpathlineto{\pgfqpoint{3.577014in}{3.418066in}}%
\pgfpathlineto{\pgfqpoint{3.563739in}{3.429895in}}%
\pgfpathlineto{\pgfqpoint{3.550465in}{3.441838in}}%
\pgfpathlineto{\pgfqpoint{3.537191in}{3.453894in}}%
\pgfpathlineto{\pgfqpoint{3.529461in}{3.431941in}}%
\pgfpathlineto{\pgfqpoint{3.521726in}{3.410340in}}%
\pgfpathlineto{\pgfqpoint{3.513986in}{3.389086in}}%
\pgfpathlineto{\pgfqpoint{3.506241in}{3.368171in}}%
\pgfpathclose%
\pgfusepath{fill}%
\end{pgfscope}%
\begin{pgfscope}%
\pgfpathrectangle{\pgfqpoint{1.150000in}{0.150000in}}{\pgfqpoint{5.700000in}{5.700000in}}%
\pgfusepath{clip}%
\pgfsetbuttcap%
\pgfsetroundjoin%
\definecolor{currentfill}{rgb}{0.327796,0.773980,0.406640}%
\pgfsetfillcolor{currentfill}%
\pgfsetfillopacity{0.700000}%
\pgfsetlinewidth{0.000000pt}%
\definecolor{currentstroke}{rgb}{0.000000,0.000000,0.000000}%
\pgfsetstrokecolor{currentstroke}%
\pgfsetdash{}{0pt}%
\pgfpathmoveto{\pgfqpoint{3.371419in}{4.360299in}}%
\pgfpathlineto{\pgfqpoint{3.384753in}{4.342882in}}%
\pgfpathlineto{\pgfqpoint{3.398083in}{4.325608in}}%
\pgfpathlineto{\pgfqpoint{3.411412in}{4.308476in}}%
\pgfpathlineto{\pgfqpoint{3.424737in}{4.291486in}}%
\pgfpathlineto{\pgfqpoint{3.432334in}{4.325339in}}%
\pgfpathlineto{\pgfqpoint{3.439926in}{4.359767in}}%
\pgfpathlineto{\pgfqpoint{3.447513in}{4.394781in}}%
\pgfpathlineto{\pgfqpoint{3.455096in}{4.430392in}}%
\pgfpathlineto{\pgfqpoint{3.441758in}{4.447920in}}%
\pgfpathlineto{\pgfqpoint{3.428416in}{4.465589in}}%
\pgfpathlineto{\pgfqpoint{3.415072in}{4.483402in}}%
\pgfpathlineto{\pgfqpoint{3.401725in}{4.501359in}}%
\pgfpathlineto{\pgfqpoint{3.394156in}{4.465199in}}%
\pgfpathlineto{\pgfqpoint{3.386582in}{4.429643in}}%
\pgfpathlineto{\pgfqpoint{3.379003in}{4.394680in}}%
\pgfpathlineto{\pgfqpoint{3.371419in}{4.360299in}}%
\pgfpathclose%
\pgfusepath{fill}%
\end{pgfscope}%
\begin{pgfscope}%
\pgfpathrectangle{\pgfqpoint{1.150000in}{0.150000in}}{\pgfqpoint{5.700000in}{5.700000in}}%
\pgfusepath{clip}%
\pgfsetbuttcap%
\pgfsetroundjoin%
\definecolor{currentfill}{rgb}{0.123463,0.581687,0.547445}%
\pgfsetfillcolor{currentfill}%
\pgfsetfillopacity{0.700000}%
\pgfsetlinewidth{0.000000pt}%
\definecolor{currentstroke}{rgb}{0.000000,0.000000,0.000000}%
\pgfsetstrokecolor{currentstroke}%
\pgfsetdash{}{0pt}%
\pgfpathmoveto{\pgfqpoint{3.850456in}{3.815106in}}%
\pgfpathlineto{\pgfqpoint{3.863756in}{3.802133in}}%
\pgfpathlineto{\pgfqpoint{3.877057in}{3.789265in}}%
\pgfpathlineto{\pgfqpoint{3.890358in}{3.776503in}}%
\pgfpathlineto{\pgfqpoint{3.903661in}{3.763844in}}%
\pgfpathlineto{\pgfqpoint{3.911339in}{3.793061in}}%
\pgfpathlineto{\pgfqpoint{3.919016in}{3.822797in}}%
\pgfpathlineto{\pgfqpoint{3.926692in}{3.853061in}}%
\pgfpathlineto{\pgfqpoint{3.934368in}{3.883863in}}%
\pgfpathlineto{\pgfqpoint{3.921059in}{3.897036in}}%
\pgfpathlineto{\pgfqpoint{3.907751in}{3.910314in}}%
\pgfpathlineto{\pgfqpoint{3.894443in}{3.923697in}}%
\pgfpathlineto{\pgfqpoint{3.881136in}{3.937187in}}%
\pgfpathlineto{\pgfqpoint{3.873468in}{3.905860in}}%
\pgfpathlineto{\pgfqpoint{3.865798in}{3.875078in}}%
\pgfpathlineto{\pgfqpoint{3.858128in}{3.844830in}}%
\pgfpathlineto{\pgfqpoint{3.850456in}{3.815106in}}%
\pgfpathclose%
\pgfusepath{fill}%
\end{pgfscope}%
\begin{pgfscope}%
\pgfpathrectangle{\pgfqpoint{1.150000in}{0.150000in}}{\pgfqpoint{5.700000in}{5.700000in}}%
\pgfusepath{clip}%
\pgfsetbuttcap%
\pgfsetroundjoin%
\definecolor{currentfill}{rgb}{0.180629,0.429975,0.557282}%
\pgfsetfillcolor{currentfill}%
\pgfsetfillopacity{0.700000}%
\pgfsetlinewidth{0.000000pt}%
\definecolor{currentstroke}{rgb}{0.000000,0.000000,0.000000}%
\pgfsetstrokecolor{currentstroke}%
\pgfsetdash{}{0pt}%
\pgfpathmoveto{\pgfqpoint{3.094256in}{3.433435in}}%
\pgfpathlineto{\pgfqpoint{3.107540in}{3.419759in}}%
\pgfpathlineto{\pgfqpoint{3.120823in}{3.406222in}}%
\pgfpathlineto{\pgfqpoint{3.134105in}{3.392822in}}%
\pgfpathlineto{\pgfqpoint{3.147385in}{3.379559in}}%
\pgfpathlineto{\pgfqpoint{3.155206in}{3.398485in}}%
\pgfpathlineto{\pgfqpoint{3.163020in}{3.417700in}}%
\pgfpathlineto{\pgfqpoint{3.170826in}{3.437208in}}%
\pgfpathlineto{\pgfqpoint{3.178625in}{3.457017in}}%
\pgfpathlineto{\pgfqpoint{3.165346in}{3.470603in}}%
\pgfpathlineto{\pgfqpoint{3.152065in}{3.484325in}}%
\pgfpathlineto{\pgfqpoint{3.138782in}{3.498186in}}%
\pgfpathlineto{\pgfqpoint{3.125497in}{3.512186in}}%
\pgfpathlineto{\pgfqpoint{3.117698in}{3.492046in}}%
\pgfpathlineto{\pgfqpoint{3.109892in}{3.472211in}}%
\pgfpathlineto{\pgfqpoint{3.102077in}{3.452676in}}%
\pgfpathlineto{\pgfqpoint{3.094256in}{3.433435in}}%
\pgfpathclose%
\pgfusepath{fill}%
\end{pgfscope}%
\begin{pgfscope}%
\pgfpathrectangle{\pgfqpoint{1.150000in}{0.150000in}}{\pgfqpoint{5.700000in}{5.700000in}}%
\pgfusepath{clip}%
\pgfsetbuttcap%
\pgfsetroundjoin%
\definecolor{currentfill}{rgb}{0.160665,0.478540,0.558115}%
\pgfsetfillcolor{currentfill}%
\pgfsetfillopacity{0.700000}%
\pgfsetlinewidth{0.000000pt}%
\definecolor{currentstroke}{rgb}{0.000000,0.000000,0.000000}%
\pgfsetstrokecolor{currentstroke}%
\pgfsetdash{}{0pt}%
\pgfpathmoveto{\pgfqpoint{3.842190in}{3.547620in}}%
\pgfpathlineto{\pgfqpoint{3.855485in}{3.536023in}}%
\pgfpathlineto{\pgfqpoint{3.868782in}{3.524529in}}%
\pgfpathlineto{\pgfqpoint{3.882081in}{3.513135in}}%
\pgfpathlineto{\pgfqpoint{3.895381in}{3.501843in}}%
\pgfpathlineto{\pgfqpoint{3.903074in}{3.526797in}}%
\pgfpathlineto{\pgfqpoint{3.910765in}{3.552189in}}%
\pgfpathlineto{\pgfqpoint{3.918455in}{3.578028in}}%
\pgfpathlineto{\pgfqpoint{3.926143in}{3.604322in}}%
\pgfpathlineto{\pgfqpoint{3.912840in}{3.616081in}}%
\pgfpathlineto{\pgfqpoint{3.899538in}{3.627940in}}%
\pgfpathlineto{\pgfqpoint{3.886238in}{3.639902in}}%
\pgfpathlineto{\pgfqpoint{3.872939in}{3.651966in}}%
\pgfpathlineto{\pgfqpoint{3.865254in}{3.625196in}}%
\pgfpathlineto{\pgfqpoint{3.857568in}{3.598888in}}%
\pgfpathlineto{\pgfqpoint{3.849880in}{3.573032in}}%
\pgfpathlineto{\pgfqpoint{3.842190in}{3.547620in}}%
\pgfpathclose%
\pgfusepath{fill}%
\end{pgfscope}%
\begin{pgfscope}%
\pgfpathrectangle{\pgfqpoint{1.150000in}{0.150000in}}{\pgfqpoint{5.700000in}{5.700000in}}%
\pgfusepath{clip}%
\pgfsetbuttcap%
\pgfsetroundjoin%
\definecolor{currentfill}{rgb}{0.197636,0.391528,0.554969}%
\pgfsetfillcolor{currentfill}%
\pgfsetfillopacity{0.700000}%
\pgfsetlinewidth{0.000000pt}%
\definecolor{currentstroke}{rgb}{0.000000,0.000000,0.000000}%
\pgfsetstrokecolor{currentstroke}%
\pgfsetdash{}{0pt}%
\pgfpathmoveto{\pgfqpoint{3.422120in}{3.334169in}}%
\pgfpathlineto{\pgfqpoint{3.435391in}{3.322392in}}%
\pgfpathlineto{\pgfqpoint{3.448663in}{3.310733in}}%
\pgfpathlineto{\pgfqpoint{3.461934in}{3.299189in}}%
\pgfpathlineto{\pgfqpoint{3.475207in}{3.287761in}}%
\pgfpathlineto{\pgfqpoint{3.482974in}{3.307390in}}%
\pgfpathlineto{\pgfqpoint{3.490735in}{3.327329in}}%
\pgfpathlineto{\pgfqpoint{3.498491in}{3.347588in}}%
\pgfpathlineto{\pgfqpoint{3.506241in}{3.368171in}}%
\pgfpathlineto{\pgfqpoint{3.492969in}{3.379960in}}%
\pgfpathlineto{\pgfqpoint{3.479697in}{3.391864in}}%
\pgfpathlineto{\pgfqpoint{3.466425in}{3.403885in}}%
\pgfpathlineto{\pgfqpoint{3.453154in}{3.416024in}}%
\pgfpathlineto{\pgfqpoint{3.445404in}{3.395071in}}%
\pgfpathlineto{\pgfqpoint{3.437649in}{3.374449in}}%
\pgfpathlineto{\pgfqpoint{3.429887in}{3.354151in}}%
\pgfpathlineto{\pgfqpoint{3.422120in}{3.334169in}}%
\pgfpathclose%
\pgfusepath{fill}%
\end{pgfscope}%
\begin{pgfscope}%
\pgfpathrectangle{\pgfqpoint{1.150000in}{0.150000in}}{\pgfqpoint{5.700000in}{5.700000in}}%
\pgfusepath{clip}%
\pgfsetbuttcap%
\pgfsetroundjoin%
\definecolor{currentfill}{rgb}{0.194100,0.399323,0.555565}%
\pgfsetfillcolor{currentfill}%
\pgfsetfillopacity{0.700000}%
\pgfsetlinewidth{0.000000pt}%
\definecolor{currentstroke}{rgb}{0.000000,0.000000,0.000000}%
\pgfsetstrokecolor{currentstroke}%
\pgfsetdash{}{0pt}%
\pgfpathmoveto{\pgfqpoint{3.284824in}{3.353109in}}%
\pgfpathlineto{\pgfqpoint{3.298095in}{3.340700in}}%
\pgfpathlineto{\pgfqpoint{3.311367in}{3.328416in}}%
\pgfpathlineto{\pgfqpoint{3.324638in}{3.316256in}}%
\pgfpathlineto{\pgfqpoint{3.337909in}{3.304219in}}%
\pgfpathlineto{\pgfqpoint{3.345700in}{3.323331in}}%
\pgfpathlineto{\pgfqpoint{3.353486in}{3.342739in}}%
\pgfpathlineto{\pgfqpoint{3.361265in}{3.362450in}}%
\pgfpathlineto{\pgfqpoint{3.369038in}{3.382469in}}%
\pgfpathlineto{\pgfqpoint{3.355767in}{3.394847in}}%
\pgfpathlineto{\pgfqpoint{3.342497in}{3.407349in}}%
\pgfpathlineto{\pgfqpoint{3.329225in}{3.419974in}}%
\pgfpathlineto{\pgfqpoint{3.315954in}{3.432725in}}%
\pgfpathlineto{\pgfqpoint{3.308181in}{3.412356in}}%
\pgfpathlineto{\pgfqpoint{3.300402in}{3.392300in}}%
\pgfpathlineto{\pgfqpoint{3.292616in}{3.372554in}}%
\pgfpathlineto{\pgfqpoint{3.284824in}{3.353109in}}%
\pgfpathclose%
\pgfusepath{fill}%
\end{pgfscope}%
\begin{pgfscope}%
\pgfpathrectangle{\pgfqpoint{1.150000in}{0.150000in}}{\pgfqpoint{5.700000in}{5.700000in}}%
\pgfusepath{clip}%
\pgfsetbuttcap%
\pgfsetroundjoin%
\definecolor{currentfill}{rgb}{0.144759,0.519093,0.556572}%
\pgfsetfillcolor{currentfill}%
\pgfsetfillopacity{0.700000}%
\pgfsetlinewidth{0.000000pt}%
\definecolor{currentstroke}{rgb}{0.000000,0.000000,0.000000}%
\pgfsetstrokecolor{currentstroke}%
\pgfsetdash{}{0pt}%
\pgfpathmoveto{\pgfqpoint{3.872939in}{3.651966in}}%
\pgfpathlineto{\pgfqpoint{3.886238in}{3.639902in}}%
\pgfpathlineto{\pgfqpoint{3.899538in}{3.627940in}}%
\pgfpathlineto{\pgfqpoint{3.912840in}{3.616081in}}%
\pgfpathlineto{\pgfqpoint{3.926143in}{3.604322in}}%
\pgfpathlineto{\pgfqpoint{3.933829in}{3.631082in}}%
\pgfpathlineto{\pgfqpoint{3.941514in}{3.658315in}}%
\pgfpathlineto{\pgfqpoint{3.949199in}{3.686032in}}%
\pgfpathlineto{\pgfqpoint{3.956883in}{3.714242in}}%
\pgfpathlineto{\pgfqpoint{3.943575in}{3.726489in}}%
\pgfpathlineto{\pgfqpoint{3.930269in}{3.738839in}}%
\pgfpathlineto{\pgfqpoint{3.916965in}{3.751290in}}%
\pgfpathlineto{\pgfqpoint{3.903661in}{3.763844in}}%
\pgfpathlineto{\pgfqpoint{3.895982in}{3.735136in}}%
\pgfpathlineto{\pgfqpoint{3.888302in}{3.706926in}}%
\pgfpathlineto{\pgfqpoint{3.880621in}{3.679206in}}%
\pgfpathlineto{\pgfqpoint{3.872939in}{3.651966in}}%
\pgfpathclose%
\pgfusepath{fill}%
\end{pgfscope}%
\begin{pgfscope}%
\pgfpathrectangle{\pgfqpoint{1.150000in}{0.150000in}}{\pgfqpoint{5.700000in}{5.700000in}}%
\pgfusepath{clip}%
\pgfsetbuttcap%
\pgfsetroundjoin%
\definecolor{currentfill}{rgb}{0.183898,0.422383,0.556944}%
\pgfsetfillcolor{currentfill}%
\pgfsetfillopacity{0.700000}%
\pgfsetlinewidth{0.000000pt}%
\definecolor{currentstroke}{rgb}{0.000000,0.000000,0.000000}%
\pgfsetstrokecolor{currentstroke}%
\pgfsetdash{}{0pt}%
\pgfpathmoveto{\pgfqpoint{3.727417in}{3.403220in}}%
\pgfpathlineto{\pgfqpoint{3.740703in}{3.392076in}}%
\pgfpathlineto{\pgfqpoint{3.753991in}{3.381036in}}%
\pgfpathlineto{\pgfqpoint{3.767281in}{3.370100in}}%
\pgfpathlineto{\pgfqpoint{3.780572in}{3.359267in}}%
\pgfpathlineto{\pgfqpoint{3.788285in}{3.381436in}}%
\pgfpathlineto{\pgfqpoint{3.795994in}{3.403981in}}%
\pgfpathlineto{\pgfqpoint{3.803701in}{3.426911in}}%
\pgfpathlineto{\pgfqpoint{3.811404in}{3.450234in}}%
\pgfpathlineto{\pgfqpoint{3.798112in}{3.461490in}}%
\pgfpathlineto{\pgfqpoint{3.784821in}{3.472848in}}%
\pgfpathlineto{\pgfqpoint{3.771532in}{3.484311in}}%
\pgfpathlineto{\pgfqpoint{3.758244in}{3.495878in}}%
\pgfpathlineto{\pgfqpoint{3.750542in}{3.472124in}}%
\pgfpathlineto{\pgfqpoint{3.742837in}{3.448768in}}%
\pgfpathlineto{\pgfqpoint{3.735129in}{3.425803in}}%
\pgfpathlineto{\pgfqpoint{3.727417in}{3.403220in}}%
\pgfpathclose%
\pgfusepath{fill}%
\end{pgfscope}%
\begin{pgfscope}%
\pgfpathrectangle{\pgfqpoint{1.150000in}{0.150000in}}{\pgfqpoint{5.700000in}{5.700000in}}%
\pgfusepath{clip}%
\pgfsetbuttcap%
\pgfsetroundjoin%
\definecolor{currentfill}{rgb}{0.190631,0.407061,0.556089}%
\pgfsetfillcolor{currentfill}%
\pgfsetfillopacity{0.700000}%
\pgfsetlinewidth{0.000000pt}%
\definecolor{currentstroke}{rgb}{0.000000,0.000000,0.000000}%
\pgfsetstrokecolor{currentstroke}%
\pgfsetdash{}{0pt}%
\pgfpathmoveto{\pgfqpoint{3.643401in}{3.360588in}}%
\pgfpathlineto{\pgfqpoint{3.656682in}{3.349420in}}%
\pgfpathlineto{\pgfqpoint{3.669963in}{3.338359in}}%
\pgfpathlineto{\pgfqpoint{3.683247in}{3.327405in}}%
\pgfpathlineto{\pgfqpoint{3.696531in}{3.316557in}}%
\pgfpathlineto{\pgfqpoint{3.704259in}{3.337688in}}%
\pgfpathlineto{\pgfqpoint{3.711982in}{3.359170in}}%
\pgfpathlineto{\pgfqpoint{3.719701in}{3.381011in}}%
\pgfpathlineto{\pgfqpoint{3.727417in}{3.403220in}}%
\pgfpathlineto{\pgfqpoint{3.714132in}{3.414469in}}%
\pgfpathlineto{\pgfqpoint{3.700848in}{3.425825in}}%
\pgfpathlineto{\pgfqpoint{3.687566in}{3.437288in}}%
\pgfpathlineto{\pgfqpoint{3.674284in}{3.448858in}}%
\pgfpathlineto{\pgfqpoint{3.666570in}{3.426239in}}%
\pgfpathlineto{\pgfqpoint{3.658851in}{3.403993in}}%
\pgfpathlineto{\pgfqpoint{3.651128in}{3.382112in}}%
\pgfpathlineto{\pgfqpoint{3.643401in}{3.360588in}}%
\pgfpathclose%
\pgfusepath{fill}%
\end{pgfscope}%
\begin{pgfscope}%
\pgfpathrectangle{\pgfqpoint{1.150000in}{0.150000in}}{\pgfqpoint{5.700000in}{5.700000in}}%
\pgfusepath{clip}%
\pgfsetbuttcap%
\pgfsetroundjoin%
\definecolor{currentfill}{rgb}{0.196571,0.711827,0.479221}%
\pgfsetfillcolor{currentfill}%
\pgfsetfillopacity{0.700000}%
\pgfsetlinewidth{0.000000pt}%
\definecolor{currentstroke}{rgb}{0.000000,0.000000,0.000000}%
\pgfsetstrokecolor{currentstroke}%
\pgfsetdash{}{0pt}%
\pgfpathmoveto{\pgfqpoint{3.668236in}{4.168178in}}%
\pgfpathlineto{\pgfqpoint{3.681545in}{4.152870in}}%
\pgfpathlineto{\pgfqpoint{3.694854in}{4.137682in}}%
\pgfpathlineto{\pgfqpoint{3.708161in}{4.122614in}}%
\pgfpathlineto{\pgfqpoint{3.721468in}{4.107665in}}%
\pgfpathlineto{\pgfqpoint{3.729109in}{4.141153in}}%
\pgfpathlineto{\pgfqpoint{3.736748in}{4.175225in}}%
\pgfpathlineto{\pgfqpoint{3.744385in}{4.209893in}}%
\pgfpathlineto{\pgfqpoint{3.752021in}{4.245168in}}%
\pgfpathlineto{\pgfqpoint{3.738703in}{4.260666in}}%
\pgfpathlineto{\pgfqpoint{3.725384in}{4.276284in}}%
\pgfpathlineto{\pgfqpoint{3.712064in}{4.292022in}}%
\pgfpathlineto{\pgfqpoint{3.698743in}{4.307882in}}%
\pgfpathlineto{\pgfqpoint{3.691119in}{4.272047in}}%
\pgfpathlineto{\pgfqpoint{3.683493in}{4.236826in}}%
\pgfpathlineto{\pgfqpoint{3.675866in}{4.202206in}}%
\pgfpathlineto{\pgfqpoint{3.668236in}{4.168178in}}%
\pgfpathclose%
\pgfusepath{fill}%
\end{pgfscope}%
\begin{pgfscope}%
\pgfpathrectangle{\pgfqpoint{1.150000in}{0.150000in}}{\pgfqpoint{5.700000in}{5.700000in}}%
\pgfusepath{clip}%
\pgfsetbuttcap%
\pgfsetroundjoin%
\definecolor{currentfill}{rgb}{0.166383,0.690856,0.496502}%
\pgfsetfillcolor{currentfill}%
\pgfsetfillopacity{0.700000}%
\pgfsetlinewidth{0.000000pt}%
\definecolor{currentstroke}{rgb}{0.000000,0.000000,0.000000}%
\pgfsetstrokecolor{currentstroke}%
\pgfsetdash{}{0pt}%
\pgfpathmoveto{\pgfqpoint{3.721468in}{4.107665in}}%
\pgfpathlineto{\pgfqpoint{3.734775in}{4.092833in}}%
\pgfpathlineto{\pgfqpoint{3.748081in}{4.078118in}}%
\pgfpathlineto{\pgfqpoint{3.761387in}{4.063518in}}%
\pgfpathlineto{\pgfqpoint{3.774692in}{4.049034in}}%
\pgfpathlineto{\pgfqpoint{3.782343in}{4.081984in}}%
\pgfpathlineto{\pgfqpoint{3.789992in}{4.115513in}}%
\pgfpathlineto{\pgfqpoint{3.797640in}{4.149630in}}%
\pgfpathlineto{\pgfqpoint{3.805288in}{4.184347in}}%
\pgfpathlineto{\pgfqpoint{3.791972in}{4.199378in}}%
\pgfpathlineto{\pgfqpoint{3.778656in}{4.214525in}}%
\pgfpathlineto{\pgfqpoint{3.765339in}{4.229788in}}%
\pgfpathlineto{\pgfqpoint{3.752021in}{4.245168in}}%
\pgfpathlineto{\pgfqpoint{3.744385in}{4.209893in}}%
\pgfpathlineto{\pgfqpoint{3.736748in}{4.175225in}}%
\pgfpathlineto{\pgfqpoint{3.729109in}{4.141153in}}%
\pgfpathlineto{\pgfqpoint{3.721468in}{4.107665in}}%
\pgfpathclose%
\pgfusepath{fill}%
\end{pgfscope}%
\begin{pgfscope}%
\pgfpathrectangle{\pgfqpoint{1.150000in}{0.150000in}}{\pgfqpoint{5.700000in}{5.700000in}}%
\pgfusepath{clip}%
\pgfsetbuttcap%
\pgfsetroundjoin%
\definecolor{currentfill}{rgb}{0.232815,0.732247,0.459277}%
\pgfsetfillcolor{currentfill}%
\pgfsetfillopacity{0.700000}%
\pgfsetlinewidth{0.000000pt}%
\definecolor{currentstroke}{rgb}{0.000000,0.000000,0.000000}%
\pgfsetstrokecolor{currentstroke}%
\pgfsetdash{}{0pt}%
\pgfpathmoveto{\pgfqpoint{3.614989in}{4.230638in}}%
\pgfpathlineto{\pgfqpoint{3.628303in}{4.214837in}}%
\pgfpathlineto{\pgfqpoint{3.641615in}{4.199160in}}%
\pgfpathlineto{\pgfqpoint{3.654926in}{4.183608in}}%
\pgfpathlineto{\pgfqpoint{3.668236in}{4.168178in}}%
\pgfpathlineto{\pgfqpoint{3.675866in}{4.202206in}}%
\pgfpathlineto{\pgfqpoint{3.683493in}{4.236826in}}%
\pgfpathlineto{\pgfqpoint{3.691119in}{4.272047in}}%
\pgfpathlineto{\pgfqpoint{3.698743in}{4.307882in}}%
\pgfpathlineto{\pgfqpoint{3.685421in}{4.323863in}}%
\pgfpathlineto{\pgfqpoint{3.672098in}{4.339968in}}%
\pgfpathlineto{\pgfqpoint{3.658773in}{4.356198in}}%
\pgfpathlineto{\pgfqpoint{3.645447in}{4.372553in}}%
\pgfpathlineto{\pgfqpoint{3.637836in}{4.336155in}}%
\pgfpathlineto{\pgfqpoint{3.630223in}{4.300378in}}%
\pgfpathlineto{\pgfqpoint{3.622607in}{4.265209in}}%
\pgfpathlineto{\pgfqpoint{3.614989in}{4.230638in}}%
\pgfpathclose%
\pgfusepath{fill}%
\end{pgfscope}%
\begin{pgfscope}%
\pgfpathrectangle{\pgfqpoint{1.150000in}{0.150000in}}{\pgfqpoint{5.700000in}{5.700000in}}%
\pgfusepath{clip}%
\pgfsetbuttcap%
\pgfsetroundjoin%
\definecolor{currentfill}{rgb}{0.175841,0.441290,0.557685}%
\pgfsetfillcolor{currentfill}%
\pgfsetfillopacity{0.700000}%
\pgfsetlinewidth{0.000000pt}%
\definecolor{currentstroke}{rgb}{0.000000,0.000000,0.000000}%
\pgfsetstrokecolor{currentstroke}%
\pgfsetdash{}{0pt}%
\pgfpathmoveto{\pgfqpoint{3.811404in}{3.450234in}}%
\pgfpathlineto{\pgfqpoint{3.824697in}{3.439082in}}%
\pgfpathlineto{\pgfqpoint{3.837993in}{3.428031in}}%
\pgfpathlineto{\pgfqpoint{3.851290in}{3.417082in}}%
\pgfpathlineto{\pgfqpoint{3.864588in}{3.406233in}}%
\pgfpathlineto{\pgfqpoint{3.872290in}{3.429522in}}%
\pgfpathlineto{\pgfqpoint{3.879990in}{3.453214in}}%
\pgfpathlineto{\pgfqpoint{3.887687in}{3.477318in}}%
\pgfpathlineto{\pgfqpoint{3.895381in}{3.501843in}}%
\pgfpathlineto{\pgfqpoint{3.882081in}{3.513135in}}%
\pgfpathlineto{\pgfqpoint{3.868782in}{3.524529in}}%
\pgfpathlineto{\pgfqpoint{3.855485in}{3.536023in}}%
\pgfpathlineto{\pgfqpoint{3.842190in}{3.547620in}}%
\pgfpathlineto{\pgfqpoint{3.834497in}{3.522643in}}%
\pgfpathlineto{\pgfqpoint{3.826802in}{3.498092in}}%
\pgfpathlineto{\pgfqpoint{3.819104in}{3.473958in}}%
\pgfpathlineto{\pgfqpoint{3.811404in}{3.450234in}}%
\pgfpathclose%
\pgfusepath{fill}%
\end{pgfscope}%
\begin{pgfscope}%
\pgfpathrectangle{\pgfqpoint{1.150000in}{0.150000in}}{\pgfqpoint{5.700000in}{5.700000in}}%
\pgfusepath{clip}%
\pgfsetbuttcap%
\pgfsetroundjoin%
\definecolor{currentfill}{rgb}{0.197636,0.391528,0.554969}%
\pgfsetfillcolor{currentfill}%
\pgfsetfillopacity{0.700000}%
\pgfsetlinewidth{0.000000pt}%
\definecolor{currentstroke}{rgb}{0.000000,0.000000,0.000000}%
\pgfsetstrokecolor{currentstroke}%
\pgfsetdash{}{0pt}%
\pgfpathmoveto{\pgfqpoint{3.559337in}{3.322154in}}%
\pgfpathlineto{\pgfqpoint{3.572613in}{3.310929in}}%
\pgfpathlineto{\pgfqpoint{3.585890in}{3.299816in}}%
\pgfpathlineto{\pgfqpoint{3.599169in}{3.288812in}}%
\pgfpathlineto{\pgfqpoint{3.612448in}{3.277916in}}%
\pgfpathlineto{\pgfqpoint{3.620193in}{3.298085in}}%
\pgfpathlineto{\pgfqpoint{3.627934in}{3.318581in}}%
\pgfpathlineto{\pgfqpoint{3.635670in}{3.339413in}}%
\pgfpathlineto{\pgfqpoint{3.643401in}{3.360588in}}%
\pgfpathlineto{\pgfqpoint{3.630122in}{3.371864in}}%
\pgfpathlineto{\pgfqpoint{3.616843in}{3.383249in}}%
\pgfpathlineto{\pgfqpoint{3.603566in}{3.394744in}}%
\pgfpathlineto{\pgfqpoint{3.590290in}{3.406349in}}%
\pgfpathlineto{\pgfqpoint{3.582559in}{3.384786in}}%
\pgfpathlineto{\pgfqpoint{3.574823in}{3.363570in}}%
\pgfpathlineto{\pgfqpoint{3.567082in}{3.342695in}}%
\pgfpathlineto{\pgfqpoint{3.559337in}{3.322154in}}%
\pgfpathclose%
\pgfusepath{fill}%
\end{pgfscope}%
\begin{pgfscope}%
\pgfpathrectangle{\pgfqpoint{1.150000in}{0.150000in}}{\pgfqpoint{5.700000in}{5.700000in}}%
\pgfusepath{clip}%
\pgfsetbuttcap%
\pgfsetroundjoin%
\definecolor{currentfill}{rgb}{0.143303,0.669459,0.511215}%
\pgfsetfillcolor{currentfill}%
\pgfsetfillopacity{0.700000}%
\pgfsetlinewidth{0.000000pt}%
\definecolor{currentstroke}{rgb}{0.000000,0.000000,0.000000}%
\pgfsetstrokecolor{currentstroke}%
\pgfsetdash{}{0pt}%
\pgfpathmoveto{\pgfqpoint{3.774692in}{4.049034in}}%
\pgfpathlineto{\pgfqpoint{3.787997in}{4.034663in}}%
\pgfpathlineto{\pgfqpoint{3.801302in}{4.020406in}}%
\pgfpathlineto{\pgfqpoint{3.814608in}{4.006261in}}%
\pgfpathlineto{\pgfqpoint{3.827913in}{3.992227in}}%
\pgfpathlineto{\pgfqpoint{3.835573in}{4.024642in}}%
\pgfpathlineto{\pgfqpoint{3.843232in}{4.057629in}}%
\pgfpathlineto{\pgfqpoint{3.850890in}{4.091198in}}%
\pgfpathlineto{\pgfqpoint{3.858548in}{4.125360in}}%
\pgfpathlineto{\pgfqpoint{3.845233in}{4.139938in}}%
\pgfpathlineto{\pgfqpoint{3.831918in}{4.154628in}}%
\pgfpathlineto{\pgfqpoint{3.818603in}{4.169431in}}%
\pgfpathlineto{\pgfqpoint{3.805288in}{4.184347in}}%
\pgfpathlineto{\pgfqpoint{3.797640in}{4.149630in}}%
\pgfpathlineto{\pgfqpoint{3.789992in}{4.115513in}}%
\pgfpathlineto{\pgfqpoint{3.782343in}{4.081984in}}%
\pgfpathlineto{\pgfqpoint{3.774692in}{4.049034in}}%
\pgfpathclose%
\pgfusepath{fill}%
\end{pgfscope}%
\begin{pgfscope}%
\pgfpathrectangle{\pgfqpoint{1.150000in}{0.150000in}}{\pgfqpoint{5.700000in}{5.700000in}}%
\pgfusepath{clip}%
\pgfsetbuttcap%
\pgfsetroundjoin%
\definecolor{currentfill}{rgb}{0.281477,0.755203,0.432552}%
\pgfsetfillcolor{currentfill}%
\pgfsetfillopacity{0.700000}%
\pgfsetlinewidth{0.000000pt}%
\definecolor{currentstroke}{rgb}{0.000000,0.000000,0.000000}%
\pgfsetstrokecolor{currentstroke}%
\pgfsetdash{}{0pt}%
\pgfpathmoveto{\pgfqpoint{3.561721in}{4.295111in}}%
\pgfpathlineto{\pgfqpoint{3.575041in}{4.278800in}}%
\pgfpathlineto{\pgfqpoint{3.588358in}{4.262618in}}%
\pgfpathlineto{\pgfqpoint{3.601674in}{4.246564in}}%
\pgfpathlineto{\pgfqpoint{3.614989in}{4.230638in}}%
\pgfpathlineto{\pgfqpoint{3.622607in}{4.265209in}}%
\pgfpathlineto{\pgfqpoint{3.630223in}{4.300378in}}%
\pgfpathlineto{\pgfqpoint{3.637836in}{4.336155in}}%
\pgfpathlineto{\pgfqpoint{3.645447in}{4.372553in}}%
\pgfpathlineto{\pgfqpoint{3.632120in}{4.389034in}}%
\pgfpathlineto{\pgfqpoint{3.618791in}{4.405644in}}%
\pgfpathlineto{\pgfqpoint{3.605460in}{4.422382in}}%
\pgfpathlineto{\pgfqpoint{3.592127in}{4.439250in}}%
\pgfpathlineto{\pgfqpoint{3.584530in}{4.402286in}}%
\pgfpathlineto{\pgfqpoint{3.576930in}{4.365949in}}%
\pgfpathlineto{\pgfqpoint{3.569327in}{4.330228in}}%
\pgfpathlineto{\pgfqpoint{3.561721in}{4.295111in}}%
\pgfpathclose%
\pgfusepath{fill}%
\end{pgfscope}%
\begin{pgfscope}%
\pgfpathrectangle{\pgfqpoint{1.150000in}{0.150000in}}{\pgfqpoint{5.700000in}{5.700000in}}%
\pgfusepath{clip}%
\pgfsetbuttcap%
\pgfsetroundjoin%
\definecolor{currentfill}{rgb}{0.188923,0.410910,0.556326}%
\pgfsetfillcolor{currentfill}%
\pgfsetfillopacity{0.700000}%
\pgfsetlinewidth{0.000000pt}%
\definecolor{currentstroke}{rgb}{0.000000,0.000000,0.000000}%
\pgfsetstrokecolor{currentstroke}%
\pgfsetdash{}{0pt}%
\pgfpathmoveto{\pgfqpoint{3.147385in}{3.379559in}}%
\pgfpathlineto{\pgfqpoint{3.160663in}{3.366431in}}%
\pgfpathlineto{\pgfqpoint{3.173941in}{3.353437in}}%
\pgfpathlineto{\pgfqpoint{3.187218in}{3.340576in}}%
\pgfpathlineto{\pgfqpoint{3.200493in}{3.327847in}}%
\pgfpathlineto{\pgfqpoint{3.208313in}{3.346459in}}%
\pgfpathlineto{\pgfqpoint{3.216126in}{3.365355in}}%
\pgfpathlineto{\pgfqpoint{3.223933in}{3.384538in}}%
\pgfpathlineto{\pgfqpoint{3.231732in}{3.404017in}}%
\pgfpathlineto{\pgfqpoint{3.218457in}{3.417068in}}%
\pgfpathlineto{\pgfqpoint{3.205181in}{3.430251in}}%
\pgfpathlineto{\pgfqpoint{3.191904in}{3.443567in}}%
\pgfpathlineto{\pgfqpoint{3.178625in}{3.457017in}}%
\pgfpathlineto{\pgfqpoint{3.170826in}{3.437208in}}%
\pgfpathlineto{\pgfqpoint{3.163020in}{3.417700in}}%
\pgfpathlineto{\pgfqpoint{3.155206in}{3.398485in}}%
\pgfpathlineto{\pgfqpoint{3.147385in}{3.379559in}}%
\pgfpathclose%
\pgfusepath{fill}%
\end{pgfscope}%
\begin{pgfscope}%
\pgfpathrectangle{\pgfqpoint{1.150000in}{0.150000in}}{\pgfqpoint{5.700000in}{5.700000in}}%
\pgfusepath{clip}%
\pgfsetbuttcap%
\pgfsetroundjoin%
\definecolor{currentfill}{rgb}{0.128729,0.563265,0.551229}%
\pgfsetfillcolor{currentfill}%
\pgfsetfillopacity{0.700000}%
\pgfsetlinewidth{0.000000pt}%
\definecolor{currentstroke}{rgb}{0.000000,0.000000,0.000000}%
\pgfsetstrokecolor{currentstroke}%
\pgfsetdash{}{0pt}%
\pgfpathmoveto{\pgfqpoint{3.903661in}{3.763844in}}%
\pgfpathlineto{\pgfqpoint{3.916965in}{3.751290in}}%
\pgfpathlineto{\pgfqpoint{3.930269in}{3.738839in}}%
\pgfpathlineto{\pgfqpoint{3.943575in}{3.726489in}}%
\pgfpathlineto{\pgfqpoint{3.956883in}{3.714242in}}%
\pgfpathlineto{\pgfqpoint{3.964566in}{3.742954in}}%
\pgfpathlineto{\pgfqpoint{3.972248in}{3.772179in}}%
\pgfpathlineto{\pgfqpoint{3.979931in}{3.801926in}}%
\pgfpathlineto{\pgfqpoint{3.987613in}{3.832206in}}%
\pgfpathlineto{\pgfqpoint{3.974300in}{3.844966in}}%
\pgfpathlineto{\pgfqpoint{3.960989in}{3.857829in}}%
\pgfpathlineto{\pgfqpoint{3.947678in}{3.870794in}}%
\pgfpathlineto{\pgfqpoint{3.934368in}{3.883863in}}%
\pgfpathlineto{\pgfqpoint{3.926692in}{3.853061in}}%
\pgfpathlineto{\pgfqpoint{3.919016in}{3.822797in}}%
\pgfpathlineto{\pgfqpoint{3.911339in}{3.793061in}}%
\pgfpathlineto{\pgfqpoint{3.903661in}{3.763844in}}%
\pgfpathclose%
\pgfusepath{fill}%
\end{pgfscope}%
\begin{pgfscope}%
\pgfpathrectangle{\pgfqpoint{1.150000in}{0.150000in}}{\pgfqpoint{5.700000in}{5.700000in}}%
\pgfusepath{clip}%
\pgfsetbuttcap%
\pgfsetroundjoin%
\definecolor{currentfill}{rgb}{0.128087,0.647749,0.523491}%
\pgfsetfillcolor{currentfill}%
\pgfsetfillopacity{0.700000}%
\pgfsetlinewidth{0.000000pt}%
\definecolor{currentstroke}{rgb}{0.000000,0.000000,0.000000}%
\pgfsetstrokecolor{currentstroke}%
\pgfsetdash{}{0pt}%
\pgfpathmoveto{\pgfqpoint{3.827913in}{3.992227in}}%
\pgfpathlineto{\pgfqpoint{3.841218in}{3.978303in}}%
\pgfpathlineto{\pgfqpoint{3.854524in}{3.964489in}}%
\pgfpathlineto{\pgfqpoint{3.867830in}{3.950784in}}%
\pgfpathlineto{\pgfqpoint{3.881136in}{3.937187in}}%
\pgfpathlineto{\pgfqpoint{3.888805in}{3.969069in}}%
\pgfpathlineto{\pgfqpoint{3.896472in}{4.001517in}}%
\pgfpathlineto{\pgfqpoint{3.904140in}{4.034540in}}%
\pgfpathlineto{\pgfqpoint{3.911808in}{4.068150in}}%
\pgfpathlineto{\pgfqpoint{3.898493in}{4.082289in}}%
\pgfpathlineto{\pgfqpoint{3.885178in}{4.096537in}}%
\pgfpathlineto{\pgfqpoint{3.871863in}{4.110893in}}%
\pgfpathlineto{\pgfqpoint{3.858548in}{4.125360in}}%
\pgfpathlineto{\pgfqpoint{3.850890in}{4.091198in}}%
\pgfpathlineto{\pgfqpoint{3.843232in}{4.057629in}}%
\pgfpathlineto{\pgfqpoint{3.835573in}{4.024642in}}%
\pgfpathlineto{\pgfqpoint{3.827913in}{3.992227in}}%
\pgfpathclose%
\pgfusepath{fill}%
\end{pgfscope}%
\begin{pgfscope}%
\pgfpathrectangle{\pgfqpoint{1.150000in}{0.150000in}}{\pgfqpoint{5.700000in}{5.700000in}}%
\pgfusepath{clip}%
\pgfsetbuttcap%
\pgfsetroundjoin%
\definecolor{currentfill}{rgb}{0.335885,0.777018,0.402049}%
\pgfsetfillcolor{currentfill}%
\pgfsetfillopacity{0.700000}%
\pgfsetlinewidth{0.000000pt}%
\definecolor{currentstroke}{rgb}{0.000000,0.000000,0.000000}%
\pgfsetstrokecolor{currentstroke}%
\pgfsetdash{}{0pt}%
\pgfpathmoveto{\pgfqpoint{3.508426in}{4.361670in}}%
\pgfpathlineto{\pgfqpoint{3.521753in}{4.344830in}}%
\pgfpathlineto{\pgfqpoint{3.535078in}{4.328125in}}%
\pgfpathlineto{\pgfqpoint{3.548400in}{4.311552in}}%
\pgfpathlineto{\pgfqpoint{3.561721in}{4.295111in}}%
\pgfpathlineto{\pgfqpoint{3.569327in}{4.330228in}}%
\pgfpathlineto{\pgfqpoint{3.576930in}{4.365949in}}%
\pgfpathlineto{\pgfqpoint{3.584530in}{4.402286in}}%
\pgfpathlineto{\pgfqpoint{3.592127in}{4.439250in}}%
\pgfpathlineto{\pgfqpoint{3.578793in}{4.456249in}}%
\pgfpathlineto{\pgfqpoint{3.565456in}{4.473380in}}%
\pgfpathlineto{\pgfqpoint{3.552118in}{4.490645in}}%
\pgfpathlineto{\pgfqpoint{3.538777in}{4.508045in}}%
\pgfpathlineto{\pgfqpoint{3.531194in}{4.470512in}}%
\pgfpathlineto{\pgfqpoint{3.523608in}{4.433613in}}%
\pgfpathlineto{\pgfqpoint{3.516019in}{4.397336in}}%
\pgfpathlineto{\pgfqpoint{3.508426in}{4.361670in}}%
\pgfpathclose%
\pgfusepath{fill}%
\end{pgfscope}%
\begin{pgfscope}%
\pgfpathrectangle{\pgfqpoint{1.150000in}{0.150000in}}{\pgfqpoint{5.700000in}{5.700000in}}%
\pgfusepath{clip}%
\pgfsetbuttcap%
\pgfsetroundjoin%
\definecolor{currentfill}{rgb}{0.201239,0.383670,0.554294}%
\pgfsetfillcolor{currentfill}%
\pgfsetfillopacity{0.700000}%
\pgfsetlinewidth{0.000000pt}%
\definecolor{currentstroke}{rgb}{0.000000,0.000000,0.000000}%
\pgfsetstrokecolor{currentstroke}%
\pgfsetdash{}{0pt}%
\pgfpathmoveto{\pgfqpoint{3.337909in}{3.304219in}}%
\pgfpathlineto{\pgfqpoint{3.351179in}{3.292305in}}%
\pgfpathlineto{\pgfqpoint{3.364450in}{3.280511in}}%
\pgfpathlineto{\pgfqpoint{3.377722in}{3.268837in}}%
\pgfpathlineto{\pgfqpoint{3.390993in}{3.257282in}}%
\pgfpathlineto{\pgfqpoint{3.398784in}{3.276061in}}%
\pgfpathlineto{\pgfqpoint{3.406569in}{3.295131in}}%
\pgfpathlineto{\pgfqpoint{3.414347in}{3.314498in}}%
\pgfpathlineto{\pgfqpoint{3.422120in}{3.334169in}}%
\pgfpathlineto{\pgfqpoint{3.408849in}{3.346065in}}%
\pgfpathlineto{\pgfqpoint{3.395579in}{3.358079in}}%
\pgfpathlineto{\pgfqpoint{3.382308in}{3.370214in}}%
\pgfpathlineto{\pgfqpoint{3.369038in}{3.382469in}}%
\pgfpathlineto{\pgfqpoint{3.361265in}{3.362450in}}%
\pgfpathlineto{\pgfqpoint{3.353486in}{3.342739in}}%
\pgfpathlineto{\pgfqpoint{3.345700in}{3.323331in}}%
\pgfpathlineto{\pgfqpoint{3.337909in}{3.304219in}}%
\pgfpathclose%
\pgfusepath{fill}%
\end{pgfscope}%
\begin{pgfscope}%
\pgfpathrectangle{\pgfqpoint{1.150000in}{0.150000in}}{\pgfqpoint{5.700000in}{5.700000in}}%
\pgfusepath{clip}%
\pgfsetbuttcap%
\pgfsetroundjoin%
\definecolor{currentfill}{rgb}{0.150476,0.504369,0.557430}%
\pgfsetfillcolor{currentfill}%
\pgfsetfillopacity{0.700000}%
\pgfsetlinewidth{0.000000pt}%
\definecolor{currentstroke}{rgb}{0.000000,0.000000,0.000000}%
\pgfsetstrokecolor{currentstroke}%
\pgfsetdash{}{0pt}%
\pgfpathmoveto{\pgfqpoint{3.926143in}{3.604322in}}%
\pgfpathlineto{\pgfqpoint{3.939447in}{3.592664in}}%
\pgfpathlineto{\pgfqpoint{3.952753in}{3.581106in}}%
\pgfpathlineto{\pgfqpoint{3.966061in}{3.569647in}}%
\pgfpathlineto{\pgfqpoint{3.979371in}{3.558287in}}%
\pgfpathlineto{\pgfqpoint{3.987061in}{3.584566in}}%
\pgfpathlineto{\pgfqpoint{3.994749in}{3.611314in}}%
\pgfpathlineto{\pgfqpoint{4.002438in}{3.638540in}}%
\pgfpathlineto{\pgfqpoint{4.010125in}{3.666253in}}%
\pgfpathlineto{\pgfqpoint{3.996812in}{3.678101in}}%
\pgfpathlineto{\pgfqpoint{3.983501in}{3.690048in}}%
\pgfpathlineto{\pgfqpoint{3.970191in}{3.702095in}}%
\pgfpathlineto{\pgfqpoint{3.956883in}{3.714242in}}%
\pgfpathlineto{\pgfqpoint{3.949199in}{3.686032in}}%
\pgfpathlineto{\pgfqpoint{3.941514in}{3.658315in}}%
\pgfpathlineto{\pgfqpoint{3.933829in}{3.631082in}}%
\pgfpathlineto{\pgfqpoint{3.926143in}{3.604322in}}%
\pgfpathclose%
\pgfusepath{fill}%
\end{pgfscope}%
\begin{pgfscope}%
\pgfpathrectangle{\pgfqpoint{1.150000in}{0.150000in}}{\pgfqpoint{5.700000in}{5.700000in}}%
\pgfusepath{clip}%
\pgfsetbuttcap%
\pgfsetroundjoin%
\definecolor{currentfill}{rgb}{0.204903,0.375746,0.553533}%
\pgfsetfillcolor{currentfill}%
\pgfsetfillopacity{0.700000}%
\pgfsetlinewidth{0.000000pt}%
\definecolor{currentstroke}{rgb}{0.000000,0.000000,0.000000}%
\pgfsetstrokecolor{currentstroke}%
\pgfsetdash{}{0pt}%
\pgfpathmoveto{\pgfqpoint{3.475207in}{3.287761in}}%
\pgfpathlineto{\pgfqpoint{3.488480in}{3.276448in}}%
\pgfpathlineto{\pgfqpoint{3.501754in}{3.265248in}}%
\pgfpathlineto{\pgfqpoint{3.515029in}{3.254161in}}%
\pgfpathlineto{\pgfqpoint{3.528305in}{3.243186in}}%
\pgfpathlineto{\pgfqpoint{3.536070in}{3.262461in}}%
\pgfpathlineto{\pgfqpoint{3.543831in}{3.282044in}}%
\pgfpathlineto{\pgfqpoint{3.551587in}{3.301939in}}%
\pgfpathlineto{\pgfqpoint{3.559337in}{3.322154in}}%
\pgfpathlineto{\pgfqpoint{3.546062in}{3.333489in}}%
\pgfpathlineto{\pgfqpoint{3.532788in}{3.344936in}}%
\pgfpathlineto{\pgfqpoint{3.519514in}{3.356497in}}%
\pgfpathlineto{\pgfqpoint{3.506241in}{3.368171in}}%
\pgfpathlineto{\pgfqpoint{3.498491in}{3.347588in}}%
\pgfpathlineto{\pgfqpoint{3.490735in}{3.327329in}}%
\pgfpathlineto{\pgfqpoint{3.482974in}{3.307390in}}%
\pgfpathlineto{\pgfqpoint{3.475207in}{3.287761in}}%
\pgfpathclose%
\pgfusepath{fill}%
\end{pgfscope}%
\begin{pgfscope}%
\pgfpathrectangle{\pgfqpoint{1.150000in}{0.150000in}}{\pgfqpoint{5.700000in}{5.700000in}}%
\pgfusepath{clip}%
\pgfsetbuttcap%
\pgfsetroundjoin%
\definecolor{currentfill}{rgb}{0.166617,0.463708,0.558119}%
\pgfsetfillcolor{currentfill}%
\pgfsetfillopacity{0.700000}%
\pgfsetlinewidth{0.000000pt}%
\definecolor{currentstroke}{rgb}{0.000000,0.000000,0.000000}%
\pgfsetstrokecolor{currentstroke}%
\pgfsetdash{}{0pt}%
\pgfpathmoveto{\pgfqpoint{3.895381in}{3.501843in}}%
\pgfpathlineto{\pgfqpoint{3.908683in}{3.490651in}}%
\pgfpathlineto{\pgfqpoint{3.921987in}{3.479558in}}%
\pgfpathlineto{\pgfqpoint{3.935293in}{3.468564in}}%
\pgfpathlineto{\pgfqpoint{3.948600in}{3.457669in}}%
\pgfpathlineto{\pgfqpoint{3.956295in}{3.482166in}}%
\pgfpathlineto{\pgfqpoint{3.963988in}{3.507096in}}%
\pgfpathlineto{\pgfqpoint{3.971680in}{3.532466in}}%
\pgfpathlineto{\pgfqpoint{3.979371in}{3.558287in}}%
\pgfpathlineto{\pgfqpoint{3.966061in}{3.569647in}}%
\pgfpathlineto{\pgfqpoint{3.952753in}{3.581106in}}%
\pgfpathlineto{\pgfqpoint{3.939447in}{3.592664in}}%
\pgfpathlineto{\pgfqpoint{3.926143in}{3.604322in}}%
\pgfpathlineto{\pgfqpoint{3.918455in}{3.578028in}}%
\pgfpathlineto{\pgfqpoint{3.910765in}{3.552189in}}%
\pgfpathlineto{\pgfqpoint{3.903074in}{3.526797in}}%
\pgfpathlineto{\pgfqpoint{3.895381in}{3.501843in}}%
\pgfpathclose%
\pgfusepath{fill}%
\end{pgfscope}%
\begin{pgfscope}%
\pgfpathrectangle{\pgfqpoint{1.150000in}{0.150000in}}{\pgfqpoint{5.700000in}{5.700000in}}%
\pgfusepath{clip}%
\pgfsetbuttcap%
\pgfsetroundjoin%
\definecolor{currentfill}{rgb}{0.121380,0.629492,0.531973}%
\pgfsetfillcolor{currentfill}%
\pgfsetfillopacity{0.700000}%
\pgfsetlinewidth{0.000000pt}%
\definecolor{currentstroke}{rgb}{0.000000,0.000000,0.000000}%
\pgfsetstrokecolor{currentstroke}%
\pgfsetdash{}{0pt}%
\pgfpathmoveto{\pgfqpoint{3.881136in}{3.937187in}}%
\pgfpathlineto{\pgfqpoint{3.894443in}{3.923697in}}%
\pgfpathlineto{\pgfqpoint{3.907751in}{3.910314in}}%
\pgfpathlineto{\pgfqpoint{3.921059in}{3.897036in}}%
\pgfpathlineto{\pgfqpoint{3.934368in}{3.883863in}}%
\pgfpathlineto{\pgfqpoint{3.942044in}{3.915214in}}%
\pgfpathlineto{\pgfqpoint{3.949720in}{3.947124in}}%
\pgfpathlineto{\pgfqpoint{3.957396in}{3.979604in}}%
\pgfpathlineto{\pgfqpoint{3.965073in}{4.012665in}}%
\pgfpathlineto{\pgfqpoint{3.951756in}{4.026377in}}%
\pgfpathlineto{\pgfqpoint{3.938439in}{4.040195in}}%
\pgfpathlineto{\pgfqpoint{3.925123in}{4.054119in}}%
\pgfpathlineto{\pgfqpoint{3.911808in}{4.068150in}}%
\pgfpathlineto{\pgfqpoint{3.904140in}{4.034540in}}%
\pgfpathlineto{\pgfqpoint{3.896472in}{4.001517in}}%
\pgfpathlineto{\pgfqpoint{3.888805in}{3.969069in}}%
\pgfpathlineto{\pgfqpoint{3.881136in}{3.937187in}}%
\pgfpathclose%
\pgfusepath{fill}%
\end{pgfscope}%
\begin{pgfscope}%
\pgfpathrectangle{\pgfqpoint{1.150000in}{0.150000in}}{\pgfqpoint{5.700000in}{5.700000in}}%
\pgfusepath{clip}%
\pgfsetbuttcap%
\pgfsetroundjoin%
\definecolor{currentfill}{rgb}{0.197636,0.391528,0.554969}%
\pgfsetfillcolor{currentfill}%
\pgfsetfillopacity{0.700000}%
\pgfsetlinewidth{0.000000pt}%
\definecolor{currentstroke}{rgb}{0.000000,0.000000,0.000000}%
\pgfsetstrokecolor{currentstroke}%
\pgfsetdash{}{0pt}%
\pgfpathmoveto{\pgfqpoint{3.200493in}{3.327847in}}%
\pgfpathlineto{\pgfqpoint{3.213768in}{3.315248in}}%
\pgfpathlineto{\pgfqpoint{3.227042in}{3.302779in}}%
\pgfpathlineto{\pgfqpoint{3.240316in}{3.290437in}}%
\pgfpathlineto{\pgfqpoint{3.253589in}{3.278223in}}%
\pgfpathlineto{\pgfqpoint{3.261407in}{3.296522in}}%
\pgfpathlineto{\pgfqpoint{3.269220in}{3.315099in}}%
\pgfpathlineto{\pgfqpoint{3.277025in}{3.333959in}}%
\pgfpathlineto{\pgfqpoint{3.284824in}{3.353109in}}%
\pgfpathlineto{\pgfqpoint{3.271552in}{3.365644in}}%
\pgfpathlineto{\pgfqpoint{3.258279in}{3.378306in}}%
\pgfpathlineto{\pgfqpoint{3.245006in}{3.391097in}}%
\pgfpathlineto{\pgfqpoint{3.231732in}{3.404017in}}%
\pgfpathlineto{\pgfqpoint{3.223933in}{3.384538in}}%
\pgfpathlineto{\pgfqpoint{3.216126in}{3.365355in}}%
\pgfpathlineto{\pgfqpoint{3.208313in}{3.346459in}}%
\pgfpathlineto{\pgfqpoint{3.200493in}{3.327847in}}%
\pgfpathclose%
\pgfusepath{fill}%
\end{pgfscope}%
\begin{pgfscope}%
\pgfpathrectangle{\pgfqpoint{1.150000in}{0.150000in}}{\pgfqpoint{5.700000in}{5.700000in}}%
\pgfusepath{clip}%
\pgfsetbuttcap%
\pgfsetroundjoin%
\definecolor{currentfill}{rgb}{0.190631,0.407061,0.556089}%
\pgfsetfillcolor{currentfill}%
\pgfsetfillopacity{0.700000}%
\pgfsetlinewidth{0.000000pt}%
\definecolor{currentstroke}{rgb}{0.000000,0.000000,0.000000}%
\pgfsetstrokecolor{currentstroke}%
\pgfsetdash{}{0pt}%
\pgfpathmoveto{\pgfqpoint{3.780572in}{3.359267in}}%
\pgfpathlineto{\pgfqpoint{3.793865in}{3.348537in}}%
\pgfpathlineto{\pgfqpoint{3.807159in}{3.337908in}}%
\pgfpathlineto{\pgfqpoint{3.820456in}{3.327380in}}%
\pgfpathlineto{\pgfqpoint{3.833754in}{3.316953in}}%
\pgfpathlineto{\pgfqpoint{3.841467in}{3.338708in}}%
\pgfpathlineto{\pgfqpoint{3.849177in}{3.360834in}}%
\pgfpathlineto{\pgfqpoint{3.856884in}{3.383340in}}%
\pgfpathlineto{\pgfqpoint{3.864588in}{3.406233in}}%
\pgfpathlineto{\pgfqpoint{3.851290in}{3.417082in}}%
\pgfpathlineto{\pgfqpoint{3.837993in}{3.428031in}}%
\pgfpathlineto{\pgfqpoint{3.824697in}{3.439082in}}%
\pgfpathlineto{\pgfqpoint{3.811404in}{3.450234in}}%
\pgfpathlineto{\pgfqpoint{3.803701in}{3.426911in}}%
\pgfpathlineto{\pgfqpoint{3.795994in}{3.403981in}}%
\pgfpathlineto{\pgfqpoint{3.788285in}{3.381436in}}%
\pgfpathlineto{\pgfqpoint{3.780572in}{3.359267in}}%
\pgfpathclose%
\pgfusepath{fill}%
\end{pgfscope}%
\begin{pgfscope}%
\pgfpathrectangle{\pgfqpoint{1.150000in}{0.150000in}}{\pgfqpoint{5.700000in}{5.700000in}}%
\pgfusepath{clip}%
\pgfsetbuttcap%
\pgfsetroundjoin%
\definecolor{currentfill}{rgb}{0.395174,0.797475,0.367757}%
\pgfsetfillcolor{currentfill}%
\pgfsetfillopacity{0.700000}%
\pgfsetlinewidth{0.000000pt}%
\definecolor{currentstroke}{rgb}{0.000000,0.000000,0.000000}%
\pgfsetstrokecolor{currentstroke}%
\pgfsetdash{}{0pt}%
\pgfpathmoveto{\pgfqpoint{3.455096in}{4.430392in}}%
\pgfpathlineto{\pgfqpoint{3.468432in}{4.413004in}}%
\pgfpathlineto{\pgfqpoint{3.481766in}{4.395756in}}%
\pgfpathlineto{\pgfqpoint{3.495097in}{4.378644in}}%
\pgfpathlineto{\pgfqpoint{3.508426in}{4.361670in}}%
\pgfpathlineto{\pgfqpoint{3.516019in}{4.397336in}}%
\pgfpathlineto{\pgfqpoint{3.523608in}{4.433613in}}%
\pgfpathlineto{\pgfqpoint{3.531194in}{4.470512in}}%
\pgfpathlineto{\pgfqpoint{3.538777in}{4.508045in}}%
\pgfpathlineto{\pgfqpoint{3.525433in}{4.525581in}}%
\pgfpathlineto{\pgfqpoint{3.512088in}{4.543254in}}%
\pgfpathlineto{\pgfqpoint{3.498740in}{4.561065in}}%
\pgfpathlineto{\pgfqpoint{3.485389in}{4.579016in}}%
\pgfpathlineto{\pgfqpoint{3.477821in}{4.540911in}}%
\pgfpathlineto{\pgfqpoint{3.470250in}{4.503446in}}%
\pgfpathlineto{\pgfqpoint{3.462675in}{4.466610in}}%
\pgfpathlineto{\pgfqpoint{3.455096in}{4.430392in}}%
\pgfpathclose%
\pgfusepath{fill}%
\end{pgfscope}%
\begin{pgfscope}%
\pgfpathrectangle{\pgfqpoint{1.150000in}{0.150000in}}{\pgfqpoint{5.700000in}{5.700000in}}%
\pgfusepath{clip}%
\pgfsetbuttcap%
\pgfsetroundjoin%
\definecolor{currentfill}{rgb}{0.197636,0.391528,0.554969}%
\pgfsetfillcolor{currentfill}%
\pgfsetfillopacity{0.700000}%
\pgfsetlinewidth{0.000000pt}%
\definecolor{currentstroke}{rgb}{0.000000,0.000000,0.000000}%
\pgfsetstrokecolor{currentstroke}%
\pgfsetdash{}{0pt}%
\pgfpathmoveto{\pgfqpoint{3.696531in}{3.316557in}}%
\pgfpathlineto{\pgfqpoint{3.709817in}{3.305814in}}%
\pgfpathlineto{\pgfqpoint{3.723105in}{3.295175in}}%
\pgfpathlineto{\pgfqpoint{3.736394in}{3.284640in}}%
\pgfpathlineto{\pgfqpoint{3.749686in}{3.274208in}}%
\pgfpathlineto{\pgfqpoint{3.757413in}{3.294946in}}%
\pgfpathlineto{\pgfqpoint{3.765136in}{3.316030in}}%
\pgfpathlineto{\pgfqpoint{3.772856in}{3.337468in}}%
\pgfpathlineto{\pgfqpoint{3.780572in}{3.359267in}}%
\pgfpathlineto{\pgfqpoint{3.767281in}{3.370100in}}%
\pgfpathlineto{\pgfqpoint{3.753991in}{3.381036in}}%
\pgfpathlineto{\pgfqpoint{3.740703in}{3.392076in}}%
\pgfpathlineto{\pgfqpoint{3.727417in}{3.403220in}}%
\pgfpathlineto{\pgfqpoint{3.719701in}{3.381011in}}%
\pgfpathlineto{\pgfqpoint{3.711982in}{3.359170in}}%
\pgfpathlineto{\pgfqpoint{3.704259in}{3.337688in}}%
\pgfpathlineto{\pgfqpoint{3.696531in}{3.316557in}}%
\pgfpathclose%
\pgfusepath{fill}%
\end{pgfscope}%
\begin{pgfscope}%
\pgfpathrectangle{\pgfqpoint{1.150000in}{0.150000in}}{\pgfqpoint{5.700000in}{5.700000in}}%
\pgfusepath{clip}%
\pgfsetbuttcap%
\pgfsetroundjoin%
\definecolor{currentfill}{rgb}{0.135066,0.544853,0.554029}%
\pgfsetfillcolor{currentfill}%
\pgfsetfillopacity{0.700000}%
\pgfsetlinewidth{0.000000pt}%
\definecolor{currentstroke}{rgb}{0.000000,0.000000,0.000000}%
\pgfsetstrokecolor{currentstroke}%
\pgfsetdash{}{0pt}%
\pgfpathmoveto{\pgfqpoint{3.956883in}{3.714242in}}%
\pgfpathlineto{\pgfqpoint{3.970191in}{3.702095in}}%
\pgfpathlineto{\pgfqpoint{3.983501in}{3.690048in}}%
\pgfpathlineto{\pgfqpoint{3.996812in}{3.678101in}}%
\pgfpathlineto{\pgfqpoint{4.010125in}{3.666253in}}%
\pgfpathlineto{\pgfqpoint{4.017813in}{3.694462in}}%
\pgfpathlineto{\pgfqpoint{4.025501in}{3.723178in}}%
\pgfpathlineto{\pgfqpoint{4.033188in}{3.752410in}}%
\pgfpathlineto{\pgfqpoint{4.040876in}{3.782169in}}%
\pgfpathlineto{\pgfqpoint{4.027559in}{3.794528in}}%
\pgfpathlineto{\pgfqpoint{4.014242in}{3.806987in}}%
\pgfpathlineto{\pgfqpoint{4.000927in}{3.819546in}}%
\pgfpathlineto{\pgfqpoint{3.987613in}{3.832206in}}%
\pgfpathlineto{\pgfqpoint{3.979931in}{3.801926in}}%
\pgfpathlineto{\pgfqpoint{3.972248in}{3.772179in}}%
\pgfpathlineto{\pgfqpoint{3.964566in}{3.742954in}}%
\pgfpathlineto{\pgfqpoint{3.956883in}{3.714242in}}%
\pgfpathclose%
\pgfusepath{fill}%
\end{pgfscope}%
\begin{pgfscope}%
\pgfpathrectangle{\pgfqpoint{1.150000in}{0.150000in}}{\pgfqpoint{5.700000in}{5.700000in}}%
\pgfusepath{clip}%
\pgfsetbuttcap%
\pgfsetroundjoin%
\definecolor{currentfill}{rgb}{0.182256,0.426184,0.557120}%
\pgfsetfillcolor{currentfill}%
\pgfsetfillopacity{0.700000}%
\pgfsetlinewidth{0.000000pt}%
\definecolor{currentstroke}{rgb}{0.000000,0.000000,0.000000}%
\pgfsetstrokecolor{currentstroke}%
\pgfsetdash{}{0pt}%
\pgfpathmoveto{\pgfqpoint{3.864588in}{3.406233in}}%
\pgfpathlineto{\pgfqpoint{3.877889in}{3.395484in}}%
\pgfpathlineto{\pgfqpoint{3.891192in}{3.384834in}}%
\pgfpathlineto{\pgfqpoint{3.904496in}{3.374283in}}%
\pgfpathlineto{\pgfqpoint{3.917803in}{3.363830in}}%
\pgfpathlineto{\pgfqpoint{3.925505in}{3.386684in}}%
\pgfpathlineto{\pgfqpoint{3.933206in}{3.409936in}}%
\pgfpathlineto{\pgfqpoint{3.940904in}{3.433595in}}%
\pgfpathlineto{\pgfqpoint{3.948600in}{3.457669in}}%
\pgfpathlineto{\pgfqpoint{3.935293in}{3.468564in}}%
\pgfpathlineto{\pgfqpoint{3.921987in}{3.479558in}}%
\pgfpathlineto{\pgfqpoint{3.908683in}{3.490651in}}%
\pgfpathlineto{\pgfqpoint{3.895381in}{3.501843in}}%
\pgfpathlineto{\pgfqpoint{3.887687in}{3.477318in}}%
\pgfpathlineto{\pgfqpoint{3.879990in}{3.453214in}}%
\pgfpathlineto{\pgfqpoint{3.872290in}{3.429522in}}%
\pgfpathlineto{\pgfqpoint{3.864588in}{3.406233in}}%
\pgfpathclose%
\pgfusepath{fill}%
\end{pgfscope}%
\begin{pgfscope}%
\pgfpathrectangle{\pgfqpoint{1.150000in}{0.150000in}}{\pgfqpoint{5.700000in}{5.700000in}}%
\pgfusepath{clip}%
\pgfsetbuttcap%
\pgfsetroundjoin%
\definecolor{currentfill}{rgb}{0.204903,0.375746,0.553533}%
\pgfsetfillcolor{currentfill}%
\pgfsetfillopacity{0.700000}%
\pgfsetlinewidth{0.000000pt}%
\definecolor{currentstroke}{rgb}{0.000000,0.000000,0.000000}%
\pgfsetstrokecolor{currentstroke}%
\pgfsetdash{}{0pt}%
\pgfpathmoveto{\pgfqpoint{3.612448in}{3.277916in}}%
\pgfpathlineto{\pgfqpoint{3.625729in}{3.267129in}}%
\pgfpathlineto{\pgfqpoint{3.639011in}{3.256448in}}%
\pgfpathlineto{\pgfqpoint{3.652295in}{3.245874in}}%
\pgfpathlineto{\pgfqpoint{3.665580in}{3.235406in}}%
\pgfpathlineto{\pgfqpoint{3.673324in}{3.255203in}}%
\pgfpathlineto{\pgfqpoint{3.681064in}{3.275322in}}%
\pgfpathlineto{\pgfqpoint{3.688800in}{3.295771in}}%
\pgfpathlineto{\pgfqpoint{3.696531in}{3.316557in}}%
\pgfpathlineto{\pgfqpoint{3.683247in}{3.327405in}}%
\pgfpathlineto{\pgfqpoint{3.669963in}{3.338359in}}%
\pgfpathlineto{\pgfqpoint{3.656682in}{3.349420in}}%
\pgfpathlineto{\pgfqpoint{3.643401in}{3.360588in}}%
\pgfpathlineto{\pgfqpoint{3.635670in}{3.339413in}}%
\pgfpathlineto{\pgfqpoint{3.627934in}{3.318581in}}%
\pgfpathlineto{\pgfqpoint{3.620193in}{3.298085in}}%
\pgfpathlineto{\pgfqpoint{3.612448in}{3.277916in}}%
\pgfpathclose%
\pgfusepath{fill}%
\end{pgfscope}%
\begin{pgfscope}%
\pgfpathrectangle{\pgfqpoint{1.150000in}{0.150000in}}{\pgfqpoint{5.700000in}{5.700000in}}%
\pgfusepath{clip}%
\pgfsetbuttcap%
\pgfsetroundjoin%
\definecolor{currentfill}{rgb}{0.119423,0.611141,0.538982}%
\pgfsetfillcolor{currentfill}%
\pgfsetfillopacity{0.700000}%
\pgfsetlinewidth{0.000000pt}%
\definecolor{currentstroke}{rgb}{0.000000,0.000000,0.000000}%
\pgfsetstrokecolor{currentstroke}%
\pgfsetdash{}{0pt}%
\pgfpathmoveto{\pgfqpoint{3.934368in}{3.883863in}}%
\pgfpathlineto{\pgfqpoint{3.947678in}{3.870794in}}%
\pgfpathlineto{\pgfqpoint{3.960989in}{3.857829in}}%
\pgfpathlineto{\pgfqpoint{3.974300in}{3.844966in}}%
\pgfpathlineto{\pgfqpoint{3.987613in}{3.832206in}}%
\pgfpathlineto{\pgfqpoint{3.995296in}{3.863028in}}%
\pgfpathlineto{\pgfqpoint{4.002979in}{3.894403in}}%
\pgfpathlineto{\pgfqpoint{4.010663in}{3.926341in}}%
\pgfpathlineto{\pgfqpoint{4.018347in}{3.958854in}}%
\pgfpathlineto{\pgfqpoint{4.005027in}{3.972152in}}%
\pgfpathlineto{\pgfqpoint{3.991708in}{3.985553in}}%
\pgfpathlineto{\pgfqpoint{3.978390in}{3.999057in}}%
\pgfpathlineto{\pgfqpoint{3.965073in}{4.012665in}}%
\pgfpathlineto{\pgfqpoint{3.957396in}{3.979604in}}%
\pgfpathlineto{\pgfqpoint{3.949720in}{3.947124in}}%
\pgfpathlineto{\pgfqpoint{3.942044in}{3.915214in}}%
\pgfpathlineto{\pgfqpoint{3.934368in}{3.883863in}}%
\pgfpathclose%
\pgfusepath{fill}%
\end{pgfscope}%
\begin{pgfscope}%
\pgfpathrectangle{\pgfqpoint{1.150000in}{0.150000in}}{\pgfqpoint{5.700000in}{5.700000in}}%
\pgfusepath{clip}%
\pgfsetbuttcap%
\pgfsetroundjoin%
\definecolor{currentfill}{rgb}{0.210503,0.363727,0.552206}%
\pgfsetfillcolor{currentfill}%
\pgfsetfillopacity{0.700000}%
\pgfsetlinewidth{0.000000pt}%
\definecolor{currentstroke}{rgb}{0.000000,0.000000,0.000000}%
\pgfsetstrokecolor{currentstroke}%
\pgfsetdash{}{0pt}%
\pgfpathmoveto{\pgfqpoint{3.390993in}{3.257282in}}%
\pgfpathlineto{\pgfqpoint{3.404265in}{3.245845in}}%
\pgfpathlineto{\pgfqpoint{3.417537in}{3.234526in}}%
\pgfpathlineto{\pgfqpoint{3.430810in}{3.223322in}}%
\pgfpathlineto{\pgfqpoint{3.444084in}{3.212233in}}%
\pgfpathlineto{\pgfqpoint{3.451873in}{3.230680in}}%
\pgfpathlineto{\pgfqpoint{3.459657in}{3.249413in}}%
\pgfpathlineto{\pgfqpoint{3.467435in}{3.268438in}}%
\pgfpathlineto{\pgfqpoint{3.475207in}{3.287761in}}%
\pgfpathlineto{\pgfqpoint{3.461934in}{3.299189in}}%
\pgfpathlineto{\pgfqpoint{3.448663in}{3.310733in}}%
\pgfpathlineto{\pgfqpoint{3.435391in}{3.322392in}}%
\pgfpathlineto{\pgfqpoint{3.422120in}{3.334169in}}%
\pgfpathlineto{\pgfqpoint{3.414347in}{3.314498in}}%
\pgfpathlineto{\pgfqpoint{3.406569in}{3.295131in}}%
\pgfpathlineto{\pgfqpoint{3.398784in}{3.276061in}}%
\pgfpathlineto{\pgfqpoint{3.390993in}{3.257282in}}%
\pgfpathclose%
\pgfusepath{fill}%
\end{pgfscope}%
\begin{pgfscope}%
\pgfpathrectangle{\pgfqpoint{1.150000in}{0.150000in}}{\pgfqpoint{5.700000in}{5.700000in}}%
\pgfusepath{clip}%
\pgfsetbuttcap%
\pgfsetroundjoin%
\definecolor{currentfill}{rgb}{0.468053,0.818921,0.323998}%
\pgfsetfillcolor{currentfill}%
\pgfsetfillopacity{0.700000}%
\pgfsetlinewidth{0.000000pt}%
\definecolor{currentstroke}{rgb}{0.000000,0.000000,0.000000}%
\pgfsetstrokecolor{currentstroke}%
\pgfsetdash{}{0pt}%
\pgfpathmoveto{\pgfqpoint{3.401725in}{4.501359in}}%
\pgfpathlineto{\pgfqpoint{3.415072in}{4.483402in}}%
\pgfpathlineto{\pgfqpoint{3.428416in}{4.465589in}}%
\pgfpathlineto{\pgfqpoint{3.441758in}{4.447920in}}%
\pgfpathlineto{\pgfqpoint{3.455096in}{4.430392in}}%
\pgfpathlineto{\pgfqpoint{3.462675in}{4.466610in}}%
\pgfpathlineto{\pgfqpoint{3.470250in}{4.503446in}}%
\pgfpathlineto{\pgfqpoint{3.477821in}{4.540911in}}%
\pgfpathlineto{\pgfqpoint{3.485389in}{4.579016in}}%
\pgfpathlineto{\pgfqpoint{3.472035in}{4.597109in}}%
\pgfpathlineto{\pgfqpoint{3.458678in}{4.615344in}}%
\pgfpathlineto{\pgfqpoint{3.445319in}{4.633723in}}%
\pgfpathlineto{\pgfqpoint{3.431956in}{4.652247in}}%
\pgfpathlineto{\pgfqpoint{3.424405in}{4.613565in}}%
\pgfpathlineto{\pgfqpoint{3.416849in}{4.575530in}}%
\pgfpathlineto{\pgfqpoint{3.409290in}{4.538132in}}%
\pgfpathlineto{\pgfqpoint{3.401725in}{4.501359in}}%
\pgfpathclose%
\pgfusepath{fill}%
\end{pgfscope}%
\begin{pgfscope}%
\pgfpathrectangle{\pgfqpoint{1.150000in}{0.150000in}}{\pgfqpoint{5.700000in}{5.700000in}}%
\pgfusepath{clip}%
\pgfsetbuttcap%
\pgfsetroundjoin%
\definecolor{currentfill}{rgb}{0.157729,0.485932,0.558013}%
\pgfsetfillcolor{currentfill}%
\pgfsetfillopacity{0.700000}%
\pgfsetlinewidth{0.000000pt}%
\definecolor{currentstroke}{rgb}{0.000000,0.000000,0.000000}%
\pgfsetstrokecolor{currentstroke}%
\pgfsetdash{}{0pt}%
\pgfpathmoveto{\pgfqpoint{3.979371in}{3.558287in}}%
\pgfpathlineto{\pgfqpoint{3.992682in}{3.547024in}}%
\pgfpathlineto{\pgfqpoint{4.005996in}{3.535859in}}%
\pgfpathlineto{\pgfqpoint{4.019311in}{3.524790in}}%
\pgfpathlineto{\pgfqpoint{4.032628in}{3.513817in}}%
\pgfpathlineto{\pgfqpoint{4.040320in}{3.539618in}}%
\pgfpathlineto{\pgfqpoint{4.048012in}{3.565882in}}%
\pgfpathlineto{\pgfqpoint{4.055703in}{3.592618in}}%
\pgfpathlineto{\pgfqpoint{4.063395in}{3.619835in}}%
\pgfpathlineto{\pgfqpoint{4.050075in}{3.631294in}}%
\pgfpathlineto{\pgfqpoint{4.036756in}{3.642850in}}%
\pgfpathlineto{\pgfqpoint{4.023440in}{3.654503in}}%
\pgfpathlineto{\pgfqpoint{4.010125in}{3.666253in}}%
\pgfpathlineto{\pgfqpoint{4.002438in}{3.638540in}}%
\pgfpathlineto{\pgfqpoint{3.994749in}{3.611314in}}%
\pgfpathlineto{\pgfqpoint{3.987061in}{3.584566in}}%
\pgfpathlineto{\pgfqpoint{3.979371in}{3.558287in}}%
\pgfpathclose%
\pgfusepath{fill}%
\end{pgfscope}%
\begin{pgfscope}%
\pgfpathrectangle{\pgfqpoint{1.150000in}{0.150000in}}{\pgfqpoint{5.700000in}{5.700000in}}%
\pgfusepath{clip}%
\pgfsetbuttcap%
\pgfsetroundjoin%
\definecolor{currentfill}{rgb}{0.206756,0.371758,0.553117}%
\pgfsetfillcolor{currentfill}%
\pgfsetfillopacity{0.700000}%
\pgfsetlinewidth{0.000000pt}%
\definecolor{currentstroke}{rgb}{0.000000,0.000000,0.000000}%
\pgfsetstrokecolor{currentstroke}%
\pgfsetdash{}{0pt}%
\pgfpathmoveto{\pgfqpoint{3.253589in}{3.278223in}}%
\pgfpathlineto{\pgfqpoint{3.266861in}{3.266135in}}%
\pgfpathlineto{\pgfqpoint{3.280134in}{3.254171in}}%
\pgfpathlineto{\pgfqpoint{3.293406in}{3.242332in}}%
\pgfpathlineto{\pgfqpoint{3.306678in}{3.230615in}}%
\pgfpathlineto{\pgfqpoint{3.314496in}{3.248602in}}%
\pgfpathlineto{\pgfqpoint{3.322306in}{3.266861in}}%
\pgfpathlineto{\pgfqpoint{3.330111in}{3.285398in}}%
\pgfpathlineto{\pgfqpoint{3.337909in}{3.304219in}}%
\pgfpathlineto{\pgfqpoint{3.324638in}{3.316256in}}%
\pgfpathlineto{\pgfqpoint{3.311367in}{3.328416in}}%
\pgfpathlineto{\pgfqpoint{3.298095in}{3.340700in}}%
\pgfpathlineto{\pgfqpoint{3.284824in}{3.353109in}}%
\pgfpathlineto{\pgfqpoint{3.277025in}{3.333959in}}%
\pgfpathlineto{\pgfqpoint{3.269220in}{3.315099in}}%
\pgfpathlineto{\pgfqpoint{3.261407in}{3.296522in}}%
\pgfpathlineto{\pgfqpoint{3.253589in}{3.278223in}}%
\pgfpathclose%
\pgfusepath{fill}%
\end{pgfscope}%
\begin{pgfscope}%
\pgfpathrectangle{\pgfqpoint{1.150000in}{0.150000in}}{\pgfqpoint{5.700000in}{5.700000in}}%
\pgfusepath{clip}%
\pgfsetbuttcap%
\pgfsetroundjoin%
\definecolor{currentfill}{rgb}{0.192357,0.403199,0.555836}%
\pgfsetfillcolor{currentfill}%
\pgfsetfillopacity{0.700000}%
\pgfsetlinewidth{0.000000pt}%
\definecolor{currentstroke}{rgb}{0.000000,0.000000,0.000000}%
\pgfsetstrokecolor{currentstroke}%
\pgfsetdash{}{0pt}%
\pgfpathmoveto{\pgfqpoint{3.062894in}{3.359289in}}%
\pgfpathlineto{\pgfqpoint{3.076179in}{3.345915in}}%
\pgfpathlineto{\pgfqpoint{3.089464in}{3.332679in}}%
\pgfpathlineto{\pgfqpoint{3.102746in}{3.319581in}}%
\pgfpathlineto{\pgfqpoint{3.116028in}{3.306619in}}%
\pgfpathlineto{\pgfqpoint{3.123878in}{3.324451in}}%
\pgfpathlineto{\pgfqpoint{3.131721in}{3.342547in}}%
\pgfpathlineto{\pgfqpoint{3.139556in}{3.360915in}}%
\pgfpathlineto{\pgfqpoint{3.147385in}{3.379559in}}%
\pgfpathlineto{\pgfqpoint{3.134105in}{3.392822in}}%
\pgfpathlineto{\pgfqpoint{3.120823in}{3.406222in}}%
\pgfpathlineto{\pgfqpoint{3.107540in}{3.419759in}}%
\pgfpathlineto{\pgfqpoint{3.094256in}{3.433435in}}%
\pgfpathlineto{\pgfqpoint{3.086427in}{3.414482in}}%
\pgfpathlineto{\pgfqpoint{3.078590in}{3.395810in}}%
\pgfpathlineto{\pgfqpoint{3.070746in}{3.377414in}}%
\pgfpathlineto{\pgfqpoint{3.062894in}{3.359289in}}%
\pgfpathclose%
\pgfusepath{fill}%
\end{pgfscope}%
\begin{pgfscope}%
\pgfpathrectangle{\pgfqpoint{1.150000in}{0.150000in}}{\pgfqpoint{5.700000in}{5.700000in}}%
\pgfusepath{clip}%
\pgfsetbuttcap%
\pgfsetroundjoin%
\definecolor{currentfill}{rgb}{0.212395,0.359683,0.551710}%
\pgfsetfillcolor{currentfill}%
\pgfsetfillopacity{0.700000}%
\pgfsetlinewidth{0.000000pt}%
\definecolor{currentstroke}{rgb}{0.000000,0.000000,0.000000}%
\pgfsetstrokecolor{currentstroke}%
\pgfsetdash{}{0pt}%
\pgfpathmoveto{\pgfqpoint{3.528305in}{3.243186in}}%
\pgfpathlineto{\pgfqpoint{3.541581in}{3.232321in}}%
\pgfpathlineto{\pgfqpoint{3.554859in}{3.221567in}}%
\pgfpathlineto{\pgfqpoint{3.568139in}{3.210923in}}%
\pgfpathlineto{\pgfqpoint{3.581419in}{3.200386in}}%
\pgfpathlineto{\pgfqpoint{3.589184in}{3.219311in}}%
\pgfpathlineto{\pgfqpoint{3.596943in}{3.238536in}}%
\pgfpathlineto{\pgfqpoint{3.604698in}{3.258069in}}%
\pgfpathlineto{\pgfqpoint{3.612448in}{3.277916in}}%
\pgfpathlineto{\pgfqpoint{3.599169in}{3.288812in}}%
\pgfpathlineto{\pgfqpoint{3.585890in}{3.299816in}}%
\pgfpathlineto{\pgfqpoint{3.572613in}{3.310929in}}%
\pgfpathlineto{\pgfqpoint{3.559337in}{3.322154in}}%
\pgfpathlineto{\pgfqpoint{3.551587in}{3.301939in}}%
\pgfpathlineto{\pgfqpoint{3.543831in}{3.282044in}}%
\pgfpathlineto{\pgfqpoint{3.536070in}{3.262461in}}%
\pgfpathlineto{\pgfqpoint{3.528305in}{3.243186in}}%
\pgfpathclose%
\pgfusepath{fill}%
\end{pgfscope}%
\begin{pgfscope}%
\pgfpathrectangle{\pgfqpoint{1.150000in}{0.150000in}}{\pgfqpoint{5.700000in}{5.700000in}}%
\pgfusepath{clip}%
\pgfsetbuttcap%
\pgfsetroundjoin%
\definecolor{currentfill}{rgb}{0.172719,0.448791,0.557885}%
\pgfsetfillcolor{currentfill}%
\pgfsetfillopacity{0.700000}%
\pgfsetlinewidth{0.000000pt}%
\definecolor{currentstroke}{rgb}{0.000000,0.000000,0.000000}%
\pgfsetstrokecolor{currentstroke}%
\pgfsetdash{}{0pt}%
\pgfpathmoveto{\pgfqpoint{3.948600in}{3.457669in}}%
\pgfpathlineto{\pgfqpoint{3.961910in}{3.446871in}}%
\pgfpathlineto{\pgfqpoint{3.975221in}{3.436169in}}%
\pgfpathlineto{\pgfqpoint{3.988535in}{3.425565in}}%
\pgfpathlineto{\pgfqpoint{4.001851in}{3.415055in}}%
\pgfpathlineto{\pgfqpoint{4.009547in}{3.439097in}}%
\pgfpathlineto{\pgfqpoint{4.017242in}{3.463565in}}%
\pgfpathlineto{\pgfqpoint{4.024936in}{3.488469in}}%
\pgfpathlineto{\pgfqpoint{4.032628in}{3.513817in}}%
\pgfpathlineto{\pgfqpoint{4.019311in}{3.524790in}}%
\pgfpathlineto{\pgfqpoint{4.005996in}{3.535859in}}%
\pgfpathlineto{\pgfqpoint{3.992682in}{3.547024in}}%
\pgfpathlineto{\pgfqpoint{3.979371in}{3.558287in}}%
\pgfpathlineto{\pgfqpoint{3.971680in}{3.532466in}}%
\pgfpathlineto{\pgfqpoint{3.963988in}{3.507096in}}%
\pgfpathlineto{\pgfqpoint{3.956295in}{3.482166in}}%
\pgfpathlineto{\pgfqpoint{3.948600in}{3.457669in}}%
\pgfpathclose%
\pgfusepath{fill}%
\end{pgfscope}%
\begin{pgfscope}%
\pgfpathrectangle{\pgfqpoint{1.150000in}{0.150000in}}{\pgfqpoint{5.700000in}{5.700000in}}%
\pgfusepath{clip}%
\pgfsetbuttcap%
\pgfsetroundjoin%
\definecolor{currentfill}{rgb}{0.140536,0.530132,0.555659}%
\pgfsetfillcolor{currentfill}%
\pgfsetfillopacity{0.700000}%
\pgfsetlinewidth{0.000000pt}%
\definecolor{currentstroke}{rgb}{0.000000,0.000000,0.000000}%
\pgfsetstrokecolor{currentstroke}%
\pgfsetdash{}{0pt}%
\pgfpathmoveto{\pgfqpoint{4.010125in}{3.666253in}}%
\pgfpathlineto{\pgfqpoint{4.023440in}{3.654503in}}%
\pgfpathlineto{\pgfqpoint{4.036756in}{3.642850in}}%
\pgfpathlineto{\pgfqpoint{4.050075in}{3.631294in}}%
\pgfpathlineto{\pgfqpoint{4.063395in}{3.619835in}}%
\pgfpathlineto{\pgfqpoint{4.071086in}{3.647543in}}%
\pgfpathlineto{\pgfqpoint{4.078778in}{3.675751in}}%
\pgfpathlineto{\pgfqpoint{4.086470in}{3.704470in}}%
\pgfpathlineto{\pgfqpoint{4.094163in}{3.733709in}}%
\pgfpathlineto{\pgfqpoint{4.080839in}{3.745678in}}%
\pgfpathlineto{\pgfqpoint{4.067516in}{3.757744in}}%
\pgfpathlineto{\pgfqpoint{4.054196in}{3.769907in}}%
\pgfpathlineto{\pgfqpoint{4.040876in}{3.782169in}}%
\pgfpathlineto{\pgfqpoint{4.033188in}{3.752410in}}%
\pgfpathlineto{\pgfqpoint{4.025501in}{3.723178in}}%
\pgfpathlineto{\pgfqpoint{4.017813in}{3.694462in}}%
\pgfpathlineto{\pgfqpoint{4.010125in}{3.666253in}}%
\pgfpathclose%
\pgfusepath{fill}%
\end{pgfscope}%
\begin{pgfscope}%
\pgfpathrectangle{\pgfqpoint{1.150000in}{0.150000in}}{\pgfqpoint{5.700000in}{5.700000in}}%
\pgfusepath{clip}%
\pgfsetbuttcap%
\pgfsetroundjoin%
\definecolor{currentfill}{rgb}{0.121148,0.592739,0.544641}%
\pgfsetfillcolor{currentfill}%
\pgfsetfillopacity{0.700000}%
\pgfsetlinewidth{0.000000pt}%
\definecolor{currentstroke}{rgb}{0.000000,0.000000,0.000000}%
\pgfsetstrokecolor{currentstroke}%
\pgfsetdash{}{0pt}%
\pgfpathmoveto{\pgfqpoint{3.987613in}{3.832206in}}%
\pgfpathlineto{\pgfqpoint{4.000927in}{3.819546in}}%
\pgfpathlineto{\pgfqpoint{4.014242in}{3.806987in}}%
\pgfpathlineto{\pgfqpoint{4.027559in}{3.794528in}}%
\pgfpathlineto{\pgfqpoint{4.040876in}{3.782169in}}%
\pgfpathlineto{\pgfqpoint{4.048565in}{3.812464in}}%
\pgfpathlineto{\pgfqpoint{4.056255in}{3.843305in}}%
\pgfpathlineto{\pgfqpoint{4.063945in}{3.874704in}}%
\pgfpathlineto{\pgfqpoint{4.071637in}{3.906671in}}%
\pgfpathlineto{\pgfqpoint{4.058313in}{3.919567in}}%
\pgfpathlineto{\pgfqpoint{4.044990in}{3.932562in}}%
\pgfpathlineto{\pgfqpoint{4.031668in}{3.945657in}}%
\pgfpathlineto{\pgfqpoint{4.018347in}{3.958854in}}%
\pgfpathlineto{\pgfqpoint{4.010663in}{3.926341in}}%
\pgfpathlineto{\pgfqpoint{4.002979in}{3.894403in}}%
\pgfpathlineto{\pgfqpoint{3.995296in}{3.863028in}}%
\pgfpathlineto{\pgfqpoint{3.987613in}{3.832206in}}%
\pgfpathclose%
\pgfusepath{fill}%
\end{pgfscope}%
\begin{pgfscope}%
\pgfpathrectangle{\pgfqpoint{1.150000in}{0.150000in}}{\pgfqpoint{5.700000in}{5.700000in}}%
\pgfusepath{clip}%
\pgfsetbuttcap%
\pgfsetroundjoin%
\definecolor{currentfill}{rgb}{0.252899,0.742211,0.448284}%
\pgfsetfillcolor{currentfill}%
\pgfsetfillopacity{0.700000}%
\pgfsetlinewidth{0.000000pt}%
\definecolor{currentstroke}{rgb}{0.000000,0.000000,0.000000}%
\pgfsetstrokecolor{currentstroke}%
\pgfsetdash{}{0pt}%
\pgfpathmoveto{\pgfqpoint{3.752021in}{4.245168in}}%
\pgfpathlineto{\pgfqpoint{3.765339in}{4.229788in}}%
\pgfpathlineto{\pgfqpoint{3.778656in}{4.214525in}}%
\pgfpathlineto{\pgfqpoint{3.791972in}{4.199378in}}%
\pgfpathlineto{\pgfqpoint{3.805288in}{4.184347in}}%
\pgfpathlineto{\pgfqpoint{3.812934in}{4.219675in}}%
\pgfpathlineto{\pgfqpoint{3.820581in}{4.255625in}}%
\pgfpathlineto{\pgfqpoint{3.828227in}{4.292209in}}%
\pgfpathlineto{\pgfqpoint{3.835873in}{4.329437in}}%
\pgfpathlineto{\pgfqpoint{3.822545in}{4.345041in}}%
\pgfpathlineto{\pgfqpoint{3.809216in}{4.360761in}}%
\pgfpathlineto{\pgfqpoint{3.795887in}{4.376598in}}%
\pgfpathlineto{\pgfqpoint{3.782557in}{4.392553in}}%
\pgfpathlineto{\pgfqpoint{3.774924in}{4.354741in}}%
\pgfpathlineto{\pgfqpoint{3.767290in}{4.317580in}}%
\pgfpathlineto{\pgfqpoint{3.759656in}{4.281060in}}%
\pgfpathlineto{\pgfqpoint{3.752021in}{4.245168in}}%
\pgfpathclose%
\pgfusepath{fill}%
\end{pgfscope}%
\begin{pgfscope}%
\pgfpathrectangle{\pgfqpoint{1.150000in}{0.150000in}}{\pgfqpoint{5.700000in}{5.700000in}}%
\pgfusepath{clip}%
\pgfsetbuttcap%
\pgfsetroundjoin%
\definecolor{currentfill}{rgb}{0.195860,0.395433,0.555276}%
\pgfsetfillcolor{currentfill}%
\pgfsetfillopacity{0.700000}%
\pgfsetlinewidth{0.000000pt}%
\definecolor{currentstroke}{rgb}{0.000000,0.000000,0.000000}%
\pgfsetstrokecolor{currentstroke}%
\pgfsetdash{}{0pt}%
\pgfpathmoveto{\pgfqpoint{3.833754in}{3.316953in}}%
\pgfpathlineto{\pgfqpoint{3.847054in}{3.306625in}}%
\pgfpathlineto{\pgfqpoint{3.860357in}{3.296396in}}%
\pgfpathlineto{\pgfqpoint{3.873661in}{3.286266in}}%
\pgfpathlineto{\pgfqpoint{3.886968in}{3.276233in}}%
\pgfpathlineto{\pgfqpoint{3.894681in}{3.297575in}}%
\pgfpathlineto{\pgfqpoint{3.902391in}{3.319284in}}%
\pgfpathlineto{\pgfqpoint{3.910098in}{3.341366in}}%
\pgfpathlineto{\pgfqpoint{3.917803in}{3.363830in}}%
\pgfpathlineto{\pgfqpoint{3.904496in}{3.374283in}}%
\pgfpathlineto{\pgfqpoint{3.891192in}{3.384834in}}%
\pgfpathlineto{\pgfqpoint{3.877889in}{3.395484in}}%
\pgfpathlineto{\pgfqpoint{3.864588in}{3.406233in}}%
\pgfpathlineto{\pgfqpoint{3.856884in}{3.383340in}}%
\pgfpathlineto{\pgfqpoint{3.849177in}{3.360834in}}%
\pgfpathlineto{\pgfqpoint{3.841467in}{3.338708in}}%
\pgfpathlineto{\pgfqpoint{3.833754in}{3.316953in}}%
\pgfpathclose%
\pgfusepath{fill}%
\end{pgfscope}%
\begin{pgfscope}%
\pgfpathrectangle{\pgfqpoint{1.150000in}{0.150000in}}{\pgfqpoint{5.700000in}{5.700000in}}%
\pgfusepath{clip}%
\pgfsetbuttcap%
\pgfsetroundjoin%
\definecolor{currentfill}{rgb}{0.214000,0.722114,0.469588}%
\pgfsetfillcolor{currentfill}%
\pgfsetfillopacity{0.700000}%
\pgfsetlinewidth{0.000000pt}%
\definecolor{currentstroke}{rgb}{0.000000,0.000000,0.000000}%
\pgfsetstrokecolor{currentstroke}%
\pgfsetdash{}{0pt}%
\pgfpathmoveto{\pgfqpoint{3.805288in}{4.184347in}}%
\pgfpathlineto{\pgfqpoint{3.818603in}{4.169431in}}%
\pgfpathlineto{\pgfqpoint{3.831918in}{4.154628in}}%
\pgfpathlineto{\pgfqpoint{3.845233in}{4.139938in}}%
\pgfpathlineto{\pgfqpoint{3.858548in}{4.125360in}}%
\pgfpathlineto{\pgfqpoint{3.866206in}{4.160127in}}%
\pgfpathlineto{\pgfqpoint{3.873864in}{4.195509in}}%
\pgfpathlineto{\pgfqpoint{3.881522in}{4.231518in}}%
\pgfpathlineto{\pgfqpoint{3.889180in}{4.268165in}}%
\pgfpathlineto{\pgfqpoint{3.875854in}{4.283313in}}%
\pgfpathlineto{\pgfqpoint{3.862528in}{4.298574in}}%
\pgfpathlineto{\pgfqpoint{3.849201in}{4.313948in}}%
\pgfpathlineto{\pgfqpoint{3.835873in}{4.329437in}}%
\pgfpathlineto{\pgfqpoint{3.828227in}{4.292209in}}%
\pgfpathlineto{\pgfqpoint{3.820581in}{4.255625in}}%
\pgfpathlineto{\pgfqpoint{3.812934in}{4.219675in}}%
\pgfpathlineto{\pgfqpoint{3.805288in}{4.184347in}}%
\pgfpathclose%
\pgfusepath{fill}%
\end{pgfscope}%
\begin{pgfscope}%
\pgfpathrectangle{\pgfqpoint{1.150000in}{0.150000in}}{\pgfqpoint{5.700000in}{5.700000in}}%
\pgfusepath{clip}%
\pgfsetbuttcap%
\pgfsetroundjoin%
\definecolor{currentfill}{rgb}{0.204903,0.375746,0.553533}%
\pgfsetfillcolor{currentfill}%
\pgfsetfillopacity{0.700000}%
\pgfsetlinewidth{0.000000pt}%
\definecolor{currentstroke}{rgb}{0.000000,0.000000,0.000000}%
\pgfsetstrokecolor{currentstroke}%
\pgfsetdash{}{0pt}%
\pgfpathmoveto{\pgfqpoint{3.749686in}{3.274208in}}%
\pgfpathlineto{\pgfqpoint{3.762979in}{3.263878in}}%
\pgfpathlineto{\pgfqpoint{3.776274in}{3.253649in}}%
\pgfpathlineto{\pgfqpoint{3.789571in}{3.243521in}}%
\pgfpathlineto{\pgfqpoint{3.802870in}{3.233493in}}%
\pgfpathlineto{\pgfqpoint{3.810596in}{3.253839in}}%
\pgfpathlineto{\pgfqpoint{3.818319in}{3.274526in}}%
\pgfpathlineto{\pgfqpoint{3.826038in}{3.295562in}}%
\pgfpathlineto{\pgfqpoint{3.833754in}{3.316953in}}%
\pgfpathlineto{\pgfqpoint{3.820456in}{3.327380in}}%
\pgfpathlineto{\pgfqpoint{3.807159in}{3.337908in}}%
\pgfpathlineto{\pgfqpoint{3.793865in}{3.348537in}}%
\pgfpathlineto{\pgfqpoint{3.780572in}{3.359267in}}%
\pgfpathlineto{\pgfqpoint{3.772856in}{3.337468in}}%
\pgfpathlineto{\pgfqpoint{3.765136in}{3.316030in}}%
\pgfpathlineto{\pgfqpoint{3.757413in}{3.294946in}}%
\pgfpathlineto{\pgfqpoint{3.749686in}{3.274208in}}%
\pgfpathclose%
\pgfusepath{fill}%
\end{pgfscope}%
\begin{pgfscope}%
\pgfpathrectangle{\pgfqpoint{1.150000in}{0.150000in}}{\pgfqpoint{5.700000in}{5.700000in}}%
\pgfusepath{clip}%
\pgfsetbuttcap%
\pgfsetroundjoin%
\definecolor{currentfill}{rgb}{0.201239,0.383670,0.554294}%
\pgfsetfillcolor{currentfill}%
\pgfsetfillopacity{0.700000}%
\pgfsetlinewidth{0.000000pt}%
\definecolor{currentstroke}{rgb}{0.000000,0.000000,0.000000}%
\pgfsetstrokecolor{currentstroke}%
\pgfsetdash{}{0pt}%
\pgfpathmoveto{\pgfqpoint{3.116028in}{3.306619in}}%
\pgfpathlineto{\pgfqpoint{3.129308in}{3.293793in}}%
\pgfpathlineto{\pgfqpoint{3.142587in}{3.281099in}}%
\pgfpathlineto{\pgfqpoint{3.155865in}{3.268539in}}%
\pgfpathlineto{\pgfqpoint{3.169143in}{3.256110in}}%
\pgfpathlineto{\pgfqpoint{3.176991in}{3.273649in}}%
\pgfpathlineto{\pgfqpoint{3.184832in}{3.291448in}}%
\pgfpathlineto{\pgfqpoint{3.192666in}{3.309512in}}%
\pgfpathlineto{\pgfqpoint{3.200493in}{3.327847in}}%
\pgfpathlineto{\pgfqpoint{3.187218in}{3.340576in}}%
\pgfpathlineto{\pgfqpoint{3.173941in}{3.353437in}}%
\pgfpathlineto{\pgfqpoint{3.160663in}{3.366431in}}%
\pgfpathlineto{\pgfqpoint{3.147385in}{3.379559in}}%
\pgfpathlineto{\pgfqpoint{3.139556in}{3.360915in}}%
\pgfpathlineto{\pgfqpoint{3.131721in}{3.342547in}}%
\pgfpathlineto{\pgfqpoint{3.123878in}{3.324451in}}%
\pgfpathlineto{\pgfqpoint{3.116028in}{3.306619in}}%
\pgfpathclose%
\pgfusepath{fill}%
\end{pgfscope}%
\begin{pgfscope}%
\pgfpathrectangle{\pgfqpoint{1.150000in}{0.150000in}}{\pgfqpoint{5.700000in}{5.700000in}}%
\pgfusepath{clip}%
\pgfsetbuttcap%
\pgfsetroundjoin%
\definecolor{currentfill}{rgb}{0.304148,0.764704,0.419943}%
\pgfsetfillcolor{currentfill}%
\pgfsetfillopacity{0.700000}%
\pgfsetlinewidth{0.000000pt}%
\definecolor{currentstroke}{rgb}{0.000000,0.000000,0.000000}%
\pgfsetstrokecolor{currentstroke}%
\pgfsetdash{}{0pt}%
\pgfpathmoveto{\pgfqpoint{3.698743in}{4.307882in}}%
\pgfpathlineto{\pgfqpoint{3.712064in}{4.292022in}}%
\pgfpathlineto{\pgfqpoint{3.725384in}{4.276284in}}%
\pgfpathlineto{\pgfqpoint{3.738703in}{4.260666in}}%
\pgfpathlineto{\pgfqpoint{3.752021in}{4.245168in}}%
\pgfpathlineto{\pgfqpoint{3.759656in}{4.281060in}}%
\pgfpathlineto{\pgfqpoint{3.767290in}{4.317580in}}%
\pgfpathlineto{\pgfqpoint{3.774924in}{4.354741in}}%
\pgfpathlineto{\pgfqpoint{3.782557in}{4.392553in}}%
\pgfpathlineto{\pgfqpoint{3.769225in}{4.408628in}}%
\pgfpathlineto{\pgfqpoint{3.755893in}{4.424822in}}%
\pgfpathlineto{\pgfqpoint{3.742560in}{4.441138in}}%
\pgfpathlineto{\pgfqpoint{3.729225in}{4.457575in}}%
\pgfpathlineto{\pgfqpoint{3.721606in}{4.419175in}}%
\pgfpathlineto{\pgfqpoint{3.713986in}{4.381434in}}%
\pgfpathlineto{\pgfqpoint{3.706365in}{4.344340in}}%
\pgfpathlineto{\pgfqpoint{3.698743in}{4.307882in}}%
\pgfpathclose%
\pgfusepath{fill}%
\end{pgfscope}%
\begin{pgfscope}%
\pgfpathrectangle{\pgfqpoint{1.150000in}{0.150000in}}{\pgfqpoint{5.700000in}{5.700000in}}%
\pgfusepath{clip}%
\pgfsetbuttcap%
\pgfsetroundjoin%
\definecolor{currentfill}{rgb}{0.180653,0.701402,0.488189}%
\pgfsetfillcolor{currentfill}%
\pgfsetfillopacity{0.700000}%
\pgfsetlinewidth{0.000000pt}%
\definecolor{currentstroke}{rgb}{0.000000,0.000000,0.000000}%
\pgfsetstrokecolor{currentstroke}%
\pgfsetdash{}{0pt}%
\pgfpathmoveto{\pgfqpoint{3.858548in}{4.125360in}}%
\pgfpathlineto{\pgfqpoint{3.871863in}{4.110893in}}%
\pgfpathlineto{\pgfqpoint{3.885178in}{4.096537in}}%
\pgfpathlineto{\pgfqpoint{3.898493in}{4.082289in}}%
\pgfpathlineto{\pgfqpoint{3.911808in}{4.068150in}}%
\pgfpathlineto{\pgfqpoint{3.919476in}{4.102358in}}%
\pgfpathlineto{\pgfqpoint{3.927144in}{4.137175in}}%
\pgfpathlineto{\pgfqpoint{3.934813in}{4.172612in}}%
\pgfpathlineto{\pgfqpoint{3.942484in}{4.208680in}}%
\pgfpathlineto{\pgfqpoint{3.929158in}{4.223387in}}%
\pgfpathlineto{\pgfqpoint{3.915832in}{4.238203in}}%
\pgfpathlineto{\pgfqpoint{3.902506in}{4.253128in}}%
\pgfpathlineto{\pgfqpoint{3.889180in}{4.268165in}}%
\pgfpathlineto{\pgfqpoint{3.881522in}{4.231518in}}%
\pgfpathlineto{\pgfqpoint{3.873864in}{4.195509in}}%
\pgfpathlineto{\pgfqpoint{3.866206in}{4.160127in}}%
\pgfpathlineto{\pgfqpoint{3.858548in}{4.125360in}}%
\pgfpathclose%
\pgfusepath{fill}%
\end{pgfscope}%
\begin{pgfscope}%
\pgfpathrectangle{\pgfqpoint{1.150000in}{0.150000in}}{\pgfqpoint{5.700000in}{5.700000in}}%
\pgfusepath{clip}%
\pgfsetbuttcap%
\pgfsetroundjoin%
\definecolor{currentfill}{rgb}{0.187231,0.414746,0.556547}%
\pgfsetfillcolor{currentfill}%
\pgfsetfillopacity{0.700000}%
\pgfsetlinewidth{0.000000pt}%
\definecolor{currentstroke}{rgb}{0.000000,0.000000,0.000000}%
\pgfsetstrokecolor{currentstroke}%
\pgfsetdash{}{0pt}%
\pgfpathmoveto{\pgfqpoint{3.917803in}{3.363830in}}%
\pgfpathlineto{\pgfqpoint{3.931112in}{3.353474in}}%
\pgfpathlineto{\pgfqpoint{3.944423in}{3.343215in}}%
\pgfpathlineto{\pgfqpoint{3.957736in}{3.333052in}}%
\pgfpathlineto{\pgfqpoint{3.971052in}{3.322984in}}%
\pgfpathlineto{\pgfqpoint{3.978754in}{3.345404in}}%
\pgfpathlineto{\pgfqpoint{3.986455in}{3.368217in}}%
\pgfpathlineto{\pgfqpoint{3.994154in}{3.391432in}}%
\pgfpathlineto{\pgfqpoint{4.001851in}{3.415055in}}%
\pgfpathlineto{\pgfqpoint{3.988535in}{3.425565in}}%
\pgfpathlineto{\pgfqpoint{3.975221in}{3.436169in}}%
\pgfpathlineto{\pgfqpoint{3.961910in}{3.446871in}}%
\pgfpathlineto{\pgfqpoint{3.948600in}{3.457669in}}%
\pgfpathlineto{\pgfqpoint{3.940904in}{3.433595in}}%
\pgfpathlineto{\pgfqpoint{3.933206in}{3.409936in}}%
\pgfpathlineto{\pgfqpoint{3.925505in}{3.386684in}}%
\pgfpathlineto{\pgfqpoint{3.917803in}{3.363830in}}%
\pgfpathclose%
\pgfusepath{fill}%
\end{pgfscope}%
\begin{pgfscope}%
\pgfpathrectangle{\pgfqpoint{1.150000in}{0.150000in}}{\pgfqpoint{5.700000in}{5.700000in}}%
\pgfusepath{clip}%
\pgfsetbuttcap%
\pgfsetroundjoin%
\definecolor{currentfill}{rgb}{0.352360,0.783011,0.392636}%
\pgfsetfillcolor{currentfill}%
\pgfsetfillopacity{0.700000}%
\pgfsetlinewidth{0.000000pt}%
\definecolor{currentstroke}{rgb}{0.000000,0.000000,0.000000}%
\pgfsetstrokecolor{currentstroke}%
\pgfsetdash{}{0pt}%
\pgfpathmoveto{\pgfqpoint{3.645447in}{4.372553in}}%
\pgfpathlineto{\pgfqpoint{3.658773in}{4.356198in}}%
\pgfpathlineto{\pgfqpoint{3.672098in}{4.339968in}}%
\pgfpathlineto{\pgfqpoint{3.685421in}{4.323863in}}%
\pgfpathlineto{\pgfqpoint{3.698743in}{4.307882in}}%
\pgfpathlineto{\pgfqpoint{3.706365in}{4.344340in}}%
\pgfpathlineto{\pgfqpoint{3.713986in}{4.381434in}}%
\pgfpathlineto{\pgfqpoint{3.721606in}{4.419175in}}%
\pgfpathlineto{\pgfqpoint{3.729225in}{4.457575in}}%
\pgfpathlineto{\pgfqpoint{3.715889in}{4.474136in}}%
\pgfpathlineto{\pgfqpoint{3.702552in}{4.490821in}}%
\pgfpathlineto{\pgfqpoint{3.689213in}{4.507631in}}%
\pgfpathlineto{\pgfqpoint{3.675872in}{4.524567in}}%
\pgfpathlineto{\pgfqpoint{3.668268in}{4.485577in}}%
\pgfpathlineto{\pgfqpoint{3.660663in}{4.447252in}}%
\pgfpathlineto{\pgfqpoint{3.653056in}{4.409581in}}%
\pgfpathlineto{\pgfqpoint{3.645447in}{4.372553in}}%
\pgfpathclose%
\pgfusepath{fill}%
\end{pgfscope}%
\begin{pgfscope}%
\pgfpathrectangle{\pgfqpoint{1.150000in}{0.150000in}}{\pgfqpoint{5.700000in}{5.700000in}}%
\pgfusepath{clip}%
\pgfsetbuttcap%
\pgfsetroundjoin%
\definecolor{currentfill}{rgb}{0.212395,0.359683,0.551710}%
\pgfsetfillcolor{currentfill}%
\pgfsetfillopacity{0.700000}%
\pgfsetlinewidth{0.000000pt}%
\definecolor{currentstroke}{rgb}{0.000000,0.000000,0.000000}%
\pgfsetstrokecolor{currentstroke}%
\pgfsetdash{}{0pt}%
\pgfpathmoveto{\pgfqpoint{3.665580in}{3.235406in}}%
\pgfpathlineto{\pgfqpoint{3.678867in}{3.225043in}}%
\pgfpathlineto{\pgfqpoint{3.692155in}{3.214783in}}%
\pgfpathlineto{\pgfqpoint{3.705446in}{3.204628in}}%
\pgfpathlineto{\pgfqpoint{3.718738in}{3.194574in}}%
\pgfpathlineto{\pgfqpoint{3.726481in}{3.213999in}}%
\pgfpathlineto{\pgfqpoint{3.734220in}{3.233742in}}%
\pgfpathlineto{\pgfqpoint{3.741955in}{3.253809in}}%
\pgfpathlineto{\pgfqpoint{3.749686in}{3.274208in}}%
\pgfpathlineto{\pgfqpoint{3.736394in}{3.284640in}}%
\pgfpathlineto{\pgfqpoint{3.723105in}{3.295175in}}%
\pgfpathlineto{\pgfqpoint{3.709817in}{3.305814in}}%
\pgfpathlineto{\pgfqpoint{3.696531in}{3.316557in}}%
\pgfpathlineto{\pgfqpoint{3.688800in}{3.295771in}}%
\pgfpathlineto{\pgfqpoint{3.681064in}{3.275322in}}%
\pgfpathlineto{\pgfqpoint{3.673324in}{3.255203in}}%
\pgfpathlineto{\pgfqpoint{3.665580in}{3.235406in}}%
\pgfpathclose%
\pgfusepath{fill}%
\end{pgfscope}%
\begin{pgfscope}%
\pgfpathrectangle{\pgfqpoint{1.150000in}{0.150000in}}{\pgfqpoint{5.700000in}{5.700000in}}%
\pgfusepath{clip}%
\pgfsetbuttcap%
\pgfsetroundjoin%
\definecolor{currentfill}{rgb}{0.214298,0.355619,0.551184}%
\pgfsetfillcolor{currentfill}%
\pgfsetfillopacity{0.700000}%
\pgfsetlinewidth{0.000000pt}%
\definecolor{currentstroke}{rgb}{0.000000,0.000000,0.000000}%
\pgfsetstrokecolor{currentstroke}%
\pgfsetdash{}{0pt}%
\pgfpathmoveto{\pgfqpoint{3.306678in}{3.230615in}}%
\pgfpathlineto{\pgfqpoint{3.319951in}{3.219020in}}%
\pgfpathlineto{\pgfqpoint{3.333223in}{3.207545in}}%
\pgfpathlineto{\pgfqpoint{3.346496in}{3.196191in}}%
\pgfpathlineto{\pgfqpoint{3.359769in}{3.184955in}}%
\pgfpathlineto{\pgfqpoint{3.367584in}{3.202631in}}%
\pgfpathlineto{\pgfqpoint{3.375393in}{3.220573in}}%
\pgfpathlineto{\pgfqpoint{3.383196in}{3.238788in}}%
\pgfpathlineto{\pgfqpoint{3.390993in}{3.257282in}}%
\pgfpathlineto{\pgfqpoint{3.377722in}{3.268837in}}%
\pgfpathlineto{\pgfqpoint{3.364450in}{3.280511in}}%
\pgfpathlineto{\pgfqpoint{3.351179in}{3.292305in}}%
\pgfpathlineto{\pgfqpoint{3.337909in}{3.304219in}}%
\pgfpathlineto{\pgfqpoint{3.330111in}{3.285398in}}%
\pgfpathlineto{\pgfqpoint{3.322306in}{3.266861in}}%
\pgfpathlineto{\pgfqpoint{3.314496in}{3.248602in}}%
\pgfpathlineto{\pgfqpoint{3.306678in}{3.230615in}}%
\pgfpathclose%
\pgfusepath{fill}%
\end{pgfscope}%
\begin{pgfscope}%
\pgfpathrectangle{\pgfqpoint{1.150000in}{0.150000in}}{\pgfqpoint{5.700000in}{5.700000in}}%
\pgfusepath{clip}%
\pgfsetbuttcap%
\pgfsetroundjoin%
\definecolor{currentfill}{rgb}{0.218130,0.347432,0.550038}%
\pgfsetfillcolor{currentfill}%
\pgfsetfillopacity{0.700000}%
\pgfsetlinewidth{0.000000pt}%
\definecolor{currentstroke}{rgb}{0.000000,0.000000,0.000000}%
\pgfsetstrokecolor{currentstroke}%
\pgfsetdash{}{0pt}%
\pgfpathmoveto{\pgfqpoint{3.444084in}{3.212233in}}%
\pgfpathlineto{\pgfqpoint{3.457358in}{3.201259in}}%
\pgfpathlineto{\pgfqpoint{3.470634in}{3.190399in}}%
\pgfpathlineto{\pgfqpoint{3.483910in}{3.179651in}}%
\pgfpathlineto{\pgfqpoint{3.497187in}{3.169014in}}%
\pgfpathlineto{\pgfqpoint{3.504975in}{3.187130in}}%
\pgfpathlineto{\pgfqpoint{3.512757in}{3.205526in}}%
\pgfpathlineto{\pgfqpoint{3.520533in}{3.224209in}}%
\pgfpathlineto{\pgfqpoint{3.528305in}{3.243186in}}%
\pgfpathlineto{\pgfqpoint{3.515029in}{3.254161in}}%
\pgfpathlineto{\pgfqpoint{3.501754in}{3.265248in}}%
\pgfpathlineto{\pgfqpoint{3.488480in}{3.276448in}}%
\pgfpathlineto{\pgfqpoint{3.475207in}{3.287761in}}%
\pgfpathlineto{\pgfqpoint{3.467435in}{3.268438in}}%
\pgfpathlineto{\pgfqpoint{3.459657in}{3.249413in}}%
\pgfpathlineto{\pgfqpoint{3.451873in}{3.230680in}}%
\pgfpathlineto{\pgfqpoint{3.444084in}{3.212233in}}%
\pgfpathclose%
\pgfusepath{fill}%
\end{pgfscope}%
\begin{pgfscope}%
\pgfpathrectangle{\pgfqpoint{1.150000in}{0.150000in}}{\pgfqpoint{5.700000in}{5.700000in}}%
\pgfusepath{clip}%
\pgfsetbuttcap%
\pgfsetroundjoin%
\definecolor{currentfill}{rgb}{0.157851,0.683765,0.501686}%
\pgfsetfillcolor{currentfill}%
\pgfsetfillopacity{0.700000}%
\pgfsetlinewidth{0.000000pt}%
\definecolor{currentstroke}{rgb}{0.000000,0.000000,0.000000}%
\pgfsetstrokecolor{currentstroke}%
\pgfsetdash{}{0pt}%
\pgfpathmoveto{\pgfqpoint{3.911808in}{4.068150in}}%
\pgfpathlineto{\pgfqpoint{3.925123in}{4.054119in}}%
\pgfpathlineto{\pgfqpoint{3.938439in}{4.040195in}}%
\pgfpathlineto{\pgfqpoint{3.951756in}{4.026377in}}%
\pgfpathlineto{\pgfqpoint{3.965073in}{4.012665in}}%
\pgfpathlineto{\pgfqpoint{3.972750in}{4.046316in}}%
\pgfpathlineto{\pgfqpoint{3.980428in}{4.080570in}}%
\pgfpathlineto{\pgfqpoint{3.988107in}{4.115438in}}%
\pgfpathlineto{\pgfqpoint{3.995788in}{4.150930in}}%
\pgfpathlineto{\pgfqpoint{3.982462in}{4.165208in}}%
\pgfpathlineto{\pgfqpoint{3.969135in}{4.179592in}}%
\pgfpathlineto{\pgfqpoint{3.955809in}{4.194082in}}%
\pgfpathlineto{\pgfqpoint{3.942484in}{4.208680in}}%
\pgfpathlineto{\pgfqpoint{3.934813in}{4.172612in}}%
\pgfpathlineto{\pgfqpoint{3.927144in}{4.137175in}}%
\pgfpathlineto{\pgfqpoint{3.919476in}{4.102358in}}%
\pgfpathlineto{\pgfqpoint{3.911808in}{4.068150in}}%
\pgfpathclose%
\pgfusepath{fill}%
\end{pgfscope}%
\begin{pgfscope}%
\pgfpathrectangle{\pgfqpoint{1.150000in}{0.150000in}}{\pgfqpoint{5.700000in}{5.700000in}}%
\pgfusepath{clip}%
\pgfsetbuttcap%
\pgfsetroundjoin%
\definecolor{currentfill}{rgb}{0.125394,0.574318,0.549086}%
\pgfsetfillcolor{currentfill}%
\pgfsetfillopacity{0.700000}%
\pgfsetlinewidth{0.000000pt}%
\definecolor{currentstroke}{rgb}{0.000000,0.000000,0.000000}%
\pgfsetstrokecolor{currentstroke}%
\pgfsetdash{}{0pt}%
\pgfpathmoveto{\pgfqpoint{4.040876in}{3.782169in}}%
\pgfpathlineto{\pgfqpoint{4.054196in}{3.769907in}}%
\pgfpathlineto{\pgfqpoint{4.067516in}{3.757744in}}%
\pgfpathlineto{\pgfqpoint{4.080839in}{3.745678in}}%
\pgfpathlineto{\pgfqpoint{4.094163in}{3.733709in}}%
\pgfpathlineto{\pgfqpoint{4.101857in}{3.763479in}}%
\pgfpathlineto{\pgfqpoint{4.109552in}{3.793789in}}%
\pgfpathlineto{\pgfqpoint{4.117248in}{3.824651in}}%
\pgfpathlineto{\pgfqpoint{4.124947in}{3.856074in}}%
\pgfpathlineto{\pgfqpoint{4.111617in}{3.868577in}}%
\pgfpathlineto{\pgfqpoint{4.098289in}{3.881177in}}%
\pgfpathlineto{\pgfqpoint{4.084962in}{3.893875in}}%
\pgfpathlineto{\pgfqpoint{4.071637in}{3.906671in}}%
\pgfpathlineto{\pgfqpoint{4.063945in}{3.874704in}}%
\pgfpathlineto{\pgfqpoint{4.056255in}{3.843305in}}%
\pgfpathlineto{\pgfqpoint{4.048565in}{3.812464in}}%
\pgfpathlineto{\pgfqpoint{4.040876in}{3.782169in}}%
\pgfpathclose%
\pgfusepath{fill}%
\end{pgfscope}%
\begin{pgfscope}%
\pgfpathrectangle{\pgfqpoint{1.150000in}{0.150000in}}{\pgfqpoint{5.700000in}{5.700000in}}%
\pgfusepath{clip}%
\pgfsetbuttcap%
\pgfsetroundjoin%
\definecolor{currentfill}{rgb}{0.163625,0.471133,0.558148}%
\pgfsetfillcolor{currentfill}%
\pgfsetfillopacity{0.700000}%
\pgfsetlinewidth{0.000000pt}%
\definecolor{currentstroke}{rgb}{0.000000,0.000000,0.000000}%
\pgfsetstrokecolor{currentstroke}%
\pgfsetdash{}{0pt}%
\pgfpathmoveto{\pgfqpoint{4.032628in}{3.513817in}}%
\pgfpathlineto{\pgfqpoint{4.045948in}{3.502939in}}%
\pgfpathlineto{\pgfqpoint{4.059270in}{3.492157in}}%
\pgfpathlineto{\pgfqpoint{4.072594in}{3.481468in}}%
\pgfpathlineto{\pgfqpoint{4.085920in}{3.470873in}}%
\pgfpathlineto{\pgfqpoint{4.093614in}{3.496198in}}%
\pgfpathlineto{\pgfqpoint{4.101307in}{3.521979in}}%
\pgfpathlineto{\pgfqpoint{4.109001in}{3.548226in}}%
\pgfpathlineto{\pgfqpoint{4.116695in}{3.574949in}}%
\pgfpathlineto{\pgfqpoint{4.103367in}{3.586029in}}%
\pgfpathlineto{\pgfqpoint{4.090041in}{3.597203in}}%
\pgfpathlineto{\pgfqpoint{4.076717in}{3.608471in}}%
\pgfpathlineto{\pgfqpoint{4.063395in}{3.619835in}}%
\pgfpathlineto{\pgfqpoint{4.055703in}{3.592618in}}%
\pgfpathlineto{\pgfqpoint{4.048012in}{3.565882in}}%
\pgfpathlineto{\pgfqpoint{4.040320in}{3.539618in}}%
\pgfpathlineto{\pgfqpoint{4.032628in}{3.513817in}}%
\pgfpathclose%
\pgfusepath{fill}%
\end{pgfscope}%
\begin{pgfscope}%
\pgfpathrectangle{\pgfqpoint{1.150000in}{0.150000in}}{\pgfqpoint{5.700000in}{5.700000in}}%
\pgfusepath{clip}%
\pgfsetbuttcap%
\pgfsetroundjoin%
\definecolor{currentfill}{rgb}{0.412913,0.803041,0.357269}%
\pgfsetfillcolor{currentfill}%
\pgfsetfillopacity{0.700000}%
\pgfsetlinewidth{0.000000pt}%
\definecolor{currentstroke}{rgb}{0.000000,0.000000,0.000000}%
\pgfsetstrokecolor{currentstroke}%
\pgfsetdash{}{0pt}%
\pgfpathmoveto{\pgfqpoint{3.592127in}{4.439250in}}%
\pgfpathlineto{\pgfqpoint{3.605460in}{4.422382in}}%
\pgfpathlineto{\pgfqpoint{3.618791in}{4.405644in}}%
\pgfpathlineto{\pgfqpoint{3.632120in}{4.389034in}}%
\pgfpathlineto{\pgfqpoint{3.645447in}{4.372553in}}%
\pgfpathlineto{\pgfqpoint{3.653056in}{4.409581in}}%
\pgfpathlineto{\pgfqpoint{3.660663in}{4.447252in}}%
\pgfpathlineto{\pgfqpoint{3.668268in}{4.485577in}}%
\pgfpathlineto{\pgfqpoint{3.675872in}{4.524567in}}%
\pgfpathlineto{\pgfqpoint{3.662530in}{4.541631in}}%
\pgfpathlineto{\pgfqpoint{3.649186in}{4.558823in}}%
\pgfpathlineto{\pgfqpoint{3.635840in}{4.576145in}}%
\pgfpathlineto{\pgfqpoint{3.622492in}{4.593597in}}%
\pgfpathlineto{\pgfqpoint{3.614904in}{4.554013in}}%
\pgfpathlineto{\pgfqpoint{3.607314in}{4.515102in}}%
\pgfpathlineto{\pgfqpoint{3.599722in}{4.476851in}}%
\pgfpathlineto{\pgfqpoint{3.592127in}{4.439250in}}%
\pgfpathclose%
\pgfusepath{fill}%
\end{pgfscope}%
\begin{pgfscope}%
\pgfpathrectangle{\pgfqpoint{1.150000in}{0.150000in}}{\pgfqpoint{5.700000in}{5.700000in}}%
\pgfusepath{clip}%
\pgfsetbuttcap%
\pgfsetroundjoin%
\definecolor{currentfill}{rgb}{0.146180,0.515413,0.556823}%
\pgfsetfillcolor{currentfill}%
\pgfsetfillopacity{0.700000}%
\pgfsetlinewidth{0.000000pt}%
\definecolor{currentstroke}{rgb}{0.000000,0.000000,0.000000}%
\pgfsetstrokecolor{currentstroke}%
\pgfsetdash{}{0pt}%
\pgfpathmoveto{\pgfqpoint{4.063395in}{3.619835in}}%
\pgfpathlineto{\pgfqpoint{4.076717in}{3.608471in}}%
\pgfpathlineto{\pgfqpoint{4.090041in}{3.597203in}}%
\pgfpathlineto{\pgfqpoint{4.103367in}{3.586029in}}%
\pgfpathlineto{\pgfqpoint{4.116695in}{3.574949in}}%
\pgfpathlineto{\pgfqpoint{4.124389in}{3.602156in}}%
\pgfpathlineto{\pgfqpoint{4.132084in}{3.629859in}}%
\pgfpathlineto{\pgfqpoint{4.139780in}{3.658066in}}%
\pgfpathlineto{\pgfqpoint{4.147477in}{3.686787in}}%
\pgfpathlineto{\pgfqpoint{4.134145in}{3.698376in}}%
\pgfpathlineto{\pgfqpoint{4.120816in}{3.710059in}}%
\pgfpathlineto{\pgfqpoint{4.107488in}{3.721836in}}%
\pgfpathlineto{\pgfqpoint{4.094163in}{3.733709in}}%
\pgfpathlineto{\pgfqpoint{4.086470in}{3.704470in}}%
\pgfpathlineto{\pgfqpoint{4.078778in}{3.675751in}}%
\pgfpathlineto{\pgfqpoint{4.071086in}{3.647543in}}%
\pgfpathlineto{\pgfqpoint{4.063395in}{3.619835in}}%
\pgfpathclose%
\pgfusepath{fill}%
\end{pgfscope}%
\begin{pgfscope}%
\pgfpathrectangle{\pgfqpoint{1.150000in}{0.150000in}}{\pgfqpoint{5.700000in}{5.700000in}}%
\pgfusepath{clip}%
\pgfsetbuttcap%
\pgfsetroundjoin%
\definecolor{currentfill}{rgb}{0.137339,0.662252,0.515571}%
\pgfsetfillcolor{currentfill}%
\pgfsetfillopacity{0.700000}%
\pgfsetlinewidth{0.000000pt}%
\definecolor{currentstroke}{rgb}{0.000000,0.000000,0.000000}%
\pgfsetstrokecolor{currentstroke}%
\pgfsetdash{}{0pt}%
\pgfpathmoveto{\pgfqpoint{3.965073in}{4.012665in}}%
\pgfpathlineto{\pgfqpoint{3.978390in}{3.999057in}}%
\pgfpathlineto{\pgfqpoint{3.991708in}{3.985553in}}%
\pgfpathlineto{\pgfqpoint{4.005027in}{3.972152in}}%
\pgfpathlineto{\pgfqpoint{4.018347in}{3.958854in}}%
\pgfpathlineto{\pgfqpoint{4.026033in}{3.991951in}}%
\pgfpathlineto{\pgfqpoint{4.033720in}{4.025645in}}%
\pgfpathlineto{\pgfqpoint{4.041409in}{4.059945in}}%
\pgfpathlineto{\pgfqpoint{4.049100in}{4.094864in}}%
\pgfpathlineto{\pgfqpoint{4.035771in}{4.108726in}}%
\pgfpathlineto{\pgfqpoint{4.022443in}{4.122690in}}%
\pgfpathlineto{\pgfqpoint{4.009115in}{4.136758in}}%
\pgfpathlineto{\pgfqpoint{3.995788in}{4.150930in}}%
\pgfpathlineto{\pgfqpoint{3.988107in}{4.115438in}}%
\pgfpathlineto{\pgfqpoint{3.980428in}{4.080570in}}%
\pgfpathlineto{\pgfqpoint{3.972750in}{4.046316in}}%
\pgfpathlineto{\pgfqpoint{3.965073in}{4.012665in}}%
\pgfpathclose%
\pgfusepath{fill}%
\end{pgfscope}%
\begin{pgfscope}%
\pgfpathrectangle{\pgfqpoint{1.150000in}{0.150000in}}{\pgfqpoint{5.700000in}{5.700000in}}%
\pgfusepath{clip}%
\pgfsetbuttcap%
\pgfsetroundjoin%
\definecolor{currentfill}{rgb}{0.208623,0.367752,0.552675}%
\pgfsetfillcolor{currentfill}%
\pgfsetfillopacity{0.700000}%
\pgfsetlinewidth{0.000000pt}%
\definecolor{currentstroke}{rgb}{0.000000,0.000000,0.000000}%
\pgfsetstrokecolor{currentstroke}%
\pgfsetdash{}{0pt}%
\pgfpathmoveto{\pgfqpoint{3.169143in}{3.256110in}}%
\pgfpathlineto{\pgfqpoint{3.182419in}{3.243811in}}%
\pgfpathlineto{\pgfqpoint{3.195695in}{3.231642in}}%
\pgfpathlineto{\pgfqpoint{3.208971in}{3.219600in}}%
\pgfpathlineto{\pgfqpoint{3.222246in}{3.207685in}}%
\pgfpathlineto{\pgfqpoint{3.230092in}{3.224932in}}%
\pgfpathlineto{\pgfqpoint{3.237931in}{3.242434in}}%
\pgfpathlineto{\pgfqpoint{3.245763in}{3.260195in}}%
\pgfpathlineto{\pgfqpoint{3.253589in}{3.278223in}}%
\pgfpathlineto{\pgfqpoint{3.240316in}{3.290437in}}%
\pgfpathlineto{\pgfqpoint{3.227042in}{3.302779in}}%
\pgfpathlineto{\pgfqpoint{3.213768in}{3.315248in}}%
\pgfpathlineto{\pgfqpoint{3.200493in}{3.327847in}}%
\pgfpathlineto{\pgfqpoint{3.192666in}{3.309512in}}%
\pgfpathlineto{\pgfqpoint{3.184832in}{3.291448in}}%
\pgfpathlineto{\pgfqpoint{3.176991in}{3.273649in}}%
\pgfpathlineto{\pgfqpoint{3.169143in}{3.256110in}}%
\pgfpathclose%
\pgfusepath{fill}%
\end{pgfscope}%
\begin{pgfscope}%
\pgfpathrectangle{\pgfqpoint{1.150000in}{0.150000in}}{\pgfqpoint{5.700000in}{5.700000in}}%
\pgfusepath{clip}%
\pgfsetbuttcap%
\pgfsetroundjoin%
\definecolor{currentfill}{rgb}{0.179019,0.433756,0.557430}%
\pgfsetfillcolor{currentfill}%
\pgfsetfillopacity{0.700000}%
\pgfsetlinewidth{0.000000pt}%
\definecolor{currentstroke}{rgb}{0.000000,0.000000,0.000000}%
\pgfsetstrokecolor{currentstroke}%
\pgfsetdash{}{0pt}%
\pgfpathmoveto{\pgfqpoint{4.001851in}{3.415055in}}%
\pgfpathlineto{\pgfqpoint{4.015170in}{3.404641in}}%
\pgfpathlineto{\pgfqpoint{4.028490in}{3.394322in}}%
\pgfpathlineto{\pgfqpoint{4.041813in}{3.384096in}}%
\pgfpathlineto{\pgfqpoint{4.055139in}{3.373964in}}%
\pgfpathlineto{\pgfqpoint{4.062836in}{3.397551in}}%
\pgfpathlineto{\pgfqpoint{4.070531in}{3.421559in}}%
\pgfpathlineto{\pgfqpoint{4.078226in}{3.445997in}}%
\pgfpathlineto{\pgfqpoint{4.085920in}{3.470873in}}%
\pgfpathlineto{\pgfqpoint{4.072594in}{3.481468in}}%
\pgfpathlineto{\pgfqpoint{4.059270in}{3.492157in}}%
\pgfpathlineto{\pgfqpoint{4.045948in}{3.502939in}}%
\pgfpathlineto{\pgfqpoint{4.032628in}{3.513817in}}%
\pgfpathlineto{\pgfqpoint{4.024936in}{3.488469in}}%
\pgfpathlineto{\pgfqpoint{4.017242in}{3.463565in}}%
\pgfpathlineto{\pgfqpoint{4.009547in}{3.439097in}}%
\pgfpathlineto{\pgfqpoint{4.001851in}{3.415055in}}%
\pgfpathclose%
\pgfusepath{fill}%
\end{pgfscope}%
\begin{pgfscope}%
\pgfpathrectangle{\pgfqpoint{1.150000in}{0.150000in}}{\pgfqpoint{5.700000in}{5.700000in}}%
\pgfusepath{clip}%
\pgfsetbuttcap%
\pgfsetroundjoin%
\definecolor{currentfill}{rgb}{0.218130,0.347432,0.550038}%
\pgfsetfillcolor{currentfill}%
\pgfsetfillopacity{0.700000}%
\pgfsetlinewidth{0.000000pt}%
\definecolor{currentstroke}{rgb}{0.000000,0.000000,0.000000}%
\pgfsetstrokecolor{currentstroke}%
\pgfsetdash{}{0pt}%
\pgfpathmoveto{\pgfqpoint{3.581419in}{3.200386in}}%
\pgfpathlineto{\pgfqpoint{3.594701in}{3.189958in}}%
\pgfpathlineto{\pgfqpoint{3.607985in}{3.179637in}}%
\pgfpathlineto{\pgfqpoint{3.621270in}{3.169421in}}%
\pgfpathlineto{\pgfqpoint{3.634556in}{3.159312in}}%
\pgfpathlineto{\pgfqpoint{3.642319in}{3.177885in}}%
\pgfpathlineto{\pgfqpoint{3.650077in}{3.196754in}}%
\pgfpathlineto{\pgfqpoint{3.657831in}{3.215926in}}%
\pgfpathlineto{\pgfqpoint{3.665580in}{3.235406in}}%
\pgfpathlineto{\pgfqpoint{3.652295in}{3.245874in}}%
\pgfpathlineto{\pgfqpoint{3.639011in}{3.256448in}}%
\pgfpathlineto{\pgfqpoint{3.625729in}{3.267129in}}%
\pgfpathlineto{\pgfqpoint{3.612448in}{3.277916in}}%
\pgfpathlineto{\pgfqpoint{3.604698in}{3.258069in}}%
\pgfpathlineto{\pgfqpoint{3.596943in}{3.238536in}}%
\pgfpathlineto{\pgfqpoint{3.589184in}{3.219311in}}%
\pgfpathlineto{\pgfqpoint{3.581419in}{3.200386in}}%
\pgfpathclose%
\pgfusepath{fill}%
\end{pgfscope}%
\begin{pgfscope}%
\pgfpathrectangle{\pgfqpoint{1.150000in}{0.150000in}}{\pgfqpoint{5.700000in}{5.700000in}}%
\pgfusepath{clip}%
\pgfsetbuttcap%
\pgfsetroundjoin%
\definecolor{currentfill}{rgb}{0.477504,0.821444,0.318195}%
\pgfsetfillcolor{currentfill}%
\pgfsetfillopacity{0.700000}%
\pgfsetlinewidth{0.000000pt}%
\definecolor{currentstroke}{rgb}{0.000000,0.000000,0.000000}%
\pgfsetstrokecolor{currentstroke}%
\pgfsetdash{}{0pt}%
\pgfpathmoveto{\pgfqpoint{3.538777in}{4.508045in}}%
\pgfpathlineto{\pgfqpoint{3.552118in}{4.490645in}}%
\pgfpathlineto{\pgfqpoint{3.565456in}{4.473380in}}%
\pgfpathlineto{\pgfqpoint{3.578793in}{4.456249in}}%
\pgfpathlineto{\pgfqpoint{3.592127in}{4.439250in}}%
\pgfpathlineto{\pgfqpoint{3.599722in}{4.476851in}}%
\pgfpathlineto{\pgfqpoint{3.607314in}{4.515102in}}%
\pgfpathlineto{\pgfqpoint{3.614904in}{4.554013in}}%
\pgfpathlineto{\pgfqpoint{3.622492in}{4.593597in}}%
\pgfpathlineto{\pgfqpoint{3.609142in}{4.611182in}}%
\pgfpathlineto{\pgfqpoint{3.595790in}{4.628900in}}%
\pgfpathlineto{\pgfqpoint{3.582435in}{4.646753in}}%
\pgfpathlineto{\pgfqpoint{3.569078in}{4.664741in}}%
\pgfpathlineto{\pgfqpoint{3.561507in}{4.624559in}}%
\pgfpathlineto{\pgfqpoint{3.553933in}{4.585057in}}%
\pgfpathlineto{\pgfqpoint{3.546356in}{4.546223in}}%
\pgfpathlineto{\pgfqpoint{3.538777in}{4.508045in}}%
\pgfpathclose%
\pgfusepath{fill}%
\end{pgfscope}%
\begin{pgfscope}%
\pgfpathrectangle{\pgfqpoint{1.150000in}{0.150000in}}{\pgfqpoint{5.700000in}{5.700000in}}%
\pgfusepath{clip}%
\pgfsetbuttcap%
\pgfsetroundjoin%
\definecolor{currentfill}{rgb}{0.126326,0.644107,0.525311}%
\pgfsetfillcolor{currentfill}%
\pgfsetfillopacity{0.700000}%
\pgfsetlinewidth{0.000000pt}%
\definecolor{currentstroke}{rgb}{0.000000,0.000000,0.000000}%
\pgfsetstrokecolor{currentstroke}%
\pgfsetdash{}{0pt}%
\pgfpathmoveto{\pgfqpoint{4.018347in}{3.958854in}}%
\pgfpathlineto{\pgfqpoint{4.031668in}{3.945657in}}%
\pgfpathlineto{\pgfqpoint{4.044990in}{3.932562in}}%
\pgfpathlineto{\pgfqpoint{4.058313in}{3.919567in}}%
\pgfpathlineto{\pgfqpoint{4.071637in}{3.906671in}}%
\pgfpathlineto{\pgfqpoint{4.079330in}{3.939217in}}%
\pgfpathlineto{\pgfqpoint{4.087026in}{3.972352in}}%
\pgfpathlineto{\pgfqpoint{4.094723in}{4.006088in}}%
\pgfpathlineto{\pgfqpoint{4.102422in}{4.040436in}}%
\pgfpathlineto{\pgfqpoint{4.089090in}{4.053892in}}%
\pgfpathlineto{\pgfqpoint{4.075759in}{4.067448in}}%
\pgfpathlineto{\pgfqpoint{4.062429in}{4.081105in}}%
\pgfpathlineto{\pgfqpoint{4.049100in}{4.094864in}}%
\pgfpathlineto{\pgfqpoint{4.041409in}{4.059945in}}%
\pgfpathlineto{\pgfqpoint{4.033720in}{4.025645in}}%
\pgfpathlineto{\pgfqpoint{4.026033in}{3.991951in}}%
\pgfpathlineto{\pgfqpoint{4.018347in}{3.958854in}}%
\pgfpathclose%
\pgfusepath{fill}%
\end{pgfscope}%
\begin{pgfscope}%
\pgfpathrectangle{\pgfqpoint{1.150000in}{0.150000in}}{\pgfqpoint{5.700000in}{5.700000in}}%
\pgfusepath{clip}%
\pgfsetbuttcap%
\pgfsetroundjoin%
\definecolor{currentfill}{rgb}{0.129933,0.559582,0.551864}%
\pgfsetfillcolor{currentfill}%
\pgfsetfillopacity{0.700000}%
\pgfsetlinewidth{0.000000pt}%
\definecolor{currentstroke}{rgb}{0.000000,0.000000,0.000000}%
\pgfsetstrokecolor{currentstroke}%
\pgfsetdash{}{0pt}%
\pgfpathmoveto{\pgfqpoint{4.094163in}{3.733709in}}%
\pgfpathlineto{\pgfqpoint{4.107488in}{3.721836in}}%
\pgfpathlineto{\pgfqpoint{4.120816in}{3.710059in}}%
\pgfpathlineto{\pgfqpoint{4.134145in}{3.698376in}}%
\pgfpathlineto{\pgfqpoint{4.147477in}{3.686787in}}%
\pgfpathlineto{\pgfqpoint{4.155175in}{3.716033in}}%
\pgfpathlineto{\pgfqpoint{4.162875in}{3.745814in}}%
\pgfpathlineto{\pgfqpoint{4.170577in}{3.776139in}}%
\pgfpathlineto{\pgfqpoint{4.178281in}{3.807021in}}%
\pgfpathlineto{\pgfqpoint{4.164945in}{3.819141in}}%
\pgfpathlineto{\pgfqpoint{4.151610in}{3.831357in}}%
\pgfpathlineto{\pgfqpoint{4.138278in}{3.843667in}}%
\pgfpathlineto{\pgfqpoint{4.124947in}{3.856074in}}%
\pgfpathlineto{\pgfqpoint{4.117248in}{3.824651in}}%
\pgfpathlineto{\pgfqpoint{4.109552in}{3.793789in}}%
\pgfpathlineto{\pgfqpoint{4.101857in}{3.763479in}}%
\pgfpathlineto{\pgfqpoint{4.094163in}{3.733709in}}%
\pgfpathclose%
\pgfusepath{fill}%
\end{pgfscope}%
\begin{pgfscope}%
\pgfpathrectangle{\pgfqpoint{1.150000in}{0.150000in}}{\pgfqpoint{5.700000in}{5.700000in}}%
\pgfusepath{clip}%
\pgfsetbuttcap%
\pgfsetroundjoin%
\definecolor{currentfill}{rgb}{0.203063,0.379716,0.553925}%
\pgfsetfillcolor{currentfill}%
\pgfsetfillopacity{0.700000}%
\pgfsetlinewidth{0.000000pt}%
\definecolor{currentstroke}{rgb}{0.000000,0.000000,0.000000}%
\pgfsetstrokecolor{currentstroke}%
\pgfsetdash{}{0pt}%
\pgfpathmoveto{\pgfqpoint{3.886968in}{3.276233in}}%
\pgfpathlineto{\pgfqpoint{3.900278in}{3.266297in}}%
\pgfpathlineto{\pgfqpoint{3.913589in}{3.256458in}}%
\pgfpathlineto{\pgfqpoint{3.926903in}{3.246715in}}%
\pgfpathlineto{\pgfqpoint{3.940220in}{3.237067in}}%
\pgfpathlineto{\pgfqpoint{3.947932in}{3.257997in}}%
\pgfpathlineto{\pgfqpoint{3.955641in}{3.279288in}}%
\pgfpathlineto{\pgfqpoint{3.963347in}{3.300948in}}%
\pgfpathlineto{\pgfqpoint{3.971052in}{3.322984in}}%
\pgfpathlineto{\pgfqpoint{3.957736in}{3.333052in}}%
\pgfpathlineto{\pgfqpoint{3.944423in}{3.343215in}}%
\pgfpathlineto{\pgfqpoint{3.931112in}{3.353474in}}%
\pgfpathlineto{\pgfqpoint{3.917803in}{3.363830in}}%
\pgfpathlineto{\pgfqpoint{3.910098in}{3.341366in}}%
\pgfpathlineto{\pgfqpoint{3.902391in}{3.319284in}}%
\pgfpathlineto{\pgfqpoint{3.894681in}{3.297575in}}%
\pgfpathlineto{\pgfqpoint{3.886968in}{3.276233in}}%
\pgfpathclose%
\pgfusepath{fill}%
\end{pgfscope}%
\begin{pgfscope}%
\pgfpathrectangle{\pgfqpoint{1.150000in}{0.150000in}}{\pgfqpoint{5.700000in}{5.700000in}}%
\pgfusepath{clip}%
\pgfsetbuttcap%
\pgfsetroundjoin%
\definecolor{currentfill}{rgb}{0.210503,0.363727,0.552206}%
\pgfsetfillcolor{currentfill}%
\pgfsetfillopacity{0.700000}%
\pgfsetlinewidth{0.000000pt}%
\definecolor{currentstroke}{rgb}{0.000000,0.000000,0.000000}%
\pgfsetstrokecolor{currentstroke}%
\pgfsetdash{}{0pt}%
\pgfpathmoveto{\pgfqpoint{3.802870in}{3.233493in}}%
\pgfpathlineto{\pgfqpoint{3.816171in}{3.223565in}}%
\pgfpathlineto{\pgfqpoint{3.829474in}{3.213736in}}%
\pgfpathlineto{\pgfqpoint{3.842780in}{3.204004in}}%
\pgfpathlineto{\pgfqpoint{3.856088in}{3.194371in}}%
\pgfpathlineto{\pgfqpoint{3.863813in}{3.214325in}}%
\pgfpathlineto{\pgfqpoint{3.871535in}{3.234616in}}%
\pgfpathlineto{\pgfqpoint{3.879253in}{3.255249in}}%
\pgfpathlineto{\pgfqpoint{3.886968in}{3.276233in}}%
\pgfpathlineto{\pgfqpoint{3.873661in}{3.286266in}}%
\pgfpathlineto{\pgfqpoint{3.860357in}{3.296396in}}%
\pgfpathlineto{\pgfqpoint{3.847054in}{3.306625in}}%
\pgfpathlineto{\pgfqpoint{3.833754in}{3.316953in}}%
\pgfpathlineto{\pgfqpoint{3.826038in}{3.295562in}}%
\pgfpathlineto{\pgfqpoint{3.818319in}{3.274526in}}%
\pgfpathlineto{\pgfqpoint{3.810596in}{3.253839in}}%
\pgfpathlineto{\pgfqpoint{3.802870in}{3.233493in}}%
\pgfpathclose%
\pgfusepath{fill}%
\end{pgfscope}%
\begin{pgfscope}%
\pgfpathrectangle{\pgfqpoint{1.150000in}{0.150000in}}{\pgfqpoint{5.700000in}{5.700000in}}%
\pgfusepath{clip}%
\pgfsetbuttcap%
\pgfsetroundjoin%
\definecolor{currentfill}{rgb}{0.221989,0.339161,0.548752}%
\pgfsetfillcolor{currentfill}%
\pgfsetfillopacity{0.700000}%
\pgfsetlinewidth{0.000000pt}%
\definecolor{currentstroke}{rgb}{0.000000,0.000000,0.000000}%
\pgfsetstrokecolor{currentstroke}%
\pgfsetdash{}{0pt}%
\pgfpathmoveto{\pgfqpoint{3.359769in}{3.184955in}}%
\pgfpathlineto{\pgfqpoint{3.373043in}{3.173837in}}%
\pgfpathlineto{\pgfqpoint{3.386317in}{3.162836in}}%
\pgfpathlineto{\pgfqpoint{3.399592in}{3.151951in}}%
\pgfpathlineto{\pgfqpoint{3.412868in}{3.141181in}}%
\pgfpathlineto{\pgfqpoint{3.420681in}{3.158546in}}%
\pgfpathlineto{\pgfqpoint{3.428488in}{3.176172in}}%
\pgfpathlineto{\pgfqpoint{3.436289in}{3.194066in}}%
\pgfpathlineto{\pgfqpoint{3.444084in}{3.212233in}}%
\pgfpathlineto{\pgfqpoint{3.430810in}{3.223322in}}%
\pgfpathlineto{\pgfqpoint{3.417537in}{3.234526in}}%
\pgfpathlineto{\pgfqpoint{3.404265in}{3.245845in}}%
\pgfpathlineto{\pgfqpoint{3.390993in}{3.257282in}}%
\pgfpathlineto{\pgfqpoint{3.383196in}{3.238788in}}%
\pgfpathlineto{\pgfqpoint{3.375393in}{3.220573in}}%
\pgfpathlineto{\pgfqpoint{3.367584in}{3.202631in}}%
\pgfpathlineto{\pgfqpoint{3.359769in}{3.184955in}}%
\pgfpathclose%
\pgfusepath{fill}%
\end{pgfscope}%
\begin{pgfscope}%
\pgfpathrectangle{\pgfqpoint{1.150000in}{0.150000in}}{\pgfqpoint{5.700000in}{5.700000in}}%
\pgfusepath{clip}%
\pgfsetbuttcap%
\pgfsetroundjoin%
\definecolor{currentfill}{rgb}{0.223925,0.334994,0.548053}%
\pgfsetfillcolor{currentfill}%
\pgfsetfillopacity{0.700000}%
\pgfsetlinewidth{0.000000pt}%
\definecolor{currentstroke}{rgb}{0.000000,0.000000,0.000000}%
\pgfsetstrokecolor{currentstroke}%
\pgfsetdash{}{0pt}%
\pgfpathmoveto{\pgfqpoint{3.497187in}{3.169014in}}%
\pgfpathlineto{\pgfqpoint{3.510466in}{3.158488in}}%
\pgfpathlineto{\pgfqpoint{3.523745in}{3.148073in}}%
\pgfpathlineto{\pgfqpoint{3.537027in}{3.137767in}}%
\pgfpathlineto{\pgfqpoint{3.550309in}{3.127569in}}%
\pgfpathlineto{\pgfqpoint{3.558095in}{3.145354in}}%
\pgfpathlineto{\pgfqpoint{3.565875in}{3.163414in}}%
\pgfpathlineto{\pgfqpoint{3.573649in}{3.181756in}}%
\pgfpathlineto{\pgfqpoint{3.581419in}{3.200386in}}%
\pgfpathlineto{\pgfqpoint{3.568139in}{3.210923in}}%
\pgfpathlineto{\pgfqpoint{3.554859in}{3.221567in}}%
\pgfpathlineto{\pgfqpoint{3.541581in}{3.232321in}}%
\pgfpathlineto{\pgfqpoint{3.528305in}{3.243186in}}%
\pgfpathlineto{\pgfqpoint{3.520533in}{3.224209in}}%
\pgfpathlineto{\pgfqpoint{3.512757in}{3.205526in}}%
\pgfpathlineto{\pgfqpoint{3.504975in}{3.187130in}}%
\pgfpathlineto{\pgfqpoint{3.497187in}{3.169014in}}%
\pgfpathclose%
\pgfusepath{fill}%
\end{pgfscope}%
\begin{pgfscope}%
\pgfpathrectangle{\pgfqpoint{1.150000in}{0.150000in}}{\pgfqpoint{5.700000in}{5.700000in}}%
\pgfusepath{clip}%
\pgfsetbuttcap%
\pgfsetroundjoin%
\definecolor{currentfill}{rgb}{0.194100,0.399323,0.555565}%
\pgfsetfillcolor{currentfill}%
\pgfsetfillopacity{0.700000}%
\pgfsetlinewidth{0.000000pt}%
\definecolor{currentstroke}{rgb}{0.000000,0.000000,0.000000}%
\pgfsetstrokecolor{currentstroke}%
\pgfsetdash{}{0pt}%
\pgfpathmoveto{\pgfqpoint{3.971052in}{3.322984in}}%
\pgfpathlineto{\pgfqpoint{3.984370in}{3.313011in}}%
\pgfpathlineto{\pgfqpoint{3.997691in}{3.303132in}}%
\pgfpathlineto{\pgfqpoint{4.011015in}{3.293348in}}%
\pgfpathlineto{\pgfqpoint{4.024341in}{3.283656in}}%
\pgfpathlineto{\pgfqpoint{4.032043in}{3.305644in}}%
\pgfpathlineto{\pgfqpoint{4.039743in}{3.328019in}}%
\pgfpathlineto{\pgfqpoint{4.047442in}{3.350789in}}%
\pgfpathlineto{\pgfqpoint{4.055139in}{3.373964in}}%
\pgfpathlineto{\pgfqpoint{4.041813in}{3.384096in}}%
\pgfpathlineto{\pgfqpoint{4.028490in}{3.394322in}}%
\pgfpathlineto{\pgfqpoint{4.015170in}{3.404641in}}%
\pgfpathlineto{\pgfqpoint{4.001851in}{3.415055in}}%
\pgfpathlineto{\pgfqpoint{3.994154in}{3.391432in}}%
\pgfpathlineto{\pgfqpoint{3.986455in}{3.368217in}}%
\pgfpathlineto{\pgfqpoint{3.978754in}{3.345404in}}%
\pgfpathlineto{\pgfqpoint{3.971052in}{3.322984in}}%
\pgfpathclose%
\pgfusepath{fill}%
\end{pgfscope}%
\begin{pgfscope}%
\pgfpathrectangle{\pgfqpoint{1.150000in}{0.150000in}}{\pgfqpoint{5.700000in}{5.700000in}}%
\pgfusepath{clip}%
\pgfsetbuttcap%
\pgfsetroundjoin%
\definecolor{currentfill}{rgb}{0.218130,0.347432,0.550038}%
\pgfsetfillcolor{currentfill}%
\pgfsetfillopacity{0.700000}%
\pgfsetlinewidth{0.000000pt}%
\definecolor{currentstroke}{rgb}{0.000000,0.000000,0.000000}%
\pgfsetstrokecolor{currentstroke}%
\pgfsetdash{}{0pt}%
\pgfpathmoveto{\pgfqpoint{3.718738in}{3.194574in}}%
\pgfpathlineto{\pgfqpoint{3.732032in}{3.184623in}}%
\pgfpathlineto{\pgfqpoint{3.745328in}{3.174773in}}%
\pgfpathlineto{\pgfqpoint{3.758627in}{3.165024in}}%
\pgfpathlineto{\pgfqpoint{3.771928in}{3.155375in}}%
\pgfpathlineto{\pgfqpoint{3.779669in}{3.174429in}}%
\pgfpathlineto{\pgfqpoint{3.787406in}{3.193795in}}%
\pgfpathlineto{\pgfqpoint{3.795140in}{3.213481in}}%
\pgfpathlineto{\pgfqpoint{3.802870in}{3.233493in}}%
\pgfpathlineto{\pgfqpoint{3.789571in}{3.243521in}}%
\pgfpathlineto{\pgfqpoint{3.776274in}{3.253649in}}%
\pgfpathlineto{\pgfqpoint{3.762979in}{3.263878in}}%
\pgfpathlineto{\pgfqpoint{3.749686in}{3.274208in}}%
\pgfpathlineto{\pgfqpoint{3.741955in}{3.253809in}}%
\pgfpathlineto{\pgfqpoint{3.734220in}{3.233742in}}%
\pgfpathlineto{\pgfqpoint{3.726481in}{3.213999in}}%
\pgfpathlineto{\pgfqpoint{3.718738in}{3.194574in}}%
\pgfpathclose%
\pgfusepath{fill}%
\end{pgfscope}%
\begin{pgfscope}%
\pgfpathrectangle{\pgfqpoint{1.150000in}{0.150000in}}{\pgfqpoint{5.700000in}{5.700000in}}%
\pgfusepath{clip}%
\pgfsetbuttcap%
\pgfsetroundjoin%
\definecolor{currentfill}{rgb}{0.168126,0.459988,0.558082}%
\pgfsetfillcolor{currentfill}%
\pgfsetfillopacity{0.700000}%
\pgfsetlinewidth{0.000000pt}%
\definecolor{currentstroke}{rgb}{0.000000,0.000000,0.000000}%
\pgfsetstrokecolor{currentstroke}%
\pgfsetdash{}{0pt}%
\pgfpathmoveto{\pgfqpoint{4.085920in}{3.470873in}}%
\pgfpathlineto{\pgfqpoint{4.099249in}{3.460371in}}%
\pgfpathlineto{\pgfqpoint{4.112580in}{3.449962in}}%
\pgfpathlineto{\pgfqpoint{4.125914in}{3.439645in}}%
\pgfpathlineto{\pgfqpoint{4.139250in}{3.429419in}}%
\pgfpathlineto{\pgfqpoint{4.146945in}{3.454267in}}%
\pgfpathlineto{\pgfqpoint{4.154640in}{3.479567in}}%
\pgfpathlineto{\pgfqpoint{4.162335in}{3.505327in}}%
\pgfpathlineto{\pgfqpoint{4.170031in}{3.531556in}}%
\pgfpathlineto{\pgfqpoint{4.156693in}{3.542266in}}%
\pgfpathlineto{\pgfqpoint{4.143358in}{3.553068in}}%
\pgfpathlineto{\pgfqpoint{4.130025in}{3.563962in}}%
\pgfpathlineto{\pgfqpoint{4.116695in}{3.574949in}}%
\pgfpathlineto{\pgfqpoint{4.109001in}{3.548226in}}%
\pgfpathlineto{\pgfqpoint{4.101307in}{3.521979in}}%
\pgfpathlineto{\pgfqpoint{4.093614in}{3.496198in}}%
\pgfpathlineto{\pgfqpoint{4.085920in}{3.470873in}}%
\pgfpathclose%
\pgfusepath{fill}%
\end{pgfscope}%
\begin{pgfscope}%
\pgfpathrectangle{\pgfqpoint{1.150000in}{0.150000in}}{\pgfqpoint{5.700000in}{5.700000in}}%
\pgfusepath{clip}%
\pgfsetbuttcap%
\pgfsetroundjoin%
\definecolor{currentfill}{rgb}{0.153364,0.497000,0.557724}%
\pgfsetfillcolor{currentfill}%
\pgfsetfillopacity{0.700000}%
\pgfsetlinewidth{0.000000pt}%
\definecolor{currentstroke}{rgb}{0.000000,0.000000,0.000000}%
\pgfsetstrokecolor{currentstroke}%
\pgfsetdash{}{0pt}%
\pgfpathmoveto{\pgfqpoint{4.116695in}{3.574949in}}%
\pgfpathlineto{\pgfqpoint{4.130025in}{3.563962in}}%
\pgfpathlineto{\pgfqpoint{4.143358in}{3.553068in}}%
\pgfpathlineto{\pgfqpoint{4.156693in}{3.542266in}}%
\pgfpathlineto{\pgfqpoint{4.170031in}{3.531556in}}%
\pgfpathlineto{\pgfqpoint{4.177727in}{3.558265in}}%
\pgfpathlineto{\pgfqpoint{4.185425in}{3.585463in}}%
\pgfpathlineto{\pgfqpoint{4.193123in}{3.613160in}}%
\pgfpathlineto{\pgfqpoint{4.200823in}{3.641365in}}%
\pgfpathlineto{\pgfqpoint{4.187484in}{3.652582in}}%
\pgfpathlineto{\pgfqpoint{4.174146in}{3.663891in}}%
\pgfpathlineto{\pgfqpoint{4.160810in}{3.675293in}}%
\pgfpathlineto{\pgfqpoint{4.147477in}{3.686787in}}%
\pgfpathlineto{\pgfqpoint{4.139780in}{3.658066in}}%
\pgfpathlineto{\pgfqpoint{4.132084in}{3.629859in}}%
\pgfpathlineto{\pgfqpoint{4.124389in}{3.602156in}}%
\pgfpathlineto{\pgfqpoint{4.116695in}{3.574949in}}%
\pgfpathclose%
\pgfusepath{fill}%
\end{pgfscope}%
\begin{pgfscope}%
\pgfpathrectangle{\pgfqpoint{1.150000in}{0.150000in}}{\pgfqpoint{5.700000in}{5.700000in}}%
\pgfusepath{clip}%
\pgfsetbuttcap%
\pgfsetroundjoin%
\definecolor{currentfill}{rgb}{0.218130,0.347432,0.550038}%
\pgfsetfillcolor{currentfill}%
\pgfsetfillopacity{0.700000}%
\pgfsetlinewidth{0.000000pt}%
\definecolor{currentstroke}{rgb}{0.000000,0.000000,0.000000}%
\pgfsetstrokecolor{currentstroke}%
\pgfsetdash{}{0pt}%
\pgfpathmoveto{\pgfqpoint{3.222246in}{3.207685in}}%
\pgfpathlineto{\pgfqpoint{3.235520in}{3.195896in}}%
\pgfpathlineto{\pgfqpoint{3.248795in}{3.184232in}}%
\pgfpathlineto{\pgfqpoint{3.262070in}{3.172691in}}%
\pgfpathlineto{\pgfqpoint{3.275344in}{3.161273in}}%
\pgfpathlineto{\pgfqpoint{3.283188in}{3.178229in}}%
\pgfpathlineto{\pgfqpoint{3.291024in}{3.195434in}}%
\pgfpathlineto{\pgfqpoint{3.298855in}{3.212894in}}%
\pgfpathlineto{\pgfqpoint{3.306678in}{3.230615in}}%
\pgfpathlineto{\pgfqpoint{3.293406in}{3.242332in}}%
\pgfpathlineto{\pgfqpoint{3.280134in}{3.254171in}}%
\pgfpathlineto{\pgfqpoint{3.266861in}{3.266135in}}%
\pgfpathlineto{\pgfqpoint{3.253589in}{3.278223in}}%
\pgfpathlineto{\pgfqpoint{3.245763in}{3.260195in}}%
\pgfpathlineto{\pgfqpoint{3.237931in}{3.242434in}}%
\pgfpathlineto{\pgfqpoint{3.230092in}{3.224932in}}%
\pgfpathlineto{\pgfqpoint{3.222246in}{3.207685in}}%
\pgfpathclose%
\pgfusepath{fill}%
\end{pgfscope}%
\begin{pgfscope}%
\pgfpathrectangle{\pgfqpoint{1.150000in}{0.150000in}}{\pgfqpoint{5.700000in}{5.700000in}}%
\pgfusepath{clip}%
\pgfsetbuttcap%
\pgfsetroundjoin%
\definecolor{currentfill}{rgb}{0.555484,0.840254,0.269281}%
\pgfsetfillcolor{currentfill}%
\pgfsetfillopacity{0.700000}%
\pgfsetlinewidth{0.000000pt}%
\definecolor{currentstroke}{rgb}{0.000000,0.000000,0.000000}%
\pgfsetstrokecolor{currentstroke}%
\pgfsetdash{}{0pt}%
\pgfpathmoveto{\pgfqpoint{3.485389in}{4.579016in}}%
\pgfpathlineto{\pgfqpoint{3.498740in}{4.561065in}}%
\pgfpathlineto{\pgfqpoint{3.512088in}{4.543254in}}%
\pgfpathlineto{\pgfqpoint{3.525433in}{4.525581in}}%
\pgfpathlineto{\pgfqpoint{3.538777in}{4.508045in}}%
\pgfpathlineto{\pgfqpoint{3.546356in}{4.546223in}}%
\pgfpathlineto{\pgfqpoint{3.553933in}{4.585057in}}%
\pgfpathlineto{\pgfqpoint{3.561507in}{4.624559in}}%
\pgfpathlineto{\pgfqpoint{3.569078in}{4.664741in}}%
\pgfpathlineto{\pgfqpoint{3.555719in}{4.682866in}}%
\pgfpathlineto{\pgfqpoint{3.542356in}{4.701129in}}%
\pgfpathlineto{\pgfqpoint{3.528991in}{4.719531in}}%
\pgfpathlineto{\pgfqpoint{3.515623in}{4.738075in}}%
\pgfpathlineto{\pgfqpoint{3.508070in}{4.697291in}}%
\pgfpathlineto{\pgfqpoint{3.500513in}{4.657195in}}%
\pgfpathlineto{\pgfqpoint{3.492952in}{4.617774in}}%
\pgfpathlineto{\pgfqpoint{3.485389in}{4.579016in}}%
\pgfpathclose%
\pgfusepath{fill}%
\end{pgfscope}%
\begin{pgfscope}%
\pgfpathrectangle{\pgfqpoint{1.150000in}{0.150000in}}{\pgfqpoint{5.700000in}{5.700000in}}%
\pgfusepath{clip}%
\pgfsetbuttcap%
\pgfsetroundjoin%
\definecolor{currentfill}{rgb}{0.203063,0.379716,0.553925}%
\pgfsetfillcolor{currentfill}%
\pgfsetfillopacity{0.700000}%
\pgfsetlinewidth{0.000000pt}%
\definecolor{currentstroke}{rgb}{0.000000,0.000000,0.000000}%
\pgfsetstrokecolor{currentstroke}%
\pgfsetdash{}{0pt}%
\pgfpathmoveto{\pgfqpoint{3.031409in}{3.289387in}}%
\pgfpathlineto{\pgfqpoint{3.044697in}{3.276293in}}%
\pgfpathlineto{\pgfqpoint{3.057983in}{3.263338in}}%
\pgfpathlineto{\pgfqpoint{3.071269in}{3.250520in}}%
\pgfpathlineto{\pgfqpoint{3.084552in}{3.237838in}}%
\pgfpathlineto{\pgfqpoint{3.092432in}{3.254662in}}%
\pgfpathlineto{\pgfqpoint{3.100305in}{3.271730in}}%
\pgfpathlineto{\pgfqpoint{3.108170in}{3.289048in}}%
\pgfpathlineto{\pgfqpoint{3.116028in}{3.306619in}}%
\pgfpathlineto{\pgfqpoint{3.102746in}{3.319581in}}%
\pgfpathlineto{\pgfqpoint{3.089464in}{3.332679in}}%
\pgfpathlineto{\pgfqpoint{3.076179in}{3.345915in}}%
\pgfpathlineto{\pgfqpoint{3.062894in}{3.359289in}}%
\pgfpathlineto{\pgfqpoint{3.055034in}{3.341430in}}%
\pgfpathlineto{\pgfqpoint{3.047167in}{3.323829in}}%
\pgfpathlineto{\pgfqpoint{3.039292in}{3.306484in}}%
\pgfpathlineto{\pgfqpoint{3.031409in}{3.289387in}}%
\pgfpathclose%
\pgfusepath{fill}%
\end{pgfscope}%
\begin{pgfscope}%
\pgfpathrectangle{\pgfqpoint{1.150000in}{0.150000in}}{\pgfqpoint{5.700000in}{5.700000in}}%
\pgfusepath{clip}%
\pgfsetbuttcap%
\pgfsetroundjoin%
\definecolor{currentfill}{rgb}{0.120638,0.625828,0.533488}%
\pgfsetfillcolor{currentfill}%
\pgfsetfillopacity{0.700000}%
\pgfsetlinewidth{0.000000pt}%
\definecolor{currentstroke}{rgb}{0.000000,0.000000,0.000000}%
\pgfsetstrokecolor{currentstroke}%
\pgfsetdash{}{0pt}%
\pgfpathmoveto{\pgfqpoint{4.071637in}{3.906671in}}%
\pgfpathlineto{\pgfqpoint{4.084962in}{3.893875in}}%
\pgfpathlineto{\pgfqpoint{4.098289in}{3.881177in}}%
\pgfpathlineto{\pgfqpoint{4.111617in}{3.868577in}}%
\pgfpathlineto{\pgfqpoint{4.124947in}{3.856074in}}%
\pgfpathlineto{\pgfqpoint{4.132647in}{3.888070in}}%
\pgfpathlineto{\pgfqpoint{4.140349in}{3.920648in}}%
\pgfpathlineto{\pgfqpoint{4.148054in}{3.953822in}}%
\pgfpathlineto{\pgfqpoint{4.155762in}{3.987600in}}%
\pgfpathlineto{\pgfqpoint{4.142425in}{4.000662in}}%
\pgfpathlineto{\pgfqpoint{4.129090in}{4.013822in}}%
\pgfpathlineto{\pgfqpoint{4.115756in}{4.027079in}}%
\pgfpathlineto{\pgfqpoint{4.102422in}{4.040436in}}%
\pgfpathlineto{\pgfqpoint{4.094723in}{4.006088in}}%
\pgfpathlineto{\pgfqpoint{4.087026in}{3.972352in}}%
\pgfpathlineto{\pgfqpoint{4.079330in}{3.939217in}}%
\pgfpathlineto{\pgfqpoint{4.071637in}{3.906671in}}%
\pgfpathclose%
\pgfusepath{fill}%
\end{pgfscope}%
\begin{pgfscope}%
\pgfpathrectangle{\pgfqpoint{1.150000in}{0.150000in}}{\pgfqpoint{5.700000in}{5.700000in}}%
\pgfusepath{clip}%
\pgfsetbuttcap%
\pgfsetroundjoin%
\definecolor{currentfill}{rgb}{0.183898,0.422383,0.556944}%
\pgfsetfillcolor{currentfill}%
\pgfsetfillopacity{0.700000}%
\pgfsetlinewidth{0.000000pt}%
\definecolor{currentstroke}{rgb}{0.000000,0.000000,0.000000}%
\pgfsetstrokecolor{currentstroke}%
\pgfsetdash{}{0pt}%
\pgfpathmoveto{\pgfqpoint{4.055139in}{3.373964in}}%
\pgfpathlineto{\pgfqpoint{4.068467in}{3.363924in}}%
\pgfpathlineto{\pgfqpoint{4.081798in}{3.353977in}}%
\pgfpathlineto{\pgfqpoint{4.095132in}{3.344122in}}%
\pgfpathlineto{\pgfqpoint{4.108469in}{3.334358in}}%
\pgfpathlineto{\pgfqpoint{4.116165in}{3.357491in}}%
\pgfpathlineto{\pgfqpoint{4.123860in}{3.381040in}}%
\pgfpathlineto{\pgfqpoint{4.131555in}{3.405013in}}%
\pgfpathlineto{\pgfqpoint{4.139250in}{3.429419in}}%
\pgfpathlineto{\pgfqpoint{4.125914in}{3.439645in}}%
\pgfpathlineto{\pgfqpoint{4.112580in}{3.449962in}}%
\pgfpathlineto{\pgfqpoint{4.099249in}{3.460371in}}%
\pgfpathlineto{\pgfqpoint{4.085920in}{3.470873in}}%
\pgfpathlineto{\pgfqpoint{4.078226in}{3.445997in}}%
\pgfpathlineto{\pgfqpoint{4.070531in}{3.421559in}}%
\pgfpathlineto{\pgfqpoint{4.062836in}{3.397551in}}%
\pgfpathlineto{\pgfqpoint{4.055139in}{3.373964in}}%
\pgfpathclose%
\pgfusepath{fill}%
\end{pgfscope}%
\begin{pgfscope}%
\pgfpathrectangle{\pgfqpoint{1.150000in}{0.150000in}}{\pgfqpoint{5.700000in}{5.700000in}}%
\pgfusepath{clip}%
\pgfsetbuttcap%
\pgfsetroundjoin%
\definecolor{currentfill}{rgb}{0.225863,0.330805,0.547314}%
\pgfsetfillcolor{currentfill}%
\pgfsetfillopacity{0.700000}%
\pgfsetlinewidth{0.000000pt}%
\definecolor{currentstroke}{rgb}{0.000000,0.000000,0.000000}%
\pgfsetstrokecolor{currentstroke}%
\pgfsetdash{}{0pt}%
\pgfpathmoveto{\pgfqpoint{3.634556in}{3.159312in}}%
\pgfpathlineto{\pgfqpoint{3.647845in}{3.149307in}}%
\pgfpathlineto{\pgfqpoint{3.661135in}{3.139406in}}%
\pgfpathlineto{\pgfqpoint{3.674428in}{3.129608in}}%
\pgfpathlineto{\pgfqpoint{3.687722in}{3.119913in}}%
\pgfpathlineto{\pgfqpoint{3.695483in}{3.138136in}}%
\pgfpathlineto{\pgfqpoint{3.703239in}{3.156649in}}%
\pgfpathlineto{\pgfqpoint{3.710990in}{3.175460in}}%
\pgfpathlineto{\pgfqpoint{3.718738in}{3.194574in}}%
\pgfpathlineto{\pgfqpoint{3.705446in}{3.204628in}}%
\pgfpathlineto{\pgfqpoint{3.692155in}{3.214783in}}%
\pgfpathlineto{\pgfqpoint{3.678867in}{3.225043in}}%
\pgfpathlineto{\pgfqpoint{3.665580in}{3.235406in}}%
\pgfpathlineto{\pgfqpoint{3.657831in}{3.215926in}}%
\pgfpathlineto{\pgfqpoint{3.650077in}{3.196754in}}%
\pgfpathlineto{\pgfqpoint{3.642319in}{3.177885in}}%
\pgfpathlineto{\pgfqpoint{3.634556in}{3.159312in}}%
\pgfpathclose%
\pgfusepath{fill}%
\end{pgfscope}%
\begin{pgfscope}%
\pgfpathrectangle{\pgfqpoint{1.150000in}{0.150000in}}{\pgfqpoint{5.700000in}{5.700000in}}%
\pgfusepath{clip}%
\pgfsetbuttcap%
\pgfsetroundjoin%
\definecolor{currentfill}{rgb}{0.136408,0.541173,0.554483}%
\pgfsetfillcolor{currentfill}%
\pgfsetfillopacity{0.700000}%
\pgfsetlinewidth{0.000000pt}%
\definecolor{currentstroke}{rgb}{0.000000,0.000000,0.000000}%
\pgfsetstrokecolor{currentstroke}%
\pgfsetdash{}{0pt}%
\pgfpathmoveto{\pgfqpoint{4.147477in}{3.686787in}}%
\pgfpathlineto{\pgfqpoint{4.160810in}{3.675293in}}%
\pgfpathlineto{\pgfqpoint{4.174146in}{3.663891in}}%
\pgfpathlineto{\pgfqpoint{4.187484in}{3.652582in}}%
\pgfpathlineto{\pgfqpoint{4.200823in}{3.641365in}}%
\pgfpathlineto{\pgfqpoint{4.208525in}{3.670088in}}%
\pgfpathlineto{\pgfqpoint{4.216229in}{3.699341in}}%
\pgfpathlineto{\pgfqpoint{4.223935in}{3.729132in}}%
\pgfpathlineto{\pgfqpoint{4.231644in}{3.759473in}}%
\pgfpathlineto{\pgfqpoint{4.218300in}{3.771221in}}%
\pgfpathlineto{\pgfqpoint{4.204958in}{3.783061in}}%
\pgfpathlineto{\pgfqpoint{4.191619in}{3.794994in}}%
\pgfpathlineto{\pgfqpoint{4.178281in}{3.807021in}}%
\pgfpathlineto{\pgfqpoint{4.170577in}{3.776139in}}%
\pgfpathlineto{\pgfqpoint{4.162875in}{3.745814in}}%
\pgfpathlineto{\pgfqpoint{4.155175in}{3.716033in}}%
\pgfpathlineto{\pgfqpoint{4.147477in}{3.686787in}}%
\pgfpathclose%
\pgfusepath{fill}%
\end{pgfscope}%
\begin{pgfscope}%
\pgfpathrectangle{\pgfqpoint{1.150000in}{0.150000in}}{\pgfqpoint{5.700000in}{5.700000in}}%
\pgfusepath{clip}%
\pgfsetbuttcap%
\pgfsetroundjoin%
\definecolor{currentfill}{rgb}{0.212395,0.359683,0.551710}%
\pgfsetfillcolor{currentfill}%
\pgfsetfillopacity{0.700000}%
\pgfsetlinewidth{0.000000pt}%
\definecolor{currentstroke}{rgb}{0.000000,0.000000,0.000000}%
\pgfsetstrokecolor{currentstroke}%
\pgfsetdash{}{0pt}%
\pgfpathmoveto{\pgfqpoint{3.084552in}{3.237838in}}%
\pgfpathlineto{\pgfqpoint{3.097835in}{3.225291in}}%
\pgfpathlineto{\pgfqpoint{3.111117in}{3.212877in}}%
\pgfpathlineto{\pgfqpoint{3.124398in}{3.200596in}}%
\pgfpathlineto{\pgfqpoint{3.137678in}{3.188446in}}%
\pgfpathlineto{\pgfqpoint{3.145555in}{3.204999in}}%
\pgfpathlineto{\pgfqpoint{3.153425in}{3.221790in}}%
\pgfpathlineto{\pgfqpoint{3.161287in}{3.238826in}}%
\pgfpathlineto{\pgfqpoint{3.169143in}{3.256110in}}%
\pgfpathlineto{\pgfqpoint{3.155865in}{3.268539in}}%
\pgfpathlineto{\pgfqpoint{3.142587in}{3.281099in}}%
\pgfpathlineto{\pgfqpoint{3.129308in}{3.293793in}}%
\pgfpathlineto{\pgfqpoint{3.116028in}{3.306619in}}%
\pgfpathlineto{\pgfqpoint{3.108170in}{3.289048in}}%
\pgfpathlineto{\pgfqpoint{3.100305in}{3.271730in}}%
\pgfpathlineto{\pgfqpoint{3.092432in}{3.254662in}}%
\pgfpathlineto{\pgfqpoint{3.084552in}{3.237838in}}%
\pgfpathclose%
\pgfusepath{fill}%
\end{pgfscope}%
\begin{pgfscope}%
\pgfpathrectangle{\pgfqpoint{1.150000in}{0.150000in}}{\pgfqpoint{5.700000in}{5.700000in}}%
\pgfusepath{clip}%
\pgfsetbuttcap%
\pgfsetroundjoin%
\definecolor{currentfill}{rgb}{0.636902,0.856542,0.216620}%
\pgfsetfillcolor{currentfill}%
\pgfsetfillopacity{0.700000}%
\pgfsetlinewidth{0.000000pt}%
\definecolor{currentstroke}{rgb}{0.000000,0.000000,0.000000}%
\pgfsetstrokecolor{currentstroke}%
\pgfsetdash{}{0pt}%
\pgfpathmoveto{\pgfqpoint{3.431956in}{4.652247in}}%
\pgfpathlineto{\pgfqpoint{3.445319in}{4.633723in}}%
\pgfpathlineto{\pgfqpoint{3.458678in}{4.615344in}}%
\pgfpathlineto{\pgfqpoint{3.472035in}{4.597109in}}%
\pgfpathlineto{\pgfqpoint{3.485389in}{4.579016in}}%
\pgfpathlineto{\pgfqpoint{3.492952in}{4.617774in}}%
\pgfpathlineto{\pgfqpoint{3.500513in}{4.657195in}}%
\pgfpathlineto{\pgfqpoint{3.508070in}{4.697291in}}%
\pgfpathlineto{\pgfqpoint{3.515623in}{4.738075in}}%
\pgfpathlineto{\pgfqpoint{3.502253in}{4.756760in}}%
\pgfpathlineto{\pgfqpoint{3.488879in}{4.775589in}}%
\pgfpathlineto{\pgfqpoint{3.475501in}{4.794563in}}%
\pgfpathlineto{\pgfqpoint{3.462121in}{4.813684in}}%
\pgfpathlineto{\pgfqpoint{3.454585in}{4.772295in}}%
\pgfpathlineto{\pgfqpoint{3.447046in}{4.731600in}}%
\pgfpathlineto{\pgfqpoint{3.439503in}{4.691588in}}%
\pgfpathlineto{\pgfqpoint{3.431956in}{4.652247in}}%
\pgfpathclose%
\pgfusepath{fill}%
\end{pgfscope}%
\begin{pgfscope}%
\pgfpathrectangle{\pgfqpoint{1.150000in}{0.150000in}}{\pgfqpoint{5.700000in}{5.700000in}}%
\pgfusepath{clip}%
\pgfsetbuttcap%
\pgfsetroundjoin%
\definecolor{currentfill}{rgb}{0.119512,0.607464,0.540218}%
\pgfsetfillcolor{currentfill}%
\pgfsetfillopacity{0.700000}%
\pgfsetlinewidth{0.000000pt}%
\definecolor{currentstroke}{rgb}{0.000000,0.000000,0.000000}%
\pgfsetstrokecolor{currentstroke}%
\pgfsetdash{}{0pt}%
\pgfpathmoveto{\pgfqpoint{4.124947in}{3.856074in}}%
\pgfpathlineto{\pgfqpoint{4.138278in}{3.843667in}}%
\pgfpathlineto{\pgfqpoint{4.151610in}{3.831357in}}%
\pgfpathlineto{\pgfqpoint{4.164945in}{3.819141in}}%
\pgfpathlineto{\pgfqpoint{4.178281in}{3.807021in}}%
\pgfpathlineto{\pgfqpoint{4.185987in}{3.838468in}}%
\pgfpathlineto{\pgfqpoint{4.193696in}{3.870493in}}%
\pgfpathlineto{\pgfqpoint{4.201407in}{3.903105in}}%
\pgfpathlineto{\pgfqpoint{4.209122in}{3.936317in}}%
\pgfpathlineto{\pgfqpoint{4.195780in}{3.948995in}}%
\pgfpathlineto{\pgfqpoint{4.182439in}{3.961767in}}%
\pgfpathlineto{\pgfqpoint{4.169100in}{3.974636in}}%
\pgfpathlineto{\pgfqpoint{4.155762in}{3.987600in}}%
\pgfpathlineto{\pgfqpoint{4.148054in}{3.953822in}}%
\pgfpathlineto{\pgfqpoint{4.140349in}{3.920648in}}%
\pgfpathlineto{\pgfqpoint{4.132647in}{3.888070in}}%
\pgfpathlineto{\pgfqpoint{4.124947in}{3.856074in}}%
\pgfpathclose%
\pgfusepath{fill}%
\end{pgfscope}%
\begin{pgfscope}%
\pgfpathrectangle{\pgfqpoint{1.150000in}{0.150000in}}{\pgfqpoint{5.700000in}{5.700000in}}%
\pgfusepath{clip}%
\pgfsetbuttcap%
\pgfsetroundjoin%
\definecolor{currentfill}{rgb}{0.229739,0.322361,0.545706}%
\pgfsetfillcolor{currentfill}%
\pgfsetfillopacity{0.700000}%
\pgfsetlinewidth{0.000000pt}%
\definecolor{currentstroke}{rgb}{0.000000,0.000000,0.000000}%
\pgfsetstrokecolor{currentstroke}%
\pgfsetdash{}{0pt}%
\pgfpathmoveto{\pgfqpoint{3.412868in}{3.141181in}}%
\pgfpathlineto{\pgfqpoint{3.426145in}{3.130525in}}%
\pgfpathlineto{\pgfqpoint{3.439422in}{3.119983in}}%
\pgfpathlineto{\pgfqpoint{3.452701in}{3.109553in}}%
\pgfpathlineto{\pgfqpoint{3.465981in}{3.099234in}}%
\pgfpathlineto{\pgfqpoint{3.473791in}{3.116289in}}%
\pgfpathlineto{\pgfqpoint{3.481595in}{3.133599in}}%
\pgfpathlineto{\pgfqpoint{3.489394in}{3.151173in}}%
\pgfpathlineto{\pgfqpoint{3.497187in}{3.169014in}}%
\pgfpathlineto{\pgfqpoint{3.483910in}{3.179651in}}%
\pgfpathlineto{\pgfqpoint{3.470634in}{3.190399in}}%
\pgfpathlineto{\pgfqpoint{3.457358in}{3.201259in}}%
\pgfpathlineto{\pgfqpoint{3.444084in}{3.212233in}}%
\pgfpathlineto{\pgfqpoint{3.436289in}{3.194066in}}%
\pgfpathlineto{\pgfqpoint{3.428488in}{3.176172in}}%
\pgfpathlineto{\pgfqpoint{3.420681in}{3.158546in}}%
\pgfpathlineto{\pgfqpoint{3.412868in}{3.141181in}}%
\pgfpathclose%
\pgfusepath{fill}%
\end{pgfscope}%
\begin{pgfscope}%
\pgfpathrectangle{\pgfqpoint{1.150000in}{0.150000in}}{\pgfqpoint{5.700000in}{5.700000in}}%
\pgfusepath{clip}%
\pgfsetbuttcap%
\pgfsetroundjoin%
\definecolor{currentfill}{rgb}{0.225863,0.330805,0.547314}%
\pgfsetfillcolor{currentfill}%
\pgfsetfillopacity{0.700000}%
\pgfsetlinewidth{0.000000pt}%
\definecolor{currentstroke}{rgb}{0.000000,0.000000,0.000000}%
\pgfsetstrokecolor{currentstroke}%
\pgfsetdash{}{0pt}%
\pgfpathmoveto{\pgfqpoint{3.275344in}{3.161273in}}%
\pgfpathlineto{\pgfqpoint{3.288619in}{3.149976in}}%
\pgfpathlineto{\pgfqpoint{3.301894in}{3.138801in}}%
\pgfpathlineto{\pgfqpoint{3.315170in}{3.127744in}}%
\pgfpathlineto{\pgfqpoint{3.328445in}{3.116807in}}%
\pgfpathlineto{\pgfqpoint{3.336286in}{3.133472in}}%
\pgfpathlineto{\pgfqpoint{3.344120in}{3.150381in}}%
\pgfpathlineto{\pgfqpoint{3.351948in}{3.167541in}}%
\pgfpathlineto{\pgfqpoint{3.359769in}{3.184955in}}%
\pgfpathlineto{\pgfqpoint{3.346496in}{3.196191in}}%
\pgfpathlineto{\pgfqpoint{3.333223in}{3.207545in}}%
\pgfpathlineto{\pgfqpoint{3.319951in}{3.219020in}}%
\pgfpathlineto{\pgfqpoint{3.306678in}{3.230615in}}%
\pgfpathlineto{\pgfqpoint{3.298855in}{3.212894in}}%
\pgfpathlineto{\pgfqpoint{3.291024in}{3.195434in}}%
\pgfpathlineto{\pgfqpoint{3.283188in}{3.178229in}}%
\pgfpathlineto{\pgfqpoint{3.275344in}{3.161273in}}%
\pgfpathclose%
\pgfusepath{fill}%
\end{pgfscope}%
\begin{pgfscope}%
\pgfpathrectangle{\pgfqpoint{1.150000in}{0.150000in}}{\pgfqpoint{5.700000in}{5.700000in}}%
\pgfusepath{clip}%
\pgfsetbuttcap%
\pgfsetroundjoin%
\definecolor{currentfill}{rgb}{0.335885,0.777018,0.402049}%
\pgfsetfillcolor{currentfill}%
\pgfsetfillopacity{0.700000}%
\pgfsetlinewidth{0.000000pt}%
\definecolor{currentstroke}{rgb}{0.000000,0.000000,0.000000}%
\pgfsetstrokecolor{currentstroke}%
\pgfsetdash{}{0pt}%
\pgfpathmoveto{\pgfqpoint{3.835873in}{4.329437in}}%
\pgfpathlineto{\pgfqpoint{3.849201in}{4.313948in}}%
\pgfpathlineto{\pgfqpoint{3.862528in}{4.298574in}}%
\pgfpathlineto{\pgfqpoint{3.875854in}{4.283313in}}%
\pgfpathlineto{\pgfqpoint{3.889180in}{4.268165in}}%
\pgfpathlineto{\pgfqpoint{3.896840in}{4.305462in}}%
\pgfpathlineto{\pgfqpoint{3.904500in}{4.343420in}}%
\pgfpathlineto{\pgfqpoint{3.912161in}{4.382052in}}%
\pgfpathlineto{\pgfqpoint{3.919824in}{4.421370in}}%
\pgfpathlineto{\pgfqpoint{3.906485in}{4.437116in}}%
\pgfpathlineto{\pgfqpoint{3.893144in}{4.452975in}}%
\pgfpathlineto{\pgfqpoint{3.879804in}{4.468948in}}%
\pgfpathlineto{\pgfqpoint{3.866462in}{4.485036in}}%
\pgfpathlineto{\pgfqpoint{3.858814in}{4.445109in}}%
\pgfpathlineto{\pgfqpoint{3.851166in}{4.405875in}}%
\pgfpathlineto{\pgfqpoint{3.843520in}{4.367322in}}%
\pgfpathlineto{\pgfqpoint{3.835873in}{4.329437in}}%
\pgfpathclose%
\pgfusepath{fill}%
\end{pgfscope}%
\begin{pgfscope}%
\pgfpathrectangle{\pgfqpoint{1.150000in}{0.150000in}}{\pgfqpoint{5.700000in}{5.700000in}}%
\pgfusepath{clip}%
\pgfsetbuttcap%
\pgfsetroundjoin%
\definecolor{currentfill}{rgb}{0.281477,0.755203,0.432552}%
\pgfsetfillcolor{currentfill}%
\pgfsetfillopacity{0.700000}%
\pgfsetlinewidth{0.000000pt}%
\definecolor{currentstroke}{rgb}{0.000000,0.000000,0.000000}%
\pgfsetstrokecolor{currentstroke}%
\pgfsetdash{}{0pt}%
\pgfpathmoveto{\pgfqpoint{3.889180in}{4.268165in}}%
\pgfpathlineto{\pgfqpoint{3.902506in}{4.253128in}}%
\pgfpathlineto{\pgfqpoint{3.915832in}{4.238203in}}%
\pgfpathlineto{\pgfqpoint{3.929158in}{4.223387in}}%
\pgfpathlineto{\pgfqpoint{3.942484in}{4.208680in}}%
\pgfpathlineto{\pgfqpoint{3.950155in}{4.245392in}}%
\pgfpathlineto{\pgfqpoint{3.957828in}{4.282758in}}%
\pgfpathlineto{\pgfqpoint{3.965502in}{4.320791in}}%
\pgfpathlineto{\pgfqpoint{3.973178in}{4.359503in}}%
\pgfpathlineto{\pgfqpoint{3.959840in}{4.374804in}}%
\pgfpathlineto{\pgfqpoint{3.946502in}{4.390215in}}%
\pgfpathlineto{\pgfqpoint{3.933163in}{4.405737in}}%
\pgfpathlineto{\pgfqpoint{3.919824in}{4.421370in}}%
\pgfpathlineto{\pgfqpoint{3.912161in}{4.382052in}}%
\pgfpathlineto{\pgfqpoint{3.904500in}{4.343420in}}%
\pgfpathlineto{\pgfqpoint{3.896840in}{4.305462in}}%
\pgfpathlineto{\pgfqpoint{3.889180in}{4.268165in}}%
\pgfpathclose%
\pgfusepath{fill}%
\end{pgfscope}%
\begin{pgfscope}%
\pgfpathrectangle{\pgfqpoint{1.150000in}{0.150000in}}{\pgfqpoint{5.700000in}{5.700000in}}%
\pgfusepath{clip}%
\pgfsetbuttcap%
\pgfsetroundjoin%
\definecolor{currentfill}{rgb}{0.208623,0.367752,0.552675}%
\pgfsetfillcolor{currentfill}%
\pgfsetfillopacity{0.700000}%
\pgfsetlinewidth{0.000000pt}%
\definecolor{currentstroke}{rgb}{0.000000,0.000000,0.000000}%
\pgfsetstrokecolor{currentstroke}%
\pgfsetdash{}{0pt}%
\pgfpathmoveto{\pgfqpoint{3.940220in}{3.237067in}}%
\pgfpathlineto{\pgfqpoint{3.953540in}{3.227514in}}%
\pgfpathlineto{\pgfqpoint{3.966862in}{3.218054in}}%
\pgfpathlineto{\pgfqpoint{3.980186in}{3.208689in}}%
\pgfpathlineto{\pgfqpoint{3.993514in}{3.199416in}}%
\pgfpathlineto{\pgfqpoint{4.001224in}{3.219935in}}%
\pgfpathlineto{\pgfqpoint{4.008932in}{3.240810in}}%
\pgfpathlineto{\pgfqpoint{4.016637in}{3.262047in}}%
\pgfpathlineto{\pgfqpoint{4.024341in}{3.283656in}}%
\pgfpathlineto{\pgfqpoint{4.011015in}{3.293348in}}%
\pgfpathlineto{\pgfqpoint{3.997691in}{3.303132in}}%
\pgfpathlineto{\pgfqpoint{3.984370in}{3.313011in}}%
\pgfpathlineto{\pgfqpoint{3.971052in}{3.322984in}}%
\pgfpathlineto{\pgfqpoint{3.963347in}{3.300948in}}%
\pgfpathlineto{\pgfqpoint{3.955641in}{3.279288in}}%
\pgfpathlineto{\pgfqpoint{3.947932in}{3.257997in}}%
\pgfpathlineto{\pgfqpoint{3.940220in}{3.237067in}}%
\pgfpathclose%
\pgfusepath{fill}%
\end{pgfscope}%
\begin{pgfscope}%
\pgfpathrectangle{\pgfqpoint{1.150000in}{0.150000in}}{\pgfqpoint{5.700000in}{5.700000in}}%
\pgfusepath{clip}%
\pgfsetbuttcap%
\pgfsetroundjoin%
\definecolor{currentfill}{rgb}{0.218130,0.347432,0.550038}%
\pgfsetfillcolor{currentfill}%
\pgfsetfillopacity{0.700000}%
\pgfsetlinewidth{0.000000pt}%
\definecolor{currentstroke}{rgb}{0.000000,0.000000,0.000000}%
\pgfsetstrokecolor{currentstroke}%
\pgfsetdash{}{0pt}%
\pgfpathmoveto{\pgfqpoint{3.856088in}{3.194371in}}%
\pgfpathlineto{\pgfqpoint{3.869399in}{3.184834in}}%
\pgfpathlineto{\pgfqpoint{3.882712in}{3.175393in}}%
\pgfpathlineto{\pgfqpoint{3.896028in}{3.166048in}}%
\pgfpathlineto{\pgfqpoint{3.909347in}{3.156799in}}%
\pgfpathlineto{\pgfqpoint{3.917069in}{3.176363in}}%
\pgfpathlineto{\pgfqpoint{3.924789in}{3.196257in}}%
\pgfpathlineto{\pgfqpoint{3.932506in}{3.216489in}}%
\pgfpathlineto{\pgfqpoint{3.940220in}{3.237067in}}%
\pgfpathlineto{\pgfqpoint{3.926903in}{3.246715in}}%
\pgfpathlineto{\pgfqpoint{3.913589in}{3.256458in}}%
\pgfpathlineto{\pgfqpoint{3.900278in}{3.266297in}}%
\pgfpathlineto{\pgfqpoint{3.886968in}{3.276233in}}%
\pgfpathlineto{\pgfqpoint{3.879253in}{3.255249in}}%
\pgfpathlineto{\pgfqpoint{3.871535in}{3.234616in}}%
\pgfpathlineto{\pgfqpoint{3.863813in}{3.214325in}}%
\pgfpathlineto{\pgfqpoint{3.856088in}{3.194371in}}%
\pgfpathclose%
\pgfusepath{fill}%
\end{pgfscope}%
\begin{pgfscope}%
\pgfpathrectangle{\pgfqpoint{1.150000in}{0.150000in}}{\pgfqpoint{5.700000in}{5.700000in}}%
\pgfusepath{clip}%
\pgfsetbuttcap%
\pgfsetroundjoin%
\definecolor{currentfill}{rgb}{0.157729,0.485932,0.558013}%
\pgfsetfillcolor{currentfill}%
\pgfsetfillopacity{0.700000}%
\pgfsetlinewidth{0.000000pt}%
\definecolor{currentstroke}{rgb}{0.000000,0.000000,0.000000}%
\pgfsetstrokecolor{currentstroke}%
\pgfsetdash{}{0pt}%
\pgfpathmoveto{\pgfqpoint{4.170031in}{3.531556in}}%
\pgfpathlineto{\pgfqpoint{4.183371in}{3.520938in}}%
\pgfpathlineto{\pgfqpoint{4.196713in}{3.510409in}}%
\pgfpathlineto{\pgfqpoint{4.210058in}{3.499971in}}%
\pgfpathlineto{\pgfqpoint{4.223406in}{3.489623in}}%
\pgfpathlineto{\pgfqpoint{4.231104in}{3.515834in}}%
\pgfpathlineto{\pgfqpoint{4.238804in}{3.542529in}}%
\pgfpathlineto{\pgfqpoint{4.246504in}{3.569716in}}%
\pgfpathlineto{\pgfqpoint{4.254207in}{3.597407in}}%
\pgfpathlineto{\pgfqpoint{4.240857in}{3.608261in}}%
\pgfpathlineto{\pgfqpoint{4.227510in}{3.619205in}}%
\pgfpathlineto{\pgfqpoint{4.214166in}{3.630239in}}%
\pgfpathlineto{\pgfqpoint{4.200823in}{3.641365in}}%
\pgfpathlineto{\pgfqpoint{4.193123in}{3.613160in}}%
\pgfpathlineto{\pgfqpoint{4.185425in}{3.585463in}}%
\pgfpathlineto{\pgfqpoint{4.177727in}{3.558265in}}%
\pgfpathlineto{\pgfqpoint{4.170031in}{3.531556in}}%
\pgfpathclose%
\pgfusepath{fill}%
\end{pgfscope}%
\begin{pgfscope}%
\pgfpathrectangle{\pgfqpoint{1.150000in}{0.150000in}}{\pgfqpoint{5.700000in}{5.700000in}}%
\pgfusepath{clip}%
\pgfsetbuttcap%
\pgfsetroundjoin%
\definecolor{currentfill}{rgb}{0.386433,0.794644,0.372886}%
\pgfsetfillcolor{currentfill}%
\pgfsetfillopacity{0.700000}%
\pgfsetlinewidth{0.000000pt}%
\definecolor{currentstroke}{rgb}{0.000000,0.000000,0.000000}%
\pgfsetstrokecolor{currentstroke}%
\pgfsetdash{}{0pt}%
\pgfpathmoveto{\pgfqpoint{3.782557in}{4.392553in}}%
\pgfpathlineto{\pgfqpoint{3.795887in}{4.376598in}}%
\pgfpathlineto{\pgfqpoint{3.809216in}{4.360761in}}%
\pgfpathlineto{\pgfqpoint{3.822545in}{4.345041in}}%
\pgfpathlineto{\pgfqpoint{3.835873in}{4.329437in}}%
\pgfpathlineto{\pgfqpoint{3.843520in}{4.367322in}}%
\pgfpathlineto{\pgfqpoint{3.851166in}{4.405875in}}%
\pgfpathlineto{\pgfqpoint{3.858814in}{4.445109in}}%
\pgfpathlineto{\pgfqpoint{3.866462in}{4.485036in}}%
\pgfpathlineto{\pgfqpoint{3.853120in}{4.501241in}}%
\pgfpathlineto{\pgfqpoint{3.839777in}{4.517562in}}%
\pgfpathlineto{\pgfqpoint{3.826433in}{4.534001in}}%
\pgfpathlineto{\pgfqpoint{3.813088in}{4.550559in}}%
\pgfpathlineto{\pgfqpoint{3.805455in}{4.510020in}}%
\pgfpathlineto{\pgfqpoint{3.797822in}{4.470181in}}%
\pgfpathlineto{\pgfqpoint{3.790189in}{4.431029in}}%
\pgfpathlineto{\pgfqpoint{3.782557in}{4.392553in}}%
\pgfpathclose%
\pgfusepath{fill}%
\end{pgfscope}%
\begin{pgfscope}%
\pgfpathrectangle{\pgfqpoint{1.150000in}{0.150000in}}{\pgfqpoint{5.700000in}{5.700000in}}%
\pgfusepath{clip}%
\pgfsetbuttcap%
\pgfsetroundjoin%
\definecolor{currentfill}{rgb}{0.239374,0.735588,0.455688}%
\pgfsetfillcolor{currentfill}%
\pgfsetfillopacity{0.700000}%
\pgfsetlinewidth{0.000000pt}%
\definecolor{currentstroke}{rgb}{0.000000,0.000000,0.000000}%
\pgfsetstrokecolor{currentstroke}%
\pgfsetdash{}{0pt}%
\pgfpathmoveto{\pgfqpoint{3.942484in}{4.208680in}}%
\pgfpathlineto{\pgfqpoint{3.955809in}{4.194082in}}%
\pgfpathlineto{\pgfqpoint{3.969135in}{4.179592in}}%
\pgfpathlineto{\pgfqpoint{3.982462in}{4.165208in}}%
\pgfpathlineto{\pgfqpoint{3.995788in}{4.150930in}}%
\pgfpathlineto{\pgfqpoint{4.003471in}{4.187059in}}%
\pgfpathlineto{\pgfqpoint{4.011155in}{4.223836in}}%
\pgfpathlineto{\pgfqpoint{4.018842in}{4.261273in}}%
\pgfpathlineto{\pgfqpoint{4.026531in}{4.299382in}}%
\pgfpathlineto{\pgfqpoint{4.013192in}{4.314252in}}%
\pgfpathlineto{\pgfqpoint{3.999854in}{4.329228in}}%
\pgfpathlineto{\pgfqpoint{3.986516in}{4.344311in}}%
\pgfpathlineto{\pgfqpoint{3.973178in}{4.359503in}}%
\pgfpathlineto{\pgfqpoint{3.965502in}{4.320791in}}%
\pgfpathlineto{\pgfqpoint{3.957828in}{4.282758in}}%
\pgfpathlineto{\pgfqpoint{3.950155in}{4.245392in}}%
\pgfpathlineto{\pgfqpoint{3.942484in}{4.208680in}}%
\pgfpathclose%
\pgfusepath{fill}%
\end{pgfscope}%
\begin{pgfscope}%
\pgfpathrectangle{\pgfqpoint{1.150000in}{0.150000in}}{\pgfqpoint{5.700000in}{5.700000in}}%
\pgfusepath{clip}%
\pgfsetbuttcap%
\pgfsetroundjoin%
\definecolor{currentfill}{rgb}{0.231674,0.318106,0.544834}%
\pgfsetfillcolor{currentfill}%
\pgfsetfillopacity{0.700000}%
\pgfsetlinewidth{0.000000pt}%
\definecolor{currentstroke}{rgb}{0.000000,0.000000,0.000000}%
\pgfsetstrokecolor{currentstroke}%
\pgfsetdash{}{0pt}%
\pgfpathmoveto{\pgfqpoint{3.550309in}{3.127569in}}%
\pgfpathlineto{\pgfqpoint{3.563593in}{3.117478in}}%
\pgfpathlineto{\pgfqpoint{3.576879in}{3.107495in}}%
\pgfpathlineto{\pgfqpoint{3.590166in}{3.097617in}}%
\pgfpathlineto{\pgfqpoint{3.603456in}{3.087845in}}%
\pgfpathlineto{\pgfqpoint{3.611238in}{3.105300in}}%
\pgfpathlineto{\pgfqpoint{3.619016in}{3.123026in}}%
\pgfpathlineto{\pgfqpoint{3.626789in}{3.141027in}}%
\pgfpathlineto{\pgfqpoint{3.634556in}{3.159312in}}%
\pgfpathlineto{\pgfqpoint{3.621270in}{3.169421in}}%
\pgfpathlineto{\pgfqpoint{3.607985in}{3.179637in}}%
\pgfpathlineto{\pgfqpoint{3.594701in}{3.189958in}}%
\pgfpathlineto{\pgfqpoint{3.581419in}{3.200386in}}%
\pgfpathlineto{\pgfqpoint{3.573649in}{3.181756in}}%
\pgfpathlineto{\pgfqpoint{3.565875in}{3.163414in}}%
\pgfpathlineto{\pgfqpoint{3.558095in}{3.145354in}}%
\pgfpathlineto{\pgfqpoint{3.550309in}{3.127569in}}%
\pgfpathclose%
\pgfusepath{fill}%
\end{pgfscope}%
\begin{pgfscope}%
\pgfpathrectangle{\pgfqpoint{1.150000in}{0.150000in}}{\pgfqpoint{5.700000in}{5.700000in}}%
\pgfusepath{clip}%
\pgfsetbuttcap%
\pgfsetroundjoin%
\definecolor{currentfill}{rgb}{0.174274,0.445044,0.557792}%
\pgfsetfillcolor{currentfill}%
\pgfsetfillopacity{0.700000}%
\pgfsetlinewidth{0.000000pt}%
\definecolor{currentstroke}{rgb}{0.000000,0.000000,0.000000}%
\pgfsetstrokecolor{currentstroke}%
\pgfsetdash{}{0pt}%
\pgfpathmoveto{\pgfqpoint{4.139250in}{3.429419in}}%
\pgfpathlineto{\pgfqpoint{4.152589in}{3.419284in}}%
\pgfpathlineto{\pgfqpoint{4.165931in}{3.409239in}}%
\pgfpathlineto{\pgfqpoint{4.179276in}{3.399285in}}%
\pgfpathlineto{\pgfqpoint{4.192623in}{3.389419in}}%
\pgfpathlineto{\pgfqpoint{4.200318in}{3.413793in}}%
\pgfpathlineto{\pgfqpoint{4.208014in}{3.438612in}}%
\pgfpathlineto{\pgfqpoint{4.215710in}{3.463885in}}%
\pgfpathlineto{\pgfqpoint{4.223406in}{3.489623in}}%
\pgfpathlineto{\pgfqpoint{4.210058in}{3.499971in}}%
\pgfpathlineto{\pgfqpoint{4.196713in}{3.510409in}}%
\pgfpathlineto{\pgfqpoint{4.183371in}{3.520938in}}%
\pgfpathlineto{\pgfqpoint{4.170031in}{3.531556in}}%
\pgfpathlineto{\pgfqpoint{4.162335in}{3.505327in}}%
\pgfpathlineto{\pgfqpoint{4.154640in}{3.479567in}}%
\pgfpathlineto{\pgfqpoint{4.146945in}{3.454267in}}%
\pgfpathlineto{\pgfqpoint{4.139250in}{3.429419in}}%
\pgfpathclose%
\pgfusepath{fill}%
\end{pgfscope}%
\begin{pgfscope}%
\pgfpathrectangle{\pgfqpoint{1.150000in}{0.150000in}}{\pgfqpoint{5.700000in}{5.700000in}}%
\pgfusepath{clip}%
\pgfsetbuttcap%
\pgfsetroundjoin%
\definecolor{currentfill}{rgb}{0.199430,0.387607,0.554642}%
\pgfsetfillcolor{currentfill}%
\pgfsetfillopacity{0.700000}%
\pgfsetlinewidth{0.000000pt}%
\definecolor{currentstroke}{rgb}{0.000000,0.000000,0.000000}%
\pgfsetstrokecolor{currentstroke}%
\pgfsetdash{}{0pt}%
\pgfpathmoveto{\pgfqpoint{4.024341in}{3.283656in}}%
\pgfpathlineto{\pgfqpoint{4.037670in}{3.274057in}}%
\pgfpathlineto{\pgfqpoint{4.051002in}{3.264550in}}%
\pgfpathlineto{\pgfqpoint{4.064336in}{3.255135in}}%
\pgfpathlineto{\pgfqpoint{4.077674in}{3.245810in}}%
\pgfpathlineto{\pgfqpoint{4.085374in}{3.267366in}}%
\pgfpathlineto{\pgfqpoint{4.093074in}{3.289304in}}%
\pgfpathlineto{\pgfqpoint{4.100772in}{3.311631in}}%
\pgfpathlineto{\pgfqpoint{4.108469in}{3.334358in}}%
\pgfpathlineto{\pgfqpoint{4.095132in}{3.344122in}}%
\pgfpathlineto{\pgfqpoint{4.081798in}{3.353977in}}%
\pgfpathlineto{\pgfqpoint{4.068467in}{3.363924in}}%
\pgfpathlineto{\pgfqpoint{4.055139in}{3.373964in}}%
\pgfpathlineto{\pgfqpoint{4.047442in}{3.350789in}}%
\pgfpathlineto{\pgfqpoint{4.039743in}{3.328019in}}%
\pgfpathlineto{\pgfqpoint{4.032043in}{3.305644in}}%
\pgfpathlineto{\pgfqpoint{4.024341in}{3.283656in}}%
\pgfpathclose%
\pgfusepath{fill}%
\end{pgfscope}%
\begin{pgfscope}%
\pgfpathrectangle{\pgfqpoint{1.150000in}{0.150000in}}{\pgfqpoint{5.700000in}{5.700000in}}%
\pgfusepath{clip}%
\pgfsetbuttcap%
\pgfsetroundjoin%
\definecolor{currentfill}{rgb}{0.208030,0.718701,0.472873}%
\pgfsetfillcolor{currentfill}%
\pgfsetfillopacity{0.700000}%
\pgfsetlinewidth{0.000000pt}%
\definecolor{currentstroke}{rgb}{0.000000,0.000000,0.000000}%
\pgfsetstrokecolor{currentstroke}%
\pgfsetdash{}{0pt}%
\pgfpathmoveto{\pgfqpoint{3.995788in}{4.150930in}}%
\pgfpathlineto{\pgfqpoint{4.009115in}{4.136758in}}%
\pgfpathlineto{\pgfqpoint{4.022443in}{4.122690in}}%
\pgfpathlineto{\pgfqpoint{4.035771in}{4.108726in}}%
\pgfpathlineto{\pgfqpoint{4.049100in}{4.094864in}}%
\pgfpathlineto{\pgfqpoint{4.056792in}{4.130413in}}%
\pgfpathlineto{\pgfqpoint{4.064487in}{4.166603in}}%
\pgfpathlineto{\pgfqpoint{4.072185in}{4.203447in}}%
\pgfpathlineto{\pgfqpoint{4.079886in}{4.240956in}}%
\pgfpathlineto{\pgfqpoint{4.066546in}{4.255406in}}%
\pgfpathlineto{\pgfqpoint{4.053207in}{4.269960in}}%
\pgfpathlineto{\pgfqpoint{4.039869in}{4.284618in}}%
\pgfpathlineto{\pgfqpoint{4.026531in}{4.299382in}}%
\pgfpathlineto{\pgfqpoint{4.018842in}{4.261273in}}%
\pgfpathlineto{\pgfqpoint{4.011155in}{4.223836in}}%
\pgfpathlineto{\pgfqpoint{4.003471in}{4.187059in}}%
\pgfpathlineto{\pgfqpoint{3.995788in}{4.150930in}}%
\pgfpathclose%
\pgfusepath{fill}%
\end{pgfscope}%
\begin{pgfscope}%
\pgfpathrectangle{\pgfqpoint{1.150000in}{0.150000in}}{\pgfqpoint{5.700000in}{5.700000in}}%
\pgfusepath{clip}%
\pgfsetbuttcap%
\pgfsetroundjoin%
\definecolor{currentfill}{rgb}{0.449368,0.813768,0.335384}%
\pgfsetfillcolor{currentfill}%
\pgfsetfillopacity{0.700000}%
\pgfsetlinewidth{0.000000pt}%
\definecolor{currentstroke}{rgb}{0.000000,0.000000,0.000000}%
\pgfsetstrokecolor{currentstroke}%
\pgfsetdash{}{0pt}%
\pgfpathmoveto{\pgfqpoint{3.729225in}{4.457575in}}%
\pgfpathlineto{\pgfqpoint{3.742560in}{4.441138in}}%
\pgfpathlineto{\pgfqpoint{3.755893in}{4.424822in}}%
\pgfpathlineto{\pgfqpoint{3.769225in}{4.408628in}}%
\pgfpathlineto{\pgfqpoint{3.782557in}{4.392553in}}%
\pgfpathlineto{\pgfqpoint{3.790189in}{4.431029in}}%
\pgfpathlineto{\pgfqpoint{3.797822in}{4.470181in}}%
\pgfpathlineto{\pgfqpoint{3.805455in}{4.510020in}}%
\pgfpathlineto{\pgfqpoint{3.813088in}{4.550559in}}%
\pgfpathlineto{\pgfqpoint{3.799741in}{4.567238in}}%
\pgfpathlineto{\pgfqpoint{3.786394in}{4.584036in}}%
\pgfpathlineto{\pgfqpoint{3.773045in}{4.600957in}}%
\pgfpathlineto{\pgfqpoint{3.759694in}{4.618001in}}%
\pgfpathlineto{\pgfqpoint{3.752077in}{4.576846in}}%
\pgfpathlineto{\pgfqpoint{3.744460in}{4.536398in}}%
\pgfpathlineto{\pgfqpoint{3.736843in}{4.496645in}}%
\pgfpathlineto{\pgfqpoint{3.729225in}{4.457575in}}%
\pgfpathclose%
\pgfusepath{fill}%
\end{pgfscope}%
\begin{pgfscope}%
\pgfpathrectangle{\pgfqpoint{1.150000in}{0.150000in}}{\pgfqpoint{5.700000in}{5.700000in}}%
\pgfusepath{clip}%
\pgfsetbuttcap%
\pgfsetroundjoin%
\definecolor{currentfill}{rgb}{0.225863,0.330805,0.547314}%
\pgfsetfillcolor{currentfill}%
\pgfsetfillopacity{0.700000}%
\pgfsetlinewidth{0.000000pt}%
\definecolor{currentstroke}{rgb}{0.000000,0.000000,0.000000}%
\pgfsetstrokecolor{currentstroke}%
\pgfsetdash{}{0pt}%
\pgfpathmoveto{\pgfqpoint{3.771928in}{3.155375in}}%
\pgfpathlineto{\pgfqpoint{3.785231in}{3.145824in}}%
\pgfpathlineto{\pgfqpoint{3.798536in}{3.136373in}}%
\pgfpathlineto{\pgfqpoint{3.811844in}{3.127020in}}%
\pgfpathlineto{\pgfqpoint{3.825154in}{3.117764in}}%
\pgfpathlineto{\pgfqpoint{3.832893in}{3.136448in}}%
\pgfpathlineto{\pgfqpoint{3.840628in}{3.155439in}}%
\pgfpathlineto{\pgfqpoint{3.848360in}{3.174744in}}%
\pgfpathlineto{\pgfqpoint{3.856088in}{3.194371in}}%
\pgfpathlineto{\pgfqpoint{3.842780in}{3.204004in}}%
\pgfpathlineto{\pgfqpoint{3.829474in}{3.213736in}}%
\pgfpathlineto{\pgfqpoint{3.816171in}{3.223565in}}%
\pgfpathlineto{\pgfqpoint{3.802870in}{3.233493in}}%
\pgfpathlineto{\pgfqpoint{3.795140in}{3.213481in}}%
\pgfpathlineto{\pgfqpoint{3.787406in}{3.193795in}}%
\pgfpathlineto{\pgfqpoint{3.779669in}{3.174429in}}%
\pgfpathlineto{\pgfqpoint{3.771928in}{3.155375in}}%
\pgfpathclose%
\pgfusepath{fill}%
\end{pgfscope}%
\begin{pgfscope}%
\pgfpathrectangle{\pgfqpoint{1.150000in}{0.150000in}}{\pgfqpoint{5.700000in}{5.700000in}}%
\pgfusepath{clip}%
\pgfsetbuttcap%
\pgfsetroundjoin%
\definecolor{currentfill}{rgb}{0.141935,0.526453,0.555991}%
\pgfsetfillcolor{currentfill}%
\pgfsetfillopacity{0.700000}%
\pgfsetlinewidth{0.000000pt}%
\definecolor{currentstroke}{rgb}{0.000000,0.000000,0.000000}%
\pgfsetstrokecolor{currentstroke}%
\pgfsetdash{}{0pt}%
\pgfpathmoveto{\pgfqpoint{4.200823in}{3.641365in}}%
\pgfpathlineto{\pgfqpoint{4.214166in}{3.630239in}}%
\pgfpathlineto{\pgfqpoint{4.227510in}{3.619205in}}%
\pgfpathlineto{\pgfqpoint{4.240857in}{3.608261in}}%
\pgfpathlineto{\pgfqpoint{4.254207in}{3.597407in}}%
\pgfpathlineto{\pgfqpoint{4.261912in}{3.625609in}}%
\pgfpathlineto{\pgfqpoint{4.269619in}{3.654335in}}%
\pgfpathlineto{\pgfqpoint{4.277328in}{3.683594in}}%
\pgfpathlineto{\pgfqpoint{4.285040in}{3.713396in}}%
\pgfpathlineto{\pgfqpoint{4.271688in}{3.724779in}}%
\pgfpathlineto{\pgfqpoint{4.258338in}{3.736253in}}%
\pgfpathlineto{\pgfqpoint{4.244990in}{3.747818in}}%
\pgfpathlineto{\pgfqpoint{4.231644in}{3.759473in}}%
\pgfpathlineto{\pgfqpoint{4.223935in}{3.729132in}}%
\pgfpathlineto{\pgfqpoint{4.216229in}{3.699341in}}%
\pgfpathlineto{\pgfqpoint{4.208525in}{3.670088in}}%
\pgfpathlineto{\pgfqpoint{4.200823in}{3.641365in}}%
\pgfpathclose%
\pgfusepath{fill}%
\end{pgfscope}%
\begin{pgfscope}%
\pgfpathrectangle{\pgfqpoint{1.150000in}{0.150000in}}{\pgfqpoint{5.700000in}{5.700000in}}%
\pgfusepath{clip}%
\pgfsetbuttcap%
\pgfsetroundjoin%
\definecolor{currentfill}{rgb}{0.220057,0.343307,0.549413}%
\pgfsetfillcolor{currentfill}%
\pgfsetfillopacity{0.700000}%
\pgfsetlinewidth{0.000000pt}%
\definecolor{currentstroke}{rgb}{0.000000,0.000000,0.000000}%
\pgfsetstrokecolor{currentstroke}%
\pgfsetdash{}{0pt}%
\pgfpathmoveto{\pgfqpoint{3.137678in}{3.188446in}}%
\pgfpathlineto{\pgfqpoint{3.150957in}{3.176426in}}%
\pgfpathlineto{\pgfqpoint{3.164236in}{3.164535in}}%
\pgfpathlineto{\pgfqpoint{3.177515in}{3.152772in}}%
\pgfpathlineto{\pgfqpoint{3.190793in}{3.141136in}}%
\pgfpathlineto{\pgfqpoint{3.198666in}{3.157418in}}%
\pgfpathlineto{\pgfqpoint{3.206533in}{3.173933in}}%
\pgfpathlineto{\pgfqpoint{3.214393in}{3.190687in}}%
\pgfpathlineto{\pgfqpoint{3.222246in}{3.207685in}}%
\pgfpathlineto{\pgfqpoint{3.208971in}{3.219600in}}%
\pgfpathlineto{\pgfqpoint{3.195695in}{3.231642in}}%
\pgfpathlineto{\pgfqpoint{3.182419in}{3.243811in}}%
\pgfpathlineto{\pgfqpoint{3.169143in}{3.256110in}}%
\pgfpathlineto{\pgfqpoint{3.161287in}{3.238826in}}%
\pgfpathlineto{\pgfqpoint{3.153425in}{3.221790in}}%
\pgfpathlineto{\pgfqpoint{3.145555in}{3.204999in}}%
\pgfpathlineto{\pgfqpoint{3.137678in}{3.188446in}}%
\pgfpathclose%
\pgfusepath{fill}%
\end{pgfscope}%
\begin{pgfscope}%
\pgfpathrectangle{\pgfqpoint{1.150000in}{0.150000in}}{\pgfqpoint{5.700000in}{5.700000in}}%
\pgfusepath{clip}%
\pgfsetbuttcap%
\pgfsetroundjoin%
\definecolor{currentfill}{rgb}{0.121831,0.589055,0.545623}%
\pgfsetfillcolor{currentfill}%
\pgfsetfillopacity{0.700000}%
\pgfsetlinewidth{0.000000pt}%
\definecolor{currentstroke}{rgb}{0.000000,0.000000,0.000000}%
\pgfsetstrokecolor{currentstroke}%
\pgfsetdash{}{0pt}%
\pgfpathmoveto{\pgfqpoint{4.178281in}{3.807021in}}%
\pgfpathlineto{\pgfqpoint{4.191619in}{3.794994in}}%
\pgfpathlineto{\pgfqpoint{4.204958in}{3.783061in}}%
\pgfpathlineto{\pgfqpoint{4.218300in}{3.771221in}}%
\pgfpathlineto{\pgfqpoint{4.231644in}{3.759473in}}%
\pgfpathlineto{\pgfqpoint{4.239355in}{3.790374in}}%
\pgfpathlineto{\pgfqpoint{4.247069in}{3.821847in}}%
\pgfpathlineto{\pgfqpoint{4.254787in}{3.853900in}}%
\pgfpathlineto{\pgfqpoint{4.262508in}{3.886547in}}%
\pgfpathlineto{\pgfqpoint{4.249159in}{3.898850in}}%
\pgfpathlineto{\pgfqpoint{4.235811in}{3.911245in}}%
\pgfpathlineto{\pgfqpoint{4.222466in}{3.923734in}}%
\pgfpathlineto{\pgfqpoint{4.209122in}{3.936317in}}%
\pgfpathlineto{\pgfqpoint{4.201407in}{3.903105in}}%
\pgfpathlineto{\pgfqpoint{4.193696in}{3.870493in}}%
\pgfpathlineto{\pgfqpoint{4.185987in}{3.838468in}}%
\pgfpathlineto{\pgfqpoint{4.178281in}{3.807021in}}%
\pgfpathclose%
\pgfusepath{fill}%
\end{pgfscope}%
\begin{pgfscope}%
\pgfpathrectangle{\pgfqpoint{1.150000in}{0.150000in}}{\pgfqpoint{5.700000in}{5.700000in}}%
\pgfusepath{clip}%
\pgfsetbuttcap%
\pgfsetroundjoin%
\definecolor{currentfill}{rgb}{0.175707,0.697900,0.491033}%
\pgfsetfillcolor{currentfill}%
\pgfsetfillopacity{0.700000}%
\pgfsetlinewidth{0.000000pt}%
\definecolor{currentstroke}{rgb}{0.000000,0.000000,0.000000}%
\pgfsetstrokecolor{currentstroke}%
\pgfsetdash{}{0pt}%
\pgfpathmoveto{\pgfqpoint{4.049100in}{4.094864in}}%
\pgfpathlineto{\pgfqpoint{4.062429in}{4.081105in}}%
\pgfpathlineto{\pgfqpoint{4.075759in}{4.067448in}}%
\pgfpathlineto{\pgfqpoint{4.089090in}{4.053892in}}%
\pgfpathlineto{\pgfqpoint{4.102422in}{4.040436in}}%
\pgfpathlineto{\pgfqpoint{4.110125in}{4.075407in}}%
\pgfpathlineto{\pgfqpoint{4.117829in}{4.111013in}}%
\pgfpathlineto{\pgfqpoint{4.125537in}{4.147265in}}%
\pgfpathlineto{\pgfqpoint{4.133249in}{4.184176in}}%
\pgfpathlineto{\pgfqpoint{4.119907in}{4.198219in}}%
\pgfpathlineto{\pgfqpoint{4.106566in}{4.212363in}}%
\pgfpathlineto{\pgfqpoint{4.093226in}{4.226608in}}%
\pgfpathlineto{\pgfqpoint{4.079886in}{4.240956in}}%
\pgfpathlineto{\pgfqpoint{4.072185in}{4.203447in}}%
\pgfpathlineto{\pgfqpoint{4.064487in}{4.166603in}}%
\pgfpathlineto{\pgfqpoint{4.056792in}{4.130413in}}%
\pgfpathlineto{\pgfqpoint{4.049100in}{4.094864in}}%
\pgfpathclose%
\pgfusepath{fill}%
\end{pgfscope}%
\begin{pgfscope}%
\pgfpathrectangle{\pgfqpoint{1.150000in}{0.150000in}}{\pgfqpoint{5.700000in}{5.700000in}}%
\pgfusepath{clip}%
\pgfsetbuttcap%
\pgfsetroundjoin%
\definecolor{currentfill}{rgb}{0.515992,0.831158,0.294279}%
\pgfsetfillcolor{currentfill}%
\pgfsetfillopacity{0.700000}%
\pgfsetlinewidth{0.000000pt}%
\definecolor{currentstroke}{rgb}{0.000000,0.000000,0.000000}%
\pgfsetstrokecolor{currentstroke}%
\pgfsetdash{}{0pt}%
\pgfpathmoveto{\pgfqpoint{3.675872in}{4.524567in}}%
\pgfpathlineto{\pgfqpoint{3.689213in}{4.507631in}}%
\pgfpathlineto{\pgfqpoint{3.702552in}{4.490821in}}%
\pgfpathlineto{\pgfqpoint{3.715889in}{4.474136in}}%
\pgfpathlineto{\pgfqpoint{3.729225in}{4.457575in}}%
\pgfpathlineto{\pgfqpoint{3.736843in}{4.496645in}}%
\pgfpathlineto{\pgfqpoint{3.744460in}{4.536398in}}%
\pgfpathlineto{\pgfqpoint{3.752077in}{4.576846in}}%
\pgfpathlineto{\pgfqpoint{3.759694in}{4.618001in}}%
\pgfpathlineto{\pgfqpoint{3.746342in}{4.635169in}}%
\pgfpathlineto{\pgfqpoint{3.732989in}{4.652462in}}%
\pgfpathlineto{\pgfqpoint{3.719634in}{4.669880in}}%
\pgfpathlineto{\pgfqpoint{3.706276in}{4.687426in}}%
\pgfpathlineto{\pgfqpoint{3.698677in}{4.645652in}}%
\pgfpathlineto{\pgfqpoint{3.691076in}{4.604593in}}%
\pgfpathlineto{\pgfqpoint{3.683475in}{4.564235in}}%
\pgfpathlineto{\pgfqpoint{3.675872in}{4.524567in}}%
\pgfpathclose%
\pgfusepath{fill}%
\end{pgfscope}%
\begin{pgfscope}%
\pgfpathrectangle{\pgfqpoint{1.150000in}{0.150000in}}{\pgfqpoint{5.700000in}{5.700000in}}%
\pgfusepath{clip}%
\pgfsetbuttcap%
\pgfsetroundjoin%
\definecolor{currentfill}{rgb}{0.188923,0.410910,0.556326}%
\pgfsetfillcolor{currentfill}%
\pgfsetfillopacity{0.700000}%
\pgfsetlinewidth{0.000000pt}%
\definecolor{currentstroke}{rgb}{0.000000,0.000000,0.000000}%
\pgfsetstrokecolor{currentstroke}%
\pgfsetdash{}{0pt}%
\pgfpathmoveto{\pgfqpoint{4.108469in}{3.334358in}}%
\pgfpathlineto{\pgfqpoint{4.121808in}{3.324684in}}%
\pgfpathlineto{\pgfqpoint{4.135150in}{3.315101in}}%
\pgfpathlineto{\pgfqpoint{4.148496in}{3.305607in}}%
\pgfpathlineto{\pgfqpoint{4.161844in}{3.296203in}}%
\pgfpathlineto{\pgfqpoint{4.169539in}{3.318883in}}%
\pgfpathlineto{\pgfqpoint{4.177234in}{3.341973in}}%
\pgfpathlineto{\pgfqpoint{4.184929in}{3.365482in}}%
\pgfpathlineto{\pgfqpoint{4.192623in}{3.389419in}}%
\pgfpathlineto{\pgfqpoint{4.179276in}{3.399285in}}%
\pgfpathlineto{\pgfqpoint{4.165931in}{3.409239in}}%
\pgfpathlineto{\pgfqpoint{4.152589in}{3.419284in}}%
\pgfpathlineto{\pgfqpoint{4.139250in}{3.429419in}}%
\pgfpathlineto{\pgfqpoint{4.131555in}{3.405013in}}%
\pgfpathlineto{\pgfqpoint{4.123860in}{3.381040in}}%
\pgfpathlineto{\pgfqpoint{4.116165in}{3.357491in}}%
\pgfpathlineto{\pgfqpoint{4.108469in}{3.334358in}}%
\pgfpathclose%
\pgfusepath{fill}%
\end{pgfscope}%
\begin{pgfscope}%
\pgfpathrectangle{\pgfqpoint{1.150000in}{0.150000in}}{\pgfqpoint{5.700000in}{5.700000in}}%
\pgfusepath{clip}%
\pgfsetbuttcap%
\pgfsetroundjoin%
\definecolor{currentfill}{rgb}{0.231674,0.318106,0.544834}%
\pgfsetfillcolor{currentfill}%
\pgfsetfillopacity{0.700000}%
\pgfsetlinewidth{0.000000pt}%
\definecolor{currentstroke}{rgb}{0.000000,0.000000,0.000000}%
\pgfsetstrokecolor{currentstroke}%
\pgfsetdash{}{0pt}%
\pgfpathmoveto{\pgfqpoint{3.687722in}{3.119913in}}%
\pgfpathlineto{\pgfqpoint{3.701018in}{3.110319in}}%
\pgfpathlineto{\pgfqpoint{3.714317in}{3.100827in}}%
\pgfpathlineto{\pgfqpoint{3.727618in}{3.091435in}}%
\pgfpathlineto{\pgfqpoint{3.740921in}{3.082143in}}%
\pgfpathlineto{\pgfqpoint{3.748679in}{3.100017in}}%
\pgfpathlineto{\pgfqpoint{3.756433in}{3.118175in}}%
\pgfpathlineto{\pgfqpoint{3.764182in}{3.136626in}}%
\pgfpathlineto{\pgfqpoint{3.771928in}{3.155375in}}%
\pgfpathlineto{\pgfqpoint{3.758627in}{3.165024in}}%
\pgfpathlineto{\pgfqpoint{3.745328in}{3.174773in}}%
\pgfpathlineto{\pgfqpoint{3.732032in}{3.184623in}}%
\pgfpathlineto{\pgfqpoint{3.718738in}{3.194574in}}%
\pgfpathlineto{\pgfqpoint{3.710990in}{3.175460in}}%
\pgfpathlineto{\pgfqpoint{3.703239in}{3.156649in}}%
\pgfpathlineto{\pgfqpoint{3.695483in}{3.138136in}}%
\pgfpathlineto{\pgfqpoint{3.687722in}{3.119913in}}%
\pgfpathclose%
\pgfusepath{fill}%
\end{pgfscope}%
\begin{pgfscope}%
\pgfpathrectangle{\pgfqpoint{1.150000in}{0.150000in}}{\pgfqpoint{5.700000in}{5.700000in}}%
\pgfusepath{clip}%
\pgfsetbuttcap%
\pgfsetroundjoin%
\definecolor{currentfill}{rgb}{0.153894,0.680203,0.504172}%
\pgfsetfillcolor{currentfill}%
\pgfsetfillopacity{0.700000}%
\pgfsetlinewidth{0.000000pt}%
\definecolor{currentstroke}{rgb}{0.000000,0.000000,0.000000}%
\pgfsetstrokecolor{currentstroke}%
\pgfsetdash{}{0pt}%
\pgfpathmoveto{\pgfqpoint{4.102422in}{4.040436in}}%
\pgfpathlineto{\pgfqpoint{4.115756in}{4.027079in}}%
\pgfpathlineto{\pgfqpoint{4.129090in}{4.013822in}}%
\pgfpathlineto{\pgfqpoint{4.142425in}{4.000662in}}%
\pgfpathlineto{\pgfqpoint{4.155762in}{3.987600in}}%
\pgfpathlineto{\pgfqpoint{4.163472in}{4.021996in}}%
\pgfpathlineto{\pgfqpoint{4.171186in}{4.057020in}}%
\pgfpathlineto{\pgfqpoint{4.178903in}{4.092684in}}%
\pgfpathlineto{\pgfqpoint{4.186625in}{4.129000in}}%
\pgfpathlineto{\pgfqpoint{4.173279in}{4.142646in}}%
\pgfpathlineto{\pgfqpoint{4.159935in}{4.156391in}}%
\pgfpathlineto{\pgfqpoint{4.146591in}{4.170234in}}%
\pgfpathlineto{\pgfqpoint{4.133249in}{4.184176in}}%
\pgfpathlineto{\pgfqpoint{4.125537in}{4.147265in}}%
\pgfpathlineto{\pgfqpoint{4.117829in}{4.111013in}}%
\pgfpathlineto{\pgfqpoint{4.110125in}{4.075407in}}%
\pgfpathlineto{\pgfqpoint{4.102422in}{4.040436in}}%
\pgfpathclose%
\pgfusepath{fill}%
\end{pgfscope}%
\begin{pgfscope}%
\pgfpathrectangle{\pgfqpoint{1.150000in}{0.150000in}}{\pgfqpoint{5.700000in}{5.700000in}}%
\pgfusepath{clip}%
\pgfsetbuttcap%
\pgfsetroundjoin%
\definecolor{currentfill}{rgb}{0.233603,0.313828,0.543914}%
\pgfsetfillcolor{currentfill}%
\pgfsetfillopacity{0.700000}%
\pgfsetlinewidth{0.000000pt}%
\definecolor{currentstroke}{rgb}{0.000000,0.000000,0.000000}%
\pgfsetstrokecolor{currentstroke}%
\pgfsetdash{}{0pt}%
\pgfpathmoveto{\pgfqpoint{3.328445in}{3.116807in}}%
\pgfpathlineto{\pgfqpoint{3.341722in}{3.105987in}}%
\pgfpathlineto{\pgfqpoint{3.354999in}{3.095284in}}%
\pgfpathlineto{\pgfqpoint{3.368277in}{3.084696in}}%
\pgfpathlineto{\pgfqpoint{3.381556in}{3.074224in}}%
\pgfpathlineto{\pgfqpoint{3.389393in}{3.090599in}}%
\pgfpathlineto{\pgfqpoint{3.397224in}{3.107214in}}%
\pgfpathlineto{\pgfqpoint{3.405049in}{3.124072in}}%
\pgfpathlineto{\pgfqpoint{3.412868in}{3.141181in}}%
\pgfpathlineto{\pgfqpoint{3.399592in}{3.151951in}}%
\pgfpathlineto{\pgfqpoint{3.386317in}{3.162836in}}%
\pgfpathlineto{\pgfqpoint{3.373043in}{3.173837in}}%
\pgfpathlineto{\pgfqpoint{3.359769in}{3.184955in}}%
\pgfpathlineto{\pgfqpoint{3.351948in}{3.167541in}}%
\pgfpathlineto{\pgfqpoint{3.344120in}{3.150381in}}%
\pgfpathlineto{\pgfqpoint{3.336286in}{3.133472in}}%
\pgfpathlineto{\pgfqpoint{3.328445in}{3.116807in}}%
\pgfpathclose%
\pgfusepath{fill}%
\end{pgfscope}%
\begin{pgfscope}%
\pgfpathrectangle{\pgfqpoint{1.150000in}{0.150000in}}{\pgfqpoint{5.700000in}{5.700000in}}%
\pgfusepath{clip}%
\pgfsetbuttcap%
\pgfsetroundjoin%
\definecolor{currentfill}{rgb}{0.235526,0.309527,0.542944}%
\pgfsetfillcolor{currentfill}%
\pgfsetfillopacity{0.700000}%
\pgfsetlinewidth{0.000000pt}%
\definecolor{currentstroke}{rgb}{0.000000,0.000000,0.000000}%
\pgfsetstrokecolor{currentstroke}%
\pgfsetdash{}{0pt}%
\pgfpathmoveto{\pgfqpoint{3.465981in}{3.099234in}}%
\pgfpathlineto{\pgfqpoint{3.479262in}{3.089026in}}%
\pgfpathlineto{\pgfqpoint{3.492544in}{3.078928in}}%
\pgfpathlineto{\pgfqpoint{3.505828in}{3.068939in}}%
\pgfpathlineto{\pgfqpoint{3.519114in}{3.059059in}}%
\pgfpathlineto{\pgfqpoint{3.526921in}{3.075803in}}%
\pgfpathlineto{\pgfqpoint{3.534722in}{3.092799in}}%
\pgfpathlineto{\pgfqpoint{3.542519in}{3.110052in}}%
\pgfpathlineto{\pgfqpoint{3.550309in}{3.127569in}}%
\pgfpathlineto{\pgfqpoint{3.537027in}{3.137767in}}%
\pgfpathlineto{\pgfqpoint{3.523745in}{3.148073in}}%
\pgfpathlineto{\pgfqpoint{3.510466in}{3.158488in}}%
\pgfpathlineto{\pgfqpoint{3.497187in}{3.169014in}}%
\pgfpathlineto{\pgfqpoint{3.489394in}{3.151173in}}%
\pgfpathlineto{\pgfqpoint{3.481595in}{3.133599in}}%
\pgfpathlineto{\pgfqpoint{3.473791in}{3.116289in}}%
\pgfpathlineto{\pgfqpoint{3.465981in}{3.099234in}}%
\pgfpathclose%
\pgfusepath{fill}%
\end{pgfscope}%
\begin{pgfscope}%
\pgfpathrectangle{\pgfqpoint{1.150000in}{0.150000in}}{\pgfqpoint{5.700000in}{5.700000in}}%
\pgfusepath{clip}%
\pgfsetbuttcap%
\pgfsetroundjoin%
\definecolor{currentfill}{rgb}{0.585678,0.846661,0.249897}%
\pgfsetfillcolor{currentfill}%
\pgfsetfillopacity{0.700000}%
\pgfsetlinewidth{0.000000pt}%
\definecolor{currentstroke}{rgb}{0.000000,0.000000,0.000000}%
\pgfsetstrokecolor{currentstroke}%
\pgfsetdash{}{0pt}%
\pgfpathmoveto{\pgfqpoint{3.622492in}{4.593597in}}%
\pgfpathlineto{\pgfqpoint{3.635840in}{4.576145in}}%
\pgfpathlineto{\pgfqpoint{3.649186in}{4.558823in}}%
\pgfpathlineto{\pgfqpoint{3.662530in}{4.541631in}}%
\pgfpathlineto{\pgfqpoint{3.675872in}{4.524567in}}%
\pgfpathlineto{\pgfqpoint{3.683475in}{4.564235in}}%
\pgfpathlineto{\pgfqpoint{3.691076in}{4.604593in}}%
\pgfpathlineto{\pgfqpoint{3.698677in}{4.645652in}}%
\pgfpathlineto{\pgfqpoint{3.706276in}{4.687426in}}%
\pgfpathlineto{\pgfqpoint{3.692917in}{4.705101in}}%
\pgfpathlineto{\pgfqpoint{3.679556in}{4.722905in}}%
\pgfpathlineto{\pgfqpoint{3.666193in}{4.740839in}}%
\pgfpathlineto{\pgfqpoint{3.652828in}{4.758906in}}%
\pgfpathlineto{\pgfqpoint{3.645246in}{4.716508in}}%
\pgfpathlineto{\pgfqpoint{3.637663in}{4.674833in}}%
\pgfpathlineto{\pgfqpoint{3.630078in}{4.633867in}}%
\pgfpathlineto{\pgfqpoint{3.622492in}{4.593597in}}%
\pgfpathclose%
\pgfusepath{fill}%
\end{pgfscope}%
\begin{pgfscope}%
\pgfpathrectangle{\pgfqpoint{1.150000in}{0.150000in}}{\pgfqpoint{5.700000in}{5.700000in}}%
\pgfusepath{clip}%
\pgfsetbuttcap%
\pgfsetroundjoin%
\definecolor{currentfill}{rgb}{0.163625,0.471133,0.558148}%
\pgfsetfillcolor{currentfill}%
\pgfsetfillopacity{0.700000}%
\pgfsetlinewidth{0.000000pt}%
\definecolor{currentstroke}{rgb}{0.000000,0.000000,0.000000}%
\pgfsetstrokecolor{currentstroke}%
\pgfsetdash{}{0pt}%
\pgfpathmoveto{\pgfqpoint{4.223406in}{3.489623in}}%
\pgfpathlineto{\pgfqpoint{4.236757in}{3.479364in}}%
\pgfpathlineto{\pgfqpoint{4.250111in}{3.469193in}}%
\pgfpathlineto{\pgfqpoint{4.263467in}{3.459111in}}%
\pgfpathlineto{\pgfqpoint{4.276827in}{3.449116in}}%
\pgfpathlineto{\pgfqpoint{4.284525in}{3.474831in}}%
\pgfpathlineto{\pgfqpoint{4.292225in}{3.501023in}}%
\pgfpathlineto{\pgfqpoint{4.299927in}{3.527703in}}%
\pgfpathlineto{\pgfqpoint{4.307632in}{3.554879in}}%
\pgfpathlineto{\pgfqpoint{4.294271in}{3.565379in}}%
\pgfpathlineto{\pgfqpoint{4.280914in}{3.575966in}}%
\pgfpathlineto{\pgfqpoint{4.267559in}{3.586642in}}%
\pgfpathlineto{\pgfqpoint{4.254207in}{3.597407in}}%
\pgfpathlineto{\pgfqpoint{4.246504in}{3.569716in}}%
\pgfpathlineto{\pgfqpoint{4.238804in}{3.542529in}}%
\pgfpathlineto{\pgfqpoint{4.231104in}{3.515834in}}%
\pgfpathlineto{\pgfqpoint{4.223406in}{3.489623in}}%
\pgfpathclose%
\pgfusepath{fill}%
\end{pgfscope}%
\begin{pgfscope}%
\pgfpathrectangle{\pgfqpoint{1.150000in}{0.150000in}}{\pgfqpoint{5.700000in}{5.700000in}}%
\pgfusepath{clip}%
\pgfsetbuttcap%
\pgfsetroundjoin%
\definecolor{currentfill}{rgb}{0.125394,0.574318,0.549086}%
\pgfsetfillcolor{currentfill}%
\pgfsetfillopacity{0.700000}%
\pgfsetlinewidth{0.000000pt}%
\definecolor{currentstroke}{rgb}{0.000000,0.000000,0.000000}%
\pgfsetstrokecolor{currentstroke}%
\pgfsetdash{}{0pt}%
\pgfpathmoveto{\pgfqpoint{4.231644in}{3.759473in}}%
\pgfpathlineto{\pgfqpoint{4.244990in}{3.747818in}}%
\pgfpathlineto{\pgfqpoint{4.258338in}{3.736253in}}%
\pgfpathlineto{\pgfqpoint{4.271688in}{3.724779in}}%
\pgfpathlineto{\pgfqpoint{4.285040in}{3.713396in}}%
\pgfpathlineto{\pgfqpoint{4.292756in}{3.743752in}}%
\pgfpathlineto{\pgfqpoint{4.300475in}{3.774674in}}%
\pgfpathlineto{\pgfqpoint{4.308197in}{3.806171in}}%
\pgfpathlineto{\pgfqpoint{4.315923in}{3.838254in}}%
\pgfpathlineto{\pgfqpoint{4.302566in}{3.850191in}}%
\pgfpathlineto{\pgfqpoint{4.289211in}{3.862218in}}%
\pgfpathlineto{\pgfqpoint{4.275859in}{3.874337in}}%
\pgfpathlineto{\pgfqpoint{4.262508in}{3.886547in}}%
\pgfpathlineto{\pgfqpoint{4.254787in}{3.853900in}}%
\pgfpathlineto{\pgfqpoint{4.247069in}{3.821847in}}%
\pgfpathlineto{\pgfqpoint{4.239355in}{3.790374in}}%
\pgfpathlineto{\pgfqpoint{4.231644in}{3.759473in}}%
\pgfpathclose%
\pgfusepath{fill}%
\end{pgfscope}%
\begin{pgfscope}%
\pgfpathrectangle{\pgfqpoint{1.150000in}{0.150000in}}{\pgfqpoint{5.700000in}{5.700000in}}%
\pgfusepath{clip}%
\pgfsetbuttcap%
\pgfsetroundjoin%
\definecolor{currentfill}{rgb}{0.227802,0.326594,0.546532}%
\pgfsetfillcolor{currentfill}%
\pgfsetfillopacity{0.700000}%
\pgfsetlinewidth{0.000000pt}%
\definecolor{currentstroke}{rgb}{0.000000,0.000000,0.000000}%
\pgfsetstrokecolor{currentstroke}%
\pgfsetdash{}{0pt}%
\pgfpathmoveto{\pgfqpoint{3.190793in}{3.141136in}}%
\pgfpathlineto{\pgfqpoint{3.204071in}{3.129625in}}%
\pgfpathlineto{\pgfqpoint{3.217348in}{3.118239in}}%
\pgfpathlineto{\pgfqpoint{3.230626in}{3.106976in}}%
\pgfpathlineto{\pgfqpoint{3.243904in}{3.095836in}}%
\pgfpathlineto{\pgfqpoint{3.251774in}{3.111848in}}%
\pgfpathlineto{\pgfqpoint{3.259638in}{3.128088in}}%
\pgfpathlineto{\pgfqpoint{3.267494in}{3.144561in}}%
\pgfpathlineto{\pgfqpoint{3.275344in}{3.161273in}}%
\pgfpathlineto{\pgfqpoint{3.262070in}{3.172691in}}%
\pgfpathlineto{\pgfqpoint{3.248795in}{3.184232in}}%
\pgfpathlineto{\pgfqpoint{3.235520in}{3.195896in}}%
\pgfpathlineto{\pgfqpoint{3.222246in}{3.207685in}}%
\pgfpathlineto{\pgfqpoint{3.214393in}{3.190687in}}%
\pgfpathlineto{\pgfqpoint{3.206533in}{3.173933in}}%
\pgfpathlineto{\pgfqpoint{3.198666in}{3.157418in}}%
\pgfpathlineto{\pgfqpoint{3.190793in}{3.141136in}}%
\pgfpathclose%
\pgfusepath{fill}%
\end{pgfscope}%
\begin{pgfscope}%
\pgfpathrectangle{\pgfqpoint{1.150000in}{0.150000in}}{\pgfqpoint{5.700000in}{5.700000in}}%
\pgfusepath{clip}%
\pgfsetbuttcap%
\pgfsetroundjoin%
\definecolor{currentfill}{rgb}{0.212395,0.359683,0.551710}%
\pgfsetfillcolor{currentfill}%
\pgfsetfillopacity{0.700000}%
\pgfsetlinewidth{0.000000pt}%
\definecolor{currentstroke}{rgb}{0.000000,0.000000,0.000000}%
\pgfsetstrokecolor{currentstroke}%
\pgfsetdash{}{0pt}%
\pgfpathmoveto{\pgfqpoint{2.999800in}{3.223391in}}%
\pgfpathlineto{\pgfqpoint{3.013092in}{3.210557in}}%
\pgfpathlineto{\pgfqpoint{3.026382in}{3.197860in}}%
\pgfpathlineto{\pgfqpoint{3.039670in}{3.185301in}}%
\pgfpathlineto{\pgfqpoint{3.052958in}{3.172878in}}%
\pgfpathlineto{\pgfqpoint{3.060868in}{3.188777in}}%
\pgfpathlineto{\pgfqpoint{3.068770in}{3.204900in}}%
\pgfpathlineto{\pgfqpoint{3.076665in}{3.221252in}}%
\pgfpathlineto{\pgfqpoint{3.084552in}{3.237838in}}%
\pgfpathlineto{\pgfqpoint{3.071269in}{3.250520in}}%
\pgfpathlineto{\pgfqpoint{3.057983in}{3.263338in}}%
\pgfpathlineto{\pgfqpoint{3.044697in}{3.276293in}}%
\pgfpathlineto{\pgfqpoint{3.031409in}{3.289387in}}%
\pgfpathlineto{\pgfqpoint{3.023518in}{3.272535in}}%
\pgfpathlineto{\pgfqpoint{3.015620in}{3.255921in}}%
\pgfpathlineto{\pgfqpoint{3.007714in}{3.239542in}}%
\pgfpathlineto{\pgfqpoint{2.999800in}{3.223391in}}%
\pgfpathclose%
\pgfusepath{fill}%
\end{pgfscope}%
\begin{pgfscope}%
\pgfpathrectangle{\pgfqpoint{1.150000in}{0.150000in}}{\pgfqpoint{5.700000in}{5.700000in}}%
\pgfusepath{clip}%
\pgfsetbuttcap%
\pgfsetroundjoin%
\definecolor{currentfill}{rgb}{0.179019,0.433756,0.557430}%
\pgfsetfillcolor{currentfill}%
\pgfsetfillopacity{0.700000}%
\pgfsetlinewidth{0.000000pt}%
\definecolor{currentstroke}{rgb}{0.000000,0.000000,0.000000}%
\pgfsetstrokecolor{currentstroke}%
\pgfsetdash{}{0pt}%
\pgfpathmoveto{\pgfqpoint{4.192623in}{3.389419in}}%
\pgfpathlineto{\pgfqpoint{4.205974in}{3.379643in}}%
\pgfpathlineto{\pgfqpoint{4.219328in}{3.369955in}}%
\pgfpathlineto{\pgfqpoint{4.232684in}{3.360355in}}%
\pgfpathlineto{\pgfqpoint{4.246044in}{3.350842in}}%
\pgfpathlineto{\pgfqpoint{4.253739in}{3.374741in}}%
\pgfpathlineto{\pgfqpoint{4.261434in}{3.399080in}}%
\pgfpathlineto{\pgfqpoint{4.269130in}{3.423869in}}%
\pgfpathlineto{\pgfqpoint{4.276827in}{3.449116in}}%
\pgfpathlineto{\pgfqpoint{4.263467in}{3.459111in}}%
\pgfpathlineto{\pgfqpoint{4.250111in}{3.469193in}}%
\pgfpathlineto{\pgfqpoint{4.236757in}{3.479364in}}%
\pgfpathlineto{\pgfqpoint{4.223406in}{3.489623in}}%
\pgfpathlineto{\pgfqpoint{4.215710in}{3.463885in}}%
\pgfpathlineto{\pgfqpoint{4.208014in}{3.438612in}}%
\pgfpathlineto{\pgfqpoint{4.200318in}{3.413793in}}%
\pgfpathlineto{\pgfqpoint{4.192623in}{3.389419in}}%
\pgfpathclose%
\pgfusepath{fill}%
\end{pgfscope}%
\begin{pgfscope}%
\pgfpathrectangle{\pgfqpoint{1.150000in}{0.150000in}}{\pgfqpoint{5.700000in}{5.700000in}}%
\pgfusepath{clip}%
\pgfsetbuttcap%
\pgfsetroundjoin%
\definecolor{currentfill}{rgb}{0.214298,0.355619,0.551184}%
\pgfsetfillcolor{currentfill}%
\pgfsetfillopacity{0.700000}%
\pgfsetlinewidth{0.000000pt}%
\definecolor{currentstroke}{rgb}{0.000000,0.000000,0.000000}%
\pgfsetstrokecolor{currentstroke}%
\pgfsetdash{}{0pt}%
\pgfpathmoveto{\pgfqpoint{3.993514in}{3.199416in}}%
\pgfpathlineto{\pgfqpoint{4.006845in}{3.190236in}}%
\pgfpathlineto{\pgfqpoint{4.020178in}{3.181147in}}%
\pgfpathlineto{\pgfqpoint{4.033515in}{3.172150in}}%
\pgfpathlineto{\pgfqpoint{4.046855in}{3.163244in}}%
\pgfpathlineto{\pgfqpoint{4.054562in}{3.183353in}}%
\pgfpathlineto{\pgfqpoint{4.062268in}{3.203812in}}%
\pgfpathlineto{\pgfqpoint{4.069972in}{3.224628in}}%
\pgfpathlineto{\pgfqpoint{4.077674in}{3.245810in}}%
\pgfpathlineto{\pgfqpoint{4.064336in}{3.255135in}}%
\pgfpathlineto{\pgfqpoint{4.051002in}{3.264550in}}%
\pgfpathlineto{\pgfqpoint{4.037670in}{3.274057in}}%
\pgfpathlineto{\pgfqpoint{4.024341in}{3.283656in}}%
\pgfpathlineto{\pgfqpoint{4.016637in}{3.262047in}}%
\pgfpathlineto{\pgfqpoint{4.008932in}{3.240810in}}%
\pgfpathlineto{\pgfqpoint{4.001224in}{3.219935in}}%
\pgfpathlineto{\pgfqpoint{3.993514in}{3.199416in}}%
\pgfpathclose%
\pgfusepath{fill}%
\end{pgfscope}%
\begin{pgfscope}%
\pgfpathrectangle{\pgfqpoint{1.150000in}{0.150000in}}{\pgfqpoint{5.700000in}{5.700000in}}%
\pgfusepath{clip}%
\pgfsetbuttcap%
\pgfsetroundjoin%
\definecolor{currentfill}{rgb}{0.237441,0.305202,0.541921}%
\pgfsetfillcolor{currentfill}%
\pgfsetfillopacity{0.700000}%
\pgfsetlinewidth{0.000000pt}%
\definecolor{currentstroke}{rgb}{0.000000,0.000000,0.000000}%
\pgfsetstrokecolor{currentstroke}%
\pgfsetdash{}{0pt}%
\pgfpathmoveto{\pgfqpoint{3.603456in}{3.087845in}}%
\pgfpathlineto{\pgfqpoint{3.616747in}{3.078177in}}%
\pgfpathlineto{\pgfqpoint{3.630040in}{3.068614in}}%
\pgfpathlineto{\pgfqpoint{3.643335in}{3.059153in}}%
\pgfpathlineto{\pgfqpoint{3.656632in}{3.049795in}}%
\pgfpathlineto{\pgfqpoint{3.664412in}{3.066921in}}%
\pgfpathlineto{\pgfqpoint{3.672187in}{3.084311in}}%
\pgfpathlineto{\pgfqpoint{3.679957in}{3.101973in}}%
\pgfpathlineto{\pgfqpoint{3.687722in}{3.119913in}}%
\pgfpathlineto{\pgfqpoint{3.674428in}{3.129608in}}%
\pgfpathlineto{\pgfqpoint{3.661135in}{3.139406in}}%
\pgfpathlineto{\pgfqpoint{3.647845in}{3.149307in}}%
\pgfpathlineto{\pgfqpoint{3.634556in}{3.159312in}}%
\pgfpathlineto{\pgfqpoint{3.626789in}{3.141027in}}%
\pgfpathlineto{\pgfqpoint{3.619016in}{3.123026in}}%
\pgfpathlineto{\pgfqpoint{3.611238in}{3.105300in}}%
\pgfpathlineto{\pgfqpoint{3.603456in}{3.087845in}}%
\pgfpathclose%
\pgfusepath{fill}%
\end{pgfscope}%
\begin{pgfscope}%
\pgfpathrectangle{\pgfqpoint{1.150000in}{0.150000in}}{\pgfqpoint{5.700000in}{5.700000in}}%
\pgfusepath{clip}%
\pgfsetbuttcap%
\pgfsetroundjoin%
\definecolor{currentfill}{rgb}{0.147607,0.511733,0.557049}%
\pgfsetfillcolor{currentfill}%
\pgfsetfillopacity{0.700000}%
\pgfsetlinewidth{0.000000pt}%
\definecolor{currentstroke}{rgb}{0.000000,0.000000,0.000000}%
\pgfsetstrokecolor{currentstroke}%
\pgfsetdash{}{0pt}%
\pgfpathmoveto{\pgfqpoint{4.254207in}{3.597407in}}%
\pgfpathlineto{\pgfqpoint{4.267559in}{3.586642in}}%
\pgfpathlineto{\pgfqpoint{4.280914in}{3.575966in}}%
\pgfpathlineto{\pgfqpoint{4.294271in}{3.565379in}}%
\pgfpathlineto{\pgfqpoint{4.307632in}{3.554879in}}%
\pgfpathlineto{\pgfqpoint{4.315338in}{3.582562in}}%
\pgfpathlineto{\pgfqpoint{4.323047in}{3.610763in}}%
\pgfpathlineto{\pgfqpoint{4.330759in}{3.639490in}}%
\pgfpathlineto{\pgfqpoint{4.338475in}{3.668755in}}%
\pgfpathlineto{\pgfqpoint{4.325112in}{3.679783in}}%
\pgfpathlineto{\pgfqpoint{4.311753in}{3.690898in}}%
\pgfpathlineto{\pgfqpoint{4.298395in}{3.702102in}}%
\pgfpathlineto{\pgfqpoint{4.285040in}{3.713396in}}%
\pgfpathlineto{\pgfqpoint{4.277328in}{3.683594in}}%
\pgfpathlineto{\pgfqpoint{4.269619in}{3.654335in}}%
\pgfpathlineto{\pgfqpoint{4.261912in}{3.625609in}}%
\pgfpathlineto{\pgfqpoint{4.254207in}{3.597407in}}%
\pgfpathclose%
\pgfusepath{fill}%
\end{pgfscope}%
\begin{pgfscope}%
\pgfpathrectangle{\pgfqpoint{1.150000in}{0.150000in}}{\pgfqpoint{5.700000in}{5.700000in}}%
\pgfusepath{clip}%
\pgfsetbuttcap%
\pgfsetroundjoin%
\definecolor{currentfill}{rgb}{0.137339,0.662252,0.515571}%
\pgfsetfillcolor{currentfill}%
\pgfsetfillopacity{0.700000}%
\pgfsetlinewidth{0.000000pt}%
\definecolor{currentstroke}{rgb}{0.000000,0.000000,0.000000}%
\pgfsetstrokecolor{currentstroke}%
\pgfsetdash{}{0pt}%
\pgfpathmoveto{\pgfqpoint{4.155762in}{3.987600in}}%
\pgfpathlineto{\pgfqpoint{4.169100in}{3.974636in}}%
\pgfpathlineto{\pgfqpoint{4.182439in}{3.961767in}}%
\pgfpathlineto{\pgfqpoint{4.195780in}{3.948995in}}%
\pgfpathlineto{\pgfqpoint{4.209122in}{3.936317in}}%
\pgfpathlineto{\pgfqpoint{4.216840in}{3.970140in}}%
\pgfpathlineto{\pgfqpoint{4.224562in}{4.004584in}}%
\pgfpathlineto{\pgfqpoint{4.232288in}{4.039662in}}%
\pgfpathlineto{\pgfqpoint{4.240018in}{4.075385in}}%
\pgfpathlineto{\pgfqpoint{4.226668in}{4.088644in}}%
\pgfpathlineto{\pgfqpoint{4.213319in}{4.102000in}}%
\pgfpathlineto{\pgfqpoint{4.199971in}{4.115451in}}%
\pgfpathlineto{\pgfqpoint{4.186625in}{4.129000in}}%
\pgfpathlineto{\pgfqpoint{4.178903in}{4.092684in}}%
\pgfpathlineto{\pgfqpoint{4.171186in}{4.057020in}}%
\pgfpathlineto{\pgfqpoint{4.163472in}{4.021996in}}%
\pgfpathlineto{\pgfqpoint{4.155762in}{3.987600in}}%
\pgfpathclose%
\pgfusepath{fill}%
\end{pgfscope}%
\begin{pgfscope}%
\pgfpathrectangle{\pgfqpoint{1.150000in}{0.150000in}}{\pgfqpoint{5.700000in}{5.700000in}}%
\pgfusepath{clip}%
\pgfsetbuttcap%
\pgfsetroundjoin%
\definecolor{currentfill}{rgb}{0.223925,0.334994,0.548053}%
\pgfsetfillcolor{currentfill}%
\pgfsetfillopacity{0.700000}%
\pgfsetlinewidth{0.000000pt}%
\definecolor{currentstroke}{rgb}{0.000000,0.000000,0.000000}%
\pgfsetstrokecolor{currentstroke}%
\pgfsetdash{}{0pt}%
\pgfpathmoveto{\pgfqpoint{3.909347in}{3.156799in}}%
\pgfpathlineto{\pgfqpoint{3.922668in}{3.147643in}}%
\pgfpathlineto{\pgfqpoint{3.935992in}{3.138582in}}%
\pgfpathlineto{\pgfqpoint{3.949319in}{3.129614in}}%
\pgfpathlineto{\pgfqpoint{3.962649in}{3.120739in}}%
\pgfpathlineto{\pgfqpoint{3.970370in}{3.139913in}}%
\pgfpathlineto{\pgfqpoint{3.978087in}{3.159413in}}%
\pgfpathlineto{\pgfqpoint{3.985802in}{3.179244in}}%
\pgfpathlineto{\pgfqpoint{3.993514in}{3.199416in}}%
\pgfpathlineto{\pgfqpoint{3.980186in}{3.208689in}}%
\pgfpathlineto{\pgfqpoint{3.966862in}{3.218054in}}%
\pgfpathlineto{\pgfqpoint{3.953540in}{3.227514in}}%
\pgfpathlineto{\pgfqpoint{3.940220in}{3.237067in}}%
\pgfpathlineto{\pgfqpoint{3.932506in}{3.216489in}}%
\pgfpathlineto{\pgfqpoint{3.924789in}{3.196257in}}%
\pgfpathlineto{\pgfqpoint{3.917069in}{3.176363in}}%
\pgfpathlineto{\pgfqpoint{3.909347in}{3.156799in}}%
\pgfpathclose%
\pgfusepath{fill}%
\end{pgfscope}%
\begin{pgfscope}%
\pgfpathrectangle{\pgfqpoint{1.150000in}{0.150000in}}{\pgfqpoint{5.700000in}{5.700000in}}%
\pgfusepath{clip}%
\pgfsetbuttcap%
\pgfsetroundjoin%
\definecolor{currentfill}{rgb}{0.657642,0.860219,0.203082}%
\pgfsetfillcolor{currentfill}%
\pgfsetfillopacity{0.700000}%
\pgfsetlinewidth{0.000000pt}%
\definecolor{currentstroke}{rgb}{0.000000,0.000000,0.000000}%
\pgfsetstrokecolor{currentstroke}%
\pgfsetdash{}{0pt}%
\pgfpathmoveto{\pgfqpoint{3.569078in}{4.664741in}}%
\pgfpathlineto{\pgfqpoint{3.582435in}{4.646753in}}%
\pgfpathlineto{\pgfqpoint{3.595790in}{4.628900in}}%
\pgfpathlineto{\pgfqpoint{3.609142in}{4.611182in}}%
\pgfpathlineto{\pgfqpoint{3.622492in}{4.593597in}}%
\pgfpathlineto{\pgfqpoint{3.630078in}{4.633867in}}%
\pgfpathlineto{\pgfqpoint{3.637663in}{4.674833in}}%
\pgfpathlineto{\pgfqpoint{3.645246in}{4.716508in}}%
\pgfpathlineto{\pgfqpoint{3.652828in}{4.758906in}}%
\pgfpathlineto{\pgfqpoint{3.639460in}{4.777105in}}%
\pgfpathlineto{\pgfqpoint{3.626090in}{4.795438in}}%
\pgfpathlineto{\pgfqpoint{3.612717in}{4.813907in}}%
\pgfpathlineto{\pgfqpoint{3.599342in}{4.832513in}}%
\pgfpathlineto{\pgfqpoint{3.591779in}{4.789488in}}%
\pgfpathlineto{\pgfqpoint{3.584214in}{4.747193in}}%
\pgfpathlineto{\pgfqpoint{3.576647in}{4.705615in}}%
\pgfpathlineto{\pgfqpoint{3.569078in}{4.664741in}}%
\pgfpathclose%
\pgfusepath{fill}%
\end{pgfscope}%
\begin{pgfscope}%
\pgfpathrectangle{\pgfqpoint{1.150000in}{0.150000in}}{\pgfqpoint{5.700000in}{5.700000in}}%
\pgfusepath{clip}%
\pgfsetbuttcap%
\pgfsetroundjoin%
\definecolor{currentfill}{rgb}{0.204903,0.375746,0.553533}%
\pgfsetfillcolor{currentfill}%
\pgfsetfillopacity{0.700000}%
\pgfsetlinewidth{0.000000pt}%
\definecolor{currentstroke}{rgb}{0.000000,0.000000,0.000000}%
\pgfsetstrokecolor{currentstroke}%
\pgfsetdash{}{0pt}%
\pgfpathmoveto{\pgfqpoint{4.077674in}{3.245810in}}%
\pgfpathlineto{\pgfqpoint{4.091015in}{3.236576in}}%
\pgfpathlineto{\pgfqpoint{4.104358in}{3.227432in}}%
\pgfpathlineto{\pgfqpoint{4.117706in}{3.218378in}}%
\pgfpathlineto{\pgfqpoint{4.131056in}{3.209412in}}%
\pgfpathlineto{\pgfqpoint{4.138754in}{3.230537in}}%
\pgfpathlineto{\pgfqpoint{4.146452in}{3.252038in}}%
\pgfpathlineto{\pgfqpoint{4.154148in}{3.273924in}}%
\pgfpathlineto{\pgfqpoint{4.161844in}{3.296203in}}%
\pgfpathlineto{\pgfqpoint{4.148496in}{3.305607in}}%
\pgfpathlineto{\pgfqpoint{4.135150in}{3.315101in}}%
\pgfpathlineto{\pgfqpoint{4.121808in}{3.324684in}}%
\pgfpathlineto{\pgfqpoint{4.108469in}{3.334358in}}%
\pgfpathlineto{\pgfqpoint{4.100772in}{3.311631in}}%
\pgfpathlineto{\pgfqpoint{4.093074in}{3.289304in}}%
\pgfpathlineto{\pgfqpoint{4.085374in}{3.267366in}}%
\pgfpathlineto{\pgfqpoint{4.077674in}{3.245810in}}%
\pgfpathclose%
\pgfusepath{fill}%
\end{pgfscope}%
\begin{pgfscope}%
\pgfpathrectangle{\pgfqpoint{1.150000in}{0.150000in}}{\pgfqpoint{5.700000in}{5.700000in}}%
\pgfusepath{clip}%
\pgfsetbuttcap%
\pgfsetroundjoin%
\definecolor{currentfill}{rgb}{0.231674,0.318106,0.544834}%
\pgfsetfillcolor{currentfill}%
\pgfsetfillopacity{0.700000}%
\pgfsetlinewidth{0.000000pt}%
\definecolor{currentstroke}{rgb}{0.000000,0.000000,0.000000}%
\pgfsetstrokecolor{currentstroke}%
\pgfsetdash{}{0pt}%
\pgfpathmoveto{\pgfqpoint{3.825154in}{3.117764in}}%
\pgfpathlineto{\pgfqpoint{3.838467in}{3.108604in}}%
\pgfpathlineto{\pgfqpoint{3.851783in}{3.099541in}}%
\pgfpathlineto{\pgfqpoint{3.865101in}{3.090574in}}%
\pgfpathlineto{\pgfqpoint{3.878423in}{3.081701in}}%
\pgfpathlineto{\pgfqpoint{3.886159in}{3.100016in}}%
\pgfpathlineto{\pgfqpoint{3.893891in}{3.118632in}}%
\pgfpathlineto{\pgfqpoint{3.901621in}{3.137558in}}%
\pgfpathlineto{\pgfqpoint{3.909347in}{3.156799in}}%
\pgfpathlineto{\pgfqpoint{3.896028in}{3.166048in}}%
\pgfpathlineto{\pgfqpoint{3.882712in}{3.175393in}}%
\pgfpathlineto{\pgfqpoint{3.869399in}{3.184834in}}%
\pgfpathlineto{\pgfqpoint{3.856088in}{3.194371in}}%
\pgfpathlineto{\pgfqpoint{3.848360in}{3.174744in}}%
\pgfpathlineto{\pgfqpoint{3.840628in}{3.155439in}}%
\pgfpathlineto{\pgfqpoint{3.832893in}{3.136448in}}%
\pgfpathlineto{\pgfqpoint{3.825154in}{3.117764in}}%
\pgfpathclose%
\pgfusepath{fill}%
\end{pgfscope}%
\begin{pgfscope}%
\pgfpathrectangle{\pgfqpoint{1.150000in}{0.150000in}}{\pgfqpoint{5.700000in}{5.700000in}}%
\pgfusepath{clip}%
\pgfsetbuttcap%
\pgfsetroundjoin%
\definecolor{currentfill}{rgb}{0.126326,0.644107,0.525311}%
\pgfsetfillcolor{currentfill}%
\pgfsetfillopacity{0.700000}%
\pgfsetlinewidth{0.000000pt}%
\definecolor{currentstroke}{rgb}{0.000000,0.000000,0.000000}%
\pgfsetstrokecolor{currentstroke}%
\pgfsetdash{}{0pt}%
\pgfpathmoveto{\pgfqpoint{4.209122in}{3.936317in}}%
\pgfpathlineto{\pgfqpoint{4.222466in}{3.923734in}}%
\pgfpathlineto{\pgfqpoint{4.235811in}{3.911245in}}%
\pgfpathlineto{\pgfqpoint{4.249159in}{3.898850in}}%
\pgfpathlineto{\pgfqpoint{4.262508in}{3.886547in}}%
\pgfpathlineto{\pgfqpoint{4.270232in}{3.919798in}}%
\pgfpathlineto{\pgfqpoint{4.277961in}{3.953665in}}%
\pgfpathlineto{\pgfqpoint{4.285695in}{3.988158in}}%
\pgfpathlineto{\pgfqpoint{4.293433in}{4.023291in}}%
\pgfpathlineto{\pgfqpoint{4.280077in}{4.036174in}}%
\pgfpathlineto{\pgfqpoint{4.266722in}{4.049150in}}%
\pgfpathlineto{\pgfqpoint{4.253369in}{4.062220in}}%
\pgfpathlineto{\pgfqpoint{4.240018in}{4.075385in}}%
\pgfpathlineto{\pgfqpoint{4.232288in}{4.039662in}}%
\pgfpathlineto{\pgfqpoint{4.224562in}{4.004584in}}%
\pgfpathlineto{\pgfqpoint{4.216840in}{3.970140in}}%
\pgfpathlineto{\pgfqpoint{4.209122in}{3.936317in}}%
\pgfpathclose%
\pgfusepath{fill}%
\end{pgfscope}%
\begin{pgfscope}%
\pgfpathrectangle{\pgfqpoint{1.150000in}{0.150000in}}{\pgfqpoint{5.700000in}{5.700000in}}%
\pgfusepath{clip}%
\pgfsetbuttcap%
\pgfsetroundjoin%
\definecolor{currentfill}{rgb}{0.195860,0.395433,0.555276}%
\pgfsetfillcolor{currentfill}%
\pgfsetfillopacity{0.700000}%
\pgfsetlinewidth{0.000000pt}%
\definecolor{currentstroke}{rgb}{0.000000,0.000000,0.000000}%
\pgfsetstrokecolor{currentstroke}%
\pgfsetdash{}{0pt}%
\pgfpathmoveto{\pgfqpoint{4.161844in}{3.296203in}}%
\pgfpathlineto{\pgfqpoint{4.175196in}{3.286887in}}%
\pgfpathlineto{\pgfqpoint{4.188550in}{3.277659in}}%
\pgfpathlineto{\pgfqpoint{4.201909in}{3.268519in}}%
\pgfpathlineto{\pgfqpoint{4.215270in}{3.259466in}}%
\pgfpathlineto{\pgfqpoint{4.222964in}{3.281695in}}%
\pgfpathlineto{\pgfqpoint{4.230657in}{3.304327in}}%
\pgfpathlineto{\pgfqpoint{4.238350in}{3.327374in}}%
\pgfpathlineto{\pgfqpoint{4.246044in}{3.350842in}}%
\pgfpathlineto{\pgfqpoint{4.232684in}{3.360355in}}%
\pgfpathlineto{\pgfqpoint{4.219328in}{3.369955in}}%
\pgfpathlineto{\pgfqpoint{4.205974in}{3.379643in}}%
\pgfpathlineto{\pgfqpoint{4.192623in}{3.389419in}}%
\pgfpathlineto{\pgfqpoint{4.184929in}{3.365482in}}%
\pgfpathlineto{\pgfqpoint{4.177234in}{3.341973in}}%
\pgfpathlineto{\pgfqpoint{4.169539in}{3.318883in}}%
\pgfpathlineto{\pgfqpoint{4.161844in}{3.296203in}}%
\pgfpathclose%
\pgfusepath{fill}%
\end{pgfscope}%
\begin{pgfscope}%
\pgfpathrectangle{\pgfqpoint{1.150000in}{0.150000in}}{\pgfqpoint{5.700000in}{5.700000in}}%
\pgfusepath{clip}%
\pgfsetbuttcap%
\pgfsetroundjoin%
\definecolor{currentfill}{rgb}{0.221989,0.339161,0.548752}%
\pgfsetfillcolor{currentfill}%
\pgfsetfillopacity{0.700000}%
\pgfsetlinewidth{0.000000pt}%
\definecolor{currentstroke}{rgb}{0.000000,0.000000,0.000000}%
\pgfsetstrokecolor{currentstroke}%
\pgfsetdash{}{0pt}%
\pgfpathmoveto{\pgfqpoint{3.052958in}{3.172878in}}%
\pgfpathlineto{\pgfqpoint{3.066244in}{3.160590in}}%
\pgfpathlineto{\pgfqpoint{3.079529in}{3.148435in}}%
\pgfpathlineto{\pgfqpoint{3.092814in}{3.136412in}}%
\pgfpathlineto{\pgfqpoint{3.106098in}{3.124520in}}%
\pgfpathlineto{\pgfqpoint{3.114004in}{3.140169in}}%
\pgfpathlineto{\pgfqpoint{3.121902in}{3.156036in}}%
\pgfpathlineto{\pgfqpoint{3.129794in}{3.172127in}}%
\pgfpathlineto{\pgfqpoint{3.137678in}{3.188446in}}%
\pgfpathlineto{\pgfqpoint{3.124398in}{3.200596in}}%
\pgfpathlineto{\pgfqpoint{3.111117in}{3.212877in}}%
\pgfpathlineto{\pgfqpoint{3.097835in}{3.225291in}}%
\pgfpathlineto{\pgfqpoint{3.084552in}{3.237838in}}%
\pgfpathlineto{\pgfqpoint{3.076665in}{3.221252in}}%
\pgfpathlineto{\pgfqpoint{3.068770in}{3.204900in}}%
\pgfpathlineto{\pgfqpoint{3.060868in}{3.188777in}}%
\pgfpathlineto{\pgfqpoint{3.052958in}{3.172878in}}%
\pgfpathclose%
\pgfusepath{fill}%
\end{pgfscope}%
\begin{pgfscope}%
\pgfpathrectangle{\pgfqpoint{1.150000in}{0.150000in}}{\pgfqpoint{5.700000in}{5.700000in}}%
\pgfusepath{clip}%
\pgfsetbuttcap%
\pgfsetroundjoin%
\definecolor{currentfill}{rgb}{0.239346,0.300855,0.540844}%
\pgfsetfillcolor{currentfill}%
\pgfsetfillopacity{0.700000}%
\pgfsetlinewidth{0.000000pt}%
\definecolor{currentstroke}{rgb}{0.000000,0.000000,0.000000}%
\pgfsetstrokecolor{currentstroke}%
\pgfsetdash{}{0pt}%
\pgfpathmoveto{\pgfqpoint{3.381556in}{3.074224in}}%
\pgfpathlineto{\pgfqpoint{3.394836in}{3.063866in}}%
\pgfpathlineto{\pgfqpoint{3.408116in}{3.053620in}}%
\pgfpathlineto{\pgfqpoint{3.421398in}{3.043487in}}%
\pgfpathlineto{\pgfqpoint{3.434682in}{3.033466in}}%
\pgfpathlineto{\pgfqpoint{3.442515in}{3.049552in}}%
\pgfpathlineto{\pgfqpoint{3.450343in}{3.065871in}}%
\pgfpathlineto{\pgfqpoint{3.458165in}{3.082430in}}%
\pgfpathlineto{\pgfqpoint{3.465981in}{3.099234in}}%
\pgfpathlineto{\pgfqpoint{3.452701in}{3.109553in}}%
\pgfpathlineto{\pgfqpoint{3.439422in}{3.119983in}}%
\pgfpathlineto{\pgfqpoint{3.426145in}{3.130525in}}%
\pgfpathlineto{\pgfqpoint{3.412868in}{3.141181in}}%
\pgfpathlineto{\pgfqpoint{3.405049in}{3.124072in}}%
\pgfpathlineto{\pgfqpoint{3.397224in}{3.107214in}}%
\pgfpathlineto{\pgfqpoint{3.389393in}{3.090599in}}%
\pgfpathlineto{\pgfqpoint{3.381556in}{3.074224in}}%
\pgfpathclose%
\pgfusepath{fill}%
\end{pgfscope}%
\begin{pgfscope}%
\pgfpathrectangle{\pgfqpoint{1.150000in}{0.150000in}}{\pgfqpoint{5.700000in}{5.700000in}}%
\pgfusepath{clip}%
\pgfsetbuttcap%
\pgfsetroundjoin%
\definecolor{currentfill}{rgb}{0.129933,0.559582,0.551864}%
\pgfsetfillcolor{currentfill}%
\pgfsetfillopacity{0.700000}%
\pgfsetlinewidth{0.000000pt}%
\definecolor{currentstroke}{rgb}{0.000000,0.000000,0.000000}%
\pgfsetstrokecolor{currentstroke}%
\pgfsetdash{}{0pt}%
\pgfpathmoveto{\pgfqpoint{4.285040in}{3.713396in}}%
\pgfpathlineto{\pgfqpoint{4.298395in}{3.702102in}}%
\pgfpathlineto{\pgfqpoint{4.311753in}{3.690898in}}%
\pgfpathlineto{\pgfqpoint{4.325112in}{3.679783in}}%
\pgfpathlineto{\pgfqpoint{4.338475in}{3.668755in}}%
\pgfpathlineto{\pgfqpoint{4.346193in}{3.698569in}}%
\pgfpathlineto{\pgfqpoint{4.353916in}{3.728941in}}%
\pgfpathlineto{\pgfqpoint{4.361642in}{3.759882in}}%
\pgfpathlineto{\pgfqpoint{4.369373in}{3.791404in}}%
\pgfpathlineto{\pgfqpoint{4.356007in}{3.802983in}}%
\pgfpathlineto{\pgfqpoint{4.342644in}{3.814651in}}%
\pgfpathlineto{\pgfqpoint{4.329282in}{3.826408in}}%
\pgfpathlineto{\pgfqpoint{4.315923in}{3.838254in}}%
\pgfpathlineto{\pgfqpoint{4.308197in}{3.806171in}}%
\pgfpathlineto{\pgfqpoint{4.300475in}{3.774674in}}%
\pgfpathlineto{\pgfqpoint{4.292756in}{3.743752in}}%
\pgfpathlineto{\pgfqpoint{4.285040in}{3.713396in}}%
\pgfpathclose%
\pgfusepath{fill}%
\end{pgfscope}%
\begin{pgfscope}%
\pgfpathrectangle{\pgfqpoint{1.150000in}{0.150000in}}{\pgfqpoint{5.700000in}{5.700000in}}%
\pgfusepath{clip}%
\pgfsetbuttcap%
\pgfsetroundjoin%
\definecolor{currentfill}{rgb}{0.237441,0.305202,0.541921}%
\pgfsetfillcolor{currentfill}%
\pgfsetfillopacity{0.700000}%
\pgfsetlinewidth{0.000000pt}%
\definecolor{currentstroke}{rgb}{0.000000,0.000000,0.000000}%
\pgfsetstrokecolor{currentstroke}%
\pgfsetdash{}{0pt}%
\pgfpathmoveto{\pgfqpoint{3.740921in}{3.082143in}}%
\pgfpathlineto{\pgfqpoint{3.754227in}{3.072950in}}%
\pgfpathlineto{\pgfqpoint{3.767535in}{3.063856in}}%
\pgfpathlineto{\pgfqpoint{3.780846in}{3.054860in}}%
\pgfpathlineto{\pgfqpoint{3.794159in}{3.045960in}}%
\pgfpathlineto{\pgfqpoint{3.801914in}{3.063485in}}%
\pgfpathlineto{\pgfqpoint{3.809665in}{3.081289in}}%
\pgfpathlineto{\pgfqpoint{3.817411in}{3.099380in}}%
\pgfpathlineto{\pgfqpoint{3.825154in}{3.117764in}}%
\pgfpathlineto{\pgfqpoint{3.811844in}{3.127020in}}%
\pgfpathlineto{\pgfqpoint{3.798536in}{3.136373in}}%
\pgfpathlineto{\pgfqpoint{3.785231in}{3.145824in}}%
\pgfpathlineto{\pgfqpoint{3.771928in}{3.155375in}}%
\pgfpathlineto{\pgfqpoint{3.764182in}{3.136626in}}%
\pgfpathlineto{\pgfqpoint{3.756433in}{3.118175in}}%
\pgfpathlineto{\pgfqpoint{3.748679in}{3.100017in}}%
\pgfpathlineto{\pgfqpoint{3.740921in}{3.082143in}}%
\pgfpathclose%
\pgfusepath{fill}%
\end{pgfscope}%
\begin{pgfscope}%
\pgfpathrectangle{\pgfqpoint{1.150000in}{0.150000in}}{\pgfqpoint{5.700000in}{5.700000in}}%
\pgfusepath{clip}%
\pgfsetbuttcap%
\pgfsetroundjoin%
\definecolor{currentfill}{rgb}{0.741388,0.873449,0.149561}%
\pgfsetfillcolor{currentfill}%
\pgfsetfillopacity{0.700000}%
\pgfsetlinewidth{0.000000pt}%
\definecolor{currentstroke}{rgb}{0.000000,0.000000,0.000000}%
\pgfsetstrokecolor{currentstroke}%
\pgfsetdash{}{0pt}%
\pgfpathmoveto{\pgfqpoint{3.515623in}{4.738075in}}%
\pgfpathlineto{\pgfqpoint{3.528991in}{4.719531in}}%
\pgfpathlineto{\pgfqpoint{3.542356in}{4.701129in}}%
\pgfpathlineto{\pgfqpoint{3.555719in}{4.682866in}}%
\pgfpathlineto{\pgfqpoint{3.569078in}{4.664741in}}%
\pgfpathlineto{\pgfqpoint{3.576647in}{4.705615in}}%
\pgfpathlineto{\pgfqpoint{3.584214in}{4.747193in}}%
\pgfpathlineto{\pgfqpoint{3.591779in}{4.789488in}}%
\pgfpathlineto{\pgfqpoint{3.599342in}{4.832513in}}%
\pgfpathlineto{\pgfqpoint{3.585963in}{4.851256in}}%
\pgfpathlineto{\pgfqpoint{3.572582in}{4.870139in}}%
\pgfpathlineto{\pgfqpoint{3.559198in}{4.889162in}}%
\pgfpathlineto{\pgfqpoint{3.545811in}{4.908327in}}%
\pgfpathlineto{\pgfqpoint{3.538268in}{4.864671in}}%
\pgfpathlineto{\pgfqpoint{3.530723in}{4.821752in}}%
\pgfpathlineto{\pgfqpoint{3.523174in}{4.779558in}}%
\pgfpathlineto{\pgfqpoint{3.515623in}{4.738075in}}%
\pgfpathclose%
\pgfusepath{fill}%
\end{pgfscope}%
\begin{pgfscope}%
\pgfpathrectangle{\pgfqpoint{1.150000in}{0.150000in}}{\pgfqpoint{5.700000in}{5.700000in}}%
\pgfusepath{clip}%
\pgfsetbuttcap%
\pgfsetroundjoin%
\definecolor{currentfill}{rgb}{0.235526,0.309527,0.542944}%
\pgfsetfillcolor{currentfill}%
\pgfsetfillopacity{0.700000}%
\pgfsetlinewidth{0.000000pt}%
\definecolor{currentstroke}{rgb}{0.000000,0.000000,0.000000}%
\pgfsetstrokecolor{currentstroke}%
\pgfsetdash{}{0pt}%
\pgfpathmoveto{\pgfqpoint{3.243904in}{3.095836in}}%
\pgfpathlineto{\pgfqpoint{3.257182in}{3.084817in}}%
\pgfpathlineto{\pgfqpoint{3.270461in}{3.073919in}}%
\pgfpathlineto{\pgfqpoint{3.283740in}{3.063141in}}%
\pgfpathlineto{\pgfqpoint{3.297019in}{3.052480in}}%
\pgfpathlineto{\pgfqpoint{3.304885in}{3.068222in}}%
\pgfpathlineto{\pgfqpoint{3.312745in}{3.084187in}}%
\pgfpathlineto{\pgfqpoint{3.320599in}{3.100380in}}%
\pgfpathlineto{\pgfqpoint{3.328445in}{3.116807in}}%
\pgfpathlineto{\pgfqpoint{3.315170in}{3.127744in}}%
\pgfpathlineto{\pgfqpoint{3.301894in}{3.138801in}}%
\pgfpathlineto{\pgfqpoint{3.288619in}{3.149976in}}%
\pgfpathlineto{\pgfqpoint{3.275344in}{3.161273in}}%
\pgfpathlineto{\pgfqpoint{3.267494in}{3.144561in}}%
\pgfpathlineto{\pgfqpoint{3.259638in}{3.128088in}}%
\pgfpathlineto{\pgfqpoint{3.251774in}{3.111848in}}%
\pgfpathlineto{\pgfqpoint{3.243904in}{3.095836in}}%
\pgfpathclose%
\pgfusepath{fill}%
\end{pgfscope}%
\begin{pgfscope}%
\pgfpathrectangle{\pgfqpoint{1.150000in}{0.150000in}}{\pgfqpoint{5.700000in}{5.700000in}}%
\pgfusepath{clip}%
\pgfsetbuttcap%
\pgfsetroundjoin%
\definecolor{currentfill}{rgb}{0.243113,0.292092,0.538516}%
\pgfsetfillcolor{currentfill}%
\pgfsetfillopacity{0.700000}%
\pgfsetlinewidth{0.000000pt}%
\definecolor{currentstroke}{rgb}{0.000000,0.000000,0.000000}%
\pgfsetstrokecolor{currentstroke}%
\pgfsetdash{}{0pt}%
\pgfpathmoveto{\pgfqpoint{3.519114in}{3.059059in}}%
\pgfpathlineto{\pgfqpoint{3.532401in}{3.049285in}}%
\pgfpathlineto{\pgfqpoint{3.545690in}{3.039619in}}%
\pgfpathlineto{\pgfqpoint{3.558980in}{3.030058in}}%
\pgfpathlineto{\pgfqpoint{3.572273in}{3.020603in}}%
\pgfpathlineto{\pgfqpoint{3.580077in}{3.037039in}}%
\pgfpathlineto{\pgfqpoint{3.587875in}{3.053720in}}%
\pgfpathlineto{\pgfqpoint{3.595668in}{3.070654in}}%
\pgfpathlineto{\pgfqpoint{3.603456in}{3.087845in}}%
\pgfpathlineto{\pgfqpoint{3.590166in}{3.097617in}}%
\pgfpathlineto{\pgfqpoint{3.576879in}{3.107495in}}%
\pgfpathlineto{\pgfqpoint{3.563593in}{3.117478in}}%
\pgfpathlineto{\pgfqpoint{3.550309in}{3.127569in}}%
\pgfpathlineto{\pgfqpoint{3.542519in}{3.110052in}}%
\pgfpathlineto{\pgfqpoint{3.534722in}{3.092799in}}%
\pgfpathlineto{\pgfqpoint{3.526921in}{3.075803in}}%
\pgfpathlineto{\pgfqpoint{3.519114in}{3.059059in}}%
\pgfpathclose%
\pgfusepath{fill}%
\end{pgfscope}%
\begin{pgfscope}%
\pgfpathrectangle{\pgfqpoint{1.150000in}{0.150000in}}{\pgfqpoint{5.700000in}{5.700000in}}%
\pgfusepath{clip}%
\pgfsetbuttcap%
\pgfsetroundjoin%
\definecolor{currentfill}{rgb}{0.168126,0.459988,0.558082}%
\pgfsetfillcolor{currentfill}%
\pgfsetfillopacity{0.700000}%
\pgfsetlinewidth{0.000000pt}%
\definecolor{currentstroke}{rgb}{0.000000,0.000000,0.000000}%
\pgfsetstrokecolor{currentstroke}%
\pgfsetdash{}{0pt}%
\pgfpathmoveto{\pgfqpoint{4.276827in}{3.449116in}}%
\pgfpathlineto{\pgfqpoint{4.290189in}{3.439208in}}%
\pgfpathlineto{\pgfqpoint{4.303555in}{3.429388in}}%
\pgfpathlineto{\pgfqpoint{4.316924in}{3.419653in}}%
\pgfpathlineto{\pgfqpoint{4.330296in}{3.410004in}}%
\pgfpathlineto{\pgfqpoint{4.337994in}{3.435224in}}%
\pgfpathlineto{\pgfqpoint{4.345694in}{3.460915in}}%
\pgfpathlineto{\pgfqpoint{4.353397in}{3.487088in}}%
\pgfpathlineto{\pgfqpoint{4.361102in}{3.513752in}}%
\pgfpathlineto{\pgfqpoint{4.347730in}{3.523904in}}%
\pgfpathlineto{\pgfqpoint{4.334361in}{3.534142in}}%
\pgfpathlineto{\pgfqpoint{4.320995in}{3.544467in}}%
\pgfpathlineto{\pgfqpoint{4.307632in}{3.554879in}}%
\pgfpathlineto{\pgfqpoint{4.299927in}{3.527703in}}%
\pgfpathlineto{\pgfqpoint{4.292225in}{3.501023in}}%
\pgfpathlineto{\pgfqpoint{4.284525in}{3.474831in}}%
\pgfpathlineto{\pgfqpoint{4.276827in}{3.449116in}}%
\pgfpathclose%
\pgfusepath{fill}%
\end{pgfscope}%
\begin{pgfscope}%
\pgfpathrectangle{\pgfqpoint{1.150000in}{0.150000in}}{\pgfqpoint{5.700000in}{5.700000in}}%
\pgfusepath{clip}%
\pgfsetbuttcap%
\pgfsetroundjoin%
\definecolor{currentfill}{rgb}{0.352360,0.783011,0.392636}%
\pgfsetfillcolor{currentfill}%
\pgfsetfillopacity{0.700000}%
\pgfsetlinewidth{0.000000pt}%
\definecolor{currentstroke}{rgb}{0.000000,0.000000,0.000000}%
\pgfsetstrokecolor{currentstroke}%
\pgfsetdash{}{0pt}%
\pgfpathmoveto{\pgfqpoint{3.973178in}{4.359503in}}%
\pgfpathlineto{\pgfqpoint{3.986516in}{4.344311in}}%
\pgfpathlineto{\pgfqpoint{3.999854in}{4.329228in}}%
\pgfpathlineto{\pgfqpoint{4.013192in}{4.314252in}}%
\pgfpathlineto{\pgfqpoint{4.026531in}{4.299382in}}%
\pgfpathlineto{\pgfqpoint{4.034222in}{4.338175in}}%
\pgfpathlineto{\pgfqpoint{4.041917in}{4.377665in}}%
\pgfpathlineto{\pgfqpoint{4.049615in}{4.417865in}}%
\pgfpathlineto{\pgfqpoint{4.036267in}{4.433196in}}%
\pgfpathlineto{\pgfqpoint{4.022918in}{4.448635in}}%
\pgfpathlineto{\pgfqpoint{4.009570in}{4.464181in}}%
\pgfpathlineto{\pgfqpoint{3.996222in}{4.479837in}}%
\pgfpathlineto{\pgfqpoint{3.988538in}{4.439014in}}%
\pgfpathlineto{\pgfqpoint{3.980857in}{4.398907in}}%
\pgfpathlineto{\pgfqpoint{3.973178in}{4.359503in}}%
\pgfpathclose%
\pgfusepath{fill}%
\end{pgfscope}%
\begin{pgfscope}%
\pgfpathrectangle{\pgfqpoint{1.150000in}{0.150000in}}{\pgfqpoint{5.700000in}{5.700000in}}%
\pgfusepath{clip}%
\pgfsetbuttcap%
\pgfsetroundjoin%
\definecolor{currentfill}{rgb}{0.404001,0.800275,0.362552}%
\pgfsetfillcolor{currentfill}%
\pgfsetfillopacity{0.700000}%
\pgfsetlinewidth{0.000000pt}%
\definecolor{currentstroke}{rgb}{0.000000,0.000000,0.000000}%
\pgfsetstrokecolor{currentstroke}%
\pgfsetdash{}{0pt}%
\pgfpathmoveto{\pgfqpoint{3.919824in}{4.421370in}}%
\pgfpathlineto{\pgfqpoint{3.933163in}{4.405737in}}%
\pgfpathlineto{\pgfqpoint{3.946502in}{4.390215in}}%
\pgfpathlineto{\pgfqpoint{3.959840in}{4.374804in}}%
\pgfpathlineto{\pgfqpoint{3.973178in}{4.359503in}}%
\pgfpathlineto{\pgfqpoint{3.980857in}{4.398907in}}%
\pgfpathlineto{\pgfqpoint{3.988538in}{4.439014in}}%
\pgfpathlineto{\pgfqpoint{3.996222in}{4.479837in}}%
\pgfpathlineto{\pgfqpoint{3.982873in}{4.495602in}}%
\pgfpathlineto{\pgfqpoint{3.969524in}{4.511478in}}%
\pgfpathlineto{\pgfqpoint{3.956174in}{4.527465in}}%
\pgfpathlineto{\pgfqpoint{3.942824in}{4.543564in}}%
\pgfpathlineto{\pgfqpoint{3.935155in}{4.502113in}}%
\pgfpathlineto{\pgfqpoint{3.927489in}{4.461386in}}%
\pgfpathlineto{\pgfqpoint{3.919824in}{4.421370in}}%
\pgfpathclose%
\pgfusepath{fill}%
\end{pgfscope}%
\begin{pgfscope}%
\pgfpathrectangle{\pgfqpoint{1.150000in}{0.150000in}}{\pgfqpoint{5.700000in}{5.700000in}}%
\pgfusepath{clip}%
\pgfsetbuttcap%
\pgfsetroundjoin%
\definecolor{currentfill}{rgb}{0.304148,0.764704,0.419943}%
\pgfsetfillcolor{currentfill}%
\pgfsetfillopacity{0.700000}%
\pgfsetlinewidth{0.000000pt}%
\definecolor{currentstroke}{rgb}{0.000000,0.000000,0.000000}%
\pgfsetstrokecolor{currentstroke}%
\pgfsetdash{}{0pt}%
\pgfpathmoveto{\pgfqpoint{4.026531in}{4.299382in}}%
\pgfpathlineto{\pgfqpoint{4.039869in}{4.284618in}}%
\pgfpathlineto{\pgfqpoint{4.053207in}{4.269960in}}%
\pgfpathlineto{\pgfqpoint{4.066546in}{4.255406in}}%
\pgfpathlineto{\pgfqpoint{4.079886in}{4.240956in}}%
\pgfpathlineto{\pgfqpoint{4.087590in}{4.279142in}}%
\pgfpathlineto{\pgfqpoint{4.095297in}{4.318018in}}%
\pgfpathlineto{\pgfqpoint{4.103008in}{4.357596in}}%
\pgfpathlineto{\pgfqpoint{4.089659in}{4.372506in}}%
\pgfpathlineto{\pgfqpoint{4.076311in}{4.387521in}}%
\pgfpathlineto{\pgfqpoint{4.062963in}{4.402640in}}%
\pgfpathlineto{\pgfqpoint{4.049615in}{4.417865in}}%
\pgfpathlineto{\pgfqpoint{4.041917in}{4.377665in}}%
\pgfpathlineto{\pgfqpoint{4.034222in}{4.338175in}}%
\pgfpathlineto{\pgfqpoint{4.026531in}{4.299382in}}%
\pgfpathclose%
\pgfusepath{fill}%
\end{pgfscope}%
\begin{pgfscope}%
\pgfpathrectangle{\pgfqpoint{1.150000in}{0.150000in}}{\pgfqpoint{5.700000in}{5.700000in}}%
\pgfusepath{clip}%
\pgfsetbuttcap%
\pgfsetroundjoin%
\definecolor{currentfill}{rgb}{0.151918,0.500685,0.557587}%
\pgfsetfillcolor{currentfill}%
\pgfsetfillopacity{0.700000}%
\pgfsetlinewidth{0.000000pt}%
\definecolor{currentstroke}{rgb}{0.000000,0.000000,0.000000}%
\pgfsetstrokecolor{currentstroke}%
\pgfsetdash{}{0pt}%
\pgfpathmoveto{\pgfqpoint{4.307632in}{3.554879in}}%
\pgfpathlineto{\pgfqpoint{4.320995in}{3.544467in}}%
\pgfpathlineto{\pgfqpoint{4.334361in}{3.534142in}}%
\pgfpathlineto{\pgfqpoint{4.347730in}{3.523904in}}%
\pgfpathlineto{\pgfqpoint{4.361102in}{3.513752in}}%
\pgfpathlineto{\pgfqpoint{4.368809in}{3.540917in}}%
\pgfpathlineto{\pgfqpoint{4.376520in}{3.568593in}}%
\pgfpathlineto{\pgfqpoint{4.384234in}{3.596790in}}%
\pgfpathlineto{\pgfqpoint{4.391951in}{3.625520in}}%
\pgfpathlineto{\pgfqpoint{4.378578in}{3.636198in}}%
\pgfpathlineto{\pgfqpoint{4.365207in}{3.646964in}}%
\pgfpathlineto{\pgfqpoint{4.351840in}{3.657816in}}%
\pgfpathlineto{\pgfqpoint{4.338475in}{3.668755in}}%
\pgfpathlineto{\pgfqpoint{4.330759in}{3.639490in}}%
\pgfpathlineto{\pgfqpoint{4.323047in}{3.610763in}}%
\pgfpathlineto{\pgfqpoint{4.315338in}{3.582562in}}%
\pgfpathlineto{\pgfqpoint{4.307632in}{3.554879in}}%
\pgfpathclose%
\pgfusepath{fill}%
\end{pgfscope}%
\begin{pgfscope}%
\pgfpathrectangle{\pgfqpoint{1.150000in}{0.150000in}}{\pgfqpoint{5.700000in}{5.700000in}}%
\pgfusepath{clip}%
\pgfsetbuttcap%
\pgfsetroundjoin%
\definecolor{currentfill}{rgb}{0.468053,0.818921,0.323998}%
\pgfsetfillcolor{currentfill}%
\pgfsetfillopacity{0.700000}%
\pgfsetlinewidth{0.000000pt}%
\definecolor{currentstroke}{rgb}{0.000000,0.000000,0.000000}%
\pgfsetstrokecolor{currentstroke}%
\pgfsetdash{}{0pt}%
\pgfpathmoveto{\pgfqpoint{3.866462in}{4.485036in}}%
\pgfpathlineto{\pgfqpoint{3.879804in}{4.468948in}}%
\pgfpathlineto{\pgfqpoint{3.893144in}{4.452975in}}%
\pgfpathlineto{\pgfqpoint{3.906485in}{4.437116in}}%
\pgfpathlineto{\pgfqpoint{3.919824in}{4.421370in}}%
\pgfpathlineto{\pgfqpoint{3.927489in}{4.461386in}}%
\pgfpathlineto{\pgfqpoint{3.935155in}{4.502113in}}%
\pgfpathlineto{\pgfqpoint{3.942824in}{4.543564in}}%
\pgfpathlineto{\pgfqpoint{3.929473in}{4.559776in}}%
\pgfpathlineto{\pgfqpoint{3.916121in}{4.576102in}}%
\pgfpathlineto{\pgfqpoint{3.902769in}{4.592543in}}%
\pgfpathlineto{\pgfqpoint{3.889416in}{4.609100in}}%
\pgfpathlineto{\pgfqpoint{3.881763in}{4.567019in}}%
\pgfpathlineto{\pgfqpoint{3.874112in}{4.525669in}}%
\pgfpathlineto{\pgfqpoint{3.866462in}{4.485036in}}%
\pgfpathclose%
\pgfusepath{fill}%
\end{pgfscope}%
\begin{pgfscope}%
\pgfpathrectangle{\pgfqpoint{1.150000in}{0.150000in}}{\pgfqpoint{5.700000in}{5.700000in}}%
\pgfusepath{clip}%
\pgfsetbuttcap%
\pgfsetroundjoin%
\definecolor{currentfill}{rgb}{0.120638,0.625828,0.533488}%
\pgfsetfillcolor{currentfill}%
\pgfsetfillopacity{0.700000}%
\pgfsetlinewidth{0.000000pt}%
\definecolor{currentstroke}{rgb}{0.000000,0.000000,0.000000}%
\pgfsetstrokecolor{currentstroke}%
\pgfsetdash{}{0pt}%
\pgfpathmoveto{\pgfqpoint{4.262508in}{3.886547in}}%
\pgfpathlineto{\pgfqpoint{4.275859in}{3.874337in}}%
\pgfpathlineto{\pgfqpoint{4.289211in}{3.862218in}}%
\pgfpathlineto{\pgfqpoint{4.302566in}{3.850191in}}%
\pgfpathlineto{\pgfqpoint{4.315923in}{3.838254in}}%
\pgfpathlineto{\pgfqpoint{4.323654in}{3.870936in}}%
\pgfpathlineto{\pgfqpoint{4.331389in}{3.904227in}}%
\pgfpathlineto{\pgfqpoint{4.339129in}{3.938138in}}%
\pgfpathlineto{\pgfqpoint{4.346874in}{3.972682in}}%
\pgfpathlineto{\pgfqpoint{4.333511in}{3.985197in}}%
\pgfpathlineto{\pgfqpoint{4.320150in}{3.997803in}}%
\pgfpathlineto{\pgfqpoint{4.306791in}{4.010501in}}%
\pgfpathlineto{\pgfqpoint{4.293433in}{4.023291in}}%
\pgfpathlineto{\pgfqpoint{4.285695in}{3.988158in}}%
\pgfpathlineto{\pgfqpoint{4.277961in}{3.953665in}}%
\pgfpathlineto{\pgfqpoint{4.270232in}{3.919798in}}%
\pgfpathlineto{\pgfqpoint{4.262508in}{3.886547in}}%
\pgfpathclose%
\pgfusepath{fill}%
\end{pgfscope}%
\begin{pgfscope}%
\pgfpathrectangle{\pgfqpoint{1.150000in}{0.150000in}}{\pgfqpoint{5.700000in}{5.700000in}}%
\pgfusepath{clip}%
\pgfsetbuttcap%
\pgfsetroundjoin%
\definecolor{currentfill}{rgb}{0.259857,0.745492,0.444467}%
\pgfsetfillcolor{currentfill}%
\pgfsetfillopacity{0.700000}%
\pgfsetlinewidth{0.000000pt}%
\definecolor{currentstroke}{rgb}{0.000000,0.000000,0.000000}%
\pgfsetstrokecolor{currentstroke}%
\pgfsetdash{}{0pt}%
\pgfpathmoveto{\pgfqpoint{4.079886in}{4.240956in}}%
\pgfpathlineto{\pgfqpoint{4.093226in}{4.226608in}}%
\pgfpathlineto{\pgfqpoint{4.106566in}{4.212363in}}%
\pgfpathlineto{\pgfqpoint{4.119907in}{4.198219in}}%
\pgfpathlineto{\pgfqpoint{4.133249in}{4.184176in}}%
\pgfpathlineto{\pgfqpoint{4.140964in}{4.221758in}}%
\pgfpathlineto{\pgfqpoint{4.148683in}{4.260023in}}%
\pgfpathlineto{\pgfqpoint{4.156406in}{4.298983in}}%
\pgfpathlineto{\pgfqpoint{4.143056in}{4.313484in}}%
\pgfpathlineto{\pgfqpoint{4.129706in}{4.328086in}}%
\pgfpathlineto{\pgfqpoint{4.116357in}{4.342789in}}%
\pgfpathlineto{\pgfqpoint{4.103008in}{4.357596in}}%
\pgfpathlineto{\pgfqpoint{4.095297in}{4.318018in}}%
\pgfpathlineto{\pgfqpoint{4.087590in}{4.279142in}}%
\pgfpathlineto{\pgfqpoint{4.079886in}{4.240956in}}%
\pgfpathclose%
\pgfusepath{fill}%
\end{pgfscope}%
\begin{pgfscope}%
\pgfpathrectangle{\pgfqpoint{1.150000in}{0.150000in}}{\pgfqpoint{5.700000in}{5.700000in}}%
\pgfusepath{clip}%
\pgfsetbuttcap%
\pgfsetroundjoin%
\definecolor{currentfill}{rgb}{0.183898,0.422383,0.556944}%
\pgfsetfillcolor{currentfill}%
\pgfsetfillopacity{0.700000}%
\pgfsetlinewidth{0.000000pt}%
\definecolor{currentstroke}{rgb}{0.000000,0.000000,0.000000}%
\pgfsetstrokecolor{currentstroke}%
\pgfsetdash{}{0pt}%
\pgfpathmoveto{\pgfqpoint{4.246044in}{3.350842in}}%
\pgfpathlineto{\pgfqpoint{4.259407in}{3.341416in}}%
\pgfpathlineto{\pgfqpoint{4.272774in}{3.332077in}}%
\pgfpathlineto{\pgfqpoint{4.286144in}{3.322824in}}%
\pgfpathlineto{\pgfqpoint{4.299517in}{3.313656in}}%
\pgfpathlineto{\pgfqpoint{4.307210in}{3.337082in}}%
\pgfpathlineto{\pgfqpoint{4.314904in}{3.360943in}}%
\pgfpathlineto{\pgfqpoint{4.322599in}{3.385247in}}%
\pgfpathlineto{\pgfqpoint{4.330296in}{3.410004in}}%
\pgfpathlineto{\pgfqpoint{4.316924in}{3.419653in}}%
\pgfpathlineto{\pgfqpoint{4.303555in}{3.429388in}}%
\pgfpathlineto{\pgfqpoint{4.290189in}{3.439208in}}%
\pgfpathlineto{\pgfqpoint{4.276827in}{3.449116in}}%
\pgfpathlineto{\pgfqpoint{4.269130in}{3.423869in}}%
\pgfpathlineto{\pgfqpoint{4.261434in}{3.399080in}}%
\pgfpathlineto{\pgfqpoint{4.253739in}{3.374741in}}%
\pgfpathlineto{\pgfqpoint{4.246044in}{3.350842in}}%
\pgfpathclose%
\pgfusepath{fill}%
\end{pgfscope}%
\begin{pgfscope}%
\pgfpathrectangle{\pgfqpoint{1.150000in}{0.150000in}}{\pgfqpoint{5.700000in}{5.700000in}}%
\pgfusepath{clip}%
\pgfsetbuttcap%
\pgfsetroundjoin%
\definecolor{currentfill}{rgb}{0.535621,0.835785,0.281908}%
\pgfsetfillcolor{currentfill}%
\pgfsetfillopacity{0.700000}%
\pgfsetlinewidth{0.000000pt}%
\definecolor{currentstroke}{rgb}{0.000000,0.000000,0.000000}%
\pgfsetstrokecolor{currentstroke}%
\pgfsetdash{}{0pt}%
\pgfpathmoveto{\pgfqpoint{3.813088in}{4.550559in}}%
\pgfpathlineto{\pgfqpoint{3.826433in}{4.534001in}}%
\pgfpathlineto{\pgfqpoint{3.839777in}{4.517562in}}%
\pgfpathlineto{\pgfqpoint{3.853120in}{4.501241in}}%
\pgfpathlineto{\pgfqpoint{3.866462in}{4.485036in}}%
\pgfpathlineto{\pgfqpoint{3.874112in}{4.525669in}}%
\pgfpathlineto{\pgfqpoint{3.881763in}{4.567019in}}%
\pgfpathlineto{\pgfqpoint{3.889416in}{4.609100in}}%
\pgfpathlineto{\pgfqpoint{3.876061in}{4.625774in}}%
\pgfpathlineto{\pgfqpoint{3.862706in}{4.642565in}}%
\pgfpathlineto{\pgfqpoint{3.849349in}{4.659474in}}%
\pgfpathlineto{\pgfqpoint{3.835992in}{4.676504in}}%
\pgfpathlineto{\pgfqpoint{3.828356in}{4.633788in}}%
\pgfpathlineto{\pgfqpoint{3.820721in}{4.591811in}}%
\pgfpathlineto{\pgfqpoint{3.813088in}{4.550559in}}%
\pgfpathclose%
\pgfusepath{fill}%
\end{pgfscope}%
\begin{pgfscope}%
\pgfpathrectangle{\pgfqpoint{1.150000in}{0.150000in}}{\pgfqpoint{5.700000in}{5.700000in}}%
\pgfusepath{clip}%
\pgfsetbuttcap%
\pgfsetroundjoin%
\definecolor{currentfill}{rgb}{0.243113,0.292092,0.538516}%
\pgfsetfillcolor{currentfill}%
\pgfsetfillopacity{0.700000}%
\pgfsetlinewidth{0.000000pt}%
\definecolor{currentstroke}{rgb}{0.000000,0.000000,0.000000}%
\pgfsetstrokecolor{currentstroke}%
\pgfsetdash{}{0pt}%
\pgfpathmoveto{\pgfqpoint{3.656632in}{3.049795in}}%
\pgfpathlineto{\pgfqpoint{3.669932in}{3.040538in}}%
\pgfpathlineto{\pgfqpoint{3.683234in}{3.031383in}}%
\pgfpathlineto{\pgfqpoint{3.696538in}{3.022328in}}%
\pgfpathlineto{\pgfqpoint{3.709845in}{3.013372in}}%
\pgfpathlineto{\pgfqpoint{3.717621in}{3.030169in}}%
\pgfpathlineto{\pgfqpoint{3.725392in}{3.047226in}}%
\pgfpathlineto{\pgfqpoint{3.733159in}{3.064548in}}%
\pgfpathlineto{\pgfqpoint{3.740921in}{3.082143in}}%
\pgfpathlineto{\pgfqpoint{3.727618in}{3.091435in}}%
\pgfpathlineto{\pgfqpoint{3.714317in}{3.100827in}}%
\pgfpathlineto{\pgfqpoint{3.701018in}{3.110319in}}%
\pgfpathlineto{\pgfqpoint{3.687722in}{3.119913in}}%
\pgfpathlineto{\pgfqpoint{3.679957in}{3.101973in}}%
\pgfpathlineto{\pgfqpoint{3.672187in}{3.084311in}}%
\pgfpathlineto{\pgfqpoint{3.664412in}{3.066921in}}%
\pgfpathlineto{\pgfqpoint{3.656632in}{3.049795in}}%
\pgfpathclose%
\pgfusepath{fill}%
\end{pgfscope}%
\begin{pgfscope}%
\pgfpathrectangle{\pgfqpoint{1.150000in}{0.150000in}}{\pgfqpoint{5.700000in}{5.700000in}}%
\pgfusepath{clip}%
\pgfsetbuttcap%
\pgfsetroundjoin%
\definecolor{currentfill}{rgb}{0.229739,0.322361,0.545706}%
\pgfsetfillcolor{currentfill}%
\pgfsetfillopacity{0.700000}%
\pgfsetlinewidth{0.000000pt}%
\definecolor{currentstroke}{rgb}{0.000000,0.000000,0.000000}%
\pgfsetstrokecolor{currentstroke}%
\pgfsetdash{}{0pt}%
\pgfpathmoveto{\pgfqpoint{3.106098in}{3.124520in}}%
\pgfpathlineto{\pgfqpoint{3.119381in}{3.112758in}}%
\pgfpathlineto{\pgfqpoint{3.132663in}{3.101126in}}%
\pgfpathlineto{\pgfqpoint{3.145946in}{3.089620in}}%
\pgfpathlineto{\pgfqpoint{3.159228in}{3.078242in}}%
\pgfpathlineto{\pgfqpoint{3.167130in}{3.093640in}}%
\pgfpathlineto{\pgfqpoint{3.175024in}{3.109252in}}%
\pgfpathlineto{\pgfqpoint{3.182912in}{3.125082in}}%
\pgfpathlineto{\pgfqpoint{3.190793in}{3.141136in}}%
\pgfpathlineto{\pgfqpoint{3.177515in}{3.152772in}}%
\pgfpathlineto{\pgfqpoint{3.164236in}{3.164535in}}%
\pgfpathlineto{\pgfqpoint{3.150957in}{3.176426in}}%
\pgfpathlineto{\pgfqpoint{3.137678in}{3.188446in}}%
\pgfpathlineto{\pgfqpoint{3.129794in}{3.172127in}}%
\pgfpathlineto{\pgfqpoint{3.121902in}{3.156036in}}%
\pgfpathlineto{\pgfqpoint{3.114004in}{3.140169in}}%
\pgfpathlineto{\pgfqpoint{3.106098in}{3.124520in}}%
\pgfpathclose%
\pgfusepath{fill}%
\end{pgfscope}%
\begin{pgfscope}%
\pgfpathrectangle{\pgfqpoint{1.150000in}{0.150000in}}{\pgfqpoint{5.700000in}{5.700000in}}%
\pgfusepath{clip}%
\pgfsetbuttcap%
\pgfsetroundjoin%
\definecolor{currentfill}{rgb}{0.220057,0.343307,0.549413}%
\pgfsetfillcolor{currentfill}%
\pgfsetfillopacity{0.700000}%
\pgfsetlinewidth{0.000000pt}%
\definecolor{currentstroke}{rgb}{0.000000,0.000000,0.000000}%
\pgfsetstrokecolor{currentstroke}%
\pgfsetdash{}{0pt}%
\pgfpathmoveto{\pgfqpoint{4.046855in}{3.163244in}}%
\pgfpathlineto{\pgfqpoint{4.060198in}{3.154428in}}%
\pgfpathlineto{\pgfqpoint{4.073544in}{3.145703in}}%
\pgfpathlineto{\pgfqpoint{4.086894in}{3.137066in}}%
\pgfpathlineto{\pgfqpoint{4.100247in}{3.128518in}}%
\pgfpathlineto{\pgfqpoint{4.107952in}{3.148217in}}%
\pgfpathlineto{\pgfqpoint{4.115655in}{3.168260in}}%
\pgfpathlineto{\pgfqpoint{4.123356in}{3.188656in}}%
\pgfpathlineto{\pgfqpoint{4.131056in}{3.209412in}}%
\pgfpathlineto{\pgfqpoint{4.117706in}{3.218378in}}%
\pgfpathlineto{\pgfqpoint{4.104358in}{3.227432in}}%
\pgfpathlineto{\pgfqpoint{4.091015in}{3.236576in}}%
\pgfpathlineto{\pgfqpoint{4.077674in}{3.245810in}}%
\pgfpathlineto{\pgfqpoint{4.069972in}{3.224628in}}%
\pgfpathlineto{\pgfqpoint{4.062268in}{3.203812in}}%
\pgfpathlineto{\pgfqpoint{4.054562in}{3.183353in}}%
\pgfpathlineto{\pgfqpoint{4.046855in}{3.163244in}}%
\pgfpathclose%
\pgfusepath{fill}%
\end{pgfscope}%
\begin{pgfscope}%
\pgfpathrectangle{\pgfqpoint{1.150000in}{0.150000in}}{\pgfqpoint{5.700000in}{5.700000in}}%
\pgfusepath{clip}%
\pgfsetbuttcap%
\pgfsetroundjoin%
\definecolor{currentfill}{rgb}{0.227802,0.326594,0.546532}%
\pgfsetfillcolor{currentfill}%
\pgfsetfillopacity{0.700000}%
\pgfsetlinewidth{0.000000pt}%
\definecolor{currentstroke}{rgb}{0.000000,0.000000,0.000000}%
\pgfsetstrokecolor{currentstroke}%
\pgfsetdash{}{0pt}%
\pgfpathmoveto{\pgfqpoint{3.962649in}{3.120739in}}%
\pgfpathlineto{\pgfqpoint{3.975983in}{3.111957in}}%
\pgfpathlineto{\pgfqpoint{3.989319in}{3.103266in}}%
\pgfpathlineto{\pgfqpoint{4.002659in}{3.094666in}}%
\pgfpathlineto{\pgfqpoint{4.016002in}{3.086157in}}%
\pgfpathlineto{\pgfqpoint{4.023719in}{3.104942in}}%
\pgfpathlineto{\pgfqpoint{4.031433in}{3.124046in}}%
\pgfpathlineto{\pgfqpoint{4.039145in}{3.143478in}}%
\pgfpathlineto{\pgfqpoint{4.046855in}{3.163244in}}%
\pgfpathlineto{\pgfqpoint{4.033515in}{3.172150in}}%
\pgfpathlineto{\pgfqpoint{4.020178in}{3.181147in}}%
\pgfpathlineto{\pgfqpoint{4.006845in}{3.190236in}}%
\pgfpathlineto{\pgfqpoint{3.993514in}{3.199416in}}%
\pgfpathlineto{\pgfqpoint{3.985802in}{3.179244in}}%
\pgfpathlineto{\pgfqpoint{3.978087in}{3.159413in}}%
\pgfpathlineto{\pgfqpoint{3.970370in}{3.139913in}}%
\pgfpathlineto{\pgfqpoint{3.962649in}{3.120739in}}%
\pgfpathclose%
\pgfusepath{fill}%
\end{pgfscope}%
\begin{pgfscope}%
\pgfpathrectangle{\pgfqpoint{1.150000in}{0.150000in}}{\pgfqpoint{5.700000in}{5.700000in}}%
\pgfusepath{clip}%
\pgfsetbuttcap%
\pgfsetroundjoin%
\definecolor{currentfill}{rgb}{0.226397,0.728888,0.462789}%
\pgfsetfillcolor{currentfill}%
\pgfsetfillopacity{0.700000}%
\pgfsetlinewidth{0.000000pt}%
\definecolor{currentstroke}{rgb}{0.000000,0.000000,0.000000}%
\pgfsetstrokecolor{currentstroke}%
\pgfsetdash{}{0pt}%
\pgfpathmoveto{\pgfqpoint{4.133249in}{4.184176in}}%
\pgfpathlineto{\pgfqpoint{4.146591in}{4.170234in}}%
\pgfpathlineto{\pgfqpoint{4.159935in}{4.156391in}}%
\pgfpathlineto{\pgfqpoint{4.173279in}{4.142646in}}%
\pgfpathlineto{\pgfqpoint{4.186625in}{4.129000in}}%
\pgfpathlineto{\pgfqpoint{4.194350in}{4.165980in}}%
\pgfpathlineto{\pgfqpoint{4.202080in}{4.203636in}}%
\pgfpathlineto{\pgfqpoint{4.209814in}{4.241981in}}%
\pgfpathlineto{\pgfqpoint{4.196461in}{4.256083in}}%
\pgfpathlineto{\pgfqpoint{4.183109in}{4.270283in}}%
\pgfpathlineto{\pgfqpoint{4.169757in}{4.284583in}}%
\pgfpathlineto{\pgfqpoint{4.156406in}{4.298983in}}%
\pgfpathlineto{\pgfqpoint{4.148683in}{4.260023in}}%
\pgfpathlineto{\pgfqpoint{4.140964in}{4.221758in}}%
\pgfpathlineto{\pgfqpoint{4.133249in}{4.184176in}}%
\pgfpathclose%
\pgfusepath{fill}%
\end{pgfscope}%
\begin{pgfscope}%
\pgfpathrectangle{\pgfqpoint{1.150000in}{0.150000in}}{\pgfqpoint{5.700000in}{5.700000in}}%
\pgfusepath{clip}%
\pgfsetbuttcap%
\pgfsetroundjoin%
\definecolor{currentfill}{rgb}{0.824940,0.884720,0.106217}%
\pgfsetfillcolor{currentfill}%
\pgfsetfillopacity{0.700000}%
\pgfsetlinewidth{0.000000pt}%
\definecolor{currentstroke}{rgb}{0.000000,0.000000,0.000000}%
\pgfsetstrokecolor{currentstroke}%
\pgfsetdash{}{0pt}%
\pgfpathmoveto{\pgfqpoint{3.462121in}{4.813684in}}%
\pgfpathlineto{\pgfqpoint{3.475501in}{4.794563in}}%
\pgfpathlineto{\pgfqpoint{3.488879in}{4.775589in}}%
\pgfpathlineto{\pgfqpoint{3.502253in}{4.756760in}}%
\pgfpathlineto{\pgfqpoint{3.515623in}{4.738075in}}%
\pgfpathlineto{\pgfqpoint{3.523174in}{4.779558in}}%
\pgfpathlineto{\pgfqpoint{3.530723in}{4.821752in}}%
\pgfpathlineto{\pgfqpoint{3.538268in}{4.864671in}}%
\pgfpathlineto{\pgfqpoint{3.545811in}{4.908327in}}%
\pgfpathlineto{\pgfqpoint{3.532421in}{4.927636in}}%
\pgfpathlineto{\pgfqpoint{3.519027in}{4.947089in}}%
\pgfpathlineto{\pgfqpoint{3.505630in}{4.966687in}}%
\pgfpathlineto{\pgfqpoint{3.492230in}{4.986434in}}%
\pgfpathlineto{\pgfqpoint{3.484707in}{4.942142in}}%
\pgfpathlineto{\pgfqpoint{3.477182in}{4.898595in}}%
\pgfpathlineto{\pgfqpoint{3.469653in}{4.855780in}}%
\pgfpathlineto{\pgfqpoint{3.462121in}{4.813684in}}%
\pgfpathclose%
\pgfusepath{fill}%
\end{pgfscope}%
\begin{pgfscope}%
\pgfpathrectangle{\pgfqpoint{1.150000in}{0.150000in}}{\pgfqpoint{5.700000in}{5.700000in}}%
\pgfusepath{clip}%
\pgfsetbuttcap%
\pgfsetroundjoin%
\definecolor{currentfill}{rgb}{0.210503,0.363727,0.552206}%
\pgfsetfillcolor{currentfill}%
\pgfsetfillopacity{0.700000}%
\pgfsetlinewidth{0.000000pt}%
\definecolor{currentstroke}{rgb}{0.000000,0.000000,0.000000}%
\pgfsetstrokecolor{currentstroke}%
\pgfsetdash{}{0pt}%
\pgfpathmoveto{\pgfqpoint{4.131056in}{3.209412in}}%
\pgfpathlineto{\pgfqpoint{4.144409in}{3.200535in}}%
\pgfpathlineto{\pgfqpoint{4.157766in}{3.191746in}}%
\pgfpathlineto{\pgfqpoint{4.171127in}{3.183045in}}%
\pgfpathlineto{\pgfqpoint{4.184491in}{3.174431in}}%
\pgfpathlineto{\pgfqpoint{4.192187in}{3.195125in}}%
\pgfpathlineto{\pgfqpoint{4.199882in}{3.216190in}}%
\pgfpathlineto{\pgfqpoint{4.207576in}{3.237634in}}%
\pgfpathlineto{\pgfqpoint{4.215270in}{3.259466in}}%
\pgfpathlineto{\pgfqpoint{4.201909in}{3.268519in}}%
\pgfpathlineto{\pgfqpoint{4.188550in}{3.277659in}}%
\pgfpathlineto{\pgfqpoint{4.175196in}{3.286887in}}%
\pgfpathlineto{\pgfqpoint{4.161844in}{3.296203in}}%
\pgfpathlineto{\pgfqpoint{4.154148in}{3.273924in}}%
\pgfpathlineto{\pgfqpoint{4.146452in}{3.252038in}}%
\pgfpathlineto{\pgfqpoint{4.138754in}{3.230537in}}%
\pgfpathlineto{\pgfqpoint{4.131056in}{3.209412in}}%
\pgfpathclose%
\pgfusepath{fill}%
\end{pgfscope}%
\begin{pgfscope}%
\pgfpathrectangle{\pgfqpoint{1.150000in}{0.150000in}}{\pgfqpoint{5.700000in}{5.700000in}}%
\pgfusepath{clip}%
\pgfsetbuttcap%
\pgfsetroundjoin%
\definecolor{currentfill}{rgb}{0.606045,0.850733,0.236712}%
\pgfsetfillcolor{currentfill}%
\pgfsetfillopacity{0.700000}%
\pgfsetlinewidth{0.000000pt}%
\definecolor{currentstroke}{rgb}{0.000000,0.000000,0.000000}%
\pgfsetstrokecolor{currentstroke}%
\pgfsetdash{}{0pt}%
\pgfpathmoveto{\pgfqpoint{3.759694in}{4.618001in}}%
\pgfpathlineto{\pgfqpoint{3.773045in}{4.600957in}}%
\pgfpathlineto{\pgfqpoint{3.786394in}{4.584036in}}%
\pgfpathlineto{\pgfqpoint{3.799741in}{4.567238in}}%
\pgfpathlineto{\pgfqpoint{3.813088in}{4.550559in}}%
\pgfpathlineto{\pgfqpoint{3.820721in}{4.591811in}}%
\pgfpathlineto{\pgfqpoint{3.828356in}{4.633788in}}%
\pgfpathlineto{\pgfqpoint{3.835992in}{4.676504in}}%
\pgfpathlineto{\pgfqpoint{3.822633in}{4.693653in}}%
\pgfpathlineto{\pgfqpoint{3.809272in}{4.710925in}}%
\pgfpathlineto{\pgfqpoint{3.795910in}{4.728319in}}%
\pgfpathlineto{\pgfqpoint{3.782546in}{4.745836in}}%
\pgfpathlineto{\pgfqpoint{3.774929in}{4.702483in}}%
\pgfpathlineto{\pgfqpoint{3.767311in}{4.659876in}}%
\pgfpathlineto{\pgfqpoint{3.759694in}{4.618001in}}%
\pgfpathclose%
\pgfusepath{fill}%
\end{pgfscope}%
\begin{pgfscope}%
\pgfpathrectangle{\pgfqpoint{1.150000in}{0.150000in}}{\pgfqpoint{5.700000in}{5.700000in}}%
\pgfusepath{clip}%
\pgfsetbuttcap%
\pgfsetroundjoin%
\definecolor{currentfill}{rgb}{0.135066,0.544853,0.554029}%
\pgfsetfillcolor{currentfill}%
\pgfsetfillopacity{0.700000}%
\pgfsetlinewidth{0.000000pt}%
\definecolor{currentstroke}{rgb}{0.000000,0.000000,0.000000}%
\pgfsetstrokecolor{currentstroke}%
\pgfsetdash{}{0pt}%
\pgfpathmoveto{\pgfqpoint{4.338475in}{3.668755in}}%
\pgfpathlineto{\pgfqpoint{4.351840in}{3.657816in}}%
\pgfpathlineto{\pgfqpoint{4.365207in}{3.646964in}}%
\pgfpathlineto{\pgfqpoint{4.378578in}{3.636198in}}%
\pgfpathlineto{\pgfqpoint{4.391951in}{3.625520in}}%
\pgfpathlineto{\pgfqpoint{4.399672in}{3.654791in}}%
\pgfpathlineto{\pgfqpoint{4.407397in}{3.684615in}}%
\pgfpathlineto{\pgfqpoint{4.415127in}{3.715003in}}%
\pgfpathlineto{\pgfqpoint{4.422862in}{3.745965in}}%
\pgfpathlineto{\pgfqpoint{4.409486in}{3.757194in}}%
\pgfpathlineto{\pgfqpoint{4.396112in}{3.768510in}}%
\pgfpathlineto{\pgfqpoint{4.382742in}{3.779913in}}%
\pgfpathlineto{\pgfqpoint{4.369373in}{3.791404in}}%
\pgfpathlineto{\pgfqpoint{4.361642in}{3.759882in}}%
\pgfpathlineto{\pgfqpoint{4.353916in}{3.728941in}}%
\pgfpathlineto{\pgfqpoint{4.346193in}{3.698569in}}%
\pgfpathlineto{\pgfqpoint{4.338475in}{3.668755in}}%
\pgfpathclose%
\pgfusepath{fill}%
\end{pgfscope}%
\begin{pgfscope}%
\pgfpathrectangle{\pgfqpoint{1.150000in}{0.150000in}}{\pgfqpoint{5.700000in}{5.700000in}}%
\pgfusepath{clip}%
\pgfsetbuttcap%
\pgfsetroundjoin%
\definecolor{currentfill}{rgb}{0.235526,0.309527,0.542944}%
\pgfsetfillcolor{currentfill}%
\pgfsetfillopacity{0.700000}%
\pgfsetlinewidth{0.000000pt}%
\definecolor{currentstroke}{rgb}{0.000000,0.000000,0.000000}%
\pgfsetstrokecolor{currentstroke}%
\pgfsetdash{}{0pt}%
\pgfpathmoveto{\pgfqpoint{3.878423in}{3.081701in}}%
\pgfpathlineto{\pgfqpoint{3.891747in}{3.072923in}}%
\pgfpathlineto{\pgfqpoint{3.905074in}{3.064239in}}%
\pgfpathlineto{\pgfqpoint{3.918405in}{3.055648in}}%
\pgfpathlineto{\pgfqpoint{3.931738in}{3.047150in}}%
\pgfpathlineto{\pgfqpoint{3.939471in}{3.065095in}}%
\pgfpathlineto{\pgfqpoint{3.947200in}{3.083337in}}%
\pgfpathlineto{\pgfqpoint{3.954926in}{3.101883in}}%
\pgfpathlineto{\pgfqpoint{3.962649in}{3.120739in}}%
\pgfpathlineto{\pgfqpoint{3.949319in}{3.129614in}}%
\pgfpathlineto{\pgfqpoint{3.935992in}{3.138582in}}%
\pgfpathlineto{\pgfqpoint{3.922668in}{3.147643in}}%
\pgfpathlineto{\pgfqpoint{3.909347in}{3.156799in}}%
\pgfpathlineto{\pgfqpoint{3.901621in}{3.137558in}}%
\pgfpathlineto{\pgfqpoint{3.893891in}{3.118632in}}%
\pgfpathlineto{\pgfqpoint{3.886159in}{3.100016in}}%
\pgfpathlineto{\pgfqpoint{3.878423in}{3.081701in}}%
\pgfpathclose%
\pgfusepath{fill}%
\end{pgfscope}%
\begin{pgfscope}%
\pgfpathrectangle{\pgfqpoint{1.150000in}{0.150000in}}{\pgfqpoint{5.700000in}{5.700000in}}%
\pgfusepath{clip}%
\pgfsetbuttcap%
\pgfsetroundjoin%
\definecolor{currentfill}{rgb}{0.191090,0.708366,0.482284}%
\pgfsetfillcolor{currentfill}%
\pgfsetfillopacity{0.700000}%
\pgfsetlinewidth{0.000000pt}%
\definecolor{currentstroke}{rgb}{0.000000,0.000000,0.000000}%
\pgfsetstrokecolor{currentstroke}%
\pgfsetdash{}{0pt}%
\pgfpathmoveto{\pgfqpoint{4.186625in}{4.129000in}}%
\pgfpathlineto{\pgfqpoint{4.199971in}{4.115451in}}%
\pgfpathlineto{\pgfqpoint{4.213319in}{4.102000in}}%
\pgfpathlineto{\pgfqpoint{4.226668in}{4.088644in}}%
\pgfpathlineto{\pgfqpoint{4.240018in}{4.075385in}}%
\pgfpathlineto{\pgfqpoint{4.247753in}{4.111765in}}%
\pgfpathlineto{\pgfqpoint{4.255492in}{4.148815in}}%
\pgfpathlineto{\pgfqpoint{4.263237in}{4.186547in}}%
\pgfpathlineto{\pgfqpoint{4.249880in}{4.200260in}}%
\pgfpathlineto{\pgfqpoint{4.236524in}{4.214070in}}%
\pgfpathlineto{\pgfqpoint{4.223169in}{4.227977in}}%
\pgfpathlineto{\pgfqpoint{4.209814in}{4.241981in}}%
\pgfpathlineto{\pgfqpoint{4.202080in}{4.203636in}}%
\pgfpathlineto{\pgfqpoint{4.194350in}{4.165980in}}%
\pgfpathlineto{\pgfqpoint{4.186625in}{4.129000in}}%
\pgfpathclose%
\pgfusepath{fill}%
\end{pgfscope}%
\begin{pgfscope}%
\pgfpathrectangle{\pgfqpoint{1.150000in}{0.150000in}}{\pgfqpoint{5.700000in}{5.700000in}}%
\pgfusepath{clip}%
\pgfsetbuttcap%
\pgfsetroundjoin%
\definecolor{currentfill}{rgb}{0.246811,0.283237,0.535941}%
\pgfsetfillcolor{currentfill}%
\pgfsetfillopacity{0.700000}%
\pgfsetlinewidth{0.000000pt}%
\definecolor{currentstroke}{rgb}{0.000000,0.000000,0.000000}%
\pgfsetstrokecolor{currentstroke}%
\pgfsetdash{}{0pt}%
\pgfpathmoveto{\pgfqpoint{3.434682in}{3.033466in}}%
\pgfpathlineto{\pgfqpoint{3.447966in}{3.023555in}}%
\pgfpathlineto{\pgfqpoint{3.461252in}{3.013754in}}%
\pgfpathlineto{\pgfqpoint{3.474540in}{3.004062in}}%
\pgfpathlineto{\pgfqpoint{3.487829in}{2.994478in}}%
\pgfpathlineto{\pgfqpoint{3.495659in}{3.010274in}}%
\pgfpathlineto{\pgfqpoint{3.503483in}{3.026300in}}%
\pgfpathlineto{\pgfqpoint{3.511301in}{3.042559in}}%
\pgfpathlineto{\pgfqpoint{3.519114in}{3.059059in}}%
\pgfpathlineto{\pgfqpoint{3.505828in}{3.068939in}}%
\pgfpathlineto{\pgfqpoint{3.492544in}{3.078928in}}%
\pgfpathlineto{\pgfqpoint{3.479262in}{3.089026in}}%
\pgfpathlineto{\pgfqpoint{3.465981in}{3.099234in}}%
\pgfpathlineto{\pgfqpoint{3.458165in}{3.082430in}}%
\pgfpathlineto{\pgfqpoint{3.450343in}{3.065871in}}%
\pgfpathlineto{\pgfqpoint{3.442515in}{3.049552in}}%
\pgfpathlineto{\pgfqpoint{3.434682in}{3.033466in}}%
\pgfpathclose%
\pgfusepath{fill}%
\end{pgfscope}%
\begin{pgfscope}%
\pgfpathrectangle{\pgfqpoint{1.150000in}{0.150000in}}{\pgfqpoint{5.700000in}{5.700000in}}%
\pgfusepath{clip}%
\pgfsetbuttcap%
\pgfsetroundjoin%
\definecolor{currentfill}{rgb}{0.243113,0.292092,0.538516}%
\pgfsetfillcolor{currentfill}%
\pgfsetfillopacity{0.700000}%
\pgfsetlinewidth{0.000000pt}%
\definecolor{currentstroke}{rgb}{0.000000,0.000000,0.000000}%
\pgfsetstrokecolor{currentstroke}%
\pgfsetdash{}{0pt}%
\pgfpathmoveto{\pgfqpoint{3.297019in}{3.052480in}}%
\pgfpathlineto{\pgfqpoint{3.310299in}{3.041938in}}%
\pgfpathlineto{\pgfqpoint{3.323580in}{3.031512in}}%
\pgfpathlineto{\pgfqpoint{3.336862in}{3.021201in}}%
\pgfpathlineto{\pgfqpoint{3.350145in}{3.011006in}}%
\pgfpathlineto{\pgfqpoint{3.358007in}{3.026478in}}%
\pgfpathlineto{\pgfqpoint{3.365863in}{3.042169in}}%
\pgfpathlineto{\pgfqpoint{3.373712in}{3.058082in}}%
\pgfpathlineto{\pgfqpoint{3.381556in}{3.074224in}}%
\pgfpathlineto{\pgfqpoint{3.368277in}{3.084696in}}%
\pgfpathlineto{\pgfqpoint{3.354999in}{3.095284in}}%
\pgfpathlineto{\pgfqpoint{3.341722in}{3.105987in}}%
\pgfpathlineto{\pgfqpoint{3.328445in}{3.116807in}}%
\pgfpathlineto{\pgfqpoint{3.320599in}{3.100380in}}%
\pgfpathlineto{\pgfqpoint{3.312745in}{3.084187in}}%
\pgfpathlineto{\pgfqpoint{3.304885in}{3.068222in}}%
\pgfpathlineto{\pgfqpoint{3.297019in}{3.052480in}}%
\pgfpathclose%
\pgfusepath{fill}%
\end{pgfscope}%
\begin{pgfscope}%
\pgfpathrectangle{\pgfqpoint{1.150000in}{0.150000in}}{\pgfqpoint{5.700000in}{5.700000in}}%
\pgfusepath{clip}%
\pgfsetbuttcap%
\pgfsetroundjoin%
\definecolor{currentfill}{rgb}{0.119512,0.607464,0.540218}%
\pgfsetfillcolor{currentfill}%
\pgfsetfillopacity{0.700000}%
\pgfsetlinewidth{0.000000pt}%
\definecolor{currentstroke}{rgb}{0.000000,0.000000,0.000000}%
\pgfsetstrokecolor{currentstroke}%
\pgfsetdash{}{0pt}%
\pgfpathmoveto{\pgfqpoint{4.315923in}{3.838254in}}%
\pgfpathlineto{\pgfqpoint{4.329282in}{3.826408in}}%
\pgfpathlineto{\pgfqpoint{4.342644in}{3.814651in}}%
\pgfpathlineto{\pgfqpoint{4.356007in}{3.802983in}}%
\pgfpathlineto{\pgfqpoint{4.369373in}{3.791404in}}%
\pgfpathlineto{\pgfqpoint{4.377109in}{3.823518in}}%
\pgfpathlineto{\pgfqpoint{4.384849in}{3.856235in}}%
\pgfpathlineto{\pgfqpoint{4.392595in}{3.889566in}}%
\pgfpathlineto{\pgfqpoint{4.400346in}{3.923524in}}%
\pgfpathlineto{\pgfqpoint{4.386975in}{3.935679in}}%
\pgfpathlineto{\pgfqpoint{4.373606in}{3.947924in}}%
\pgfpathlineto{\pgfqpoint{4.360239in}{3.960258in}}%
\pgfpathlineto{\pgfqpoint{4.346874in}{3.972682in}}%
\pgfpathlineto{\pgfqpoint{4.339129in}{3.938138in}}%
\pgfpathlineto{\pgfqpoint{4.331389in}{3.904227in}}%
\pgfpathlineto{\pgfqpoint{4.323654in}{3.870936in}}%
\pgfpathlineto{\pgfqpoint{4.315923in}{3.838254in}}%
\pgfpathclose%
\pgfusepath{fill}%
\end{pgfscope}%
\begin{pgfscope}%
\pgfpathrectangle{\pgfqpoint{1.150000in}{0.150000in}}{\pgfqpoint{5.700000in}{5.700000in}}%
\pgfusepath{clip}%
\pgfsetbuttcap%
\pgfsetroundjoin%
\definecolor{currentfill}{rgb}{0.199430,0.387607,0.554642}%
\pgfsetfillcolor{currentfill}%
\pgfsetfillopacity{0.700000}%
\pgfsetlinewidth{0.000000pt}%
\definecolor{currentstroke}{rgb}{0.000000,0.000000,0.000000}%
\pgfsetstrokecolor{currentstroke}%
\pgfsetdash{}{0pt}%
\pgfpathmoveto{\pgfqpoint{4.215270in}{3.259466in}}%
\pgfpathlineto{\pgfqpoint{4.228635in}{3.250500in}}%
\pgfpathlineto{\pgfqpoint{4.242003in}{3.241621in}}%
\pgfpathlineto{\pgfqpoint{4.255375in}{3.232827in}}%
\pgfpathlineto{\pgfqpoint{4.268751in}{3.224119in}}%
\pgfpathlineto{\pgfqpoint{4.276442in}{3.245896in}}%
\pgfpathlineto{\pgfqpoint{4.284133in}{3.268072in}}%
\pgfpathlineto{\pgfqpoint{4.291825in}{3.290656in}}%
\pgfpathlineto{\pgfqpoint{4.299517in}{3.313656in}}%
\pgfpathlineto{\pgfqpoint{4.286144in}{3.322824in}}%
\pgfpathlineto{\pgfqpoint{4.272774in}{3.332077in}}%
\pgfpathlineto{\pgfqpoint{4.259407in}{3.341416in}}%
\pgfpathlineto{\pgfqpoint{4.246044in}{3.350842in}}%
\pgfpathlineto{\pgfqpoint{4.238350in}{3.327374in}}%
\pgfpathlineto{\pgfqpoint{4.230657in}{3.304327in}}%
\pgfpathlineto{\pgfqpoint{4.222964in}{3.281695in}}%
\pgfpathlineto{\pgfqpoint{4.215270in}{3.259466in}}%
\pgfpathclose%
\pgfusepath{fill}%
\end{pgfscope}%
\begin{pgfscope}%
\pgfpathrectangle{\pgfqpoint{1.150000in}{0.150000in}}{\pgfqpoint{5.700000in}{5.700000in}}%
\pgfusepath{clip}%
\pgfsetbuttcap%
\pgfsetroundjoin%
\definecolor{currentfill}{rgb}{0.678489,0.863742,0.189503}%
\pgfsetfillcolor{currentfill}%
\pgfsetfillopacity{0.700000}%
\pgfsetlinewidth{0.000000pt}%
\definecolor{currentstroke}{rgb}{0.000000,0.000000,0.000000}%
\pgfsetstrokecolor{currentstroke}%
\pgfsetdash{}{0pt}%
\pgfpathmoveto{\pgfqpoint{3.706276in}{4.687426in}}%
\pgfpathlineto{\pgfqpoint{3.719634in}{4.669880in}}%
\pgfpathlineto{\pgfqpoint{3.732989in}{4.652462in}}%
\pgfpathlineto{\pgfqpoint{3.746342in}{4.635169in}}%
\pgfpathlineto{\pgfqpoint{3.759694in}{4.618001in}}%
\pgfpathlineto{\pgfqpoint{3.767311in}{4.659876in}}%
\pgfpathlineto{\pgfqpoint{3.774929in}{4.702483in}}%
\pgfpathlineto{\pgfqpoint{3.782546in}{4.745836in}}%
\pgfpathlineto{\pgfqpoint{3.769181in}{4.763478in}}%
\pgfpathlineto{\pgfqpoint{3.755814in}{4.781247in}}%
\pgfpathlineto{\pgfqpoint{3.742445in}{4.799141in}}%
\pgfpathlineto{\pgfqpoint{3.729074in}{4.817164in}}%
\pgfpathlineto{\pgfqpoint{3.721475in}{4.773169in}}%
\pgfpathlineto{\pgfqpoint{3.713876in}{4.729928in}}%
\pgfpathlineto{\pgfqpoint{3.706276in}{4.687426in}}%
\pgfpathclose%
\pgfusepath{fill}%
\end{pgfscope}%
\begin{pgfscope}%
\pgfpathrectangle{\pgfqpoint{1.150000in}{0.150000in}}{\pgfqpoint{5.700000in}{5.700000in}}%
\pgfusepath{clip}%
\pgfsetbuttcap%
\pgfsetroundjoin%
\definecolor{currentfill}{rgb}{0.248629,0.278775,0.534556}%
\pgfsetfillcolor{currentfill}%
\pgfsetfillopacity{0.700000}%
\pgfsetlinewidth{0.000000pt}%
\definecolor{currentstroke}{rgb}{0.000000,0.000000,0.000000}%
\pgfsetstrokecolor{currentstroke}%
\pgfsetdash{}{0pt}%
\pgfpathmoveto{\pgfqpoint{3.572273in}{3.020603in}}%
\pgfpathlineto{\pgfqpoint{3.585568in}{3.011252in}}%
\pgfpathlineto{\pgfqpoint{3.598864in}{3.002005in}}%
\pgfpathlineto{\pgfqpoint{3.612163in}{2.992861in}}%
\pgfpathlineto{\pgfqpoint{3.625464in}{2.983819in}}%
\pgfpathlineto{\pgfqpoint{3.633264in}{2.999946in}}%
\pgfpathlineto{\pgfqpoint{3.641058in}{3.016313in}}%
\pgfpathlineto{\pgfqpoint{3.648848in}{3.032928in}}%
\pgfpathlineto{\pgfqpoint{3.656632in}{3.049795in}}%
\pgfpathlineto{\pgfqpoint{3.643335in}{3.059153in}}%
\pgfpathlineto{\pgfqpoint{3.630040in}{3.068614in}}%
\pgfpathlineto{\pgfqpoint{3.616747in}{3.078177in}}%
\pgfpathlineto{\pgfqpoint{3.603456in}{3.087845in}}%
\pgfpathlineto{\pgfqpoint{3.595668in}{3.070654in}}%
\pgfpathlineto{\pgfqpoint{3.587875in}{3.053720in}}%
\pgfpathlineto{\pgfqpoint{3.580077in}{3.037039in}}%
\pgfpathlineto{\pgfqpoint{3.572273in}{3.020603in}}%
\pgfpathclose%
\pgfusepath{fill}%
\end{pgfscope}%
\begin{pgfscope}%
\pgfpathrectangle{\pgfqpoint{1.150000in}{0.150000in}}{\pgfqpoint{5.700000in}{5.700000in}}%
\pgfusepath{clip}%
\pgfsetbuttcap%
\pgfsetroundjoin%
\definecolor{currentfill}{rgb}{0.243113,0.292092,0.538516}%
\pgfsetfillcolor{currentfill}%
\pgfsetfillopacity{0.700000}%
\pgfsetlinewidth{0.000000pt}%
\definecolor{currentstroke}{rgb}{0.000000,0.000000,0.000000}%
\pgfsetstrokecolor{currentstroke}%
\pgfsetdash{}{0pt}%
\pgfpathmoveto{\pgfqpoint{3.794159in}{3.045960in}}%
\pgfpathlineto{\pgfqpoint{3.807476in}{3.037158in}}%
\pgfpathlineto{\pgfqpoint{3.820795in}{3.028452in}}%
\pgfpathlineto{\pgfqpoint{3.834117in}{3.019841in}}%
\pgfpathlineto{\pgfqpoint{3.847442in}{3.011325in}}%
\pgfpathlineto{\pgfqpoint{3.855193in}{3.028500in}}%
\pgfpathlineto{\pgfqpoint{3.862940in}{3.045950in}}%
\pgfpathlineto{\pgfqpoint{3.870683in}{3.063682in}}%
\pgfpathlineto{\pgfqpoint{3.878423in}{3.081701in}}%
\pgfpathlineto{\pgfqpoint{3.865101in}{3.090574in}}%
\pgfpathlineto{\pgfqpoint{3.851783in}{3.099541in}}%
\pgfpathlineto{\pgfqpoint{3.838467in}{3.108604in}}%
\pgfpathlineto{\pgfqpoint{3.825154in}{3.117764in}}%
\pgfpathlineto{\pgfqpoint{3.817411in}{3.099380in}}%
\pgfpathlineto{\pgfqpoint{3.809665in}{3.081289in}}%
\pgfpathlineto{\pgfqpoint{3.801914in}{3.063485in}}%
\pgfpathlineto{\pgfqpoint{3.794159in}{3.045960in}}%
\pgfpathclose%
\pgfusepath{fill}%
\end{pgfscope}%
\begin{pgfscope}%
\pgfpathrectangle{\pgfqpoint{1.150000in}{0.150000in}}{\pgfqpoint{5.700000in}{5.700000in}}%
\pgfusepath{clip}%
\pgfsetbuttcap%
\pgfsetroundjoin%
\definecolor{currentfill}{rgb}{0.157729,0.485932,0.558013}%
\pgfsetfillcolor{currentfill}%
\pgfsetfillopacity{0.700000}%
\pgfsetlinewidth{0.000000pt}%
\definecolor{currentstroke}{rgb}{0.000000,0.000000,0.000000}%
\pgfsetstrokecolor{currentstroke}%
\pgfsetdash{}{0pt}%
\pgfpathmoveto{\pgfqpoint{4.361102in}{3.513752in}}%
\pgfpathlineto{\pgfqpoint{4.374477in}{3.503685in}}%
\pgfpathlineto{\pgfqpoint{4.387855in}{3.493704in}}%
\pgfpathlineto{\pgfqpoint{4.401236in}{3.483807in}}%
\pgfpathlineto{\pgfqpoint{4.414621in}{3.473995in}}%
\pgfpathlineto{\pgfqpoint{4.422329in}{3.500643in}}%
\pgfpathlineto{\pgfqpoint{4.430040in}{3.527796in}}%
\pgfpathlineto{\pgfqpoint{4.437755in}{3.555465in}}%
\pgfpathlineto{\pgfqpoint{4.445473in}{3.583659in}}%
\pgfpathlineto{\pgfqpoint{4.432088in}{3.593997in}}%
\pgfpathlineto{\pgfqpoint{4.418706in}{3.604419in}}%
\pgfpathlineto{\pgfqpoint{4.405327in}{3.614927in}}%
\pgfpathlineto{\pgfqpoint{4.391951in}{3.625520in}}%
\pgfpathlineto{\pgfqpoint{4.384234in}{3.596790in}}%
\pgfpathlineto{\pgfqpoint{4.376520in}{3.568593in}}%
\pgfpathlineto{\pgfqpoint{4.368809in}{3.540917in}}%
\pgfpathlineto{\pgfqpoint{4.361102in}{3.513752in}}%
\pgfpathclose%
\pgfusepath{fill}%
\end{pgfscope}%
\begin{pgfscope}%
\pgfpathrectangle{\pgfqpoint{1.150000in}{0.150000in}}{\pgfqpoint{5.700000in}{5.700000in}}%
\pgfusepath{clip}%
\pgfsetbuttcap%
\pgfsetroundjoin%
\definecolor{currentfill}{rgb}{0.174274,0.445044,0.557792}%
\pgfsetfillcolor{currentfill}%
\pgfsetfillopacity{0.700000}%
\pgfsetlinewidth{0.000000pt}%
\definecolor{currentstroke}{rgb}{0.000000,0.000000,0.000000}%
\pgfsetstrokecolor{currentstroke}%
\pgfsetdash{}{0pt}%
\pgfpathmoveto{\pgfqpoint{4.330296in}{3.410004in}}%
\pgfpathlineto{\pgfqpoint{4.343671in}{3.400441in}}%
\pgfpathlineto{\pgfqpoint{4.357050in}{3.390963in}}%
\pgfpathlineto{\pgfqpoint{4.370432in}{3.381569in}}%
\pgfpathlineto{\pgfqpoint{4.383817in}{3.372259in}}%
\pgfpathlineto{\pgfqpoint{4.391515in}{3.396984in}}%
\pgfpathlineto{\pgfqpoint{4.399214in}{3.422175in}}%
\pgfpathlineto{\pgfqpoint{4.406916in}{3.447842in}}%
\pgfpathlineto{\pgfqpoint{4.414621in}{3.473995in}}%
\pgfpathlineto{\pgfqpoint{4.401236in}{3.483807in}}%
\pgfpathlineto{\pgfqpoint{4.387855in}{3.493704in}}%
\pgfpathlineto{\pgfqpoint{4.374477in}{3.503685in}}%
\pgfpathlineto{\pgfqpoint{4.361102in}{3.513752in}}%
\pgfpathlineto{\pgfqpoint{4.353397in}{3.487088in}}%
\pgfpathlineto{\pgfqpoint{4.345694in}{3.460915in}}%
\pgfpathlineto{\pgfqpoint{4.337994in}{3.435224in}}%
\pgfpathlineto{\pgfqpoint{4.330296in}{3.410004in}}%
\pgfpathclose%
\pgfusepath{fill}%
\end{pgfscope}%
\begin{pgfscope}%
\pgfpathrectangle{\pgfqpoint{1.150000in}{0.150000in}}{\pgfqpoint{5.700000in}{5.700000in}}%
\pgfusepath{clip}%
\pgfsetbuttcap%
\pgfsetroundjoin%
\definecolor{currentfill}{rgb}{0.166383,0.690856,0.496502}%
\pgfsetfillcolor{currentfill}%
\pgfsetfillopacity{0.700000}%
\pgfsetlinewidth{0.000000pt}%
\definecolor{currentstroke}{rgb}{0.000000,0.000000,0.000000}%
\pgfsetstrokecolor{currentstroke}%
\pgfsetdash{}{0pt}%
\pgfpathmoveto{\pgfqpoint{4.240018in}{4.075385in}}%
\pgfpathlineto{\pgfqpoint{4.253369in}{4.062220in}}%
\pgfpathlineto{\pgfqpoint{4.266722in}{4.049150in}}%
\pgfpathlineto{\pgfqpoint{4.280077in}{4.036174in}}%
\pgfpathlineto{\pgfqpoint{4.293433in}{4.023291in}}%
\pgfpathlineto{\pgfqpoint{4.301176in}{4.059074in}}%
\pgfpathlineto{\pgfqpoint{4.308925in}{4.095520in}}%
\pgfpathlineto{\pgfqpoint{4.316679in}{4.132642in}}%
\pgfpathlineto{\pgfqpoint{4.303316in}{4.145977in}}%
\pgfpathlineto{\pgfqpoint{4.289955in}{4.159405in}}%
\pgfpathlineto{\pgfqpoint{4.276596in}{4.172929in}}%
\pgfpathlineto{\pgfqpoint{4.263237in}{4.186547in}}%
\pgfpathlineto{\pgfqpoint{4.255492in}{4.148815in}}%
\pgfpathlineto{\pgfqpoint{4.247753in}{4.111765in}}%
\pgfpathlineto{\pgfqpoint{4.240018in}{4.075385in}}%
\pgfpathclose%
\pgfusepath{fill}%
\end{pgfscope}%
\begin{pgfscope}%
\pgfpathrectangle{\pgfqpoint{1.150000in}{0.150000in}}{\pgfqpoint{5.700000in}{5.700000in}}%
\pgfusepath{clip}%
\pgfsetbuttcap%
\pgfsetroundjoin%
\definecolor{currentfill}{rgb}{0.237441,0.305202,0.541921}%
\pgfsetfillcolor{currentfill}%
\pgfsetfillopacity{0.700000}%
\pgfsetlinewidth{0.000000pt}%
\definecolor{currentstroke}{rgb}{0.000000,0.000000,0.000000}%
\pgfsetstrokecolor{currentstroke}%
\pgfsetdash{}{0pt}%
\pgfpathmoveto{\pgfqpoint{3.159228in}{3.078242in}}%
\pgfpathlineto{\pgfqpoint{3.172510in}{3.066988in}}%
\pgfpathlineto{\pgfqpoint{3.185792in}{3.055860in}}%
\pgfpathlineto{\pgfqpoint{3.199073in}{3.044854in}}%
\pgfpathlineto{\pgfqpoint{3.212356in}{3.033971in}}%
\pgfpathlineto{\pgfqpoint{3.220253in}{3.049120in}}%
\pgfpathlineto{\pgfqpoint{3.228143in}{3.064477in}}%
\pgfpathlineto{\pgfqpoint{3.236027in}{3.080048in}}%
\pgfpathlineto{\pgfqpoint{3.243904in}{3.095836in}}%
\pgfpathlineto{\pgfqpoint{3.230626in}{3.106976in}}%
\pgfpathlineto{\pgfqpoint{3.217348in}{3.118239in}}%
\pgfpathlineto{\pgfqpoint{3.204071in}{3.129625in}}%
\pgfpathlineto{\pgfqpoint{3.190793in}{3.141136in}}%
\pgfpathlineto{\pgfqpoint{3.182912in}{3.125082in}}%
\pgfpathlineto{\pgfqpoint{3.175024in}{3.109252in}}%
\pgfpathlineto{\pgfqpoint{3.167130in}{3.093640in}}%
\pgfpathlineto{\pgfqpoint{3.159228in}{3.078242in}}%
\pgfpathclose%
\pgfusepath{fill}%
\end{pgfscope}%
\begin{pgfscope}%
\pgfpathrectangle{\pgfqpoint{1.150000in}{0.150000in}}{\pgfqpoint{5.700000in}{5.700000in}}%
\pgfusepath{clip}%
\pgfsetbuttcap%
\pgfsetroundjoin%
\definecolor{currentfill}{rgb}{0.221989,0.339161,0.548752}%
\pgfsetfillcolor{currentfill}%
\pgfsetfillopacity{0.700000}%
\pgfsetlinewidth{0.000000pt}%
\definecolor{currentstroke}{rgb}{0.000000,0.000000,0.000000}%
\pgfsetstrokecolor{currentstroke}%
\pgfsetdash{}{0pt}%
\pgfpathmoveto{\pgfqpoint{2.968067in}{3.160990in}}%
\pgfpathlineto{\pgfqpoint{2.981363in}{3.148394in}}%
\pgfpathlineto{\pgfqpoint{2.994657in}{3.135936in}}%
\pgfpathlineto{\pgfqpoint{3.007950in}{3.123615in}}%
\pgfpathlineto{\pgfqpoint{3.021242in}{3.111429in}}%
\pgfpathlineto{\pgfqpoint{3.029182in}{3.126479in}}%
\pgfpathlineto{\pgfqpoint{3.037115in}{3.141733in}}%
\pgfpathlineto{\pgfqpoint{3.045040in}{3.157199in}}%
\pgfpathlineto{\pgfqpoint{3.052958in}{3.172878in}}%
\pgfpathlineto{\pgfqpoint{3.039670in}{3.185301in}}%
\pgfpathlineto{\pgfqpoint{3.026382in}{3.197860in}}%
\pgfpathlineto{\pgfqpoint{3.013092in}{3.210557in}}%
\pgfpathlineto{\pgfqpoint{2.999800in}{3.223391in}}%
\pgfpathlineto{\pgfqpoint{2.991879in}{3.207466in}}%
\pgfpathlineto{\pgfqpoint{2.983949in}{3.191760in}}%
\pgfpathlineto{\pgfqpoint{2.976012in}{3.176269in}}%
\pgfpathlineto{\pgfqpoint{2.968067in}{3.160990in}}%
\pgfpathclose%
\pgfusepath{fill}%
\end{pgfscope}%
\begin{pgfscope}%
\pgfpathrectangle{\pgfqpoint{1.150000in}{0.150000in}}{\pgfqpoint{5.700000in}{5.700000in}}%
\pgfusepath{clip}%
\pgfsetbuttcap%
\pgfsetroundjoin%
\definecolor{currentfill}{rgb}{0.751884,0.874951,0.143228}%
\pgfsetfillcolor{currentfill}%
\pgfsetfillopacity{0.700000}%
\pgfsetlinewidth{0.000000pt}%
\definecolor{currentstroke}{rgb}{0.000000,0.000000,0.000000}%
\pgfsetstrokecolor{currentstroke}%
\pgfsetdash{}{0pt}%
\pgfpathmoveto{\pgfqpoint{3.652828in}{4.758906in}}%
\pgfpathlineto{\pgfqpoint{3.666193in}{4.740839in}}%
\pgfpathlineto{\pgfqpoint{3.679556in}{4.722905in}}%
\pgfpathlineto{\pgfqpoint{3.692917in}{4.705101in}}%
\pgfpathlineto{\pgfqpoint{3.706276in}{4.687426in}}%
\pgfpathlineto{\pgfqpoint{3.713876in}{4.729928in}}%
\pgfpathlineto{\pgfqpoint{3.721475in}{4.773169in}}%
\pgfpathlineto{\pgfqpoint{3.729074in}{4.817164in}}%
\pgfpathlineto{\pgfqpoint{3.715701in}{4.835316in}}%
\pgfpathlineto{\pgfqpoint{3.702325in}{4.853598in}}%
\pgfpathlineto{\pgfqpoint{3.688948in}{4.872012in}}%
\pgfpathlineto{\pgfqpoint{3.675568in}{4.890558in}}%
\pgfpathlineto{\pgfqpoint{3.667988in}{4.845917in}}%
\pgfpathlineto{\pgfqpoint{3.660408in}{4.802038in}}%
\pgfpathlineto{\pgfqpoint{3.652828in}{4.758906in}}%
\pgfpathclose%
\pgfusepath{fill}%
\end{pgfscope}%
\begin{pgfscope}%
\pgfpathrectangle{\pgfqpoint{1.150000in}{0.150000in}}{\pgfqpoint{5.700000in}{5.700000in}}%
\pgfusepath{clip}%
\pgfsetbuttcap%
\pgfsetroundjoin%
\definecolor{currentfill}{rgb}{0.188923,0.410910,0.556326}%
\pgfsetfillcolor{currentfill}%
\pgfsetfillopacity{0.700000}%
\pgfsetlinewidth{0.000000pt}%
\definecolor{currentstroke}{rgb}{0.000000,0.000000,0.000000}%
\pgfsetstrokecolor{currentstroke}%
\pgfsetdash{}{0pt}%
\pgfpathmoveto{\pgfqpoint{4.299517in}{3.313656in}}%
\pgfpathlineto{\pgfqpoint{4.312894in}{3.304574in}}%
\pgfpathlineto{\pgfqpoint{4.326274in}{3.295577in}}%
\pgfpathlineto{\pgfqpoint{4.339658in}{3.286663in}}%
\pgfpathlineto{\pgfqpoint{4.353046in}{3.277834in}}%
\pgfpathlineto{\pgfqpoint{4.360736in}{3.300788in}}%
\pgfpathlineto{\pgfqpoint{4.368428in}{3.324171in}}%
\pgfpathlineto{\pgfqpoint{4.376122in}{3.347991in}}%
\pgfpathlineto{\pgfqpoint{4.383817in}{3.372259in}}%
\pgfpathlineto{\pgfqpoint{4.370432in}{3.381569in}}%
\pgfpathlineto{\pgfqpoint{4.357050in}{3.390963in}}%
\pgfpathlineto{\pgfqpoint{4.343671in}{3.400441in}}%
\pgfpathlineto{\pgfqpoint{4.330296in}{3.410004in}}%
\pgfpathlineto{\pgfqpoint{4.322599in}{3.385247in}}%
\pgfpathlineto{\pgfqpoint{4.314904in}{3.360943in}}%
\pgfpathlineto{\pgfqpoint{4.307210in}{3.337082in}}%
\pgfpathlineto{\pgfqpoint{4.299517in}{3.313656in}}%
\pgfpathclose%
\pgfusepath{fill}%
\end{pgfscope}%
\begin{pgfscope}%
\pgfpathrectangle{\pgfqpoint{1.150000in}{0.150000in}}{\pgfqpoint{5.700000in}{5.700000in}}%
\pgfusepath{clip}%
\pgfsetbuttcap%
\pgfsetroundjoin%
\definecolor{currentfill}{rgb}{0.140536,0.530132,0.555659}%
\pgfsetfillcolor{currentfill}%
\pgfsetfillopacity{0.700000}%
\pgfsetlinewidth{0.000000pt}%
\definecolor{currentstroke}{rgb}{0.000000,0.000000,0.000000}%
\pgfsetstrokecolor{currentstroke}%
\pgfsetdash{}{0pt}%
\pgfpathmoveto{\pgfqpoint{4.391951in}{3.625520in}}%
\pgfpathlineto{\pgfqpoint{4.405327in}{3.614927in}}%
\pgfpathlineto{\pgfqpoint{4.418706in}{3.604419in}}%
\pgfpathlineto{\pgfqpoint{4.432088in}{3.593997in}}%
\pgfpathlineto{\pgfqpoint{4.445473in}{3.583659in}}%
\pgfpathlineto{\pgfqpoint{4.453196in}{3.612390in}}%
\pgfpathlineto{\pgfqpoint{4.460923in}{3.641668in}}%
\pgfpathlineto{\pgfqpoint{4.468655in}{3.671504in}}%
\pgfpathlineto{\pgfqpoint{4.476392in}{3.701907in}}%
\pgfpathlineto{\pgfqpoint{4.463006in}{3.712794in}}%
\pgfpathlineto{\pgfqpoint{4.449621in}{3.723766in}}%
\pgfpathlineto{\pgfqpoint{4.436240in}{3.734822in}}%
\pgfpathlineto{\pgfqpoint{4.422862in}{3.745965in}}%
\pgfpathlineto{\pgfqpoint{4.415127in}{3.715003in}}%
\pgfpathlineto{\pgfqpoint{4.407397in}{3.684615in}}%
\pgfpathlineto{\pgfqpoint{4.399672in}{3.654791in}}%
\pgfpathlineto{\pgfqpoint{4.391951in}{3.625520in}}%
\pgfpathclose%
\pgfusepath{fill}%
\end{pgfscope}%
\begin{pgfscope}%
\pgfpathrectangle{\pgfqpoint{1.150000in}{0.150000in}}{\pgfqpoint{5.700000in}{5.700000in}}%
\pgfusepath{clip}%
\pgfsetbuttcap%
\pgfsetroundjoin%
\definecolor{currentfill}{rgb}{0.121148,0.592739,0.544641}%
\pgfsetfillcolor{currentfill}%
\pgfsetfillopacity{0.700000}%
\pgfsetlinewidth{0.000000pt}%
\definecolor{currentstroke}{rgb}{0.000000,0.000000,0.000000}%
\pgfsetstrokecolor{currentstroke}%
\pgfsetdash{}{0pt}%
\pgfpathmoveto{\pgfqpoint{4.369373in}{3.791404in}}%
\pgfpathlineto{\pgfqpoint{4.382742in}{3.779913in}}%
\pgfpathlineto{\pgfqpoint{4.396112in}{3.768510in}}%
\pgfpathlineto{\pgfqpoint{4.409486in}{3.757194in}}%
\pgfpathlineto{\pgfqpoint{4.422862in}{3.745965in}}%
\pgfpathlineto{\pgfqpoint{4.430601in}{3.777513in}}%
\pgfpathlineto{\pgfqpoint{4.438346in}{3.809658in}}%
\pgfpathlineto{\pgfqpoint{4.446096in}{3.842411in}}%
\pgfpathlineto{\pgfqpoint{4.453853in}{3.875783in}}%
\pgfpathlineto{\pgfqpoint{4.440472in}{3.887587in}}%
\pgfpathlineto{\pgfqpoint{4.427095in}{3.899478in}}%
\pgfpathlineto{\pgfqpoint{4.413719in}{3.911457in}}%
\pgfpathlineto{\pgfqpoint{4.400346in}{3.923524in}}%
\pgfpathlineto{\pgfqpoint{4.392595in}{3.889566in}}%
\pgfpathlineto{\pgfqpoint{4.384849in}{3.856235in}}%
\pgfpathlineto{\pgfqpoint{4.377109in}{3.823518in}}%
\pgfpathlineto{\pgfqpoint{4.369373in}{3.791404in}}%
\pgfpathclose%
\pgfusepath{fill}%
\end{pgfscope}%
\begin{pgfscope}%
\pgfpathrectangle{\pgfqpoint{1.150000in}{0.150000in}}{\pgfqpoint{5.700000in}{5.700000in}}%
\pgfusepath{clip}%
\pgfsetbuttcap%
\pgfsetroundjoin%
\definecolor{currentfill}{rgb}{0.248629,0.278775,0.534556}%
\pgfsetfillcolor{currentfill}%
\pgfsetfillopacity{0.700000}%
\pgfsetlinewidth{0.000000pt}%
\definecolor{currentstroke}{rgb}{0.000000,0.000000,0.000000}%
\pgfsetstrokecolor{currentstroke}%
\pgfsetdash{}{0pt}%
\pgfpathmoveto{\pgfqpoint{3.709845in}{3.013372in}}%
\pgfpathlineto{\pgfqpoint{3.723154in}{3.004516in}}%
\pgfpathlineto{\pgfqpoint{3.736466in}{2.995758in}}%
\pgfpathlineto{\pgfqpoint{3.749780in}{2.987098in}}%
\pgfpathlineto{\pgfqpoint{3.763098in}{2.978535in}}%
\pgfpathlineto{\pgfqpoint{3.770870in}{2.995004in}}%
\pgfpathlineto{\pgfqpoint{3.778637in}{3.011726in}}%
\pgfpathlineto{\pgfqpoint{3.786400in}{3.028710in}}%
\pgfpathlineto{\pgfqpoint{3.794159in}{3.045960in}}%
\pgfpathlineto{\pgfqpoint{3.780846in}{3.054860in}}%
\pgfpathlineto{\pgfqpoint{3.767535in}{3.063856in}}%
\pgfpathlineto{\pgfqpoint{3.754227in}{3.072950in}}%
\pgfpathlineto{\pgfqpoint{3.740921in}{3.082143in}}%
\pgfpathlineto{\pgfqpoint{3.733159in}{3.064548in}}%
\pgfpathlineto{\pgfqpoint{3.725392in}{3.047226in}}%
\pgfpathlineto{\pgfqpoint{3.717621in}{3.030169in}}%
\pgfpathlineto{\pgfqpoint{3.709845in}{3.013372in}}%
\pgfpathclose%
\pgfusepath{fill}%
\end{pgfscope}%
\begin{pgfscope}%
\pgfpathrectangle{\pgfqpoint{1.150000in}{0.150000in}}{\pgfqpoint{5.700000in}{5.700000in}}%
\pgfusepath{clip}%
\pgfsetbuttcap%
\pgfsetroundjoin%
\definecolor{currentfill}{rgb}{0.146616,0.673050,0.508936}%
\pgfsetfillcolor{currentfill}%
\pgfsetfillopacity{0.700000}%
\pgfsetlinewidth{0.000000pt}%
\definecolor{currentstroke}{rgb}{0.000000,0.000000,0.000000}%
\pgfsetstrokecolor{currentstroke}%
\pgfsetdash{}{0pt}%
\pgfpathmoveto{\pgfqpoint{4.293433in}{4.023291in}}%
\pgfpathlineto{\pgfqpoint{4.306791in}{4.010501in}}%
\pgfpathlineto{\pgfqpoint{4.320150in}{3.997803in}}%
\pgfpathlineto{\pgfqpoint{4.333511in}{3.985197in}}%
\pgfpathlineto{\pgfqpoint{4.346874in}{3.972682in}}%
\pgfpathlineto{\pgfqpoint{4.354625in}{4.007870in}}%
\pgfpathlineto{\pgfqpoint{4.362381in}{4.043715in}}%
\pgfpathlineto{\pgfqpoint{4.370144in}{4.080228in}}%
\pgfpathlineto{\pgfqpoint{4.356775in}{4.093193in}}%
\pgfpathlineto{\pgfqpoint{4.343408in}{4.106250in}}%
\pgfpathlineto{\pgfqpoint{4.330043in}{4.119400in}}%
\pgfpathlineto{\pgfqpoint{4.316679in}{4.132642in}}%
\pgfpathlineto{\pgfqpoint{4.308925in}{4.095520in}}%
\pgfpathlineto{\pgfqpoint{4.301176in}{4.059074in}}%
\pgfpathlineto{\pgfqpoint{4.293433in}{4.023291in}}%
\pgfpathclose%
\pgfusepath{fill}%
\end{pgfscope}%
\begin{pgfscope}%
\pgfpathrectangle{\pgfqpoint{1.150000in}{0.150000in}}{\pgfqpoint{5.700000in}{5.700000in}}%
\pgfusepath{clip}%
\pgfsetbuttcap%
\pgfsetroundjoin%
\definecolor{currentfill}{rgb}{0.225863,0.330805,0.547314}%
\pgfsetfillcolor{currentfill}%
\pgfsetfillopacity{0.700000}%
\pgfsetlinewidth{0.000000pt}%
\definecolor{currentstroke}{rgb}{0.000000,0.000000,0.000000}%
\pgfsetstrokecolor{currentstroke}%
\pgfsetdash{}{0pt}%
\pgfpathmoveto{\pgfqpoint{4.100247in}{3.128518in}}%
\pgfpathlineto{\pgfqpoint{4.113604in}{3.120059in}}%
\pgfpathlineto{\pgfqpoint{4.126964in}{3.111687in}}%
\pgfpathlineto{\pgfqpoint{4.140328in}{3.103403in}}%
\pgfpathlineto{\pgfqpoint{4.153695in}{3.095206in}}%
\pgfpathlineto{\pgfqpoint{4.161396in}{3.114496in}}%
\pgfpathlineto{\pgfqpoint{4.169096in}{3.134124in}}%
\pgfpathlineto{\pgfqpoint{4.176794in}{3.154100in}}%
\pgfpathlineto{\pgfqpoint{4.184491in}{3.174431in}}%
\pgfpathlineto{\pgfqpoint{4.171127in}{3.183045in}}%
\pgfpathlineto{\pgfqpoint{4.157766in}{3.191746in}}%
\pgfpathlineto{\pgfqpoint{4.144409in}{3.200535in}}%
\pgfpathlineto{\pgfqpoint{4.131056in}{3.209412in}}%
\pgfpathlineto{\pgfqpoint{4.123356in}{3.188656in}}%
\pgfpathlineto{\pgfqpoint{4.115655in}{3.168260in}}%
\pgfpathlineto{\pgfqpoint{4.107952in}{3.148217in}}%
\pgfpathlineto{\pgfqpoint{4.100247in}{3.128518in}}%
\pgfpathclose%
\pgfusepath{fill}%
\end{pgfscope}%
\begin{pgfscope}%
\pgfpathrectangle{\pgfqpoint{1.150000in}{0.150000in}}{\pgfqpoint{5.700000in}{5.700000in}}%
\pgfusepath{clip}%
\pgfsetbuttcap%
\pgfsetroundjoin%
\definecolor{currentfill}{rgb}{0.233603,0.313828,0.543914}%
\pgfsetfillcolor{currentfill}%
\pgfsetfillopacity{0.700000}%
\pgfsetlinewidth{0.000000pt}%
\definecolor{currentstroke}{rgb}{0.000000,0.000000,0.000000}%
\pgfsetstrokecolor{currentstroke}%
\pgfsetdash{}{0pt}%
\pgfpathmoveto{\pgfqpoint{4.016002in}{3.086157in}}%
\pgfpathlineto{\pgfqpoint{4.029348in}{3.077739in}}%
\pgfpathlineto{\pgfqpoint{4.042698in}{3.069410in}}%
\pgfpathlineto{\pgfqpoint{4.056051in}{3.061170in}}%
\pgfpathlineto{\pgfqpoint{4.069408in}{3.053019in}}%
\pgfpathlineto{\pgfqpoint{4.077121in}{3.071414in}}%
\pgfpathlineto{\pgfqpoint{4.084832in}{3.090124in}}%
\pgfpathlineto{\pgfqpoint{4.092541in}{3.109157in}}%
\pgfpathlineto{\pgfqpoint{4.100247in}{3.128518in}}%
\pgfpathlineto{\pgfqpoint{4.086894in}{3.137066in}}%
\pgfpathlineto{\pgfqpoint{4.073544in}{3.145703in}}%
\pgfpathlineto{\pgfqpoint{4.060198in}{3.154428in}}%
\pgfpathlineto{\pgfqpoint{4.046855in}{3.163244in}}%
\pgfpathlineto{\pgfqpoint{4.039145in}{3.143478in}}%
\pgfpathlineto{\pgfqpoint{4.031433in}{3.124046in}}%
\pgfpathlineto{\pgfqpoint{4.023719in}{3.104942in}}%
\pgfpathlineto{\pgfqpoint{4.016002in}{3.086157in}}%
\pgfpathclose%
\pgfusepath{fill}%
\end{pgfscope}%
\begin{pgfscope}%
\pgfpathrectangle{\pgfqpoint{1.150000in}{0.150000in}}{\pgfqpoint{5.700000in}{5.700000in}}%
\pgfusepath{clip}%
\pgfsetbuttcap%
\pgfsetroundjoin%
\definecolor{currentfill}{rgb}{0.250425,0.274290,0.533103}%
\pgfsetfillcolor{currentfill}%
\pgfsetfillopacity{0.700000}%
\pgfsetlinewidth{0.000000pt}%
\definecolor{currentstroke}{rgb}{0.000000,0.000000,0.000000}%
\pgfsetstrokecolor{currentstroke}%
\pgfsetdash{}{0pt}%
\pgfpathmoveto{\pgfqpoint{3.350145in}{3.011006in}}%
\pgfpathlineto{\pgfqpoint{3.363428in}{3.000924in}}%
\pgfpathlineto{\pgfqpoint{3.376713in}{2.990956in}}%
\pgfpathlineto{\pgfqpoint{3.389999in}{2.981099in}}%
\pgfpathlineto{\pgfqpoint{3.403287in}{2.971355in}}%
\pgfpathlineto{\pgfqpoint{3.411145in}{2.986558in}}%
\pgfpathlineto{\pgfqpoint{3.418996in}{3.001974in}}%
\pgfpathlineto{\pgfqpoint{3.426842in}{3.017609in}}%
\pgfpathlineto{\pgfqpoint{3.434682in}{3.033466in}}%
\pgfpathlineto{\pgfqpoint{3.421398in}{3.043487in}}%
\pgfpathlineto{\pgfqpoint{3.408116in}{3.053620in}}%
\pgfpathlineto{\pgfqpoint{3.394836in}{3.063866in}}%
\pgfpathlineto{\pgfqpoint{3.381556in}{3.074224in}}%
\pgfpathlineto{\pgfqpoint{3.373712in}{3.058082in}}%
\pgfpathlineto{\pgfqpoint{3.365863in}{3.042169in}}%
\pgfpathlineto{\pgfqpoint{3.358007in}{3.026478in}}%
\pgfpathlineto{\pgfqpoint{3.350145in}{3.011006in}}%
\pgfpathclose%
\pgfusepath{fill}%
\end{pgfscope}%
\begin{pgfscope}%
\pgfpathrectangle{\pgfqpoint{1.150000in}{0.150000in}}{\pgfqpoint{5.700000in}{5.700000in}}%
\pgfusepath{clip}%
\pgfsetbuttcap%
\pgfsetroundjoin%
\definecolor{currentfill}{rgb}{0.216210,0.351535,0.550627}%
\pgfsetfillcolor{currentfill}%
\pgfsetfillopacity{0.700000}%
\pgfsetlinewidth{0.000000pt}%
\definecolor{currentstroke}{rgb}{0.000000,0.000000,0.000000}%
\pgfsetstrokecolor{currentstroke}%
\pgfsetdash{}{0pt}%
\pgfpathmoveto{\pgfqpoint{4.184491in}{3.174431in}}%
\pgfpathlineto{\pgfqpoint{4.197859in}{3.165903in}}%
\pgfpathlineto{\pgfqpoint{4.211230in}{3.157462in}}%
\pgfpathlineto{\pgfqpoint{4.224605in}{3.149106in}}%
\pgfpathlineto{\pgfqpoint{4.237984in}{3.140836in}}%
\pgfpathlineto{\pgfqpoint{4.245676in}{3.161100in}}%
\pgfpathlineto{\pgfqpoint{4.253368in}{3.181729in}}%
\pgfpathlineto{\pgfqpoint{4.261060in}{3.202733in}}%
\pgfpathlineto{\pgfqpoint{4.268751in}{3.224119in}}%
\pgfpathlineto{\pgfqpoint{4.255375in}{3.232827in}}%
\pgfpathlineto{\pgfqpoint{4.242003in}{3.241621in}}%
\pgfpathlineto{\pgfqpoint{4.228635in}{3.250500in}}%
\pgfpathlineto{\pgfqpoint{4.215270in}{3.259466in}}%
\pgfpathlineto{\pgfqpoint{4.207576in}{3.237634in}}%
\pgfpathlineto{\pgfqpoint{4.199882in}{3.216190in}}%
\pgfpathlineto{\pgfqpoint{4.192187in}{3.195125in}}%
\pgfpathlineto{\pgfqpoint{4.184491in}{3.174431in}}%
\pgfpathclose%
\pgfusepath{fill}%
\end{pgfscope}%
\begin{pgfscope}%
\pgfpathrectangle{\pgfqpoint{1.150000in}{0.150000in}}{\pgfqpoint{5.700000in}{5.700000in}}%
\pgfusepath{clip}%
\pgfsetbuttcap%
\pgfsetroundjoin%
\definecolor{currentfill}{rgb}{0.231674,0.318106,0.544834}%
\pgfsetfillcolor{currentfill}%
\pgfsetfillopacity{0.700000}%
\pgfsetlinewidth{0.000000pt}%
\definecolor{currentstroke}{rgb}{0.000000,0.000000,0.000000}%
\pgfsetstrokecolor{currentstroke}%
\pgfsetdash{}{0pt}%
\pgfpathmoveto{\pgfqpoint{3.021242in}{3.111429in}}%
\pgfpathlineto{\pgfqpoint{3.034533in}{3.099379in}}%
\pgfpathlineto{\pgfqpoint{3.047823in}{3.087461in}}%
\pgfpathlineto{\pgfqpoint{3.061112in}{3.075676in}}%
\pgfpathlineto{\pgfqpoint{3.074400in}{3.064022in}}%
\pgfpathlineto{\pgfqpoint{3.082336in}{3.078841in}}%
\pgfpathlineto{\pgfqpoint{3.090264in}{3.093861in}}%
\pgfpathlineto{\pgfqpoint{3.098184in}{3.109086in}}%
\pgfpathlineto{\pgfqpoint{3.106098in}{3.124520in}}%
\pgfpathlineto{\pgfqpoint{3.092814in}{3.136412in}}%
\pgfpathlineto{\pgfqpoint{3.079529in}{3.148435in}}%
\pgfpathlineto{\pgfqpoint{3.066244in}{3.160590in}}%
\pgfpathlineto{\pgfqpoint{3.052958in}{3.172878in}}%
\pgfpathlineto{\pgfqpoint{3.045040in}{3.157199in}}%
\pgfpathlineto{\pgfqpoint{3.037115in}{3.141733in}}%
\pgfpathlineto{\pgfqpoint{3.029182in}{3.126479in}}%
\pgfpathlineto{\pgfqpoint{3.021242in}{3.111429in}}%
\pgfpathclose%
\pgfusepath{fill}%
\end{pgfscope}%
\begin{pgfscope}%
\pgfpathrectangle{\pgfqpoint{1.150000in}{0.150000in}}{\pgfqpoint{5.700000in}{5.700000in}}%
\pgfusepath{clip}%
\pgfsetbuttcap%
\pgfsetroundjoin%
\definecolor{currentfill}{rgb}{0.252194,0.269783,0.531579}%
\pgfsetfillcolor{currentfill}%
\pgfsetfillopacity{0.700000}%
\pgfsetlinewidth{0.000000pt}%
\definecolor{currentstroke}{rgb}{0.000000,0.000000,0.000000}%
\pgfsetstrokecolor{currentstroke}%
\pgfsetdash{}{0pt}%
\pgfpathmoveto{\pgfqpoint{3.487829in}{2.994478in}}%
\pgfpathlineto{\pgfqpoint{3.501120in}{2.985001in}}%
\pgfpathlineto{\pgfqpoint{3.514413in}{2.975631in}}%
\pgfpathlineto{\pgfqpoint{3.527708in}{2.966367in}}%
\pgfpathlineto{\pgfqpoint{3.541004in}{2.957208in}}%
\pgfpathlineto{\pgfqpoint{3.548830in}{2.972716in}}%
\pgfpathlineto{\pgfqpoint{3.556650in}{2.988447in}}%
\pgfpathlineto{\pgfqpoint{3.564464in}{3.004408in}}%
\pgfpathlineto{\pgfqpoint{3.572273in}{3.020603in}}%
\pgfpathlineto{\pgfqpoint{3.558980in}{3.030058in}}%
\pgfpathlineto{\pgfqpoint{3.545690in}{3.039619in}}%
\pgfpathlineto{\pgfqpoint{3.532401in}{3.049285in}}%
\pgfpathlineto{\pgfqpoint{3.519114in}{3.059059in}}%
\pgfpathlineto{\pgfqpoint{3.511301in}{3.042559in}}%
\pgfpathlineto{\pgfqpoint{3.503483in}{3.026300in}}%
\pgfpathlineto{\pgfqpoint{3.495659in}{3.010274in}}%
\pgfpathlineto{\pgfqpoint{3.487829in}{2.994478in}}%
\pgfpathclose%
\pgfusepath{fill}%
\end{pgfscope}%
\begin{pgfscope}%
\pgfpathrectangle{\pgfqpoint{1.150000in}{0.150000in}}{\pgfqpoint{5.700000in}{5.700000in}}%
\pgfusepath{clip}%
\pgfsetbuttcap%
\pgfsetroundjoin%
\definecolor{currentfill}{rgb}{0.835270,0.886029,0.102646}%
\pgfsetfillcolor{currentfill}%
\pgfsetfillopacity{0.700000}%
\pgfsetlinewidth{0.000000pt}%
\definecolor{currentstroke}{rgb}{0.000000,0.000000,0.000000}%
\pgfsetstrokecolor{currentstroke}%
\pgfsetdash{}{0pt}%
\pgfpathmoveto{\pgfqpoint{3.599342in}{4.832513in}}%
\pgfpathlineto{\pgfqpoint{3.612717in}{4.813907in}}%
\pgfpathlineto{\pgfqpoint{3.626090in}{4.795438in}}%
\pgfpathlineto{\pgfqpoint{3.639460in}{4.777105in}}%
\pgfpathlineto{\pgfqpoint{3.652828in}{4.758906in}}%
\pgfpathlineto{\pgfqpoint{3.660408in}{4.802038in}}%
\pgfpathlineto{\pgfqpoint{3.667988in}{4.845917in}}%
\pgfpathlineto{\pgfqpoint{3.675568in}{4.890558in}}%
\pgfpathlineto{\pgfqpoint{3.662185in}{4.909238in}}%
\pgfpathlineto{\pgfqpoint{3.648800in}{4.928053in}}%
\pgfpathlineto{\pgfqpoint{3.635412in}{4.947004in}}%
\pgfpathlineto{\pgfqpoint{3.622021in}{4.966092in}}%
\pgfpathlineto{\pgfqpoint{3.614463in}{4.920802in}}%
\pgfpathlineto{\pgfqpoint{3.606903in}{4.876279in}}%
\pgfpathlineto{\pgfqpoint{3.599342in}{4.832513in}}%
\pgfpathclose%
\pgfusepath{fill}%
\end{pgfscope}%
\begin{pgfscope}%
\pgfpathrectangle{\pgfqpoint{1.150000in}{0.150000in}}{\pgfqpoint{5.700000in}{5.700000in}}%
\pgfusepath{clip}%
\pgfsetbuttcap%
\pgfsetroundjoin%
\definecolor{currentfill}{rgb}{0.241237,0.296485,0.539709}%
\pgfsetfillcolor{currentfill}%
\pgfsetfillopacity{0.700000}%
\pgfsetlinewidth{0.000000pt}%
\definecolor{currentstroke}{rgb}{0.000000,0.000000,0.000000}%
\pgfsetstrokecolor{currentstroke}%
\pgfsetdash{}{0pt}%
\pgfpathmoveto{\pgfqpoint{3.931738in}{3.047150in}}%
\pgfpathlineto{\pgfqpoint{3.945075in}{3.038744in}}%
\pgfpathlineto{\pgfqpoint{3.958415in}{3.030430in}}%
\pgfpathlineto{\pgfqpoint{3.971758in}{3.022206in}}%
\pgfpathlineto{\pgfqpoint{3.985105in}{3.014074in}}%
\pgfpathlineto{\pgfqpoint{3.992834in}{3.031651in}}%
\pgfpathlineto{\pgfqpoint{4.000559in}{3.049519in}}%
\pgfpathlineto{\pgfqpoint{4.008282in}{3.067685in}}%
\pgfpathlineto{\pgfqpoint{4.016002in}{3.086157in}}%
\pgfpathlineto{\pgfqpoint{4.002659in}{3.094666in}}%
\pgfpathlineto{\pgfqpoint{3.989319in}{3.103266in}}%
\pgfpathlineto{\pgfqpoint{3.975983in}{3.111957in}}%
\pgfpathlineto{\pgfqpoint{3.962649in}{3.120739in}}%
\pgfpathlineto{\pgfqpoint{3.954926in}{3.101883in}}%
\pgfpathlineto{\pgfqpoint{3.947200in}{3.083337in}}%
\pgfpathlineto{\pgfqpoint{3.939471in}{3.065095in}}%
\pgfpathlineto{\pgfqpoint{3.931738in}{3.047150in}}%
\pgfpathclose%
\pgfusepath{fill}%
\end{pgfscope}%
\begin{pgfscope}%
\pgfpathrectangle{\pgfqpoint{1.150000in}{0.150000in}}{\pgfqpoint{5.700000in}{5.700000in}}%
\pgfusepath{clip}%
\pgfsetbuttcap%
\pgfsetroundjoin%
\definecolor{currentfill}{rgb}{0.244972,0.287675,0.537260}%
\pgfsetfillcolor{currentfill}%
\pgfsetfillopacity{0.700000}%
\pgfsetlinewidth{0.000000pt}%
\definecolor{currentstroke}{rgb}{0.000000,0.000000,0.000000}%
\pgfsetstrokecolor{currentstroke}%
\pgfsetdash{}{0pt}%
\pgfpathmoveto{\pgfqpoint{3.212356in}{3.033971in}}%
\pgfpathlineto{\pgfqpoint{3.225638in}{3.023210in}}%
\pgfpathlineto{\pgfqpoint{3.238921in}{3.012569in}}%
\pgfpathlineto{\pgfqpoint{3.252204in}{3.002047in}}%
\pgfpathlineto{\pgfqpoint{3.265488in}{2.991643in}}%
\pgfpathlineto{\pgfqpoint{3.273381in}{3.006543in}}%
\pgfpathlineto{\pgfqpoint{3.281267in}{3.021645in}}%
\pgfpathlineto{\pgfqpoint{3.289146in}{3.036956in}}%
\pgfpathlineto{\pgfqpoint{3.297019in}{3.052480in}}%
\pgfpathlineto{\pgfqpoint{3.283740in}{3.063141in}}%
\pgfpathlineto{\pgfqpoint{3.270461in}{3.073919in}}%
\pgfpathlineto{\pgfqpoint{3.257182in}{3.084817in}}%
\pgfpathlineto{\pgfqpoint{3.243904in}{3.095836in}}%
\pgfpathlineto{\pgfqpoint{3.236027in}{3.080048in}}%
\pgfpathlineto{\pgfqpoint{3.228143in}{3.064477in}}%
\pgfpathlineto{\pgfqpoint{3.220253in}{3.049120in}}%
\pgfpathlineto{\pgfqpoint{3.212356in}{3.033971in}}%
\pgfpathclose%
\pgfusepath{fill}%
\end{pgfscope}%
\begin{pgfscope}%
\pgfpathrectangle{\pgfqpoint{1.150000in}{0.150000in}}{\pgfqpoint{5.700000in}{5.700000in}}%
\pgfusepath{clip}%
\pgfsetbuttcap%
\pgfsetroundjoin%
\definecolor{currentfill}{rgb}{0.162142,0.474838,0.558140}%
\pgfsetfillcolor{currentfill}%
\pgfsetfillopacity{0.700000}%
\pgfsetlinewidth{0.000000pt}%
\definecolor{currentstroke}{rgb}{0.000000,0.000000,0.000000}%
\pgfsetstrokecolor{currentstroke}%
\pgfsetdash{}{0pt}%
\pgfpathmoveto{\pgfqpoint{4.414621in}{3.473995in}}%
\pgfpathlineto{\pgfqpoint{4.428009in}{3.464267in}}%
\pgfpathlineto{\pgfqpoint{4.441401in}{3.454622in}}%
\pgfpathlineto{\pgfqpoint{4.454796in}{3.445060in}}%
\pgfpathlineto{\pgfqpoint{4.468194in}{3.435582in}}%
\pgfpathlineto{\pgfqpoint{4.475902in}{3.461713in}}%
\pgfpathlineto{\pgfqpoint{4.483612in}{3.488345in}}%
\pgfpathlineto{\pgfqpoint{4.491327in}{3.515486in}}%
\pgfpathlineto{\pgfqpoint{4.499046in}{3.543147in}}%
\pgfpathlineto{\pgfqpoint{4.485648in}{3.553150in}}%
\pgfpathlineto{\pgfqpoint{4.472253in}{3.563236in}}%
\pgfpathlineto{\pgfqpoint{4.458862in}{3.573406in}}%
\pgfpathlineto{\pgfqpoint{4.445473in}{3.583659in}}%
\pgfpathlineto{\pgfqpoint{4.437755in}{3.555465in}}%
\pgfpathlineto{\pgfqpoint{4.430040in}{3.527796in}}%
\pgfpathlineto{\pgfqpoint{4.422329in}{3.500643in}}%
\pgfpathlineto{\pgfqpoint{4.414621in}{3.473995in}}%
\pgfpathclose%
\pgfusepath{fill}%
\end{pgfscope}%
\begin{pgfscope}%
\pgfpathrectangle{\pgfqpoint{1.150000in}{0.150000in}}{\pgfqpoint{5.700000in}{5.700000in}}%
\pgfusepath{clip}%
\pgfsetbuttcap%
\pgfsetroundjoin%
\definecolor{currentfill}{rgb}{0.132268,0.655014,0.519661}%
\pgfsetfillcolor{currentfill}%
\pgfsetfillopacity{0.700000}%
\pgfsetlinewidth{0.000000pt}%
\definecolor{currentstroke}{rgb}{0.000000,0.000000,0.000000}%
\pgfsetstrokecolor{currentstroke}%
\pgfsetdash{}{0pt}%
\pgfpathmoveto{\pgfqpoint{4.346874in}{3.972682in}}%
\pgfpathlineto{\pgfqpoint{4.360239in}{3.960258in}}%
\pgfpathlineto{\pgfqpoint{4.373606in}{3.947924in}}%
\pgfpathlineto{\pgfqpoint{4.386975in}{3.935679in}}%
\pgfpathlineto{\pgfqpoint{4.400346in}{3.923524in}}%
\pgfpathlineto{\pgfqpoint{4.408103in}{3.958119in}}%
\pgfpathlineto{\pgfqpoint{4.415867in}{3.993364in}}%
\pgfpathlineto{\pgfqpoint{4.423637in}{4.029271in}}%
\pgfpathlineto{\pgfqpoint{4.410261in}{4.041875in}}%
\pgfpathlineto{\pgfqpoint{4.396887in}{4.054569in}}%
\pgfpathlineto{\pgfqpoint{4.383514in}{4.067353in}}%
\pgfpathlineto{\pgfqpoint{4.370144in}{4.080228in}}%
\pgfpathlineto{\pgfqpoint{4.362381in}{4.043715in}}%
\pgfpathlineto{\pgfqpoint{4.354625in}{4.007870in}}%
\pgfpathlineto{\pgfqpoint{4.346874in}{3.972682in}}%
\pgfpathclose%
\pgfusepath{fill}%
\end{pgfscope}%
\begin{pgfscope}%
\pgfpathrectangle{\pgfqpoint{1.150000in}{0.150000in}}{\pgfqpoint{5.700000in}{5.700000in}}%
\pgfusepath{clip}%
\pgfsetbuttcap%
\pgfsetroundjoin%
\definecolor{currentfill}{rgb}{0.204903,0.375746,0.553533}%
\pgfsetfillcolor{currentfill}%
\pgfsetfillopacity{0.700000}%
\pgfsetlinewidth{0.000000pt}%
\definecolor{currentstroke}{rgb}{0.000000,0.000000,0.000000}%
\pgfsetstrokecolor{currentstroke}%
\pgfsetdash{}{0pt}%
\pgfpathmoveto{\pgfqpoint{4.268751in}{3.224119in}}%
\pgfpathlineto{\pgfqpoint{4.282130in}{3.215496in}}%
\pgfpathlineto{\pgfqpoint{4.295513in}{3.206958in}}%
\pgfpathlineto{\pgfqpoint{4.308900in}{3.198503in}}%
\pgfpathlineto{\pgfqpoint{4.322291in}{3.190133in}}%
\pgfpathlineto{\pgfqpoint{4.329979in}{3.211459in}}%
\pgfpathlineto{\pgfqpoint{4.337667in}{3.233178in}}%
\pgfpathlineto{\pgfqpoint{4.345356in}{3.255301in}}%
\pgfpathlineto{\pgfqpoint{4.353046in}{3.277834in}}%
\pgfpathlineto{\pgfqpoint{4.339658in}{3.286663in}}%
\pgfpathlineto{\pgfqpoint{4.326274in}{3.295577in}}%
\pgfpathlineto{\pgfqpoint{4.312894in}{3.304574in}}%
\pgfpathlineto{\pgfqpoint{4.299517in}{3.313656in}}%
\pgfpathlineto{\pgfqpoint{4.291825in}{3.290656in}}%
\pgfpathlineto{\pgfqpoint{4.284133in}{3.268072in}}%
\pgfpathlineto{\pgfqpoint{4.276442in}{3.245896in}}%
\pgfpathlineto{\pgfqpoint{4.268751in}{3.224119in}}%
\pgfpathclose%
\pgfusepath{fill}%
\end{pgfscope}%
\begin{pgfscope}%
\pgfpathrectangle{\pgfqpoint{1.150000in}{0.150000in}}{\pgfqpoint{5.700000in}{5.700000in}}%
\pgfusepath{clip}%
\pgfsetbuttcap%
\pgfsetroundjoin%
\definecolor{currentfill}{rgb}{0.179019,0.433756,0.557430}%
\pgfsetfillcolor{currentfill}%
\pgfsetfillopacity{0.700000}%
\pgfsetlinewidth{0.000000pt}%
\definecolor{currentstroke}{rgb}{0.000000,0.000000,0.000000}%
\pgfsetstrokecolor{currentstroke}%
\pgfsetdash{}{0pt}%
\pgfpathmoveto{\pgfqpoint{4.383817in}{3.372259in}}%
\pgfpathlineto{\pgfqpoint{4.397207in}{3.363034in}}%
\pgfpathlineto{\pgfqpoint{4.410599in}{3.353891in}}%
\pgfpathlineto{\pgfqpoint{4.423996in}{3.344832in}}%
\pgfpathlineto{\pgfqpoint{4.437396in}{3.335855in}}%
\pgfpathlineto{\pgfqpoint{4.445092in}{3.360086in}}%
\pgfpathlineto{\pgfqpoint{4.452790in}{3.384778in}}%
\pgfpathlineto{\pgfqpoint{4.460490in}{3.409940in}}%
\pgfpathlineto{\pgfqpoint{4.468194in}{3.435582in}}%
\pgfpathlineto{\pgfqpoint{4.454796in}{3.445060in}}%
\pgfpathlineto{\pgfqpoint{4.441401in}{3.454622in}}%
\pgfpathlineto{\pgfqpoint{4.428009in}{3.464267in}}%
\pgfpathlineto{\pgfqpoint{4.414621in}{3.473995in}}%
\pgfpathlineto{\pgfqpoint{4.406916in}{3.447842in}}%
\pgfpathlineto{\pgfqpoint{4.399214in}{3.422175in}}%
\pgfpathlineto{\pgfqpoint{4.391515in}{3.396984in}}%
\pgfpathlineto{\pgfqpoint{4.383817in}{3.372259in}}%
\pgfpathclose%
\pgfusepath{fill}%
\end{pgfscope}%
\begin{pgfscope}%
\pgfpathrectangle{\pgfqpoint{1.150000in}{0.150000in}}{\pgfqpoint{5.700000in}{5.700000in}}%
\pgfusepath{clip}%
\pgfsetbuttcap%
\pgfsetroundjoin%
\definecolor{currentfill}{rgb}{0.253935,0.265254,0.529983}%
\pgfsetfillcolor{currentfill}%
\pgfsetfillopacity{0.700000}%
\pgfsetlinewidth{0.000000pt}%
\definecolor{currentstroke}{rgb}{0.000000,0.000000,0.000000}%
\pgfsetstrokecolor{currentstroke}%
\pgfsetdash{}{0pt}%
\pgfpathmoveto{\pgfqpoint{3.625464in}{2.983819in}}%
\pgfpathlineto{\pgfqpoint{3.638767in}{2.974879in}}%
\pgfpathlineto{\pgfqpoint{3.652073in}{2.966040in}}%
\pgfpathlineto{\pgfqpoint{3.665382in}{2.957301in}}%
\pgfpathlineto{\pgfqpoint{3.678692in}{2.948662in}}%
\pgfpathlineto{\pgfqpoint{3.686488in}{2.964480in}}%
\pgfpathlineto{\pgfqpoint{3.694278in}{2.980534in}}%
\pgfpathlineto{\pgfqpoint{3.702064in}{2.996829in}}%
\pgfpathlineto{\pgfqpoint{3.709845in}{3.013372in}}%
\pgfpathlineto{\pgfqpoint{3.696538in}{3.022328in}}%
\pgfpathlineto{\pgfqpoint{3.683234in}{3.031383in}}%
\pgfpathlineto{\pgfqpoint{3.669932in}{3.040538in}}%
\pgfpathlineto{\pgfqpoint{3.656632in}{3.049795in}}%
\pgfpathlineto{\pgfqpoint{3.648848in}{3.032928in}}%
\pgfpathlineto{\pgfqpoint{3.641058in}{3.016313in}}%
\pgfpathlineto{\pgfqpoint{3.633264in}{2.999946in}}%
\pgfpathlineto{\pgfqpoint{3.625464in}{2.983819in}}%
\pgfpathclose%
\pgfusepath{fill}%
\end{pgfscope}%
\begin{pgfscope}%
\pgfpathrectangle{\pgfqpoint{1.150000in}{0.150000in}}{\pgfqpoint{5.700000in}{5.700000in}}%
\pgfusepath{clip}%
\pgfsetbuttcap%
\pgfsetroundjoin%
\definecolor{currentfill}{rgb}{0.124395,0.578002,0.548287}%
\pgfsetfillcolor{currentfill}%
\pgfsetfillopacity{0.700000}%
\pgfsetlinewidth{0.000000pt}%
\definecolor{currentstroke}{rgb}{0.000000,0.000000,0.000000}%
\pgfsetstrokecolor{currentstroke}%
\pgfsetdash{}{0pt}%
\pgfpathmoveto{\pgfqpoint{4.422862in}{3.745965in}}%
\pgfpathlineto{\pgfqpoint{4.436240in}{3.734822in}}%
\pgfpathlineto{\pgfqpoint{4.449621in}{3.723766in}}%
\pgfpathlineto{\pgfqpoint{4.463006in}{3.712794in}}%
\pgfpathlineto{\pgfqpoint{4.476392in}{3.701907in}}%
\pgfpathlineto{\pgfqpoint{4.484135in}{3.732891in}}%
\pgfpathlineto{\pgfqpoint{4.491883in}{3.764465in}}%
\pgfpathlineto{\pgfqpoint{4.499637in}{3.796641in}}%
\pgfpathlineto{\pgfqpoint{4.507398in}{3.829431in}}%
\pgfpathlineto{\pgfqpoint{4.494008in}{3.840890in}}%
\pgfpathlineto{\pgfqpoint{4.480620in}{3.852435in}}%
\pgfpathlineto{\pgfqpoint{4.467235in}{3.864066in}}%
\pgfpathlineto{\pgfqpoint{4.453853in}{3.875783in}}%
\pgfpathlineto{\pgfqpoint{4.446096in}{3.842411in}}%
\pgfpathlineto{\pgfqpoint{4.438346in}{3.809658in}}%
\pgfpathlineto{\pgfqpoint{4.430601in}{3.777513in}}%
\pgfpathlineto{\pgfqpoint{4.422862in}{3.745965in}}%
\pgfpathclose%
\pgfusepath{fill}%
\end{pgfscope}%
\begin{pgfscope}%
\pgfpathrectangle{\pgfqpoint{1.150000in}{0.150000in}}{\pgfqpoint{5.700000in}{5.700000in}}%
\pgfusepath{clip}%
\pgfsetbuttcap%
\pgfsetroundjoin%
\definecolor{currentfill}{rgb}{0.246811,0.283237,0.535941}%
\pgfsetfillcolor{currentfill}%
\pgfsetfillopacity{0.700000}%
\pgfsetlinewidth{0.000000pt}%
\definecolor{currentstroke}{rgb}{0.000000,0.000000,0.000000}%
\pgfsetstrokecolor{currentstroke}%
\pgfsetdash{}{0pt}%
\pgfpathmoveto{\pgfqpoint{3.847442in}{3.011325in}}%
\pgfpathlineto{\pgfqpoint{3.860770in}{3.002903in}}%
\pgfpathlineto{\pgfqpoint{3.874101in}{2.994575in}}%
\pgfpathlineto{\pgfqpoint{3.887435in}{2.986340in}}%
\pgfpathlineto{\pgfqpoint{3.900773in}{2.978198in}}%
\pgfpathlineto{\pgfqpoint{3.908520in}{2.995025in}}%
\pgfpathlineto{\pgfqpoint{3.916263in}{3.012121in}}%
\pgfpathlineto{\pgfqpoint{3.924002in}{3.029494in}}%
\pgfpathlineto{\pgfqpoint{3.931738in}{3.047150in}}%
\pgfpathlineto{\pgfqpoint{3.918405in}{3.055648in}}%
\pgfpathlineto{\pgfqpoint{3.905074in}{3.064239in}}%
\pgfpathlineto{\pgfqpoint{3.891747in}{3.072923in}}%
\pgfpathlineto{\pgfqpoint{3.878423in}{3.081701in}}%
\pgfpathlineto{\pgfqpoint{3.870683in}{3.063682in}}%
\pgfpathlineto{\pgfqpoint{3.862940in}{3.045950in}}%
\pgfpathlineto{\pgfqpoint{3.855193in}{3.028500in}}%
\pgfpathlineto{\pgfqpoint{3.847442in}{3.011325in}}%
\pgfpathclose%
\pgfusepath{fill}%
\end{pgfscope}%
\begin{pgfscope}%
\pgfpathrectangle{\pgfqpoint{1.150000in}{0.150000in}}{\pgfqpoint{5.700000in}{5.700000in}}%
\pgfusepath{clip}%
\pgfsetbuttcap%
\pgfsetroundjoin%
\definecolor{currentfill}{rgb}{0.146180,0.515413,0.556823}%
\pgfsetfillcolor{currentfill}%
\pgfsetfillopacity{0.700000}%
\pgfsetlinewidth{0.000000pt}%
\definecolor{currentstroke}{rgb}{0.000000,0.000000,0.000000}%
\pgfsetstrokecolor{currentstroke}%
\pgfsetdash{}{0pt}%
\pgfpathmoveto{\pgfqpoint{4.445473in}{3.583659in}}%
\pgfpathlineto{\pgfqpoint{4.458862in}{3.573406in}}%
\pgfpathlineto{\pgfqpoint{4.472253in}{3.563236in}}%
\pgfpathlineto{\pgfqpoint{4.485648in}{3.553150in}}%
\pgfpathlineto{\pgfqpoint{4.499046in}{3.543147in}}%
\pgfpathlineto{\pgfqpoint{4.506769in}{3.571339in}}%
\pgfpathlineto{\pgfqpoint{4.514498in}{3.600071in}}%
\pgfpathlineto{\pgfqpoint{4.522231in}{3.629356in}}%
\pgfpathlineto{\pgfqpoint{4.529970in}{3.659203in}}%
\pgfpathlineto{\pgfqpoint{4.516571in}{3.669754in}}%
\pgfpathlineto{\pgfqpoint{4.503175in}{3.680388in}}%
\pgfpathlineto{\pgfqpoint{4.489782in}{3.691105in}}%
\pgfpathlineto{\pgfqpoint{4.476392in}{3.701907in}}%
\pgfpathlineto{\pgfqpoint{4.468655in}{3.671504in}}%
\pgfpathlineto{\pgfqpoint{4.460923in}{3.641668in}}%
\pgfpathlineto{\pgfqpoint{4.453196in}{3.612390in}}%
\pgfpathlineto{\pgfqpoint{4.445473in}{3.583659in}}%
\pgfpathclose%
\pgfusepath{fill}%
\end{pgfscope}%
\begin{pgfscope}%
\pgfpathrectangle{\pgfqpoint{1.150000in}{0.150000in}}{\pgfqpoint{5.700000in}{5.700000in}}%
\pgfusepath{clip}%
\pgfsetbuttcap%
\pgfsetroundjoin%
\definecolor{currentfill}{rgb}{0.239346,0.300855,0.540844}%
\pgfsetfillcolor{currentfill}%
\pgfsetfillopacity{0.700000}%
\pgfsetlinewidth{0.000000pt}%
\definecolor{currentstroke}{rgb}{0.000000,0.000000,0.000000}%
\pgfsetstrokecolor{currentstroke}%
\pgfsetdash{}{0pt}%
\pgfpathmoveto{\pgfqpoint{3.074400in}{3.064022in}}%
\pgfpathlineto{\pgfqpoint{3.087688in}{3.052498in}}%
\pgfpathlineto{\pgfqpoint{3.100976in}{3.041102in}}%
\pgfpathlineto{\pgfqpoint{3.114263in}{3.029834in}}%
\pgfpathlineto{\pgfqpoint{3.127550in}{3.018692in}}%
\pgfpathlineto{\pgfqpoint{3.135480in}{3.033282in}}%
\pgfpathlineto{\pgfqpoint{3.143403in}{3.048067in}}%
\pgfpathlineto{\pgfqpoint{3.151319in}{3.063052in}}%
\pgfpathlineto{\pgfqpoint{3.159228in}{3.078242in}}%
\pgfpathlineto{\pgfqpoint{3.145946in}{3.089620in}}%
\pgfpathlineto{\pgfqpoint{3.132663in}{3.101126in}}%
\pgfpathlineto{\pgfqpoint{3.119381in}{3.112758in}}%
\pgfpathlineto{\pgfqpoint{3.106098in}{3.124520in}}%
\pgfpathlineto{\pgfqpoint{3.098184in}{3.109086in}}%
\pgfpathlineto{\pgfqpoint{3.090264in}{3.093861in}}%
\pgfpathlineto{\pgfqpoint{3.082336in}{3.078841in}}%
\pgfpathlineto{\pgfqpoint{3.074400in}{3.064022in}}%
\pgfpathclose%
\pgfusepath{fill}%
\end{pgfscope}%
\begin{pgfscope}%
\pgfpathrectangle{\pgfqpoint{1.150000in}{0.150000in}}{\pgfqpoint{5.700000in}{5.700000in}}%
\pgfusepath{clip}%
\pgfsetbuttcap%
\pgfsetroundjoin%
\definecolor{currentfill}{rgb}{0.916242,0.896091,0.100717}%
\pgfsetfillcolor{currentfill}%
\pgfsetfillopacity{0.700000}%
\pgfsetlinewidth{0.000000pt}%
\definecolor{currentstroke}{rgb}{0.000000,0.000000,0.000000}%
\pgfsetstrokecolor{currentstroke}%
\pgfsetdash{}{0pt}%
\pgfpathmoveto{\pgfqpoint{3.545811in}{4.908327in}}%
\pgfpathlineto{\pgfqpoint{3.559198in}{4.889162in}}%
\pgfpathlineto{\pgfqpoint{3.572582in}{4.870139in}}%
\pgfpathlineto{\pgfqpoint{3.585963in}{4.851256in}}%
\pgfpathlineto{\pgfqpoint{3.599342in}{4.832513in}}%
\pgfpathlineto{\pgfqpoint{3.606903in}{4.876279in}}%
\pgfpathlineto{\pgfqpoint{3.614463in}{4.920802in}}%
\pgfpathlineto{\pgfqpoint{3.622021in}{4.966092in}}%
\pgfpathlineto{\pgfqpoint{3.608628in}{4.985320in}}%
\pgfpathlineto{\pgfqpoint{3.595231in}{5.004687in}}%
\pgfpathlineto{\pgfqpoint{3.581831in}{5.024196in}}%
\pgfpathlineto{\pgfqpoint{3.568428in}{5.043847in}}%
\pgfpathlineto{\pgfqpoint{3.560891in}{4.997902in}}%
\pgfpathlineto{\pgfqpoint{3.553352in}{4.952733in}}%
\pgfpathlineto{\pgfqpoint{3.545811in}{4.908327in}}%
\pgfpathclose%
\pgfusepath{fill}%
\end{pgfscope}%
\begin{pgfscope}%
\pgfpathrectangle{\pgfqpoint{1.150000in}{0.150000in}}{\pgfqpoint{5.700000in}{5.700000in}}%
\pgfusepath{clip}%
\pgfsetbuttcap%
\pgfsetroundjoin%
\definecolor{currentfill}{rgb}{0.194100,0.399323,0.555565}%
\pgfsetfillcolor{currentfill}%
\pgfsetfillopacity{0.700000}%
\pgfsetlinewidth{0.000000pt}%
\definecolor{currentstroke}{rgb}{0.000000,0.000000,0.000000}%
\pgfsetstrokecolor{currentstroke}%
\pgfsetdash{}{0pt}%
\pgfpathmoveto{\pgfqpoint{4.353046in}{3.277834in}}%
\pgfpathlineto{\pgfqpoint{4.366437in}{3.269089in}}%
\pgfpathlineto{\pgfqpoint{4.379832in}{3.260426in}}%
\pgfpathlineto{\pgfqpoint{4.393231in}{3.251846in}}%
\pgfpathlineto{\pgfqpoint{4.406634in}{3.243349in}}%
\pgfpathlineto{\pgfqpoint{4.414322in}{3.265831in}}%
\pgfpathlineto{\pgfqpoint{4.422012in}{3.288736in}}%
\pgfpathlineto{\pgfqpoint{4.429703in}{3.312075in}}%
\pgfpathlineto{\pgfqpoint{4.437396in}{3.335855in}}%
\pgfpathlineto{\pgfqpoint{4.423996in}{3.344832in}}%
\pgfpathlineto{\pgfqpoint{4.410599in}{3.353891in}}%
\pgfpathlineto{\pgfqpoint{4.397207in}{3.363034in}}%
\pgfpathlineto{\pgfqpoint{4.383817in}{3.372259in}}%
\pgfpathlineto{\pgfqpoint{4.376122in}{3.347991in}}%
\pgfpathlineto{\pgfqpoint{4.368428in}{3.324171in}}%
\pgfpathlineto{\pgfqpoint{4.360736in}{3.300788in}}%
\pgfpathlineto{\pgfqpoint{4.353046in}{3.277834in}}%
\pgfpathclose%
\pgfusepath{fill}%
\end{pgfscope}%
\begin{pgfscope}%
\pgfpathrectangle{\pgfqpoint{1.150000in}{0.150000in}}{\pgfqpoint{5.700000in}{5.700000in}}%
\pgfusepath{clip}%
\pgfsetbuttcap%
\pgfsetroundjoin%
\definecolor{currentfill}{rgb}{0.124780,0.640461,0.527068}%
\pgfsetfillcolor{currentfill}%
\pgfsetfillopacity{0.700000}%
\pgfsetlinewidth{0.000000pt}%
\definecolor{currentstroke}{rgb}{0.000000,0.000000,0.000000}%
\pgfsetstrokecolor{currentstroke}%
\pgfsetdash{}{0pt}%
\pgfpathmoveto{\pgfqpoint{4.400346in}{3.923524in}}%
\pgfpathlineto{\pgfqpoint{4.413719in}{3.911457in}}%
\pgfpathlineto{\pgfqpoint{4.427095in}{3.899478in}}%
\pgfpathlineto{\pgfqpoint{4.440472in}{3.887587in}}%
\pgfpathlineto{\pgfqpoint{4.453853in}{3.875783in}}%
\pgfpathlineto{\pgfqpoint{4.461615in}{3.909787in}}%
\pgfpathlineto{\pgfqpoint{4.469385in}{3.944435in}}%
\pgfpathlineto{\pgfqpoint{4.477161in}{3.979738in}}%
\pgfpathlineto{\pgfqpoint{4.463777in}{3.991989in}}%
\pgfpathlineto{\pgfqpoint{4.450395in}{4.004328in}}%
\pgfpathlineto{\pgfqpoint{4.437015in}{4.016755in}}%
\pgfpathlineto{\pgfqpoint{4.423637in}{4.029271in}}%
\pgfpathlineto{\pgfqpoint{4.415867in}{3.993364in}}%
\pgfpathlineto{\pgfqpoint{4.408103in}{3.958119in}}%
\pgfpathlineto{\pgfqpoint{4.400346in}{3.923524in}}%
\pgfpathclose%
\pgfusepath{fill}%
\end{pgfscope}%
\begin{pgfscope}%
\pgfpathrectangle{\pgfqpoint{1.150000in}{0.150000in}}{\pgfqpoint{5.700000in}{5.700000in}}%
\pgfusepath{clip}%
\pgfsetbuttcap%
\pgfsetroundjoin%
\definecolor{currentfill}{rgb}{0.253935,0.265254,0.529983}%
\pgfsetfillcolor{currentfill}%
\pgfsetfillopacity{0.700000}%
\pgfsetlinewidth{0.000000pt}%
\definecolor{currentstroke}{rgb}{0.000000,0.000000,0.000000}%
\pgfsetstrokecolor{currentstroke}%
\pgfsetdash{}{0pt}%
\pgfpathmoveto{\pgfqpoint{3.763098in}{2.978535in}}%
\pgfpathlineto{\pgfqpoint{3.776418in}{2.970069in}}%
\pgfpathlineto{\pgfqpoint{3.789741in}{2.961699in}}%
\pgfpathlineto{\pgfqpoint{3.803067in}{2.953424in}}%
\pgfpathlineto{\pgfqpoint{3.816397in}{2.945244in}}%
\pgfpathlineto{\pgfqpoint{3.824164in}{2.961384in}}%
\pgfpathlineto{\pgfqpoint{3.831928in}{2.977773in}}%
\pgfpathlineto{\pgfqpoint{3.839687in}{2.994418in}}%
\pgfpathlineto{\pgfqpoint{3.847442in}{3.011325in}}%
\pgfpathlineto{\pgfqpoint{3.834117in}{3.019841in}}%
\pgfpathlineto{\pgfqpoint{3.820795in}{3.028452in}}%
\pgfpathlineto{\pgfqpoint{3.807476in}{3.037158in}}%
\pgfpathlineto{\pgfqpoint{3.794159in}{3.045960in}}%
\pgfpathlineto{\pgfqpoint{3.786400in}{3.028710in}}%
\pgfpathlineto{\pgfqpoint{3.778637in}{3.011726in}}%
\pgfpathlineto{\pgfqpoint{3.770870in}{2.995004in}}%
\pgfpathlineto{\pgfqpoint{3.763098in}{2.978535in}}%
\pgfpathclose%
\pgfusepath{fill}%
\end{pgfscope}%
\begin{pgfscope}%
\pgfpathrectangle{\pgfqpoint{1.150000in}{0.150000in}}{\pgfqpoint{5.700000in}{5.700000in}}%
\pgfusepath{clip}%
\pgfsetbuttcap%
\pgfsetroundjoin%
\definecolor{currentfill}{rgb}{0.255645,0.260703,0.528312}%
\pgfsetfillcolor{currentfill}%
\pgfsetfillopacity{0.700000}%
\pgfsetlinewidth{0.000000pt}%
\definecolor{currentstroke}{rgb}{0.000000,0.000000,0.000000}%
\pgfsetstrokecolor{currentstroke}%
\pgfsetdash{}{0pt}%
\pgfpathmoveto{\pgfqpoint{3.403287in}{2.971355in}}%
\pgfpathlineto{\pgfqpoint{3.416576in}{2.961720in}}%
\pgfpathlineto{\pgfqpoint{3.429866in}{2.952195in}}%
\pgfpathlineto{\pgfqpoint{3.443158in}{2.942779in}}%
\pgfpathlineto{\pgfqpoint{3.456452in}{2.933472in}}%
\pgfpathlineto{\pgfqpoint{3.464305in}{2.948406in}}%
\pgfpathlineto{\pgfqpoint{3.472152in}{2.963549in}}%
\pgfpathlineto{\pgfqpoint{3.479994in}{2.978904in}}%
\pgfpathlineto{\pgfqpoint{3.487829in}{2.994478in}}%
\pgfpathlineto{\pgfqpoint{3.474540in}{3.004062in}}%
\pgfpathlineto{\pgfqpoint{3.461252in}{3.013754in}}%
\pgfpathlineto{\pgfqpoint{3.447966in}{3.023555in}}%
\pgfpathlineto{\pgfqpoint{3.434682in}{3.033466in}}%
\pgfpathlineto{\pgfqpoint{3.426842in}{3.017609in}}%
\pgfpathlineto{\pgfqpoint{3.418996in}{3.001974in}}%
\pgfpathlineto{\pgfqpoint{3.411145in}{2.986558in}}%
\pgfpathlineto{\pgfqpoint{3.403287in}{2.971355in}}%
\pgfpathclose%
\pgfusepath{fill}%
\end{pgfscope}%
\begin{pgfscope}%
\pgfpathrectangle{\pgfqpoint{1.150000in}{0.150000in}}{\pgfqpoint{5.700000in}{5.700000in}}%
\pgfusepath{clip}%
\pgfsetbuttcap%
\pgfsetroundjoin%
\definecolor{currentfill}{rgb}{0.252194,0.269783,0.531579}%
\pgfsetfillcolor{currentfill}%
\pgfsetfillopacity{0.700000}%
\pgfsetlinewidth{0.000000pt}%
\definecolor{currentstroke}{rgb}{0.000000,0.000000,0.000000}%
\pgfsetstrokecolor{currentstroke}%
\pgfsetdash{}{0pt}%
\pgfpathmoveto{\pgfqpoint{3.265488in}{2.991643in}}%
\pgfpathlineto{\pgfqpoint{3.278773in}{2.981357in}}%
\pgfpathlineto{\pgfqpoint{3.292058in}{2.971188in}}%
\pgfpathlineto{\pgfqpoint{3.305345in}{2.961134in}}%
\pgfpathlineto{\pgfqpoint{3.318632in}{2.951195in}}%
\pgfpathlineto{\pgfqpoint{3.326520in}{2.965846in}}%
\pgfpathlineto{\pgfqpoint{3.334401in}{2.980694in}}%
\pgfpathlineto{\pgfqpoint{3.342276in}{2.995746in}}%
\pgfpathlineto{\pgfqpoint{3.350145in}{3.011006in}}%
\pgfpathlineto{\pgfqpoint{3.336862in}{3.021201in}}%
\pgfpathlineto{\pgfqpoint{3.323580in}{3.031512in}}%
\pgfpathlineto{\pgfqpoint{3.310299in}{3.041938in}}%
\pgfpathlineto{\pgfqpoint{3.297019in}{3.052480in}}%
\pgfpathlineto{\pgfqpoint{3.289146in}{3.036956in}}%
\pgfpathlineto{\pgfqpoint{3.281267in}{3.021645in}}%
\pgfpathlineto{\pgfqpoint{3.273381in}{3.006543in}}%
\pgfpathlineto{\pgfqpoint{3.265488in}{2.991643in}}%
\pgfpathclose%
\pgfusepath{fill}%
\end{pgfscope}%
\begin{pgfscope}%
\pgfpathrectangle{\pgfqpoint{1.150000in}{0.150000in}}{\pgfqpoint{5.700000in}{5.700000in}}%
\pgfusepath{clip}%
\pgfsetbuttcap%
\pgfsetroundjoin%
\definecolor{currentfill}{rgb}{0.229739,0.322361,0.545706}%
\pgfsetfillcolor{currentfill}%
\pgfsetfillopacity{0.700000}%
\pgfsetlinewidth{0.000000pt}%
\definecolor{currentstroke}{rgb}{0.000000,0.000000,0.000000}%
\pgfsetstrokecolor{currentstroke}%
\pgfsetdash{}{0pt}%
\pgfpathmoveto{\pgfqpoint{4.153695in}{3.095206in}}%
\pgfpathlineto{\pgfqpoint{4.167067in}{3.087096in}}%
\pgfpathlineto{\pgfqpoint{4.180442in}{3.079072in}}%
\pgfpathlineto{\pgfqpoint{4.193821in}{3.071133in}}%
\pgfpathlineto{\pgfqpoint{4.207204in}{3.063279in}}%
\pgfpathlineto{\pgfqpoint{4.214901in}{3.082159in}}%
\pgfpathlineto{\pgfqpoint{4.222596in}{3.101374in}}%
\pgfpathlineto{\pgfqpoint{4.230290in}{3.120930in}}%
\pgfpathlineto{\pgfqpoint{4.237984in}{3.140836in}}%
\pgfpathlineto{\pgfqpoint{4.224605in}{3.149106in}}%
\pgfpathlineto{\pgfqpoint{4.211230in}{3.157462in}}%
\pgfpathlineto{\pgfqpoint{4.197859in}{3.165903in}}%
\pgfpathlineto{\pgfqpoint{4.184491in}{3.174431in}}%
\pgfpathlineto{\pgfqpoint{4.176794in}{3.154100in}}%
\pgfpathlineto{\pgfqpoint{4.169096in}{3.134124in}}%
\pgfpathlineto{\pgfqpoint{4.161396in}{3.114496in}}%
\pgfpathlineto{\pgfqpoint{4.153695in}{3.095206in}}%
\pgfpathclose%
\pgfusepath{fill}%
\end{pgfscope}%
\begin{pgfscope}%
\pgfpathrectangle{\pgfqpoint{1.150000in}{0.150000in}}{\pgfqpoint{5.700000in}{5.700000in}}%
\pgfusepath{clip}%
\pgfsetbuttcap%
\pgfsetroundjoin%
\definecolor{currentfill}{rgb}{0.257322,0.256130,0.526563}%
\pgfsetfillcolor{currentfill}%
\pgfsetfillopacity{0.700000}%
\pgfsetlinewidth{0.000000pt}%
\definecolor{currentstroke}{rgb}{0.000000,0.000000,0.000000}%
\pgfsetstrokecolor{currentstroke}%
\pgfsetdash{}{0pt}%
\pgfpathmoveto{\pgfqpoint{3.541004in}{2.957208in}}%
\pgfpathlineto{\pgfqpoint{3.554303in}{2.948153in}}%
\pgfpathlineto{\pgfqpoint{3.567604in}{2.939202in}}%
\pgfpathlineto{\pgfqpoint{3.580907in}{2.930354in}}%
\pgfpathlineto{\pgfqpoint{3.594213in}{2.921608in}}%
\pgfpathlineto{\pgfqpoint{3.602034in}{2.936828in}}%
\pgfpathlineto{\pgfqpoint{3.609849in}{2.952266in}}%
\pgfpathlineto{\pgfqpoint{3.617659in}{2.967928in}}%
\pgfpathlineto{\pgfqpoint{3.625464in}{2.983819in}}%
\pgfpathlineto{\pgfqpoint{3.612163in}{2.992861in}}%
\pgfpathlineto{\pgfqpoint{3.598864in}{3.002005in}}%
\pgfpathlineto{\pgfqpoint{3.585568in}{3.011252in}}%
\pgfpathlineto{\pgfqpoint{3.572273in}{3.020603in}}%
\pgfpathlineto{\pgfqpoint{3.564464in}{3.004408in}}%
\pgfpathlineto{\pgfqpoint{3.556650in}{2.988447in}}%
\pgfpathlineto{\pgfqpoint{3.548830in}{2.972716in}}%
\pgfpathlineto{\pgfqpoint{3.541004in}{2.957208in}}%
\pgfpathclose%
\pgfusepath{fill}%
\end{pgfscope}%
\begin{pgfscope}%
\pgfpathrectangle{\pgfqpoint{1.150000in}{0.150000in}}{\pgfqpoint{5.700000in}{5.700000in}}%
\pgfusepath{clip}%
\pgfsetbuttcap%
\pgfsetroundjoin%
\definecolor{currentfill}{rgb}{0.128729,0.563265,0.551229}%
\pgfsetfillcolor{currentfill}%
\pgfsetfillopacity{0.700000}%
\pgfsetlinewidth{0.000000pt}%
\definecolor{currentstroke}{rgb}{0.000000,0.000000,0.000000}%
\pgfsetstrokecolor{currentstroke}%
\pgfsetdash{}{0pt}%
\pgfpathmoveto{\pgfqpoint{4.476392in}{3.701907in}}%
\pgfpathlineto{\pgfqpoint{4.489782in}{3.691105in}}%
\pgfpathlineto{\pgfqpoint{4.503175in}{3.680388in}}%
\pgfpathlineto{\pgfqpoint{4.516571in}{3.669754in}}%
\pgfpathlineto{\pgfqpoint{4.529970in}{3.659203in}}%
\pgfpathlineto{\pgfqpoint{4.537714in}{3.689623in}}%
\pgfpathlineto{\pgfqpoint{4.545465in}{3.720628in}}%
\pgfpathlineto{\pgfqpoint{4.553222in}{3.752229in}}%
\pgfpathlineto{\pgfqpoint{4.560986in}{3.784437in}}%
\pgfpathlineto{\pgfqpoint{4.547585in}{3.795560in}}%
\pgfpathlineto{\pgfqpoint{4.534186in}{3.806766in}}%
\pgfpathlineto{\pgfqpoint{4.520791in}{3.818056in}}%
\pgfpathlineto{\pgfqpoint{4.507398in}{3.829431in}}%
\pgfpathlineto{\pgfqpoint{4.499637in}{3.796641in}}%
\pgfpathlineto{\pgfqpoint{4.491883in}{3.764465in}}%
\pgfpathlineto{\pgfqpoint{4.484135in}{3.732891in}}%
\pgfpathlineto{\pgfqpoint{4.476392in}{3.701907in}}%
\pgfpathclose%
\pgfusepath{fill}%
\end{pgfscope}%
\begin{pgfscope}%
\pgfpathrectangle{\pgfqpoint{1.150000in}{0.150000in}}{\pgfqpoint{5.700000in}{5.700000in}}%
\pgfusepath{clip}%
\pgfsetbuttcap%
\pgfsetroundjoin%
\definecolor{currentfill}{rgb}{0.237441,0.305202,0.541921}%
\pgfsetfillcolor{currentfill}%
\pgfsetfillopacity{0.700000}%
\pgfsetlinewidth{0.000000pt}%
\definecolor{currentstroke}{rgb}{0.000000,0.000000,0.000000}%
\pgfsetstrokecolor{currentstroke}%
\pgfsetdash{}{0pt}%
\pgfpathmoveto{\pgfqpoint{4.069408in}{3.053019in}}%
\pgfpathlineto{\pgfqpoint{4.082768in}{3.044956in}}%
\pgfpathlineto{\pgfqpoint{4.096133in}{3.036981in}}%
\pgfpathlineto{\pgfqpoint{4.109501in}{3.029094in}}%
\pgfpathlineto{\pgfqpoint{4.122872in}{3.021293in}}%
\pgfpathlineto{\pgfqpoint{4.130581in}{3.039300in}}%
\pgfpathlineto{\pgfqpoint{4.138288in}{3.057616in}}%
\pgfpathlineto{\pgfqpoint{4.145993in}{3.076249in}}%
\pgfpathlineto{\pgfqpoint{4.153695in}{3.095206in}}%
\pgfpathlineto{\pgfqpoint{4.140328in}{3.103403in}}%
\pgfpathlineto{\pgfqpoint{4.126964in}{3.111687in}}%
\pgfpathlineto{\pgfqpoint{4.113604in}{3.120059in}}%
\pgfpathlineto{\pgfqpoint{4.100247in}{3.128518in}}%
\pgfpathlineto{\pgfqpoint{4.092541in}{3.109157in}}%
\pgfpathlineto{\pgfqpoint{4.084832in}{3.090124in}}%
\pgfpathlineto{\pgfqpoint{4.077121in}{3.071414in}}%
\pgfpathlineto{\pgfqpoint{4.069408in}{3.053019in}}%
\pgfpathclose%
\pgfusepath{fill}%
\end{pgfscope}%
\begin{pgfscope}%
\pgfpathrectangle{\pgfqpoint{1.150000in}{0.150000in}}{\pgfqpoint{5.700000in}{5.700000in}}%
\pgfusepath{clip}%
\pgfsetbuttcap%
\pgfsetroundjoin%
\definecolor{currentfill}{rgb}{0.220057,0.343307,0.549413}%
\pgfsetfillcolor{currentfill}%
\pgfsetfillopacity{0.700000}%
\pgfsetlinewidth{0.000000pt}%
\definecolor{currentstroke}{rgb}{0.000000,0.000000,0.000000}%
\pgfsetstrokecolor{currentstroke}%
\pgfsetdash{}{0pt}%
\pgfpathmoveto{\pgfqpoint{4.237984in}{3.140836in}}%
\pgfpathlineto{\pgfqpoint{4.251367in}{3.132651in}}%
\pgfpathlineto{\pgfqpoint{4.264753in}{3.124550in}}%
\pgfpathlineto{\pgfqpoint{4.278144in}{3.116533in}}%
\pgfpathlineto{\pgfqpoint{4.291539in}{3.108600in}}%
\pgfpathlineto{\pgfqpoint{4.299227in}{3.128434in}}%
\pgfpathlineto{\pgfqpoint{4.306915in}{3.148629in}}%
\pgfpathlineto{\pgfqpoint{4.314603in}{3.169192in}}%
\pgfpathlineto{\pgfqpoint{4.322291in}{3.190133in}}%
\pgfpathlineto{\pgfqpoint{4.308900in}{3.198503in}}%
\pgfpathlineto{\pgfqpoint{4.295513in}{3.206958in}}%
\pgfpathlineto{\pgfqpoint{4.282130in}{3.215496in}}%
\pgfpathlineto{\pgfqpoint{4.268751in}{3.224119in}}%
\pgfpathlineto{\pgfqpoint{4.261060in}{3.202733in}}%
\pgfpathlineto{\pgfqpoint{4.253368in}{3.181729in}}%
\pgfpathlineto{\pgfqpoint{4.245676in}{3.161100in}}%
\pgfpathlineto{\pgfqpoint{4.237984in}{3.140836in}}%
\pgfpathclose%
\pgfusepath{fill}%
\end{pgfscope}%
\begin{pgfscope}%
\pgfpathrectangle{\pgfqpoint{1.150000in}{0.150000in}}{\pgfqpoint{5.700000in}{5.700000in}}%
\pgfusepath{clip}%
\pgfsetbuttcap%
\pgfsetroundjoin%
\definecolor{currentfill}{rgb}{0.166617,0.463708,0.558119}%
\pgfsetfillcolor{currentfill}%
\pgfsetfillopacity{0.700000}%
\pgfsetlinewidth{0.000000pt}%
\definecolor{currentstroke}{rgb}{0.000000,0.000000,0.000000}%
\pgfsetstrokecolor{currentstroke}%
\pgfsetdash{}{0pt}%
\pgfpathmoveto{\pgfqpoint{4.468194in}{3.435582in}}%
\pgfpathlineto{\pgfqpoint{4.481597in}{3.426185in}}%
\pgfpathlineto{\pgfqpoint{4.495002in}{3.416871in}}%
\pgfpathlineto{\pgfqpoint{4.508412in}{3.407638in}}%
\pgfpathlineto{\pgfqpoint{4.521825in}{3.398486in}}%
\pgfpathlineto{\pgfqpoint{4.529531in}{3.424103in}}%
\pgfpathlineto{\pgfqpoint{4.537241in}{3.450213in}}%
\pgfpathlineto{\pgfqpoint{4.544954in}{3.476828in}}%
\pgfpathlineto{\pgfqpoint{4.552673in}{3.503957in}}%
\pgfpathlineto{\pgfqpoint{4.539261in}{3.513632in}}%
\pgfpathlineto{\pgfqpoint{4.525852in}{3.523388in}}%
\pgfpathlineto{\pgfqpoint{4.512448in}{3.533227in}}%
\pgfpathlineto{\pgfqpoint{4.499046in}{3.543147in}}%
\pgfpathlineto{\pgfqpoint{4.491327in}{3.515486in}}%
\pgfpathlineto{\pgfqpoint{4.483612in}{3.488345in}}%
\pgfpathlineto{\pgfqpoint{4.475902in}{3.461713in}}%
\pgfpathlineto{\pgfqpoint{4.468194in}{3.435582in}}%
\pgfpathclose%
\pgfusepath{fill}%
\end{pgfscope}%
\begin{pgfscope}%
\pgfpathrectangle{\pgfqpoint{1.150000in}{0.150000in}}{\pgfqpoint{5.700000in}{5.700000in}}%
\pgfusepath{clip}%
\pgfsetbuttcap%
\pgfsetroundjoin%
\definecolor{currentfill}{rgb}{0.120081,0.622161,0.534946}%
\pgfsetfillcolor{currentfill}%
\pgfsetfillopacity{0.700000}%
\pgfsetlinewidth{0.000000pt}%
\definecolor{currentstroke}{rgb}{0.000000,0.000000,0.000000}%
\pgfsetstrokecolor{currentstroke}%
\pgfsetdash{}{0pt}%
\pgfpathmoveto{\pgfqpoint{4.453853in}{3.875783in}}%
\pgfpathlineto{\pgfqpoint{4.467235in}{3.864066in}}%
\pgfpathlineto{\pgfqpoint{4.480620in}{3.852435in}}%
\pgfpathlineto{\pgfqpoint{4.494008in}{3.840890in}}%
\pgfpathlineto{\pgfqpoint{4.507398in}{3.829431in}}%
\pgfpathlineto{\pgfqpoint{4.515165in}{3.862845in}}%
\pgfpathlineto{\pgfqpoint{4.522940in}{3.896897in}}%
\pgfpathlineto{\pgfqpoint{4.530722in}{3.931598in}}%
\pgfpathlineto{\pgfqpoint{4.517328in}{3.943504in}}%
\pgfpathlineto{\pgfqpoint{4.503937in}{3.955496in}}%
\pgfpathlineto{\pgfqpoint{4.490548in}{3.967574in}}%
\pgfpathlineto{\pgfqpoint{4.477161in}{3.979738in}}%
\pgfpathlineto{\pgfqpoint{4.469385in}{3.944435in}}%
\pgfpathlineto{\pgfqpoint{4.461615in}{3.909787in}}%
\pgfpathlineto{\pgfqpoint{4.453853in}{3.875783in}}%
\pgfpathclose%
\pgfusepath{fill}%
\end{pgfscope}%
\begin{pgfscope}%
\pgfpathrectangle{\pgfqpoint{1.150000in}{0.150000in}}{\pgfqpoint{5.700000in}{5.700000in}}%
\pgfusepath{clip}%
\pgfsetbuttcap%
\pgfsetroundjoin%
\definecolor{currentfill}{rgb}{0.244972,0.287675,0.537260}%
\pgfsetfillcolor{currentfill}%
\pgfsetfillopacity{0.700000}%
\pgfsetlinewidth{0.000000pt}%
\definecolor{currentstroke}{rgb}{0.000000,0.000000,0.000000}%
\pgfsetstrokecolor{currentstroke}%
\pgfsetdash{}{0pt}%
\pgfpathmoveto{\pgfqpoint{3.985105in}{3.014074in}}%
\pgfpathlineto{\pgfqpoint{3.998456in}{3.006032in}}%
\pgfpathlineto{\pgfqpoint{4.011810in}{2.998079in}}%
\pgfpathlineto{\pgfqpoint{4.025167in}{2.990215in}}%
\pgfpathlineto{\pgfqpoint{4.038529in}{2.982440in}}%
\pgfpathlineto{\pgfqpoint{4.046253in}{2.999649in}}%
\pgfpathlineto{\pgfqpoint{4.053974in}{3.017143in}}%
\pgfpathlineto{\pgfqpoint{4.061692in}{3.034931in}}%
\pgfpathlineto{\pgfqpoint{4.069408in}{3.053019in}}%
\pgfpathlineto{\pgfqpoint{4.056051in}{3.061170in}}%
\pgfpathlineto{\pgfqpoint{4.042698in}{3.069410in}}%
\pgfpathlineto{\pgfqpoint{4.029348in}{3.077739in}}%
\pgfpathlineto{\pgfqpoint{4.016002in}{3.086157in}}%
\pgfpathlineto{\pgfqpoint{4.008282in}{3.067685in}}%
\pgfpathlineto{\pgfqpoint{4.000559in}{3.049519in}}%
\pgfpathlineto{\pgfqpoint{3.992834in}{3.031651in}}%
\pgfpathlineto{\pgfqpoint{3.985105in}{3.014074in}}%
\pgfpathclose%
\pgfusepath{fill}%
\end{pgfscope}%
\begin{pgfscope}%
\pgfpathrectangle{\pgfqpoint{1.150000in}{0.150000in}}{\pgfqpoint{5.700000in}{5.700000in}}%
\pgfusepath{clip}%
\pgfsetbuttcap%
\pgfsetroundjoin%
\definecolor{currentfill}{rgb}{0.246811,0.283237,0.535941}%
\pgfsetfillcolor{currentfill}%
\pgfsetfillopacity{0.700000}%
\pgfsetlinewidth{0.000000pt}%
\definecolor{currentstroke}{rgb}{0.000000,0.000000,0.000000}%
\pgfsetstrokecolor{currentstroke}%
\pgfsetdash{}{0pt}%
\pgfpathmoveto{\pgfqpoint{3.127550in}{3.018692in}}%
\pgfpathlineto{\pgfqpoint{3.140836in}{3.007676in}}%
\pgfpathlineto{\pgfqpoint{3.154123in}{2.996784in}}%
\pgfpathlineto{\pgfqpoint{3.167410in}{2.986015in}}%
\pgfpathlineto{\pgfqpoint{3.180697in}{2.975369in}}%
\pgfpathlineto{\pgfqpoint{3.188622in}{2.989730in}}%
\pgfpathlineto{\pgfqpoint{3.196540in}{3.004281in}}%
\pgfpathlineto{\pgfqpoint{3.204451in}{3.019027in}}%
\pgfpathlineto{\pgfqpoint{3.212356in}{3.033971in}}%
\pgfpathlineto{\pgfqpoint{3.199073in}{3.044854in}}%
\pgfpathlineto{\pgfqpoint{3.185792in}{3.055860in}}%
\pgfpathlineto{\pgfqpoint{3.172510in}{3.066988in}}%
\pgfpathlineto{\pgfqpoint{3.159228in}{3.078242in}}%
\pgfpathlineto{\pgfqpoint{3.151319in}{3.063052in}}%
\pgfpathlineto{\pgfqpoint{3.143403in}{3.048067in}}%
\pgfpathlineto{\pgfqpoint{3.135480in}{3.033282in}}%
\pgfpathlineto{\pgfqpoint{3.127550in}{3.018692in}}%
\pgfpathclose%
\pgfusepath{fill}%
\end{pgfscope}%
\begin{pgfscope}%
\pgfpathrectangle{\pgfqpoint{1.150000in}{0.150000in}}{\pgfqpoint{5.700000in}{5.700000in}}%
\pgfusepath{clip}%
\pgfsetbuttcap%
\pgfsetroundjoin%
\definecolor{currentfill}{rgb}{0.150476,0.504369,0.557430}%
\pgfsetfillcolor{currentfill}%
\pgfsetfillopacity{0.700000}%
\pgfsetlinewidth{0.000000pt}%
\definecolor{currentstroke}{rgb}{0.000000,0.000000,0.000000}%
\pgfsetstrokecolor{currentstroke}%
\pgfsetdash{}{0pt}%
\pgfpathmoveto{\pgfqpoint{4.499046in}{3.543147in}}%
\pgfpathlineto{\pgfqpoint{4.512448in}{3.533227in}}%
\pgfpathlineto{\pgfqpoint{4.525852in}{3.523388in}}%
\pgfpathlineto{\pgfqpoint{4.539261in}{3.513632in}}%
\pgfpathlineto{\pgfqpoint{4.552673in}{3.503957in}}%
\pgfpathlineto{\pgfqpoint{4.560396in}{3.531610in}}%
\pgfpathlineto{\pgfqpoint{4.568124in}{3.559799in}}%
\pgfpathlineto{\pgfqpoint{4.575858in}{3.588534in}}%
\pgfpathlineto{\pgfqpoint{4.583598in}{3.617825in}}%
\pgfpathlineto{\pgfqpoint{4.570186in}{3.628046in}}%
\pgfpathlineto{\pgfqpoint{4.556777in}{3.638350in}}%
\pgfpathlineto{\pgfqpoint{4.543372in}{3.648735in}}%
\pgfpathlineto{\pgfqpoint{4.529970in}{3.659203in}}%
\pgfpathlineto{\pgfqpoint{4.522231in}{3.629356in}}%
\pgfpathlineto{\pgfqpoint{4.514498in}{3.600071in}}%
\pgfpathlineto{\pgfqpoint{4.506769in}{3.571339in}}%
\pgfpathlineto{\pgfqpoint{4.499046in}{3.543147in}}%
\pgfpathclose%
\pgfusepath{fill}%
\end{pgfscope}%
\begin{pgfscope}%
\pgfpathrectangle{\pgfqpoint{1.150000in}{0.150000in}}{\pgfqpoint{5.700000in}{5.700000in}}%
\pgfusepath{clip}%
\pgfsetbuttcap%
\pgfsetroundjoin%
\definecolor{currentfill}{rgb}{0.993248,0.906157,0.143936}%
\pgfsetfillcolor{currentfill}%
\pgfsetfillopacity{0.700000}%
\pgfsetlinewidth{0.000000pt}%
\definecolor{currentstroke}{rgb}{0.000000,0.000000,0.000000}%
\pgfsetstrokecolor{currentstroke}%
\pgfsetdash{}{0pt}%
\pgfpathmoveto{\pgfqpoint{3.492230in}{4.986434in}}%
\pgfpathlineto{\pgfqpoint{3.505630in}{4.966687in}}%
\pgfpathlineto{\pgfqpoint{3.519027in}{4.947089in}}%
\pgfpathlineto{\pgfqpoint{3.532421in}{4.927636in}}%
\pgfpathlineto{\pgfqpoint{3.545811in}{4.908327in}}%
\pgfpathlineto{\pgfqpoint{3.553352in}{4.952733in}}%
\pgfpathlineto{\pgfqpoint{3.560891in}{4.997902in}}%
\pgfpathlineto{\pgfqpoint{3.568428in}{5.043847in}}%
\pgfpathlineto{\pgfqpoint{3.555022in}{5.063643in}}%
\pgfpathlineto{\pgfqpoint{3.541612in}{5.083584in}}%
\pgfpathlineto{\pgfqpoint{3.528198in}{5.103672in}}%
\pgfpathlineto{\pgfqpoint{3.514781in}{5.123909in}}%
\pgfpathlineto{\pgfqpoint{3.507266in}{5.077304in}}%
\pgfpathlineto{\pgfqpoint{3.499749in}{5.031483in}}%
\pgfpathlineto{\pgfqpoint{3.492230in}{4.986434in}}%
\pgfpathclose%
\pgfusepath{fill}%
\end{pgfscope}%
\begin{pgfscope}%
\pgfpathrectangle{\pgfqpoint{1.150000in}{0.150000in}}{\pgfqpoint{5.700000in}{5.700000in}}%
\pgfusepath{clip}%
\pgfsetbuttcap%
\pgfsetroundjoin%
\definecolor{currentfill}{rgb}{0.183898,0.422383,0.556944}%
\pgfsetfillcolor{currentfill}%
\pgfsetfillopacity{0.700000}%
\pgfsetlinewidth{0.000000pt}%
\definecolor{currentstroke}{rgb}{0.000000,0.000000,0.000000}%
\pgfsetstrokecolor{currentstroke}%
\pgfsetdash{}{0pt}%
\pgfpathmoveto{\pgfqpoint{4.437396in}{3.335855in}}%
\pgfpathlineto{\pgfqpoint{4.450800in}{3.326960in}}%
\pgfpathlineto{\pgfqpoint{4.464208in}{3.318147in}}%
\pgfpathlineto{\pgfqpoint{4.477620in}{3.309415in}}%
\pgfpathlineto{\pgfqpoint{4.491036in}{3.300764in}}%
\pgfpathlineto{\pgfqpoint{4.498729in}{3.324503in}}%
\pgfpathlineto{\pgfqpoint{4.506424in}{3.348696in}}%
\pgfpathlineto{\pgfqpoint{4.514123in}{3.373354in}}%
\pgfpathlineto{\pgfqpoint{4.521825in}{3.398486in}}%
\pgfpathlineto{\pgfqpoint{4.508412in}{3.407638in}}%
\pgfpathlineto{\pgfqpoint{4.495002in}{3.416871in}}%
\pgfpathlineto{\pgfqpoint{4.481597in}{3.426185in}}%
\pgfpathlineto{\pgfqpoint{4.468194in}{3.435582in}}%
\pgfpathlineto{\pgfqpoint{4.460490in}{3.409940in}}%
\pgfpathlineto{\pgfqpoint{4.452790in}{3.384778in}}%
\pgfpathlineto{\pgfqpoint{4.445092in}{3.360086in}}%
\pgfpathlineto{\pgfqpoint{4.437396in}{3.335855in}}%
\pgfpathclose%
\pgfusepath{fill}%
\end{pgfscope}%
\begin{pgfscope}%
\pgfpathrectangle{\pgfqpoint{1.150000in}{0.150000in}}{\pgfqpoint{5.700000in}{5.700000in}}%
\pgfusepath{clip}%
\pgfsetbuttcap%
\pgfsetroundjoin%
\definecolor{currentfill}{rgb}{0.231674,0.318106,0.544834}%
\pgfsetfillcolor{currentfill}%
\pgfsetfillopacity{0.700000}%
\pgfsetlinewidth{0.000000pt}%
\definecolor{currentstroke}{rgb}{0.000000,0.000000,0.000000}%
\pgfsetstrokecolor{currentstroke}%
\pgfsetdash{}{0pt}%
\pgfpathmoveto{\pgfqpoint{2.936208in}{3.101894in}}%
\pgfpathlineto{\pgfqpoint{2.949509in}{3.089515in}}%
\pgfpathlineto{\pgfqpoint{2.962809in}{3.077275in}}%
\pgfpathlineto{\pgfqpoint{2.976107in}{3.065171in}}%
\pgfpathlineto{\pgfqpoint{2.989405in}{3.053203in}}%
\pgfpathlineto{\pgfqpoint{2.997375in}{3.067472in}}%
\pgfpathlineto{\pgfqpoint{3.005339in}{3.081930in}}%
\pgfpathlineto{\pgfqpoint{3.013294in}{3.096581in}}%
\pgfpathlineto{\pgfqpoint{3.021242in}{3.111429in}}%
\pgfpathlineto{\pgfqpoint{3.007950in}{3.123615in}}%
\pgfpathlineto{\pgfqpoint{2.994657in}{3.135936in}}%
\pgfpathlineto{\pgfqpoint{2.981363in}{3.148394in}}%
\pgfpathlineto{\pgfqpoint{2.968067in}{3.160990in}}%
\pgfpathlineto{\pgfqpoint{2.960114in}{3.145917in}}%
\pgfpathlineto{\pgfqpoint{2.952153in}{3.131046in}}%
\pgfpathlineto{\pgfqpoint{2.944185in}{3.116373in}}%
\pgfpathlineto{\pgfqpoint{2.936208in}{3.101894in}}%
\pgfpathclose%
\pgfusepath{fill}%
\end{pgfscope}%
\begin{pgfscope}%
\pgfpathrectangle{\pgfqpoint{1.150000in}{0.150000in}}{\pgfqpoint{5.700000in}{5.700000in}}%
\pgfusepath{clip}%
\pgfsetbuttcap%
\pgfsetroundjoin%
\definecolor{currentfill}{rgb}{0.210503,0.363727,0.552206}%
\pgfsetfillcolor{currentfill}%
\pgfsetfillopacity{0.700000}%
\pgfsetlinewidth{0.000000pt}%
\definecolor{currentstroke}{rgb}{0.000000,0.000000,0.000000}%
\pgfsetstrokecolor{currentstroke}%
\pgfsetdash{}{0pt}%
\pgfpathmoveto{\pgfqpoint{4.322291in}{3.190133in}}%
\pgfpathlineto{\pgfqpoint{4.335685in}{3.181846in}}%
\pgfpathlineto{\pgfqpoint{4.349084in}{3.173642in}}%
\pgfpathlineto{\pgfqpoint{4.362487in}{3.165520in}}%
\pgfpathlineto{\pgfqpoint{4.375893in}{3.157481in}}%
\pgfpathlineto{\pgfqpoint{4.383578in}{3.178357in}}%
\pgfpathlineto{\pgfqpoint{4.391262in}{3.199621in}}%
\pgfpathlineto{\pgfqpoint{4.398948in}{3.221282in}}%
\pgfpathlineto{\pgfqpoint{4.406634in}{3.243349in}}%
\pgfpathlineto{\pgfqpoint{4.393231in}{3.251846in}}%
\pgfpathlineto{\pgfqpoint{4.379832in}{3.260426in}}%
\pgfpathlineto{\pgfqpoint{4.366437in}{3.269089in}}%
\pgfpathlineto{\pgfqpoint{4.353046in}{3.277834in}}%
\pgfpathlineto{\pgfqpoint{4.345356in}{3.255301in}}%
\pgfpathlineto{\pgfqpoint{4.337667in}{3.233178in}}%
\pgfpathlineto{\pgfqpoint{4.329979in}{3.211459in}}%
\pgfpathlineto{\pgfqpoint{4.322291in}{3.190133in}}%
\pgfpathclose%
\pgfusepath{fill}%
\end{pgfscope}%
\begin{pgfscope}%
\pgfpathrectangle{\pgfqpoint{1.150000in}{0.150000in}}{\pgfqpoint{5.700000in}{5.700000in}}%
\pgfusepath{clip}%
\pgfsetbuttcap%
\pgfsetroundjoin%
\definecolor{currentfill}{rgb}{0.257322,0.256130,0.526563}%
\pgfsetfillcolor{currentfill}%
\pgfsetfillopacity{0.700000}%
\pgfsetlinewidth{0.000000pt}%
\definecolor{currentstroke}{rgb}{0.000000,0.000000,0.000000}%
\pgfsetstrokecolor{currentstroke}%
\pgfsetdash{}{0pt}%
\pgfpathmoveto{\pgfqpoint{3.678692in}{2.948662in}}%
\pgfpathlineto{\pgfqpoint{3.692006in}{2.940121in}}%
\pgfpathlineto{\pgfqpoint{3.705322in}{2.931680in}}%
\pgfpathlineto{\pgfqpoint{3.718641in}{2.923335in}}%
\pgfpathlineto{\pgfqpoint{3.731963in}{2.915088in}}%
\pgfpathlineto{\pgfqpoint{3.739754in}{2.930598in}}%
\pgfpathlineto{\pgfqpoint{3.747540in}{2.946339in}}%
\pgfpathlineto{\pgfqpoint{3.755321in}{2.962316in}}%
\pgfpathlineto{\pgfqpoint{3.763098in}{2.978535in}}%
\pgfpathlineto{\pgfqpoint{3.749780in}{2.987098in}}%
\pgfpathlineto{\pgfqpoint{3.736466in}{2.995758in}}%
\pgfpathlineto{\pgfqpoint{3.723154in}{3.004516in}}%
\pgfpathlineto{\pgfqpoint{3.709845in}{3.013372in}}%
\pgfpathlineto{\pgfqpoint{3.702064in}{2.996829in}}%
\pgfpathlineto{\pgfqpoint{3.694278in}{2.980534in}}%
\pgfpathlineto{\pgfqpoint{3.686488in}{2.964480in}}%
\pgfpathlineto{\pgfqpoint{3.678692in}{2.948662in}}%
\pgfpathclose%
\pgfusepath{fill}%
\end{pgfscope}%
\begin{pgfscope}%
\pgfpathrectangle{\pgfqpoint{1.150000in}{0.150000in}}{\pgfqpoint{5.700000in}{5.700000in}}%
\pgfusepath{clip}%
\pgfsetbuttcap%
\pgfsetroundjoin%
\definecolor{currentfill}{rgb}{0.252194,0.269783,0.531579}%
\pgfsetfillcolor{currentfill}%
\pgfsetfillopacity{0.700000}%
\pgfsetlinewidth{0.000000pt}%
\definecolor{currentstroke}{rgb}{0.000000,0.000000,0.000000}%
\pgfsetstrokecolor{currentstroke}%
\pgfsetdash{}{0pt}%
\pgfpathmoveto{\pgfqpoint{3.900773in}{2.978198in}}%
\pgfpathlineto{\pgfqpoint{3.914114in}{2.970148in}}%
\pgfpathlineto{\pgfqpoint{3.927458in}{2.962190in}}%
\pgfpathlineto{\pgfqpoint{3.940806in}{2.954323in}}%
\pgfpathlineto{\pgfqpoint{3.954158in}{2.946546in}}%
\pgfpathlineto{\pgfqpoint{3.961900in}{2.963024in}}%
\pgfpathlineto{\pgfqpoint{3.969638in}{2.979768in}}%
\pgfpathlineto{\pgfqpoint{3.977374in}{2.996782in}}%
\pgfpathlineto{\pgfqpoint{3.985105in}{3.014074in}}%
\pgfpathlineto{\pgfqpoint{3.971758in}{3.022206in}}%
\pgfpathlineto{\pgfqpoint{3.958415in}{3.030430in}}%
\pgfpathlineto{\pgfqpoint{3.945075in}{3.038744in}}%
\pgfpathlineto{\pgfqpoint{3.931738in}{3.047150in}}%
\pgfpathlineto{\pgfqpoint{3.924002in}{3.029494in}}%
\pgfpathlineto{\pgfqpoint{3.916263in}{3.012121in}}%
\pgfpathlineto{\pgfqpoint{3.908520in}{2.995025in}}%
\pgfpathlineto{\pgfqpoint{3.900773in}{2.978198in}}%
\pgfpathclose%
\pgfusepath{fill}%
\end{pgfscope}%
\begin{pgfscope}%
\pgfpathrectangle{\pgfqpoint{1.150000in}{0.150000in}}{\pgfqpoint{5.700000in}{5.700000in}}%
\pgfusepath{clip}%
\pgfsetbuttcap%
\pgfsetroundjoin%
\definecolor{currentfill}{rgb}{0.199430,0.387607,0.554642}%
\pgfsetfillcolor{currentfill}%
\pgfsetfillopacity{0.700000}%
\pgfsetlinewidth{0.000000pt}%
\definecolor{currentstroke}{rgb}{0.000000,0.000000,0.000000}%
\pgfsetstrokecolor{currentstroke}%
\pgfsetdash{}{0pt}%
\pgfpathmoveto{\pgfqpoint{4.406634in}{3.243349in}}%
\pgfpathlineto{\pgfqpoint{4.420041in}{3.234934in}}%
\pgfpathlineto{\pgfqpoint{4.433453in}{3.226600in}}%
\pgfpathlineto{\pgfqpoint{4.446868in}{3.218348in}}%
\pgfpathlineto{\pgfqpoint{4.460287in}{3.210176in}}%
\pgfpathlineto{\pgfqpoint{4.467972in}{3.232187in}}%
\pgfpathlineto{\pgfqpoint{4.475658in}{3.254616in}}%
\pgfpathlineto{\pgfqpoint{4.483346in}{3.277472in}}%
\pgfpathlineto{\pgfqpoint{4.491036in}{3.300764in}}%
\pgfpathlineto{\pgfqpoint{4.477620in}{3.309415in}}%
\pgfpathlineto{\pgfqpoint{4.464208in}{3.318147in}}%
\pgfpathlineto{\pgfqpoint{4.450800in}{3.326960in}}%
\pgfpathlineto{\pgfqpoint{4.437396in}{3.335855in}}%
\pgfpathlineto{\pgfqpoint{4.429703in}{3.312075in}}%
\pgfpathlineto{\pgfqpoint{4.422012in}{3.288736in}}%
\pgfpathlineto{\pgfqpoint{4.414322in}{3.265831in}}%
\pgfpathlineto{\pgfqpoint{4.406634in}{3.243349in}}%
\pgfpathclose%
\pgfusepath{fill}%
\end{pgfscope}%
\begin{pgfscope}%
\pgfpathrectangle{\pgfqpoint{1.150000in}{0.150000in}}{\pgfqpoint{5.700000in}{5.700000in}}%
\pgfusepath{clip}%
\pgfsetbuttcap%
\pgfsetroundjoin%
\definecolor{currentfill}{rgb}{0.133743,0.548535,0.553541}%
\pgfsetfillcolor{currentfill}%
\pgfsetfillopacity{0.700000}%
\pgfsetlinewidth{0.000000pt}%
\definecolor{currentstroke}{rgb}{0.000000,0.000000,0.000000}%
\pgfsetstrokecolor{currentstroke}%
\pgfsetdash{}{0pt}%
\pgfpathmoveto{\pgfqpoint{4.529970in}{3.659203in}}%
\pgfpathlineto{\pgfqpoint{4.543372in}{3.648735in}}%
\pgfpathlineto{\pgfqpoint{4.556777in}{3.638350in}}%
\pgfpathlineto{\pgfqpoint{4.570186in}{3.628046in}}%
\pgfpathlineto{\pgfqpoint{4.583598in}{3.617825in}}%
\pgfpathlineto{\pgfqpoint{4.591343in}{3.647684in}}%
\pgfpathlineto{\pgfqpoint{4.599096in}{3.678121in}}%
\pgfpathlineto{\pgfqpoint{4.606855in}{3.709148in}}%
\pgfpathlineto{\pgfqpoint{4.614621in}{3.740776in}}%
\pgfpathlineto{\pgfqpoint{4.601207in}{3.751568in}}%
\pgfpathlineto{\pgfqpoint{4.587797in}{3.762442in}}%
\pgfpathlineto{\pgfqpoint{4.574390in}{3.773398in}}%
\pgfpathlineto{\pgfqpoint{4.560986in}{3.784437in}}%
\pgfpathlineto{\pgfqpoint{4.553222in}{3.752229in}}%
\pgfpathlineto{\pgfqpoint{4.545465in}{3.720628in}}%
\pgfpathlineto{\pgfqpoint{4.537714in}{3.689623in}}%
\pgfpathlineto{\pgfqpoint{4.529970in}{3.659203in}}%
\pgfpathclose%
\pgfusepath{fill}%
\end{pgfscope}%
\begin{pgfscope}%
\pgfpathrectangle{\pgfqpoint{1.150000in}{0.150000in}}{\pgfqpoint{5.700000in}{5.700000in}}%
\pgfusepath{clip}%
\pgfsetbuttcap%
\pgfsetroundjoin%
\definecolor{currentfill}{rgb}{0.257322,0.256130,0.526563}%
\pgfsetfillcolor{currentfill}%
\pgfsetfillopacity{0.700000}%
\pgfsetlinewidth{0.000000pt}%
\definecolor{currentstroke}{rgb}{0.000000,0.000000,0.000000}%
\pgfsetstrokecolor{currentstroke}%
\pgfsetdash{}{0pt}%
\pgfpathmoveto{\pgfqpoint{3.318632in}{2.951195in}}%
\pgfpathlineto{\pgfqpoint{3.331921in}{2.941370in}}%
\pgfpathlineto{\pgfqpoint{3.345210in}{2.931658in}}%
\pgfpathlineto{\pgfqpoint{3.358501in}{2.922058in}}%
\pgfpathlineto{\pgfqpoint{3.371794in}{2.912569in}}%
\pgfpathlineto{\pgfqpoint{3.379676in}{2.926971in}}%
\pgfpathlineto{\pgfqpoint{3.387553in}{2.941566in}}%
\pgfpathlineto{\pgfqpoint{3.395423in}{2.956359in}}%
\pgfpathlineto{\pgfqpoint{3.403287in}{2.971355in}}%
\pgfpathlineto{\pgfqpoint{3.389999in}{2.981099in}}%
\pgfpathlineto{\pgfqpoint{3.376713in}{2.990956in}}%
\pgfpathlineto{\pgfqpoint{3.363428in}{3.000924in}}%
\pgfpathlineto{\pgfqpoint{3.350145in}{3.011006in}}%
\pgfpathlineto{\pgfqpoint{3.342276in}{2.995746in}}%
\pgfpathlineto{\pgfqpoint{3.334401in}{2.980694in}}%
\pgfpathlineto{\pgfqpoint{3.326520in}{2.965846in}}%
\pgfpathlineto{\pgfqpoint{3.318632in}{2.951195in}}%
\pgfpathclose%
\pgfusepath{fill}%
\end{pgfscope}%
\begin{pgfscope}%
\pgfpathrectangle{\pgfqpoint{1.150000in}{0.150000in}}{\pgfqpoint{5.700000in}{5.700000in}}%
\pgfusepath{clip}%
\pgfsetbuttcap%
\pgfsetroundjoin%
\definecolor{currentfill}{rgb}{0.260571,0.246922,0.522828}%
\pgfsetfillcolor{currentfill}%
\pgfsetfillopacity{0.700000}%
\pgfsetlinewidth{0.000000pt}%
\definecolor{currentstroke}{rgb}{0.000000,0.000000,0.000000}%
\pgfsetstrokecolor{currentstroke}%
\pgfsetdash{}{0pt}%
\pgfpathmoveto{\pgfqpoint{3.456452in}{2.933472in}}%
\pgfpathlineto{\pgfqpoint{3.469748in}{2.924271in}}%
\pgfpathlineto{\pgfqpoint{3.483045in}{2.915177in}}%
\pgfpathlineto{\pgfqpoint{3.496345in}{2.906189in}}%
\pgfpathlineto{\pgfqpoint{3.509646in}{2.897306in}}%
\pgfpathlineto{\pgfqpoint{3.517494in}{2.911972in}}%
\pgfpathlineto{\pgfqpoint{3.525337in}{2.926841in}}%
\pgfpathlineto{\pgfqpoint{3.533173in}{2.941918in}}%
\pgfpathlineto{\pgfqpoint{3.541004in}{2.957208in}}%
\pgfpathlineto{\pgfqpoint{3.527708in}{2.966367in}}%
\pgfpathlineto{\pgfqpoint{3.514413in}{2.975631in}}%
\pgfpathlineto{\pgfqpoint{3.501120in}{2.985001in}}%
\pgfpathlineto{\pgfqpoint{3.487829in}{2.994478in}}%
\pgfpathlineto{\pgfqpoint{3.479994in}{2.978904in}}%
\pgfpathlineto{\pgfqpoint{3.472152in}{2.963549in}}%
\pgfpathlineto{\pgfqpoint{3.464305in}{2.948406in}}%
\pgfpathlineto{\pgfqpoint{3.456452in}{2.933472in}}%
\pgfpathclose%
\pgfusepath{fill}%
\end{pgfscope}%
\begin{pgfscope}%
\pgfpathrectangle{\pgfqpoint{1.150000in}{0.150000in}}{\pgfqpoint{5.700000in}{5.700000in}}%
\pgfusepath{clip}%
\pgfsetbuttcap%
\pgfsetroundjoin%
\definecolor{currentfill}{rgb}{0.119512,0.607464,0.540218}%
\pgfsetfillcolor{currentfill}%
\pgfsetfillopacity{0.700000}%
\pgfsetlinewidth{0.000000pt}%
\definecolor{currentstroke}{rgb}{0.000000,0.000000,0.000000}%
\pgfsetstrokecolor{currentstroke}%
\pgfsetdash{}{0pt}%
\pgfpathmoveto{\pgfqpoint{4.507398in}{3.829431in}}%
\pgfpathlineto{\pgfqpoint{4.520791in}{3.818056in}}%
\pgfpathlineto{\pgfqpoint{4.534186in}{3.806766in}}%
\pgfpathlineto{\pgfqpoint{4.547585in}{3.795560in}}%
\pgfpathlineto{\pgfqpoint{4.560986in}{3.784437in}}%
\pgfpathlineto{\pgfqpoint{4.568757in}{3.817264in}}%
\pgfpathlineto{\pgfqpoint{4.576535in}{3.850722in}}%
\pgfpathlineto{\pgfqpoint{4.584322in}{3.884823in}}%
\pgfpathlineto{\pgfqpoint{4.570918in}{3.896390in}}%
\pgfpathlineto{\pgfqpoint{4.557516in}{3.908042in}}%
\pgfpathlineto{\pgfqpoint{4.544118in}{3.919778in}}%
\pgfpathlineto{\pgfqpoint{4.530722in}{3.931598in}}%
\pgfpathlineto{\pgfqpoint{4.522940in}{3.896897in}}%
\pgfpathlineto{\pgfqpoint{4.515165in}{3.862845in}}%
\pgfpathlineto{\pgfqpoint{4.507398in}{3.829431in}}%
\pgfpathclose%
\pgfusepath{fill}%
\end{pgfscope}%
\begin{pgfscope}%
\pgfpathrectangle{\pgfqpoint{1.150000in}{0.150000in}}{\pgfqpoint{5.700000in}{5.700000in}}%
\pgfusepath{clip}%
\pgfsetbuttcap%
\pgfsetroundjoin%
\definecolor{currentfill}{rgb}{0.239346,0.300855,0.540844}%
\pgfsetfillcolor{currentfill}%
\pgfsetfillopacity{0.700000}%
\pgfsetlinewidth{0.000000pt}%
\definecolor{currentstroke}{rgb}{0.000000,0.000000,0.000000}%
\pgfsetstrokecolor{currentstroke}%
\pgfsetdash{}{0pt}%
\pgfpathmoveto{\pgfqpoint{2.989405in}{3.053203in}}%
\pgfpathlineto{\pgfqpoint{3.002701in}{3.041370in}}%
\pgfpathlineto{\pgfqpoint{3.015996in}{3.029669in}}%
\pgfpathlineto{\pgfqpoint{3.029291in}{3.018101in}}%
\pgfpathlineto{\pgfqpoint{3.042585in}{3.006664in}}%
\pgfpathlineto{\pgfqpoint{3.050550in}{3.020724in}}%
\pgfpathlineto{\pgfqpoint{3.058507in}{3.034967in}}%
\pgfpathlineto{\pgfqpoint{3.066458in}{3.049399in}}%
\pgfpathlineto{\pgfqpoint{3.074400in}{3.064022in}}%
\pgfpathlineto{\pgfqpoint{3.061112in}{3.075676in}}%
\pgfpathlineto{\pgfqpoint{3.047823in}{3.087461in}}%
\pgfpathlineto{\pgfqpoint{3.034533in}{3.099379in}}%
\pgfpathlineto{\pgfqpoint{3.021242in}{3.111429in}}%
\pgfpathlineto{\pgfqpoint{3.013294in}{3.096581in}}%
\pgfpathlineto{\pgfqpoint{3.005339in}{3.081930in}}%
\pgfpathlineto{\pgfqpoint{2.997375in}{3.067472in}}%
\pgfpathlineto{\pgfqpoint{2.989405in}{3.053203in}}%
\pgfpathclose%
\pgfusepath{fill}%
\end{pgfscope}%
\begin{pgfscope}%
\pgfpathrectangle{\pgfqpoint{1.150000in}{0.150000in}}{\pgfqpoint{5.700000in}{5.700000in}}%
\pgfusepath{clip}%
\pgfsetbuttcap%
\pgfsetroundjoin%
\definecolor{currentfill}{rgb}{0.257322,0.256130,0.526563}%
\pgfsetfillcolor{currentfill}%
\pgfsetfillopacity{0.700000}%
\pgfsetlinewidth{0.000000pt}%
\definecolor{currentstroke}{rgb}{0.000000,0.000000,0.000000}%
\pgfsetstrokecolor{currentstroke}%
\pgfsetdash{}{0pt}%
\pgfpathmoveto{\pgfqpoint{3.816397in}{2.945244in}}%
\pgfpathlineto{\pgfqpoint{3.829729in}{2.937158in}}%
\pgfpathlineto{\pgfqpoint{3.843065in}{2.929166in}}%
\pgfpathlineto{\pgfqpoint{3.856404in}{2.921267in}}%
\pgfpathlineto{\pgfqpoint{3.869747in}{2.913461in}}%
\pgfpathlineto{\pgfqpoint{3.877509in}{2.929272in}}%
\pgfpathlineto{\pgfqpoint{3.885268in}{2.945328in}}%
\pgfpathlineto{\pgfqpoint{3.893022in}{2.961635in}}%
\pgfpathlineto{\pgfqpoint{3.900773in}{2.978198in}}%
\pgfpathlineto{\pgfqpoint{3.887435in}{2.986340in}}%
\pgfpathlineto{\pgfqpoint{3.874101in}{2.994575in}}%
\pgfpathlineto{\pgfqpoint{3.860770in}{3.002903in}}%
\pgfpathlineto{\pgfqpoint{3.847442in}{3.011325in}}%
\pgfpathlineto{\pgfqpoint{3.839687in}{2.994418in}}%
\pgfpathlineto{\pgfqpoint{3.831928in}{2.977773in}}%
\pgfpathlineto{\pgfqpoint{3.824164in}{2.961384in}}%
\pgfpathlineto{\pgfqpoint{3.816397in}{2.945244in}}%
\pgfpathclose%
\pgfusepath{fill}%
\end{pgfscope}%
\begin{pgfscope}%
\pgfpathrectangle{\pgfqpoint{1.150000in}{0.150000in}}{\pgfqpoint{5.700000in}{5.700000in}}%
\pgfusepath{clip}%
\pgfsetbuttcap%
\pgfsetroundjoin%
\definecolor{currentfill}{rgb}{0.253935,0.265254,0.529983}%
\pgfsetfillcolor{currentfill}%
\pgfsetfillopacity{0.700000}%
\pgfsetlinewidth{0.000000pt}%
\definecolor{currentstroke}{rgb}{0.000000,0.000000,0.000000}%
\pgfsetstrokecolor{currentstroke}%
\pgfsetdash{}{0pt}%
\pgfpathmoveto{\pgfqpoint{3.180697in}{2.975369in}}%
\pgfpathlineto{\pgfqpoint{3.193985in}{2.964844in}}%
\pgfpathlineto{\pgfqpoint{3.207273in}{2.954439in}}%
\pgfpathlineto{\pgfqpoint{3.220561in}{2.944154in}}%
\pgfpathlineto{\pgfqpoint{3.233850in}{2.933987in}}%
\pgfpathlineto{\pgfqpoint{3.241770in}{2.948119in}}%
\pgfpathlineto{\pgfqpoint{3.249683in}{2.962436in}}%
\pgfpathlineto{\pgfqpoint{3.257589in}{2.976943in}}%
\pgfpathlineto{\pgfqpoint{3.265488in}{2.991643in}}%
\pgfpathlineto{\pgfqpoint{3.252204in}{3.002047in}}%
\pgfpathlineto{\pgfqpoint{3.238921in}{3.012569in}}%
\pgfpathlineto{\pgfqpoint{3.225638in}{3.023210in}}%
\pgfpathlineto{\pgfqpoint{3.212356in}{3.033971in}}%
\pgfpathlineto{\pgfqpoint{3.204451in}{3.019027in}}%
\pgfpathlineto{\pgfqpoint{3.196540in}{3.004281in}}%
\pgfpathlineto{\pgfqpoint{3.188622in}{2.989730in}}%
\pgfpathlineto{\pgfqpoint{3.180697in}{2.975369in}}%
\pgfpathclose%
\pgfusepath{fill}%
\end{pgfscope}%
\begin{pgfscope}%
\pgfpathrectangle{\pgfqpoint{1.150000in}{0.150000in}}{\pgfqpoint{5.700000in}{5.700000in}}%
\pgfusepath{clip}%
\pgfsetbuttcap%
\pgfsetroundjoin%
\definecolor{currentfill}{rgb}{0.262138,0.242286,0.520837}%
\pgfsetfillcolor{currentfill}%
\pgfsetfillopacity{0.700000}%
\pgfsetlinewidth{0.000000pt}%
\definecolor{currentstroke}{rgb}{0.000000,0.000000,0.000000}%
\pgfsetstrokecolor{currentstroke}%
\pgfsetdash{}{0pt}%
\pgfpathmoveto{\pgfqpoint{3.594213in}{2.921608in}}%
\pgfpathlineto{\pgfqpoint{3.607521in}{2.912964in}}%
\pgfpathlineto{\pgfqpoint{3.620832in}{2.904421in}}%
\pgfpathlineto{\pgfqpoint{3.634145in}{2.895978in}}%
\pgfpathlineto{\pgfqpoint{3.647460in}{2.887634in}}%
\pgfpathlineto{\pgfqpoint{3.655276in}{2.902565in}}%
\pgfpathlineto{\pgfqpoint{3.663087in}{2.917710in}}%
\pgfpathlineto{\pgfqpoint{3.670892in}{2.933074in}}%
\pgfpathlineto{\pgfqpoint{3.678692in}{2.948662in}}%
\pgfpathlineto{\pgfqpoint{3.665382in}{2.957301in}}%
\pgfpathlineto{\pgfqpoint{3.652073in}{2.966040in}}%
\pgfpathlineto{\pgfqpoint{3.638767in}{2.974879in}}%
\pgfpathlineto{\pgfqpoint{3.625464in}{2.983819in}}%
\pgfpathlineto{\pgfqpoint{3.617659in}{2.967928in}}%
\pgfpathlineto{\pgfqpoint{3.609849in}{2.952266in}}%
\pgfpathlineto{\pgfqpoint{3.602034in}{2.936828in}}%
\pgfpathlineto{\pgfqpoint{3.594213in}{2.921608in}}%
\pgfpathclose%
\pgfusepath{fill}%
\end{pgfscope}%
\begin{pgfscope}%
\pgfpathrectangle{\pgfqpoint{1.150000in}{0.150000in}}{\pgfqpoint{5.700000in}{5.700000in}}%
\pgfusepath{clip}%
\pgfsetbuttcap%
\pgfsetroundjoin%
\definecolor{currentfill}{rgb}{0.171176,0.452530,0.557965}%
\pgfsetfillcolor{currentfill}%
\pgfsetfillopacity{0.700000}%
\pgfsetlinewidth{0.000000pt}%
\definecolor{currentstroke}{rgb}{0.000000,0.000000,0.000000}%
\pgfsetstrokecolor{currentstroke}%
\pgfsetdash{}{0pt}%
\pgfpathmoveto{\pgfqpoint{4.521825in}{3.398486in}}%
\pgfpathlineto{\pgfqpoint{4.535242in}{3.389416in}}%
\pgfpathlineto{\pgfqpoint{4.548663in}{3.380425in}}%
\pgfpathlineto{\pgfqpoint{4.562088in}{3.371515in}}%
\pgfpathlineto{\pgfqpoint{4.575518in}{3.362685in}}%
\pgfpathlineto{\pgfqpoint{4.583221in}{3.387787in}}%
\pgfpathlineto{\pgfqpoint{4.590928in}{3.413378in}}%
\pgfpathlineto{\pgfqpoint{4.598641in}{3.439467in}}%
\pgfpathlineto{\pgfqpoint{4.606357in}{3.466064in}}%
\pgfpathlineto{\pgfqpoint{4.592931in}{3.475417in}}%
\pgfpathlineto{\pgfqpoint{4.579508in}{3.484850in}}%
\pgfpathlineto{\pgfqpoint{4.566088in}{3.494363in}}%
\pgfpathlineto{\pgfqpoint{4.552673in}{3.503957in}}%
\pgfpathlineto{\pgfqpoint{4.544954in}{3.476828in}}%
\pgfpathlineto{\pgfqpoint{4.537241in}{3.450213in}}%
\pgfpathlineto{\pgfqpoint{4.529531in}{3.424103in}}%
\pgfpathlineto{\pgfqpoint{4.521825in}{3.398486in}}%
\pgfpathclose%
\pgfusepath{fill}%
\end{pgfscope}%
\begin{pgfscope}%
\pgfpathrectangle{\pgfqpoint{1.150000in}{0.150000in}}{\pgfqpoint{5.700000in}{5.700000in}}%
\pgfusepath{clip}%
\pgfsetbuttcap%
\pgfsetroundjoin%
\definecolor{currentfill}{rgb}{0.154815,0.493313,0.557840}%
\pgfsetfillcolor{currentfill}%
\pgfsetfillopacity{0.700000}%
\pgfsetlinewidth{0.000000pt}%
\definecolor{currentstroke}{rgb}{0.000000,0.000000,0.000000}%
\pgfsetstrokecolor{currentstroke}%
\pgfsetdash{}{0pt}%
\pgfpathmoveto{\pgfqpoint{4.552673in}{3.503957in}}%
\pgfpathlineto{\pgfqpoint{4.566088in}{3.494363in}}%
\pgfpathlineto{\pgfqpoint{4.579508in}{3.484850in}}%
\pgfpathlineto{\pgfqpoint{4.592931in}{3.475417in}}%
\pgfpathlineto{\pgfqpoint{4.606357in}{3.466064in}}%
\pgfpathlineto{\pgfqpoint{4.614080in}{3.493181in}}%
\pgfpathlineto{\pgfqpoint{4.621807in}{3.520827in}}%
\pgfpathlineto{\pgfqpoint{4.629540in}{3.549013in}}%
\pgfpathlineto{\pgfqpoint{4.637280in}{3.577749in}}%
\pgfpathlineto{\pgfqpoint{4.623854in}{3.587648in}}%
\pgfpathlineto{\pgfqpoint{4.610432in}{3.597626in}}%
\pgfpathlineto{\pgfqpoint{4.597013in}{3.607685in}}%
\pgfpathlineto{\pgfqpoint{4.583598in}{3.617825in}}%
\pgfpathlineto{\pgfqpoint{4.575858in}{3.588534in}}%
\pgfpathlineto{\pgfqpoint{4.568124in}{3.559799in}}%
\pgfpathlineto{\pgfqpoint{4.560396in}{3.531610in}}%
\pgfpathlineto{\pgfqpoint{4.552673in}{3.503957in}}%
\pgfpathclose%
\pgfusepath{fill}%
\end{pgfscope}%
\begin{pgfscope}%
\pgfpathrectangle{\pgfqpoint{1.150000in}{0.150000in}}{\pgfqpoint{5.700000in}{5.700000in}}%
\pgfusepath{clip}%
\pgfsetbuttcap%
\pgfsetroundjoin%
\definecolor{currentfill}{rgb}{0.233603,0.313828,0.543914}%
\pgfsetfillcolor{currentfill}%
\pgfsetfillopacity{0.700000}%
\pgfsetlinewidth{0.000000pt}%
\definecolor{currentstroke}{rgb}{0.000000,0.000000,0.000000}%
\pgfsetstrokecolor{currentstroke}%
\pgfsetdash{}{0pt}%
\pgfpathmoveto{\pgfqpoint{4.207204in}{3.063279in}}%
\pgfpathlineto{\pgfqpoint{4.220591in}{3.055511in}}%
\pgfpathlineto{\pgfqpoint{4.233982in}{3.047827in}}%
\pgfpathlineto{\pgfqpoint{4.247377in}{3.040226in}}%
\pgfpathlineto{\pgfqpoint{4.260777in}{3.032710in}}%
\pgfpathlineto{\pgfqpoint{4.268469in}{3.051181in}}%
\pgfpathlineto{\pgfqpoint{4.276160in}{3.069981in}}%
\pgfpathlineto{\pgfqpoint{4.283849in}{3.089118in}}%
\pgfpathlineto{\pgfqpoint{4.291539in}{3.108600in}}%
\pgfpathlineto{\pgfqpoint{4.278144in}{3.116533in}}%
\pgfpathlineto{\pgfqpoint{4.264753in}{3.124550in}}%
\pgfpathlineto{\pgfqpoint{4.251367in}{3.132651in}}%
\pgfpathlineto{\pgfqpoint{4.237984in}{3.140836in}}%
\pgfpathlineto{\pgfqpoint{4.230290in}{3.120930in}}%
\pgfpathlineto{\pgfqpoint{4.222596in}{3.101374in}}%
\pgfpathlineto{\pgfqpoint{4.214901in}{3.082159in}}%
\pgfpathlineto{\pgfqpoint{4.207204in}{3.063279in}}%
\pgfpathclose%
\pgfusepath{fill}%
\end{pgfscope}%
\begin{pgfscope}%
\pgfpathrectangle{\pgfqpoint{1.150000in}{0.150000in}}{\pgfqpoint{5.700000in}{5.700000in}}%
\pgfusepath{clip}%
\pgfsetbuttcap%
\pgfsetroundjoin%
\definecolor{currentfill}{rgb}{0.243113,0.292092,0.538516}%
\pgfsetfillcolor{currentfill}%
\pgfsetfillopacity{0.700000}%
\pgfsetlinewidth{0.000000pt}%
\definecolor{currentstroke}{rgb}{0.000000,0.000000,0.000000}%
\pgfsetstrokecolor{currentstroke}%
\pgfsetdash{}{0pt}%
\pgfpathmoveto{\pgfqpoint{4.122872in}{3.021293in}}%
\pgfpathlineto{\pgfqpoint{4.136248in}{3.013579in}}%
\pgfpathlineto{\pgfqpoint{4.149628in}{3.005951in}}%
\pgfpathlineto{\pgfqpoint{4.163012in}{2.998408in}}%
\pgfpathlineto{\pgfqpoint{4.176400in}{2.990951in}}%
\pgfpathlineto{\pgfqpoint{4.184104in}{3.008569in}}%
\pgfpathlineto{\pgfqpoint{4.191806in}{3.026492in}}%
\pgfpathlineto{\pgfqpoint{4.199506in}{3.044726in}}%
\pgfpathlineto{\pgfqpoint{4.207204in}{3.063279in}}%
\pgfpathlineto{\pgfqpoint{4.193821in}{3.071133in}}%
\pgfpathlineto{\pgfqpoint{4.180442in}{3.079072in}}%
\pgfpathlineto{\pgfqpoint{4.167067in}{3.087096in}}%
\pgfpathlineto{\pgfqpoint{4.153695in}{3.095206in}}%
\pgfpathlineto{\pgfqpoint{4.145993in}{3.076249in}}%
\pgfpathlineto{\pgfqpoint{4.138288in}{3.057616in}}%
\pgfpathlineto{\pgfqpoint{4.130581in}{3.039300in}}%
\pgfpathlineto{\pgfqpoint{4.122872in}{3.021293in}}%
\pgfpathclose%
\pgfusepath{fill}%
\end{pgfscope}%
\begin{pgfscope}%
\pgfpathrectangle{\pgfqpoint{1.150000in}{0.150000in}}{\pgfqpoint{5.700000in}{5.700000in}}%
\pgfusepath{clip}%
\pgfsetbuttcap%
\pgfsetroundjoin%
\definecolor{currentfill}{rgb}{0.223925,0.334994,0.548053}%
\pgfsetfillcolor{currentfill}%
\pgfsetfillopacity{0.700000}%
\pgfsetlinewidth{0.000000pt}%
\definecolor{currentstroke}{rgb}{0.000000,0.000000,0.000000}%
\pgfsetstrokecolor{currentstroke}%
\pgfsetdash{}{0pt}%
\pgfpathmoveto{\pgfqpoint{4.291539in}{3.108600in}}%
\pgfpathlineto{\pgfqpoint{4.304937in}{3.100750in}}%
\pgfpathlineto{\pgfqpoint{4.318340in}{3.092983in}}%
\pgfpathlineto{\pgfqpoint{4.331748in}{3.085299in}}%
\pgfpathlineto{\pgfqpoint{4.345159in}{3.077697in}}%
\pgfpathlineto{\pgfqpoint{4.352843in}{3.097102in}}%
\pgfpathlineto{\pgfqpoint{4.360526in}{3.116862in}}%
\pgfpathlineto{\pgfqpoint{4.368210in}{3.136986in}}%
\pgfpathlineto{\pgfqpoint{4.375893in}{3.157481in}}%
\pgfpathlineto{\pgfqpoint{4.362487in}{3.165520in}}%
\pgfpathlineto{\pgfqpoint{4.349084in}{3.173642in}}%
\pgfpathlineto{\pgfqpoint{4.335685in}{3.181846in}}%
\pgfpathlineto{\pgfqpoint{4.322291in}{3.190133in}}%
\pgfpathlineto{\pgfqpoint{4.314603in}{3.169192in}}%
\pgfpathlineto{\pgfqpoint{4.306915in}{3.148629in}}%
\pgfpathlineto{\pgfqpoint{4.299227in}{3.128434in}}%
\pgfpathlineto{\pgfqpoint{4.291539in}{3.108600in}}%
\pgfpathclose%
\pgfusepath{fill}%
\end{pgfscope}%
\begin{pgfscope}%
\pgfpathrectangle{\pgfqpoint{1.150000in}{0.150000in}}{\pgfqpoint{5.700000in}{5.700000in}}%
\pgfusepath{clip}%
\pgfsetbuttcap%
\pgfsetroundjoin%
\definecolor{currentfill}{rgb}{0.187231,0.414746,0.556547}%
\pgfsetfillcolor{currentfill}%
\pgfsetfillopacity{0.700000}%
\pgfsetlinewidth{0.000000pt}%
\definecolor{currentstroke}{rgb}{0.000000,0.000000,0.000000}%
\pgfsetstrokecolor{currentstroke}%
\pgfsetdash{}{0pt}%
\pgfpathmoveto{\pgfqpoint{4.491036in}{3.300764in}}%
\pgfpathlineto{\pgfqpoint{4.504456in}{3.292194in}}%
\pgfpathlineto{\pgfqpoint{4.517880in}{3.283705in}}%
\pgfpathlineto{\pgfqpoint{4.531308in}{3.275295in}}%
\pgfpathlineto{\pgfqpoint{4.544740in}{3.266965in}}%
\pgfpathlineto{\pgfqpoint{4.552430in}{3.290211in}}%
\pgfpathlineto{\pgfqpoint{4.560122in}{3.313907in}}%
\pgfpathlineto{\pgfqpoint{4.567818in}{3.338062in}}%
\pgfpathlineto{\pgfqpoint{4.575518in}{3.362685in}}%
\pgfpathlineto{\pgfqpoint{4.562088in}{3.371515in}}%
\pgfpathlineto{\pgfqpoint{4.548663in}{3.380425in}}%
\pgfpathlineto{\pgfqpoint{4.535242in}{3.389416in}}%
\pgfpathlineto{\pgfqpoint{4.521825in}{3.398486in}}%
\pgfpathlineto{\pgfqpoint{4.514123in}{3.373354in}}%
\pgfpathlineto{\pgfqpoint{4.506424in}{3.348696in}}%
\pgfpathlineto{\pgfqpoint{4.498729in}{3.324503in}}%
\pgfpathlineto{\pgfqpoint{4.491036in}{3.300764in}}%
\pgfpathclose%
\pgfusepath{fill}%
\end{pgfscope}%
\begin{pgfscope}%
\pgfpathrectangle{\pgfqpoint{1.150000in}{0.150000in}}{\pgfqpoint{5.700000in}{5.700000in}}%
\pgfusepath{clip}%
\pgfsetbuttcap%
\pgfsetroundjoin%
\definecolor{currentfill}{rgb}{0.248629,0.278775,0.534556}%
\pgfsetfillcolor{currentfill}%
\pgfsetfillopacity{0.700000}%
\pgfsetlinewidth{0.000000pt}%
\definecolor{currentstroke}{rgb}{0.000000,0.000000,0.000000}%
\pgfsetstrokecolor{currentstroke}%
\pgfsetdash{}{0pt}%
\pgfpathmoveto{\pgfqpoint{4.038529in}{2.982440in}}%
\pgfpathlineto{\pgfqpoint{4.051894in}{2.974754in}}%
\pgfpathlineto{\pgfqpoint{4.065263in}{2.967155in}}%
\pgfpathlineto{\pgfqpoint{4.078636in}{2.959643in}}%
\pgfpathlineto{\pgfqpoint{4.092013in}{2.952218in}}%
\pgfpathlineto{\pgfqpoint{4.099732in}{2.969058in}}%
\pgfpathlineto{\pgfqpoint{4.107448in}{2.986179in}}%
\pgfpathlineto{\pgfqpoint{4.115161in}{3.003589in}}%
\pgfpathlineto{\pgfqpoint{4.122872in}{3.021293in}}%
\pgfpathlineto{\pgfqpoint{4.109501in}{3.029094in}}%
\pgfpathlineto{\pgfqpoint{4.096133in}{3.036981in}}%
\pgfpathlineto{\pgfqpoint{4.082768in}{3.044956in}}%
\pgfpathlineto{\pgfqpoint{4.069408in}{3.053019in}}%
\pgfpathlineto{\pgfqpoint{4.061692in}{3.034931in}}%
\pgfpathlineto{\pgfqpoint{4.053974in}{3.017143in}}%
\pgfpathlineto{\pgfqpoint{4.046253in}{2.999649in}}%
\pgfpathlineto{\pgfqpoint{4.038529in}{2.982440in}}%
\pgfpathclose%
\pgfusepath{fill}%
\end{pgfscope}%
\begin{pgfscope}%
\pgfpathrectangle{\pgfqpoint{1.150000in}{0.150000in}}{\pgfqpoint{5.700000in}{5.700000in}}%
\pgfusepath{clip}%
\pgfsetbuttcap%
\pgfsetroundjoin%
\definecolor{currentfill}{rgb}{0.121148,0.592739,0.544641}%
\pgfsetfillcolor{currentfill}%
\pgfsetfillopacity{0.700000}%
\pgfsetlinewidth{0.000000pt}%
\definecolor{currentstroke}{rgb}{0.000000,0.000000,0.000000}%
\pgfsetstrokecolor{currentstroke}%
\pgfsetdash{}{0pt}%
\pgfpathmoveto{\pgfqpoint{4.560986in}{3.784437in}}%
\pgfpathlineto{\pgfqpoint{4.574390in}{3.773398in}}%
\pgfpathlineto{\pgfqpoint{4.587797in}{3.762442in}}%
\pgfpathlineto{\pgfqpoint{4.601207in}{3.751568in}}%
\pgfpathlineto{\pgfqpoint{4.614621in}{3.740776in}}%
\pgfpathlineto{\pgfqpoint{4.622394in}{3.773018in}}%
\pgfpathlineto{\pgfqpoint{4.630176in}{3.805883in}}%
\pgfpathlineto{\pgfqpoint{4.637966in}{3.839385in}}%
\pgfpathlineto{\pgfqpoint{4.624550in}{3.850621in}}%
\pgfpathlineto{\pgfqpoint{4.611138in}{3.861939in}}%
\pgfpathlineto{\pgfqpoint{4.597729in}{3.873339in}}%
\pgfpathlineto{\pgfqpoint{4.584322in}{3.884823in}}%
\pgfpathlineto{\pgfqpoint{4.576535in}{3.850722in}}%
\pgfpathlineto{\pgfqpoint{4.568757in}{3.817264in}}%
\pgfpathlineto{\pgfqpoint{4.560986in}{3.784437in}}%
\pgfpathclose%
\pgfusepath{fill}%
\end{pgfscope}%
\begin{pgfscope}%
\pgfpathrectangle{\pgfqpoint{1.150000in}{0.150000in}}{\pgfqpoint{5.700000in}{5.700000in}}%
\pgfusepath{clip}%
\pgfsetbuttcap%
\pgfsetroundjoin%
\definecolor{currentfill}{rgb}{0.214298,0.355619,0.551184}%
\pgfsetfillcolor{currentfill}%
\pgfsetfillopacity{0.700000}%
\pgfsetlinewidth{0.000000pt}%
\definecolor{currentstroke}{rgb}{0.000000,0.000000,0.000000}%
\pgfsetstrokecolor{currentstroke}%
\pgfsetdash{}{0pt}%
\pgfpathmoveto{\pgfqpoint{4.375893in}{3.157481in}}%
\pgfpathlineto{\pgfqpoint{4.389305in}{3.149524in}}%
\pgfpathlineto{\pgfqpoint{4.402720in}{3.141648in}}%
\pgfpathlineto{\pgfqpoint{4.416140in}{3.133853in}}%
\pgfpathlineto{\pgfqpoint{4.429564in}{3.126139in}}%
\pgfpathlineto{\pgfqpoint{4.437243in}{3.146565in}}%
\pgfpathlineto{\pgfqpoint{4.444923in}{3.167374in}}%
\pgfpathlineto{\pgfqpoint{4.452605in}{3.188575in}}%
\pgfpathlineto{\pgfqpoint{4.460287in}{3.210176in}}%
\pgfpathlineto{\pgfqpoint{4.446868in}{3.218348in}}%
\pgfpathlineto{\pgfqpoint{4.433453in}{3.226600in}}%
\pgfpathlineto{\pgfqpoint{4.420041in}{3.234934in}}%
\pgfpathlineto{\pgfqpoint{4.406634in}{3.243349in}}%
\pgfpathlineto{\pgfqpoint{4.398948in}{3.221282in}}%
\pgfpathlineto{\pgfqpoint{4.391262in}{3.199621in}}%
\pgfpathlineto{\pgfqpoint{4.383578in}{3.178357in}}%
\pgfpathlineto{\pgfqpoint{4.375893in}{3.157481in}}%
\pgfpathclose%
\pgfusepath{fill}%
\end{pgfscope}%
\begin{pgfscope}%
\pgfpathrectangle{\pgfqpoint{1.150000in}{0.150000in}}{\pgfqpoint{5.700000in}{5.700000in}}%
\pgfusepath{clip}%
\pgfsetbuttcap%
\pgfsetroundjoin%
\definecolor{currentfill}{rgb}{0.246811,0.283237,0.535941}%
\pgfsetfillcolor{currentfill}%
\pgfsetfillopacity{0.700000}%
\pgfsetlinewidth{0.000000pt}%
\definecolor{currentstroke}{rgb}{0.000000,0.000000,0.000000}%
\pgfsetstrokecolor{currentstroke}%
\pgfsetdash{}{0pt}%
\pgfpathmoveto{\pgfqpoint{3.042585in}{3.006664in}}%
\pgfpathlineto{\pgfqpoint{3.055878in}{2.995356in}}%
\pgfpathlineto{\pgfqpoint{3.069171in}{2.984177in}}%
\pgfpathlineto{\pgfqpoint{3.082464in}{2.973125in}}%
\pgfpathlineto{\pgfqpoint{3.095757in}{2.962200in}}%
\pgfpathlineto{\pgfqpoint{3.103716in}{2.976051in}}%
\pgfpathlineto{\pgfqpoint{3.111668in}{2.990081in}}%
\pgfpathlineto{\pgfqpoint{3.119612in}{3.004293in}}%
\pgfpathlineto{\pgfqpoint{3.127550in}{3.018692in}}%
\pgfpathlineto{\pgfqpoint{3.114263in}{3.029834in}}%
\pgfpathlineto{\pgfqpoint{3.100976in}{3.041102in}}%
\pgfpathlineto{\pgfqpoint{3.087688in}{3.052498in}}%
\pgfpathlineto{\pgfqpoint{3.074400in}{3.064022in}}%
\pgfpathlineto{\pgfqpoint{3.066458in}{3.049399in}}%
\pgfpathlineto{\pgfqpoint{3.058507in}{3.034967in}}%
\pgfpathlineto{\pgfqpoint{3.050550in}{3.020724in}}%
\pgfpathlineto{\pgfqpoint{3.042585in}{3.006664in}}%
\pgfpathclose%
\pgfusepath{fill}%
\end{pgfscope}%
\begin{pgfscope}%
\pgfpathrectangle{\pgfqpoint{1.150000in}{0.150000in}}{\pgfqpoint{5.700000in}{5.700000in}}%
\pgfusepath{clip}%
\pgfsetbuttcap%
\pgfsetroundjoin%
\definecolor{currentfill}{rgb}{0.262138,0.242286,0.520837}%
\pgfsetfillcolor{currentfill}%
\pgfsetfillopacity{0.700000}%
\pgfsetlinewidth{0.000000pt}%
\definecolor{currentstroke}{rgb}{0.000000,0.000000,0.000000}%
\pgfsetstrokecolor{currentstroke}%
\pgfsetdash{}{0pt}%
\pgfpathmoveto{\pgfqpoint{3.731963in}{2.915088in}}%
\pgfpathlineto{\pgfqpoint{3.745288in}{2.906938in}}%
\pgfpathlineto{\pgfqpoint{3.758617in}{2.898883in}}%
\pgfpathlineto{\pgfqpoint{3.771948in}{2.890924in}}%
\pgfpathlineto{\pgfqpoint{3.785282in}{2.883059in}}%
\pgfpathlineto{\pgfqpoint{3.793068in}{2.898261in}}%
\pgfpathlineto{\pgfqpoint{3.800848in}{2.913689in}}%
\pgfpathlineto{\pgfqpoint{3.808625in}{2.929348in}}%
\pgfpathlineto{\pgfqpoint{3.816397in}{2.945244in}}%
\pgfpathlineto{\pgfqpoint{3.803067in}{2.953424in}}%
\pgfpathlineto{\pgfqpoint{3.789741in}{2.961699in}}%
\pgfpathlineto{\pgfqpoint{3.776418in}{2.970069in}}%
\pgfpathlineto{\pgfqpoint{3.763098in}{2.978535in}}%
\pgfpathlineto{\pgfqpoint{3.755321in}{2.962316in}}%
\pgfpathlineto{\pgfqpoint{3.747540in}{2.946339in}}%
\pgfpathlineto{\pgfqpoint{3.739754in}{2.930598in}}%
\pgfpathlineto{\pgfqpoint{3.731963in}{2.915088in}}%
\pgfpathclose%
\pgfusepath{fill}%
\end{pgfscope}%
\begin{pgfscope}%
\pgfpathrectangle{\pgfqpoint{1.150000in}{0.150000in}}{\pgfqpoint{5.700000in}{5.700000in}}%
\pgfusepath{clip}%
\pgfsetbuttcap%
\pgfsetroundjoin%
\definecolor{currentfill}{rgb}{0.137770,0.537492,0.554906}%
\pgfsetfillcolor{currentfill}%
\pgfsetfillopacity{0.700000}%
\pgfsetlinewidth{0.000000pt}%
\definecolor{currentstroke}{rgb}{0.000000,0.000000,0.000000}%
\pgfsetstrokecolor{currentstroke}%
\pgfsetdash{}{0pt}%
\pgfpathmoveto{\pgfqpoint{4.583598in}{3.617825in}}%
\pgfpathlineto{\pgfqpoint{4.597013in}{3.607685in}}%
\pgfpathlineto{\pgfqpoint{4.610432in}{3.597626in}}%
\pgfpathlineto{\pgfqpoint{4.623854in}{3.587648in}}%
\pgfpathlineto{\pgfqpoint{4.637280in}{3.577749in}}%
\pgfpathlineto{\pgfqpoint{4.645026in}{3.607048in}}%
\pgfpathlineto{\pgfqpoint{4.652779in}{3.636919in}}%
\pgfpathlineto{\pgfqpoint{4.660538in}{3.667373in}}%
\pgfpathlineto{\pgfqpoint{4.668306in}{3.698423in}}%
\pgfpathlineto{\pgfqpoint{4.654880in}{3.708890in}}%
\pgfpathlineto{\pgfqpoint{4.641457in}{3.719438in}}%
\pgfpathlineto{\pgfqpoint{4.628037in}{3.730067in}}%
\pgfpathlineto{\pgfqpoint{4.614621in}{3.740776in}}%
\pgfpathlineto{\pgfqpoint{4.606855in}{3.709148in}}%
\pgfpathlineto{\pgfqpoint{4.599096in}{3.678121in}}%
\pgfpathlineto{\pgfqpoint{4.591343in}{3.647684in}}%
\pgfpathlineto{\pgfqpoint{4.583598in}{3.617825in}}%
\pgfpathclose%
\pgfusepath{fill}%
\end{pgfscope}%
\begin{pgfscope}%
\pgfpathrectangle{\pgfqpoint{1.150000in}{0.150000in}}{\pgfqpoint{5.700000in}{5.700000in}}%
\pgfusepath{clip}%
\pgfsetbuttcap%
\pgfsetroundjoin%
\definecolor{currentfill}{rgb}{0.262138,0.242286,0.520837}%
\pgfsetfillcolor{currentfill}%
\pgfsetfillopacity{0.700000}%
\pgfsetlinewidth{0.000000pt}%
\definecolor{currentstroke}{rgb}{0.000000,0.000000,0.000000}%
\pgfsetstrokecolor{currentstroke}%
\pgfsetdash{}{0pt}%
\pgfpathmoveto{\pgfqpoint{3.371794in}{2.912569in}}%
\pgfpathlineto{\pgfqpoint{3.385088in}{2.903190in}}%
\pgfpathlineto{\pgfqpoint{3.398383in}{2.893922in}}%
\pgfpathlineto{\pgfqpoint{3.411681in}{2.884762in}}%
\pgfpathlineto{\pgfqpoint{3.424980in}{2.875710in}}%
\pgfpathlineto{\pgfqpoint{3.432857in}{2.889863in}}%
\pgfpathlineto{\pgfqpoint{3.440728in}{2.904205in}}%
\pgfpathlineto{\pgfqpoint{3.448593in}{2.918739in}}%
\pgfpathlineto{\pgfqpoint{3.456452in}{2.933472in}}%
\pgfpathlineto{\pgfqpoint{3.443158in}{2.942779in}}%
\pgfpathlineto{\pgfqpoint{3.429866in}{2.952195in}}%
\pgfpathlineto{\pgfqpoint{3.416576in}{2.961720in}}%
\pgfpathlineto{\pgfqpoint{3.403287in}{2.971355in}}%
\pgfpathlineto{\pgfqpoint{3.395423in}{2.956359in}}%
\pgfpathlineto{\pgfqpoint{3.387553in}{2.941566in}}%
\pgfpathlineto{\pgfqpoint{3.379676in}{2.926971in}}%
\pgfpathlineto{\pgfqpoint{3.371794in}{2.912569in}}%
\pgfpathclose%
\pgfusepath{fill}%
\end{pgfscope}%
\begin{pgfscope}%
\pgfpathrectangle{\pgfqpoint{1.150000in}{0.150000in}}{\pgfqpoint{5.700000in}{5.700000in}}%
\pgfusepath{clip}%
\pgfsetbuttcap%
\pgfsetroundjoin%
\definecolor{currentfill}{rgb}{0.255645,0.260703,0.528312}%
\pgfsetfillcolor{currentfill}%
\pgfsetfillopacity{0.700000}%
\pgfsetlinewidth{0.000000pt}%
\definecolor{currentstroke}{rgb}{0.000000,0.000000,0.000000}%
\pgfsetstrokecolor{currentstroke}%
\pgfsetdash{}{0pt}%
\pgfpathmoveto{\pgfqpoint{3.954158in}{2.946546in}}%
\pgfpathlineto{\pgfqpoint{3.967513in}{2.938859in}}%
\pgfpathlineto{\pgfqpoint{3.980872in}{2.931262in}}%
\pgfpathlineto{\pgfqpoint{3.994235in}{2.923754in}}%
\pgfpathlineto{\pgfqpoint{4.007601in}{2.916335in}}%
\pgfpathlineto{\pgfqpoint{4.015338in}{2.932466in}}%
\pgfpathlineto{\pgfqpoint{4.023072in}{2.948856in}}%
\pgfpathlineto{\pgfqpoint{4.030802in}{2.965512in}}%
\pgfpathlineto{\pgfqpoint{4.038529in}{2.982440in}}%
\pgfpathlineto{\pgfqpoint{4.025167in}{2.990215in}}%
\pgfpathlineto{\pgfqpoint{4.011810in}{2.998079in}}%
\pgfpathlineto{\pgfqpoint{3.998456in}{3.006032in}}%
\pgfpathlineto{\pgfqpoint{3.985105in}{3.014074in}}%
\pgfpathlineto{\pgfqpoint{3.977374in}{2.996782in}}%
\pgfpathlineto{\pgfqpoint{3.969638in}{2.979768in}}%
\pgfpathlineto{\pgfqpoint{3.961900in}{2.963024in}}%
\pgfpathlineto{\pgfqpoint{3.954158in}{2.946546in}}%
\pgfpathclose%
\pgfusepath{fill}%
\end{pgfscope}%
\begin{pgfscope}%
\pgfpathrectangle{\pgfqpoint{1.150000in}{0.150000in}}{\pgfqpoint{5.700000in}{5.700000in}}%
\pgfusepath{clip}%
\pgfsetbuttcap%
\pgfsetroundjoin%
\definecolor{currentfill}{rgb}{0.263663,0.237631,0.518762}%
\pgfsetfillcolor{currentfill}%
\pgfsetfillopacity{0.700000}%
\pgfsetlinewidth{0.000000pt}%
\definecolor{currentstroke}{rgb}{0.000000,0.000000,0.000000}%
\pgfsetstrokecolor{currentstroke}%
\pgfsetdash{}{0pt}%
\pgfpathmoveto{\pgfqpoint{3.509646in}{2.897306in}}%
\pgfpathlineto{\pgfqpoint{3.522950in}{2.888527in}}%
\pgfpathlineto{\pgfqpoint{3.536256in}{2.879852in}}%
\pgfpathlineto{\pgfqpoint{3.549565in}{2.871280in}}%
\pgfpathlineto{\pgfqpoint{3.562875in}{2.862810in}}%
\pgfpathlineto{\pgfqpoint{3.570718in}{2.877208in}}%
\pgfpathlineto{\pgfqpoint{3.578555in}{2.891804in}}%
\pgfpathlineto{\pgfqpoint{3.586387in}{2.906602in}}%
\pgfpathlineto{\pgfqpoint{3.594213in}{2.921608in}}%
\pgfpathlineto{\pgfqpoint{3.580907in}{2.930354in}}%
\pgfpathlineto{\pgfqpoint{3.567604in}{2.939202in}}%
\pgfpathlineto{\pgfqpoint{3.554303in}{2.948153in}}%
\pgfpathlineto{\pgfqpoint{3.541004in}{2.957208in}}%
\pgfpathlineto{\pgfqpoint{3.533173in}{2.941918in}}%
\pgfpathlineto{\pgfqpoint{3.525337in}{2.926841in}}%
\pgfpathlineto{\pgfqpoint{3.517494in}{2.911972in}}%
\pgfpathlineto{\pgfqpoint{3.509646in}{2.897306in}}%
\pgfpathclose%
\pgfusepath{fill}%
\end{pgfscope}%
\begin{pgfscope}%
\pgfpathrectangle{\pgfqpoint{1.150000in}{0.150000in}}{\pgfqpoint{5.700000in}{5.700000in}}%
\pgfusepath{clip}%
\pgfsetbuttcap%
\pgfsetroundjoin%
\definecolor{currentfill}{rgb}{0.258965,0.251537,0.524736}%
\pgfsetfillcolor{currentfill}%
\pgfsetfillopacity{0.700000}%
\pgfsetlinewidth{0.000000pt}%
\definecolor{currentstroke}{rgb}{0.000000,0.000000,0.000000}%
\pgfsetstrokecolor{currentstroke}%
\pgfsetdash{}{0pt}%
\pgfpathmoveto{\pgfqpoint{3.233850in}{2.933987in}}%
\pgfpathlineto{\pgfqpoint{3.247140in}{2.923937in}}%
\pgfpathlineto{\pgfqpoint{3.260431in}{2.914004in}}%
\pgfpathlineto{\pgfqpoint{3.273723in}{2.904186in}}%
\pgfpathlineto{\pgfqpoint{3.287016in}{2.894483in}}%
\pgfpathlineto{\pgfqpoint{3.294930in}{2.908387in}}%
\pgfpathlineto{\pgfqpoint{3.302837in}{2.922470in}}%
\pgfpathlineto{\pgfqpoint{3.310738in}{2.936738in}}%
\pgfpathlineto{\pgfqpoint{3.318632in}{2.951195in}}%
\pgfpathlineto{\pgfqpoint{3.305345in}{2.961134in}}%
\pgfpathlineto{\pgfqpoint{3.292058in}{2.971188in}}%
\pgfpathlineto{\pgfqpoint{3.278773in}{2.981357in}}%
\pgfpathlineto{\pgfqpoint{3.265488in}{2.991643in}}%
\pgfpathlineto{\pgfqpoint{3.257589in}{2.976943in}}%
\pgfpathlineto{\pgfqpoint{3.249683in}{2.962436in}}%
\pgfpathlineto{\pgfqpoint{3.241770in}{2.948119in}}%
\pgfpathlineto{\pgfqpoint{3.233850in}{2.933987in}}%
\pgfpathclose%
\pgfusepath{fill}%
\end{pgfscope}%
\begin{pgfscope}%
\pgfpathrectangle{\pgfqpoint{1.150000in}{0.150000in}}{\pgfqpoint{5.700000in}{5.700000in}}%
\pgfusepath{clip}%
\pgfsetbuttcap%
\pgfsetroundjoin%
\definecolor{currentfill}{rgb}{0.203063,0.379716,0.553925}%
\pgfsetfillcolor{currentfill}%
\pgfsetfillopacity{0.700000}%
\pgfsetlinewidth{0.000000pt}%
\definecolor{currentstroke}{rgb}{0.000000,0.000000,0.000000}%
\pgfsetstrokecolor{currentstroke}%
\pgfsetdash{}{0pt}%
\pgfpathmoveto{\pgfqpoint{4.460287in}{3.210176in}}%
\pgfpathlineto{\pgfqpoint{4.473711in}{3.202085in}}%
\pgfpathlineto{\pgfqpoint{4.487139in}{3.194074in}}%
\pgfpathlineto{\pgfqpoint{4.500572in}{3.186143in}}%
\pgfpathlineto{\pgfqpoint{4.514008in}{3.178292in}}%
\pgfpathlineto{\pgfqpoint{4.521688in}{3.199832in}}%
\pgfpathlineto{\pgfqpoint{4.529370in}{3.221785in}}%
\pgfpathlineto{\pgfqpoint{4.537054in}{3.244160in}}%
\pgfpathlineto{\pgfqpoint{4.544740in}{3.266965in}}%
\pgfpathlineto{\pgfqpoint{4.531308in}{3.275295in}}%
\pgfpathlineto{\pgfqpoint{4.517880in}{3.283705in}}%
\pgfpathlineto{\pgfqpoint{4.504456in}{3.292194in}}%
\pgfpathlineto{\pgfqpoint{4.491036in}{3.300764in}}%
\pgfpathlineto{\pgfqpoint{4.483346in}{3.277472in}}%
\pgfpathlineto{\pgfqpoint{4.475658in}{3.254616in}}%
\pgfpathlineto{\pgfqpoint{4.467972in}{3.232187in}}%
\pgfpathlineto{\pgfqpoint{4.460287in}{3.210176in}}%
\pgfpathclose%
\pgfusepath{fill}%
\end{pgfscope}%
\begin{pgfscope}%
\pgfpathrectangle{\pgfqpoint{1.150000in}{0.150000in}}{\pgfqpoint{5.700000in}{5.700000in}}%
\pgfusepath{clip}%
\pgfsetbuttcap%
\pgfsetroundjoin%
\definecolor{currentfill}{rgb}{0.160665,0.478540,0.558115}%
\pgfsetfillcolor{currentfill}%
\pgfsetfillopacity{0.700000}%
\pgfsetlinewidth{0.000000pt}%
\definecolor{currentstroke}{rgb}{0.000000,0.000000,0.000000}%
\pgfsetstrokecolor{currentstroke}%
\pgfsetdash{}{0pt}%
\pgfpathmoveto{\pgfqpoint{4.606357in}{3.466064in}}%
\pgfpathlineto{\pgfqpoint{4.619788in}{3.456791in}}%
\pgfpathlineto{\pgfqpoint{4.633223in}{3.447598in}}%
\pgfpathlineto{\pgfqpoint{4.646661in}{3.438483in}}%
\pgfpathlineto{\pgfqpoint{4.660104in}{3.429447in}}%
\pgfpathlineto{\pgfqpoint{4.667824in}{3.456027in}}%
\pgfpathlineto{\pgfqpoint{4.675550in}{3.483132in}}%
\pgfpathlineto{\pgfqpoint{4.683282in}{3.510770in}}%
\pgfpathlineto{\pgfqpoint{4.691020in}{3.538953in}}%
\pgfpathlineto{\pgfqpoint{4.677579in}{3.548534in}}%
\pgfpathlineto{\pgfqpoint{4.664143in}{3.558193in}}%
\pgfpathlineto{\pgfqpoint{4.650709in}{3.567931in}}%
\pgfpathlineto{\pgfqpoint{4.637280in}{3.577749in}}%
\pgfpathlineto{\pgfqpoint{4.629540in}{3.549013in}}%
\pgfpathlineto{\pgfqpoint{4.621807in}{3.520827in}}%
\pgfpathlineto{\pgfqpoint{4.614080in}{3.493181in}}%
\pgfpathlineto{\pgfqpoint{4.606357in}{3.466064in}}%
\pgfpathclose%
\pgfusepath{fill}%
\end{pgfscope}%
\begin{pgfscope}%
\pgfpathrectangle{\pgfqpoint{1.150000in}{0.150000in}}{\pgfqpoint{5.700000in}{5.700000in}}%
\pgfusepath{clip}%
\pgfsetbuttcap%
\pgfsetroundjoin%
\definecolor{currentfill}{rgb}{0.175841,0.441290,0.557685}%
\pgfsetfillcolor{currentfill}%
\pgfsetfillopacity{0.700000}%
\pgfsetlinewidth{0.000000pt}%
\definecolor{currentstroke}{rgb}{0.000000,0.000000,0.000000}%
\pgfsetstrokecolor{currentstroke}%
\pgfsetdash{}{0pt}%
\pgfpathmoveto{\pgfqpoint{4.575518in}{3.362685in}}%
\pgfpathlineto{\pgfqpoint{4.588951in}{3.353934in}}%
\pgfpathlineto{\pgfqpoint{4.602388in}{3.345263in}}%
\pgfpathlineto{\pgfqpoint{4.615829in}{3.336670in}}%
\pgfpathlineto{\pgfqpoint{4.629275in}{3.328156in}}%
\pgfpathlineto{\pgfqpoint{4.636975in}{3.352744in}}%
\pgfpathlineto{\pgfqpoint{4.644680in}{3.377816in}}%
\pgfpathlineto{\pgfqpoint{4.652389in}{3.403380in}}%
\pgfpathlineto{\pgfqpoint{4.660104in}{3.429447in}}%
\pgfpathlineto{\pgfqpoint{4.646661in}{3.438483in}}%
\pgfpathlineto{\pgfqpoint{4.633223in}{3.447598in}}%
\pgfpathlineto{\pgfqpoint{4.619788in}{3.456791in}}%
\pgfpathlineto{\pgfqpoint{4.606357in}{3.466064in}}%
\pgfpathlineto{\pgfqpoint{4.598641in}{3.439467in}}%
\pgfpathlineto{\pgfqpoint{4.590928in}{3.413378in}}%
\pgfpathlineto{\pgfqpoint{4.583221in}{3.387787in}}%
\pgfpathlineto{\pgfqpoint{4.575518in}{3.362685in}}%
\pgfpathclose%
\pgfusepath{fill}%
\end{pgfscope}%
\begin{pgfscope}%
\pgfpathrectangle{\pgfqpoint{1.150000in}{0.150000in}}{\pgfqpoint{5.700000in}{5.700000in}}%
\pgfusepath{clip}%
\pgfsetbuttcap%
\pgfsetroundjoin%
\definecolor{currentfill}{rgb}{0.260571,0.246922,0.522828}%
\pgfsetfillcolor{currentfill}%
\pgfsetfillopacity{0.700000}%
\pgfsetlinewidth{0.000000pt}%
\definecolor{currentstroke}{rgb}{0.000000,0.000000,0.000000}%
\pgfsetstrokecolor{currentstroke}%
\pgfsetdash{}{0pt}%
\pgfpathmoveto{\pgfqpoint{3.869747in}{2.913461in}}%
\pgfpathlineto{\pgfqpoint{3.883093in}{2.905746in}}%
\pgfpathlineto{\pgfqpoint{3.896442in}{2.898123in}}%
\pgfpathlineto{\pgfqpoint{3.909796in}{2.890592in}}%
\pgfpathlineto{\pgfqpoint{3.923152in}{2.883151in}}%
\pgfpathlineto{\pgfqpoint{3.930910in}{2.898634in}}%
\pgfpathlineto{\pgfqpoint{3.938663in}{2.914357in}}%
\pgfpathlineto{\pgfqpoint{3.946412in}{2.930326in}}%
\pgfpathlineto{\pgfqpoint{3.954158in}{2.946546in}}%
\pgfpathlineto{\pgfqpoint{3.940806in}{2.954323in}}%
\pgfpathlineto{\pgfqpoint{3.927458in}{2.962190in}}%
\pgfpathlineto{\pgfqpoint{3.914114in}{2.970148in}}%
\pgfpathlineto{\pgfqpoint{3.900773in}{2.978198in}}%
\pgfpathlineto{\pgfqpoint{3.893022in}{2.961635in}}%
\pgfpathlineto{\pgfqpoint{3.885268in}{2.945328in}}%
\pgfpathlineto{\pgfqpoint{3.877509in}{2.929272in}}%
\pgfpathlineto{\pgfqpoint{3.869747in}{2.913461in}}%
\pgfpathclose%
\pgfusepath{fill}%
\end{pgfscope}%
\begin{pgfscope}%
\pgfpathrectangle{\pgfqpoint{1.150000in}{0.150000in}}{\pgfqpoint{5.700000in}{5.700000in}}%
\pgfusepath{clip}%
\pgfsetbuttcap%
\pgfsetroundjoin%
\definecolor{currentfill}{rgb}{0.265145,0.232956,0.516599}%
\pgfsetfillcolor{currentfill}%
\pgfsetfillopacity{0.700000}%
\pgfsetlinewidth{0.000000pt}%
\definecolor{currentstroke}{rgb}{0.000000,0.000000,0.000000}%
\pgfsetstrokecolor{currentstroke}%
\pgfsetdash{}{0pt}%
\pgfpathmoveto{\pgfqpoint{3.647460in}{2.887634in}}%
\pgfpathlineto{\pgfqpoint{3.660779in}{2.879390in}}%
\pgfpathlineto{\pgfqpoint{3.674100in}{2.871243in}}%
\pgfpathlineto{\pgfqpoint{3.687425in}{2.863195in}}%
\pgfpathlineto{\pgfqpoint{3.700752in}{2.855243in}}%
\pgfpathlineto{\pgfqpoint{3.708562in}{2.869887in}}%
\pgfpathlineto{\pgfqpoint{3.716368in}{2.884738in}}%
\pgfpathlineto{\pgfqpoint{3.724168in}{2.899804in}}%
\pgfpathlineto{\pgfqpoint{3.731963in}{2.915088in}}%
\pgfpathlineto{\pgfqpoint{3.718641in}{2.923335in}}%
\pgfpathlineto{\pgfqpoint{3.705322in}{2.931680in}}%
\pgfpathlineto{\pgfqpoint{3.692006in}{2.940121in}}%
\pgfpathlineto{\pgfqpoint{3.678692in}{2.948662in}}%
\pgfpathlineto{\pgfqpoint{3.670892in}{2.933074in}}%
\pgfpathlineto{\pgfqpoint{3.663087in}{2.917710in}}%
\pgfpathlineto{\pgfqpoint{3.655276in}{2.902565in}}%
\pgfpathlineto{\pgfqpoint{3.647460in}{2.887634in}}%
\pgfpathclose%
\pgfusepath{fill}%
\end{pgfscope}%
\begin{pgfscope}%
\pgfpathrectangle{\pgfqpoint{1.150000in}{0.150000in}}{\pgfqpoint{5.700000in}{5.700000in}}%
\pgfusepath{clip}%
\pgfsetbuttcap%
\pgfsetroundjoin%
\definecolor{currentfill}{rgb}{0.124395,0.578002,0.548287}%
\pgfsetfillcolor{currentfill}%
\pgfsetfillopacity{0.700000}%
\pgfsetlinewidth{0.000000pt}%
\definecolor{currentstroke}{rgb}{0.000000,0.000000,0.000000}%
\pgfsetstrokecolor{currentstroke}%
\pgfsetdash{}{0pt}%
\pgfpathmoveto{\pgfqpoint{4.614621in}{3.740776in}}%
\pgfpathlineto{\pgfqpoint{4.628037in}{3.730067in}}%
\pgfpathlineto{\pgfqpoint{4.641457in}{3.719438in}}%
\pgfpathlineto{\pgfqpoint{4.654880in}{3.708890in}}%
\pgfpathlineto{\pgfqpoint{4.668306in}{3.698423in}}%
\pgfpathlineto{\pgfqpoint{4.676081in}{3.730080in}}%
\pgfpathlineto{\pgfqpoint{4.683865in}{3.762355in}}%
\pgfpathlineto{\pgfqpoint{4.691657in}{3.795260in}}%
\pgfpathlineto{\pgfqpoint{4.678230in}{3.806170in}}%
\pgfpathlineto{\pgfqpoint{4.664805in}{3.817160in}}%
\pgfpathlineto{\pgfqpoint{4.651384in}{3.828232in}}%
\pgfpathlineto{\pgfqpoint{4.637966in}{3.839385in}}%
\pgfpathlineto{\pgfqpoint{4.630176in}{3.805883in}}%
\pgfpathlineto{\pgfqpoint{4.622394in}{3.773018in}}%
\pgfpathlineto{\pgfqpoint{4.614621in}{3.740776in}}%
\pgfpathclose%
\pgfusepath{fill}%
\end{pgfscope}%
\begin{pgfscope}%
\pgfpathrectangle{\pgfqpoint{1.150000in}{0.150000in}}{\pgfqpoint{5.700000in}{5.700000in}}%
\pgfusepath{clip}%
\pgfsetbuttcap%
\pgfsetroundjoin%
\definecolor{currentfill}{rgb}{0.253935,0.265254,0.529983}%
\pgfsetfillcolor{currentfill}%
\pgfsetfillopacity{0.700000}%
\pgfsetlinewidth{0.000000pt}%
\definecolor{currentstroke}{rgb}{0.000000,0.000000,0.000000}%
\pgfsetstrokecolor{currentstroke}%
\pgfsetdash{}{0pt}%
\pgfpathmoveto{\pgfqpoint{3.095757in}{2.962200in}}%
\pgfpathlineto{\pgfqpoint{3.109049in}{2.951401in}}%
\pgfpathlineto{\pgfqpoint{3.122342in}{2.940725in}}%
\pgfpathlineto{\pgfqpoint{3.135635in}{2.930173in}}%
\pgfpathlineto{\pgfqpoint{3.148928in}{2.919743in}}%
\pgfpathlineto{\pgfqpoint{3.156881in}{2.933385in}}%
\pgfpathlineto{\pgfqpoint{3.164826in}{2.947201in}}%
\pgfpathlineto{\pgfqpoint{3.172765in}{2.961194in}}%
\pgfpathlineto{\pgfqpoint{3.180697in}{2.975369in}}%
\pgfpathlineto{\pgfqpoint{3.167410in}{2.986015in}}%
\pgfpathlineto{\pgfqpoint{3.154123in}{2.996784in}}%
\pgfpathlineto{\pgfqpoint{3.140836in}{3.007676in}}%
\pgfpathlineto{\pgfqpoint{3.127550in}{3.018692in}}%
\pgfpathlineto{\pgfqpoint{3.119612in}{3.004293in}}%
\pgfpathlineto{\pgfqpoint{3.111668in}{2.990081in}}%
\pgfpathlineto{\pgfqpoint{3.103716in}{2.976051in}}%
\pgfpathlineto{\pgfqpoint{3.095757in}{2.962200in}}%
\pgfpathclose%
\pgfusepath{fill}%
\end{pgfscope}%
\begin{pgfscope}%
\pgfpathrectangle{\pgfqpoint{1.150000in}{0.150000in}}{\pgfqpoint{5.700000in}{5.700000in}}%
\pgfusepath{clip}%
\pgfsetbuttcap%
\pgfsetroundjoin%
\definecolor{currentfill}{rgb}{0.239346,0.300855,0.540844}%
\pgfsetfillcolor{currentfill}%
\pgfsetfillopacity{0.700000}%
\pgfsetlinewidth{0.000000pt}%
\definecolor{currentstroke}{rgb}{0.000000,0.000000,0.000000}%
\pgfsetstrokecolor{currentstroke}%
\pgfsetdash{}{0pt}%
\pgfpathmoveto{\pgfqpoint{2.904223in}{3.045838in}}%
\pgfpathlineto{\pgfqpoint{2.917530in}{3.033656in}}%
\pgfpathlineto{\pgfqpoint{2.930836in}{3.021612in}}%
\pgfpathlineto{\pgfqpoint{2.944141in}{3.009706in}}%
\pgfpathlineto{\pgfqpoint{2.957445in}{2.997934in}}%
\pgfpathlineto{\pgfqpoint{2.965446in}{3.011488in}}%
\pgfpathlineto{\pgfqpoint{2.973440in}{3.025215in}}%
\pgfpathlineto{\pgfqpoint{2.981426in}{3.039118in}}%
\pgfpathlineto{\pgfqpoint{2.989405in}{3.053203in}}%
\pgfpathlineto{\pgfqpoint{2.976107in}{3.065171in}}%
\pgfpathlineto{\pgfqpoint{2.962809in}{3.077275in}}%
\pgfpathlineto{\pgfqpoint{2.949509in}{3.089515in}}%
\pgfpathlineto{\pgfqpoint{2.936208in}{3.101894in}}%
\pgfpathlineto{\pgfqpoint{2.928224in}{3.087605in}}%
\pgfpathlineto{\pgfqpoint{2.920231in}{3.073502in}}%
\pgfpathlineto{\pgfqpoint{2.912231in}{3.059581in}}%
\pgfpathlineto{\pgfqpoint{2.904223in}{3.045838in}}%
\pgfpathclose%
\pgfusepath{fill}%
\end{pgfscope}%
\begin{pgfscope}%
\pgfpathrectangle{\pgfqpoint{1.150000in}{0.150000in}}{\pgfqpoint{5.700000in}{5.700000in}}%
\pgfusepath{clip}%
\pgfsetbuttcap%
\pgfsetroundjoin%
\definecolor{currentfill}{rgb}{0.143343,0.522773,0.556295}%
\pgfsetfillcolor{currentfill}%
\pgfsetfillopacity{0.700000}%
\pgfsetlinewidth{0.000000pt}%
\definecolor{currentstroke}{rgb}{0.000000,0.000000,0.000000}%
\pgfsetstrokecolor{currentstroke}%
\pgfsetdash{}{0pt}%
\pgfpathmoveto{\pgfqpoint{4.637280in}{3.577749in}}%
\pgfpathlineto{\pgfqpoint{4.650709in}{3.567931in}}%
\pgfpathlineto{\pgfqpoint{4.664143in}{3.558193in}}%
\pgfpathlineto{\pgfqpoint{4.677579in}{3.548534in}}%
\pgfpathlineto{\pgfqpoint{4.691020in}{3.538953in}}%
\pgfpathlineto{\pgfqpoint{4.698765in}{3.567692in}}%
\pgfpathlineto{\pgfqpoint{4.706518in}{3.596998in}}%
\pgfpathlineto{\pgfqpoint{4.714278in}{3.626882in}}%
\pgfpathlineto{\pgfqpoint{4.722046in}{3.657354in}}%
\pgfpathlineto{\pgfqpoint{4.708605in}{3.667502in}}%
\pgfpathlineto{\pgfqpoint{4.695169in}{3.677730in}}%
\pgfpathlineto{\pgfqpoint{4.681736in}{3.688037in}}%
\pgfpathlineto{\pgfqpoint{4.668306in}{3.698423in}}%
\pgfpathlineto{\pgfqpoint{4.660538in}{3.667373in}}%
\pgfpathlineto{\pgfqpoint{4.652779in}{3.636919in}}%
\pgfpathlineto{\pgfqpoint{4.645026in}{3.607048in}}%
\pgfpathlineto{\pgfqpoint{4.637280in}{3.577749in}}%
\pgfpathclose%
\pgfusepath{fill}%
\end{pgfscope}%
\begin{pgfscope}%
\pgfpathrectangle{\pgfqpoint{1.150000in}{0.150000in}}{\pgfqpoint{5.700000in}{5.700000in}}%
\pgfusepath{clip}%
\pgfsetbuttcap%
\pgfsetroundjoin%
\definecolor{currentfill}{rgb}{0.192357,0.403199,0.555836}%
\pgfsetfillcolor{currentfill}%
\pgfsetfillopacity{0.700000}%
\pgfsetlinewidth{0.000000pt}%
\definecolor{currentstroke}{rgb}{0.000000,0.000000,0.000000}%
\pgfsetstrokecolor{currentstroke}%
\pgfsetdash{}{0pt}%
\pgfpathmoveto{\pgfqpoint{4.544740in}{3.266965in}}%
\pgfpathlineto{\pgfqpoint{4.558177in}{3.258715in}}%
\pgfpathlineto{\pgfqpoint{4.571618in}{3.250543in}}%
\pgfpathlineto{\pgfqpoint{4.585064in}{3.242450in}}%
\pgfpathlineto{\pgfqpoint{4.598514in}{3.234435in}}%
\pgfpathlineto{\pgfqpoint{4.606199in}{3.257190in}}%
\pgfpathlineto{\pgfqpoint{4.613887in}{3.280388in}}%
\pgfpathlineto{\pgfqpoint{4.621579in}{3.304040in}}%
\pgfpathlineto{\pgfqpoint{4.629275in}{3.328156in}}%
\pgfpathlineto{\pgfqpoint{4.615829in}{3.336670in}}%
\pgfpathlineto{\pgfqpoint{4.602388in}{3.345263in}}%
\pgfpathlineto{\pgfqpoint{4.588951in}{3.353934in}}%
\pgfpathlineto{\pgfqpoint{4.575518in}{3.362685in}}%
\pgfpathlineto{\pgfqpoint{4.567818in}{3.338062in}}%
\pgfpathlineto{\pgfqpoint{4.560122in}{3.313907in}}%
\pgfpathlineto{\pgfqpoint{4.552430in}{3.290211in}}%
\pgfpathlineto{\pgfqpoint{4.544740in}{3.266965in}}%
\pgfpathclose%
\pgfusepath{fill}%
\end{pgfscope}%
\begin{pgfscope}%
\pgfpathrectangle{\pgfqpoint{1.150000in}{0.150000in}}{\pgfqpoint{5.700000in}{5.700000in}}%
\pgfusepath{clip}%
\pgfsetbuttcap%
\pgfsetroundjoin%
\definecolor{currentfill}{rgb}{0.237441,0.305202,0.541921}%
\pgfsetfillcolor{currentfill}%
\pgfsetfillopacity{0.700000}%
\pgfsetlinewidth{0.000000pt}%
\definecolor{currentstroke}{rgb}{0.000000,0.000000,0.000000}%
\pgfsetstrokecolor{currentstroke}%
\pgfsetdash{}{0pt}%
\pgfpathmoveto{\pgfqpoint{4.260777in}{3.032710in}}%
\pgfpathlineto{\pgfqpoint{4.274181in}{3.025276in}}%
\pgfpathlineto{\pgfqpoint{4.287589in}{3.017926in}}%
\pgfpathlineto{\pgfqpoint{4.301002in}{3.010658in}}%
\pgfpathlineto{\pgfqpoint{4.314419in}{3.003472in}}%
\pgfpathlineto{\pgfqpoint{4.322105in}{3.021534in}}%
\pgfpathlineto{\pgfqpoint{4.329791in}{3.039921in}}%
\pgfpathlineto{\pgfqpoint{4.337475in}{3.058639in}}%
\pgfpathlineto{\pgfqpoint{4.345159in}{3.077697in}}%
\pgfpathlineto{\pgfqpoint{4.331748in}{3.085299in}}%
\pgfpathlineto{\pgfqpoint{4.318340in}{3.092983in}}%
\pgfpathlineto{\pgfqpoint{4.304937in}{3.100750in}}%
\pgfpathlineto{\pgfqpoint{4.291539in}{3.108600in}}%
\pgfpathlineto{\pgfqpoint{4.283849in}{3.089118in}}%
\pgfpathlineto{\pgfqpoint{4.276160in}{3.069981in}}%
\pgfpathlineto{\pgfqpoint{4.268469in}{3.051181in}}%
\pgfpathlineto{\pgfqpoint{4.260777in}{3.032710in}}%
\pgfpathclose%
\pgfusepath{fill}%
\end{pgfscope}%
\begin{pgfscope}%
\pgfpathrectangle{\pgfqpoint{1.150000in}{0.150000in}}{\pgfqpoint{5.700000in}{5.700000in}}%
\pgfusepath{clip}%
\pgfsetbuttcap%
\pgfsetroundjoin%
\definecolor{currentfill}{rgb}{0.229739,0.322361,0.545706}%
\pgfsetfillcolor{currentfill}%
\pgfsetfillopacity{0.700000}%
\pgfsetlinewidth{0.000000pt}%
\definecolor{currentstroke}{rgb}{0.000000,0.000000,0.000000}%
\pgfsetstrokecolor{currentstroke}%
\pgfsetdash{}{0pt}%
\pgfpathmoveto{\pgfqpoint{4.345159in}{3.077697in}}%
\pgfpathlineto{\pgfqpoint{4.358575in}{3.070176in}}%
\pgfpathlineto{\pgfqpoint{4.371996in}{3.062737in}}%
\pgfpathlineto{\pgfqpoint{4.385421in}{3.055379in}}%
\pgfpathlineto{\pgfqpoint{4.398850in}{3.048102in}}%
\pgfpathlineto{\pgfqpoint{4.406528in}{3.067078in}}%
\pgfpathlineto{\pgfqpoint{4.414206in}{3.086404in}}%
\pgfpathlineto{\pgfqpoint{4.421885in}{3.106089in}}%
\pgfpathlineto{\pgfqpoint{4.429564in}{3.126139in}}%
\pgfpathlineto{\pgfqpoint{4.416140in}{3.133853in}}%
\pgfpathlineto{\pgfqpoint{4.402720in}{3.141648in}}%
\pgfpathlineto{\pgfqpoint{4.389305in}{3.149524in}}%
\pgfpathlineto{\pgfqpoint{4.375893in}{3.157481in}}%
\pgfpathlineto{\pgfqpoint{4.368210in}{3.136986in}}%
\pgfpathlineto{\pgfqpoint{4.360526in}{3.116862in}}%
\pgfpathlineto{\pgfqpoint{4.352843in}{3.097102in}}%
\pgfpathlineto{\pgfqpoint{4.345159in}{3.077697in}}%
\pgfpathclose%
\pgfusepath{fill}%
\end{pgfscope}%
\begin{pgfscope}%
\pgfpathrectangle{\pgfqpoint{1.150000in}{0.150000in}}{\pgfqpoint{5.700000in}{5.700000in}}%
\pgfusepath{clip}%
\pgfsetbuttcap%
\pgfsetroundjoin%
\definecolor{currentfill}{rgb}{0.246811,0.283237,0.535941}%
\pgfsetfillcolor{currentfill}%
\pgfsetfillopacity{0.700000}%
\pgfsetlinewidth{0.000000pt}%
\definecolor{currentstroke}{rgb}{0.000000,0.000000,0.000000}%
\pgfsetstrokecolor{currentstroke}%
\pgfsetdash{}{0pt}%
\pgfpathmoveto{\pgfqpoint{4.176400in}{2.990951in}}%
\pgfpathlineto{\pgfqpoint{4.189792in}{2.983578in}}%
\pgfpathlineto{\pgfqpoint{4.203189in}{2.976290in}}%
\pgfpathlineto{\pgfqpoint{4.216589in}{2.969086in}}%
\pgfpathlineto{\pgfqpoint{4.229995in}{2.961965in}}%
\pgfpathlineto{\pgfqpoint{4.237693in}{2.979195in}}%
\pgfpathlineto{\pgfqpoint{4.245389in}{2.996724in}}%
\pgfpathlineto{\pgfqpoint{4.253084in}{3.014560in}}%
\pgfpathlineto{\pgfqpoint{4.260777in}{3.032710in}}%
\pgfpathlineto{\pgfqpoint{4.247377in}{3.040226in}}%
\pgfpathlineto{\pgfqpoint{4.233982in}{3.047827in}}%
\pgfpathlineto{\pgfqpoint{4.220591in}{3.055511in}}%
\pgfpathlineto{\pgfqpoint{4.207204in}{3.063279in}}%
\pgfpathlineto{\pgfqpoint{4.199506in}{3.044726in}}%
\pgfpathlineto{\pgfqpoint{4.191806in}{3.026492in}}%
\pgfpathlineto{\pgfqpoint{4.184104in}{3.008569in}}%
\pgfpathlineto{\pgfqpoint{4.176400in}{2.990951in}}%
\pgfpathclose%
\pgfusepath{fill}%
\end{pgfscope}%
\begin{pgfscope}%
\pgfpathrectangle{\pgfqpoint{1.150000in}{0.150000in}}{\pgfqpoint{5.700000in}{5.700000in}}%
\pgfusepath{clip}%
\pgfsetbuttcap%
\pgfsetroundjoin%
\definecolor{currentfill}{rgb}{0.218130,0.347432,0.550038}%
\pgfsetfillcolor{currentfill}%
\pgfsetfillopacity{0.700000}%
\pgfsetlinewidth{0.000000pt}%
\definecolor{currentstroke}{rgb}{0.000000,0.000000,0.000000}%
\pgfsetstrokecolor{currentstroke}%
\pgfsetdash{}{0pt}%
\pgfpathmoveto{\pgfqpoint{4.429564in}{3.126139in}}%
\pgfpathlineto{\pgfqpoint{4.442992in}{3.118506in}}%
\pgfpathlineto{\pgfqpoint{4.456425in}{3.110953in}}%
\pgfpathlineto{\pgfqpoint{4.469863in}{3.103479in}}%
\pgfpathlineto{\pgfqpoint{4.483305in}{3.096085in}}%
\pgfpathlineto{\pgfqpoint{4.490979in}{3.116061in}}%
\pgfpathlineto{\pgfqpoint{4.498654in}{3.136415in}}%
\pgfpathlineto{\pgfqpoint{4.506330in}{3.157156in}}%
\pgfpathlineto{\pgfqpoint{4.514008in}{3.178292in}}%
\pgfpathlineto{\pgfqpoint{4.500572in}{3.186143in}}%
\pgfpathlineto{\pgfqpoint{4.487139in}{3.194074in}}%
\pgfpathlineto{\pgfqpoint{4.473711in}{3.202085in}}%
\pgfpathlineto{\pgfqpoint{4.460287in}{3.210176in}}%
\pgfpathlineto{\pgfqpoint{4.452605in}{3.188575in}}%
\pgfpathlineto{\pgfqpoint{4.444923in}{3.167374in}}%
\pgfpathlineto{\pgfqpoint{4.437243in}{3.146565in}}%
\pgfpathlineto{\pgfqpoint{4.429564in}{3.126139in}}%
\pgfpathclose%
\pgfusepath{fill}%
\end{pgfscope}%
\begin{pgfscope}%
\pgfpathrectangle{\pgfqpoint{1.150000in}{0.150000in}}{\pgfqpoint{5.700000in}{5.700000in}}%
\pgfusepath{clip}%
\pgfsetbuttcap%
\pgfsetroundjoin%
\definecolor{currentfill}{rgb}{0.252194,0.269783,0.531579}%
\pgfsetfillcolor{currentfill}%
\pgfsetfillopacity{0.700000}%
\pgfsetlinewidth{0.000000pt}%
\definecolor{currentstroke}{rgb}{0.000000,0.000000,0.000000}%
\pgfsetstrokecolor{currentstroke}%
\pgfsetdash{}{0pt}%
\pgfpathmoveto{\pgfqpoint{4.092013in}{2.952218in}}%
\pgfpathlineto{\pgfqpoint{4.105394in}{2.944880in}}%
\pgfpathlineto{\pgfqpoint{4.118779in}{2.937627in}}%
\pgfpathlineto{\pgfqpoint{4.132169in}{2.930461in}}%
\pgfpathlineto{\pgfqpoint{4.145562in}{2.923379in}}%
\pgfpathlineto{\pgfqpoint{4.153275in}{2.939851in}}%
\pgfpathlineto{\pgfqpoint{4.160986in}{2.956599in}}%
\pgfpathlineto{\pgfqpoint{4.168694in}{2.973630in}}%
\pgfpathlineto{\pgfqpoint{4.176400in}{2.990951in}}%
\pgfpathlineto{\pgfqpoint{4.163012in}{2.998408in}}%
\pgfpathlineto{\pgfqpoint{4.149628in}{3.005951in}}%
\pgfpathlineto{\pgfqpoint{4.136248in}{3.013579in}}%
\pgfpathlineto{\pgfqpoint{4.122872in}{3.021293in}}%
\pgfpathlineto{\pgfqpoint{4.115161in}{3.003589in}}%
\pgfpathlineto{\pgfqpoint{4.107448in}{2.986179in}}%
\pgfpathlineto{\pgfqpoint{4.099732in}{2.969058in}}%
\pgfpathlineto{\pgfqpoint{4.092013in}{2.952218in}}%
\pgfpathclose%
\pgfusepath{fill}%
\end{pgfscope}%
\begin{pgfscope}%
\pgfpathrectangle{\pgfqpoint{1.150000in}{0.150000in}}{\pgfqpoint{5.700000in}{5.700000in}}%
\pgfusepath{clip}%
\pgfsetbuttcap%
\pgfsetroundjoin%
\definecolor{currentfill}{rgb}{0.265145,0.232956,0.516599}%
\pgfsetfillcolor{currentfill}%
\pgfsetfillopacity{0.700000}%
\pgfsetlinewidth{0.000000pt}%
\definecolor{currentstroke}{rgb}{0.000000,0.000000,0.000000}%
\pgfsetstrokecolor{currentstroke}%
\pgfsetdash{}{0pt}%
\pgfpathmoveto{\pgfqpoint{3.785282in}{2.883059in}}%
\pgfpathlineto{\pgfqpoint{3.798620in}{2.875289in}}%
\pgfpathlineto{\pgfqpoint{3.811961in}{2.867613in}}%
\pgfpathlineto{\pgfqpoint{3.825306in}{2.860029in}}%
\pgfpathlineto{\pgfqpoint{3.838654in}{2.852538in}}%
\pgfpathlineto{\pgfqpoint{3.846433in}{2.867432in}}%
\pgfpathlineto{\pgfqpoint{3.854209in}{2.882546in}}%
\pgfpathlineto{\pgfqpoint{3.861980in}{2.897887in}}%
\pgfpathlineto{\pgfqpoint{3.869747in}{2.913461in}}%
\pgfpathlineto{\pgfqpoint{3.856404in}{2.921267in}}%
\pgfpathlineto{\pgfqpoint{3.843065in}{2.929166in}}%
\pgfpathlineto{\pgfqpoint{3.829729in}{2.937158in}}%
\pgfpathlineto{\pgfqpoint{3.816397in}{2.945244in}}%
\pgfpathlineto{\pgfqpoint{3.808625in}{2.929348in}}%
\pgfpathlineto{\pgfqpoint{3.800848in}{2.913689in}}%
\pgfpathlineto{\pgfqpoint{3.793068in}{2.898261in}}%
\pgfpathlineto{\pgfqpoint{3.785282in}{2.883059in}}%
\pgfpathclose%
\pgfusepath{fill}%
\end{pgfscope}%
\begin{pgfscope}%
\pgfpathrectangle{\pgfqpoint{1.150000in}{0.150000in}}{\pgfqpoint{5.700000in}{5.700000in}}%
\pgfusepath{clip}%
\pgfsetbuttcap%
\pgfsetroundjoin%
\definecolor{currentfill}{rgb}{0.266580,0.228262,0.514349}%
\pgfsetfillcolor{currentfill}%
\pgfsetfillopacity{0.700000}%
\pgfsetlinewidth{0.000000pt}%
\definecolor{currentstroke}{rgb}{0.000000,0.000000,0.000000}%
\pgfsetstrokecolor{currentstroke}%
\pgfsetdash{}{0pt}%
\pgfpathmoveto{\pgfqpoint{3.424980in}{2.875710in}}%
\pgfpathlineto{\pgfqpoint{3.438281in}{2.866765in}}%
\pgfpathlineto{\pgfqpoint{3.451584in}{2.857927in}}%
\pgfpathlineto{\pgfqpoint{3.464889in}{2.849194in}}%
\pgfpathlineto{\pgfqpoint{3.478196in}{2.840567in}}%
\pgfpathlineto{\pgfqpoint{3.486067in}{2.854472in}}%
\pgfpathlineto{\pgfqpoint{3.493933in}{2.868561in}}%
\pgfpathlineto{\pgfqpoint{3.501793in}{2.882837in}}%
\pgfpathlineto{\pgfqpoint{3.509646in}{2.897306in}}%
\pgfpathlineto{\pgfqpoint{3.496345in}{2.906189in}}%
\pgfpathlineto{\pgfqpoint{3.483045in}{2.915177in}}%
\pgfpathlineto{\pgfqpoint{3.469748in}{2.924271in}}%
\pgfpathlineto{\pgfqpoint{3.456452in}{2.933472in}}%
\pgfpathlineto{\pgfqpoint{3.448593in}{2.918739in}}%
\pgfpathlineto{\pgfqpoint{3.440728in}{2.904205in}}%
\pgfpathlineto{\pgfqpoint{3.432857in}{2.889863in}}%
\pgfpathlineto{\pgfqpoint{3.424980in}{2.875710in}}%
\pgfpathclose%
\pgfusepath{fill}%
\end{pgfscope}%
\begin{pgfscope}%
\pgfpathrectangle{\pgfqpoint{1.150000in}{0.150000in}}{\pgfqpoint{5.700000in}{5.700000in}}%
\pgfusepath{clip}%
\pgfsetbuttcap%
\pgfsetroundjoin%
\definecolor{currentfill}{rgb}{0.263663,0.237631,0.518762}%
\pgfsetfillcolor{currentfill}%
\pgfsetfillopacity{0.700000}%
\pgfsetlinewidth{0.000000pt}%
\definecolor{currentstroke}{rgb}{0.000000,0.000000,0.000000}%
\pgfsetstrokecolor{currentstroke}%
\pgfsetdash{}{0pt}%
\pgfpathmoveto{\pgfqpoint{3.287016in}{2.894483in}}%
\pgfpathlineto{\pgfqpoint{3.300310in}{2.884894in}}%
\pgfpathlineto{\pgfqpoint{3.313606in}{2.875418in}}%
\pgfpathlineto{\pgfqpoint{3.326902in}{2.866054in}}%
\pgfpathlineto{\pgfqpoint{3.340200in}{2.856801in}}%
\pgfpathlineto{\pgfqpoint{3.348108in}{2.870476in}}%
\pgfpathlineto{\pgfqpoint{3.356010in}{2.884326in}}%
\pgfpathlineto{\pgfqpoint{3.363905in}{2.898355in}}%
\pgfpathlineto{\pgfqpoint{3.371794in}{2.912569in}}%
\pgfpathlineto{\pgfqpoint{3.358501in}{2.922058in}}%
\pgfpathlineto{\pgfqpoint{3.345210in}{2.931658in}}%
\pgfpathlineto{\pgfqpoint{3.331921in}{2.941370in}}%
\pgfpathlineto{\pgfqpoint{3.318632in}{2.951195in}}%
\pgfpathlineto{\pgfqpoint{3.310738in}{2.936738in}}%
\pgfpathlineto{\pgfqpoint{3.302837in}{2.922470in}}%
\pgfpathlineto{\pgfqpoint{3.294930in}{2.908387in}}%
\pgfpathlineto{\pgfqpoint{3.287016in}{2.894483in}}%
\pgfpathclose%
\pgfusepath{fill}%
\end{pgfscope}%
\begin{pgfscope}%
\pgfpathrectangle{\pgfqpoint{1.150000in}{0.150000in}}{\pgfqpoint{5.700000in}{5.700000in}}%
\pgfusepath{clip}%
\pgfsetbuttcap%
\pgfsetroundjoin%
\definecolor{currentfill}{rgb}{0.246811,0.283237,0.535941}%
\pgfsetfillcolor{currentfill}%
\pgfsetfillopacity{0.700000}%
\pgfsetlinewidth{0.000000pt}%
\definecolor{currentstroke}{rgb}{0.000000,0.000000,0.000000}%
\pgfsetstrokecolor{currentstroke}%
\pgfsetdash{}{0pt}%
\pgfpathmoveto{\pgfqpoint{2.957445in}{2.997934in}}%
\pgfpathlineto{\pgfqpoint{2.970747in}{2.986297in}}%
\pgfpathlineto{\pgfqpoint{2.984049in}{2.974794in}}%
\pgfpathlineto{\pgfqpoint{2.997350in}{2.963422in}}%
\pgfpathlineto{\pgfqpoint{3.010650in}{2.952181in}}%
\pgfpathlineto{\pgfqpoint{3.018645in}{2.965546in}}%
\pgfpathlineto{\pgfqpoint{3.026633in}{2.979078in}}%
\pgfpathlineto{\pgfqpoint{3.034612in}{2.992783in}}%
\pgfpathlineto{\pgfqpoint{3.042585in}{3.006664in}}%
\pgfpathlineto{\pgfqpoint{3.029291in}{3.018101in}}%
\pgfpathlineto{\pgfqpoint{3.015996in}{3.029669in}}%
\pgfpathlineto{\pgfqpoint{3.002701in}{3.041370in}}%
\pgfpathlineto{\pgfqpoint{2.989405in}{3.053203in}}%
\pgfpathlineto{\pgfqpoint{2.981426in}{3.039118in}}%
\pgfpathlineto{\pgfqpoint{2.973440in}{3.025215in}}%
\pgfpathlineto{\pgfqpoint{2.965446in}{3.011488in}}%
\pgfpathlineto{\pgfqpoint{2.957445in}{2.997934in}}%
\pgfpathclose%
\pgfusepath{fill}%
\end{pgfscope}%
\begin{pgfscope}%
\pgfpathrectangle{\pgfqpoint{1.150000in}{0.150000in}}{\pgfqpoint{5.700000in}{5.700000in}}%
\pgfusepath{clip}%
\pgfsetbuttcap%
\pgfsetroundjoin%
\definecolor{currentfill}{rgb}{0.127568,0.566949,0.550556}%
\pgfsetfillcolor{currentfill}%
\pgfsetfillopacity{0.700000}%
\pgfsetlinewidth{0.000000pt}%
\definecolor{currentstroke}{rgb}{0.000000,0.000000,0.000000}%
\pgfsetstrokecolor{currentstroke}%
\pgfsetdash{}{0pt}%
\pgfpathmoveto{\pgfqpoint{4.668306in}{3.698423in}}%
\pgfpathlineto{\pgfqpoint{4.681736in}{3.688037in}}%
\pgfpathlineto{\pgfqpoint{4.695169in}{3.677730in}}%
\pgfpathlineto{\pgfqpoint{4.708605in}{3.667502in}}%
\pgfpathlineto{\pgfqpoint{4.722046in}{3.657354in}}%
\pgfpathlineto{\pgfqpoint{4.729822in}{3.688428in}}%
\pgfpathlineto{\pgfqpoint{4.737606in}{3.720114in}}%
\pgfpathlineto{\pgfqpoint{4.745400in}{3.752423in}}%
\pgfpathlineto{\pgfqpoint{4.731959in}{3.763013in}}%
\pgfpathlineto{\pgfqpoint{4.718522in}{3.773682in}}%
\pgfpathlineto{\pgfqpoint{4.705088in}{3.784431in}}%
\pgfpathlineto{\pgfqpoint{4.691657in}{3.795260in}}%
\pgfpathlineto{\pgfqpoint{4.683865in}{3.762355in}}%
\pgfpathlineto{\pgfqpoint{4.676081in}{3.730080in}}%
\pgfpathlineto{\pgfqpoint{4.668306in}{3.698423in}}%
\pgfpathclose%
\pgfusepath{fill}%
\end{pgfscope}%
\begin{pgfscope}%
\pgfpathrectangle{\pgfqpoint{1.150000in}{0.150000in}}{\pgfqpoint{5.700000in}{5.700000in}}%
\pgfusepath{clip}%
\pgfsetbuttcap%
\pgfsetroundjoin%
\definecolor{currentfill}{rgb}{0.267968,0.223549,0.512008}%
\pgfsetfillcolor{currentfill}%
\pgfsetfillopacity{0.700000}%
\pgfsetlinewidth{0.000000pt}%
\definecolor{currentstroke}{rgb}{0.000000,0.000000,0.000000}%
\pgfsetstrokecolor{currentstroke}%
\pgfsetdash{}{0pt}%
\pgfpathmoveto{\pgfqpoint{3.562875in}{2.862810in}}%
\pgfpathlineto{\pgfqpoint{3.576189in}{2.854441in}}%
\pgfpathlineto{\pgfqpoint{3.589505in}{2.846173in}}%
\pgfpathlineto{\pgfqpoint{3.602823in}{2.838006in}}%
\pgfpathlineto{\pgfqpoint{3.616145in}{2.829938in}}%
\pgfpathlineto{\pgfqpoint{3.623982in}{2.844068in}}%
\pgfpathlineto{\pgfqpoint{3.631813in}{2.858391in}}%
\pgfpathlineto{\pgfqpoint{3.639639in}{2.872911in}}%
\pgfpathlineto{\pgfqpoint{3.647460in}{2.887634in}}%
\pgfpathlineto{\pgfqpoint{3.634145in}{2.895978in}}%
\pgfpathlineto{\pgfqpoint{3.620832in}{2.904421in}}%
\pgfpathlineto{\pgfqpoint{3.607521in}{2.912964in}}%
\pgfpathlineto{\pgfqpoint{3.594213in}{2.921608in}}%
\pgfpathlineto{\pgfqpoint{3.586387in}{2.906602in}}%
\pgfpathlineto{\pgfqpoint{3.578555in}{2.891804in}}%
\pgfpathlineto{\pgfqpoint{3.570718in}{2.877208in}}%
\pgfpathlineto{\pgfqpoint{3.562875in}{2.862810in}}%
\pgfpathclose%
\pgfusepath{fill}%
\end{pgfscope}%
\begin{pgfscope}%
\pgfpathrectangle{\pgfqpoint{1.150000in}{0.150000in}}{\pgfqpoint{5.700000in}{5.700000in}}%
\pgfusepath{clip}%
\pgfsetbuttcap%
\pgfsetroundjoin%
\definecolor{currentfill}{rgb}{0.258965,0.251537,0.524736}%
\pgfsetfillcolor{currentfill}%
\pgfsetfillopacity{0.700000}%
\pgfsetlinewidth{0.000000pt}%
\definecolor{currentstroke}{rgb}{0.000000,0.000000,0.000000}%
\pgfsetstrokecolor{currentstroke}%
\pgfsetdash{}{0pt}%
\pgfpathmoveto{\pgfqpoint{4.007601in}{2.916335in}}%
\pgfpathlineto{\pgfqpoint{4.020972in}{2.909004in}}%
\pgfpathlineto{\pgfqpoint{4.034347in}{2.901761in}}%
\pgfpathlineto{\pgfqpoint{4.047725in}{2.894605in}}%
\pgfpathlineto{\pgfqpoint{4.061108in}{2.887535in}}%
\pgfpathlineto{\pgfqpoint{4.068839in}{2.903318in}}%
\pgfpathlineto{\pgfqpoint{4.076567in}{2.919355in}}%
\pgfpathlineto{\pgfqpoint{4.084291in}{2.935653in}}%
\pgfpathlineto{\pgfqpoint{4.092013in}{2.952218in}}%
\pgfpathlineto{\pgfqpoint{4.078636in}{2.959643in}}%
\pgfpathlineto{\pgfqpoint{4.065263in}{2.967155in}}%
\pgfpathlineto{\pgfqpoint{4.051894in}{2.974754in}}%
\pgfpathlineto{\pgfqpoint{4.038529in}{2.982440in}}%
\pgfpathlineto{\pgfqpoint{4.030802in}{2.965512in}}%
\pgfpathlineto{\pgfqpoint{4.023072in}{2.948856in}}%
\pgfpathlineto{\pgfqpoint{4.015338in}{2.932466in}}%
\pgfpathlineto{\pgfqpoint{4.007601in}{2.916335in}}%
\pgfpathclose%
\pgfusepath{fill}%
\end{pgfscope}%
\begin{pgfscope}%
\pgfpathrectangle{\pgfqpoint{1.150000in}{0.150000in}}{\pgfqpoint{5.700000in}{5.700000in}}%
\pgfusepath{clip}%
\pgfsetbuttcap%
\pgfsetroundjoin%
\definecolor{currentfill}{rgb}{0.260571,0.246922,0.522828}%
\pgfsetfillcolor{currentfill}%
\pgfsetfillopacity{0.700000}%
\pgfsetlinewidth{0.000000pt}%
\definecolor{currentstroke}{rgb}{0.000000,0.000000,0.000000}%
\pgfsetstrokecolor{currentstroke}%
\pgfsetdash{}{0pt}%
\pgfpathmoveto{\pgfqpoint{3.148928in}{2.919743in}}%
\pgfpathlineto{\pgfqpoint{3.162221in}{2.909434in}}%
\pgfpathlineto{\pgfqpoint{3.175515in}{2.899245in}}%
\pgfpathlineto{\pgfqpoint{3.188810in}{2.889176in}}%
\pgfpathlineto{\pgfqpoint{3.202105in}{2.879225in}}%
\pgfpathlineto{\pgfqpoint{3.210051in}{2.892659in}}%
\pgfpathlineto{\pgfqpoint{3.217991in}{2.906261in}}%
\pgfpathlineto{\pgfqpoint{3.225924in}{2.920036in}}%
\pgfpathlineto{\pgfqpoint{3.233850in}{2.933987in}}%
\pgfpathlineto{\pgfqpoint{3.220561in}{2.944154in}}%
\pgfpathlineto{\pgfqpoint{3.207273in}{2.954439in}}%
\pgfpathlineto{\pgfqpoint{3.193985in}{2.964844in}}%
\pgfpathlineto{\pgfqpoint{3.180697in}{2.975369in}}%
\pgfpathlineto{\pgfqpoint{3.172765in}{2.961194in}}%
\pgfpathlineto{\pgfqpoint{3.164826in}{2.947201in}}%
\pgfpathlineto{\pgfqpoint{3.156881in}{2.933385in}}%
\pgfpathlineto{\pgfqpoint{3.148928in}{2.919743in}}%
\pgfpathclose%
\pgfusepath{fill}%
\end{pgfscope}%
\begin{pgfscope}%
\pgfpathrectangle{\pgfqpoint{1.150000in}{0.150000in}}{\pgfqpoint{5.700000in}{5.700000in}}%
\pgfusepath{clip}%
\pgfsetbuttcap%
\pgfsetroundjoin%
\definecolor{currentfill}{rgb}{0.163625,0.471133,0.558148}%
\pgfsetfillcolor{currentfill}%
\pgfsetfillopacity{0.700000}%
\pgfsetlinewidth{0.000000pt}%
\definecolor{currentstroke}{rgb}{0.000000,0.000000,0.000000}%
\pgfsetstrokecolor{currentstroke}%
\pgfsetdash{}{0pt}%
\pgfpathmoveto{\pgfqpoint{4.660104in}{3.429447in}}%
\pgfpathlineto{\pgfqpoint{4.673551in}{3.420489in}}%
\pgfpathlineto{\pgfqpoint{4.687002in}{3.411610in}}%
\pgfpathlineto{\pgfqpoint{4.700457in}{3.402808in}}%
\pgfpathlineto{\pgfqpoint{4.713916in}{3.394084in}}%
\pgfpathlineto{\pgfqpoint{4.721633in}{3.420129in}}%
\pgfpathlineto{\pgfqpoint{4.729356in}{3.446692in}}%
\pgfpathlineto{\pgfqpoint{4.737086in}{3.473784in}}%
\pgfpathlineto{\pgfqpoint{4.744822in}{3.501415in}}%
\pgfpathlineto{\pgfqpoint{4.731366in}{3.510682in}}%
\pgfpathlineto{\pgfqpoint{4.717913in}{3.520028in}}%
\pgfpathlineto{\pgfqpoint{4.704465in}{3.529451in}}%
\pgfpathlineto{\pgfqpoint{4.691020in}{3.538953in}}%
\pgfpathlineto{\pgfqpoint{4.683282in}{3.510770in}}%
\pgfpathlineto{\pgfqpoint{4.675550in}{3.483132in}}%
\pgfpathlineto{\pgfqpoint{4.667824in}{3.456027in}}%
\pgfpathlineto{\pgfqpoint{4.660104in}{3.429447in}}%
\pgfpathclose%
\pgfusepath{fill}%
\end{pgfscope}%
\begin{pgfscope}%
\pgfpathrectangle{\pgfqpoint{1.150000in}{0.150000in}}{\pgfqpoint{5.700000in}{5.700000in}}%
\pgfusepath{clip}%
\pgfsetbuttcap%
\pgfsetroundjoin%
\definecolor{currentfill}{rgb}{0.208623,0.367752,0.552675}%
\pgfsetfillcolor{currentfill}%
\pgfsetfillopacity{0.700000}%
\pgfsetlinewidth{0.000000pt}%
\definecolor{currentstroke}{rgb}{0.000000,0.000000,0.000000}%
\pgfsetstrokecolor{currentstroke}%
\pgfsetdash{}{0pt}%
\pgfpathmoveto{\pgfqpoint{4.514008in}{3.178292in}}%
\pgfpathlineto{\pgfqpoint{4.527450in}{3.170520in}}%
\pgfpathlineto{\pgfqpoint{4.540896in}{3.162827in}}%
\pgfpathlineto{\pgfqpoint{4.554346in}{3.155212in}}%
\pgfpathlineto{\pgfqpoint{4.567802in}{3.147675in}}%
\pgfpathlineto{\pgfqpoint{4.575476in}{3.168745in}}%
\pgfpathlineto{\pgfqpoint{4.583153in}{3.190222in}}%
\pgfpathlineto{\pgfqpoint{4.590832in}{3.212116in}}%
\pgfpathlineto{\pgfqpoint{4.598514in}{3.234435in}}%
\pgfpathlineto{\pgfqpoint{4.585064in}{3.242450in}}%
\pgfpathlineto{\pgfqpoint{4.571618in}{3.250543in}}%
\pgfpathlineto{\pgfqpoint{4.558177in}{3.258715in}}%
\pgfpathlineto{\pgfqpoint{4.544740in}{3.266965in}}%
\pgfpathlineto{\pgfqpoint{4.537054in}{3.244160in}}%
\pgfpathlineto{\pgfqpoint{4.529370in}{3.221785in}}%
\pgfpathlineto{\pgfqpoint{4.521688in}{3.199832in}}%
\pgfpathlineto{\pgfqpoint{4.514008in}{3.178292in}}%
\pgfpathclose%
\pgfusepath{fill}%
\end{pgfscope}%
\begin{pgfscope}%
\pgfpathrectangle{\pgfqpoint{1.150000in}{0.150000in}}{\pgfqpoint{5.700000in}{5.700000in}}%
\pgfusepath{clip}%
\pgfsetbuttcap%
\pgfsetroundjoin%
\definecolor{currentfill}{rgb}{0.180629,0.429975,0.557282}%
\pgfsetfillcolor{currentfill}%
\pgfsetfillopacity{0.700000}%
\pgfsetlinewidth{0.000000pt}%
\definecolor{currentstroke}{rgb}{0.000000,0.000000,0.000000}%
\pgfsetstrokecolor{currentstroke}%
\pgfsetdash{}{0pt}%
\pgfpathmoveto{\pgfqpoint{4.629275in}{3.328156in}}%
\pgfpathlineto{\pgfqpoint{4.642725in}{3.319720in}}%
\pgfpathlineto{\pgfqpoint{4.656180in}{3.311361in}}%
\pgfpathlineto{\pgfqpoint{4.669639in}{3.303081in}}%
\pgfpathlineto{\pgfqpoint{4.683102in}{3.294877in}}%
\pgfpathlineto{\pgfqpoint{4.690798in}{3.318953in}}%
\pgfpathlineto{\pgfqpoint{4.698499in}{3.343506in}}%
\pgfpathlineto{\pgfqpoint{4.706205in}{3.368546in}}%
\pgfpathlineto{\pgfqpoint{4.713916in}{3.394084in}}%
\pgfpathlineto{\pgfqpoint{4.700457in}{3.402808in}}%
\pgfpathlineto{\pgfqpoint{4.687002in}{3.411610in}}%
\pgfpathlineto{\pgfqpoint{4.673551in}{3.420489in}}%
\pgfpathlineto{\pgfqpoint{4.660104in}{3.429447in}}%
\pgfpathlineto{\pgfqpoint{4.652389in}{3.403380in}}%
\pgfpathlineto{\pgfqpoint{4.644680in}{3.377816in}}%
\pgfpathlineto{\pgfqpoint{4.636975in}{3.352744in}}%
\pgfpathlineto{\pgfqpoint{4.629275in}{3.328156in}}%
\pgfpathclose%
\pgfusepath{fill}%
\end{pgfscope}%
\begin{pgfscope}%
\pgfpathrectangle{\pgfqpoint{1.150000in}{0.150000in}}{\pgfqpoint{5.700000in}{5.700000in}}%
\pgfusepath{clip}%
\pgfsetbuttcap%
\pgfsetroundjoin%
\definecolor{currentfill}{rgb}{0.147607,0.511733,0.557049}%
\pgfsetfillcolor{currentfill}%
\pgfsetfillopacity{0.700000}%
\pgfsetlinewidth{0.000000pt}%
\definecolor{currentstroke}{rgb}{0.000000,0.000000,0.000000}%
\pgfsetstrokecolor{currentstroke}%
\pgfsetdash{}{0pt}%
\pgfpathmoveto{\pgfqpoint{4.691020in}{3.538953in}}%
\pgfpathlineto{\pgfqpoint{4.704465in}{3.529451in}}%
\pgfpathlineto{\pgfqpoint{4.717913in}{3.520028in}}%
\pgfpathlineto{\pgfqpoint{4.731366in}{3.510682in}}%
\pgfpathlineto{\pgfqpoint{4.744822in}{3.501415in}}%
\pgfpathlineto{\pgfqpoint{4.752566in}{3.529595in}}%
\pgfpathlineto{\pgfqpoint{4.760317in}{3.558337in}}%
\pgfpathlineto{\pgfqpoint{4.768076in}{3.587651in}}%
\pgfpathlineto{\pgfqpoint{4.775843in}{3.617548in}}%
\pgfpathlineto{\pgfqpoint{4.762388in}{3.627382in}}%
\pgfpathlineto{\pgfqpoint{4.748937in}{3.637295in}}%
\pgfpathlineto{\pgfqpoint{4.735489in}{3.647285in}}%
\pgfpathlineto{\pgfqpoint{4.722046in}{3.657354in}}%
\pgfpathlineto{\pgfqpoint{4.714278in}{3.626882in}}%
\pgfpathlineto{\pgfqpoint{4.706518in}{3.596998in}}%
\pgfpathlineto{\pgfqpoint{4.698765in}{3.567692in}}%
\pgfpathlineto{\pgfqpoint{4.691020in}{3.538953in}}%
\pgfpathclose%
\pgfusepath{fill}%
\end{pgfscope}%
\begin{pgfscope}%
\pgfpathrectangle{\pgfqpoint{1.150000in}{0.150000in}}{\pgfqpoint{5.700000in}{5.700000in}}%
\pgfusepath{clip}%
\pgfsetbuttcap%
\pgfsetroundjoin%
\definecolor{currentfill}{rgb}{0.267968,0.223549,0.512008}%
\pgfsetfillcolor{currentfill}%
\pgfsetfillopacity{0.700000}%
\pgfsetlinewidth{0.000000pt}%
\definecolor{currentstroke}{rgb}{0.000000,0.000000,0.000000}%
\pgfsetstrokecolor{currentstroke}%
\pgfsetdash{}{0pt}%
\pgfpathmoveto{\pgfqpoint{3.700752in}{2.855243in}}%
\pgfpathlineto{\pgfqpoint{3.714082in}{2.847388in}}%
\pgfpathlineto{\pgfqpoint{3.727416in}{2.839629in}}%
\pgfpathlineto{\pgfqpoint{3.740753in}{2.831966in}}%
\pgfpathlineto{\pgfqpoint{3.754093in}{2.824397in}}%
\pgfpathlineto{\pgfqpoint{3.761898in}{2.838752in}}%
\pgfpathlineto{\pgfqpoint{3.769697in}{2.853310in}}%
\pgfpathlineto{\pgfqpoint{3.777492in}{2.868078in}}%
\pgfpathlineto{\pgfqpoint{3.785282in}{2.883059in}}%
\pgfpathlineto{\pgfqpoint{3.771948in}{2.890924in}}%
\pgfpathlineto{\pgfqpoint{3.758617in}{2.898883in}}%
\pgfpathlineto{\pgfqpoint{3.745288in}{2.906938in}}%
\pgfpathlineto{\pgfqpoint{3.731963in}{2.915088in}}%
\pgfpathlineto{\pgfqpoint{3.724168in}{2.899804in}}%
\pgfpathlineto{\pgfqpoint{3.716368in}{2.884738in}}%
\pgfpathlineto{\pgfqpoint{3.708562in}{2.869887in}}%
\pgfpathlineto{\pgfqpoint{3.700752in}{2.855243in}}%
\pgfpathclose%
\pgfusepath{fill}%
\end{pgfscope}%
\begin{pgfscope}%
\pgfpathrectangle{\pgfqpoint{1.150000in}{0.150000in}}{\pgfqpoint{5.700000in}{5.700000in}}%
\pgfusepath{clip}%
\pgfsetbuttcap%
\pgfsetroundjoin%
\definecolor{currentfill}{rgb}{0.263663,0.237631,0.518762}%
\pgfsetfillcolor{currentfill}%
\pgfsetfillopacity{0.700000}%
\pgfsetlinewidth{0.000000pt}%
\definecolor{currentstroke}{rgb}{0.000000,0.000000,0.000000}%
\pgfsetstrokecolor{currentstroke}%
\pgfsetdash{}{0pt}%
\pgfpathmoveto{\pgfqpoint{3.923152in}{2.883151in}}%
\pgfpathlineto{\pgfqpoint{3.936513in}{2.875799in}}%
\pgfpathlineto{\pgfqpoint{3.949878in}{2.868538in}}%
\pgfpathlineto{\pgfqpoint{3.963246in}{2.861365in}}%
\pgfpathlineto{\pgfqpoint{3.976619in}{2.854282in}}%
\pgfpathlineto{\pgfqpoint{3.984370in}{2.869437in}}%
\pgfpathlineto{\pgfqpoint{3.992117in}{2.884827in}}%
\pgfpathlineto{\pgfqpoint{3.999861in}{2.900458in}}%
\pgfpathlineto{\pgfqpoint{4.007601in}{2.916335in}}%
\pgfpathlineto{\pgfqpoint{3.994235in}{2.923754in}}%
\pgfpathlineto{\pgfqpoint{3.980872in}{2.931262in}}%
\pgfpathlineto{\pgfqpoint{3.967513in}{2.938859in}}%
\pgfpathlineto{\pgfqpoint{3.954158in}{2.946546in}}%
\pgfpathlineto{\pgfqpoint{3.946412in}{2.930326in}}%
\pgfpathlineto{\pgfqpoint{3.938663in}{2.914357in}}%
\pgfpathlineto{\pgfqpoint{3.930910in}{2.898634in}}%
\pgfpathlineto{\pgfqpoint{3.923152in}{2.883151in}}%
\pgfpathclose%
\pgfusepath{fill}%
\end{pgfscope}%
\begin{pgfscope}%
\pgfpathrectangle{\pgfqpoint{1.150000in}{0.150000in}}{\pgfqpoint{5.700000in}{5.700000in}}%
\pgfusepath{clip}%
\pgfsetbuttcap%
\pgfsetroundjoin%
\definecolor{currentfill}{rgb}{0.195860,0.395433,0.555276}%
\pgfsetfillcolor{currentfill}%
\pgfsetfillopacity{0.700000}%
\pgfsetlinewidth{0.000000pt}%
\definecolor{currentstroke}{rgb}{0.000000,0.000000,0.000000}%
\pgfsetstrokecolor{currentstroke}%
\pgfsetdash{}{0pt}%
\pgfpathmoveto{\pgfqpoint{4.598514in}{3.234435in}}%
\pgfpathlineto{\pgfqpoint{4.611968in}{3.226499in}}%
\pgfpathlineto{\pgfqpoint{4.625428in}{3.218640in}}%
\pgfpathlineto{\pgfqpoint{4.638891in}{3.210858in}}%
\pgfpathlineto{\pgfqpoint{4.652360in}{3.203154in}}%
\pgfpathlineto{\pgfqpoint{4.660040in}{3.225417in}}%
\pgfpathlineto{\pgfqpoint{4.667723in}{3.248119in}}%
\pgfpathlineto{\pgfqpoint{4.675411in}{3.271269in}}%
\pgfpathlineto{\pgfqpoint{4.683102in}{3.294877in}}%
\pgfpathlineto{\pgfqpoint{4.669639in}{3.303081in}}%
\pgfpathlineto{\pgfqpoint{4.656180in}{3.311361in}}%
\pgfpathlineto{\pgfqpoint{4.642725in}{3.319720in}}%
\pgfpathlineto{\pgfqpoint{4.629275in}{3.328156in}}%
\pgfpathlineto{\pgfqpoint{4.621579in}{3.304040in}}%
\pgfpathlineto{\pgfqpoint{4.613887in}{3.280388in}}%
\pgfpathlineto{\pgfqpoint{4.606199in}{3.257190in}}%
\pgfpathlineto{\pgfqpoint{4.598514in}{3.234435in}}%
\pgfpathclose%
\pgfusepath{fill}%
\end{pgfscope}%
\begin{pgfscope}%
\pgfpathrectangle{\pgfqpoint{1.150000in}{0.150000in}}{\pgfqpoint{5.700000in}{5.700000in}}%
\pgfusepath{clip}%
\pgfsetbuttcap%
\pgfsetroundjoin%
\definecolor{currentfill}{rgb}{0.253935,0.265254,0.529983}%
\pgfsetfillcolor{currentfill}%
\pgfsetfillopacity{0.700000}%
\pgfsetlinewidth{0.000000pt}%
\definecolor{currentstroke}{rgb}{0.000000,0.000000,0.000000}%
\pgfsetstrokecolor{currentstroke}%
\pgfsetdash{}{0pt}%
\pgfpathmoveto{\pgfqpoint{3.010650in}{2.952181in}}%
\pgfpathlineto{\pgfqpoint{3.023950in}{2.941070in}}%
\pgfpathlineto{\pgfqpoint{3.037250in}{2.930087in}}%
\pgfpathlineto{\pgfqpoint{3.050549in}{2.919232in}}%
\pgfpathlineto{\pgfqpoint{3.063848in}{2.908503in}}%
\pgfpathlineto{\pgfqpoint{3.071836in}{2.921679in}}%
\pgfpathlineto{\pgfqpoint{3.079817in}{2.935018in}}%
\pgfpathlineto{\pgfqpoint{3.087790in}{2.948524in}}%
\pgfpathlineto{\pgfqpoint{3.095757in}{2.962200in}}%
\pgfpathlineto{\pgfqpoint{3.082464in}{2.973125in}}%
\pgfpathlineto{\pgfqpoint{3.069171in}{2.984177in}}%
\pgfpathlineto{\pgfqpoint{3.055878in}{2.995356in}}%
\pgfpathlineto{\pgfqpoint{3.042585in}{3.006664in}}%
\pgfpathlineto{\pgfqpoint{3.034612in}{2.992783in}}%
\pgfpathlineto{\pgfqpoint{3.026633in}{2.979078in}}%
\pgfpathlineto{\pgfqpoint{3.018645in}{2.965546in}}%
\pgfpathlineto{\pgfqpoint{3.010650in}{2.952181in}}%
\pgfpathclose%
\pgfusepath{fill}%
\end{pgfscope}%
\begin{pgfscope}%
\pgfpathrectangle{\pgfqpoint{1.150000in}{0.150000in}}{\pgfqpoint{5.700000in}{5.700000in}}%
\pgfusepath{clip}%
\pgfsetbuttcap%
\pgfsetroundjoin%
\definecolor{currentfill}{rgb}{0.241237,0.296485,0.539709}%
\pgfsetfillcolor{currentfill}%
\pgfsetfillopacity{0.700000}%
\pgfsetlinewidth{0.000000pt}%
\definecolor{currentstroke}{rgb}{0.000000,0.000000,0.000000}%
\pgfsetstrokecolor{currentstroke}%
\pgfsetdash{}{0pt}%
\pgfpathmoveto{\pgfqpoint{4.314419in}{3.003472in}}%
\pgfpathlineto{\pgfqpoint{4.327841in}{2.996368in}}%
\pgfpathlineto{\pgfqpoint{4.341267in}{2.989345in}}%
\pgfpathlineto{\pgfqpoint{4.354698in}{2.982403in}}%
\pgfpathlineto{\pgfqpoint{4.368134in}{2.975542in}}%
\pgfpathlineto{\pgfqpoint{4.375814in}{2.993196in}}%
\pgfpathlineto{\pgfqpoint{4.383493in}{3.011169in}}%
\pgfpathlineto{\pgfqpoint{4.391172in}{3.029468in}}%
\pgfpathlineto{\pgfqpoint{4.398850in}{3.048102in}}%
\pgfpathlineto{\pgfqpoint{4.385421in}{3.055379in}}%
\pgfpathlineto{\pgfqpoint{4.371996in}{3.062737in}}%
\pgfpathlineto{\pgfqpoint{4.358575in}{3.070176in}}%
\pgfpathlineto{\pgfqpoint{4.345159in}{3.077697in}}%
\pgfpathlineto{\pgfqpoint{4.337475in}{3.058639in}}%
\pgfpathlineto{\pgfqpoint{4.329791in}{3.039921in}}%
\pgfpathlineto{\pgfqpoint{4.322105in}{3.021534in}}%
\pgfpathlineto{\pgfqpoint{4.314419in}{3.003472in}}%
\pgfpathclose%
\pgfusepath{fill}%
\end{pgfscope}%
\begin{pgfscope}%
\pgfpathrectangle{\pgfqpoint{1.150000in}{0.150000in}}{\pgfqpoint{5.700000in}{5.700000in}}%
\pgfusepath{clip}%
\pgfsetbuttcap%
\pgfsetroundjoin%
\definecolor{currentfill}{rgb}{0.267968,0.223549,0.512008}%
\pgfsetfillcolor{currentfill}%
\pgfsetfillopacity{0.700000}%
\pgfsetlinewidth{0.000000pt}%
\definecolor{currentstroke}{rgb}{0.000000,0.000000,0.000000}%
\pgfsetstrokecolor{currentstroke}%
\pgfsetdash{}{0pt}%
\pgfpathmoveto{\pgfqpoint{3.340200in}{2.856801in}}%
\pgfpathlineto{\pgfqpoint{3.353500in}{2.847658in}}%
\pgfpathlineto{\pgfqpoint{3.366802in}{2.838625in}}%
\pgfpathlineto{\pgfqpoint{3.380105in}{2.829701in}}%
\pgfpathlineto{\pgfqpoint{3.393410in}{2.820885in}}%
\pgfpathlineto{\pgfqpoint{3.401312in}{2.834331in}}%
\pgfpathlineto{\pgfqpoint{3.409207in}{2.847948in}}%
\pgfpathlineto{\pgfqpoint{3.417097in}{2.861740in}}%
\pgfpathlineto{\pgfqpoint{3.424980in}{2.875710in}}%
\pgfpathlineto{\pgfqpoint{3.411681in}{2.884762in}}%
\pgfpathlineto{\pgfqpoint{3.398383in}{2.893922in}}%
\pgfpathlineto{\pgfqpoint{3.385088in}{2.903190in}}%
\pgfpathlineto{\pgfqpoint{3.371794in}{2.912569in}}%
\pgfpathlineto{\pgfqpoint{3.363905in}{2.898355in}}%
\pgfpathlineto{\pgfqpoint{3.356010in}{2.884326in}}%
\pgfpathlineto{\pgfqpoint{3.348108in}{2.870476in}}%
\pgfpathlineto{\pgfqpoint{3.340200in}{2.856801in}}%
\pgfpathclose%
\pgfusepath{fill}%
\end{pgfscope}%
\begin{pgfscope}%
\pgfpathrectangle{\pgfqpoint{1.150000in}{0.150000in}}{\pgfqpoint{5.700000in}{5.700000in}}%
\pgfusepath{clip}%
\pgfsetbuttcap%
\pgfsetroundjoin%
\definecolor{currentfill}{rgb}{0.132444,0.552216,0.553018}%
\pgfsetfillcolor{currentfill}%
\pgfsetfillopacity{0.700000}%
\pgfsetlinewidth{0.000000pt}%
\definecolor{currentstroke}{rgb}{0.000000,0.000000,0.000000}%
\pgfsetstrokecolor{currentstroke}%
\pgfsetdash{}{0pt}%
\pgfpathmoveto{\pgfqpoint{4.722046in}{3.657354in}}%
\pgfpathlineto{\pgfqpoint{4.735489in}{3.647285in}}%
\pgfpathlineto{\pgfqpoint{4.748937in}{3.637295in}}%
\pgfpathlineto{\pgfqpoint{4.762388in}{3.627382in}}%
\pgfpathlineto{\pgfqpoint{4.775843in}{3.617548in}}%
\pgfpathlineto{\pgfqpoint{4.783619in}{3.648039in}}%
\pgfpathlineto{\pgfqpoint{4.791404in}{3.679137in}}%
\pgfpathlineto{\pgfqpoint{4.799198in}{3.710853in}}%
\pgfpathlineto{\pgfqpoint{4.785743in}{3.721128in}}%
\pgfpathlineto{\pgfqpoint{4.772292in}{3.731481in}}%
\pgfpathlineto{\pgfqpoint{4.758844in}{3.741913in}}%
\pgfpathlineto{\pgfqpoint{4.745400in}{3.752423in}}%
\pgfpathlineto{\pgfqpoint{4.737606in}{3.720114in}}%
\pgfpathlineto{\pgfqpoint{4.729822in}{3.688428in}}%
\pgfpathlineto{\pgfqpoint{4.722046in}{3.657354in}}%
\pgfpathclose%
\pgfusepath{fill}%
\end{pgfscope}%
\begin{pgfscope}%
\pgfpathrectangle{\pgfqpoint{1.150000in}{0.150000in}}{\pgfqpoint{5.700000in}{5.700000in}}%
\pgfusepath{clip}%
\pgfsetbuttcap%
\pgfsetroundjoin%
\definecolor{currentfill}{rgb}{0.231674,0.318106,0.544834}%
\pgfsetfillcolor{currentfill}%
\pgfsetfillopacity{0.700000}%
\pgfsetlinewidth{0.000000pt}%
\definecolor{currentstroke}{rgb}{0.000000,0.000000,0.000000}%
\pgfsetstrokecolor{currentstroke}%
\pgfsetdash{}{0pt}%
\pgfpathmoveto{\pgfqpoint{4.398850in}{3.048102in}}%
\pgfpathlineto{\pgfqpoint{4.412284in}{3.040905in}}%
\pgfpathlineto{\pgfqpoint{4.425723in}{3.033789in}}%
\pgfpathlineto{\pgfqpoint{4.439167in}{3.026751in}}%
\pgfpathlineto{\pgfqpoint{4.452615in}{3.019794in}}%
\pgfpathlineto{\pgfqpoint{4.460287in}{3.038341in}}%
\pgfpathlineto{\pgfqpoint{4.467959in}{3.057233in}}%
\pgfpathlineto{\pgfqpoint{4.475632in}{3.076478in}}%
\pgfpathlineto{\pgfqpoint{4.483305in}{3.096085in}}%
\pgfpathlineto{\pgfqpoint{4.469863in}{3.103479in}}%
\pgfpathlineto{\pgfqpoint{4.456425in}{3.110953in}}%
\pgfpathlineto{\pgfqpoint{4.442992in}{3.118506in}}%
\pgfpathlineto{\pgfqpoint{4.429564in}{3.126139in}}%
\pgfpathlineto{\pgfqpoint{4.421885in}{3.106089in}}%
\pgfpathlineto{\pgfqpoint{4.414206in}{3.086404in}}%
\pgfpathlineto{\pgfqpoint{4.406528in}{3.067078in}}%
\pgfpathlineto{\pgfqpoint{4.398850in}{3.048102in}}%
\pgfpathclose%
\pgfusepath{fill}%
\end{pgfscope}%
\begin{pgfscope}%
\pgfpathrectangle{\pgfqpoint{1.150000in}{0.150000in}}{\pgfqpoint{5.700000in}{5.700000in}}%
\pgfusepath{clip}%
\pgfsetbuttcap%
\pgfsetroundjoin%
\definecolor{currentfill}{rgb}{0.248629,0.278775,0.534556}%
\pgfsetfillcolor{currentfill}%
\pgfsetfillopacity{0.700000}%
\pgfsetlinewidth{0.000000pt}%
\definecolor{currentstroke}{rgb}{0.000000,0.000000,0.000000}%
\pgfsetstrokecolor{currentstroke}%
\pgfsetdash{}{0pt}%
\pgfpathmoveto{\pgfqpoint{4.229995in}{2.961965in}}%
\pgfpathlineto{\pgfqpoint{4.243404in}{2.954927in}}%
\pgfpathlineto{\pgfqpoint{4.256818in}{2.947973in}}%
\pgfpathlineto{\pgfqpoint{4.270237in}{2.941100in}}%
\pgfpathlineto{\pgfqpoint{4.283661in}{2.934310in}}%
\pgfpathlineto{\pgfqpoint{4.291352in}{2.951152in}}%
\pgfpathlineto{\pgfqpoint{4.299043in}{2.968288in}}%
\pgfpathlineto{\pgfqpoint{4.306731in}{2.985725in}}%
\pgfpathlineto{\pgfqpoint{4.314419in}{3.003472in}}%
\pgfpathlineto{\pgfqpoint{4.301002in}{3.010658in}}%
\pgfpathlineto{\pgfqpoint{4.287589in}{3.017926in}}%
\pgfpathlineto{\pgfqpoint{4.274181in}{3.025276in}}%
\pgfpathlineto{\pgfqpoint{4.260777in}{3.032710in}}%
\pgfpathlineto{\pgfqpoint{4.253084in}{3.014560in}}%
\pgfpathlineto{\pgfqpoint{4.245389in}{2.996724in}}%
\pgfpathlineto{\pgfqpoint{4.237693in}{2.979195in}}%
\pgfpathlineto{\pgfqpoint{4.229995in}{2.961965in}}%
\pgfpathclose%
\pgfusepath{fill}%
\end{pgfscope}%
\begin{pgfscope}%
\pgfpathrectangle{\pgfqpoint{1.150000in}{0.150000in}}{\pgfqpoint{5.700000in}{5.700000in}}%
\pgfusepath{clip}%
\pgfsetbuttcap%
\pgfsetroundjoin%
\definecolor{currentfill}{rgb}{0.270595,0.214069,0.507052}%
\pgfsetfillcolor{currentfill}%
\pgfsetfillopacity{0.700000}%
\pgfsetlinewidth{0.000000pt}%
\definecolor{currentstroke}{rgb}{0.000000,0.000000,0.000000}%
\pgfsetstrokecolor{currentstroke}%
\pgfsetdash{}{0pt}%
\pgfpathmoveto{\pgfqpoint{3.478196in}{2.840567in}}%
\pgfpathlineto{\pgfqpoint{3.491505in}{2.832044in}}%
\pgfpathlineto{\pgfqpoint{3.504817in}{2.823624in}}%
\pgfpathlineto{\pgfqpoint{3.518131in}{2.815308in}}%
\pgfpathlineto{\pgfqpoint{3.531448in}{2.807093in}}%
\pgfpathlineto{\pgfqpoint{3.539313in}{2.820750in}}%
\pgfpathlineto{\pgfqpoint{3.547173in}{2.834586in}}%
\pgfpathlineto{\pgfqpoint{3.555027in}{2.848604in}}%
\pgfpathlineto{\pgfqpoint{3.562875in}{2.862810in}}%
\pgfpathlineto{\pgfqpoint{3.549565in}{2.871280in}}%
\pgfpathlineto{\pgfqpoint{3.536256in}{2.879852in}}%
\pgfpathlineto{\pgfqpoint{3.522950in}{2.888527in}}%
\pgfpathlineto{\pgfqpoint{3.509646in}{2.897306in}}%
\pgfpathlineto{\pgfqpoint{3.501793in}{2.882837in}}%
\pgfpathlineto{\pgfqpoint{3.493933in}{2.868561in}}%
\pgfpathlineto{\pgfqpoint{3.486067in}{2.854472in}}%
\pgfpathlineto{\pgfqpoint{3.478196in}{2.840567in}}%
\pgfpathclose%
\pgfusepath{fill}%
\end{pgfscope}%
\begin{pgfscope}%
\pgfpathrectangle{\pgfqpoint{1.150000in}{0.150000in}}{\pgfqpoint{5.700000in}{5.700000in}}%
\pgfusepath{clip}%
\pgfsetbuttcap%
\pgfsetroundjoin%
\definecolor{currentfill}{rgb}{0.265145,0.232956,0.516599}%
\pgfsetfillcolor{currentfill}%
\pgfsetfillopacity{0.700000}%
\pgfsetlinewidth{0.000000pt}%
\definecolor{currentstroke}{rgb}{0.000000,0.000000,0.000000}%
\pgfsetstrokecolor{currentstroke}%
\pgfsetdash{}{0pt}%
\pgfpathmoveto{\pgfqpoint{3.202105in}{2.879225in}}%
\pgfpathlineto{\pgfqpoint{3.215401in}{2.869391in}}%
\pgfpathlineto{\pgfqpoint{3.228698in}{2.859674in}}%
\pgfpathlineto{\pgfqpoint{3.241996in}{2.850072in}}%
\pgfpathlineto{\pgfqpoint{3.255295in}{2.840585in}}%
\pgfpathlineto{\pgfqpoint{3.263235in}{2.853811in}}%
\pgfpathlineto{\pgfqpoint{3.271169in}{2.867199in}}%
\pgfpathlineto{\pgfqpoint{3.279096in}{2.880756in}}%
\pgfpathlineto{\pgfqpoint{3.287016in}{2.894483in}}%
\pgfpathlineto{\pgfqpoint{3.273723in}{2.904186in}}%
\pgfpathlineto{\pgfqpoint{3.260431in}{2.914004in}}%
\pgfpathlineto{\pgfqpoint{3.247140in}{2.923937in}}%
\pgfpathlineto{\pgfqpoint{3.233850in}{2.933987in}}%
\pgfpathlineto{\pgfqpoint{3.225924in}{2.920036in}}%
\pgfpathlineto{\pgfqpoint{3.217991in}{2.906261in}}%
\pgfpathlineto{\pgfqpoint{3.210051in}{2.892659in}}%
\pgfpathlineto{\pgfqpoint{3.202105in}{2.879225in}}%
\pgfpathclose%
\pgfusepath{fill}%
\end{pgfscope}%
\begin{pgfscope}%
\pgfpathrectangle{\pgfqpoint{1.150000in}{0.150000in}}{\pgfqpoint{5.700000in}{5.700000in}}%
\pgfusepath{clip}%
\pgfsetbuttcap%
\pgfsetroundjoin%
\definecolor{currentfill}{rgb}{0.221989,0.339161,0.548752}%
\pgfsetfillcolor{currentfill}%
\pgfsetfillopacity{0.700000}%
\pgfsetlinewidth{0.000000pt}%
\definecolor{currentstroke}{rgb}{0.000000,0.000000,0.000000}%
\pgfsetstrokecolor{currentstroke}%
\pgfsetdash{}{0pt}%
\pgfpathmoveto{\pgfqpoint{4.483305in}{3.096085in}}%
\pgfpathlineto{\pgfqpoint{4.496752in}{3.088770in}}%
\pgfpathlineto{\pgfqpoint{4.510204in}{3.081534in}}%
\pgfpathlineto{\pgfqpoint{4.523660in}{3.074376in}}%
\pgfpathlineto{\pgfqpoint{4.537122in}{3.067296in}}%
\pgfpathlineto{\pgfqpoint{4.544789in}{3.086823in}}%
\pgfpathlineto{\pgfqpoint{4.552459in}{3.106723in}}%
\pgfpathlineto{\pgfqpoint{4.560129in}{3.127004in}}%
\pgfpathlineto{\pgfqpoint{4.567802in}{3.147675in}}%
\pgfpathlineto{\pgfqpoint{4.554346in}{3.155212in}}%
\pgfpathlineto{\pgfqpoint{4.540896in}{3.162827in}}%
\pgfpathlineto{\pgfqpoint{4.527450in}{3.170520in}}%
\pgfpathlineto{\pgfqpoint{4.514008in}{3.178292in}}%
\pgfpathlineto{\pgfqpoint{4.506330in}{3.157156in}}%
\pgfpathlineto{\pgfqpoint{4.498654in}{3.136415in}}%
\pgfpathlineto{\pgfqpoint{4.490979in}{3.116061in}}%
\pgfpathlineto{\pgfqpoint{4.483305in}{3.096085in}}%
\pgfpathclose%
\pgfusepath{fill}%
\end{pgfscope}%
\begin{pgfscope}%
\pgfpathrectangle{\pgfqpoint{1.150000in}{0.150000in}}{\pgfqpoint{5.700000in}{5.700000in}}%
\pgfusepath{clip}%
\pgfsetbuttcap%
\pgfsetroundjoin%
\definecolor{currentfill}{rgb}{0.267968,0.223549,0.512008}%
\pgfsetfillcolor{currentfill}%
\pgfsetfillopacity{0.700000}%
\pgfsetlinewidth{0.000000pt}%
\definecolor{currentstroke}{rgb}{0.000000,0.000000,0.000000}%
\pgfsetstrokecolor{currentstroke}%
\pgfsetdash{}{0pt}%
\pgfpathmoveto{\pgfqpoint{3.838654in}{2.852538in}}%
\pgfpathlineto{\pgfqpoint{3.852005in}{2.845139in}}%
\pgfpathlineto{\pgfqpoint{3.865361in}{2.837832in}}%
\pgfpathlineto{\pgfqpoint{3.878720in}{2.830616in}}%
\pgfpathlineto{\pgfqpoint{3.892083in}{2.823490in}}%
\pgfpathlineto{\pgfqpoint{3.899857in}{2.838076in}}%
\pgfpathlineto{\pgfqpoint{3.907626in}{2.852877in}}%
\pgfpathlineto{\pgfqpoint{3.915391in}{2.867900in}}%
\pgfpathlineto{\pgfqpoint{3.923152in}{2.883151in}}%
\pgfpathlineto{\pgfqpoint{3.909796in}{2.890592in}}%
\pgfpathlineto{\pgfqpoint{3.896442in}{2.898123in}}%
\pgfpathlineto{\pgfqpoint{3.883093in}{2.905746in}}%
\pgfpathlineto{\pgfqpoint{3.869747in}{2.913461in}}%
\pgfpathlineto{\pgfqpoint{3.861980in}{2.897887in}}%
\pgfpathlineto{\pgfqpoint{3.854209in}{2.882546in}}%
\pgfpathlineto{\pgfqpoint{3.846433in}{2.867432in}}%
\pgfpathlineto{\pgfqpoint{3.838654in}{2.852538in}}%
\pgfpathclose%
\pgfusepath{fill}%
\end{pgfscope}%
\begin{pgfscope}%
\pgfpathrectangle{\pgfqpoint{1.150000in}{0.150000in}}{\pgfqpoint{5.700000in}{5.700000in}}%
\pgfusepath{clip}%
\pgfsetbuttcap%
\pgfsetroundjoin%
\definecolor{currentfill}{rgb}{0.255645,0.260703,0.528312}%
\pgfsetfillcolor{currentfill}%
\pgfsetfillopacity{0.700000}%
\pgfsetlinewidth{0.000000pt}%
\definecolor{currentstroke}{rgb}{0.000000,0.000000,0.000000}%
\pgfsetstrokecolor{currentstroke}%
\pgfsetdash{}{0pt}%
\pgfpathmoveto{\pgfqpoint{4.145562in}{2.923379in}}%
\pgfpathlineto{\pgfqpoint{4.158960in}{2.916382in}}%
\pgfpathlineto{\pgfqpoint{4.172363in}{2.909469in}}%
\pgfpathlineto{\pgfqpoint{4.185770in}{2.902640in}}%
\pgfpathlineto{\pgfqpoint{4.199181in}{2.895895in}}%
\pgfpathlineto{\pgfqpoint{4.206888in}{2.911999in}}%
\pgfpathlineto{\pgfqpoint{4.214592in}{2.928374in}}%
\pgfpathlineto{\pgfqpoint{4.222295in}{2.945027in}}%
\pgfpathlineto{\pgfqpoint{4.229995in}{2.961965in}}%
\pgfpathlineto{\pgfqpoint{4.216589in}{2.969086in}}%
\pgfpathlineto{\pgfqpoint{4.203189in}{2.976290in}}%
\pgfpathlineto{\pgfqpoint{4.189792in}{2.983578in}}%
\pgfpathlineto{\pgfqpoint{4.176400in}{2.990951in}}%
\pgfpathlineto{\pgfqpoint{4.168694in}{2.973630in}}%
\pgfpathlineto{\pgfqpoint{4.160986in}{2.956599in}}%
\pgfpathlineto{\pgfqpoint{4.153275in}{2.939851in}}%
\pgfpathlineto{\pgfqpoint{4.145562in}{2.923379in}}%
\pgfpathclose%
\pgfusepath{fill}%
\end{pgfscope}%
\begin{pgfscope}%
\pgfpathrectangle{\pgfqpoint{1.150000in}{0.150000in}}{\pgfqpoint{5.700000in}{5.700000in}}%
\pgfusepath{clip}%
\pgfsetbuttcap%
\pgfsetroundjoin%
\definecolor{currentfill}{rgb}{0.168126,0.459988,0.558082}%
\pgfsetfillcolor{currentfill}%
\pgfsetfillopacity{0.700000}%
\pgfsetlinewidth{0.000000pt}%
\definecolor{currentstroke}{rgb}{0.000000,0.000000,0.000000}%
\pgfsetstrokecolor{currentstroke}%
\pgfsetdash{}{0pt}%
\pgfpathmoveto{\pgfqpoint{4.713916in}{3.394084in}}%
\pgfpathlineto{\pgfqpoint{4.727380in}{3.385436in}}%
\pgfpathlineto{\pgfqpoint{4.740848in}{3.376866in}}%
\pgfpathlineto{\pgfqpoint{4.754321in}{3.368372in}}%
\pgfpathlineto{\pgfqpoint{4.767798in}{3.359954in}}%
\pgfpathlineto{\pgfqpoint{4.775511in}{3.385465in}}%
\pgfpathlineto{\pgfqpoint{4.783231in}{3.411488in}}%
\pgfpathlineto{\pgfqpoint{4.790957in}{3.438034in}}%
\pgfpathlineto{\pgfqpoint{4.798690in}{3.465114in}}%
\pgfpathlineto{\pgfqpoint{4.785217in}{3.474074in}}%
\pgfpathlineto{\pgfqpoint{4.771748in}{3.483111in}}%
\pgfpathlineto{\pgfqpoint{4.758283in}{3.492224in}}%
\pgfpathlineto{\pgfqpoint{4.744822in}{3.501415in}}%
\pgfpathlineto{\pgfqpoint{4.737086in}{3.473784in}}%
\pgfpathlineto{\pgfqpoint{4.729356in}{3.446692in}}%
\pgfpathlineto{\pgfqpoint{4.721633in}{3.420129in}}%
\pgfpathlineto{\pgfqpoint{4.713916in}{3.394084in}}%
\pgfpathclose%
\pgfusepath{fill}%
\end{pgfscope}%
\begin{pgfscope}%
\pgfpathrectangle{\pgfqpoint{1.150000in}{0.150000in}}{\pgfqpoint{5.700000in}{5.700000in}}%
\pgfusepath{clip}%
\pgfsetbuttcap%
\pgfsetroundjoin%
\definecolor{currentfill}{rgb}{0.270595,0.214069,0.507052}%
\pgfsetfillcolor{currentfill}%
\pgfsetfillopacity{0.700000}%
\pgfsetlinewidth{0.000000pt}%
\definecolor{currentstroke}{rgb}{0.000000,0.000000,0.000000}%
\pgfsetstrokecolor{currentstroke}%
\pgfsetdash{}{0pt}%
\pgfpathmoveto{\pgfqpoint{3.616145in}{2.829938in}}%
\pgfpathlineto{\pgfqpoint{3.629469in}{2.821969in}}%
\pgfpathlineto{\pgfqpoint{3.642796in}{2.814098in}}%
\pgfpathlineto{\pgfqpoint{3.656126in}{2.806325in}}%
\pgfpathlineto{\pgfqpoint{3.669459in}{2.798649in}}%
\pgfpathlineto{\pgfqpoint{3.677290in}{2.812511in}}%
\pgfpathlineto{\pgfqpoint{3.685116in}{2.826561in}}%
\pgfpathlineto{\pgfqpoint{3.692937in}{2.840803in}}%
\pgfpathlineto{\pgfqpoint{3.700752in}{2.855243in}}%
\pgfpathlineto{\pgfqpoint{3.687425in}{2.863195in}}%
\pgfpathlineto{\pgfqpoint{3.674100in}{2.871243in}}%
\pgfpathlineto{\pgfqpoint{3.660779in}{2.879390in}}%
\pgfpathlineto{\pgfqpoint{3.647460in}{2.887634in}}%
\pgfpathlineto{\pgfqpoint{3.639639in}{2.872911in}}%
\pgfpathlineto{\pgfqpoint{3.631813in}{2.858391in}}%
\pgfpathlineto{\pgfqpoint{3.623982in}{2.844068in}}%
\pgfpathlineto{\pgfqpoint{3.616145in}{2.829938in}}%
\pgfpathclose%
\pgfusepath{fill}%
\end{pgfscope}%
\begin{pgfscope}%
\pgfpathrectangle{\pgfqpoint{1.150000in}{0.150000in}}{\pgfqpoint{5.700000in}{5.700000in}}%
\pgfusepath{clip}%
\pgfsetbuttcap%
\pgfsetroundjoin%
\definecolor{currentfill}{rgb}{0.151918,0.500685,0.557587}%
\pgfsetfillcolor{currentfill}%
\pgfsetfillopacity{0.700000}%
\pgfsetlinewidth{0.000000pt}%
\definecolor{currentstroke}{rgb}{0.000000,0.000000,0.000000}%
\pgfsetstrokecolor{currentstroke}%
\pgfsetdash{}{0pt}%
\pgfpathmoveto{\pgfqpoint{4.744822in}{3.501415in}}%
\pgfpathlineto{\pgfqpoint{4.758283in}{3.492224in}}%
\pgfpathlineto{\pgfqpoint{4.771748in}{3.483111in}}%
\pgfpathlineto{\pgfqpoint{4.785217in}{3.474074in}}%
\pgfpathlineto{\pgfqpoint{4.798690in}{3.465114in}}%
\pgfpathlineto{\pgfqpoint{4.806431in}{3.492737in}}%
\pgfpathlineto{\pgfqpoint{4.814180in}{3.520916in}}%
\pgfpathlineto{\pgfqpoint{4.821937in}{3.549661in}}%
\pgfpathlineto{\pgfqpoint{4.829702in}{3.578983in}}%
\pgfpathlineto{\pgfqpoint{4.816232in}{3.588509in}}%
\pgfpathlineto{\pgfqpoint{4.802765in}{3.598111in}}%
\pgfpathlineto{\pgfqpoint{4.789302in}{3.607791in}}%
\pgfpathlineto{\pgfqpoint{4.775843in}{3.617548in}}%
\pgfpathlineto{\pgfqpoint{4.768076in}{3.587651in}}%
\pgfpathlineto{\pgfqpoint{4.760317in}{3.558337in}}%
\pgfpathlineto{\pgfqpoint{4.752566in}{3.529595in}}%
\pgfpathlineto{\pgfqpoint{4.744822in}{3.501415in}}%
\pgfpathclose%
\pgfusepath{fill}%
\end{pgfscope}%
\begin{pgfscope}%
\pgfpathrectangle{\pgfqpoint{1.150000in}{0.150000in}}{\pgfqpoint{5.700000in}{5.700000in}}%
\pgfusepath{clip}%
\pgfsetbuttcap%
\pgfsetroundjoin%
\definecolor{currentfill}{rgb}{0.183898,0.422383,0.556944}%
\pgfsetfillcolor{currentfill}%
\pgfsetfillopacity{0.700000}%
\pgfsetlinewidth{0.000000pt}%
\definecolor{currentstroke}{rgb}{0.000000,0.000000,0.000000}%
\pgfsetstrokecolor{currentstroke}%
\pgfsetdash{}{0pt}%
\pgfpathmoveto{\pgfqpoint{4.683102in}{3.294877in}}%
\pgfpathlineto{\pgfqpoint{4.696570in}{3.286751in}}%
\pgfpathlineto{\pgfqpoint{4.710043in}{3.278701in}}%
\pgfpathlineto{\pgfqpoint{4.723520in}{3.270728in}}%
\pgfpathlineto{\pgfqpoint{4.737002in}{3.262831in}}%
\pgfpathlineto{\pgfqpoint{4.744693in}{3.286394in}}%
\pgfpathlineto{\pgfqpoint{4.752389in}{3.310429in}}%
\pgfpathlineto{\pgfqpoint{4.760091in}{3.334946in}}%
\pgfpathlineto{\pgfqpoint{4.767798in}{3.359954in}}%
\pgfpathlineto{\pgfqpoint{4.754321in}{3.368372in}}%
\pgfpathlineto{\pgfqpoint{4.740848in}{3.376866in}}%
\pgfpathlineto{\pgfqpoint{4.727380in}{3.385436in}}%
\pgfpathlineto{\pgfqpoint{4.713916in}{3.394084in}}%
\pgfpathlineto{\pgfqpoint{4.706205in}{3.368546in}}%
\pgfpathlineto{\pgfqpoint{4.698499in}{3.343506in}}%
\pgfpathlineto{\pgfqpoint{4.690798in}{3.318953in}}%
\pgfpathlineto{\pgfqpoint{4.683102in}{3.294877in}}%
\pgfpathclose%
\pgfusepath{fill}%
\end{pgfscope}%
\begin{pgfscope}%
\pgfpathrectangle{\pgfqpoint{1.150000in}{0.150000in}}{\pgfqpoint{5.700000in}{5.700000in}}%
\pgfusepath{clip}%
\pgfsetbuttcap%
\pgfsetroundjoin%
\definecolor{currentfill}{rgb}{0.212395,0.359683,0.551710}%
\pgfsetfillcolor{currentfill}%
\pgfsetfillopacity{0.700000}%
\pgfsetlinewidth{0.000000pt}%
\definecolor{currentstroke}{rgb}{0.000000,0.000000,0.000000}%
\pgfsetstrokecolor{currentstroke}%
\pgfsetdash{}{0pt}%
\pgfpathmoveto{\pgfqpoint{4.567802in}{3.147675in}}%
\pgfpathlineto{\pgfqpoint{4.581262in}{3.140217in}}%
\pgfpathlineto{\pgfqpoint{4.594726in}{3.132836in}}%
\pgfpathlineto{\pgfqpoint{4.608196in}{3.125532in}}%
\pgfpathlineto{\pgfqpoint{4.621671in}{3.118305in}}%
\pgfpathlineto{\pgfqpoint{4.629339in}{3.138905in}}%
\pgfpathlineto{\pgfqpoint{4.637010in}{3.159907in}}%
\pgfpathlineto{\pgfqpoint{4.644683in}{3.181321in}}%
\pgfpathlineto{\pgfqpoint{4.652360in}{3.203154in}}%
\pgfpathlineto{\pgfqpoint{4.638891in}{3.210858in}}%
\pgfpathlineto{\pgfqpoint{4.625428in}{3.218640in}}%
\pgfpathlineto{\pgfqpoint{4.611968in}{3.226499in}}%
\pgfpathlineto{\pgfqpoint{4.598514in}{3.234435in}}%
\pgfpathlineto{\pgfqpoint{4.590832in}{3.212116in}}%
\pgfpathlineto{\pgfqpoint{4.583153in}{3.190222in}}%
\pgfpathlineto{\pgfqpoint{4.575476in}{3.168745in}}%
\pgfpathlineto{\pgfqpoint{4.567802in}{3.147675in}}%
\pgfpathclose%
\pgfusepath{fill}%
\end{pgfscope}%
\begin{pgfscope}%
\pgfpathrectangle{\pgfqpoint{1.150000in}{0.150000in}}{\pgfqpoint{5.700000in}{5.700000in}}%
\pgfusepath{clip}%
\pgfsetbuttcap%
\pgfsetroundjoin%
\definecolor{currentfill}{rgb}{0.260571,0.246922,0.522828}%
\pgfsetfillcolor{currentfill}%
\pgfsetfillopacity{0.700000}%
\pgfsetlinewidth{0.000000pt}%
\definecolor{currentstroke}{rgb}{0.000000,0.000000,0.000000}%
\pgfsetstrokecolor{currentstroke}%
\pgfsetdash{}{0pt}%
\pgfpathmoveto{\pgfqpoint{3.063848in}{2.908503in}}%
\pgfpathlineto{\pgfqpoint{3.077147in}{2.897899in}}%
\pgfpathlineto{\pgfqpoint{3.090447in}{2.887419in}}%
\pgfpathlineto{\pgfqpoint{3.103746in}{2.877063in}}%
\pgfpathlineto{\pgfqpoint{3.117046in}{2.866829in}}%
\pgfpathlineto{\pgfqpoint{3.125027in}{2.879817in}}%
\pgfpathlineto{\pgfqpoint{3.133001in}{2.892962in}}%
\pgfpathlineto{\pgfqpoint{3.140968in}{2.906270in}}%
\pgfpathlineto{\pgfqpoint{3.148928in}{2.919743in}}%
\pgfpathlineto{\pgfqpoint{3.135635in}{2.930173in}}%
\pgfpathlineto{\pgfqpoint{3.122342in}{2.940725in}}%
\pgfpathlineto{\pgfqpoint{3.109049in}{2.951401in}}%
\pgfpathlineto{\pgfqpoint{3.095757in}{2.962200in}}%
\pgfpathlineto{\pgfqpoint{3.087790in}{2.948524in}}%
\pgfpathlineto{\pgfqpoint{3.079817in}{2.935018in}}%
\pgfpathlineto{\pgfqpoint{3.071836in}{2.921679in}}%
\pgfpathlineto{\pgfqpoint{3.063848in}{2.908503in}}%
\pgfpathclose%
\pgfusepath{fill}%
\end{pgfscope}%
\begin{pgfscope}%
\pgfpathrectangle{\pgfqpoint{1.150000in}{0.150000in}}{\pgfqpoint{5.700000in}{5.700000in}}%
\pgfusepath{clip}%
\pgfsetbuttcap%
\pgfsetroundjoin%
\definecolor{currentfill}{rgb}{0.262138,0.242286,0.520837}%
\pgfsetfillcolor{currentfill}%
\pgfsetfillopacity{0.700000}%
\pgfsetlinewidth{0.000000pt}%
\definecolor{currentstroke}{rgb}{0.000000,0.000000,0.000000}%
\pgfsetstrokecolor{currentstroke}%
\pgfsetdash{}{0pt}%
\pgfpathmoveto{\pgfqpoint{4.061108in}{2.887535in}}%
\pgfpathlineto{\pgfqpoint{4.074495in}{2.880552in}}%
\pgfpathlineto{\pgfqpoint{4.087886in}{2.873655in}}%
\pgfpathlineto{\pgfqpoint{4.101282in}{2.866844in}}%
\pgfpathlineto{\pgfqpoint{4.114682in}{2.860117in}}%
\pgfpathlineto{\pgfqpoint{4.122406in}{2.875552in}}%
\pgfpathlineto{\pgfqpoint{4.130128in}{2.891236in}}%
\pgfpathlineto{\pgfqpoint{4.137846in}{2.907176in}}%
\pgfpathlineto{\pgfqpoint{4.145562in}{2.923379in}}%
\pgfpathlineto{\pgfqpoint{4.132169in}{2.930461in}}%
\pgfpathlineto{\pgfqpoint{4.118779in}{2.937627in}}%
\pgfpathlineto{\pgfqpoint{4.105394in}{2.944880in}}%
\pgfpathlineto{\pgfqpoint{4.092013in}{2.952218in}}%
\pgfpathlineto{\pgfqpoint{4.084291in}{2.935653in}}%
\pgfpathlineto{\pgfqpoint{4.076567in}{2.919355in}}%
\pgfpathlineto{\pgfqpoint{4.068839in}{2.903318in}}%
\pgfpathlineto{\pgfqpoint{4.061108in}{2.887535in}}%
\pgfpathclose%
\pgfusepath{fill}%
\end{pgfscope}%
\begin{pgfscope}%
\pgfpathrectangle{\pgfqpoint{1.150000in}{0.150000in}}{\pgfqpoint{5.700000in}{5.700000in}}%
\pgfusepath{clip}%
\pgfsetbuttcap%
\pgfsetroundjoin%
\definecolor{currentfill}{rgb}{0.246811,0.283237,0.535941}%
\pgfsetfillcolor{currentfill}%
\pgfsetfillopacity{0.700000}%
\pgfsetlinewidth{0.000000pt}%
\definecolor{currentstroke}{rgb}{0.000000,0.000000,0.000000}%
\pgfsetstrokecolor{currentstroke}%
\pgfsetdash{}{0pt}%
\pgfpathmoveto{\pgfqpoint{2.872111in}{2.992579in}}%
\pgfpathlineto{\pgfqpoint{2.885426in}{2.980574in}}%
\pgfpathlineto{\pgfqpoint{2.898739in}{2.968707in}}%
\pgfpathlineto{\pgfqpoint{2.912051in}{2.956976in}}%
\pgfpathlineto{\pgfqpoint{2.925362in}{2.945381in}}%
\pgfpathlineto{\pgfqpoint{2.933394in}{2.958277in}}%
\pgfpathlineto{\pgfqpoint{2.941419in}{2.971333in}}%
\pgfpathlineto{\pgfqpoint{2.949435in}{2.984550in}}%
\pgfpathlineto{\pgfqpoint{2.957445in}{2.997934in}}%
\pgfpathlineto{\pgfqpoint{2.944141in}{3.009706in}}%
\pgfpathlineto{\pgfqpoint{2.930836in}{3.021612in}}%
\pgfpathlineto{\pgfqpoint{2.917530in}{3.033656in}}%
\pgfpathlineto{\pgfqpoint{2.904223in}{3.045838in}}%
\pgfpathlineto{\pgfqpoint{2.896207in}{3.032270in}}%
\pgfpathlineto{\pgfqpoint{2.888183in}{3.018873in}}%
\pgfpathlineto{\pgfqpoint{2.880151in}{3.005644in}}%
\pgfpathlineto{\pgfqpoint{2.872111in}{2.992579in}}%
\pgfpathclose%
\pgfusepath{fill}%
\end{pgfscope}%
\begin{pgfscope}%
\pgfpathrectangle{\pgfqpoint{1.150000in}{0.150000in}}{\pgfqpoint{5.700000in}{5.700000in}}%
\pgfusepath{clip}%
\pgfsetbuttcap%
\pgfsetroundjoin%
\definecolor{currentfill}{rgb}{0.136408,0.541173,0.554483}%
\pgfsetfillcolor{currentfill}%
\pgfsetfillopacity{0.700000}%
\pgfsetlinewidth{0.000000pt}%
\definecolor{currentstroke}{rgb}{0.000000,0.000000,0.000000}%
\pgfsetstrokecolor{currentstroke}%
\pgfsetdash{}{0pt}%
\pgfpathmoveto{\pgfqpoint{4.775843in}{3.617548in}}%
\pgfpathlineto{\pgfqpoint{4.789302in}{3.607791in}}%
\pgfpathlineto{\pgfqpoint{4.802765in}{3.598111in}}%
\pgfpathlineto{\pgfqpoint{4.816232in}{3.588509in}}%
\pgfpathlineto{\pgfqpoint{4.829702in}{3.578983in}}%
\pgfpathlineto{\pgfqpoint{4.837477in}{3.608894in}}%
\pgfpathlineto{\pgfqpoint{4.845261in}{3.639405in}}%
\pgfpathlineto{\pgfqpoint{4.853055in}{3.670528in}}%
\pgfpathlineto{\pgfqpoint{4.839585in}{3.680493in}}%
\pgfpathlineto{\pgfqpoint{4.826119in}{3.690536in}}%
\pgfpathlineto{\pgfqpoint{4.812657in}{3.700656in}}%
\pgfpathlineto{\pgfqpoint{4.799198in}{3.710853in}}%
\pgfpathlineto{\pgfqpoint{4.791404in}{3.679137in}}%
\pgfpathlineto{\pgfqpoint{4.783619in}{3.648039in}}%
\pgfpathlineto{\pgfqpoint{4.775843in}{3.617548in}}%
\pgfpathclose%
\pgfusepath{fill}%
\end{pgfscope}%
\begin{pgfscope}%
\pgfpathrectangle{\pgfqpoint{1.150000in}{0.150000in}}{\pgfqpoint{5.700000in}{5.700000in}}%
\pgfusepath{clip}%
\pgfsetbuttcap%
\pgfsetroundjoin%
\definecolor{currentfill}{rgb}{0.270595,0.214069,0.507052}%
\pgfsetfillcolor{currentfill}%
\pgfsetfillopacity{0.700000}%
\pgfsetlinewidth{0.000000pt}%
\definecolor{currentstroke}{rgb}{0.000000,0.000000,0.000000}%
\pgfsetstrokecolor{currentstroke}%
\pgfsetdash{}{0pt}%
\pgfpathmoveto{\pgfqpoint{3.754093in}{2.824397in}}%
\pgfpathlineto{\pgfqpoint{3.767437in}{2.816922in}}%
\pgfpathlineto{\pgfqpoint{3.780784in}{2.809541in}}%
\pgfpathlineto{\pgfqpoint{3.794134in}{2.802252in}}%
\pgfpathlineto{\pgfqpoint{3.807489in}{2.795057in}}%
\pgfpathlineto{\pgfqpoint{3.815287in}{2.809124in}}%
\pgfpathlineto{\pgfqpoint{3.823080in}{2.823390in}}%
\pgfpathlineto{\pgfqpoint{3.830869in}{2.837859in}}%
\pgfpathlineto{\pgfqpoint{3.838654in}{2.852538in}}%
\pgfpathlineto{\pgfqpoint{3.825306in}{2.860029in}}%
\pgfpathlineto{\pgfqpoint{3.811961in}{2.867613in}}%
\pgfpathlineto{\pgfqpoint{3.798620in}{2.875289in}}%
\pgfpathlineto{\pgfqpoint{3.785282in}{2.883059in}}%
\pgfpathlineto{\pgfqpoint{3.777492in}{2.868078in}}%
\pgfpathlineto{\pgfqpoint{3.769697in}{2.853310in}}%
\pgfpathlineto{\pgfqpoint{3.761898in}{2.838752in}}%
\pgfpathlineto{\pgfqpoint{3.754093in}{2.824397in}}%
\pgfpathclose%
\pgfusepath{fill}%
\end{pgfscope}%
\begin{pgfscope}%
\pgfpathrectangle{\pgfqpoint{1.150000in}{0.150000in}}{\pgfqpoint{5.700000in}{5.700000in}}%
\pgfusepath{clip}%
\pgfsetbuttcap%
\pgfsetroundjoin%
\definecolor{currentfill}{rgb}{0.199430,0.387607,0.554642}%
\pgfsetfillcolor{currentfill}%
\pgfsetfillopacity{0.700000}%
\pgfsetlinewidth{0.000000pt}%
\definecolor{currentstroke}{rgb}{0.000000,0.000000,0.000000}%
\pgfsetstrokecolor{currentstroke}%
\pgfsetdash{}{0pt}%
\pgfpathmoveto{\pgfqpoint{4.652360in}{3.203154in}}%
\pgfpathlineto{\pgfqpoint{4.665833in}{3.195527in}}%
\pgfpathlineto{\pgfqpoint{4.679311in}{3.187976in}}%
\pgfpathlineto{\pgfqpoint{4.692794in}{3.180501in}}%
\pgfpathlineto{\pgfqpoint{4.706282in}{3.173102in}}%
\pgfpathlineto{\pgfqpoint{4.713956in}{3.194875in}}%
\pgfpathlineto{\pgfqpoint{4.721634in}{3.217081in}}%
\pgfpathlineto{\pgfqpoint{4.729316in}{3.239729in}}%
\pgfpathlineto{\pgfqpoint{4.737002in}{3.262831in}}%
\pgfpathlineto{\pgfqpoint{4.723520in}{3.270728in}}%
\pgfpathlineto{\pgfqpoint{4.710043in}{3.278701in}}%
\pgfpathlineto{\pgfqpoint{4.696570in}{3.286751in}}%
\pgfpathlineto{\pgfqpoint{4.683102in}{3.294877in}}%
\pgfpathlineto{\pgfqpoint{4.675411in}{3.271269in}}%
\pgfpathlineto{\pgfqpoint{4.667723in}{3.248119in}}%
\pgfpathlineto{\pgfqpoint{4.660040in}{3.225417in}}%
\pgfpathlineto{\pgfqpoint{4.652360in}{3.203154in}}%
\pgfpathclose%
\pgfusepath{fill}%
\end{pgfscope}%
\begin{pgfscope}%
\pgfpathrectangle{\pgfqpoint{1.150000in}{0.150000in}}{\pgfqpoint{5.700000in}{5.700000in}}%
\pgfusepath{clip}%
\pgfsetbuttcap%
\pgfsetroundjoin%
\definecolor{currentfill}{rgb}{0.266580,0.228262,0.514349}%
\pgfsetfillcolor{currentfill}%
\pgfsetfillopacity{0.700000}%
\pgfsetlinewidth{0.000000pt}%
\definecolor{currentstroke}{rgb}{0.000000,0.000000,0.000000}%
\pgfsetstrokecolor{currentstroke}%
\pgfsetdash{}{0pt}%
\pgfpathmoveto{\pgfqpoint{3.976619in}{2.854282in}}%
\pgfpathlineto{\pgfqpoint{3.989995in}{2.847286in}}%
\pgfpathlineto{\pgfqpoint{4.003376in}{2.840378in}}%
\pgfpathlineto{\pgfqpoint{4.016761in}{2.833557in}}%
\pgfpathlineto{\pgfqpoint{4.030150in}{2.826823in}}%
\pgfpathlineto{\pgfqpoint{4.037895in}{2.841651in}}%
\pgfpathlineto{\pgfqpoint{4.045636in}{2.856708in}}%
\pgfpathlineto{\pgfqpoint{4.053374in}{2.872001in}}%
\pgfpathlineto{\pgfqpoint{4.061108in}{2.887535in}}%
\pgfpathlineto{\pgfqpoint{4.047725in}{2.894605in}}%
\pgfpathlineto{\pgfqpoint{4.034347in}{2.901761in}}%
\pgfpathlineto{\pgfqpoint{4.020972in}{2.909004in}}%
\pgfpathlineto{\pgfqpoint{4.007601in}{2.916335in}}%
\pgfpathlineto{\pgfqpoint{3.999861in}{2.900458in}}%
\pgfpathlineto{\pgfqpoint{3.992117in}{2.884827in}}%
\pgfpathlineto{\pgfqpoint{3.984370in}{2.869437in}}%
\pgfpathlineto{\pgfqpoint{3.976619in}{2.854282in}}%
\pgfpathclose%
\pgfusepath{fill}%
\end{pgfscope}%
\begin{pgfscope}%
\pgfpathrectangle{\pgfqpoint{1.150000in}{0.150000in}}{\pgfqpoint{5.700000in}{5.700000in}}%
\pgfusepath{clip}%
\pgfsetbuttcap%
\pgfsetroundjoin%
\definecolor{currentfill}{rgb}{0.271828,0.209303,0.504434}%
\pgfsetfillcolor{currentfill}%
\pgfsetfillopacity{0.700000}%
\pgfsetlinewidth{0.000000pt}%
\definecolor{currentstroke}{rgb}{0.000000,0.000000,0.000000}%
\pgfsetstrokecolor{currentstroke}%
\pgfsetdash{}{0pt}%
\pgfpathmoveto{\pgfqpoint{3.393410in}{2.820885in}}%
\pgfpathlineto{\pgfqpoint{3.406717in}{2.812176in}}%
\pgfpathlineto{\pgfqpoint{3.420026in}{2.803573in}}%
\pgfpathlineto{\pgfqpoint{3.433337in}{2.795076in}}%
\pgfpathlineto{\pgfqpoint{3.446651in}{2.786685in}}%
\pgfpathlineto{\pgfqpoint{3.454546in}{2.799903in}}%
\pgfpathlineto{\pgfqpoint{3.462435in}{2.813287in}}%
\pgfpathlineto{\pgfqpoint{3.470319in}{2.826840in}}%
\pgfpathlineto{\pgfqpoint{3.478196in}{2.840567in}}%
\pgfpathlineto{\pgfqpoint{3.464889in}{2.849194in}}%
\pgfpathlineto{\pgfqpoint{3.451584in}{2.857927in}}%
\pgfpathlineto{\pgfqpoint{3.438281in}{2.866765in}}%
\pgfpathlineto{\pgfqpoint{3.424980in}{2.875710in}}%
\pgfpathlineto{\pgfqpoint{3.417097in}{2.861740in}}%
\pgfpathlineto{\pgfqpoint{3.409207in}{2.847948in}}%
\pgfpathlineto{\pgfqpoint{3.401312in}{2.834331in}}%
\pgfpathlineto{\pgfqpoint{3.393410in}{2.820885in}}%
\pgfpathclose%
\pgfusepath{fill}%
\end{pgfscope}%
\begin{pgfscope}%
\pgfpathrectangle{\pgfqpoint{1.150000in}{0.150000in}}{\pgfqpoint{5.700000in}{5.700000in}}%
\pgfusepath{clip}%
\pgfsetbuttcap%
\pgfsetroundjoin%
\definecolor{currentfill}{rgb}{0.269308,0.218818,0.509577}%
\pgfsetfillcolor{currentfill}%
\pgfsetfillopacity{0.700000}%
\pgfsetlinewidth{0.000000pt}%
\definecolor{currentstroke}{rgb}{0.000000,0.000000,0.000000}%
\pgfsetstrokecolor{currentstroke}%
\pgfsetdash{}{0pt}%
\pgfpathmoveto{\pgfqpoint{3.255295in}{2.840585in}}%
\pgfpathlineto{\pgfqpoint{3.268595in}{2.831212in}}%
\pgfpathlineto{\pgfqpoint{3.281897in}{2.821951in}}%
\pgfpathlineto{\pgfqpoint{3.295200in}{2.812803in}}%
\pgfpathlineto{\pgfqpoint{3.308505in}{2.803765in}}%
\pgfpathlineto{\pgfqpoint{3.316439in}{2.816783in}}%
\pgfpathlineto{\pgfqpoint{3.324366in}{2.829958in}}%
\pgfpathlineto{\pgfqpoint{3.332286in}{2.843296in}}%
\pgfpathlineto{\pgfqpoint{3.340200in}{2.856801in}}%
\pgfpathlineto{\pgfqpoint{3.326902in}{2.866054in}}%
\pgfpathlineto{\pgfqpoint{3.313606in}{2.875418in}}%
\pgfpathlineto{\pgfqpoint{3.300310in}{2.884894in}}%
\pgfpathlineto{\pgfqpoint{3.287016in}{2.894483in}}%
\pgfpathlineto{\pgfqpoint{3.279096in}{2.880756in}}%
\pgfpathlineto{\pgfqpoint{3.271169in}{2.867199in}}%
\pgfpathlineto{\pgfqpoint{3.263235in}{2.853811in}}%
\pgfpathlineto{\pgfqpoint{3.255295in}{2.840585in}}%
\pgfpathclose%
\pgfusepath{fill}%
\end{pgfscope}%
\begin{pgfscope}%
\pgfpathrectangle{\pgfqpoint{1.150000in}{0.150000in}}{\pgfqpoint{5.700000in}{5.700000in}}%
\pgfusepath{clip}%
\pgfsetbuttcap%
\pgfsetroundjoin%
\definecolor{currentfill}{rgb}{0.273006,0.204520,0.501721}%
\pgfsetfillcolor{currentfill}%
\pgfsetfillopacity{0.700000}%
\pgfsetlinewidth{0.000000pt}%
\definecolor{currentstroke}{rgb}{0.000000,0.000000,0.000000}%
\pgfsetstrokecolor{currentstroke}%
\pgfsetdash{}{0pt}%
\pgfpathmoveto{\pgfqpoint{3.531448in}{2.807093in}}%
\pgfpathlineto{\pgfqpoint{3.544767in}{2.798980in}}%
\pgfpathlineto{\pgfqpoint{3.558089in}{2.790968in}}%
\pgfpathlineto{\pgfqpoint{3.571414in}{2.783056in}}%
\pgfpathlineto{\pgfqpoint{3.584742in}{2.775244in}}%
\pgfpathlineto{\pgfqpoint{3.592601in}{2.788653in}}%
\pgfpathlineto{\pgfqpoint{3.600454in}{2.802235in}}%
\pgfpathlineto{\pgfqpoint{3.608302in}{2.815995in}}%
\pgfpathlineto{\pgfqpoint{3.616145in}{2.829938in}}%
\pgfpathlineto{\pgfqpoint{3.602823in}{2.838006in}}%
\pgfpathlineto{\pgfqpoint{3.589505in}{2.846173in}}%
\pgfpathlineto{\pgfqpoint{3.576189in}{2.854441in}}%
\pgfpathlineto{\pgfqpoint{3.562875in}{2.862810in}}%
\pgfpathlineto{\pgfqpoint{3.555027in}{2.848604in}}%
\pgfpathlineto{\pgfqpoint{3.547173in}{2.834586in}}%
\pgfpathlineto{\pgfqpoint{3.539313in}{2.820750in}}%
\pgfpathlineto{\pgfqpoint{3.531448in}{2.807093in}}%
\pgfpathclose%
\pgfusepath{fill}%
\end{pgfscope}%
\begin{pgfscope}%
\pgfpathrectangle{\pgfqpoint{1.150000in}{0.150000in}}{\pgfqpoint{5.700000in}{5.700000in}}%
\pgfusepath{clip}%
\pgfsetbuttcap%
\pgfsetroundjoin%
\definecolor{currentfill}{rgb}{0.253935,0.265254,0.529983}%
\pgfsetfillcolor{currentfill}%
\pgfsetfillopacity{0.700000}%
\pgfsetlinewidth{0.000000pt}%
\definecolor{currentstroke}{rgb}{0.000000,0.000000,0.000000}%
\pgfsetstrokecolor{currentstroke}%
\pgfsetdash{}{0pt}%
\pgfpathmoveto{\pgfqpoint{2.925362in}{2.945381in}}%
\pgfpathlineto{\pgfqpoint{2.938671in}{2.933920in}}%
\pgfpathlineto{\pgfqpoint{2.951980in}{2.922593in}}%
\pgfpathlineto{\pgfqpoint{2.965289in}{2.911397in}}%
\pgfpathlineto{\pgfqpoint{2.978596in}{2.900332in}}%
\pgfpathlineto{\pgfqpoint{2.986621in}{2.913060in}}%
\pgfpathlineto{\pgfqpoint{2.994638in}{2.925941in}}%
\pgfpathlineto{\pgfqpoint{3.002648in}{2.938981in}}%
\pgfpathlineto{\pgfqpoint{3.010650in}{2.952181in}}%
\pgfpathlineto{\pgfqpoint{2.997350in}{2.963422in}}%
\pgfpathlineto{\pgfqpoint{2.984049in}{2.974794in}}%
\pgfpathlineto{\pgfqpoint{2.970747in}{2.986297in}}%
\pgfpathlineto{\pgfqpoint{2.957445in}{2.997934in}}%
\pgfpathlineto{\pgfqpoint{2.949435in}{2.984550in}}%
\pgfpathlineto{\pgfqpoint{2.941419in}{2.971333in}}%
\pgfpathlineto{\pgfqpoint{2.933394in}{2.958277in}}%
\pgfpathlineto{\pgfqpoint{2.925362in}{2.945381in}}%
\pgfpathclose%
\pgfusepath{fill}%
\end{pgfscope}%
\begin{pgfscope}%
\pgfpathrectangle{\pgfqpoint{1.150000in}{0.150000in}}{\pgfqpoint{5.700000in}{5.700000in}}%
\pgfusepath{clip}%
\pgfsetbuttcap%
\pgfsetroundjoin%
\definecolor{currentfill}{rgb}{0.244972,0.287675,0.537260}%
\pgfsetfillcolor{currentfill}%
\pgfsetfillopacity{0.700000}%
\pgfsetlinewidth{0.000000pt}%
\definecolor{currentstroke}{rgb}{0.000000,0.000000,0.000000}%
\pgfsetstrokecolor{currentstroke}%
\pgfsetdash{}{0pt}%
\pgfpathmoveto{\pgfqpoint{4.368134in}{2.975542in}}%
\pgfpathlineto{\pgfqpoint{4.381574in}{2.968761in}}%
\pgfpathlineto{\pgfqpoint{4.395020in}{2.962060in}}%
\pgfpathlineto{\pgfqpoint{4.408470in}{2.955438in}}%
\pgfpathlineto{\pgfqpoint{4.421925in}{2.948897in}}%
\pgfpathlineto{\pgfqpoint{4.429599in}{2.966143in}}%
\pgfpathlineto{\pgfqpoint{4.437271in}{2.983702in}}%
\pgfpathlineto{\pgfqpoint{4.444943in}{3.001583in}}%
\pgfpathlineto{\pgfqpoint{4.452615in}{3.019794in}}%
\pgfpathlineto{\pgfqpoint{4.439167in}{3.026751in}}%
\pgfpathlineto{\pgfqpoint{4.425723in}{3.033789in}}%
\pgfpathlineto{\pgfqpoint{4.412284in}{3.040905in}}%
\pgfpathlineto{\pgfqpoint{4.398850in}{3.048102in}}%
\pgfpathlineto{\pgfqpoint{4.391172in}{3.029468in}}%
\pgfpathlineto{\pgfqpoint{4.383493in}{3.011169in}}%
\pgfpathlineto{\pgfqpoint{4.375814in}{2.993196in}}%
\pgfpathlineto{\pgfqpoint{4.368134in}{2.975542in}}%
\pgfpathclose%
\pgfusepath{fill}%
\end{pgfscope}%
\begin{pgfscope}%
\pgfpathrectangle{\pgfqpoint{1.150000in}{0.150000in}}{\pgfqpoint{5.700000in}{5.700000in}}%
\pgfusepath{clip}%
\pgfsetbuttcap%
\pgfsetroundjoin%
\definecolor{currentfill}{rgb}{0.235526,0.309527,0.542944}%
\pgfsetfillcolor{currentfill}%
\pgfsetfillopacity{0.700000}%
\pgfsetlinewidth{0.000000pt}%
\definecolor{currentstroke}{rgb}{0.000000,0.000000,0.000000}%
\pgfsetstrokecolor{currentstroke}%
\pgfsetdash{}{0pt}%
\pgfpathmoveto{\pgfqpoint{4.452615in}{3.019794in}}%
\pgfpathlineto{\pgfqpoint{4.466069in}{3.012915in}}%
\pgfpathlineto{\pgfqpoint{4.479527in}{3.006115in}}%
\pgfpathlineto{\pgfqpoint{4.492990in}{2.999394in}}%
\pgfpathlineto{\pgfqpoint{4.506458in}{2.992750in}}%
\pgfpathlineto{\pgfqpoint{4.514123in}{3.010869in}}%
\pgfpathlineto{\pgfqpoint{4.521789in}{3.029327in}}%
\pgfpathlineto{\pgfqpoint{4.529455in}{3.048134in}}%
\pgfpathlineto{\pgfqpoint{4.537122in}{3.067296in}}%
\pgfpathlineto{\pgfqpoint{4.523660in}{3.074376in}}%
\pgfpathlineto{\pgfqpoint{4.510204in}{3.081534in}}%
\pgfpathlineto{\pgfqpoint{4.496752in}{3.088770in}}%
\pgfpathlineto{\pgfqpoint{4.483305in}{3.096085in}}%
\pgfpathlineto{\pgfqpoint{4.475632in}{3.076478in}}%
\pgfpathlineto{\pgfqpoint{4.467959in}{3.057233in}}%
\pgfpathlineto{\pgfqpoint{4.460287in}{3.038341in}}%
\pgfpathlineto{\pgfqpoint{4.452615in}{3.019794in}}%
\pgfpathclose%
\pgfusepath{fill}%
\end{pgfscope}%
\begin{pgfscope}%
\pgfpathrectangle{\pgfqpoint{1.150000in}{0.150000in}}{\pgfqpoint{5.700000in}{5.700000in}}%
\pgfusepath{clip}%
\pgfsetbuttcap%
\pgfsetroundjoin%
\definecolor{currentfill}{rgb}{0.172719,0.448791,0.557885}%
\pgfsetfillcolor{currentfill}%
\pgfsetfillopacity{0.700000}%
\pgfsetlinewidth{0.000000pt}%
\definecolor{currentstroke}{rgb}{0.000000,0.000000,0.000000}%
\pgfsetstrokecolor{currentstroke}%
\pgfsetdash{}{0pt}%
\pgfpathmoveto{\pgfqpoint{4.767798in}{3.359954in}}%
\pgfpathlineto{\pgfqpoint{4.781279in}{3.351613in}}%
\pgfpathlineto{\pgfqpoint{4.794766in}{3.343347in}}%
\pgfpathlineto{\pgfqpoint{4.808257in}{3.335156in}}%
\pgfpathlineto{\pgfqpoint{4.821752in}{3.327041in}}%
\pgfpathlineto{\pgfqpoint{4.829461in}{3.352018in}}%
\pgfpathlineto{\pgfqpoint{4.837176in}{3.377502in}}%
\pgfpathlineto{\pgfqpoint{4.844898in}{3.403503in}}%
\pgfpathlineto{\pgfqpoint{4.852627in}{3.430031in}}%
\pgfpathlineto{\pgfqpoint{4.839136in}{3.438689in}}%
\pgfpathlineto{\pgfqpoint{4.825650in}{3.447421in}}%
\pgfpathlineto{\pgfqpoint{4.812168in}{3.456230in}}%
\pgfpathlineto{\pgfqpoint{4.798690in}{3.465114in}}%
\pgfpathlineto{\pgfqpoint{4.790957in}{3.438034in}}%
\pgfpathlineto{\pgfqpoint{4.783231in}{3.411488in}}%
\pgfpathlineto{\pgfqpoint{4.775511in}{3.385465in}}%
\pgfpathlineto{\pgfqpoint{4.767798in}{3.359954in}}%
\pgfpathclose%
\pgfusepath{fill}%
\end{pgfscope}%
\begin{pgfscope}%
\pgfpathrectangle{\pgfqpoint{1.150000in}{0.150000in}}{\pgfqpoint{5.700000in}{5.700000in}}%
\pgfusepath{clip}%
\pgfsetbuttcap%
\pgfsetroundjoin%
\definecolor{currentfill}{rgb}{0.265145,0.232956,0.516599}%
\pgfsetfillcolor{currentfill}%
\pgfsetfillopacity{0.700000}%
\pgfsetlinewidth{0.000000pt}%
\definecolor{currentstroke}{rgb}{0.000000,0.000000,0.000000}%
\pgfsetstrokecolor{currentstroke}%
\pgfsetdash{}{0pt}%
\pgfpathmoveto{\pgfqpoint{3.117046in}{2.866829in}}%
\pgfpathlineto{\pgfqpoint{3.130346in}{2.856716in}}%
\pgfpathlineto{\pgfqpoint{3.143647in}{2.846723in}}%
\pgfpathlineto{\pgfqpoint{3.156948in}{2.836850in}}%
\pgfpathlineto{\pgfqpoint{3.170250in}{2.827095in}}%
\pgfpathlineto{\pgfqpoint{3.178224in}{2.839894in}}%
\pgfpathlineto{\pgfqpoint{3.186191in}{2.852846in}}%
\pgfpathlineto{\pgfqpoint{3.194151in}{2.865955in}}%
\pgfpathlineto{\pgfqpoint{3.202105in}{2.879225in}}%
\pgfpathlineto{\pgfqpoint{3.188810in}{2.889176in}}%
\pgfpathlineto{\pgfqpoint{3.175515in}{2.899245in}}%
\pgfpathlineto{\pgfqpoint{3.162221in}{2.909434in}}%
\pgfpathlineto{\pgfqpoint{3.148928in}{2.919743in}}%
\pgfpathlineto{\pgfqpoint{3.140968in}{2.906270in}}%
\pgfpathlineto{\pgfqpoint{3.133001in}{2.892962in}}%
\pgfpathlineto{\pgfqpoint{3.125027in}{2.879817in}}%
\pgfpathlineto{\pgfqpoint{3.117046in}{2.866829in}}%
\pgfpathclose%
\pgfusepath{fill}%
\end{pgfscope}%
\begin{pgfscope}%
\pgfpathrectangle{\pgfqpoint{1.150000in}{0.150000in}}{\pgfqpoint{5.700000in}{5.700000in}}%
\pgfusepath{clip}%
\pgfsetbuttcap%
\pgfsetroundjoin%
\definecolor{currentfill}{rgb}{0.252194,0.269783,0.531579}%
\pgfsetfillcolor{currentfill}%
\pgfsetfillopacity{0.700000}%
\pgfsetlinewidth{0.000000pt}%
\definecolor{currentstroke}{rgb}{0.000000,0.000000,0.000000}%
\pgfsetstrokecolor{currentstroke}%
\pgfsetdash{}{0pt}%
\pgfpathmoveto{\pgfqpoint{4.283661in}{2.934310in}}%
\pgfpathlineto{\pgfqpoint{4.297089in}{2.927601in}}%
\pgfpathlineto{\pgfqpoint{4.310522in}{2.920974in}}%
\pgfpathlineto{\pgfqpoint{4.323960in}{2.914428in}}%
\pgfpathlineto{\pgfqpoint{4.337402in}{2.907962in}}%
\pgfpathlineto{\pgfqpoint{4.345087in}{2.924416in}}%
\pgfpathlineto{\pgfqpoint{4.352771in}{2.941159in}}%
\pgfpathlineto{\pgfqpoint{4.360453in}{2.958198in}}%
\pgfpathlineto{\pgfqpoint{4.368134in}{2.975542in}}%
\pgfpathlineto{\pgfqpoint{4.354698in}{2.982403in}}%
\pgfpathlineto{\pgfqpoint{4.341267in}{2.989345in}}%
\pgfpathlineto{\pgfqpoint{4.327841in}{2.996368in}}%
\pgfpathlineto{\pgfqpoint{4.314419in}{3.003472in}}%
\pgfpathlineto{\pgfqpoint{4.306731in}{2.985725in}}%
\pgfpathlineto{\pgfqpoint{4.299043in}{2.968288in}}%
\pgfpathlineto{\pgfqpoint{4.291352in}{2.951152in}}%
\pgfpathlineto{\pgfqpoint{4.283661in}{2.934310in}}%
\pgfpathclose%
\pgfusepath{fill}%
\end{pgfscope}%
\begin{pgfscope}%
\pgfpathrectangle{\pgfqpoint{1.150000in}{0.150000in}}{\pgfqpoint{5.700000in}{5.700000in}}%
\pgfusepath{clip}%
\pgfsetbuttcap%
\pgfsetroundjoin%
\definecolor{currentfill}{rgb}{0.156270,0.489624,0.557936}%
\pgfsetfillcolor{currentfill}%
\pgfsetfillopacity{0.700000}%
\pgfsetlinewidth{0.000000pt}%
\definecolor{currentstroke}{rgb}{0.000000,0.000000,0.000000}%
\pgfsetstrokecolor{currentstroke}%
\pgfsetdash{}{0pt}%
\pgfpathmoveto{\pgfqpoint{4.798690in}{3.465114in}}%
\pgfpathlineto{\pgfqpoint{4.812168in}{3.456230in}}%
\pgfpathlineto{\pgfqpoint{4.825650in}{3.447421in}}%
\pgfpathlineto{\pgfqpoint{4.839136in}{3.438689in}}%
\pgfpathlineto{\pgfqpoint{4.852627in}{3.430031in}}%
\pgfpathlineto{\pgfqpoint{4.860365in}{3.457099in}}%
\pgfpathlineto{\pgfqpoint{4.868110in}{3.484715in}}%
\pgfpathlineto{\pgfqpoint{4.875864in}{3.512892in}}%
\pgfpathlineto{\pgfqpoint{4.883627in}{3.541641in}}%
\pgfpathlineto{\pgfqpoint{4.870140in}{3.550863in}}%
\pgfpathlineto{\pgfqpoint{4.856656in}{3.560160in}}%
\pgfpathlineto{\pgfqpoint{4.843177in}{3.569534in}}%
\pgfpathlineto{\pgfqpoint{4.829702in}{3.578983in}}%
\pgfpathlineto{\pgfqpoint{4.821937in}{3.549661in}}%
\pgfpathlineto{\pgfqpoint{4.814180in}{3.520916in}}%
\pgfpathlineto{\pgfqpoint{4.806431in}{3.492737in}}%
\pgfpathlineto{\pgfqpoint{4.798690in}{3.465114in}}%
\pgfpathclose%
\pgfusepath{fill}%
\end{pgfscope}%
\begin{pgfscope}%
\pgfpathrectangle{\pgfqpoint{1.150000in}{0.150000in}}{\pgfqpoint{5.700000in}{5.700000in}}%
\pgfusepath{clip}%
\pgfsetbuttcap%
\pgfsetroundjoin%
\definecolor{currentfill}{rgb}{0.225863,0.330805,0.547314}%
\pgfsetfillcolor{currentfill}%
\pgfsetfillopacity{0.700000}%
\pgfsetlinewidth{0.000000pt}%
\definecolor{currentstroke}{rgb}{0.000000,0.000000,0.000000}%
\pgfsetstrokecolor{currentstroke}%
\pgfsetdash{}{0pt}%
\pgfpathmoveto{\pgfqpoint{4.537122in}{3.067296in}}%
\pgfpathlineto{\pgfqpoint{4.550588in}{3.060295in}}%
\pgfpathlineto{\pgfqpoint{4.564059in}{3.053370in}}%
\pgfpathlineto{\pgfqpoint{4.577536in}{3.046524in}}%
\pgfpathlineto{\pgfqpoint{4.591017in}{3.039754in}}%
\pgfpathlineto{\pgfqpoint{4.598678in}{3.058832in}}%
\pgfpathlineto{\pgfqpoint{4.606340in}{3.078277in}}%
\pgfpathlineto{\pgfqpoint{4.614005in}{3.098099in}}%
\pgfpathlineto{\pgfqpoint{4.621671in}{3.118305in}}%
\pgfpathlineto{\pgfqpoint{4.608196in}{3.125532in}}%
\pgfpathlineto{\pgfqpoint{4.594726in}{3.132836in}}%
\pgfpathlineto{\pgfqpoint{4.581262in}{3.140217in}}%
\pgfpathlineto{\pgfqpoint{4.567802in}{3.147675in}}%
\pgfpathlineto{\pgfqpoint{4.560129in}{3.127004in}}%
\pgfpathlineto{\pgfqpoint{4.552459in}{3.106723in}}%
\pgfpathlineto{\pgfqpoint{4.544789in}{3.086823in}}%
\pgfpathlineto{\pgfqpoint{4.537122in}{3.067296in}}%
\pgfpathclose%
\pgfusepath{fill}%
\end{pgfscope}%
\begin{pgfscope}%
\pgfpathrectangle{\pgfqpoint{1.150000in}{0.150000in}}{\pgfqpoint{5.700000in}{5.700000in}}%
\pgfusepath{clip}%
\pgfsetbuttcap%
\pgfsetroundjoin%
\definecolor{currentfill}{rgb}{0.270595,0.214069,0.507052}%
\pgfsetfillcolor{currentfill}%
\pgfsetfillopacity{0.700000}%
\pgfsetlinewidth{0.000000pt}%
\definecolor{currentstroke}{rgb}{0.000000,0.000000,0.000000}%
\pgfsetstrokecolor{currentstroke}%
\pgfsetdash{}{0pt}%
\pgfpathmoveto{\pgfqpoint{3.892083in}{2.823490in}}%
\pgfpathlineto{\pgfqpoint{3.905450in}{2.816454in}}%
\pgfpathlineto{\pgfqpoint{3.918820in}{2.809508in}}%
\pgfpathlineto{\pgfqpoint{3.932195in}{2.802651in}}%
\pgfpathlineto{\pgfqpoint{3.945574in}{2.795882in}}%
\pgfpathlineto{\pgfqpoint{3.953341in}{2.810160in}}%
\pgfpathlineto{\pgfqpoint{3.961105in}{2.824649in}}%
\pgfpathlineto{\pgfqpoint{3.968864in}{2.839354in}}%
\pgfpathlineto{\pgfqpoint{3.976619in}{2.854282in}}%
\pgfpathlineto{\pgfqpoint{3.963246in}{2.861365in}}%
\pgfpathlineto{\pgfqpoint{3.949878in}{2.868538in}}%
\pgfpathlineto{\pgfqpoint{3.936513in}{2.875799in}}%
\pgfpathlineto{\pgfqpoint{3.923152in}{2.883151in}}%
\pgfpathlineto{\pgfqpoint{3.915391in}{2.867900in}}%
\pgfpathlineto{\pgfqpoint{3.907626in}{2.852877in}}%
\pgfpathlineto{\pgfqpoint{3.899857in}{2.838076in}}%
\pgfpathlineto{\pgfqpoint{3.892083in}{2.823490in}}%
\pgfpathclose%
\pgfusepath{fill}%
\end{pgfscope}%
\begin{pgfscope}%
\pgfpathrectangle{\pgfqpoint{1.150000in}{0.150000in}}{\pgfqpoint{5.700000in}{5.700000in}}%
\pgfusepath{clip}%
\pgfsetbuttcap%
\pgfsetroundjoin%
\definecolor{currentfill}{rgb}{0.273006,0.204520,0.501721}%
\pgfsetfillcolor{currentfill}%
\pgfsetfillopacity{0.700000}%
\pgfsetlinewidth{0.000000pt}%
\definecolor{currentstroke}{rgb}{0.000000,0.000000,0.000000}%
\pgfsetstrokecolor{currentstroke}%
\pgfsetdash{}{0pt}%
\pgfpathmoveto{\pgfqpoint{3.669459in}{2.798649in}}%
\pgfpathlineto{\pgfqpoint{3.682796in}{2.791070in}}%
\pgfpathlineto{\pgfqpoint{3.696136in}{2.783586in}}%
\pgfpathlineto{\pgfqpoint{3.709479in}{2.776197in}}%
\pgfpathlineto{\pgfqpoint{3.722825in}{2.768904in}}%
\pgfpathlineto{\pgfqpoint{3.730650in}{2.782498in}}%
\pgfpathlineto{\pgfqpoint{3.738469in}{2.796275in}}%
\pgfpathlineto{\pgfqpoint{3.746284in}{2.810239in}}%
\pgfpathlineto{\pgfqpoint{3.754093in}{2.824397in}}%
\pgfpathlineto{\pgfqpoint{3.740753in}{2.831966in}}%
\pgfpathlineto{\pgfqpoint{3.727416in}{2.839629in}}%
\pgfpathlineto{\pgfqpoint{3.714082in}{2.847388in}}%
\pgfpathlineto{\pgfqpoint{3.700752in}{2.855243in}}%
\pgfpathlineto{\pgfqpoint{3.692937in}{2.840803in}}%
\pgfpathlineto{\pgfqpoint{3.685116in}{2.826561in}}%
\pgfpathlineto{\pgfqpoint{3.677290in}{2.812511in}}%
\pgfpathlineto{\pgfqpoint{3.669459in}{2.798649in}}%
\pgfpathclose%
\pgfusepath{fill}%
\end{pgfscope}%
\begin{pgfscope}%
\pgfpathrectangle{\pgfqpoint{1.150000in}{0.150000in}}{\pgfqpoint{5.700000in}{5.700000in}}%
\pgfusepath{clip}%
\pgfsetbuttcap%
\pgfsetroundjoin%
\definecolor{currentfill}{rgb}{0.258965,0.251537,0.524736}%
\pgfsetfillcolor{currentfill}%
\pgfsetfillopacity{0.700000}%
\pgfsetlinewidth{0.000000pt}%
\definecolor{currentstroke}{rgb}{0.000000,0.000000,0.000000}%
\pgfsetstrokecolor{currentstroke}%
\pgfsetdash{}{0pt}%
\pgfpathmoveto{\pgfqpoint{4.199181in}{2.895895in}}%
\pgfpathlineto{\pgfqpoint{4.212597in}{2.889233in}}%
\pgfpathlineto{\pgfqpoint{4.226018in}{2.882654in}}%
\pgfpathlineto{\pgfqpoint{4.239444in}{2.876157in}}%
\pgfpathlineto{\pgfqpoint{4.252874in}{2.869742in}}%
\pgfpathlineto{\pgfqpoint{4.260574in}{2.885478in}}%
\pgfpathlineto{\pgfqpoint{4.268272in}{2.901480in}}%
\pgfpathlineto{\pgfqpoint{4.275967in}{2.917755in}}%
\pgfpathlineto{\pgfqpoint{4.283661in}{2.934310in}}%
\pgfpathlineto{\pgfqpoint{4.270237in}{2.941100in}}%
\pgfpathlineto{\pgfqpoint{4.256818in}{2.947973in}}%
\pgfpathlineto{\pgfqpoint{4.243404in}{2.954927in}}%
\pgfpathlineto{\pgfqpoint{4.229995in}{2.961965in}}%
\pgfpathlineto{\pgfqpoint{4.222295in}{2.945027in}}%
\pgfpathlineto{\pgfqpoint{4.214592in}{2.928374in}}%
\pgfpathlineto{\pgfqpoint{4.206888in}{2.911999in}}%
\pgfpathlineto{\pgfqpoint{4.199181in}{2.895895in}}%
\pgfpathclose%
\pgfusepath{fill}%
\end{pgfscope}%
\begin{pgfscope}%
\pgfpathrectangle{\pgfqpoint{1.150000in}{0.150000in}}{\pgfqpoint{5.700000in}{5.700000in}}%
\pgfusepath{clip}%
\pgfsetbuttcap%
\pgfsetroundjoin%
\definecolor{currentfill}{rgb}{0.188923,0.410910,0.556326}%
\pgfsetfillcolor{currentfill}%
\pgfsetfillopacity{0.700000}%
\pgfsetlinewidth{0.000000pt}%
\definecolor{currentstroke}{rgb}{0.000000,0.000000,0.000000}%
\pgfsetstrokecolor{currentstroke}%
\pgfsetdash{}{0pt}%
\pgfpathmoveto{\pgfqpoint{4.737002in}{3.262831in}}%
\pgfpathlineto{\pgfqpoint{4.750489in}{3.255009in}}%
\pgfpathlineto{\pgfqpoint{4.763980in}{3.247263in}}%
\pgfpathlineto{\pgfqpoint{4.777477in}{3.239593in}}%
\pgfpathlineto{\pgfqpoint{4.790979in}{3.231997in}}%
\pgfpathlineto{\pgfqpoint{4.798664in}{3.255048in}}%
\pgfpathlineto{\pgfqpoint{4.806354in}{3.278566in}}%
\pgfpathlineto{\pgfqpoint{4.814050in}{3.302560in}}%
\pgfpathlineto{\pgfqpoint{4.821752in}{3.327041in}}%
\pgfpathlineto{\pgfqpoint{4.808257in}{3.335156in}}%
\pgfpathlineto{\pgfqpoint{4.794766in}{3.343347in}}%
\pgfpathlineto{\pgfqpoint{4.781279in}{3.351613in}}%
\pgfpathlineto{\pgfqpoint{4.767798in}{3.359954in}}%
\pgfpathlineto{\pgfqpoint{4.760091in}{3.334946in}}%
\pgfpathlineto{\pgfqpoint{4.752389in}{3.310429in}}%
\pgfpathlineto{\pgfqpoint{4.744693in}{3.286394in}}%
\pgfpathlineto{\pgfqpoint{4.737002in}{3.262831in}}%
\pgfpathclose%
\pgfusepath{fill}%
\end{pgfscope}%
\begin{pgfscope}%
\pgfpathrectangle{\pgfqpoint{1.150000in}{0.150000in}}{\pgfqpoint{5.700000in}{5.700000in}}%
\pgfusepath{clip}%
\pgfsetbuttcap%
\pgfsetroundjoin%
\definecolor{currentfill}{rgb}{0.140536,0.530132,0.555659}%
\pgfsetfillcolor{currentfill}%
\pgfsetfillopacity{0.700000}%
\pgfsetlinewidth{0.000000pt}%
\definecolor{currentstroke}{rgb}{0.000000,0.000000,0.000000}%
\pgfsetstrokecolor{currentstroke}%
\pgfsetdash{}{0pt}%
\pgfpathmoveto{\pgfqpoint{4.829702in}{3.578983in}}%
\pgfpathlineto{\pgfqpoint{4.843177in}{3.569534in}}%
\pgfpathlineto{\pgfqpoint{4.856656in}{3.560160in}}%
\pgfpathlineto{\pgfqpoint{4.870140in}{3.550863in}}%
\pgfpathlineto{\pgfqpoint{4.883627in}{3.541641in}}%
\pgfpathlineto{\pgfqpoint{4.891399in}{3.570972in}}%
\pgfpathlineto{\pgfqpoint{4.899182in}{3.600898in}}%
\pgfpathlineto{\pgfqpoint{4.906974in}{3.631429in}}%
\pgfpathlineto{\pgfqpoint{4.893488in}{3.641090in}}%
\pgfpathlineto{\pgfqpoint{4.880006in}{3.650826in}}%
\pgfpathlineto{\pgfqpoint{4.866529in}{3.660639in}}%
\pgfpathlineto{\pgfqpoint{4.853055in}{3.670528in}}%
\pgfpathlineto{\pgfqpoint{4.845261in}{3.639405in}}%
\pgfpathlineto{\pgfqpoint{4.837477in}{3.608894in}}%
\pgfpathlineto{\pgfqpoint{4.829702in}{3.578983in}}%
\pgfpathclose%
\pgfusepath{fill}%
\end{pgfscope}%
\begin{pgfscope}%
\pgfpathrectangle{\pgfqpoint{1.150000in}{0.150000in}}{\pgfqpoint{5.700000in}{5.700000in}}%
\pgfusepath{clip}%
\pgfsetbuttcap%
\pgfsetroundjoin%
\definecolor{currentfill}{rgb}{0.216210,0.351535,0.550627}%
\pgfsetfillcolor{currentfill}%
\pgfsetfillopacity{0.700000}%
\pgfsetlinewidth{0.000000pt}%
\definecolor{currentstroke}{rgb}{0.000000,0.000000,0.000000}%
\pgfsetstrokecolor{currentstroke}%
\pgfsetdash{}{0pt}%
\pgfpathmoveto{\pgfqpoint{4.621671in}{3.118305in}}%
\pgfpathlineto{\pgfqpoint{4.635150in}{3.111155in}}%
\pgfpathlineto{\pgfqpoint{4.648635in}{3.104082in}}%
\pgfpathlineto{\pgfqpoint{4.662125in}{3.097085in}}%
\pgfpathlineto{\pgfqpoint{4.675619in}{3.090163in}}%
\pgfpathlineto{\pgfqpoint{4.683281in}{3.110294in}}%
\pgfpathlineto{\pgfqpoint{4.690945in}{3.130821in}}%
\pgfpathlineto{\pgfqpoint{4.698612in}{3.151754in}}%
\pgfpathlineto{\pgfqpoint{4.706282in}{3.173102in}}%
\pgfpathlineto{\pgfqpoint{4.692794in}{3.180501in}}%
\pgfpathlineto{\pgfqpoint{4.679311in}{3.187976in}}%
\pgfpathlineto{\pgfqpoint{4.665833in}{3.195527in}}%
\pgfpathlineto{\pgfqpoint{4.652360in}{3.203154in}}%
\pgfpathlineto{\pgfqpoint{4.644683in}{3.181321in}}%
\pgfpathlineto{\pgfqpoint{4.637010in}{3.159907in}}%
\pgfpathlineto{\pgfqpoint{4.629339in}{3.138905in}}%
\pgfpathlineto{\pgfqpoint{4.621671in}{3.118305in}}%
\pgfpathclose%
\pgfusepath{fill}%
\end{pgfscope}%
\begin{pgfscope}%
\pgfpathrectangle{\pgfqpoint{1.150000in}{0.150000in}}{\pgfqpoint{5.700000in}{5.700000in}}%
\pgfusepath{clip}%
\pgfsetbuttcap%
\pgfsetroundjoin%
\definecolor{currentfill}{rgb}{0.263663,0.237631,0.518762}%
\pgfsetfillcolor{currentfill}%
\pgfsetfillopacity{0.700000}%
\pgfsetlinewidth{0.000000pt}%
\definecolor{currentstroke}{rgb}{0.000000,0.000000,0.000000}%
\pgfsetstrokecolor{currentstroke}%
\pgfsetdash{}{0pt}%
\pgfpathmoveto{\pgfqpoint{4.114682in}{2.860117in}}%
\pgfpathlineto{\pgfqpoint{4.128086in}{2.853476in}}%
\pgfpathlineto{\pgfqpoint{4.141496in}{2.846918in}}%
\pgfpathlineto{\pgfqpoint{4.154909in}{2.840445in}}%
\pgfpathlineto{\pgfqpoint{4.168328in}{2.834055in}}%
\pgfpathlineto{\pgfqpoint{4.176045in}{2.849142in}}%
\pgfpathlineto{\pgfqpoint{4.183760in}{2.864473in}}%
\pgfpathlineto{\pgfqpoint{4.191472in}{2.880055in}}%
\pgfpathlineto{\pgfqpoint{4.199181in}{2.895895in}}%
\pgfpathlineto{\pgfqpoint{4.185770in}{2.902640in}}%
\pgfpathlineto{\pgfqpoint{4.172363in}{2.909469in}}%
\pgfpathlineto{\pgfqpoint{4.158960in}{2.916382in}}%
\pgfpathlineto{\pgfqpoint{4.145562in}{2.923379in}}%
\pgfpathlineto{\pgfqpoint{4.137846in}{2.907176in}}%
\pgfpathlineto{\pgfqpoint{4.130128in}{2.891236in}}%
\pgfpathlineto{\pgfqpoint{4.122406in}{2.875552in}}%
\pgfpathlineto{\pgfqpoint{4.114682in}{2.860117in}}%
\pgfpathclose%
\pgfusepath{fill}%
\end{pgfscope}%
\begin{pgfscope}%
\pgfpathrectangle{\pgfqpoint{1.150000in}{0.150000in}}{\pgfqpoint{5.700000in}{5.700000in}}%
\pgfusepath{clip}%
\pgfsetbuttcap%
\pgfsetroundjoin%
\definecolor{currentfill}{rgb}{0.260571,0.246922,0.522828}%
\pgfsetfillcolor{currentfill}%
\pgfsetfillopacity{0.700000}%
\pgfsetlinewidth{0.000000pt}%
\definecolor{currentstroke}{rgb}{0.000000,0.000000,0.000000}%
\pgfsetstrokecolor{currentstroke}%
\pgfsetdash{}{0pt}%
\pgfpathmoveto{\pgfqpoint{2.978596in}{2.900332in}}%
\pgfpathlineto{\pgfqpoint{2.991904in}{2.889397in}}%
\pgfpathlineto{\pgfqpoint{3.005210in}{2.878590in}}%
\pgfpathlineto{\pgfqpoint{3.018517in}{2.867910in}}%
\pgfpathlineto{\pgfqpoint{3.031824in}{2.857357in}}%
\pgfpathlineto{\pgfqpoint{3.039841in}{2.869917in}}%
\pgfpathlineto{\pgfqpoint{3.047851in}{2.882625in}}%
\pgfpathlineto{\pgfqpoint{3.055853in}{2.895486in}}%
\pgfpathlineto{\pgfqpoint{3.063848in}{2.908503in}}%
\pgfpathlineto{\pgfqpoint{3.050549in}{2.919232in}}%
\pgfpathlineto{\pgfqpoint{3.037250in}{2.930087in}}%
\pgfpathlineto{\pgfqpoint{3.023950in}{2.941070in}}%
\pgfpathlineto{\pgfqpoint{3.010650in}{2.952181in}}%
\pgfpathlineto{\pgfqpoint{3.002648in}{2.938981in}}%
\pgfpathlineto{\pgfqpoint{2.994638in}{2.925941in}}%
\pgfpathlineto{\pgfqpoint{2.986621in}{2.913060in}}%
\pgfpathlineto{\pgfqpoint{2.978596in}{2.900332in}}%
\pgfpathclose%
\pgfusepath{fill}%
\end{pgfscope}%
\begin{pgfscope}%
\pgfpathrectangle{\pgfqpoint{1.150000in}{0.150000in}}{\pgfqpoint{5.700000in}{5.700000in}}%
\pgfusepath{clip}%
\pgfsetbuttcap%
\pgfsetroundjoin%
\definecolor{currentfill}{rgb}{0.273006,0.204520,0.501721}%
\pgfsetfillcolor{currentfill}%
\pgfsetfillopacity{0.700000}%
\pgfsetlinewidth{0.000000pt}%
\definecolor{currentstroke}{rgb}{0.000000,0.000000,0.000000}%
\pgfsetstrokecolor{currentstroke}%
\pgfsetdash{}{0pt}%
\pgfpathmoveto{\pgfqpoint{3.308505in}{2.803765in}}%
\pgfpathlineto{\pgfqpoint{3.321811in}{2.794838in}}%
\pgfpathlineto{\pgfqpoint{3.335119in}{2.786021in}}%
\pgfpathlineto{\pgfqpoint{3.348429in}{2.777313in}}%
\pgfpathlineto{\pgfqpoint{3.361741in}{2.768712in}}%
\pgfpathlineto{\pgfqpoint{3.369668in}{2.781521in}}%
\pgfpathlineto{\pgfqpoint{3.377588in}{2.794484in}}%
\pgfpathlineto{\pgfqpoint{3.385502in}{2.807603in}}%
\pgfpathlineto{\pgfqpoint{3.393410in}{2.820885in}}%
\pgfpathlineto{\pgfqpoint{3.380105in}{2.829701in}}%
\pgfpathlineto{\pgfqpoint{3.366802in}{2.838625in}}%
\pgfpathlineto{\pgfqpoint{3.353500in}{2.847658in}}%
\pgfpathlineto{\pgfqpoint{3.340200in}{2.856801in}}%
\pgfpathlineto{\pgfqpoint{3.332286in}{2.843296in}}%
\pgfpathlineto{\pgfqpoint{3.324366in}{2.829958in}}%
\pgfpathlineto{\pgfqpoint{3.316439in}{2.816783in}}%
\pgfpathlineto{\pgfqpoint{3.308505in}{2.803765in}}%
\pgfpathclose%
\pgfusepath{fill}%
\end{pgfscope}%
\begin{pgfscope}%
\pgfpathrectangle{\pgfqpoint{1.150000in}{0.150000in}}{\pgfqpoint{5.700000in}{5.700000in}}%
\pgfusepath{clip}%
\pgfsetbuttcap%
\pgfsetroundjoin%
\definecolor{currentfill}{rgb}{0.274128,0.199721,0.498911}%
\pgfsetfillcolor{currentfill}%
\pgfsetfillopacity{0.700000}%
\pgfsetlinewidth{0.000000pt}%
\definecolor{currentstroke}{rgb}{0.000000,0.000000,0.000000}%
\pgfsetstrokecolor{currentstroke}%
\pgfsetdash{}{0pt}%
\pgfpathmoveto{\pgfqpoint{3.446651in}{2.786685in}}%
\pgfpathlineto{\pgfqpoint{3.459967in}{2.778397in}}%
\pgfpathlineto{\pgfqpoint{3.473285in}{2.770213in}}%
\pgfpathlineto{\pgfqpoint{3.486605in}{2.762132in}}%
\pgfpathlineto{\pgfqpoint{3.499929in}{2.754153in}}%
\pgfpathlineto{\pgfqpoint{3.507817in}{2.767144in}}%
\pgfpathlineto{\pgfqpoint{3.515700in}{2.780294in}}%
\pgfpathlineto{\pgfqpoint{3.523577in}{2.793609in}}%
\pgfpathlineto{\pgfqpoint{3.531448in}{2.807093in}}%
\pgfpathlineto{\pgfqpoint{3.518131in}{2.815308in}}%
\pgfpathlineto{\pgfqpoint{3.504817in}{2.823624in}}%
\pgfpathlineto{\pgfqpoint{3.491505in}{2.832044in}}%
\pgfpathlineto{\pgfqpoint{3.478196in}{2.840567in}}%
\pgfpathlineto{\pgfqpoint{3.470319in}{2.826840in}}%
\pgfpathlineto{\pgfqpoint{3.462435in}{2.813287in}}%
\pgfpathlineto{\pgfqpoint{3.454546in}{2.799903in}}%
\pgfpathlineto{\pgfqpoint{3.446651in}{2.786685in}}%
\pgfpathclose%
\pgfusepath{fill}%
\end{pgfscope}%
\begin{pgfscope}%
\pgfpathrectangle{\pgfqpoint{1.150000in}{0.150000in}}{\pgfqpoint{5.700000in}{5.700000in}}%
\pgfusepath{clip}%
\pgfsetbuttcap%
\pgfsetroundjoin%
\definecolor{currentfill}{rgb}{0.273006,0.204520,0.501721}%
\pgfsetfillcolor{currentfill}%
\pgfsetfillopacity{0.700000}%
\pgfsetlinewidth{0.000000pt}%
\definecolor{currentstroke}{rgb}{0.000000,0.000000,0.000000}%
\pgfsetstrokecolor{currentstroke}%
\pgfsetdash{}{0pt}%
\pgfpathmoveto{\pgfqpoint{3.807489in}{2.795057in}}%
\pgfpathlineto{\pgfqpoint{3.820847in}{2.787953in}}%
\pgfpathlineto{\pgfqpoint{3.834208in}{2.780941in}}%
\pgfpathlineto{\pgfqpoint{3.847574in}{2.774020in}}%
\pgfpathlineto{\pgfqpoint{3.860943in}{2.767190in}}%
\pgfpathlineto{\pgfqpoint{3.868735in}{2.780969in}}%
\pgfpathlineto{\pgfqpoint{3.876522in}{2.794942in}}%
\pgfpathlineto{\pgfqpoint{3.884305in}{2.809114in}}%
\pgfpathlineto{\pgfqpoint{3.892083in}{2.823490in}}%
\pgfpathlineto{\pgfqpoint{3.878720in}{2.830616in}}%
\pgfpathlineto{\pgfqpoint{3.865361in}{2.837832in}}%
\pgfpathlineto{\pgfqpoint{3.852005in}{2.845139in}}%
\pgfpathlineto{\pgfqpoint{3.838654in}{2.852538in}}%
\pgfpathlineto{\pgfqpoint{3.830869in}{2.837859in}}%
\pgfpathlineto{\pgfqpoint{3.823080in}{2.823390in}}%
\pgfpathlineto{\pgfqpoint{3.815287in}{2.809124in}}%
\pgfpathlineto{\pgfqpoint{3.807489in}{2.795057in}}%
\pgfpathclose%
\pgfusepath{fill}%
\end{pgfscope}%
\begin{pgfscope}%
\pgfpathrectangle{\pgfqpoint{1.150000in}{0.150000in}}{\pgfqpoint{5.700000in}{5.700000in}}%
\pgfusepath{clip}%
\pgfsetbuttcap%
\pgfsetroundjoin%
\definecolor{currentfill}{rgb}{0.203063,0.379716,0.553925}%
\pgfsetfillcolor{currentfill}%
\pgfsetfillopacity{0.700000}%
\pgfsetlinewidth{0.000000pt}%
\definecolor{currentstroke}{rgb}{0.000000,0.000000,0.000000}%
\pgfsetstrokecolor{currentstroke}%
\pgfsetdash{}{0pt}%
\pgfpathmoveto{\pgfqpoint{4.706282in}{3.173102in}}%
\pgfpathlineto{\pgfqpoint{4.719775in}{3.165779in}}%
\pgfpathlineto{\pgfqpoint{4.733273in}{3.158532in}}%
\pgfpathlineto{\pgfqpoint{4.746777in}{3.151360in}}%
\pgfpathlineto{\pgfqpoint{4.760285in}{3.144262in}}%
\pgfpathlineto{\pgfqpoint{4.767952in}{3.165544in}}%
\pgfpathlineto{\pgfqpoint{4.775623in}{3.187255in}}%
\pgfpathlineto{\pgfqpoint{4.783298in}{3.209402in}}%
\pgfpathlineto{\pgfqpoint{4.790979in}{3.231997in}}%
\pgfpathlineto{\pgfqpoint{4.777477in}{3.239593in}}%
\pgfpathlineto{\pgfqpoint{4.763980in}{3.247263in}}%
\pgfpathlineto{\pgfqpoint{4.750489in}{3.255009in}}%
\pgfpathlineto{\pgfqpoint{4.737002in}{3.262831in}}%
\pgfpathlineto{\pgfqpoint{4.729316in}{3.239729in}}%
\pgfpathlineto{\pgfqpoint{4.721634in}{3.217081in}}%
\pgfpathlineto{\pgfqpoint{4.713956in}{3.194875in}}%
\pgfpathlineto{\pgfqpoint{4.706282in}{3.173102in}}%
\pgfpathclose%
\pgfusepath{fill}%
\end{pgfscope}%
\begin{pgfscope}%
\pgfpathrectangle{\pgfqpoint{1.150000in}{0.150000in}}{\pgfqpoint{5.700000in}{5.700000in}}%
\pgfusepath{clip}%
\pgfsetbuttcap%
\pgfsetroundjoin%
\definecolor{currentfill}{rgb}{0.269308,0.218818,0.509577}%
\pgfsetfillcolor{currentfill}%
\pgfsetfillopacity{0.700000}%
\pgfsetlinewidth{0.000000pt}%
\definecolor{currentstroke}{rgb}{0.000000,0.000000,0.000000}%
\pgfsetstrokecolor{currentstroke}%
\pgfsetdash{}{0pt}%
\pgfpathmoveto{\pgfqpoint{3.170250in}{2.827095in}}%
\pgfpathlineto{\pgfqpoint{3.183553in}{2.817457in}}%
\pgfpathlineto{\pgfqpoint{3.196857in}{2.807935in}}%
\pgfpathlineto{\pgfqpoint{3.210162in}{2.798529in}}%
\pgfpathlineto{\pgfqpoint{3.223468in}{2.789238in}}%
\pgfpathlineto{\pgfqpoint{3.231435in}{2.801849in}}%
\pgfpathlineto{\pgfqpoint{3.239395in}{2.814608in}}%
\pgfpathlineto{\pgfqpoint{3.247348in}{2.827519in}}%
\pgfpathlineto{\pgfqpoint{3.255295in}{2.840585in}}%
\pgfpathlineto{\pgfqpoint{3.241996in}{2.850072in}}%
\pgfpathlineto{\pgfqpoint{3.228698in}{2.859674in}}%
\pgfpathlineto{\pgfqpoint{3.215401in}{2.869391in}}%
\pgfpathlineto{\pgfqpoint{3.202105in}{2.879225in}}%
\pgfpathlineto{\pgfqpoint{3.194151in}{2.865955in}}%
\pgfpathlineto{\pgfqpoint{3.186191in}{2.852846in}}%
\pgfpathlineto{\pgfqpoint{3.178224in}{2.839894in}}%
\pgfpathlineto{\pgfqpoint{3.170250in}{2.827095in}}%
\pgfpathclose%
\pgfusepath{fill}%
\end{pgfscope}%
\begin{pgfscope}%
\pgfpathrectangle{\pgfqpoint{1.150000in}{0.150000in}}{\pgfqpoint{5.700000in}{5.700000in}}%
\pgfusepath{clip}%
\pgfsetbuttcap%
\pgfsetroundjoin%
\definecolor{currentfill}{rgb}{0.267968,0.223549,0.512008}%
\pgfsetfillcolor{currentfill}%
\pgfsetfillopacity{0.700000}%
\pgfsetlinewidth{0.000000pt}%
\definecolor{currentstroke}{rgb}{0.000000,0.000000,0.000000}%
\pgfsetstrokecolor{currentstroke}%
\pgfsetdash{}{0pt}%
\pgfpathmoveto{\pgfqpoint{4.030150in}{2.826823in}}%
\pgfpathlineto{\pgfqpoint{4.043544in}{2.820175in}}%
\pgfpathlineto{\pgfqpoint{4.056942in}{2.813614in}}%
\pgfpathlineto{\pgfqpoint{4.070344in}{2.807137in}}%
\pgfpathlineto{\pgfqpoint{4.083751in}{2.800746in}}%
\pgfpathlineto{\pgfqpoint{4.091489in}{2.815247in}}%
\pgfpathlineto{\pgfqpoint{4.099223in}{2.829971in}}%
\pgfpathlineto{\pgfqpoint{4.106954in}{2.844926in}}%
\pgfpathlineto{\pgfqpoint{4.114682in}{2.860117in}}%
\pgfpathlineto{\pgfqpoint{4.101282in}{2.866844in}}%
\pgfpathlineto{\pgfqpoint{4.087886in}{2.873655in}}%
\pgfpathlineto{\pgfqpoint{4.074495in}{2.880552in}}%
\pgfpathlineto{\pgfqpoint{4.061108in}{2.887535in}}%
\pgfpathlineto{\pgfqpoint{4.053374in}{2.872001in}}%
\pgfpathlineto{\pgfqpoint{4.045636in}{2.856708in}}%
\pgfpathlineto{\pgfqpoint{4.037895in}{2.841651in}}%
\pgfpathlineto{\pgfqpoint{4.030150in}{2.826823in}}%
\pgfpathclose%
\pgfusepath{fill}%
\end{pgfscope}%
\begin{pgfscope}%
\pgfpathrectangle{\pgfqpoint{1.150000in}{0.150000in}}{\pgfqpoint{5.700000in}{5.700000in}}%
\pgfusepath{clip}%
\pgfsetbuttcap%
\pgfsetroundjoin%
\definecolor{currentfill}{rgb}{0.275191,0.194905,0.496005}%
\pgfsetfillcolor{currentfill}%
\pgfsetfillopacity{0.700000}%
\pgfsetlinewidth{0.000000pt}%
\definecolor{currentstroke}{rgb}{0.000000,0.000000,0.000000}%
\pgfsetstrokecolor{currentstroke}%
\pgfsetdash{}{0pt}%
\pgfpathmoveto{\pgfqpoint{3.584742in}{2.775244in}}%
\pgfpathlineto{\pgfqpoint{3.598072in}{2.767530in}}%
\pgfpathlineto{\pgfqpoint{3.611406in}{2.759915in}}%
\pgfpathlineto{\pgfqpoint{3.624742in}{2.752397in}}%
\pgfpathlineto{\pgfqpoint{3.638082in}{2.744977in}}%
\pgfpathlineto{\pgfqpoint{3.645935in}{2.758138in}}%
\pgfpathlineto{\pgfqpoint{3.653782in}{2.771467in}}%
\pgfpathlineto{\pgfqpoint{3.661623in}{2.784969in}}%
\pgfpathlineto{\pgfqpoint{3.669459in}{2.798649in}}%
\pgfpathlineto{\pgfqpoint{3.656126in}{2.806325in}}%
\pgfpathlineto{\pgfqpoint{3.642796in}{2.814098in}}%
\pgfpathlineto{\pgfqpoint{3.629469in}{2.821969in}}%
\pgfpathlineto{\pgfqpoint{3.616145in}{2.829938in}}%
\pgfpathlineto{\pgfqpoint{3.608302in}{2.815995in}}%
\pgfpathlineto{\pgfqpoint{3.600454in}{2.802235in}}%
\pgfpathlineto{\pgfqpoint{3.592601in}{2.788653in}}%
\pgfpathlineto{\pgfqpoint{3.584742in}{2.775244in}}%
\pgfpathclose%
\pgfusepath{fill}%
\end{pgfscope}%
\begin{pgfscope}%
\pgfpathrectangle{\pgfqpoint{1.150000in}{0.150000in}}{\pgfqpoint{5.700000in}{5.700000in}}%
\pgfusepath{clip}%
\pgfsetbuttcap%
\pgfsetroundjoin%
\definecolor{currentfill}{rgb}{0.160665,0.478540,0.558115}%
\pgfsetfillcolor{currentfill}%
\pgfsetfillopacity{0.700000}%
\pgfsetlinewidth{0.000000pt}%
\definecolor{currentstroke}{rgb}{0.000000,0.000000,0.000000}%
\pgfsetstrokecolor{currentstroke}%
\pgfsetdash{}{0pt}%
\pgfpathmoveto{\pgfqpoint{4.852627in}{3.430031in}}%
\pgfpathlineto{\pgfqpoint{4.866123in}{3.421449in}}%
\pgfpathlineto{\pgfqpoint{4.879623in}{3.412942in}}%
\pgfpathlineto{\pgfqpoint{4.893128in}{3.404509in}}%
\pgfpathlineto{\pgfqpoint{4.906637in}{3.396150in}}%
\pgfpathlineto{\pgfqpoint{4.914370in}{3.422662in}}%
\pgfpathlineto{\pgfqpoint{4.922111in}{3.449717in}}%
\pgfpathlineto{\pgfqpoint{4.929861in}{3.477327in}}%
\pgfpathlineto{\pgfqpoint{4.937621in}{3.505503in}}%
\pgfpathlineto{\pgfqpoint{4.924116in}{3.514426in}}%
\pgfpathlineto{\pgfqpoint{4.910615in}{3.523422in}}%
\pgfpathlineto{\pgfqpoint{4.897119in}{3.532494in}}%
\pgfpathlineto{\pgfqpoint{4.883627in}{3.541641in}}%
\pgfpathlineto{\pgfqpoint{4.875864in}{3.512892in}}%
\pgfpathlineto{\pgfqpoint{4.868110in}{3.484715in}}%
\pgfpathlineto{\pgfqpoint{4.860365in}{3.457099in}}%
\pgfpathlineto{\pgfqpoint{4.852627in}{3.430031in}}%
\pgfpathclose%
\pgfusepath{fill}%
\end{pgfscope}%
\begin{pgfscope}%
\pgfpathrectangle{\pgfqpoint{1.150000in}{0.150000in}}{\pgfqpoint{5.700000in}{5.700000in}}%
\pgfusepath{clip}%
\pgfsetbuttcap%
\pgfsetroundjoin%
\definecolor{currentfill}{rgb}{0.175841,0.441290,0.557685}%
\pgfsetfillcolor{currentfill}%
\pgfsetfillopacity{0.700000}%
\pgfsetlinewidth{0.000000pt}%
\definecolor{currentstroke}{rgb}{0.000000,0.000000,0.000000}%
\pgfsetstrokecolor{currentstroke}%
\pgfsetdash{}{0pt}%
\pgfpathmoveto{\pgfqpoint{4.821752in}{3.327041in}}%
\pgfpathlineto{\pgfqpoint{4.835253in}{3.319000in}}%
\pgfpathlineto{\pgfqpoint{4.848758in}{3.311035in}}%
\pgfpathlineto{\pgfqpoint{4.862268in}{3.303143in}}%
\pgfpathlineto{\pgfqpoint{4.875783in}{3.295326in}}%
\pgfpathlineto{\pgfqpoint{4.883486in}{3.319770in}}%
\pgfpathlineto{\pgfqpoint{4.891196in}{3.344715in}}%
\pgfpathlineto{\pgfqpoint{4.898913in}{3.370171in}}%
\pgfpathlineto{\pgfqpoint{4.906637in}{3.396150in}}%
\pgfpathlineto{\pgfqpoint{4.893128in}{3.404509in}}%
\pgfpathlineto{\pgfqpoint{4.879623in}{3.412942in}}%
\pgfpathlineto{\pgfqpoint{4.866123in}{3.421449in}}%
\pgfpathlineto{\pgfqpoint{4.852627in}{3.430031in}}%
\pgfpathlineto{\pgfqpoint{4.844898in}{3.403503in}}%
\pgfpathlineto{\pgfqpoint{4.837176in}{3.377502in}}%
\pgfpathlineto{\pgfqpoint{4.829461in}{3.352018in}}%
\pgfpathlineto{\pgfqpoint{4.821752in}{3.327041in}}%
\pgfpathclose%
\pgfusepath{fill}%
\end{pgfscope}%
\begin{pgfscope}%
\pgfpathrectangle{\pgfqpoint{1.150000in}{0.150000in}}{\pgfqpoint{5.700000in}{5.700000in}}%
\pgfusepath{clip}%
\pgfsetbuttcap%
\pgfsetroundjoin%
\definecolor{currentfill}{rgb}{0.246811,0.283237,0.535941}%
\pgfsetfillcolor{currentfill}%
\pgfsetfillopacity{0.700000}%
\pgfsetlinewidth{0.000000pt}%
\definecolor{currentstroke}{rgb}{0.000000,0.000000,0.000000}%
\pgfsetstrokecolor{currentstroke}%
\pgfsetdash{}{0pt}%
\pgfpathmoveto{\pgfqpoint{4.421925in}{2.948897in}}%
\pgfpathlineto{\pgfqpoint{4.435386in}{2.942434in}}%
\pgfpathlineto{\pgfqpoint{4.448852in}{2.936050in}}%
\pgfpathlineto{\pgfqpoint{4.462322in}{2.929744in}}%
\pgfpathlineto{\pgfqpoint{4.475798in}{2.923516in}}%
\pgfpathlineto{\pgfqpoint{4.483464in}{2.940354in}}%
\pgfpathlineto{\pgfqpoint{4.491129in}{2.957501in}}%
\pgfpathlineto{\pgfqpoint{4.498794in}{2.974963in}}%
\pgfpathlineto{\pgfqpoint{4.506458in}{2.992750in}}%
\pgfpathlineto{\pgfqpoint{4.492990in}{2.999394in}}%
\pgfpathlineto{\pgfqpoint{4.479527in}{3.006115in}}%
\pgfpathlineto{\pgfqpoint{4.466069in}{3.012915in}}%
\pgfpathlineto{\pgfqpoint{4.452615in}{3.019794in}}%
\pgfpathlineto{\pgfqpoint{4.444943in}{3.001583in}}%
\pgfpathlineto{\pgfqpoint{4.437271in}{2.983702in}}%
\pgfpathlineto{\pgfqpoint{4.429599in}{2.966143in}}%
\pgfpathlineto{\pgfqpoint{4.421925in}{2.948897in}}%
\pgfpathclose%
\pgfusepath{fill}%
\end{pgfscope}%
\begin{pgfscope}%
\pgfpathrectangle{\pgfqpoint{1.150000in}{0.150000in}}{\pgfqpoint{5.700000in}{5.700000in}}%
\pgfusepath{clip}%
\pgfsetbuttcap%
\pgfsetroundjoin%
\definecolor{currentfill}{rgb}{0.239346,0.300855,0.540844}%
\pgfsetfillcolor{currentfill}%
\pgfsetfillopacity{0.700000}%
\pgfsetlinewidth{0.000000pt}%
\definecolor{currentstroke}{rgb}{0.000000,0.000000,0.000000}%
\pgfsetstrokecolor{currentstroke}%
\pgfsetdash{}{0pt}%
\pgfpathmoveto{\pgfqpoint{4.506458in}{2.992750in}}%
\pgfpathlineto{\pgfqpoint{4.519932in}{2.986185in}}%
\pgfpathlineto{\pgfqpoint{4.533411in}{2.979696in}}%
\pgfpathlineto{\pgfqpoint{4.546894in}{2.973286in}}%
\pgfpathlineto{\pgfqpoint{4.560384in}{2.966952in}}%
\pgfpathlineto{\pgfqpoint{4.568041in}{2.984642in}}%
\pgfpathlineto{\pgfqpoint{4.575699in}{3.002667in}}%
\pgfpathlineto{\pgfqpoint{4.583357in}{3.021035in}}%
\pgfpathlineto{\pgfqpoint{4.591017in}{3.039754in}}%
\pgfpathlineto{\pgfqpoint{4.577536in}{3.046524in}}%
\pgfpathlineto{\pgfqpoint{4.564059in}{3.053370in}}%
\pgfpathlineto{\pgfqpoint{4.550588in}{3.060295in}}%
\pgfpathlineto{\pgfqpoint{4.537122in}{3.067296in}}%
\pgfpathlineto{\pgfqpoint{4.529455in}{3.048134in}}%
\pgfpathlineto{\pgfqpoint{4.521789in}{3.029327in}}%
\pgfpathlineto{\pgfqpoint{4.514123in}{3.010869in}}%
\pgfpathlineto{\pgfqpoint{4.506458in}{2.992750in}}%
\pgfpathclose%
\pgfusepath{fill}%
\end{pgfscope}%
\begin{pgfscope}%
\pgfpathrectangle{\pgfqpoint{1.150000in}{0.150000in}}{\pgfqpoint{5.700000in}{5.700000in}}%
\pgfusepath{clip}%
\pgfsetbuttcap%
\pgfsetroundjoin%
\definecolor{currentfill}{rgb}{0.144759,0.519093,0.556572}%
\pgfsetfillcolor{currentfill}%
\pgfsetfillopacity{0.700000}%
\pgfsetlinewidth{0.000000pt}%
\definecolor{currentstroke}{rgb}{0.000000,0.000000,0.000000}%
\pgfsetstrokecolor{currentstroke}%
\pgfsetdash{}{0pt}%
\pgfpathmoveto{\pgfqpoint{4.883627in}{3.541641in}}%
\pgfpathlineto{\pgfqpoint{4.897119in}{3.532494in}}%
\pgfpathlineto{\pgfqpoint{4.910615in}{3.523422in}}%
\pgfpathlineto{\pgfqpoint{4.924116in}{3.514426in}}%
\pgfpathlineto{\pgfqpoint{4.937621in}{3.505503in}}%
\pgfpathlineto{\pgfqpoint{4.945390in}{3.534256in}}%
\pgfpathlineto{\pgfqpoint{4.953169in}{3.563597in}}%
\pgfpathlineto{\pgfqpoint{4.960959in}{3.593538in}}%
\pgfpathlineto{\pgfqpoint{4.947456in}{3.602898in}}%
\pgfpathlineto{\pgfqpoint{4.933958in}{3.612334in}}%
\pgfpathlineto{\pgfqpoint{4.920464in}{3.621844in}}%
\pgfpathlineto{\pgfqpoint{4.906974in}{3.631429in}}%
\pgfpathlineto{\pgfqpoint{4.899182in}{3.600898in}}%
\pgfpathlineto{\pgfqpoint{4.891399in}{3.570972in}}%
\pgfpathlineto{\pgfqpoint{4.883627in}{3.541641in}}%
\pgfpathclose%
\pgfusepath{fill}%
\end{pgfscope}%
\begin{pgfscope}%
\pgfpathrectangle{\pgfqpoint{1.150000in}{0.150000in}}{\pgfqpoint{5.700000in}{5.700000in}}%
\pgfusepath{clip}%
\pgfsetbuttcap%
\pgfsetroundjoin%
\definecolor{currentfill}{rgb}{0.265145,0.232956,0.516599}%
\pgfsetfillcolor{currentfill}%
\pgfsetfillopacity{0.700000}%
\pgfsetlinewidth{0.000000pt}%
\definecolor{currentstroke}{rgb}{0.000000,0.000000,0.000000}%
\pgfsetstrokecolor{currentstroke}%
\pgfsetdash{}{0pt}%
\pgfpathmoveto{\pgfqpoint{3.031824in}{2.857357in}}%
\pgfpathlineto{\pgfqpoint{3.045130in}{2.846929in}}%
\pgfpathlineto{\pgfqpoint{3.058437in}{2.836626in}}%
\pgfpathlineto{\pgfqpoint{3.071743in}{2.826445in}}%
\pgfpathlineto{\pgfqpoint{3.085051in}{2.816387in}}%
\pgfpathlineto{\pgfqpoint{3.093060in}{2.828778in}}%
\pgfpathlineto{\pgfqpoint{3.101062in}{2.841313in}}%
\pgfpathlineto{\pgfqpoint{3.109058in}{2.853996in}}%
\pgfpathlineto{\pgfqpoint{3.117046in}{2.866829in}}%
\pgfpathlineto{\pgfqpoint{3.103746in}{2.877063in}}%
\pgfpathlineto{\pgfqpoint{3.090447in}{2.887419in}}%
\pgfpathlineto{\pgfqpoint{3.077147in}{2.897899in}}%
\pgfpathlineto{\pgfqpoint{3.063848in}{2.908503in}}%
\pgfpathlineto{\pgfqpoint{3.055853in}{2.895486in}}%
\pgfpathlineto{\pgfqpoint{3.047851in}{2.882625in}}%
\pgfpathlineto{\pgfqpoint{3.039841in}{2.869917in}}%
\pgfpathlineto{\pgfqpoint{3.031824in}{2.857357in}}%
\pgfpathclose%
\pgfusepath{fill}%
\end{pgfscope}%
\begin{pgfscope}%
\pgfpathrectangle{\pgfqpoint{1.150000in}{0.150000in}}{\pgfqpoint{5.700000in}{5.700000in}}%
\pgfusepath{clip}%
\pgfsetbuttcap%
\pgfsetroundjoin%
\definecolor{currentfill}{rgb}{0.255645,0.260703,0.528312}%
\pgfsetfillcolor{currentfill}%
\pgfsetfillopacity{0.700000}%
\pgfsetlinewidth{0.000000pt}%
\definecolor{currentstroke}{rgb}{0.000000,0.000000,0.000000}%
\pgfsetstrokecolor{currentstroke}%
\pgfsetdash{}{0pt}%
\pgfpathmoveto{\pgfqpoint{4.337402in}{2.907962in}}%
\pgfpathlineto{\pgfqpoint{4.350850in}{2.901577in}}%
\pgfpathlineto{\pgfqpoint{4.364303in}{2.895272in}}%
\pgfpathlineto{\pgfqpoint{4.377760in}{2.889046in}}%
\pgfpathlineto{\pgfqpoint{4.391223in}{2.882900in}}%
\pgfpathlineto{\pgfqpoint{4.398901in}{2.898965in}}%
\pgfpathlineto{\pgfqpoint{4.406577in}{2.915315in}}%
\pgfpathlineto{\pgfqpoint{4.414252in}{2.931957in}}%
\pgfpathlineto{\pgfqpoint{4.421925in}{2.948897in}}%
\pgfpathlineto{\pgfqpoint{4.408470in}{2.955438in}}%
\pgfpathlineto{\pgfqpoint{4.395020in}{2.962060in}}%
\pgfpathlineto{\pgfqpoint{4.381574in}{2.968761in}}%
\pgfpathlineto{\pgfqpoint{4.368134in}{2.975542in}}%
\pgfpathlineto{\pgfqpoint{4.360453in}{2.958198in}}%
\pgfpathlineto{\pgfqpoint{4.352771in}{2.941159in}}%
\pgfpathlineto{\pgfqpoint{4.345087in}{2.924416in}}%
\pgfpathlineto{\pgfqpoint{4.337402in}{2.907962in}}%
\pgfpathclose%
\pgfusepath{fill}%
\end{pgfscope}%
\begin{pgfscope}%
\pgfpathrectangle{\pgfqpoint{1.150000in}{0.150000in}}{\pgfqpoint{5.700000in}{5.700000in}}%
\pgfusepath{clip}%
\pgfsetbuttcap%
\pgfsetroundjoin%
\definecolor{currentfill}{rgb}{0.229739,0.322361,0.545706}%
\pgfsetfillcolor{currentfill}%
\pgfsetfillopacity{0.700000}%
\pgfsetlinewidth{0.000000pt}%
\definecolor{currentstroke}{rgb}{0.000000,0.000000,0.000000}%
\pgfsetstrokecolor{currentstroke}%
\pgfsetdash{}{0pt}%
\pgfpathmoveto{\pgfqpoint{4.591017in}{3.039754in}}%
\pgfpathlineto{\pgfqpoint{4.604504in}{3.033060in}}%
\pgfpathlineto{\pgfqpoint{4.617996in}{3.026444in}}%
\pgfpathlineto{\pgfqpoint{4.631493in}{3.019903in}}%
\pgfpathlineto{\pgfqpoint{4.644996in}{3.013438in}}%
\pgfpathlineto{\pgfqpoint{4.652649in}{3.032067in}}%
\pgfpathlineto{\pgfqpoint{4.660304in}{3.051059in}}%
\pgfpathlineto{\pgfqpoint{4.667960in}{3.070421in}}%
\pgfpathlineto{\pgfqpoint{4.675619in}{3.090163in}}%
\pgfpathlineto{\pgfqpoint{4.662125in}{3.097085in}}%
\pgfpathlineto{\pgfqpoint{4.648635in}{3.104082in}}%
\pgfpathlineto{\pgfqpoint{4.635150in}{3.111155in}}%
\pgfpathlineto{\pgfqpoint{4.621671in}{3.118305in}}%
\pgfpathlineto{\pgfqpoint{4.614005in}{3.098099in}}%
\pgfpathlineto{\pgfqpoint{4.606340in}{3.078277in}}%
\pgfpathlineto{\pgfqpoint{4.598678in}{3.058832in}}%
\pgfpathlineto{\pgfqpoint{4.591017in}{3.039754in}}%
\pgfpathclose%
\pgfusepath{fill}%
\end{pgfscope}%
\begin{pgfscope}%
\pgfpathrectangle{\pgfqpoint{1.150000in}{0.150000in}}{\pgfqpoint{5.700000in}{5.700000in}}%
\pgfusepath{clip}%
\pgfsetbuttcap%
\pgfsetroundjoin%
\definecolor{currentfill}{rgb}{0.192357,0.403199,0.555836}%
\pgfsetfillcolor{currentfill}%
\pgfsetfillopacity{0.700000}%
\pgfsetlinewidth{0.000000pt}%
\definecolor{currentstroke}{rgb}{0.000000,0.000000,0.000000}%
\pgfsetstrokecolor{currentstroke}%
\pgfsetdash{}{0pt}%
\pgfpathmoveto{\pgfqpoint{4.790979in}{3.231997in}}%
\pgfpathlineto{\pgfqpoint{4.804485in}{3.224476in}}%
\pgfpathlineto{\pgfqpoint{4.817997in}{3.217030in}}%
\pgfpathlineto{\pgfqpoint{4.831513in}{3.209658in}}%
\pgfpathlineto{\pgfqpoint{4.845035in}{3.202360in}}%
\pgfpathlineto{\pgfqpoint{4.852714in}{3.224900in}}%
\pgfpathlineto{\pgfqpoint{4.860398in}{3.247901in}}%
\pgfpathlineto{\pgfqpoint{4.868087in}{3.271373in}}%
\pgfpathlineto{\pgfqpoint{4.875783in}{3.295326in}}%
\pgfpathlineto{\pgfqpoint{4.862268in}{3.303143in}}%
\pgfpathlineto{\pgfqpoint{4.848758in}{3.311035in}}%
\pgfpathlineto{\pgfqpoint{4.835253in}{3.319000in}}%
\pgfpathlineto{\pgfqpoint{4.821752in}{3.327041in}}%
\pgfpathlineto{\pgfqpoint{4.814050in}{3.302560in}}%
\pgfpathlineto{\pgfqpoint{4.806354in}{3.278566in}}%
\pgfpathlineto{\pgfqpoint{4.798664in}{3.255048in}}%
\pgfpathlineto{\pgfqpoint{4.790979in}{3.231997in}}%
\pgfpathclose%
\pgfusepath{fill}%
\end{pgfscope}%
\begin{pgfscope}%
\pgfpathrectangle{\pgfqpoint{1.150000in}{0.150000in}}{\pgfqpoint{5.700000in}{5.700000in}}%
\pgfusepath{clip}%
\pgfsetbuttcap%
\pgfsetroundjoin%
\definecolor{currentfill}{rgb}{0.275191,0.194905,0.496005}%
\pgfsetfillcolor{currentfill}%
\pgfsetfillopacity{0.700000}%
\pgfsetlinewidth{0.000000pt}%
\definecolor{currentstroke}{rgb}{0.000000,0.000000,0.000000}%
\pgfsetstrokecolor{currentstroke}%
\pgfsetdash{}{0pt}%
\pgfpathmoveto{\pgfqpoint{3.722825in}{2.768904in}}%
\pgfpathlineto{\pgfqpoint{3.736175in}{2.761704in}}%
\pgfpathlineto{\pgfqpoint{3.749529in}{2.754599in}}%
\pgfpathlineto{\pgfqpoint{3.762886in}{2.747586in}}%
\pgfpathlineto{\pgfqpoint{3.776247in}{2.740666in}}%
\pgfpathlineto{\pgfqpoint{3.784065in}{2.753992in}}%
\pgfpathlineto{\pgfqpoint{3.791878in}{2.767496in}}%
\pgfpathlineto{\pgfqpoint{3.799685in}{2.781183in}}%
\pgfpathlineto{\pgfqpoint{3.807489in}{2.795057in}}%
\pgfpathlineto{\pgfqpoint{3.794134in}{2.802252in}}%
\pgfpathlineto{\pgfqpoint{3.780784in}{2.809541in}}%
\pgfpathlineto{\pgfqpoint{3.767437in}{2.816922in}}%
\pgfpathlineto{\pgfqpoint{3.754093in}{2.824397in}}%
\pgfpathlineto{\pgfqpoint{3.746284in}{2.810239in}}%
\pgfpathlineto{\pgfqpoint{3.738469in}{2.796275in}}%
\pgfpathlineto{\pgfqpoint{3.730650in}{2.782498in}}%
\pgfpathlineto{\pgfqpoint{3.722825in}{2.768904in}}%
\pgfpathclose%
\pgfusepath{fill}%
\end{pgfscope}%
\begin{pgfscope}%
\pgfpathrectangle{\pgfqpoint{1.150000in}{0.150000in}}{\pgfqpoint{5.700000in}{5.700000in}}%
\pgfusepath{clip}%
\pgfsetbuttcap%
\pgfsetroundjoin%
\definecolor{currentfill}{rgb}{0.252194,0.269783,0.531579}%
\pgfsetfillcolor{currentfill}%
\pgfsetfillopacity{0.700000}%
\pgfsetlinewidth{0.000000pt}%
\definecolor{currentstroke}{rgb}{0.000000,0.000000,0.000000}%
\pgfsetstrokecolor{currentstroke}%
\pgfsetdash{}{0pt}%
\pgfpathmoveto{\pgfqpoint{2.839872in}{2.941898in}}%
\pgfpathlineto{\pgfqpoint{2.853195in}{2.930049in}}%
\pgfpathlineto{\pgfqpoint{2.866516in}{2.918337in}}%
\pgfpathlineto{\pgfqpoint{2.879836in}{2.906762in}}%
\pgfpathlineto{\pgfqpoint{2.893155in}{2.895323in}}%
\pgfpathlineto{\pgfqpoint{2.901218in}{2.907615in}}%
\pgfpathlineto{\pgfqpoint{2.909274in}{2.920053in}}%
\pgfpathlineto{\pgfqpoint{2.917321in}{2.932641in}}%
\pgfpathlineto{\pgfqpoint{2.925362in}{2.945381in}}%
\pgfpathlineto{\pgfqpoint{2.912051in}{2.956976in}}%
\pgfpathlineto{\pgfqpoint{2.898739in}{2.968707in}}%
\pgfpathlineto{\pgfqpoint{2.885426in}{2.980574in}}%
\pgfpathlineto{\pgfqpoint{2.872111in}{2.992579in}}%
\pgfpathlineto{\pgfqpoint{2.864063in}{2.979676in}}%
\pgfpathlineto{\pgfqpoint{2.856008in}{2.966929in}}%
\pgfpathlineto{\pgfqpoint{2.847944in}{2.954338in}}%
\pgfpathlineto{\pgfqpoint{2.839872in}{2.941898in}}%
\pgfpathclose%
\pgfusepath{fill}%
\end{pgfscope}%
\begin{pgfscope}%
\pgfpathrectangle{\pgfqpoint{1.150000in}{0.150000in}}{\pgfqpoint{5.700000in}{5.700000in}}%
\pgfusepath{clip}%
\pgfsetbuttcap%
\pgfsetroundjoin%
\definecolor{currentfill}{rgb}{0.271828,0.209303,0.504434}%
\pgfsetfillcolor{currentfill}%
\pgfsetfillopacity{0.700000}%
\pgfsetlinewidth{0.000000pt}%
\definecolor{currentstroke}{rgb}{0.000000,0.000000,0.000000}%
\pgfsetstrokecolor{currentstroke}%
\pgfsetdash{}{0pt}%
\pgfpathmoveto{\pgfqpoint{3.945574in}{2.795882in}}%
\pgfpathlineto{\pgfqpoint{3.958957in}{2.789202in}}%
\pgfpathlineto{\pgfqpoint{3.972345in}{2.782609in}}%
\pgfpathlineto{\pgfqpoint{3.985737in}{2.776103in}}%
\pgfpathlineto{\pgfqpoint{3.999133in}{2.769685in}}%
\pgfpathlineto{\pgfqpoint{4.006893in}{2.783655in}}%
\pgfpathlineto{\pgfqpoint{4.014649in}{2.797831in}}%
\pgfpathlineto{\pgfqpoint{4.022402in}{2.812218in}}%
\pgfpathlineto{\pgfqpoint{4.030150in}{2.826823in}}%
\pgfpathlineto{\pgfqpoint{4.016761in}{2.833557in}}%
\pgfpathlineto{\pgfqpoint{4.003376in}{2.840378in}}%
\pgfpathlineto{\pgfqpoint{3.989995in}{2.847286in}}%
\pgfpathlineto{\pgfqpoint{3.976619in}{2.854282in}}%
\pgfpathlineto{\pgfqpoint{3.968864in}{2.839354in}}%
\pgfpathlineto{\pgfqpoint{3.961105in}{2.824649in}}%
\pgfpathlineto{\pgfqpoint{3.953341in}{2.810160in}}%
\pgfpathlineto{\pgfqpoint{3.945574in}{2.795882in}}%
\pgfpathclose%
\pgfusepath{fill}%
\end{pgfscope}%
\begin{pgfscope}%
\pgfpathrectangle{\pgfqpoint{1.150000in}{0.150000in}}{\pgfqpoint{5.700000in}{5.700000in}}%
\pgfusepath{clip}%
\pgfsetbuttcap%
\pgfsetroundjoin%
\definecolor{currentfill}{rgb}{0.260571,0.246922,0.522828}%
\pgfsetfillcolor{currentfill}%
\pgfsetfillopacity{0.700000}%
\pgfsetlinewidth{0.000000pt}%
\definecolor{currentstroke}{rgb}{0.000000,0.000000,0.000000}%
\pgfsetstrokecolor{currentstroke}%
\pgfsetdash{}{0pt}%
\pgfpathmoveto{\pgfqpoint{4.252874in}{2.869742in}}%
\pgfpathlineto{\pgfqpoint{4.266309in}{2.863409in}}%
\pgfpathlineto{\pgfqpoint{4.279750in}{2.857157in}}%
\pgfpathlineto{\pgfqpoint{4.293195in}{2.850986in}}%
\pgfpathlineto{\pgfqpoint{4.306645in}{2.844896in}}%
\pgfpathlineto{\pgfqpoint{4.314337in}{2.860264in}}%
\pgfpathlineto{\pgfqpoint{4.322027in}{2.875893in}}%
\pgfpathlineto{\pgfqpoint{4.329716in}{2.891790in}}%
\pgfpathlineto{\pgfqpoint{4.337402in}{2.907962in}}%
\pgfpathlineto{\pgfqpoint{4.323960in}{2.914428in}}%
\pgfpathlineto{\pgfqpoint{4.310522in}{2.920974in}}%
\pgfpathlineto{\pgfqpoint{4.297089in}{2.927601in}}%
\pgfpathlineto{\pgfqpoint{4.283661in}{2.934310in}}%
\pgfpathlineto{\pgfqpoint{4.275967in}{2.917755in}}%
\pgfpathlineto{\pgfqpoint{4.268272in}{2.901480in}}%
\pgfpathlineto{\pgfqpoint{4.260574in}{2.885478in}}%
\pgfpathlineto{\pgfqpoint{4.252874in}{2.869742in}}%
\pgfpathclose%
\pgfusepath{fill}%
\end{pgfscope}%
\begin{pgfscope}%
\pgfpathrectangle{\pgfqpoint{1.150000in}{0.150000in}}{\pgfqpoint{5.700000in}{5.700000in}}%
\pgfusepath{clip}%
\pgfsetbuttcap%
\pgfsetroundjoin%
\definecolor{currentfill}{rgb}{0.218130,0.347432,0.550038}%
\pgfsetfillcolor{currentfill}%
\pgfsetfillopacity{0.700000}%
\pgfsetlinewidth{0.000000pt}%
\definecolor{currentstroke}{rgb}{0.000000,0.000000,0.000000}%
\pgfsetstrokecolor{currentstroke}%
\pgfsetdash{}{0pt}%
\pgfpathmoveto{\pgfqpoint{4.675619in}{3.090163in}}%
\pgfpathlineto{\pgfqpoint{4.689119in}{3.083318in}}%
\pgfpathlineto{\pgfqpoint{4.702625in}{3.076547in}}%
\pgfpathlineto{\pgfqpoint{4.716136in}{3.069852in}}%
\pgfpathlineto{\pgfqpoint{4.729652in}{3.063232in}}%
\pgfpathlineto{\pgfqpoint{4.737305in}{3.082893in}}%
\pgfpathlineto{\pgfqpoint{4.744962in}{3.102946in}}%
\pgfpathlineto{\pgfqpoint{4.752622in}{3.123399in}}%
\pgfpathlineto{\pgfqpoint{4.760285in}{3.144262in}}%
\pgfpathlineto{\pgfqpoint{4.746777in}{3.151360in}}%
\pgfpathlineto{\pgfqpoint{4.733273in}{3.158532in}}%
\pgfpathlineto{\pgfqpoint{4.719775in}{3.165779in}}%
\pgfpathlineto{\pgfqpoint{4.706282in}{3.173102in}}%
\pgfpathlineto{\pgfqpoint{4.698612in}{3.151754in}}%
\pgfpathlineto{\pgfqpoint{4.690945in}{3.130821in}}%
\pgfpathlineto{\pgfqpoint{4.683281in}{3.110294in}}%
\pgfpathlineto{\pgfqpoint{4.675619in}{3.090163in}}%
\pgfpathclose%
\pgfusepath{fill}%
\end{pgfscope}%
\begin{pgfscope}%
\pgfpathrectangle{\pgfqpoint{1.150000in}{0.150000in}}{\pgfqpoint{5.700000in}{5.700000in}}%
\pgfusepath{clip}%
\pgfsetbuttcap%
\pgfsetroundjoin%
\definecolor{currentfill}{rgb}{0.275191,0.194905,0.496005}%
\pgfsetfillcolor{currentfill}%
\pgfsetfillopacity{0.700000}%
\pgfsetlinewidth{0.000000pt}%
\definecolor{currentstroke}{rgb}{0.000000,0.000000,0.000000}%
\pgfsetstrokecolor{currentstroke}%
\pgfsetdash{}{0pt}%
\pgfpathmoveto{\pgfqpoint{3.361741in}{2.768712in}}%
\pgfpathlineto{\pgfqpoint{3.375055in}{2.760219in}}%
\pgfpathlineto{\pgfqpoint{3.388370in}{2.751832in}}%
\pgfpathlineto{\pgfqpoint{3.401688in}{2.743550in}}%
\pgfpathlineto{\pgfqpoint{3.415009in}{2.735374in}}%
\pgfpathlineto{\pgfqpoint{3.422928in}{2.747975in}}%
\pgfpathlineto{\pgfqpoint{3.430842in}{2.760725in}}%
\pgfpathlineto{\pgfqpoint{3.438749in}{2.773626in}}%
\pgfpathlineto{\pgfqpoint{3.446651in}{2.786685in}}%
\pgfpathlineto{\pgfqpoint{3.433337in}{2.795076in}}%
\pgfpathlineto{\pgfqpoint{3.420026in}{2.803573in}}%
\pgfpathlineto{\pgfqpoint{3.406717in}{2.812176in}}%
\pgfpathlineto{\pgfqpoint{3.393410in}{2.820885in}}%
\pgfpathlineto{\pgfqpoint{3.385502in}{2.807603in}}%
\pgfpathlineto{\pgfqpoint{3.377588in}{2.794484in}}%
\pgfpathlineto{\pgfqpoint{3.369668in}{2.781521in}}%
\pgfpathlineto{\pgfqpoint{3.361741in}{2.768712in}}%
\pgfpathclose%
\pgfusepath{fill}%
\end{pgfscope}%
\begin{pgfscope}%
\pgfpathrectangle{\pgfqpoint{1.150000in}{0.150000in}}{\pgfqpoint{5.700000in}{5.700000in}}%
\pgfusepath{clip}%
\pgfsetbuttcap%
\pgfsetroundjoin%
\definecolor{currentfill}{rgb}{0.273006,0.204520,0.501721}%
\pgfsetfillcolor{currentfill}%
\pgfsetfillopacity{0.700000}%
\pgfsetlinewidth{0.000000pt}%
\definecolor{currentstroke}{rgb}{0.000000,0.000000,0.000000}%
\pgfsetstrokecolor{currentstroke}%
\pgfsetdash{}{0pt}%
\pgfpathmoveto{\pgfqpoint{3.223468in}{2.789238in}}%
\pgfpathlineto{\pgfqpoint{3.236775in}{2.780060in}}%
\pgfpathlineto{\pgfqpoint{3.250084in}{2.770995in}}%
\pgfpathlineto{\pgfqpoint{3.263394in}{2.762042in}}%
\pgfpathlineto{\pgfqpoint{3.276706in}{2.753201in}}%
\pgfpathlineto{\pgfqpoint{3.284665in}{2.765624in}}%
\pgfpathlineto{\pgfqpoint{3.292618in}{2.778190in}}%
\pgfpathlineto{\pgfqpoint{3.300565in}{2.790902in}}%
\pgfpathlineto{\pgfqpoint{3.308505in}{2.803765in}}%
\pgfpathlineto{\pgfqpoint{3.295200in}{2.812803in}}%
\pgfpathlineto{\pgfqpoint{3.281897in}{2.821951in}}%
\pgfpathlineto{\pgfqpoint{3.268595in}{2.831212in}}%
\pgfpathlineto{\pgfqpoint{3.255295in}{2.840585in}}%
\pgfpathlineto{\pgfqpoint{3.247348in}{2.827519in}}%
\pgfpathlineto{\pgfqpoint{3.239395in}{2.814608in}}%
\pgfpathlineto{\pgfqpoint{3.231435in}{2.801849in}}%
\pgfpathlineto{\pgfqpoint{3.223468in}{2.789238in}}%
\pgfpathclose%
\pgfusepath{fill}%
\end{pgfscope}%
\begin{pgfscope}%
\pgfpathrectangle{\pgfqpoint{1.150000in}{0.150000in}}{\pgfqpoint{5.700000in}{5.700000in}}%
\pgfusepath{clip}%
\pgfsetbuttcap%
\pgfsetroundjoin%
\definecolor{currentfill}{rgb}{0.277134,0.185228,0.489898}%
\pgfsetfillcolor{currentfill}%
\pgfsetfillopacity{0.700000}%
\pgfsetlinewidth{0.000000pt}%
\definecolor{currentstroke}{rgb}{0.000000,0.000000,0.000000}%
\pgfsetstrokecolor{currentstroke}%
\pgfsetdash{}{0pt}%
\pgfpathmoveto{\pgfqpoint{3.499929in}{2.754153in}}%
\pgfpathlineto{\pgfqpoint{3.513254in}{2.746275in}}%
\pgfpathlineto{\pgfqpoint{3.526583in}{2.738499in}}%
\pgfpathlineto{\pgfqpoint{3.539915in}{2.730822in}}%
\pgfpathlineto{\pgfqpoint{3.553249in}{2.723245in}}%
\pgfpathlineto{\pgfqpoint{3.561131in}{2.736008in}}%
\pgfpathlineto{\pgfqpoint{3.569007in}{2.748926in}}%
\pgfpathlineto{\pgfqpoint{3.576877in}{2.762003in}}%
\pgfpathlineto{\pgfqpoint{3.584742in}{2.775244in}}%
\pgfpathlineto{\pgfqpoint{3.571414in}{2.783056in}}%
\pgfpathlineto{\pgfqpoint{3.558089in}{2.790968in}}%
\pgfpathlineto{\pgfqpoint{3.544767in}{2.798980in}}%
\pgfpathlineto{\pgfqpoint{3.531448in}{2.807093in}}%
\pgfpathlineto{\pgfqpoint{3.523577in}{2.793609in}}%
\pgfpathlineto{\pgfqpoint{3.515700in}{2.780294in}}%
\pgfpathlineto{\pgfqpoint{3.507817in}{2.767144in}}%
\pgfpathlineto{\pgfqpoint{3.499929in}{2.754153in}}%
\pgfpathclose%
\pgfusepath{fill}%
\end{pgfscope}%
\begin{pgfscope}%
\pgfpathrectangle{\pgfqpoint{1.150000in}{0.150000in}}{\pgfqpoint{5.700000in}{5.700000in}}%
\pgfusepath{clip}%
\pgfsetbuttcap%
\pgfsetroundjoin%
\definecolor{currentfill}{rgb}{0.266580,0.228262,0.514349}%
\pgfsetfillcolor{currentfill}%
\pgfsetfillopacity{0.700000}%
\pgfsetlinewidth{0.000000pt}%
\definecolor{currentstroke}{rgb}{0.000000,0.000000,0.000000}%
\pgfsetstrokecolor{currentstroke}%
\pgfsetdash{}{0pt}%
\pgfpathmoveto{\pgfqpoint{4.168328in}{2.834055in}}%
\pgfpathlineto{\pgfqpoint{4.181751in}{2.827748in}}%
\pgfpathlineto{\pgfqpoint{4.195179in}{2.821524in}}%
\pgfpathlineto{\pgfqpoint{4.208612in}{2.815383in}}%
\pgfpathlineto{\pgfqpoint{4.222050in}{2.809323in}}%
\pgfpathlineto{\pgfqpoint{4.229760in}{2.824062in}}%
\pgfpathlineto{\pgfqpoint{4.237467in}{2.839041in}}%
\pgfpathlineto{\pgfqpoint{4.245172in}{2.854265in}}%
\pgfpathlineto{\pgfqpoint{4.252874in}{2.869742in}}%
\pgfpathlineto{\pgfqpoint{4.239444in}{2.876157in}}%
\pgfpathlineto{\pgfqpoint{4.226018in}{2.882654in}}%
\pgfpathlineto{\pgfqpoint{4.212597in}{2.889233in}}%
\pgfpathlineto{\pgfqpoint{4.199181in}{2.895895in}}%
\pgfpathlineto{\pgfqpoint{4.191472in}{2.880055in}}%
\pgfpathlineto{\pgfqpoint{4.183760in}{2.864473in}}%
\pgfpathlineto{\pgfqpoint{4.176045in}{2.849142in}}%
\pgfpathlineto{\pgfqpoint{4.168328in}{2.834055in}}%
\pgfpathclose%
\pgfusepath{fill}%
\end{pgfscope}%
\begin{pgfscope}%
\pgfpathrectangle{\pgfqpoint{1.150000in}{0.150000in}}{\pgfqpoint{5.700000in}{5.700000in}}%
\pgfusepath{clip}%
\pgfsetbuttcap%
\pgfsetroundjoin%
\definecolor{currentfill}{rgb}{0.275191,0.194905,0.496005}%
\pgfsetfillcolor{currentfill}%
\pgfsetfillopacity{0.700000}%
\pgfsetlinewidth{0.000000pt}%
\definecolor{currentstroke}{rgb}{0.000000,0.000000,0.000000}%
\pgfsetstrokecolor{currentstroke}%
\pgfsetdash{}{0pt}%
\pgfpathmoveto{\pgfqpoint{3.860943in}{2.767190in}}%
\pgfpathlineto{\pgfqpoint{3.874317in}{2.760449in}}%
\pgfpathlineto{\pgfqpoint{3.887694in}{2.753799in}}%
\pgfpathlineto{\pgfqpoint{3.901076in}{2.747237in}}%
\pgfpathlineto{\pgfqpoint{3.914462in}{2.740764in}}%
\pgfpathlineto{\pgfqpoint{3.922247in}{2.754255in}}%
\pgfpathlineto{\pgfqpoint{3.930027in}{2.767935in}}%
\pgfpathlineto{\pgfqpoint{3.937803in}{2.781809in}}%
\pgfpathlineto{\pgfqpoint{3.945574in}{2.795882in}}%
\pgfpathlineto{\pgfqpoint{3.932195in}{2.802651in}}%
\pgfpathlineto{\pgfqpoint{3.918820in}{2.809508in}}%
\pgfpathlineto{\pgfqpoint{3.905450in}{2.816454in}}%
\pgfpathlineto{\pgfqpoint{3.892083in}{2.823490in}}%
\pgfpathlineto{\pgfqpoint{3.884305in}{2.809114in}}%
\pgfpathlineto{\pgfqpoint{3.876522in}{2.794942in}}%
\pgfpathlineto{\pgfqpoint{3.868735in}{2.780969in}}%
\pgfpathlineto{\pgfqpoint{3.860943in}{2.767190in}}%
\pgfpathclose%
\pgfusepath{fill}%
\end{pgfscope}%
\begin{pgfscope}%
\pgfpathrectangle{\pgfqpoint{1.150000in}{0.150000in}}{\pgfqpoint{5.700000in}{5.700000in}}%
\pgfusepath{clip}%
\pgfsetbuttcap%
\pgfsetroundjoin%
\definecolor{currentfill}{rgb}{0.258965,0.251537,0.524736}%
\pgfsetfillcolor{currentfill}%
\pgfsetfillopacity{0.700000}%
\pgfsetlinewidth{0.000000pt}%
\definecolor{currentstroke}{rgb}{0.000000,0.000000,0.000000}%
\pgfsetstrokecolor{currentstroke}%
\pgfsetdash{}{0pt}%
\pgfpathmoveto{\pgfqpoint{2.893155in}{2.895323in}}%
\pgfpathlineto{\pgfqpoint{2.906473in}{2.884018in}}%
\pgfpathlineto{\pgfqpoint{2.919790in}{2.872846in}}%
\pgfpathlineto{\pgfqpoint{2.933106in}{2.861806in}}%
\pgfpathlineto{\pgfqpoint{2.946422in}{2.850897in}}%
\pgfpathlineto{\pgfqpoint{2.954477in}{2.863041in}}%
\pgfpathlineto{\pgfqpoint{2.962524in}{2.875326in}}%
\pgfpathlineto{\pgfqpoint{2.970564in}{2.887755in}}%
\pgfpathlineto{\pgfqpoint{2.978596in}{2.900332in}}%
\pgfpathlineto{\pgfqpoint{2.965289in}{2.911397in}}%
\pgfpathlineto{\pgfqpoint{2.951980in}{2.922593in}}%
\pgfpathlineto{\pgfqpoint{2.938671in}{2.933920in}}%
\pgfpathlineto{\pgfqpoint{2.925362in}{2.945381in}}%
\pgfpathlineto{\pgfqpoint{2.917321in}{2.932641in}}%
\pgfpathlineto{\pgfqpoint{2.909274in}{2.920053in}}%
\pgfpathlineto{\pgfqpoint{2.901218in}{2.907615in}}%
\pgfpathlineto{\pgfqpoint{2.893155in}{2.895323in}}%
\pgfpathclose%
\pgfusepath{fill}%
\end{pgfscope}%
\begin{pgfscope}%
\pgfpathrectangle{\pgfqpoint{1.150000in}{0.150000in}}{\pgfqpoint{5.700000in}{5.700000in}}%
\pgfusepath{clip}%
\pgfsetbuttcap%
\pgfsetroundjoin%
\definecolor{currentfill}{rgb}{0.206756,0.371758,0.553117}%
\pgfsetfillcolor{currentfill}%
\pgfsetfillopacity{0.700000}%
\pgfsetlinewidth{0.000000pt}%
\definecolor{currentstroke}{rgb}{0.000000,0.000000,0.000000}%
\pgfsetstrokecolor{currentstroke}%
\pgfsetdash{}{0pt}%
\pgfpathmoveto{\pgfqpoint{4.760285in}{3.144262in}}%
\pgfpathlineto{\pgfqpoint{4.773798in}{3.137240in}}%
\pgfpathlineto{\pgfqpoint{4.787317in}{3.130291in}}%
\pgfpathlineto{\pgfqpoint{4.800842in}{3.123417in}}%
\pgfpathlineto{\pgfqpoint{4.814371in}{3.116617in}}%
\pgfpathlineto{\pgfqpoint{4.822031in}{3.137409in}}%
\pgfpathlineto{\pgfqpoint{4.829694in}{3.158624in}}%
\pgfpathlineto{\pgfqpoint{4.837362in}{3.180271in}}%
\pgfpathlineto{\pgfqpoint{4.845035in}{3.202360in}}%
\pgfpathlineto{\pgfqpoint{4.831513in}{3.209658in}}%
\pgfpathlineto{\pgfqpoint{4.817997in}{3.217030in}}%
\pgfpathlineto{\pgfqpoint{4.804485in}{3.224476in}}%
\pgfpathlineto{\pgfqpoint{4.790979in}{3.231997in}}%
\pgfpathlineto{\pgfqpoint{4.783298in}{3.209402in}}%
\pgfpathlineto{\pgfqpoint{4.775623in}{3.187255in}}%
\pgfpathlineto{\pgfqpoint{4.767952in}{3.165544in}}%
\pgfpathlineto{\pgfqpoint{4.760285in}{3.144262in}}%
\pgfpathclose%
\pgfusepath{fill}%
\end{pgfscope}%
\begin{pgfscope}%
\pgfpathrectangle{\pgfqpoint{1.150000in}{0.150000in}}{\pgfqpoint{5.700000in}{5.700000in}}%
\pgfusepath{clip}%
\pgfsetbuttcap%
\pgfsetroundjoin%
\definecolor{currentfill}{rgb}{0.163625,0.471133,0.558148}%
\pgfsetfillcolor{currentfill}%
\pgfsetfillopacity{0.700000}%
\pgfsetlinewidth{0.000000pt}%
\definecolor{currentstroke}{rgb}{0.000000,0.000000,0.000000}%
\pgfsetstrokecolor{currentstroke}%
\pgfsetdash{}{0pt}%
\pgfpathmoveto{\pgfqpoint{4.906637in}{3.396150in}}%
\pgfpathlineto{\pgfqpoint{4.920152in}{3.387866in}}%
\pgfpathlineto{\pgfqpoint{4.933671in}{3.379655in}}%
\pgfpathlineto{\pgfqpoint{4.947195in}{3.371518in}}%
\pgfpathlineto{\pgfqpoint{4.960724in}{3.363454in}}%
\pgfpathlineto{\pgfqpoint{4.968451in}{3.389411in}}%
\pgfpathlineto{\pgfqpoint{4.976187in}{3.415906in}}%
\pgfpathlineto{\pgfqpoint{4.983932in}{3.442950in}}%
\pgfpathlineto{\pgfqpoint{4.991687in}{3.470553in}}%
\pgfpathlineto{\pgfqpoint{4.978163in}{3.479180in}}%
\pgfpathlineto{\pgfqpoint{4.964645in}{3.487881in}}%
\pgfpathlineto{\pgfqpoint{4.951130in}{3.496655in}}%
\pgfpathlineto{\pgfqpoint{4.937621in}{3.505503in}}%
\pgfpathlineto{\pgfqpoint{4.929861in}{3.477327in}}%
\pgfpathlineto{\pgfqpoint{4.922111in}{3.449717in}}%
\pgfpathlineto{\pgfqpoint{4.914370in}{3.422662in}}%
\pgfpathlineto{\pgfqpoint{4.906637in}{3.396150in}}%
\pgfpathclose%
\pgfusepath{fill}%
\end{pgfscope}%
\begin{pgfscope}%
\pgfpathrectangle{\pgfqpoint{1.150000in}{0.150000in}}{\pgfqpoint{5.700000in}{5.700000in}}%
\pgfusepath{clip}%
\pgfsetbuttcap%
\pgfsetroundjoin%
\definecolor{currentfill}{rgb}{0.270595,0.214069,0.507052}%
\pgfsetfillcolor{currentfill}%
\pgfsetfillopacity{0.700000}%
\pgfsetlinewidth{0.000000pt}%
\definecolor{currentstroke}{rgb}{0.000000,0.000000,0.000000}%
\pgfsetstrokecolor{currentstroke}%
\pgfsetdash{}{0pt}%
\pgfpathmoveto{\pgfqpoint{3.085051in}{2.816387in}}%
\pgfpathlineto{\pgfqpoint{3.098358in}{2.806450in}}%
\pgfpathlineto{\pgfqpoint{3.111666in}{2.796633in}}%
\pgfpathlineto{\pgfqpoint{3.124975in}{2.786935in}}%
\pgfpathlineto{\pgfqpoint{3.138285in}{2.777355in}}%
\pgfpathlineto{\pgfqpoint{3.146286in}{2.789578in}}%
\pgfpathlineto{\pgfqpoint{3.154281in}{2.801940in}}%
\pgfpathlineto{\pgfqpoint{3.162269in}{2.814445in}}%
\pgfpathlineto{\pgfqpoint{3.170250in}{2.827095in}}%
\pgfpathlineto{\pgfqpoint{3.156948in}{2.836850in}}%
\pgfpathlineto{\pgfqpoint{3.143647in}{2.846723in}}%
\pgfpathlineto{\pgfqpoint{3.130346in}{2.856716in}}%
\pgfpathlineto{\pgfqpoint{3.117046in}{2.866829in}}%
\pgfpathlineto{\pgfqpoint{3.109058in}{2.853996in}}%
\pgfpathlineto{\pgfqpoint{3.101062in}{2.841313in}}%
\pgfpathlineto{\pgfqpoint{3.093060in}{2.828778in}}%
\pgfpathlineto{\pgfqpoint{3.085051in}{2.816387in}}%
\pgfpathclose%
\pgfusepath{fill}%
\end{pgfscope}%
\begin{pgfscope}%
\pgfpathrectangle{\pgfqpoint{1.150000in}{0.150000in}}{\pgfqpoint{5.700000in}{5.700000in}}%
\pgfusepath{clip}%
\pgfsetbuttcap%
\pgfsetroundjoin%
\definecolor{currentfill}{rgb}{0.180629,0.429975,0.557282}%
\pgfsetfillcolor{currentfill}%
\pgfsetfillopacity{0.700000}%
\pgfsetlinewidth{0.000000pt}%
\definecolor{currentstroke}{rgb}{0.000000,0.000000,0.000000}%
\pgfsetstrokecolor{currentstroke}%
\pgfsetdash{}{0pt}%
\pgfpathmoveto{\pgfqpoint{4.875783in}{3.295326in}}%
\pgfpathlineto{\pgfqpoint{4.889304in}{3.287582in}}%
\pgfpathlineto{\pgfqpoint{4.902829in}{3.279913in}}%
\pgfpathlineto{\pgfqpoint{4.916359in}{3.272316in}}%
\pgfpathlineto{\pgfqpoint{4.929895in}{3.264793in}}%
\pgfpathlineto{\pgfqpoint{4.937591in}{3.288704in}}%
\pgfpathlineto{\pgfqpoint{4.945294in}{3.313111in}}%
\pgfpathlineto{\pgfqpoint{4.953005in}{3.338024in}}%
\pgfpathlineto{\pgfqpoint{4.960724in}{3.363454in}}%
\pgfpathlineto{\pgfqpoint{4.947195in}{3.371518in}}%
\pgfpathlineto{\pgfqpoint{4.933671in}{3.379655in}}%
\pgfpathlineto{\pgfqpoint{4.920152in}{3.387866in}}%
\pgfpathlineto{\pgfqpoint{4.906637in}{3.396150in}}%
\pgfpathlineto{\pgfqpoint{4.898913in}{3.370171in}}%
\pgfpathlineto{\pgfqpoint{4.891196in}{3.344715in}}%
\pgfpathlineto{\pgfqpoint{4.883486in}{3.319770in}}%
\pgfpathlineto{\pgfqpoint{4.875783in}{3.295326in}}%
\pgfpathclose%
\pgfusepath{fill}%
\end{pgfscope}%
\begin{pgfscope}%
\pgfpathrectangle{\pgfqpoint{1.150000in}{0.150000in}}{\pgfqpoint{5.700000in}{5.700000in}}%
\pgfusepath{clip}%
\pgfsetbuttcap%
\pgfsetroundjoin%
\definecolor{currentfill}{rgb}{0.277134,0.185228,0.489898}%
\pgfsetfillcolor{currentfill}%
\pgfsetfillopacity{0.700000}%
\pgfsetlinewidth{0.000000pt}%
\definecolor{currentstroke}{rgb}{0.000000,0.000000,0.000000}%
\pgfsetstrokecolor{currentstroke}%
\pgfsetdash{}{0pt}%
\pgfpathmoveto{\pgfqpoint{3.638082in}{2.744977in}}%
\pgfpathlineto{\pgfqpoint{3.651425in}{2.737652in}}%
\pgfpathlineto{\pgfqpoint{3.664772in}{2.730424in}}%
\pgfpathlineto{\pgfqpoint{3.678122in}{2.723291in}}%
\pgfpathlineto{\pgfqpoint{3.691475in}{2.716253in}}%
\pgfpathlineto{\pgfqpoint{3.699320in}{2.729166in}}%
\pgfpathlineto{\pgfqpoint{3.707161in}{2.742243in}}%
\pgfpathlineto{\pgfqpoint{3.714996in}{2.755487in}}%
\pgfpathlineto{\pgfqpoint{3.722825in}{2.768904in}}%
\pgfpathlineto{\pgfqpoint{3.709479in}{2.776197in}}%
\pgfpathlineto{\pgfqpoint{3.696136in}{2.783586in}}%
\pgfpathlineto{\pgfqpoint{3.682796in}{2.791070in}}%
\pgfpathlineto{\pgfqpoint{3.669459in}{2.798649in}}%
\pgfpathlineto{\pgfqpoint{3.661623in}{2.784969in}}%
\pgfpathlineto{\pgfqpoint{3.653782in}{2.771467in}}%
\pgfpathlineto{\pgfqpoint{3.645935in}{2.758138in}}%
\pgfpathlineto{\pgfqpoint{3.638082in}{2.744977in}}%
\pgfpathclose%
\pgfusepath{fill}%
\end{pgfscope}%
\begin{pgfscope}%
\pgfpathrectangle{\pgfqpoint{1.150000in}{0.150000in}}{\pgfqpoint{5.700000in}{5.700000in}}%
\pgfusepath{clip}%
\pgfsetbuttcap%
\pgfsetroundjoin%
\definecolor{currentfill}{rgb}{0.149039,0.508051,0.557250}%
\pgfsetfillcolor{currentfill}%
\pgfsetfillopacity{0.700000}%
\pgfsetlinewidth{0.000000pt}%
\definecolor{currentstroke}{rgb}{0.000000,0.000000,0.000000}%
\pgfsetstrokecolor{currentstroke}%
\pgfsetdash{}{0pt}%
\pgfpathmoveto{\pgfqpoint{4.937621in}{3.505503in}}%
\pgfpathlineto{\pgfqpoint{4.951130in}{3.496655in}}%
\pgfpathlineto{\pgfqpoint{4.964645in}{3.487881in}}%
\pgfpathlineto{\pgfqpoint{4.978163in}{3.479180in}}%
\pgfpathlineto{\pgfqpoint{4.991687in}{3.470553in}}%
\pgfpathlineto{\pgfqpoint{4.999452in}{3.498728in}}%
\pgfpathlineto{\pgfqpoint{5.007227in}{3.527486in}}%
\pgfpathlineto{\pgfqpoint{5.015013in}{3.556838in}}%
\pgfpathlineto{\pgfqpoint{5.001493in}{3.565902in}}%
\pgfpathlineto{\pgfqpoint{4.987977in}{3.575040in}}%
\pgfpathlineto{\pgfqpoint{4.974466in}{3.584252in}}%
\pgfpathlineto{\pgfqpoint{4.960959in}{3.593538in}}%
\pgfpathlineto{\pgfqpoint{4.953169in}{3.563597in}}%
\pgfpathlineto{\pgfqpoint{4.945390in}{3.534256in}}%
\pgfpathlineto{\pgfqpoint{4.937621in}{3.505503in}}%
\pgfpathclose%
\pgfusepath{fill}%
\end{pgfscope}%
\begin{pgfscope}%
\pgfpathrectangle{\pgfqpoint{1.150000in}{0.150000in}}{\pgfqpoint{5.700000in}{5.700000in}}%
\pgfusepath{clip}%
\pgfsetbuttcap%
\pgfsetroundjoin%
\definecolor{currentfill}{rgb}{0.270595,0.214069,0.507052}%
\pgfsetfillcolor{currentfill}%
\pgfsetfillopacity{0.700000}%
\pgfsetlinewidth{0.000000pt}%
\definecolor{currentstroke}{rgb}{0.000000,0.000000,0.000000}%
\pgfsetstrokecolor{currentstroke}%
\pgfsetdash{}{0pt}%
\pgfpathmoveto{\pgfqpoint{4.083751in}{2.800746in}}%
\pgfpathlineto{\pgfqpoint{4.097163in}{2.794440in}}%
\pgfpathlineto{\pgfqpoint{4.110579in}{2.788218in}}%
\pgfpathlineto{\pgfqpoint{4.124000in}{2.782080in}}%
\pgfpathlineto{\pgfqpoint{4.137426in}{2.776025in}}%
\pgfpathlineto{\pgfqpoint{4.145156in}{2.790198in}}%
\pgfpathlineto{\pgfqpoint{4.152883in}{2.804589in}}%
\pgfpathlineto{\pgfqpoint{4.160607in}{2.819206in}}%
\pgfpathlineto{\pgfqpoint{4.168328in}{2.834055in}}%
\pgfpathlineto{\pgfqpoint{4.154909in}{2.840445in}}%
\pgfpathlineto{\pgfqpoint{4.141496in}{2.846918in}}%
\pgfpathlineto{\pgfqpoint{4.128086in}{2.853476in}}%
\pgfpathlineto{\pgfqpoint{4.114682in}{2.860117in}}%
\pgfpathlineto{\pgfqpoint{4.106954in}{2.844926in}}%
\pgfpathlineto{\pgfqpoint{4.099223in}{2.829971in}}%
\pgfpathlineto{\pgfqpoint{4.091489in}{2.815247in}}%
\pgfpathlineto{\pgfqpoint{4.083751in}{2.800746in}}%
\pgfpathclose%
\pgfusepath{fill}%
\end{pgfscope}%
\begin{pgfscope}%
\pgfpathrectangle{\pgfqpoint{1.150000in}{0.150000in}}{\pgfqpoint{5.700000in}{5.700000in}}%
\pgfusepath{clip}%
\pgfsetbuttcap%
\pgfsetroundjoin%
\definecolor{currentfill}{rgb}{0.241237,0.296485,0.539709}%
\pgfsetfillcolor{currentfill}%
\pgfsetfillopacity{0.700000}%
\pgfsetlinewidth{0.000000pt}%
\definecolor{currentstroke}{rgb}{0.000000,0.000000,0.000000}%
\pgfsetstrokecolor{currentstroke}%
\pgfsetdash{}{0pt}%
\pgfpathmoveto{\pgfqpoint{4.560384in}{2.966952in}}%
\pgfpathlineto{\pgfqpoint{4.573878in}{2.960695in}}%
\pgfpathlineto{\pgfqpoint{4.587378in}{2.954514in}}%
\pgfpathlineto{\pgfqpoint{4.600884in}{2.948409in}}%
\pgfpathlineto{\pgfqpoint{4.614395in}{2.942380in}}%
\pgfpathlineto{\pgfqpoint{4.622044in}{2.959642in}}%
\pgfpathlineto{\pgfqpoint{4.629693in}{2.977234in}}%
\pgfpathlineto{\pgfqpoint{4.637344in}{2.995163in}}%
\pgfpathlineto{\pgfqpoint{4.644996in}{3.013438in}}%
\pgfpathlineto{\pgfqpoint{4.631493in}{3.019903in}}%
\pgfpathlineto{\pgfqpoint{4.617996in}{3.026444in}}%
\pgfpathlineto{\pgfqpoint{4.604504in}{3.033060in}}%
\pgfpathlineto{\pgfqpoint{4.591017in}{3.039754in}}%
\pgfpathlineto{\pgfqpoint{4.583357in}{3.021035in}}%
\pgfpathlineto{\pgfqpoint{4.575699in}{3.002667in}}%
\pgfpathlineto{\pgfqpoint{4.568041in}{2.984642in}}%
\pgfpathlineto{\pgfqpoint{4.560384in}{2.966952in}}%
\pgfpathclose%
\pgfusepath{fill}%
\end{pgfscope}%
\begin{pgfscope}%
\pgfpathrectangle{\pgfqpoint{1.150000in}{0.150000in}}{\pgfqpoint{5.700000in}{5.700000in}}%
\pgfusepath{clip}%
\pgfsetbuttcap%
\pgfsetroundjoin%
\definecolor{currentfill}{rgb}{0.195860,0.395433,0.555276}%
\pgfsetfillcolor{currentfill}%
\pgfsetfillopacity{0.700000}%
\pgfsetlinewidth{0.000000pt}%
\definecolor{currentstroke}{rgb}{0.000000,0.000000,0.000000}%
\pgfsetstrokecolor{currentstroke}%
\pgfsetdash{}{0pt}%
\pgfpathmoveto{\pgfqpoint{4.845035in}{3.202360in}}%
\pgfpathlineto{\pgfqpoint{4.858563in}{3.195136in}}%
\pgfpathlineto{\pgfqpoint{4.872095in}{3.187985in}}%
\pgfpathlineto{\pgfqpoint{4.885633in}{3.180907in}}%
\pgfpathlineto{\pgfqpoint{4.899176in}{3.173903in}}%
\pgfpathlineto{\pgfqpoint{4.906847in}{3.195932in}}%
\pgfpathlineto{\pgfqpoint{4.914523in}{3.218417in}}%
\pgfpathlineto{\pgfqpoint{4.922206in}{3.241367in}}%
\pgfpathlineto{\pgfqpoint{4.929895in}{3.264793in}}%
\pgfpathlineto{\pgfqpoint{4.916359in}{3.272316in}}%
\pgfpathlineto{\pgfqpoint{4.902829in}{3.279913in}}%
\pgfpathlineto{\pgfqpoint{4.889304in}{3.287582in}}%
\pgfpathlineto{\pgfqpoint{4.875783in}{3.295326in}}%
\pgfpathlineto{\pgfqpoint{4.868087in}{3.271373in}}%
\pgfpathlineto{\pgfqpoint{4.860398in}{3.247901in}}%
\pgfpathlineto{\pgfqpoint{4.852714in}{3.224900in}}%
\pgfpathlineto{\pgfqpoint{4.845035in}{3.202360in}}%
\pgfpathclose%
\pgfusepath{fill}%
\end{pgfscope}%
\begin{pgfscope}%
\pgfpathrectangle{\pgfqpoint{1.150000in}{0.150000in}}{\pgfqpoint{5.700000in}{5.700000in}}%
\pgfusepath{clip}%
\pgfsetbuttcap%
\pgfsetroundjoin%
\definecolor{currentfill}{rgb}{0.250425,0.274290,0.533103}%
\pgfsetfillcolor{currentfill}%
\pgfsetfillopacity{0.700000}%
\pgfsetlinewidth{0.000000pt}%
\definecolor{currentstroke}{rgb}{0.000000,0.000000,0.000000}%
\pgfsetstrokecolor{currentstroke}%
\pgfsetdash{}{0pt}%
\pgfpathmoveto{\pgfqpoint{4.475798in}{2.923516in}}%
\pgfpathlineto{\pgfqpoint{4.489280in}{2.917366in}}%
\pgfpathlineto{\pgfqpoint{4.502766in}{2.911294in}}%
\pgfpathlineto{\pgfqpoint{4.516258in}{2.905299in}}%
\pgfpathlineto{\pgfqpoint{4.529756in}{2.899381in}}%
\pgfpathlineto{\pgfqpoint{4.537413in}{2.915811in}}%
\pgfpathlineto{\pgfqpoint{4.545070in}{2.932544in}}%
\pgfpathlineto{\pgfqpoint{4.552727in}{2.949588in}}%
\pgfpathlineto{\pgfqpoint{4.560384in}{2.966952in}}%
\pgfpathlineto{\pgfqpoint{4.546894in}{2.973286in}}%
\pgfpathlineto{\pgfqpoint{4.533411in}{2.979696in}}%
\pgfpathlineto{\pgfqpoint{4.519932in}{2.986185in}}%
\pgfpathlineto{\pgfqpoint{4.506458in}{2.992750in}}%
\pgfpathlineto{\pgfqpoint{4.498794in}{2.974963in}}%
\pgfpathlineto{\pgfqpoint{4.491129in}{2.957501in}}%
\pgfpathlineto{\pgfqpoint{4.483464in}{2.940354in}}%
\pgfpathlineto{\pgfqpoint{4.475798in}{2.923516in}}%
\pgfpathclose%
\pgfusepath{fill}%
\end{pgfscope}%
\begin{pgfscope}%
\pgfpathrectangle{\pgfqpoint{1.150000in}{0.150000in}}{\pgfqpoint{5.700000in}{5.700000in}}%
\pgfusepath{clip}%
\pgfsetbuttcap%
\pgfsetroundjoin%
\definecolor{currentfill}{rgb}{0.277134,0.185228,0.489898}%
\pgfsetfillcolor{currentfill}%
\pgfsetfillopacity{0.700000}%
\pgfsetlinewidth{0.000000pt}%
\definecolor{currentstroke}{rgb}{0.000000,0.000000,0.000000}%
\pgfsetstrokecolor{currentstroke}%
\pgfsetdash{}{0pt}%
\pgfpathmoveto{\pgfqpoint{3.776247in}{2.740666in}}%
\pgfpathlineto{\pgfqpoint{3.789612in}{2.733838in}}%
\pgfpathlineto{\pgfqpoint{3.802980in}{2.727101in}}%
\pgfpathlineto{\pgfqpoint{3.816353in}{2.720455in}}%
\pgfpathlineto{\pgfqpoint{3.829730in}{2.713900in}}%
\pgfpathlineto{\pgfqpoint{3.837540in}{2.726959in}}%
\pgfpathlineto{\pgfqpoint{3.845346in}{2.740190in}}%
\pgfpathlineto{\pgfqpoint{3.853147in}{2.753598in}}%
\pgfpathlineto{\pgfqpoint{3.860943in}{2.767190in}}%
\pgfpathlineto{\pgfqpoint{3.847574in}{2.774020in}}%
\pgfpathlineto{\pgfqpoint{3.834208in}{2.780941in}}%
\pgfpathlineto{\pgfqpoint{3.820847in}{2.787953in}}%
\pgfpathlineto{\pgfqpoint{3.807489in}{2.795057in}}%
\pgfpathlineto{\pgfqpoint{3.799685in}{2.781183in}}%
\pgfpathlineto{\pgfqpoint{3.791878in}{2.767496in}}%
\pgfpathlineto{\pgfqpoint{3.784065in}{2.753992in}}%
\pgfpathlineto{\pgfqpoint{3.776247in}{2.740666in}}%
\pgfpathclose%
\pgfusepath{fill}%
\end{pgfscope}%
\begin{pgfscope}%
\pgfpathrectangle{\pgfqpoint{1.150000in}{0.150000in}}{\pgfqpoint{5.700000in}{5.700000in}}%
\pgfusepath{clip}%
\pgfsetbuttcap%
\pgfsetroundjoin%
\definecolor{currentfill}{rgb}{0.231674,0.318106,0.544834}%
\pgfsetfillcolor{currentfill}%
\pgfsetfillopacity{0.700000}%
\pgfsetlinewidth{0.000000pt}%
\definecolor{currentstroke}{rgb}{0.000000,0.000000,0.000000}%
\pgfsetstrokecolor{currentstroke}%
\pgfsetdash{}{0pt}%
\pgfpathmoveto{\pgfqpoint{4.644996in}{3.013438in}}%
\pgfpathlineto{\pgfqpoint{4.658504in}{3.007049in}}%
\pgfpathlineto{\pgfqpoint{4.672017in}{3.000735in}}%
\pgfpathlineto{\pgfqpoint{4.685536in}{2.994496in}}%
\pgfpathlineto{\pgfqpoint{4.699061in}{2.988332in}}%
\pgfpathlineto{\pgfqpoint{4.706705in}{3.006513in}}%
\pgfpathlineto{\pgfqpoint{4.714352in}{3.025051in}}%
\pgfpathlineto{\pgfqpoint{4.722001in}{3.043954in}}%
\pgfpathlineto{\pgfqpoint{4.729652in}{3.063232in}}%
\pgfpathlineto{\pgfqpoint{4.716136in}{3.069852in}}%
\pgfpathlineto{\pgfqpoint{4.702625in}{3.076547in}}%
\pgfpathlineto{\pgfqpoint{4.689119in}{3.083318in}}%
\pgfpathlineto{\pgfqpoint{4.675619in}{3.090163in}}%
\pgfpathlineto{\pgfqpoint{4.667960in}{3.070421in}}%
\pgfpathlineto{\pgfqpoint{4.660304in}{3.051059in}}%
\pgfpathlineto{\pgfqpoint{4.652649in}{3.032067in}}%
\pgfpathlineto{\pgfqpoint{4.644996in}{3.013438in}}%
\pgfpathclose%
\pgfusepath{fill}%
\end{pgfscope}%
\begin{pgfscope}%
\pgfpathrectangle{\pgfqpoint{1.150000in}{0.150000in}}{\pgfqpoint{5.700000in}{5.700000in}}%
\pgfusepath{clip}%
\pgfsetbuttcap%
\pgfsetroundjoin%
\definecolor{currentfill}{rgb}{0.265145,0.232956,0.516599}%
\pgfsetfillcolor{currentfill}%
\pgfsetfillopacity{0.700000}%
\pgfsetlinewidth{0.000000pt}%
\definecolor{currentstroke}{rgb}{0.000000,0.000000,0.000000}%
\pgfsetstrokecolor{currentstroke}%
\pgfsetdash{}{0pt}%
\pgfpathmoveto{\pgfqpoint{2.946422in}{2.850897in}}%
\pgfpathlineto{\pgfqpoint{2.959737in}{2.840118in}}%
\pgfpathlineto{\pgfqpoint{2.973052in}{2.829466in}}%
\pgfpathlineto{\pgfqpoint{2.986367in}{2.818943in}}%
\pgfpathlineto{\pgfqpoint{2.999682in}{2.808545in}}%
\pgfpathlineto{\pgfqpoint{3.007728in}{2.820541in}}%
\pgfpathlineto{\pgfqpoint{3.015767in}{2.832672in}}%
\pgfpathlineto{\pgfqpoint{3.023799in}{2.844944in}}%
\pgfpathlineto{\pgfqpoint{3.031824in}{2.857357in}}%
\pgfpathlineto{\pgfqpoint{3.018517in}{2.867910in}}%
\pgfpathlineto{\pgfqpoint{3.005210in}{2.878590in}}%
\pgfpathlineto{\pgfqpoint{2.991904in}{2.889397in}}%
\pgfpathlineto{\pgfqpoint{2.978596in}{2.900332in}}%
\pgfpathlineto{\pgfqpoint{2.970564in}{2.887755in}}%
\pgfpathlineto{\pgfqpoint{2.962524in}{2.875326in}}%
\pgfpathlineto{\pgfqpoint{2.954477in}{2.863041in}}%
\pgfpathlineto{\pgfqpoint{2.946422in}{2.850897in}}%
\pgfpathclose%
\pgfusepath{fill}%
\end{pgfscope}%
\begin{pgfscope}%
\pgfpathrectangle{\pgfqpoint{1.150000in}{0.150000in}}{\pgfqpoint{5.700000in}{5.700000in}}%
\pgfusepath{clip}%
\pgfsetbuttcap%
\pgfsetroundjoin%
\definecolor{currentfill}{rgb}{0.257322,0.256130,0.526563}%
\pgfsetfillcolor{currentfill}%
\pgfsetfillopacity{0.700000}%
\pgfsetlinewidth{0.000000pt}%
\definecolor{currentstroke}{rgb}{0.000000,0.000000,0.000000}%
\pgfsetstrokecolor{currentstroke}%
\pgfsetdash{}{0pt}%
\pgfpathmoveto{\pgfqpoint{4.391223in}{2.882900in}}%
\pgfpathlineto{\pgfqpoint{4.404692in}{2.876832in}}%
\pgfpathlineto{\pgfqpoint{4.418165in}{2.870844in}}%
\pgfpathlineto{\pgfqpoint{4.431644in}{2.864933in}}%
\pgfpathlineto{\pgfqpoint{4.445128in}{2.859101in}}%
\pgfpathlineto{\pgfqpoint{4.452797in}{2.874779in}}%
\pgfpathlineto{\pgfqpoint{4.460465in}{2.890736in}}%
\pgfpathlineto{\pgfqpoint{4.468132in}{2.906979in}}%
\pgfpathlineto{\pgfqpoint{4.475798in}{2.923516in}}%
\pgfpathlineto{\pgfqpoint{4.462322in}{2.929744in}}%
\pgfpathlineto{\pgfqpoint{4.448852in}{2.936050in}}%
\pgfpathlineto{\pgfqpoint{4.435386in}{2.942434in}}%
\pgfpathlineto{\pgfqpoint{4.421925in}{2.948897in}}%
\pgfpathlineto{\pgfqpoint{4.414252in}{2.931957in}}%
\pgfpathlineto{\pgfqpoint{4.406577in}{2.915315in}}%
\pgfpathlineto{\pgfqpoint{4.398901in}{2.898965in}}%
\pgfpathlineto{\pgfqpoint{4.391223in}{2.882900in}}%
\pgfpathclose%
\pgfusepath{fill}%
\end{pgfscope}%
\begin{pgfscope}%
\pgfpathrectangle{\pgfqpoint{1.150000in}{0.150000in}}{\pgfqpoint{5.700000in}{5.700000in}}%
\pgfusepath{clip}%
\pgfsetbuttcap%
\pgfsetroundjoin%
\definecolor{currentfill}{rgb}{0.276194,0.190074,0.493001}%
\pgfsetfillcolor{currentfill}%
\pgfsetfillopacity{0.700000}%
\pgfsetlinewidth{0.000000pt}%
\definecolor{currentstroke}{rgb}{0.000000,0.000000,0.000000}%
\pgfsetstrokecolor{currentstroke}%
\pgfsetdash{}{0pt}%
\pgfpathmoveto{\pgfqpoint{3.276706in}{2.753201in}}%
\pgfpathlineto{\pgfqpoint{3.290019in}{2.744469in}}%
\pgfpathlineto{\pgfqpoint{3.303335in}{2.735847in}}%
\pgfpathlineto{\pgfqpoint{3.316652in}{2.727334in}}%
\pgfpathlineto{\pgfqpoint{3.329971in}{2.718929in}}%
\pgfpathlineto{\pgfqpoint{3.337923in}{2.731164in}}%
\pgfpathlineto{\pgfqpoint{3.345868in}{2.743537in}}%
\pgfpathlineto{\pgfqpoint{3.353808in}{2.756052in}}%
\pgfpathlineto{\pgfqpoint{3.361741in}{2.768712in}}%
\pgfpathlineto{\pgfqpoint{3.348429in}{2.777313in}}%
\pgfpathlineto{\pgfqpoint{3.335119in}{2.786021in}}%
\pgfpathlineto{\pgfqpoint{3.321811in}{2.794838in}}%
\pgfpathlineto{\pgfqpoint{3.308505in}{2.803765in}}%
\pgfpathlineto{\pgfqpoint{3.300565in}{2.790902in}}%
\pgfpathlineto{\pgfqpoint{3.292618in}{2.778190in}}%
\pgfpathlineto{\pgfqpoint{3.284665in}{2.765624in}}%
\pgfpathlineto{\pgfqpoint{3.276706in}{2.753201in}}%
\pgfpathclose%
\pgfusepath{fill}%
\end{pgfscope}%
\begin{pgfscope}%
\pgfpathrectangle{\pgfqpoint{1.150000in}{0.150000in}}{\pgfqpoint{5.700000in}{5.700000in}}%
\pgfusepath{clip}%
\pgfsetbuttcap%
\pgfsetroundjoin%
\definecolor{currentfill}{rgb}{0.274128,0.199721,0.498911}%
\pgfsetfillcolor{currentfill}%
\pgfsetfillopacity{0.700000}%
\pgfsetlinewidth{0.000000pt}%
\definecolor{currentstroke}{rgb}{0.000000,0.000000,0.000000}%
\pgfsetstrokecolor{currentstroke}%
\pgfsetdash{}{0pt}%
\pgfpathmoveto{\pgfqpoint{3.999133in}{2.769685in}}%
\pgfpathlineto{\pgfqpoint{4.012533in}{2.763353in}}%
\pgfpathlineto{\pgfqpoint{4.025939in}{2.757106in}}%
\pgfpathlineto{\pgfqpoint{4.039348in}{2.750945in}}%
\pgfpathlineto{\pgfqpoint{4.052763in}{2.744870in}}%
\pgfpathlineto{\pgfqpoint{4.060516in}{2.758532in}}%
\pgfpathlineto{\pgfqpoint{4.068264in}{2.772395in}}%
\pgfpathlineto{\pgfqpoint{4.076010in}{2.786465in}}%
\pgfpathlineto{\pgfqpoint{4.083751in}{2.800746in}}%
\pgfpathlineto{\pgfqpoint{4.070344in}{2.807137in}}%
\pgfpathlineto{\pgfqpoint{4.056942in}{2.813614in}}%
\pgfpathlineto{\pgfqpoint{4.043544in}{2.820175in}}%
\pgfpathlineto{\pgfqpoint{4.030150in}{2.826823in}}%
\pgfpathlineto{\pgfqpoint{4.022402in}{2.812218in}}%
\pgfpathlineto{\pgfqpoint{4.014649in}{2.797831in}}%
\pgfpathlineto{\pgfqpoint{4.006893in}{2.783655in}}%
\pgfpathlineto{\pgfqpoint{3.999133in}{2.769685in}}%
\pgfpathclose%
\pgfusepath{fill}%
\end{pgfscope}%
\begin{pgfscope}%
\pgfpathrectangle{\pgfqpoint{1.150000in}{0.150000in}}{\pgfqpoint{5.700000in}{5.700000in}}%
\pgfusepath{clip}%
\pgfsetbuttcap%
\pgfsetroundjoin%
\definecolor{currentfill}{rgb}{0.278012,0.180367,0.486697}%
\pgfsetfillcolor{currentfill}%
\pgfsetfillopacity{0.700000}%
\pgfsetlinewidth{0.000000pt}%
\definecolor{currentstroke}{rgb}{0.000000,0.000000,0.000000}%
\pgfsetstrokecolor{currentstroke}%
\pgfsetdash{}{0pt}%
\pgfpathmoveto{\pgfqpoint{3.415009in}{2.735374in}}%
\pgfpathlineto{\pgfqpoint{3.428331in}{2.727302in}}%
\pgfpathlineto{\pgfqpoint{3.441657in}{2.719334in}}%
\pgfpathlineto{\pgfqpoint{3.454984in}{2.711468in}}%
\pgfpathlineto{\pgfqpoint{3.468315in}{2.703705in}}%
\pgfpathlineto{\pgfqpoint{3.476227in}{2.716098in}}%
\pgfpathlineto{\pgfqpoint{3.484133in}{2.728634in}}%
\pgfpathlineto{\pgfqpoint{3.492034in}{2.741318in}}%
\pgfpathlineto{\pgfqpoint{3.499929in}{2.754153in}}%
\pgfpathlineto{\pgfqpoint{3.486605in}{2.762132in}}%
\pgfpathlineto{\pgfqpoint{3.473285in}{2.770213in}}%
\pgfpathlineto{\pgfqpoint{3.459967in}{2.778397in}}%
\pgfpathlineto{\pgfqpoint{3.446651in}{2.786685in}}%
\pgfpathlineto{\pgfqpoint{3.438749in}{2.773626in}}%
\pgfpathlineto{\pgfqpoint{3.430842in}{2.760725in}}%
\pgfpathlineto{\pgfqpoint{3.422928in}{2.747975in}}%
\pgfpathlineto{\pgfqpoint{3.415009in}{2.735374in}}%
\pgfpathclose%
\pgfusepath{fill}%
\end{pgfscope}%
\begin{pgfscope}%
\pgfpathrectangle{\pgfqpoint{1.150000in}{0.150000in}}{\pgfqpoint{5.700000in}{5.700000in}}%
\pgfusepath{clip}%
\pgfsetbuttcap%
\pgfsetroundjoin%
\definecolor{currentfill}{rgb}{0.221989,0.339161,0.548752}%
\pgfsetfillcolor{currentfill}%
\pgfsetfillopacity{0.700000}%
\pgfsetlinewidth{0.000000pt}%
\definecolor{currentstroke}{rgb}{0.000000,0.000000,0.000000}%
\pgfsetstrokecolor{currentstroke}%
\pgfsetdash{}{0pt}%
\pgfpathmoveto{\pgfqpoint{4.729652in}{3.063232in}}%
\pgfpathlineto{\pgfqpoint{4.743173in}{3.056686in}}%
\pgfpathlineto{\pgfqpoint{4.756700in}{3.050215in}}%
\pgfpathlineto{\pgfqpoint{4.770233in}{3.043818in}}%
\pgfpathlineto{\pgfqpoint{4.783771in}{3.037495in}}%
\pgfpathlineto{\pgfqpoint{4.791416in}{3.056687in}}%
\pgfpathlineto{\pgfqpoint{4.799064in}{3.076265in}}%
\pgfpathlineto{\pgfqpoint{4.806716in}{3.096239in}}%
\pgfpathlineto{\pgfqpoint{4.814371in}{3.116617in}}%
\pgfpathlineto{\pgfqpoint{4.800842in}{3.123417in}}%
\pgfpathlineto{\pgfqpoint{4.787317in}{3.130291in}}%
\pgfpathlineto{\pgfqpoint{4.773798in}{3.137240in}}%
\pgfpathlineto{\pgfqpoint{4.760285in}{3.144262in}}%
\pgfpathlineto{\pgfqpoint{4.752622in}{3.123399in}}%
\pgfpathlineto{\pgfqpoint{4.744962in}{3.102946in}}%
\pgfpathlineto{\pgfqpoint{4.737305in}{3.082893in}}%
\pgfpathlineto{\pgfqpoint{4.729652in}{3.063232in}}%
\pgfpathclose%
\pgfusepath{fill}%
\end{pgfscope}%
\begin{pgfscope}%
\pgfpathrectangle{\pgfqpoint{1.150000in}{0.150000in}}{\pgfqpoint{5.700000in}{5.700000in}}%
\pgfusepath{clip}%
\pgfsetbuttcap%
\pgfsetroundjoin%
\definecolor{currentfill}{rgb}{0.263663,0.237631,0.518762}%
\pgfsetfillcolor{currentfill}%
\pgfsetfillopacity{0.700000}%
\pgfsetlinewidth{0.000000pt}%
\definecolor{currentstroke}{rgb}{0.000000,0.000000,0.000000}%
\pgfsetstrokecolor{currentstroke}%
\pgfsetdash{}{0pt}%
\pgfpathmoveto{\pgfqpoint{4.306645in}{2.844896in}}%
\pgfpathlineto{\pgfqpoint{4.320100in}{2.838886in}}%
\pgfpathlineto{\pgfqpoint{4.333561in}{2.832956in}}%
\pgfpathlineto{\pgfqpoint{4.347027in}{2.827106in}}%
\pgfpathlineto{\pgfqpoint{4.360498in}{2.821335in}}%
\pgfpathlineto{\pgfqpoint{4.368182in}{2.836335in}}%
\pgfpathlineto{\pgfqpoint{4.375864in}{2.851591in}}%
\pgfpathlineto{\pgfqpoint{4.383545in}{2.867111in}}%
\pgfpathlineto{\pgfqpoint{4.391223in}{2.882900in}}%
\pgfpathlineto{\pgfqpoint{4.377760in}{2.889046in}}%
\pgfpathlineto{\pgfqpoint{4.364303in}{2.895272in}}%
\pgfpathlineto{\pgfqpoint{4.350850in}{2.901577in}}%
\pgfpathlineto{\pgfqpoint{4.337402in}{2.907962in}}%
\pgfpathlineto{\pgfqpoint{4.329716in}{2.891790in}}%
\pgfpathlineto{\pgfqpoint{4.322027in}{2.875893in}}%
\pgfpathlineto{\pgfqpoint{4.314337in}{2.860264in}}%
\pgfpathlineto{\pgfqpoint{4.306645in}{2.844896in}}%
\pgfpathclose%
\pgfusepath{fill}%
\end{pgfscope}%
\begin{pgfscope}%
\pgfpathrectangle{\pgfqpoint{1.150000in}{0.150000in}}{\pgfqpoint{5.700000in}{5.700000in}}%
\pgfusepath{clip}%
\pgfsetbuttcap%
\pgfsetroundjoin%
\definecolor{currentfill}{rgb}{0.274128,0.199721,0.498911}%
\pgfsetfillcolor{currentfill}%
\pgfsetfillopacity{0.700000}%
\pgfsetlinewidth{0.000000pt}%
\definecolor{currentstroke}{rgb}{0.000000,0.000000,0.000000}%
\pgfsetstrokecolor{currentstroke}%
\pgfsetdash{}{0pt}%
\pgfpathmoveto{\pgfqpoint{3.138285in}{2.777355in}}%
\pgfpathlineto{\pgfqpoint{3.151595in}{2.767893in}}%
\pgfpathlineto{\pgfqpoint{3.164907in}{2.758547in}}%
\pgfpathlineto{\pgfqpoint{3.178219in}{2.749316in}}%
\pgfpathlineto{\pgfqpoint{3.191533in}{2.740201in}}%
\pgfpathlineto{\pgfqpoint{3.199527in}{2.752256in}}%
\pgfpathlineto{\pgfqpoint{3.207514in}{2.764444in}}%
\pgfpathlineto{\pgfqpoint{3.215494in}{2.776771in}}%
\pgfpathlineto{\pgfqpoint{3.223468in}{2.789238in}}%
\pgfpathlineto{\pgfqpoint{3.210162in}{2.798529in}}%
\pgfpathlineto{\pgfqpoint{3.196857in}{2.807935in}}%
\pgfpathlineto{\pgfqpoint{3.183553in}{2.817457in}}%
\pgfpathlineto{\pgfqpoint{3.170250in}{2.827095in}}%
\pgfpathlineto{\pgfqpoint{3.162269in}{2.814445in}}%
\pgfpathlineto{\pgfqpoint{3.154281in}{2.801940in}}%
\pgfpathlineto{\pgfqpoint{3.146286in}{2.789578in}}%
\pgfpathlineto{\pgfqpoint{3.138285in}{2.777355in}}%
\pgfpathclose%
\pgfusepath{fill}%
\end{pgfscope}%
\begin{pgfscope}%
\pgfpathrectangle{\pgfqpoint{1.150000in}{0.150000in}}{\pgfqpoint{5.700000in}{5.700000in}}%
\pgfusepath{clip}%
\pgfsetbuttcap%
\pgfsetroundjoin%
\definecolor{currentfill}{rgb}{0.278826,0.175490,0.483397}%
\pgfsetfillcolor{currentfill}%
\pgfsetfillopacity{0.700000}%
\pgfsetlinewidth{0.000000pt}%
\definecolor{currentstroke}{rgb}{0.000000,0.000000,0.000000}%
\pgfsetstrokecolor{currentstroke}%
\pgfsetdash{}{0pt}%
\pgfpathmoveto{\pgfqpoint{3.553249in}{2.723245in}}%
\pgfpathlineto{\pgfqpoint{3.566586in}{2.715767in}}%
\pgfpathlineto{\pgfqpoint{3.579927in}{2.708387in}}%
\pgfpathlineto{\pgfqpoint{3.593270in}{2.701105in}}%
\pgfpathlineto{\pgfqpoint{3.606617in}{2.693920in}}%
\pgfpathlineto{\pgfqpoint{3.614492in}{2.706455in}}%
\pgfpathlineto{\pgfqpoint{3.622361in}{2.719140in}}%
\pgfpathlineto{\pgfqpoint{3.630224in}{2.731979in}}%
\pgfpathlineto{\pgfqpoint{3.638082in}{2.744977in}}%
\pgfpathlineto{\pgfqpoint{3.624742in}{2.752397in}}%
\pgfpathlineto{\pgfqpoint{3.611406in}{2.759915in}}%
\pgfpathlineto{\pgfqpoint{3.598072in}{2.767530in}}%
\pgfpathlineto{\pgfqpoint{3.584742in}{2.775244in}}%
\pgfpathlineto{\pgfqpoint{3.576877in}{2.762003in}}%
\pgfpathlineto{\pgfqpoint{3.569007in}{2.748926in}}%
\pgfpathlineto{\pgfqpoint{3.561131in}{2.736008in}}%
\pgfpathlineto{\pgfqpoint{3.553249in}{2.723245in}}%
\pgfpathclose%
\pgfusepath{fill}%
\end{pgfscope}%
\begin{pgfscope}%
\pgfpathrectangle{\pgfqpoint{1.150000in}{0.150000in}}{\pgfqpoint{5.700000in}{5.700000in}}%
\pgfusepath{clip}%
\pgfsetbuttcap%
\pgfsetroundjoin%
\definecolor{currentfill}{rgb}{0.168126,0.459988,0.558082}%
\pgfsetfillcolor{currentfill}%
\pgfsetfillopacity{0.700000}%
\pgfsetlinewidth{0.000000pt}%
\definecolor{currentstroke}{rgb}{0.000000,0.000000,0.000000}%
\pgfsetstrokecolor{currentstroke}%
\pgfsetdash{}{0pt}%
\pgfpathmoveto{\pgfqpoint{4.960724in}{3.363454in}}%
\pgfpathlineto{\pgfqpoint{4.974258in}{3.355463in}}%
\pgfpathlineto{\pgfqpoint{4.987797in}{3.347545in}}%
\pgfpathlineto{\pgfqpoint{5.001341in}{3.339700in}}%
\pgfpathlineto{\pgfqpoint{5.014890in}{3.331927in}}%
\pgfpathlineto{\pgfqpoint{5.022611in}{3.357330in}}%
\pgfpathlineto{\pgfqpoint{5.030341in}{3.383265in}}%
\pgfpathlineto{\pgfqpoint{5.038080in}{3.409743in}}%
\pgfpathlineto{\pgfqpoint{5.045829in}{3.436776in}}%
\pgfpathlineto{\pgfqpoint{5.032286in}{3.445111in}}%
\pgfpathlineto{\pgfqpoint{5.018748in}{3.453519in}}%
\pgfpathlineto{\pgfqpoint{5.005215in}{3.462000in}}%
\pgfpathlineto{\pgfqpoint{4.991687in}{3.470553in}}%
\pgfpathlineto{\pgfqpoint{4.983932in}{3.442950in}}%
\pgfpathlineto{\pgfqpoint{4.976187in}{3.415906in}}%
\pgfpathlineto{\pgfqpoint{4.968451in}{3.389411in}}%
\pgfpathlineto{\pgfqpoint{4.960724in}{3.363454in}}%
\pgfpathclose%
\pgfusepath{fill}%
\end{pgfscope}%
\begin{pgfscope}%
\pgfpathrectangle{\pgfqpoint{1.150000in}{0.150000in}}{\pgfqpoint{5.700000in}{5.700000in}}%
\pgfusepath{clip}%
\pgfsetbuttcap%
\pgfsetroundjoin%
\definecolor{currentfill}{rgb}{0.153364,0.497000,0.557724}%
\pgfsetfillcolor{currentfill}%
\pgfsetfillopacity{0.700000}%
\pgfsetlinewidth{0.000000pt}%
\definecolor{currentstroke}{rgb}{0.000000,0.000000,0.000000}%
\pgfsetstrokecolor{currentstroke}%
\pgfsetdash{}{0pt}%
\pgfpathmoveto{\pgfqpoint{4.991687in}{3.470553in}}%
\pgfpathlineto{\pgfqpoint{5.005215in}{3.462000in}}%
\pgfpathlineto{\pgfqpoint{5.018748in}{3.453519in}}%
\pgfpathlineto{\pgfqpoint{5.032286in}{3.445111in}}%
\pgfpathlineto{\pgfqpoint{5.045829in}{3.436776in}}%
\pgfpathlineto{\pgfqpoint{5.053589in}{3.464374in}}%
\pgfpathlineto{\pgfqpoint{5.061359in}{3.492549in}}%
\pgfpathlineto{\pgfqpoint{5.069140in}{3.521312in}}%
\pgfpathlineto{\pgfqpoint{5.055601in}{3.530084in}}%
\pgfpathlineto{\pgfqpoint{5.042067in}{3.538929in}}%
\pgfpathlineto{\pgfqpoint{5.028538in}{3.547847in}}%
\pgfpathlineto{\pgfqpoint{5.015013in}{3.556838in}}%
\pgfpathlineto{\pgfqpoint{5.007227in}{3.527486in}}%
\pgfpathlineto{\pgfqpoint{4.999452in}{3.498728in}}%
\pgfpathlineto{\pgfqpoint{4.991687in}{3.470553in}}%
\pgfpathclose%
\pgfusepath{fill}%
\end{pgfscope}%
\begin{pgfscope}%
\pgfpathrectangle{\pgfqpoint{1.150000in}{0.150000in}}{\pgfqpoint{5.700000in}{5.700000in}}%
\pgfusepath{clip}%
\pgfsetbuttcap%
\pgfsetroundjoin%
\definecolor{currentfill}{rgb}{0.267968,0.223549,0.512008}%
\pgfsetfillcolor{currentfill}%
\pgfsetfillopacity{0.700000}%
\pgfsetlinewidth{0.000000pt}%
\definecolor{currentstroke}{rgb}{0.000000,0.000000,0.000000}%
\pgfsetstrokecolor{currentstroke}%
\pgfsetdash{}{0pt}%
\pgfpathmoveto{\pgfqpoint{4.222050in}{2.809323in}}%
\pgfpathlineto{\pgfqpoint{4.235493in}{2.803346in}}%
\pgfpathlineto{\pgfqpoint{4.248941in}{2.797449in}}%
\pgfpathlineto{\pgfqpoint{4.262394in}{2.791634in}}%
\pgfpathlineto{\pgfqpoint{4.275852in}{2.785899in}}%
\pgfpathlineto{\pgfqpoint{4.283554in}{2.800290in}}%
\pgfpathlineto{\pgfqpoint{4.291254in}{2.814915in}}%
\pgfpathlineto{\pgfqpoint{4.298950in}{2.829782in}}%
\pgfpathlineto{\pgfqpoint{4.306645in}{2.844896in}}%
\pgfpathlineto{\pgfqpoint{4.293195in}{2.850986in}}%
\pgfpathlineto{\pgfqpoint{4.279750in}{2.857157in}}%
\pgfpathlineto{\pgfqpoint{4.266309in}{2.863409in}}%
\pgfpathlineto{\pgfqpoint{4.252874in}{2.869742in}}%
\pgfpathlineto{\pgfqpoint{4.245172in}{2.854265in}}%
\pgfpathlineto{\pgfqpoint{4.237467in}{2.839041in}}%
\pgfpathlineto{\pgfqpoint{4.229760in}{2.824062in}}%
\pgfpathlineto{\pgfqpoint{4.222050in}{2.809323in}}%
\pgfpathclose%
\pgfusepath{fill}%
\end{pgfscope}%
\begin{pgfscope}%
\pgfpathrectangle{\pgfqpoint{1.150000in}{0.150000in}}{\pgfqpoint{5.700000in}{5.700000in}}%
\pgfusepath{clip}%
\pgfsetbuttcap%
\pgfsetroundjoin%
\definecolor{currentfill}{rgb}{0.210503,0.363727,0.552206}%
\pgfsetfillcolor{currentfill}%
\pgfsetfillopacity{0.700000}%
\pgfsetlinewidth{0.000000pt}%
\definecolor{currentstroke}{rgb}{0.000000,0.000000,0.000000}%
\pgfsetstrokecolor{currentstroke}%
\pgfsetdash{}{0pt}%
\pgfpathmoveto{\pgfqpoint{4.814371in}{3.116617in}}%
\pgfpathlineto{\pgfqpoint{4.827906in}{3.109891in}}%
\pgfpathlineto{\pgfqpoint{4.841447in}{3.103238in}}%
\pgfpathlineto{\pgfqpoint{4.854993in}{3.096658in}}%
\pgfpathlineto{\pgfqpoint{4.868545in}{3.090151in}}%
\pgfpathlineto{\pgfqpoint{4.876196in}{3.110454in}}%
\pgfpathlineto{\pgfqpoint{4.883851in}{3.131173in}}%
\pgfpathlineto{\pgfqpoint{4.891511in}{3.152320in}}%
\pgfpathlineto{\pgfqpoint{4.899176in}{3.173903in}}%
\pgfpathlineto{\pgfqpoint{4.885633in}{3.180907in}}%
\pgfpathlineto{\pgfqpoint{4.872095in}{3.187985in}}%
\pgfpathlineto{\pgfqpoint{4.858563in}{3.195136in}}%
\pgfpathlineto{\pgfqpoint{4.845035in}{3.202360in}}%
\pgfpathlineto{\pgfqpoint{4.837362in}{3.180271in}}%
\pgfpathlineto{\pgfqpoint{4.829694in}{3.158624in}}%
\pgfpathlineto{\pgfqpoint{4.822031in}{3.137409in}}%
\pgfpathlineto{\pgfqpoint{4.814371in}{3.116617in}}%
\pgfpathclose%
\pgfusepath{fill}%
\end{pgfscope}%
\begin{pgfscope}%
\pgfpathrectangle{\pgfqpoint{1.150000in}{0.150000in}}{\pgfqpoint{5.700000in}{5.700000in}}%
\pgfusepath{clip}%
\pgfsetbuttcap%
\pgfsetroundjoin%
\definecolor{currentfill}{rgb}{0.183898,0.422383,0.556944}%
\pgfsetfillcolor{currentfill}%
\pgfsetfillopacity{0.700000}%
\pgfsetlinewidth{0.000000pt}%
\definecolor{currentstroke}{rgb}{0.000000,0.000000,0.000000}%
\pgfsetstrokecolor{currentstroke}%
\pgfsetdash{}{0pt}%
\pgfpathmoveto{\pgfqpoint{4.929895in}{3.264793in}}%
\pgfpathlineto{\pgfqpoint{4.943436in}{3.257343in}}%
\pgfpathlineto{\pgfqpoint{4.956982in}{3.249965in}}%
\pgfpathlineto{\pgfqpoint{4.970533in}{3.242660in}}%
\pgfpathlineto{\pgfqpoint{4.984090in}{3.235427in}}%
\pgfpathlineto{\pgfqpoint{4.991779in}{3.258806in}}%
\pgfpathlineto{\pgfqpoint{4.999475in}{3.282676in}}%
\pgfpathlineto{\pgfqpoint{5.007178in}{3.307046in}}%
\pgfpathlineto{\pgfqpoint{5.014890in}{3.331927in}}%
\pgfpathlineto{\pgfqpoint{5.001341in}{3.339700in}}%
\pgfpathlineto{\pgfqpoint{4.987797in}{3.347545in}}%
\pgfpathlineto{\pgfqpoint{4.974258in}{3.355463in}}%
\pgfpathlineto{\pgfqpoint{4.960724in}{3.363454in}}%
\pgfpathlineto{\pgfqpoint{4.953005in}{3.338024in}}%
\pgfpathlineto{\pgfqpoint{4.945294in}{3.313111in}}%
\pgfpathlineto{\pgfqpoint{4.937591in}{3.288704in}}%
\pgfpathlineto{\pgfqpoint{4.929895in}{3.264793in}}%
\pgfpathclose%
\pgfusepath{fill}%
\end{pgfscope}%
\begin{pgfscope}%
\pgfpathrectangle{\pgfqpoint{1.150000in}{0.150000in}}{\pgfqpoint{5.700000in}{5.700000in}}%
\pgfusepath{clip}%
\pgfsetbuttcap%
\pgfsetroundjoin%
\definecolor{currentfill}{rgb}{0.276194,0.190074,0.493001}%
\pgfsetfillcolor{currentfill}%
\pgfsetfillopacity{0.700000}%
\pgfsetlinewidth{0.000000pt}%
\definecolor{currentstroke}{rgb}{0.000000,0.000000,0.000000}%
\pgfsetstrokecolor{currentstroke}%
\pgfsetdash{}{0pt}%
\pgfpathmoveto{\pgfqpoint{3.914462in}{2.740764in}}%
\pgfpathlineto{\pgfqpoint{3.927853in}{2.734379in}}%
\pgfpathlineto{\pgfqpoint{3.941247in}{2.728081in}}%
\pgfpathlineto{\pgfqpoint{3.954646in}{2.721871in}}%
\pgfpathlineto{\pgfqpoint{3.968050in}{2.715748in}}%
\pgfpathlineto{\pgfqpoint{3.975827in}{2.728952in}}%
\pgfpathlineto{\pgfqpoint{3.983600in}{2.742339in}}%
\pgfpathlineto{\pgfqpoint{3.991368in}{2.755915in}}%
\pgfpathlineto{\pgfqpoint{3.999133in}{2.769685in}}%
\pgfpathlineto{\pgfqpoint{3.985737in}{2.776103in}}%
\pgfpathlineto{\pgfqpoint{3.972345in}{2.782609in}}%
\pgfpathlineto{\pgfqpoint{3.958957in}{2.789202in}}%
\pgfpathlineto{\pgfqpoint{3.945574in}{2.795882in}}%
\pgfpathlineto{\pgfqpoint{3.937803in}{2.781809in}}%
\pgfpathlineto{\pgfqpoint{3.930027in}{2.767935in}}%
\pgfpathlineto{\pgfqpoint{3.922247in}{2.754255in}}%
\pgfpathlineto{\pgfqpoint{3.914462in}{2.740764in}}%
\pgfpathclose%
\pgfusepath{fill}%
\end{pgfscope}%
\begin{pgfscope}%
\pgfpathrectangle{\pgfqpoint{1.150000in}{0.150000in}}{\pgfqpoint{5.700000in}{5.700000in}}%
\pgfusepath{clip}%
\pgfsetbuttcap%
\pgfsetroundjoin%
\definecolor{currentfill}{rgb}{0.278826,0.175490,0.483397}%
\pgfsetfillcolor{currentfill}%
\pgfsetfillopacity{0.700000}%
\pgfsetlinewidth{0.000000pt}%
\definecolor{currentstroke}{rgb}{0.000000,0.000000,0.000000}%
\pgfsetstrokecolor{currentstroke}%
\pgfsetdash{}{0pt}%
\pgfpathmoveto{\pgfqpoint{3.691475in}{2.716253in}}%
\pgfpathlineto{\pgfqpoint{3.704832in}{2.709309in}}%
\pgfpathlineto{\pgfqpoint{3.718193in}{2.702459in}}%
\pgfpathlineto{\pgfqpoint{3.731557in}{2.695702in}}%
\pgfpathlineto{\pgfqpoint{3.744925in}{2.689037in}}%
\pgfpathlineto{\pgfqpoint{3.752763in}{2.701702in}}%
\pgfpathlineto{\pgfqpoint{3.760596in}{2.714526in}}%
\pgfpathlineto{\pgfqpoint{3.768424in}{2.727512in}}%
\pgfpathlineto{\pgfqpoint{3.776247in}{2.740666in}}%
\pgfpathlineto{\pgfqpoint{3.762886in}{2.747586in}}%
\pgfpathlineto{\pgfqpoint{3.749529in}{2.754599in}}%
\pgfpathlineto{\pgfqpoint{3.736175in}{2.761704in}}%
\pgfpathlineto{\pgfqpoint{3.722825in}{2.768904in}}%
\pgfpathlineto{\pgfqpoint{3.714996in}{2.755487in}}%
\pgfpathlineto{\pgfqpoint{3.707161in}{2.742243in}}%
\pgfpathlineto{\pgfqpoint{3.699320in}{2.729166in}}%
\pgfpathlineto{\pgfqpoint{3.691475in}{2.716253in}}%
\pgfpathclose%
\pgfusepath{fill}%
\end{pgfscope}%
\begin{pgfscope}%
\pgfpathrectangle{\pgfqpoint{1.150000in}{0.150000in}}{\pgfqpoint{5.700000in}{5.700000in}}%
\pgfusepath{clip}%
\pgfsetbuttcap%
\pgfsetroundjoin%
\definecolor{currentfill}{rgb}{0.269308,0.218818,0.509577}%
\pgfsetfillcolor{currentfill}%
\pgfsetfillopacity{0.700000}%
\pgfsetlinewidth{0.000000pt}%
\definecolor{currentstroke}{rgb}{0.000000,0.000000,0.000000}%
\pgfsetstrokecolor{currentstroke}%
\pgfsetdash{}{0pt}%
\pgfpathmoveto{\pgfqpoint{2.999682in}{2.808545in}}%
\pgfpathlineto{\pgfqpoint{3.012996in}{2.798273in}}%
\pgfpathlineto{\pgfqpoint{3.026311in}{2.788125in}}%
\pgfpathlineto{\pgfqpoint{3.039626in}{2.778100in}}%
\pgfpathlineto{\pgfqpoint{3.052941in}{2.768197in}}%
\pgfpathlineto{\pgfqpoint{3.060979in}{2.780044in}}%
\pgfpathlineto{\pgfqpoint{3.069010in}{2.792023in}}%
\pgfpathlineto{\pgfqpoint{3.077034in}{2.804136in}}%
\pgfpathlineto{\pgfqpoint{3.085051in}{2.816387in}}%
\pgfpathlineto{\pgfqpoint{3.071743in}{2.826445in}}%
\pgfpathlineto{\pgfqpoint{3.058437in}{2.836626in}}%
\pgfpathlineto{\pgfqpoint{3.045130in}{2.846929in}}%
\pgfpathlineto{\pgfqpoint{3.031824in}{2.857357in}}%
\pgfpathlineto{\pgfqpoint{3.023799in}{2.844944in}}%
\pgfpathlineto{\pgfqpoint{3.015767in}{2.832672in}}%
\pgfpathlineto{\pgfqpoint{3.007728in}{2.820541in}}%
\pgfpathlineto{\pgfqpoint{2.999682in}{2.808545in}}%
\pgfpathclose%
\pgfusepath{fill}%
\end{pgfscope}%
\begin{pgfscope}%
\pgfpathrectangle{\pgfqpoint{1.150000in}{0.150000in}}{\pgfqpoint{5.700000in}{5.700000in}}%
\pgfusepath{clip}%
\pgfsetbuttcap%
\pgfsetroundjoin%
\definecolor{currentfill}{rgb}{0.257322,0.256130,0.526563}%
\pgfsetfillcolor{currentfill}%
\pgfsetfillopacity{0.700000}%
\pgfsetlinewidth{0.000000pt}%
\definecolor{currentstroke}{rgb}{0.000000,0.000000,0.000000}%
\pgfsetstrokecolor{currentstroke}%
\pgfsetdash{}{0pt}%
\pgfpathmoveto{\pgfqpoint{2.807505in}{2.893596in}}%
\pgfpathlineto{\pgfqpoint{2.820836in}{2.881882in}}%
\pgfpathlineto{\pgfqpoint{2.834167in}{2.870306in}}%
\pgfpathlineto{\pgfqpoint{2.847496in}{2.858867in}}%
\pgfpathlineto{\pgfqpoint{2.860823in}{2.847563in}}%
\pgfpathlineto{\pgfqpoint{2.868918in}{2.859298in}}%
\pgfpathlineto{\pgfqpoint{2.877005in}{2.871167in}}%
\pgfpathlineto{\pgfqpoint{2.885084in}{2.883175in}}%
\pgfpathlineto{\pgfqpoint{2.893155in}{2.895323in}}%
\pgfpathlineto{\pgfqpoint{2.879836in}{2.906762in}}%
\pgfpathlineto{\pgfqpoint{2.866516in}{2.918337in}}%
\pgfpathlineto{\pgfqpoint{2.853195in}{2.930049in}}%
\pgfpathlineto{\pgfqpoint{2.839872in}{2.941898in}}%
\pgfpathlineto{\pgfqpoint{2.831792in}{2.929607in}}%
\pgfpathlineto{\pgfqpoint{2.823704in}{2.917461in}}%
\pgfpathlineto{\pgfqpoint{2.815609in}{2.905458in}}%
\pgfpathlineto{\pgfqpoint{2.807505in}{2.893596in}}%
\pgfpathclose%
\pgfusepath{fill}%
\end{pgfscope}%
\begin{pgfscope}%
\pgfpathrectangle{\pgfqpoint{1.150000in}{0.150000in}}{\pgfqpoint{5.700000in}{5.700000in}}%
\pgfusepath{clip}%
\pgfsetbuttcap%
\pgfsetroundjoin%
\definecolor{currentfill}{rgb}{0.271828,0.209303,0.504434}%
\pgfsetfillcolor{currentfill}%
\pgfsetfillopacity{0.700000}%
\pgfsetlinewidth{0.000000pt}%
\definecolor{currentstroke}{rgb}{0.000000,0.000000,0.000000}%
\pgfsetstrokecolor{currentstroke}%
\pgfsetdash{}{0pt}%
\pgfpathmoveto{\pgfqpoint{4.137426in}{2.776025in}}%
\pgfpathlineto{\pgfqpoint{4.150857in}{2.770054in}}%
\pgfpathlineto{\pgfqpoint{4.164293in}{2.764166in}}%
\pgfpathlineto{\pgfqpoint{4.177733in}{2.758359in}}%
\pgfpathlineto{\pgfqpoint{4.191179in}{2.752635in}}%
\pgfpathlineto{\pgfqpoint{4.198902in}{2.766480in}}%
\pgfpathlineto{\pgfqpoint{4.206621in}{2.780538in}}%
\pgfpathlineto{\pgfqpoint{4.214337in}{2.794817in}}%
\pgfpathlineto{\pgfqpoint{4.222050in}{2.809323in}}%
\pgfpathlineto{\pgfqpoint{4.208612in}{2.815383in}}%
\pgfpathlineto{\pgfqpoint{4.195179in}{2.821524in}}%
\pgfpathlineto{\pgfqpoint{4.181751in}{2.827748in}}%
\pgfpathlineto{\pgfqpoint{4.168328in}{2.834055in}}%
\pgfpathlineto{\pgfqpoint{4.160607in}{2.819206in}}%
\pgfpathlineto{\pgfqpoint{4.152883in}{2.804589in}}%
\pgfpathlineto{\pgfqpoint{4.145156in}{2.790198in}}%
\pgfpathlineto{\pgfqpoint{4.137426in}{2.776025in}}%
\pgfpathclose%
\pgfusepath{fill}%
\end{pgfscope}%
\begin{pgfscope}%
\pgfpathrectangle{\pgfqpoint{1.150000in}{0.150000in}}{\pgfqpoint{5.700000in}{5.700000in}}%
\pgfusepath{clip}%
\pgfsetbuttcap%
\pgfsetroundjoin%
\definecolor{currentfill}{rgb}{0.199430,0.387607,0.554642}%
\pgfsetfillcolor{currentfill}%
\pgfsetfillopacity{0.700000}%
\pgfsetlinewidth{0.000000pt}%
\definecolor{currentstroke}{rgb}{0.000000,0.000000,0.000000}%
\pgfsetstrokecolor{currentstroke}%
\pgfsetdash{}{0pt}%
\pgfpathmoveto{\pgfqpoint{4.899176in}{3.173903in}}%
\pgfpathlineto{\pgfqpoint{4.912725in}{3.166972in}}%
\pgfpathlineto{\pgfqpoint{4.926279in}{3.160113in}}%
\pgfpathlineto{\pgfqpoint{4.939839in}{3.153326in}}%
\pgfpathlineto{\pgfqpoint{4.953405in}{3.146612in}}%
\pgfpathlineto{\pgfqpoint{4.961067in}{3.168131in}}%
\pgfpathlineto{\pgfqpoint{4.968735in}{3.190099in}}%
\pgfpathlineto{\pgfqpoint{4.976409in}{3.212528in}}%
\pgfpathlineto{\pgfqpoint{4.984090in}{3.235427in}}%
\pgfpathlineto{\pgfqpoint{4.970533in}{3.242660in}}%
\pgfpathlineto{\pgfqpoint{4.956982in}{3.249965in}}%
\pgfpathlineto{\pgfqpoint{4.943436in}{3.257343in}}%
\pgfpathlineto{\pgfqpoint{4.929895in}{3.264793in}}%
\pgfpathlineto{\pgfqpoint{4.922206in}{3.241367in}}%
\pgfpathlineto{\pgfqpoint{4.914523in}{3.218417in}}%
\pgfpathlineto{\pgfqpoint{4.906847in}{3.195932in}}%
\pgfpathlineto{\pgfqpoint{4.899176in}{3.173903in}}%
\pgfpathclose%
\pgfusepath{fill}%
\end{pgfscope}%
\begin{pgfscope}%
\pgfpathrectangle{\pgfqpoint{1.150000in}{0.150000in}}{\pgfqpoint{5.700000in}{5.700000in}}%
\pgfusepath{clip}%
\pgfsetbuttcap%
\pgfsetroundjoin%
\definecolor{currentfill}{rgb}{0.278826,0.175490,0.483397}%
\pgfsetfillcolor{currentfill}%
\pgfsetfillopacity{0.700000}%
\pgfsetlinewidth{0.000000pt}%
\definecolor{currentstroke}{rgb}{0.000000,0.000000,0.000000}%
\pgfsetstrokecolor{currentstroke}%
\pgfsetdash{}{0pt}%
\pgfpathmoveto{\pgfqpoint{3.329971in}{2.718929in}}%
\pgfpathlineto{\pgfqpoint{3.343292in}{2.710632in}}%
\pgfpathlineto{\pgfqpoint{3.356615in}{2.702440in}}%
\pgfpathlineto{\pgfqpoint{3.369940in}{2.694355in}}%
\pgfpathlineto{\pgfqpoint{3.383268in}{2.686374in}}%
\pgfpathlineto{\pgfqpoint{3.391213in}{2.698421in}}%
\pgfpathlineto{\pgfqpoint{3.399151in}{2.710601in}}%
\pgfpathlineto{\pgfqpoint{3.407083in}{2.722917in}}%
\pgfpathlineto{\pgfqpoint{3.415009in}{2.735374in}}%
\pgfpathlineto{\pgfqpoint{3.401688in}{2.743550in}}%
\pgfpathlineto{\pgfqpoint{3.388370in}{2.751832in}}%
\pgfpathlineto{\pgfqpoint{3.375055in}{2.760219in}}%
\pgfpathlineto{\pgfqpoint{3.361741in}{2.768712in}}%
\pgfpathlineto{\pgfqpoint{3.353808in}{2.756052in}}%
\pgfpathlineto{\pgfqpoint{3.345868in}{2.743537in}}%
\pgfpathlineto{\pgfqpoint{3.337923in}{2.731164in}}%
\pgfpathlineto{\pgfqpoint{3.329971in}{2.718929in}}%
\pgfpathclose%
\pgfusepath{fill}%
\end{pgfscope}%
\begin{pgfscope}%
\pgfpathrectangle{\pgfqpoint{1.150000in}{0.150000in}}{\pgfqpoint{5.700000in}{5.700000in}}%
\pgfusepath{clip}%
\pgfsetbuttcap%
\pgfsetroundjoin%
\definecolor{currentfill}{rgb}{0.276194,0.190074,0.493001}%
\pgfsetfillcolor{currentfill}%
\pgfsetfillopacity{0.700000}%
\pgfsetlinewidth{0.000000pt}%
\definecolor{currentstroke}{rgb}{0.000000,0.000000,0.000000}%
\pgfsetstrokecolor{currentstroke}%
\pgfsetdash{}{0pt}%
\pgfpathmoveto{\pgfqpoint{3.191533in}{2.740201in}}%
\pgfpathlineto{\pgfqpoint{3.204848in}{2.731198in}}%
\pgfpathlineto{\pgfqpoint{3.218165in}{2.722309in}}%
\pgfpathlineto{\pgfqpoint{3.231483in}{2.713532in}}%
\pgfpathlineto{\pgfqpoint{3.244802in}{2.704866in}}%
\pgfpathlineto{\pgfqpoint{3.252788in}{2.716753in}}%
\pgfpathlineto{\pgfqpoint{3.260767in}{2.728769in}}%
\pgfpathlineto{\pgfqpoint{3.268740in}{2.740917in}}%
\pgfpathlineto{\pgfqpoint{3.276706in}{2.753201in}}%
\pgfpathlineto{\pgfqpoint{3.263394in}{2.762042in}}%
\pgfpathlineto{\pgfqpoint{3.250084in}{2.770995in}}%
\pgfpathlineto{\pgfqpoint{3.236775in}{2.780060in}}%
\pgfpathlineto{\pgfqpoint{3.223468in}{2.789238in}}%
\pgfpathlineto{\pgfqpoint{3.215494in}{2.776771in}}%
\pgfpathlineto{\pgfqpoint{3.207514in}{2.764444in}}%
\pgfpathlineto{\pgfqpoint{3.199527in}{2.752256in}}%
\pgfpathlineto{\pgfqpoint{3.191533in}{2.740201in}}%
\pgfpathclose%
\pgfusepath{fill}%
\end{pgfscope}%
\begin{pgfscope}%
\pgfpathrectangle{\pgfqpoint{1.150000in}{0.150000in}}{\pgfqpoint{5.700000in}{5.700000in}}%
\pgfusepath{clip}%
\pgfsetbuttcap%
\pgfsetroundjoin%
\definecolor{currentfill}{rgb}{0.279574,0.170599,0.479997}%
\pgfsetfillcolor{currentfill}%
\pgfsetfillopacity{0.700000}%
\pgfsetlinewidth{0.000000pt}%
\definecolor{currentstroke}{rgb}{0.000000,0.000000,0.000000}%
\pgfsetstrokecolor{currentstroke}%
\pgfsetdash{}{0pt}%
\pgfpathmoveto{\pgfqpoint{3.468315in}{2.703705in}}%
\pgfpathlineto{\pgfqpoint{3.481648in}{2.696043in}}%
\pgfpathlineto{\pgfqpoint{3.494983in}{2.688482in}}%
\pgfpathlineto{\pgfqpoint{3.508322in}{2.681021in}}%
\pgfpathlineto{\pgfqpoint{3.521664in}{2.673659in}}%
\pgfpathlineto{\pgfqpoint{3.529569in}{2.685844in}}%
\pgfpathlineto{\pgfqpoint{3.537468in}{2.698168in}}%
\pgfpathlineto{\pgfqpoint{3.545361in}{2.710633in}}%
\pgfpathlineto{\pgfqpoint{3.553249in}{2.723245in}}%
\pgfpathlineto{\pgfqpoint{3.539915in}{2.730822in}}%
\pgfpathlineto{\pgfqpoint{3.526583in}{2.738499in}}%
\pgfpathlineto{\pgfqpoint{3.513254in}{2.746275in}}%
\pgfpathlineto{\pgfqpoint{3.499929in}{2.754153in}}%
\pgfpathlineto{\pgfqpoint{3.492034in}{2.741318in}}%
\pgfpathlineto{\pgfqpoint{3.484133in}{2.728634in}}%
\pgfpathlineto{\pgfqpoint{3.476227in}{2.716098in}}%
\pgfpathlineto{\pgfqpoint{3.468315in}{2.703705in}}%
\pgfpathclose%
\pgfusepath{fill}%
\end{pgfscope}%
\begin{pgfscope}%
\pgfpathrectangle{\pgfqpoint{1.150000in}{0.150000in}}{\pgfqpoint{5.700000in}{5.700000in}}%
\pgfusepath{clip}%
\pgfsetbuttcap%
\pgfsetroundjoin%
\definecolor{currentfill}{rgb}{0.243113,0.292092,0.538516}%
\pgfsetfillcolor{currentfill}%
\pgfsetfillopacity{0.700000}%
\pgfsetlinewidth{0.000000pt}%
\definecolor{currentstroke}{rgb}{0.000000,0.000000,0.000000}%
\pgfsetstrokecolor{currentstroke}%
\pgfsetdash{}{0pt}%
\pgfpathmoveto{\pgfqpoint{4.614395in}{2.942380in}}%
\pgfpathlineto{\pgfqpoint{4.627912in}{2.936427in}}%
\pgfpathlineto{\pgfqpoint{4.641434in}{2.930549in}}%
\pgfpathlineto{\pgfqpoint{4.654962in}{2.924747in}}%
\pgfpathlineto{\pgfqpoint{4.668496in}{2.919019in}}%
\pgfpathlineto{\pgfqpoint{4.676136in}{2.935852in}}%
\pgfpathlineto{\pgfqpoint{4.683776in}{2.953010in}}%
\pgfpathlineto{\pgfqpoint{4.691418in}{2.970501in}}%
\pgfpathlineto{\pgfqpoint{4.699061in}{2.988332in}}%
\pgfpathlineto{\pgfqpoint{4.685536in}{2.994496in}}%
\pgfpathlineto{\pgfqpoint{4.672017in}{3.000735in}}%
\pgfpathlineto{\pgfqpoint{4.658504in}{3.007049in}}%
\pgfpathlineto{\pgfqpoint{4.644996in}{3.013438in}}%
\pgfpathlineto{\pgfqpoint{4.637344in}{2.995163in}}%
\pgfpathlineto{\pgfqpoint{4.629693in}{2.977234in}}%
\pgfpathlineto{\pgfqpoint{4.622044in}{2.959642in}}%
\pgfpathlineto{\pgfqpoint{4.614395in}{2.942380in}}%
\pgfpathclose%
\pgfusepath{fill}%
\end{pgfscope}%
\begin{pgfscope}%
\pgfpathrectangle{\pgfqpoint{1.150000in}{0.150000in}}{\pgfqpoint{5.700000in}{5.700000in}}%
\pgfusepath{clip}%
\pgfsetbuttcap%
\pgfsetroundjoin%
\definecolor{currentfill}{rgb}{0.252194,0.269783,0.531579}%
\pgfsetfillcolor{currentfill}%
\pgfsetfillopacity{0.700000}%
\pgfsetlinewidth{0.000000pt}%
\definecolor{currentstroke}{rgb}{0.000000,0.000000,0.000000}%
\pgfsetstrokecolor{currentstroke}%
\pgfsetdash{}{0pt}%
\pgfpathmoveto{\pgfqpoint{4.529756in}{2.899381in}}%
\pgfpathlineto{\pgfqpoint{4.543259in}{2.893539in}}%
\pgfpathlineto{\pgfqpoint{4.556768in}{2.887774in}}%
\pgfpathlineto{\pgfqpoint{4.570282in}{2.882085in}}%
\pgfpathlineto{\pgfqpoint{4.583802in}{2.876472in}}%
\pgfpathlineto{\pgfqpoint{4.591451in}{2.892494in}}%
\pgfpathlineto{\pgfqpoint{4.599099in}{2.908814in}}%
\pgfpathlineto{\pgfqpoint{4.606747in}{2.925440in}}%
\pgfpathlineto{\pgfqpoint{4.614395in}{2.942380in}}%
\pgfpathlineto{\pgfqpoint{4.600884in}{2.948409in}}%
\pgfpathlineto{\pgfqpoint{4.587378in}{2.954514in}}%
\pgfpathlineto{\pgfqpoint{4.573878in}{2.960695in}}%
\pgfpathlineto{\pgfqpoint{4.560384in}{2.966952in}}%
\pgfpathlineto{\pgfqpoint{4.552727in}{2.949588in}}%
\pgfpathlineto{\pgfqpoint{4.545070in}{2.932544in}}%
\pgfpathlineto{\pgfqpoint{4.537413in}{2.915811in}}%
\pgfpathlineto{\pgfqpoint{4.529756in}{2.899381in}}%
\pgfpathclose%
\pgfusepath{fill}%
\end{pgfscope}%
\begin{pgfscope}%
\pgfpathrectangle{\pgfqpoint{1.150000in}{0.150000in}}{\pgfqpoint{5.700000in}{5.700000in}}%
\pgfusepath{clip}%
\pgfsetbuttcap%
\pgfsetroundjoin%
\definecolor{currentfill}{rgb}{0.278012,0.180367,0.486697}%
\pgfsetfillcolor{currentfill}%
\pgfsetfillopacity{0.700000}%
\pgfsetlinewidth{0.000000pt}%
\definecolor{currentstroke}{rgb}{0.000000,0.000000,0.000000}%
\pgfsetstrokecolor{currentstroke}%
\pgfsetdash{}{0pt}%
\pgfpathmoveto{\pgfqpoint{3.829730in}{2.713900in}}%
\pgfpathlineto{\pgfqpoint{3.843110in}{2.707435in}}%
\pgfpathlineto{\pgfqpoint{3.856495in}{2.701060in}}%
\pgfpathlineto{\pgfqpoint{3.869884in}{2.694774in}}%
\pgfpathlineto{\pgfqpoint{3.883278in}{2.688576in}}%
\pgfpathlineto{\pgfqpoint{3.891081in}{2.701366in}}%
\pgfpathlineto{\pgfqpoint{3.898879in}{2.714324in}}%
\pgfpathlineto{\pgfqpoint{3.906673in}{2.727455in}}%
\pgfpathlineto{\pgfqpoint{3.914462in}{2.740764in}}%
\pgfpathlineto{\pgfqpoint{3.901076in}{2.747237in}}%
\pgfpathlineto{\pgfqpoint{3.887694in}{2.753799in}}%
\pgfpathlineto{\pgfqpoint{3.874317in}{2.760449in}}%
\pgfpathlineto{\pgfqpoint{3.860943in}{2.767190in}}%
\pgfpathlineto{\pgfqpoint{3.853147in}{2.753598in}}%
\pgfpathlineto{\pgfqpoint{3.845346in}{2.740190in}}%
\pgfpathlineto{\pgfqpoint{3.837540in}{2.726959in}}%
\pgfpathlineto{\pgfqpoint{3.829730in}{2.713900in}}%
\pgfpathclose%
\pgfusepath{fill}%
\end{pgfscope}%
\begin{pgfscope}%
\pgfpathrectangle{\pgfqpoint{1.150000in}{0.150000in}}{\pgfqpoint{5.700000in}{5.700000in}}%
\pgfusepath{clip}%
\pgfsetbuttcap%
\pgfsetroundjoin%
\definecolor{currentfill}{rgb}{0.235526,0.309527,0.542944}%
\pgfsetfillcolor{currentfill}%
\pgfsetfillopacity{0.700000}%
\pgfsetlinewidth{0.000000pt}%
\definecolor{currentstroke}{rgb}{0.000000,0.000000,0.000000}%
\pgfsetstrokecolor{currentstroke}%
\pgfsetdash{}{0pt}%
\pgfpathmoveto{\pgfqpoint{4.699061in}{2.988332in}}%
\pgfpathlineto{\pgfqpoint{4.712591in}{2.982243in}}%
\pgfpathlineto{\pgfqpoint{4.726127in}{2.976228in}}%
\pgfpathlineto{\pgfqpoint{4.739669in}{2.970288in}}%
\pgfpathlineto{\pgfqpoint{4.753216in}{2.964421in}}%
\pgfpathlineto{\pgfqpoint{4.760852in}{2.982152in}}%
\pgfpathlineto{\pgfqpoint{4.768489in}{3.000236in}}%
\pgfpathlineto{\pgfqpoint{4.776129in}{3.018681in}}%
\pgfpathlineto{\pgfqpoint{4.783771in}{3.037495in}}%
\pgfpathlineto{\pgfqpoint{4.770233in}{3.043818in}}%
\pgfpathlineto{\pgfqpoint{4.756700in}{3.050215in}}%
\pgfpathlineto{\pgfqpoint{4.743173in}{3.056686in}}%
\pgfpathlineto{\pgfqpoint{4.729652in}{3.063232in}}%
\pgfpathlineto{\pgfqpoint{4.722001in}{3.043954in}}%
\pgfpathlineto{\pgfqpoint{4.714352in}{3.025051in}}%
\pgfpathlineto{\pgfqpoint{4.706705in}{3.006513in}}%
\pgfpathlineto{\pgfqpoint{4.699061in}{2.988332in}}%
\pgfpathclose%
\pgfusepath{fill}%
\end{pgfscope}%
\begin{pgfscope}%
\pgfpathrectangle{\pgfqpoint{1.150000in}{0.150000in}}{\pgfqpoint{5.700000in}{5.700000in}}%
\pgfusepath{clip}%
\pgfsetbuttcap%
\pgfsetroundjoin%
\definecolor{currentfill}{rgb}{0.258965,0.251537,0.524736}%
\pgfsetfillcolor{currentfill}%
\pgfsetfillopacity{0.700000}%
\pgfsetlinewidth{0.000000pt}%
\definecolor{currentstroke}{rgb}{0.000000,0.000000,0.000000}%
\pgfsetstrokecolor{currentstroke}%
\pgfsetdash{}{0pt}%
\pgfpathmoveto{\pgfqpoint{4.445128in}{2.859101in}}%
\pgfpathlineto{\pgfqpoint{4.458618in}{2.853347in}}%
\pgfpathlineto{\pgfqpoint{4.472113in}{2.847670in}}%
\pgfpathlineto{\pgfqpoint{4.485614in}{2.842071in}}%
\pgfpathlineto{\pgfqpoint{4.499121in}{2.836548in}}%
\pgfpathlineto{\pgfqpoint{4.506781in}{2.851838in}}%
\pgfpathlineto{\pgfqpoint{4.514440in}{2.867402in}}%
\pgfpathlineto{\pgfqpoint{4.522098in}{2.883247in}}%
\pgfpathlineto{\pgfqpoint{4.529756in}{2.899381in}}%
\pgfpathlineto{\pgfqpoint{4.516258in}{2.905299in}}%
\pgfpathlineto{\pgfqpoint{4.502766in}{2.911294in}}%
\pgfpathlineto{\pgfqpoint{4.489280in}{2.917366in}}%
\pgfpathlineto{\pgfqpoint{4.475798in}{2.923516in}}%
\pgfpathlineto{\pgfqpoint{4.468132in}{2.906979in}}%
\pgfpathlineto{\pgfqpoint{4.460465in}{2.890736in}}%
\pgfpathlineto{\pgfqpoint{4.452797in}{2.874779in}}%
\pgfpathlineto{\pgfqpoint{4.445128in}{2.859101in}}%
\pgfpathclose%
\pgfusepath{fill}%
\end{pgfscope}%
\begin{pgfscope}%
\pgfpathrectangle{\pgfqpoint{1.150000in}{0.150000in}}{\pgfqpoint{5.700000in}{5.700000in}}%
\pgfusepath{clip}%
\pgfsetbuttcap%
\pgfsetroundjoin%
\definecolor{currentfill}{rgb}{0.275191,0.194905,0.496005}%
\pgfsetfillcolor{currentfill}%
\pgfsetfillopacity{0.700000}%
\pgfsetlinewidth{0.000000pt}%
\definecolor{currentstroke}{rgb}{0.000000,0.000000,0.000000}%
\pgfsetstrokecolor{currentstroke}%
\pgfsetdash{}{0pt}%
\pgfpathmoveto{\pgfqpoint{4.052763in}{2.744870in}}%
\pgfpathlineto{\pgfqpoint{4.066182in}{2.738879in}}%
\pgfpathlineto{\pgfqpoint{4.079606in}{2.732972in}}%
\pgfpathlineto{\pgfqpoint{4.093035in}{2.727150in}}%
\pgfpathlineto{\pgfqpoint{4.106469in}{2.721411in}}%
\pgfpathlineto{\pgfqpoint{4.114214in}{2.734765in}}%
\pgfpathlineto{\pgfqpoint{4.121955in}{2.748315in}}%
\pgfpathlineto{\pgfqpoint{4.129692in}{2.762067in}}%
\pgfpathlineto{\pgfqpoint{4.137426in}{2.776025in}}%
\pgfpathlineto{\pgfqpoint{4.124000in}{2.782080in}}%
\pgfpathlineto{\pgfqpoint{4.110579in}{2.788218in}}%
\pgfpathlineto{\pgfqpoint{4.097163in}{2.794440in}}%
\pgfpathlineto{\pgfqpoint{4.083751in}{2.800746in}}%
\pgfpathlineto{\pgfqpoint{4.076010in}{2.786465in}}%
\pgfpathlineto{\pgfqpoint{4.068264in}{2.772395in}}%
\pgfpathlineto{\pgfqpoint{4.060516in}{2.758532in}}%
\pgfpathlineto{\pgfqpoint{4.052763in}{2.744870in}}%
\pgfpathclose%
\pgfusepath{fill}%
\end{pgfscope}%
\begin{pgfscope}%
\pgfpathrectangle{\pgfqpoint{1.150000in}{0.150000in}}{\pgfqpoint{5.700000in}{5.700000in}}%
\pgfusepath{clip}%
\pgfsetbuttcap%
\pgfsetroundjoin%
\definecolor{currentfill}{rgb}{0.156270,0.489624,0.557936}%
\pgfsetfillcolor{currentfill}%
\pgfsetfillopacity{0.700000}%
\pgfsetlinewidth{0.000000pt}%
\definecolor{currentstroke}{rgb}{0.000000,0.000000,0.000000}%
\pgfsetstrokecolor{currentstroke}%
\pgfsetdash{}{0pt}%
\pgfpathmoveto{\pgfqpoint{5.045829in}{3.436776in}}%
\pgfpathlineto{\pgfqpoint{5.059377in}{3.428513in}}%
\pgfpathlineto{\pgfqpoint{5.072930in}{3.420322in}}%
\pgfpathlineto{\pgfqpoint{5.086488in}{3.412203in}}%
\pgfpathlineto{\pgfqpoint{5.100051in}{3.404156in}}%
\pgfpathlineto{\pgfqpoint{5.107804in}{3.431178in}}%
\pgfpathlineto{\pgfqpoint{5.115568in}{3.458771in}}%
\pgfpathlineto{\pgfqpoint{5.123344in}{3.486947in}}%
\pgfpathlineto{\pgfqpoint{5.109786in}{3.495430in}}%
\pgfpathlineto{\pgfqpoint{5.096232in}{3.503986in}}%
\pgfpathlineto{\pgfqpoint{5.082684in}{3.512613in}}%
\pgfpathlineto{\pgfqpoint{5.069140in}{3.521312in}}%
\pgfpathlineto{\pgfqpoint{5.061359in}{3.492549in}}%
\pgfpathlineto{\pgfqpoint{5.053589in}{3.464374in}}%
\pgfpathlineto{\pgfqpoint{5.045829in}{3.436776in}}%
\pgfpathclose%
\pgfusepath{fill}%
\end{pgfscope}%
\begin{pgfscope}%
\pgfpathrectangle{\pgfqpoint{1.150000in}{0.150000in}}{\pgfqpoint{5.700000in}{5.700000in}}%
\pgfusepath{clip}%
\pgfsetbuttcap%
\pgfsetroundjoin%
\definecolor{currentfill}{rgb}{0.263663,0.237631,0.518762}%
\pgfsetfillcolor{currentfill}%
\pgfsetfillopacity{0.700000}%
\pgfsetlinewidth{0.000000pt}%
\definecolor{currentstroke}{rgb}{0.000000,0.000000,0.000000}%
\pgfsetstrokecolor{currentstroke}%
\pgfsetdash{}{0pt}%
\pgfpathmoveto{\pgfqpoint{2.860823in}{2.847563in}}%
\pgfpathlineto{\pgfqpoint{2.874150in}{2.836394in}}%
\pgfpathlineto{\pgfqpoint{2.887476in}{2.825357in}}%
\pgfpathlineto{\pgfqpoint{2.900802in}{2.814453in}}%
\pgfpathlineto{\pgfqpoint{2.914126in}{2.803679in}}%
\pgfpathlineto{\pgfqpoint{2.922212in}{2.815286in}}%
\pgfpathlineto{\pgfqpoint{2.930289in}{2.827023in}}%
\pgfpathlineto{\pgfqpoint{2.938359in}{2.838892in}}%
\pgfpathlineto{\pgfqpoint{2.946422in}{2.850897in}}%
\pgfpathlineto{\pgfqpoint{2.933106in}{2.861806in}}%
\pgfpathlineto{\pgfqpoint{2.919790in}{2.872846in}}%
\pgfpathlineto{\pgfqpoint{2.906473in}{2.884018in}}%
\pgfpathlineto{\pgfqpoint{2.893155in}{2.895323in}}%
\pgfpathlineto{\pgfqpoint{2.885084in}{2.883175in}}%
\pgfpathlineto{\pgfqpoint{2.877005in}{2.871167in}}%
\pgfpathlineto{\pgfqpoint{2.868918in}{2.859298in}}%
\pgfpathlineto{\pgfqpoint{2.860823in}{2.847563in}}%
\pgfpathclose%
\pgfusepath{fill}%
\end{pgfscope}%
\begin{pgfscope}%
\pgfpathrectangle{\pgfqpoint{1.150000in}{0.150000in}}{\pgfqpoint{5.700000in}{5.700000in}}%
\pgfusepath{clip}%
\pgfsetbuttcap%
\pgfsetroundjoin%
\definecolor{currentfill}{rgb}{0.171176,0.452530,0.557965}%
\pgfsetfillcolor{currentfill}%
\pgfsetfillopacity{0.700000}%
\pgfsetlinewidth{0.000000pt}%
\definecolor{currentstroke}{rgb}{0.000000,0.000000,0.000000}%
\pgfsetstrokecolor{currentstroke}%
\pgfsetdash{}{0pt}%
\pgfpathmoveto{\pgfqpoint{5.014890in}{3.331927in}}%
\pgfpathlineto{\pgfqpoint{5.028445in}{3.324226in}}%
\pgfpathlineto{\pgfqpoint{5.042005in}{3.316597in}}%
\pgfpathlineto{\pgfqpoint{5.055570in}{3.309040in}}%
\pgfpathlineto{\pgfqpoint{5.069140in}{3.301555in}}%
\pgfpathlineto{\pgfqpoint{5.076854in}{3.326404in}}%
\pgfpathlineto{\pgfqpoint{5.084576in}{3.351780in}}%
\pgfpathlineto{\pgfqpoint{5.092308in}{3.377694in}}%
\pgfpathlineto{\pgfqpoint{5.100051in}{3.404156in}}%
\pgfpathlineto{\pgfqpoint{5.086488in}{3.412203in}}%
\pgfpathlineto{\pgfqpoint{5.072930in}{3.420322in}}%
\pgfpathlineto{\pgfqpoint{5.059377in}{3.428513in}}%
\pgfpathlineto{\pgfqpoint{5.045829in}{3.436776in}}%
\pgfpathlineto{\pgfqpoint{5.038080in}{3.409743in}}%
\pgfpathlineto{\pgfqpoint{5.030341in}{3.383265in}}%
\pgfpathlineto{\pgfqpoint{5.022611in}{3.357330in}}%
\pgfpathlineto{\pgfqpoint{5.014890in}{3.331927in}}%
\pgfpathclose%
\pgfusepath{fill}%
\end{pgfscope}%
\begin{pgfscope}%
\pgfpathrectangle{\pgfqpoint{1.150000in}{0.150000in}}{\pgfqpoint{5.700000in}{5.700000in}}%
\pgfusepath{clip}%
\pgfsetbuttcap%
\pgfsetroundjoin%
\definecolor{currentfill}{rgb}{0.279574,0.170599,0.479997}%
\pgfsetfillcolor{currentfill}%
\pgfsetfillopacity{0.700000}%
\pgfsetlinewidth{0.000000pt}%
\definecolor{currentstroke}{rgb}{0.000000,0.000000,0.000000}%
\pgfsetstrokecolor{currentstroke}%
\pgfsetdash{}{0pt}%
\pgfpathmoveto{\pgfqpoint{3.606617in}{2.693920in}}%
\pgfpathlineto{\pgfqpoint{3.619968in}{2.686832in}}%
\pgfpathlineto{\pgfqpoint{3.633321in}{2.679839in}}%
\pgfpathlineto{\pgfqpoint{3.646678in}{2.672942in}}%
\pgfpathlineto{\pgfqpoint{3.660039in}{2.666139in}}%
\pgfpathlineto{\pgfqpoint{3.667906in}{2.678446in}}%
\pgfpathlineto{\pgfqpoint{3.675768in}{2.690897in}}%
\pgfpathlineto{\pgfqpoint{3.683624in}{2.703498in}}%
\pgfpathlineto{\pgfqpoint{3.691475in}{2.716253in}}%
\pgfpathlineto{\pgfqpoint{3.678122in}{2.723291in}}%
\pgfpathlineto{\pgfqpoint{3.664772in}{2.730424in}}%
\pgfpathlineto{\pgfqpoint{3.651425in}{2.737652in}}%
\pgfpathlineto{\pgfqpoint{3.638082in}{2.744977in}}%
\pgfpathlineto{\pgfqpoint{3.630224in}{2.731979in}}%
\pgfpathlineto{\pgfqpoint{3.622361in}{2.719140in}}%
\pgfpathlineto{\pgfqpoint{3.614492in}{2.706455in}}%
\pgfpathlineto{\pgfqpoint{3.606617in}{2.693920in}}%
\pgfpathclose%
\pgfusepath{fill}%
\end{pgfscope}%
\begin{pgfscope}%
\pgfpathrectangle{\pgfqpoint{1.150000in}{0.150000in}}{\pgfqpoint{5.700000in}{5.700000in}}%
\pgfusepath{clip}%
\pgfsetbuttcap%
\pgfsetroundjoin%
\definecolor{currentfill}{rgb}{0.223925,0.334994,0.548053}%
\pgfsetfillcolor{currentfill}%
\pgfsetfillopacity{0.700000}%
\pgfsetlinewidth{0.000000pt}%
\definecolor{currentstroke}{rgb}{0.000000,0.000000,0.000000}%
\pgfsetstrokecolor{currentstroke}%
\pgfsetdash{}{0pt}%
\pgfpathmoveto{\pgfqpoint{4.783771in}{3.037495in}}%
\pgfpathlineto{\pgfqpoint{4.797315in}{3.031245in}}%
\pgfpathlineto{\pgfqpoint{4.810865in}{3.025069in}}%
\pgfpathlineto{\pgfqpoint{4.824420in}{3.018966in}}%
\pgfpathlineto{\pgfqpoint{4.837981in}{3.012937in}}%
\pgfpathlineto{\pgfqpoint{4.845617in}{3.031659in}}%
\pgfpathlineto{\pgfqpoint{4.853256in}{3.050763in}}%
\pgfpathlineto{\pgfqpoint{4.860899in}{3.070258in}}%
\pgfpathlineto{\pgfqpoint{4.868545in}{3.090151in}}%
\pgfpathlineto{\pgfqpoint{4.854993in}{3.096658in}}%
\pgfpathlineto{\pgfqpoint{4.841447in}{3.103238in}}%
\pgfpathlineto{\pgfqpoint{4.827906in}{3.109891in}}%
\pgfpathlineto{\pgfqpoint{4.814371in}{3.116617in}}%
\pgfpathlineto{\pgfqpoint{4.806716in}{3.096239in}}%
\pgfpathlineto{\pgfqpoint{4.799064in}{3.076265in}}%
\pgfpathlineto{\pgfqpoint{4.791416in}{3.056687in}}%
\pgfpathlineto{\pgfqpoint{4.783771in}{3.037495in}}%
\pgfpathclose%
\pgfusepath{fill}%
\end{pgfscope}%
\begin{pgfscope}%
\pgfpathrectangle{\pgfqpoint{1.150000in}{0.150000in}}{\pgfqpoint{5.700000in}{5.700000in}}%
\pgfusepath{clip}%
\pgfsetbuttcap%
\pgfsetroundjoin%
\definecolor{currentfill}{rgb}{0.274128,0.199721,0.498911}%
\pgfsetfillcolor{currentfill}%
\pgfsetfillopacity{0.700000}%
\pgfsetlinewidth{0.000000pt}%
\definecolor{currentstroke}{rgb}{0.000000,0.000000,0.000000}%
\pgfsetstrokecolor{currentstroke}%
\pgfsetdash{}{0pt}%
\pgfpathmoveto{\pgfqpoint{3.052941in}{2.768197in}}%
\pgfpathlineto{\pgfqpoint{3.066257in}{2.758415in}}%
\pgfpathlineto{\pgfqpoint{3.079574in}{2.748754in}}%
\pgfpathlineto{\pgfqpoint{3.092891in}{2.739212in}}%
\pgfpathlineto{\pgfqpoint{3.106208in}{2.729787in}}%
\pgfpathlineto{\pgfqpoint{3.114238in}{2.741487in}}%
\pgfpathlineto{\pgfqpoint{3.122261in}{2.753313in}}%
\pgfpathlineto{\pgfqpoint{3.130276in}{2.765268in}}%
\pgfpathlineto{\pgfqpoint{3.138285in}{2.777355in}}%
\pgfpathlineto{\pgfqpoint{3.124975in}{2.786935in}}%
\pgfpathlineto{\pgfqpoint{3.111666in}{2.796633in}}%
\pgfpathlineto{\pgfqpoint{3.098358in}{2.806450in}}%
\pgfpathlineto{\pgfqpoint{3.085051in}{2.816387in}}%
\pgfpathlineto{\pgfqpoint{3.077034in}{2.804136in}}%
\pgfpathlineto{\pgfqpoint{3.069010in}{2.792023in}}%
\pgfpathlineto{\pgfqpoint{3.060979in}{2.780044in}}%
\pgfpathlineto{\pgfqpoint{3.052941in}{2.768197in}}%
\pgfpathclose%
\pgfusepath{fill}%
\end{pgfscope}%
\begin{pgfscope}%
\pgfpathrectangle{\pgfqpoint{1.150000in}{0.150000in}}{\pgfqpoint{5.700000in}{5.700000in}}%
\pgfusepath{clip}%
\pgfsetbuttcap%
\pgfsetroundjoin%
\definecolor{currentfill}{rgb}{0.265145,0.232956,0.516599}%
\pgfsetfillcolor{currentfill}%
\pgfsetfillopacity{0.700000}%
\pgfsetlinewidth{0.000000pt}%
\definecolor{currentstroke}{rgb}{0.000000,0.000000,0.000000}%
\pgfsetstrokecolor{currentstroke}%
\pgfsetdash{}{0pt}%
\pgfpathmoveto{\pgfqpoint{4.360498in}{2.821335in}}%
\pgfpathlineto{\pgfqpoint{4.373975in}{2.815643in}}%
\pgfpathlineto{\pgfqpoint{4.387457in}{2.810030in}}%
\pgfpathlineto{\pgfqpoint{4.400944in}{2.804495in}}%
\pgfpathlineto{\pgfqpoint{4.414437in}{2.799039in}}%
\pgfpathlineto{\pgfqpoint{4.422113in}{2.813671in}}%
\pgfpathlineto{\pgfqpoint{4.429786in}{2.828554in}}%
\pgfpathlineto{\pgfqpoint{4.437458in}{2.843695in}}%
\pgfpathlineto{\pgfqpoint{4.445128in}{2.859101in}}%
\pgfpathlineto{\pgfqpoint{4.431644in}{2.864933in}}%
\pgfpathlineto{\pgfqpoint{4.418165in}{2.870844in}}%
\pgfpathlineto{\pgfqpoint{4.404692in}{2.876832in}}%
\pgfpathlineto{\pgfqpoint{4.391223in}{2.882900in}}%
\pgfpathlineto{\pgfqpoint{4.383545in}{2.867111in}}%
\pgfpathlineto{\pgfqpoint{4.375864in}{2.851591in}}%
\pgfpathlineto{\pgfqpoint{4.368182in}{2.836335in}}%
\pgfpathlineto{\pgfqpoint{4.360498in}{2.821335in}}%
\pgfpathclose%
\pgfusepath{fill}%
\end{pgfscope}%
\begin{pgfscope}%
\pgfpathrectangle{\pgfqpoint{1.150000in}{0.150000in}}{\pgfqpoint{5.700000in}{5.700000in}}%
\pgfusepath{clip}%
\pgfsetbuttcap%
\pgfsetroundjoin%
\definecolor{currentfill}{rgb}{0.187231,0.414746,0.556547}%
\pgfsetfillcolor{currentfill}%
\pgfsetfillopacity{0.700000}%
\pgfsetlinewidth{0.000000pt}%
\definecolor{currentstroke}{rgb}{0.000000,0.000000,0.000000}%
\pgfsetstrokecolor{currentstroke}%
\pgfsetdash{}{0pt}%
\pgfpathmoveto{\pgfqpoint{4.984090in}{3.235427in}}%
\pgfpathlineto{\pgfqpoint{4.997653in}{3.228267in}}%
\pgfpathlineto{\pgfqpoint{5.011221in}{3.221178in}}%
\pgfpathlineto{\pgfqpoint{5.024794in}{3.214161in}}%
\pgfpathlineto{\pgfqpoint{5.038373in}{3.207215in}}%
\pgfpathlineto{\pgfqpoint{5.046053in}{3.230062in}}%
\pgfpathlineto{\pgfqpoint{5.053740in}{3.253394in}}%
\pgfpathlineto{\pgfqpoint{5.061436in}{3.277222in}}%
\pgfpathlineto{\pgfqpoint{5.069140in}{3.301555in}}%
\pgfpathlineto{\pgfqpoint{5.055570in}{3.309040in}}%
\pgfpathlineto{\pgfqpoint{5.042005in}{3.316597in}}%
\pgfpathlineto{\pgfqpoint{5.028445in}{3.324226in}}%
\pgfpathlineto{\pgfqpoint{5.014890in}{3.331927in}}%
\pgfpathlineto{\pgfqpoint{5.007178in}{3.307046in}}%
\pgfpathlineto{\pgfqpoint{4.999475in}{3.282676in}}%
\pgfpathlineto{\pgfqpoint{4.991779in}{3.258806in}}%
\pgfpathlineto{\pgfqpoint{4.984090in}{3.235427in}}%
\pgfpathclose%
\pgfusepath{fill}%
\end{pgfscope}%
\begin{pgfscope}%
\pgfpathrectangle{\pgfqpoint{1.150000in}{0.150000in}}{\pgfqpoint{5.700000in}{5.700000in}}%
\pgfusepath{clip}%
\pgfsetbuttcap%
\pgfsetroundjoin%
\definecolor{currentfill}{rgb}{0.214298,0.355619,0.551184}%
\pgfsetfillcolor{currentfill}%
\pgfsetfillopacity{0.700000}%
\pgfsetlinewidth{0.000000pt}%
\definecolor{currentstroke}{rgb}{0.000000,0.000000,0.000000}%
\pgfsetstrokecolor{currentstroke}%
\pgfsetdash{}{0pt}%
\pgfpathmoveto{\pgfqpoint{4.868545in}{3.090151in}}%
\pgfpathlineto{\pgfqpoint{4.882103in}{3.083718in}}%
\pgfpathlineto{\pgfqpoint{4.895666in}{3.077356in}}%
\pgfpathlineto{\pgfqpoint{4.909235in}{3.071067in}}%
\pgfpathlineto{\pgfqpoint{4.922810in}{3.064851in}}%
\pgfpathlineto{\pgfqpoint{4.930451in}{3.084663in}}%
\pgfpathlineto{\pgfqpoint{4.938097in}{3.104888in}}%
\pgfpathlineto{\pgfqpoint{4.945748in}{3.125535in}}%
\pgfpathlineto{\pgfqpoint{4.953405in}{3.146612in}}%
\pgfpathlineto{\pgfqpoint{4.939839in}{3.153326in}}%
\pgfpathlineto{\pgfqpoint{4.926279in}{3.160113in}}%
\pgfpathlineto{\pgfqpoint{4.912725in}{3.166972in}}%
\pgfpathlineto{\pgfqpoint{4.899176in}{3.173903in}}%
\pgfpathlineto{\pgfqpoint{4.891511in}{3.152320in}}%
\pgfpathlineto{\pgfqpoint{4.883851in}{3.131173in}}%
\pgfpathlineto{\pgfqpoint{4.876196in}{3.110454in}}%
\pgfpathlineto{\pgfqpoint{4.868545in}{3.090151in}}%
\pgfpathclose%
\pgfusepath{fill}%
\end{pgfscope}%
\begin{pgfscope}%
\pgfpathrectangle{\pgfqpoint{1.150000in}{0.150000in}}{\pgfqpoint{5.700000in}{5.700000in}}%
\pgfusepath{clip}%
\pgfsetbuttcap%
\pgfsetroundjoin%
\definecolor{currentfill}{rgb}{0.269308,0.218818,0.509577}%
\pgfsetfillcolor{currentfill}%
\pgfsetfillopacity{0.700000}%
\pgfsetlinewidth{0.000000pt}%
\definecolor{currentstroke}{rgb}{0.000000,0.000000,0.000000}%
\pgfsetstrokecolor{currentstroke}%
\pgfsetdash{}{0pt}%
\pgfpathmoveto{\pgfqpoint{4.275852in}{2.785899in}}%
\pgfpathlineto{\pgfqpoint{4.289316in}{2.780245in}}%
\pgfpathlineto{\pgfqpoint{4.302785in}{2.774670in}}%
\pgfpathlineto{\pgfqpoint{4.316259in}{2.769175in}}%
\pgfpathlineto{\pgfqpoint{4.329739in}{2.763760in}}%
\pgfpathlineto{\pgfqpoint{4.337432in}{2.777803in}}%
\pgfpathlineto{\pgfqpoint{4.345123in}{2.792075in}}%
\pgfpathlineto{\pgfqpoint{4.352812in}{2.806584in}}%
\pgfpathlineto{\pgfqpoint{4.360498in}{2.821335in}}%
\pgfpathlineto{\pgfqpoint{4.347027in}{2.827106in}}%
\pgfpathlineto{\pgfqpoint{4.333561in}{2.832956in}}%
\pgfpathlineto{\pgfqpoint{4.320100in}{2.838886in}}%
\pgfpathlineto{\pgfqpoint{4.306645in}{2.844896in}}%
\pgfpathlineto{\pgfqpoint{4.298950in}{2.829782in}}%
\pgfpathlineto{\pgfqpoint{4.291254in}{2.814915in}}%
\pgfpathlineto{\pgfqpoint{4.283554in}{2.800290in}}%
\pgfpathlineto{\pgfqpoint{4.275852in}{2.785899in}}%
\pgfpathclose%
\pgfusepath{fill}%
\end{pgfscope}%
\begin{pgfscope}%
\pgfpathrectangle{\pgfqpoint{1.150000in}{0.150000in}}{\pgfqpoint{5.700000in}{5.700000in}}%
\pgfusepath{clip}%
\pgfsetbuttcap%
\pgfsetroundjoin%
\definecolor{currentfill}{rgb}{0.279574,0.170599,0.479997}%
\pgfsetfillcolor{currentfill}%
\pgfsetfillopacity{0.700000}%
\pgfsetlinewidth{0.000000pt}%
\definecolor{currentstroke}{rgb}{0.000000,0.000000,0.000000}%
\pgfsetstrokecolor{currentstroke}%
\pgfsetdash{}{0pt}%
\pgfpathmoveto{\pgfqpoint{3.744925in}{2.689037in}}%
\pgfpathlineto{\pgfqpoint{3.758297in}{2.682464in}}%
\pgfpathlineto{\pgfqpoint{3.771673in}{2.675983in}}%
\pgfpathlineto{\pgfqpoint{3.785053in}{2.669593in}}%
\pgfpathlineto{\pgfqpoint{3.798437in}{2.663294in}}%
\pgfpathlineto{\pgfqpoint{3.806268in}{2.675711in}}%
\pgfpathlineto{\pgfqpoint{3.814094in}{2.688281in}}%
\pgfpathlineto{\pgfqpoint{3.821914in}{2.701009in}}%
\pgfpathlineto{\pgfqpoint{3.829730in}{2.713900in}}%
\pgfpathlineto{\pgfqpoint{3.816353in}{2.720455in}}%
\pgfpathlineto{\pgfqpoint{3.802980in}{2.727101in}}%
\pgfpathlineto{\pgfqpoint{3.789612in}{2.733838in}}%
\pgfpathlineto{\pgfqpoint{3.776247in}{2.740666in}}%
\pgfpathlineto{\pgfqpoint{3.768424in}{2.727512in}}%
\pgfpathlineto{\pgfqpoint{3.760596in}{2.714526in}}%
\pgfpathlineto{\pgfqpoint{3.752763in}{2.701702in}}%
\pgfpathlineto{\pgfqpoint{3.744925in}{2.689037in}}%
\pgfpathclose%
\pgfusepath{fill}%
\end{pgfscope}%
\begin{pgfscope}%
\pgfpathrectangle{\pgfqpoint{1.150000in}{0.150000in}}{\pgfqpoint{5.700000in}{5.700000in}}%
\pgfusepath{clip}%
\pgfsetbuttcap%
\pgfsetroundjoin%
\definecolor{currentfill}{rgb}{0.277134,0.185228,0.489898}%
\pgfsetfillcolor{currentfill}%
\pgfsetfillopacity{0.700000}%
\pgfsetlinewidth{0.000000pt}%
\definecolor{currentstroke}{rgb}{0.000000,0.000000,0.000000}%
\pgfsetstrokecolor{currentstroke}%
\pgfsetdash{}{0pt}%
\pgfpathmoveto{\pgfqpoint{3.968050in}{2.715748in}}%
\pgfpathlineto{\pgfqpoint{3.981458in}{2.709711in}}%
\pgfpathlineto{\pgfqpoint{3.994871in}{2.703760in}}%
\pgfpathlineto{\pgfqpoint{4.008289in}{2.697895in}}%
\pgfpathlineto{\pgfqpoint{4.021711in}{2.692115in}}%
\pgfpathlineto{\pgfqpoint{4.029480in}{2.705030in}}%
\pgfpathlineto{\pgfqpoint{4.037245in}{2.718124in}}%
\pgfpathlineto{\pgfqpoint{4.045006in}{2.731402in}}%
\pgfpathlineto{\pgfqpoint{4.052763in}{2.744870in}}%
\pgfpathlineto{\pgfqpoint{4.039348in}{2.750945in}}%
\pgfpathlineto{\pgfqpoint{4.025939in}{2.757106in}}%
\pgfpathlineto{\pgfqpoint{4.012533in}{2.763353in}}%
\pgfpathlineto{\pgfqpoint{3.999133in}{2.769685in}}%
\pgfpathlineto{\pgfqpoint{3.991368in}{2.755915in}}%
\pgfpathlineto{\pgfqpoint{3.983600in}{2.742339in}}%
\pgfpathlineto{\pgfqpoint{3.975827in}{2.728952in}}%
\pgfpathlineto{\pgfqpoint{3.968050in}{2.715748in}}%
\pgfpathclose%
\pgfusepath{fill}%
\end{pgfscope}%
\begin{pgfscope}%
\pgfpathrectangle{\pgfqpoint{1.150000in}{0.150000in}}{\pgfqpoint{5.700000in}{5.700000in}}%
\pgfusepath{clip}%
\pgfsetbuttcap%
\pgfsetroundjoin%
\definecolor{currentfill}{rgb}{0.269308,0.218818,0.509577}%
\pgfsetfillcolor{currentfill}%
\pgfsetfillopacity{0.700000}%
\pgfsetlinewidth{0.000000pt}%
\definecolor{currentstroke}{rgb}{0.000000,0.000000,0.000000}%
\pgfsetstrokecolor{currentstroke}%
\pgfsetdash{}{0pt}%
\pgfpathmoveto{\pgfqpoint{2.914126in}{2.803679in}}%
\pgfpathlineto{\pgfqpoint{2.927451in}{2.793035in}}%
\pgfpathlineto{\pgfqpoint{2.940775in}{2.782519in}}%
\pgfpathlineto{\pgfqpoint{2.954098in}{2.772131in}}%
\pgfpathlineto{\pgfqpoint{2.967422in}{2.761869in}}%
\pgfpathlineto{\pgfqpoint{2.975498in}{2.773348in}}%
\pgfpathlineto{\pgfqpoint{2.983566in}{2.784951in}}%
\pgfpathlineto{\pgfqpoint{2.991628in}{2.796683in}}%
\pgfpathlineto{\pgfqpoint{2.999682in}{2.808545in}}%
\pgfpathlineto{\pgfqpoint{2.986367in}{2.818943in}}%
\pgfpathlineto{\pgfqpoint{2.973052in}{2.829466in}}%
\pgfpathlineto{\pgfqpoint{2.959737in}{2.840118in}}%
\pgfpathlineto{\pgfqpoint{2.946422in}{2.850897in}}%
\pgfpathlineto{\pgfqpoint{2.938359in}{2.838892in}}%
\pgfpathlineto{\pgfqpoint{2.930289in}{2.827023in}}%
\pgfpathlineto{\pgfqpoint{2.922212in}{2.815286in}}%
\pgfpathlineto{\pgfqpoint{2.914126in}{2.803679in}}%
\pgfpathclose%
\pgfusepath{fill}%
\end{pgfscope}%
\begin{pgfscope}%
\pgfpathrectangle{\pgfqpoint{1.150000in}{0.150000in}}{\pgfqpoint{5.700000in}{5.700000in}}%
\pgfusepath{clip}%
\pgfsetbuttcap%
\pgfsetroundjoin%
\definecolor{currentfill}{rgb}{0.280255,0.165693,0.476498}%
\pgfsetfillcolor{currentfill}%
\pgfsetfillopacity{0.700000}%
\pgfsetlinewidth{0.000000pt}%
\definecolor{currentstroke}{rgb}{0.000000,0.000000,0.000000}%
\pgfsetstrokecolor{currentstroke}%
\pgfsetdash{}{0pt}%
\pgfpathmoveto{\pgfqpoint{3.383268in}{2.686374in}}%
\pgfpathlineto{\pgfqpoint{3.396598in}{2.678498in}}%
\pgfpathlineto{\pgfqpoint{3.409931in}{2.670725in}}%
\pgfpathlineto{\pgfqpoint{3.423266in}{2.663055in}}%
\pgfpathlineto{\pgfqpoint{3.436604in}{2.655487in}}%
\pgfpathlineto{\pgfqpoint{3.444541in}{2.667346in}}%
\pgfpathlineto{\pgfqpoint{3.452472in}{2.679332in}}%
\pgfpathlineto{\pgfqpoint{3.460396in}{2.691451in}}%
\pgfpathlineto{\pgfqpoint{3.468315in}{2.703705in}}%
\pgfpathlineto{\pgfqpoint{3.454984in}{2.711468in}}%
\pgfpathlineto{\pgfqpoint{3.441657in}{2.719334in}}%
\pgfpathlineto{\pgfqpoint{3.428331in}{2.727302in}}%
\pgfpathlineto{\pgfqpoint{3.415009in}{2.735374in}}%
\pgfpathlineto{\pgfqpoint{3.407083in}{2.722917in}}%
\pgfpathlineto{\pgfqpoint{3.399151in}{2.710601in}}%
\pgfpathlineto{\pgfqpoint{3.391213in}{2.698421in}}%
\pgfpathlineto{\pgfqpoint{3.383268in}{2.686374in}}%
\pgfpathclose%
\pgfusepath{fill}%
\end{pgfscope}%
\begin{pgfscope}%
\pgfpathrectangle{\pgfqpoint{1.150000in}{0.150000in}}{\pgfqpoint{5.700000in}{5.700000in}}%
\pgfusepath{clip}%
\pgfsetbuttcap%
\pgfsetroundjoin%
\definecolor{currentfill}{rgb}{0.278826,0.175490,0.483397}%
\pgfsetfillcolor{currentfill}%
\pgfsetfillopacity{0.700000}%
\pgfsetlinewidth{0.000000pt}%
\definecolor{currentstroke}{rgb}{0.000000,0.000000,0.000000}%
\pgfsetstrokecolor{currentstroke}%
\pgfsetdash{}{0pt}%
\pgfpathmoveto{\pgfqpoint{3.244802in}{2.704866in}}%
\pgfpathlineto{\pgfqpoint{3.258123in}{2.696310in}}%
\pgfpathlineto{\pgfqpoint{3.271446in}{2.687864in}}%
\pgfpathlineto{\pgfqpoint{3.284771in}{2.679526in}}%
\pgfpathlineto{\pgfqpoint{3.298098in}{2.671297in}}%
\pgfpathlineto{\pgfqpoint{3.306076in}{2.683016in}}%
\pgfpathlineto{\pgfqpoint{3.314047in}{2.694859in}}%
\pgfpathlineto{\pgfqpoint{3.322012in}{2.706829in}}%
\pgfpathlineto{\pgfqpoint{3.329971in}{2.718929in}}%
\pgfpathlineto{\pgfqpoint{3.316652in}{2.727334in}}%
\pgfpathlineto{\pgfqpoint{3.303335in}{2.735847in}}%
\pgfpathlineto{\pgfqpoint{3.290019in}{2.744469in}}%
\pgfpathlineto{\pgfqpoint{3.276706in}{2.753201in}}%
\pgfpathlineto{\pgfqpoint{3.268740in}{2.740917in}}%
\pgfpathlineto{\pgfqpoint{3.260767in}{2.728769in}}%
\pgfpathlineto{\pgfqpoint{3.252788in}{2.716753in}}%
\pgfpathlineto{\pgfqpoint{3.244802in}{2.704866in}}%
\pgfpathclose%
\pgfusepath{fill}%
\end{pgfscope}%
\begin{pgfscope}%
\pgfpathrectangle{\pgfqpoint{1.150000in}{0.150000in}}{\pgfqpoint{5.700000in}{5.700000in}}%
\pgfusepath{clip}%
\pgfsetbuttcap%
\pgfsetroundjoin%
\definecolor{currentfill}{rgb}{0.201239,0.383670,0.554294}%
\pgfsetfillcolor{currentfill}%
\pgfsetfillopacity{0.700000}%
\pgfsetlinewidth{0.000000pt}%
\definecolor{currentstroke}{rgb}{0.000000,0.000000,0.000000}%
\pgfsetstrokecolor{currentstroke}%
\pgfsetdash{}{0pt}%
\pgfpathmoveto{\pgfqpoint{4.953405in}{3.146612in}}%
\pgfpathlineto{\pgfqpoint{4.966976in}{3.139970in}}%
\pgfpathlineto{\pgfqpoint{4.980553in}{3.133400in}}%
\pgfpathlineto{\pgfqpoint{4.994135in}{3.126901in}}%
\pgfpathlineto{\pgfqpoint{5.007724in}{3.120474in}}%
\pgfpathlineto{\pgfqpoint{5.015377in}{3.141481in}}%
\pgfpathlineto{\pgfqpoint{5.023035in}{3.162934in}}%
\pgfpathlineto{\pgfqpoint{5.030701in}{3.184842in}}%
\pgfpathlineto{\pgfqpoint{5.038373in}{3.207215in}}%
\pgfpathlineto{\pgfqpoint{5.024794in}{3.214161in}}%
\pgfpathlineto{\pgfqpoint{5.011221in}{3.221178in}}%
\pgfpathlineto{\pgfqpoint{4.997653in}{3.228267in}}%
\pgfpathlineto{\pgfqpoint{4.984090in}{3.235427in}}%
\pgfpathlineto{\pgfqpoint{4.976409in}{3.212528in}}%
\pgfpathlineto{\pgfqpoint{4.968735in}{3.190099in}}%
\pgfpathlineto{\pgfqpoint{4.961067in}{3.168131in}}%
\pgfpathlineto{\pgfqpoint{4.953405in}{3.146612in}}%
\pgfpathclose%
\pgfusepath{fill}%
\end{pgfscope}%
\begin{pgfscope}%
\pgfpathrectangle{\pgfqpoint{1.150000in}{0.150000in}}{\pgfqpoint{5.700000in}{5.700000in}}%
\pgfusepath{clip}%
\pgfsetbuttcap%
\pgfsetroundjoin%
\definecolor{currentfill}{rgb}{0.273006,0.204520,0.501721}%
\pgfsetfillcolor{currentfill}%
\pgfsetfillopacity{0.700000}%
\pgfsetlinewidth{0.000000pt}%
\definecolor{currentstroke}{rgb}{0.000000,0.000000,0.000000}%
\pgfsetstrokecolor{currentstroke}%
\pgfsetdash{}{0pt}%
\pgfpathmoveto{\pgfqpoint{4.191179in}{2.752635in}}%
\pgfpathlineto{\pgfqpoint{4.204630in}{2.746993in}}%
\pgfpathlineto{\pgfqpoint{4.218087in}{2.741432in}}%
\pgfpathlineto{\pgfqpoint{4.231548in}{2.735952in}}%
\pgfpathlineto{\pgfqpoint{4.245015in}{2.730553in}}%
\pgfpathlineto{\pgfqpoint{4.252729in}{2.744069in}}%
\pgfpathlineto{\pgfqpoint{4.260440in}{2.757795in}}%
\pgfpathlineto{\pgfqpoint{4.268147in}{2.771736in}}%
\pgfpathlineto{\pgfqpoint{4.275852in}{2.785899in}}%
\pgfpathlineto{\pgfqpoint{4.262394in}{2.791634in}}%
\pgfpathlineto{\pgfqpoint{4.248941in}{2.797449in}}%
\pgfpathlineto{\pgfqpoint{4.235493in}{2.803346in}}%
\pgfpathlineto{\pgfqpoint{4.222050in}{2.809323in}}%
\pgfpathlineto{\pgfqpoint{4.214337in}{2.794817in}}%
\pgfpathlineto{\pgfqpoint{4.206621in}{2.780538in}}%
\pgfpathlineto{\pgfqpoint{4.198902in}{2.766480in}}%
\pgfpathlineto{\pgfqpoint{4.191179in}{2.752635in}}%
\pgfpathclose%
\pgfusepath{fill}%
\end{pgfscope}%
\begin{pgfscope}%
\pgfpathrectangle{\pgfqpoint{1.150000in}{0.150000in}}{\pgfqpoint{5.700000in}{5.700000in}}%
\pgfusepath{clip}%
\pgfsetbuttcap%
\pgfsetroundjoin%
\definecolor{currentfill}{rgb}{0.280868,0.160771,0.472899}%
\pgfsetfillcolor{currentfill}%
\pgfsetfillopacity{0.700000}%
\pgfsetlinewidth{0.000000pt}%
\definecolor{currentstroke}{rgb}{0.000000,0.000000,0.000000}%
\pgfsetstrokecolor{currentstroke}%
\pgfsetdash{}{0pt}%
\pgfpathmoveto{\pgfqpoint{3.521664in}{2.673659in}}%
\pgfpathlineto{\pgfqpoint{3.535009in}{2.666397in}}%
\pgfpathlineto{\pgfqpoint{3.548357in}{2.659233in}}%
\pgfpathlineto{\pgfqpoint{3.561708in}{2.652166in}}%
\pgfpathlineto{\pgfqpoint{3.575062in}{2.645197in}}%
\pgfpathlineto{\pgfqpoint{3.582960in}{2.657174in}}%
\pgfpathlineto{\pgfqpoint{3.590851in}{2.669284in}}%
\pgfpathlineto{\pgfqpoint{3.598737in}{2.681531in}}%
\pgfpathlineto{\pgfqpoint{3.606617in}{2.693920in}}%
\pgfpathlineto{\pgfqpoint{3.593270in}{2.701105in}}%
\pgfpathlineto{\pgfqpoint{3.579927in}{2.708387in}}%
\pgfpathlineto{\pgfqpoint{3.566586in}{2.715767in}}%
\pgfpathlineto{\pgfqpoint{3.553249in}{2.723245in}}%
\pgfpathlineto{\pgfqpoint{3.545361in}{2.710633in}}%
\pgfpathlineto{\pgfqpoint{3.537468in}{2.698168in}}%
\pgfpathlineto{\pgfqpoint{3.529569in}{2.685844in}}%
\pgfpathlineto{\pgfqpoint{3.521664in}{2.673659in}}%
\pgfpathclose%
\pgfusepath{fill}%
\end{pgfscope}%
\begin{pgfscope}%
\pgfpathrectangle{\pgfqpoint{1.150000in}{0.150000in}}{\pgfqpoint{5.700000in}{5.700000in}}%
\pgfusepath{clip}%
\pgfsetbuttcap%
\pgfsetroundjoin%
\definecolor{currentfill}{rgb}{0.276194,0.190074,0.493001}%
\pgfsetfillcolor{currentfill}%
\pgfsetfillopacity{0.700000}%
\pgfsetlinewidth{0.000000pt}%
\definecolor{currentstroke}{rgb}{0.000000,0.000000,0.000000}%
\pgfsetstrokecolor{currentstroke}%
\pgfsetdash{}{0pt}%
\pgfpathmoveto{\pgfqpoint{3.106208in}{2.729787in}}%
\pgfpathlineto{\pgfqpoint{3.119527in}{2.720481in}}%
\pgfpathlineto{\pgfqpoint{3.132847in}{2.711290in}}%
\pgfpathlineto{\pgfqpoint{3.146168in}{2.702215in}}%
\pgfpathlineto{\pgfqpoint{3.159490in}{2.693255in}}%
\pgfpathlineto{\pgfqpoint{3.167511in}{2.704807in}}%
\pgfpathlineto{\pgfqpoint{3.175525in}{2.716479in}}%
\pgfpathlineto{\pgfqpoint{3.183532in}{2.728276in}}%
\pgfpathlineto{\pgfqpoint{3.191533in}{2.740201in}}%
\pgfpathlineto{\pgfqpoint{3.178219in}{2.749316in}}%
\pgfpathlineto{\pgfqpoint{3.164907in}{2.758547in}}%
\pgfpathlineto{\pgfqpoint{3.151595in}{2.767893in}}%
\pgfpathlineto{\pgfqpoint{3.138285in}{2.777355in}}%
\pgfpathlineto{\pgfqpoint{3.130276in}{2.765268in}}%
\pgfpathlineto{\pgfqpoint{3.122261in}{2.753313in}}%
\pgfpathlineto{\pgfqpoint{3.114238in}{2.741487in}}%
\pgfpathlineto{\pgfqpoint{3.106208in}{2.729787in}}%
\pgfpathclose%
\pgfusepath{fill}%
\end{pgfscope}%
\begin{pgfscope}%
\pgfpathrectangle{\pgfqpoint{1.150000in}{0.150000in}}{\pgfqpoint{5.700000in}{5.700000in}}%
\pgfusepath{clip}%
\pgfsetbuttcap%
\pgfsetroundjoin%
\definecolor{currentfill}{rgb}{0.160665,0.478540,0.558115}%
\pgfsetfillcolor{currentfill}%
\pgfsetfillopacity{0.700000}%
\pgfsetlinewidth{0.000000pt}%
\definecolor{currentstroke}{rgb}{0.000000,0.000000,0.000000}%
\pgfsetstrokecolor{currentstroke}%
\pgfsetdash{}{0pt}%
\pgfpathmoveto{\pgfqpoint{5.100051in}{3.404156in}}%
\pgfpathlineto{\pgfqpoint{5.113619in}{3.396180in}}%
\pgfpathlineto{\pgfqpoint{5.127193in}{3.388276in}}%
\pgfpathlineto{\pgfqpoint{5.140771in}{3.380443in}}%
\pgfpathlineto{\pgfqpoint{5.154356in}{3.372680in}}%
\pgfpathlineto{\pgfqpoint{5.162101in}{3.399127in}}%
\pgfpathlineto{\pgfqpoint{5.169859in}{3.426139in}}%
\pgfpathlineto{\pgfqpoint{5.177628in}{3.453728in}}%
\pgfpathlineto{\pgfqpoint{5.164049in}{3.461926in}}%
\pgfpathlineto{\pgfqpoint{5.150476in}{3.470195in}}%
\pgfpathlineto{\pgfqpoint{5.136907in}{3.478535in}}%
\pgfpathlineto{\pgfqpoint{5.123344in}{3.486947in}}%
\pgfpathlineto{\pgfqpoint{5.115568in}{3.458771in}}%
\pgfpathlineto{\pgfqpoint{5.107804in}{3.431178in}}%
\pgfpathlineto{\pgfqpoint{5.100051in}{3.404156in}}%
\pgfpathclose%
\pgfusepath{fill}%
\end{pgfscope}%
\begin{pgfscope}%
\pgfpathrectangle{\pgfqpoint{1.150000in}{0.150000in}}{\pgfqpoint{5.700000in}{5.700000in}}%
\pgfusepath{clip}%
\pgfsetbuttcap%
\pgfsetroundjoin%
\definecolor{currentfill}{rgb}{0.174274,0.445044,0.557792}%
\pgfsetfillcolor{currentfill}%
\pgfsetfillopacity{0.700000}%
\pgfsetlinewidth{0.000000pt}%
\definecolor{currentstroke}{rgb}{0.000000,0.000000,0.000000}%
\pgfsetstrokecolor{currentstroke}%
\pgfsetdash{}{0pt}%
\pgfpathmoveto{\pgfqpoint{5.069140in}{3.301555in}}%
\pgfpathlineto{\pgfqpoint{5.082716in}{3.294141in}}%
\pgfpathlineto{\pgfqpoint{5.096298in}{3.286798in}}%
\pgfpathlineto{\pgfqpoint{5.109885in}{3.279526in}}%
\pgfpathlineto{\pgfqpoint{5.123477in}{3.272325in}}%
\pgfpathlineto{\pgfqpoint{5.131182in}{3.296621in}}%
\pgfpathlineto{\pgfqpoint{5.138896in}{3.321438in}}%
\pgfpathlineto{\pgfqpoint{5.146621in}{3.346788in}}%
\pgfpathlineto{\pgfqpoint{5.154356in}{3.372680in}}%
\pgfpathlineto{\pgfqpoint{5.140771in}{3.380443in}}%
\pgfpathlineto{\pgfqpoint{5.127193in}{3.388276in}}%
\pgfpathlineto{\pgfqpoint{5.113619in}{3.396180in}}%
\pgfpathlineto{\pgfqpoint{5.100051in}{3.404156in}}%
\pgfpathlineto{\pgfqpoint{5.092308in}{3.377694in}}%
\pgfpathlineto{\pgfqpoint{5.084576in}{3.351780in}}%
\pgfpathlineto{\pgfqpoint{5.076854in}{3.326404in}}%
\pgfpathlineto{\pgfqpoint{5.069140in}{3.301555in}}%
\pgfpathclose%
\pgfusepath{fill}%
\end{pgfscope}%
\begin{pgfscope}%
\pgfpathrectangle{\pgfqpoint{1.150000in}{0.150000in}}{\pgfqpoint{5.700000in}{5.700000in}}%
\pgfusepath{clip}%
\pgfsetbuttcap%
\pgfsetroundjoin%
\definecolor{currentfill}{rgb}{0.279574,0.170599,0.479997}%
\pgfsetfillcolor{currentfill}%
\pgfsetfillopacity{0.700000}%
\pgfsetlinewidth{0.000000pt}%
\definecolor{currentstroke}{rgb}{0.000000,0.000000,0.000000}%
\pgfsetstrokecolor{currentstroke}%
\pgfsetdash{}{0pt}%
\pgfpathmoveto{\pgfqpoint{3.883278in}{2.688576in}}%
\pgfpathlineto{\pgfqpoint{3.896676in}{2.682466in}}%
\pgfpathlineto{\pgfqpoint{3.910078in}{2.676445in}}%
\pgfpathlineto{\pgfqpoint{3.923485in}{2.670510in}}%
\pgfpathlineto{\pgfqpoint{3.936897in}{2.664663in}}%
\pgfpathlineto{\pgfqpoint{3.944692in}{2.677185in}}%
\pgfpathlineto{\pgfqpoint{3.952482in}{2.689870in}}%
\pgfpathlineto{\pgfqpoint{3.960268in}{2.702722in}}%
\pgfpathlineto{\pgfqpoint{3.968050in}{2.715748in}}%
\pgfpathlineto{\pgfqpoint{3.954646in}{2.721871in}}%
\pgfpathlineto{\pgfqpoint{3.941247in}{2.728081in}}%
\pgfpathlineto{\pgfqpoint{3.927853in}{2.734379in}}%
\pgfpathlineto{\pgfqpoint{3.914462in}{2.740764in}}%
\pgfpathlineto{\pgfqpoint{3.906673in}{2.727455in}}%
\pgfpathlineto{\pgfqpoint{3.898879in}{2.714324in}}%
\pgfpathlineto{\pgfqpoint{3.891081in}{2.701366in}}%
\pgfpathlineto{\pgfqpoint{3.883278in}{2.688576in}}%
\pgfpathclose%
\pgfusepath{fill}%
\end{pgfscope}%
\begin{pgfscope}%
\pgfpathrectangle{\pgfqpoint{1.150000in}{0.150000in}}{\pgfqpoint{5.700000in}{5.700000in}}%
\pgfusepath{clip}%
\pgfsetbuttcap%
\pgfsetroundjoin%
\definecolor{currentfill}{rgb}{0.246811,0.283237,0.535941}%
\pgfsetfillcolor{currentfill}%
\pgfsetfillopacity{0.700000}%
\pgfsetlinewidth{0.000000pt}%
\definecolor{currentstroke}{rgb}{0.000000,0.000000,0.000000}%
\pgfsetstrokecolor{currentstroke}%
\pgfsetdash{}{0pt}%
\pgfpathmoveto{\pgfqpoint{4.668496in}{2.919019in}}%
\pgfpathlineto{\pgfqpoint{4.682036in}{2.913366in}}%
\pgfpathlineto{\pgfqpoint{4.695581in}{2.907787in}}%
\pgfpathlineto{\pgfqpoint{4.709133in}{2.902282in}}%
\pgfpathlineto{\pgfqpoint{4.722690in}{2.896851in}}%
\pgfpathlineto{\pgfqpoint{4.730320in}{2.913256in}}%
\pgfpathlineto{\pgfqpoint{4.737951in}{2.929981in}}%
\pgfpathlineto{\pgfqpoint{4.745583in}{2.947033in}}%
\pgfpathlineto{\pgfqpoint{4.753216in}{2.964421in}}%
\pgfpathlineto{\pgfqpoint{4.739669in}{2.970288in}}%
\pgfpathlineto{\pgfqpoint{4.726127in}{2.976228in}}%
\pgfpathlineto{\pgfqpoint{4.712591in}{2.982243in}}%
\pgfpathlineto{\pgfqpoint{4.699061in}{2.988332in}}%
\pgfpathlineto{\pgfqpoint{4.691418in}{2.970501in}}%
\pgfpathlineto{\pgfqpoint{4.683776in}{2.953010in}}%
\pgfpathlineto{\pgfqpoint{4.676136in}{2.935852in}}%
\pgfpathlineto{\pgfqpoint{4.668496in}{2.919019in}}%
\pgfpathclose%
\pgfusepath{fill}%
\end{pgfscope}%
\begin{pgfscope}%
\pgfpathrectangle{\pgfqpoint{1.150000in}{0.150000in}}{\pgfqpoint{5.700000in}{5.700000in}}%
\pgfusepath{clip}%
\pgfsetbuttcap%
\pgfsetroundjoin%
\definecolor{currentfill}{rgb}{0.253935,0.265254,0.529983}%
\pgfsetfillcolor{currentfill}%
\pgfsetfillopacity{0.700000}%
\pgfsetlinewidth{0.000000pt}%
\definecolor{currentstroke}{rgb}{0.000000,0.000000,0.000000}%
\pgfsetstrokecolor{currentstroke}%
\pgfsetdash{}{0pt}%
\pgfpathmoveto{\pgfqpoint{4.583802in}{2.876472in}}%
\pgfpathlineto{\pgfqpoint{4.597328in}{2.870935in}}%
\pgfpathlineto{\pgfqpoint{4.610860in}{2.865473in}}%
\pgfpathlineto{\pgfqpoint{4.624398in}{2.860086in}}%
\pgfpathlineto{\pgfqpoint{4.637942in}{2.854774in}}%
\pgfpathlineto{\pgfqpoint{4.645580in}{2.870387in}}%
\pgfpathlineto{\pgfqpoint{4.653218in}{2.886294in}}%
\pgfpathlineto{\pgfqpoint{4.660857in}{2.902502in}}%
\pgfpathlineto{\pgfqpoint{4.668496in}{2.919019in}}%
\pgfpathlineto{\pgfqpoint{4.654962in}{2.924747in}}%
\pgfpathlineto{\pgfqpoint{4.641434in}{2.930549in}}%
\pgfpathlineto{\pgfqpoint{4.627912in}{2.936427in}}%
\pgfpathlineto{\pgfqpoint{4.614395in}{2.942380in}}%
\pgfpathlineto{\pgfqpoint{4.606747in}{2.925440in}}%
\pgfpathlineto{\pgfqpoint{4.599099in}{2.908814in}}%
\pgfpathlineto{\pgfqpoint{4.591451in}{2.892494in}}%
\pgfpathlineto{\pgfqpoint{4.583802in}{2.876472in}}%
\pgfpathclose%
\pgfusepath{fill}%
\end{pgfscope}%
\begin{pgfscope}%
\pgfpathrectangle{\pgfqpoint{1.150000in}{0.150000in}}{\pgfqpoint{5.700000in}{5.700000in}}%
\pgfusepath{clip}%
\pgfsetbuttcap%
\pgfsetroundjoin%
\definecolor{currentfill}{rgb}{0.237441,0.305202,0.541921}%
\pgfsetfillcolor{currentfill}%
\pgfsetfillopacity{0.700000}%
\pgfsetlinewidth{0.000000pt}%
\definecolor{currentstroke}{rgb}{0.000000,0.000000,0.000000}%
\pgfsetstrokecolor{currentstroke}%
\pgfsetdash{}{0pt}%
\pgfpathmoveto{\pgfqpoint{4.753216in}{2.964421in}}%
\pgfpathlineto{\pgfqpoint{4.766770in}{2.958627in}}%
\pgfpathlineto{\pgfqpoint{4.780329in}{2.952908in}}%
\pgfpathlineto{\pgfqpoint{4.793895in}{2.947261in}}%
\pgfpathlineto{\pgfqpoint{4.807466in}{2.941688in}}%
\pgfpathlineto{\pgfqpoint{4.815091in}{2.958971in}}%
\pgfpathlineto{\pgfqpoint{4.822719in}{2.976601in}}%
\pgfpathlineto{\pgfqpoint{4.830349in}{2.994587in}}%
\pgfpathlineto{\pgfqpoint{4.837981in}{3.012937in}}%
\pgfpathlineto{\pgfqpoint{4.824420in}{3.018966in}}%
\pgfpathlineto{\pgfqpoint{4.810865in}{3.025069in}}%
\pgfpathlineto{\pgfqpoint{4.797315in}{3.031245in}}%
\pgfpathlineto{\pgfqpoint{4.783771in}{3.037495in}}%
\pgfpathlineto{\pgfqpoint{4.776129in}{3.018681in}}%
\pgfpathlineto{\pgfqpoint{4.768489in}{3.000236in}}%
\pgfpathlineto{\pgfqpoint{4.760852in}{2.982152in}}%
\pgfpathlineto{\pgfqpoint{4.753216in}{2.964421in}}%
\pgfpathclose%
\pgfusepath{fill}%
\end{pgfscope}%
\begin{pgfscope}%
\pgfpathrectangle{\pgfqpoint{1.150000in}{0.150000in}}{\pgfqpoint{5.700000in}{5.700000in}}%
\pgfusepath{clip}%
\pgfsetbuttcap%
\pgfsetroundjoin%
\definecolor{currentfill}{rgb}{0.280868,0.160771,0.472899}%
\pgfsetfillcolor{currentfill}%
\pgfsetfillopacity{0.700000}%
\pgfsetlinewidth{0.000000pt}%
\definecolor{currentstroke}{rgb}{0.000000,0.000000,0.000000}%
\pgfsetstrokecolor{currentstroke}%
\pgfsetdash{}{0pt}%
\pgfpathmoveto{\pgfqpoint{3.660039in}{2.666139in}}%
\pgfpathlineto{\pgfqpoint{3.673404in}{2.659431in}}%
\pgfpathlineto{\pgfqpoint{3.686772in}{2.652816in}}%
\pgfpathlineto{\pgfqpoint{3.700144in}{2.646294in}}%
\pgfpathlineto{\pgfqpoint{3.713520in}{2.639865in}}%
\pgfpathlineto{\pgfqpoint{3.721379in}{2.651944in}}%
\pgfpathlineto{\pgfqpoint{3.729233in}{2.664162in}}%
\pgfpathlineto{\pgfqpoint{3.737082in}{2.676525in}}%
\pgfpathlineto{\pgfqpoint{3.744925in}{2.689037in}}%
\pgfpathlineto{\pgfqpoint{3.731557in}{2.695702in}}%
\pgfpathlineto{\pgfqpoint{3.718193in}{2.702459in}}%
\pgfpathlineto{\pgfqpoint{3.704832in}{2.709309in}}%
\pgfpathlineto{\pgfqpoint{3.691475in}{2.716253in}}%
\pgfpathlineto{\pgfqpoint{3.683624in}{2.703498in}}%
\pgfpathlineto{\pgfqpoint{3.675768in}{2.690897in}}%
\pgfpathlineto{\pgfqpoint{3.667906in}{2.678446in}}%
\pgfpathlineto{\pgfqpoint{3.660039in}{2.666139in}}%
\pgfpathclose%
\pgfusepath{fill}%
\end{pgfscope}%
\begin{pgfscope}%
\pgfpathrectangle{\pgfqpoint{1.150000in}{0.150000in}}{\pgfqpoint{5.700000in}{5.700000in}}%
\pgfusepath{clip}%
\pgfsetbuttcap%
\pgfsetroundjoin%
\definecolor{currentfill}{rgb}{0.260571,0.246922,0.522828}%
\pgfsetfillcolor{currentfill}%
\pgfsetfillopacity{0.700000}%
\pgfsetlinewidth{0.000000pt}%
\definecolor{currentstroke}{rgb}{0.000000,0.000000,0.000000}%
\pgfsetstrokecolor{currentstroke}%
\pgfsetdash{}{0pt}%
\pgfpathmoveto{\pgfqpoint{4.499121in}{2.836548in}}%
\pgfpathlineto{\pgfqpoint{4.512633in}{2.831103in}}%
\pgfpathlineto{\pgfqpoint{4.526151in}{2.825733in}}%
\pgfpathlineto{\pgfqpoint{4.539675in}{2.820440in}}%
\pgfpathlineto{\pgfqpoint{4.553205in}{2.815223in}}%
\pgfpathlineto{\pgfqpoint{4.560856in}{2.830124in}}%
\pgfpathlineto{\pgfqpoint{4.568505in}{2.845295in}}%
\pgfpathlineto{\pgfqpoint{4.576154in}{2.860742in}}%
\pgfpathlineto{\pgfqpoint{4.583802in}{2.876472in}}%
\pgfpathlineto{\pgfqpoint{4.570282in}{2.882085in}}%
\pgfpathlineto{\pgfqpoint{4.556768in}{2.887774in}}%
\pgfpathlineto{\pgfqpoint{4.543259in}{2.893539in}}%
\pgfpathlineto{\pgfqpoint{4.529756in}{2.899381in}}%
\pgfpathlineto{\pgfqpoint{4.522098in}{2.883247in}}%
\pgfpathlineto{\pgfqpoint{4.514440in}{2.867402in}}%
\pgfpathlineto{\pgfqpoint{4.506781in}{2.851838in}}%
\pgfpathlineto{\pgfqpoint{4.499121in}{2.836548in}}%
\pgfpathclose%
\pgfusepath{fill}%
\end{pgfscope}%
\begin{pgfscope}%
\pgfpathrectangle{\pgfqpoint{1.150000in}{0.150000in}}{\pgfqpoint{5.700000in}{5.700000in}}%
\pgfusepath{clip}%
\pgfsetbuttcap%
\pgfsetroundjoin%
\definecolor{currentfill}{rgb}{0.276194,0.190074,0.493001}%
\pgfsetfillcolor{currentfill}%
\pgfsetfillopacity{0.700000}%
\pgfsetlinewidth{0.000000pt}%
\definecolor{currentstroke}{rgb}{0.000000,0.000000,0.000000}%
\pgfsetstrokecolor{currentstroke}%
\pgfsetdash{}{0pt}%
\pgfpathmoveto{\pgfqpoint{4.106469in}{2.721411in}}%
\pgfpathlineto{\pgfqpoint{4.119908in}{2.715755in}}%
\pgfpathlineto{\pgfqpoint{4.133352in}{2.710182in}}%
\pgfpathlineto{\pgfqpoint{4.146801in}{2.704691in}}%
\pgfpathlineto{\pgfqpoint{4.160255in}{2.699283in}}%
\pgfpathlineto{\pgfqpoint{4.167992in}{2.712329in}}%
\pgfpathlineto{\pgfqpoint{4.175724in}{2.725566in}}%
\pgfpathlineto{\pgfqpoint{4.183453in}{2.738999in}}%
\pgfpathlineto{\pgfqpoint{4.191179in}{2.752635in}}%
\pgfpathlineto{\pgfqpoint{4.177733in}{2.758359in}}%
\pgfpathlineto{\pgfqpoint{4.164293in}{2.764166in}}%
\pgfpathlineto{\pgfqpoint{4.150857in}{2.770054in}}%
\pgfpathlineto{\pgfqpoint{4.137426in}{2.776025in}}%
\pgfpathlineto{\pgfqpoint{4.129692in}{2.762067in}}%
\pgfpathlineto{\pgfqpoint{4.121955in}{2.748315in}}%
\pgfpathlineto{\pgfqpoint{4.114214in}{2.734765in}}%
\pgfpathlineto{\pgfqpoint{4.106469in}{2.721411in}}%
\pgfpathclose%
\pgfusepath{fill}%
\end{pgfscope}%
\begin{pgfscope}%
\pgfpathrectangle{\pgfqpoint{1.150000in}{0.150000in}}{\pgfqpoint{5.700000in}{5.700000in}}%
\pgfusepath{clip}%
\pgfsetbuttcap%
\pgfsetroundjoin%
\definecolor{currentfill}{rgb}{0.227802,0.326594,0.546532}%
\pgfsetfillcolor{currentfill}%
\pgfsetfillopacity{0.700000}%
\pgfsetlinewidth{0.000000pt}%
\definecolor{currentstroke}{rgb}{0.000000,0.000000,0.000000}%
\pgfsetstrokecolor{currentstroke}%
\pgfsetdash{}{0pt}%
\pgfpathmoveto{\pgfqpoint{4.837981in}{3.012937in}}%
\pgfpathlineto{\pgfqpoint{4.851549in}{3.006980in}}%
\pgfpathlineto{\pgfqpoint{4.865122in}{3.001095in}}%
\pgfpathlineto{\pgfqpoint{4.878701in}{2.995283in}}%
\pgfpathlineto{\pgfqpoint{4.892286in}{2.989543in}}%
\pgfpathlineto{\pgfqpoint{4.899912in}{3.007796in}}%
\pgfpathlineto{\pgfqpoint{4.907541in}{3.026426in}}%
\pgfpathlineto{\pgfqpoint{4.915173in}{3.045441in}}%
\pgfpathlineto{\pgfqpoint{4.922810in}{3.064851in}}%
\pgfpathlineto{\pgfqpoint{4.909235in}{3.071067in}}%
\pgfpathlineto{\pgfqpoint{4.895666in}{3.077356in}}%
\pgfpathlineto{\pgfqpoint{4.882103in}{3.083718in}}%
\pgfpathlineto{\pgfqpoint{4.868545in}{3.090151in}}%
\pgfpathlineto{\pgfqpoint{4.860899in}{3.070258in}}%
\pgfpathlineto{\pgfqpoint{4.853256in}{3.050763in}}%
\pgfpathlineto{\pgfqpoint{4.845617in}{3.031659in}}%
\pgfpathlineto{\pgfqpoint{4.837981in}{3.012937in}}%
\pgfpathclose%
\pgfusepath{fill}%
\end{pgfscope}%
\begin{pgfscope}%
\pgfpathrectangle{\pgfqpoint{1.150000in}{0.150000in}}{\pgfqpoint{5.700000in}{5.700000in}}%
\pgfusepath{clip}%
\pgfsetbuttcap%
\pgfsetroundjoin%
\definecolor{currentfill}{rgb}{0.273006,0.204520,0.501721}%
\pgfsetfillcolor{currentfill}%
\pgfsetfillopacity{0.700000}%
\pgfsetlinewidth{0.000000pt}%
\definecolor{currentstroke}{rgb}{0.000000,0.000000,0.000000}%
\pgfsetstrokecolor{currentstroke}%
\pgfsetdash{}{0pt}%
\pgfpathmoveto{\pgfqpoint{2.967422in}{2.761869in}}%
\pgfpathlineto{\pgfqpoint{2.980745in}{2.751732in}}%
\pgfpathlineto{\pgfqpoint{2.994069in}{2.741719in}}%
\pgfpathlineto{\pgfqpoint{3.007393in}{2.731830in}}%
\pgfpathlineto{\pgfqpoint{3.020717in}{2.722063in}}%
\pgfpathlineto{\pgfqpoint{3.028784in}{2.733413in}}%
\pgfpathlineto{\pgfqpoint{3.036844in}{2.744884in}}%
\pgfpathlineto{\pgfqpoint{3.044896in}{2.756478in}}%
\pgfpathlineto{\pgfqpoint{3.052941in}{2.768197in}}%
\pgfpathlineto{\pgfqpoint{3.039626in}{2.778100in}}%
\pgfpathlineto{\pgfqpoint{3.026311in}{2.788125in}}%
\pgfpathlineto{\pgfqpoint{3.012996in}{2.798273in}}%
\pgfpathlineto{\pgfqpoint{2.999682in}{2.808545in}}%
\pgfpathlineto{\pgfqpoint{2.991628in}{2.796683in}}%
\pgfpathlineto{\pgfqpoint{2.983566in}{2.784951in}}%
\pgfpathlineto{\pgfqpoint{2.975498in}{2.773348in}}%
\pgfpathlineto{\pgfqpoint{2.967422in}{2.761869in}}%
\pgfpathclose%
\pgfusepath{fill}%
\end{pgfscope}%
\begin{pgfscope}%
\pgfpathrectangle{\pgfqpoint{1.150000in}{0.150000in}}{\pgfqpoint{5.700000in}{5.700000in}}%
\pgfusepath{clip}%
\pgfsetbuttcap%
\pgfsetroundjoin%
\definecolor{currentfill}{rgb}{0.188923,0.410910,0.556326}%
\pgfsetfillcolor{currentfill}%
\pgfsetfillopacity{0.700000}%
\pgfsetlinewidth{0.000000pt}%
\definecolor{currentstroke}{rgb}{0.000000,0.000000,0.000000}%
\pgfsetstrokecolor{currentstroke}%
\pgfsetdash{}{0pt}%
\pgfpathmoveto{\pgfqpoint{5.038373in}{3.207215in}}%
\pgfpathlineto{\pgfqpoint{5.051958in}{3.200340in}}%
\pgfpathlineto{\pgfqpoint{5.065548in}{3.193537in}}%
\pgfpathlineto{\pgfqpoint{5.079145in}{3.186805in}}%
\pgfpathlineto{\pgfqpoint{5.092747in}{3.180143in}}%
\pgfpathlineto{\pgfqpoint{5.100417in}{3.202459in}}%
\pgfpathlineto{\pgfqpoint{5.108095in}{3.225254in}}%
\pgfpathlineto{\pgfqpoint{5.115782in}{3.248539in}}%
\pgfpathlineto{\pgfqpoint{5.123477in}{3.272325in}}%
\pgfpathlineto{\pgfqpoint{5.109885in}{3.279526in}}%
\pgfpathlineto{\pgfqpoint{5.096298in}{3.286798in}}%
\pgfpathlineto{\pgfqpoint{5.082716in}{3.294141in}}%
\pgfpathlineto{\pgfqpoint{5.069140in}{3.301555in}}%
\pgfpathlineto{\pgfqpoint{5.061436in}{3.277222in}}%
\pgfpathlineto{\pgfqpoint{5.053740in}{3.253394in}}%
\pgfpathlineto{\pgfqpoint{5.046053in}{3.230062in}}%
\pgfpathlineto{\pgfqpoint{5.038373in}{3.207215in}}%
\pgfpathclose%
\pgfusepath{fill}%
\end{pgfscope}%
\begin{pgfscope}%
\pgfpathrectangle{\pgfqpoint{1.150000in}{0.150000in}}{\pgfqpoint{5.700000in}{5.700000in}}%
\pgfusepath{clip}%
\pgfsetbuttcap%
\pgfsetroundjoin%
\definecolor{currentfill}{rgb}{0.262138,0.242286,0.520837}%
\pgfsetfillcolor{currentfill}%
\pgfsetfillopacity{0.700000}%
\pgfsetlinewidth{0.000000pt}%
\definecolor{currentstroke}{rgb}{0.000000,0.000000,0.000000}%
\pgfsetstrokecolor{currentstroke}%
\pgfsetdash{}{0pt}%
\pgfpathmoveto{\pgfqpoint{2.775009in}{2.847496in}}%
\pgfpathlineto{\pgfqpoint{2.788351in}{2.835898in}}%
\pgfpathlineto{\pgfqpoint{2.801691in}{2.824437in}}%
\pgfpathlineto{\pgfqpoint{2.815029in}{2.813113in}}%
\pgfpathlineto{\pgfqpoint{2.828367in}{2.801925in}}%
\pgfpathlineto{\pgfqpoint{2.836493in}{2.813144in}}%
\pgfpathlineto{\pgfqpoint{2.844611in}{2.824489in}}%
\pgfpathlineto{\pgfqpoint{2.852721in}{2.835961in}}%
\pgfpathlineto{\pgfqpoint{2.860823in}{2.847563in}}%
\pgfpathlineto{\pgfqpoint{2.847496in}{2.858867in}}%
\pgfpathlineto{\pgfqpoint{2.834167in}{2.870306in}}%
\pgfpathlineto{\pgfqpoint{2.820836in}{2.881882in}}%
\pgfpathlineto{\pgfqpoint{2.807505in}{2.893596in}}%
\pgfpathlineto{\pgfqpoint{2.799393in}{2.881871in}}%
\pgfpathlineto{\pgfqpoint{2.791273in}{2.870281in}}%
\pgfpathlineto{\pgfqpoint{2.783145in}{2.858823in}}%
\pgfpathlineto{\pgfqpoint{2.775009in}{2.847496in}}%
\pgfpathclose%
\pgfusepath{fill}%
\end{pgfscope}%
\begin{pgfscope}%
\pgfpathrectangle{\pgfqpoint{1.150000in}{0.150000in}}{\pgfqpoint{5.700000in}{5.700000in}}%
\pgfusepath{clip}%
\pgfsetbuttcap%
\pgfsetroundjoin%
\definecolor{currentfill}{rgb}{0.266580,0.228262,0.514349}%
\pgfsetfillcolor{currentfill}%
\pgfsetfillopacity{0.700000}%
\pgfsetlinewidth{0.000000pt}%
\definecolor{currentstroke}{rgb}{0.000000,0.000000,0.000000}%
\pgfsetstrokecolor{currentstroke}%
\pgfsetdash{}{0pt}%
\pgfpathmoveto{\pgfqpoint{4.414437in}{2.799039in}}%
\pgfpathlineto{\pgfqpoint{4.427936in}{2.793660in}}%
\pgfpathlineto{\pgfqpoint{4.441440in}{2.788359in}}%
\pgfpathlineto{\pgfqpoint{4.454951in}{2.783135in}}%
\pgfpathlineto{\pgfqpoint{4.468467in}{2.777989in}}%
\pgfpathlineto{\pgfqpoint{4.476133in}{2.792253in}}%
\pgfpathlineto{\pgfqpoint{4.483797in}{2.806763in}}%
\pgfpathlineto{\pgfqpoint{4.491460in}{2.821526in}}%
\pgfpathlineto{\pgfqpoint{4.499121in}{2.836548in}}%
\pgfpathlineto{\pgfqpoint{4.485614in}{2.842071in}}%
\pgfpathlineto{\pgfqpoint{4.472113in}{2.847670in}}%
\pgfpathlineto{\pgfqpoint{4.458618in}{2.853347in}}%
\pgfpathlineto{\pgfqpoint{4.445128in}{2.859101in}}%
\pgfpathlineto{\pgfqpoint{4.437458in}{2.843695in}}%
\pgfpathlineto{\pgfqpoint{4.429786in}{2.828554in}}%
\pgfpathlineto{\pgfqpoint{4.422113in}{2.813671in}}%
\pgfpathlineto{\pgfqpoint{4.414437in}{2.799039in}}%
\pgfpathclose%
\pgfusepath{fill}%
\end{pgfscope}%
\begin{pgfscope}%
\pgfpathrectangle{\pgfqpoint{1.150000in}{0.150000in}}{\pgfqpoint{5.700000in}{5.700000in}}%
\pgfusepath{clip}%
\pgfsetbuttcap%
\pgfsetroundjoin%
\definecolor{currentfill}{rgb}{0.216210,0.351535,0.550627}%
\pgfsetfillcolor{currentfill}%
\pgfsetfillopacity{0.700000}%
\pgfsetlinewidth{0.000000pt}%
\definecolor{currentstroke}{rgb}{0.000000,0.000000,0.000000}%
\pgfsetstrokecolor{currentstroke}%
\pgfsetdash{}{0pt}%
\pgfpathmoveto{\pgfqpoint{4.922810in}{3.064851in}}%
\pgfpathlineto{\pgfqpoint{4.936391in}{3.058706in}}%
\pgfpathlineto{\pgfqpoint{4.949978in}{3.052633in}}%
\pgfpathlineto{\pgfqpoint{4.963571in}{3.046631in}}%
\pgfpathlineto{\pgfqpoint{4.977170in}{3.040701in}}%
\pgfpathlineto{\pgfqpoint{4.984801in}{3.060024in}}%
\pgfpathlineto{\pgfqpoint{4.992437in}{3.079754in}}%
\pgfpathlineto{\pgfqpoint{5.000077in}{3.099901in}}%
\pgfpathlineto{\pgfqpoint{5.007724in}{3.120474in}}%
\pgfpathlineto{\pgfqpoint{4.994135in}{3.126901in}}%
\pgfpathlineto{\pgfqpoint{4.980553in}{3.133400in}}%
\pgfpathlineto{\pgfqpoint{4.966976in}{3.139970in}}%
\pgfpathlineto{\pgfqpoint{4.953405in}{3.146612in}}%
\pgfpathlineto{\pgfqpoint{4.945748in}{3.125535in}}%
\pgfpathlineto{\pgfqpoint{4.938097in}{3.104888in}}%
\pgfpathlineto{\pgfqpoint{4.930451in}{3.084663in}}%
\pgfpathlineto{\pgfqpoint{4.922810in}{3.064851in}}%
\pgfpathclose%
\pgfusepath{fill}%
\end{pgfscope}%
\begin{pgfscope}%
\pgfpathrectangle{\pgfqpoint{1.150000in}{0.150000in}}{\pgfqpoint{5.700000in}{5.700000in}}%
\pgfusepath{clip}%
\pgfsetbuttcap%
\pgfsetroundjoin%
\definecolor{currentfill}{rgb}{0.280255,0.165693,0.476498}%
\pgfsetfillcolor{currentfill}%
\pgfsetfillopacity{0.700000}%
\pgfsetlinewidth{0.000000pt}%
\definecolor{currentstroke}{rgb}{0.000000,0.000000,0.000000}%
\pgfsetstrokecolor{currentstroke}%
\pgfsetdash{}{0pt}%
\pgfpathmoveto{\pgfqpoint{3.298098in}{2.671297in}}%
\pgfpathlineto{\pgfqpoint{3.311427in}{2.663175in}}%
\pgfpathlineto{\pgfqpoint{3.324758in}{2.655159in}}%
\pgfpathlineto{\pgfqpoint{3.338092in}{2.647249in}}%
\pgfpathlineto{\pgfqpoint{3.351428in}{2.639444in}}%
\pgfpathlineto{\pgfqpoint{3.359397in}{2.650995in}}%
\pgfpathlineto{\pgfqpoint{3.367361in}{2.662664in}}%
\pgfpathlineto{\pgfqpoint{3.375318in}{2.674456in}}%
\pgfpathlineto{\pgfqpoint{3.383268in}{2.686374in}}%
\pgfpathlineto{\pgfqpoint{3.369940in}{2.694355in}}%
\pgfpathlineto{\pgfqpoint{3.356615in}{2.702440in}}%
\pgfpathlineto{\pgfqpoint{3.343292in}{2.710632in}}%
\pgfpathlineto{\pgfqpoint{3.329971in}{2.718929in}}%
\pgfpathlineto{\pgfqpoint{3.322012in}{2.706829in}}%
\pgfpathlineto{\pgfqpoint{3.314047in}{2.694859in}}%
\pgfpathlineto{\pgfqpoint{3.306076in}{2.683016in}}%
\pgfpathlineto{\pgfqpoint{3.298098in}{2.671297in}}%
\pgfpathclose%
\pgfusepath{fill}%
\end{pgfscope}%
\begin{pgfscope}%
\pgfpathrectangle{\pgfqpoint{1.150000in}{0.150000in}}{\pgfqpoint{5.700000in}{5.700000in}}%
\pgfusepath{clip}%
\pgfsetbuttcap%
\pgfsetroundjoin%
\definecolor{currentfill}{rgb}{0.280868,0.160771,0.472899}%
\pgfsetfillcolor{currentfill}%
\pgfsetfillopacity{0.700000}%
\pgfsetlinewidth{0.000000pt}%
\definecolor{currentstroke}{rgb}{0.000000,0.000000,0.000000}%
\pgfsetstrokecolor{currentstroke}%
\pgfsetdash{}{0pt}%
\pgfpathmoveto{\pgfqpoint{3.798437in}{2.663294in}}%
\pgfpathlineto{\pgfqpoint{3.811826in}{2.657084in}}%
\pgfpathlineto{\pgfqpoint{3.825218in}{2.650965in}}%
\pgfpathlineto{\pgfqpoint{3.838615in}{2.644934in}}%
\pgfpathlineto{\pgfqpoint{3.852017in}{2.638992in}}%
\pgfpathlineto{\pgfqpoint{3.859840in}{2.651161in}}%
\pgfpathlineto{\pgfqpoint{3.867657in}{2.663478in}}%
\pgfpathlineto{\pgfqpoint{3.875470in}{2.675948in}}%
\pgfpathlineto{\pgfqpoint{3.883278in}{2.688576in}}%
\pgfpathlineto{\pgfqpoint{3.869884in}{2.694774in}}%
\pgfpathlineto{\pgfqpoint{3.856495in}{2.701060in}}%
\pgfpathlineto{\pgfqpoint{3.843110in}{2.707435in}}%
\pgfpathlineto{\pgfqpoint{3.829730in}{2.713900in}}%
\pgfpathlineto{\pgfqpoint{3.821914in}{2.701009in}}%
\pgfpathlineto{\pgfqpoint{3.814094in}{2.688281in}}%
\pgfpathlineto{\pgfqpoint{3.806268in}{2.675711in}}%
\pgfpathlineto{\pgfqpoint{3.798437in}{2.663294in}}%
\pgfpathclose%
\pgfusepath{fill}%
\end{pgfscope}%
\begin{pgfscope}%
\pgfpathrectangle{\pgfqpoint{1.150000in}{0.150000in}}{\pgfqpoint{5.700000in}{5.700000in}}%
\pgfusepath{clip}%
\pgfsetbuttcap%
\pgfsetroundjoin%
\definecolor{currentfill}{rgb}{0.281412,0.155834,0.469201}%
\pgfsetfillcolor{currentfill}%
\pgfsetfillopacity{0.700000}%
\pgfsetlinewidth{0.000000pt}%
\definecolor{currentstroke}{rgb}{0.000000,0.000000,0.000000}%
\pgfsetstrokecolor{currentstroke}%
\pgfsetdash{}{0pt}%
\pgfpathmoveto{\pgfqpoint{3.436604in}{2.655487in}}%
\pgfpathlineto{\pgfqpoint{3.449945in}{2.648021in}}%
\pgfpathlineto{\pgfqpoint{3.463289in}{2.640655in}}%
\pgfpathlineto{\pgfqpoint{3.476635in}{2.633390in}}%
\pgfpathlineto{\pgfqpoint{3.489985in}{2.626224in}}%
\pgfpathlineto{\pgfqpoint{3.497914in}{2.637895in}}%
\pgfpathlineto{\pgfqpoint{3.505836in}{2.649688in}}%
\pgfpathlineto{\pgfqpoint{3.513753in}{2.661609in}}%
\pgfpathlineto{\pgfqpoint{3.521664in}{2.673659in}}%
\pgfpathlineto{\pgfqpoint{3.508322in}{2.681021in}}%
\pgfpathlineto{\pgfqpoint{3.494983in}{2.688482in}}%
\pgfpathlineto{\pgfqpoint{3.481648in}{2.696043in}}%
\pgfpathlineto{\pgfqpoint{3.468315in}{2.703705in}}%
\pgfpathlineto{\pgfqpoint{3.460396in}{2.691451in}}%
\pgfpathlineto{\pgfqpoint{3.452472in}{2.679332in}}%
\pgfpathlineto{\pgfqpoint{3.444541in}{2.667346in}}%
\pgfpathlineto{\pgfqpoint{3.436604in}{2.655487in}}%
\pgfpathclose%
\pgfusepath{fill}%
\end{pgfscope}%
\begin{pgfscope}%
\pgfpathrectangle{\pgfqpoint{1.150000in}{0.150000in}}{\pgfqpoint{5.700000in}{5.700000in}}%
\pgfusepath{clip}%
\pgfsetbuttcap%
\pgfsetroundjoin%
\definecolor{currentfill}{rgb}{0.278826,0.175490,0.483397}%
\pgfsetfillcolor{currentfill}%
\pgfsetfillopacity{0.700000}%
\pgfsetlinewidth{0.000000pt}%
\definecolor{currentstroke}{rgb}{0.000000,0.000000,0.000000}%
\pgfsetstrokecolor{currentstroke}%
\pgfsetdash{}{0pt}%
\pgfpathmoveto{\pgfqpoint{3.159490in}{2.693255in}}%
\pgfpathlineto{\pgfqpoint{3.172813in}{2.684409in}}%
\pgfpathlineto{\pgfqpoint{3.186138in}{2.675675in}}%
\pgfpathlineto{\pgfqpoint{3.199465in}{2.667053in}}%
\pgfpathlineto{\pgfqpoint{3.212792in}{2.658543in}}%
\pgfpathlineto{\pgfqpoint{3.220805in}{2.669946in}}%
\pgfpathlineto{\pgfqpoint{3.228811in}{2.681466in}}%
\pgfpathlineto{\pgfqpoint{3.236810in}{2.693104in}}%
\pgfpathlineto{\pgfqpoint{3.244802in}{2.704866in}}%
\pgfpathlineto{\pgfqpoint{3.231483in}{2.713532in}}%
\pgfpathlineto{\pgfqpoint{3.218165in}{2.722309in}}%
\pgfpathlineto{\pgfqpoint{3.204848in}{2.731198in}}%
\pgfpathlineto{\pgfqpoint{3.191533in}{2.740201in}}%
\pgfpathlineto{\pgfqpoint{3.183532in}{2.728276in}}%
\pgfpathlineto{\pgfqpoint{3.175525in}{2.716479in}}%
\pgfpathlineto{\pgfqpoint{3.167511in}{2.704807in}}%
\pgfpathlineto{\pgfqpoint{3.159490in}{2.693255in}}%
\pgfpathclose%
\pgfusepath{fill}%
\end{pgfscope}%
\begin{pgfscope}%
\pgfpathrectangle{\pgfqpoint{1.150000in}{0.150000in}}{\pgfqpoint{5.700000in}{5.700000in}}%
\pgfusepath{clip}%
\pgfsetbuttcap%
\pgfsetroundjoin%
\definecolor{currentfill}{rgb}{0.278012,0.180367,0.486697}%
\pgfsetfillcolor{currentfill}%
\pgfsetfillopacity{0.700000}%
\pgfsetlinewidth{0.000000pt}%
\definecolor{currentstroke}{rgb}{0.000000,0.000000,0.000000}%
\pgfsetstrokecolor{currentstroke}%
\pgfsetdash{}{0pt}%
\pgfpathmoveto{\pgfqpoint{4.021711in}{2.692115in}}%
\pgfpathlineto{\pgfqpoint{4.035138in}{2.686420in}}%
\pgfpathlineto{\pgfqpoint{4.048570in}{2.680809in}}%
\pgfpathlineto{\pgfqpoint{4.062007in}{2.675282in}}%
\pgfpathlineto{\pgfqpoint{4.075450in}{2.669839in}}%
\pgfpathlineto{\pgfqpoint{4.083211in}{2.682466in}}%
\pgfpathlineto{\pgfqpoint{4.090967in}{2.695267in}}%
\pgfpathlineto{\pgfqpoint{4.098720in}{2.708246in}}%
\pgfpathlineto{\pgfqpoint{4.106469in}{2.721411in}}%
\pgfpathlineto{\pgfqpoint{4.093035in}{2.727150in}}%
\pgfpathlineto{\pgfqpoint{4.079606in}{2.732972in}}%
\pgfpathlineto{\pgfqpoint{4.066182in}{2.738879in}}%
\pgfpathlineto{\pgfqpoint{4.052763in}{2.744870in}}%
\pgfpathlineto{\pgfqpoint{4.045006in}{2.731402in}}%
\pgfpathlineto{\pgfqpoint{4.037245in}{2.718124in}}%
\pgfpathlineto{\pgfqpoint{4.029480in}{2.705030in}}%
\pgfpathlineto{\pgfqpoint{4.021711in}{2.692115in}}%
\pgfpathclose%
\pgfusepath{fill}%
\end{pgfscope}%
\begin{pgfscope}%
\pgfpathrectangle{\pgfqpoint{1.150000in}{0.150000in}}{\pgfqpoint{5.700000in}{5.700000in}}%
\pgfusepath{clip}%
\pgfsetbuttcap%
\pgfsetroundjoin%
\definecolor{currentfill}{rgb}{0.270595,0.214069,0.507052}%
\pgfsetfillcolor{currentfill}%
\pgfsetfillopacity{0.700000}%
\pgfsetlinewidth{0.000000pt}%
\definecolor{currentstroke}{rgb}{0.000000,0.000000,0.000000}%
\pgfsetstrokecolor{currentstroke}%
\pgfsetdash{}{0pt}%
\pgfpathmoveto{\pgfqpoint{4.329739in}{2.763760in}}%
\pgfpathlineto{\pgfqpoint{4.343224in}{2.758424in}}%
\pgfpathlineto{\pgfqpoint{4.356715in}{2.753167in}}%
\pgfpathlineto{\pgfqpoint{4.370212in}{2.747988in}}%
\pgfpathlineto{\pgfqpoint{4.383715in}{2.742887in}}%
\pgfpathlineto{\pgfqpoint{4.391399in}{2.756581in}}%
\pgfpathlineto{\pgfqpoint{4.399080in}{2.770501in}}%
\pgfpathlineto{\pgfqpoint{4.406760in}{2.784651in}}%
\pgfpathlineto{\pgfqpoint{4.414437in}{2.799039in}}%
\pgfpathlineto{\pgfqpoint{4.400944in}{2.804495in}}%
\pgfpathlineto{\pgfqpoint{4.387457in}{2.810030in}}%
\pgfpathlineto{\pgfqpoint{4.373975in}{2.815643in}}%
\pgfpathlineto{\pgfqpoint{4.360498in}{2.821335in}}%
\pgfpathlineto{\pgfqpoint{4.352812in}{2.806584in}}%
\pgfpathlineto{\pgfqpoint{4.345123in}{2.792075in}}%
\pgfpathlineto{\pgfqpoint{4.337432in}{2.777803in}}%
\pgfpathlineto{\pgfqpoint{4.329739in}{2.763760in}}%
\pgfpathclose%
\pgfusepath{fill}%
\end{pgfscope}%
\begin{pgfscope}%
\pgfpathrectangle{\pgfqpoint{1.150000in}{0.150000in}}{\pgfqpoint{5.700000in}{5.700000in}}%
\pgfusepath{clip}%
\pgfsetbuttcap%
\pgfsetroundjoin%
\definecolor{currentfill}{rgb}{0.163625,0.471133,0.558148}%
\pgfsetfillcolor{currentfill}%
\pgfsetfillopacity{0.700000}%
\pgfsetlinewidth{0.000000pt}%
\definecolor{currentstroke}{rgb}{0.000000,0.000000,0.000000}%
\pgfsetstrokecolor{currentstroke}%
\pgfsetdash{}{0pt}%
\pgfpathmoveto{\pgfqpoint{5.154356in}{3.372680in}}%
\pgfpathlineto{\pgfqpoint{5.167945in}{3.364988in}}%
\pgfpathlineto{\pgfqpoint{5.181540in}{3.357367in}}%
\pgfpathlineto{\pgfqpoint{5.195141in}{3.349816in}}%
\pgfpathlineto{\pgfqpoint{5.208747in}{3.342335in}}%
\pgfpathlineto{\pgfqpoint{5.216484in}{3.368207in}}%
\pgfpathlineto{\pgfqpoint{5.224233in}{3.394639in}}%
\pgfpathlineto{\pgfqpoint{5.231995in}{3.421642in}}%
\pgfpathlineto{\pgfqpoint{5.218395in}{3.429558in}}%
\pgfpathlineto{\pgfqpoint{5.204801in}{3.437544in}}%
\pgfpathlineto{\pgfqpoint{5.191211in}{3.445601in}}%
\pgfpathlineto{\pgfqpoint{5.177628in}{3.453728in}}%
\pgfpathlineto{\pgfqpoint{5.169859in}{3.426139in}}%
\pgfpathlineto{\pgfqpoint{5.162101in}{3.399127in}}%
\pgfpathlineto{\pgfqpoint{5.154356in}{3.372680in}}%
\pgfpathclose%
\pgfusepath{fill}%
\end{pgfscope}%
\begin{pgfscope}%
\pgfpathrectangle{\pgfqpoint{1.150000in}{0.150000in}}{\pgfqpoint{5.700000in}{5.700000in}}%
\pgfusepath{clip}%
\pgfsetbuttcap%
\pgfsetroundjoin%
\definecolor{currentfill}{rgb}{0.204903,0.375746,0.553533}%
\pgfsetfillcolor{currentfill}%
\pgfsetfillopacity{0.700000}%
\pgfsetlinewidth{0.000000pt}%
\definecolor{currentstroke}{rgb}{0.000000,0.000000,0.000000}%
\pgfsetstrokecolor{currentstroke}%
\pgfsetdash{}{0pt}%
\pgfpathmoveto{\pgfqpoint{5.007724in}{3.120474in}}%
\pgfpathlineto{\pgfqpoint{5.021319in}{3.114117in}}%
\pgfpathlineto{\pgfqpoint{5.034919in}{3.107832in}}%
\pgfpathlineto{\pgfqpoint{5.048526in}{3.101618in}}%
\pgfpathlineto{\pgfqpoint{5.062138in}{3.095474in}}%
\pgfpathlineto{\pgfqpoint{5.069780in}{3.115972in}}%
\pgfpathlineto{\pgfqpoint{5.077429in}{3.136909in}}%
\pgfpathlineto{\pgfqpoint{5.085084in}{3.158296in}}%
\pgfpathlineto{\pgfqpoint{5.092747in}{3.180143in}}%
\pgfpathlineto{\pgfqpoint{5.079145in}{3.186805in}}%
\pgfpathlineto{\pgfqpoint{5.065548in}{3.193537in}}%
\pgfpathlineto{\pgfqpoint{5.051958in}{3.200340in}}%
\pgfpathlineto{\pgfqpoint{5.038373in}{3.207215in}}%
\pgfpathlineto{\pgfqpoint{5.030701in}{3.184842in}}%
\pgfpathlineto{\pgfqpoint{5.023035in}{3.162934in}}%
\pgfpathlineto{\pgfqpoint{5.015377in}{3.141481in}}%
\pgfpathlineto{\pgfqpoint{5.007724in}{3.120474in}}%
\pgfpathclose%
\pgfusepath{fill}%
\end{pgfscope}%
\begin{pgfscope}%
\pgfpathrectangle{\pgfqpoint{1.150000in}{0.150000in}}{\pgfqpoint{5.700000in}{5.700000in}}%
\pgfusepath{clip}%
\pgfsetbuttcap%
\pgfsetroundjoin%
\definecolor{currentfill}{rgb}{0.281412,0.155834,0.469201}%
\pgfsetfillcolor{currentfill}%
\pgfsetfillopacity{0.700000}%
\pgfsetlinewidth{0.000000pt}%
\definecolor{currentstroke}{rgb}{0.000000,0.000000,0.000000}%
\pgfsetstrokecolor{currentstroke}%
\pgfsetdash{}{0pt}%
\pgfpathmoveto{\pgfqpoint{3.575062in}{2.645197in}}%
\pgfpathlineto{\pgfqpoint{3.588420in}{2.638324in}}%
\pgfpathlineto{\pgfqpoint{3.601782in}{2.631547in}}%
\pgfpathlineto{\pgfqpoint{3.615147in}{2.624865in}}%
\pgfpathlineto{\pgfqpoint{3.628515in}{2.618278in}}%
\pgfpathlineto{\pgfqpoint{3.636405in}{2.630047in}}%
\pgfpathlineto{\pgfqpoint{3.644288in}{2.641944in}}%
\pgfpathlineto{\pgfqpoint{3.652167in}{2.653973in}}%
\pgfpathlineto{\pgfqpoint{3.660039in}{2.666139in}}%
\pgfpathlineto{\pgfqpoint{3.646678in}{2.672942in}}%
\pgfpathlineto{\pgfqpoint{3.633321in}{2.679839in}}%
\pgfpathlineto{\pgfqpoint{3.619968in}{2.686832in}}%
\pgfpathlineto{\pgfqpoint{3.606617in}{2.693920in}}%
\pgfpathlineto{\pgfqpoint{3.598737in}{2.681531in}}%
\pgfpathlineto{\pgfqpoint{3.590851in}{2.669284in}}%
\pgfpathlineto{\pgfqpoint{3.582960in}{2.657174in}}%
\pgfpathlineto{\pgfqpoint{3.575062in}{2.645197in}}%
\pgfpathclose%
\pgfusepath{fill}%
\end{pgfscope}%
\begin{pgfscope}%
\pgfpathrectangle{\pgfqpoint{1.150000in}{0.150000in}}{\pgfqpoint{5.700000in}{5.700000in}}%
\pgfusepath{clip}%
\pgfsetbuttcap%
\pgfsetroundjoin%
\definecolor{currentfill}{rgb}{0.266580,0.228262,0.514349}%
\pgfsetfillcolor{currentfill}%
\pgfsetfillopacity{0.700000}%
\pgfsetlinewidth{0.000000pt}%
\definecolor{currentstroke}{rgb}{0.000000,0.000000,0.000000}%
\pgfsetstrokecolor{currentstroke}%
\pgfsetdash{}{0pt}%
\pgfpathmoveto{\pgfqpoint{2.828367in}{2.801925in}}%
\pgfpathlineto{\pgfqpoint{2.841704in}{2.790871in}}%
\pgfpathlineto{\pgfqpoint{2.855040in}{2.779949in}}%
\pgfpathlineto{\pgfqpoint{2.868375in}{2.769160in}}%
\pgfpathlineto{\pgfqpoint{2.881709in}{2.758502in}}%
\pgfpathlineto{\pgfqpoint{2.889825in}{2.769613in}}%
\pgfpathlineto{\pgfqpoint{2.897933in}{2.780845in}}%
\pgfpathlineto{\pgfqpoint{2.906034in}{2.792200in}}%
\pgfpathlineto{\pgfqpoint{2.914126in}{2.803679in}}%
\pgfpathlineto{\pgfqpoint{2.900802in}{2.814453in}}%
\pgfpathlineto{\pgfqpoint{2.887476in}{2.825357in}}%
\pgfpathlineto{\pgfqpoint{2.874150in}{2.836394in}}%
\pgfpathlineto{\pgfqpoint{2.860823in}{2.847563in}}%
\pgfpathlineto{\pgfqpoint{2.852721in}{2.835961in}}%
\pgfpathlineto{\pgfqpoint{2.844611in}{2.824489in}}%
\pgfpathlineto{\pgfqpoint{2.836493in}{2.813144in}}%
\pgfpathlineto{\pgfqpoint{2.828367in}{2.801925in}}%
\pgfpathclose%
\pgfusepath{fill}%
\end{pgfscope}%
\begin{pgfscope}%
\pgfpathrectangle{\pgfqpoint{1.150000in}{0.150000in}}{\pgfqpoint{5.700000in}{5.700000in}}%
\pgfusepath{clip}%
\pgfsetbuttcap%
\pgfsetroundjoin%
\definecolor{currentfill}{rgb}{0.177423,0.437527,0.557565}%
\pgfsetfillcolor{currentfill}%
\pgfsetfillopacity{0.700000}%
\pgfsetlinewidth{0.000000pt}%
\definecolor{currentstroke}{rgb}{0.000000,0.000000,0.000000}%
\pgfsetstrokecolor{currentstroke}%
\pgfsetdash{}{0pt}%
\pgfpathmoveto{\pgfqpoint{5.123477in}{3.272325in}}%
\pgfpathlineto{\pgfqpoint{5.137076in}{3.265194in}}%
\pgfpathlineto{\pgfqpoint{5.150680in}{3.258134in}}%
\pgfpathlineto{\pgfqpoint{5.164289in}{3.251144in}}%
\pgfpathlineto{\pgfqpoint{5.177905in}{3.244224in}}%
\pgfpathlineto{\pgfqpoint{5.185600in}{3.267967in}}%
\pgfpathlineto{\pgfqpoint{5.193305in}{3.292226in}}%
\pgfpathlineto{\pgfqpoint{5.201021in}{3.317012in}}%
\pgfpathlineto{\pgfqpoint{5.208747in}{3.342335in}}%
\pgfpathlineto{\pgfqpoint{5.195141in}{3.349816in}}%
\pgfpathlineto{\pgfqpoint{5.181540in}{3.357367in}}%
\pgfpathlineto{\pgfqpoint{5.167945in}{3.364988in}}%
\pgfpathlineto{\pgfqpoint{5.154356in}{3.372680in}}%
\pgfpathlineto{\pgfqpoint{5.146621in}{3.346788in}}%
\pgfpathlineto{\pgfqpoint{5.138896in}{3.321438in}}%
\pgfpathlineto{\pgfqpoint{5.131182in}{3.296621in}}%
\pgfpathlineto{\pgfqpoint{5.123477in}{3.272325in}}%
\pgfpathclose%
\pgfusepath{fill}%
\end{pgfscope}%
\begin{pgfscope}%
\pgfpathrectangle{\pgfqpoint{1.150000in}{0.150000in}}{\pgfqpoint{5.700000in}{5.700000in}}%
\pgfusepath{clip}%
\pgfsetbuttcap%
\pgfsetroundjoin%
\definecolor{currentfill}{rgb}{0.276194,0.190074,0.493001}%
\pgfsetfillcolor{currentfill}%
\pgfsetfillopacity{0.700000}%
\pgfsetlinewidth{0.000000pt}%
\definecolor{currentstroke}{rgb}{0.000000,0.000000,0.000000}%
\pgfsetstrokecolor{currentstroke}%
\pgfsetdash{}{0pt}%
\pgfpathmoveto{\pgfqpoint{3.020717in}{2.722063in}}%
\pgfpathlineto{\pgfqpoint{3.034042in}{2.712416in}}%
\pgfpathlineto{\pgfqpoint{3.047367in}{2.702890in}}%
\pgfpathlineto{\pgfqpoint{3.060693in}{2.693483in}}%
\pgfpathlineto{\pgfqpoint{3.074020in}{2.684195in}}%
\pgfpathlineto{\pgfqpoint{3.082078in}{2.695418in}}%
\pgfpathlineto{\pgfqpoint{3.090128in}{2.706756in}}%
\pgfpathlineto{\pgfqpoint{3.098172in}{2.718211in}}%
\pgfpathlineto{\pgfqpoint{3.106208in}{2.729787in}}%
\pgfpathlineto{\pgfqpoint{3.092891in}{2.739212in}}%
\pgfpathlineto{\pgfqpoint{3.079574in}{2.748754in}}%
\pgfpathlineto{\pgfqpoint{3.066257in}{2.758415in}}%
\pgfpathlineto{\pgfqpoint{3.052941in}{2.768197in}}%
\pgfpathlineto{\pgfqpoint{3.044896in}{2.756478in}}%
\pgfpathlineto{\pgfqpoint{3.036844in}{2.744884in}}%
\pgfpathlineto{\pgfqpoint{3.028784in}{2.733413in}}%
\pgfpathlineto{\pgfqpoint{3.020717in}{2.722063in}}%
\pgfpathclose%
\pgfusepath{fill}%
\end{pgfscope}%
\begin{pgfscope}%
\pgfpathrectangle{\pgfqpoint{1.150000in}{0.150000in}}{\pgfqpoint{5.700000in}{5.700000in}}%
\pgfusepath{clip}%
\pgfsetbuttcap%
\pgfsetroundjoin%
\definecolor{currentfill}{rgb}{0.274128,0.199721,0.498911}%
\pgfsetfillcolor{currentfill}%
\pgfsetfillopacity{0.700000}%
\pgfsetlinewidth{0.000000pt}%
\definecolor{currentstroke}{rgb}{0.000000,0.000000,0.000000}%
\pgfsetstrokecolor{currentstroke}%
\pgfsetdash{}{0pt}%
\pgfpathmoveto{\pgfqpoint{4.245015in}{2.730553in}}%
\pgfpathlineto{\pgfqpoint{4.258487in}{2.725234in}}%
\pgfpathlineto{\pgfqpoint{4.271965in}{2.719996in}}%
\pgfpathlineto{\pgfqpoint{4.285449in}{2.714837in}}%
\pgfpathlineto{\pgfqpoint{4.298938in}{2.709757in}}%
\pgfpathlineto{\pgfqpoint{4.306642in}{2.722945in}}%
\pgfpathlineto{\pgfqpoint{4.314344in}{2.736337in}}%
\pgfpathlineto{\pgfqpoint{4.322043in}{2.749940in}}%
\pgfpathlineto{\pgfqpoint{4.329739in}{2.763760in}}%
\pgfpathlineto{\pgfqpoint{4.316259in}{2.769175in}}%
\pgfpathlineto{\pgfqpoint{4.302785in}{2.774670in}}%
\pgfpathlineto{\pgfqpoint{4.289316in}{2.780245in}}%
\pgfpathlineto{\pgfqpoint{4.275852in}{2.785899in}}%
\pgfpathlineto{\pgfqpoint{4.268147in}{2.771736in}}%
\pgfpathlineto{\pgfqpoint{4.260440in}{2.757795in}}%
\pgfpathlineto{\pgfqpoint{4.252729in}{2.744069in}}%
\pgfpathlineto{\pgfqpoint{4.245015in}{2.730553in}}%
\pgfpathclose%
\pgfusepath{fill}%
\end{pgfscope}%
\begin{pgfscope}%
\pgfpathrectangle{\pgfqpoint{1.150000in}{0.150000in}}{\pgfqpoint{5.700000in}{5.700000in}}%
\pgfusepath{clip}%
\pgfsetbuttcap%
\pgfsetroundjoin%
\definecolor{currentfill}{rgb}{0.280255,0.165693,0.476498}%
\pgfsetfillcolor{currentfill}%
\pgfsetfillopacity{0.700000}%
\pgfsetlinewidth{0.000000pt}%
\definecolor{currentstroke}{rgb}{0.000000,0.000000,0.000000}%
\pgfsetstrokecolor{currentstroke}%
\pgfsetdash{}{0pt}%
\pgfpathmoveto{\pgfqpoint{3.936897in}{2.664663in}}%
\pgfpathlineto{\pgfqpoint{3.950313in}{2.658902in}}%
\pgfpathlineto{\pgfqpoint{3.963734in}{2.653226in}}%
\pgfpathlineto{\pgfqpoint{3.977160in}{2.647637in}}%
\pgfpathlineto{\pgfqpoint{3.990590in}{2.642133in}}%
\pgfpathlineto{\pgfqpoint{3.998377in}{2.654387in}}%
\pgfpathlineto{\pgfqpoint{4.006159in}{2.666798in}}%
\pgfpathlineto{\pgfqpoint{4.013937in}{2.679373in}}%
\pgfpathlineto{\pgfqpoint{4.021711in}{2.692115in}}%
\pgfpathlineto{\pgfqpoint{4.008289in}{2.697895in}}%
\pgfpathlineto{\pgfqpoint{3.994871in}{2.703760in}}%
\pgfpathlineto{\pgfqpoint{3.981458in}{2.709711in}}%
\pgfpathlineto{\pgfqpoint{3.968050in}{2.715748in}}%
\pgfpathlineto{\pgfqpoint{3.960268in}{2.702722in}}%
\pgfpathlineto{\pgfqpoint{3.952482in}{2.689870in}}%
\pgfpathlineto{\pgfqpoint{3.944692in}{2.677185in}}%
\pgfpathlineto{\pgfqpoint{3.936897in}{2.664663in}}%
\pgfpathclose%
\pgfusepath{fill}%
\end{pgfscope}%
\begin{pgfscope}%
\pgfpathrectangle{\pgfqpoint{1.150000in}{0.150000in}}{\pgfqpoint{5.700000in}{5.700000in}}%
\pgfusepath{clip}%
\pgfsetbuttcap%
\pgfsetroundjoin%
\definecolor{currentfill}{rgb}{0.246811,0.283237,0.535941}%
\pgfsetfillcolor{currentfill}%
\pgfsetfillopacity{0.700000}%
\pgfsetlinewidth{0.000000pt}%
\definecolor{currentstroke}{rgb}{0.000000,0.000000,0.000000}%
\pgfsetstrokecolor{currentstroke}%
\pgfsetdash{}{0pt}%
\pgfpathmoveto{\pgfqpoint{4.722690in}{2.896851in}}%
\pgfpathlineto{\pgfqpoint{4.736254in}{2.891494in}}%
\pgfpathlineto{\pgfqpoint{4.749824in}{2.886210in}}%
\pgfpathlineto{\pgfqpoint{4.763400in}{2.881000in}}%
\pgfpathlineto{\pgfqpoint{4.776982in}{2.875863in}}%
\pgfpathlineto{\pgfqpoint{4.784601in}{2.891839in}}%
\pgfpathlineto{\pgfqpoint{4.792221in}{2.908130in}}%
\pgfpathlineto{\pgfqpoint{4.799843in}{2.924744in}}%
\pgfpathlineto{\pgfqpoint{4.807466in}{2.941688in}}%
\pgfpathlineto{\pgfqpoint{4.793895in}{2.947261in}}%
\pgfpathlineto{\pgfqpoint{4.780329in}{2.952908in}}%
\pgfpathlineto{\pgfqpoint{4.766770in}{2.958627in}}%
\pgfpathlineto{\pgfqpoint{4.753216in}{2.964421in}}%
\pgfpathlineto{\pgfqpoint{4.745583in}{2.947033in}}%
\pgfpathlineto{\pgfqpoint{4.737951in}{2.929981in}}%
\pgfpathlineto{\pgfqpoint{4.730320in}{2.913256in}}%
\pgfpathlineto{\pgfqpoint{4.722690in}{2.896851in}}%
\pgfpathclose%
\pgfusepath{fill}%
\end{pgfscope}%
\begin{pgfscope}%
\pgfpathrectangle{\pgfqpoint{1.150000in}{0.150000in}}{\pgfqpoint{5.700000in}{5.700000in}}%
\pgfusepath{clip}%
\pgfsetbuttcap%
\pgfsetroundjoin%
\definecolor{currentfill}{rgb}{0.281412,0.155834,0.469201}%
\pgfsetfillcolor{currentfill}%
\pgfsetfillopacity{0.700000}%
\pgfsetlinewidth{0.000000pt}%
\definecolor{currentstroke}{rgb}{0.000000,0.000000,0.000000}%
\pgfsetstrokecolor{currentstroke}%
\pgfsetdash{}{0pt}%
\pgfpathmoveto{\pgfqpoint{3.713520in}{2.639865in}}%
\pgfpathlineto{\pgfqpoint{3.726899in}{2.633528in}}%
\pgfpathlineto{\pgfqpoint{3.740283in}{2.627283in}}%
\pgfpathlineto{\pgfqpoint{3.753671in}{2.621129in}}%
\pgfpathlineto{\pgfqpoint{3.767063in}{2.615065in}}%
\pgfpathlineto{\pgfqpoint{3.774915in}{2.626915in}}%
\pgfpathlineto{\pgfqpoint{3.782761in}{2.638900in}}%
\pgfpathlineto{\pgfqpoint{3.790602in}{2.651025in}}%
\pgfpathlineto{\pgfqpoint{3.798437in}{2.663294in}}%
\pgfpathlineto{\pgfqpoint{3.785053in}{2.669593in}}%
\pgfpathlineto{\pgfqpoint{3.771673in}{2.675983in}}%
\pgfpathlineto{\pgfqpoint{3.758297in}{2.682464in}}%
\pgfpathlineto{\pgfqpoint{3.744925in}{2.689037in}}%
\pgfpathlineto{\pgfqpoint{3.737082in}{2.676525in}}%
\pgfpathlineto{\pgfqpoint{3.729233in}{2.664162in}}%
\pgfpathlineto{\pgfqpoint{3.721379in}{2.651944in}}%
\pgfpathlineto{\pgfqpoint{3.713520in}{2.639865in}}%
\pgfpathclose%
\pgfusepath{fill}%
\end{pgfscope}%
\begin{pgfscope}%
\pgfpathrectangle{\pgfqpoint{1.150000in}{0.150000in}}{\pgfqpoint{5.700000in}{5.700000in}}%
\pgfusepath{clip}%
\pgfsetbuttcap%
\pgfsetroundjoin%
\definecolor{currentfill}{rgb}{0.239346,0.300855,0.540844}%
\pgfsetfillcolor{currentfill}%
\pgfsetfillopacity{0.700000}%
\pgfsetlinewidth{0.000000pt}%
\definecolor{currentstroke}{rgb}{0.000000,0.000000,0.000000}%
\pgfsetstrokecolor{currentstroke}%
\pgfsetdash{}{0pt}%
\pgfpathmoveto{\pgfqpoint{4.807466in}{2.941688in}}%
\pgfpathlineto{\pgfqpoint{4.821044in}{2.936187in}}%
\pgfpathlineto{\pgfqpoint{4.834628in}{2.930759in}}%
\pgfpathlineto{\pgfqpoint{4.848218in}{2.925403in}}%
\pgfpathlineto{\pgfqpoint{4.861814in}{2.920120in}}%
\pgfpathlineto{\pgfqpoint{4.869428in}{2.936954in}}%
\pgfpathlineto{\pgfqpoint{4.877045in}{2.954130in}}%
\pgfpathlineto{\pgfqpoint{4.884664in}{2.971657in}}%
\pgfpathlineto{\pgfqpoint{4.892286in}{2.989543in}}%
\pgfpathlineto{\pgfqpoint{4.878701in}{2.995283in}}%
\pgfpathlineto{\pgfqpoint{4.865122in}{3.001095in}}%
\pgfpathlineto{\pgfqpoint{4.851549in}{3.006980in}}%
\pgfpathlineto{\pgfqpoint{4.837981in}{3.012937in}}%
\pgfpathlineto{\pgfqpoint{4.830349in}{2.994587in}}%
\pgfpathlineto{\pgfqpoint{4.822719in}{2.976601in}}%
\pgfpathlineto{\pgfqpoint{4.815091in}{2.958971in}}%
\pgfpathlineto{\pgfqpoint{4.807466in}{2.941688in}}%
\pgfpathclose%
\pgfusepath{fill}%
\end{pgfscope}%
\begin{pgfscope}%
\pgfpathrectangle{\pgfqpoint{1.150000in}{0.150000in}}{\pgfqpoint{5.700000in}{5.700000in}}%
\pgfusepath{clip}%
\pgfsetbuttcap%
\pgfsetroundjoin%
\definecolor{currentfill}{rgb}{0.255645,0.260703,0.528312}%
\pgfsetfillcolor{currentfill}%
\pgfsetfillopacity{0.700000}%
\pgfsetlinewidth{0.000000pt}%
\definecolor{currentstroke}{rgb}{0.000000,0.000000,0.000000}%
\pgfsetstrokecolor{currentstroke}%
\pgfsetdash{}{0pt}%
\pgfpathmoveto{\pgfqpoint{4.637942in}{2.854774in}}%
\pgfpathlineto{\pgfqpoint{4.651491in}{2.849537in}}%
\pgfpathlineto{\pgfqpoint{4.665047in}{2.844374in}}%
\pgfpathlineto{\pgfqpoint{4.678609in}{2.839285in}}%
\pgfpathlineto{\pgfqpoint{4.692177in}{2.834270in}}%
\pgfpathlineto{\pgfqpoint{4.699805in}{2.849475in}}%
\pgfpathlineto{\pgfqpoint{4.707433in}{2.864968in}}%
\pgfpathlineto{\pgfqpoint{4.715061in}{2.880758in}}%
\pgfpathlineto{\pgfqpoint{4.722690in}{2.896851in}}%
\pgfpathlineto{\pgfqpoint{4.709133in}{2.902282in}}%
\pgfpathlineto{\pgfqpoint{4.695581in}{2.907787in}}%
\pgfpathlineto{\pgfqpoint{4.682036in}{2.913366in}}%
\pgfpathlineto{\pgfqpoint{4.668496in}{2.919019in}}%
\pgfpathlineto{\pgfqpoint{4.660857in}{2.902502in}}%
\pgfpathlineto{\pgfqpoint{4.653218in}{2.886294in}}%
\pgfpathlineto{\pgfqpoint{4.645580in}{2.870387in}}%
\pgfpathlineto{\pgfqpoint{4.637942in}{2.854774in}}%
\pgfpathclose%
\pgfusepath{fill}%
\end{pgfscope}%
\begin{pgfscope}%
\pgfpathrectangle{\pgfqpoint{1.150000in}{0.150000in}}{\pgfqpoint{5.700000in}{5.700000in}}%
\pgfusepath{clip}%
\pgfsetbuttcap%
\pgfsetroundjoin%
\definecolor{currentfill}{rgb}{0.192357,0.403199,0.555836}%
\pgfsetfillcolor{currentfill}%
\pgfsetfillopacity{0.700000}%
\pgfsetlinewidth{0.000000pt}%
\definecolor{currentstroke}{rgb}{0.000000,0.000000,0.000000}%
\pgfsetstrokecolor{currentstroke}%
\pgfsetdash{}{0pt}%
\pgfpathmoveto{\pgfqpoint{5.092747in}{3.180143in}}%
\pgfpathlineto{\pgfqpoint{5.106355in}{3.173551in}}%
\pgfpathlineto{\pgfqpoint{5.119969in}{3.167030in}}%
\pgfpathlineto{\pgfqpoint{5.133589in}{3.160580in}}%
\pgfpathlineto{\pgfqpoint{5.147215in}{3.154199in}}%
\pgfpathlineto{\pgfqpoint{5.154875in}{3.175983in}}%
\pgfpathlineto{\pgfqpoint{5.162543in}{3.198242in}}%
\pgfpathlineto{\pgfqpoint{5.170219in}{3.220986in}}%
\pgfpathlineto{\pgfqpoint{5.177905in}{3.244224in}}%
\pgfpathlineto{\pgfqpoint{5.164289in}{3.251144in}}%
\pgfpathlineto{\pgfqpoint{5.150680in}{3.258134in}}%
\pgfpathlineto{\pgfqpoint{5.137076in}{3.265194in}}%
\pgfpathlineto{\pgfqpoint{5.123477in}{3.272325in}}%
\pgfpathlineto{\pgfqpoint{5.115782in}{3.248539in}}%
\pgfpathlineto{\pgfqpoint{5.108095in}{3.225254in}}%
\pgfpathlineto{\pgfqpoint{5.100417in}{3.202459in}}%
\pgfpathlineto{\pgfqpoint{5.092747in}{3.180143in}}%
\pgfpathclose%
\pgfusepath{fill}%
\end{pgfscope}%
\begin{pgfscope}%
\pgfpathrectangle{\pgfqpoint{1.150000in}{0.150000in}}{\pgfqpoint{5.700000in}{5.700000in}}%
\pgfusepath{clip}%
\pgfsetbuttcap%
\pgfsetroundjoin%
\definecolor{currentfill}{rgb}{0.229739,0.322361,0.545706}%
\pgfsetfillcolor{currentfill}%
\pgfsetfillopacity{0.700000}%
\pgfsetlinewidth{0.000000pt}%
\definecolor{currentstroke}{rgb}{0.000000,0.000000,0.000000}%
\pgfsetstrokecolor{currentstroke}%
\pgfsetdash{}{0pt}%
\pgfpathmoveto{\pgfqpoint{4.892286in}{2.989543in}}%
\pgfpathlineto{\pgfqpoint{4.905878in}{2.983875in}}%
\pgfpathlineto{\pgfqpoint{4.919475in}{2.978279in}}%
\pgfpathlineto{\pgfqpoint{4.933079in}{2.972754in}}%
\pgfpathlineto{\pgfqpoint{4.946690in}{2.967301in}}%
\pgfpathlineto{\pgfqpoint{4.954304in}{2.985085in}}%
\pgfpathlineto{\pgfqpoint{4.961922in}{3.003241in}}%
\pgfpathlineto{\pgfqpoint{4.969544in}{3.021777in}}%
\pgfpathlineto{\pgfqpoint{4.977170in}{3.040701in}}%
\pgfpathlineto{\pgfqpoint{4.963571in}{3.046631in}}%
\pgfpathlineto{\pgfqpoint{4.949978in}{3.052633in}}%
\pgfpathlineto{\pgfqpoint{4.936391in}{3.058706in}}%
\pgfpathlineto{\pgfqpoint{4.922810in}{3.064851in}}%
\pgfpathlineto{\pgfqpoint{4.915173in}{3.045441in}}%
\pgfpathlineto{\pgfqpoint{4.907541in}{3.026426in}}%
\pgfpathlineto{\pgfqpoint{4.899912in}{3.007796in}}%
\pgfpathlineto{\pgfqpoint{4.892286in}{2.989543in}}%
\pgfpathclose%
\pgfusepath{fill}%
\end{pgfscope}%
\begin{pgfscope}%
\pgfpathrectangle{\pgfqpoint{1.150000in}{0.150000in}}{\pgfqpoint{5.700000in}{5.700000in}}%
\pgfusepath{clip}%
\pgfsetbuttcap%
\pgfsetroundjoin%
\definecolor{currentfill}{rgb}{0.277134,0.185228,0.489898}%
\pgfsetfillcolor{currentfill}%
\pgfsetfillopacity{0.700000}%
\pgfsetlinewidth{0.000000pt}%
\definecolor{currentstroke}{rgb}{0.000000,0.000000,0.000000}%
\pgfsetstrokecolor{currentstroke}%
\pgfsetdash{}{0pt}%
\pgfpathmoveto{\pgfqpoint{4.160255in}{2.699283in}}%
\pgfpathlineto{\pgfqpoint{4.173715in}{2.693956in}}%
\pgfpathlineto{\pgfqpoint{4.187180in}{2.688711in}}%
\pgfpathlineto{\pgfqpoint{4.200650in}{2.683547in}}%
\pgfpathlineto{\pgfqpoint{4.214126in}{2.678464in}}%
\pgfpathlineto{\pgfqpoint{4.221854in}{2.691202in}}%
\pgfpathlineto{\pgfqpoint{4.229577in}{2.704125in}}%
\pgfpathlineto{\pgfqpoint{4.237298in}{2.717240in}}%
\pgfpathlineto{\pgfqpoint{4.245015in}{2.730553in}}%
\pgfpathlineto{\pgfqpoint{4.231548in}{2.735952in}}%
\pgfpathlineto{\pgfqpoint{4.218087in}{2.741432in}}%
\pgfpathlineto{\pgfqpoint{4.204630in}{2.746993in}}%
\pgfpathlineto{\pgfqpoint{4.191179in}{2.752635in}}%
\pgfpathlineto{\pgfqpoint{4.183453in}{2.738999in}}%
\pgfpathlineto{\pgfqpoint{4.175724in}{2.725566in}}%
\pgfpathlineto{\pgfqpoint{4.167992in}{2.712329in}}%
\pgfpathlineto{\pgfqpoint{4.160255in}{2.699283in}}%
\pgfpathclose%
\pgfusepath{fill}%
\end{pgfscope}%
\begin{pgfscope}%
\pgfpathrectangle{\pgfqpoint{1.150000in}{0.150000in}}{\pgfqpoint{5.700000in}{5.700000in}}%
\pgfusepath{clip}%
\pgfsetbuttcap%
\pgfsetroundjoin%
\definecolor{currentfill}{rgb}{0.262138,0.242286,0.520837}%
\pgfsetfillcolor{currentfill}%
\pgfsetfillopacity{0.700000}%
\pgfsetlinewidth{0.000000pt}%
\definecolor{currentstroke}{rgb}{0.000000,0.000000,0.000000}%
\pgfsetstrokecolor{currentstroke}%
\pgfsetdash{}{0pt}%
\pgfpathmoveto{\pgfqpoint{4.553205in}{2.815223in}}%
\pgfpathlineto{\pgfqpoint{4.566741in}{2.810081in}}%
\pgfpathlineto{\pgfqpoint{4.580283in}{2.805015in}}%
\pgfpathlineto{\pgfqpoint{4.593831in}{2.800024in}}%
\pgfpathlineto{\pgfqpoint{4.607385in}{2.795108in}}%
\pgfpathlineto{\pgfqpoint{4.615025in}{2.809621in}}%
\pgfpathlineto{\pgfqpoint{4.622664in}{2.824399in}}%
\pgfpathlineto{\pgfqpoint{4.630303in}{2.839447in}}%
\pgfpathlineto{\pgfqpoint{4.637942in}{2.854774in}}%
\pgfpathlineto{\pgfqpoint{4.624398in}{2.860086in}}%
\pgfpathlineto{\pgfqpoint{4.610860in}{2.865473in}}%
\pgfpathlineto{\pgfqpoint{4.597328in}{2.870935in}}%
\pgfpathlineto{\pgfqpoint{4.583802in}{2.876472in}}%
\pgfpathlineto{\pgfqpoint{4.576154in}{2.860742in}}%
\pgfpathlineto{\pgfqpoint{4.568505in}{2.845295in}}%
\pgfpathlineto{\pgfqpoint{4.560856in}{2.830124in}}%
\pgfpathlineto{\pgfqpoint{4.553205in}{2.815223in}}%
\pgfpathclose%
\pgfusepath{fill}%
\end{pgfscope}%
\begin{pgfscope}%
\pgfpathrectangle{\pgfqpoint{1.150000in}{0.150000in}}{\pgfqpoint{5.700000in}{5.700000in}}%
\pgfusepath{clip}%
\pgfsetbuttcap%
\pgfsetroundjoin%
\definecolor{currentfill}{rgb}{0.281412,0.155834,0.469201}%
\pgfsetfillcolor{currentfill}%
\pgfsetfillopacity{0.700000}%
\pgfsetlinewidth{0.000000pt}%
\definecolor{currentstroke}{rgb}{0.000000,0.000000,0.000000}%
\pgfsetstrokecolor{currentstroke}%
\pgfsetdash{}{0pt}%
\pgfpathmoveto{\pgfqpoint{3.351428in}{2.639444in}}%
\pgfpathlineto{\pgfqpoint{3.364766in}{2.631743in}}%
\pgfpathlineto{\pgfqpoint{3.378107in}{2.624146in}}%
\pgfpathlineto{\pgfqpoint{3.391450in}{2.616652in}}%
\pgfpathlineto{\pgfqpoint{3.404797in}{2.609260in}}%
\pgfpathlineto{\pgfqpoint{3.412758in}{2.620642in}}%
\pgfpathlineto{\pgfqpoint{3.420713in}{2.632139in}}%
\pgfpathlineto{\pgfqpoint{3.428662in}{2.643752in}}%
\pgfpathlineto{\pgfqpoint{3.436604in}{2.655487in}}%
\pgfpathlineto{\pgfqpoint{3.423266in}{2.663055in}}%
\pgfpathlineto{\pgfqpoint{3.409931in}{2.670725in}}%
\pgfpathlineto{\pgfqpoint{3.396598in}{2.678498in}}%
\pgfpathlineto{\pgfqpoint{3.383268in}{2.686374in}}%
\pgfpathlineto{\pgfqpoint{3.375318in}{2.674456in}}%
\pgfpathlineto{\pgfqpoint{3.367361in}{2.662664in}}%
\pgfpathlineto{\pgfqpoint{3.359397in}{2.650995in}}%
\pgfpathlineto{\pgfqpoint{3.351428in}{2.639444in}}%
\pgfpathclose%
\pgfusepath{fill}%
\end{pgfscope}%
\begin{pgfscope}%
\pgfpathrectangle{\pgfqpoint{1.150000in}{0.150000in}}{\pgfqpoint{5.700000in}{5.700000in}}%
\pgfusepath{clip}%
\pgfsetbuttcap%
\pgfsetroundjoin%
\definecolor{currentfill}{rgb}{0.271828,0.209303,0.504434}%
\pgfsetfillcolor{currentfill}%
\pgfsetfillopacity{0.700000}%
\pgfsetlinewidth{0.000000pt}%
\definecolor{currentstroke}{rgb}{0.000000,0.000000,0.000000}%
\pgfsetstrokecolor{currentstroke}%
\pgfsetdash{}{0pt}%
\pgfpathmoveto{\pgfqpoint{2.881709in}{2.758502in}}%
\pgfpathlineto{\pgfqpoint{2.895043in}{2.747973in}}%
\pgfpathlineto{\pgfqpoint{2.908377in}{2.737572in}}%
\pgfpathlineto{\pgfqpoint{2.921710in}{2.727299in}}%
\pgfpathlineto{\pgfqpoint{2.935044in}{2.717153in}}%
\pgfpathlineto{\pgfqpoint{2.943149in}{2.728157in}}%
\pgfpathlineto{\pgfqpoint{2.951248in}{2.739276in}}%
\pgfpathlineto{\pgfqpoint{2.959338in}{2.750512in}}%
\pgfpathlineto{\pgfqpoint{2.967422in}{2.761869in}}%
\pgfpathlineto{\pgfqpoint{2.954098in}{2.772131in}}%
\pgfpathlineto{\pgfqpoint{2.940775in}{2.782519in}}%
\pgfpathlineto{\pgfqpoint{2.927451in}{2.793035in}}%
\pgfpathlineto{\pgfqpoint{2.914126in}{2.803679in}}%
\pgfpathlineto{\pgfqpoint{2.906034in}{2.792200in}}%
\pgfpathlineto{\pgfqpoint{2.897933in}{2.780845in}}%
\pgfpathlineto{\pgfqpoint{2.889825in}{2.769613in}}%
\pgfpathlineto{\pgfqpoint{2.881709in}{2.758502in}}%
\pgfpathclose%
\pgfusepath{fill}%
\end{pgfscope}%
\begin{pgfscope}%
\pgfpathrectangle{\pgfqpoint{1.150000in}{0.150000in}}{\pgfqpoint{5.700000in}{5.700000in}}%
\pgfusepath{clip}%
\pgfsetbuttcap%
\pgfsetroundjoin%
\definecolor{currentfill}{rgb}{0.280868,0.160771,0.472899}%
\pgfsetfillcolor{currentfill}%
\pgfsetfillopacity{0.700000}%
\pgfsetlinewidth{0.000000pt}%
\definecolor{currentstroke}{rgb}{0.000000,0.000000,0.000000}%
\pgfsetstrokecolor{currentstroke}%
\pgfsetdash{}{0pt}%
\pgfpathmoveto{\pgfqpoint{3.212792in}{2.658543in}}%
\pgfpathlineto{\pgfqpoint{3.226122in}{2.650143in}}%
\pgfpathlineto{\pgfqpoint{3.239454in}{2.641852in}}%
\pgfpathlineto{\pgfqpoint{3.252787in}{2.633670in}}%
\pgfpathlineto{\pgfqpoint{3.266123in}{2.625597in}}%
\pgfpathlineto{\pgfqpoint{3.274126in}{2.636852in}}%
\pgfpathlineto{\pgfqpoint{3.282123in}{2.648218in}}%
\pgfpathlineto{\pgfqpoint{3.290114in}{2.659699in}}%
\pgfpathlineto{\pgfqpoint{3.298098in}{2.671297in}}%
\pgfpathlineto{\pgfqpoint{3.284771in}{2.679526in}}%
\pgfpathlineto{\pgfqpoint{3.271446in}{2.687864in}}%
\pgfpathlineto{\pgfqpoint{3.258123in}{2.696310in}}%
\pgfpathlineto{\pgfqpoint{3.244802in}{2.704866in}}%
\pgfpathlineto{\pgfqpoint{3.236810in}{2.693104in}}%
\pgfpathlineto{\pgfqpoint{3.228811in}{2.681466in}}%
\pgfpathlineto{\pgfqpoint{3.220805in}{2.669946in}}%
\pgfpathlineto{\pgfqpoint{3.212792in}{2.658543in}}%
\pgfpathclose%
\pgfusepath{fill}%
\end{pgfscope}%
\begin{pgfscope}%
\pgfpathrectangle{\pgfqpoint{1.150000in}{0.150000in}}{\pgfqpoint{5.700000in}{5.700000in}}%
\pgfusepath{clip}%
\pgfsetbuttcap%
\pgfsetroundjoin%
\definecolor{currentfill}{rgb}{0.218130,0.347432,0.550038}%
\pgfsetfillcolor{currentfill}%
\pgfsetfillopacity{0.700000}%
\pgfsetlinewidth{0.000000pt}%
\definecolor{currentstroke}{rgb}{0.000000,0.000000,0.000000}%
\pgfsetstrokecolor{currentstroke}%
\pgfsetdash{}{0pt}%
\pgfpathmoveto{\pgfqpoint{4.977170in}{3.040701in}}%
\pgfpathlineto{\pgfqpoint{4.990775in}{3.034843in}}%
\pgfpathlineto{\pgfqpoint{5.004387in}{3.029055in}}%
\pgfpathlineto{\pgfqpoint{5.018004in}{3.023338in}}%
\pgfpathlineto{\pgfqpoint{5.031628in}{3.017691in}}%
\pgfpathlineto{\pgfqpoint{5.039248in}{3.036525in}}%
\pgfpathlineto{\pgfqpoint{5.046872in}{3.055760in}}%
\pgfpathlineto{\pgfqpoint{5.054502in}{3.075407in}}%
\pgfpathlineto{\pgfqpoint{5.062138in}{3.095474in}}%
\pgfpathlineto{\pgfqpoint{5.048526in}{3.101618in}}%
\pgfpathlineto{\pgfqpoint{5.034919in}{3.107832in}}%
\pgfpathlineto{\pgfqpoint{5.021319in}{3.114117in}}%
\pgfpathlineto{\pgfqpoint{5.007724in}{3.120474in}}%
\pgfpathlineto{\pgfqpoint{5.000077in}{3.099901in}}%
\pgfpathlineto{\pgfqpoint{4.992437in}{3.079754in}}%
\pgfpathlineto{\pgfqpoint{4.984801in}{3.060024in}}%
\pgfpathlineto{\pgfqpoint{4.977170in}{3.040701in}}%
\pgfpathclose%
\pgfusepath{fill}%
\end{pgfscope}%
\begin{pgfscope}%
\pgfpathrectangle{\pgfqpoint{1.150000in}{0.150000in}}{\pgfqpoint{5.700000in}{5.700000in}}%
\pgfusepath{clip}%
\pgfsetbuttcap%
\pgfsetroundjoin%
\definecolor{currentfill}{rgb}{0.281887,0.150881,0.465405}%
\pgfsetfillcolor{currentfill}%
\pgfsetfillopacity{0.700000}%
\pgfsetlinewidth{0.000000pt}%
\definecolor{currentstroke}{rgb}{0.000000,0.000000,0.000000}%
\pgfsetstrokecolor{currentstroke}%
\pgfsetdash{}{0pt}%
\pgfpathmoveto{\pgfqpoint{3.489985in}{2.626224in}}%
\pgfpathlineto{\pgfqpoint{3.503338in}{2.619158in}}%
\pgfpathlineto{\pgfqpoint{3.516694in}{2.612189in}}%
\pgfpathlineto{\pgfqpoint{3.530053in}{2.605318in}}%
\pgfpathlineto{\pgfqpoint{3.543415in}{2.598545in}}%
\pgfpathlineto{\pgfqpoint{3.551336in}{2.610027in}}%
\pgfpathlineto{\pgfqpoint{3.559251in}{2.621627in}}%
\pgfpathlineto{\pgfqpoint{3.567159in}{2.633349in}}%
\pgfpathlineto{\pgfqpoint{3.575062in}{2.645197in}}%
\pgfpathlineto{\pgfqpoint{3.561708in}{2.652166in}}%
\pgfpathlineto{\pgfqpoint{3.548357in}{2.659233in}}%
\pgfpathlineto{\pgfqpoint{3.535009in}{2.666397in}}%
\pgfpathlineto{\pgfqpoint{3.521664in}{2.673659in}}%
\pgfpathlineto{\pgfqpoint{3.513753in}{2.661609in}}%
\pgfpathlineto{\pgfqpoint{3.505836in}{2.649688in}}%
\pgfpathlineto{\pgfqpoint{3.497914in}{2.637895in}}%
\pgfpathlineto{\pgfqpoint{3.489985in}{2.626224in}}%
\pgfpathclose%
\pgfusepath{fill}%
\end{pgfscope}%
\begin{pgfscope}%
\pgfpathrectangle{\pgfqpoint{1.150000in}{0.150000in}}{\pgfqpoint{5.700000in}{5.700000in}}%
\pgfusepath{clip}%
\pgfsetbuttcap%
\pgfsetroundjoin%
\definecolor{currentfill}{rgb}{0.266580,0.228262,0.514349}%
\pgfsetfillcolor{currentfill}%
\pgfsetfillopacity{0.700000}%
\pgfsetlinewidth{0.000000pt}%
\definecolor{currentstroke}{rgb}{0.000000,0.000000,0.000000}%
\pgfsetstrokecolor{currentstroke}%
\pgfsetdash{}{0pt}%
\pgfpathmoveto{\pgfqpoint{4.468467in}{2.777989in}}%
\pgfpathlineto{\pgfqpoint{4.481989in}{2.772919in}}%
\pgfpathlineto{\pgfqpoint{4.495516in}{2.767925in}}%
\pgfpathlineto{\pgfqpoint{4.509050in}{2.763008in}}%
\pgfpathlineto{\pgfqpoint{4.522590in}{2.758167in}}%
\pgfpathlineto{\pgfqpoint{4.530246in}{2.772062in}}%
\pgfpathlineto{\pgfqpoint{4.537901in}{2.786199in}}%
\pgfpathlineto{\pgfqpoint{4.545553in}{2.800583in}}%
\pgfpathlineto{\pgfqpoint{4.553205in}{2.815223in}}%
\pgfpathlineto{\pgfqpoint{4.539675in}{2.820440in}}%
\pgfpathlineto{\pgfqpoint{4.526151in}{2.825733in}}%
\pgfpathlineto{\pgfqpoint{4.512633in}{2.831103in}}%
\pgfpathlineto{\pgfqpoint{4.499121in}{2.836548in}}%
\pgfpathlineto{\pgfqpoint{4.491460in}{2.821526in}}%
\pgfpathlineto{\pgfqpoint{4.483797in}{2.806763in}}%
\pgfpathlineto{\pgfqpoint{4.476133in}{2.792253in}}%
\pgfpathlineto{\pgfqpoint{4.468467in}{2.777989in}}%
\pgfpathclose%
\pgfusepath{fill}%
\end{pgfscope}%
\begin{pgfscope}%
\pgfpathrectangle{\pgfqpoint{1.150000in}{0.150000in}}{\pgfqpoint{5.700000in}{5.700000in}}%
\pgfusepath{clip}%
\pgfsetbuttcap%
\pgfsetroundjoin%
\definecolor{currentfill}{rgb}{0.278826,0.175490,0.483397}%
\pgfsetfillcolor{currentfill}%
\pgfsetfillopacity{0.700000}%
\pgfsetlinewidth{0.000000pt}%
\definecolor{currentstroke}{rgb}{0.000000,0.000000,0.000000}%
\pgfsetstrokecolor{currentstroke}%
\pgfsetdash{}{0pt}%
\pgfpathmoveto{\pgfqpoint{3.074020in}{2.684195in}}%
\pgfpathlineto{\pgfqpoint{3.087348in}{2.675024in}}%
\pgfpathlineto{\pgfqpoint{3.100677in}{2.665969in}}%
\pgfpathlineto{\pgfqpoint{3.114007in}{2.657029in}}%
\pgfpathlineto{\pgfqpoint{3.127338in}{2.648205in}}%
\pgfpathlineto{\pgfqpoint{3.135386in}{2.659300in}}%
\pgfpathlineto{\pgfqpoint{3.143427in}{2.670505in}}%
\pgfpathlineto{\pgfqpoint{3.151462in}{2.681822in}}%
\pgfpathlineto{\pgfqpoint{3.159490in}{2.693255in}}%
\pgfpathlineto{\pgfqpoint{3.146168in}{2.702215in}}%
\pgfpathlineto{\pgfqpoint{3.132847in}{2.711290in}}%
\pgfpathlineto{\pgfqpoint{3.119527in}{2.720481in}}%
\pgfpathlineto{\pgfqpoint{3.106208in}{2.729787in}}%
\pgfpathlineto{\pgfqpoint{3.098172in}{2.718211in}}%
\pgfpathlineto{\pgfqpoint{3.090128in}{2.706756in}}%
\pgfpathlineto{\pgfqpoint{3.082078in}{2.695418in}}%
\pgfpathlineto{\pgfqpoint{3.074020in}{2.684195in}}%
\pgfpathclose%
\pgfusepath{fill}%
\end{pgfscope}%
\begin{pgfscope}%
\pgfpathrectangle{\pgfqpoint{1.150000in}{0.150000in}}{\pgfqpoint{5.700000in}{5.700000in}}%
\pgfusepath{clip}%
\pgfsetbuttcap%
\pgfsetroundjoin%
\definecolor{currentfill}{rgb}{0.281412,0.155834,0.469201}%
\pgfsetfillcolor{currentfill}%
\pgfsetfillopacity{0.700000}%
\pgfsetlinewidth{0.000000pt}%
\definecolor{currentstroke}{rgb}{0.000000,0.000000,0.000000}%
\pgfsetstrokecolor{currentstroke}%
\pgfsetdash{}{0pt}%
\pgfpathmoveto{\pgfqpoint{3.852017in}{2.638992in}}%
\pgfpathlineto{\pgfqpoint{3.865423in}{2.633138in}}%
\pgfpathlineto{\pgfqpoint{3.878833in}{2.627373in}}%
\pgfpathlineto{\pgfqpoint{3.892248in}{2.621694in}}%
\pgfpathlineto{\pgfqpoint{3.905668in}{2.616102in}}%
\pgfpathlineto{\pgfqpoint{3.913483in}{2.628023in}}%
\pgfpathlineto{\pgfqpoint{3.921292in}{2.640086in}}%
\pgfpathlineto{\pgfqpoint{3.929097in}{2.652298in}}%
\pgfpathlineto{\pgfqpoint{3.936897in}{2.664663in}}%
\pgfpathlineto{\pgfqpoint{3.923485in}{2.670510in}}%
\pgfpathlineto{\pgfqpoint{3.910078in}{2.676445in}}%
\pgfpathlineto{\pgfqpoint{3.896676in}{2.682466in}}%
\pgfpathlineto{\pgfqpoint{3.883278in}{2.688576in}}%
\pgfpathlineto{\pgfqpoint{3.875470in}{2.675948in}}%
\pgfpathlineto{\pgfqpoint{3.867657in}{2.663478in}}%
\pgfpathlineto{\pgfqpoint{3.859840in}{2.651161in}}%
\pgfpathlineto{\pgfqpoint{3.852017in}{2.638992in}}%
\pgfpathclose%
\pgfusepath{fill}%
\end{pgfscope}%
\begin{pgfscope}%
\pgfpathrectangle{\pgfqpoint{1.150000in}{0.150000in}}{\pgfqpoint{5.700000in}{5.700000in}}%
\pgfusepath{clip}%
\pgfsetbuttcap%
\pgfsetroundjoin%
\definecolor{currentfill}{rgb}{0.166617,0.463708,0.558119}%
\pgfsetfillcolor{currentfill}%
\pgfsetfillopacity{0.700000}%
\pgfsetlinewidth{0.000000pt}%
\definecolor{currentstroke}{rgb}{0.000000,0.000000,0.000000}%
\pgfsetstrokecolor{currentstroke}%
\pgfsetdash{}{0pt}%
\pgfpathmoveto{\pgfqpoint{5.208747in}{3.342335in}}%
\pgfpathlineto{\pgfqpoint{5.222359in}{3.334925in}}%
\pgfpathlineto{\pgfqpoint{5.235976in}{3.327584in}}%
\pgfpathlineto{\pgfqpoint{5.249599in}{3.320312in}}%
\pgfpathlineto{\pgfqpoint{5.263228in}{3.313110in}}%
\pgfpathlineto{\pgfqpoint{5.270956in}{3.338408in}}%
\pgfpathlineto{\pgfqpoint{5.278696in}{3.364260in}}%
\pgfpathlineto{\pgfqpoint{5.286448in}{3.390677in}}%
\pgfpathlineto{\pgfqpoint{5.272827in}{3.398314in}}%
\pgfpathlineto{\pgfqpoint{5.259210in}{3.406020in}}%
\pgfpathlineto{\pgfqpoint{5.245600in}{3.413796in}}%
\pgfpathlineto{\pgfqpoint{5.231995in}{3.421642in}}%
\pgfpathlineto{\pgfqpoint{5.224233in}{3.394639in}}%
\pgfpathlineto{\pgfqpoint{5.216484in}{3.368207in}}%
\pgfpathlineto{\pgfqpoint{5.208747in}{3.342335in}}%
\pgfpathclose%
\pgfusepath{fill}%
\end{pgfscope}%
\begin{pgfscope}%
\pgfpathrectangle{\pgfqpoint{1.150000in}{0.150000in}}{\pgfqpoint{5.700000in}{5.700000in}}%
\pgfusepath{clip}%
\pgfsetbuttcap%
\pgfsetroundjoin%
\definecolor{currentfill}{rgb}{0.278826,0.175490,0.483397}%
\pgfsetfillcolor{currentfill}%
\pgfsetfillopacity{0.700000}%
\pgfsetlinewidth{0.000000pt}%
\definecolor{currentstroke}{rgb}{0.000000,0.000000,0.000000}%
\pgfsetstrokecolor{currentstroke}%
\pgfsetdash{}{0pt}%
\pgfpathmoveto{\pgfqpoint{4.075450in}{2.669839in}}%
\pgfpathlineto{\pgfqpoint{4.088897in}{2.664479in}}%
\pgfpathlineto{\pgfqpoint{4.102350in}{2.659201in}}%
\pgfpathlineto{\pgfqpoint{4.115808in}{2.654007in}}%
\pgfpathlineto{\pgfqpoint{4.129271in}{2.648894in}}%
\pgfpathlineto{\pgfqpoint{4.137023in}{2.661233in}}%
\pgfpathlineto{\pgfqpoint{4.144771in}{2.673741in}}%
\pgfpathlineto{\pgfqpoint{4.152515in}{2.686422in}}%
\pgfpathlineto{\pgfqpoint{4.160255in}{2.699283in}}%
\pgfpathlineto{\pgfqpoint{4.146801in}{2.704691in}}%
\pgfpathlineto{\pgfqpoint{4.133352in}{2.710182in}}%
\pgfpathlineto{\pgfqpoint{4.119908in}{2.715755in}}%
\pgfpathlineto{\pgfqpoint{4.106469in}{2.721411in}}%
\pgfpathlineto{\pgfqpoint{4.098720in}{2.708246in}}%
\pgfpathlineto{\pgfqpoint{4.090967in}{2.695267in}}%
\pgfpathlineto{\pgfqpoint{4.083211in}{2.682466in}}%
\pgfpathlineto{\pgfqpoint{4.075450in}{2.669839in}}%
\pgfpathclose%
\pgfusepath{fill}%
\end{pgfscope}%
\begin{pgfscope}%
\pgfpathrectangle{\pgfqpoint{1.150000in}{0.150000in}}{\pgfqpoint{5.700000in}{5.700000in}}%
\pgfusepath{clip}%
\pgfsetbuttcap%
\pgfsetroundjoin%
\definecolor{currentfill}{rgb}{0.271828,0.209303,0.504434}%
\pgfsetfillcolor{currentfill}%
\pgfsetfillopacity{0.700000}%
\pgfsetlinewidth{0.000000pt}%
\definecolor{currentstroke}{rgb}{0.000000,0.000000,0.000000}%
\pgfsetstrokecolor{currentstroke}%
\pgfsetdash{}{0pt}%
\pgfpathmoveto{\pgfqpoint{4.383715in}{2.742887in}}%
\pgfpathlineto{\pgfqpoint{4.397223in}{2.737864in}}%
\pgfpathlineto{\pgfqpoint{4.410737in}{2.732919in}}%
\pgfpathlineto{\pgfqpoint{4.424257in}{2.728051in}}%
\pgfpathlineto{\pgfqpoint{4.437783in}{2.723260in}}%
\pgfpathlineto{\pgfqpoint{4.445457in}{2.736606in}}%
\pgfpathlineto{\pgfqpoint{4.453129in}{2.750172in}}%
\pgfpathlineto{\pgfqpoint{4.460799in}{2.763964in}}%
\pgfpathlineto{\pgfqpoint{4.468467in}{2.777989in}}%
\pgfpathlineto{\pgfqpoint{4.454951in}{2.783135in}}%
\pgfpathlineto{\pgfqpoint{4.441440in}{2.788359in}}%
\pgfpathlineto{\pgfqpoint{4.427936in}{2.793660in}}%
\pgfpathlineto{\pgfqpoint{4.414437in}{2.799039in}}%
\pgfpathlineto{\pgfqpoint{4.406760in}{2.784651in}}%
\pgfpathlineto{\pgfqpoint{4.399080in}{2.770501in}}%
\pgfpathlineto{\pgfqpoint{4.391399in}{2.756581in}}%
\pgfpathlineto{\pgfqpoint{4.383715in}{2.742887in}}%
\pgfpathclose%
\pgfusepath{fill}%
\end{pgfscope}%
\begin{pgfscope}%
\pgfpathrectangle{\pgfqpoint{1.150000in}{0.150000in}}{\pgfqpoint{5.700000in}{5.700000in}}%
\pgfusepath{clip}%
\pgfsetbuttcap%
\pgfsetroundjoin%
\definecolor{currentfill}{rgb}{0.282290,0.145912,0.461510}%
\pgfsetfillcolor{currentfill}%
\pgfsetfillopacity{0.700000}%
\pgfsetlinewidth{0.000000pt}%
\definecolor{currentstroke}{rgb}{0.000000,0.000000,0.000000}%
\pgfsetstrokecolor{currentstroke}%
\pgfsetdash{}{0pt}%
\pgfpathmoveto{\pgfqpoint{3.628515in}{2.618278in}}%
\pgfpathlineto{\pgfqpoint{3.641888in}{2.611786in}}%
\pgfpathlineto{\pgfqpoint{3.655264in}{2.605387in}}%
\pgfpathlineto{\pgfqpoint{3.668644in}{2.599081in}}%
\pgfpathlineto{\pgfqpoint{3.682028in}{2.592868in}}%
\pgfpathlineto{\pgfqpoint{3.689909in}{2.604428in}}%
\pgfpathlineto{\pgfqpoint{3.697785in}{2.616112in}}%
\pgfpathlineto{\pgfqpoint{3.705655in}{2.627923in}}%
\pgfpathlineto{\pgfqpoint{3.713520in}{2.639865in}}%
\pgfpathlineto{\pgfqpoint{3.700144in}{2.646294in}}%
\pgfpathlineto{\pgfqpoint{3.686772in}{2.652816in}}%
\pgfpathlineto{\pgfqpoint{3.673404in}{2.659431in}}%
\pgfpathlineto{\pgfqpoint{3.660039in}{2.666139in}}%
\pgfpathlineto{\pgfqpoint{3.652167in}{2.653973in}}%
\pgfpathlineto{\pgfqpoint{3.644288in}{2.641944in}}%
\pgfpathlineto{\pgfqpoint{3.636405in}{2.630047in}}%
\pgfpathlineto{\pgfqpoint{3.628515in}{2.618278in}}%
\pgfpathclose%
\pgfusepath{fill}%
\end{pgfscope}%
\begin{pgfscope}%
\pgfpathrectangle{\pgfqpoint{1.150000in}{0.150000in}}{\pgfqpoint{5.700000in}{5.700000in}}%
\pgfusepath{clip}%
\pgfsetbuttcap%
\pgfsetroundjoin%
\definecolor{currentfill}{rgb}{0.206756,0.371758,0.553117}%
\pgfsetfillcolor{currentfill}%
\pgfsetfillopacity{0.700000}%
\pgfsetlinewidth{0.000000pt}%
\definecolor{currentstroke}{rgb}{0.000000,0.000000,0.000000}%
\pgfsetstrokecolor{currentstroke}%
\pgfsetdash{}{0pt}%
\pgfpathmoveto{\pgfqpoint{5.062138in}{3.095474in}}%
\pgfpathlineto{\pgfqpoint{5.075757in}{3.089401in}}%
\pgfpathlineto{\pgfqpoint{5.089382in}{3.083398in}}%
\pgfpathlineto{\pgfqpoint{5.103013in}{3.077465in}}%
\pgfpathlineto{\pgfqpoint{5.116651in}{3.071602in}}%
\pgfpathlineto{\pgfqpoint{5.124281in}{3.091590in}}%
\pgfpathlineto{\pgfqpoint{5.131919in}{3.112012in}}%
\pgfpathlineto{\pgfqpoint{5.139563in}{3.132878in}}%
\pgfpathlineto{\pgfqpoint{5.147215in}{3.154199in}}%
\pgfpathlineto{\pgfqpoint{5.133589in}{3.160580in}}%
\pgfpathlineto{\pgfqpoint{5.119969in}{3.167030in}}%
\pgfpathlineto{\pgfqpoint{5.106355in}{3.173551in}}%
\pgfpathlineto{\pgfqpoint{5.092747in}{3.180143in}}%
\pgfpathlineto{\pgfqpoint{5.085084in}{3.158296in}}%
\pgfpathlineto{\pgfqpoint{5.077429in}{3.136909in}}%
\pgfpathlineto{\pgfqpoint{5.069780in}{3.115972in}}%
\pgfpathlineto{\pgfqpoint{5.062138in}{3.095474in}}%
\pgfpathclose%
\pgfusepath{fill}%
\end{pgfscope}%
\begin{pgfscope}%
\pgfpathrectangle{\pgfqpoint{1.150000in}{0.150000in}}{\pgfqpoint{5.700000in}{5.700000in}}%
\pgfusepath{clip}%
\pgfsetbuttcap%
\pgfsetroundjoin%
\definecolor{currentfill}{rgb}{0.180629,0.429975,0.557282}%
\pgfsetfillcolor{currentfill}%
\pgfsetfillopacity{0.700000}%
\pgfsetlinewidth{0.000000pt}%
\definecolor{currentstroke}{rgb}{0.000000,0.000000,0.000000}%
\pgfsetstrokecolor{currentstroke}%
\pgfsetdash{}{0pt}%
\pgfpathmoveto{\pgfqpoint{5.177905in}{3.244224in}}%
\pgfpathlineto{\pgfqpoint{5.191526in}{3.237374in}}%
\pgfpathlineto{\pgfqpoint{5.205154in}{3.230593in}}%
\pgfpathlineto{\pgfqpoint{5.218787in}{3.223882in}}%
\pgfpathlineto{\pgfqpoint{5.232427in}{3.217241in}}%
\pgfpathlineto{\pgfqpoint{5.240111in}{3.240432in}}%
\pgfpathlineto{\pgfqpoint{5.247806in}{3.264133in}}%
\pgfpathlineto{\pgfqpoint{5.255512in}{3.288356in}}%
\pgfpathlineto{\pgfqpoint{5.263228in}{3.313110in}}%
\pgfpathlineto{\pgfqpoint{5.249599in}{3.320312in}}%
\pgfpathlineto{\pgfqpoint{5.235976in}{3.327584in}}%
\pgfpathlineto{\pgfqpoint{5.222359in}{3.334925in}}%
\pgfpathlineto{\pgfqpoint{5.208747in}{3.342335in}}%
\pgfpathlineto{\pgfqpoint{5.201021in}{3.317012in}}%
\pgfpathlineto{\pgfqpoint{5.193305in}{3.292226in}}%
\pgfpathlineto{\pgfqpoint{5.185600in}{3.267967in}}%
\pgfpathlineto{\pgfqpoint{5.177905in}{3.244224in}}%
\pgfpathclose%
\pgfusepath{fill}%
\end{pgfscope}%
\begin{pgfscope}%
\pgfpathrectangle{\pgfqpoint{1.150000in}{0.150000in}}{\pgfqpoint{5.700000in}{5.700000in}}%
\pgfusepath{clip}%
\pgfsetbuttcap%
\pgfsetroundjoin%
\definecolor{currentfill}{rgb}{0.275191,0.194905,0.496005}%
\pgfsetfillcolor{currentfill}%
\pgfsetfillopacity{0.700000}%
\pgfsetlinewidth{0.000000pt}%
\definecolor{currentstroke}{rgb}{0.000000,0.000000,0.000000}%
\pgfsetstrokecolor{currentstroke}%
\pgfsetdash{}{0pt}%
\pgfpathmoveto{\pgfqpoint{2.935044in}{2.717153in}}%
\pgfpathlineto{\pgfqpoint{2.948377in}{2.707131in}}%
\pgfpathlineto{\pgfqpoint{2.961710in}{2.697234in}}%
\pgfpathlineto{\pgfqpoint{2.975044in}{2.687460in}}%
\pgfpathlineto{\pgfqpoint{2.988378in}{2.677808in}}%
\pgfpathlineto{\pgfqpoint{2.996474in}{2.688704in}}%
\pgfpathlineto{\pgfqpoint{3.004562in}{2.699710in}}%
\pgfpathlineto{\pgfqpoint{3.012643in}{2.710829in}}%
\pgfpathlineto{\pgfqpoint{3.020717in}{2.722063in}}%
\pgfpathlineto{\pgfqpoint{3.007393in}{2.731830in}}%
\pgfpathlineto{\pgfqpoint{2.994069in}{2.741719in}}%
\pgfpathlineto{\pgfqpoint{2.980745in}{2.751732in}}%
\pgfpathlineto{\pgfqpoint{2.967422in}{2.761869in}}%
\pgfpathlineto{\pgfqpoint{2.959338in}{2.750512in}}%
\pgfpathlineto{\pgfqpoint{2.951248in}{2.739276in}}%
\pgfpathlineto{\pgfqpoint{2.943149in}{2.728157in}}%
\pgfpathlineto{\pgfqpoint{2.935044in}{2.717153in}}%
\pgfpathclose%
\pgfusepath{fill}%
\end{pgfscope}%
\begin{pgfscope}%
\pgfpathrectangle{\pgfqpoint{1.150000in}{0.150000in}}{\pgfqpoint{5.700000in}{5.700000in}}%
\pgfusepath{clip}%
\pgfsetbuttcap%
\pgfsetroundjoin%
\definecolor{currentfill}{rgb}{0.275191,0.194905,0.496005}%
\pgfsetfillcolor{currentfill}%
\pgfsetfillopacity{0.700000}%
\pgfsetlinewidth{0.000000pt}%
\definecolor{currentstroke}{rgb}{0.000000,0.000000,0.000000}%
\pgfsetstrokecolor{currentstroke}%
\pgfsetdash{}{0pt}%
\pgfpathmoveto{\pgfqpoint{4.298938in}{2.709757in}}%
\pgfpathlineto{\pgfqpoint{4.312432in}{2.704757in}}%
\pgfpathlineto{\pgfqpoint{4.325933in}{2.699835in}}%
\pgfpathlineto{\pgfqpoint{4.339439in}{2.694993in}}%
\pgfpathlineto{\pgfqpoint{4.352951in}{2.690228in}}%
\pgfpathlineto{\pgfqpoint{4.360646in}{2.703087in}}%
\pgfpathlineto{\pgfqpoint{4.368338in}{2.716146in}}%
\pgfpathlineto{\pgfqpoint{4.376028in}{2.729410in}}%
\pgfpathlineto{\pgfqpoint{4.383715in}{2.742887in}}%
\pgfpathlineto{\pgfqpoint{4.370212in}{2.747988in}}%
\pgfpathlineto{\pgfqpoint{4.356715in}{2.753167in}}%
\pgfpathlineto{\pgfqpoint{4.343224in}{2.758424in}}%
\pgfpathlineto{\pgfqpoint{4.329739in}{2.763760in}}%
\pgfpathlineto{\pgfqpoint{4.322043in}{2.749940in}}%
\pgfpathlineto{\pgfqpoint{4.314344in}{2.736337in}}%
\pgfpathlineto{\pgfqpoint{4.306642in}{2.722945in}}%
\pgfpathlineto{\pgfqpoint{4.298938in}{2.709757in}}%
\pgfpathclose%
\pgfusepath{fill}%
\end{pgfscope}%
\begin{pgfscope}%
\pgfpathrectangle{\pgfqpoint{1.150000in}{0.150000in}}{\pgfqpoint{5.700000in}{5.700000in}}%
\pgfusepath{clip}%
\pgfsetbuttcap%
\pgfsetroundjoin%
\definecolor{currentfill}{rgb}{0.265145,0.232956,0.516599}%
\pgfsetfillcolor{currentfill}%
\pgfsetfillopacity{0.700000}%
\pgfsetlinewidth{0.000000pt}%
\definecolor{currentstroke}{rgb}{0.000000,0.000000,0.000000}%
\pgfsetstrokecolor{currentstroke}%
\pgfsetdash{}{0pt}%
\pgfpathmoveto{\pgfqpoint{2.742384in}{2.803445in}}%
\pgfpathlineto{\pgfqpoint{2.755736in}{2.791942in}}%
\pgfpathlineto{\pgfqpoint{2.769087in}{2.780576in}}%
\pgfpathlineto{\pgfqpoint{2.782437in}{2.769347in}}%
\pgfpathlineto{\pgfqpoint{2.795785in}{2.758254in}}%
\pgfpathlineto{\pgfqpoint{2.803942in}{2.768995in}}%
\pgfpathlineto{\pgfqpoint{2.812092in}{2.779852in}}%
\pgfpathlineto{\pgfqpoint{2.820233in}{2.790828in}}%
\pgfpathlineto{\pgfqpoint{2.828367in}{2.801925in}}%
\pgfpathlineto{\pgfqpoint{2.815029in}{2.813113in}}%
\pgfpathlineto{\pgfqpoint{2.801691in}{2.824437in}}%
\pgfpathlineto{\pgfqpoint{2.788351in}{2.835898in}}%
\pgfpathlineto{\pgfqpoint{2.775009in}{2.847496in}}%
\pgfpathlineto{\pgfqpoint{2.766865in}{2.836297in}}%
\pgfpathlineto{\pgfqpoint{2.758713in}{2.825224in}}%
\pgfpathlineto{\pgfqpoint{2.750553in}{2.814274in}}%
\pgfpathlineto{\pgfqpoint{2.742384in}{2.803445in}}%
\pgfpathclose%
\pgfusepath{fill}%
\end{pgfscope}%
\begin{pgfscope}%
\pgfpathrectangle{\pgfqpoint{1.150000in}{0.150000in}}{\pgfqpoint{5.700000in}{5.700000in}}%
\pgfusepath{clip}%
\pgfsetbuttcap%
\pgfsetroundjoin%
\definecolor{currentfill}{rgb}{0.281887,0.150881,0.465405}%
\pgfsetfillcolor{currentfill}%
\pgfsetfillopacity{0.700000}%
\pgfsetlinewidth{0.000000pt}%
\definecolor{currentstroke}{rgb}{0.000000,0.000000,0.000000}%
\pgfsetstrokecolor{currentstroke}%
\pgfsetdash{}{0pt}%
\pgfpathmoveto{\pgfqpoint{3.767063in}{2.615065in}}%
\pgfpathlineto{\pgfqpoint{3.780460in}{2.609092in}}%
\pgfpathlineto{\pgfqpoint{3.793861in}{2.603208in}}%
\pgfpathlineto{\pgfqpoint{3.807266in}{2.597413in}}%
\pgfpathlineto{\pgfqpoint{3.820676in}{2.591707in}}%
\pgfpathlineto{\pgfqpoint{3.828519in}{2.603329in}}%
\pgfpathlineto{\pgfqpoint{3.836357in}{2.615081in}}%
\pgfpathlineto{\pgfqpoint{3.844189in}{2.626967in}}%
\pgfpathlineto{\pgfqpoint{3.852017in}{2.638992in}}%
\pgfpathlineto{\pgfqpoint{3.838615in}{2.644934in}}%
\pgfpathlineto{\pgfqpoint{3.825218in}{2.650965in}}%
\pgfpathlineto{\pgfqpoint{3.811826in}{2.657084in}}%
\pgfpathlineto{\pgfqpoint{3.798437in}{2.663294in}}%
\pgfpathlineto{\pgfqpoint{3.790602in}{2.651025in}}%
\pgfpathlineto{\pgfqpoint{3.782761in}{2.638900in}}%
\pgfpathlineto{\pgfqpoint{3.774915in}{2.626915in}}%
\pgfpathlineto{\pgfqpoint{3.767063in}{2.615065in}}%
\pgfpathclose%
\pgfusepath{fill}%
\end{pgfscope}%
\begin{pgfscope}%
\pgfpathrectangle{\pgfqpoint{1.150000in}{0.150000in}}{\pgfqpoint{5.700000in}{5.700000in}}%
\pgfusepath{clip}%
\pgfsetbuttcap%
\pgfsetroundjoin%
\definecolor{currentfill}{rgb}{0.280868,0.160771,0.472899}%
\pgfsetfillcolor{currentfill}%
\pgfsetfillopacity{0.700000}%
\pgfsetlinewidth{0.000000pt}%
\definecolor{currentstroke}{rgb}{0.000000,0.000000,0.000000}%
\pgfsetstrokecolor{currentstroke}%
\pgfsetdash{}{0pt}%
\pgfpathmoveto{\pgfqpoint{3.990590in}{2.642133in}}%
\pgfpathlineto{\pgfqpoint{4.004026in}{2.636713in}}%
\pgfpathlineto{\pgfqpoint{4.017467in}{2.631378in}}%
\pgfpathlineto{\pgfqpoint{4.030912in}{2.626127in}}%
\pgfpathlineto{\pgfqpoint{4.044363in}{2.620960in}}%
\pgfpathlineto{\pgfqpoint{4.052142in}{2.632946in}}%
\pgfpathlineto{\pgfqpoint{4.059915in}{2.645084in}}%
\pgfpathlineto{\pgfqpoint{4.067685in}{2.657380in}}%
\pgfpathlineto{\pgfqpoint{4.075450in}{2.669839in}}%
\pgfpathlineto{\pgfqpoint{4.062007in}{2.675282in}}%
\pgfpathlineto{\pgfqpoint{4.048570in}{2.680809in}}%
\pgfpathlineto{\pgfqpoint{4.035138in}{2.686420in}}%
\pgfpathlineto{\pgfqpoint{4.021711in}{2.692115in}}%
\pgfpathlineto{\pgfqpoint{4.013937in}{2.679373in}}%
\pgfpathlineto{\pgfqpoint{4.006159in}{2.666798in}}%
\pgfpathlineto{\pgfqpoint{3.998377in}{2.654387in}}%
\pgfpathlineto{\pgfqpoint{3.990590in}{2.642133in}}%
\pgfpathclose%
\pgfusepath{fill}%
\end{pgfscope}%
\begin{pgfscope}%
\pgfpathrectangle{\pgfqpoint{1.150000in}{0.150000in}}{\pgfqpoint{5.700000in}{5.700000in}}%
\pgfusepath{clip}%
\pgfsetbuttcap%
\pgfsetroundjoin%
\definecolor{currentfill}{rgb}{0.281887,0.150881,0.465405}%
\pgfsetfillcolor{currentfill}%
\pgfsetfillopacity{0.700000}%
\pgfsetlinewidth{0.000000pt}%
\definecolor{currentstroke}{rgb}{0.000000,0.000000,0.000000}%
\pgfsetstrokecolor{currentstroke}%
\pgfsetdash{}{0pt}%
\pgfpathmoveto{\pgfqpoint{3.266123in}{2.625597in}}%
\pgfpathlineto{\pgfqpoint{3.279460in}{2.617630in}}%
\pgfpathlineto{\pgfqpoint{3.292800in}{2.609770in}}%
\pgfpathlineto{\pgfqpoint{3.306142in}{2.602016in}}%
\pgfpathlineto{\pgfqpoint{3.319486in}{2.594367in}}%
\pgfpathlineto{\pgfqpoint{3.327481in}{2.605473in}}%
\pgfpathlineto{\pgfqpoint{3.335470in}{2.616687in}}%
\pgfpathlineto{\pgfqpoint{3.343452in}{2.628009in}}%
\pgfpathlineto{\pgfqpoint{3.351428in}{2.639444in}}%
\pgfpathlineto{\pgfqpoint{3.338092in}{2.647249in}}%
\pgfpathlineto{\pgfqpoint{3.324758in}{2.655159in}}%
\pgfpathlineto{\pgfqpoint{3.311427in}{2.663175in}}%
\pgfpathlineto{\pgfqpoint{3.298098in}{2.671297in}}%
\pgfpathlineto{\pgfqpoint{3.290114in}{2.659699in}}%
\pgfpathlineto{\pgfqpoint{3.282123in}{2.648218in}}%
\pgfpathlineto{\pgfqpoint{3.274126in}{2.636852in}}%
\pgfpathlineto{\pgfqpoint{3.266123in}{2.625597in}}%
\pgfpathclose%
\pgfusepath{fill}%
\end{pgfscope}%
\begin{pgfscope}%
\pgfpathrectangle{\pgfqpoint{1.150000in}{0.150000in}}{\pgfqpoint{5.700000in}{5.700000in}}%
\pgfusepath{clip}%
\pgfsetbuttcap%
\pgfsetroundjoin%
\definecolor{currentfill}{rgb}{0.194100,0.399323,0.555565}%
\pgfsetfillcolor{currentfill}%
\pgfsetfillopacity{0.700000}%
\pgfsetlinewidth{0.000000pt}%
\definecolor{currentstroke}{rgb}{0.000000,0.000000,0.000000}%
\pgfsetstrokecolor{currentstroke}%
\pgfsetdash{}{0pt}%
\pgfpathmoveto{\pgfqpoint{5.147215in}{3.154199in}}%
\pgfpathlineto{\pgfqpoint{5.160847in}{3.147888in}}%
\pgfpathlineto{\pgfqpoint{5.174486in}{3.141646in}}%
\pgfpathlineto{\pgfqpoint{5.188130in}{3.135474in}}%
\pgfpathlineto{\pgfqpoint{5.201781in}{3.129372in}}%
\pgfpathlineto{\pgfqpoint{5.209429in}{3.150625in}}%
\pgfpathlineto{\pgfqpoint{5.217086in}{3.172348in}}%
\pgfpathlineto{\pgfqpoint{5.224752in}{3.194550in}}%
\pgfpathlineto{\pgfqpoint{5.232427in}{3.217241in}}%
\pgfpathlineto{\pgfqpoint{5.218787in}{3.223882in}}%
\pgfpathlineto{\pgfqpoint{5.205154in}{3.230593in}}%
\pgfpathlineto{\pgfqpoint{5.191526in}{3.237374in}}%
\pgfpathlineto{\pgfqpoint{5.177905in}{3.244224in}}%
\pgfpathlineto{\pgfqpoint{5.170219in}{3.220986in}}%
\pgfpathlineto{\pgfqpoint{5.162543in}{3.198242in}}%
\pgfpathlineto{\pgfqpoint{5.154875in}{3.175983in}}%
\pgfpathlineto{\pgfqpoint{5.147215in}{3.154199in}}%
\pgfpathclose%
\pgfusepath{fill}%
\end{pgfscope}%
\begin{pgfscope}%
\pgfpathrectangle{\pgfqpoint{1.150000in}{0.150000in}}{\pgfqpoint{5.700000in}{5.700000in}}%
\pgfusepath{clip}%
\pgfsetbuttcap%
\pgfsetroundjoin%
\definecolor{currentfill}{rgb}{0.248629,0.278775,0.534556}%
\pgfsetfillcolor{currentfill}%
\pgfsetfillopacity{0.700000}%
\pgfsetlinewidth{0.000000pt}%
\definecolor{currentstroke}{rgb}{0.000000,0.000000,0.000000}%
\pgfsetstrokecolor{currentstroke}%
\pgfsetdash{}{0pt}%
\pgfpathmoveto{\pgfqpoint{4.776982in}{2.875863in}}%
\pgfpathlineto{\pgfqpoint{4.790571in}{2.870798in}}%
\pgfpathlineto{\pgfqpoint{4.804166in}{2.865807in}}%
\pgfpathlineto{\pgfqpoint{4.817767in}{2.860887in}}%
\pgfpathlineto{\pgfqpoint{4.831375in}{2.856040in}}%
\pgfpathlineto{\pgfqpoint{4.838982in}{2.871588in}}%
\pgfpathlineto{\pgfqpoint{4.846591in}{2.887445in}}%
\pgfpathlineto{\pgfqpoint{4.854202in}{2.903619in}}%
\pgfpathlineto{\pgfqpoint{4.861814in}{2.920120in}}%
\pgfpathlineto{\pgfqpoint{4.848218in}{2.925403in}}%
\pgfpathlineto{\pgfqpoint{4.834628in}{2.930759in}}%
\pgfpathlineto{\pgfqpoint{4.821044in}{2.936187in}}%
\pgfpathlineto{\pgfqpoint{4.807466in}{2.941688in}}%
\pgfpathlineto{\pgfqpoint{4.799843in}{2.924744in}}%
\pgfpathlineto{\pgfqpoint{4.792221in}{2.908130in}}%
\pgfpathlineto{\pgfqpoint{4.784601in}{2.891839in}}%
\pgfpathlineto{\pgfqpoint{4.776982in}{2.875863in}}%
\pgfpathclose%
\pgfusepath{fill}%
\end{pgfscope}%
\begin{pgfscope}%
\pgfpathrectangle{\pgfqpoint{1.150000in}{0.150000in}}{\pgfqpoint{5.700000in}{5.700000in}}%
\pgfusepath{clip}%
\pgfsetbuttcap%
\pgfsetroundjoin%
\definecolor{currentfill}{rgb}{0.282290,0.145912,0.461510}%
\pgfsetfillcolor{currentfill}%
\pgfsetfillopacity{0.700000}%
\pgfsetlinewidth{0.000000pt}%
\definecolor{currentstroke}{rgb}{0.000000,0.000000,0.000000}%
\pgfsetstrokecolor{currentstroke}%
\pgfsetdash{}{0pt}%
\pgfpathmoveto{\pgfqpoint{3.404797in}{2.609260in}}%
\pgfpathlineto{\pgfqpoint{3.418146in}{2.601969in}}%
\pgfpathlineto{\pgfqpoint{3.431497in}{2.594780in}}%
\pgfpathlineto{\pgfqpoint{3.444852in}{2.587690in}}%
\pgfpathlineto{\pgfqpoint{3.458210in}{2.580701in}}%
\pgfpathlineto{\pgfqpoint{3.466163in}{2.591915in}}%
\pgfpathlineto{\pgfqpoint{3.474110in}{2.603238in}}%
\pgfpathlineto{\pgfqpoint{3.482050in}{2.614673in}}%
\pgfpathlineto{\pgfqpoint{3.489985in}{2.626224in}}%
\pgfpathlineto{\pgfqpoint{3.476635in}{2.633390in}}%
\pgfpathlineto{\pgfqpoint{3.463289in}{2.640655in}}%
\pgfpathlineto{\pgfqpoint{3.449945in}{2.648021in}}%
\pgfpathlineto{\pgfqpoint{3.436604in}{2.655487in}}%
\pgfpathlineto{\pgfqpoint{3.428662in}{2.643752in}}%
\pgfpathlineto{\pgfqpoint{3.420713in}{2.632139in}}%
\pgfpathlineto{\pgfqpoint{3.412758in}{2.620642in}}%
\pgfpathlineto{\pgfqpoint{3.404797in}{2.609260in}}%
\pgfpathclose%
\pgfusepath{fill}%
\end{pgfscope}%
\begin{pgfscope}%
\pgfpathrectangle{\pgfqpoint{1.150000in}{0.150000in}}{\pgfqpoint{5.700000in}{5.700000in}}%
\pgfusepath{clip}%
\pgfsetbuttcap%
\pgfsetroundjoin%
\definecolor{currentfill}{rgb}{0.241237,0.296485,0.539709}%
\pgfsetfillcolor{currentfill}%
\pgfsetfillopacity{0.700000}%
\pgfsetlinewidth{0.000000pt}%
\definecolor{currentstroke}{rgb}{0.000000,0.000000,0.000000}%
\pgfsetstrokecolor{currentstroke}%
\pgfsetdash{}{0pt}%
\pgfpathmoveto{\pgfqpoint{4.861814in}{2.920120in}}%
\pgfpathlineto{\pgfqpoint{4.875417in}{2.914908in}}%
\pgfpathlineto{\pgfqpoint{4.889026in}{2.909769in}}%
\pgfpathlineto{\pgfqpoint{4.902641in}{2.904700in}}%
\pgfpathlineto{\pgfqpoint{4.916263in}{2.899704in}}%
\pgfpathlineto{\pgfqpoint{4.923866in}{2.916089in}}%
\pgfpathlineto{\pgfqpoint{4.931471in}{2.932811in}}%
\pgfpathlineto{\pgfqpoint{4.939079in}{2.949879in}}%
\pgfpathlineto{\pgfqpoint{4.946690in}{2.967301in}}%
\pgfpathlineto{\pgfqpoint{4.933079in}{2.972754in}}%
\pgfpathlineto{\pgfqpoint{4.919475in}{2.978279in}}%
\pgfpathlineto{\pgfqpoint{4.905878in}{2.983875in}}%
\pgfpathlineto{\pgfqpoint{4.892286in}{2.989543in}}%
\pgfpathlineto{\pgfqpoint{4.884664in}{2.971657in}}%
\pgfpathlineto{\pgfqpoint{4.877045in}{2.954130in}}%
\pgfpathlineto{\pgfqpoint{4.869428in}{2.936954in}}%
\pgfpathlineto{\pgfqpoint{4.861814in}{2.920120in}}%
\pgfpathclose%
\pgfusepath{fill}%
\end{pgfscope}%
\begin{pgfscope}%
\pgfpathrectangle{\pgfqpoint{1.150000in}{0.150000in}}{\pgfqpoint{5.700000in}{5.700000in}}%
\pgfusepath{clip}%
\pgfsetbuttcap%
\pgfsetroundjoin%
\definecolor{currentfill}{rgb}{0.280868,0.160771,0.472899}%
\pgfsetfillcolor{currentfill}%
\pgfsetfillopacity{0.700000}%
\pgfsetlinewidth{0.000000pt}%
\definecolor{currentstroke}{rgb}{0.000000,0.000000,0.000000}%
\pgfsetstrokecolor{currentstroke}%
\pgfsetdash{}{0pt}%
\pgfpathmoveto{\pgfqpoint{3.127338in}{2.648205in}}%
\pgfpathlineto{\pgfqpoint{3.140670in}{2.639494in}}%
\pgfpathlineto{\pgfqpoint{3.154004in}{2.630896in}}%
\pgfpathlineto{\pgfqpoint{3.167339in}{2.622410in}}%
\pgfpathlineto{\pgfqpoint{3.180676in}{2.614035in}}%
\pgfpathlineto{\pgfqpoint{3.188715in}{2.625001in}}%
\pgfpathlineto{\pgfqpoint{3.196748in}{2.636073in}}%
\pgfpathlineto{\pgfqpoint{3.204773in}{2.647253in}}%
\pgfpathlineto{\pgfqpoint{3.212792in}{2.658543in}}%
\pgfpathlineto{\pgfqpoint{3.199465in}{2.667053in}}%
\pgfpathlineto{\pgfqpoint{3.186138in}{2.675675in}}%
\pgfpathlineto{\pgfqpoint{3.172813in}{2.684409in}}%
\pgfpathlineto{\pgfqpoint{3.159490in}{2.693255in}}%
\pgfpathlineto{\pgfqpoint{3.151462in}{2.681822in}}%
\pgfpathlineto{\pgfqpoint{3.143427in}{2.670505in}}%
\pgfpathlineto{\pgfqpoint{3.135386in}{2.659300in}}%
\pgfpathlineto{\pgfqpoint{3.127338in}{2.648205in}}%
\pgfpathclose%
\pgfusepath{fill}%
\end{pgfscope}%
\begin{pgfscope}%
\pgfpathrectangle{\pgfqpoint{1.150000in}{0.150000in}}{\pgfqpoint{5.700000in}{5.700000in}}%
\pgfusepath{clip}%
\pgfsetbuttcap%
\pgfsetroundjoin%
\definecolor{currentfill}{rgb}{0.255645,0.260703,0.528312}%
\pgfsetfillcolor{currentfill}%
\pgfsetfillopacity{0.700000}%
\pgfsetlinewidth{0.000000pt}%
\definecolor{currentstroke}{rgb}{0.000000,0.000000,0.000000}%
\pgfsetstrokecolor{currentstroke}%
\pgfsetdash{}{0pt}%
\pgfpathmoveto{\pgfqpoint{4.692177in}{2.834270in}}%
\pgfpathlineto{\pgfqpoint{4.705752in}{2.829329in}}%
\pgfpathlineto{\pgfqpoint{4.719333in}{2.824462in}}%
\pgfpathlineto{\pgfqpoint{4.732920in}{2.819668in}}%
\pgfpathlineto{\pgfqpoint{4.746513in}{2.814947in}}%
\pgfpathlineto{\pgfqpoint{4.754130in}{2.829743in}}%
\pgfpathlineto{\pgfqpoint{4.761747in}{2.844822in}}%
\pgfpathlineto{\pgfqpoint{4.769364in}{2.860193in}}%
\pgfpathlineto{\pgfqpoint{4.776982in}{2.875863in}}%
\pgfpathlineto{\pgfqpoint{4.763400in}{2.881000in}}%
\pgfpathlineto{\pgfqpoint{4.749824in}{2.886210in}}%
\pgfpathlineto{\pgfqpoint{4.736254in}{2.891494in}}%
\pgfpathlineto{\pgfqpoint{4.722690in}{2.896851in}}%
\pgfpathlineto{\pgfqpoint{4.715061in}{2.880758in}}%
\pgfpathlineto{\pgfqpoint{4.707433in}{2.864968in}}%
\pgfpathlineto{\pgfqpoint{4.699805in}{2.849475in}}%
\pgfpathlineto{\pgfqpoint{4.692177in}{2.834270in}}%
\pgfpathclose%
\pgfusepath{fill}%
\end{pgfscope}%
\begin{pgfscope}%
\pgfpathrectangle{\pgfqpoint{1.150000in}{0.150000in}}{\pgfqpoint{5.700000in}{5.700000in}}%
\pgfusepath{clip}%
\pgfsetbuttcap%
\pgfsetroundjoin%
\definecolor{currentfill}{rgb}{0.231674,0.318106,0.544834}%
\pgfsetfillcolor{currentfill}%
\pgfsetfillopacity{0.700000}%
\pgfsetlinewidth{0.000000pt}%
\definecolor{currentstroke}{rgb}{0.000000,0.000000,0.000000}%
\pgfsetstrokecolor{currentstroke}%
\pgfsetdash{}{0pt}%
\pgfpathmoveto{\pgfqpoint{4.946690in}{2.967301in}}%
\pgfpathlineto{\pgfqpoint{4.960306in}{2.961919in}}%
\pgfpathlineto{\pgfqpoint{4.973929in}{2.956608in}}%
\pgfpathlineto{\pgfqpoint{4.987559in}{2.951368in}}%
\pgfpathlineto{\pgfqpoint{5.001195in}{2.946198in}}%
\pgfpathlineto{\pgfqpoint{5.008797in}{2.963513in}}%
\pgfpathlineto{\pgfqpoint{5.016403in}{2.981195in}}%
\pgfpathlineto{\pgfqpoint{5.024013in}{2.999251in}}%
\pgfpathlineto{\pgfqpoint{5.031628in}{3.017691in}}%
\pgfpathlineto{\pgfqpoint{5.018004in}{3.023338in}}%
\pgfpathlineto{\pgfqpoint{5.004387in}{3.029055in}}%
\pgfpathlineto{\pgfqpoint{4.990775in}{3.034843in}}%
\pgfpathlineto{\pgfqpoint{4.977170in}{3.040701in}}%
\pgfpathlineto{\pgfqpoint{4.969544in}{3.021777in}}%
\pgfpathlineto{\pgfqpoint{4.961922in}{3.003241in}}%
\pgfpathlineto{\pgfqpoint{4.954304in}{2.985085in}}%
\pgfpathlineto{\pgfqpoint{4.946690in}{2.967301in}}%
\pgfpathclose%
\pgfusepath{fill}%
\end{pgfscope}%
\begin{pgfscope}%
\pgfpathrectangle{\pgfqpoint{1.150000in}{0.150000in}}{\pgfqpoint{5.700000in}{5.700000in}}%
\pgfusepath{clip}%
\pgfsetbuttcap%
\pgfsetroundjoin%
\definecolor{currentfill}{rgb}{0.277134,0.185228,0.489898}%
\pgfsetfillcolor{currentfill}%
\pgfsetfillopacity{0.700000}%
\pgfsetlinewidth{0.000000pt}%
\definecolor{currentstroke}{rgb}{0.000000,0.000000,0.000000}%
\pgfsetstrokecolor{currentstroke}%
\pgfsetdash{}{0pt}%
\pgfpathmoveto{\pgfqpoint{4.214126in}{2.678464in}}%
\pgfpathlineto{\pgfqpoint{4.227607in}{2.673461in}}%
\pgfpathlineto{\pgfqpoint{4.241095in}{2.668538in}}%
\pgfpathlineto{\pgfqpoint{4.254587in}{2.663696in}}%
\pgfpathlineto{\pgfqpoint{4.268086in}{2.658932in}}%
\pgfpathlineto{\pgfqpoint{4.275804in}{2.671361in}}%
\pgfpathlineto{\pgfqpoint{4.283518in}{2.683971in}}%
\pgfpathlineto{\pgfqpoint{4.291230in}{2.696768in}}%
\pgfpathlineto{\pgfqpoint{4.298938in}{2.709757in}}%
\pgfpathlineto{\pgfqpoint{4.285449in}{2.714837in}}%
\pgfpathlineto{\pgfqpoint{4.271965in}{2.719996in}}%
\pgfpathlineto{\pgfqpoint{4.258487in}{2.725234in}}%
\pgfpathlineto{\pgfqpoint{4.245015in}{2.730553in}}%
\pgfpathlineto{\pgfqpoint{4.237298in}{2.717240in}}%
\pgfpathlineto{\pgfqpoint{4.229577in}{2.704125in}}%
\pgfpathlineto{\pgfqpoint{4.221854in}{2.691202in}}%
\pgfpathlineto{\pgfqpoint{4.214126in}{2.678464in}}%
\pgfpathclose%
\pgfusepath{fill}%
\end{pgfscope}%
\begin{pgfscope}%
\pgfpathrectangle{\pgfqpoint{1.150000in}{0.150000in}}{\pgfqpoint{5.700000in}{5.700000in}}%
\pgfusepath{clip}%
\pgfsetbuttcap%
\pgfsetroundjoin%
\definecolor{currentfill}{rgb}{0.282623,0.140926,0.457517}%
\pgfsetfillcolor{currentfill}%
\pgfsetfillopacity{0.700000}%
\pgfsetlinewidth{0.000000pt}%
\definecolor{currentstroke}{rgb}{0.000000,0.000000,0.000000}%
\pgfsetstrokecolor{currentstroke}%
\pgfsetdash{}{0pt}%
\pgfpathmoveto{\pgfqpoint{3.543415in}{2.598545in}}%
\pgfpathlineto{\pgfqpoint{3.556781in}{2.591868in}}%
\pgfpathlineto{\pgfqpoint{3.570151in}{2.585287in}}%
\pgfpathlineto{\pgfqpoint{3.583524in}{2.578801in}}%
\pgfpathlineto{\pgfqpoint{3.596901in}{2.572410in}}%
\pgfpathlineto{\pgfqpoint{3.604813in}{2.583704in}}%
\pgfpathlineto{\pgfqpoint{3.612720in}{2.595111in}}%
\pgfpathlineto{\pgfqpoint{3.620620in}{2.606634in}}%
\pgfpathlineto{\pgfqpoint{3.628515in}{2.618278in}}%
\pgfpathlineto{\pgfqpoint{3.615147in}{2.624865in}}%
\pgfpathlineto{\pgfqpoint{3.601782in}{2.631547in}}%
\pgfpathlineto{\pgfqpoint{3.588420in}{2.638324in}}%
\pgfpathlineto{\pgfqpoint{3.575062in}{2.645197in}}%
\pgfpathlineto{\pgfqpoint{3.567159in}{2.633349in}}%
\pgfpathlineto{\pgfqpoint{3.559251in}{2.621627in}}%
\pgfpathlineto{\pgfqpoint{3.551336in}{2.610027in}}%
\pgfpathlineto{\pgfqpoint{3.543415in}{2.598545in}}%
\pgfpathclose%
\pgfusepath{fill}%
\end{pgfscope}%
\begin{pgfscope}%
\pgfpathrectangle{\pgfqpoint{1.150000in}{0.150000in}}{\pgfqpoint{5.700000in}{5.700000in}}%
\pgfusepath{clip}%
\pgfsetbuttcap%
\pgfsetroundjoin%
\definecolor{currentfill}{rgb}{0.262138,0.242286,0.520837}%
\pgfsetfillcolor{currentfill}%
\pgfsetfillopacity{0.700000}%
\pgfsetlinewidth{0.000000pt}%
\definecolor{currentstroke}{rgb}{0.000000,0.000000,0.000000}%
\pgfsetstrokecolor{currentstroke}%
\pgfsetdash{}{0pt}%
\pgfpathmoveto{\pgfqpoint{4.607385in}{2.795108in}}%
\pgfpathlineto{\pgfqpoint{4.620945in}{2.790267in}}%
\pgfpathlineto{\pgfqpoint{4.634511in}{2.785500in}}%
\pgfpathlineto{\pgfqpoint{4.648084in}{2.780808in}}%
\pgfpathlineto{\pgfqpoint{4.661663in}{2.776189in}}%
\pgfpathlineto{\pgfqpoint{4.669293in}{2.790314in}}%
\pgfpathlineto{\pgfqpoint{4.676921in}{2.804697in}}%
\pgfpathlineto{\pgfqpoint{4.684550in}{2.819347in}}%
\pgfpathlineto{\pgfqpoint{4.692177in}{2.834270in}}%
\pgfpathlineto{\pgfqpoint{4.678609in}{2.839285in}}%
\pgfpathlineto{\pgfqpoint{4.665047in}{2.844374in}}%
\pgfpathlineto{\pgfqpoint{4.651491in}{2.849537in}}%
\pgfpathlineto{\pgfqpoint{4.637942in}{2.854774in}}%
\pgfpathlineto{\pgfqpoint{4.630303in}{2.839447in}}%
\pgfpathlineto{\pgfqpoint{4.622664in}{2.824399in}}%
\pgfpathlineto{\pgfqpoint{4.615025in}{2.809621in}}%
\pgfpathlineto{\pgfqpoint{4.607385in}{2.795108in}}%
\pgfpathclose%
\pgfusepath{fill}%
\end{pgfscope}%
\begin{pgfscope}%
\pgfpathrectangle{\pgfqpoint{1.150000in}{0.150000in}}{\pgfqpoint{5.700000in}{5.700000in}}%
\pgfusepath{clip}%
\pgfsetbuttcap%
\pgfsetroundjoin%
\definecolor{currentfill}{rgb}{0.169646,0.456262,0.558030}%
\pgfsetfillcolor{currentfill}%
\pgfsetfillopacity{0.700000}%
\pgfsetlinewidth{0.000000pt}%
\definecolor{currentstroke}{rgb}{0.000000,0.000000,0.000000}%
\pgfsetstrokecolor{currentstroke}%
\pgfsetdash{}{0pt}%
\pgfpathmoveto{\pgfqpoint{5.263228in}{3.313110in}}%
\pgfpathlineto{\pgfqpoint{5.276863in}{3.305978in}}%
\pgfpathlineto{\pgfqpoint{5.290503in}{3.298914in}}%
\pgfpathlineto{\pgfqpoint{5.304150in}{3.291919in}}%
\pgfpathlineto{\pgfqpoint{5.317802in}{3.284993in}}%
\pgfpathlineto{\pgfqpoint{5.325520in}{3.309717in}}%
\pgfpathlineto{\pgfqpoint{5.333250in}{3.334990in}}%
\pgfpathlineto{\pgfqpoint{5.340992in}{3.360822in}}%
\pgfpathlineto{\pgfqpoint{5.327348in}{3.368182in}}%
\pgfpathlineto{\pgfqpoint{5.313709in}{3.375611in}}%
\pgfpathlineto{\pgfqpoint{5.300076in}{3.383110in}}%
\pgfpathlineto{\pgfqpoint{5.286448in}{3.390677in}}%
\pgfpathlineto{\pgfqpoint{5.278696in}{3.364260in}}%
\pgfpathlineto{\pgfqpoint{5.270956in}{3.338408in}}%
\pgfpathlineto{\pgfqpoint{5.263228in}{3.313110in}}%
\pgfpathclose%
\pgfusepath{fill}%
\end{pgfscope}%
\begin{pgfscope}%
\pgfpathrectangle{\pgfqpoint{1.150000in}{0.150000in}}{\pgfqpoint{5.700000in}{5.700000in}}%
\pgfusepath{clip}%
\pgfsetbuttcap%
\pgfsetroundjoin%
\definecolor{currentfill}{rgb}{0.220057,0.343307,0.549413}%
\pgfsetfillcolor{currentfill}%
\pgfsetfillopacity{0.700000}%
\pgfsetlinewidth{0.000000pt}%
\definecolor{currentstroke}{rgb}{0.000000,0.000000,0.000000}%
\pgfsetstrokecolor{currentstroke}%
\pgfsetdash{}{0pt}%
\pgfpathmoveto{\pgfqpoint{5.031628in}{3.017691in}}%
\pgfpathlineto{\pgfqpoint{5.045258in}{3.012116in}}%
\pgfpathlineto{\pgfqpoint{5.058895in}{3.006610in}}%
\pgfpathlineto{\pgfqpoint{5.072538in}{3.001174in}}%
\pgfpathlineto{\pgfqpoint{5.086188in}{2.995809in}}%
\pgfpathlineto{\pgfqpoint{5.093796in}{3.014153in}}%
\pgfpathlineto{\pgfqpoint{5.101408in}{3.032893in}}%
\pgfpathlineto{\pgfqpoint{5.109026in}{3.052040in}}%
\pgfpathlineto{\pgfqpoint{5.116651in}{3.071602in}}%
\pgfpathlineto{\pgfqpoint{5.103013in}{3.077465in}}%
\pgfpathlineto{\pgfqpoint{5.089382in}{3.083398in}}%
\pgfpathlineto{\pgfqpoint{5.075757in}{3.089401in}}%
\pgfpathlineto{\pgfqpoint{5.062138in}{3.095474in}}%
\pgfpathlineto{\pgfqpoint{5.054502in}{3.075407in}}%
\pgfpathlineto{\pgfqpoint{5.046872in}{3.055760in}}%
\pgfpathlineto{\pgfqpoint{5.039248in}{3.036525in}}%
\pgfpathlineto{\pgfqpoint{5.031628in}{3.017691in}}%
\pgfpathclose%
\pgfusepath{fill}%
\end{pgfscope}%
\begin{pgfscope}%
\pgfpathrectangle{\pgfqpoint{1.150000in}{0.150000in}}{\pgfqpoint{5.700000in}{5.700000in}}%
\pgfusepath{clip}%
\pgfsetbuttcap%
\pgfsetroundjoin%
\definecolor{currentfill}{rgb}{0.270595,0.214069,0.507052}%
\pgfsetfillcolor{currentfill}%
\pgfsetfillopacity{0.700000}%
\pgfsetlinewidth{0.000000pt}%
\definecolor{currentstroke}{rgb}{0.000000,0.000000,0.000000}%
\pgfsetstrokecolor{currentstroke}%
\pgfsetdash{}{0pt}%
\pgfpathmoveto{\pgfqpoint{2.795785in}{2.758254in}}%
\pgfpathlineto{\pgfqpoint{2.809132in}{2.747295in}}%
\pgfpathlineto{\pgfqpoint{2.822479in}{2.736469in}}%
\pgfpathlineto{\pgfqpoint{2.835825in}{2.725774in}}%
\pgfpathlineto{\pgfqpoint{2.849170in}{2.715211in}}%
\pgfpathlineto{\pgfqpoint{2.857316in}{2.725864in}}%
\pgfpathlineto{\pgfqpoint{2.865455in}{2.736629in}}%
\pgfpathlineto{\pgfqpoint{2.873586in}{2.747508in}}%
\pgfpathlineto{\pgfqpoint{2.881709in}{2.758502in}}%
\pgfpathlineto{\pgfqpoint{2.868375in}{2.769160in}}%
\pgfpathlineto{\pgfqpoint{2.855040in}{2.779949in}}%
\pgfpathlineto{\pgfqpoint{2.841704in}{2.790871in}}%
\pgfpathlineto{\pgfqpoint{2.828367in}{2.801925in}}%
\pgfpathlineto{\pgfqpoint{2.820233in}{2.790828in}}%
\pgfpathlineto{\pgfqpoint{2.812092in}{2.779852in}}%
\pgfpathlineto{\pgfqpoint{2.803942in}{2.768995in}}%
\pgfpathlineto{\pgfqpoint{2.795785in}{2.758254in}}%
\pgfpathclose%
\pgfusepath{fill}%
\end{pgfscope}%
\begin{pgfscope}%
\pgfpathrectangle{\pgfqpoint{1.150000in}{0.150000in}}{\pgfqpoint{5.700000in}{5.700000in}}%
\pgfusepath{clip}%
\pgfsetbuttcap%
\pgfsetroundjoin%
\definecolor{currentfill}{rgb}{0.278012,0.180367,0.486697}%
\pgfsetfillcolor{currentfill}%
\pgfsetfillopacity{0.700000}%
\pgfsetlinewidth{0.000000pt}%
\definecolor{currentstroke}{rgb}{0.000000,0.000000,0.000000}%
\pgfsetstrokecolor{currentstroke}%
\pgfsetdash{}{0pt}%
\pgfpathmoveto{\pgfqpoint{2.988378in}{2.677808in}}%
\pgfpathlineto{\pgfqpoint{3.001712in}{2.668277in}}%
\pgfpathlineto{\pgfqpoint{3.015047in}{2.658866in}}%
\pgfpathlineto{\pgfqpoint{3.028383in}{2.649575in}}%
\pgfpathlineto{\pgfqpoint{3.041719in}{2.640402in}}%
\pgfpathlineto{\pgfqpoint{3.049805in}{2.651190in}}%
\pgfpathlineto{\pgfqpoint{3.057884in}{2.662084in}}%
\pgfpathlineto{\pgfqpoint{3.065956in}{2.673084in}}%
\pgfpathlineto{\pgfqpoint{3.074020in}{2.684195in}}%
\pgfpathlineto{\pgfqpoint{3.060693in}{2.693483in}}%
\pgfpathlineto{\pgfqpoint{3.047367in}{2.702890in}}%
\pgfpathlineto{\pgfqpoint{3.034042in}{2.712416in}}%
\pgfpathlineto{\pgfqpoint{3.020717in}{2.722063in}}%
\pgfpathlineto{\pgfqpoint{3.012643in}{2.710829in}}%
\pgfpathlineto{\pgfqpoint{3.004562in}{2.699710in}}%
\pgfpathlineto{\pgfqpoint{2.996474in}{2.688704in}}%
\pgfpathlineto{\pgfqpoint{2.988378in}{2.677808in}}%
\pgfpathclose%
\pgfusepath{fill}%
\end{pgfscope}%
\begin{pgfscope}%
\pgfpathrectangle{\pgfqpoint{1.150000in}{0.150000in}}{\pgfqpoint{5.700000in}{5.700000in}}%
\pgfusepath{clip}%
\pgfsetbuttcap%
\pgfsetroundjoin%
\definecolor{currentfill}{rgb}{0.267968,0.223549,0.512008}%
\pgfsetfillcolor{currentfill}%
\pgfsetfillopacity{0.700000}%
\pgfsetlinewidth{0.000000pt}%
\definecolor{currentstroke}{rgb}{0.000000,0.000000,0.000000}%
\pgfsetstrokecolor{currentstroke}%
\pgfsetdash{}{0pt}%
\pgfpathmoveto{\pgfqpoint{4.522590in}{2.758167in}}%
\pgfpathlineto{\pgfqpoint{4.536136in}{2.753401in}}%
\pgfpathlineto{\pgfqpoint{4.549689in}{2.748711in}}%
\pgfpathlineto{\pgfqpoint{4.563247in}{2.744096in}}%
\pgfpathlineto{\pgfqpoint{4.576812in}{2.739557in}}%
\pgfpathlineto{\pgfqpoint{4.584457in}{2.753083in}}%
\pgfpathlineto{\pgfqpoint{4.592101in}{2.766846in}}%
\pgfpathlineto{\pgfqpoint{4.599743in}{2.780852in}}%
\pgfpathlineto{\pgfqpoint{4.607385in}{2.795108in}}%
\pgfpathlineto{\pgfqpoint{4.593831in}{2.800024in}}%
\pgfpathlineto{\pgfqpoint{4.580283in}{2.805015in}}%
\pgfpathlineto{\pgfqpoint{4.566741in}{2.810081in}}%
\pgfpathlineto{\pgfqpoint{4.553205in}{2.815223in}}%
\pgfpathlineto{\pgfqpoint{4.545553in}{2.800583in}}%
\pgfpathlineto{\pgfqpoint{4.537901in}{2.786199in}}%
\pgfpathlineto{\pgfqpoint{4.530246in}{2.772062in}}%
\pgfpathlineto{\pgfqpoint{4.522590in}{2.758167in}}%
\pgfpathclose%
\pgfusepath{fill}%
\end{pgfscope}%
\begin{pgfscope}%
\pgfpathrectangle{\pgfqpoint{1.150000in}{0.150000in}}{\pgfqpoint{5.700000in}{5.700000in}}%
\pgfusepath{clip}%
\pgfsetbuttcap%
\pgfsetroundjoin%
\definecolor{currentfill}{rgb}{0.281887,0.150881,0.465405}%
\pgfsetfillcolor{currentfill}%
\pgfsetfillopacity{0.700000}%
\pgfsetlinewidth{0.000000pt}%
\definecolor{currentstroke}{rgb}{0.000000,0.000000,0.000000}%
\pgfsetstrokecolor{currentstroke}%
\pgfsetdash{}{0pt}%
\pgfpathmoveto{\pgfqpoint{3.905668in}{2.616102in}}%
\pgfpathlineto{\pgfqpoint{3.919093in}{2.610597in}}%
\pgfpathlineto{\pgfqpoint{3.932522in}{2.605178in}}%
\pgfpathlineto{\pgfqpoint{3.945957in}{2.599845in}}%
\pgfpathlineto{\pgfqpoint{3.959396in}{2.594597in}}%
\pgfpathlineto{\pgfqpoint{3.967202in}{2.606268in}}%
\pgfpathlineto{\pgfqpoint{3.975003in}{2.618079in}}%
\pgfpathlineto{\pgfqpoint{3.982799in}{2.630032in}}%
\pgfpathlineto{\pgfqpoint{3.990590in}{2.642133in}}%
\pgfpathlineto{\pgfqpoint{3.977160in}{2.647637in}}%
\pgfpathlineto{\pgfqpoint{3.963734in}{2.653226in}}%
\pgfpathlineto{\pgfqpoint{3.950313in}{2.658902in}}%
\pgfpathlineto{\pgfqpoint{3.936897in}{2.664663in}}%
\pgfpathlineto{\pgfqpoint{3.929097in}{2.652298in}}%
\pgfpathlineto{\pgfqpoint{3.921292in}{2.640086in}}%
\pgfpathlineto{\pgfqpoint{3.913483in}{2.628023in}}%
\pgfpathlineto{\pgfqpoint{3.905668in}{2.616102in}}%
\pgfpathclose%
\pgfusepath{fill}%
\end{pgfscope}%
\begin{pgfscope}%
\pgfpathrectangle{\pgfqpoint{1.150000in}{0.150000in}}{\pgfqpoint{5.700000in}{5.700000in}}%
\pgfusepath{clip}%
\pgfsetbuttcap%
\pgfsetroundjoin%
\definecolor{currentfill}{rgb}{0.182256,0.426184,0.557120}%
\pgfsetfillcolor{currentfill}%
\pgfsetfillopacity{0.700000}%
\pgfsetlinewidth{0.000000pt}%
\definecolor{currentstroke}{rgb}{0.000000,0.000000,0.000000}%
\pgfsetstrokecolor{currentstroke}%
\pgfsetdash{}{0pt}%
\pgfpathmoveto{\pgfqpoint{5.232427in}{3.217241in}}%
\pgfpathlineto{\pgfqpoint{5.246072in}{3.210668in}}%
\pgfpathlineto{\pgfqpoint{5.259724in}{3.204165in}}%
\pgfpathlineto{\pgfqpoint{5.273382in}{3.197731in}}%
\pgfpathlineto{\pgfqpoint{5.287046in}{3.191365in}}%
\pgfpathlineto{\pgfqpoint{5.294719in}{3.214004in}}%
\pgfpathlineto{\pgfqpoint{5.302402in}{3.237148in}}%
\pgfpathlineto{\pgfqpoint{5.310097in}{3.260807in}}%
\pgfpathlineto{\pgfqpoint{5.317802in}{3.284993in}}%
\pgfpathlineto{\pgfqpoint{5.304150in}{3.291919in}}%
\pgfpathlineto{\pgfqpoint{5.290503in}{3.298914in}}%
\pgfpathlineto{\pgfqpoint{5.276863in}{3.305978in}}%
\pgfpathlineto{\pgfqpoint{5.263228in}{3.313110in}}%
\pgfpathlineto{\pgfqpoint{5.255512in}{3.288356in}}%
\pgfpathlineto{\pgfqpoint{5.247806in}{3.264133in}}%
\pgfpathlineto{\pgfqpoint{5.240111in}{3.240432in}}%
\pgfpathlineto{\pgfqpoint{5.232427in}{3.217241in}}%
\pgfpathclose%
\pgfusepath{fill}%
\end{pgfscope}%
\begin{pgfscope}%
\pgfpathrectangle{\pgfqpoint{1.150000in}{0.150000in}}{\pgfqpoint{5.700000in}{5.700000in}}%
\pgfusepath{clip}%
\pgfsetbuttcap%
\pgfsetroundjoin%
\definecolor{currentfill}{rgb}{0.282623,0.140926,0.457517}%
\pgfsetfillcolor{currentfill}%
\pgfsetfillopacity{0.700000}%
\pgfsetlinewidth{0.000000pt}%
\definecolor{currentstroke}{rgb}{0.000000,0.000000,0.000000}%
\pgfsetstrokecolor{currentstroke}%
\pgfsetdash{}{0pt}%
\pgfpathmoveto{\pgfqpoint{3.682028in}{2.592868in}}%
\pgfpathlineto{\pgfqpoint{3.695416in}{2.586747in}}%
\pgfpathlineto{\pgfqpoint{3.708808in}{2.580718in}}%
\pgfpathlineto{\pgfqpoint{3.722204in}{2.574780in}}%
\pgfpathlineto{\pgfqpoint{3.735605in}{2.568932in}}%
\pgfpathlineto{\pgfqpoint{3.743478in}{2.580284in}}%
\pgfpathlineto{\pgfqpoint{3.751345in}{2.591754in}}%
\pgfpathlineto{\pgfqpoint{3.759207in}{2.603346in}}%
\pgfpathlineto{\pgfqpoint{3.767063in}{2.615065in}}%
\pgfpathlineto{\pgfqpoint{3.753671in}{2.621129in}}%
\pgfpathlineto{\pgfqpoint{3.740283in}{2.627283in}}%
\pgfpathlineto{\pgfqpoint{3.726899in}{2.633528in}}%
\pgfpathlineto{\pgfqpoint{3.713520in}{2.639865in}}%
\pgfpathlineto{\pgfqpoint{3.705655in}{2.627923in}}%
\pgfpathlineto{\pgfqpoint{3.697785in}{2.616112in}}%
\pgfpathlineto{\pgfqpoint{3.689909in}{2.604428in}}%
\pgfpathlineto{\pgfqpoint{3.682028in}{2.592868in}}%
\pgfpathclose%
\pgfusepath{fill}%
\end{pgfscope}%
\begin{pgfscope}%
\pgfpathrectangle{\pgfqpoint{1.150000in}{0.150000in}}{\pgfqpoint{5.700000in}{5.700000in}}%
\pgfusepath{clip}%
\pgfsetbuttcap%
\pgfsetroundjoin%
\definecolor{currentfill}{rgb}{0.208623,0.367752,0.552675}%
\pgfsetfillcolor{currentfill}%
\pgfsetfillopacity{0.700000}%
\pgfsetlinewidth{0.000000pt}%
\definecolor{currentstroke}{rgb}{0.000000,0.000000,0.000000}%
\pgfsetstrokecolor{currentstroke}%
\pgfsetdash{}{0pt}%
\pgfpathmoveto{\pgfqpoint{5.116651in}{3.071602in}}%
\pgfpathlineto{\pgfqpoint{5.130295in}{3.065809in}}%
\pgfpathlineto{\pgfqpoint{5.143945in}{3.060086in}}%
\pgfpathlineto{\pgfqpoint{5.157602in}{3.054432in}}%
\pgfpathlineto{\pgfqpoint{5.171265in}{3.048847in}}%
\pgfpathlineto{\pgfqpoint{5.178883in}{3.068325in}}%
\pgfpathlineto{\pgfqpoint{5.186509in}{3.088231in}}%
\pgfpathlineto{\pgfqpoint{5.194141in}{3.108577in}}%
\pgfpathlineto{\pgfqpoint{5.201781in}{3.129372in}}%
\pgfpathlineto{\pgfqpoint{5.188130in}{3.135474in}}%
\pgfpathlineto{\pgfqpoint{5.174486in}{3.141646in}}%
\pgfpathlineto{\pgfqpoint{5.160847in}{3.147888in}}%
\pgfpathlineto{\pgfqpoint{5.147215in}{3.154199in}}%
\pgfpathlineto{\pgfqpoint{5.139563in}{3.132878in}}%
\pgfpathlineto{\pgfqpoint{5.131919in}{3.112012in}}%
\pgfpathlineto{\pgfqpoint{5.124281in}{3.091590in}}%
\pgfpathlineto{\pgfqpoint{5.116651in}{3.071602in}}%
\pgfpathclose%
\pgfusepath{fill}%
\end{pgfscope}%
\begin{pgfscope}%
\pgfpathrectangle{\pgfqpoint{1.150000in}{0.150000in}}{\pgfqpoint{5.700000in}{5.700000in}}%
\pgfusepath{clip}%
\pgfsetbuttcap%
\pgfsetroundjoin%
\definecolor{currentfill}{rgb}{0.279574,0.170599,0.479997}%
\pgfsetfillcolor{currentfill}%
\pgfsetfillopacity{0.700000}%
\pgfsetlinewidth{0.000000pt}%
\definecolor{currentstroke}{rgb}{0.000000,0.000000,0.000000}%
\pgfsetstrokecolor{currentstroke}%
\pgfsetdash{}{0pt}%
\pgfpathmoveto{\pgfqpoint{4.129271in}{2.648894in}}%
\pgfpathlineto{\pgfqpoint{4.142739in}{2.643864in}}%
\pgfpathlineto{\pgfqpoint{4.156213in}{2.638915in}}%
\pgfpathlineto{\pgfqpoint{4.169693in}{2.634047in}}%
\pgfpathlineto{\pgfqpoint{4.183178in}{2.629260in}}%
\pgfpathlineto{\pgfqpoint{4.190921in}{2.641310in}}%
\pgfpathlineto{\pgfqpoint{4.198660in}{2.653524in}}%
\pgfpathlineto{\pgfqpoint{4.206395in}{2.665906in}}%
\pgfpathlineto{\pgfqpoint{4.214126in}{2.678464in}}%
\pgfpathlineto{\pgfqpoint{4.200650in}{2.683547in}}%
\pgfpathlineto{\pgfqpoint{4.187180in}{2.688711in}}%
\pgfpathlineto{\pgfqpoint{4.173715in}{2.693956in}}%
\pgfpathlineto{\pgfqpoint{4.160255in}{2.699283in}}%
\pgfpathlineto{\pgfqpoint{4.152515in}{2.686422in}}%
\pgfpathlineto{\pgfqpoint{4.144771in}{2.673741in}}%
\pgfpathlineto{\pgfqpoint{4.137023in}{2.661233in}}%
\pgfpathlineto{\pgfqpoint{4.129271in}{2.648894in}}%
\pgfpathclose%
\pgfusepath{fill}%
\end{pgfscope}%
\begin{pgfscope}%
\pgfpathrectangle{\pgfqpoint{1.150000in}{0.150000in}}{\pgfqpoint{5.700000in}{5.700000in}}%
\pgfusepath{clip}%
\pgfsetbuttcap%
\pgfsetroundjoin%
\definecolor{currentfill}{rgb}{0.271828,0.209303,0.504434}%
\pgfsetfillcolor{currentfill}%
\pgfsetfillopacity{0.700000}%
\pgfsetlinewidth{0.000000pt}%
\definecolor{currentstroke}{rgb}{0.000000,0.000000,0.000000}%
\pgfsetstrokecolor{currentstroke}%
\pgfsetdash{}{0pt}%
\pgfpathmoveto{\pgfqpoint{4.437783in}{2.723260in}}%
\pgfpathlineto{\pgfqpoint{4.451315in}{2.718546in}}%
\pgfpathlineto{\pgfqpoint{4.464853in}{2.713909in}}%
\pgfpathlineto{\pgfqpoint{4.478397in}{2.709348in}}%
\pgfpathlineto{\pgfqpoint{4.491948in}{2.704863in}}%
\pgfpathlineto{\pgfqpoint{4.499611in}{2.717860in}}%
\pgfpathlineto{\pgfqpoint{4.507273in}{2.731072in}}%
\pgfpathlineto{\pgfqpoint{4.514933in}{2.744506in}}%
\pgfpathlineto{\pgfqpoint{4.522590in}{2.758167in}}%
\pgfpathlineto{\pgfqpoint{4.509050in}{2.763008in}}%
\pgfpathlineto{\pgfqpoint{4.495516in}{2.767925in}}%
\pgfpathlineto{\pgfqpoint{4.481989in}{2.772919in}}%
\pgfpathlineto{\pgfqpoint{4.468467in}{2.777989in}}%
\pgfpathlineto{\pgfqpoint{4.460799in}{2.763964in}}%
\pgfpathlineto{\pgfqpoint{4.453129in}{2.750172in}}%
\pgfpathlineto{\pgfqpoint{4.445457in}{2.736606in}}%
\pgfpathlineto{\pgfqpoint{4.437783in}{2.723260in}}%
\pgfpathclose%
\pgfusepath{fill}%
\end{pgfscope}%
\begin{pgfscope}%
\pgfpathrectangle{\pgfqpoint{1.150000in}{0.150000in}}{\pgfqpoint{5.700000in}{5.700000in}}%
\pgfusepath{clip}%
\pgfsetbuttcap%
\pgfsetroundjoin%
\definecolor{currentfill}{rgb}{0.282623,0.140926,0.457517}%
\pgfsetfillcolor{currentfill}%
\pgfsetfillopacity{0.700000}%
\pgfsetlinewidth{0.000000pt}%
\definecolor{currentstroke}{rgb}{0.000000,0.000000,0.000000}%
\pgfsetstrokecolor{currentstroke}%
\pgfsetdash{}{0pt}%
\pgfpathmoveto{\pgfqpoint{3.319486in}{2.594367in}}%
\pgfpathlineto{\pgfqpoint{3.332833in}{2.586822in}}%
\pgfpathlineto{\pgfqpoint{3.346183in}{2.579380in}}%
\pgfpathlineto{\pgfqpoint{3.359535in}{2.572042in}}%
\pgfpathlineto{\pgfqpoint{3.372890in}{2.564806in}}%
\pgfpathlineto{\pgfqpoint{3.380876in}{2.575764in}}%
\pgfpathlineto{\pgfqpoint{3.388856in}{2.586824in}}%
\pgfpathlineto{\pgfqpoint{3.396829in}{2.597988in}}%
\pgfpathlineto{\pgfqpoint{3.404797in}{2.609260in}}%
\pgfpathlineto{\pgfqpoint{3.391450in}{2.616652in}}%
\pgfpathlineto{\pgfqpoint{3.378107in}{2.624146in}}%
\pgfpathlineto{\pgfqpoint{3.364766in}{2.631743in}}%
\pgfpathlineto{\pgfqpoint{3.351428in}{2.639444in}}%
\pgfpathlineto{\pgfqpoint{3.343452in}{2.628009in}}%
\pgfpathlineto{\pgfqpoint{3.335470in}{2.616687in}}%
\pgfpathlineto{\pgfqpoint{3.327481in}{2.605473in}}%
\pgfpathlineto{\pgfqpoint{3.319486in}{2.594367in}}%
\pgfpathclose%
\pgfusepath{fill}%
\end{pgfscope}%
\begin{pgfscope}%
\pgfpathrectangle{\pgfqpoint{1.150000in}{0.150000in}}{\pgfqpoint{5.700000in}{5.700000in}}%
\pgfusepath{clip}%
\pgfsetbuttcap%
\pgfsetroundjoin%
\definecolor{currentfill}{rgb}{0.281887,0.150881,0.465405}%
\pgfsetfillcolor{currentfill}%
\pgfsetfillopacity{0.700000}%
\pgfsetlinewidth{0.000000pt}%
\definecolor{currentstroke}{rgb}{0.000000,0.000000,0.000000}%
\pgfsetstrokecolor{currentstroke}%
\pgfsetdash{}{0pt}%
\pgfpathmoveto{\pgfqpoint{3.180676in}{2.614035in}}%
\pgfpathlineto{\pgfqpoint{3.194015in}{2.605770in}}%
\pgfpathlineto{\pgfqpoint{3.207356in}{2.597615in}}%
\pgfpathlineto{\pgfqpoint{3.220698in}{2.589569in}}%
\pgfpathlineto{\pgfqpoint{3.234043in}{2.581631in}}%
\pgfpathlineto{\pgfqpoint{3.242072in}{2.592470in}}%
\pgfpathlineto{\pgfqpoint{3.250096in}{2.603409in}}%
\pgfpathlineto{\pgfqpoint{3.258112in}{2.614450in}}%
\pgfpathlineto{\pgfqpoint{3.266123in}{2.625597in}}%
\pgfpathlineto{\pgfqpoint{3.252787in}{2.633670in}}%
\pgfpathlineto{\pgfqpoint{3.239454in}{2.641852in}}%
\pgfpathlineto{\pgfqpoint{3.226122in}{2.650143in}}%
\pgfpathlineto{\pgfqpoint{3.212792in}{2.658543in}}%
\pgfpathlineto{\pgfqpoint{3.204773in}{2.647253in}}%
\pgfpathlineto{\pgfqpoint{3.196748in}{2.636073in}}%
\pgfpathlineto{\pgfqpoint{3.188715in}{2.625001in}}%
\pgfpathlineto{\pgfqpoint{3.180676in}{2.614035in}}%
\pgfpathclose%
\pgfusepath{fill}%
\end{pgfscope}%
\begin{pgfscope}%
\pgfpathrectangle{\pgfqpoint{1.150000in}{0.150000in}}{\pgfqpoint{5.700000in}{5.700000in}}%
\pgfusepath{clip}%
\pgfsetbuttcap%
\pgfsetroundjoin%
\definecolor{currentfill}{rgb}{0.274128,0.199721,0.498911}%
\pgfsetfillcolor{currentfill}%
\pgfsetfillopacity{0.700000}%
\pgfsetlinewidth{0.000000pt}%
\definecolor{currentstroke}{rgb}{0.000000,0.000000,0.000000}%
\pgfsetstrokecolor{currentstroke}%
\pgfsetdash{}{0pt}%
\pgfpathmoveto{\pgfqpoint{2.849170in}{2.715211in}}%
\pgfpathlineto{\pgfqpoint{2.862514in}{2.704777in}}%
\pgfpathlineto{\pgfqpoint{2.875858in}{2.694472in}}%
\pgfpathlineto{\pgfqpoint{2.889202in}{2.684294in}}%
\pgfpathlineto{\pgfqpoint{2.902546in}{2.674243in}}%
\pgfpathlineto{\pgfqpoint{2.910682in}{2.684808in}}%
\pgfpathlineto{\pgfqpoint{2.918810in}{2.695481in}}%
\pgfpathlineto{\pgfqpoint{2.926930in}{2.706261in}}%
\pgfpathlineto{\pgfqpoint{2.935044in}{2.717153in}}%
\pgfpathlineto{\pgfqpoint{2.921710in}{2.727299in}}%
\pgfpathlineto{\pgfqpoint{2.908377in}{2.737572in}}%
\pgfpathlineto{\pgfqpoint{2.895043in}{2.747973in}}%
\pgfpathlineto{\pgfqpoint{2.881709in}{2.758502in}}%
\pgfpathlineto{\pgfqpoint{2.873586in}{2.747508in}}%
\pgfpathlineto{\pgfqpoint{2.865455in}{2.736629in}}%
\pgfpathlineto{\pgfqpoint{2.857316in}{2.725864in}}%
\pgfpathlineto{\pgfqpoint{2.849170in}{2.715211in}}%
\pgfpathclose%
\pgfusepath{fill}%
\end{pgfscope}%
\begin{pgfscope}%
\pgfpathrectangle{\pgfqpoint{1.150000in}{0.150000in}}{\pgfqpoint{5.700000in}{5.700000in}}%
\pgfusepath{clip}%
\pgfsetbuttcap%
\pgfsetroundjoin%
\definecolor{currentfill}{rgb}{0.282884,0.135920,0.453427}%
\pgfsetfillcolor{currentfill}%
\pgfsetfillopacity{0.700000}%
\pgfsetlinewidth{0.000000pt}%
\definecolor{currentstroke}{rgb}{0.000000,0.000000,0.000000}%
\pgfsetstrokecolor{currentstroke}%
\pgfsetdash{}{0pt}%
\pgfpathmoveto{\pgfqpoint{3.458210in}{2.580701in}}%
\pgfpathlineto{\pgfqpoint{3.471571in}{2.573810in}}%
\pgfpathlineto{\pgfqpoint{3.484936in}{2.567017in}}%
\pgfpathlineto{\pgfqpoint{3.498304in}{2.560323in}}%
\pgfpathlineto{\pgfqpoint{3.511675in}{2.553725in}}%
\pgfpathlineto{\pgfqpoint{3.519619in}{2.564771in}}%
\pgfpathlineto{\pgfqpoint{3.527557in}{2.575920in}}%
\pgfpathlineto{\pgfqpoint{3.535489in}{2.587177in}}%
\pgfpathlineto{\pgfqpoint{3.543415in}{2.598545in}}%
\pgfpathlineto{\pgfqpoint{3.530053in}{2.605318in}}%
\pgfpathlineto{\pgfqpoint{3.516694in}{2.612189in}}%
\pgfpathlineto{\pgfqpoint{3.503338in}{2.619158in}}%
\pgfpathlineto{\pgfqpoint{3.489985in}{2.626224in}}%
\pgfpathlineto{\pgfqpoint{3.482050in}{2.614673in}}%
\pgfpathlineto{\pgfqpoint{3.474110in}{2.603238in}}%
\pgfpathlineto{\pgfqpoint{3.466163in}{2.591915in}}%
\pgfpathlineto{\pgfqpoint{3.458210in}{2.580701in}}%
\pgfpathclose%
\pgfusepath{fill}%
\end{pgfscope}%
\begin{pgfscope}%
\pgfpathrectangle{\pgfqpoint{1.150000in}{0.150000in}}{\pgfqpoint{5.700000in}{5.700000in}}%
\pgfusepath{clip}%
\pgfsetbuttcap%
\pgfsetroundjoin%
\definecolor{currentfill}{rgb}{0.197636,0.391528,0.554969}%
\pgfsetfillcolor{currentfill}%
\pgfsetfillopacity{0.700000}%
\pgfsetlinewidth{0.000000pt}%
\definecolor{currentstroke}{rgb}{0.000000,0.000000,0.000000}%
\pgfsetstrokecolor{currentstroke}%
\pgfsetdash{}{0pt}%
\pgfpathmoveto{\pgfqpoint{5.201781in}{3.129372in}}%
\pgfpathlineto{\pgfqpoint{5.215438in}{3.123338in}}%
\pgfpathlineto{\pgfqpoint{5.229102in}{3.117374in}}%
\pgfpathlineto{\pgfqpoint{5.242772in}{3.111478in}}%
\pgfpathlineto{\pgfqpoint{5.256449in}{3.105651in}}%
\pgfpathlineto{\pgfqpoint{5.264085in}{3.126374in}}%
\pgfpathlineto{\pgfqpoint{5.271729in}{3.147560in}}%
\pgfpathlineto{\pgfqpoint{5.279383in}{3.169221in}}%
\pgfpathlineto{\pgfqpoint{5.287046in}{3.191365in}}%
\pgfpathlineto{\pgfqpoint{5.273382in}{3.197731in}}%
\pgfpathlineto{\pgfqpoint{5.259724in}{3.204165in}}%
\pgfpathlineto{\pgfqpoint{5.246072in}{3.210668in}}%
\pgfpathlineto{\pgfqpoint{5.232427in}{3.217241in}}%
\pgfpathlineto{\pgfqpoint{5.224752in}{3.194550in}}%
\pgfpathlineto{\pgfqpoint{5.217086in}{3.172348in}}%
\pgfpathlineto{\pgfqpoint{5.209429in}{3.150625in}}%
\pgfpathlineto{\pgfqpoint{5.201781in}{3.129372in}}%
\pgfpathclose%
\pgfusepath{fill}%
\end{pgfscope}%
\begin{pgfscope}%
\pgfpathrectangle{\pgfqpoint{1.150000in}{0.150000in}}{\pgfqpoint{5.700000in}{5.700000in}}%
\pgfusepath{clip}%
\pgfsetbuttcap%
\pgfsetroundjoin%
\definecolor{currentfill}{rgb}{0.275191,0.194905,0.496005}%
\pgfsetfillcolor{currentfill}%
\pgfsetfillopacity{0.700000}%
\pgfsetlinewidth{0.000000pt}%
\definecolor{currentstroke}{rgb}{0.000000,0.000000,0.000000}%
\pgfsetstrokecolor{currentstroke}%
\pgfsetdash{}{0pt}%
\pgfpathmoveto{\pgfqpoint{4.352951in}{2.690228in}}%
\pgfpathlineto{\pgfqpoint{4.366469in}{2.685541in}}%
\pgfpathlineto{\pgfqpoint{4.379993in}{2.680932in}}%
\pgfpathlineto{\pgfqpoint{4.393523in}{2.676401in}}%
\pgfpathlineto{\pgfqpoint{4.407059in}{2.671946in}}%
\pgfpathlineto{\pgfqpoint{4.414744in}{2.684477in}}%
\pgfpathlineto{\pgfqpoint{4.422427in}{2.697202in}}%
\pgfpathlineto{\pgfqpoint{4.430106in}{2.710127in}}%
\pgfpathlineto{\pgfqpoint{4.437783in}{2.723260in}}%
\pgfpathlineto{\pgfqpoint{4.424257in}{2.728051in}}%
\pgfpathlineto{\pgfqpoint{4.410737in}{2.732919in}}%
\pgfpathlineto{\pgfqpoint{4.397223in}{2.737864in}}%
\pgfpathlineto{\pgfqpoint{4.383715in}{2.742887in}}%
\pgfpathlineto{\pgfqpoint{4.376028in}{2.729410in}}%
\pgfpathlineto{\pgfqpoint{4.368338in}{2.716146in}}%
\pgfpathlineto{\pgfqpoint{4.360646in}{2.703087in}}%
\pgfpathlineto{\pgfqpoint{4.352951in}{2.690228in}}%
\pgfpathclose%
\pgfusepath{fill}%
\end{pgfscope}%
\begin{pgfscope}%
\pgfpathrectangle{\pgfqpoint{1.150000in}{0.150000in}}{\pgfqpoint{5.700000in}{5.700000in}}%
\pgfusepath{clip}%
\pgfsetbuttcap%
\pgfsetroundjoin%
\definecolor{currentfill}{rgb}{0.282290,0.145912,0.461510}%
\pgfsetfillcolor{currentfill}%
\pgfsetfillopacity{0.700000}%
\pgfsetlinewidth{0.000000pt}%
\definecolor{currentstroke}{rgb}{0.000000,0.000000,0.000000}%
\pgfsetstrokecolor{currentstroke}%
\pgfsetdash{}{0pt}%
\pgfpathmoveto{\pgfqpoint{3.820676in}{2.591707in}}%
\pgfpathlineto{\pgfqpoint{3.834090in}{2.586090in}}%
\pgfpathlineto{\pgfqpoint{3.847509in}{2.580560in}}%
\pgfpathlineto{\pgfqpoint{3.860933in}{2.575118in}}%
\pgfpathlineto{\pgfqpoint{3.874361in}{2.569762in}}%
\pgfpathlineto{\pgfqpoint{3.882196in}{2.581155in}}%
\pgfpathlineto{\pgfqpoint{3.890025in}{2.592673in}}%
\pgfpathlineto{\pgfqpoint{3.897849in}{2.604321in}}%
\pgfpathlineto{\pgfqpoint{3.905668in}{2.616102in}}%
\pgfpathlineto{\pgfqpoint{3.892248in}{2.621694in}}%
\pgfpathlineto{\pgfqpoint{3.878833in}{2.627373in}}%
\pgfpathlineto{\pgfqpoint{3.865423in}{2.633138in}}%
\pgfpathlineto{\pgfqpoint{3.852017in}{2.638992in}}%
\pgfpathlineto{\pgfqpoint{3.844189in}{2.626967in}}%
\pgfpathlineto{\pgfqpoint{3.836357in}{2.615081in}}%
\pgfpathlineto{\pgfqpoint{3.828519in}{2.603329in}}%
\pgfpathlineto{\pgfqpoint{3.820676in}{2.591707in}}%
\pgfpathclose%
\pgfusepath{fill}%
\end{pgfscope}%
\begin{pgfscope}%
\pgfpathrectangle{\pgfqpoint{1.150000in}{0.150000in}}{\pgfqpoint{5.700000in}{5.700000in}}%
\pgfusepath{clip}%
\pgfsetbuttcap%
\pgfsetroundjoin%
\definecolor{currentfill}{rgb}{0.280868,0.160771,0.472899}%
\pgfsetfillcolor{currentfill}%
\pgfsetfillopacity{0.700000}%
\pgfsetlinewidth{0.000000pt}%
\definecolor{currentstroke}{rgb}{0.000000,0.000000,0.000000}%
\pgfsetstrokecolor{currentstroke}%
\pgfsetdash{}{0pt}%
\pgfpathmoveto{\pgfqpoint{4.044363in}{2.620960in}}%
\pgfpathlineto{\pgfqpoint{4.057820in}{2.615876in}}%
\pgfpathlineto{\pgfqpoint{4.071281in}{2.610875in}}%
\pgfpathlineto{\pgfqpoint{4.084748in}{2.605957in}}%
\pgfpathlineto{\pgfqpoint{4.098220in}{2.601121in}}%
\pgfpathlineto{\pgfqpoint{4.105989in}{2.612837in}}%
\pgfpathlineto{\pgfqpoint{4.113754in}{2.624702in}}%
\pgfpathlineto{\pgfqpoint{4.121515in}{2.636719in}}%
\pgfpathlineto{\pgfqpoint{4.129271in}{2.648894in}}%
\pgfpathlineto{\pgfqpoint{4.115808in}{2.654007in}}%
\pgfpathlineto{\pgfqpoint{4.102350in}{2.659201in}}%
\pgfpathlineto{\pgfqpoint{4.088897in}{2.664479in}}%
\pgfpathlineto{\pgfqpoint{4.075450in}{2.669839in}}%
\pgfpathlineto{\pgfqpoint{4.067685in}{2.657380in}}%
\pgfpathlineto{\pgfqpoint{4.059915in}{2.645084in}}%
\pgfpathlineto{\pgfqpoint{4.052142in}{2.632946in}}%
\pgfpathlineto{\pgfqpoint{4.044363in}{2.620960in}}%
\pgfpathclose%
\pgfusepath{fill}%
\end{pgfscope}%
\begin{pgfscope}%
\pgfpathrectangle{\pgfqpoint{1.150000in}{0.150000in}}{\pgfqpoint{5.700000in}{5.700000in}}%
\pgfusepath{clip}%
\pgfsetbuttcap%
\pgfsetroundjoin%
\definecolor{currentfill}{rgb}{0.280255,0.165693,0.476498}%
\pgfsetfillcolor{currentfill}%
\pgfsetfillopacity{0.700000}%
\pgfsetlinewidth{0.000000pt}%
\definecolor{currentstroke}{rgb}{0.000000,0.000000,0.000000}%
\pgfsetstrokecolor{currentstroke}%
\pgfsetdash{}{0pt}%
\pgfpathmoveto{\pgfqpoint{3.041719in}{2.640402in}}%
\pgfpathlineto{\pgfqpoint{3.055057in}{2.631346in}}%
\pgfpathlineto{\pgfqpoint{3.068395in}{2.622407in}}%
\pgfpathlineto{\pgfqpoint{3.081735in}{2.613583in}}%
\pgfpathlineto{\pgfqpoint{3.095075in}{2.604874in}}%
\pgfpathlineto{\pgfqpoint{3.103151in}{2.615554in}}%
\pgfpathlineto{\pgfqpoint{3.111220in}{2.626334in}}%
\pgfpathlineto{\pgfqpoint{3.119282in}{2.637217in}}%
\pgfpathlineto{\pgfqpoint{3.127338in}{2.648205in}}%
\pgfpathlineto{\pgfqpoint{3.114007in}{2.657029in}}%
\pgfpathlineto{\pgfqpoint{3.100677in}{2.665969in}}%
\pgfpathlineto{\pgfqpoint{3.087348in}{2.675024in}}%
\pgfpathlineto{\pgfqpoint{3.074020in}{2.684195in}}%
\pgfpathlineto{\pgfqpoint{3.065956in}{2.673084in}}%
\pgfpathlineto{\pgfqpoint{3.057884in}{2.662084in}}%
\pgfpathlineto{\pgfqpoint{3.049805in}{2.651190in}}%
\pgfpathlineto{\pgfqpoint{3.041719in}{2.640402in}}%
\pgfpathclose%
\pgfusepath{fill}%
\end{pgfscope}%
\begin{pgfscope}%
\pgfpathrectangle{\pgfqpoint{1.150000in}{0.150000in}}{\pgfqpoint{5.700000in}{5.700000in}}%
\pgfusepath{clip}%
\pgfsetbuttcap%
\pgfsetroundjoin%
\definecolor{currentfill}{rgb}{0.241237,0.296485,0.539709}%
\pgfsetfillcolor{currentfill}%
\pgfsetfillopacity{0.700000}%
\pgfsetlinewidth{0.000000pt}%
\definecolor{currentstroke}{rgb}{0.000000,0.000000,0.000000}%
\pgfsetstrokecolor{currentstroke}%
\pgfsetdash{}{0pt}%
\pgfpathmoveto{\pgfqpoint{4.916263in}{2.899704in}}%
\pgfpathlineto{\pgfqpoint{4.929892in}{2.894778in}}%
\pgfpathlineto{\pgfqpoint{4.943527in}{2.889924in}}%
\pgfpathlineto{\pgfqpoint{4.957169in}{2.885141in}}%
\pgfpathlineto{\pgfqpoint{4.970818in}{2.880428in}}%
\pgfpathlineto{\pgfqpoint{4.978408in}{2.896364in}}%
\pgfpathlineto{\pgfqpoint{4.986000in}{2.912632in}}%
\pgfpathlineto{\pgfqpoint{4.993596in}{2.929241in}}%
\pgfpathlineto{\pgfqpoint{5.001195in}{2.946198in}}%
\pgfpathlineto{\pgfqpoint{4.987559in}{2.951368in}}%
\pgfpathlineto{\pgfqpoint{4.973929in}{2.956608in}}%
\pgfpathlineto{\pgfqpoint{4.960306in}{2.961919in}}%
\pgfpathlineto{\pgfqpoint{4.946690in}{2.967301in}}%
\pgfpathlineto{\pgfqpoint{4.939079in}{2.949879in}}%
\pgfpathlineto{\pgfqpoint{4.931471in}{2.932811in}}%
\pgfpathlineto{\pgfqpoint{4.923866in}{2.916089in}}%
\pgfpathlineto{\pgfqpoint{4.916263in}{2.899704in}}%
\pgfpathclose%
\pgfusepath{fill}%
\end{pgfscope}%
\begin{pgfscope}%
\pgfpathrectangle{\pgfqpoint{1.150000in}{0.150000in}}{\pgfqpoint{5.700000in}{5.700000in}}%
\pgfusepath{clip}%
\pgfsetbuttcap%
\pgfsetroundjoin%
\definecolor{currentfill}{rgb}{0.250425,0.274290,0.533103}%
\pgfsetfillcolor{currentfill}%
\pgfsetfillopacity{0.700000}%
\pgfsetlinewidth{0.000000pt}%
\definecolor{currentstroke}{rgb}{0.000000,0.000000,0.000000}%
\pgfsetstrokecolor{currentstroke}%
\pgfsetdash{}{0pt}%
\pgfpathmoveto{\pgfqpoint{4.831375in}{2.856040in}}%
\pgfpathlineto{\pgfqpoint{4.844989in}{2.851265in}}%
\pgfpathlineto{\pgfqpoint{4.858610in}{2.846562in}}%
\pgfpathlineto{\pgfqpoint{4.872238in}{2.841930in}}%
\pgfpathlineto{\pgfqpoint{4.885873in}{2.837370in}}%
\pgfpathlineto{\pgfqpoint{4.893468in}{2.852489in}}%
\pgfpathlineto{\pgfqpoint{4.901064in}{2.867912in}}%
\pgfpathlineto{\pgfqpoint{4.908663in}{2.883647in}}%
\pgfpathlineto{\pgfqpoint{4.916263in}{2.899704in}}%
\pgfpathlineto{\pgfqpoint{4.902641in}{2.904700in}}%
\pgfpathlineto{\pgfqpoint{4.889026in}{2.909769in}}%
\pgfpathlineto{\pgfqpoint{4.875417in}{2.914908in}}%
\pgfpathlineto{\pgfqpoint{4.861814in}{2.920120in}}%
\pgfpathlineto{\pgfqpoint{4.854202in}{2.903619in}}%
\pgfpathlineto{\pgfqpoint{4.846591in}{2.887445in}}%
\pgfpathlineto{\pgfqpoint{4.838982in}{2.871588in}}%
\pgfpathlineto{\pgfqpoint{4.831375in}{2.856040in}}%
\pgfpathclose%
\pgfusepath{fill}%
\end{pgfscope}%
\begin{pgfscope}%
\pgfpathrectangle{\pgfqpoint{1.150000in}{0.150000in}}{\pgfqpoint{5.700000in}{5.700000in}}%
\pgfusepath{clip}%
\pgfsetbuttcap%
\pgfsetroundjoin%
\definecolor{currentfill}{rgb}{0.172719,0.448791,0.557885}%
\pgfsetfillcolor{currentfill}%
\pgfsetfillopacity{0.700000}%
\pgfsetlinewidth{0.000000pt}%
\definecolor{currentstroke}{rgb}{0.000000,0.000000,0.000000}%
\pgfsetstrokecolor{currentstroke}%
\pgfsetdash{}{0pt}%
\pgfpathmoveto{\pgfqpoint{5.317802in}{3.284993in}}%
\pgfpathlineto{\pgfqpoint{5.331461in}{3.278136in}}%
\pgfpathlineto{\pgfqpoint{5.345126in}{3.271348in}}%
\pgfpathlineto{\pgfqpoint{5.358797in}{3.264627in}}%
\pgfpathlineto{\pgfqpoint{5.372474in}{3.257975in}}%
\pgfpathlineto{\pgfqpoint{5.380180in}{3.282125in}}%
\pgfpathlineto{\pgfqpoint{5.387898in}{3.306819in}}%
\pgfpathlineto{\pgfqpoint{5.395630in}{3.332067in}}%
\pgfpathlineto{\pgfqpoint{5.381961in}{3.339153in}}%
\pgfpathlineto{\pgfqpoint{5.368299in}{3.346308in}}%
\pgfpathlineto{\pgfqpoint{5.354643in}{3.353531in}}%
\pgfpathlineto{\pgfqpoint{5.340992in}{3.360822in}}%
\pgfpathlineto{\pgfqpoint{5.333250in}{3.334990in}}%
\pgfpathlineto{\pgfqpoint{5.325520in}{3.309717in}}%
\pgfpathlineto{\pgfqpoint{5.317802in}{3.284993in}}%
\pgfpathclose%
\pgfusepath{fill}%
\end{pgfscope}%
\begin{pgfscope}%
\pgfpathrectangle{\pgfqpoint{1.150000in}{0.150000in}}{\pgfqpoint{5.700000in}{5.700000in}}%
\pgfusepath{clip}%
\pgfsetbuttcap%
\pgfsetroundjoin%
\definecolor{currentfill}{rgb}{0.282884,0.135920,0.453427}%
\pgfsetfillcolor{currentfill}%
\pgfsetfillopacity{0.700000}%
\pgfsetlinewidth{0.000000pt}%
\definecolor{currentstroke}{rgb}{0.000000,0.000000,0.000000}%
\pgfsetstrokecolor{currentstroke}%
\pgfsetdash{}{0pt}%
\pgfpathmoveto{\pgfqpoint{3.596901in}{2.572410in}}%
\pgfpathlineto{\pgfqpoint{3.610282in}{2.566114in}}%
\pgfpathlineto{\pgfqpoint{3.623666in}{2.559911in}}%
\pgfpathlineto{\pgfqpoint{3.637055in}{2.553801in}}%
\pgfpathlineto{\pgfqpoint{3.650447in}{2.547784in}}%
\pgfpathlineto{\pgfqpoint{3.658351in}{2.558889in}}%
\pgfpathlineto{\pgfqpoint{3.666249in}{2.570102in}}%
\pgfpathlineto{\pgfqpoint{3.674141in}{2.581427in}}%
\pgfpathlineto{\pgfqpoint{3.682028in}{2.592868in}}%
\pgfpathlineto{\pgfqpoint{3.668644in}{2.599081in}}%
\pgfpathlineto{\pgfqpoint{3.655264in}{2.605387in}}%
\pgfpathlineto{\pgfqpoint{3.641888in}{2.611786in}}%
\pgfpathlineto{\pgfqpoint{3.628515in}{2.618278in}}%
\pgfpathlineto{\pgfqpoint{3.620620in}{2.606634in}}%
\pgfpathlineto{\pgfqpoint{3.612720in}{2.595111in}}%
\pgfpathlineto{\pgfqpoint{3.604813in}{2.583704in}}%
\pgfpathlineto{\pgfqpoint{3.596901in}{2.572410in}}%
\pgfpathclose%
\pgfusepath{fill}%
\end{pgfscope}%
\begin{pgfscope}%
\pgfpathrectangle{\pgfqpoint{1.150000in}{0.150000in}}{\pgfqpoint{5.700000in}{5.700000in}}%
\pgfusepath{clip}%
\pgfsetbuttcap%
\pgfsetroundjoin%
\definecolor{currentfill}{rgb}{0.231674,0.318106,0.544834}%
\pgfsetfillcolor{currentfill}%
\pgfsetfillopacity{0.700000}%
\pgfsetlinewidth{0.000000pt}%
\definecolor{currentstroke}{rgb}{0.000000,0.000000,0.000000}%
\pgfsetstrokecolor{currentstroke}%
\pgfsetdash{}{0pt}%
\pgfpathmoveto{\pgfqpoint{5.001195in}{2.946198in}}%
\pgfpathlineto{\pgfqpoint{5.014837in}{2.941099in}}%
\pgfpathlineto{\pgfqpoint{5.028486in}{2.936071in}}%
\pgfpathlineto{\pgfqpoint{5.042142in}{2.931112in}}%
\pgfpathlineto{\pgfqpoint{5.055805in}{2.926224in}}%
\pgfpathlineto{\pgfqpoint{5.063395in}{2.943069in}}%
\pgfpathlineto{\pgfqpoint{5.070988in}{2.960276in}}%
\pgfpathlineto{\pgfqpoint{5.078586in}{2.977853in}}%
\pgfpathlineto{\pgfqpoint{5.086188in}{2.995809in}}%
\pgfpathlineto{\pgfqpoint{5.072538in}{3.001174in}}%
\pgfpathlineto{\pgfqpoint{5.058895in}{3.006610in}}%
\pgfpathlineto{\pgfqpoint{5.045258in}{3.012116in}}%
\pgfpathlineto{\pgfqpoint{5.031628in}{3.017691in}}%
\pgfpathlineto{\pgfqpoint{5.024013in}{2.999251in}}%
\pgfpathlineto{\pgfqpoint{5.016403in}{2.981195in}}%
\pgfpathlineto{\pgfqpoint{5.008797in}{2.963513in}}%
\pgfpathlineto{\pgfqpoint{5.001195in}{2.946198in}}%
\pgfpathclose%
\pgfusepath{fill}%
\end{pgfscope}%
\begin{pgfscope}%
\pgfpathrectangle{\pgfqpoint{1.150000in}{0.150000in}}{\pgfqpoint{5.700000in}{5.700000in}}%
\pgfusepath{clip}%
\pgfsetbuttcap%
\pgfsetroundjoin%
\definecolor{currentfill}{rgb}{0.257322,0.256130,0.526563}%
\pgfsetfillcolor{currentfill}%
\pgfsetfillopacity{0.700000}%
\pgfsetlinewidth{0.000000pt}%
\definecolor{currentstroke}{rgb}{0.000000,0.000000,0.000000}%
\pgfsetstrokecolor{currentstroke}%
\pgfsetdash{}{0pt}%
\pgfpathmoveto{\pgfqpoint{4.746513in}{2.814947in}}%
\pgfpathlineto{\pgfqpoint{4.760113in}{2.810299in}}%
\pgfpathlineto{\pgfqpoint{4.773720in}{2.805723in}}%
\pgfpathlineto{\pgfqpoint{4.787333in}{2.801220in}}%
\pgfpathlineto{\pgfqpoint{4.800953in}{2.796790in}}%
\pgfpathlineto{\pgfqpoint{4.808558in}{2.811177in}}%
\pgfpathlineto{\pgfqpoint{4.816163in}{2.825842in}}%
\pgfpathlineto{\pgfqpoint{4.823769in}{2.840794in}}%
\pgfpathlineto{\pgfqpoint{4.831375in}{2.856040in}}%
\pgfpathlineto{\pgfqpoint{4.817767in}{2.860887in}}%
\pgfpathlineto{\pgfqpoint{4.804166in}{2.865807in}}%
\pgfpathlineto{\pgfqpoint{4.790571in}{2.870798in}}%
\pgfpathlineto{\pgfqpoint{4.776982in}{2.875863in}}%
\pgfpathlineto{\pgfqpoint{4.769364in}{2.860193in}}%
\pgfpathlineto{\pgfqpoint{4.761747in}{2.844822in}}%
\pgfpathlineto{\pgfqpoint{4.754130in}{2.829743in}}%
\pgfpathlineto{\pgfqpoint{4.746513in}{2.814947in}}%
\pgfpathclose%
\pgfusepath{fill}%
\end{pgfscope}%
\begin{pgfscope}%
\pgfpathrectangle{\pgfqpoint{1.150000in}{0.150000in}}{\pgfqpoint{5.700000in}{5.700000in}}%
\pgfusepath{clip}%
\pgfsetbuttcap%
\pgfsetroundjoin%
\definecolor{currentfill}{rgb}{0.278012,0.180367,0.486697}%
\pgfsetfillcolor{currentfill}%
\pgfsetfillopacity{0.700000}%
\pgfsetlinewidth{0.000000pt}%
\definecolor{currentstroke}{rgb}{0.000000,0.000000,0.000000}%
\pgfsetstrokecolor{currentstroke}%
\pgfsetdash{}{0pt}%
\pgfpathmoveto{\pgfqpoint{4.268086in}{2.658932in}}%
\pgfpathlineto{\pgfqpoint{4.281590in}{2.654248in}}%
\pgfpathlineto{\pgfqpoint{4.295100in}{2.649643in}}%
\pgfpathlineto{\pgfqpoint{4.308616in}{2.645116in}}%
\pgfpathlineto{\pgfqpoint{4.322139in}{2.640668in}}%
\pgfpathlineto{\pgfqpoint{4.329847in}{2.652788in}}%
\pgfpathlineto{\pgfqpoint{4.337551in}{2.665084in}}%
\pgfpathlineto{\pgfqpoint{4.345253in}{2.677562in}}%
\pgfpathlineto{\pgfqpoint{4.352951in}{2.690228in}}%
\pgfpathlineto{\pgfqpoint{4.339439in}{2.694993in}}%
\pgfpathlineto{\pgfqpoint{4.325933in}{2.699835in}}%
\pgfpathlineto{\pgfqpoint{4.312432in}{2.704757in}}%
\pgfpathlineto{\pgfqpoint{4.298938in}{2.709757in}}%
\pgfpathlineto{\pgfqpoint{4.291230in}{2.696768in}}%
\pgfpathlineto{\pgfqpoint{4.283518in}{2.683971in}}%
\pgfpathlineto{\pgfqpoint{4.275804in}{2.671361in}}%
\pgfpathlineto{\pgfqpoint{4.268086in}{2.658932in}}%
\pgfpathclose%
\pgfusepath{fill}%
\end{pgfscope}%
\begin{pgfscope}%
\pgfpathrectangle{\pgfqpoint{1.150000in}{0.150000in}}{\pgfqpoint{5.700000in}{5.700000in}}%
\pgfusepath{clip}%
\pgfsetbuttcap%
\pgfsetroundjoin%
\definecolor{currentfill}{rgb}{0.221989,0.339161,0.548752}%
\pgfsetfillcolor{currentfill}%
\pgfsetfillopacity{0.700000}%
\pgfsetlinewidth{0.000000pt}%
\definecolor{currentstroke}{rgb}{0.000000,0.000000,0.000000}%
\pgfsetstrokecolor{currentstroke}%
\pgfsetdash{}{0pt}%
\pgfpathmoveto{\pgfqpoint{5.086188in}{2.995809in}}%
\pgfpathlineto{\pgfqpoint{5.099845in}{2.990513in}}%
\pgfpathlineto{\pgfqpoint{5.113508in}{2.985287in}}%
\pgfpathlineto{\pgfqpoint{5.127177in}{2.980131in}}%
\pgfpathlineto{\pgfqpoint{5.140854in}{2.975044in}}%
\pgfpathlineto{\pgfqpoint{5.148448in}{2.992897in}}%
\pgfpathlineto{\pgfqpoint{5.156048in}{3.011143in}}%
\pgfpathlineto{\pgfqpoint{5.163653in}{3.029790in}}%
\pgfpathlineto{\pgfqpoint{5.171265in}{3.048847in}}%
\pgfpathlineto{\pgfqpoint{5.157602in}{3.054432in}}%
\pgfpathlineto{\pgfqpoint{5.143945in}{3.060086in}}%
\pgfpathlineto{\pgfqpoint{5.130295in}{3.065809in}}%
\pgfpathlineto{\pgfqpoint{5.116651in}{3.071602in}}%
\pgfpathlineto{\pgfqpoint{5.109026in}{3.052040in}}%
\pgfpathlineto{\pgfqpoint{5.101408in}{3.032893in}}%
\pgfpathlineto{\pgfqpoint{5.093796in}{3.014153in}}%
\pgfpathlineto{\pgfqpoint{5.086188in}{2.995809in}}%
\pgfpathclose%
\pgfusepath{fill}%
\end{pgfscope}%
\begin{pgfscope}%
\pgfpathrectangle{\pgfqpoint{1.150000in}{0.150000in}}{\pgfqpoint{5.700000in}{5.700000in}}%
\pgfusepath{clip}%
\pgfsetbuttcap%
\pgfsetroundjoin%
\definecolor{currentfill}{rgb}{0.263663,0.237631,0.518762}%
\pgfsetfillcolor{currentfill}%
\pgfsetfillopacity{0.700000}%
\pgfsetlinewidth{0.000000pt}%
\definecolor{currentstroke}{rgb}{0.000000,0.000000,0.000000}%
\pgfsetstrokecolor{currentstroke}%
\pgfsetdash{}{0pt}%
\pgfpathmoveto{\pgfqpoint{4.661663in}{2.776189in}}%
\pgfpathlineto{\pgfqpoint{4.675249in}{2.771645in}}%
\pgfpathlineto{\pgfqpoint{4.688841in}{2.767173in}}%
\pgfpathlineto{\pgfqpoint{4.702440in}{2.762776in}}%
\pgfpathlineto{\pgfqpoint{4.716045in}{2.758451in}}%
\pgfpathlineto{\pgfqpoint{4.723663in}{2.772187in}}%
\pgfpathlineto{\pgfqpoint{4.731280in}{2.786176in}}%
\pgfpathlineto{\pgfqpoint{4.738897in}{2.800427in}}%
\pgfpathlineto{\pgfqpoint{4.746513in}{2.814947in}}%
\pgfpathlineto{\pgfqpoint{4.732920in}{2.819668in}}%
\pgfpathlineto{\pgfqpoint{4.719333in}{2.824462in}}%
\pgfpathlineto{\pgfqpoint{4.705752in}{2.829329in}}%
\pgfpathlineto{\pgfqpoint{4.692177in}{2.834270in}}%
\pgfpathlineto{\pgfqpoint{4.684550in}{2.819347in}}%
\pgfpathlineto{\pgfqpoint{4.676921in}{2.804697in}}%
\pgfpathlineto{\pgfqpoint{4.669293in}{2.790314in}}%
\pgfpathlineto{\pgfqpoint{4.661663in}{2.776189in}}%
\pgfpathclose%
\pgfusepath{fill}%
\end{pgfscope}%
\begin{pgfscope}%
\pgfpathrectangle{\pgfqpoint{1.150000in}{0.150000in}}{\pgfqpoint{5.700000in}{5.700000in}}%
\pgfusepath{clip}%
\pgfsetbuttcap%
\pgfsetroundjoin%
\definecolor{currentfill}{rgb}{0.185556,0.418570,0.556753}%
\pgfsetfillcolor{currentfill}%
\pgfsetfillopacity{0.700000}%
\pgfsetlinewidth{0.000000pt}%
\definecolor{currentstroke}{rgb}{0.000000,0.000000,0.000000}%
\pgfsetstrokecolor{currentstroke}%
\pgfsetdash{}{0pt}%
\pgfpathmoveto{\pgfqpoint{5.287046in}{3.191365in}}%
\pgfpathlineto{\pgfqpoint{5.300716in}{3.185068in}}%
\pgfpathlineto{\pgfqpoint{5.314393in}{3.178839in}}%
\pgfpathlineto{\pgfqpoint{5.328076in}{3.172679in}}%
\pgfpathlineto{\pgfqpoint{5.341766in}{3.166586in}}%
\pgfpathlineto{\pgfqpoint{5.349426in}{3.188673in}}%
\pgfpathlineto{\pgfqpoint{5.357097in}{3.211260in}}%
\pgfpathlineto{\pgfqpoint{5.364780in}{3.234357in}}%
\pgfpathlineto{\pgfqpoint{5.372474in}{3.257975in}}%
\pgfpathlineto{\pgfqpoint{5.358797in}{3.264627in}}%
\pgfpathlineto{\pgfqpoint{5.345126in}{3.271348in}}%
\pgfpathlineto{\pgfqpoint{5.331461in}{3.278136in}}%
\pgfpathlineto{\pgfqpoint{5.317802in}{3.284993in}}%
\pgfpathlineto{\pgfqpoint{5.310097in}{3.260807in}}%
\pgfpathlineto{\pgfqpoint{5.302402in}{3.237148in}}%
\pgfpathlineto{\pgfqpoint{5.294719in}{3.214004in}}%
\pgfpathlineto{\pgfqpoint{5.287046in}{3.191365in}}%
\pgfpathclose%
\pgfusepath{fill}%
\end{pgfscope}%
\begin{pgfscope}%
\pgfpathrectangle{\pgfqpoint{1.150000in}{0.150000in}}{\pgfqpoint{5.700000in}{5.700000in}}%
\pgfusepath{clip}%
\pgfsetbuttcap%
\pgfsetroundjoin%
\definecolor{currentfill}{rgb}{0.278012,0.180367,0.486697}%
\pgfsetfillcolor{currentfill}%
\pgfsetfillopacity{0.700000}%
\pgfsetlinewidth{0.000000pt}%
\definecolor{currentstroke}{rgb}{0.000000,0.000000,0.000000}%
\pgfsetstrokecolor{currentstroke}%
\pgfsetdash{}{0pt}%
\pgfpathmoveto{\pgfqpoint{2.902546in}{2.674243in}}%
\pgfpathlineto{\pgfqpoint{2.915890in}{2.664316in}}%
\pgfpathlineto{\pgfqpoint{2.929234in}{2.654514in}}%
\pgfpathlineto{\pgfqpoint{2.942578in}{2.644835in}}%
\pgfpathlineto{\pgfqpoint{2.955922in}{2.635279in}}%
\pgfpathlineto{\pgfqpoint{2.964047in}{2.645757in}}%
\pgfpathlineto{\pgfqpoint{2.972165in}{2.656336in}}%
\pgfpathlineto{\pgfqpoint{2.980275in}{2.667019in}}%
\pgfpathlineto{\pgfqpoint{2.988378in}{2.677808in}}%
\pgfpathlineto{\pgfqpoint{2.975044in}{2.687460in}}%
\pgfpathlineto{\pgfqpoint{2.961710in}{2.697234in}}%
\pgfpathlineto{\pgfqpoint{2.948377in}{2.707131in}}%
\pgfpathlineto{\pgfqpoint{2.935044in}{2.717153in}}%
\pgfpathlineto{\pgfqpoint{2.926930in}{2.706261in}}%
\pgfpathlineto{\pgfqpoint{2.918810in}{2.695481in}}%
\pgfpathlineto{\pgfqpoint{2.910682in}{2.684808in}}%
\pgfpathlineto{\pgfqpoint{2.902546in}{2.674243in}}%
\pgfpathclose%
\pgfusepath{fill}%
\end{pgfscope}%
\begin{pgfscope}%
\pgfpathrectangle{\pgfqpoint{1.150000in}{0.150000in}}{\pgfqpoint{5.700000in}{5.700000in}}%
\pgfusepath{clip}%
\pgfsetbuttcap%
\pgfsetroundjoin%
\definecolor{currentfill}{rgb}{0.281887,0.150881,0.465405}%
\pgfsetfillcolor{currentfill}%
\pgfsetfillopacity{0.700000}%
\pgfsetlinewidth{0.000000pt}%
\definecolor{currentstroke}{rgb}{0.000000,0.000000,0.000000}%
\pgfsetstrokecolor{currentstroke}%
\pgfsetdash{}{0pt}%
\pgfpathmoveto{\pgfqpoint{3.959396in}{2.594597in}}%
\pgfpathlineto{\pgfqpoint{3.972841in}{2.589433in}}%
\pgfpathlineto{\pgfqpoint{3.986290in}{2.584355in}}%
\pgfpathlineto{\pgfqpoint{3.999745in}{2.579360in}}%
\pgfpathlineto{\pgfqpoint{4.013205in}{2.574449in}}%
\pgfpathlineto{\pgfqpoint{4.021002in}{2.585872in}}%
\pgfpathlineto{\pgfqpoint{4.028793in}{2.597428in}}%
\pgfpathlineto{\pgfqpoint{4.036581in}{2.609123in}}%
\pgfpathlineto{\pgfqpoint{4.044363in}{2.620960in}}%
\pgfpathlineto{\pgfqpoint{4.030912in}{2.626127in}}%
\pgfpathlineto{\pgfqpoint{4.017467in}{2.631378in}}%
\pgfpathlineto{\pgfqpoint{4.004026in}{2.636713in}}%
\pgfpathlineto{\pgfqpoint{3.990590in}{2.642133in}}%
\pgfpathlineto{\pgfqpoint{3.982799in}{2.630032in}}%
\pgfpathlineto{\pgfqpoint{3.975003in}{2.618079in}}%
\pgfpathlineto{\pgfqpoint{3.967202in}{2.606268in}}%
\pgfpathlineto{\pgfqpoint{3.959396in}{2.594597in}}%
\pgfpathclose%
\pgfusepath{fill}%
\end{pgfscope}%
\begin{pgfscope}%
\pgfpathrectangle{\pgfqpoint{1.150000in}{0.150000in}}{\pgfqpoint{5.700000in}{5.700000in}}%
\pgfusepath{clip}%
\pgfsetbuttcap%
\pgfsetroundjoin%
\definecolor{currentfill}{rgb}{0.282884,0.135920,0.453427}%
\pgfsetfillcolor{currentfill}%
\pgfsetfillopacity{0.700000}%
\pgfsetlinewidth{0.000000pt}%
\definecolor{currentstroke}{rgb}{0.000000,0.000000,0.000000}%
\pgfsetstrokecolor{currentstroke}%
\pgfsetdash{}{0pt}%
\pgfpathmoveto{\pgfqpoint{3.735605in}{2.568932in}}%
\pgfpathlineto{\pgfqpoint{3.749010in}{2.563175in}}%
\pgfpathlineto{\pgfqpoint{3.762419in}{2.557507in}}%
\pgfpathlineto{\pgfqpoint{3.775833in}{2.551929in}}%
\pgfpathlineto{\pgfqpoint{3.789251in}{2.546439in}}%
\pgfpathlineto{\pgfqpoint{3.797116in}{2.557582in}}%
\pgfpathlineto{\pgfqpoint{3.804974in}{2.568838in}}%
\pgfpathlineto{\pgfqpoint{3.812828in}{2.580212in}}%
\pgfpathlineto{\pgfqpoint{3.820676in}{2.591707in}}%
\pgfpathlineto{\pgfqpoint{3.807266in}{2.597413in}}%
\pgfpathlineto{\pgfqpoint{3.793861in}{2.603208in}}%
\pgfpathlineto{\pgfqpoint{3.780460in}{2.609092in}}%
\pgfpathlineto{\pgfqpoint{3.767063in}{2.615065in}}%
\pgfpathlineto{\pgfqpoint{3.759207in}{2.603346in}}%
\pgfpathlineto{\pgfqpoint{3.751345in}{2.591754in}}%
\pgfpathlineto{\pgfqpoint{3.743478in}{2.580284in}}%
\pgfpathlineto{\pgfqpoint{3.735605in}{2.568932in}}%
\pgfpathclose%
\pgfusepath{fill}%
\end{pgfscope}%
\begin{pgfscope}%
\pgfpathrectangle{\pgfqpoint{1.150000in}{0.150000in}}{\pgfqpoint{5.700000in}{5.700000in}}%
\pgfusepath{clip}%
\pgfsetbuttcap%
\pgfsetroundjoin%
\definecolor{currentfill}{rgb}{0.267968,0.223549,0.512008}%
\pgfsetfillcolor{currentfill}%
\pgfsetfillopacity{0.700000}%
\pgfsetlinewidth{0.000000pt}%
\definecolor{currentstroke}{rgb}{0.000000,0.000000,0.000000}%
\pgfsetstrokecolor{currentstroke}%
\pgfsetdash{}{0pt}%
\pgfpathmoveto{\pgfqpoint{4.576812in}{2.739557in}}%
\pgfpathlineto{\pgfqpoint{4.590383in}{2.735092in}}%
\pgfpathlineto{\pgfqpoint{4.603961in}{2.730701in}}%
\pgfpathlineto{\pgfqpoint{4.617545in}{2.726385in}}%
\pgfpathlineto{\pgfqpoint{4.631136in}{2.722143in}}%
\pgfpathlineto{\pgfqpoint{4.638770in}{2.735301in}}%
\pgfpathlineto{\pgfqpoint{4.646402in}{2.748690in}}%
\pgfpathlineto{\pgfqpoint{4.654033in}{2.762317in}}%
\pgfpathlineto{\pgfqpoint{4.661663in}{2.776189in}}%
\pgfpathlineto{\pgfqpoint{4.648084in}{2.780808in}}%
\pgfpathlineto{\pgfqpoint{4.634511in}{2.785500in}}%
\pgfpathlineto{\pgfqpoint{4.620945in}{2.790267in}}%
\pgfpathlineto{\pgfqpoint{4.607385in}{2.795108in}}%
\pgfpathlineto{\pgfqpoint{4.599743in}{2.780852in}}%
\pgfpathlineto{\pgfqpoint{4.592101in}{2.766846in}}%
\pgfpathlineto{\pgfqpoint{4.584457in}{2.753083in}}%
\pgfpathlineto{\pgfqpoint{4.576812in}{2.739557in}}%
\pgfpathclose%
\pgfusepath{fill}%
\end{pgfscope}%
\begin{pgfscope}%
\pgfpathrectangle{\pgfqpoint{1.150000in}{0.150000in}}{\pgfqpoint{5.700000in}{5.700000in}}%
\pgfusepath{clip}%
\pgfsetbuttcap%
\pgfsetroundjoin%
\definecolor{currentfill}{rgb}{0.267968,0.223549,0.512008}%
\pgfsetfillcolor{currentfill}%
\pgfsetfillopacity{0.700000}%
\pgfsetlinewidth{0.000000pt}%
\definecolor{currentstroke}{rgb}{0.000000,0.000000,0.000000}%
\pgfsetstrokecolor{currentstroke}%
\pgfsetdash{}{0pt}%
\pgfpathmoveto{\pgfqpoint{2.709629in}{2.761310in}}%
\pgfpathlineto{\pgfqpoint{2.722993in}{2.749881in}}%
\pgfpathlineto{\pgfqpoint{2.736355in}{2.738590in}}%
\pgfpathlineto{\pgfqpoint{2.749716in}{2.727436in}}%
\pgfpathlineto{\pgfqpoint{2.763076in}{2.716417in}}%
\pgfpathlineto{\pgfqpoint{2.771265in}{2.726711in}}%
\pgfpathlineto{\pgfqpoint{2.779447in}{2.737114in}}%
\pgfpathlineto{\pgfqpoint{2.787620in}{2.747627in}}%
\pgfpathlineto{\pgfqpoint{2.795785in}{2.758254in}}%
\pgfpathlineto{\pgfqpoint{2.782437in}{2.769347in}}%
\pgfpathlineto{\pgfqpoint{2.769087in}{2.780576in}}%
\pgfpathlineto{\pgfqpoint{2.755736in}{2.791942in}}%
\pgfpathlineto{\pgfqpoint{2.742384in}{2.803445in}}%
\pgfpathlineto{\pgfqpoint{2.734208in}{2.792737in}}%
\pgfpathlineto{\pgfqpoint{2.726023in}{2.782146in}}%
\pgfpathlineto{\pgfqpoint{2.717830in}{2.771671in}}%
\pgfpathlineto{\pgfqpoint{2.709629in}{2.761310in}}%
\pgfpathclose%
\pgfusepath{fill}%
\end{pgfscope}%
\begin{pgfscope}%
\pgfpathrectangle{\pgfqpoint{1.150000in}{0.150000in}}{\pgfqpoint{5.700000in}{5.700000in}}%
\pgfusepath{clip}%
\pgfsetbuttcap%
\pgfsetroundjoin%
\definecolor{currentfill}{rgb}{0.210503,0.363727,0.552206}%
\pgfsetfillcolor{currentfill}%
\pgfsetfillopacity{0.700000}%
\pgfsetlinewidth{0.000000pt}%
\definecolor{currentstroke}{rgb}{0.000000,0.000000,0.000000}%
\pgfsetstrokecolor{currentstroke}%
\pgfsetdash{}{0pt}%
\pgfpathmoveto{\pgfqpoint{5.171265in}{3.048847in}}%
\pgfpathlineto{\pgfqpoint{5.184935in}{3.043332in}}%
\pgfpathlineto{\pgfqpoint{5.198612in}{3.037885in}}%
\pgfpathlineto{\pgfqpoint{5.212295in}{3.032508in}}%
\pgfpathlineto{\pgfqpoint{5.225985in}{3.027199in}}%
\pgfpathlineto{\pgfqpoint{5.233590in}{3.046166in}}%
\pgfpathlineto{\pgfqpoint{5.241202in}{3.065557in}}%
\pgfpathlineto{\pgfqpoint{5.248821in}{3.085382in}}%
\pgfpathlineto{\pgfqpoint{5.256449in}{3.105651in}}%
\pgfpathlineto{\pgfqpoint{5.242772in}{3.111478in}}%
\pgfpathlineto{\pgfqpoint{5.229102in}{3.117374in}}%
\pgfpathlineto{\pgfqpoint{5.215438in}{3.123338in}}%
\pgfpathlineto{\pgfqpoint{5.201781in}{3.129372in}}%
\pgfpathlineto{\pgfqpoint{5.194141in}{3.108577in}}%
\pgfpathlineto{\pgfqpoint{5.186509in}{3.088231in}}%
\pgfpathlineto{\pgfqpoint{5.178883in}{3.068325in}}%
\pgfpathlineto{\pgfqpoint{5.171265in}{3.048847in}}%
\pgfpathclose%
\pgfusepath{fill}%
\end{pgfscope}%
\begin{pgfscope}%
\pgfpathrectangle{\pgfqpoint{1.150000in}{0.150000in}}{\pgfqpoint{5.700000in}{5.700000in}}%
\pgfusepath{clip}%
\pgfsetbuttcap%
\pgfsetroundjoin%
\definecolor{currentfill}{rgb}{0.282623,0.140926,0.457517}%
\pgfsetfillcolor{currentfill}%
\pgfsetfillopacity{0.700000}%
\pgfsetlinewidth{0.000000pt}%
\definecolor{currentstroke}{rgb}{0.000000,0.000000,0.000000}%
\pgfsetstrokecolor{currentstroke}%
\pgfsetdash{}{0pt}%
\pgfpathmoveto{\pgfqpoint{3.234043in}{2.581631in}}%
\pgfpathlineto{\pgfqpoint{3.247389in}{2.573801in}}%
\pgfpathlineto{\pgfqpoint{3.260738in}{2.566076in}}%
\pgfpathlineto{\pgfqpoint{3.274089in}{2.558458in}}%
\pgfpathlineto{\pgfqpoint{3.287443in}{2.550945in}}%
\pgfpathlineto{\pgfqpoint{3.295463in}{2.561655in}}%
\pgfpathlineto{\pgfqpoint{3.303477in}{2.572460in}}%
\pgfpathlineto{\pgfqpoint{3.311485in}{2.583363in}}%
\pgfpathlineto{\pgfqpoint{3.319486in}{2.594367in}}%
\pgfpathlineto{\pgfqpoint{3.306142in}{2.602016in}}%
\pgfpathlineto{\pgfqpoint{3.292800in}{2.609770in}}%
\pgfpathlineto{\pgfqpoint{3.279460in}{2.617630in}}%
\pgfpathlineto{\pgfqpoint{3.266123in}{2.625597in}}%
\pgfpathlineto{\pgfqpoint{3.258112in}{2.614450in}}%
\pgfpathlineto{\pgfqpoint{3.250096in}{2.603409in}}%
\pgfpathlineto{\pgfqpoint{3.242072in}{2.592470in}}%
\pgfpathlineto{\pgfqpoint{3.234043in}{2.581631in}}%
\pgfpathclose%
\pgfusepath{fill}%
\end{pgfscope}%
\begin{pgfscope}%
\pgfpathrectangle{\pgfqpoint{1.150000in}{0.150000in}}{\pgfqpoint{5.700000in}{5.700000in}}%
\pgfusepath{clip}%
\pgfsetbuttcap%
\pgfsetroundjoin%
\definecolor{currentfill}{rgb}{0.283072,0.130895,0.449241}%
\pgfsetfillcolor{currentfill}%
\pgfsetfillopacity{0.700000}%
\pgfsetlinewidth{0.000000pt}%
\definecolor{currentstroke}{rgb}{0.000000,0.000000,0.000000}%
\pgfsetstrokecolor{currentstroke}%
\pgfsetdash{}{0pt}%
\pgfpathmoveto{\pgfqpoint{3.372890in}{2.564806in}}%
\pgfpathlineto{\pgfqpoint{3.386248in}{2.557671in}}%
\pgfpathlineto{\pgfqpoint{3.399608in}{2.550638in}}%
\pgfpathlineto{\pgfqpoint{3.412972in}{2.543704in}}%
\pgfpathlineto{\pgfqpoint{3.426339in}{2.536870in}}%
\pgfpathlineto{\pgfqpoint{3.434316in}{2.547680in}}%
\pgfpathlineto{\pgfqpoint{3.442287in}{2.558587in}}%
\pgfpathlineto{\pgfqpoint{3.450252in}{2.569592in}}%
\pgfpathlineto{\pgfqpoint{3.458210in}{2.580701in}}%
\pgfpathlineto{\pgfqpoint{3.444852in}{2.587690in}}%
\pgfpathlineto{\pgfqpoint{3.431497in}{2.594780in}}%
\pgfpathlineto{\pgfqpoint{3.418146in}{2.601969in}}%
\pgfpathlineto{\pgfqpoint{3.404797in}{2.609260in}}%
\pgfpathlineto{\pgfqpoint{3.396829in}{2.597988in}}%
\pgfpathlineto{\pgfqpoint{3.388856in}{2.586824in}}%
\pgfpathlineto{\pgfqpoint{3.380876in}{2.575764in}}%
\pgfpathlineto{\pgfqpoint{3.372890in}{2.564806in}}%
\pgfpathclose%
\pgfusepath{fill}%
\end{pgfscope}%
\begin{pgfscope}%
\pgfpathrectangle{\pgfqpoint{1.150000in}{0.150000in}}{\pgfqpoint{5.700000in}{5.700000in}}%
\pgfusepath{clip}%
\pgfsetbuttcap%
\pgfsetroundjoin%
\definecolor{currentfill}{rgb}{0.279574,0.170599,0.479997}%
\pgfsetfillcolor{currentfill}%
\pgfsetfillopacity{0.700000}%
\pgfsetlinewidth{0.000000pt}%
\definecolor{currentstroke}{rgb}{0.000000,0.000000,0.000000}%
\pgfsetstrokecolor{currentstroke}%
\pgfsetdash{}{0pt}%
\pgfpathmoveto{\pgfqpoint{4.183178in}{2.629260in}}%
\pgfpathlineto{\pgfqpoint{4.196669in}{2.624553in}}%
\pgfpathlineto{\pgfqpoint{4.210166in}{2.619927in}}%
\pgfpathlineto{\pgfqpoint{4.223668in}{2.615381in}}%
\pgfpathlineto{\pgfqpoint{4.237177in}{2.610914in}}%
\pgfpathlineto{\pgfqpoint{4.244910in}{2.622675in}}%
\pgfpathlineto{\pgfqpoint{4.252639in}{2.634594in}}%
\pgfpathlineto{\pgfqpoint{4.260364in}{2.646678in}}%
\pgfpathlineto{\pgfqpoint{4.268086in}{2.658932in}}%
\pgfpathlineto{\pgfqpoint{4.254587in}{2.663696in}}%
\pgfpathlineto{\pgfqpoint{4.241095in}{2.668538in}}%
\pgfpathlineto{\pgfqpoint{4.227607in}{2.673461in}}%
\pgfpathlineto{\pgfqpoint{4.214126in}{2.678464in}}%
\pgfpathlineto{\pgfqpoint{4.206395in}{2.665906in}}%
\pgfpathlineto{\pgfqpoint{4.198660in}{2.653524in}}%
\pgfpathlineto{\pgfqpoint{4.190921in}{2.641310in}}%
\pgfpathlineto{\pgfqpoint{4.183178in}{2.629260in}}%
\pgfpathclose%
\pgfusepath{fill}%
\end{pgfscope}%
\begin{pgfscope}%
\pgfpathrectangle{\pgfqpoint{1.150000in}{0.150000in}}{\pgfqpoint{5.700000in}{5.700000in}}%
\pgfusepath{clip}%
\pgfsetbuttcap%
\pgfsetroundjoin%
\definecolor{currentfill}{rgb}{0.281887,0.150881,0.465405}%
\pgfsetfillcolor{currentfill}%
\pgfsetfillopacity{0.700000}%
\pgfsetlinewidth{0.000000pt}%
\definecolor{currentstroke}{rgb}{0.000000,0.000000,0.000000}%
\pgfsetstrokecolor{currentstroke}%
\pgfsetdash{}{0pt}%
\pgfpathmoveto{\pgfqpoint{3.095075in}{2.604874in}}%
\pgfpathlineto{\pgfqpoint{3.108417in}{2.596278in}}%
\pgfpathlineto{\pgfqpoint{3.121761in}{2.587796in}}%
\pgfpathlineto{\pgfqpoint{3.135106in}{2.579425in}}%
\pgfpathlineto{\pgfqpoint{3.148453in}{2.571166in}}%
\pgfpathlineto{\pgfqpoint{3.156519in}{2.581738in}}%
\pgfpathlineto{\pgfqpoint{3.164578in}{2.592406in}}%
\pgfpathlineto{\pgfqpoint{3.172630in}{2.603170in}}%
\pgfpathlineto{\pgfqpoint{3.180676in}{2.614035in}}%
\pgfpathlineto{\pgfqpoint{3.167339in}{2.622410in}}%
\pgfpathlineto{\pgfqpoint{3.154004in}{2.630896in}}%
\pgfpathlineto{\pgfqpoint{3.140670in}{2.639494in}}%
\pgfpathlineto{\pgfqpoint{3.127338in}{2.648205in}}%
\pgfpathlineto{\pgfqpoint{3.119282in}{2.637217in}}%
\pgfpathlineto{\pgfqpoint{3.111220in}{2.626334in}}%
\pgfpathlineto{\pgfqpoint{3.103151in}{2.615554in}}%
\pgfpathlineto{\pgfqpoint{3.095075in}{2.604874in}}%
\pgfpathclose%
\pgfusepath{fill}%
\end{pgfscope}%
\begin{pgfscope}%
\pgfpathrectangle{\pgfqpoint{1.150000in}{0.150000in}}{\pgfqpoint{5.700000in}{5.700000in}}%
\pgfusepath{clip}%
\pgfsetbuttcap%
\pgfsetroundjoin%
\definecolor{currentfill}{rgb}{0.271828,0.209303,0.504434}%
\pgfsetfillcolor{currentfill}%
\pgfsetfillopacity{0.700000}%
\pgfsetlinewidth{0.000000pt}%
\definecolor{currentstroke}{rgb}{0.000000,0.000000,0.000000}%
\pgfsetstrokecolor{currentstroke}%
\pgfsetdash{}{0pt}%
\pgfpathmoveto{\pgfqpoint{4.491948in}{2.704863in}}%
\pgfpathlineto{\pgfqpoint{4.505504in}{2.700454in}}%
\pgfpathlineto{\pgfqpoint{4.519067in}{2.696120in}}%
\pgfpathlineto{\pgfqpoint{4.532637in}{2.691862in}}%
\pgfpathlineto{\pgfqpoint{4.546213in}{2.687679in}}%
\pgfpathlineto{\pgfqpoint{4.553866in}{2.700327in}}%
\pgfpathlineto{\pgfqpoint{4.561516in}{2.713185in}}%
\pgfpathlineto{\pgfqpoint{4.569165in}{2.726260in}}%
\pgfpathlineto{\pgfqpoint{4.576812in}{2.739557in}}%
\pgfpathlineto{\pgfqpoint{4.563247in}{2.744096in}}%
\pgfpathlineto{\pgfqpoint{4.549689in}{2.748711in}}%
\pgfpathlineto{\pgfqpoint{4.536136in}{2.753401in}}%
\pgfpathlineto{\pgfqpoint{4.522590in}{2.758167in}}%
\pgfpathlineto{\pgfqpoint{4.514933in}{2.744506in}}%
\pgfpathlineto{\pgfqpoint{4.507273in}{2.731072in}}%
\pgfpathlineto{\pgfqpoint{4.499611in}{2.717860in}}%
\pgfpathlineto{\pgfqpoint{4.491948in}{2.704863in}}%
\pgfpathclose%
\pgfusepath{fill}%
\end{pgfscope}%
\begin{pgfscope}%
\pgfpathrectangle{\pgfqpoint{1.150000in}{0.150000in}}{\pgfqpoint{5.700000in}{5.700000in}}%
\pgfusepath{clip}%
\pgfsetbuttcap%
\pgfsetroundjoin%
\definecolor{currentfill}{rgb}{0.283072,0.130895,0.449241}%
\pgfsetfillcolor{currentfill}%
\pgfsetfillopacity{0.700000}%
\pgfsetlinewidth{0.000000pt}%
\definecolor{currentstroke}{rgb}{0.000000,0.000000,0.000000}%
\pgfsetstrokecolor{currentstroke}%
\pgfsetdash{}{0pt}%
\pgfpathmoveto{\pgfqpoint{3.511675in}{2.553725in}}%
\pgfpathlineto{\pgfqpoint{3.525049in}{2.547224in}}%
\pgfpathlineto{\pgfqpoint{3.538428in}{2.540819in}}%
\pgfpathlineto{\pgfqpoint{3.551809in}{2.534510in}}%
\pgfpathlineto{\pgfqpoint{3.565195in}{2.528295in}}%
\pgfpathlineto{\pgfqpoint{3.573130in}{2.539172in}}%
\pgfpathlineto{\pgfqpoint{3.581060in}{2.550148in}}%
\pgfpathlineto{\pgfqpoint{3.588983in}{2.561226in}}%
\pgfpathlineto{\pgfqpoint{3.596901in}{2.572410in}}%
\pgfpathlineto{\pgfqpoint{3.583524in}{2.578801in}}%
\pgfpathlineto{\pgfqpoint{3.570151in}{2.585287in}}%
\pgfpathlineto{\pgfqpoint{3.556781in}{2.591868in}}%
\pgfpathlineto{\pgfqpoint{3.543415in}{2.598545in}}%
\pgfpathlineto{\pgfqpoint{3.535489in}{2.587177in}}%
\pgfpathlineto{\pgfqpoint{3.527557in}{2.575920in}}%
\pgfpathlineto{\pgfqpoint{3.519619in}{2.564771in}}%
\pgfpathlineto{\pgfqpoint{3.511675in}{2.553725in}}%
\pgfpathclose%
\pgfusepath{fill}%
\end{pgfscope}%
\begin{pgfscope}%
\pgfpathrectangle{\pgfqpoint{1.150000in}{0.150000in}}{\pgfqpoint{5.700000in}{5.700000in}}%
\pgfusepath{clip}%
\pgfsetbuttcap%
\pgfsetroundjoin%
\definecolor{currentfill}{rgb}{0.199430,0.387607,0.554642}%
\pgfsetfillcolor{currentfill}%
\pgfsetfillopacity{0.700000}%
\pgfsetlinewidth{0.000000pt}%
\definecolor{currentstroke}{rgb}{0.000000,0.000000,0.000000}%
\pgfsetstrokecolor{currentstroke}%
\pgfsetdash{}{0pt}%
\pgfpathmoveto{\pgfqpoint{5.256449in}{3.105651in}}%
\pgfpathlineto{\pgfqpoint{5.270132in}{3.099893in}}%
\pgfpathlineto{\pgfqpoint{5.283822in}{3.094203in}}%
\pgfpathlineto{\pgfqpoint{5.297518in}{3.088581in}}%
\pgfpathlineto{\pgfqpoint{5.311221in}{3.083028in}}%
\pgfpathlineto{\pgfqpoint{5.318844in}{3.103220in}}%
\pgfpathlineto{\pgfqpoint{5.326475in}{3.123870in}}%
\pgfpathlineto{\pgfqpoint{5.334115in}{3.144989in}}%
\pgfpathlineto{\pgfqpoint{5.341766in}{3.166586in}}%
\pgfpathlineto{\pgfqpoint{5.328076in}{3.172679in}}%
\pgfpathlineto{\pgfqpoint{5.314393in}{3.178839in}}%
\pgfpathlineto{\pgfqpoint{5.300716in}{3.185068in}}%
\pgfpathlineto{\pgfqpoint{5.287046in}{3.191365in}}%
\pgfpathlineto{\pgfqpoint{5.279383in}{3.169221in}}%
\pgfpathlineto{\pgfqpoint{5.271729in}{3.147560in}}%
\pgfpathlineto{\pgfqpoint{5.264085in}{3.126374in}}%
\pgfpathlineto{\pgfqpoint{5.256449in}{3.105651in}}%
\pgfpathclose%
\pgfusepath{fill}%
\end{pgfscope}%
\begin{pgfscope}%
\pgfpathrectangle{\pgfqpoint{1.150000in}{0.150000in}}{\pgfqpoint{5.700000in}{5.700000in}}%
\pgfusepath{clip}%
\pgfsetbuttcap%
\pgfsetroundjoin%
\definecolor{currentfill}{rgb}{0.282623,0.140926,0.457517}%
\pgfsetfillcolor{currentfill}%
\pgfsetfillopacity{0.700000}%
\pgfsetlinewidth{0.000000pt}%
\definecolor{currentstroke}{rgb}{0.000000,0.000000,0.000000}%
\pgfsetstrokecolor{currentstroke}%
\pgfsetdash{}{0pt}%
\pgfpathmoveto{\pgfqpoint{3.874361in}{2.569762in}}%
\pgfpathlineto{\pgfqpoint{3.887795in}{2.564493in}}%
\pgfpathlineto{\pgfqpoint{3.901233in}{2.559311in}}%
\pgfpathlineto{\pgfqpoint{3.914676in}{2.554214in}}%
\pgfpathlineto{\pgfqpoint{3.928125in}{2.549202in}}%
\pgfpathlineto{\pgfqpoint{3.935950in}{2.560366in}}%
\pgfpathlineto{\pgfqpoint{3.943770in}{2.571650in}}%
\pgfpathlineto{\pgfqpoint{3.951586in}{2.583059in}}%
\pgfpathlineto{\pgfqpoint{3.959396in}{2.594597in}}%
\pgfpathlineto{\pgfqpoint{3.945957in}{2.599845in}}%
\pgfpathlineto{\pgfqpoint{3.932522in}{2.605178in}}%
\pgfpathlineto{\pgfqpoint{3.919093in}{2.610597in}}%
\pgfpathlineto{\pgfqpoint{3.905668in}{2.616102in}}%
\pgfpathlineto{\pgfqpoint{3.897849in}{2.604321in}}%
\pgfpathlineto{\pgfqpoint{3.890025in}{2.592673in}}%
\pgfpathlineto{\pgfqpoint{3.882196in}{2.581155in}}%
\pgfpathlineto{\pgfqpoint{3.874361in}{2.569762in}}%
\pgfpathclose%
\pgfusepath{fill}%
\end{pgfscope}%
\begin{pgfscope}%
\pgfpathrectangle{\pgfqpoint{1.150000in}{0.150000in}}{\pgfqpoint{5.700000in}{5.700000in}}%
\pgfusepath{clip}%
\pgfsetbuttcap%
\pgfsetroundjoin%
\definecolor{currentfill}{rgb}{0.273006,0.204520,0.501721}%
\pgfsetfillcolor{currentfill}%
\pgfsetfillopacity{0.700000}%
\pgfsetlinewidth{0.000000pt}%
\definecolor{currentstroke}{rgb}{0.000000,0.000000,0.000000}%
\pgfsetstrokecolor{currentstroke}%
\pgfsetdash{}{0pt}%
\pgfpathmoveto{\pgfqpoint{2.763076in}{2.716417in}}%
\pgfpathlineto{\pgfqpoint{2.776435in}{2.705533in}}%
\pgfpathlineto{\pgfqpoint{2.789793in}{2.694782in}}%
\pgfpathlineto{\pgfqpoint{2.803150in}{2.684163in}}%
\pgfpathlineto{\pgfqpoint{2.816507in}{2.673674in}}%
\pgfpathlineto{\pgfqpoint{2.824684in}{2.683900in}}%
\pgfpathlineto{\pgfqpoint{2.832854in}{2.694231in}}%
\pgfpathlineto{\pgfqpoint{2.841016in}{2.704667in}}%
\pgfpathlineto{\pgfqpoint{2.849170in}{2.715211in}}%
\pgfpathlineto{\pgfqpoint{2.835825in}{2.725774in}}%
\pgfpathlineto{\pgfqpoint{2.822479in}{2.736469in}}%
\pgfpathlineto{\pgfqpoint{2.809132in}{2.747295in}}%
\pgfpathlineto{\pgfqpoint{2.795785in}{2.758254in}}%
\pgfpathlineto{\pgfqpoint{2.787620in}{2.747627in}}%
\pgfpathlineto{\pgfqpoint{2.779447in}{2.737114in}}%
\pgfpathlineto{\pgfqpoint{2.771265in}{2.726711in}}%
\pgfpathlineto{\pgfqpoint{2.763076in}{2.716417in}}%
\pgfpathclose%
\pgfusepath{fill}%
\end{pgfscope}%
\begin{pgfscope}%
\pgfpathrectangle{\pgfqpoint{1.150000in}{0.150000in}}{\pgfqpoint{5.700000in}{5.700000in}}%
\pgfusepath{clip}%
\pgfsetbuttcap%
\pgfsetroundjoin%
\definecolor{currentfill}{rgb}{0.174274,0.445044,0.557792}%
\pgfsetfillcolor{currentfill}%
\pgfsetfillopacity{0.700000}%
\pgfsetlinewidth{0.000000pt}%
\definecolor{currentstroke}{rgb}{0.000000,0.000000,0.000000}%
\pgfsetstrokecolor{currentstroke}%
\pgfsetdash{}{0pt}%
\pgfpathmoveto{\pgfqpoint{5.372474in}{3.257975in}}%
\pgfpathlineto{\pgfqpoint{5.386157in}{3.251391in}}%
\pgfpathlineto{\pgfqpoint{5.399847in}{3.244875in}}%
\pgfpathlineto{\pgfqpoint{5.413543in}{3.238426in}}%
\pgfpathlineto{\pgfqpoint{5.427245in}{3.232045in}}%
\pgfpathlineto{\pgfqpoint{5.434939in}{3.255622in}}%
\pgfpathlineto{\pgfqpoint{5.442645in}{3.279737in}}%
\pgfpathlineto{\pgfqpoint{5.450364in}{3.304401in}}%
\pgfpathlineto{\pgfqpoint{5.436671in}{3.311216in}}%
\pgfpathlineto{\pgfqpoint{5.422985in}{3.318098in}}%
\pgfpathlineto{\pgfqpoint{5.409304in}{3.325048in}}%
\pgfpathlineto{\pgfqpoint{5.395630in}{3.332067in}}%
\pgfpathlineto{\pgfqpoint{5.387898in}{3.306819in}}%
\pgfpathlineto{\pgfqpoint{5.380180in}{3.282125in}}%
\pgfpathlineto{\pgfqpoint{5.372474in}{3.257975in}}%
\pgfpathclose%
\pgfusepath{fill}%
\end{pgfscope}%
\begin{pgfscope}%
\pgfpathrectangle{\pgfqpoint{1.150000in}{0.150000in}}{\pgfqpoint{5.700000in}{5.700000in}}%
\pgfusepath{clip}%
\pgfsetbuttcap%
\pgfsetroundjoin%
\definecolor{currentfill}{rgb}{0.280255,0.165693,0.476498}%
\pgfsetfillcolor{currentfill}%
\pgfsetfillopacity{0.700000}%
\pgfsetlinewidth{0.000000pt}%
\definecolor{currentstroke}{rgb}{0.000000,0.000000,0.000000}%
\pgfsetstrokecolor{currentstroke}%
\pgfsetdash{}{0pt}%
\pgfpathmoveto{\pgfqpoint{2.955922in}{2.635279in}}%
\pgfpathlineto{\pgfqpoint{2.969267in}{2.625843in}}%
\pgfpathlineto{\pgfqpoint{2.982612in}{2.616528in}}%
\pgfpathlineto{\pgfqpoint{2.995958in}{2.607332in}}%
\pgfpathlineto{\pgfqpoint{3.009305in}{2.598254in}}%
\pgfpathlineto{\pgfqpoint{3.017419in}{2.608645in}}%
\pgfpathlineto{\pgfqpoint{3.025527in}{2.619131in}}%
\pgfpathlineto{\pgfqpoint{3.033627in}{2.629716in}}%
\pgfpathlineto{\pgfqpoint{3.041719in}{2.640402in}}%
\pgfpathlineto{\pgfqpoint{3.028383in}{2.649575in}}%
\pgfpathlineto{\pgfqpoint{3.015047in}{2.658866in}}%
\pgfpathlineto{\pgfqpoint{3.001712in}{2.668277in}}%
\pgfpathlineto{\pgfqpoint{2.988378in}{2.677808in}}%
\pgfpathlineto{\pgfqpoint{2.980275in}{2.667019in}}%
\pgfpathlineto{\pgfqpoint{2.972165in}{2.656336in}}%
\pgfpathlineto{\pgfqpoint{2.964047in}{2.645757in}}%
\pgfpathlineto{\pgfqpoint{2.955922in}{2.635279in}}%
\pgfpathclose%
\pgfusepath{fill}%
\end{pgfscope}%
\begin{pgfscope}%
\pgfpathrectangle{\pgfqpoint{1.150000in}{0.150000in}}{\pgfqpoint{5.700000in}{5.700000in}}%
\pgfusepath{clip}%
\pgfsetbuttcap%
\pgfsetroundjoin%
\definecolor{currentfill}{rgb}{0.275191,0.194905,0.496005}%
\pgfsetfillcolor{currentfill}%
\pgfsetfillopacity{0.700000}%
\pgfsetlinewidth{0.000000pt}%
\definecolor{currentstroke}{rgb}{0.000000,0.000000,0.000000}%
\pgfsetstrokecolor{currentstroke}%
\pgfsetdash{}{0pt}%
\pgfpathmoveto{\pgfqpoint{4.407059in}{2.671946in}}%
\pgfpathlineto{\pgfqpoint{4.420602in}{2.667569in}}%
\pgfpathlineto{\pgfqpoint{4.434150in}{2.663268in}}%
\pgfpathlineto{\pgfqpoint{4.447705in}{2.659044in}}%
\pgfpathlineto{\pgfqpoint{4.461267in}{2.654896in}}%
\pgfpathlineto{\pgfqpoint{4.468941in}{2.667097in}}%
\pgfpathlineto{\pgfqpoint{4.476612in}{2.679487in}}%
\pgfpathlineto{\pgfqpoint{4.484281in}{2.692074in}}%
\pgfpathlineto{\pgfqpoint{4.491948in}{2.704863in}}%
\pgfpathlineto{\pgfqpoint{4.478397in}{2.709348in}}%
\pgfpathlineto{\pgfqpoint{4.464853in}{2.713909in}}%
\pgfpathlineto{\pgfqpoint{4.451315in}{2.718546in}}%
\pgfpathlineto{\pgfqpoint{4.437783in}{2.723260in}}%
\pgfpathlineto{\pgfqpoint{4.430106in}{2.710127in}}%
\pgfpathlineto{\pgfqpoint{4.422427in}{2.697202in}}%
\pgfpathlineto{\pgfqpoint{4.414744in}{2.684477in}}%
\pgfpathlineto{\pgfqpoint{4.407059in}{2.671946in}}%
\pgfpathclose%
\pgfusepath{fill}%
\end{pgfscope}%
\begin{pgfscope}%
\pgfpathrectangle{\pgfqpoint{1.150000in}{0.150000in}}{\pgfqpoint{5.700000in}{5.700000in}}%
\pgfusepath{clip}%
\pgfsetbuttcap%
\pgfsetroundjoin%
\definecolor{currentfill}{rgb}{0.281412,0.155834,0.469201}%
\pgfsetfillcolor{currentfill}%
\pgfsetfillopacity{0.700000}%
\pgfsetlinewidth{0.000000pt}%
\definecolor{currentstroke}{rgb}{0.000000,0.000000,0.000000}%
\pgfsetstrokecolor{currentstroke}%
\pgfsetdash{}{0pt}%
\pgfpathmoveto{\pgfqpoint{4.098220in}{2.601121in}}%
\pgfpathlineto{\pgfqpoint{4.111698in}{2.596367in}}%
\pgfpathlineto{\pgfqpoint{4.125182in}{2.591694in}}%
\pgfpathlineto{\pgfqpoint{4.138671in}{2.587103in}}%
\pgfpathlineto{\pgfqpoint{4.152166in}{2.582593in}}%
\pgfpathlineto{\pgfqpoint{4.159925in}{2.594040in}}%
\pgfpathlineto{\pgfqpoint{4.167680in}{2.605630in}}%
\pgfpathlineto{\pgfqpoint{4.175431in}{2.617368in}}%
\pgfpathlineto{\pgfqpoint{4.183178in}{2.629260in}}%
\pgfpathlineto{\pgfqpoint{4.169693in}{2.634047in}}%
\pgfpathlineto{\pgfqpoint{4.156213in}{2.638915in}}%
\pgfpathlineto{\pgfqpoint{4.142739in}{2.643864in}}%
\pgfpathlineto{\pgfqpoint{4.129271in}{2.648894in}}%
\pgfpathlineto{\pgfqpoint{4.121515in}{2.636719in}}%
\pgfpathlineto{\pgfqpoint{4.113754in}{2.624702in}}%
\pgfpathlineto{\pgfqpoint{4.105989in}{2.612837in}}%
\pgfpathlineto{\pgfqpoint{4.098220in}{2.601121in}}%
\pgfpathclose%
\pgfusepath{fill}%
\end{pgfscope}%
\begin{pgfscope}%
\pgfpathrectangle{\pgfqpoint{1.150000in}{0.150000in}}{\pgfqpoint{5.700000in}{5.700000in}}%
\pgfusepath{clip}%
\pgfsetbuttcap%
\pgfsetroundjoin%
\definecolor{currentfill}{rgb}{0.283072,0.130895,0.449241}%
\pgfsetfillcolor{currentfill}%
\pgfsetfillopacity{0.700000}%
\pgfsetlinewidth{0.000000pt}%
\definecolor{currentstroke}{rgb}{0.000000,0.000000,0.000000}%
\pgfsetstrokecolor{currentstroke}%
\pgfsetdash{}{0pt}%
\pgfpathmoveto{\pgfqpoint{3.650447in}{2.547784in}}%
\pgfpathlineto{\pgfqpoint{3.663844in}{2.541859in}}%
\pgfpathlineto{\pgfqpoint{3.677245in}{2.536026in}}%
\pgfpathlineto{\pgfqpoint{3.690650in}{2.530285in}}%
\pgfpathlineto{\pgfqpoint{3.704059in}{2.524633in}}%
\pgfpathlineto{\pgfqpoint{3.711954in}{2.535550in}}%
\pgfpathlineto{\pgfqpoint{3.719843in}{2.546569in}}%
\pgfpathlineto{\pgfqpoint{3.727727in}{2.557695in}}%
\pgfpathlineto{\pgfqpoint{3.735605in}{2.568932in}}%
\pgfpathlineto{\pgfqpoint{3.722204in}{2.574780in}}%
\pgfpathlineto{\pgfqpoint{3.708808in}{2.580718in}}%
\pgfpathlineto{\pgfqpoint{3.695416in}{2.586747in}}%
\pgfpathlineto{\pgfqpoint{3.682028in}{2.592868in}}%
\pgfpathlineto{\pgfqpoint{3.674141in}{2.581427in}}%
\pgfpathlineto{\pgfqpoint{3.666249in}{2.570102in}}%
\pgfpathlineto{\pgfqpoint{3.658351in}{2.558889in}}%
\pgfpathlineto{\pgfqpoint{3.650447in}{2.547784in}}%
\pgfpathclose%
\pgfusepath{fill}%
\end{pgfscope}%
\begin{pgfscope}%
\pgfpathrectangle{\pgfqpoint{1.150000in}{0.150000in}}{\pgfqpoint{5.700000in}{5.700000in}}%
\pgfusepath{clip}%
\pgfsetbuttcap%
\pgfsetroundjoin%
\definecolor{currentfill}{rgb}{0.243113,0.292092,0.538516}%
\pgfsetfillcolor{currentfill}%
\pgfsetfillopacity{0.700000}%
\pgfsetlinewidth{0.000000pt}%
\definecolor{currentstroke}{rgb}{0.000000,0.000000,0.000000}%
\pgfsetstrokecolor{currentstroke}%
\pgfsetdash{}{0pt}%
\pgfpathmoveto{\pgfqpoint{4.970818in}{2.880428in}}%
\pgfpathlineto{\pgfqpoint{4.984473in}{2.875786in}}%
\pgfpathlineto{\pgfqpoint{4.998135in}{2.871214in}}%
\pgfpathlineto{\pgfqpoint{5.011805in}{2.866713in}}%
\pgfpathlineto{\pgfqpoint{5.025481in}{2.862281in}}%
\pgfpathlineto{\pgfqpoint{5.033057in}{2.877768in}}%
\pgfpathlineto{\pgfqpoint{5.040637in}{2.893581in}}%
\pgfpathlineto{\pgfqpoint{5.048219in}{2.909731in}}%
\pgfpathlineto{\pgfqpoint{5.055805in}{2.926224in}}%
\pgfpathlineto{\pgfqpoint{5.042142in}{2.931112in}}%
\pgfpathlineto{\pgfqpoint{5.028486in}{2.936071in}}%
\pgfpathlineto{\pgfqpoint{5.014837in}{2.941099in}}%
\pgfpathlineto{\pgfqpoint{5.001195in}{2.946198in}}%
\pgfpathlineto{\pgfqpoint{4.993596in}{2.929241in}}%
\pgfpathlineto{\pgfqpoint{4.986000in}{2.912632in}}%
\pgfpathlineto{\pgfqpoint{4.978408in}{2.896364in}}%
\pgfpathlineto{\pgfqpoint{4.970818in}{2.880428in}}%
\pgfpathclose%
\pgfusepath{fill}%
\end{pgfscope}%
\begin{pgfscope}%
\pgfpathrectangle{\pgfqpoint{1.150000in}{0.150000in}}{\pgfqpoint{5.700000in}{5.700000in}}%
\pgfusepath{clip}%
\pgfsetbuttcap%
\pgfsetroundjoin%
\definecolor{currentfill}{rgb}{0.250425,0.274290,0.533103}%
\pgfsetfillcolor{currentfill}%
\pgfsetfillopacity{0.700000}%
\pgfsetlinewidth{0.000000pt}%
\definecolor{currentstroke}{rgb}{0.000000,0.000000,0.000000}%
\pgfsetstrokecolor{currentstroke}%
\pgfsetdash{}{0pt}%
\pgfpathmoveto{\pgfqpoint{4.885873in}{2.837370in}}%
\pgfpathlineto{\pgfqpoint{4.899514in}{2.832882in}}%
\pgfpathlineto{\pgfqpoint{4.913162in}{2.828464in}}%
\pgfpathlineto{\pgfqpoint{4.926817in}{2.824117in}}%
\pgfpathlineto{\pgfqpoint{4.940479in}{2.819842in}}%
\pgfpathlineto{\pgfqpoint{4.948061in}{2.834531in}}%
\pgfpathlineto{\pgfqpoint{4.955644in}{2.849519in}}%
\pgfpathlineto{\pgfqpoint{4.963230in}{2.864816in}}%
\pgfpathlineto{\pgfqpoint{4.970818in}{2.880428in}}%
\pgfpathlineto{\pgfqpoint{4.957169in}{2.885141in}}%
\pgfpathlineto{\pgfqpoint{4.943527in}{2.889924in}}%
\pgfpathlineto{\pgfqpoint{4.929892in}{2.894778in}}%
\pgfpathlineto{\pgfqpoint{4.916263in}{2.899704in}}%
\pgfpathlineto{\pgfqpoint{4.908663in}{2.883647in}}%
\pgfpathlineto{\pgfqpoint{4.901064in}{2.867912in}}%
\pgfpathlineto{\pgfqpoint{4.893468in}{2.852489in}}%
\pgfpathlineto{\pgfqpoint{4.885873in}{2.837370in}}%
\pgfpathclose%
\pgfusepath{fill}%
\end{pgfscope}%
\begin{pgfscope}%
\pgfpathrectangle{\pgfqpoint{1.150000in}{0.150000in}}{\pgfqpoint{5.700000in}{5.700000in}}%
\pgfusepath{clip}%
\pgfsetbuttcap%
\pgfsetroundjoin%
\definecolor{currentfill}{rgb}{0.233603,0.313828,0.543914}%
\pgfsetfillcolor{currentfill}%
\pgfsetfillopacity{0.700000}%
\pgfsetlinewidth{0.000000pt}%
\definecolor{currentstroke}{rgb}{0.000000,0.000000,0.000000}%
\pgfsetstrokecolor{currentstroke}%
\pgfsetdash{}{0pt}%
\pgfpathmoveto{\pgfqpoint{5.055805in}{2.926224in}}%
\pgfpathlineto{\pgfqpoint{5.069475in}{2.921405in}}%
\pgfpathlineto{\pgfqpoint{5.083151in}{2.916656in}}%
\pgfpathlineto{\pgfqpoint{5.096834in}{2.911977in}}%
\pgfpathlineto{\pgfqpoint{5.110525in}{2.907367in}}%
\pgfpathlineto{\pgfqpoint{5.118100in}{2.923743in}}%
\pgfpathlineto{\pgfqpoint{5.125680in}{2.940475in}}%
\pgfpathlineto{\pgfqpoint{5.133265in}{2.957573in}}%
\pgfpathlineto{\pgfqpoint{5.140854in}{2.975044in}}%
\pgfpathlineto{\pgfqpoint{5.127177in}{2.980131in}}%
\pgfpathlineto{\pgfqpoint{5.113508in}{2.985287in}}%
\pgfpathlineto{\pgfqpoint{5.099845in}{2.990513in}}%
\pgfpathlineto{\pgfqpoint{5.086188in}{2.995809in}}%
\pgfpathlineto{\pgfqpoint{5.078586in}{2.977853in}}%
\pgfpathlineto{\pgfqpoint{5.070988in}{2.960276in}}%
\pgfpathlineto{\pgfqpoint{5.063395in}{2.943069in}}%
\pgfpathlineto{\pgfqpoint{5.055805in}{2.926224in}}%
\pgfpathclose%
\pgfusepath{fill}%
\end{pgfscope}%
\begin{pgfscope}%
\pgfpathrectangle{\pgfqpoint{1.150000in}{0.150000in}}{\pgfqpoint{5.700000in}{5.700000in}}%
\pgfusepath{clip}%
\pgfsetbuttcap%
\pgfsetroundjoin%
\definecolor{currentfill}{rgb}{0.257322,0.256130,0.526563}%
\pgfsetfillcolor{currentfill}%
\pgfsetfillopacity{0.700000}%
\pgfsetlinewidth{0.000000pt}%
\definecolor{currentstroke}{rgb}{0.000000,0.000000,0.000000}%
\pgfsetstrokecolor{currentstroke}%
\pgfsetdash{}{0pt}%
\pgfpathmoveto{\pgfqpoint{4.800953in}{2.796790in}}%
\pgfpathlineto{\pgfqpoint{4.814580in}{2.792431in}}%
\pgfpathlineto{\pgfqpoint{4.828214in}{2.788145in}}%
\pgfpathlineto{\pgfqpoint{4.841854in}{2.783930in}}%
\pgfpathlineto{\pgfqpoint{4.855501in}{2.779787in}}%
\pgfpathlineto{\pgfqpoint{4.863093in}{2.793765in}}%
\pgfpathlineto{\pgfqpoint{4.870685in}{2.808016in}}%
\pgfpathlineto{\pgfqpoint{4.878279in}{2.822549in}}%
\pgfpathlineto{\pgfqpoint{4.885873in}{2.837370in}}%
\pgfpathlineto{\pgfqpoint{4.872238in}{2.841930in}}%
\pgfpathlineto{\pgfqpoint{4.858610in}{2.846562in}}%
\pgfpathlineto{\pgfqpoint{4.844989in}{2.851265in}}%
\pgfpathlineto{\pgfqpoint{4.831375in}{2.856040in}}%
\pgfpathlineto{\pgfqpoint{4.823769in}{2.840794in}}%
\pgfpathlineto{\pgfqpoint{4.816163in}{2.825842in}}%
\pgfpathlineto{\pgfqpoint{4.808558in}{2.811177in}}%
\pgfpathlineto{\pgfqpoint{4.800953in}{2.796790in}}%
\pgfpathclose%
\pgfusepath{fill}%
\end{pgfscope}%
\begin{pgfscope}%
\pgfpathrectangle{\pgfqpoint{1.150000in}{0.150000in}}{\pgfqpoint{5.700000in}{5.700000in}}%
\pgfusepath{clip}%
\pgfsetbuttcap%
\pgfsetroundjoin%
\definecolor{currentfill}{rgb}{0.187231,0.414746,0.556547}%
\pgfsetfillcolor{currentfill}%
\pgfsetfillopacity{0.700000}%
\pgfsetlinewidth{0.000000pt}%
\definecolor{currentstroke}{rgb}{0.000000,0.000000,0.000000}%
\pgfsetstrokecolor{currentstroke}%
\pgfsetdash{}{0pt}%
\pgfpathmoveto{\pgfqpoint{5.341766in}{3.166586in}}%
\pgfpathlineto{\pgfqpoint{5.355462in}{3.160562in}}%
\pgfpathlineto{\pgfqpoint{5.369165in}{3.154606in}}%
\pgfpathlineto{\pgfqpoint{5.382874in}{3.148717in}}%
\pgfpathlineto{\pgfqpoint{5.396590in}{3.142896in}}%
\pgfpathlineto{\pgfqpoint{5.404237in}{3.164431in}}%
\pgfpathlineto{\pgfqpoint{5.411895in}{3.186460in}}%
\pgfpathlineto{\pgfqpoint{5.419564in}{3.208995in}}%
\pgfpathlineto{\pgfqpoint{5.427245in}{3.232045in}}%
\pgfpathlineto{\pgfqpoint{5.413543in}{3.238426in}}%
\pgfpathlineto{\pgfqpoint{5.399847in}{3.244875in}}%
\pgfpathlineto{\pgfqpoint{5.386157in}{3.251391in}}%
\pgfpathlineto{\pgfqpoint{5.372474in}{3.257975in}}%
\pgfpathlineto{\pgfqpoint{5.364780in}{3.234357in}}%
\pgfpathlineto{\pgfqpoint{5.357097in}{3.211260in}}%
\pgfpathlineto{\pgfqpoint{5.349426in}{3.188673in}}%
\pgfpathlineto{\pgfqpoint{5.341766in}{3.166586in}}%
\pgfpathclose%
\pgfusepath{fill}%
\end{pgfscope}%
\begin{pgfscope}%
\pgfpathrectangle{\pgfqpoint{1.150000in}{0.150000in}}{\pgfqpoint{5.700000in}{5.700000in}}%
\pgfusepath{clip}%
\pgfsetbuttcap%
\pgfsetroundjoin%
\definecolor{currentfill}{rgb}{0.223925,0.334994,0.548053}%
\pgfsetfillcolor{currentfill}%
\pgfsetfillopacity{0.700000}%
\pgfsetlinewidth{0.000000pt}%
\definecolor{currentstroke}{rgb}{0.000000,0.000000,0.000000}%
\pgfsetstrokecolor{currentstroke}%
\pgfsetdash{}{0pt}%
\pgfpathmoveto{\pgfqpoint{5.140854in}{2.975044in}}%
\pgfpathlineto{\pgfqpoint{5.154537in}{2.970026in}}%
\pgfpathlineto{\pgfqpoint{5.168227in}{2.965077in}}%
\pgfpathlineto{\pgfqpoint{5.181924in}{2.960196in}}%
\pgfpathlineto{\pgfqpoint{5.195628in}{2.955385in}}%
\pgfpathlineto{\pgfqpoint{5.203209in}{2.972749in}}%
\pgfpathlineto{\pgfqpoint{5.210795in}{2.990500in}}%
\pgfpathlineto{\pgfqpoint{5.218386in}{3.008647in}}%
\pgfpathlineto{\pgfqpoint{5.225985in}{3.027199in}}%
\pgfpathlineto{\pgfqpoint{5.212295in}{3.032508in}}%
\pgfpathlineto{\pgfqpoint{5.198612in}{3.037885in}}%
\pgfpathlineto{\pgfqpoint{5.184935in}{3.043332in}}%
\pgfpathlineto{\pgfqpoint{5.171265in}{3.048847in}}%
\pgfpathlineto{\pgfqpoint{5.163653in}{3.029790in}}%
\pgfpathlineto{\pgfqpoint{5.156048in}{3.011143in}}%
\pgfpathlineto{\pgfqpoint{5.148448in}{2.992897in}}%
\pgfpathlineto{\pgfqpoint{5.140854in}{2.975044in}}%
\pgfpathclose%
\pgfusepath{fill}%
\end{pgfscope}%
\begin{pgfscope}%
\pgfpathrectangle{\pgfqpoint{1.150000in}{0.150000in}}{\pgfqpoint{5.700000in}{5.700000in}}%
\pgfusepath{clip}%
\pgfsetbuttcap%
\pgfsetroundjoin%
\definecolor{currentfill}{rgb}{0.278012,0.180367,0.486697}%
\pgfsetfillcolor{currentfill}%
\pgfsetfillopacity{0.700000}%
\pgfsetlinewidth{0.000000pt}%
\definecolor{currentstroke}{rgb}{0.000000,0.000000,0.000000}%
\pgfsetstrokecolor{currentstroke}%
\pgfsetdash{}{0pt}%
\pgfpathmoveto{\pgfqpoint{4.322139in}{2.640668in}}%
\pgfpathlineto{\pgfqpoint{4.335667in}{2.636298in}}%
\pgfpathlineto{\pgfqpoint{4.349201in}{2.632006in}}%
\pgfpathlineto{\pgfqpoint{4.362742in}{2.627791in}}%
\pgfpathlineto{\pgfqpoint{4.376288in}{2.623653in}}%
\pgfpathlineto{\pgfqpoint{4.383986in}{2.635464in}}%
\pgfpathlineto{\pgfqpoint{4.391680in}{2.647446in}}%
\pgfpathlineto{\pgfqpoint{4.399371in}{2.659605in}}%
\pgfpathlineto{\pgfqpoint{4.407059in}{2.671946in}}%
\pgfpathlineto{\pgfqpoint{4.393523in}{2.676401in}}%
\pgfpathlineto{\pgfqpoint{4.379993in}{2.680932in}}%
\pgfpathlineto{\pgfqpoint{4.366469in}{2.685541in}}%
\pgfpathlineto{\pgfqpoint{4.352951in}{2.690228in}}%
\pgfpathlineto{\pgfqpoint{4.345253in}{2.677562in}}%
\pgfpathlineto{\pgfqpoint{4.337551in}{2.665084in}}%
\pgfpathlineto{\pgfqpoint{4.329847in}{2.652788in}}%
\pgfpathlineto{\pgfqpoint{4.322139in}{2.640668in}}%
\pgfpathclose%
\pgfusepath{fill}%
\end{pgfscope}%
\begin{pgfscope}%
\pgfpathrectangle{\pgfqpoint{1.150000in}{0.150000in}}{\pgfqpoint{5.700000in}{5.700000in}}%
\pgfusepath{clip}%
\pgfsetbuttcap%
\pgfsetroundjoin%
\definecolor{currentfill}{rgb}{0.283072,0.130895,0.449241}%
\pgfsetfillcolor{currentfill}%
\pgfsetfillopacity{0.700000}%
\pgfsetlinewidth{0.000000pt}%
\definecolor{currentstroke}{rgb}{0.000000,0.000000,0.000000}%
\pgfsetstrokecolor{currentstroke}%
\pgfsetdash{}{0pt}%
\pgfpathmoveto{\pgfqpoint{3.287443in}{2.550945in}}%
\pgfpathlineto{\pgfqpoint{3.300799in}{2.543536in}}%
\pgfpathlineto{\pgfqpoint{3.314158in}{2.536230in}}%
\pgfpathlineto{\pgfqpoint{3.327519in}{2.529028in}}%
\pgfpathlineto{\pgfqpoint{3.340883in}{2.521928in}}%
\pgfpathlineto{\pgfqpoint{3.348894in}{2.532510in}}%
\pgfpathlineto{\pgfqpoint{3.356899in}{2.543181in}}%
\pgfpathlineto{\pgfqpoint{3.364897in}{2.553946in}}%
\pgfpathlineto{\pgfqpoint{3.372890in}{2.564806in}}%
\pgfpathlineto{\pgfqpoint{3.359535in}{2.572042in}}%
\pgfpathlineto{\pgfqpoint{3.346183in}{2.579380in}}%
\pgfpathlineto{\pgfqpoint{3.332833in}{2.586822in}}%
\pgfpathlineto{\pgfqpoint{3.319486in}{2.594367in}}%
\pgfpathlineto{\pgfqpoint{3.311485in}{2.583363in}}%
\pgfpathlineto{\pgfqpoint{3.303477in}{2.572460in}}%
\pgfpathlineto{\pgfqpoint{3.295463in}{2.561655in}}%
\pgfpathlineto{\pgfqpoint{3.287443in}{2.550945in}}%
\pgfpathclose%
\pgfusepath{fill}%
\end{pgfscope}%
\begin{pgfscope}%
\pgfpathrectangle{\pgfqpoint{1.150000in}{0.150000in}}{\pgfqpoint{5.700000in}{5.700000in}}%
\pgfusepath{clip}%
\pgfsetbuttcap%
\pgfsetroundjoin%
\definecolor{currentfill}{rgb}{0.263663,0.237631,0.518762}%
\pgfsetfillcolor{currentfill}%
\pgfsetfillopacity{0.700000}%
\pgfsetlinewidth{0.000000pt}%
\definecolor{currentstroke}{rgb}{0.000000,0.000000,0.000000}%
\pgfsetstrokecolor{currentstroke}%
\pgfsetdash{}{0pt}%
\pgfpathmoveto{\pgfqpoint{4.716045in}{2.758451in}}%
\pgfpathlineto{\pgfqpoint{4.729658in}{2.754200in}}%
\pgfpathlineto{\pgfqpoint{4.743276in}{2.750021in}}%
\pgfpathlineto{\pgfqpoint{4.756902in}{2.745915in}}%
\pgfpathlineto{\pgfqpoint{4.770534in}{2.741881in}}%
\pgfpathlineto{\pgfqpoint{4.778140in}{2.755227in}}%
\pgfpathlineto{\pgfqpoint{4.785744in}{2.768822in}}%
\pgfpathlineto{\pgfqpoint{4.793349in}{2.782674in}}%
\pgfpathlineto{\pgfqpoint{4.800953in}{2.796790in}}%
\pgfpathlineto{\pgfqpoint{4.787333in}{2.801220in}}%
\pgfpathlineto{\pgfqpoint{4.773720in}{2.805723in}}%
\pgfpathlineto{\pgfqpoint{4.760113in}{2.810299in}}%
\pgfpathlineto{\pgfqpoint{4.746513in}{2.814947in}}%
\pgfpathlineto{\pgfqpoint{4.738897in}{2.800427in}}%
\pgfpathlineto{\pgfqpoint{4.731280in}{2.786176in}}%
\pgfpathlineto{\pgfqpoint{4.723663in}{2.772187in}}%
\pgfpathlineto{\pgfqpoint{4.716045in}{2.758451in}}%
\pgfpathclose%
\pgfusepath{fill}%
\end{pgfscope}%
\begin{pgfscope}%
\pgfpathrectangle{\pgfqpoint{1.150000in}{0.150000in}}{\pgfqpoint{5.700000in}{5.700000in}}%
\pgfusepath{clip}%
\pgfsetbuttcap%
\pgfsetroundjoin%
\definecolor{currentfill}{rgb}{0.282623,0.140926,0.457517}%
\pgfsetfillcolor{currentfill}%
\pgfsetfillopacity{0.700000}%
\pgfsetlinewidth{0.000000pt}%
\definecolor{currentstroke}{rgb}{0.000000,0.000000,0.000000}%
\pgfsetstrokecolor{currentstroke}%
\pgfsetdash{}{0pt}%
\pgfpathmoveto{\pgfqpoint{3.148453in}{2.571166in}}%
\pgfpathlineto{\pgfqpoint{3.161801in}{2.563018in}}%
\pgfpathlineto{\pgfqpoint{3.175151in}{2.554979in}}%
\pgfpathlineto{\pgfqpoint{3.188503in}{2.547048in}}%
\pgfpathlineto{\pgfqpoint{3.201858in}{2.539226in}}%
\pgfpathlineto{\pgfqpoint{3.209914in}{2.549690in}}%
\pgfpathlineto{\pgfqpoint{3.217963in}{2.560244in}}%
\pgfpathlineto{\pgfqpoint{3.226006in}{2.570890in}}%
\pgfpathlineto{\pgfqpoint{3.234043in}{2.581631in}}%
\pgfpathlineto{\pgfqpoint{3.220698in}{2.589569in}}%
\pgfpathlineto{\pgfqpoint{3.207356in}{2.597615in}}%
\pgfpathlineto{\pgfqpoint{3.194015in}{2.605770in}}%
\pgfpathlineto{\pgfqpoint{3.180676in}{2.614035in}}%
\pgfpathlineto{\pgfqpoint{3.172630in}{2.603170in}}%
\pgfpathlineto{\pgfqpoint{3.164578in}{2.592406in}}%
\pgfpathlineto{\pgfqpoint{3.156519in}{2.581738in}}%
\pgfpathlineto{\pgfqpoint{3.148453in}{2.571166in}}%
\pgfpathclose%
\pgfusepath{fill}%
\end{pgfscope}%
\begin{pgfscope}%
\pgfpathrectangle{\pgfqpoint{1.150000in}{0.150000in}}{\pgfqpoint{5.700000in}{5.700000in}}%
\pgfusepath{clip}%
\pgfsetbuttcap%
\pgfsetroundjoin%
\definecolor{currentfill}{rgb}{0.276194,0.190074,0.493001}%
\pgfsetfillcolor{currentfill}%
\pgfsetfillopacity{0.700000}%
\pgfsetlinewidth{0.000000pt}%
\definecolor{currentstroke}{rgb}{0.000000,0.000000,0.000000}%
\pgfsetstrokecolor{currentstroke}%
\pgfsetdash{}{0pt}%
\pgfpathmoveto{\pgfqpoint{2.816507in}{2.673674in}}%
\pgfpathlineto{\pgfqpoint{2.829863in}{2.663315in}}%
\pgfpathlineto{\pgfqpoint{2.843218in}{2.653085in}}%
\pgfpathlineto{\pgfqpoint{2.856573in}{2.642982in}}%
\pgfpathlineto{\pgfqpoint{2.869928in}{2.633006in}}%
\pgfpathlineto{\pgfqpoint{2.878094in}{2.643165in}}%
\pgfpathlineto{\pgfqpoint{2.886252in}{2.653423in}}%
\pgfpathlineto{\pgfqpoint{2.894403in}{2.663781in}}%
\pgfpathlineto{\pgfqpoint{2.902546in}{2.674243in}}%
\pgfpathlineto{\pgfqpoint{2.889202in}{2.684294in}}%
\pgfpathlineto{\pgfqpoint{2.875858in}{2.694472in}}%
\pgfpathlineto{\pgfqpoint{2.862514in}{2.704777in}}%
\pgfpathlineto{\pgfqpoint{2.849170in}{2.715211in}}%
\pgfpathlineto{\pgfqpoint{2.841016in}{2.704667in}}%
\pgfpathlineto{\pgfqpoint{2.832854in}{2.694231in}}%
\pgfpathlineto{\pgfqpoint{2.824684in}{2.683900in}}%
\pgfpathlineto{\pgfqpoint{2.816507in}{2.673674in}}%
\pgfpathclose%
\pgfusepath{fill}%
\end{pgfscope}%
\begin{pgfscope}%
\pgfpathrectangle{\pgfqpoint{1.150000in}{0.150000in}}{\pgfqpoint{5.700000in}{5.700000in}}%
\pgfusepath{clip}%
\pgfsetbuttcap%
\pgfsetroundjoin%
\definecolor{currentfill}{rgb}{0.283187,0.125848,0.444960}%
\pgfsetfillcolor{currentfill}%
\pgfsetfillopacity{0.700000}%
\pgfsetlinewidth{0.000000pt}%
\definecolor{currentstroke}{rgb}{0.000000,0.000000,0.000000}%
\pgfsetstrokecolor{currentstroke}%
\pgfsetdash{}{0pt}%
\pgfpathmoveto{\pgfqpoint{3.426339in}{2.536870in}}%
\pgfpathlineto{\pgfqpoint{3.439709in}{2.530136in}}%
\pgfpathlineto{\pgfqpoint{3.453082in}{2.523499in}}%
\pgfpathlineto{\pgfqpoint{3.466459in}{2.516961in}}%
\pgfpathlineto{\pgfqpoint{3.479839in}{2.510520in}}%
\pgfpathlineto{\pgfqpoint{3.487807in}{2.521181in}}%
\pgfpathlineto{\pgfqpoint{3.495769in}{2.531933in}}%
\pgfpathlineto{\pgfqpoint{3.503725in}{2.542780in}}%
\pgfpathlineto{\pgfqpoint{3.511675in}{2.553725in}}%
\pgfpathlineto{\pgfqpoint{3.498304in}{2.560323in}}%
\pgfpathlineto{\pgfqpoint{3.484936in}{2.567017in}}%
\pgfpathlineto{\pgfqpoint{3.471571in}{2.573810in}}%
\pgfpathlineto{\pgfqpoint{3.458210in}{2.580701in}}%
\pgfpathlineto{\pgfqpoint{3.450252in}{2.569592in}}%
\pgfpathlineto{\pgfqpoint{3.442287in}{2.558587in}}%
\pgfpathlineto{\pgfqpoint{3.434316in}{2.547680in}}%
\pgfpathlineto{\pgfqpoint{3.426339in}{2.536870in}}%
\pgfpathclose%
\pgfusepath{fill}%
\end{pgfscope}%
\begin{pgfscope}%
\pgfpathrectangle{\pgfqpoint{1.150000in}{0.150000in}}{\pgfqpoint{5.700000in}{5.700000in}}%
\pgfusepath{clip}%
\pgfsetbuttcap%
\pgfsetroundjoin%
\definecolor{currentfill}{rgb}{0.283072,0.130895,0.449241}%
\pgfsetfillcolor{currentfill}%
\pgfsetfillopacity{0.700000}%
\pgfsetlinewidth{0.000000pt}%
\definecolor{currentstroke}{rgb}{0.000000,0.000000,0.000000}%
\pgfsetstrokecolor{currentstroke}%
\pgfsetdash{}{0pt}%
\pgfpathmoveto{\pgfqpoint{3.789251in}{2.546439in}}%
\pgfpathlineto{\pgfqpoint{3.802675in}{2.541038in}}%
\pgfpathlineto{\pgfqpoint{3.816102in}{2.535725in}}%
\pgfpathlineto{\pgfqpoint{3.829535in}{2.530499in}}%
\pgfpathlineto{\pgfqpoint{3.842972in}{2.525360in}}%
\pgfpathlineto{\pgfqpoint{3.850827in}{2.536293in}}%
\pgfpathlineto{\pgfqpoint{3.858677in}{2.547336in}}%
\pgfpathlineto{\pgfqpoint{3.866522in}{2.558491in}}%
\pgfpathlineto{\pgfqpoint{3.874361in}{2.569762in}}%
\pgfpathlineto{\pgfqpoint{3.860933in}{2.575118in}}%
\pgfpathlineto{\pgfqpoint{3.847509in}{2.580560in}}%
\pgfpathlineto{\pgfqpoint{3.834090in}{2.586090in}}%
\pgfpathlineto{\pgfqpoint{3.820676in}{2.591707in}}%
\pgfpathlineto{\pgfqpoint{3.812828in}{2.580212in}}%
\pgfpathlineto{\pgfqpoint{3.804974in}{2.568838in}}%
\pgfpathlineto{\pgfqpoint{3.797116in}{2.557582in}}%
\pgfpathlineto{\pgfqpoint{3.789251in}{2.546439in}}%
\pgfpathclose%
\pgfusepath{fill}%
\end{pgfscope}%
\begin{pgfscope}%
\pgfpathrectangle{\pgfqpoint{1.150000in}{0.150000in}}{\pgfqpoint{5.700000in}{5.700000in}}%
\pgfusepath{clip}%
\pgfsetbuttcap%
\pgfsetroundjoin%
\definecolor{currentfill}{rgb}{0.212395,0.359683,0.551710}%
\pgfsetfillcolor{currentfill}%
\pgfsetfillopacity{0.700000}%
\pgfsetlinewidth{0.000000pt}%
\definecolor{currentstroke}{rgb}{0.000000,0.000000,0.000000}%
\pgfsetstrokecolor{currentstroke}%
\pgfsetdash{}{0pt}%
\pgfpathmoveto{\pgfqpoint{5.225985in}{3.027199in}}%
\pgfpathlineto{\pgfqpoint{5.239681in}{3.021959in}}%
\pgfpathlineto{\pgfqpoint{5.253385in}{3.016787in}}%
\pgfpathlineto{\pgfqpoint{5.267096in}{3.011683in}}%
\pgfpathlineto{\pgfqpoint{5.280813in}{3.006648in}}%
\pgfpathlineto{\pgfqpoint{5.288404in}{3.025104in}}%
\pgfpathlineto{\pgfqpoint{5.296002in}{3.043980in}}%
\pgfpathlineto{\pgfqpoint{5.303608in}{3.063285in}}%
\pgfpathlineto{\pgfqpoint{5.311221in}{3.083028in}}%
\pgfpathlineto{\pgfqpoint{5.297518in}{3.088581in}}%
\pgfpathlineto{\pgfqpoint{5.283822in}{3.094203in}}%
\pgfpathlineto{\pgfqpoint{5.270132in}{3.099893in}}%
\pgfpathlineto{\pgfqpoint{5.256449in}{3.105651in}}%
\pgfpathlineto{\pgfqpoint{5.248821in}{3.085382in}}%
\pgfpathlineto{\pgfqpoint{5.241202in}{3.065557in}}%
\pgfpathlineto{\pgfqpoint{5.233590in}{3.046166in}}%
\pgfpathlineto{\pgfqpoint{5.225985in}{3.027199in}}%
\pgfpathclose%
\pgfusepath{fill}%
\end{pgfscope}%
\begin{pgfscope}%
\pgfpathrectangle{\pgfqpoint{1.150000in}{0.150000in}}{\pgfqpoint{5.700000in}{5.700000in}}%
\pgfusepath{clip}%
\pgfsetbuttcap%
\pgfsetroundjoin%
\definecolor{currentfill}{rgb}{0.282290,0.145912,0.461510}%
\pgfsetfillcolor{currentfill}%
\pgfsetfillopacity{0.700000}%
\pgfsetlinewidth{0.000000pt}%
\definecolor{currentstroke}{rgb}{0.000000,0.000000,0.000000}%
\pgfsetstrokecolor{currentstroke}%
\pgfsetdash{}{0pt}%
\pgfpathmoveto{\pgfqpoint{4.013205in}{2.574449in}}%
\pgfpathlineto{\pgfqpoint{4.026670in}{2.569622in}}%
\pgfpathlineto{\pgfqpoint{4.040141in}{2.564878in}}%
\pgfpathlineto{\pgfqpoint{4.053617in}{2.560216in}}%
\pgfpathlineto{\pgfqpoint{4.067099in}{2.555637in}}%
\pgfpathlineto{\pgfqpoint{4.074886in}{2.566810in}}%
\pgfpathlineto{\pgfqpoint{4.082669in}{2.578112in}}%
\pgfpathlineto{\pgfqpoint{4.090447in}{2.589548in}}%
\pgfpathlineto{\pgfqpoint{4.098220in}{2.601121in}}%
\pgfpathlineto{\pgfqpoint{4.084748in}{2.605957in}}%
\pgfpathlineto{\pgfqpoint{4.071281in}{2.610875in}}%
\pgfpathlineto{\pgfqpoint{4.057820in}{2.615876in}}%
\pgfpathlineto{\pgfqpoint{4.044363in}{2.620960in}}%
\pgfpathlineto{\pgfqpoint{4.036581in}{2.609123in}}%
\pgfpathlineto{\pgfqpoint{4.028793in}{2.597428in}}%
\pgfpathlineto{\pgfqpoint{4.021002in}{2.585872in}}%
\pgfpathlineto{\pgfqpoint{4.013205in}{2.574449in}}%
\pgfpathclose%
\pgfusepath{fill}%
\end{pgfscope}%
\begin{pgfscope}%
\pgfpathrectangle{\pgfqpoint{1.150000in}{0.150000in}}{\pgfqpoint{5.700000in}{5.700000in}}%
\pgfusepath{clip}%
\pgfsetbuttcap%
\pgfsetroundjoin%
\definecolor{currentfill}{rgb}{0.267968,0.223549,0.512008}%
\pgfsetfillcolor{currentfill}%
\pgfsetfillopacity{0.700000}%
\pgfsetlinewidth{0.000000pt}%
\definecolor{currentstroke}{rgb}{0.000000,0.000000,0.000000}%
\pgfsetstrokecolor{currentstroke}%
\pgfsetdash{}{0pt}%
\pgfpathmoveto{\pgfqpoint{4.631136in}{2.722143in}}%
\pgfpathlineto{\pgfqpoint{4.644733in}{2.717975in}}%
\pgfpathlineto{\pgfqpoint{4.658337in}{2.713881in}}%
\pgfpathlineto{\pgfqpoint{4.671948in}{2.709860in}}%
\pgfpathlineto{\pgfqpoint{4.685565in}{2.705913in}}%
\pgfpathlineto{\pgfqpoint{4.693187in}{2.718701in}}%
\pgfpathlineto{\pgfqpoint{4.700808in}{2.731715in}}%
\pgfpathlineto{\pgfqpoint{4.708427in}{2.744963in}}%
\pgfpathlineto{\pgfqpoint{4.716045in}{2.758451in}}%
\pgfpathlineto{\pgfqpoint{4.702440in}{2.762776in}}%
\pgfpathlineto{\pgfqpoint{4.688841in}{2.767173in}}%
\pgfpathlineto{\pgfqpoint{4.675249in}{2.771645in}}%
\pgfpathlineto{\pgfqpoint{4.661663in}{2.776189in}}%
\pgfpathlineto{\pgfqpoint{4.654033in}{2.762317in}}%
\pgfpathlineto{\pgfqpoint{4.646402in}{2.748690in}}%
\pgfpathlineto{\pgfqpoint{4.638770in}{2.735301in}}%
\pgfpathlineto{\pgfqpoint{4.631136in}{2.722143in}}%
\pgfpathclose%
\pgfusepath{fill}%
\end{pgfscope}%
\begin{pgfscope}%
\pgfpathrectangle{\pgfqpoint{1.150000in}{0.150000in}}{\pgfqpoint{5.700000in}{5.700000in}}%
\pgfusepath{clip}%
\pgfsetbuttcap%
\pgfsetroundjoin%
\definecolor{currentfill}{rgb}{0.281412,0.155834,0.469201}%
\pgfsetfillcolor{currentfill}%
\pgfsetfillopacity{0.700000}%
\pgfsetlinewidth{0.000000pt}%
\definecolor{currentstroke}{rgb}{0.000000,0.000000,0.000000}%
\pgfsetstrokecolor{currentstroke}%
\pgfsetdash{}{0pt}%
\pgfpathmoveto{\pgfqpoint{3.009305in}{2.598254in}}%
\pgfpathlineto{\pgfqpoint{3.022653in}{2.589294in}}%
\pgfpathlineto{\pgfqpoint{3.036002in}{2.580450in}}%
\pgfpathlineto{\pgfqpoint{3.049351in}{2.571722in}}%
\pgfpathlineto{\pgfqpoint{3.062702in}{2.563108in}}%
\pgfpathlineto{\pgfqpoint{3.070806in}{2.573411in}}%
\pgfpathlineto{\pgfqpoint{3.078903in}{2.583804in}}%
\pgfpathlineto{\pgfqpoint{3.086993in}{2.594291in}}%
\pgfpathlineto{\pgfqpoint{3.095075in}{2.604874in}}%
\pgfpathlineto{\pgfqpoint{3.081735in}{2.613583in}}%
\pgfpathlineto{\pgfqpoint{3.068395in}{2.622407in}}%
\pgfpathlineto{\pgfqpoint{3.055057in}{2.631346in}}%
\pgfpathlineto{\pgfqpoint{3.041719in}{2.640402in}}%
\pgfpathlineto{\pgfqpoint{3.033627in}{2.629716in}}%
\pgfpathlineto{\pgfqpoint{3.025527in}{2.619131in}}%
\pgfpathlineto{\pgfqpoint{3.017419in}{2.608645in}}%
\pgfpathlineto{\pgfqpoint{3.009305in}{2.598254in}}%
\pgfpathclose%
\pgfusepath{fill}%
\end{pgfscope}%
\begin{pgfscope}%
\pgfpathrectangle{\pgfqpoint{1.150000in}{0.150000in}}{\pgfqpoint{5.700000in}{5.700000in}}%
\pgfusepath{clip}%
\pgfsetbuttcap%
\pgfsetroundjoin%
\definecolor{currentfill}{rgb}{0.283229,0.120777,0.440584}%
\pgfsetfillcolor{currentfill}%
\pgfsetfillopacity{0.700000}%
\pgfsetlinewidth{0.000000pt}%
\definecolor{currentstroke}{rgb}{0.000000,0.000000,0.000000}%
\pgfsetstrokecolor{currentstroke}%
\pgfsetdash{}{0pt}%
\pgfpathmoveto{\pgfqpoint{3.565195in}{2.528295in}}%
\pgfpathlineto{\pgfqpoint{3.578584in}{2.522175in}}%
\pgfpathlineto{\pgfqpoint{3.591978in}{2.516148in}}%
\pgfpathlineto{\pgfqpoint{3.605375in}{2.510215in}}%
\pgfpathlineto{\pgfqpoint{3.618777in}{2.504374in}}%
\pgfpathlineto{\pgfqpoint{3.626703in}{2.515082in}}%
\pgfpathlineto{\pgfqpoint{3.634623in}{2.525884in}}%
\pgfpathlineto{\pgfqpoint{3.642538in}{2.536784in}}%
\pgfpathlineto{\pgfqpoint{3.650447in}{2.547784in}}%
\pgfpathlineto{\pgfqpoint{3.637055in}{2.553801in}}%
\pgfpathlineto{\pgfqpoint{3.623666in}{2.559911in}}%
\pgfpathlineto{\pgfqpoint{3.610282in}{2.566114in}}%
\pgfpathlineto{\pgfqpoint{3.596901in}{2.572410in}}%
\pgfpathlineto{\pgfqpoint{3.588983in}{2.561226in}}%
\pgfpathlineto{\pgfqpoint{3.581060in}{2.550148in}}%
\pgfpathlineto{\pgfqpoint{3.573130in}{2.539172in}}%
\pgfpathlineto{\pgfqpoint{3.565195in}{2.528295in}}%
\pgfpathclose%
\pgfusepath{fill}%
\end{pgfscope}%
\begin{pgfscope}%
\pgfpathrectangle{\pgfqpoint{1.150000in}{0.150000in}}{\pgfqpoint{5.700000in}{5.700000in}}%
\pgfusepath{clip}%
\pgfsetbuttcap%
\pgfsetroundjoin%
\definecolor{currentfill}{rgb}{0.280255,0.165693,0.476498}%
\pgfsetfillcolor{currentfill}%
\pgfsetfillopacity{0.700000}%
\pgfsetlinewidth{0.000000pt}%
\definecolor{currentstroke}{rgb}{0.000000,0.000000,0.000000}%
\pgfsetstrokecolor{currentstroke}%
\pgfsetdash{}{0pt}%
\pgfpathmoveto{\pgfqpoint{4.237177in}{2.610914in}}%
\pgfpathlineto{\pgfqpoint{4.250691in}{2.606526in}}%
\pgfpathlineto{\pgfqpoint{4.264211in}{2.602218in}}%
\pgfpathlineto{\pgfqpoint{4.277737in}{2.597988in}}%
\pgfpathlineto{\pgfqpoint{4.291270in}{2.593837in}}%
\pgfpathlineto{\pgfqpoint{4.298993in}{2.605308in}}%
\pgfpathlineto{\pgfqpoint{4.306712in}{2.616934in}}%
\pgfpathlineto{\pgfqpoint{4.314427in}{2.628719in}}%
\pgfpathlineto{\pgfqpoint{4.322139in}{2.640668in}}%
\pgfpathlineto{\pgfqpoint{4.308616in}{2.645116in}}%
\pgfpathlineto{\pgfqpoint{4.295100in}{2.649643in}}%
\pgfpathlineto{\pgfqpoint{4.281590in}{2.654248in}}%
\pgfpathlineto{\pgfqpoint{4.268086in}{2.658932in}}%
\pgfpathlineto{\pgfqpoint{4.260364in}{2.646678in}}%
\pgfpathlineto{\pgfqpoint{4.252639in}{2.634594in}}%
\pgfpathlineto{\pgfqpoint{4.244910in}{2.622675in}}%
\pgfpathlineto{\pgfqpoint{4.237177in}{2.610914in}}%
\pgfpathclose%
\pgfusepath{fill}%
\end{pgfscope}%
\begin{pgfscope}%
\pgfpathrectangle{\pgfqpoint{1.150000in}{0.150000in}}{\pgfqpoint{5.700000in}{5.700000in}}%
\pgfusepath{clip}%
\pgfsetbuttcap%
\pgfsetroundjoin%
\definecolor{currentfill}{rgb}{0.201239,0.383670,0.554294}%
\pgfsetfillcolor{currentfill}%
\pgfsetfillopacity{0.700000}%
\pgfsetlinewidth{0.000000pt}%
\definecolor{currentstroke}{rgb}{0.000000,0.000000,0.000000}%
\pgfsetstrokecolor{currentstroke}%
\pgfsetdash{}{0pt}%
\pgfpathmoveto{\pgfqpoint{5.311221in}{3.083028in}}%
\pgfpathlineto{\pgfqpoint{5.324931in}{3.077542in}}%
\pgfpathlineto{\pgfqpoint{5.338648in}{3.072125in}}%
\pgfpathlineto{\pgfqpoint{5.352372in}{3.066775in}}%
\pgfpathlineto{\pgfqpoint{5.366102in}{3.061493in}}%
\pgfpathlineto{\pgfqpoint{5.373710in}{3.081153in}}%
\pgfpathlineto{\pgfqpoint{5.381327in}{3.101267in}}%
\pgfpathlineto{\pgfqpoint{5.388954in}{3.121845in}}%
\pgfpathlineto{\pgfqpoint{5.396590in}{3.142896in}}%
\pgfpathlineto{\pgfqpoint{5.382874in}{3.148717in}}%
\pgfpathlineto{\pgfqpoint{5.369165in}{3.154606in}}%
\pgfpathlineto{\pgfqpoint{5.355462in}{3.160562in}}%
\pgfpathlineto{\pgfqpoint{5.341766in}{3.166586in}}%
\pgfpathlineto{\pgfqpoint{5.334115in}{3.144989in}}%
\pgfpathlineto{\pgfqpoint{5.326475in}{3.123870in}}%
\pgfpathlineto{\pgfqpoint{5.318844in}{3.103220in}}%
\pgfpathlineto{\pgfqpoint{5.311221in}{3.083028in}}%
\pgfpathclose%
\pgfusepath{fill}%
\end{pgfscope}%
\begin{pgfscope}%
\pgfpathrectangle{\pgfqpoint{1.150000in}{0.150000in}}{\pgfqpoint{5.700000in}{5.700000in}}%
\pgfusepath{clip}%
\pgfsetbuttcap%
\pgfsetroundjoin%
\definecolor{currentfill}{rgb}{0.273006,0.204520,0.501721}%
\pgfsetfillcolor{currentfill}%
\pgfsetfillopacity{0.700000}%
\pgfsetlinewidth{0.000000pt}%
\definecolor{currentstroke}{rgb}{0.000000,0.000000,0.000000}%
\pgfsetstrokecolor{currentstroke}%
\pgfsetdash{}{0pt}%
\pgfpathmoveto{\pgfqpoint{4.546213in}{2.687679in}}%
\pgfpathlineto{\pgfqpoint{4.559795in}{2.683571in}}%
\pgfpathlineto{\pgfqpoint{4.573385in}{2.679537in}}%
\pgfpathlineto{\pgfqpoint{4.586980in}{2.675578in}}%
\pgfpathlineto{\pgfqpoint{4.600583in}{2.671693in}}%
\pgfpathlineto{\pgfqpoint{4.608224in}{2.683992in}}%
\pgfpathlineto{\pgfqpoint{4.615863in}{2.696495in}}%
\pgfpathlineto{\pgfqpoint{4.623500in}{2.709210in}}%
\pgfpathlineto{\pgfqpoint{4.631136in}{2.722143in}}%
\pgfpathlineto{\pgfqpoint{4.617545in}{2.726385in}}%
\pgfpathlineto{\pgfqpoint{4.603961in}{2.730701in}}%
\pgfpathlineto{\pgfqpoint{4.590383in}{2.735092in}}%
\pgfpathlineto{\pgfqpoint{4.576812in}{2.739557in}}%
\pgfpathlineto{\pgfqpoint{4.569165in}{2.726260in}}%
\pgfpathlineto{\pgfqpoint{4.561516in}{2.713185in}}%
\pgfpathlineto{\pgfqpoint{4.553866in}{2.700327in}}%
\pgfpathlineto{\pgfqpoint{4.546213in}{2.687679in}}%
\pgfpathclose%
\pgfusepath{fill}%
\end{pgfscope}%
\begin{pgfscope}%
\pgfpathrectangle{\pgfqpoint{1.150000in}{0.150000in}}{\pgfqpoint{5.700000in}{5.700000in}}%
\pgfusepath{clip}%
\pgfsetbuttcap%
\pgfsetroundjoin%
\definecolor{currentfill}{rgb}{0.177423,0.437527,0.557565}%
\pgfsetfillcolor{currentfill}%
\pgfsetfillopacity{0.700000}%
\pgfsetlinewidth{0.000000pt}%
\definecolor{currentstroke}{rgb}{0.000000,0.000000,0.000000}%
\pgfsetstrokecolor{currentstroke}%
\pgfsetdash{}{0pt}%
\pgfpathmoveto{\pgfqpoint{5.427245in}{3.232045in}}%
\pgfpathlineto{\pgfqpoint{5.440954in}{3.225732in}}%
\pgfpathlineto{\pgfqpoint{5.454670in}{3.219486in}}%
\pgfpathlineto{\pgfqpoint{5.468392in}{3.213307in}}%
\pgfpathlineto{\pgfqpoint{5.482120in}{3.207196in}}%
\pgfpathlineto{\pgfqpoint{5.489800in}{3.230200in}}%
\pgfpathlineto{\pgfqpoint{5.497493in}{3.253736in}}%
\pgfpathlineto{\pgfqpoint{5.505199in}{3.277816in}}%
\pgfpathlineto{\pgfqpoint{5.491481in}{3.284362in}}%
\pgfpathlineto{\pgfqpoint{5.477769in}{3.290974in}}%
\pgfpathlineto{\pgfqpoint{5.464063in}{3.297654in}}%
\pgfpathlineto{\pgfqpoint{5.450364in}{3.304401in}}%
\pgfpathlineto{\pgfqpoint{5.442645in}{3.279737in}}%
\pgfpathlineto{\pgfqpoint{5.434939in}{3.255622in}}%
\pgfpathlineto{\pgfqpoint{5.427245in}{3.232045in}}%
\pgfpathclose%
\pgfusepath{fill}%
\end{pgfscope}%
\begin{pgfscope}%
\pgfpathrectangle{\pgfqpoint{1.150000in}{0.150000in}}{\pgfqpoint{5.700000in}{5.700000in}}%
\pgfusepath{clip}%
\pgfsetbuttcap%
\pgfsetroundjoin%
\definecolor{currentfill}{rgb}{0.282884,0.135920,0.453427}%
\pgfsetfillcolor{currentfill}%
\pgfsetfillopacity{0.700000}%
\pgfsetlinewidth{0.000000pt}%
\definecolor{currentstroke}{rgb}{0.000000,0.000000,0.000000}%
\pgfsetstrokecolor{currentstroke}%
\pgfsetdash{}{0pt}%
\pgfpathmoveto{\pgfqpoint{3.928125in}{2.549202in}}%
\pgfpathlineto{\pgfqpoint{3.941578in}{2.544275in}}%
\pgfpathlineto{\pgfqpoint{3.955037in}{2.539433in}}%
\pgfpathlineto{\pgfqpoint{3.968501in}{2.534676in}}%
\pgfpathlineto{\pgfqpoint{3.981970in}{2.530002in}}%
\pgfpathlineto{\pgfqpoint{3.989786in}{2.540936in}}%
\pgfpathlineto{\pgfqpoint{3.997597in}{2.551986in}}%
\pgfpathlineto{\pgfqpoint{4.005404in}{2.563155in}}%
\pgfpathlineto{\pgfqpoint{4.013205in}{2.574449in}}%
\pgfpathlineto{\pgfqpoint{3.999745in}{2.579360in}}%
\pgfpathlineto{\pgfqpoint{3.986290in}{2.584355in}}%
\pgfpathlineto{\pgfqpoint{3.972841in}{2.589433in}}%
\pgfpathlineto{\pgfqpoint{3.959396in}{2.594597in}}%
\pgfpathlineto{\pgfqpoint{3.951586in}{2.583059in}}%
\pgfpathlineto{\pgfqpoint{3.943770in}{2.571650in}}%
\pgfpathlineto{\pgfqpoint{3.935950in}{2.560366in}}%
\pgfpathlineto{\pgfqpoint{3.928125in}{2.549202in}}%
\pgfpathclose%
\pgfusepath{fill}%
\end{pgfscope}%
\begin{pgfscope}%
\pgfpathrectangle{\pgfqpoint{1.150000in}{0.150000in}}{\pgfqpoint{5.700000in}{5.700000in}}%
\pgfusepath{clip}%
\pgfsetbuttcap%
\pgfsetroundjoin%
\definecolor{currentfill}{rgb}{0.279574,0.170599,0.479997}%
\pgfsetfillcolor{currentfill}%
\pgfsetfillopacity{0.700000}%
\pgfsetlinewidth{0.000000pt}%
\definecolor{currentstroke}{rgb}{0.000000,0.000000,0.000000}%
\pgfsetstrokecolor{currentstroke}%
\pgfsetdash{}{0pt}%
\pgfpathmoveto{\pgfqpoint{2.869928in}{2.633006in}}%
\pgfpathlineto{\pgfqpoint{2.883283in}{2.623155in}}%
\pgfpathlineto{\pgfqpoint{2.896638in}{2.613428in}}%
\pgfpathlineto{\pgfqpoint{2.909994in}{2.603824in}}%
\pgfpathlineto{\pgfqpoint{2.923349in}{2.594343in}}%
\pgfpathlineto{\pgfqpoint{2.931503in}{2.604434in}}%
\pgfpathlineto{\pgfqpoint{2.939650in}{2.614619in}}%
\pgfpathlineto{\pgfqpoint{2.947790in}{2.624900in}}%
\pgfpathlineto{\pgfqpoint{2.955922in}{2.635279in}}%
\pgfpathlineto{\pgfqpoint{2.942578in}{2.644835in}}%
\pgfpathlineto{\pgfqpoint{2.929234in}{2.654514in}}%
\pgfpathlineto{\pgfqpoint{2.915890in}{2.664316in}}%
\pgfpathlineto{\pgfqpoint{2.902546in}{2.674243in}}%
\pgfpathlineto{\pgfqpoint{2.894403in}{2.663781in}}%
\pgfpathlineto{\pgfqpoint{2.886252in}{2.653423in}}%
\pgfpathlineto{\pgfqpoint{2.878094in}{2.643165in}}%
\pgfpathlineto{\pgfqpoint{2.869928in}{2.633006in}}%
\pgfpathclose%
\pgfusepath{fill}%
\end{pgfscope}%
\begin{pgfscope}%
\pgfpathrectangle{\pgfqpoint{1.150000in}{0.150000in}}{\pgfqpoint{5.700000in}{5.700000in}}%
\pgfusepath{clip}%
\pgfsetbuttcap%
\pgfsetroundjoin%
\definecolor{currentfill}{rgb}{0.283187,0.125848,0.444960}%
\pgfsetfillcolor{currentfill}%
\pgfsetfillopacity{0.700000}%
\pgfsetlinewidth{0.000000pt}%
\definecolor{currentstroke}{rgb}{0.000000,0.000000,0.000000}%
\pgfsetstrokecolor{currentstroke}%
\pgfsetdash{}{0pt}%
\pgfpathmoveto{\pgfqpoint{3.704059in}{2.524633in}}%
\pgfpathlineto{\pgfqpoint{3.717473in}{2.519073in}}%
\pgfpathlineto{\pgfqpoint{3.730891in}{2.513601in}}%
\pgfpathlineto{\pgfqpoint{3.744314in}{2.508220in}}%
\pgfpathlineto{\pgfqpoint{3.757742in}{2.502927in}}%
\pgfpathlineto{\pgfqpoint{3.765627in}{2.513654in}}%
\pgfpathlineto{\pgfqpoint{3.773507in}{2.524479in}}%
\pgfpathlineto{\pgfqpoint{3.781382in}{2.535406in}}%
\pgfpathlineto{\pgfqpoint{3.789251in}{2.546439in}}%
\pgfpathlineto{\pgfqpoint{3.775833in}{2.551929in}}%
\pgfpathlineto{\pgfqpoint{3.762419in}{2.557507in}}%
\pgfpathlineto{\pgfqpoint{3.749010in}{2.563175in}}%
\pgfpathlineto{\pgfqpoint{3.735605in}{2.568932in}}%
\pgfpathlineto{\pgfqpoint{3.727727in}{2.557695in}}%
\pgfpathlineto{\pgfqpoint{3.719843in}{2.546569in}}%
\pgfpathlineto{\pgfqpoint{3.711954in}{2.535550in}}%
\pgfpathlineto{\pgfqpoint{3.704059in}{2.524633in}}%
\pgfpathclose%
\pgfusepath{fill}%
\end{pgfscope}%
\begin{pgfscope}%
\pgfpathrectangle{\pgfqpoint{1.150000in}{0.150000in}}{\pgfqpoint{5.700000in}{5.700000in}}%
\pgfusepath{clip}%
\pgfsetbuttcap%
\pgfsetroundjoin%
\definecolor{currentfill}{rgb}{0.281412,0.155834,0.469201}%
\pgfsetfillcolor{currentfill}%
\pgfsetfillopacity{0.700000}%
\pgfsetlinewidth{0.000000pt}%
\definecolor{currentstroke}{rgb}{0.000000,0.000000,0.000000}%
\pgfsetstrokecolor{currentstroke}%
\pgfsetdash{}{0pt}%
\pgfpathmoveto{\pgfqpoint{4.152166in}{2.582593in}}%
\pgfpathlineto{\pgfqpoint{4.165666in}{2.578163in}}%
\pgfpathlineto{\pgfqpoint{4.179173in}{2.573813in}}%
\pgfpathlineto{\pgfqpoint{4.192685in}{2.569544in}}%
\pgfpathlineto{\pgfqpoint{4.206204in}{2.565354in}}%
\pgfpathlineto{\pgfqpoint{4.213953in}{2.576532in}}%
\pgfpathlineto{\pgfqpoint{4.221698in}{2.587848in}}%
\pgfpathlineto{\pgfqpoint{4.229439in}{2.599307in}}%
\pgfpathlineto{\pgfqpoint{4.237177in}{2.610914in}}%
\pgfpathlineto{\pgfqpoint{4.223668in}{2.615381in}}%
\pgfpathlineto{\pgfqpoint{4.210166in}{2.619927in}}%
\pgfpathlineto{\pgfqpoint{4.196669in}{2.624553in}}%
\pgfpathlineto{\pgfqpoint{4.183178in}{2.629260in}}%
\pgfpathlineto{\pgfqpoint{4.175431in}{2.617368in}}%
\pgfpathlineto{\pgfqpoint{4.167680in}{2.605630in}}%
\pgfpathlineto{\pgfqpoint{4.159925in}{2.594040in}}%
\pgfpathlineto{\pgfqpoint{4.152166in}{2.582593in}}%
\pgfpathclose%
\pgfusepath{fill}%
\end{pgfscope}%
\begin{pgfscope}%
\pgfpathrectangle{\pgfqpoint{1.150000in}{0.150000in}}{\pgfqpoint{5.700000in}{5.700000in}}%
\pgfusepath{clip}%
\pgfsetbuttcap%
\pgfsetroundjoin%
\definecolor{currentfill}{rgb}{0.276194,0.190074,0.493001}%
\pgfsetfillcolor{currentfill}%
\pgfsetfillopacity{0.700000}%
\pgfsetlinewidth{0.000000pt}%
\definecolor{currentstroke}{rgb}{0.000000,0.000000,0.000000}%
\pgfsetstrokecolor{currentstroke}%
\pgfsetdash{}{0pt}%
\pgfpathmoveto{\pgfqpoint{4.461267in}{2.654896in}}%
\pgfpathlineto{\pgfqpoint{4.474835in}{2.650823in}}%
\pgfpathlineto{\pgfqpoint{4.488409in}{2.646827in}}%
\pgfpathlineto{\pgfqpoint{4.501990in}{2.642905in}}%
\pgfpathlineto{\pgfqpoint{4.515577in}{2.639059in}}%
\pgfpathlineto{\pgfqpoint{4.523240in}{2.650931in}}%
\pgfpathlineto{\pgfqpoint{4.530900in}{2.662987in}}%
\pgfpathlineto{\pgfqpoint{4.538558in}{2.675235in}}%
\pgfpathlineto{\pgfqpoint{4.546213in}{2.687679in}}%
\pgfpathlineto{\pgfqpoint{4.532637in}{2.691862in}}%
\pgfpathlineto{\pgfqpoint{4.519067in}{2.696120in}}%
\pgfpathlineto{\pgfqpoint{4.505504in}{2.700454in}}%
\pgfpathlineto{\pgfqpoint{4.491948in}{2.704863in}}%
\pgfpathlineto{\pgfqpoint{4.484281in}{2.692074in}}%
\pgfpathlineto{\pgfqpoint{4.476612in}{2.679487in}}%
\pgfpathlineto{\pgfqpoint{4.468941in}{2.667097in}}%
\pgfpathlineto{\pgfqpoint{4.461267in}{2.654896in}}%
\pgfpathclose%
\pgfusepath{fill}%
\end{pgfscope}%
\begin{pgfscope}%
\pgfpathrectangle{\pgfqpoint{1.150000in}{0.150000in}}{\pgfqpoint{5.700000in}{5.700000in}}%
\pgfusepath{clip}%
\pgfsetbuttcap%
\pgfsetroundjoin%
\definecolor{currentfill}{rgb}{0.270595,0.214069,0.507052}%
\pgfsetfillcolor{currentfill}%
\pgfsetfillopacity{0.700000}%
\pgfsetlinewidth{0.000000pt}%
\definecolor{currentstroke}{rgb}{0.000000,0.000000,0.000000}%
\pgfsetstrokecolor{currentstroke}%
\pgfsetdash{}{0pt}%
\pgfpathmoveto{\pgfqpoint{2.676742in}{2.720979in}}%
\pgfpathlineto{\pgfqpoint{2.690119in}{2.709605in}}%
\pgfpathlineto{\pgfqpoint{2.703494in}{2.698368in}}%
\pgfpathlineto{\pgfqpoint{2.716867in}{2.687269in}}%
\pgfpathlineto{\pgfqpoint{2.730240in}{2.676304in}}%
\pgfpathlineto{\pgfqpoint{2.738461in}{2.686176in}}%
\pgfpathlineto{\pgfqpoint{2.746674in}{2.696152in}}%
\pgfpathlineto{\pgfqpoint{2.754879in}{2.706232in}}%
\pgfpathlineto{\pgfqpoint{2.763076in}{2.716417in}}%
\pgfpathlineto{\pgfqpoint{2.749716in}{2.727436in}}%
\pgfpathlineto{\pgfqpoint{2.736355in}{2.738590in}}%
\pgfpathlineto{\pgfqpoint{2.722993in}{2.749881in}}%
\pgfpathlineto{\pgfqpoint{2.709629in}{2.761310in}}%
\pgfpathlineto{\pgfqpoint{2.701420in}{2.751062in}}%
\pgfpathlineto{\pgfqpoint{2.693202in}{2.740925in}}%
\pgfpathlineto{\pgfqpoint{2.684976in}{2.730898in}}%
\pgfpathlineto{\pgfqpoint{2.676742in}{2.720979in}}%
\pgfpathclose%
\pgfusepath{fill}%
\end{pgfscope}%
\begin{pgfscope}%
\pgfpathrectangle{\pgfqpoint{1.150000in}{0.150000in}}{\pgfqpoint{5.700000in}{5.700000in}}%
\pgfusepath{clip}%
\pgfsetbuttcap%
\pgfsetroundjoin%
\definecolor{currentfill}{rgb}{0.283072,0.130895,0.449241}%
\pgfsetfillcolor{currentfill}%
\pgfsetfillopacity{0.700000}%
\pgfsetlinewidth{0.000000pt}%
\definecolor{currentstroke}{rgb}{0.000000,0.000000,0.000000}%
\pgfsetstrokecolor{currentstroke}%
\pgfsetdash{}{0pt}%
\pgfpathmoveto{\pgfqpoint{3.201858in}{2.539226in}}%
\pgfpathlineto{\pgfqpoint{3.215214in}{2.531511in}}%
\pgfpathlineto{\pgfqpoint{3.228572in}{2.523903in}}%
\pgfpathlineto{\pgfqpoint{3.241933in}{2.516401in}}%
\pgfpathlineto{\pgfqpoint{3.255297in}{2.509003in}}%
\pgfpathlineto{\pgfqpoint{3.263343in}{2.519359in}}%
\pgfpathlineto{\pgfqpoint{3.271383in}{2.529799in}}%
\pgfpathlineto{\pgfqpoint{3.279416in}{2.540327in}}%
\pgfpathlineto{\pgfqpoint{3.287443in}{2.550945in}}%
\pgfpathlineto{\pgfqpoint{3.274089in}{2.558458in}}%
\pgfpathlineto{\pgfqpoint{3.260738in}{2.566076in}}%
\pgfpathlineto{\pgfqpoint{3.247389in}{2.573801in}}%
\pgfpathlineto{\pgfqpoint{3.234043in}{2.581631in}}%
\pgfpathlineto{\pgfqpoint{3.226006in}{2.570890in}}%
\pgfpathlineto{\pgfqpoint{3.217963in}{2.560244in}}%
\pgfpathlineto{\pgfqpoint{3.209914in}{2.549690in}}%
\pgfpathlineto{\pgfqpoint{3.201858in}{2.539226in}}%
\pgfpathclose%
\pgfusepath{fill}%
\end{pgfscope}%
\begin{pgfscope}%
\pgfpathrectangle{\pgfqpoint{1.150000in}{0.150000in}}{\pgfqpoint{5.700000in}{5.700000in}}%
\pgfusepath{clip}%
\pgfsetbuttcap%
\pgfsetroundjoin%
\definecolor{currentfill}{rgb}{0.243113,0.292092,0.538516}%
\pgfsetfillcolor{currentfill}%
\pgfsetfillopacity{0.700000}%
\pgfsetlinewidth{0.000000pt}%
\definecolor{currentstroke}{rgb}{0.000000,0.000000,0.000000}%
\pgfsetstrokecolor{currentstroke}%
\pgfsetdash{}{0pt}%
\pgfpathmoveto{\pgfqpoint{5.025481in}{2.862281in}}%
\pgfpathlineto{\pgfqpoint{5.039164in}{2.857920in}}%
\pgfpathlineto{\pgfqpoint{5.052854in}{2.853628in}}%
\pgfpathlineto{\pgfqpoint{5.066552in}{2.849406in}}%
\pgfpathlineto{\pgfqpoint{5.080257in}{2.845254in}}%
\pgfpathlineto{\pgfqpoint{5.087819in}{2.860290in}}%
\pgfpathlineto{\pgfqpoint{5.095384in}{2.875649in}}%
\pgfpathlineto{\pgfqpoint{5.102953in}{2.891339in}}%
\pgfpathlineto{\pgfqpoint{5.110525in}{2.907367in}}%
\pgfpathlineto{\pgfqpoint{5.096834in}{2.911977in}}%
\pgfpathlineto{\pgfqpoint{5.083151in}{2.916656in}}%
\pgfpathlineto{\pgfqpoint{5.069475in}{2.921405in}}%
\pgfpathlineto{\pgfqpoint{5.055805in}{2.926224in}}%
\pgfpathlineto{\pgfqpoint{5.048219in}{2.909731in}}%
\pgfpathlineto{\pgfqpoint{5.040637in}{2.893581in}}%
\pgfpathlineto{\pgfqpoint{5.033057in}{2.877768in}}%
\pgfpathlineto{\pgfqpoint{5.025481in}{2.862281in}}%
\pgfpathclose%
\pgfusepath{fill}%
\end{pgfscope}%
\begin{pgfscope}%
\pgfpathrectangle{\pgfqpoint{1.150000in}{0.150000in}}{\pgfqpoint{5.700000in}{5.700000in}}%
\pgfusepath{clip}%
\pgfsetbuttcap%
\pgfsetroundjoin%
\definecolor{currentfill}{rgb}{0.188923,0.410910,0.556326}%
\pgfsetfillcolor{currentfill}%
\pgfsetfillopacity{0.700000}%
\pgfsetlinewidth{0.000000pt}%
\definecolor{currentstroke}{rgb}{0.000000,0.000000,0.000000}%
\pgfsetstrokecolor{currentstroke}%
\pgfsetdash{}{0pt}%
\pgfpathmoveto{\pgfqpoint{5.396590in}{3.142896in}}%
\pgfpathlineto{\pgfqpoint{5.410313in}{3.137142in}}%
\pgfpathlineto{\pgfqpoint{5.424043in}{3.131456in}}%
\pgfpathlineto{\pgfqpoint{5.437779in}{3.125837in}}%
\pgfpathlineto{\pgfqpoint{5.451522in}{3.120285in}}%
\pgfpathlineto{\pgfqpoint{5.459155in}{3.141268in}}%
\pgfpathlineto{\pgfqpoint{5.466798in}{3.162741in}}%
\pgfpathlineto{\pgfqpoint{5.474453in}{3.184713in}}%
\pgfpathlineto{\pgfqpoint{5.482120in}{3.207196in}}%
\pgfpathlineto{\pgfqpoint{5.468392in}{3.213307in}}%
\pgfpathlineto{\pgfqpoint{5.454670in}{3.219486in}}%
\pgfpathlineto{\pgfqpoint{5.440954in}{3.225732in}}%
\pgfpathlineto{\pgfqpoint{5.427245in}{3.232045in}}%
\pgfpathlineto{\pgfqpoint{5.419564in}{3.208995in}}%
\pgfpathlineto{\pgfqpoint{5.411895in}{3.186460in}}%
\pgfpathlineto{\pgfqpoint{5.404237in}{3.164431in}}%
\pgfpathlineto{\pgfqpoint{5.396590in}{3.142896in}}%
\pgfpathclose%
\pgfusepath{fill}%
\end{pgfscope}%
\begin{pgfscope}%
\pgfpathrectangle{\pgfqpoint{1.150000in}{0.150000in}}{\pgfqpoint{5.700000in}{5.700000in}}%
\pgfusepath{clip}%
\pgfsetbuttcap%
\pgfsetroundjoin%
\definecolor{currentfill}{rgb}{0.283229,0.120777,0.440584}%
\pgfsetfillcolor{currentfill}%
\pgfsetfillopacity{0.700000}%
\pgfsetlinewidth{0.000000pt}%
\definecolor{currentstroke}{rgb}{0.000000,0.000000,0.000000}%
\pgfsetstrokecolor{currentstroke}%
\pgfsetdash{}{0pt}%
\pgfpathmoveto{\pgfqpoint{3.340883in}{2.521928in}}%
\pgfpathlineto{\pgfqpoint{3.354250in}{2.514929in}}%
\pgfpathlineto{\pgfqpoint{3.367620in}{2.508032in}}%
\pgfpathlineto{\pgfqpoint{3.380993in}{2.501235in}}%
\pgfpathlineto{\pgfqpoint{3.394369in}{2.494537in}}%
\pgfpathlineto{\pgfqpoint{3.402371in}{2.504990in}}%
\pgfpathlineto{\pgfqpoint{3.410366in}{2.515528in}}%
\pgfpathlineto{\pgfqpoint{3.418356in}{2.526154in}}%
\pgfpathlineto{\pgfqpoint{3.426339in}{2.536870in}}%
\pgfpathlineto{\pgfqpoint{3.412972in}{2.543704in}}%
\pgfpathlineto{\pgfqpoint{3.399608in}{2.550638in}}%
\pgfpathlineto{\pgfqpoint{3.386248in}{2.557671in}}%
\pgfpathlineto{\pgfqpoint{3.372890in}{2.564806in}}%
\pgfpathlineto{\pgfqpoint{3.364897in}{2.553946in}}%
\pgfpathlineto{\pgfqpoint{3.356899in}{2.543181in}}%
\pgfpathlineto{\pgfqpoint{3.348894in}{2.532510in}}%
\pgfpathlineto{\pgfqpoint{3.340883in}{2.521928in}}%
\pgfpathclose%
\pgfusepath{fill}%
\end{pgfscope}%
\begin{pgfscope}%
\pgfpathrectangle{\pgfqpoint{1.150000in}{0.150000in}}{\pgfqpoint{5.700000in}{5.700000in}}%
\pgfusepath{clip}%
\pgfsetbuttcap%
\pgfsetroundjoin%
\definecolor{currentfill}{rgb}{0.233603,0.313828,0.543914}%
\pgfsetfillcolor{currentfill}%
\pgfsetfillopacity{0.700000}%
\pgfsetlinewidth{0.000000pt}%
\definecolor{currentstroke}{rgb}{0.000000,0.000000,0.000000}%
\pgfsetstrokecolor{currentstroke}%
\pgfsetdash{}{0pt}%
\pgfpathmoveto{\pgfqpoint{5.110525in}{2.907367in}}%
\pgfpathlineto{\pgfqpoint{5.124222in}{2.902826in}}%
\pgfpathlineto{\pgfqpoint{5.137926in}{2.898355in}}%
\pgfpathlineto{\pgfqpoint{5.151638in}{2.893952in}}%
\pgfpathlineto{\pgfqpoint{5.165357in}{2.889618in}}%
\pgfpathlineto{\pgfqpoint{5.172918in}{2.905525in}}%
\pgfpathlineto{\pgfqpoint{5.180483in}{2.921782in}}%
\pgfpathlineto{\pgfqpoint{5.188053in}{2.938399in}}%
\pgfpathlineto{\pgfqpoint{5.195628in}{2.955385in}}%
\pgfpathlineto{\pgfqpoint{5.181924in}{2.960196in}}%
\pgfpathlineto{\pgfqpoint{5.168227in}{2.965077in}}%
\pgfpathlineto{\pgfqpoint{5.154537in}{2.970026in}}%
\pgfpathlineto{\pgfqpoint{5.140854in}{2.975044in}}%
\pgfpathlineto{\pgfqpoint{5.133265in}{2.957573in}}%
\pgfpathlineto{\pgfqpoint{5.125680in}{2.940475in}}%
\pgfpathlineto{\pgfqpoint{5.118100in}{2.923743in}}%
\pgfpathlineto{\pgfqpoint{5.110525in}{2.907367in}}%
\pgfpathclose%
\pgfusepath{fill}%
\end{pgfscope}%
\begin{pgfscope}%
\pgfpathrectangle{\pgfqpoint{1.150000in}{0.150000in}}{\pgfqpoint{5.700000in}{5.700000in}}%
\pgfusepath{clip}%
\pgfsetbuttcap%
\pgfsetroundjoin%
\definecolor{currentfill}{rgb}{0.250425,0.274290,0.533103}%
\pgfsetfillcolor{currentfill}%
\pgfsetfillopacity{0.700000}%
\pgfsetlinewidth{0.000000pt}%
\definecolor{currentstroke}{rgb}{0.000000,0.000000,0.000000}%
\pgfsetstrokecolor{currentstroke}%
\pgfsetdash{}{0pt}%
\pgfpathmoveto{\pgfqpoint{4.940479in}{2.819842in}}%
\pgfpathlineto{\pgfqpoint{4.954147in}{2.815637in}}%
\pgfpathlineto{\pgfqpoint{4.967823in}{2.811502in}}%
\pgfpathlineto{\pgfqpoint{4.981506in}{2.807438in}}%
\pgfpathlineto{\pgfqpoint{4.995197in}{2.803444in}}%
\pgfpathlineto{\pgfqpoint{5.002765in}{2.817703in}}%
\pgfpathlineto{\pgfqpoint{5.010335in}{2.832257in}}%
\pgfpathlineto{\pgfqpoint{5.017907in}{2.847114in}}%
\pgfpathlineto{\pgfqpoint{5.025481in}{2.862281in}}%
\pgfpathlineto{\pgfqpoint{5.011805in}{2.866713in}}%
\pgfpathlineto{\pgfqpoint{4.998135in}{2.871214in}}%
\pgfpathlineto{\pgfqpoint{4.984473in}{2.875786in}}%
\pgfpathlineto{\pgfqpoint{4.970818in}{2.880428in}}%
\pgfpathlineto{\pgfqpoint{4.963230in}{2.864816in}}%
\pgfpathlineto{\pgfqpoint{4.955644in}{2.849519in}}%
\pgfpathlineto{\pgfqpoint{4.948061in}{2.834531in}}%
\pgfpathlineto{\pgfqpoint{4.940479in}{2.819842in}}%
\pgfpathclose%
\pgfusepath{fill}%
\end{pgfscope}%
\begin{pgfscope}%
\pgfpathrectangle{\pgfqpoint{1.150000in}{0.150000in}}{\pgfqpoint{5.700000in}{5.700000in}}%
\pgfusepath{clip}%
\pgfsetbuttcap%
\pgfsetroundjoin%
\definecolor{currentfill}{rgb}{0.282623,0.140926,0.457517}%
\pgfsetfillcolor{currentfill}%
\pgfsetfillopacity{0.700000}%
\pgfsetlinewidth{0.000000pt}%
\definecolor{currentstroke}{rgb}{0.000000,0.000000,0.000000}%
\pgfsetstrokecolor{currentstroke}%
\pgfsetdash{}{0pt}%
\pgfpathmoveto{\pgfqpoint{3.062702in}{2.563108in}}%
\pgfpathlineto{\pgfqpoint{3.076055in}{2.554609in}}%
\pgfpathlineto{\pgfqpoint{3.089409in}{2.546222in}}%
\pgfpathlineto{\pgfqpoint{3.102764in}{2.537947in}}%
\pgfpathlineto{\pgfqpoint{3.116121in}{2.529784in}}%
\pgfpathlineto{\pgfqpoint{3.124214in}{2.539998in}}%
\pgfpathlineto{\pgfqpoint{3.132300in}{2.550298in}}%
\pgfpathlineto{\pgfqpoint{3.140380in}{2.560687in}}%
\pgfpathlineto{\pgfqpoint{3.148453in}{2.571166in}}%
\pgfpathlineto{\pgfqpoint{3.135106in}{2.579425in}}%
\pgfpathlineto{\pgfqpoint{3.121761in}{2.587796in}}%
\pgfpathlineto{\pgfqpoint{3.108417in}{2.596278in}}%
\pgfpathlineto{\pgfqpoint{3.095075in}{2.604874in}}%
\pgfpathlineto{\pgfqpoint{3.086993in}{2.594291in}}%
\pgfpathlineto{\pgfqpoint{3.078903in}{2.583804in}}%
\pgfpathlineto{\pgfqpoint{3.070806in}{2.573411in}}%
\pgfpathlineto{\pgfqpoint{3.062702in}{2.563108in}}%
\pgfpathclose%
\pgfusepath{fill}%
\end{pgfscope}%
\begin{pgfscope}%
\pgfpathrectangle{\pgfqpoint{1.150000in}{0.150000in}}{\pgfqpoint{5.700000in}{5.700000in}}%
\pgfusepath{clip}%
\pgfsetbuttcap%
\pgfsetroundjoin%
\definecolor{currentfill}{rgb}{0.223925,0.334994,0.548053}%
\pgfsetfillcolor{currentfill}%
\pgfsetfillopacity{0.700000}%
\pgfsetlinewidth{0.000000pt}%
\definecolor{currentstroke}{rgb}{0.000000,0.000000,0.000000}%
\pgfsetstrokecolor{currentstroke}%
\pgfsetdash{}{0pt}%
\pgfpathmoveto{\pgfqpoint{5.195628in}{2.955385in}}%
\pgfpathlineto{\pgfqpoint{5.209340in}{2.950643in}}%
\pgfpathlineto{\pgfqpoint{5.223058in}{2.945968in}}%
\pgfpathlineto{\pgfqpoint{5.236783in}{2.941362in}}%
\pgfpathlineto{\pgfqpoint{5.250516in}{2.936825in}}%
\pgfpathlineto{\pgfqpoint{5.258081in}{2.953699in}}%
\pgfpathlineto{\pgfqpoint{5.265652in}{2.970954in}}%
\pgfpathlineto{\pgfqpoint{5.273229in}{2.988601in}}%
\pgfpathlineto{\pgfqpoint{5.280813in}{3.006648in}}%
\pgfpathlineto{\pgfqpoint{5.267096in}{3.011683in}}%
\pgfpathlineto{\pgfqpoint{5.253385in}{3.016787in}}%
\pgfpathlineto{\pgfqpoint{5.239681in}{3.021959in}}%
\pgfpathlineto{\pgfqpoint{5.225985in}{3.027199in}}%
\pgfpathlineto{\pgfqpoint{5.218386in}{3.008647in}}%
\pgfpathlineto{\pgfqpoint{5.210795in}{2.990500in}}%
\pgfpathlineto{\pgfqpoint{5.203209in}{2.972749in}}%
\pgfpathlineto{\pgfqpoint{5.195628in}{2.955385in}}%
\pgfpathclose%
\pgfusepath{fill}%
\end{pgfscope}%
\begin{pgfscope}%
\pgfpathrectangle{\pgfqpoint{1.150000in}{0.150000in}}{\pgfqpoint{5.700000in}{5.700000in}}%
\pgfusepath{clip}%
\pgfsetbuttcap%
\pgfsetroundjoin%
\definecolor{currentfill}{rgb}{0.257322,0.256130,0.526563}%
\pgfsetfillcolor{currentfill}%
\pgfsetfillopacity{0.700000}%
\pgfsetlinewidth{0.000000pt}%
\definecolor{currentstroke}{rgb}{0.000000,0.000000,0.000000}%
\pgfsetstrokecolor{currentstroke}%
\pgfsetdash{}{0pt}%
\pgfpathmoveto{\pgfqpoint{4.855501in}{2.779787in}}%
\pgfpathlineto{\pgfqpoint{4.869155in}{2.775715in}}%
\pgfpathlineto{\pgfqpoint{4.882817in}{2.771715in}}%
\pgfpathlineto{\pgfqpoint{4.896485in}{2.767786in}}%
\pgfpathlineto{\pgfqpoint{4.910161in}{2.763927in}}%
\pgfpathlineto{\pgfqpoint{4.917739in}{2.777495in}}%
\pgfpathlineto{\pgfqpoint{4.925318in}{2.791331in}}%
\pgfpathlineto{\pgfqpoint{4.932898in}{2.805444in}}%
\pgfpathlineto{\pgfqpoint{4.940479in}{2.819842in}}%
\pgfpathlineto{\pgfqpoint{4.926817in}{2.824117in}}%
\pgfpathlineto{\pgfqpoint{4.913162in}{2.828464in}}%
\pgfpathlineto{\pgfqpoint{4.899514in}{2.832882in}}%
\pgfpathlineto{\pgfqpoint{4.885873in}{2.837370in}}%
\pgfpathlineto{\pgfqpoint{4.878279in}{2.822549in}}%
\pgfpathlineto{\pgfqpoint{4.870685in}{2.808016in}}%
\pgfpathlineto{\pgfqpoint{4.863093in}{2.793765in}}%
\pgfpathlineto{\pgfqpoint{4.855501in}{2.779787in}}%
\pgfpathclose%
\pgfusepath{fill}%
\end{pgfscope}%
\begin{pgfscope}%
\pgfpathrectangle{\pgfqpoint{1.150000in}{0.150000in}}{\pgfqpoint{5.700000in}{5.700000in}}%
\pgfusepath{clip}%
\pgfsetbuttcap%
\pgfsetroundjoin%
\definecolor{currentfill}{rgb}{0.283229,0.120777,0.440584}%
\pgfsetfillcolor{currentfill}%
\pgfsetfillopacity{0.700000}%
\pgfsetlinewidth{0.000000pt}%
\definecolor{currentstroke}{rgb}{0.000000,0.000000,0.000000}%
\pgfsetstrokecolor{currentstroke}%
\pgfsetdash{}{0pt}%
\pgfpathmoveto{\pgfqpoint{3.479839in}{2.510520in}}%
\pgfpathlineto{\pgfqpoint{3.493222in}{2.504175in}}%
\pgfpathlineto{\pgfqpoint{3.506610in}{2.497926in}}%
\pgfpathlineto{\pgfqpoint{3.520001in}{2.491773in}}%
\pgfpathlineto{\pgfqpoint{3.533395in}{2.485715in}}%
\pgfpathlineto{\pgfqpoint{3.541354in}{2.496227in}}%
\pgfpathlineto{\pgfqpoint{3.549307in}{2.506826in}}%
\pgfpathlineto{\pgfqpoint{3.557254in}{2.517514in}}%
\pgfpathlineto{\pgfqpoint{3.565195in}{2.528295in}}%
\pgfpathlineto{\pgfqpoint{3.551809in}{2.534510in}}%
\pgfpathlineto{\pgfqpoint{3.538428in}{2.540819in}}%
\pgfpathlineto{\pgfqpoint{3.525049in}{2.547224in}}%
\pgfpathlineto{\pgfqpoint{3.511675in}{2.553725in}}%
\pgfpathlineto{\pgfqpoint{3.503725in}{2.542780in}}%
\pgfpathlineto{\pgfqpoint{3.495769in}{2.531933in}}%
\pgfpathlineto{\pgfqpoint{3.487807in}{2.521181in}}%
\pgfpathlineto{\pgfqpoint{3.479839in}{2.510520in}}%
\pgfpathclose%
\pgfusepath{fill}%
\end{pgfscope}%
\begin{pgfscope}%
\pgfpathrectangle{\pgfqpoint{1.150000in}{0.150000in}}{\pgfqpoint{5.700000in}{5.700000in}}%
\pgfusepath{clip}%
\pgfsetbuttcap%
\pgfsetroundjoin%
\definecolor{currentfill}{rgb}{0.278012,0.180367,0.486697}%
\pgfsetfillcolor{currentfill}%
\pgfsetfillopacity{0.700000}%
\pgfsetlinewidth{0.000000pt}%
\definecolor{currentstroke}{rgb}{0.000000,0.000000,0.000000}%
\pgfsetstrokecolor{currentstroke}%
\pgfsetdash{}{0pt}%
\pgfpathmoveto{\pgfqpoint{4.376288in}{2.623653in}}%
\pgfpathlineto{\pgfqpoint{4.389841in}{2.619593in}}%
\pgfpathlineto{\pgfqpoint{4.403401in}{2.615609in}}%
\pgfpathlineto{\pgfqpoint{4.416967in}{2.611702in}}%
\pgfpathlineto{\pgfqpoint{4.430539in}{2.607871in}}%
\pgfpathlineto{\pgfqpoint{4.438226in}{2.619372in}}%
\pgfpathlineto{\pgfqpoint{4.445909in}{2.631039in}}%
\pgfpathlineto{\pgfqpoint{4.453590in}{2.642878in}}%
\pgfpathlineto{\pgfqpoint{4.461267in}{2.654896in}}%
\pgfpathlineto{\pgfqpoint{4.447705in}{2.659044in}}%
\pgfpathlineto{\pgfqpoint{4.434150in}{2.663268in}}%
\pgfpathlineto{\pgfqpoint{4.420602in}{2.667569in}}%
\pgfpathlineto{\pgfqpoint{4.407059in}{2.671946in}}%
\pgfpathlineto{\pgfqpoint{4.399371in}{2.659605in}}%
\pgfpathlineto{\pgfqpoint{4.391680in}{2.647446in}}%
\pgfpathlineto{\pgfqpoint{4.383986in}{2.635464in}}%
\pgfpathlineto{\pgfqpoint{4.376288in}{2.623653in}}%
\pgfpathclose%
\pgfusepath{fill}%
\end{pgfscope}%
\begin{pgfscope}%
\pgfpathrectangle{\pgfqpoint{1.150000in}{0.150000in}}{\pgfqpoint{5.700000in}{5.700000in}}%
\pgfusepath{clip}%
\pgfsetbuttcap%
\pgfsetroundjoin%
\definecolor{currentfill}{rgb}{0.214298,0.355619,0.551184}%
\pgfsetfillcolor{currentfill}%
\pgfsetfillopacity{0.700000}%
\pgfsetlinewidth{0.000000pt}%
\definecolor{currentstroke}{rgb}{0.000000,0.000000,0.000000}%
\pgfsetstrokecolor{currentstroke}%
\pgfsetdash{}{0pt}%
\pgfpathmoveto{\pgfqpoint{5.280813in}{3.006648in}}%
\pgfpathlineto{\pgfqpoint{5.294538in}{3.001680in}}%
\pgfpathlineto{\pgfqpoint{5.308269in}{2.996781in}}%
\pgfpathlineto{\pgfqpoint{5.322008in}{2.991949in}}%
\pgfpathlineto{\pgfqpoint{5.335754in}{2.987185in}}%
\pgfpathlineto{\pgfqpoint{5.343330in}{3.005131in}}%
\pgfpathlineto{\pgfqpoint{5.350912in}{3.023491in}}%
\pgfpathlineto{\pgfqpoint{5.358503in}{3.042275in}}%
\pgfpathlineto{\pgfqpoint{5.366102in}{3.061493in}}%
\pgfpathlineto{\pgfqpoint{5.352372in}{3.066775in}}%
\pgfpathlineto{\pgfqpoint{5.338648in}{3.072125in}}%
\pgfpathlineto{\pgfqpoint{5.324931in}{3.077542in}}%
\pgfpathlineto{\pgfqpoint{5.311221in}{3.083028in}}%
\pgfpathlineto{\pgfqpoint{5.303608in}{3.063285in}}%
\pgfpathlineto{\pgfqpoint{5.296002in}{3.043980in}}%
\pgfpathlineto{\pgfqpoint{5.288404in}{3.025104in}}%
\pgfpathlineto{\pgfqpoint{5.280813in}{3.006648in}}%
\pgfpathclose%
\pgfusepath{fill}%
\end{pgfscope}%
\begin{pgfscope}%
\pgfpathrectangle{\pgfqpoint{1.150000in}{0.150000in}}{\pgfqpoint{5.700000in}{5.700000in}}%
\pgfusepath{clip}%
\pgfsetbuttcap%
\pgfsetroundjoin%
\definecolor{currentfill}{rgb}{0.283072,0.130895,0.449241}%
\pgfsetfillcolor{currentfill}%
\pgfsetfillopacity{0.700000}%
\pgfsetlinewidth{0.000000pt}%
\definecolor{currentstroke}{rgb}{0.000000,0.000000,0.000000}%
\pgfsetstrokecolor{currentstroke}%
\pgfsetdash{}{0pt}%
\pgfpathmoveto{\pgfqpoint{3.842972in}{2.525360in}}%
\pgfpathlineto{\pgfqpoint{3.856415in}{2.520308in}}%
\pgfpathlineto{\pgfqpoint{3.869862in}{2.515342in}}%
\pgfpathlineto{\pgfqpoint{3.883315in}{2.510461in}}%
\pgfpathlineto{\pgfqpoint{3.896772in}{2.505667in}}%
\pgfpathlineto{\pgfqpoint{3.904618in}{2.516391in}}%
\pgfpathlineto{\pgfqpoint{3.912459in}{2.527219in}}%
\pgfpathlineto{\pgfqpoint{3.920294in}{2.538155in}}%
\pgfpathlineto{\pgfqpoint{3.928125in}{2.549202in}}%
\pgfpathlineto{\pgfqpoint{3.914676in}{2.554214in}}%
\pgfpathlineto{\pgfqpoint{3.901233in}{2.559311in}}%
\pgfpathlineto{\pgfqpoint{3.887795in}{2.564493in}}%
\pgfpathlineto{\pgfqpoint{3.874361in}{2.569762in}}%
\pgfpathlineto{\pgfqpoint{3.866522in}{2.558491in}}%
\pgfpathlineto{\pgfqpoint{3.858677in}{2.547336in}}%
\pgfpathlineto{\pgfqpoint{3.850827in}{2.536293in}}%
\pgfpathlineto{\pgfqpoint{3.842972in}{2.525360in}}%
\pgfpathclose%
\pgfusepath{fill}%
\end{pgfscope}%
\begin{pgfscope}%
\pgfpathrectangle{\pgfqpoint{1.150000in}{0.150000in}}{\pgfqpoint{5.700000in}{5.700000in}}%
\pgfusepath{clip}%
\pgfsetbuttcap%
\pgfsetroundjoin%
\definecolor{currentfill}{rgb}{0.263663,0.237631,0.518762}%
\pgfsetfillcolor{currentfill}%
\pgfsetfillopacity{0.700000}%
\pgfsetlinewidth{0.000000pt}%
\definecolor{currentstroke}{rgb}{0.000000,0.000000,0.000000}%
\pgfsetstrokecolor{currentstroke}%
\pgfsetdash{}{0pt}%
\pgfpathmoveto{\pgfqpoint{4.770534in}{2.741881in}}%
\pgfpathlineto{\pgfqpoint{4.784174in}{2.737920in}}%
\pgfpathlineto{\pgfqpoint{4.797820in}{2.734031in}}%
\pgfpathlineto{\pgfqpoint{4.811474in}{2.730213in}}%
\pgfpathlineto{\pgfqpoint{4.825134in}{2.726467in}}%
\pgfpathlineto{\pgfqpoint{4.832726in}{2.739423in}}%
\pgfpathlineto{\pgfqpoint{4.840318in}{2.752624in}}%
\pgfpathlineto{\pgfqpoint{4.847910in}{2.766076in}}%
\pgfpathlineto{\pgfqpoint{4.855501in}{2.779787in}}%
\pgfpathlineto{\pgfqpoint{4.841854in}{2.783930in}}%
\pgfpathlineto{\pgfqpoint{4.828214in}{2.788145in}}%
\pgfpathlineto{\pgfqpoint{4.814580in}{2.792431in}}%
\pgfpathlineto{\pgfqpoint{4.800953in}{2.796790in}}%
\pgfpathlineto{\pgfqpoint{4.793349in}{2.782674in}}%
\pgfpathlineto{\pgfqpoint{4.785744in}{2.768822in}}%
\pgfpathlineto{\pgfqpoint{4.778140in}{2.755227in}}%
\pgfpathlineto{\pgfqpoint{4.770534in}{2.741881in}}%
\pgfpathclose%
\pgfusepath{fill}%
\end{pgfscope}%
\begin{pgfscope}%
\pgfpathrectangle{\pgfqpoint{1.150000in}{0.150000in}}{\pgfqpoint{5.700000in}{5.700000in}}%
\pgfusepath{clip}%
\pgfsetbuttcap%
\pgfsetroundjoin%
\definecolor{currentfill}{rgb}{0.282290,0.145912,0.461510}%
\pgfsetfillcolor{currentfill}%
\pgfsetfillopacity{0.700000}%
\pgfsetlinewidth{0.000000pt}%
\definecolor{currentstroke}{rgb}{0.000000,0.000000,0.000000}%
\pgfsetstrokecolor{currentstroke}%
\pgfsetdash{}{0pt}%
\pgfpathmoveto{\pgfqpoint{4.067099in}{2.555637in}}%
\pgfpathlineto{\pgfqpoint{4.080587in}{2.551139in}}%
\pgfpathlineto{\pgfqpoint{4.094080in}{2.546723in}}%
\pgfpathlineto{\pgfqpoint{4.107579in}{2.542389in}}%
\pgfpathlineto{\pgfqpoint{4.121083in}{2.538136in}}%
\pgfpathlineto{\pgfqpoint{4.128861in}{2.549060in}}%
\pgfpathlineto{\pgfqpoint{4.136634in}{2.560107in}}%
\pgfpathlineto{\pgfqpoint{4.144402in}{2.571283in}}%
\pgfpathlineto{\pgfqpoint{4.152166in}{2.582593in}}%
\pgfpathlineto{\pgfqpoint{4.138671in}{2.587103in}}%
\pgfpathlineto{\pgfqpoint{4.125182in}{2.591694in}}%
\pgfpathlineto{\pgfqpoint{4.111698in}{2.596367in}}%
\pgfpathlineto{\pgfqpoint{4.098220in}{2.601121in}}%
\pgfpathlineto{\pgfqpoint{4.090447in}{2.589548in}}%
\pgfpathlineto{\pgfqpoint{4.082669in}{2.578112in}}%
\pgfpathlineto{\pgfqpoint{4.074886in}{2.566810in}}%
\pgfpathlineto{\pgfqpoint{4.067099in}{2.555637in}}%
\pgfpathclose%
\pgfusepath{fill}%
\end{pgfscope}%
\begin{pgfscope}%
\pgfpathrectangle{\pgfqpoint{1.150000in}{0.150000in}}{\pgfqpoint{5.700000in}{5.700000in}}%
\pgfusepath{clip}%
\pgfsetbuttcap%
\pgfsetroundjoin%
\definecolor{currentfill}{rgb}{0.275191,0.194905,0.496005}%
\pgfsetfillcolor{currentfill}%
\pgfsetfillopacity{0.700000}%
\pgfsetlinewidth{0.000000pt}%
\definecolor{currentstroke}{rgb}{0.000000,0.000000,0.000000}%
\pgfsetstrokecolor{currentstroke}%
\pgfsetdash{}{0pt}%
\pgfpathmoveto{\pgfqpoint{2.730240in}{2.676304in}}%
\pgfpathlineto{\pgfqpoint{2.743611in}{2.665475in}}%
\pgfpathlineto{\pgfqpoint{2.756981in}{2.654778in}}%
\pgfpathlineto{\pgfqpoint{2.770351in}{2.644214in}}%
\pgfpathlineto{\pgfqpoint{2.783719in}{2.633780in}}%
\pgfpathlineto{\pgfqpoint{2.791928in}{2.643605in}}%
\pgfpathlineto{\pgfqpoint{2.800129in}{2.653528in}}%
\pgfpathlineto{\pgfqpoint{2.808322in}{2.663551in}}%
\pgfpathlineto{\pgfqpoint{2.816507in}{2.673674in}}%
\pgfpathlineto{\pgfqpoint{2.803150in}{2.684163in}}%
\pgfpathlineto{\pgfqpoint{2.789793in}{2.694782in}}%
\pgfpathlineto{\pgfqpoint{2.776435in}{2.705533in}}%
\pgfpathlineto{\pgfqpoint{2.763076in}{2.716417in}}%
\pgfpathlineto{\pgfqpoint{2.754879in}{2.706232in}}%
\pgfpathlineto{\pgfqpoint{2.746674in}{2.696152in}}%
\pgfpathlineto{\pgfqpoint{2.738461in}{2.686176in}}%
\pgfpathlineto{\pgfqpoint{2.730240in}{2.676304in}}%
\pgfpathclose%
\pgfusepath{fill}%
\end{pgfscope}%
\begin{pgfscope}%
\pgfpathrectangle{\pgfqpoint{1.150000in}{0.150000in}}{\pgfqpoint{5.700000in}{5.700000in}}%
\pgfusepath{clip}%
\pgfsetbuttcap%
\pgfsetroundjoin%
\definecolor{currentfill}{rgb}{0.280868,0.160771,0.472899}%
\pgfsetfillcolor{currentfill}%
\pgfsetfillopacity{0.700000}%
\pgfsetlinewidth{0.000000pt}%
\definecolor{currentstroke}{rgb}{0.000000,0.000000,0.000000}%
\pgfsetstrokecolor{currentstroke}%
\pgfsetdash{}{0pt}%
\pgfpathmoveto{\pgfqpoint{2.923349in}{2.594343in}}%
\pgfpathlineto{\pgfqpoint{2.936705in}{2.584983in}}%
\pgfpathlineto{\pgfqpoint{2.950062in}{2.575743in}}%
\pgfpathlineto{\pgfqpoint{2.963419in}{2.566622in}}%
\pgfpathlineto{\pgfqpoint{2.976777in}{2.557620in}}%
\pgfpathlineto{\pgfqpoint{2.984920in}{2.567643in}}%
\pgfpathlineto{\pgfqpoint{2.993055in}{2.577756in}}%
\pgfpathlineto{\pgfqpoint{3.001184in}{2.587959in}}%
\pgfpathlineto{\pgfqpoint{3.009305in}{2.598254in}}%
\pgfpathlineto{\pgfqpoint{2.995958in}{2.607332in}}%
\pgfpathlineto{\pgfqpoint{2.982612in}{2.616528in}}%
\pgfpathlineto{\pgfqpoint{2.969267in}{2.625843in}}%
\pgfpathlineto{\pgfqpoint{2.955922in}{2.635279in}}%
\pgfpathlineto{\pgfqpoint{2.947790in}{2.624900in}}%
\pgfpathlineto{\pgfqpoint{2.939650in}{2.614619in}}%
\pgfpathlineto{\pgfqpoint{2.931503in}{2.604434in}}%
\pgfpathlineto{\pgfqpoint{2.923349in}{2.594343in}}%
\pgfpathclose%
\pgfusepath{fill}%
\end{pgfscope}%
\begin{pgfscope}%
\pgfpathrectangle{\pgfqpoint{1.150000in}{0.150000in}}{\pgfqpoint{5.700000in}{5.700000in}}%
\pgfusepath{clip}%
\pgfsetbuttcap%
\pgfsetroundjoin%
\definecolor{currentfill}{rgb}{0.283197,0.115680,0.436115}%
\pgfsetfillcolor{currentfill}%
\pgfsetfillopacity{0.700000}%
\pgfsetlinewidth{0.000000pt}%
\definecolor{currentstroke}{rgb}{0.000000,0.000000,0.000000}%
\pgfsetstrokecolor{currentstroke}%
\pgfsetdash{}{0pt}%
\pgfpathmoveto{\pgfqpoint{3.618777in}{2.504374in}}%
\pgfpathlineto{\pgfqpoint{3.632182in}{2.498626in}}%
\pgfpathlineto{\pgfqpoint{3.645592in}{2.492969in}}%
\pgfpathlineto{\pgfqpoint{3.659006in}{2.487404in}}%
\pgfpathlineto{\pgfqpoint{3.672425in}{2.481930in}}%
\pgfpathlineto{\pgfqpoint{3.680342in}{2.492468in}}%
\pgfpathlineto{\pgfqpoint{3.688253in}{2.503096in}}%
\pgfpathlineto{\pgfqpoint{3.696159in}{2.513817in}}%
\pgfpathlineto{\pgfqpoint{3.704059in}{2.524633in}}%
\pgfpathlineto{\pgfqpoint{3.690650in}{2.530285in}}%
\pgfpathlineto{\pgfqpoint{3.677245in}{2.536026in}}%
\pgfpathlineto{\pgfqpoint{3.663844in}{2.541859in}}%
\pgfpathlineto{\pgfqpoint{3.650447in}{2.547784in}}%
\pgfpathlineto{\pgfqpoint{3.642538in}{2.536784in}}%
\pgfpathlineto{\pgfqpoint{3.634623in}{2.525884in}}%
\pgfpathlineto{\pgfqpoint{3.626703in}{2.515082in}}%
\pgfpathlineto{\pgfqpoint{3.618777in}{2.504374in}}%
\pgfpathclose%
\pgfusepath{fill}%
\end{pgfscope}%
\begin{pgfscope}%
\pgfpathrectangle{\pgfqpoint{1.150000in}{0.150000in}}{\pgfqpoint{5.700000in}{5.700000in}}%
\pgfusepath{clip}%
\pgfsetbuttcap%
\pgfsetroundjoin%
\definecolor{currentfill}{rgb}{0.269308,0.218818,0.509577}%
\pgfsetfillcolor{currentfill}%
\pgfsetfillopacity{0.700000}%
\pgfsetlinewidth{0.000000pt}%
\definecolor{currentstroke}{rgb}{0.000000,0.000000,0.000000}%
\pgfsetstrokecolor{currentstroke}%
\pgfsetdash{}{0pt}%
\pgfpathmoveto{\pgfqpoint{4.685565in}{2.705913in}}%
\pgfpathlineto{\pgfqpoint{4.699190in}{2.702038in}}%
\pgfpathlineto{\pgfqpoint{4.712821in}{2.698237in}}%
\pgfpathlineto{\pgfqpoint{4.726459in}{2.694508in}}%
\pgfpathlineto{\pgfqpoint{4.740105in}{2.690852in}}%
\pgfpathlineto{\pgfqpoint{4.747714in}{2.703270in}}%
\pgfpathlineto{\pgfqpoint{4.755322in}{2.715910in}}%
\pgfpathlineto{\pgfqpoint{4.762928in}{2.728778in}}%
\pgfpathlineto{\pgfqpoint{4.770534in}{2.741881in}}%
\pgfpathlineto{\pgfqpoint{4.756902in}{2.745915in}}%
\pgfpathlineto{\pgfqpoint{4.743276in}{2.750021in}}%
\pgfpathlineto{\pgfqpoint{4.729658in}{2.754200in}}%
\pgfpathlineto{\pgfqpoint{4.716045in}{2.758451in}}%
\pgfpathlineto{\pgfqpoint{4.708427in}{2.744963in}}%
\pgfpathlineto{\pgfqpoint{4.700808in}{2.731715in}}%
\pgfpathlineto{\pgfqpoint{4.693187in}{2.718701in}}%
\pgfpathlineto{\pgfqpoint{4.685565in}{2.705913in}}%
\pgfpathclose%
\pgfusepath{fill}%
\end{pgfscope}%
\begin{pgfscope}%
\pgfpathrectangle{\pgfqpoint{1.150000in}{0.150000in}}{\pgfqpoint{5.700000in}{5.700000in}}%
\pgfusepath{clip}%
\pgfsetbuttcap%
\pgfsetroundjoin%
\definecolor{currentfill}{rgb}{0.179019,0.433756,0.557430}%
\pgfsetfillcolor{currentfill}%
\pgfsetfillopacity{0.700000}%
\pgfsetlinewidth{0.000000pt}%
\definecolor{currentstroke}{rgb}{0.000000,0.000000,0.000000}%
\pgfsetstrokecolor{currentstroke}%
\pgfsetdash{}{0pt}%
\pgfpathmoveto{\pgfqpoint{5.482120in}{3.207196in}}%
\pgfpathlineto{\pgfqpoint{5.495856in}{3.201151in}}%
\pgfpathlineto{\pgfqpoint{5.509598in}{3.195174in}}%
\pgfpathlineto{\pgfqpoint{5.523347in}{3.189263in}}%
\pgfpathlineto{\pgfqpoint{5.537102in}{3.183418in}}%
\pgfpathlineto{\pgfqpoint{5.544767in}{3.205850in}}%
\pgfpathlineto{\pgfqpoint{5.552445in}{3.228808in}}%
\pgfpathlineto{\pgfqpoint{5.560138in}{3.252304in}}%
\pgfpathlineto{\pgfqpoint{5.546393in}{3.258582in}}%
\pgfpathlineto{\pgfqpoint{5.532655in}{3.264927in}}%
\pgfpathlineto{\pgfqpoint{5.518924in}{3.271338in}}%
\pgfpathlineto{\pgfqpoint{5.505199in}{3.277816in}}%
\pgfpathlineto{\pgfqpoint{5.497493in}{3.253736in}}%
\pgfpathlineto{\pgfqpoint{5.489800in}{3.230200in}}%
\pgfpathlineto{\pgfqpoint{5.482120in}{3.207196in}}%
\pgfpathclose%
\pgfusepath{fill}%
\end{pgfscope}%
\begin{pgfscope}%
\pgfpathrectangle{\pgfqpoint{1.150000in}{0.150000in}}{\pgfqpoint{5.700000in}{5.700000in}}%
\pgfusepath{clip}%
\pgfsetbuttcap%
\pgfsetroundjoin%
\definecolor{currentfill}{rgb}{0.203063,0.379716,0.553925}%
\pgfsetfillcolor{currentfill}%
\pgfsetfillopacity{0.700000}%
\pgfsetlinewidth{0.000000pt}%
\definecolor{currentstroke}{rgb}{0.000000,0.000000,0.000000}%
\pgfsetstrokecolor{currentstroke}%
\pgfsetdash{}{0pt}%
\pgfpathmoveto{\pgfqpoint{5.366102in}{3.061493in}}%
\pgfpathlineto{\pgfqpoint{5.379840in}{3.056278in}}%
\pgfpathlineto{\pgfqpoint{5.393585in}{3.051130in}}%
\pgfpathlineto{\pgfqpoint{5.407336in}{3.046050in}}%
\pgfpathlineto{\pgfqpoint{5.421095in}{3.041037in}}%
\pgfpathlineto{\pgfqpoint{5.428688in}{3.060167in}}%
\pgfpathlineto{\pgfqpoint{5.436289in}{3.079745in}}%
\pgfpathlineto{\pgfqpoint{5.443901in}{3.099781in}}%
\pgfpathlineto{\pgfqpoint{5.451522in}{3.120285in}}%
\pgfpathlineto{\pgfqpoint{5.437779in}{3.125837in}}%
\pgfpathlineto{\pgfqpoint{5.424043in}{3.131456in}}%
\pgfpathlineto{\pgfqpoint{5.410313in}{3.137142in}}%
\pgfpathlineto{\pgfqpoint{5.396590in}{3.142896in}}%
\pgfpathlineto{\pgfqpoint{5.388954in}{3.121845in}}%
\pgfpathlineto{\pgfqpoint{5.381327in}{3.101267in}}%
\pgfpathlineto{\pgfqpoint{5.373710in}{3.081153in}}%
\pgfpathlineto{\pgfqpoint{5.366102in}{3.061493in}}%
\pgfpathclose%
\pgfusepath{fill}%
\end{pgfscope}%
\begin{pgfscope}%
\pgfpathrectangle{\pgfqpoint{1.150000in}{0.150000in}}{\pgfqpoint{5.700000in}{5.700000in}}%
\pgfusepath{clip}%
\pgfsetbuttcap%
\pgfsetroundjoin%
\definecolor{currentfill}{rgb}{0.280255,0.165693,0.476498}%
\pgfsetfillcolor{currentfill}%
\pgfsetfillopacity{0.700000}%
\pgfsetlinewidth{0.000000pt}%
\definecolor{currentstroke}{rgb}{0.000000,0.000000,0.000000}%
\pgfsetstrokecolor{currentstroke}%
\pgfsetdash{}{0pt}%
\pgfpathmoveto{\pgfqpoint{4.291270in}{2.593837in}}%
\pgfpathlineto{\pgfqpoint{4.304808in}{2.589764in}}%
\pgfpathlineto{\pgfqpoint{4.318353in}{2.585769in}}%
\pgfpathlineto{\pgfqpoint{4.331904in}{2.581851in}}%
\pgfpathlineto{\pgfqpoint{4.345462in}{2.578011in}}%
\pgfpathlineto{\pgfqpoint{4.353174in}{2.589192in}}%
\pgfpathlineto{\pgfqpoint{4.360883in}{2.600523in}}%
\pgfpathlineto{\pgfqpoint{4.368587in}{2.612008in}}%
\pgfpathlineto{\pgfqpoint{4.376288in}{2.623653in}}%
\pgfpathlineto{\pgfqpoint{4.362742in}{2.627791in}}%
\pgfpathlineto{\pgfqpoint{4.349201in}{2.632006in}}%
\pgfpathlineto{\pgfqpoint{4.335667in}{2.636298in}}%
\pgfpathlineto{\pgfqpoint{4.322139in}{2.640668in}}%
\pgfpathlineto{\pgfqpoint{4.314427in}{2.628719in}}%
\pgfpathlineto{\pgfqpoint{4.306712in}{2.616934in}}%
\pgfpathlineto{\pgfqpoint{4.298993in}{2.605308in}}%
\pgfpathlineto{\pgfqpoint{4.291270in}{2.593837in}}%
\pgfpathclose%
\pgfusepath{fill}%
\end{pgfscope}%
\begin{pgfscope}%
\pgfpathrectangle{\pgfqpoint{1.150000in}{0.150000in}}{\pgfqpoint{5.700000in}{5.700000in}}%
\pgfusepath{clip}%
\pgfsetbuttcap%
\pgfsetroundjoin%
\definecolor{currentfill}{rgb}{0.283229,0.120777,0.440584}%
\pgfsetfillcolor{currentfill}%
\pgfsetfillopacity{0.700000}%
\pgfsetlinewidth{0.000000pt}%
\definecolor{currentstroke}{rgb}{0.000000,0.000000,0.000000}%
\pgfsetstrokecolor{currentstroke}%
\pgfsetdash{}{0pt}%
\pgfpathmoveto{\pgfqpoint{3.255297in}{2.509003in}}%
\pgfpathlineto{\pgfqpoint{3.268662in}{2.501710in}}%
\pgfpathlineto{\pgfqpoint{3.282031in}{2.494521in}}%
\pgfpathlineto{\pgfqpoint{3.295402in}{2.487435in}}%
\pgfpathlineto{\pgfqpoint{3.308776in}{2.480451in}}%
\pgfpathlineto{\pgfqpoint{3.316812in}{2.490698in}}%
\pgfpathlineto{\pgfqpoint{3.324842in}{2.501024in}}%
\pgfpathlineto{\pgfqpoint{3.332866in}{2.511434in}}%
\pgfpathlineto{\pgfqpoint{3.340883in}{2.521928in}}%
\pgfpathlineto{\pgfqpoint{3.327519in}{2.529028in}}%
\pgfpathlineto{\pgfqpoint{3.314158in}{2.536230in}}%
\pgfpathlineto{\pgfqpoint{3.300799in}{2.543536in}}%
\pgfpathlineto{\pgfqpoint{3.287443in}{2.550945in}}%
\pgfpathlineto{\pgfqpoint{3.279416in}{2.540327in}}%
\pgfpathlineto{\pgfqpoint{3.271383in}{2.529799in}}%
\pgfpathlineto{\pgfqpoint{3.263343in}{2.519359in}}%
\pgfpathlineto{\pgfqpoint{3.255297in}{2.509003in}}%
\pgfpathclose%
\pgfusepath{fill}%
\end{pgfscope}%
\begin{pgfscope}%
\pgfpathrectangle{\pgfqpoint{1.150000in}{0.150000in}}{\pgfqpoint{5.700000in}{5.700000in}}%
\pgfusepath{clip}%
\pgfsetbuttcap%
\pgfsetroundjoin%
\definecolor{currentfill}{rgb}{0.273006,0.204520,0.501721}%
\pgfsetfillcolor{currentfill}%
\pgfsetfillopacity{0.700000}%
\pgfsetlinewidth{0.000000pt}%
\definecolor{currentstroke}{rgb}{0.000000,0.000000,0.000000}%
\pgfsetstrokecolor{currentstroke}%
\pgfsetdash{}{0pt}%
\pgfpathmoveto{\pgfqpoint{4.600583in}{2.671693in}}%
\pgfpathlineto{\pgfqpoint{4.614192in}{2.667882in}}%
\pgfpathlineto{\pgfqpoint{4.627808in}{2.664145in}}%
\pgfpathlineto{\pgfqpoint{4.641431in}{2.660482in}}%
\pgfpathlineto{\pgfqpoint{4.655061in}{2.656892in}}%
\pgfpathlineto{\pgfqpoint{4.662690in}{2.668840in}}%
\pgfpathlineto{\pgfqpoint{4.670317in}{2.680989in}}%
\pgfpathlineto{\pgfqpoint{4.677942in}{2.693344in}}%
\pgfpathlineto{\pgfqpoint{4.685565in}{2.705913in}}%
\pgfpathlineto{\pgfqpoint{4.671948in}{2.709860in}}%
\pgfpathlineto{\pgfqpoint{4.658337in}{2.713881in}}%
\pgfpathlineto{\pgfqpoint{4.644733in}{2.717975in}}%
\pgfpathlineto{\pgfqpoint{4.631136in}{2.722143in}}%
\pgfpathlineto{\pgfqpoint{4.623500in}{2.709210in}}%
\pgfpathlineto{\pgfqpoint{4.615863in}{2.696495in}}%
\pgfpathlineto{\pgfqpoint{4.608224in}{2.683992in}}%
\pgfpathlineto{\pgfqpoint{4.600583in}{2.671693in}}%
\pgfpathclose%
\pgfusepath{fill}%
\end{pgfscope}%
\begin{pgfscope}%
\pgfpathrectangle{\pgfqpoint{1.150000in}{0.150000in}}{\pgfqpoint{5.700000in}{5.700000in}}%
\pgfusepath{clip}%
\pgfsetbuttcap%
\pgfsetroundjoin%
\definecolor{currentfill}{rgb}{0.283072,0.130895,0.449241}%
\pgfsetfillcolor{currentfill}%
\pgfsetfillopacity{0.700000}%
\pgfsetlinewidth{0.000000pt}%
\definecolor{currentstroke}{rgb}{0.000000,0.000000,0.000000}%
\pgfsetstrokecolor{currentstroke}%
\pgfsetdash{}{0pt}%
\pgfpathmoveto{\pgfqpoint{3.116121in}{2.529784in}}%
\pgfpathlineto{\pgfqpoint{3.129479in}{2.521731in}}%
\pgfpathlineto{\pgfqpoint{3.142840in}{2.513788in}}%
\pgfpathlineto{\pgfqpoint{3.156202in}{2.505953in}}%
\pgfpathlineto{\pgfqpoint{3.169567in}{2.498227in}}%
\pgfpathlineto{\pgfqpoint{3.177649in}{2.508353in}}%
\pgfpathlineto{\pgfqpoint{3.185725in}{2.518560in}}%
\pgfpathlineto{\pgfqpoint{3.193795in}{2.528850in}}%
\pgfpathlineto{\pgfqpoint{3.201858in}{2.539226in}}%
\pgfpathlineto{\pgfqpoint{3.188503in}{2.547048in}}%
\pgfpathlineto{\pgfqpoint{3.175151in}{2.554979in}}%
\pgfpathlineto{\pgfqpoint{3.161801in}{2.563018in}}%
\pgfpathlineto{\pgfqpoint{3.148453in}{2.571166in}}%
\pgfpathlineto{\pgfqpoint{3.140380in}{2.560687in}}%
\pgfpathlineto{\pgfqpoint{3.132300in}{2.550298in}}%
\pgfpathlineto{\pgfqpoint{3.124214in}{2.539998in}}%
\pgfpathlineto{\pgfqpoint{3.116121in}{2.529784in}}%
\pgfpathclose%
\pgfusepath{fill}%
\end{pgfscope}%
\begin{pgfscope}%
\pgfpathrectangle{\pgfqpoint{1.150000in}{0.150000in}}{\pgfqpoint{5.700000in}{5.700000in}}%
\pgfusepath{clip}%
\pgfsetbuttcap%
\pgfsetroundjoin%
\definecolor{currentfill}{rgb}{0.282884,0.135920,0.453427}%
\pgfsetfillcolor{currentfill}%
\pgfsetfillopacity{0.700000}%
\pgfsetlinewidth{0.000000pt}%
\definecolor{currentstroke}{rgb}{0.000000,0.000000,0.000000}%
\pgfsetstrokecolor{currentstroke}%
\pgfsetdash{}{0pt}%
\pgfpathmoveto{\pgfqpoint{3.981970in}{2.530002in}}%
\pgfpathlineto{\pgfqpoint{3.995445in}{2.525411in}}%
\pgfpathlineto{\pgfqpoint{4.008925in}{2.520904in}}%
\pgfpathlineto{\pgfqpoint{4.022411in}{2.516479in}}%
\pgfpathlineto{\pgfqpoint{4.035903in}{2.512137in}}%
\pgfpathlineto{\pgfqpoint{4.043709in}{2.522841in}}%
\pgfpathlineto{\pgfqpoint{4.051511in}{2.533657in}}%
\pgfpathlineto{\pgfqpoint{4.059307in}{2.544587in}}%
\pgfpathlineto{\pgfqpoint{4.067099in}{2.555637in}}%
\pgfpathlineto{\pgfqpoint{4.053617in}{2.560216in}}%
\pgfpathlineto{\pgfqpoint{4.040141in}{2.564878in}}%
\pgfpathlineto{\pgfqpoint{4.026670in}{2.569622in}}%
\pgfpathlineto{\pgfqpoint{4.013205in}{2.574449in}}%
\pgfpathlineto{\pgfqpoint{4.005404in}{2.563155in}}%
\pgfpathlineto{\pgfqpoint{3.997597in}{2.551986in}}%
\pgfpathlineto{\pgfqpoint{3.989786in}{2.540936in}}%
\pgfpathlineto{\pgfqpoint{3.981970in}{2.530002in}}%
\pgfpathclose%
\pgfusepath{fill}%
\end{pgfscope}%
\begin{pgfscope}%
\pgfpathrectangle{\pgfqpoint{1.150000in}{0.150000in}}{\pgfqpoint{5.700000in}{5.700000in}}%
\pgfusepath{clip}%
\pgfsetbuttcap%
\pgfsetroundjoin%
\definecolor{currentfill}{rgb}{0.283229,0.120777,0.440584}%
\pgfsetfillcolor{currentfill}%
\pgfsetfillopacity{0.700000}%
\pgfsetlinewidth{0.000000pt}%
\definecolor{currentstroke}{rgb}{0.000000,0.000000,0.000000}%
\pgfsetstrokecolor{currentstroke}%
\pgfsetdash{}{0pt}%
\pgfpathmoveto{\pgfqpoint{3.757742in}{2.502927in}}%
\pgfpathlineto{\pgfqpoint{3.771174in}{2.497722in}}%
\pgfpathlineto{\pgfqpoint{3.784611in}{2.492605in}}%
\pgfpathlineto{\pgfqpoint{3.798052in}{2.487576in}}%
\pgfpathlineto{\pgfqpoint{3.811499in}{2.482634in}}%
\pgfpathlineto{\pgfqpoint{3.819376in}{2.493172in}}%
\pgfpathlineto{\pgfqpoint{3.827247in}{2.503803in}}%
\pgfpathlineto{\pgfqpoint{3.835112in}{2.514531in}}%
\pgfpathlineto{\pgfqpoint{3.842972in}{2.525360in}}%
\pgfpathlineto{\pgfqpoint{3.829535in}{2.530499in}}%
\pgfpathlineto{\pgfqpoint{3.816102in}{2.535725in}}%
\pgfpathlineto{\pgfqpoint{3.802675in}{2.541038in}}%
\pgfpathlineto{\pgfqpoint{3.789251in}{2.546439in}}%
\pgfpathlineto{\pgfqpoint{3.781382in}{2.535406in}}%
\pgfpathlineto{\pgfqpoint{3.773507in}{2.524479in}}%
\pgfpathlineto{\pgfqpoint{3.765627in}{2.513654in}}%
\pgfpathlineto{\pgfqpoint{3.757742in}{2.502927in}}%
\pgfpathclose%
\pgfusepath{fill}%
\end{pgfscope}%
\begin{pgfscope}%
\pgfpathrectangle{\pgfqpoint{1.150000in}{0.150000in}}{\pgfqpoint{5.700000in}{5.700000in}}%
\pgfusepath{clip}%
\pgfsetbuttcap%
\pgfsetroundjoin%
\definecolor{currentfill}{rgb}{0.283197,0.115680,0.436115}%
\pgfsetfillcolor{currentfill}%
\pgfsetfillopacity{0.700000}%
\pgfsetlinewidth{0.000000pt}%
\definecolor{currentstroke}{rgb}{0.000000,0.000000,0.000000}%
\pgfsetstrokecolor{currentstroke}%
\pgfsetdash{}{0pt}%
\pgfpathmoveto{\pgfqpoint{3.394369in}{2.494537in}}%
\pgfpathlineto{\pgfqpoint{3.407749in}{2.487939in}}%
\pgfpathlineto{\pgfqpoint{3.421131in}{2.481439in}}%
\pgfpathlineto{\pgfqpoint{3.434517in}{2.475037in}}%
\pgfpathlineto{\pgfqpoint{3.447907in}{2.468732in}}%
\pgfpathlineto{\pgfqpoint{3.455899in}{2.479056in}}%
\pgfpathlineto{\pgfqpoint{3.463885in}{2.489460in}}%
\pgfpathlineto{\pgfqpoint{3.471865in}{2.499947in}}%
\pgfpathlineto{\pgfqpoint{3.479839in}{2.510520in}}%
\pgfpathlineto{\pgfqpoint{3.466459in}{2.516961in}}%
\pgfpathlineto{\pgfqpoint{3.453082in}{2.523499in}}%
\pgfpathlineto{\pgfqpoint{3.439709in}{2.530136in}}%
\pgfpathlineto{\pgfqpoint{3.426339in}{2.536870in}}%
\pgfpathlineto{\pgfqpoint{3.418356in}{2.526154in}}%
\pgfpathlineto{\pgfqpoint{3.410366in}{2.515528in}}%
\pgfpathlineto{\pgfqpoint{3.402371in}{2.504990in}}%
\pgfpathlineto{\pgfqpoint{3.394369in}{2.494537in}}%
\pgfpathclose%
\pgfusepath{fill}%
\end{pgfscope}%
\begin{pgfscope}%
\pgfpathrectangle{\pgfqpoint{1.150000in}{0.150000in}}{\pgfqpoint{5.700000in}{5.700000in}}%
\pgfusepath{clip}%
\pgfsetbuttcap%
\pgfsetroundjoin%
\definecolor{currentfill}{rgb}{0.278012,0.180367,0.486697}%
\pgfsetfillcolor{currentfill}%
\pgfsetfillopacity{0.700000}%
\pgfsetlinewidth{0.000000pt}%
\definecolor{currentstroke}{rgb}{0.000000,0.000000,0.000000}%
\pgfsetstrokecolor{currentstroke}%
\pgfsetdash{}{0pt}%
\pgfpathmoveto{\pgfqpoint{2.783719in}{2.633780in}}%
\pgfpathlineto{\pgfqpoint{2.797087in}{2.623476in}}%
\pgfpathlineto{\pgfqpoint{2.810455in}{2.613301in}}%
\pgfpathlineto{\pgfqpoint{2.823822in}{2.603253in}}%
\pgfpathlineto{\pgfqpoint{2.837190in}{2.593331in}}%
\pgfpathlineto{\pgfqpoint{2.845386in}{2.603109in}}%
\pgfpathlineto{\pgfqpoint{2.853574in}{2.612980in}}%
\pgfpathlineto{\pgfqpoint{2.861755in}{2.622945in}}%
\pgfpathlineto{\pgfqpoint{2.869928in}{2.633006in}}%
\pgfpathlineto{\pgfqpoint{2.856573in}{2.642982in}}%
\pgfpathlineto{\pgfqpoint{2.843218in}{2.653085in}}%
\pgfpathlineto{\pgfqpoint{2.829863in}{2.663315in}}%
\pgfpathlineto{\pgfqpoint{2.816507in}{2.673674in}}%
\pgfpathlineto{\pgfqpoint{2.808322in}{2.663551in}}%
\pgfpathlineto{\pgfqpoint{2.800129in}{2.653528in}}%
\pgfpathlineto{\pgfqpoint{2.791928in}{2.643605in}}%
\pgfpathlineto{\pgfqpoint{2.783719in}{2.633780in}}%
\pgfpathclose%
\pgfusepath{fill}%
\end{pgfscope}%
\begin{pgfscope}%
\pgfpathrectangle{\pgfqpoint{1.150000in}{0.150000in}}{\pgfqpoint{5.700000in}{5.700000in}}%
\pgfusepath{clip}%
\pgfsetbuttcap%
\pgfsetroundjoin%
\definecolor{currentfill}{rgb}{0.190631,0.407061,0.556089}%
\pgfsetfillcolor{currentfill}%
\pgfsetfillopacity{0.700000}%
\pgfsetlinewidth{0.000000pt}%
\definecolor{currentstroke}{rgb}{0.000000,0.000000,0.000000}%
\pgfsetstrokecolor{currentstroke}%
\pgfsetdash{}{0pt}%
\pgfpathmoveto{\pgfqpoint{5.451522in}{3.120285in}}%
\pgfpathlineto{\pgfqpoint{5.465273in}{3.114800in}}%
\pgfpathlineto{\pgfqpoint{5.479030in}{3.109382in}}%
\pgfpathlineto{\pgfqpoint{5.492794in}{3.104031in}}%
\pgfpathlineto{\pgfqpoint{5.506566in}{3.098746in}}%
\pgfpathlineto{\pgfqpoint{5.514182in}{3.119178in}}%
\pgfpathlineto{\pgfqpoint{5.521810in}{3.140093in}}%
\pgfpathlineto{\pgfqpoint{5.529450in}{3.161503in}}%
\pgfpathlineto{\pgfqpoint{5.537102in}{3.183418in}}%
\pgfpathlineto{\pgfqpoint{5.523347in}{3.189263in}}%
\pgfpathlineto{\pgfqpoint{5.509598in}{3.195174in}}%
\pgfpathlineto{\pgfqpoint{5.495856in}{3.201151in}}%
\pgfpathlineto{\pgfqpoint{5.482120in}{3.207196in}}%
\pgfpathlineto{\pgfqpoint{5.474453in}{3.184713in}}%
\pgfpathlineto{\pgfqpoint{5.466798in}{3.162741in}}%
\pgfpathlineto{\pgfqpoint{5.459155in}{3.141268in}}%
\pgfpathlineto{\pgfqpoint{5.451522in}{3.120285in}}%
\pgfpathclose%
\pgfusepath{fill}%
\end{pgfscope}%
\begin{pgfscope}%
\pgfpathrectangle{\pgfqpoint{1.150000in}{0.150000in}}{\pgfqpoint{5.700000in}{5.700000in}}%
\pgfusepath{clip}%
\pgfsetbuttcap%
\pgfsetroundjoin%
\definecolor{currentfill}{rgb}{0.281412,0.155834,0.469201}%
\pgfsetfillcolor{currentfill}%
\pgfsetfillopacity{0.700000}%
\pgfsetlinewidth{0.000000pt}%
\definecolor{currentstroke}{rgb}{0.000000,0.000000,0.000000}%
\pgfsetstrokecolor{currentstroke}%
\pgfsetdash{}{0pt}%
\pgfpathmoveto{\pgfqpoint{4.206204in}{2.565354in}}%
\pgfpathlineto{\pgfqpoint{4.219728in}{2.561244in}}%
\pgfpathlineto{\pgfqpoint{4.233259in}{2.557212in}}%
\pgfpathlineto{\pgfqpoint{4.246795in}{2.553260in}}%
\pgfpathlineto{\pgfqpoint{4.260338in}{2.549386in}}%
\pgfpathlineto{\pgfqpoint{4.268077in}{2.560294in}}%
\pgfpathlineto{\pgfqpoint{4.275812in}{2.571335in}}%
\pgfpathlineto{\pgfqpoint{4.283543in}{2.582514in}}%
\pgfpathlineto{\pgfqpoint{4.291270in}{2.593837in}}%
\pgfpathlineto{\pgfqpoint{4.277737in}{2.597988in}}%
\pgfpathlineto{\pgfqpoint{4.264211in}{2.602218in}}%
\pgfpathlineto{\pgfqpoint{4.250691in}{2.606526in}}%
\pgfpathlineto{\pgfqpoint{4.237177in}{2.610914in}}%
\pgfpathlineto{\pgfqpoint{4.229439in}{2.599307in}}%
\pgfpathlineto{\pgfqpoint{4.221698in}{2.587848in}}%
\pgfpathlineto{\pgfqpoint{4.213953in}{2.576532in}}%
\pgfpathlineto{\pgfqpoint{4.206204in}{2.565354in}}%
\pgfpathclose%
\pgfusepath{fill}%
\end{pgfscope}%
\begin{pgfscope}%
\pgfpathrectangle{\pgfqpoint{1.150000in}{0.150000in}}{\pgfqpoint{5.700000in}{5.700000in}}%
\pgfusepath{clip}%
\pgfsetbuttcap%
\pgfsetroundjoin%
\definecolor{currentfill}{rgb}{0.282290,0.145912,0.461510}%
\pgfsetfillcolor{currentfill}%
\pgfsetfillopacity{0.700000}%
\pgfsetlinewidth{0.000000pt}%
\definecolor{currentstroke}{rgb}{0.000000,0.000000,0.000000}%
\pgfsetstrokecolor{currentstroke}%
\pgfsetdash{}{0pt}%
\pgfpathmoveto{\pgfqpoint{2.976777in}{2.557620in}}%
\pgfpathlineto{\pgfqpoint{2.990135in}{2.548735in}}%
\pgfpathlineto{\pgfqpoint{3.003495in}{2.539967in}}%
\pgfpathlineto{\pgfqpoint{3.016856in}{2.531314in}}%
\pgfpathlineto{\pgfqpoint{3.030218in}{2.522776in}}%
\pgfpathlineto{\pgfqpoint{3.038350in}{2.532731in}}%
\pgfpathlineto{\pgfqpoint{3.046474in}{2.542771in}}%
\pgfpathlineto{\pgfqpoint{3.054592in}{2.552896in}}%
\pgfpathlineto{\pgfqpoint{3.062702in}{2.563108in}}%
\pgfpathlineto{\pgfqpoint{3.049351in}{2.571722in}}%
\pgfpathlineto{\pgfqpoint{3.036002in}{2.580450in}}%
\pgfpathlineto{\pgfqpoint{3.022653in}{2.589294in}}%
\pgfpathlineto{\pgfqpoint{3.009305in}{2.598254in}}%
\pgfpathlineto{\pgfqpoint{3.001184in}{2.587959in}}%
\pgfpathlineto{\pgfqpoint{2.993055in}{2.577756in}}%
\pgfpathlineto{\pgfqpoint{2.984920in}{2.567643in}}%
\pgfpathlineto{\pgfqpoint{2.976777in}{2.557620in}}%
\pgfpathclose%
\pgfusepath{fill}%
\end{pgfscope}%
\begin{pgfscope}%
\pgfpathrectangle{\pgfqpoint{1.150000in}{0.150000in}}{\pgfqpoint{5.700000in}{5.700000in}}%
\pgfusepath{clip}%
\pgfsetbuttcap%
\pgfsetroundjoin%
\definecolor{currentfill}{rgb}{0.276194,0.190074,0.493001}%
\pgfsetfillcolor{currentfill}%
\pgfsetfillopacity{0.700000}%
\pgfsetlinewidth{0.000000pt}%
\definecolor{currentstroke}{rgb}{0.000000,0.000000,0.000000}%
\pgfsetstrokecolor{currentstroke}%
\pgfsetdash{}{0pt}%
\pgfpathmoveto{\pgfqpoint{4.515577in}{2.639059in}}%
\pgfpathlineto{\pgfqpoint{4.529171in}{2.635288in}}%
\pgfpathlineto{\pgfqpoint{4.542772in}{2.631592in}}%
\pgfpathlineto{\pgfqpoint{4.556380in}{2.627971in}}%
\pgfpathlineto{\pgfqpoint{4.569994in}{2.624423in}}%
\pgfpathlineto{\pgfqpoint{4.577645in}{2.635965in}}%
\pgfpathlineto{\pgfqpoint{4.585294in}{2.647686in}}%
\pgfpathlineto{\pgfqpoint{4.592939in}{2.659593in}}%
\pgfpathlineto{\pgfqpoint{4.600583in}{2.671693in}}%
\pgfpathlineto{\pgfqpoint{4.586980in}{2.675578in}}%
\pgfpathlineto{\pgfqpoint{4.573385in}{2.679537in}}%
\pgfpathlineto{\pgfqpoint{4.559795in}{2.683571in}}%
\pgfpathlineto{\pgfqpoint{4.546213in}{2.687679in}}%
\pgfpathlineto{\pgfqpoint{4.538558in}{2.675235in}}%
\pgfpathlineto{\pgfqpoint{4.530900in}{2.662987in}}%
\pgfpathlineto{\pgfqpoint{4.523240in}{2.650931in}}%
\pgfpathlineto{\pgfqpoint{4.515577in}{2.639059in}}%
\pgfpathclose%
\pgfusepath{fill}%
\end{pgfscope}%
\begin{pgfscope}%
\pgfpathrectangle{\pgfqpoint{1.150000in}{0.150000in}}{\pgfqpoint{5.700000in}{5.700000in}}%
\pgfusepath{clip}%
\pgfsetbuttcap%
\pgfsetroundjoin%
\definecolor{currentfill}{rgb}{0.243113,0.292092,0.538516}%
\pgfsetfillcolor{currentfill}%
\pgfsetfillopacity{0.700000}%
\pgfsetlinewidth{0.000000pt}%
\definecolor{currentstroke}{rgb}{0.000000,0.000000,0.000000}%
\pgfsetstrokecolor{currentstroke}%
\pgfsetdash{}{0pt}%
\pgfpathmoveto{\pgfqpoint{5.080257in}{2.845254in}}%
\pgfpathlineto{\pgfqpoint{5.093969in}{2.841170in}}%
\pgfpathlineto{\pgfqpoint{5.107688in}{2.837156in}}%
\pgfpathlineto{\pgfqpoint{5.121414in}{2.833211in}}%
\pgfpathlineto{\pgfqpoint{5.135148in}{2.829335in}}%
\pgfpathlineto{\pgfqpoint{5.142696in}{2.843922in}}%
\pgfpathlineto{\pgfqpoint{5.150246in}{2.858826in}}%
\pgfpathlineto{\pgfqpoint{5.157800in}{2.874055in}}%
\pgfpathlineto{\pgfqpoint{5.165357in}{2.889618in}}%
\pgfpathlineto{\pgfqpoint{5.151638in}{2.893952in}}%
\pgfpathlineto{\pgfqpoint{5.137926in}{2.898355in}}%
\pgfpathlineto{\pgfqpoint{5.124222in}{2.902826in}}%
\pgfpathlineto{\pgfqpoint{5.110525in}{2.907367in}}%
\pgfpathlineto{\pgfqpoint{5.102953in}{2.891339in}}%
\pgfpathlineto{\pgfqpoint{5.095384in}{2.875649in}}%
\pgfpathlineto{\pgfqpoint{5.087819in}{2.860290in}}%
\pgfpathlineto{\pgfqpoint{5.080257in}{2.845254in}}%
\pgfpathclose%
\pgfusepath{fill}%
\end{pgfscope}%
\begin{pgfscope}%
\pgfpathrectangle{\pgfqpoint{1.150000in}{0.150000in}}{\pgfqpoint{5.700000in}{5.700000in}}%
\pgfusepath{clip}%
\pgfsetbuttcap%
\pgfsetroundjoin%
\definecolor{currentfill}{rgb}{0.283091,0.110553,0.431554}%
\pgfsetfillcolor{currentfill}%
\pgfsetfillopacity{0.700000}%
\pgfsetlinewidth{0.000000pt}%
\definecolor{currentstroke}{rgb}{0.000000,0.000000,0.000000}%
\pgfsetstrokecolor{currentstroke}%
\pgfsetdash{}{0pt}%
\pgfpathmoveto{\pgfqpoint{3.533395in}{2.485715in}}%
\pgfpathlineto{\pgfqpoint{3.546794in}{2.479751in}}%
\pgfpathlineto{\pgfqpoint{3.560197in}{2.473881in}}%
\pgfpathlineto{\pgfqpoint{3.573603in}{2.468104in}}%
\pgfpathlineto{\pgfqpoint{3.587014in}{2.462420in}}%
\pgfpathlineto{\pgfqpoint{3.594963in}{2.472783in}}%
\pgfpathlineto{\pgfqpoint{3.602907in}{2.483228in}}%
\pgfpathlineto{\pgfqpoint{3.610845in}{2.493757in}}%
\pgfpathlineto{\pgfqpoint{3.618777in}{2.504374in}}%
\pgfpathlineto{\pgfqpoint{3.605375in}{2.510215in}}%
\pgfpathlineto{\pgfqpoint{3.591978in}{2.516148in}}%
\pgfpathlineto{\pgfqpoint{3.578584in}{2.522175in}}%
\pgfpathlineto{\pgfqpoint{3.565195in}{2.528295in}}%
\pgfpathlineto{\pgfqpoint{3.557254in}{2.517514in}}%
\pgfpathlineto{\pgfqpoint{3.549307in}{2.506826in}}%
\pgfpathlineto{\pgfqpoint{3.541354in}{2.496227in}}%
\pgfpathlineto{\pgfqpoint{3.533395in}{2.485715in}}%
\pgfpathclose%
\pgfusepath{fill}%
\end{pgfscope}%
\begin{pgfscope}%
\pgfpathrectangle{\pgfqpoint{1.150000in}{0.150000in}}{\pgfqpoint{5.700000in}{5.700000in}}%
\pgfusepath{clip}%
\pgfsetbuttcap%
\pgfsetroundjoin%
\definecolor{currentfill}{rgb}{0.235526,0.309527,0.542944}%
\pgfsetfillcolor{currentfill}%
\pgfsetfillopacity{0.700000}%
\pgfsetlinewidth{0.000000pt}%
\definecolor{currentstroke}{rgb}{0.000000,0.000000,0.000000}%
\pgfsetstrokecolor{currentstroke}%
\pgfsetdash{}{0pt}%
\pgfpathmoveto{\pgfqpoint{5.165357in}{2.889618in}}%
\pgfpathlineto{\pgfqpoint{5.179083in}{2.885353in}}%
\pgfpathlineto{\pgfqpoint{5.192816in}{2.881157in}}%
\pgfpathlineto{\pgfqpoint{5.206557in}{2.877029in}}%
\pgfpathlineto{\pgfqpoint{5.220305in}{2.872969in}}%
\pgfpathlineto{\pgfqpoint{5.227851in}{2.888405in}}%
\pgfpathlineto{\pgfqpoint{5.235401in}{2.904187in}}%
\pgfpathlineto{\pgfqpoint{5.242956in}{2.920324in}}%
\pgfpathlineto{\pgfqpoint{5.250516in}{2.936825in}}%
\pgfpathlineto{\pgfqpoint{5.236783in}{2.941362in}}%
\pgfpathlineto{\pgfqpoint{5.223058in}{2.945968in}}%
\pgfpathlineto{\pgfqpoint{5.209340in}{2.950643in}}%
\pgfpathlineto{\pgfqpoint{5.195628in}{2.955385in}}%
\pgfpathlineto{\pgfqpoint{5.188053in}{2.938399in}}%
\pgfpathlineto{\pgfqpoint{5.180483in}{2.921782in}}%
\pgfpathlineto{\pgfqpoint{5.172918in}{2.905525in}}%
\pgfpathlineto{\pgfqpoint{5.165357in}{2.889618in}}%
\pgfpathclose%
\pgfusepath{fill}%
\end{pgfscope}%
\begin{pgfscope}%
\pgfpathrectangle{\pgfqpoint{1.150000in}{0.150000in}}{\pgfqpoint{5.700000in}{5.700000in}}%
\pgfusepath{clip}%
\pgfsetbuttcap%
\pgfsetroundjoin%
\definecolor{currentfill}{rgb}{0.252194,0.269783,0.531579}%
\pgfsetfillcolor{currentfill}%
\pgfsetfillopacity{0.700000}%
\pgfsetlinewidth{0.000000pt}%
\definecolor{currentstroke}{rgb}{0.000000,0.000000,0.000000}%
\pgfsetstrokecolor{currentstroke}%
\pgfsetdash{}{0pt}%
\pgfpathmoveto{\pgfqpoint{4.995197in}{2.803444in}}%
\pgfpathlineto{\pgfqpoint{5.008894in}{2.799520in}}%
\pgfpathlineto{\pgfqpoint{5.022599in}{2.795665in}}%
\pgfpathlineto{\pgfqpoint{5.036311in}{2.791881in}}%
\pgfpathlineto{\pgfqpoint{5.050031in}{2.788166in}}%
\pgfpathlineto{\pgfqpoint{5.057584in}{2.801995in}}%
\pgfpathlineto{\pgfqpoint{5.065139in}{2.816114in}}%
\pgfpathlineto{\pgfqpoint{5.072697in}{2.830531in}}%
\pgfpathlineto{\pgfqpoint{5.080257in}{2.845254in}}%
\pgfpathlineto{\pgfqpoint{5.066552in}{2.849406in}}%
\pgfpathlineto{\pgfqpoint{5.052854in}{2.853628in}}%
\pgfpathlineto{\pgfqpoint{5.039164in}{2.857920in}}%
\pgfpathlineto{\pgfqpoint{5.025481in}{2.862281in}}%
\pgfpathlineto{\pgfqpoint{5.017907in}{2.847114in}}%
\pgfpathlineto{\pgfqpoint{5.010335in}{2.832257in}}%
\pgfpathlineto{\pgfqpoint{5.002765in}{2.817703in}}%
\pgfpathlineto{\pgfqpoint{4.995197in}{2.803444in}}%
\pgfpathclose%
\pgfusepath{fill}%
\end{pgfscope}%
\begin{pgfscope}%
\pgfpathrectangle{\pgfqpoint{1.150000in}{0.150000in}}{\pgfqpoint{5.700000in}{5.700000in}}%
\pgfusepath{clip}%
\pgfsetbuttcap%
\pgfsetroundjoin%
\definecolor{currentfill}{rgb}{0.225863,0.330805,0.547314}%
\pgfsetfillcolor{currentfill}%
\pgfsetfillopacity{0.700000}%
\pgfsetlinewidth{0.000000pt}%
\definecolor{currentstroke}{rgb}{0.000000,0.000000,0.000000}%
\pgfsetstrokecolor{currentstroke}%
\pgfsetdash{}{0pt}%
\pgfpathmoveto{\pgfqpoint{5.250516in}{2.936825in}}%
\pgfpathlineto{\pgfqpoint{5.264255in}{2.932355in}}%
\pgfpathlineto{\pgfqpoint{5.278003in}{2.927954in}}%
\pgfpathlineto{\pgfqpoint{5.291757in}{2.923620in}}%
\pgfpathlineto{\pgfqpoint{5.305519in}{2.919354in}}%
\pgfpathlineto{\pgfqpoint{5.313068in}{2.935738in}}%
\pgfpathlineto{\pgfqpoint{5.320624in}{2.952498in}}%
\pgfpathlineto{\pgfqpoint{5.328186in}{2.969644in}}%
\pgfpathlineto{\pgfqpoint{5.335754in}{2.987185in}}%
\pgfpathlineto{\pgfqpoint{5.322008in}{2.991949in}}%
\pgfpathlineto{\pgfqpoint{5.308269in}{2.996781in}}%
\pgfpathlineto{\pgfqpoint{5.294538in}{3.001680in}}%
\pgfpathlineto{\pgfqpoint{5.280813in}{3.006648in}}%
\pgfpathlineto{\pgfqpoint{5.273229in}{2.988601in}}%
\pgfpathlineto{\pgfqpoint{5.265652in}{2.970954in}}%
\pgfpathlineto{\pgfqpoint{5.258081in}{2.953699in}}%
\pgfpathlineto{\pgfqpoint{5.250516in}{2.936825in}}%
\pgfpathclose%
\pgfusepath{fill}%
\end{pgfscope}%
\begin{pgfscope}%
\pgfpathrectangle{\pgfqpoint{1.150000in}{0.150000in}}{\pgfqpoint{5.700000in}{5.700000in}}%
\pgfusepath{clip}%
\pgfsetbuttcap%
\pgfsetroundjoin%
\definecolor{currentfill}{rgb}{0.258965,0.251537,0.524736}%
\pgfsetfillcolor{currentfill}%
\pgfsetfillopacity{0.700000}%
\pgfsetlinewidth{0.000000pt}%
\definecolor{currentstroke}{rgb}{0.000000,0.000000,0.000000}%
\pgfsetstrokecolor{currentstroke}%
\pgfsetdash{}{0pt}%
\pgfpathmoveto{\pgfqpoint{4.910161in}{2.763927in}}%
\pgfpathlineto{\pgfqpoint{4.923843in}{2.760140in}}%
\pgfpathlineto{\pgfqpoint{4.937533in}{2.756422in}}%
\pgfpathlineto{\pgfqpoint{4.951231in}{2.752776in}}%
\pgfpathlineto{\pgfqpoint{4.964935in}{2.749199in}}%
\pgfpathlineto{\pgfqpoint{4.972499in}{2.762357in}}%
\pgfpathlineto{\pgfqpoint{4.980064in}{2.775778in}}%
\pgfpathlineto{\pgfqpoint{4.987630in}{2.789471in}}%
\pgfpathlineto{\pgfqpoint{4.995197in}{2.803444in}}%
\pgfpathlineto{\pgfqpoint{4.981506in}{2.807438in}}%
\pgfpathlineto{\pgfqpoint{4.967823in}{2.811502in}}%
\pgfpathlineto{\pgfqpoint{4.954147in}{2.815637in}}%
\pgfpathlineto{\pgfqpoint{4.940479in}{2.819842in}}%
\pgfpathlineto{\pgfqpoint{4.932898in}{2.805444in}}%
\pgfpathlineto{\pgfqpoint{4.925318in}{2.791331in}}%
\pgfpathlineto{\pgfqpoint{4.917739in}{2.777495in}}%
\pgfpathlineto{\pgfqpoint{4.910161in}{2.763927in}}%
\pgfpathclose%
\pgfusepath{fill}%
\end{pgfscope}%
\begin{pgfscope}%
\pgfpathrectangle{\pgfqpoint{1.150000in}{0.150000in}}{\pgfqpoint{5.700000in}{5.700000in}}%
\pgfusepath{clip}%
\pgfsetbuttcap%
\pgfsetroundjoin%
\definecolor{currentfill}{rgb}{0.214298,0.355619,0.551184}%
\pgfsetfillcolor{currentfill}%
\pgfsetfillopacity{0.700000}%
\pgfsetlinewidth{0.000000pt}%
\definecolor{currentstroke}{rgb}{0.000000,0.000000,0.000000}%
\pgfsetstrokecolor{currentstroke}%
\pgfsetdash{}{0pt}%
\pgfpathmoveto{\pgfqpoint{5.335754in}{2.987185in}}%
\pgfpathlineto{\pgfqpoint{5.349507in}{2.982489in}}%
\pgfpathlineto{\pgfqpoint{5.363268in}{2.977859in}}%
\pgfpathlineto{\pgfqpoint{5.377035in}{2.973298in}}%
\pgfpathlineto{\pgfqpoint{5.390811in}{2.968803in}}%
\pgfpathlineto{\pgfqpoint{5.398370in}{2.986238in}}%
\pgfpathlineto{\pgfqpoint{5.405937in}{3.004083in}}%
\pgfpathlineto{\pgfqpoint{5.413512in}{3.022346in}}%
\pgfpathlineto{\pgfqpoint{5.421095in}{3.041037in}}%
\pgfpathlineto{\pgfqpoint{5.407336in}{3.046050in}}%
\pgfpathlineto{\pgfqpoint{5.393585in}{3.051130in}}%
\pgfpathlineto{\pgfqpoint{5.379840in}{3.056278in}}%
\pgfpathlineto{\pgfqpoint{5.366102in}{3.061493in}}%
\pgfpathlineto{\pgfqpoint{5.358503in}{3.042275in}}%
\pgfpathlineto{\pgfqpoint{5.350912in}{3.023491in}}%
\pgfpathlineto{\pgfqpoint{5.343330in}{3.005131in}}%
\pgfpathlineto{\pgfqpoint{5.335754in}{2.987185in}}%
\pgfpathclose%
\pgfusepath{fill}%
\end{pgfscope}%
\begin{pgfscope}%
\pgfpathrectangle{\pgfqpoint{1.150000in}{0.150000in}}{\pgfqpoint{5.700000in}{5.700000in}}%
\pgfusepath{clip}%
\pgfsetbuttcap%
\pgfsetroundjoin%
\definecolor{currentfill}{rgb}{0.283187,0.125848,0.444960}%
\pgfsetfillcolor{currentfill}%
\pgfsetfillopacity{0.700000}%
\pgfsetlinewidth{0.000000pt}%
\definecolor{currentstroke}{rgb}{0.000000,0.000000,0.000000}%
\pgfsetstrokecolor{currentstroke}%
\pgfsetdash{}{0pt}%
\pgfpathmoveto{\pgfqpoint{3.896772in}{2.505667in}}%
\pgfpathlineto{\pgfqpoint{3.910235in}{2.500957in}}%
\pgfpathlineto{\pgfqpoint{3.923703in}{2.496332in}}%
\pgfpathlineto{\pgfqpoint{3.937177in}{2.491791in}}%
\pgfpathlineto{\pgfqpoint{3.950656in}{2.487334in}}%
\pgfpathlineto{\pgfqpoint{3.958492in}{2.497849in}}%
\pgfpathlineto{\pgfqpoint{3.966323in}{2.508462in}}%
\pgfpathlineto{\pgfqpoint{3.974149in}{2.519178in}}%
\pgfpathlineto{\pgfqpoint{3.981970in}{2.530002in}}%
\pgfpathlineto{\pgfqpoint{3.968501in}{2.534676in}}%
\pgfpathlineto{\pgfqpoint{3.955037in}{2.539433in}}%
\pgfpathlineto{\pgfqpoint{3.941578in}{2.544275in}}%
\pgfpathlineto{\pgfqpoint{3.928125in}{2.549202in}}%
\pgfpathlineto{\pgfqpoint{3.920294in}{2.538155in}}%
\pgfpathlineto{\pgfqpoint{3.912459in}{2.527219in}}%
\pgfpathlineto{\pgfqpoint{3.904618in}{2.516391in}}%
\pgfpathlineto{\pgfqpoint{3.896772in}{2.505667in}}%
\pgfpathclose%
\pgfusepath{fill}%
\end{pgfscope}%
\begin{pgfscope}%
\pgfpathrectangle{\pgfqpoint{1.150000in}{0.150000in}}{\pgfqpoint{5.700000in}{5.700000in}}%
\pgfusepath{clip}%
\pgfsetbuttcap%
\pgfsetroundjoin%
\definecolor{currentfill}{rgb}{0.278012,0.180367,0.486697}%
\pgfsetfillcolor{currentfill}%
\pgfsetfillopacity{0.700000}%
\pgfsetlinewidth{0.000000pt}%
\definecolor{currentstroke}{rgb}{0.000000,0.000000,0.000000}%
\pgfsetstrokecolor{currentstroke}%
\pgfsetdash{}{0pt}%
\pgfpathmoveto{\pgfqpoint{4.430539in}{2.607871in}}%
\pgfpathlineto{\pgfqpoint{4.444118in}{2.604116in}}%
\pgfpathlineto{\pgfqpoint{4.457704in}{2.600437in}}%
\pgfpathlineto{\pgfqpoint{4.471296in}{2.596833in}}%
\pgfpathlineto{\pgfqpoint{4.484895in}{2.593305in}}%
\pgfpathlineto{\pgfqpoint{4.492571in}{2.604495in}}%
\pgfpathlineto{\pgfqpoint{4.500243in}{2.615848in}}%
\pgfpathlineto{\pgfqpoint{4.507911in}{2.627367in}}%
\pgfpathlineto{\pgfqpoint{4.515577in}{2.639059in}}%
\pgfpathlineto{\pgfqpoint{4.501990in}{2.642905in}}%
\pgfpathlineto{\pgfqpoint{4.488409in}{2.646827in}}%
\pgfpathlineto{\pgfqpoint{4.474835in}{2.650823in}}%
\pgfpathlineto{\pgfqpoint{4.461267in}{2.654896in}}%
\pgfpathlineto{\pgfqpoint{4.453590in}{2.642878in}}%
\pgfpathlineto{\pgfqpoint{4.445909in}{2.631039in}}%
\pgfpathlineto{\pgfqpoint{4.438226in}{2.619372in}}%
\pgfpathlineto{\pgfqpoint{4.430539in}{2.607871in}}%
\pgfpathclose%
\pgfusepath{fill}%
\end{pgfscope}%
\begin{pgfscope}%
\pgfpathrectangle{\pgfqpoint{1.150000in}{0.150000in}}{\pgfqpoint{5.700000in}{5.700000in}}%
\pgfusepath{clip}%
\pgfsetbuttcap%
\pgfsetroundjoin%
\definecolor{currentfill}{rgb}{0.283197,0.115680,0.436115}%
\pgfsetfillcolor{currentfill}%
\pgfsetfillopacity{0.700000}%
\pgfsetlinewidth{0.000000pt}%
\definecolor{currentstroke}{rgb}{0.000000,0.000000,0.000000}%
\pgfsetstrokecolor{currentstroke}%
\pgfsetdash{}{0pt}%
\pgfpathmoveto{\pgfqpoint{3.672425in}{2.481930in}}%
\pgfpathlineto{\pgfqpoint{3.685847in}{2.476546in}}%
\pgfpathlineto{\pgfqpoint{3.699275in}{2.471251in}}%
\pgfpathlineto{\pgfqpoint{3.712707in}{2.466046in}}%
\pgfpathlineto{\pgfqpoint{3.726144in}{2.460930in}}%
\pgfpathlineto{\pgfqpoint{3.734052in}{2.471299in}}%
\pgfpathlineto{\pgfqpoint{3.741954in}{2.481753in}}%
\pgfpathlineto{\pgfqpoint{3.749850in}{2.492294in}}%
\pgfpathlineto{\pgfqpoint{3.757742in}{2.502927in}}%
\pgfpathlineto{\pgfqpoint{3.744314in}{2.508220in}}%
\pgfpathlineto{\pgfqpoint{3.730891in}{2.513601in}}%
\pgfpathlineto{\pgfqpoint{3.717473in}{2.519073in}}%
\pgfpathlineto{\pgfqpoint{3.704059in}{2.524633in}}%
\pgfpathlineto{\pgfqpoint{3.696159in}{2.513817in}}%
\pgfpathlineto{\pgfqpoint{3.688253in}{2.503096in}}%
\pgfpathlineto{\pgfqpoint{3.680342in}{2.492468in}}%
\pgfpathlineto{\pgfqpoint{3.672425in}{2.481930in}}%
\pgfpathclose%
\pgfusepath{fill}%
\end{pgfscope}%
\begin{pgfscope}%
\pgfpathrectangle{\pgfqpoint{1.150000in}{0.150000in}}{\pgfqpoint{5.700000in}{5.700000in}}%
\pgfusepath{clip}%
\pgfsetbuttcap%
\pgfsetroundjoin%
\definecolor{currentfill}{rgb}{0.263663,0.237631,0.518762}%
\pgfsetfillcolor{currentfill}%
\pgfsetfillopacity{0.700000}%
\pgfsetlinewidth{0.000000pt}%
\definecolor{currentstroke}{rgb}{0.000000,0.000000,0.000000}%
\pgfsetstrokecolor{currentstroke}%
\pgfsetdash{}{0pt}%
\pgfpathmoveto{\pgfqpoint{4.825134in}{2.726467in}}%
\pgfpathlineto{\pgfqpoint{4.838802in}{2.722793in}}%
\pgfpathlineto{\pgfqpoint{4.852477in}{2.719190in}}%
\pgfpathlineto{\pgfqpoint{4.866159in}{2.715659in}}%
\pgfpathlineto{\pgfqpoint{4.879848in}{2.712198in}}%
\pgfpathlineto{\pgfqpoint{4.887427in}{2.724764in}}%
\pgfpathlineto{\pgfqpoint{4.895005in}{2.737569in}}%
\pgfpathlineto{\pgfqpoint{4.902583in}{2.750621in}}%
\pgfpathlineto{\pgfqpoint{4.910161in}{2.763927in}}%
\pgfpathlineto{\pgfqpoint{4.896485in}{2.767786in}}%
\pgfpathlineto{\pgfqpoint{4.882817in}{2.771715in}}%
\pgfpathlineto{\pgfqpoint{4.869155in}{2.775715in}}%
\pgfpathlineto{\pgfqpoint{4.855501in}{2.779787in}}%
\pgfpathlineto{\pgfqpoint{4.847910in}{2.766076in}}%
\pgfpathlineto{\pgfqpoint{4.840318in}{2.752624in}}%
\pgfpathlineto{\pgfqpoint{4.832726in}{2.739423in}}%
\pgfpathlineto{\pgfqpoint{4.825134in}{2.726467in}}%
\pgfpathclose%
\pgfusepath{fill}%
\end{pgfscope}%
\begin{pgfscope}%
\pgfpathrectangle{\pgfqpoint{1.150000in}{0.150000in}}{\pgfqpoint{5.700000in}{5.700000in}}%
\pgfusepath{clip}%
\pgfsetbuttcap%
\pgfsetroundjoin%
\definecolor{currentfill}{rgb}{0.280255,0.165693,0.476498}%
\pgfsetfillcolor{currentfill}%
\pgfsetfillopacity{0.700000}%
\pgfsetlinewidth{0.000000pt}%
\definecolor{currentstroke}{rgb}{0.000000,0.000000,0.000000}%
\pgfsetstrokecolor{currentstroke}%
\pgfsetdash{}{0pt}%
\pgfpathmoveto{\pgfqpoint{2.837190in}{2.593331in}}%
\pgfpathlineto{\pgfqpoint{2.850557in}{2.583535in}}%
\pgfpathlineto{\pgfqpoint{2.863924in}{2.573863in}}%
\pgfpathlineto{\pgfqpoint{2.877291in}{2.564315in}}%
\pgfpathlineto{\pgfqpoint{2.890658in}{2.554889in}}%
\pgfpathlineto{\pgfqpoint{2.898842in}{2.564619in}}%
\pgfpathlineto{\pgfqpoint{2.907019in}{2.574437in}}%
\pgfpathlineto{\pgfqpoint{2.915188in}{2.584344in}}%
\pgfpathlineto{\pgfqpoint{2.923349in}{2.594343in}}%
\pgfpathlineto{\pgfqpoint{2.909994in}{2.603824in}}%
\pgfpathlineto{\pgfqpoint{2.896638in}{2.613428in}}%
\pgfpathlineto{\pgfqpoint{2.883283in}{2.623155in}}%
\pgfpathlineto{\pgfqpoint{2.869928in}{2.633006in}}%
\pgfpathlineto{\pgfqpoint{2.861755in}{2.622945in}}%
\pgfpathlineto{\pgfqpoint{2.853574in}{2.612980in}}%
\pgfpathlineto{\pgfqpoint{2.845386in}{2.603109in}}%
\pgfpathlineto{\pgfqpoint{2.837190in}{2.593331in}}%
\pgfpathclose%
\pgfusepath{fill}%
\end{pgfscope}%
\begin{pgfscope}%
\pgfpathrectangle{\pgfqpoint{1.150000in}{0.150000in}}{\pgfqpoint{5.700000in}{5.700000in}}%
\pgfusepath{clip}%
\pgfsetbuttcap%
\pgfsetroundjoin%
\definecolor{currentfill}{rgb}{0.180629,0.429975,0.557282}%
\pgfsetfillcolor{currentfill}%
\pgfsetfillopacity{0.700000}%
\pgfsetlinewidth{0.000000pt}%
\definecolor{currentstroke}{rgb}{0.000000,0.000000,0.000000}%
\pgfsetstrokecolor{currentstroke}%
\pgfsetdash{}{0pt}%
\pgfpathmoveto{\pgfqpoint{5.537102in}{3.183418in}}%
\pgfpathlineto{\pgfqpoint{5.550865in}{3.177641in}}%
\pgfpathlineto{\pgfqpoint{5.564635in}{3.171929in}}%
\pgfpathlineto{\pgfqpoint{5.578411in}{3.166284in}}%
\pgfpathlineto{\pgfqpoint{5.592195in}{3.160706in}}%
\pgfpathlineto{\pgfqpoint{5.599844in}{3.182564in}}%
\pgfpathlineto{\pgfqpoint{5.607507in}{3.204944in}}%
\pgfpathlineto{\pgfqpoint{5.615183in}{3.227857in}}%
\pgfpathlineto{\pgfqpoint{5.601412in}{3.233869in}}%
\pgfpathlineto{\pgfqpoint{5.587647in}{3.239948in}}%
\pgfpathlineto{\pgfqpoint{5.573889in}{3.246093in}}%
\pgfpathlineto{\pgfqpoint{5.560138in}{3.252304in}}%
\pgfpathlineto{\pgfqpoint{5.552445in}{3.228808in}}%
\pgfpathlineto{\pgfqpoint{5.544767in}{3.205850in}}%
\pgfpathlineto{\pgfqpoint{5.537102in}{3.183418in}}%
\pgfpathclose%
\pgfusepath{fill}%
\end{pgfscope}%
\begin{pgfscope}%
\pgfpathrectangle{\pgfqpoint{1.150000in}{0.150000in}}{\pgfqpoint{5.700000in}{5.700000in}}%
\pgfusepath{clip}%
\pgfsetbuttcap%
\pgfsetroundjoin%
\definecolor{currentfill}{rgb}{0.282290,0.145912,0.461510}%
\pgfsetfillcolor{currentfill}%
\pgfsetfillopacity{0.700000}%
\pgfsetlinewidth{0.000000pt}%
\definecolor{currentstroke}{rgb}{0.000000,0.000000,0.000000}%
\pgfsetstrokecolor{currentstroke}%
\pgfsetdash{}{0pt}%
\pgfpathmoveto{\pgfqpoint{4.121083in}{2.538136in}}%
\pgfpathlineto{\pgfqpoint{4.134594in}{2.533963in}}%
\pgfpathlineto{\pgfqpoint{4.148111in}{2.529871in}}%
\pgfpathlineto{\pgfqpoint{4.161633in}{2.525859in}}%
\pgfpathlineto{\pgfqpoint{4.175162in}{2.521926in}}%
\pgfpathlineto{\pgfqpoint{4.182929in}{2.532600in}}%
\pgfpathlineto{\pgfqpoint{4.190692in}{2.543393in}}%
\pgfpathlineto{\pgfqpoint{4.198450in}{2.554309in}}%
\pgfpathlineto{\pgfqpoint{4.206204in}{2.565354in}}%
\pgfpathlineto{\pgfqpoint{4.192685in}{2.569544in}}%
\pgfpathlineto{\pgfqpoint{4.179173in}{2.573813in}}%
\pgfpathlineto{\pgfqpoint{4.165666in}{2.578163in}}%
\pgfpathlineto{\pgfqpoint{4.152166in}{2.582593in}}%
\pgfpathlineto{\pgfqpoint{4.144402in}{2.571283in}}%
\pgfpathlineto{\pgfqpoint{4.136634in}{2.560107in}}%
\pgfpathlineto{\pgfqpoint{4.128861in}{2.549060in}}%
\pgfpathlineto{\pgfqpoint{4.121083in}{2.538136in}}%
\pgfpathclose%
\pgfusepath{fill}%
\end{pgfscope}%
\begin{pgfscope}%
\pgfpathrectangle{\pgfqpoint{1.150000in}{0.150000in}}{\pgfqpoint{5.700000in}{5.700000in}}%
\pgfusepath{clip}%
\pgfsetbuttcap%
\pgfsetroundjoin%
\definecolor{currentfill}{rgb}{0.204903,0.375746,0.553533}%
\pgfsetfillcolor{currentfill}%
\pgfsetfillopacity{0.700000}%
\pgfsetlinewidth{0.000000pt}%
\definecolor{currentstroke}{rgb}{0.000000,0.000000,0.000000}%
\pgfsetstrokecolor{currentstroke}%
\pgfsetdash{}{0pt}%
\pgfpathmoveto{\pgfqpoint{5.421095in}{3.041037in}}%
\pgfpathlineto{\pgfqpoint{5.434861in}{3.036091in}}%
\pgfpathlineto{\pgfqpoint{5.448635in}{3.031212in}}%
\pgfpathlineto{\pgfqpoint{5.462416in}{3.026400in}}%
\pgfpathlineto{\pgfqpoint{5.476204in}{3.021654in}}%
\pgfpathlineto{\pgfqpoint{5.483780in}{3.040253in}}%
\pgfpathlineto{\pgfqpoint{5.491365in}{3.059294in}}%
\pgfpathlineto{\pgfqpoint{5.498960in}{3.078789in}}%
\pgfpathlineto{\pgfqpoint{5.506566in}{3.098746in}}%
\pgfpathlineto{\pgfqpoint{5.492794in}{3.104031in}}%
\pgfpathlineto{\pgfqpoint{5.479030in}{3.109382in}}%
\pgfpathlineto{\pgfqpoint{5.465273in}{3.114800in}}%
\pgfpathlineto{\pgfqpoint{5.451522in}{3.120285in}}%
\pgfpathlineto{\pgfqpoint{5.443901in}{3.099781in}}%
\pgfpathlineto{\pgfqpoint{5.436289in}{3.079745in}}%
\pgfpathlineto{\pgfqpoint{5.428688in}{3.060167in}}%
\pgfpathlineto{\pgfqpoint{5.421095in}{3.041037in}}%
\pgfpathclose%
\pgfusepath{fill}%
\end{pgfscope}%
\begin{pgfscope}%
\pgfpathrectangle{\pgfqpoint{1.150000in}{0.150000in}}{\pgfqpoint{5.700000in}{5.700000in}}%
\pgfusepath{clip}%
\pgfsetbuttcap%
\pgfsetroundjoin%
\definecolor{currentfill}{rgb}{0.283229,0.120777,0.440584}%
\pgfsetfillcolor{currentfill}%
\pgfsetfillopacity{0.700000}%
\pgfsetlinewidth{0.000000pt}%
\definecolor{currentstroke}{rgb}{0.000000,0.000000,0.000000}%
\pgfsetstrokecolor{currentstroke}%
\pgfsetdash{}{0pt}%
\pgfpathmoveto{\pgfqpoint{3.169567in}{2.498227in}}%
\pgfpathlineto{\pgfqpoint{3.182933in}{2.490608in}}%
\pgfpathlineto{\pgfqpoint{3.196302in}{2.483096in}}%
\pgfpathlineto{\pgfqpoint{3.209673in}{2.475690in}}%
\pgfpathlineto{\pgfqpoint{3.223047in}{2.468388in}}%
\pgfpathlineto{\pgfqpoint{3.231119in}{2.478425in}}%
\pgfpathlineto{\pgfqpoint{3.239185in}{2.488539in}}%
\pgfpathlineto{\pgfqpoint{3.247244in}{2.498731in}}%
\pgfpathlineto{\pgfqpoint{3.255297in}{2.509003in}}%
\pgfpathlineto{\pgfqpoint{3.241933in}{2.516401in}}%
\pgfpathlineto{\pgfqpoint{3.228572in}{2.523903in}}%
\pgfpathlineto{\pgfqpoint{3.215214in}{2.531511in}}%
\pgfpathlineto{\pgfqpoint{3.201858in}{2.539226in}}%
\pgfpathlineto{\pgfqpoint{3.193795in}{2.528850in}}%
\pgfpathlineto{\pgfqpoint{3.185725in}{2.518560in}}%
\pgfpathlineto{\pgfqpoint{3.177649in}{2.508353in}}%
\pgfpathlineto{\pgfqpoint{3.169567in}{2.498227in}}%
\pgfpathclose%
\pgfusepath{fill}%
\end{pgfscope}%
\begin{pgfscope}%
\pgfpathrectangle{\pgfqpoint{1.150000in}{0.150000in}}{\pgfqpoint{5.700000in}{5.700000in}}%
\pgfusepath{clip}%
\pgfsetbuttcap%
\pgfsetroundjoin%
\definecolor{currentfill}{rgb}{0.283091,0.110553,0.431554}%
\pgfsetfillcolor{currentfill}%
\pgfsetfillopacity{0.700000}%
\pgfsetlinewidth{0.000000pt}%
\definecolor{currentstroke}{rgb}{0.000000,0.000000,0.000000}%
\pgfsetstrokecolor{currentstroke}%
\pgfsetdash{}{0pt}%
\pgfpathmoveto{\pgfqpoint{3.308776in}{2.480451in}}%
\pgfpathlineto{\pgfqpoint{3.322152in}{2.473569in}}%
\pgfpathlineto{\pgfqpoint{3.335532in}{2.466788in}}%
\pgfpathlineto{\pgfqpoint{3.348915in}{2.460107in}}%
\pgfpathlineto{\pgfqpoint{3.362301in}{2.453526in}}%
\pgfpathlineto{\pgfqpoint{3.370327in}{2.463663in}}%
\pgfpathlineto{\pgfqpoint{3.378347in}{2.473876in}}%
\pgfpathlineto{\pgfqpoint{3.386361in}{2.484167in}}%
\pgfpathlineto{\pgfqpoint{3.394369in}{2.494537in}}%
\pgfpathlineto{\pgfqpoint{3.380993in}{2.501235in}}%
\pgfpathlineto{\pgfqpoint{3.367620in}{2.508032in}}%
\pgfpathlineto{\pgfqpoint{3.354250in}{2.514929in}}%
\pgfpathlineto{\pgfqpoint{3.340883in}{2.521928in}}%
\pgfpathlineto{\pgfqpoint{3.332866in}{2.511434in}}%
\pgfpathlineto{\pgfqpoint{3.324842in}{2.501024in}}%
\pgfpathlineto{\pgfqpoint{3.316812in}{2.490698in}}%
\pgfpathlineto{\pgfqpoint{3.308776in}{2.480451in}}%
\pgfpathclose%
\pgfusepath{fill}%
\end{pgfscope}%
\begin{pgfscope}%
\pgfpathrectangle{\pgfqpoint{1.150000in}{0.150000in}}{\pgfqpoint{5.700000in}{5.700000in}}%
\pgfusepath{clip}%
\pgfsetbuttcap%
\pgfsetroundjoin%
\definecolor{currentfill}{rgb}{0.271828,0.209303,0.504434}%
\pgfsetfillcolor{currentfill}%
\pgfsetfillopacity{0.700000}%
\pgfsetlinewidth{0.000000pt}%
\definecolor{currentstroke}{rgb}{0.000000,0.000000,0.000000}%
\pgfsetstrokecolor{currentstroke}%
\pgfsetdash{}{0pt}%
\pgfpathmoveto{\pgfqpoint{2.643723in}{2.682362in}}%
\pgfpathlineto{\pgfqpoint{2.657113in}{2.671022in}}%
\pgfpathlineto{\pgfqpoint{2.670501in}{2.659820in}}%
\pgfpathlineto{\pgfqpoint{2.683888in}{2.648754in}}%
\pgfpathlineto{\pgfqpoint{2.697274in}{2.637825in}}%
\pgfpathlineto{\pgfqpoint{2.705527in}{2.647296in}}%
\pgfpathlineto{\pgfqpoint{2.713773in}{2.656865in}}%
\pgfpathlineto{\pgfqpoint{2.722010in}{2.666534in}}%
\pgfpathlineto{\pgfqpoint{2.730240in}{2.676304in}}%
\pgfpathlineto{\pgfqpoint{2.716867in}{2.687269in}}%
\pgfpathlineto{\pgfqpoint{2.703494in}{2.698368in}}%
\pgfpathlineto{\pgfqpoint{2.690119in}{2.709605in}}%
\pgfpathlineto{\pgfqpoint{2.676742in}{2.720979in}}%
\pgfpathlineto{\pgfqpoint{2.668500in}{2.711167in}}%
\pgfpathlineto{\pgfqpoint{2.660249in}{2.701461in}}%
\pgfpathlineto{\pgfqpoint{2.651990in}{2.691860in}}%
\pgfpathlineto{\pgfqpoint{2.643723in}{2.682362in}}%
\pgfpathclose%
\pgfusepath{fill}%
\end{pgfscope}%
\begin{pgfscope}%
\pgfpathrectangle{\pgfqpoint{1.150000in}{0.150000in}}{\pgfqpoint{5.700000in}{5.700000in}}%
\pgfusepath{clip}%
\pgfsetbuttcap%
\pgfsetroundjoin%
\definecolor{currentfill}{rgb}{0.267968,0.223549,0.512008}%
\pgfsetfillcolor{currentfill}%
\pgfsetfillopacity{0.700000}%
\pgfsetlinewidth{0.000000pt}%
\definecolor{currentstroke}{rgb}{0.000000,0.000000,0.000000}%
\pgfsetstrokecolor{currentstroke}%
\pgfsetdash{}{0pt}%
\pgfpathmoveto{\pgfqpoint{4.740105in}{2.690852in}}%
\pgfpathlineto{\pgfqpoint{4.753757in}{2.687268in}}%
\pgfpathlineto{\pgfqpoint{4.767417in}{2.683756in}}%
\pgfpathlineto{\pgfqpoint{4.781083in}{2.680316in}}%
\pgfpathlineto{\pgfqpoint{4.794757in}{2.676949in}}%
\pgfpathlineto{\pgfqpoint{4.802353in}{2.688996in}}%
\pgfpathlineto{\pgfqpoint{4.809948in}{2.701261in}}%
\pgfpathlineto{\pgfqpoint{4.817541in}{2.713749in}}%
\pgfpathlineto{\pgfqpoint{4.825134in}{2.726467in}}%
\pgfpathlineto{\pgfqpoint{4.811474in}{2.730213in}}%
\pgfpathlineto{\pgfqpoint{4.797820in}{2.734031in}}%
\pgfpathlineto{\pgfqpoint{4.784174in}{2.737920in}}%
\pgfpathlineto{\pgfqpoint{4.770534in}{2.741881in}}%
\pgfpathlineto{\pgfqpoint{4.762928in}{2.728778in}}%
\pgfpathlineto{\pgfqpoint{4.755322in}{2.715910in}}%
\pgfpathlineto{\pgfqpoint{4.747714in}{2.703270in}}%
\pgfpathlineto{\pgfqpoint{4.740105in}{2.690852in}}%
\pgfpathclose%
\pgfusepath{fill}%
\end{pgfscope}%
\begin{pgfscope}%
\pgfpathrectangle{\pgfqpoint{1.150000in}{0.150000in}}{\pgfqpoint{5.700000in}{5.700000in}}%
\pgfusepath{clip}%
\pgfsetbuttcap%
\pgfsetroundjoin%
\definecolor{currentfill}{rgb}{0.283072,0.130895,0.449241}%
\pgfsetfillcolor{currentfill}%
\pgfsetfillopacity{0.700000}%
\pgfsetlinewidth{0.000000pt}%
\definecolor{currentstroke}{rgb}{0.000000,0.000000,0.000000}%
\pgfsetstrokecolor{currentstroke}%
\pgfsetdash{}{0pt}%
\pgfpathmoveto{\pgfqpoint{3.030218in}{2.522776in}}%
\pgfpathlineto{\pgfqpoint{3.043581in}{2.514352in}}%
\pgfpathlineto{\pgfqpoint{3.056946in}{2.506041in}}%
\pgfpathlineto{\pgfqpoint{3.070312in}{2.497842in}}%
\pgfpathlineto{\pgfqpoint{3.083680in}{2.489754in}}%
\pgfpathlineto{\pgfqpoint{3.091801in}{2.499641in}}%
\pgfpathlineto{\pgfqpoint{3.099914in}{2.509607in}}%
\pgfpathlineto{\pgfqpoint{3.108021in}{2.519654in}}%
\pgfpathlineto{\pgfqpoint{3.116121in}{2.529784in}}%
\pgfpathlineto{\pgfqpoint{3.102764in}{2.537947in}}%
\pgfpathlineto{\pgfqpoint{3.089409in}{2.546222in}}%
\pgfpathlineto{\pgfqpoint{3.076055in}{2.554609in}}%
\pgfpathlineto{\pgfqpoint{3.062702in}{2.563108in}}%
\pgfpathlineto{\pgfqpoint{3.054592in}{2.552896in}}%
\pgfpathlineto{\pgfqpoint{3.046474in}{2.542771in}}%
\pgfpathlineto{\pgfqpoint{3.038350in}{2.532731in}}%
\pgfpathlineto{\pgfqpoint{3.030218in}{2.522776in}}%
\pgfpathclose%
\pgfusepath{fill}%
\end{pgfscope}%
\begin{pgfscope}%
\pgfpathrectangle{\pgfqpoint{1.150000in}{0.150000in}}{\pgfqpoint{5.700000in}{5.700000in}}%
\pgfusepath{clip}%
\pgfsetbuttcap%
\pgfsetroundjoin%
\definecolor{currentfill}{rgb}{0.283091,0.110553,0.431554}%
\pgfsetfillcolor{currentfill}%
\pgfsetfillopacity{0.700000}%
\pgfsetlinewidth{0.000000pt}%
\definecolor{currentstroke}{rgb}{0.000000,0.000000,0.000000}%
\pgfsetstrokecolor{currentstroke}%
\pgfsetdash{}{0pt}%
\pgfpathmoveto{\pgfqpoint{3.447907in}{2.468732in}}%
\pgfpathlineto{\pgfqpoint{3.461300in}{2.462523in}}%
\pgfpathlineto{\pgfqpoint{3.474697in}{2.456411in}}%
\pgfpathlineto{\pgfqpoint{3.488097in}{2.450395in}}%
\pgfpathlineto{\pgfqpoint{3.501501in}{2.444473in}}%
\pgfpathlineto{\pgfqpoint{3.509484in}{2.454668in}}%
\pgfpathlineto{\pgfqpoint{3.517460in}{2.464938in}}%
\pgfpathlineto{\pgfqpoint{3.525431in}{2.475286in}}%
\pgfpathlineto{\pgfqpoint{3.533395in}{2.485715in}}%
\pgfpathlineto{\pgfqpoint{3.520001in}{2.491773in}}%
\pgfpathlineto{\pgfqpoint{3.506610in}{2.497926in}}%
\pgfpathlineto{\pgfqpoint{3.493222in}{2.504175in}}%
\pgfpathlineto{\pgfqpoint{3.479839in}{2.510520in}}%
\pgfpathlineto{\pgfqpoint{3.471865in}{2.499947in}}%
\pgfpathlineto{\pgfqpoint{3.463885in}{2.489460in}}%
\pgfpathlineto{\pgfqpoint{3.455899in}{2.479056in}}%
\pgfpathlineto{\pgfqpoint{3.447907in}{2.468732in}}%
\pgfpathclose%
\pgfusepath{fill}%
\end{pgfscope}%
\begin{pgfscope}%
\pgfpathrectangle{\pgfqpoint{1.150000in}{0.150000in}}{\pgfqpoint{5.700000in}{5.700000in}}%
\pgfusepath{clip}%
\pgfsetbuttcap%
\pgfsetroundjoin%
\definecolor{currentfill}{rgb}{0.280255,0.165693,0.476498}%
\pgfsetfillcolor{currentfill}%
\pgfsetfillopacity{0.700000}%
\pgfsetlinewidth{0.000000pt}%
\definecolor{currentstroke}{rgb}{0.000000,0.000000,0.000000}%
\pgfsetstrokecolor{currentstroke}%
\pgfsetdash{}{0pt}%
\pgfpathmoveto{\pgfqpoint{4.345462in}{2.578011in}}%
\pgfpathlineto{\pgfqpoint{4.359026in}{2.574248in}}%
\pgfpathlineto{\pgfqpoint{4.372597in}{2.570561in}}%
\pgfpathlineto{\pgfqpoint{4.386174in}{2.566952in}}%
\pgfpathlineto{\pgfqpoint{4.399757in}{2.563418in}}%
\pgfpathlineto{\pgfqpoint{4.407458in}{2.574310in}}%
\pgfpathlineto{\pgfqpoint{4.415156in}{2.585345in}}%
\pgfpathlineto{\pgfqpoint{4.422849in}{2.596530in}}%
\pgfpathlineto{\pgfqpoint{4.430539in}{2.607871in}}%
\pgfpathlineto{\pgfqpoint{4.416967in}{2.611702in}}%
\pgfpathlineto{\pgfqpoint{4.403401in}{2.615609in}}%
\pgfpathlineto{\pgfqpoint{4.389841in}{2.619593in}}%
\pgfpathlineto{\pgfqpoint{4.376288in}{2.623653in}}%
\pgfpathlineto{\pgfqpoint{4.368587in}{2.612008in}}%
\pgfpathlineto{\pgfqpoint{4.360883in}{2.600523in}}%
\pgfpathlineto{\pgfqpoint{4.353174in}{2.589192in}}%
\pgfpathlineto{\pgfqpoint{4.345462in}{2.578011in}}%
\pgfpathclose%
\pgfusepath{fill}%
\end{pgfscope}%
\begin{pgfscope}%
\pgfpathrectangle{\pgfqpoint{1.150000in}{0.150000in}}{\pgfqpoint{5.700000in}{5.700000in}}%
\pgfusepath{clip}%
\pgfsetbuttcap%
\pgfsetroundjoin%
\definecolor{currentfill}{rgb}{0.283197,0.115680,0.436115}%
\pgfsetfillcolor{currentfill}%
\pgfsetfillopacity{0.700000}%
\pgfsetlinewidth{0.000000pt}%
\definecolor{currentstroke}{rgb}{0.000000,0.000000,0.000000}%
\pgfsetstrokecolor{currentstroke}%
\pgfsetdash{}{0pt}%
\pgfpathmoveto{\pgfqpoint{3.811499in}{2.482634in}}%
\pgfpathlineto{\pgfqpoint{3.824951in}{2.477779in}}%
\pgfpathlineto{\pgfqpoint{3.838408in}{2.473010in}}%
\pgfpathlineto{\pgfqpoint{3.851870in}{2.468327in}}%
\pgfpathlineto{\pgfqpoint{3.865337in}{2.463729in}}%
\pgfpathlineto{\pgfqpoint{3.873204in}{2.474077in}}%
\pgfpathlineto{\pgfqpoint{3.881065in}{2.484514in}}%
\pgfpathlineto{\pgfqpoint{3.888921in}{2.495042in}}%
\pgfpathlineto{\pgfqpoint{3.896772in}{2.505667in}}%
\pgfpathlineto{\pgfqpoint{3.883315in}{2.510461in}}%
\pgfpathlineto{\pgfqpoint{3.869862in}{2.515342in}}%
\pgfpathlineto{\pgfqpoint{3.856415in}{2.520308in}}%
\pgfpathlineto{\pgfqpoint{3.842972in}{2.525360in}}%
\pgfpathlineto{\pgfqpoint{3.835112in}{2.514531in}}%
\pgfpathlineto{\pgfqpoint{3.827247in}{2.503803in}}%
\pgfpathlineto{\pgfqpoint{3.819376in}{2.493172in}}%
\pgfpathlineto{\pgfqpoint{3.811499in}{2.482634in}}%
\pgfpathclose%
\pgfusepath{fill}%
\end{pgfscope}%
\begin{pgfscope}%
\pgfpathrectangle{\pgfqpoint{1.150000in}{0.150000in}}{\pgfqpoint{5.700000in}{5.700000in}}%
\pgfusepath{clip}%
\pgfsetbuttcap%
\pgfsetroundjoin%
\definecolor{currentfill}{rgb}{0.192357,0.403199,0.555836}%
\pgfsetfillcolor{currentfill}%
\pgfsetfillopacity{0.700000}%
\pgfsetlinewidth{0.000000pt}%
\definecolor{currentstroke}{rgb}{0.000000,0.000000,0.000000}%
\pgfsetstrokecolor{currentstroke}%
\pgfsetdash{}{0pt}%
\pgfpathmoveto{\pgfqpoint{5.506566in}{3.098746in}}%
\pgfpathlineto{\pgfqpoint{5.520345in}{3.093528in}}%
\pgfpathlineto{\pgfqpoint{5.534130in}{3.088377in}}%
\pgfpathlineto{\pgfqpoint{5.547924in}{3.083292in}}%
\pgfpathlineto{\pgfqpoint{5.561724in}{3.078273in}}%
\pgfpathlineto{\pgfqpoint{5.569324in}{3.098152in}}%
\pgfpathlineto{\pgfqpoint{5.576935in}{3.118510in}}%
\pgfpathlineto{\pgfqpoint{5.584559in}{3.139358in}}%
\pgfpathlineto{\pgfqpoint{5.592195in}{3.160706in}}%
\pgfpathlineto{\pgfqpoint{5.578411in}{3.166284in}}%
\pgfpathlineto{\pgfqpoint{5.564635in}{3.171929in}}%
\pgfpathlineto{\pgfqpoint{5.550865in}{3.177641in}}%
\pgfpathlineto{\pgfqpoint{5.537102in}{3.183418in}}%
\pgfpathlineto{\pgfqpoint{5.529450in}{3.161503in}}%
\pgfpathlineto{\pgfqpoint{5.521810in}{3.140093in}}%
\pgfpathlineto{\pgfqpoint{5.514182in}{3.119178in}}%
\pgfpathlineto{\pgfqpoint{5.506566in}{3.098746in}}%
\pgfpathclose%
\pgfusepath{fill}%
\end{pgfscope}%
\begin{pgfscope}%
\pgfpathrectangle{\pgfqpoint{1.150000in}{0.150000in}}{\pgfqpoint{5.700000in}{5.700000in}}%
\pgfusepath{clip}%
\pgfsetbuttcap%
\pgfsetroundjoin%
\definecolor{currentfill}{rgb}{0.273006,0.204520,0.501721}%
\pgfsetfillcolor{currentfill}%
\pgfsetfillopacity{0.700000}%
\pgfsetlinewidth{0.000000pt}%
\definecolor{currentstroke}{rgb}{0.000000,0.000000,0.000000}%
\pgfsetstrokecolor{currentstroke}%
\pgfsetdash{}{0pt}%
\pgfpathmoveto{\pgfqpoint{4.655061in}{2.656892in}}%
\pgfpathlineto{\pgfqpoint{4.668698in}{2.653375in}}%
\pgfpathlineto{\pgfqpoint{4.682342in}{2.649931in}}%
\pgfpathlineto{\pgfqpoint{4.695993in}{2.646560in}}%
\pgfpathlineto{\pgfqpoint{4.709652in}{2.643262in}}%
\pgfpathlineto{\pgfqpoint{4.717268in}{2.654860in}}%
\pgfpathlineto{\pgfqpoint{4.724882in}{2.666653in}}%
\pgfpathlineto{\pgfqpoint{4.732494in}{2.678649in}}%
\pgfpathlineto{\pgfqpoint{4.740105in}{2.690852in}}%
\pgfpathlineto{\pgfqpoint{4.726459in}{2.694508in}}%
\pgfpathlineto{\pgfqpoint{4.712821in}{2.698237in}}%
\pgfpathlineto{\pgfqpoint{4.699190in}{2.702038in}}%
\pgfpathlineto{\pgfqpoint{4.685565in}{2.705913in}}%
\pgfpathlineto{\pgfqpoint{4.677942in}{2.693344in}}%
\pgfpathlineto{\pgfqpoint{4.670317in}{2.680989in}}%
\pgfpathlineto{\pgfqpoint{4.662690in}{2.668840in}}%
\pgfpathlineto{\pgfqpoint{4.655061in}{2.656892in}}%
\pgfpathclose%
\pgfusepath{fill}%
\end{pgfscope}%
\begin{pgfscope}%
\pgfpathrectangle{\pgfqpoint{1.150000in}{0.150000in}}{\pgfqpoint{5.700000in}{5.700000in}}%
\pgfusepath{clip}%
\pgfsetbuttcap%
\pgfsetroundjoin%
\definecolor{currentfill}{rgb}{0.282884,0.135920,0.453427}%
\pgfsetfillcolor{currentfill}%
\pgfsetfillopacity{0.700000}%
\pgfsetlinewidth{0.000000pt}%
\definecolor{currentstroke}{rgb}{0.000000,0.000000,0.000000}%
\pgfsetstrokecolor{currentstroke}%
\pgfsetdash{}{0pt}%
\pgfpathmoveto{\pgfqpoint{4.035903in}{2.512137in}}%
\pgfpathlineto{\pgfqpoint{4.049400in}{2.507877in}}%
\pgfpathlineto{\pgfqpoint{4.062903in}{2.503698in}}%
\pgfpathlineto{\pgfqpoint{4.076412in}{2.499601in}}%
\pgfpathlineto{\pgfqpoint{4.089927in}{2.495585in}}%
\pgfpathlineto{\pgfqpoint{4.097723in}{2.506060in}}%
\pgfpathlineto{\pgfqpoint{4.105515in}{2.516640in}}%
\pgfpathlineto{\pgfqpoint{4.113301in}{2.527331in}}%
\pgfpathlineto{\pgfqpoint{4.121083in}{2.538136in}}%
\pgfpathlineto{\pgfqpoint{4.107579in}{2.542389in}}%
\pgfpathlineto{\pgfqpoint{4.094080in}{2.546723in}}%
\pgfpathlineto{\pgfqpoint{4.080587in}{2.551139in}}%
\pgfpathlineto{\pgfqpoint{4.067099in}{2.555637in}}%
\pgfpathlineto{\pgfqpoint{4.059307in}{2.544587in}}%
\pgfpathlineto{\pgfqpoint{4.051511in}{2.533657in}}%
\pgfpathlineto{\pgfqpoint{4.043709in}{2.522841in}}%
\pgfpathlineto{\pgfqpoint{4.035903in}{2.512137in}}%
\pgfpathclose%
\pgfusepath{fill}%
\end{pgfscope}%
\begin{pgfscope}%
\pgfpathrectangle{\pgfqpoint{1.150000in}{0.150000in}}{\pgfqpoint{5.700000in}{5.700000in}}%
\pgfusepath{clip}%
\pgfsetbuttcap%
\pgfsetroundjoin%
\definecolor{currentfill}{rgb}{0.282910,0.105393,0.426902}%
\pgfsetfillcolor{currentfill}%
\pgfsetfillopacity{0.700000}%
\pgfsetlinewidth{0.000000pt}%
\definecolor{currentstroke}{rgb}{0.000000,0.000000,0.000000}%
\pgfsetstrokecolor{currentstroke}%
\pgfsetdash{}{0pt}%
\pgfpathmoveto{\pgfqpoint{3.587014in}{2.462420in}}%
\pgfpathlineto{\pgfqpoint{3.600429in}{2.456829in}}%
\pgfpathlineto{\pgfqpoint{3.613848in}{2.451329in}}%
\pgfpathlineto{\pgfqpoint{3.627271in}{2.445921in}}%
\pgfpathlineto{\pgfqpoint{3.640699in}{2.440603in}}%
\pgfpathlineto{\pgfqpoint{3.648639in}{2.450817in}}%
\pgfpathlineto{\pgfqpoint{3.656573in}{2.461107in}}%
\pgfpathlineto{\pgfqpoint{3.664502in}{2.471477in}}%
\pgfpathlineto{\pgfqpoint{3.672425in}{2.481930in}}%
\pgfpathlineto{\pgfqpoint{3.659006in}{2.487404in}}%
\pgfpathlineto{\pgfqpoint{3.645592in}{2.492969in}}%
\pgfpathlineto{\pgfqpoint{3.632182in}{2.498626in}}%
\pgfpathlineto{\pgfqpoint{3.618777in}{2.504374in}}%
\pgfpathlineto{\pgfqpoint{3.610845in}{2.493757in}}%
\pgfpathlineto{\pgfqpoint{3.602907in}{2.483228in}}%
\pgfpathlineto{\pgfqpoint{3.594963in}{2.472783in}}%
\pgfpathlineto{\pgfqpoint{3.587014in}{2.462420in}}%
\pgfpathclose%
\pgfusepath{fill}%
\end{pgfscope}%
\begin{pgfscope}%
\pgfpathrectangle{\pgfqpoint{1.150000in}{0.150000in}}{\pgfqpoint{5.700000in}{5.700000in}}%
\pgfusepath{clip}%
\pgfsetbuttcap%
\pgfsetroundjoin%
\definecolor{currentfill}{rgb}{0.281887,0.150881,0.465405}%
\pgfsetfillcolor{currentfill}%
\pgfsetfillopacity{0.700000}%
\pgfsetlinewidth{0.000000pt}%
\definecolor{currentstroke}{rgb}{0.000000,0.000000,0.000000}%
\pgfsetstrokecolor{currentstroke}%
\pgfsetdash{}{0pt}%
\pgfpathmoveto{\pgfqpoint{2.890658in}{2.554889in}}%
\pgfpathlineto{\pgfqpoint{2.904026in}{2.545584in}}%
\pgfpathlineto{\pgfqpoint{2.917394in}{2.536399in}}%
\pgfpathlineto{\pgfqpoint{2.930763in}{2.527334in}}%
\pgfpathlineto{\pgfqpoint{2.944133in}{2.518387in}}%
\pgfpathlineto{\pgfqpoint{2.952305in}{2.528069in}}%
\pgfpathlineto{\pgfqpoint{2.960469in}{2.537834in}}%
\pgfpathlineto{\pgfqpoint{2.968627in}{2.547684in}}%
\pgfpathlineto{\pgfqpoint{2.976777in}{2.557620in}}%
\pgfpathlineto{\pgfqpoint{2.963419in}{2.566622in}}%
\pgfpathlineto{\pgfqpoint{2.950062in}{2.575743in}}%
\pgfpathlineto{\pgfqpoint{2.936705in}{2.584983in}}%
\pgfpathlineto{\pgfqpoint{2.923349in}{2.594343in}}%
\pgfpathlineto{\pgfqpoint{2.915188in}{2.584344in}}%
\pgfpathlineto{\pgfqpoint{2.907019in}{2.574437in}}%
\pgfpathlineto{\pgfqpoint{2.898842in}{2.564619in}}%
\pgfpathlineto{\pgfqpoint{2.890658in}{2.554889in}}%
\pgfpathclose%
\pgfusepath{fill}%
\end{pgfscope}%
\begin{pgfscope}%
\pgfpathrectangle{\pgfqpoint{1.150000in}{0.150000in}}{\pgfqpoint{5.700000in}{5.700000in}}%
\pgfusepath{clip}%
\pgfsetbuttcap%
\pgfsetroundjoin%
\definecolor{currentfill}{rgb}{0.276194,0.190074,0.493001}%
\pgfsetfillcolor{currentfill}%
\pgfsetfillopacity{0.700000}%
\pgfsetlinewidth{0.000000pt}%
\definecolor{currentstroke}{rgb}{0.000000,0.000000,0.000000}%
\pgfsetstrokecolor{currentstroke}%
\pgfsetdash{}{0pt}%
\pgfpathmoveto{\pgfqpoint{2.697274in}{2.637825in}}%
\pgfpathlineto{\pgfqpoint{2.710658in}{2.627029in}}%
\pgfpathlineto{\pgfqpoint{2.724042in}{2.616367in}}%
\pgfpathlineto{\pgfqpoint{2.737424in}{2.605837in}}%
\pgfpathlineto{\pgfqpoint{2.750806in}{2.595438in}}%
\pgfpathlineto{\pgfqpoint{2.759046in}{2.604882in}}%
\pgfpathlineto{\pgfqpoint{2.767278in}{2.614420in}}%
\pgfpathlineto{\pgfqpoint{2.775503in}{2.624052in}}%
\pgfpathlineto{\pgfqpoint{2.783719in}{2.633780in}}%
\pgfpathlineto{\pgfqpoint{2.770351in}{2.644214in}}%
\pgfpathlineto{\pgfqpoint{2.756981in}{2.654778in}}%
\pgfpathlineto{\pgfqpoint{2.743611in}{2.665475in}}%
\pgfpathlineto{\pgfqpoint{2.730240in}{2.676304in}}%
\pgfpathlineto{\pgfqpoint{2.722010in}{2.666534in}}%
\pgfpathlineto{\pgfqpoint{2.713773in}{2.656865in}}%
\pgfpathlineto{\pgfqpoint{2.705527in}{2.647296in}}%
\pgfpathlineto{\pgfqpoint{2.697274in}{2.637825in}}%
\pgfpathclose%
\pgfusepath{fill}%
\end{pgfscope}%
\begin{pgfscope}%
\pgfpathrectangle{\pgfqpoint{1.150000in}{0.150000in}}{\pgfqpoint{5.700000in}{5.700000in}}%
\pgfusepath{clip}%
\pgfsetbuttcap%
\pgfsetroundjoin%
\definecolor{currentfill}{rgb}{0.281412,0.155834,0.469201}%
\pgfsetfillcolor{currentfill}%
\pgfsetfillopacity{0.700000}%
\pgfsetlinewidth{0.000000pt}%
\definecolor{currentstroke}{rgb}{0.000000,0.000000,0.000000}%
\pgfsetstrokecolor{currentstroke}%
\pgfsetdash{}{0pt}%
\pgfpathmoveto{\pgfqpoint{4.260338in}{2.549386in}}%
\pgfpathlineto{\pgfqpoint{4.273887in}{2.545590in}}%
\pgfpathlineto{\pgfqpoint{4.287443in}{2.541873in}}%
\pgfpathlineto{\pgfqpoint{4.301005in}{2.538233in}}%
\pgfpathlineto{\pgfqpoint{4.314574in}{2.534670in}}%
\pgfpathlineto{\pgfqpoint{4.322302in}{2.545308in}}%
\pgfpathlineto{\pgfqpoint{4.330026in}{2.556074in}}%
\pgfpathlineto{\pgfqpoint{4.337746in}{2.566973in}}%
\pgfpathlineto{\pgfqpoint{4.345462in}{2.578011in}}%
\pgfpathlineto{\pgfqpoint{4.331904in}{2.581851in}}%
\pgfpathlineto{\pgfqpoint{4.318353in}{2.585769in}}%
\pgfpathlineto{\pgfqpoint{4.304808in}{2.589764in}}%
\pgfpathlineto{\pgfqpoint{4.291270in}{2.593837in}}%
\pgfpathlineto{\pgfqpoint{4.283543in}{2.582514in}}%
\pgfpathlineto{\pgfqpoint{4.275812in}{2.571335in}}%
\pgfpathlineto{\pgfqpoint{4.268077in}{2.560294in}}%
\pgfpathlineto{\pgfqpoint{4.260338in}{2.549386in}}%
\pgfpathclose%
\pgfusepath{fill}%
\end{pgfscope}%
\begin{pgfscope}%
\pgfpathrectangle{\pgfqpoint{1.150000in}{0.150000in}}{\pgfqpoint{5.700000in}{5.700000in}}%
\pgfusepath{clip}%
\pgfsetbuttcap%
\pgfsetroundjoin%
\definecolor{currentfill}{rgb}{0.235526,0.309527,0.542944}%
\pgfsetfillcolor{currentfill}%
\pgfsetfillopacity{0.700000}%
\pgfsetlinewidth{0.000000pt}%
\definecolor{currentstroke}{rgb}{0.000000,0.000000,0.000000}%
\pgfsetstrokecolor{currentstroke}%
\pgfsetdash{}{0pt}%
\pgfpathmoveto{\pgfqpoint{5.220305in}{2.872969in}}%
\pgfpathlineto{\pgfqpoint{5.234061in}{2.868978in}}%
\pgfpathlineto{\pgfqpoint{5.247824in}{2.865054in}}%
\pgfpathlineto{\pgfqpoint{5.261595in}{2.861199in}}%
\pgfpathlineto{\pgfqpoint{5.275373in}{2.857411in}}%
\pgfpathlineto{\pgfqpoint{5.282903in}{2.872376in}}%
\pgfpathlineto{\pgfqpoint{5.290436in}{2.887683in}}%
\pgfpathlineto{\pgfqpoint{5.297975in}{2.903339in}}%
\pgfpathlineto{\pgfqpoint{5.305519in}{2.919354in}}%
\pgfpathlineto{\pgfqpoint{5.291757in}{2.923620in}}%
\pgfpathlineto{\pgfqpoint{5.278003in}{2.927954in}}%
\pgfpathlineto{\pgfqpoint{5.264255in}{2.932355in}}%
\pgfpathlineto{\pgfqpoint{5.250516in}{2.936825in}}%
\pgfpathlineto{\pgfqpoint{5.242956in}{2.920324in}}%
\pgfpathlineto{\pgfqpoint{5.235401in}{2.904187in}}%
\pgfpathlineto{\pgfqpoint{5.227851in}{2.888405in}}%
\pgfpathlineto{\pgfqpoint{5.220305in}{2.872969in}}%
\pgfpathclose%
\pgfusepath{fill}%
\end{pgfscope}%
\begin{pgfscope}%
\pgfpathrectangle{\pgfqpoint{1.150000in}{0.150000in}}{\pgfqpoint{5.700000in}{5.700000in}}%
\pgfusepath{clip}%
\pgfsetbuttcap%
\pgfsetroundjoin%
\definecolor{currentfill}{rgb}{0.243113,0.292092,0.538516}%
\pgfsetfillcolor{currentfill}%
\pgfsetfillopacity{0.700000}%
\pgfsetlinewidth{0.000000pt}%
\definecolor{currentstroke}{rgb}{0.000000,0.000000,0.000000}%
\pgfsetstrokecolor{currentstroke}%
\pgfsetdash{}{0pt}%
\pgfpathmoveto{\pgfqpoint{5.135148in}{2.829335in}}%
\pgfpathlineto{\pgfqpoint{5.148890in}{2.825528in}}%
\pgfpathlineto{\pgfqpoint{5.162639in}{2.821790in}}%
\pgfpathlineto{\pgfqpoint{5.176396in}{2.818120in}}%
\pgfpathlineto{\pgfqpoint{5.190160in}{2.814518in}}%
\pgfpathlineto{\pgfqpoint{5.197691in}{2.828654in}}%
\pgfpathlineto{\pgfqpoint{5.205226in}{2.843102in}}%
\pgfpathlineto{\pgfqpoint{5.212764in}{2.857871in}}%
\pgfpathlineto{\pgfqpoint{5.220305in}{2.872969in}}%
\pgfpathlineto{\pgfqpoint{5.206557in}{2.877029in}}%
\pgfpathlineto{\pgfqpoint{5.192816in}{2.881157in}}%
\pgfpathlineto{\pgfqpoint{5.179083in}{2.885353in}}%
\pgfpathlineto{\pgfqpoint{5.165357in}{2.889618in}}%
\pgfpathlineto{\pgfqpoint{5.157800in}{2.874055in}}%
\pgfpathlineto{\pgfqpoint{5.150246in}{2.858826in}}%
\pgfpathlineto{\pgfqpoint{5.142696in}{2.843922in}}%
\pgfpathlineto{\pgfqpoint{5.135148in}{2.829335in}}%
\pgfpathclose%
\pgfusepath{fill}%
\end{pgfscope}%
\begin{pgfscope}%
\pgfpathrectangle{\pgfqpoint{1.150000in}{0.150000in}}{\pgfqpoint{5.700000in}{5.700000in}}%
\pgfusepath{clip}%
\pgfsetbuttcap%
\pgfsetroundjoin%
\definecolor{currentfill}{rgb}{0.275191,0.194905,0.496005}%
\pgfsetfillcolor{currentfill}%
\pgfsetfillopacity{0.700000}%
\pgfsetlinewidth{0.000000pt}%
\definecolor{currentstroke}{rgb}{0.000000,0.000000,0.000000}%
\pgfsetstrokecolor{currentstroke}%
\pgfsetdash{}{0pt}%
\pgfpathmoveto{\pgfqpoint{4.569994in}{2.624423in}}%
\pgfpathlineto{\pgfqpoint{4.583616in}{2.620950in}}%
\pgfpathlineto{\pgfqpoint{4.597244in}{2.617551in}}%
\pgfpathlineto{\pgfqpoint{4.610880in}{2.614225in}}%
\pgfpathlineto{\pgfqpoint{4.624522in}{2.610973in}}%
\pgfpathlineto{\pgfqpoint{4.632161in}{2.622184in}}%
\pgfpathlineto{\pgfqpoint{4.639797in}{2.633570in}}%
\pgfpathlineto{\pgfqpoint{4.647430in}{2.645137in}}%
\pgfpathlineto{\pgfqpoint{4.655061in}{2.656892in}}%
\pgfpathlineto{\pgfqpoint{4.641431in}{2.660482in}}%
\pgfpathlineto{\pgfqpoint{4.627808in}{2.664145in}}%
\pgfpathlineto{\pgfqpoint{4.614192in}{2.667882in}}%
\pgfpathlineto{\pgfqpoint{4.600583in}{2.671693in}}%
\pgfpathlineto{\pgfqpoint{4.592939in}{2.659593in}}%
\pgfpathlineto{\pgfqpoint{4.585294in}{2.647686in}}%
\pgfpathlineto{\pgfqpoint{4.577645in}{2.635965in}}%
\pgfpathlineto{\pgfqpoint{4.569994in}{2.624423in}}%
\pgfpathclose%
\pgfusepath{fill}%
\end{pgfscope}%
\begin{pgfscope}%
\pgfpathrectangle{\pgfqpoint{1.150000in}{0.150000in}}{\pgfqpoint{5.700000in}{5.700000in}}%
\pgfusepath{clip}%
\pgfsetbuttcap%
\pgfsetroundjoin%
\definecolor{currentfill}{rgb}{0.225863,0.330805,0.547314}%
\pgfsetfillcolor{currentfill}%
\pgfsetfillopacity{0.700000}%
\pgfsetlinewidth{0.000000pt}%
\definecolor{currentstroke}{rgb}{0.000000,0.000000,0.000000}%
\pgfsetstrokecolor{currentstroke}%
\pgfsetdash{}{0pt}%
\pgfpathmoveto{\pgfqpoint{5.305519in}{2.919354in}}%
\pgfpathlineto{\pgfqpoint{5.319288in}{2.915156in}}%
\pgfpathlineto{\pgfqpoint{5.333065in}{2.911025in}}%
\pgfpathlineto{\pgfqpoint{5.346850in}{2.906962in}}%
\pgfpathlineto{\pgfqpoint{5.360642in}{2.902966in}}%
\pgfpathlineto{\pgfqpoint{5.368175in}{2.918858in}}%
\pgfpathlineto{\pgfqpoint{5.375713in}{2.935122in}}%
\pgfpathlineto{\pgfqpoint{5.383259in}{2.951768in}}%
\pgfpathlineto{\pgfqpoint{5.390811in}{2.968803in}}%
\pgfpathlineto{\pgfqpoint{5.377035in}{2.973298in}}%
\pgfpathlineto{\pgfqpoint{5.363268in}{2.977859in}}%
\pgfpathlineto{\pgfqpoint{5.349507in}{2.982489in}}%
\pgfpathlineto{\pgfqpoint{5.335754in}{2.987185in}}%
\pgfpathlineto{\pgfqpoint{5.328186in}{2.969644in}}%
\pgfpathlineto{\pgfqpoint{5.320624in}{2.952498in}}%
\pgfpathlineto{\pgfqpoint{5.313068in}{2.935738in}}%
\pgfpathlineto{\pgfqpoint{5.305519in}{2.919354in}}%
\pgfpathclose%
\pgfusepath{fill}%
\end{pgfscope}%
\begin{pgfscope}%
\pgfpathrectangle{\pgfqpoint{1.150000in}{0.150000in}}{\pgfqpoint{5.700000in}{5.700000in}}%
\pgfusepath{clip}%
\pgfsetbuttcap%
\pgfsetroundjoin%
\definecolor{currentfill}{rgb}{0.252194,0.269783,0.531579}%
\pgfsetfillcolor{currentfill}%
\pgfsetfillopacity{0.700000}%
\pgfsetlinewidth{0.000000pt}%
\definecolor{currentstroke}{rgb}{0.000000,0.000000,0.000000}%
\pgfsetstrokecolor{currentstroke}%
\pgfsetdash{}{0pt}%
\pgfpathmoveto{\pgfqpoint{5.050031in}{2.788166in}}%
\pgfpathlineto{\pgfqpoint{5.063758in}{2.784521in}}%
\pgfpathlineto{\pgfqpoint{5.077492in}{2.780945in}}%
\pgfpathlineto{\pgfqpoint{5.091234in}{2.777438in}}%
\pgfpathlineto{\pgfqpoint{5.104984in}{2.774000in}}%
\pgfpathlineto{\pgfqpoint{5.112522in}{2.787399in}}%
\pgfpathlineto{\pgfqpoint{5.120062in}{2.801082in}}%
\pgfpathlineto{\pgfqpoint{5.127604in}{2.815058in}}%
\pgfpathlineto{\pgfqpoint{5.135148in}{2.829335in}}%
\pgfpathlineto{\pgfqpoint{5.121414in}{2.833211in}}%
\pgfpathlineto{\pgfqpoint{5.107688in}{2.837156in}}%
\pgfpathlineto{\pgfqpoint{5.093969in}{2.841170in}}%
\pgfpathlineto{\pgfqpoint{5.080257in}{2.845254in}}%
\pgfpathlineto{\pgfqpoint{5.072697in}{2.830531in}}%
\pgfpathlineto{\pgfqpoint{5.065139in}{2.816114in}}%
\pgfpathlineto{\pgfqpoint{5.057584in}{2.801995in}}%
\pgfpathlineto{\pgfqpoint{5.050031in}{2.788166in}}%
\pgfpathclose%
\pgfusepath{fill}%
\end{pgfscope}%
\begin{pgfscope}%
\pgfpathrectangle{\pgfqpoint{1.150000in}{0.150000in}}{\pgfqpoint{5.700000in}{5.700000in}}%
\pgfusepath{clip}%
\pgfsetbuttcap%
\pgfsetroundjoin%
\definecolor{currentfill}{rgb}{0.216210,0.351535,0.550627}%
\pgfsetfillcolor{currentfill}%
\pgfsetfillopacity{0.700000}%
\pgfsetlinewidth{0.000000pt}%
\definecolor{currentstroke}{rgb}{0.000000,0.000000,0.000000}%
\pgfsetstrokecolor{currentstroke}%
\pgfsetdash{}{0pt}%
\pgfpathmoveto{\pgfqpoint{5.390811in}{2.968803in}}%
\pgfpathlineto{\pgfqpoint{5.404593in}{2.964376in}}%
\pgfpathlineto{\pgfqpoint{5.418384in}{2.960015in}}%
\pgfpathlineto{\pgfqpoint{5.432181in}{2.955722in}}%
\pgfpathlineto{\pgfqpoint{5.445987in}{2.951495in}}%
\pgfpathlineto{\pgfqpoint{5.453529in}{2.968419in}}%
\pgfpathlineto{\pgfqpoint{5.461079in}{2.985747in}}%
\pgfpathlineto{\pgfqpoint{5.468637in}{3.003489in}}%
\pgfpathlineto{\pgfqpoint{5.476204in}{3.021654in}}%
\pgfpathlineto{\pgfqpoint{5.462416in}{3.026400in}}%
\pgfpathlineto{\pgfqpoint{5.448635in}{3.031212in}}%
\pgfpathlineto{\pgfqpoint{5.434861in}{3.036091in}}%
\pgfpathlineto{\pgfqpoint{5.421095in}{3.041037in}}%
\pgfpathlineto{\pgfqpoint{5.413512in}{3.022346in}}%
\pgfpathlineto{\pgfqpoint{5.405937in}{3.004083in}}%
\pgfpathlineto{\pgfqpoint{5.398370in}{2.986238in}}%
\pgfpathlineto{\pgfqpoint{5.390811in}{2.968803in}}%
\pgfpathclose%
\pgfusepath{fill}%
\end{pgfscope}%
\begin{pgfscope}%
\pgfpathrectangle{\pgfqpoint{1.150000in}{0.150000in}}{\pgfqpoint{5.700000in}{5.700000in}}%
\pgfusepath{clip}%
\pgfsetbuttcap%
\pgfsetroundjoin%
\definecolor{currentfill}{rgb}{0.283091,0.110553,0.431554}%
\pgfsetfillcolor{currentfill}%
\pgfsetfillopacity{0.700000}%
\pgfsetlinewidth{0.000000pt}%
\definecolor{currentstroke}{rgb}{0.000000,0.000000,0.000000}%
\pgfsetstrokecolor{currentstroke}%
\pgfsetdash{}{0pt}%
\pgfpathmoveto{\pgfqpoint{3.223047in}{2.468388in}}%
\pgfpathlineto{\pgfqpoint{3.236423in}{2.461192in}}%
\pgfpathlineto{\pgfqpoint{3.249801in}{2.454099in}}%
\pgfpathlineto{\pgfqpoint{3.263183in}{2.447109in}}%
\pgfpathlineto{\pgfqpoint{3.276567in}{2.440221in}}%
\pgfpathlineto{\pgfqpoint{3.284629in}{2.450169in}}%
\pgfpathlineto{\pgfqpoint{3.292684in}{2.460189in}}%
\pgfpathlineto{\pgfqpoint{3.300733in}{2.470282in}}%
\pgfpathlineto{\pgfqpoint{3.308776in}{2.480451in}}%
\pgfpathlineto{\pgfqpoint{3.295402in}{2.487435in}}%
\pgfpathlineto{\pgfqpoint{3.282031in}{2.494521in}}%
\pgfpathlineto{\pgfqpoint{3.268662in}{2.501710in}}%
\pgfpathlineto{\pgfqpoint{3.255297in}{2.509003in}}%
\pgfpathlineto{\pgfqpoint{3.247244in}{2.498731in}}%
\pgfpathlineto{\pgfqpoint{3.239185in}{2.488539in}}%
\pgfpathlineto{\pgfqpoint{3.231119in}{2.478425in}}%
\pgfpathlineto{\pgfqpoint{3.223047in}{2.468388in}}%
\pgfpathclose%
\pgfusepath{fill}%
\end{pgfscope}%
\begin{pgfscope}%
\pgfpathrectangle{\pgfqpoint{1.150000in}{0.150000in}}{\pgfqpoint{5.700000in}{5.700000in}}%
\pgfusepath{clip}%
\pgfsetbuttcap%
\pgfsetroundjoin%
\definecolor{currentfill}{rgb}{0.182256,0.426184,0.557120}%
\pgfsetfillcolor{currentfill}%
\pgfsetfillopacity{0.700000}%
\pgfsetlinewidth{0.000000pt}%
\definecolor{currentstroke}{rgb}{0.000000,0.000000,0.000000}%
\pgfsetstrokecolor{currentstroke}%
\pgfsetdash{}{0pt}%
\pgfpathmoveto{\pgfqpoint{5.592195in}{3.160706in}}%
\pgfpathlineto{\pgfqpoint{5.605986in}{3.155193in}}%
\pgfpathlineto{\pgfqpoint{5.619784in}{3.149746in}}%
\pgfpathlineto{\pgfqpoint{5.633589in}{3.144365in}}%
\pgfpathlineto{\pgfqpoint{5.647401in}{3.139051in}}%
\pgfpathlineto{\pgfqpoint{5.655033in}{3.160336in}}%
\pgfpathlineto{\pgfqpoint{5.662679in}{3.182138in}}%
\pgfpathlineto{\pgfqpoint{5.670340in}{3.204468in}}%
\pgfpathlineto{\pgfqpoint{5.656540in}{3.210216in}}%
\pgfpathlineto{\pgfqpoint{5.642747in}{3.216030in}}%
\pgfpathlineto{\pgfqpoint{5.628962in}{3.221911in}}%
\pgfpathlineto{\pgfqpoint{5.615183in}{3.227857in}}%
\pgfpathlineto{\pgfqpoint{5.607507in}{3.204944in}}%
\pgfpathlineto{\pgfqpoint{5.599844in}{3.182564in}}%
\pgfpathlineto{\pgfqpoint{5.592195in}{3.160706in}}%
\pgfpathclose%
\pgfusepath{fill}%
\end{pgfscope}%
\begin{pgfscope}%
\pgfpathrectangle{\pgfqpoint{1.150000in}{0.150000in}}{\pgfqpoint{5.700000in}{5.700000in}}%
\pgfusepath{clip}%
\pgfsetbuttcap%
\pgfsetroundjoin%
\definecolor{currentfill}{rgb}{0.257322,0.256130,0.526563}%
\pgfsetfillcolor{currentfill}%
\pgfsetfillopacity{0.700000}%
\pgfsetlinewidth{0.000000pt}%
\definecolor{currentstroke}{rgb}{0.000000,0.000000,0.000000}%
\pgfsetstrokecolor{currentstroke}%
\pgfsetdash{}{0pt}%
\pgfpathmoveto{\pgfqpoint{4.964935in}{2.749199in}}%
\pgfpathlineto{\pgfqpoint{4.978647in}{2.745693in}}%
\pgfpathlineto{\pgfqpoint{4.992367in}{2.742257in}}%
\pgfpathlineto{\pgfqpoint{5.006094in}{2.738891in}}%
\pgfpathlineto{\pgfqpoint{5.019829in}{2.735594in}}%
\pgfpathlineto{\pgfqpoint{5.027378in}{2.748341in}}%
\pgfpathlineto{\pgfqpoint{5.034928in}{2.761347in}}%
\pgfpathlineto{\pgfqpoint{5.042478in}{2.774619in}}%
\pgfpathlineto{\pgfqpoint{5.050031in}{2.788166in}}%
\pgfpathlineto{\pgfqpoint{5.036311in}{2.791881in}}%
\pgfpathlineto{\pgfqpoint{5.022599in}{2.795665in}}%
\pgfpathlineto{\pgfqpoint{5.008894in}{2.799520in}}%
\pgfpathlineto{\pgfqpoint{4.995197in}{2.803444in}}%
\pgfpathlineto{\pgfqpoint{4.987630in}{2.789471in}}%
\pgfpathlineto{\pgfqpoint{4.980064in}{2.775778in}}%
\pgfpathlineto{\pgfqpoint{4.972499in}{2.762357in}}%
\pgfpathlineto{\pgfqpoint{4.964935in}{2.749199in}}%
\pgfpathclose%
\pgfusepath{fill}%
\end{pgfscope}%
\begin{pgfscope}%
\pgfpathrectangle{\pgfqpoint{1.150000in}{0.150000in}}{\pgfqpoint{5.700000in}{5.700000in}}%
\pgfusepath{clip}%
\pgfsetbuttcap%
\pgfsetroundjoin%
\definecolor{currentfill}{rgb}{0.283187,0.125848,0.444960}%
\pgfsetfillcolor{currentfill}%
\pgfsetfillopacity{0.700000}%
\pgfsetlinewidth{0.000000pt}%
\definecolor{currentstroke}{rgb}{0.000000,0.000000,0.000000}%
\pgfsetstrokecolor{currentstroke}%
\pgfsetdash{}{0pt}%
\pgfpathmoveto{\pgfqpoint{3.950656in}{2.487334in}}%
\pgfpathlineto{\pgfqpoint{3.964141in}{2.482961in}}%
\pgfpathlineto{\pgfqpoint{3.977631in}{2.478671in}}%
\pgfpathlineto{\pgfqpoint{3.991126in}{2.474464in}}%
\pgfpathlineto{\pgfqpoint{4.004628in}{2.470339in}}%
\pgfpathlineto{\pgfqpoint{4.012454in}{2.480643in}}%
\pgfpathlineto{\pgfqpoint{4.020275in}{2.491042in}}%
\pgfpathlineto{\pgfqpoint{4.028092in}{2.501538in}}%
\pgfpathlineto{\pgfqpoint{4.035903in}{2.512137in}}%
\pgfpathlineto{\pgfqpoint{4.022411in}{2.516479in}}%
\pgfpathlineto{\pgfqpoint{4.008925in}{2.520904in}}%
\pgfpathlineto{\pgfqpoint{3.995445in}{2.525411in}}%
\pgfpathlineto{\pgfqpoint{3.981970in}{2.530002in}}%
\pgfpathlineto{\pgfqpoint{3.974149in}{2.519178in}}%
\pgfpathlineto{\pgfqpoint{3.966323in}{2.508462in}}%
\pgfpathlineto{\pgfqpoint{3.958492in}{2.497849in}}%
\pgfpathlineto{\pgfqpoint{3.950656in}{2.487334in}}%
\pgfpathclose%
\pgfusepath{fill}%
\end{pgfscope}%
\begin{pgfscope}%
\pgfpathrectangle{\pgfqpoint{1.150000in}{0.150000in}}{\pgfqpoint{5.700000in}{5.700000in}}%
\pgfusepath{clip}%
\pgfsetbuttcap%
\pgfsetroundjoin%
\definecolor{currentfill}{rgb}{0.283091,0.110553,0.431554}%
\pgfsetfillcolor{currentfill}%
\pgfsetfillopacity{0.700000}%
\pgfsetlinewidth{0.000000pt}%
\definecolor{currentstroke}{rgb}{0.000000,0.000000,0.000000}%
\pgfsetstrokecolor{currentstroke}%
\pgfsetdash{}{0pt}%
\pgfpathmoveto{\pgfqpoint{3.726144in}{2.460930in}}%
\pgfpathlineto{\pgfqpoint{3.739585in}{2.455902in}}%
\pgfpathlineto{\pgfqpoint{3.753032in}{2.450963in}}%
\pgfpathlineto{\pgfqpoint{3.766483in}{2.446111in}}%
\pgfpathlineto{\pgfqpoint{3.779939in}{2.441346in}}%
\pgfpathlineto{\pgfqpoint{3.787837in}{2.451546in}}%
\pgfpathlineto{\pgfqpoint{3.795730in}{2.461825in}}%
\pgfpathlineto{\pgfqpoint{3.803617in}{2.472186in}}%
\pgfpathlineto{\pgfqpoint{3.811499in}{2.482634in}}%
\pgfpathlineto{\pgfqpoint{3.798052in}{2.487576in}}%
\pgfpathlineto{\pgfqpoint{3.784611in}{2.492605in}}%
\pgfpathlineto{\pgfqpoint{3.771174in}{2.497722in}}%
\pgfpathlineto{\pgfqpoint{3.757742in}{2.502927in}}%
\pgfpathlineto{\pgfqpoint{3.749850in}{2.492294in}}%
\pgfpathlineto{\pgfqpoint{3.741954in}{2.481753in}}%
\pgfpathlineto{\pgfqpoint{3.734052in}{2.471299in}}%
\pgfpathlineto{\pgfqpoint{3.726144in}{2.460930in}}%
\pgfpathclose%
\pgfusepath{fill}%
\end{pgfscope}%
\begin{pgfscope}%
\pgfpathrectangle{\pgfqpoint{1.150000in}{0.150000in}}{\pgfqpoint{5.700000in}{5.700000in}}%
\pgfusepath{clip}%
\pgfsetbuttcap%
\pgfsetroundjoin%
\definecolor{currentfill}{rgb}{0.283229,0.120777,0.440584}%
\pgfsetfillcolor{currentfill}%
\pgfsetfillopacity{0.700000}%
\pgfsetlinewidth{0.000000pt}%
\definecolor{currentstroke}{rgb}{0.000000,0.000000,0.000000}%
\pgfsetstrokecolor{currentstroke}%
\pgfsetdash{}{0pt}%
\pgfpathmoveto{\pgfqpoint{3.083680in}{2.489754in}}%
\pgfpathlineto{\pgfqpoint{3.097050in}{2.481777in}}%
\pgfpathlineto{\pgfqpoint{3.110421in}{2.473910in}}%
\pgfpathlineto{\pgfqpoint{3.123794in}{2.466152in}}%
\pgfpathlineto{\pgfqpoint{3.137169in}{2.458501in}}%
\pgfpathlineto{\pgfqpoint{3.145279in}{2.468320in}}%
\pgfpathlineto{\pgfqpoint{3.153381in}{2.478213in}}%
\pgfpathlineto{\pgfqpoint{3.161477in}{2.488181in}}%
\pgfpathlineto{\pgfqpoint{3.169567in}{2.498227in}}%
\pgfpathlineto{\pgfqpoint{3.156202in}{2.505953in}}%
\pgfpathlineto{\pgfqpoint{3.142840in}{2.513788in}}%
\pgfpathlineto{\pgfqpoint{3.129479in}{2.521731in}}%
\pgfpathlineto{\pgfqpoint{3.116121in}{2.529784in}}%
\pgfpathlineto{\pgfqpoint{3.108021in}{2.519654in}}%
\pgfpathlineto{\pgfqpoint{3.099914in}{2.509607in}}%
\pgfpathlineto{\pgfqpoint{3.091801in}{2.499641in}}%
\pgfpathlineto{\pgfqpoint{3.083680in}{2.489754in}}%
\pgfpathclose%
\pgfusepath{fill}%
\end{pgfscope}%
\begin{pgfscope}%
\pgfpathrectangle{\pgfqpoint{1.150000in}{0.150000in}}{\pgfqpoint{5.700000in}{5.700000in}}%
\pgfusepath{clip}%
\pgfsetbuttcap%
\pgfsetroundjoin%
\definecolor{currentfill}{rgb}{0.282910,0.105393,0.426902}%
\pgfsetfillcolor{currentfill}%
\pgfsetfillopacity{0.700000}%
\pgfsetlinewidth{0.000000pt}%
\definecolor{currentstroke}{rgb}{0.000000,0.000000,0.000000}%
\pgfsetstrokecolor{currentstroke}%
\pgfsetdash{}{0pt}%
\pgfpathmoveto{\pgfqpoint{3.362301in}{2.453526in}}%
\pgfpathlineto{\pgfqpoint{3.375690in}{2.447044in}}%
\pgfpathlineto{\pgfqpoint{3.389082in}{2.440660in}}%
\pgfpathlineto{\pgfqpoint{3.402478in}{2.434374in}}%
\pgfpathlineto{\pgfqpoint{3.415878in}{2.428186in}}%
\pgfpathlineto{\pgfqpoint{3.423894in}{2.438215in}}%
\pgfpathlineto{\pgfqpoint{3.431904in}{2.448314in}}%
\pgfpathlineto{\pgfqpoint{3.439909in}{2.458485in}}%
\pgfpathlineto{\pgfqpoint{3.447907in}{2.468732in}}%
\pgfpathlineto{\pgfqpoint{3.434517in}{2.475037in}}%
\pgfpathlineto{\pgfqpoint{3.421131in}{2.481439in}}%
\pgfpathlineto{\pgfqpoint{3.407749in}{2.487939in}}%
\pgfpathlineto{\pgfqpoint{3.394369in}{2.494537in}}%
\pgfpathlineto{\pgfqpoint{3.386361in}{2.484167in}}%
\pgfpathlineto{\pgfqpoint{3.378347in}{2.473876in}}%
\pgfpathlineto{\pgfqpoint{3.370327in}{2.463663in}}%
\pgfpathlineto{\pgfqpoint{3.362301in}{2.453526in}}%
\pgfpathclose%
\pgfusepath{fill}%
\end{pgfscope}%
\begin{pgfscope}%
\pgfpathrectangle{\pgfqpoint{1.150000in}{0.150000in}}{\pgfqpoint{5.700000in}{5.700000in}}%
\pgfusepath{clip}%
\pgfsetbuttcap%
\pgfsetroundjoin%
\definecolor{currentfill}{rgb}{0.278012,0.180367,0.486697}%
\pgfsetfillcolor{currentfill}%
\pgfsetfillopacity{0.700000}%
\pgfsetlinewidth{0.000000pt}%
\definecolor{currentstroke}{rgb}{0.000000,0.000000,0.000000}%
\pgfsetstrokecolor{currentstroke}%
\pgfsetdash{}{0pt}%
\pgfpathmoveto{\pgfqpoint{4.484895in}{2.593305in}}%
\pgfpathlineto{\pgfqpoint{4.498501in}{2.589851in}}%
\pgfpathlineto{\pgfqpoint{4.512114in}{2.586473in}}%
\pgfpathlineto{\pgfqpoint{4.525734in}{2.583170in}}%
\pgfpathlineto{\pgfqpoint{4.539361in}{2.579940in}}%
\pgfpathlineto{\pgfqpoint{4.547024in}{2.590820in}}%
\pgfpathlineto{\pgfqpoint{4.554684in}{2.601857in}}%
\pgfpathlineto{\pgfqpoint{4.562340in}{2.613056in}}%
\pgfpathlineto{\pgfqpoint{4.569994in}{2.624423in}}%
\pgfpathlineto{\pgfqpoint{4.556380in}{2.627971in}}%
\pgfpathlineto{\pgfqpoint{4.542772in}{2.631592in}}%
\pgfpathlineto{\pgfqpoint{4.529171in}{2.635288in}}%
\pgfpathlineto{\pgfqpoint{4.515577in}{2.639059in}}%
\pgfpathlineto{\pgfqpoint{4.507911in}{2.627367in}}%
\pgfpathlineto{\pgfqpoint{4.500243in}{2.615848in}}%
\pgfpathlineto{\pgfqpoint{4.492571in}{2.604495in}}%
\pgfpathlineto{\pgfqpoint{4.484895in}{2.593305in}}%
\pgfpathclose%
\pgfusepath{fill}%
\end{pgfscope}%
\begin{pgfscope}%
\pgfpathrectangle{\pgfqpoint{1.150000in}{0.150000in}}{\pgfqpoint{5.700000in}{5.700000in}}%
\pgfusepath{clip}%
\pgfsetbuttcap%
\pgfsetroundjoin%
\definecolor{currentfill}{rgb}{0.279574,0.170599,0.479997}%
\pgfsetfillcolor{currentfill}%
\pgfsetfillopacity{0.700000}%
\pgfsetlinewidth{0.000000pt}%
\definecolor{currentstroke}{rgb}{0.000000,0.000000,0.000000}%
\pgfsetstrokecolor{currentstroke}%
\pgfsetdash{}{0pt}%
\pgfpathmoveto{\pgfqpoint{2.750806in}{2.595438in}}%
\pgfpathlineto{\pgfqpoint{2.764187in}{2.585169in}}%
\pgfpathlineto{\pgfqpoint{2.777568in}{2.575028in}}%
\pgfpathlineto{\pgfqpoint{2.790948in}{2.565015in}}%
\pgfpathlineto{\pgfqpoint{2.804328in}{2.555129in}}%
\pgfpathlineto{\pgfqpoint{2.812555in}{2.564545in}}%
\pgfpathlineto{\pgfqpoint{2.820774in}{2.574051in}}%
\pgfpathlineto{\pgfqpoint{2.828986in}{2.583646in}}%
\pgfpathlineto{\pgfqpoint{2.837190in}{2.593331in}}%
\pgfpathlineto{\pgfqpoint{2.823822in}{2.603253in}}%
\pgfpathlineto{\pgfqpoint{2.810455in}{2.613301in}}%
\pgfpathlineto{\pgfqpoint{2.797087in}{2.623476in}}%
\pgfpathlineto{\pgfqpoint{2.783719in}{2.633780in}}%
\pgfpathlineto{\pgfqpoint{2.775503in}{2.624052in}}%
\pgfpathlineto{\pgfqpoint{2.767278in}{2.614420in}}%
\pgfpathlineto{\pgfqpoint{2.759046in}{2.604882in}}%
\pgfpathlineto{\pgfqpoint{2.750806in}{2.595438in}}%
\pgfpathclose%
\pgfusepath{fill}%
\end{pgfscope}%
\begin{pgfscope}%
\pgfpathrectangle{\pgfqpoint{1.150000in}{0.150000in}}{\pgfqpoint{5.700000in}{5.700000in}}%
\pgfusepath{clip}%
\pgfsetbuttcap%
\pgfsetroundjoin%
\definecolor{currentfill}{rgb}{0.204903,0.375746,0.553533}%
\pgfsetfillcolor{currentfill}%
\pgfsetfillopacity{0.700000}%
\pgfsetlinewidth{0.000000pt}%
\definecolor{currentstroke}{rgb}{0.000000,0.000000,0.000000}%
\pgfsetstrokecolor{currentstroke}%
\pgfsetdash{}{0pt}%
\pgfpathmoveto{\pgfqpoint{5.476204in}{3.021654in}}%
\pgfpathlineto{\pgfqpoint{5.489999in}{3.016975in}}%
\pgfpathlineto{\pgfqpoint{5.503802in}{3.012363in}}%
\pgfpathlineto{\pgfqpoint{5.517613in}{3.007817in}}%
\pgfpathlineto{\pgfqpoint{5.531431in}{3.003337in}}%
\pgfpathlineto{\pgfqpoint{5.538989in}{3.021404in}}%
\pgfpathlineto{\pgfqpoint{5.546557in}{3.039909in}}%
\pgfpathlineto{\pgfqpoint{5.554135in}{3.058862in}}%
\pgfpathlineto{\pgfqpoint{5.561724in}{3.078273in}}%
\pgfpathlineto{\pgfqpoint{5.547924in}{3.083292in}}%
\pgfpathlineto{\pgfqpoint{5.534130in}{3.088377in}}%
\pgfpathlineto{\pgfqpoint{5.520345in}{3.093528in}}%
\pgfpathlineto{\pgfqpoint{5.506566in}{3.098746in}}%
\pgfpathlineto{\pgfqpoint{5.498960in}{3.078789in}}%
\pgfpathlineto{\pgfqpoint{5.491365in}{3.059294in}}%
\pgfpathlineto{\pgfqpoint{5.483780in}{3.040253in}}%
\pgfpathlineto{\pgfqpoint{5.476204in}{3.021654in}}%
\pgfpathclose%
\pgfusepath{fill}%
\end{pgfscope}%
\begin{pgfscope}%
\pgfpathrectangle{\pgfqpoint{1.150000in}{0.150000in}}{\pgfqpoint{5.700000in}{5.700000in}}%
\pgfusepath{clip}%
\pgfsetbuttcap%
\pgfsetroundjoin%
\definecolor{currentfill}{rgb}{0.282290,0.145912,0.461510}%
\pgfsetfillcolor{currentfill}%
\pgfsetfillopacity{0.700000}%
\pgfsetlinewidth{0.000000pt}%
\definecolor{currentstroke}{rgb}{0.000000,0.000000,0.000000}%
\pgfsetstrokecolor{currentstroke}%
\pgfsetdash{}{0pt}%
\pgfpathmoveto{\pgfqpoint{4.175162in}{2.521926in}}%
\pgfpathlineto{\pgfqpoint{4.188697in}{2.518074in}}%
\pgfpathlineto{\pgfqpoint{4.202238in}{2.514300in}}%
\pgfpathlineto{\pgfqpoint{4.215785in}{2.510605in}}%
\pgfpathlineto{\pgfqpoint{4.229338in}{2.506989in}}%
\pgfpathlineto{\pgfqpoint{4.237095in}{2.517413in}}%
\pgfpathlineto{\pgfqpoint{4.244847in}{2.527950in}}%
\pgfpathlineto{\pgfqpoint{4.252595in}{2.538606in}}%
\pgfpathlineto{\pgfqpoint{4.260338in}{2.549386in}}%
\pgfpathlineto{\pgfqpoint{4.246795in}{2.553260in}}%
\pgfpathlineto{\pgfqpoint{4.233259in}{2.557212in}}%
\pgfpathlineto{\pgfqpoint{4.219728in}{2.561244in}}%
\pgfpathlineto{\pgfqpoint{4.206204in}{2.565354in}}%
\pgfpathlineto{\pgfqpoint{4.198450in}{2.554309in}}%
\pgfpathlineto{\pgfqpoint{4.190692in}{2.543393in}}%
\pgfpathlineto{\pgfqpoint{4.182929in}{2.532600in}}%
\pgfpathlineto{\pgfqpoint{4.175162in}{2.521926in}}%
\pgfpathclose%
\pgfusepath{fill}%
\end{pgfscope}%
\begin{pgfscope}%
\pgfpathrectangle{\pgfqpoint{1.150000in}{0.150000in}}{\pgfqpoint{5.700000in}{5.700000in}}%
\pgfusepath{clip}%
\pgfsetbuttcap%
\pgfsetroundjoin%
\definecolor{currentfill}{rgb}{0.263663,0.237631,0.518762}%
\pgfsetfillcolor{currentfill}%
\pgfsetfillopacity{0.700000}%
\pgfsetlinewidth{0.000000pt}%
\definecolor{currentstroke}{rgb}{0.000000,0.000000,0.000000}%
\pgfsetstrokecolor{currentstroke}%
\pgfsetdash{}{0pt}%
\pgfpathmoveto{\pgfqpoint{4.879848in}{2.712198in}}%
\pgfpathlineto{\pgfqpoint{4.893545in}{2.708808in}}%
\pgfpathlineto{\pgfqpoint{4.907249in}{2.705489in}}%
\pgfpathlineto{\pgfqpoint{4.920961in}{2.702241in}}%
\pgfpathlineto{\pgfqpoint{4.934681in}{2.699063in}}%
\pgfpathlineto{\pgfqpoint{4.942244in}{2.711238in}}%
\pgfpathlineto{\pgfqpoint{4.949808in}{2.723647in}}%
\pgfpathlineto{\pgfqpoint{4.957371in}{2.736299in}}%
\pgfpathlineto{\pgfqpoint{4.964935in}{2.749199in}}%
\pgfpathlineto{\pgfqpoint{4.951231in}{2.752776in}}%
\pgfpathlineto{\pgfqpoint{4.937533in}{2.756422in}}%
\pgfpathlineto{\pgfqpoint{4.923843in}{2.760140in}}%
\pgfpathlineto{\pgfqpoint{4.910161in}{2.763927in}}%
\pgfpathlineto{\pgfqpoint{4.902583in}{2.750621in}}%
\pgfpathlineto{\pgfqpoint{4.895005in}{2.737569in}}%
\pgfpathlineto{\pgfqpoint{4.887427in}{2.724764in}}%
\pgfpathlineto{\pgfqpoint{4.879848in}{2.712198in}}%
\pgfpathclose%
\pgfusepath{fill}%
\end{pgfscope}%
\begin{pgfscope}%
\pgfpathrectangle{\pgfqpoint{1.150000in}{0.150000in}}{\pgfqpoint{5.700000in}{5.700000in}}%
\pgfusepath{clip}%
\pgfsetbuttcap%
\pgfsetroundjoin%
\definecolor{currentfill}{rgb}{0.282910,0.105393,0.426902}%
\pgfsetfillcolor{currentfill}%
\pgfsetfillopacity{0.700000}%
\pgfsetlinewidth{0.000000pt}%
\definecolor{currentstroke}{rgb}{0.000000,0.000000,0.000000}%
\pgfsetstrokecolor{currentstroke}%
\pgfsetdash{}{0pt}%
\pgfpathmoveto{\pgfqpoint{3.501501in}{2.444473in}}%
\pgfpathlineto{\pgfqpoint{3.514909in}{2.438646in}}%
\pgfpathlineto{\pgfqpoint{3.528322in}{2.432913in}}%
\pgfpathlineto{\pgfqpoint{3.541738in}{2.427273in}}%
\pgfpathlineto{\pgfqpoint{3.555158in}{2.421726in}}%
\pgfpathlineto{\pgfqpoint{3.563131in}{2.431792in}}%
\pgfpathlineto{\pgfqpoint{3.571098in}{2.441927in}}%
\pgfpathlineto{\pgfqpoint{3.579059in}{2.452136in}}%
\pgfpathlineto{\pgfqpoint{3.587014in}{2.462420in}}%
\pgfpathlineto{\pgfqpoint{3.573603in}{2.468104in}}%
\pgfpathlineto{\pgfqpoint{3.560197in}{2.473881in}}%
\pgfpathlineto{\pgfqpoint{3.546794in}{2.479751in}}%
\pgfpathlineto{\pgfqpoint{3.533395in}{2.485715in}}%
\pgfpathlineto{\pgfqpoint{3.525431in}{2.475286in}}%
\pgfpathlineto{\pgfqpoint{3.517460in}{2.464938in}}%
\pgfpathlineto{\pgfqpoint{3.509484in}{2.454668in}}%
\pgfpathlineto{\pgfqpoint{3.501501in}{2.444473in}}%
\pgfpathclose%
\pgfusepath{fill}%
\end{pgfscope}%
\begin{pgfscope}%
\pgfpathrectangle{\pgfqpoint{1.150000in}{0.150000in}}{\pgfqpoint{5.700000in}{5.700000in}}%
\pgfusepath{clip}%
\pgfsetbuttcap%
\pgfsetroundjoin%
\definecolor{currentfill}{rgb}{0.282884,0.135920,0.453427}%
\pgfsetfillcolor{currentfill}%
\pgfsetfillopacity{0.700000}%
\pgfsetlinewidth{0.000000pt}%
\definecolor{currentstroke}{rgb}{0.000000,0.000000,0.000000}%
\pgfsetstrokecolor{currentstroke}%
\pgfsetdash{}{0pt}%
\pgfpathmoveto{\pgfqpoint{2.944133in}{2.518387in}}%
\pgfpathlineto{\pgfqpoint{2.957504in}{2.509558in}}%
\pgfpathlineto{\pgfqpoint{2.970875in}{2.500845in}}%
\pgfpathlineto{\pgfqpoint{2.984247in}{2.492248in}}%
\pgfpathlineto{\pgfqpoint{2.997621in}{2.483765in}}%
\pgfpathlineto{\pgfqpoint{3.005781in}{2.493399in}}%
\pgfpathlineto{\pgfqpoint{3.013934in}{2.503112in}}%
\pgfpathlineto{\pgfqpoint{3.022079in}{2.512903in}}%
\pgfpathlineto{\pgfqpoint{3.030218in}{2.522776in}}%
\pgfpathlineto{\pgfqpoint{3.016856in}{2.531314in}}%
\pgfpathlineto{\pgfqpoint{3.003495in}{2.539967in}}%
\pgfpathlineto{\pgfqpoint{2.990135in}{2.548735in}}%
\pgfpathlineto{\pgfqpoint{2.976777in}{2.557620in}}%
\pgfpathlineto{\pgfqpoint{2.968627in}{2.547684in}}%
\pgfpathlineto{\pgfqpoint{2.960469in}{2.537834in}}%
\pgfpathlineto{\pgfqpoint{2.952305in}{2.528069in}}%
\pgfpathlineto{\pgfqpoint{2.944133in}{2.518387in}}%
\pgfpathclose%
\pgfusepath{fill}%
\end{pgfscope}%
\begin{pgfscope}%
\pgfpathrectangle{\pgfqpoint{1.150000in}{0.150000in}}{\pgfqpoint{5.700000in}{5.700000in}}%
\pgfusepath{clip}%
\pgfsetbuttcap%
\pgfsetroundjoin%
\definecolor{currentfill}{rgb}{0.267968,0.223549,0.512008}%
\pgfsetfillcolor{currentfill}%
\pgfsetfillopacity{0.700000}%
\pgfsetlinewidth{0.000000pt}%
\definecolor{currentstroke}{rgb}{0.000000,0.000000,0.000000}%
\pgfsetstrokecolor{currentstroke}%
\pgfsetdash{}{0pt}%
\pgfpathmoveto{\pgfqpoint{4.794757in}{2.676949in}}%
\pgfpathlineto{\pgfqpoint{4.808439in}{2.673652in}}%
\pgfpathlineto{\pgfqpoint{4.822127in}{2.670428in}}%
\pgfpathlineto{\pgfqpoint{4.835824in}{2.667274in}}%
\pgfpathlineto{\pgfqpoint{4.849527in}{2.664192in}}%
\pgfpathlineto{\pgfqpoint{4.857109in}{2.675869in}}%
\pgfpathlineto{\pgfqpoint{4.864690in}{2.687758in}}%
\pgfpathlineto{\pgfqpoint{4.872269in}{2.699865in}}%
\pgfpathlineto{\pgfqpoint{4.879848in}{2.712198in}}%
\pgfpathlineto{\pgfqpoint{4.866159in}{2.715659in}}%
\pgfpathlineto{\pgfqpoint{4.852477in}{2.719190in}}%
\pgfpathlineto{\pgfqpoint{4.838802in}{2.722793in}}%
\pgfpathlineto{\pgfqpoint{4.825134in}{2.726467in}}%
\pgfpathlineto{\pgfqpoint{4.817541in}{2.713749in}}%
\pgfpathlineto{\pgfqpoint{4.809948in}{2.701261in}}%
\pgfpathlineto{\pgfqpoint{4.802353in}{2.688996in}}%
\pgfpathlineto{\pgfqpoint{4.794757in}{2.676949in}}%
\pgfpathclose%
\pgfusepath{fill}%
\end{pgfscope}%
\begin{pgfscope}%
\pgfpathrectangle{\pgfqpoint{1.150000in}{0.150000in}}{\pgfqpoint{5.700000in}{5.700000in}}%
\pgfusepath{clip}%
\pgfsetbuttcap%
\pgfsetroundjoin%
\definecolor{currentfill}{rgb}{0.194100,0.399323,0.555565}%
\pgfsetfillcolor{currentfill}%
\pgfsetfillopacity{0.700000}%
\pgfsetlinewidth{0.000000pt}%
\definecolor{currentstroke}{rgb}{0.000000,0.000000,0.000000}%
\pgfsetstrokecolor{currentstroke}%
\pgfsetdash{}{0pt}%
\pgfpathmoveto{\pgfqpoint{5.561724in}{3.078273in}}%
\pgfpathlineto{\pgfqpoint{5.575532in}{3.073320in}}%
\pgfpathlineto{\pgfqpoint{5.589347in}{3.068433in}}%
\pgfpathlineto{\pgfqpoint{5.603170in}{3.063612in}}%
\pgfpathlineto{\pgfqpoint{5.617000in}{3.058857in}}%
\pgfpathlineto{\pgfqpoint{5.624582in}{3.078184in}}%
\pgfpathlineto{\pgfqpoint{5.632176in}{3.097985in}}%
\pgfpathlineto{\pgfqpoint{5.639782in}{3.118271in}}%
\pgfpathlineto{\pgfqpoint{5.647401in}{3.139051in}}%
\pgfpathlineto{\pgfqpoint{5.633589in}{3.144365in}}%
\pgfpathlineto{\pgfqpoint{5.619784in}{3.149746in}}%
\pgfpathlineto{\pgfqpoint{5.605986in}{3.155193in}}%
\pgfpathlineto{\pgfqpoint{5.592195in}{3.160706in}}%
\pgfpathlineto{\pgfqpoint{5.584559in}{3.139358in}}%
\pgfpathlineto{\pgfqpoint{5.576935in}{3.118510in}}%
\pgfpathlineto{\pgfqpoint{5.569324in}{3.098152in}}%
\pgfpathlineto{\pgfqpoint{5.561724in}{3.078273in}}%
\pgfpathclose%
\pgfusepath{fill}%
\end{pgfscope}%
\begin{pgfscope}%
\pgfpathrectangle{\pgfqpoint{1.150000in}{0.150000in}}{\pgfqpoint{5.700000in}{5.700000in}}%
\pgfusepath{clip}%
\pgfsetbuttcap%
\pgfsetroundjoin%
\definecolor{currentfill}{rgb}{0.280255,0.165693,0.476498}%
\pgfsetfillcolor{currentfill}%
\pgfsetfillopacity{0.700000}%
\pgfsetlinewidth{0.000000pt}%
\definecolor{currentstroke}{rgb}{0.000000,0.000000,0.000000}%
\pgfsetstrokecolor{currentstroke}%
\pgfsetdash{}{0pt}%
\pgfpathmoveto{\pgfqpoint{4.399757in}{2.563418in}}%
\pgfpathlineto{\pgfqpoint{4.413348in}{2.559961in}}%
\pgfpathlineto{\pgfqpoint{4.426945in}{2.556580in}}%
\pgfpathlineto{\pgfqpoint{4.440549in}{2.553275in}}%
\pgfpathlineto{\pgfqpoint{4.454160in}{2.550044in}}%
\pgfpathlineto{\pgfqpoint{4.461849in}{2.560645in}}%
\pgfpathlineto{\pgfqpoint{4.469535in}{2.571385in}}%
\pgfpathlineto{\pgfqpoint{4.477217in}{2.582270in}}%
\pgfpathlineto{\pgfqpoint{4.484895in}{2.593305in}}%
\pgfpathlineto{\pgfqpoint{4.471296in}{2.596833in}}%
\pgfpathlineto{\pgfqpoint{4.457704in}{2.600437in}}%
\pgfpathlineto{\pgfqpoint{4.444118in}{2.604116in}}%
\pgfpathlineto{\pgfqpoint{4.430539in}{2.607871in}}%
\pgfpathlineto{\pgfqpoint{4.422849in}{2.596530in}}%
\pgfpathlineto{\pgfqpoint{4.415156in}{2.585345in}}%
\pgfpathlineto{\pgfqpoint{4.407458in}{2.574310in}}%
\pgfpathlineto{\pgfqpoint{4.399757in}{2.563418in}}%
\pgfpathclose%
\pgfusepath{fill}%
\end{pgfscope}%
\begin{pgfscope}%
\pgfpathrectangle{\pgfqpoint{1.150000in}{0.150000in}}{\pgfqpoint{5.700000in}{5.700000in}}%
\pgfusepath{clip}%
\pgfsetbuttcap%
\pgfsetroundjoin%
\definecolor{currentfill}{rgb}{0.283197,0.115680,0.436115}%
\pgfsetfillcolor{currentfill}%
\pgfsetfillopacity{0.700000}%
\pgfsetlinewidth{0.000000pt}%
\definecolor{currentstroke}{rgb}{0.000000,0.000000,0.000000}%
\pgfsetstrokecolor{currentstroke}%
\pgfsetdash{}{0pt}%
\pgfpathmoveto{\pgfqpoint{3.865337in}{2.463729in}}%
\pgfpathlineto{\pgfqpoint{3.878809in}{2.459217in}}%
\pgfpathlineto{\pgfqpoint{3.892287in}{2.454789in}}%
\pgfpathlineto{\pgfqpoint{3.905770in}{2.450446in}}%
\pgfpathlineto{\pgfqpoint{3.919259in}{2.446187in}}%
\pgfpathlineto{\pgfqpoint{3.927116in}{2.456345in}}%
\pgfpathlineto{\pgfqpoint{3.934968in}{2.466586in}}%
\pgfpathlineto{\pgfqpoint{3.942815in}{2.476915in}}%
\pgfpathlineto{\pgfqpoint{3.950656in}{2.487334in}}%
\pgfpathlineto{\pgfqpoint{3.937177in}{2.491791in}}%
\pgfpathlineto{\pgfqpoint{3.923703in}{2.496332in}}%
\pgfpathlineto{\pgfqpoint{3.910235in}{2.500957in}}%
\pgfpathlineto{\pgfqpoint{3.896772in}{2.505667in}}%
\pgfpathlineto{\pgfqpoint{3.888921in}{2.495042in}}%
\pgfpathlineto{\pgfqpoint{3.881065in}{2.484514in}}%
\pgfpathlineto{\pgfqpoint{3.873204in}{2.474077in}}%
\pgfpathlineto{\pgfqpoint{3.865337in}{2.463729in}}%
\pgfpathclose%
\pgfusepath{fill}%
\end{pgfscope}%
\begin{pgfscope}%
\pgfpathrectangle{\pgfqpoint{1.150000in}{0.150000in}}{\pgfqpoint{5.700000in}{5.700000in}}%
\pgfusepath{clip}%
\pgfsetbuttcap%
\pgfsetroundjoin%
\definecolor{currentfill}{rgb}{0.282910,0.105393,0.426902}%
\pgfsetfillcolor{currentfill}%
\pgfsetfillopacity{0.700000}%
\pgfsetlinewidth{0.000000pt}%
\definecolor{currentstroke}{rgb}{0.000000,0.000000,0.000000}%
\pgfsetstrokecolor{currentstroke}%
\pgfsetdash{}{0pt}%
\pgfpathmoveto{\pgfqpoint{3.640699in}{2.440603in}}%
\pgfpathlineto{\pgfqpoint{3.654132in}{2.435376in}}%
\pgfpathlineto{\pgfqpoint{3.667569in}{2.430239in}}%
\pgfpathlineto{\pgfqpoint{3.681010in}{2.425191in}}%
\pgfpathlineto{\pgfqpoint{3.694457in}{2.420232in}}%
\pgfpathlineto{\pgfqpoint{3.702387in}{2.430296in}}%
\pgfpathlineto{\pgfqpoint{3.710312in}{2.440432in}}%
\pgfpathlineto{\pgfqpoint{3.718230in}{2.450642in}}%
\pgfpathlineto{\pgfqpoint{3.726144in}{2.460930in}}%
\pgfpathlineto{\pgfqpoint{3.712707in}{2.466046in}}%
\pgfpathlineto{\pgfqpoint{3.699275in}{2.471251in}}%
\pgfpathlineto{\pgfqpoint{3.685847in}{2.476546in}}%
\pgfpathlineto{\pgfqpoint{3.672425in}{2.481930in}}%
\pgfpathlineto{\pgfqpoint{3.664502in}{2.471477in}}%
\pgfpathlineto{\pgfqpoint{3.656573in}{2.461107in}}%
\pgfpathlineto{\pgfqpoint{3.648639in}{2.450817in}}%
\pgfpathlineto{\pgfqpoint{3.640699in}{2.440603in}}%
\pgfpathclose%
\pgfusepath{fill}%
\end{pgfscope}%
\begin{pgfscope}%
\pgfpathrectangle{\pgfqpoint{1.150000in}{0.150000in}}{\pgfqpoint{5.700000in}{5.700000in}}%
\pgfusepath{clip}%
\pgfsetbuttcap%
\pgfsetroundjoin%
\definecolor{currentfill}{rgb}{0.282884,0.135920,0.453427}%
\pgfsetfillcolor{currentfill}%
\pgfsetfillopacity{0.700000}%
\pgfsetlinewidth{0.000000pt}%
\definecolor{currentstroke}{rgb}{0.000000,0.000000,0.000000}%
\pgfsetstrokecolor{currentstroke}%
\pgfsetdash{}{0pt}%
\pgfpathmoveto{\pgfqpoint{4.089927in}{2.495585in}}%
\pgfpathlineto{\pgfqpoint{4.103448in}{2.491650in}}%
\pgfpathlineto{\pgfqpoint{4.116974in}{2.487796in}}%
\pgfpathlineto{\pgfqpoint{4.130507in}{2.484021in}}%
\pgfpathlineto{\pgfqpoint{4.144046in}{2.480327in}}%
\pgfpathlineto{\pgfqpoint{4.151832in}{2.490571in}}%
\pgfpathlineto{\pgfqpoint{4.159613in}{2.500916in}}%
\pgfpathlineto{\pgfqpoint{4.167390in}{2.511367in}}%
\pgfpathlineto{\pgfqpoint{4.175162in}{2.521926in}}%
\pgfpathlineto{\pgfqpoint{4.161633in}{2.525859in}}%
\pgfpathlineto{\pgfqpoint{4.148111in}{2.529871in}}%
\pgfpathlineto{\pgfqpoint{4.134594in}{2.533963in}}%
\pgfpathlineto{\pgfqpoint{4.121083in}{2.538136in}}%
\pgfpathlineto{\pgfqpoint{4.113301in}{2.527331in}}%
\pgfpathlineto{\pgfqpoint{4.105515in}{2.516640in}}%
\pgfpathlineto{\pgfqpoint{4.097723in}{2.506060in}}%
\pgfpathlineto{\pgfqpoint{4.089927in}{2.495585in}}%
\pgfpathclose%
\pgfusepath{fill}%
\end{pgfscope}%
\begin{pgfscope}%
\pgfpathrectangle{\pgfqpoint{1.150000in}{0.150000in}}{\pgfqpoint{5.700000in}{5.700000in}}%
\pgfusepath{clip}%
\pgfsetbuttcap%
\pgfsetroundjoin%
\definecolor{currentfill}{rgb}{0.271828,0.209303,0.504434}%
\pgfsetfillcolor{currentfill}%
\pgfsetfillopacity{0.700000}%
\pgfsetlinewidth{0.000000pt}%
\definecolor{currentstroke}{rgb}{0.000000,0.000000,0.000000}%
\pgfsetstrokecolor{currentstroke}%
\pgfsetdash{}{0pt}%
\pgfpathmoveto{\pgfqpoint{4.709652in}{2.643262in}}%
\pgfpathlineto{\pgfqpoint{4.723317in}{2.640036in}}%
\pgfpathlineto{\pgfqpoint{4.736990in}{2.636883in}}%
\pgfpathlineto{\pgfqpoint{4.750671in}{2.633802in}}%
\pgfpathlineto{\pgfqpoint{4.764358in}{2.630792in}}%
\pgfpathlineto{\pgfqpoint{4.771961in}{2.642039in}}%
\pgfpathlineto{\pgfqpoint{4.779561in}{2.653477in}}%
\pgfpathlineto{\pgfqpoint{4.787160in}{2.665111in}}%
\pgfpathlineto{\pgfqpoint{4.794757in}{2.676949in}}%
\pgfpathlineto{\pgfqpoint{4.781083in}{2.680316in}}%
\pgfpathlineto{\pgfqpoint{4.767417in}{2.683756in}}%
\pgfpathlineto{\pgfqpoint{4.753757in}{2.687268in}}%
\pgfpathlineto{\pgfqpoint{4.740105in}{2.690852in}}%
\pgfpathlineto{\pgfqpoint{4.732494in}{2.678649in}}%
\pgfpathlineto{\pgfqpoint{4.724882in}{2.666653in}}%
\pgfpathlineto{\pgfqpoint{4.717268in}{2.654860in}}%
\pgfpathlineto{\pgfqpoint{4.709652in}{2.643262in}}%
\pgfpathclose%
\pgfusepath{fill}%
\end{pgfscope}%
\begin{pgfscope}%
\pgfpathrectangle{\pgfqpoint{1.150000in}{0.150000in}}{\pgfqpoint{5.700000in}{5.700000in}}%
\pgfusepath{clip}%
\pgfsetbuttcap%
\pgfsetroundjoin%
\definecolor{currentfill}{rgb}{0.281412,0.155834,0.469201}%
\pgfsetfillcolor{currentfill}%
\pgfsetfillopacity{0.700000}%
\pgfsetlinewidth{0.000000pt}%
\definecolor{currentstroke}{rgb}{0.000000,0.000000,0.000000}%
\pgfsetstrokecolor{currentstroke}%
\pgfsetdash{}{0pt}%
\pgfpathmoveto{\pgfqpoint{2.804328in}{2.555129in}}%
\pgfpathlineto{\pgfqpoint{2.817708in}{2.545367in}}%
\pgfpathlineto{\pgfqpoint{2.831088in}{2.535731in}}%
\pgfpathlineto{\pgfqpoint{2.844468in}{2.526217in}}%
\pgfpathlineto{\pgfqpoint{2.857848in}{2.516826in}}%
\pgfpathlineto{\pgfqpoint{2.866062in}{2.526215in}}%
\pgfpathlineto{\pgfqpoint{2.874268in}{2.535688in}}%
\pgfpathlineto{\pgfqpoint{2.882467in}{2.545246in}}%
\pgfpathlineto{\pgfqpoint{2.890658in}{2.554889in}}%
\pgfpathlineto{\pgfqpoint{2.877291in}{2.564315in}}%
\pgfpathlineto{\pgfqpoint{2.863924in}{2.573863in}}%
\pgfpathlineto{\pgfqpoint{2.850557in}{2.583535in}}%
\pgfpathlineto{\pgfqpoint{2.837190in}{2.593331in}}%
\pgfpathlineto{\pgfqpoint{2.828986in}{2.583646in}}%
\pgfpathlineto{\pgfqpoint{2.820774in}{2.574051in}}%
\pgfpathlineto{\pgfqpoint{2.812555in}{2.564545in}}%
\pgfpathlineto{\pgfqpoint{2.804328in}{2.555129in}}%
\pgfpathclose%
\pgfusepath{fill}%
\end{pgfscope}%
\begin{pgfscope}%
\pgfpathrectangle{\pgfqpoint{1.150000in}{0.150000in}}{\pgfqpoint{5.700000in}{5.700000in}}%
\pgfusepath{clip}%
\pgfsetbuttcap%
\pgfsetroundjoin%
\definecolor{currentfill}{rgb}{0.283091,0.110553,0.431554}%
\pgfsetfillcolor{currentfill}%
\pgfsetfillopacity{0.700000}%
\pgfsetlinewidth{0.000000pt}%
\definecolor{currentstroke}{rgb}{0.000000,0.000000,0.000000}%
\pgfsetstrokecolor{currentstroke}%
\pgfsetdash{}{0pt}%
\pgfpathmoveto{\pgfqpoint{3.137169in}{2.458501in}}%
\pgfpathlineto{\pgfqpoint{3.150547in}{2.450959in}}%
\pgfpathlineto{\pgfqpoint{3.163926in}{2.443522in}}%
\pgfpathlineto{\pgfqpoint{3.177308in}{2.436192in}}%
\pgfpathlineto{\pgfqpoint{3.190693in}{2.428967in}}%
\pgfpathlineto{\pgfqpoint{3.198791in}{2.438717in}}%
\pgfpathlineto{\pgfqpoint{3.206883in}{2.448536in}}%
\pgfpathlineto{\pgfqpoint{3.214968in}{2.458426in}}%
\pgfpathlineto{\pgfqpoint{3.223047in}{2.468388in}}%
\pgfpathlineto{\pgfqpoint{3.209673in}{2.475690in}}%
\pgfpathlineto{\pgfqpoint{3.196302in}{2.483096in}}%
\pgfpathlineto{\pgfqpoint{3.182933in}{2.490608in}}%
\pgfpathlineto{\pgfqpoint{3.169567in}{2.498227in}}%
\pgfpathlineto{\pgfqpoint{3.161477in}{2.488181in}}%
\pgfpathlineto{\pgfqpoint{3.153381in}{2.478213in}}%
\pgfpathlineto{\pgfqpoint{3.145279in}{2.468320in}}%
\pgfpathlineto{\pgfqpoint{3.137169in}{2.458501in}}%
\pgfpathclose%
\pgfusepath{fill}%
\end{pgfscope}%
\begin{pgfscope}%
\pgfpathrectangle{\pgfqpoint{1.150000in}{0.150000in}}{\pgfqpoint{5.700000in}{5.700000in}}%
\pgfusepath{clip}%
\pgfsetbuttcap%
\pgfsetroundjoin%
\definecolor{currentfill}{rgb}{0.282910,0.105393,0.426902}%
\pgfsetfillcolor{currentfill}%
\pgfsetfillopacity{0.700000}%
\pgfsetlinewidth{0.000000pt}%
\definecolor{currentstroke}{rgb}{0.000000,0.000000,0.000000}%
\pgfsetstrokecolor{currentstroke}%
\pgfsetdash{}{0pt}%
\pgfpathmoveto{\pgfqpoint{3.276567in}{2.440221in}}%
\pgfpathlineto{\pgfqpoint{3.289954in}{2.433435in}}%
\pgfpathlineto{\pgfqpoint{3.303344in}{2.426750in}}%
\pgfpathlineto{\pgfqpoint{3.316737in}{2.420166in}}%
\pgfpathlineto{\pgfqpoint{3.330133in}{2.413682in}}%
\pgfpathlineto{\pgfqpoint{3.338184in}{2.423541in}}%
\pgfpathlineto{\pgfqpoint{3.346229in}{2.433466in}}%
\pgfpathlineto{\pgfqpoint{3.354268in}{2.443461in}}%
\pgfpathlineto{\pgfqpoint{3.362301in}{2.453526in}}%
\pgfpathlineto{\pgfqpoint{3.348915in}{2.460107in}}%
\pgfpathlineto{\pgfqpoint{3.335532in}{2.466788in}}%
\pgfpathlineto{\pgfqpoint{3.322152in}{2.473569in}}%
\pgfpathlineto{\pgfqpoint{3.308776in}{2.480451in}}%
\pgfpathlineto{\pgfqpoint{3.300733in}{2.470282in}}%
\pgfpathlineto{\pgfqpoint{3.292684in}{2.460189in}}%
\pgfpathlineto{\pgfqpoint{3.284629in}{2.450169in}}%
\pgfpathlineto{\pgfqpoint{3.276567in}{2.440221in}}%
\pgfpathclose%
\pgfusepath{fill}%
\end{pgfscope}%
\begin{pgfscope}%
\pgfpathrectangle{\pgfqpoint{1.150000in}{0.150000in}}{\pgfqpoint{5.700000in}{5.700000in}}%
\pgfusepath{clip}%
\pgfsetbuttcap%
\pgfsetroundjoin%
\definecolor{currentfill}{rgb}{0.235526,0.309527,0.542944}%
\pgfsetfillcolor{currentfill}%
\pgfsetfillopacity{0.700000}%
\pgfsetlinewidth{0.000000pt}%
\definecolor{currentstroke}{rgb}{0.000000,0.000000,0.000000}%
\pgfsetstrokecolor{currentstroke}%
\pgfsetdash{}{0pt}%
\pgfpathmoveto{\pgfqpoint{5.275373in}{2.857411in}}%
\pgfpathlineto{\pgfqpoint{5.289160in}{2.853691in}}%
\pgfpathlineto{\pgfqpoint{5.302954in}{2.850039in}}%
\pgfpathlineto{\pgfqpoint{5.316755in}{2.846454in}}%
\pgfpathlineto{\pgfqpoint{5.330565in}{2.842937in}}%
\pgfpathlineto{\pgfqpoint{5.338077in}{2.857431in}}%
\pgfpathlineto{\pgfqpoint{5.345594in}{2.872261in}}%
\pgfpathlineto{\pgfqpoint{5.353115in}{2.887436in}}%
\pgfpathlineto{\pgfqpoint{5.360642in}{2.902966in}}%
\pgfpathlineto{\pgfqpoint{5.346850in}{2.906962in}}%
\pgfpathlineto{\pgfqpoint{5.333065in}{2.911025in}}%
\pgfpathlineto{\pgfqpoint{5.319288in}{2.915156in}}%
\pgfpathlineto{\pgfqpoint{5.305519in}{2.919354in}}%
\pgfpathlineto{\pgfqpoint{5.297975in}{2.903339in}}%
\pgfpathlineto{\pgfqpoint{5.290436in}{2.887683in}}%
\pgfpathlineto{\pgfqpoint{5.282903in}{2.872376in}}%
\pgfpathlineto{\pgfqpoint{5.275373in}{2.857411in}}%
\pgfpathclose%
\pgfusepath{fill}%
\end{pgfscope}%
\begin{pgfscope}%
\pgfpathrectangle{\pgfqpoint{1.150000in}{0.150000in}}{\pgfqpoint{5.700000in}{5.700000in}}%
\pgfusepath{clip}%
\pgfsetbuttcap%
\pgfsetroundjoin%
\definecolor{currentfill}{rgb}{0.274128,0.199721,0.498911}%
\pgfsetfillcolor{currentfill}%
\pgfsetfillopacity{0.700000}%
\pgfsetlinewidth{0.000000pt}%
\definecolor{currentstroke}{rgb}{0.000000,0.000000,0.000000}%
\pgfsetstrokecolor{currentstroke}%
\pgfsetdash{}{0pt}%
\pgfpathmoveto{\pgfqpoint{2.610568in}{2.645391in}}%
\pgfpathlineto{\pgfqpoint{2.623972in}{2.634065in}}%
\pgfpathlineto{\pgfqpoint{2.637375in}{2.622876in}}%
\pgfpathlineto{\pgfqpoint{2.650776in}{2.611825in}}%
\pgfpathlineto{\pgfqpoint{2.664176in}{2.600909in}}%
\pgfpathlineto{\pgfqpoint{2.672463in}{2.609995in}}%
\pgfpathlineto{\pgfqpoint{2.680741in}{2.619175in}}%
\pgfpathlineto{\pgfqpoint{2.689012in}{2.628451in}}%
\pgfpathlineto{\pgfqpoint{2.697274in}{2.637825in}}%
\pgfpathlineto{\pgfqpoint{2.683888in}{2.648754in}}%
\pgfpathlineto{\pgfqpoint{2.670501in}{2.659820in}}%
\pgfpathlineto{\pgfqpoint{2.657113in}{2.671022in}}%
\pgfpathlineto{\pgfqpoint{2.643723in}{2.682362in}}%
\pgfpathlineto{\pgfqpoint{2.635447in}{2.672967in}}%
\pgfpathlineto{\pgfqpoint{2.627162in}{2.663674in}}%
\pgfpathlineto{\pgfqpoint{2.618869in}{2.654483in}}%
\pgfpathlineto{\pgfqpoint{2.610568in}{2.645391in}}%
\pgfpathclose%
\pgfusepath{fill}%
\end{pgfscope}%
\begin{pgfscope}%
\pgfpathrectangle{\pgfqpoint{1.150000in}{0.150000in}}{\pgfqpoint{5.700000in}{5.700000in}}%
\pgfusepath{clip}%
\pgfsetbuttcap%
\pgfsetroundjoin%
\definecolor{currentfill}{rgb}{0.281412,0.155834,0.469201}%
\pgfsetfillcolor{currentfill}%
\pgfsetfillopacity{0.700000}%
\pgfsetlinewidth{0.000000pt}%
\definecolor{currentstroke}{rgb}{0.000000,0.000000,0.000000}%
\pgfsetstrokecolor{currentstroke}%
\pgfsetdash{}{0pt}%
\pgfpathmoveto{\pgfqpoint{4.314574in}{2.534670in}}%
\pgfpathlineto{\pgfqpoint{4.328149in}{2.531185in}}%
\pgfpathlineto{\pgfqpoint{4.341731in}{2.527777in}}%
\pgfpathlineto{\pgfqpoint{4.355319in}{2.524445in}}%
\pgfpathlineto{\pgfqpoint{4.368914in}{2.521190in}}%
\pgfpathlineto{\pgfqpoint{4.376631in}{2.531557in}}%
\pgfpathlineto{\pgfqpoint{4.384344in}{2.542047in}}%
\pgfpathlineto{\pgfqpoint{4.392053in}{2.552666in}}%
\pgfpathlineto{\pgfqpoint{4.399757in}{2.563418in}}%
\pgfpathlineto{\pgfqpoint{4.386174in}{2.566952in}}%
\pgfpathlineto{\pgfqpoint{4.372597in}{2.570561in}}%
\pgfpathlineto{\pgfqpoint{4.359026in}{2.574248in}}%
\pgfpathlineto{\pgfqpoint{4.345462in}{2.578011in}}%
\pgfpathlineto{\pgfqpoint{4.337746in}{2.566973in}}%
\pgfpathlineto{\pgfqpoint{4.330026in}{2.556074in}}%
\pgfpathlineto{\pgfqpoint{4.322302in}{2.545308in}}%
\pgfpathlineto{\pgfqpoint{4.314574in}{2.534670in}}%
\pgfpathclose%
\pgfusepath{fill}%
\end{pgfscope}%
\begin{pgfscope}%
\pgfpathrectangle{\pgfqpoint{1.150000in}{0.150000in}}{\pgfqpoint{5.700000in}{5.700000in}}%
\pgfusepath{clip}%
\pgfsetbuttcap%
\pgfsetroundjoin%
\definecolor{currentfill}{rgb}{0.225863,0.330805,0.547314}%
\pgfsetfillcolor{currentfill}%
\pgfsetfillopacity{0.700000}%
\pgfsetlinewidth{0.000000pt}%
\definecolor{currentstroke}{rgb}{0.000000,0.000000,0.000000}%
\pgfsetstrokecolor{currentstroke}%
\pgfsetdash{}{0pt}%
\pgfpathmoveto{\pgfqpoint{5.360642in}{2.902966in}}%
\pgfpathlineto{\pgfqpoint{5.374442in}{2.899037in}}%
\pgfpathlineto{\pgfqpoint{5.388250in}{2.895175in}}%
\pgfpathlineto{\pgfqpoint{5.402065in}{2.891380in}}%
\pgfpathlineto{\pgfqpoint{5.415888in}{2.887653in}}%
\pgfpathlineto{\pgfqpoint{5.423403in}{2.903054in}}%
\pgfpathlineto{\pgfqpoint{5.430924in}{2.918821in}}%
\pgfpathlineto{\pgfqpoint{5.438452in}{2.934966in}}%
\pgfpathlineto{\pgfqpoint{5.445987in}{2.951495in}}%
\pgfpathlineto{\pgfqpoint{5.432181in}{2.955722in}}%
\pgfpathlineto{\pgfqpoint{5.418384in}{2.960015in}}%
\pgfpathlineto{\pgfqpoint{5.404593in}{2.964376in}}%
\pgfpathlineto{\pgfqpoint{5.390811in}{2.968803in}}%
\pgfpathlineto{\pgfqpoint{5.383259in}{2.951768in}}%
\pgfpathlineto{\pgfqpoint{5.375713in}{2.935122in}}%
\pgfpathlineto{\pgfqpoint{5.368175in}{2.918858in}}%
\pgfpathlineto{\pgfqpoint{5.360642in}{2.902966in}}%
\pgfpathclose%
\pgfusepath{fill}%
\end{pgfscope}%
\begin{pgfscope}%
\pgfpathrectangle{\pgfqpoint{1.150000in}{0.150000in}}{\pgfqpoint{5.700000in}{5.700000in}}%
\pgfusepath{clip}%
\pgfsetbuttcap%
\pgfsetroundjoin%
\definecolor{currentfill}{rgb}{0.243113,0.292092,0.538516}%
\pgfsetfillcolor{currentfill}%
\pgfsetfillopacity{0.700000}%
\pgfsetlinewidth{0.000000pt}%
\definecolor{currentstroke}{rgb}{0.000000,0.000000,0.000000}%
\pgfsetstrokecolor{currentstroke}%
\pgfsetdash{}{0pt}%
\pgfpathmoveto{\pgfqpoint{5.190160in}{2.814518in}}%
\pgfpathlineto{\pgfqpoint{5.203932in}{2.810985in}}%
\pgfpathlineto{\pgfqpoint{5.217712in}{2.807520in}}%
\pgfpathlineto{\pgfqpoint{5.231500in}{2.804124in}}%
\pgfpathlineto{\pgfqpoint{5.245295in}{2.800795in}}%
\pgfpathlineto{\pgfqpoint{5.252809in}{2.814479in}}%
\pgfpathlineto{\pgfqpoint{5.260327in}{2.828471in}}%
\pgfpathlineto{\pgfqpoint{5.267848in}{2.842779in}}%
\pgfpathlineto{\pgfqpoint{5.275373in}{2.857411in}}%
\pgfpathlineto{\pgfqpoint{5.261595in}{2.861199in}}%
\pgfpathlineto{\pgfqpoint{5.247824in}{2.865054in}}%
\pgfpathlineto{\pgfqpoint{5.234061in}{2.868978in}}%
\pgfpathlineto{\pgfqpoint{5.220305in}{2.872969in}}%
\pgfpathlineto{\pgfqpoint{5.212764in}{2.857871in}}%
\pgfpathlineto{\pgfqpoint{5.205226in}{2.843102in}}%
\pgfpathlineto{\pgfqpoint{5.197691in}{2.828654in}}%
\pgfpathlineto{\pgfqpoint{5.190160in}{2.814518in}}%
\pgfpathclose%
\pgfusepath{fill}%
\end{pgfscope}%
\begin{pgfscope}%
\pgfpathrectangle{\pgfqpoint{1.150000in}{0.150000in}}{\pgfqpoint{5.700000in}{5.700000in}}%
\pgfusepath{clip}%
\pgfsetbuttcap%
\pgfsetroundjoin%
\definecolor{currentfill}{rgb}{0.183898,0.422383,0.556944}%
\pgfsetfillcolor{currentfill}%
\pgfsetfillopacity{0.700000}%
\pgfsetlinewidth{0.000000pt}%
\definecolor{currentstroke}{rgb}{0.000000,0.000000,0.000000}%
\pgfsetstrokecolor{currentstroke}%
\pgfsetdash{}{0pt}%
\pgfpathmoveto{\pgfqpoint{5.647401in}{3.139051in}}%
\pgfpathlineto{\pgfqpoint{5.661221in}{3.133801in}}%
\pgfpathlineto{\pgfqpoint{5.675048in}{3.128618in}}%
\pgfpathlineto{\pgfqpoint{5.688883in}{3.123500in}}%
\pgfpathlineto{\pgfqpoint{5.702725in}{3.118447in}}%
\pgfpathlineto{\pgfqpoint{5.710339in}{3.139160in}}%
\pgfpathlineto{\pgfqpoint{5.717967in}{3.160384in}}%
\pgfpathlineto{\pgfqpoint{5.725610in}{3.182130in}}%
\pgfpathlineto{\pgfqpoint{5.711781in}{3.187616in}}%
\pgfpathlineto{\pgfqpoint{5.697960in}{3.193168in}}%
\pgfpathlineto{\pgfqpoint{5.684146in}{3.198785in}}%
\pgfpathlineto{\pgfqpoint{5.670340in}{3.204468in}}%
\pgfpathlineto{\pgfqpoint{5.662679in}{3.182138in}}%
\pgfpathlineto{\pgfqpoint{5.655033in}{3.160336in}}%
\pgfpathlineto{\pgfqpoint{5.647401in}{3.139051in}}%
\pgfpathclose%
\pgfusepath{fill}%
\end{pgfscope}%
\begin{pgfscope}%
\pgfpathrectangle{\pgfqpoint{1.150000in}{0.150000in}}{\pgfqpoint{5.700000in}{5.700000in}}%
\pgfusepath{clip}%
\pgfsetbuttcap%
\pgfsetroundjoin%
\definecolor{currentfill}{rgb}{0.282656,0.100196,0.422160}%
\pgfsetfillcolor{currentfill}%
\pgfsetfillopacity{0.700000}%
\pgfsetlinewidth{0.000000pt}%
\definecolor{currentstroke}{rgb}{0.000000,0.000000,0.000000}%
\pgfsetstrokecolor{currentstroke}%
\pgfsetdash{}{0pt}%
\pgfpathmoveto{\pgfqpoint{3.415878in}{2.428186in}}%
\pgfpathlineto{\pgfqpoint{3.429281in}{2.422095in}}%
\pgfpathlineto{\pgfqpoint{3.442687in}{2.416099in}}%
\pgfpathlineto{\pgfqpoint{3.456097in}{2.410199in}}%
\pgfpathlineto{\pgfqpoint{3.469512in}{2.404395in}}%
\pgfpathlineto{\pgfqpoint{3.477518in}{2.414314in}}%
\pgfpathlineto{\pgfqpoint{3.485518in}{2.424299in}}%
\pgfpathlineto{\pgfqpoint{3.493513in}{2.434351in}}%
\pgfpathlineto{\pgfqpoint{3.501501in}{2.444473in}}%
\pgfpathlineto{\pgfqpoint{3.488097in}{2.450395in}}%
\pgfpathlineto{\pgfqpoint{3.474697in}{2.456411in}}%
\pgfpathlineto{\pgfqpoint{3.461300in}{2.462523in}}%
\pgfpathlineto{\pgfqpoint{3.447907in}{2.468732in}}%
\pgfpathlineto{\pgfqpoint{3.439909in}{2.458485in}}%
\pgfpathlineto{\pgfqpoint{3.431904in}{2.448314in}}%
\pgfpathlineto{\pgfqpoint{3.423894in}{2.438215in}}%
\pgfpathlineto{\pgfqpoint{3.415878in}{2.428186in}}%
\pgfpathclose%
\pgfusepath{fill}%
\end{pgfscope}%
\begin{pgfscope}%
\pgfpathrectangle{\pgfqpoint{1.150000in}{0.150000in}}{\pgfqpoint{5.700000in}{5.700000in}}%
\pgfusepath{clip}%
\pgfsetbuttcap%
\pgfsetroundjoin%
\definecolor{currentfill}{rgb}{0.283187,0.125848,0.444960}%
\pgfsetfillcolor{currentfill}%
\pgfsetfillopacity{0.700000}%
\pgfsetlinewidth{0.000000pt}%
\definecolor{currentstroke}{rgb}{0.000000,0.000000,0.000000}%
\pgfsetstrokecolor{currentstroke}%
\pgfsetdash{}{0pt}%
\pgfpathmoveto{\pgfqpoint{2.997621in}{2.483765in}}%
\pgfpathlineto{\pgfqpoint{3.010996in}{2.475397in}}%
\pgfpathlineto{\pgfqpoint{3.024372in}{2.467141in}}%
\pgfpathlineto{\pgfqpoint{3.037750in}{2.458998in}}%
\pgfpathlineto{\pgfqpoint{3.051129in}{2.450967in}}%
\pgfpathlineto{\pgfqpoint{3.059277in}{2.460552in}}%
\pgfpathlineto{\pgfqpoint{3.067419in}{2.470211in}}%
\pgfpathlineto{\pgfqpoint{3.075553in}{2.479945in}}%
\pgfpathlineto{\pgfqpoint{3.083680in}{2.489754in}}%
\pgfpathlineto{\pgfqpoint{3.070312in}{2.497842in}}%
\pgfpathlineto{\pgfqpoint{3.056946in}{2.506041in}}%
\pgfpathlineto{\pgfqpoint{3.043581in}{2.514352in}}%
\pgfpathlineto{\pgfqpoint{3.030218in}{2.522776in}}%
\pgfpathlineto{\pgfqpoint{3.022079in}{2.512903in}}%
\pgfpathlineto{\pgfqpoint{3.013934in}{2.503112in}}%
\pgfpathlineto{\pgfqpoint{3.005781in}{2.493399in}}%
\pgfpathlineto{\pgfqpoint{2.997621in}{2.483765in}}%
\pgfpathclose%
\pgfusepath{fill}%
\end{pgfscope}%
\begin{pgfscope}%
\pgfpathrectangle{\pgfqpoint{1.150000in}{0.150000in}}{\pgfqpoint{5.700000in}{5.700000in}}%
\pgfusepath{clip}%
\pgfsetbuttcap%
\pgfsetroundjoin%
\definecolor{currentfill}{rgb}{0.275191,0.194905,0.496005}%
\pgfsetfillcolor{currentfill}%
\pgfsetfillopacity{0.700000}%
\pgfsetlinewidth{0.000000pt}%
\definecolor{currentstroke}{rgb}{0.000000,0.000000,0.000000}%
\pgfsetstrokecolor{currentstroke}%
\pgfsetdash{}{0pt}%
\pgfpathmoveto{\pgfqpoint{4.624522in}{2.610973in}}%
\pgfpathlineto{\pgfqpoint{4.638172in}{2.607795in}}%
\pgfpathlineto{\pgfqpoint{4.651829in}{2.604689in}}%
\pgfpathlineto{\pgfqpoint{4.665493in}{2.601657in}}%
\pgfpathlineto{\pgfqpoint{4.679165in}{2.598697in}}%
\pgfpathlineto{\pgfqpoint{4.686790in}{2.609577in}}%
\pgfpathlineto{\pgfqpoint{4.694413in}{2.620627in}}%
\pgfpathlineto{\pgfqpoint{4.702034in}{2.631853in}}%
\pgfpathlineto{\pgfqpoint{4.709652in}{2.643262in}}%
\pgfpathlineto{\pgfqpoint{4.695993in}{2.646560in}}%
\pgfpathlineto{\pgfqpoint{4.682342in}{2.649931in}}%
\pgfpathlineto{\pgfqpoint{4.668698in}{2.653375in}}%
\pgfpathlineto{\pgfqpoint{4.655061in}{2.656892in}}%
\pgfpathlineto{\pgfqpoint{4.647430in}{2.645137in}}%
\pgfpathlineto{\pgfqpoint{4.639797in}{2.633570in}}%
\pgfpathlineto{\pgfqpoint{4.632161in}{2.622184in}}%
\pgfpathlineto{\pgfqpoint{4.624522in}{2.610973in}}%
\pgfpathclose%
\pgfusepath{fill}%
\end{pgfscope}%
\begin{pgfscope}%
\pgfpathrectangle{\pgfqpoint{1.150000in}{0.150000in}}{\pgfqpoint{5.700000in}{5.700000in}}%
\pgfusepath{clip}%
\pgfsetbuttcap%
\pgfsetroundjoin%
\definecolor{currentfill}{rgb}{0.250425,0.274290,0.533103}%
\pgfsetfillcolor{currentfill}%
\pgfsetfillopacity{0.700000}%
\pgfsetlinewidth{0.000000pt}%
\definecolor{currentstroke}{rgb}{0.000000,0.000000,0.000000}%
\pgfsetstrokecolor{currentstroke}%
\pgfsetdash{}{0pt}%
\pgfpathmoveto{\pgfqpoint{5.104984in}{2.774000in}}%
\pgfpathlineto{\pgfqpoint{5.118741in}{2.770631in}}%
\pgfpathlineto{\pgfqpoint{5.132506in}{2.767331in}}%
\pgfpathlineto{\pgfqpoint{5.146279in}{2.764100in}}%
\pgfpathlineto{\pgfqpoint{5.160060in}{2.760937in}}%
\pgfpathlineto{\pgfqpoint{5.167582in}{2.773904in}}%
\pgfpathlineto{\pgfqpoint{5.175106in}{2.787152in}}%
\pgfpathlineto{\pgfqpoint{5.182632in}{2.800687in}}%
\pgfpathlineto{\pgfqpoint{5.190160in}{2.814518in}}%
\pgfpathlineto{\pgfqpoint{5.176396in}{2.818120in}}%
\pgfpathlineto{\pgfqpoint{5.162639in}{2.821790in}}%
\pgfpathlineto{\pgfqpoint{5.148890in}{2.825528in}}%
\pgfpathlineto{\pgfqpoint{5.135148in}{2.829335in}}%
\pgfpathlineto{\pgfqpoint{5.127604in}{2.815058in}}%
\pgfpathlineto{\pgfqpoint{5.120062in}{2.801082in}}%
\pgfpathlineto{\pgfqpoint{5.112522in}{2.787399in}}%
\pgfpathlineto{\pgfqpoint{5.104984in}{2.774000in}}%
\pgfpathclose%
\pgfusepath{fill}%
\end{pgfscope}%
\begin{pgfscope}%
\pgfpathrectangle{\pgfqpoint{1.150000in}{0.150000in}}{\pgfqpoint{5.700000in}{5.700000in}}%
\pgfusepath{clip}%
\pgfsetbuttcap%
\pgfsetroundjoin%
\definecolor{currentfill}{rgb}{0.216210,0.351535,0.550627}%
\pgfsetfillcolor{currentfill}%
\pgfsetfillopacity{0.700000}%
\pgfsetlinewidth{0.000000pt}%
\definecolor{currentstroke}{rgb}{0.000000,0.000000,0.000000}%
\pgfsetstrokecolor{currentstroke}%
\pgfsetdash{}{0pt}%
\pgfpathmoveto{\pgfqpoint{5.445987in}{2.951495in}}%
\pgfpathlineto{\pgfqpoint{5.459800in}{2.947335in}}%
\pgfpathlineto{\pgfqpoint{5.473621in}{2.943241in}}%
\pgfpathlineto{\pgfqpoint{5.487449in}{2.939214in}}%
\pgfpathlineto{\pgfqpoint{5.501286in}{2.935254in}}%
\pgfpathlineto{\pgfqpoint{5.508810in}{2.951666in}}%
\pgfpathlineto{\pgfqpoint{5.516342in}{2.968478in}}%
\pgfpathlineto{\pgfqpoint{5.523882in}{2.985698in}}%
\pgfpathlineto{\pgfqpoint{5.531431in}{3.003337in}}%
\pgfpathlineto{\pgfqpoint{5.517613in}{3.007817in}}%
\pgfpathlineto{\pgfqpoint{5.503802in}{3.012363in}}%
\pgfpathlineto{\pgfqpoint{5.489999in}{3.016975in}}%
\pgfpathlineto{\pgfqpoint{5.476204in}{3.021654in}}%
\pgfpathlineto{\pgfqpoint{5.468637in}{3.003489in}}%
\pgfpathlineto{\pgfqpoint{5.461079in}{2.985747in}}%
\pgfpathlineto{\pgfqpoint{5.453529in}{2.968419in}}%
\pgfpathlineto{\pgfqpoint{5.445987in}{2.951495in}}%
\pgfpathclose%
\pgfusepath{fill}%
\end{pgfscope}%
\begin{pgfscope}%
\pgfpathrectangle{\pgfqpoint{1.150000in}{0.150000in}}{\pgfqpoint{5.700000in}{5.700000in}}%
\pgfusepath{clip}%
\pgfsetbuttcap%
\pgfsetroundjoin%
\definecolor{currentfill}{rgb}{0.283091,0.110553,0.431554}%
\pgfsetfillcolor{currentfill}%
\pgfsetfillopacity{0.700000}%
\pgfsetlinewidth{0.000000pt}%
\definecolor{currentstroke}{rgb}{0.000000,0.000000,0.000000}%
\pgfsetstrokecolor{currentstroke}%
\pgfsetdash{}{0pt}%
\pgfpathmoveto{\pgfqpoint{3.779939in}{2.441346in}}%
\pgfpathlineto{\pgfqpoint{3.793401in}{2.436668in}}%
\pgfpathlineto{\pgfqpoint{3.806867in}{2.432077in}}%
\pgfpathlineto{\pgfqpoint{3.820339in}{2.427571in}}%
\pgfpathlineto{\pgfqpoint{3.833816in}{2.423151in}}%
\pgfpathlineto{\pgfqpoint{3.841704in}{2.433181in}}%
\pgfpathlineto{\pgfqpoint{3.849587in}{2.443285in}}%
\pgfpathlineto{\pgfqpoint{3.857465in}{2.453466in}}%
\pgfpathlineto{\pgfqpoint{3.865337in}{2.463729in}}%
\pgfpathlineto{\pgfqpoint{3.851870in}{2.468327in}}%
\pgfpathlineto{\pgfqpoint{3.838408in}{2.473010in}}%
\pgfpathlineto{\pgfqpoint{3.824951in}{2.477779in}}%
\pgfpathlineto{\pgfqpoint{3.811499in}{2.482634in}}%
\pgfpathlineto{\pgfqpoint{3.803617in}{2.472186in}}%
\pgfpathlineto{\pgfqpoint{3.795730in}{2.461825in}}%
\pgfpathlineto{\pgfqpoint{3.787837in}{2.451546in}}%
\pgfpathlineto{\pgfqpoint{3.779939in}{2.441346in}}%
\pgfpathclose%
\pgfusepath{fill}%
\end{pgfscope}%
\begin{pgfscope}%
\pgfpathrectangle{\pgfqpoint{1.150000in}{0.150000in}}{\pgfqpoint{5.700000in}{5.700000in}}%
\pgfusepath{clip}%
\pgfsetbuttcap%
\pgfsetroundjoin%
\definecolor{currentfill}{rgb}{0.283187,0.125848,0.444960}%
\pgfsetfillcolor{currentfill}%
\pgfsetfillopacity{0.700000}%
\pgfsetlinewidth{0.000000pt}%
\definecolor{currentstroke}{rgb}{0.000000,0.000000,0.000000}%
\pgfsetstrokecolor{currentstroke}%
\pgfsetdash{}{0pt}%
\pgfpathmoveto{\pgfqpoint{4.004628in}{2.470339in}}%
\pgfpathlineto{\pgfqpoint{4.018135in}{2.466296in}}%
\pgfpathlineto{\pgfqpoint{4.031648in}{2.462336in}}%
\pgfpathlineto{\pgfqpoint{4.045167in}{2.458457in}}%
\pgfpathlineto{\pgfqpoint{4.058692in}{2.454659in}}%
\pgfpathlineto{\pgfqpoint{4.066508in}{2.464753in}}%
\pgfpathlineto{\pgfqpoint{4.074319in}{2.474936in}}%
\pgfpathlineto{\pgfqpoint{4.082126in}{2.485212in}}%
\pgfpathlineto{\pgfqpoint{4.089927in}{2.495585in}}%
\pgfpathlineto{\pgfqpoint{4.076412in}{2.499601in}}%
\pgfpathlineto{\pgfqpoint{4.062903in}{2.503698in}}%
\pgfpathlineto{\pgfqpoint{4.049400in}{2.507877in}}%
\pgfpathlineto{\pgfqpoint{4.035903in}{2.512137in}}%
\pgfpathlineto{\pgfqpoint{4.028092in}{2.501538in}}%
\pgfpathlineto{\pgfqpoint{4.020275in}{2.491042in}}%
\pgfpathlineto{\pgfqpoint{4.012454in}{2.480643in}}%
\pgfpathlineto{\pgfqpoint{4.004628in}{2.470339in}}%
\pgfpathclose%
\pgfusepath{fill}%
\end{pgfscope}%
\begin{pgfscope}%
\pgfpathrectangle{\pgfqpoint{1.150000in}{0.150000in}}{\pgfqpoint{5.700000in}{5.700000in}}%
\pgfusepath{clip}%
\pgfsetbuttcap%
\pgfsetroundjoin%
\definecolor{currentfill}{rgb}{0.257322,0.256130,0.526563}%
\pgfsetfillcolor{currentfill}%
\pgfsetfillopacity{0.700000}%
\pgfsetlinewidth{0.000000pt}%
\definecolor{currentstroke}{rgb}{0.000000,0.000000,0.000000}%
\pgfsetstrokecolor{currentstroke}%
\pgfsetdash{}{0pt}%
\pgfpathmoveto{\pgfqpoint{5.019829in}{2.735594in}}%
\pgfpathlineto{\pgfqpoint{5.033571in}{2.732367in}}%
\pgfpathlineto{\pgfqpoint{5.047321in}{2.729210in}}%
\pgfpathlineto{\pgfqpoint{5.061079in}{2.726122in}}%
\pgfpathlineto{\pgfqpoint{5.074845in}{2.723103in}}%
\pgfpathlineto{\pgfqpoint{5.082378in}{2.735438in}}%
\pgfpathlineto{\pgfqpoint{5.089912in}{2.748028in}}%
\pgfpathlineto{\pgfqpoint{5.097447in}{2.760879in}}%
\pgfpathlineto{\pgfqpoint{5.104984in}{2.774000in}}%
\pgfpathlineto{\pgfqpoint{5.091234in}{2.777438in}}%
\pgfpathlineto{\pgfqpoint{5.077492in}{2.780945in}}%
\pgfpathlineto{\pgfqpoint{5.063758in}{2.784521in}}%
\pgfpathlineto{\pgfqpoint{5.050031in}{2.788166in}}%
\pgfpathlineto{\pgfqpoint{5.042478in}{2.774619in}}%
\pgfpathlineto{\pgfqpoint{5.034928in}{2.761347in}}%
\pgfpathlineto{\pgfqpoint{5.027378in}{2.748341in}}%
\pgfpathlineto{\pgfqpoint{5.019829in}{2.735594in}}%
\pgfpathclose%
\pgfusepath{fill}%
\end{pgfscope}%
\begin{pgfscope}%
\pgfpathrectangle{\pgfqpoint{1.150000in}{0.150000in}}{\pgfqpoint{5.700000in}{5.700000in}}%
\pgfusepath{clip}%
\pgfsetbuttcap%
\pgfsetroundjoin%
\definecolor{currentfill}{rgb}{0.206756,0.371758,0.553117}%
\pgfsetfillcolor{currentfill}%
\pgfsetfillopacity{0.700000}%
\pgfsetlinewidth{0.000000pt}%
\definecolor{currentstroke}{rgb}{0.000000,0.000000,0.000000}%
\pgfsetstrokecolor{currentstroke}%
\pgfsetdash{}{0pt}%
\pgfpathmoveto{\pgfqpoint{5.531431in}{3.003337in}}%
\pgfpathlineto{\pgfqpoint{5.545257in}{2.998923in}}%
\pgfpathlineto{\pgfqpoint{5.559090in}{2.994576in}}%
\pgfpathlineto{\pgfqpoint{5.572931in}{2.990295in}}%
\pgfpathlineto{\pgfqpoint{5.586780in}{2.986079in}}%
\pgfpathlineto{\pgfqpoint{5.594320in}{3.003614in}}%
\pgfpathlineto{\pgfqpoint{5.601870in}{3.021582in}}%
\pgfpathlineto{\pgfqpoint{5.609430in}{3.039993in}}%
\pgfpathlineto{\pgfqpoint{5.617000in}{3.058857in}}%
\pgfpathlineto{\pgfqpoint{5.603170in}{3.063612in}}%
\pgfpathlineto{\pgfqpoint{5.589347in}{3.068433in}}%
\pgfpathlineto{\pgfqpoint{5.575532in}{3.073320in}}%
\pgfpathlineto{\pgfqpoint{5.561724in}{3.078273in}}%
\pgfpathlineto{\pgfqpoint{5.554135in}{3.058862in}}%
\pgfpathlineto{\pgfqpoint{5.546557in}{3.039909in}}%
\pgfpathlineto{\pgfqpoint{5.538989in}{3.021404in}}%
\pgfpathlineto{\pgfqpoint{5.531431in}{3.003337in}}%
\pgfpathclose%
\pgfusepath{fill}%
\end{pgfscope}%
\begin{pgfscope}%
\pgfpathrectangle{\pgfqpoint{1.150000in}{0.150000in}}{\pgfqpoint{5.700000in}{5.700000in}}%
\pgfusepath{clip}%
\pgfsetbuttcap%
\pgfsetroundjoin%
\definecolor{currentfill}{rgb}{0.282656,0.100196,0.422160}%
\pgfsetfillcolor{currentfill}%
\pgfsetfillopacity{0.700000}%
\pgfsetlinewidth{0.000000pt}%
\definecolor{currentstroke}{rgb}{0.000000,0.000000,0.000000}%
\pgfsetstrokecolor{currentstroke}%
\pgfsetdash{}{0pt}%
\pgfpathmoveto{\pgfqpoint{3.555158in}{2.421726in}}%
\pgfpathlineto{\pgfqpoint{3.568583in}{2.416272in}}%
\pgfpathlineto{\pgfqpoint{3.582011in}{2.410909in}}%
\pgfpathlineto{\pgfqpoint{3.595445in}{2.405638in}}%
\pgfpathlineto{\pgfqpoint{3.608882in}{2.400457in}}%
\pgfpathlineto{\pgfqpoint{3.616845in}{2.410393in}}%
\pgfpathlineto{\pgfqpoint{3.624802in}{2.420394in}}%
\pgfpathlineto{\pgfqpoint{3.632754in}{2.430463in}}%
\pgfpathlineto{\pgfqpoint{3.640699in}{2.440603in}}%
\pgfpathlineto{\pgfqpoint{3.627271in}{2.445921in}}%
\pgfpathlineto{\pgfqpoint{3.613848in}{2.451329in}}%
\pgfpathlineto{\pgfqpoint{3.600429in}{2.456829in}}%
\pgfpathlineto{\pgfqpoint{3.587014in}{2.462420in}}%
\pgfpathlineto{\pgfqpoint{3.579059in}{2.452136in}}%
\pgfpathlineto{\pgfqpoint{3.571098in}{2.441927in}}%
\pgfpathlineto{\pgfqpoint{3.563131in}{2.431792in}}%
\pgfpathlineto{\pgfqpoint{3.555158in}{2.421726in}}%
\pgfpathclose%
\pgfusepath{fill}%
\end{pgfscope}%
\begin{pgfscope}%
\pgfpathrectangle{\pgfqpoint{1.150000in}{0.150000in}}{\pgfqpoint{5.700000in}{5.700000in}}%
\pgfusepath{clip}%
\pgfsetbuttcap%
\pgfsetroundjoin%
\definecolor{currentfill}{rgb}{0.282623,0.140926,0.457517}%
\pgfsetfillcolor{currentfill}%
\pgfsetfillopacity{0.700000}%
\pgfsetlinewidth{0.000000pt}%
\definecolor{currentstroke}{rgb}{0.000000,0.000000,0.000000}%
\pgfsetstrokecolor{currentstroke}%
\pgfsetdash{}{0pt}%
\pgfpathmoveto{\pgfqpoint{2.857848in}{2.516826in}}%
\pgfpathlineto{\pgfqpoint{2.871228in}{2.507556in}}%
\pgfpathlineto{\pgfqpoint{2.884609in}{2.498407in}}%
\pgfpathlineto{\pgfqpoint{2.897991in}{2.489377in}}%
\pgfpathlineto{\pgfqpoint{2.911373in}{2.480465in}}%
\pgfpathlineto{\pgfqpoint{2.919574in}{2.489827in}}%
\pgfpathlineto{\pgfqpoint{2.927768in}{2.499267in}}%
\pgfpathlineto{\pgfqpoint{2.935954in}{2.508787in}}%
\pgfpathlineto{\pgfqpoint{2.944133in}{2.518387in}}%
\pgfpathlineto{\pgfqpoint{2.930763in}{2.527334in}}%
\pgfpathlineto{\pgfqpoint{2.917394in}{2.536399in}}%
\pgfpathlineto{\pgfqpoint{2.904026in}{2.545584in}}%
\pgfpathlineto{\pgfqpoint{2.890658in}{2.554889in}}%
\pgfpathlineto{\pgfqpoint{2.882467in}{2.545246in}}%
\pgfpathlineto{\pgfqpoint{2.874268in}{2.535688in}}%
\pgfpathlineto{\pgfqpoint{2.866062in}{2.526215in}}%
\pgfpathlineto{\pgfqpoint{2.857848in}{2.516826in}}%
\pgfpathclose%
\pgfusepath{fill}%
\end{pgfscope}%
\begin{pgfscope}%
\pgfpathrectangle{\pgfqpoint{1.150000in}{0.150000in}}{\pgfqpoint{5.700000in}{5.700000in}}%
\pgfusepath{clip}%
\pgfsetbuttcap%
\pgfsetroundjoin%
\definecolor{currentfill}{rgb}{0.278012,0.180367,0.486697}%
\pgfsetfillcolor{currentfill}%
\pgfsetfillopacity{0.700000}%
\pgfsetlinewidth{0.000000pt}%
\definecolor{currentstroke}{rgb}{0.000000,0.000000,0.000000}%
\pgfsetstrokecolor{currentstroke}%
\pgfsetdash{}{0pt}%
\pgfpathmoveto{\pgfqpoint{4.539361in}{2.579940in}}%
\pgfpathlineto{\pgfqpoint{4.552995in}{2.576785in}}%
\pgfpathlineto{\pgfqpoint{4.566636in}{2.573705in}}%
\pgfpathlineto{\pgfqpoint{4.580284in}{2.570698in}}%
\pgfpathlineto{\pgfqpoint{4.593939in}{2.567765in}}%
\pgfpathlineto{\pgfqpoint{4.601589in}{2.578333in}}%
\pgfpathlineto{\pgfqpoint{4.609237in}{2.589054in}}%
\pgfpathlineto{\pgfqpoint{4.616881in}{2.599932in}}%
\pgfpathlineto{\pgfqpoint{4.624522in}{2.610973in}}%
\pgfpathlineto{\pgfqpoint{4.610880in}{2.614225in}}%
\pgfpathlineto{\pgfqpoint{4.597244in}{2.617551in}}%
\pgfpathlineto{\pgfqpoint{4.583616in}{2.620950in}}%
\pgfpathlineto{\pgfqpoint{4.569994in}{2.624423in}}%
\pgfpathlineto{\pgfqpoint{4.562340in}{2.613056in}}%
\pgfpathlineto{\pgfqpoint{4.554684in}{2.601857in}}%
\pgfpathlineto{\pgfqpoint{4.547024in}{2.590820in}}%
\pgfpathlineto{\pgfqpoint{4.539361in}{2.579940in}}%
\pgfpathclose%
\pgfusepath{fill}%
\end{pgfscope}%
\begin{pgfscope}%
\pgfpathrectangle{\pgfqpoint{1.150000in}{0.150000in}}{\pgfqpoint{5.700000in}{5.700000in}}%
\pgfusepath{clip}%
\pgfsetbuttcap%
\pgfsetroundjoin%
\definecolor{currentfill}{rgb}{0.282290,0.145912,0.461510}%
\pgfsetfillcolor{currentfill}%
\pgfsetfillopacity{0.700000}%
\pgfsetlinewidth{0.000000pt}%
\definecolor{currentstroke}{rgb}{0.000000,0.000000,0.000000}%
\pgfsetstrokecolor{currentstroke}%
\pgfsetdash{}{0pt}%
\pgfpathmoveto{\pgfqpoint{4.229338in}{2.506989in}}%
\pgfpathlineto{\pgfqpoint{4.242899in}{2.503452in}}%
\pgfpathlineto{\pgfqpoint{4.256465in}{2.499992in}}%
\pgfpathlineto{\pgfqpoint{4.270038in}{2.496610in}}%
\pgfpathlineto{\pgfqpoint{4.283618in}{2.493306in}}%
\pgfpathlineto{\pgfqpoint{4.291363in}{2.503479in}}%
\pgfpathlineto{\pgfqpoint{4.299105in}{2.513761in}}%
\pgfpathlineto{\pgfqpoint{4.306841in}{2.524156in}}%
\pgfpathlineto{\pgfqpoint{4.314574in}{2.534670in}}%
\pgfpathlineto{\pgfqpoint{4.301005in}{2.538233in}}%
\pgfpathlineto{\pgfqpoint{4.287443in}{2.541873in}}%
\pgfpathlineto{\pgfqpoint{4.273887in}{2.545590in}}%
\pgfpathlineto{\pgfqpoint{4.260338in}{2.549386in}}%
\pgfpathlineto{\pgfqpoint{4.252595in}{2.538606in}}%
\pgfpathlineto{\pgfqpoint{4.244847in}{2.527950in}}%
\pgfpathlineto{\pgfqpoint{4.237095in}{2.517413in}}%
\pgfpathlineto{\pgfqpoint{4.229338in}{2.506989in}}%
\pgfpathclose%
\pgfusepath{fill}%
\end{pgfscope}%
\begin{pgfscope}%
\pgfpathrectangle{\pgfqpoint{1.150000in}{0.150000in}}{\pgfqpoint{5.700000in}{5.700000in}}%
\pgfusepath{clip}%
\pgfsetbuttcap%
\pgfsetroundjoin%
\definecolor{currentfill}{rgb}{0.277134,0.185228,0.489898}%
\pgfsetfillcolor{currentfill}%
\pgfsetfillopacity{0.700000}%
\pgfsetlinewidth{0.000000pt}%
\definecolor{currentstroke}{rgb}{0.000000,0.000000,0.000000}%
\pgfsetstrokecolor{currentstroke}%
\pgfsetdash{}{0pt}%
\pgfpathmoveto{\pgfqpoint{2.664176in}{2.600909in}}%
\pgfpathlineto{\pgfqpoint{2.677575in}{2.590128in}}%
\pgfpathlineto{\pgfqpoint{2.690973in}{2.579480in}}%
\pgfpathlineto{\pgfqpoint{2.704369in}{2.568965in}}%
\pgfpathlineto{\pgfqpoint{2.717765in}{2.558580in}}%
\pgfpathlineto{\pgfqpoint{2.726037in}{2.567659in}}%
\pgfpathlineto{\pgfqpoint{2.734302in}{2.576827in}}%
\pgfpathlineto{\pgfqpoint{2.742558in}{2.586087in}}%
\pgfpathlineto{\pgfqpoint{2.750806in}{2.595438in}}%
\pgfpathlineto{\pgfqpoint{2.737424in}{2.605837in}}%
\pgfpathlineto{\pgfqpoint{2.724042in}{2.616367in}}%
\pgfpathlineto{\pgfqpoint{2.710658in}{2.627029in}}%
\pgfpathlineto{\pgfqpoint{2.697274in}{2.637825in}}%
\pgfpathlineto{\pgfqpoint{2.689012in}{2.628451in}}%
\pgfpathlineto{\pgfqpoint{2.680741in}{2.619175in}}%
\pgfpathlineto{\pgfqpoint{2.672463in}{2.609995in}}%
\pgfpathlineto{\pgfqpoint{2.664176in}{2.600909in}}%
\pgfpathclose%
\pgfusepath{fill}%
\end{pgfscope}%
\begin{pgfscope}%
\pgfpathrectangle{\pgfqpoint{1.150000in}{0.150000in}}{\pgfqpoint{5.700000in}{5.700000in}}%
\pgfusepath{clip}%
\pgfsetbuttcap%
\pgfsetroundjoin%
\definecolor{currentfill}{rgb}{0.263663,0.237631,0.518762}%
\pgfsetfillcolor{currentfill}%
\pgfsetfillopacity{0.700000}%
\pgfsetlinewidth{0.000000pt}%
\definecolor{currentstroke}{rgb}{0.000000,0.000000,0.000000}%
\pgfsetstrokecolor{currentstroke}%
\pgfsetdash{}{0pt}%
\pgfpathmoveto{\pgfqpoint{4.934681in}{2.699063in}}%
\pgfpathlineto{\pgfqpoint{4.948408in}{2.695955in}}%
\pgfpathlineto{\pgfqpoint{4.962142in}{2.692918in}}%
\pgfpathlineto{\pgfqpoint{4.975885in}{2.689950in}}%
\pgfpathlineto{\pgfqpoint{4.989635in}{2.687052in}}%
\pgfpathlineto{\pgfqpoint{4.997183in}{2.698836in}}%
\pgfpathlineto{\pgfqpoint{5.004732in}{2.710849in}}%
\pgfpathlineto{\pgfqpoint{5.012280in}{2.723099in}}%
\pgfpathlineto{\pgfqpoint{5.019829in}{2.735594in}}%
\pgfpathlineto{\pgfqpoint{5.006094in}{2.738891in}}%
\pgfpathlineto{\pgfqpoint{4.992367in}{2.742257in}}%
\pgfpathlineto{\pgfqpoint{4.978647in}{2.745693in}}%
\pgfpathlineto{\pgfqpoint{4.964935in}{2.749199in}}%
\pgfpathlineto{\pgfqpoint{4.957371in}{2.736299in}}%
\pgfpathlineto{\pgfqpoint{4.949808in}{2.723647in}}%
\pgfpathlineto{\pgfqpoint{4.942244in}{2.711238in}}%
\pgfpathlineto{\pgfqpoint{4.934681in}{2.699063in}}%
\pgfpathclose%
\pgfusepath{fill}%
\end{pgfscope}%
\begin{pgfscope}%
\pgfpathrectangle{\pgfqpoint{1.150000in}{0.150000in}}{\pgfqpoint{5.700000in}{5.700000in}}%
\pgfusepath{clip}%
\pgfsetbuttcap%
\pgfsetroundjoin%
\definecolor{currentfill}{rgb}{0.194100,0.399323,0.555565}%
\pgfsetfillcolor{currentfill}%
\pgfsetfillopacity{0.700000}%
\pgfsetlinewidth{0.000000pt}%
\definecolor{currentstroke}{rgb}{0.000000,0.000000,0.000000}%
\pgfsetstrokecolor{currentstroke}%
\pgfsetdash{}{0pt}%
\pgfpathmoveto{\pgfqpoint{5.617000in}{3.058857in}}%
\pgfpathlineto{\pgfqpoint{5.630838in}{3.054168in}}%
\pgfpathlineto{\pgfqpoint{5.644684in}{3.049544in}}%
\pgfpathlineto{\pgfqpoint{5.658537in}{3.044986in}}%
\pgfpathlineto{\pgfqpoint{5.672398in}{3.040494in}}%
\pgfpathlineto{\pgfqpoint{5.679962in}{3.059269in}}%
\pgfpathlineto{\pgfqpoint{5.687537in}{3.078512in}}%
\pgfpathlineto{\pgfqpoint{5.695124in}{3.098235in}}%
\pgfpathlineto{\pgfqpoint{5.702725in}{3.118447in}}%
\pgfpathlineto{\pgfqpoint{5.688883in}{3.123500in}}%
\pgfpathlineto{\pgfqpoint{5.675048in}{3.128618in}}%
\pgfpathlineto{\pgfqpoint{5.661221in}{3.133801in}}%
\pgfpathlineto{\pgfqpoint{5.647401in}{3.139051in}}%
\pgfpathlineto{\pgfqpoint{5.639782in}{3.118271in}}%
\pgfpathlineto{\pgfqpoint{5.632176in}{3.097985in}}%
\pgfpathlineto{\pgfqpoint{5.624582in}{3.078184in}}%
\pgfpathlineto{\pgfqpoint{5.617000in}{3.058857in}}%
\pgfpathclose%
\pgfusepath{fill}%
\end{pgfscope}%
\begin{pgfscope}%
\pgfpathrectangle{\pgfqpoint{1.150000in}{0.150000in}}{\pgfqpoint{5.700000in}{5.700000in}}%
\pgfusepath{clip}%
\pgfsetbuttcap%
\pgfsetroundjoin%
\definecolor{currentfill}{rgb}{0.267968,0.223549,0.512008}%
\pgfsetfillcolor{currentfill}%
\pgfsetfillopacity{0.700000}%
\pgfsetlinewidth{0.000000pt}%
\definecolor{currentstroke}{rgb}{0.000000,0.000000,0.000000}%
\pgfsetstrokecolor{currentstroke}%
\pgfsetdash{}{0pt}%
\pgfpathmoveto{\pgfqpoint{4.849527in}{2.664192in}}%
\pgfpathlineto{\pgfqpoint{4.863239in}{2.661181in}}%
\pgfpathlineto{\pgfqpoint{4.876957in}{2.658241in}}%
\pgfpathlineto{\pgfqpoint{4.890684in}{2.655371in}}%
\pgfpathlineto{\pgfqpoint{4.904418in}{2.652572in}}%
\pgfpathlineto{\pgfqpoint{4.911985in}{2.663877in}}%
\pgfpathlineto{\pgfqpoint{4.919551in}{2.675390in}}%
\pgfpathlineto{\pgfqpoint{4.927116in}{2.687116in}}%
\pgfpathlineto{\pgfqpoint{4.934681in}{2.699063in}}%
\pgfpathlineto{\pgfqpoint{4.920961in}{2.702241in}}%
\pgfpathlineto{\pgfqpoint{4.907249in}{2.705489in}}%
\pgfpathlineto{\pgfqpoint{4.893545in}{2.708808in}}%
\pgfpathlineto{\pgfqpoint{4.879848in}{2.712198in}}%
\pgfpathlineto{\pgfqpoint{4.872269in}{2.699865in}}%
\pgfpathlineto{\pgfqpoint{4.864690in}{2.687758in}}%
\pgfpathlineto{\pgfqpoint{4.857109in}{2.675869in}}%
\pgfpathlineto{\pgfqpoint{4.849527in}{2.664192in}}%
\pgfpathclose%
\pgfusepath{fill}%
\end{pgfscope}%
\begin{pgfscope}%
\pgfpathrectangle{\pgfqpoint{1.150000in}{0.150000in}}{\pgfqpoint{5.700000in}{5.700000in}}%
\pgfusepath{clip}%
\pgfsetbuttcap%
\pgfsetroundjoin%
\definecolor{currentfill}{rgb}{0.283197,0.115680,0.436115}%
\pgfsetfillcolor{currentfill}%
\pgfsetfillopacity{0.700000}%
\pgfsetlinewidth{0.000000pt}%
\definecolor{currentstroke}{rgb}{0.000000,0.000000,0.000000}%
\pgfsetstrokecolor{currentstroke}%
\pgfsetdash{}{0pt}%
\pgfpathmoveto{\pgfqpoint{3.919259in}{2.446187in}}%
\pgfpathlineto{\pgfqpoint{3.932754in}{2.442011in}}%
\pgfpathlineto{\pgfqpoint{3.946254in}{2.437919in}}%
\pgfpathlineto{\pgfqpoint{3.959759in}{2.433909in}}%
\pgfpathlineto{\pgfqpoint{3.973271in}{2.429983in}}%
\pgfpathlineto{\pgfqpoint{3.981118in}{2.439950in}}%
\pgfpathlineto{\pgfqpoint{3.988960in}{2.449996in}}%
\pgfpathlineto{\pgfqpoint{3.996796in}{2.460125in}}%
\pgfpathlineto{\pgfqpoint{4.004628in}{2.470339in}}%
\pgfpathlineto{\pgfqpoint{3.991126in}{2.474464in}}%
\pgfpathlineto{\pgfqpoint{3.977631in}{2.478671in}}%
\pgfpathlineto{\pgfqpoint{3.964141in}{2.482961in}}%
\pgfpathlineto{\pgfqpoint{3.950656in}{2.487334in}}%
\pgfpathlineto{\pgfqpoint{3.942815in}{2.476915in}}%
\pgfpathlineto{\pgfqpoint{3.934968in}{2.466586in}}%
\pgfpathlineto{\pgfqpoint{3.927116in}{2.456345in}}%
\pgfpathlineto{\pgfqpoint{3.919259in}{2.446187in}}%
\pgfpathclose%
\pgfusepath{fill}%
\end{pgfscope}%
\begin{pgfscope}%
\pgfpathrectangle{\pgfqpoint{1.150000in}{0.150000in}}{\pgfqpoint{5.700000in}{5.700000in}}%
\pgfusepath{clip}%
\pgfsetbuttcap%
\pgfsetroundjoin%
\definecolor{currentfill}{rgb}{0.282910,0.105393,0.426902}%
\pgfsetfillcolor{currentfill}%
\pgfsetfillopacity{0.700000}%
\pgfsetlinewidth{0.000000pt}%
\definecolor{currentstroke}{rgb}{0.000000,0.000000,0.000000}%
\pgfsetstrokecolor{currentstroke}%
\pgfsetdash{}{0pt}%
\pgfpathmoveto{\pgfqpoint{3.190693in}{2.428967in}}%
\pgfpathlineto{\pgfqpoint{3.204079in}{2.421847in}}%
\pgfpathlineto{\pgfqpoint{3.217469in}{2.414830in}}%
\pgfpathlineto{\pgfqpoint{3.230861in}{2.407917in}}%
\pgfpathlineto{\pgfqpoint{3.244256in}{2.401105in}}%
\pgfpathlineto{\pgfqpoint{3.252343in}{2.410786in}}%
\pgfpathlineto{\pgfqpoint{3.260424in}{2.420531in}}%
\pgfpathlineto{\pgfqpoint{3.268499in}{2.430342in}}%
\pgfpathlineto{\pgfqpoint{3.276567in}{2.440221in}}%
\pgfpathlineto{\pgfqpoint{3.263183in}{2.447109in}}%
\pgfpathlineto{\pgfqpoint{3.249801in}{2.454099in}}%
\pgfpathlineto{\pgfqpoint{3.236423in}{2.461192in}}%
\pgfpathlineto{\pgfqpoint{3.223047in}{2.468388in}}%
\pgfpathlineto{\pgfqpoint{3.214968in}{2.458426in}}%
\pgfpathlineto{\pgfqpoint{3.206883in}{2.448536in}}%
\pgfpathlineto{\pgfqpoint{3.198791in}{2.438717in}}%
\pgfpathlineto{\pgfqpoint{3.190693in}{2.428967in}}%
\pgfpathclose%
\pgfusepath{fill}%
\end{pgfscope}%
\begin{pgfscope}%
\pgfpathrectangle{\pgfqpoint{1.150000in}{0.150000in}}{\pgfqpoint{5.700000in}{5.700000in}}%
\pgfusepath{clip}%
\pgfsetbuttcap%
\pgfsetroundjoin%
\definecolor{currentfill}{rgb}{0.282656,0.100196,0.422160}%
\pgfsetfillcolor{currentfill}%
\pgfsetfillopacity{0.700000}%
\pgfsetlinewidth{0.000000pt}%
\definecolor{currentstroke}{rgb}{0.000000,0.000000,0.000000}%
\pgfsetstrokecolor{currentstroke}%
\pgfsetdash{}{0pt}%
\pgfpathmoveto{\pgfqpoint{3.694457in}{2.420232in}}%
\pgfpathlineto{\pgfqpoint{3.707908in}{2.415362in}}%
\pgfpathlineto{\pgfqpoint{3.721364in}{2.410579in}}%
\pgfpathlineto{\pgfqpoint{3.734825in}{2.405885in}}%
\pgfpathlineto{\pgfqpoint{3.748291in}{2.401278in}}%
\pgfpathlineto{\pgfqpoint{3.756211in}{2.411192in}}%
\pgfpathlineto{\pgfqpoint{3.764126in}{2.421172in}}%
\pgfpathlineto{\pgfqpoint{3.772036in}{2.431223in}}%
\pgfpathlineto{\pgfqpoint{3.779939in}{2.441346in}}%
\pgfpathlineto{\pgfqpoint{3.766483in}{2.446111in}}%
\pgfpathlineto{\pgfqpoint{3.753032in}{2.450963in}}%
\pgfpathlineto{\pgfqpoint{3.739585in}{2.455902in}}%
\pgfpathlineto{\pgfqpoint{3.726144in}{2.460930in}}%
\pgfpathlineto{\pgfqpoint{3.718230in}{2.450642in}}%
\pgfpathlineto{\pgfqpoint{3.710312in}{2.440432in}}%
\pgfpathlineto{\pgfqpoint{3.702387in}{2.430296in}}%
\pgfpathlineto{\pgfqpoint{3.694457in}{2.420232in}}%
\pgfpathclose%
\pgfusepath{fill}%
\end{pgfscope}%
\begin{pgfscope}%
\pgfpathrectangle{\pgfqpoint{1.150000in}{0.150000in}}{\pgfqpoint{5.700000in}{5.700000in}}%
\pgfusepath{clip}%
\pgfsetbuttcap%
\pgfsetroundjoin%
\definecolor{currentfill}{rgb}{0.279574,0.170599,0.479997}%
\pgfsetfillcolor{currentfill}%
\pgfsetfillopacity{0.700000}%
\pgfsetlinewidth{0.000000pt}%
\definecolor{currentstroke}{rgb}{0.000000,0.000000,0.000000}%
\pgfsetstrokecolor{currentstroke}%
\pgfsetdash{}{0pt}%
\pgfpathmoveto{\pgfqpoint{4.454160in}{2.550044in}}%
\pgfpathlineto{\pgfqpoint{4.467778in}{2.546890in}}%
\pgfpathlineto{\pgfqpoint{4.481403in}{2.543810in}}%
\pgfpathlineto{\pgfqpoint{4.495035in}{2.540805in}}%
\pgfpathlineto{\pgfqpoint{4.508674in}{2.537874in}}%
\pgfpathlineto{\pgfqpoint{4.516351in}{2.548184in}}%
\pgfpathlineto{\pgfqpoint{4.524024in}{2.558628in}}%
\pgfpathlineto{\pgfqpoint{4.531694in}{2.569211in}}%
\pgfpathlineto{\pgfqpoint{4.539361in}{2.579940in}}%
\pgfpathlineto{\pgfqpoint{4.525734in}{2.583170in}}%
\pgfpathlineto{\pgfqpoint{4.512114in}{2.586473in}}%
\pgfpathlineto{\pgfqpoint{4.498501in}{2.589851in}}%
\pgfpathlineto{\pgfqpoint{4.484895in}{2.593305in}}%
\pgfpathlineto{\pgfqpoint{4.477217in}{2.582270in}}%
\pgfpathlineto{\pgfqpoint{4.469535in}{2.571385in}}%
\pgfpathlineto{\pgfqpoint{4.461849in}{2.560645in}}%
\pgfpathlineto{\pgfqpoint{4.454160in}{2.550044in}}%
\pgfpathclose%
\pgfusepath{fill}%
\end{pgfscope}%
\begin{pgfscope}%
\pgfpathrectangle{\pgfqpoint{1.150000in}{0.150000in}}{\pgfqpoint{5.700000in}{5.700000in}}%
\pgfusepath{clip}%
\pgfsetbuttcap%
\pgfsetroundjoin%
\definecolor{currentfill}{rgb}{0.282327,0.094955,0.417331}%
\pgfsetfillcolor{currentfill}%
\pgfsetfillopacity{0.700000}%
\pgfsetlinewidth{0.000000pt}%
\definecolor{currentstroke}{rgb}{0.000000,0.000000,0.000000}%
\pgfsetstrokecolor{currentstroke}%
\pgfsetdash{}{0pt}%
\pgfpathmoveto{\pgfqpoint{3.330133in}{2.413682in}}%
\pgfpathlineto{\pgfqpoint{3.343532in}{2.407296in}}%
\pgfpathlineto{\pgfqpoint{3.356935in}{2.401009in}}%
\pgfpathlineto{\pgfqpoint{3.370341in}{2.394820in}}%
\pgfpathlineto{\pgfqpoint{3.383751in}{2.388729in}}%
\pgfpathlineto{\pgfqpoint{3.391792in}{2.398499in}}%
\pgfpathlineto{\pgfqpoint{3.399826in}{2.408330in}}%
\pgfpathlineto{\pgfqpoint{3.407855in}{2.418225in}}%
\pgfpathlineto{\pgfqpoint{3.415878in}{2.428186in}}%
\pgfpathlineto{\pgfqpoint{3.402478in}{2.434374in}}%
\pgfpathlineto{\pgfqpoint{3.389082in}{2.440660in}}%
\pgfpathlineto{\pgfqpoint{3.375690in}{2.447044in}}%
\pgfpathlineto{\pgfqpoint{3.362301in}{2.453526in}}%
\pgfpathlineto{\pgfqpoint{3.354268in}{2.443461in}}%
\pgfpathlineto{\pgfqpoint{3.346229in}{2.433466in}}%
\pgfpathlineto{\pgfqpoint{3.338184in}{2.423541in}}%
\pgfpathlineto{\pgfqpoint{3.330133in}{2.413682in}}%
\pgfpathclose%
\pgfusepath{fill}%
\end{pgfscope}%
\begin{pgfscope}%
\pgfpathrectangle{\pgfqpoint{1.150000in}{0.150000in}}{\pgfqpoint{5.700000in}{5.700000in}}%
\pgfusepath{clip}%
\pgfsetbuttcap%
\pgfsetroundjoin%
\definecolor{currentfill}{rgb}{0.283197,0.115680,0.436115}%
\pgfsetfillcolor{currentfill}%
\pgfsetfillopacity{0.700000}%
\pgfsetlinewidth{0.000000pt}%
\definecolor{currentstroke}{rgb}{0.000000,0.000000,0.000000}%
\pgfsetstrokecolor{currentstroke}%
\pgfsetdash{}{0pt}%
\pgfpathmoveto{\pgfqpoint{3.051129in}{2.450967in}}%
\pgfpathlineto{\pgfqpoint{3.064510in}{2.443045in}}%
\pgfpathlineto{\pgfqpoint{3.077893in}{2.435234in}}%
\pgfpathlineto{\pgfqpoint{3.091278in}{2.427532in}}%
\pgfpathlineto{\pgfqpoint{3.104665in}{2.419937in}}%
\pgfpathlineto{\pgfqpoint{3.112801in}{2.429475in}}%
\pgfpathlineto{\pgfqpoint{3.120931in}{2.439080in}}%
\pgfpathlineto{\pgfqpoint{3.129053in}{2.448755in}}%
\pgfpathlineto{\pgfqpoint{3.137169in}{2.458501in}}%
\pgfpathlineto{\pgfqpoint{3.123794in}{2.466152in}}%
\pgfpathlineto{\pgfqpoint{3.110421in}{2.473910in}}%
\pgfpathlineto{\pgfqpoint{3.097050in}{2.481777in}}%
\pgfpathlineto{\pgfqpoint{3.083680in}{2.489754in}}%
\pgfpathlineto{\pgfqpoint{3.075553in}{2.479945in}}%
\pgfpathlineto{\pgfqpoint{3.067419in}{2.470211in}}%
\pgfpathlineto{\pgfqpoint{3.059277in}{2.460552in}}%
\pgfpathlineto{\pgfqpoint{3.051129in}{2.450967in}}%
\pgfpathclose%
\pgfusepath{fill}%
\end{pgfscope}%
\begin{pgfscope}%
\pgfpathrectangle{\pgfqpoint{1.150000in}{0.150000in}}{\pgfqpoint{5.700000in}{5.700000in}}%
\pgfusepath{clip}%
\pgfsetbuttcap%
\pgfsetroundjoin%
\definecolor{currentfill}{rgb}{0.282884,0.135920,0.453427}%
\pgfsetfillcolor{currentfill}%
\pgfsetfillopacity{0.700000}%
\pgfsetlinewidth{0.000000pt}%
\definecolor{currentstroke}{rgb}{0.000000,0.000000,0.000000}%
\pgfsetstrokecolor{currentstroke}%
\pgfsetdash{}{0pt}%
\pgfpathmoveto{\pgfqpoint{4.144046in}{2.480327in}}%
\pgfpathlineto{\pgfqpoint{4.157592in}{2.476712in}}%
\pgfpathlineto{\pgfqpoint{4.171143in}{2.473177in}}%
\pgfpathlineto{\pgfqpoint{4.184701in}{2.469720in}}%
\pgfpathlineto{\pgfqpoint{4.198266in}{2.466342in}}%
\pgfpathlineto{\pgfqpoint{4.206041in}{2.476356in}}%
\pgfpathlineto{\pgfqpoint{4.213811in}{2.486465in}}%
\pgfpathlineto{\pgfqpoint{4.221577in}{2.496675in}}%
\pgfpathlineto{\pgfqpoint{4.229338in}{2.506989in}}%
\pgfpathlineto{\pgfqpoint{4.215785in}{2.510605in}}%
\pgfpathlineto{\pgfqpoint{4.202238in}{2.514300in}}%
\pgfpathlineto{\pgfqpoint{4.188697in}{2.518074in}}%
\pgfpathlineto{\pgfqpoint{4.175162in}{2.521926in}}%
\pgfpathlineto{\pgfqpoint{4.167390in}{2.511367in}}%
\pgfpathlineto{\pgfqpoint{4.159613in}{2.500916in}}%
\pgfpathlineto{\pgfqpoint{4.151832in}{2.490571in}}%
\pgfpathlineto{\pgfqpoint{4.144046in}{2.480327in}}%
\pgfpathclose%
\pgfusepath{fill}%
\end{pgfscope}%
\begin{pgfscope}%
\pgfpathrectangle{\pgfqpoint{1.150000in}{0.150000in}}{\pgfqpoint{5.700000in}{5.700000in}}%
\pgfusepath{clip}%
\pgfsetbuttcap%
\pgfsetroundjoin%
\definecolor{currentfill}{rgb}{0.280255,0.165693,0.476498}%
\pgfsetfillcolor{currentfill}%
\pgfsetfillopacity{0.700000}%
\pgfsetlinewidth{0.000000pt}%
\definecolor{currentstroke}{rgb}{0.000000,0.000000,0.000000}%
\pgfsetstrokecolor{currentstroke}%
\pgfsetdash{}{0pt}%
\pgfpathmoveto{\pgfqpoint{2.717765in}{2.558580in}}%
\pgfpathlineto{\pgfqpoint{2.731160in}{2.548325in}}%
\pgfpathlineto{\pgfqpoint{2.744555in}{2.538199in}}%
\pgfpathlineto{\pgfqpoint{2.757949in}{2.528201in}}%
\pgfpathlineto{\pgfqpoint{2.771342in}{2.518329in}}%
\pgfpathlineto{\pgfqpoint{2.779600in}{2.527400in}}%
\pgfpathlineto{\pgfqpoint{2.787851in}{2.536557in}}%
\pgfpathlineto{\pgfqpoint{2.796093in}{2.545799in}}%
\pgfpathlineto{\pgfqpoint{2.804328in}{2.555129in}}%
\pgfpathlineto{\pgfqpoint{2.790948in}{2.565015in}}%
\pgfpathlineto{\pgfqpoint{2.777568in}{2.575028in}}%
\pgfpathlineto{\pgfqpoint{2.764187in}{2.585169in}}%
\pgfpathlineto{\pgfqpoint{2.750806in}{2.595438in}}%
\pgfpathlineto{\pgfqpoint{2.742558in}{2.586087in}}%
\pgfpathlineto{\pgfqpoint{2.734302in}{2.576827in}}%
\pgfpathlineto{\pgfqpoint{2.726037in}{2.567659in}}%
\pgfpathlineto{\pgfqpoint{2.717765in}{2.558580in}}%
\pgfpathclose%
\pgfusepath{fill}%
\end{pgfscope}%
\begin{pgfscope}%
\pgfpathrectangle{\pgfqpoint{1.150000in}{0.150000in}}{\pgfqpoint{5.700000in}{5.700000in}}%
\pgfusepath{clip}%
\pgfsetbuttcap%
\pgfsetroundjoin%
\definecolor{currentfill}{rgb}{0.271828,0.209303,0.504434}%
\pgfsetfillcolor{currentfill}%
\pgfsetfillopacity{0.700000}%
\pgfsetlinewidth{0.000000pt}%
\definecolor{currentstroke}{rgb}{0.000000,0.000000,0.000000}%
\pgfsetstrokecolor{currentstroke}%
\pgfsetdash{}{0pt}%
\pgfpathmoveto{\pgfqpoint{4.764358in}{2.630792in}}%
\pgfpathlineto{\pgfqpoint{4.778054in}{2.627855in}}%
\pgfpathlineto{\pgfqpoint{4.791757in}{2.624989in}}%
\pgfpathlineto{\pgfqpoint{4.805467in}{2.622194in}}%
\pgfpathlineto{\pgfqpoint{4.819185in}{2.619471in}}%
\pgfpathlineto{\pgfqpoint{4.826773in}{2.630367in}}%
\pgfpathlineto{\pgfqpoint{4.834360in}{2.641448in}}%
\pgfpathlineto{\pgfqpoint{4.841944in}{2.652720in}}%
\pgfpathlineto{\pgfqpoint{4.849527in}{2.664192in}}%
\pgfpathlineto{\pgfqpoint{4.835824in}{2.667274in}}%
\pgfpathlineto{\pgfqpoint{4.822127in}{2.670428in}}%
\pgfpathlineto{\pgfqpoint{4.808439in}{2.673652in}}%
\pgfpathlineto{\pgfqpoint{4.794757in}{2.676949in}}%
\pgfpathlineto{\pgfqpoint{4.787160in}{2.665111in}}%
\pgfpathlineto{\pgfqpoint{4.779561in}{2.653477in}}%
\pgfpathlineto{\pgfqpoint{4.771961in}{2.642039in}}%
\pgfpathlineto{\pgfqpoint{4.764358in}{2.630792in}}%
\pgfpathclose%
\pgfusepath{fill}%
\end{pgfscope}%
\begin{pgfscope}%
\pgfpathrectangle{\pgfqpoint{1.150000in}{0.150000in}}{\pgfqpoint{5.700000in}{5.700000in}}%
\pgfusepath{clip}%
\pgfsetbuttcap%
\pgfsetroundjoin%
\definecolor{currentfill}{rgb}{0.282327,0.094955,0.417331}%
\pgfsetfillcolor{currentfill}%
\pgfsetfillopacity{0.700000}%
\pgfsetlinewidth{0.000000pt}%
\definecolor{currentstroke}{rgb}{0.000000,0.000000,0.000000}%
\pgfsetstrokecolor{currentstroke}%
\pgfsetdash{}{0pt}%
\pgfpathmoveto{\pgfqpoint{3.469512in}{2.404395in}}%
\pgfpathlineto{\pgfqpoint{3.482930in}{2.398685in}}%
\pgfpathlineto{\pgfqpoint{3.496352in}{2.393068in}}%
\pgfpathlineto{\pgfqpoint{3.509778in}{2.387546in}}%
\pgfpathlineto{\pgfqpoint{3.523208in}{2.382116in}}%
\pgfpathlineto{\pgfqpoint{3.531204in}{2.391926in}}%
\pgfpathlineto{\pgfqpoint{3.539195in}{2.401796in}}%
\pgfpathlineto{\pgfqpoint{3.547179in}{2.411728in}}%
\pgfpathlineto{\pgfqpoint{3.555158in}{2.421726in}}%
\pgfpathlineto{\pgfqpoint{3.541738in}{2.427273in}}%
\pgfpathlineto{\pgfqpoint{3.528322in}{2.432913in}}%
\pgfpathlineto{\pgfqpoint{3.514909in}{2.438646in}}%
\pgfpathlineto{\pgfqpoint{3.501501in}{2.444473in}}%
\pgfpathlineto{\pgfqpoint{3.493513in}{2.434351in}}%
\pgfpathlineto{\pgfqpoint{3.485518in}{2.424299in}}%
\pgfpathlineto{\pgfqpoint{3.477518in}{2.414314in}}%
\pgfpathlineto{\pgfqpoint{3.469512in}{2.404395in}}%
\pgfpathclose%
\pgfusepath{fill}%
\end{pgfscope}%
\begin{pgfscope}%
\pgfpathrectangle{\pgfqpoint{1.150000in}{0.150000in}}{\pgfqpoint{5.700000in}{5.700000in}}%
\pgfusepath{clip}%
\pgfsetbuttcap%
\pgfsetroundjoin%
\definecolor{currentfill}{rgb}{0.185556,0.418570,0.556753}%
\pgfsetfillcolor{currentfill}%
\pgfsetfillopacity{0.700000}%
\pgfsetlinewidth{0.000000pt}%
\definecolor{currentstroke}{rgb}{0.000000,0.000000,0.000000}%
\pgfsetstrokecolor{currentstroke}%
\pgfsetdash{}{0pt}%
\pgfpathmoveto{\pgfqpoint{5.702725in}{3.118447in}}%
\pgfpathlineto{\pgfqpoint{5.716574in}{3.113460in}}%
\pgfpathlineto{\pgfqpoint{5.730432in}{3.108538in}}%
\pgfpathlineto{\pgfqpoint{5.744297in}{3.103681in}}%
\pgfpathlineto{\pgfqpoint{5.758169in}{3.098890in}}%
\pgfpathlineto{\pgfqpoint{5.765764in}{3.119029in}}%
\pgfpathlineto{\pgfqpoint{5.773373in}{3.139675in}}%
\pgfpathlineto{\pgfqpoint{5.780997in}{3.160838in}}%
\pgfpathlineto{\pgfqpoint{5.767139in}{3.166063in}}%
\pgfpathlineto{\pgfqpoint{5.753289in}{3.171353in}}%
\pgfpathlineto{\pgfqpoint{5.739445in}{3.176709in}}%
\pgfpathlineto{\pgfqpoint{5.725610in}{3.182130in}}%
\pgfpathlineto{\pgfqpoint{5.717967in}{3.160384in}}%
\pgfpathlineto{\pgfqpoint{5.710339in}{3.139160in}}%
\pgfpathlineto{\pgfqpoint{5.702725in}{3.118447in}}%
\pgfpathclose%
\pgfusepath{fill}%
\end{pgfscope}%
\begin{pgfscope}%
\pgfpathrectangle{\pgfqpoint{1.150000in}{0.150000in}}{\pgfqpoint{5.700000in}{5.700000in}}%
\pgfusepath{clip}%
\pgfsetbuttcap%
\pgfsetroundjoin%
\definecolor{currentfill}{rgb}{0.283072,0.130895,0.449241}%
\pgfsetfillcolor{currentfill}%
\pgfsetfillopacity{0.700000}%
\pgfsetlinewidth{0.000000pt}%
\definecolor{currentstroke}{rgb}{0.000000,0.000000,0.000000}%
\pgfsetstrokecolor{currentstroke}%
\pgfsetdash{}{0pt}%
\pgfpathmoveto{\pgfqpoint{2.911373in}{2.480465in}}%
\pgfpathlineto{\pgfqpoint{2.924756in}{2.471671in}}%
\pgfpathlineto{\pgfqpoint{2.938140in}{2.462994in}}%
\pgfpathlineto{\pgfqpoint{2.951525in}{2.454432in}}%
\pgfpathlineto{\pgfqpoint{2.964911in}{2.445986in}}%
\pgfpathlineto{\pgfqpoint{2.973099in}{2.455319in}}%
\pgfpathlineto{\pgfqpoint{2.981280in}{2.464726in}}%
\pgfpathlineto{\pgfqpoint{2.989454in}{2.474208in}}%
\pgfpathlineto{\pgfqpoint{2.997621in}{2.483765in}}%
\pgfpathlineto{\pgfqpoint{2.984247in}{2.492248in}}%
\pgfpathlineto{\pgfqpoint{2.970875in}{2.500845in}}%
\pgfpathlineto{\pgfqpoint{2.957504in}{2.509558in}}%
\pgfpathlineto{\pgfqpoint{2.944133in}{2.518387in}}%
\pgfpathlineto{\pgfqpoint{2.935954in}{2.508787in}}%
\pgfpathlineto{\pgfqpoint{2.927768in}{2.499267in}}%
\pgfpathlineto{\pgfqpoint{2.919574in}{2.489827in}}%
\pgfpathlineto{\pgfqpoint{2.911373in}{2.480465in}}%
\pgfpathclose%
\pgfusepath{fill}%
\end{pgfscope}%
\begin{pgfscope}%
\pgfpathrectangle{\pgfqpoint{1.150000in}{0.150000in}}{\pgfqpoint{5.700000in}{5.700000in}}%
\pgfusepath{clip}%
\pgfsetbuttcap%
\pgfsetroundjoin%
\definecolor{currentfill}{rgb}{0.235526,0.309527,0.542944}%
\pgfsetfillcolor{currentfill}%
\pgfsetfillopacity{0.700000}%
\pgfsetlinewidth{0.000000pt}%
\definecolor{currentstroke}{rgb}{0.000000,0.000000,0.000000}%
\pgfsetstrokecolor{currentstroke}%
\pgfsetdash{}{0pt}%
\pgfpathmoveto{\pgfqpoint{5.330565in}{2.842937in}}%
\pgfpathlineto{\pgfqpoint{5.344383in}{2.839487in}}%
\pgfpathlineto{\pgfqpoint{5.358208in}{2.836104in}}%
\pgfpathlineto{\pgfqpoint{5.372042in}{2.832789in}}%
\pgfpathlineto{\pgfqpoint{5.385884in}{2.829540in}}%
\pgfpathlineto{\pgfqpoint{5.393377in}{2.843562in}}%
\pgfpathlineto{\pgfqpoint{5.400876in}{2.857916in}}%
\pgfpathlineto{\pgfqpoint{5.408379in}{2.872610in}}%
\pgfpathlineto{\pgfqpoint{5.415888in}{2.887653in}}%
\pgfpathlineto{\pgfqpoint{5.402065in}{2.891380in}}%
\pgfpathlineto{\pgfqpoint{5.388250in}{2.895175in}}%
\pgfpathlineto{\pgfqpoint{5.374442in}{2.899037in}}%
\pgfpathlineto{\pgfqpoint{5.360642in}{2.902966in}}%
\pgfpathlineto{\pgfqpoint{5.353115in}{2.887436in}}%
\pgfpathlineto{\pgfqpoint{5.345594in}{2.872261in}}%
\pgfpathlineto{\pgfqpoint{5.338077in}{2.857431in}}%
\pgfpathlineto{\pgfqpoint{5.330565in}{2.842937in}}%
\pgfpathclose%
\pgfusepath{fill}%
\end{pgfscope}%
\begin{pgfscope}%
\pgfpathrectangle{\pgfqpoint{1.150000in}{0.150000in}}{\pgfqpoint{5.700000in}{5.700000in}}%
\pgfusepath{clip}%
\pgfsetbuttcap%
\pgfsetroundjoin%
\definecolor{currentfill}{rgb}{0.225863,0.330805,0.547314}%
\pgfsetfillcolor{currentfill}%
\pgfsetfillopacity{0.700000}%
\pgfsetlinewidth{0.000000pt}%
\definecolor{currentstroke}{rgb}{0.000000,0.000000,0.000000}%
\pgfsetstrokecolor{currentstroke}%
\pgfsetdash{}{0pt}%
\pgfpathmoveto{\pgfqpoint{5.415888in}{2.887653in}}%
\pgfpathlineto{\pgfqpoint{5.429720in}{2.883992in}}%
\pgfpathlineto{\pgfqpoint{5.443559in}{2.880397in}}%
\pgfpathlineto{\pgfqpoint{5.457406in}{2.876869in}}%
\pgfpathlineto{\pgfqpoint{5.471262in}{2.873408in}}%
\pgfpathlineto{\pgfqpoint{5.478758in}{2.888317in}}%
\pgfpathlineto{\pgfqpoint{5.486260in}{2.903588in}}%
\pgfpathlineto{\pgfqpoint{5.493769in}{2.919231in}}%
\pgfpathlineto{\pgfqpoint{5.501286in}{2.935254in}}%
\pgfpathlineto{\pgfqpoint{5.487449in}{2.939214in}}%
\pgfpathlineto{\pgfqpoint{5.473621in}{2.943241in}}%
\pgfpathlineto{\pgfqpoint{5.459800in}{2.947335in}}%
\pgfpathlineto{\pgfqpoint{5.445987in}{2.951495in}}%
\pgfpathlineto{\pgfqpoint{5.438452in}{2.934966in}}%
\pgfpathlineto{\pgfqpoint{5.430924in}{2.918821in}}%
\pgfpathlineto{\pgfqpoint{5.423403in}{2.903054in}}%
\pgfpathlineto{\pgfqpoint{5.415888in}{2.887653in}}%
\pgfpathclose%
\pgfusepath{fill}%
\end{pgfscope}%
\begin{pgfscope}%
\pgfpathrectangle{\pgfqpoint{1.150000in}{0.150000in}}{\pgfqpoint{5.700000in}{5.700000in}}%
\pgfusepath{clip}%
\pgfsetbuttcap%
\pgfsetroundjoin%
\definecolor{currentfill}{rgb}{0.243113,0.292092,0.538516}%
\pgfsetfillcolor{currentfill}%
\pgfsetfillopacity{0.700000}%
\pgfsetlinewidth{0.000000pt}%
\definecolor{currentstroke}{rgb}{0.000000,0.000000,0.000000}%
\pgfsetstrokecolor{currentstroke}%
\pgfsetdash{}{0pt}%
\pgfpathmoveto{\pgfqpoint{5.245295in}{2.800795in}}%
\pgfpathlineto{\pgfqpoint{5.259098in}{2.797534in}}%
\pgfpathlineto{\pgfqpoint{5.272910in}{2.794340in}}%
\pgfpathlineto{\pgfqpoint{5.286729in}{2.791215in}}%
\pgfpathlineto{\pgfqpoint{5.300557in}{2.788157in}}%
\pgfpathlineto{\pgfqpoint{5.308054in}{2.801390in}}%
\pgfpathlineto{\pgfqpoint{5.315554in}{2.814925in}}%
\pgfpathlineto{\pgfqpoint{5.323057in}{2.828771in}}%
\pgfpathlineto{\pgfqpoint{5.330565in}{2.842937in}}%
\pgfpathlineto{\pgfqpoint{5.316755in}{2.846454in}}%
\pgfpathlineto{\pgfqpoint{5.302954in}{2.850039in}}%
\pgfpathlineto{\pgfqpoint{5.289160in}{2.853691in}}%
\pgfpathlineto{\pgfqpoint{5.275373in}{2.857411in}}%
\pgfpathlineto{\pgfqpoint{5.267848in}{2.842779in}}%
\pgfpathlineto{\pgfqpoint{5.260327in}{2.828471in}}%
\pgfpathlineto{\pgfqpoint{5.252809in}{2.814479in}}%
\pgfpathlineto{\pgfqpoint{5.245295in}{2.800795in}}%
\pgfpathclose%
\pgfusepath{fill}%
\end{pgfscope}%
\begin{pgfscope}%
\pgfpathrectangle{\pgfqpoint{1.150000in}{0.150000in}}{\pgfqpoint{5.700000in}{5.700000in}}%
\pgfusepath{clip}%
\pgfsetbuttcap%
\pgfsetroundjoin%
\definecolor{currentfill}{rgb}{0.281412,0.155834,0.469201}%
\pgfsetfillcolor{currentfill}%
\pgfsetfillopacity{0.700000}%
\pgfsetlinewidth{0.000000pt}%
\definecolor{currentstroke}{rgb}{0.000000,0.000000,0.000000}%
\pgfsetstrokecolor{currentstroke}%
\pgfsetdash{}{0pt}%
\pgfpathmoveto{\pgfqpoint{4.368914in}{2.521190in}}%
\pgfpathlineto{\pgfqpoint{4.382516in}{2.518012in}}%
\pgfpathlineto{\pgfqpoint{4.396125in}{2.514909in}}%
\pgfpathlineto{\pgfqpoint{4.409741in}{2.511882in}}%
\pgfpathlineto{\pgfqpoint{4.423364in}{2.508931in}}%
\pgfpathlineto{\pgfqpoint{4.431069in}{2.519026in}}%
\pgfpathlineto{\pgfqpoint{4.438770in}{2.529240in}}%
\pgfpathlineto{\pgfqpoint{4.446467in}{2.539578in}}%
\pgfpathlineto{\pgfqpoint{4.454160in}{2.550044in}}%
\pgfpathlineto{\pgfqpoint{4.440549in}{2.553275in}}%
\pgfpathlineto{\pgfqpoint{4.426945in}{2.556580in}}%
\pgfpathlineto{\pgfqpoint{4.413348in}{2.559961in}}%
\pgfpathlineto{\pgfqpoint{4.399757in}{2.563418in}}%
\pgfpathlineto{\pgfqpoint{4.392053in}{2.552666in}}%
\pgfpathlineto{\pgfqpoint{4.384344in}{2.542047in}}%
\pgfpathlineto{\pgfqpoint{4.376631in}{2.531557in}}%
\pgfpathlineto{\pgfqpoint{4.368914in}{2.521190in}}%
\pgfpathclose%
\pgfusepath{fill}%
\end{pgfscope}%
\begin{pgfscope}%
\pgfpathrectangle{\pgfqpoint{1.150000in}{0.150000in}}{\pgfqpoint{5.700000in}{5.700000in}}%
\pgfusepath{clip}%
\pgfsetbuttcap%
\pgfsetroundjoin%
\definecolor{currentfill}{rgb}{0.216210,0.351535,0.550627}%
\pgfsetfillcolor{currentfill}%
\pgfsetfillopacity{0.700000}%
\pgfsetlinewidth{0.000000pt}%
\definecolor{currentstroke}{rgb}{0.000000,0.000000,0.000000}%
\pgfsetstrokecolor{currentstroke}%
\pgfsetdash{}{0pt}%
\pgfpathmoveto{\pgfqpoint{5.501286in}{2.935254in}}%
\pgfpathlineto{\pgfqpoint{5.515130in}{2.931359in}}%
\pgfpathlineto{\pgfqpoint{5.528982in}{2.927531in}}%
\pgfpathlineto{\pgfqpoint{5.542843in}{2.923769in}}%
\pgfpathlineto{\pgfqpoint{5.556711in}{2.920073in}}%
\pgfpathlineto{\pgfqpoint{5.564216in}{2.935974in}}%
\pgfpathlineto{\pgfqpoint{5.571729in}{2.952269in}}%
\pgfpathlineto{\pgfqpoint{5.579250in}{2.968967in}}%
\pgfpathlineto{\pgfqpoint{5.586780in}{2.986079in}}%
\pgfpathlineto{\pgfqpoint{5.572931in}{2.990295in}}%
\pgfpathlineto{\pgfqpoint{5.559090in}{2.994576in}}%
\pgfpathlineto{\pgfqpoint{5.545257in}{2.998923in}}%
\pgfpathlineto{\pgfqpoint{5.531431in}{3.003337in}}%
\pgfpathlineto{\pgfqpoint{5.523882in}{2.985698in}}%
\pgfpathlineto{\pgfqpoint{5.516342in}{2.968478in}}%
\pgfpathlineto{\pgfqpoint{5.508810in}{2.951666in}}%
\pgfpathlineto{\pgfqpoint{5.501286in}{2.935254in}}%
\pgfpathclose%
\pgfusepath{fill}%
\end{pgfscope}%
\begin{pgfscope}%
\pgfpathrectangle{\pgfqpoint{1.150000in}{0.150000in}}{\pgfqpoint{5.700000in}{5.700000in}}%
\pgfusepath{clip}%
\pgfsetbuttcap%
\pgfsetroundjoin%
\definecolor{currentfill}{rgb}{0.282910,0.105393,0.426902}%
\pgfsetfillcolor{currentfill}%
\pgfsetfillopacity{0.700000}%
\pgfsetlinewidth{0.000000pt}%
\definecolor{currentstroke}{rgb}{0.000000,0.000000,0.000000}%
\pgfsetstrokecolor{currentstroke}%
\pgfsetdash{}{0pt}%
\pgfpathmoveto{\pgfqpoint{3.833816in}{2.423151in}}%
\pgfpathlineto{\pgfqpoint{3.847298in}{2.418816in}}%
\pgfpathlineto{\pgfqpoint{3.860786in}{2.414566in}}%
\pgfpathlineto{\pgfqpoint{3.874279in}{2.410401in}}%
\pgfpathlineto{\pgfqpoint{3.887778in}{2.406319in}}%
\pgfpathlineto{\pgfqpoint{3.895656in}{2.416178in}}%
\pgfpathlineto{\pgfqpoint{3.903529in}{2.426107in}}%
\pgfpathlineto{\pgfqpoint{3.911397in}{2.436109in}}%
\pgfpathlineto{\pgfqpoint{3.919259in}{2.446187in}}%
\pgfpathlineto{\pgfqpoint{3.905770in}{2.450446in}}%
\pgfpathlineto{\pgfqpoint{3.892287in}{2.454789in}}%
\pgfpathlineto{\pgfqpoint{3.878809in}{2.459217in}}%
\pgfpathlineto{\pgfqpoint{3.865337in}{2.463729in}}%
\pgfpathlineto{\pgfqpoint{3.857465in}{2.453466in}}%
\pgfpathlineto{\pgfqpoint{3.849587in}{2.443285in}}%
\pgfpathlineto{\pgfqpoint{3.841704in}{2.433181in}}%
\pgfpathlineto{\pgfqpoint{3.833816in}{2.423151in}}%
\pgfpathclose%
\pgfusepath{fill}%
\end{pgfscope}%
\begin{pgfscope}%
\pgfpathrectangle{\pgfqpoint{1.150000in}{0.150000in}}{\pgfqpoint{5.700000in}{5.700000in}}%
\pgfusepath{clip}%
\pgfsetbuttcap%
\pgfsetroundjoin%
\definecolor{currentfill}{rgb}{0.250425,0.274290,0.533103}%
\pgfsetfillcolor{currentfill}%
\pgfsetfillopacity{0.700000}%
\pgfsetlinewidth{0.000000pt}%
\definecolor{currentstroke}{rgb}{0.000000,0.000000,0.000000}%
\pgfsetstrokecolor{currentstroke}%
\pgfsetdash{}{0pt}%
\pgfpathmoveto{\pgfqpoint{5.160060in}{2.760937in}}%
\pgfpathlineto{\pgfqpoint{5.173849in}{2.757843in}}%
\pgfpathlineto{\pgfqpoint{5.187646in}{2.754817in}}%
\pgfpathlineto{\pgfqpoint{5.201451in}{2.751860in}}%
\pgfpathlineto{\pgfqpoint{5.215263in}{2.748970in}}%
\pgfpathlineto{\pgfqpoint{5.222768in}{2.761505in}}%
\pgfpathlineto{\pgfqpoint{5.230275in}{2.774316in}}%
\pgfpathlineto{\pgfqpoint{5.237784in}{2.787410in}}%
\pgfpathlineto{\pgfqpoint{5.245295in}{2.800795in}}%
\pgfpathlineto{\pgfqpoint{5.231500in}{2.804124in}}%
\pgfpathlineto{\pgfqpoint{5.217712in}{2.807520in}}%
\pgfpathlineto{\pgfqpoint{5.203932in}{2.810985in}}%
\pgfpathlineto{\pgfqpoint{5.190160in}{2.814518in}}%
\pgfpathlineto{\pgfqpoint{5.182632in}{2.800687in}}%
\pgfpathlineto{\pgfqpoint{5.175106in}{2.787152in}}%
\pgfpathlineto{\pgfqpoint{5.167582in}{2.773904in}}%
\pgfpathlineto{\pgfqpoint{5.160060in}{2.760937in}}%
\pgfpathclose%
\pgfusepath{fill}%
\end{pgfscope}%
\begin{pgfscope}%
\pgfpathrectangle{\pgfqpoint{1.150000in}{0.150000in}}{\pgfqpoint{5.700000in}{5.700000in}}%
\pgfusepath{clip}%
\pgfsetbuttcap%
\pgfsetroundjoin%
\definecolor{currentfill}{rgb}{0.275191,0.194905,0.496005}%
\pgfsetfillcolor{currentfill}%
\pgfsetfillopacity{0.700000}%
\pgfsetlinewidth{0.000000pt}%
\definecolor{currentstroke}{rgb}{0.000000,0.000000,0.000000}%
\pgfsetstrokecolor{currentstroke}%
\pgfsetdash{}{0pt}%
\pgfpathmoveto{\pgfqpoint{4.679165in}{2.598697in}}%
\pgfpathlineto{\pgfqpoint{4.692844in}{2.595810in}}%
\pgfpathlineto{\pgfqpoint{4.706531in}{2.592996in}}%
\pgfpathlineto{\pgfqpoint{4.720225in}{2.590253in}}%
\pgfpathlineto{\pgfqpoint{4.733926in}{2.587583in}}%
\pgfpathlineto{\pgfqpoint{4.741538in}{2.598131in}}%
\pgfpathlineto{\pgfqpoint{4.749147in}{2.608844in}}%
\pgfpathlineto{\pgfqpoint{4.756754in}{2.619729in}}%
\pgfpathlineto{\pgfqpoint{4.764358in}{2.630792in}}%
\pgfpathlineto{\pgfqpoint{4.750671in}{2.633802in}}%
\pgfpathlineto{\pgfqpoint{4.736990in}{2.636883in}}%
\pgfpathlineto{\pgfqpoint{4.723317in}{2.640036in}}%
\pgfpathlineto{\pgfqpoint{4.709652in}{2.643262in}}%
\pgfpathlineto{\pgfqpoint{4.702034in}{2.631853in}}%
\pgfpathlineto{\pgfqpoint{4.694413in}{2.620627in}}%
\pgfpathlineto{\pgfqpoint{4.686790in}{2.609577in}}%
\pgfpathlineto{\pgfqpoint{4.679165in}{2.598697in}}%
\pgfpathclose%
\pgfusepath{fill}%
\end{pgfscope}%
\begin{pgfscope}%
\pgfpathrectangle{\pgfqpoint{1.150000in}{0.150000in}}{\pgfqpoint{5.700000in}{5.700000in}}%
\pgfusepath{clip}%
\pgfsetbuttcap%
\pgfsetroundjoin%
\definecolor{currentfill}{rgb}{0.283187,0.125848,0.444960}%
\pgfsetfillcolor{currentfill}%
\pgfsetfillopacity{0.700000}%
\pgfsetlinewidth{0.000000pt}%
\definecolor{currentstroke}{rgb}{0.000000,0.000000,0.000000}%
\pgfsetstrokecolor{currentstroke}%
\pgfsetdash{}{0pt}%
\pgfpathmoveto{\pgfqpoint{4.058692in}{2.454659in}}%
\pgfpathlineto{\pgfqpoint{4.072223in}{2.450942in}}%
\pgfpathlineto{\pgfqpoint{4.085760in}{2.447305in}}%
\pgfpathlineto{\pgfqpoint{4.099304in}{2.443749in}}%
\pgfpathlineto{\pgfqpoint{4.112853in}{2.440273in}}%
\pgfpathlineto{\pgfqpoint{4.120659in}{2.450156in}}%
\pgfpathlineto{\pgfqpoint{4.128460in}{2.460124in}}%
\pgfpathlineto{\pgfqpoint{4.136255in}{2.470179in}}%
\pgfpathlineto{\pgfqpoint{4.144046in}{2.480327in}}%
\pgfpathlineto{\pgfqpoint{4.130507in}{2.484021in}}%
\pgfpathlineto{\pgfqpoint{4.116974in}{2.487796in}}%
\pgfpathlineto{\pgfqpoint{4.103448in}{2.491650in}}%
\pgfpathlineto{\pgfqpoint{4.089927in}{2.495585in}}%
\pgfpathlineto{\pgfqpoint{4.082126in}{2.485212in}}%
\pgfpathlineto{\pgfqpoint{4.074319in}{2.474936in}}%
\pgfpathlineto{\pgfqpoint{4.066508in}{2.464753in}}%
\pgfpathlineto{\pgfqpoint{4.058692in}{2.454659in}}%
\pgfpathclose%
\pgfusepath{fill}%
\end{pgfscope}%
\begin{pgfscope}%
\pgfpathrectangle{\pgfqpoint{1.150000in}{0.150000in}}{\pgfqpoint{5.700000in}{5.700000in}}%
\pgfusepath{clip}%
\pgfsetbuttcap%
\pgfsetroundjoin%
\definecolor{currentfill}{rgb}{0.282327,0.094955,0.417331}%
\pgfsetfillcolor{currentfill}%
\pgfsetfillopacity{0.700000}%
\pgfsetlinewidth{0.000000pt}%
\definecolor{currentstroke}{rgb}{0.000000,0.000000,0.000000}%
\pgfsetstrokecolor{currentstroke}%
\pgfsetdash{}{0pt}%
\pgfpathmoveto{\pgfqpoint{3.608882in}{2.400457in}}%
\pgfpathlineto{\pgfqpoint{3.622324in}{2.395367in}}%
\pgfpathlineto{\pgfqpoint{3.635771in}{2.390367in}}%
\pgfpathlineto{\pgfqpoint{3.649223in}{2.385457in}}%
\pgfpathlineto{\pgfqpoint{3.662679in}{2.380635in}}%
\pgfpathlineto{\pgfqpoint{3.670632in}{2.390441in}}%
\pgfpathlineto{\pgfqpoint{3.678579in}{2.400307in}}%
\pgfpathlineto{\pgfqpoint{3.686521in}{2.410237in}}%
\pgfpathlineto{\pgfqpoint{3.694457in}{2.420232in}}%
\pgfpathlineto{\pgfqpoint{3.681010in}{2.425191in}}%
\pgfpathlineto{\pgfqpoint{3.667569in}{2.430239in}}%
\pgfpathlineto{\pgfqpoint{3.654132in}{2.435376in}}%
\pgfpathlineto{\pgfqpoint{3.640699in}{2.440603in}}%
\pgfpathlineto{\pgfqpoint{3.632754in}{2.430463in}}%
\pgfpathlineto{\pgfqpoint{3.624802in}{2.420394in}}%
\pgfpathlineto{\pgfqpoint{3.616845in}{2.410393in}}%
\pgfpathlineto{\pgfqpoint{3.608882in}{2.400457in}}%
\pgfpathclose%
\pgfusepath{fill}%
\end{pgfscope}%
\begin{pgfscope}%
\pgfpathrectangle{\pgfqpoint{1.150000in}{0.150000in}}{\pgfqpoint{5.700000in}{5.700000in}}%
\pgfusepath{clip}%
\pgfsetbuttcap%
\pgfsetroundjoin%
\definecolor{currentfill}{rgb}{0.206756,0.371758,0.553117}%
\pgfsetfillcolor{currentfill}%
\pgfsetfillopacity{0.700000}%
\pgfsetlinewidth{0.000000pt}%
\definecolor{currentstroke}{rgb}{0.000000,0.000000,0.000000}%
\pgfsetstrokecolor{currentstroke}%
\pgfsetdash{}{0pt}%
\pgfpathmoveto{\pgfqpoint{5.586780in}{2.986079in}}%
\pgfpathlineto{\pgfqpoint{5.600637in}{2.981929in}}%
\pgfpathlineto{\pgfqpoint{5.614502in}{2.977846in}}%
\pgfpathlineto{\pgfqpoint{5.628375in}{2.973827in}}%
\pgfpathlineto{\pgfqpoint{5.642256in}{2.969875in}}%
\pgfpathlineto{\pgfqpoint{5.649776in}{2.986878in}}%
\pgfpathlineto{\pgfqpoint{5.657306in}{3.004308in}}%
\pgfpathlineto{\pgfqpoint{5.664847in}{3.022177in}}%
\pgfpathlineto{\pgfqpoint{5.672398in}{3.040494in}}%
\pgfpathlineto{\pgfqpoint{5.658537in}{3.044986in}}%
\pgfpathlineto{\pgfqpoint{5.644684in}{3.049544in}}%
\pgfpathlineto{\pgfqpoint{5.630838in}{3.054168in}}%
\pgfpathlineto{\pgfqpoint{5.617000in}{3.058857in}}%
\pgfpathlineto{\pgfqpoint{5.609430in}{3.039993in}}%
\pgfpathlineto{\pgfqpoint{5.601870in}{3.021582in}}%
\pgfpathlineto{\pgfqpoint{5.594320in}{3.003614in}}%
\pgfpathlineto{\pgfqpoint{5.586780in}{2.986079in}}%
\pgfpathclose%
\pgfusepath{fill}%
\end{pgfscope}%
\begin{pgfscope}%
\pgfpathrectangle{\pgfqpoint{1.150000in}{0.150000in}}{\pgfqpoint{5.700000in}{5.700000in}}%
\pgfusepath{clip}%
\pgfsetbuttcap%
\pgfsetroundjoin%
\definecolor{currentfill}{rgb}{0.257322,0.256130,0.526563}%
\pgfsetfillcolor{currentfill}%
\pgfsetfillopacity{0.700000}%
\pgfsetlinewidth{0.000000pt}%
\definecolor{currentstroke}{rgb}{0.000000,0.000000,0.000000}%
\pgfsetstrokecolor{currentstroke}%
\pgfsetdash{}{0pt}%
\pgfpathmoveto{\pgfqpoint{5.074845in}{2.723103in}}%
\pgfpathlineto{\pgfqpoint{5.088618in}{2.720153in}}%
\pgfpathlineto{\pgfqpoint{5.102400in}{2.717272in}}%
\pgfpathlineto{\pgfqpoint{5.116190in}{2.714460in}}%
\pgfpathlineto{\pgfqpoint{5.129987in}{2.711717in}}%
\pgfpathlineto{\pgfqpoint{5.137504in}{2.723640in}}%
\pgfpathlineto{\pgfqpoint{5.145021in}{2.735813in}}%
\pgfpathlineto{\pgfqpoint{5.152540in}{2.748243in}}%
\pgfpathlineto{\pgfqpoint{5.160060in}{2.760937in}}%
\pgfpathlineto{\pgfqpoint{5.146279in}{2.764100in}}%
\pgfpathlineto{\pgfqpoint{5.132506in}{2.767331in}}%
\pgfpathlineto{\pgfqpoint{5.118741in}{2.770631in}}%
\pgfpathlineto{\pgfqpoint{5.104984in}{2.774000in}}%
\pgfpathlineto{\pgfqpoint{5.097447in}{2.760879in}}%
\pgfpathlineto{\pgfqpoint{5.089912in}{2.748028in}}%
\pgfpathlineto{\pgfqpoint{5.082378in}{2.735438in}}%
\pgfpathlineto{\pgfqpoint{5.074845in}{2.723103in}}%
\pgfpathclose%
\pgfusepath{fill}%
\end{pgfscope}%
\begin{pgfscope}%
\pgfpathrectangle{\pgfqpoint{1.150000in}{0.150000in}}{\pgfqpoint{5.700000in}{5.700000in}}%
\pgfusepath{clip}%
\pgfsetbuttcap%
\pgfsetroundjoin%
\definecolor{currentfill}{rgb}{0.281887,0.150881,0.465405}%
\pgfsetfillcolor{currentfill}%
\pgfsetfillopacity{0.700000}%
\pgfsetlinewidth{0.000000pt}%
\definecolor{currentstroke}{rgb}{0.000000,0.000000,0.000000}%
\pgfsetstrokecolor{currentstroke}%
\pgfsetdash{}{0pt}%
\pgfpathmoveto{\pgfqpoint{2.771342in}{2.518329in}}%
\pgfpathlineto{\pgfqpoint{2.784736in}{2.508582in}}%
\pgfpathlineto{\pgfqpoint{2.798129in}{2.498960in}}%
\pgfpathlineto{\pgfqpoint{2.811523in}{2.489462in}}%
\pgfpathlineto{\pgfqpoint{2.824916in}{2.480086in}}%
\pgfpathlineto{\pgfqpoint{2.833161in}{2.489150in}}%
\pgfpathlineto{\pgfqpoint{2.841397in}{2.498294in}}%
\pgfpathlineto{\pgfqpoint{2.849626in}{2.507519in}}%
\pgfpathlineto{\pgfqpoint{2.857848in}{2.516826in}}%
\pgfpathlineto{\pgfqpoint{2.844468in}{2.526217in}}%
\pgfpathlineto{\pgfqpoint{2.831088in}{2.535731in}}%
\pgfpathlineto{\pgfqpoint{2.817708in}{2.545367in}}%
\pgfpathlineto{\pgfqpoint{2.804328in}{2.555129in}}%
\pgfpathlineto{\pgfqpoint{2.796093in}{2.545799in}}%
\pgfpathlineto{\pgfqpoint{2.787851in}{2.536557in}}%
\pgfpathlineto{\pgfqpoint{2.779600in}{2.527400in}}%
\pgfpathlineto{\pgfqpoint{2.771342in}{2.518329in}}%
\pgfpathclose%
\pgfusepath{fill}%
\end{pgfscope}%
\begin{pgfscope}%
\pgfpathrectangle{\pgfqpoint{1.150000in}{0.150000in}}{\pgfqpoint{5.700000in}{5.700000in}}%
\pgfusepath{clip}%
\pgfsetbuttcap%
\pgfsetroundjoin%
\definecolor{currentfill}{rgb}{0.282327,0.094955,0.417331}%
\pgfsetfillcolor{currentfill}%
\pgfsetfillopacity{0.700000}%
\pgfsetlinewidth{0.000000pt}%
\definecolor{currentstroke}{rgb}{0.000000,0.000000,0.000000}%
\pgfsetstrokecolor{currentstroke}%
\pgfsetdash{}{0pt}%
\pgfpathmoveto{\pgfqpoint{3.244256in}{2.401105in}}%
\pgfpathlineto{\pgfqpoint{3.257653in}{2.394396in}}%
\pgfpathlineto{\pgfqpoint{3.271054in}{2.387788in}}%
\pgfpathlineto{\pgfqpoint{3.284458in}{2.381280in}}%
\pgfpathlineto{\pgfqpoint{3.297864in}{2.374872in}}%
\pgfpathlineto{\pgfqpoint{3.305941in}{2.384484in}}%
\pgfpathlineto{\pgfqpoint{3.314011in}{2.394155in}}%
\pgfpathlineto{\pgfqpoint{3.322075in}{2.403887in}}%
\pgfpathlineto{\pgfqpoint{3.330133in}{2.413682in}}%
\pgfpathlineto{\pgfqpoint{3.316737in}{2.420166in}}%
\pgfpathlineto{\pgfqpoint{3.303344in}{2.426750in}}%
\pgfpathlineto{\pgfqpoint{3.289954in}{2.433435in}}%
\pgfpathlineto{\pgfqpoint{3.276567in}{2.440221in}}%
\pgfpathlineto{\pgfqpoint{3.268499in}{2.430342in}}%
\pgfpathlineto{\pgfqpoint{3.260424in}{2.420531in}}%
\pgfpathlineto{\pgfqpoint{3.252343in}{2.410786in}}%
\pgfpathlineto{\pgfqpoint{3.244256in}{2.401105in}}%
\pgfpathclose%
\pgfusepath{fill}%
\end{pgfscope}%
\begin{pgfscope}%
\pgfpathrectangle{\pgfqpoint{1.150000in}{0.150000in}}{\pgfqpoint{5.700000in}{5.700000in}}%
\pgfusepath{clip}%
\pgfsetbuttcap%
\pgfsetroundjoin%
\definecolor{currentfill}{rgb}{0.282910,0.105393,0.426902}%
\pgfsetfillcolor{currentfill}%
\pgfsetfillopacity{0.700000}%
\pgfsetlinewidth{0.000000pt}%
\definecolor{currentstroke}{rgb}{0.000000,0.000000,0.000000}%
\pgfsetstrokecolor{currentstroke}%
\pgfsetdash{}{0pt}%
\pgfpathmoveto{\pgfqpoint{3.104665in}{2.419937in}}%
\pgfpathlineto{\pgfqpoint{3.118053in}{2.412451in}}%
\pgfpathlineto{\pgfqpoint{3.131444in}{2.405071in}}%
\pgfpathlineto{\pgfqpoint{3.144838in}{2.397797in}}%
\pgfpathlineto{\pgfqpoint{3.158233in}{2.390628in}}%
\pgfpathlineto{\pgfqpoint{3.166358in}{2.400117in}}%
\pgfpathlineto{\pgfqpoint{3.174476in}{2.409669in}}%
\pgfpathlineto{\pgfqpoint{3.182588in}{2.419285in}}%
\pgfpathlineto{\pgfqpoint{3.190693in}{2.428967in}}%
\pgfpathlineto{\pgfqpoint{3.177308in}{2.436192in}}%
\pgfpathlineto{\pgfqpoint{3.163926in}{2.443522in}}%
\pgfpathlineto{\pgfqpoint{3.150547in}{2.450959in}}%
\pgfpathlineto{\pgfqpoint{3.137169in}{2.458501in}}%
\pgfpathlineto{\pgfqpoint{3.129053in}{2.448755in}}%
\pgfpathlineto{\pgfqpoint{3.120931in}{2.439080in}}%
\pgfpathlineto{\pgfqpoint{3.112801in}{2.429475in}}%
\pgfpathlineto{\pgfqpoint{3.104665in}{2.419937in}}%
\pgfpathclose%
\pgfusepath{fill}%
\end{pgfscope}%
\begin{pgfscope}%
\pgfpathrectangle{\pgfqpoint{1.150000in}{0.150000in}}{\pgfqpoint{5.700000in}{5.700000in}}%
\pgfusepath{clip}%
\pgfsetbuttcap%
\pgfsetroundjoin%
\definecolor{currentfill}{rgb}{0.282290,0.145912,0.461510}%
\pgfsetfillcolor{currentfill}%
\pgfsetfillopacity{0.700000}%
\pgfsetlinewidth{0.000000pt}%
\definecolor{currentstroke}{rgb}{0.000000,0.000000,0.000000}%
\pgfsetstrokecolor{currentstroke}%
\pgfsetdash{}{0pt}%
\pgfpathmoveto{\pgfqpoint{4.283618in}{2.493306in}}%
\pgfpathlineto{\pgfqpoint{4.297204in}{2.490079in}}%
\pgfpathlineto{\pgfqpoint{4.310797in}{2.486930in}}%
\pgfpathlineto{\pgfqpoint{4.324397in}{2.483857in}}%
\pgfpathlineto{\pgfqpoint{4.338004in}{2.480861in}}%
\pgfpathlineto{\pgfqpoint{4.345738in}{2.490782in}}%
\pgfpathlineto{\pgfqpoint{4.353468in}{2.500808in}}%
\pgfpathlineto{\pgfqpoint{4.361193in}{2.510942in}}%
\pgfpathlineto{\pgfqpoint{4.368914in}{2.521190in}}%
\pgfpathlineto{\pgfqpoint{4.355319in}{2.524445in}}%
\pgfpathlineto{\pgfqpoint{4.341731in}{2.527777in}}%
\pgfpathlineto{\pgfqpoint{4.328149in}{2.531185in}}%
\pgfpathlineto{\pgfqpoint{4.314574in}{2.534670in}}%
\pgfpathlineto{\pgfqpoint{4.306841in}{2.524156in}}%
\pgfpathlineto{\pgfqpoint{4.299105in}{2.513761in}}%
\pgfpathlineto{\pgfqpoint{4.291363in}{2.503479in}}%
\pgfpathlineto{\pgfqpoint{4.283618in}{2.493306in}}%
\pgfpathclose%
\pgfusepath{fill}%
\end{pgfscope}%
\begin{pgfscope}%
\pgfpathrectangle{\pgfqpoint{1.150000in}{0.150000in}}{\pgfqpoint{5.700000in}{5.700000in}}%
\pgfusepath{clip}%
\pgfsetbuttcap%
\pgfsetroundjoin%
\definecolor{currentfill}{rgb}{0.195860,0.395433,0.555276}%
\pgfsetfillcolor{currentfill}%
\pgfsetfillopacity{0.700000}%
\pgfsetlinewidth{0.000000pt}%
\definecolor{currentstroke}{rgb}{0.000000,0.000000,0.000000}%
\pgfsetstrokecolor{currentstroke}%
\pgfsetdash{}{0pt}%
\pgfpathmoveto{\pgfqpoint{5.672398in}{3.040494in}}%
\pgfpathlineto{\pgfqpoint{5.686267in}{3.036067in}}%
\pgfpathlineto{\pgfqpoint{5.700144in}{3.031705in}}%
\pgfpathlineto{\pgfqpoint{5.714029in}{3.027409in}}%
\pgfpathlineto{\pgfqpoint{5.727922in}{3.023177in}}%
\pgfpathlineto{\pgfqpoint{5.735465in}{3.041400in}}%
\pgfpathlineto{\pgfqpoint{5.743020in}{3.060086in}}%
\pgfpathlineto{\pgfqpoint{5.750588in}{3.079245in}}%
\pgfpathlineto{\pgfqpoint{5.758169in}{3.098890in}}%
\pgfpathlineto{\pgfqpoint{5.744297in}{3.103681in}}%
\pgfpathlineto{\pgfqpoint{5.730432in}{3.108538in}}%
\pgfpathlineto{\pgfqpoint{5.716574in}{3.113460in}}%
\pgfpathlineto{\pgfqpoint{5.702725in}{3.118447in}}%
\pgfpathlineto{\pgfqpoint{5.695124in}{3.098235in}}%
\pgfpathlineto{\pgfqpoint{5.687537in}{3.078512in}}%
\pgfpathlineto{\pgfqpoint{5.679962in}{3.059269in}}%
\pgfpathlineto{\pgfqpoint{5.672398in}{3.040494in}}%
\pgfpathclose%
\pgfusepath{fill}%
\end{pgfscope}%
\begin{pgfscope}%
\pgfpathrectangle{\pgfqpoint{1.150000in}{0.150000in}}{\pgfqpoint{5.700000in}{5.700000in}}%
\pgfusepath{clip}%
\pgfsetbuttcap%
\pgfsetroundjoin%
\definecolor{currentfill}{rgb}{0.277134,0.185228,0.489898}%
\pgfsetfillcolor{currentfill}%
\pgfsetfillopacity{0.700000}%
\pgfsetlinewidth{0.000000pt}%
\definecolor{currentstroke}{rgb}{0.000000,0.000000,0.000000}%
\pgfsetstrokecolor{currentstroke}%
\pgfsetdash{}{0pt}%
\pgfpathmoveto{\pgfqpoint{4.593939in}{2.567765in}}%
\pgfpathlineto{\pgfqpoint{4.607602in}{2.564905in}}%
\pgfpathlineto{\pgfqpoint{4.621272in}{2.562118in}}%
\pgfpathlineto{\pgfqpoint{4.634949in}{2.559405in}}%
\pgfpathlineto{\pgfqpoint{4.648634in}{2.556765in}}%
\pgfpathlineto{\pgfqpoint{4.656272in}{2.567022in}}%
\pgfpathlineto{\pgfqpoint{4.663906in}{2.577426in}}%
\pgfpathlineto{\pgfqpoint{4.671537in}{2.587982in}}%
\pgfpathlineto{\pgfqpoint{4.679165in}{2.598697in}}%
\pgfpathlineto{\pgfqpoint{4.665493in}{2.601657in}}%
\pgfpathlineto{\pgfqpoint{4.651829in}{2.604689in}}%
\pgfpathlineto{\pgfqpoint{4.638172in}{2.607795in}}%
\pgfpathlineto{\pgfqpoint{4.624522in}{2.610973in}}%
\pgfpathlineto{\pgfqpoint{4.616881in}{2.599932in}}%
\pgfpathlineto{\pgfqpoint{4.609237in}{2.589054in}}%
\pgfpathlineto{\pgfqpoint{4.601589in}{2.578333in}}%
\pgfpathlineto{\pgfqpoint{4.593939in}{2.567765in}}%
\pgfpathclose%
\pgfusepath{fill}%
\end{pgfscope}%
\begin{pgfscope}%
\pgfpathrectangle{\pgfqpoint{1.150000in}{0.150000in}}{\pgfqpoint{5.700000in}{5.700000in}}%
\pgfusepath{clip}%
\pgfsetbuttcap%
\pgfsetroundjoin%
\definecolor{currentfill}{rgb}{0.262138,0.242286,0.520837}%
\pgfsetfillcolor{currentfill}%
\pgfsetfillopacity{0.700000}%
\pgfsetlinewidth{0.000000pt}%
\definecolor{currentstroke}{rgb}{0.000000,0.000000,0.000000}%
\pgfsetstrokecolor{currentstroke}%
\pgfsetdash{}{0pt}%
\pgfpathmoveto{\pgfqpoint{4.989635in}{2.687052in}}%
\pgfpathlineto{\pgfqpoint{5.003393in}{2.684225in}}%
\pgfpathlineto{\pgfqpoint{5.017159in}{2.681466in}}%
\pgfpathlineto{\pgfqpoint{5.030933in}{2.678778in}}%
\pgfpathlineto{\pgfqpoint{5.044715in}{2.676158in}}%
\pgfpathlineto{\pgfqpoint{5.052247in}{2.687549in}}%
\pgfpathlineto{\pgfqpoint{5.059780in}{2.699166in}}%
\pgfpathlineto{\pgfqpoint{5.067312in}{2.711015in}}%
\pgfpathlineto{\pgfqpoint{5.074845in}{2.723103in}}%
\pgfpathlineto{\pgfqpoint{5.061079in}{2.726122in}}%
\pgfpathlineto{\pgfqpoint{5.047321in}{2.729210in}}%
\pgfpathlineto{\pgfqpoint{5.033571in}{2.732367in}}%
\pgfpathlineto{\pgfqpoint{5.019829in}{2.735594in}}%
\pgfpathlineto{\pgfqpoint{5.012280in}{2.723099in}}%
\pgfpathlineto{\pgfqpoint{5.004732in}{2.710849in}}%
\pgfpathlineto{\pgfqpoint{4.997183in}{2.698836in}}%
\pgfpathlineto{\pgfqpoint{4.989635in}{2.687052in}}%
\pgfpathclose%
\pgfusepath{fill}%
\end{pgfscope}%
\begin{pgfscope}%
\pgfpathrectangle{\pgfqpoint{1.150000in}{0.150000in}}{\pgfqpoint{5.700000in}{5.700000in}}%
\pgfusepath{clip}%
\pgfsetbuttcap%
\pgfsetroundjoin%
\definecolor{currentfill}{rgb}{0.275191,0.194905,0.496005}%
\pgfsetfillcolor{currentfill}%
\pgfsetfillopacity{0.700000}%
\pgfsetlinewidth{0.000000pt}%
\definecolor{currentstroke}{rgb}{0.000000,0.000000,0.000000}%
\pgfsetstrokecolor{currentstroke}%
\pgfsetdash{}{0pt}%
\pgfpathmoveto{\pgfqpoint{2.577276in}{2.610018in}}%
\pgfpathlineto{\pgfqpoint{2.590696in}{2.598685in}}%
\pgfpathlineto{\pgfqpoint{2.604114in}{2.587490in}}%
\pgfpathlineto{\pgfqpoint{2.617530in}{2.576433in}}%
\pgfpathlineto{\pgfqpoint{2.630945in}{2.565511in}}%
\pgfpathlineto{\pgfqpoint{2.639266in}{2.574220in}}%
\pgfpathlineto{\pgfqpoint{2.647578in}{2.583023in}}%
\pgfpathlineto{\pgfqpoint{2.655881in}{2.591919in}}%
\pgfpathlineto{\pgfqpoint{2.664176in}{2.600909in}}%
\pgfpathlineto{\pgfqpoint{2.650776in}{2.611825in}}%
\pgfpathlineto{\pgfqpoint{2.637375in}{2.622876in}}%
\pgfpathlineto{\pgfqpoint{2.623972in}{2.634065in}}%
\pgfpathlineto{\pgfqpoint{2.610568in}{2.645391in}}%
\pgfpathlineto{\pgfqpoint{2.602258in}{2.636399in}}%
\pgfpathlineto{\pgfqpoint{2.593939in}{2.627507in}}%
\pgfpathlineto{\pgfqpoint{2.585612in}{2.618713in}}%
\pgfpathlineto{\pgfqpoint{2.577276in}{2.610018in}}%
\pgfpathclose%
\pgfusepath{fill}%
\end{pgfscope}%
\begin{pgfscope}%
\pgfpathrectangle{\pgfqpoint{1.150000in}{0.150000in}}{\pgfqpoint{5.700000in}{5.700000in}}%
\pgfusepath{clip}%
\pgfsetbuttcap%
\pgfsetroundjoin%
\definecolor{currentfill}{rgb}{0.281924,0.089666,0.412415}%
\pgfsetfillcolor{currentfill}%
\pgfsetfillopacity{0.700000}%
\pgfsetlinewidth{0.000000pt}%
\definecolor{currentstroke}{rgb}{0.000000,0.000000,0.000000}%
\pgfsetstrokecolor{currentstroke}%
\pgfsetdash{}{0pt}%
\pgfpathmoveto{\pgfqpoint{3.383751in}{2.388729in}}%
\pgfpathlineto{\pgfqpoint{3.397164in}{2.382734in}}%
\pgfpathlineto{\pgfqpoint{3.410581in}{2.376836in}}%
\pgfpathlineto{\pgfqpoint{3.424001in}{2.371033in}}%
\pgfpathlineto{\pgfqpoint{3.437426in}{2.365325in}}%
\pgfpathlineto{\pgfqpoint{3.445456in}{2.375006in}}%
\pgfpathlineto{\pgfqpoint{3.453481in}{2.384743in}}%
\pgfpathlineto{\pgfqpoint{3.461499in}{2.394538in}}%
\pgfpathlineto{\pgfqpoint{3.469512in}{2.404395in}}%
\pgfpathlineto{\pgfqpoint{3.456097in}{2.410199in}}%
\pgfpathlineto{\pgfqpoint{3.442687in}{2.416099in}}%
\pgfpathlineto{\pgfqpoint{3.429281in}{2.422095in}}%
\pgfpathlineto{\pgfqpoint{3.415878in}{2.428186in}}%
\pgfpathlineto{\pgfqpoint{3.407855in}{2.418225in}}%
\pgfpathlineto{\pgfqpoint{3.399826in}{2.408330in}}%
\pgfpathlineto{\pgfqpoint{3.391792in}{2.398499in}}%
\pgfpathlineto{\pgfqpoint{3.383751in}{2.388729in}}%
\pgfpathclose%
\pgfusepath{fill}%
\end{pgfscope}%
\begin{pgfscope}%
\pgfpathrectangle{\pgfqpoint{1.150000in}{0.150000in}}{\pgfqpoint{5.700000in}{5.700000in}}%
\pgfusepath{clip}%
\pgfsetbuttcap%
\pgfsetroundjoin%
\definecolor{currentfill}{rgb}{0.283197,0.115680,0.436115}%
\pgfsetfillcolor{currentfill}%
\pgfsetfillopacity{0.700000}%
\pgfsetlinewidth{0.000000pt}%
\definecolor{currentstroke}{rgb}{0.000000,0.000000,0.000000}%
\pgfsetstrokecolor{currentstroke}%
\pgfsetdash{}{0pt}%
\pgfpathmoveto{\pgfqpoint{2.964911in}{2.445986in}}%
\pgfpathlineto{\pgfqpoint{2.978298in}{2.437653in}}%
\pgfpathlineto{\pgfqpoint{2.991686in}{2.429433in}}%
\pgfpathlineto{\pgfqpoint{3.005076in}{2.421326in}}%
\pgfpathlineto{\pgfqpoint{3.018468in}{2.413330in}}%
\pgfpathlineto{\pgfqpoint{3.026644in}{2.422635in}}%
\pgfpathlineto{\pgfqpoint{3.034813in}{2.432009in}}%
\pgfpathlineto{\pgfqpoint{3.042974in}{2.441453in}}%
\pgfpathlineto{\pgfqpoint{3.051129in}{2.450967in}}%
\pgfpathlineto{\pgfqpoint{3.037750in}{2.458998in}}%
\pgfpathlineto{\pgfqpoint{3.024372in}{2.467141in}}%
\pgfpathlineto{\pgfqpoint{3.010996in}{2.475397in}}%
\pgfpathlineto{\pgfqpoint{2.997621in}{2.483765in}}%
\pgfpathlineto{\pgfqpoint{2.989454in}{2.474208in}}%
\pgfpathlineto{\pgfqpoint{2.981280in}{2.464726in}}%
\pgfpathlineto{\pgfqpoint{2.973099in}{2.455319in}}%
\pgfpathlineto{\pgfqpoint{2.964911in}{2.445986in}}%
\pgfpathclose%
\pgfusepath{fill}%
\end{pgfscope}%
\begin{pgfscope}%
\pgfpathrectangle{\pgfqpoint{1.150000in}{0.150000in}}{\pgfqpoint{5.700000in}{5.700000in}}%
\pgfusepath{clip}%
\pgfsetbuttcap%
\pgfsetroundjoin%
\definecolor{currentfill}{rgb}{0.282656,0.100196,0.422160}%
\pgfsetfillcolor{currentfill}%
\pgfsetfillopacity{0.700000}%
\pgfsetlinewidth{0.000000pt}%
\definecolor{currentstroke}{rgb}{0.000000,0.000000,0.000000}%
\pgfsetstrokecolor{currentstroke}%
\pgfsetdash{}{0pt}%
\pgfpathmoveto{\pgfqpoint{3.748291in}{2.401278in}}%
\pgfpathlineto{\pgfqpoint{3.761762in}{2.396757in}}%
\pgfpathlineto{\pgfqpoint{3.775238in}{2.392323in}}%
\pgfpathlineto{\pgfqpoint{3.788720in}{2.387976in}}%
\pgfpathlineto{\pgfqpoint{3.802207in}{2.383713in}}%
\pgfpathlineto{\pgfqpoint{3.810117in}{2.393477in}}%
\pgfpathlineto{\pgfqpoint{3.818022in}{2.403302in}}%
\pgfpathlineto{\pgfqpoint{3.825922in}{2.413193in}}%
\pgfpathlineto{\pgfqpoint{3.833816in}{2.423151in}}%
\pgfpathlineto{\pgfqpoint{3.820339in}{2.427571in}}%
\pgfpathlineto{\pgfqpoint{3.806867in}{2.432077in}}%
\pgfpathlineto{\pgfqpoint{3.793401in}{2.436668in}}%
\pgfpathlineto{\pgfqpoint{3.779939in}{2.441346in}}%
\pgfpathlineto{\pgfqpoint{3.772036in}{2.431223in}}%
\pgfpathlineto{\pgfqpoint{3.764126in}{2.421172in}}%
\pgfpathlineto{\pgfqpoint{3.756211in}{2.411192in}}%
\pgfpathlineto{\pgfqpoint{3.748291in}{2.401278in}}%
\pgfpathclose%
\pgfusepath{fill}%
\end{pgfscope}%
\begin{pgfscope}%
\pgfpathrectangle{\pgfqpoint{1.150000in}{0.150000in}}{\pgfqpoint{5.700000in}{5.700000in}}%
\pgfusepath{clip}%
\pgfsetbuttcap%
\pgfsetroundjoin%
\definecolor{currentfill}{rgb}{0.283197,0.115680,0.436115}%
\pgfsetfillcolor{currentfill}%
\pgfsetfillopacity{0.700000}%
\pgfsetlinewidth{0.000000pt}%
\definecolor{currentstroke}{rgb}{0.000000,0.000000,0.000000}%
\pgfsetstrokecolor{currentstroke}%
\pgfsetdash{}{0pt}%
\pgfpathmoveto{\pgfqpoint{3.973271in}{2.429983in}}%
\pgfpathlineto{\pgfqpoint{3.986788in}{2.426138in}}%
\pgfpathlineto{\pgfqpoint{4.000312in}{2.422375in}}%
\pgfpathlineto{\pgfqpoint{4.013841in}{2.418695in}}%
\pgfpathlineto{\pgfqpoint{4.027376in}{2.415095in}}%
\pgfpathlineto{\pgfqpoint{4.035213in}{2.424872in}}%
\pgfpathlineto{\pgfqpoint{4.043044in}{2.434722in}}%
\pgfpathlineto{\pgfqpoint{4.050871in}{2.444650in}}%
\pgfpathlineto{\pgfqpoint{4.058692in}{2.454659in}}%
\pgfpathlineto{\pgfqpoint{4.045167in}{2.458457in}}%
\pgfpathlineto{\pgfqpoint{4.031648in}{2.462336in}}%
\pgfpathlineto{\pgfqpoint{4.018135in}{2.466296in}}%
\pgfpathlineto{\pgfqpoint{4.004628in}{2.470339in}}%
\pgfpathlineto{\pgfqpoint{3.996796in}{2.460125in}}%
\pgfpathlineto{\pgfqpoint{3.988960in}{2.449996in}}%
\pgfpathlineto{\pgfqpoint{3.981118in}{2.439950in}}%
\pgfpathlineto{\pgfqpoint{3.973271in}{2.429983in}}%
\pgfpathclose%
\pgfusepath{fill}%
\end{pgfscope}%
\begin{pgfscope}%
\pgfpathrectangle{\pgfqpoint{1.150000in}{0.150000in}}{\pgfqpoint{5.700000in}{5.700000in}}%
\pgfusepath{clip}%
\pgfsetbuttcap%
\pgfsetroundjoin%
\definecolor{currentfill}{rgb}{0.266580,0.228262,0.514349}%
\pgfsetfillcolor{currentfill}%
\pgfsetfillopacity{0.700000}%
\pgfsetlinewidth{0.000000pt}%
\definecolor{currentstroke}{rgb}{0.000000,0.000000,0.000000}%
\pgfsetstrokecolor{currentstroke}%
\pgfsetdash{}{0pt}%
\pgfpathmoveto{\pgfqpoint{4.904418in}{2.652572in}}%
\pgfpathlineto{\pgfqpoint{4.918160in}{2.649844in}}%
\pgfpathlineto{\pgfqpoint{4.931910in}{2.647185in}}%
\pgfpathlineto{\pgfqpoint{4.945668in}{2.644597in}}%
\pgfpathlineto{\pgfqpoint{4.959434in}{2.642079in}}%
\pgfpathlineto{\pgfqpoint{4.966986in}{2.653012in}}%
\pgfpathlineto{\pgfqpoint{4.974536in}{2.664148in}}%
\pgfpathlineto{\pgfqpoint{4.982086in}{2.675492in}}%
\pgfpathlineto{\pgfqpoint{4.989635in}{2.687052in}}%
\pgfpathlineto{\pgfqpoint{4.975885in}{2.689950in}}%
\pgfpathlineto{\pgfqpoint{4.962142in}{2.692918in}}%
\pgfpathlineto{\pgfqpoint{4.948408in}{2.695955in}}%
\pgfpathlineto{\pgfqpoint{4.934681in}{2.699063in}}%
\pgfpathlineto{\pgfqpoint{4.927116in}{2.687116in}}%
\pgfpathlineto{\pgfqpoint{4.919551in}{2.675390in}}%
\pgfpathlineto{\pgfqpoint{4.911985in}{2.663877in}}%
\pgfpathlineto{\pgfqpoint{4.904418in}{2.652572in}}%
\pgfpathclose%
\pgfusepath{fill}%
\end{pgfscope}%
\begin{pgfscope}%
\pgfpathrectangle{\pgfqpoint{1.150000in}{0.150000in}}{\pgfqpoint{5.700000in}{5.700000in}}%
\pgfusepath{clip}%
\pgfsetbuttcap%
\pgfsetroundjoin%
\definecolor{currentfill}{rgb}{0.279574,0.170599,0.479997}%
\pgfsetfillcolor{currentfill}%
\pgfsetfillopacity{0.700000}%
\pgfsetlinewidth{0.000000pt}%
\definecolor{currentstroke}{rgb}{0.000000,0.000000,0.000000}%
\pgfsetstrokecolor{currentstroke}%
\pgfsetdash{}{0pt}%
\pgfpathmoveto{\pgfqpoint{4.508674in}{2.537874in}}%
\pgfpathlineto{\pgfqpoint{4.522320in}{2.535019in}}%
\pgfpathlineto{\pgfqpoint{4.535974in}{2.532237in}}%
\pgfpathlineto{\pgfqpoint{4.549635in}{2.529529in}}%
\pgfpathlineto{\pgfqpoint{4.563303in}{2.526895in}}%
\pgfpathlineto{\pgfqpoint{4.570967in}{2.536913in}}%
\pgfpathlineto{\pgfqpoint{4.578628in}{2.547060in}}%
\pgfpathlineto{\pgfqpoint{4.586285in}{2.557342in}}%
\pgfpathlineto{\pgfqpoint{4.593939in}{2.567765in}}%
\pgfpathlineto{\pgfqpoint{4.580284in}{2.570698in}}%
\pgfpathlineto{\pgfqpoint{4.566636in}{2.573705in}}%
\pgfpathlineto{\pgfqpoint{4.552995in}{2.576785in}}%
\pgfpathlineto{\pgfqpoint{4.539361in}{2.579940in}}%
\pgfpathlineto{\pgfqpoint{4.531694in}{2.569211in}}%
\pgfpathlineto{\pgfqpoint{4.524024in}{2.558628in}}%
\pgfpathlineto{\pgfqpoint{4.516351in}{2.548184in}}%
\pgfpathlineto{\pgfqpoint{4.508674in}{2.537874in}}%
\pgfpathclose%
\pgfusepath{fill}%
\end{pgfscope}%
\begin{pgfscope}%
\pgfpathrectangle{\pgfqpoint{1.150000in}{0.150000in}}{\pgfqpoint{5.700000in}{5.700000in}}%
\pgfusepath{clip}%
\pgfsetbuttcap%
\pgfsetroundjoin%
\definecolor{currentfill}{rgb}{0.281924,0.089666,0.412415}%
\pgfsetfillcolor{currentfill}%
\pgfsetfillopacity{0.700000}%
\pgfsetlinewidth{0.000000pt}%
\definecolor{currentstroke}{rgb}{0.000000,0.000000,0.000000}%
\pgfsetstrokecolor{currentstroke}%
\pgfsetdash{}{0pt}%
\pgfpathmoveto{\pgfqpoint{3.523208in}{2.382116in}}%
\pgfpathlineto{\pgfqpoint{3.536643in}{2.376778in}}%
\pgfpathlineto{\pgfqpoint{3.550081in}{2.371533in}}%
\pgfpathlineto{\pgfqpoint{3.563525in}{2.366379in}}%
\pgfpathlineto{\pgfqpoint{3.576972in}{2.361316in}}%
\pgfpathlineto{\pgfqpoint{3.584959in}{2.371016in}}%
\pgfpathlineto{\pgfqpoint{3.592939in}{2.380771in}}%
\pgfpathlineto{\pgfqpoint{3.600913in}{2.390584in}}%
\pgfpathlineto{\pgfqpoint{3.608882in}{2.400457in}}%
\pgfpathlineto{\pgfqpoint{3.595445in}{2.405638in}}%
\pgfpathlineto{\pgfqpoint{3.582011in}{2.410909in}}%
\pgfpathlineto{\pgfqpoint{3.568583in}{2.416272in}}%
\pgfpathlineto{\pgfqpoint{3.555158in}{2.421726in}}%
\pgfpathlineto{\pgfqpoint{3.547179in}{2.411728in}}%
\pgfpathlineto{\pgfqpoint{3.539195in}{2.401796in}}%
\pgfpathlineto{\pgfqpoint{3.531204in}{2.391926in}}%
\pgfpathlineto{\pgfqpoint{3.523208in}{2.382116in}}%
\pgfpathclose%
\pgfusepath{fill}%
\end{pgfscope}%
\begin{pgfscope}%
\pgfpathrectangle{\pgfqpoint{1.150000in}{0.150000in}}{\pgfqpoint{5.700000in}{5.700000in}}%
\pgfusepath{clip}%
\pgfsetbuttcap%
\pgfsetroundjoin%
\definecolor{currentfill}{rgb}{0.282884,0.135920,0.453427}%
\pgfsetfillcolor{currentfill}%
\pgfsetfillopacity{0.700000}%
\pgfsetlinewidth{0.000000pt}%
\definecolor{currentstroke}{rgb}{0.000000,0.000000,0.000000}%
\pgfsetstrokecolor{currentstroke}%
\pgfsetdash{}{0pt}%
\pgfpathmoveto{\pgfqpoint{4.198266in}{2.466342in}}%
\pgfpathlineto{\pgfqpoint{4.211837in}{2.463043in}}%
\pgfpathlineto{\pgfqpoint{4.225414in}{2.459822in}}%
\pgfpathlineto{\pgfqpoint{4.238998in}{2.456679in}}%
\pgfpathlineto{\pgfqpoint{4.252589in}{2.453614in}}%
\pgfpathlineto{\pgfqpoint{4.260353in}{2.463396in}}%
\pgfpathlineto{\pgfqpoint{4.268113in}{2.473269in}}%
\pgfpathlineto{\pgfqpoint{4.275868in}{2.483238in}}%
\pgfpathlineto{\pgfqpoint{4.283618in}{2.493306in}}%
\pgfpathlineto{\pgfqpoint{4.270038in}{2.496610in}}%
\pgfpathlineto{\pgfqpoint{4.256465in}{2.499992in}}%
\pgfpathlineto{\pgfqpoint{4.242899in}{2.503452in}}%
\pgfpathlineto{\pgfqpoint{4.229338in}{2.506989in}}%
\pgfpathlineto{\pgfqpoint{4.221577in}{2.496675in}}%
\pgfpathlineto{\pgfqpoint{4.213811in}{2.486465in}}%
\pgfpathlineto{\pgfqpoint{4.206041in}{2.476356in}}%
\pgfpathlineto{\pgfqpoint{4.198266in}{2.466342in}}%
\pgfpathclose%
\pgfusepath{fill}%
\end{pgfscope}%
\begin{pgfscope}%
\pgfpathrectangle{\pgfqpoint{1.150000in}{0.150000in}}{\pgfqpoint{5.700000in}{5.700000in}}%
\pgfusepath{clip}%
\pgfsetbuttcap%
\pgfsetroundjoin%
\definecolor{currentfill}{rgb}{0.185556,0.418570,0.556753}%
\pgfsetfillcolor{currentfill}%
\pgfsetfillopacity{0.700000}%
\pgfsetlinewidth{0.000000pt}%
\definecolor{currentstroke}{rgb}{0.000000,0.000000,0.000000}%
\pgfsetstrokecolor{currentstroke}%
\pgfsetdash{}{0pt}%
\pgfpathmoveto{\pgfqpoint{5.758169in}{3.098890in}}%
\pgfpathlineto{\pgfqpoint{5.772050in}{3.094163in}}%
\pgfpathlineto{\pgfqpoint{5.785938in}{3.089502in}}%
\pgfpathlineto{\pgfqpoint{5.799834in}{3.084905in}}%
\pgfpathlineto{\pgfqpoint{5.813738in}{3.080373in}}%
\pgfpathlineto{\pgfqpoint{5.821312in}{3.099939in}}%
\pgfpathlineto{\pgfqpoint{5.828901in}{3.120007in}}%
\pgfpathlineto{\pgfqpoint{5.836505in}{3.140586in}}%
\pgfpathlineto{\pgfqpoint{5.822617in}{3.145551in}}%
\pgfpathlineto{\pgfqpoint{5.808736in}{3.150582in}}%
\pgfpathlineto{\pgfqpoint{5.794863in}{3.155677in}}%
\pgfpathlineto{\pgfqpoint{5.780997in}{3.160838in}}%
\pgfpathlineto{\pgfqpoint{5.773373in}{3.139675in}}%
\pgfpathlineto{\pgfqpoint{5.765764in}{3.119029in}}%
\pgfpathlineto{\pgfqpoint{5.758169in}{3.098890in}}%
\pgfpathclose%
\pgfusepath{fill}%
\end{pgfscope}%
\begin{pgfscope}%
\pgfpathrectangle{\pgfqpoint{1.150000in}{0.150000in}}{\pgfqpoint{5.700000in}{5.700000in}}%
\pgfusepath{clip}%
\pgfsetbuttcap%
\pgfsetroundjoin%
\definecolor{currentfill}{rgb}{0.282884,0.135920,0.453427}%
\pgfsetfillcolor{currentfill}%
\pgfsetfillopacity{0.700000}%
\pgfsetlinewidth{0.000000pt}%
\definecolor{currentstroke}{rgb}{0.000000,0.000000,0.000000}%
\pgfsetstrokecolor{currentstroke}%
\pgfsetdash{}{0pt}%
\pgfpathmoveto{\pgfqpoint{2.824916in}{2.480086in}}%
\pgfpathlineto{\pgfqpoint{2.838310in}{2.470831in}}%
\pgfpathlineto{\pgfqpoint{2.851705in}{2.461697in}}%
\pgfpathlineto{\pgfqpoint{2.865099in}{2.452682in}}%
\pgfpathlineto{\pgfqpoint{2.878495in}{2.443786in}}%
\pgfpathlineto{\pgfqpoint{2.886726in}{2.452842in}}%
\pgfpathlineto{\pgfqpoint{2.894949in}{2.461974in}}%
\pgfpathlineto{\pgfqpoint{2.903165in}{2.471181in}}%
\pgfpathlineto{\pgfqpoint{2.911373in}{2.480465in}}%
\pgfpathlineto{\pgfqpoint{2.897991in}{2.489377in}}%
\pgfpathlineto{\pgfqpoint{2.884609in}{2.498407in}}%
\pgfpathlineto{\pgfqpoint{2.871228in}{2.507556in}}%
\pgfpathlineto{\pgfqpoint{2.857848in}{2.516826in}}%
\pgfpathlineto{\pgfqpoint{2.849626in}{2.507519in}}%
\pgfpathlineto{\pgfqpoint{2.841397in}{2.498294in}}%
\pgfpathlineto{\pgfqpoint{2.833161in}{2.489150in}}%
\pgfpathlineto{\pgfqpoint{2.824916in}{2.480086in}}%
\pgfpathclose%
\pgfusepath{fill}%
\end{pgfscope}%
\begin{pgfscope}%
\pgfpathrectangle{\pgfqpoint{1.150000in}{0.150000in}}{\pgfqpoint{5.700000in}{5.700000in}}%
\pgfusepath{clip}%
\pgfsetbuttcap%
\pgfsetroundjoin%
\definecolor{currentfill}{rgb}{0.278826,0.175490,0.483397}%
\pgfsetfillcolor{currentfill}%
\pgfsetfillopacity{0.700000}%
\pgfsetlinewidth{0.000000pt}%
\definecolor{currentstroke}{rgb}{0.000000,0.000000,0.000000}%
\pgfsetstrokecolor{currentstroke}%
\pgfsetdash{}{0pt}%
\pgfpathmoveto{\pgfqpoint{2.630945in}{2.565511in}}%
\pgfpathlineto{\pgfqpoint{2.644359in}{2.554724in}}%
\pgfpathlineto{\pgfqpoint{2.657772in}{2.544070in}}%
\pgfpathlineto{\pgfqpoint{2.671183in}{2.533548in}}%
\pgfpathlineto{\pgfqpoint{2.684594in}{2.523158in}}%
\pgfpathlineto{\pgfqpoint{2.692899in}{2.531881in}}%
\pgfpathlineto{\pgfqpoint{2.701196in}{2.540692in}}%
\pgfpathlineto{\pgfqpoint{2.709485in}{2.549591in}}%
\pgfpathlineto{\pgfqpoint{2.717765in}{2.558580in}}%
\pgfpathlineto{\pgfqpoint{2.704369in}{2.568965in}}%
\pgfpathlineto{\pgfqpoint{2.690973in}{2.579480in}}%
\pgfpathlineto{\pgfqpoint{2.677575in}{2.590128in}}%
\pgfpathlineto{\pgfqpoint{2.664176in}{2.600909in}}%
\pgfpathlineto{\pgfqpoint{2.655881in}{2.591919in}}%
\pgfpathlineto{\pgfqpoint{2.647578in}{2.583023in}}%
\pgfpathlineto{\pgfqpoint{2.639266in}{2.574220in}}%
\pgfpathlineto{\pgfqpoint{2.630945in}{2.565511in}}%
\pgfpathclose%
\pgfusepath{fill}%
\end{pgfscope}%
\begin{pgfscope}%
\pgfpathrectangle{\pgfqpoint{1.150000in}{0.150000in}}{\pgfqpoint{5.700000in}{5.700000in}}%
\pgfusepath{clip}%
\pgfsetbuttcap%
\pgfsetroundjoin%
\definecolor{currentfill}{rgb}{0.270595,0.214069,0.507052}%
\pgfsetfillcolor{currentfill}%
\pgfsetfillopacity{0.700000}%
\pgfsetlinewidth{0.000000pt}%
\definecolor{currentstroke}{rgb}{0.000000,0.000000,0.000000}%
\pgfsetstrokecolor{currentstroke}%
\pgfsetdash{}{0pt}%
\pgfpathmoveto{\pgfqpoint{4.819185in}{2.619471in}}%
\pgfpathlineto{\pgfqpoint{4.832911in}{2.616819in}}%
\pgfpathlineto{\pgfqpoint{4.846645in}{2.614239in}}%
\pgfpathlineto{\pgfqpoint{4.860386in}{2.611729in}}%
\pgfpathlineto{\pgfqpoint{4.874135in}{2.609290in}}%
\pgfpathlineto{\pgfqpoint{4.881709in}{2.619832in}}%
\pgfpathlineto{\pgfqpoint{4.889280in}{2.630556in}}%
\pgfpathlineto{\pgfqpoint{4.896850in}{2.641467in}}%
\pgfpathlineto{\pgfqpoint{4.904418in}{2.652572in}}%
\pgfpathlineto{\pgfqpoint{4.890684in}{2.655371in}}%
\pgfpathlineto{\pgfqpoint{4.876957in}{2.658241in}}%
\pgfpathlineto{\pgfqpoint{4.863239in}{2.661181in}}%
\pgfpathlineto{\pgfqpoint{4.849527in}{2.664192in}}%
\pgfpathlineto{\pgfqpoint{4.841944in}{2.652720in}}%
\pgfpathlineto{\pgfqpoint{4.834360in}{2.641448in}}%
\pgfpathlineto{\pgfqpoint{4.826773in}{2.630367in}}%
\pgfpathlineto{\pgfqpoint{4.819185in}{2.619471in}}%
\pgfpathclose%
\pgfusepath{fill}%
\end{pgfscope}%
\begin{pgfscope}%
\pgfpathrectangle{\pgfqpoint{1.150000in}{0.150000in}}{\pgfqpoint{5.700000in}{5.700000in}}%
\pgfusepath{clip}%
\pgfsetbuttcap%
\pgfsetroundjoin%
\definecolor{currentfill}{rgb}{0.225863,0.330805,0.547314}%
\pgfsetfillcolor{currentfill}%
\pgfsetfillopacity{0.700000}%
\pgfsetlinewidth{0.000000pt}%
\definecolor{currentstroke}{rgb}{0.000000,0.000000,0.000000}%
\pgfsetstrokecolor{currentstroke}%
\pgfsetdash{}{0pt}%
\pgfpathmoveto{\pgfqpoint{5.471262in}{2.873408in}}%
\pgfpathlineto{\pgfqpoint{5.485125in}{2.870013in}}%
\pgfpathlineto{\pgfqpoint{5.498997in}{2.866685in}}%
\pgfpathlineto{\pgfqpoint{5.512877in}{2.863423in}}%
\pgfpathlineto{\pgfqpoint{5.526765in}{2.860227in}}%
\pgfpathlineto{\pgfqpoint{5.534242in}{2.874643in}}%
\pgfpathlineto{\pgfqpoint{5.541724in}{2.889417in}}%
\pgfpathlineto{\pgfqpoint{5.549214in}{2.904558in}}%
\pgfpathlineto{\pgfqpoint{5.556711in}{2.920073in}}%
\pgfpathlineto{\pgfqpoint{5.542843in}{2.923769in}}%
\pgfpathlineto{\pgfqpoint{5.528982in}{2.927531in}}%
\pgfpathlineto{\pgfqpoint{5.515130in}{2.931359in}}%
\pgfpathlineto{\pgfqpoint{5.501286in}{2.935254in}}%
\pgfpathlineto{\pgfqpoint{5.493769in}{2.919231in}}%
\pgfpathlineto{\pgfqpoint{5.486260in}{2.903588in}}%
\pgfpathlineto{\pgfqpoint{5.478758in}{2.888317in}}%
\pgfpathlineto{\pgfqpoint{5.471262in}{2.873408in}}%
\pgfpathclose%
\pgfusepath{fill}%
\end{pgfscope}%
\begin{pgfscope}%
\pgfpathrectangle{\pgfqpoint{1.150000in}{0.150000in}}{\pgfqpoint{5.700000in}{5.700000in}}%
\pgfusepath{clip}%
\pgfsetbuttcap%
\pgfsetroundjoin%
\definecolor{currentfill}{rgb}{0.235526,0.309527,0.542944}%
\pgfsetfillcolor{currentfill}%
\pgfsetfillopacity{0.700000}%
\pgfsetlinewidth{0.000000pt}%
\definecolor{currentstroke}{rgb}{0.000000,0.000000,0.000000}%
\pgfsetstrokecolor{currentstroke}%
\pgfsetdash{}{0pt}%
\pgfpathmoveto{\pgfqpoint{5.385884in}{2.829540in}}%
\pgfpathlineto{\pgfqpoint{5.399734in}{2.826359in}}%
\pgfpathlineto{\pgfqpoint{5.413592in}{2.823244in}}%
\pgfpathlineto{\pgfqpoint{5.427458in}{2.820196in}}%
\pgfpathlineto{\pgfqpoint{5.441333in}{2.817215in}}%
\pgfpathlineto{\pgfqpoint{5.448807in}{2.830764in}}%
\pgfpathlineto{\pgfqpoint{5.456287in}{2.844640in}}%
\pgfpathlineto{\pgfqpoint{5.463771in}{2.858852in}}%
\pgfpathlineto{\pgfqpoint{5.471262in}{2.873408in}}%
\pgfpathlineto{\pgfqpoint{5.457406in}{2.876869in}}%
\pgfpathlineto{\pgfqpoint{5.443559in}{2.880397in}}%
\pgfpathlineto{\pgfqpoint{5.429720in}{2.883992in}}%
\pgfpathlineto{\pgfqpoint{5.415888in}{2.887653in}}%
\pgfpathlineto{\pgfqpoint{5.408379in}{2.872610in}}%
\pgfpathlineto{\pgfqpoint{5.400876in}{2.857916in}}%
\pgfpathlineto{\pgfqpoint{5.393377in}{2.843562in}}%
\pgfpathlineto{\pgfqpoint{5.385884in}{2.829540in}}%
\pgfpathclose%
\pgfusepath{fill}%
\end{pgfscope}%
\begin{pgfscope}%
\pgfpathrectangle{\pgfqpoint{1.150000in}{0.150000in}}{\pgfqpoint{5.700000in}{5.700000in}}%
\pgfusepath{clip}%
\pgfsetbuttcap%
\pgfsetroundjoin%
\definecolor{currentfill}{rgb}{0.216210,0.351535,0.550627}%
\pgfsetfillcolor{currentfill}%
\pgfsetfillopacity{0.700000}%
\pgfsetlinewidth{0.000000pt}%
\definecolor{currentstroke}{rgb}{0.000000,0.000000,0.000000}%
\pgfsetstrokecolor{currentstroke}%
\pgfsetdash{}{0pt}%
\pgfpathmoveto{\pgfqpoint{5.556711in}{2.920073in}}%
\pgfpathlineto{\pgfqpoint{5.570588in}{2.916443in}}%
\pgfpathlineto{\pgfqpoint{5.584472in}{2.912879in}}%
\pgfpathlineto{\pgfqpoint{5.598365in}{2.909381in}}%
\pgfpathlineto{\pgfqpoint{5.612267in}{2.905949in}}%
\pgfpathlineto{\pgfqpoint{5.619751in}{2.921337in}}%
\pgfpathlineto{\pgfqpoint{5.627244in}{2.937114in}}%
\pgfpathlineto{\pgfqpoint{5.634745in}{2.953290in}}%
\pgfpathlineto{\pgfqpoint{5.642256in}{2.969875in}}%
\pgfpathlineto{\pgfqpoint{5.628375in}{2.973827in}}%
\pgfpathlineto{\pgfqpoint{5.614502in}{2.977846in}}%
\pgfpathlineto{\pgfqpoint{5.600637in}{2.981929in}}%
\pgfpathlineto{\pgfqpoint{5.586780in}{2.986079in}}%
\pgfpathlineto{\pgfqpoint{5.579250in}{2.968967in}}%
\pgfpathlineto{\pgfqpoint{5.571729in}{2.952269in}}%
\pgfpathlineto{\pgfqpoint{5.564216in}{2.935974in}}%
\pgfpathlineto{\pgfqpoint{5.556711in}{2.920073in}}%
\pgfpathclose%
\pgfusepath{fill}%
\end{pgfscope}%
\begin{pgfscope}%
\pgfpathrectangle{\pgfqpoint{1.150000in}{0.150000in}}{\pgfqpoint{5.700000in}{5.700000in}}%
\pgfusepath{clip}%
\pgfsetbuttcap%
\pgfsetroundjoin%
\definecolor{currentfill}{rgb}{0.282910,0.105393,0.426902}%
\pgfsetfillcolor{currentfill}%
\pgfsetfillopacity{0.700000}%
\pgfsetlinewidth{0.000000pt}%
\definecolor{currentstroke}{rgb}{0.000000,0.000000,0.000000}%
\pgfsetstrokecolor{currentstroke}%
\pgfsetdash{}{0pt}%
\pgfpathmoveto{\pgfqpoint{3.887778in}{2.406319in}}%
\pgfpathlineto{\pgfqpoint{3.901282in}{2.402322in}}%
\pgfpathlineto{\pgfqpoint{3.914792in}{2.398407in}}%
\pgfpathlineto{\pgfqpoint{3.928308in}{2.394576in}}%
\pgfpathlineto{\pgfqpoint{3.941830in}{2.390828in}}%
\pgfpathlineto{\pgfqpoint{3.949698in}{2.400516in}}%
\pgfpathlineto{\pgfqpoint{3.957561in}{2.410269in}}%
\pgfpathlineto{\pgfqpoint{3.965419in}{2.420090in}}%
\pgfpathlineto{\pgfqpoint{3.973271in}{2.429983in}}%
\pgfpathlineto{\pgfqpoint{3.959759in}{2.433909in}}%
\pgfpathlineto{\pgfqpoint{3.946254in}{2.437919in}}%
\pgfpathlineto{\pgfqpoint{3.932754in}{2.442011in}}%
\pgfpathlineto{\pgfqpoint{3.919259in}{2.446187in}}%
\pgfpathlineto{\pgfqpoint{3.911397in}{2.436109in}}%
\pgfpathlineto{\pgfqpoint{3.903529in}{2.426107in}}%
\pgfpathlineto{\pgfqpoint{3.895656in}{2.416178in}}%
\pgfpathlineto{\pgfqpoint{3.887778in}{2.406319in}}%
\pgfpathclose%
\pgfusepath{fill}%
\end{pgfscope}%
\begin{pgfscope}%
\pgfpathrectangle{\pgfqpoint{1.150000in}{0.150000in}}{\pgfqpoint{5.700000in}{5.700000in}}%
\pgfusepath{clip}%
\pgfsetbuttcap%
\pgfsetroundjoin%
\definecolor{currentfill}{rgb}{0.243113,0.292092,0.538516}%
\pgfsetfillcolor{currentfill}%
\pgfsetfillopacity{0.700000}%
\pgfsetlinewidth{0.000000pt}%
\definecolor{currentstroke}{rgb}{0.000000,0.000000,0.000000}%
\pgfsetstrokecolor{currentstroke}%
\pgfsetdash{}{0pt}%
\pgfpathmoveto{\pgfqpoint{5.300557in}{2.788157in}}%
\pgfpathlineto{\pgfqpoint{5.314393in}{2.785166in}}%
\pgfpathlineto{\pgfqpoint{5.328237in}{2.782243in}}%
\pgfpathlineto{\pgfqpoint{5.342089in}{2.779388in}}%
\pgfpathlineto{\pgfqpoint{5.355949in}{2.776599in}}%
\pgfpathlineto{\pgfqpoint{5.363428in}{2.789379in}}%
\pgfpathlineto{\pgfqpoint{5.370909in}{2.802457in}}%
\pgfpathlineto{\pgfqpoint{5.378394in}{2.815841in}}%
\pgfpathlineto{\pgfqpoint{5.385884in}{2.829540in}}%
\pgfpathlineto{\pgfqpoint{5.372042in}{2.832789in}}%
\pgfpathlineto{\pgfqpoint{5.358208in}{2.836104in}}%
\pgfpathlineto{\pgfqpoint{5.344383in}{2.839487in}}%
\pgfpathlineto{\pgfqpoint{5.330565in}{2.842937in}}%
\pgfpathlineto{\pgfqpoint{5.323057in}{2.828771in}}%
\pgfpathlineto{\pgfqpoint{5.315554in}{2.814925in}}%
\pgfpathlineto{\pgfqpoint{5.308054in}{2.801390in}}%
\pgfpathlineto{\pgfqpoint{5.300557in}{2.788157in}}%
\pgfpathclose%
\pgfusepath{fill}%
\end{pgfscope}%
\begin{pgfscope}%
\pgfpathrectangle{\pgfqpoint{1.150000in}{0.150000in}}{\pgfqpoint{5.700000in}{5.700000in}}%
\pgfusepath{clip}%
\pgfsetbuttcap%
\pgfsetroundjoin%
\definecolor{currentfill}{rgb}{0.280868,0.160771,0.472899}%
\pgfsetfillcolor{currentfill}%
\pgfsetfillopacity{0.700000}%
\pgfsetlinewidth{0.000000pt}%
\definecolor{currentstroke}{rgb}{0.000000,0.000000,0.000000}%
\pgfsetstrokecolor{currentstroke}%
\pgfsetdash{}{0pt}%
\pgfpathmoveto{\pgfqpoint{4.423364in}{2.508931in}}%
\pgfpathlineto{\pgfqpoint{4.436994in}{2.506055in}}%
\pgfpathlineto{\pgfqpoint{4.450631in}{2.503254in}}%
\pgfpathlineto{\pgfqpoint{4.464275in}{2.500529in}}%
\pgfpathlineto{\pgfqpoint{4.477927in}{2.497878in}}%
\pgfpathlineto{\pgfqpoint{4.485620in}{2.507701in}}%
\pgfpathlineto{\pgfqpoint{4.493308in}{2.517638in}}%
\pgfpathlineto{\pgfqpoint{4.500993in}{2.527694in}}%
\pgfpathlineto{\pgfqpoint{4.508674in}{2.537874in}}%
\pgfpathlineto{\pgfqpoint{4.495035in}{2.540805in}}%
\pgfpathlineto{\pgfqpoint{4.481403in}{2.543810in}}%
\pgfpathlineto{\pgfqpoint{4.467778in}{2.546890in}}%
\pgfpathlineto{\pgfqpoint{4.454160in}{2.550044in}}%
\pgfpathlineto{\pgfqpoint{4.446467in}{2.539578in}}%
\pgfpathlineto{\pgfqpoint{4.438770in}{2.529240in}}%
\pgfpathlineto{\pgfqpoint{4.431069in}{2.519026in}}%
\pgfpathlineto{\pgfqpoint{4.423364in}{2.508931in}}%
\pgfpathclose%
\pgfusepath{fill}%
\end{pgfscope}%
\begin{pgfscope}%
\pgfpathrectangle{\pgfqpoint{1.150000in}{0.150000in}}{\pgfqpoint{5.700000in}{5.700000in}}%
\pgfusepath{clip}%
\pgfsetbuttcap%
\pgfsetroundjoin%
\definecolor{currentfill}{rgb}{0.282327,0.094955,0.417331}%
\pgfsetfillcolor{currentfill}%
\pgfsetfillopacity{0.700000}%
\pgfsetlinewidth{0.000000pt}%
\definecolor{currentstroke}{rgb}{0.000000,0.000000,0.000000}%
\pgfsetstrokecolor{currentstroke}%
\pgfsetdash{}{0pt}%
\pgfpathmoveto{\pgfqpoint{3.662679in}{2.380635in}}%
\pgfpathlineto{\pgfqpoint{3.676140in}{2.375902in}}%
\pgfpathlineto{\pgfqpoint{3.689606in}{2.371258in}}%
\pgfpathlineto{\pgfqpoint{3.703077in}{2.366701in}}%
\pgfpathlineto{\pgfqpoint{3.716553in}{2.362232in}}%
\pgfpathlineto{\pgfqpoint{3.724496in}{2.371907in}}%
\pgfpathlineto{\pgfqpoint{3.732433in}{2.381638in}}%
\pgfpathlineto{\pgfqpoint{3.740365in}{2.391428in}}%
\pgfpathlineto{\pgfqpoint{3.748291in}{2.401278in}}%
\pgfpathlineto{\pgfqpoint{3.734825in}{2.405885in}}%
\pgfpathlineto{\pgfqpoint{3.721364in}{2.410579in}}%
\pgfpathlineto{\pgfqpoint{3.707908in}{2.415362in}}%
\pgfpathlineto{\pgfqpoint{3.694457in}{2.420232in}}%
\pgfpathlineto{\pgfqpoint{3.686521in}{2.410237in}}%
\pgfpathlineto{\pgfqpoint{3.678579in}{2.400307in}}%
\pgfpathlineto{\pgfqpoint{3.670632in}{2.390441in}}%
\pgfpathlineto{\pgfqpoint{3.662679in}{2.380635in}}%
\pgfpathclose%
\pgfusepath{fill}%
\end{pgfscope}%
\begin{pgfscope}%
\pgfpathrectangle{\pgfqpoint{1.150000in}{0.150000in}}{\pgfqpoint{5.700000in}{5.700000in}}%
\pgfusepath{clip}%
\pgfsetbuttcap%
\pgfsetroundjoin%
\definecolor{currentfill}{rgb}{0.282327,0.094955,0.417331}%
\pgfsetfillcolor{currentfill}%
\pgfsetfillopacity{0.700000}%
\pgfsetlinewidth{0.000000pt}%
\definecolor{currentstroke}{rgb}{0.000000,0.000000,0.000000}%
\pgfsetstrokecolor{currentstroke}%
\pgfsetdash{}{0pt}%
\pgfpathmoveto{\pgfqpoint{3.158233in}{2.390628in}}%
\pgfpathlineto{\pgfqpoint{3.171631in}{2.383564in}}%
\pgfpathlineto{\pgfqpoint{3.185032in}{2.376604in}}%
\pgfpathlineto{\pgfqpoint{3.198435in}{2.369747in}}%
\pgfpathlineto{\pgfqpoint{3.211841in}{2.362992in}}%
\pgfpathlineto{\pgfqpoint{3.219955in}{2.372432in}}%
\pgfpathlineto{\pgfqpoint{3.228061in}{2.381930in}}%
\pgfpathlineto{\pgfqpoint{3.236162in}{2.391487in}}%
\pgfpathlineto{\pgfqpoint{3.244256in}{2.401105in}}%
\pgfpathlineto{\pgfqpoint{3.230861in}{2.407917in}}%
\pgfpathlineto{\pgfqpoint{3.217469in}{2.414830in}}%
\pgfpathlineto{\pgfqpoint{3.204079in}{2.421847in}}%
\pgfpathlineto{\pgfqpoint{3.190693in}{2.428967in}}%
\pgfpathlineto{\pgfqpoint{3.182588in}{2.419285in}}%
\pgfpathlineto{\pgfqpoint{3.174476in}{2.409669in}}%
\pgfpathlineto{\pgfqpoint{3.166358in}{2.400117in}}%
\pgfpathlineto{\pgfqpoint{3.158233in}{2.390628in}}%
\pgfpathclose%
\pgfusepath{fill}%
\end{pgfscope}%
\begin{pgfscope}%
\pgfpathrectangle{\pgfqpoint{1.150000in}{0.150000in}}{\pgfqpoint{5.700000in}{5.700000in}}%
\pgfusepath{clip}%
\pgfsetbuttcap%
\pgfsetroundjoin%
\definecolor{currentfill}{rgb}{0.206756,0.371758,0.553117}%
\pgfsetfillcolor{currentfill}%
\pgfsetfillopacity{0.700000}%
\pgfsetlinewidth{0.000000pt}%
\definecolor{currentstroke}{rgb}{0.000000,0.000000,0.000000}%
\pgfsetstrokecolor{currentstroke}%
\pgfsetdash{}{0pt}%
\pgfpathmoveto{\pgfqpoint{5.642256in}{2.969875in}}%
\pgfpathlineto{\pgfqpoint{5.656145in}{2.965988in}}%
\pgfpathlineto{\pgfqpoint{5.670042in}{2.962166in}}%
\pgfpathlineto{\pgfqpoint{5.683947in}{2.958410in}}%
\pgfpathlineto{\pgfqpoint{5.697861in}{2.954719in}}%
\pgfpathlineto{\pgfqpoint{5.705360in}{2.971189in}}%
\pgfpathlineto{\pgfqpoint{5.712870in}{2.988082in}}%
\pgfpathlineto{\pgfqpoint{5.720390in}{3.005408in}}%
\pgfpathlineto{\pgfqpoint{5.727922in}{3.023177in}}%
\pgfpathlineto{\pgfqpoint{5.714029in}{3.027409in}}%
\pgfpathlineto{\pgfqpoint{5.700144in}{3.031705in}}%
\pgfpathlineto{\pgfqpoint{5.686267in}{3.036067in}}%
\pgfpathlineto{\pgfqpoint{5.672398in}{3.040494in}}%
\pgfpathlineto{\pgfqpoint{5.664847in}{3.022177in}}%
\pgfpathlineto{\pgfqpoint{5.657306in}{3.004308in}}%
\pgfpathlineto{\pgfqpoint{5.649776in}{2.986878in}}%
\pgfpathlineto{\pgfqpoint{5.642256in}{2.969875in}}%
\pgfpathclose%
\pgfusepath{fill}%
\end{pgfscope}%
\begin{pgfscope}%
\pgfpathrectangle{\pgfqpoint{1.150000in}{0.150000in}}{\pgfqpoint{5.700000in}{5.700000in}}%
\pgfusepath{clip}%
\pgfsetbuttcap%
\pgfsetroundjoin%
\definecolor{currentfill}{rgb}{0.281924,0.089666,0.412415}%
\pgfsetfillcolor{currentfill}%
\pgfsetfillopacity{0.700000}%
\pgfsetlinewidth{0.000000pt}%
\definecolor{currentstroke}{rgb}{0.000000,0.000000,0.000000}%
\pgfsetstrokecolor{currentstroke}%
\pgfsetdash{}{0pt}%
\pgfpathmoveto{\pgfqpoint{3.297864in}{2.374872in}}%
\pgfpathlineto{\pgfqpoint{3.311274in}{2.368564in}}%
\pgfpathlineto{\pgfqpoint{3.324688in}{2.362354in}}%
\pgfpathlineto{\pgfqpoint{3.338105in}{2.356242in}}%
\pgfpathlineto{\pgfqpoint{3.351525in}{2.350227in}}%
\pgfpathlineto{\pgfqpoint{3.359591in}{2.359769in}}%
\pgfpathlineto{\pgfqpoint{3.367650in}{2.369366in}}%
\pgfpathlineto{\pgfqpoint{3.375703in}{2.379019in}}%
\pgfpathlineto{\pgfqpoint{3.383751in}{2.388729in}}%
\pgfpathlineto{\pgfqpoint{3.370341in}{2.394820in}}%
\pgfpathlineto{\pgfqpoint{3.356935in}{2.401009in}}%
\pgfpathlineto{\pgfqpoint{3.343532in}{2.407296in}}%
\pgfpathlineto{\pgfqpoint{3.330133in}{2.413682in}}%
\pgfpathlineto{\pgfqpoint{3.322075in}{2.403887in}}%
\pgfpathlineto{\pgfqpoint{3.314011in}{2.394155in}}%
\pgfpathlineto{\pgfqpoint{3.305941in}{2.384484in}}%
\pgfpathlineto{\pgfqpoint{3.297864in}{2.374872in}}%
\pgfpathclose%
\pgfusepath{fill}%
\end{pgfscope}%
\begin{pgfscope}%
\pgfpathrectangle{\pgfqpoint{1.150000in}{0.150000in}}{\pgfqpoint{5.700000in}{5.700000in}}%
\pgfusepath{clip}%
\pgfsetbuttcap%
\pgfsetroundjoin%
\definecolor{currentfill}{rgb}{0.250425,0.274290,0.533103}%
\pgfsetfillcolor{currentfill}%
\pgfsetfillopacity{0.700000}%
\pgfsetlinewidth{0.000000pt}%
\definecolor{currentstroke}{rgb}{0.000000,0.000000,0.000000}%
\pgfsetstrokecolor{currentstroke}%
\pgfsetdash{}{0pt}%
\pgfpathmoveto{\pgfqpoint{5.215263in}{2.748970in}}%
\pgfpathlineto{\pgfqpoint{5.229084in}{2.746149in}}%
\pgfpathlineto{\pgfqpoint{5.242914in}{2.743395in}}%
\pgfpathlineto{\pgfqpoint{5.256751in}{2.740710in}}%
\pgfpathlineto{\pgfqpoint{5.270597in}{2.738092in}}%
\pgfpathlineto{\pgfqpoint{5.278083in}{2.750194in}}%
\pgfpathlineto{\pgfqpoint{5.285572in}{2.762567in}}%
\pgfpathlineto{\pgfqpoint{5.293063in}{2.775219in}}%
\pgfpathlineto{\pgfqpoint{5.300557in}{2.788157in}}%
\pgfpathlineto{\pgfqpoint{5.286729in}{2.791215in}}%
\pgfpathlineto{\pgfqpoint{5.272910in}{2.794340in}}%
\pgfpathlineto{\pgfqpoint{5.259098in}{2.797534in}}%
\pgfpathlineto{\pgfqpoint{5.245295in}{2.800795in}}%
\pgfpathlineto{\pgfqpoint{5.237784in}{2.787410in}}%
\pgfpathlineto{\pgfqpoint{5.230275in}{2.774316in}}%
\pgfpathlineto{\pgfqpoint{5.222768in}{2.761505in}}%
\pgfpathlineto{\pgfqpoint{5.215263in}{2.748970in}}%
\pgfpathclose%
\pgfusepath{fill}%
\end{pgfscope}%
\begin{pgfscope}%
\pgfpathrectangle{\pgfqpoint{1.150000in}{0.150000in}}{\pgfqpoint{5.700000in}{5.700000in}}%
\pgfusepath{clip}%
\pgfsetbuttcap%
\pgfsetroundjoin%
\definecolor{currentfill}{rgb}{0.282910,0.105393,0.426902}%
\pgfsetfillcolor{currentfill}%
\pgfsetfillopacity{0.700000}%
\pgfsetlinewidth{0.000000pt}%
\definecolor{currentstroke}{rgb}{0.000000,0.000000,0.000000}%
\pgfsetstrokecolor{currentstroke}%
\pgfsetdash{}{0pt}%
\pgfpathmoveto{\pgfqpoint{3.018468in}{2.413330in}}%
\pgfpathlineto{\pgfqpoint{3.031861in}{2.405445in}}%
\pgfpathlineto{\pgfqpoint{3.045256in}{2.397669in}}%
\pgfpathlineto{\pgfqpoint{3.058653in}{2.390003in}}%
\pgfpathlineto{\pgfqpoint{3.072051in}{2.382445in}}%
\pgfpathlineto{\pgfqpoint{3.080215in}{2.391722in}}%
\pgfpathlineto{\pgfqpoint{3.088372in}{2.401062in}}%
\pgfpathlineto{\pgfqpoint{3.096522in}{2.410467in}}%
\pgfpathlineto{\pgfqpoint{3.104665in}{2.419937in}}%
\pgfpathlineto{\pgfqpoint{3.091278in}{2.427532in}}%
\pgfpathlineto{\pgfqpoint{3.077893in}{2.435234in}}%
\pgfpathlineto{\pgfqpoint{3.064510in}{2.443045in}}%
\pgfpathlineto{\pgfqpoint{3.051129in}{2.450967in}}%
\pgfpathlineto{\pgfqpoint{3.042974in}{2.441453in}}%
\pgfpathlineto{\pgfqpoint{3.034813in}{2.432009in}}%
\pgfpathlineto{\pgfqpoint{3.026644in}{2.422635in}}%
\pgfpathlineto{\pgfqpoint{3.018468in}{2.413330in}}%
\pgfpathclose%
\pgfusepath{fill}%
\end{pgfscope}%
\begin{pgfscope}%
\pgfpathrectangle{\pgfqpoint{1.150000in}{0.150000in}}{\pgfqpoint{5.700000in}{5.700000in}}%
\pgfusepath{clip}%
\pgfsetbuttcap%
\pgfsetroundjoin%
\definecolor{currentfill}{rgb}{0.283187,0.125848,0.444960}%
\pgfsetfillcolor{currentfill}%
\pgfsetfillopacity{0.700000}%
\pgfsetlinewidth{0.000000pt}%
\definecolor{currentstroke}{rgb}{0.000000,0.000000,0.000000}%
\pgfsetstrokecolor{currentstroke}%
\pgfsetdash{}{0pt}%
\pgfpathmoveto{\pgfqpoint{4.112853in}{2.440273in}}%
\pgfpathlineto{\pgfqpoint{4.126409in}{2.436877in}}%
\pgfpathlineto{\pgfqpoint{4.139972in}{2.433560in}}%
\pgfpathlineto{\pgfqpoint{4.153540in}{2.430322in}}%
\pgfpathlineto{\pgfqpoint{4.167116in}{2.427163in}}%
\pgfpathlineto{\pgfqpoint{4.174911in}{2.436835in}}%
\pgfpathlineto{\pgfqpoint{4.182701in}{2.446586in}}%
\pgfpathlineto{\pgfqpoint{4.190486in}{2.456421in}}%
\pgfpathlineto{\pgfqpoint{4.198266in}{2.466342in}}%
\pgfpathlineto{\pgfqpoint{4.184701in}{2.469720in}}%
\pgfpathlineto{\pgfqpoint{4.171143in}{2.473177in}}%
\pgfpathlineto{\pgfqpoint{4.157592in}{2.476712in}}%
\pgfpathlineto{\pgfqpoint{4.144046in}{2.480327in}}%
\pgfpathlineto{\pgfqpoint{4.136255in}{2.470179in}}%
\pgfpathlineto{\pgfqpoint{4.128460in}{2.460124in}}%
\pgfpathlineto{\pgfqpoint{4.120659in}{2.450156in}}%
\pgfpathlineto{\pgfqpoint{4.112853in}{2.440273in}}%
\pgfpathclose%
\pgfusepath{fill}%
\end{pgfscope}%
\begin{pgfscope}%
\pgfpathrectangle{\pgfqpoint{1.150000in}{0.150000in}}{\pgfqpoint{5.700000in}{5.700000in}}%
\pgfusepath{clip}%
\pgfsetbuttcap%
\pgfsetroundjoin%
\definecolor{currentfill}{rgb}{0.274128,0.199721,0.498911}%
\pgfsetfillcolor{currentfill}%
\pgfsetfillopacity{0.700000}%
\pgfsetlinewidth{0.000000pt}%
\definecolor{currentstroke}{rgb}{0.000000,0.000000,0.000000}%
\pgfsetstrokecolor{currentstroke}%
\pgfsetdash{}{0pt}%
\pgfpathmoveto{\pgfqpoint{4.733926in}{2.587583in}}%
\pgfpathlineto{\pgfqpoint{4.747636in}{2.584985in}}%
\pgfpathlineto{\pgfqpoint{4.761353in}{2.582459in}}%
\pgfpathlineto{\pgfqpoint{4.775077in}{2.580004in}}%
\pgfpathlineto{\pgfqpoint{4.788810in}{2.577621in}}%
\pgfpathlineto{\pgfqpoint{4.796407in}{2.587836in}}%
\pgfpathlineto{\pgfqpoint{4.804002in}{2.598212in}}%
\pgfpathlineto{\pgfqpoint{4.811595in}{2.608755in}}%
\pgfpathlineto{\pgfqpoint{4.819185in}{2.619471in}}%
\pgfpathlineto{\pgfqpoint{4.805467in}{2.622194in}}%
\pgfpathlineto{\pgfqpoint{4.791757in}{2.624989in}}%
\pgfpathlineto{\pgfqpoint{4.778054in}{2.627855in}}%
\pgfpathlineto{\pgfqpoint{4.764358in}{2.630792in}}%
\pgfpathlineto{\pgfqpoint{4.756754in}{2.619729in}}%
\pgfpathlineto{\pgfqpoint{4.749147in}{2.608844in}}%
\pgfpathlineto{\pgfqpoint{4.741538in}{2.598131in}}%
\pgfpathlineto{\pgfqpoint{4.733926in}{2.587583in}}%
\pgfpathclose%
\pgfusepath{fill}%
\end{pgfscope}%
\begin{pgfscope}%
\pgfpathrectangle{\pgfqpoint{1.150000in}{0.150000in}}{\pgfqpoint{5.700000in}{5.700000in}}%
\pgfusepath{clip}%
\pgfsetbuttcap%
\pgfsetroundjoin%
\definecolor{currentfill}{rgb}{0.255645,0.260703,0.528312}%
\pgfsetfillcolor{currentfill}%
\pgfsetfillopacity{0.700000}%
\pgfsetlinewidth{0.000000pt}%
\definecolor{currentstroke}{rgb}{0.000000,0.000000,0.000000}%
\pgfsetstrokecolor{currentstroke}%
\pgfsetdash{}{0pt}%
\pgfpathmoveto{\pgfqpoint{5.129987in}{2.711717in}}%
\pgfpathlineto{\pgfqpoint{5.143793in}{2.709042in}}%
\pgfpathlineto{\pgfqpoint{5.157607in}{2.706436in}}%
\pgfpathlineto{\pgfqpoint{5.171429in}{2.703899in}}%
\pgfpathlineto{\pgfqpoint{5.185260in}{2.701429in}}%
\pgfpathlineto{\pgfqpoint{5.192759in}{2.712940in}}%
\pgfpathlineto{\pgfqpoint{5.200259in}{2.724695in}}%
\pgfpathlineto{\pgfqpoint{5.207761in}{2.736703in}}%
\pgfpathlineto{\pgfqpoint{5.215263in}{2.748970in}}%
\pgfpathlineto{\pgfqpoint{5.201451in}{2.751860in}}%
\pgfpathlineto{\pgfqpoint{5.187646in}{2.754817in}}%
\pgfpathlineto{\pgfqpoint{5.173849in}{2.757843in}}%
\pgfpathlineto{\pgfqpoint{5.160060in}{2.760937in}}%
\pgfpathlineto{\pgfqpoint{5.152540in}{2.748243in}}%
\pgfpathlineto{\pgfqpoint{5.145021in}{2.735813in}}%
\pgfpathlineto{\pgfqpoint{5.137504in}{2.723640in}}%
\pgfpathlineto{\pgfqpoint{5.129987in}{2.711717in}}%
\pgfpathclose%
\pgfusepath{fill}%
\end{pgfscope}%
\begin{pgfscope}%
\pgfpathrectangle{\pgfqpoint{1.150000in}{0.150000in}}{\pgfqpoint{5.700000in}{5.700000in}}%
\pgfusepath{clip}%
\pgfsetbuttcap%
\pgfsetroundjoin%
\definecolor{currentfill}{rgb}{0.281446,0.084320,0.407414}%
\pgfsetfillcolor{currentfill}%
\pgfsetfillopacity{0.700000}%
\pgfsetlinewidth{0.000000pt}%
\definecolor{currentstroke}{rgb}{0.000000,0.000000,0.000000}%
\pgfsetstrokecolor{currentstroke}%
\pgfsetdash{}{0pt}%
\pgfpathmoveto{\pgfqpoint{3.437426in}{2.365325in}}%
\pgfpathlineto{\pgfqpoint{3.450854in}{2.359712in}}%
\pgfpathlineto{\pgfqpoint{3.464286in}{2.354193in}}%
\pgfpathlineto{\pgfqpoint{3.477723in}{2.348768in}}%
\pgfpathlineto{\pgfqpoint{3.491163in}{2.343435in}}%
\pgfpathlineto{\pgfqpoint{3.499183in}{2.353026in}}%
\pgfpathlineto{\pgfqpoint{3.507198in}{2.362668in}}%
\pgfpathlineto{\pgfqpoint{3.515206in}{2.372364in}}%
\pgfpathlineto{\pgfqpoint{3.523208in}{2.382116in}}%
\pgfpathlineto{\pgfqpoint{3.509778in}{2.387546in}}%
\pgfpathlineto{\pgfqpoint{3.496352in}{2.393068in}}%
\pgfpathlineto{\pgfqpoint{3.482930in}{2.398685in}}%
\pgfpathlineto{\pgfqpoint{3.469512in}{2.404395in}}%
\pgfpathlineto{\pgfqpoint{3.461499in}{2.394538in}}%
\pgfpathlineto{\pgfqpoint{3.453481in}{2.384743in}}%
\pgfpathlineto{\pgfqpoint{3.445456in}{2.375006in}}%
\pgfpathlineto{\pgfqpoint{3.437426in}{2.365325in}}%
\pgfpathclose%
\pgfusepath{fill}%
\end{pgfscope}%
\begin{pgfscope}%
\pgfpathrectangle{\pgfqpoint{1.150000in}{0.150000in}}{\pgfqpoint{5.700000in}{5.700000in}}%
\pgfusepath{clip}%
\pgfsetbuttcap%
\pgfsetroundjoin%
\definecolor{currentfill}{rgb}{0.195860,0.395433,0.555276}%
\pgfsetfillcolor{currentfill}%
\pgfsetfillopacity{0.700000}%
\pgfsetlinewidth{0.000000pt}%
\definecolor{currentstroke}{rgb}{0.000000,0.000000,0.000000}%
\pgfsetstrokecolor{currentstroke}%
\pgfsetdash{}{0pt}%
\pgfpathmoveto{\pgfqpoint{5.727922in}{3.023177in}}%
\pgfpathlineto{\pgfqpoint{5.741822in}{3.019011in}}%
\pgfpathlineto{\pgfqpoint{5.755731in}{3.014910in}}%
\pgfpathlineto{\pgfqpoint{5.769648in}{3.010874in}}%
\pgfpathlineto{\pgfqpoint{5.783573in}{3.006903in}}%
\pgfpathlineto{\pgfqpoint{5.791095in}{3.024572in}}%
\pgfpathlineto{\pgfqpoint{5.798630in}{3.042700in}}%
\pgfpathlineto{\pgfqpoint{5.806177in}{3.061297in}}%
\pgfpathlineto{\pgfqpoint{5.813738in}{3.080373in}}%
\pgfpathlineto{\pgfqpoint{5.799834in}{3.084905in}}%
\pgfpathlineto{\pgfqpoint{5.785938in}{3.089502in}}%
\pgfpathlineto{\pgfqpoint{5.772050in}{3.094163in}}%
\pgfpathlineto{\pgfqpoint{5.758169in}{3.098890in}}%
\pgfpathlineto{\pgfqpoint{5.750588in}{3.079245in}}%
\pgfpathlineto{\pgfqpoint{5.743020in}{3.060086in}}%
\pgfpathlineto{\pgfqpoint{5.735465in}{3.041400in}}%
\pgfpathlineto{\pgfqpoint{5.727922in}{3.023177in}}%
\pgfpathclose%
\pgfusepath{fill}%
\end{pgfscope}%
\begin{pgfscope}%
\pgfpathrectangle{\pgfqpoint{1.150000in}{0.150000in}}{\pgfqpoint{5.700000in}{5.700000in}}%
\pgfusepath{clip}%
\pgfsetbuttcap%
\pgfsetroundjoin%
\definecolor{currentfill}{rgb}{0.280868,0.160771,0.472899}%
\pgfsetfillcolor{currentfill}%
\pgfsetfillopacity{0.700000}%
\pgfsetlinewidth{0.000000pt}%
\definecolor{currentstroke}{rgb}{0.000000,0.000000,0.000000}%
\pgfsetstrokecolor{currentstroke}%
\pgfsetdash{}{0pt}%
\pgfpathmoveto{\pgfqpoint{2.684594in}{2.523158in}}%
\pgfpathlineto{\pgfqpoint{2.698004in}{2.512897in}}%
\pgfpathlineto{\pgfqpoint{2.711413in}{2.502766in}}%
\pgfpathlineto{\pgfqpoint{2.724822in}{2.492762in}}%
\pgfpathlineto{\pgfqpoint{2.738230in}{2.482884in}}%
\pgfpathlineto{\pgfqpoint{2.746520in}{2.491620in}}%
\pgfpathlineto{\pgfqpoint{2.754802in}{2.500439in}}%
\pgfpathlineto{\pgfqpoint{2.763076in}{2.509342in}}%
\pgfpathlineto{\pgfqpoint{2.771342in}{2.518329in}}%
\pgfpathlineto{\pgfqpoint{2.757949in}{2.528201in}}%
\pgfpathlineto{\pgfqpoint{2.744555in}{2.538199in}}%
\pgfpathlineto{\pgfqpoint{2.731160in}{2.548325in}}%
\pgfpathlineto{\pgfqpoint{2.717765in}{2.558580in}}%
\pgfpathlineto{\pgfqpoint{2.709485in}{2.549591in}}%
\pgfpathlineto{\pgfqpoint{2.701196in}{2.540692in}}%
\pgfpathlineto{\pgfqpoint{2.692899in}{2.531881in}}%
\pgfpathlineto{\pgfqpoint{2.684594in}{2.523158in}}%
\pgfpathclose%
\pgfusepath{fill}%
\end{pgfscope}%
\begin{pgfscope}%
\pgfpathrectangle{\pgfqpoint{1.150000in}{0.150000in}}{\pgfqpoint{5.700000in}{5.700000in}}%
\pgfusepath{clip}%
\pgfsetbuttcap%
\pgfsetroundjoin%
\definecolor{currentfill}{rgb}{0.283187,0.125848,0.444960}%
\pgfsetfillcolor{currentfill}%
\pgfsetfillopacity{0.700000}%
\pgfsetlinewidth{0.000000pt}%
\definecolor{currentstroke}{rgb}{0.000000,0.000000,0.000000}%
\pgfsetstrokecolor{currentstroke}%
\pgfsetdash{}{0pt}%
\pgfpathmoveto{\pgfqpoint{2.878495in}{2.443786in}}%
\pgfpathlineto{\pgfqpoint{2.891891in}{2.435007in}}%
\pgfpathlineto{\pgfqpoint{2.905288in}{2.426345in}}%
\pgfpathlineto{\pgfqpoint{2.918686in}{2.417799in}}%
\pgfpathlineto{\pgfqpoint{2.932085in}{2.409368in}}%
\pgfpathlineto{\pgfqpoint{2.940302in}{2.418416in}}%
\pgfpathlineto{\pgfqpoint{2.948512in}{2.427535in}}%
\pgfpathlineto{\pgfqpoint{2.956715in}{2.436724in}}%
\pgfpathlineto{\pgfqpoint{2.964911in}{2.445986in}}%
\pgfpathlineto{\pgfqpoint{2.951525in}{2.454432in}}%
\pgfpathlineto{\pgfqpoint{2.938140in}{2.462994in}}%
\pgfpathlineto{\pgfqpoint{2.924756in}{2.471671in}}%
\pgfpathlineto{\pgfqpoint{2.911373in}{2.480465in}}%
\pgfpathlineto{\pgfqpoint{2.903165in}{2.471181in}}%
\pgfpathlineto{\pgfqpoint{2.894949in}{2.461974in}}%
\pgfpathlineto{\pgfqpoint{2.886726in}{2.452842in}}%
\pgfpathlineto{\pgfqpoint{2.878495in}{2.443786in}}%
\pgfpathclose%
\pgfusepath{fill}%
\end{pgfscope}%
\begin{pgfscope}%
\pgfpathrectangle{\pgfqpoint{1.150000in}{0.150000in}}{\pgfqpoint{5.700000in}{5.700000in}}%
\pgfusepath{clip}%
\pgfsetbuttcap%
\pgfsetroundjoin%
\definecolor{currentfill}{rgb}{0.281887,0.150881,0.465405}%
\pgfsetfillcolor{currentfill}%
\pgfsetfillopacity{0.700000}%
\pgfsetlinewidth{0.000000pt}%
\definecolor{currentstroke}{rgb}{0.000000,0.000000,0.000000}%
\pgfsetstrokecolor{currentstroke}%
\pgfsetdash{}{0pt}%
\pgfpathmoveto{\pgfqpoint{4.338004in}{2.480861in}}%
\pgfpathlineto{\pgfqpoint{4.351618in}{2.477941in}}%
\pgfpathlineto{\pgfqpoint{4.365238in}{2.475098in}}%
\pgfpathlineto{\pgfqpoint{4.378866in}{2.472330in}}%
\pgfpathlineto{\pgfqpoint{4.392501in}{2.469638in}}%
\pgfpathlineto{\pgfqpoint{4.400223in}{2.479308in}}%
\pgfpathlineto{\pgfqpoint{4.407941in}{2.489076in}}%
\pgfpathlineto{\pgfqpoint{4.415655in}{2.498949in}}%
\pgfpathlineto{\pgfqpoint{4.423364in}{2.508931in}}%
\pgfpathlineto{\pgfqpoint{4.409741in}{2.511882in}}%
\pgfpathlineto{\pgfqpoint{4.396125in}{2.514909in}}%
\pgfpathlineto{\pgfqpoint{4.382516in}{2.518012in}}%
\pgfpathlineto{\pgfqpoint{4.368914in}{2.521190in}}%
\pgfpathlineto{\pgfqpoint{4.361193in}{2.510942in}}%
\pgfpathlineto{\pgfqpoint{4.353468in}{2.500808in}}%
\pgfpathlineto{\pgfqpoint{4.345738in}{2.490782in}}%
\pgfpathlineto{\pgfqpoint{4.338004in}{2.480861in}}%
\pgfpathclose%
\pgfusepath{fill}%
\end{pgfscope}%
\begin{pgfscope}%
\pgfpathrectangle{\pgfqpoint{1.150000in}{0.150000in}}{\pgfqpoint{5.700000in}{5.700000in}}%
\pgfusepath{clip}%
\pgfsetbuttcap%
\pgfsetroundjoin%
\definecolor{currentfill}{rgb}{0.277134,0.185228,0.489898}%
\pgfsetfillcolor{currentfill}%
\pgfsetfillopacity{0.700000}%
\pgfsetlinewidth{0.000000pt}%
\definecolor{currentstroke}{rgb}{0.000000,0.000000,0.000000}%
\pgfsetstrokecolor{currentstroke}%
\pgfsetdash{}{0pt}%
\pgfpathmoveto{\pgfqpoint{4.648634in}{2.556765in}}%
\pgfpathlineto{\pgfqpoint{4.662327in}{2.554197in}}%
\pgfpathlineto{\pgfqpoint{4.676027in}{2.551702in}}%
\pgfpathlineto{\pgfqpoint{4.689735in}{2.549280in}}%
\pgfpathlineto{\pgfqpoint{4.703451in}{2.546929in}}%
\pgfpathlineto{\pgfqpoint{4.711074in}{2.556874in}}%
\pgfpathlineto{\pgfqpoint{4.718694in}{2.566961in}}%
\pgfpathlineto{\pgfqpoint{4.726312in}{2.577195in}}%
\pgfpathlineto{\pgfqpoint{4.733926in}{2.587583in}}%
\pgfpathlineto{\pgfqpoint{4.720225in}{2.590253in}}%
\pgfpathlineto{\pgfqpoint{4.706531in}{2.592996in}}%
\pgfpathlineto{\pgfqpoint{4.692844in}{2.595810in}}%
\pgfpathlineto{\pgfqpoint{4.679165in}{2.598697in}}%
\pgfpathlineto{\pgfqpoint{4.671537in}{2.587982in}}%
\pgfpathlineto{\pgfqpoint{4.663906in}{2.577426in}}%
\pgfpathlineto{\pgfqpoint{4.656272in}{2.567022in}}%
\pgfpathlineto{\pgfqpoint{4.648634in}{2.556765in}}%
\pgfpathclose%
\pgfusepath{fill}%
\end{pgfscope}%
\begin{pgfscope}%
\pgfpathrectangle{\pgfqpoint{1.150000in}{0.150000in}}{\pgfqpoint{5.700000in}{5.700000in}}%
\pgfusepath{clip}%
\pgfsetbuttcap%
\pgfsetroundjoin%
\definecolor{currentfill}{rgb}{0.260571,0.246922,0.522828}%
\pgfsetfillcolor{currentfill}%
\pgfsetfillopacity{0.700000}%
\pgfsetlinewidth{0.000000pt}%
\definecolor{currentstroke}{rgb}{0.000000,0.000000,0.000000}%
\pgfsetstrokecolor{currentstroke}%
\pgfsetdash{}{0pt}%
\pgfpathmoveto{\pgfqpoint{5.044715in}{2.676158in}}%
\pgfpathlineto{\pgfqpoint{5.058505in}{2.673608in}}%
\pgfpathlineto{\pgfqpoint{5.072303in}{2.671127in}}%
\pgfpathlineto{\pgfqpoint{5.086109in}{2.668715in}}%
\pgfpathlineto{\pgfqpoint{5.099924in}{2.666372in}}%
\pgfpathlineto{\pgfqpoint{5.107440in}{2.677371in}}%
\pgfpathlineto{\pgfqpoint{5.114955in}{2.688589in}}%
\pgfpathlineto{\pgfqpoint{5.122471in}{2.700036in}}%
\pgfpathlineto{\pgfqpoint{5.129987in}{2.711717in}}%
\pgfpathlineto{\pgfqpoint{5.116190in}{2.714460in}}%
\pgfpathlineto{\pgfqpoint{5.102400in}{2.717272in}}%
\pgfpathlineto{\pgfqpoint{5.088618in}{2.720153in}}%
\pgfpathlineto{\pgfqpoint{5.074845in}{2.723103in}}%
\pgfpathlineto{\pgfqpoint{5.067312in}{2.711015in}}%
\pgfpathlineto{\pgfqpoint{5.059780in}{2.699166in}}%
\pgfpathlineto{\pgfqpoint{5.052247in}{2.687549in}}%
\pgfpathlineto{\pgfqpoint{5.044715in}{2.676158in}}%
\pgfpathclose%
\pgfusepath{fill}%
\end{pgfscope}%
\begin{pgfscope}%
\pgfpathrectangle{\pgfqpoint{1.150000in}{0.150000in}}{\pgfqpoint{5.700000in}{5.700000in}}%
\pgfusepath{clip}%
\pgfsetbuttcap%
\pgfsetroundjoin%
\definecolor{currentfill}{rgb}{0.282656,0.100196,0.422160}%
\pgfsetfillcolor{currentfill}%
\pgfsetfillopacity{0.700000}%
\pgfsetlinewidth{0.000000pt}%
\definecolor{currentstroke}{rgb}{0.000000,0.000000,0.000000}%
\pgfsetstrokecolor{currentstroke}%
\pgfsetdash{}{0pt}%
\pgfpathmoveto{\pgfqpoint{3.802207in}{2.383713in}}%
\pgfpathlineto{\pgfqpoint{3.815699in}{2.379536in}}%
\pgfpathlineto{\pgfqpoint{3.829197in}{2.375444in}}%
\pgfpathlineto{\pgfqpoint{3.842700in}{2.371437in}}%
\pgfpathlineto{\pgfqpoint{3.856209in}{2.367514in}}%
\pgfpathlineto{\pgfqpoint{3.864110in}{2.377127in}}%
\pgfpathlineto{\pgfqpoint{3.872004in}{2.386797in}}%
\pgfpathlineto{\pgfqpoint{3.879894in}{2.396526in}}%
\pgfpathlineto{\pgfqpoint{3.887778in}{2.406319in}}%
\pgfpathlineto{\pgfqpoint{3.874279in}{2.410401in}}%
\pgfpathlineto{\pgfqpoint{3.860786in}{2.414566in}}%
\pgfpathlineto{\pgfqpoint{3.847298in}{2.418816in}}%
\pgfpathlineto{\pgfqpoint{3.833816in}{2.423151in}}%
\pgfpathlineto{\pgfqpoint{3.825922in}{2.413193in}}%
\pgfpathlineto{\pgfqpoint{3.818022in}{2.403302in}}%
\pgfpathlineto{\pgfqpoint{3.810117in}{2.393477in}}%
\pgfpathlineto{\pgfqpoint{3.802207in}{2.383713in}}%
\pgfpathclose%
\pgfusepath{fill}%
\end{pgfscope}%
\begin{pgfscope}%
\pgfpathrectangle{\pgfqpoint{1.150000in}{0.150000in}}{\pgfqpoint{5.700000in}{5.700000in}}%
\pgfusepath{clip}%
\pgfsetbuttcap%
\pgfsetroundjoin%
\definecolor{currentfill}{rgb}{0.283197,0.115680,0.436115}%
\pgfsetfillcolor{currentfill}%
\pgfsetfillopacity{0.700000}%
\pgfsetlinewidth{0.000000pt}%
\definecolor{currentstroke}{rgb}{0.000000,0.000000,0.000000}%
\pgfsetstrokecolor{currentstroke}%
\pgfsetdash{}{0pt}%
\pgfpathmoveto{\pgfqpoint{4.027376in}{2.415095in}}%
\pgfpathlineto{\pgfqpoint{4.040918in}{2.411576in}}%
\pgfpathlineto{\pgfqpoint{4.054465in}{2.408138in}}%
\pgfpathlineto{\pgfqpoint{4.068019in}{2.404781in}}%
\pgfpathlineto{\pgfqpoint{4.081580in}{2.401504in}}%
\pgfpathlineto{\pgfqpoint{4.089406in}{2.411089in}}%
\pgfpathlineto{\pgfqpoint{4.097227in}{2.420743in}}%
\pgfpathlineto{\pgfqpoint{4.105043in}{2.430470in}}%
\pgfpathlineto{\pgfqpoint{4.112853in}{2.440273in}}%
\pgfpathlineto{\pgfqpoint{4.099304in}{2.443749in}}%
\pgfpathlineto{\pgfqpoint{4.085760in}{2.447305in}}%
\pgfpathlineto{\pgfqpoint{4.072223in}{2.450942in}}%
\pgfpathlineto{\pgfqpoint{4.058692in}{2.454659in}}%
\pgfpathlineto{\pgfqpoint{4.050871in}{2.444650in}}%
\pgfpathlineto{\pgfqpoint{4.043044in}{2.434722in}}%
\pgfpathlineto{\pgfqpoint{4.035213in}{2.424872in}}%
\pgfpathlineto{\pgfqpoint{4.027376in}{2.415095in}}%
\pgfpathclose%
\pgfusepath{fill}%
\end{pgfscope}%
\begin{pgfscope}%
\pgfpathrectangle{\pgfqpoint{1.150000in}{0.150000in}}{\pgfqpoint{5.700000in}{5.700000in}}%
\pgfusepath{clip}%
\pgfsetbuttcap%
\pgfsetroundjoin%
\definecolor{currentfill}{rgb}{0.281924,0.089666,0.412415}%
\pgfsetfillcolor{currentfill}%
\pgfsetfillopacity{0.700000}%
\pgfsetlinewidth{0.000000pt}%
\definecolor{currentstroke}{rgb}{0.000000,0.000000,0.000000}%
\pgfsetstrokecolor{currentstroke}%
\pgfsetdash{}{0pt}%
\pgfpathmoveto{\pgfqpoint{3.576972in}{2.361316in}}%
\pgfpathlineto{\pgfqpoint{3.590425in}{2.356343in}}%
\pgfpathlineto{\pgfqpoint{3.603881in}{2.351461in}}%
\pgfpathlineto{\pgfqpoint{3.617343in}{2.346668in}}%
\pgfpathlineto{\pgfqpoint{3.630809in}{2.341964in}}%
\pgfpathlineto{\pgfqpoint{3.638785in}{2.351554in}}%
\pgfpathlineto{\pgfqpoint{3.646756in}{2.361194in}}%
\pgfpathlineto{\pgfqpoint{3.654720in}{2.370887in}}%
\pgfpathlineto{\pgfqpoint{3.662679in}{2.380635in}}%
\pgfpathlineto{\pgfqpoint{3.649223in}{2.385457in}}%
\pgfpathlineto{\pgfqpoint{3.635771in}{2.390367in}}%
\pgfpathlineto{\pgfqpoint{3.622324in}{2.395367in}}%
\pgfpathlineto{\pgfqpoint{3.608882in}{2.400457in}}%
\pgfpathlineto{\pgfqpoint{3.600913in}{2.390584in}}%
\pgfpathlineto{\pgfqpoint{3.592939in}{2.380771in}}%
\pgfpathlineto{\pgfqpoint{3.584959in}{2.371016in}}%
\pgfpathlineto{\pgfqpoint{3.576972in}{2.361316in}}%
\pgfpathclose%
\pgfusepath{fill}%
\end{pgfscope}%
\begin{pgfscope}%
\pgfpathrectangle{\pgfqpoint{1.150000in}{0.150000in}}{\pgfqpoint{5.700000in}{5.700000in}}%
\pgfusepath{clip}%
\pgfsetbuttcap%
\pgfsetroundjoin%
\definecolor{currentfill}{rgb}{0.187231,0.414746,0.556547}%
\pgfsetfillcolor{currentfill}%
\pgfsetfillopacity{0.700000}%
\pgfsetlinewidth{0.000000pt}%
\definecolor{currentstroke}{rgb}{0.000000,0.000000,0.000000}%
\pgfsetstrokecolor{currentstroke}%
\pgfsetdash{}{0pt}%
\pgfpathmoveto{\pgfqpoint{5.813738in}{3.080373in}}%
\pgfpathlineto{\pgfqpoint{5.827650in}{3.075906in}}%
\pgfpathlineto{\pgfqpoint{5.841570in}{3.071504in}}%
\pgfpathlineto{\pgfqpoint{5.855498in}{3.067166in}}%
\pgfpathlineto{\pgfqpoint{5.869434in}{3.062893in}}%
\pgfpathlineto{\pgfqpoint{5.876987in}{3.081885in}}%
\pgfpathlineto{\pgfqpoint{5.884555in}{3.101374in}}%
\pgfpathlineto{\pgfqpoint{5.892138in}{3.121369in}}%
\pgfpathlineto{\pgfqpoint{5.878218in}{3.126076in}}%
\pgfpathlineto{\pgfqpoint{5.864306in}{3.130848in}}%
\pgfpathlineto{\pgfqpoint{5.850402in}{3.135685in}}%
\pgfpathlineto{\pgfqpoint{5.836505in}{3.140586in}}%
\pgfpathlineto{\pgfqpoint{5.828901in}{3.120007in}}%
\pgfpathlineto{\pgfqpoint{5.821312in}{3.099939in}}%
\pgfpathlineto{\pgfqpoint{5.813738in}{3.080373in}}%
\pgfpathclose%
\pgfusepath{fill}%
\end{pgfscope}%
\begin{pgfscope}%
\pgfpathrectangle{\pgfqpoint{1.150000in}{0.150000in}}{\pgfqpoint{5.700000in}{5.700000in}}%
\pgfusepath{clip}%
\pgfsetbuttcap%
\pgfsetroundjoin%
\definecolor{currentfill}{rgb}{0.266580,0.228262,0.514349}%
\pgfsetfillcolor{currentfill}%
\pgfsetfillopacity{0.700000}%
\pgfsetlinewidth{0.000000pt}%
\definecolor{currentstroke}{rgb}{0.000000,0.000000,0.000000}%
\pgfsetstrokecolor{currentstroke}%
\pgfsetdash{}{0pt}%
\pgfpathmoveto{\pgfqpoint{4.959434in}{2.642079in}}%
\pgfpathlineto{\pgfqpoint{4.973208in}{2.639631in}}%
\pgfpathlineto{\pgfqpoint{4.986990in}{2.637253in}}%
\pgfpathlineto{\pgfqpoint{5.000780in}{2.634945in}}%
\pgfpathlineto{\pgfqpoint{5.014579in}{2.632706in}}%
\pgfpathlineto{\pgfqpoint{5.022114in}{2.643266in}}%
\pgfpathlineto{\pgfqpoint{5.029648in}{2.654023in}}%
\pgfpathlineto{\pgfqpoint{5.037182in}{2.664985in}}%
\pgfpathlineto{\pgfqpoint{5.044715in}{2.676158in}}%
\pgfpathlineto{\pgfqpoint{5.030933in}{2.678778in}}%
\pgfpathlineto{\pgfqpoint{5.017159in}{2.681466in}}%
\pgfpathlineto{\pgfqpoint{5.003393in}{2.684225in}}%
\pgfpathlineto{\pgfqpoint{4.989635in}{2.687052in}}%
\pgfpathlineto{\pgfqpoint{4.982086in}{2.675492in}}%
\pgfpathlineto{\pgfqpoint{4.974536in}{2.664148in}}%
\pgfpathlineto{\pgfqpoint{4.966986in}{2.653012in}}%
\pgfpathlineto{\pgfqpoint{4.959434in}{2.642079in}}%
\pgfpathclose%
\pgfusepath{fill}%
\end{pgfscope}%
\begin{pgfscope}%
\pgfpathrectangle{\pgfqpoint{1.150000in}{0.150000in}}{\pgfqpoint{5.700000in}{5.700000in}}%
\pgfusepath{clip}%
\pgfsetbuttcap%
\pgfsetroundjoin%
\definecolor{currentfill}{rgb}{0.278826,0.175490,0.483397}%
\pgfsetfillcolor{currentfill}%
\pgfsetfillopacity{0.700000}%
\pgfsetlinewidth{0.000000pt}%
\definecolor{currentstroke}{rgb}{0.000000,0.000000,0.000000}%
\pgfsetstrokecolor{currentstroke}%
\pgfsetdash{}{0pt}%
\pgfpathmoveto{\pgfqpoint{4.563303in}{2.526895in}}%
\pgfpathlineto{\pgfqpoint{4.576979in}{2.524335in}}%
\pgfpathlineto{\pgfqpoint{4.590662in}{2.521848in}}%
\pgfpathlineto{\pgfqpoint{4.604353in}{2.519435in}}%
\pgfpathlineto{\pgfqpoint{4.618051in}{2.517094in}}%
\pgfpathlineto{\pgfqpoint{4.625702in}{2.526819in}}%
\pgfpathlineto{\pgfqpoint{4.633350in}{2.536669in}}%
\pgfpathlineto{\pgfqpoint{4.640994in}{2.546649in}}%
\pgfpathlineto{\pgfqpoint{4.648634in}{2.556765in}}%
\pgfpathlineto{\pgfqpoint{4.634949in}{2.559405in}}%
\pgfpathlineto{\pgfqpoint{4.621272in}{2.562118in}}%
\pgfpathlineto{\pgfqpoint{4.607602in}{2.564905in}}%
\pgfpathlineto{\pgfqpoint{4.593939in}{2.567765in}}%
\pgfpathlineto{\pgfqpoint{4.586285in}{2.557342in}}%
\pgfpathlineto{\pgfqpoint{4.578628in}{2.547060in}}%
\pgfpathlineto{\pgfqpoint{4.570967in}{2.536913in}}%
\pgfpathlineto{\pgfqpoint{4.563303in}{2.526895in}}%
\pgfpathclose%
\pgfusepath{fill}%
\end{pgfscope}%
\begin{pgfscope}%
\pgfpathrectangle{\pgfqpoint{1.150000in}{0.150000in}}{\pgfqpoint{5.700000in}{5.700000in}}%
\pgfusepath{clip}%
\pgfsetbuttcap%
\pgfsetroundjoin%
\definecolor{currentfill}{rgb}{0.282623,0.140926,0.457517}%
\pgfsetfillcolor{currentfill}%
\pgfsetfillopacity{0.700000}%
\pgfsetlinewidth{0.000000pt}%
\definecolor{currentstroke}{rgb}{0.000000,0.000000,0.000000}%
\pgfsetstrokecolor{currentstroke}%
\pgfsetdash{}{0pt}%
\pgfpathmoveto{\pgfqpoint{4.252589in}{2.453614in}}%
\pgfpathlineto{\pgfqpoint{4.266187in}{2.450626in}}%
\pgfpathlineto{\pgfqpoint{4.279791in}{2.447716in}}%
\pgfpathlineto{\pgfqpoint{4.293403in}{2.444882in}}%
\pgfpathlineto{\pgfqpoint{4.307021in}{2.442125in}}%
\pgfpathlineto{\pgfqpoint{4.314774in}{2.451676in}}%
\pgfpathlineto{\pgfqpoint{4.322522in}{2.461312in}}%
\pgfpathlineto{\pgfqpoint{4.330265in}{2.471039in}}%
\pgfpathlineto{\pgfqpoint{4.338004in}{2.480861in}}%
\pgfpathlineto{\pgfqpoint{4.324397in}{2.483857in}}%
\pgfpathlineto{\pgfqpoint{4.310797in}{2.486930in}}%
\pgfpathlineto{\pgfqpoint{4.297204in}{2.490079in}}%
\pgfpathlineto{\pgfqpoint{4.283618in}{2.493306in}}%
\pgfpathlineto{\pgfqpoint{4.275868in}{2.483238in}}%
\pgfpathlineto{\pgfqpoint{4.268113in}{2.473269in}}%
\pgfpathlineto{\pgfqpoint{4.260353in}{2.463396in}}%
\pgfpathlineto{\pgfqpoint{4.252589in}{2.453614in}}%
\pgfpathclose%
\pgfusepath{fill}%
\end{pgfscope}%
\begin{pgfscope}%
\pgfpathrectangle{\pgfqpoint{1.150000in}{0.150000in}}{\pgfqpoint{5.700000in}{5.700000in}}%
\pgfusepath{clip}%
\pgfsetbuttcap%
\pgfsetroundjoin%
\definecolor{currentfill}{rgb}{0.282290,0.145912,0.461510}%
\pgfsetfillcolor{currentfill}%
\pgfsetfillopacity{0.700000}%
\pgfsetlinewidth{0.000000pt}%
\definecolor{currentstroke}{rgb}{0.000000,0.000000,0.000000}%
\pgfsetstrokecolor{currentstroke}%
\pgfsetdash{}{0pt}%
\pgfpathmoveto{\pgfqpoint{2.738230in}{2.482884in}}%
\pgfpathlineto{\pgfqpoint{2.751638in}{2.473132in}}%
\pgfpathlineto{\pgfqpoint{2.765046in}{2.463505in}}%
\pgfpathlineto{\pgfqpoint{2.778454in}{2.454001in}}%
\pgfpathlineto{\pgfqpoint{2.791862in}{2.444620in}}%
\pgfpathlineto{\pgfqpoint{2.800137in}{2.453369in}}%
\pgfpathlineto{\pgfqpoint{2.808405in}{2.462196in}}%
\pgfpathlineto{\pgfqpoint{2.816664in}{2.471101in}}%
\pgfpathlineto{\pgfqpoint{2.824916in}{2.480086in}}%
\pgfpathlineto{\pgfqpoint{2.811523in}{2.489462in}}%
\pgfpathlineto{\pgfqpoint{2.798129in}{2.498960in}}%
\pgfpathlineto{\pgfqpoint{2.784736in}{2.508582in}}%
\pgfpathlineto{\pgfqpoint{2.771342in}{2.518329in}}%
\pgfpathlineto{\pgfqpoint{2.763076in}{2.509342in}}%
\pgfpathlineto{\pgfqpoint{2.754802in}{2.500439in}}%
\pgfpathlineto{\pgfqpoint{2.746520in}{2.491620in}}%
\pgfpathlineto{\pgfqpoint{2.738230in}{2.482884in}}%
\pgfpathclose%
\pgfusepath{fill}%
\end{pgfscope}%
\begin{pgfscope}%
\pgfpathrectangle{\pgfqpoint{1.150000in}{0.150000in}}{\pgfqpoint{5.700000in}{5.700000in}}%
\pgfusepath{clip}%
\pgfsetbuttcap%
\pgfsetroundjoin%
\definecolor{currentfill}{rgb}{0.281924,0.089666,0.412415}%
\pgfsetfillcolor{currentfill}%
\pgfsetfillopacity{0.700000}%
\pgfsetlinewidth{0.000000pt}%
\definecolor{currentstroke}{rgb}{0.000000,0.000000,0.000000}%
\pgfsetstrokecolor{currentstroke}%
\pgfsetdash{}{0pt}%
\pgfpathmoveto{\pgfqpoint{3.211841in}{2.362992in}}%
\pgfpathlineto{\pgfqpoint{3.225250in}{2.356339in}}%
\pgfpathlineto{\pgfqpoint{3.238662in}{2.349788in}}%
\pgfpathlineto{\pgfqpoint{3.252077in}{2.343337in}}%
\pgfpathlineto{\pgfqpoint{3.265495in}{2.336986in}}%
\pgfpathlineto{\pgfqpoint{3.273597in}{2.346376in}}%
\pgfpathlineto{\pgfqpoint{3.281692in}{2.355820in}}%
\pgfpathlineto{\pgfqpoint{3.289782in}{2.365318in}}%
\pgfpathlineto{\pgfqpoint{3.297864in}{2.374872in}}%
\pgfpathlineto{\pgfqpoint{3.284458in}{2.381280in}}%
\pgfpathlineto{\pgfqpoint{3.271054in}{2.387788in}}%
\pgfpathlineto{\pgfqpoint{3.257653in}{2.394396in}}%
\pgfpathlineto{\pgfqpoint{3.244256in}{2.401105in}}%
\pgfpathlineto{\pgfqpoint{3.236162in}{2.391487in}}%
\pgfpathlineto{\pgfqpoint{3.228061in}{2.381930in}}%
\pgfpathlineto{\pgfqpoint{3.219955in}{2.372432in}}%
\pgfpathlineto{\pgfqpoint{3.211841in}{2.362992in}}%
\pgfpathclose%
\pgfusepath{fill}%
\end{pgfscope}%
\begin{pgfscope}%
\pgfpathrectangle{\pgfqpoint{1.150000in}{0.150000in}}{\pgfqpoint{5.700000in}{5.700000in}}%
\pgfusepath{clip}%
\pgfsetbuttcap%
\pgfsetroundjoin%
\definecolor{currentfill}{rgb}{0.282656,0.100196,0.422160}%
\pgfsetfillcolor{currentfill}%
\pgfsetfillopacity{0.700000}%
\pgfsetlinewidth{0.000000pt}%
\definecolor{currentstroke}{rgb}{0.000000,0.000000,0.000000}%
\pgfsetstrokecolor{currentstroke}%
\pgfsetdash{}{0pt}%
\pgfpathmoveto{\pgfqpoint{3.072051in}{2.382445in}}%
\pgfpathlineto{\pgfqpoint{3.085452in}{2.374994in}}%
\pgfpathlineto{\pgfqpoint{3.098855in}{2.367650in}}%
\pgfpathlineto{\pgfqpoint{3.112260in}{2.360413in}}%
\pgfpathlineto{\pgfqpoint{3.125668in}{2.353280in}}%
\pgfpathlineto{\pgfqpoint{3.133819in}{2.362529in}}%
\pgfpathlineto{\pgfqpoint{3.141964in}{2.371835in}}%
\pgfpathlineto{\pgfqpoint{3.150102in}{2.381201in}}%
\pgfpathlineto{\pgfqpoint{3.158233in}{2.390628in}}%
\pgfpathlineto{\pgfqpoint{3.144838in}{2.397797in}}%
\pgfpathlineto{\pgfqpoint{3.131444in}{2.405071in}}%
\pgfpathlineto{\pgfqpoint{3.118053in}{2.412451in}}%
\pgfpathlineto{\pgfqpoint{3.104665in}{2.419937in}}%
\pgfpathlineto{\pgfqpoint{3.096522in}{2.410467in}}%
\pgfpathlineto{\pgfqpoint{3.088372in}{2.401062in}}%
\pgfpathlineto{\pgfqpoint{3.080215in}{2.391722in}}%
\pgfpathlineto{\pgfqpoint{3.072051in}{2.382445in}}%
\pgfpathclose%
\pgfusepath{fill}%
\end{pgfscope}%
\begin{pgfscope}%
\pgfpathrectangle{\pgfqpoint{1.150000in}{0.150000in}}{\pgfqpoint{5.700000in}{5.700000in}}%
\pgfusepath{clip}%
\pgfsetbuttcap%
\pgfsetroundjoin%
\definecolor{currentfill}{rgb}{0.281446,0.084320,0.407414}%
\pgfsetfillcolor{currentfill}%
\pgfsetfillopacity{0.700000}%
\pgfsetlinewidth{0.000000pt}%
\definecolor{currentstroke}{rgb}{0.000000,0.000000,0.000000}%
\pgfsetstrokecolor{currentstroke}%
\pgfsetdash{}{0pt}%
\pgfpathmoveto{\pgfqpoint{3.351525in}{2.350227in}}%
\pgfpathlineto{\pgfqpoint{3.364949in}{2.344309in}}%
\pgfpathlineto{\pgfqpoint{3.378376in}{2.338488in}}%
\pgfpathlineto{\pgfqpoint{3.391807in}{2.332762in}}%
\pgfpathlineto{\pgfqpoint{3.405243in}{2.327132in}}%
\pgfpathlineto{\pgfqpoint{3.413297in}{2.336605in}}%
\pgfpathlineto{\pgfqpoint{3.421346in}{2.346127in}}%
\pgfpathlineto{\pgfqpoint{3.429389in}{2.355700in}}%
\pgfpathlineto{\pgfqpoint{3.437426in}{2.365325in}}%
\pgfpathlineto{\pgfqpoint{3.424001in}{2.371033in}}%
\pgfpathlineto{\pgfqpoint{3.410581in}{2.376836in}}%
\pgfpathlineto{\pgfqpoint{3.397164in}{2.382734in}}%
\pgfpathlineto{\pgfqpoint{3.383751in}{2.388729in}}%
\pgfpathlineto{\pgfqpoint{3.375703in}{2.379019in}}%
\pgfpathlineto{\pgfqpoint{3.367650in}{2.369366in}}%
\pgfpathlineto{\pgfqpoint{3.359591in}{2.359769in}}%
\pgfpathlineto{\pgfqpoint{3.351525in}{2.350227in}}%
\pgfpathclose%
\pgfusepath{fill}%
\end{pgfscope}%
\begin{pgfscope}%
\pgfpathrectangle{\pgfqpoint{1.150000in}{0.150000in}}{\pgfqpoint{5.700000in}{5.700000in}}%
\pgfusepath{clip}%
\pgfsetbuttcap%
\pgfsetroundjoin%
\definecolor{currentfill}{rgb}{0.225863,0.330805,0.547314}%
\pgfsetfillcolor{currentfill}%
\pgfsetfillopacity{0.700000}%
\pgfsetlinewidth{0.000000pt}%
\definecolor{currentstroke}{rgb}{0.000000,0.000000,0.000000}%
\pgfsetstrokecolor{currentstroke}%
\pgfsetdash{}{0pt}%
\pgfpathmoveto{\pgfqpoint{5.526765in}{2.860227in}}%
\pgfpathlineto{\pgfqpoint{5.540662in}{2.857097in}}%
\pgfpathlineto{\pgfqpoint{5.554567in}{2.854033in}}%
\pgfpathlineto{\pgfqpoint{5.568480in}{2.851035in}}%
\pgfpathlineto{\pgfqpoint{5.582403in}{2.848103in}}%
\pgfpathlineto{\pgfqpoint{5.589858in}{2.862027in}}%
\pgfpathlineto{\pgfqpoint{5.597321in}{2.876303in}}%
\pgfpathlineto{\pgfqpoint{5.604790in}{2.890941in}}%
\pgfpathlineto{\pgfqpoint{5.612267in}{2.905949in}}%
\pgfpathlineto{\pgfqpoint{5.598365in}{2.909381in}}%
\pgfpathlineto{\pgfqpoint{5.584472in}{2.912879in}}%
\pgfpathlineto{\pgfqpoint{5.570588in}{2.916443in}}%
\pgfpathlineto{\pgfqpoint{5.556711in}{2.920073in}}%
\pgfpathlineto{\pgfqpoint{5.549214in}{2.904558in}}%
\pgfpathlineto{\pgfqpoint{5.541724in}{2.889417in}}%
\pgfpathlineto{\pgfqpoint{5.534242in}{2.874643in}}%
\pgfpathlineto{\pgfqpoint{5.526765in}{2.860227in}}%
\pgfpathclose%
\pgfusepath{fill}%
\end{pgfscope}%
\begin{pgfscope}%
\pgfpathrectangle{\pgfqpoint{1.150000in}{0.150000in}}{\pgfqpoint{5.700000in}{5.700000in}}%
\pgfusepath{clip}%
\pgfsetbuttcap%
\pgfsetroundjoin%
\definecolor{currentfill}{rgb}{0.269308,0.218818,0.509577}%
\pgfsetfillcolor{currentfill}%
\pgfsetfillopacity{0.700000}%
\pgfsetlinewidth{0.000000pt}%
\definecolor{currentstroke}{rgb}{0.000000,0.000000,0.000000}%
\pgfsetstrokecolor{currentstroke}%
\pgfsetdash{}{0pt}%
\pgfpathmoveto{\pgfqpoint{4.874135in}{2.609290in}}%
\pgfpathlineto{\pgfqpoint{4.887893in}{2.606921in}}%
\pgfpathlineto{\pgfqpoint{4.901658in}{2.604623in}}%
\pgfpathlineto{\pgfqpoint{4.915432in}{2.602395in}}%
\pgfpathlineto{\pgfqpoint{4.929213in}{2.600238in}}%
\pgfpathlineto{\pgfqpoint{4.936771in}{2.610428in}}%
\pgfpathlineto{\pgfqpoint{4.944327in}{2.620794in}}%
\pgfpathlineto{\pgfqpoint{4.951881in}{2.631342in}}%
\pgfpathlineto{\pgfqpoint{4.959434in}{2.642079in}}%
\pgfpathlineto{\pgfqpoint{4.945668in}{2.644597in}}%
\pgfpathlineto{\pgfqpoint{4.931910in}{2.647185in}}%
\pgfpathlineto{\pgfqpoint{4.918160in}{2.649844in}}%
\pgfpathlineto{\pgfqpoint{4.904418in}{2.652572in}}%
\pgfpathlineto{\pgfqpoint{4.896850in}{2.641467in}}%
\pgfpathlineto{\pgfqpoint{4.889280in}{2.630556in}}%
\pgfpathlineto{\pgfqpoint{4.881709in}{2.619832in}}%
\pgfpathlineto{\pgfqpoint{4.874135in}{2.609290in}}%
\pgfpathclose%
\pgfusepath{fill}%
\end{pgfscope}%
\begin{pgfscope}%
\pgfpathrectangle{\pgfqpoint{1.150000in}{0.150000in}}{\pgfqpoint{5.700000in}{5.700000in}}%
\pgfusepath{clip}%
\pgfsetbuttcap%
\pgfsetroundjoin%
\definecolor{currentfill}{rgb}{0.216210,0.351535,0.550627}%
\pgfsetfillcolor{currentfill}%
\pgfsetfillopacity{0.700000}%
\pgfsetlinewidth{0.000000pt}%
\definecolor{currentstroke}{rgb}{0.000000,0.000000,0.000000}%
\pgfsetstrokecolor{currentstroke}%
\pgfsetdash{}{0pt}%
\pgfpathmoveto{\pgfqpoint{5.612267in}{2.905949in}}%
\pgfpathlineto{\pgfqpoint{5.626176in}{2.902582in}}%
\pgfpathlineto{\pgfqpoint{5.640094in}{2.899281in}}%
\pgfpathlineto{\pgfqpoint{5.654021in}{2.896045in}}%
\pgfpathlineto{\pgfqpoint{5.667955in}{2.892875in}}%
\pgfpathlineto{\pgfqpoint{5.675419in}{2.907750in}}%
\pgfpathlineto{\pgfqpoint{5.682890in}{2.923009in}}%
\pgfpathlineto{\pgfqpoint{5.690371in}{2.938662in}}%
\pgfpathlineto{\pgfqpoint{5.697861in}{2.954719in}}%
\pgfpathlineto{\pgfqpoint{5.683947in}{2.958410in}}%
\pgfpathlineto{\pgfqpoint{5.670042in}{2.962166in}}%
\pgfpathlineto{\pgfqpoint{5.656145in}{2.965988in}}%
\pgfpathlineto{\pgfqpoint{5.642256in}{2.969875in}}%
\pgfpathlineto{\pgfqpoint{5.634745in}{2.953290in}}%
\pgfpathlineto{\pgfqpoint{5.627244in}{2.937114in}}%
\pgfpathlineto{\pgfqpoint{5.619751in}{2.921337in}}%
\pgfpathlineto{\pgfqpoint{5.612267in}{2.905949in}}%
\pgfpathclose%
\pgfusepath{fill}%
\end{pgfscope}%
\begin{pgfscope}%
\pgfpathrectangle{\pgfqpoint{1.150000in}{0.150000in}}{\pgfqpoint{5.700000in}{5.700000in}}%
\pgfusepath{clip}%
\pgfsetbuttcap%
\pgfsetroundjoin%
\definecolor{currentfill}{rgb}{0.233603,0.313828,0.543914}%
\pgfsetfillcolor{currentfill}%
\pgfsetfillopacity{0.700000}%
\pgfsetlinewidth{0.000000pt}%
\definecolor{currentstroke}{rgb}{0.000000,0.000000,0.000000}%
\pgfsetstrokecolor{currentstroke}%
\pgfsetdash{}{0pt}%
\pgfpathmoveto{\pgfqpoint{5.441333in}{2.817215in}}%
\pgfpathlineto{\pgfqpoint{5.455216in}{2.814300in}}%
\pgfpathlineto{\pgfqpoint{5.469108in}{2.811452in}}%
\pgfpathlineto{\pgfqpoint{5.483008in}{2.808670in}}%
\pgfpathlineto{\pgfqpoint{5.496916in}{2.805955in}}%
\pgfpathlineto{\pgfqpoint{5.504371in}{2.819031in}}%
\pgfpathlineto{\pgfqpoint{5.511830in}{2.832429in}}%
\pgfpathlineto{\pgfqpoint{5.519295in}{2.846158in}}%
\pgfpathlineto{\pgfqpoint{5.526765in}{2.860227in}}%
\pgfpathlineto{\pgfqpoint{5.512877in}{2.863423in}}%
\pgfpathlineto{\pgfqpoint{5.498997in}{2.866685in}}%
\pgfpathlineto{\pgfqpoint{5.485125in}{2.870013in}}%
\pgfpathlineto{\pgfqpoint{5.471262in}{2.873408in}}%
\pgfpathlineto{\pgfqpoint{5.463771in}{2.858852in}}%
\pgfpathlineto{\pgfqpoint{5.456287in}{2.844640in}}%
\pgfpathlineto{\pgfqpoint{5.448807in}{2.830764in}}%
\pgfpathlineto{\pgfqpoint{5.441333in}{2.817215in}}%
\pgfpathclose%
\pgfusepath{fill}%
\end{pgfscope}%
\begin{pgfscope}%
\pgfpathrectangle{\pgfqpoint{1.150000in}{0.150000in}}{\pgfqpoint{5.700000in}{5.700000in}}%
\pgfusepath{clip}%
\pgfsetbuttcap%
\pgfsetroundjoin%
\definecolor{currentfill}{rgb}{0.281924,0.089666,0.412415}%
\pgfsetfillcolor{currentfill}%
\pgfsetfillopacity{0.700000}%
\pgfsetlinewidth{0.000000pt}%
\definecolor{currentstroke}{rgb}{0.000000,0.000000,0.000000}%
\pgfsetstrokecolor{currentstroke}%
\pgfsetdash{}{0pt}%
\pgfpathmoveto{\pgfqpoint{3.716553in}{2.362232in}}%
\pgfpathlineto{\pgfqpoint{3.730034in}{2.357849in}}%
\pgfpathlineto{\pgfqpoint{3.743521in}{2.353553in}}%
\pgfpathlineto{\pgfqpoint{3.757012in}{2.349343in}}%
\pgfpathlineto{\pgfqpoint{3.770509in}{2.345219in}}%
\pgfpathlineto{\pgfqpoint{3.778442in}{2.354764in}}%
\pgfpathlineto{\pgfqpoint{3.786369in}{2.364360in}}%
\pgfpathlineto{\pgfqpoint{3.794291in}{2.374009in}}%
\pgfpathlineto{\pgfqpoint{3.802207in}{2.383713in}}%
\pgfpathlineto{\pgfqpoint{3.788720in}{2.387976in}}%
\pgfpathlineto{\pgfqpoint{3.775238in}{2.392323in}}%
\pgfpathlineto{\pgfqpoint{3.761762in}{2.396757in}}%
\pgfpathlineto{\pgfqpoint{3.748291in}{2.401278in}}%
\pgfpathlineto{\pgfqpoint{3.740365in}{2.391428in}}%
\pgfpathlineto{\pgfqpoint{3.732433in}{2.381638in}}%
\pgfpathlineto{\pgfqpoint{3.724496in}{2.371907in}}%
\pgfpathlineto{\pgfqpoint{3.716553in}{2.362232in}}%
\pgfpathclose%
\pgfusepath{fill}%
\end{pgfscope}%
\begin{pgfscope}%
\pgfpathrectangle{\pgfqpoint{1.150000in}{0.150000in}}{\pgfqpoint{5.700000in}{5.700000in}}%
\pgfusepath{clip}%
\pgfsetbuttcap%
\pgfsetroundjoin%
\definecolor{currentfill}{rgb}{0.283091,0.110553,0.431554}%
\pgfsetfillcolor{currentfill}%
\pgfsetfillopacity{0.700000}%
\pgfsetlinewidth{0.000000pt}%
\definecolor{currentstroke}{rgb}{0.000000,0.000000,0.000000}%
\pgfsetstrokecolor{currentstroke}%
\pgfsetdash{}{0pt}%
\pgfpathmoveto{\pgfqpoint{2.932085in}{2.409368in}}%
\pgfpathlineto{\pgfqpoint{2.945485in}{2.401050in}}%
\pgfpathlineto{\pgfqpoint{2.958887in}{2.392846in}}%
\pgfpathlineto{\pgfqpoint{2.972290in}{2.384755in}}%
\pgfpathlineto{\pgfqpoint{2.985694in}{2.376775in}}%
\pgfpathlineto{\pgfqpoint{2.993898in}{2.385815in}}%
\pgfpathlineto{\pgfqpoint{3.002095in}{2.394921in}}%
\pgfpathlineto{\pgfqpoint{3.010285in}{2.404092in}}%
\pgfpathlineto{\pgfqpoint{3.018468in}{2.413330in}}%
\pgfpathlineto{\pgfqpoint{3.005076in}{2.421326in}}%
\pgfpathlineto{\pgfqpoint{2.991686in}{2.429433in}}%
\pgfpathlineto{\pgfqpoint{2.978298in}{2.437653in}}%
\pgfpathlineto{\pgfqpoint{2.964911in}{2.445986in}}%
\pgfpathlineto{\pgfqpoint{2.956715in}{2.436724in}}%
\pgfpathlineto{\pgfqpoint{2.948512in}{2.427535in}}%
\pgfpathlineto{\pgfqpoint{2.940302in}{2.418416in}}%
\pgfpathlineto{\pgfqpoint{2.932085in}{2.409368in}}%
\pgfpathclose%
\pgfusepath{fill}%
\end{pgfscope}%
\begin{pgfscope}%
\pgfpathrectangle{\pgfqpoint{1.150000in}{0.150000in}}{\pgfqpoint{5.700000in}{5.700000in}}%
\pgfusepath{clip}%
\pgfsetbuttcap%
\pgfsetroundjoin%
\definecolor{currentfill}{rgb}{0.276194,0.190074,0.493001}%
\pgfsetfillcolor{currentfill}%
\pgfsetfillopacity{0.700000}%
\pgfsetlinewidth{0.000000pt}%
\definecolor{currentstroke}{rgb}{0.000000,0.000000,0.000000}%
\pgfsetstrokecolor{currentstroke}%
\pgfsetdash{}{0pt}%
\pgfpathmoveto{\pgfqpoint{2.543844in}{2.576216in}}%
\pgfpathlineto{\pgfqpoint{2.557280in}{2.564857in}}%
\pgfpathlineto{\pgfqpoint{2.570714in}{2.553635in}}%
\pgfpathlineto{\pgfqpoint{2.584147in}{2.542551in}}%
\pgfpathlineto{\pgfqpoint{2.597578in}{2.531603in}}%
\pgfpathlineto{\pgfqpoint{2.605933in}{2.539941in}}%
\pgfpathlineto{\pgfqpoint{2.614279in}{2.548371in}}%
\pgfpathlineto{\pgfqpoint{2.622617in}{2.556895in}}%
\pgfpathlineto{\pgfqpoint{2.630945in}{2.565511in}}%
\pgfpathlineto{\pgfqpoint{2.617530in}{2.576433in}}%
\pgfpathlineto{\pgfqpoint{2.604114in}{2.587490in}}%
\pgfpathlineto{\pgfqpoint{2.590696in}{2.598685in}}%
\pgfpathlineto{\pgfqpoint{2.577276in}{2.610018in}}%
\pgfpathlineto{\pgfqpoint{2.568931in}{2.601421in}}%
\pgfpathlineto{\pgfqpoint{2.560577in}{2.592921in}}%
\pgfpathlineto{\pgfqpoint{2.552215in}{2.584520in}}%
\pgfpathlineto{\pgfqpoint{2.543844in}{2.576216in}}%
\pgfpathclose%
\pgfusepath{fill}%
\end{pgfscope}%
\begin{pgfscope}%
\pgfpathrectangle{\pgfqpoint{1.150000in}{0.150000in}}{\pgfqpoint{5.700000in}{5.700000in}}%
\pgfusepath{clip}%
\pgfsetbuttcap%
\pgfsetroundjoin%
\definecolor{currentfill}{rgb}{0.282910,0.105393,0.426902}%
\pgfsetfillcolor{currentfill}%
\pgfsetfillopacity{0.700000}%
\pgfsetlinewidth{0.000000pt}%
\definecolor{currentstroke}{rgb}{0.000000,0.000000,0.000000}%
\pgfsetstrokecolor{currentstroke}%
\pgfsetdash{}{0pt}%
\pgfpathmoveto{\pgfqpoint{3.941830in}{2.390828in}}%
\pgfpathlineto{\pgfqpoint{3.955357in}{2.387161in}}%
\pgfpathlineto{\pgfqpoint{3.968891in}{2.383577in}}%
\pgfpathlineto{\pgfqpoint{3.982431in}{2.380075in}}%
\pgfpathlineto{\pgfqpoint{3.995977in}{2.376654in}}%
\pgfpathlineto{\pgfqpoint{4.003835in}{2.386171in}}%
\pgfpathlineto{\pgfqpoint{4.011687in}{2.395748in}}%
\pgfpathlineto{\pgfqpoint{4.019534in}{2.405388in}}%
\pgfpathlineto{\pgfqpoint{4.027376in}{2.415095in}}%
\pgfpathlineto{\pgfqpoint{4.013841in}{2.418695in}}%
\pgfpathlineto{\pgfqpoint{4.000312in}{2.422375in}}%
\pgfpathlineto{\pgfqpoint{3.986788in}{2.426138in}}%
\pgfpathlineto{\pgfqpoint{3.973271in}{2.429983in}}%
\pgfpathlineto{\pgfqpoint{3.965419in}{2.420090in}}%
\pgfpathlineto{\pgfqpoint{3.957561in}{2.410269in}}%
\pgfpathlineto{\pgfqpoint{3.949698in}{2.400516in}}%
\pgfpathlineto{\pgfqpoint{3.941830in}{2.390828in}}%
\pgfpathclose%
\pgfusepath{fill}%
\end{pgfscope}%
\begin{pgfscope}%
\pgfpathrectangle{\pgfqpoint{1.150000in}{0.150000in}}{\pgfqpoint{5.700000in}{5.700000in}}%
\pgfusepath{clip}%
\pgfsetbuttcap%
\pgfsetroundjoin%
\definecolor{currentfill}{rgb}{0.241237,0.296485,0.539709}%
\pgfsetfillcolor{currentfill}%
\pgfsetfillopacity{0.700000}%
\pgfsetlinewidth{0.000000pt}%
\definecolor{currentstroke}{rgb}{0.000000,0.000000,0.000000}%
\pgfsetstrokecolor{currentstroke}%
\pgfsetdash{}{0pt}%
\pgfpathmoveto{\pgfqpoint{5.355949in}{2.776599in}}%
\pgfpathlineto{\pgfqpoint{5.369818in}{2.773877in}}%
\pgfpathlineto{\pgfqpoint{5.383696in}{2.771223in}}%
\pgfpathlineto{\pgfqpoint{5.397582in}{2.768635in}}%
\pgfpathlineto{\pgfqpoint{5.411476in}{2.766115in}}%
\pgfpathlineto{\pgfqpoint{5.418935in}{2.778442in}}%
\pgfpathlineto{\pgfqpoint{5.426397in}{2.791062in}}%
\pgfpathlineto{\pgfqpoint{5.433863in}{2.803983in}}%
\pgfpathlineto{\pgfqpoint{5.441333in}{2.817215in}}%
\pgfpathlineto{\pgfqpoint{5.427458in}{2.820196in}}%
\pgfpathlineto{\pgfqpoint{5.413592in}{2.823244in}}%
\pgfpathlineto{\pgfqpoint{5.399734in}{2.826359in}}%
\pgfpathlineto{\pgfqpoint{5.385884in}{2.829540in}}%
\pgfpathlineto{\pgfqpoint{5.378394in}{2.815841in}}%
\pgfpathlineto{\pgfqpoint{5.370909in}{2.802457in}}%
\pgfpathlineto{\pgfqpoint{5.363428in}{2.789379in}}%
\pgfpathlineto{\pgfqpoint{5.355949in}{2.776599in}}%
\pgfpathclose%
\pgfusepath{fill}%
\end{pgfscope}%
\begin{pgfscope}%
\pgfpathrectangle{\pgfqpoint{1.150000in}{0.150000in}}{\pgfqpoint{5.700000in}{5.700000in}}%
\pgfusepath{clip}%
\pgfsetbuttcap%
\pgfsetroundjoin%
\definecolor{currentfill}{rgb}{0.206756,0.371758,0.553117}%
\pgfsetfillcolor{currentfill}%
\pgfsetfillopacity{0.700000}%
\pgfsetlinewidth{0.000000pt}%
\definecolor{currentstroke}{rgb}{0.000000,0.000000,0.000000}%
\pgfsetstrokecolor{currentstroke}%
\pgfsetdash{}{0pt}%
\pgfpathmoveto{\pgfqpoint{5.697861in}{2.954719in}}%
\pgfpathlineto{\pgfqpoint{5.711782in}{2.951094in}}%
\pgfpathlineto{\pgfqpoint{5.725713in}{2.947533in}}%
\pgfpathlineto{\pgfqpoint{5.739651in}{2.944038in}}%
\pgfpathlineto{\pgfqpoint{5.753598in}{2.940608in}}%
\pgfpathlineto{\pgfqpoint{5.761076in}{2.956544in}}%
\pgfpathlineto{\pgfqpoint{5.768564in}{2.972899in}}%
\pgfpathlineto{\pgfqpoint{5.776063in}{2.989682in}}%
\pgfpathlineto{\pgfqpoint{5.783573in}{3.006903in}}%
\pgfpathlineto{\pgfqpoint{5.769648in}{3.010874in}}%
\pgfpathlineto{\pgfqpoint{5.755731in}{3.014910in}}%
\pgfpathlineto{\pgfqpoint{5.741822in}{3.019011in}}%
\pgfpathlineto{\pgfqpoint{5.727922in}{3.023177in}}%
\pgfpathlineto{\pgfqpoint{5.720390in}{3.005408in}}%
\pgfpathlineto{\pgfqpoint{5.712870in}{2.988082in}}%
\pgfpathlineto{\pgfqpoint{5.705360in}{2.971189in}}%
\pgfpathlineto{\pgfqpoint{5.697861in}{2.954719in}}%
\pgfpathclose%
\pgfusepath{fill}%
\end{pgfscope}%
\begin{pgfscope}%
\pgfpathrectangle{\pgfqpoint{1.150000in}{0.150000in}}{\pgfqpoint{5.700000in}{5.700000in}}%
\pgfusepath{clip}%
\pgfsetbuttcap%
\pgfsetroundjoin%
\definecolor{currentfill}{rgb}{0.280255,0.165693,0.476498}%
\pgfsetfillcolor{currentfill}%
\pgfsetfillopacity{0.700000}%
\pgfsetlinewidth{0.000000pt}%
\definecolor{currentstroke}{rgb}{0.000000,0.000000,0.000000}%
\pgfsetstrokecolor{currentstroke}%
\pgfsetdash{}{0pt}%
\pgfpathmoveto{\pgfqpoint{4.477927in}{2.497878in}}%
\pgfpathlineto{\pgfqpoint{4.491586in}{2.495301in}}%
\pgfpathlineto{\pgfqpoint{4.505252in}{2.492799in}}%
\pgfpathlineto{\pgfqpoint{4.518925in}{2.490371in}}%
\pgfpathlineto{\pgfqpoint{4.532607in}{2.488017in}}%
\pgfpathlineto{\pgfqpoint{4.540287in}{2.497568in}}%
\pgfpathlineto{\pgfqpoint{4.547963in}{2.507228in}}%
\pgfpathlineto{\pgfqpoint{4.555635in}{2.517002in}}%
\pgfpathlineto{\pgfqpoint{4.563303in}{2.526895in}}%
\pgfpathlineto{\pgfqpoint{4.549635in}{2.529529in}}%
\pgfpathlineto{\pgfqpoint{4.535974in}{2.532237in}}%
\pgfpathlineto{\pgfqpoint{4.522320in}{2.535019in}}%
\pgfpathlineto{\pgfqpoint{4.508674in}{2.537874in}}%
\pgfpathlineto{\pgfqpoint{4.500993in}{2.527694in}}%
\pgfpathlineto{\pgfqpoint{4.493308in}{2.517638in}}%
\pgfpathlineto{\pgfqpoint{4.485620in}{2.507701in}}%
\pgfpathlineto{\pgfqpoint{4.477927in}{2.497878in}}%
\pgfpathclose%
\pgfusepath{fill}%
\end{pgfscope}%
\begin{pgfscope}%
\pgfpathrectangle{\pgfqpoint{1.150000in}{0.150000in}}{\pgfqpoint{5.700000in}{5.700000in}}%
\pgfusepath{clip}%
\pgfsetbuttcap%
\pgfsetroundjoin%
\definecolor{currentfill}{rgb}{0.281446,0.084320,0.407414}%
\pgfsetfillcolor{currentfill}%
\pgfsetfillopacity{0.700000}%
\pgfsetlinewidth{0.000000pt}%
\definecolor{currentstroke}{rgb}{0.000000,0.000000,0.000000}%
\pgfsetstrokecolor{currentstroke}%
\pgfsetdash{}{0pt}%
\pgfpathmoveto{\pgfqpoint{3.491163in}{2.343435in}}%
\pgfpathlineto{\pgfqpoint{3.504608in}{2.338195in}}%
\pgfpathlineto{\pgfqpoint{3.518057in}{2.333047in}}%
\pgfpathlineto{\pgfqpoint{3.531511in}{2.327991in}}%
\pgfpathlineto{\pgfqpoint{3.544969in}{2.323025in}}%
\pgfpathlineto{\pgfqpoint{3.552978in}{2.332526in}}%
\pgfpathlineto{\pgfqpoint{3.560982in}{2.342073in}}%
\pgfpathlineto{\pgfqpoint{3.568980in}{2.351669in}}%
\pgfpathlineto{\pgfqpoint{3.576972in}{2.361316in}}%
\pgfpathlineto{\pgfqpoint{3.563525in}{2.366379in}}%
\pgfpathlineto{\pgfqpoint{3.550081in}{2.371533in}}%
\pgfpathlineto{\pgfqpoint{3.536643in}{2.376778in}}%
\pgfpathlineto{\pgfqpoint{3.523208in}{2.382116in}}%
\pgfpathlineto{\pgfqpoint{3.515206in}{2.372364in}}%
\pgfpathlineto{\pgfqpoint{3.507198in}{2.362668in}}%
\pgfpathlineto{\pgfqpoint{3.499183in}{2.353026in}}%
\pgfpathlineto{\pgfqpoint{3.491163in}{2.343435in}}%
\pgfpathclose%
\pgfusepath{fill}%
\end{pgfscope}%
\begin{pgfscope}%
\pgfpathrectangle{\pgfqpoint{1.150000in}{0.150000in}}{\pgfqpoint{5.700000in}{5.700000in}}%
\pgfusepath{clip}%
\pgfsetbuttcap%
\pgfsetroundjoin%
\definecolor{currentfill}{rgb}{0.283187,0.125848,0.444960}%
\pgfsetfillcolor{currentfill}%
\pgfsetfillopacity{0.700000}%
\pgfsetlinewidth{0.000000pt}%
\definecolor{currentstroke}{rgb}{0.000000,0.000000,0.000000}%
\pgfsetstrokecolor{currentstroke}%
\pgfsetdash{}{0pt}%
\pgfpathmoveto{\pgfqpoint{4.167116in}{2.427163in}}%
\pgfpathlineto{\pgfqpoint{4.180698in}{2.424083in}}%
\pgfpathlineto{\pgfqpoint{4.194286in}{2.421081in}}%
\pgfpathlineto{\pgfqpoint{4.207882in}{2.418157in}}%
\pgfpathlineto{\pgfqpoint{4.221484in}{2.415311in}}%
\pgfpathlineto{\pgfqpoint{4.229268in}{2.424771in}}%
\pgfpathlineto{\pgfqpoint{4.237046in}{2.434306in}}%
\pgfpathlineto{\pgfqpoint{4.244820in}{2.443919in}}%
\pgfpathlineto{\pgfqpoint{4.252589in}{2.453614in}}%
\pgfpathlineto{\pgfqpoint{4.238998in}{2.456679in}}%
\pgfpathlineto{\pgfqpoint{4.225414in}{2.459822in}}%
\pgfpathlineto{\pgfqpoint{4.211837in}{2.463043in}}%
\pgfpathlineto{\pgfqpoint{4.198266in}{2.466342in}}%
\pgfpathlineto{\pgfqpoint{4.190486in}{2.456421in}}%
\pgfpathlineto{\pgfqpoint{4.182701in}{2.446586in}}%
\pgfpathlineto{\pgfqpoint{4.174911in}{2.436835in}}%
\pgfpathlineto{\pgfqpoint{4.167116in}{2.427163in}}%
\pgfpathclose%
\pgfusepath{fill}%
\end{pgfscope}%
\begin{pgfscope}%
\pgfpathrectangle{\pgfqpoint{1.150000in}{0.150000in}}{\pgfqpoint{5.700000in}{5.700000in}}%
\pgfusepath{clip}%
\pgfsetbuttcap%
\pgfsetroundjoin%
\definecolor{currentfill}{rgb}{0.248629,0.278775,0.534556}%
\pgfsetfillcolor{currentfill}%
\pgfsetfillopacity{0.700000}%
\pgfsetlinewidth{0.000000pt}%
\definecolor{currentstroke}{rgb}{0.000000,0.000000,0.000000}%
\pgfsetstrokecolor{currentstroke}%
\pgfsetdash{}{0pt}%
\pgfpathmoveto{\pgfqpoint{5.270597in}{2.738092in}}%
\pgfpathlineto{\pgfqpoint{5.284451in}{2.735541in}}%
\pgfpathlineto{\pgfqpoint{5.298314in}{2.733059in}}%
\pgfpathlineto{\pgfqpoint{5.312185in}{2.730643in}}%
\pgfpathlineto{\pgfqpoint{5.326064in}{2.728296in}}%
\pgfpathlineto{\pgfqpoint{5.333532in}{2.739965in}}%
\pgfpathlineto{\pgfqpoint{5.341002in}{2.751900in}}%
\pgfpathlineto{\pgfqpoint{5.348474in}{2.764109in}}%
\pgfpathlineto{\pgfqpoint{5.355949in}{2.776599in}}%
\pgfpathlineto{\pgfqpoint{5.342089in}{2.779388in}}%
\pgfpathlineto{\pgfqpoint{5.328237in}{2.782243in}}%
\pgfpathlineto{\pgfqpoint{5.314393in}{2.785166in}}%
\pgfpathlineto{\pgfqpoint{5.300557in}{2.788157in}}%
\pgfpathlineto{\pgfqpoint{5.293063in}{2.775219in}}%
\pgfpathlineto{\pgfqpoint{5.285572in}{2.762567in}}%
\pgfpathlineto{\pgfqpoint{5.278083in}{2.750194in}}%
\pgfpathlineto{\pgfqpoint{5.270597in}{2.738092in}}%
\pgfpathclose%
\pgfusepath{fill}%
\end{pgfscope}%
\begin{pgfscope}%
\pgfpathrectangle{\pgfqpoint{1.150000in}{0.150000in}}{\pgfqpoint{5.700000in}{5.700000in}}%
\pgfusepath{clip}%
\pgfsetbuttcap%
\pgfsetroundjoin%
\definecolor{currentfill}{rgb}{0.195860,0.395433,0.555276}%
\pgfsetfillcolor{currentfill}%
\pgfsetfillopacity{0.700000}%
\pgfsetlinewidth{0.000000pt}%
\definecolor{currentstroke}{rgb}{0.000000,0.000000,0.000000}%
\pgfsetstrokecolor{currentstroke}%
\pgfsetdash{}{0pt}%
\pgfpathmoveto{\pgfqpoint{5.783573in}{3.006903in}}%
\pgfpathlineto{\pgfqpoint{5.797507in}{3.002997in}}%
\pgfpathlineto{\pgfqpoint{5.811448in}{2.999156in}}%
\pgfpathlineto{\pgfqpoint{5.825398in}{2.995379in}}%
\pgfpathlineto{\pgfqpoint{5.839357in}{2.991667in}}%
\pgfpathlineto{\pgfqpoint{5.846857in}{3.008782in}}%
\pgfpathlineto{\pgfqpoint{5.854369in}{3.026352in}}%
\pgfpathlineto{\pgfqpoint{5.861895in}{3.044385in}}%
\pgfpathlineto{\pgfqpoint{5.869434in}{3.062893in}}%
\pgfpathlineto{\pgfqpoint{5.855498in}{3.067166in}}%
\pgfpathlineto{\pgfqpoint{5.841570in}{3.071504in}}%
\pgfpathlineto{\pgfqpoint{5.827650in}{3.075906in}}%
\pgfpathlineto{\pgfqpoint{5.813738in}{3.080373in}}%
\pgfpathlineto{\pgfqpoint{5.806177in}{3.061297in}}%
\pgfpathlineto{\pgfqpoint{5.798630in}{3.042700in}}%
\pgfpathlineto{\pgfqpoint{5.791095in}{3.024572in}}%
\pgfpathlineto{\pgfqpoint{5.783573in}{3.006903in}}%
\pgfpathclose%
\pgfusepath{fill}%
\end{pgfscope}%
\begin{pgfscope}%
\pgfpathrectangle{\pgfqpoint{1.150000in}{0.150000in}}{\pgfqpoint{5.700000in}{5.700000in}}%
\pgfusepath{clip}%
\pgfsetbuttcap%
\pgfsetroundjoin%
\definecolor{currentfill}{rgb}{0.273006,0.204520,0.501721}%
\pgfsetfillcolor{currentfill}%
\pgfsetfillopacity{0.700000}%
\pgfsetlinewidth{0.000000pt}%
\definecolor{currentstroke}{rgb}{0.000000,0.000000,0.000000}%
\pgfsetstrokecolor{currentstroke}%
\pgfsetdash{}{0pt}%
\pgfpathmoveto{\pgfqpoint{4.788810in}{2.577621in}}%
\pgfpathlineto{\pgfqpoint{4.802550in}{2.575309in}}%
\pgfpathlineto{\pgfqpoint{4.816299in}{2.573068in}}%
\pgfpathlineto{\pgfqpoint{4.830055in}{2.570899in}}%
\pgfpathlineto{\pgfqpoint{4.843820in}{2.568800in}}%
\pgfpathlineto{\pgfqpoint{4.851402in}{2.578682in}}%
\pgfpathlineto{\pgfqpoint{4.858982in}{2.588721in}}%
\pgfpathlineto{\pgfqpoint{4.866560in}{2.598921in}}%
\pgfpathlineto{\pgfqpoint{4.874135in}{2.609290in}}%
\pgfpathlineto{\pgfqpoint{4.860386in}{2.611729in}}%
\pgfpathlineto{\pgfqpoint{4.846645in}{2.614239in}}%
\pgfpathlineto{\pgfqpoint{4.832911in}{2.616819in}}%
\pgfpathlineto{\pgfqpoint{4.819185in}{2.619471in}}%
\pgfpathlineto{\pgfqpoint{4.811595in}{2.608755in}}%
\pgfpathlineto{\pgfqpoint{4.804002in}{2.598212in}}%
\pgfpathlineto{\pgfqpoint{4.796407in}{2.587836in}}%
\pgfpathlineto{\pgfqpoint{4.788810in}{2.577621in}}%
\pgfpathclose%
\pgfusepath{fill}%
\end{pgfscope}%
\begin{pgfscope}%
\pgfpathrectangle{\pgfqpoint{1.150000in}{0.150000in}}{\pgfqpoint{5.700000in}{5.700000in}}%
\pgfusepath{clip}%
\pgfsetbuttcap%
\pgfsetroundjoin%
\definecolor{currentfill}{rgb}{0.283072,0.130895,0.449241}%
\pgfsetfillcolor{currentfill}%
\pgfsetfillopacity{0.700000}%
\pgfsetlinewidth{0.000000pt}%
\definecolor{currentstroke}{rgb}{0.000000,0.000000,0.000000}%
\pgfsetstrokecolor{currentstroke}%
\pgfsetdash{}{0pt}%
\pgfpathmoveto{\pgfqpoint{2.791862in}{2.444620in}}%
\pgfpathlineto{\pgfqpoint{2.805270in}{2.435360in}}%
\pgfpathlineto{\pgfqpoint{2.818679in}{2.426221in}}%
\pgfpathlineto{\pgfqpoint{2.832088in}{2.417201in}}%
\pgfpathlineto{\pgfqpoint{2.845497in}{2.408300in}}%
\pgfpathlineto{\pgfqpoint{2.853758in}{2.417061in}}%
\pgfpathlineto{\pgfqpoint{2.862011in}{2.425896in}}%
\pgfpathlineto{\pgfqpoint{2.870257in}{2.434804in}}%
\pgfpathlineto{\pgfqpoint{2.878495in}{2.443786in}}%
\pgfpathlineto{\pgfqpoint{2.865099in}{2.452682in}}%
\pgfpathlineto{\pgfqpoint{2.851705in}{2.461697in}}%
\pgfpathlineto{\pgfqpoint{2.838310in}{2.470831in}}%
\pgfpathlineto{\pgfqpoint{2.824916in}{2.480086in}}%
\pgfpathlineto{\pgfqpoint{2.816664in}{2.471101in}}%
\pgfpathlineto{\pgfqpoint{2.808405in}{2.462196in}}%
\pgfpathlineto{\pgfqpoint{2.800137in}{2.453369in}}%
\pgfpathlineto{\pgfqpoint{2.791862in}{2.444620in}}%
\pgfpathclose%
\pgfusepath{fill}%
\end{pgfscope}%
\begin{pgfscope}%
\pgfpathrectangle{\pgfqpoint{1.150000in}{0.150000in}}{\pgfqpoint{5.700000in}{5.700000in}}%
\pgfusepath{clip}%
\pgfsetbuttcap%
\pgfsetroundjoin%
\definecolor{currentfill}{rgb}{0.253935,0.265254,0.529983}%
\pgfsetfillcolor{currentfill}%
\pgfsetfillopacity{0.700000}%
\pgfsetlinewidth{0.000000pt}%
\definecolor{currentstroke}{rgb}{0.000000,0.000000,0.000000}%
\pgfsetstrokecolor{currentstroke}%
\pgfsetdash{}{0pt}%
\pgfpathmoveto{\pgfqpoint{5.185260in}{2.701429in}}%
\pgfpathlineto{\pgfqpoint{5.199098in}{2.699028in}}%
\pgfpathlineto{\pgfqpoint{5.212946in}{2.696695in}}%
\pgfpathlineto{\pgfqpoint{5.226801in}{2.694430in}}%
\pgfpathlineto{\pgfqpoint{5.240666in}{2.692233in}}%
\pgfpathlineto{\pgfqpoint{5.248147in}{2.703330in}}%
\pgfpathlineto{\pgfqpoint{5.255629in}{2.714667in}}%
\pgfpathlineto{\pgfqpoint{5.263112in}{2.726252in}}%
\pgfpathlineto{\pgfqpoint{5.270597in}{2.738092in}}%
\pgfpathlineto{\pgfqpoint{5.256751in}{2.740710in}}%
\pgfpathlineto{\pgfqpoint{5.242914in}{2.743395in}}%
\pgfpathlineto{\pgfqpoint{5.229084in}{2.746149in}}%
\pgfpathlineto{\pgfqpoint{5.215263in}{2.748970in}}%
\pgfpathlineto{\pgfqpoint{5.207761in}{2.736703in}}%
\pgfpathlineto{\pgfqpoint{5.200259in}{2.724695in}}%
\pgfpathlineto{\pgfqpoint{5.192759in}{2.712940in}}%
\pgfpathlineto{\pgfqpoint{5.185260in}{2.701429in}}%
\pgfpathclose%
\pgfusepath{fill}%
\end{pgfscope}%
\begin{pgfscope}%
\pgfpathrectangle{\pgfqpoint{1.150000in}{0.150000in}}{\pgfqpoint{5.700000in}{5.700000in}}%
\pgfusepath{clip}%
\pgfsetbuttcap%
\pgfsetroundjoin%
\definecolor{currentfill}{rgb}{0.279574,0.170599,0.479997}%
\pgfsetfillcolor{currentfill}%
\pgfsetfillopacity{0.700000}%
\pgfsetlinewidth{0.000000pt}%
\definecolor{currentstroke}{rgb}{0.000000,0.000000,0.000000}%
\pgfsetstrokecolor{currentstroke}%
\pgfsetdash{}{0pt}%
\pgfpathmoveto{\pgfqpoint{2.597578in}{2.531603in}}%
\pgfpathlineto{\pgfqpoint{2.611008in}{2.520789in}}%
\pgfpathlineto{\pgfqpoint{2.624437in}{2.510109in}}%
\pgfpathlineto{\pgfqpoint{2.637864in}{2.499562in}}%
\pgfpathlineto{\pgfqpoint{2.651291in}{2.489145in}}%
\pgfpathlineto{\pgfqpoint{2.659629in}{2.497517in}}%
\pgfpathlineto{\pgfqpoint{2.667959in}{2.505976in}}%
\pgfpathlineto{\pgfqpoint{2.676281in}{2.514523in}}%
\pgfpathlineto{\pgfqpoint{2.684594in}{2.523158in}}%
\pgfpathlineto{\pgfqpoint{2.671183in}{2.533548in}}%
\pgfpathlineto{\pgfqpoint{2.657772in}{2.544070in}}%
\pgfpathlineto{\pgfqpoint{2.644359in}{2.554724in}}%
\pgfpathlineto{\pgfqpoint{2.630945in}{2.565511in}}%
\pgfpathlineto{\pgfqpoint{2.622617in}{2.556895in}}%
\pgfpathlineto{\pgfqpoint{2.614279in}{2.548371in}}%
\pgfpathlineto{\pgfqpoint{2.605933in}{2.539941in}}%
\pgfpathlineto{\pgfqpoint{2.597578in}{2.531603in}}%
\pgfpathclose%
\pgfusepath{fill}%
\end{pgfscope}%
\begin{pgfscope}%
\pgfpathrectangle{\pgfqpoint{1.150000in}{0.150000in}}{\pgfqpoint{5.700000in}{5.700000in}}%
\pgfusepath{clip}%
\pgfsetbuttcap%
\pgfsetroundjoin%
\definecolor{currentfill}{rgb}{0.282656,0.100196,0.422160}%
\pgfsetfillcolor{currentfill}%
\pgfsetfillopacity{0.700000}%
\pgfsetlinewidth{0.000000pt}%
\definecolor{currentstroke}{rgb}{0.000000,0.000000,0.000000}%
\pgfsetstrokecolor{currentstroke}%
\pgfsetdash{}{0pt}%
\pgfpathmoveto{\pgfqpoint{3.856209in}{2.367514in}}%
\pgfpathlineto{\pgfqpoint{3.869724in}{2.363674in}}%
\pgfpathlineto{\pgfqpoint{3.883244in}{2.359919in}}%
\pgfpathlineto{\pgfqpoint{3.896770in}{2.356246in}}%
\pgfpathlineto{\pgfqpoint{3.910302in}{2.352656in}}%
\pgfpathlineto{\pgfqpoint{3.918192in}{2.362118in}}%
\pgfpathlineto{\pgfqpoint{3.926077in}{2.371632in}}%
\pgfpathlineto{\pgfqpoint{3.933956in}{2.381201in}}%
\pgfpathlineto{\pgfqpoint{3.941830in}{2.390828in}}%
\pgfpathlineto{\pgfqpoint{3.928308in}{2.394576in}}%
\pgfpathlineto{\pgfqpoint{3.914792in}{2.398407in}}%
\pgfpathlineto{\pgfqpoint{3.901282in}{2.402322in}}%
\pgfpathlineto{\pgfqpoint{3.887778in}{2.406319in}}%
\pgfpathlineto{\pgfqpoint{3.879894in}{2.396526in}}%
\pgfpathlineto{\pgfqpoint{3.872004in}{2.386797in}}%
\pgfpathlineto{\pgfqpoint{3.864110in}{2.377127in}}%
\pgfpathlineto{\pgfqpoint{3.856209in}{2.367514in}}%
\pgfpathclose%
\pgfusepath{fill}%
\end{pgfscope}%
\begin{pgfscope}%
\pgfpathrectangle{\pgfqpoint{1.150000in}{0.150000in}}{\pgfqpoint{5.700000in}{5.700000in}}%
\pgfusepath{clip}%
\pgfsetbuttcap%
\pgfsetroundjoin%
\definecolor{currentfill}{rgb}{0.281887,0.150881,0.465405}%
\pgfsetfillcolor{currentfill}%
\pgfsetfillopacity{0.700000}%
\pgfsetlinewidth{0.000000pt}%
\definecolor{currentstroke}{rgb}{0.000000,0.000000,0.000000}%
\pgfsetstrokecolor{currentstroke}%
\pgfsetdash{}{0pt}%
\pgfpathmoveto{\pgfqpoint{4.392501in}{2.469638in}}%
\pgfpathlineto{\pgfqpoint{4.406143in}{2.467022in}}%
\pgfpathlineto{\pgfqpoint{4.419792in}{2.464481in}}%
\pgfpathlineto{\pgfqpoint{4.433449in}{2.462015in}}%
\pgfpathlineto{\pgfqpoint{4.447113in}{2.459623in}}%
\pgfpathlineto{\pgfqpoint{4.454823in}{2.469041in}}%
\pgfpathlineto{\pgfqpoint{4.462528in}{2.478552in}}%
\pgfpathlineto{\pgfqpoint{4.470230in}{2.488163in}}%
\pgfpathlineto{\pgfqpoint{4.477927in}{2.497878in}}%
\pgfpathlineto{\pgfqpoint{4.464275in}{2.500529in}}%
\pgfpathlineto{\pgfqpoint{4.450631in}{2.503254in}}%
\pgfpathlineto{\pgfqpoint{4.436994in}{2.506055in}}%
\pgfpathlineto{\pgfqpoint{4.423364in}{2.508931in}}%
\pgfpathlineto{\pgfqpoint{4.415655in}{2.498949in}}%
\pgfpathlineto{\pgfqpoint{4.407941in}{2.489076in}}%
\pgfpathlineto{\pgfqpoint{4.400223in}{2.479308in}}%
\pgfpathlineto{\pgfqpoint{4.392501in}{2.469638in}}%
\pgfpathclose%
\pgfusepath{fill}%
\end{pgfscope}%
\begin{pgfscope}%
\pgfpathrectangle{\pgfqpoint{1.150000in}{0.150000in}}{\pgfqpoint{5.700000in}{5.700000in}}%
\pgfusepath{clip}%
\pgfsetbuttcap%
\pgfsetroundjoin%
\definecolor{currentfill}{rgb}{0.281446,0.084320,0.407414}%
\pgfsetfillcolor{currentfill}%
\pgfsetfillopacity{0.700000}%
\pgfsetlinewidth{0.000000pt}%
\definecolor{currentstroke}{rgb}{0.000000,0.000000,0.000000}%
\pgfsetstrokecolor{currentstroke}%
\pgfsetdash{}{0pt}%
\pgfpathmoveto{\pgfqpoint{3.630809in}{2.341964in}}%
\pgfpathlineto{\pgfqpoint{3.644281in}{2.337349in}}%
\pgfpathlineto{\pgfqpoint{3.657757in}{2.332823in}}%
\pgfpathlineto{\pgfqpoint{3.671238in}{2.328384in}}%
\pgfpathlineto{\pgfqpoint{3.684724in}{2.324032in}}%
\pgfpathlineto{\pgfqpoint{3.692690in}{2.333512in}}%
\pgfpathlineto{\pgfqpoint{3.700650in}{2.343036in}}%
\pgfpathlineto{\pgfqpoint{3.708604in}{2.352609in}}%
\pgfpathlineto{\pgfqpoint{3.716553in}{2.362232in}}%
\pgfpathlineto{\pgfqpoint{3.703077in}{2.366701in}}%
\pgfpathlineto{\pgfqpoint{3.689606in}{2.371258in}}%
\pgfpathlineto{\pgfqpoint{3.676140in}{2.375902in}}%
\pgfpathlineto{\pgfqpoint{3.662679in}{2.380635in}}%
\pgfpathlineto{\pgfqpoint{3.654720in}{2.370887in}}%
\pgfpathlineto{\pgfqpoint{3.646756in}{2.361194in}}%
\pgfpathlineto{\pgfqpoint{3.638785in}{2.351554in}}%
\pgfpathlineto{\pgfqpoint{3.630809in}{2.341964in}}%
\pgfpathclose%
\pgfusepath{fill}%
\end{pgfscope}%
\begin{pgfscope}%
\pgfpathrectangle{\pgfqpoint{1.150000in}{0.150000in}}{\pgfqpoint{5.700000in}{5.700000in}}%
\pgfusepath{clip}%
\pgfsetbuttcap%
\pgfsetroundjoin%
\definecolor{currentfill}{rgb}{0.281924,0.089666,0.412415}%
\pgfsetfillcolor{currentfill}%
\pgfsetfillopacity{0.700000}%
\pgfsetlinewidth{0.000000pt}%
\definecolor{currentstroke}{rgb}{0.000000,0.000000,0.000000}%
\pgfsetstrokecolor{currentstroke}%
\pgfsetdash{}{0pt}%
\pgfpathmoveto{\pgfqpoint{3.125668in}{2.353280in}}%
\pgfpathlineto{\pgfqpoint{3.139078in}{2.346253in}}%
\pgfpathlineto{\pgfqpoint{3.152490in}{2.339329in}}%
\pgfpathlineto{\pgfqpoint{3.165905in}{2.332508in}}%
\pgfpathlineto{\pgfqpoint{3.179323in}{2.325790in}}%
\pgfpathlineto{\pgfqpoint{3.187463in}{2.335009in}}%
\pgfpathlineto{\pgfqpoint{3.195595in}{2.344282in}}%
\pgfpathlineto{\pgfqpoint{3.203722in}{2.353609in}}%
\pgfpathlineto{\pgfqpoint{3.211841in}{2.362992in}}%
\pgfpathlineto{\pgfqpoint{3.198435in}{2.369747in}}%
\pgfpathlineto{\pgfqpoint{3.185032in}{2.376604in}}%
\pgfpathlineto{\pgfqpoint{3.171631in}{2.383564in}}%
\pgfpathlineto{\pgfqpoint{3.158233in}{2.390628in}}%
\pgfpathlineto{\pgfqpoint{3.150102in}{2.381201in}}%
\pgfpathlineto{\pgfqpoint{3.141964in}{2.371835in}}%
\pgfpathlineto{\pgfqpoint{3.133819in}{2.362529in}}%
\pgfpathlineto{\pgfqpoint{3.125668in}{2.353280in}}%
\pgfpathclose%
\pgfusepath{fill}%
\end{pgfscope}%
\begin{pgfscope}%
\pgfpathrectangle{\pgfqpoint{1.150000in}{0.150000in}}{\pgfqpoint{5.700000in}{5.700000in}}%
\pgfusepath{clip}%
\pgfsetbuttcap%
\pgfsetroundjoin%
\definecolor{currentfill}{rgb}{0.276194,0.190074,0.493001}%
\pgfsetfillcolor{currentfill}%
\pgfsetfillopacity{0.700000}%
\pgfsetlinewidth{0.000000pt}%
\definecolor{currentstroke}{rgb}{0.000000,0.000000,0.000000}%
\pgfsetstrokecolor{currentstroke}%
\pgfsetdash{}{0pt}%
\pgfpathmoveto{\pgfqpoint{4.703451in}{2.546929in}}%
\pgfpathlineto{\pgfqpoint{4.717174in}{2.544651in}}%
\pgfpathlineto{\pgfqpoint{4.730905in}{2.542445in}}%
\pgfpathlineto{\pgfqpoint{4.744644in}{2.540311in}}%
\pgfpathlineto{\pgfqpoint{4.758392in}{2.538248in}}%
\pgfpathlineto{\pgfqpoint{4.766001in}{2.547880in}}%
\pgfpathlineto{\pgfqpoint{4.773607in}{2.557649in}}%
\pgfpathlineto{\pgfqpoint{4.781210in}{2.567560in}}%
\pgfpathlineto{\pgfqpoint{4.788810in}{2.577621in}}%
\pgfpathlineto{\pgfqpoint{4.775077in}{2.580004in}}%
\pgfpathlineto{\pgfqpoint{4.761353in}{2.582459in}}%
\pgfpathlineto{\pgfqpoint{4.747636in}{2.584985in}}%
\pgfpathlineto{\pgfqpoint{4.733926in}{2.587583in}}%
\pgfpathlineto{\pgfqpoint{4.726312in}{2.577195in}}%
\pgfpathlineto{\pgfqpoint{4.718694in}{2.566961in}}%
\pgfpathlineto{\pgfqpoint{4.711074in}{2.556874in}}%
\pgfpathlineto{\pgfqpoint{4.703451in}{2.546929in}}%
\pgfpathclose%
\pgfusepath{fill}%
\end{pgfscope}%
\begin{pgfscope}%
\pgfpathrectangle{\pgfqpoint{1.150000in}{0.150000in}}{\pgfqpoint{5.700000in}{5.700000in}}%
\pgfusepath{clip}%
\pgfsetbuttcap%
\pgfsetroundjoin%
\definecolor{currentfill}{rgb}{0.260571,0.246922,0.522828}%
\pgfsetfillcolor{currentfill}%
\pgfsetfillopacity{0.700000}%
\pgfsetlinewidth{0.000000pt}%
\definecolor{currentstroke}{rgb}{0.000000,0.000000,0.000000}%
\pgfsetstrokecolor{currentstroke}%
\pgfsetdash{}{0pt}%
\pgfpathmoveto{\pgfqpoint{5.099924in}{2.666372in}}%
\pgfpathlineto{\pgfqpoint{5.113747in}{2.664098in}}%
\pgfpathlineto{\pgfqpoint{5.127578in}{2.661893in}}%
\pgfpathlineto{\pgfqpoint{5.141418in}{2.659756in}}%
\pgfpathlineto{\pgfqpoint{5.155266in}{2.657687in}}%
\pgfpathlineto{\pgfqpoint{5.162765in}{2.668292in}}%
\pgfpathlineto{\pgfqpoint{5.170263in}{2.679113in}}%
\pgfpathlineto{\pgfqpoint{5.177761in}{2.690156in}}%
\pgfpathlineto{\pgfqpoint{5.185260in}{2.701429in}}%
\pgfpathlineto{\pgfqpoint{5.171429in}{2.703899in}}%
\pgfpathlineto{\pgfqpoint{5.157607in}{2.706436in}}%
\pgfpathlineto{\pgfqpoint{5.143793in}{2.709042in}}%
\pgfpathlineto{\pgfqpoint{5.129987in}{2.711717in}}%
\pgfpathlineto{\pgfqpoint{5.122471in}{2.700036in}}%
\pgfpathlineto{\pgfqpoint{5.114955in}{2.688589in}}%
\pgfpathlineto{\pgfqpoint{5.107440in}{2.677371in}}%
\pgfpathlineto{\pgfqpoint{5.099924in}{2.666372in}}%
\pgfpathclose%
\pgfusepath{fill}%
\end{pgfscope}%
\begin{pgfscope}%
\pgfpathrectangle{\pgfqpoint{1.150000in}{0.150000in}}{\pgfqpoint{5.700000in}{5.700000in}}%
\pgfusepath{clip}%
\pgfsetbuttcap%
\pgfsetroundjoin%
\definecolor{currentfill}{rgb}{0.187231,0.414746,0.556547}%
\pgfsetfillcolor{currentfill}%
\pgfsetfillopacity{0.700000}%
\pgfsetlinewidth{0.000000pt}%
\definecolor{currentstroke}{rgb}{0.000000,0.000000,0.000000}%
\pgfsetstrokecolor{currentstroke}%
\pgfsetdash{}{0pt}%
\pgfpathmoveto{\pgfqpoint{5.869434in}{3.062893in}}%
\pgfpathlineto{\pgfqpoint{5.883378in}{3.058684in}}%
\pgfpathlineto{\pgfqpoint{5.897331in}{3.054540in}}%
\pgfpathlineto{\pgfqpoint{5.911292in}{3.050460in}}%
\pgfpathlineto{\pgfqpoint{5.925261in}{3.046444in}}%
\pgfpathlineto{\pgfqpoint{5.932791in}{3.064863in}}%
\pgfpathlineto{\pgfqpoint{5.940337in}{3.083772in}}%
\pgfpathlineto{\pgfqpoint{5.947898in}{3.103183in}}%
\pgfpathlineto{\pgfqpoint{5.933946in}{3.107633in}}%
\pgfpathlineto{\pgfqpoint{5.920002in}{3.112147in}}%
\pgfpathlineto{\pgfqpoint{5.906066in}{3.116726in}}%
\pgfpathlineto{\pgfqpoint{5.892138in}{3.121369in}}%
\pgfpathlineto{\pgfqpoint{5.884555in}{3.101374in}}%
\pgfpathlineto{\pgfqpoint{5.876987in}{3.081885in}}%
\pgfpathlineto{\pgfqpoint{5.869434in}{3.062893in}}%
\pgfpathclose%
\pgfusepath{fill}%
\end{pgfscope}%
\begin{pgfscope}%
\pgfpathrectangle{\pgfqpoint{1.150000in}{0.150000in}}{\pgfqpoint{5.700000in}{5.700000in}}%
\pgfusepath{clip}%
\pgfsetbuttcap%
\pgfsetroundjoin%
\definecolor{currentfill}{rgb}{0.281446,0.084320,0.407414}%
\pgfsetfillcolor{currentfill}%
\pgfsetfillopacity{0.700000}%
\pgfsetlinewidth{0.000000pt}%
\definecolor{currentstroke}{rgb}{0.000000,0.000000,0.000000}%
\pgfsetstrokecolor{currentstroke}%
\pgfsetdash{}{0pt}%
\pgfpathmoveto{\pgfqpoint{3.265495in}{2.336986in}}%
\pgfpathlineto{\pgfqpoint{3.278916in}{2.330734in}}%
\pgfpathlineto{\pgfqpoint{3.292341in}{2.324581in}}%
\pgfpathlineto{\pgfqpoint{3.305769in}{2.318526in}}%
\pgfpathlineto{\pgfqpoint{3.319200in}{2.312568in}}%
\pgfpathlineto{\pgfqpoint{3.327291in}{2.321909in}}%
\pgfpathlineto{\pgfqpoint{3.335375in}{2.331298in}}%
\pgfpathlineto{\pgfqpoint{3.343453in}{2.340737in}}%
\pgfpathlineto{\pgfqpoint{3.351525in}{2.350227in}}%
\pgfpathlineto{\pgfqpoint{3.338105in}{2.356242in}}%
\pgfpathlineto{\pgfqpoint{3.324688in}{2.362354in}}%
\pgfpathlineto{\pgfqpoint{3.311274in}{2.368564in}}%
\pgfpathlineto{\pgfqpoint{3.297864in}{2.374872in}}%
\pgfpathlineto{\pgfqpoint{3.289782in}{2.365318in}}%
\pgfpathlineto{\pgfqpoint{3.281692in}{2.355820in}}%
\pgfpathlineto{\pgfqpoint{3.273597in}{2.346376in}}%
\pgfpathlineto{\pgfqpoint{3.265495in}{2.336986in}}%
\pgfpathclose%
\pgfusepath{fill}%
\end{pgfscope}%
\begin{pgfscope}%
\pgfpathrectangle{\pgfqpoint{1.150000in}{0.150000in}}{\pgfqpoint{5.700000in}{5.700000in}}%
\pgfusepath{clip}%
\pgfsetbuttcap%
\pgfsetroundjoin%
\definecolor{currentfill}{rgb}{0.283229,0.120777,0.440584}%
\pgfsetfillcolor{currentfill}%
\pgfsetfillopacity{0.700000}%
\pgfsetlinewidth{0.000000pt}%
\definecolor{currentstroke}{rgb}{0.000000,0.000000,0.000000}%
\pgfsetstrokecolor{currentstroke}%
\pgfsetdash{}{0pt}%
\pgfpathmoveto{\pgfqpoint{4.081580in}{2.401504in}}%
\pgfpathlineto{\pgfqpoint{4.095146in}{2.398306in}}%
\pgfpathlineto{\pgfqpoint{4.108720in}{2.395188in}}%
\pgfpathlineto{\pgfqpoint{4.122299in}{2.392149in}}%
\pgfpathlineto{\pgfqpoint{4.135886in}{2.389190in}}%
\pgfpathlineto{\pgfqpoint{4.143701in}{2.398583in}}%
\pgfpathlineto{\pgfqpoint{4.151511in}{2.408041in}}%
\pgfpathlineto{\pgfqpoint{4.159316in}{2.417566in}}%
\pgfpathlineto{\pgfqpoint{4.167116in}{2.427163in}}%
\pgfpathlineto{\pgfqpoint{4.153540in}{2.430322in}}%
\pgfpathlineto{\pgfqpoint{4.139972in}{2.433560in}}%
\pgfpathlineto{\pgfqpoint{4.126409in}{2.436877in}}%
\pgfpathlineto{\pgfqpoint{4.112853in}{2.440273in}}%
\pgfpathlineto{\pgfqpoint{4.105043in}{2.430470in}}%
\pgfpathlineto{\pgfqpoint{4.097227in}{2.420743in}}%
\pgfpathlineto{\pgfqpoint{4.089406in}{2.411089in}}%
\pgfpathlineto{\pgfqpoint{4.081580in}{2.401504in}}%
\pgfpathclose%
\pgfusepath{fill}%
\end{pgfscope}%
\begin{pgfscope}%
\pgfpathrectangle{\pgfqpoint{1.150000in}{0.150000in}}{\pgfqpoint{5.700000in}{5.700000in}}%
\pgfusepath{clip}%
\pgfsetbuttcap%
\pgfsetroundjoin%
\definecolor{currentfill}{rgb}{0.282656,0.100196,0.422160}%
\pgfsetfillcolor{currentfill}%
\pgfsetfillopacity{0.700000}%
\pgfsetlinewidth{0.000000pt}%
\definecolor{currentstroke}{rgb}{0.000000,0.000000,0.000000}%
\pgfsetstrokecolor{currentstroke}%
\pgfsetdash{}{0pt}%
\pgfpathmoveto{\pgfqpoint{2.985694in}{2.376775in}}%
\pgfpathlineto{\pgfqpoint{2.999100in}{2.368905in}}%
\pgfpathlineto{\pgfqpoint{3.012508in}{2.361146in}}%
\pgfpathlineto{\pgfqpoint{3.025917in}{2.353496in}}%
\pgfpathlineto{\pgfqpoint{3.039329in}{2.345954in}}%
\pgfpathlineto{\pgfqpoint{3.047520in}{2.354985in}}%
\pgfpathlineto{\pgfqpoint{3.055704in}{2.364077in}}%
\pgfpathlineto{\pgfqpoint{3.063881in}{2.373230in}}%
\pgfpathlineto{\pgfqpoint{3.072051in}{2.382445in}}%
\pgfpathlineto{\pgfqpoint{3.058653in}{2.390003in}}%
\pgfpathlineto{\pgfqpoint{3.045256in}{2.397669in}}%
\pgfpathlineto{\pgfqpoint{3.031861in}{2.405445in}}%
\pgfpathlineto{\pgfqpoint{3.018468in}{2.413330in}}%
\pgfpathlineto{\pgfqpoint{3.010285in}{2.404092in}}%
\pgfpathlineto{\pgfqpoint{3.002095in}{2.394921in}}%
\pgfpathlineto{\pgfqpoint{2.993898in}{2.385815in}}%
\pgfpathlineto{\pgfqpoint{2.985694in}{2.376775in}}%
\pgfpathclose%
\pgfusepath{fill}%
\end{pgfscope}%
\begin{pgfscope}%
\pgfpathrectangle{\pgfqpoint{1.150000in}{0.150000in}}{\pgfqpoint{5.700000in}{5.700000in}}%
\pgfusepath{clip}%
\pgfsetbuttcap%
\pgfsetroundjoin%
\definecolor{currentfill}{rgb}{0.280894,0.078907,0.402329}%
\pgfsetfillcolor{currentfill}%
\pgfsetfillopacity{0.700000}%
\pgfsetlinewidth{0.000000pt}%
\definecolor{currentstroke}{rgb}{0.000000,0.000000,0.000000}%
\pgfsetstrokecolor{currentstroke}%
\pgfsetdash{}{0pt}%
\pgfpathmoveto{\pgfqpoint{3.405243in}{2.327132in}}%
\pgfpathlineto{\pgfqpoint{3.418681in}{2.321596in}}%
\pgfpathlineto{\pgfqpoint{3.432124in}{2.316154in}}%
\pgfpathlineto{\pgfqpoint{3.445572in}{2.310806in}}%
\pgfpathlineto{\pgfqpoint{3.459023in}{2.305551in}}%
\pgfpathlineto{\pgfqpoint{3.467067in}{2.314954in}}%
\pgfpathlineto{\pgfqpoint{3.475105in}{2.324401in}}%
\pgfpathlineto{\pgfqpoint{3.483137in}{2.333894in}}%
\pgfpathlineto{\pgfqpoint{3.491163in}{2.343435in}}%
\pgfpathlineto{\pgfqpoint{3.477723in}{2.348768in}}%
\pgfpathlineto{\pgfqpoint{3.464286in}{2.354193in}}%
\pgfpathlineto{\pgfqpoint{3.450854in}{2.359712in}}%
\pgfpathlineto{\pgfqpoint{3.437426in}{2.365325in}}%
\pgfpathlineto{\pgfqpoint{3.429389in}{2.355700in}}%
\pgfpathlineto{\pgfqpoint{3.421346in}{2.346127in}}%
\pgfpathlineto{\pgfqpoint{3.413297in}{2.336605in}}%
\pgfpathlineto{\pgfqpoint{3.405243in}{2.327132in}}%
\pgfpathclose%
\pgfusepath{fill}%
\end{pgfscope}%
\begin{pgfscope}%
\pgfpathrectangle{\pgfqpoint{1.150000in}{0.150000in}}{\pgfqpoint{5.700000in}{5.700000in}}%
\pgfusepath{clip}%
\pgfsetbuttcap%
\pgfsetroundjoin%
\definecolor{currentfill}{rgb}{0.265145,0.232956,0.516599}%
\pgfsetfillcolor{currentfill}%
\pgfsetfillopacity{0.700000}%
\pgfsetlinewidth{0.000000pt}%
\definecolor{currentstroke}{rgb}{0.000000,0.000000,0.000000}%
\pgfsetstrokecolor{currentstroke}%
\pgfsetdash{}{0pt}%
\pgfpathmoveto{\pgfqpoint{5.014579in}{2.632706in}}%
\pgfpathlineto{\pgfqpoint{5.028385in}{2.630536in}}%
\pgfpathlineto{\pgfqpoint{5.042200in}{2.628436in}}%
\pgfpathlineto{\pgfqpoint{5.056023in}{2.626405in}}%
\pgfpathlineto{\pgfqpoint{5.069855in}{2.624443in}}%
\pgfpathlineto{\pgfqpoint{5.077374in}{2.634630in}}%
\pgfpathlineto{\pgfqpoint{5.084891in}{2.645009in}}%
\pgfpathlineto{\pgfqpoint{5.092408in}{2.655587in}}%
\pgfpathlineto{\pgfqpoint{5.099924in}{2.666372in}}%
\pgfpathlineto{\pgfqpoint{5.086109in}{2.668715in}}%
\pgfpathlineto{\pgfqpoint{5.072303in}{2.671127in}}%
\pgfpathlineto{\pgfqpoint{5.058505in}{2.673608in}}%
\pgfpathlineto{\pgfqpoint{5.044715in}{2.676158in}}%
\pgfpathlineto{\pgfqpoint{5.037182in}{2.664985in}}%
\pgfpathlineto{\pgfqpoint{5.029648in}{2.654023in}}%
\pgfpathlineto{\pgfqpoint{5.022114in}{2.643266in}}%
\pgfpathlineto{\pgfqpoint{5.014579in}{2.632706in}}%
\pgfpathclose%
\pgfusepath{fill}%
\end{pgfscope}%
\begin{pgfscope}%
\pgfpathrectangle{\pgfqpoint{1.150000in}{0.150000in}}{\pgfqpoint{5.700000in}{5.700000in}}%
\pgfusepath{clip}%
\pgfsetbuttcap%
\pgfsetroundjoin%
\definecolor{currentfill}{rgb}{0.281412,0.155834,0.469201}%
\pgfsetfillcolor{currentfill}%
\pgfsetfillopacity{0.700000}%
\pgfsetlinewidth{0.000000pt}%
\definecolor{currentstroke}{rgb}{0.000000,0.000000,0.000000}%
\pgfsetstrokecolor{currentstroke}%
\pgfsetdash{}{0pt}%
\pgfpathmoveto{\pgfqpoint{2.651291in}{2.489145in}}%
\pgfpathlineto{\pgfqpoint{2.664716in}{2.478858in}}%
\pgfpathlineto{\pgfqpoint{2.678141in}{2.468701in}}%
\pgfpathlineto{\pgfqpoint{2.691566in}{2.458671in}}%
\pgfpathlineto{\pgfqpoint{2.704989in}{2.448768in}}%
\pgfpathlineto{\pgfqpoint{2.713312in}{2.457173in}}%
\pgfpathlineto{\pgfqpoint{2.721626in}{2.465661in}}%
\pgfpathlineto{\pgfqpoint{2.729932in}{2.474231in}}%
\pgfpathlineto{\pgfqpoint{2.738230in}{2.482884in}}%
\pgfpathlineto{\pgfqpoint{2.724822in}{2.492762in}}%
\pgfpathlineto{\pgfqpoint{2.711413in}{2.502766in}}%
\pgfpathlineto{\pgfqpoint{2.698004in}{2.512897in}}%
\pgfpathlineto{\pgfqpoint{2.684594in}{2.523158in}}%
\pgfpathlineto{\pgfqpoint{2.676281in}{2.514523in}}%
\pgfpathlineto{\pgfqpoint{2.667959in}{2.505976in}}%
\pgfpathlineto{\pgfqpoint{2.659629in}{2.497517in}}%
\pgfpathlineto{\pgfqpoint{2.651291in}{2.489145in}}%
\pgfpathclose%
\pgfusepath{fill}%
\end{pgfscope}%
\begin{pgfscope}%
\pgfpathrectangle{\pgfqpoint{1.150000in}{0.150000in}}{\pgfqpoint{5.700000in}{5.700000in}}%
\pgfusepath{clip}%
\pgfsetbuttcap%
\pgfsetroundjoin%
\definecolor{currentfill}{rgb}{0.282623,0.140926,0.457517}%
\pgfsetfillcolor{currentfill}%
\pgfsetfillopacity{0.700000}%
\pgfsetlinewidth{0.000000pt}%
\definecolor{currentstroke}{rgb}{0.000000,0.000000,0.000000}%
\pgfsetstrokecolor{currentstroke}%
\pgfsetdash{}{0pt}%
\pgfpathmoveto{\pgfqpoint{4.307021in}{2.442125in}}%
\pgfpathlineto{\pgfqpoint{4.320646in}{2.439445in}}%
\pgfpathlineto{\pgfqpoint{4.334279in}{2.436841in}}%
\pgfpathlineto{\pgfqpoint{4.347919in}{2.434313in}}%
\pgfpathlineto{\pgfqpoint{4.361565in}{2.431861in}}%
\pgfpathlineto{\pgfqpoint{4.369306in}{2.441179in}}%
\pgfpathlineto{\pgfqpoint{4.377042in}{2.450578in}}%
\pgfpathlineto{\pgfqpoint{4.384774in}{2.460063in}}%
\pgfpathlineto{\pgfqpoint{4.392501in}{2.469638in}}%
\pgfpathlineto{\pgfqpoint{4.378866in}{2.472330in}}%
\pgfpathlineto{\pgfqpoint{4.365238in}{2.475098in}}%
\pgfpathlineto{\pgfqpoint{4.351618in}{2.477941in}}%
\pgfpathlineto{\pgfqpoint{4.338004in}{2.480861in}}%
\pgfpathlineto{\pgfqpoint{4.330265in}{2.471039in}}%
\pgfpathlineto{\pgfqpoint{4.322522in}{2.461312in}}%
\pgfpathlineto{\pgfqpoint{4.314774in}{2.451676in}}%
\pgfpathlineto{\pgfqpoint{4.307021in}{2.442125in}}%
\pgfpathclose%
\pgfusepath{fill}%
\end{pgfscope}%
\begin{pgfscope}%
\pgfpathrectangle{\pgfqpoint{1.150000in}{0.150000in}}{\pgfqpoint{5.700000in}{5.700000in}}%
\pgfusepath{clip}%
\pgfsetbuttcap%
\pgfsetroundjoin%
\definecolor{currentfill}{rgb}{0.278012,0.180367,0.486697}%
\pgfsetfillcolor{currentfill}%
\pgfsetfillopacity{0.700000}%
\pgfsetlinewidth{0.000000pt}%
\definecolor{currentstroke}{rgb}{0.000000,0.000000,0.000000}%
\pgfsetstrokecolor{currentstroke}%
\pgfsetdash{}{0pt}%
\pgfpathmoveto{\pgfqpoint{4.618051in}{2.517094in}}%
\pgfpathlineto{\pgfqpoint{4.631757in}{2.514827in}}%
\pgfpathlineto{\pgfqpoint{4.645471in}{2.512632in}}%
\pgfpathlineto{\pgfqpoint{4.659193in}{2.510510in}}%
\pgfpathlineto{\pgfqpoint{4.672922in}{2.508460in}}%
\pgfpathlineto{\pgfqpoint{4.680560in}{2.517892in}}%
\pgfpathlineto{\pgfqpoint{4.688194in}{2.527444in}}%
\pgfpathlineto{\pgfqpoint{4.695824in}{2.537121in}}%
\pgfpathlineto{\pgfqpoint{4.703451in}{2.546929in}}%
\pgfpathlineto{\pgfqpoint{4.689735in}{2.549280in}}%
\pgfpathlineto{\pgfqpoint{4.676027in}{2.551702in}}%
\pgfpathlineto{\pgfqpoint{4.662327in}{2.554197in}}%
\pgfpathlineto{\pgfqpoint{4.648634in}{2.556765in}}%
\pgfpathlineto{\pgfqpoint{4.640994in}{2.546649in}}%
\pgfpathlineto{\pgfqpoint{4.633350in}{2.536669in}}%
\pgfpathlineto{\pgfqpoint{4.625702in}{2.526819in}}%
\pgfpathlineto{\pgfqpoint{4.618051in}{2.517094in}}%
\pgfpathclose%
\pgfusepath{fill}%
\end{pgfscope}%
\begin{pgfscope}%
\pgfpathrectangle{\pgfqpoint{1.150000in}{0.150000in}}{\pgfqpoint{5.700000in}{5.700000in}}%
\pgfusepath{clip}%
\pgfsetbuttcap%
\pgfsetroundjoin%
\definecolor{currentfill}{rgb}{0.283229,0.120777,0.440584}%
\pgfsetfillcolor{currentfill}%
\pgfsetfillopacity{0.700000}%
\pgfsetlinewidth{0.000000pt}%
\definecolor{currentstroke}{rgb}{0.000000,0.000000,0.000000}%
\pgfsetstrokecolor{currentstroke}%
\pgfsetdash{}{0pt}%
\pgfpathmoveto{\pgfqpoint{2.845497in}{2.408300in}}%
\pgfpathlineto{\pgfqpoint{2.858907in}{2.399517in}}%
\pgfpathlineto{\pgfqpoint{2.872318in}{2.390850in}}%
\pgfpathlineto{\pgfqpoint{2.885730in}{2.382299in}}%
\pgfpathlineto{\pgfqpoint{2.899143in}{2.373863in}}%
\pgfpathlineto{\pgfqpoint{2.907389in}{2.382637in}}%
\pgfpathlineto{\pgfqpoint{2.915629in}{2.391478in}}%
\pgfpathlineto{\pgfqpoint{2.923861in}{2.400389in}}%
\pgfpathlineto{\pgfqpoint{2.932085in}{2.409368in}}%
\pgfpathlineto{\pgfqpoint{2.918686in}{2.417799in}}%
\pgfpathlineto{\pgfqpoint{2.905288in}{2.426345in}}%
\pgfpathlineto{\pgfqpoint{2.891891in}{2.435007in}}%
\pgfpathlineto{\pgfqpoint{2.878495in}{2.443786in}}%
\pgfpathlineto{\pgfqpoint{2.870257in}{2.434804in}}%
\pgfpathlineto{\pgfqpoint{2.862011in}{2.425896in}}%
\pgfpathlineto{\pgfqpoint{2.853758in}{2.417061in}}%
\pgfpathlineto{\pgfqpoint{2.845497in}{2.408300in}}%
\pgfpathclose%
\pgfusepath{fill}%
\end{pgfscope}%
\begin{pgfscope}%
\pgfpathrectangle{\pgfqpoint{1.150000in}{0.150000in}}{\pgfqpoint{5.700000in}{5.700000in}}%
\pgfusepath{clip}%
\pgfsetbuttcap%
\pgfsetroundjoin%
\definecolor{currentfill}{rgb}{0.281924,0.089666,0.412415}%
\pgfsetfillcolor{currentfill}%
\pgfsetfillopacity{0.700000}%
\pgfsetlinewidth{0.000000pt}%
\definecolor{currentstroke}{rgb}{0.000000,0.000000,0.000000}%
\pgfsetstrokecolor{currentstroke}%
\pgfsetdash{}{0pt}%
\pgfpathmoveto{\pgfqpoint{3.770509in}{2.345219in}}%
\pgfpathlineto{\pgfqpoint{3.784012in}{2.341181in}}%
\pgfpathlineto{\pgfqpoint{3.797520in}{2.337227in}}%
\pgfpathlineto{\pgfqpoint{3.811033in}{2.333358in}}%
\pgfpathlineto{\pgfqpoint{3.824553in}{2.329573in}}%
\pgfpathlineto{\pgfqpoint{3.832475in}{2.338987in}}%
\pgfpathlineto{\pgfqpoint{3.840392in}{2.348447in}}%
\pgfpathlineto{\pgfqpoint{3.848303in}{2.357955in}}%
\pgfpathlineto{\pgfqpoint{3.856209in}{2.367514in}}%
\pgfpathlineto{\pgfqpoint{3.842700in}{2.371437in}}%
\pgfpathlineto{\pgfqpoint{3.829197in}{2.375444in}}%
\pgfpathlineto{\pgfqpoint{3.815699in}{2.379536in}}%
\pgfpathlineto{\pgfqpoint{3.802207in}{2.383713in}}%
\pgfpathlineto{\pgfqpoint{3.794291in}{2.374009in}}%
\pgfpathlineto{\pgfqpoint{3.786369in}{2.364360in}}%
\pgfpathlineto{\pgfqpoint{3.778442in}{2.354764in}}%
\pgfpathlineto{\pgfqpoint{3.770509in}{2.345219in}}%
\pgfpathclose%
\pgfusepath{fill}%
\end{pgfscope}%
\begin{pgfscope}%
\pgfpathrectangle{\pgfqpoint{1.150000in}{0.150000in}}{\pgfqpoint{5.700000in}{5.700000in}}%
\pgfusepath{clip}%
\pgfsetbuttcap%
\pgfsetroundjoin%
\definecolor{currentfill}{rgb}{0.223925,0.334994,0.548053}%
\pgfsetfillcolor{currentfill}%
\pgfsetfillopacity{0.700000}%
\pgfsetlinewidth{0.000000pt}%
\definecolor{currentstroke}{rgb}{0.000000,0.000000,0.000000}%
\pgfsetstrokecolor{currentstroke}%
\pgfsetdash{}{0pt}%
\pgfpathmoveto{\pgfqpoint{5.582403in}{2.848103in}}%
\pgfpathlineto{\pgfqpoint{5.596333in}{2.845237in}}%
\pgfpathlineto{\pgfqpoint{5.610273in}{2.842437in}}%
\pgfpathlineto{\pgfqpoint{5.624221in}{2.839702in}}%
\pgfpathlineto{\pgfqpoint{5.638177in}{2.837033in}}%
\pgfpathlineto{\pgfqpoint{5.645612in}{2.850463in}}%
\pgfpathlineto{\pgfqpoint{5.653052in}{2.864241in}}%
\pgfpathlineto{\pgfqpoint{5.660500in}{2.878375in}}%
\pgfpathlineto{\pgfqpoint{5.667955in}{2.892875in}}%
\pgfpathlineto{\pgfqpoint{5.654021in}{2.896045in}}%
\pgfpathlineto{\pgfqpoint{5.640094in}{2.899281in}}%
\pgfpathlineto{\pgfqpoint{5.626176in}{2.902582in}}%
\pgfpathlineto{\pgfqpoint{5.612267in}{2.905949in}}%
\pgfpathlineto{\pgfqpoint{5.604790in}{2.890941in}}%
\pgfpathlineto{\pgfqpoint{5.597321in}{2.876303in}}%
\pgfpathlineto{\pgfqpoint{5.589858in}{2.862027in}}%
\pgfpathlineto{\pgfqpoint{5.582403in}{2.848103in}}%
\pgfpathclose%
\pgfusepath{fill}%
\end{pgfscope}%
\begin{pgfscope}%
\pgfpathrectangle{\pgfqpoint{1.150000in}{0.150000in}}{\pgfqpoint{5.700000in}{5.700000in}}%
\pgfusepath{clip}%
\pgfsetbuttcap%
\pgfsetroundjoin%
\definecolor{currentfill}{rgb}{0.216210,0.351535,0.550627}%
\pgfsetfillcolor{currentfill}%
\pgfsetfillopacity{0.700000}%
\pgfsetlinewidth{0.000000pt}%
\definecolor{currentstroke}{rgb}{0.000000,0.000000,0.000000}%
\pgfsetstrokecolor{currentstroke}%
\pgfsetdash{}{0pt}%
\pgfpathmoveto{\pgfqpoint{5.667955in}{2.892875in}}%
\pgfpathlineto{\pgfqpoint{5.681899in}{2.889770in}}%
\pgfpathlineto{\pgfqpoint{5.695851in}{2.886730in}}%
\pgfpathlineto{\pgfqpoint{5.709812in}{2.883756in}}%
\pgfpathlineto{\pgfqpoint{5.723781in}{2.880847in}}%
\pgfpathlineto{\pgfqpoint{5.731222in}{2.895208in}}%
\pgfpathlineto{\pgfqpoint{5.738672in}{2.909949in}}%
\pgfpathlineto{\pgfqpoint{5.746130in}{2.925079in}}%
\pgfpathlineto{\pgfqpoint{5.753598in}{2.940608in}}%
\pgfpathlineto{\pgfqpoint{5.739651in}{2.944038in}}%
\pgfpathlineto{\pgfqpoint{5.725713in}{2.947533in}}%
\pgfpathlineto{\pgfqpoint{5.711782in}{2.951094in}}%
\pgfpathlineto{\pgfqpoint{5.697861in}{2.954719in}}%
\pgfpathlineto{\pgfqpoint{5.690371in}{2.938662in}}%
\pgfpathlineto{\pgfqpoint{5.682890in}{2.923009in}}%
\pgfpathlineto{\pgfqpoint{5.675419in}{2.907750in}}%
\pgfpathlineto{\pgfqpoint{5.667955in}{2.892875in}}%
\pgfpathclose%
\pgfusepath{fill}%
\end{pgfscope}%
\begin{pgfscope}%
\pgfpathrectangle{\pgfqpoint{1.150000in}{0.150000in}}{\pgfqpoint{5.700000in}{5.700000in}}%
\pgfusepath{clip}%
\pgfsetbuttcap%
\pgfsetroundjoin%
\definecolor{currentfill}{rgb}{0.283091,0.110553,0.431554}%
\pgfsetfillcolor{currentfill}%
\pgfsetfillopacity{0.700000}%
\pgfsetlinewidth{0.000000pt}%
\definecolor{currentstroke}{rgb}{0.000000,0.000000,0.000000}%
\pgfsetstrokecolor{currentstroke}%
\pgfsetdash{}{0pt}%
\pgfpathmoveto{\pgfqpoint{3.995977in}{2.376654in}}%
\pgfpathlineto{\pgfqpoint{4.009529in}{2.373314in}}%
\pgfpathlineto{\pgfqpoint{4.023087in}{2.370055in}}%
\pgfpathlineto{\pgfqpoint{4.036652in}{2.366877in}}%
\pgfpathlineto{\pgfqpoint{4.050223in}{2.363778in}}%
\pgfpathlineto{\pgfqpoint{4.058070in}{2.373124in}}%
\pgfpathlineto{\pgfqpoint{4.065912in}{2.382525in}}%
\pgfpathlineto{\pgfqpoint{4.073748in}{2.391983in}}%
\pgfpathlineto{\pgfqpoint{4.081580in}{2.401504in}}%
\pgfpathlineto{\pgfqpoint{4.068019in}{2.404781in}}%
\pgfpathlineto{\pgfqpoint{4.054465in}{2.408138in}}%
\pgfpathlineto{\pgfqpoint{4.040918in}{2.411576in}}%
\pgfpathlineto{\pgfqpoint{4.027376in}{2.415095in}}%
\pgfpathlineto{\pgfqpoint{4.019534in}{2.405388in}}%
\pgfpathlineto{\pgfqpoint{4.011687in}{2.395748in}}%
\pgfpathlineto{\pgfqpoint{4.003835in}{2.386171in}}%
\pgfpathlineto{\pgfqpoint{3.995977in}{2.376654in}}%
\pgfpathclose%
\pgfusepath{fill}%
\end{pgfscope}%
\begin{pgfscope}%
\pgfpathrectangle{\pgfqpoint{1.150000in}{0.150000in}}{\pgfqpoint{5.700000in}{5.700000in}}%
\pgfusepath{clip}%
\pgfsetbuttcap%
\pgfsetroundjoin%
\definecolor{currentfill}{rgb}{0.233603,0.313828,0.543914}%
\pgfsetfillcolor{currentfill}%
\pgfsetfillopacity{0.700000}%
\pgfsetlinewidth{0.000000pt}%
\definecolor{currentstroke}{rgb}{0.000000,0.000000,0.000000}%
\pgfsetstrokecolor{currentstroke}%
\pgfsetdash{}{0pt}%
\pgfpathmoveto{\pgfqpoint{5.496916in}{2.805955in}}%
\pgfpathlineto{\pgfqpoint{5.510833in}{2.803305in}}%
\pgfpathlineto{\pgfqpoint{5.524759in}{2.800723in}}%
\pgfpathlineto{\pgfqpoint{5.538694in}{2.798206in}}%
\pgfpathlineto{\pgfqpoint{5.552637in}{2.795755in}}%
\pgfpathlineto{\pgfqpoint{5.560071in}{2.808358in}}%
\pgfpathlineto{\pgfqpoint{5.567509in}{2.821278in}}%
\pgfpathlineto{\pgfqpoint{5.574953in}{2.834523in}}%
\pgfpathlineto{\pgfqpoint{5.582403in}{2.848103in}}%
\pgfpathlineto{\pgfqpoint{5.568480in}{2.851035in}}%
\pgfpathlineto{\pgfqpoint{5.554567in}{2.854033in}}%
\pgfpathlineto{\pgfqpoint{5.540662in}{2.857097in}}%
\pgfpathlineto{\pgfqpoint{5.526765in}{2.860227in}}%
\pgfpathlineto{\pgfqpoint{5.519295in}{2.846158in}}%
\pgfpathlineto{\pgfqpoint{5.511830in}{2.832429in}}%
\pgfpathlineto{\pgfqpoint{5.504371in}{2.819031in}}%
\pgfpathlineto{\pgfqpoint{5.496916in}{2.805955in}}%
\pgfpathclose%
\pgfusepath{fill}%
\end{pgfscope}%
\begin{pgfscope}%
\pgfpathrectangle{\pgfqpoint{1.150000in}{0.150000in}}{\pgfqpoint{5.700000in}{5.700000in}}%
\pgfusepath{clip}%
\pgfsetbuttcap%
\pgfsetroundjoin%
\definecolor{currentfill}{rgb}{0.280894,0.078907,0.402329}%
\pgfsetfillcolor{currentfill}%
\pgfsetfillopacity{0.700000}%
\pgfsetlinewidth{0.000000pt}%
\definecolor{currentstroke}{rgb}{0.000000,0.000000,0.000000}%
\pgfsetstrokecolor{currentstroke}%
\pgfsetdash{}{0pt}%
\pgfpathmoveto{\pgfqpoint{3.544969in}{2.323025in}}%
\pgfpathlineto{\pgfqpoint{3.558431in}{2.318150in}}%
\pgfpathlineto{\pgfqpoint{3.571899in}{2.313366in}}%
\pgfpathlineto{\pgfqpoint{3.585371in}{2.308671in}}%
\pgfpathlineto{\pgfqpoint{3.598847in}{2.304065in}}%
\pgfpathlineto{\pgfqpoint{3.606847in}{2.313475in}}%
\pgfpathlineto{\pgfqpoint{3.614840in}{2.322927in}}%
\pgfpathlineto{\pgfqpoint{3.622828in}{2.332423in}}%
\pgfpathlineto{\pgfqpoint{3.630809in}{2.341964in}}%
\pgfpathlineto{\pgfqpoint{3.617343in}{2.346668in}}%
\pgfpathlineto{\pgfqpoint{3.603881in}{2.351461in}}%
\pgfpathlineto{\pgfqpoint{3.590425in}{2.356343in}}%
\pgfpathlineto{\pgfqpoint{3.576972in}{2.361316in}}%
\pgfpathlineto{\pgfqpoint{3.568980in}{2.351669in}}%
\pgfpathlineto{\pgfqpoint{3.560982in}{2.342073in}}%
\pgfpathlineto{\pgfqpoint{3.552978in}{2.332526in}}%
\pgfpathlineto{\pgfqpoint{3.544969in}{2.323025in}}%
\pgfpathclose%
\pgfusepath{fill}%
\end{pgfscope}%
\begin{pgfscope}%
\pgfpathrectangle{\pgfqpoint{1.150000in}{0.150000in}}{\pgfqpoint{5.700000in}{5.700000in}}%
\pgfusepath{clip}%
\pgfsetbuttcap%
\pgfsetroundjoin%
\definecolor{currentfill}{rgb}{0.206756,0.371758,0.553117}%
\pgfsetfillcolor{currentfill}%
\pgfsetfillopacity{0.700000}%
\pgfsetlinewidth{0.000000pt}%
\definecolor{currentstroke}{rgb}{0.000000,0.000000,0.000000}%
\pgfsetstrokecolor{currentstroke}%
\pgfsetdash{}{0pt}%
\pgfpathmoveto{\pgfqpoint{5.753598in}{2.940608in}}%
\pgfpathlineto{\pgfqpoint{5.767554in}{2.937242in}}%
\pgfpathlineto{\pgfqpoint{5.781518in}{2.933942in}}%
\pgfpathlineto{\pgfqpoint{5.795491in}{2.930706in}}%
\pgfpathlineto{\pgfqpoint{5.809472in}{2.927536in}}%
\pgfpathlineto{\pgfqpoint{5.816927in}{2.942939in}}%
\pgfpathlineto{\pgfqpoint{5.824393in}{2.958755in}}%
\pgfpathlineto{\pgfqpoint{5.831869in}{2.974994in}}%
\pgfpathlineto{\pgfqpoint{5.839357in}{2.991667in}}%
\pgfpathlineto{\pgfqpoint{5.825398in}{2.995379in}}%
\pgfpathlineto{\pgfqpoint{5.811448in}{2.999156in}}%
\pgfpathlineto{\pgfqpoint{5.797507in}{3.002997in}}%
\pgfpathlineto{\pgfqpoint{5.783573in}{3.006903in}}%
\pgfpathlineto{\pgfqpoint{5.776063in}{2.989682in}}%
\pgfpathlineto{\pgfqpoint{5.768564in}{2.972899in}}%
\pgfpathlineto{\pgfqpoint{5.761076in}{2.956544in}}%
\pgfpathlineto{\pgfqpoint{5.753598in}{2.940608in}}%
\pgfpathclose%
\pgfusepath{fill}%
\end{pgfscope}%
\begin{pgfscope}%
\pgfpathrectangle{\pgfqpoint{1.150000in}{0.150000in}}{\pgfqpoint{5.700000in}{5.700000in}}%
\pgfusepath{clip}%
\pgfsetbuttcap%
\pgfsetroundjoin%
\definecolor{currentfill}{rgb}{0.269308,0.218818,0.509577}%
\pgfsetfillcolor{currentfill}%
\pgfsetfillopacity{0.700000}%
\pgfsetlinewidth{0.000000pt}%
\definecolor{currentstroke}{rgb}{0.000000,0.000000,0.000000}%
\pgfsetstrokecolor{currentstroke}%
\pgfsetdash{}{0pt}%
\pgfpathmoveto{\pgfqpoint{4.929213in}{2.600238in}}%
\pgfpathlineto{\pgfqpoint{4.943003in}{2.598150in}}%
\pgfpathlineto{\pgfqpoint{4.956801in}{2.596133in}}%
\pgfpathlineto{\pgfqpoint{4.970608in}{2.594185in}}%
\pgfpathlineto{\pgfqpoint{4.984422in}{2.592308in}}%
\pgfpathlineto{\pgfqpoint{4.991964in}{2.602144in}}%
\pgfpathlineto{\pgfqpoint{4.999504in}{2.612151in}}%
\pgfpathlineto{\pgfqpoint{5.007042in}{2.622336in}}%
\pgfpathlineto{\pgfqpoint{5.014579in}{2.632706in}}%
\pgfpathlineto{\pgfqpoint{5.000780in}{2.634945in}}%
\pgfpathlineto{\pgfqpoint{4.986990in}{2.637253in}}%
\pgfpathlineto{\pgfqpoint{4.973208in}{2.639631in}}%
\pgfpathlineto{\pgfqpoint{4.959434in}{2.642079in}}%
\pgfpathlineto{\pgfqpoint{4.951881in}{2.631342in}}%
\pgfpathlineto{\pgfqpoint{4.944327in}{2.620794in}}%
\pgfpathlineto{\pgfqpoint{4.936771in}{2.610428in}}%
\pgfpathlineto{\pgfqpoint{4.929213in}{2.600238in}}%
\pgfpathclose%
\pgfusepath{fill}%
\end{pgfscope}%
\begin{pgfscope}%
\pgfpathrectangle{\pgfqpoint{1.150000in}{0.150000in}}{\pgfqpoint{5.700000in}{5.700000in}}%
\pgfusepath{clip}%
\pgfsetbuttcap%
\pgfsetroundjoin%
\definecolor{currentfill}{rgb}{0.241237,0.296485,0.539709}%
\pgfsetfillcolor{currentfill}%
\pgfsetfillopacity{0.700000}%
\pgfsetlinewidth{0.000000pt}%
\definecolor{currentstroke}{rgb}{0.000000,0.000000,0.000000}%
\pgfsetstrokecolor{currentstroke}%
\pgfsetdash{}{0pt}%
\pgfpathmoveto{\pgfqpoint{5.411476in}{2.766115in}}%
\pgfpathlineto{\pgfqpoint{5.425379in}{2.763661in}}%
\pgfpathlineto{\pgfqpoint{5.439291in}{2.761274in}}%
\pgfpathlineto{\pgfqpoint{5.453211in}{2.758953in}}%
\pgfpathlineto{\pgfqpoint{5.467141in}{2.756699in}}%
\pgfpathlineto{\pgfqpoint{5.474579in}{2.768572in}}%
\pgfpathlineto{\pgfqpoint{5.482021in}{2.780734in}}%
\pgfpathlineto{\pgfqpoint{5.489466in}{2.793192in}}%
\pgfpathlineto{\pgfqpoint{5.496916in}{2.805955in}}%
\pgfpathlineto{\pgfqpoint{5.483008in}{2.808670in}}%
\pgfpathlineto{\pgfqpoint{5.469108in}{2.811452in}}%
\pgfpathlineto{\pgfqpoint{5.455216in}{2.814300in}}%
\pgfpathlineto{\pgfqpoint{5.441333in}{2.817215in}}%
\pgfpathlineto{\pgfqpoint{5.433863in}{2.803983in}}%
\pgfpathlineto{\pgfqpoint{5.426397in}{2.791062in}}%
\pgfpathlineto{\pgfqpoint{5.418935in}{2.778442in}}%
\pgfpathlineto{\pgfqpoint{5.411476in}{2.766115in}}%
\pgfpathclose%
\pgfusepath{fill}%
\end{pgfscope}%
\begin{pgfscope}%
\pgfpathrectangle{\pgfqpoint{1.150000in}{0.150000in}}{\pgfqpoint{5.700000in}{5.700000in}}%
\pgfusepath{clip}%
\pgfsetbuttcap%
\pgfsetroundjoin%
\definecolor{currentfill}{rgb}{0.195860,0.395433,0.555276}%
\pgfsetfillcolor{currentfill}%
\pgfsetfillopacity{0.700000}%
\pgfsetlinewidth{0.000000pt}%
\definecolor{currentstroke}{rgb}{0.000000,0.000000,0.000000}%
\pgfsetstrokecolor{currentstroke}%
\pgfsetdash{}{0pt}%
\pgfpathmoveto{\pgfqpoint{5.839357in}{2.991667in}}%
\pgfpathlineto{\pgfqpoint{5.853324in}{2.988019in}}%
\pgfpathlineto{\pgfqpoint{5.867299in}{2.984436in}}%
\pgfpathlineto{\pgfqpoint{5.881284in}{2.980918in}}%
\pgfpathlineto{\pgfqpoint{5.895276in}{2.977464in}}%
\pgfpathlineto{\pgfqpoint{5.902753in}{2.994026in}}%
\pgfpathlineto{\pgfqpoint{5.910242in}{3.011036in}}%
\pgfpathlineto{\pgfqpoint{5.917745in}{3.028505in}}%
\pgfpathlineto{\pgfqpoint{5.925261in}{3.046444in}}%
\pgfpathlineto{\pgfqpoint{5.911292in}{3.050460in}}%
\pgfpathlineto{\pgfqpoint{5.897331in}{3.054540in}}%
\pgfpathlineto{\pgfqpoint{5.883378in}{3.058684in}}%
\pgfpathlineto{\pgfqpoint{5.869434in}{3.062893in}}%
\pgfpathlineto{\pgfqpoint{5.861895in}{3.044385in}}%
\pgfpathlineto{\pgfqpoint{5.854369in}{3.026352in}}%
\pgfpathlineto{\pgfqpoint{5.846857in}{3.008782in}}%
\pgfpathlineto{\pgfqpoint{5.839357in}{2.991667in}}%
\pgfpathclose%
\pgfusepath{fill}%
\end{pgfscope}%
\begin{pgfscope}%
\pgfpathrectangle{\pgfqpoint{1.150000in}{0.150000in}}{\pgfqpoint{5.700000in}{5.700000in}}%
\pgfusepath{clip}%
\pgfsetbuttcap%
\pgfsetroundjoin%
\definecolor{currentfill}{rgb}{0.280255,0.165693,0.476498}%
\pgfsetfillcolor{currentfill}%
\pgfsetfillopacity{0.700000}%
\pgfsetlinewidth{0.000000pt}%
\definecolor{currentstroke}{rgb}{0.000000,0.000000,0.000000}%
\pgfsetstrokecolor{currentstroke}%
\pgfsetdash{}{0pt}%
\pgfpathmoveto{\pgfqpoint{4.532607in}{2.488017in}}%
\pgfpathlineto{\pgfqpoint{4.546295in}{2.485737in}}%
\pgfpathlineto{\pgfqpoint{4.559992in}{2.483530in}}%
\pgfpathlineto{\pgfqpoint{4.573696in}{2.481397in}}%
\pgfpathlineto{\pgfqpoint{4.587408in}{2.479338in}}%
\pgfpathlineto{\pgfqpoint{4.595075in}{2.488616in}}%
\pgfpathlineto{\pgfqpoint{4.602737in}{2.497998in}}%
\pgfpathlineto{\pgfqpoint{4.610396in}{2.507489in}}%
\pgfpathlineto{\pgfqpoint{4.618051in}{2.517094in}}%
\pgfpathlineto{\pgfqpoint{4.604353in}{2.519435in}}%
\pgfpathlineto{\pgfqpoint{4.590662in}{2.521848in}}%
\pgfpathlineto{\pgfqpoint{4.576979in}{2.524335in}}%
\pgfpathlineto{\pgfqpoint{4.563303in}{2.526895in}}%
\pgfpathlineto{\pgfqpoint{4.555635in}{2.517002in}}%
\pgfpathlineto{\pgfqpoint{4.547963in}{2.507228in}}%
\pgfpathlineto{\pgfqpoint{4.540287in}{2.497568in}}%
\pgfpathlineto{\pgfqpoint{4.532607in}{2.488017in}}%
\pgfpathclose%
\pgfusepath{fill}%
\end{pgfscope}%
\begin{pgfscope}%
\pgfpathrectangle{\pgfqpoint{1.150000in}{0.150000in}}{\pgfqpoint{5.700000in}{5.700000in}}%
\pgfusepath{clip}%
\pgfsetbuttcap%
\pgfsetroundjoin%
\definecolor{currentfill}{rgb}{0.283072,0.130895,0.449241}%
\pgfsetfillcolor{currentfill}%
\pgfsetfillopacity{0.700000}%
\pgfsetlinewidth{0.000000pt}%
\definecolor{currentstroke}{rgb}{0.000000,0.000000,0.000000}%
\pgfsetstrokecolor{currentstroke}%
\pgfsetdash{}{0pt}%
\pgfpathmoveto{\pgfqpoint{4.221484in}{2.415311in}}%
\pgfpathlineto{\pgfqpoint{4.235093in}{2.412542in}}%
\pgfpathlineto{\pgfqpoint{4.248709in}{2.409851in}}%
\pgfpathlineto{\pgfqpoint{4.262332in}{2.407238in}}%
\pgfpathlineto{\pgfqpoint{4.275962in}{2.404701in}}%
\pgfpathlineto{\pgfqpoint{4.283734in}{2.413948in}}%
\pgfpathlineto{\pgfqpoint{4.291501in}{2.423266in}}%
\pgfpathlineto{\pgfqpoint{4.299264in}{2.432657in}}%
\pgfpathlineto{\pgfqpoint{4.307021in}{2.442125in}}%
\pgfpathlineto{\pgfqpoint{4.293403in}{2.444882in}}%
\pgfpathlineto{\pgfqpoint{4.279791in}{2.447716in}}%
\pgfpathlineto{\pgfqpoint{4.266187in}{2.450626in}}%
\pgfpathlineto{\pgfqpoint{4.252589in}{2.453614in}}%
\pgfpathlineto{\pgfqpoint{4.244820in}{2.443919in}}%
\pgfpathlineto{\pgfqpoint{4.237046in}{2.434306in}}%
\pgfpathlineto{\pgfqpoint{4.229268in}{2.424771in}}%
\pgfpathlineto{\pgfqpoint{4.221484in}{2.415311in}}%
\pgfpathclose%
\pgfusepath{fill}%
\end{pgfscope}%
\begin{pgfscope}%
\pgfpathrectangle{\pgfqpoint{1.150000in}{0.150000in}}{\pgfqpoint{5.700000in}{5.700000in}}%
\pgfusepath{clip}%
\pgfsetbuttcap%
\pgfsetroundjoin%
\definecolor{currentfill}{rgb}{0.246811,0.283237,0.535941}%
\pgfsetfillcolor{currentfill}%
\pgfsetfillopacity{0.700000}%
\pgfsetlinewidth{0.000000pt}%
\definecolor{currentstroke}{rgb}{0.000000,0.000000,0.000000}%
\pgfsetstrokecolor{currentstroke}%
\pgfsetdash{}{0pt}%
\pgfpathmoveto{\pgfqpoint{5.326064in}{2.728296in}}%
\pgfpathlineto{\pgfqpoint{5.339953in}{2.726015in}}%
\pgfpathlineto{\pgfqpoint{5.353850in}{2.723802in}}%
\pgfpathlineto{\pgfqpoint{5.367755in}{2.721655in}}%
\pgfpathlineto{\pgfqpoint{5.381670in}{2.719576in}}%
\pgfpathlineto{\pgfqpoint{5.389118in}{2.730811in}}%
\pgfpathlineto{\pgfqpoint{5.396568in}{2.742308in}}%
\pgfpathlineto{\pgfqpoint{5.404021in}{2.754073in}}%
\pgfpathlineto{\pgfqpoint{5.411476in}{2.766115in}}%
\pgfpathlineto{\pgfqpoint{5.397582in}{2.768635in}}%
\pgfpathlineto{\pgfqpoint{5.383696in}{2.771223in}}%
\pgfpathlineto{\pgfqpoint{5.369818in}{2.773877in}}%
\pgfpathlineto{\pgfqpoint{5.355949in}{2.776599in}}%
\pgfpathlineto{\pgfqpoint{5.348474in}{2.764109in}}%
\pgfpathlineto{\pgfqpoint{5.341002in}{2.751900in}}%
\pgfpathlineto{\pgfqpoint{5.333532in}{2.739965in}}%
\pgfpathlineto{\pgfqpoint{5.326064in}{2.728296in}}%
\pgfpathclose%
\pgfusepath{fill}%
\end{pgfscope}%
\begin{pgfscope}%
\pgfpathrectangle{\pgfqpoint{1.150000in}{0.150000in}}{\pgfqpoint{5.700000in}{5.700000in}}%
\pgfusepath{clip}%
\pgfsetbuttcap%
\pgfsetroundjoin%
\definecolor{currentfill}{rgb}{0.280894,0.078907,0.402329}%
\pgfsetfillcolor{currentfill}%
\pgfsetfillopacity{0.700000}%
\pgfsetlinewidth{0.000000pt}%
\definecolor{currentstroke}{rgb}{0.000000,0.000000,0.000000}%
\pgfsetstrokecolor{currentstroke}%
\pgfsetdash{}{0pt}%
\pgfpathmoveto{\pgfqpoint{3.179323in}{2.325790in}}%
\pgfpathlineto{\pgfqpoint{3.192744in}{2.319174in}}%
\pgfpathlineto{\pgfqpoint{3.206167in}{2.312659in}}%
\pgfpathlineto{\pgfqpoint{3.219594in}{2.306245in}}%
\pgfpathlineto{\pgfqpoint{3.233024in}{2.299931in}}%
\pgfpathlineto{\pgfqpoint{3.241151in}{2.309121in}}%
\pgfpathlineto{\pgfqpoint{3.249272in}{2.318359in}}%
\pgfpathlineto{\pgfqpoint{3.257387in}{2.327647in}}%
\pgfpathlineto{\pgfqpoint{3.265495in}{2.336986in}}%
\pgfpathlineto{\pgfqpoint{3.252077in}{2.343337in}}%
\pgfpathlineto{\pgfqpoint{3.238662in}{2.349788in}}%
\pgfpathlineto{\pgfqpoint{3.225250in}{2.356339in}}%
\pgfpathlineto{\pgfqpoint{3.211841in}{2.362992in}}%
\pgfpathlineto{\pgfqpoint{3.203722in}{2.353609in}}%
\pgfpathlineto{\pgfqpoint{3.195595in}{2.344282in}}%
\pgfpathlineto{\pgfqpoint{3.187463in}{2.335009in}}%
\pgfpathlineto{\pgfqpoint{3.179323in}{2.325790in}}%
\pgfpathclose%
\pgfusepath{fill}%
\end{pgfscope}%
\begin{pgfscope}%
\pgfpathrectangle{\pgfqpoint{1.150000in}{0.150000in}}{\pgfqpoint{5.700000in}{5.700000in}}%
\pgfusepath{clip}%
\pgfsetbuttcap%
\pgfsetroundjoin%
\definecolor{currentfill}{rgb}{0.281924,0.089666,0.412415}%
\pgfsetfillcolor{currentfill}%
\pgfsetfillopacity{0.700000}%
\pgfsetlinewidth{0.000000pt}%
\definecolor{currentstroke}{rgb}{0.000000,0.000000,0.000000}%
\pgfsetstrokecolor{currentstroke}%
\pgfsetdash{}{0pt}%
\pgfpathmoveto{\pgfqpoint{3.039329in}{2.345954in}}%
\pgfpathlineto{\pgfqpoint{3.052742in}{2.338519in}}%
\pgfpathlineto{\pgfqpoint{3.066157in}{2.331192in}}%
\pgfpathlineto{\pgfqpoint{3.079575in}{2.323970in}}%
\pgfpathlineto{\pgfqpoint{3.092995in}{2.316854in}}%
\pgfpathlineto{\pgfqpoint{3.101173in}{2.325877in}}%
\pgfpathlineto{\pgfqpoint{3.109345in}{2.334956in}}%
\pgfpathlineto{\pgfqpoint{3.117510in}{2.344090in}}%
\pgfpathlineto{\pgfqpoint{3.125668in}{2.353280in}}%
\pgfpathlineto{\pgfqpoint{3.112260in}{2.360413in}}%
\pgfpathlineto{\pgfqpoint{3.098855in}{2.367650in}}%
\pgfpathlineto{\pgfqpoint{3.085452in}{2.374994in}}%
\pgfpathlineto{\pgfqpoint{3.072051in}{2.382445in}}%
\pgfpathlineto{\pgfqpoint{3.063881in}{2.373230in}}%
\pgfpathlineto{\pgfqpoint{3.055704in}{2.364077in}}%
\pgfpathlineto{\pgfqpoint{3.047520in}{2.354985in}}%
\pgfpathlineto{\pgfqpoint{3.039329in}{2.345954in}}%
\pgfpathclose%
\pgfusepath{fill}%
\end{pgfscope}%
\begin{pgfscope}%
\pgfpathrectangle{\pgfqpoint{1.150000in}{0.150000in}}{\pgfqpoint{5.700000in}{5.700000in}}%
\pgfusepath{clip}%
\pgfsetbuttcap%
\pgfsetroundjoin%
\definecolor{currentfill}{rgb}{0.282623,0.140926,0.457517}%
\pgfsetfillcolor{currentfill}%
\pgfsetfillopacity{0.700000}%
\pgfsetlinewidth{0.000000pt}%
\definecolor{currentstroke}{rgb}{0.000000,0.000000,0.000000}%
\pgfsetstrokecolor{currentstroke}%
\pgfsetdash{}{0pt}%
\pgfpathmoveto{\pgfqpoint{2.704989in}{2.448768in}}%
\pgfpathlineto{\pgfqpoint{2.718413in}{2.438990in}}%
\pgfpathlineto{\pgfqpoint{2.731836in}{2.429338in}}%
\pgfpathlineto{\pgfqpoint{2.745259in}{2.419808in}}%
\pgfpathlineto{\pgfqpoint{2.758682in}{2.410402in}}%
\pgfpathlineto{\pgfqpoint{2.766989in}{2.418840in}}%
\pgfpathlineto{\pgfqpoint{2.775288in}{2.427356in}}%
\pgfpathlineto{\pgfqpoint{2.783579in}{2.435949in}}%
\pgfpathlineto{\pgfqpoint{2.791862in}{2.444620in}}%
\pgfpathlineto{\pgfqpoint{2.778454in}{2.454001in}}%
\pgfpathlineto{\pgfqpoint{2.765046in}{2.463505in}}%
\pgfpathlineto{\pgfqpoint{2.751638in}{2.473132in}}%
\pgfpathlineto{\pgfqpoint{2.738230in}{2.482884in}}%
\pgfpathlineto{\pgfqpoint{2.729932in}{2.474231in}}%
\pgfpathlineto{\pgfqpoint{2.721626in}{2.465661in}}%
\pgfpathlineto{\pgfqpoint{2.713312in}{2.457173in}}%
\pgfpathlineto{\pgfqpoint{2.704989in}{2.448768in}}%
\pgfpathclose%
\pgfusepath{fill}%
\end{pgfscope}%
\begin{pgfscope}%
\pgfpathrectangle{\pgfqpoint{1.150000in}{0.150000in}}{\pgfqpoint{5.700000in}{5.700000in}}%
\pgfusepath{clip}%
\pgfsetbuttcap%
\pgfsetroundjoin%
\definecolor{currentfill}{rgb}{0.271828,0.209303,0.504434}%
\pgfsetfillcolor{currentfill}%
\pgfsetfillopacity{0.700000}%
\pgfsetlinewidth{0.000000pt}%
\definecolor{currentstroke}{rgb}{0.000000,0.000000,0.000000}%
\pgfsetstrokecolor{currentstroke}%
\pgfsetdash{}{0pt}%
\pgfpathmoveto{\pgfqpoint{4.843820in}{2.568800in}}%
\pgfpathlineto{\pgfqpoint{4.857593in}{2.566772in}}%
\pgfpathlineto{\pgfqpoint{4.871374in}{2.564815in}}%
\pgfpathlineto{\pgfqpoint{4.885163in}{2.562929in}}%
\pgfpathlineto{\pgfqpoint{4.898960in}{2.561112in}}%
\pgfpathlineto{\pgfqpoint{4.906527in}{2.570661in}}%
\pgfpathlineto{\pgfqpoint{4.914091in}{2.580361in}}%
\pgfpathlineto{\pgfqpoint{4.921653in}{2.590217in}}%
\pgfpathlineto{\pgfqpoint{4.929213in}{2.600238in}}%
\pgfpathlineto{\pgfqpoint{4.915432in}{2.602395in}}%
\pgfpathlineto{\pgfqpoint{4.901658in}{2.604623in}}%
\pgfpathlineto{\pgfqpoint{4.887893in}{2.606921in}}%
\pgfpathlineto{\pgfqpoint{4.874135in}{2.609290in}}%
\pgfpathlineto{\pgfqpoint{4.866560in}{2.598921in}}%
\pgfpathlineto{\pgfqpoint{4.858982in}{2.588721in}}%
\pgfpathlineto{\pgfqpoint{4.851402in}{2.578682in}}%
\pgfpathlineto{\pgfqpoint{4.843820in}{2.568800in}}%
\pgfpathclose%
\pgfusepath{fill}%
\end{pgfscope}%
\begin{pgfscope}%
\pgfpathrectangle{\pgfqpoint{1.150000in}{0.150000in}}{\pgfqpoint{5.700000in}{5.700000in}}%
\pgfusepath{clip}%
\pgfsetbuttcap%
\pgfsetroundjoin%
\definecolor{currentfill}{rgb}{0.280894,0.078907,0.402329}%
\pgfsetfillcolor{currentfill}%
\pgfsetfillopacity{0.700000}%
\pgfsetlinewidth{0.000000pt}%
\definecolor{currentstroke}{rgb}{0.000000,0.000000,0.000000}%
\pgfsetstrokecolor{currentstroke}%
\pgfsetdash{}{0pt}%
\pgfpathmoveto{\pgfqpoint{3.319200in}{2.312568in}}%
\pgfpathlineto{\pgfqpoint{3.332635in}{2.306708in}}%
\pgfpathlineto{\pgfqpoint{3.346073in}{2.300943in}}%
\pgfpathlineto{\pgfqpoint{3.359516in}{2.295275in}}%
\pgfpathlineto{\pgfqpoint{3.372962in}{2.289702in}}%
\pgfpathlineto{\pgfqpoint{3.381041in}{2.298993in}}%
\pgfpathlineto{\pgfqpoint{3.389114in}{2.308327in}}%
\pgfpathlineto{\pgfqpoint{3.397182in}{2.317707in}}%
\pgfpathlineto{\pgfqpoint{3.405243in}{2.327132in}}%
\pgfpathlineto{\pgfqpoint{3.391807in}{2.332762in}}%
\pgfpathlineto{\pgfqpoint{3.378376in}{2.338488in}}%
\pgfpathlineto{\pgfqpoint{3.364949in}{2.344309in}}%
\pgfpathlineto{\pgfqpoint{3.351525in}{2.350227in}}%
\pgfpathlineto{\pgfqpoint{3.343453in}{2.340737in}}%
\pgfpathlineto{\pgfqpoint{3.335375in}{2.331298in}}%
\pgfpathlineto{\pgfqpoint{3.327291in}{2.321909in}}%
\pgfpathlineto{\pgfqpoint{3.319200in}{2.312568in}}%
\pgfpathclose%
\pgfusepath{fill}%
\end{pgfscope}%
\begin{pgfscope}%
\pgfpathrectangle{\pgfqpoint{1.150000in}{0.150000in}}{\pgfqpoint{5.700000in}{5.700000in}}%
\pgfusepath{clip}%
\pgfsetbuttcap%
\pgfsetroundjoin%
\definecolor{currentfill}{rgb}{0.281446,0.084320,0.407414}%
\pgfsetfillcolor{currentfill}%
\pgfsetfillopacity{0.700000}%
\pgfsetlinewidth{0.000000pt}%
\definecolor{currentstroke}{rgb}{0.000000,0.000000,0.000000}%
\pgfsetstrokecolor{currentstroke}%
\pgfsetdash{}{0pt}%
\pgfpathmoveto{\pgfqpoint{3.684724in}{2.324032in}}%
\pgfpathlineto{\pgfqpoint{3.698216in}{2.319768in}}%
\pgfpathlineto{\pgfqpoint{3.711712in}{2.315590in}}%
\pgfpathlineto{\pgfqpoint{3.725214in}{2.311498in}}%
\pgfpathlineto{\pgfqpoint{3.738722in}{2.307493in}}%
\pgfpathlineto{\pgfqpoint{3.746677in}{2.316861in}}%
\pgfpathlineto{\pgfqpoint{3.754627in}{2.326270in}}%
\pgfpathlineto{\pgfqpoint{3.762571in}{2.335722in}}%
\pgfpathlineto{\pgfqpoint{3.770509in}{2.345219in}}%
\pgfpathlineto{\pgfqpoint{3.757012in}{2.349343in}}%
\pgfpathlineto{\pgfqpoint{3.743521in}{2.353553in}}%
\pgfpathlineto{\pgfqpoint{3.730034in}{2.357849in}}%
\pgfpathlineto{\pgfqpoint{3.716553in}{2.362232in}}%
\pgfpathlineto{\pgfqpoint{3.708604in}{2.352609in}}%
\pgfpathlineto{\pgfqpoint{3.700650in}{2.343036in}}%
\pgfpathlineto{\pgfqpoint{3.692690in}{2.333512in}}%
\pgfpathlineto{\pgfqpoint{3.684724in}{2.324032in}}%
\pgfpathclose%
\pgfusepath{fill}%
\end{pgfscope}%
\begin{pgfscope}%
\pgfpathrectangle{\pgfqpoint{1.150000in}{0.150000in}}{\pgfqpoint{5.700000in}{5.700000in}}%
\pgfusepath{clip}%
\pgfsetbuttcap%
\pgfsetroundjoin%
\definecolor{currentfill}{rgb}{0.253935,0.265254,0.529983}%
\pgfsetfillcolor{currentfill}%
\pgfsetfillopacity{0.700000}%
\pgfsetlinewidth{0.000000pt}%
\definecolor{currentstroke}{rgb}{0.000000,0.000000,0.000000}%
\pgfsetstrokecolor{currentstroke}%
\pgfsetdash{}{0pt}%
\pgfpathmoveto{\pgfqpoint{5.240666in}{2.692233in}}%
\pgfpathlineto{\pgfqpoint{5.254539in}{2.690104in}}%
\pgfpathlineto{\pgfqpoint{5.268420in}{2.688042in}}%
\pgfpathlineto{\pgfqpoint{5.282310in}{2.686049in}}%
\pgfpathlineto{\pgfqpoint{5.296209in}{2.684122in}}%
\pgfpathlineto{\pgfqpoint{5.303671in}{2.694805in}}%
\pgfpathlineto{\pgfqpoint{5.311134in}{2.705723in}}%
\pgfpathlineto{\pgfqpoint{5.318599in}{2.716884in}}%
\pgfpathlineto{\pgfqpoint{5.326064in}{2.728296in}}%
\pgfpathlineto{\pgfqpoint{5.312185in}{2.730643in}}%
\pgfpathlineto{\pgfqpoint{5.298314in}{2.733059in}}%
\pgfpathlineto{\pgfqpoint{5.284451in}{2.735541in}}%
\pgfpathlineto{\pgfqpoint{5.270597in}{2.738092in}}%
\pgfpathlineto{\pgfqpoint{5.263112in}{2.726252in}}%
\pgfpathlineto{\pgfqpoint{5.255629in}{2.714667in}}%
\pgfpathlineto{\pgfqpoint{5.248147in}{2.703330in}}%
\pgfpathlineto{\pgfqpoint{5.240666in}{2.692233in}}%
\pgfpathclose%
\pgfusepath{fill}%
\end{pgfscope}%
\begin{pgfscope}%
\pgfpathrectangle{\pgfqpoint{1.150000in}{0.150000in}}{\pgfqpoint{5.700000in}{5.700000in}}%
\pgfusepath{clip}%
\pgfsetbuttcap%
\pgfsetroundjoin%
\definecolor{currentfill}{rgb}{0.282656,0.100196,0.422160}%
\pgfsetfillcolor{currentfill}%
\pgfsetfillopacity{0.700000}%
\pgfsetlinewidth{0.000000pt}%
\definecolor{currentstroke}{rgb}{0.000000,0.000000,0.000000}%
\pgfsetstrokecolor{currentstroke}%
\pgfsetdash{}{0pt}%
\pgfpathmoveto{\pgfqpoint{3.910302in}{2.352656in}}%
\pgfpathlineto{\pgfqpoint{3.923841in}{2.349148in}}%
\pgfpathlineto{\pgfqpoint{3.937385in}{2.345723in}}%
\pgfpathlineto{\pgfqpoint{3.950935in}{2.342379in}}%
\pgfpathlineto{\pgfqpoint{3.964491in}{2.339117in}}%
\pgfpathlineto{\pgfqpoint{3.972371in}{2.348428in}}%
\pgfpathlineto{\pgfqpoint{3.980245in}{2.357786in}}%
\pgfpathlineto{\pgfqpoint{3.988113in}{2.367193in}}%
\pgfpathlineto{\pgfqpoint{3.995977in}{2.376654in}}%
\pgfpathlineto{\pgfqpoint{3.982431in}{2.380075in}}%
\pgfpathlineto{\pgfqpoint{3.968891in}{2.383577in}}%
\pgfpathlineto{\pgfqpoint{3.955357in}{2.387161in}}%
\pgfpathlineto{\pgfqpoint{3.941830in}{2.390828in}}%
\pgfpathlineto{\pgfqpoint{3.933956in}{2.381201in}}%
\pgfpathlineto{\pgfqpoint{3.926077in}{2.371632in}}%
\pgfpathlineto{\pgfqpoint{3.918192in}{2.362118in}}%
\pgfpathlineto{\pgfqpoint{3.910302in}{2.352656in}}%
\pgfpathclose%
\pgfusepath{fill}%
\end{pgfscope}%
\begin{pgfscope}%
\pgfpathrectangle{\pgfqpoint{1.150000in}{0.150000in}}{\pgfqpoint{5.700000in}{5.700000in}}%
\pgfusepath{clip}%
\pgfsetbuttcap%
\pgfsetroundjoin%
\definecolor{currentfill}{rgb}{0.282910,0.105393,0.426902}%
\pgfsetfillcolor{currentfill}%
\pgfsetfillopacity{0.700000}%
\pgfsetlinewidth{0.000000pt}%
\definecolor{currentstroke}{rgb}{0.000000,0.000000,0.000000}%
\pgfsetstrokecolor{currentstroke}%
\pgfsetdash{}{0pt}%
\pgfpathmoveto{\pgfqpoint{2.899143in}{2.373863in}}%
\pgfpathlineto{\pgfqpoint{2.912557in}{2.365542in}}%
\pgfpathlineto{\pgfqpoint{2.925972in}{2.357333in}}%
\pgfpathlineto{\pgfqpoint{2.939388in}{2.349237in}}%
\pgfpathlineto{\pgfqpoint{2.952806in}{2.341253in}}%
\pgfpathlineto{\pgfqpoint{2.961039in}{2.350038in}}%
\pgfpathlineto{\pgfqpoint{2.969265in}{2.358887in}}%
\pgfpathlineto{\pgfqpoint{2.977483in}{2.367799in}}%
\pgfpathlineto{\pgfqpoint{2.985694in}{2.376775in}}%
\pgfpathlineto{\pgfqpoint{2.972290in}{2.384755in}}%
\pgfpathlineto{\pgfqpoint{2.958887in}{2.392846in}}%
\pgfpathlineto{\pgfqpoint{2.945485in}{2.401050in}}%
\pgfpathlineto{\pgfqpoint{2.932085in}{2.409368in}}%
\pgfpathlineto{\pgfqpoint{2.923861in}{2.400389in}}%
\pgfpathlineto{\pgfqpoint{2.915629in}{2.391478in}}%
\pgfpathlineto{\pgfqpoint{2.907389in}{2.382637in}}%
\pgfpathlineto{\pgfqpoint{2.899143in}{2.373863in}}%
\pgfpathclose%
\pgfusepath{fill}%
\end{pgfscope}%
\begin{pgfscope}%
\pgfpathrectangle{\pgfqpoint{1.150000in}{0.150000in}}{\pgfqpoint{5.700000in}{5.700000in}}%
\pgfusepath{clip}%
\pgfsetbuttcap%
\pgfsetroundjoin%
\definecolor{currentfill}{rgb}{0.187231,0.414746,0.556547}%
\pgfsetfillcolor{currentfill}%
\pgfsetfillopacity{0.700000}%
\pgfsetlinewidth{0.000000pt}%
\definecolor{currentstroke}{rgb}{0.000000,0.000000,0.000000}%
\pgfsetstrokecolor{currentstroke}%
\pgfsetdash{}{0pt}%
\pgfpathmoveto{\pgfqpoint{5.925261in}{3.046444in}}%
\pgfpathlineto{\pgfqpoint{5.939239in}{3.042493in}}%
\pgfpathlineto{\pgfqpoint{5.953225in}{3.038605in}}%
\pgfpathlineto{\pgfqpoint{5.967219in}{3.034782in}}%
\pgfpathlineto{\pgfqpoint{5.981223in}{3.031024in}}%
\pgfpathlineto{\pgfqpoint{5.988729in}{3.048868in}}%
\pgfpathlineto{\pgfqpoint{5.996251in}{3.067198in}}%
\pgfpathlineto{\pgfqpoint{6.003789in}{3.086024in}}%
\pgfpathlineto{\pgfqpoint{5.989804in}{3.090218in}}%
\pgfpathlineto{\pgfqpoint{5.975827in}{3.094476in}}%
\pgfpathlineto{\pgfqpoint{5.961858in}{3.098797in}}%
\pgfpathlineto{\pgfqpoint{5.947898in}{3.103183in}}%
\pgfpathlineto{\pgfqpoint{5.940337in}{3.083772in}}%
\pgfpathlineto{\pgfqpoint{5.932791in}{3.064863in}}%
\pgfpathlineto{\pgfqpoint{5.925261in}{3.046444in}}%
\pgfpathclose%
\pgfusepath{fill}%
\end{pgfscope}%
\begin{pgfscope}%
\pgfpathrectangle{\pgfqpoint{1.150000in}{0.150000in}}{\pgfqpoint{5.700000in}{5.700000in}}%
\pgfusepath{clip}%
\pgfsetbuttcap%
\pgfsetroundjoin%
\definecolor{currentfill}{rgb}{0.277134,0.185228,0.489898}%
\pgfsetfillcolor{currentfill}%
\pgfsetfillopacity{0.700000}%
\pgfsetlinewidth{0.000000pt}%
\definecolor{currentstroke}{rgb}{0.000000,0.000000,0.000000}%
\pgfsetstrokecolor{currentstroke}%
\pgfsetdash{}{0pt}%
\pgfpathmoveto{\pgfqpoint{2.510267in}{2.543982in}}%
\pgfpathlineto{\pgfqpoint{2.523721in}{2.532575in}}%
\pgfpathlineto{\pgfqpoint{2.537173in}{2.521307in}}%
\pgfpathlineto{\pgfqpoint{2.550623in}{2.510176in}}%
\pgfpathlineto{\pgfqpoint{2.564072in}{2.499181in}}%
\pgfpathlineto{\pgfqpoint{2.572462in}{2.507146in}}%
\pgfpathlineto{\pgfqpoint{2.580843in}{2.515206in}}%
\pgfpathlineto{\pgfqpoint{2.589215in}{2.523358in}}%
\pgfpathlineto{\pgfqpoint{2.597578in}{2.531603in}}%
\pgfpathlineto{\pgfqpoint{2.584147in}{2.542551in}}%
\pgfpathlineto{\pgfqpoint{2.570714in}{2.553635in}}%
\pgfpathlineto{\pgfqpoint{2.557280in}{2.564857in}}%
\pgfpathlineto{\pgfqpoint{2.543844in}{2.576216in}}%
\pgfpathlineto{\pgfqpoint{2.535463in}{2.568011in}}%
\pgfpathlineto{\pgfqpoint{2.527074in}{2.559903in}}%
\pgfpathlineto{\pgfqpoint{2.518675in}{2.551893in}}%
\pgfpathlineto{\pgfqpoint{2.510267in}{2.543982in}}%
\pgfpathclose%
\pgfusepath{fill}%
\end{pgfscope}%
\begin{pgfscope}%
\pgfpathrectangle{\pgfqpoint{1.150000in}{0.150000in}}{\pgfqpoint{5.700000in}{5.700000in}}%
\pgfusepath{clip}%
\pgfsetbuttcap%
\pgfsetroundjoin%
\definecolor{currentfill}{rgb}{0.281412,0.155834,0.469201}%
\pgfsetfillcolor{currentfill}%
\pgfsetfillopacity{0.700000}%
\pgfsetlinewidth{0.000000pt}%
\definecolor{currentstroke}{rgb}{0.000000,0.000000,0.000000}%
\pgfsetstrokecolor{currentstroke}%
\pgfsetdash{}{0pt}%
\pgfpathmoveto{\pgfqpoint{4.447113in}{2.459623in}}%
\pgfpathlineto{\pgfqpoint{4.460784in}{2.457307in}}%
\pgfpathlineto{\pgfqpoint{4.474463in}{2.455065in}}%
\pgfpathlineto{\pgfqpoint{4.488149in}{2.452898in}}%
\pgfpathlineto{\pgfqpoint{4.501843in}{2.450804in}}%
\pgfpathlineto{\pgfqpoint{4.509541in}{2.459968in}}%
\pgfpathlineto{\pgfqpoint{4.517234in}{2.469222in}}%
\pgfpathlineto{\pgfqpoint{4.524922in}{2.478570in}}%
\pgfpathlineto{\pgfqpoint{4.532607in}{2.488017in}}%
\pgfpathlineto{\pgfqpoint{4.518925in}{2.490371in}}%
\pgfpathlineto{\pgfqpoint{4.505252in}{2.492799in}}%
\pgfpathlineto{\pgfqpoint{4.491586in}{2.495301in}}%
\pgfpathlineto{\pgfqpoint{4.477927in}{2.497878in}}%
\pgfpathlineto{\pgfqpoint{4.470230in}{2.488163in}}%
\pgfpathlineto{\pgfqpoint{4.462528in}{2.478552in}}%
\pgfpathlineto{\pgfqpoint{4.454823in}{2.469041in}}%
\pgfpathlineto{\pgfqpoint{4.447113in}{2.459623in}}%
\pgfpathclose%
\pgfusepath{fill}%
\end{pgfscope}%
\begin{pgfscope}%
\pgfpathrectangle{\pgfqpoint{1.150000in}{0.150000in}}{\pgfqpoint{5.700000in}{5.700000in}}%
\pgfusepath{clip}%
\pgfsetbuttcap%
\pgfsetroundjoin%
\definecolor{currentfill}{rgb}{0.280267,0.073417,0.397163}%
\pgfsetfillcolor{currentfill}%
\pgfsetfillopacity{0.700000}%
\pgfsetlinewidth{0.000000pt}%
\definecolor{currentstroke}{rgb}{0.000000,0.000000,0.000000}%
\pgfsetstrokecolor{currentstroke}%
\pgfsetdash{}{0pt}%
\pgfpathmoveto{\pgfqpoint{3.459023in}{2.305551in}}%
\pgfpathlineto{\pgfqpoint{3.472478in}{2.300388in}}%
\pgfpathlineto{\pgfqpoint{3.485938in}{2.295318in}}%
\pgfpathlineto{\pgfqpoint{3.499402in}{2.290339in}}%
\pgfpathlineto{\pgfqpoint{3.512871in}{2.285452in}}%
\pgfpathlineto{\pgfqpoint{3.520904in}{2.294784in}}%
\pgfpathlineto{\pgfqpoint{3.528932in}{2.304156in}}%
\pgfpathlineto{\pgfqpoint{3.536953in}{2.313569in}}%
\pgfpathlineto{\pgfqpoint{3.544969in}{2.323025in}}%
\pgfpathlineto{\pgfqpoint{3.531511in}{2.327991in}}%
\pgfpathlineto{\pgfqpoint{3.518057in}{2.333047in}}%
\pgfpathlineto{\pgfqpoint{3.504608in}{2.338195in}}%
\pgfpathlineto{\pgfqpoint{3.491163in}{2.343435in}}%
\pgfpathlineto{\pgfqpoint{3.483137in}{2.333894in}}%
\pgfpathlineto{\pgfqpoint{3.475105in}{2.324401in}}%
\pgfpathlineto{\pgfqpoint{3.467067in}{2.314954in}}%
\pgfpathlineto{\pgfqpoint{3.459023in}{2.305551in}}%
\pgfpathclose%
\pgfusepath{fill}%
\end{pgfscope}%
\begin{pgfscope}%
\pgfpathrectangle{\pgfqpoint{1.150000in}{0.150000in}}{\pgfqpoint{5.700000in}{5.700000in}}%
\pgfusepath{clip}%
\pgfsetbuttcap%
\pgfsetroundjoin%
\definecolor{currentfill}{rgb}{0.283229,0.120777,0.440584}%
\pgfsetfillcolor{currentfill}%
\pgfsetfillopacity{0.700000}%
\pgfsetlinewidth{0.000000pt}%
\definecolor{currentstroke}{rgb}{0.000000,0.000000,0.000000}%
\pgfsetstrokecolor{currentstroke}%
\pgfsetdash{}{0pt}%
\pgfpathmoveto{\pgfqpoint{4.135886in}{2.389190in}}%
\pgfpathlineto{\pgfqpoint{4.149479in}{2.386309in}}%
\pgfpathlineto{\pgfqpoint{4.163078in}{2.383506in}}%
\pgfpathlineto{\pgfqpoint{4.176685in}{2.380782in}}%
\pgfpathlineto{\pgfqpoint{4.190298in}{2.378135in}}%
\pgfpathlineto{\pgfqpoint{4.198102in}{2.387337in}}%
\pgfpathlineto{\pgfqpoint{4.205901in}{2.396598in}}%
\pgfpathlineto{\pgfqpoint{4.213695in}{2.405921in}}%
\pgfpathlineto{\pgfqpoint{4.221484in}{2.415311in}}%
\pgfpathlineto{\pgfqpoint{4.207882in}{2.418157in}}%
\pgfpathlineto{\pgfqpoint{4.194286in}{2.421081in}}%
\pgfpathlineto{\pgfqpoint{4.180698in}{2.424083in}}%
\pgfpathlineto{\pgfqpoint{4.167116in}{2.427163in}}%
\pgfpathlineto{\pgfqpoint{4.159316in}{2.417566in}}%
\pgfpathlineto{\pgfqpoint{4.151511in}{2.408041in}}%
\pgfpathlineto{\pgfqpoint{4.143701in}{2.398583in}}%
\pgfpathlineto{\pgfqpoint{4.135886in}{2.389190in}}%
\pgfpathclose%
\pgfusepath{fill}%
\end{pgfscope}%
\begin{pgfscope}%
\pgfpathrectangle{\pgfqpoint{1.150000in}{0.150000in}}{\pgfqpoint{5.700000in}{5.700000in}}%
\pgfusepath{clip}%
\pgfsetbuttcap%
\pgfsetroundjoin%
\definecolor{currentfill}{rgb}{0.275191,0.194905,0.496005}%
\pgfsetfillcolor{currentfill}%
\pgfsetfillopacity{0.700000}%
\pgfsetlinewidth{0.000000pt}%
\definecolor{currentstroke}{rgb}{0.000000,0.000000,0.000000}%
\pgfsetstrokecolor{currentstroke}%
\pgfsetdash{}{0pt}%
\pgfpathmoveto{\pgfqpoint{4.758392in}{2.538248in}}%
\pgfpathlineto{\pgfqpoint{4.772147in}{2.536257in}}%
\pgfpathlineto{\pgfqpoint{4.785910in}{2.534338in}}%
\pgfpathlineto{\pgfqpoint{4.799682in}{2.532489in}}%
\pgfpathlineto{\pgfqpoint{4.813461in}{2.530712in}}%
\pgfpathlineto{\pgfqpoint{4.821056in}{2.540029in}}%
\pgfpathlineto{\pgfqpoint{4.828647in}{2.549480in}}%
\pgfpathlineto{\pgfqpoint{4.836235in}{2.559068in}}%
\pgfpathlineto{\pgfqpoint{4.843820in}{2.568800in}}%
\pgfpathlineto{\pgfqpoint{4.830055in}{2.570899in}}%
\pgfpathlineto{\pgfqpoint{4.816299in}{2.573068in}}%
\pgfpathlineto{\pgfqpoint{4.802550in}{2.575309in}}%
\pgfpathlineto{\pgfqpoint{4.788810in}{2.577621in}}%
\pgfpathlineto{\pgfqpoint{4.781210in}{2.567560in}}%
\pgfpathlineto{\pgfqpoint{4.773607in}{2.557649in}}%
\pgfpathlineto{\pgfqpoint{4.766001in}{2.547880in}}%
\pgfpathlineto{\pgfqpoint{4.758392in}{2.538248in}}%
\pgfpathclose%
\pgfusepath{fill}%
\end{pgfscope}%
\begin{pgfscope}%
\pgfpathrectangle{\pgfqpoint{1.150000in}{0.150000in}}{\pgfqpoint{5.700000in}{5.700000in}}%
\pgfusepath{clip}%
\pgfsetbuttcap%
\pgfsetroundjoin%
\definecolor{currentfill}{rgb}{0.258965,0.251537,0.524736}%
\pgfsetfillcolor{currentfill}%
\pgfsetfillopacity{0.700000}%
\pgfsetlinewidth{0.000000pt}%
\definecolor{currentstroke}{rgb}{0.000000,0.000000,0.000000}%
\pgfsetstrokecolor{currentstroke}%
\pgfsetdash{}{0pt}%
\pgfpathmoveto{\pgfqpoint{5.155266in}{2.657687in}}%
\pgfpathlineto{\pgfqpoint{5.169123in}{2.655688in}}%
\pgfpathlineto{\pgfqpoint{5.182989in}{2.653756in}}%
\pgfpathlineto{\pgfqpoint{5.196863in}{2.651893in}}%
\pgfpathlineto{\pgfqpoint{5.210746in}{2.650097in}}%
\pgfpathlineto{\pgfqpoint{5.218226in}{2.660308in}}%
\pgfpathlineto{\pgfqpoint{5.225705in}{2.670729in}}%
\pgfpathlineto{\pgfqpoint{5.233185in}{2.681368in}}%
\pgfpathlineto{\pgfqpoint{5.240666in}{2.692233in}}%
\pgfpathlineto{\pgfqpoint{5.226801in}{2.694430in}}%
\pgfpathlineto{\pgfqpoint{5.212946in}{2.696695in}}%
\pgfpathlineto{\pgfqpoint{5.199098in}{2.699028in}}%
\pgfpathlineto{\pgfqpoint{5.185260in}{2.701429in}}%
\pgfpathlineto{\pgfqpoint{5.177761in}{2.690156in}}%
\pgfpathlineto{\pgfqpoint{5.170263in}{2.679113in}}%
\pgfpathlineto{\pgfqpoint{5.162765in}{2.668292in}}%
\pgfpathlineto{\pgfqpoint{5.155266in}{2.657687in}}%
\pgfpathclose%
\pgfusepath{fill}%
\end{pgfscope}%
\begin{pgfscope}%
\pgfpathrectangle{\pgfqpoint{1.150000in}{0.150000in}}{\pgfqpoint{5.700000in}{5.700000in}}%
\pgfusepath{clip}%
\pgfsetbuttcap%
\pgfsetroundjoin%
\definecolor{currentfill}{rgb}{0.283187,0.125848,0.444960}%
\pgfsetfillcolor{currentfill}%
\pgfsetfillopacity{0.700000}%
\pgfsetlinewidth{0.000000pt}%
\definecolor{currentstroke}{rgb}{0.000000,0.000000,0.000000}%
\pgfsetstrokecolor{currentstroke}%
\pgfsetdash{}{0pt}%
\pgfpathmoveto{\pgfqpoint{2.758682in}{2.410402in}}%
\pgfpathlineto{\pgfqpoint{2.772106in}{2.401117in}}%
\pgfpathlineto{\pgfqpoint{2.785529in}{2.391952in}}%
\pgfpathlineto{\pgfqpoint{2.798953in}{2.382907in}}%
\pgfpathlineto{\pgfqpoint{2.812377in}{2.373981in}}%
\pgfpathlineto{\pgfqpoint{2.820669in}{2.382452in}}%
\pgfpathlineto{\pgfqpoint{2.828953in}{2.390996in}}%
\pgfpathlineto{\pgfqpoint{2.837229in}{2.399611in}}%
\pgfpathlineto{\pgfqpoint{2.845497in}{2.408300in}}%
\pgfpathlineto{\pgfqpoint{2.832088in}{2.417201in}}%
\pgfpathlineto{\pgfqpoint{2.818679in}{2.426221in}}%
\pgfpathlineto{\pgfqpoint{2.805270in}{2.435360in}}%
\pgfpathlineto{\pgfqpoint{2.791862in}{2.444620in}}%
\pgfpathlineto{\pgfqpoint{2.783579in}{2.435949in}}%
\pgfpathlineto{\pgfqpoint{2.775288in}{2.427356in}}%
\pgfpathlineto{\pgfqpoint{2.766989in}{2.418840in}}%
\pgfpathlineto{\pgfqpoint{2.758682in}{2.410402in}}%
\pgfpathclose%
\pgfusepath{fill}%
\end{pgfscope}%
\begin{pgfscope}%
\pgfpathrectangle{\pgfqpoint{1.150000in}{0.150000in}}{\pgfqpoint{5.700000in}{5.700000in}}%
\pgfusepath{clip}%
\pgfsetbuttcap%
\pgfsetroundjoin%
\definecolor{currentfill}{rgb}{0.282327,0.094955,0.417331}%
\pgfsetfillcolor{currentfill}%
\pgfsetfillopacity{0.700000}%
\pgfsetlinewidth{0.000000pt}%
\definecolor{currentstroke}{rgb}{0.000000,0.000000,0.000000}%
\pgfsetstrokecolor{currentstroke}%
\pgfsetdash{}{0pt}%
\pgfpathmoveto{\pgfqpoint{3.824553in}{2.329573in}}%
\pgfpathlineto{\pgfqpoint{3.838078in}{2.325872in}}%
\pgfpathlineto{\pgfqpoint{3.851608in}{2.322255in}}%
\pgfpathlineto{\pgfqpoint{3.865145in}{2.318721in}}%
\pgfpathlineto{\pgfqpoint{3.878687in}{2.315270in}}%
\pgfpathlineto{\pgfqpoint{3.886600in}{2.324553in}}%
\pgfpathlineto{\pgfqpoint{3.894506in}{2.333876in}}%
\pgfpathlineto{\pgfqpoint{3.902407in}{2.343243in}}%
\pgfpathlineto{\pgfqpoint{3.910302in}{2.352656in}}%
\pgfpathlineto{\pgfqpoint{3.896770in}{2.356246in}}%
\pgfpathlineto{\pgfqpoint{3.883244in}{2.359919in}}%
\pgfpathlineto{\pgfqpoint{3.869724in}{2.363674in}}%
\pgfpathlineto{\pgfqpoint{3.856209in}{2.367514in}}%
\pgfpathlineto{\pgfqpoint{3.848303in}{2.357955in}}%
\pgfpathlineto{\pgfqpoint{3.840392in}{2.348447in}}%
\pgfpathlineto{\pgfqpoint{3.832475in}{2.338987in}}%
\pgfpathlineto{\pgfqpoint{3.824553in}{2.329573in}}%
\pgfpathclose%
\pgfusepath{fill}%
\end{pgfscope}%
\begin{pgfscope}%
\pgfpathrectangle{\pgfqpoint{1.150000in}{0.150000in}}{\pgfqpoint{5.700000in}{5.700000in}}%
\pgfusepath{clip}%
\pgfsetbuttcap%
\pgfsetroundjoin%
\definecolor{currentfill}{rgb}{0.263663,0.237631,0.518762}%
\pgfsetfillcolor{currentfill}%
\pgfsetfillopacity{0.700000}%
\pgfsetlinewidth{0.000000pt}%
\definecolor{currentstroke}{rgb}{0.000000,0.000000,0.000000}%
\pgfsetstrokecolor{currentstroke}%
\pgfsetdash{}{0pt}%
\pgfpathmoveto{\pgfqpoint{5.069855in}{2.624443in}}%
\pgfpathlineto{\pgfqpoint{5.083696in}{2.622550in}}%
\pgfpathlineto{\pgfqpoint{5.097544in}{2.620726in}}%
\pgfpathlineto{\pgfqpoint{5.111402in}{2.618971in}}%
\pgfpathlineto{\pgfqpoint{5.125268in}{2.617285in}}%
\pgfpathlineto{\pgfqpoint{5.132769in}{2.627097in}}%
\pgfpathlineto{\pgfqpoint{5.140269in}{2.637097in}}%
\pgfpathlineto{\pgfqpoint{5.147768in}{2.647291in}}%
\pgfpathlineto{\pgfqpoint{5.155266in}{2.657687in}}%
\pgfpathlineto{\pgfqpoint{5.141418in}{2.659756in}}%
\pgfpathlineto{\pgfqpoint{5.127578in}{2.661893in}}%
\pgfpathlineto{\pgfqpoint{5.113747in}{2.664098in}}%
\pgfpathlineto{\pgfqpoint{5.099924in}{2.666372in}}%
\pgfpathlineto{\pgfqpoint{5.092408in}{2.655587in}}%
\pgfpathlineto{\pgfqpoint{5.084891in}{2.645009in}}%
\pgfpathlineto{\pgfqpoint{5.077374in}{2.634630in}}%
\pgfpathlineto{\pgfqpoint{5.069855in}{2.624443in}}%
\pgfpathclose%
\pgfusepath{fill}%
\end{pgfscope}%
\begin{pgfscope}%
\pgfpathrectangle{\pgfqpoint{1.150000in}{0.150000in}}{\pgfqpoint{5.700000in}{5.700000in}}%
\pgfusepath{clip}%
\pgfsetbuttcap%
\pgfsetroundjoin%
\definecolor{currentfill}{rgb}{0.282290,0.145912,0.461510}%
\pgfsetfillcolor{currentfill}%
\pgfsetfillopacity{0.700000}%
\pgfsetlinewidth{0.000000pt}%
\definecolor{currentstroke}{rgb}{0.000000,0.000000,0.000000}%
\pgfsetstrokecolor{currentstroke}%
\pgfsetdash{}{0pt}%
\pgfpathmoveto{\pgfqpoint{4.361565in}{2.431861in}}%
\pgfpathlineto{\pgfqpoint{4.375220in}{2.429485in}}%
\pgfpathlineto{\pgfqpoint{4.388881in}{2.427184in}}%
\pgfpathlineto{\pgfqpoint{4.402550in}{2.424958in}}%
\pgfpathlineto{\pgfqpoint{4.416226in}{2.422807in}}%
\pgfpathlineto{\pgfqpoint{4.423955in}{2.431892in}}%
\pgfpathlineto{\pgfqpoint{4.431679in}{2.441054in}}%
\pgfpathlineto{\pgfqpoint{4.439398in}{2.450296in}}%
\pgfpathlineto{\pgfqpoint{4.447113in}{2.459623in}}%
\pgfpathlineto{\pgfqpoint{4.433449in}{2.462015in}}%
\pgfpathlineto{\pgfqpoint{4.419792in}{2.464481in}}%
\pgfpathlineto{\pgfqpoint{4.406143in}{2.467022in}}%
\pgfpathlineto{\pgfqpoint{4.392501in}{2.469638in}}%
\pgfpathlineto{\pgfqpoint{4.384774in}{2.460063in}}%
\pgfpathlineto{\pgfqpoint{4.377042in}{2.450578in}}%
\pgfpathlineto{\pgfqpoint{4.369306in}{2.441179in}}%
\pgfpathlineto{\pgfqpoint{4.361565in}{2.431861in}}%
\pgfpathclose%
\pgfusepath{fill}%
\end{pgfscope}%
\begin{pgfscope}%
\pgfpathrectangle{\pgfqpoint{1.150000in}{0.150000in}}{\pgfqpoint{5.700000in}{5.700000in}}%
\pgfusepath{clip}%
\pgfsetbuttcap%
\pgfsetroundjoin%
\definecolor{currentfill}{rgb}{0.279574,0.170599,0.479997}%
\pgfsetfillcolor{currentfill}%
\pgfsetfillopacity{0.700000}%
\pgfsetlinewidth{0.000000pt}%
\definecolor{currentstroke}{rgb}{0.000000,0.000000,0.000000}%
\pgfsetstrokecolor{currentstroke}%
\pgfsetdash{}{0pt}%
\pgfpathmoveto{\pgfqpoint{2.564072in}{2.499181in}}%
\pgfpathlineto{\pgfqpoint{2.577519in}{2.488320in}}%
\pgfpathlineto{\pgfqpoint{2.590964in}{2.477594in}}%
\pgfpathlineto{\pgfqpoint{2.604409in}{2.466999in}}%
\pgfpathlineto{\pgfqpoint{2.617852in}{2.456536in}}%
\pgfpathlineto{\pgfqpoint{2.626225in}{2.464556in}}%
\pgfpathlineto{\pgfqpoint{2.634589in}{2.472665in}}%
\pgfpathlineto{\pgfqpoint{2.642944in}{2.480861in}}%
\pgfpathlineto{\pgfqpoint{2.651291in}{2.489145in}}%
\pgfpathlineto{\pgfqpoint{2.637864in}{2.499562in}}%
\pgfpathlineto{\pgfqpoint{2.624437in}{2.510109in}}%
\pgfpathlineto{\pgfqpoint{2.611008in}{2.520789in}}%
\pgfpathlineto{\pgfqpoint{2.597578in}{2.531603in}}%
\pgfpathlineto{\pgfqpoint{2.589215in}{2.523358in}}%
\pgfpathlineto{\pgfqpoint{2.580843in}{2.515206in}}%
\pgfpathlineto{\pgfqpoint{2.572462in}{2.507146in}}%
\pgfpathlineto{\pgfqpoint{2.564072in}{2.499181in}}%
\pgfpathclose%
\pgfusepath{fill}%
\end{pgfscope}%
\begin{pgfscope}%
\pgfpathrectangle{\pgfqpoint{1.150000in}{0.150000in}}{\pgfqpoint{5.700000in}{5.700000in}}%
\pgfusepath{clip}%
\pgfsetbuttcap%
\pgfsetroundjoin%
\definecolor{currentfill}{rgb}{0.280894,0.078907,0.402329}%
\pgfsetfillcolor{currentfill}%
\pgfsetfillopacity{0.700000}%
\pgfsetlinewidth{0.000000pt}%
\definecolor{currentstroke}{rgb}{0.000000,0.000000,0.000000}%
\pgfsetstrokecolor{currentstroke}%
\pgfsetdash{}{0pt}%
\pgfpathmoveto{\pgfqpoint{3.598847in}{2.304065in}}%
\pgfpathlineto{\pgfqpoint{3.612329in}{2.299548in}}%
\pgfpathlineto{\pgfqpoint{3.625816in}{2.295119in}}%
\pgfpathlineto{\pgfqpoint{3.639307in}{2.290778in}}%
\pgfpathlineto{\pgfqpoint{3.652804in}{2.286525in}}%
\pgfpathlineto{\pgfqpoint{3.660793in}{2.295845in}}%
\pgfpathlineto{\pgfqpoint{3.668776in}{2.305201in}}%
\pgfpathlineto{\pgfqpoint{3.676753in}{2.314596in}}%
\pgfpathlineto{\pgfqpoint{3.684724in}{2.324032in}}%
\pgfpathlineto{\pgfqpoint{3.671238in}{2.328384in}}%
\pgfpathlineto{\pgfqpoint{3.657757in}{2.332823in}}%
\pgfpathlineto{\pgfqpoint{3.644281in}{2.337349in}}%
\pgfpathlineto{\pgfqpoint{3.630809in}{2.341964in}}%
\pgfpathlineto{\pgfqpoint{3.622828in}{2.332423in}}%
\pgfpathlineto{\pgfqpoint{3.614840in}{2.322927in}}%
\pgfpathlineto{\pgfqpoint{3.606847in}{2.313475in}}%
\pgfpathlineto{\pgfqpoint{3.598847in}{2.304065in}}%
\pgfpathclose%
\pgfusepath{fill}%
\end{pgfscope}%
\begin{pgfscope}%
\pgfpathrectangle{\pgfqpoint{1.150000in}{0.150000in}}{\pgfqpoint{5.700000in}{5.700000in}}%
\pgfusepath{clip}%
\pgfsetbuttcap%
\pgfsetroundjoin%
\definecolor{currentfill}{rgb}{0.277134,0.185228,0.489898}%
\pgfsetfillcolor{currentfill}%
\pgfsetfillopacity{0.700000}%
\pgfsetlinewidth{0.000000pt}%
\definecolor{currentstroke}{rgb}{0.000000,0.000000,0.000000}%
\pgfsetstrokecolor{currentstroke}%
\pgfsetdash{}{0pt}%
\pgfpathmoveto{\pgfqpoint{4.672922in}{2.508460in}}%
\pgfpathlineto{\pgfqpoint{4.686660in}{2.506483in}}%
\pgfpathlineto{\pgfqpoint{4.700405in}{2.504578in}}%
\pgfpathlineto{\pgfqpoint{4.714159in}{2.502745in}}%
\pgfpathlineto{\pgfqpoint{4.727921in}{2.500983in}}%
\pgfpathlineto{\pgfqpoint{4.735544in}{2.510121in}}%
\pgfpathlineto{\pgfqpoint{4.743163in}{2.519375in}}%
\pgfpathlineto{\pgfqpoint{4.750779in}{2.528748in}}%
\pgfpathlineto{\pgfqpoint{4.758392in}{2.538248in}}%
\pgfpathlineto{\pgfqpoint{4.744644in}{2.540311in}}%
\pgfpathlineto{\pgfqpoint{4.730905in}{2.542445in}}%
\pgfpathlineto{\pgfqpoint{4.717174in}{2.544651in}}%
\pgfpathlineto{\pgfqpoint{4.703451in}{2.546929in}}%
\pgfpathlineto{\pgfqpoint{4.695824in}{2.537121in}}%
\pgfpathlineto{\pgfqpoint{4.688194in}{2.527444in}}%
\pgfpathlineto{\pgfqpoint{4.680560in}{2.517892in}}%
\pgfpathlineto{\pgfqpoint{4.672922in}{2.508460in}}%
\pgfpathclose%
\pgfusepath{fill}%
\end{pgfscope}%
\begin{pgfscope}%
\pgfpathrectangle{\pgfqpoint{1.150000in}{0.150000in}}{\pgfqpoint{5.700000in}{5.700000in}}%
\pgfusepath{clip}%
\pgfsetbuttcap%
\pgfsetroundjoin%
\definecolor{currentfill}{rgb}{0.214298,0.355619,0.551184}%
\pgfsetfillcolor{currentfill}%
\pgfsetfillopacity{0.700000}%
\pgfsetlinewidth{0.000000pt}%
\definecolor{currentstroke}{rgb}{0.000000,0.000000,0.000000}%
\pgfsetstrokecolor{currentstroke}%
\pgfsetdash{}{0pt}%
\pgfpathmoveto{\pgfqpoint{5.723781in}{2.880847in}}%
\pgfpathlineto{\pgfqpoint{5.737760in}{2.878003in}}%
\pgfpathlineto{\pgfqpoint{5.751747in}{2.875225in}}%
\pgfpathlineto{\pgfqpoint{5.765743in}{2.872511in}}%
\pgfpathlineto{\pgfqpoint{5.779748in}{2.869862in}}%
\pgfpathlineto{\pgfqpoint{5.787166in}{2.883709in}}%
\pgfpathlineto{\pgfqpoint{5.794592in}{2.897930in}}%
\pgfpathlineto{\pgfqpoint{5.802027in}{2.912536in}}%
\pgfpathlineto{\pgfqpoint{5.809472in}{2.927536in}}%
\pgfpathlineto{\pgfqpoint{5.795491in}{2.930706in}}%
\pgfpathlineto{\pgfqpoint{5.781518in}{2.933942in}}%
\pgfpathlineto{\pgfqpoint{5.767554in}{2.937242in}}%
\pgfpathlineto{\pgfqpoint{5.753598in}{2.940608in}}%
\pgfpathlineto{\pgfqpoint{5.746130in}{2.925079in}}%
\pgfpathlineto{\pgfqpoint{5.738672in}{2.909949in}}%
\pgfpathlineto{\pgfqpoint{5.731222in}{2.895208in}}%
\pgfpathlineto{\pgfqpoint{5.723781in}{2.880847in}}%
\pgfpathclose%
\pgfusepath{fill}%
\end{pgfscope}%
\begin{pgfscope}%
\pgfpathrectangle{\pgfqpoint{1.150000in}{0.150000in}}{\pgfqpoint{5.700000in}{5.700000in}}%
\pgfusepath{clip}%
\pgfsetbuttcap%
\pgfsetroundjoin%
\definecolor{currentfill}{rgb}{0.281446,0.084320,0.407414}%
\pgfsetfillcolor{currentfill}%
\pgfsetfillopacity{0.700000}%
\pgfsetlinewidth{0.000000pt}%
\definecolor{currentstroke}{rgb}{0.000000,0.000000,0.000000}%
\pgfsetstrokecolor{currentstroke}%
\pgfsetdash{}{0pt}%
\pgfpathmoveto{\pgfqpoint{3.092995in}{2.316854in}}%
\pgfpathlineto{\pgfqpoint{3.106418in}{2.309843in}}%
\pgfpathlineto{\pgfqpoint{3.119842in}{2.302936in}}%
\pgfpathlineto{\pgfqpoint{3.133270in}{2.296132in}}%
\pgfpathlineto{\pgfqpoint{3.146700in}{2.289430in}}%
\pgfpathlineto{\pgfqpoint{3.154866in}{2.298444in}}%
\pgfpathlineto{\pgfqpoint{3.163025in}{2.307509in}}%
\pgfpathlineto{\pgfqpoint{3.171177in}{2.316624in}}%
\pgfpathlineto{\pgfqpoint{3.179323in}{2.325790in}}%
\pgfpathlineto{\pgfqpoint{3.165905in}{2.332508in}}%
\pgfpathlineto{\pgfqpoint{3.152490in}{2.339329in}}%
\pgfpathlineto{\pgfqpoint{3.139078in}{2.346253in}}%
\pgfpathlineto{\pgfqpoint{3.125668in}{2.353280in}}%
\pgfpathlineto{\pgfqpoint{3.117510in}{2.344090in}}%
\pgfpathlineto{\pgfqpoint{3.109345in}{2.334956in}}%
\pgfpathlineto{\pgfqpoint{3.101173in}{2.325877in}}%
\pgfpathlineto{\pgfqpoint{3.092995in}{2.316854in}}%
\pgfpathclose%
\pgfusepath{fill}%
\end{pgfscope}%
\begin{pgfscope}%
\pgfpathrectangle{\pgfqpoint{1.150000in}{0.150000in}}{\pgfqpoint{5.700000in}{5.700000in}}%
\pgfusepath{clip}%
\pgfsetbuttcap%
\pgfsetroundjoin%
\definecolor{currentfill}{rgb}{0.223925,0.334994,0.548053}%
\pgfsetfillcolor{currentfill}%
\pgfsetfillopacity{0.700000}%
\pgfsetlinewidth{0.000000pt}%
\definecolor{currentstroke}{rgb}{0.000000,0.000000,0.000000}%
\pgfsetstrokecolor{currentstroke}%
\pgfsetdash{}{0pt}%
\pgfpathmoveto{\pgfqpoint{5.638177in}{2.837033in}}%
\pgfpathlineto{\pgfqpoint{5.652143in}{2.834430in}}%
\pgfpathlineto{\pgfqpoint{5.666118in}{2.831892in}}%
\pgfpathlineto{\pgfqpoint{5.680101in}{2.829420in}}%
\pgfpathlineto{\pgfqpoint{5.694094in}{2.827013in}}%
\pgfpathlineto{\pgfqpoint{5.701505in}{2.839948in}}%
\pgfpathlineto{\pgfqpoint{5.708923in}{2.853226in}}%
\pgfpathlineto{\pgfqpoint{5.716348in}{2.866856in}}%
\pgfpathlineto{\pgfqpoint{5.723781in}{2.880847in}}%
\pgfpathlineto{\pgfqpoint{5.709812in}{2.883756in}}%
\pgfpathlineto{\pgfqpoint{5.695851in}{2.886730in}}%
\pgfpathlineto{\pgfqpoint{5.681899in}{2.889770in}}%
\pgfpathlineto{\pgfqpoint{5.667955in}{2.892875in}}%
\pgfpathlineto{\pgfqpoint{5.660500in}{2.878375in}}%
\pgfpathlineto{\pgfqpoint{5.653052in}{2.864241in}}%
\pgfpathlineto{\pgfqpoint{5.645612in}{2.850463in}}%
\pgfpathlineto{\pgfqpoint{5.638177in}{2.837033in}}%
\pgfpathclose%
\pgfusepath{fill}%
\end{pgfscope}%
\begin{pgfscope}%
\pgfpathrectangle{\pgfqpoint{1.150000in}{0.150000in}}{\pgfqpoint{5.700000in}{5.700000in}}%
\pgfusepath{clip}%
\pgfsetbuttcap%
\pgfsetroundjoin%
\definecolor{currentfill}{rgb}{0.283091,0.110553,0.431554}%
\pgfsetfillcolor{currentfill}%
\pgfsetfillopacity{0.700000}%
\pgfsetlinewidth{0.000000pt}%
\definecolor{currentstroke}{rgb}{0.000000,0.000000,0.000000}%
\pgfsetstrokecolor{currentstroke}%
\pgfsetdash{}{0pt}%
\pgfpathmoveto{\pgfqpoint{4.050223in}{2.363778in}}%
\pgfpathlineto{\pgfqpoint{4.063800in}{2.360760in}}%
\pgfpathlineto{\pgfqpoint{4.077384in}{2.357821in}}%
\pgfpathlineto{\pgfqpoint{4.090975in}{2.354962in}}%
\pgfpathlineto{\pgfqpoint{4.104572in}{2.352182in}}%
\pgfpathlineto{\pgfqpoint{4.112408in}{2.361356in}}%
\pgfpathlineto{\pgfqpoint{4.120239in}{2.370579in}}%
\pgfpathlineto{\pgfqpoint{4.128065in}{2.379856in}}%
\pgfpathlineto{\pgfqpoint{4.135886in}{2.389190in}}%
\pgfpathlineto{\pgfqpoint{4.122299in}{2.392149in}}%
\pgfpathlineto{\pgfqpoint{4.108720in}{2.395188in}}%
\pgfpathlineto{\pgfqpoint{4.095146in}{2.398306in}}%
\pgfpathlineto{\pgfqpoint{4.081580in}{2.401504in}}%
\pgfpathlineto{\pgfqpoint{4.073748in}{2.391983in}}%
\pgfpathlineto{\pgfqpoint{4.065912in}{2.382525in}}%
\pgfpathlineto{\pgfqpoint{4.058070in}{2.373124in}}%
\pgfpathlineto{\pgfqpoint{4.050223in}{2.363778in}}%
\pgfpathclose%
\pgfusepath{fill}%
\end{pgfscope}%
\begin{pgfscope}%
\pgfpathrectangle{\pgfqpoint{1.150000in}{0.150000in}}{\pgfqpoint{5.700000in}{5.700000in}}%
\pgfusepath{clip}%
\pgfsetbuttcap%
\pgfsetroundjoin%
\definecolor{currentfill}{rgb}{0.280267,0.073417,0.397163}%
\pgfsetfillcolor{currentfill}%
\pgfsetfillopacity{0.700000}%
\pgfsetlinewidth{0.000000pt}%
\definecolor{currentstroke}{rgb}{0.000000,0.000000,0.000000}%
\pgfsetstrokecolor{currentstroke}%
\pgfsetdash{}{0pt}%
\pgfpathmoveto{\pgfqpoint{3.233024in}{2.299931in}}%
\pgfpathlineto{\pgfqpoint{3.246456in}{2.293716in}}%
\pgfpathlineto{\pgfqpoint{3.259893in}{2.287600in}}%
\pgfpathlineto{\pgfqpoint{3.273332in}{2.281582in}}%
\pgfpathlineto{\pgfqpoint{3.286775in}{2.275661in}}%
\pgfpathlineto{\pgfqpoint{3.294891in}{2.284822in}}%
\pgfpathlineto{\pgfqpoint{3.303000in}{2.294026in}}%
\pgfpathlineto{\pgfqpoint{3.311103in}{2.303274in}}%
\pgfpathlineto{\pgfqpoint{3.319200in}{2.312568in}}%
\pgfpathlineto{\pgfqpoint{3.305769in}{2.318526in}}%
\pgfpathlineto{\pgfqpoint{3.292341in}{2.324581in}}%
\pgfpathlineto{\pgfqpoint{3.278916in}{2.330734in}}%
\pgfpathlineto{\pgfqpoint{3.265495in}{2.336986in}}%
\pgfpathlineto{\pgfqpoint{3.257387in}{2.327647in}}%
\pgfpathlineto{\pgfqpoint{3.249272in}{2.318359in}}%
\pgfpathlineto{\pgfqpoint{3.241151in}{2.309121in}}%
\pgfpathlineto{\pgfqpoint{3.233024in}{2.299931in}}%
\pgfpathclose%
\pgfusepath{fill}%
\end{pgfscope}%
\begin{pgfscope}%
\pgfpathrectangle{\pgfqpoint{1.150000in}{0.150000in}}{\pgfqpoint{5.700000in}{5.700000in}}%
\pgfusepath{clip}%
\pgfsetbuttcap%
\pgfsetroundjoin%
\definecolor{currentfill}{rgb}{0.206756,0.371758,0.553117}%
\pgfsetfillcolor{currentfill}%
\pgfsetfillopacity{0.700000}%
\pgfsetlinewidth{0.000000pt}%
\definecolor{currentstroke}{rgb}{0.000000,0.000000,0.000000}%
\pgfsetstrokecolor{currentstroke}%
\pgfsetdash{}{0pt}%
\pgfpathmoveto{\pgfqpoint{5.809472in}{2.927536in}}%
\pgfpathlineto{\pgfqpoint{5.823463in}{2.924430in}}%
\pgfpathlineto{\pgfqpoint{5.837462in}{2.921389in}}%
\pgfpathlineto{\pgfqpoint{5.851470in}{2.918412in}}%
\pgfpathlineto{\pgfqpoint{5.865486in}{2.915500in}}%
\pgfpathlineto{\pgfqpoint{5.872917in}{2.930369in}}%
\pgfpathlineto{\pgfqpoint{5.880359in}{2.945646in}}%
\pgfpathlineto{\pgfqpoint{5.887812in}{2.961341in}}%
\pgfpathlineto{\pgfqpoint{5.895276in}{2.977464in}}%
\pgfpathlineto{\pgfqpoint{5.881284in}{2.980918in}}%
\pgfpathlineto{\pgfqpoint{5.867299in}{2.984436in}}%
\pgfpathlineto{\pgfqpoint{5.853324in}{2.988019in}}%
\pgfpathlineto{\pgfqpoint{5.839357in}{2.991667in}}%
\pgfpathlineto{\pgfqpoint{5.831869in}{2.974994in}}%
\pgfpathlineto{\pgfqpoint{5.824393in}{2.958755in}}%
\pgfpathlineto{\pgfqpoint{5.816927in}{2.942939in}}%
\pgfpathlineto{\pgfqpoint{5.809472in}{2.927536in}}%
\pgfpathclose%
\pgfusepath{fill}%
\end{pgfscope}%
\begin{pgfscope}%
\pgfpathrectangle{\pgfqpoint{1.150000in}{0.150000in}}{\pgfqpoint{5.700000in}{5.700000in}}%
\pgfusepath{clip}%
\pgfsetbuttcap%
\pgfsetroundjoin%
\definecolor{currentfill}{rgb}{0.231674,0.318106,0.544834}%
\pgfsetfillcolor{currentfill}%
\pgfsetfillopacity{0.700000}%
\pgfsetlinewidth{0.000000pt}%
\definecolor{currentstroke}{rgb}{0.000000,0.000000,0.000000}%
\pgfsetstrokecolor{currentstroke}%
\pgfsetdash{}{0pt}%
\pgfpathmoveto{\pgfqpoint{5.552637in}{2.795755in}}%
\pgfpathlineto{\pgfqpoint{5.566590in}{2.793371in}}%
\pgfpathlineto{\pgfqpoint{5.580551in}{2.791052in}}%
\pgfpathlineto{\pgfqpoint{5.594521in}{2.788799in}}%
\pgfpathlineto{\pgfqpoint{5.608500in}{2.786612in}}%
\pgfpathlineto{\pgfqpoint{5.615911in}{2.798740in}}%
\pgfpathlineto{\pgfqpoint{5.623328in}{2.811181in}}%
\pgfpathlineto{\pgfqpoint{5.630750in}{2.823942in}}%
\pgfpathlineto{\pgfqpoint{5.638177in}{2.837033in}}%
\pgfpathlineto{\pgfqpoint{5.624221in}{2.839702in}}%
\pgfpathlineto{\pgfqpoint{5.610273in}{2.842437in}}%
\pgfpathlineto{\pgfqpoint{5.596333in}{2.845237in}}%
\pgfpathlineto{\pgfqpoint{5.582403in}{2.848103in}}%
\pgfpathlineto{\pgfqpoint{5.574953in}{2.834523in}}%
\pgfpathlineto{\pgfqpoint{5.567509in}{2.821278in}}%
\pgfpathlineto{\pgfqpoint{5.560071in}{2.808358in}}%
\pgfpathlineto{\pgfqpoint{5.552637in}{2.795755in}}%
\pgfpathclose%
\pgfusepath{fill}%
\end{pgfscope}%
\begin{pgfscope}%
\pgfpathrectangle{\pgfqpoint{1.150000in}{0.150000in}}{\pgfqpoint{5.700000in}{5.700000in}}%
\pgfusepath{clip}%
\pgfsetbuttcap%
\pgfsetroundjoin%
\definecolor{currentfill}{rgb}{0.282327,0.094955,0.417331}%
\pgfsetfillcolor{currentfill}%
\pgfsetfillopacity{0.700000}%
\pgfsetlinewidth{0.000000pt}%
\definecolor{currentstroke}{rgb}{0.000000,0.000000,0.000000}%
\pgfsetstrokecolor{currentstroke}%
\pgfsetdash{}{0pt}%
\pgfpathmoveto{\pgfqpoint{2.952806in}{2.341253in}}%
\pgfpathlineto{\pgfqpoint{2.966226in}{2.333379in}}%
\pgfpathlineto{\pgfqpoint{2.979647in}{2.325616in}}%
\pgfpathlineto{\pgfqpoint{2.993070in}{2.317961in}}%
\pgfpathlineto{\pgfqpoint{3.006495in}{2.310416in}}%
\pgfpathlineto{\pgfqpoint{3.014714in}{2.319213in}}%
\pgfpathlineto{\pgfqpoint{3.022926in}{2.328068in}}%
\pgfpathlineto{\pgfqpoint{3.031131in}{2.336981in}}%
\pgfpathlineto{\pgfqpoint{3.039329in}{2.345954in}}%
\pgfpathlineto{\pgfqpoint{3.025917in}{2.353496in}}%
\pgfpathlineto{\pgfqpoint{3.012508in}{2.361146in}}%
\pgfpathlineto{\pgfqpoint{2.999100in}{2.368905in}}%
\pgfpathlineto{\pgfqpoint{2.985694in}{2.376775in}}%
\pgfpathlineto{\pgfqpoint{2.977483in}{2.367799in}}%
\pgfpathlineto{\pgfqpoint{2.969265in}{2.358887in}}%
\pgfpathlineto{\pgfqpoint{2.961039in}{2.350038in}}%
\pgfpathlineto{\pgfqpoint{2.952806in}{2.341253in}}%
\pgfpathclose%
\pgfusepath{fill}%
\end{pgfscope}%
\begin{pgfscope}%
\pgfpathrectangle{\pgfqpoint{1.150000in}{0.150000in}}{\pgfqpoint{5.700000in}{5.700000in}}%
\pgfusepath{clip}%
\pgfsetbuttcap%
\pgfsetroundjoin%
\definecolor{currentfill}{rgb}{0.195860,0.395433,0.555276}%
\pgfsetfillcolor{currentfill}%
\pgfsetfillopacity{0.700000}%
\pgfsetlinewidth{0.000000pt}%
\definecolor{currentstroke}{rgb}{0.000000,0.000000,0.000000}%
\pgfsetstrokecolor{currentstroke}%
\pgfsetdash{}{0pt}%
\pgfpathmoveto{\pgfqpoint{5.895276in}{2.977464in}}%
\pgfpathlineto{\pgfqpoint{5.909278in}{2.974074in}}%
\pgfpathlineto{\pgfqpoint{5.923288in}{2.970749in}}%
\pgfpathlineto{\pgfqpoint{5.937307in}{2.967488in}}%
\pgfpathlineto{\pgfqpoint{5.951335in}{2.964292in}}%
\pgfpathlineto{\pgfqpoint{5.958787in}{2.980299in}}%
\pgfpathlineto{\pgfqpoint{5.966252in}{2.996750in}}%
\pgfpathlineto{\pgfqpoint{5.973730in}{3.013654in}}%
\pgfpathlineto{\pgfqpoint{5.981223in}{3.031024in}}%
\pgfpathlineto{\pgfqpoint{5.967219in}{3.034782in}}%
\pgfpathlineto{\pgfqpoint{5.953225in}{3.038605in}}%
\pgfpathlineto{\pgfqpoint{5.939239in}{3.042493in}}%
\pgfpathlineto{\pgfqpoint{5.925261in}{3.046444in}}%
\pgfpathlineto{\pgfqpoint{5.917745in}{3.028505in}}%
\pgfpathlineto{\pgfqpoint{5.910242in}{3.011036in}}%
\pgfpathlineto{\pgfqpoint{5.902753in}{2.994026in}}%
\pgfpathlineto{\pgfqpoint{5.895276in}{2.977464in}}%
\pgfpathclose%
\pgfusepath{fill}%
\end{pgfscope}%
\begin{pgfscope}%
\pgfpathrectangle{\pgfqpoint{1.150000in}{0.150000in}}{\pgfqpoint{5.700000in}{5.700000in}}%
\pgfusepath{clip}%
\pgfsetbuttcap%
\pgfsetroundjoin%
\definecolor{currentfill}{rgb}{0.266580,0.228262,0.514349}%
\pgfsetfillcolor{currentfill}%
\pgfsetfillopacity{0.700000}%
\pgfsetlinewidth{0.000000pt}%
\definecolor{currentstroke}{rgb}{0.000000,0.000000,0.000000}%
\pgfsetstrokecolor{currentstroke}%
\pgfsetdash{}{0pt}%
\pgfpathmoveto{\pgfqpoint{4.984422in}{2.592308in}}%
\pgfpathlineto{\pgfqpoint{4.998246in}{2.590499in}}%
\pgfpathlineto{\pgfqpoint{5.012077in}{2.588761in}}%
\pgfpathlineto{\pgfqpoint{5.025918in}{2.587092in}}%
\pgfpathlineto{\pgfqpoint{5.039767in}{2.585492in}}%
\pgfpathlineto{\pgfqpoint{5.047291in}{2.594974in}}%
\pgfpathlineto{\pgfqpoint{5.054814in}{2.604622in}}%
\pgfpathlineto{\pgfqpoint{5.062335in}{2.614443in}}%
\pgfpathlineto{\pgfqpoint{5.069855in}{2.624443in}}%
\pgfpathlineto{\pgfqpoint{5.056023in}{2.626405in}}%
\pgfpathlineto{\pgfqpoint{5.042200in}{2.628436in}}%
\pgfpathlineto{\pgfqpoint{5.028385in}{2.630536in}}%
\pgfpathlineto{\pgfqpoint{5.014579in}{2.632706in}}%
\pgfpathlineto{\pgfqpoint{5.007042in}{2.622336in}}%
\pgfpathlineto{\pgfqpoint{4.999504in}{2.612151in}}%
\pgfpathlineto{\pgfqpoint{4.991964in}{2.602144in}}%
\pgfpathlineto{\pgfqpoint{4.984422in}{2.592308in}}%
\pgfpathclose%
\pgfusepath{fill}%
\end{pgfscope}%
\begin{pgfscope}%
\pgfpathrectangle{\pgfqpoint{1.150000in}{0.150000in}}{\pgfqpoint{5.700000in}{5.700000in}}%
\pgfusepath{clip}%
\pgfsetbuttcap%
\pgfsetroundjoin%
\definecolor{currentfill}{rgb}{0.239346,0.300855,0.540844}%
\pgfsetfillcolor{currentfill}%
\pgfsetfillopacity{0.700000}%
\pgfsetlinewidth{0.000000pt}%
\definecolor{currentstroke}{rgb}{0.000000,0.000000,0.000000}%
\pgfsetstrokecolor{currentstroke}%
\pgfsetdash{}{0pt}%
\pgfpathmoveto{\pgfqpoint{5.467141in}{2.756699in}}%
\pgfpathlineto{\pgfqpoint{5.481079in}{2.754512in}}%
\pgfpathlineto{\pgfqpoint{5.495026in}{2.752390in}}%
\pgfpathlineto{\pgfqpoint{5.508982in}{2.750336in}}%
\pgfpathlineto{\pgfqpoint{5.522947in}{2.748347in}}%
\pgfpathlineto{\pgfqpoint{5.530364in}{2.759766in}}%
\pgfpathlineto{\pgfqpoint{5.537784in}{2.771467in}}%
\pgfpathlineto{\pgfqpoint{5.545209in}{2.783461in}}%
\pgfpathlineto{\pgfqpoint{5.552637in}{2.795755in}}%
\pgfpathlineto{\pgfqpoint{5.538694in}{2.798206in}}%
\pgfpathlineto{\pgfqpoint{5.524759in}{2.800723in}}%
\pgfpathlineto{\pgfqpoint{5.510833in}{2.803305in}}%
\pgfpathlineto{\pgfqpoint{5.496916in}{2.805955in}}%
\pgfpathlineto{\pgfqpoint{5.489466in}{2.793192in}}%
\pgfpathlineto{\pgfqpoint{5.482021in}{2.780734in}}%
\pgfpathlineto{\pgfqpoint{5.474579in}{2.768572in}}%
\pgfpathlineto{\pgfqpoint{5.467141in}{2.756699in}}%
\pgfpathclose%
\pgfusepath{fill}%
\end{pgfscope}%
\begin{pgfscope}%
\pgfpathrectangle{\pgfqpoint{1.150000in}{0.150000in}}{\pgfqpoint{5.700000in}{5.700000in}}%
\pgfusepath{clip}%
\pgfsetbuttcap%
\pgfsetroundjoin%
\definecolor{currentfill}{rgb}{0.280267,0.073417,0.397163}%
\pgfsetfillcolor{currentfill}%
\pgfsetfillopacity{0.700000}%
\pgfsetlinewidth{0.000000pt}%
\definecolor{currentstroke}{rgb}{0.000000,0.000000,0.000000}%
\pgfsetstrokecolor{currentstroke}%
\pgfsetdash{}{0pt}%
\pgfpathmoveto{\pgfqpoint{3.372962in}{2.289702in}}%
\pgfpathlineto{\pgfqpoint{3.386412in}{2.284223in}}%
\pgfpathlineto{\pgfqpoint{3.399866in}{2.278839in}}%
\pgfpathlineto{\pgfqpoint{3.413324in}{2.273548in}}%
\pgfpathlineto{\pgfqpoint{3.426786in}{2.268351in}}%
\pgfpathlineto{\pgfqpoint{3.434854in}{2.277592in}}%
\pgfpathlineto{\pgfqpoint{3.442916in}{2.286872in}}%
\pgfpathlineto{\pgfqpoint{3.450973in}{2.296191in}}%
\pgfpathlineto{\pgfqpoint{3.459023in}{2.305551in}}%
\pgfpathlineto{\pgfqpoint{3.445572in}{2.310806in}}%
\pgfpathlineto{\pgfqpoint{3.432124in}{2.316154in}}%
\pgfpathlineto{\pgfqpoint{3.418681in}{2.321596in}}%
\pgfpathlineto{\pgfqpoint{3.405243in}{2.327132in}}%
\pgfpathlineto{\pgfqpoint{3.397182in}{2.317707in}}%
\pgfpathlineto{\pgfqpoint{3.389114in}{2.308327in}}%
\pgfpathlineto{\pgfqpoint{3.381041in}{2.298993in}}%
\pgfpathlineto{\pgfqpoint{3.372962in}{2.289702in}}%
\pgfpathclose%
\pgfusepath{fill}%
\end{pgfscope}%
\begin{pgfscope}%
\pgfpathrectangle{\pgfqpoint{1.150000in}{0.150000in}}{\pgfqpoint{5.700000in}{5.700000in}}%
\pgfusepath{clip}%
\pgfsetbuttcap%
\pgfsetroundjoin%
\definecolor{currentfill}{rgb}{0.282884,0.135920,0.453427}%
\pgfsetfillcolor{currentfill}%
\pgfsetfillopacity{0.700000}%
\pgfsetlinewidth{0.000000pt}%
\definecolor{currentstroke}{rgb}{0.000000,0.000000,0.000000}%
\pgfsetstrokecolor{currentstroke}%
\pgfsetdash{}{0pt}%
\pgfpathmoveto{\pgfqpoint{4.275962in}{2.404701in}}%
\pgfpathlineto{\pgfqpoint{4.289599in}{2.402240in}}%
\pgfpathlineto{\pgfqpoint{4.303243in}{2.399856in}}%
\pgfpathlineto{\pgfqpoint{4.316894in}{2.397549in}}%
\pgfpathlineto{\pgfqpoint{4.330553in}{2.395317in}}%
\pgfpathlineto{\pgfqpoint{4.338314in}{2.404352in}}%
\pgfpathlineto{\pgfqpoint{4.346069in}{2.413452in}}%
\pgfpathlineto{\pgfqpoint{4.353820in}{2.422620in}}%
\pgfpathlineto{\pgfqpoint{4.361565in}{2.431861in}}%
\pgfpathlineto{\pgfqpoint{4.347919in}{2.434313in}}%
\pgfpathlineto{\pgfqpoint{4.334279in}{2.436841in}}%
\pgfpathlineto{\pgfqpoint{4.320646in}{2.439445in}}%
\pgfpathlineto{\pgfqpoint{4.307021in}{2.442125in}}%
\pgfpathlineto{\pgfqpoint{4.299264in}{2.432657in}}%
\pgfpathlineto{\pgfqpoint{4.291501in}{2.423266in}}%
\pgfpathlineto{\pgfqpoint{4.283734in}{2.413948in}}%
\pgfpathlineto{\pgfqpoint{4.275962in}{2.404701in}}%
\pgfpathclose%
\pgfusepath{fill}%
\end{pgfscope}%
\begin{pgfscope}%
\pgfpathrectangle{\pgfqpoint{1.150000in}{0.150000in}}{\pgfqpoint{5.700000in}{5.700000in}}%
\pgfusepath{clip}%
\pgfsetbuttcap%
\pgfsetroundjoin%
\definecolor{currentfill}{rgb}{0.279574,0.170599,0.479997}%
\pgfsetfillcolor{currentfill}%
\pgfsetfillopacity{0.700000}%
\pgfsetlinewidth{0.000000pt}%
\definecolor{currentstroke}{rgb}{0.000000,0.000000,0.000000}%
\pgfsetstrokecolor{currentstroke}%
\pgfsetdash{}{0pt}%
\pgfpathmoveto{\pgfqpoint{4.587408in}{2.479338in}}%
\pgfpathlineto{\pgfqpoint{4.601127in}{2.477351in}}%
\pgfpathlineto{\pgfqpoint{4.614855in}{2.475437in}}%
\pgfpathlineto{\pgfqpoint{4.628590in}{2.473596in}}%
\pgfpathlineto{\pgfqpoint{4.642334in}{2.471828in}}%
\pgfpathlineto{\pgfqpoint{4.649987in}{2.480832in}}%
\pgfpathlineto{\pgfqpoint{4.657636in}{2.489935in}}%
\pgfpathlineto{\pgfqpoint{4.665281in}{2.499143in}}%
\pgfpathlineto{\pgfqpoint{4.672922in}{2.508460in}}%
\pgfpathlineto{\pgfqpoint{4.659193in}{2.510510in}}%
\pgfpathlineto{\pgfqpoint{4.645471in}{2.512632in}}%
\pgfpathlineto{\pgfqpoint{4.631757in}{2.514827in}}%
\pgfpathlineto{\pgfqpoint{4.618051in}{2.517094in}}%
\pgfpathlineto{\pgfqpoint{4.610396in}{2.507489in}}%
\pgfpathlineto{\pgfqpoint{4.602737in}{2.497998in}}%
\pgfpathlineto{\pgfqpoint{4.595075in}{2.488616in}}%
\pgfpathlineto{\pgfqpoint{4.587408in}{2.479338in}}%
\pgfpathclose%
\pgfusepath{fill}%
\end{pgfscope}%
\begin{pgfscope}%
\pgfpathrectangle{\pgfqpoint{1.150000in}{0.150000in}}{\pgfqpoint{5.700000in}{5.700000in}}%
\pgfusepath{clip}%
\pgfsetbuttcap%
\pgfsetroundjoin%
\definecolor{currentfill}{rgb}{0.244972,0.287675,0.537260}%
\pgfsetfillcolor{currentfill}%
\pgfsetfillopacity{0.700000}%
\pgfsetlinewidth{0.000000pt}%
\definecolor{currentstroke}{rgb}{0.000000,0.000000,0.000000}%
\pgfsetstrokecolor{currentstroke}%
\pgfsetdash{}{0pt}%
\pgfpathmoveto{\pgfqpoint{5.381670in}{2.719576in}}%
\pgfpathlineto{\pgfqpoint{5.395593in}{2.717564in}}%
\pgfpathlineto{\pgfqpoint{5.409525in}{2.715619in}}%
\pgfpathlineto{\pgfqpoint{5.423466in}{2.713740in}}%
\pgfpathlineto{\pgfqpoint{5.437416in}{2.711929in}}%
\pgfpathlineto{\pgfqpoint{5.444844in}{2.722729in}}%
\pgfpathlineto{\pgfqpoint{5.452274in}{2.733785in}}%
\pgfpathlineto{\pgfqpoint{5.459706in}{2.745106in}}%
\pgfpathlineto{\pgfqpoint{5.467141in}{2.756699in}}%
\pgfpathlineto{\pgfqpoint{5.453211in}{2.758953in}}%
\pgfpathlineto{\pgfqpoint{5.439291in}{2.761274in}}%
\pgfpathlineto{\pgfqpoint{5.425379in}{2.763661in}}%
\pgfpathlineto{\pgfqpoint{5.411476in}{2.766115in}}%
\pgfpathlineto{\pgfqpoint{5.404021in}{2.754073in}}%
\pgfpathlineto{\pgfqpoint{5.396568in}{2.742308in}}%
\pgfpathlineto{\pgfqpoint{5.389118in}{2.730811in}}%
\pgfpathlineto{\pgfqpoint{5.381670in}{2.719576in}}%
\pgfpathclose%
\pgfusepath{fill}%
\end{pgfscope}%
\begin{pgfscope}%
\pgfpathrectangle{\pgfqpoint{1.150000in}{0.150000in}}{\pgfqpoint{5.700000in}{5.700000in}}%
\pgfusepath{clip}%
\pgfsetbuttcap%
\pgfsetroundjoin%
\definecolor{currentfill}{rgb}{0.281887,0.150881,0.465405}%
\pgfsetfillcolor{currentfill}%
\pgfsetfillopacity{0.700000}%
\pgfsetlinewidth{0.000000pt}%
\definecolor{currentstroke}{rgb}{0.000000,0.000000,0.000000}%
\pgfsetstrokecolor{currentstroke}%
\pgfsetdash{}{0pt}%
\pgfpathmoveto{\pgfqpoint{2.617852in}{2.456536in}}%
\pgfpathlineto{\pgfqpoint{2.631294in}{2.446203in}}%
\pgfpathlineto{\pgfqpoint{2.644736in}{2.436000in}}%
\pgfpathlineto{\pgfqpoint{2.658177in}{2.425924in}}%
\pgfpathlineto{\pgfqpoint{2.671617in}{2.415975in}}%
\pgfpathlineto{\pgfqpoint{2.679973in}{2.424048in}}%
\pgfpathlineto{\pgfqpoint{2.688320in}{2.432205in}}%
\pgfpathlineto{\pgfqpoint{2.696659in}{2.440445in}}%
\pgfpathlineto{\pgfqpoint{2.704989in}{2.448768in}}%
\pgfpathlineto{\pgfqpoint{2.691566in}{2.458671in}}%
\pgfpathlineto{\pgfqpoint{2.678141in}{2.468701in}}%
\pgfpathlineto{\pgfqpoint{2.664716in}{2.478858in}}%
\pgfpathlineto{\pgfqpoint{2.651291in}{2.489145in}}%
\pgfpathlineto{\pgfqpoint{2.642944in}{2.480861in}}%
\pgfpathlineto{\pgfqpoint{2.634589in}{2.472665in}}%
\pgfpathlineto{\pgfqpoint{2.626225in}{2.464556in}}%
\pgfpathlineto{\pgfqpoint{2.617852in}{2.456536in}}%
\pgfpathclose%
\pgfusepath{fill}%
\end{pgfscope}%
\begin{pgfscope}%
\pgfpathrectangle{\pgfqpoint{1.150000in}{0.150000in}}{\pgfqpoint{5.700000in}{5.700000in}}%
\pgfusepath{clip}%
\pgfsetbuttcap%
\pgfsetroundjoin%
\definecolor{currentfill}{rgb}{0.281446,0.084320,0.407414}%
\pgfsetfillcolor{currentfill}%
\pgfsetfillopacity{0.700000}%
\pgfsetlinewidth{0.000000pt}%
\definecolor{currentstroke}{rgb}{0.000000,0.000000,0.000000}%
\pgfsetstrokecolor{currentstroke}%
\pgfsetdash{}{0pt}%
\pgfpathmoveto{\pgfqpoint{3.738722in}{2.307493in}}%
\pgfpathlineto{\pgfqpoint{3.752235in}{2.303572in}}%
\pgfpathlineto{\pgfqpoint{3.765753in}{2.299737in}}%
\pgfpathlineto{\pgfqpoint{3.779277in}{2.295987in}}%
\pgfpathlineto{\pgfqpoint{3.792807in}{2.292321in}}%
\pgfpathlineto{\pgfqpoint{3.800752in}{2.301578in}}%
\pgfpathlineto{\pgfqpoint{3.808691in}{2.310871in}}%
\pgfpathlineto{\pgfqpoint{3.816625in}{2.320202in}}%
\pgfpathlineto{\pgfqpoint{3.824553in}{2.329573in}}%
\pgfpathlineto{\pgfqpoint{3.811033in}{2.333358in}}%
\pgfpathlineto{\pgfqpoint{3.797520in}{2.337227in}}%
\pgfpathlineto{\pgfqpoint{3.784012in}{2.341181in}}%
\pgfpathlineto{\pgfqpoint{3.770509in}{2.345219in}}%
\pgfpathlineto{\pgfqpoint{3.762571in}{2.335722in}}%
\pgfpathlineto{\pgfqpoint{3.754627in}{2.326270in}}%
\pgfpathlineto{\pgfqpoint{3.746677in}{2.316861in}}%
\pgfpathlineto{\pgfqpoint{3.738722in}{2.307493in}}%
\pgfpathclose%
\pgfusepath{fill}%
\end{pgfscope}%
\begin{pgfscope}%
\pgfpathrectangle{\pgfqpoint{1.150000in}{0.150000in}}{\pgfqpoint{5.700000in}{5.700000in}}%
\pgfusepath{clip}%
\pgfsetbuttcap%
\pgfsetroundjoin%
\definecolor{currentfill}{rgb}{0.283197,0.115680,0.436115}%
\pgfsetfillcolor{currentfill}%
\pgfsetfillopacity{0.700000}%
\pgfsetlinewidth{0.000000pt}%
\definecolor{currentstroke}{rgb}{0.000000,0.000000,0.000000}%
\pgfsetstrokecolor{currentstroke}%
\pgfsetdash{}{0pt}%
\pgfpathmoveto{\pgfqpoint{2.812377in}{2.373981in}}%
\pgfpathlineto{\pgfqpoint{2.825802in}{2.365173in}}%
\pgfpathlineto{\pgfqpoint{2.839228in}{2.356481in}}%
\pgfpathlineto{\pgfqpoint{2.852654in}{2.347906in}}%
\pgfpathlineto{\pgfqpoint{2.866082in}{2.339446in}}%
\pgfpathlineto{\pgfqpoint{2.874358in}{2.347949in}}%
\pgfpathlineto{\pgfqpoint{2.882627in}{2.356519in}}%
\pgfpathlineto{\pgfqpoint{2.890889in}{2.365158in}}%
\pgfpathlineto{\pgfqpoint{2.899143in}{2.373863in}}%
\pgfpathlineto{\pgfqpoint{2.885730in}{2.382299in}}%
\pgfpathlineto{\pgfqpoint{2.872318in}{2.390850in}}%
\pgfpathlineto{\pgfqpoint{2.858907in}{2.399517in}}%
\pgfpathlineto{\pgfqpoint{2.845497in}{2.408300in}}%
\pgfpathlineto{\pgfqpoint{2.837229in}{2.399611in}}%
\pgfpathlineto{\pgfqpoint{2.828953in}{2.390996in}}%
\pgfpathlineto{\pgfqpoint{2.820669in}{2.382452in}}%
\pgfpathlineto{\pgfqpoint{2.812377in}{2.373981in}}%
\pgfpathclose%
\pgfusepath{fill}%
\end{pgfscope}%
\begin{pgfscope}%
\pgfpathrectangle{\pgfqpoint{1.150000in}{0.150000in}}{\pgfqpoint{5.700000in}{5.700000in}}%
\pgfusepath{clip}%
\pgfsetbuttcap%
\pgfsetroundjoin%
\definecolor{currentfill}{rgb}{0.187231,0.414746,0.556547}%
\pgfsetfillcolor{currentfill}%
\pgfsetfillopacity{0.700000}%
\pgfsetlinewidth{0.000000pt}%
\definecolor{currentstroke}{rgb}{0.000000,0.000000,0.000000}%
\pgfsetstrokecolor{currentstroke}%
\pgfsetdash{}{0pt}%
\pgfpathmoveto{\pgfqpoint{5.981223in}{3.031024in}}%
\pgfpathlineto{\pgfqpoint{5.995235in}{3.027329in}}%
\pgfpathlineto{\pgfqpoint{6.009255in}{3.023698in}}%
\pgfpathlineto{\pgfqpoint{6.023284in}{3.020131in}}%
\pgfpathlineto{\pgfqpoint{6.037322in}{3.016628in}}%
\pgfpathlineto{\pgfqpoint{6.044804in}{3.033897in}}%
\pgfpathlineto{\pgfqpoint{6.052302in}{3.051648in}}%
\pgfpathlineto{\pgfqpoint{6.059815in}{3.069889in}}%
\pgfpathlineto{\pgfqpoint{6.045795in}{3.073827in}}%
\pgfpathlineto{\pgfqpoint{6.031785in}{3.077829in}}%
\pgfpathlineto{\pgfqpoint{6.017783in}{3.081895in}}%
\pgfpathlineto{\pgfqpoint{6.003789in}{3.086024in}}%
\pgfpathlineto{\pgfqpoint{5.996251in}{3.067198in}}%
\pgfpathlineto{\pgfqpoint{5.988729in}{3.048868in}}%
\pgfpathlineto{\pgfqpoint{5.981223in}{3.031024in}}%
\pgfpathclose%
\pgfusepath{fill}%
\end{pgfscope}%
\begin{pgfscope}%
\pgfpathrectangle{\pgfqpoint{1.150000in}{0.150000in}}{\pgfqpoint{5.700000in}{5.700000in}}%
\pgfusepath{clip}%
\pgfsetbuttcap%
\pgfsetroundjoin%
\definecolor{currentfill}{rgb}{0.282910,0.105393,0.426902}%
\pgfsetfillcolor{currentfill}%
\pgfsetfillopacity{0.700000}%
\pgfsetlinewidth{0.000000pt}%
\definecolor{currentstroke}{rgb}{0.000000,0.000000,0.000000}%
\pgfsetstrokecolor{currentstroke}%
\pgfsetdash{}{0pt}%
\pgfpathmoveto{\pgfqpoint{3.964491in}{2.339117in}}%
\pgfpathlineto{\pgfqpoint{3.978054in}{2.335937in}}%
\pgfpathlineto{\pgfqpoint{3.991623in}{2.332837in}}%
\pgfpathlineto{\pgfqpoint{4.005198in}{2.329818in}}%
\pgfpathlineto{\pgfqpoint{4.018780in}{2.326879in}}%
\pgfpathlineto{\pgfqpoint{4.026649in}{2.336037in}}%
\pgfpathlineto{\pgfqpoint{4.034512in}{2.345238in}}%
\pgfpathlineto{\pgfqpoint{4.042370in}{2.354484in}}%
\pgfpathlineto{\pgfqpoint{4.050223in}{2.363778in}}%
\pgfpathlineto{\pgfqpoint{4.036652in}{2.366877in}}%
\pgfpathlineto{\pgfqpoint{4.023087in}{2.370055in}}%
\pgfpathlineto{\pgfqpoint{4.009529in}{2.373314in}}%
\pgfpathlineto{\pgfqpoint{3.995977in}{2.376654in}}%
\pgfpathlineto{\pgfqpoint{3.988113in}{2.367193in}}%
\pgfpathlineto{\pgfqpoint{3.980245in}{2.357786in}}%
\pgfpathlineto{\pgfqpoint{3.972371in}{2.348428in}}%
\pgfpathlineto{\pgfqpoint{3.964491in}{2.339117in}}%
\pgfpathclose%
\pgfusepath{fill}%
\end{pgfscope}%
\begin{pgfscope}%
\pgfpathrectangle{\pgfqpoint{1.150000in}{0.150000in}}{\pgfqpoint{5.700000in}{5.700000in}}%
\pgfusepath{clip}%
\pgfsetbuttcap%
\pgfsetroundjoin%
\definecolor{currentfill}{rgb}{0.270595,0.214069,0.507052}%
\pgfsetfillcolor{currentfill}%
\pgfsetfillopacity{0.700000}%
\pgfsetlinewidth{0.000000pt}%
\definecolor{currentstroke}{rgb}{0.000000,0.000000,0.000000}%
\pgfsetstrokecolor{currentstroke}%
\pgfsetdash{}{0pt}%
\pgfpathmoveto{\pgfqpoint{4.898960in}{2.561112in}}%
\pgfpathlineto{\pgfqpoint{4.912766in}{2.559366in}}%
\pgfpathlineto{\pgfqpoint{4.926580in}{2.557691in}}%
\pgfpathlineto{\pgfqpoint{4.940403in}{2.556085in}}%
\pgfpathlineto{\pgfqpoint{4.954234in}{2.554549in}}%
\pgfpathlineto{\pgfqpoint{4.961785in}{2.563763in}}%
\pgfpathlineto{\pgfqpoint{4.969333in}{2.573124in}}%
\pgfpathlineto{\pgfqpoint{4.976879in}{2.582636in}}%
\pgfpathlineto{\pgfqpoint{4.984422in}{2.592308in}}%
\pgfpathlineto{\pgfqpoint{4.970608in}{2.594185in}}%
\pgfpathlineto{\pgfqpoint{4.956801in}{2.596133in}}%
\pgfpathlineto{\pgfqpoint{4.943003in}{2.598150in}}%
\pgfpathlineto{\pgfqpoint{4.929213in}{2.600238in}}%
\pgfpathlineto{\pgfqpoint{4.921653in}{2.590217in}}%
\pgfpathlineto{\pgfqpoint{4.914091in}{2.580361in}}%
\pgfpathlineto{\pgfqpoint{4.906527in}{2.570661in}}%
\pgfpathlineto{\pgfqpoint{4.898960in}{2.561112in}}%
\pgfpathclose%
\pgfusepath{fill}%
\end{pgfscope}%
\begin{pgfscope}%
\pgfpathrectangle{\pgfqpoint{1.150000in}{0.150000in}}{\pgfqpoint{5.700000in}{5.700000in}}%
\pgfusepath{clip}%
\pgfsetbuttcap%
\pgfsetroundjoin%
\definecolor{currentfill}{rgb}{0.280267,0.073417,0.397163}%
\pgfsetfillcolor{currentfill}%
\pgfsetfillopacity{0.700000}%
\pgfsetlinewidth{0.000000pt}%
\definecolor{currentstroke}{rgb}{0.000000,0.000000,0.000000}%
\pgfsetstrokecolor{currentstroke}%
\pgfsetdash{}{0pt}%
\pgfpathmoveto{\pgfqpoint{3.512871in}{2.285452in}}%
\pgfpathlineto{\pgfqpoint{3.526344in}{2.280655in}}%
\pgfpathlineto{\pgfqpoint{3.539822in}{2.275948in}}%
\pgfpathlineto{\pgfqpoint{3.553305in}{2.271331in}}%
\pgfpathlineto{\pgfqpoint{3.566792in}{2.266803in}}%
\pgfpathlineto{\pgfqpoint{3.574815in}{2.276065in}}%
\pgfpathlineto{\pgfqpoint{3.582831in}{2.285362in}}%
\pgfpathlineto{\pgfqpoint{3.590842in}{2.294694in}}%
\pgfpathlineto{\pgfqpoint{3.598847in}{2.304065in}}%
\pgfpathlineto{\pgfqpoint{3.585371in}{2.308671in}}%
\pgfpathlineto{\pgfqpoint{3.571899in}{2.313366in}}%
\pgfpathlineto{\pgfqpoint{3.558431in}{2.318150in}}%
\pgfpathlineto{\pgfqpoint{3.544969in}{2.323025in}}%
\pgfpathlineto{\pgfqpoint{3.536953in}{2.313569in}}%
\pgfpathlineto{\pgfqpoint{3.528932in}{2.304156in}}%
\pgfpathlineto{\pgfqpoint{3.520904in}{2.294784in}}%
\pgfpathlineto{\pgfqpoint{3.512871in}{2.285452in}}%
\pgfpathclose%
\pgfusepath{fill}%
\end{pgfscope}%
\begin{pgfscope}%
\pgfpathrectangle{\pgfqpoint{1.150000in}{0.150000in}}{\pgfqpoint{5.700000in}{5.700000in}}%
\pgfusepath{clip}%
\pgfsetbuttcap%
\pgfsetroundjoin%
\definecolor{currentfill}{rgb}{0.252194,0.269783,0.531579}%
\pgfsetfillcolor{currentfill}%
\pgfsetfillopacity{0.700000}%
\pgfsetlinewidth{0.000000pt}%
\definecolor{currentstroke}{rgb}{0.000000,0.000000,0.000000}%
\pgfsetstrokecolor{currentstroke}%
\pgfsetdash{}{0pt}%
\pgfpathmoveto{\pgfqpoint{5.296209in}{2.684122in}}%
\pgfpathlineto{\pgfqpoint{5.310117in}{2.682264in}}%
\pgfpathlineto{\pgfqpoint{5.324034in}{2.680472in}}%
\pgfpathlineto{\pgfqpoint{5.337959in}{2.678748in}}%
\pgfpathlineto{\pgfqpoint{5.351894in}{2.677092in}}%
\pgfpathlineto{\pgfqpoint{5.359336in}{2.687360in}}%
\pgfpathlineto{\pgfqpoint{5.366779in}{2.697858in}}%
\pgfpathlineto{\pgfqpoint{5.374224in}{2.708594in}}%
\pgfpathlineto{\pgfqpoint{5.381670in}{2.719576in}}%
\pgfpathlineto{\pgfqpoint{5.367755in}{2.721655in}}%
\pgfpathlineto{\pgfqpoint{5.353850in}{2.723802in}}%
\pgfpathlineto{\pgfqpoint{5.339953in}{2.726015in}}%
\pgfpathlineto{\pgfqpoint{5.326064in}{2.728296in}}%
\pgfpathlineto{\pgfqpoint{5.318599in}{2.716884in}}%
\pgfpathlineto{\pgfqpoint{5.311134in}{2.705723in}}%
\pgfpathlineto{\pgfqpoint{5.303671in}{2.694805in}}%
\pgfpathlineto{\pgfqpoint{5.296209in}{2.684122in}}%
\pgfpathclose%
\pgfusepath{fill}%
\end{pgfscope}%
\begin{pgfscope}%
\pgfpathrectangle{\pgfqpoint{1.150000in}{0.150000in}}{\pgfqpoint{5.700000in}{5.700000in}}%
\pgfusepath{clip}%
\pgfsetbuttcap%
\pgfsetroundjoin%
\definecolor{currentfill}{rgb}{0.280868,0.160771,0.472899}%
\pgfsetfillcolor{currentfill}%
\pgfsetfillopacity{0.700000}%
\pgfsetlinewidth{0.000000pt}%
\definecolor{currentstroke}{rgb}{0.000000,0.000000,0.000000}%
\pgfsetstrokecolor{currentstroke}%
\pgfsetdash{}{0pt}%
\pgfpathmoveto{\pgfqpoint{4.501843in}{2.450804in}}%
\pgfpathlineto{\pgfqpoint{4.515545in}{2.448785in}}%
\pgfpathlineto{\pgfqpoint{4.529255in}{2.446839in}}%
\pgfpathlineto{\pgfqpoint{4.542972in}{2.444967in}}%
\pgfpathlineto{\pgfqpoint{4.556697in}{2.443168in}}%
\pgfpathlineto{\pgfqpoint{4.564381in}{2.452079in}}%
\pgfpathlineto{\pgfqpoint{4.572061in}{2.461074in}}%
\pgfpathlineto{\pgfqpoint{4.579737in}{2.470159in}}%
\pgfpathlineto{\pgfqpoint{4.587408in}{2.479338in}}%
\pgfpathlineto{\pgfqpoint{4.573696in}{2.481397in}}%
\pgfpathlineto{\pgfqpoint{4.559992in}{2.483530in}}%
\pgfpathlineto{\pgfqpoint{4.546295in}{2.485737in}}%
\pgfpathlineto{\pgfqpoint{4.532607in}{2.488017in}}%
\pgfpathlineto{\pgfqpoint{4.524922in}{2.478570in}}%
\pgfpathlineto{\pgfqpoint{4.517234in}{2.469222in}}%
\pgfpathlineto{\pgfqpoint{4.509541in}{2.459968in}}%
\pgfpathlineto{\pgfqpoint{4.501843in}{2.450804in}}%
\pgfpathclose%
\pgfusepath{fill}%
\end{pgfscope}%
\begin{pgfscope}%
\pgfpathrectangle{\pgfqpoint{1.150000in}{0.150000in}}{\pgfqpoint{5.700000in}{5.700000in}}%
\pgfusepath{clip}%
\pgfsetbuttcap%
\pgfsetroundjoin%
\definecolor{currentfill}{rgb}{0.283187,0.125848,0.444960}%
\pgfsetfillcolor{currentfill}%
\pgfsetfillopacity{0.700000}%
\pgfsetlinewidth{0.000000pt}%
\definecolor{currentstroke}{rgb}{0.000000,0.000000,0.000000}%
\pgfsetstrokecolor{currentstroke}%
\pgfsetdash{}{0pt}%
\pgfpathmoveto{\pgfqpoint{4.190298in}{2.378135in}}%
\pgfpathlineto{\pgfqpoint{4.203918in}{2.375567in}}%
\pgfpathlineto{\pgfqpoint{4.217546in}{2.373076in}}%
\pgfpathlineto{\pgfqpoint{4.231180in}{2.370662in}}%
\pgfpathlineto{\pgfqpoint{4.244822in}{2.368325in}}%
\pgfpathlineto{\pgfqpoint{4.252614in}{2.377334in}}%
\pgfpathlineto{\pgfqpoint{4.260402in}{2.386397in}}%
\pgfpathlineto{\pgfqpoint{4.268184in}{2.395518in}}%
\pgfpathlineto{\pgfqpoint{4.275962in}{2.404701in}}%
\pgfpathlineto{\pgfqpoint{4.262332in}{2.407238in}}%
\pgfpathlineto{\pgfqpoint{4.248709in}{2.409851in}}%
\pgfpathlineto{\pgfqpoint{4.235093in}{2.412542in}}%
\pgfpathlineto{\pgfqpoint{4.221484in}{2.415311in}}%
\pgfpathlineto{\pgfqpoint{4.213695in}{2.405921in}}%
\pgfpathlineto{\pgfqpoint{4.205901in}{2.396598in}}%
\pgfpathlineto{\pgfqpoint{4.198102in}{2.387337in}}%
\pgfpathlineto{\pgfqpoint{4.190298in}{2.378135in}}%
\pgfpathclose%
\pgfusepath{fill}%
\end{pgfscope}%
\begin{pgfscope}%
\pgfpathrectangle{\pgfqpoint{1.150000in}{0.150000in}}{\pgfqpoint{5.700000in}{5.700000in}}%
\pgfusepath{clip}%
\pgfsetbuttcap%
\pgfsetroundjoin%
\definecolor{currentfill}{rgb}{0.280267,0.073417,0.397163}%
\pgfsetfillcolor{currentfill}%
\pgfsetfillopacity{0.700000}%
\pgfsetlinewidth{0.000000pt}%
\definecolor{currentstroke}{rgb}{0.000000,0.000000,0.000000}%
\pgfsetstrokecolor{currentstroke}%
\pgfsetdash{}{0pt}%
\pgfpathmoveto{\pgfqpoint{3.146700in}{2.289430in}}%
\pgfpathlineto{\pgfqpoint{3.160133in}{2.282831in}}%
\pgfpathlineto{\pgfqpoint{3.173569in}{2.276333in}}%
\pgfpathlineto{\pgfqpoint{3.187008in}{2.269935in}}%
\pgfpathlineto{\pgfqpoint{3.200449in}{2.263638in}}%
\pgfpathlineto{\pgfqpoint{3.208603in}{2.272643in}}%
\pgfpathlineto{\pgfqpoint{3.216749in}{2.281693in}}%
\pgfpathlineto{\pgfqpoint{3.224890in}{2.290788in}}%
\pgfpathlineto{\pgfqpoint{3.233024in}{2.299931in}}%
\pgfpathlineto{\pgfqpoint{3.219594in}{2.306245in}}%
\pgfpathlineto{\pgfqpoint{3.206167in}{2.312659in}}%
\pgfpathlineto{\pgfqpoint{3.192744in}{2.319174in}}%
\pgfpathlineto{\pgfqpoint{3.179323in}{2.325790in}}%
\pgfpathlineto{\pgfqpoint{3.171177in}{2.316624in}}%
\pgfpathlineto{\pgfqpoint{3.163025in}{2.307509in}}%
\pgfpathlineto{\pgfqpoint{3.154866in}{2.298444in}}%
\pgfpathlineto{\pgfqpoint{3.146700in}{2.289430in}}%
\pgfpathclose%
\pgfusepath{fill}%
\end{pgfscope}%
\begin{pgfscope}%
\pgfpathrectangle{\pgfqpoint{1.150000in}{0.150000in}}{\pgfqpoint{5.700000in}{5.700000in}}%
\pgfusepath{clip}%
\pgfsetbuttcap%
\pgfsetroundjoin%
\definecolor{currentfill}{rgb}{0.274128,0.199721,0.498911}%
\pgfsetfillcolor{currentfill}%
\pgfsetfillopacity{0.700000}%
\pgfsetlinewidth{0.000000pt}%
\definecolor{currentstroke}{rgb}{0.000000,0.000000,0.000000}%
\pgfsetstrokecolor{currentstroke}%
\pgfsetdash{}{0pt}%
\pgfpathmoveto{\pgfqpoint{4.813461in}{2.530712in}}%
\pgfpathlineto{\pgfqpoint{4.827249in}{2.529005in}}%
\pgfpathlineto{\pgfqpoint{4.841046in}{2.527370in}}%
\pgfpathlineto{\pgfqpoint{4.854850in}{2.525805in}}%
\pgfpathlineto{\pgfqpoint{4.868664in}{2.524311in}}%
\pgfpathlineto{\pgfqpoint{4.876242in}{2.533314in}}%
\pgfpathlineto{\pgfqpoint{4.883818in}{2.542445in}}%
\pgfpathlineto{\pgfqpoint{4.891390in}{2.551709in}}%
\pgfpathlineto{\pgfqpoint{4.898960in}{2.561112in}}%
\pgfpathlineto{\pgfqpoint{4.885163in}{2.562929in}}%
\pgfpathlineto{\pgfqpoint{4.871374in}{2.564815in}}%
\pgfpathlineto{\pgfqpoint{4.857593in}{2.566772in}}%
\pgfpathlineto{\pgfqpoint{4.843820in}{2.568800in}}%
\pgfpathlineto{\pgfqpoint{4.836235in}{2.559068in}}%
\pgfpathlineto{\pgfqpoint{4.828647in}{2.549480in}}%
\pgfpathlineto{\pgfqpoint{4.821056in}{2.540029in}}%
\pgfpathlineto{\pgfqpoint{4.813461in}{2.530712in}}%
\pgfpathclose%
\pgfusepath{fill}%
\end{pgfscope}%
\begin{pgfscope}%
\pgfpathrectangle{\pgfqpoint{1.150000in}{0.150000in}}{\pgfqpoint{5.700000in}{5.700000in}}%
\pgfusepath{clip}%
\pgfsetbuttcap%
\pgfsetroundjoin%
\definecolor{currentfill}{rgb}{0.257322,0.256130,0.526563}%
\pgfsetfillcolor{currentfill}%
\pgfsetfillopacity{0.700000}%
\pgfsetlinewidth{0.000000pt}%
\definecolor{currentstroke}{rgb}{0.000000,0.000000,0.000000}%
\pgfsetstrokecolor{currentstroke}%
\pgfsetdash{}{0pt}%
\pgfpathmoveto{\pgfqpoint{5.210746in}{2.650097in}}%
\pgfpathlineto{\pgfqpoint{5.224637in}{2.648370in}}%
\pgfpathlineto{\pgfqpoint{5.238538in}{2.646711in}}%
\pgfpathlineto{\pgfqpoint{5.252447in}{2.645119in}}%
\pgfpathlineto{\pgfqpoint{5.266366in}{2.643596in}}%
\pgfpathlineto{\pgfqpoint{5.273826in}{2.653411in}}%
\pgfpathlineto{\pgfqpoint{5.281287in}{2.663433in}}%
\pgfpathlineto{\pgfqpoint{5.288748in}{2.673667in}}%
\pgfpathlineto{\pgfqpoint{5.296209in}{2.684122in}}%
\pgfpathlineto{\pgfqpoint{5.282310in}{2.686049in}}%
\pgfpathlineto{\pgfqpoint{5.268420in}{2.688042in}}%
\pgfpathlineto{\pgfqpoint{5.254539in}{2.690104in}}%
\pgfpathlineto{\pgfqpoint{5.240666in}{2.692233in}}%
\pgfpathlineto{\pgfqpoint{5.233185in}{2.681368in}}%
\pgfpathlineto{\pgfqpoint{5.225705in}{2.670729in}}%
\pgfpathlineto{\pgfqpoint{5.218226in}{2.660308in}}%
\pgfpathlineto{\pgfqpoint{5.210746in}{2.650097in}}%
\pgfpathclose%
\pgfusepath{fill}%
\end{pgfscope}%
\begin{pgfscope}%
\pgfpathrectangle{\pgfqpoint{1.150000in}{0.150000in}}{\pgfqpoint{5.700000in}{5.700000in}}%
\pgfusepath{clip}%
\pgfsetbuttcap%
\pgfsetroundjoin%
\definecolor{currentfill}{rgb}{0.281446,0.084320,0.407414}%
\pgfsetfillcolor{currentfill}%
\pgfsetfillopacity{0.700000}%
\pgfsetlinewidth{0.000000pt}%
\definecolor{currentstroke}{rgb}{0.000000,0.000000,0.000000}%
\pgfsetstrokecolor{currentstroke}%
\pgfsetdash{}{0pt}%
\pgfpathmoveto{\pgfqpoint{3.006495in}{2.310416in}}%
\pgfpathlineto{\pgfqpoint{3.019921in}{2.302977in}}%
\pgfpathlineto{\pgfqpoint{3.033350in}{2.295646in}}%
\pgfpathlineto{\pgfqpoint{3.046781in}{2.288421in}}%
\pgfpathlineto{\pgfqpoint{3.060214in}{2.281301in}}%
\pgfpathlineto{\pgfqpoint{3.068419in}{2.290109in}}%
\pgfpathlineto{\pgfqpoint{3.076618in}{2.298971in}}%
\pgfpathlineto{\pgfqpoint{3.084810in}{2.307886in}}%
\pgfpathlineto{\pgfqpoint{3.092995in}{2.316854in}}%
\pgfpathlineto{\pgfqpoint{3.079575in}{2.323970in}}%
\pgfpathlineto{\pgfqpoint{3.066157in}{2.331192in}}%
\pgfpathlineto{\pgfqpoint{3.052742in}{2.338519in}}%
\pgfpathlineto{\pgfqpoint{3.039329in}{2.345954in}}%
\pgfpathlineto{\pgfqpoint{3.031131in}{2.336981in}}%
\pgfpathlineto{\pgfqpoint{3.022926in}{2.328068in}}%
\pgfpathlineto{\pgfqpoint{3.014714in}{2.319213in}}%
\pgfpathlineto{\pgfqpoint{3.006495in}{2.310416in}}%
\pgfpathclose%
\pgfusepath{fill}%
\end{pgfscope}%
\begin{pgfscope}%
\pgfpathrectangle{\pgfqpoint{1.150000in}{0.150000in}}{\pgfqpoint{5.700000in}{5.700000in}}%
\pgfusepath{clip}%
\pgfsetbuttcap%
\pgfsetroundjoin%
\definecolor{currentfill}{rgb}{0.279566,0.067836,0.391917}%
\pgfsetfillcolor{currentfill}%
\pgfsetfillopacity{0.700000}%
\pgfsetlinewidth{0.000000pt}%
\definecolor{currentstroke}{rgb}{0.000000,0.000000,0.000000}%
\pgfsetstrokecolor{currentstroke}%
\pgfsetdash{}{0pt}%
\pgfpathmoveto{\pgfqpoint{3.286775in}{2.275661in}}%
\pgfpathlineto{\pgfqpoint{3.300221in}{2.269838in}}%
\pgfpathlineto{\pgfqpoint{3.313671in}{2.264111in}}%
\pgfpathlineto{\pgfqpoint{3.327125in}{2.258480in}}%
\pgfpathlineto{\pgfqpoint{3.340583in}{2.252944in}}%
\pgfpathlineto{\pgfqpoint{3.348687in}{2.262075in}}%
\pgfpathlineto{\pgfqpoint{3.356785in}{2.271244in}}%
\pgfpathlineto{\pgfqpoint{3.364876in}{2.280452in}}%
\pgfpathlineto{\pgfqpoint{3.372962in}{2.289702in}}%
\pgfpathlineto{\pgfqpoint{3.359516in}{2.295275in}}%
\pgfpathlineto{\pgfqpoint{3.346073in}{2.300943in}}%
\pgfpathlineto{\pgfqpoint{3.332635in}{2.306708in}}%
\pgfpathlineto{\pgfqpoint{3.319200in}{2.312568in}}%
\pgfpathlineto{\pgfqpoint{3.311103in}{2.303274in}}%
\pgfpathlineto{\pgfqpoint{3.303000in}{2.294026in}}%
\pgfpathlineto{\pgfqpoint{3.294891in}{2.284822in}}%
\pgfpathlineto{\pgfqpoint{3.286775in}{2.275661in}}%
\pgfpathclose%
\pgfusepath{fill}%
\end{pgfscope}%
\begin{pgfscope}%
\pgfpathrectangle{\pgfqpoint{1.150000in}{0.150000in}}{\pgfqpoint{5.700000in}{5.700000in}}%
\pgfusepath{clip}%
\pgfsetbuttcap%
\pgfsetroundjoin%
\definecolor{currentfill}{rgb}{0.282884,0.135920,0.453427}%
\pgfsetfillcolor{currentfill}%
\pgfsetfillopacity{0.700000}%
\pgfsetlinewidth{0.000000pt}%
\definecolor{currentstroke}{rgb}{0.000000,0.000000,0.000000}%
\pgfsetstrokecolor{currentstroke}%
\pgfsetdash{}{0pt}%
\pgfpathmoveto{\pgfqpoint{2.671617in}{2.415975in}}%
\pgfpathlineto{\pgfqpoint{2.685057in}{2.406151in}}%
\pgfpathlineto{\pgfqpoint{2.698496in}{2.396453in}}%
\pgfpathlineto{\pgfqpoint{2.711935in}{2.386878in}}%
\pgfpathlineto{\pgfqpoint{2.725375in}{2.377426in}}%
\pgfpathlineto{\pgfqpoint{2.733714in}{2.385553in}}%
\pgfpathlineto{\pgfqpoint{2.742045in}{2.393758in}}%
\pgfpathlineto{\pgfqpoint{2.750368in}{2.402041in}}%
\pgfpathlineto{\pgfqpoint{2.758682in}{2.410402in}}%
\pgfpathlineto{\pgfqpoint{2.745259in}{2.419808in}}%
\pgfpathlineto{\pgfqpoint{2.731836in}{2.429338in}}%
\pgfpathlineto{\pgfqpoint{2.718413in}{2.438990in}}%
\pgfpathlineto{\pgfqpoint{2.704989in}{2.448768in}}%
\pgfpathlineto{\pgfqpoint{2.696659in}{2.440445in}}%
\pgfpathlineto{\pgfqpoint{2.688320in}{2.432205in}}%
\pgfpathlineto{\pgfqpoint{2.679973in}{2.424048in}}%
\pgfpathlineto{\pgfqpoint{2.671617in}{2.415975in}}%
\pgfpathclose%
\pgfusepath{fill}%
\end{pgfscope}%
\begin{pgfscope}%
\pgfpathrectangle{\pgfqpoint{1.150000in}{0.150000in}}{\pgfqpoint{5.700000in}{5.700000in}}%
\pgfusepath{clip}%
\pgfsetbuttcap%
\pgfsetroundjoin%
\definecolor{currentfill}{rgb}{0.280894,0.078907,0.402329}%
\pgfsetfillcolor{currentfill}%
\pgfsetfillopacity{0.700000}%
\pgfsetlinewidth{0.000000pt}%
\definecolor{currentstroke}{rgb}{0.000000,0.000000,0.000000}%
\pgfsetstrokecolor{currentstroke}%
\pgfsetdash{}{0pt}%
\pgfpathmoveto{\pgfqpoint{3.652804in}{2.286525in}}%
\pgfpathlineto{\pgfqpoint{3.666306in}{2.282359in}}%
\pgfpathlineto{\pgfqpoint{3.679813in}{2.278279in}}%
\pgfpathlineto{\pgfqpoint{3.693325in}{2.274286in}}%
\pgfpathlineto{\pgfqpoint{3.706843in}{2.270379in}}%
\pgfpathlineto{\pgfqpoint{3.714821in}{2.279608in}}%
\pgfpathlineto{\pgfqpoint{3.722794in}{2.288868in}}%
\pgfpathlineto{\pgfqpoint{3.730761in}{2.298162in}}%
\pgfpathlineto{\pgfqpoint{3.738722in}{2.307493in}}%
\pgfpathlineto{\pgfqpoint{3.725214in}{2.311498in}}%
\pgfpathlineto{\pgfqpoint{3.711712in}{2.315590in}}%
\pgfpathlineto{\pgfqpoint{3.698216in}{2.319768in}}%
\pgfpathlineto{\pgfqpoint{3.684724in}{2.324032in}}%
\pgfpathlineto{\pgfqpoint{3.676753in}{2.314596in}}%
\pgfpathlineto{\pgfqpoint{3.668776in}{2.305201in}}%
\pgfpathlineto{\pgfqpoint{3.660793in}{2.295845in}}%
\pgfpathlineto{\pgfqpoint{3.652804in}{2.286525in}}%
\pgfpathclose%
\pgfusepath{fill}%
\end{pgfscope}%
\begin{pgfscope}%
\pgfpathrectangle{\pgfqpoint{1.150000in}{0.150000in}}{\pgfqpoint{5.700000in}{5.700000in}}%
\pgfusepath{clip}%
\pgfsetbuttcap%
\pgfsetroundjoin%
\definecolor{currentfill}{rgb}{0.282327,0.094955,0.417331}%
\pgfsetfillcolor{currentfill}%
\pgfsetfillopacity{0.700000}%
\pgfsetlinewidth{0.000000pt}%
\definecolor{currentstroke}{rgb}{0.000000,0.000000,0.000000}%
\pgfsetstrokecolor{currentstroke}%
\pgfsetdash{}{0pt}%
\pgfpathmoveto{\pgfqpoint{3.878687in}{2.315270in}}%
\pgfpathlineto{\pgfqpoint{3.892236in}{2.311901in}}%
\pgfpathlineto{\pgfqpoint{3.905791in}{2.308615in}}%
\pgfpathlineto{\pgfqpoint{3.919351in}{2.305410in}}%
\pgfpathlineto{\pgfqpoint{3.932918in}{2.302288in}}%
\pgfpathlineto{\pgfqpoint{3.940820in}{2.311439in}}%
\pgfpathlineto{\pgfqpoint{3.948716in}{2.320626in}}%
\pgfpathlineto{\pgfqpoint{3.956606in}{2.329851in}}%
\pgfpathlineto{\pgfqpoint{3.964491in}{2.339117in}}%
\pgfpathlineto{\pgfqpoint{3.950935in}{2.342379in}}%
\pgfpathlineto{\pgfqpoint{3.937385in}{2.345723in}}%
\pgfpathlineto{\pgfqpoint{3.923841in}{2.349148in}}%
\pgfpathlineto{\pgfqpoint{3.910302in}{2.352656in}}%
\pgfpathlineto{\pgfqpoint{3.902407in}{2.343243in}}%
\pgfpathlineto{\pgfqpoint{3.894506in}{2.333876in}}%
\pgfpathlineto{\pgfqpoint{3.886600in}{2.324553in}}%
\pgfpathlineto{\pgfqpoint{3.878687in}{2.315270in}}%
\pgfpathclose%
\pgfusepath{fill}%
\end{pgfscope}%
\begin{pgfscope}%
\pgfpathrectangle{\pgfqpoint{1.150000in}{0.150000in}}{\pgfqpoint{5.700000in}{5.700000in}}%
\pgfusepath{clip}%
\pgfsetbuttcap%
\pgfsetroundjoin%
\definecolor{currentfill}{rgb}{0.282656,0.100196,0.422160}%
\pgfsetfillcolor{currentfill}%
\pgfsetfillopacity{0.700000}%
\pgfsetlinewidth{0.000000pt}%
\definecolor{currentstroke}{rgb}{0.000000,0.000000,0.000000}%
\pgfsetstrokecolor{currentstroke}%
\pgfsetdash{}{0pt}%
\pgfpathmoveto{\pgfqpoint{2.866082in}{2.339446in}}%
\pgfpathlineto{\pgfqpoint{2.879510in}{2.331099in}}%
\pgfpathlineto{\pgfqpoint{2.892940in}{2.322866in}}%
\pgfpathlineto{\pgfqpoint{2.906371in}{2.314746in}}%
\pgfpathlineto{\pgfqpoint{2.919803in}{2.306737in}}%
\pgfpathlineto{\pgfqpoint{2.928065in}{2.315273in}}%
\pgfpathlineto{\pgfqpoint{2.936319in}{2.323870in}}%
\pgfpathlineto{\pgfqpoint{2.944566in}{2.332530in}}%
\pgfpathlineto{\pgfqpoint{2.952806in}{2.341253in}}%
\pgfpathlineto{\pgfqpoint{2.939388in}{2.349237in}}%
\pgfpathlineto{\pgfqpoint{2.925972in}{2.357333in}}%
\pgfpathlineto{\pgfqpoint{2.912557in}{2.365542in}}%
\pgfpathlineto{\pgfqpoint{2.899143in}{2.373863in}}%
\pgfpathlineto{\pgfqpoint{2.890889in}{2.365158in}}%
\pgfpathlineto{\pgfqpoint{2.882627in}{2.356519in}}%
\pgfpathlineto{\pgfqpoint{2.874358in}{2.347949in}}%
\pgfpathlineto{\pgfqpoint{2.866082in}{2.339446in}}%
\pgfpathclose%
\pgfusepath{fill}%
\end{pgfscope}%
\begin{pgfscope}%
\pgfpathrectangle{\pgfqpoint{1.150000in}{0.150000in}}{\pgfqpoint{5.700000in}{5.700000in}}%
\pgfusepath{clip}%
\pgfsetbuttcap%
\pgfsetroundjoin%
\definecolor{currentfill}{rgb}{0.281887,0.150881,0.465405}%
\pgfsetfillcolor{currentfill}%
\pgfsetfillopacity{0.700000}%
\pgfsetlinewidth{0.000000pt}%
\definecolor{currentstroke}{rgb}{0.000000,0.000000,0.000000}%
\pgfsetstrokecolor{currentstroke}%
\pgfsetdash{}{0pt}%
\pgfpathmoveto{\pgfqpoint{4.416226in}{2.422807in}}%
\pgfpathlineto{\pgfqpoint{4.429910in}{2.420732in}}%
\pgfpathlineto{\pgfqpoint{4.443602in}{2.418731in}}%
\pgfpathlineto{\pgfqpoint{4.457301in}{2.416804in}}%
\pgfpathlineto{\pgfqpoint{4.471008in}{2.414952in}}%
\pgfpathlineto{\pgfqpoint{4.478724in}{2.423803in}}%
\pgfpathlineto{\pgfqpoint{4.486435in}{2.432726in}}%
\pgfpathlineto{\pgfqpoint{4.494141in}{2.441725in}}%
\pgfpathlineto{\pgfqpoint{4.501843in}{2.450804in}}%
\pgfpathlineto{\pgfqpoint{4.488149in}{2.452898in}}%
\pgfpathlineto{\pgfqpoint{4.474463in}{2.455065in}}%
\pgfpathlineto{\pgfqpoint{4.460784in}{2.457307in}}%
\pgfpathlineto{\pgfqpoint{4.447113in}{2.459623in}}%
\pgfpathlineto{\pgfqpoint{4.439398in}{2.450296in}}%
\pgfpathlineto{\pgfqpoint{4.431679in}{2.441054in}}%
\pgfpathlineto{\pgfqpoint{4.423955in}{2.431892in}}%
\pgfpathlineto{\pgfqpoint{4.416226in}{2.422807in}}%
\pgfpathclose%
\pgfusepath{fill}%
\end{pgfscope}%
\begin{pgfscope}%
\pgfpathrectangle{\pgfqpoint{1.150000in}{0.150000in}}{\pgfqpoint{5.700000in}{5.700000in}}%
\pgfusepath{clip}%
\pgfsetbuttcap%
\pgfsetroundjoin%
\definecolor{currentfill}{rgb}{0.260571,0.246922,0.522828}%
\pgfsetfillcolor{currentfill}%
\pgfsetfillopacity{0.700000}%
\pgfsetlinewidth{0.000000pt}%
\definecolor{currentstroke}{rgb}{0.000000,0.000000,0.000000}%
\pgfsetstrokecolor{currentstroke}%
\pgfsetdash{}{0pt}%
\pgfpathmoveto{\pgfqpoint{5.125268in}{2.617285in}}%
\pgfpathlineto{\pgfqpoint{5.139143in}{2.615667in}}%
\pgfpathlineto{\pgfqpoint{5.153027in}{2.614118in}}%
\pgfpathlineto{\pgfqpoint{5.166919in}{2.612637in}}%
\pgfpathlineto{\pgfqpoint{5.180821in}{2.611224in}}%
\pgfpathlineto{\pgfqpoint{5.188303in}{2.620661in}}%
\pgfpathlineto{\pgfqpoint{5.195785in}{2.630281in}}%
\pgfpathlineto{\pgfqpoint{5.203266in}{2.640091in}}%
\pgfpathlineto{\pgfqpoint{5.210746in}{2.650097in}}%
\pgfpathlineto{\pgfqpoint{5.196863in}{2.651893in}}%
\pgfpathlineto{\pgfqpoint{5.182989in}{2.653756in}}%
\pgfpathlineto{\pgfqpoint{5.169123in}{2.655688in}}%
\pgfpathlineto{\pgfqpoint{5.155266in}{2.657687in}}%
\pgfpathlineto{\pgfqpoint{5.147768in}{2.647291in}}%
\pgfpathlineto{\pgfqpoint{5.140269in}{2.637097in}}%
\pgfpathlineto{\pgfqpoint{5.132769in}{2.627097in}}%
\pgfpathlineto{\pgfqpoint{5.125268in}{2.617285in}}%
\pgfpathclose%
\pgfusepath{fill}%
\end{pgfscope}%
\begin{pgfscope}%
\pgfpathrectangle{\pgfqpoint{1.150000in}{0.150000in}}{\pgfqpoint{5.700000in}{5.700000in}}%
\pgfusepath{clip}%
\pgfsetbuttcap%
\pgfsetroundjoin%
\definecolor{currentfill}{rgb}{0.214298,0.355619,0.551184}%
\pgfsetfillcolor{currentfill}%
\pgfsetfillopacity{0.700000}%
\pgfsetlinewidth{0.000000pt}%
\definecolor{currentstroke}{rgb}{0.000000,0.000000,0.000000}%
\pgfsetstrokecolor{currentstroke}%
\pgfsetdash{}{0pt}%
\pgfpathmoveto{\pgfqpoint{5.779748in}{2.869862in}}%
\pgfpathlineto{\pgfqpoint{5.793762in}{2.867278in}}%
\pgfpathlineto{\pgfqpoint{5.807785in}{2.864759in}}%
\pgfpathlineto{\pgfqpoint{5.821817in}{2.862305in}}%
\pgfpathlineto{\pgfqpoint{5.835859in}{2.859916in}}%
\pgfpathlineto{\pgfqpoint{5.843252in}{2.873248in}}%
\pgfpathlineto{\pgfqpoint{5.850654in}{2.886949in}}%
\pgfpathlineto{\pgfqpoint{5.858066in}{2.901030in}}%
\pgfpathlineto{\pgfqpoint{5.865486in}{2.915500in}}%
\pgfpathlineto{\pgfqpoint{5.851470in}{2.918412in}}%
\pgfpathlineto{\pgfqpoint{5.837462in}{2.921389in}}%
\pgfpathlineto{\pgfqpoint{5.823463in}{2.924430in}}%
\pgfpathlineto{\pgfqpoint{5.809472in}{2.927536in}}%
\pgfpathlineto{\pgfqpoint{5.802027in}{2.912536in}}%
\pgfpathlineto{\pgfqpoint{5.794592in}{2.897930in}}%
\pgfpathlineto{\pgfqpoint{5.787166in}{2.883709in}}%
\pgfpathlineto{\pgfqpoint{5.779748in}{2.869862in}}%
\pgfpathclose%
\pgfusepath{fill}%
\end{pgfscope}%
\begin{pgfscope}%
\pgfpathrectangle{\pgfqpoint{1.150000in}{0.150000in}}{\pgfqpoint{5.700000in}{5.700000in}}%
\pgfusepath{clip}%
\pgfsetbuttcap%
\pgfsetroundjoin%
\definecolor{currentfill}{rgb}{0.204903,0.375746,0.553533}%
\pgfsetfillcolor{currentfill}%
\pgfsetfillopacity{0.700000}%
\pgfsetlinewidth{0.000000pt}%
\definecolor{currentstroke}{rgb}{0.000000,0.000000,0.000000}%
\pgfsetstrokecolor{currentstroke}%
\pgfsetdash{}{0pt}%
\pgfpathmoveto{\pgfqpoint{5.865486in}{2.915500in}}%
\pgfpathlineto{\pgfqpoint{5.879512in}{2.912653in}}%
\pgfpathlineto{\pgfqpoint{5.893547in}{2.909870in}}%
\pgfpathlineto{\pgfqpoint{5.907591in}{2.907152in}}%
\pgfpathlineto{\pgfqpoint{5.921644in}{2.904498in}}%
\pgfpathlineto{\pgfqpoint{5.929050in}{2.918831in}}%
\pgfpathlineto{\pgfqpoint{5.936467in}{2.933568in}}%
\pgfpathlineto{\pgfqpoint{5.943895in}{2.948718in}}%
\pgfpathlineto{\pgfqpoint{5.951335in}{2.964292in}}%
\pgfpathlineto{\pgfqpoint{5.937307in}{2.967488in}}%
\pgfpathlineto{\pgfqpoint{5.923288in}{2.970749in}}%
\pgfpathlineto{\pgfqpoint{5.909278in}{2.974074in}}%
\pgfpathlineto{\pgfqpoint{5.895276in}{2.977464in}}%
\pgfpathlineto{\pgfqpoint{5.887812in}{2.961341in}}%
\pgfpathlineto{\pgfqpoint{5.880359in}{2.945646in}}%
\pgfpathlineto{\pgfqpoint{5.872917in}{2.930369in}}%
\pgfpathlineto{\pgfqpoint{5.865486in}{2.915500in}}%
\pgfpathclose%
\pgfusepath{fill}%
\end{pgfscope}%
\begin{pgfscope}%
\pgfpathrectangle{\pgfqpoint{1.150000in}{0.150000in}}{\pgfqpoint{5.700000in}{5.700000in}}%
\pgfusepath{clip}%
\pgfsetbuttcap%
\pgfsetroundjoin%
\definecolor{currentfill}{rgb}{0.279566,0.067836,0.391917}%
\pgfsetfillcolor{currentfill}%
\pgfsetfillopacity{0.700000}%
\pgfsetlinewidth{0.000000pt}%
\definecolor{currentstroke}{rgb}{0.000000,0.000000,0.000000}%
\pgfsetstrokecolor{currentstroke}%
\pgfsetdash{}{0pt}%
\pgfpathmoveto{\pgfqpoint{3.426786in}{2.268351in}}%
\pgfpathlineto{\pgfqpoint{3.440253in}{2.263246in}}%
\pgfpathlineto{\pgfqpoint{3.453723in}{2.258233in}}%
\pgfpathlineto{\pgfqpoint{3.467198in}{2.253312in}}%
\pgfpathlineto{\pgfqpoint{3.480678in}{2.248482in}}%
\pgfpathlineto{\pgfqpoint{3.488735in}{2.257673in}}%
\pgfpathlineto{\pgfqpoint{3.496786in}{2.266898in}}%
\pgfpathlineto{\pgfqpoint{3.504832in}{2.276157in}}%
\pgfpathlineto{\pgfqpoint{3.512871in}{2.285452in}}%
\pgfpathlineto{\pgfqpoint{3.499402in}{2.290339in}}%
\pgfpathlineto{\pgfqpoint{3.485938in}{2.295318in}}%
\pgfpathlineto{\pgfqpoint{3.472478in}{2.300388in}}%
\pgfpathlineto{\pgfqpoint{3.459023in}{2.305551in}}%
\pgfpathlineto{\pgfqpoint{3.450973in}{2.296191in}}%
\pgfpathlineto{\pgfqpoint{3.442916in}{2.286872in}}%
\pgfpathlineto{\pgfqpoint{3.434854in}{2.277592in}}%
\pgfpathlineto{\pgfqpoint{3.426786in}{2.268351in}}%
\pgfpathclose%
\pgfusepath{fill}%
\end{pgfscope}%
\begin{pgfscope}%
\pgfpathrectangle{\pgfqpoint{1.150000in}{0.150000in}}{\pgfqpoint{5.700000in}{5.700000in}}%
\pgfusepath{clip}%
\pgfsetbuttcap%
\pgfsetroundjoin%
\definecolor{currentfill}{rgb}{0.221989,0.339161,0.548752}%
\pgfsetfillcolor{currentfill}%
\pgfsetfillopacity{0.700000}%
\pgfsetlinewidth{0.000000pt}%
\definecolor{currentstroke}{rgb}{0.000000,0.000000,0.000000}%
\pgfsetstrokecolor{currentstroke}%
\pgfsetdash{}{0pt}%
\pgfpathmoveto{\pgfqpoint{5.694094in}{2.827013in}}%
\pgfpathlineto{\pgfqpoint{5.708095in}{2.824671in}}%
\pgfpathlineto{\pgfqpoint{5.722106in}{2.822394in}}%
\pgfpathlineto{\pgfqpoint{5.736125in}{2.820183in}}%
\pgfpathlineto{\pgfqpoint{5.750154in}{2.818037in}}%
\pgfpathlineto{\pgfqpoint{5.757542in}{2.830477in}}%
\pgfpathlineto{\pgfqpoint{5.764937in}{2.843256in}}%
\pgfpathlineto{\pgfqpoint{5.772338in}{2.856381in}}%
\pgfpathlineto{\pgfqpoint{5.779748in}{2.869862in}}%
\pgfpathlineto{\pgfqpoint{5.765743in}{2.872511in}}%
\pgfpathlineto{\pgfqpoint{5.751747in}{2.875225in}}%
\pgfpathlineto{\pgfqpoint{5.737760in}{2.878003in}}%
\pgfpathlineto{\pgfqpoint{5.723781in}{2.880847in}}%
\pgfpathlineto{\pgfqpoint{5.716348in}{2.866856in}}%
\pgfpathlineto{\pgfqpoint{5.708923in}{2.853226in}}%
\pgfpathlineto{\pgfqpoint{5.701505in}{2.839948in}}%
\pgfpathlineto{\pgfqpoint{5.694094in}{2.827013in}}%
\pgfpathclose%
\pgfusepath{fill}%
\end{pgfscope}%
\begin{pgfscope}%
\pgfpathrectangle{\pgfqpoint{1.150000in}{0.150000in}}{\pgfqpoint{5.700000in}{5.700000in}}%
\pgfusepath{clip}%
\pgfsetbuttcap%
\pgfsetroundjoin%
\definecolor{currentfill}{rgb}{0.283197,0.115680,0.436115}%
\pgfsetfillcolor{currentfill}%
\pgfsetfillopacity{0.700000}%
\pgfsetlinewidth{0.000000pt}%
\definecolor{currentstroke}{rgb}{0.000000,0.000000,0.000000}%
\pgfsetstrokecolor{currentstroke}%
\pgfsetdash{}{0pt}%
\pgfpathmoveto{\pgfqpoint{4.104572in}{2.352182in}}%
\pgfpathlineto{\pgfqpoint{4.118176in}{2.349481in}}%
\pgfpathlineto{\pgfqpoint{4.131787in}{2.346858in}}%
\pgfpathlineto{\pgfqpoint{4.145405in}{2.344313in}}%
\pgfpathlineto{\pgfqpoint{4.159029in}{2.341847in}}%
\pgfpathlineto{\pgfqpoint{4.166854in}{2.350848in}}%
\pgfpathlineto{\pgfqpoint{4.174674in}{2.359895in}}%
\pgfpathlineto{\pgfqpoint{4.182489in}{2.368989in}}%
\pgfpathlineto{\pgfqpoint{4.190298in}{2.378135in}}%
\pgfpathlineto{\pgfqpoint{4.176685in}{2.380782in}}%
\pgfpathlineto{\pgfqpoint{4.163078in}{2.383506in}}%
\pgfpathlineto{\pgfqpoint{4.149479in}{2.386309in}}%
\pgfpathlineto{\pgfqpoint{4.135886in}{2.389190in}}%
\pgfpathlineto{\pgfqpoint{4.128065in}{2.379856in}}%
\pgfpathlineto{\pgfqpoint{4.120239in}{2.370579in}}%
\pgfpathlineto{\pgfqpoint{4.112408in}{2.361356in}}%
\pgfpathlineto{\pgfqpoint{4.104572in}{2.352182in}}%
\pgfpathclose%
\pgfusepath{fill}%
\end{pgfscope}%
\begin{pgfscope}%
\pgfpathrectangle{\pgfqpoint{1.150000in}{0.150000in}}{\pgfqpoint{5.700000in}{5.700000in}}%
\pgfusepath{clip}%
\pgfsetbuttcap%
\pgfsetroundjoin%
\definecolor{currentfill}{rgb}{0.276194,0.190074,0.493001}%
\pgfsetfillcolor{currentfill}%
\pgfsetfillopacity{0.700000}%
\pgfsetlinewidth{0.000000pt}%
\definecolor{currentstroke}{rgb}{0.000000,0.000000,0.000000}%
\pgfsetstrokecolor{currentstroke}%
\pgfsetdash{}{0pt}%
\pgfpathmoveto{\pgfqpoint{4.727921in}{2.500983in}}%
\pgfpathlineto{\pgfqpoint{4.741691in}{2.499294in}}%
\pgfpathlineto{\pgfqpoint{4.755469in}{2.497676in}}%
\pgfpathlineto{\pgfqpoint{4.769255in}{2.496129in}}%
\pgfpathlineto{\pgfqpoint{4.783050in}{2.494654in}}%
\pgfpathlineto{\pgfqpoint{4.790658in}{2.503497in}}%
\pgfpathlineto{\pgfqpoint{4.798263in}{2.512451in}}%
\pgfpathlineto{\pgfqpoint{4.805864in}{2.521521in}}%
\pgfpathlineto{\pgfqpoint{4.813461in}{2.530712in}}%
\pgfpathlineto{\pgfqpoint{4.799682in}{2.532489in}}%
\pgfpathlineto{\pgfqpoint{4.785910in}{2.534338in}}%
\pgfpathlineto{\pgfqpoint{4.772147in}{2.536257in}}%
\pgfpathlineto{\pgfqpoint{4.758392in}{2.538248in}}%
\pgfpathlineto{\pgfqpoint{4.750779in}{2.528748in}}%
\pgfpathlineto{\pgfqpoint{4.743163in}{2.519375in}}%
\pgfpathlineto{\pgfqpoint{4.735544in}{2.510121in}}%
\pgfpathlineto{\pgfqpoint{4.727921in}{2.500983in}}%
\pgfpathclose%
\pgfusepath{fill}%
\end{pgfscope}%
\begin{pgfscope}%
\pgfpathrectangle{\pgfqpoint{1.150000in}{0.150000in}}{\pgfqpoint{5.700000in}{5.700000in}}%
\pgfusepath{clip}%
\pgfsetbuttcap%
\pgfsetroundjoin%
\definecolor{currentfill}{rgb}{0.277134,0.185228,0.489898}%
\pgfsetfillcolor{currentfill}%
\pgfsetfillopacity{0.700000}%
\pgfsetlinewidth{0.000000pt}%
\definecolor{currentstroke}{rgb}{0.000000,0.000000,0.000000}%
\pgfsetstrokecolor{currentstroke}%
\pgfsetdash{}{0pt}%
\pgfpathmoveto{\pgfqpoint{2.476543in}{2.513331in}}%
\pgfpathlineto{\pgfqpoint{2.490016in}{2.501856in}}%
\pgfpathlineto{\pgfqpoint{2.503486in}{2.490521in}}%
\pgfpathlineto{\pgfqpoint{2.516954in}{2.479322in}}%
\pgfpathlineto{\pgfqpoint{2.530421in}{2.468260in}}%
\pgfpathlineto{\pgfqpoint{2.538848in}{2.475847in}}%
\pgfpathlineto{\pgfqpoint{2.547265in}{2.483531in}}%
\pgfpathlineto{\pgfqpoint{2.555673in}{2.491308in}}%
\pgfpathlineto{\pgfqpoint{2.564072in}{2.499181in}}%
\pgfpathlineto{\pgfqpoint{2.550623in}{2.510176in}}%
\pgfpathlineto{\pgfqpoint{2.537173in}{2.521307in}}%
\pgfpathlineto{\pgfqpoint{2.523721in}{2.532575in}}%
\pgfpathlineto{\pgfqpoint{2.510267in}{2.543982in}}%
\pgfpathlineto{\pgfqpoint{2.501850in}{2.536170in}}%
\pgfpathlineto{\pgfqpoint{2.493424in}{2.528457in}}%
\pgfpathlineto{\pgfqpoint{2.484988in}{2.520844in}}%
\pgfpathlineto{\pgfqpoint{2.476543in}{2.513331in}}%
\pgfpathclose%
\pgfusepath{fill}%
\end{pgfscope}%
\begin{pgfscope}%
\pgfpathrectangle{\pgfqpoint{1.150000in}{0.150000in}}{\pgfqpoint{5.700000in}{5.700000in}}%
\pgfusepath{clip}%
\pgfsetbuttcap%
\pgfsetroundjoin%
\definecolor{currentfill}{rgb}{0.195860,0.395433,0.555276}%
\pgfsetfillcolor{currentfill}%
\pgfsetfillopacity{0.700000}%
\pgfsetlinewidth{0.000000pt}%
\definecolor{currentstroke}{rgb}{0.000000,0.000000,0.000000}%
\pgfsetstrokecolor{currentstroke}%
\pgfsetdash{}{0pt}%
\pgfpathmoveto{\pgfqpoint{5.951335in}{2.964292in}}%
\pgfpathlineto{\pgfqpoint{5.965372in}{2.961159in}}%
\pgfpathlineto{\pgfqpoint{5.979417in}{2.958091in}}%
\pgfpathlineto{\pgfqpoint{5.993472in}{2.955087in}}%
\pgfpathlineto{\pgfqpoint{6.007536in}{2.952147in}}%
\pgfpathlineto{\pgfqpoint{6.014962in}{2.967598in}}%
\pgfpathlineto{\pgfqpoint{6.022402in}{2.983489in}}%
\pgfpathlineto{\pgfqpoint{6.029855in}{2.999829in}}%
\pgfpathlineto{\pgfqpoint{6.037322in}{3.016628in}}%
\pgfpathlineto{\pgfqpoint{6.023284in}{3.020131in}}%
\pgfpathlineto{\pgfqpoint{6.009255in}{3.023698in}}%
\pgfpathlineto{\pgfqpoint{5.995235in}{3.027329in}}%
\pgfpathlineto{\pgfqpoint{5.981223in}{3.031024in}}%
\pgfpathlineto{\pgfqpoint{5.973730in}{3.013654in}}%
\pgfpathlineto{\pgfqpoint{5.966252in}{2.996750in}}%
\pgfpathlineto{\pgfqpoint{5.958787in}{2.980299in}}%
\pgfpathlineto{\pgfqpoint{5.951335in}{2.964292in}}%
\pgfpathclose%
\pgfusepath{fill}%
\end{pgfscope}%
\begin{pgfscope}%
\pgfpathrectangle{\pgfqpoint{1.150000in}{0.150000in}}{\pgfqpoint{5.700000in}{5.700000in}}%
\pgfusepath{clip}%
\pgfsetbuttcap%
\pgfsetroundjoin%
\definecolor{currentfill}{rgb}{0.229739,0.322361,0.545706}%
\pgfsetfillcolor{currentfill}%
\pgfsetfillopacity{0.700000}%
\pgfsetlinewidth{0.000000pt}%
\definecolor{currentstroke}{rgb}{0.000000,0.000000,0.000000}%
\pgfsetstrokecolor{currentstroke}%
\pgfsetdash{}{0pt}%
\pgfpathmoveto{\pgfqpoint{5.608500in}{2.786612in}}%
\pgfpathlineto{\pgfqpoint{5.622488in}{2.784491in}}%
\pgfpathlineto{\pgfqpoint{5.636485in}{2.782436in}}%
\pgfpathlineto{\pgfqpoint{5.650492in}{2.780446in}}%
\pgfpathlineto{\pgfqpoint{5.664508in}{2.778522in}}%
\pgfpathlineto{\pgfqpoint{5.671896in}{2.790174in}}%
\pgfpathlineto{\pgfqpoint{5.679289in}{2.802134in}}%
\pgfpathlineto{\pgfqpoint{5.686689in}{2.814411in}}%
\pgfpathlineto{\pgfqpoint{5.694094in}{2.827013in}}%
\pgfpathlineto{\pgfqpoint{5.680101in}{2.829420in}}%
\pgfpathlineto{\pgfqpoint{5.666118in}{2.831892in}}%
\pgfpathlineto{\pgfqpoint{5.652143in}{2.834430in}}%
\pgfpathlineto{\pgfqpoint{5.638177in}{2.837033in}}%
\pgfpathlineto{\pgfqpoint{5.630750in}{2.823942in}}%
\pgfpathlineto{\pgfqpoint{5.623328in}{2.811181in}}%
\pgfpathlineto{\pgfqpoint{5.615911in}{2.798740in}}%
\pgfpathlineto{\pgfqpoint{5.608500in}{2.786612in}}%
\pgfpathclose%
\pgfusepath{fill}%
\end{pgfscope}%
\begin{pgfscope}%
\pgfpathrectangle{\pgfqpoint{1.150000in}{0.150000in}}{\pgfqpoint{5.700000in}{5.700000in}}%
\pgfusepath{clip}%
\pgfsetbuttcap%
\pgfsetroundjoin%
\definecolor{currentfill}{rgb}{0.237441,0.305202,0.541921}%
\pgfsetfillcolor{currentfill}%
\pgfsetfillopacity{0.700000}%
\pgfsetlinewidth{0.000000pt}%
\definecolor{currentstroke}{rgb}{0.000000,0.000000,0.000000}%
\pgfsetstrokecolor{currentstroke}%
\pgfsetdash{}{0pt}%
\pgfpathmoveto{\pgfqpoint{5.522947in}{2.748347in}}%
\pgfpathlineto{\pgfqpoint{5.536921in}{2.746425in}}%
\pgfpathlineto{\pgfqpoint{5.550904in}{2.744569in}}%
\pgfpathlineto{\pgfqpoint{5.564896in}{2.742779in}}%
\pgfpathlineto{\pgfqpoint{5.578898in}{2.741055in}}%
\pgfpathlineto{\pgfqpoint{5.586293in}{2.752018in}}%
\pgfpathlineto{\pgfqpoint{5.593691in}{2.763259in}}%
\pgfpathlineto{\pgfqpoint{5.601093in}{2.774788in}}%
\pgfpathlineto{\pgfqpoint{5.608500in}{2.786612in}}%
\pgfpathlineto{\pgfqpoint{5.594521in}{2.788799in}}%
\pgfpathlineto{\pgfqpoint{5.580551in}{2.791052in}}%
\pgfpathlineto{\pgfqpoint{5.566590in}{2.793371in}}%
\pgfpathlineto{\pgfqpoint{5.552637in}{2.795755in}}%
\pgfpathlineto{\pgfqpoint{5.545209in}{2.783461in}}%
\pgfpathlineto{\pgfqpoint{5.537784in}{2.771467in}}%
\pgfpathlineto{\pgfqpoint{5.530364in}{2.759766in}}%
\pgfpathlineto{\pgfqpoint{5.522947in}{2.748347in}}%
\pgfpathclose%
\pgfusepath{fill}%
\end{pgfscope}%
\begin{pgfscope}%
\pgfpathrectangle{\pgfqpoint{1.150000in}{0.150000in}}{\pgfqpoint{5.700000in}{5.700000in}}%
\pgfusepath{clip}%
\pgfsetbuttcap%
\pgfsetroundjoin%
\definecolor{currentfill}{rgb}{0.265145,0.232956,0.516599}%
\pgfsetfillcolor{currentfill}%
\pgfsetfillopacity{0.700000}%
\pgfsetlinewidth{0.000000pt}%
\definecolor{currentstroke}{rgb}{0.000000,0.000000,0.000000}%
\pgfsetstrokecolor{currentstroke}%
\pgfsetdash{}{0pt}%
\pgfpathmoveto{\pgfqpoint{5.039767in}{2.585492in}}%
\pgfpathlineto{\pgfqpoint{5.053624in}{2.583961in}}%
\pgfpathlineto{\pgfqpoint{5.067491in}{2.582500in}}%
\pgfpathlineto{\pgfqpoint{5.081366in}{2.581107in}}%
\pgfpathlineto{\pgfqpoint{5.095250in}{2.579784in}}%
\pgfpathlineto{\pgfqpoint{5.102757in}{2.588910in}}%
\pgfpathlineto{\pgfqpoint{5.110262in}{2.598198in}}%
\pgfpathlineto{\pgfqpoint{5.117766in}{2.607654in}}%
\pgfpathlineto{\pgfqpoint{5.125268in}{2.617285in}}%
\pgfpathlineto{\pgfqpoint{5.111402in}{2.618971in}}%
\pgfpathlineto{\pgfqpoint{5.097544in}{2.620726in}}%
\pgfpathlineto{\pgfqpoint{5.083696in}{2.622550in}}%
\pgfpathlineto{\pgfqpoint{5.069855in}{2.624443in}}%
\pgfpathlineto{\pgfqpoint{5.062335in}{2.614443in}}%
\pgfpathlineto{\pgfqpoint{5.054814in}{2.604622in}}%
\pgfpathlineto{\pgfqpoint{5.047291in}{2.594974in}}%
\pgfpathlineto{\pgfqpoint{5.039767in}{2.585492in}}%
\pgfpathclose%
\pgfusepath{fill}%
\end{pgfscope}%
\begin{pgfscope}%
\pgfpathrectangle{\pgfqpoint{1.150000in}{0.150000in}}{\pgfqpoint{5.700000in}{5.700000in}}%
\pgfusepath{clip}%
\pgfsetbuttcap%
\pgfsetroundjoin%
\definecolor{currentfill}{rgb}{0.283229,0.120777,0.440584}%
\pgfsetfillcolor{currentfill}%
\pgfsetfillopacity{0.700000}%
\pgfsetlinewidth{0.000000pt}%
\definecolor{currentstroke}{rgb}{0.000000,0.000000,0.000000}%
\pgfsetstrokecolor{currentstroke}%
\pgfsetdash{}{0pt}%
\pgfpathmoveto{\pgfqpoint{2.725375in}{2.377426in}}%
\pgfpathlineto{\pgfqpoint{2.738814in}{2.368095in}}%
\pgfpathlineto{\pgfqpoint{2.752253in}{2.358885in}}%
\pgfpathlineto{\pgfqpoint{2.765693in}{2.349795in}}%
\pgfpathlineto{\pgfqpoint{2.779133in}{2.340824in}}%
\pgfpathlineto{\pgfqpoint{2.787456in}{2.349004in}}%
\pgfpathlineto{\pgfqpoint{2.795771in}{2.357257in}}%
\pgfpathlineto{\pgfqpoint{2.804078in}{2.365583in}}%
\pgfpathlineto{\pgfqpoint{2.812377in}{2.373981in}}%
\pgfpathlineto{\pgfqpoint{2.798953in}{2.382907in}}%
\pgfpathlineto{\pgfqpoint{2.785529in}{2.391952in}}%
\pgfpathlineto{\pgfqpoint{2.772106in}{2.401117in}}%
\pgfpathlineto{\pgfqpoint{2.758682in}{2.410402in}}%
\pgfpathlineto{\pgfqpoint{2.750368in}{2.402041in}}%
\pgfpathlineto{\pgfqpoint{2.742045in}{2.393758in}}%
\pgfpathlineto{\pgfqpoint{2.733714in}{2.385553in}}%
\pgfpathlineto{\pgfqpoint{2.725375in}{2.377426in}}%
\pgfpathclose%
\pgfusepath{fill}%
\end{pgfscope}%
\begin{pgfscope}%
\pgfpathrectangle{\pgfqpoint{1.150000in}{0.150000in}}{\pgfqpoint{5.700000in}{5.700000in}}%
\pgfusepath{clip}%
\pgfsetbuttcap%
\pgfsetroundjoin%
\definecolor{currentfill}{rgb}{0.281446,0.084320,0.407414}%
\pgfsetfillcolor{currentfill}%
\pgfsetfillopacity{0.700000}%
\pgfsetlinewidth{0.000000pt}%
\definecolor{currentstroke}{rgb}{0.000000,0.000000,0.000000}%
\pgfsetstrokecolor{currentstroke}%
\pgfsetdash{}{0pt}%
\pgfpathmoveto{\pgfqpoint{3.792807in}{2.292321in}}%
\pgfpathlineto{\pgfqpoint{3.806342in}{2.288739in}}%
\pgfpathlineto{\pgfqpoint{3.819883in}{2.285240in}}%
\pgfpathlineto{\pgfqpoint{3.833430in}{2.281825in}}%
\pgfpathlineto{\pgfqpoint{3.846984in}{2.278493in}}%
\pgfpathlineto{\pgfqpoint{3.854918in}{2.287639in}}%
\pgfpathlineto{\pgfqpoint{3.862847in}{2.296815in}}%
\pgfpathlineto{\pgfqpoint{3.870770in}{2.306025in}}%
\pgfpathlineto{\pgfqpoint{3.878687in}{2.315270in}}%
\pgfpathlineto{\pgfqpoint{3.865145in}{2.318721in}}%
\pgfpathlineto{\pgfqpoint{3.851608in}{2.322255in}}%
\pgfpathlineto{\pgfqpoint{3.838078in}{2.325872in}}%
\pgfpathlineto{\pgfqpoint{3.824553in}{2.329573in}}%
\pgfpathlineto{\pgfqpoint{3.816625in}{2.320202in}}%
\pgfpathlineto{\pgfqpoint{3.808691in}{2.310871in}}%
\pgfpathlineto{\pgfqpoint{3.800752in}{2.301578in}}%
\pgfpathlineto{\pgfqpoint{3.792807in}{2.292321in}}%
\pgfpathclose%
\pgfusepath{fill}%
\end{pgfscope}%
\begin{pgfscope}%
\pgfpathrectangle{\pgfqpoint{1.150000in}{0.150000in}}{\pgfqpoint{5.700000in}{5.700000in}}%
\pgfusepath{clip}%
\pgfsetbuttcap%
\pgfsetroundjoin%
\definecolor{currentfill}{rgb}{0.282623,0.140926,0.457517}%
\pgfsetfillcolor{currentfill}%
\pgfsetfillopacity{0.700000}%
\pgfsetlinewidth{0.000000pt}%
\definecolor{currentstroke}{rgb}{0.000000,0.000000,0.000000}%
\pgfsetstrokecolor{currentstroke}%
\pgfsetdash{}{0pt}%
\pgfpathmoveto{\pgfqpoint{4.330553in}{2.395317in}}%
\pgfpathlineto{\pgfqpoint{4.344219in}{2.393161in}}%
\pgfpathlineto{\pgfqpoint{4.357893in}{2.391081in}}%
\pgfpathlineto{\pgfqpoint{4.371574in}{2.389076in}}%
\pgfpathlineto{\pgfqpoint{4.385263in}{2.387146in}}%
\pgfpathlineto{\pgfqpoint{4.393011in}{2.395968in}}%
\pgfpathlineto{\pgfqpoint{4.400754in}{2.404849in}}%
\pgfpathlineto{\pgfqpoint{4.408493in}{2.413794in}}%
\pgfpathlineto{\pgfqpoint{4.416226in}{2.422807in}}%
\pgfpathlineto{\pgfqpoint{4.402550in}{2.424958in}}%
\pgfpathlineto{\pgfqpoint{4.388881in}{2.427184in}}%
\pgfpathlineto{\pgfqpoint{4.375220in}{2.429485in}}%
\pgfpathlineto{\pgfqpoint{4.361565in}{2.431861in}}%
\pgfpathlineto{\pgfqpoint{4.353820in}{2.422620in}}%
\pgfpathlineto{\pgfqpoint{4.346069in}{2.413452in}}%
\pgfpathlineto{\pgfqpoint{4.338314in}{2.404352in}}%
\pgfpathlineto{\pgfqpoint{4.330553in}{2.395317in}}%
\pgfpathclose%
\pgfusepath{fill}%
\end{pgfscope}%
\begin{pgfscope}%
\pgfpathrectangle{\pgfqpoint{1.150000in}{0.150000in}}{\pgfqpoint{5.700000in}{5.700000in}}%
\pgfusepath{clip}%
\pgfsetbuttcap%
\pgfsetroundjoin%
\definecolor{currentfill}{rgb}{0.280267,0.073417,0.397163}%
\pgfsetfillcolor{currentfill}%
\pgfsetfillopacity{0.700000}%
\pgfsetlinewidth{0.000000pt}%
\definecolor{currentstroke}{rgb}{0.000000,0.000000,0.000000}%
\pgfsetstrokecolor{currentstroke}%
\pgfsetdash{}{0pt}%
\pgfpathmoveto{\pgfqpoint{3.566792in}{2.266803in}}%
\pgfpathlineto{\pgfqpoint{3.580284in}{2.262364in}}%
\pgfpathlineto{\pgfqpoint{3.593782in}{2.258013in}}%
\pgfpathlineto{\pgfqpoint{3.607284in}{2.253751in}}%
\pgfpathlineto{\pgfqpoint{3.620791in}{2.249576in}}%
\pgfpathlineto{\pgfqpoint{3.628803in}{2.258767in}}%
\pgfpathlineto{\pgfqpoint{3.636809in}{2.267988in}}%
\pgfpathlineto{\pgfqpoint{3.644809in}{2.277240in}}%
\pgfpathlineto{\pgfqpoint{3.652804in}{2.286525in}}%
\pgfpathlineto{\pgfqpoint{3.639307in}{2.290778in}}%
\pgfpathlineto{\pgfqpoint{3.625816in}{2.295119in}}%
\pgfpathlineto{\pgfqpoint{3.612329in}{2.299548in}}%
\pgfpathlineto{\pgfqpoint{3.598847in}{2.304065in}}%
\pgfpathlineto{\pgfqpoint{3.590842in}{2.294694in}}%
\pgfpathlineto{\pgfqpoint{3.582831in}{2.285362in}}%
\pgfpathlineto{\pgfqpoint{3.574815in}{2.276065in}}%
\pgfpathlineto{\pgfqpoint{3.566792in}{2.266803in}}%
\pgfpathclose%
\pgfusepath{fill}%
\end{pgfscope}%
\begin{pgfscope}%
\pgfpathrectangle{\pgfqpoint{1.150000in}{0.150000in}}{\pgfqpoint{5.700000in}{5.700000in}}%
\pgfusepath{clip}%
\pgfsetbuttcap%
\pgfsetroundjoin%
\definecolor{currentfill}{rgb}{0.187231,0.414746,0.556547}%
\pgfsetfillcolor{currentfill}%
\pgfsetfillopacity{0.700000}%
\pgfsetlinewidth{0.000000pt}%
\definecolor{currentstroke}{rgb}{0.000000,0.000000,0.000000}%
\pgfsetstrokecolor{currentstroke}%
\pgfsetdash{}{0pt}%
\pgfpathmoveto{\pgfqpoint{6.037322in}{3.016628in}}%
\pgfpathlineto{\pgfqpoint{6.051369in}{3.013189in}}%
\pgfpathlineto{\pgfqpoint{6.065425in}{3.009814in}}%
\pgfpathlineto{\pgfqpoint{6.079490in}{3.006502in}}%
\pgfpathlineto{\pgfqpoint{6.093564in}{3.003255in}}%
\pgfpathlineto{\pgfqpoint{6.101020in}{3.019948in}}%
\pgfpathlineto{\pgfqpoint{6.108491in}{3.037118in}}%
\pgfpathlineto{\pgfqpoint{6.115978in}{3.054774in}}%
\pgfpathlineto{\pgfqpoint{6.101924in}{3.058458in}}%
\pgfpathlineto{\pgfqpoint{6.087879in}{3.062204in}}%
\pgfpathlineto{\pgfqpoint{6.073842in}{3.066015in}}%
\pgfpathlineto{\pgfqpoint{6.059815in}{3.069889in}}%
\pgfpathlineto{\pgfqpoint{6.052302in}{3.051648in}}%
\pgfpathlineto{\pgfqpoint{6.044804in}{3.033897in}}%
\pgfpathlineto{\pgfqpoint{6.037322in}{3.016628in}}%
\pgfpathclose%
\pgfusepath{fill}%
\end{pgfscope}%
\begin{pgfscope}%
\pgfpathrectangle{\pgfqpoint{1.150000in}{0.150000in}}{\pgfqpoint{5.700000in}{5.700000in}}%
\pgfusepath{clip}%
\pgfsetbuttcap%
\pgfsetroundjoin%
\definecolor{currentfill}{rgb}{0.278012,0.180367,0.486697}%
\pgfsetfillcolor{currentfill}%
\pgfsetfillopacity{0.700000}%
\pgfsetlinewidth{0.000000pt}%
\definecolor{currentstroke}{rgb}{0.000000,0.000000,0.000000}%
\pgfsetstrokecolor{currentstroke}%
\pgfsetdash{}{0pt}%
\pgfpathmoveto{\pgfqpoint{4.642334in}{2.471828in}}%
\pgfpathlineto{\pgfqpoint{4.656085in}{2.470132in}}%
\pgfpathlineto{\pgfqpoint{4.669845in}{2.468508in}}%
\pgfpathlineto{\pgfqpoint{4.683613in}{2.466957in}}%
\pgfpathlineto{\pgfqpoint{4.697389in}{2.465478in}}%
\pgfpathlineto{\pgfqpoint{4.705028in}{2.474207in}}%
\pgfpathlineto{\pgfqpoint{4.712663in}{2.483032in}}%
\pgfpathlineto{\pgfqpoint{4.720294in}{2.491955in}}%
\pgfpathlineto{\pgfqpoint{4.727921in}{2.500983in}}%
\pgfpathlineto{\pgfqpoint{4.714159in}{2.502745in}}%
\pgfpathlineto{\pgfqpoint{4.700405in}{2.504578in}}%
\pgfpathlineto{\pgfqpoint{4.686660in}{2.506483in}}%
\pgfpathlineto{\pgfqpoint{4.672922in}{2.508460in}}%
\pgfpathlineto{\pgfqpoint{4.665281in}{2.499143in}}%
\pgfpathlineto{\pgfqpoint{4.657636in}{2.489935in}}%
\pgfpathlineto{\pgfqpoint{4.649987in}{2.480832in}}%
\pgfpathlineto{\pgfqpoint{4.642334in}{2.471828in}}%
\pgfpathclose%
\pgfusepath{fill}%
\end{pgfscope}%
\begin{pgfscope}%
\pgfpathrectangle{\pgfqpoint{1.150000in}{0.150000in}}{\pgfqpoint{5.700000in}{5.700000in}}%
\pgfusepath{clip}%
\pgfsetbuttcap%
\pgfsetroundjoin%
\definecolor{currentfill}{rgb}{0.243113,0.292092,0.538516}%
\pgfsetfillcolor{currentfill}%
\pgfsetfillopacity{0.700000}%
\pgfsetlinewidth{0.000000pt}%
\definecolor{currentstroke}{rgb}{0.000000,0.000000,0.000000}%
\pgfsetstrokecolor{currentstroke}%
\pgfsetdash{}{0pt}%
\pgfpathmoveto{\pgfqpoint{5.437416in}{2.711929in}}%
\pgfpathlineto{\pgfqpoint{5.451376in}{2.710184in}}%
\pgfpathlineto{\pgfqpoint{5.465344in}{2.708505in}}%
\pgfpathlineto{\pgfqpoint{5.479322in}{2.706894in}}%
\pgfpathlineto{\pgfqpoint{5.493308in}{2.705349in}}%
\pgfpathlineto{\pgfqpoint{5.500714in}{2.715713in}}%
\pgfpathlineto{\pgfqpoint{5.508122in}{2.726329in}}%
\pgfpathlineto{\pgfqpoint{5.515533in}{2.737204in}}%
\pgfpathlineto{\pgfqpoint{5.522947in}{2.748347in}}%
\pgfpathlineto{\pgfqpoint{5.508982in}{2.750336in}}%
\pgfpathlineto{\pgfqpoint{5.495026in}{2.752390in}}%
\pgfpathlineto{\pgfqpoint{5.481079in}{2.754512in}}%
\pgfpathlineto{\pgfqpoint{5.467141in}{2.756699in}}%
\pgfpathlineto{\pgfqpoint{5.459706in}{2.745106in}}%
\pgfpathlineto{\pgfqpoint{5.452274in}{2.733785in}}%
\pgfpathlineto{\pgfqpoint{5.444844in}{2.722729in}}%
\pgfpathlineto{\pgfqpoint{5.437416in}{2.711929in}}%
\pgfpathclose%
\pgfusepath{fill}%
\end{pgfscope}%
\begin{pgfscope}%
\pgfpathrectangle{\pgfqpoint{1.150000in}{0.150000in}}{\pgfqpoint{5.700000in}{5.700000in}}%
\pgfusepath{clip}%
\pgfsetbuttcap%
\pgfsetroundjoin%
\definecolor{currentfill}{rgb}{0.280894,0.078907,0.402329}%
\pgfsetfillcolor{currentfill}%
\pgfsetfillopacity{0.700000}%
\pgfsetlinewidth{0.000000pt}%
\definecolor{currentstroke}{rgb}{0.000000,0.000000,0.000000}%
\pgfsetstrokecolor{currentstroke}%
\pgfsetdash{}{0pt}%
\pgfpathmoveto{\pgfqpoint{3.060214in}{2.281301in}}%
\pgfpathlineto{\pgfqpoint{3.073649in}{2.274286in}}%
\pgfpathlineto{\pgfqpoint{3.087087in}{2.267375in}}%
\pgfpathlineto{\pgfqpoint{3.100528in}{2.260568in}}%
\pgfpathlineto{\pgfqpoint{3.113971in}{2.253863in}}%
\pgfpathlineto{\pgfqpoint{3.122163in}{2.262682in}}%
\pgfpathlineto{\pgfqpoint{3.130349in}{2.271550in}}%
\pgfpathlineto{\pgfqpoint{3.138528in}{2.280465in}}%
\pgfpathlineto{\pgfqpoint{3.146700in}{2.289430in}}%
\pgfpathlineto{\pgfqpoint{3.133270in}{2.296132in}}%
\pgfpathlineto{\pgfqpoint{3.119842in}{2.302936in}}%
\pgfpathlineto{\pgfqpoint{3.106418in}{2.309843in}}%
\pgfpathlineto{\pgfqpoint{3.092995in}{2.316854in}}%
\pgfpathlineto{\pgfqpoint{3.084810in}{2.307886in}}%
\pgfpathlineto{\pgfqpoint{3.076618in}{2.298971in}}%
\pgfpathlineto{\pgfqpoint{3.068419in}{2.290109in}}%
\pgfpathlineto{\pgfqpoint{3.060214in}{2.281301in}}%
\pgfpathclose%
\pgfusepath{fill}%
\end{pgfscope}%
\begin{pgfscope}%
\pgfpathrectangle{\pgfqpoint{1.150000in}{0.150000in}}{\pgfqpoint{5.700000in}{5.700000in}}%
\pgfusepath{clip}%
\pgfsetbuttcap%
\pgfsetroundjoin%
\definecolor{currentfill}{rgb}{0.282910,0.105393,0.426902}%
\pgfsetfillcolor{currentfill}%
\pgfsetfillopacity{0.700000}%
\pgfsetlinewidth{0.000000pt}%
\definecolor{currentstroke}{rgb}{0.000000,0.000000,0.000000}%
\pgfsetstrokecolor{currentstroke}%
\pgfsetdash{}{0pt}%
\pgfpathmoveto{\pgfqpoint{4.018780in}{2.326879in}}%
\pgfpathlineto{\pgfqpoint{4.032368in}{2.324020in}}%
\pgfpathlineto{\pgfqpoint{4.045963in}{2.321241in}}%
\pgfpathlineto{\pgfqpoint{4.059565in}{2.318541in}}%
\pgfpathlineto{\pgfqpoint{4.073173in}{2.315921in}}%
\pgfpathlineto{\pgfqpoint{4.081031in}{2.324927in}}%
\pgfpathlineto{\pgfqpoint{4.088883in}{2.333971in}}%
\pgfpathlineto{\pgfqpoint{4.096730in}{2.343055in}}%
\pgfpathlineto{\pgfqpoint{4.104572in}{2.352182in}}%
\pgfpathlineto{\pgfqpoint{4.090975in}{2.354962in}}%
\pgfpathlineto{\pgfqpoint{4.077384in}{2.357821in}}%
\pgfpathlineto{\pgfqpoint{4.063800in}{2.360760in}}%
\pgfpathlineto{\pgfqpoint{4.050223in}{2.363778in}}%
\pgfpathlineto{\pgfqpoint{4.042370in}{2.354484in}}%
\pgfpathlineto{\pgfqpoint{4.034512in}{2.345238in}}%
\pgfpathlineto{\pgfqpoint{4.026649in}{2.336037in}}%
\pgfpathlineto{\pgfqpoint{4.018780in}{2.326879in}}%
\pgfpathclose%
\pgfusepath{fill}%
\end{pgfscope}%
\begin{pgfscope}%
\pgfpathrectangle{\pgfqpoint{1.150000in}{0.150000in}}{\pgfqpoint{5.700000in}{5.700000in}}%
\pgfusepath{clip}%
\pgfsetbuttcap%
\pgfsetroundjoin%
\definecolor{currentfill}{rgb}{0.280255,0.165693,0.476498}%
\pgfsetfillcolor{currentfill}%
\pgfsetfillopacity{0.700000}%
\pgfsetlinewidth{0.000000pt}%
\definecolor{currentstroke}{rgb}{0.000000,0.000000,0.000000}%
\pgfsetstrokecolor{currentstroke}%
\pgfsetdash{}{0pt}%
\pgfpathmoveto{\pgfqpoint{2.530421in}{2.468260in}}%
\pgfpathlineto{\pgfqpoint{2.543886in}{2.457332in}}%
\pgfpathlineto{\pgfqpoint{2.557350in}{2.446539in}}%
\pgfpathlineto{\pgfqpoint{2.570812in}{2.435877in}}%
\pgfpathlineto{\pgfqpoint{2.584274in}{2.425348in}}%
\pgfpathlineto{\pgfqpoint{2.592681in}{2.433010in}}%
\pgfpathlineto{\pgfqpoint{2.601080in}{2.440763in}}%
\pgfpathlineto{\pgfqpoint{2.609471in}{2.448605in}}%
\pgfpathlineto{\pgfqpoint{2.617852in}{2.456536in}}%
\pgfpathlineto{\pgfqpoint{2.604409in}{2.466999in}}%
\pgfpathlineto{\pgfqpoint{2.590964in}{2.477594in}}%
\pgfpathlineto{\pgfqpoint{2.577519in}{2.488320in}}%
\pgfpathlineto{\pgfqpoint{2.564072in}{2.499181in}}%
\pgfpathlineto{\pgfqpoint{2.555673in}{2.491308in}}%
\pgfpathlineto{\pgfqpoint{2.547265in}{2.483531in}}%
\pgfpathlineto{\pgfqpoint{2.538848in}{2.475847in}}%
\pgfpathlineto{\pgfqpoint{2.530421in}{2.468260in}}%
\pgfpathclose%
\pgfusepath{fill}%
\end{pgfscope}%
\begin{pgfscope}%
\pgfpathrectangle{\pgfqpoint{1.150000in}{0.150000in}}{\pgfqpoint{5.700000in}{5.700000in}}%
\pgfusepath{clip}%
\pgfsetbuttcap%
\pgfsetroundjoin%
\definecolor{currentfill}{rgb}{0.279566,0.067836,0.391917}%
\pgfsetfillcolor{currentfill}%
\pgfsetfillopacity{0.700000}%
\pgfsetlinewidth{0.000000pt}%
\definecolor{currentstroke}{rgb}{0.000000,0.000000,0.000000}%
\pgfsetstrokecolor{currentstroke}%
\pgfsetdash{}{0pt}%
\pgfpathmoveto{\pgfqpoint{3.200449in}{2.263638in}}%
\pgfpathlineto{\pgfqpoint{3.213894in}{2.257440in}}%
\pgfpathlineto{\pgfqpoint{3.227342in}{2.251341in}}%
\pgfpathlineto{\pgfqpoint{3.240794in}{2.245340in}}%
\pgfpathlineto{\pgfqpoint{3.254249in}{2.239437in}}%
\pgfpathlineto{\pgfqpoint{3.262390in}{2.248432in}}%
\pgfpathlineto{\pgfqpoint{3.270525in}{2.257467in}}%
\pgfpathlineto{\pgfqpoint{3.278653in}{2.266544in}}%
\pgfpathlineto{\pgfqpoint{3.286775in}{2.275661in}}%
\pgfpathlineto{\pgfqpoint{3.273332in}{2.281582in}}%
\pgfpathlineto{\pgfqpoint{3.259893in}{2.287600in}}%
\pgfpathlineto{\pgfqpoint{3.246456in}{2.293716in}}%
\pgfpathlineto{\pgfqpoint{3.233024in}{2.299931in}}%
\pgfpathlineto{\pgfqpoint{3.224890in}{2.290788in}}%
\pgfpathlineto{\pgfqpoint{3.216749in}{2.281693in}}%
\pgfpathlineto{\pgfqpoint{3.208603in}{2.272643in}}%
\pgfpathlineto{\pgfqpoint{3.200449in}{2.263638in}}%
\pgfpathclose%
\pgfusepath{fill}%
\end{pgfscope}%
\begin{pgfscope}%
\pgfpathrectangle{\pgfqpoint{1.150000in}{0.150000in}}{\pgfqpoint{5.700000in}{5.700000in}}%
\pgfusepath{clip}%
\pgfsetbuttcap%
\pgfsetroundjoin%
\definecolor{currentfill}{rgb}{0.269308,0.218818,0.509577}%
\pgfsetfillcolor{currentfill}%
\pgfsetfillopacity{0.700000}%
\pgfsetlinewidth{0.000000pt}%
\definecolor{currentstroke}{rgb}{0.000000,0.000000,0.000000}%
\pgfsetstrokecolor{currentstroke}%
\pgfsetdash{}{0pt}%
\pgfpathmoveto{\pgfqpoint{4.954234in}{2.554549in}}%
\pgfpathlineto{\pgfqpoint{4.968074in}{2.553083in}}%
\pgfpathlineto{\pgfqpoint{4.981923in}{2.551687in}}%
\pgfpathlineto{\pgfqpoint{4.995780in}{2.550361in}}%
\pgfpathlineto{\pgfqpoint{5.009646in}{2.549104in}}%
\pgfpathlineto{\pgfqpoint{5.017180in}{2.557982in}}%
\pgfpathlineto{\pgfqpoint{5.024711in}{2.567002in}}%
\pgfpathlineto{\pgfqpoint{5.032240in}{2.576170in}}%
\pgfpathlineto{\pgfqpoint{5.039767in}{2.585492in}}%
\pgfpathlineto{\pgfqpoint{5.025918in}{2.587092in}}%
\pgfpathlineto{\pgfqpoint{5.012077in}{2.588761in}}%
\pgfpathlineto{\pgfqpoint{4.998246in}{2.590499in}}%
\pgfpathlineto{\pgfqpoint{4.984422in}{2.592308in}}%
\pgfpathlineto{\pgfqpoint{4.976879in}{2.582636in}}%
\pgfpathlineto{\pgfqpoint{4.969333in}{2.573124in}}%
\pgfpathlineto{\pgfqpoint{4.961785in}{2.563763in}}%
\pgfpathlineto{\pgfqpoint{4.954234in}{2.554549in}}%
\pgfpathclose%
\pgfusepath{fill}%
\end{pgfscope}%
\begin{pgfscope}%
\pgfpathrectangle{\pgfqpoint{1.150000in}{0.150000in}}{\pgfqpoint{5.700000in}{5.700000in}}%
\pgfusepath{clip}%
\pgfsetbuttcap%
\pgfsetroundjoin%
\definecolor{currentfill}{rgb}{0.281924,0.089666,0.412415}%
\pgfsetfillcolor{currentfill}%
\pgfsetfillopacity{0.700000}%
\pgfsetlinewidth{0.000000pt}%
\definecolor{currentstroke}{rgb}{0.000000,0.000000,0.000000}%
\pgfsetstrokecolor{currentstroke}%
\pgfsetdash{}{0pt}%
\pgfpathmoveto{\pgfqpoint{2.919803in}{2.306737in}}%
\pgfpathlineto{\pgfqpoint{2.933237in}{2.298839in}}%
\pgfpathlineto{\pgfqpoint{2.946672in}{2.291052in}}%
\pgfpathlineto{\pgfqpoint{2.960109in}{2.283373in}}%
\pgfpathlineto{\pgfqpoint{2.973548in}{2.275803in}}%
\pgfpathlineto{\pgfqpoint{2.981795in}{2.284370in}}%
\pgfpathlineto{\pgfqpoint{2.990035in}{2.292995in}}%
\pgfpathlineto{\pgfqpoint{2.998268in}{2.301676in}}%
\pgfpathlineto{\pgfqpoint{3.006495in}{2.310416in}}%
\pgfpathlineto{\pgfqpoint{2.993070in}{2.317961in}}%
\pgfpathlineto{\pgfqpoint{2.979647in}{2.325616in}}%
\pgfpathlineto{\pgfqpoint{2.966226in}{2.333379in}}%
\pgfpathlineto{\pgfqpoint{2.952806in}{2.341253in}}%
\pgfpathlineto{\pgfqpoint{2.944566in}{2.332530in}}%
\pgfpathlineto{\pgfqpoint{2.936319in}{2.323870in}}%
\pgfpathlineto{\pgfqpoint{2.928065in}{2.315273in}}%
\pgfpathlineto{\pgfqpoint{2.919803in}{2.306737in}}%
\pgfpathclose%
\pgfusepath{fill}%
\end{pgfscope}%
\begin{pgfscope}%
\pgfpathrectangle{\pgfqpoint{1.150000in}{0.150000in}}{\pgfqpoint{5.700000in}{5.700000in}}%
\pgfusepath{clip}%
\pgfsetbuttcap%
\pgfsetroundjoin%
\definecolor{currentfill}{rgb}{0.279566,0.067836,0.391917}%
\pgfsetfillcolor{currentfill}%
\pgfsetfillopacity{0.700000}%
\pgfsetlinewidth{0.000000pt}%
\definecolor{currentstroke}{rgb}{0.000000,0.000000,0.000000}%
\pgfsetstrokecolor{currentstroke}%
\pgfsetdash{}{0pt}%
\pgfpathmoveto{\pgfqpoint{3.340583in}{2.252944in}}%
\pgfpathlineto{\pgfqpoint{3.354044in}{2.247503in}}%
\pgfpathlineto{\pgfqpoint{3.367510in}{2.242156in}}%
\pgfpathlineto{\pgfqpoint{3.380979in}{2.236903in}}%
\pgfpathlineto{\pgfqpoint{3.394453in}{2.231743in}}%
\pgfpathlineto{\pgfqpoint{3.402545in}{2.240844in}}%
\pgfpathlineto{\pgfqpoint{3.410631in}{2.249978in}}%
\pgfpathlineto{\pgfqpoint{3.418712in}{2.259146in}}%
\pgfpathlineto{\pgfqpoint{3.426786in}{2.268351in}}%
\pgfpathlineto{\pgfqpoint{3.413324in}{2.273548in}}%
\pgfpathlineto{\pgfqpoint{3.399866in}{2.278839in}}%
\pgfpathlineto{\pgfqpoint{3.386412in}{2.284223in}}%
\pgfpathlineto{\pgfqpoint{3.372962in}{2.289702in}}%
\pgfpathlineto{\pgfqpoint{3.364876in}{2.280452in}}%
\pgfpathlineto{\pgfqpoint{3.356785in}{2.271244in}}%
\pgfpathlineto{\pgfqpoint{3.348687in}{2.262075in}}%
\pgfpathlineto{\pgfqpoint{3.340583in}{2.252944in}}%
\pgfpathclose%
\pgfusepath{fill}%
\end{pgfscope}%
\begin{pgfscope}%
\pgfpathrectangle{\pgfqpoint{1.150000in}{0.150000in}}{\pgfqpoint{5.700000in}{5.700000in}}%
\pgfusepath{clip}%
\pgfsetbuttcap%
\pgfsetroundjoin%
\definecolor{currentfill}{rgb}{0.248629,0.278775,0.534556}%
\pgfsetfillcolor{currentfill}%
\pgfsetfillopacity{0.700000}%
\pgfsetlinewidth{0.000000pt}%
\definecolor{currentstroke}{rgb}{0.000000,0.000000,0.000000}%
\pgfsetstrokecolor{currentstroke}%
\pgfsetdash{}{0pt}%
\pgfpathmoveto{\pgfqpoint{5.351894in}{2.677092in}}%
\pgfpathlineto{\pgfqpoint{5.365837in}{2.675502in}}%
\pgfpathlineto{\pgfqpoint{5.379790in}{2.673980in}}%
\pgfpathlineto{\pgfqpoint{5.393752in}{2.672525in}}%
\pgfpathlineto{\pgfqpoint{5.407723in}{2.671137in}}%
\pgfpathlineto{\pgfqpoint{5.415145in}{2.680988in}}%
\pgfpathlineto{\pgfqpoint{5.422567in}{2.691066in}}%
\pgfpathlineto{\pgfqpoint{5.429991in}{2.701377in}}%
\pgfpathlineto{\pgfqpoint{5.437416in}{2.711929in}}%
\pgfpathlineto{\pgfqpoint{5.423466in}{2.713740in}}%
\pgfpathlineto{\pgfqpoint{5.409525in}{2.715619in}}%
\pgfpathlineto{\pgfqpoint{5.395593in}{2.717564in}}%
\pgfpathlineto{\pgfqpoint{5.381670in}{2.719576in}}%
\pgfpathlineto{\pgfqpoint{5.374224in}{2.708594in}}%
\pgfpathlineto{\pgfqpoint{5.366779in}{2.697858in}}%
\pgfpathlineto{\pgfqpoint{5.359336in}{2.687360in}}%
\pgfpathlineto{\pgfqpoint{5.351894in}{2.677092in}}%
\pgfpathclose%
\pgfusepath{fill}%
\end{pgfscope}%
\begin{pgfscope}%
\pgfpathrectangle{\pgfqpoint{1.150000in}{0.150000in}}{\pgfqpoint{5.700000in}{5.700000in}}%
\pgfusepath{clip}%
\pgfsetbuttcap%
\pgfsetroundjoin%
\definecolor{currentfill}{rgb}{0.283072,0.130895,0.449241}%
\pgfsetfillcolor{currentfill}%
\pgfsetfillopacity{0.700000}%
\pgfsetlinewidth{0.000000pt}%
\definecolor{currentstroke}{rgb}{0.000000,0.000000,0.000000}%
\pgfsetstrokecolor{currentstroke}%
\pgfsetdash{}{0pt}%
\pgfpathmoveto{\pgfqpoint{4.244822in}{2.368325in}}%
\pgfpathlineto{\pgfqpoint{4.258470in}{2.366065in}}%
\pgfpathlineto{\pgfqpoint{4.272126in}{2.363882in}}%
\pgfpathlineto{\pgfqpoint{4.285790in}{2.361774in}}%
\pgfpathlineto{\pgfqpoint{4.299460in}{2.359743in}}%
\pgfpathlineto{\pgfqpoint{4.307241in}{2.368559in}}%
\pgfpathlineto{\pgfqpoint{4.315017in}{2.377424in}}%
\pgfpathlineto{\pgfqpoint{4.322788in}{2.386342in}}%
\pgfpathlineto{\pgfqpoint{4.330553in}{2.395317in}}%
\pgfpathlineto{\pgfqpoint{4.316894in}{2.397549in}}%
\pgfpathlineto{\pgfqpoint{4.303243in}{2.399856in}}%
\pgfpathlineto{\pgfqpoint{4.289599in}{2.402240in}}%
\pgfpathlineto{\pgfqpoint{4.275962in}{2.404701in}}%
\pgfpathlineto{\pgfqpoint{4.268184in}{2.395518in}}%
\pgfpathlineto{\pgfqpoint{4.260402in}{2.386397in}}%
\pgfpathlineto{\pgfqpoint{4.252614in}{2.377334in}}%
\pgfpathlineto{\pgfqpoint{4.244822in}{2.368325in}}%
\pgfpathclose%
\pgfusepath{fill}%
\end{pgfscope}%
\begin{pgfscope}%
\pgfpathrectangle{\pgfqpoint{1.150000in}{0.150000in}}{\pgfqpoint{5.700000in}{5.700000in}}%
\pgfusepath{clip}%
\pgfsetbuttcap%
\pgfsetroundjoin%
\definecolor{currentfill}{rgb}{0.280255,0.165693,0.476498}%
\pgfsetfillcolor{currentfill}%
\pgfsetfillopacity{0.700000}%
\pgfsetlinewidth{0.000000pt}%
\definecolor{currentstroke}{rgb}{0.000000,0.000000,0.000000}%
\pgfsetstrokecolor{currentstroke}%
\pgfsetdash{}{0pt}%
\pgfpathmoveto{\pgfqpoint{4.556697in}{2.443168in}}%
\pgfpathlineto{\pgfqpoint{4.570430in}{2.441443in}}%
\pgfpathlineto{\pgfqpoint{4.584171in}{2.439791in}}%
\pgfpathlineto{\pgfqpoint{4.597921in}{2.438212in}}%
\pgfpathlineto{\pgfqpoint{4.611678in}{2.436705in}}%
\pgfpathlineto{\pgfqpoint{4.619349in}{2.445361in}}%
\pgfpathlineto{\pgfqpoint{4.627015in}{2.454097in}}%
\pgfpathlineto{\pgfqpoint{4.634677in}{2.462918in}}%
\pgfpathlineto{\pgfqpoint{4.642334in}{2.471828in}}%
\pgfpathlineto{\pgfqpoint{4.628590in}{2.473596in}}%
\pgfpathlineto{\pgfqpoint{4.614855in}{2.475437in}}%
\pgfpathlineto{\pgfqpoint{4.601127in}{2.477351in}}%
\pgfpathlineto{\pgfqpoint{4.587408in}{2.479338in}}%
\pgfpathlineto{\pgfqpoint{4.579737in}{2.470159in}}%
\pgfpathlineto{\pgfqpoint{4.572061in}{2.461074in}}%
\pgfpathlineto{\pgfqpoint{4.564381in}{2.452079in}}%
\pgfpathlineto{\pgfqpoint{4.556697in}{2.443168in}}%
\pgfpathclose%
\pgfusepath{fill}%
\end{pgfscope}%
\begin{pgfscope}%
\pgfpathrectangle{\pgfqpoint{1.150000in}{0.150000in}}{\pgfqpoint{5.700000in}{5.700000in}}%
\pgfusepath{clip}%
\pgfsetbuttcap%
\pgfsetroundjoin%
\definecolor{currentfill}{rgb}{0.280894,0.078907,0.402329}%
\pgfsetfillcolor{currentfill}%
\pgfsetfillopacity{0.700000}%
\pgfsetlinewidth{0.000000pt}%
\definecolor{currentstroke}{rgb}{0.000000,0.000000,0.000000}%
\pgfsetstrokecolor{currentstroke}%
\pgfsetdash{}{0pt}%
\pgfpathmoveto{\pgfqpoint{3.706843in}{2.270379in}}%
\pgfpathlineto{\pgfqpoint{3.720367in}{2.266557in}}%
\pgfpathlineto{\pgfqpoint{3.733896in}{2.262821in}}%
\pgfpathlineto{\pgfqpoint{3.747430in}{2.259169in}}%
\pgfpathlineto{\pgfqpoint{3.760970in}{2.255602in}}%
\pgfpathlineto{\pgfqpoint{3.768938in}{2.264739in}}%
\pgfpathlineto{\pgfqpoint{3.776900in}{2.273903in}}%
\pgfpathlineto{\pgfqpoint{3.784856in}{2.283097in}}%
\pgfpathlineto{\pgfqpoint{3.792807in}{2.292321in}}%
\pgfpathlineto{\pgfqpoint{3.779277in}{2.295987in}}%
\pgfpathlineto{\pgfqpoint{3.765753in}{2.299737in}}%
\pgfpathlineto{\pgfqpoint{3.752235in}{2.303572in}}%
\pgfpathlineto{\pgfqpoint{3.738722in}{2.307493in}}%
\pgfpathlineto{\pgfqpoint{3.730761in}{2.298162in}}%
\pgfpathlineto{\pgfqpoint{3.722794in}{2.288868in}}%
\pgfpathlineto{\pgfqpoint{3.714821in}{2.279608in}}%
\pgfpathlineto{\pgfqpoint{3.706843in}{2.270379in}}%
\pgfpathclose%
\pgfusepath{fill}%
\end{pgfscope}%
\begin{pgfscope}%
\pgfpathrectangle{\pgfqpoint{1.150000in}{0.150000in}}{\pgfqpoint{5.700000in}{5.700000in}}%
\pgfusepath{clip}%
\pgfsetbuttcap%
\pgfsetroundjoin%
\definecolor{currentfill}{rgb}{0.253935,0.265254,0.529983}%
\pgfsetfillcolor{currentfill}%
\pgfsetfillopacity{0.700000}%
\pgfsetlinewidth{0.000000pt}%
\definecolor{currentstroke}{rgb}{0.000000,0.000000,0.000000}%
\pgfsetstrokecolor{currentstroke}%
\pgfsetdash{}{0pt}%
\pgfpathmoveto{\pgfqpoint{5.266366in}{2.643596in}}%
\pgfpathlineto{\pgfqpoint{5.280293in}{2.642140in}}%
\pgfpathlineto{\pgfqpoint{5.294229in}{2.640752in}}%
\pgfpathlineto{\pgfqpoint{5.308175in}{2.639431in}}%
\pgfpathlineto{\pgfqpoint{5.322130in}{2.638178in}}%
\pgfpathlineto{\pgfqpoint{5.329571in}{2.647597in}}%
\pgfpathlineto{\pgfqpoint{5.337011in}{2.657218in}}%
\pgfpathlineto{\pgfqpoint{5.344452in}{2.667047in}}%
\pgfpathlineto{\pgfqpoint{5.351894in}{2.677092in}}%
\pgfpathlineto{\pgfqpoint{5.337959in}{2.678748in}}%
\pgfpathlineto{\pgfqpoint{5.324034in}{2.680472in}}%
\pgfpathlineto{\pgfqpoint{5.310117in}{2.682264in}}%
\pgfpathlineto{\pgfqpoint{5.296209in}{2.684122in}}%
\pgfpathlineto{\pgfqpoint{5.288748in}{2.673667in}}%
\pgfpathlineto{\pgfqpoint{5.281287in}{2.663433in}}%
\pgfpathlineto{\pgfqpoint{5.273826in}{2.653411in}}%
\pgfpathlineto{\pgfqpoint{5.266366in}{2.643596in}}%
\pgfpathclose%
\pgfusepath{fill}%
\end{pgfscope}%
\begin{pgfscope}%
\pgfpathrectangle{\pgfqpoint{1.150000in}{0.150000in}}{\pgfqpoint{5.700000in}{5.700000in}}%
\pgfusepath{clip}%
\pgfsetbuttcap%
\pgfsetroundjoin%
\definecolor{currentfill}{rgb}{0.283091,0.110553,0.431554}%
\pgfsetfillcolor{currentfill}%
\pgfsetfillopacity{0.700000}%
\pgfsetlinewidth{0.000000pt}%
\definecolor{currentstroke}{rgb}{0.000000,0.000000,0.000000}%
\pgfsetstrokecolor{currentstroke}%
\pgfsetdash{}{0pt}%
\pgfpathmoveto{\pgfqpoint{2.779133in}{2.340824in}}%
\pgfpathlineto{\pgfqpoint{2.792574in}{2.331970in}}%
\pgfpathlineto{\pgfqpoint{2.806015in}{2.323234in}}%
\pgfpathlineto{\pgfqpoint{2.819457in}{2.314613in}}%
\pgfpathlineto{\pgfqpoint{2.832899in}{2.306108in}}%
\pgfpathlineto{\pgfqpoint{2.841207in}{2.314341in}}%
\pgfpathlineto{\pgfqpoint{2.849506in}{2.322641in}}%
\pgfpathlineto{\pgfqpoint{2.857798in}{2.331010in}}%
\pgfpathlineto{\pgfqpoint{2.866082in}{2.339446in}}%
\pgfpathlineto{\pgfqpoint{2.852654in}{2.347906in}}%
\pgfpathlineto{\pgfqpoint{2.839228in}{2.356481in}}%
\pgfpathlineto{\pgfqpoint{2.825802in}{2.365173in}}%
\pgfpathlineto{\pgfqpoint{2.812377in}{2.373981in}}%
\pgfpathlineto{\pgfqpoint{2.804078in}{2.365583in}}%
\pgfpathlineto{\pgfqpoint{2.795771in}{2.357257in}}%
\pgfpathlineto{\pgfqpoint{2.787456in}{2.349004in}}%
\pgfpathlineto{\pgfqpoint{2.779133in}{2.340824in}}%
\pgfpathclose%
\pgfusepath{fill}%
\end{pgfscope}%
\begin{pgfscope}%
\pgfpathrectangle{\pgfqpoint{1.150000in}{0.150000in}}{\pgfqpoint{5.700000in}{5.700000in}}%
\pgfusepath{clip}%
\pgfsetbuttcap%
\pgfsetroundjoin%
\definecolor{currentfill}{rgb}{0.271828,0.209303,0.504434}%
\pgfsetfillcolor{currentfill}%
\pgfsetfillopacity{0.700000}%
\pgfsetlinewidth{0.000000pt}%
\definecolor{currentstroke}{rgb}{0.000000,0.000000,0.000000}%
\pgfsetstrokecolor{currentstroke}%
\pgfsetdash{}{0pt}%
\pgfpathmoveto{\pgfqpoint{4.868664in}{2.524311in}}%
\pgfpathlineto{\pgfqpoint{4.882485in}{2.522887in}}%
\pgfpathlineto{\pgfqpoint{4.896316in}{2.521534in}}%
\pgfpathlineto{\pgfqpoint{4.910154in}{2.520251in}}%
\pgfpathlineto{\pgfqpoint{4.924002in}{2.519038in}}%
\pgfpathlineto{\pgfqpoint{4.931565in}{2.527726in}}%
\pgfpathlineto{\pgfqpoint{4.939124in}{2.536537in}}%
\pgfpathlineto{\pgfqpoint{4.946681in}{2.545476in}}%
\pgfpathlineto{\pgfqpoint{4.954234in}{2.554549in}}%
\pgfpathlineto{\pgfqpoint{4.940403in}{2.556085in}}%
\pgfpathlineto{\pgfqpoint{4.926580in}{2.557691in}}%
\pgfpathlineto{\pgfqpoint{4.912766in}{2.559366in}}%
\pgfpathlineto{\pgfqpoint{4.898960in}{2.561112in}}%
\pgfpathlineto{\pgfqpoint{4.891390in}{2.551709in}}%
\pgfpathlineto{\pgfqpoint{4.883818in}{2.542445in}}%
\pgfpathlineto{\pgfqpoint{4.876242in}{2.533314in}}%
\pgfpathlineto{\pgfqpoint{4.868664in}{2.524311in}}%
\pgfpathclose%
\pgfusepath{fill}%
\end{pgfscope}%
\begin{pgfscope}%
\pgfpathrectangle{\pgfqpoint{1.150000in}{0.150000in}}{\pgfqpoint{5.700000in}{5.700000in}}%
\pgfusepath{clip}%
\pgfsetbuttcap%
\pgfsetroundjoin%
\definecolor{currentfill}{rgb}{0.282327,0.094955,0.417331}%
\pgfsetfillcolor{currentfill}%
\pgfsetfillopacity{0.700000}%
\pgfsetlinewidth{0.000000pt}%
\definecolor{currentstroke}{rgb}{0.000000,0.000000,0.000000}%
\pgfsetstrokecolor{currentstroke}%
\pgfsetdash{}{0pt}%
\pgfpathmoveto{\pgfqpoint{3.932918in}{2.302288in}}%
\pgfpathlineto{\pgfqpoint{3.946492in}{2.299246in}}%
\pgfpathlineto{\pgfqpoint{3.960071in}{2.296286in}}%
\pgfpathlineto{\pgfqpoint{3.973658in}{2.293406in}}%
\pgfpathlineto{\pgfqpoint{3.987250in}{2.290607in}}%
\pgfpathlineto{\pgfqpoint{3.995141in}{2.299626in}}%
\pgfpathlineto{\pgfqpoint{4.003026in}{2.308676in}}%
\pgfpathlineto{\pgfqpoint{4.010906in}{2.317759in}}%
\pgfpathlineto{\pgfqpoint{4.018780in}{2.326879in}}%
\pgfpathlineto{\pgfqpoint{4.005198in}{2.329818in}}%
\pgfpathlineto{\pgfqpoint{3.991623in}{2.332837in}}%
\pgfpathlineto{\pgfqpoint{3.978054in}{2.335937in}}%
\pgfpathlineto{\pgfqpoint{3.964491in}{2.339117in}}%
\pgfpathlineto{\pgfqpoint{3.956606in}{2.329851in}}%
\pgfpathlineto{\pgfqpoint{3.948716in}{2.320626in}}%
\pgfpathlineto{\pgfqpoint{3.940820in}{2.311439in}}%
\pgfpathlineto{\pgfqpoint{3.932918in}{2.302288in}}%
\pgfpathclose%
\pgfusepath{fill}%
\end{pgfscope}%
\begin{pgfscope}%
\pgfpathrectangle{\pgfqpoint{1.150000in}{0.150000in}}{\pgfqpoint{5.700000in}{5.700000in}}%
\pgfusepath{clip}%
\pgfsetbuttcap%
\pgfsetroundjoin%
\definecolor{currentfill}{rgb}{0.281887,0.150881,0.465405}%
\pgfsetfillcolor{currentfill}%
\pgfsetfillopacity{0.700000}%
\pgfsetlinewidth{0.000000pt}%
\definecolor{currentstroke}{rgb}{0.000000,0.000000,0.000000}%
\pgfsetstrokecolor{currentstroke}%
\pgfsetdash{}{0pt}%
\pgfpathmoveto{\pgfqpoint{2.584274in}{2.425348in}}%
\pgfpathlineto{\pgfqpoint{2.597734in}{2.414948in}}%
\pgfpathlineto{\pgfqpoint{2.611193in}{2.404678in}}%
\pgfpathlineto{\pgfqpoint{2.624651in}{2.394536in}}%
\pgfpathlineto{\pgfqpoint{2.638109in}{2.384520in}}%
\pgfpathlineto{\pgfqpoint{2.646499in}{2.392257in}}%
\pgfpathlineto{\pgfqpoint{2.654880in}{2.400078in}}%
\pgfpathlineto{\pgfqpoint{2.663253in}{2.407984in}}%
\pgfpathlineto{\pgfqpoint{2.671617in}{2.415975in}}%
\pgfpathlineto{\pgfqpoint{2.658177in}{2.425924in}}%
\pgfpathlineto{\pgfqpoint{2.644736in}{2.436000in}}%
\pgfpathlineto{\pgfqpoint{2.631294in}{2.446203in}}%
\pgfpathlineto{\pgfqpoint{2.617852in}{2.456536in}}%
\pgfpathlineto{\pgfqpoint{2.609471in}{2.448605in}}%
\pgfpathlineto{\pgfqpoint{2.601080in}{2.440763in}}%
\pgfpathlineto{\pgfqpoint{2.592681in}{2.433010in}}%
\pgfpathlineto{\pgfqpoint{2.584274in}{2.425348in}}%
\pgfpathclose%
\pgfusepath{fill}%
\end{pgfscope}%
\begin{pgfscope}%
\pgfpathrectangle{\pgfqpoint{1.150000in}{0.150000in}}{\pgfqpoint{5.700000in}{5.700000in}}%
\pgfusepath{clip}%
\pgfsetbuttcap%
\pgfsetroundjoin%
\definecolor{currentfill}{rgb}{0.279566,0.067836,0.391917}%
\pgfsetfillcolor{currentfill}%
\pgfsetfillopacity{0.700000}%
\pgfsetlinewidth{0.000000pt}%
\definecolor{currentstroke}{rgb}{0.000000,0.000000,0.000000}%
\pgfsetstrokecolor{currentstroke}%
\pgfsetdash{}{0pt}%
\pgfpathmoveto{\pgfqpoint{3.480678in}{2.248482in}}%
\pgfpathlineto{\pgfqpoint{3.494162in}{2.243743in}}%
\pgfpathlineto{\pgfqpoint{3.507651in}{2.239094in}}%
\pgfpathlineto{\pgfqpoint{3.521145in}{2.234535in}}%
\pgfpathlineto{\pgfqpoint{3.534643in}{2.230066in}}%
\pgfpathlineto{\pgfqpoint{3.542689in}{2.239206in}}%
\pgfpathlineto{\pgfqpoint{3.550729in}{2.248375in}}%
\pgfpathlineto{\pgfqpoint{3.558764in}{2.257573in}}%
\pgfpathlineto{\pgfqpoint{3.566792in}{2.266803in}}%
\pgfpathlineto{\pgfqpoint{3.553305in}{2.271331in}}%
\pgfpathlineto{\pgfqpoint{3.539822in}{2.275948in}}%
\pgfpathlineto{\pgfqpoint{3.526344in}{2.280655in}}%
\pgfpathlineto{\pgfqpoint{3.512871in}{2.285452in}}%
\pgfpathlineto{\pgfqpoint{3.504832in}{2.276157in}}%
\pgfpathlineto{\pgfqpoint{3.496786in}{2.266898in}}%
\pgfpathlineto{\pgfqpoint{3.488735in}{2.257673in}}%
\pgfpathlineto{\pgfqpoint{3.480678in}{2.248482in}}%
\pgfpathclose%
\pgfusepath{fill}%
\end{pgfscope}%
\begin{pgfscope}%
\pgfpathrectangle{\pgfqpoint{1.150000in}{0.150000in}}{\pgfqpoint{5.700000in}{5.700000in}}%
\pgfusepath{clip}%
\pgfsetbuttcap%
\pgfsetroundjoin%
\definecolor{currentfill}{rgb}{0.212395,0.359683,0.551710}%
\pgfsetfillcolor{currentfill}%
\pgfsetfillopacity{0.700000}%
\pgfsetlinewidth{0.000000pt}%
\definecolor{currentstroke}{rgb}{0.000000,0.000000,0.000000}%
\pgfsetstrokecolor{currentstroke}%
\pgfsetdash{}{0pt}%
\pgfpathmoveto{\pgfqpoint{5.835859in}{2.859916in}}%
\pgfpathlineto{\pgfqpoint{5.849910in}{2.857592in}}%
\pgfpathlineto{\pgfqpoint{5.863969in}{2.855332in}}%
\pgfpathlineto{\pgfqpoint{5.878039in}{2.853137in}}%
\pgfpathlineto{\pgfqpoint{5.892118in}{2.851007in}}%
\pgfpathlineto{\pgfqpoint{5.899485in}{2.863822in}}%
\pgfpathlineto{\pgfqpoint{5.906862in}{2.877003in}}%
\pgfpathlineto{\pgfqpoint{5.914248in}{2.890558in}}%
\pgfpathlineto{\pgfqpoint{5.921644in}{2.904498in}}%
\pgfpathlineto{\pgfqpoint{5.907591in}{2.907152in}}%
\pgfpathlineto{\pgfqpoint{5.893547in}{2.909870in}}%
\pgfpathlineto{\pgfqpoint{5.879512in}{2.912653in}}%
\pgfpathlineto{\pgfqpoint{5.865486in}{2.915500in}}%
\pgfpathlineto{\pgfqpoint{5.858066in}{2.901030in}}%
\pgfpathlineto{\pgfqpoint{5.850654in}{2.886949in}}%
\pgfpathlineto{\pgfqpoint{5.843252in}{2.873248in}}%
\pgfpathlineto{\pgfqpoint{5.835859in}{2.859916in}}%
\pgfpathclose%
\pgfusepath{fill}%
\end{pgfscope}%
\begin{pgfscope}%
\pgfpathrectangle{\pgfqpoint{1.150000in}{0.150000in}}{\pgfqpoint{5.700000in}{5.700000in}}%
\pgfusepath{clip}%
\pgfsetbuttcap%
\pgfsetroundjoin%
\definecolor{currentfill}{rgb}{0.204903,0.375746,0.553533}%
\pgfsetfillcolor{currentfill}%
\pgfsetfillopacity{0.700000}%
\pgfsetlinewidth{0.000000pt}%
\definecolor{currentstroke}{rgb}{0.000000,0.000000,0.000000}%
\pgfsetstrokecolor{currentstroke}%
\pgfsetdash{}{0pt}%
\pgfpathmoveto{\pgfqpoint{5.921644in}{2.904498in}}%
\pgfpathlineto{\pgfqpoint{5.935706in}{2.901908in}}%
\pgfpathlineto{\pgfqpoint{5.949778in}{2.899383in}}%
\pgfpathlineto{\pgfqpoint{5.963859in}{2.896922in}}%
\pgfpathlineto{\pgfqpoint{5.977949in}{2.894525in}}%
\pgfpathlineto{\pgfqpoint{5.985329in}{2.908323in}}%
\pgfpathlineto{\pgfqpoint{5.992719in}{2.922519in}}%
\pgfpathlineto{\pgfqpoint{6.000122in}{2.937123in}}%
\pgfpathlineto{\pgfqpoint{6.007536in}{2.952147in}}%
\pgfpathlineto{\pgfqpoint{5.993472in}{2.955087in}}%
\pgfpathlineto{\pgfqpoint{5.979417in}{2.958091in}}%
\pgfpathlineto{\pgfqpoint{5.965372in}{2.961159in}}%
\pgfpathlineto{\pgfqpoint{5.951335in}{2.964292in}}%
\pgfpathlineto{\pgfqpoint{5.943895in}{2.948718in}}%
\pgfpathlineto{\pgfqpoint{5.936467in}{2.933568in}}%
\pgfpathlineto{\pgfqpoint{5.929050in}{2.918831in}}%
\pgfpathlineto{\pgfqpoint{5.921644in}{2.904498in}}%
\pgfpathclose%
\pgfusepath{fill}%
\end{pgfscope}%
\begin{pgfscope}%
\pgfpathrectangle{\pgfqpoint{1.150000in}{0.150000in}}{\pgfqpoint{5.700000in}{5.700000in}}%
\pgfusepath{clip}%
\pgfsetbuttcap%
\pgfsetroundjoin%
\definecolor{currentfill}{rgb}{0.281412,0.155834,0.469201}%
\pgfsetfillcolor{currentfill}%
\pgfsetfillopacity{0.700000}%
\pgfsetlinewidth{0.000000pt}%
\definecolor{currentstroke}{rgb}{0.000000,0.000000,0.000000}%
\pgfsetstrokecolor{currentstroke}%
\pgfsetdash{}{0pt}%
\pgfpathmoveto{\pgfqpoint{4.471008in}{2.414952in}}%
\pgfpathlineto{\pgfqpoint{4.484722in}{2.413173in}}%
\pgfpathlineto{\pgfqpoint{4.498445in}{2.411469in}}%
\pgfpathlineto{\pgfqpoint{4.512175in}{2.409838in}}%
\pgfpathlineto{\pgfqpoint{4.525914in}{2.408282in}}%
\pgfpathlineto{\pgfqpoint{4.533617in}{2.416899in}}%
\pgfpathlineto{\pgfqpoint{4.541315in}{2.425583in}}%
\pgfpathlineto{\pgfqpoint{4.549008in}{2.434338in}}%
\pgfpathlineto{\pgfqpoint{4.556697in}{2.443168in}}%
\pgfpathlineto{\pgfqpoint{4.542972in}{2.444967in}}%
\pgfpathlineto{\pgfqpoint{4.529255in}{2.446839in}}%
\pgfpathlineto{\pgfqpoint{4.515545in}{2.448785in}}%
\pgfpathlineto{\pgfqpoint{4.501843in}{2.450804in}}%
\pgfpathlineto{\pgfqpoint{4.494141in}{2.441725in}}%
\pgfpathlineto{\pgfqpoint{4.486435in}{2.432726in}}%
\pgfpathlineto{\pgfqpoint{4.478724in}{2.423803in}}%
\pgfpathlineto{\pgfqpoint{4.471008in}{2.414952in}}%
\pgfpathclose%
\pgfusepath{fill}%
\end{pgfscope}%
\begin{pgfscope}%
\pgfpathrectangle{\pgfqpoint{1.150000in}{0.150000in}}{\pgfqpoint{5.700000in}{5.700000in}}%
\pgfusepath{clip}%
\pgfsetbuttcap%
\pgfsetroundjoin%
\definecolor{currentfill}{rgb}{0.220057,0.343307,0.549413}%
\pgfsetfillcolor{currentfill}%
\pgfsetfillopacity{0.700000}%
\pgfsetlinewidth{0.000000pt}%
\definecolor{currentstroke}{rgb}{0.000000,0.000000,0.000000}%
\pgfsetstrokecolor{currentstroke}%
\pgfsetdash{}{0pt}%
\pgfpathmoveto{\pgfqpoint{5.750154in}{2.818037in}}%
\pgfpathlineto{\pgfqpoint{5.764193in}{2.815957in}}%
\pgfpathlineto{\pgfqpoint{5.778240in}{2.813941in}}%
\pgfpathlineto{\pgfqpoint{5.792297in}{2.811990in}}%
\pgfpathlineto{\pgfqpoint{5.806364in}{2.810105in}}%
\pgfpathlineto{\pgfqpoint{5.813727in}{2.822049in}}%
\pgfpathlineto{\pgfqpoint{5.821096in}{2.834326in}}%
\pgfpathlineto{\pgfqpoint{5.828474in}{2.846945in}}%
\pgfpathlineto{\pgfqpoint{5.835859in}{2.859916in}}%
\pgfpathlineto{\pgfqpoint{5.821817in}{2.862305in}}%
\pgfpathlineto{\pgfqpoint{5.807785in}{2.864759in}}%
\pgfpathlineto{\pgfqpoint{5.793762in}{2.867278in}}%
\pgfpathlineto{\pgfqpoint{5.779748in}{2.869862in}}%
\pgfpathlineto{\pgfqpoint{5.772338in}{2.856381in}}%
\pgfpathlineto{\pgfqpoint{5.764937in}{2.843256in}}%
\pgfpathlineto{\pgfqpoint{5.757542in}{2.830477in}}%
\pgfpathlineto{\pgfqpoint{5.750154in}{2.818037in}}%
\pgfpathclose%
\pgfusepath{fill}%
\end{pgfscope}%
\begin{pgfscope}%
\pgfpathrectangle{\pgfqpoint{1.150000in}{0.150000in}}{\pgfqpoint{5.700000in}{5.700000in}}%
\pgfusepath{clip}%
\pgfsetbuttcap%
\pgfsetroundjoin%
\definecolor{currentfill}{rgb}{0.283229,0.120777,0.440584}%
\pgfsetfillcolor{currentfill}%
\pgfsetfillopacity{0.700000}%
\pgfsetlinewidth{0.000000pt}%
\definecolor{currentstroke}{rgb}{0.000000,0.000000,0.000000}%
\pgfsetstrokecolor{currentstroke}%
\pgfsetdash{}{0pt}%
\pgfpathmoveto{\pgfqpoint{4.159029in}{2.341847in}}%
\pgfpathlineto{\pgfqpoint{4.172661in}{2.339459in}}%
\pgfpathlineto{\pgfqpoint{4.186300in}{2.337148in}}%
\pgfpathlineto{\pgfqpoint{4.199945in}{2.334914in}}%
\pgfpathlineto{\pgfqpoint{4.213598in}{2.332758in}}%
\pgfpathlineto{\pgfqpoint{4.221412in}{2.341586in}}%
\pgfpathlineto{\pgfqpoint{4.229221in}{2.350455in}}%
\pgfpathlineto{\pgfqpoint{4.237024in}{2.359366in}}%
\pgfpathlineto{\pgfqpoint{4.244822in}{2.368325in}}%
\pgfpathlineto{\pgfqpoint{4.231180in}{2.370662in}}%
\pgfpathlineto{\pgfqpoint{4.217546in}{2.373076in}}%
\pgfpathlineto{\pgfqpoint{4.203918in}{2.375567in}}%
\pgfpathlineto{\pgfqpoint{4.190298in}{2.378135in}}%
\pgfpathlineto{\pgfqpoint{4.182489in}{2.368989in}}%
\pgfpathlineto{\pgfqpoint{4.174674in}{2.359895in}}%
\pgfpathlineto{\pgfqpoint{4.166854in}{2.350848in}}%
\pgfpathlineto{\pgfqpoint{4.159029in}{2.341847in}}%
\pgfpathclose%
\pgfusepath{fill}%
\end{pgfscope}%
\begin{pgfscope}%
\pgfpathrectangle{\pgfqpoint{1.150000in}{0.150000in}}{\pgfqpoint{5.700000in}{5.700000in}}%
\pgfusepath{clip}%
\pgfsetbuttcap%
\pgfsetroundjoin%
\definecolor{currentfill}{rgb}{0.195860,0.395433,0.555276}%
\pgfsetfillcolor{currentfill}%
\pgfsetfillopacity{0.700000}%
\pgfsetlinewidth{0.000000pt}%
\definecolor{currentstroke}{rgb}{0.000000,0.000000,0.000000}%
\pgfsetstrokecolor{currentstroke}%
\pgfsetdash{}{0pt}%
\pgfpathmoveto{\pgfqpoint{6.007536in}{2.952147in}}%
\pgfpathlineto{\pgfqpoint{6.021609in}{2.949271in}}%
\pgfpathlineto{\pgfqpoint{6.035691in}{2.946459in}}%
\pgfpathlineto{\pgfqpoint{6.049782in}{2.943710in}}%
\pgfpathlineto{\pgfqpoint{6.063883in}{2.941026in}}%
\pgfpathlineto{\pgfqpoint{6.071283in}{2.955922in}}%
\pgfpathlineto{\pgfqpoint{6.078696in}{2.971252in}}%
\pgfpathlineto{\pgfqpoint{6.086123in}{2.987026in}}%
\pgfpathlineto{\pgfqpoint{6.093564in}{3.003255in}}%
\pgfpathlineto{\pgfqpoint{6.079490in}{3.006502in}}%
\pgfpathlineto{\pgfqpoint{6.065425in}{3.009814in}}%
\pgfpathlineto{\pgfqpoint{6.051369in}{3.013189in}}%
\pgfpathlineto{\pgfqpoint{6.037322in}{3.016628in}}%
\pgfpathlineto{\pgfqpoint{6.029855in}{2.999829in}}%
\pgfpathlineto{\pgfqpoint{6.022402in}{2.983489in}}%
\pgfpathlineto{\pgfqpoint{6.014962in}{2.967598in}}%
\pgfpathlineto{\pgfqpoint{6.007536in}{2.952147in}}%
\pgfpathclose%
\pgfusepath{fill}%
\end{pgfscope}%
\begin{pgfscope}%
\pgfpathrectangle{\pgfqpoint{1.150000in}{0.150000in}}{\pgfqpoint{5.700000in}{5.700000in}}%
\pgfusepath{clip}%
\pgfsetbuttcap%
\pgfsetroundjoin%
\definecolor{currentfill}{rgb}{0.258965,0.251537,0.524736}%
\pgfsetfillcolor{currentfill}%
\pgfsetfillopacity{0.700000}%
\pgfsetlinewidth{0.000000pt}%
\definecolor{currentstroke}{rgb}{0.000000,0.000000,0.000000}%
\pgfsetstrokecolor{currentstroke}%
\pgfsetdash{}{0pt}%
\pgfpathmoveto{\pgfqpoint{5.180821in}{2.611224in}}%
\pgfpathlineto{\pgfqpoint{5.194731in}{2.609880in}}%
\pgfpathlineto{\pgfqpoint{5.208651in}{2.608604in}}%
\pgfpathlineto{\pgfqpoint{5.222579in}{2.607396in}}%
\pgfpathlineto{\pgfqpoint{5.236517in}{2.606256in}}%
\pgfpathlineto{\pgfqpoint{5.243980in}{2.615317in}}%
\pgfpathlineto{\pgfqpoint{5.251443in}{2.624556in}}%
\pgfpathlineto{\pgfqpoint{5.258904in}{2.633980in}}%
\pgfpathlineto{\pgfqpoint{5.266366in}{2.643596in}}%
\pgfpathlineto{\pgfqpoint{5.252447in}{2.645119in}}%
\pgfpathlineto{\pgfqpoint{5.238538in}{2.646711in}}%
\pgfpathlineto{\pgfqpoint{5.224637in}{2.648370in}}%
\pgfpathlineto{\pgfqpoint{5.210746in}{2.650097in}}%
\pgfpathlineto{\pgfqpoint{5.203266in}{2.640091in}}%
\pgfpathlineto{\pgfqpoint{5.195785in}{2.630281in}}%
\pgfpathlineto{\pgfqpoint{5.188303in}{2.620661in}}%
\pgfpathlineto{\pgfqpoint{5.180821in}{2.611224in}}%
\pgfpathclose%
\pgfusepath{fill}%
\end{pgfscope}%
\begin{pgfscope}%
\pgfpathrectangle{\pgfqpoint{1.150000in}{0.150000in}}{\pgfqpoint{5.700000in}{5.700000in}}%
\pgfusepath{clip}%
\pgfsetbuttcap%
\pgfsetroundjoin%
\definecolor{currentfill}{rgb}{0.279566,0.067836,0.391917}%
\pgfsetfillcolor{currentfill}%
\pgfsetfillopacity{0.700000}%
\pgfsetlinewidth{0.000000pt}%
\definecolor{currentstroke}{rgb}{0.000000,0.000000,0.000000}%
\pgfsetstrokecolor{currentstroke}%
\pgfsetdash{}{0pt}%
\pgfpathmoveto{\pgfqpoint{3.113971in}{2.253863in}}%
\pgfpathlineto{\pgfqpoint{3.127416in}{2.247260in}}%
\pgfpathlineto{\pgfqpoint{3.140865in}{2.240759in}}%
\pgfpathlineto{\pgfqpoint{3.154317in}{2.234358in}}%
\pgfpathlineto{\pgfqpoint{3.167771in}{2.228058in}}%
\pgfpathlineto{\pgfqpoint{3.175950in}{2.236888in}}%
\pgfpathlineto{\pgfqpoint{3.184123in}{2.245761in}}%
\pgfpathlineto{\pgfqpoint{3.192290in}{2.254677in}}%
\pgfpathlineto{\pgfqpoint{3.200449in}{2.263638in}}%
\pgfpathlineto{\pgfqpoint{3.187008in}{2.269935in}}%
\pgfpathlineto{\pgfqpoint{3.173569in}{2.276333in}}%
\pgfpathlineto{\pgfqpoint{3.160133in}{2.282831in}}%
\pgfpathlineto{\pgfqpoint{3.146700in}{2.289430in}}%
\pgfpathlineto{\pgfqpoint{3.138528in}{2.280465in}}%
\pgfpathlineto{\pgfqpoint{3.130349in}{2.271550in}}%
\pgfpathlineto{\pgfqpoint{3.122163in}{2.262682in}}%
\pgfpathlineto{\pgfqpoint{3.113971in}{2.253863in}}%
\pgfpathclose%
\pgfusepath{fill}%
\end{pgfscope}%
\begin{pgfscope}%
\pgfpathrectangle{\pgfqpoint{1.150000in}{0.150000in}}{\pgfqpoint{5.700000in}{5.700000in}}%
\pgfusepath{clip}%
\pgfsetbuttcap%
\pgfsetroundjoin%
\definecolor{currentfill}{rgb}{0.275191,0.194905,0.496005}%
\pgfsetfillcolor{currentfill}%
\pgfsetfillopacity{0.700000}%
\pgfsetlinewidth{0.000000pt}%
\definecolor{currentstroke}{rgb}{0.000000,0.000000,0.000000}%
\pgfsetstrokecolor{currentstroke}%
\pgfsetdash{}{0pt}%
\pgfpathmoveto{\pgfqpoint{4.783050in}{2.494654in}}%
\pgfpathlineto{\pgfqpoint{4.796853in}{2.493250in}}%
\pgfpathlineto{\pgfqpoint{4.810665in}{2.491917in}}%
\pgfpathlineto{\pgfqpoint{4.824485in}{2.490655in}}%
\pgfpathlineto{\pgfqpoint{4.838314in}{2.489463in}}%
\pgfpathlineto{\pgfqpoint{4.845907in}{2.498012in}}%
\pgfpathlineto{\pgfqpoint{4.853496in}{2.506665in}}%
\pgfpathlineto{\pgfqpoint{4.861081in}{2.515430in}}%
\pgfpathlineto{\pgfqpoint{4.868664in}{2.524311in}}%
\pgfpathlineto{\pgfqpoint{4.854850in}{2.525805in}}%
\pgfpathlineto{\pgfqpoint{4.841046in}{2.527370in}}%
\pgfpathlineto{\pgfqpoint{4.827249in}{2.529005in}}%
\pgfpathlineto{\pgfqpoint{4.813461in}{2.530712in}}%
\pgfpathlineto{\pgfqpoint{4.805864in}{2.521521in}}%
\pgfpathlineto{\pgfqpoint{4.798263in}{2.512451in}}%
\pgfpathlineto{\pgfqpoint{4.790658in}{2.503497in}}%
\pgfpathlineto{\pgfqpoint{4.783050in}{2.494654in}}%
\pgfpathclose%
\pgfusepath{fill}%
\end{pgfscope}%
\begin{pgfscope}%
\pgfpathrectangle{\pgfqpoint{1.150000in}{0.150000in}}{\pgfqpoint{5.700000in}{5.700000in}}%
\pgfusepath{clip}%
\pgfsetbuttcap%
\pgfsetroundjoin%
\definecolor{currentfill}{rgb}{0.227802,0.326594,0.546532}%
\pgfsetfillcolor{currentfill}%
\pgfsetfillopacity{0.700000}%
\pgfsetlinewidth{0.000000pt}%
\definecolor{currentstroke}{rgb}{0.000000,0.000000,0.000000}%
\pgfsetstrokecolor{currentstroke}%
\pgfsetdash{}{0pt}%
\pgfpathmoveto{\pgfqpoint{5.664508in}{2.778522in}}%
\pgfpathlineto{\pgfqpoint{5.678533in}{2.776663in}}%
\pgfpathlineto{\pgfqpoint{5.692567in}{2.774870in}}%
\pgfpathlineto{\pgfqpoint{5.706611in}{2.773143in}}%
\pgfpathlineto{\pgfqpoint{5.720664in}{2.771481in}}%
\pgfpathlineto{\pgfqpoint{5.728028in}{2.782657in}}%
\pgfpathlineto{\pgfqpoint{5.735398in}{2.794136in}}%
\pgfpathlineto{\pgfqpoint{5.742773in}{2.805926in}}%
\pgfpathlineto{\pgfqpoint{5.750154in}{2.818037in}}%
\pgfpathlineto{\pgfqpoint{5.736125in}{2.820183in}}%
\pgfpathlineto{\pgfqpoint{5.722106in}{2.822394in}}%
\pgfpathlineto{\pgfqpoint{5.708095in}{2.824671in}}%
\pgfpathlineto{\pgfqpoint{5.694094in}{2.827013in}}%
\pgfpathlineto{\pgfqpoint{5.686689in}{2.814411in}}%
\pgfpathlineto{\pgfqpoint{5.679289in}{2.802134in}}%
\pgfpathlineto{\pgfqpoint{5.671896in}{2.790174in}}%
\pgfpathlineto{\pgfqpoint{5.664508in}{2.778522in}}%
\pgfpathclose%
\pgfusepath{fill}%
\end{pgfscope}%
\begin{pgfscope}%
\pgfpathrectangle{\pgfqpoint{1.150000in}{0.150000in}}{\pgfqpoint{5.700000in}{5.700000in}}%
\pgfusepath{clip}%
\pgfsetbuttcap%
\pgfsetroundjoin%
\definecolor{currentfill}{rgb}{0.280894,0.078907,0.402329}%
\pgfsetfillcolor{currentfill}%
\pgfsetfillopacity{0.700000}%
\pgfsetlinewidth{0.000000pt}%
\definecolor{currentstroke}{rgb}{0.000000,0.000000,0.000000}%
\pgfsetstrokecolor{currentstroke}%
\pgfsetdash{}{0pt}%
\pgfpathmoveto{\pgfqpoint{2.973548in}{2.275803in}}%
\pgfpathlineto{\pgfqpoint{2.986988in}{2.268341in}}%
\pgfpathlineto{\pgfqpoint{3.000431in}{2.260986in}}%
\pgfpathlineto{\pgfqpoint{3.013875in}{2.253737in}}%
\pgfpathlineto{\pgfqpoint{3.027322in}{2.246593in}}%
\pgfpathlineto{\pgfqpoint{3.035556in}{2.255192in}}%
\pgfpathlineto{\pgfqpoint{3.043782in}{2.263842in}}%
\pgfpathlineto{\pgfqpoint{3.052001in}{2.272545in}}%
\pgfpathlineto{\pgfqpoint{3.060214in}{2.281301in}}%
\pgfpathlineto{\pgfqpoint{3.046781in}{2.288421in}}%
\pgfpathlineto{\pgfqpoint{3.033350in}{2.295646in}}%
\pgfpathlineto{\pgfqpoint{3.019921in}{2.302977in}}%
\pgfpathlineto{\pgfqpoint{3.006495in}{2.310416in}}%
\pgfpathlineto{\pgfqpoint{2.998268in}{2.301676in}}%
\pgfpathlineto{\pgfqpoint{2.990035in}{2.292995in}}%
\pgfpathlineto{\pgfqpoint{2.981795in}{2.284370in}}%
\pgfpathlineto{\pgfqpoint{2.973548in}{2.275803in}}%
\pgfpathclose%
\pgfusepath{fill}%
\end{pgfscope}%
\begin{pgfscope}%
\pgfpathrectangle{\pgfqpoint{1.150000in}{0.150000in}}{\pgfqpoint{5.700000in}{5.700000in}}%
\pgfusepath{clip}%
\pgfsetbuttcap%
\pgfsetroundjoin%
\definecolor{currentfill}{rgb}{0.278791,0.062145,0.386592}%
\pgfsetfillcolor{currentfill}%
\pgfsetfillopacity{0.700000}%
\pgfsetlinewidth{0.000000pt}%
\definecolor{currentstroke}{rgb}{0.000000,0.000000,0.000000}%
\pgfsetstrokecolor{currentstroke}%
\pgfsetdash{}{0pt}%
\pgfpathmoveto{\pgfqpoint{3.254249in}{2.239437in}}%
\pgfpathlineto{\pgfqpoint{3.267707in}{2.233630in}}%
\pgfpathlineto{\pgfqpoint{3.281169in}{2.227921in}}%
\pgfpathlineto{\pgfqpoint{3.294635in}{2.222307in}}%
\pgfpathlineto{\pgfqpoint{3.308104in}{2.216788in}}%
\pgfpathlineto{\pgfqpoint{3.316233in}{2.225774in}}%
\pgfpathlineto{\pgfqpoint{3.324356in}{2.234794in}}%
\pgfpathlineto{\pgfqpoint{3.332472in}{2.243851in}}%
\pgfpathlineto{\pgfqpoint{3.340583in}{2.252944in}}%
\pgfpathlineto{\pgfqpoint{3.327125in}{2.258480in}}%
\pgfpathlineto{\pgfqpoint{3.313671in}{2.264111in}}%
\pgfpathlineto{\pgfqpoint{3.300221in}{2.269838in}}%
\pgfpathlineto{\pgfqpoint{3.286775in}{2.275661in}}%
\pgfpathlineto{\pgfqpoint{3.278653in}{2.266544in}}%
\pgfpathlineto{\pgfqpoint{3.270525in}{2.257467in}}%
\pgfpathlineto{\pgfqpoint{3.262390in}{2.248432in}}%
\pgfpathlineto{\pgfqpoint{3.254249in}{2.239437in}}%
\pgfpathclose%
\pgfusepath{fill}%
\end{pgfscope}%
\begin{pgfscope}%
\pgfpathrectangle{\pgfqpoint{1.150000in}{0.150000in}}{\pgfqpoint{5.700000in}{5.700000in}}%
\pgfusepath{clip}%
\pgfsetbuttcap%
\pgfsetroundjoin%
\definecolor{currentfill}{rgb}{0.280267,0.073417,0.397163}%
\pgfsetfillcolor{currentfill}%
\pgfsetfillopacity{0.700000}%
\pgfsetlinewidth{0.000000pt}%
\definecolor{currentstroke}{rgb}{0.000000,0.000000,0.000000}%
\pgfsetstrokecolor{currentstroke}%
\pgfsetdash{}{0pt}%
\pgfpathmoveto{\pgfqpoint{3.620791in}{2.249576in}}%
\pgfpathlineto{\pgfqpoint{3.634304in}{2.245488in}}%
\pgfpathlineto{\pgfqpoint{3.647822in}{2.241487in}}%
\pgfpathlineto{\pgfqpoint{3.661345in}{2.237572in}}%
\pgfpathlineto{\pgfqpoint{3.674873in}{2.233744in}}%
\pgfpathlineto{\pgfqpoint{3.682874in}{2.242864in}}%
\pgfpathlineto{\pgfqpoint{3.690870in}{2.252009in}}%
\pgfpathlineto{\pgfqpoint{3.698859in}{2.261180in}}%
\pgfpathlineto{\pgfqpoint{3.706843in}{2.270379in}}%
\pgfpathlineto{\pgfqpoint{3.693325in}{2.274286in}}%
\pgfpathlineto{\pgfqpoint{3.679813in}{2.278279in}}%
\pgfpathlineto{\pgfqpoint{3.666306in}{2.282359in}}%
\pgfpathlineto{\pgfqpoint{3.652804in}{2.286525in}}%
\pgfpathlineto{\pgfqpoint{3.644809in}{2.277240in}}%
\pgfpathlineto{\pgfqpoint{3.636809in}{2.267988in}}%
\pgfpathlineto{\pgfqpoint{3.628803in}{2.258767in}}%
\pgfpathlineto{\pgfqpoint{3.620791in}{2.249576in}}%
\pgfpathclose%
\pgfusepath{fill}%
\end{pgfscope}%
\begin{pgfscope}%
\pgfpathrectangle{\pgfqpoint{1.150000in}{0.150000in}}{\pgfqpoint{5.700000in}{5.700000in}}%
\pgfusepath{clip}%
\pgfsetbuttcap%
\pgfsetroundjoin%
\definecolor{currentfill}{rgb}{0.281924,0.089666,0.412415}%
\pgfsetfillcolor{currentfill}%
\pgfsetfillopacity{0.700000}%
\pgfsetlinewidth{0.000000pt}%
\definecolor{currentstroke}{rgb}{0.000000,0.000000,0.000000}%
\pgfsetstrokecolor{currentstroke}%
\pgfsetdash{}{0pt}%
\pgfpathmoveto{\pgfqpoint{3.846984in}{2.278493in}}%
\pgfpathlineto{\pgfqpoint{3.860543in}{2.275244in}}%
\pgfpathlineto{\pgfqpoint{3.874108in}{2.272076in}}%
\pgfpathlineto{\pgfqpoint{3.887679in}{2.268991in}}%
\pgfpathlineto{\pgfqpoint{3.901257in}{2.265988in}}%
\pgfpathlineto{\pgfqpoint{3.909181in}{2.275022in}}%
\pgfpathlineto{\pgfqpoint{3.917099in}{2.284082in}}%
\pgfpathlineto{\pgfqpoint{3.925011in}{2.293170in}}%
\pgfpathlineto{\pgfqpoint{3.932918in}{2.302288in}}%
\pgfpathlineto{\pgfqpoint{3.919351in}{2.305410in}}%
\pgfpathlineto{\pgfqpoint{3.905791in}{2.308615in}}%
\pgfpathlineto{\pgfqpoint{3.892236in}{2.311901in}}%
\pgfpathlineto{\pgfqpoint{3.878687in}{2.315270in}}%
\pgfpathlineto{\pgfqpoint{3.870770in}{2.306025in}}%
\pgfpathlineto{\pgfqpoint{3.862847in}{2.296815in}}%
\pgfpathlineto{\pgfqpoint{3.854918in}{2.287639in}}%
\pgfpathlineto{\pgfqpoint{3.846984in}{2.278493in}}%
\pgfpathclose%
\pgfusepath{fill}%
\end{pgfscope}%
\begin{pgfscope}%
\pgfpathrectangle{\pgfqpoint{1.150000in}{0.150000in}}{\pgfqpoint{5.700000in}{5.700000in}}%
\pgfusepath{clip}%
\pgfsetbuttcap%
\pgfsetroundjoin%
\definecolor{currentfill}{rgb}{0.187231,0.414746,0.556547}%
\pgfsetfillcolor{currentfill}%
\pgfsetfillopacity{0.700000}%
\pgfsetlinewidth{0.000000pt}%
\definecolor{currentstroke}{rgb}{0.000000,0.000000,0.000000}%
\pgfsetstrokecolor{currentstroke}%
\pgfsetdash{}{0pt}%
\pgfpathmoveto{\pgfqpoint{6.093564in}{3.003255in}}%
\pgfpathlineto{\pgfqpoint{6.107647in}{3.000071in}}%
\pgfpathlineto{\pgfqpoint{6.121739in}{2.996950in}}%
\pgfpathlineto{\pgfqpoint{6.135840in}{2.993894in}}%
\pgfpathlineto{\pgfqpoint{6.149950in}{2.990901in}}%
\pgfpathlineto{\pgfqpoint{6.157379in}{3.007018in}}%
\pgfpathlineto{\pgfqpoint{6.164823in}{3.023607in}}%
\pgfpathlineto{\pgfqpoint{6.172283in}{3.040677in}}%
\pgfpathlineto{\pgfqpoint{6.158194in}{3.044106in}}%
\pgfpathlineto{\pgfqpoint{6.144113in}{3.047599in}}%
\pgfpathlineto{\pgfqpoint{6.130041in}{3.051155in}}%
\pgfpathlineto{\pgfqpoint{6.115978in}{3.054774in}}%
\pgfpathlineto{\pgfqpoint{6.108491in}{3.037118in}}%
\pgfpathlineto{\pgfqpoint{6.101020in}{3.019948in}}%
\pgfpathlineto{\pgfqpoint{6.093564in}{3.003255in}}%
\pgfpathclose%
\pgfusepath{fill}%
\end{pgfscope}%
\begin{pgfscope}%
\pgfpathrectangle{\pgfqpoint{1.150000in}{0.150000in}}{\pgfqpoint{5.700000in}{5.700000in}}%
\pgfusepath{clip}%
\pgfsetbuttcap%
\pgfsetroundjoin%
\definecolor{currentfill}{rgb}{0.282884,0.135920,0.453427}%
\pgfsetfillcolor{currentfill}%
\pgfsetfillopacity{0.700000}%
\pgfsetlinewidth{0.000000pt}%
\definecolor{currentstroke}{rgb}{0.000000,0.000000,0.000000}%
\pgfsetstrokecolor{currentstroke}%
\pgfsetdash{}{0pt}%
\pgfpathmoveto{\pgfqpoint{2.638109in}{2.384520in}}%
\pgfpathlineto{\pgfqpoint{2.651566in}{2.374631in}}%
\pgfpathlineto{\pgfqpoint{2.665023in}{2.364866in}}%
\pgfpathlineto{\pgfqpoint{2.678479in}{2.355225in}}%
\pgfpathlineto{\pgfqpoint{2.691935in}{2.345707in}}%
\pgfpathlineto{\pgfqpoint{2.700308in}{2.353517in}}%
\pgfpathlineto{\pgfqpoint{2.708672in}{2.361407in}}%
\pgfpathlineto{\pgfqpoint{2.717027in}{2.369377in}}%
\pgfpathlineto{\pgfqpoint{2.725375in}{2.377426in}}%
\pgfpathlineto{\pgfqpoint{2.711935in}{2.386878in}}%
\pgfpathlineto{\pgfqpoint{2.698496in}{2.396453in}}%
\pgfpathlineto{\pgfqpoint{2.685057in}{2.406151in}}%
\pgfpathlineto{\pgfqpoint{2.671617in}{2.415975in}}%
\pgfpathlineto{\pgfqpoint{2.663253in}{2.407984in}}%
\pgfpathlineto{\pgfqpoint{2.654880in}{2.400078in}}%
\pgfpathlineto{\pgfqpoint{2.646499in}{2.392257in}}%
\pgfpathlineto{\pgfqpoint{2.638109in}{2.384520in}}%
\pgfpathclose%
\pgfusepath{fill}%
\end{pgfscope}%
\begin{pgfscope}%
\pgfpathrectangle{\pgfqpoint{1.150000in}{0.150000in}}{\pgfqpoint{5.700000in}{5.700000in}}%
\pgfusepath{clip}%
\pgfsetbuttcap%
\pgfsetroundjoin%
\definecolor{currentfill}{rgb}{0.235526,0.309527,0.542944}%
\pgfsetfillcolor{currentfill}%
\pgfsetfillopacity{0.700000}%
\pgfsetlinewidth{0.000000pt}%
\definecolor{currentstroke}{rgb}{0.000000,0.000000,0.000000}%
\pgfsetstrokecolor{currentstroke}%
\pgfsetdash{}{0pt}%
\pgfpathmoveto{\pgfqpoint{5.578898in}{2.741055in}}%
\pgfpathlineto{\pgfqpoint{5.592909in}{2.739397in}}%
\pgfpathlineto{\pgfqpoint{5.606929in}{2.737806in}}%
\pgfpathlineto{\pgfqpoint{5.620959in}{2.736280in}}%
\pgfpathlineto{\pgfqpoint{5.634998in}{2.734820in}}%
\pgfpathlineto{\pgfqpoint{5.642370in}{2.745326in}}%
\pgfpathlineto{\pgfqpoint{5.649745in}{2.756106in}}%
\pgfpathlineto{\pgfqpoint{5.657124in}{2.767168in}}%
\pgfpathlineto{\pgfqpoint{5.664508in}{2.778522in}}%
\pgfpathlineto{\pgfqpoint{5.650492in}{2.780446in}}%
\pgfpathlineto{\pgfqpoint{5.636485in}{2.782436in}}%
\pgfpathlineto{\pgfqpoint{5.622488in}{2.784491in}}%
\pgfpathlineto{\pgfqpoint{5.608500in}{2.786612in}}%
\pgfpathlineto{\pgfqpoint{5.601093in}{2.774788in}}%
\pgfpathlineto{\pgfqpoint{5.593691in}{2.763259in}}%
\pgfpathlineto{\pgfqpoint{5.586293in}{2.752018in}}%
\pgfpathlineto{\pgfqpoint{5.578898in}{2.741055in}}%
\pgfpathclose%
\pgfusepath{fill}%
\end{pgfscope}%
\begin{pgfscope}%
\pgfpathrectangle{\pgfqpoint{1.150000in}{0.150000in}}{\pgfqpoint{5.700000in}{5.700000in}}%
\pgfusepath{clip}%
\pgfsetbuttcap%
\pgfsetroundjoin%
\definecolor{currentfill}{rgb}{0.263663,0.237631,0.518762}%
\pgfsetfillcolor{currentfill}%
\pgfsetfillopacity{0.700000}%
\pgfsetlinewidth{0.000000pt}%
\definecolor{currentstroke}{rgb}{0.000000,0.000000,0.000000}%
\pgfsetstrokecolor{currentstroke}%
\pgfsetdash{}{0pt}%
\pgfpathmoveto{\pgfqpoint{5.095250in}{2.579784in}}%
\pgfpathlineto{\pgfqpoint{5.109143in}{2.578529in}}%
\pgfpathlineto{\pgfqpoint{5.123045in}{2.577343in}}%
\pgfpathlineto{\pgfqpoint{5.136956in}{2.576225in}}%
\pgfpathlineto{\pgfqpoint{5.150876in}{2.575177in}}%
\pgfpathlineto{\pgfqpoint{5.158365in}{2.583947in}}%
\pgfpathlineto{\pgfqpoint{5.165852in}{2.592874in}}%
\pgfpathlineto{\pgfqpoint{5.173337in}{2.601964in}}%
\pgfpathlineto{\pgfqpoint{5.180821in}{2.611224in}}%
\pgfpathlineto{\pgfqpoint{5.166919in}{2.612637in}}%
\pgfpathlineto{\pgfqpoint{5.153027in}{2.614118in}}%
\pgfpathlineto{\pgfqpoint{5.139143in}{2.615667in}}%
\pgfpathlineto{\pgfqpoint{5.125268in}{2.617285in}}%
\pgfpathlineto{\pgfqpoint{5.117766in}{2.607654in}}%
\pgfpathlineto{\pgfqpoint{5.110262in}{2.598198in}}%
\pgfpathlineto{\pgfqpoint{5.102757in}{2.588910in}}%
\pgfpathlineto{\pgfqpoint{5.095250in}{2.579784in}}%
\pgfpathclose%
\pgfusepath{fill}%
\end{pgfscope}%
\begin{pgfscope}%
\pgfpathrectangle{\pgfqpoint{1.150000in}{0.150000in}}{\pgfqpoint{5.700000in}{5.700000in}}%
\pgfusepath{clip}%
\pgfsetbuttcap%
\pgfsetroundjoin%
\definecolor{currentfill}{rgb}{0.282290,0.145912,0.461510}%
\pgfsetfillcolor{currentfill}%
\pgfsetfillopacity{0.700000}%
\pgfsetlinewidth{0.000000pt}%
\definecolor{currentstroke}{rgb}{0.000000,0.000000,0.000000}%
\pgfsetstrokecolor{currentstroke}%
\pgfsetdash{}{0pt}%
\pgfpathmoveto{\pgfqpoint{4.385263in}{2.387146in}}%
\pgfpathlineto{\pgfqpoint{4.398959in}{2.385291in}}%
\pgfpathlineto{\pgfqpoint{4.412663in}{2.383511in}}%
\pgfpathlineto{\pgfqpoint{4.426375in}{2.381806in}}%
\pgfpathlineto{\pgfqpoint{4.440095in}{2.380175in}}%
\pgfpathlineto{\pgfqpoint{4.447830in}{2.388783in}}%
\pgfpathlineto{\pgfqpoint{4.455561in}{2.397446in}}%
\pgfpathlineto{\pgfqpoint{4.463287in}{2.406167in}}%
\pgfpathlineto{\pgfqpoint{4.471008in}{2.414952in}}%
\pgfpathlineto{\pgfqpoint{4.457301in}{2.416804in}}%
\pgfpathlineto{\pgfqpoint{4.443602in}{2.418731in}}%
\pgfpathlineto{\pgfqpoint{4.429910in}{2.420732in}}%
\pgfpathlineto{\pgfqpoint{4.416226in}{2.422807in}}%
\pgfpathlineto{\pgfqpoint{4.408493in}{2.413794in}}%
\pgfpathlineto{\pgfqpoint{4.400754in}{2.404849in}}%
\pgfpathlineto{\pgfqpoint{4.393011in}{2.395968in}}%
\pgfpathlineto{\pgfqpoint{4.385263in}{2.387146in}}%
\pgfpathclose%
\pgfusepath{fill}%
\end{pgfscope}%
\begin{pgfscope}%
\pgfpathrectangle{\pgfqpoint{1.150000in}{0.150000in}}{\pgfqpoint{5.700000in}{5.700000in}}%
\pgfusepath{clip}%
\pgfsetbuttcap%
\pgfsetroundjoin%
\definecolor{currentfill}{rgb}{0.282656,0.100196,0.422160}%
\pgfsetfillcolor{currentfill}%
\pgfsetfillopacity{0.700000}%
\pgfsetlinewidth{0.000000pt}%
\definecolor{currentstroke}{rgb}{0.000000,0.000000,0.000000}%
\pgfsetstrokecolor{currentstroke}%
\pgfsetdash{}{0pt}%
\pgfpathmoveto{\pgfqpoint{2.832899in}{2.306108in}}%
\pgfpathlineto{\pgfqpoint{2.846343in}{2.297717in}}%
\pgfpathlineto{\pgfqpoint{2.859788in}{2.289440in}}%
\pgfpathlineto{\pgfqpoint{2.873234in}{2.281275in}}%
\pgfpathlineto{\pgfqpoint{2.886681in}{2.273221in}}%
\pgfpathlineto{\pgfqpoint{2.894973in}{2.281506in}}%
\pgfpathlineto{\pgfqpoint{2.903257in}{2.289854in}}%
\pgfpathlineto{\pgfqpoint{2.911534in}{2.298264in}}%
\pgfpathlineto{\pgfqpoint{2.919803in}{2.306737in}}%
\pgfpathlineto{\pgfqpoint{2.906371in}{2.314746in}}%
\pgfpathlineto{\pgfqpoint{2.892940in}{2.322866in}}%
\pgfpathlineto{\pgfqpoint{2.879510in}{2.331099in}}%
\pgfpathlineto{\pgfqpoint{2.866082in}{2.339446in}}%
\pgfpathlineto{\pgfqpoint{2.857798in}{2.331010in}}%
\pgfpathlineto{\pgfqpoint{2.849506in}{2.322641in}}%
\pgfpathlineto{\pgfqpoint{2.841207in}{2.314341in}}%
\pgfpathlineto{\pgfqpoint{2.832899in}{2.306108in}}%
\pgfpathclose%
\pgfusepath{fill}%
\end{pgfscope}%
\begin{pgfscope}%
\pgfpathrectangle{\pgfqpoint{1.150000in}{0.150000in}}{\pgfqpoint{5.700000in}{5.700000in}}%
\pgfusepath{clip}%
\pgfsetbuttcap%
\pgfsetroundjoin%
\definecolor{currentfill}{rgb}{0.278791,0.062145,0.386592}%
\pgfsetfillcolor{currentfill}%
\pgfsetfillopacity{0.700000}%
\pgfsetlinewidth{0.000000pt}%
\definecolor{currentstroke}{rgb}{0.000000,0.000000,0.000000}%
\pgfsetstrokecolor{currentstroke}%
\pgfsetdash{}{0pt}%
\pgfpathmoveto{\pgfqpoint{3.394453in}{2.231743in}}%
\pgfpathlineto{\pgfqpoint{3.407930in}{2.226676in}}%
\pgfpathlineto{\pgfqpoint{3.421412in}{2.221701in}}%
\pgfpathlineto{\pgfqpoint{3.434899in}{2.216818in}}%
\pgfpathlineto{\pgfqpoint{3.448390in}{2.212026in}}%
\pgfpathlineto{\pgfqpoint{3.456471in}{2.221096in}}%
\pgfpathlineto{\pgfqpoint{3.464546in}{2.230195in}}%
\pgfpathlineto{\pgfqpoint{3.472615in}{2.239323in}}%
\pgfpathlineto{\pgfqpoint{3.480678in}{2.248482in}}%
\pgfpathlineto{\pgfqpoint{3.467198in}{2.253312in}}%
\pgfpathlineto{\pgfqpoint{3.453723in}{2.258233in}}%
\pgfpathlineto{\pgfqpoint{3.440253in}{2.263246in}}%
\pgfpathlineto{\pgfqpoint{3.426786in}{2.268351in}}%
\pgfpathlineto{\pgfqpoint{3.418712in}{2.259146in}}%
\pgfpathlineto{\pgfqpoint{3.410631in}{2.249978in}}%
\pgfpathlineto{\pgfqpoint{3.402545in}{2.240844in}}%
\pgfpathlineto{\pgfqpoint{3.394453in}{2.231743in}}%
\pgfpathclose%
\pgfusepath{fill}%
\end{pgfscope}%
\begin{pgfscope}%
\pgfpathrectangle{\pgfqpoint{1.150000in}{0.150000in}}{\pgfqpoint{5.700000in}{5.700000in}}%
\pgfusepath{clip}%
\pgfsetbuttcap%
\pgfsetroundjoin%
\definecolor{currentfill}{rgb}{0.283091,0.110553,0.431554}%
\pgfsetfillcolor{currentfill}%
\pgfsetfillopacity{0.700000}%
\pgfsetlinewidth{0.000000pt}%
\definecolor{currentstroke}{rgb}{0.000000,0.000000,0.000000}%
\pgfsetstrokecolor{currentstroke}%
\pgfsetdash{}{0pt}%
\pgfpathmoveto{\pgfqpoint{4.073173in}{2.315921in}}%
\pgfpathlineto{\pgfqpoint{4.086788in}{2.313380in}}%
\pgfpathlineto{\pgfqpoint{4.100410in}{2.310917in}}%
\pgfpathlineto{\pgfqpoint{4.114039in}{2.308533in}}%
\pgfpathlineto{\pgfqpoint{4.127675in}{2.306227in}}%
\pgfpathlineto{\pgfqpoint{4.135521in}{2.315080in}}%
\pgfpathlineto{\pgfqpoint{4.143363in}{2.323966in}}%
\pgfpathlineto{\pgfqpoint{4.151199in}{2.332887in}}%
\pgfpathlineto{\pgfqpoint{4.159029in}{2.341847in}}%
\pgfpathlineto{\pgfqpoint{4.145405in}{2.344313in}}%
\pgfpathlineto{\pgfqpoint{4.131787in}{2.346858in}}%
\pgfpathlineto{\pgfqpoint{4.118176in}{2.349481in}}%
\pgfpathlineto{\pgfqpoint{4.104572in}{2.352182in}}%
\pgfpathlineto{\pgfqpoint{4.096730in}{2.343055in}}%
\pgfpathlineto{\pgfqpoint{4.088883in}{2.333971in}}%
\pgfpathlineto{\pgfqpoint{4.081031in}{2.324927in}}%
\pgfpathlineto{\pgfqpoint{4.073173in}{2.315921in}}%
\pgfpathclose%
\pgfusepath{fill}%
\end{pgfscope}%
\begin{pgfscope}%
\pgfpathrectangle{\pgfqpoint{1.150000in}{0.150000in}}{\pgfqpoint{5.700000in}{5.700000in}}%
\pgfusepath{clip}%
\pgfsetbuttcap%
\pgfsetroundjoin%
\definecolor{currentfill}{rgb}{0.277134,0.185228,0.489898}%
\pgfsetfillcolor{currentfill}%
\pgfsetfillopacity{0.700000}%
\pgfsetlinewidth{0.000000pt}%
\definecolor{currentstroke}{rgb}{0.000000,0.000000,0.000000}%
\pgfsetstrokecolor{currentstroke}%
\pgfsetdash{}{0pt}%
\pgfpathmoveto{\pgfqpoint{4.697389in}{2.465478in}}%
\pgfpathlineto{\pgfqpoint{4.711174in}{2.464070in}}%
\pgfpathlineto{\pgfqpoint{4.724966in}{2.462735in}}%
\pgfpathlineto{\pgfqpoint{4.738768in}{2.461470in}}%
\pgfpathlineto{\pgfqpoint{4.752577in}{2.460278in}}%
\pgfpathlineto{\pgfqpoint{4.760202in}{2.468732in}}%
\pgfpathlineto{\pgfqpoint{4.767822in}{2.477277in}}%
\pgfpathlineto{\pgfqpoint{4.775438in}{2.485915in}}%
\pgfpathlineto{\pgfqpoint{4.783050in}{2.494654in}}%
\pgfpathlineto{\pgfqpoint{4.769255in}{2.496129in}}%
\pgfpathlineto{\pgfqpoint{4.755469in}{2.497676in}}%
\pgfpathlineto{\pgfqpoint{4.741691in}{2.499294in}}%
\pgfpathlineto{\pgfqpoint{4.727921in}{2.500983in}}%
\pgfpathlineto{\pgfqpoint{4.720294in}{2.491955in}}%
\pgfpathlineto{\pgfqpoint{4.712663in}{2.483032in}}%
\pgfpathlineto{\pgfqpoint{4.705028in}{2.474207in}}%
\pgfpathlineto{\pgfqpoint{4.697389in}{2.465478in}}%
\pgfpathclose%
\pgfusepath{fill}%
\end{pgfscope}%
\begin{pgfscope}%
\pgfpathrectangle{\pgfqpoint{1.150000in}{0.150000in}}{\pgfqpoint{5.700000in}{5.700000in}}%
\pgfusepath{clip}%
\pgfsetbuttcap%
\pgfsetroundjoin%
\definecolor{currentfill}{rgb}{0.241237,0.296485,0.539709}%
\pgfsetfillcolor{currentfill}%
\pgfsetfillopacity{0.700000}%
\pgfsetlinewidth{0.000000pt}%
\definecolor{currentstroke}{rgb}{0.000000,0.000000,0.000000}%
\pgfsetstrokecolor{currentstroke}%
\pgfsetdash{}{0pt}%
\pgfpathmoveto{\pgfqpoint{5.493308in}{2.705349in}}%
\pgfpathlineto{\pgfqpoint{5.507304in}{2.703870in}}%
\pgfpathlineto{\pgfqpoint{5.521310in}{2.702458in}}%
\pgfpathlineto{\pgfqpoint{5.535325in}{2.701112in}}%
\pgfpathlineto{\pgfqpoint{5.549349in}{2.699832in}}%
\pgfpathlineto{\pgfqpoint{5.556732in}{2.709760in}}%
\pgfpathlineto{\pgfqpoint{5.564118in}{2.719934in}}%
\pgfpathlineto{\pgfqpoint{5.571507in}{2.730363in}}%
\pgfpathlineto{\pgfqpoint{5.578898in}{2.741055in}}%
\pgfpathlineto{\pgfqpoint{5.564896in}{2.742779in}}%
\pgfpathlineto{\pgfqpoint{5.550904in}{2.744569in}}%
\pgfpathlineto{\pgfqpoint{5.536921in}{2.746425in}}%
\pgfpathlineto{\pgfqpoint{5.522947in}{2.748347in}}%
\pgfpathlineto{\pgfqpoint{5.515533in}{2.737204in}}%
\pgfpathlineto{\pgfqpoint{5.508122in}{2.726329in}}%
\pgfpathlineto{\pgfqpoint{5.500714in}{2.715713in}}%
\pgfpathlineto{\pgfqpoint{5.493308in}{2.705349in}}%
\pgfpathclose%
\pgfusepath{fill}%
\end{pgfscope}%
\begin{pgfscope}%
\pgfpathrectangle{\pgfqpoint{1.150000in}{0.150000in}}{\pgfqpoint{5.700000in}{5.700000in}}%
\pgfusepath{clip}%
\pgfsetbuttcap%
\pgfsetroundjoin%
\definecolor{currentfill}{rgb}{0.277134,0.185228,0.489898}%
\pgfsetfillcolor{currentfill}%
\pgfsetfillopacity{0.700000}%
\pgfsetlinewidth{0.000000pt}%
\definecolor{currentstroke}{rgb}{0.000000,0.000000,0.000000}%
\pgfsetstrokecolor{currentstroke}%
\pgfsetdash{}{0pt}%
\pgfpathmoveto{\pgfqpoint{2.442666in}{2.484301in}}%
\pgfpathlineto{\pgfqpoint{2.456158in}{2.472739in}}%
\pgfpathlineto{\pgfqpoint{2.469648in}{2.461315in}}%
\pgfpathlineto{\pgfqpoint{2.483136in}{2.450028in}}%
\pgfpathlineto{\pgfqpoint{2.496622in}{2.438878in}}%
\pgfpathlineto{\pgfqpoint{2.505086in}{2.446076in}}%
\pgfpathlineto{\pgfqpoint{2.513541in}{2.453373in}}%
\pgfpathlineto{\pgfqpoint{2.521986in}{2.460768in}}%
\pgfpathlineto{\pgfqpoint{2.530421in}{2.468260in}}%
\pgfpathlineto{\pgfqpoint{2.516954in}{2.479322in}}%
\pgfpathlineto{\pgfqpoint{2.503486in}{2.490521in}}%
\pgfpathlineto{\pgfqpoint{2.490016in}{2.501856in}}%
\pgfpathlineto{\pgfqpoint{2.476543in}{2.513331in}}%
\pgfpathlineto{\pgfqpoint{2.468088in}{2.505920in}}%
\pgfpathlineto{\pgfqpoint{2.459624in}{2.498610in}}%
\pgfpathlineto{\pgfqpoint{2.451150in}{2.491404in}}%
\pgfpathlineto{\pgfqpoint{2.442666in}{2.484301in}}%
\pgfpathclose%
\pgfusepath{fill}%
\end{pgfscope}%
\begin{pgfscope}%
\pgfpathrectangle{\pgfqpoint{1.150000in}{0.150000in}}{\pgfqpoint{5.700000in}{5.700000in}}%
\pgfusepath{clip}%
\pgfsetbuttcap%
\pgfsetroundjoin%
\definecolor{currentfill}{rgb}{0.266580,0.228262,0.514349}%
\pgfsetfillcolor{currentfill}%
\pgfsetfillopacity{0.700000}%
\pgfsetlinewidth{0.000000pt}%
\definecolor{currentstroke}{rgb}{0.000000,0.000000,0.000000}%
\pgfsetstrokecolor{currentstroke}%
\pgfsetdash{}{0pt}%
\pgfpathmoveto{\pgfqpoint{5.009646in}{2.549104in}}%
\pgfpathlineto{\pgfqpoint{5.023521in}{2.547916in}}%
\pgfpathlineto{\pgfqpoint{5.037405in}{2.546798in}}%
\pgfpathlineto{\pgfqpoint{5.051297in}{2.545749in}}%
\pgfpathlineto{\pgfqpoint{5.065199in}{2.544769in}}%
\pgfpathlineto{\pgfqpoint{5.072716in}{2.553312in}}%
\pgfpathlineto{\pgfqpoint{5.080229in}{2.561991in}}%
\pgfpathlineto{\pgfqpoint{5.087741in}{2.570813in}}%
\pgfpathlineto{\pgfqpoint{5.095250in}{2.579784in}}%
\pgfpathlineto{\pgfqpoint{5.081366in}{2.581107in}}%
\pgfpathlineto{\pgfqpoint{5.067491in}{2.582500in}}%
\pgfpathlineto{\pgfqpoint{5.053624in}{2.583961in}}%
\pgfpathlineto{\pgfqpoint{5.039767in}{2.585492in}}%
\pgfpathlineto{\pgfqpoint{5.032240in}{2.576170in}}%
\pgfpathlineto{\pgfqpoint{5.024711in}{2.567002in}}%
\pgfpathlineto{\pgfqpoint{5.017180in}{2.557982in}}%
\pgfpathlineto{\pgfqpoint{5.009646in}{2.549104in}}%
\pgfpathclose%
\pgfusepath{fill}%
\end{pgfscope}%
\begin{pgfscope}%
\pgfpathrectangle{\pgfqpoint{1.150000in}{0.150000in}}{\pgfqpoint{5.700000in}{5.700000in}}%
\pgfusepath{clip}%
\pgfsetbuttcap%
\pgfsetroundjoin%
\definecolor{currentfill}{rgb}{0.246811,0.283237,0.535941}%
\pgfsetfillcolor{currentfill}%
\pgfsetfillopacity{0.700000}%
\pgfsetlinewidth{0.000000pt}%
\definecolor{currentstroke}{rgb}{0.000000,0.000000,0.000000}%
\pgfsetstrokecolor{currentstroke}%
\pgfsetdash{}{0pt}%
\pgfpathmoveto{\pgfqpoint{5.407723in}{2.671137in}}%
\pgfpathlineto{\pgfqpoint{5.421704in}{2.669815in}}%
\pgfpathlineto{\pgfqpoint{5.435693in}{2.668561in}}%
\pgfpathlineto{\pgfqpoint{5.449692in}{2.667373in}}%
\pgfpathlineto{\pgfqpoint{5.463701in}{2.666253in}}%
\pgfpathlineto{\pgfqpoint{5.471101in}{2.675687in}}%
\pgfpathlineto{\pgfqpoint{5.478502in}{2.685343in}}%
\pgfpathlineto{\pgfqpoint{5.485904in}{2.695228in}}%
\pgfpathlineto{\pgfqpoint{5.493308in}{2.705349in}}%
\pgfpathlineto{\pgfqpoint{5.479322in}{2.706894in}}%
\pgfpathlineto{\pgfqpoint{5.465344in}{2.708505in}}%
\pgfpathlineto{\pgfqpoint{5.451376in}{2.710184in}}%
\pgfpathlineto{\pgfqpoint{5.437416in}{2.711929in}}%
\pgfpathlineto{\pgfqpoint{5.429991in}{2.701377in}}%
\pgfpathlineto{\pgfqpoint{5.422567in}{2.691066in}}%
\pgfpathlineto{\pgfqpoint{5.415145in}{2.680988in}}%
\pgfpathlineto{\pgfqpoint{5.407723in}{2.671137in}}%
\pgfpathclose%
\pgfusepath{fill}%
\end{pgfscope}%
\begin{pgfscope}%
\pgfpathrectangle{\pgfqpoint{1.150000in}{0.150000in}}{\pgfqpoint{5.700000in}{5.700000in}}%
\pgfusepath{clip}%
\pgfsetbuttcap%
\pgfsetroundjoin%
\definecolor{currentfill}{rgb}{0.280894,0.078907,0.402329}%
\pgfsetfillcolor{currentfill}%
\pgfsetfillopacity{0.700000}%
\pgfsetlinewidth{0.000000pt}%
\definecolor{currentstroke}{rgb}{0.000000,0.000000,0.000000}%
\pgfsetstrokecolor{currentstroke}%
\pgfsetdash{}{0pt}%
\pgfpathmoveto{\pgfqpoint{3.760970in}{2.255602in}}%
\pgfpathlineto{\pgfqpoint{3.774516in}{2.252119in}}%
\pgfpathlineto{\pgfqpoint{3.788068in}{2.248719in}}%
\pgfpathlineto{\pgfqpoint{3.801626in}{2.245403in}}%
\pgfpathlineto{\pgfqpoint{3.815190in}{2.242170in}}%
\pgfpathlineto{\pgfqpoint{3.823147in}{2.251216in}}%
\pgfpathlineto{\pgfqpoint{3.831098in}{2.260284in}}%
\pgfpathlineto{\pgfqpoint{3.839044in}{2.269375in}}%
\pgfpathlineto{\pgfqpoint{3.846984in}{2.278493in}}%
\pgfpathlineto{\pgfqpoint{3.833430in}{2.281825in}}%
\pgfpathlineto{\pgfqpoint{3.819883in}{2.285240in}}%
\pgfpathlineto{\pgfqpoint{3.806342in}{2.288739in}}%
\pgfpathlineto{\pgfqpoint{3.792807in}{2.292321in}}%
\pgfpathlineto{\pgfqpoint{3.784856in}{2.283097in}}%
\pgfpathlineto{\pgfqpoint{3.776900in}{2.273903in}}%
\pgfpathlineto{\pgfqpoint{3.768938in}{2.264739in}}%
\pgfpathlineto{\pgfqpoint{3.760970in}{2.255602in}}%
\pgfpathclose%
\pgfusepath{fill}%
\end{pgfscope}%
\begin{pgfscope}%
\pgfpathrectangle{\pgfqpoint{1.150000in}{0.150000in}}{\pgfqpoint{5.700000in}{5.700000in}}%
\pgfusepath{clip}%
\pgfsetbuttcap%
\pgfsetroundjoin%
\definecolor{currentfill}{rgb}{0.279566,0.067836,0.391917}%
\pgfsetfillcolor{currentfill}%
\pgfsetfillopacity{0.700000}%
\pgfsetlinewidth{0.000000pt}%
\definecolor{currentstroke}{rgb}{0.000000,0.000000,0.000000}%
\pgfsetstrokecolor{currentstroke}%
\pgfsetdash{}{0pt}%
\pgfpathmoveto{\pgfqpoint{3.534643in}{2.230066in}}%
\pgfpathlineto{\pgfqpoint{3.548146in}{2.225685in}}%
\pgfpathlineto{\pgfqpoint{3.561654in}{2.221392in}}%
\pgfpathlineto{\pgfqpoint{3.575167in}{2.217188in}}%
\pgfpathlineto{\pgfqpoint{3.588686in}{2.213072in}}%
\pgfpathlineto{\pgfqpoint{3.596721in}{2.222161in}}%
\pgfpathlineto{\pgfqpoint{3.604750in}{2.231274in}}%
\pgfpathlineto{\pgfqpoint{3.612774in}{2.240412in}}%
\pgfpathlineto{\pgfqpoint{3.620791in}{2.249576in}}%
\pgfpathlineto{\pgfqpoint{3.607284in}{2.253751in}}%
\pgfpathlineto{\pgfqpoint{3.593782in}{2.258013in}}%
\pgfpathlineto{\pgfqpoint{3.580284in}{2.262364in}}%
\pgfpathlineto{\pgfqpoint{3.566792in}{2.266803in}}%
\pgfpathlineto{\pgfqpoint{3.558764in}{2.257573in}}%
\pgfpathlineto{\pgfqpoint{3.550729in}{2.248375in}}%
\pgfpathlineto{\pgfqpoint{3.542689in}{2.239206in}}%
\pgfpathlineto{\pgfqpoint{3.534643in}{2.230066in}}%
\pgfpathclose%
\pgfusepath{fill}%
\end{pgfscope}%
\begin{pgfscope}%
\pgfpathrectangle{\pgfqpoint{1.150000in}{0.150000in}}{\pgfqpoint{5.700000in}{5.700000in}}%
\pgfusepath{clip}%
\pgfsetbuttcap%
\pgfsetroundjoin%
\definecolor{currentfill}{rgb}{0.282884,0.135920,0.453427}%
\pgfsetfillcolor{currentfill}%
\pgfsetfillopacity{0.700000}%
\pgfsetlinewidth{0.000000pt}%
\definecolor{currentstroke}{rgb}{0.000000,0.000000,0.000000}%
\pgfsetstrokecolor{currentstroke}%
\pgfsetdash{}{0pt}%
\pgfpathmoveto{\pgfqpoint{4.299460in}{2.359743in}}%
\pgfpathlineto{\pgfqpoint{4.313139in}{2.357788in}}%
\pgfpathlineto{\pgfqpoint{4.326824in}{2.355909in}}%
\pgfpathlineto{\pgfqpoint{4.340518in}{2.354105in}}%
\pgfpathlineto{\pgfqpoint{4.354218in}{2.352377in}}%
\pgfpathlineto{\pgfqpoint{4.361987in}{2.360999in}}%
\pgfpathlineto{\pgfqpoint{4.369751in}{2.369666in}}%
\pgfpathlineto{\pgfqpoint{4.377509in}{2.378380in}}%
\pgfpathlineto{\pgfqpoint{4.385263in}{2.387146in}}%
\pgfpathlineto{\pgfqpoint{4.371574in}{2.389076in}}%
\pgfpathlineto{\pgfqpoint{4.357893in}{2.391081in}}%
\pgfpathlineto{\pgfqpoint{4.344219in}{2.393161in}}%
\pgfpathlineto{\pgfqpoint{4.330553in}{2.395317in}}%
\pgfpathlineto{\pgfqpoint{4.322788in}{2.386342in}}%
\pgfpathlineto{\pgfqpoint{4.315017in}{2.377424in}}%
\pgfpathlineto{\pgfqpoint{4.307241in}{2.368559in}}%
\pgfpathlineto{\pgfqpoint{4.299460in}{2.359743in}}%
\pgfpathclose%
\pgfusepath{fill}%
\end{pgfscope}%
\begin{pgfscope}%
\pgfpathrectangle{\pgfqpoint{1.150000in}{0.150000in}}{\pgfqpoint{5.700000in}{5.700000in}}%
\pgfusepath{clip}%
\pgfsetbuttcap%
\pgfsetroundjoin%
\definecolor{currentfill}{rgb}{0.283229,0.120777,0.440584}%
\pgfsetfillcolor{currentfill}%
\pgfsetfillopacity{0.700000}%
\pgfsetlinewidth{0.000000pt}%
\definecolor{currentstroke}{rgb}{0.000000,0.000000,0.000000}%
\pgfsetstrokecolor{currentstroke}%
\pgfsetdash{}{0pt}%
\pgfpathmoveto{\pgfqpoint{2.691935in}{2.345707in}}%
\pgfpathlineto{\pgfqpoint{2.705391in}{2.336311in}}%
\pgfpathlineto{\pgfqpoint{2.718848in}{2.327035in}}%
\pgfpathlineto{\pgfqpoint{2.732304in}{2.317879in}}%
\pgfpathlineto{\pgfqpoint{2.745761in}{2.308843in}}%
\pgfpathlineto{\pgfqpoint{2.754116in}{2.316726in}}%
\pgfpathlineto{\pgfqpoint{2.762463in}{2.324684in}}%
\pgfpathlineto{\pgfqpoint{2.770802in}{2.332717in}}%
\pgfpathlineto{\pgfqpoint{2.779133in}{2.340824in}}%
\pgfpathlineto{\pgfqpoint{2.765693in}{2.349795in}}%
\pgfpathlineto{\pgfqpoint{2.752253in}{2.358885in}}%
\pgfpathlineto{\pgfqpoint{2.738814in}{2.368095in}}%
\pgfpathlineto{\pgfqpoint{2.725375in}{2.377426in}}%
\pgfpathlineto{\pgfqpoint{2.717027in}{2.369377in}}%
\pgfpathlineto{\pgfqpoint{2.708672in}{2.361407in}}%
\pgfpathlineto{\pgfqpoint{2.700308in}{2.353517in}}%
\pgfpathlineto{\pgfqpoint{2.691935in}{2.345707in}}%
\pgfpathclose%
\pgfusepath{fill}%
\end{pgfscope}%
\begin{pgfscope}%
\pgfpathrectangle{\pgfqpoint{1.150000in}{0.150000in}}{\pgfqpoint{5.700000in}{5.700000in}}%
\pgfusepath{clip}%
\pgfsetbuttcap%
\pgfsetroundjoin%
\definecolor{currentfill}{rgb}{0.278826,0.175490,0.483397}%
\pgfsetfillcolor{currentfill}%
\pgfsetfillopacity{0.700000}%
\pgfsetlinewidth{0.000000pt}%
\definecolor{currentstroke}{rgb}{0.000000,0.000000,0.000000}%
\pgfsetstrokecolor{currentstroke}%
\pgfsetdash{}{0pt}%
\pgfpathmoveto{\pgfqpoint{4.611678in}{2.436705in}}%
\pgfpathlineto{\pgfqpoint{4.625443in}{2.435271in}}%
\pgfpathlineto{\pgfqpoint{4.639217in}{2.433910in}}%
\pgfpathlineto{\pgfqpoint{4.652999in}{2.432621in}}%
\pgfpathlineto{\pgfqpoint{4.666790in}{2.431404in}}%
\pgfpathlineto{\pgfqpoint{4.674446in}{2.439805in}}%
\pgfpathlineto{\pgfqpoint{4.682098in}{2.448281in}}%
\pgfpathlineto{\pgfqpoint{4.689746in}{2.456837in}}%
\pgfpathlineto{\pgfqpoint{4.697389in}{2.465478in}}%
\pgfpathlineto{\pgfqpoint{4.683613in}{2.466957in}}%
\pgfpathlineto{\pgfqpoint{4.669845in}{2.468508in}}%
\pgfpathlineto{\pgfqpoint{4.656085in}{2.470132in}}%
\pgfpathlineto{\pgfqpoint{4.642334in}{2.471828in}}%
\pgfpathlineto{\pgfqpoint{4.634677in}{2.462918in}}%
\pgfpathlineto{\pgfqpoint{4.627015in}{2.454097in}}%
\pgfpathlineto{\pgfqpoint{4.619349in}{2.445361in}}%
\pgfpathlineto{\pgfqpoint{4.611678in}{2.436705in}}%
\pgfpathclose%
\pgfusepath{fill}%
\end{pgfscope}%
\begin{pgfscope}%
\pgfpathrectangle{\pgfqpoint{1.150000in}{0.150000in}}{\pgfqpoint{5.700000in}{5.700000in}}%
\pgfusepath{clip}%
\pgfsetbuttcap%
\pgfsetroundjoin%
\definecolor{currentfill}{rgb}{0.282656,0.100196,0.422160}%
\pgfsetfillcolor{currentfill}%
\pgfsetfillopacity{0.700000}%
\pgfsetlinewidth{0.000000pt}%
\definecolor{currentstroke}{rgb}{0.000000,0.000000,0.000000}%
\pgfsetstrokecolor{currentstroke}%
\pgfsetdash{}{0pt}%
\pgfpathmoveto{\pgfqpoint{3.987250in}{2.290607in}}%
\pgfpathlineto{\pgfqpoint{4.000849in}{2.287888in}}%
\pgfpathlineto{\pgfqpoint{4.014455in}{2.285249in}}%
\pgfpathlineto{\pgfqpoint{4.028068in}{2.282689in}}%
\pgfpathlineto{\pgfqpoint{4.041687in}{2.280209in}}%
\pgfpathlineto{\pgfqpoint{4.049567in}{2.289095in}}%
\pgfpathlineto{\pgfqpoint{4.057441in}{2.298008in}}%
\pgfpathlineto{\pgfqpoint{4.065310in}{2.306949in}}%
\pgfpathlineto{\pgfqpoint{4.073173in}{2.315921in}}%
\pgfpathlineto{\pgfqpoint{4.059565in}{2.318541in}}%
\pgfpathlineto{\pgfqpoint{4.045963in}{2.321241in}}%
\pgfpathlineto{\pgfqpoint{4.032368in}{2.324020in}}%
\pgfpathlineto{\pgfqpoint{4.018780in}{2.326879in}}%
\pgfpathlineto{\pgfqpoint{4.010906in}{2.317759in}}%
\pgfpathlineto{\pgfqpoint{4.003026in}{2.308676in}}%
\pgfpathlineto{\pgfqpoint{3.995141in}{2.299626in}}%
\pgfpathlineto{\pgfqpoint{3.987250in}{2.290607in}}%
\pgfpathclose%
\pgfusepath{fill}%
\end{pgfscope}%
\begin{pgfscope}%
\pgfpathrectangle{\pgfqpoint{1.150000in}{0.150000in}}{\pgfqpoint{5.700000in}{5.700000in}}%
\pgfusepath{clip}%
\pgfsetbuttcap%
\pgfsetroundjoin%
\definecolor{currentfill}{rgb}{0.280267,0.073417,0.397163}%
\pgfsetfillcolor{currentfill}%
\pgfsetfillopacity{0.700000}%
\pgfsetlinewidth{0.000000pt}%
\definecolor{currentstroke}{rgb}{0.000000,0.000000,0.000000}%
\pgfsetstrokecolor{currentstroke}%
\pgfsetdash{}{0pt}%
\pgfpathmoveto{\pgfqpoint{3.027322in}{2.246593in}}%
\pgfpathlineto{\pgfqpoint{3.040771in}{2.239555in}}%
\pgfpathlineto{\pgfqpoint{3.054223in}{2.232620in}}%
\pgfpathlineto{\pgfqpoint{3.067677in}{2.225789in}}%
\pgfpathlineto{\pgfqpoint{3.081133in}{2.219061in}}%
\pgfpathlineto{\pgfqpoint{3.089353in}{2.227690in}}%
\pgfpathlineto{\pgfqpoint{3.097566in}{2.236367in}}%
\pgfpathlineto{\pgfqpoint{3.105772in}{2.245091in}}%
\pgfpathlineto{\pgfqpoint{3.113971in}{2.253863in}}%
\pgfpathlineto{\pgfqpoint{3.100528in}{2.260568in}}%
\pgfpathlineto{\pgfqpoint{3.087087in}{2.267375in}}%
\pgfpathlineto{\pgfqpoint{3.073649in}{2.274286in}}%
\pgfpathlineto{\pgfqpoint{3.060214in}{2.281301in}}%
\pgfpathlineto{\pgfqpoint{3.052001in}{2.272545in}}%
\pgfpathlineto{\pgfqpoint{3.043782in}{2.263842in}}%
\pgfpathlineto{\pgfqpoint{3.035556in}{2.255192in}}%
\pgfpathlineto{\pgfqpoint{3.027322in}{2.246593in}}%
\pgfpathclose%
\pgfusepath{fill}%
\end{pgfscope}%
\begin{pgfscope}%
\pgfpathrectangle{\pgfqpoint{1.150000in}{0.150000in}}{\pgfqpoint{5.700000in}{5.700000in}}%
\pgfusepath{clip}%
\pgfsetbuttcap%
\pgfsetroundjoin%
\definecolor{currentfill}{rgb}{0.278791,0.062145,0.386592}%
\pgfsetfillcolor{currentfill}%
\pgfsetfillopacity{0.700000}%
\pgfsetlinewidth{0.000000pt}%
\definecolor{currentstroke}{rgb}{0.000000,0.000000,0.000000}%
\pgfsetstrokecolor{currentstroke}%
\pgfsetdash{}{0pt}%
\pgfpathmoveto{\pgfqpoint{3.167771in}{2.228058in}}%
\pgfpathlineto{\pgfqpoint{3.181229in}{2.221857in}}%
\pgfpathlineto{\pgfqpoint{3.194689in}{2.215755in}}%
\pgfpathlineto{\pgfqpoint{3.208153in}{2.209751in}}%
\pgfpathlineto{\pgfqpoint{3.221621in}{2.203845in}}%
\pgfpathlineto{\pgfqpoint{3.229788in}{2.212685in}}%
\pgfpathlineto{\pgfqpoint{3.237948in}{2.221564in}}%
\pgfpathlineto{\pgfqpoint{3.246101in}{2.230481in}}%
\pgfpathlineto{\pgfqpoint{3.254249in}{2.239437in}}%
\pgfpathlineto{\pgfqpoint{3.240794in}{2.245340in}}%
\pgfpathlineto{\pgfqpoint{3.227342in}{2.251341in}}%
\pgfpathlineto{\pgfqpoint{3.213894in}{2.257440in}}%
\pgfpathlineto{\pgfqpoint{3.200449in}{2.263638in}}%
\pgfpathlineto{\pgfqpoint{3.192290in}{2.254677in}}%
\pgfpathlineto{\pgfqpoint{3.184123in}{2.245761in}}%
\pgfpathlineto{\pgfqpoint{3.175950in}{2.236888in}}%
\pgfpathlineto{\pgfqpoint{3.167771in}{2.228058in}}%
\pgfpathclose%
\pgfusepath{fill}%
\end{pgfscope}%
\begin{pgfscope}%
\pgfpathrectangle{\pgfqpoint{1.150000in}{0.150000in}}{\pgfqpoint{5.700000in}{5.700000in}}%
\pgfusepath{clip}%
\pgfsetbuttcap%
\pgfsetroundjoin%
\definecolor{currentfill}{rgb}{0.252194,0.269783,0.531579}%
\pgfsetfillcolor{currentfill}%
\pgfsetfillopacity{0.700000}%
\pgfsetlinewidth{0.000000pt}%
\definecolor{currentstroke}{rgb}{0.000000,0.000000,0.000000}%
\pgfsetstrokecolor{currentstroke}%
\pgfsetdash{}{0pt}%
\pgfpathmoveto{\pgfqpoint{5.322130in}{2.638178in}}%
\pgfpathlineto{\pgfqpoint{5.336094in}{2.636993in}}%
\pgfpathlineto{\pgfqpoint{5.350067in}{2.635874in}}%
\pgfpathlineto{\pgfqpoint{5.364050in}{2.634823in}}%
\pgfpathlineto{\pgfqpoint{5.378042in}{2.633840in}}%
\pgfpathlineto{\pgfqpoint{5.385462in}{2.642862in}}%
\pgfpathlineto{\pgfqpoint{5.392882in}{2.652080in}}%
\pgfpathlineto{\pgfqpoint{5.400302in}{2.661503in}}%
\pgfpathlineto{\pgfqpoint{5.407723in}{2.671137in}}%
\pgfpathlineto{\pgfqpoint{5.393752in}{2.672525in}}%
\pgfpathlineto{\pgfqpoint{5.379790in}{2.673980in}}%
\pgfpathlineto{\pgfqpoint{5.365837in}{2.675502in}}%
\pgfpathlineto{\pgfqpoint{5.351894in}{2.677092in}}%
\pgfpathlineto{\pgfqpoint{5.344452in}{2.667047in}}%
\pgfpathlineto{\pgfqpoint{5.337011in}{2.657218in}}%
\pgfpathlineto{\pgfqpoint{5.329571in}{2.647597in}}%
\pgfpathlineto{\pgfqpoint{5.322130in}{2.638178in}}%
\pgfpathclose%
\pgfusepath{fill}%
\end{pgfscope}%
\begin{pgfscope}%
\pgfpathrectangle{\pgfqpoint{1.150000in}{0.150000in}}{\pgfqpoint{5.700000in}{5.700000in}}%
\pgfusepath{clip}%
\pgfsetbuttcap%
\pgfsetroundjoin%
\definecolor{currentfill}{rgb}{0.270595,0.214069,0.507052}%
\pgfsetfillcolor{currentfill}%
\pgfsetfillopacity{0.700000}%
\pgfsetlinewidth{0.000000pt}%
\definecolor{currentstroke}{rgb}{0.000000,0.000000,0.000000}%
\pgfsetstrokecolor{currentstroke}%
\pgfsetdash{}{0pt}%
\pgfpathmoveto{\pgfqpoint{4.924002in}{2.519038in}}%
\pgfpathlineto{\pgfqpoint{4.937859in}{2.517896in}}%
\pgfpathlineto{\pgfqpoint{4.951724in}{2.516823in}}%
\pgfpathlineto{\pgfqpoint{4.965598in}{2.515820in}}%
\pgfpathlineto{\pgfqpoint{4.979481in}{2.514886in}}%
\pgfpathlineto{\pgfqpoint{4.987027in}{2.523258in}}%
\pgfpathlineto{\pgfqpoint{4.994570in}{2.531748in}}%
\pgfpathlineto{\pgfqpoint{5.002109in}{2.540361in}}%
\pgfpathlineto{\pgfqpoint{5.009646in}{2.549104in}}%
\pgfpathlineto{\pgfqpoint{4.995780in}{2.550361in}}%
\pgfpathlineto{\pgfqpoint{4.981923in}{2.551687in}}%
\pgfpathlineto{\pgfqpoint{4.968074in}{2.553083in}}%
\pgfpathlineto{\pgfqpoint{4.954234in}{2.554549in}}%
\pgfpathlineto{\pgfqpoint{4.946681in}{2.545476in}}%
\pgfpathlineto{\pgfqpoint{4.939124in}{2.536537in}}%
\pgfpathlineto{\pgfqpoint{4.931565in}{2.527726in}}%
\pgfpathlineto{\pgfqpoint{4.924002in}{2.519038in}}%
\pgfpathclose%
\pgfusepath{fill}%
\end{pgfscope}%
\begin{pgfscope}%
\pgfpathrectangle{\pgfqpoint{1.150000in}{0.150000in}}{\pgfqpoint{5.700000in}{5.700000in}}%
\pgfusepath{clip}%
\pgfsetbuttcap%
\pgfsetroundjoin%
\definecolor{currentfill}{rgb}{0.280255,0.165693,0.476498}%
\pgfsetfillcolor{currentfill}%
\pgfsetfillopacity{0.700000}%
\pgfsetlinewidth{0.000000pt}%
\definecolor{currentstroke}{rgb}{0.000000,0.000000,0.000000}%
\pgfsetstrokecolor{currentstroke}%
\pgfsetdash{}{0pt}%
\pgfpathmoveto{\pgfqpoint{2.496622in}{2.438878in}}%
\pgfpathlineto{\pgfqpoint{2.510107in}{2.427863in}}%
\pgfpathlineto{\pgfqpoint{2.523590in}{2.416982in}}%
\pgfpathlineto{\pgfqpoint{2.537071in}{2.406233in}}%
\pgfpathlineto{\pgfqpoint{2.550551in}{2.395616in}}%
\pgfpathlineto{\pgfqpoint{2.558996in}{2.402910in}}%
\pgfpathlineto{\pgfqpoint{2.567431in}{2.410297in}}%
\pgfpathlineto{\pgfqpoint{2.575857in}{2.417776in}}%
\pgfpathlineto{\pgfqpoint{2.584274in}{2.425348in}}%
\pgfpathlineto{\pgfqpoint{2.570812in}{2.435877in}}%
\pgfpathlineto{\pgfqpoint{2.557350in}{2.446539in}}%
\pgfpathlineto{\pgfqpoint{2.543886in}{2.457332in}}%
\pgfpathlineto{\pgfqpoint{2.530421in}{2.468260in}}%
\pgfpathlineto{\pgfqpoint{2.521986in}{2.460768in}}%
\pgfpathlineto{\pgfqpoint{2.513541in}{2.453373in}}%
\pgfpathlineto{\pgfqpoint{2.505086in}{2.446076in}}%
\pgfpathlineto{\pgfqpoint{2.496622in}{2.438878in}}%
\pgfpathclose%
\pgfusepath{fill}%
\end{pgfscope}%
\begin{pgfscope}%
\pgfpathrectangle{\pgfqpoint{1.150000in}{0.150000in}}{\pgfqpoint{5.700000in}{5.700000in}}%
\pgfusepath{clip}%
\pgfsetbuttcap%
\pgfsetroundjoin%
\definecolor{currentfill}{rgb}{0.281446,0.084320,0.407414}%
\pgfsetfillcolor{currentfill}%
\pgfsetfillopacity{0.700000}%
\pgfsetlinewidth{0.000000pt}%
\definecolor{currentstroke}{rgb}{0.000000,0.000000,0.000000}%
\pgfsetstrokecolor{currentstroke}%
\pgfsetdash{}{0pt}%
\pgfpathmoveto{\pgfqpoint{2.886681in}{2.273221in}}%
\pgfpathlineto{\pgfqpoint{2.900130in}{2.265279in}}%
\pgfpathlineto{\pgfqpoint{2.913580in}{2.257447in}}%
\pgfpathlineto{\pgfqpoint{2.927032in}{2.249724in}}%
\pgfpathlineto{\pgfqpoint{2.940485in}{2.242110in}}%
\pgfpathlineto{\pgfqpoint{2.948762in}{2.250447in}}%
\pgfpathlineto{\pgfqpoint{2.957031in}{2.258841in}}%
\pgfpathlineto{\pgfqpoint{2.965293in}{2.267294in}}%
\pgfpathlineto{\pgfqpoint{2.973548in}{2.275803in}}%
\pgfpathlineto{\pgfqpoint{2.960109in}{2.283373in}}%
\pgfpathlineto{\pgfqpoint{2.946672in}{2.291052in}}%
\pgfpathlineto{\pgfqpoint{2.933237in}{2.298839in}}%
\pgfpathlineto{\pgfqpoint{2.919803in}{2.306737in}}%
\pgfpathlineto{\pgfqpoint{2.911534in}{2.298264in}}%
\pgfpathlineto{\pgfqpoint{2.903257in}{2.289854in}}%
\pgfpathlineto{\pgfqpoint{2.894973in}{2.281506in}}%
\pgfpathlineto{\pgfqpoint{2.886681in}{2.273221in}}%
\pgfpathclose%
\pgfusepath{fill}%
\end{pgfscope}%
\begin{pgfscope}%
\pgfpathrectangle{\pgfqpoint{1.150000in}{0.150000in}}{\pgfqpoint{5.700000in}{5.700000in}}%
\pgfusepath{clip}%
\pgfsetbuttcap%
\pgfsetroundjoin%
\definecolor{currentfill}{rgb}{0.203063,0.379716,0.553925}%
\pgfsetfillcolor{currentfill}%
\pgfsetfillopacity{0.700000}%
\pgfsetlinewidth{0.000000pt}%
\definecolor{currentstroke}{rgb}{0.000000,0.000000,0.000000}%
\pgfsetstrokecolor{currentstroke}%
\pgfsetdash{}{0pt}%
\pgfpathmoveto{\pgfqpoint{5.977949in}{2.894525in}}%
\pgfpathlineto{\pgfqpoint{5.992048in}{2.892193in}}%
\pgfpathlineto{\pgfqpoint{6.006157in}{2.889925in}}%
\pgfpathlineto{\pgfqpoint{6.020276in}{2.887721in}}%
\pgfpathlineto{\pgfqpoint{6.034404in}{2.885581in}}%
\pgfpathlineto{\pgfqpoint{6.041757in}{2.898842in}}%
\pgfpathlineto{\pgfqpoint{6.049121in}{2.912496in}}%
\pgfpathlineto{\pgfqpoint{6.056496in}{2.926554in}}%
\pgfpathlineto{\pgfqpoint{6.063883in}{2.941026in}}%
\pgfpathlineto{\pgfqpoint{6.049782in}{2.943710in}}%
\pgfpathlineto{\pgfqpoint{6.035691in}{2.946459in}}%
\pgfpathlineto{\pgfqpoint{6.021609in}{2.949271in}}%
\pgfpathlineto{\pgfqpoint{6.007536in}{2.952147in}}%
\pgfpathlineto{\pgfqpoint{6.000122in}{2.937123in}}%
\pgfpathlineto{\pgfqpoint{5.992719in}{2.922519in}}%
\pgfpathlineto{\pgfqpoint{5.985329in}{2.908323in}}%
\pgfpathlineto{\pgfqpoint{5.977949in}{2.894525in}}%
\pgfpathclose%
\pgfusepath{fill}%
\end{pgfscope}%
\begin{pgfscope}%
\pgfpathrectangle{\pgfqpoint{1.150000in}{0.150000in}}{\pgfqpoint{5.700000in}{5.700000in}}%
\pgfusepath{clip}%
\pgfsetbuttcap%
\pgfsetroundjoin%
\definecolor{currentfill}{rgb}{0.278791,0.062145,0.386592}%
\pgfsetfillcolor{currentfill}%
\pgfsetfillopacity{0.700000}%
\pgfsetlinewidth{0.000000pt}%
\definecolor{currentstroke}{rgb}{0.000000,0.000000,0.000000}%
\pgfsetstrokecolor{currentstroke}%
\pgfsetdash{}{0pt}%
\pgfpathmoveto{\pgfqpoint{3.308104in}{2.216788in}}%
\pgfpathlineto{\pgfqpoint{3.321578in}{2.211365in}}%
\pgfpathlineto{\pgfqpoint{3.335055in}{2.206035in}}%
\pgfpathlineto{\pgfqpoint{3.348536in}{2.200800in}}%
\pgfpathlineto{\pgfqpoint{3.362021in}{2.195658in}}%
\pgfpathlineto{\pgfqpoint{3.370138in}{2.204633in}}%
\pgfpathlineto{\pgfqpoint{3.378249in}{2.213639in}}%
\pgfpathlineto{\pgfqpoint{3.386354in}{2.222675in}}%
\pgfpathlineto{\pgfqpoint{3.394453in}{2.231743in}}%
\pgfpathlineto{\pgfqpoint{3.380979in}{2.236903in}}%
\pgfpathlineto{\pgfqpoint{3.367510in}{2.242156in}}%
\pgfpathlineto{\pgfqpoint{3.354044in}{2.247503in}}%
\pgfpathlineto{\pgfqpoint{3.340583in}{2.252944in}}%
\pgfpathlineto{\pgfqpoint{3.332472in}{2.243851in}}%
\pgfpathlineto{\pgfqpoint{3.324356in}{2.234794in}}%
\pgfpathlineto{\pgfqpoint{3.316233in}{2.225774in}}%
\pgfpathlineto{\pgfqpoint{3.308104in}{2.216788in}}%
\pgfpathclose%
\pgfusepath{fill}%
\end{pgfscope}%
\begin{pgfscope}%
\pgfpathrectangle{\pgfqpoint{1.150000in}{0.150000in}}{\pgfqpoint{5.700000in}{5.700000in}}%
\pgfusepath{clip}%
\pgfsetbuttcap%
\pgfsetroundjoin%
\definecolor{currentfill}{rgb}{0.210503,0.363727,0.552206}%
\pgfsetfillcolor{currentfill}%
\pgfsetfillopacity{0.700000}%
\pgfsetlinewidth{0.000000pt}%
\definecolor{currentstroke}{rgb}{0.000000,0.000000,0.000000}%
\pgfsetstrokecolor{currentstroke}%
\pgfsetdash{}{0pt}%
\pgfpathmoveto{\pgfqpoint{5.892118in}{2.851007in}}%
\pgfpathlineto{\pgfqpoint{5.906206in}{2.848941in}}%
\pgfpathlineto{\pgfqpoint{5.920303in}{2.846940in}}%
\pgfpathlineto{\pgfqpoint{5.934411in}{2.845003in}}%
\pgfpathlineto{\pgfqpoint{5.948528in}{2.843131in}}%
\pgfpathlineto{\pgfqpoint{5.955869in}{2.855429in}}%
\pgfpathlineto{\pgfqpoint{5.963219in}{2.868088in}}%
\pgfpathlineto{\pgfqpoint{5.970579in}{2.881117in}}%
\pgfpathlineto{\pgfqpoint{5.977949in}{2.894525in}}%
\pgfpathlineto{\pgfqpoint{5.963859in}{2.896922in}}%
\pgfpathlineto{\pgfqpoint{5.949778in}{2.899383in}}%
\pgfpathlineto{\pgfqpoint{5.935706in}{2.901908in}}%
\pgfpathlineto{\pgfqpoint{5.921644in}{2.904498in}}%
\pgfpathlineto{\pgfqpoint{5.914248in}{2.890558in}}%
\pgfpathlineto{\pgfqpoint{5.906862in}{2.877003in}}%
\pgfpathlineto{\pgfqpoint{5.899485in}{2.863822in}}%
\pgfpathlineto{\pgfqpoint{5.892118in}{2.851007in}}%
\pgfpathclose%
\pgfusepath{fill}%
\end{pgfscope}%
\begin{pgfscope}%
\pgfpathrectangle{\pgfqpoint{1.150000in}{0.150000in}}{\pgfqpoint{5.700000in}{5.700000in}}%
\pgfusepath{clip}%
\pgfsetbuttcap%
\pgfsetroundjoin%
\definecolor{currentfill}{rgb}{0.194100,0.399323,0.555565}%
\pgfsetfillcolor{currentfill}%
\pgfsetfillopacity{0.700000}%
\pgfsetlinewidth{0.000000pt}%
\definecolor{currentstroke}{rgb}{0.000000,0.000000,0.000000}%
\pgfsetstrokecolor{currentstroke}%
\pgfsetdash{}{0pt}%
\pgfpathmoveto{\pgfqpoint{6.063883in}{2.941026in}}%
\pgfpathlineto{\pgfqpoint{6.077993in}{2.938406in}}%
\pgfpathlineto{\pgfqpoint{6.092113in}{2.935850in}}%
\pgfpathlineto{\pgfqpoint{6.106242in}{2.933357in}}%
\pgfpathlineto{\pgfqpoint{6.120380in}{2.930929in}}%
\pgfpathlineto{\pgfqpoint{6.127752in}{2.945268in}}%
\pgfpathlineto{\pgfqpoint{6.135138in}{2.960036in}}%
\pgfpathlineto{\pgfqpoint{6.142537in}{2.975243in}}%
\pgfpathlineto{\pgfqpoint{6.149950in}{2.990901in}}%
\pgfpathlineto{\pgfqpoint{6.135840in}{2.993894in}}%
\pgfpathlineto{\pgfqpoint{6.121739in}{2.996950in}}%
\pgfpathlineto{\pgfqpoint{6.107647in}{3.000071in}}%
\pgfpathlineto{\pgfqpoint{6.093564in}{3.003255in}}%
\pgfpathlineto{\pgfqpoint{6.086123in}{2.987026in}}%
\pgfpathlineto{\pgfqpoint{6.078696in}{2.971252in}}%
\pgfpathlineto{\pgfqpoint{6.071283in}{2.955922in}}%
\pgfpathlineto{\pgfqpoint{6.063883in}{2.941026in}}%
\pgfpathclose%
\pgfusepath{fill}%
\end{pgfscope}%
\begin{pgfscope}%
\pgfpathrectangle{\pgfqpoint{1.150000in}{0.150000in}}{\pgfqpoint{5.700000in}{5.700000in}}%
\pgfusepath{clip}%
\pgfsetbuttcap%
\pgfsetroundjoin%
\definecolor{currentfill}{rgb}{0.218130,0.347432,0.550038}%
\pgfsetfillcolor{currentfill}%
\pgfsetfillopacity{0.700000}%
\pgfsetlinewidth{0.000000pt}%
\definecolor{currentstroke}{rgb}{0.000000,0.000000,0.000000}%
\pgfsetstrokecolor{currentstroke}%
\pgfsetdash{}{0pt}%
\pgfpathmoveto{\pgfqpoint{5.806364in}{2.810105in}}%
\pgfpathlineto{\pgfqpoint{5.820439in}{2.808284in}}%
\pgfpathlineto{\pgfqpoint{5.834525in}{2.806529in}}%
\pgfpathlineto{\pgfqpoint{5.848620in}{2.804838in}}%
\pgfpathlineto{\pgfqpoint{5.862725in}{2.803212in}}%
\pgfpathlineto{\pgfqpoint{5.870062in}{2.814659in}}%
\pgfpathlineto{\pgfqpoint{5.877406in}{2.826434in}}%
\pgfpathlineto{\pgfqpoint{5.884758in}{2.838547in}}%
\pgfpathlineto{\pgfqpoint{5.892118in}{2.851007in}}%
\pgfpathlineto{\pgfqpoint{5.878039in}{2.853137in}}%
\pgfpathlineto{\pgfqpoint{5.863969in}{2.855332in}}%
\pgfpathlineto{\pgfqpoint{5.849910in}{2.857592in}}%
\pgfpathlineto{\pgfqpoint{5.835859in}{2.859916in}}%
\pgfpathlineto{\pgfqpoint{5.828474in}{2.846945in}}%
\pgfpathlineto{\pgfqpoint{5.821096in}{2.834326in}}%
\pgfpathlineto{\pgfqpoint{5.813727in}{2.822049in}}%
\pgfpathlineto{\pgfqpoint{5.806364in}{2.810105in}}%
\pgfpathclose%
\pgfusepath{fill}%
\end{pgfscope}%
\begin{pgfscope}%
\pgfpathrectangle{\pgfqpoint{1.150000in}{0.150000in}}{\pgfqpoint{5.700000in}{5.700000in}}%
\pgfusepath{clip}%
\pgfsetbuttcap%
\pgfsetroundjoin%
\definecolor{currentfill}{rgb}{0.283187,0.125848,0.444960}%
\pgfsetfillcolor{currentfill}%
\pgfsetfillopacity{0.700000}%
\pgfsetlinewidth{0.000000pt}%
\definecolor{currentstroke}{rgb}{0.000000,0.000000,0.000000}%
\pgfsetstrokecolor{currentstroke}%
\pgfsetdash{}{0pt}%
\pgfpathmoveto{\pgfqpoint{4.213598in}{2.332758in}}%
\pgfpathlineto{\pgfqpoint{4.227259in}{2.330679in}}%
\pgfpathlineto{\pgfqpoint{4.240926in}{2.328676in}}%
\pgfpathlineto{\pgfqpoint{4.254601in}{2.326750in}}%
\pgfpathlineto{\pgfqpoint{4.268284in}{2.324900in}}%
\pgfpathlineto{\pgfqpoint{4.276086in}{2.333555in}}%
\pgfpathlineto{\pgfqpoint{4.283883in}{2.342245in}}%
\pgfpathlineto{\pgfqpoint{4.291674in}{2.350973in}}%
\pgfpathlineto{\pgfqpoint{4.299460in}{2.359743in}}%
\pgfpathlineto{\pgfqpoint{4.285790in}{2.361774in}}%
\pgfpathlineto{\pgfqpoint{4.272126in}{2.363882in}}%
\pgfpathlineto{\pgfqpoint{4.258470in}{2.366065in}}%
\pgfpathlineto{\pgfqpoint{4.244822in}{2.368325in}}%
\pgfpathlineto{\pgfqpoint{4.237024in}{2.359366in}}%
\pgfpathlineto{\pgfqpoint{4.229221in}{2.350455in}}%
\pgfpathlineto{\pgfqpoint{4.221412in}{2.341586in}}%
\pgfpathlineto{\pgfqpoint{4.213598in}{2.332758in}}%
\pgfpathclose%
\pgfusepath{fill}%
\end{pgfscope}%
\begin{pgfscope}%
\pgfpathrectangle{\pgfqpoint{1.150000in}{0.150000in}}{\pgfqpoint{5.700000in}{5.700000in}}%
\pgfusepath{clip}%
\pgfsetbuttcap%
\pgfsetroundjoin%
\definecolor{currentfill}{rgb}{0.185556,0.418570,0.556753}%
\pgfsetfillcolor{currentfill}%
\pgfsetfillopacity{0.700000}%
\pgfsetlinewidth{0.000000pt}%
\definecolor{currentstroke}{rgb}{0.000000,0.000000,0.000000}%
\pgfsetstrokecolor{currentstroke}%
\pgfsetdash{}{0pt}%
\pgfpathmoveto{\pgfqpoint{6.149950in}{2.990901in}}%
\pgfpathlineto{\pgfqpoint{6.164070in}{2.987971in}}%
\pgfpathlineto{\pgfqpoint{6.178199in}{2.985105in}}%
\pgfpathlineto{\pgfqpoint{6.192337in}{2.982303in}}%
\pgfpathlineto{\pgfqpoint{6.199745in}{2.997988in}}%
\pgfpathlineto{\pgfqpoint{6.207168in}{3.014140in}}%
\pgfpathlineto{\pgfqpoint{6.214607in}{3.030771in}}%
\pgfpathlineto{\pgfqpoint{6.200490in}{3.034009in}}%
\pgfpathlineto{\pgfqpoint{6.186382in}{3.037312in}}%
\pgfpathlineto{\pgfqpoint{6.172283in}{3.040677in}}%
\pgfpathlineto{\pgfqpoint{6.164823in}{3.023607in}}%
\pgfpathlineto{\pgfqpoint{6.157379in}{3.007018in}}%
\pgfpathlineto{\pgfqpoint{6.149950in}{2.990901in}}%
\pgfpathclose%
\pgfusepath{fill}%
\end{pgfscope}%
\begin{pgfscope}%
\pgfpathrectangle{\pgfqpoint{1.150000in}{0.150000in}}{\pgfqpoint{5.700000in}{5.700000in}}%
\pgfusepath{clip}%
\pgfsetbuttcap%
\pgfsetroundjoin%
\definecolor{currentfill}{rgb}{0.280868,0.160771,0.472899}%
\pgfsetfillcolor{currentfill}%
\pgfsetfillopacity{0.700000}%
\pgfsetlinewidth{0.000000pt}%
\definecolor{currentstroke}{rgb}{0.000000,0.000000,0.000000}%
\pgfsetstrokecolor{currentstroke}%
\pgfsetdash{}{0pt}%
\pgfpathmoveto{\pgfqpoint{4.525914in}{2.408282in}}%
\pgfpathlineto{\pgfqpoint{4.539660in}{2.406798in}}%
\pgfpathlineto{\pgfqpoint{4.553415in}{2.405388in}}%
\pgfpathlineto{\pgfqpoint{4.567178in}{2.404051in}}%
\pgfpathlineto{\pgfqpoint{4.580949in}{2.402787in}}%
\pgfpathlineto{\pgfqpoint{4.588638in}{2.411169in}}%
\pgfpathlineto{\pgfqpoint{4.596323in}{2.419614in}}%
\pgfpathlineto{\pgfqpoint{4.604003in}{2.428124in}}%
\pgfpathlineto{\pgfqpoint{4.611678in}{2.436705in}}%
\pgfpathlineto{\pgfqpoint{4.597921in}{2.438212in}}%
\pgfpathlineto{\pgfqpoint{4.584171in}{2.439791in}}%
\pgfpathlineto{\pgfqpoint{4.570430in}{2.441443in}}%
\pgfpathlineto{\pgfqpoint{4.556697in}{2.443168in}}%
\pgfpathlineto{\pgfqpoint{4.549008in}{2.434338in}}%
\pgfpathlineto{\pgfqpoint{4.541315in}{2.425583in}}%
\pgfpathlineto{\pgfqpoint{4.533617in}{2.416899in}}%
\pgfpathlineto{\pgfqpoint{4.525914in}{2.408282in}}%
\pgfpathclose%
\pgfusepath{fill}%
\end{pgfscope}%
\begin{pgfscope}%
\pgfpathrectangle{\pgfqpoint{1.150000in}{0.150000in}}{\pgfqpoint{5.700000in}{5.700000in}}%
\pgfusepath{clip}%
\pgfsetbuttcap%
\pgfsetroundjoin%
\definecolor{currentfill}{rgb}{0.280267,0.073417,0.397163}%
\pgfsetfillcolor{currentfill}%
\pgfsetfillopacity{0.700000}%
\pgfsetlinewidth{0.000000pt}%
\definecolor{currentstroke}{rgb}{0.000000,0.000000,0.000000}%
\pgfsetstrokecolor{currentstroke}%
\pgfsetdash{}{0pt}%
\pgfpathmoveto{\pgfqpoint{3.674873in}{2.233744in}}%
\pgfpathlineto{\pgfqpoint{3.688407in}{2.230001in}}%
\pgfpathlineto{\pgfqpoint{3.701947in}{2.226343in}}%
\pgfpathlineto{\pgfqpoint{3.715492in}{2.222770in}}%
\pgfpathlineto{\pgfqpoint{3.729043in}{2.219282in}}%
\pgfpathlineto{\pgfqpoint{3.737034in}{2.228331in}}%
\pgfpathlineto{\pgfqpoint{3.745018in}{2.237399in}}%
\pgfpathlineto{\pgfqpoint{3.752997in}{2.246489in}}%
\pgfpathlineto{\pgfqpoint{3.760970in}{2.255602in}}%
\pgfpathlineto{\pgfqpoint{3.747430in}{2.259169in}}%
\pgfpathlineto{\pgfqpoint{3.733896in}{2.262821in}}%
\pgfpathlineto{\pgfqpoint{3.720367in}{2.266557in}}%
\pgfpathlineto{\pgfqpoint{3.706843in}{2.270379in}}%
\pgfpathlineto{\pgfqpoint{3.698859in}{2.261180in}}%
\pgfpathlineto{\pgfqpoint{3.690870in}{2.252009in}}%
\pgfpathlineto{\pgfqpoint{3.682874in}{2.242864in}}%
\pgfpathlineto{\pgfqpoint{3.674873in}{2.233744in}}%
\pgfpathclose%
\pgfusepath{fill}%
\end{pgfscope}%
\begin{pgfscope}%
\pgfpathrectangle{\pgfqpoint{1.150000in}{0.150000in}}{\pgfqpoint{5.700000in}{5.700000in}}%
\pgfusepath{clip}%
\pgfsetbuttcap%
\pgfsetroundjoin%
\definecolor{currentfill}{rgb}{0.257322,0.256130,0.526563}%
\pgfsetfillcolor{currentfill}%
\pgfsetfillopacity{0.700000}%
\pgfsetlinewidth{0.000000pt}%
\definecolor{currentstroke}{rgb}{0.000000,0.000000,0.000000}%
\pgfsetstrokecolor{currentstroke}%
\pgfsetdash{}{0pt}%
\pgfpathmoveto{\pgfqpoint{5.236517in}{2.606256in}}%
\pgfpathlineto{\pgfqpoint{5.250464in}{2.605184in}}%
\pgfpathlineto{\pgfqpoint{5.264420in}{2.604180in}}%
\pgfpathlineto{\pgfqpoint{5.278386in}{2.603244in}}%
\pgfpathlineto{\pgfqpoint{5.292361in}{2.602376in}}%
\pgfpathlineto{\pgfqpoint{5.299804in}{2.611059in}}%
\pgfpathlineto{\pgfqpoint{5.307247in}{2.619916in}}%
\pgfpathlineto{\pgfqpoint{5.314688in}{2.628953in}}%
\pgfpathlineto{\pgfqpoint{5.322130in}{2.638178in}}%
\pgfpathlineto{\pgfqpoint{5.308175in}{2.639431in}}%
\pgfpathlineto{\pgfqpoint{5.294229in}{2.640752in}}%
\pgfpathlineto{\pgfqpoint{5.280293in}{2.642140in}}%
\pgfpathlineto{\pgfqpoint{5.266366in}{2.643596in}}%
\pgfpathlineto{\pgfqpoint{5.258904in}{2.633980in}}%
\pgfpathlineto{\pgfqpoint{5.251443in}{2.624556in}}%
\pgfpathlineto{\pgfqpoint{5.243980in}{2.615317in}}%
\pgfpathlineto{\pgfqpoint{5.236517in}{2.606256in}}%
\pgfpathclose%
\pgfusepath{fill}%
\end{pgfscope}%
\begin{pgfscope}%
\pgfpathrectangle{\pgfqpoint{1.150000in}{0.150000in}}{\pgfqpoint{5.700000in}{5.700000in}}%
\pgfusepath{clip}%
\pgfsetbuttcap%
\pgfsetroundjoin%
\definecolor{currentfill}{rgb}{0.225863,0.330805,0.547314}%
\pgfsetfillcolor{currentfill}%
\pgfsetfillopacity{0.700000}%
\pgfsetlinewidth{0.000000pt}%
\definecolor{currentstroke}{rgb}{0.000000,0.000000,0.000000}%
\pgfsetstrokecolor{currentstroke}%
\pgfsetdash{}{0pt}%
\pgfpathmoveto{\pgfqpoint{5.720664in}{2.771481in}}%
\pgfpathlineto{\pgfqpoint{5.734727in}{2.769884in}}%
\pgfpathlineto{\pgfqpoint{5.748799in}{2.768352in}}%
\pgfpathlineto{\pgfqpoint{5.762881in}{2.766886in}}%
\pgfpathlineto{\pgfqpoint{5.776973in}{2.765486in}}%
\pgfpathlineto{\pgfqpoint{5.784312in}{2.776184in}}%
\pgfpathlineto{\pgfqpoint{5.791657in}{2.787181in}}%
\pgfpathlineto{\pgfqpoint{5.799007in}{2.798485in}}%
\pgfpathlineto{\pgfqpoint{5.806364in}{2.810105in}}%
\pgfpathlineto{\pgfqpoint{5.792297in}{2.811990in}}%
\pgfpathlineto{\pgfqpoint{5.778240in}{2.813941in}}%
\pgfpathlineto{\pgfqpoint{5.764193in}{2.815957in}}%
\pgfpathlineto{\pgfqpoint{5.750154in}{2.818037in}}%
\pgfpathlineto{\pgfqpoint{5.742773in}{2.805926in}}%
\pgfpathlineto{\pgfqpoint{5.735398in}{2.794136in}}%
\pgfpathlineto{\pgfqpoint{5.728028in}{2.782657in}}%
\pgfpathlineto{\pgfqpoint{5.720664in}{2.771481in}}%
\pgfpathclose%
\pgfusepath{fill}%
\end{pgfscope}%
\begin{pgfscope}%
\pgfpathrectangle{\pgfqpoint{1.150000in}{0.150000in}}{\pgfqpoint{5.700000in}{5.700000in}}%
\pgfusepath{clip}%
\pgfsetbuttcap%
\pgfsetroundjoin%
\definecolor{currentfill}{rgb}{0.281924,0.089666,0.412415}%
\pgfsetfillcolor{currentfill}%
\pgfsetfillopacity{0.700000}%
\pgfsetlinewidth{0.000000pt}%
\definecolor{currentstroke}{rgb}{0.000000,0.000000,0.000000}%
\pgfsetstrokecolor{currentstroke}%
\pgfsetdash{}{0pt}%
\pgfpathmoveto{\pgfqpoint{3.901257in}{2.265988in}}%
\pgfpathlineto{\pgfqpoint{3.914841in}{2.263066in}}%
\pgfpathlineto{\pgfqpoint{3.928431in}{2.260225in}}%
\pgfpathlineto{\pgfqpoint{3.942028in}{2.257465in}}%
\pgfpathlineto{\pgfqpoint{3.955632in}{2.254786in}}%
\pgfpathlineto{\pgfqpoint{3.963545in}{2.263708in}}%
\pgfpathlineto{\pgfqpoint{3.971452in}{2.272650in}}%
\pgfpathlineto{\pgfqpoint{3.979354in}{2.281616in}}%
\pgfpathlineto{\pgfqpoint{3.987250in}{2.290607in}}%
\pgfpathlineto{\pgfqpoint{3.973658in}{2.293406in}}%
\pgfpathlineto{\pgfqpoint{3.960071in}{2.296286in}}%
\pgfpathlineto{\pgfqpoint{3.946492in}{2.299246in}}%
\pgfpathlineto{\pgfqpoint{3.932918in}{2.302288in}}%
\pgfpathlineto{\pgfqpoint{3.925011in}{2.293170in}}%
\pgfpathlineto{\pgfqpoint{3.917099in}{2.284082in}}%
\pgfpathlineto{\pgfqpoint{3.909181in}{2.275022in}}%
\pgfpathlineto{\pgfqpoint{3.901257in}{2.265988in}}%
\pgfpathclose%
\pgfusepath{fill}%
\end{pgfscope}%
\begin{pgfscope}%
\pgfpathrectangle{\pgfqpoint{1.150000in}{0.150000in}}{\pgfqpoint{5.700000in}{5.700000in}}%
\pgfusepath{clip}%
\pgfsetbuttcap%
\pgfsetroundjoin%
\definecolor{currentfill}{rgb}{0.273006,0.204520,0.501721}%
\pgfsetfillcolor{currentfill}%
\pgfsetfillopacity{0.700000}%
\pgfsetlinewidth{0.000000pt}%
\definecolor{currentstroke}{rgb}{0.000000,0.000000,0.000000}%
\pgfsetstrokecolor{currentstroke}%
\pgfsetdash{}{0pt}%
\pgfpathmoveto{\pgfqpoint{4.838314in}{2.489463in}}%
\pgfpathlineto{\pgfqpoint{4.852151in}{2.488343in}}%
\pgfpathlineto{\pgfqpoint{4.865998in}{2.487293in}}%
\pgfpathlineto{\pgfqpoint{4.879853in}{2.486313in}}%
\pgfpathlineto{\pgfqpoint{4.893717in}{2.485404in}}%
\pgfpathlineto{\pgfqpoint{4.901294in}{2.493656in}}%
\pgfpathlineto{\pgfqpoint{4.908867in}{2.502009in}}%
\pgfpathlineto{\pgfqpoint{4.916436in}{2.510468in}}%
\pgfpathlineto{\pgfqpoint{4.924002in}{2.519038in}}%
\pgfpathlineto{\pgfqpoint{4.910154in}{2.520251in}}%
\pgfpathlineto{\pgfqpoint{4.896316in}{2.521534in}}%
\pgfpathlineto{\pgfqpoint{4.882485in}{2.522887in}}%
\pgfpathlineto{\pgfqpoint{4.868664in}{2.524311in}}%
\pgfpathlineto{\pgfqpoint{4.861081in}{2.515430in}}%
\pgfpathlineto{\pgfqpoint{4.853496in}{2.506665in}}%
\pgfpathlineto{\pgfqpoint{4.845907in}{2.498012in}}%
\pgfpathlineto{\pgfqpoint{4.838314in}{2.489463in}}%
\pgfpathclose%
\pgfusepath{fill}%
\end{pgfscope}%
\begin{pgfscope}%
\pgfpathrectangle{\pgfqpoint{1.150000in}{0.150000in}}{\pgfqpoint{5.700000in}{5.700000in}}%
\pgfusepath{clip}%
\pgfsetbuttcap%
\pgfsetroundjoin%
\definecolor{currentfill}{rgb}{0.278791,0.062145,0.386592}%
\pgfsetfillcolor{currentfill}%
\pgfsetfillopacity{0.700000}%
\pgfsetlinewidth{0.000000pt}%
\definecolor{currentstroke}{rgb}{0.000000,0.000000,0.000000}%
\pgfsetstrokecolor{currentstroke}%
\pgfsetdash{}{0pt}%
\pgfpathmoveto{\pgfqpoint{3.448390in}{2.212026in}}%
\pgfpathlineto{\pgfqpoint{3.461885in}{2.207325in}}%
\pgfpathlineto{\pgfqpoint{3.475385in}{2.202714in}}%
\pgfpathlineto{\pgfqpoint{3.488890in}{2.198193in}}%
\pgfpathlineto{\pgfqpoint{3.502400in}{2.193761in}}%
\pgfpathlineto{\pgfqpoint{3.510469in}{2.202801in}}%
\pgfpathlineto{\pgfqpoint{3.518533in}{2.211864in}}%
\pgfpathlineto{\pgfqpoint{3.526591in}{2.220952in}}%
\pgfpathlineto{\pgfqpoint{3.534643in}{2.230066in}}%
\pgfpathlineto{\pgfqpoint{3.521145in}{2.234535in}}%
\pgfpathlineto{\pgfqpoint{3.507651in}{2.239094in}}%
\pgfpathlineto{\pgfqpoint{3.494162in}{2.243743in}}%
\pgfpathlineto{\pgfqpoint{3.480678in}{2.248482in}}%
\pgfpathlineto{\pgfqpoint{3.472615in}{2.239323in}}%
\pgfpathlineto{\pgfqpoint{3.464546in}{2.230195in}}%
\pgfpathlineto{\pgfqpoint{3.456471in}{2.221096in}}%
\pgfpathlineto{\pgfqpoint{3.448390in}{2.212026in}}%
\pgfpathclose%
\pgfusepath{fill}%
\end{pgfscope}%
\begin{pgfscope}%
\pgfpathrectangle{\pgfqpoint{1.150000in}{0.150000in}}{\pgfqpoint{5.700000in}{5.700000in}}%
\pgfusepath{clip}%
\pgfsetbuttcap%
\pgfsetroundjoin%
\definecolor{currentfill}{rgb}{0.282910,0.105393,0.426902}%
\pgfsetfillcolor{currentfill}%
\pgfsetfillopacity{0.700000}%
\pgfsetlinewidth{0.000000pt}%
\definecolor{currentstroke}{rgb}{0.000000,0.000000,0.000000}%
\pgfsetstrokecolor{currentstroke}%
\pgfsetdash{}{0pt}%
\pgfpathmoveto{\pgfqpoint{2.745761in}{2.308843in}}%
\pgfpathlineto{\pgfqpoint{2.759218in}{2.299924in}}%
\pgfpathlineto{\pgfqpoint{2.772675in}{2.291122in}}%
\pgfpathlineto{\pgfqpoint{2.786134in}{2.282437in}}%
\pgfpathlineto{\pgfqpoint{2.799593in}{2.273866in}}%
\pgfpathlineto{\pgfqpoint{2.807931in}{2.281822in}}%
\pgfpathlineto{\pgfqpoint{2.816262in}{2.289848in}}%
\pgfpathlineto{\pgfqpoint{2.824585in}{2.297944in}}%
\pgfpathlineto{\pgfqpoint{2.832899in}{2.306108in}}%
\pgfpathlineto{\pgfqpoint{2.819457in}{2.314613in}}%
\pgfpathlineto{\pgfqpoint{2.806015in}{2.323234in}}%
\pgfpathlineto{\pgfqpoint{2.792574in}{2.331970in}}%
\pgfpathlineto{\pgfqpoint{2.779133in}{2.340824in}}%
\pgfpathlineto{\pgfqpoint{2.770802in}{2.332717in}}%
\pgfpathlineto{\pgfqpoint{2.762463in}{2.324684in}}%
\pgfpathlineto{\pgfqpoint{2.754116in}{2.316726in}}%
\pgfpathlineto{\pgfqpoint{2.745761in}{2.308843in}}%
\pgfpathclose%
\pgfusepath{fill}%
\end{pgfscope}%
\begin{pgfscope}%
\pgfpathrectangle{\pgfqpoint{1.150000in}{0.150000in}}{\pgfqpoint{5.700000in}{5.700000in}}%
\pgfusepath{clip}%
\pgfsetbuttcap%
\pgfsetroundjoin%
\definecolor{currentfill}{rgb}{0.281887,0.150881,0.465405}%
\pgfsetfillcolor{currentfill}%
\pgfsetfillopacity{0.700000}%
\pgfsetlinewidth{0.000000pt}%
\definecolor{currentstroke}{rgb}{0.000000,0.000000,0.000000}%
\pgfsetstrokecolor{currentstroke}%
\pgfsetdash{}{0pt}%
\pgfpathmoveto{\pgfqpoint{2.550551in}{2.395616in}}%
\pgfpathlineto{\pgfqpoint{2.564030in}{2.385130in}}%
\pgfpathlineto{\pgfqpoint{2.577508in}{2.374773in}}%
\pgfpathlineto{\pgfqpoint{2.590985in}{2.364543in}}%
\pgfpathlineto{\pgfqpoint{2.604461in}{2.354441in}}%
\pgfpathlineto{\pgfqpoint{2.612886in}{2.361829in}}%
\pgfpathlineto{\pgfqpoint{2.621303in}{2.369306in}}%
\pgfpathlineto{\pgfqpoint{2.629710in}{2.376870in}}%
\pgfpathlineto{\pgfqpoint{2.638109in}{2.384520in}}%
\pgfpathlineto{\pgfqpoint{2.624651in}{2.394536in}}%
\pgfpathlineto{\pgfqpoint{2.611193in}{2.404678in}}%
\pgfpathlineto{\pgfqpoint{2.597734in}{2.414948in}}%
\pgfpathlineto{\pgfqpoint{2.584274in}{2.425348in}}%
\pgfpathlineto{\pgfqpoint{2.575857in}{2.417776in}}%
\pgfpathlineto{\pgfqpoint{2.567431in}{2.410297in}}%
\pgfpathlineto{\pgfqpoint{2.558996in}{2.402910in}}%
\pgfpathlineto{\pgfqpoint{2.550551in}{2.395616in}}%
\pgfpathclose%
\pgfusepath{fill}%
\end{pgfscope}%
\begin{pgfscope}%
\pgfpathrectangle{\pgfqpoint{1.150000in}{0.150000in}}{\pgfqpoint{5.700000in}{5.700000in}}%
\pgfusepath{clip}%
\pgfsetbuttcap%
\pgfsetroundjoin%
\definecolor{currentfill}{rgb}{0.231674,0.318106,0.544834}%
\pgfsetfillcolor{currentfill}%
\pgfsetfillopacity{0.700000}%
\pgfsetlinewidth{0.000000pt}%
\definecolor{currentstroke}{rgb}{0.000000,0.000000,0.000000}%
\pgfsetstrokecolor{currentstroke}%
\pgfsetdash{}{0pt}%
\pgfpathmoveto{\pgfqpoint{5.634998in}{2.734820in}}%
\pgfpathlineto{\pgfqpoint{5.649047in}{2.733425in}}%
\pgfpathlineto{\pgfqpoint{5.663105in}{2.732097in}}%
\pgfpathlineto{\pgfqpoint{5.677173in}{2.730834in}}%
\pgfpathlineto{\pgfqpoint{5.691251in}{2.729637in}}%
\pgfpathlineto{\pgfqpoint{5.698598in}{2.739685in}}%
\pgfpathlineto{\pgfqpoint{5.705949in}{2.750003in}}%
\pgfpathlineto{\pgfqpoint{5.713304in}{2.760599in}}%
\pgfpathlineto{\pgfqpoint{5.720664in}{2.771481in}}%
\pgfpathlineto{\pgfqpoint{5.706611in}{2.773143in}}%
\pgfpathlineto{\pgfqpoint{5.692567in}{2.774870in}}%
\pgfpathlineto{\pgfqpoint{5.678533in}{2.776663in}}%
\pgfpathlineto{\pgfqpoint{5.664508in}{2.778522in}}%
\pgfpathlineto{\pgfqpoint{5.657124in}{2.767168in}}%
\pgfpathlineto{\pgfqpoint{5.649745in}{2.756106in}}%
\pgfpathlineto{\pgfqpoint{5.642370in}{2.745326in}}%
\pgfpathlineto{\pgfqpoint{5.634998in}{2.734820in}}%
\pgfpathclose%
\pgfusepath{fill}%
\end{pgfscope}%
\begin{pgfscope}%
\pgfpathrectangle{\pgfqpoint{1.150000in}{0.150000in}}{\pgfqpoint{5.700000in}{5.700000in}}%
\pgfusepath{clip}%
\pgfsetbuttcap%
\pgfsetroundjoin%
\definecolor{currentfill}{rgb}{0.260571,0.246922,0.522828}%
\pgfsetfillcolor{currentfill}%
\pgfsetfillopacity{0.700000}%
\pgfsetlinewidth{0.000000pt}%
\definecolor{currentstroke}{rgb}{0.000000,0.000000,0.000000}%
\pgfsetstrokecolor{currentstroke}%
\pgfsetdash{}{0pt}%
\pgfpathmoveto{\pgfqpoint{5.150876in}{2.575177in}}%
\pgfpathlineto{\pgfqpoint{5.164805in}{2.574196in}}%
\pgfpathlineto{\pgfqpoint{5.178744in}{2.573285in}}%
\pgfpathlineto{\pgfqpoint{5.192691in}{2.572441in}}%
\pgfpathlineto{\pgfqpoint{5.206648in}{2.571666in}}%
\pgfpathlineto{\pgfqpoint{5.214118in}{2.580079in}}%
\pgfpathlineto{\pgfqpoint{5.221586in}{2.588644in}}%
\pgfpathlineto{\pgfqpoint{5.229052in}{2.597367in}}%
\pgfpathlineto{\pgfqpoint{5.236517in}{2.606256in}}%
\pgfpathlineto{\pgfqpoint{5.222579in}{2.607396in}}%
\pgfpathlineto{\pgfqpoint{5.208651in}{2.608604in}}%
\pgfpathlineto{\pgfqpoint{5.194731in}{2.609880in}}%
\pgfpathlineto{\pgfqpoint{5.180821in}{2.611224in}}%
\pgfpathlineto{\pgfqpoint{5.173337in}{2.601964in}}%
\pgfpathlineto{\pgfqpoint{5.165852in}{2.592874in}}%
\pgfpathlineto{\pgfqpoint{5.158365in}{2.583947in}}%
\pgfpathlineto{\pgfqpoint{5.150876in}{2.575177in}}%
\pgfpathclose%
\pgfusepath{fill}%
\end{pgfscope}%
\begin{pgfscope}%
\pgfpathrectangle{\pgfqpoint{1.150000in}{0.150000in}}{\pgfqpoint{5.700000in}{5.700000in}}%
\pgfusepath{clip}%
\pgfsetbuttcap%
\pgfsetroundjoin%
\definecolor{currentfill}{rgb}{0.281887,0.150881,0.465405}%
\pgfsetfillcolor{currentfill}%
\pgfsetfillopacity{0.700000}%
\pgfsetlinewidth{0.000000pt}%
\definecolor{currentstroke}{rgb}{0.000000,0.000000,0.000000}%
\pgfsetstrokecolor{currentstroke}%
\pgfsetdash{}{0pt}%
\pgfpathmoveto{\pgfqpoint{4.440095in}{2.380175in}}%
\pgfpathlineto{\pgfqpoint{4.453822in}{2.378619in}}%
\pgfpathlineto{\pgfqpoint{4.467557in}{2.377136in}}%
\pgfpathlineto{\pgfqpoint{4.481301in}{2.375728in}}%
\pgfpathlineto{\pgfqpoint{4.495052in}{2.374393in}}%
\pgfpathlineto{\pgfqpoint{4.502775in}{2.382786in}}%
\pgfpathlineto{\pgfqpoint{4.510493in}{2.391229in}}%
\pgfpathlineto{\pgfqpoint{4.518206in}{2.399726in}}%
\pgfpathlineto{\pgfqpoint{4.525914in}{2.408282in}}%
\pgfpathlineto{\pgfqpoint{4.512175in}{2.409838in}}%
\pgfpathlineto{\pgfqpoint{4.498445in}{2.411469in}}%
\pgfpathlineto{\pgfqpoint{4.484722in}{2.413173in}}%
\pgfpathlineto{\pgfqpoint{4.471008in}{2.414952in}}%
\pgfpathlineto{\pgfqpoint{4.463287in}{2.406167in}}%
\pgfpathlineto{\pgfqpoint{4.455561in}{2.397446in}}%
\pgfpathlineto{\pgfqpoint{4.447830in}{2.388783in}}%
\pgfpathlineto{\pgfqpoint{4.440095in}{2.380175in}}%
\pgfpathclose%
\pgfusepath{fill}%
\end{pgfscope}%
\begin{pgfscope}%
\pgfpathrectangle{\pgfqpoint{1.150000in}{0.150000in}}{\pgfqpoint{5.700000in}{5.700000in}}%
\pgfusepath{clip}%
\pgfsetbuttcap%
\pgfsetroundjoin%
\definecolor{currentfill}{rgb}{0.283197,0.115680,0.436115}%
\pgfsetfillcolor{currentfill}%
\pgfsetfillopacity{0.700000}%
\pgfsetlinewidth{0.000000pt}%
\definecolor{currentstroke}{rgb}{0.000000,0.000000,0.000000}%
\pgfsetstrokecolor{currentstroke}%
\pgfsetdash{}{0pt}%
\pgfpathmoveto{\pgfqpoint{4.127675in}{2.306227in}}%
\pgfpathlineto{\pgfqpoint{4.141318in}{2.303999in}}%
\pgfpathlineto{\pgfqpoint{4.154968in}{2.301848in}}%
\pgfpathlineto{\pgfqpoint{4.168625in}{2.299776in}}%
\pgfpathlineto{\pgfqpoint{4.182289in}{2.297780in}}%
\pgfpathlineto{\pgfqpoint{4.190125in}{2.306481in}}%
\pgfpathlineto{\pgfqpoint{4.197955in}{2.315208in}}%
\pgfpathlineto{\pgfqpoint{4.205779in}{2.323966in}}%
\pgfpathlineto{\pgfqpoint{4.213598in}{2.332758in}}%
\pgfpathlineto{\pgfqpoint{4.199945in}{2.334914in}}%
\pgfpathlineto{\pgfqpoint{4.186300in}{2.337148in}}%
\pgfpathlineto{\pgfqpoint{4.172661in}{2.339459in}}%
\pgfpathlineto{\pgfqpoint{4.159029in}{2.341847in}}%
\pgfpathlineto{\pgfqpoint{4.151199in}{2.332887in}}%
\pgfpathlineto{\pgfqpoint{4.143363in}{2.323966in}}%
\pgfpathlineto{\pgfqpoint{4.135521in}{2.315080in}}%
\pgfpathlineto{\pgfqpoint{4.127675in}{2.306227in}}%
\pgfpathclose%
\pgfusepath{fill}%
\end{pgfscope}%
\begin{pgfscope}%
\pgfpathrectangle{\pgfqpoint{1.150000in}{0.150000in}}{\pgfqpoint{5.700000in}{5.700000in}}%
\pgfusepath{clip}%
\pgfsetbuttcap%
\pgfsetroundjoin%
\definecolor{currentfill}{rgb}{0.278791,0.062145,0.386592}%
\pgfsetfillcolor{currentfill}%
\pgfsetfillopacity{0.700000}%
\pgfsetlinewidth{0.000000pt}%
\definecolor{currentstroke}{rgb}{0.000000,0.000000,0.000000}%
\pgfsetstrokecolor{currentstroke}%
\pgfsetdash{}{0pt}%
\pgfpathmoveto{\pgfqpoint{3.081133in}{2.219061in}}%
\pgfpathlineto{\pgfqpoint{3.094593in}{2.212435in}}%
\pgfpathlineto{\pgfqpoint{3.108055in}{2.205910in}}%
\pgfpathlineto{\pgfqpoint{3.121519in}{2.199486in}}%
\pgfpathlineto{\pgfqpoint{3.134987in}{2.193162in}}%
\pgfpathlineto{\pgfqpoint{3.143193in}{2.201823in}}%
\pgfpathlineto{\pgfqpoint{3.151392in}{2.210525in}}%
\pgfpathlineto{\pgfqpoint{3.159585in}{2.219270in}}%
\pgfpathlineto{\pgfqpoint{3.167771in}{2.228058in}}%
\pgfpathlineto{\pgfqpoint{3.154317in}{2.234358in}}%
\pgfpathlineto{\pgfqpoint{3.140865in}{2.240759in}}%
\pgfpathlineto{\pgfqpoint{3.127416in}{2.247260in}}%
\pgfpathlineto{\pgfqpoint{3.113971in}{2.253863in}}%
\pgfpathlineto{\pgfqpoint{3.105772in}{2.245091in}}%
\pgfpathlineto{\pgfqpoint{3.097566in}{2.236367in}}%
\pgfpathlineto{\pgfqpoint{3.089353in}{2.227690in}}%
\pgfpathlineto{\pgfqpoint{3.081133in}{2.219061in}}%
\pgfpathclose%
\pgfusepath{fill}%
\end{pgfscope}%
\begin{pgfscope}%
\pgfpathrectangle{\pgfqpoint{1.150000in}{0.150000in}}{\pgfqpoint{5.700000in}{5.700000in}}%
\pgfusepath{clip}%
\pgfsetbuttcap%
\pgfsetroundjoin%
\definecolor{currentfill}{rgb}{0.239346,0.300855,0.540844}%
\pgfsetfillcolor{currentfill}%
\pgfsetfillopacity{0.700000}%
\pgfsetlinewidth{0.000000pt}%
\definecolor{currentstroke}{rgb}{0.000000,0.000000,0.000000}%
\pgfsetstrokecolor{currentstroke}%
\pgfsetdash{}{0pt}%
\pgfpathmoveto{\pgfqpoint{5.549349in}{2.699832in}}%
\pgfpathlineto{\pgfqpoint{5.563383in}{2.698619in}}%
\pgfpathlineto{\pgfqpoint{5.577426in}{2.697471in}}%
\pgfpathlineto{\pgfqpoint{5.591479in}{2.696390in}}%
\pgfpathlineto{\pgfqpoint{5.605542in}{2.695375in}}%
\pgfpathlineto{\pgfqpoint{5.612902in}{2.704865in}}%
\pgfpathlineto{\pgfqpoint{5.620265in}{2.714597in}}%
\pgfpathlineto{\pgfqpoint{5.627630in}{2.724579in}}%
\pgfpathlineto{\pgfqpoint{5.634998in}{2.734820in}}%
\pgfpathlineto{\pgfqpoint{5.620959in}{2.736280in}}%
\pgfpathlineto{\pgfqpoint{5.606929in}{2.737806in}}%
\pgfpathlineto{\pgfqpoint{5.592909in}{2.739397in}}%
\pgfpathlineto{\pgfqpoint{5.578898in}{2.741055in}}%
\pgfpathlineto{\pgfqpoint{5.571507in}{2.730363in}}%
\pgfpathlineto{\pgfqpoint{5.564118in}{2.719934in}}%
\pgfpathlineto{\pgfqpoint{5.556732in}{2.709760in}}%
\pgfpathlineto{\pgfqpoint{5.549349in}{2.699832in}}%
\pgfpathclose%
\pgfusepath{fill}%
\end{pgfscope}%
\begin{pgfscope}%
\pgfpathrectangle{\pgfqpoint{1.150000in}{0.150000in}}{\pgfqpoint{5.700000in}{5.700000in}}%
\pgfusepath{clip}%
\pgfsetbuttcap%
\pgfsetroundjoin%
\definecolor{currentfill}{rgb}{0.276194,0.190074,0.493001}%
\pgfsetfillcolor{currentfill}%
\pgfsetfillopacity{0.700000}%
\pgfsetlinewidth{0.000000pt}%
\definecolor{currentstroke}{rgb}{0.000000,0.000000,0.000000}%
\pgfsetstrokecolor{currentstroke}%
\pgfsetdash{}{0pt}%
\pgfpathmoveto{\pgfqpoint{4.752577in}{2.460278in}}%
\pgfpathlineto{\pgfqpoint{4.766396in}{2.459157in}}%
\pgfpathlineto{\pgfqpoint{4.780223in}{2.458107in}}%
\pgfpathlineto{\pgfqpoint{4.794058in}{2.457128in}}%
\pgfpathlineto{\pgfqpoint{4.807903in}{2.456220in}}%
\pgfpathlineto{\pgfqpoint{4.815512in}{2.464399in}}%
\pgfpathlineto{\pgfqpoint{4.823116in}{2.472662in}}%
\pgfpathlineto{\pgfqpoint{4.830717in}{2.481015in}}%
\pgfpathlineto{\pgfqpoint{4.838314in}{2.489463in}}%
\pgfpathlineto{\pgfqpoint{4.824485in}{2.490655in}}%
\pgfpathlineto{\pgfqpoint{4.810665in}{2.491917in}}%
\pgfpathlineto{\pgfqpoint{4.796853in}{2.493250in}}%
\pgfpathlineto{\pgfqpoint{4.783050in}{2.494654in}}%
\pgfpathlineto{\pgfqpoint{4.775438in}{2.485915in}}%
\pgfpathlineto{\pgfqpoint{4.767822in}{2.477277in}}%
\pgfpathlineto{\pgfqpoint{4.760202in}{2.468732in}}%
\pgfpathlineto{\pgfqpoint{4.752577in}{2.460278in}}%
\pgfpathclose%
\pgfusepath{fill}%
\end{pgfscope}%
\begin{pgfscope}%
\pgfpathrectangle{\pgfqpoint{1.150000in}{0.150000in}}{\pgfqpoint{5.700000in}{5.700000in}}%
\pgfusepath{clip}%
\pgfsetbuttcap%
\pgfsetroundjoin%
\definecolor{currentfill}{rgb}{0.277941,0.056324,0.381191}%
\pgfsetfillcolor{currentfill}%
\pgfsetfillopacity{0.700000}%
\pgfsetlinewidth{0.000000pt}%
\definecolor{currentstroke}{rgb}{0.000000,0.000000,0.000000}%
\pgfsetstrokecolor{currentstroke}%
\pgfsetdash{}{0pt}%
\pgfpathmoveto{\pgfqpoint{3.221621in}{2.203845in}}%
\pgfpathlineto{\pgfqpoint{3.235092in}{2.198035in}}%
\pgfpathlineto{\pgfqpoint{3.248566in}{2.192323in}}%
\pgfpathlineto{\pgfqpoint{3.262044in}{2.186707in}}%
\pgfpathlineto{\pgfqpoint{3.275526in}{2.181185in}}%
\pgfpathlineto{\pgfqpoint{3.283680in}{2.190036in}}%
\pgfpathlineto{\pgfqpoint{3.291828in}{2.198920in}}%
\pgfpathlineto{\pgfqpoint{3.299969in}{2.207837in}}%
\pgfpathlineto{\pgfqpoint{3.308104in}{2.216788in}}%
\pgfpathlineto{\pgfqpoint{3.294635in}{2.222307in}}%
\pgfpathlineto{\pgfqpoint{3.281169in}{2.227921in}}%
\pgfpathlineto{\pgfqpoint{3.267707in}{2.233630in}}%
\pgfpathlineto{\pgfqpoint{3.254249in}{2.239437in}}%
\pgfpathlineto{\pgfqpoint{3.246101in}{2.230481in}}%
\pgfpathlineto{\pgfqpoint{3.237948in}{2.221564in}}%
\pgfpathlineto{\pgfqpoint{3.229788in}{2.212685in}}%
\pgfpathlineto{\pgfqpoint{3.221621in}{2.203845in}}%
\pgfpathclose%
\pgfusepath{fill}%
\end{pgfscope}%
\begin{pgfscope}%
\pgfpathrectangle{\pgfqpoint{1.150000in}{0.150000in}}{\pgfqpoint{5.700000in}{5.700000in}}%
\pgfusepath{clip}%
\pgfsetbuttcap%
\pgfsetroundjoin%
\definecolor{currentfill}{rgb}{0.280894,0.078907,0.402329}%
\pgfsetfillcolor{currentfill}%
\pgfsetfillopacity{0.700000}%
\pgfsetlinewidth{0.000000pt}%
\definecolor{currentstroke}{rgb}{0.000000,0.000000,0.000000}%
\pgfsetstrokecolor{currentstroke}%
\pgfsetdash{}{0pt}%
\pgfpathmoveto{\pgfqpoint{2.940485in}{2.242110in}}%
\pgfpathlineto{\pgfqpoint{2.953941in}{2.234604in}}%
\pgfpathlineto{\pgfqpoint{2.967398in}{2.227204in}}%
\pgfpathlineto{\pgfqpoint{2.980857in}{2.219912in}}%
\pgfpathlineto{\pgfqpoint{2.994318in}{2.212724in}}%
\pgfpathlineto{\pgfqpoint{3.002580in}{2.221112in}}%
\pgfpathlineto{\pgfqpoint{3.010834in}{2.229554in}}%
\pgfpathlineto{\pgfqpoint{3.019082in}{2.238047in}}%
\pgfpathlineto{\pgfqpoint{3.027322in}{2.246593in}}%
\pgfpathlineto{\pgfqpoint{3.013875in}{2.253737in}}%
\pgfpathlineto{\pgfqpoint{3.000431in}{2.260986in}}%
\pgfpathlineto{\pgfqpoint{2.986988in}{2.268341in}}%
\pgfpathlineto{\pgfqpoint{2.973548in}{2.275803in}}%
\pgfpathlineto{\pgfqpoint{2.965293in}{2.267294in}}%
\pgfpathlineto{\pgfqpoint{2.957031in}{2.258841in}}%
\pgfpathlineto{\pgfqpoint{2.948762in}{2.250447in}}%
\pgfpathlineto{\pgfqpoint{2.940485in}{2.242110in}}%
\pgfpathclose%
\pgfusepath{fill}%
\end{pgfscope}%
\begin{pgfscope}%
\pgfpathrectangle{\pgfqpoint{1.150000in}{0.150000in}}{\pgfqpoint{5.700000in}{5.700000in}}%
\pgfusepath{clip}%
\pgfsetbuttcap%
\pgfsetroundjoin%
\definecolor{currentfill}{rgb}{0.279566,0.067836,0.391917}%
\pgfsetfillcolor{currentfill}%
\pgfsetfillopacity{0.700000}%
\pgfsetlinewidth{0.000000pt}%
\definecolor{currentstroke}{rgb}{0.000000,0.000000,0.000000}%
\pgfsetstrokecolor{currentstroke}%
\pgfsetdash{}{0pt}%
\pgfpathmoveto{\pgfqpoint{3.588686in}{2.213072in}}%
\pgfpathlineto{\pgfqpoint{3.602209in}{2.209042in}}%
\pgfpathlineto{\pgfqpoint{3.615738in}{2.205100in}}%
\pgfpathlineto{\pgfqpoint{3.629272in}{2.201244in}}%
\pgfpathlineto{\pgfqpoint{3.642811in}{2.197474in}}%
\pgfpathlineto{\pgfqpoint{3.650835in}{2.206513in}}%
\pgfpathlineto{\pgfqpoint{3.658854in}{2.215569in}}%
\pgfpathlineto{\pgfqpoint{3.666866in}{2.224646in}}%
\pgfpathlineto{\pgfqpoint{3.674873in}{2.233744in}}%
\pgfpathlineto{\pgfqpoint{3.661345in}{2.237572in}}%
\pgfpathlineto{\pgfqpoint{3.647822in}{2.241487in}}%
\pgfpathlineto{\pgfqpoint{3.634304in}{2.245488in}}%
\pgfpathlineto{\pgfqpoint{3.620791in}{2.249576in}}%
\pgfpathlineto{\pgfqpoint{3.612774in}{2.240412in}}%
\pgfpathlineto{\pgfqpoint{3.604750in}{2.231274in}}%
\pgfpathlineto{\pgfqpoint{3.596721in}{2.222161in}}%
\pgfpathlineto{\pgfqpoint{3.588686in}{2.213072in}}%
\pgfpathclose%
\pgfusepath{fill}%
\end{pgfscope}%
\begin{pgfscope}%
\pgfpathrectangle{\pgfqpoint{1.150000in}{0.150000in}}{\pgfqpoint{5.700000in}{5.700000in}}%
\pgfusepath{clip}%
\pgfsetbuttcap%
\pgfsetroundjoin%
\definecolor{currentfill}{rgb}{0.281446,0.084320,0.407414}%
\pgfsetfillcolor{currentfill}%
\pgfsetfillopacity{0.700000}%
\pgfsetlinewidth{0.000000pt}%
\definecolor{currentstroke}{rgb}{0.000000,0.000000,0.000000}%
\pgfsetstrokecolor{currentstroke}%
\pgfsetdash{}{0pt}%
\pgfpathmoveto{\pgfqpoint{3.815190in}{2.242170in}}%
\pgfpathlineto{\pgfqpoint{3.828760in}{2.239020in}}%
\pgfpathlineto{\pgfqpoint{3.842336in}{2.235952in}}%
\pgfpathlineto{\pgfqpoint{3.855918in}{2.232967in}}%
\pgfpathlineto{\pgfqpoint{3.869506in}{2.230063in}}%
\pgfpathlineto{\pgfqpoint{3.877452in}{2.239017in}}%
\pgfpathlineto{\pgfqpoint{3.885393in}{2.247988in}}%
\pgfpathlineto{\pgfqpoint{3.893328in}{2.256977in}}%
\pgfpathlineto{\pgfqpoint{3.901257in}{2.265988in}}%
\pgfpathlineto{\pgfqpoint{3.887679in}{2.268991in}}%
\pgfpathlineto{\pgfqpoint{3.874108in}{2.272076in}}%
\pgfpathlineto{\pgfqpoint{3.860543in}{2.275244in}}%
\pgfpathlineto{\pgfqpoint{3.846984in}{2.278493in}}%
\pgfpathlineto{\pgfqpoint{3.839044in}{2.269375in}}%
\pgfpathlineto{\pgfqpoint{3.831098in}{2.260284in}}%
\pgfpathlineto{\pgfqpoint{3.823147in}{2.251216in}}%
\pgfpathlineto{\pgfqpoint{3.815190in}{2.242170in}}%
\pgfpathclose%
\pgfusepath{fill}%
\end{pgfscope}%
\begin{pgfscope}%
\pgfpathrectangle{\pgfqpoint{1.150000in}{0.150000in}}{\pgfqpoint{5.700000in}{5.700000in}}%
\pgfusepath{clip}%
\pgfsetbuttcap%
\pgfsetroundjoin%
\definecolor{currentfill}{rgb}{0.265145,0.232956,0.516599}%
\pgfsetfillcolor{currentfill}%
\pgfsetfillopacity{0.700000}%
\pgfsetlinewidth{0.000000pt}%
\definecolor{currentstroke}{rgb}{0.000000,0.000000,0.000000}%
\pgfsetstrokecolor{currentstroke}%
\pgfsetdash{}{0pt}%
\pgfpathmoveto{\pgfqpoint{5.065199in}{2.544769in}}%
\pgfpathlineto{\pgfqpoint{5.079110in}{2.543859in}}%
\pgfpathlineto{\pgfqpoint{5.093030in}{2.543017in}}%
\pgfpathlineto{\pgfqpoint{5.106959in}{2.542244in}}%
\pgfpathlineto{\pgfqpoint{5.120898in}{2.541540in}}%
\pgfpathlineto{\pgfqpoint{5.128396in}{2.549745in}}%
\pgfpathlineto{\pgfqpoint{5.135892in}{2.558082in}}%
\pgfpathlineto{\pgfqpoint{5.143385in}{2.566557in}}%
\pgfpathlineto{\pgfqpoint{5.150876in}{2.575177in}}%
\pgfpathlineto{\pgfqpoint{5.136956in}{2.576225in}}%
\pgfpathlineto{\pgfqpoint{5.123045in}{2.577343in}}%
\pgfpathlineto{\pgfqpoint{5.109143in}{2.578529in}}%
\pgfpathlineto{\pgfqpoint{5.095250in}{2.579784in}}%
\pgfpathlineto{\pgfqpoint{5.087741in}{2.570813in}}%
\pgfpathlineto{\pgfqpoint{5.080229in}{2.561991in}}%
\pgfpathlineto{\pgfqpoint{5.072716in}{2.553312in}}%
\pgfpathlineto{\pgfqpoint{5.065199in}{2.544769in}}%
\pgfpathclose%
\pgfusepath{fill}%
\end{pgfscope}%
\begin{pgfscope}%
\pgfpathrectangle{\pgfqpoint{1.150000in}{0.150000in}}{\pgfqpoint{5.700000in}{5.700000in}}%
\pgfusepath{clip}%
\pgfsetbuttcap%
\pgfsetroundjoin%
\definecolor{currentfill}{rgb}{0.244972,0.287675,0.537260}%
\pgfsetfillcolor{currentfill}%
\pgfsetfillopacity{0.700000}%
\pgfsetlinewidth{0.000000pt}%
\definecolor{currentstroke}{rgb}{0.000000,0.000000,0.000000}%
\pgfsetstrokecolor{currentstroke}%
\pgfsetdash{}{0pt}%
\pgfpathmoveto{\pgfqpoint{5.463701in}{2.666253in}}%
\pgfpathlineto{\pgfqpoint{5.477719in}{2.665198in}}%
\pgfpathlineto{\pgfqpoint{5.491747in}{2.664211in}}%
\pgfpathlineto{\pgfqpoint{5.505784in}{2.663290in}}%
\pgfpathlineto{\pgfqpoint{5.519831in}{2.662436in}}%
\pgfpathlineto{\pgfqpoint{5.527209in}{2.671453in}}%
\pgfpathlineto{\pgfqpoint{5.534587in}{2.680686in}}%
\pgfpathlineto{\pgfqpoint{5.541967in}{2.690143in}}%
\pgfpathlineto{\pgfqpoint{5.549349in}{2.699832in}}%
\pgfpathlineto{\pgfqpoint{5.535325in}{2.701112in}}%
\pgfpathlineto{\pgfqpoint{5.521310in}{2.702458in}}%
\pgfpathlineto{\pgfqpoint{5.507304in}{2.703870in}}%
\pgfpathlineto{\pgfqpoint{5.493308in}{2.705349in}}%
\pgfpathlineto{\pgfqpoint{5.485904in}{2.695228in}}%
\pgfpathlineto{\pgfqpoint{5.478502in}{2.685343in}}%
\pgfpathlineto{\pgfqpoint{5.471101in}{2.675687in}}%
\pgfpathlineto{\pgfqpoint{5.463701in}{2.666253in}}%
\pgfpathclose%
\pgfusepath{fill}%
\end{pgfscope}%
\begin{pgfscope}%
\pgfpathrectangle{\pgfqpoint{1.150000in}{0.150000in}}{\pgfqpoint{5.700000in}{5.700000in}}%
\pgfusepath{clip}%
\pgfsetbuttcap%
\pgfsetroundjoin%
\definecolor{currentfill}{rgb}{0.283072,0.130895,0.449241}%
\pgfsetfillcolor{currentfill}%
\pgfsetfillopacity{0.700000}%
\pgfsetlinewidth{0.000000pt}%
\definecolor{currentstroke}{rgb}{0.000000,0.000000,0.000000}%
\pgfsetstrokecolor{currentstroke}%
\pgfsetdash{}{0pt}%
\pgfpathmoveto{\pgfqpoint{2.604461in}{2.354441in}}%
\pgfpathlineto{\pgfqpoint{2.617936in}{2.344465in}}%
\pgfpathlineto{\pgfqpoint{2.631411in}{2.334614in}}%
\pgfpathlineto{\pgfqpoint{2.644886in}{2.324887in}}%
\pgfpathlineto{\pgfqpoint{2.658360in}{2.315283in}}%
\pgfpathlineto{\pgfqpoint{2.666767in}{2.322765in}}%
\pgfpathlineto{\pgfqpoint{2.675165in}{2.330330in}}%
\pgfpathlineto{\pgfqpoint{2.683554in}{2.337978in}}%
\pgfpathlineto{\pgfqpoint{2.691935in}{2.345707in}}%
\pgfpathlineto{\pgfqpoint{2.678479in}{2.355225in}}%
\pgfpathlineto{\pgfqpoint{2.665023in}{2.364866in}}%
\pgfpathlineto{\pgfqpoint{2.651566in}{2.374631in}}%
\pgfpathlineto{\pgfqpoint{2.638109in}{2.384520in}}%
\pgfpathlineto{\pgfqpoint{2.629710in}{2.376870in}}%
\pgfpathlineto{\pgfqpoint{2.621303in}{2.369306in}}%
\pgfpathlineto{\pgfqpoint{2.612886in}{2.361829in}}%
\pgfpathlineto{\pgfqpoint{2.604461in}{2.354441in}}%
\pgfpathclose%
\pgfusepath{fill}%
\end{pgfscope}%
\begin{pgfscope}%
\pgfpathrectangle{\pgfqpoint{1.150000in}{0.150000in}}{\pgfqpoint{5.700000in}{5.700000in}}%
\pgfusepath{clip}%
\pgfsetbuttcap%
\pgfsetroundjoin%
\definecolor{currentfill}{rgb}{0.282623,0.140926,0.457517}%
\pgfsetfillcolor{currentfill}%
\pgfsetfillopacity{0.700000}%
\pgfsetlinewidth{0.000000pt}%
\definecolor{currentstroke}{rgb}{0.000000,0.000000,0.000000}%
\pgfsetstrokecolor{currentstroke}%
\pgfsetdash{}{0pt}%
\pgfpathmoveto{\pgfqpoint{4.354218in}{2.352377in}}%
\pgfpathlineto{\pgfqpoint{4.367927in}{2.350724in}}%
\pgfpathlineto{\pgfqpoint{4.381644in}{2.349145in}}%
\pgfpathlineto{\pgfqpoint{4.395368in}{2.347642in}}%
\pgfpathlineto{\pgfqpoint{4.409100in}{2.346213in}}%
\pgfpathlineto{\pgfqpoint{4.416856in}{2.354641in}}%
\pgfpathlineto{\pgfqpoint{4.424608in}{2.363108in}}%
\pgfpathlineto{\pgfqpoint{4.432354in}{2.371618in}}%
\pgfpathlineto{\pgfqpoint{4.440095in}{2.380175in}}%
\pgfpathlineto{\pgfqpoint{4.426375in}{2.381806in}}%
\pgfpathlineto{\pgfqpoint{4.412663in}{2.383511in}}%
\pgfpathlineto{\pgfqpoint{4.398959in}{2.385291in}}%
\pgfpathlineto{\pgfqpoint{4.385263in}{2.387146in}}%
\pgfpathlineto{\pgfqpoint{4.377509in}{2.378380in}}%
\pgfpathlineto{\pgfqpoint{4.369751in}{2.369666in}}%
\pgfpathlineto{\pgfqpoint{4.361987in}{2.360999in}}%
\pgfpathlineto{\pgfqpoint{4.354218in}{2.352377in}}%
\pgfpathclose%
\pgfusepath{fill}%
\end{pgfscope}%
\begin{pgfscope}%
\pgfpathrectangle{\pgfqpoint{1.150000in}{0.150000in}}{\pgfqpoint{5.700000in}{5.700000in}}%
\pgfusepath{clip}%
\pgfsetbuttcap%
\pgfsetroundjoin%
\definecolor{currentfill}{rgb}{0.277941,0.056324,0.381191}%
\pgfsetfillcolor{currentfill}%
\pgfsetfillopacity{0.700000}%
\pgfsetlinewidth{0.000000pt}%
\definecolor{currentstroke}{rgb}{0.000000,0.000000,0.000000}%
\pgfsetstrokecolor{currentstroke}%
\pgfsetdash{}{0pt}%
\pgfpathmoveto{\pgfqpoint{3.362021in}{2.195658in}}%
\pgfpathlineto{\pgfqpoint{3.375511in}{2.190608in}}%
\pgfpathlineto{\pgfqpoint{3.389005in}{2.185651in}}%
\pgfpathlineto{\pgfqpoint{3.402503in}{2.180786in}}%
\pgfpathlineto{\pgfqpoint{3.416005in}{2.176012in}}%
\pgfpathlineto{\pgfqpoint{3.424110in}{2.184977in}}%
\pgfpathlineto{\pgfqpoint{3.432210in}{2.193967in}}%
\pgfpathlineto{\pgfqpoint{3.440303in}{2.202983in}}%
\pgfpathlineto{\pgfqpoint{3.448390in}{2.212026in}}%
\pgfpathlineto{\pgfqpoint{3.434899in}{2.216818in}}%
\pgfpathlineto{\pgfqpoint{3.421412in}{2.221701in}}%
\pgfpathlineto{\pgfqpoint{3.407930in}{2.226676in}}%
\pgfpathlineto{\pgfqpoint{3.394453in}{2.231743in}}%
\pgfpathlineto{\pgfqpoint{3.386354in}{2.222675in}}%
\pgfpathlineto{\pgfqpoint{3.378249in}{2.213639in}}%
\pgfpathlineto{\pgfqpoint{3.370138in}{2.204633in}}%
\pgfpathlineto{\pgfqpoint{3.362021in}{2.195658in}}%
\pgfpathclose%
\pgfusepath{fill}%
\end{pgfscope}%
\begin{pgfscope}%
\pgfpathrectangle{\pgfqpoint{1.150000in}{0.150000in}}{\pgfqpoint{5.700000in}{5.700000in}}%
\pgfusepath{clip}%
\pgfsetbuttcap%
\pgfsetroundjoin%
\definecolor{currentfill}{rgb}{0.282910,0.105393,0.426902}%
\pgfsetfillcolor{currentfill}%
\pgfsetfillopacity{0.700000}%
\pgfsetlinewidth{0.000000pt}%
\definecolor{currentstroke}{rgb}{0.000000,0.000000,0.000000}%
\pgfsetstrokecolor{currentstroke}%
\pgfsetdash{}{0pt}%
\pgfpathmoveto{\pgfqpoint{4.041687in}{2.280209in}}%
\pgfpathlineto{\pgfqpoint{4.055313in}{2.277807in}}%
\pgfpathlineto{\pgfqpoint{4.068946in}{2.275485in}}%
\pgfpathlineto{\pgfqpoint{4.082586in}{2.273241in}}%
\pgfpathlineto{\pgfqpoint{4.096233in}{2.271076in}}%
\pgfpathlineto{\pgfqpoint{4.104102in}{2.279830in}}%
\pgfpathlineto{\pgfqpoint{4.111965in}{2.288604in}}%
\pgfpathlineto{\pgfqpoint{4.119823in}{2.297402in}}%
\pgfpathlineto{\pgfqpoint{4.127675in}{2.306227in}}%
\pgfpathlineto{\pgfqpoint{4.114039in}{2.308533in}}%
\pgfpathlineto{\pgfqpoint{4.100410in}{2.310917in}}%
\pgfpathlineto{\pgfqpoint{4.086788in}{2.313380in}}%
\pgfpathlineto{\pgfqpoint{4.073173in}{2.315921in}}%
\pgfpathlineto{\pgfqpoint{4.065310in}{2.306949in}}%
\pgfpathlineto{\pgfqpoint{4.057441in}{2.298008in}}%
\pgfpathlineto{\pgfqpoint{4.049567in}{2.289095in}}%
\pgfpathlineto{\pgfqpoint{4.041687in}{2.280209in}}%
\pgfpathclose%
\pgfusepath{fill}%
\end{pgfscope}%
\begin{pgfscope}%
\pgfpathrectangle{\pgfqpoint{1.150000in}{0.150000in}}{\pgfqpoint{5.700000in}{5.700000in}}%
\pgfusepath{clip}%
\pgfsetbuttcap%
\pgfsetroundjoin%
\definecolor{currentfill}{rgb}{0.282327,0.094955,0.417331}%
\pgfsetfillcolor{currentfill}%
\pgfsetfillopacity{0.700000}%
\pgfsetlinewidth{0.000000pt}%
\definecolor{currentstroke}{rgb}{0.000000,0.000000,0.000000}%
\pgfsetstrokecolor{currentstroke}%
\pgfsetdash{}{0pt}%
\pgfpathmoveto{\pgfqpoint{2.799593in}{2.273866in}}%
\pgfpathlineto{\pgfqpoint{2.813053in}{2.265410in}}%
\pgfpathlineto{\pgfqpoint{2.826513in}{2.257068in}}%
\pgfpathlineto{\pgfqpoint{2.839975in}{2.248838in}}%
\pgfpathlineto{\pgfqpoint{2.853439in}{2.240720in}}%
\pgfpathlineto{\pgfqpoint{2.861761in}{2.248749in}}%
\pgfpathlineto{\pgfqpoint{2.870075in}{2.256842in}}%
\pgfpathlineto{\pgfqpoint{2.878382in}{2.265000in}}%
\pgfpathlineto{\pgfqpoint{2.886681in}{2.273221in}}%
\pgfpathlineto{\pgfqpoint{2.873234in}{2.281275in}}%
\pgfpathlineto{\pgfqpoint{2.859788in}{2.289440in}}%
\pgfpathlineto{\pgfqpoint{2.846343in}{2.297717in}}%
\pgfpathlineto{\pgfqpoint{2.832899in}{2.306108in}}%
\pgfpathlineto{\pgfqpoint{2.824585in}{2.297944in}}%
\pgfpathlineto{\pgfqpoint{2.816262in}{2.289848in}}%
\pgfpathlineto{\pgfqpoint{2.807931in}{2.281822in}}%
\pgfpathlineto{\pgfqpoint{2.799593in}{2.273866in}}%
\pgfpathclose%
\pgfusepath{fill}%
\end{pgfscope}%
\begin{pgfscope}%
\pgfpathrectangle{\pgfqpoint{1.150000in}{0.150000in}}{\pgfqpoint{5.700000in}{5.700000in}}%
\pgfusepath{clip}%
\pgfsetbuttcap%
\pgfsetroundjoin%
\definecolor{currentfill}{rgb}{0.278012,0.180367,0.486697}%
\pgfsetfillcolor{currentfill}%
\pgfsetfillopacity{0.700000}%
\pgfsetlinewidth{0.000000pt}%
\definecolor{currentstroke}{rgb}{0.000000,0.000000,0.000000}%
\pgfsetstrokecolor{currentstroke}%
\pgfsetdash{}{0pt}%
\pgfpathmoveto{\pgfqpoint{4.666790in}{2.431404in}}%
\pgfpathlineto{\pgfqpoint{4.680589in}{2.430259in}}%
\pgfpathlineto{\pgfqpoint{4.694396in}{2.429187in}}%
\pgfpathlineto{\pgfqpoint{4.708212in}{2.428186in}}%
\pgfpathlineto{\pgfqpoint{4.722036in}{2.427257in}}%
\pgfpathlineto{\pgfqpoint{4.729678in}{2.435402in}}%
\pgfpathlineto{\pgfqpoint{4.737316in}{2.443618in}}%
\pgfpathlineto{\pgfqpoint{4.744949in}{2.451908in}}%
\pgfpathlineto{\pgfqpoint{4.752577in}{2.460278in}}%
\pgfpathlineto{\pgfqpoint{4.738768in}{2.461470in}}%
\pgfpathlineto{\pgfqpoint{4.724966in}{2.462735in}}%
\pgfpathlineto{\pgfqpoint{4.711174in}{2.464070in}}%
\pgfpathlineto{\pgfqpoint{4.697389in}{2.465478in}}%
\pgfpathlineto{\pgfqpoint{4.689746in}{2.456837in}}%
\pgfpathlineto{\pgfqpoint{4.682098in}{2.448281in}}%
\pgfpathlineto{\pgfqpoint{4.674446in}{2.439805in}}%
\pgfpathlineto{\pgfqpoint{4.666790in}{2.431404in}}%
\pgfpathclose%
\pgfusepath{fill}%
\end{pgfscope}%
\begin{pgfscope}%
\pgfpathrectangle{\pgfqpoint{1.150000in}{0.150000in}}{\pgfqpoint{5.700000in}{5.700000in}}%
\pgfusepath{clip}%
\pgfsetbuttcap%
\pgfsetroundjoin%
\definecolor{currentfill}{rgb}{0.192357,0.403199,0.555836}%
\pgfsetfillcolor{currentfill}%
\pgfsetfillopacity{0.700000}%
\pgfsetlinewidth{0.000000pt}%
\definecolor{currentstroke}{rgb}{0.000000,0.000000,0.000000}%
\pgfsetstrokecolor{currentstroke}%
\pgfsetdash{}{0pt}%
\pgfpathmoveto{\pgfqpoint{6.120380in}{2.930929in}}%
\pgfpathlineto{\pgfqpoint{6.134528in}{2.928564in}}%
\pgfpathlineto{\pgfqpoint{6.148686in}{2.926263in}}%
\pgfpathlineto{\pgfqpoint{6.162853in}{2.924025in}}%
\pgfpathlineto{\pgfqpoint{6.170204in}{2.937946in}}%
\pgfpathlineto{\pgfqpoint{6.177568in}{2.952292in}}%
\pgfpathlineto{\pgfqpoint{6.184945in}{2.967075in}}%
\pgfpathlineto{\pgfqpoint{6.192337in}{2.982303in}}%
\pgfpathlineto{\pgfqpoint{6.178199in}{2.985105in}}%
\pgfpathlineto{\pgfqpoint{6.164070in}{2.987971in}}%
\pgfpathlineto{\pgfqpoint{6.149950in}{2.990901in}}%
\pgfpathlineto{\pgfqpoint{6.142537in}{2.975243in}}%
\pgfpathlineto{\pgfqpoint{6.135138in}{2.960036in}}%
\pgfpathlineto{\pgfqpoint{6.127752in}{2.945268in}}%
\pgfpathlineto{\pgfqpoint{6.120380in}{2.930929in}}%
\pgfpathclose%
\pgfusepath{fill}%
\end{pgfscope}%
\begin{pgfscope}%
\pgfpathrectangle{\pgfqpoint{1.150000in}{0.150000in}}{\pgfqpoint{5.700000in}{5.700000in}}%
\pgfusepath{clip}%
\pgfsetbuttcap%
\pgfsetroundjoin%
\definecolor{currentfill}{rgb}{0.201239,0.383670,0.554294}%
\pgfsetfillcolor{currentfill}%
\pgfsetfillopacity{0.700000}%
\pgfsetlinewidth{0.000000pt}%
\definecolor{currentstroke}{rgb}{0.000000,0.000000,0.000000}%
\pgfsetstrokecolor{currentstroke}%
\pgfsetdash{}{0pt}%
\pgfpathmoveto{\pgfqpoint{6.034404in}{2.885581in}}%
\pgfpathlineto{\pgfqpoint{6.048542in}{2.883506in}}%
\pgfpathlineto{\pgfqpoint{6.062689in}{2.881494in}}%
\pgfpathlineto{\pgfqpoint{6.076847in}{2.879546in}}%
\pgfpathlineto{\pgfqpoint{6.091014in}{2.877663in}}%
\pgfpathlineto{\pgfqpoint{6.098338in}{2.890386in}}%
\pgfpathlineto{\pgfqpoint{6.105674in}{2.903498in}}%
\pgfpathlineto{\pgfqpoint{6.113021in}{2.917009in}}%
\pgfpathlineto{\pgfqpoint{6.120380in}{2.930929in}}%
\pgfpathlineto{\pgfqpoint{6.106242in}{2.933357in}}%
\pgfpathlineto{\pgfqpoint{6.092113in}{2.935850in}}%
\pgfpathlineto{\pgfqpoint{6.077993in}{2.938406in}}%
\pgfpathlineto{\pgfqpoint{6.063883in}{2.941026in}}%
\pgfpathlineto{\pgfqpoint{6.056496in}{2.926554in}}%
\pgfpathlineto{\pgfqpoint{6.049121in}{2.912496in}}%
\pgfpathlineto{\pgfqpoint{6.041757in}{2.898842in}}%
\pgfpathlineto{\pgfqpoint{6.034404in}{2.885581in}}%
\pgfpathclose%
\pgfusepath{fill}%
\end{pgfscope}%
\begin{pgfscope}%
\pgfpathrectangle{\pgfqpoint{1.150000in}{0.150000in}}{\pgfqpoint{5.700000in}{5.700000in}}%
\pgfusepath{clip}%
\pgfsetbuttcap%
\pgfsetroundjoin%
\definecolor{currentfill}{rgb}{0.248629,0.278775,0.534556}%
\pgfsetfillcolor{currentfill}%
\pgfsetfillopacity{0.700000}%
\pgfsetlinewidth{0.000000pt}%
\definecolor{currentstroke}{rgb}{0.000000,0.000000,0.000000}%
\pgfsetstrokecolor{currentstroke}%
\pgfsetdash{}{0pt}%
\pgfpathmoveto{\pgfqpoint{5.378042in}{2.633840in}}%
\pgfpathlineto{\pgfqpoint{5.392043in}{2.632923in}}%
\pgfpathlineto{\pgfqpoint{5.406055in}{2.632074in}}%
\pgfpathlineto{\pgfqpoint{5.420075in}{2.631291in}}%
\pgfpathlineto{\pgfqpoint{5.434106in}{2.630576in}}%
\pgfpathlineto{\pgfqpoint{5.441504in}{2.639200in}}%
\pgfpathlineto{\pgfqpoint{5.448903in}{2.648016in}}%
\pgfpathlineto{\pgfqpoint{5.456302in}{2.657031in}}%
\pgfpathlineto{\pgfqpoint{5.463701in}{2.666253in}}%
\pgfpathlineto{\pgfqpoint{5.449692in}{2.667373in}}%
\pgfpathlineto{\pgfqpoint{5.435693in}{2.668561in}}%
\pgfpathlineto{\pgfqpoint{5.421704in}{2.669815in}}%
\pgfpathlineto{\pgfqpoint{5.407723in}{2.671137in}}%
\pgfpathlineto{\pgfqpoint{5.400302in}{2.661503in}}%
\pgfpathlineto{\pgfqpoint{5.392882in}{2.652080in}}%
\pgfpathlineto{\pgfqpoint{5.385462in}{2.642862in}}%
\pgfpathlineto{\pgfqpoint{5.378042in}{2.633840in}}%
\pgfpathclose%
\pgfusepath{fill}%
\end{pgfscope}%
\begin{pgfscope}%
\pgfpathrectangle{\pgfqpoint{1.150000in}{0.150000in}}{\pgfqpoint{5.700000in}{5.700000in}}%
\pgfusepath{clip}%
\pgfsetbuttcap%
\pgfsetroundjoin%
\definecolor{currentfill}{rgb}{0.208623,0.367752,0.552675}%
\pgfsetfillcolor{currentfill}%
\pgfsetfillopacity{0.700000}%
\pgfsetlinewidth{0.000000pt}%
\definecolor{currentstroke}{rgb}{0.000000,0.000000,0.000000}%
\pgfsetstrokecolor{currentstroke}%
\pgfsetdash{}{0pt}%
\pgfpathmoveto{\pgfqpoint{5.948528in}{2.843131in}}%
\pgfpathlineto{\pgfqpoint{5.962654in}{2.841323in}}%
\pgfpathlineto{\pgfqpoint{5.976791in}{2.839580in}}%
\pgfpathlineto{\pgfqpoint{5.990937in}{2.837901in}}%
\pgfpathlineto{\pgfqpoint{6.005093in}{2.836287in}}%
\pgfpathlineto{\pgfqpoint{6.012407in}{2.848067in}}%
\pgfpathlineto{\pgfqpoint{6.019730in}{2.860204in}}%
\pgfpathlineto{\pgfqpoint{6.027062in}{2.872705in}}%
\pgfpathlineto{\pgfqpoint{6.034404in}{2.885581in}}%
\pgfpathlineto{\pgfqpoint{6.020276in}{2.887721in}}%
\pgfpathlineto{\pgfqpoint{6.006157in}{2.889925in}}%
\pgfpathlineto{\pgfqpoint{5.992048in}{2.892193in}}%
\pgfpathlineto{\pgfqpoint{5.977949in}{2.894525in}}%
\pgfpathlineto{\pgfqpoint{5.970579in}{2.881117in}}%
\pgfpathlineto{\pgfqpoint{5.963219in}{2.868088in}}%
\pgfpathlineto{\pgfqpoint{5.955869in}{2.855429in}}%
\pgfpathlineto{\pgfqpoint{5.948528in}{2.843131in}}%
\pgfpathclose%
\pgfusepath{fill}%
\end{pgfscope}%
\begin{pgfscope}%
\pgfpathrectangle{\pgfqpoint{1.150000in}{0.150000in}}{\pgfqpoint{5.700000in}{5.700000in}}%
\pgfusepath{clip}%
\pgfsetbuttcap%
\pgfsetroundjoin%
\definecolor{currentfill}{rgb}{0.267968,0.223549,0.512008}%
\pgfsetfillcolor{currentfill}%
\pgfsetfillopacity{0.700000}%
\pgfsetlinewidth{0.000000pt}%
\definecolor{currentstroke}{rgb}{0.000000,0.000000,0.000000}%
\pgfsetstrokecolor{currentstroke}%
\pgfsetdash{}{0pt}%
\pgfpathmoveto{\pgfqpoint{4.979481in}{2.514886in}}%
\pgfpathlineto{\pgfqpoint{4.993373in}{2.514023in}}%
\pgfpathlineto{\pgfqpoint{5.007274in}{2.513229in}}%
\pgfpathlineto{\pgfqpoint{5.021185in}{2.512504in}}%
\pgfpathlineto{\pgfqpoint{5.035104in}{2.511849in}}%
\pgfpathlineto{\pgfqpoint{5.042633in}{2.519903in}}%
\pgfpathlineto{\pgfqpoint{5.050158in}{2.528071in}}%
\pgfpathlineto{\pgfqpoint{5.057680in}{2.536358in}}%
\pgfpathlineto{\pgfqpoint{5.065199in}{2.544769in}}%
\pgfpathlineto{\pgfqpoint{5.051297in}{2.545749in}}%
\pgfpathlineto{\pgfqpoint{5.037405in}{2.546798in}}%
\pgfpathlineto{\pgfqpoint{5.023521in}{2.547916in}}%
\pgfpathlineto{\pgfqpoint{5.009646in}{2.549104in}}%
\pgfpathlineto{\pgfqpoint{5.002109in}{2.540361in}}%
\pgfpathlineto{\pgfqpoint{4.994570in}{2.531748in}}%
\pgfpathlineto{\pgfqpoint{4.987027in}{2.523258in}}%
\pgfpathlineto{\pgfqpoint{4.979481in}{2.514886in}}%
\pgfpathclose%
\pgfusepath{fill}%
\end{pgfscope}%
\begin{pgfscope}%
\pgfpathrectangle{\pgfqpoint{1.150000in}{0.150000in}}{\pgfqpoint{5.700000in}{5.700000in}}%
\pgfusepath{clip}%
\pgfsetbuttcap%
\pgfsetroundjoin%
\definecolor{currentfill}{rgb}{0.280267,0.073417,0.397163}%
\pgfsetfillcolor{currentfill}%
\pgfsetfillopacity{0.700000}%
\pgfsetlinewidth{0.000000pt}%
\definecolor{currentstroke}{rgb}{0.000000,0.000000,0.000000}%
\pgfsetstrokecolor{currentstroke}%
\pgfsetdash{}{0pt}%
\pgfpathmoveto{\pgfqpoint{3.729043in}{2.219282in}}%
\pgfpathlineto{\pgfqpoint{3.742600in}{2.215878in}}%
\pgfpathlineto{\pgfqpoint{3.756163in}{2.212557in}}%
\pgfpathlineto{\pgfqpoint{3.769731in}{2.209321in}}%
\pgfpathlineto{\pgfqpoint{3.783306in}{2.206167in}}%
\pgfpathlineto{\pgfqpoint{3.791285in}{2.215144in}}%
\pgfpathlineto{\pgfqpoint{3.799259in}{2.224136in}}%
\pgfpathlineto{\pgfqpoint{3.807227in}{2.233144in}}%
\pgfpathlineto{\pgfqpoint{3.815190in}{2.242170in}}%
\pgfpathlineto{\pgfqpoint{3.801626in}{2.245403in}}%
\pgfpathlineto{\pgfqpoint{3.788068in}{2.248719in}}%
\pgfpathlineto{\pgfqpoint{3.774516in}{2.252119in}}%
\pgfpathlineto{\pgfqpoint{3.760970in}{2.255602in}}%
\pgfpathlineto{\pgfqpoint{3.752997in}{2.246489in}}%
\pgfpathlineto{\pgfqpoint{3.745018in}{2.237399in}}%
\pgfpathlineto{\pgfqpoint{3.737034in}{2.228331in}}%
\pgfpathlineto{\pgfqpoint{3.729043in}{2.219282in}}%
\pgfpathclose%
\pgfusepath{fill}%
\end{pgfscope}%
\begin{pgfscope}%
\pgfpathrectangle{\pgfqpoint{1.150000in}{0.150000in}}{\pgfqpoint{5.700000in}{5.700000in}}%
\pgfusepath{clip}%
\pgfsetbuttcap%
\pgfsetroundjoin%
\definecolor{currentfill}{rgb}{0.278791,0.062145,0.386592}%
\pgfsetfillcolor{currentfill}%
\pgfsetfillopacity{0.700000}%
\pgfsetlinewidth{0.000000pt}%
\definecolor{currentstroke}{rgb}{0.000000,0.000000,0.000000}%
\pgfsetstrokecolor{currentstroke}%
\pgfsetdash{}{0pt}%
\pgfpathmoveto{\pgfqpoint{3.502400in}{2.193761in}}%
\pgfpathlineto{\pgfqpoint{3.515914in}{2.189419in}}%
\pgfpathlineto{\pgfqpoint{3.529433in}{2.185165in}}%
\pgfpathlineto{\pgfqpoint{3.542958in}{2.180999in}}%
\pgfpathlineto{\pgfqpoint{3.556487in}{2.176921in}}%
\pgfpathlineto{\pgfqpoint{3.564545in}{2.185930in}}%
\pgfpathlineto{\pgfqpoint{3.572598in}{2.194957in}}%
\pgfpathlineto{\pgfqpoint{3.580645in}{2.204004in}}%
\pgfpathlineto{\pgfqpoint{3.588686in}{2.213072in}}%
\pgfpathlineto{\pgfqpoint{3.575167in}{2.217188in}}%
\pgfpathlineto{\pgfqpoint{3.561654in}{2.221392in}}%
\pgfpathlineto{\pgfqpoint{3.548146in}{2.225685in}}%
\pgfpathlineto{\pgfqpoint{3.534643in}{2.230066in}}%
\pgfpathlineto{\pgfqpoint{3.526591in}{2.220952in}}%
\pgfpathlineto{\pgfqpoint{3.518533in}{2.211864in}}%
\pgfpathlineto{\pgfqpoint{3.510469in}{2.202801in}}%
\pgfpathlineto{\pgfqpoint{3.502400in}{2.193761in}}%
\pgfpathclose%
\pgfusepath{fill}%
\end{pgfscope}%
\begin{pgfscope}%
\pgfpathrectangle{\pgfqpoint{1.150000in}{0.150000in}}{\pgfqpoint{5.700000in}{5.700000in}}%
\pgfusepath{clip}%
\pgfsetbuttcap%
\pgfsetroundjoin%
\definecolor{currentfill}{rgb}{0.277134,0.185228,0.489898}%
\pgfsetfillcolor{currentfill}%
\pgfsetfillopacity{0.700000}%
\pgfsetlinewidth{0.000000pt}%
\definecolor{currentstroke}{rgb}{0.000000,0.000000,0.000000}%
\pgfsetstrokecolor{currentstroke}%
\pgfsetdash{}{0pt}%
\pgfpathmoveto{\pgfqpoint{2.408630in}{2.456952in}}%
\pgfpathlineto{\pgfqpoint{2.422143in}{2.445281in}}%
\pgfpathlineto{\pgfqpoint{2.435654in}{2.433748in}}%
\pgfpathlineto{\pgfqpoint{2.449163in}{2.422353in}}%
\pgfpathlineto{\pgfqpoint{2.462670in}{2.411095in}}%
\pgfpathlineto{\pgfqpoint{2.471173in}{2.417887in}}%
\pgfpathlineto{\pgfqpoint{2.479666in}{2.424782in}}%
\pgfpathlineto{\pgfqpoint{2.488149in}{2.431780in}}%
\pgfpathlineto{\pgfqpoint{2.496622in}{2.438878in}}%
\pgfpathlineto{\pgfqpoint{2.483136in}{2.450028in}}%
\pgfpathlineto{\pgfqpoint{2.469648in}{2.461315in}}%
\pgfpathlineto{\pgfqpoint{2.456158in}{2.472739in}}%
\pgfpathlineto{\pgfqpoint{2.442666in}{2.484301in}}%
\pgfpathlineto{\pgfqpoint{2.434172in}{2.477304in}}%
\pgfpathlineto{\pgfqpoint{2.425668in}{2.470412in}}%
\pgfpathlineto{\pgfqpoint{2.417154in}{2.463628in}}%
\pgfpathlineto{\pgfqpoint{2.408630in}{2.456952in}}%
\pgfpathclose%
\pgfusepath{fill}%
\end{pgfscope}%
\begin{pgfscope}%
\pgfpathrectangle{\pgfqpoint{1.150000in}{0.150000in}}{\pgfqpoint{5.700000in}{5.700000in}}%
\pgfusepath{clip}%
\pgfsetbuttcap%
\pgfsetroundjoin%
\definecolor{currentfill}{rgb}{0.216210,0.351535,0.550627}%
\pgfsetfillcolor{currentfill}%
\pgfsetfillopacity{0.700000}%
\pgfsetlinewidth{0.000000pt}%
\definecolor{currentstroke}{rgb}{0.000000,0.000000,0.000000}%
\pgfsetstrokecolor{currentstroke}%
\pgfsetdash{}{0pt}%
\pgfpathmoveto{\pgfqpoint{5.862725in}{2.803212in}}%
\pgfpathlineto{\pgfqpoint{5.876839in}{2.801651in}}%
\pgfpathlineto{\pgfqpoint{5.890964in}{2.800155in}}%
\pgfpathlineto{\pgfqpoint{5.905098in}{2.798724in}}%
\pgfpathlineto{\pgfqpoint{5.919242in}{2.797357in}}%
\pgfpathlineto{\pgfqpoint{5.926552in}{2.808306in}}%
\pgfpathlineto{\pgfqpoint{5.933870in}{2.819578in}}%
\pgfpathlineto{\pgfqpoint{5.941195in}{2.831183in}}%
\pgfpathlineto{\pgfqpoint{5.948528in}{2.843131in}}%
\pgfpathlineto{\pgfqpoint{5.934411in}{2.845003in}}%
\pgfpathlineto{\pgfqpoint{5.920303in}{2.846940in}}%
\pgfpathlineto{\pgfqpoint{5.906206in}{2.848941in}}%
\pgfpathlineto{\pgfqpoint{5.892118in}{2.851007in}}%
\pgfpathlineto{\pgfqpoint{5.884758in}{2.838547in}}%
\pgfpathlineto{\pgfqpoint{5.877406in}{2.826434in}}%
\pgfpathlineto{\pgfqpoint{5.870062in}{2.814659in}}%
\pgfpathlineto{\pgfqpoint{5.862725in}{2.803212in}}%
\pgfpathclose%
\pgfusepath{fill}%
\end{pgfscope}%
\begin{pgfscope}%
\pgfpathrectangle{\pgfqpoint{1.150000in}{0.150000in}}{\pgfqpoint{5.700000in}{5.700000in}}%
\pgfusepath{clip}%
\pgfsetbuttcap%
\pgfsetroundjoin%
\definecolor{currentfill}{rgb}{0.283072,0.130895,0.449241}%
\pgfsetfillcolor{currentfill}%
\pgfsetfillopacity{0.700000}%
\pgfsetlinewidth{0.000000pt}%
\definecolor{currentstroke}{rgb}{0.000000,0.000000,0.000000}%
\pgfsetstrokecolor{currentstroke}%
\pgfsetdash{}{0pt}%
\pgfpathmoveto{\pgfqpoint{4.268284in}{2.324900in}}%
\pgfpathlineto{\pgfqpoint{4.281974in}{2.323126in}}%
\pgfpathlineto{\pgfqpoint{4.295672in}{2.321428in}}%
\pgfpathlineto{\pgfqpoint{4.309377in}{2.319806in}}%
\pgfpathlineto{\pgfqpoint{4.323090in}{2.318259in}}%
\pgfpathlineto{\pgfqpoint{4.330880in}{2.326740in}}%
\pgfpathlineto{\pgfqpoint{4.338665in}{2.335251in}}%
\pgfpathlineto{\pgfqpoint{4.346444in}{2.343795in}}%
\pgfpathlineto{\pgfqpoint{4.354218in}{2.352377in}}%
\pgfpathlineto{\pgfqpoint{4.340518in}{2.354105in}}%
\pgfpathlineto{\pgfqpoint{4.326824in}{2.355909in}}%
\pgfpathlineto{\pgfqpoint{4.313139in}{2.357788in}}%
\pgfpathlineto{\pgfqpoint{4.299460in}{2.359743in}}%
\pgfpathlineto{\pgfqpoint{4.291674in}{2.350973in}}%
\pgfpathlineto{\pgfqpoint{4.283883in}{2.342245in}}%
\pgfpathlineto{\pgfqpoint{4.276086in}{2.333555in}}%
\pgfpathlineto{\pgfqpoint{4.268284in}{2.324900in}}%
\pgfpathclose%
\pgfusepath{fill}%
\end{pgfscope}%
\begin{pgfscope}%
\pgfpathrectangle{\pgfqpoint{1.150000in}{0.150000in}}{\pgfqpoint{5.700000in}{5.700000in}}%
\pgfusepath{clip}%
\pgfsetbuttcap%
\pgfsetroundjoin%
\definecolor{currentfill}{rgb}{0.279574,0.170599,0.479997}%
\pgfsetfillcolor{currentfill}%
\pgfsetfillopacity{0.700000}%
\pgfsetlinewidth{0.000000pt}%
\definecolor{currentstroke}{rgb}{0.000000,0.000000,0.000000}%
\pgfsetstrokecolor{currentstroke}%
\pgfsetdash{}{0pt}%
\pgfpathmoveto{\pgfqpoint{4.580949in}{2.402787in}}%
\pgfpathlineto{\pgfqpoint{4.594728in}{2.401595in}}%
\pgfpathlineto{\pgfqpoint{4.608516in}{2.400477in}}%
\pgfpathlineto{\pgfqpoint{4.622312in}{2.399431in}}%
\pgfpathlineto{\pgfqpoint{4.636116in}{2.398457in}}%
\pgfpathlineto{\pgfqpoint{4.643792in}{2.406604in}}%
\pgfpathlineto{\pgfqpoint{4.651463in}{2.414808in}}%
\pgfpathlineto{\pgfqpoint{4.659128in}{2.423073in}}%
\pgfpathlineto{\pgfqpoint{4.666790in}{2.431404in}}%
\pgfpathlineto{\pgfqpoint{4.652999in}{2.432621in}}%
\pgfpathlineto{\pgfqpoint{4.639217in}{2.433910in}}%
\pgfpathlineto{\pgfqpoint{4.625443in}{2.435271in}}%
\pgfpathlineto{\pgfqpoint{4.611678in}{2.436705in}}%
\pgfpathlineto{\pgfqpoint{4.604003in}{2.428124in}}%
\pgfpathlineto{\pgfqpoint{4.596323in}{2.419614in}}%
\pgfpathlineto{\pgfqpoint{4.588638in}{2.411169in}}%
\pgfpathlineto{\pgfqpoint{4.580949in}{2.402787in}}%
\pgfpathclose%
\pgfusepath{fill}%
\end{pgfscope}%
\begin{pgfscope}%
\pgfpathrectangle{\pgfqpoint{1.150000in}{0.150000in}}{\pgfqpoint{5.700000in}{5.700000in}}%
\pgfusepath{clip}%
\pgfsetbuttcap%
\pgfsetroundjoin%
\definecolor{currentfill}{rgb}{0.282327,0.094955,0.417331}%
\pgfsetfillcolor{currentfill}%
\pgfsetfillopacity{0.700000}%
\pgfsetlinewidth{0.000000pt}%
\definecolor{currentstroke}{rgb}{0.000000,0.000000,0.000000}%
\pgfsetstrokecolor{currentstroke}%
\pgfsetdash{}{0pt}%
\pgfpathmoveto{\pgfqpoint{3.955632in}{2.254786in}}%
\pgfpathlineto{\pgfqpoint{3.969242in}{2.252187in}}%
\pgfpathlineto{\pgfqpoint{3.982858in}{2.249668in}}%
\pgfpathlineto{\pgfqpoint{3.996482in}{2.247228in}}%
\pgfpathlineto{\pgfqpoint{4.010112in}{2.244868in}}%
\pgfpathlineto{\pgfqpoint{4.018014in}{2.253677in}}%
\pgfpathlineto{\pgfqpoint{4.025911in}{2.262502in}}%
\pgfpathlineto{\pgfqpoint{4.033801in}{2.271345in}}%
\pgfpathlineto{\pgfqpoint{4.041687in}{2.280209in}}%
\pgfpathlineto{\pgfqpoint{4.028068in}{2.282689in}}%
\pgfpathlineto{\pgfqpoint{4.014455in}{2.285249in}}%
\pgfpathlineto{\pgfqpoint{4.000849in}{2.287888in}}%
\pgfpathlineto{\pgfqpoint{3.987250in}{2.290607in}}%
\pgfpathlineto{\pgfqpoint{3.979354in}{2.281616in}}%
\pgfpathlineto{\pgfqpoint{3.971452in}{2.272650in}}%
\pgfpathlineto{\pgfqpoint{3.963545in}{2.263708in}}%
\pgfpathlineto{\pgfqpoint{3.955632in}{2.254786in}}%
\pgfpathclose%
\pgfusepath{fill}%
\end{pgfscope}%
\begin{pgfscope}%
\pgfpathrectangle{\pgfqpoint{1.150000in}{0.150000in}}{\pgfqpoint{5.700000in}{5.700000in}}%
\pgfusepath{clip}%
\pgfsetbuttcap%
\pgfsetroundjoin%
\definecolor{currentfill}{rgb}{0.283229,0.120777,0.440584}%
\pgfsetfillcolor{currentfill}%
\pgfsetfillopacity{0.700000}%
\pgfsetlinewidth{0.000000pt}%
\definecolor{currentstroke}{rgb}{0.000000,0.000000,0.000000}%
\pgfsetstrokecolor{currentstroke}%
\pgfsetdash{}{0pt}%
\pgfpathmoveto{\pgfqpoint{2.658360in}{2.315283in}}%
\pgfpathlineto{\pgfqpoint{2.671834in}{2.305800in}}%
\pgfpathlineto{\pgfqpoint{2.685308in}{2.296439in}}%
\pgfpathlineto{\pgfqpoint{2.698782in}{2.287197in}}%
\pgfpathlineto{\pgfqpoint{2.712257in}{2.278075in}}%
\pgfpathlineto{\pgfqpoint{2.720645in}{2.285650in}}%
\pgfpathlineto{\pgfqpoint{2.729025in}{2.293304in}}%
\pgfpathlineto{\pgfqpoint{2.737397in}{2.301035in}}%
\pgfpathlineto{\pgfqpoint{2.745761in}{2.308843in}}%
\pgfpathlineto{\pgfqpoint{2.732304in}{2.317879in}}%
\pgfpathlineto{\pgfqpoint{2.718848in}{2.327035in}}%
\pgfpathlineto{\pgfqpoint{2.705391in}{2.336311in}}%
\pgfpathlineto{\pgfqpoint{2.691935in}{2.345707in}}%
\pgfpathlineto{\pgfqpoint{2.683554in}{2.337978in}}%
\pgfpathlineto{\pgfqpoint{2.675165in}{2.330330in}}%
\pgfpathlineto{\pgfqpoint{2.666767in}{2.322765in}}%
\pgfpathlineto{\pgfqpoint{2.658360in}{2.315283in}}%
\pgfpathclose%
\pgfusepath{fill}%
\end{pgfscope}%
\begin{pgfscope}%
\pgfpathrectangle{\pgfqpoint{1.150000in}{0.150000in}}{\pgfqpoint{5.700000in}{5.700000in}}%
\pgfusepath{clip}%
\pgfsetbuttcap%
\pgfsetroundjoin%
\definecolor{currentfill}{rgb}{0.277941,0.056324,0.381191}%
\pgfsetfillcolor{currentfill}%
\pgfsetfillopacity{0.700000}%
\pgfsetlinewidth{0.000000pt}%
\definecolor{currentstroke}{rgb}{0.000000,0.000000,0.000000}%
\pgfsetstrokecolor{currentstroke}%
\pgfsetdash{}{0pt}%
\pgfpathmoveto{\pgfqpoint{3.134987in}{2.193162in}}%
\pgfpathlineto{\pgfqpoint{3.148458in}{2.186938in}}%
\pgfpathlineto{\pgfqpoint{3.161932in}{2.180813in}}%
\pgfpathlineto{\pgfqpoint{3.175409in}{2.174786in}}%
\pgfpathlineto{\pgfqpoint{3.188890in}{2.168857in}}%
\pgfpathlineto{\pgfqpoint{3.197082in}{2.177548in}}%
\pgfpathlineto{\pgfqpoint{3.205268in}{2.186276in}}%
\pgfpathlineto{\pgfqpoint{3.213448in}{2.195042in}}%
\pgfpathlineto{\pgfqpoint{3.221621in}{2.203845in}}%
\pgfpathlineto{\pgfqpoint{3.208153in}{2.209751in}}%
\pgfpathlineto{\pgfqpoint{3.194689in}{2.215755in}}%
\pgfpathlineto{\pgfqpoint{3.181229in}{2.221857in}}%
\pgfpathlineto{\pgfqpoint{3.167771in}{2.228058in}}%
\pgfpathlineto{\pgfqpoint{3.159585in}{2.219270in}}%
\pgfpathlineto{\pgfqpoint{3.151392in}{2.210525in}}%
\pgfpathlineto{\pgfqpoint{3.143193in}{2.201823in}}%
\pgfpathlineto{\pgfqpoint{3.134987in}{2.193162in}}%
\pgfpathclose%
\pgfusepath{fill}%
\end{pgfscope}%
\begin{pgfscope}%
\pgfpathrectangle{\pgfqpoint{1.150000in}{0.150000in}}{\pgfqpoint{5.700000in}{5.700000in}}%
\pgfusepath{clip}%
\pgfsetbuttcap%
\pgfsetroundjoin%
\definecolor{currentfill}{rgb}{0.223925,0.334994,0.548053}%
\pgfsetfillcolor{currentfill}%
\pgfsetfillopacity{0.700000}%
\pgfsetlinewidth{0.000000pt}%
\definecolor{currentstroke}{rgb}{0.000000,0.000000,0.000000}%
\pgfsetstrokecolor{currentstroke}%
\pgfsetdash{}{0pt}%
\pgfpathmoveto{\pgfqpoint{5.776973in}{2.765486in}}%
\pgfpathlineto{\pgfqpoint{5.791074in}{2.764150in}}%
\pgfpathlineto{\pgfqpoint{5.805185in}{2.762880in}}%
\pgfpathlineto{\pgfqpoint{5.819306in}{2.761675in}}%
\pgfpathlineto{\pgfqpoint{5.833437in}{2.760535in}}%
\pgfpathlineto{\pgfqpoint{5.840751in}{2.770755in}}%
\pgfpathlineto{\pgfqpoint{5.848070in}{2.781269in}}%
\pgfpathlineto{\pgfqpoint{5.855394in}{2.792085in}}%
\pgfpathlineto{\pgfqpoint{5.862725in}{2.803212in}}%
\pgfpathlineto{\pgfqpoint{5.848620in}{2.804838in}}%
\pgfpathlineto{\pgfqpoint{5.834525in}{2.806529in}}%
\pgfpathlineto{\pgfqpoint{5.820439in}{2.808284in}}%
\pgfpathlineto{\pgfqpoint{5.806364in}{2.810105in}}%
\pgfpathlineto{\pgfqpoint{5.799007in}{2.798485in}}%
\pgfpathlineto{\pgfqpoint{5.791657in}{2.787181in}}%
\pgfpathlineto{\pgfqpoint{5.784312in}{2.776184in}}%
\pgfpathlineto{\pgfqpoint{5.776973in}{2.765486in}}%
\pgfpathclose%
\pgfusepath{fill}%
\end{pgfscope}%
\begin{pgfscope}%
\pgfpathrectangle{\pgfqpoint{1.150000in}{0.150000in}}{\pgfqpoint{5.700000in}{5.700000in}}%
\pgfusepath{clip}%
\pgfsetbuttcap%
\pgfsetroundjoin%
\definecolor{currentfill}{rgb}{0.279566,0.067836,0.391917}%
\pgfsetfillcolor{currentfill}%
\pgfsetfillopacity{0.700000}%
\pgfsetlinewidth{0.000000pt}%
\definecolor{currentstroke}{rgb}{0.000000,0.000000,0.000000}%
\pgfsetstrokecolor{currentstroke}%
\pgfsetdash{}{0pt}%
\pgfpathmoveto{\pgfqpoint{2.994318in}{2.212724in}}%
\pgfpathlineto{\pgfqpoint{3.007782in}{2.205642in}}%
\pgfpathlineto{\pgfqpoint{3.021248in}{2.198664in}}%
\pgfpathlineto{\pgfqpoint{3.034716in}{2.191789in}}%
\pgfpathlineto{\pgfqpoint{3.048187in}{2.185017in}}%
\pgfpathlineto{\pgfqpoint{3.056434in}{2.193457in}}%
\pgfpathlineto{\pgfqpoint{3.064674in}{2.201944in}}%
\pgfpathlineto{\pgfqpoint{3.072907in}{2.210479in}}%
\pgfpathlineto{\pgfqpoint{3.081133in}{2.219061in}}%
\pgfpathlineto{\pgfqpoint{3.067677in}{2.225789in}}%
\pgfpathlineto{\pgfqpoint{3.054223in}{2.232620in}}%
\pgfpathlineto{\pgfqpoint{3.040771in}{2.239555in}}%
\pgfpathlineto{\pgfqpoint{3.027322in}{2.246593in}}%
\pgfpathlineto{\pgfqpoint{3.019082in}{2.238047in}}%
\pgfpathlineto{\pgfqpoint{3.010834in}{2.229554in}}%
\pgfpathlineto{\pgfqpoint{3.002580in}{2.221112in}}%
\pgfpathlineto{\pgfqpoint{2.994318in}{2.212724in}}%
\pgfpathclose%
\pgfusepath{fill}%
\end{pgfscope}%
\begin{pgfscope}%
\pgfpathrectangle{\pgfqpoint{1.150000in}{0.150000in}}{\pgfqpoint{5.700000in}{5.700000in}}%
\pgfusepath{clip}%
\pgfsetbuttcap%
\pgfsetroundjoin%
\definecolor{currentfill}{rgb}{0.253935,0.265254,0.529983}%
\pgfsetfillcolor{currentfill}%
\pgfsetfillopacity{0.700000}%
\pgfsetlinewidth{0.000000pt}%
\definecolor{currentstroke}{rgb}{0.000000,0.000000,0.000000}%
\pgfsetstrokecolor{currentstroke}%
\pgfsetdash{}{0pt}%
\pgfpathmoveto{\pgfqpoint{5.292361in}{2.602376in}}%
\pgfpathlineto{\pgfqpoint{5.306345in}{2.601575in}}%
\pgfpathlineto{\pgfqpoint{5.320339in}{2.600842in}}%
\pgfpathlineto{\pgfqpoint{5.334342in}{2.600176in}}%
\pgfpathlineto{\pgfqpoint{5.348355in}{2.599578in}}%
\pgfpathlineto{\pgfqpoint{5.355778in}{2.607883in}}%
\pgfpathlineto{\pgfqpoint{5.363200in}{2.616357in}}%
\pgfpathlineto{\pgfqpoint{5.370621in}{2.625007in}}%
\pgfpathlineto{\pgfqpoint{5.378042in}{2.633840in}}%
\pgfpathlineto{\pgfqpoint{5.364050in}{2.634823in}}%
\pgfpathlineto{\pgfqpoint{5.350067in}{2.635874in}}%
\pgfpathlineto{\pgfqpoint{5.336094in}{2.636993in}}%
\pgfpathlineto{\pgfqpoint{5.322130in}{2.638178in}}%
\pgfpathlineto{\pgfqpoint{5.314688in}{2.628953in}}%
\pgfpathlineto{\pgfqpoint{5.307247in}{2.619916in}}%
\pgfpathlineto{\pgfqpoint{5.299804in}{2.611059in}}%
\pgfpathlineto{\pgfqpoint{5.292361in}{2.602376in}}%
\pgfpathclose%
\pgfusepath{fill}%
\end{pgfscope}%
\begin{pgfscope}%
\pgfpathrectangle{\pgfqpoint{1.150000in}{0.150000in}}{\pgfqpoint{5.700000in}{5.700000in}}%
\pgfusepath{clip}%
\pgfsetbuttcap%
\pgfsetroundjoin%
\definecolor{currentfill}{rgb}{0.271828,0.209303,0.504434}%
\pgfsetfillcolor{currentfill}%
\pgfsetfillopacity{0.700000}%
\pgfsetlinewidth{0.000000pt}%
\definecolor{currentstroke}{rgb}{0.000000,0.000000,0.000000}%
\pgfsetstrokecolor{currentstroke}%
\pgfsetdash{}{0pt}%
\pgfpathmoveto{\pgfqpoint{4.893717in}{2.485404in}}%
\pgfpathlineto{\pgfqpoint{4.907589in}{2.484565in}}%
\pgfpathlineto{\pgfqpoint{4.921471in}{2.483796in}}%
\pgfpathlineto{\pgfqpoint{4.935362in}{2.483098in}}%
\pgfpathlineto{\pgfqpoint{4.949262in}{2.482469in}}%
\pgfpathlineto{\pgfqpoint{4.956822in}{2.490424in}}%
\pgfpathlineto{\pgfqpoint{4.964379in}{2.498475in}}%
\pgfpathlineto{\pgfqpoint{4.971932in}{2.506627in}}%
\pgfpathlineto{\pgfqpoint{4.979481in}{2.514886in}}%
\pgfpathlineto{\pgfqpoint{4.965598in}{2.515820in}}%
\pgfpathlineto{\pgfqpoint{4.951724in}{2.516823in}}%
\pgfpathlineto{\pgfqpoint{4.937859in}{2.517896in}}%
\pgfpathlineto{\pgfqpoint{4.924002in}{2.519038in}}%
\pgfpathlineto{\pgfqpoint{4.916436in}{2.510468in}}%
\pgfpathlineto{\pgfqpoint{4.908867in}{2.502009in}}%
\pgfpathlineto{\pgfqpoint{4.901294in}{2.493656in}}%
\pgfpathlineto{\pgfqpoint{4.893717in}{2.485404in}}%
\pgfpathclose%
\pgfusepath{fill}%
\end{pgfscope}%
\begin{pgfscope}%
\pgfpathrectangle{\pgfqpoint{1.150000in}{0.150000in}}{\pgfqpoint{5.700000in}{5.700000in}}%
\pgfusepath{clip}%
\pgfsetbuttcap%
\pgfsetroundjoin%
\definecolor{currentfill}{rgb}{0.277941,0.056324,0.381191}%
\pgfsetfillcolor{currentfill}%
\pgfsetfillopacity{0.700000}%
\pgfsetlinewidth{0.000000pt}%
\definecolor{currentstroke}{rgb}{0.000000,0.000000,0.000000}%
\pgfsetstrokecolor{currentstroke}%
\pgfsetdash{}{0pt}%
\pgfpathmoveto{\pgfqpoint{3.275526in}{2.181185in}}%
\pgfpathlineto{\pgfqpoint{3.289012in}{2.175759in}}%
\pgfpathlineto{\pgfqpoint{3.302501in}{2.170428in}}%
\pgfpathlineto{\pgfqpoint{3.315994in}{2.165190in}}%
\pgfpathlineto{\pgfqpoint{3.329492in}{2.160045in}}%
\pgfpathlineto{\pgfqpoint{3.337634in}{2.168906in}}%
\pgfpathlineto{\pgfqpoint{3.345769in}{2.177795in}}%
\pgfpathlineto{\pgfqpoint{3.353898in}{2.186712in}}%
\pgfpathlineto{\pgfqpoint{3.362021in}{2.195658in}}%
\pgfpathlineto{\pgfqpoint{3.348536in}{2.200800in}}%
\pgfpathlineto{\pgfqpoint{3.335055in}{2.206035in}}%
\pgfpathlineto{\pgfqpoint{3.321578in}{2.211365in}}%
\pgfpathlineto{\pgfqpoint{3.308104in}{2.216788in}}%
\pgfpathlineto{\pgfqpoint{3.299969in}{2.207837in}}%
\pgfpathlineto{\pgfqpoint{3.291828in}{2.198920in}}%
\pgfpathlineto{\pgfqpoint{3.283680in}{2.190036in}}%
\pgfpathlineto{\pgfqpoint{3.275526in}{2.181185in}}%
\pgfpathclose%
\pgfusepath{fill}%
\end{pgfscope}%
\begin{pgfscope}%
\pgfpathrectangle{\pgfqpoint{1.150000in}{0.150000in}}{\pgfqpoint{5.700000in}{5.700000in}}%
\pgfusepath{clip}%
\pgfsetbuttcap%
\pgfsetroundjoin%
\definecolor{currentfill}{rgb}{0.281446,0.084320,0.407414}%
\pgfsetfillcolor{currentfill}%
\pgfsetfillopacity{0.700000}%
\pgfsetlinewidth{0.000000pt}%
\definecolor{currentstroke}{rgb}{0.000000,0.000000,0.000000}%
\pgfsetstrokecolor{currentstroke}%
\pgfsetdash{}{0pt}%
\pgfpathmoveto{\pgfqpoint{2.853439in}{2.240720in}}%
\pgfpathlineto{\pgfqpoint{2.866903in}{2.232713in}}%
\pgfpathlineto{\pgfqpoint{2.880369in}{2.224817in}}%
\pgfpathlineto{\pgfqpoint{2.893836in}{2.217029in}}%
\pgfpathlineto{\pgfqpoint{2.907305in}{2.209351in}}%
\pgfpathlineto{\pgfqpoint{2.915612in}{2.217452in}}%
\pgfpathlineto{\pgfqpoint{2.923910in}{2.225612in}}%
\pgfpathlineto{\pgfqpoint{2.932202in}{2.233832in}}%
\pgfpathlineto{\pgfqpoint{2.940485in}{2.242110in}}%
\pgfpathlineto{\pgfqpoint{2.927032in}{2.249724in}}%
\pgfpathlineto{\pgfqpoint{2.913580in}{2.257447in}}%
\pgfpathlineto{\pgfqpoint{2.900130in}{2.265279in}}%
\pgfpathlineto{\pgfqpoint{2.886681in}{2.273221in}}%
\pgfpathlineto{\pgfqpoint{2.878382in}{2.265000in}}%
\pgfpathlineto{\pgfqpoint{2.870075in}{2.256842in}}%
\pgfpathlineto{\pgfqpoint{2.861761in}{2.248749in}}%
\pgfpathlineto{\pgfqpoint{2.853439in}{2.240720in}}%
\pgfpathclose%
\pgfusepath{fill}%
\end{pgfscope}%
\begin{pgfscope}%
\pgfpathrectangle{\pgfqpoint{1.150000in}{0.150000in}}{\pgfqpoint{5.700000in}{5.700000in}}%
\pgfusepath{clip}%
\pgfsetbuttcap%
\pgfsetroundjoin%
\definecolor{currentfill}{rgb}{0.229739,0.322361,0.545706}%
\pgfsetfillcolor{currentfill}%
\pgfsetfillopacity{0.700000}%
\pgfsetlinewidth{0.000000pt}%
\definecolor{currentstroke}{rgb}{0.000000,0.000000,0.000000}%
\pgfsetstrokecolor{currentstroke}%
\pgfsetdash{}{0pt}%
\pgfpathmoveto{\pgfqpoint{5.691251in}{2.729637in}}%
\pgfpathlineto{\pgfqpoint{5.705338in}{2.728505in}}%
\pgfpathlineto{\pgfqpoint{5.719435in}{2.727439in}}%
\pgfpathlineto{\pgfqpoint{5.733542in}{2.726439in}}%
\pgfpathlineto{\pgfqpoint{5.747660in}{2.725504in}}%
\pgfpathlineto{\pgfqpoint{5.754982in}{2.735094in}}%
\pgfpathlineto{\pgfqpoint{5.762308in}{2.744949in}}%
\pgfpathlineto{\pgfqpoint{5.769638in}{2.755077in}}%
\pgfpathlineto{\pgfqpoint{5.776973in}{2.765486in}}%
\pgfpathlineto{\pgfqpoint{5.762881in}{2.766886in}}%
\pgfpathlineto{\pgfqpoint{5.748799in}{2.768352in}}%
\pgfpathlineto{\pgfqpoint{5.734727in}{2.769884in}}%
\pgfpathlineto{\pgfqpoint{5.720664in}{2.771481in}}%
\pgfpathlineto{\pgfqpoint{5.713304in}{2.760599in}}%
\pgfpathlineto{\pgfqpoint{5.705949in}{2.750003in}}%
\pgfpathlineto{\pgfqpoint{5.698598in}{2.739685in}}%
\pgfpathlineto{\pgfqpoint{5.691251in}{2.729637in}}%
\pgfpathclose%
\pgfusepath{fill}%
\end{pgfscope}%
\begin{pgfscope}%
\pgfpathrectangle{\pgfqpoint{1.150000in}{0.150000in}}{\pgfqpoint{5.700000in}{5.700000in}}%
\pgfusepath{clip}%
\pgfsetbuttcap%
\pgfsetroundjoin%
\definecolor{currentfill}{rgb}{0.280255,0.165693,0.476498}%
\pgfsetfillcolor{currentfill}%
\pgfsetfillopacity{0.700000}%
\pgfsetlinewidth{0.000000pt}%
\definecolor{currentstroke}{rgb}{0.000000,0.000000,0.000000}%
\pgfsetstrokecolor{currentstroke}%
\pgfsetdash{}{0pt}%
\pgfpathmoveto{\pgfqpoint{2.462670in}{2.411095in}}%
\pgfpathlineto{\pgfqpoint{2.476175in}{2.399971in}}%
\pgfpathlineto{\pgfqpoint{2.489678in}{2.388982in}}%
\pgfpathlineto{\pgfqpoint{2.503179in}{2.378126in}}%
\pgfpathlineto{\pgfqpoint{2.516680in}{2.367401in}}%
\pgfpathlineto{\pgfqpoint{2.525162in}{2.374309in}}%
\pgfpathlineto{\pgfqpoint{2.533634in}{2.381315in}}%
\pgfpathlineto{\pgfqpoint{2.542098in}{2.388418in}}%
\pgfpathlineto{\pgfqpoint{2.550551in}{2.395616in}}%
\pgfpathlineto{\pgfqpoint{2.537071in}{2.406233in}}%
\pgfpathlineto{\pgfqpoint{2.523590in}{2.416982in}}%
\pgfpathlineto{\pgfqpoint{2.510107in}{2.427863in}}%
\pgfpathlineto{\pgfqpoint{2.496622in}{2.438878in}}%
\pgfpathlineto{\pgfqpoint{2.488149in}{2.431780in}}%
\pgfpathlineto{\pgfqpoint{2.479666in}{2.424782in}}%
\pgfpathlineto{\pgfqpoint{2.471173in}{2.417887in}}%
\pgfpathlineto{\pgfqpoint{2.462670in}{2.411095in}}%
\pgfpathclose%
\pgfusepath{fill}%
\end{pgfscope}%
\begin{pgfscope}%
\pgfpathrectangle{\pgfqpoint{1.150000in}{0.150000in}}{\pgfqpoint{5.700000in}{5.700000in}}%
\pgfusepath{clip}%
\pgfsetbuttcap%
\pgfsetroundjoin%
\definecolor{currentfill}{rgb}{0.283229,0.120777,0.440584}%
\pgfsetfillcolor{currentfill}%
\pgfsetfillopacity{0.700000}%
\pgfsetlinewidth{0.000000pt}%
\definecolor{currentstroke}{rgb}{0.000000,0.000000,0.000000}%
\pgfsetstrokecolor{currentstroke}%
\pgfsetdash{}{0pt}%
\pgfpathmoveto{\pgfqpoint{4.182289in}{2.297780in}}%
\pgfpathlineto{\pgfqpoint{4.195961in}{2.295862in}}%
\pgfpathlineto{\pgfqpoint{4.209640in}{2.294020in}}%
\pgfpathlineto{\pgfqpoint{4.223327in}{2.292255in}}%
\pgfpathlineto{\pgfqpoint{4.237021in}{2.290567in}}%
\pgfpathlineto{\pgfqpoint{4.244845in}{2.299113in}}%
\pgfpathlineto{\pgfqpoint{4.252664in}{2.307682in}}%
\pgfpathlineto{\pgfqpoint{4.260477in}{2.316277in}}%
\pgfpathlineto{\pgfqpoint{4.268284in}{2.324900in}}%
\pgfpathlineto{\pgfqpoint{4.254601in}{2.326750in}}%
\pgfpathlineto{\pgfqpoint{4.240926in}{2.328676in}}%
\pgfpathlineto{\pgfqpoint{4.227259in}{2.330679in}}%
\pgfpathlineto{\pgfqpoint{4.213598in}{2.332758in}}%
\pgfpathlineto{\pgfqpoint{4.205779in}{2.323966in}}%
\pgfpathlineto{\pgfqpoint{4.197955in}{2.315208in}}%
\pgfpathlineto{\pgfqpoint{4.190125in}{2.306481in}}%
\pgfpathlineto{\pgfqpoint{4.182289in}{2.297780in}}%
\pgfpathclose%
\pgfusepath{fill}%
\end{pgfscope}%
\begin{pgfscope}%
\pgfpathrectangle{\pgfqpoint{1.150000in}{0.150000in}}{\pgfqpoint{5.700000in}{5.700000in}}%
\pgfusepath{clip}%
\pgfsetbuttcap%
\pgfsetroundjoin%
\definecolor{currentfill}{rgb}{0.279566,0.067836,0.391917}%
\pgfsetfillcolor{currentfill}%
\pgfsetfillopacity{0.700000}%
\pgfsetlinewidth{0.000000pt}%
\definecolor{currentstroke}{rgb}{0.000000,0.000000,0.000000}%
\pgfsetstrokecolor{currentstroke}%
\pgfsetdash{}{0pt}%
\pgfpathmoveto{\pgfqpoint{3.642811in}{2.197474in}}%
\pgfpathlineto{\pgfqpoint{3.656356in}{2.193790in}}%
\pgfpathlineto{\pgfqpoint{3.669907in}{2.190191in}}%
\pgfpathlineto{\pgfqpoint{3.683463in}{2.186677in}}%
\pgfpathlineto{\pgfqpoint{3.697025in}{2.183248in}}%
\pgfpathlineto{\pgfqpoint{3.705038in}{2.192235in}}%
\pgfpathlineto{\pgfqpoint{3.713045in}{2.201235in}}%
\pgfpathlineto{\pgfqpoint{3.721047in}{2.210251in}}%
\pgfpathlineto{\pgfqpoint{3.729043in}{2.219282in}}%
\pgfpathlineto{\pgfqpoint{3.715492in}{2.222770in}}%
\pgfpathlineto{\pgfqpoint{3.701947in}{2.226343in}}%
\pgfpathlineto{\pgfqpoint{3.688407in}{2.230001in}}%
\pgfpathlineto{\pgfqpoint{3.674873in}{2.233744in}}%
\pgfpathlineto{\pgfqpoint{3.666866in}{2.224646in}}%
\pgfpathlineto{\pgfqpoint{3.658854in}{2.215569in}}%
\pgfpathlineto{\pgfqpoint{3.650835in}{2.206513in}}%
\pgfpathlineto{\pgfqpoint{3.642811in}{2.197474in}}%
\pgfpathclose%
\pgfusepath{fill}%
\end{pgfscope}%
\begin{pgfscope}%
\pgfpathrectangle{\pgfqpoint{1.150000in}{0.150000in}}{\pgfqpoint{5.700000in}{5.700000in}}%
\pgfusepath{clip}%
\pgfsetbuttcap%
\pgfsetroundjoin%
\definecolor{currentfill}{rgb}{0.280868,0.160771,0.472899}%
\pgfsetfillcolor{currentfill}%
\pgfsetfillopacity{0.700000}%
\pgfsetlinewidth{0.000000pt}%
\definecolor{currentstroke}{rgb}{0.000000,0.000000,0.000000}%
\pgfsetstrokecolor{currentstroke}%
\pgfsetdash{}{0pt}%
\pgfpathmoveto{\pgfqpoint{4.495052in}{2.374393in}}%
\pgfpathlineto{\pgfqpoint{4.508812in}{2.373132in}}%
\pgfpathlineto{\pgfqpoint{4.522580in}{2.371944in}}%
\pgfpathlineto{\pgfqpoint{4.536356in}{2.370830in}}%
\pgfpathlineto{\pgfqpoint{4.550141in}{2.369789in}}%
\pgfpathlineto{\pgfqpoint{4.557850in}{2.377967in}}%
\pgfpathlineto{\pgfqpoint{4.565555in}{2.386190in}}%
\pgfpathlineto{\pgfqpoint{4.573254in}{2.394462in}}%
\pgfpathlineto{\pgfqpoint{4.580949in}{2.402787in}}%
\pgfpathlineto{\pgfqpoint{4.567178in}{2.404051in}}%
\pgfpathlineto{\pgfqpoint{4.553415in}{2.405388in}}%
\pgfpathlineto{\pgfqpoint{4.539660in}{2.406798in}}%
\pgfpathlineto{\pgfqpoint{4.525914in}{2.408282in}}%
\pgfpathlineto{\pgfqpoint{4.518206in}{2.399726in}}%
\pgfpathlineto{\pgfqpoint{4.510493in}{2.391229in}}%
\pgfpathlineto{\pgfqpoint{4.502775in}{2.382786in}}%
\pgfpathlineto{\pgfqpoint{4.495052in}{2.374393in}}%
\pgfpathclose%
\pgfusepath{fill}%
\end{pgfscope}%
\begin{pgfscope}%
\pgfpathrectangle{\pgfqpoint{1.150000in}{0.150000in}}{\pgfqpoint{5.700000in}{5.700000in}}%
\pgfusepath{clip}%
\pgfsetbuttcap%
\pgfsetroundjoin%
\definecolor{currentfill}{rgb}{0.258965,0.251537,0.524736}%
\pgfsetfillcolor{currentfill}%
\pgfsetfillopacity{0.700000}%
\pgfsetlinewidth{0.000000pt}%
\definecolor{currentstroke}{rgb}{0.000000,0.000000,0.000000}%
\pgfsetstrokecolor{currentstroke}%
\pgfsetdash{}{0pt}%
\pgfpathmoveto{\pgfqpoint{5.206648in}{2.571666in}}%
\pgfpathlineto{\pgfqpoint{5.220615in}{2.570959in}}%
\pgfpathlineto{\pgfqpoint{5.234591in}{2.570320in}}%
\pgfpathlineto{\pgfqpoint{5.248576in}{2.569749in}}%
\pgfpathlineto{\pgfqpoint{5.262571in}{2.569246in}}%
\pgfpathlineto{\pgfqpoint{5.270021in}{2.577301in}}%
\pgfpathlineto{\pgfqpoint{5.277469in}{2.585503in}}%
\pgfpathlineto{\pgfqpoint{5.284916in}{2.593859in}}%
\pgfpathlineto{\pgfqpoint{5.292361in}{2.602376in}}%
\pgfpathlineto{\pgfqpoint{5.278386in}{2.603244in}}%
\pgfpathlineto{\pgfqpoint{5.264420in}{2.604180in}}%
\pgfpathlineto{\pgfqpoint{5.250464in}{2.605184in}}%
\pgfpathlineto{\pgfqpoint{5.236517in}{2.606256in}}%
\pgfpathlineto{\pgfqpoint{5.229052in}{2.597367in}}%
\pgfpathlineto{\pgfqpoint{5.221586in}{2.588644in}}%
\pgfpathlineto{\pgfqpoint{5.214118in}{2.580079in}}%
\pgfpathlineto{\pgfqpoint{5.206648in}{2.571666in}}%
\pgfpathclose%
\pgfusepath{fill}%
\end{pgfscope}%
\begin{pgfscope}%
\pgfpathrectangle{\pgfqpoint{1.150000in}{0.150000in}}{\pgfqpoint{5.700000in}{5.700000in}}%
\pgfusepath{clip}%
\pgfsetbuttcap%
\pgfsetroundjoin%
\definecolor{currentfill}{rgb}{0.281446,0.084320,0.407414}%
\pgfsetfillcolor{currentfill}%
\pgfsetfillopacity{0.700000}%
\pgfsetlinewidth{0.000000pt}%
\definecolor{currentstroke}{rgb}{0.000000,0.000000,0.000000}%
\pgfsetstrokecolor{currentstroke}%
\pgfsetdash{}{0pt}%
\pgfpathmoveto{\pgfqpoint{3.869506in}{2.230063in}}%
\pgfpathlineto{\pgfqpoint{3.883101in}{2.227241in}}%
\pgfpathlineto{\pgfqpoint{3.896702in}{2.224500in}}%
\pgfpathlineto{\pgfqpoint{3.910310in}{2.221839in}}%
\pgfpathlineto{\pgfqpoint{3.923924in}{2.219260in}}%
\pgfpathlineto{\pgfqpoint{3.931859in}{2.228122in}}%
\pgfpathlineto{\pgfqpoint{3.939789in}{2.236995in}}%
\pgfpathlineto{\pgfqpoint{3.947713in}{2.245882in}}%
\pgfpathlineto{\pgfqpoint{3.955632in}{2.254786in}}%
\pgfpathlineto{\pgfqpoint{3.942028in}{2.257465in}}%
\pgfpathlineto{\pgfqpoint{3.928431in}{2.260225in}}%
\pgfpathlineto{\pgfqpoint{3.914841in}{2.263066in}}%
\pgfpathlineto{\pgfqpoint{3.901257in}{2.265988in}}%
\pgfpathlineto{\pgfqpoint{3.893328in}{2.256977in}}%
\pgfpathlineto{\pgfqpoint{3.885393in}{2.247988in}}%
\pgfpathlineto{\pgfqpoint{3.877452in}{2.239017in}}%
\pgfpathlineto{\pgfqpoint{3.869506in}{2.230063in}}%
\pgfpathclose%
\pgfusepath{fill}%
\end{pgfscope}%
\begin{pgfscope}%
\pgfpathrectangle{\pgfqpoint{1.150000in}{0.150000in}}{\pgfqpoint{5.700000in}{5.700000in}}%
\pgfusepath{clip}%
\pgfsetbuttcap%
\pgfsetroundjoin%
\definecolor{currentfill}{rgb}{0.277941,0.056324,0.381191}%
\pgfsetfillcolor{currentfill}%
\pgfsetfillopacity{0.700000}%
\pgfsetlinewidth{0.000000pt}%
\definecolor{currentstroke}{rgb}{0.000000,0.000000,0.000000}%
\pgfsetstrokecolor{currentstroke}%
\pgfsetdash{}{0pt}%
\pgfpathmoveto{\pgfqpoint{3.416005in}{2.176012in}}%
\pgfpathlineto{\pgfqpoint{3.429512in}{2.171329in}}%
\pgfpathlineto{\pgfqpoint{3.443024in}{2.166736in}}%
\pgfpathlineto{\pgfqpoint{3.456540in}{2.162233in}}%
\pgfpathlineto{\pgfqpoint{3.470061in}{2.157820in}}%
\pgfpathlineto{\pgfqpoint{3.478155in}{2.166774in}}%
\pgfpathlineto{\pgfqpoint{3.486242in}{2.175749in}}%
\pgfpathlineto{\pgfqpoint{3.494324in}{2.184744in}}%
\pgfpathlineto{\pgfqpoint{3.502400in}{2.193761in}}%
\pgfpathlineto{\pgfqpoint{3.488890in}{2.198193in}}%
\pgfpathlineto{\pgfqpoint{3.475385in}{2.202714in}}%
\pgfpathlineto{\pgfqpoint{3.461885in}{2.207325in}}%
\pgfpathlineto{\pgfqpoint{3.448390in}{2.212026in}}%
\pgfpathlineto{\pgfqpoint{3.440303in}{2.202983in}}%
\pgfpathlineto{\pgfqpoint{3.432210in}{2.193967in}}%
\pgfpathlineto{\pgfqpoint{3.424110in}{2.184977in}}%
\pgfpathlineto{\pgfqpoint{3.416005in}{2.176012in}}%
\pgfpathclose%
\pgfusepath{fill}%
\end{pgfscope}%
\begin{pgfscope}%
\pgfpathrectangle{\pgfqpoint{1.150000in}{0.150000in}}{\pgfqpoint{5.700000in}{5.700000in}}%
\pgfusepath{clip}%
\pgfsetbuttcap%
\pgfsetroundjoin%
\definecolor{currentfill}{rgb}{0.235526,0.309527,0.542944}%
\pgfsetfillcolor{currentfill}%
\pgfsetfillopacity{0.700000}%
\pgfsetlinewidth{0.000000pt}%
\definecolor{currentstroke}{rgb}{0.000000,0.000000,0.000000}%
\pgfsetstrokecolor{currentstroke}%
\pgfsetdash{}{0pt}%
\pgfpathmoveto{\pgfqpoint{5.605542in}{2.695375in}}%
\pgfpathlineto{\pgfqpoint{5.619614in}{2.694427in}}%
\pgfpathlineto{\pgfqpoint{5.633697in}{2.693544in}}%
\pgfpathlineto{\pgfqpoint{5.647789in}{2.692727in}}%
\pgfpathlineto{\pgfqpoint{5.661891in}{2.691976in}}%
\pgfpathlineto{\pgfqpoint{5.669227in}{2.701027in}}%
\pgfpathlineto{\pgfqpoint{5.676566in}{2.710316in}}%
\pgfpathlineto{\pgfqpoint{5.683907in}{2.719850in}}%
\pgfpathlineto{\pgfqpoint{5.691251in}{2.729637in}}%
\pgfpathlineto{\pgfqpoint{5.677173in}{2.730834in}}%
\pgfpathlineto{\pgfqpoint{5.663105in}{2.732097in}}%
\pgfpathlineto{\pgfqpoint{5.649047in}{2.733425in}}%
\pgfpathlineto{\pgfqpoint{5.634998in}{2.734820in}}%
\pgfpathlineto{\pgfqpoint{5.627630in}{2.724579in}}%
\pgfpathlineto{\pgfqpoint{5.620265in}{2.714597in}}%
\pgfpathlineto{\pgfqpoint{5.612902in}{2.704865in}}%
\pgfpathlineto{\pgfqpoint{5.605542in}{2.695375in}}%
\pgfpathclose%
\pgfusepath{fill}%
\end{pgfscope}%
\begin{pgfscope}%
\pgfpathrectangle{\pgfqpoint{1.150000in}{0.150000in}}{\pgfqpoint{5.700000in}{5.700000in}}%
\pgfusepath{clip}%
\pgfsetbuttcap%
\pgfsetroundjoin%
\definecolor{currentfill}{rgb}{0.274128,0.199721,0.498911}%
\pgfsetfillcolor{currentfill}%
\pgfsetfillopacity{0.700000}%
\pgfsetlinewidth{0.000000pt}%
\definecolor{currentstroke}{rgb}{0.000000,0.000000,0.000000}%
\pgfsetstrokecolor{currentstroke}%
\pgfsetdash{}{0pt}%
\pgfpathmoveto{\pgfqpoint{4.807903in}{2.456220in}}%
\pgfpathlineto{\pgfqpoint{4.821756in}{2.455383in}}%
\pgfpathlineto{\pgfqpoint{4.835618in}{2.454617in}}%
\pgfpathlineto{\pgfqpoint{4.849489in}{2.453922in}}%
\pgfpathlineto{\pgfqpoint{4.863369in}{2.453297in}}%
\pgfpathlineto{\pgfqpoint{4.870962in}{2.461199in}}%
\pgfpathlineto{\pgfqpoint{4.878551in}{2.469181in}}%
\pgfpathlineto{\pgfqpoint{4.886136in}{2.477247in}}%
\pgfpathlineto{\pgfqpoint{4.893717in}{2.485404in}}%
\pgfpathlineto{\pgfqpoint{4.879853in}{2.486313in}}%
\pgfpathlineto{\pgfqpoint{4.865998in}{2.487293in}}%
\pgfpathlineto{\pgfqpoint{4.852151in}{2.488343in}}%
\pgfpathlineto{\pgfqpoint{4.838314in}{2.489463in}}%
\pgfpathlineto{\pgfqpoint{4.830717in}{2.481015in}}%
\pgfpathlineto{\pgfqpoint{4.823116in}{2.472662in}}%
\pgfpathlineto{\pgfqpoint{4.815512in}{2.464399in}}%
\pgfpathlineto{\pgfqpoint{4.807903in}{2.456220in}}%
\pgfpathclose%
\pgfusepath{fill}%
\end{pgfscope}%
\begin{pgfscope}%
\pgfpathrectangle{\pgfqpoint{1.150000in}{0.150000in}}{\pgfqpoint{5.700000in}{5.700000in}}%
\pgfusepath{clip}%
\pgfsetbuttcap%
\pgfsetroundjoin%
\definecolor{currentfill}{rgb}{0.282910,0.105393,0.426902}%
\pgfsetfillcolor{currentfill}%
\pgfsetfillopacity{0.700000}%
\pgfsetlinewidth{0.000000pt}%
\definecolor{currentstroke}{rgb}{0.000000,0.000000,0.000000}%
\pgfsetstrokecolor{currentstroke}%
\pgfsetdash{}{0pt}%
\pgfpathmoveto{\pgfqpoint{2.712257in}{2.278075in}}%
\pgfpathlineto{\pgfqpoint{2.725731in}{2.269070in}}%
\pgfpathlineto{\pgfqpoint{2.739206in}{2.260183in}}%
\pgfpathlineto{\pgfqpoint{2.752682in}{2.251412in}}%
\pgfpathlineto{\pgfqpoint{2.766158in}{2.242756in}}%
\pgfpathlineto{\pgfqpoint{2.774529in}{2.250425in}}%
\pgfpathlineto{\pgfqpoint{2.782892in}{2.258167in}}%
\pgfpathlineto{\pgfqpoint{2.791246in}{2.265981in}}%
\pgfpathlineto{\pgfqpoint{2.799593in}{2.273866in}}%
\pgfpathlineto{\pgfqpoint{2.786134in}{2.282437in}}%
\pgfpathlineto{\pgfqpoint{2.772675in}{2.291122in}}%
\pgfpathlineto{\pgfqpoint{2.759218in}{2.299924in}}%
\pgfpathlineto{\pgfqpoint{2.745761in}{2.308843in}}%
\pgfpathlineto{\pgfqpoint{2.737397in}{2.301035in}}%
\pgfpathlineto{\pgfqpoint{2.729025in}{2.293304in}}%
\pgfpathlineto{\pgfqpoint{2.720645in}{2.285650in}}%
\pgfpathlineto{\pgfqpoint{2.712257in}{2.278075in}}%
\pgfpathclose%
\pgfusepath{fill}%
\end{pgfscope}%
\begin{pgfscope}%
\pgfpathrectangle{\pgfqpoint{1.150000in}{0.150000in}}{\pgfqpoint{5.700000in}{5.700000in}}%
\pgfusepath{clip}%
\pgfsetbuttcap%
\pgfsetroundjoin%
\definecolor{currentfill}{rgb}{0.282290,0.145912,0.461510}%
\pgfsetfillcolor{currentfill}%
\pgfsetfillopacity{0.700000}%
\pgfsetlinewidth{0.000000pt}%
\definecolor{currentstroke}{rgb}{0.000000,0.000000,0.000000}%
\pgfsetstrokecolor{currentstroke}%
\pgfsetdash{}{0pt}%
\pgfpathmoveto{\pgfqpoint{2.516680in}{2.367401in}}%
\pgfpathlineto{\pgfqpoint{2.530178in}{2.356807in}}%
\pgfpathlineto{\pgfqpoint{2.543676in}{2.346342in}}%
\pgfpathlineto{\pgfqpoint{2.557173in}{2.336006in}}%
\pgfpathlineto{\pgfqpoint{2.570668in}{2.325797in}}%
\pgfpathlineto{\pgfqpoint{2.579130in}{2.332819in}}%
\pgfpathlineto{\pgfqpoint{2.587583in}{2.339935in}}%
\pgfpathlineto{\pgfqpoint{2.596027in}{2.347143in}}%
\pgfpathlineto{\pgfqpoint{2.604461in}{2.354441in}}%
\pgfpathlineto{\pgfqpoint{2.590985in}{2.364543in}}%
\pgfpathlineto{\pgfqpoint{2.577508in}{2.374773in}}%
\pgfpathlineto{\pgfqpoint{2.564030in}{2.385130in}}%
\pgfpathlineto{\pgfqpoint{2.550551in}{2.395616in}}%
\pgfpathlineto{\pgfqpoint{2.542098in}{2.388418in}}%
\pgfpathlineto{\pgfqpoint{2.533634in}{2.381315in}}%
\pgfpathlineto{\pgfqpoint{2.525162in}{2.374309in}}%
\pgfpathlineto{\pgfqpoint{2.516680in}{2.367401in}}%
\pgfpathclose%
\pgfusepath{fill}%
\end{pgfscope}%
\begin{pgfscope}%
\pgfpathrectangle{\pgfqpoint{1.150000in}{0.150000in}}{\pgfqpoint{5.700000in}{5.700000in}}%
\pgfusepath{clip}%
\pgfsetbuttcap%
\pgfsetroundjoin%
\definecolor{currentfill}{rgb}{0.283091,0.110553,0.431554}%
\pgfsetfillcolor{currentfill}%
\pgfsetfillopacity{0.700000}%
\pgfsetlinewidth{0.000000pt}%
\definecolor{currentstroke}{rgb}{0.000000,0.000000,0.000000}%
\pgfsetstrokecolor{currentstroke}%
\pgfsetdash{}{0pt}%
\pgfpathmoveto{\pgfqpoint{4.096233in}{2.271076in}}%
\pgfpathlineto{\pgfqpoint{4.109887in}{2.268988in}}%
\pgfpathlineto{\pgfqpoint{4.123548in}{2.266979in}}%
\pgfpathlineto{\pgfqpoint{4.137217in}{2.265047in}}%
\pgfpathlineto{\pgfqpoint{4.150893in}{2.263193in}}%
\pgfpathlineto{\pgfqpoint{4.158750in}{2.271813in}}%
\pgfpathlineto{\pgfqpoint{4.166602in}{2.280449in}}%
\pgfpathlineto{\pgfqpoint{4.174449in}{2.289104in}}%
\pgfpathlineto{\pgfqpoint{4.182289in}{2.297780in}}%
\pgfpathlineto{\pgfqpoint{4.168625in}{2.299776in}}%
\pgfpathlineto{\pgfqpoint{4.154968in}{2.301848in}}%
\pgfpathlineto{\pgfqpoint{4.141318in}{2.303999in}}%
\pgfpathlineto{\pgfqpoint{4.127675in}{2.306227in}}%
\pgfpathlineto{\pgfqpoint{4.119823in}{2.297402in}}%
\pgfpathlineto{\pgfqpoint{4.111965in}{2.288604in}}%
\pgfpathlineto{\pgfqpoint{4.104102in}{2.279830in}}%
\pgfpathlineto{\pgfqpoint{4.096233in}{2.271076in}}%
\pgfpathclose%
\pgfusepath{fill}%
\end{pgfscope}%
\begin{pgfscope}%
\pgfpathrectangle{\pgfqpoint{1.150000in}{0.150000in}}{\pgfqpoint{5.700000in}{5.700000in}}%
\pgfusepath{clip}%
\pgfsetbuttcap%
\pgfsetroundjoin%
\definecolor{currentfill}{rgb}{0.262138,0.242286,0.520837}%
\pgfsetfillcolor{currentfill}%
\pgfsetfillopacity{0.700000}%
\pgfsetlinewidth{0.000000pt}%
\definecolor{currentstroke}{rgb}{0.000000,0.000000,0.000000}%
\pgfsetstrokecolor{currentstroke}%
\pgfsetdash{}{0pt}%
\pgfpathmoveto{\pgfqpoint{5.120898in}{2.541540in}}%
\pgfpathlineto{\pgfqpoint{5.134846in}{2.540904in}}%
\pgfpathlineto{\pgfqpoint{5.148803in}{2.540338in}}%
\pgfpathlineto{\pgfqpoint{5.162770in}{2.539840in}}%
\pgfpathlineto{\pgfqpoint{5.176746in}{2.539410in}}%
\pgfpathlineto{\pgfqpoint{5.184225in}{2.547277in}}%
\pgfpathlineto{\pgfqpoint{5.191702in}{2.555271in}}%
\pgfpathlineto{\pgfqpoint{5.199176in}{2.563399in}}%
\pgfpathlineto{\pgfqpoint{5.206648in}{2.571666in}}%
\pgfpathlineto{\pgfqpoint{5.192691in}{2.572441in}}%
\pgfpathlineto{\pgfqpoint{5.178744in}{2.573285in}}%
\pgfpathlineto{\pgfqpoint{5.164805in}{2.574196in}}%
\pgfpathlineto{\pgfqpoint{5.150876in}{2.575177in}}%
\pgfpathlineto{\pgfqpoint{5.143385in}{2.566557in}}%
\pgfpathlineto{\pgfqpoint{5.135892in}{2.558082in}}%
\pgfpathlineto{\pgfqpoint{5.128396in}{2.549745in}}%
\pgfpathlineto{\pgfqpoint{5.120898in}{2.541540in}}%
\pgfpathclose%
\pgfusepath{fill}%
\end{pgfscope}%
\begin{pgfscope}%
\pgfpathrectangle{\pgfqpoint{1.150000in}{0.150000in}}{\pgfqpoint{5.700000in}{5.700000in}}%
\pgfusepath{clip}%
\pgfsetbuttcap%
\pgfsetroundjoin%
\definecolor{currentfill}{rgb}{0.282290,0.145912,0.461510}%
\pgfsetfillcolor{currentfill}%
\pgfsetfillopacity{0.700000}%
\pgfsetlinewidth{0.000000pt}%
\definecolor{currentstroke}{rgb}{0.000000,0.000000,0.000000}%
\pgfsetstrokecolor{currentstroke}%
\pgfsetdash{}{0pt}%
\pgfpathmoveto{\pgfqpoint{4.409100in}{2.346213in}}%
\pgfpathlineto{\pgfqpoint{4.422840in}{2.344859in}}%
\pgfpathlineto{\pgfqpoint{4.436588in}{2.343578in}}%
\pgfpathlineto{\pgfqpoint{4.450345in}{2.342372in}}%
\pgfpathlineto{\pgfqpoint{4.464109in}{2.341240in}}%
\pgfpathlineto{\pgfqpoint{4.471853in}{2.349473in}}%
\pgfpathlineto{\pgfqpoint{4.479591in}{2.357740in}}%
\pgfpathlineto{\pgfqpoint{4.487324in}{2.366046in}}%
\pgfpathlineto{\pgfqpoint{4.495052in}{2.374393in}}%
\pgfpathlineto{\pgfqpoint{4.481301in}{2.375728in}}%
\pgfpathlineto{\pgfqpoint{4.467557in}{2.377136in}}%
\pgfpathlineto{\pgfqpoint{4.453822in}{2.378619in}}%
\pgfpathlineto{\pgfqpoint{4.440095in}{2.380175in}}%
\pgfpathlineto{\pgfqpoint{4.432354in}{2.371618in}}%
\pgfpathlineto{\pgfqpoint{4.424608in}{2.363108in}}%
\pgfpathlineto{\pgfqpoint{4.416856in}{2.354641in}}%
\pgfpathlineto{\pgfqpoint{4.409100in}{2.346213in}}%
\pgfpathclose%
\pgfusepath{fill}%
\end{pgfscope}%
\begin{pgfscope}%
\pgfpathrectangle{\pgfqpoint{1.150000in}{0.150000in}}{\pgfqpoint{5.700000in}{5.700000in}}%
\pgfusepath{clip}%
\pgfsetbuttcap%
\pgfsetroundjoin%
\definecolor{currentfill}{rgb}{0.241237,0.296485,0.539709}%
\pgfsetfillcolor{currentfill}%
\pgfsetfillopacity{0.700000}%
\pgfsetlinewidth{0.000000pt}%
\definecolor{currentstroke}{rgb}{0.000000,0.000000,0.000000}%
\pgfsetstrokecolor{currentstroke}%
\pgfsetdash{}{0pt}%
\pgfpathmoveto{\pgfqpoint{5.519831in}{2.662436in}}%
\pgfpathlineto{\pgfqpoint{5.533888in}{2.661648in}}%
\pgfpathlineto{\pgfqpoint{5.547955in}{2.660927in}}%
\pgfpathlineto{\pgfqpoint{5.562031in}{2.660272in}}%
\pgfpathlineto{\pgfqpoint{5.576117in}{2.659684in}}%
\pgfpathlineto{\pgfqpoint{5.583472in}{2.668282in}}%
\pgfpathlineto{\pgfqpoint{5.590827in}{2.677091in}}%
\pgfpathlineto{\pgfqpoint{5.598184in}{2.686120in}}%
\pgfpathlineto{\pgfqpoint{5.605542in}{2.695375in}}%
\pgfpathlineto{\pgfqpoint{5.591479in}{2.696390in}}%
\pgfpathlineto{\pgfqpoint{5.577426in}{2.697471in}}%
\pgfpathlineto{\pgfqpoint{5.563383in}{2.698619in}}%
\pgfpathlineto{\pgfqpoint{5.549349in}{2.699832in}}%
\pgfpathlineto{\pgfqpoint{5.541967in}{2.690143in}}%
\pgfpathlineto{\pgfqpoint{5.534587in}{2.680686in}}%
\pgfpathlineto{\pgfqpoint{5.527209in}{2.671453in}}%
\pgfpathlineto{\pgfqpoint{5.519831in}{2.662436in}}%
\pgfpathclose%
\pgfusepath{fill}%
\end{pgfscope}%
\begin{pgfscope}%
\pgfpathrectangle{\pgfqpoint{1.150000in}{0.150000in}}{\pgfqpoint{5.700000in}{5.700000in}}%
\pgfusepath{clip}%
\pgfsetbuttcap%
\pgfsetroundjoin%
\definecolor{currentfill}{rgb}{0.199430,0.387607,0.554642}%
\pgfsetfillcolor{currentfill}%
\pgfsetfillopacity{0.700000}%
\pgfsetlinewidth{0.000000pt}%
\definecolor{currentstroke}{rgb}{0.000000,0.000000,0.000000}%
\pgfsetstrokecolor{currentstroke}%
\pgfsetdash{}{0pt}%
\pgfpathmoveto{\pgfqpoint{6.091014in}{2.877663in}}%
\pgfpathlineto{\pgfqpoint{6.105191in}{2.875844in}}%
\pgfpathlineto{\pgfqpoint{6.119378in}{2.874088in}}%
\pgfpathlineto{\pgfqpoint{6.133575in}{2.872397in}}%
\pgfpathlineto{\pgfqpoint{6.140877in}{2.884715in}}%
\pgfpathlineto{\pgfqpoint{6.148191in}{2.897420in}}%
\pgfpathlineto{\pgfqpoint{6.155516in}{2.910520in}}%
\pgfpathlineto{\pgfqpoint{6.162853in}{2.924025in}}%
\pgfpathlineto{\pgfqpoint{6.148686in}{2.926263in}}%
\pgfpathlineto{\pgfqpoint{6.134528in}{2.928564in}}%
\pgfpathlineto{\pgfqpoint{6.120380in}{2.930929in}}%
\pgfpathlineto{\pgfqpoint{6.113021in}{2.917009in}}%
\pgfpathlineto{\pgfqpoint{6.105674in}{2.903498in}}%
\pgfpathlineto{\pgfqpoint{6.098338in}{2.890386in}}%
\pgfpathlineto{\pgfqpoint{6.091014in}{2.877663in}}%
\pgfpathclose%
\pgfusepath{fill}%
\end{pgfscope}%
\begin{pgfscope}%
\pgfpathrectangle{\pgfqpoint{1.150000in}{0.150000in}}{\pgfqpoint{5.700000in}{5.700000in}}%
\pgfusepath{clip}%
\pgfsetbuttcap%
\pgfsetroundjoin%
\definecolor{currentfill}{rgb}{0.278791,0.062145,0.386592}%
\pgfsetfillcolor{currentfill}%
\pgfsetfillopacity{0.700000}%
\pgfsetlinewidth{0.000000pt}%
\definecolor{currentstroke}{rgb}{0.000000,0.000000,0.000000}%
\pgfsetstrokecolor{currentstroke}%
\pgfsetdash{}{0pt}%
\pgfpathmoveto{\pgfqpoint{3.048187in}{2.185017in}}%
\pgfpathlineto{\pgfqpoint{3.061660in}{2.178347in}}%
\pgfpathlineto{\pgfqpoint{3.075136in}{2.171779in}}%
\pgfpathlineto{\pgfqpoint{3.088615in}{2.165312in}}%
\pgfpathlineto{\pgfqpoint{3.102096in}{2.158945in}}%
\pgfpathlineto{\pgfqpoint{3.110329in}{2.167435in}}%
\pgfpathlineto{\pgfqpoint{3.118555in}{2.175969in}}%
\pgfpathlineto{\pgfqpoint{3.126775in}{2.184544in}}%
\pgfpathlineto{\pgfqpoint{3.134987in}{2.193162in}}%
\pgfpathlineto{\pgfqpoint{3.121519in}{2.199486in}}%
\pgfpathlineto{\pgfqpoint{3.108055in}{2.205910in}}%
\pgfpathlineto{\pgfqpoint{3.094593in}{2.212435in}}%
\pgfpathlineto{\pgfqpoint{3.081133in}{2.219061in}}%
\pgfpathlineto{\pgfqpoint{3.072907in}{2.210479in}}%
\pgfpathlineto{\pgfqpoint{3.064674in}{2.201944in}}%
\pgfpathlineto{\pgfqpoint{3.056434in}{2.193457in}}%
\pgfpathlineto{\pgfqpoint{3.048187in}{2.185017in}}%
\pgfpathclose%
\pgfusepath{fill}%
\end{pgfscope}%
\begin{pgfscope}%
\pgfpathrectangle{\pgfqpoint{1.150000in}{0.150000in}}{\pgfqpoint{5.700000in}{5.700000in}}%
\pgfusepath{clip}%
\pgfsetbuttcap%
\pgfsetroundjoin%
\definecolor{currentfill}{rgb}{0.276194,0.190074,0.493001}%
\pgfsetfillcolor{currentfill}%
\pgfsetfillopacity{0.700000}%
\pgfsetlinewidth{0.000000pt}%
\definecolor{currentstroke}{rgb}{0.000000,0.000000,0.000000}%
\pgfsetstrokecolor{currentstroke}%
\pgfsetdash{}{0pt}%
\pgfpathmoveto{\pgfqpoint{4.722036in}{2.427257in}}%
\pgfpathlineto{\pgfqpoint{4.735870in}{2.426399in}}%
\pgfpathlineto{\pgfqpoint{4.749712in}{2.425613in}}%
\pgfpathlineto{\pgfqpoint{4.763562in}{2.424898in}}%
\pgfpathlineto{\pgfqpoint{4.777422in}{2.424254in}}%
\pgfpathlineto{\pgfqpoint{4.785049in}{2.432143in}}%
\pgfpathlineto{\pgfqpoint{4.792671in}{2.440097in}}%
\pgfpathlineto{\pgfqpoint{4.800289in}{2.448121in}}%
\pgfpathlineto{\pgfqpoint{4.807903in}{2.456220in}}%
\pgfpathlineto{\pgfqpoint{4.794058in}{2.457128in}}%
\pgfpathlineto{\pgfqpoint{4.780223in}{2.458107in}}%
\pgfpathlineto{\pgfqpoint{4.766396in}{2.459157in}}%
\pgfpathlineto{\pgfqpoint{4.752577in}{2.460278in}}%
\pgfpathlineto{\pgfqpoint{4.744949in}{2.451908in}}%
\pgfpathlineto{\pgfqpoint{4.737316in}{2.443618in}}%
\pgfpathlineto{\pgfqpoint{4.729678in}{2.435402in}}%
\pgfpathlineto{\pgfqpoint{4.722036in}{2.427257in}}%
\pgfpathclose%
\pgfusepath{fill}%
\end{pgfscope}%
\begin{pgfscope}%
\pgfpathrectangle{\pgfqpoint{1.150000in}{0.150000in}}{\pgfqpoint{5.700000in}{5.700000in}}%
\pgfusepath{clip}%
\pgfsetbuttcap%
\pgfsetroundjoin%
\definecolor{currentfill}{rgb}{0.278791,0.062145,0.386592}%
\pgfsetfillcolor{currentfill}%
\pgfsetfillopacity{0.700000}%
\pgfsetlinewidth{0.000000pt}%
\definecolor{currentstroke}{rgb}{0.000000,0.000000,0.000000}%
\pgfsetstrokecolor{currentstroke}%
\pgfsetdash{}{0pt}%
\pgfpathmoveto{\pgfqpoint{3.556487in}{2.176921in}}%
\pgfpathlineto{\pgfqpoint{3.570021in}{2.172930in}}%
\pgfpathlineto{\pgfqpoint{3.583561in}{2.169026in}}%
\pgfpathlineto{\pgfqpoint{3.597106in}{2.165209in}}%
\pgfpathlineto{\pgfqpoint{3.610657in}{2.161478in}}%
\pgfpathlineto{\pgfqpoint{3.618704in}{2.170455in}}%
\pgfpathlineto{\pgfqpoint{3.626746in}{2.179447in}}%
\pgfpathlineto{\pgfqpoint{3.634781in}{2.188452in}}%
\pgfpathlineto{\pgfqpoint{3.642811in}{2.197474in}}%
\pgfpathlineto{\pgfqpoint{3.629272in}{2.201244in}}%
\pgfpathlineto{\pgfqpoint{3.615738in}{2.205100in}}%
\pgfpathlineto{\pgfqpoint{3.602209in}{2.209042in}}%
\pgfpathlineto{\pgfqpoint{3.588686in}{2.213072in}}%
\pgfpathlineto{\pgfqpoint{3.580645in}{2.204004in}}%
\pgfpathlineto{\pgfqpoint{3.572598in}{2.194957in}}%
\pgfpathlineto{\pgfqpoint{3.564545in}{2.185930in}}%
\pgfpathlineto{\pgfqpoint{3.556487in}{2.176921in}}%
\pgfpathclose%
\pgfusepath{fill}%
\end{pgfscope}%
\begin{pgfscope}%
\pgfpathrectangle{\pgfqpoint{1.150000in}{0.150000in}}{\pgfqpoint{5.700000in}{5.700000in}}%
\pgfusepath{clip}%
\pgfsetbuttcap%
\pgfsetroundjoin%
\definecolor{currentfill}{rgb}{0.277018,0.050344,0.375715}%
\pgfsetfillcolor{currentfill}%
\pgfsetfillopacity{0.700000}%
\pgfsetlinewidth{0.000000pt}%
\definecolor{currentstroke}{rgb}{0.000000,0.000000,0.000000}%
\pgfsetstrokecolor{currentstroke}%
\pgfsetdash{}{0pt}%
\pgfpathmoveto{\pgfqpoint{3.188890in}{2.168857in}}%
\pgfpathlineto{\pgfqpoint{3.202373in}{2.163025in}}%
\pgfpathlineto{\pgfqpoint{3.215861in}{2.157290in}}%
\pgfpathlineto{\pgfqpoint{3.229352in}{2.151651in}}%
\pgfpathlineto{\pgfqpoint{3.242846in}{2.146107in}}%
\pgfpathlineto{\pgfqpoint{3.251026in}{2.154828in}}%
\pgfpathlineto{\pgfqpoint{3.259199in}{2.163582in}}%
\pgfpathlineto{\pgfqpoint{3.267366in}{2.172367in}}%
\pgfpathlineto{\pgfqpoint{3.275526in}{2.181185in}}%
\pgfpathlineto{\pgfqpoint{3.262044in}{2.186707in}}%
\pgfpathlineto{\pgfqpoint{3.248566in}{2.192323in}}%
\pgfpathlineto{\pgfqpoint{3.235092in}{2.198035in}}%
\pgfpathlineto{\pgfqpoint{3.221621in}{2.203845in}}%
\pgfpathlineto{\pgfqpoint{3.213448in}{2.195042in}}%
\pgfpathlineto{\pgfqpoint{3.205268in}{2.186276in}}%
\pgfpathlineto{\pgfqpoint{3.197082in}{2.177548in}}%
\pgfpathlineto{\pgfqpoint{3.188890in}{2.168857in}}%
\pgfpathclose%
\pgfusepath{fill}%
\end{pgfscope}%
\begin{pgfscope}%
\pgfpathrectangle{\pgfqpoint{1.150000in}{0.150000in}}{\pgfqpoint{5.700000in}{5.700000in}}%
\pgfusepath{clip}%
\pgfsetbuttcap%
\pgfsetroundjoin%
\definecolor{currentfill}{rgb}{0.280894,0.078907,0.402329}%
\pgfsetfillcolor{currentfill}%
\pgfsetfillopacity{0.700000}%
\pgfsetlinewidth{0.000000pt}%
\definecolor{currentstroke}{rgb}{0.000000,0.000000,0.000000}%
\pgfsetstrokecolor{currentstroke}%
\pgfsetdash{}{0pt}%
\pgfpathmoveto{\pgfqpoint{3.783306in}{2.206167in}}%
\pgfpathlineto{\pgfqpoint{3.796886in}{2.203096in}}%
\pgfpathlineto{\pgfqpoint{3.810473in}{2.200108in}}%
\pgfpathlineto{\pgfqpoint{3.824066in}{2.197202in}}%
\pgfpathlineto{\pgfqpoint{3.837665in}{2.194378in}}%
\pgfpathlineto{\pgfqpoint{3.845634in}{2.203283in}}%
\pgfpathlineto{\pgfqpoint{3.853597in}{2.212198in}}%
\pgfpathlineto{\pgfqpoint{3.861554in}{2.221124in}}%
\pgfpathlineto{\pgfqpoint{3.869506in}{2.230063in}}%
\pgfpathlineto{\pgfqpoint{3.855918in}{2.232967in}}%
\pgfpathlineto{\pgfqpoint{3.842336in}{2.235952in}}%
\pgfpathlineto{\pgfqpoint{3.828760in}{2.239020in}}%
\pgfpathlineto{\pgfqpoint{3.815190in}{2.242170in}}%
\pgfpathlineto{\pgfqpoint{3.807227in}{2.233144in}}%
\pgfpathlineto{\pgfqpoint{3.799259in}{2.224136in}}%
\pgfpathlineto{\pgfqpoint{3.791285in}{2.215144in}}%
\pgfpathlineto{\pgfqpoint{3.783306in}{2.206167in}}%
\pgfpathclose%
\pgfusepath{fill}%
\end{pgfscope}%
\begin{pgfscope}%
\pgfpathrectangle{\pgfqpoint{1.150000in}{0.150000in}}{\pgfqpoint{5.700000in}{5.700000in}}%
\pgfusepath{clip}%
\pgfsetbuttcap%
\pgfsetroundjoin%
\definecolor{currentfill}{rgb}{0.280267,0.073417,0.397163}%
\pgfsetfillcolor{currentfill}%
\pgfsetfillopacity{0.700000}%
\pgfsetlinewidth{0.000000pt}%
\definecolor{currentstroke}{rgb}{0.000000,0.000000,0.000000}%
\pgfsetstrokecolor{currentstroke}%
\pgfsetdash{}{0pt}%
\pgfpathmoveto{\pgfqpoint{2.907305in}{2.209351in}}%
\pgfpathlineto{\pgfqpoint{2.920776in}{2.201780in}}%
\pgfpathlineto{\pgfqpoint{2.934249in}{2.194317in}}%
\pgfpathlineto{\pgfqpoint{2.947723in}{2.186960in}}%
\pgfpathlineto{\pgfqpoint{2.961199in}{2.179708in}}%
\pgfpathlineto{\pgfqpoint{2.969490in}{2.187881in}}%
\pgfpathlineto{\pgfqpoint{2.977773in}{2.196108in}}%
\pgfpathlineto{\pgfqpoint{2.986049in}{2.204389in}}%
\pgfpathlineto{\pgfqpoint{2.994318in}{2.212724in}}%
\pgfpathlineto{\pgfqpoint{2.980857in}{2.219912in}}%
\pgfpathlineto{\pgfqpoint{2.967398in}{2.227204in}}%
\pgfpathlineto{\pgfqpoint{2.953941in}{2.234604in}}%
\pgfpathlineto{\pgfqpoint{2.940485in}{2.242110in}}%
\pgfpathlineto{\pgfqpoint{2.932202in}{2.233832in}}%
\pgfpathlineto{\pgfqpoint{2.923910in}{2.225612in}}%
\pgfpathlineto{\pgfqpoint{2.915612in}{2.217452in}}%
\pgfpathlineto{\pgfqpoint{2.907305in}{2.209351in}}%
\pgfpathclose%
\pgfusepath{fill}%
\end{pgfscope}%
\begin{pgfscope}%
\pgfpathrectangle{\pgfqpoint{1.150000in}{0.150000in}}{\pgfqpoint{5.700000in}{5.700000in}}%
\pgfusepath{clip}%
\pgfsetbuttcap%
\pgfsetroundjoin%
\definecolor{currentfill}{rgb}{0.206756,0.371758,0.553117}%
\pgfsetfillcolor{currentfill}%
\pgfsetfillopacity{0.700000}%
\pgfsetlinewidth{0.000000pt}%
\definecolor{currentstroke}{rgb}{0.000000,0.000000,0.000000}%
\pgfsetstrokecolor{currentstroke}%
\pgfsetdash{}{0pt}%
\pgfpathmoveto{\pgfqpoint{6.005093in}{2.836287in}}%
\pgfpathlineto{\pgfqpoint{6.019259in}{2.834737in}}%
\pgfpathlineto{\pgfqpoint{6.033435in}{2.833251in}}%
\pgfpathlineto{\pgfqpoint{6.047621in}{2.831830in}}%
\pgfpathlineto{\pgfqpoint{6.061817in}{2.830472in}}%
\pgfpathlineto{\pgfqpoint{6.069102in}{2.841734in}}%
\pgfpathlineto{\pgfqpoint{6.076396in}{2.853347in}}%
\pgfpathlineto{\pgfqpoint{6.083700in}{2.865320in}}%
\pgfpathlineto{\pgfqpoint{6.091014in}{2.877663in}}%
\pgfpathlineto{\pgfqpoint{6.076847in}{2.879546in}}%
\pgfpathlineto{\pgfqpoint{6.062689in}{2.881494in}}%
\pgfpathlineto{\pgfqpoint{6.048542in}{2.883506in}}%
\pgfpathlineto{\pgfqpoint{6.034404in}{2.885581in}}%
\pgfpathlineto{\pgfqpoint{6.027062in}{2.872705in}}%
\pgfpathlineto{\pgfqpoint{6.019730in}{2.860204in}}%
\pgfpathlineto{\pgfqpoint{6.012407in}{2.848067in}}%
\pgfpathlineto{\pgfqpoint{6.005093in}{2.836287in}}%
\pgfpathclose%
\pgfusepath{fill}%
\end{pgfscope}%
\begin{pgfscope}%
\pgfpathrectangle{\pgfqpoint{1.150000in}{0.150000in}}{\pgfqpoint{5.700000in}{5.700000in}}%
\pgfusepath{clip}%
\pgfsetbuttcap%
\pgfsetroundjoin%
\definecolor{currentfill}{rgb}{0.246811,0.283237,0.535941}%
\pgfsetfillcolor{currentfill}%
\pgfsetfillopacity{0.700000}%
\pgfsetlinewidth{0.000000pt}%
\definecolor{currentstroke}{rgb}{0.000000,0.000000,0.000000}%
\pgfsetstrokecolor{currentstroke}%
\pgfsetdash{}{0pt}%
\pgfpathmoveto{\pgfqpoint{5.434106in}{2.630576in}}%
\pgfpathlineto{\pgfqpoint{5.448146in}{2.629928in}}%
\pgfpathlineto{\pgfqpoint{5.462196in}{2.629346in}}%
\pgfpathlineto{\pgfqpoint{5.476256in}{2.628832in}}%
\pgfpathlineto{\pgfqpoint{5.490325in}{2.628384in}}%
\pgfpathlineto{\pgfqpoint{5.497702in}{2.636609in}}%
\pgfpathlineto{\pgfqpoint{5.505078in}{2.645021in}}%
\pgfpathlineto{\pgfqpoint{5.512454in}{2.653628in}}%
\pgfpathlineto{\pgfqpoint{5.519831in}{2.662436in}}%
\pgfpathlineto{\pgfqpoint{5.505784in}{2.663290in}}%
\pgfpathlineto{\pgfqpoint{5.491747in}{2.664211in}}%
\pgfpathlineto{\pgfqpoint{5.477719in}{2.665198in}}%
\pgfpathlineto{\pgfqpoint{5.463701in}{2.666253in}}%
\pgfpathlineto{\pgfqpoint{5.456302in}{2.657031in}}%
\pgfpathlineto{\pgfqpoint{5.448903in}{2.648016in}}%
\pgfpathlineto{\pgfqpoint{5.441504in}{2.639200in}}%
\pgfpathlineto{\pgfqpoint{5.434106in}{2.630576in}}%
\pgfpathclose%
\pgfusepath{fill}%
\end{pgfscope}%
\begin{pgfscope}%
\pgfpathrectangle{\pgfqpoint{1.150000in}{0.150000in}}{\pgfqpoint{5.700000in}{5.700000in}}%
\pgfusepath{clip}%
\pgfsetbuttcap%
\pgfsetroundjoin%
\definecolor{currentfill}{rgb}{0.266580,0.228262,0.514349}%
\pgfsetfillcolor{currentfill}%
\pgfsetfillopacity{0.700000}%
\pgfsetlinewidth{0.000000pt}%
\definecolor{currentstroke}{rgb}{0.000000,0.000000,0.000000}%
\pgfsetstrokecolor{currentstroke}%
\pgfsetdash{}{0pt}%
\pgfpathmoveto{\pgfqpoint{5.035104in}{2.511849in}}%
\pgfpathlineto{\pgfqpoint{5.049033in}{2.511263in}}%
\pgfpathlineto{\pgfqpoint{5.062971in}{2.510746in}}%
\pgfpathlineto{\pgfqpoint{5.076918in}{2.510299in}}%
\pgfpathlineto{\pgfqpoint{5.090875in}{2.509920in}}%
\pgfpathlineto{\pgfqpoint{5.098386in}{2.517657in}}%
\pgfpathlineto{\pgfqpoint{5.105893in}{2.525502in}}%
\pgfpathlineto{\pgfqpoint{5.113397in}{2.533461in}}%
\pgfpathlineto{\pgfqpoint{5.120898in}{2.541540in}}%
\pgfpathlineto{\pgfqpoint{5.106959in}{2.542244in}}%
\pgfpathlineto{\pgfqpoint{5.093030in}{2.543017in}}%
\pgfpathlineto{\pgfqpoint{5.079110in}{2.543859in}}%
\pgfpathlineto{\pgfqpoint{5.065199in}{2.544769in}}%
\pgfpathlineto{\pgfqpoint{5.057680in}{2.536358in}}%
\pgfpathlineto{\pgfqpoint{5.050158in}{2.528071in}}%
\pgfpathlineto{\pgfqpoint{5.042633in}{2.519903in}}%
\pgfpathlineto{\pgfqpoint{5.035104in}{2.511849in}}%
\pgfpathclose%
\pgfusepath{fill}%
\end{pgfscope}%
\begin{pgfscope}%
\pgfpathrectangle{\pgfqpoint{1.150000in}{0.150000in}}{\pgfqpoint{5.700000in}{5.700000in}}%
\pgfusepath{clip}%
\pgfsetbuttcap%
\pgfsetroundjoin%
\definecolor{currentfill}{rgb}{0.214298,0.355619,0.551184}%
\pgfsetfillcolor{currentfill}%
\pgfsetfillopacity{0.700000}%
\pgfsetlinewidth{0.000000pt}%
\definecolor{currentstroke}{rgb}{0.000000,0.000000,0.000000}%
\pgfsetstrokecolor{currentstroke}%
\pgfsetdash{}{0pt}%
\pgfpathmoveto{\pgfqpoint{5.919242in}{2.797357in}}%
\pgfpathlineto{\pgfqpoint{5.933396in}{2.796055in}}%
\pgfpathlineto{\pgfqpoint{5.947560in}{2.794818in}}%
\pgfpathlineto{\pgfqpoint{5.961734in}{2.793646in}}%
\pgfpathlineto{\pgfqpoint{5.975918in}{2.792538in}}%
\pgfpathlineto{\pgfqpoint{5.983201in}{2.802987in}}%
\pgfpathlineto{\pgfqpoint{5.990490in}{2.813756in}}%
\pgfpathlineto{\pgfqpoint{5.997788in}{2.824853in}}%
\pgfpathlineto{\pgfqpoint{6.005093in}{2.836287in}}%
\pgfpathlineto{\pgfqpoint{5.990937in}{2.837901in}}%
\pgfpathlineto{\pgfqpoint{5.976791in}{2.839580in}}%
\pgfpathlineto{\pgfqpoint{5.962654in}{2.841323in}}%
\pgfpathlineto{\pgfqpoint{5.948528in}{2.843131in}}%
\pgfpathlineto{\pgfqpoint{5.941195in}{2.831183in}}%
\pgfpathlineto{\pgfqpoint{5.933870in}{2.819578in}}%
\pgfpathlineto{\pgfqpoint{5.926552in}{2.808306in}}%
\pgfpathlineto{\pgfqpoint{5.919242in}{2.797357in}}%
\pgfpathclose%
\pgfusepath{fill}%
\end{pgfscope}%
\begin{pgfscope}%
\pgfpathrectangle{\pgfqpoint{1.150000in}{0.150000in}}{\pgfqpoint{5.700000in}{5.700000in}}%
\pgfusepath{clip}%
\pgfsetbuttcap%
\pgfsetroundjoin%
\definecolor{currentfill}{rgb}{0.277018,0.050344,0.375715}%
\pgfsetfillcolor{currentfill}%
\pgfsetfillopacity{0.700000}%
\pgfsetlinewidth{0.000000pt}%
\definecolor{currentstroke}{rgb}{0.000000,0.000000,0.000000}%
\pgfsetstrokecolor{currentstroke}%
\pgfsetdash{}{0pt}%
\pgfpathmoveto{\pgfqpoint{3.329492in}{2.160045in}}%
\pgfpathlineto{\pgfqpoint{3.342994in}{2.154993in}}%
\pgfpathlineto{\pgfqpoint{3.356499in}{2.150034in}}%
\pgfpathlineto{\pgfqpoint{3.370009in}{2.145166in}}%
\pgfpathlineto{\pgfqpoint{3.383524in}{2.140390in}}%
\pgfpathlineto{\pgfqpoint{3.391653in}{2.149261in}}%
\pgfpathlineto{\pgfqpoint{3.399777in}{2.158154in}}%
\pgfpathlineto{\pgfqpoint{3.407894in}{2.167071in}}%
\pgfpathlineto{\pgfqpoint{3.416005in}{2.176012in}}%
\pgfpathlineto{\pgfqpoint{3.402503in}{2.180786in}}%
\pgfpathlineto{\pgfqpoint{3.389005in}{2.185651in}}%
\pgfpathlineto{\pgfqpoint{3.375511in}{2.190608in}}%
\pgfpathlineto{\pgfqpoint{3.362021in}{2.195658in}}%
\pgfpathlineto{\pgfqpoint{3.353898in}{2.186712in}}%
\pgfpathlineto{\pgfqpoint{3.345769in}{2.177795in}}%
\pgfpathlineto{\pgfqpoint{3.337634in}{2.168906in}}%
\pgfpathlineto{\pgfqpoint{3.329492in}{2.160045in}}%
\pgfpathclose%
\pgfusepath{fill}%
\end{pgfscope}%
\begin{pgfscope}%
\pgfpathrectangle{\pgfqpoint{1.150000in}{0.150000in}}{\pgfqpoint{5.700000in}{5.700000in}}%
\pgfusepath{clip}%
\pgfsetbuttcap%
\pgfsetroundjoin%
\definecolor{currentfill}{rgb}{0.282884,0.135920,0.453427}%
\pgfsetfillcolor{currentfill}%
\pgfsetfillopacity{0.700000}%
\pgfsetlinewidth{0.000000pt}%
\definecolor{currentstroke}{rgb}{0.000000,0.000000,0.000000}%
\pgfsetstrokecolor{currentstroke}%
\pgfsetdash{}{0pt}%
\pgfpathmoveto{\pgfqpoint{4.323090in}{2.318259in}}%
\pgfpathlineto{\pgfqpoint{4.336811in}{2.316788in}}%
\pgfpathlineto{\pgfqpoint{4.350539in}{2.315391in}}%
\pgfpathlineto{\pgfqpoint{4.364276in}{2.314070in}}%
\pgfpathlineto{\pgfqpoint{4.378021in}{2.312823in}}%
\pgfpathlineto{\pgfqpoint{4.385799in}{2.321129in}}%
\pgfpathlineto{\pgfqpoint{4.393571in}{2.329461in}}%
\pgfpathlineto{\pgfqpoint{4.401338in}{2.337821in}}%
\pgfpathlineto{\pgfqpoint{4.409100in}{2.346213in}}%
\pgfpathlineto{\pgfqpoint{4.395368in}{2.347642in}}%
\pgfpathlineto{\pgfqpoint{4.381644in}{2.349145in}}%
\pgfpathlineto{\pgfqpoint{4.367927in}{2.350724in}}%
\pgfpathlineto{\pgfqpoint{4.354218in}{2.352377in}}%
\pgfpathlineto{\pgfqpoint{4.346444in}{2.343795in}}%
\pgfpathlineto{\pgfqpoint{4.338665in}{2.335251in}}%
\pgfpathlineto{\pgfqpoint{4.330880in}{2.326740in}}%
\pgfpathlineto{\pgfqpoint{4.323090in}{2.318259in}}%
\pgfpathclose%
\pgfusepath{fill}%
\end{pgfscope}%
\begin{pgfscope}%
\pgfpathrectangle{\pgfqpoint{1.150000in}{0.150000in}}{\pgfqpoint{5.700000in}{5.700000in}}%
\pgfusepath{clip}%
\pgfsetbuttcap%
\pgfsetroundjoin%
\definecolor{currentfill}{rgb}{0.282656,0.100196,0.422160}%
\pgfsetfillcolor{currentfill}%
\pgfsetfillopacity{0.700000}%
\pgfsetlinewidth{0.000000pt}%
\definecolor{currentstroke}{rgb}{0.000000,0.000000,0.000000}%
\pgfsetstrokecolor{currentstroke}%
\pgfsetdash{}{0pt}%
\pgfpathmoveto{\pgfqpoint{4.010112in}{2.244868in}}%
\pgfpathlineto{\pgfqpoint{4.023749in}{2.242587in}}%
\pgfpathlineto{\pgfqpoint{4.037393in}{2.240385in}}%
\pgfpathlineto{\pgfqpoint{4.051044in}{2.238262in}}%
\pgfpathlineto{\pgfqpoint{4.064702in}{2.236217in}}%
\pgfpathlineto{\pgfqpoint{4.072593in}{2.244913in}}%
\pgfpathlineto{\pgfqpoint{4.080479in}{2.253620in}}%
\pgfpathlineto{\pgfqpoint{4.088359in}{2.262340in}}%
\pgfpathlineto{\pgfqpoint{4.096233in}{2.271076in}}%
\pgfpathlineto{\pgfqpoint{4.082586in}{2.273241in}}%
\pgfpathlineto{\pgfqpoint{4.068946in}{2.275485in}}%
\pgfpathlineto{\pgfqpoint{4.055313in}{2.277807in}}%
\pgfpathlineto{\pgfqpoint{4.041687in}{2.280209in}}%
\pgfpathlineto{\pgfqpoint{4.033801in}{2.271345in}}%
\pgfpathlineto{\pgfqpoint{4.025911in}{2.262502in}}%
\pgfpathlineto{\pgfqpoint{4.018014in}{2.253677in}}%
\pgfpathlineto{\pgfqpoint{4.010112in}{2.244868in}}%
\pgfpathclose%
\pgfusepath{fill}%
\end{pgfscope}%
\begin{pgfscope}%
\pgfpathrectangle{\pgfqpoint{1.150000in}{0.150000in}}{\pgfqpoint{5.700000in}{5.700000in}}%
\pgfusepath{clip}%
\pgfsetbuttcap%
\pgfsetroundjoin%
\definecolor{currentfill}{rgb}{0.283072,0.130895,0.449241}%
\pgfsetfillcolor{currentfill}%
\pgfsetfillopacity{0.700000}%
\pgfsetlinewidth{0.000000pt}%
\definecolor{currentstroke}{rgb}{0.000000,0.000000,0.000000}%
\pgfsetstrokecolor{currentstroke}%
\pgfsetdash{}{0pt}%
\pgfpathmoveto{\pgfqpoint{2.570668in}{2.325797in}}%
\pgfpathlineto{\pgfqpoint{2.584163in}{2.315713in}}%
\pgfpathlineto{\pgfqpoint{2.597658in}{2.305755in}}%
\pgfpathlineto{\pgfqpoint{2.611151in}{2.295921in}}%
\pgfpathlineto{\pgfqpoint{2.624645in}{2.286211in}}%
\pgfpathlineto{\pgfqpoint{2.633087in}{2.293348in}}%
\pgfpathlineto{\pgfqpoint{2.641520in}{2.300573in}}%
\pgfpathlineto{\pgfqpoint{2.649945in}{2.307885in}}%
\pgfpathlineto{\pgfqpoint{2.658360in}{2.315283in}}%
\pgfpathlineto{\pgfqpoint{2.644886in}{2.324887in}}%
\pgfpathlineto{\pgfqpoint{2.631411in}{2.334614in}}%
\pgfpathlineto{\pgfqpoint{2.617936in}{2.344465in}}%
\pgfpathlineto{\pgfqpoint{2.604461in}{2.354441in}}%
\pgfpathlineto{\pgfqpoint{2.596027in}{2.347143in}}%
\pgfpathlineto{\pgfqpoint{2.587583in}{2.339935in}}%
\pgfpathlineto{\pgfqpoint{2.579130in}{2.332819in}}%
\pgfpathlineto{\pgfqpoint{2.570668in}{2.325797in}}%
\pgfpathclose%
\pgfusepath{fill}%
\end{pgfscope}%
\begin{pgfscope}%
\pgfpathrectangle{\pgfqpoint{1.150000in}{0.150000in}}{\pgfqpoint{5.700000in}{5.700000in}}%
\pgfusepath{clip}%
\pgfsetbuttcap%
\pgfsetroundjoin%
\definecolor{currentfill}{rgb}{0.282327,0.094955,0.417331}%
\pgfsetfillcolor{currentfill}%
\pgfsetfillopacity{0.700000}%
\pgfsetlinewidth{0.000000pt}%
\definecolor{currentstroke}{rgb}{0.000000,0.000000,0.000000}%
\pgfsetstrokecolor{currentstroke}%
\pgfsetdash{}{0pt}%
\pgfpathmoveto{\pgfqpoint{2.766158in}{2.242756in}}%
\pgfpathlineto{\pgfqpoint{2.779635in}{2.234215in}}%
\pgfpathlineto{\pgfqpoint{2.793113in}{2.225787in}}%
\pgfpathlineto{\pgfqpoint{2.806591in}{2.217472in}}%
\pgfpathlineto{\pgfqpoint{2.820071in}{2.209270in}}%
\pgfpathlineto{\pgfqpoint{2.828425in}{2.217031in}}%
\pgfpathlineto{\pgfqpoint{2.836771in}{2.224860in}}%
\pgfpathlineto{\pgfqpoint{2.845109in}{2.232757in}}%
\pgfpathlineto{\pgfqpoint{2.853439in}{2.240720in}}%
\pgfpathlineto{\pgfqpoint{2.839975in}{2.248838in}}%
\pgfpathlineto{\pgfqpoint{2.826513in}{2.257068in}}%
\pgfpathlineto{\pgfqpoint{2.813053in}{2.265410in}}%
\pgfpathlineto{\pgfqpoint{2.799593in}{2.273866in}}%
\pgfpathlineto{\pgfqpoint{2.791246in}{2.265981in}}%
\pgfpathlineto{\pgfqpoint{2.782892in}{2.258167in}}%
\pgfpathlineto{\pgfqpoint{2.774529in}{2.250425in}}%
\pgfpathlineto{\pgfqpoint{2.766158in}{2.242756in}}%
\pgfpathclose%
\pgfusepath{fill}%
\end{pgfscope}%
\begin{pgfscope}%
\pgfpathrectangle{\pgfqpoint{1.150000in}{0.150000in}}{\pgfqpoint{5.700000in}{5.700000in}}%
\pgfusepath{clip}%
\pgfsetbuttcap%
\pgfsetroundjoin%
\definecolor{currentfill}{rgb}{0.278826,0.175490,0.483397}%
\pgfsetfillcolor{currentfill}%
\pgfsetfillopacity{0.700000}%
\pgfsetlinewidth{0.000000pt}%
\definecolor{currentstroke}{rgb}{0.000000,0.000000,0.000000}%
\pgfsetstrokecolor{currentstroke}%
\pgfsetdash{}{0pt}%
\pgfpathmoveto{\pgfqpoint{4.636116in}{2.398457in}}%
\pgfpathlineto{\pgfqpoint{4.649929in}{2.397556in}}%
\pgfpathlineto{\pgfqpoint{4.663751in}{2.396727in}}%
\pgfpathlineto{\pgfqpoint{4.677581in}{2.395969in}}%
\pgfpathlineto{\pgfqpoint{4.691420in}{2.395284in}}%
\pgfpathlineto{\pgfqpoint{4.699082in}{2.403195in}}%
\pgfpathlineto{\pgfqpoint{4.706738in}{2.411158in}}%
\pgfpathlineto{\pgfqpoint{4.714390in}{2.419177in}}%
\pgfpathlineto{\pgfqpoint{4.722036in}{2.427257in}}%
\pgfpathlineto{\pgfqpoint{4.708212in}{2.428186in}}%
\pgfpathlineto{\pgfqpoint{4.694396in}{2.429187in}}%
\pgfpathlineto{\pgfqpoint{4.680589in}{2.430259in}}%
\pgfpathlineto{\pgfqpoint{4.666790in}{2.431404in}}%
\pgfpathlineto{\pgfqpoint{4.659128in}{2.423073in}}%
\pgfpathlineto{\pgfqpoint{4.651463in}{2.414808in}}%
\pgfpathlineto{\pgfqpoint{4.643792in}{2.406604in}}%
\pgfpathlineto{\pgfqpoint{4.636116in}{2.398457in}}%
\pgfpathclose%
\pgfusepath{fill}%
\end{pgfscope}%
\begin{pgfscope}%
\pgfpathrectangle{\pgfqpoint{1.150000in}{0.150000in}}{\pgfqpoint{5.700000in}{5.700000in}}%
\pgfusepath{clip}%
\pgfsetbuttcap%
\pgfsetroundjoin%
\definecolor{currentfill}{rgb}{0.220057,0.343307,0.549413}%
\pgfsetfillcolor{currentfill}%
\pgfsetfillopacity{0.700000}%
\pgfsetlinewidth{0.000000pt}%
\definecolor{currentstroke}{rgb}{0.000000,0.000000,0.000000}%
\pgfsetstrokecolor{currentstroke}%
\pgfsetdash{}{0pt}%
\pgfpathmoveto{\pgfqpoint{5.833437in}{2.760535in}}%
\pgfpathlineto{\pgfqpoint{5.847578in}{2.759460in}}%
\pgfpathlineto{\pgfqpoint{5.861729in}{2.758450in}}%
\pgfpathlineto{\pgfqpoint{5.875890in}{2.757505in}}%
\pgfpathlineto{\pgfqpoint{5.890062in}{2.756626in}}%
\pgfpathlineto{\pgfqpoint{5.897348in}{2.766366in}}%
\pgfpathlineto{\pgfqpoint{5.904640in}{2.776396in}}%
\pgfpathlineto{\pgfqpoint{5.911938in}{2.786724in}}%
\pgfpathlineto{\pgfqpoint{5.919242in}{2.797357in}}%
\pgfpathlineto{\pgfqpoint{5.905098in}{2.798724in}}%
\pgfpathlineto{\pgfqpoint{5.890964in}{2.800155in}}%
\pgfpathlineto{\pgfqpoint{5.876839in}{2.801651in}}%
\pgfpathlineto{\pgfqpoint{5.862725in}{2.803212in}}%
\pgfpathlineto{\pgfqpoint{5.855394in}{2.792085in}}%
\pgfpathlineto{\pgfqpoint{5.848070in}{2.781269in}}%
\pgfpathlineto{\pgfqpoint{5.840751in}{2.770755in}}%
\pgfpathlineto{\pgfqpoint{5.833437in}{2.760535in}}%
\pgfpathclose%
\pgfusepath{fill}%
\end{pgfscope}%
\begin{pgfscope}%
\pgfpathrectangle{\pgfqpoint{1.150000in}{0.150000in}}{\pgfqpoint{5.700000in}{5.700000in}}%
\pgfusepath{clip}%
\pgfsetbuttcap%
\pgfsetroundjoin%
\definecolor{currentfill}{rgb}{0.250425,0.274290,0.533103}%
\pgfsetfillcolor{currentfill}%
\pgfsetfillopacity{0.700000}%
\pgfsetlinewidth{0.000000pt}%
\definecolor{currentstroke}{rgb}{0.000000,0.000000,0.000000}%
\pgfsetstrokecolor{currentstroke}%
\pgfsetdash{}{0pt}%
\pgfpathmoveto{\pgfqpoint{5.348355in}{2.599578in}}%
\pgfpathlineto{\pgfqpoint{5.362378in}{2.599047in}}%
\pgfpathlineto{\pgfqpoint{5.376411in}{2.598584in}}%
\pgfpathlineto{\pgfqpoint{5.390453in}{2.598188in}}%
\pgfpathlineto{\pgfqpoint{5.404505in}{2.597860in}}%
\pgfpathlineto{\pgfqpoint{5.411907in}{2.605785in}}%
\pgfpathlineto{\pgfqpoint{5.419307in}{2.613876in}}%
\pgfpathlineto{\pgfqpoint{5.426707in}{2.622137in}}%
\pgfpathlineto{\pgfqpoint{5.434106in}{2.630576in}}%
\pgfpathlineto{\pgfqpoint{5.420075in}{2.631291in}}%
\pgfpathlineto{\pgfqpoint{5.406055in}{2.632074in}}%
\pgfpathlineto{\pgfqpoint{5.392043in}{2.632923in}}%
\pgfpathlineto{\pgfqpoint{5.378042in}{2.633840in}}%
\pgfpathlineto{\pgfqpoint{5.370621in}{2.625007in}}%
\pgfpathlineto{\pgfqpoint{5.363200in}{2.616357in}}%
\pgfpathlineto{\pgfqpoint{5.355778in}{2.607883in}}%
\pgfpathlineto{\pgfqpoint{5.348355in}{2.599578in}}%
\pgfpathclose%
\pgfusepath{fill}%
\end{pgfscope}%
\begin{pgfscope}%
\pgfpathrectangle{\pgfqpoint{1.150000in}{0.150000in}}{\pgfqpoint{5.700000in}{5.700000in}}%
\pgfusepath{clip}%
\pgfsetbuttcap%
\pgfsetroundjoin%
\definecolor{currentfill}{rgb}{0.279566,0.067836,0.391917}%
\pgfsetfillcolor{currentfill}%
\pgfsetfillopacity{0.700000}%
\pgfsetlinewidth{0.000000pt}%
\definecolor{currentstroke}{rgb}{0.000000,0.000000,0.000000}%
\pgfsetstrokecolor{currentstroke}%
\pgfsetdash{}{0pt}%
\pgfpathmoveto{\pgfqpoint{3.697025in}{2.183248in}}%
\pgfpathlineto{\pgfqpoint{3.710592in}{2.179903in}}%
\pgfpathlineto{\pgfqpoint{3.724166in}{2.176642in}}%
\pgfpathlineto{\pgfqpoint{3.737745in}{2.173464in}}%
\pgfpathlineto{\pgfqpoint{3.751330in}{2.170370in}}%
\pgfpathlineto{\pgfqpoint{3.759333in}{2.179305in}}%
\pgfpathlineto{\pgfqpoint{3.767329in}{2.188249in}}%
\pgfpathlineto{\pgfqpoint{3.775320in}{2.197202in}}%
\pgfpathlineto{\pgfqpoint{3.783306in}{2.206167in}}%
\pgfpathlineto{\pgfqpoint{3.769731in}{2.209321in}}%
\pgfpathlineto{\pgfqpoint{3.756163in}{2.212557in}}%
\pgfpathlineto{\pgfqpoint{3.742600in}{2.215878in}}%
\pgfpathlineto{\pgfqpoint{3.729043in}{2.219282in}}%
\pgfpathlineto{\pgfqpoint{3.721047in}{2.210251in}}%
\pgfpathlineto{\pgfqpoint{3.713045in}{2.201235in}}%
\pgfpathlineto{\pgfqpoint{3.705038in}{2.192235in}}%
\pgfpathlineto{\pgfqpoint{3.697025in}{2.183248in}}%
\pgfpathclose%
\pgfusepath{fill}%
\end{pgfscope}%
\begin{pgfscope}%
\pgfpathrectangle{\pgfqpoint{1.150000in}{0.150000in}}{\pgfqpoint{5.700000in}{5.700000in}}%
\pgfusepath{clip}%
\pgfsetbuttcap%
\pgfsetroundjoin%
\definecolor{currentfill}{rgb}{0.269308,0.218818,0.509577}%
\pgfsetfillcolor{currentfill}%
\pgfsetfillopacity{0.700000}%
\pgfsetlinewidth{0.000000pt}%
\definecolor{currentstroke}{rgb}{0.000000,0.000000,0.000000}%
\pgfsetstrokecolor{currentstroke}%
\pgfsetdash{}{0pt}%
\pgfpathmoveto{\pgfqpoint{4.949262in}{2.482469in}}%
\pgfpathlineto{\pgfqpoint{4.963171in}{2.481910in}}%
\pgfpathlineto{\pgfqpoint{4.977089in}{2.481421in}}%
\pgfpathlineto{\pgfqpoint{4.991017in}{2.481001in}}%
\pgfpathlineto{\pgfqpoint{5.004954in}{2.480652in}}%
\pgfpathlineto{\pgfqpoint{5.012497in}{2.488309in}}%
\pgfpathlineto{\pgfqpoint{5.020036in}{2.496057in}}%
\pgfpathlineto{\pgfqpoint{5.027572in}{2.503902in}}%
\pgfpathlineto{\pgfqpoint{5.035104in}{2.511849in}}%
\pgfpathlineto{\pgfqpoint{5.021185in}{2.512504in}}%
\pgfpathlineto{\pgfqpoint{5.007274in}{2.513229in}}%
\pgfpathlineto{\pgfqpoint{4.993373in}{2.514023in}}%
\pgfpathlineto{\pgfqpoint{4.979481in}{2.514886in}}%
\pgfpathlineto{\pgfqpoint{4.971932in}{2.506627in}}%
\pgfpathlineto{\pgfqpoint{4.964379in}{2.498475in}}%
\pgfpathlineto{\pgfqpoint{4.956822in}{2.490424in}}%
\pgfpathlineto{\pgfqpoint{4.949262in}{2.482469in}}%
\pgfpathclose%
\pgfusepath{fill}%
\end{pgfscope}%
\begin{pgfscope}%
\pgfpathrectangle{\pgfqpoint{1.150000in}{0.150000in}}{\pgfqpoint{5.700000in}{5.700000in}}%
\pgfusepath{clip}%
\pgfsetbuttcap%
\pgfsetroundjoin%
\definecolor{currentfill}{rgb}{0.277941,0.056324,0.381191}%
\pgfsetfillcolor{currentfill}%
\pgfsetfillopacity{0.700000}%
\pgfsetlinewidth{0.000000pt}%
\definecolor{currentstroke}{rgb}{0.000000,0.000000,0.000000}%
\pgfsetstrokecolor{currentstroke}%
\pgfsetdash{}{0pt}%
\pgfpathmoveto{\pgfqpoint{3.470061in}{2.157820in}}%
\pgfpathlineto{\pgfqpoint{3.483587in}{2.153495in}}%
\pgfpathlineto{\pgfqpoint{3.497118in}{2.149260in}}%
\pgfpathlineto{\pgfqpoint{3.510654in}{2.145112in}}%
\pgfpathlineto{\pgfqpoint{3.524194in}{2.141052in}}%
\pgfpathlineto{\pgfqpoint{3.532276in}{2.149996in}}%
\pgfpathlineto{\pgfqpoint{3.540352in}{2.158955in}}%
\pgfpathlineto{\pgfqpoint{3.548423in}{2.167930in}}%
\pgfpathlineto{\pgfqpoint{3.556487in}{2.176921in}}%
\pgfpathlineto{\pgfqpoint{3.542958in}{2.180999in}}%
\pgfpathlineto{\pgfqpoint{3.529433in}{2.185165in}}%
\pgfpathlineto{\pgfqpoint{3.515914in}{2.189419in}}%
\pgfpathlineto{\pgfqpoint{3.502400in}{2.193761in}}%
\pgfpathlineto{\pgfqpoint{3.494324in}{2.184744in}}%
\pgfpathlineto{\pgfqpoint{3.486242in}{2.175749in}}%
\pgfpathlineto{\pgfqpoint{3.478155in}{2.166774in}}%
\pgfpathlineto{\pgfqpoint{3.470061in}{2.157820in}}%
\pgfpathclose%
\pgfusepath{fill}%
\end{pgfscope}%
\begin{pgfscope}%
\pgfpathrectangle{\pgfqpoint{1.150000in}{0.150000in}}{\pgfqpoint{5.700000in}{5.700000in}}%
\pgfusepath{clip}%
\pgfsetbuttcap%
\pgfsetroundjoin%
\definecolor{currentfill}{rgb}{0.227802,0.326594,0.546532}%
\pgfsetfillcolor{currentfill}%
\pgfsetfillopacity{0.700000}%
\pgfsetlinewidth{0.000000pt}%
\definecolor{currentstroke}{rgb}{0.000000,0.000000,0.000000}%
\pgfsetstrokecolor{currentstroke}%
\pgfsetdash{}{0pt}%
\pgfpathmoveto{\pgfqpoint{5.747660in}{2.725504in}}%
\pgfpathlineto{\pgfqpoint{5.761787in}{2.724635in}}%
\pgfpathlineto{\pgfqpoint{5.775924in}{2.723832in}}%
\pgfpathlineto{\pgfqpoint{5.790071in}{2.723093in}}%
\pgfpathlineto{\pgfqpoint{5.804228in}{2.722420in}}%
\pgfpathlineto{\pgfqpoint{5.811525in}{2.731551in}}%
\pgfpathlineto{\pgfqpoint{5.818825in}{2.740941in}}%
\pgfpathlineto{\pgfqpoint{5.826129in}{2.750599in}}%
\pgfpathlineto{\pgfqpoint{5.833437in}{2.760535in}}%
\pgfpathlineto{\pgfqpoint{5.819306in}{2.761675in}}%
\pgfpathlineto{\pgfqpoint{5.805185in}{2.762880in}}%
\pgfpathlineto{\pgfqpoint{5.791074in}{2.764150in}}%
\pgfpathlineto{\pgfqpoint{5.776973in}{2.765486in}}%
\pgfpathlineto{\pgfqpoint{5.769638in}{2.755077in}}%
\pgfpathlineto{\pgfqpoint{5.762308in}{2.744949in}}%
\pgfpathlineto{\pgfqpoint{5.754982in}{2.735094in}}%
\pgfpathlineto{\pgfqpoint{5.747660in}{2.725504in}}%
\pgfpathclose%
\pgfusepath{fill}%
\end{pgfscope}%
\begin{pgfscope}%
\pgfpathrectangle{\pgfqpoint{1.150000in}{0.150000in}}{\pgfqpoint{5.700000in}{5.700000in}}%
\pgfusepath{clip}%
\pgfsetbuttcap%
\pgfsetroundjoin%
\definecolor{currentfill}{rgb}{0.283187,0.125848,0.444960}%
\pgfsetfillcolor{currentfill}%
\pgfsetfillopacity{0.700000}%
\pgfsetlinewidth{0.000000pt}%
\definecolor{currentstroke}{rgb}{0.000000,0.000000,0.000000}%
\pgfsetstrokecolor{currentstroke}%
\pgfsetdash{}{0pt}%
\pgfpathmoveto{\pgfqpoint{4.237021in}{2.290567in}}%
\pgfpathlineto{\pgfqpoint{4.250723in}{2.288954in}}%
\pgfpathlineto{\pgfqpoint{4.264433in}{2.287418in}}%
\pgfpathlineto{\pgfqpoint{4.278150in}{2.285957in}}%
\pgfpathlineto{\pgfqpoint{4.291875in}{2.284572in}}%
\pgfpathlineto{\pgfqpoint{4.299687in}{2.292965in}}%
\pgfpathlineto{\pgfqpoint{4.307493in}{2.301375in}}%
\pgfpathlineto{\pgfqpoint{4.315294in}{2.309805in}}%
\pgfpathlineto{\pgfqpoint{4.323090in}{2.318259in}}%
\pgfpathlineto{\pgfqpoint{4.309377in}{2.319806in}}%
\pgfpathlineto{\pgfqpoint{4.295672in}{2.321428in}}%
\pgfpathlineto{\pgfqpoint{4.281974in}{2.323126in}}%
\pgfpathlineto{\pgfqpoint{4.268284in}{2.324900in}}%
\pgfpathlineto{\pgfqpoint{4.260477in}{2.316277in}}%
\pgfpathlineto{\pgfqpoint{4.252664in}{2.307682in}}%
\pgfpathlineto{\pgfqpoint{4.244845in}{2.299113in}}%
\pgfpathlineto{\pgfqpoint{4.237021in}{2.290567in}}%
\pgfpathclose%
\pgfusepath{fill}%
\end{pgfscope}%
\begin{pgfscope}%
\pgfpathrectangle{\pgfqpoint{1.150000in}{0.150000in}}{\pgfqpoint{5.700000in}{5.700000in}}%
\pgfusepath{clip}%
\pgfsetbuttcap%
\pgfsetroundjoin%
\definecolor{currentfill}{rgb}{0.281924,0.089666,0.412415}%
\pgfsetfillcolor{currentfill}%
\pgfsetfillopacity{0.700000}%
\pgfsetlinewidth{0.000000pt}%
\definecolor{currentstroke}{rgb}{0.000000,0.000000,0.000000}%
\pgfsetstrokecolor{currentstroke}%
\pgfsetdash{}{0pt}%
\pgfpathmoveto{\pgfqpoint{3.923924in}{2.219260in}}%
\pgfpathlineto{\pgfqpoint{3.937545in}{2.216761in}}%
\pgfpathlineto{\pgfqpoint{3.951172in}{2.214342in}}%
\pgfpathlineto{\pgfqpoint{3.964807in}{2.212002in}}%
\pgfpathlineto{\pgfqpoint{3.978448in}{2.209743in}}%
\pgfpathlineto{\pgfqpoint{3.986372in}{2.218512in}}%
\pgfpathlineto{\pgfqpoint{3.994291in}{2.227287in}}%
\pgfpathlineto{\pgfqpoint{4.002204in}{2.236072in}}%
\pgfpathlineto{\pgfqpoint{4.010112in}{2.244868in}}%
\pgfpathlineto{\pgfqpoint{3.996482in}{2.247228in}}%
\pgfpathlineto{\pgfqpoint{3.982858in}{2.249668in}}%
\pgfpathlineto{\pgfqpoint{3.969242in}{2.252187in}}%
\pgfpathlineto{\pgfqpoint{3.955632in}{2.254786in}}%
\pgfpathlineto{\pgfqpoint{3.947713in}{2.245882in}}%
\pgfpathlineto{\pgfqpoint{3.939789in}{2.236995in}}%
\pgfpathlineto{\pgfqpoint{3.931859in}{2.228122in}}%
\pgfpathlineto{\pgfqpoint{3.923924in}{2.219260in}}%
\pgfpathclose%
\pgfusepath{fill}%
\end{pgfscope}%
\begin{pgfscope}%
\pgfpathrectangle{\pgfqpoint{1.150000in}{0.150000in}}{\pgfqpoint{5.700000in}{5.700000in}}%
\pgfusepath{clip}%
\pgfsetbuttcap%
\pgfsetroundjoin%
\definecolor{currentfill}{rgb}{0.280255,0.165693,0.476498}%
\pgfsetfillcolor{currentfill}%
\pgfsetfillopacity{0.700000}%
\pgfsetlinewidth{0.000000pt}%
\definecolor{currentstroke}{rgb}{0.000000,0.000000,0.000000}%
\pgfsetstrokecolor{currentstroke}%
\pgfsetdash{}{0pt}%
\pgfpathmoveto{\pgfqpoint{4.550141in}{2.369789in}}%
\pgfpathlineto{\pgfqpoint{4.563933in}{2.368820in}}%
\pgfpathlineto{\pgfqpoint{4.577735in}{2.367925in}}%
\pgfpathlineto{\pgfqpoint{4.591545in}{2.367102in}}%
\pgfpathlineto{\pgfqpoint{4.605363in}{2.366352in}}%
\pgfpathlineto{\pgfqpoint{4.613059in}{2.374314in}}%
\pgfpathlineto{\pgfqpoint{4.620750in}{2.382316in}}%
\pgfpathlineto{\pgfqpoint{4.628436in}{2.390363in}}%
\pgfpathlineto{\pgfqpoint{4.636116in}{2.398457in}}%
\pgfpathlineto{\pgfqpoint{4.622312in}{2.399431in}}%
\pgfpathlineto{\pgfqpoint{4.608516in}{2.400477in}}%
\pgfpathlineto{\pgfqpoint{4.594728in}{2.401595in}}%
\pgfpathlineto{\pgfqpoint{4.580949in}{2.402787in}}%
\pgfpathlineto{\pgfqpoint{4.573254in}{2.394462in}}%
\pgfpathlineto{\pgfqpoint{4.565555in}{2.386190in}}%
\pgfpathlineto{\pgfqpoint{4.557850in}{2.377967in}}%
\pgfpathlineto{\pgfqpoint{4.550141in}{2.369789in}}%
\pgfpathclose%
\pgfusepath{fill}%
\end{pgfscope}%
\begin{pgfscope}%
\pgfpathrectangle{\pgfqpoint{1.150000in}{0.150000in}}{\pgfqpoint{5.700000in}{5.700000in}}%
\pgfusepath{clip}%
\pgfsetbuttcap%
\pgfsetroundjoin%
\definecolor{currentfill}{rgb}{0.277941,0.056324,0.381191}%
\pgfsetfillcolor{currentfill}%
\pgfsetfillopacity{0.700000}%
\pgfsetlinewidth{0.000000pt}%
\definecolor{currentstroke}{rgb}{0.000000,0.000000,0.000000}%
\pgfsetstrokecolor{currentstroke}%
\pgfsetdash{}{0pt}%
\pgfpathmoveto{\pgfqpoint{3.102096in}{2.158945in}}%
\pgfpathlineto{\pgfqpoint{3.115581in}{2.152678in}}%
\pgfpathlineto{\pgfqpoint{3.129069in}{2.146509in}}%
\pgfpathlineto{\pgfqpoint{3.142559in}{2.140439in}}%
\pgfpathlineto{\pgfqpoint{3.156053in}{2.134467in}}%
\pgfpathlineto{\pgfqpoint{3.164272in}{2.143008in}}%
\pgfpathlineto{\pgfqpoint{3.172485in}{2.151587in}}%
\pgfpathlineto{\pgfqpoint{3.180690in}{2.160203in}}%
\pgfpathlineto{\pgfqpoint{3.188890in}{2.168857in}}%
\pgfpathlineto{\pgfqpoint{3.175409in}{2.174786in}}%
\pgfpathlineto{\pgfqpoint{3.161932in}{2.180813in}}%
\pgfpathlineto{\pgfqpoint{3.148458in}{2.186938in}}%
\pgfpathlineto{\pgfqpoint{3.134987in}{2.193162in}}%
\pgfpathlineto{\pgfqpoint{3.126775in}{2.184544in}}%
\pgfpathlineto{\pgfqpoint{3.118555in}{2.175969in}}%
\pgfpathlineto{\pgfqpoint{3.110329in}{2.167435in}}%
\pgfpathlineto{\pgfqpoint{3.102096in}{2.158945in}}%
\pgfpathclose%
\pgfusepath{fill}%
\end{pgfscope}%
\begin{pgfscope}%
\pgfpathrectangle{\pgfqpoint{1.150000in}{0.150000in}}{\pgfqpoint{5.700000in}{5.700000in}}%
\pgfusepath{clip}%
\pgfsetbuttcap%
\pgfsetroundjoin%
\definecolor{currentfill}{rgb}{0.277134,0.185228,0.489898}%
\pgfsetfillcolor{currentfill}%
\pgfsetfillopacity{0.700000}%
\pgfsetlinewidth{0.000000pt}%
\definecolor{currentstroke}{rgb}{0.000000,0.000000,0.000000}%
\pgfsetstrokecolor{currentstroke}%
\pgfsetdash{}{0pt}%
\pgfpathmoveto{\pgfqpoint{2.374429in}{2.431364in}}%
\pgfpathlineto{\pgfqpoint{2.387965in}{2.419563in}}%
\pgfpathlineto{\pgfqpoint{2.401498in}{2.407901in}}%
\pgfpathlineto{\pgfqpoint{2.415028in}{2.396377in}}%
\pgfpathlineto{\pgfqpoint{2.428557in}{2.384990in}}%
\pgfpathlineto{\pgfqpoint{2.437101in}{2.391354in}}%
\pgfpathlineto{\pgfqpoint{2.445634in}{2.397827in}}%
\pgfpathlineto{\pgfqpoint{2.454157in}{2.404407in}}%
\pgfpathlineto{\pgfqpoint{2.462670in}{2.411095in}}%
\pgfpathlineto{\pgfqpoint{2.449163in}{2.422353in}}%
\pgfpathlineto{\pgfqpoint{2.435654in}{2.433748in}}%
\pgfpathlineto{\pgfqpoint{2.422143in}{2.445281in}}%
\pgfpathlineto{\pgfqpoint{2.408630in}{2.456952in}}%
\pgfpathlineto{\pgfqpoint{2.400096in}{2.450386in}}%
\pgfpathlineto{\pgfqpoint{2.391551in}{2.443932in}}%
\pgfpathlineto{\pgfqpoint{2.382995in}{2.437591in}}%
\pgfpathlineto{\pgfqpoint{2.374429in}{2.431364in}}%
\pgfpathclose%
\pgfusepath{fill}%
\end{pgfscope}%
\begin{pgfscope}%
\pgfpathrectangle{\pgfqpoint{1.150000in}{0.150000in}}{\pgfqpoint{5.700000in}{5.700000in}}%
\pgfusepath{clip}%
\pgfsetbuttcap%
\pgfsetroundjoin%
\definecolor{currentfill}{rgb}{0.278791,0.062145,0.386592}%
\pgfsetfillcolor{currentfill}%
\pgfsetfillopacity{0.700000}%
\pgfsetlinewidth{0.000000pt}%
\definecolor{currentstroke}{rgb}{0.000000,0.000000,0.000000}%
\pgfsetstrokecolor{currentstroke}%
\pgfsetdash{}{0pt}%
\pgfpathmoveto{\pgfqpoint{2.961199in}{2.179708in}}%
\pgfpathlineto{\pgfqpoint{2.974678in}{2.172562in}}%
\pgfpathlineto{\pgfqpoint{2.988159in}{2.165520in}}%
\pgfpathlineto{\pgfqpoint{3.001642in}{2.158581in}}%
\pgfpathlineto{\pgfqpoint{3.015128in}{2.151746in}}%
\pgfpathlineto{\pgfqpoint{3.023403in}{2.159990in}}%
\pgfpathlineto{\pgfqpoint{3.031671in}{2.168283in}}%
\pgfpathlineto{\pgfqpoint{3.039932in}{2.176626in}}%
\pgfpathlineto{\pgfqpoint{3.048187in}{2.185017in}}%
\pgfpathlineto{\pgfqpoint{3.034716in}{2.191789in}}%
\pgfpathlineto{\pgfqpoint{3.021248in}{2.198664in}}%
\pgfpathlineto{\pgfqpoint{3.007782in}{2.205642in}}%
\pgfpathlineto{\pgfqpoint{2.994318in}{2.212724in}}%
\pgfpathlineto{\pgfqpoint{2.986049in}{2.204389in}}%
\pgfpathlineto{\pgfqpoint{2.977773in}{2.196108in}}%
\pgfpathlineto{\pgfqpoint{2.969490in}{2.187881in}}%
\pgfpathlineto{\pgfqpoint{2.961199in}{2.179708in}}%
\pgfpathclose%
\pgfusepath{fill}%
\end{pgfscope}%
\begin{pgfscope}%
\pgfpathrectangle{\pgfqpoint{1.150000in}{0.150000in}}{\pgfqpoint{5.700000in}{5.700000in}}%
\pgfusepath{clip}%
\pgfsetbuttcap%
\pgfsetroundjoin%
\definecolor{currentfill}{rgb}{0.255645,0.260703,0.528312}%
\pgfsetfillcolor{currentfill}%
\pgfsetfillopacity{0.700000}%
\pgfsetlinewidth{0.000000pt}%
\definecolor{currentstroke}{rgb}{0.000000,0.000000,0.000000}%
\pgfsetstrokecolor{currentstroke}%
\pgfsetdash{}{0pt}%
\pgfpathmoveto{\pgfqpoint{5.262571in}{2.569246in}}%
\pgfpathlineto{\pgfqpoint{5.276576in}{2.568812in}}%
\pgfpathlineto{\pgfqpoint{5.290590in}{2.568445in}}%
\pgfpathlineto{\pgfqpoint{5.304614in}{2.568145in}}%
\pgfpathlineto{\pgfqpoint{5.318648in}{2.567914in}}%
\pgfpathlineto{\pgfqpoint{5.326078in}{2.575610in}}%
\pgfpathlineto{\pgfqpoint{5.333505in}{2.583448in}}%
\pgfpathlineto{\pgfqpoint{5.340931in}{2.591435in}}%
\pgfpathlineto{\pgfqpoint{5.348355in}{2.599578in}}%
\pgfpathlineto{\pgfqpoint{5.334342in}{2.600176in}}%
\pgfpathlineto{\pgfqpoint{5.320339in}{2.600842in}}%
\pgfpathlineto{\pgfqpoint{5.306345in}{2.601575in}}%
\pgfpathlineto{\pgfqpoint{5.292361in}{2.602376in}}%
\pgfpathlineto{\pgfqpoint{5.284916in}{2.593859in}}%
\pgfpathlineto{\pgfqpoint{5.277469in}{2.585503in}}%
\pgfpathlineto{\pgfqpoint{5.270021in}{2.577301in}}%
\pgfpathlineto{\pgfqpoint{5.262571in}{2.569246in}}%
\pgfpathclose%
\pgfusepath{fill}%
\end{pgfscope}%
\begin{pgfscope}%
\pgfpathrectangle{\pgfqpoint{1.150000in}{0.150000in}}{\pgfqpoint{5.700000in}{5.700000in}}%
\pgfusepath{clip}%
\pgfsetbuttcap%
\pgfsetroundjoin%
\definecolor{currentfill}{rgb}{0.283197,0.115680,0.436115}%
\pgfsetfillcolor{currentfill}%
\pgfsetfillopacity{0.700000}%
\pgfsetlinewidth{0.000000pt}%
\definecolor{currentstroke}{rgb}{0.000000,0.000000,0.000000}%
\pgfsetstrokecolor{currentstroke}%
\pgfsetdash{}{0pt}%
\pgfpathmoveto{\pgfqpoint{2.624645in}{2.286211in}}%
\pgfpathlineto{\pgfqpoint{2.638138in}{2.276622in}}%
\pgfpathlineto{\pgfqpoint{2.651630in}{2.267154in}}%
\pgfpathlineto{\pgfqpoint{2.665123in}{2.257806in}}%
\pgfpathlineto{\pgfqpoint{2.678616in}{2.248577in}}%
\pgfpathlineto{\pgfqpoint{2.687039in}{2.255828in}}%
\pgfpathlineto{\pgfqpoint{2.695454in}{2.263163in}}%
\pgfpathlineto{\pgfqpoint{2.703859in}{2.270578in}}%
\pgfpathlineto{\pgfqpoint{2.712257in}{2.278075in}}%
\pgfpathlineto{\pgfqpoint{2.698782in}{2.287197in}}%
\pgfpathlineto{\pgfqpoint{2.685308in}{2.296439in}}%
\pgfpathlineto{\pgfqpoint{2.671834in}{2.305800in}}%
\pgfpathlineto{\pgfqpoint{2.658360in}{2.315283in}}%
\pgfpathlineto{\pgfqpoint{2.649945in}{2.307885in}}%
\pgfpathlineto{\pgfqpoint{2.641520in}{2.300573in}}%
\pgfpathlineto{\pgfqpoint{2.633087in}{2.293348in}}%
\pgfpathlineto{\pgfqpoint{2.624645in}{2.286211in}}%
\pgfpathclose%
\pgfusepath{fill}%
\end{pgfscope}%
\begin{pgfscope}%
\pgfpathrectangle{\pgfqpoint{1.150000in}{0.150000in}}{\pgfqpoint{5.700000in}{5.700000in}}%
\pgfusepath{clip}%
\pgfsetbuttcap%
\pgfsetroundjoin%
\definecolor{currentfill}{rgb}{0.233603,0.313828,0.543914}%
\pgfsetfillcolor{currentfill}%
\pgfsetfillopacity{0.700000}%
\pgfsetlinewidth{0.000000pt}%
\definecolor{currentstroke}{rgb}{0.000000,0.000000,0.000000}%
\pgfsetstrokecolor{currentstroke}%
\pgfsetdash{}{0pt}%
\pgfpathmoveto{\pgfqpoint{5.661891in}{2.691976in}}%
\pgfpathlineto{\pgfqpoint{5.676003in}{2.691291in}}%
\pgfpathlineto{\pgfqpoint{5.690126in}{2.690672in}}%
\pgfpathlineto{\pgfqpoint{5.704258in}{2.690119in}}%
\pgfpathlineto{\pgfqpoint{5.718401in}{2.689632in}}%
\pgfpathlineto{\pgfqpoint{5.725711in}{2.698243in}}%
\pgfpathlineto{\pgfqpoint{5.733025in}{2.707087in}}%
\pgfpathlineto{\pgfqpoint{5.740341in}{2.716171in}}%
\pgfpathlineto{\pgfqpoint{5.747660in}{2.725504in}}%
\pgfpathlineto{\pgfqpoint{5.733542in}{2.726439in}}%
\pgfpathlineto{\pgfqpoint{5.719435in}{2.727439in}}%
\pgfpathlineto{\pgfqpoint{5.705338in}{2.728505in}}%
\pgfpathlineto{\pgfqpoint{5.691251in}{2.729637in}}%
\pgfpathlineto{\pgfqpoint{5.683907in}{2.719850in}}%
\pgfpathlineto{\pgfqpoint{5.676566in}{2.710316in}}%
\pgfpathlineto{\pgfqpoint{5.669227in}{2.701027in}}%
\pgfpathlineto{\pgfqpoint{5.661891in}{2.691976in}}%
\pgfpathclose%
\pgfusepath{fill}%
\end{pgfscope}%
\begin{pgfscope}%
\pgfpathrectangle{\pgfqpoint{1.150000in}{0.150000in}}{\pgfqpoint{5.700000in}{5.700000in}}%
\pgfusepath{clip}%
\pgfsetbuttcap%
\pgfsetroundjoin%
\definecolor{currentfill}{rgb}{0.277018,0.050344,0.375715}%
\pgfsetfillcolor{currentfill}%
\pgfsetfillopacity{0.700000}%
\pgfsetlinewidth{0.000000pt}%
\definecolor{currentstroke}{rgb}{0.000000,0.000000,0.000000}%
\pgfsetstrokecolor{currentstroke}%
\pgfsetdash{}{0pt}%
\pgfpathmoveto{\pgfqpoint{3.242846in}{2.146107in}}%
\pgfpathlineto{\pgfqpoint{3.256345in}{2.140658in}}%
\pgfpathlineto{\pgfqpoint{3.269847in}{2.135304in}}%
\pgfpathlineto{\pgfqpoint{3.283353in}{2.130043in}}%
\pgfpathlineto{\pgfqpoint{3.296863in}{2.124876in}}%
\pgfpathlineto{\pgfqpoint{3.305030in}{2.133628in}}%
\pgfpathlineto{\pgfqpoint{3.313190in}{2.142406in}}%
\pgfpathlineto{\pgfqpoint{3.321344in}{2.151212in}}%
\pgfpathlineto{\pgfqpoint{3.329492in}{2.160045in}}%
\pgfpathlineto{\pgfqpoint{3.315994in}{2.165190in}}%
\pgfpathlineto{\pgfqpoint{3.302501in}{2.170428in}}%
\pgfpathlineto{\pgfqpoint{3.289012in}{2.175759in}}%
\pgfpathlineto{\pgfqpoint{3.275526in}{2.181185in}}%
\pgfpathlineto{\pgfqpoint{3.267366in}{2.172367in}}%
\pgfpathlineto{\pgfqpoint{3.259199in}{2.163582in}}%
\pgfpathlineto{\pgfqpoint{3.251026in}{2.154828in}}%
\pgfpathlineto{\pgfqpoint{3.242846in}{2.146107in}}%
\pgfpathclose%
\pgfusepath{fill}%
\end{pgfscope}%
\begin{pgfscope}%
\pgfpathrectangle{\pgfqpoint{1.150000in}{0.150000in}}{\pgfqpoint{5.700000in}{5.700000in}}%
\pgfusepath{clip}%
\pgfsetbuttcap%
\pgfsetroundjoin%
\definecolor{currentfill}{rgb}{0.271828,0.209303,0.504434}%
\pgfsetfillcolor{currentfill}%
\pgfsetfillopacity{0.700000}%
\pgfsetlinewidth{0.000000pt}%
\definecolor{currentstroke}{rgb}{0.000000,0.000000,0.000000}%
\pgfsetstrokecolor{currentstroke}%
\pgfsetdash{}{0pt}%
\pgfpathmoveto{\pgfqpoint{4.863369in}{2.453297in}}%
\pgfpathlineto{\pgfqpoint{4.877258in}{2.452743in}}%
\pgfpathlineto{\pgfqpoint{4.891156in}{2.452259in}}%
\pgfpathlineto{\pgfqpoint{4.905063in}{2.451845in}}%
\pgfpathlineto{\pgfqpoint{4.918979in}{2.451501in}}%
\pgfpathlineto{\pgfqpoint{4.926556in}{2.459126in}}%
\pgfpathlineto{\pgfqpoint{4.934129in}{2.466825in}}%
\pgfpathlineto{\pgfqpoint{4.941698in}{2.474604in}}%
\pgfpathlineto{\pgfqpoint{4.949262in}{2.482469in}}%
\pgfpathlineto{\pgfqpoint{4.935362in}{2.483098in}}%
\pgfpathlineto{\pgfqpoint{4.921471in}{2.483796in}}%
\pgfpathlineto{\pgfqpoint{4.907589in}{2.484565in}}%
\pgfpathlineto{\pgfqpoint{4.893717in}{2.485404in}}%
\pgfpathlineto{\pgfqpoint{4.886136in}{2.477247in}}%
\pgfpathlineto{\pgfqpoint{4.878551in}{2.469181in}}%
\pgfpathlineto{\pgfqpoint{4.870962in}{2.461199in}}%
\pgfpathlineto{\pgfqpoint{4.863369in}{2.453297in}}%
\pgfpathclose%
\pgfusepath{fill}%
\end{pgfscope}%
\begin{pgfscope}%
\pgfpathrectangle{\pgfqpoint{1.150000in}{0.150000in}}{\pgfqpoint{5.700000in}{5.700000in}}%
\pgfusepath{clip}%
\pgfsetbuttcap%
\pgfsetroundjoin%
\definecolor{currentfill}{rgb}{0.280894,0.078907,0.402329}%
\pgfsetfillcolor{currentfill}%
\pgfsetfillopacity{0.700000}%
\pgfsetlinewidth{0.000000pt}%
\definecolor{currentstroke}{rgb}{0.000000,0.000000,0.000000}%
\pgfsetstrokecolor{currentstroke}%
\pgfsetdash{}{0pt}%
\pgfpathmoveto{\pgfqpoint{2.820071in}{2.209270in}}%
\pgfpathlineto{\pgfqpoint{2.833552in}{2.201178in}}%
\pgfpathlineto{\pgfqpoint{2.847035in}{2.193196in}}%
\pgfpathlineto{\pgfqpoint{2.860519in}{2.185324in}}%
\pgfpathlineto{\pgfqpoint{2.874004in}{2.177561in}}%
\pgfpathlineto{\pgfqpoint{2.882341in}{2.185415in}}%
\pgfpathlineto{\pgfqpoint{2.890670in}{2.193332in}}%
\pgfpathlineto{\pgfqpoint{2.898991in}{2.201311in}}%
\pgfpathlineto{\pgfqpoint{2.907305in}{2.209351in}}%
\pgfpathlineto{\pgfqpoint{2.893836in}{2.217029in}}%
\pgfpathlineto{\pgfqpoint{2.880369in}{2.224817in}}%
\pgfpathlineto{\pgfqpoint{2.866903in}{2.232713in}}%
\pgfpathlineto{\pgfqpoint{2.853439in}{2.240720in}}%
\pgfpathlineto{\pgfqpoint{2.845109in}{2.232757in}}%
\pgfpathlineto{\pgfqpoint{2.836771in}{2.224860in}}%
\pgfpathlineto{\pgfqpoint{2.828425in}{2.217031in}}%
\pgfpathlineto{\pgfqpoint{2.820071in}{2.209270in}}%
\pgfpathclose%
\pgfusepath{fill}%
\end{pgfscope}%
\begin{pgfscope}%
\pgfpathrectangle{\pgfqpoint{1.150000in}{0.150000in}}{\pgfqpoint{5.700000in}{5.700000in}}%
\pgfusepath{clip}%
\pgfsetbuttcap%
\pgfsetroundjoin%
\definecolor{currentfill}{rgb}{0.283197,0.115680,0.436115}%
\pgfsetfillcolor{currentfill}%
\pgfsetfillopacity{0.700000}%
\pgfsetlinewidth{0.000000pt}%
\definecolor{currentstroke}{rgb}{0.000000,0.000000,0.000000}%
\pgfsetstrokecolor{currentstroke}%
\pgfsetdash{}{0pt}%
\pgfpathmoveto{\pgfqpoint{4.150893in}{2.263193in}}%
\pgfpathlineto{\pgfqpoint{4.164576in}{2.261415in}}%
\pgfpathlineto{\pgfqpoint{4.178267in}{2.259715in}}%
\pgfpathlineto{\pgfqpoint{4.191965in}{2.258091in}}%
\pgfpathlineto{\pgfqpoint{4.205671in}{2.256544in}}%
\pgfpathlineto{\pgfqpoint{4.213517in}{2.265031in}}%
\pgfpathlineto{\pgfqpoint{4.221357in}{2.273528in}}%
\pgfpathlineto{\pgfqpoint{4.229192in}{2.282039in}}%
\pgfpathlineto{\pgfqpoint{4.237021in}{2.290567in}}%
\pgfpathlineto{\pgfqpoint{4.223327in}{2.292255in}}%
\pgfpathlineto{\pgfqpoint{4.209640in}{2.294020in}}%
\pgfpathlineto{\pgfqpoint{4.195961in}{2.295862in}}%
\pgfpathlineto{\pgfqpoint{4.182289in}{2.297780in}}%
\pgfpathlineto{\pgfqpoint{4.174449in}{2.289104in}}%
\pgfpathlineto{\pgfqpoint{4.166602in}{2.280449in}}%
\pgfpathlineto{\pgfqpoint{4.158750in}{2.271813in}}%
\pgfpathlineto{\pgfqpoint{4.150893in}{2.263193in}}%
\pgfpathclose%
\pgfusepath{fill}%
\end{pgfscope}%
\begin{pgfscope}%
\pgfpathrectangle{\pgfqpoint{1.150000in}{0.150000in}}{\pgfqpoint{5.700000in}{5.700000in}}%
\pgfusepath{clip}%
\pgfsetbuttcap%
\pgfsetroundjoin%
\definecolor{currentfill}{rgb}{0.278791,0.062145,0.386592}%
\pgfsetfillcolor{currentfill}%
\pgfsetfillopacity{0.700000}%
\pgfsetlinewidth{0.000000pt}%
\definecolor{currentstroke}{rgb}{0.000000,0.000000,0.000000}%
\pgfsetstrokecolor{currentstroke}%
\pgfsetdash{}{0pt}%
\pgfpathmoveto{\pgfqpoint{3.610657in}{2.161478in}}%
\pgfpathlineto{\pgfqpoint{3.624213in}{2.157832in}}%
\pgfpathlineto{\pgfqpoint{3.637774in}{2.154272in}}%
\pgfpathlineto{\pgfqpoint{3.651342in}{2.150797in}}%
\pgfpathlineto{\pgfqpoint{3.664914in}{2.147407in}}%
\pgfpathlineto{\pgfqpoint{3.672951in}{2.156353in}}%
\pgfpathlineto{\pgfqpoint{3.680981in}{2.165308in}}%
\pgfpathlineto{\pgfqpoint{3.689006in}{2.174273in}}%
\pgfpathlineto{\pgfqpoint{3.697025in}{2.183248in}}%
\pgfpathlineto{\pgfqpoint{3.683463in}{2.186677in}}%
\pgfpathlineto{\pgfqpoint{3.669907in}{2.190191in}}%
\pgfpathlineto{\pgfqpoint{3.656356in}{2.193790in}}%
\pgfpathlineto{\pgfqpoint{3.642811in}{2.197474in}}%
\pgfpathlineto{\pgfqpoint{3.634781in}{2.188452in}}%
\pgfpathlineto{\pgfqpoint{3.626746in}{2.179447in}}%
\pgfpathlineto{\pgfqpoint{3.618704in}{2.170455in}}%
\pgfpathlineto{\pgfqpoint{3.610657in}{2.161478in}}%
\pgfpathclose%
\pgfusepath{fill}%
\end{pgfscope}%
\begin{pgfscope}%
\pgfpathrectangle{\pgfqpoint{1.150000in}{0.150000in}}{\pgfqpoint{5.700000in}{5.700000in}}%
\pgfusepath{clip}%
\pgfsetbuttcap%
\pgfsetroundjoin%
\definecolor{currentfill}{rgb}{0.281412,0.155834,0.469201}%
\pgfsetfillcolor{currentfill}%
\pgfsetfillopacity{0.700000}%
\pgfsetlinewidth{0.000000pt}%
\definecolor{currentstroke}{rgb}{0.000000,0.000000,0.000000}%
\pgfsetstrokecolor{currentstroke}%
\pgfsetdash{}{0pt}%
\pgfpathmoveto{\pgfqpoint{4.464109in}{2.341240in}}%
\pgfpathlineto{\pgfqpoint{4.477882in}{2.340182in}}%
\pgfpathlineto{\pgfqpoint{4.491663in}{2.339197in}}%
\pgfpathlineto{\pgfqpoint{4.505452in}{2.338285in}}%
\pgfpathlineto{\pgfqpoint{4.519250in}{2.337447in}}%
\pgfpathlineto{\pgfqpoint{4.526980in}{2.345485in}}%
\pgfpathlineto{\pgfqpoint{4.534706in}{2.353551in}}%
\pgfpathlineto{\pgfqpoint{4.542426in}{2.361652in}}%
\pgfpathlineto{\pgfqpoint{4.550141in}{2.369789in}}%
\pgfpathlineto{\pgfqpoint{4.536356in}{2.370830in}}%
\pgfpathlineto{\pgfqpoint{4.522580in}{2.371944in}}%
\pgfpathlineto{\pgfqpoint{4.508812in}{2.373132in}}%
\pgfpathlineto{\pgfqpoint{4.495052in}{2.374393in}}%
\pgfpathlineto{\pgfqpoint{4.487324in}{2.366046in}}%
\pgfpathlineto{\pgfqpoint{4.479591in}{2.357740in}}%
\pgfpathlineto{\pgfqpoint{4.471853in}{2.349473in}}%
\pgfpathlineto{\pgfqpoint{4.464109in}{2.341240in}}%
\pgfpathclose%
\pgfusepath{fill}%
\end{pgfscope}%
\begin{pgfscope}%
\pgfpathrectangle{\pgfqpoint{1.150000in}{0.150000in}}{\pgfqpoint{5.700000in}{5.700000in}}%
\pgfusepath{clip}%
\pgfsetbuttcap%
\pgfsetroundjoin%
\definecolor{currentfill}{rgb}{0.204903,0.375746,0.553533}%
\pgfsetfillcolor{currentfill}%
\pgfsetfillopacity{0.700000}%
\pgfsetlinewidth{0.000000pt}%
\definecolor{currentstroke}{rgb}{0.000000,0.000000,0.000000}%
\pgfsetstrokecolor{currentstroke}%
\pgfsetdash{}{0pt}%
\pgfpathmoveto{\pgfqpoint{6.061817in}{2.830472in}}%
\pgfpathlineto{\pgfqpoint{6.076023in}{2.829180in}}%
\pgfpathlineto{\pgfqpoint{6.090240in}{2.827951in}}%
\pgfpathlineto{\pgfqpoint{6.104466in}{2.826787in}}%
\pgfpathlineto{\pgfqpoint{6.111729in}{2.837659in}}%
\pgfpathlineto{\pgfqpoint{6.119001in}{2.848878in}}%
\pgfpathlineto{\pgfqpoint{6.126283in}{2.860454in}}%
\pgfpathlineto{\pgfqpoint{6.133575in}{2.872397in}}%
\pgfpathlineto{\pgfqpoint{6.119378in}{2.874088in}}%
\pgfpathlineto{\pgfqpoint{6.105191in}{2.875844in}}%
\pgfpathlineto{\pgfqpoint{6.091014in}{2.877663in}}%
\pgfpathlineto{\pgfqpoint{6.083700in}{2.865320in}}%
\pgfpathlineto{\pgfqpoint{6.076396in}{2.853347in}}%
\pgfpathlineto{\pgfqpoint{6.069102in}{2.841734in}}%
\pgfpathlineto{\pgfqpoint{6.061817in}{2.830472in}}%
\pgfpathclose%
\pgfusepath{fill}%
\end{pgfscope}%
\begin{pgfscope}%
\pgfpathrectangle{\pgfqpoint{1.150000in}{0.150000in}}{\pgfqpoint{5.700000in}{5.700000in}}%
\pgfusepath{clip}%
\pgfsetbuttcap%
\pgfsetroundjoin%
\definecolor{currentfill}{rgb}{0.237441,0.305202,0.541921}%
\pgfsetfillcolor{currentfill}%
\pgfsetfillopacity{0.700000}%
\pgfsetlinewidth{0.000000pt}%
\definecolor{currentstroke}{rgb}{0.000000,0.000000,0.000000}%
\pgfsetstrokecolor{currentstroke}%
\pgfsetdash{}{0pt}%
\pgfpathmoveto{\pgfqpoint{5.576117in}{2.659684in}}%
\pgfpathlineto{\pgfqpoint{5.590214in}{2.659161in}}%
\pgfpathlineto{\pgfqpoint{5.604320in}{2.658706in}}%
\pgfpathlineto{\pgfqpoint{5.618437in}{2.658316in}}%
\pgfpathlineto{\pgfqpoint{5.632563in}{2.657993in}}%
\pgfpathlineto{\pgfqpoint{5.639893in}{2.666171in}}%
\pgfpathlineto{\pgfqpoint{5.647225in}{2.674556in}}%
\pgfpathlineto{\pgfqpoint{5.654557in}{2.683155in}}%
\pgfpathlineto{\pgfqpoint{5.661891in}{2.691976in}}%
\pgfpathlineto{\pgfqpoint{5.647789in}{2.692727in}}%
\pgfpathlineto{\pgfqpoint{5.633697in}{2.693544in}}%
\pgfpathlineto{\pgfqpoint{5.619614in}{2.694427in}}%
\pgfpathlineto{\pgfqpoint{5.605542in}{2.695375in}}%
\pgfpathlineto{\pgfqpoint{5.598184in}{2.686120in}}%
\pgfpathlineto{\pgfqpoint{5.590827in}{2.677091in}}%
\pgfpathlineto{\pgfqpoint{5.583472in}{2.668282in}}%
\pgfpathlineto{\pgfqpoint{5.576117in}{2.659684in}}%
\pgfpathclose%
\pgfusepath{fill}%
\end{pgfscope}%
\begin{pgfscope}%
\pgfpathrectangle{\pgfqpoint{1.150000in}{0.150000in}}{\pgfqpoint{5.700000in}{5.700000in}}%
\pgfusepath{clip}%
\pgfsetbuttcap%
\pgfsetroundjoin%
\definecolor{currentfill}{rgb}{0.258965,0.251537,0.524736}%
\pgfsetfillcolor{currentfill}%
\pgfsetfillopacity{0.700000}%
\pgfsetlinewidth{0.000000pt}%
\definecolor{currentstroke}{rgb}{0.000000,0.000000,0.000000}%
\pgfsetstrokecolor{currentstroke}%
\pgfsetdash{}{0pt}%
\pgfpathmoveto{\pgfqpoint{5.176746in}{2.539410in}}%
\pgfpathlineto{\pgfqpoint{5.190732in}{2.539049in}}%
\pgfpathlineto{\pgfqpoint{5.204727in}{2.538756in}}%
\pgfpathlineto{\pgfqpoint{5.218732in}{2.538532in}}%
\pgfpathlineto{\pgfqpoint{5.232747in}{2.538376in}}%
\pgfpathlineto{\pgfqpoint{5.240207in}{2.545904in}}%
\pgfpathlineto{\pgfqpoint{5.247664in}{2.553554in}}%
\pgfpathlineto{\pgfqpoint{5.255119in}{2.561333in}}%
\pgfpathlineto{\pgfqpoint{5.262571in}{2.569246in}}%
\pgfpathlineto{\pgfqpoint{5.248576in}{2.569749in}}%
\pgfpathlineto{\pgfqpoint{5.234591in}{2.570320in}}%
\pgfpathlineto{\pgfqpoint{5.220615in}{2.570959in}}%
\pgfpathlineto{\pgfqpoint{5.206648in}{2.571666in}}%
\pgfpathlineto{\pgfqpoint{5.199176in}{2.563399in}}%
\pgfpathlineto{\pgfqpoint{5.191702in}{2.555271in}}%
\pgfpathlineto{\pgfqpoint{5.184225in}{2.547277in}}%
\pgfpathlineto{\pgfqpoint{5.176746in}{2.539410in}}%
\pgfpathclose%
\pgfusepath{fill}%
\end{pgfscope}%
\begin{pgfscope}%
\pgfpathrectangle{\pgfqpoint{1.150000in}{0.150000in}}{\pgfqpoint{5.700000in}{5.700000in}}%
\pgfusepath{clip}%
\pgfsetbuttcap%
\pgfsetroundjoin%
\definecolor{currentfill}{rgb}{0.280894,0.078907,0.402329}%
\pgfsetfillcolor{currentfill}%
\pgfsetfillopacity{0.700000}%
\pgfsetlinewidth{0.000000pt}%
\definecolor{currentstroke}{rgb}{0.000000,0.000000,0.000000}%
\pgfsetstrokecolor{currentstroke}%
\pgfsetdash{}{0pt}%
\pgfpathmoveto{\pgfqpoint{3.837665in}{2.194378in}}%
\pgfpathlineto{\pgfqpoint{3.851271in}{2.191635in}}%
\pgfpathlineto{\pgfqpoint{3.864883in}{2.188974in}}%
\pgfpathlineto{\pgfqpoint{3.878501in}{2.186394in}}%
\pgfpathlineto{\pgfqpoint{3.892126in}{2.183894in}}%
\pgfpathlineto{\pgfqpoint{3.900084in}{2.192727in}}%
\pgfpathlineto{\pgfqpoint{3.908036in}{2.201565in}}%
\pgfpathlineto{\pgfqpoint{3.915983in}{2.210408in}}%
\pgfpathlineto{\pgfqpoint{3.923924in}{2.219260in}}%
\pgfpathlineto{\pgfqpoint{3.910310in}{2.221839in}}%
\pgfpathlineto{\pgfqpoint{3.896702in}{2.224500in}}%
\pgfpathlineto{\pgfqpoint{3.883101in}{2.227241in}}%
\pgfpathlineto{\pgfqpoint{3.869506in}{2.230063in}}%
\pgfpathlineto{\pgfqpoint{3.861554in}{2.221124in}}%
\pgfpathlineto{\pgfqpoint{3.853597in}{2.212198in}}%
\pgfpathlineto{\pgfqpoint{3.845634in}{2.203283in}}%
\pgfpathlineto{\pgfqpoint{3.837665in}{2.194378in}}%
\pgfpathclose%
\pgfusepath{fill}%
\end{pgfscope}%
\begin{pgfscope}%
\pgfpathrectangle{\pgfqpoint{1.150000in}{0.150000in}}{\pgfqpoint{5.700000in}{5.700000in}}%
\pgfusepath{clip}%
\pgfsetbuttcap%
\pgfsetroundjoin%
\definecolor{currentfill}{rgb}{0.280255,0.165693,0.476498}%
\pgfsetfillcolor{currentfill}%
\pgfsetfillopacity{0.700000}%
\pgfsetlinewidth{0.000000pt}%
\definecolor{currentstroke}{rgb}{0.000000,0.000000,0.000000}%
\pgfsetstrokecolor{currentstroke}%
\pgfsetdash{}{0pt}%
\pgfpathmoveto{\pgfqpoint{2.428557in}{2.384990in}}%
\pgfpathlineto{\pgfqpoint{2.442084in}{2.373738in}}%
\pgfpathlineto{\pgfqpoint{2.455608in}{2.362620in}}%
\pgfpathlineto{\pgfqpoint{2.469131in}{2.351635in}}%
\pgfpathlineto{\pgfqpoint{2.482653in}{2.340782in}}%
\pgfpathlineto{\pgfqpoint{2.491174in}{2.347282in}}%
\pgfpathlineto{\pgfqpoint{2.499686in}{2.353886in}}%
\pgfpathlineto{\pgfqpoint{2.508188in}{2.360593in}}%
\pgfpathlineto{\pgfqpoint{2.516680in}{2.367401in}}%
\pgfpathlineto{\pgfqpoint{2.503179in}{2.378126in}}%
\pgfpathlineto{\pgfqpoint{2.489678in}{2.388982in}}%
\pgfpathlineto{\pgfqpoint{2.476175in}{2.399971in}}%
\pgfpathlineto{\pgfqpoint{2.462670in}{2.411095in}}%
\pgfpathlineto{\pgfqpoint{2.454157in}{2.404407in}}%
\pgfpathlineto{\pgfqpoint{2.445634in}{2.397827in}}%
\pgfpathlineto{\pgfqpoint{2.437101in}{2.391354in}}%
\pgfpathlineto{\pgfqpoint{2.428557in}{2.384990in}}%
\pgfpathclose%
\pgfusepath{fill}%
\end{pgfscope}%
\begin{pgfscope}%
\pgfpathrectangle{\pgfqpoint{1.150000in}{0.150000in}}{\pgfqpoint{5.700000in}{5.700000in}}%
\pgfusepath{clip}%
\pgfsetbuttcap%
\pgfsetroundjoin%
\definecolor{currentfill}{rgb}{0.277018,0.050344,0.375715}%
\pgfsetfillcolor{currentfill}%
\pgfsetfillopacity{0.700000}%
\pgfsetlinewidth{0.000000pt}%
\definecolor{currentstroke}{rgb}{0.000000,0.000000,0.000000}%
\pgfsetstrokecolor{currentstroke}%
\pgfsetdash{}{0pt}%
\pgfpathmoveto{\pgfqpoint{3.383524in}{2.140390in}}%
\pgfpathlineto{\pgfqpoint{3.397043in}{2.135705in}}%
\pgfpathlineto{\pgfqpoint{3.410566in}{2.131110in}}%
\pgfpathlineto{\pgfqpoint{3.424095in}{2.126605in}}%
\pgfpathlineto{\pgfqpoint{3.437628in}{2.122190in}}%
\pgfpathlineto{\pgfqpoint{3.445745in}{2.131070in}}%
\pgfpathlineto{\pgfqpoint{3.453856in}{2.139968in}}%
\pgfpathlineto{\pgfqpoint{3.461962in}{2.148884in}}%
\pgfpathlineto{\pgfqpoint{3.470061in}{2.157820in}}%
\pgfpathlineto{\pgfqpoint{3.456540in}{2.162233in}}%
\pgfpathlineto{\pgfqpoint{3.443024in}{2.166736in}}%
\pgfpathlineto{\pgfqpoint{3.429512in}{2.171329in}}%
\pgfpathlineto{\pgfqpoint{3.416005in}{2.176012in}}%
\pgfpathlineto{\pgfqpoint{3.407894in}{2.167071in}}%
\pgfpathlineto{\pgfqpoint{3.399777in}{2.158154in}}%
\pgfpathlineto{\pgfqpoint{3.391653in}{2.149261in}}%
\pgfpathlineto{\pgfqpoint{3.383524in}{2.140390in}}%
\pgfpathclose%
\pgfusepath{fill}%
\end{pgfscope}%
\begin{pgfscope}%
\pgfpathrectangle{\pgfqpoint{1.150000in}{0.150000in}}{\pgfqpoint{5.700000in}{5.700000in}}%
\pgfusepath{clip}%
\pgfsetbuttcap%
\pgfsetroundjoin%
\definecolor{currentfill}{rgb}{0.274128,0.199721,0.498911}%
\pgfsetfillcolor{currentfill}%
\pgfsetfillopacity{0.700000}%
\pgfsetlinewidth{0.000000pt}%
\definecolor{currentstroke}{rgb}{0.000000,0.000000,0.000000}%
\pgfsetstrokecolor{currentstroke}%
\pgfsetdash{}{0pt}%
\pgfpathmoveto{\pgfqpoint{4.777422in}{2.424254in}}%
\pgfpathlineto{\pgfqpoint{4.791290in}{2.423682in}}%
\pgfpathlineto{\pgfqpoint{4.805168in}{2.423180in}}%
\pgfpathlineto{\pgfqpoint{4.819055in}{2.422749in}}%
\pgfpathlineto{\pgfqpoint{4.832950in}{2.422390in}}%
\pgfpathlineto{\pgfqpoint{4.840562in}{2.430021in}}%
\pgfpathlineto{\pgfqpoint{4.848169in}{2.437713in}}%
\pgfpathlineto{\pgfqpoint{4.855771in}{2.445470in}}%
\pgfpathlineto{\pgfqpoint{4.863369in}{2.453297in}}%
\pgfpathlineto{\pgfqpoint{4.849489in}{2.453922in}}%
\pgfpathlineto{\pgfqpoint{4.835618in}{2.454617in}}%
\pgfpathlineto{\pgfqpoint{4.821756in}{2.455383in}}%
\pgfpathlineto{\pgfqpoint{4.807903in}{2.456220in}}%
\pgfpathlineto{\pgfqpoint{4.800289in}{2.448121in}}%
\pgfpathlineto{\pgfqpoint{4.792671in}{2.440097in}}%
\pgfpathlineto{\pgfqpoint{4.785049in}{2.432143in}}%
\pgfpathlineto{\pgfqpoint{4.777422in}{2.424254in}}%
\pgfpathclose%
\pgfusepath{fill}%
\end{pgfscope}%
\begin{pgfscope}%
\pgfpathrectangle{\pgfqpoint{1.150000in}{0.150000in}}{\pgfqpoint{5.700000in}{5.700000in}}%
\pgfusepath{clip}%
\pgfsetbuttcap%
\pgfsetroundjoin%
\definecolor{currentfill}{rgb}{0.282910,0.105393,0.426902}%
\pgfsetfillcolor{currentfill}%
\pgfsetfillopacity{0.700000}%
\pgfsetlinewidth{0.000000pt}%
\definecolor{currentstroke}{rgb}{0.000000,0.000000,0.000000}%
\pgfsetstrokecolor{currentstroke}%
\pgfsetdash{}{0pt}%
\pgfpathmoveto{\pgfqpoint{2.678616in}{2.248577in}}%
\pgfpathlineto{\pgfqpoint{2.692109in}{2.239467in}}%
\pgfpathlineto{\pgfqpoint{2.705603in}{2.230473in}}%
\pgfpathlineto{\pgfqpoint{2.719096in}{2.221597in}}%
\pgfpathlineto{\pgfqpoint{2.732591in}{2.212835in}}%
\pgfpathlineto{\pgfqpoint{2.740995in}{2.220200in}}%
\pgfpathlineto{\pgfqpoint{2.749391in}{2.227643in}}%
\pgfpathlineto{\pgfqpoint{2.757779in}{2.235162in}}%
\pgfpathlineto{\pgfqpoint{2.766158in}{2.242756in}}%
\pgfpathlineto{\pgfqpoint{2.752682in}{2.251412in}}%
\pgfpathlineto{\pgfqpoint{2.739206in}{2.260183in}}%
\pgfpathlineto{\pgfqpoint{2.725731in}{2.269070in}}%
\pgfpathlineto{\pgfqpoint{2.712257in}{2.278075in}}%
\pgfpathlineto{\pgfqpoint{2.703859in}{2.270578in}}%
\pgfpathlineto{\pgfqpoint{2.695454in}{2.263163in}}%
\pgfpathlineto{\pgfqpoint{2.687039in}{2.255828in}}%
\pgfpathlineto{\pgfqpoint{2.678616in}{2.248577in}}%
\pgfpathclose%
\pgfusepath{fill}%
\end{pgfscope}%
\begin{pgfscope}%
\pgfpathrectangle{\pgfqpoint{1.150000in}{0.150000in}}{\pgfqpoint{5.700000in}{5.700000in}}%
\pgfusepath{clip}%
\pgfsetbuttcap%
\pgfsetroundjoin%
\definecolor{currentfill}{rgb}{0.210503,0.363727,0.552206}%
\pgfsetfillcolor{currentfill}%
\pgfsetfillopacity{0.700000}%
\pgfsetlinewidth{0.000000pt}%
\definecolor{currentstroke}{rgb}{0.000000,0.000000,0.000000}%
\pgfsetstrokecolor{currentstroke}%
\pgfsetdash{}{0pt}%
\pgfpathmoveto{\pgfqpoint{5.975918in}{2.792538in}}%
\pgfpathlineto{\pgfqpoint{5.990112in}{2.791495in}}%
\pgfpathlineto{\pgfqpoint{6.004317in}{2.790516in}}%
\pgfpathlineto{\pgfqpoint{6.018532in}{2.789602in}}%
\pgfpathlineto{\pgfqpoint{6.032757in}{2.788753in}}%
\pgfpathlineto{\pgfqpoint{6.040011in}{2.798702in}}%
\pgfpathlineto{\pgfqpoint{6.047272in}{2.808966in}}%
\pgfpathlineto{\pgfqpoint{6.054540in}{2.819553in}}%
\pgfpathlineto{\pgfqpoint{6.061817in}{2.830472in}}%
\pgfpathlineto{\pgfqpoint{6.047621in}{2.831830in}}%
\pgfpathlineto{\pgfqpoint{6.033435in}{2.833251in}}%
\pgfpathlineto{\pgfqpoint{6.019259in}{2.834737in}}%
\pgfpathlineto{\pgfqpoint{6.005093in}{2.836287in}}%
\pgfpathlineto{\pgfqpoint{5.997788in}{2.824853in}}%
\pgfpathlineto{\pgfqpoint{5.990490in}{2.813756in}}%
\pgfpathlineto{\pgfqpoint{5.983201in}{2.802987in}}%
\pgfpathlineto{\pgfqpoint{5.975918in}{2.792538in}}%
\pgfpathclose%
\pgfusepath{fill}%
\end{pgfscope}%
\begin{pgfscope}%
\pgfpathrectangle{\pgfqpoint{1.150000in}{0.150000in}}{\pgfqpoint{5.700000in}{5.700000in}}%
\pgfusepath{clip}%
\pgfsetbuttcap%
\pgfsetroundjoin%
\definecolor{currentfill}{rgb}{0.282910,0.105393,0.426902}%
\pgfsetfillcolor{currentfill}%
\pgfsetfillopacity{0.700000}%
\pgfsetlinewidth{0.000000pt}%
\definecolor{currentstroke}{rgb}{0.000000,0.000000,0.000000}%
\pgfsetstrokecolor{currentstroke}%
\pgfsetdash{}{0pt}%
\pgfpathmoveto{\pgfqpoint{4.064702in}{2.236217in}}%
\pgfpathlineto{\pgfqpoint{4.078368in}{2.234251in}}%
\pgfpathlineto{\pgfqpoint{4.092040in}{2.232362in}}%
\pgfpathlineto{\pgfqpoint{4.105720in}{2.230551in}}%
\pgfpathlineto{\pgfqpoint{4.119407in}{2.228818in}}%
\pgfpathlineto{\pgfqpoint{4.127287in}{2.237400in}}%
\pgfpathlineto{\pgfqpoint{4.135161in}{2.245989in}}%
\pgfpathlineto{\pgfqpoint{4.143030in}{2.254585in}}%
\pgfpathlineto{\pgfqpoint{4.150893in}{2.263193in}}%
\pgfpathlineto{\pgfqpoint{4.137217in}{2.265047in}}%
\pgfpathlineto{\pgfqpoint{4.123548in}{2.266979in}}%
\pgfpathlineto{\pgfqpoint{4.109887in}{2.268988in}}%
\pgfpathlineto{\pgfqpoint{4.096233in}{2.271076in}}%
\pgfpathlineto{\pgfqpoint{4.088359in}{2.262340in}}%
\pgfpathlineto{\pgfqpoint{4.080479in}{2.253620in}}%
\pgfpathlineto{\pgfqpoint{4.072593in}{2.244913in}}%
\pgfpathlineto{\pgfqpoint{4.064702in}{2.236217in}}%
\pgfpathclose%
\pgfusepath{fill}%
\end{pgfscope}%
\begin{pgfscope}%
\pgfpathrectangle{\pgfqpoint{1.150000in}{0.150000in}}{\pgfqpoint{5.700000in}{5.700000in}}%
\pgfusepath{clip}%
\pgfsetbuttcap%
\pgfsetroundjoin%
\definecolor{currentfill}{rgb}{0.243113,0.292092,0.538516}%
\pgfsetfillcolor{currentfill}%
\pgfsetfillopacity{0.700000}%
\pgfsetlinewidth{0.000000pt}%
\definecolor{currentstroke}{rgb}{0.000000,0.000000,0.000000}%
\pgfsetstrokecolor{currentstroke}%
\pgfsetdash{}{0pt}%
\pgfpathmoveto{\pgfqpoint{5.490325in}{2.628384in}}%
\pgfpathlineto{\pgfqpoint{5.504405in}{2.628003in}}%
\pgfpathlineto{\pgfqpoint{5.518495in}{2.627689in}}%
\pgfpathlineto{\pgfqpoint{5.532595in}{2.627442in}}%
\pgfpathlineto{\pgfqpoint{5.546705in}{2.627261in}}%
\pgfpathlineto{\pgfqpoint{5.554058in}{2.635086in}}%
\pgfpathlineto{\pgfqpoint{5.561411in}{2.643093in}}%
\pgfpathlineto{\pgfqpoint{5.568764in}{2.651290in}}%
\pgfpathlineto{\pgfqpoint{5.576117in}{2.659684in}}%
\pgfpathlineto{\pgfqpoint{5.562031in}{2.660272in}}%
\pgfpathlineto{\pgfqpoint{5.547955in}{2.660927in}}%
\pgfpathlineto{\pgfqpoint{5.533888in}{2.661648in}}%
\pgfpathlineto{\pgfqpoint{5.519831in}{2.662436in}}%
\pgfpathlineto{\pgfqpoint{5.512454in}{2.653628in}}%
\pgfpathlineto{\pgfqpoint{5.505078in}{2.645021in}}%
\pgfpathlineto{\pgfqpoint{5.497702in}{2.636609in}}%
\pgfpathlineto{\pgfqpoint{5.490325in}{2.628384in}}%
\pgfpathclose%
\pgfusepath{fill}%
\end{pgfscope}%
\begin{pgfscope}%
\pgfpathrectangle{\pgfqpoint{1.150000in}{0.150000in}}{\pgfqpoint{5.700000in}{5.700000in}}%
\pgfusepath{clip}%
\pgfsetbuttcap%
\pgfsetroundjoin%
\definecolor{currentfill}{rgb}{0.282290,0.145912,0.461510}%
\pgfsetfillcolor{currentfill}%
\pgfsetfillopacity{0.700000}%
\pgfsetlinewidth{0.000000pt}%
\definecolor{currentstroke}{rgb}{0.000000,0.000000,0.000000}%
\pgfsetstrokecolor{currentstroke}%
\pgfsetdash{}{0pt}%
\pgfpathmoveto{\pgfqpoint{4.378021in}{2.312823in}}%
\pgfpathlineto{\pgfqpoint{4.391773in}{2.311651in}}%
\pgfpathlineto{\pgfqpoint{4.405534in}{2.310553in}}%
\pgfpathlineto{\pgfqpoint{4.419303in}{2.309530in}}%
\pgfpathlineto{\pgfqpoint{4.433080in}{2.308581in}}%
\pgfpathlineto{\pgfqpoint{4.440845in}{2.316712in}}%
\pgfpathlineto{\pgfqpoint{4.448605in}{2.324863in}}%
\pgfpathlineto{\pgfqpoint{4.456360in}{2.333038in}}%
\pgfpathlineto{\pgfqpoint{4.464109in}{2.341240in}}%
\pgfpathlineto{\pgfqpoint{4.450345in}{2.342372in}}%
\pgfpathlineto{\pgfqpoint{4.436588in}{2.343578in}}%
\pgfpathlineto{\pgfqpoint{4.422840in}{2.344859in}}%
\pgfpathlineto{\pgfqpoint{4.409100in}{2.346213in}}%
\pgfpathlineto{\pgfqpoint{4.401338in}{2.337821in}}%
\pgfpathlineto{\pgfqpoint{4.393571in}{2.329461in}}%
\pgfpathlineto{\pgfqpoint{4.385799in}{2.321129in}}%
\pgfpathlineto{\pgfqpoint{4.378021in}{2.312823in}}%
\pgfpathclose%
\pgfusepath{fill}%
\end{pgfscope}%
\begin{pgfscope}%
\pgfpathrectangle{\pgfqpoint{1.150000in}{0.150000in}}{\pgfqpoint{5.700000in}{5.700000in}}%
\pgfusepath{clip}%
\pgfsetbuttcap%
\pgfsetroundjoin%
\definecolor{currentfill}{rgb}{0.263663,0.237631,0.518762}%
\pgfsetfillcolor{currentfill}%
\pgfsetfillopacity{0.700000}%
\pgfsetlinewidth{0.000000pt}%
\definecolor{currentstroke}{rgb}{0.000000,0.000000,0.000000}%
\pgfsetstrokecolor{currentstroke}%
\pgfsetdash{}{0pt}%
\pgfpathmoveto{\pgfqpoint{5.090875in}{2.509920in}}%
\pgfpathlineto{\pgfqpoint{5.104841in}{2.509611in}}%
\pgfpathlineto{\pgfqpoint{5.118817in}{2.509370in}}%
\pgfpathlineto{\pgfqpoint{5.132803in}{2.509198in}}%
\pgfpathlineto{\pgfqpoint{5.146798in}{2.509095in}}%
\pgfpathlineto{\pgfqpoint{5.154290in}{2.516513in}}%
\pgfpathlineto{\pgfqpoint{5.161778in}{2.524034in}}%
\pgfpathlineto{\pgfqpoint{5.169264in}{2.531664in}}%
\pgfpathlineto{\pgfqpoint{5.176746in}{2.539410in}}%
\pgfpathlineto{\pgfqpoint{5.162770in}{2.539840in}}%
\pgfpathlineto{\pgfqpoint{5.148803in}{2.540338in}}%
\pgfpathlineto{\pgfqpoint{5.134846in}{2.540904in}}%
\pgfpathlineto{\pgfqpoint{5.120898in}{2.541540in}}%
\pgfpathlineto{\pgfqpoint{5.113397in}{2.533461in}}%
\pgfpathlineto{\pgfqpoint{5.105893in}{2.525502in}}%
\pgfpathlineto{\pgfqpoint{5.098386in}{2.517657in}}%
\pgfpathlineto{\pgfqpoint{5.090875in}{2.509920in}}%
\pgfpathclose%
\pgfusepath{fill}%
\end{pgfscope}%
\begin{pgfscope}%
\pgfpathrectangle{\pgfqpoint{1.150000in}{0.150000in}}{\pgfqpoint{5.700000in}{5.700000in}}%
\pgfusepath{clip}%
\pgfsetbuttcap%
\pgfsetroundjoin%
\definecolor{currentfill}{rgb}{0.277941,0.056324,0.381191}%
\pgfsetfillcolor{currentfill}%
\pgfsetfillopacity{0.700000}%
\pgfsetlinewidth{0.000000pt}%
\definecolor{currentstroke}{rgb}{0.000000,0.000000,0.000000}%
\pgfsetstrokecolor{currentstroke}%
\pgfsetdash{}{0pt}%
\pgfpathmoveto{\pgfqpoint{3.015128in}{2.151746in}}%
\pgfpathlineto{\pgfqpoint{3.028616in}{2.145012in}}%
\pgfpathlineto{\pgfqpoint{3.042106in}{2.138381in}}%
\pgfpathlineto{\pgfqpoint{3.055600in}{2.131850in}}%
\pgfpathlineto{\pgfqpoint{3.069096in}{2.125419in}}%
\pgfpathlineto{\pgfqpoint{3.077356in}{2.133734in}}%
\pgfpathlineto{\pgfqpoint{3.085610in}{2.142094in}}%
\pgfpathlineto{\pgfqpoint{3.093857in}{2.150498in}}%
\pgfpathlineto{\pgfqpoint{3.102096in}{2.158945in}}%
\pgfpathlineto{\pgfqpoint{3.088615in}{2.165312in}}%
\pgfpathlineto{\pgfqpoint{3.075136in}{2.171779in}}%
\pgfpathlineto{\pgfqpoint{3.061660in}{2.178347in}}%
\pgfpathlineto{\pgfqpoint{3.048187in}{2.185017in}}%
\pgfpathlineto{\pgfqpoint{3.039932in}{2.176626in}}%
\pgfpathlineto{\pgfqpoint{3.031671in}{2.168283in}}%
\pgfpathlineto{\pgfqpoint{3.023403in}{2.159990in}}%
\pgfpathlineto{\pgfqpoint{3.015128in}{2.151746in}}%
\pgfpathclose%
\pgfusepath{fill}%
\end{pgfscope}%
\begin{pgfscope}%
\pgfpathrectangle{\pgfqpoint{1.150000in}{0.150000in}}{\pgfqpoint{5.700000in}{5.700000in}}%
\pgfusepath{clip}%
\pgfsetbuttcap%
\pgfsetroundjoin%
\definecolor{currentfill}{rgb}{0.277941,0.056324,0.381191}%
\pgfsetfillcolor{currentfill}%
\pgfsetfillopacity{0.700000}%
\pgfsetlinewidth{0.000000pt}%
\definecolor{currentstroke}{rgb}{0.000000,0.000000,0.000000}%
\pgfsetstrokecolor{currentstroke}%
\pgfsetdash{}{0pt}%
\pgfpathmoveto{\pgfqpoint{3.524194in}{2.141052in}}%
\pgfpathlineto{\pgfqpoint{3.537740in}{2.137080in}}%
\pgfpathlineto{\pgfqpoint{3.551292in}{2.133195in}}%
\pgfpathlineto{\pgfqpoint{3.564848in}{2.129396in}}%
\pgfpathlineto{\pgfqpoint{3.578410in}{2.125684in}}%
\pgfpathlineto{\pgfqpoint{3.586480in}{2.134616in}}%
\pgfpathlineto{\pgfqpoint{3.594545in}{2.143559in}}%
\pgfpathlineto{\pgfqpoint{3.602604in}{2.152513in}}%
\pgfpathlineto{\pgfqpoint{3.610657in}{2.161478in}}%
\pgfpathlineto{\pgfqpoint{3.597106in}{2.165209in}}%
\pgfpathlineto{\pgfqpoint{3.583561in}{2.169026in}}%
\pgfpathlineto{\pgfqpoint{3.570021in}{2.172930in}}%
\pgfpathlineto{\pgfqpoint{3.556487in}{2.176921in}}%
\pgfpathlineto{\pgfqpoint{3.548423in}{2.167930in}}%
\pgfpathlineto{\pgfqpoint{3.540352in}{2.158955in}}%
\pgfpathlineto{\pgfqpoint{3.532276in}{2.149996in}}%
\pgfpathlineto{\pgfqpoint{3.524194in}{2.141052in}}%
\pgfpathclose%
\pgfusepath{fill}%
\end{pgfscope}%
\begin{pgfscope}%
\pgfpathrectangle{\pgfqpoint{1.150000in}{0.150000in}}{\pgfqpoint{5.700000in}{5.700000in}}%
\pgfusepath{clip}%
\pgfsetbuttcap%
\pgfsetroundjoin%
\definecolor{currentfill}{rgb}{0.281887,0.150881,0.465405}%
\pgfsetfillcolor{currentfill}%
\pgfsetfillopacity{0.700000}%
\pgfsetlinewidth{0.000000pt}%
\definecolor{currentstroke}{rgb}{0.000000,0.000000,0.000000}%
\pgfsetstrokecolor{currentstroke}%
\pgfsetdash{}{0pt}%
\pgfpathmoveto{\pgfqpoint{2.482653in}{2.340782in}}%
\pgfpathlineto{\pgfqpoint{2.496173in}{2.330060in}}%
\pgfpathlineto{\pgfqpoint{2.509691in}{2.319467in}}%
\pgfpathlineto{\pgfqpoint{2.523209in}{2.309003in}}%
\pgfpathlineto{\pgfqpoint{2.536725in}{2.298666in}}%
\pgfpathlineto{\pgfqpoint{2.545226in}{2.305301in}}%
\pgfpathlineto{\pgfqpoint{2.553716in}{2.312036in}}%
\pgfpathlineto{\pgfqpoint{2.562197in}{2.318868in}}%
\pgfpathlineto{\pgfqpoint{2.570668in}{2.325797in}}%
\pgfpathlineto{\pgfqpoint{2.557173in}{2.336006in}}%
\pgfpathlineto{\pgfqpoint{2.543676in}{2.346342in}}%
\pgfpathlineto{\pgfqpoint{2.530178in}{2.356807in}}%
\pgfpathlineto{\pgfqpoint{2.516680in}{2.367401in}}%
\pgfpathlineto{\pgfqpoint{2.508188in}{2.360593in}}%
\pgfpathlineto{\pgfqpoint{2.499686in}{2.353886in}}%
\pgfpathlineto{\pgfqpoint{2.491174in}{2.347282in}}%
\pgfpathlineto{\pgfqpoint{2.482653in}{2.340782in}}%
\pgfpathclose%
\pgfusepath{fill}%
\end{pgfscope}%
\begin{pgfscope}%
\pgfpathrectangle{\pgfqpoint{1.150000in}{0.150000in}}{\pgfqpoint{5.700000in}{5.700000in}}%
\pgfusepath{clip}%
\pgfsetbuttcap%
\pgfsetroundjoin%
\definecolor{currentfill}{rgb}{0.277018,0.050344,0.375715}%
\pgfsetfillcolor{currentfill}%
\pgfsetfillopacity{0.700000}%
\pgfsetlinewidth{0.000000pt}%
\definecolor{currentstroke}{rgb}{0.000000,0.000000,0.000000}%
\pgfsetstrokecolor{currentstroke}%
\pgfsetdash{}{0pt}%
\pgfpathmoveto{\pgfqpoint{3.156053in}{2.134467in}}%
\pgfpathlineto{\pgfqpoint{3.169551in}{2.128592in}}%
\pgfpathlineto{\pgfqpoint{3.183052in}{2.122814in}}%
\pgfpathlineto{\pgfqpoint{3.196556in}{2.117132in}}%
\pgfpathlineto{\pgfqpoint{3.210064in}{2.111545in}}%
\pgfpathlineto{\pgfqpoint{3.218269in}{2.120137in}}%
\pgfpathlineto{\pgfqpoint{3.226468in}{2.128761in}}%
\pgfpathlineto{\pgfqpoint{3.234660in}{2.137418in}}%
\pgfpathlineto{\pgfqpoint{3.242846in}{2.146107in}}%
\pgfpathlineto{\pgfqpoint{3.229352in}{2.151651in}}%
\pgfpathlineto{\pgfqpoint{3.215861in}{2.157290in}}%
\pgfpathlineto{\pgfqpoint{3.202373in}{2.163025in}}%
\pgfpathlineto{\pgfqpoint{3.188890in}{2.168857in}}%
\pgfpathlineto{\pgfqpoint{3.180690in}{2.160203in}}%
\pgfpathlineto{\pgfqpoint{3.172485in}{2.151587in}}%
\pgfpathlineto{\pgfqpoint{3.164272in}{2.143008in}}%
\pgfpathlineto{\pgfqpoint{3.156053in}{2.134467in}}%
\pgfpathclose%
\pgfusepath{fill}%
\end{pgfscope}%
\begin{pgfscope}%
\pgfpathrectangle{\pgfqpoint{1.150000in}{0.150000in}}{\pgfqpoint{5.700000in}{5.700000in}}%
\pgfusepath{clip}%
\pgfsetbuttcap%
\pgfsetroundjoin%
\definecolor{currentfill}{rgb}{0.280267,0.073417,0.397163}%
\pgfsetfillcolor{currentfill}%
\pgfsetfillopacity{0.700000}%
\pgfsetlinewidth{0.000000pt}%
\definecolor{currentstroke}{rgb}{0.000000,0.000000,0.000000}%
\pgfsetstrokecolor{currentstroke}%
\pgfsetdash{}{0pt}%
\pgfpathmoveto{\pgfqpoint{3.751330in}{2.170370in}}%
\pgfpathlineto{\pgfqpoint{3.764922in}{2.167358in}}%
\pgfpathlineto{\pgfqpoint{3.778520in}{2.164429in}}%
\pgfpathlineto{\pgfqpoint{3.792123in}{2.161583in}}%
\pgfpathlineto{\pgfqpoint{3.805733in}{2.158818in}}%
\pgfpathlineto{\pgfqpoint{3.813725in}{2.167702in}}%
\pgfpathlineto{\pgfqpoint{3.821711in}{2.176589in}}%
\pgfpathlineto{\pgfqpoint{3.829691in}{2.185480in}}%
\pgfpathlineto{\pgfqpoint{3.837665in}{2.194378in}}%
\pgfpathlineto{\pgfqpoint{3.824066in}{2.197202in}}%
\pgfpathlineto{\pgfqpoint{3.810473in}{2.200108in}}%
\pgfpathlineto{\pgfqpoint{3.796886in}{2.203096in}}%
\pgfpathlineto{\pgfqpoint{3.783306in}{2.206167in}}%
\pgfpathlineto{\pgfqpoint{3.775320in}{2.197202in}}%
\pgfpathlineto{\pgfqpoint{3.767329in}{2.188249in}}%
\pgfpathlineto{\pgfqpoint{3.759333in}{2.179305in}}%
\pgfpathlineto{\pgfqpoint{3.751330in}{2.170370in}}%
\pgfpathclose%
\pgfusepath{fill}%
\end{pgfscope}%
\begin{pgfscope}%
\pgfpathrectangle{\pgfqpoint{1.150000in}{0.150000in}}{\pgfqpoint{5.700000in}{5.700000in}}%
\pgfusepath{clip}%
\pgfsetbuttcap%
\pgfsetroundjoin%
\definecolor{currentfill}{rgb}{0.218130,0.347432,0.550038}%
\pgfsetfillcolor{currentfill}%
\pgfsetfillopacity{0.700000}%
\pgfsetlinewidth{0.000000pt}%
\definecolor{currentstroke}{rgb}{0.000000,0.000000,0.000000}%
\pgfsetstrokecolor{currentstroke}%
\pgfsetdash{}{0pt}%
\pgfpathmoveto{\pgfqpoint{5.890062in}{2.756626in}}%
\pgfpathlineto{\pgfqpoint{5.904243in}{2.755811in}}%
\pgfpathlineto{\pgfqpoint{5.918435in}{2.755061in}}%
\pgfpathlineto{\pgfqpoint{5.932637in}{2.754377in}}%
\pgfpathlineto{\pgfqpoint{5.946850in}{2.753757in}}%
\pgfpathlineto{\pgfqpoint{5.954108in}{2.763017in}}%
\pgfpathlineto{\pgfqpoint{5.961372in}{2.772562in}}%
\pgfpathlineto{\pgfqpoint{5.968642in}{2.782399in}}%
\pgfpathlineto{\pgfqpoint{5.975918in}{2.792538in}}%
\pgfpathlineto{\pgfqpoint{5.961734in}{2.793646in}}%
\pgfpathlineto{\pgfqpoint{5.947560in}{2.794818in}}%
\pgfpathlineto{\pgfqpoint{5.933396in}{2.796055in}}%
\pgfpathlineto{\pgfqpoint{5.919242in}{2.797357in}}%
\pgfpathlineto{\pgfqpoint{5.911938in}{2.786724in}}%
\pgfpathlineto{\pgfqpoint{5.904640in}{2.776396in}}%
\pgfpathlineto{\pgfqpoint{5.897348in}{2.766366in}}%
\pgfpathlineto{\pgfqpoint{5.890062in}{2.756626in}}%
\pgfpathclose%
\pgfusepath{fill}%
\end{pgfscope}%
\begin{pgfscope}%
\pgfpathrectangle{\pgfqpoint{1.150000in}{0.150000in}}{\pgfqpoint{5.700000in}{5.700000in}}%
\pgfusepath{clip}%
\pgfsetbuttcap%
\pgfsetroundjoin%
\definecolor{currentfill}{rgb}{0.277134,0.185228,0.489898}%
\pgfsetfillcolor{currentfill}%
\pgfsetfillopacity{0.700000}%
\pgfsetlinewidth{0.000000pt}%
\definecolor{currentstroke}{rgb}{0.000000,0.000000,0.000000}%
\pgfsetstrokecolor{currentstroke}%
\pgfsetdash{}{0pt}%
\pgfpathmoveto{\pgfqpoint{4.691420in}{2.395284in}}%
\pgfpathlineto{\pgfqpoint{4.705268in}{2.394670in}}%
\pgfpathlineto{\pgfqpoint{4.719125in}{2.394128in}}%
\pgfpathlineto{\pgfqpoint{4.732991in}{2.393658in}}%
\pgfpathlineto{\pgfqpoint{4.746865in}{2.393259in}}%
\pgfpathlineto{\pgfqpoint{4.754512in}{2.400933in}}%
\pgfpathlineto{\pgfqpoint{4.762153in}{2.408654in}}%
\pgfpathlineto{\pgfqpoint{4.769790in}{2.416426in}}%
\pgfpathlineto{\pgfqpoint{4.777422in}{2.424254in}}%
\pgfpathlineto{\pgfqpoint{4.763562in}{2.424898in}}%
\pgfpathlineto{\pgfqpoint{4.749712in}{2.425613in}}%
\pgfpathlineto{\pgfqpoint{4.735870in}{2.426399in}}%
\pgfpathlineto{\pgfqpoint{4.722036in}{2.427257in}}%
\pgfpathlineto{\pgfqpoint{4.714390in}{2.419177in}}%
\pgfpathlineto{\pgfqpoint{4.706738in}{2.411158in}}%
\pgfpathlineto{\pgfqpoint{4.699082in}{2.403195in}}%
\pgfpathlineto{\pgfqpoint{4.691420in}{2.395284in}}%
\pgfpathclose%
\pgfusepath{fill}%
\end{pgfscope}%
\begin{pgfscope}%
\pgfpathrectangle{\pgfqpoint{1.150000in}{0.150000in}}{\pgfqpoint{5.700000in}{5.700000in}}%
\pgfusepath{clip}%
\pgfsetbuttcap%
\pgfsetroundjoin%
\definecolor{currentfill}{rgb}{0.280267,0.073417,0.397163}%
\pgfsetfillcolor{currentfill}%
\pgfsetfillopacity{0.700000}%
\pgfsetlinewidth{0.000000pt}%
\definecolor{currentstroke}{rgb}{0.000000,0.000000,0.000000}%
\pgfsetstrokecolor{currentstroke}%
\pgfsetdash{}{0pt}%
\pgfpathmoveto{\pgfqpoint{2.874004in}{2.177561in}}%
\pgfpathlineto{\pgfqpoint{2.887491in}{2.169906in}}%
\pgfpathlineto{\pgfqpoint{2.900980in}{2.162358in}}%
\pgfpathlineto{\pgfqpoint{2.914470in}{2.154917in}}%
\pgfpathlineto{\pgfqpoint{2.927963in}{2.147581in}}%
\pgfpathlineto{\pgfqpoint{2.936283in}{2.155527in}}%
\pgfpathlineto{\pgfqpoint{2.944596in}{2.163531in}}%
\pgfpathlineto{\pgfqpoint{2.952901in}{2.171591in}}%
\pgfpathlineto{\pgfqpoint{2.961199in}{2.179708in}}%
\pgfpathlineto{\pgfqpoint{2.947723in}{2.186960in}}%
\pgfpathlineto{\pgfqpoint{2.934249in}{2.194317in}}%
\pgfpathlineto{\pgfqpoint{2.920776in}{2.201780in}}%
\pgfpathlineto{\pgfqpoint{2.907305in}{2.209351in}}%
\pgfpathlineto{\pgfqpoint{2.898991in}{2.201311in}}%
\pgfpathlineto{\pgfqpoint{2.890670in}{2.193332in}}%
\pgfpathlineto{\pgfqpoint{2.882341in}{2.185415in}}%
\pgfpathlineto{\pgfqpoint{2.874004in}{2.177561in}}%
\pgfpathclose%
\pgfusepath{fill}%
\end{pgfscope}%
\begin{pgfscope}%
\pgfpathrectangle{\pgfqpoint{1.150000in}{0.150000in}}{\pgfqpoint{5.700000in}{5.700000in}}%
\pgfusepath{clip}%
\pgfsetbuttcap%
\pgfsetroundjoin%
\definecolor{currentfill}{rgb}{0.248629,0.278775,0.534556}%
\pgfsetfillcolor{currentfill}%
\pgfsetfillopacity{0.700000}%
\pgfsetlinewidth{0.000000pt}%
\definecolor{currentstroke}{rgb}{0.000000,0.000000,0.000000}%
\pgfsetstrokecolor{currentstroke}%
\pgfsetdash{}{0pt}%
\pgfpathmoveto{\pgfqpoint{5.404505in}{2.597860in}}%
\pgfpathlineto{\pgfqpoint{5.418567in}{2.597598in}}%
\pgfpathlineto{\pgfqpoint{5.432639in}{2.597404in}}%
\pgfpathlineto{\pgfqpoint{5.446722in}{2.597277in}}%
\pgfpathlineto{\pgfqpoint{5.460814in}{2.597217in}}%
\pgfpathlineto{\pgfqpoint{5.468193in}{2.604763in}}%
\pgfpathlineto{\pgfqpoint{5.475571in}{2.612468in}}%
\pgfpathlineto{\pgfqpoint{5.482949in}{2.620339in}}%
\pgfpathlineto{\pgfqpoint{5.490325in}{2.628384in}}%
\pgfpathlineto{\pgfqpoint{5.476256in}{2.628832in}}%
\pgfpathlineto{\pgfqpoint{5.462196in}{2.629346in}}%
\pgfpathlineto{\pgfqpoint{5.448146in}{2.629928in}}%
\pgfpathlineto{\pgfqpoint{5.434106in}{2.630576in}}%
\pgfpathlineto{\pgfqpoint{5.426707in}{2.622137in}}%
\pgfpathlineto{\pgfqpoint{5.419307in}{2.613876in}}%
\pgfpathlineto{\pgfqpoint{5.411907in}{2.605785in}}%
\pgfpathlineto{\pgfqpoint{5.404505in}{2.597860in}}%
\pgfpathclose%
\pgfusepath{fill}%
\end{pgfscope}%
\begin{pgfscope}%
\pgfpathrectangle{\pgfqpoint{1.150000in}{0.150000in}}{\pgfqpoint{5.700000in}{5.700000in}}%
\pgfusepath{clip}%
\pgfsetbuttcap%
\pgfsetroundjoin%
\definecolor{currentfill}{rgb}{0.276022,0.044167,0.370164}%
\pgfsetfillcolor{currentfill}%
\pgfsetfillopacity{0.700000}%
\pgfsetlinewidth{0.000000pt}%
\definecolor{currentstroke}{rgb}{0.000000,0.000000,0.000000}%
\pgfsetstrokecolor{currentstroke}%
\pgfsetdash{}{0pt}%
\pgfpathmoveto{\pgfqpoint{3.296863in}{2.124876in}}%
\pgfpathlineto{\pgfqpoint{3.310377in}{2.119802in}}%
\pgfpathlineto{\pgfqpoint{3.323895in}{2.114821in}}%
\pgfpathlineto{\pgfqpoint{3.337418in}{2.109931in}}%
\pgfpathlineto{\pgfqpoint{3.350945in}{2.105132in}}%
\pgfpathlineto{\pgfqpoint{3.359099in}{2.113914in}}%
\pgfpathlineto{\pgfqpoint{3.367247in}{2.122717in}}%
\pgfpathlineto{\pgfqpoint{3.375388in}{2.131542in}}%
\pgfpathlineto{\pgfqpoint{3.383524in}{2.140390in}}%
\pgfpathlineto{\pgfqpoint{3.370009in}{2.145166in}}%
\pgfpathlineto{\pgfqpoint{3.356499in}{2.150034in}}%
\pgfpathlineto{\pgfqpoint{3.342994in}{2.154993in}}%
\pgfpathlineto{\pgfqpoint{3.329492in}{2.160045in}}%
\pgfpathlineto{\pgfqpoint{3.321344in}{2.151212in}}%
\pgfpathlineto{\pgfqpoint{3.313190in}{2.142406in}}%
\pgfpathlineto{\pgfqpoint{3.305030in}{2.133628in}}%
\pgfpathlineto{\pgfqpoint{3.296863in}{2.124876in}}%
\pgfpathclose%
\pgfusepath{fill}%
\end{pgfscope}%
\begin{pgfscope}%
\pgfpathrectangle{\pgfqpoint{1.150000in}{0.150000in}}{\pgfqpoint{5.700000in}{5.700000in}}%
\pgfusepath{clip}%
\pgfsetbuttcap%
\pgfsetroundjoin%
\definecolor{currentfill}{rgb}{0.266580,0.228262,0.514349}%
\pgfsetfillcolor{currentfill}%
\pgfsetfillopacity{0.700000}%
\pgfsetlinewidth{0.000000pt}%
\definecolor{currentstroke}{rgb}{0.000000,0.000000,0.000000}%
\pgfsetstrokecolor{currentstroke}%
\pgfsetdash{}{0pt}%
\pgfpathmoveto{\pgfqpoint{5.004954in}{2.480652in}}%
\pgfpathlineto{\pgfqpoint{5.018900in}{2.480371in}}%
\pgfpathlineto{\pgfqpoint{5.032856in}{2.480161in}}%
\pgfpathlineto{\pgfqpoint{5.046821in}{2.480019in}}%
\pgfpathlineto{\pgfqpoint{5.060796in}{2.479947in}}%
\pgfpathlineto{\pgfqpoint{5.068321in}{2.487306in}}%
\pgfpathlineto{\pgfqpoint{5.075843in}{2.494750in}}%
\pgfpathlineto{\pgfqpoint{5.083361in}{2.502287in}}%
\pgfpathlineto{\pgfqpoint{5.090875in}{2.509920in}}%
\pgfpathlineto{\pgfqpoint{5.076918in}{2.510299in}}%
\pgfpathlineto{\pgfqpoint{5.062971in}{2.510746in}}%
\pgfpathlineto{\pgfqpoint{5.049033in}{2.511263in}}%
\pgfpathlineto{\pgfqpoint{5.035104in}{2.511849in}}%
\pgfpathlineto{\pgfqpoint{5.027572in}{2.503902in}}%
\pgfpathlineto{\pgfqpoint{5.020036in}{2.496057in}}%
\pgfpathlineto{\pgfqpoint{5.012497in}{2.488309in}}%
\pgfpathlineto{\pgfqpoint{5.004954in}{2.480652in}}%
\pgfpathclose%
\pgfusepath{fill}%
\end{pgfscope}%
\begin{pgfscope}%
\pgfpathrectangle{\pgfqpoint{1.150000in}{0.150000in}}{\pgfqpoint{5.700000in}{5.700000in}}%
\pgfusepath{clip}%
\pgfsetbuttcap%
\pgfsetroundjoin%
\definecolor{currentfill}{rgb}{0.282327,0.094955,0.417331}%
\pgfsetfillcolor{currentfill}%
\pgfsetfillopacity{0.700000}%
\pgfsetlinewidth{0.000000pt}%
\definecolor{currentstroke}{rgb}{0.000000,0.000000,0.000000}%
\pgfsetstrokecolor{currentstroke}%
\pgfsetdash{}{0pt}%
\pgfpathmoveto{\pgfqpoint{3.978448in}{2.209743in}}%
\pgfpathlineto{\pgfqpoint{3.992096in}{2.207562in}}%
\pgfpathlineto{\pgfqpoint{4.005751in}{2.205461in}}%
\pgfpathlineto{\pgfqpoint{4.019413in}{2.203438in}}%
\pgfpathlineto{\pgfqpoint{4.033082in}{2.201494in}}%
\pgfpathlineto{\pgfqpoint{4.040996in}{2.210170in}}%
\pgfpathlineto{\pgfqpoint{4.048903in}{2.218848in}}%
\pgfpathlineto{\pgfqpoint{4.056806in}{2.227529in}}%
\pgfpathlineto{\pgfqpoint{4.064702in}{2.236217in}}%
\pgfpathlineto{\pgfqpoint{4.051044in}{2.238262in}}%
\pgfpathlineto{\pgfqpoint{4.037393in}{2.240385in}}%
\pgfpathlineto{\pgfqpoint{4.023749in}{2.242587in}}%
\pgfpathlineto{\pgfqpoint{4.010112in}{2.244868in}}%
\pgfpathlineto{\pgfqpoint{4.002204in}{2.236072in}}%
\pgfpathlineto{\pgfqpoint{3.994291in}{2.227287in}}%
\pgfpathlineto{\pgfqpoint{3.986372in}{2.218512in}}%
\pgfpathlineto{\pgfqpoint{3.978448in}{2.209743in}}%
\pgfpathclose%
\pgfusepath{fill}%
\end{pgfscope}%
\begin{pgfscope}%
\pgfpathrectangle{\pgfqpoint{1.150000in}{0.150000in}}{\pgfqpoint{5.700000in}{5.700000in}}%
\pgfusepath{clip}%
\pgfsetbuttcap%
\pgfsetroundjoin%
\definecolor{currentfill}{rgb}{0.282884,0.135920,0.453427}%
\pgfsetfillcolor{currentfill}%
\pgfsetfillopacity{0.700000}%
\pgfsetlinewidth{0.000000pt}%
\definecolor{currentstroke}{rgb}{0.000000,0.000000,0.000000}%
\pgfsetstrokecolor{currentstroke}%
\pgfsetdash{}{0pt}%
\pgfpathmoveto{\pgfqpoint{4.291875in}{2.284572in}}%
\pgfpathlineto{\pgfqpoint{4.305608in}{2.283263in}}%
\pgfpathlineto{\pgfqpoint{4.319348in}{2.282029in}}%
\pgfpathlineto{\pgfqpoint{4.333097in}{2.280870in}}%
\pgfpathlineto{\pgfqpoint{4.346854in}{2.279785in}}%
\pgfpathlineto{\pgfqpoint{4.354654in}{2.288023in}}%
\pgfpathlineto{\pgfqpoint{4.362448in}{2.296273in}}%
\pgfpathlineto{\pgfqpoint{4.370237in}{2.304539in}}%
\pgfpathlineto{\pgfqpoint{4.378021in}{2.312823in}}%
\pgfpathlineto{\pgfqpoint{4.364276in}{2.314070in}}%
\pgfpathlineto{\pgfqpoint{4.350539in}{2.315391in}}%
\pgfpathlineto{\pgfqpoint{4.336811in}{2.316788in}}%
\pgfpathlineto{\pgfqpoint{4.323090in}{2.318259in}}%
\pgfpathlineto{\pgfqpoint{4.315294in}{2.309805in}}%
\pgfpathlineto{\pgfqpoint{4.307493in}{2.301375in}}%
\pgfpathlineto{\pgfqpoint{4.299687in}{2.292965in}}%
\pgfpathlineto{\pgfqpoint{4.291875in}{2.284572in}}%
\pgfpathclose%
\pgfusepath{fill}%
\end{pgfscope}%
\begin{pgfscope}%
\pgfpathrectangle{\pgfqpoint{1.150000in}{0.150000in}}{\pgfqpoint{5.700000in}{5.700000in}}%
\pgfusepath{clip}%
\pgfsetbuttcap%
\pgfsetroundjoin%
\definecolor{currentfill}{rgb}{0.223925,0.334994,0.548053}%
\pgfsetfillcolor{currentfill}%
\pgfsetfillopacity{0.700000}%
\pgfsetlinewidth{0.000000pt}%
\definecolor{currentstroke}{rgb}{0.000000,0.000000,0.000000}%
\pgfsetstrokecolor{currentstroke}%
\pgfsetdash{}{0pt}%
\pgfpathmoveto{\pgfqpoint{5.804228in}{2.722420in}}%
\pgfpathlineto{\pgfqpoint{5.818396in}{2.721813in}}%
\pgfpathlineto{\pgfqpoint{5.832574in}{2.721271in}}%
\pgfpathlineto{\pgfqpoint{5.846762in}{2.720794in}}%
\pgfpathlineto{\pgfqpoint{5.860961in}{2.720383in}}%
\pgfpathlineto{\pgfqpoint{5.868230in}{2.729052in}}%
\pgfpathlineto{\pgfqpoint{5.875503in}{2.737977in}}%
\pgfpathlineto{\pgfqpoint{5.882780in}{2.747165in}}%
\pgfpathlineto{\pgfqpoint{5.890062in}{2.756626in}}%
\pgfpathlineto{\pgfqpoint{5.875890in}{2.757505in}}%
\pgfpathlineto{\pgfqpoint{5.861729in}{2.758450in}}%
\pgfpathlineto{\pgfqpoint{5.847578in}{2.759460in}}%
\pgfpathlineto{\pgfqpoint{5.833437in}{2.760535in}}%
\pgfpathlineto{\pgfqpoint{5.826129in}{2.750599in}}%
\pgfpathlineto{\pgfqpoint{5.818825in}{2.740941in}}%
\pgfpathlineto{\pgfqpoint{5.811525in}{2.731551in}}%
\pgfpathlineto{\pgfqpoint{5.804228in}{2.722420in}}%
\pgfpathclose%
\pgfusepath{fill}%
\end{pgfscope}%
\begin{pgfscope}%
\pgfpathrectangle{\pgfqpoint{1.150000in}{0.150000in}}{\pgfqpoint{5.700000in}{5.700000in}}%
\pgfusepath{clip}%
\pgfsetbuttcap%
\pgfsetroundjoin%
\definecolor{currentfill}{rgb}{0.278826,0.175490,0.483397}%
\pgfsetfillcolor{currentfill}%
\pgfsetfillopacity{0.700000}%
\pgfsetlinewidth{0.000000pt}%
\definecolor{currentstroke}{rgb}{0.000000,0.000000,0.000000}%
\pgfsetstrokecolor{currentstroke}%
\pgfsetdash{}{0pt}%
\pgfpathmoveto{\pgfqpoint{4.605363in}{2.366352in}}%
\pgfpathlineto{\pgfqpoint{4.619190in}{2.365675in}}%
\pgfpathlineto{\pgfqpoint{4.633026in}{2.365069in}}%
\pgfpathlineto{\pgfqpoint{4.646870in}{2.364536in}}%
\pgfpathlineto{\pgfqpoint{4.660724in}{2.364075in}}%
\pgfpathlineto{\pgfqpoint{4.668406in}{2.371820in}}%
\pgfpathlineto{\pgfqpoint{4.676082in}{2.379601in}}%
\pgfpathlineto{\pgfqpoint{4.683754in}{2.387421in}}%
\pgfpathlineto{\pgfqpoint{4.691420in}{2.395284in}}%
\pgfpathlineto{\pgfqpoint{4.677581in}{2.395969in}}%
\pgfpathlineto{\pgfqpoint{4.663751in}{2.396727in}}%
\pgfpathlineto{\pgfqpoint{4.649929in}{2.397556in}}%
\pgfpathlineto{\pgfqpoint{4.636116in}{2.398457in}}%
\pgfpathlineto{\pgfqpoint{4.628436in}{2.390363in}}%
\pgfpathlineto{\pgfqpoint{4.620750in}{2.382316in}}%
\pgfpathlineto{\pgfqpoint{4.613059in}{2.374314in}}%
\pgfpathlineto{\pgfqpoint{4.605363in}{2.366352in}}%
\pgfpathclose%
\pgfusepath{fill}%
\end{pgfscope}%
\begin{pgfscope}%
\pgfpathrectangle{\pgfqpoint{1.150000in}{0.150000in}}{\pgfqpoint{5.700000in}{5.700000in}}%
\pgfusepath{clip}%
\pgfsetbuttcap%
\pgfsetroundjoin%
\definecolor{currentfill}{rgb}{0.281924,0.089666,0.412415}%
\pgfsetfillcolor{currentfill}%
\pgfsetfillopacity{0.700000}%
\pgfsetlinewidth{0.000000pt}%
\definecolor{currentstroke}{rgb}{0.000000,0.000000,0.000000}%
\pgfsetstrokecolor{currentstroke}%
\pgfsetdash{}{0pt}%
\pgfpathmoveto{\pgfqpoint{2.732591in}{2.212835in}}%
\pgfpathlineto{\pgfqpoint{2.746086in}{2.204188in}}%
\pgfpathlineto{\pgfqpoint{2.759581in}{2.195655in}}%
\pgfpathlineto{\pgfqpoint{2.773078in}{2.187235in}}%
\pgfpathlineto{\pgfqpoint{2.786576in}{2.178927in}}%
\pgfpathlineto{\pgfqpoint{2.794962in}{2.186405in}}%
\pgfpathlineto{\pgfqpoint{2.803340in}{2.193955in}}%
\pgfpathlineto{\pgfqpoint{2.811710in}{2.201577in}}%
\pgfpathlineto{\pgfqpoint{2.820071in}{2.209270in}}%
\pgfpathlineto{\pgfqpoint{2.806591in}{2.217472in}}%
\pgfpathlineto{\pgfqpoint{2.793113in}{2.225787in}}%
\pgfpathlineto{\pgfqpoint{2.779635in}{2.234215in}}%
\pgfpathlineto{\pgfqpoint{2.766158in}{2.242756in}}%
\pgfpathlineto{\pgfqpoint{2.757779in}{2.235162in}}%
\pgfpathlineto{\pgfqpoint{2.749391in}{2.227643in}}%
\pgfpathlineto{\pgfqpoint{2.740995in}{2.220200in}}%
\pgfpathlineto{\pgfqpoint{2.732591in}{2.212835in}}%
\pgfpathclose%
\pgfusepath{fill}%
\end{pgfscope}%
\begin{pgfscope}%
\pgfpathrectangle{\pgfqpoint{1.150000in}{0.150000in}}{\pgfqpoint{5.700000in}{5.700000in}}%
\pgfusepath{clip}%
\pgfsetbuttcap%
\pgfsetroundjoin%
\definecolor{currentfill}{rgb}{0.283072,0.130895,0.449241}%
\pgfsetfillcolor{currentfill}%
\pgfsetfillopacity{0.700000}%
\pgfsetlinewidth{0.000000pt}%
\definecolor{currentstroke}{rgb}{0.000000,0.000000,0.000000}%
\pgfsetstrokecolor{currentstroke}%
\pgfsetdash{}{0pt}%
\pgfpathmoveto{\pgfqpoint{2.536725in}{2.298666in}}%
\pgfpathlineto{\pgfqpoint{2.550241in}{2.288455in}}%
\pgfpathlineto{\pgfqpoint{2.563756in}{2.278370in}}%
\pgfpathlineto{\pgfqpoint{2.577270in}{2.268408in}}%
\pgfpathlineto{\pgfqpoint{2.590783in}{2.258570in}}%
\pgfpathlineto{\pgfqpoint{2.599263in}{2.265341in}}%
\pgfpathlineto{\pgfqpoint{2.607732in}{2.272206in}}%
\pgfpathlineto{\pgfqpoint{2.616193in}{2.279163in}}%
\pgfpathlineto{\pgfqpoint{2.624645in}{2.286211in}}%
\pgfpathlineto{\pgfqpoint{2.611151in}{2.295921in}}%
\pgfpathlineto{\pgfqpoint{2.597658in}{2.305755in}}%
\pgfpathlineto{\pgfqpoint{2.584163in}{2.315713in}}%
\pgfpathlineto{\pgfqpoint{2.570668in}{2.325797in}}%
\pgfpathlineto{\pgfqpoint{2.562197in}{2.318868in}}%
\pgfpathlineto{\pgfqpoint{2.553716in}{2.312036in}}%
\pgfpathlineto{\pgfqpoint{2.545226in}{2.305301in}}%
\pgfpathlineto{\pgfqpoint{2.536725in}{2.298666in}}%
\pgfpathclose%
\pgfusepath{fill}%
\end{pgfscope}%
\begin{pgfscope}%
\pgfpathrectangle{\pgfqpoint{1.150000in}{0.150000in}}{\pgfqpoint{5.700000in}{5.700000in}}%
\pgfusepath{clip}%
\pgfsetbuttcap%
\pgfsetroundjoin%
\definecolor{currentfill}{rgb}{0.278791,0.062145,0.386592}%
\pgfsetfillcolor{currentfill}%
\pgfsetfillopacity{0.700000}%
\pgfsetlinewidth{0.000000pt}%
\definecolor{currentstroke}{rgb}{0.000000,0.000000,0.000000}%
\pgfsetstrokecolor{currentstroke}%
\pgfsetdash{}{0pt}%
\pgfpathmoveto{\pgfqpoint{3.664914in}{2.147407in}}%
\pgfpathlineto{\pgfqpoint{3.678493in}{2.144101in}}%
\pgfpathlineto{\pgfqpoint{3.692078in}{2.140879in}}%
\pgfpathlineto{\pgfqpoint{3.705668in}{2.137741in}}%
\pgfpathlineto{\pgfqpoint{3.719264in}{2.134686in}}%
\pgfpathlineto{\pgfqpoint{3.727289in}{2.143600in}}%
\pgfpathlineto{\pgfqpoint{3.735309in}{2.152518in}}%
\pgfpathlineto{\pgfqpoint{3.743322in}{2.161441in}}%
\pgfpathlineto{\pgfqpoint{3.751330in}{2.170370in}}%
\pgfpathlineto{\pgfqpoint{3.737745in}{2.173464in}}%
\pgfpathlineto{\pgfqpoint{3.724166in}{2.176642in}}%
\pgfpathlineto{\pgfqpoint{3.710592in}{2.179903in}}%
\pgfpathlineto{\pgfqpoint{3.697025in}{2.183248in}}%
\pgfpathlineto{\pgfqpoint{3.689006in}{2.174273in}}%
\pgfpathlineto{\pgfqpoint{3.680981in}{2.165308in}}%
\pgfpathlineto{\pgfqpoint{3.672951in}{2.156353in}}%
\pgfpathlineto{\pgfqpoint{3.664914in}{2.147407in}}%
\pgfpathclose%
\pgfusepath{fill}%
\end{pgfscope}%
\begin{pgfscope}%
\pgfpathrectangle{\pgfqpoint{1.150000in}{0.150000in}}{\pgfqpoint{5.700000in}{5.700000in}}%
\pgfusepath{clip}%
\pgfsetbuttcap%
\pgfsetroundjoin%
\definecolor{currentfill}{rgb}{0.252194,0.269783,0.531579}%
\pgfsetfillcolor{currentfill}%
\pgfsetfillopacity{0.700000}%
\pgfsetlinewidth{0.000000pt}%
\definecolor{currentstroke}{rgb}{0.000000,0.000000,0.000000}%
\pgfsetstrokecolor{currentstroke}%
\pgfsetdash{}{0pt}%
\pgfpathmoveto{\pgfqpoint{5.318648in}{2.567914in}}%
\pgfpathlineto{\pgfqpoint{5.332692in}{2.567750in}}%
\pgfpathlineto{\pgfqpoint{5.346745in}{2.567654in}}%
\pgfpathlineto{\pgfqpoint{5.360809in}{2.567626in}}%
\pgfpathlineto{\pgfqpoint{5.374883in}{2.567665in}}%
\pgfpathlineto{\pgfqpoint{5.382291in}{2.575000in}}%
\pgfpathlineto{\pgfqpoint{5.389698in}{2.582474in}}%
\pgfpathlineto{\pgfqpoint{5.397102in}{2.590091in}}%
\pgfpathlineto{\pgfqpoint{5.404505in}{2.597860in}}%
\pgfpathlineto{\pgfqpoint{5.390453in}{2.598188in}}%
\pgfpathlineto{\pgfqpoint{5.376411in}{2.598584in}}%
\pgfpathlineto{\pgfqpoint{5.362378in}{2.599047in}}%
\pgfpathlineto{\pgfqpoint{5.348355in}{2.599578in}}%
\pgfpathlineto{\pgfqpoint{5.340931in}{2.591435in}}%
\pgfpathlineto{\pgfqpoint{5.333505in}{2.583448in}}%
\pgfpathlineto{\pgfqpoint{5.326078in}{2.575610in}}%
\pgfpathlineto{\pgfqpoint{5.318648in}{2.567914in}}%
\pgfpathclose%
\pgfusepath{fill}%
\end{pgfscope}%
\begin{pgfscope}%
\pgfpathrectangle{\pgfqpoint{1.150000in}{0.150000in}}{\pgfqpoint{5.700000in}{5.700000in}}%
\pgfusepath{clip}%
\pgfsetbuttcap%
\pgfsetroundjoin%
\definecolor{currentfill}{rgb}{0.229739,0.322361,0.545706}%
\pgfsetfillcolor{currentfill}%
\pgfsetfillopacity{0.700000}%
\pgfsetlinewidth{0.000000pt}%
\definecolor{currentstroke}{rgb}{0.000000,0.000000,0.000000}%
\pgfsetstrokecolor{currentstroke}%
\pgfsetdash{}{0pt}%
\pgfpathmoveto{\pgfqpoint{5.718401in}{2.689632in}}%
\pgfpathlineto{\pgfqpoint{5.732553in}{2.689210in}}%
\pgfpathlineto{\pgfqpoint{5.746716in}{2.688854in}}%
\pgfpathlineto{\pgfqpoint{5.760890in}{2.688564in}}%
\pgfpathlineto{\pgfqpoint{5.775074in}{2.688340in}}%
\pgfpathlineto{\pgfqpoint{5.782359in}{2.696510in}}%
\pgfpathlineto{\pgfqpoint{5.789646in}{2.704908in}}%
\pgfpathlineto{\pgfqpoint{5.796936in}{2.713542in}}%
\pgfpathlineto{\pgfqpoint{5.804228in}{2.722420in}}%
\pgfpathlineto{\pgfqpoint{5.790071in}{2.723093in}}%
\pgfpathlineto{\pgfqpoint{5.775924in}{2.723832in}}%
\pgfpathlineto{\pgfqpoint{5.761787in}{2.724635in}}%
\pgfpathlineto{\pgfqpoint{5.747660in}{2.725504in}}%
\pgfpathlineto{\pgfqpoint{5.740341in}{2.716171in}}%
\pgfpathlineto{\pgfqpoint{5.733025in}{2.707087in}}%
\pgfpathlineto{\pgfqpoint{5.725711in}{2.698243in}}%
\pgfpathlineto{\pgfqpoint{5.718401in}{2.689632in}}%
\pgfpathclose%
\pgfusepath{fill}%
\end{pgfscope}%
\begin{pgfscope}%
\pgfpathrectangle{\pgfqpoint{1.150000in}{0.150000in}}{\pgfqpoint{5.700000in}{5.700000in}}%
\pgfusepath{clip}%
\pgfsetbuttcap%
\pgfsetroundjoin%
\definecolor{currentfill}{rgb}{0.277018,0.050344,0.375715}%
\pgfsetfillcolor{currentfill}%
\pgfsetfillopacity{0.700000}%
\pgfsetlinewidth{0.000000pt}%
\definecolor{currentstroke}{rgb}{0.000000,0.000000,0.000000}%
\pgfsetstrokecolor{currentstroke}%
\pgfsetdash{}{0pt}%
\pgfpathmoveto{\pgfqpoint{3.437628in}{2.122190in}}%
\pgfpathlineto{\pgfqpoint{3.451165in}{2.117864in}}%
\pgfpathlineto{\pgfqpoint{3.464708in}{2.113627in}}%
\pgfpathlineto{\pgfqpoint{3.478255in}{2.109477in}}%
\pgfpathlineto{\pgfqpoint{3.491808in}{2.105416in}}%
\pgfpathlineto{\pgfqpoint{3.499913in}{2.114305in}}%
\pgfpathlineto{\pgfqpoint{3.508013in}{2.123208in}}%
\pgfpathlineto{\pgfqpoint{3.516107in}{2.132123in}}%
\pgfpathlineto{\pgfqpoint{3.524194in}{2.141052in}}%
\pgfpathlineto{\pgfqpoint{3.510654in}{2.145112in}}%
\pgfpathlineto{\pgfqpoint{3.497118in}{2.149260in}}%
\pgfpathlineto{\pgfqpoint{3.483587in}{2.153495in}}%
\pgfpathlineto{\pgfqpoint{3.470061in}{2.157820in}}%
\pgfpathlineto{\pgfqpoint{3.461962in}{2.148884in}}%
\pgfpathlineto{\pgfqpoint{3.453856in}{2.139968in}}%
\pgfpathlineto{\pgfqpoint{3.445745in}{2.131070in}}%
\pgfpathlineto{\pgfqpoint{3.437628in}{2.122190in}}%
\pgfpathclose%
\pgfusepath{fill}%
\end{pgfscope}%
\begin{pgfscope}%
\pgfpathrectangle{\pgfqpoint{1.150000in}{0.150000in}}{\pgfqpoint{5.700000in}{5.700000in}}%
\pgfusepath{clip}%
\pgfsetbuttcap%
\pgfsetroundjoin%
\definecolor{currentfill}{rgb}{0.269308,0.218818,0.509577}%
\pgfsetfillcolor{currentfill}%
\pgfsetfillopacity{0.700000}%
\pgfsetlinewidth{0.000000pt}%
\definecolor{currentstroke}{rgb}{0.000000,0.000000,0.000000}%
\pgfsetstrokecolor{currentstroke}%
\pgfsetdash{}{0pt}%
\pgfpathmoveto{\pgfqpoint{4.918979in}{2.451501in}}%
\pgfpathlineto{\pgfqpoint{4.932905in}{2.451228in}}%
\pgfpathlineto{\pgfqpoint{4.946840in}{2.451025in}}%
\pgfpathlineto{\pgfqpoint{4.960785in}{2.450891in}}%
\pgfpathlineto{\pgfqpoint{4.974739in}{2.450827in}}%
\pgfpathlineto{\pgfqpoint{4.982299in}{2.458173in}}%
\pgfpathlineto{\pgfqpoint{4.989855in}{2.465589in}}%
\pgfpathlineto{\pgfqpoint{4.997406in}{2.473080in}}%
\pgfpathlineto{\pgfqpoint{5.004954in}{2.480652in}}%
\pgfpathlineto{\pgfqpoint{4.991017in}{2.481001in}}%
\pgfpathlineto{\pgfqpoint{4.977089in}{2.481421in}}%
\pgfpathlineto{\pgfqpoint{4.963171in}{2.481910in}}%
\pgfpathlineto{\pgfqpoint{4.949262in}{2.482469in}}%
\pgfpathlineto{\pgfqpoint{4.941698in}{2.474604in}}%
\pgfpathlineto{\pgfqpoint{4.934129in}{2.466825in}}%
\pgfpathlineto{\pgfqpoint{4.926556in}{2.459126in}}%
\pgfpathlineto{\pgfqpoint{4.918979in}{2.451501in}}%
\pgfpathclose%
\pgfusepath{fill}%
\end{pgfscope}%
\begin{pgfscope}%
\pgfpathrectangle{\pgfqpoint{1.150000in}{0.150000in}}{\pgfqpoint{5.700000in}{5.700000in}}%
\pgfusepath{clip}%
\pgfsetbuttcap%
\pgfsetroundjoin%
\definecolor{currentfill}{rgb}{0.283229,0.120777,0.440584}%
\pgfsetfillcolor{currentfill}%
\pgfsetfillopacity{0.700000}%
\pgfsetlinewidth{0.000000pt}%
\definecolor{currentstroke}{rgb}{0.000000,0.000000,0.000000}%
\pgfsetstrokecolor{currentstroke}%
\pgfsetdash{}{0pt}%
\pgfpathmoveto{\pgfqpoint{4.205671in}{2.256544in}}%
\pgfpathlineto{\pgfqpoint{4.219384in}{2.255073in}}%
\pgfpathlineto{\pgfqpoint{4.233105in}{2.253679in}}%
\pgfpathlineto{\pgfqpoint{4.246834in}{2.252360in}}%
\pgfpathlineto{\pgfqpoint{4.260571in}{2.251117in}}%
\pgfpathlineto{\pgfqpoint{4.268405in}{2.259469in}}%
\pgfpathlineto{\pgfqpoint{4.276234in}{2.267828in}}%
\pgfpathlineto{\pgfqpoint{4.284057in}{2.276194in}}%
\pgfpathlineto{\pgfqpoint{4.291875in}{2.284572in}}%
\pgfpathlineto{\pgfqpoint{4.278150in}{2.285957in}}%
\pgfpathlineto{\pgfqpoint{4.264433in}{2.287418in}}%
\pgfpathlineto{\pgfqpoint{4.250723in}{2.288954in}}%
\pgfpathlineto{\pgfqpoint{4.237021in}{2.290567in}}%
\pgfpathlineto{\pgfqpoint{4.229192in}{2.282039in}}%
\pgfpathlineto{\pgfqpoint{4.221357in}{2.273528in}}%
\pgfpathlineto{\pgfqpoint{4.213517in}{2.265031in}}%
\pgfpathlineto{\pgfqpoint{4.205671in}{2.256544in}}%
\pgfpathclose%
\pgfusepath{fill}%
\end{pgfscope}%
\begin{pgfscope}%
\pgfpathrectangle{\pgfqpoint{1.150000in}{0.150000in}}{\pgfqpoint{5.700000in}{5.700000in}}%
\pgfusepath{clip}%
\pgfsetbuttcap%
\pgfsetroundjoin%
\definecolor{currentfill}{rgb}{0.281446,0.084320,0.407414}%
\pgfsetfillcolor{currentfill}%
\pgfsetfillopacity{0.700000}%
\pgfsetlinewidth{0.000000pt}%
\definecolor{currentstroke}{rgb}{0.000000,0.000000,0.000000}%
\pgfsetstrokecolor{currentstroke}%
\pgfsetdash{}{0pt}%
\pgfpathmoveto{\pgfqpoint{3.892126in}{2.183894in}}%
\pgfpathlineto{\pgfqpoint{3.905758in}{2.181475in}}%
\pgfpathlineto{\pgfqpoint{3.919396in}{2.179136in}}%
\pgfpathlineto{\pgfqpoint{3.933042in}{2.176877in}}%
\pgfpathlineto{\pgfqpoint{3.946694in}{2.174698in}}%
\pgfpathlineto{\pgfqpoint{3.954641in}{2.183458in}}%
\pgfpathlineto{\pgfqpoint{3.962582in}{2.192218in}}%
\pgfpathlineto{\pgfqpoint{3.970518in}{2.200979in}}%
\pgfpathlineto{\pgfqpoint{3.978448in}{2.209743in}}%
\pgfpathlineto{\pgfqpoint{3.964807in}{2.212002in}}%
\pgfpathlineto{\pgfqpoint{3.951172in}{2.214342in}}%
\pgfpathlineto{\pgfqpoint{3.937545in}{2.216761in}}%
\pgfpathlineto{\pgfqpoint{3.923924in}{2.219260in}}%
\pgfpathlineto{\pgfqpoint{3.915983in}{2.210408in}}%
\pgfpathlineto{\pgfqpoint{3.908036in}{2.201565in}}%
\pgfpathlineto{\pgfqpoint{3.900084in}{2.192727in}}%
\pgfpathlineto{\pgfqpoint{3.892126in}{2.183894in}}%
\pgfpathclose%
\pgfusepath{fill}%
\end{pgfscope}%
\begin{pgfscope}%
\pgfpathrectangle{\pgfqpoint{1.150000in}{0.150000in}}{\pgfqpoint{5.700000in}{5.700000in}}%
\pgfusepath{clip}%
\pgfsetbuttcap%
\pgfsetroundjoin%
\definecolor{currentfill}{rgb}{0.277018,0.050344,0.375715}%
\pgfsetfillcolor{currentfill}%
\pgfsetfillopacity{0.700000}%
\pgfsetlinewidth{0.000000pt}%
\definecolor{currentstroke}{rgb}{0.000000,0.000000,0.000000}%
\pgfsetstrokecolor{currentstroke}%
\pgfsetdash{}{0pt}%
\pgfpathmoveto{\pgfqpoint{3.069096in}{2.125419in}}%
\pgfpathlineto{\pgfqpoint{3.082595in}{2.119089in}}%
\pgfpathlineto{\pgfqpoint{3.096097in}{2.112857in}}%
\pgfpathlineto{\pgfqpoint{3.109602in}{2.106724in}}%
\pgfpathlineto{\pgfqpoint{3.123110in}{2.100688in}}%
\pgfpathlineto{\pgfqpoint{3.131356in}{2.109074in}}%
\pgfpathlineto{\pgfqpoint{3.139595in}{2.117500in}}%
\pgfpathlineto{\pgfqpoint{3.147828in}{2.125964in}}%
\pgfpathlineto{\pgfqpoint{3.156053in}{2.134467in}}%
\pgfpathlineto{\pgfqpoint{3.142559in}{2.140439in}}%
\pgfpathlineto{\pgfqpoint{3.129069in}{2.146509in}}%
\pgfpathlineto{\pgfqpoint{3.115581in}{2.152678in}}%
\pgfpathlineto{\pgfqpoint{3.102096in}{2.158945in}}%
\pgfpathlineto{\pgfqpoint{3.093857in}{2.150498in}}%
\pgfpathlineto{\pgfqpoint{3.085610in}{2.142094in}}%
\pgfpathlineto{\pgfqpoint{3.077356in}{2.133734in}}%
\pgfpathlineto{\pgfqpoint{3.069096in}{2.125419in}}%
\pgfpathclose%
\pgfusepath{fill}%
\end{pgfscope}%
\begin{pgfscope}%
\pgfpathrectangle{\pgfqpoint{1.150000in}{0.150000in}}{\pgfqpoint{5.700000in}{5.700000in}}%
\pgfusepath{clip}%
\pgfsetbuttcap%
\pgfsetroundjoin%
\definecolor{currentfill}{rgb}{0.280255,0.165693,0.476498}%
\pgfsetfillcolor{currentfill}%
\pgfsetfillopacity{0.700000}%
\pgfsetlinewidth{0.000000pt}%
\definecolor{currentstroke}{rgb}{0.000000,0.000000,0.000000}%
\pgfsetstrokecolor{currentstroke}%
\pgfsetdash{}{0pt}%
\pgfpathmoveto{\pgfqpoint{4.519250in}{2.337447in}}%
\pgfpathlineto{\pgfqpoint{4.533056in}{2.336682in}}%
\pgfpathlineto{\pgfqpoint{4.546871in}{2.335991in}}%
\pgfpathlineto{\pgfqpoint{4.560694in}{2.335372in}}%
\pgfpathlineto{\pgfqpoint{4.574526in}{2.334825in}}%
\pgfpathlineto{\pgfqpoint{4.582243in}{2.342667in}}%
\pgfpathlineto{\pgfqpoint{4.589955in}{2.350532in}}%
\pgfpathlineto{\pgfqpoint{4.597662in}{2.358426in}}%
\pgfpathlineto{\pgfqpoint{4.605363in}{2.366352in}}%
\pgfpathlineto{\pgfqpoint{4.591545in}{2.367102in}}%
\pgfpathlineto{\pgfqpoint{4.577735in}{2.367925in}}%
\pgfpathlineto{\pgfqpoint{4.563933in}{2.368820in}}%
\pgfpathlineto{\pgfqpoint{4.550141in}{2.369789in}}%
\pgfpathlineto{\pgfqpoint{4.542426in}{2.361652in}}%
\pgfpathlineto{\pgfqpoint{4.534706in}{2.353551in}}%
\pgfpathlineto{\pgfqpoint{4.526980in}{2.345485in}}%
\pgfpathlineto{\pgfqpoint{4.519250in}{2.337447in}}%
\pgfpathclose%
\pgfusepath{fill}%
\end{pgfscope}%
\begin{pgfscope}%
\pgfpathrectangle{\pgfqpoint{1.150000in}{0.150000in}}{\pgfqpoint{5.700000in}{5.700000in}}%
\pgfusepath{clip}%
\pgfsetbuttcap%
\pgfsetroundjoin%
\definecolor{currentfill}{rgb}{0.278791,0.062145,0.386592}%
\pgfsetfillcolor{currentfill}%
\pgfsetfillopacity{0.700000}%
\pgfsetlinewidth{0.000000pt}%
\definecolor{currentstroke}{rgb}{0.000000,0.000000,0.000000}%
\pgfsetstrokecolor{currentstroke}%
\pgfsetdash{}{0pt}%
\pgfpathmoveto{\pgfqpoint{2.927963in}{2.147581in}}%
\pgfpathlineto{\pgfqpoint{2.941457in}{2.140351in}}%
\pgfpathlineto{\pgfqpoint{2.954954in}{2.133225in}}%
\pgfpathlineto{\pgfqpoint{2.968453in}{2.126202in}}%
\pgfpathlineto{\pgfqpoint{2.981954in}{2.119282in}}%
\pgfpathlineto{\pgfqpoint{2.990258in}{2.127320in}}%
\pgfpathlineto{\pgfqpoint{2.998555in}{2.135410in}}%
\pgfpathlineto{\pgfqpoint{3.006845in}{2.143552in}}%
\pgfpathlineto{\pgfqpoint{3.015128in}{2.151746in}}%
\pgfpathlineto{\pgfqpoint{3.001642in}{2.158581in}}%
\pgfpathlineto{\pgfqpoint{2.988159in}{2.165520in}}%
\pgfpathlineto{\pgfqpoint{2.974678in}{2.172562in}}%
\pgfpathlineto{\pgfqpoint{2.961199in}{2.179708in}}%
\pgfpathlineto{\pgfqpoint{2.952901in}{2.171591in}}%
\pgfpathlineto{\pgfqpoint{2.944596in}{2.163531in}}%
\pgfpathlineto{\pgfqpoint{2.936283in}{2.155527in}}%
\pgfpathlineto{\pgfqpoint{2.927963in}{2.147581in}}%
\pgfpathclose%
\pgfusepath{fill}%
\end{pgfscope}%
\begin{pgfscope}%
\pgfpathrectangle{\pgfqpoint{1.150000in}{0.150000in}}{\pgfqpoint{5.700000in}{5.700000in}}%
\pgfusepath{clip}%
\pgfsetbuttcap%
\pgfsetroundjoin%
\definecolor{currentfill}{rgb}{0.233603,0.313828,0.543914}%
\pgfsetfillcolor{currentfill}%
\pgfsetfillopacity{0.700000}%
\pgfsetlinewidth{0.000000pt}%
\definecolor{currentstroke}{rgb}{0.000000,0.000000,0.000000}%
\pgfsetstrokecolor{currentstroke}%
\pgfsetdash{}{0pt}%
\pgfpathmoveto{\pgfqpoint{5.632563in}{2.657993in}}%
\pgfpathlineto{\pgfqpoint{5.646700in}{2.657736in}}%
\pgfpathlineto{\pgfqpoint{5.660847in}{2.657545in}}%
\pgfpathlineto{\pgfqpoint{5.675005in}{2.657420in}}%
\pgfpathlineto{\pgfqpoint{5.689173in}{2.657362in}}%
\pgfpathlineto{\pgfqpoint{5.696478in}{2.665118in}}%
\pgfpathlineto{\pgfqpoint{5.703784in}{2.673077in}}%
\pgfpathlineto{\pgfqpoint{5.711092in}{2.681246in}}%
\pgfpathlineto{\pgfqpoint{5.718401in}{2.689632in}}%
\pgfpathlineto{\pgfqpoint{5.704258in}{2.690119in}}%
\pgfpathlineto{\pgfqpoint{5.690126in}{2.690672in}}%
\pgfpathlineto{\pgfqpoint{5.676003in}{2.691291in}}%
\pgfpathlineto{\pgfqpoint{5.661891in}{2.691976in}}%
\pgfpathlineto{\pgfqpoint{5.654557in}{2.683155in}}%
\pgfpathlineto{\pgfqpoint{5.647225in}{2.674556in}}%
\pgfpathlineto{\pgfqpoint{5.639893in}{2.666171in}}%
\pgfpathlineto{\pgfqpoint{5.632563in}{2.657993in}}%
\pgfpathclose%
\pgfusepath{fill}%
\end{pgfscope}%
\begin{pgfscope}%
\pgfpathrectangle{\pgfqpoint{1.150000in}{0.150000in}}{\pgfqpoint{5.700000in}{5.700000in}}%
\pgfusepath{clip}%
\pgfsetbuttcap%
\pgfsetroundjoin%
\definecolor{currentfill}{rgb}{0.255645,0.260703,0.528312}%
\pgfsetfillcolor{currentfill}%
\pgfsetfillopacity{0.700000}%
\pgfsetlinewidth{0.000000pt}%
\definecolor{currentstroke}{rgb}{0.000000,0.000000,0.000000}%
\pgfsetstrokecolor{currentstroke}%
\pgfsetdash{}{0pt}%
\pgfpathmoveto{\pgfqpoint{5.232747in}{2.538376in}}%
\pgfpathlineto{\pgfqpoint{5.246772in}{2.538288in}}%
\pgfpathlineto{\pgfqpoint{5.260806in}{2.538268in}}%
\pgfpathlineto{\pgfqpoint{5.274851in}{2.538316in}}%
\pgfpathlineto{\pgfqpoint{5.288905in}{2.538433in}}%
\pgfpathlineto{\pgfqpoint{5.296345in}{2.545620in}}%
\pgfpathlineto{\pgfqpoint{5.303782in}{2.552926in}}%
\pgfpathlineto{\pgfqpoint{5.311216in}{2.560355in}}%
\pgfpathlineto{\pgfqpoint{5.318648in}{2.567914in}}%
\pgfpathlineto{\pgfqpoint{5.304614in}{2.568145in}}%
\pgfpathlineto{\pgfqpoint{5.290590in}{2.568445in}}%
\pgfpathlineto{\pgfqpoint{5.276576in}{2.568812in}}%
\pgfpathlineto{\pgfqpoint{5.262571in}{2.569246in}}%
\pgfpathlineto{\pgfqpoint{5.255119in}{2.561333in}}%
\pgfpathlineto{\pgfqpoint{5.247664in}{2.553554in}}%
\pgfpathlineto{\pgfqpoint{5.240207in}{2.545904in}}%
\pgfpathlineto{\pgfqpoint{5.232747in}{2.538376in}}%
\pgfpathclose%
\pgfusepath{fill}%
\end{pgfscope}%
\begin{pgfscope}%
\pgfpathrectangle{\pgfqpoint{1.150000in}{0.150000in}}{\pgfqpoint{5.700000in}{5.700000in}}%
\pgfusepath{clip}%
\pgfsetbuttcap%
\pgfsetroundjoin%
\definecolor{currentfill}{rgb}{0.276022,0.044167,0.370164}%
\pgfsetfillcolor{currentfill}%
\pgfsetfillopacity{0.700000}%
\pgfsetlinewidth{0.000000pt}%
\definecolor{currentstroke}{rgb}{0.000000,0.000000,0.000000}%
\pgfsetstrokecolor{currentstroke}%
\pgfsetdash{}{0pt}%
\pgfpathmoveto{\pgfqpoint{3.210064in}{2.111545in}}%
\pgfpathlineto{\pgfqpoint{3.223575in}{2.106054in}}%
\pgfpathlineto{\pgfqpoint{3.237091in}{2.100657in}}%
\pgfpathlineto{\pgfqpoint{3.250610in}{2.095354in}}%
\pgfpathlineto{\pgfqpoint{3.264133in}{2.090144in}}%
\pgfpathlineto{\pgfqpoint{3.272325in}{2.098786in}}%
\pgfpathlineto{\pgfqpoint{3.280511in}{2.107455in}}%
\pgfpathlineto{\pgfqpoint{3.288690in}{2.116152in}}%
\pgfpathlineto{\pgfqpoint{3.296863in}{2.124876in}}%
\pgfpathlineto{\pgfqpoint{3.283353in}{2.130043in}}%
\pgfpathlineto{\pgfqpoint{3.269847in}{2.135304in}}%
\pgfpathlineto{\pgfqpoint{3.256345in}{2.140658in}}%
\pgfpathlineto{\pgfqpoint{3.242846in}{2.146107in}}%
\pgfpathlineto{\pgfqpoint{3.234660in}{2.137418in}}%
\pgfpathlineto{\pgfqpoint{3.226468in}{2.128761in}}%
\pgfpathlineto{\pgfqpoint{3.218269in}{2.120137in}}%
\pgfpathlineto{\pgfqpoint{3.210064in}{2.111545in}}%
\pgfpathclose%
\pgfusepath{fill}%
\end{pgfscope}%
\begin{pgfscope}%
\pgfpathrectangle{\pgfqpoint{1.150000in}{0.150000in}}{\pgfqpoint{5.700000in}{5.700000in}}%
\pgfusepath{clip}%
\pgfsetbuttcap%
\pgfsetroundjoin%
\definecolor{currentfill}{rgb}{0.208623,0.367752,0.552675}%
\pgfsetfillcolor{currentfill}%
\pgfsetfillopacity{0.700000}%
\pgfsetlinewidth{0.000000pt}%
\definecolor{currentstroke}{rgb}{0.000000,0.000000,0.000000}%
\pgfsetstrokecolor{currentstroke}%
\pgfsetdash{}{0pt}%
\pgfpathmoveto{\pgfqpoint{6.032757in}{2.788753in}}%
\pgfpathlineto{\pgfqpoint{6.046993in}{2.787968in}}%
\pgfpathlineto{\pgfqpoint{6.061239in}{2.787248in}}%
\pgfpathlineto{\pgfqpoint{6.075496in}{2.786593in}}%
\pgfpathlineto{\pgfqpoint{6.082727in}{2.796166in}}%
\pgfpathlineto{\pgfqpoint{6.089966in}{2.806050in}}%
\pgfpathlineto{\pgfqpoint{6.097212in}{2.816254in}}%
\pgfpathlineto{\pgfqpoint{6.104466in}{2.826787in}}%
\pgfpathlineto{\pgfqpoint{6.090240in}{2.827951in}}%
\pgfpathlineto{\pgfqpoint{6.076023in}{2.829180in}}%
\pgfpathlineto{\pgfqpoint{6.061817in}{2.830472in}}%
\pgfpathlineto{\pgfqpoint{6.054540in}{2.819553in}}%
\pgfpathlineto{\pgfqpoint{6.047272in}{2.808966in}}%
\pgfpathlineto{\pgfqpoint{6.040011in}{2.798702in}}%
\pgfpathlineto{\pgfqpoint{6.032757in}{2.788753in}}%
\pgfpathclose%
\pgfusepath{fill}%
\end{pgfscope}%
\begin{pgfscope}%
\pgfpathrectangle{\pgfqpoint{1.150000in}{0.150000in}}{\pgfqpoint{5.700000in}{5.700000in}}%
\pgfusepath{clip}%
\pgfsetbuttcap%
\pgfsetroundjoin%
\definecolor{currentfill}{rgb}{0.277134,0.185228,0.489898}%
\pgfsetfillcolor{currentfill}%
\pgfsetfillopacity{0.700000}%
\pgfsetlinewidth{0.000000pt}%
\definecolor{currentstroke}{rgb}{0.000000,0.000000,0.000000}%
\pgfsetstrokecolor{currentstroke}%
\pgfsetdash{}{0pt}%
\pgfpathmoveto{\pgfqpoint{2.340055in}{2.407641in}}%
\pgfpathlineto{\pgfqpoint{2.353614in}{2.395689in}}%
\pgfpathlineto{\pgfqpoint{2.367171in}{2.383877in}}%
\pgfpathlineto{\pgfqpoint{2.380725in}{2.372203in}}%
\pgfpathlineto{\pgfqpoint{2.394277in}{2.360666in}}%
\pgfpathlineto{\pgfqpoint{2.402863in}{2.366574in}}%
\pgfpathlineto{\pgfqpoint{2.411438in}{2.372598in}}%
\pgfpathlineto{\pgfqpoint{2.420003in}{2.378737in}}%
\pgfpathlineto{\pgfqpoint{2.428557in}{2.384990in}}%
\pgfpathlineto{\pgfqpoint{2.415028in}{2.396377in}}%
\pgfpathlineto{\pgfqpoint{2.401498in}{2.407901in}}%
\pgfpathlineto{\pgfqpoint{2.387965in}{2.419563in}}%
\pgfpathlineto{\pgfqpoint{2.374429in}{2.431364in}}%
\pgfpathlineto{\pgfqpoint{2.365852in}{2.425254in}}%
\pgfpathlineto{\pgfqpoint{2.357264in}{2.419262in}}%
\pgfpathlineto{\pgfqpoint{2.348665in}{2.413390in}}%
\pgfpathlineto{\pgfqpoint{2.340055in}{2.407641in}}%
\pgfpathclose%
\pgfusepath{fill}%
\end{pgfscope}%
\begin{pgfscope}%
\pgfpathrectangle{\pgfqpoint{1.150000in}{0.150000in}}{\pgfqpoint{5.700000in}{5.700000in}}%
\pgfusepath{clip}%
\pgfsetbuttcap%
\pgfsetroundjoin%
\definecolor{currentfill}{rgb}{0.283197,0.115680,0.436115}%
\pgfsetfillcolor{currentfill}%
\pgfsetfillopacity{0.700000}%
\pgfsetlinewidth{0.000000pt}%
\definecolor{currentstroke}{rgb}{0.000000,0.000000,0.000000}%
\pgfsetstrokecolor{currentstroke}%
\pgfsetdash{}{0pt}%
\pgfpathmoveto{\pgfqpoint{2.590783in}{2.258570in}}%
\pgfpathlineto{\pgfqpoint{2.604296in}{2.248854in}}%
\pgfpathlineto{\pgfqpoint{2.617809in}{2.239260in}}%
\pgfpathlineto{\pgfqpoint{2.631322in}{2.229785in}}%
\pgfpathlineto{\pgfqpoint{2.644834in}{2.220430in}}%
\pgfpathlineto{\pgfqpoint{2.653293in}{2.227335in}}%
\pgfpathlineto{\pgfqpoint{2.661743in}{2.234329in}}%
\pgfpathlineto{\pgfqpoint{2.670184in}{2.241410in}}%
\pgfpathlineto{\pgfqpoint{2.678616in}{2.248577in}}%
\pgfpathlineto{\pgfqpoint{2.665123in}{2.257806in}}%
\pgfpathlineto{\pgfqpoint{2.651630in}{2.267154in}}%
\pgfpathlineto{\pgfqpoint{2.638138in}{2.276622in}}%
\pgfpathlineto{\pgfqpoint{2.624645in}{2.286211in}}%
\pgfpathlineto{\pgfqpoint{2.616193in}{2.279163in}}%
\pgfpathlineto{\pgfqpoint{2.607732in}{2.272206in}}%
\pgfpathlineto{\pgfqpoint{2.599263in}{2.265341in}}%
\pgfpathlineto{\pgfqpoint{2.590783in}{2.258570in}}%
\pgfpathclose%
\pgfusepath{fill}%
\end{pgfscope}%
\begin{pgfscope}%
\pgfpathrectangle{\pgfqpoint{1.150000in}{0.150000in}}{\pgfqpoint{5.700000in}{5.700000in}}%
\pgfusepath{clip}%
\pgfsetbuttcap%
\pgfsetroundjoin%
\definecolor{currentfill}{rgb}{0.273006,0.204520,0.501721}%
\pgfsetfillcolor{currentfill}%
\pgfsetfillopacity{0.700000}%
\pgfsetlinewidth{0.000000pt}%
\definecolor{currentstroke}{rgb}{0.000000,0.000000,0.000000}%
\pgfsetstrokecolor{currentstroke}%
\pgfsetdash{}{0pt}%
\pgfpathmoveto{\pgfqpoint{4.832950in}{2.422390in}}%
\pgfpathlineto{\pgfqpoint{4.846855in}{2.422100in}}%
\pgfpathlineto{\pgfqpoint{4.860769in}{2.421882in}}%
\pgfpathlineto{\pgfqpoint{4.874693in}{2.421734in}}%
\pgfpathlineto{\pgfqpoint{4.888626in}{2.421656in}}%
\pgfpathlineto{\pgfqpoint{4.896221in}{2.429029in}}%
\pgfpathlineto{\pgfqpoint{4.903812in}{2.436458in}}%
\pgfpathlineto{\pgfqpoint{4.911398in}{2.443947in}}%
\pgfpathlineto{\pgfqpoint{4.918979in}{2.451501in}}%
\pgfpathlineto{\pgfqpoint{4.905063in}{2.451845in}}%
\pgfpathlineto{\pgfqpoint{4.891156in}{2.452259in}}%
\pgfpathlineto{\pgfqpoint{4.877258in}{2.452743in}}%
\pgfpathlineto{\pgfqpoint{4.863369in}{2.453297in}}%
\pgfpathlineto{\pgfqpoint{4.855771in}{2.445470in}}%
\pgfpathlineto{\pgfqpoint{4.848169in}{2.437713in}}%
\pgfpathlineto{\pgfqpoint{4.840562in}{2.430021in}}%
\pgfpathlineto{\pgfqpoint{4.832950in}{2.422390in}}%
\pgfpathclose%
\pgfusepath{fill}%
\end{pgfscope}%
\begin{pgfscope}%
\pgfpathrectangle{\pgfqpoint{1.150000in}{0.150000in}}{\pgfqpoint{5.700000in}{5.700000in}}%
\pgfusepath{clip}%
\pgfsetbuttcap%
\pgfsetroundjoin%
\definecolor{currentfill}{rgb}{0.277941,0.056324,0.381191}%
\pgfsetfillcolor{currentfill}%
\pgfsetfillopacity{0.700000}%
\pgfsetlinewidth{0.000000pt}%
\definecolor{currentstroke}{rgb}{0.000000,0.000000,0.000000}%
\pgfsetstrokecolor{currentstroke}%
\pgfsetdash{}{0pt}%
\pgfpathmoveto{\pgfqpoint{3.578410in}{2.125684in}}%
\pgfpathlineto{\pgfqpoint{3.591977in}{2.122057in}}%
\pgfpathlineto{\pgfqpoint{3.605550in}{2.118516in}}%
\pgfpathlineto{\pgfqpoint{3.619128in}{2.115060in}}%
\pgfpathlineto{\pgfqpoint{3.632712in}{2.111688in}}%
\pgfpathlineto{\pgfqpoint{3.640771in}{2.120610in}}%
\pgfpathlineto{\pgfqpoint{3.648825in}{2.129536in}}%
\pgfpathlineto{\pgfqpoint{3.656873in}{2.138468in}}%
\pgfpathlineto{\pgfqpoint{3.664914in}{2.147407in}}%
\pgfpathlineto{\pgfqpoint{3.651342in}{2.150797in}}%
\pgfpathlineto{\pgfqpoint{3.637774in}{2.154272in}}%
\pgfpathlineto{\pgfqpoint{3.624213in}{2.157832in}}%
\pgfpathlineto{\pgfqpoint{3.610657in}{2.161478in}}%
\pgfpathlineto{\pgfqpoint{3.602604in}{2.152513in}}%
\pgfpathlineto{\pgfqpoint{3.594545in}{2.143559in}}%
\pgfpathlineto{\pgfqpoint{3.586480in}{2.134616in}}%
\pgfpathlineto{\pgfqpoint{3.578410in}{2.125684in}}%
\pgfpathclose%
\pgfusepath{fill}%
\end{pgfscope}%
\begin{pgfscope}%
\pgfpathrectangle{\pgfqpoint{1.150000in}{0.150000in}}{\pgfqpoint{5.700000in}{5.700000in}}%
\pgfusepath{clip}%
\pgfsetbuttcap%
\pgfsetroundjoin%
\definecolor{currentfill}{rgb}{0.283091,0.110553,0.431554}%
\pgfsetfillcolor{currentfill}%
\pgfsetfillopacity{0.700000}%
\pgfsetlinewidth{0.000000pt}%
\definecolor{currentstroke}{rgb}{0.000000,0.000000,0.000000}%
\pgfsetstrokecolor{currentstroke}%
\pgfsetdash{}{0pt}%
\pgfpathmoveto{\pgfqpoint{4.119407in}{2.228818in}}%
\pgfpathlineto{\pgfqpoint{4.133101in}{2.227162in}}%
\pgfpathlineto{\pgfqpoint{4.146803in}{2.225583in}}%
\pgfpathlineto{\pgfqpoint{4.160513in}{2.224080in}}%
\pgfpathlineto{\pgfqpoint{4.174230in}{2.222655in}}%
\pgfpathlineto{\pgfqpoint{4.182099in}{2.231124in}}%
\pgfpathlineto{\pgfqpoint{4.189962in}{2.239593in}}%
\pgfpathlineto{\pgfqpoint{4.197819in}{2.248066in}}%
\pgfpathlineto{\pgfqpoint{4.205671in}{2.256544in}}%
\pgfpathlineto{\pgfqpoint{4.191965in}{2.258091in}}%
\pgfpathlineto{\pgfqpoint{4.178267in}{2.259715in}}%
\pgfpathlineto{\pgfqpoint{4.164576in}{2.261415in}}%
\pgfpathlineto{\pgfqpoint{4.150893in}{2.263193in}}%
\pgfpathlineto{\pgfqpoint{4.143030in}{2.254585in}}%
\pgfpathlineto{\pgfqpoint{4.135161in}{2.245989in}}%
\pgfpathlineto{\pgfqpoint{4.127287in}{2.237400in}}%
\pgfpathlineto{\pgfqpoint{4.119407in}{2.228818in}}%
\pgfpathclose%
\pgfusepath{fill}%
\end{pgfscope}%
\begin{pgfscope}%
\pgfpathrectangle{\pgfqpoint{1.150000in}{0.150000in}}{\pgfqpoint{5.700000in}{5.700000in}}%
\pgfusepath{clip}%
\pgfsetbuttcap%
\pgfsetroundjoin%
\definecolor{currentfill}{rgb}{0.280894,0.078907,0.402329}%
\pgfsetfillcolor{currentfill}%
\pgfsetfillopacity{0.700000}%
\pgfsetlinewidth{0.000000pt}%
\definecolor{currentstroke}{rgb}{0.000000,0.000000,0.000000}%
\pgfsetstrokecolor{currentstroke}%
\pgfsetdash{}{0pt}%
\pgfpathmoveto{\pgfqpoint{2.786576in}{2.178927in}}%
\pgfpathlineto{\pgfqpoint{2.800074in}{2.170730in}}%
\pgfpathlineto{\pgfqpoint{2.813574in}{2.162643in}}%
\pgfpathlineto{\pgfqpoint{2.827075in}{2.154667in}}%
\pgfpathlineto{\pgfqpoint{2.840578in}{2.146798in}}%
\pgfpathlineto{\pgfqpoint{2.848946in}{2.154389in}}%
\pgfpathlineto{\pgfqpoint{2.857307in}{2.162047in}}%
\pgfpathlineto{\pgfqpoint{2.865659in}{2.169772in}}%
\pgfpathlineto{\pgfqpoint{2.874004in}{2.177561in}}%
\pgfpathlineto{\pgfqpoint{2.860519in}{2.185324in}}%
\pgfpathlineto{\pgfqpoint{2.847035in}{2.193196in}}%
\pgfpathlineto{\pgfqpoint{2.833552in}{2.201178in}}%
\pgfpathlineto{\pgfqpoint{2.820071in}{2.209270in}}%
\pgfpathlineto{\pgfqpoint{2.811710in}{2.201577in}}%
\pgfpathlineto{\pgfqpoint{2.803340in}{2.193955in}}%
\pgfpathlineto{\pgfqpoint{2.794962in}{2.186405in}}%
\pgfpathlineto{\pgfqpoint{2.786576in}{2.178927in}}%
\pgfpathclose%
\pgfusepath{fill}%
\end{pgfscope}%
\begin{pgfscope}%
\pgfpathrectangle{\pgfqpoint{1.150000in}{0.150000in}}{\pgfqpoint{5.700000in}{5.700000in}}%
\pgfusepath{clip}%
\pgfsetbuttcap%
\pgfsetroundjoin%
\definecolor{currentfill}{rgb}{0.280894,0.078907,0.402329}%
\pgfsetfillcolor{currentfill}%
\pgfsetfillopacity{0.700000}%
\pgfsetlinewidth{0.000000pt}%
\definecolor{currentstroke}{rgb}{0.000000,0.000000,0.000000}%
\pgfsetstrokecolor{currentstroke}%
\pgfsetdash{}{0pt}%
\pgfpathmoveto{\pgfqpoint{3.805733in}{2.158818in}}%
\pgfpathlineto{\pgfqpoint{3.819350in}{2.156136in}}%
\pgfpathlineto{\pgfqpoint{3.832973in}{2.153534in}}%
\pgfpathlineto{\pgfqpoint{3.846602in}{2.151014in}}%
\pgfpathlineto{\pgfqpoint{3.860238in}{2.148574in}}%
\pgfpathlineto{\pgfqpoint{3.868219in}{2.157405in}}%
\pgfpathlineto{\pgfqpoint{3.876193in}{2.166235in}}%
\pgfpathlineto{\pgfqpoint{3.884163in}{2.175064in}}%
\pgfpathlineto{\pgfqpoint{3.892126in}{2.183894in}}%
\pgfpathlineto{\pgfqpoint{3.878501in}{2.186394in}}%
\pgfpathlineto{\pgfqpoint{3.864883in}{2.188974in}}%
\pgfpathlineto{\pgfqpoint{3.851271in}{2.191635in}}%
\pgfpathlineto{\pgfqpoint{3.837665in}{2.194378in}}%
\pgfpathlineto{\pgfqpoint{3.829691in}{2.185480in}}%
\pgfpathlineto{\pgfqpoint{3.821711in}{2.176589in}}%
\pgfpathlineto{\pgfqpoint{3.813725in}{2.167702in}}%
\pgfpathlineto{\pgfqpoint{3.805733in}{2.158818in}}%
\pgfpathclose%
\pgfusepath{fill}%
\end{pgfscope}%
\begin{pgfscope}%
\pgfpathrectangle{\pgfqpoint{1.150000in}{0.150000in}}{\pgfqpoint{5.700000in}{5.700000in}}%
\pgfusepath{clip}%
\pgfsetbuttcap%
\pgfsetroundjoin%
\definecolor{currentfill}{rgb}{0.281887,0.150881,0.465405}%
\pgfsetfillcolor{currentfill}%
\pgfsetfillopacity{0.700000}%
\pgfsetlinewidth{0.000000pt}%
\definecolor{currentstroke}{rgb}{0.000000,0.000000,0.000000}%
\pgfsetstrokecolor{currentstroke}%
\pgfsetdash{}{0pt}%
\pgfpathmoveto{\pgfqpoint{4.433080in}{2.308581in}}%
\pgfpathlineto{\pgfqpoint{4.446865in}{2.307705in}}%
\pgfpathlineto{\pgfqpoint{4.460659in}{2.306903in}}%
\pgfpathlineto{\pgfqpoint{4.474461in}{2.306175in}}%
\pgfpathlineto{\pgfqpoint{4.488272in}{2.305521in}}%
\pgfpathlineto{\pgfqpoint{4.496025in}{2.313476in}}%
\pgfpathlineto{\pgfqpoint{4.503772in}{2.321447in}}%
\pgfpathlineto{\pgfqpoint{4.511514in}{2.329436in}}%
\pgfpathlineto{\pgfqpoint{4.519250in}{2.337447in}}%
\pgfpathlineto{\pgfqpoint{4.505452in}{2.338285in}}%
\pgfpathlineto{\pgfqpoint{4.491663in}{2.339197in}}%
\pgfpathlineto{\pgfqpoint{4.477882in}{2.340182in}}%
\pgfpathlineto{\pgfqpoint{4.464109in}{2.341240in}}%
\pgfpathlineto{\pgfqpoint{4.456360in}{2.333038in}}%
\pgfpathlineto{\pgfqpoint{4.448605in}{2.324863in}}%
\pgfpathlineto{\pgfqpoint{4.440845in}{2.316712in}}%
\pgfpathlineto{\pgfqpoint{4.433080in}{2.308581in}}%
\pgfpathclose%
\pgfusepath{fill}%
\end{pgfscope}%
\begin{pgfscope}%
\pgfpathrectangle{\pgfqpoint{1.150000in}{0.150000in}}{\pgfqpoint{5.700000in}{5.700000in}}%
\pgfusepath{clip}%
\pgfsetbuttcap%
\pgfsetroundjoin%
\definecolor{currentfill}{rgb}{0.276022,0.044167,0.370164}%
\pgfsetfillcolor{currentfill}%
\pgfsetfillopacity{0.700000}%
\pgfsetlinewidth{0.000000pt}%
\definecolor{currentstroke}{rgb}{0.000000,0.000000,0.000000}%
\pgfsetstrokecolor{currentstroke}%
\pgfsetdash{}{0pt}%
\pgfpathmoveto{\pgfqpoint{3.350945in}{2.105132in}}%
\pgfpathlineto{\pgfqpoint{3.364476in}{2.100425in}}%
\pgfpathlineto{\pgfqpoint{3.378012in}{2.095808in}}%
\pgfpathlineto{\pgfqpoint{3.391552in}{2.091281in}}%
\pgfpathlineto{\pgfqpoint{3.405097in}{2.086844in}}%
\pgfpathlineto{\pgfqpoint{3.413239in}{2.095655in}}%
\pgfpathlineto{\pgfqpoint{3.421375in}{2.104483in}}%
\pgfpathlineto{\pgfqpoint{3.429504in}{2.113328in}}%
\pgfpathlineto{\pgfqpoint{3.437628in}{2.122190in}}%
\pgfpathlineto{\pgfqpoint{3.424095in}{2.126605in}}%
\pgfpathlineto{\pgfqpoint{3.410566in}{2.131110in}}%
\pgfpathlineto{\pgfqpoint{3.397043in}{2.135705in}}%
\pgfpathlineto{\pgfqpoint{3.383524in}{2.140390in}}%
\pgfpathlineto{\pgfqpoint{3.375388in}{2.131542in}}%
\pgfpathlineto{\pgfqpoint{3.367247in}{2.122717in}}%
\pgfpathlineto{\pgfqpoint{3.359099in}{2.113914in}}%
\pgfpathlineto{\pgfqpoint{3.350945in}{2.105132in}}%
\pgfpathclose%
\pgfusepath{fill}%
\end{pgfscope}%
\begin{pgfscope}%
\pgfpathrectangle{\pgfqpoint{1.150000in}{0.150000in}}{\pgfqpoint{5.700000in}{5.700000in}}%
\pgfusepath{clip}%
\pgfsetbuttcap%
\pgfsetroundjoin%
\definecolor{currentfill}{rgb}{0.239346,0.300855,0.540844}%
\pgfsetfillcolor{currentfill}%
\pgfsetfillopacity{0.700000}%
\pgfsetlinewidth{0.000000pt}%
\definecolor{currentstroke}{rgb}{0.000000,0.000000,0.000000}%
\pgfsetstrokecolor{currentstroke}%
\pgfsetdash{}{0pt}%
\pgfpathmoveto{\pgfqpoint{5.546705in}{2.627261in}}%
\pgfpathlineto{\pgfqpoint{5.560825in}{2.627147in}}%
\pgfpathlineto{\pgfqpoint{5.574955in}{2.627099in}}%
\pgfpathlineto{\pgfqpoint{5.589096in}{2.627118in}}%
\pgfpathlineto{\pgfqpoint{5.603247in}{2.627204in}}%
\pgfpathlineto{\pgfqpoint{5.610576in}{2.634627in}}%
\pgfpathlineto{\pgfqpoint{5.617905in}{2.642228in}}%
\pgfpathlineto{\pgfqpoint{5.625234in}{2.650015in}}%
\pgfpathlineto{\pgfqpoint{5.632563in}{2.657993in}}%
\pgfpathlineto{\pgfqpoint{5.618437in}{2.658316in}}%
\pgfpathlineto{\pgfqpoint{5.604320in}{2.658706in}}%
\pgfpathlineto{\pgfqpoint{5.590214in}{2.659161in}}%
\pgfpathlineto{\pgfqpoint{5.576117in}{2.659684in}}%
\pgfpathlineto{\pgfqpoint{5.568764in}{2.651290in}}%
\pgfpathlineto{\pgfqpoint{5.561411in}{2.643093in}}%
\pgfpathlineto{\pgfqpoint{5.554058in}{2.635086in}}%
\pgfpathlineto{\pgfqpoint{5.546705in}{2.627261in}}%
\pgfpathclose%
\pgfusepath{fill}%
\end{pgfscope}%
\begin{pgfscope}%
\pgfpathrectangle{\pgfqpoint{1.150000in}{0.150000in}}{\pgfqpoint{5.700000in}{5.700000in}}%
\pgfusepath{clip}%
\pgfsetbuttcap%
\pgfsetroundjoin%
\definecolor{currentfill}{rgb}{0.260571,0.246922,0.522828}%
\pgfsetfillcolor{currentfill}%
\pgfsetfillopacity{0.700000}%
\pgfsetlinewidth{0.000000pt}%
\definecolor{currentstroke}{rgb}{0.000000,0.000000,0.000000}%
\pgfsetstrokecolor{currentstroke}%
\pgfsetdash{}{0pt}%
\pgfpathmoveto{\pgfqpoint{5.146798in}{2.509095in}}%
\pgfpathlineto{\pgfqpoint{5.160803in}{2.509061in}}%
\pgfpathlineto{\pgfqpoint{5.174817in}{2.509095in}}%
\pgfpathlineto{\pgfqpoint{5.188842in}{2.509198in}}%
\pgfpathlineto{\pgfqpoint{5.202876in}{2.509370in}}%
\pgfpathlineto{\pgfqpoint{5.210349in}{2.516467in}}%
\pgfpathlineto{\pgfqpoint{5.217818in}{2.523663in}}%
\pgfpathlineto{\pgfqpoint{5.225284in}{2.530964in}}%
\pgfpathlineto{\pgfqpoint{5.232747in}{2.538376in}}%
\pgfpathlineto{\pgfqpoint{5.218732in}{2.538532in}}%
\pgfpathlineto{\pgfqpoint{5.204727in}{2.538756in}}%
\pgfpathlineto{\pgfqpoint{5.190732in}{2.539049in}}%
\pgfpathlineto{\pgfqpoint{5.176746in}{2.539410in}}%
\pgfpathlineto{\pgfqpoint{5.169264in}{2.531664in}}%
\pgfpathlineto{\pgfqpoint{5.161778in}{2.524034in}}%
\pgfpathlineto{\pgfqpoint{5.154290in}{2.516513in}}%
\pgfpathlineto{\pgfqpoint{5.146798in}{2.509095in}}%
\pgfpathclose%
\pgfusepath{fill}%
\end{pgfscope}%
\begin{pgfscope}%
\pgfpathrectangle{\pgfqpoint{1.150000in}{0.150000in}}{\pgfqpoint{5.700000in}{5.700000in}}%
\pgfusepath{clip}%
\pgfsetbuttcap%
\pgfsetroundjoin%
\definecolor{currentfill}{rgb}{0.214298,0.355619,0.551184}%
\pgfsetfillcolor{currentfill}%
\pgfsetfillopacity{0.700000}%
\pgfsetlinewidth{0.000000pt}%
\definecolor{currentstroke}{rgb}{0.000000,0.000000,0.000000}%
\pgfsetstrokecolor{currentstroke}%
\pgfsetdash{}{0pt}%
\pgfpathmoveto{\pgfqpoint{5.946850in}{2.753757in}}%
\pgfpathlineto{\pgfqpoint{5.961073in}{2.753202in}}%
\pgfpathlineto{\pgfqpoint{5.975306in}{2.752712in}}%
\pgfpathlineto{\pgfqpoint{5.989551in}{2.752287in}}%
\pgfpathlineto{\pgfqpoint{6.003805in}{2.751927in}}%
\pgfpathlineto{\pgfqpoint{6.011035in}{2.760706in}}%
\pgfpathlineto{\pgfqpoint{6.018270in}{2.769764in}}%
\pgfpathlineto{\pgfqpoint{6.025510in}{2.779110in}}%
\pgfpathlineto{\pgfqpoint{6.032757in}{2.788753in}}%
\pgfpathlineto{\pgfqpoint{6.018532in}{2.789602in}}%
\pgfpathlineto{\pgfqpoint{6.004317in}{2.790516in}}%
\pgfpathlineto{\pgfqpoint{5.990112in}{2.791495in}}%
\pgfpathlineto{\pgfqpoint{5.975918in}{2.792538in}}%
\pgfpathlineto{\pgfqpoint{5.968642in}{2.782399in}}%
\pgfpathlineto{\pgfqpoint{5.961372in}{2.772562in}}%
\pgfpathlineto{\pgfqpoint{5.954108in}{2.763017in}}%
\pgfpathlineto{\pgfqpoint{5.946850in}{2.753757in}}%
\pgfpathclose%
\pgfusepath{fill}%
\end{pgfscope}%
\begin{pgfscope}%
\pgfpathrectangle{\pgfqpoint{1.150000in}{0.150000in}}{\pgfqpoint{5.700000in}{5.700000in}}%
\pgfusepath{clip}%
\pgfsetbuttcap%
\pgfsetroundjoin%
\definecolor{currentfill}{rgb}{0.280255,0.165693,0.476498}%
\pgfsetfillcolor{currentfill}%
\pgfsetfillopacity{0.700000}%
\pgfsetlinewidth{0.000000pt}%
\definecolor{currentstroke}{rgb}{0.000000,0.000000,0.000000}%
\pgfsetstrokecolor{currentstroke}%
\pgfsetdash{}{0pt}%
\pgfpathmoveto{\pgfqpoint{2.394277in}{2.360666in}}%
\pgfpathlineto{\pgfqpoint{2.407826in}{2.349264in}}%
\pgfpathlineto{\pgfqpoint{2.421374in}{2.337997in}}%
\pgfpathlineto{\pgfqpoint{2.434920in}{2.326863in}}%
\pgfpathlineto{\pgfqpoint{2.448464in}{2.315861in}}%
\pgfpathlineto{\pgfqpoint{2.457027in}{2.321926in}}%
\pgfpathlineto{\pgfqpoint{2.465579in}{2.328102in}}%
\pgfpathlineto{\pgfqpoint{2.474121in}{2.334388in}}%
\pgfpathlineto{\pgfqpoint{2.482653in}{2.340782in}}%
\pgfpathlineto{\pgfqpoint{2.469131in}{2.351635in}}%
\pgfpathlineto{\pgfqpoint{2.455608in}{2.362620in}}%
\pgfpathlineto{\pgfqpoint{2.442084in}{2.373738in}}%
\pgfpathlineto{\pgfqpoint{2.428557in}{2.384990in}}%
\pgfpathlineto{\pgfqpoint{2.420003in}{2.378737in}}%
\pgfpathlineto{\pgfqpoint{2.411438in}{2.372598in}}%
\pgfpathlineto{\pgfqpoint{2.402863in}{2.366574in}}%
\pgfpathlineto{\pgfqpoint{2.394277in}{2.360666in}}%
\pgfpathclose%
\pgfusepath{fill}%
\end{pgfscope}%
\begin{pgfscope}%
\pgfpathrectangle{\pgfqpoint{1.150000in}{0.150000in}}{\pgfqpoint{5.700000in}{5.700000in}}%
\pgfusepath{clip}%
\pgfsetbuttcap%
\pgfsetroundjoin%
\definecolor{currentfill}{rgb}{0.275191,0.194905,0.496005}%
\pgfsetfillcolor{currentfill}%
\pgfsetfillopacity{0.700000}%
\pgfsetlinewidth{0.000000pt}%
\definecolor{currentstroke}{rgb}{0.000000,0.000000,0.000000}%
\pgfsetstrokecolor{currentstroke}%
\pgfsetdash{}{0pt}%
\pgfpathmoveto{\pgfqpoint{4.746865in}{2.393259in}}%
\pgfpathlineto{\pgfqpoint{4.760749in}{2.392931in}}%
\pgfpathlineto{\pgfqpoint{4.774642in}{2.392675in}}%
\pgfpathlineto{\pgfqpoint{4.788544in}{2.392489in}}%
\pgfpathlineto{\pgfqpoint{4.802455in}{2.392375in}}%
\pgfpathlineto{\pgfqpoint{4.810086in}{2.399811in}}%
\pgfpathlineto{\pgfqpoint{4.817713in}{2.407289in}}%
\pgfpathlineto{\pgfqpoint{4.825334in}{2.414814in}}%
\pgfpathlineto{\pgfqpoint{4.832950in}{2.422390in}}%
\pgfpathlineto{\pgfqpoint{4.819055in}{2.422749in}}%
\pgfpathlineto{\pgfqpoint{4.805168in}{2.423180in}}%
\pgfpathlineto{\pgfqpoint{4.791290in}{2.423682in}}%
\pgfpathlineto{\pgfqpoint{4.777422in}{2.424254in}}%
\pgfpathlineto{\pgfqpoint{4.769790in}{2.416426in}}%
\pgfpathlineto{\pgfqpoint{4.762153in}{2.408654in}}%
\pgfpathlineto{\pgfqpoint{4.754512in}{2.400933in}}%
\pgfpathlineto{\pgfqpoint{4.746865in}{2.393259in}}%
\pgfpathclose%
\pgfusepath{fill}%
\end{pgfscope}%
\begin{pgfscope}%
\pgfpathrectangle{\pgfqpoint{1.150000in}{0.150000in}}{\pgfqpoint{5.700000in}{5.700000in}}%
\pgfusepath{clip}%
\pgfsetbuttcap%
\pgfsetroundjoin%
\definecolor{currentfill}{rgb}{0.282656,0.100196,0.422160}%
\pgfsetfillcolor{currentfill}%
\pgfsetfillopacity{0.700000}%
\pgfsetlinewidth{0.000000pt}%
\definecolor{currentstroke}{rgb}{0.000000,0.000000,0.000000}%
\pgfsetstrokecolor{currentstroke}%
\pgfsetdash{}{0pt}%
\pgfpathmoveto{\pgfqpoint{4.033082in}{2.201494in}}%
\pgfpathlineto{\pgfqpoint{4.046758in}{2.199628in}}%
\pgfpathlineto{\pgfqpoint{4.060442in}{2.197841in}}%
\pgfpathlineto{\pgfqpoint{4.074133in}{2.196131in}}%
\pgfpathlineto{\pgfqpoint{4.087831in}{2.194499in}}%
\pgfpathlineto{\pgfqpoint{4.095734in}{2.203081in}}%
\pgfpathlineto{\pgfqpoint{4.103630in}{2.211660in}}%
\pgfpathlineto{\pgfqpoint{4.111521in}{2.220238in}}%
\pgfpathlineto{\pgfqpoint{4.119407in}{2.228818in}}%
\pgfpathlineto{\pgfqpoint{4.105720in}{2.230551in}}%
\pgfpathlineto{\pgfqpoint{4.092040in}{2.232362in}}%
\pgfpathlineto{\pgfqpoint{4.078368in}{2.234251in}}%
\pgfpathlineto{\pgfqpoint{4.064702in}{2.236217in}}%
\pgfpathlineto{\pgfqpoint{4.056806in}{2.227529in}}%
\pgfpathlineto{\pgfqpoint{4.048903in}{2.218848in}}%
\pgfpathlineto{\pgfqpoint{4.040996in}{2.210170in}}%
\pgfpathlineto{\pgfqpoint{4.033082in}{2.201494in}}%
\pgfpathclose%
\pgfusepath{fill}%
\end{pgfscope}%
\begin{pgfscope}%
\pgfpathrectangle{\pgfqpoint{1.150000in}{0.150000in}}{\pgfqpoint{5.700000in}{5.700000in}}%
\pgfusepath{clip}%
\pgfsetbuttcap%
\pgfsetroundjoin%
\definecolor{currentfill}{rgb}{0.243113,0.292092,0.538516}%
\pgfsetfillcolor{currentfill}%
\pgfsetfillopacity{0.700000}%
\pgfsetlinewidth{0.000000pt}%
\definecolor{currentstroke}{rgb}{0.000000,0.000000,0.000000}%
\pgfsetstrokecolor{currentstroke}%
\pgfsetdash{}{0pt}%
\pgfpathmoveto{\pgfqpoint{5.460814in}{2.597217in}}%
\pgfpathlineto{\pgfqpoint{5.474916in}{2.597224in}}%
\pgfpathlineto{\pgfqpoint{5.489029in}{2.597298in}}%
\pgfpathlineto{\pgfqpoint{5.503152in}{2.597440in}}%
\pgfpathlineto{\pgfqpoint{5.517285in}{2.597648in}}%
\pgfpathlineto{\pgfqpoint{5.524642in}{2.604812in}}%
\pgfpathlineto{\pgfqpoint{5.531997in}{2.612131in}}%
\pgfpathlineto{\pgfqpoint{5.539351in}{2.619612in}}%
\pgfpathlineto{\pgfqpoint{5.546705in}{2.627261in}}%
\pgfpathlineto{\pgfqpoint{5.532595in}{2.627442in}}%
\pgfpathlineto{\pgfqpoint{5.518495in}{2.627689in}}%
\pgfpathlineto{\pgfqpoint{5.504405in}{2.628003in}}%
\pgfpathlineto{\pgfqpoint{5.490325in}{2.628384in}}%
\pgfpathlineto{\pgfqpoint{5.482949in}{2.620339in}}%
\pgfpathlineto{\pgfqpoint{5.475571in}{2.612468in}}%
\pgfpathlineto{\pgfqpoint{5.468193in}{2.604763in}}%
\pgfpathlineto{\pgfqpoint{5.460814in}{2.597217in}}%
\pgfpathclose%
\pgfusepath{fill}%
\end{pgfscope}%
\begin{pgfscope}%
\pgfpathrectangle{\pgfqpoint{1.150000in}{0.150000in}}{\pgfqpoint{5.700000in}{5.700000in}}%
\pgfusepath{clip}%
\pgfsetbuttcap%
\pgfsetroundjoin%
\definecolor{currentfill}{rgb}{0.282623,0.140926,0.457517}%
\pgfsetfillcolor{currentfill}%
\pgfsetfillopacity{0.700000}%
\pgfsetlinewidth{0.000000pt}%
\definecolor{currentstroke}{rgb}{0.000000,0.000000,0.000000}%
\pgfsetstrokecolor{currentstroke}%
\pgfsetdash{}{0pt}%
\pgfpathmoveto{\pgfqpoint{4.346854in}{2.279785in}}%
\pgfpathlineto{\pgfqpoint{4.360619in}{2.278776in}}%
\pgfpathlineto{\pgfqpoint{4.374392in}{2.277841in}}%
\pgfpathlineto{\pgfqpoint{4.388173in}{2.276980in}}%
\pgfpathlineto{\pgfqpoint{4.401962in}{2.276194in}}%
\pgfpathlineto{\pgfqpoint{4.409750in}{2.284276in}}%
\pgfpathlineto{\pgfqpoint{4.417532in}{2.292366in}}%
\pgfpathlineto{\pgfqpoint{4.425309in}{2.300466in}}%
\pgfpathlineto{\pgfqpoint{4.433080in}{2.308581in}}%
\pgfpathlineto{\pgfqpoint{4.419303in}{2.309530in}}%
\pgfpathlineto{\pgfqpoint{4.405534in}{2.310553in}}%
\pgfpathlineto{\pgfqpoint{4.391773in}{2.311651in}}%
\pgfpathlineto{\pgfqpoint{4.378021in}{2.312823in}}%
\pgfpathlineto{\pgfqpoint{4.370237in}{2.304539in}}%
\pgfpathlineto{\pgfqpoint{4.362448in}{2.296273in}}%
\pgfpathlineto{\pgfqpoint{4.354654in}{2.288023in}}%
\pgfpathlineto{\pgfqpoint{4.346854in}{2.279785in}}%
\pgfpathclose%
\pgfusepath{fill}%
\end{pgfscope}%
\begin{pgfscope}%
\pgfpathrectangle{\pgfqpoint{1.150000in}{0.150000in}}{\pgfqpoint{5.700000in}{5.700000in}}%
\pgfusepath{clip}%
\pgfsetbuttcap%
\pgfsetroundjoin%
\definecolor{currentfill}{rgb}{0.282910,0.105393,0.426902}%
\pgfsetfillcolor{currentfill}%
\pgfsetfillopacity{0.700000}%
\pgfsetlinewidth{0.000000pt}%
\definecolor{currentstroke}{rgb}{0.000000,0.000000,0.000000}%
\pgfsetstrokecolor{currentstroke}%
\pgfsetdash{}{0pt}%
\pgfpathmoveto{\pgfqpoint{2.644834in}{2.220430in}}%
\pgfpathlineto{\pgfqpoint{2.658347in}{2.211193in}}%
\pgfpathlineto{\pgfqpoint{2.671859in}{2.202073in}}%
\pgfpathlineto{\pgfqpoint{2.685373in}{2.193070in}}%
\pgfpathlineto{\pgfqpoint{2.698886in}{2.184182in}}%
\pgfpathlineto{\pgfqpoint{2.707325in}{2.191222in}}%
\pgfpathlineto{\pgfqpoint{2.715756in}{2.198345in}}%
\pgfpathlineto{\pgfqpoint{2.724178in}{2.205550in}}%
\pgfpathlineto{\pgfqpoint{2.732591in}{2.212835in}}%
\pgfpathlineto{\pgfqpoint{2.719096in}{2.221597in}}%
\pgfpathlineto{\pgfqpoint{2.705603in}{2.230473in}}%
\pgfpathlineto{\pgfqpoint{2.692109in}{2.239467in}}%
\pgfpathlineto{\pgfqpoint{2.678616in}{2.248577in}}%
\pgfpathlineto{\pgfqpoint{2.670184in}{2.241410in}}%
\pgfpathlineto{\pgfqpoint{2.661743in}{2.234329in}}%
\pgfpathlineto{\pgfqpoint{2.653293in}{2.227335in}}%
\pgfpathlineto{\pgfqpoint{2.644834in}{2.220430in}}%
\pgfpathclose%
\pgfusepath{fill}%
\end{pgfscope}%
\begin{pgfscope}%
\pgfpathrectangle{\pgfqpoint{1.150000in}{0.150000in}}{\pgfqpoint{5.700000in}{5.700000in}}%
\pgfusepath{clip}%
\pgfsetbuttcap%
\pgfsetroundjoin%
\definecolor{currentfill}{rgb}{0.277018,0.050344,0.375715}%
\pgfsetfillcolor{currentfill}%
\pgfsetfillopacity{0.700000}%
\pgfsetlinewidth{0.000000pt}%
\definecolor{currentstroke}{rgb}{0.000000,0.000000,0.000000}%
\pgfsetstrokecolor{currentstroke}%
\pgfsetdash{}{0pt}%
\pgfpathmoveto{\pgfqpoint{3.491808in}{2.105416in}}%
\pgfpathlineto{\pgfqpoint{3.505365in}{2.101442in}}%
\pgfpathlineto{\pgfqpoint{3.518928in}{2.097555in}}%
\pgfpathlineto{\pgfqpoint{3.532496in}{2.093755in}}%
\pgfpathlineto{\pgfqpoint{3.546069in}{2.090041in}}%
\pgfpathlineto{\pgfqpoint{3.554163in}{2.098940in}}%
\pgfpathlineto{\pgfqpoint{3.562251in}{2.107846in}}%
\pgfpathlineto{\pgfqpoint{3.570333in}{2.116760in}}%
\pgfpathlineto{\pgfqpoint{3.578410in}{2.125684in}}%
\pgfpathlineto{\pgfqpoint{3.564848in}{2.129396in}}%
\pgfpathlineto{\pgfqpoint{3.551292in}{2.133195in}}%
\pgfpathlineto{\pgfqpoint{3.537740in}{2.137080in}}%
\pgfpathlineto{\pgfqpoint{3.524194in}{2.141052in}}%
\pgfpathlineto{\pgfqpoint{3.516107in}{2.132123in}}%
\pgfpathlineto{\pgfqpoint{3.508013in}{2.123208in}}%
\pgfpathlineto{\pgfqpoint{3.499913in}{2.114305in}}%
\pgfpathlineto{\pgfqpoint{3.491808in}{2.105416in}}%
\pgfpathclose%
\pgfusepath{fill}%
\end{pgfscope}%
\begin{pgfscope}%
\pgfpathrectangle{\pgfqpoint{1.150000in}{0.150000in}}{\pgfqpoint{5.700000in}{5.700000in}}%
\pgfusepath{clip}%
\pgfsetbuttcap%
\pgfsetroundjoin%
\definecolor{currentfill}{rgb}{0.220057,0.343307,0.549413}%
\pgfsetfillcolor{currentfill}%
\pgfsetfillopacity{0.700000}%
\pgfsetlinewidth{0.000000pt}%
\definecolor{currentstroke}{rgb}{0.000000,0.000000,0.000000}%
\pgfsetstrokecolor{currentstroke}%
\pgfsetdash{}{0pt}%
\pgfpathmoveto{\pgfqpoint{5.860961in}{2.720383in}}%
\pgfpathlineto{\pgfqpoint{5.875170in}{2.720037in}}%
\pgfpathlineto{\pgfqpoint{5.889390in}{2.719756in}}%
\pgfpathlineto{\pgfqpoint{5.903620in}{2.719541in}}%
\pgfpathlineto{\pgfqpoint{5.917861in}{2.719391in}}%
\pgfpathlineto{\pgfqpoint{5.925102in}{2.727598in}}%
\pgfpathlineto{\pgfqpoint{5.932347in}{2.736056in}}%
\pgfpathlineto{\pgfqpoint{5.939596in}{2.744773in}}%
\pgfpathlineto{\pgfqpoint{5.946850in}{2.753757in}}%
\pgfpathlineto{\pgfqpoint{5.932637in}{2.754377in}}%
\pgfpathlineto{\pgfqpoint{5.918435in}{2.755061in}}%
\pgfpathlineto{\pgfqpoint{5.904243in}{2.755811in}}%
\pgfpathlineto{\pgfqpoint{5.890062in}{2.756626in}}%
\pgfpathlineto{\pgfqpoint{5.882780in}{2.747165in}}%
\pgfpathlineto{\pgfqpoint{5.875503in}{2.737977in}}%
\pgfpathlineto{\pgfqpoint{5.868230in}{2.729052in}}%
\pgfpathlineto{\pgfqpoint{5.860961in}{2.720383in}}%
\pgfpathclose%
\pgfusepath{fill}%
\end{pgfscope}%
\begin{pgfscope}%
\pgfpathrectangle{\pgfqpoint{1.150000in}{0.150000in}}{\pgfqpoint{5.700000in}{5.700000in}}%
\pgfusepath{clip}%
\pgfsetbuttcap%
\pgfsetroundjoin%
\definecolor{currentfill}{rgb}{0.263663,0.237631,0.518762}%
\pgfsetfillcolor{currentfill}%
\pgfsetfillopacity{0.700000}%
\pgfsetlinewidth{0.000000pt}%
\definecolor{currentstroke}{rgb}{0.000000,0.000000,0.000000}%
\pgfsetstrokecolor{currentstroke}%
\pgfsetdash{}{0pt}%
\pgfpathmoveto{\pgfqpoint{5.060796in}{2.479947in}}%
\pgfpathlineto{\pgfqpoint{5.074780in}{2.479944in}}%
\pgfpathlineto{\pgfqpoint{5.088775in}{2.480010in}}%
\pgfpathlineto{\pgfqpoint{5.102779in}{2.480146in}}%
\pgfpathlineto{\pgfqpoint{5.116792in}{2.480350in}}%
\pgfpathlineto{\pgfqpoint{5.124300in}{2.487409in}}%
\pgfpathlineto{\pgfqpoint{5.131803in}{2.494549in}}%
\pgfpathlineto{\pgfqpoint{5.139302in}{2.501776in}}%
\pgfpathlineto{\pgfqpoint{5.146798in}{2.509095in}}%
\pgfpathlineto{\pgfqpoint{5.132803in}{2.509198in}}%
\pgfpathlineto{\pgfqpoint{5.118817in}{2.509370in}}%
\pgfpathlineto{\pgfqpoint{5.104841in}{2.509611in}}%
\pgfpathlineto{\pgfqpoint{5.090875in}{2.509920in}}%
\pgfpathlineto{\pgfqpoint{5.083361in}{2.502287in}}%
\pgfpathlineto{\pgfqpoint{5.075843in}{2.494750in}}%
\pgfpathlineto{\pgfqpoint{5.068321in}{2.487306in}}%
\pgfpathlineto{\pgfqpoint{5.060796in}{2.479947in}}%
\pgfpathclose%
\pgfusepath{fill}%
\end{pgfscope}%
\begin{pgfscope}%
\pgfpathrectangle{\pgfqpoint{1.150000in}{0.150000in}}{\pgfqpoint{5.700000in}{5.700000in}}%
\pgfusepath{clip}%
\pgfsetbuttcap%
\pgfsetroundjoin%
\definecolor{currentfill}{rgb}{0.279566,0.067836,0.391917}%
\pgfsetfillcolor{currentfill}%
\pgfsetfillopacity{0.700000}%
\pgfsetlinewidth{0.000000pt}%
\definecolor{currentstroke}{rgb}{0.000000,0.000000,0.000000}%
\pgfsetstrokecolor{currentstroke}%
\pgfsetdash{}{0pt}%
\pgfpathmoveto{\pgfqpoint{3.719264in}{2.134686in}}%
\pgfpathlineto{\pgfqpoint{3.732867in}{2.131714in}}%
\pgfpathlineto{\pgfqpoint{3.746475in}{2.128824in}}%
\pgfpathlineto{\pgfqpoint{3.760090in}{2.126017in}}%
\pgfpathlineto{\pgfqpoint{3.773711in}{2.123292in}}%
\pgfpathlineto{\pgfqpoint{3.781725in}{2.132175in}}%
\pgfpathlineto{\pgfqpoint{3.789734in}{2.141056in}}%
\pgfpathlineto{\pgfqpoint{3.797736in}{2.149937in}}%
\pgfpathlineto{\pgfqpoint{3.805733in}{2.158818in}}%
\pgfpathlineto{\pgfqpoint{3.792123in}{2.161583in}}%
\pgfpathlineto{\pgfqpoint{3.778520in}{2.164429in}}%
\pgfpathlineto{\pgfqpoint{3.764922in}{2.167358in}}%
\pgfpathlineto{\pgfqpoint{3.751330in}{2.170370in}}%
\pgfpathlineto{\pgfqpoint{3.743322in}{2.161441in}}%
\pgfpathlineto{\pgfqpoint{3.735309in}{2.152518in}}%
\pgfpathlineto{\pgfqpoint{3.727289in}{2.143600in}}%
\pgfpathlineto{\pgfqpoint{3.719264in}{2.134686in}}%
\pgfpathclose%
\pgfusepath{fill}%
\end{pgfscope}%
\begin{pgfscope}%
\pgfpathrectangle{\pgfqpoint{1.150000in}{0.150000in}}{\pgfqpoint{5.700000in}{5.700000in}}%
\pgfusepath{clip}%
\pgfsetbuttcap%
\pgfsetroundjoin%
\definecolor{currentfill}{rgb}{0.277018,0.050344,0.375715}%
\pgfsetfillcolor{currentfill}%
\pgfsetfillopacity{0.700000}%
\pgfsetlinewidth{0.000000pt}%
\definecolor{currentstroke}{rgb}{0.000000,0.000000,0.000000}%
\pgfsetstrokecolor{currentstroke}%
\pgfsetdash{}{0pt}%
\pgfpathmoveto{\pgfqpoint{2.981954in}{2.119282in}}%
\pgfpathlineto{\pgfqpoint{2.995457in}{2.112465in}}%
\pgfpathlineto{\pgfqpoint{3.008963in}{2.105750in}}%
\pgfpathlineto{\pgfqpoint{3.022472in}{2.099135in}}%
\pgfpathlineto{\pgfqpoint{3.035983in}{2.092621in}}%
\pgfpathlineto{\pgfqpoint{3.044272in}{2.100750in}}%
\pgfpathlineto{\pgfqpoint{3.052554in}{2.108926in}}%
\pgfpathlineto{\pgfqpoint{3.060828in}{2.117150in}}%
\pgfpathlineto{\pgfqpoint{3.069096in}{2.125419in}}%
\pgfpathlineto{\pgfqpoint{3.055600in}{2.131850in}}%
\pgfpathlineto{\pgfqpoint{3.042106in}{2.138381in}}%
\pgfpathlineto{\pgfqpoint{3.028616in}{2.145012in}}%
\pgfpathlineto{\pgfqpoint{3.015128in}{2.151746in}}%
\pgfpathlineto{\pgfqpoint{3.006845in}{2.143552in}}%
\pgfpathlineto{\pgfqpoint{2.998555in}{2.135410in}}%
\pgfpathlineto{\pgfqpoint{2.990258in}{2.127320in}}%
\pgfpathlineto{\pgfqpoint{2.981954in}{2.119282in}}%
\pgfpathclose%
\pgfusepath{fill}%
\end{pgfscope}%
\begin{pgfscope}%
\pgfpathrectangle{\pgfqpoint{1.150000in}{0.150000in}}{\pgfqpoint{5.700000in}{5.700000in}}%
\pgfusepath{clip}%
\pgfsetbuttcap%
\pgfsetroundjoin%
\definecolor{currentfill}{rgb}{0.276022,0.044167,0.370164}%
\pgfsetfillcolor{currentfill}%
\pgfsetfillopacity{0.700000}%
\pgfsetlinewidth{0.000000pt}%
\definecolor{currentstroke}{rgb}{0.000000,0.000000,0.000000}%
\pgfsetstrokecolor{currentstroke}%
\pgfsetdash{}{0pt}%
\pgfpathmoveto{\pgfqpoint{3.123110in}{2.100688in}}%
\pgfpathlineto{\pgfqpoint{3.136622in}{2.094750in}}%
\pgfpathlineto{\pgfqpoint{3.150137in}{2.088909in}}%
\pgfpathlineto{\pgfqpoint{3.163655in}{2.083164in}}%
\pgfpathlineto{\pgfqpoint{3.177177in}{2.077515in}}%
\pgfpathlineto{\pgfqpoint{3.185409in}{2.085971in}}%
\pgfpathlineto{\pgfqpoint{3.193634in}{2.094462in}}%
\pgfpathlineto{\pgfqpoint{3.201852in}{2.102987in}}%
\pgfpathlineto{\pgfqpoint{3.210064in}{2.111545in}}%
\pgfpathlineto{\pgfqpoint{3.196556in}{2.117132in}}%
\pgfpathlineto{\pgfqpoint{3.183052in}{2.122814in}}%
\pgfpathlineto{\pgfqpoint{3.169551in}{2.128592in}}%
\pgfpathlineto{\pgfqpoint{3.156053in}{2.134467in}}%
\pgfpathlineto{\pgfqpoint{3.147828in}{2.125964in}}%
\pgfpathlineto{\pgfqpoint{3.139595in}{2.117500in}}%
\pgfpathlineto{\pgfqpoint{3.131356in}{2.109074in}}%
\pgfpathlineto{\pgfqpoint{3.123110in}{2.100688in}}%
\pgfpathclose%
\pgfusepath{fill}%
\end{pgfscope}%
\begin{pgfscope}%
\pgfpathrectangle{\pgfqpoint{1.150000in}{0.150000in}}{\pgfqpoint{5.700000in}{5.700000in}}%
\pgfusepath{clip}%
\pgfsetbuttcap%
\pgfsetroundjoin%
\definecolor{currentfill}{rgb}{0.277134,0.185228,0.489898}%
\pgfsetfillcolor{currentfill}%
\pgfsetfillopacity{0.700000}%
\pgfsetlinewidth{0.000000pt}%
\definecolor{currentstroke}{rgb}{0.000000,0.000000,0.000000}%
\pgfsetstrokecolor{currentstroke}%
\pgfsetdash{}{0pt}%
\pgfpathmoveto{\pgfqpoint{4.660724in}{2.364075in}}%
\pgfpathlineto{\pgfqpoint{4.674586in}{2.363686in}}%
\pgfpathlineto{\pgfqpoint{4.688457in}{2.363369in}}%
\pgfpathlineto{\pgfqpoint{4.702338in}{2.363123in}}%
\pgfpathlineto{\pgfqpoint{4.716227in}{2.362949in}}%
\pgfpathlineto{\pgfqpoint{4.723894in}{2.370477in}}%
\pgfpathlineto{\pgfqpoint{4.731557in}{2.378035in}}%
\pgfpathlineto{\pgfqpoint{4.739214in}{2.385628in}}%
\pgfpathlineto{\pgfqpoint{4.746865in}{2.393259in}}%
\pgfpathlineto{\pgfqpoint{4.732991in}{2.393658in}}%
\pgfpathlineto{\pgfqpoint{4.719125in}{2.394128in}}%
\pgfpathlineto{\pgfqpoint{4.705268in}{2.394670in}}%
\pgfpathlineto{\pgfqpoint{4.691420in}{2.395284in}}%
\pgfpathlineto{\pgfqpoint{4.683754in}{2.387421in}}%
\pgfpathlineto{\pgfqpoint{4.676082in}{2.379601in}}%
\pgfpathlineto{\pgfqpoint{4.668406in}{2.371820in}}%
\pgfpathlineto{\pgfqpoint{4.660724in}{2.364075in}}%
\pgfpathclose%
\pgfusepath{fill}%
\end{pgfscope}%
\begin{pgfscope}%
\pgfpathrectangle{\pgfqpoint{1.150000in}{0.150000in}}{\pgfqpoint{5.700000in}{5.700000in}}%
\pgfusepath{clip}%
\pgfsetbuttcap%
\pgfsetroundjoin%
\definecolor{currentfill}{rgb}{0.281887,0.150881,0.465405}%
\pgfsetfillcolor{currentfill}%
\pgfsetfillopacity{0.700000}%
\pgfsetlinewidth{0.000000pt}%
\definecolor{currentstroke}{rgb}{0.000000,0.000000,0.000000}%
\pgfsetstrokecolor{currentstroke}%
\pgfsetdash{}{0pt}%
\pgfpathmoveto{\pgfqpoint{2.448464in}{2.315861in}}%
\pgfpathlineto{\pgfqpoint{2.462006in}{2.304990in}}%
\pgfpathlineto{\pgfqpoint{2.475547in}{2.294248in}}%
\pgfpathlineto{\pgfqpoint{2.489087in}{2.283635in}}%
\pgfpathlineto{\pgfqpoint{2.502625in}{2.273150in}}%
\pgfpathlineto{\pgfqpoint{2.511165in}{2.279371in}}%
\pgfpathlineto{\pgfqpoint{2.519695in}{2.285699in}}%
\pgfpathlineto{\pgfqpoint{2.528215in}{2.292131in}}%
\pgfpathlineto{\pgfqpoint{2.536725in}{2.298666in}}%
\pgfpathlineto{\pgfqpoint{2.523209in}{2.309003in}}%
\pgfpathlineto{\pgfqpoint{2.509691in}{2.319467in}}%
\pgfpathlineto{\pgfqpoint{2.496173in}{2.330060in}}%
\pgfpathlineto{\pgfqpoint{2.482653in}{2.340782in}}%
\pgfpathlineto{\pgfqpoint{2.474121in}{2.334388in}}%
\pgfpathlineto{\pgfqpoint{2.465579in}{2.328102in}}%
\pgfpathlineto{\pgfqpoint{2.457027in}{2.321926in}}%
\pgfpathlineto{\pgfqpoint{2.448464in}{2.315861in}}%
\pgfpathclose%
\pgfusepath{fill}%
\end{pgfscope}%
\begin{pgfscope}%
\pgfpathrectangle{\pgfqpoint{1.150000in}{0.150000in}}{\pgfqpoint{5.700000in}{5.700000in}}%
\pgfusepath{clip}%
\pgfsetbuttcap%
\pgfsetroundjoin%
\definecolor{currentfill}{rgb}{0.279566,0.067836,0.391917}%
\pgfsetfillcolor{currentfill}%
\pgfsetfillopacity{0.700000}%
\pgfsetlinewidth{0.000000pt}%
\definecolor{currentstroke}{rgb}{0.000000,0.000000,0.000000}%
\pgfsetstrokecolor{currentstroke}%
\pgfsetdash{}{0pt}%
\pgfpathmoveto{\pgfqpoint{2.840578in}{2.146798in}}%
\pgfpathlineto{\pgfqpoint{2.854082in}{2.139039in}}%
\pgfpathlineto{\pgfqpoint{2.867588in}{2.131386in}}%
\pgfpathlineto{\pgfqpoint{2.881095in}{2.123840in}}%
\pgfpathlineto{\pgfqpoint{2.894604in}{2.116400in}}%
\pgfpathlineto{\pgfqpoint{2.902956in}{2.124103in}}%
\pgfpathlineto{\pgfqpoint{2.911299in}{2.131868in}}%
\pgfpathlineto{\pgfqpoint{2.919635in}{2.139695in}}%
\pgfpathlineto{\pgfqpoint{2.927963in}{2.147581in}}%
\pgfpathlineto{\pgfqpoint{2.914470in}{2.154917in}}%
\pgfpathlineto{\pgfqpoint{2.900980in}{2.162358in}}%
\pgfpathlineto{\pgfqpoint{2.887491in}{2.169906in}}%
\pgfpathlineto{\pgfqpoint{2.874004in}{2.177561in}}%
\pgfpathlineto{\pgfqpoint{2.865659in}{2.169772in}}%
\pgfpathlineto{\pgfqpoint{2.857307in}{2.162047in}}%
\pgfpathlineto{\pgfqpoint{2.848946in}{2.154389in}}%
\pgfpathlineto{\pgfqpoint{2.840578in}{2.146798in}}%
\pgfpathclose%
\pgfusepath{fill}%
\end{pgfscope}%
\begin{pgfscope}%
\pgfpathrectangle{\pgfqpoint{1.150000in}{0.150000in}}{\pgfqpoint{5.700000in}{5.700000in}}%
\pgfusepath{clip}%
\pgfsetbuttcap%
\pgfsetroundjoin%
\definecolor{currentfill}{rgb}{0.274952,0.037752,0.364543}%
\pgfsetfillcolor{currentfill}%
\pgfsetfillopacity{0.700000}%
\pgfsetlinewidth{0.000000pt}%
\definecolor{currentstroke}{rgb}{0.000000,0.000000,0.000000}%
\pgfsetstrokecolor{currentstroke}%
\pgfsetdash{}{0pt}%
\pgfpathmoveto{\pgfqpoint{3.264133in}{2.090144in}}%
\pgfpathlineto{\pgfqpoint{3.277660in}{2.085028in}}%
\pgfpathlineto{\pgfqpoint{3.291192in}{2.080004in}}%
\pgfpathlineto{\pgfqpoint{3.304727in}{2.075071in}}%
\pgfpathlineto{\pgfqpoint{3.318267in}{2.070231in}}%
\pgfpathlineto{\pgfqpoint{3.326446in}{2.078922in}}%
\pgfpathlineto{\pgfqpoint{3.334618in}{2.087637in}}%
\pgfpathlineto{\pgfqpoint{3.342785in}{2.096373in}}%
\pgfpathlineto{\pgfqpoint{3.350945in}{2.105132in}}%
\pgfpathlineto{\pgfqpoint{3.337418in}{2.109931in}}%
\pgfpathlineto{\pgfqpoint{3.323895in}{2.114821in}}%
\pgfpathlineto{\pgfqpoint{3.310377in}{2.119802in}}%
\pgfpathlineto{\pgfqpoint{3.296863in}{2.124876in}}%
\pgfpathlineto{\pgfqpoint{3.288690in}{2.116152in}}%
\pgfpathlineto{\pgfqpoint{3.280511in}{2.107455in}}%
\pgfpathlineto{\pgfqpoint{3.272325in}{2.098786in}}%
\pgfpathlineto{\pgfqpoint{3.264133in}{2.090144in}}%
\pgfpathclose%
\pgfusepath{fill}%
\end{pgfscope}%
\begin{pgfscope}%
\pgfpathrectangle{\pgfqpoint{1.150000in}{0.150000in}}{\pgfqpoint{5.700000in}{5.700000in}}%
\pgfusepath{clip}%
\pgfsetbuttcap%
\pgfsetroundjoin%
\definecolor{currentfill}{rgb}{0.248629,0.278775,0.534556}%
\pgfsetfillcolor{currentfill}%
\pgfsetfillopacity{0.700000}%
\pgfsetlinewidth{0.000000pt}%
\definecolor{currentstroke}{rgb}{0.000000,0.000000,0.000000}%
\pgfsetstrokecolor{currentstroke}%
\pgfsetdash{}{0pt}%
\pgfpathmoveto{\pgfqpoint{5.374883in}{2.567665in}}%
\pgfpathlineto{\pgfqpoint{5.388967in}{2.567772in}}%
\pgfpathlineto{\pgfqpoint{5.403061in}{2.567946in}}%
\pgfpathlineto{\pgfqpoint{5.417165in}{2.568188in}}%
\pgfpathlineto{\pgfqpoint{5.431280in}{2.568497in}}%
\pgfpathlineto{\pgfqpoint{5.438666in}{2.575470in}}%
\pgfpathlineto{\pgfqpoint{5.446051in}{2.582578in}}%
\pgfpathlineto{\pgfqpoint{5.453433in}{2.589824in}}%
\pgfpathlineto{\pgfqpoint{5.460814in}{2.597217in}}%
\pgfpathlineto{\pgfqpoint{5.446722in}{2.597277in}}%
\pgfpathlineto{\pgfqpoint{5.432639in}{2.597404in}}%
\pgfpathlineto{\pgfqpoint{5.418567in}{2.597598in}}%
\pgfpathlineto{\pgfqpoint{5.404505in}{2.597860in}}%
\pgfpathlineto{\pgfqpoint{5.397102in}{2.590091in}}%
\pgfpathlineto{\pgfqpoint{5.389698in}{2.582474in}}%
\pgfpathlineto{\pgfqpoint{5.382291in}{2.575000in}}%
\pgfpathlineto{\pgfqpoint{5.374883in}{2.567665in}}%
\pgfpathclose%
\pgfusepath{fill}%
\end{pgfscope}%
\begin{pgfscope}%
\pgfpathrectangle{\pgfqpoint{1.150000in}{0.150000in}}{\pgfqpoint{5.700000in}{5.700000in}}%
\pgfusepath{clip}%
\pgfsetbuttcap%
\pgfsetroundjoin%
\definecolor{currentfill}{rgb}{0.225863,0.330805,0.547314}%
\pgfsetfillcolor{currentfill}%
\pgfsetfillopacity{0.700000}%
\pgfsetlinewidth{0.000000pt}%
\definecolor{currentstroke}{rgb}{0.000000,0.000000,0.000000}%
\pgfsetstrokecolor{currentstroke}%
\pgfsetdash{}{0pt}%
\pgfpathmoveto{\pgfqpoint{5.775074in}{2.688340in}}%
\pgfpathlineto{\pgfqpoint{5.789268in}{2.688182in}}%
\pgfpathlineto{\pgfqpoint{5.803473in}{2.688089in}}%
\pgfpathlineto{\pgfqpoint{5.817688in}{2.688062in}}%
\pgfpathlineto{\pgfqpoint{5.831914in}{2.688100in}}%
\pgfpathlineto{\pgfqpoint{5.839172in}{2.695828in}}%
\pgfpathlineto{\pgfqpoint{5.846432in}{2.703779in}}%
\pgfpathlineto{\pgfqpoint{5.853695in}{2.711961in}}%
\pgfpathlineto{\pgfqpoint{5.860961in}{2.720383in}}%
\pgfpathlineto{\pgfqpoint{5.846762in}{2.720794in}}%
\pgfpathlineto{\pgfqpoint{5.832574in}{2.721271in}}%
\pgfpathlineto{\pgfqpoint{5.818396in}{2.721813in}}%
\pgfpathlineto{\pgfqpoint{5.804228in}{2.722420in}}%
\pgfpathlineto{\pgfqpoint{5.796936in}{2.713542in}}%
\pgfpathlineto{\pgfqpoint{5.789646in}{2.704908in}}%
\pgfpathlineto{\pgfqpoint{5.782359in}{2.696510in}}%
\pgfpathlineto{\pgfqpoint{5.775074in}{2.688340in}}%
\pgfpathclose%
\pgfusepath{fill}%
\end{pgfscope}%
\begin{pgfscope}%
\pgfpathrectangle{\pgfqpoint{1.150000in}{0.150000in}}{\pgfqpoint{5.700000in}{5.700000in}}%
\pgfusepath{clip}%
\pgfsetbuttcap%
\pgfsetroundjoin%
\definecolor{currentfill}{rgb}{0.281924,0.089666,0.412415}%
\pgfsetfillcolor{currentfill}%
\pgfsetfillopacity{0.700000}%
\pgfsetlinewidth{0.000000pt}%
\definecolor{currentstroke}{rgb}{0.000000,0.000000,0.000000}%
\pgfsetstrokecolor{currentstroke}%
\pgfsetdash{}{0pt}%
\pgfpathmoveto{\pgfqpoint{3.946694in}{2.174698in}}%
\pgfpathlineto{\pgfqpoint{3.960353in}{2.172598in}}%
\pgfpathlineto{\pgfqpoint{3.974018in}{2.170577in}}%
\pgfpathlineto{\pgfqpoint{3.987691in}{2.168635in}}%
\pgfpathlineto{\pgfqpoint{4.001372in}{2.166771in}}%
\pgfpathlineto{\pgfqpoint{4.009308in}{2.175459in}}%
\pgfpathlineto{\pgfqpoint{4.017238in}{2.184140in}}%
\pgfpathlineto{\pgfqpoint{4.025163in}{2.192818in}}%
\pgfpathlineto{\pgfqpoint{4.033082in}{2.201494in}}%
\pgfpathlineto{\pgfqpoint{4.019413in}{2.203438in}}%
\pgfpathlineto{\pgfqpoint{4.005751in}{2.205461in}}%
\pgfpathlineto{\pgfqpoint{3.992096in}{2.207562in}}%
\pgfpathlineto{\pgfqpoint{3.978448in}{2.209743in}}%
\pgfpathlineto{\pgfqpoint{3.970518in}{2.200979in}}%
\pgfpathlineto{\pgfqpoint{3.962582in}{2.192218in}}%
\pgfpathlineto{\pgfqpoint{3.954641in}{2.183458in}}%
\pgfpathlineto{\pgfqpoint{3.946694in}{2.174698in}}%
\pgfpathclose%
\pgfusepath{fill}%
\end{pgfscope}%
\begin{pgfscope}%
\pgfpathrectangle{\pgfqpoint{1.150000in}{0.150000in}}{\pgfqpoint{5.700000in}{5.700000in}}%
\pgfusepath{clip}%
\pgfsetbuttcap%
\pgfsetroundjoin%
\definecolor{currentfill}{rgb}{0.283072,0.130895,0.449241}%
\pgfsetfillcolor{currentfill}%
\pgfsetfillopacity{0.700000}%
\pgfsetlinewidth{0.000000pt}%
\definecolor{currentstroke}{rgb}{0.000000,0.000000,0.000000}%
\pgfsetstrokecolor{currentstroke}%
\pgfsetdash{}{0pt}%
\pgfpathmoveto{\pgfqpoint{4.260571in}{2.251117in}}%
\pgfpathlineto{\pgfqpoint{4.274315in}{2.249950in}}%
\pgfpathlineto{\pgfqpoint{4.288068in}{2.248858in}}%
\pgfpathlineto{\pgfqpoint{4.301829in}{2.247841in}}%
\pgfpathlineto{\pgfqpoint{4.315597in}{2.246899in}}%
\pgfpathlineto{\pgfqpoint{4.323420in}{2.255117in}}%
\pgfpathlineto{\pgfqpoint{4.331237in}{2.263335in}}%
\pgfpathlineto{\pgfqpoint{4.339048in}{2.271557in}}%
\pgfpathlineto{\pgfqpoint{4.346854in}{2.279785in}}%
\pgfpathlineto{\pgfqpoint{4.333097in}{2.280870in}}%
\pgfpathlineto{\pgfqpoint{4.319348in}{2.282029in}}%
\pgfpathlineto{\pgfqpoint{4.305608in}{2.283263in}}%
\pgfpathlineto{\pgfqpoint{4.291875in}{2.284572in}}%
\pgfpathlineto{\pgfqpoint{4.284057in}{2.276194in}}%
\pgfpathlineto{\pgfqpoint{4.276234in}{2.267828in}}%
\pgfpathlineto{\pgfqpoint{4.268405in}{2.259469in}}%
\pgfpathlineto{\pgfqpoint{4.260571in}{2.251117in}}%
\pgfpathclose%
\pgfusepath{fill}%
\end{pgfscope}%
\begin{pgfscope}%
\pgfpathrectangle{\pgfqpoint{1.150000in}{0.150000in}}{\pgfqpoint{5.700000in}{5.700000in}}%
\pgfusepath{clip}%
\pgfsetbuttcap%
\pgfsetroundjoin%
\definecolor{currentfill}{rgb}{0.266580,0.228262,0.514349}%
\pgfsetfillcolor{currentfill}%
\pgfsetfillopacity{0.700000}%
\pgfsetlinewidth{0.000000pt}%
\definecolor{currentstroke}{rgb}{0.000000,0.000000,0.000000}%
\pgfsetstrokecolor{currentstroke}%
\pgfsetdash{}{0pt}%
\pgfpathmoveto{\pgfqpoint{4.974739in}{2.450827in}}%
\pgfpathlineto{\pgfqpoint{4.988703in}{2.450833in}}%
\pgfpathlineto{\pgfqpoint{5.002676in}{2.450909in}}%
\pgfpathlineto{\pgfqpoint{5.016659in}{2.451055in}}%
\pgfpathlineto{\pgfqpoint{5.030651in}{2.451270in}}%
\pgfpathlineto{\pgfqpoint{5.038194in}{2.458336in}}%
\pgfpathlineto{\pgfqpoint{5.045732in}{2.465467in}}%
\pgfpathlineto{\pgfqpoint{5.053266in}{2.472669in}}%
\pgfpathlineto{\pgfqpoint{5.060796in}{2.479947in}}%
\pgfpathlineto{\pgfqpoint{5.046821in}{2.480019in}}%
\pgfpathlineto{\pgfqpoint{5.032856in}{2.480161in}}%
\pgfpathlineto{\pgfqpoint{5.018900in}{2.480371in}}%
\pgfpathlineto{\pgfqpoint{5.004954in}{2.480652in}}%
\pgfpathlineto{\pgfqpoint{4.997406in}{2.473080in}}%
\pgfpathlineto{\pgfqpoint{4.989855in}{2.465589in}}%
\pgfpathlineto{\pgfqpoint{4.982299in}{2.458173in}}%
\pgfpathlineto{\pgfqpoint{4.974739in}{2.450827in}}%
\pgfpathclose%
\pgfusepath{fill}%
\end{pgfscope}%
\begin{pgfscope}%
\pgfpathrectangle{\pgfqpoint{1.150000in}{0.150000in}}{\pgfqpoint{5.700000in}{5.700000in}}%
\pgfusepath{clip}%
\pgfsetbuttcap%
\pgfsetroundjoin%
\definecolor{currentfill}{rgb}{0.278826,0.175490,0.483397}%
\pgfsetfillcolor{currentfill}%
\pgfsetfillopacity{0.700000}%
\pgfsetlinewidth{0.000000pt}%
\definecolor{currentstroke}{rgb}{0.000000,0.000000,0.000000}%
\pgfsetstrokecolor{currentstroke}%
\pgfsetdash{}{0pt}%
\pgfpathmoveto{\pgfqpoint{4.574526in}{2.334825in}}%
\pgfpathlineto{\pgfqpoint{4.588367in}{2.334352in}}%
\pgfpathlineto{\pgfqpoint{4.602216in}{2.333951in}}%
\pgfpathlineto{\pgfqpoint{4.616075in}{2.333622in}}%
\pgfpathlineto{\pgfqpoint{4.629942in}{2.333366in}}%
\pgfpathlineto{\pgfqpoint{4.637646in}{2.341010in}}%
\pgfpathlineto{\pgfqpoint{4.645344in}{2.348674in}}%
\pgfpathlineto{\pgfqpoint{4.653037in}{2.356361in}}%
\pgfpathlineto{\pgfqpoint{4.660724in}{2.364075in}}%
\pgfpathlineto{\pgfqpoint{4.646870in}{2.364536in}}%
\pgfpathlineto{\pgfqpoint{4.633026in}{2.365069in}}%
\pgfpathlineto{\pgfqpoint{4.619190in}{2.365675in}}%
\pgfpathlineto{\pgfqpoint{4.605363in}{2.366352in}}%
\pgfpathlineto{\pgfqpoint{4.597662in}{2.358426in}}%
\pgfpathlineto{\pgfqpoint{4.589955in}{2.350532in}}%
\pgfpathlineto{\pgfqpoint{4.582243in}{2.342667in}}%
\pgfpathlineto{\pgfqpoint{4.574526in}{2.334825in}}%
\pgfpathclose%
\pgfusepath{fill}%
\end{pgfscope}%
\begin{pgfscope}%
\pgfpathrectangle{\pgfqpoint{1.150000in}{0.150000in}}{\pgfqpoint{5.700000in}{5.700000in}}%
\pgfusepath{clip}%
\pgfsetbuttcap%
\pgfsetroundjoin%
\definecolor{currentfill}{rgb}{0.277941,0.056324,0.381191}%
\pgfsetfillcolor{currentfill}%
\pgfsetfillopacity{0.700000}%
\pgfsetlinewidth{0.000000pt}%
\definecolor{currentstroke}{rgb}{0.000000,0.000000,0.000000}%
\pgfsetstrokecolor{currentstroke}%
\pgfsetdash{}{0pt}%
\pgfpathmoveto{\pgfqpoint{3.632712in}{2.111688in}}%
\pgfpathlineto{\pgfqpoint{3.646302in}{2.108401in}}%
\pgfpathlineto{\pgfqpoint{3.659898in}{2.105198in}}%
\pgfpathlineto{\pgfqpoint{3.673499in}{2.102079in}}%
\pgfpathlineto{\pgfqpoint{3.687107in}{2.099043in}}%
\pgfpathlineto{\pgfqpoint{3.695155in}{2.107953in}}%
\pgfpathlineto{\pgfqpoint{3.703197in}{2.116863in}}%
\pgfpathlineto{\pgfqpoint{3.711233in}{2.125774in}}%
\pgfpathlineto{\pgfqpoint{3.719264in}{2.134686in}}%
\pgfpathlineto{\pgfqpoint{3.705668in}{2.137741in}}%
\pgfpathlineto{\pgfqpoint{3.692078in}{2.140879in}}%
\pgfpathlineto{\pgfqpoint{3.678493in}{2.144101in}}%
\pgfpathlineto{\pgfqpoint{3.664914in}{2.147407in}}%
\pgfpathlineto{\pgfqpoint{3.656873in}{2.138468in}}%
\pgfpathlineto{\pgfqpoint{3.648825in}{2.129536in}}%
\pgfpathlineto{\pgfqpoint{3.640771in}{2.120610in}}%
\pgfpathlineto{\pgfqpoint{3.632712in}{2.111688in}}%
\pgfpathclose%
\pgfusepath{fill}%
\end{pgfscope}%
\begin{pgfscope}%
\pgfpathrectangle{\pgfqpoint{1.150000in}{0.150000in}}{\pgfqpoint{5.700000in}{5.700000in}}%
\pgfusepath{clip}%
\pgfsetbuttcap%
\pgfsetroundjoin%
\definecolor{currentfill}{rgb}{0.276022,0.044167,0.370164}%
\pgfsetfillcolor{currentfill}%
\pgfsetfillopacity{0.700000}%
\pgfsetlinewidth{0.000000pt}%
\definecolor{currentstroke}{rgb}{0.000000,0.000000,0.000000}%
\pgfsetstrokecolor{currentstroke}%
\pgfsetdash{}{0pt}%
\pgfpathmoveto{\pgfqpoint{3.405097in}{2.086844in}}%
\pgfpathlineto{\pgfqpoint{3.418647in}{2.082496in}}%
\pgfpathlineto{\pgfqpoint{3.432202in}{2.078237in}}%
\pgfpathlineto{\pgfqpoint{3.445761in}{2.074066in}}%
\pgfpathlineto{\pgfqpoint{3.459326in}{2.069983in}}%
\pgfpathlineto{\pgfqpoint{3.467455in}{2.078823in}}%
\pgfpathlineto{\pgfqpoint{3.475579in}{2.087675in}}%
\pgfpathlineto{\pgfqpoint{3.483696in}{2.096539in}}%
\pgfpathlineto{\pgfqpoint{3.491808in}{2.105416in}}%
\pgfpathlineto{\pgfqpoint{3.478255in}{2.109477in}}%
\pgfpathlineto{\pgfqpoint{3.464708in}{2.113627in}}%
\pgfpathlineto{\pgfqpoint{3.451165in}{2.117864in}}%
\pgfpathlineto{\pgfqpoint{3.437628in}{2.122190in}}%
\pgfpathlineto{\pgfqpoint{3.429504in}{2.113328in}}%
\pgfpathlineto{\pgfqpoint{3.421375in}{2.104483in}}%
\pgfpathlineto{\pgfqpoint{3.413239in}{2.095655in}}%
\pgfpathlineto{\pgfqpoint{3.405097in}{2.086844in}}%
\pgfpathclose%
\pgfusepath{fill}%
\end{pgfscope}%
\begin{pgfscope}%
\pgfpathrectangle{\pgfqpoint{1.150000in}{0.150000in}}{\pgfqpoint{5.700000in}{5.700000in}}%
\pgfusepath{clip}%
\pgfsetbuttcap%
\pgfsetroundjoin%
\definecolor{currentfill}{rgb}{0.281924,0.089666,0.412415}%
\pgfsetfillcolor{currentfill}%
\pgfsetfillopacity{0.700000}%
\pgfsetlinewidth{0.000000pt}%
\definecolor{currentstroke}{rgb}{0.000000,0.000000,0.000000}%
\pgfsetstrokecolor{currentstroke}%
\pgfsetdash{}{0pt}%
\pgfpathmoveto{\pgfqpoint{2.698886in}{2.184182in}}%
\pgfpathlineto{\pgfqpoint{2.712400in}{2.175410in}}%
\pgfpathlineto{\pgfqpoint{2.725915in}{2.166751in}}%
\pgfpathlineto{\pgfqpoint{2.739430in}{2.158205in}}%
\pgfpathlineto{\pgfqpoint{2.752946in}{2.149771in}}%
\pgfpathlineto{\pgfqpoint{2.761366in}{2.156943in}}%
\pgfpathlineto{\pgfqpoint{2.769778in}{2.164195in}}%
\pgfpathlineto{\pgfqpoint{2.778181in}{2.171523in}}%
\pgfpathlineto{\pgfqpoint{2.786576in}{2.178927in}}%
\pgfpathlineto{\pgfqpoint{2.773078in}{2.187235in}}%
\pgfpathlineto{\pgfqpoint{2.759581in}{2.195655in}}%
\pgfpathlineto{\pgfqpoint{2.746086in}{2.204188in}}%
\pgfpathlineto{\pgfqpoint{2.732591in}{2.212835in}}%
\pgfpathlineto{\pgfqpoint{2.724178in}{2.205550in}}%
\pgfpathlineto{\pgfqpoint{2.715756in}{2.198345in}}%
\pgfpathlineto{\pgfqpoint{2.707325in}{2.191222in}}%
\pgfpathlineto{\pgfqpoint{2.698886in}{2.184182in}}%
\pgfpathclose%
\pgfusepath{fill}%
\end{pgfscope}%
\begin{pgfscope}%
\pgfpathrectangle{\pgfqpoint{1.150000in}{0.150000in}}{\pgfqpoint{5.700000in}{5.700000in}}%
\pgfusepath{clip}%
\pgfsetbuttcap%
\pgfsetroundjoin%
\definecolor{currentfill}{rgb}{0.229739,0.322361,0.545706}%
\pgfsetfillcolor{currentfill}%
\pgfsetfillopacity{0.700000}%
\pgfsetlinewidth{0.000000pt}%
\definecolor{currentstroke}{rgb}{0.000000,0.000000,0.000000}%
\pgfsetstrokecolor{currentstroke}%
\pgfsetdash{}{0pt}%
\pgfpathmoveto{\pgfqpoint{5.689173in}{2.657362in}}%
\pgfpathlineto{\pgfqpoint{5.703351in}{2.657369in}}%
\pgfpathlineto{\pgfqpoint{5.717540in}{2.657443in}}%
\pgfpathlineto{\pgfqpoint{5.731740in}{2.657582in}}%
\pgfpathlineto{\pgfqpoint{5.745950in}{2.657788in}}%
\pgfpathlineto{\pgfqpoint{5.753229in}{2.665122in}}%
\pgfpathlineto{\pgfqpoint{5.760509in}{2.672654in}}%
\pgfpathlineto{\pgfqpoint{5.767791in}{2.680391in}}%
\pgfpathlineto{\pgfqpoint{5.775074in}{2.688340in}}%
\pgfpathlineto{\pgfqpoint{5.760890in}{2.688564in}}%
\pgfpathlineto{\pgfqpoint{5.746716in}{2.688854in}}%
\pgfpathlineto{\pgfqpoint{5.732553in}{2.689210in}}%
\pgfpathlineto{\pgfqpoint{5.718401in}{2.689632in}}%
\pgfpathlineto{\pgfqpoint{5.711092in}{2.681246in}}%
\pgfpathlineto{\pgfqpoint{5.703784in}{2.673077in}}%
\pgfpathlineto{\pgfqpoint{5.696478in}{2.665118in}}%
\pgfpathlineto{\pgfqpoint{5.689173in}{2.657362in}}%
\pgfpathclose%
\pgfusepath{fill}%
\end{pgfscope}%
\begin{pgfscope}%
\pgfpathrectangle{\pgfqpoint{1.150000in}{0.150000in}}{\pgfqpoint{5.700000in}{5.700000in}}%
\pgfusepath{clip}%
\pgfsetbuttcap%
\pgfsetroundjoin%
\definecolor{currentfill}{rgb}{0.252194,0.269783,0.531579}%
\pgfsetfillcolor{currentfill}%
\pgfsetfillopacity{0.700000}%
\pgfsetlinewidth{0.000000pt}%
\definecolor{currentstroke}{rgb}{0.000000,0.000000,0.000000}%
\pgfsetstrokecolor{currentstroke}%
\pgfsetdash{}{0pt}%
\pgfpathmoveto{\pgfqpoint{5.288905in}{2.538433in}}%
\pgfpathlineto{\pgfqpoint{5.302970in}{2.538617in}}%
\pgfpathlineto{\pgfqpoint{5.317045in}{2.538869in}}%
\pgfpathlineto{\pgfqpoint{5.331129in}{2.539189in}}%
\pgfpathlineto{\pgfqpoint{5.345224in}{2.539577in}}%
\pgfpathlineto{\pgfqpoint{5.352643in}{2.546424in}}%
\pgfpathlineto{\pgfqpoint{5.360059in}{2.553383in}}%
\pgfpathlineto{\pgfqpoint{5.367472in}{2.560461in}}%
\pgfpathlineto{\pgfqpoint{5.374883in}{2.567665in}}%
\pgfpathlineto{\pgfqpoint{5.360809in}{2.567626in}}%
\pgfpathlineto{\pgfqpoint{5.346745in}{2.567654in}}%
\pgfpathlineto{\pgfqpoint{5.332692in}{2.567750in}}%
\pgfpathlineto{\pgfqpoint{5.318648in}{2.567914in}}%
\pgfpathlineto{\pgfqpoint{5.311216in}{2.560355in}}%
\pgfpathlineto{\pgfqpoint{5.303782in}{2.552926in}}%
\pgfpathlineto{\pgfqpoint{5.296345in}{2.545620in}}%
\pgfpathlineto{\pgfqpoint{5.288905in}{2.538433in}}%
\pgfpathclose%
\pgfusepath{fill}%
\end{pgfscope}%
\begin{pgfscope}%
\pgfpathrectangle{\pgfqpoint{1.150000in}{0.150000in}}{\pgfqpoint{5.700000in}{5.700000in}}%
\pgfusepath{clip}%
\pgfsetbuttcap%
\pgfsetroundjoin%
\definecolor{currentfill}{rgb}{0.282884,0.135920,0.453427}%
\pgfsetfillcolor{currentfill}%
\pgfsetfillopacity{0.700000}%
\pgfsetlinewidth{0.000000pt}%
\definecolor{currentstroke}{rgb}{0.000000,0.000000,0.000000}%
\pgfsetstrokecolor{currentstroke}%
\pgfsetdash{}{0pt}%
\pgfpathmoveto{\pgfqpoint{2.502625in}{2.273150in}}%
\pgfpathlineto{\pgfqpoint{2.516162in}{2.262791in}}%
\pgfpathlineto{\pgfqpoint{2.529699in}{2.252558in}}%
\pgfpathlineto{\pgfqpoint{2.543234in}{2.242449in}}%
\pgfpathlineto{\pgfqpoint{2.556769in}{2.232463in}}%
\pgfpathlineto{\pgfqpoint{2.565287in}{2.238840in}}%
\pgfpathlineto{\pgfqpoint{2.573796in}{2.245318in}}%
\pgfpathlineto{\pgfqpoint{2.582294in}{2.251895in}}%
\pgfpathlineto{\pgfqpoint{2.590783in}{2.258570in}}%
\pgfpathlineto{\pgfqpoint{2.577270in}{2.268408in}}%
\pgfpathlineto{\pgfqpoint{2.563756in}{2.278370in}}%
\pgfpathlineto{\pgfqpoint{2.550241in}{2.288455in}}%
\pgfpathlineto{\pgfqpoint{2.536725in}{2.298666in}}%
\pgfpathlineto{\pgfqpoint{2.528215in}{2.292131in}}%
\pgfpathlineto{\pgfqpoint{2.519695in}{2.285699in}}%
\pgfpathlineto{\pgfqpoint{2.511165in}{2.279371in}}%
\pgfpathlineto{\pgfqpoint{2.502625in}{2.273150in}}%
\pgfpathclose%
\pgfusepath{fill}%
\end{pgfscope}%
\begin{pgfscope}%
\pgfpathrectangle{\pgfqpoint{1.150000in}{0.150000in}}{\pgfqpoint{5.700000in}{5.700000in}}%
\pgfusepath{clip}%
\pgfsetbuttcap%
\pgfsetroundjoin%
\definecolor{currentfill}{rgb}{0.283229,0.120777,0.440584}%
\pgfsetfillcolor{currentfill}%
\pgfsetfillopacity{0.700000}%
\pgfsetlinewidth{0.000000pt}%
\definecolor{currentstroke}{rgb}{0.000000,0.000000,0.000000}%
\pgfsetstrokecolor{currentstroke}%
\pgfsetdash{}{0pt}%
\pgfpathmoveto{\pgfqpoint{4.174230in}{2.222655in}}%
\pgfpathlineto{\pgfqpoint{4.187955in}{2.221306in}}%
\pgfpathlineto{\pgfqpoint{4.201688in}{2.220033in}}%
\pgfpathlineto{\pgfqpoint{4.215428in}{2.218836in}}%
\pgfpathlineto{\pgfqpoint{4.229176in}{2.217715in}}%
\pgfpathlineto{\pgfqpoint{4.237033in}{2.226070in}}%
\pgfpathlineto{\pgfqpoint{4.244885in}{2.234420in}}%
\pgfpathlineto{\pgfqpoint{4.252731in}{2.242768in}}%
\pgfpathlineto{\pgfqpoint{4.260571in}{2.251117in}}%
\pgfpathlineto{\pgfqpoint{4.246834in}{2.252360in}}%
\pgfpathlineto{\pgfqpoint{4.233105in}{2.253679in}}%
\pgfpathlineto{\pgfqpoint{4.219384in}{2.255073in}}%
\pgfpathlineto{\pgfqpoint{4.205671in}{2.256544in}}%
\pgfpathlineto{\pgfqpoint{4.197819in}{2.248066in}}%
\pgfpathlineto{\pgfqpoint{4.189962in}{2.239593in}}%
\pgfpathlineto{\pgfqpoint{4.182099in}{2.231124in}}%
\pgfpathlineto{\pgfqpoint{4.174230in}{2.222655in}}%
\pgfpathclose%
\pgfusepath{fill}%
\end{pgfscope}%
\begin{pgfscope}%
\pgfpathrectangle{\pgfqpoint{1.150000in}{0.150000in}}{\pgfqpoint{5.700000in}{5.700000in}}%
\pgfusepath{clip}%
\pgfsetbuttcap%
\pgfsetroundjoin%
\definecolor{currentfill}{rgb}{0.280894,0.078907,0.402329}%
\pgfsetfillcolor{currentfill}%
\pgfsetfillopacity{0.700000}%
\pgfsetlinewidth{0.000000pt}%
\definecolor{currentstroke}{rgb}{0.000000,0.000000,0.000000}%
\pgfsetstrokecolor{currentstroke}%
\pgfsetdash{}{0pt}%
\pgfpathmoveto{\pgfqpoint{3.860238in}{2.148574in}}%
\pgfpathlineto{\pgfqpoint{3.873881in}{2.146215in}}%
\pgfpathlineto{\pgfqpoint{3.887530in}{2.143936in}}%
\pgfpathlineto{\pgfqpoint{3.901186in}{2.141737in}}%
\pgfpathlineto{\pgfqpoint{3.914849in}{2.139618in}}%
\pgfpathlineto{\pgfqpoint{3.922819in}{2.148397in}}%
\pgfpathlineto{\pgfqpoint{3.930783in}{2.157169in}}%
\pgfpathlineto{\pgfqpoint{3.938741in}{2.165935in}}%
\pgfpathlineto{\pgfqpoint{3.946694in}{2.174698in}}%
\pgfpathlineto{\pgfqpoint{3.933042in}{2.176877in}}%
\pgfpathlineto{\pgfqpoint{3.919396in}{2.179136in}}%
\pgfpathlineto{\pgfqpoint{3.905758in}{2.181475in}}%
\pgfpathlineto{\pgfqpoint{3.892126in}{2.183894in}}%
\pgfpathlineto{\pgfqpoint{3.884163in}{2.175064in}}%
\pgfpathlineto{\pgfqpoint{3.876193in}{2.166235in}}%
\pgfpathlineto{\pgfqpoint{3.868219in}{2.157405in}}%
\pgfpathlineto{\pgfqpoint{3.860238in}{2.148574in}}%
\pgfpathclose%
\pgfusepath{fill}%
\end{pgfscope}%
\begin{pgfscope}%
\pgfpathrectangle{\pgfqpoint{1.150000in}{0.150000in}}{\pgfqpoint{5.700000in}{5.700000in}}%
\pgfusepath{clip}%
\pgfsetbuttcap%
\pgfsetroundjoin%
\definecolor{currentfill}{rgb}{0.269308,0.218818,0.509577}%
\pgfsetfillcolor{currentfill}%
\pgfsetfillopacity{0.700000}%
\pgfsetlinewidth{0.000000pt}%
\definecolor{currentstroke}{rgb}{0.000000,0.000000,0.000000}%
\pgfsetstrokecolor{currentstroke}%
\pgfsetdash{}{0pt}%
\pgfpathmoveto{\pgfqpoint{4.888626in}{2.421656in}}%
\pgfpathlineto{\pgfqpoint{4.902568in}{2.421649in}}%
\pgfpathlineto{\pgfqpoint{4.916519in}{2.421711in}}%
\pgfpathlineto{\pgfqpoint{4.930481in}{2.421844in}}%
\pgfpathlineto{\pgfqpoint{4.944451in}{2.422047in}}%
\pgfpathlineto{\pgfqpoint{4.952031in}{2.429162in}}%
\pgfpathlineto{\pgfqpoint{4.959605in}{2.436327in}}%
\pgfpathlineto{\pgfqpoint{4.967174in}{2.443547in}}%
\pgfpathlineto{\pgfqpoint{4.974739in}{2.450827in}}%
\pgfpathlineto{\pgfqpoint{4.960785in}{2.450891in}}%
\pgfpathlineto{\pgfqpoint{4.946840in}{2.451025in}}%
\pgfpathlineto{\pgfqpoint{4.932905in}{2.451228in}}%
\pgfpathlineto{\pgfqpoint{4.918979in}{2.451501in}}%
\pgfpathlineto{\pgfqpoint{4.911398in}{2.443947in}}%
\pgfpathlineto{\pgfqpoint{4.903812in}{2.436458in}}%
\pgfpathlineto{\pgfqpoint{4.896221in}{2.429029in}}%
\pgfpathlineto{\pgfqpoint{4.888626in}{2.421656in}}%
\pgfpathclose%
\pgfusepath{fill}%
\end{pgfscope}%
\begin{pgfscope}%
\pgfpathrectangle{\pgfqpoint{1.150000in}{0.150000in}}{\pgfqpoint{5.700000in}{5.700000in}}%
\pgfusepath{clip}%
\pgfsetbuttcap%
\pgfsetroundjoin%
\definecolor{currentfill}{rgb}{0.276022,0.044167,0.370164}%
\pgfsetfillcolor{currentfill}%
\pgfsetfillopacity{0.700000}%
\pgfsetlinewidth{0.000000pt}%
\definecolor{currentstroke}{rgb}{0.000000,0.000000,0.000000}%
\pgfsetstrokecolor{currentstroke}%
\pgfsetdash{}{0pt}%
\pgfpathmoveto{\pgfqpoint{3.035983in}{2.092621in}}%
\pgfpathlineto{\pgfqpoint{3.049498in}{2.086207in}}%
\pgfpathlineto{\pgfqpoint{3.063015in}{2.079891in}}%
\pgfpathlineto{\pgfqpoint{3.076535in}{2.073675in}}%
\pgfpathlineto{\pgfqpoint{3.090058in}{2.067556in}}%
\pgfpathlineto{\pgfqpoint{3.098331in}{2.075776in}}%
\pgfpathlineto{\pgfqpoint{3.106598in}{2.084038in}}%
\pgfpathlineto{\pgfqpoint{3.114858in}{2.092343in}}%
\pgfpathlineto{\pgfqpoint{3.123110in}{2.100688in}}%
\pgfpathlineto{\pgfqpoint{3.109602in}{2.106724in}}%
\pgfpathlineto{\pgfqpoint{3.096097in}{2.112857in}}%
\pgfpathlineto{\pgfqpoint{3.082595in}{2.119089in}}%
\pgfpathlineto{\pgfqpoint{3.069096in}{2.125419in}}%
\pgfpathlineto{\pgfqpoint{3.060828in}{2.117150in}}%
\pgfpathlineto{\pgfqpoint{3.052554in}{2.108926in}}%
\pgfpathlineto{\pgfqpoint{3.044272in}{2.100750in}}%
\pgfpathlineto{\pgfqpoint{3.035983in}{2.092621in}}%
\pgfpathclose%
\pgfusepath{fill}%
\end{pgfscope}%
\begin{pgfscope}%
\pgfpathrectangle{\pgfqpoint{1.150000in}{0.150000in}}{\pgfqpoint{5.700000in}{5.700000in}}%
\pgfusepath{clip}%
\pgfsetbuttcap%
\pgfsetroundjoin%
\definecolor{currentfill}{rgb}{0.212395,0.359683,0.551710}%
\pgfsetfillcolor{currentfill}%
\pgfsetfillopacity{0.700000}%
\pgfsetlinewidth{0.000000pt}%
\definecolor{currentstroke}{rgb}{0.000000,0.000000,0.000000}%
\pgfsetstrokecolor{currentstroke}%
\pgfsetdash{}{0pt}%
\pgfpathmoveto{\pgfqpoint{6.003805in}{2.751927in}}%
\pgfpathlineto{\pgfqpoint{6.018071in}{2.751632in}}%
\pgfpathlineto{\pgfqpoint{6.032347in}{2.751402in}}%
\pgfpathlineto{\pgfqpoint{6.046634in}{2.751237in}}%
\pgfpathlineto{\pgfqpoint{6.053841in}{2.759653in}}%
\pgfpathlineto{\pgfqpoint{6.061053in}{2.768345in}}%
\pgfpathlineto{\pgfqpoint{6.068272in}{2.777322in}}%
\pgfpathlineto{\pgfqpoint{6.075496in}{2.786593in}}%
\pgfpathlineto{\pgfqpoint{6.061239in}{2.787248in}}%
\pgfpathlineto{\pgfqpoint{6.046993in}{2.787968in}}%
\pgfpathlineto{\pgfqpoint{6.032757in}{2.788753in}}%
\pgfpathlineto{\pgfqpoint{6.025510in}{2.779110in}}%
\pgfpathlineto{\pgfqpoint{6.018270in}{2.769764in}}%
\pgfpathlineto{\pgfqpoint{6.011035in}{2.760706in}}%
\pgfpathlineto{\pgfqpoint{6.003805in}{2.751927in}}%
\pgfpathclose%
\pgfusepath{fill}%
\end{pgfscope}%
\begin{pgfscope}%
\pgfpathrectangle{\pgfqpoint{1.150000in}{0.150000in}}{\pgfqpoint{5.700000in}{5.700000in}}%
\pgfusepath{clip}%
\pgfsetbuttcap%
\pgfsetroundjoin%
\definecolor{currentfill}{rgb}{0.280868,0.160771,0.472899}%
\pgfsetfillcolor{currentfill}%
\pgfsetfillopacity{0.700000}%
\pgfsetlinewidth{0.000000pt}%
\definecolor{currentstroke}{rgb}{0.000000,0.000000,0.000000}%
\pgfsetstrokecolor{currentstroke}%
\pgfsetdash{}{0pt}%
\pgfpathmoveto{\pgfqpoint{4.488272in}{2.305521in}}%
\pgfpathlineto{\pgfqpoint{4.502091in}{2.304940in}}%
\pgfpathlineto{\pgfqpoint{4.515919in}{2.304431in}}%
\pgfpathlineto{\pgfqpoint{4.529756in}{2.303996in}}%
\pgfpathlineto{\pgfqpoint{4.543601in}{2.303634in}}%
\pgfpathlineto{\pgfqpoint{4.551341in}{2.311413in}}%
\pgfpathlineto{\pgfqpoint{4.559075in}{2.319202in}}%
\pgfpathlineto{\pgfqpoint{4.566803in}{2.327005in}}%
\pgfpathlineto{\pgfqpoint{4.574526in}{2.334825in}}%
\pgfpathlineto{\pgfqpoint{4.560694in}{2.335372in}}%
\pgfpathlineto{\pgfqpoint{4.546871in}{2.335991in}}%
\pgfpathlineto{\pgfqpoint{4.533056in}{2.336682in}}%
\pgfpathlineto{\pgfqpoint{4.519250in}{2.337447in}}%
\pgfpathlineto{\pgfqpoint{4.511514in}{2.329436in}}%
\pgfpathlineto{\pgfqpoint{4.503772in}{2.321447in}}%
\pgfpathlineto{\pgfqpoint{4.496025in}{2.313476in}}%
\pgfpathlineto{\pgfqpoint{4.488272in}{2.305521in}}%
\pgfpathclose%
\pgfusepath{fill}%
\end{pgfscope}%
\begin{pgfscope}%
\pgfpathrectangle{\pgfqpoint{1.150000in}{0.150000in}}{\pgfqpoint{5.700000in}{5.700000in}}%
\pgfusepath{clip}%
\pgfsetbuttcap%
\pgfsetroundjoin%
\definecolor{currentfill}{rgb}{0.235526,0.309527,0.542944}%
\pgfsetfillcolor{currentfill}%
\pgfsetfillopacity{0.700000}%
\pgfsetlinewidth{0.000000pt}%
\definecolor{currentstroke}{rgb}{0.000000,0.000000,0.000000}%
\pgfsetstrokecolor{currentstroke}%
\pgfsetdash{}{0pt}%
\pgfpathmoveto{\pgfqpoint{5.603247in}{2.627204in}}%
\pgfpathlineto{\pgfqpoint{5.617408in}{2.627356in}}%
\pgfpathlineto{\pgfqpoint{5.631580in}{2.627575in}}%
\pgfpathlineto{\pgfqpoint{5.645763in}{2.627860in}}%
\pgfpathlineto{\pgfqpoint{5.659956in}{2.628211in}}%
\pgfpathlineto{\pgfqpoint{5.667260in}{2.635232in}}%
\pgfpathlineto{\pgfqpoint{5.674564in}{2.642426in}}%
\pgfpathlineto{\pgfqpoint{5.681869in}{2.649800in}}%
\pgfpathlineto{\pgfqpoint{5.689173in}{2.657362in}}%
\pgfpathlineto{\pgfqpoint{5.675005in}{2.657420in}}%
\pgfpathlineto{\pgfqpoint{5.660847in}{2.657545in}}%
\pgfpathlineto{\pgfqpoint{5.646700in}{2.657736in}}%
\pgfpathlineto{\pgfqpoint{5.632563in}{2.657993in}}%
\pgfpathlineto{\pgfqpoint{5.625234in}{2.650015in}}%
\pgfpathlineto{\pgfqpoint{5.617905in}{2.642228in}}%
\pgfpathlineto{\pgfqpoint{5.610576in}{2.634627in}}%
\pgfpathlineto{\pgfqpoint{5.603247in}{2.627204in}}%
\pgfpathclose%
\pgfusepath{fill}%
\end{pgfscope}%
\begin{pgfscope}%
\pgfpathrectangle{\pgfqpoint{1.150000in}{0.150000in}}{\pgfqpoint{5.700000in}{5.700000in}}%
\pgfusepath{clip}%
\pgfsetbuttcap%
\pgfsetroundjoin%
\definecolor{currentfill}{rgb}{0.274952,0.037752,0.364543}%
\pgfsetfillcolor{currentfill}%
\pgfsetfillopacity{0.700000}%
\pgfsetlinewidth{0.000000pt}%
\definecolor{currentstroke}{rgb}{0.000000,0.000000,0.000000}%
\pgfsetstrokecolor{currentstroke}%
\pgfsetdash{}{0pt}%
\pgfpathmoveto{\pgfqpoint{3.177177in}{2.077515in}}%
\pgfpathlineto{\pgfqpoint{3.190702in}{2.071960in}}%
\pgfpathlineto{\pgfqpoint{3.204231in}{2.066500in}}%
\pgfpathlineto{\pgfqpoint{3.217764in}{2.061135in}}%
\pgfpathlineto{\pgfqpoint{3.231301in}{2.055862in}}%
\pgfpathlineto{\pgfqpoint{3.239519in}{2.064389in}}%
\pgfpathlineto{\pgfqpoint{3.247730in}{2.072945in}}%
\pgfpathlineto{\pgfqpoint{3.255935in}{2.081531in}}%
\pgfpathlineto{\pgfqpoint{3.264133in}{2.090144in}}%
\pgfpathlineto{\pgfqpoint{3.250610in}{2.095354in}}%
\pgfpathlineto{\pgfqpoint{3.237091in}{2.100657in}}%
\pgfpathlineto{\pgfqpoint{3.223575in}{2.106054in}}%
\pgfpathlineto{\pgfqpoint{3.210064in}{2.111545in}}%
\pgfpathlineto{\pgfqpoint{3.201852in}{2.102987in}}%
\pgfpathlineto{\pgfqpoint{3.193634in}{2.094462in}}%
\pgfpathlineto{\pgfqpoint{3.185409in}{2.085971in}}%
\pgfpathlineto{\pgfqpoint{3.177177in}{2.077515in}}%
\pgfpathclose%
\pgfusepath{fill}%
\end{pgfscope}%
\begin{pgfscope}%
\pgfpathrectangle{\pgfqpoint{1.150000in}{0.150000in}}{\pgfqpoint{5.700000in}{5.700000in}}%
\pgfusepath{clip}%
\pgfsetbuttcap%
\pgfsetroundjoin%
\definecolor{currentfill}{rgb}{0.277941,0.056324,0.381191}%
\pgfsetfillcolor{currentfill}%
\pgfsetfillopacity{0.700000}%
\pgfsetlinewidth{0.000000pt}%
\definecolor{currentstroke}{rgb}{0.000000,0.000000,0.000000}%
\pgfsetstrokecolor{currentstroke}%
\pgfsetdash{}{0pt}%
\pgfpathmoveto{\pgfqpoint{2.894604in}{2.116400in}}%
\pgfpathlineto{\pgfqpoint{2.908115in}{2.109065in}}%
\pgfpathlineto{\pgfqpoint{2.921629in}{2.101834in}}%
\pgfpathlineto{\pgfqpoint{2.935144in}{2.094707in}}%
\pgfpathlineto{\pgfqpoint{2.948662in}{2.087683in}}%
\pgfpathlineto{\pgfqpoint{2.956996in}{2.095498in}}%
\pgfpathlineto{\pgfqpoint{2.965323in}{2.103371in}}%
\pgfpathlineto{\pgfqpoint{2.973642in}{2.111299in}}%
\pgfpathlineto{\pgfqpoint{2.981954in}{2.119282in}}%
\pgfpathlineto{\pgfqpoint{2.968453in}{2.126202in}}%
\pgfpathlineto{\pgfqpoint{2.954954in}{2.133225in}}%
\pgfpathlineto{\pgfqpoint{2.941457in}{2.140351in}}%
\pgfpathlineto{\pgfqpoint{2.927963in}{2.147581in}}%
\pgfpathlineto{\pgfqpoint{2.919635in}{2.139695in}}%
\pgfpathlineto{\pgfqpoint{2.911299in}{2.131868in}}%
\pgfpathlineto{\pgfqpoint{2.902956in}{2.124103in}}%
\pgfpathlineto{\pgfqpoint{2.894604in}{2.116400in}}%
\pgfpathclose%
\pgfusepath{fill}%
\end{pgfscope}%
\begin{pgfscope}%
\pgfpathrectangle{\pgfqpoint{1.150000in}{0.150000in}}{\pgfqpoint{5.700000in}{5.700000in}}%
\pgfusepath{clip}%
\pgfsetbuttcap%
\pgfsetroundjoin%
\definecolor{currentfill}{rgb}{0.255645,0.260703,0.528312}%
\pgfsetfillcolor{currentfill}%
\pgfsetfillopacity{0.700000}%
\pgfsetlinewidth{0.000000pt}%
\definecolor{currentstroke}{rgb}{0.000000,0.000000,0.000000}%
\pgfsetstrokecolor{currentstroke}%
\pgfsetdash{}{0pt}%
\pgfpathmoveto{\pgfqpoint{5.202876in}{2.509370in}}%
\pgfpathlineto{\pgfqpoint{5.216921in}{2.509610in}}%
\pgfpathlineto{\pgfqpoint{5.230975in}{2.509918in}}%
\pgfpathlineto{\pgfqpoint{5.245040in}{2.510295in}}%
\pgfpathlineto{\pgfqpoint{5.259115in}{2.510739in}}%
\pgfpathlineto{\pgfqpoint{5.266567in}{2.517516in}}%
\pgfpathlineto{\pgfqpoint{5.274017in}{2.524386in}}%
\pgfpathlineto{\pgfqpoint{5.281463in}{2.531356in}}%
\pgfpathlineto{\pgfqpoint{5.288905in}{2.538433in}}%
\pgfpathlineto{\pgfqpoint{5.274851in}{2.538316in}}%
\pgfpathlineto{\pgfqpoint{5.260806in}{2.538268in}}%
\pgfpathlineto{\pgfqpoint{5.246772in}{2.538288in}}%
\pgfpathlineto{\pgfqpoint{5.232747in}{2.538376in}}%
\pgfpathlineto{\pgfqpoint{5.225284in}{2.530964in}}%
\pgfpathlineto{\pgfqpoint{5.217818in}{2.523663in}}%
\pgfpathlineto{\pgfqpoint{5.210349in}{2.516467in}}%
\pgfpathlineto{\pgfqpoint{5.202876in}{2.509370in}}%
\pgfpathclose%
\pgfusepath{fill}%
\end{pgfscope}%
\begin{pgfscope}%
\pgfpathrectangle{\pgfqpoint{1.150000in}{0.150000in}}{\pgfqpoint{5.700000in}{5.700000in}}%
\pgfusepath{clip}%
\pgfsetbuttcap%
\pgfsetroundjoin%
\definecolor{currentfill}{rgb}{0.277018,0.050344,0.375715}%
\pgfsetfillcolor{currentfill}%
\pgfsetfillopacity{0.700000}%
\pgfsetlinewidth{0.000000pt}%
\definecolor{currentstroke}{rgb}{0.000000,0.000000,0.000000}%
\pgfsetstrokecolor{currentstroke}%
\pgfsetdash{}{0pt}%
\pgfpathmoveto{\pgfqpoint{3.546069in}{2.090041in}}%
\pgfpathlineto{\pgfqpoint{3.559648in}{2.086413in}}%
\pgfpathlineto{\pgfqpoint{3.573232in}{2.082871in}}%
\pgfpathlineto{\pgfqpoint{3.586822in}{2.079413in}}%
\pgfpathlineto{\pgfqpoint{3.600418in}{2.076041in}}%
\pgfpathlineto{\pgfqpoint{3.608500in}{2.084948in}}%
\pgfpathlineto{\pgfqpoint{3.616577in}{2.093858in}}%
\pgfpathlineto{\pgfqpoint{3.624647in}{2.102771in}}%
\pgfpathlineto{\pgfqpoint{3.632712in}{2.111688in}}%
\pgfpathlineto{\pgfqpoint{3.619128in}{2.115060in}}%
\pgfpathlineto{\pgfqpoint{3.605550in}{2.118516in}}%
\pgfpathlineto{\pgfqpoint{3.591977in}{2.122057in}}%
\pgfpathlineto{\pgfqpoint{3.578410in}{2.125684in}}%
\pgfpathlineto{\pgfqpoint{3.570333in}{2.116760in}}%
\pgfpathlineto{\pgfqpoint{3.562251in}{2.107846in}}%
\pgfpathlineto{\pgfqpoint{3.554163in}{2.098940in}}%
\pgfpathlineto{\pgfqpoint{3.546069in}{2.090041in}}%
\pgfpathclose%
\pgfusepath{fill}%
\end{pgfscope}%
\begin{pgfscope}%
\pgfpathrectangle{\pgfqpoint{1.150000in}{0.150000in}}{\pgfqpoint{5.700000in}{5.700000in}}%
\pgfusepath{clip}%
\pgfsetbuttcap%
\pgfsetroundjoin%
\definecolor{currentfill}{rgb}{0.283091,0.110553,0.431554}%
\pgfsetfillcolor{currentfill}%
\pgfsetfillopacity{0.700000}%
\pgfsetlinewidth{0.000000pt}%
\definecolor{currentstroke}{rgb}{0.000000,0.000000,0.000000}%
\pgfsetstrokecolor{currentstroke}%
\pgfsetdash{}{0pt}%
\pgfpathmoveto{\pgfqpoint{4.087831in}{2.194499in}}%
\pgfpathlineto{\pgfqpoint{4.101537in}{2.192944in}}%
\pgfpathlineto{\pgfqpoint{4.115250in}{2.191466in}}%
\pgfpathlineto{\pgfqpoint{4.128971in}{2.190065in}}%
\pgfpathlineto{\pgfqpoint{4.142699in}{2.188741in}}%
\pgfpathlineto{\pgfqpoint{4.150590in}{2.197230in}}%
\pgfpathlineto{\pgfqpoint{4.158476in}{2.205710in}}%
\pgfpathlineto{\pgfqpoint{4.166356in}{2.214184in}}%
\pgfpathlineto{\pgfqpoint{4.174230in}{2.222655in}}%
\pgfpathlineto{\pgfqpoint{4.160513in}{2.224080in}}%
\pgfpathlineto{\pgfqpoint{4.146803in}{2.225583in}}%
\pgfpathlineto{\pgfqpoint{4.133101in}{2.227162in}}%
\pgfpathlineto{\pgfqpoint{4.119407in}{2.228818in}}%
\pgfpathlineto{\pgfqpoint{4.111521in}{2.220238in}}%
\pgfpathlineto{\pgfqpoint{4.103630in}{2.211660in}}%
\pgfpathlineto{\pgfqpoint{4.095734in}{2.203081in}}%
\pgfpathlineto{\pgfqpoint{4.087831in}{2.194499in}}%
\pgfpathclose%
\pgfusepath{fill}%
\end{pgfscope}%
\begin{pgfscope}%
\pgfpathrectangle{\pgfqpoint{1.150000in}{0.150000in}}{\pgfqpoint{5.700000in}{5.700000in}}%
\pgfusepath{clip}%
\pgfsetbuttcap%
\pgfsetroundjoin%
\definecolor{currentfill}{rgb}{0.273006,0.204520,0.501721}%
\pgfsetfillcolor{currentfill}%
\pgfsetfillopacity{0.700000}%
\pgfsetlinewidth{0.000000pt}%
\definecolor{currentstroke}{rgb}{0.000000,0.000000,0.000000}%
\pgfsetstrokecolor{currentstroke}%
\pgfsetdash{}{0pt}%
\pgfpathmoveto{\pgfqpoint{4.802455in}{2.392375in}}%
\pgfpathlineto{\pgfqpoint{4.816375in}{2.392331in}}%
\pgfpathlineto{\pgfqpoint{4.830305in}{2.392359in}}%
\pgfpathlineto{\pgfqpoint{4.844244in}{2.392457in}}%
\pgfpathlineto{\pgfqpoint{4.858193in}{2.392625in}}%
\pgfpathlineto{\pgfqpoint{4.865809in}{2.399823in}}%
\pgfpathlineto{\pgfqpoint{4.873420in}{2.407057in}}%
\pgfpathlineto{\pgfqpoint{4.881025in}{2.414333in}}%
\pgfpathlineto{\pgfqpoint{4.888626in}{2.421656in}}%
\pgfpathlineto{\pgfqpoint{4.874693in}{2.421734in}}%
\pgfpathlineto{\pgfqpoint{4.860769in}{2.421882in}}%
\pgfpathlineto{\pgfqpoint{4.846855in}{2.422100in}}%
\pgfpathlineto{\pgfqpoint{4.832950in}{2.422390in}}%
\pgfpathlineto{\pgfqpoint{4.825334in}{2.414814in}}%
\pgfpathlineto{\pgfqpoint{4.817713in}{2.407289in}}%
\pgfpathlineto{\pgfqpoint{4.810086in}{2.399811in}}%
\pgfpathlineto{\pgfqpoint{4.802455in}{2.392375in}}%
\pgfpathclose%
\pgfusepath{fill}%
\end{pgfscope}%
\begin{pgfscope}%
\pgfpathrectangle{\pgfqpoint{1.150000in}{0.150000in}}{\pgfqpoint{5.700000in}{5.700000in}}%
\pgfusepath{clip}%
\pgfsetbuttcap%
\pgfsetroundjoin%
\definecolor{currentfill}{rgb}{0.280267,0.073417,0.397163}%
\pgfsetfillcolor{currentfill}%
\pgfsetfillopacity{0.700000}%
\pgfsetlinewidth{0.000000pt}%
\definecolor{currentstroke}{rgb}{0.000000,0.000000,0.000000}%
\pgfsetstrokecolor{currentstroke}%
\pgfsetdash{}{0pt}%
\pgfpathmoveto{\pgfqpoint{3.773711in}{2.123292in}}%
\pgfpathlineto{\pgfqpoint{3.787338in}{2.120649in}}%
\pgfpathlineto{\pgfqpoint{3.800972in}{2.118087in}}%
\pgfpathlineto{\pgfqpoint{3.814612in}{2.115607in}}%
\pgfpathlineto{\pgfqpoint{3.828259in}{2.113207in}}%
\pgfpathlineto{\pgfqpoint{3.836262in}{2.122058in}}%
\pgfpathlineto{\pgfqpoint{3.844260in}{2.130901in}}%
\pgfpathlineto{\pgfqpoint{3.852252in}{2.139740in}}%
\pgfpathlineto{\pgfqpoint{3.860238in}{2.148574in}}%
\pgfpathlineto{\pgfqpoint{3.846602in}{2.151014in}}%
\pgfpathlineto{\pgfqpoint{3.832973in}{2.153534in}}%
\pgfpathlineto{\pgfqpoint{3.819350in}{2.156136in}}%
\pgfpathlineto{\pgfqpoint{3.805733in}{2.158818in}}%
\pgfpathlineto{\pgfqpoint{3.797736in}{2.149937in}}%
\pgfpathlineto{\pgfqpoint{3.789734in}{2.141056in}}%
\pgfpathlineto{\pgfqpoint{3.781725in}{2.132175in}}%
\pgfpathlineto{\pgfqpoint{3.773711in}{2.123292in}}%
\pgfpathclose%
\pgfusepath{fill}%
\end{pgfscope}%
\begin{pgfscope}%
\pgfpathrectangle{\pgfqpoint{1.150000in}{0.150000in}}{\pgfqpoint{5.700000in}{5.700000in}}%
\pgfusepath{clip}%
\pgfsetbuttcap%
\pgfsetroundjoin%
\definecolor{currentfill}{rgb}{0.283229,0.120777,0.440584}%
\pgfsetfillcolor{currentfill}%
\pgfsetfillopacity{0.700000}%
\pgfsetlinewidth{0.000000pt}%
\definecolor{currentstroke}{rgb}{0.000000,0.000000,0.000000}%
\pgfsetstrokecolor{currentstroke}%
\pgfsetdash{}{0pt}%
\pgfpathmoveto{\pgfqpoint{2.556769in}{2.232463in}}%
\pgfpathlineto{\pgfqpoint{2.570303in}{2.222599in}}%
\pgfpathlineto{\pgfqpoint{2.583837in}{2.212857in}}%
\pgfpathlineto{\pgfqpoint{2.597371in}{2.203235in}}%
\pgfpathlineto{\pgfqpoint{2.610904in}{2.193733in}}%
\pgfpathlineto{\pgfqpoint{2.619401in}{2.200265in}}%
\pgfpathlineto{\pgfqpoint{2.627888in}{2.206893in}}%
\pgfpathlineto{\pgfqpoint{2.636366in}{2.213615in}}%
\pgfpathlineto{\pgfqpoint{2.644834in}{2.220430in}}%
\pgfpathlineto{\pgfqpoint{2.631322in}{2.229785in}}%
\pgfpathlineto{\pgfqpoint{2.617809in}{2.239260in}}%
\pgfpathlineto{\pgfqpoint{2.604296in}{2.248854in}}%
\pgfpathlineto{\pgfqpoint{2.590783in}{2.258570in}}%
\pgfpathlineto{\pgfqpoint{2.582294in}{2.251895in}}%
\pgfpathlineto{\pgfqpoint{2.573796in}{2.245318in}}%
\pgfpathlineto{\pgfqpoint{2.565287in}{2.238840in}}%
\pgfpathlineto{\pgfqpoint{2.556769in}{2.232463in}}%
\pgfpathclose%
\pgfusepath{fill}%
\end{pgfscope}%
\begin{pgfscope}%
\pgfpathrectangle{\pgfqpoint{1.150000in}{0.150000in}}{\pgfqpoint{5.700000in}{5.700000in}}%
\pgfusepath{clip}%
\pgfsetbuttcap%
\pgfsetroundjoin%
\definecolor{currentfill}{rgb}{0.274952,0.037752,0.364543}%
\pgfsetfillcolor{currentfill}%
\pgfsetfillopacity{0.700000}%
\pgfsetlinewidth{0.000000pt}%
\definecolor{currentstroke}{rgb}{0.000000,0.000000,0.000000}%
\pgfsetstrokecolor{currentstroke}%
\pgfsetdash{}{0pt}%
\pgfpathmoveto{\pgfqpoint{3.318267in}{2.070231in}}%
\pgfpathlineto{\pgfqpoint{3.331811in}{2.065481in}}%
\pgfpathlineto{\pgfqpoint{3.345359in}{2.060822in}}%
\pgfpathlineto{\pgfqpoint{3.358912in}{2.056253in}}%
\pgfpathlineto{\pgfqpoint{3.372470in}{2.051774in}}%
\pgfpathlineto{\pgfqpoint{3.380636in}{2.060515in}}%
\pgfpathlineto{\pgfqpoint{3.388796in}{2.069274in}}%
\pgfpathlineto{\pgfqpoint{3.396950in}{2.078050in}}%
\pgfpathlineto{\pgfqpoint{3.405097in}{2.086844in}}%
\pgfpathlineto{\pgfqpoint{3.391552in}{2.091281in}}%
\pgfpathlineto{\pgfqpoint{3.378012in}{2.095808in}}%
\pgfpathlineto{\pgfqpoint{3.364476in}{2.100425in}}%
\pgfpathlineto{\pgfqpoint{3.350945in}{2.105132in}}%
\pgfpathlineto{\pgfqpoint{3.342785in}{2.096373in}}%
\pgfpathlineto{\pgfqpoint{3.334618in}{2.087637in}}%
\pgfpathlineto{\pgfqpoint{3.326446in}{2.078922in}}%
\pgfpathlineto{\pgfqpoint{3.318267in}{2.070231in}}%
\pgfpathclose%
\pgfusepath{fill}%
\end{pgfscope}%
\begin{pgfscope}%
\pgfpathrectangle{\pgfqpoint{1.150000in}{0.150000in}}{\pgfqpoint{5.700000in}{5.700000in}}%
\pgfusepath{clip}%
\pgfsetbuttcap%
\pgfsetroundjoin%
\definecolor{currentfill}{rgb}{0.280894,0.078907,0.402329}%
\pgfsetfillcolor{currentfill}%
\pgfsetfillopacity{0.700000}%
\pgfsetlinewidth{0.000000pt}%
\definecolor{currentstroke}{rgb}{0.000000,0.000000,0.000000}%
\pgfsetstrokecolor{currentstroke}%
\pgfsetdash{}{0pt}%
\pgfpathmoveto{\pgfqpoint{2.752946in}{2.149771in}}%
\pgfpathlineto{\pgfqpoint{2.766464in}{2.141448in}}%
\pgfpathlineto{\pgfqpoint{2.779982in}{2.133236in}}%
\pgfpathlineto{\pgfqpoint{2.793501in}{2.125134in}}%
\pgfpathlineto{\pgfqpoint{2.807022in}{2.117141in}}%
\pgfpathlineto{\pgfqpoint{2.815423in}{2.124446in}}%
\pgfpathlineto{\pgfqpoint{2.823816in}{2.131826in}}%
\pgfpathlineto{\pgfqpoint{2.832201in}{2.139277in}}%
\pgfpathlineto{\pgfqpoint{2.840578in}{2.146798in}}%
\pgfpathlineto{\pgfqpoint{2.827075in}{2.154667in}}%
\pgfpathlineto{\pgfqpoint{2.813574in}{2.162643in}}%
\pgfpathlineto{\pgfqpoint{2.800074in}{2.170730in}}%
\pgfpathlineto{\pgfqpoint{2.786576in}{2.178927in}}%
\pgfpathlineto{\pgfqpoint{2.778181in}{2.171523in}}%
\pgfpathlineto{\pgfqpoint{2.769778in}{2.164195in}}%
\pgfpathlineto{\pgfqpoint{2.761366in}{2.156943in}}%
\pgfpathlineto{\pgfqpoint{2.752946in}{2.149771in}}%
\pgfpathclose%
\pgfusepath{fill}%
\end{pgfscope}%
\begin{pgfscope}%
\pgfpathrectangle{\pgfqpoint{1.150000in}{0.150000in}}{\pgfqpoint{5.700000in}{5.700000in}}%
\pgfusepath{clip}%
\pgfsetbuttcap%
\pgfsetroundjoin%
\definecolor{currentfill}{rgb}{0.281887,0.150881,0.465405}%
\pgfsetfillcolor{currentfill}%
\pgfsetfillopacity{0.700000}%
\pgfsetlinewidth{0.000000pt}%
\definecolor{currentstroke}{rgb}{0.000000,0.000000,0.000000}%
\pgfsetstrokecolor{currentstroke}%
\pgfsetdash{}{0pt}%
\pgfpathmoveto{\pgfqpoint{4.401962in}{2.276194in}}%
\pgfpathlineto{\pgfqpoint{4.415760in}{2.275481in}}%
\pgfpathlineto{\pgfqpoint{4.429567in}{2.274843in}}%
\pgfpathlineto{\pgfqpoint{4.443382in}{2.274278in}}%
\pgfpathlineto{\pgfqpoint{4.457205in}{2.273788in}}%
\pgfpathlineto{\pgfqpoint{4.464980in}{2.281714in}}%
\pgfpathlineto{\pgfqpoint{4.472750in}{2.289643in}}%
\pgfpathlineto{\pgfqpoint{4.480514in}{2.297578in}}%
\pgfpathlineto{\pgfqpoint{4.488272in}{2.305521in}}%
\pgfpathlineto{\pgfqpoint{4.474461in}{2.306175in}}%
\pgfpathlineto{\pgfqpoint{4.460659in}{2.306903in}}%
\pgfpathlineto{\pgfqpoint{4.446865in}{2.307705in}}%
\pgfpathlineto{\pgfqpoint{4.433080in}{2.308581in}}%
\pgfpathlineto{\pgfqpoint{4.425309in}{2.300466in}}%
\pgfpathlineto{\pgfqpoint{4.417532in}{2.292366in}}%
\pgfpathlineto{\pgfqpoint{4.409750in}{2.284276in}}%
\pgfpathlineto{\pgfqpoint{4.401962in}{2.276194in}}%
\pgfpathclose%
\pgfusepath{fill}%
\end{pgfscope}%
\begin{pgfscope}%
\pgfpathrectangle{\pgfqpoint{1.150000in}{0.150000in}}{\pgfqpoint{5.700000in}{5.700000in}}%
\pgfusepath{clip}%
\pgfsetbuttcap%
\pgfsetroundjoin%
\definecolor{currentfill}{rgb}{0.276194,0.190074,0.493001}%
\pgfsetfillcolor{currentfill}%
\pgfsetfillopacity{0.700000}%
\pgfsetlinewidth{0.000000pt}%
\definecolor{currentstroke}{rgb}{0.000000,0.000000,0.000000}%
\pgfsetstrokecolor{currentstroke}%
\pgfsetdash{}{0pt}%
\pgfpathmoveto{\pgfqpoint{2.305500in}{2.385905in}}%
\pgfpathlineto{\pgfqpoint{2.319084in}{2.373782in}}%
\pgfpathlineto{\pgfqpoint{2.332665in}{2.361799in}}%
\pgfpathlineto{\pgfqpoint{2.346244in}{2.349954in}}%
\pgfpathlineto{\pgfqpoint{2.359821in}{2.338247in}}%
\pgfpathlineto{\pgfqpoint{2.368452in}{2.343665in}}%
\pgfpathlineto{\pgfqpoint{2.377071in}{2.349210in}}%
\pgfpathlineto{\pgfqpoint{2.385680in}{2.354877in}}%
\pgfpathlineto{\pgfqpoint{2.394277in}{2.360666in}}%
\pgfpathlineto{\pgfqpoint{2.380725in}{2.372203in}}%
\pgfpathlineto{\pgfqpoint{2.367171in}{2.383877in}}%
\pgfpathlineto{\pgfqpoint{2.353614in}{2.395689in}}%
\pgfpathlineto{\pgfqpoint{2.340055in}{2.407641in}}%
\pgfpathlineto{\pgfqpoint{2.331434in}{2.402015in}}%
\pgfpathlineto{\pgfqpoint{2.322801in}{2.396516in}}%
\pgfpathlineto{\pgfqpoint{2.314156in}{2.391145in}}%
\pgfpathlineto{\pgfqpoint{2.305500in}{2.385905in}}%
\pgfpathclose%
\pgfusepath{fill}%
\end{pgfscope}%
\begin{pgfscope}%
\pgfpathrectangle{\pgfqpoint{1.150000in}{0.150000in}}{\pgfqpoint{5.700000in}{5.700000in}}%
\pgfusepath{clip}%
\pgfsetbuttcap%
\pgfsetroundjoin%
\definecolor{currentfill}{rgb}{0.239346,0.300855,0.540844}%
\pgfsetfillcolor{currentfill}%
\pgfsetfillopacity{0.700000}%
\pgfsetlinewidth{0.000000pt}%
\definecolor{currentstroke}{rgb}{0.000000,0.000000,0.000000}%
\pgfsetstrokecolor{currentstroke}%
\pgfsetdash{}{0pt}%
\pgfpathmoveto{\pgfqpoint{5.517285in}{2.597648in}}%
\pgfpathlineto{\pgfqpoint{5.531429in}{2.597923in}}%
\pgfpathlineto{\pgfqpoint{5.545583in}{2.598265in}}%
\pgfpathlineto{\pgfqpoint{5.559748in}{2.598674in}}%
\pgfpathlineto{\pgfqpoint{5.573923in}{2.599150in}}%
\pgfpathlineto{\pgfqpoint{5.581256in}{2.605931in}}%
\pgfpathlineto{\pgfqpoint{5.588587in}{2.612863in}}%
\pgfpathlineto{\pgfqpoint{5.595917in}{2.619952in}}%
\pgfpathlineto{\pgfqpoint{5.603247in}{2.627204in}}%
\pgfpathlineto{\pgfqpoint{5.589096in}{2.627118in}}%
\pgfpathlineto{\pgfqpoint{5.574955in}{2.627099in}}%
\pgfpathlineto{\pgfqpoint{5.560825in}{2.627147in}}%
\pgfpathlineto{\pgfqpoint{5.546705in}{2.627261in}}%
\pgfpathlineto{\pgfqpoint{5.539351in}{2.619612in}}%
\pgfpathlineto{\pgfqpoint{5.531997in}{2.612131in}}%
\pgfpathlineto{\pgfqpoint{5.524642in}{2.604812in}}%
\pgfpathlineto{\pgfqpoint{5.517285in}{2.597648in}}%
\pgfpathclose%
\pgfusepath{fill}%
\end{pgfscope}%
\begin{pgfscope}%
\pgfpathrectangle{\pgfqpoint{1.150000in}{0.150000in}}{\pgfqpoint{5.700000in}{5.700000in}}%
\pgfusepath{clip}%
\pgfsetbuttcap%
\pgfsetroundjoin%
\definecolor{currentfill}{rgb}{0.216210,0.351535,0.550627}%
\pgfsetfillcolor{currentfill}%
\pgfsetfillopacity{0.700000}%
\pgfsetlinewidth{0.000000pt}%
\definecolor{currentstroke}{rgb}{0.000000,0.000000,0.000000}%
\pgfsetstrokecolor{currentstroke}%
\pgfsetdash{}{0pt}%
\pgfpathmoveto{\pgfqpoint{5.917861in}{2.719391in}}%
\pgfpathlineto{\pgfqpoint{5.932113in}{2.719306in}}%
\pgfpathlineto{\pgfqpoint{5.946375in}{2.719286in}}%
\pgfpathlineto{\pgfqpoint{5.960649in}{2.719332in}}%
\pgfpathlineto{\pgfqpoint{5.974933in}{2.719443in}}%
\pgfpathlineto{\pgfqpoint{5.982145in}{2.727186in}}%
\pgfpathlineto{\pgfqpoint{5.989361in}{2.735176in}}%
\pgfpathlineto{\pgfqpoint{5.996581in}{2.743420in}}%
\pgfpathlineto{\pgfqpoint{6.003805in}{2.751927in}}%
\pgfpathlineto{\pgfqpoint{5.989551in}{2.752287in}}%
\pgfpathlineto{\pgfqpoint{5.975306in}{2.752712in}}%
\pgfpathlineto{\pgfqpoint{5.961073in}{2.753202in}}%
\pgfpathlineto{\pgfqpoint{5.946850in}{2.753757in}}%
\pgfpathlineto{\pgfqpoint{5.939596in}{2.744773in}}%
\pgfpathlineto{\pgfqpoint{5.932347in}{2.736056in}}%
\pgfpathlineto{\pgfqpoint{5.925102in}{2.727598in}}%
\pgfpathlineto{\pgfqpoint{5.917861in}{2.719391in}}%
\pgfpathclose%
\pgfusepath{fill}%
\end{pgfscope}%
\begin{pgfscope}%
\pgfpathrectangle{\pgfqpoint{1.150000in}{0.150000in}}{\pgfqpoint{5.700000in}{5.700000in}}%
\pgfusepath{clip}%
\pgfsetbuttcap%
\pgfsetroundjoin%
\definecolor{currentfill}{rgb}{0.260571,0.246922,0.522828}%
\pgfsetfillcolor{currentfill}%
\pgfsetfillopacity{0.700000}%
\pgfsetlinewidth{0.000000pt}%
\definecolor{currentstroke}{rgb}{0.000000,0.000000,0.000000}%
\pgfsetstrokecolor{currentstroke}%
\pgfsetdash{}{0pt}%
\pgfpathmoveto{\pgfqpoint{5.116792in}{2.480350in}}%
\pgfpathlineto{\pgfqpoint{5.130816in}{2.480623in}}%
\pgfpathlineto{\pgfqpoint{5.144850in}{2.480965in}}%
\pgfpathlineto{\pgfqpoint{5.158893in}{2.481376in}}%
\pgfpathlineto{\pgfqpoint{5.172947in}{2.481856in}}%
\pgfpathlineto{\pgfqpoint{5.180435in}{2.488614in}}%
\pgfpathlineto{\pgfqpoint{5.187920in}{2.495449in}}%
\pgfpathlineto{\pgfqpoint{5.195400in}{2.502365in}}%
\pgfpathlineto{\pgfqpoint{5.202876in}{2.509370in}}%
\pgfpathlineto{\pgfqpoint{5.188842in}{2.509198in}}%
\pgfpathlineto{\pgfqpoint{5.174817in}{2.509095in}}%
\pgfpathlineto{\pgfqpoint{5.160803in}{2.509061in}}%
\pgfpathlineto{\pgfqpoint{5.146798in}{2.509095in}}%
\pgfpathlineto{\pgfqpoint{5.139302in}{2.501776in}}%
\pgfpathlineto{\pgfqpoint{5.131803in}{2.494549in}}%
\pgfpathlineto{\pgfqpoint{5.124300in}{2.487409in}}%
\pgfpathlineto{\pgfqpoint{5.116792in}{2.480350in}}%
\pgfpathclose%
\pgfusepath{fill}%
\end{pgfscope}%
\begin{pgfscope}%
\pgfpathrectangle{\pgfqpoint{1.150000in}{0.150000in}}{\pgfqpoint{5.700000in}{5.700000in}}%
\pgfusepath{clip}%
\pgfsetbuttcap%
\pgfsetroundjoin%
\definecolor{currentfill}{rgb}{0.275191,0.194905,0.496005}%
\pgfsetfillcolor{currentfill}%
\pgfsetfillopacity{0.700000}%
\pgfsetlinewidth{0.000000pt}%
\definecolor{currentstroke}{rgb}{0.000000,0.000000,0.000000}%
\pgfsetstrokecolor{currentstroke}%
\pgfsetdash{}{0pt}%
\pgfpathmoveto{\pgfqpoint{4.716227in}{2.362949in}}%
\pgfpathlineto{\pgfqpoint{4.730125in}{2.362847in}}%
\pgfpathlineto{\pgfqpoint{4.744033in}{2.362816in}}%
\pgfpathlineto{\pgfqpoint{4.757950in}{2.362856in}}%
\pgfpathlineto{\pgfqpoint{4.771876in}{2.362967in}}%
\pgfpathlineto{\pgfqpoint{4.779529in}{2.370277in}}%
\pgfpathlineto{\pgfqpoint{4.787176in}{2.377612in}}%
\pgfpathlineto{\pgfqpoint{4.794818in}{2.384977in}}%
\pgfpathlineto{\pgfqpoint{4.802455in}{2.392375in}}%
\pgfpathlineto{\pgfqpoint{4.788544in}{2.392489in}}%
\pgfpathlineto{\pgfqpoint{4.774642in}{2.392675in}}%
\pgfpathlineto{\pgfqpoint{4.760749in}{2.392931in}}%
\pgfpathlineto{\pgfqpoint{4.746865in}{2.393259in}}%
\pgfpathlineto{\pgfqpoint{4.739214in}{2.385628in}}%
\pgfpathlineto{\pgfqpoint{4.731557in}{2.378035in}}%
\pgfpathlineto{\pgfqpoint{4.723894in}{2.370477in}}%
\pgfpathlineto{\pgfqpoint{4.716227in}{2.362949in}}%
\pgfpathclose%
\pgfusepath{fill}%
\end{pgfscope}%
\begin{pgfscope}%
\pgfpathrectangle{\pgfqpoint{1.150000in}{0.150000in}}{\pgfqpoint{5.700000in}{5.700000in}}%
\pgfusepath{clip}%
\pgfsetbuttcap%
\pgfsetroundjoin%
\definecolor{currentfill}{rgb}{0.282656,0.100196,0.422160}%
\pgfsetfillcolor{currentfill}%
\pgfsetfillopacity{0.700000}%
\pgfsetlinewidth{0.000000pt}%
\definecolor{currentstroke}{rgb}{0.000000,0.000000,0.000000}%
\pgfsetstrokecolor{currentstroke}%
\pgfsetdash{}{0pt}%
\pgfpathmoveto{\pgfqpoint{4.001372in}{2.166771in}}%
\pgfpathlineto{\pgfqpoint{4.015059in}{2.164986in}}%
\pgfpathlineto{\pgfqpoint{4.028753in}{2.163280in}}%
\pgfpathlineto{\pgfqpoint{4.042455in}{2.161651in}}%
\pgfpathlineto{\pgfqpoint{4.056165in}{2.160100in}}%
\pgfpathlineto{\pgfqpoint{4.064090in}{2.168713in}}%
\pgfpathlineto{\pgfqpoint{4.072009in}{2.177317in}}%
\pgfpathlineto{\pgfqpoint{4.079923in}{2.185911in}}%
\pgfpathlineto{\pgfqpoint{4.087831in}{2.194499in}}%
\pgfpathlineto{\pgfqpoint{4.074133in}{2.196131in}}%
\pgfpathlineto{\pgfqpoint{4.060442in}{2.197841in}}%
\pgfpathlineto{\pgfqpoint{4.046758in}{2.199628in}}%
\pgfpathlineto{\pgfqpoint{4.033082in}{2.201494in}}%
\pgfpathlineto{\pgfqpoint{4.025163in}{2.192818in}}%
\pgfpathlineto{\pgfqpoint{4.017238in}{2.184140in}}%
\pgfpathlineto{\pgfqpoint{4.009308in}{2.175459in}}%
\pgfpathlineto{\pgfqpoint{4.001372in}{2.166771in}}%
\pgfpathclose%
\pgfusepath{fill}%
\end{pgfscope}%
\begin{pgfscope}%
\pgfpathrectangle{\pgfqpoint{1.150000in}{0.150000in}}{\pgfqpoint{5.700000in}{5.700000in}}%
\pgfusepath{clip}%
\pgfsetbuttcap%
\pgfsetroundjoin%
\definecolor{currentfill}{rgb}{0.276022,0.044167,0.370164}%
\pgfsetfillcolor{currentfill}%
\pgfsetfillopacity{0.700000}%
\pgfsetlinewidth{0.000000pt}%
\definecolor{currentstroke}{rgb}{0.000000,0.000000,0.000000}%
\pgfsetstrokecolor{currentstroke}%
\pgfsetdash{}{0pt}%
\pgfpathmoveto{\pgfqpoint{3.459326in}{2.069983in}}%
\pgfpathlineto{\pgfqpoint{3.472895in}{2.065987in}}%
\pgfpathlineto{\pgfqpoint{3.486470in}{2.062079in}}%
\pgfpathlineto{\pgfqpoint{3.500050in}{2.058257in}}%
\pgfpathlineto{\pgfqpoint{3.513635in}{2.054521in}}%
\pgfpathlineto{\pgfqpoint{3.521753in}{2.063391in}}%
\pgfpathlineto{\pgfqpoint{3.529864in}{2.072267in}}%
\pgfpathlineto{\pgfqpoint{3.537970in}{2.081151in}}%
\pgfpathlineto{\pgfqpoint{3.546069in}{2.090041in}}%
\pgfpathlineto{\pgfqpoint{3.532496in}{2.093755in}}%
\pgfpathlineto{\pgfqpoint{3.518928in}{2.097555in}}%
\pgfpathlineto{\pgfqpoint{3.505365in}{2.101442in}}%
\pgfpathlineto{\pgfqpoint{3.491808in}{2.105416in}}%
\pgfpathlineto{\pgfqpoint{3.483696in}{2.096539in}}%
\pgfpathlineto{\pgfqpoint{3.475579in}{2.087675in}}%
\pgfpathlineto{\pgfqpoint{3.467455in}{2.078823in}}%
\pgfpathlineto{\pgfqpoint{3.459326in}{2.069983in}}%
\pgfpathclose%
\pgfusepath{fill}%
\end{pgfscope}%
\begin{pgfscope}%
\pgfpathrectangle{\pgfqpoint{1.150000in}{0.150000in}}{\pgfqpoint{5.700000in}{5.700000in}}%
\pgfusepath{clip}%
\pgfsetbuttcap%
\pgfsetroundjoin%
\definecolor{currentfill}{rgb}{0.282623,0.140926,0.457517}%
\pgfsetfillcolor{currentfill}%
\pgfsetfillopacity{0.700000}%
\pgfsetlinewidth{0.000000pt}%
\definecolor{currentstroke}{rgb}{0.000000,0.000000,0.000000}%
\pgfsetstrokecolor{currentstroke}%
\pgfsetdash{}{0pt}%
\pgfpathmoveto{\pgfqpoint{4.315597in}{2.246899in}}%
\pgfpathlineto{\pgfqpoint{4.329374in}{2.246033in}}%
\pgfpathlineto{\pgfqpoint{4.343159in}{2.245240in}}%
\pgfpathlineto{\pgfqpoint{4.356953in}{2.244523in}}%
\pgfpathlineto{\pgfqpoint{4.370754in}{2.243879in}}%
\pgfpathlineto{\pgfqpoint{4.378565in}{2.251961in}}%
\pgfpathlineto{\pgfqpoint{4.386370in}{2.260039in}}%
\pgfpathlineto{\pgfqpoint{4.394169in}{2.268116in}}%
\pgfpathlineto{\pgfqpoint{4.401962in}{2.276194in}}%
\pgfpathlineto{\pgfqpoint{4.388173in}{2.276980in}}%
\pgfpathlineto{\pgfqpoint{4.374392in}{2.277841in}}%
\pgfpathlineto{\pgfqpoint{4.360619in}{2.278776in}}%
\pgfpathlineto{\pgfqpoint{4.346854in}{2.279785in}}%
\pgfpathlineto{\pgfqpoint{4.339048in}{2.271557in}}%
\pgfpathlineto{\pgfqpoint{4.331237in}{2.263335in}}%
\pgfpathlineto{\pgfqpoint{4.323420in}{2.255117in}}%
\pgfpathlineto{\pgfqpoint{4.315597in}{2.246899in}}%
\pgfpathclose%
\pgfusepath{fill}%
\end{pgfscope}%
\begin{pgfscope}%
\pgfpathrectangle{\pgfqpoint{1.150000in}{0.150000in}}{\pgfqpoint{5.700000in}{5.700000in}}%
\pgfusepath{clip}%
\pgfsetbuttcap%
\pgfsetroundjoin%
\definecolor{currentfill}{rgb}{0.221989,0.339161,0.548752}%
\pgfsetfillcolor{currentfill}%
\pgfsetfillopacity{0.700000}%
\pgfsetlinewidth{0.000000pt}%
\definecolor{currentstroke}{rgb}{0.000000,0.000000,0.000000}%
\pgfsetstrokecolor{currentstroke}%
\pgfsetdash{}{0pt}%
\pgfpathmoveto{\pgfqpoint{5.831914in}{2.688100in}}%
\pgfpathlineto{\pgfqpoint{5.846151in}{2.688204in}}%
\pgfpathlineto{\pgfqpoint{5.860399in}{2.688374in}}%
\pgfpathlineto{\pgfqpoint{5.874657in}{2.688609in}}%
\pgfpathlineto{\pgfqpoint{5.888926in}{2.688910in}}%
\pgfpathlineto{\pgfqpoint{5.896156in}{2.696194in}}%
\pgfpathlineto{\pgfqpoint{5.903389in}{2.703697in}}%
\pgfpathlineto{\pgfqpoint{5.910623in}{2.711426in}}%
\pgfpathlineto{\pgfqpoint{5.917861in}{2.719391in}}%
\pgfpathlineto{\pgfqpoint{5.903620in}{2.719541in}}%
\pgfpathlineto{\pgfqpoint{5.889390in}{2.719756in}}%
\pgfpathlineto{\pgfqpoint{5.875170in}{2.720037in}}%
\pgfpathlineto{\pgfqpoint{5.860961in}{2.720383in}}%
\pgfpathlineto{\pgfqpoint{5.853695in}{2.711961in}}%
\pgfpathlineto{\pgfqpoint{5.846432in}{2.703779in}}%
\pgfpathlineto{\pgfqpoint{5.839172in}{2.695828in}}%
\pgfpathlineto{\pgfqpoint{5.831914in}{2.688100in}}%
\pgfpathclose%
\pgfusepath{fill}%
\end{pgfscope}%
\begin{pgfscope}%
\pgfpathrectangle{\pgfqpoint{1.150000in}{0.150000in}}{\pgfqpoint{5.700000in}{5.700000in}}%
\pgfusepath{clip}%
\pgfsetbuttcap%
\pgfsetroundjoin%
\definecolor{currentfill}{rgb}{0.279574,0.170599,0.479997}%
\pgfsetfillcolor{currentfill}%
\pgfsetfillopacity{0.700000}%
\pgfsetlinewidth{0.000000pt}%
\definecolor{currentstroke}{rgb}{0.000000,0.000000,0.000000}%
\pgfsetstrokecolor{currentstroke}%
\pgfsetdash{}{0pt}%
\pgfpathmoveto{\pgfqpoint{2.359821in}{2.338247in}}%
\pgfpathlineto{\pgfqpoint{2.373395in}{2.326675in}}%
\pgfpathlineto{\pgfqpoint{2.386967in}{2.315237in}}%
\pgfpathlineto{\pgfqpoint{2.400536in}{2.303933in}}%
\pgfpathlineto{\pgfqpoint{2.414104in}{2.292761in}}%
\pgfpathlineto{\pgfqpoint{2.422711in}{2.298358in}}%
\pgfpathlineto{\pgfqpoint{2.431306in}{2.304075in}}%
\pgfpathlineto{\pgfqpoint{2.439890in}{2.309910in}}%
\pgfpathlineto{\pgfqpoint{2.448464in}{2.315861in}}%
\pgfpathlineto{\pgfqpoint{2.434920in}{2.326863in}}%
\pgfpathlineto{\pgfqpoint{2.421374in}{2.337997in}}%
\pgfpathlineto{\pgfqpoint{2.407826in}{2.349264in}}%
\pgfpathlineto{\pgfqpoint{2.394277in}{2.360666in}}%
\pgfpathlineto{\pgfqpoint{2.385680in}{2.354877in}}%
\pgfpathlineto{\pgfqpoint{2.377071in}{2.349210in}}%
\pgfpathlineto{\pgfqpoint{2.368452in}{2.343665in}}%
\pgfpathlineto{\pgfqpoint{2.359821in}{2.338247in}}%
\pgfpathclose%
\pgfusepath{fill}%
\end{pgfscope}%
\begin{pgfscope}%
\pgfpathrectangle{\pgfqpoint{1.150000in}{0.150000in}}{\pgfqpoint{5.700000in}{5.700000in}}%
\pgfusepath{clip}%
\pgfsetbuttcap%
\pgfsetroundjoin%
\definecolor{currentfill}{rgb}{0.278791,0.062145,0.386592}%
\pgfsetfillcolor{currentfill}%
\pgfsetfillopacity{0.700000}%
\pgfsetlinewidth{0.000000pt}%
\definecolor{currentstroke}{rgb}{0.000000,0.000000,0.000000}%
\pgfsetstrokecolor{currentstroke}%
\pgfsetdash{}{0pt}%
\pgfpathmoveto{\pgfqpoint{3.687107in}{2.099043in}}%
\pgfpathlineto{\pgfqpoint{3.700720in}{2.096091in}}%
\pgfpathlineto{\pgfqpoint{3.714340in}{2.093221in}}%
\pgfpathlineto{\pgfqpoint{3.727966in}{2.090433in}}%
\pgfpathlineto{\pgfqpoint{3.741598in}{2.087727in}}%
\pgfpathlineto{\pgfqpoint{3.749634in}{2.096625in}}%
\pgfpathlineto{\pgfqpoint{3.757666in}{2.105518in}}%
\pgfpathlineto{\pgfqpoint{3.765691in}{2.114407in}}%
\pgfpathlineto{\pgfqpoint{3.773711in}{2.123292in}}%
\pgfpathlineto{\pgfqpoint{3.760090in}{2.126017in}}%
\pgfpathlineto{\pgfqpoint{3.746475in}{2.128824in}}%
\pgfpathlineto{\pgfqpoint{3.732867in}{2.131714in}}%
\pgfpathlineto{\pgfqpoint{3.719264in}{2.134686in}}%
\pgfpathlineto{\pgfqpoint{3.711233in}{2.125774in}}%
\pgfpathlineto{\pgfqpoint{3.703197in}{2.116863in}}%
\pgfpathlineto{\pgfqpoint{3.695155in}{2.107953in}}%
\pgfpathlineto{\pgfqpoint{3.687107in}{2.099043in}}%
\pgfpathclose%
\pgfusepath{fill}%
\end{pgfscope}%
\begin{pgfscope}%
\pgfpathrectangle{\pgfqpoint{1.150000in}{0.150000in}}{\pgfqpoint{5.700000in}{5.700000in}}%
\pgfusepath{clip}%
\pgfsetbuttcap%
\pgfsetroundjoin%
\definecolor{currentfill}{rgb}{0.244972,0.287675,0.537260}%
\pgfsetfillcolor{currentfill}%
\pgfsetfillopacity{0.700000}%
\pgfsetlinewidth{0.000000pt}%
\definecolor{currentstroke}{rgb}{0.000000,0.000000,0.000000}%
\pgfsetstrokecolor{currentstroke}%
\pgfsetdash{}{0pt}%
\pgfpathmoveto{\pgfqpoint{5.431280in}{2.568497in}}%
\pgfpathlineto{\pgfqpoint{5.445405in}{2.568873in}}%
\pgfpathlineto{\pgfqpoint{5.459540in}{2.569317in}}%
\pgfpathlineto{\pgfqpoint{5.473686in}{2.569828in}}%
\pgfpathlineto{\pgfqpoint{5.487842in}{2.570406in}}%
\pgfpathlineto{\pgfqpoint{5.495206in}{2.577017in}}%
\pgfpathlineto{\pgfqpoint{5.502568in}{2.583757in}}%
\pgfpathlineto{\pgfqpoint{5.509927in}{2.590632in}}%
\pgfpathlineto{\pgfqpoint{5.517285in}{2.597648in}}%
\pgfpathlineto{\pgfqpoint{5.503152in}{2.597440in}}%
\pgfpathlineto{\pgfqpoint{5.489029in}{2.597298in}}%
\pgfpathlineto{\pgfqpoint{5.474916in}{2.597224in}}%
\pgfpathlineto{\pgfqpoint{5.460814in}{2.597217in}}%
\pgfpathlineto{\pgfqpoint{5.453433in}{2.589824in}}%
\pgfpathlineto{\pgfqpoint{5.446051in}{2.582578in}}%
\pgfpathlineto{\pgfqpoint{5.438666in}{2.575470in}}%
\pgfpathlineto{\pgfqpoint{5.431280in}{2.568497in}}%
\pgfpathclose%
\pgfusepath{fill}%
\end{pgfscope}%
\begin{pgfscope}%
\pgfpathrectangle{\pgfqpoint{1.150000in}{0.150000in}}{\pgfqpoint{5.700000in}{5.700000in}}%
\pgfusepath{clip}%
\pgfsetbuttcap%
\pgfsetroundjoin%
\definecolor{currentfill}{rgb}{0.274952,0.037752,0.364543}%
\pgfsetfillcolor{currentfill}%
\pgfsetfillopacity{0.700000}%
\pgfsetlinewidth{0.000000pt}%
\definecolor{currentstroke}{rgb}{0.000000,0.000000,0.000000}%
\pgfsetstrokecolor{currentstroke}%
\pgfsetdash{}{0pt}%
\pgfpathmoveto{\pgfqpoint{3.090058in}{2.067556in}}%
\pgfpathlineto{\pgfqpoint{3.103584in}{2.061535in}}%
\pgfpathlineto{\pgfqpoint{3.117114in}{2.055610in}}%
\pgfpathlineto{\pgfqpoint{3.130647in}{2.049782in}}%
\pgfpathlineto{\pgfqpoint{3.144183in}{2.044049in}}%
\pgfpathlineto{\pgfqpoint{3.152442in}{2.052360in}}%
\pgfpathlineto{\pgfqpoint{3.160693in}{2.060708in}}%
\pgfpathlineto{\pgfqpoint{3.168938in}{2.069093in}}%
\pgfpathlineto{\pgfqpoint{3.177177in}{2.077515in}}%
\pgfpathlineto{\pgfqpoint{3.163655in}{2.083164in}}%
\pgfpathlineto{\pgfqpoint{3.150137in}{2.088909in}}%
\pgfpathlineto{\pgfqpoint{3.136622in}{2.094750in}}%
\pgfpathlineto{\pgfqpoint{3.123110in}{2.100688in}}%
\pgfpathlineto{\pgfqpoint{3.114858in}{2.092343in}}%
\pgfpathlineto{\pgfqpoint{3.106598in}{2.084038in}}%
\pgfpathlineto{\pgfqpoint{3.098331in}{2.075776in}}%
\pgfpathlineto{\pgfqpoint{3.090058in}{2.067556in}}%
\pgfpathclose%
\pgfusepath{fill}%
\end{pgfscope}%
\begin{pgfscope}%
\pgfpathrectangle{\pgfqpoint{1.150000in}{0.150000in}}{\pgfqpoint{5.700000in}{5.700000in}}%
\pgfusepath{clip}%
\pgfsetbuttcap%
\pgfsetroundjoin%
\definecolor{currentfill}{rgb}{0.277018,0.050344,0.375715}%
\pgfsetfillcolor{currentfill}%
\pgfsetfillopacity{0.700000}%
\pgfsetlinewidth{0.000000pt}%
\definecolor{currentstroke}{rgb}{0.000000,0.000000,0.000000}%
\pgfsetstrokecolor{currentstroke}%
\pgfsetdash{}{0pt}%
\pgfpathmoveto{\pgfqpoint{2.948662in}{2.087683in}}%
\pgfpathlineto{\pgfqpoint{2.962181in}{2.080762in}}%
\pgfpathlineto{\pgfqpoint{2.975704in}{2.073942in}}%
\pgfpathlineto{\pgfqpoint{2.989228in}{2.067224in}}%
\pgfpathlineto{\pgfqpoint{3.002756in}{2.060606in}}%
\pgfpathlineto{\pgfqpoint{3.011074in}{2.068532in}}%
\pgfpathlineto{\pgfqpoint{3.019384in}{2.076511in}}%
\pgfpathlineto{\pgfqpoint{3.027687in}{2.084541in}}%
\pgfpathlineto{\pgfqpoint{3.035983in}{2.092621in}}%
\pgfpathlineto{\pgfqpoint{3.022472in}{2.099135in}}%
\pgfpathlineto{\pgfqpoint{3.008963in}{2.105750in}}%
\pgfpathlineto{\pgfqpoint{2.995457in}{2.112465in}}%
\pgfpathlineto{\pgfqpoint{2.981954in}{2.119282in}}%
\pgfpathlineto{\pgfqpoint{2.973642in}{2.111299in}}%
\pgfpathlineto{\pgfqpoint{2.965323in}{2.103371in}}%
\pgfpathlineto{\pgfqpoint{2.956996in}{2.095498in}}%
\pgfpathlineto{\pgfqpoint{2.948662in}{2.087683in}}%
\pgfpathclose%
\pgfusepath{fill}%
\end{pgfscope}%
\begin{pgfscope}%
\pgfpathrectangle{\pgfqpoint{1.150000in}{0.150000in}}{\pgfqpoint{5.700000in}{5.700000in}}%
\pgfusepath{clip}%
\pgfsetbuttcap%
\pgfsetroundjoin%
\definecolor{currentfill}{rgb}{0.282910,0.105393,0.426902}%
\pgfsetfillcolor{currentfill}%
\pgfsetfillopacity{0.700000}%
\pgfsetlinewidth{0.000000pt}%
\definecolor{currentstroke}{rgb}{0.000000,0.000000,0.000000}%
\pgfsetstrokecolor{currentstroke}%
\pgfsetdash{}{0pt}%
\pgfpathmoveto{\pgfqpoint{2.610904in}{2.193733in}}%
\pgfpathlineto{\pgfqpoint{2.624438in}{2.184349in}}%
\pgfpathlineto{\pgfqpoint{2.637971in}{2.175082in}}%
\pgfpathlineto{\pgfqpoint{2.651504in}{2.165932in}}%
\pgfpathlineto{\pgfqpoint{2.665038in}{2.156898in}}%
\pgfpathlineto{\pgfqpoint{2.673514in}{2.163584in}}%
\pgfpathlineto{\pgfqpoint{2.681980in}{2.170362in}}%
\pgfpathlineto{\pgfqpoint{2.690438in}{2.177229in}}%
\pgfpathlineto{\pgfqpoint{2.698886in}{2.184182in}}%
\pgfpathlineto{\pgfqpoint{2.685373in}{2.193070in}}%
\pgfpathlineto{\pgfqpoint{2.671859in}{2.202073in}}%
\pgfpathlineto{\pgfqpoint{2.658347in}{2.211193in}}%
\pgfpathlineto{\pgfqpoint{2.644834in}{2.220430in}}%
\pgfpathlineto{\pgfqpoint{2.636366in}{2.213615in}}%
\pgfpathlineto{\pgfqpoint{2.627888in}{2.206893in}}%
\pgfpathlineto{\pgfqpoint{2.619401in}{2.200265in}}%
\pgfpathlineto{\pgfqpoint{2.610904in}{2.193733in}}%
\pgfpathclose%
\pgfusepath{fill}%
\end{pgfscope}%
\begin{pgfscope}%
\pgfpathrectangle{\pgfqpoint{1.150000in}{0.150000in}}{\pgfqpoint{5.700000in}{5.700000in}}%
\pgfusepath{clip}%
\pgfsetbuttcap%
\pgfsetroundjoin%
\definecolor{currentfill}{rgb}{0.263663,0.237631,0.518762}%
\pgfsetfillcolor{currentfill}%
\pgfsetfillopacity{0.700000}%
\pgfsetlinewidth{0.000000pt}%
\definecolor{currentstroke}{rgb}{0.000000,0.000000,0.000000}%
\pgfsetstrokecolor{currentstroke}%
\pgfsetdash{}{0pt}%
\pgfpathmoveto{\pgfqpoint{5.030651in}{2.451270in}}%
\pgfpathlineto{\pgfqpoint{5.044653in}{2.451554in}}%
\pgfpathlineto{\pgfqpoint{5.058666in}{2.451908in}}%
\pgfpathlineto{\pgfqpoint{5.072688in}{2.452331in}}%
\pgfpathlineto{\pgfqpoint{5.086720in}{2.452823in}}%
\pgfpathlineto{\pgfqpoint{5.094245in}{2.459609in}}%
\pgfpathlineto{\pgfqpoint{5.101765in}{2.466455in}}%
\pgfpathlineto{\pgfqpoint{5.109281in}{2.473367in}}%
\pgfpathlineto{\pgfqpoint{5.116792in}{2.480350in}}%
\pgfpathlineto{\pgfqpoint{5.102779in}{2.480146in}}%
\pgfpathlineto{\pgfqpoint{5.088775in}{2.480010in}}%
\pgfpathlineto{\pgfqpoint{5.074780in}{2.479944in}}%
\pgfpathlineto{\pgfqpoint{5.060796in}{2.479947in}}%
\pgfpathlineto{\pgfqpoint{5.053266in}{2.472669in}}%
\pgfpathlineto{\pgfqpoint{5.045732in}{2.465467in}}%
\pgfpathlineto{\pgfqpoint{5.038194in}{2.458336in}}%
\pgfpathlineto{\pgfqpoint{5.030651in}{2.451270in}}%
\pgfpathclose%
\pgfusepath{fill}%
\end{pgfscope}%
\begin{pgfscope}%
\pgfpathrectangle{\pgfqpoint{1.150000in}{0.150000in}}{\pgfqpoint{5.700000in}{5.700000in}}%
\pgfusepath{clip}%
\pgfsetbuttcap%
\pgfsetroundjoin%
\definecolor{currentfill}{rgb}{0.277134,0.185228,0.489898}%
\pgfsetfillcolor{currentfill}%
\pgfsetfillopacity{0.700000}%
\pgfsetlinewidth{0.000000pt}%
\definecolor{currentstroke}{rgb}{0.000000,0.000000,0.000000}%
\pgfsetstrokecolor{currentstroke}%
\pgfsetdash{}{0pt}%
\pgfpathmoveto{\pgfqpoint{4.629942in}{2.333366in}}%
\pgfpathlineto{\pgfqpoint{4.643819in}{2.333181in}}%
\pgfpathlineto{\pgfqpoint{4.657704in}{2.333069in}}%
\pgfpathlineto{\pgfqpoint{4.671598in}{2.333029in}}%
\pgfpathlineto{\pgfqpoint{4.685502in}{2.333060in}}%
\pgfpathlineto{\pgfqpoint{4.693192in}{2.340506in}}%
\pgfpathlineto{\pgfqpoint{4.700876in}{2.347968in}}%
\pgfpathlineto{\pgfqpoint{4.708554in}{2.355447in}}%
\pgfpathlineto{\pgfqpoint{4.716227in}{2.362949in}}%
\pgfpathlineto{\pgfqpoint{4.702338in}{2.363123in}}%
\pgfpathlineto{\pgfqpoint{4.688457in}{2.363369in}}%
\pgfpathlineto{\pgfqpoint{4.674586in}{2.363686in}}%
\pgfpathlineto{\pgfqpoint{4.660724in}{2.364075in}}%
\pgfpathlineto{\pgfqpoint{4.653037in}{2.356361in}}%
\pgfpathlineto{\pgfqpoint{4.645344in}{2.348674in}}%
\pgfpathlineto{\pgfqpoint{4.637646in}{2.341010in}}%
\pgfpathlineto{\pgfqpoint{4.629942in}{2.333366in}}%
\pgfpathclose%
\pgfusepath{fill}%
\end{pgfscope}%
\begin{pgfscope}%
\pgfpathrectangle{\pgfqpoint{1.150000in}{0.150000in}}{\pgfqpoint{5.700000in}{5.700000in}}%
\pgfusepath{clip}%
\pgfsetbuttcap%
\pgfsetroundjoin%
\definecolor{currentfill}{rgb}{0.274952,0.037752,0.364543}%
\pgfsetfillcolor{currentfill}%
\pgfsetfillopacity{0.700000}%
\pgfsetlinewidth{0.000000pt}%
\definecolor{currentstroke}{rgb}{0.000000,0.000000,0.000000}%
\pgfsetstrokecolor{currentstroke}%
\pgfsetdash{}{0pt}%
\pgfpathmoveto{\pgfqpoint{3.231301in}{2.055862in}}%
\pgfpathlineto{\pgfqpoint{3.244842in}{2.050683in}}%
\pgfpathlineto{\pgfqpoint{3.258386in}{2.045596in}}%
\pgfpathlineto{\pgfqpoint{3.271935in}{2.040602in}}%
\pgfpathlineto{\pgfqpoint{3.285488in}{2.035698in}}%
\pgfpathlineto{\pgfqpoint{3.293692in}{2.044295in}}%
\pgfpathlineto{\pgfqpoint{3.301890in}{2.052917in}}%
\pgfpathlineto{\pgfqpoint{3.310082in}{2.061562in}}%
\pgfpathlineto{\pgfqpoint{3.318267in}{2.070231in}}%
\pgfpathlineto{\pgfqpoint{3.304727in}{2.075071in}}%
\pgfpathlineto{\pgfqpoint{3.291192in}{2.080004in}}%
\pgfpathlineto{\pgfqpoint{3.277660in}{2.085028in}}%
\pgfpathlineto{\pgfqpoint{3.264133in}{2.090144in}}%
\pgfpathlineto{\pgfqpoint{3.255935in}{2.081531in}}%
\pgfpathlineto{\pgfqpoint{3.247730in}{2.072945in}}%
\pgfpathlineto{\pgfqpoint{3.239519in}{2.064389in}}%
\pgfpathlineto{\pgfqpoint{3.231301in}{2.055862in}}%
\pgfpathclose%
\pgfusepath{fill}%
\end{pgfscope}%
\begin{pgfscope}%
\pgfpathrectangle{\pgfqpoint{1.150000in}{0.150000in}}{\pgfqpoint{5.700000in}{5.700000in}}%
\pgfusepath{clip}%
\pgfsetbuttcap%
\pgfsetroundjoin%
\definecolor{currentfill}{rgb}{0.281924,0.089666,0.412415}%
\pgfsetfillcolor{currentfill}%
\pgfsetfillopacity{0.700000}%
\pgfsetlinewidth{0.000000pt}%
\definecolor{currentstroke}{rgb}{0.000000,0.000000,0.000000}%
\pgfsetstrokecolor{currentstroke}%
\pgfsetdash{}{0pt}%
\pgfpathmoveto{\pgfqpoint{3.914849in}{2.139618in}}%
\pgfpathlineto{\pgfqpoint{3.928519in}{2.137579in}}%
\pgfpathlineto{\pgfqpoint{3.942196in}{2.135618in}}%
\pgfpathlineto{\pgfqpoint{3.955879in}{2.133737in}}%
\pgfpathlineto{\pgfqpoint{3.969571in}{2.131934in}}%
\pgfpathlineto{\pgfqpoint{3.977529in}{2.140660in}}%
\pgfpathlineto{\pgfqpoint{3.985482in}{2.149373in}}%
\pgfpathlineto{\pgfqpoint{3.993430in}{2.158077in}}%
\pgfpathlineto{\pgfqpoint{4.001372in}{2.166771in}}%
\pgfpathlineto{\pgfqpoint{3.987691in}{2.168635in}}%
\pgfpathlineto{\pgfqpoint{3.974018in}{2.170577in}}%
\pgfpathlineto{\pgfqpoint{3.960353in}{2.172598in}}%
\pgfpathlineto{\pgfqpoint{3.946694in}{2.174698in}}%
\pgfpathlineto{\pgfqpoint{3.938741in}{2.165935in}}%
\pgfpathlineto{\pgfqpoint{3.930783in}{2.157169in}}%
\pgfpathlineto{\pgfqpoint{3.922819in}{2.148397in}}%
\pgfpathlineto{\pgfqpoint{3.914849in}{2.139618in}}%
\pgfpathclose%
\pgfusepath{fill}%
\end{pgfscope}%
\begin{pgfscope}%
\pgfpathrectangle{\pgfqpoint{1.150000in}{0.150000in}}{\pgfqpoint{5.700000in}{5.700000in}}%
\pgfusepath{clip}%
\pgfsetbuttcap%
\pgfsetroundjoin%
\definecolor{currentfill}{rgb}{0.279566,0.067836,0.391917}%
\pgfsetfillcolor{currentfill}%
\pgfsetfillopacity{0.700000}%
\pgfsetlinewidth{0.000000pt}%
\definecolor{currentstroke}{rgb}{0.000000,0.000000,0.000000}%
\pgfsetstrokecolor{currentstroke}%
\pgfsetdash{}{0pt}%
\pgfpathmoveto{\pgfqpoint{2.807022in}{2.117141in}}%
\pgfpathlineto{\pgfqpoint{2.820544in}{2.109255in}}%
\pgfpathlineto{\pgfqpoint{2.834068in}{2.101478in}}%
\pgfpathlineto{\pgfqpoint{2.847593in}{2.093807in}}%
\pgfpathlineto{\pgfqpoint{2.861120in}{2.086242in}}%
\pgfpathlineto{\pgfqpoint{2.869503in}{2.093680in}}%
\pgfpathlineto{\pgfqpoint{2.877878in}{2.101187in}}%
\pgfpathlineto{\pgfqpoint{2.886245in}{2.108761in}}%
\pgfpathlineto{\pgfqpoint{2.894604in}{2.116400in}}%
\pgfpathlineto{\pgfqpoint{2.881095in}{2.123840in}}%
\pgfpathlineto{\pgfqpoint{2.867588in}{2.131386in}}%
\pgfpathlineto{\pgfqpoint{2.854082in}{2.139039in}}%
\pgfpathlineto{\pgfqpoint{2.840578in}{2.146798in}}%
\pgfpathlineto{\pgfqpoint{2.832201in}{2.139277in}}%
\pgfpathlineto{\pgfqpoint{2.823816in}{2.131826in}}%
\pgfpathlineto{\pgfqpoint{2.815423in}{2.124446in}}%
\pgfpathlineto{\pgfqpoint{2.807022in}{2.117141in}}%
\pgfpathclose%
\pgfusepath{fill}%
\end{pgfscope}%
\begin{pgfscope}%
\pgfpathrectangle{\pgfqpoint{1.150000in}{0.150000in}}{\pgfqpoint{5.700000in}{5.700000in}}%
\pgfusepath{clip}%
\pgfsetbuttcap%
\pgfsetroundjoin%
\definecolor{currentfill}{rgb}{0.225863,0.330805,0.547314}%
\pgfsetfillcolor{currentfill}%
\pgfsetfillopacity{0.700000}%
\pgfsetlinewidth{0.000000pt}%
\definecolor{currentstroke}{rgb}{0.000000,0.000000,0.000000}%
\pgfsetstrokecolor{currentstroke}%
\pgfsetdash{}{0pt}%
\pgfpathmoveto{\pgfqpoint{5.745950in}{2.657788in}}%
\pgfpathlineto{\pgfqpoint{5.760171in}{2.658060in}}%
\pgfpathlineto{\pgfqpoint{5.774403in}{2.658398in}}%
\pgfpathlineto{\pgfqpoint{5.788645in}{2.658801in}}%
\pgfpathlineto{\pgfqpoint{5.802898in}{2.659271in}}%
\pgfpathlineto{\pgfqpoint{5.810151in}{2.666181in}}%
\pgfpathlineto{\pgfqpoint{5.817404in}{2.673285in}}%
\pgfpathlineto{\pgfqpoint{5.824658in}{2.680588in}}%
\pgfpathlineto{\pgfqpoint{5.831914in}{2.688100in}}%
\pgfpathlineto{\pgfqpoint{5.817688in}{2.688062in}}%
\pgfpathlineto{\pgfqpoint{5.803473in}{2.688089in}}%
\pgfpathlineto{\pgfqpoint{5.789268in}{2.688182in}}%
\pgfpathlineto{\pgfqpoint{5.775074in}{2.688340in}}%
\pgfpathlineto{\pgfqpoint{5.767791in}{2.680391in}}%
\pgfpathlineto{\pgfqpoint{5.760509in}{2.672654in}}%
\pgfpathlineto{\pgfqpoint{5.753229in}{2.665122in}}%
\pgfpathlineto{\pgfqpoint{5.745950in}{2.657788in}}%
\pgfpathclose%
\pgfusepath{fill}%
\end{pgfscope}%
\begin{pgfscope}%
\pgfpathrectangle{\pgfqpoint{1.150000in}{0.150000in}}{\pgfqpoint{5.700000in}{5.700000in}}%
\pgfusepath{clip}%
\pgfsetbuttcap%
\pgfsetroundjoin%
\definecolor{currentfill}{rgb}{0.283187,0.125848,0.444960}%
\pgfsetfillcolor{currentfill}%
\pgfsetfillopacity{0.700000}%
\pgfsetlinewidth{0.000000pt}%
\definecolor{currentstroke}{rgb}{0.000000,0.000000,0.000000}%
\pgfsetstrokecolor{currentstroke}%
\pgfsetdash{}{0pt}%
\pgfpathmoveto{\pgfqpoint{4.229176in}{2.217715in}}%
\pgfpathlineto{\pgfqpoint{4.242933in}{2.216670in}}%
\pgfpathlineto{\pgfqpoint{4.256697in}{2.215701in}}%
\pgfpathlineto{\pgfqpoint{4.270469in}{2.214806in}}%
\pgfpathlineto{\pgfqpoint{4.284250in}{2.213987in}}%
\pgfpathlineto{\pgfqpoint{4.292095in}{2.222227in}}%
\pgfpathlineto{\pgfqpoint{4.299935in}{2.230457in}}%
\pgfpathlineto{\pgfqpoint{4.307769in}{2.238680in}}%
\pgfpathlineto{\pgfqpoint{4.315597in}{2.246899in}}%
\pgfpathlineto{\pgfqpoint{4.301829in}{2.247841in}}%
\pgfpathlineto{\pgfqpoint{4.288068in}{2.248858in}}%
\pgfpathlineto{\pgfqpoint{4.274315in}{2.249950in}}%
\pgfpathlineto{\pgfqpoint{4.260571in}{2.251117in}}%
\pgfpathlineto{\pgfqpoint{4.252731in}{2.242768in}}%
\pgfpathlineto{\pgfqpoint{4.244885in}{2.234420in}}%
\pgfpathlineto{\pgfqpoint{4.237033in}{2.226070in}}%
\pgfpathlineto{\pgfqpoint{4.229176in}{2.217715in}}%
\pgfpathclose%
\pgfusepath{fill}%
\end{pgfscope}%
\begin{pgfscope}%
\pgfpathrectangle{\pgfqpoint{1.150000in}{0.150000in}}{\pgfqpoint{5.700000in}{5.700000in}}%
\pgfusepath{clip}%
\pgfsetbuttcap%
\pgfsetroundjoin%
\definecolor{currentfill}{rgb}{0.248629,0.278775,0.534556}%
\pgfsetfillcolor{currentfill}%
\pgfsetfillopacity{0.700000}%
\pgfsetlinewidth{0.000000pt}%
\definecolor{currentstroke}{rgb}{0.000000,0.000000,0.000000}%
\pgfsetstrokecolor{currentstroke}%
\pgfsetdash{}{0pt}%
\pgfpathmoveto{\pgfqpoint{5.345224in}{2.539577in}}%
\pgfpathlineto{\pgfqpoint{5.359330in}{2.540033in}}%
\pgfpathlineto{\pgfqpoint{5.373445in}{2.540557in}}%
\pgfpathlineto{\pgfqpoint{5.387572in}{2.541148in}}%
\pgfpathlineto{\pgfqpoint{5.401708in}{2.541807in}}%
\pgfpathlineto{\pgfqpoint{5.409105in}{2.548311in}}%
\pgfpathlineto{\pgfqpoint{5.416500in}{2.554923in}}%
\pgfpathlineto{\pgfqpoint{5.423891in}{2.561650in}}%
\pgfpathlineto{\pgfqpoint{5.431280in}{2.568497in}}%
\pgfpathlineto{\pgfqpoint{5.417165in}{2.568188in}}%
\pgfpathlineto{\pgfqpoint{5.403061in}{2.567946in}}%
\pgfpathlineto{\pgfqpoint{5.388967in}{2.567772in}}%
\pgfpathlineto{\pgfqpoint{5.374883in}{2.567665in}}%
\pgfpathlineto{\pgfqpoint{5.367472in}{2.560461in}}%
\pgfpathlineto{\pgfqpoint{5.360059in}{2.553383in}}%
\pgfpathlineto{\pgfqpoint{5.352643in}{2.546424in}}%
\pgfpathlineto{\pgfqpoint{5.345224in}{2.539577in}}%
\pgfpathclose%
\pgfusepath{fill}%
\end{pgfscope}%
\begin{pgfscope}%
\pgfpathrectangle{\pgfqpoint{1.150000in}{0.150000in}}{\pgfqpoint{5.700000in}{5.700000in}}%
\pgfusepath{clip}%
\pgfsetbuttcap%
\pgfsetroundjoin%
\definecolor{currentfill}{rgb}{0.281887,0.150881,0.465405}%
\pgfsetfillcolor{currentfill}%
\pgfsetfillopacity{0.700000}%
\pgfsetlinewidth{0.000000pt}%
\definecolor{currentstroke}{rgb}{0.000000,0.000000,0.000000}%
\pgfsetstrokecolor{currentstroke}%
\pgfsetdash{}{0pt}%
\pgfpathmoveto{\pgfqpoint{2.414104in}{2.292761in}}%
\pgfpathlineto{\pgfqpoint{2.427670in}{2.281720in}}%
\pgfpathlineto{\pgfqpoint{2.441235in}{2.270809in}}%
\pgfpathlineto{\pgfqpoint{2.454798in}{2.260027in}}%
\pgfpathlineto{\pgfqpoint{2.468360in}{2.249373in}}%
\pgfpathlineto{\pgfqpoint{2.476942in}{2.255147in}}%
\pgfpathlineto{\pgfqpoint{2.485514in}{2.261036in}}%
\pgfpathlineto{\pgfqpoint{2.494074in}{2.267038in}}%
\pgfpathlineto{\pgfqpoint{2.502625in}{2.273150in}}%
\pgfpathlineto{\pgfqpoint{2.489087in}{2.283635in}}%
\pgfpathlineto{\pgfqpoint{2.475547in}{2.294248in}}%
\pgfpathlineto{\pgfqpoint{2.462006in}{2.304990in}}%
\pgfpathlineto{\pgfqpoint{2.448464in}{2.315861in}}%
\pgfpathlineto{\pgfqpoint{2.439890in}{2.309910in}}%
\pgfpathlineto{\pgfqpoint{2.431306in}{2.304075in}}%
\pgfpathlineto{\pgfqpoint{2.422711in}{2.298358in}}%
\pgfpathlineto{\pgfqpoint{2.414104in}{2.292761in}}%
\pgfpathclose%
\pgfusepath{fill}%
\end{pgfscope}%
\begin{pgfscope}%
\pgfpathrectangle{\pgfqpoint{1.150000in}{0.150000in}}{\pgfqpoint{5.700000in}{5.700000in}}%
\pgfusepath{clip}%
\pgfsetbuttcap%
\pgfsetroundjoin%
\definecolor{currentfill}{rgb}{0.277018,0.050344,0.375715}%
\pgfsetfillcolor{currentfill}%
\pgfsetfillopacity{0.700000}%
\pgfsetlinewidth{0.000000pt}%
\definecolor{currentstroke}{rgb}{0.000000,0.000000,0.000000}%
\pgfsetstrokecolor{currentstroke}%
\pgfsetdash{}{0pt}%
\pgfpathmoveto{\pgfqpoint{3.600418in}{2.076041in}}%
\pgfpathlineto{\pgfqpoint{3.614019in}{2.072753in}}%
\pgfpathlineto{\pgfqpoint{3.627626in}{2.069549in}}%
\pgfpathlineto{\pgfqpoint{3.641239in}{2.066429in}}%
\pgfpathlineto{\pgfqpoint{3.654857in}{2.063392in}}%
\pgfpathlineto{\pgfqpoint{3.662928in}{2.072308in}}%
\pgfpathlineto{\pgfqpoint{3.670993in}{2.081221in}}%
\pgfpathlineto{\pgfqpoint{3.679053in}{2.090133in}}%
\pgfpathlineto{\pgfqpoint{3.687107in}{2.099043in}}%
\pgfpathlineto{\pgfqpoint{3.673499in}{2.102079in}}%
\pgfpathlineto{\pgfqpoint{3.659898in}{2.105198in}}%
\pgfpathlineto{\pgfqpoint{3.646302in}{2.108401in}}%
\pgfpathlineto{\pgfqpoint{3.632712in}{2.111688in}}%
\pgfpathlineto{\pgfqpoint{3.624647in}{2.102771in}}%
\pgfpathlineto{\pgfqpoint{3.616577in}{2.093858in}}%
\pgfpathlineto{\pgfqpoint{3.608500in}{2.084948in}}%
\pgfpathlineto{\pgfqpoint{3.600418in}{2.076041in}}%
\pgfpathclose%
\pgfusepath{fill}%
\end{pgfscope}%
\begin{pgfscope}%
\pgfpathrectangle{\pgfqpoint{1.150000in}{0.150000in}}{\pgfqpoint{5.700000in}{5.700000in}}%
\pgfusepath{clip}%
\pgfsetbuttcap%
\pgfsetroundjoin%
\definecolor{currentfill}{rgb}{0.266580,0.228262,0.514349}%
\pgfsetfillcolor{currentfill}%
\pgfsetfillopacity{0.700000}%
\pgfsetlinewidth{0.000000pt}%
\definecolor{currentstroke}{rgb}{0.000000,0.000000,0.000000}%
\pgfsetstrokecolor{currentstroke}%
\pgfsetdash{}{0pt}%
\pgfpathmoveto{\pgfqpoint{4.944451in}{2.422047in}}%
\pgfpathlineto{\pgfqpoint{4.958432in}{2.422321in}}%
\pgfpathlineto{\pgfqpoint{4.972422in}{2.422664in}}%
\pgfpathlineto{\pgfqpoint{4.986422in}{2.423076in}}%
\pgfpathlineto{\pgfqpoint{5.000432in}{2.423559in}}%
\pgfpathlineto{\pgfqpoint{5.007994in}{2.430413in}}%
\pgfpathlineto{\pgfqpoint{5.015551in}{2.437313in}}%
\pgfpathlineto{\pgfqpoint{5.023104in}{2.444264in}}%
\pgfpathlineto{\pgfqpoint{5.030651in}{2.451270in}}%
\pgfpathlineto{\pgfqpoint{5.016659in}{2.451055in}}%
\pgfpathlineto{\pgfqpoint{5.002676in}{2.450909in}}%
\pgfpathlineto{\pgfqpoint{4.988703in}{2.450833in}}%
\pgfpathlineto{\pgfqpoint{4.974739in}{2.450827in}}%
\pgfpathlineto{\pgfqpoint{4.967174in}{2.443547in}}%
\pgfpathlineto{\pgfqpoint{4.959605in}{2.436327in}}%
\pgfpathlineto{\pgfqpoint{4.952031in}{2.429162in}}%
\pgfpathlineto{\pgfqpoint{4.944451in}{2.422047in}}%
\pgfpathclose%
\pgfusepath{fill}%
\end{pgfscope}%
\begin{pgfscope}%
\pgfpathrectangle{\pgfqpoint{1.150000in}{0.150000in}}{\pgfqpoint{5.700000in}{5.700000in}}%
\pgfusepath{clip}%
\pgfsetbuttcap%
\pgfsetroundjoin%
\definecolor{currentfill}{rgb}{0.274952,0.037752,0.364543}%
\pgfsetfillcolor{currentfill}%
\pgfsetfillopacity{0.700000}%
\pgfsetlinewidth{0.000000pt}%
\definecolor{currentstroke}{rgb}{0.000000,0.000000,0.000000}%
\pgfsetstrokecolor{currentstroke}%
\pgfsetdash{}{0pt}%
\pgfpathmoveto{\pgfqpoint{3.372470in}{2.051774in}}%
\pgfpathlineto{\pgfqpoint{3.386032in}{2.047384in}}%
\pgfpathlineto{\pgfqpoint{3.399600in}{2.043082in}}%
\pgfpathlineto{\pgfqpoint{3.413171in}{2.038869in}}%
\pgfpathlineto{\pgfqpoint{3.426748in}{2.034744in}}%
\pgfpathlineto{\pgfqpoint{3.434902in}{2.043535in}}%
\pgfpathlineto{\pgfqpoint{3.443049in}{2.052339in}}%
\pgfpathlineto{\pgfqpoint{3.451191in}{2.061155in}}%
\pgfpathlineto{\pgfqpoint{3.459326in}{2.069983in}}%
\pgfpathlineto{\pgfqpoint{3.445761in}{2.074066in}}%
\pgfpathlineto{\pgfqpoint{3.432202in}{2.078237in}}%
\pgfpathlineto{\pgfqpoint{3.418647in}{2.082496in}}%
\pgfpathlineto{\pgfqpoint{3.405097in}{2.086844in}}%
\pgfpathlineto{\pgfqpoint{3.396950in}{2.078050in}}%
\pgfpathlineto{\pgfqpoint{3.388796in}{2.069274in}}%
\pgfpathlineto{\pgfqpoint{3.380636in}{2.060515in}}%
\pgfpathlineto{\pgfqpoint{3.372470in}{2.051774in}}%
\pgfpathclose%
\pgfusepath{fill}%
\end{pgfscope}%
\begin{pgfscope}%
\pgfpathrectangle{\pgfqpoint{1.150000in}{0.150000in}}{\pgfqpoint{5.700000in}{5.700000in}}%
\pgfusepath{clip}%
\pgfsetbuttcap%
\pgfsetroundjoin%
\definecolor{currentfill}{rgb}{0.279574,0.170599,0.479997}%
\pgfsetfillcolor{currentfill}%
\pgfsetfillopacity{0.700000}%
\pgfsetlinewidth{0.000000pt}%
\definecolor{currentstroke}{rgb}{0.000000,0.000000,0.000000}%
\pgfsetstrokecolor{currentstroke}%
\pgfsetdash{}{0pt}%
\pgfpathmoveto{\pgfqpoint{4.543601in}{2.303634in}}%
\pgfpathlineto{\pgfqpoint{4.557456in}{2.303345in}}%
\pgfpathlineto{\pgfqpoint{4.571319in}{2.303128in}}%
\pgfpathlineto{\pgfqpoint{4.585191in}{2.302984in}}%
\pgfpathlineto{\pgfqpoint{4.599072in}{2.302913in}}%
\pgfpathlineto{\pgfqpoint{4.606798in}{2.310514in}}%
\pgfpathlineto{\pgfqpoint{4.614518in}{2.318122in}}%
\pgfpathlineto{\pgfqpoint{4.622233in}{2.325737in}}%
\pgfpathlineto{\pgfqpoint{4.629942in}{2.333366in}}%
\pgfpathlineto{\pgfqpoint{4.616075in}{2.333622in}}%
\pgfpathlineto{\pgfqpoint{4.602216in}{2.333951in}}%
\pgfpathlineto{\pgfqpoint{4.588367in}{2.334352in}}%
\pgfpathlineto{\pgfqpoint{4.574526in}{2.334825in}}%
\pgfpathlineto{\pgfqpoint{4.566803in}{2.327005in}}%
\pgfpathlineto{\pgfqpoint{4.559075in}{2.319202in}}%
\pgfpathlineto{\pgfqpoint{4.551341in}{2.311413in}}%
\pgfpathlineto{\pgfqpoint{4.543601in}{2.303634in}}%
\pgfpathclose%
\pgfusepath{fill}%
\end{pgfscope}%
\begin{pgfscope}%
\pgfpathrectangle{\pgfqpoint{1.150000in}{0.150000in}}{\pgfqpoint{5.700000in}{5.700000in}}%
\pgfusepath{clip}%
\pgfsetbuttcap%
\pgfsetroundjoin%
\definecolor{currentfill}{rgb}{0.231674,0.318106,0.544834}%
\pgfsetfillcolor{currentfill}%
\pgfsetfillopacity{0.700000}%
\pgfsetlinewidth{0.000000pt}%
\definecolor{currentstroke}{rgb}{0.000000,0.000000,0.000000}%
\pgfsetstrokecolor{currentstroke}%
\pgfsetdash{}{0pt}%
\pgfpathmoveto{\pgfqpoint{5.659956in}{2.628211in}}%
\pgfpathlineto{\pgfqpoint{5.674160in}{2.628629in}}%
\pgfpathlineto{\pgfqpoint{5.688375in}{2.629114in}}%
\pgfpathlineto{\pgfqpoint{5.702600in}{2.629665in}}%
\pgfpathlineto{\pgfqpoint{5.716836in}{2.630282in}}%
\pgfpathlineto{\pgfqpoint{5.724115in}{2.636898in}}%
\pgfpathlineto{\pgfqpoint{5.731393in}{2.643684in}}%
\pgfpathlineto{\pgfqpoint{5.738671in}{2.650644in}}%
\pgfpathlineto{\pgfqpoint{5.745950in}{2.657788in}}%
\pgfpathlineto{\pgfqpoint{5.731740in}{2.657582in}}%
\pgfpathlineto{\pgfqpoint{5.717540in}{2.657443in}}%
\pgfpathlineto{\pgfqpoint{5.703351in}{2.657369in}}%
\pgfpathlineto{\pgfqpoint{5.689173in}{2.657362in}}%
\pgfpathlineto{\pgfqpoint{5.681869in}{2.649800in}}%
\pgfpathlineto{\pgfqpoint{5.674564in}{2.642426in}}%
\pgfpathlineto{\pgfqpoint{5.667260in}{2.635232in}}%
\pgfpathlineto{\pgfqpoint{5.659956in}{2.628211in}}%
\pgfpathclose%
\pgfusepath{fill}%
\end{pgfscope}%
\begin{pgfscope}%
\pgfpathrectangle{\pgfqpoint{1.150000in}{0.150000in}}{\pgfqpoint{5.700000in}{5.700000in}}%
\pgfusepath{clip}%
\pgfsetbuttcap%
\pgfsetroundjoin%
\definecolor{currentfill}{rgb}{0.280894,0.078907,0.402329}%
\pgfsetfillcolor{currentfill}%
\pgfsetfillopacity{0.700000}%
\pgfsetlinewidth{0.000000pt}%
\definecolor{currentstroke}{rgb}{0.000000,0.000000,0.000000}%
\pgfsetstrokecolor{currentstroke}%
\pgfsetdash{}{0pt}%
\pgfpathmoveto{\pgfqpoint{3.828259in}{2.113207in}}%
\pgfpathlineto{\pgfqpoint{3.841913in}{2.110888in}}%
\pgfpathlineto{\pgfqpoint{3.855573in}{2.108649in}}%
\pgfpathlineto{\pgfqpoint{3.869240in}{2.106490in}}%
\pgfpathlineto{\pgfqpoint{3.882914in}{2.104411in}}%
\pgfpathlineto{\pgfqpoint{3.890906in}{2.113229in}}%
\pgfpathlineto{\pgfqpoint{3.898893in}{2.122036in}}%
\pgfpathlineto{\pgfqpoint{3.906874in}{2.130832in}}%
\pgfpathlineto{\pgfqpoint{3.914849in}{2.139618in}}%
\pgfpathlineto{\pgfqpoint{3.901186in}{2.141737in}}%
\pgfpathlineto{\pgfqpoint{3.887530in}{2.143936in}}%
\pgfpathlineto{\pgfqpoint{3.873881in}{2.146215in}}%
\pgfpathlineto{\pgfqpoint{3.860238in}{2.148574in}}%
\pgfpathlineto{\pgfqpoint{3.852252in}{2.139740in}}%
\pgfpathlineto{\pgfqpoint{3.844260in}{2.130901in}}%
\pgfpathlineto{\pgfqpoint{3.836262in}{2.122058in}}%
\pgfpathlineto{\pgfqpoint{3.828259in}{2.113207in}}%
\pgfpathclose%
\pgfusepath{fill}%
\end{pgfscope}%
\begin{pgfscope}%
\pgfpathrectangle{\pgfqpoint{1.150000in}{0.150000in}}{\pgfqpoint{5.700000in}{5.700000in}}%
\pgfusepath{clip}%
\pgfsetbuttcap%
\pgfsetroundjoin%
\definecolor{currentfill}{rgb}{0.281924,0.089666,0.412415}%
\pgfsetfillcolor{currentfill}%
\pgfsetfillopacity{0.700000}%
\pgfsetlinewidth{0.000000pt}%
\definecolor{currentstroke}{rgb}{0.000000,0.000000,0.000000}%
\pgfsetstrokecolor{currentstroke}%
\pgfsetdash{}{0pt}%
\pgfpathmoveto{\pgfqpoint{2.665038in}{2.156898in}}%
\pgfpathlineto{\pgfqpoint{2.678572in}{2.147979in}}%
\pgfpathlineto{\pgfqpoint{2.692107in}{2.139173in}}%
\pgfpathlineto{\pgfqpoint{2.705642in}{2.130481in}}%
\pgfpathlineto{\pgfqpoint{2.719178in}{2.121901in}}%
\pgfpathlineto{\pgfqpoint{2.727634in}{2.128741in}}%
\pgfpathlineto{\pgfqpoint{2.736080in}{2.135668in}}%
\pgfpathlineto{\pgfqpoint{2.744518in}{2.142678in}}%
\pgfpathlineto{\pgfqpoint{2.752946in}{2.149771in}}%
\pgfpathlineto{\pgfqpoint{2.739430in}{2.158205in}}%
\pgfpathlineto{\pgfqpoint{2.725915in}{2.166751in}}%
\pgfpathlineto{\pgfqpoint{2.712400in}{2.175410in}}%
\pgfpathlineto{\pgfqpoint{2.698886in}{2.184182in}}%
\pgfpathlineto{\pgfqpoint{2.690438in}{2.177229in}}%
\pgfpathlineto{\pgfqpoint{2.681980in}{2.170362in}}%
\pgfpathlineto{\pgfqpoint{2.673514in}{2.163584in}}%
\pgfpathlineto{\pgfqpoint{2.665038in}{2.156898in}}%
\pgfpathclose%
\pgfusepath{fill}%
\end{pgfscope}%
\begin{pgfscope}%
\pgfpathrectangle{\pgfqpoint{1.150000in}{0.150000in}}{\pgfqpoint{5.700000in}{5.700000in}}%
\pgfusepath{clip}%
\pgfsetbuttcap%
\pgfsetroundjoin%
\definecolor{currentfill}{rgb}{0.283197,0.115680,0.436115}%
\pgfsetfillcolor{currentfill}%
\pgfsetfillopacity{0.700000}%
\pgfsetlinewidth{0.000000pt}%
\definecolor{currentstroke}{rgb}{0.000000,0.000000,0.000000}%
\pgfsetstrokecolor{currentstroke}%
\pgfsetdash{}{0pt}%
\pgfpathmoveto{\pgfqpoint{4.142699in}{2.188741in}}%
\pgfpathlineto{\pgfqpoint{4.156435in}{2.187494in}}%
\pgfpathlineto{\pgfqpoint{4.170179in}{2.186323in}}%
\pgfpathlineto{\pgfqpoint{4.183931in}{2.185228in}}%
\pgfpathlineto{\pgfqpoint{4.197691in}{2.184209in}}%
\pgfpathlineto{\pgfqpoint{4.205571in}{2.192604in}}%
\pgfpathlineto{\pgfqpoint{4.213445in}{2.200985in}}%
\pgfpathlineto{\pgfqpoint{4.221314in}{2.209354in}}%
\pgfpathlineto{\pgfqpoint{4.229176in}{2.217715in}}%
\pgfpathlineto{\pgfqpoint{4.215428in}{2.218836in}}%
\pgfpathlineto{\pgfqpoint{4.201688in}{2.220033in}}%
\pgfpathlineto{\pgfqpoint{4.187955in}{2.221306in}}%
\pgfpathlineto{\pgfqpoint{4.174230in}{2.222655in}}%
\pgfpathlineto{\pgfqpoint{4.166356in}{2.214184in}}%
\pgfpathlineto{\pgfqpoint{4.158476in}{2.205710in}}%
\pgfpathlineto{\pgfqpoint{4.150590in}{2.197230in}}%
\pgfpathlineto{\pgfqpoint{4.142699in}{2.188741in}}%
\pgfpathclose%
\pgfusepath{fill}%
\end{pgfscope}%
\begin{pgfscope}%
\pgfpathrectangle{\pgfqpoint{1.150000in}{0.150000in}}{\pgfqpoint{5.700000in}{5.700000in}}%
\pgfusepath{clip}%
\pgfsetbuttcap%
\pgfsetroundjoin%
\definecolor{currentfill}{rgb}{0.252194,0.269783,0.531579}%
\pgfsetfillcolor{currentfill}%
\pgfsetfillopacity{0.700000}%
\pgfsetlinewidth{0.000000pt}%
\definecolor{currentstroke}{rgb}{0.000000,0.000000,0.000000}%
\pgfsetstrokecolor{currentstroke}%
\pgfsetdash{}{0pt}%
\pgfpathmoveto{\pgfqpoint{5.259115in}{2.510739in}}%
\pgfpathlineto{\pgfqpoint{5.273200in}{2.511253in}}%
\pgfpathlineto{\pgfqpoint{5.287295in}{2.511834in}}%
\pgfpathlineto{\pgfqpoint{5.301400in}{2.512484in}}%
\pgfpathlineto{\pgfqpoint{5.315516in}{2.513202in}}%
\pgfpathlineto{\pgfqpoint{5.322949in}{2.519656in}}%
\pgfpathlineto{\pgfqpoint{5.330377in}{2.526199in}}%
\pgfpathlineto{\pgfqpoint{5.337803in}{2.532838in}}%
\pgfpathlineto{\pgfqpoint{5.345224in}{2.539577in}}%
\pgfpathlineto{\pgfqpoint{5.331129in}{2.539189in}}%
\pgfpathlineto{\pgfqpoint{5.317045in}{2.538869in}}%
\pgfpathlineto{\pgfqpoint{5.302970in}{2.538617in}}%
\pgfpathlineto{\pgfqpoint{5.288905in}{2.538433in}}%
\pgfpathlineto{\pgfqpoint{5.281463in}{2.531356in}}%
\pgfpathlineto{\pgfqpoint{5.274017in}{2.524386in}}%
\pgfpathlineto{\pgfqpoint{5.266567in}{2.517516in}}%
\pgfpathlineto{\pgfqpoint{5.259115in}{2.510739in}}%
\pgfpathclose%
\pgfusepath{fill}%
\end{pgfscope}%
\begin{pgfscope}%
\pgfpathrectangle{\pgfqpoint{1.150000in}{0.150000in}}{\pgfqpoint{5.700000in}{5.700000in}}%
\pgfusepath{clip}%
\pgfsetbuttcap%
\pgfsetroundjoin%
\definecolor{currentfill}{rgb}{0.270595,0.214069,0.507052}%
\pgfsetfillcolor{currentfill}%
\pgfsetfillopacity{0.700000}%
\pgfsetlinewidth{0.000000pt}%
\definecolor{currentstroke}{rgb}{0.000000,0.000000,0.000000}%
\pgfsetstrokecolor{currentstroke}%
\pgfsetdash{}{0pt}%
\pgfpathmoveto{\pgfqpoint{4.858193in}{2.392625in}}%
\pgfpathlineto{\pgfqpoint{4.872151in}{2.392864in}}%
\pgfpathlineto{\pgfqpoint{4.886119in}{2.393174in}}%
\pgfpathlineto{\pgfqpoint{4.900097in}{2.393554in}}%
\pgfpathlineto{\pgfqpoint{4.914084in}{2.394004in}}%
\pgfpathlineto{\pgfqpoint{4.921684in}{2.400962in}}%
\pgfpathlineto{\pgfqpoint{4.929278in}{2.407952in}}%
\pgfpathlineto{\pgfqpoint{4.936867in}{2.414979in}}%
\pgfpathlineto{\pgfqpoint{4.944451in}{2.422047in}}%
\pgfpathlineto{\pgfqpoint{4.930481in}{2.421844in}}%
\pgfpathlineto{\pgfqpoint{4.916519in}{2.421711in}}%
\pgfpathlineto{\pgfqpoint{4.902568in}{2.421649in}}%
\pgfpathlineto{\pgfqpoint{4.888626in}{2.421656in}}%
\pgfpathlineto{\pgfqpoint{4.881025in}{2.414333in}}%
\pgfpathlineto{\pgfqpoint{4.873420in}{2.407057in}}%
\pgfpathlineto{\pgfqpoint{4.865809in}{2.399823in}}%
\pgfpathlineto{\pgfqpoint{4.858193in}{2.392625in}}%
\pgfpathclose%
\pgfusepath{fill}%
\end{pgfscope}%
\begin{pgfscope}%
\pgfpathrectangle{\pgfqpoint{1.150000in}{0.150000in}}{\pgfqpoint{5.700000in}{5.700000in}}%
\pgfusepath{clip}%
\pgfsetbuttcap%
\pgfsetroundjoin%
\definecolor{currentfill}{rgb}{0.212395,0.359683,0.551710}%
\pgfsetfillcolor{currentfill}%
\pgfsetfillopacity{0.700000}%
\pgfsetlinewidth{0.000000pt}%
\definecolor{currentstroke}{rgb}{0.000000,0.000000,0.000000}%
\pgfsetstrokecolor{currentstroke}%
\pgfsetdash{}{0pt}%
\pgfpathmoveto{\pgfqpoint{5.974933in}{2.719443in}}%
\pgfpathlineto{\pgfqpoint{5.989228in}{2.719619in}}%
\pgfpathlineto{\pgfqpoint{6.003534in}{2.719860in}}%
\pgfpathlineto{\pgfqpoint{6.017851in}{2.720166in}}%
\pgfpathlineto{\pgfqpoint{6.025041in}{2.727562in}}%
\pgfpathlineto{\pgfqpoint{6.032234in}{2.735199in}}%
\pgfpathlineto{\pgfqpoint{6.039432in}{2.743088in}}%
\pgfpathlineto{\pgfqpoint{6.046634in}{2.751237in}}%
\pgfpathlineto{\pgfqpoint{6.032347in}{2.751402in}}%
\pgfpathlineto{\pgfqpoint{6.018071in}{2.751632in}}%
\pgfpathlineto{\pgfqpoint{6.003805in}{2.751927in}}%
\pgfpathlineto{\pgfqpoint{5.996581in}{2.743420in}}%
\pgfpathlineto{\pgfqpoint{5.989361in}{2.735176in}}%
\pgfpathlineto{\pgfqpoint{5.982145in}{2.727186in}}%
\pgfpathlineto{\pgfqpoint{5.974933in}{2.719443in}}%
\pgfpathclose%
\pgfusepath{fill}%
\end{pgfscope}%
\begin{pgfscope}%
\pgfpathrectangle{\pgfqpoint{1.150000in}{0.150000in}}{\pgfqpoint{5.700000in}{5.700000in}}%
\pgfusepath{clip}%
\pgfsetbuttcap%
\pgfsetroundjoin%
\definecolor{currentfill}{rgb}{0.282884,0.135920,0.453427}%
\pgfsetfillcolor{currentfill}%
\pgfsetfillopacity{0.700000}%
\pgfsetlinewidth{0.000000pt}%
\definecolor{currentstroke}{rgb}{0.000000,0.000000,0.000000}%
\pgfsetstrokecolor{currentstroke}%
\pgfsetdash{}{0pt}%
\pgfpathmoveto{\pgfqpoint{2.468360in}{2.249373in}}%
\pgfpathlineto{\pgfqpoint{2.481920in}{2.238845in}}%
\pgfpathlineto{\pgfqpoint{2.495480in}{2.228443in}}%
\pgfpathlineto{\pgfqpoint{2.509038in}{2.218165in}}%
\pgfpathlineto{\pgfqpoint{2.522595in}{2.208011in}}%
\pgfpathlineto{\pgfqpoint{2.531154in}{2.213961in}}%
\pgfpathlineto{\pgfqpoint{2.539703in}{2.220021in}}%
\pgfpathlineto{\pgfqpoint{2.548241in}{2.226189in}}%
\pgfpathlineto{\pgfqpoint{2.556769in}{2.232463in}}%
\pgfpathlineto{\pgfqpoint{2.543234in}{2.242449in}}%
\pgfpathlineto{\pgfqpoint{2.529699in}{2.252558in}}%
\pgfpathlineto{\pgfqpoint{2.516162in}{2.262791in}}%
\pgfpathlineto{\pgfqpoint{2.502625in}{2.273150in}}%
\pgfpathlineto{\pgfqpoint{2.494074in}{2.267038in}}%
\pgfpathlineto{\pgfqpoint{2.485514in}{2.261036in}}%
\pgfpathlineto{\pgfqpoint{2.476942in}{2.255147in}}%
\pgfpathlineto{\pgfqpoint{2.468360in}{2.249373in}}%
\pgfpathclose%
\pgfusepath{fill}%
\end{pgfscope}%
\begin{pgfscope}%
\pgfpathrectangle{\pgfqpoint{1.150000in}{0.150000in}}{\pgfqpoint{5.700000in}{5.700000in}}%
\pgfusepath{clip}%
\pgfsetbuttcap%
\pgfsetroundjoin%
\definecolor{currentfill}{rgb}{0.276022,0.044167,0.370164}%
\pgfsetfillcolor{currentfill}%
\pgfsetfillopacity{0.700000}%
\pgfsetlinewidth{0.000000pt}%
\definecolor{currentstroke}{rgb}{0.000000,0.000000,0.000000}%
\pgfsetstrokecolor{currentstroke}%
\pgfsetdash{}{0pt}%
\pgfpathmoveto{\pgfqpoint{3.002756in}{2.060606in}}%
\pgfpathlineto{\pgfqpoint{3.016286in}{2.054088in}}%
\pgfpathlineto{\pgfqpoint{3.029819in}{2.047669in}}%
\pgfpathlineto{\pgfqpoint{3.043354in}{2.041348in}}%
\pgfpathlineto{\pgfqpoint{3.056893in}{2.035126in}}%
\pgfpathlineto{\pgfqpoint{3.065195in}{2.043164in}}%
\pgfpathlineto{\pgfqpoint{3.073490in}{2.051249in}}%
\pgfpathlineto{\pgfqpoint{3.081777in}{2.059380in}}%
\pgfpathlineto{\pgfqpoint{3.090058in}{2.067556in}}%
\pgfpathlineto{\pgfqpoint{3.076535in}{2.073675in}}%
\pgfpathlineto{\pgfqpoint{3.063015in}{2.079891in}}%
\pgfpathlineto{\pgfqpoint{3.049498in}{2.086207in}}%
\pgfpathlineto{\pgfqpoint{3.035983in}{2.092621in}}%
\pgfpathlineto{\pgfqpoint{3.027687in}{2.084541in}}%
\pgfpathlineto{\pgfqpoint{3.019384in}{2.076511in}}%
\pgfpathlineto{\pgfqpoint{3.011074in}{2.068532in}}%
\pgfpathlineto{\pgfqpoint{3.002756in}{2.060606in}}%
\pgfpathclose%
\pgfusepath{fill}%
\end{pgfscope}%
\begin{pgfscope}%
\pgfpathrectangle{\pgfqpoint{1.150000in}{0.150000in}}{\pgfqpoint{5.700000in}{5.700000in}}%
\pgfusepath{clip}%
\pgfsetbuttcap%
\pgfsetroundjoin%
\definecolor{currentfill}{rgb}{0.280868,0.160771,0.472899}%
\pgfsetfillcolor{currentfill}%
\pgfsetfillopacity{0.700000}%
\pgfsetlinewidth{0.000000pt}%
\definecolor{currentstroke}{rgb}{0.000000,0.000000,0.000000}%
\pgfsetstrokecolor{currentstroke}%
\pgfsetdash{}{0pt}%
\pgfpathmoveto{\pgfqpoint{4.457205in}{2.273788in}}%
\pgfpathlineto{\pgfqpoint{4.471037in}{2.273370in}}%
\pgfpathlineto{\pgfqpoint{4.484878in}{2.273026in}}%
\pgfpathlineto{\pgfqpoint{4.498728in}{2.272755in}}%
\pgfpathlineto{\pgfqpoint{4.512586in}{2.272557in}}%
\pgfpathlineto{\pgfqpoint{4.520348in}{2.280327in}}%
\pgfpathlineto{\pgfqpoint{4.528105in}{2.288095in}}%
\pgfpathlineto{\pgfqpoint{4.535856in}{2.295863in}}%
\pgfpathlineto{\pgfqpoint{4.543601in}{2.303634in}}%
\pgfpathlineto{\pgfqpoint{4.529756in}{2.303996in}}%
\pgfpathlineto{\pgfqpoint{4.515919in}{2.304431in}}%
\pgfpathlineto{\pgfqpoint{4.502091in}{2.304940in}}%
\pgfpathlineto{\pgfqpoint{4.488272in}{2.305521in}}%
\pgfpathlineto{\pgfqpoint{4.480514in}{2.297578in}}%
\pgfpathlineto{\pgfqpoint{4.472750in}{2.289643in}}%
\pgfpathlineto{\pgfqpoint{4.464980in}{2.281714in}}%
\pgfpathlineto{\pgfqpoint{4.457205in}{2.273788in}}%
\pgfpathclose%
\pgfusepath{fill}%
\end{pgfscope}%
\begin{pgfscope}%
\pgfpathrectangle{\pgfqpoint{1.150000in}{0.150000in}}{\pgfqpoint{5.700000in}{5.700000in}}%
\pgfusepath{clip}%
\pgfsetbuttcap%
\pgfsetroundjoin%
\definecolor{currentfill}{rgb}{0.273809,0.031497,0.358853}%
\pgfsetfillcolor{currentfill}%
\pgfsetfillopacity{0.700000}%
\pgfsetlinewidth{0.000000pt}%
\definecolor{currentstroke}{rgb}{0.000000,0.000000,0.000000}%
\pgfsetstrokecolor{currentstroke}%
\pgfsetdash{}{0pt}%
\pgfpathmoveto{\pgfqpoint{3.144183in}{2.044049in}}%
\pgfpathlineto{\pgfqpoint{3.157723in}{2.038412in}}%
\pgfpathlineto{\pgfqpoint{3.171266in}{2.032869in}}%
\pgfpathlineto{\pgfqpoint{3.184813in}{2.027420in}}%
\pgfpathlineto{\pgfqpoint{3.198364in}{2.022065in}}%
\pgfpathlineto{\pgfqpoint{3.206608in}{2.030466in}}%
\pgfpathlineto{\pgfqpoint{3.214846in}{2.038900in}}%
\pgfpathlineto{\pgfqpoint{3.223077in}{2.047366in}}%
\pgfpathlineto{\pgfqpoint{3.231301in}{2.055862in}}%
\pgfpathlineto{\pgfqpoint{3.217764in}{2.061135in}}%
\pgfpathlineto{\pgfqpoint{3.204231in}{2.066500in}}%
\pgfpathlineto{\pgfqpoint{3.190702in}{2.071960in}}%
\pgfpathlineto{\pgfqpoint{3.177177in}{2.077515in}}%
\pgfpathlineto{\pgfqpoint{3.168938in}{2.069093in}}%
\pgfpathlineto{\pgfqpoint{3.160693in}{2.060708in}}%
\pgfpathlineto{\pgfqpoint{3.152442in}{2.052360in}}%
\pgfpathlineto{\pgfqpoint{3.144183in}{2.044049in}}%
\pgfpathclose%
\pgfusepath{fill}%
\end{pgfscope}%
\begin{pgfscope}%
\pgfpathrectangle{\pgfqpoint{1.150000in}{0.150000in}}{\pgfqpoint{5.700000in}{5.700000in}}%
\pgfusepath{clip}%
\pgfsetbuttcap%
\pgfsetroundjoin%
\definecolor{currentfill}{rgb}{0.276022,0.044167,0.370164}%
\pgfsetfillcolor{currentfill}%
\pgfsetfillopacity{0.700000}%
\pgfsetlinewidth{0.000000pt}%
\definecolor{currentstroke}{rgb}{0.000000,0.000000,0.000000}%
\pgfsetstrokecolor{currentstroke}%
\pgfsetdash{}{0pt}%
\pgfpathmoveto{\pgfqpoint{3.513635in}{2.054521in}}%
\pgfpathlineto{\pgfqpoint{3.527226in}{2.050872in}}%
\pgfpathlineto{\pgfqpoint{3.540822in}{2.047308in}}%
\pgfpathlineto{\pgfqpoint{3.554423in}{2.043829in}}%
\pgfpathlineto{\pgfqpoint{3.568030in}{2.040436in}}%
\pgfpathlineto{\pgfqpoint{3.576136in}{2.049334in}}%
\pgfpathlineto{\pgfqpoint{3.584236in}{2.058234in}}%
\pgfpathlineto{\pgfqpoint{3.592330in}{2.067136in}}%
\pgfpathlineto{\pgfqpoint{3.600418in}{2.076041in}}%
\pgfpathlineto{\pgfqpoint{3.586822in}{2.079413in}}%
\pgfpathlineto{\pgfqpoint{3.573232in}{2.082871in}}%
\pgfpathlineto{\pgfqpoint{3.559648in}{2.086413in}}%
\pgfpathlineto{\pgfqpoint{3.546069in}{2.090041in}}%
\pgfpathlineto{\pgfqpoint{3.537970in}{2.081151in}}%
\pgfpathlineto{\pgfqpoint{3.529864in}{2.072267in}}%
\pgfpathlineto{\pgfqpoint{3.521753in}{2.063391in}}%
\pgfpathlineto{\pgfqpoint{3.513635in}{2.054521in}}%
\pgfpathclose%
\pgfusepath{fill}%
\end{pgfscope}%
\begin{pgfscope}%
\pgfpathrectangle{\pgfqpoint{1.150000in}{0.150000in}}{\pgfqpoint{5.700000in}{5.700000in}}%
\pgfusepath{clip}%
\pgfsetbuttcap%
\pgfsetroundjoin%
\definecolor{currentfill}{rgb}{0.277941,0.056324,0.381191}%
\pgfsetfillcolor{currentfill}%
\pgfsetfillopacity{0.700000}%
\pgfsetlinewidth{0.000000pt}%
\definecolor{currentstroke}{rgb}{0.000000,0.000000,0.000000}%
\pgfsetstrokecolor{currentstroke}%
\pgfsetdash{}{0pt}%
\pgfpathmoveto{\pgfqpoint{2.861120in}{2.086242in}}%
\pgfpathlineto{\pgfqpoint{2.874649in}{2.078782in}}%
\pgfpathlineto{\pgfqpoint{2.888180in}{2.071426in}}%
\pgfpathlineto{\pgfqpoint{2.901712in}{2.064175in}}%
\pgfpathlineto{\pgfqpoint{2.915247in}{2.057027in}}%
\pgfpathlineto{\pgfqpoint{2.923612in}{2.064597in}}%
\pgfpathlineto{\pgfqpoint{2.931970in}{2.072232in}}%
\pgfpathlineto{\pgfqpoint{2.940320in}{2.079927in}}%
\pgfpathlineto{\pgfqpoint{2.948662in}{2.087683in}}%
\pgfpathlineto{\pgfqpoint{2.935144in}{2.094707in}}%
\pgfpathlineto{\pgfqpoint{2.921629in}{2.101834in}}%
\pgfpathlineto{\pgfqpoint{2.908115in}{2.109065in}}%
\pgfpathlineto{\pgfqpoint{2.894604in}{2.116400in}}%
\pgfpathlineto{\pgfqpoint{2.886245in}{2.108761in}}%
\pgfpathlineto{\pgfqpoint{2.877878in}{2.101187in}}%
\pgfpathlineto{\pgfqpoint{2.869503in}{2.093680in}}%
\pgfpathlineto{\pgfqpoint{2.861120in}{2.086242in}}%
\pgfpathclose%
\pgfusepath{fill}%
\end{pgfscope}%
\begin{pgfscope}%
\pgfpathrectangle{\pgfqpoint{1.150000in}{0.150000in}}{\pgfqpoint{5.700000in}{5.700000in}}%
\pgfusepath{clip}%
\pgfsetbuttcap%
\pgfsetroundjoin%
\definecolor{currentfill}{rgb}{0.235526,0.309527,0.542944}%
\pgfsetfillcolor{currentfill}%
\pgfsetfillopacity{0.700000}%
\pgfsetlinewidth{0.000000pt}%
\definecolor{currentstroke}{rgb}{0.000000,0.000000,0.000000}%
\pgfsetstrokecolor{currentstroke}%
\pgfsetdash{}{0pt}%
\pgfpathmoveto{\pgfqpoint{5.573923in}{2.599150in}}%
\pgfpathlineto{\pgfqpoint{5.588109in}{2.599692in}}%
\pgfpathlineto{\pgfqpoint{5.602305in}{2.600302in}}%
\pgfpathlineto{\pgfqpoint{5.616513in}{2.600978in}}%
\pgfpathlineto{\pgfqpoint{5.630731in}{2.601721in}}%
\pgfpathlineto{\pgfqpoint{5.638039in}{2.608119in}}%
\pgfpathlineto{\pgfqpoint{5.645346in}{2.614662in}}%
\pgfpathlineto{\pgfqpoint{5.652651in}{2.621357in}}%
\pgfpathlineto{\pgfqpoint{5.659956in}{2.628211in}}%
\pgfpathlineto{\pgfqpoint{5.645763in}{2.627860in}}%
\pgfpathlineto{\pgfqpoint{5.631580in}{2.627575in}}%
\pgfpathlineto{\pgfqpoint{5.617408in}{2.627356in}}%
\pgfpathlineto{\pgfqpoint{5.603247in}{2.627204in}}%
\pgfpathlineto{\pgfqpoint{5.595917in}{2.619952in}}%
\pgfpathlineto{\pgfqpoint{5.588587in}{2.612863in}}%
\pgfpathlineto{\pgfqpoint{5.581256in}{2.605931in}}%
\pgfpathlineto{\pgfqpoint{5.573923in}{2.599150in}}%
\pgfpathclose%
\pgfusepath{fill}%
\end{pgfscope}%
\begin{pgfscope}%
\pgfpathrectangle{\pgfqpoint{1.150000in}{0.150000in}}{\pgfqpoint{5.700000in}{5.700000in}}%
\pgfusepath{clip}%
\pgfsetbuttcap%
\pgfsetroundjoin%
\definecolor{currentfill}{rgb}{0.282910,0.105393,0.426902}%
\pgfsetfillcolor{currentfill}%
\pgfsetfillopacity{0.700000}%
\pgfsetlinewidth{0.000000pt}%
\definecolor{currentstroke}{rgb}{0.000000,0.000000,0.000000}%
\pgfsetstrokecolor{currentstroke}%
\pgfsetdash{}{0pt}%
\pgfpathmoveto{\pgfqpoint{4.056165in}{2.160100in}}%
\pgfpathlineto{\pgfqpoint{4.069881in}{2.158626in}}%
\pgfpathlineto{\pgfqpoint{4.083606in}{2.157230in}}%
\pgfpathlineto{\pgfqpoint{4.097338in}{2.155910in}}%
\pgfpathlineto{\pgfqpoint{4.111077in}{2.154668in}}%
\pgfpathlineto{\pgfqpoint{4.118991in}{2.163208in}}%
\pgfpathlineto{\pgfqpoint{4.126900in}{2.171732in}}%
\pgfpathlineto{\pgfqpoint{4.134802in}{2.180243in}}%
\pgfpathlineto{\pgfqpoint{4.142699in}{2.188741in}}%
\pgfpathlineto{\pgfqpoint{4.128971in}{2.190065in}}%
\pgfpathlineto{\pgfqpoint{4.115250in}{2.191466in}}%
\pgfpathlineto{\pgfqpoint{4.101537in}{2.192944in}}%
\pgfpathlineto{\pgfqpoint{4.087831in}{2.194499in}}%
\pgfpathlineto{\pgfqpoint{4.079923in}{2.185911in}}%
\pgfpathlineto{\pgfqpoint{4.072009in}{2.177317in}}%
\pgfpathlineto{\pgfqpoint{4.064090in}{2.168713in}}%
\pgfpathlineto{\pgfqpoint{4.056165in}{2.160100in}}%
\pgfpathclose%
\pgfusepath{fill}%
\end{pgfscope}%
\begin{pgfscope}%
\pgfpathrectangle{\pgfqpoint{1.150000in}{0.150000in}}{\pgfqpoint{5.700000in}{5.700000in}}%
\pgfusepath{clip}%
\pgfsetbuttcap%
\pgfsetroundjoin%
\definecolor{currentfill}{rgb}{0.279566,0.067836,0.391917}%
\pgfsetfillcolor{currentfill}%
\pgfsetfillopacity{0.700000}%
\pgfsetlinewidth{0.000000pt}%
\definecolor{currentstroke}{rgb}{0.000000,0.000000,0.000000}%
\pgfsetstrokecolor{currentstroke}%
\pgfsetdash{}{0pt}%
\pgfpathmoveto{\pgfqpoint{3.741598in}{2.087727in}}%
\pgfpathlineto{\pgfqpoint{3.755236in}{2.085104in}}%
\pgfpathlineto{\pgfqpoint{3.768881in}{2.082562in}}%
\pgfpathlineto{\pgfqpoint{3.782532in}{2.080101in}}%
\pgfpathlineto{\pgfqpoint{3.796190in}{2.077721in}}%
\pgfpathlineto{\pgfqpoint{3.804216in}{2.086607in}}%
\pgfpathlineto{\pgfqpoint{3.812236in}{2.095482in}}%
\pgfpathlineto{\pgfqpoint{3.820250in}{2.104349in}}%
\pgfpathlineto{\pgfqpoint{3.828259in}{2.113207in}}%
\pgfpathlineto{\pgfqpoint{3.814612in}{2.115607in}}%
\pgfpathlineto{\pgfqpoint{3.800972in}{2.118087in}}%
\pgfpathlineto{\pgfqpoint{3.787338in}{2.120649in}}%
\pgfpathlineto{\pgfqpoint{3.773711in}{2.123292in}}%
\pgfpathlineto{\pgfqpoint{3.765691in}{2.114407in}}%
\pgfpathlineto{\pgfqpoint{3.757666in}{2.105518in}}%
\pgfpathlineto{\pgfqpoint{3.749634in}{2.096625in}}%
\pgfpathlineto{\pgfqpoint{3.741598in}{2.087727in}}%
\pgfpathclose%
\pgfusepath{fill}%
\end{pgfscope}%
\begin{pgfscope}%
\pgfpathrectangle{\pgfqpoint{1.150000in}{0.150000in}}{\pgfqpoint{5.700000in}{5.700000in}}%
\pgfusepath{clip}%
\pgfsetbuttcap%
\pgfsetroundjoin%
\definecolor{currentfill}{rgb}{0.255645,0.260703,0.528312}%
\pgfsetfillcolor{currentfill}%
\pgfsetfillopacity{0.700000}%
\pgfsetlinewidth{0.000000pt}%
\definecolor{currentstroke}{rgb}{0.000000,0.000000,0.000000}%
\pgfsetstrokecolor{currentstroke}%
\pgfsetdash{}{0pt}%
\pgfpathmoveto{\pgfqpoint{5.172947in}{2.481856in}}%
\pgfpathlineto{\pgfqpoint{5.187011in}{2.482405in}}%
\pgfpathlineto{\pgfqpoint{5.201085in}{2.483022in}}%
\pgfpathlineto{\pgfqpoint{5.215169in}{2.483708in}}%
\pgfpathlineto{\pgfqpoint{5.229264in}{2.484462in}}%
\pgfpathlineto{\pgfqpoint{5.236733in}{2.490918in}}%
\pgfpathlineto{\pgfqpoint{5.244197in}{2.497446in}}%
\pgfpathlineto{\pgfqpoint{5.251658in}{2.504052in}}%
\pgfpathlineto{\pgfqpoint{5.259115in}{2.510739in}}%
\pgfpathlineto{\pgfqpoint{5.245040in}{2.510295in}}%
\pgfpathlineto{\pgfqpoint{5.230975in}{2.509918in}}%
\pgfpathlineto{\pgfqpoint{5.216921in}{2.509610in}}%
\pgfpathlineto{\pgfqpoint{5.202876in}{2.509370in}}%
\pgfpathlineto{\pgfqpoint{5.195400in}{2.502365in}}%
\pgfpathlineto{\pgfqpoint{5.187920in}{2.495449in}}%
\pgfpathlineto{\pgfqpoint{5.180435in}{2.488614in}}%
\pgfpathlineto{\pgfqpoint{5.172947in}{2.481856in}}%
\pgfpathclose%
\pgfusepath{fill}%
\end{pgfscope}%
\begin{pgfscope}%
\pgfpathrectangle{\pgfqpoint{1.150000in}{0.150000in}}{\pgfqpoint{5.700000in}{5.700000in}}%
\pgfusepath{clip}%
\pgfsetbuttcap%
\pgfsetroundjoin%
\definecolor{currentfill}{rgb}{0.273809,0.031497,0.358853}%
\pgfsetfillcolor{currentfill}%
\pgfsetfillopacity{0.700000}%
\pgfsetlinewidth{0.000000pt}%
\definecolor{currentstroke}{rgb}{0.000000,0.000000,0.000000}%
\pgfsetstrokecolor{currentstroke}%
\pgfsetdash{}{0pt}%
\pgfpathmoveto{\pgfqpoint{3.285488in}{2.035698in}}%
\pgfpathlineto{\pgfqpoint{3.299046in}{2.030886in}}%
\pgfpathlineto{\pgfqpoint{3.312607in}{2.026165in}}%
\pgfpathlineto{\pgfqpoint{3.326174in}{2.021533in}}%
\pgfpathlineto{\pgfqpoint{3.339744in}{2.016992in}}%
\pgfpathlineto{\pgfqpoint{3.347935in}{2.025659in}}%
\pgfpathlineto{\pgfqpoint{3.356120in}{2.034345in}}%
\pgfpathlineto{\pgfqpoint{3.364298in}{2.043050in}}%
\pgfpathlineto{\pgfqpoint{3.372470in}{2.051774in}}%
\pgfpathlineto{\pgfqpoint{3.358912in}{2.056253in}}%
\pgfpathlineto{\pgfqpoint{3.345359in}{2.060822in}}%
\pgfpathlineto{\pgfqpoint{3.331811in}{2.065481in}}%
\pgfpathlineto{\pgfqpoint{3.318267in}{2.070231in}}%
\pgfpathlineto{\pgfqpoint{3.310082in}{2.061562in}}%
\pgfpathlineto{\pgfqpoint{3.301890in}{2.052917in}}%
\pgfpathlineto{\pgfqpoint{3.293692in}{2.044295in}}%
\pgfpathlineto{\pgfqpoint{3.285488in}{2.035698in}}%
\pgfpathclose%
\pgfusepath{fill}%
\end{pgfscope}%
\begin{pgfscope}%
\pgfpathrectangle{\pgfqpoint{1.150000in}{0.150000in}}{\pgfqpoint{5.700000in}{5.700000in}}%
\pgfusepath{clip}%
\pgfsetbuttcap%
\pgfsetroundjoin%
\definecolor{currentfill}{rgb}{0.273006,0.204520,0.501721}%
\pgfsetfillcolor{currentfill}%
\pgfsetfillopacity{0.700000}%
\pgfsetlinewidth{0.000000pt}%
\definecolor{currentstroke}{rgb}{0.000000,0.000000,0.000000}%
\pgfsetstrokecolor{currentstroke}%
\pgfsetdash{}{0pt}%
\pgfpathmoveto{\pgfqpoint{4.771876in}{2.362967in}}%
\pgfpathlineto{\pgfqpoint{4.785812in}{2.363150in}}%
\pgfpathlineto{\pgfqpoint{4.799757in}{2.363403in}}%
\pgfpathlineto{\pgfqpoint{4.813712in}{2.363728in}}%
\pgfpathlineto{\pgfqpoint{4.827676in}{2.364123in}}%
\pgfpathlineto{\pgfqpoint{4.835314in}{2.371214in}}%
\pgfpathlineto{\pgfqpoint{4.842946in}{2.378325in}}%
\pgfpathlineto{\pgfqpoint{4.850572in}{2.385461in}}%
\pgfpathlineto{\pgfqpoint{4.858193in}{2.392625in}}%
\pgfpathlineto{\pgfqpoint{4.844244in}{2.392457in}}%
\pgfpathlineto{\pgfqpoint{4.830305in}{2.392359in}}%
\pgfpathlineto{\pgfqpoint{4.816375in}{2.392331in}}%
\pgfpathlineto{\pgfqpoint{4.802455in}{2.392375in}}%
\pgfpathlineto{\pgfqpoint{4.794818in}{2.384977in}}%
\pgfpathlineto{\pgfqpoint{4.787176in}{2.377612in}}%
\pgfpathlineto{\pgfqpoint{4.779529in}{2.370277in}}%
\pgfpathlineto{\pgfqpoint{4.771876in}{2.362967in}}%
\pgfpathclose%
\pgfusepath{fill}%
\end{pgfscope}%
\begin{pgfscope}%
\pgfpathrectangle{\pgfqpoint{1.150000in}{0.150000in}}{\pgfqpoint{5.700000in}{5.700000in}}%
\pgfusepath{clip}%
\pgfsetbuttcap%
\pgfsetroundjoin%
\definecolor{currentfill}{rgb}{0.281887,0.150881,0.465405}%
\pgfsetfillcolor{currentfill}%
\pgfsetfillopacity{0.700000}%
\pgfsetlinewidth{0.000000pt}%
\definecolor{currentstroke}{rgb}{0.000000,0.000000,0.000000}%
\pgfsetstrokecolor{currentstroke}%
\pgfsetdash{}{0pt}%
\pgfpathmoveto{\pgfqpoint{4.370754in}{2.243879in}}%
\pgfpathlineto{\pgfqpoint{4.384565in}{2.243310in}}%
\pgfpathlineto{\pgfqpoint{4.398383in}{2.242815in}}%
\pgfpathlineto{\pgfqpoint{4.412211in}{2.242394in}}%
\pgfpathlineto{\pgfqpoint{4.426046in}{2.242047in}}%
\pgfpathlineto{\pgfqpoint{4.433845in}{2.249993in}}%
\pgfpathlineto{\pgfqpoint{4.441637in}{2.257930in}}%
\pgfpathlineto{\pgfqpoint{4.449424in}{2.265860in}}%
\pgfpathlineto{\pgfqpoint{4.457205in}{2.273788in}}%
\pgfpathlineto{\pgfqpoint{4.443382in}{2.274278in}}%
\pgfpathlineto{\pgfqpoint{4.429567in}{2.274843in}}%
\pgfpathlineto{\pgfqpoint{4.415760in}{2.275481in}}%
\pgfpathlineto{\pgfqpoint{4.401962in}{2.276194in}}%
\pgfpathlineto{\pgfqpoint{4.394169in}{2.268116in}}%
\pgfpathlineto{\pgfqpoint{4.386370in}{2.260039in}}%
\pgfpathlineto{\pgfqpoint{4.378565in}{2.251961in}}%
\pgfpathlineto{\pgfqpoint{4.370754in}{2.243879in}}%
\pgfpathclose%
\pgfusepath{fill}%
\end{pgfscope}%
\begin{pgfscope}%
\pgfpathrectangle{\pgfqpoint{1.150000in}{0.150000in}}{\pgfqpoint{5.700000in}{5.700000in}}%
\pgfusepath{clip}%
\pgfsetbuttcap%
\pgfsetroundjoin%
\definecolor{currentfill}{rgb}{0.280894,0.078907,0.402329}%
\pgfsetfillcolor{currentfill}%
\pgfsetfillopacity{0.700000}%
\pgfsetlinewidth{0.000000pt}%
\definecolor{currentstroke}{rgb}{0.000000,0.000000,0.000000}%
\pgfsetstrokecolor{currentstroke}%
\pgfsetdash{}{0pt}%
\pgfpathmoveto{\pgfqpoint{2.719178in}{2.121901in}}%
\pgfpathlineto{\pgfqpoint{2.732715in}{2.113432in}}%
\pgfpathlineto{\pgfqpoint{2.746253in}{2.105074in}}%
\pgfpathlineto{\pgfqpoint{2.759791in}{2.096826in}}%
\pgfpathlineto{\pgfqpoint{2.773331in}{2.088687in}}%
\pgfpathlineto{\pgfqpoint{2.781767in}{2.095681in}}%
\pgfpathlineto{\pgfqpoint{2.790194in}{2.102756in}}%
\pgfpathlineto{\pgfqpoint{2.798612in}{2.109910in}}%
\pgfpathlineto{\pgfqpoint{2.807022in}{2.117141in}}%
\pgfpathlineto{\pgfqpoint{2.793501in}{2.125134in}}%
\pgfpathlineto{\pgfqpoint{2.779982in}{2.133236in}}%
\pgfpathlineto{\pgfqpoint{2.766464in}{2.141448in}}%
\pgfpathlineto{\pgfqpoint{2.752946in}{2.149771in}}%
\pgfpathlineto{\pgfqpoint{2.744518in}{2.142678in}}%
\pgfpathlineto{\pgfqpoint{2.736080in}{2.135668in}}%
\pgfpathlineto{\pgfqpoint{2.727634in}{2.128741in}}%
\pgfpathlineto{\pgfqpoint{2.719178in}{2.121901in}}%
\pgfpathclose%
\pgfusepath{fill}%
\end{pgfscope}%
\begin{pgfscope}%
\pgfpathrectangle{\pgfqpoint{1.150000in}{0.150000in}}{\pgfqpoint{5.700000in}{5.700000in}}%
\pgfusepath{clip}%
\pgfsetbuttcap%
\pgfsetroundjoin%
\definecolor{currentfill}{rgb}{0.218130,0.347432,0.550038}%
\pgfsetfillcolor{currentfill}%
\pgfsetfillopacity{0.700000}%
\pgfsetlinewidth{0.000000pt}%
\definecolor{currentstroke}{rgb}{0.000000,0.000000,0.000000}%
\pgfsetstrokecolor{currentstroke}%
\pgfsetdash{}{0pt}%
\pgfpathmoveto{\pgfqpoint{5.888926in}{2.688910in}}%
\pgfpathlineto{\pgfqpoint{5.903207in}{2.689277in}}%
\pgfpathlineto{\pgfqpoint{5.917498in}{2.689709in}}%
\pgfpathlineto{\pgfqpoint{5.931800in}{2.690207in}}%
\pgfpathlineto{\pgfqpoint{5.946114in}{2.690770in}}%
\pgfpathlineto{\pgfqpoint{5.953315in}{2.697609in}}%
\pgfpathlineto{\pgfqpoint{5.960518in}{2.704662in}}%
\pgfpathlineto{\pgfqpoint{5.967724in}{2.711937in}}%
\pgfpathlineto{\pgfqpoint{5.974933in}{2.719443in}}%
\pgfpathlineto{\pgfqpoint{5.960649in}{2.719332in}}%
\pgfpathlineto{\pgfqpoint{5.946375in}{2.719286in}}%
\pgfpathlineto{\pgfqpoint{5.932113in}{2.719306in}}%
\pgfpathlineto{\pgfqpoint{5.917861in}{2.719391in}}%
\pgfpathlineto{\pgfqpoint{5.910623in}{2.711426in}}%
\pgfpathlineto{\pgfqpoint{5.903389in}{2.703697in}}%
\pgfpathlineto{\pgfqpoint{5.896156in}{2.696194in}}%
\pgfpathlineto{\pgfqpoint{5.888926in}{2.688910in}}%
\pgfpathclose%
\pgfusepath{fill}%
\end{pgfscope}%
\begin{pgfscope}%
\pgfpathrectangle{\pgfqpoint{1.150000in}{0.150000in}}{\pgfqpoint{5.700000in}{5.700000in}}%
\pgfusepath{clip}%
\pgfsetbuttcap%
\pgfsetroundjoin%
\definecolor{currentfill}{rgb}{0.283229,0.120777,0.440584}%
\pgfsetfillcolor{currentfill}%
\pgfsetfillopacity{0.700000}%
\pgfsetlinewidth{0.000000pt}%
\definecolor{currentstroke}{rgb}{0.000000,0.000000,0.000000}%
\pgfsetstrokecolor{currentstroke}%
\pgfsetdash{}{0pt}%
\pgfpathmoveto{\pgfqpoint{2.522595in}{2.208011in}}%
\pgfpathlineto{\pgfqpoint{2.536152in}{2.197979in}}%
\pgfpathlineto{\pgfqpoint{2.549708in}{2.188068in}}%
\pgfpathlineto{\pgfqpoint{2.563264in}{2.178279in}}%
\pgfpathlineto{\pgfqpoint{2.576820in}{2.168608in}}%
\pgfpathlineto{\pgfqpoint{2.585356in}{2.174734in}}%
\pgfpathlineto{\pgfqpoint{2.593882in}{2.180965in}}%
\pgfpathlineto{\pgfqpoint{2.602398in}{2.187299in}}%
\pgfpathlineto{\pgfqpoint{2.610904in}{2.193733in}}%
\pgfpathlineto{\pgfqpoint{2.597371in}{2.203235in}}%
\pgfpathlineto{\pgfqpoint{2.583837in}{2.212857in}}%
\pgfpathlineto{\pgfqpoint{2.570303in}{2.222599in}}%
\pgfpathlineto{\pgfqpoint{2.556769in}{2.232463in}}%
\pgfpathlineto{\pgfqpoint{2.548241in}{2.226189in}}%
\pgfpathlineto{\pgfqpoint{2.539703in}{2.220021in}}%
\pgfpathlineto{\pgfqpoint{2.531154in}{2.213961in}}%
\pgfpathlineto{\pgfqpoint{2.522595in}{2.208011in}}%
\pgfpathclose%
\pgfusepath{fill}%
\end{pgfscope}%
\begin{pgfscope}%
\pgfpathrectangle{\pgfqpoint{1.150000in}{0.150000in}}{\pgfqpoint{5.700000in}{5.700000in}}%
\pgfusepath{clip}%
\pgfsetbuttcap%
\pgfsetroundjoin%
\definecolor{currentfill}{rgb}{0.239346,0.300855,0.540844}%
\pgfsetfillcolor{currentfill}%
\pgfsetfillopacity{0.700000}%
\pgfsetlinewidth{0.000000pt}%
\definecolor{currentstroke}{rgb}{0.000000,0.000000,0.000000}%
\pgfsetstrokecolor{currentstroke}%
\pgfsetdash{}{0pt}%
\pgfpathmoveto{\pgfqpoint{5.487842in}{2.570406in}}%
\pgfpathlineto{\pgfqpoint{5.502009in}{2.571052in}}%
\pgfpathlineto{\pgfqpoint{5.516187in}{2.571764in}}%
\pgfpathlineto{\pgfqpoint{5.530375in}{2.572544in}}%
\pgfpathlineto{\pgfqpoint{5.544574in}{2.573391in}}%
\pgfpathlineto{\pgfqpoint{5.551915in}{2.579639in}}%
\pgfpathlineto{\pgfqpoint{5.559253in}{2.586010in}}%
\pgfpathlineto{\pgfqpoint{5.566589in}{2.592512in}}%
\pgfpathlineto{\pgfqpoint{5.573923in}{2.599150in}}%
\pgfpathlineto{\pgfqpoint{5.559748in}{2.598674in}}%
\pgfpathlineto{\pgfqpoint{5.545583in}{2.598265in}}%
\pgfpathlineto{\pgfqpoint{5.531429in}{2.597923in}}%
\pgfpathlineto{\pgfqpoint{5.517285in}{2.597648in}}%
\pgfpathlineto{\pgfqpoint{5.509927in}{2.590632in}}%
\pgfpathlineto{\pgfqpoint{5.502568in}{2.583757in}}%
\pgfpathlineto{\pgfqpoint{5.495206in}{2.577017in}}%
\pgfpathlineto{\pgfqpoint{5.487842in}{2.570406in}}%
\pgfpathclose%
\pgfusepath{fill}%
\end{pgfscope}%
\begin{pgfscope}%
\pgfpathrectangle{\pgfqpoint{1.150000in}{0.150000in}}{\pgfqpoint{5.700000in}{5.700000in}}%
\pgfusepath{clip}%
\pgfsetbuttcap%
\pgfsetroundjoin%
\definecolor{currentfill}{rgb}{0.282327,0.094955,0.417331}%
\pgfsetfillcolor{currentfill}%
\pgfsetfillopacity{0.700000}%
\pgfsetlinewidth{0.000000pt}%
\definecolor{currentstroke}{rgb}{0.000000,0.000000,0.000000}%
\pgfsetstrokecolor{currentstroke}%
\pgfsetdash{}{0pt}%
\pgfpathmoveto{\pgfqpoint{3.969571in}{2.131934in}}%
\pgfpathlineto{\pgfqpoint{3.983269in}{2.130210in}}%
\pgfpathlineto{\pgfqpoint{3.996974in}{2.128564in}}%
\pgfpathlineto{\pgfqpoint{4.010687in}{2.126996in}}%
\pgfpathlineto{\pgfqpoint{4.024407in}{2.125506in}}%
\pgfpathlineto{\pgfqpoint{4.032355in}{2.134178in}}%
\pgfpathlineto{\pgfqpoint{4.040297in}{2.142833in}}%
\pgfpathlineto{\pgfqpoint{4.048234in}{2.151473in}}%
\pgfpathlineto{\pgfqpoint{4.056165in}{2.160100in}}%
\pgfpathlineto{\pgfqpoint{4.042455in}{2.161651in}}%
\pgfpathlineto{\pgfqpoint{4.028753in}{2.163280in}}%
\pgfpathlineto{\pgfqpoint{4.015059in}{2.164986in}}%
\pgfpathlineto{\pgfqpoint{4.001372in}{2.166771in}}%
\pgfpathlineto{\pgfqpoint{3.993430in}{2.158077in}}%
\pgfpathlineto{\pgfqpoint{3.985482in}{2.149373in}}%
\pgfpathlineto{\pgfqpoint{3.977529in}{2.140660in}}%
\pgfpathlineto{\pgfqpoint{3.969571in}{2.131934in}}%
\pgfpathclose%
\pgfusepath{fill}%
\end{pgfscope}%
\begin{pgfscope}%
\pgfpathrectangle{\pgfqpoint{1.150000in}{0.150000in}}{\pgfqpoint{5.700000in}{5.700000in}}%
\pgfusepath{clip}%
\pgfsetbuttcap%
\pgfsetroundjoin%
\definecolor{currentfill}{rgb}{0.260571,0.246922,0.522828}%
\pgfsetfillcolor{currentfill}%
\pgfsetfillopacity{0.700000}%
\pgfsetlinewidth{0.000000pt}%
\definecolor{currentstroke}{rgb}{0.000000,0.000000,0.000000}%
\pgfsetstrokecolor{currentstroke}%
\pgfsetdash{}{0pt}%
\pgfpathmoveto{\pgfqpoint{5.086720in}{2.452823in}}%
\pgfpathlineto{\pgfqpoint{5.100762in}{2.453385in}}%
\pgfpathlineto{\pgfqpoint{5.114814in}{2.454016in}}%
\pgfpathlineto{\pgfqpoint{5.128877in}{2.454715in}}%
\pgfpathlineto{\pgfqpoint{5.142949in}{2.455484in}}%
\pgfpathlineto{\pgfqpoint{5.150456in}{2.461989in}}%
\pgfpathlineto{\pgfqpoint{5.157957in}{2.468549in}}%
\pgfpathlineto{\pgfqpoint{5.165454in}{2.475170in}}%
\pgfpathlineto{\pgfqpoint{5.172947in}{2.481856in}}%
\pgfpathlineto{\pgfqpoint{5.158893in}{2.481376in}}%
\pgfpathlineto{\pgfqpoint{5.144850in}{2.480965in}}%
\pgfpathlineto{\pgfqpoint{5.130816in}{2.480623in}}%
\pgfpathlineto{\pgfqpoint{5.116792in}{2.480350in}}%
\pgfpathlineto{\pgfqpoint{5.109281in}{2.473367in}}%
\pgfpathlineto{\pgfqpoint{5.101765in}{2.466455in}}%
\pgfpathlineto{\pgfqpoint{5.094245in}{2.459609in}}%
\pgfpathlineto{\pgfqpoint{5.086720in}{2.452823in}}%
\pgfpathclose%
\pgfusepath{fill}%
\end{pgfscope}%
\begin{pgfscope}%
\pgfpathrectangle{\pgfqpoint{1.150000in}{0.150000in}}{\pgfqpoint{5.700000in}{5.700000in}}%
\pgfusepath{clip}%
\pgfsetbuttcap%
\pgfsetroundjoin%
\definecolor{currentfill}{rgb}{0.275191,0.194905,0.496005}%
\pgfsetfillcolor{currentfill}%
\pgfsetfillopacity{0.700000}%
\pgfsetlinewidth{0.000000pt}%
\definecolor{currentstroke}{rgb}{0.000000,0.000000,0.000000}%
\pgfsetstrokecolor{currentstroke}%
\pgfsetdash{}{0pt}%
\pgfpathmoveto{\pgfqpoint{2.270753in}{2.366305in}}%
\pgfpathlineto{\pgfqpoint{2.284364in}{2.353989in}}%
\pgfpathlineto{\pgfqpoint{2.297972in}{2.341814in}}%
\pgfpathlineto{\pgfqpoint{2.311577in}{2.329778in}}%
\pgfpathlineto{\pgfqpoint{2.325179in}{2.317878in}}%
\pgfpathlineto{\pgfqpoint{2.333858in}{2.322769in}}%
\pgfpathlineto{\pgfqpoint{2.342524in}{2.327796in}}%
\pgfpathlineto{\pgfqpoint{2.351178in}{2.332956in}}%
\pgfpathlineto{\pgfqpoint{2.359821in}{2.338247in}}%
\pgfpathlineto{\pgfqpoint{2.346244in}{2.349954in}}%
\pgfpathlineto{\pgfqpoint{2.332665in}{2.361799in}}%
\pgfpathlineto{\pgfqpoint{2.319084in}{2.373782in}}%
\pgfpathlineto{\pgfqpoint{2.305500in}{2.385905in}}%
\pgfpathlineto{\pgfqpoint{2.296831in}{2.380799in}}%
\pgfpathlineto{\pgfqpoint{2.288151in}{2.375828in}}%
\pgfpathlineto{\pgfqpoint{2.279458in}{2.370996in}}%
\pgfpathlineto{\pgfqpoint{2.270753in}{2.366305in}}%
\pgfpathclose%
\pgfusepath{fill}%
\end{pgfscope}%
\begin{pgfscope}%
\pgfpathrectangle{\pgfqpoint{1.150000in}{0.150000in}}{\pgfqpoint{5.700000in}{5.700000in}}%
\pgfusepath{clip}%
\pgfsetbuttcap%
\pgfsetroundjoin%
\definecolor{currentfill}{rgb}{0.274952,0.037752,0.364543}%
\pgfsetfillcolor{currentfill}%
\pgfsetfillopacity{0.700000}%
\pgfsetlinewidth{0.000000pt}%
\definecolor{currentstroke}{rgb}{0.000000,0.000000,0.000000}%
\pgfsetstrokecolor{currentstroke}%
\pgfsetdash{}{0pt}%
\pgfpathmoveto{\pgfqpoint{3.426748in}{2.034744in}}%
\pgfpathlineto{\pgfqpoint{3.440330in}{2.030707in}}%
\pgfpathlineto{\pgfqpoint{3.453917in}{2.026757in}}%
\pgfpathlineto{\pgfqpoint{3.467509in}{2.022893in}}%
\pgfpathlineto{\pgfqpoint{3.481106in}{2.019116in}}%
\pgfpathlineto{\pgfqpoint{3.489248in}{2.027956in}}%
\pgfpathlineto{\pgfqpoint{3.497383in}{2.036804in}}%
\pgfpathlineto{\pgfqpoint{3.505512in}{2.045659in}}%
\pgfpathlineto{\pgfqpoint{3.513635in}{2.054521in}}%
\pgfpathlineto{\pgfqpoint{3.500050in}{2.058257in}}%
\pgfpathlineto{\pgfqpoint{3.486470in}{2.062079in}}%
\pgfpathlineto{\pgfqpoint{3.472895in}{2.065987in}}%
\pgfpathlineto{\pgfqpoint{3.459326in}{2.069983in}}%
\pgfpathlineto{\pgfqpoint{3.451191in}{2.061155in}}%
\pgfpathlineto{\pgfqpoint{3.443049in}{2.052339in}}%
\pgfpathlineto{\pgfqpoint{3.434902in}{2.043535in}}%
\pgfpathlineto{\pgfqpoint{3.426748in}{2.034744in}}%
\pgfpathclose%
\pgfusepath{fill}%
\end{pgfscope}%
\begin{pgfscope}%
\pgfpathrectangle{\pgfqpoint{1.150000in}{0.150000in}}{\pgfqpoint{5.700000in}{5.700000in}}%
\pgfusepath{clip}%
\pgfsetbuttcap%
\pgfsetroundjoin%
\definecolor{currentfill}{rgb}{0.275191,0.194905,0.496005}%
\pgfsetfillcolor{currentfill}%
\pgfsetfillopacity{0.700000}%
\pgfsetlinewidth{0.000000pt}%
\definecolor{currentstroke}{rgb}{0.000000,0.000000,0.000000}%
\pgfsetstrokecolor{currentstroke}%
\pgfsetdash{}{0pt}%
\pgfpathmoveto{\pgfqpoint{4.685502in}{2.333060in}}%
\pgfpathlineto{\pgfqpoint{4.699415in}{2.333163in}}%
\pgfpathlineto{\pgfqpoint{4.713337in}{2.333338in}}%
\pgfpathlineto{\pgfqpoint{4.727269in}{2.333584in}}%
\pgfpathlineto{\pgfqpoint{4.741210in}{2.333901in}}%
\pgfpathlineto{\pgfqpoint{4.748885in}{2.341150in}}%
\pgfpathlineto{\pgfqpoint{4.756554in}{2.348407in}}%
\pgfpathlineto{\pgfqpoint{4.764218in}{2.355679in}}%
\pgfpathlineto{\pgfqpoint{4.771876in}{2.362967in}}%
\pgfpathlineto{\pgfqpoint{4.757950in}{2.362856in}}%
\pgfpathlineto{\pgfqpoint{4.744033in}{2.362816in}}%
\pgfpathlineto{\pgfqpoint{4.730125in}{2.362847in}}%
\pgfpathlineto{\pgfqpoint{4.716227in}{2.362949in}}%
\pgfpathlineto{\pgfqpoint{4.708554in}{2.355447in}}%
\pgfpathlineto{\pgfqpoint{4.700876in}{2.347968in}}%
\pgfpathlineto{\pgfqpoint{4.693192in}{2.340506in}}%
\pgfpathlineto{\pgfqpoint{4.685502in}{2.333060in}}%
\pgfpathclose%
\pgfusepath{fill}%
\end{pgfscope}%
\begin{pgfscope}%
\pgfpathrectangle{\pgfqpoint{1.150000in}{0.150000in}}{\pgfqpoint{5.700000in}{5.700000in}}%
\pgfusepath{clip}%
\pgfsetbuttcap%
\pgfsetroundjoin%
\definecolor{currentfill}{rgb}{0.277941,0.056324,0.381191}%
\pgfsetfillcolor{currentfill}%
\pgfsetfillopacity{0.700000}%
\pgfsetlinewidth{0.000000pt}%
\definecolor{currentstroke}{rgb}{0.000000,0.000000,0.000000}%
\pgfsetstrokecolor{currentstroke}%
\pgfsetdash{}{0pt}%
\pgfpathmoveto{\pgfqpoint{3.654857in}{2.063392in}}%
\pgfpathlineto{\pgfqpoint{3.668482in}{2.060438in}}%
\pgfpathlineto{\pgfqpoint{3.682113in}{2.057567in}}%
\pgfpathlineto{\pgfqpoint{3.695750in}{2.054779in}}%
\pgfpathlineto{\pgfqpoint{3.709393in}{2.052073in}}%
\pgfpathlineto{\pgfqpoint{3.717453in}{2.060997in}}%
\pgfpathlineto{\pgfqpoint{3.725507in}{2.069914in}}%
\pgfpathlineto{\pgfqpoint{3.733555in}{2.078824in}}%
\pgfpathlineto{\pgfqpoint{3.741598in}{2.087727in}}%
\pgfpathlineto{\pgfqpoint{3.727966in}{2.090433in}}%
\pgfpathlineto{\pgfqpoint{3.714340in}{2.093221in}}%
\pgfpathlineto{\pgfqpoint{3.700720in}{2.096091in}}%
\pgfpathlineto{\pgfqpoint{3.687107in}{2.099043in}}%
\pgfpathlineto{\pgfqpoint{3.679053in}{2.090133in}}%
\pgfpathlineto{\pgfqpoint{3.670993in}{2.081221in}}%
\pgfpathlineto{\pgfqpoint{3.662928in}{2.072308in}}%
\pgfpathlineto{\pgfqpoint{3.654857in}{2.063392in}}%
\pgfpathclose%
\pgfusepath{fill}%
\end{pgfscope}%
\begin{pgfscope}%
\pgfpathrectangle{\pgfqpoint{1.150000in}{0.150000in}}{\pgfqpoint{5.700000in}{5.700000in}}%
\pgfusepath{clip}%
\pgfsetbuttcap%
\pgfsetroundjoin%
\definecolor{currentfill}{rgb}{0.282884,0.135920,0.453427}%
\pgfsetfillcolor{currentfill}%
\pgfsetfillopacity{0.700000}%
\pgfsetlinewidth{0.000000pt}%
\definecolor{currentstroke}{rgb}{0.000000,0.000000,0.000000}%
\pgfsetstrokecolor{currentstroke}%
\pgfsetdash{}{0pt}%
\pgfpathmoveto{\pgfqpoint{4.284250in}{2.213987in}}%
\pgfpathlineto{\pgfqpoint{4.298038in}{2.213243in}}%
\pgfpathlineto{\pgfqpoint{4.311835in}{2.212574in}}%
\pgfpathlineto{\pgfqpoint{4.325641in}{2.211979in}}%
\pgfpathlineto{\pgfqpoint{4.339454in}{2.211459in}}%
\pgfpathlineto{\pgfqpoint{4.347288in}{2.219583in}}%
\pgfpathlineto{\pgfqpoint{4.355116in}{2.227693in}}%
\pgfpathlineto{\pgfqpoint{4.362938in}{2.235791in}}%
\pgfpathlineto{\pgfqpoint{4.370754in}{2.243879in}}%
\pgfpathlineto{\pgfqpoint{4.356953in}{2.244523in}}%
\pgfpathlineto{\pgfqpoint{4.343159in}{2.245240in}}%
\pgfpathlineto{\pgfqpoint{4.329374in}{2.246033in}}%
\pgfpathlineto{\pgfqpoint{4.315597in}{2.246899in}}%
\pgfpathlineto{\pgfqpoint{4.307769in}{2.238680in}}%
\pgfpathlineto{\pgfqpoint{4.299935in}{2.230457in}}%
\pgfpathlineto{\pgfqpoint{4.292095in}{2.222227in}}%
\pgfpathlineto{\pgfqpoint{4.284250in}{2.213987in}}%
\pgfpathclose%
\pgfusepath{fill}%
\end{pgfscope}%
\begin{pgfscope}%
\pgfpathrectangle{\pgfqpoint{1.150000in}{0.150000in}}{\pgfqpoint{5.700000in}{5.700000in}}%
\pgfusepath{clip}%
\pgfsetbuttcap%
\pgfsetroundjoin%
\definecolor{currentfill}{rgb}{0.221989,0.339161,0.548752}%
\pgfsetfillcolor{currentfill}%
\pgfsetfillopacity{0.700000}%
\pgfsetlinewidth{0.000000pt}%
\definecolor{currentstroke}{rgb}{0.000000,0.000000,0.000000}%
\pgfsetstrokecolor{currentstroke}%
\pgfsetdash{}{0pt}%
\pgfpathmoveto{\pgfqpoint{5.802898in}{2.659271in}}%
\pgfpathlineto{\pgfqpoint{5.817163in}{2.659807in}}%
\pgfpathlineto{\pgfqpoint{5.831438in}{2.660409in}}%
\pgfpathlineto{\pgfqpoint{5.845724in}{2.661076in}}%
\pgfpathlineto{\pgfqpoint{5.860022in}{2.661810in}}%
\pgfpathlineto{\pgfqpoint{5.867246in}{2.668295in}}%
\pgfpathlineto{\pgfqpoint{5.874472in}{2.674968in}}%
\pgfpathlineto{\pgfqpoint{5.881698in}{2.681837in}}%
\pgfpathlineto{\pgfqpoint{5.888926in}{2.688910in}}%
\pgfpathlineto{\pgfqpoint{5.874657in}{2.688609in}}%
\pgfpathlineto{\pgfqpoint{5.860399in}{2.688374in}}%
\pgfpathlineto{\pgfqpoint{5.846151in}{2.688204in}}%
\pgfpathlineto{\pgfqpoint{5.831914in}{2.688100in}}%
\pgfpathlineto{\pgfqpoint{5.824658in}{2.680588in}}%
\pgfpathlineto{\pgfqpoint{5.817404in}{2.673285in}}%
\pgfpathlineto{\pgfqpoint{5.810151in}{2.666181in}}%
\pgfpathlineto{\pgfqpoint{5.802898in}{2.659271in}}%
\pgfpathclose%
\pgfusepath{fill}%
\end{pgfscope}%
\begin{pgfscope}%
\pgfpathrectangle{\pgfqpoint{1.150000in}{0.150000in}}{\pgfqpoint{5.700000in}{5.700000in}}%
\pgfusepath{clip}%
\pgfsetbuttcap%
\pgfsetroundjoin%
\definecolor{currentfill}{rgb}{0.274952,0.037752,0.364543}%
\pgfsetfillcolor{currentfill}%
\pgfsetfillopacity{0.700000}%
\pgfsetlinewidth{0.000000pt}%
\definecolor{currentstroke}{rgb}{0.000000,0.000000,0.000000}%
\pgfsetstrokecolor{currentstroke}%
\pgfsetdash{}{0pt}%
\pgfpathmoveto{\pgfqpoint{3.056893in}{2.035126in}}%
\pgfpathlineto{\pgfqpoint{3.070435in}{2.029001in}}%
\pgfpathlineto{\pgfqpoint{3.083980in}{2.022973in}}%
\pgfpathlineto{\pgfqpoint{3.097528in}{2.017041in}}%
\pgfpathlineto{\pgfqpoint{3.111080in}{2.011205in}}%
\pgfpathlineto{\pgfqpoint{3.119366in}{2.019354in}}%
\pgfpathlineto{\pgfqpoint{3.127645in}{2.027545in}}%
\pgfpathlineto{\pgfqpoint{3.135918in}{2.035777in}}%
\pgfpathlineto{\pgfqpoint{3.144183in}{2.044049in}}%
\pgfpathlineto{\pgfqpoint{3.130647in}{2.049782in}}%
\pgfpathlineto{\pgfqpoint{3.117114in}{2.055610in}}%
\pgfpathlineto{\pgfqpoint{3.103584in}{2.061535in}}%
\pgfpathlineto{\pgfqpoint{3.090058in}{2.067556in}}%
\pgfpathlineto{\pgfqpoint{3.081777in}{2.059380in}}%
\pgfpathlineto{\pgfqpoint{3.073490in}{2.051249in}}%
\pgfpathlineto{\pgfqpoint{3.065195in}{2.043164in}}%
\pgfpathlineto{\pgfqpoint{3.056893in}{2.035126in}}%
\pgfpathclose%
\pgfusepath{fill}%
\end{pgfscope}%
\begin{pgfscope}%
\pgfpathrectangle{\pgfqpoint{1.150000in}{0.150000in}}{\pgfqpoint{5.700000in}{5.700000in}}%
\pgfusepath{clip}%
\pgfsetbuttcap%
\pgfsetroundjoin%
\definecolor{currentfill}{rgb}{0.277018,0.050344,0.375715}%
\pgfsetfillcolor{currentfill}%
\pgfsetfillopacity{0.700000}%
\pgfsetlinewidth{0.000000pt}%
\definecolor{currentstroke}{rgb}{0.000000,0.000000,0.000000}%
\pgfsetstrokecolor{currentstroke}%
\pgfsetdash{}{0pt}%
\pgfpathmoveto{\pgfqpoint{2.915247in}{2.057027in}}%
\pgfpathlineto{\pgfqpoint{2.928784in}{2.049981in}}%
\pgfpathlineto{\pgfqpoint{2.942323in}{2.043037in}}%
\pgfpathlineto{\pgfqpoint{2.955865in}{2.036194in}}%
\pgfpathlineto{\pgfqpoint{2.969409in}{2.029452in}}%
\pgfpathlineto{\pgfqpoint{2.977757in}{2.037154in}}%
\pgfpathlineto{\pgfqpoint{2.986098in}{2.044915in}}%
\pgfpathlineto{\pgfqpoint{2.994430in}{2.052733in}}%
\pgfpathlineto{\pgfqpoint{3.002756in}{2.060606in}}%
\pgfpathlineto{\pgfqpoint{2.989228in}{2.067224in}}%
\pgfpathlineto{\pgfqpoint{2.975704in}{2.073942in}}%
\pgfpathlineto{\pgfqpoint{2.962181in}{2.080762in}}%
\pgfpathlineto{\pgfqpoint{2.948662in}{2.087683in}}%
\pgfpathlineto{\pgfqpoint{2.940320in}{2.079927in}}%
\pgfpathlineto{\pgfqpoint{2.931970in}{2.072232in}}%
\pgfpathlineto{\pgfqpoint{2.923612in}{2.064597in}}%
\pgfpathlineto{\pgfqpoint{2.915247in}{2.057027in}}%
\pgfpathclose%
\pgfusepath{fill}%
\end{pgfscope}%
\begin{pgfscope}%
\pgfpathrectangle{\pgfqpoint{1.150000in}{0.150000in}}{\pgfqpoint{5.700000in}{5.700000in}}%
\pgfusepath{clip}%
\pgfsetbuttcap%
\pgfsetroundjoin%
\definecolor{currentfill}{rgb}{0.243113,0.292092,0.538516}%
\pgfsetfillcolor{currentfill}%
\pgfsetfillopacity{0.700000}%
\pgfsetlinewidth{0.000000pt}%
\definecolor{currentstroke}{rgb}{0.000000,0.000000,0.000000}%
\pgfsetstrokecolor{currentstroke}%
\pgfsetdash{}{0pt}%
\pgfpathmoveto{\pgfqpoint{5.401708in}{2.541807in}}%
\pgfpathlineto{\pgfqpoint{5.415855in}{2.542534in}}%
\pgfpathlineto{\pgfqpoint{5.430013in}{2.543328in}}%
\pgfpathlineto{\pgfqpoint{5.444182in}{2.544190in}}%
\pgfpathlineto{\pgfqpoint{5.458361in}{2.545119in}}%
\pgfpathlineto{\pgfqpoint{5.465735in}{2.551280in}}%
\pgfpathlineto{\pgfqpoint{5.473107in}{2.557543in}}%
\pgfpathlineto{\pgfqpoint{5.480476in}{2.563917in}}%
\pgfpathlineto{\pgfqpoint{5.487842in}{2.570406in}}%
\pgfpathlineto{\pgfqpoint{5.473686in}{2.569828in}}%
\pgfpathlineto{\pgfqpoint{5.459540in}{2.569317in}}%
\pgfpathlineto{\pgfqpoint{5.445405in}{2.568873in}}%
\pgfpathlineto{\pgfqpoint{5.431280in}{2.568497in}}%
\pgfpathlineto{\pgfqpoint{5.423891in}{2.561650in}}%
\pgfpathlineto{\pgfqpoint{5.416500in}{2.554923in}}%
\pgfpathlineto{\pgfqpoint{5.409105in}{2.548311in}}%
\pgfpathlineto{\pgfqpoint{5.401708in}{2.541807in}}%
\pgfpathclose%
\pgfusepath{fill}%
\end{pgfscope}%
\begin{pgfscope}%
\pgfpathrectangle{\pgfqpoint{1.150000in}{0.150000in}}{\pgfqpoint{5.700000in}{5.700000in}}%
\pgfusepath{clip}%
\pgfsetbuttcap%
\pgfsetroundjoin%
\definecolor{currentfill}{rgb}{0.273809,0.031497,0.358853}%
\pgfsetfillcolor{currentfill}%
\pgfsetfillopacity{0.700000}%
\pgfsetlinewidth{0.000000pt}%
\definecolor{currentstroke}{rgb}{0.000000,0.000000,0.000000}%
\pgfsetstrokecolor{currentstroke}%
\pgfsetdash{}{0pt}%
\pgfpathmoveto{\pgfqpoint{3.198364in}{2.022065in}}%
\pgfpathlineto{\pgfqpoint{3.211919in}{2.016803in}}%
\pgfpathlineto{\pgfqpoint{3.225478in}{2.011633in}}%
\pgfpathlineto{\pgfqpoint{3.239041in}{2.006556in}}%
\pgfpathlineto{\pgfqpoint{3.252608in}{2.001570in}}%
\pgfpathlineto{\pgfqpoint{3.260837in}{2.010061in}}%
\pgfpathlineto{\pgfqpoint{3.269061in}{2.018581in}}%
\pgfpathlineto{\pgfqpoint{3.277278in}{2.027127in}}%
\pgfpathlineto{\pgfqpoint{3.285488in}{2.035698in}}%
\pgfpathlineto{\pgfqpoint{3.271935in}{2.040602in}}%
\pgfpathlineto{\pgfqpoint{3.258386in}{2.045596in}}%
\pgfpathlineto{\pgfqpoint{3.244842in}{2.050683in}}%
\pgfpathlineto{\pgfqpoint{3.231301in}{2.055862in}}%
\pgfpathlineto{\pgfqpoint{3.223077in}{2.047366in}}%
\pgfpathlineto{\pgfqpoint{3.214846in}{2.038900in}}%
\pgfpathlineto{\pgfqpoint{3.206608in}{2.030466in}}%
\pgfpathlineto{\pgfqpoint{3.198364in}{2.022065in}}%
\pgfpathclose%
\pgfusepath{fill}%
\end{pgfscope}%
\begin{pgfscope}%
\pgfpathrectangle{\pgfqpoint{1.150000in}{0.150000in}}{\pgfqpoint{5.700000in}{5.700000in}}%
\pgfusepath{clip}%
\pgfsetbuttcap%
\pgfsetroundjoin%
\definecolor{currentfill}{rgb}{0.263663,0.237631,0.518762}%
\pgfsetfillcolor{currentfill}%
\pgfsetfillopacity{0.700000}%
\pgfsetlinewidth{0.000000pt}%
\definecolor{currentstroke}{rgb}{0.000000,0.000000,0.000000}%
\pgfsetstrokecolor{currentstroke}%
\pgfsetdash{}{0pt}%
\pgfpathmoveto{\pgfqpoint{5.000432in}{2.423559in}}%
\pgfpathlineto{\pgfqpoint{5.014452in}{2.424111in}}%
\pgfpathlineto{\pgfqpoint{5.028482in}{2.424733in}}%
\pgfpathlineto{\pgfqpoint{5.042522in}{2.425425in}}%
\pgfpathlineto{\pgfqpoint{5.056572in}{2.426186in}}%
\pgfpathlineto{\pgfqpoint{5.064116in}{2.432779in}}%
\pgfpathlineto{\pgfqpoint{5.071656in}{2.439413in}}%
\pgfpathlineto{\pgfqpoint{5.079190in}{2.446093in}}%
\pgfpathlineto{\pgfqpoint{5.086720in}{2.452823in}}%
\pgfpathlineto{\pgfqpoint{5.072688in}{2.452331in}}%
\pgfpathlineto{\pgfqpoint{5.058666in}{2.451908in}}%
\pgfpathlineto{\pgfqpoint{5.044653in}{2.451554in}}%
\pgfpathlineto{\pgfqpoint{5.030651in}{2.451270in}}%
\pgfpathlineto{\pgfqpoint{5.023104in}{2.444264in}}%
\pgfpathlineto{\pgfqpoint{5.015551in}{2.437313in}}%
\pgfpathlineto{\pgfqpoint{5.007994in}{2.430413in}}%
\pgfpathlineto{\pgfqpoint{5.000432in}{2.423559in}}%
\pgfpathclose%
\pgfusepath{fill}%
\end{pgfscope}%
\begin{pgfscope}%
\pgfpathrectangle{\pgfqpoint{1.150000in}{0.150000in}}{\pgfqpoint{5.700000in}{5.700000in}}%
\pgfusepath{clip}%
\pgfsetbuttcap%
\pgfsetroundjoin%
\definecolor{currentfill}{rgb}{0.278826,0.175490,0.483397}%
\pgfsetfillcolor{currentfill}%
\pgfsetfillopacity{0.700000}%
\pgfsetlinewidth{0.000000pt}%
\definecolor{currentstroke}{rgb}{0.000000,0.000000,0.000000}%
\pgfsetstrokecolor{currentstroke}%
\pgfsetdash{}{0pt}%
\pgfpathmoveto{\pgfqpoint{2.325179in}{2.317878in}}%
\pgfpathlineto{\pgfqpoint{2.338779in}{2.306115in}}%
\pgfpathlineto{\pgfqpoint{2.352377in}{2.294486in}}%
\pgfpathlineto{\pgfqpoint{2.365972in}{2.282991in}}%
\pgfpathlineto{\pgfqpoint{2.379566in}{2.271628in}}%
\pgfpathlineto{\pgfqpoint{2.388218in}{2.276718in}}%
\pgfpathlineto{\pgfqpoint{2.396858in}{2.281939in}}%
\pgfpathlineto{\pgfqpoint{2.405487in}{2.287287in}}%
\pgfpathlineto{\pgfqpoint{2.414104in}{2.292761in}}%
\pgfpathlineto{\pgfqpoint{2.400536in}{2.303933in}}%
\pgfpathlineto{\pgfqpoint{2.386967in}{2.315237in}}%
\pgfpathlineto{\pgfqpoint{2.373395in}{2.326675in}}%
\pgfpathlineto{\pgfqpoint{2.359821in}{2.338247in}}%
\pgfpathlineto{\pgfqpoint{2.351178in}{2.332956in}}%
\pgfpathlineto{\pgfqpoint{2.342524in}{2.327796in}}%
\pgfpathlineto{\pgfqpoint{2.333858in}{2.322769in}}%
\pgfpathlineto{\pgfqpoint{2.325179in}{2.317878in}}%
\pgfpathclose%
\pgfusepath{fill}%
\end{pgfscope}%
\begin{pgfscope}%
\pgfpathrectangle{\pgfqpoint{1.150000in}{0.150000in}}{\pgfqpoint{5.700000in}{5.700000in}}%
\pgfusepath{clip}%
\pgfsetbuttcap%
\pgfsetroundjoin%
\definecolor{currentfill}{rgb}{0.281446,0.084320,0.407414}%
\pgfsetfillcolor{currentfill}%
\pgfsetfillopacity{0.700000}%
\pgfsetlinewidth{0.000000pt}%
\definecolor{currentstroke}{rgb}{0.000000,0.000000,0.000000}%
\pgfsetstrokecolor{currentstroke}%
\pgfsetdash{}{0pt}%
\pgfpathmoveto{\pgfqpoint{3.882914in}{2.104411in}}%
\pgfpathlineto{\pgfqpoint{3.896594in}{2.102412in}}%
\pgfpathlineto{\pgfqpoint{3.910282in}{2.100492in}}%
\pgfpathlineto{\pgfqpoint{3.923977in}{2.098651in}}%
\pgfpathlineto{\pgfqpoint{3.937679in}{2.096889in}}%
\pgfpathlineto{\pgfqpoint{3.945660in}{2.105674in}}%
\pgfpathlineto{\pgfqpoint{3.953636in}{2.114442in}}%
\pgfpathlineto{\pgfqpoint{3.961606in}{2.123195in}}%
\pgfpathlineto{\pgfqpoint{3.969571in}{2.131934in}}%
\pgfpathlineto{\pgfqpoint{3.955879in}{2.133737in}}%
\pgfpathlineto{\pgfqpoint{3.942196in}{2.135618in}}%
\pgfpathlineto{\pgfqpoint{3.928519in}{2.137579in}}%
\pgfpathlineto{\pgfqpoint{3.914849in}{2.139618in}}%
\pgfpathlineto{\pgfqpoint{3.906874in}{2.130832in}}%
\pgfpathlineto{\pgfqpoint{3.898893in}{2.122036in}}%
\pgfpathlineto{\pgfqpoint{3.890906in}{2.113229in}}%
\pgfpathlineto{\pgfqpoint{3.882914in}{2.104411in}}%
\pgfpathclose%
\pgfusepath{fill}%
\end{pgfscope}%
\begin{pgfscope}%
\pgfpathrectangle{\pgfqpoint{1.150000in}{0.150000in}}{\pgfqpoint{5.700000in}{5.700000in}}%
\pgfusepath{clip}%
\pgfsetbuttcap%
\pgfsetroundjoin%
\definecolor{currentfill}{rgb}{0.282910,0.105393,0.426902}%
\pgfsetfillcolor{currentfill}%
\pgfsetfillopacity{0.700000}%
\pgfsetlinewidth{0.000000pt}%
\definecolor{currentstroke}{rgb}{0.000000,0.000000,0.000000}%
\pgfsetstrokecolor{currentstroke}%
\pgfsetdash{}{0pt}%
\pgfpathmoveto{\pgfqpoint{2.576820in}{2.168608in}}%
\pgfpathlineto{\pgfqpoint{2.590375in}{2.159057in}}%
\pgfpathlineto{\pgfqpoint{2.603930in}{2.149622in}}%
\pgfpathlineto{\pgfqpoint{2.617485in}{2.140305in}}%
\pgfpathlineto{\pgfqpoint{2.631040in}{2.131104in}}%
\pgfpathlineto{\pgfqpoint{2.639554in}{2.137405in}}%
\pgfpathlineto{\pgfqpoint{2.648058in}{2.143806in}}%
\pgfpathlineto{\pgfqpoint{2.656553in}{2.150304in}}%
\pgfpathlineto{\pgfqpoint{2.665038in}{2.156898in}}%
\pgfpathlineto{\pgfqpoint{2.651504in}{2.165932in}}%
\pgfpathlineto{\pgfqpoint{2.637971in}{2.175082in}}%
\pgfpathlineto{\pgfqpoint{2.624438in}{2.184349in}}%
\pgfpathlineto{\pgfqpoint{2.610904in}{2.193733in}}%
\pgfpathlineto{\pgfqpoint{2.602398in}{2.187299in}}%
\pgfpathlineto{\pgfqpoint{2.593882in}{2.180965in}}%
\pgfpathlineto{\pgfqpoint{2.585356in}{2.174734in}}%
\pgfpathlineto{\pgfqpoint{2.576820in}{2.168608in}}%
\pgfpathclose%
\pgfusepath{fill}%
\end{pgfscope}%
\begin{pgfscope}%
\pgfpathrectangle{\pgfqpoint{1.150000in}{0.150000in}}{\pgfqpoint{5.700000in}{5.700000in}}%
\pgfusepath{clip}%
\pgfsetbuttcap%
\pgfsetroundjoin%
\definecolor{currentfill}{rgb}{0.278012,0.180367,0.486697}%
\pgfsetfillcolor{currentfill}%
\pgfsetfillopacity{0.700000}%
\pgfsetlinewidth{0.000000pt}%
\definecolor{currentstroke}{rgb}{0.000000,0.000000,0.000000}%
\pgfsetstrokecolor{currentstroke}%
\pgfsetdash{}{0pt}%
\pgfpathmoveto{\pgfqpoint{4.599072in}{2.302913in}}%
\pgfpathlineto{\pgfqpoint{4.612962in}{2.302913in}}%
\pgfpathlineto{\pgfqpoint{4.626861in}{2.302986in}}%
\pgfpathlineto{\pgfqpoint{4.640769in}{2.303131in}}%
\pgfpathlineto{\pgfqpoint{4.654687in}{2.303348in}}%
\pgfpathlineto{\pgfqpoint{4.662399in}{2.310772in}}%
\pgfpathlineto{\pgfqpoint{4.670106in}{2.318196in}}%
\pgfpathlineto{\pgfqpoint{4.677807in}{2.325624in}}%
\pgfpathlineto{\pgfqpoint{4.685502in}{2.333060in}}%
\pgfpathlineto{\pgfqpoint{4.671598in}{2.333029in}}%
\pgfpathlineto{\pgfqpoint{4.657704in}{2.333069in}}%
\pgfpathlineto{\pgfqpoint{4.643819in}{2.333181in}}%
\pgfpathlineto{\pgfqpoint{4.629942in}{2.333366in}}%
\pgfpathlineto{\pgfqpoint{4.622233in}{2.325737in}}%
\pgfpathlineto{\pgfqpoint{4.614518in}{2.318122in}}%
\pgfpathlineto{\pgfqpoint{4.606798in}{2.310514in}}%
\pgfpathlineto{\pgfqpoint{4.599072in}{2.302913in}}%
\pgfpathclose%
\pgfusepath{fill}%
\end{pgfscope}%
\begin{pgfscope}%
\pgfpathrectangle{\pgfqpoint{1.150000in}{0.150000in}}{\pgfqpoint{5.700000in}{5.700000in}}%
\pgfusepath{clip}%
\pgfsetbuttcap%
\pgfsetroundjoin%
\definecolor{currentfill}{rgb}{0.283187,0.125848,0.444960}%
\pgfsetfillcolor{currentfill}%
\pgfsetfillopacity{0.700000}%
\pgfsetlinewidth{0.000000pt}%
\definecolor{currentstroke}{rgb}{0.000000,0.000000,0.000000}%
\pgfsetstrokecolor{currentstroke}%
\pgfsetdash{}{0pt}%
\pgfpathmoveto{\pgfqpoint{4.197691in}{2.184209in}}%
\pgfpathlineto{\pgfqpoint{4.211458in}{2.183266in}}%
\pgfpathlineto{\pgfqpoint{4.225234in}{2.182399in}}%
\pgfpathlineto{\pgfqpoint{4.239018in}{2.181607in}}%
\pgfpathlineto{\pgfqpoint{4.252810in}{2.180890in}}%
\pgfpathlineto{\pgfqpoint{4.260678in}{2.189190in}}%
\pgfpathlineto{\pgfqpoint{4.268541in}{2.197471in}}%
\pgfpathlineto{\pgfqpoint{4.276398in}{2.205736in}}%
\pgfpathlineto{\pgfqpoint{4.284250in}{2.213987in}}%
\pgfpathlineto{\pgfqpoint{4.270469in}{2.214806in}}%
\pgfpathlineto{\pgfqpoint{4.256697in}{2.215701in}}%
\pgfpathlineto{\pgfqpoint{4.242933in}{2.216670in}}%
\pgfpathlineto{\pgfqpoint{4.229176in}{2.217715in}}%
\pgfpathlineto{\pgfqpoint{4.221314in}{2.209354in}}%
\pgfpathlineto{\pgfqpoint{4.213445in}{2.200985in}}%
\pgfpathlineto{\pgfqpoint{4.205571in}{2.192604in}}%
\pgfpathlineto{\pgfqpoint{4.197691in}{2.184209in}}%
\pgfpathclose%
\pgfusepath{fill}%
\end{pgfscope}%
\begin{pgfscope}%
\pgfpathrectangle{\pgfqpoint{1.150000in}{0.150000in}}{\pgfqpoint{5.700000in}{5.700000in}}%
\pgfusepath{clip}%
\pgfsetbuttcap%
\pgfsetroundjoin%
\definecolor{currentfill}{rgb}{0.279566,0.067836,0.391917}%
\pgfsetfillcolor{currentfill}%
\pgfsetfillopacity{0.700000}%
\pgfsetlinewidth{0.000000pt}%
\definecolor{currentstroke}{rgb}{0.000000,0.000000,0.000000}%
\pgfsetstrokecolor{currentstroke}%
\pgfsetdash{}{0pt}%
\pgfpathmoveto{\pgfqpoint{2.773331in}{2.088687in}}%
\pgfpathlineto{\pgfqpoint{2.786873in}{2.080656in}}%
\pgfpathlineto{\pgfqpoint{2.800415in}{2.072733in}}%
\pgfpathlineto{\pgfqpoint{2.813959in}{2.064917in}}%
\pgfpathlineto{\pgfqpoint{2.827505in}{2.057206in}}%
\pgfpathlineto{\pgfqpoint{2.835921in}{2.064354in}}%
\pgfpathlineto{\pgfqpoint{2.844329in}{2.071577in}}%
\pgfpathlineto{\pgfqpoint{2.852729in}{2.078873in}}%
\pgfpathlineto{\pgfqpoint{2.861120in}{2.086242in}}%
\pgfpathlineto{\pgfqpoint{2.847593in}{2.093807in}}%
\pgfpathlineto{\pgfqpoint{2.834068in}{2.101478in}}%
\pgfpathlineto{\pgfqpoint{2.820544in}{2.109255in}}%
\pgfpathlineto{\pgfqpoint{2.807022in}{2.117141in}}%
\pgfpathlineto{\pgfqpoint{2.798612in}{2.109910in}}%
\pgfpathlineto{\pgfqpoint{2.790194in}{2.102756in}}%
\pgfpathlineto{\pgfqpoint{2.781767in}{2.095681in}}%
\pgfpathlineto{\pgfqpoint{2.773331in}{2.088687in}}%
\pgfpathclose%
\pgfusepath{fill}%
\end{pgfscope}%
\begin{pgfscope}%
\pgfpathrectangle{\pgfqpoint{1.150000in}{0.150000in}}{\pgfqpoint{5.700000in}{5.700000in}}%
\pgfusepath{clip}%
\pgfsetbuttcap%
\pgfsetroundjoin%
\definecolor{currentfill}{rgb}{0.225863,0.330805,0.547314}%
\pgfsetfillcolor{currentfill}%
\pgfsetfillopacity{0.700000}%
\pgfsetlinewidth{0.000000pt}%
\definecolor{currentstroke}{rgb}{0.000000,0.000000,0.000000}%
\pgfsetstrokecolor{currentstroke}%
\pgfsetdash{}{0pt}%
\pgfpathmoveto{\pgfqpoint{5.716836in}{2.630282in}}%
\pgfpathlineto{\pgfqpoint{5.731084in}{2.630965in}}%
\pgfpathlineto{\pgfqpoint{5.745342in}{2.631715in}}%
\pgfpathlineto{\pgfqpoint{5.759611in}{2.632531in}}%
\pgfpathlineto{\pgfqpoint{5.773892in}{2.633414in}}%
\pgfpathlineto{\pgfqpoint{5.781144in}{2.639625in}}%
\pgfpathlineto{\pgfqpoint{5.788395in}{2.646000in}}%
\pgfpathlineto{\pgfqpoint{5.795647in}{2.652547in}}%
\pgfpathlineto{\pgfqpoint{5.802898in}{2.659271in}}%
\pgfpathlineto{\pgfqpoint{5.788645in}{2.658801in}}%
\pgfpathlineto{\pgfqpoint{5.774403in}{2.658398in}}%
\pgfpathlineto{\pgfqpoint{5.760171in}{2.658060in}}%
\pgfpathlineto{\pgfqpoint{5.745950in}{2.657788in}}%
\pgfpathlineto{\pgfqpoint{5.738671in}{2.650644in}}%
\pgfpathlineto{\pgfqpoint{5.731393in}{2.643684in}}%
\pgfpathlineto{\pgfqpoint{5.724115in}{2.636898in}}%
\pgfpathlineto{\pgfqpoint{5.716836in}{2.630282in}}%
\pgfpathclose%
\pgfusepath{fill}%
\end{pgfscope}%
\begin{pgfscope}%
\pgfpathrectangle{\pgfqpoint{1.150000in}{0.150000in}}{\pgfqpoint{5.700000in}{5.700000in}}%
\pgfusepath{clip}%
\pgfsetbuttcap%
\pgfsetroundjoin%
\definecolor{currentfill}{rgb}{0.277018,0.050344,0.375715}%
\pgfsetfillcolor{currentfill}%
\pgfsetfillopacity{0.700000}%
\pgfsetlinewidth{0.000000pt}%
\definecolor{currentstroke}{rgb}{0.000000,0.000000,0.000000}%
\pgfsetstrokecolor{currentstroke}%
\pgfsetdash{}{0pt}%
\pgfpathmoveto{\pgfqpoint{3.568030in}{2.040436in}}%
\pgfpathlineto{\pgfqpoint{3.581643in}{2.037126in}}%
\pgfpathlineto{\pgfqpoint{3.595262in}{2.033901in}}%
\pgfpathlineto{\pgfqpoint{3.608886in}{2.030760in}}%
\pgfpathlineto{\pgfqpoint{3.622516in}{2.027702in}}%
\pgfpathlineto{\pgfqpoint{3.630610in}{2.036629in}}%
\pgfpathlineto{\pgfqpoint{3.638698in}{2.045553in}}%
\pgfpathlineto{\pgfqpoint{3.646781in}{2.054474in}}%
\pgfpathlineto{\pgfqpoint{3.654857in}{2.063392in}}%
\pgfpathlineto{\pgfqpoint{3.641239in}{2.066429in}}%
\pgfpathlineto{\pgfqpoint{3.627626in}{2.069549in}}%
\pgfpathlineto{\pgfqpoint{3.614019in}{2.072753in}}%
\pgfpathlineto{\pgfqpoint{3.600418in}{2.076041in}}%
\pgfpathlineto{\pgfqpoint{3.592330in}{2.067136in}}%
\pgfpathlineto{\pgfqpoint{3.584236in}{2.058234in}}%
\pgfpathlineto{\pgfqpoint{3.576136in}{2.049334in}}%
\pgfpathlineto{\pgfqpoint{3.568030in}{2.040436in}}%
\pgfpathclose%
\pgfusepath{fill}%
\end{pgfscope}%
\begin{pgfscope}%
\pgfpathrectangle{\pgfqpoint{1.150000in}{0.150000in}}{\pgfqpoint{5.700000in}{5.700000in}}%
\pgfusepath{clip}%
\pgfsetbuttcap%
\pgfsetroundjoin%
\definecolor{currentfill}{rgb}{0.273809,0.031497,0.358853}%
\pgfsetfillcolor{currentfill}%
\pgfsetfillopacity{0.700000}%
\pgfsetlinewidth{0.000000pt}%
\definecolor{currentstroke}{rgb}{0.000000,0.000000,0.000000}%
\pgfsetstrokecolor{currentstroke}%
\pgfsetdash{}{0pt}%
\pgfpathmoveto{\pgfqpoint{3.339744in}{2.016992in}}%
\pgfpathlineto{\pgfqpoint{3.353319in}{2.012539in}}%
\pgfpathlineto{\pgfqpoint{3.366899in}{2.008176in}}%
\pgfpathlineto{\pgfqpoint{3.380484in}{2.003901in}}%
\pgfpathlineto{\pgfqpoint{3.394074in}{1.999714in}}%
\pgfpathlineto{\pgfqpoint{3.402252in}{2.008450in}}%
\pgfpathlineto{\pgfqpoint{3.410423in}{2.017201in}}%
\pgfpathlineto{\pgfqpoint{3.418589in}{2.025966in}}%
\pgfpathlineto{\pgfqpoint{3.426748in}{2.034744in}}%
\pgfpathlineto{\pgfqpoint{3.413171in}{2.038869in}}%
\pgfpathlineto{\pgfqpoint{3.399600in}{2.043082in}}%
\pgfpathlineto{\pgfqpoint{3.386032in}{2.047384in}}%
\pgfpathlineto{\pgfqpoint{3.372470in}{2.051774in}}%
\pgfpathlineto{\pgfqpoint{3.364298in}{2.043050in}}%
\pgfpathlineto{\pgfqpoint{3.356120in}{2.034345in}}%
\pgfpathlineto{\pgfqpoint{3.347935in}{2.025659in}}%
\pgfpathlineto{\pgfqpoint{3.339744in}{2.016992in}}%
\pgfpathclose%
\pgfusepath{fill}%
\end{pgfscope}%
\begin{pgfscope}%
\pgfpathrectangle{\pgfqpoint{1.150000in}{0.150000in}}{\pgfqpoint{5.700000in}{5.700000in}}%
\pgfusepath{clip}%
\pgfsetbuttcap%
\pgfsetroundjoin%
\definecolor{currentfill}{rgb}{0.248629,0.278775,0.534556}%
\pgfsetfillcolor{currentfill}%
\pgfsetfillopacity{0.700000}%
\pgfsetlinewidth{0.000000pt}%
\definecolor{currentstroke}{rgb}{0.000000,0.000000,0.000000}%
\pgfsetstrokecolor{currentstroke}%
\pgfsetdash{}{0pt}%
\pgfpathmoveto{\pgfqpoint{5.315516in}{2.513202in}}%
\pgfpathlineto{\pgfqpoint{5.329643in}{2.513987in}}%
\pgfpathlineto{\pgfqpoint{5.343780in}{2.514841in}}%
\pgfpathlineto{\pgfqpoint{5.357927in}{2.515763in}}%
\pgfpathlineto{\pgfqpoint{5.372086in}{2.516753in}}%
\pgfpathlineto{\pgfqpoint{5.379497in}{2.522884in}}%
\pgfpathlineto{\pgfqpoint{5.386904in}{2.529099in}}%
\pgfpathlineto{\pgfqpoint{5.394308in}{2.535405in}}%
\pgfpathlineto{\pgfqpoint{5.401708in}{2.541807in}}%
\pgfpathlineto{\pgfqpoint{5.387572in}{2.541148in}}%
\pgfpathlineto{\pgfqpoint{5.373445in}{2.540557in}}%
\pgfpathlineto{\pgfqpoint{5.359330in}{2.540033in}}%
\pgfpathlineto{\pgfqpoint{5.345224in}{2.539577in}}%
\pgfpathlineto{\pgfqpoint{5.337803in}{2.532838in}}%
\pgfpathlineto{\pgfqpoint{5.330377in}{2.526199in}}%
\pgfpathlineto{\pgfqpoint{5.322949in}{2.519656in}}%
\pgfpathlineto{\pgfqpoint{5.315516in}{2.513202in}}%
\pgfpathclose%
\pgfusepath{fill}%
\end{pgfscope}%
\begin{pgfscope}%
\pgfpathrectangle{\pgfqpoint{1.150000in}{0.150000in}}{\pgfqpoint{5.700000in}{5.700000in}}%
\pgfusepath{clip}%
\pgfsetbuttcap%
\pgfsetroundjoin%
\definecolor{currentfill}{rgb}{0.266580,0.228262,0.514349}%
\pgfsetfillcolor{currentfill}%
\pgfsetfillopacity{0.700000}%
\pgfsetlinewidth{0.000000pt}%
\definecolor{currentstroke}{rgb}{0.000000,0.000000,0.000000}%
\pgfsetstrokecolor{currentstroke}%
\pgfsetdash{}{0pt}%
\pgfpathmoveto{\pgfqpoint{4.914084in}{2.394004in}}%
\pgfpathlineto{\pgfqpoint{4.928081in}{2.394525in}}%
\pgfpathlineto{\pgfqpoint{4.942088in}{2.395115in}}%
\pgfpathlineto{\pgfqpoint{4.956105in}{2.395776in}}%
\pgfpathlineto{\pgfqpoint{4.970132in}{2.396507in}}%
\pgfpathlineto{\pgfqpoint{4.977715in}{2.403224in}}%
\pgfpathlineto{\pgfqpoint{4.985293in}{2.409969in}}%
\pgfpathlineto{\pgfqpoint{4.992865in}{2.416746in}}%
\pgfpathlineto{\pgfqpoint{5.000432in}{2.423559in}}%
\pgfpathlineto{\pgfqpoint{4.986422in}{2.423076in}}%
\pgfpathlineto{\pgfqpoint{4.972422in}{2.422664in}}%
\pgfpathlineto{\pgfqpoint{4.958432in}{2.422321in}}%
\pgfpathlineto{\pgfqpoint{4.944451in}{2.422047in}}%
\pgfpathlineto{\pgfqpoint{4.936867in}{2.414979in}}%
\pgfpathlineto{\pgfqpoint{4.929278in}{2.407952in}}%
\pgfpathlineto{\pgfqpoint{4.921684in}{2.400962in}}%
\pgfpathlineto{\pgfqpoint{4.914084in}{2.394004in}}%
\pgfpathclose%
\pgfusepath{fill}%
\end{pgfscope}%
\begin{pgfscope}%
\pgfpathrectangle{\pgfqpoint{1.150000in}{0.150000in}}{\pgfqpoint{5.700000in}{5.700000in}}%
\pgfusepath{clip}%
\pgfsetbuttcap%
\pgfsetroundjoin%
\definecolor{currentfill}{rgb}{0.279574,0.170599,0.479997}%
\pgfsetfillcolor{currentfill}%
\pgfsetfillopacity{0.700000}%
\pgfsetlinewidth{0.000000pt}%
\definecolor{currentstroke}{rgb}{0.000000,0.000000,0.000000}%
\pgfsetstrokecolor{currentstroke}%
\pgfsetdash{}{0pt}%
\pgfpathmoveto{\pgfqpoint{4.512586in}{2.272557in}}%
\pgfpathlineto{\pgfqpoint{4.526453in}{2.272432in}}%
\pgfpathlineto{\pgfqpoint{4.540329in}{2.272380in}}%
\pgfpathlineto{\pgfqpoint{4.554215in}{2.272401in}}%
\pgfpathlineto{\pgfqpoint{4.568109in}{2.272494in}}%
\pgfpathlineto{\pgfqpoint{4.575858in}{2.280107in}}%
\pgfpathlineto{\pgfqpoint{4.583602in}{2.287712in}}%
\pgfpathlineto{\pgfqpoint{4.591340in}{2.295313in}}%
\pgfpathlineto{\pgfqpoint{4.599072in}{2.302913in}}%
\pgfpathlineto{\pgfqpoint{4.585191in}{2.302984in}}%
\pgfpathlineto{\pgfqpoint{4.571319in}{2.303128in}}%
\pgfpathlineto{\pgfqpoint{4.557456in}{2.303345in}}%
\pgfpathlineto{\pgfqpoint{4.543601in}{2.303634in}}%
\pgfpathlineto{\pgfqpoint{4.535856in}{2.295863in}}%
\pgfpathlineto{\pgfqpoint{4.528105in}{2.288095in}}%
\pgfpathlineto{\pgfqpoint{4.520348in}{2.280327in}}%
\pgfpathlineto{\pgfqpoint{4.512586in}{2.272557in}}%
\pgfpathclose%
\pgfusepath{fill}%
\end{pgfscope}%
\begin{pgfscope}%
\pgfpathrectangle{\pgfqpoint{1.150000in}{0.150000in}}{\pgfqpoint{5.700000in}{5.700000in}}%
\pgfusepath{clip}%
\pgfsetbuttcap%
\pgfsetroundjoin%
\definecolor{currentfill}{rgb}{0.281412,0.155834,0.469201}%
\pgfsetfillcolor{currentfill}%
\pgfsetfillopacity{0.700000}%
\pgfsetlinewidth{0.000000pt}%
\definecolor{currentstroke}{rgb}{0.000000,0.000000,0.000000}%
\pgfsetstrokecolor{currentstroke}%
\pgfsetdash{}{0pt}%
\pgfpathmoveto{\pgfqpoint{2.379566in}{2.271628in}}%
\pgfpathlineto{\pgfqpoint{2.393157in}{2.260397in}}%
\pgfpathlineto{\pgfqpoint{2.406747in}{2.249295in}}%
\pgfpathlineto{\pgfqpoint{2.420334in}{2.238323in}}%
\pgfpathlineto{\pgfqpoint{2.433921in}{2.227478in}}%
\pgfpathlineto{\pgfqpoint{2.442547in}{2.232767in}}%
\pgfpathlineto{\pgfqpoint{2.451163in}{2.238180in}}%
\pgfpathlineto{\pgfqpoint{2.459767in}{2.243716in}}%
\pgfpathlineto{\pgfqpoint{2.468360in}{2.249373in}}%
\pgfpathlineto{\pgfqpoint{2.454798in}{2.260027in}}%
\pgfpathlineto{\pgfqpoint{2.441235in}{2.270809in}}%
\pgfpathlineto{\pgfqpoint{2.427670in}{2.281720in}}%
\pgfpathlineto{\pgfqpoint{2.414104in}{2.292761in}}%
\pgfpathlineto{\pgfqpoint{2.405487in}{2.287287in}}%
\pgfpathlineto{\pgfqpoint{2.396858in}{2.281939in}}%
\pgfpathlineto{\pgfqpoint{2.388218in}{2.276718in}}%
\pgfpathlineto{\pgfqpoint{2.379566in}{2.271628in}}%
\pgfpathclose%
\pgfusepath{fill}%
\end{pgfscope}%
\begin{pgfscope}%
\pgfpathrectangle{\pgfqpoint{1.150000in}{0.150000in}}{\pgfqpoint{5.700000in}{5.700000in}}%
\pgfusepath{clip}%
\pgfsetbuttcap%
\pgfsetroundjoin%
\definecolor{currentfill}{rgb}{0.280267,0.073417,0.397163}%
\pgfsetfillcolor{currentfill}%
\pgfsetfillopacity{0.700000}%
\pgfsetlinewidth{0.000000pt}%
\definecolor{currentstroke}{rgb}{0.000000,0.000000,0.000000}%
\pgfsetstrokecolor{currentstroke}%
\pgfsetdash{}{0pt}%
\pgfpathmoveto{\pgfqpoint{3.796190in}{2.077721in}}%
\pgfpathlineto{\pgfqpoint{3.809854in}{2.075421in}}%
\pgfpathlineto{\pgfqpoint{3.823525in}{2.073202in}}%
\pgfpathlineto{\pgfqpoint{3.837203in}{2.071064in}}%
\pgfpathlineto{\pgfqpoint{3.850888in}{2.069005in}}%
\pgfpathlineto{\pgfqpoint{3.858903in}{2.077878in}}%
\pgfpathlineto{\pgfqpoint{3.866912in}{2.086737in}}%
\pgfpathlineto{\pgfqpoint{3.874916in}{2.095581in}}%
\pgfpathlineto{\pgfqpoint{3.882914in}{2.104411in}}%
\pgfpathlineto{\pgfqpoint{3.869240in}{2.106490in}}%
\pgfpathlineto{\pgfqpoint{3.855573in}{2.108649in}}%
\pgfpathlineto{\pgfqpoint{3.841913in}{2.110888in}}%
\pgfpathlineto{\pgfqpoint{3.828259in}{2.113207in}}%
\pgfpathlineto{\pgfqpoint{3.820250in}{2.104349in}}%
\pgfpathlineto{\pgfqpoint{3.812236in}{2.095482in}}%
\pgfpathlineto{\pgfqpoint{3.804216in}{2.086607in}}%
\pgfpathlineto{\pgfqpoint{3.796190in}{2.077721in}}%
\pgfpathclose%
\pgfusepath{fill}%
\end{pgfscope}%
\begin{pgfscope}%
\pgfpathrectangle{\pgfqpoint{1.150000in}{0.150000in}}{\pgfqpoint{5.700000in}{5.700000in}}%
\pgfusepath{clip}%
\pgfsetbuttcap%
\pgfsetroundjoin%
\definecolor{currentfill}{rgb}{0.283197,0.115680,0.436115}%
\pgfsetfillcolor{currentfill}%
\pgfsetfillopacity{0.700000}%
\pgfsetlinewidth{0.000000pt}%
\definecolor{currentstroke}{rgb}{0.000000,0.000000,0.000000}%
\pgfsetstrokecolor{currentstroke}%
\pgfsetdash{}{0pt}%
\pgfpathmoveto{\pgfqpoint{4.111077in}{2.154668in}}%
\pgfpathlineto{\pgfqpoint{4.124824in}{2.153502in}}%
\pgfpathlineto{\pgfqpoint{4.138579in}{2.152413in}}%
\pgfpathlineto{\pgfqpoint{4.152342in}{2.151400in}}%
\pgfpathlineto{\pgfqpoint{4.166113in}{2.150463in}}%
\pgfpathlineto{\pgfqpoint{4.174016in}{2.158929in}}%
\pgfpathlineto{\pgfqpoint{4.181913in}{2.167374in}}%
\pgfpathlineto{\pgfqpoint{4.189805in}{2.175800in}}%
\pgfpathlineto{\pgfqpoint{4.197691in}{2.184209in}}%
\pgfpathlineto{\pgfqpoint{4.183931in}{2.185228in}}%
\pgfpathlineto{\pgfqpoint{4.170179in}{2.186323in}}%
\pgfpathlineto{\pgfqpoint{4.156435in}{2.187494in}}%
\pgfpathlineto{\pgfqpoint{4.142699in}{2.188741in}}%
\pgfpathlineto{\pgfqpoint{4.134802in}{2.180243in}}%
\pgfpathlineto{\pgfqpoint{4.126900in}{2.171732in}}%
\pgfpathlineto{\pgfqpoint{4.118991in}{2.163208in}}%
\pgfpathlineto{\pgfqpoint{4.111077in}{2.154668in}}%
\pgfpathclose%
\pgfusepath{fill}%
\end{pgfscope}%
\begin{pgfscope}%
\pgfpathrectangle{\pgfqpoint{1.150000in}{0.150000in}}{\pgfqpoint{5.700000in}{5.700000in}}%
\pgfusepath{clip}%
\pgfsetbuttcap%
\pgfsetroundjoin%
\definecolor{currentfill}{rgb}{0.231674,0.318106,0.544834}%
\pgfsetfillcolor{currentfill}%
\pgfsetfillopacity{0.700000}%
\pgfsetlinewidth{0.000000pt}%
\definecolor{currentstroke}{rgb}{0.000000,0.000000,0.000000}%
\pgfsetstrokecolor{currentstroke}%
\pgfsetdash{}{0pt}%
\pgfpathmoveto{\pgfqpoint{5.630731in}{2.601721in}}%
\pgfpathlineto{\pgfqpoint{5.644960in}{2.602531in}}%
\pgfpathlineto{\pgfqpoint{5.659200in}{2.603407in}}%
\pgfpathlineto{\pgfqpoint{5.673451in}{2.604350in}}%
\pgfpathlineto{\pgfqpoint{5.687714in}{2.605360in}}%
\pgfpathlineto{\pgfqpoint{5.694996in}{2.611372in}}%
\pgfpathlineto{\pgfqpoint{5.702277in}{2.617526in}}%
\pgfpathlineto{\pgfqpoint{5.709557in}{2.623826in}}%
\pgfpathlineto{\pgfqpoint{5.716836in}{2.630282in}}%
\pgfpathlineto{\pgfqpoint{5.702600in}{2.629665in}}%
\pgfpathlineto{\pgfqpoint{5.688375in}{2.629114in}}%
\pgfpathlineto{\pgfqpoint{5.674160in}{2.628629in}}%
\pgfpathlineto{\pgfqpoint{5.659956in}{2.628211in}}%
\pgfpathlineto{\pgfqpoint{5.652651in}{2.621357in}}%
\pgfpathlineto{\pgfqpoint{5.645346in}{2.614662in}}%
\pgfpathlineto{\pgfqpoint{5.638039in}{2.608119in}}%
\pgfpathlineto{\pgfqpoint{5.630731in}{2.601721in}}%
\pgfpathclose%
\pgfusepath{fill}%
\end{pgfscope}%
\begin{pgfscope}%
\pgfpathrectangle{\pgfqpoint{1.150000in}{0.150000in}}{\pgfqpoint{5.700000in}{5.700000in}}%
\pgfusepath{clip}%
\pgfsetbuttcap%
\pgfsetroundjoin%
\definecolor{currentfill}{rgb}{0.281924,0.089666,0.412415}%
\pgfsetfillcolor{currentfill}%
\pgfsetfillopacity{0.700000}%
\pgfsetlinewidth{0.000000pt}%
\definecolor{currentstroke}{rgb}{0.000000,0.000000,0.000000}%
\pgfsetstrokecolor{currentstroke}%
\pgfsetdash{}{0pt}%
\pgfpathmoveto{\pgfqpoint{2.631040in}{2.131104in}}%
\pgfpathlineto{\pgfqpoint{2.644596in}{2.122017in}}%
\pgfpathlineto{\pgfqpoint{2.658151in}{2.113045in}}%
\pgfpathlineto{\pgfqpoint{2.671708in}{2.104186in}}%
\pgfpathlineto{\pgfqpoint{2.685264in}{2.095439in}}%
\pgfpathlineto{\pgfqpoint{2.693757in}{2.101915in}}%
\pgfpathlineto{\pgfqpoint{2.702240in}{2.108485in}}%
\pgfpathlineto{\pgfqpoint{2.710714in}{2.115148in}}%
\pgfpathlineto{\pgfqpoint{2.719178in}{2.121901in}}%
\pgfpathlineto{\pgfqpoint{2.705642in}{2.130481in}}%
\pgfpathlineto{\pgfqpoint{2.692107in}{2.139173in}}%
\pgfpathlineto{\pgfqpoint{2.678572in}{2.147979in}}%
\pgfpathlineto{\pgfqpoint{2.665038in}{2.156898in}}%
\pgfpathlineto{\pgfqpoint{2.656553in}{2.150304in}}%
\pgfpathlineto{\pgfqpoint{2.648058in}{2.143806in}}%
\pgfpathlineto{\pgfqpoint{2.639554in}{2.137405in}}%
\pgfpathlineto{\pgfqpoint{2.631040in}{2.131104in}}%
\pgfpathclose%
\pgfusepath{fill}%
\end{pgfscope}%
\begin{pgfscope}%
\pgfpathrectangle{\pgfqpoint{1.150000in}{0.150000in}}{\pgfqpoint{5.700000in}{5.700000in}}%
\pgfusepath{clip}%
\pgfsetbuttcap%
\pgfsetroundjoin%
\definecolor{currentfill}{rgb}{0.252194,0.269783,0.531579}%
\pgfsetfillcolor{currentfill}%
\pgfsetfillopacity{0.700000}%
\pgfsetlinewidth{0.000000pt}%
\definecolor{currentstroke}{rgb}{0.000000,0.000000,0.000000}%
\pgfsetstrokecolor{currentstroke}%
\pgfsetdash{}{0pt}%
\pgfpathmoveto{\pgfqpoint{5.229264in}{2.484462in}}%
\pgfpathlineto{\pgfqpoint{5.243369in}{2.485285in}}%
\pgfpathlineto{\pgfqpoint{5.257484in}{2.486177in}}%
\pgfpathlineto{\pgfqpoint{5.271610in}{2.487137in}}%
\pgfpathlineto{\pgfqpoint{5.285747in}{2.488165in}}%
\pgfpathlineto{\pgfqpoint{5.293196in}{2.494318in}}%
\pgfpathlineto{\pgfqpoint{5.300640in}{2.500538in}}%
\pgfpathlineto{\pgfqpoint{5.308080in}{2.506831in}}%
\pgfpathlineto{\pgfqpoint{5.315516in}{2.513202in}}%
\pgfpathlineto{\pgfqpoint{5.301400in}{2.512484in}}%
\pgfpathlineto{\pgfqpoint{5.287295in}{2.511834in}}%
\pgfpathlineto{\pgfqpoint{5.273200in}{2.511253in}}%
\pgfpathlineto{\pgfqpoint{5.259115in}{2.510739in}}%
\pgfpathlineto{\pgfqpoint{5.251658in}{2.504052in}}%
\pgfpathlineto{\pgfqpoint{5.244197in}{2.497446in}}%
\pgfpathlineto{\pgfqpoint{5.236733in}{2.490918in}}%
\pgfpathlineto{\pgfqpoint{5.229264in}{2.484462in}}%
\pgfpathclose%
\pgfusepath{fill}%
\end{pgfscope}%
\begin{pgfscope}%
\pgfpathrectangle{\pgfqpoint{1.150000in}{0.150000in}}{\pgfqpoint{5.700000in}{5.700000in}}%
\pgfusepath{clip}%
\pgfsetbuttcap%
\pgfsetroundjoin%
\definecolor{currentfill}{rgb}{0.214298,0.355619,0.551184}%
\pgfsetfillcolor{currentfill}%
\pgfsetfillopacity{0.700000}%
\pgfsetlinewidth{0.000000pt}%
\definecolor{currentstroke}{rgb}{0.000000,0.000000,0.000000}%
\pgfsetstrokecolor{currentstroke}%
\pgfsetdash{}{0pt}%
\pgfpathmoveto{\pgfqpoint{5.946114in}{2.690770in}}%
\pgfpathlineto{\pgfqpoint{5.960438in}{2.691399in}}%
\pgfpathlineto{\pgfqpoint{5.974774in}{2.692093in}}%
\pgfpathlineto{\pgfqpoint{5.989122in}{2.692853in}}%
\pgfpathlineto{\pgfqpoint{5.996300in}{2.699357in}}%
\pgfpathlineto{\pgfqpoint{6.003481in}{2.706072in}}%
\pgfpathlineto{\pgfqpoint{6.010665in}{2.713006in}}%
\pgfpathlineto{\pgfqpoint{6.017851in}{2.720166in}}%
\pgfpathlineto{\pgfqpoint{6.003534in}{2.719860in}}%
\pgfpathlineto{\pgfqpoint{5.989228in}{2.719619in}}%
\pgfpathlineto{\pgfqpoint{5.974933in}{2.719443in}}%
\pgfpathlineto{\pgfqpoint{5.967724in}{2.711937in}}%
\pgfpathlineto{\pgfqpoint{5.960518in}{2.704662in}}%
\pgfpathlineto{\pgfqpoint{5.953315in}{2.697609in}}%
\pgfpathlineto{\pgfqpoint{5.946114in}{2.690770in}}%
\pgfpathclose%
\pgfusepath{fill}%
\end{pgfscope}%
\begin{pgfscope}%
\pgfpathrectangle{\pgfqpoint{1.150000in}{0.150000in}}{\pgfqpoint{5.700000in}{5.700000in}}%
\pgfusepath{clip}%
\pgfsetbuttcap%
\pgfsetroundjoin%
\definecolor{currentfill}{rgb}{0.274952,0.037752,0.364543}%
\pgfsetfillcolor{currentfill}%
\pgfsetfillopacity{0.700000}%
\pgfsetlinewidth{0.000000pt}%
\definecolor{currentstroke}{rgb}{0.000000,0.000000,0.000000}%
\pgfsetstrokecolor{currentstroke}%
\pgfsetdash{}{0pt}%
\pgfpathmoveto{\pgfqpoint{2.969409in}{2.029452in}}%
\pgfpathlineto{\pgfqpoint{2.982956in}{2.022809in}}%
\pgfpathlineto{\pgfqpoint{2.996505in}{2.016266in}}%
\pgfpathlineto{\pgfqpoint{3.010058in}{2.009822in}}%
\pgfpathlineto{\pgfqpoint{3.023613in}{2.003475in}}%
\pgfpathlineto{\pgfqpoint{3.031944in}{2.011310in}}%
\pgfpathlineto{\pgfqpoint{3.040268in}{2.019197in}}%
\pgfpathlineto{\pgfqpoint{3.048584in}{2.027137in}}%
\pgfpathlineto{\pgfqpoint{3.056893in}{2.035126in}}%
\pgfpathlineto{\pgfqpoint{3.043354in}{2.041348in}}%
\pgfpathlineto{\pgfqpoint{3.029819in}{2.047669in}}%
\pgfpathlineto{\pgfqpoint{3.016286in}{2.054088in}}%
\pgfpathlineto{\pgfqpoint{3.002756in}{2.060606in}}%
\pgfpathlineto{\pgfqpoint{2.994430in}{2.052733in}}%
\pgfpathlineto{\pgfqpoint{2.986098in}{2.044915in}}%
\pgfpathlineto{\pgfqpoint{2.977757in}{2.037154in}}%
\pgfpathlineto{\pgfqpoint{2.969409in}{2.029452in}}%
\pgfpathclose%
\pgfusepath{fill}%
\end{pgfscope}%
\begin{pgfscope}%
\pgfpathrectangle{\pgfqpoint{1.150000in}{0.150000in}}{\pgfqpoint{5.700000in}{5.700000in}}%
\pgfusepath{clip}%
\pgfsetbuttcap%
\pgfsetroundjoin%
\definecolor{currentfill}{rgb}{0.273809,0.031497,0.358853}%
\pgfsetfillcolor{currentfill}%
\pgfsetfillopacity{0.700000}%
\pgfsetlinewidth{0.000000pt}%
\definecolor{currentstroke}{rgb}{0.000000,0.000000,0.000000}%
\pgfsetstrokecolor{currentstroke}%
\pgfsetdash{}{0pt}%
\pgfpathmoveto{\pgfqpoint{3.111080in}{2.011205in}}%
\pgfpathlineto{\pgfqpoint{3.124635in}{2.005464in}}%
\pgfpathlineto{\pgfqpoint{3.138193in}{1.999818in}}%
\pgfpathlineto{\pgfqpoint{3.151755in}{1.994266in}}%
\pgfpathlineto{\pgfqpoint{3.165321in}{1.988808in}}%
\pgfpathlineto{\pgfqpoint{3.173592in}{1.997067in}}%
\pgfpathlineto{\pgfqpoint{3.181856in}{2.005364in}}%
\pgfpathlineto{\pgfqpoint{3.190114in}{2.013697in}}%
\pgfpathlineto{\pgfqpoint{3.198364in}{2.022065in}}%
\pgfpathlineto{\pgfqpoint{3.184813in}{2.027420in}}%
\pgfpathlineto{\pgfqpoint{3.171266in}{2.032869in}}%
\pgfpathlineto{\pgfqpoint{3.157723in}{2.038412in}}%
\pgfpathlineto{\pgfqpoint{3.144183in}{2.044049in}}%
\pgfpathlineto{\pgfqpoint{3.135918in}{2.035777in}}%
\pgfpathlineto{\pgfqpoint{3.127645in}{2.027545in}}%
\pgfpathlineto{\pgfqpoint{3.119366in}{2.019354in}}%
\pgfpathlineto{\pgfqpoint{3.111080in}{2.011205in}}%
\pgfpathclose%
\pgfusepath{fill}%
\end{pgfscope}%
\begin{pgfscope}%
\pgfpathrectangle{\pgfqpoint{1.150000in}{0.150000in}}{\pgfqpoint{5.700000in}{5.700000in}}%
\pgfusepath{clip}%
\pgfsetbuttcap%
\pgfsetroundjoin%
\definecolor{currentfill}{rgb}{0.270595,0.214069,0.507052}%
\pgfsetfillcolor{currentfill}%
\pgfsetfillopacity{0.700000}%
\pgfsetlinewidth{0.000000pt}%
\definecolor{currentstroke}{rgb}{0.000000,0.000000,0.000000}%
\pgfsetstrokecolor{currentstroke}%
\pgfsetdash{}{0pt}%
\pgfpathmoveto{\pgfqpoint{4.827676in}{2.364123in}}%
\pgfpathlineto{\pgfqpoint{4.841650in}{2.364589in}}%
\pgfpathlineto{\pgfqpoint{4.855634in}{2.365126in}}%
\pgfpathlineto{\pgfqpoint{4.869627in}{2.365733in}}%
\pgfpathlineto{\pgfqpoint{4.883630in}{2.366411in}}%
\pgfpathlineto{\pgfqpoint{4.891252in}{2.373282in}}%
\pgfpathlineto{\pgfqpoint{4.898868in}{2.380169in}}%
\pgfpathlineto{\pgfqpoint{4.906479in}{2.387075in}}%
\pgfpathlineto{\pgfqpoint{4.914084in}{2.394004in}}%
\pgfpathlineto{\pgfqpoint{4.900097in}{2.393554in}}%
\pgfpathlineto{\pgfqpoint{4.886119in}{2.393174in}}%
\pgfpathlineto{\pgfqpoint{4.872151in}{2.392864in}}%
\pgfpathlineto{\pgfqpoint{4.858193in}{2.392625in}}%
\pgfpathlineto{\pgfqpoint{4.850572in}{2.385461in}}%
\pgfpathlineto{\pgfqpoint{4.842946in}{2.378325in}}%
\pgfpathlineto{\pgfqpoint{4.835314in}{2.371214in}}%
\pgfpathlineto{\pgfqpoint{4.827676in}{2.364123in}}%
\pgfpathclose%
\pgfusepath{fill}%
\end{pgfscope}%
\begin{pgfscope}%
\pgfpathrectangle{\pgfqpoint{1.150000in}{0.150000in}}{\pgfqpoint{5.700000in}{5.700000in}}%
\pgfusepath{clip}%
\pgfsetbuttcap%
\pgfsetroundjoin%
\definecolor{currentfill}{rgb}{0.274952,0.037752,0.364543}%
\pgfsetfillcolor{currentfill}%
\pgfsetfillopacity{0.700000}%
\pgfsetlinewidth{0.000000pt}%
\definecolor{currentstroke}{rgb}{0.000000,0.000000,0.000000}%
\pgfsetstrokecolor{currentstroke}%
\pgfsetdash{}{0pt}%
\pgfpathmoveto{\pgfqpoint{3.481106in}{2.019116in}}%
\pgfpathlineto{\pgfqpoint{3.494709in}{2.015425in}}%
\pgfpathlineto{\pgfqpoint{3.508317in}{2.011819in}}%
\pgfpathlineto{\pgfqpoint{3.521931in}{2.008299in}}%
\pgfpathlineto{\pgfqpoint{3.535550in}{2.004864in}}%
\pgfpathlineto{\pgfqpoint{3.543678in}{2.013753in}}%
\pgfpathlineto{\pgfqpoint{3.551802in}{2.022645in}}%
\pgfpathlineto{\pgfqpoint{3.559919in}{2.031539in}}%
\pgfpathlineto{\pgfqpoint{3.568030in}{2.040436in}}%
\pgfpathlineto{\pgfqpoint{3.554423in}{2.043829in}}%
\pgfpathlineto{\pgfqpoint{3.540822in}{2.047308in}}%
\pgfpathlineto{\pgfqpoint{3.527226in}{2.050872in}}%
\pgfpathlineto{\pgfqpoint{3.513635in}{2.054521in}}%
\pgfpathlineto{\pgfqpoint{3.505512in}{2.045659in}}%
\pgfpathlineto{\pgfqpoint{3.497383in}{2.036804in}}%
\pgfpathlineto{\pgfqpoint{3.489248in}{2.027956in}}%
\pgfpathlineto{\pgfqpoint{3.481106in}{2.019116in}}%
\pgfpathclose%
\pgfusepath{fill}%
\end{pgfscope}%
\begin{pgfscope}%
\pgfpathrectangle{\pgfqpoint{1.150000in}{0.150000in}}{\pgfqpoint{5.700000in}{5.700000in}}%
\pgfusepath{clip}%
\pgfsetbuttcap%
\pgfsetroundjoin%
\definecolor{currentfill}{rgb}{0.280868,0.160771,0.472899}%
\pgfsetfillcolor{currentfill}%
\pgfsetfillopacity{0.700000}%
\pgfsetlinewidth{0.000000pt}%
\definecolor{currentstroke}{rgb}{0.000000,0.000000,0.000000}%
\pgfsetstrokecolor{currentstroke}%
\pgfsetdash{}{0pt}%
\pgfpathmoveto{\pgfqpoint{4.426046in}{2.242047in}}%
\pgfpathlineto{\pgfqpoint{4.439891in}{2.241773in}}%
\pgfpathlineto{\pgfqpoint{4.453744in}{2.241573in}}%
\pgfpathlineto{\pgfqpoint{4.467606in}{2.241446in}}%
\pgfpathlineto{\pgfqpoint{4.481477in}{2.241393in}}%
\pgfpathlineto{\pgfqpoint{4.489263in}{2.249202in}}%
\pgfpathlineto{\pgfqpoint{4.497043in}{2.256998in}}%
\pgfpathlineto{\pgfqpoint{4.504818in}{2.264782in}}%
\pgfpathlineto{\pgfqpoint{4.512586in}{2.272557in}}%
\pgfpathlineto{\pgfqpoint{4.498728in}{2.272755in}}%
\pgfpathlineto{\pgfqpoint{4.484878in}{2.273026in}}%
\pgfpathlineto{\pgfqpoint{4.471037in}{2.273370in}}%
\pgfpathlineto{\pgfqpoint{4.457205in}{2.273788in}}%
\pgfpathlineto{\pgfqpoint{4.449424in}{2.265860in}}%
\pgfpathlineto{\pgfqpoint{4.441637in}{2.257930in}}%
\pgfpathlineto{\pgfqpoint{4.433845in}{2.249993in}}%
\pgfpathlineto{\pgfqpoint{4.426046in}{2.242047in}}%
\pgfpathclose%
\pgfusepath{fill}%
\end{pgfscope}%
\begin{pgfscope}%
\pgfpathrectangle{\pgfqpoint{1.150000in}{0.150000in}}{\pgfqpoint{5.700000in}{5.700000in}}%
\pgfusepath{clip}%
\pgfsetbuttcap%
\pgfsetroundjoin%
\definecolor{currentfill}{rgb}{0.282656,0.100196,0.422160}%
\pgfsetfillcolor{currentfill}%
\pgfsetfillopacity{0.700000}%
\pgfsetlinewidth{0.000000pt}%
\definecolor{currentstroke}{rgb}{0.000000,0.000000,0.000000}%
\pgfsetstrokecolor{currentstroke}%
\pgfsetdash{}{0pt}%
\pgfpathmoveto{\pgfqpoint{4.024407in}{2.125506in}}%
\pgfpathlineto{\pgfqpoint{4.038135in}{2.124093in}}%
\pgfpathlineto{\pgfqpoint{4.051870in}{2.122758in}}%
\pgfpathlineto{\pgfqpoint{4.065613in}{2.121500in}}%
\pgfpathlineto{\pgfqpoint{4.079363in}{2.120319in}}%
\pgfpathlineto{\pgfqpoint{4.087300in}{2.128937in}}%
\pgfpathlineto{\pgfqpoint{4.095232in}{2.137534in}}%
\pgfpathlineto{\pgfqpoint{4.103157in}{2.146110in}}%
\pgfpathlineto{\pgfqpoint{4.111077in}{2.154668in}}%
\pgfpathlineto{\pgfqpoint{4.097338in}{2.155910in}}%
\pgfpathlineto{\pgfqpoint{4.083606in}{2.157230in}}%
\pgfpathlineto{\pgfqpoint{4.069881in}{2.158626in}}%
\pgfpathlineto{\pgfqpoint{4.056165in}{2.160100in}}%
\pgfpathlineto{\pgfqpoint{4.048234in}{2.151473in}}%
\pgfpathlineto{\pgfqpoint{4.040297in}{2.142833in}}%
\pgfpathlineto{\pgfqpoint{4.032355in}{2.134178in}}%
\pgfpathlineto{\pgfqpoint{4.024407in}{2.125506in}}%
\pgfpathclose%
\pgfusepath{fill}%
\end{pgfscope}%
\begin{pgfscope}%
\pgfpathrectangle{\pgfqpoint{1.150000in}{0.150000in}}{\pgfqpoint{5.700000in}{5.700000in}}%
\pgfusepath{clip}%
\pgfsetbuttcap%
\pgfsetroundjoin%
\definecolor{currentfill}{rgb}{0.282623,0.140926,0.457517}%
\pgfsetfillcolor{currentfill}%
\pgfsetfillopacity{0.700000}%
\pgfsetlinewidth{0.000000pt}%
\definecolor{currentstroke}{rgb}{0.000000,0.000000,0.000000}%
\pgfsetstrokecolor{currentstroke}%
\pgfsetdash{}{0pt}%
\pgfpathmoveto{\pgfqpoint{2.433921in}{2.227478in}}%
\pgfpathlineto{\pgfqpoint{2.447506in}{2.216761in}}%
\pgfpathlineto{\pgfqpoint{2.461089in}{2.206169in}}%
\pgfpathlineto{\pgfqpoint{2.474672in}{2.195702in}}%
\pgfpathlineto{\pgfqpoint{2.488254in}{2.185358in}}%
\pgfpathlineto{\pgfqpoint{2.496855in}{2.190844in}}%
\pgfpathlineto{\pgfqpoint{2.505446in}{2.196450in}}%
\pgfpathlineto{\pgfqpoint{2.514026in}{2.202173in}}%
\pgfpathlineto{\pgfqpoint{2.522595in}{2.208011in}}%
\pgfpathlineto{\pgfqpoint{2.509038in}{2.218165in}}%
\pgfpathlineto{\pgfqpoint{2.495480in}{2.228443in}}%
\pgfpathlineto{\pgfqpoint{2.481920in}{2.238845in}}%
\pgfpathlineto{\pgfqpoint{2.468360in}{2.249373in}}%
\pgfpathlineto{\pgfqpoint{2.459767in}{2.243716in}}%
\pgfpathlineto{\pgfqpoint{2.451163in}{2.238180in}}%
\pgfpathlineto{\pgfqpoint{2.442547in}{2.232767in}}%
\pgfpathlineto{\pgfqpoint{2.433921in}{2.227478in}}%
\pgfpathclose%
\pgfusepath{fill}%
\end{pgfscope}%
\begin{pgfscope}%
\pgfpathrectangle{\pgfqpoint{1.150000in}{0.150000in}}{\pgfqpoint{5.700000in}{5.700000in}}%
\pgfusepath{clip}%
\pgfsetbuttcap%
\pgfsetroundjoin%
\definecolor{currentfill}{rgb}{0.277941,0.056324,0.381191}%
\pgfsetfillcolor{currentfill}%
\pgfsetfillopacity{0.700000}%
\pgfsetlinewidth{0.000000pt}%
\definecolor{currentstroke}{rgb}{0.000000,0.000000,0.000000}%
\pgfsetstrokecolor{currentstroke}%
\pgfsetdash{}{0pt}%
\pgfpathmoveto{\pgfqpoint{2.827505in}{2.057206in}}%
\pgfpathlineto{\pgfqpoint{2.841052in}{2.049601in}}%
\pgfpathlineto{\pgfqpoint{2.854601in}{2.042101in}}%
\pgfpathlineto{\pgfqpoint{2.868152in}{2.034704in}}%
\pgfpathlineto{\pgfqpoint{2.881705in}{2.027411in}}%
\pgfpathlineto{\pgfqpoint{2.890103in}{2.034711in}}%
\pgfpathlineto{\pgfqpoint{2.898492in}{2.042082in}}%
\pgfpathlineto{\pgfqpoint{2.906874in}{2.049521in}}%
\pgfpathlineto{\pgfqpoint{2.915247in}{2.057027in}}%
\pgfpathlineto{\pgfqpoint{2.901712in}{2.064175in}}%
\pgfpathlineto{\pgfqpoint{2.888180in}{2.071426in}}%
\pgfpathlineto{\pgfqpoint{2.874649in}{2.078782in}}%
\pgfpathlineto{\pgfqpoint{2.861120in}{2.086242in}}%
\pgfpathlineto{\pgfqpoint{2.852729in}{2.078873in}}%
\pgfpathlineto{\pgfqpoint{2.844329in}{2.071577in}}%
\pgfpathlineto{\pgfqpoint{2.835921in}{2.064354in}}%
\pgfpathlineto{\pgfqpoint{2.827505in}{2.057206in}}%
\pgfpathclose%
\pgfusepath{fill}%
\end{pgfscope}%
\begin{pgfscope}%
\pgfpathrectangle{\pgfqpoint{1.150000in}{0.150000in}}{\pgfqpoint{5.700000in}{5.700000in}}%
\pgfusepath{clip}%
\pgfsetbuttcap%
\pgfsetroundjoin%
\definecolor{currentfill}{rgb}{0.278791,0.062145,0.386592}%
\pgfsetfillcolor{currentfill}%
\pgfsetfillopacity{0.700000}%
\pgfsetlinewidth{0.000000pt}%
\definecolor{currentstroke}{rgb}{0.000000,0.000000,0.000000}%
\pgfsetstrokecolor{currentstroke}%
\pgfsetdash{}{0pt}%
\pgfpathmoveto{\pgfqpoint{3.709393in}{2.052073in}}%
\pgfpathlineto{\pgfqpoint{3.723043in}{2.049449in}}%
\pgfpathlineto{\pgfqpoint{3.736698in}{2.046906in}}%
\pgfpathlineto{\pgfqpoint{3.750361in}{2.044444in}}%
\pgfpathlineto{\pgfqpoint{3.764030in}{2.042064in}}%
\pgfpathlineto{\pgfqpoint{3.772078in}{2.050996in}}%
\pgfpathlineto{\pgfqpoint{3.780121in}{2.059916in}}%
\pgfpathlineto{\pgfqpoint{3.788158in}{2.068824in}}%
\pgfpathlineto{\pgfqpoint{3.796190in}{2.077721in}}%
\pgfpathlineto{\pgfqpoint{3.782532in}{2.080101in}}%
\pgfpathlineto{\pgfqpoint{3.768881in}{2.082562in}}%
\pgfpathlineto{\pgfqpoint{3.755236in}{2.085104in}}%
\pgfpathlineto{\pgfqpoint{3.741598in}{2.087727in}}%
\pgfpathlineto{\pgfqpoint{3.733555in}{2.078824in}}%
\pgfpathlineto{\pgfqpoint{3.725507in}{2.069914in}}%
\pgfpathlineto{\pgfqpoint{3.717453in}{2.060997in}}%
\pgfpathlineto{\pgfqpoint{3.709393in}{2.052073in}}%
\pgfpathclose%
\pgfusepath{fill}%
\end{pgfscope}%
\begin{pgfscope}%
\pgfpathrectangle{\pgfqpoint{1.150000in}{0.150000in}}{\pgfqpoint{5.700000in}{5.700000in}}%
\pgfusepath{clip}%
\pgfsetbuttcap%
\pgfsetroundjoin%
\definecolor{currentfill}{rgb}{0.273809,0.031497,0.358853}%
\pgfsetfillcolor{currentfill}%
\pgfsetfillopacity{0.700000}%
\pgfsetlinewidth{0.000000pt}%
\definecolor{currentstroke}{rgb}{0.000000,0.000000,0.000000}%
\pgfsetstrokecolor{currentstroke}%
\pgfsetdash{}{0pt}%
\pgfpathmoveto{\pgfqpoint{3.252608in}{2.001570in}}%
\pgfpathlineto{\pgfqpoint{3.266179in}{1.996675in}}%
\pgfpathlineto{\pgfqpoint{3.279754in}{1.991871in}}%
\pgfpathlineto{\pgfqpoint{3.293334in}{1.987157in}}%
\pgfpathlineto{\pgfqpoint{3.306918in}{1.982533in}}%
\pgfpathlineto{\pgfqpoint{3.315134in}{1.991115in}}%
\pgfpathlineto{\pgfqpoint{3.323344in}{1.999719in}}%
\pgfpathlineto{\pgfqpoint{3.331547in}{2.008345in}}%
\pgfpathlineto{\pgfqpoint{3.339744in}{2.016992in}}%
\pgfpathlineto{\pgfqpoint{3.326174in}{2.021533in}}%
\pgfpathlineto{\pgfqpoint{3.312607in}{2.026165in}}%
\pgfpathlineto{\pgfqpoint{3.299046in}{2.030886in}}%
\pgfpathlineto{\pgfqpoint{3.285488in}{2.035698in}}%
\pgfpathlineto{\pgfqpoint{3.277278in}{2.027127in}}%
\pgfpathlineto{\pgfqpoint{3.269061in}{2.018581in}}%
\pgfpathlineto{\pgfqpoint{3.260837in}{2.010061in}}%
\pgfpathlineto{\pgfqpoint{3.252608in}{2.001570in}}%
\pgfpathclose%
\pgfusepath{fill}%
\end{pgfscope}%
\begin{pgfscope}%
\pgfpathrectangle{\pgfqpoint{1.150000in}{0.150000in}}{\pgfqpoint{5.700000in}{5.700000in}}%
\pgfusepath{clip}%
\pgfsetbuttcap%
\pgfsetroundjoin%
\definecolor{currentfill}{rgb}{0.235526,0.309527,0.542944}%
\pgfsetfillcolor{currentfill}%
\pgfsetfillopacity{0.700000}%
\pgfsetlinewidth{0.000000pt}%
\definecolor{currentstroke}{rgb}{0.000000,0.000000,0.000000}%
\pgfsetstrokecolor{currentstroke}%
\pgfsetdash{}{0pt}%
\pgfpathmoveto{\pgfqpoint{5.544574in}{2.573391in}}%
\pgfpathlineto{\pgfqpoint{5.558784in}{2.574306in}}%
\pgfpathlineto{\pgfqpoint{5.573005in}{2.575287in}}%
\pgfpathlineto{\pgfqpoint{5.587237in}{2.576336in}}%
\pgfpathlineto{\pgfqpoint{5.601480in}{2.577451in}}%
\pgfpathlineto{\pgfqpoint{5.608796in}{2.583333in}}%
\pgfpathlineto{\pgfqpoint{5.616110in}{2.589335in}}%
\pgfpathlineto{\pgfqpoint{5.623421in}{2.595462in}}%
\pgfpathlineto{\pgfqpoint{5.630731in}{2.601721in}}%
\pgfpathlineto{\pgfqpoint{5.616513in}{2.600978in}}%
\pgfpathlineto{\pgfqpoint{5.602305in}{2.600302in}}%
\pgfpathlineto{\pgfqpoint{5.588109in}{2.599692in}}%
\pgfpathlineto{\pgfqpoint{5.573923in}{2.599150in}}%
\pgfpathlineto{\pgfqpoint{5.566589in}{2.592512in}}%
\pgfpathlineto{\pgfqpoint{5.559253in}{2.586010in}}%
\pgfpathlineto{\pgfqpoint{5.551915in}{2.579639in}}%
\pgfpathlineto{\pgfqpoint{5.544574in}{2.573391in}}%
\pgfpathclose%
\pgfusepath{fill}%
\end{pgfscope}%
\begin{pgfscope}%
\pgfpathrectangle{\pgfqpoint{1.150000in}{0.150000in}}{\pgfqpoint{5.700000in}{5.700000in}}%
\pgfusepath{clip}%
\pgfsetbuttcap%
\pgfsetroundjoin%
\definecolor{currentfill}{rgb}{0.255645,0.260703,0.528312}%
\pgfsetfillcolor{currentfill}%
\pgfsetfillopacity{0.700000}%
\pgfsetlinewidth{0.000000pt}%
\definecolor{currentstroke}{rgb}{0.000000,0.000000,0.000000}%
\pgfsetstrokecolor{currentstroke}%
\pgfsetdash{}{0pt}%
\pgfpathmoveto{\pgfqpoint{5.142949in}{2.455484in}}%
\pgfpathlineto{\pgfqpoint{5.157032in}{2.456322in}}%
\pgfpathlineto{\pgfqpoint{5.171125in}{2.457229in}}%
\pgfpathlineto{\pgfqpoint{5.185229in}{2.458205in}}%
\pgfpathlineto{\pgfqpoint{5.199343in}{2.459249in}}%
\pgfpathlineto{\pgfqpoint{5.206830in}{2.465471in}}%
\pgfpathlineto{\pgfqpoint{5.214313in}{2.471744in}}%
\pgfpathlineto{\pgfqpoint{5.221791in}{2.478073in}}%
\pgfpathlineto{\pgfqpoint{5.229264in}{2.484462in}}%
\pgfpathlineto{\pgfqpoint{5.215169in}{2.483708in}}%
\pgfpathlineto{\pgfqpoint{5.201085in}{2.483022in}}%
\pgfpathlineto{\pgfqpoint{5.187011in}{2.482405in}}%
\pgfpathlineto{\pgfqpoint{5.172947in}{2.481856in}}%
\pgfpathlineto{\pgfqpoint{5.165454in}{2.475170in}}%
\pgfpathlineto{\pgfqpoint{5.157957in}{2.468549in}}%
\pgfpathlineto{\pgfqpoint{5.150456in}{2.461989in}}%
\pgfpathlineto{\pgfqpoint{5.142949in}{2.455484in}}%
\pgfpathclose%
\pgfusepath{fill}%
\end{pgfscope}%
\begin{pgfscope}%
\pgfpathrectangle{\pgfqpoint{1.150000in}{0.150000in}}{\pgfqpoint{5.700000in}{5.700000in}}%
\pgfusepath{clip}%
\pgfsetbuttcap%
\pgfsetroundjoin%
\definecolor{currentfill}{rgb}{0.273006,0.204520,0.501721}%
\pgfsetfillcolor{currentfill}%
\pgfsetfillopacity{0.700000}%
\pgfsetlinewidth{0.000000pt}%
\definecolor{currentstroke}{rgb}{0.000000,0.000000,0.000000}%
\pgfsetstrokecolor{currentstroke}%
\pgfsetdash{}{0pt}%
\pgfpathmoveto{\pgfqpoint{4.741210in}{2.333901in}}%
\pgfpathlineto{\pgfqpoint{4.755160in}{2.334290in}}%
\pgfpathlineto{\pgfqpoint{4.769121in}{2.334750in}}%
\pgfpathlineto{\pgfqpoint{4.783090in}{2.335282in}}%
\pgfpathlineto{\pgfqpoint{4.797070in}{2.335884in}}%
\pgfpathlineto{\pgfqpoint{4.804730in}{2.342933in}}%
\pgfpathlineto{\pgfqpoint{4.812384in}{2.349986in}}%
\pgfpathlineto{\pgfqpoint{4.820033in}{2.357049in}}%
\pgfpathlineto{\pgfqpoint{4.827676in}{2.364123in}}%
\pgfpathlineto{\pgfqpoint{4.813712in}{2.363728in}}%
\pgfpathlineto{\pgfqpoint{4.799757in}{2.363403in}}%
\pgfpathlineto{\pgfqpoint{4.785812in}{2.363150in}}%
\pgfpathlineto{\pgfqpoint{4.771876in}{2.362967in}}%
\pgfpathlineto{\pgfqpoint{4.764218in}{2.355679in}}%
\pgfpathlineto{\pgfqpoint{4.756554in}{2.348407in}}%
\pgfpathlineto{\pgfqpoint{4.748885in}{2.341150in}}%
\pgfpathlineto{\pgfqpoint{4.741210in}{2.333901in}}%
\pgfpathclose%
\pgfusepath{fill}%
\end{pgfscope}%
\begin{pgfscope}%
\pgfpathrectangle{\pgfqpoint{1.150000in}{0.150000in}}{\pgfqpoint{5.700000in}{5.700000in}}%
\pgfusepath{clip}%
\pgfsetbuttcap%
\pgfsetroundjoin%
\definecolor{currentfill}{rgb}{0.282290,0.145912,0.461510}%
\pgfsetfillcolor{currentfill}%
\pgfsetfillopacity{0.700000}%
\pgfsetlinewidth{0.000000pt}%
\definecolor{currentstroke}{rgb}{0.000000,0.000000,0.000000}%
\pgfsetstrokecolor{currentstroke}%
\pgfsetdash{}{0pt}%
\pgfpathmoveto{\pgfqpoint{4.339454in}{2.211459in}}%
\pgfpathlineto{\pgfqpoint{4.353276in}{2.211013in}}%
\pgfpathlineto{\pgfqpoint{4.367107in}{2.210641in}}%
\pgfpathlineto{\pgfqpoint{4.380946in}{2.210344in}}%
\pgfpathlineto{\pgfqpoint{4.394794in}{2.210120in}}%
\pgfpathlineto{\pgfqpoint{4.402616in}{2.218128in}}%
\pgfpathlineto{\pgfqpoint{4.410432in}{2.226117in}}%
\pgfpathlineto{\pgfqpoint{4.418242in}{2.234089in}}%
\pgfpathlineto{\pgfqpoint{4.426046in}{2.242047in}}%
\pgfpathlineto{\pgfqpoint{4.412211in}{2.242394in}}%
\pgfpathlineto{\pgfqpoint{4.398383in}{2.242815in}}%
\pgfpathlineto{\pgfqpoint{4.384565in}{2.243310in}}%
\pgfpathlineto{\pgfqpoint{4.370754in}{2.243879in}}%
\pgfpathlineto{\pgfqpoint{4.362938in}{2.235791in}}%
\pgfpathlineto{\pgfqpoint{4.355116in}{2.227693in}}%
\pgfpathlineto{\pgfqpoint{4.347288in}{2.219583in}}%
\pgfpathlineto{\pgfqpoint{4.339454in}{2.211459in}}%
\pgfpathclose%
\pgfusepath{fill}%
\end{pgfscope}%
\begin{pgfscope}%
\pgfpathrectangle{\pgfqpoint{1.150000in}{0.150000in}}{\pgfqpoint{5.700000in}{5.700000in}}%
\pgfusepath{clip}%
\pgfsetbuttcap%
\pgfsetroundjoin%
\definecolor{currentfill}{rgb}{0.218130,0.347432,0.550038}%
\pgfsetfillcolor{currentfill}%
\pgfsetfillopacity{0.700000}%
\pgfsetlinewidth{0.000000pt}%
\definecolor{currentstroke}{rgb}{0.000000,0.000000,0.000000}%
\pgfsetstrokecolor{currentstroke}%
\pgfsetdash{}{0pt}%
\pgfpathmoveto{\pgfqpoint{5.860022in}{2.661810in}}%
\pgfpathlineto{\pgfqpoint{5.874330in}{2.662609in}}%
\pgfpathlineto{\pgfqpoint{5.888650in}{2.663475in}}%
\pgfpathlineto{\pgfqpoint{5.902982in}{2.664406in}}%
\pgfpathlineto{\pgfqpoint{5.917324in}{2.665404in}}%
\pgfpathlineto{\pgfqpoint{5.924520in}{2.671462in}}%
\pgfpathlineto{\pgfqpoint{5.931717in}{2.677704in}}%
\pgfpathlineto{\pgfqpoint{5.938915in}{2.684138in}}%
\pgfpathlineto{\pgfqpoint{5.946114in}{2.690770in}}%
\pgfpathlineto{\pgfqpoint{5.931800in}{2.690207in}}%
\pgfpathlineto{\pgfqpoint{5.917498in}{2.689709in}}%
\pgfpathlineto{\pgfqpoint{5.903207in}{2.689277in}}%
\pgfpathlineto{\pgfqpoint{5.888926in}{2.688910in}}%
\pgfpathlineto{\pgfqpoint{5.881698in}{2.681837in}}%
\pgfpathlineto{\pgfqpoint{5.874472in}{2.674968in}}%
\pgfpathlineto{\pgfqpoint{5.867246in}{2.668295in}}%
\pgfpathlineto{\pgfqpoint{5.860022in}{2.661810in}}%
\pgfpathclose%
\pgfusepath{fill}%
\end{pgfscope}%
\begin{pgfscope}%
\pgfpathrectangle{\pgfqpoint{1.150000in}{0.150000in}}{\pgfqpoint{5.700000in}{5.700000in}}%
\pgfusepath{clip}%
\pgfsetbuttcap%
\pgfsetroundjoin%
\definecolor{currentfill}{rgb}{0.281924,0.089666,0.412415}%
\pgfsetfillcolor{currentfill}%
\pgfsetfillopacity{0.700000}%
\pgfsetlinewidth{0.000000pt}%
\definecolor{currentstroke}{rgb}{0.000000,0.000000,0.000000}%
\pgfsetstrokecolor{currentstroke}%
\pgfsetdash{}{0pt}%
\pgfpathmoveto{\pgfqpoint{3.937679in}{2.096889in}}%
\pgfpathlineto{\pgfqpoint{3.951388in}{2.095205in}}%
\pgfpathlineto{\pgfqpoint{3.965104in}{2.093600in}}%
\pgfpathlineto{\pgfqpoint{3.978828in}{2.092072in}}%
\pgfpathlineto{\pgfqpoint{3.992559in}{2.090623in}}%
\pgfpathlineto{\pgfqpoint{4.000529in}{2.099375in}}%
\pgfpathlineto{\pgfqpoint{4.008494in}{2.108106in}}%
\pgfpathlineto{\pgfqpoint{4.016454in}{2.116815in}}%
\pgfpathlineto{\pgfqpoint{4.024407in}{2.125506in}}%
\pgfpathlineto{\pgfqpoint{4.010687in}{2.126996in}}%
\pgfpathlineto{\pgfqpoint{3.996974in}{2.128564in}}%
\pgfpathlineto{\pgfqpoint{3.983269in}{2.130210in}}%
\pgfpathlineto{\pgfqpoint{3.969571in}{2.131934in}}%
\pgfpathlineto{\pgfqpoint{3.961606in}{2.123195in}}%
\pgfpathlineto{\pgfqpoint{3.953636in}{2.114442in}}%
\pgfpathlineto{\pgfqpoint{3.945660in}{2.105674in}}%
\pgfpathlineto{\pgfqpoint{3.937679in}{2.096889in}}%
\pgfpathclose%
\pgfusepath{fill}%
\end{pgfscope}%
\begin{pgfscope}%
\pgfpathrectangle{\pgfqpoint{1.150000in}{0.150000in}}{\pgfqpoint{5.700000in}{5.700000in}}%
\pgfusepath{clip}%
\pgfsetbuttcap%
\pgfsetroundjoin%
\definecolor{currentfill}{rgb}{0.280894,0.078907,0.402329}%
\pgfsetfillcolor{currentfill}%
\pgfsetfillopacity{0.700000}%
\pgfsetlinewidth{0.000000pt}%
\definecolor{currentstroke}{rgb}{0.000000,0.000000,0.000000}%
\pgfsetstrokecolor{currentstroke}%
\pgfsetdash{}{0pt}%
\pgfpathmoveto{\pgfqpoint{2.685264in}{2.095439in}}%
\pgfpathlineto{\pgfqpoint{2.698822in}{2.086804in}}%
\pgfpathlineto{\pgfqpoint{2.712380in}{2.078279in}}%
\pgfpathlineto{\pgfqpoint{2.725940in}{2.069865in}}%
\pgfpathlineto{\pgfqpoint{2.739500in}{2.061560in}}%
\pgfpathlineto{\pgfqpoint{2.747971in}{2.068210in}}%
\pgfpathlineto{\pgfqpoint{2.756434in}{2.074949in}}%
\pgfpathlineto{\pgfqpoint{2.764887in}{2.081776in}}%
\pgfpathlineto{\pgfqpoint{2.773331in}{2.088687in}}%
\pgfpathlineto{\pgfqpoint{2.759791in}{2.096826in}}%
\pgfpathlineto{\pgfqpoint{2.746253in}{2.105074in}}%
\pgfpathlineto{\pgfqpoint{2.732715in}{2.113432in}}%
\pgfpathlineto{\pgfqpoint{2.719178in}{2.121901in}}%
\pgfpathlineto{\pgfqpoint{2.710714in}{2.115148in}}%
\pgfpathlineto{\pgfqpoint{2.702240in}{2.108485in}}%
\pgfpathlineto{\pgfqpoint{2.693757in}{2.101915in}}%
\pgfpathlineto{\pgfqpoint{2.685264in}{2.095439in}}%
\pgfpathclose%
\pgfusepath{fill}%
\end{pgfscope}%
\begin{pgfscope}%
\pgfpathrectangle{\pgfqpoint{1.150000in}{0.150000in}}{\pgfqpoint{5.700000in}{5.700000in}}%
\pgfusepath{clip}%
\pgfsetbuttcap%
\pgfsetroundjoin%
\definecolor{currentfill}{rgb}{0.273809,0.031497,0.358853}%
\pgfsetfillcolor{currentfill}%
\pgfsetfillopacity{0.700000}%
\pgfsetlinewidth{0.000000pt}%
\definecolor{currentstroke}{rgb}{0.000000,0.000000,0.000000}%
\pgfsetstrokecolor{currentstroke}%
\pgfsetdash{}{0pt}%
\pgfpathmoveto{\pgfqpoint{3.394074in}{1.999714in}}%
\pgfpathlineto{\pgfqpoint{3.407668in}{1.995614in}}%
\pgfpathlineto{\pgfqpoint{3.421268in}{1.991602in}}%
\pgfpathlineto{\pgfqpoint{3.434872in}{1.987677in}}%
\pgfpathlineto{\pgfqpoint{3.448482in}{1.983837in}}%
\pgfpathlineto{\pgfqpoint{3.456647in}{1.992644in}}%
\pgfpathlineto{\pgfqpoint{3.464806in}{2.001459in}}%
\pgfpathlineto{\pgfqpoint{3.472959in}{2.010283in}}%
\pgfpathlineto{\pgfqpoint{3.481106in}{2.019116in}}%
\pgfpathlineto{\pgfqpoint{3.467509in}{2.022893in}}%
\pgfpathlineto{\pgfqpoint{3.453917in}{2.026757in}}%
\pgfpathlineto{\pgfqpoint{3.440330in}{2.030707in}}%
\pgfpathlineto{\pgfqpoint{3.426748in}{2.034744in}}%
\pgfpathlineto{\pgfqpoint{3.418589in}{2.025966in}}%
\pgfpathlineto{\pgfqpoint{3.410423in}{2.017201in}}%
\pgfpathlineto{\pgfqpoint{3.402252in}{2.008450in}}%
\pgfpathlineto{\pgfqpoint{3.394074in}{1.999714in}}%
\pgfpathclose%
\pgfusepath{fill}%
\end{pgfscope}%
\begin{pgfscope}%
\pgfpathrectangle{\pgfqpoint{1.150000in}{0.150000in}}{\pgfqpoint{5.700000in}{5.700000in}}%
\pgfusepath{clip}%
\pgfsetbuttcap%
\pgfsetroundjoin%
\definecolor{currentfill}{rgb}{0.277018,0.050344,0.375715}%
\pgfsetfillcolor{currentfill}%
\pgfsetfillopacity{0.700000}%
\pgfsetlinewidth{0.000000pt}%
\definecolor{currentstroke}{rgb}{0.000000,0.000000,0.000000}%
\pgfsetstrokecolor{currentstroke}%
\pgfsetdash{}{0pt}%
\pgfpathmoveto{\pgfqpoint{3.622516in}{2.027702in}}%
\pgfpathlineto{\pgfqpoint{3.636152in}{2.024727in}}%
\pgfpathlineto{\pgfqpoint{3.649794in}{2.021835in}}%
\pgfpathlineto{\pgfqpoint{3.663443in}{2.019026in}}%
\pgfpathlineto{\pgfqpoint{3.677097in}{2.016299in}}%
\pgfpathlineto{\pgfqpoint{3.685180in}{2.025254in}}%
\pgfpathlineto{\pgfqpoint{3.693257in}{2.034202in}}%
\pgfpathlineto{\pgfqpoint{3.701328in}{2.043141in}}%
\pgfpathlineto{\pgfqpoint{3.709393in}{2.052073in}}%
\pgfpathlineto{\pgfqpoint{3.695750in}{2.054779in}}%
\pgfpathlineto{\pgfqpoint{3.682113in}{2.057567in}}%
\pgfpathlineto{\pgfqpoint{3.668482in}{2.060438in}}%
\pgfpathlineto{\pgfqpoint{3.654857in}{2.063392in}}%
\pgfpathlineto{\pgfqpoint{3.646781in}{2.054474in}}%
\pgfpathlineto{\pgfqpoint{3.638698in}{2.045553in}}%
\pgfpathlineto{\pgfqpoint{3.630610in}{2.036629in}}%
\pgfpathlineto{\pgfqpoint{3.622516in}{2.027702in}}%
\pgfpathclose%
\pgfusepath{fill}%
\end{pgfscope}%
\begin{pgfscope}%
\pgfpathrectangle{\pgfqpoint{1.150000in}{0.150000in}}{\pgfqpoint{5.700000in}{5.700000in}}%
\pgfusepath{clip}%
\pgfsetbuttcap%
\pgfsetroundjoin%
\definecolor{currentfill}{rgb}{0.239346,0.300855,0.540844}%
\pgfsetfillcolor{currentfill}%
\pgfsetfillopacity{0.700000}%
\pgfsetlinewidth{0.000000pt}%
\definecolor{currentstroke}{rgb}{0.000000,0.000000,0.000000}%
\pgfsetstrokecolor{currentstroke}%
\pgfsetdash{}{0pt}%
\pgfpathmoveto{\pgfqpoint{5.458361in}{2.545119in}}%
\pgfpathlineto{\pgfqpoint{5.472551in}{2.546116in}}%
\pgfpathlineto{\pgfqpoint{5.486751in}{2.547181in}}%
\pgfpathlineto{\pgfqpoint{5.500963in}{2.548313in}}%
\pgfpathlineto{\pgfqpoint{5.515186in}{2.549512in}}%
\pgfpathlineto{\pgfqpoint{5.522537in}{2.555328in}}%
\pgfpathlineto{\pgfqpoint{5.529886in}{2.561242in}}%
\pgfpathlineto{\pgfqpoint{5.537232in}{2.567261in}}%
\pgfpathlineto{\pgfqpoint{5.544574in}{2.573391in}}%
\pgfpathlineto{\pgfqpoint{5.530375in}{2.572544in}}%
\pgfpathlineto{\pgfqpoint{5.516187in}{2.571764in}}%
\pgfpathlineto{\pgfqpoint{5.502009in}{2.571052in}}%
\pgfpathlineto{\pgfqpoint{5.487842in}{2.570406in}}%
\pgfpathlineto{\pgfqpoint{5.480476in}{2.563917in}}%
\pgfpathlineto{\pgfqpoint{5.473107in}{2.557543in}}%
\pgfpathlineto{\pgfqpoint{5.465735in}{2.551280in}}%
\pgfpathlineto{\pgfqpoint{5.458361in}{2.545119in}}%
\pgfpathclose%
\pgfusepath{fill}%
\end{pgfscope}%
\begin{pgfscope}%
\pgfpathrectangle{\pgfqpoint{1.150000in}{0.150000in}}{\pgfqpoint{5.700000in}{5.700000in}}%
\pgfusepath{clip}%
\pgfsetbuttcap%
\pgfsetroundjoin%
\definecolor{currentfill}{rgb}{0.283187,0.125848,0.444960}%
\pgfsetfillcolor{currentfill}%
\pgfsetfillopacity{0.700000}%
\pgfsetlinewidth{0.000000pt}%
\definecolor{currentstroke}{rgb}{0.000000,0.000000,0.000000}%
\pgfsetstrokecolor{currentstroke}%
\pgfsetdash{}{0pt}%
\pgfpathmoveto{\pgfqpoint{2.488254in}{2.185358in}}%
\pgfpathlineto{\pgfqpoint{2.501834in}{2.175137in}}%
\pgfpathlineto{\pgfqpoint{2.515414in}{2.165038in}}%
\pgfpathlineto{\pgfqpoint{2.528993in}{2.155059in}}%
\pgfpathlineto{\pgfqpoint{2.542572in}{2.145201in}}%
\pgfpathlineto{\pgfqpoint{2.551150in}{2.150883in}}%
\pgfpathlineto{\pgfqpoint{2.559717in}{2.156680in}}%
\pgfpathlineto{\pgfqpoint{2.568273in}{2.162589in}}%
\pgfpathlineto{\pgfqpoint{2.576820in}{2.168608in}}%
\pgfpathlineto{\pgfqpoint{2.563264in}{2.178279in}}%
\pgfpathlineto{\pgfqpoint{2.549708in}{2.188068in}}%
\pgfpathlineto{\pgfqpoint{2.536152in}{2.197979in}}%
\pgfpathlineto{\pgfqpoint{2.522595in}{2.208011in}}%
\pgfpathlineto{\pgfqpoint{2.514026in}{2.202173in}}%
\pgfpathlineto{\pgfqpoint{2.505446in}{2.196450in}}%
\pgfpathlineto{\pgfqpoint{2.496855in}{2.190844in}}%
\pgfpathlineto{\pgfqpoint{2.488254in}{2.185358in}}%
\pgfpathclose%
\pgfusepath{fill}%
\end{pgfscope}%
\begin{pgfscope}%
\pgfpathrectangle{\pgfqpoint{1.150000in}{0.150000in}}{\pgfqpoint{5.700000in}{5.700000in}}%
\pgfusepath{clip}%
\pgfsetbuttcap%
\pgfsetroundjoin%
\definecolor{currentfill}{rgb}{0.282884,0.135920,0.453427}%
\pgfsetfillcolor{currentfill}%
\pgfsetfillopacity{0.700000}%
\pgfsetlinewidth{0.000000pt}%
\definecolor{currentstroke}{rgb}{0.000000,0.000000,0.000000}%
\pgfsetstrokecolor{currentstroke}%
\pgfsetdash{}{0pt}%
\pgfpathmoveto{\pgfqpoint{4.252810in}{2.180890in}}%
\pgfpathlineto{\pgfqpoint{4.266610in}{2.180249in}}%
\pgfpathlineto{\pgfqpoint{4.280419in}{2.179682in}}%
\pgfpathlineto{\pgfqpoint{4.294236in}{2.179190in}}%
\pgfpathlineto{\pgfqpoint{4.308061in}{2.178773in}}%
\pgfpathlineto{\pgfqpoint{4.315918in}{2.186977in}}%
\pgfpathlineto{\pgfqpoint{4.323769in}{2.195158in}}%
\pgfpathlineto{\pgfqpoint{4.331615in}{2.203318in}}%
\pgfpathlineto{\pgfqpoint{4.339454in}{2.211459in}}%
\pgfpathlineto{\pgfqpoint{4.325641in}{2.211979in}}%
\pgfpathlineto{\pgfqpoint{4.311835in}{2.212574in}}%
\pgfpathlineto{\pgfqpoint{4.298038in}{2.213243in}}%
\pgfpathlineto{\pgfqpoint{4.284250in}{2.213987in}}%
\pgfpathlineto{\pgfqpoint{4.276398in}{2.205736in}}%
\pgfpathlineto{\pgfqpoint{4.268541in}{2.197471in}}%
\pgfpathlineto{\pgfqpoint{4.260678in}{2.189190in}}%
\pgfpathlineto{\pgfqpoint{4.252810in}{2.180890in}}%
\pgfpathclose%
\pgfusepath{fill}%
\end{pgfscope}%
\begin{pgfscope}%
\pgfpathrectangle{\pgfqpoint{1.150000in}{0.150000in}}{\pgfqpoint{5.700000in}{5.700000in}}%
\pgfusepath{clip}%
\pgfsetbuttcap%
\pgfsetroundjoin%
\definecolor{currentfill}{rgb}{0.258965,0.251537,0.524736}%
\pgfsetfillcolor{currentfill}%
\pgfsetfillopacity{0.700000}%
\pgfsetlinewidth{0.000000pt}%
\definecolor{currentstroke}{rgb}{0.000000,0.000000,0.000000}%
\pgfsetstrokecolor{currentstroke}%
\pgfsetdash{}{0pt}%
\pgfpathmoveto{\pgfqpoint{5.056572in}{2.426186in}}%
\pgfpathlineto{\pgfqpoint{5.070632in}{2.427016in}}%
\pgfpathlineto{\pgfqpoint{5.084702in}{2.427916in}}%
\pgfpathlineto{\pgfqpoint{5.098783in}{2.428885in}}%
\pgfpathlineto{\pgfqpoint{5.112874in}{2.429924in}}%
\pgfpathlineto{\pgfqpoint{5.120400in}{2.436255in}}%
\pgfpathlineto{\pgfqpoint{5.127922in}{2.442623in}}%
\pgfpathlineto{\pgfqpoint{5.135438in}{2.449031in}}%
\pgfpathlineto{\pgfqpoint{5.142949in}{2.455484in}}%
\pgfpathlineto{\pgfqpoint{5.128877in}{2.454715in}}%
\pgfpathlineto{\pgfqpoint{5.114814in}{2.454016in}}%
\pgfpathlineto{\pgfqpoint{5.100762in}{2.453385in}}%
\pgfpathlineto{\pgfqpoint{5.086720in}{2.452823in}}%
\pgfpathlineto{\pgfqpoint{5.079190in}{2.446093in}}%
\pgfpathlineto{\pgfqpoint{5.071656in}{2.439413in}}%
\pgfpathlineto{\pgfqpoint{5.064116in}{2.432779in}}%
\pgfpathlineto{\pgfqpoint{5.056572in}{2.426186in}}%
\pgfpathclose%
\pgfusepath{fill}%
\end{pgfscope}%
\begin{pgfscope}%
\pgfpathrectangle{\pgfqpoint{1.150000in}{0.150000in}}{\pgfqpoint{5.700000in}{5.700000in}}%
\pgfusepath{clip}%
\pgfsetbuttcap%
\pgfsetroundjoin%
\definecolor{currentfill}{rgb}{0.275191,0.194905,0.496005}%
\pgfsetfillcolor{currentfill}%
\pgfsetfillopacity{0.700000}%
\pgfsetlinewidth{0.000000pt}%
\definecolor{currentstroke}{rgb}{0.000000,0.000000,0.000000}%
\pgfsetstrokecolor{currentstroke}%
\pgfsetdash{}{0pt}%
\pgfpathmoveto{\pgfqpoint{4.654687in}{2.303348in}}%
\pgfpathlineto{\pgfqpoint{4.668614in}{2.303637in}}%
\pgfpathlineto{\pgfqpoint{4.682550in}{2.303998in}}%
\pgfpathlineto{\pgfqpoint{4.696496in}{2.304430in}}%
\pgfpathlineto{\pgfqpoint{4.710451in}{2.304934in}}%
\pgfpathlineto{\pgfqpoint{4.718150in}{2.312179in}}%
\pgfpathlineto{\pgfqpoint{4.725842in}{2.319420in}}%
\pgfpathlineto{\pgfqpoint{4.733529in}{2.326659in}}%
\pgfpathlineto{\pgfqpoint{4.741210in}{2.333901in}}%
\pgfpathlineto{\pgfqpoint{4.727269in}{2.333584in}}%
\pgfpathlineto{\pgfqpoint{4.713337in}{2.333338in}}%
\pgfpathlineto{\pgfqpoint{4.699415in}{2.333163in}}%
\pgfpathlineto{\pgfqpoint{4.685502in}{2.333060in}}%
\pgfpathlineto{\pgfqpoint{4.677807in}{2.325624in}}%
\pgfpathlineto{\pgfqpoint{4.670106in}{2.318196in}}%
\pgfpathlineto{\pgfqpoint{4.662399in}{2.310772in}}%
\pgfpathlineto{\pgfqpoint{4.654687in}{2.303348in}}%
\pgfpathclose%
\pgfusepath{fill}%
\end{pgfscope}%
\begin{pgfscope}%
\pgfpathrectangle{\pgfqpoint{1.150000in}{0.150000in}}{\pgfqpoint{5.700000in}{5.700000in}}%
\pgfusepath{clip}%
\pgfsetbuttcap%
\pgfsetroundjoin%
\definecolor{currentfill}{rgb}{0.273809,0.031497,0.358853}%
\pgfsetfillcolor{currentfill}%
\pgfsetfillopacity{0.700000}%
\pgfsetlinewidth{0.000000pt}%
\definecolor{currentstroke}{rgb}{0.000000,0.000000,0.000000}%
\pgfsetstrokecolor{currentstroke}%
\pgfsetdash{}{0pt}%
\pgfpathmoveto{\pgfqpoint{3.023613in}{2.003475in}}%
\pgfpathlineto{\pgfqpoint{3.037171in}{1.997226in}}%
\pgfpathlineto{\pgfqpoint{3.050732in}{1.991074in}}%
\pgfpathlineto{\pgfqpoint{3.064296in}{1.985019in}}%
\pgfpathlineto{\pgfqpoint{3.077863in}{1.979059in}}%
\pgfpathlineto{\pgfqpoint{3.086178in}{1.987025in}}%
\pgfpathlineto{\pgfqpoint{3.094486in}{1.995039in}}%
\pgfpathlineto{\pgfqpoint{3.102786in}{2.003100in}}%
\pgfpathlineto{\pgfqpoint{3.111080in}{2.011205in}}%
\pgfpathlineto{\pgfqpoint{3.097528in}{2.017041in}}%
\pgfpathlineto{\pgfqpoint{3.083980in}{2.022973in}}%
\pgfpathlineto{\pgfqpoint{3.070435in}{2.029001in}}%
\pgfpathlineto{\pgfqpoint{3.056893in}{2.035126in}}%
\pgfpathlineto{\pgfqpoint{3.048584in}{2.027137in}}%
\pgfpathlineto{\pgfqpoint{3.040268in}{2.019197in}}%
\pgfpathlineto{\pgfqpoint{3.031944in}{2.011310in}}%
\pgfpathlineto{\pgfqpoint{3.023613in}{2.003475in}}%
\pgfpathclose%
\pgfusepath{fill}%
\end{pgfscope}%
\begin{pgfscope}%
\pgfpathrectangle{\pgfqpoint{1.150000in}{0.150000in}}{\pgfqpoint{5.700000in}{5.700000in}}%
\pgfusepath{clip}%
\pgfsetbuttcap%
\pgfsetroundjoin%
\definecolor{currentfill}{rgb}{0.221989,0.339161,0.548752}%
\pgfsetfillcolor{currentfill}%
\pgfsetfillopacity{0.700000}%
\pgfsetlinewidth{0.000000pt}%
\definecolor{currentstroke}{rgb}{0.000000,0.000000,0.000000}%
\pgfsetstrokecolor{currentstroke}%
\pgfsetdash{}{0pt}%
\pgfpathmoveto{\pgfqpoint{5.773892in}{2.633414in}}%
\pgfpathlineto{\pgfqpoint{5.788183in}{2.634362in}}%
\pgfpathlineto{\pgfqpoint{5.802486in}{2.635377in}}%
\pgfpathlineto{\pgfqpoint{5.816800in}{2.636459in}}%
\pgfpathlineto{\pgfqpoint{5.831125in}{2.637607in}}%
\pgfpathlineto{\pgfqpoint{5.838350in}{2.643411in}}%
\pgfpathlineto{\pgfqpoint{5.845574in}{2.649375in}}%
\pgfpathlineto{\pgfqpoint{5.852798in}{2.655506in}}%
\pgfpathlineto{\pgfqpoint{5.860022in}{2.661810in}}%
\pgfpathlineto{\pgfqpoint{5.845724in}{2.661076in}}%
\pgfpathlineto{\pgfqpoint{5.831438in}{2.660409in}}%
\pgfpathlineto{\pgfqpoint{5.817163in}{2.659807in}}%
\pgfpathlineto{\pgfqpoint{5.802898in}{2.659271in}}%
\pgfpathlineto{\pgfqpoint{5.795647in}{2.652547in}}%
\pgfpathlineto{\pgfqpoint{5.788395in}{2.646000in}}%
\pgfpathlineto{\pgfqpoint{5.781144in}{2.639625in}}%
\pgfpathlineto{\pgfqpoint{5.773892in}{2.633414in}}%
\pgfpathclose%
\pgfusepath{fill}%
\end{pgfscope}%
\begin{pgfscope}%
\pgfpathrectangle{\pgfqpoint{1.150000in}{0.150000in}}{\pgfqpoint{5.700000in}{5.700000in}}%
\pgfusepath{clip}%
\pgfsetbuttcap%
\pgfsetroundjoin%
\definecolor{currentfill}{rgb}{0.274128,0.199721,0.498911}%
\pgfsetfillcolor{currentfill}%
\pgfsetfillopacity{0.700000}%
\pgfsetlinewidth{0.000000pt}%
\definecolor{currentstroke}{rgb}{0.000000,0.000000,0.000000}%
\pgfsetstrokecolor{currentstroke}%
\pgfsetdash{}{0pt}%
\pgfpathmoveto{\pgfqpoint{2.235804in}{2.349008in}}%
\pgfpathlineto{\pgfqpoint{2.249443in}{2.336479in}}%
\pgfpathlineto{\pgfqpoint{2.263079in}{2.324090in}}%
\pgfpathlineto{\pgfqpoint{2.276712in}{2.311841in}}%
\pgfpathlineto{\pgfqpoint{2.290342in}{2.299728in}}%
\pgfpathlineto{\pgfqpoint{2.299071in}{2.304048in}}%
\pgfpathlineto{\pgfqpoint{2.307786in}{2.308514in}}%
\pgfpathlineto{\pgfqpoint{2.316489in}{2.313125in}}%
\pgfpathlineto{\pgfqpoint{2.325179in}{2.317878in}}%
\pgfpathlineto{\pgfqpoint{2.311577in}{2.329778in}}%
\pgfpathlineto{\pgfqpoint{2.297972in}{2.341814in}}%
\pgfpathlineto{\pgfqpoint{2.284364in}{2.353989in}}%
\pgfpathlineto{\pgfqpoint{2.270753in}{2.366305in}}%
\pgfpathlineto{\pgfqpoint{2.262035in}{2.361757in}}%
\pgfpathlineto{\pgfqpoint{2.253305in}{2.357357in}}%
\pgfpathlineto{\pgfqpoint{2.244561in}{2.353106in}}%
\pgfpathlineto{\pgfqpoint{2.235804in}{2.349008in}}%
\pgfpathclose%
\pgfusepath{fill}%
\end{pgfscope}%
\begin{pgfscope}%
\pgfpathrectangle{\pgfqpoint{1.150000in}{0.150000in}}{\pgfqpoint{5.700000in}{5.700000in}}%
\pgfusepath{clip}%
\pgfsetbuttcap%
\pgfsetroundjoin%
\definecolor{currentfill}{rgb}{0.276022,0.044167,0.370164}%
\pgfsetfillcolor{currentfill}%
\pgfsetfillopacity{0.700000}%
\pgfsetlinewidth{0.000000pt}%
\definecolor{currentstroke}{rgb}{0.000000,0.000000,0.000000}%
\pgfsetstrokecolor{currentstroke}%
\pgfsetdash{}{0pt}%
\pgfpathmoveto{\pgfqpoint{2.881705in}{2.027411in}}%
\pgfpathlineto{\pgfqpoint{2.895260in}{2.020220in}}%
\pgfpathlineto{\pgfqpoint{2.908818in}{2.013131in}}%
\pgfpathlineto{\pgfqpoint{2.922377in}{2.006144in}}%
\pgfpathlineto{\pgfqpoint{2.935939in}{1.999257in}}%
\pgfpathlineto{\pgfqpoint{2.944318in}{2.006709in}}%
\pgfpathlineto{\pgfqpoint{2.952690in}{2.014227in}}%
\pgfpathlineto{\pgfqpoint{2.961053in}{2.021809in}}%
\pgfpathlineto{\pgfqpoint{2.969409in}{2.029452in}}%
\pgfpathlineto{\pgfqpoint{2.955865in}{2.036194in}}%
\pgfpathlineto{\pgfqpoint{2.942323in}{2.043037in}}%
\pgfpathlineto{\pgfqpoint{2.928784in}{2.049981in}}%
\pgfpathlineto{\pgfqpoint{2.915247in}{2.057027in}}%
\pgfpathlineto{\pgfqpoint{2.906874in}{2.049521in}}%
\pgfpathlineto{\pgfqpoint{2.898492in}{2.042082in}}%
\pgfpathlineto{\pgfqpoint{2.890103in}{2.034711in}}%
\pgfpathlineto{\pgfqpoint{2.881705in}{2.027411in}}%
\pgfpathclose%
\pgfusepath{fill}%
\end{pgfscope}%
\begin{pgfscope}%
\pgfpathrectangle{\pgfqpoint{1.150000in}{0.150000in}}{\pgfqpoint{5.700000in}{5.700000in}}%
\pgfusepath{clip}%
\pgfsetbuttcap%
\pgfsetroundjoin%
\definecolor{currentfill}{rgb}{0.280894,0.078907,0.402329}%
\pgfsetfillcolor{currentfill}%
\pgfsetfillopacity{0.700000}%
\pgfsetlinewidth{0.000000pt}%
\definecolor{currentstroke}{rgb}{0.000000,0.000000,0.000000}%
\pgfsetstrokecolor{currentstroke}%
\pgfsetdash{}{0pt}%
\pgfpathmoveto{\pgfqpoint{3.850888in}{2.069005in}}%
\pgfpathlineto{\pgfqpoint{3.864579in}{2.067025in}}%
\pgfpathlineto{\pgfqpoint{3.878278in}{2.065125in}}%
\pgfpathlineto{\pgfqpoint{3.891984in}{2.063305in}}%
\pgfpathlineto{\pgfqpoint{3.905696in}{2.061563in}}%
\pgfpathlineto{\pgfqpoint{3.913700in}{2.070424in}}%
\pgfpathlineto{\pgfqpoint{3.921699in}{2.079264in}}%
\pgfpathlineto{\pgfqpoint{3.929692in}{2.088086in}}%
\pgfpathlineto{\pgfqpoint{3.937679in}{2.096889in}}%
\pgfpathlineto{\pgfqpoint{3.923977in}{2.098651in}}%
\pgfpathlineto{\pgfqpoint{3.910282in}{2.100492in}}%
\pgfpathlineto{\pgfqpoint{3.896594in}{2.102412in}}%
\pgfpathlineto{\pgfqpoint{3.882914in}{2.104411in}}%
\pgfpathlineto{\pgfqpoint{3.874916in}{2.095581in}}%
\pgfpathlineto{\pgfqpoint{3.866912in}{2.086737in}}%
\pgfpathlineto{\pgfqpoint{3.858903in}{2.077878in}}%
\pgfpathlineto{\pgfqpoint{3.850888in}{2.069005in}}%
\pgfpathclose%
\pgfusepath{fill}%
\end{pgfscope}%
\begin{pgfscope}%
\pgfpathrectangle{\pgfqpoint{1.150000in}{0.150000in}}{\pgfqpoint{5.700000in}{5.700000in}}%
\pgfusepath{clip}%
\pgfsetbuttcap%
\pgfsetroundjoin%
\definecolor{currentfill}{rgb}{0.272594,0.025563,0.353093}%
\pgfsetfillcolor{currentfill}%
\pgfsetfillopacity{0.700000}%
\pgfsetlinewidth{0.000000pt}%
\definecolor{currentstroke}{rgb}{0.000000,0.000000,0.000000}%
\pgfsetstrokecolor{currentstroke}%
\pgfsetdash{}{0pt}%
\pgfpathmoveto{\pgfqpoint{3.165321in}{1.988808in}}%
\pgfpathlineto{\pgfqpoint{3.178890in}{1.983442in}}%
\pgfpathlineto{\pgfqpoint{3.192464in}{1.978170in}}%
\pgfpathlineto{\pgfqpoint{3.206041in}{1.972989in}}%
\pgfpathlineto{\pgfqpoint{3.219622in}{1.967900in}}%
\pgfpathlineto{\pgfqpoint{3.227879in}{1.976271in}}%
\pgfpathlineto{\pgfqpoint{3.236128in}{1.984673in}}%
\pgfpathlineto{\pgfqpoint{3.244371in}{1.993107in}}%
\pgfpathlineto{\pgfqpoint{3.252608in}{2.001570in}}%
\pgfpathlineto{\pgfqpoint{3.239041in}{2.006556in}}%
\pgfpathlineto{\pgfqpoint{3.225478in}{2.011633in}}%
\pgfpathlineto{\pgfqpoint{3.211919in}{2.016803in}}%
\pgfpathlineto{\pgfqpoint{3.198364in}{2.022065in}}%
\pgfpathlineto{\pgfqpoint{3.190114in}{2.013697in}}%
\pgfpathlineto{\pgfqpoint{3.181856in}{2.005364in}}%
\pgfpathlineto{\pgfqpoint{3.173592in}{1.997067in}}%
\pgfpathlineto{\pgfqpoint{3.165321in}{1.988808in}}%
\pgfpathclose%
\pgfusepath{fill}%
\end{pgfscope}%
\begin{pgfscope}%
\pgfpathrectangle{\pgfqpoint{1.150000in}{0.150000in}}{\pgfqpoint{5.700000in}{5.700000in}}%
\pgfusepath{clip}%
\pgfsetbuttcap%
\pgfsetroundjoin%
\definecolor{currentfill}{rgb}{0.243113,0.292092,0.538516}%
\pgfsetfillcolor{currentfill}%
\pgfsetfillopacity{0.700000}%
\pgfsetlinewidth{0.000000pt}%
\definecolor{currentstroke}{rgb}{0.000000,0.000000,0.000000}%
\pgfsetstrokecolor{currentstroke}%
\pgfsetdash{}{0pt}%
\pgfpathmoveto{\pgfqpoint{5.372086in}{2.516753in}}%
\pgfpathlineto{\pgfqpoint{5.386255in}{2.517811in}}%
\pgfpathlineto{\pgfqpoint{5.400434in}{2.518937in}}%
\pgfpathlineto{\pgfqpoint{5.414625in}{2.520131in}}%
\pgfpathlineto{\pgfqpoint{5.428827in}{2.521392in}}%
\pgfpathlineto{\pgfqpoint{5.436216in}{2.527199in}}%
\pgfpathlineto{\pgfqpoint{5.443601in}{2.533085in}}%
\pgfpathlineto{\pgfqpoint{5.450983in}{2.539056in}}%
\pgfpathlineto{\pgfqpoint{5.458361in}{2.545119in}}%
\pgfpathlineto{\pgfqpoint{5.444182in}{2.544190in}}%
\pgfpathlineto{\pgfqpoint{5.430013in}{2.543328in}}%
\pgfpathlineto{\pgfqpoint{5.415855in}{2.542534in}}%
\pgfpathlineto{\pgfqpoint{5.401708in}{2.541807in}}%
\pgfpathlineto{\pgfqpoint{5.394308in}{2.535405in}}%
\pgfpathlineto{\pgfqpoint{5.386904in}{2.529099in}}%
\pgfpathlineto{\pgfqpoint{5.379497in}{2.522884in}}%
\pgfpathlineto{\pgfqpoint{5.372086in}{2.516753in}}%
\pgfpathclose%
\pgfusepath{fill}%
\end{pgfscope}%
\begin{pgfscope}%
\pgfpathrectangle{\pgfqpoint{1.150000in}{0.150000in}}{\pgfqpoint{5.700000in}{5.700000in}}%
\pgfusepath{clip}%
\pgfsetbuttcap%
\pgfsetroundjoin%
\definecolor{currentfill}{rgb}{0.278012,0.180367,0.486697}%
\pgfsetfillcolor{currentfill}%
\pgfsetfillopacity{0.700000}%
\pgfsetlinewidth{0.000000pt}%
\definecolor{currentstroke}{rgb}{0.000000,0.000000,0.000000}%
\pgfsetstrokecolor{currentstroke}%
\pgfsetdash{}{0pt}%
\pgfpathmoveto{\pgfqpoint{4.568109in}{2.272494in}}%
\pgfpathlineto{\pgfqpoint{4.582012in}{2.272660in}}%
\pgfpathlineto{\pgfqpoint{4.595925in}{2.272898in}}%
\pgfpathlineto{\pgfqpoint{4.609847in}{2.273208in}}%
\pgfpathlineto{\pgfqpoint{4.623778in}{2.273591in}}%
\pgfpathlineto{\pgfqpoint{4.631514in}{2.281046in}}%
\pgfpathlineto{\pgfqpoint{4.639244in}{2.288488in}}%
\pgfpathlineto{\pgfqpoint{4.646969in}{2.295921in}}%
\pgfpathlineto{\pgfqpoint{4.654687in}{2.303348in}}%
\pgfpathlineto{\pgfqpoint{4.640769in}{2.303131in}}%
\pgfpathlineto{\pgfqpoint{4.626861in}{2.302986in}}%
\pgfpathlineto{\pgfqpoint{4.612962in}{2.302913in}}%
\pgfpathlineto{\pgfqpoint{4.599072in}{2.302913in}}%
\pgfpathlineto{\pgfqpoint{4.591340in}{2.295313in}}%
\pgfpathlineto{\pgfqpoint{4.583602in}{2.287712in}}%
\pgfpathlineto{\pgfqpoint{4.575858in}{2.280107in}}%
\pgfpathlineto{\pgfqpoint{4.568109in}{2.272494in}}%
\pgfpathclose%
\pgfusepath{fill}%
\end{pgfscope}%
\begin{pgfscope}%
\pgfpathrectangle{\pgfqpoint{1.150000in}{0.150000in}}{\pgfqpoint{5.700000in}{5.700000in}}%
\pgfusepath{clip}%
\pgfsetbuttcap%
\pgfsetroundjoin%
\definecolor{currentfill}{rgb}{0.283229,0.120777,0.440584}%
\pgfsetfillcolor{currentfill}%
\pgfsetfillopacity{0.700000}%
\pgfsetlinewidth{0.000000pt}%
\definecolor{currentstroke}{rgb}{0.000000,0.000000,0.000000}%
\pgfsetstrokecolor{currentstroke}%
\pgfsetdash{}{0pt}%
\pgfpathmoveto{\pgfqpoint{4.166113in}{2.150463in}}%
\pgfpathlineto{\pgfqpoint{4.179892in}{2.149602in}}%
\pgfpathlineto{\pgfqpoint{4.193679in}{2.148817in}}%
\pgfpathlineto{\pgfqpoint{4.207474in}{2.148107in}}%
\pgfpathlineto{\pgfqpoint{4.221277in}{2.147472in}}%
\pgfpathlineto{\pgfqpoint{4.229169in}{2.155864in}}%
\pgfpathlineto{\pgfqpoint{4.237055in}{2.164229in}}%
\pgfpathlineto{\pgfqpoint{4.244935in}{2.172571in}}%
\pgfpathlineto{\pgfqpoint{4.252810in}{2.180890in}}%
\pgfpathlineto{\pgfqpoint{4.239018in}{2.181607in}}%
\pgfpathlineto{\pgfqpoint{4.225234in}{2.182399in}}%
\pgfpathlineto{\pgfqpoint{4.211458in}{2.183266in}}%
\pgfpathlineto{\pgfqpoint{4.197691in}{2.184209in}}%
\pgfpathlineto{\pgfqpoint{4.189805in}{2.175800in}}%
\pgfpathlineto{\pgfqpoint{4.181913in}{2.167374in}}%
\pgfpathlineto{\pgfqpoint{4.174016in}{2.158929in}}%
\pgfpathlineto{\pgfqpoint{4.166113in}{2.150463in}}%
\pgfpathclose%
\pgfusepath{fill}%
\end{pgfscope}%
\begin{pgfscope}%
\pgfpathrectangle{\pgfqpoint{1.150000in}{0.150000in}}{\pgfqpoint{5.700000in}{5.700000in}}%
\pgfusepath{clip}%
\pgfsetbuttcap%
\pgfsetroundjoin%
\definecolor{currentfill}{rgb}{0.263663,0.237631,0.518762}%
\pgfsetfillcolor{currentfill}%
\pgfsetfillopacity{0.700000}%
\pgfsetlinewidth{0.000000pt}%
\definecolor{currentstroke}{rgb}{0.000000,0.000000,0.000000}%
\pgfsetstrokecolor{currentstroke}%
\pgfsetdash{}{0pt}%
\pgfpathmoveto{\pgfqpoint{4.970132in}{2.396507in}}%
\pgfpathlineto{\pgfqpoint{4.984168in}{2.397307in}}%
\pgfpathlineto{\pgfqpoint{4.998215in}{2.398178in}}%
\pgfpathlineto{\pgfqpoint{5.012273in}{2.399118in}}%
\pgfpathlineto{\pgfqpoint{5.026340in}{2.400128in}}%
\pgfpathlineto{\pgfqpoint{5.033906in}{2.406604in}}%
\pgfpathlineto{\pgfqpoint{5.041467in}{2.413103in}}%
\pgfpathlineto{\pgfqpoint{5.049022in}{2.419629in}}%
\pgfpathlineto{\pgfqpoint{5.056572in}{2.426186in}}%
\pgfpathlineto{\pgfqpoint{5.042522in}{2.425425in}}%
\pgfpathlineto{\pgfqpoint{5.028482in}{2.424733in}}%
\pgfpathlineto{\pgfqpoint{5.014452in}{2.424111in}}%
\pgfpathlineto{\pgfqpoint{5.000432in}{2.423559in}}%
\pgfpathlineto{\pgfqpoint{4.992865in}{2.416746in}}%
\pgfpathlineto{\pgfqpoint{4.985293in}{2.409969in}}%
\pgfpathlineto{\pgfqpoint{4.977715in}{2.403224in}}%
\pgfpathlineto{\pgfqpoint{4.970132in}{2.396507in}}%
\pgfpathclose%
\pgfusepath{fill}%
\end{pgfscope}%
\begin{pgfscope}%
\pgfpathrectangle{\pgfqpoint{1.150000in}{0.150000in}}{\pgfqpoint{5.700000in}{5.700000in}}%
\pgfusepath{clip}%
\pgfsetbuttcap%
\pgfsetroundjoin%
\definecolor{currentfill}{rgb}{0.276022,0.044167,0.370164}%
\pgfsetfillcolor{currentfill}%
\pgfsetfillopacity{0.700000}%
\pgfsetlinewidth{0.000000pt}%
\definecolor{currentstroke}{rgb}{0.000000,0.000000,0.000000}%
\pgfsetstrokecolor{currentstroke}%
\pgfsetdash{}{0pt}%
\pgfpathmoveto{\pgfqpoint{3.535550in}{2.004864in}}%
\pgfpathlineto{\pgfqpoint{3.549174in}{2.001513in}}%
\pgfpathlineto{\pgfqpoint{3.562804in}{1.998246in}}%
\pgfpathlineto{\pgfqpoint{3.576440in}{1.995064in}}%
\pgfpathlineto{\pgfqpoint{3.590082in}{1.991964in}}%
\pgfpathlineto{\pgfqpoint{3.598199in}{2.000903in}}%
\pgfpathlineto{\pgfqpoint{3.606311in}{2.009839in}}%
\pgfpathlineto{\pgfqpoint{3.614416in}{2.018772in}}%
\pgfpathlineto{\pgfqpoint{3.622516in}{2.027702in}}%
\pgfpathlineto{\pgfqpoint{3.608886in}{2.030760in}}%
\pgfpathlineto{\pgfqpoint{3.595262in}{2.033901in}}%
\pgfpathlineto{\pgfqpoint{3.581643in}{2.037126in}}%
\pgfpathlineto{\pgfqpoint{3.568030in}{2.040436in}}%
\pgfpathlineto{\pgfqpoint{3.559919in}{2.031539in}}%
\pgfpathlineto{\pgfqpoint{3.551802in}{2.022645in}}%
\pgfpathlineto{\pgfqpoint{3.543678in}{2.013753in}}%
\pgfpathlineto{\pgfqpoint{3.535550in}{2.004864in}}%
\pgfpathclose%
\pgfusepath{fill}%
\end{pgfscope}%
\begin{pgfscope}%
\pgfpathrectangle{\pgfqpoint{1.150000in}{0.150000in}}{\pgfqpoint{5.700000in}{5.700000in}}%
\pgfusepath{clip}%
\pgfsetbuttcap%
\pgfsetroundjoin%
\definecolor{currentfill}{rgb}{0.283091,0.110553,0.431554}%
\pgfsetfillcolor{currentfill}%
\pgfsetfillopacity{0.700000}%
\pgfsetlinewidth{0.000000pt}%
\definecolor{currentstroke}{rgb}{0.000000,0.000000,0.000000}%
\pgfsetstrokecolor{currentstroke}%
\pgfsetdash{}{0pt}%
\pgfpathmoveto{\pgfqpoint{2.542572in}{2.145201in}}%
\pgfpathlineto{\pgfqpoint{2.556150in}{2.135460in}}%
\pgfpathlineto{\pgfqpoint{2.569729in}{2.125838in}}%
\pgfpathlineto{\pgfqpoint{2.583307in}{2.116333in}}%
\pgfpathlineto{\pgfqpoint{2.596884in}{2.106943in}}%
\pgfpathlineto{\pgfqpoint{2.605439in}{2.112822in}}%
\pgfpathlineto{\pgfqpoint{2.613983in}{2.118809in}}%
\pgfpathlineto{\pgfqpoint{2.622516in}{2.124904in}}%
\pgfpathlineto{\pgfqpoint{2.631040in}{2.131104in}}%
\pgfpathlineto{\pgfqpoint{2.617485in}{2.140305in}}%
\pgfpathlineto{\pgfqpoint{2.603930in}{2.149622in}}%
\pgfpathlineto{\pgfqpoint{2.590375in}{2.159057in}}%
\pgfpathlineto{\pgfqpoint{2.576820in}{2.168608in}}%
\pgfpathlineto{\pgfqpoint{2.568273in}{2.162589in}}%
\pgfpathlineto{\pgfqpoint{2.559717in}{2.156680in}}%
\pgfpathlineto{\pgfqpoint{2.551150in}{2.150883in}}%
\pgfpathlineto{\pgfqpoint{2.542572in}{2.145201in}}%
\pgfpathclose%
\pgfusepath{fill}%
\end{pgfscope}%
\begin{pgfscope}%
\pgfpathrectangle{\pgfqpoint{1.150000in}{0.150000in}}{\pgfqpoint{5.700000in}{5.700000in}}%
\pgfusepath{clip}%
\pgfsetbuttcap%
\pgfsetroundjoin%
\definecolor{currentfill}{rgb}{0.272594,0.025563,0.353093}%
\pgfsetfillcolor{currentfill}%
\pgfsetfillopacity{0.700000}%
\pgfsetlinewidth{0.000000pt}%
\definecolor{currentstroke}{rgb}{0.000000,0.000000,0.000000}%
\pgfsetstrokecolor{currentstroke}%
\pgfsetdash{}{0pt}%
\pgfpathmoveto{\pgfqpoint{3.306918in}{1.982533in}}%
\pgfpathlineto{\pgfqpoint{3.320507in}{1.977998in}}%
\pgfpathlineto{\pgfqpoint{3.334100in}{1.973552in}}%
\pgfpathlineto{\pgfqpoint{3.347698in}{1.969195in}}%
\pgfpathlineto{\pgfqpoint{3.361301in}{1.964925in}}%
\pgfpathlineto{\pgfqpoint{3.369503in}{1.973597in}}%
\pgfpathlineto{\pgfqpoint{3.377700in}{1.982286in}}%
\pgfpathlineto{\pgfqpoint{3.385890in}{1.990992in}}%
\pgfpathlineto{\pgfqpoint{3.394074in}{1.999714in}}%
\pgfpathlineto{\pgfqpoint{3.380484in}{2.003901in}}%
\pgfpathlineto{\pgfqpoint{3.366899in}{2.008176in}}%
\pgfpathlineto{\pgfqpoint{3.353319in}{2.012539in}}%
\pgfpathlineto{\pgfqpoint{3.339744in}{2.016992in}}%
\pgfpathlineto{\pgfqpoint{3.331547in}{2.008345in}}%
\pgfpathlineto{\pgfqpoint{3.323344in}{1.999719in}}%
\pgfpathlineto{\pgfqpoint{3.315134in}{1.991115in}}%
\pgfpathlineto{\pgfqpoint{3.306918in}{1.982533in}}%
\pgfpathclose%
\pgfusepath{fill}%
\end{pgfscope}%
\begin{pgfscope}%
\pgfpathrectangle{\pgfqpoint{1.150000in}{0.150000in}}{\pgfqpoint{5.700000in}{5.700000in}}%
\pgfusepath{clip}%
\pgfsetbuttcap%
\pgfsetroundjoin%
\definecolor{currentfill}{rgb}{0.279566,0.067836,0.391917}%
\pgfsetfillcolor{currentfill}%
\pgfsetfillopacity{0.700000}%
\pgfsetlinewidth{0.000000pt}%
\definecolor{currentstroke}{rgb}{0.000000,0.000000,0.000000}%
\pgfsetstrokecolor{currentstroke}%
\pgfsetdash{}{0pt}%
\pgfpathmoveto{\pgfqpoint{2.739500in}{2.061560in}}%
\pgfpathlineto{\pgfqpoint{2.753061in}{2.053363in}}%
\pgfpathlineto{\pgfqpoint{2.766624in}{2.045273in}}%
\pgfpathlineto{\pgfqpoint{2.780188in}{2.037291in}}%
\pgfpathlineto{\pgfqpoint{2.793753in}{2.029415in}}%
\pgfpathlineto{\pgfqpoint{2.802204in}{2.036239in}}%
\pgfpathlineto{\pgfqpoint{2.810647in}{2.043147in}}%
\pgfpathlineto{\pgfqpoint{2.819080in}{2.050137in}}%
\pgfpathlineto{\pgfqpoint{2.827505in}{2.057206in}}%
\pgfpathlineto{\pgfqpoint{2.813959in}{2.064917in}}%
\pgfpathlineto{\pgfqpoint{2.800415in}{2.072733in}}%
\pgfpathlineto{\pgfqpoint{2.786873in}{2.080656in}}%
\pgfpathlineto{\pgfqpoint{2.773331in}{2.088687in}}%
\pgfpathlineto{\pgfqpoint{2.764887in}{2.081776in}}%
\pgfpathlineto{\pgfqpoint{2.756434in}{2.074949in}}%
\pgfpathlineto{\pgfqpoint{2.747971in}{2.068210in}}%
\pgfpathlineto{\pgfqpoint{2.739500in}{2.061560in}}%
\pgfpathclose%
\pgfusepath{fill}%
\end{pgfscope}%
\begin{pgfscope}%
\pgfpathrectangle{\pgfqpoint{1.150000in}{0.150000in}}{\pgfqpoint{5.700000in}{5.700000in}}%
\pgfusepath{clip}%
\pgfsetbuttcap%
\pgfsetroundjoin%
\definecolor{currentfill}{rgb}{0.278012,0.180367,0.486697}%
\pgfsetfillcolor{currentfill}%
\pgfsetfillopacity{0.700000}%
\pgfsetlinewidth{0.000000pt}%
\definecolor{currentstroke}{rgb}{0.000000,0.000000,0.000000}%
\pgfsetstrokecolor{currentstroke}%
\pgfsetdash{}{0pt}%
\pgfpathmoveto{\pgfqpoint{2.290342in}{2.299728in}}%
\pgfpathlineto{\pgfqpoint{2.303970in}{2.287752in}}%
\pgfpathlineto{\pgfqpoint{2.317595in}{2.275911in}}%
\pgfpathlineto{\pgfqpoint{2.331217in}{2.264204in}}%
\pgfpathlineto{\pgfqpoint{2.344837in}{2.252629in}}%
\pgfpathlineto{\pgfqpoint{2.353538in}{2.257169in}}%
\pgfpathlineto{\pgfqpoint{2.362226in}{2.261851in}}%
\pgfpathlineto{\pgfqpoint{2.370902in}{2.266671in}}%
\pgfpathlineto{\pgfqpoint{2.379566in}{2.271628in}}%
\pgfpathlineto{\pgfqpoint{2.365972in}{2.282991in}}%
\pgfpathlineto{\pgfqpoint{2.352377in}{2.294486in}}%
\pgfpathlineto{\pgfqpoint{2.338779in}{2.306115in}}%
\pgfpathlineto{\pgfqpoint{2.325179in}{2.317878in}}%
\pgfpathlineto{\pgfqpoint{2.316489in}{2.313125in}}%
\pgfpathlineto{\pgfqpoint{2.307786in}{2.308514in}}%
\pgfpathlineto{\pgfqpoint{2.299071in}{2.304048in}}%
\pgfpathlineto{\pgfqpoint{2.290342in}{2.299728in}}%
\pgfpathclose%
\pgfusepath{fill}%
\end{pgfscope}%
\begin{pgfscope}%
\pgfpathrectangle{\pgfqpoint{1.150000in}{0.150000in}}{\pgfqpoint{5.700000in}{5.700000in}}%
\pgfusepath{clip}%
\pgfsetbuttcap%
\pgfsetroundjoin%
\definecolor{currentfill}{rgb}{0.225863,0.330805,0.547314}%
\pgfsetfillcolor{currentfill}%
\pgfsetfillopacity{0.700000}%
\pgfsetlinewidth{0.000000pt}%
\definecolor{currentstroke}{rgb}{0.000000,0.000000,0.000000}%
\pgfsetstrokecolor{currentstroke}%
\pgfsetdash{}{0pt}%
\pgfpathmoveto{\pgfqpoint{5.687714in}{2.605360in}}%
\pgfpathlineto{\pgfqpoint{5.701987in}{2.606436in}}%
\pgfpathlineto{\pgfqpoint{5.716271in}{2.607580in}}%
\pgfpathlineto{\pgfqpoint{5.730567in}{2.608790in}}%
\pgfpathlineto{\pgfqpoint{5.744874in}{2.610066in}}%
\pgfpathlineto{\pgfqpoint{5.752131in}{2.615692in}}%
\pgfpathlineto{\pgfqpoint{5.759385in}{2.621454in}}%
\pgfpathlineto{\pgfqpoint{5.766639in}{2.627359in}}%
\pgfpathlineto{\pgfqpoint{5.773892in}{2.633414in}}%
\pgfpathlineto{\pgfqpoint{5.759611in}{2.632531in}}%
\pgfpathlineto{\pgfqpoint{5.745342in}{2.631715in}}%
\pgfpathlineto{\pgfqpoint{5.731084in}{2.630965in}}%
\pgfpathlineto{\pgfqpoint{5.716836in}{2.630282in}}%
\pgfpathlineto{\pgfqpoint{5.709557in}{2.623826in}}%
\pgfpathlineto{\pgfqpoint{5.702277in}{2.617526in}}%
\pgfpathlineto{\pgfqpoint{5.694996in}{2.611372in}}%
\pgfpathlineto{\pgfqpoint{5.687714in}{2.605360in}}%
\pgfpathclose%
\pgfusepath{fill}%
\end{pgfscope}%
\begin{pgfscope}%
\pgfpathrectangle{\pgfqpoint{1.150000in}{0.150000in}}{\pgfqpoint{5.700000in}{5.700000in}}%
\pgfusepath{clip}%
\pgfsetbuttcap%
\pgfsetroundjoin%
\definecolor{currentfill}{rgb}{0.279566,0.067836,0.391917}%
\pgfsetfillcolor{currentfill}%
\pgfsetfillopacity{0.700000}%
\pgfsetlinewidth{0.000000pt}%
\definecolor{currentstroke}{rgb}{0.000000,0.000000,0.000000}%
\pgfsetstrokecolor{currentstroke}%
\pgfsetdash{}{0pt}%
\pgfpathmoveto{\pgfqpoint{3.764030in}{2.042064in}}%
\pgfpathlineto{\pgfqpoint{3.777705in}{2.039764in}}%
\pgfpathlineto{\pgfqpoint{3.791387in}{2.037545in}}%
\pgfpathlineto{\pgfqpoint{3.805076in}{2.035406in}}%
\pgfpathlineto{\pgfqpoint{3.818771in}{2.033347in}}%
\pgfpathlineto{\pgfqpoint{3.826809in}{2.042287in}}%
\pgfpathlineto{\pgfqpoint{3.834841in}{2.051210in}}%
\pgfpathlineto{\pgfqpoint{3.842867in}{2.060115in}}%
\pgfpathlineto{\pgfqpoint{3.850888in}{2.069005in}}%
\pgfpathlineto{\pgfqpoint{3.837203in}{2.071064in}}%
\pgfpathlineto{\pgfqpoint{3.823525in}{2.073202in}}%
\pgfpathlineto{\pgfqpoint{3.809854in}{2.075421in}}%
\pgfpathlineto{\pgfqpoint{3.796190in}{2.077721in}}%
\pgfpathlineto{\pgfqpoint{3.788158in}{2.068824in}}%
\pgfpathlineto{\pgfqpoint{3.780121in}{2.059916in}}%
\pgfpathlineto{\pgfqpoint{3.772078in}{2.050996in}}%
\pgfpathlineto{\pgfqpoint{3.764030in}{2.042064in}}%
\pgfpathclose%
\pgfusepath{fill}%
\end{pgfscope}%
\begin{pgfscope}%
\pgfpathrectangle{\pgfqpoint{1.150000in}{0.150000in}}{\pgfqpoint{5.700000in}{5.700000in}}%
\pgfusepath{clip}%
\pgfsetbuttcap%
\pgfsetroundjoin%
\definecolor{currentfill}{rgb}{0.279574,0.170599,0.479997}%
\pgfsetfillcolor{currentfill}%
\pgfsetfillopacity{0.700000}%
\pgfsetlinewidth{0.000000pt}%
\definecolor{currentstroke}{rgb}{0.000000,0.000000,0.000000}%
\pgfsetstrokecolor{currentstroke}%
\pgfsetdash{}{0pt}%
\pgfpathmoveto{\pgfqpoint{4.481477in}{2.241393in}}%
\pgfpathlineto{\pgfqpoint{4.495357in}{2.241412in}}%
\pgfpathlineto{\pgfqpoint{4.509246in}{2.241505in}}%
\pgfpathlineto{\pgfqpoint{4.523144in}{2.241670in}}%
\pgfpathlineto{\pgfqpoint{4.537052in}{2.241908in}}%
\pgfpathlineto{\pgfqpoint{4.544825in}{2.249581in}}%
\pgfpathlineto{\pgfqpoint{4.552592in}{2.257234in}}%
\pgfpathlineto{\pgfqpoint{4.560353in}{2.264871in}}%
\pgfpathlineto{\pgfqpoint{4.568109in}{2.272494in}}%
\pgfpathlineto{\pgfqpoint{4.554215in}{2.272401in}}%
\pgfpathlineto{\pgfqpoint{4.540329in}{2.272380in}}%
\pgfpathlineto{\pgfqpoint{4.526453in}{2.272432in}}%
\pgfpathlineto{\pgfqpoint{4.512586in}{2.272557in}}%
\pgfpathlineto{\pgfqpoint{4.504818in}{2.264782in}}%
\pgfpathlineto{\pgfqpoint{4.497043in}{2.256998in}}%
\pgfpathlineto{\pgfqpoint{4.489263in}{2.249202in}}%
\pgfpathlineto{\pgfqpoint{4.481477in}{2.241393in}}%
\pgfpathclose%
\pgfusepath{fill}%
\end{pgfscope}%
\begin{pgfscope}%
\pgfpathrectangle{\pgfqpoint{1.150000in}{0.150000in}}{\pgfqpoint{5.700000in}{5.700000in}}%
\pgfusepath{clip}%
\pgfsetbuttcap%
\pgfsetroundjoin%
\definecolor{currentfill}{rgb}{0.246811,0.283237,0.535941}%
\pgfsetfillcolor{currentfill}%
\pgfsetfillopacity{0.700000}%
\pgfsetlinewidth{0.000000pt}%
\definecolor{currentstroke}{rgb}{0.000000,0.000000,0.000000}%
\pgfsetstrokecolor{currentstroke}%
\pgfsetdash{}{0pt}%
\pgfpathmoveto{\pgfqpoint{5.285747in}{2.488165in}}%
\pgfpathlineto{\pgfqpoint{5.299894in}{2.489262in}}%
\pgfpathlineto{\pgfqpoint{5.314052in}{2.490427in}}%
\pgfpathlineto{\pgfqpoint{5.328221in}{2.491660in}}%
\pgfpathlineto{\pgfqpoint{5.342400in}{2.492962in}}%
\pgfpathlineto{\pgfqpoint{5.349828in}{2.498811in}}%
\pgfpathlineto{\pgfqpoint{5.357252in}{2.504722in}}%
\pgfpathlineto{\pgfqpoint{5.364671in}{2.510701in}}%
\pgfpathlineto{\pgfqpoint{5.372086in}{2.516753in}}%
\pgfpathlineto{\pgfqpoint{5.357927in}{2.515763in}}%
\pgfpathlineto{\pgfqpoint{5.343780in}{2.514841in}}%
\pgfpathlineto{\pgfqpoint{5.329643in}{2.513987in}}%
\pgfpathlineto{\pgfqpoint{5.315516in}{2.513202in}}%
\pgfpathlineto{\pgfqpoint{5.308080in}{2.506831in}}%
\pgfpathlineto{\pgfqpoint{5.300640in}{2.500538in}}%
\pgfpathlineto{\pgfqpoint{5.293196in}{2.494318in}}%
\pgfpathlineto{\pgfqpoint{5.285747in}{2.488165in}}%
\pgfpathclose%
\pgfusepath{fill}%
\end{pgfscope}%
\begin{pgfscope}%
\pgfpathrectangle{\pgfqpoint{1.150000in}{0.150000in}}{\pgfqpoint{5.700000in}{5.700000in}}%
\pgfusepath{clip}%
\pgfsetbuttcap%
\pgfsetroundjoin%
\definecolor{currentfill}{rgb}{0.283091,0.110553,0.431554}%
\pgfsetfillcolor{currentfill}%
\pgfsetfillopacity{0.700000}%
\pgfsetlinewidth{0.000000pt}%
\definecolor{currentstroke}{rgb}{0.000000,0.000000,0.000000}%
\pgfsetstrokecolor{currentstroke}%
\pgfsetdash{}{0pt}%
\pgfpathmoveto{\pgfqpoint{4.079363in}{2.120319in}}%
\pgfpathlineto{\pgfqpoint{4.093122in}{2.119214in}}%
\pgfpathlineto{\pgfqpoint{4.106888in}{2.118187in}}%
\pgfpathlineto{\pgfqpoint{4.120662in}{2.117235in}}%
\pgfpathlineto{\pgfqpoint{4.134443in}{2.116360in}}%
\pgfpathlineto{\pgfqpoint{4.142370in}{2.124925in}}%
\pgfpathlineto{\pgfqpoint{4.150290in}{2.133462in}}%
\pgfpathlineto{\pgfqpoint{4.158204in}{2.141975in}}%
\pgfpathlineto{\pgfqpoint{4.166113in}{2.150463in}}%
\pgfpathlineto{\pgfqpoint{4.152342in}{2.151400in}}%
\pgfpathlineto{\pgfqpoint{4.138579in}{2.152413in}}%
\pgfpathlineto{\pgfqpoint{4.124824in}{2.153502in}}%
\pgfpathlineto{\pgfqpoint{4.111077in}{2.154668in}}%
\pgfpathlineto{\pgfqpoint{4.103157in}{2.146110in}}%
\pgfpathlineto{\pgfqpoint{4.095232in}{2.137534in}}%
\pgfpathlineto{\pgfqpoint{4.087300in}{2.128937in}}%
\pgfpathlineto{\pgfqpoint{4.079363in}{2.120319in}}%
\pgfpathclose%
\pgfusepath{fill}%
\end{pgfscope}%
\begin{pgfscope}%
\pgfpathrectangle{\pgfqpoint{1.150000in}{0.150000in}}{\pgfqpoint{5.700000in}{5.700000in}}%
\pgfusepath{clip}%
\pgfsetbuttcap%
\pgfsetroundjoin%
\definecolor{currentfill}{rgb}{0.266580,0.228262,0.514349}%
\pgfsetfillcolor{currentfill}%
\pgfsetfillopacity{0.700000}%
\pgfsetlinewidth{0.000000pt}%
\definecolor{currentstroke}{rgb}{0.000000,0.000000,0.000000}%
\pgfsetstrokecolor{currentstroke}%
\pgfsetdash{}{0pt}%
\pgfpathmoveto{\pgfqpoint{4.883630in}{2.366411in}}%
\pgfpathlineto{\pgfqpoint{4.897644in}{2.367159in}}%
\pgfpathlineto{\pgfqpoint{4.911667in}{2.367978in}}%
\pgfpathlineto{\pgfqpoint{4.925700in}{2.368867in}}%
\pgfpathlineto{\pgfqpoint{4.939743in}{2.369826in}}%
\pgfpathlineto{\pgfqpoint{4.947349in}{2.376476in}}%
\pgfpathlineto{\pgfqpoint{4.954949in}{2.383137in}}%
\pgfpathlineto{\pgfqpoint{4.962543in}{2.389813in}}%
\pgfpathlineto{\pgfqpoint{4.970132in}{2.396507in}}%
\pgfpathlineto{\pgfqpoint{4.956105in}{2.395776in}}%
\pgfpathlineto{\pgfqpoint{4.942088in}{2.395115in}}%
\pgfpathlineto{\pgfqpoint{4.928081in}{2.394525in}}%
\pgfpathlineto{\pgfqpoint{4.914084in}{2.394004in}}%
\pgfpathlineto{\pgfqpoint{4.906479in}{2.387075in}}%
\pgfpathlineto{\pgfqpoint{4.898868in}{2.380169in}}%
\pgfpathlineto{\pgfqpoint{4.891252in}{2.373282in}}%
\pgfpathlineto{\pgfqpoint{4.883630in}{2.366411in}}%
\pgfpathclose%
\pgfusepath{fill}%
\end{pgfscope}%
\begin{pgfscope}%
\pgfpathrectangle{\pgfqpoint{1.150000in}{0.150000in}}{\pgfqpoint{5.700000in}{5.700000in}}%
\pgfusepath{clip}%
\pgfsetbuttcap%
\pgfsetroundjoin%
\definecolor{currentfill}{rgb}{0.280868,0.160771,0.472899}%
\pgfsetfillcolor{currentfill}%
\pgfsetfillopacity{0.700000}%
\pgfsetlinewidth{0.000000pt}%
\definecolor{currentstroke}{rgb}{0.000000,0.000000,0.000000}%
\pgfsetstrokecolor{currentstroke}%
\pgfsetdash{}{0pt}%
\pgfpathmoveto{\pgfqpoint{2.344837in}{2.252629in}}%
\pgfpathlineto{\pgfqpoint{2.358455in}{2.241186in}}%
\pgfpathlineto{\pgfqpoint{2.372072in}{2.229873in}}%
\pgfpathlineto{\pgfqpoint{2.385686in}{2.218690in}}%
\pgfpathlineto{\pgfqpoint{2.399298in}{2.207634in}}%
\pgfpathlineto{\pgfqpoint{2.407972in}{2.212393in}}%
\pgfpathlineto{\pgfqpoint{2.416633in}{2.217289in}}%
\pgfpathlineto{\pgfqpoint{2.425283in}{2.222318in}}%
\pgfpathlineto{\pgfqpoint{2.433921in}{2.227478in}}%
\pgfpathlineto{\pgfqpoint{2.420334in}{2.238323in}}%
\pgfpathlineto{\pgfqpoint{2.406747in}{2.249295in}}%
\pgfpathlineto{\pgfqpoint{2.393157in}{2.260397in}}%
\pgfpathlineto{\pgfqpoint{2.379566in}{2.271628in}}%
\pgfpathlineto{\pgfqpoint{2.370902in}{2.266671in}}%
\pgfpathlineto{\pgfqpoint{2.362226in}{2.261851in}}%
\pgfpathlineto{\pgfqpoint{2.353538in}{2.257169in}}%
\pgfpathlineto{\pgfqpoint{2.344837in}{2.252629in}}%
\pgfpathclose%
\pgfusepath{fill}%
\end{pgfscope}%
\begin{pgfscope}%
\pgfpathrectangle{\pgfqpoint{1.150000in}{0.150000in}}{\pgfqpoint{5.700000in}{5.700000in}}%
\pgfusepath{clip}%
\pgfsetbuttcap%
\pgfsetroundjoin%
\definecolor{currentfill}{rgb}{0.273809,0.031497,0.358853}%
\pgfsetfillcolor{currentfill}%
\pgfsetfillopacity{0.700000}%
\pgfsetlinewidth{0.000000pt}%
\definecolor{currentstroke}{rgb}{0.000000,0.000000,0.000000}%
\pgfsetstrokecolor{currentstroke}%
\pgfsetdash{}{0pt}%
\pgfpathmoveto{\pgfqpoint{3.448482in}{1.983837in}}%
\pgfpathlineto{\pgfqpoint{3.462097in}{1.980084in}}%
\pgfpathlineto{\pgfqpoint{3.475718in}{1.976417in}}%
\pgfpathlineto{\pgfqpoint{3.489343in}{1.972835in}}%
\pgfpathlineto{\pgfqpoint{3.502975in}{1.969338in}}%
\pgfpathlineto{\pgfqpoint{3.511127in}{1.978214in}}%
\pgfpathlineto{\pgfqpoint{3.519274in}{1.987093in}}%
\pgfpathlineto{\pgfqpoint{3.527415in}{1.995977in}}%
\pgfpathlineto{\pgfqpoint{3.535550in}{2.004864in}}%
\pgfpathlineto{\pgfqpoint{3.521931in}{2.008299in}}%
\pgfpathlineto{\pgfqpoint{3.508317in}{2.011819in}}%
\pgfpathlineto{\pgfqpoint{3.494709in}{2.015425in}}%
\pgfpathlineto{\pgfqpoint{3.481106in}{2.019116in}}%
\pgfpathlineto{\pgfqpoint{3.472959in}{2.010283in}}%
\pgfpathlineto{\pgfqpoint{3.464806in}{2.001459in}}%
\pgfpathlineto{\pgfqpoint{3.456647in}{1.992644in}}%
\pgfpathlineto{\pgfqpoint{3.448482in}{1.983837in}}%
\pgfpathclose%
\pgfusepath{fill}%
\end{pgfscope}%
\begin{pgfscope}%
\pgfpathrectangle{\pgfqpoint{1.150000in}{0.150000in}}{\pgfqpoint{5.700000in}{5.700000in}}%
\pgfusepath{clip}%
\pgfsetbuttcap%
\pgfsetroundjoin%
\definecolor{currentfill}{rgb}{0.272594,0.025563,0.353093}%
\pgfsetfillcolor{currentfill}%
\pgfsetfillopacity{0.700000}%
\pgfsetlinewidth{0.000000pt}%
\definecolor{currentstroke}{rgb}{0.000000,0.000000,0.000000}%
\pgfsetstrokecolor{currentstroke}%
\pgfsetdash{}{0pt}%
\pgfpathmoveto{\pgfqpoint{3.077863in}{1.979059in}}%
\pgfpathlineto{\pgfqpoint{3.091434in}{1.973195in}}%
\pgfpathlineto{\pgfqpoint{3.105008in}{1.967425in}}%
\pgfpathlineto{\pgfqpoint{3.118586in}{1.961749in}}%
\pgfpathlineto{\pgfqpoint{3.132167in}{1.956168in}}%
\pgfpathlineto{\pgfqpoint{3.140466in}{1.964265in}}%
\pgfpathlineto{\pgfqpoint{3.148758in}{1.972405in}}%
\pgfpathlineto{\pgfqpoint{3.157043in}{1.980586in}}%
\pgfpathlineto{\pgfqpoint{3.165321in}{1.988808in}}%
\pgfpathlineto{\pgfqpoint{3.151755in}{1.994266in}}%
\pgfpathlineto{\pgfqpoint{3.138193in}{1.999818in}}%
\pgfpathlineto{\pgfqpoint{3.124635in}{2.005464in}}%
\pgfpathlineto{\pgfqpoint{3.111080in}{2.011205in}}%
\pgfpathlineto{\pgfqpoint{3.102786in}{2.003100in}}%
\pgfpathlineto{\pgfqpoint{3.094486in}{1.995039in}}%
\pgfpathlineto{\pgfqpoint{3.086178in}{1.987025in}}%
\pgfpathlineto{\pgfqpoint{3.077863in}{1.979059in}}%
\pgfpathclose%
\pgfusepath{fill}%
\end{pgfscope}%
\begin{pgfscope}%
\pgfpathrectangle{\pgfqpoint{1.150000in}{0.150000in}}{\pgfqpoint{5.700000in}{5.700000in}}%
\pgfusepath{clip}%
\pgfsetbuttcap%
\pgfsetroundjoin%
\definecolor{currentfill}{rgb}{0.229739,0.322361,0.545706}%
\pgfsetfillcolor{currentfill}%
\pgfsetfillopacity{0.700000}%
\pgfsetlinewidth{0.000000pt}%
\definecolor{currentstroke}{rgb}{0.000000,0.000000,0.000000}%
\pgfsetstrokecolor{currentstroke}%
\pgfsetdash{}{0pt}%
\pgfpathmoveto{\pgfqpoint{5.601480in}{2.577451in}}%
\pgfpathlineto{\pgfqpoint{5.615734in}{2.578634in}}%
\pgfpathlineto{\pgfqpoint{5.629999in}{2.579883in}}%
\pgfpathlineto{\pgfqpoint{5.644276in}{2.581200in}}%
\pgfpathlineto{\pgfqpoint{5.658563in}{2.582584in}}%
\pgfpathlineto{\pgfqpoint{5.665854in}{2.588100in}}%
\pgfpathlineto{\pgfqpoint{5.673143in}{2.593730in}}%
\pgfpathlineto{\pgfqpoint{5.680429in}{2.599481in}}%
\pgfpathlineto{\pgfqpoint{5.687714in}{2.605360in}}%
\pgfpathlineto{\pgfqpoint{5.673451in}{2.604350in}}%
\pgfpathlineto{\pgfqpoint{5.659200in}{2.603407in}}%
\pgfpathlineto{\pgfqpoint{5.644960in}{2.602531in}}%
\pgfpathlineto{\pgfqpoint{5.630731in}{2.601721in}}%
\pgfpathlineto{\pgfqpoint{5.623421in}{2.595462in}}%
\pgfpathlineto{\pgfqpoint{5.616110in}{2.589335in}}%
\pgfpathlineto{\pgfqpoint{5.608796in}{2.583333in}}%
\pgfpathlineto{\pgfqpoint{5.601480in}{2.577451in}}%
\pgfpathclose%
\pgfusepath{fill}%
\end{pgfscope}%
\begin{pgfscope}%
\pgfpathrectangle{\pgfqpoint{1.150000in}{0.150000in}}{\pgfqpoint{5.700000in}{5.700000in}}%
\pgfusepath{clip}%
\pgfsetbuttcap%
\pgfsetroundjoin%
\definecolor{currentfill}{rgb}{0.274952,0.037752,0.364543}%
\pgfsetfillcolor{currentfill}%
\pgfsetfillopacity{0.700000}%
\pgfsetlinewidth{0.000000pt}%
\definecolor{currentstroke}{rgb}{0.000000,0.000000,0.000000}%
\pgfsetstrokecolor{currentstroke}%
\pgfsetdash{}{0pt}%
\pgfpathmoveto{\pgfqpoint{2.935939in}{1.999257in}}%
\pgfpathlineto{\pgfqpoint{2.949503in}{1.992470in}}%
\pgfpathlineto{\pgfqpoint{2.963070in}{1.985782in}}%
\pgfpathlineto{\pgfqpoint{2.976640in}{1.979194in}}%
\pgfpathlineto{\pgfqpoint{2.990212in}{1.972703in}}%
\pgfpathlineto{\pgfqpoint{2.998574in}{1.980307in}}%
\pgfpathlineto{\pgfqpoint{3.006928in}{1.987972in}}%
\pgfpathlineto{\pgfqpoint{3.015274in}{1.995695in}}%
\pgfpathlineto{\pgfqpoint{3.023613in}{2.003475in}}%
\pgfpathlineto{\pgfqpoint{3.010058in}{2.009822in}}%
\pgfpathlineto{\pgfqpoint{2.996505in}{2.016266in}}%
\pgfpathlineto{\pgfqpoint{2.982956in}{2.022809in}}%
\pgfpathlineto{\pgfqpoint{2.969409in}{2.029452in}}%
\pgfpathlineto{\pgfqpoint{2.961053in}{2.021809in}}%
\pgfpathlineto{\pgfqpoint{2.952690in}{2.014227in}}%
\pgfpathlineto{\pgfqpoint{2.944318in}{2.006709in}}%
\pgfpathlineto{\pgfqpoint{2.935939in}{1.999257in}}%
\pgfpathclose%
\pgfusepath{fill}%
\end{pgfscope}%
\begin{pgfscope}%
\pgfpathrectangle{\pgfqpoint{1.150000in}{0.150000in}}{\pgfqpoint{5.700000in}{5.700000in}}%
\pgfusepath{clip}%
\pgfsetbuttcap%
\pgfsetroundjoin%
\definecolor{currentfill}{rgb}{0.281412,0.155834,0.469201}%
\pgfsetfillcolor{currentfill}%
\pgfsetfillopacity{0.700000}%
\pgfsetlinewidth{0.000000pt}%
\definecolor{currentstroke}{rgb}{0.000000,0.000000,0.000000}%
\pgfsetstrokecolor{currentstroke}%
\pgfsetdash{}{0pt}%
\pgfpathmoveto{\pgfqpoint{4.394794in}{2.210120in}}%
\pgfpathlineto{\pgfqpoint{4.408651in}{2.209970in}}%
\pgfpathlineto{\pgfqpoint{4.422517in}{2.209894in}}%
\pgfpathlineto{\pgfqpoint{4.436391in}{2.209891in}}%
\pgfpathlineto{\pgfqpoint{4.450274in}{2.209962in}}%
\pgfpathlineto{\pgfqpoint{4.458084in}{2.217853in}}%
\pgfpathlineto{\pgfqpoint{4.465888in}{2.225721in}}%
\pgfpathlineto{\pgfqpoint{4.473686in}{2.233567in}}%
\pgfpathlineto{\pgfqpoint{4.481477in}{2.241393in}}%
\pgfpathlineto{\pgfqpoint{4.467606in}{2.241446in}}%
\pgfpathlineto{\pgfqpoint{4.453744in}{2.241573in}}%
\pgfpathlineto{\pgfqpoint{4.439891in}{2.241773in}}%
\pgfpathlineto{\pgfqpoint{4.426046in}{2.242047in}}%
\pgfpathlineto{\pgfqpoint{4.418242in}{2.234089in}}%
\pgfpathlineto{\pgfqpoint{4.410432in}{2.226117in}}%
\pgfpathlineto{\pgfqpoint{4.402616in}{2.218128in}}%
\pgfpathlineto{\pgfqpoint{4.394794in}{2.210120in}}%
\pgfpathclose%
\pgfusepath{fill}%
\end{pgfscope}%
\begin{pgfscope}%
\pgfpathrectangle{\pgfqpoint{1.150000in}{0.150000in}}{\pgfqpoint{5.700000in}{5.700000in}}%
\pgfusepath{clip}%
\pgfsetbuttcap%
\pgfsetroundjoin%
\definecolor{currentfill}{rgb}{0.282327,0.094955,0.417331}%
\pgfsetfillcolor{currentfill}%
\pgfsetfillopacity{0.700000}%
\pgfsetlinewidth{0.000000pt}%
\definecolor{currentstroke}{rgb}{0.000000,0.000000,0.000000}%
\pgfsetstrokecolor{currentstroke}%
\pgfsetdash{}{0pt}%
\pgfpathmoveto{\pgfqpoint{2.596884in}{2.106943in}}%
\pgfpathlineto{\pgfqpoint{2.610462in}{2.097669in}}%
\pgfpathlineto{\pgfqpoint{2.624041in}{2.088509in}}%
\pgfpathlineto{\pgfqpoint{2.637619in}{2.079462in}}%
\pgfpathlineto{\pgfqpoint{2.651198in}{2.070528in}}%
\pgfpathlineto{\pgfqpoint{2.659729in}{2.076602in}}%
\pgfpathlineto{\pgfqpoint{2.668251in}{2.082780in}}%
\pgfpathlineto{\pgfqpoint{2.676763in}{2.089060in}}%
\pgfpathlineto{\pgfqpoint{2.685264in}{2.095439in}}%
\pgfpathlineto{\pgfqpoint{2.671708in}{2.104186in}}%
\pgfpathlineto{\pgfqpoint{2.658151in}{2.113045in}}%
\pgfpathlineto{\pgfqpoint{2.644596in}{2.122017in}}%
\pgfpathlineto{\pgfqpoint{2.631040in}{2.131104in}}%
\pgfpathlineto{\pgfqpoint{2.622516in}{2.124904in}}%
\pgfpathlineto{\pgfqpoint{2.613983in}{2.118809in}}%
\pgfpathlineto{\pgfqpoint{2.605439in}{2.112822in}}%
\pgfpathlineto{\pgfqpoint{2.596884in}{2.106943in}}%
\pgfpathclose%
\pgfusepath{fill}%
\end{pgfscope}%
\begin{pgfscope}%
\pgfpathrectangle{\pgfqpoint{1.150000in}{0.150000in}}{\pgfqpoint{5.700000in}{5.700000in}}%
\pgfusepath{clip}%
\pgfsetbuttcap%
\pgfsetroundjoin%
\definecolor{currentfill}{rgb}{0.277941,0.056324,0.381191}%
\pgfsetfillcolor{currentfill}%
\pgfsetfillopacity{0.700000}%
\pgfsetlinewidth{0.000000pt}%
\definecolor{currentstroke}{rgb}{0.000000,0.000000,0.000000}%
\pgfsetstrokecolor{currentstroke}%
\pgfsetdash{}{0pt}%
\pgfpathmoveto{\pgfqpoint{3.677097in}{2.016299in}}%
\pgfpathlineto{\pgfqpoint{3.690758in}{2.013654in}}%
\pgfpathlineto{\pgfqpoint{3.704425in}{2.011090in}}%
\pgfpathlineto{\pgfqpoint{3.718099in}{2.008608in}}%
\pgfpathlineto{\pgfqpoint{3.731778in}{2.006207in}}%
\pgfpathlineto{\pgfqpoint{3.739850in}{2.015191in}}%
\pgfpathlineto{\pgfqpoint{3.747915in}{2.024161in}}%
\pgfpathlineto{\pgfqpoint{3.755975in}{2.033119in}}%
\pgfpathlineto{\pgfqpoint{3.764030in}{2.042064in}}%
\pgfpathlineto{\pgfqpoint{3.750361in}{2.044444in}}%
\pgfpathlineto{\pgfqpoint{3.736698in}{2.046906in}}%
\pgfpathlineto{\pgfqpoint{3.723043in}{2.049449in}}%
\pgfpathlineto{\pgfqpoint{3.709393in}{2.052073in}}%
\pgfpathlineto{\pgfqpoint{3.701328in}{2.043141in}}%
\pgfpathlineto{\pgfqpoint{3.693257in}{2.034202in}}%
\pgfpathlineto{\pgfqpoint{3.685180in}{2.025254in}}%
\pgfpathlineto{\pgfqpoint{3.677097in}{2.016299in}}%
\pgfpathclose%
\pgfusepath{fill}%
\end{pgfscope}%
\begin{pgfscope}%
\pgfpathrectangle{\pgfqpoint{1.150000in}{0.150000in}}{\pgfqpoint{5.700000in}{5.700000in}}%
\pgfusepath{clip}%
\pgfsetbuttcap%
\pgfsetroundjoin%
\definecolor{currentfill}{rgb}{0.212395,0.359683,0.551710}%
\pgfsetfillcolor{currentfill}%
\pgfsetfillopacity{0.700000}%
\pgfsetlinewidth{0.000000pt}%
\definecolor{currentstroke}{rgb}{0.000000,0.000000,0.000000}%
\pgfsetstrokecolor{currentstroke}%
\pgfsetdash{}{0pt}%
\pgfpathmoveto{\pgfqpoint{5.917324in}{2.665404in}}%
\pgfpathlineto{\pgfqpoint{5.931678in}{2.666467in}}%
\pgfpathlineto{\pgfqpoint{5.946044in}{2.667596in}}%
\pgfpathlineto{\pgfqpoint{5.960421in}{2.668791in}}%
\pgfpathlineto{\pgfqpoint{5.967595in}{2.674529in}}%
\pgfpathlineto{\pgfqpoint{5.974769in}{2.680447in}}%
\pgfpathlineto{\pgfqpoint{5.981945in}{2.686552in}}%
\pgfpathlineto{\pgfqpoint{5.989122in}{2.692853in}}%
\pgfpathlineto{\pgfqpoint{5.974774in}{2.692093in}}%
\pgfpathlineto{\pgfqpoint{5.960438in}{2.691399in}}%
\pgfpathlineto{\pgfqpoint{5.946114in}{2.690770in}}%
\pgfpathlineto{\pgfqpoint{5.938915in}{2.684138in}}%
\pgfpathlineto{\pgfqpoint{5.931717in}{2.677704in}}%
\pgfpathlineto{\pgfqpoint{5.924520in}{2.671462in}}%
\pgfpathlineto{\pgfqpoint{5.917324in}{2.665404in}}%
\pgfpathclose%
\pgfusepath{fill}%
\end{pgfscope}%
\begin{pgfscope}%
\pgfpathrectangle{\pgfqpoint{1.150000in}{0.150000in}}{\pgfqpoint{5.700000in}{5.700000in}}%
\pgfusepath{clip}%
\pgfsetbuttcap%
\pgfsetroundjoin%
\definecolor{currentfill}{rgb}{0.282327,0.094955,0.417331}%
\pgfsetfillcolor{currentfill}%
\pgfsetfillopacity{0.700000}%
\pgfsetlinewidth{0.000000pt}%
\definecolor{currentstroke}{rgb}{0.000000,0.000000,0.000000}%
\pgfsetstrokecolor{currentstroke}%
\pgfsetdash{}{0pt}%
\pgfpathmoveto{\pgfqpoint{3.992559in}{2.090623in}}%
\pgfpathlineto{\pgfqpoint{4.006297in}{2.089251in}}%
\pgfpathlineto{\pgfqpoint{4.020043in}{2.087957in}}%
\pgfpathlineto{\pgfqpoint{4.033797in}{2.086740in}}%
\pgfpathlineto{\pgfqpoint{4.047558in}{2.085600in}}%
\pgfpathlineto{\pgfqpoint{4.055518in}{2.094319in}}%
\pgfpathlineto{\pgfqpoint{4.063472in}{2.103011in}}%
\pgfpathlineto{\pgfqpoint{4.071421in}{2.111677in}}%
\pgfpathlineto{\pgfqpoint{4.079363in}{2.120319in}}%
\pgfpathlineto{\pgfqpoint{4.065613in}{2.121500in}}%
\pgfpathlineto{\pgfqpoint{4.051870in}{2.122758in}}%
\pgfpathlineto{\pgfqpoint{4.038135in}{2.124093in}}%
\pgfpathlineto{\pgfqpoint{4.024407in}{2.125506in}}%
\pgfpathlineto{\pgfqpoint{4.016454in}{2.116815in}}%
\pgfpathlineto{\pgfqpoint{4.008494in}{2.108106in}}%
\pgfpathlineto{\pgfqpoint{4.000529in}{2.099375in}}%
\pgfpathlineto{\pgfqpoint{3.992559in}{2.090623in}}%
\pgfpathclose%
\pgfusepath{fill}%
\end{pgfscope}%
\begin{pgfscope}%
\pgfpathrectangle{\pgfqpoint{1.150000in}{0.150000in}}{\pgfqpoint{5.700000in}{5.700000in}}%
\pgfusepath{clip}%
\pgfsetbuttcap%
\pgfsetroundjoin%
\definecolor{currentfill}{rgb}{0.269308,0.218818,0.509577}%
\pgfsetfillcolor{currentfill}%
\pgfsetfillopacity{0.700000}%
\pgfsetlinewidth{0.000000pt}%
\definecolor{currentstroke}{rgb}{0.000000,0.000000,0.000000}%
\pgfsetstrokecolor{currentstroke}%
\pgfsetdash{}{0pt}%
\pgfpathmoveto{\pgfqpoint{4.797070in}{2.335884in}}%
\pgfpathlineto{\pgfqpoint{4.811059in}{2.336557in}}%
\pgfpathlineto{\pgfqpoint{4.825058in}{2.337301in}}%
\pgfpathlineto{\pgfqpoint{4.839067in}{2.338116in}}%
\pgfpathlineto{\pgfqpoint{4.853085in}{2.339002in}}%
\pgfpathlineto{\pgfqpoint{4.860730in}{2.345850in}}%
\pgfpathlineto{\pgfqpoint{4.868370in}{2.352699in}}%
\pgfpathlineto{\pgfqpoint{4.876003in}{2.359551in}}%
\pgfpathlineto{\pgfqpoint{4.883630in}{2.366411in}}%
\pgfpathlineto{\pgfqpoint{4.869627in}{2.365733in}}%
\pgfpathlineto{\pgfqpoint{4.855634in}{2.365126in}}%
\pgfpathlineto{\pgfqpoint{4.841650in}{2.364589in}}%
\pgfpathlineto{\pgfqpoint{4.827676in}{2.364123in}}%
\pgfpathlineto{\pgfqpoint{4.820033in}{2.357049in}}%
\pgfpathlineto{\pgfqpoint{4.812384in}{2.349986in}}%
\pgfpathlineto{\pgfqpoint{4.804730in}{2.342933in}}%
\pgfpathlineto{\pgfqpoint{4.797070in}{2.335884in}}%
\pgfpathclose%
\pgfusepath{fill}%
\end{pgfscope}%
\begin{pgfscope}%
\pgfpathrectangle{\pgfqpoint{1.150000in}{0.150000in}}{\pgfqpoint{5.700000in}{5.700000in}}%
\pgfusepath{clip}%
\pgfsetbuttcap%
\pgfsetroundjoin%
\definecolor{currentfill}{rgb}{0.250425,0.274290,0.533103}%
\pgfsetfillcolor{currentfill}%
\pgfsetfillopacity{0.700000}%
\pgfsetlinewidth{0.000000pt}%
\definecolor{currentstroke}{rgb}{0.000000,0.000000,0.000000}%
\pgfsetstrokecolor{currentstroke}%
\pgfsetdash{}{0pt}%
\pgfpathmoveto{\pgfqpoint{5.199343in}{2.459249in}}%
\pgfpathlineto{\pgfqpoint{5.213468in}{2.460363in}}%
\pgfpathlineto{\pgfqpoint{5.227603in}{2.461545in}}%
\pgfpathlineto{\pgfqpoint{5.241749in}{2.462796in}}%
\pgfpathlineto{\pgfqpoint{5.255905in}{2.464115in}}%
\pgfpathlineto{\pgfqpoint{5.263373in}{2.470054in}}%
\pgfpathlineto{\pgfqpoint{5.270836in}{2.476038in}}%
\pgfpathlineto{\pgfqpoint{5.278294in}{2.482073in}}%
\pgfpathlineto{\pgfqpoint{5.285747in}{2.488165in}}%
\pgfpathlineto{\pgfqpoint{5.271610in}{2.487137in}}%
\pgfpathlineto{\pgfqpoint{5.257484in}{2.486177in}}%
\pgfpathlineto{\pgfqpoint{5.243369in}{2.485285in}}%
\pgfpathlineto{\pgfqpoint{5.229264in}{2.484462in}}%
\pgfpathlineto{\pgfqpoint{5.221791in}{2.478073in}}%
\pgfpathlineto{\pgfqpoint{5.214313in}{2.471744in}}%
\pgfpathlineto{\pgfqpoint{5.206830in}{2.465471in}}%
\pgfpathlineto{\pgfqpoint{5.199343in}{2.459249in}}%
\pgfpathclose%
\pgfusepath{fill}%
\end{pgfscope}%
\begin{pgfscope}%
\pgfpathrectangle{\pgfqpoint{1.150000in}{0.150000in}}{\pgfqpoint{5.700000in}{5.700000in}}%
\pgfusepath{clip}%
\pgfsetbuttcap%
\pgfsetroundjoin%
\definecolor{currentfill}{rgb}{0.272594,0.025563,0.353093}%
\pgfsetfillcolor{currentfill}%
\pgfsetfillopacity{0.700000}%
\pgfsetlinewidth{0.000000pt}%
\definecolor{currentstroke}{rgb}{0.000000,0.000000,0.000000}%
\pgfsetstrokecolor{currentstroke}%
\pgfsetdash{}{0pt}%
\pgfpathmoveto{\pgfqpoint{3.219622in}{1.967900in}}%
\pgfpathlineto{\pgfqpoint{3.233208in}{1.962903in}}%
\pgfpathlineto{\pgfqpoint{3.246797in}{1.957996in}}%
\pgfpathlineto{\pgfqpoint{3.260391in}{1.953179in}}%
\pgfpathlineto{\pgfqpoint{3.273989in}{1.948452in}}%
\pgfpathlineto{\pgfqpoint{3.282231in}{1.956933in}}%
\pgfpathlineto{\pgfqpoint{3.290467in}{1.965441in}}%
\pgfpathlineto{\pgfqpoint{3.298696in}{1.973974in}}%
\pgfpathlineto{\pgfqpoint{3.306918in}{1.982533in}}%
\pgfpathlineto{\pgfqpoint{3.293334in}{1.987157in}}%
\pgfpathlineto{\pgfqpoint{3.279754in}{1.991871in}}%
\pgfpathlineto{\pgfqpoint{3.266179in}{1.996675in}}%
\pgfpathlineto{\pgfqpoint{3.252608in}{2.001570in}}%
\pgfpathlineto{\pgfqpoint{3.244371in}{1.993107in}}%
\pgfpathlineto{\pgfqpoint{3.236128in}{1.984673in}}%
\pgfpathlineto{\pgfqpoint{3.227879in}{1.976271in}}%
\pgfpathlineto{\pgfqpoint{3.219622in}{1.967900in}}%
\pgfpathclose%
\pgfusepath{fill}%
\end{pgfscope}%
\begin{pgfscope}%
\pgfpathrectangle{\pgfqpoint{1.150000in}{0.150000in}}{\pgfqpoint{5.700000in}{5.700000in}}%
\pgfusepath{clip}%
\pgfsetbuttcap%
\pgfsetroundjoin%
\definecolor{currentfill}{rgb}{0.277941,0.056324,0.381191}%
\pgfsetfillcolor{currentfill}%
\pgfsetfillopacity{0.700000}%
\pgfsetlinewidth{0.000000pt}%
\definecolor{currentstroke}{rgb}{0.000000,0.000000,0.000000}%
\pgfsetstrokecolor{currentstroke}%
\pgfsetdash{}{0pt}%
\pgfpathmoveto{\pgfqpoint{2.793753in}{2.029415in}}%
\pgfpathlineto{\pgfqpoint{2.807320in}{2.021644in}}%
\pgfpathlineto{\pgfqpoint{2.820889in}{2.013978in}}%
\pgfpathlineto{\pgfqpoint{2.834459in}{2.006416in}}%
\pgfpathlineto{\pgfqpoint{2.848031in}{1.998957in}}%
\pgfpathlineto{\pgfqpoint{2.856462in}{2.005954in}}%
\pgfpathlineto{\pgfqpoint{2.864885in}{2.013030in}}%
\pgfpathlineto{\pgfqpoint{2.873299in}{2.020183in}}%
\pgfpathlineto{\pgfqpoint{2.881705in}{2.027411in}}%
\pgfpathlineto{\pgfqpoint{2.868152in}{2.034704in}}%
\pgfpathlineto{\pgfqpoint{2.854601in}{2.042101in}}%
\pgfpathlineto{\pgfqpoint{2.841052in}{2.049601in}}%
\pgfpathlineto{\pgfqpoint{2.827505in}{2.057206in}}%
\pgfpathlineto{\pgfqpoint{2.819080in}{2.050137in}}%
\pgfpathlineto{\pgfqpoint{2.810647in}{2.043147in}}%
\pgfpathlineto{\pgfqpoint{2.802204in}{2.036239in}}%
\pgfpathlineto{\pgfqpoint{2.793753in}{2.029415in}}%
\pgfpathclose%
\pgfusepath{fill}%
\end{pgfscope}%
\begin{pgfscope}%
\pgfpathrectangle{\pgfqpoint{1.150000in}{0.150000in}}{\pgfqpoint{5.700000in}{5.700000in}}%
\pgfusepath{clip}%
\pgfsetbuttcap%
\pgfsetroundjoin%
\definecolor{currentfill}{rgb}{0.282290,0.145912,0.461510}%
\pgfsetfillcolor{currentfill}%
\pgfsetfillopacity{0.700000}%
\pgfsetlinewidth{0.000000pt}%
\definecolor{currentstroke}{rgb}{0.000000,0.000000,0.000000}%
\pgfsetstrokecolor{currentstroke}%
\pgfsetdash{}{0pt}%
\pgfpathmoveto{\pgfqpoint{4.308061in}{2.178773in}}%
\pgfpathlineto{\pgfqpoint{4.321895in}{2.178430in}}%
\pgfpathlineto{\pgfqpoint{4.335737in}{2.178162in}}%
\pgfpathlineto{\pgfqpoint{4.349588in}{2.177968in}}%
\pgfpathlineto{\pgfqpoint{4.363448in}{2.177847in}}%
\pgfpathlineto{\pgfqpoint{4.371293in}{2.185956in}}%
\pgfpathlineto{\pgfqpoint{4.379133in}{2.194036in}}%
\pgfpathlineto{\pgfqpoint{4.386967in}{2.202090in}}%
\pgfpathlineto{\pgfqpoint{4.394794in}{2.210120in}}%
\pgfpathlineto{\pgfqpoint{4.380946in}{2.210344in}}%
\pgfpathlineto{\pgfqpoint{4.367107in}{2.210641in}}%
\pgfpathlineto{\pgfqpoint{4.353276in}{2.211013in}}%
\pgfpathlineto{\pgfqpoint{4.339454in}{2.211459in}}%
\pgfpathlineto{\pgfqpoint{4.331615in}{2.203318in}}%
\pgfpathlineto{\pgfqpoint{4.323769in}{2.195158in}}%
\pgfpathlineto{\pgfqpoint{4.315918in}{2.186977in}}%
\pgfpathlineto{\pgfqpoint{4.308061in}{2.178773in}}%
\pgfpathclose%
\pgfusepath{fill}%
\end{pgfscope}%
\begin{pgfscope}%
\pgfpathrectangle{\pgfqpoint{1.150000in}{0.150000in}}{\pgfqpoint{5.700000in}{5.700000in}}%
\pgfusepath{clip}%
\pgfsetbuttcap%
\pgfsetroundjoin%
\definecolor{currentfill}{rgb}{0.282290,0.145912,0.461510}%
\pgfsetfillcolor{currentfill}%
\pgfsetfillopacity{0.700000}%
\pgfsetlinewidth{0.000000pt}%
\definecolor{currentstroke}{rgb}{0.000000,0.000000,0.000000}%
\pgfsetstrokecolor{currentstroke}%
\pgfsetdash{}{0pt}%
\pgfpathmoveto{\pgfqpoint{2.399298in}{2.207634in}}%
\pgfpathlineto{\pgfqpoint{2.412909in}{2.196706in}}%
\pgfpathlineto{\pgfqpoint{2.426519in}{2.185903in}}%
\pgfpathlineto{\pgfqpoint{2.440127in}{2.175226in}}%
\pgfpathlineto{\pgfqpoint{2.453734in}{2.164672in}}%
\pgfpathlineto{\pgfqpoint{2.462381in}{2.169649in}}%
\pgfpathlineto{\pgfqpoint{2.471017in}{2.174758in}}%
\pgfpathlineto{\pgfqpoint{2.479641in}{2.179995in}}%
\pgfpathlineto{\pgfqpoint{2.488254in}{2.185358in}}%
\pgfpathlineto{\pgfqpoint{2.474672in}{2.195702in}}%
\pgfpathlineto{\pgfqpoint{2.461089in}{2.206169in}}%
\pgfpathlineto{\pgfqpoint{2.447506in}{2.216761in}}%
\pgfpathlineto{\pgfqpoint{2.433921in}{2.227478in}}%
\pgfpathlineto{\pgfqpoint{2.425283in}{2.222318in}}%
\pgfpathlineto{\pgfqpoint{2.416633in}{2.217289in}}%
\pgfpathlineto{\pgfqpoint{2.407972in}{2.212393in}}%
\pgfpathlineto{\pgfqpoint{2.399298in}{2.207634in}}%
\pgfpathclose%
\pgfusepath{fill}%
\end{pgfscope}%
\begin{pgfscope}%
\pgfpathrectangle{\pgfqpoint{1.150000in}{0.150000in}}{\pgfqpoint{5.700000in}{5.700000in}}%
\pgfusepath{clip}%
\pgfsetbuttcap%
\pgfsetroundjoin%
\definecolor{currentfill}{rgb}{0.233603,0.313828,0.543914}%
\pgfsetfillcolor{currentfill}%
\pgfsetfillopacity{0.700000}%
\pgfsetlinewidth{0.000000pt}%
\definecolor{currentstroke}{rgb}{0.000000,0.000000,0.000000}%
\pgfsetstrokecolor{currentstroke}%
\pgfsetdash{}{0pt}%
\pgfpathmoveto{\pgfqpoint{5.515186in}{2.549512in}}%
\pgfpathlineto{\pgfqpoint{5.529419in}{2.550779in}}%
\pgfpathlineto{\pgfqpoint{5.543664in}{2.552114in}}%
\pgfpathlineto{\pgfqpoint{5.557920in}{2.553516in}}%
\pgfpathlineto{\pgfqpoint{5.572187in}{2.554985in}}%
\pgfpathlineto{\pgfqpoint{5.579515in}{2.560454in}}%
\pgfpathlineto{\pgfqpoint{5.586840in}{2.566018in}}%
\pgfpathlineto{\pgfqpoint{5.594161in}{2.571681in}}%
\pgfpathlineto{\pgfqpoint{5.601480in}{2.577451in}}%
\pgfpathlineto{\pgfqpoint{5.587237in}{2.576336in}}%
\pgfpathlineto{\pgfqpoint{5.573005in}{2.575287in}}%
\pgfpathlineto{\pgfqpoint{5.558784in}{2.574306in}}%
\pgfpathlineto{\pgfqpoint{5.544574in}{2.573391in}}%
\pgfpathlineto{\pgfqpoint{5.537232in}{2.567261in}}%
\pgfpathlineto{\pgfqpoint{5.529886in}{2.561242in}}%
\pgfpathlineto{\pgfqpoint{5.522537in}{2.555328in}}%
\pgfpathlineto{\pgfqpoint{5.515186in}{2.549512in}}%
\pgfpathclose%
\pgfusepath{fill}%
\end{pgfscope}%
\begin{pgfscope}%
\pgfpathrectangle{\pgfqpoint{1.150000in}{0.150000in}}{\pgfqpoint{5.700000in}{5.700000in}}%
\pgfusepath{clip}%
\pgfsetbuttcap%
\pgfsetroundjoin%
\definecolor{currentfill}{rgb}{0.273006,0.204520,0.501721}%
\pgfsetfillcolor{currentfill}%
\pgfsetfillopacity{0.700000}%
\pgfsetlinewidth{0.000000pt}%
\definecolor{currentstroke}{rgb}{0.000000,0.000000,0.000000}%
\pgfsetstrokecolor{currentstroke}%
\pgfsetdash{}{0pt}%
\pgfpathmoveto{\pgfqpoint{4.710451in}{2.304934in}}%
\pgfpathlineto{\pgfqpoint{4.724416in}{2.305509in}}%
\pgfpathlineto{\pgfqpoint{4.738391in}{2.306156in}}%
\pgfpathlineto{\pgfqpoint{4.752375in}{2.306874in}}%
\pgfpathlineto{\pgfqpoint{4.766369in}{2.307663in}}%
\pgfpathlineto{\pgfqpoint{4.774053in}{2.314729in}}%
\pgfpathlineto{\pgfqpoint{4.781731in}{2.321786in}}%
\pgfpathlineto{\pgfqpoint{4.789403in}{2.328836in}}%
\pgfpathlineto{\pgfqpoint{4.797070in}{2.335884in}}%
\pgfpathlineto{\pgfqpoint{4.783090in}{2.335282in}}%
\pgfpathlineto{\pgfqpoint{4.769121in}{2.334750in}}%
\pgfpathlineto{\pgfqpoint{4.755160in}{2.334290in}}%
\pgfpathlineto{\pgfqpoint{4.741210in}{2.333901in}}%
\pgfpathlineto{\pgfqpoint{4.733529in}{2.326659in}}%
\pgfpathlineto{\pgfqpoint{4.725842in}{2.319420in}}%
\pgfpathlineto{\pgfqpoint{4.718150in}{2.312179in}}%
\pgfpathlineto{\pgfqpoint{4.710451in}{2.304934in}}%
\pgfpathclose%
\pgfusepath{fill}%
\end{pgfscope}%
\begin{pgfscope}%
\pgfpathrectangle{\pgfqpoint{1.150000in}{0.150000in}}{\pgfqpoint{5.700000in}{5.700000in}}%
\pgfusepath{clip}%
\pgfsetbuttcap%
\pgfsetroundjoin%
\definecolor{currentfill}{rgb}{0.281446,0.084320,0.407414}%
\pgfsetfillcolor{currentfill}%
\pgfsetfillopacity{0.700000}%
\pgfsetlinewidth{0.000000pt}%
\definecolor{currentstroke}{rgb}{0.000000,0.000000,0.000000}%
\pgfsetstrokecolor{currentstroke}%
\pgfsetdash{}{0pt}%
\pgfpathmoveto{\pgfqpoint{3.905696in}{2.061563in}}%
\pgfpathlineto{\pgfqpoint{3.919416in}{2.059899in}}%
\pgfpathlineto{\pgfqpoint{3.933143in}{2.058314in}}%
\pgfpathlineto{\pgfqpoint{3.946878in}{2.056808in}}%
\pgfpathlineto{\pgfqpoint{3.960619in}{2.055379in}}%
\pgfpathlineto{\pgfqpoint{3.968613in}{2.064227in}}%
\pgfpathlineto{\pgfqpoint{3.976600in}{2.073050in}}%
\pgfpathlineto{\pgfqpoint{3.984582in}{2.081848in}}%
\pgfpathlineto{\pgfqpoint{3.992559in}{2.090623in}}%
\pgfpathlineto{\pgfqpoint{3.978828in}{2.092072in}}%
\pgfpathlineto{\pgfqpoint{3.965104in}{2.093600in}}%
\pgfpathlineto{\pgfqpoint{3.951388in}{2.095205in}}%
\pgfpathlineto{\pgfqpoint{3.937679in}{2.096889in}}%
\pgfpathlineto{\pgfqpoint{3.929692in}{2.088086in}}%
\pgfpathlineto{\pgfqpoint{3.921699in}{2.079264in}}%
\pgfpathlineto{\pgfqpoint{3.913700in}{2.070424in}}%
\pgfpathlineto{\pgfqpoint{3.905696in}{2.061563in}}%
\pgfpathclose%
\pgfusepath{fill}%
\end{pgfscope}%
\begin{pgfscope}%
\pgfpathrectangle{\pgfqpoint{1.150000in}{0.150000in}}{\pgfqpoint{5.700000in}{5.700000in}}%
\pgfusepath{clip}%
\pgfsetbuttcap%
\pgfsetroundjoin%
\definecolor{currentfill}{rgb}{0.255645,0.260703,0.528312}%
\pgfsetfillcolor{currentfill}%
\pgfsetfillopacity{0.700000}%
\pgfsetlinewidth{0.000000pt}%
\definecolor{currentstroke}{rgb}{0.000000,0.000000,0.000000}%
\pgfsetstrokecolor{currentstroke}%
\pgfsetdash{}{0pt}%
\pgfpathmoveto{\pgfqpoint{5.112874in}{2.429924in}}%
\pgfpathlineto{\pgfqpoint{5.126975in}{2.431032in}}%
\pgfpathlineto{\pgfqpoint{5.141087in}{2.432209in}}%
\pgfpathlineto{\pgfqpoint{5.155210in}{2.433455in}}%
\pgfpathlineto{\pgfqpoint{5.169343in}{2.434770in}}%
\pgfpathlineto{\pgfqpoint{5.176851in}{2.440838in}}%
\pgfpathlineto{\pgfqpoint{5.184353in}{2.446938in}}%
\pgfpathlineto{\pgfqpoint{5.191851in}{2.453073in}}%
\pgfpathlineto{\pgfqpoint{5.199343in}{2.459249in}}%
\pgfpathlineto{\pgfqpoint{5.185229in}{2.458205in}}%
\pgfpathlineto{\pgfqpoint{5.171125in}{2.457229in}}%
\pgfpathlineto{\pgfqpoint{5.157032in}{2.456322in}}%
\pgfpathlineto{\pgfqpoint{5.142949in}{2.455484in}}%
\pgfpathlineto{\pgfqpoint{5.135438in}{2.449031in}}%
\pgfpathlineto{\pgfqpoint{5.127922in}{2.442623in}}%
\pgfpathlineto{\pgfqpoint{5.120400in}{2.436255in}}%
\pgfpathlineto{\pgfqpoint{5.112874in}{2.429924in}}%
\pgfpathclose%
\pgfusepath{fill}%
\end{pgfscope}%
\begin{pgfscope}%
\pgfpathrectangle{\pgfqpoint{1.150000in}{0.150000in}}{\pgfqpoint{5.700000in}{5.700000in}}%
\pgfusepath{clip}%
\pgfsetbuttcap%
\pgfsetroundjoin%
\definecolor{currentfill}{rgb}{0.272594,0.025563,0.353093}%
\pgfsetfillcolor{currentfill}%
\pgfsetfillopacity{0.700000}%
\pgfsetlinewidth{0.000000pt}%
\definecolor{currentstroke}{rgb}{0.000000,0.000000,0.000000}%
\pgfsetstrokecolor{currentstroke}%
\pgfsetdash{}{0pt}%
\pgfpathmoveto{\pgfqpoint{3.361301in}{1.964925in}}%
\pgfpathlineto{\pgfqpoint{3.374909in}{1.960743in}}%
\pgfpathlineto{\pgfqpoint{3.388521in}{1.956649in}}%
\pgfpathlineto{\pgfqpoint{3.402139in}{1.952641in}}%
\pgfpathlineto{\pgfqpoint{3.415761in}{1.948720in}}%
\pgfpathlineto{\pgfqpoint{3.423951in}{1.957481in}}%
\pgfpathlineto{\pgfqpoint{3.432134in}{1.966256in}}%
\pgfpathlineto{\pgfqpoint{3.440311in}{1.975041in}}%
\pgfpathlineto{\pgfqpoint{3.448482in}{1.983837in}}%
\pgfpathlineto{\pgfqpoint{3.434872in}{1.987677in}}%
\pgfpathlineto{\pgfqpoint{3.421268in}{1.991602in}}%
\pgfpathlineto{\pgfqpoint{3.407668in}{1.995614in}}%
\pgfpathlineto{\pgfqpoint{3.394074in}{1.999714in}}%
\pgfpathlineto{\pgfqpoint{3.385890in}{1.990992in}}%
\pgfpathlineto{\pgfqpoint{3.377700in}{1.982286in}}%
\pgfpathlineto{\pgfqpoint{3.369503in}{1.973597in}}%
\pgfpathlineto{\pgfqpoint{3.361301in}{1.964925in}}%
\pgfpathclose%
\pgfusepath{fill}%
\end{pgfscope}%
\begin{pgfscope}%
\pgfpathrectangle{\pgfqpoint{1.150000in}{0.150000in}}{\pgfqpoint{5.700000in}{5.700000in}}%
\pgfusepath{clip}%
\pgfsetbuttcap%
\pgfsetroundjoin%
\definecolor{currentfill}{rgb}{0.276022,0.044167,0.370164}%
\pgfsetfillcolor{currentfill}%
\pgfsetfillopacity{0.700000}%
\pgfsetlinewidth{0.000000pt}%
\definecolor{currentstroke}{rgb}{0.000000,0.000000,0.000000}%
\pgfsetstrokecolor{currentstroke}%
\pgfsetdash{}{0pt}%
\pgfpathmoveto{\pgfqpoint{3.590082in}{1.991964in}}%
\pgfpathlineto{\pgfqpoint{3.603730in}{1.988948in}}%
\pgfpathlineto{\pgfqpoint{3.617384in}{1.986015in}}%
\pgfpathlineto{\pgfqpoint{3.631044in}{1.983165in}}%
\pgfpathlineto{\pgfqpoint{3.644710in}{1.980396in}}%
\pgfpathlineto{\pgfqpoint{3.652815in}{1.989384in}}%
\pgfpathlineto{\pgfqpoint{3.660915in}{1.998363in}}%
\pgfpathlineto{\pgfqpoint{3.669009in}{2.007335in}}%
\pgfpathlineto{\pgfqpoint{3.677097in}{2.016299in}}%
\pgfpathlineto{\pgfqpoint{3.663443in}{2.019026in}}%
\pgfpathlineto{\pgfqpoint{3.649794in}{2.021835in}}%
\pgfpathlineto{\pgfqpoint{3.636152in}{2.024727in}}%
\pgfpathlineto{\pgfqpoint{3.622516in}{2.027702in}}%
\pgfpathlineto{\pgfqpoint{3.614416in}{2.018772in}}%
\pgfpathlineto{\pgfqpoint{3.606311in}{2.009839in}}%
\pgfpathlineto{\pgfqpoint{3.598199in}{2.000903in}}%
\pgfpathlineto{\pgfqpoint{3.590082in}{1.991964in}}%
\pgfpathclose%
\pgfusepath{fill}%
\end{pgfscope}%
\begin{pgfscope}%
\pgfpathrectangle{\pgfqpoint{1.150000in}{0.150000in}}{\pgfqpoint{5.700000in}{5.700000in}}%
\pgfusepath{clip}%
\pgfsetbuttcap%
\pgfsetroundjoin%
\definecolor{currentfill}{rgb}{0.216210,0.351535,0.550627}%
\pgfsetfillcolor{currentfill}%
\pgfsetfillopacity{0.700000}%
\pgfsetlinewidth{0.000000pt}%
\definecolor{currentstroke}{rgb}{0.000000,0.000000,0.000000}%
\pgfsetstrokecolor{currentstroke}%
\pgfsetdash{}{0pt}%
\pgfpathmoveto{\pgfqpoint{5.831125in}{2.637607in}}%
\pgfpathlineto{\pgfqpoint{5.845462in}{2.638821in}}%
\pgfpathlineto{\pgfqpoint{5.859811in}{2.640101in}}%
\pgfpathlineto{\pgfqpoint{5.874171in}{2.641448in}}%
\pgfpathlineto{\pgfqpoint{5.888542in}{2.642860in}}%
\pgfpathlineto{\pgfqpoint{5.895738in}{2.648257in}}%
\pgfpathlineto{\pgfqpoint{5.902934in}{2.653809in}}%
\pgfpathlineto{\pgfqpoint{5.910129in}{2.659522in}}%
\pgfpathlineto{\pgfqpoint{5.917324in}{2.665404in}}%
\pgfpathlineto{\pgfqpoint{5.902982in}{2.664406in}}%
\pgfpathlineto{\pgfqpoint{5.888650in}{2.663475in}}%
\pgfpathlineto{\pgfqpoint{5.874330in}{2.662609in}}%
\pgfpathlineto{\pgfqpoint{5.860022in}{2.661810in}}%
\pgfpathlineto{\pgfqpoint{5.852798in}{2.655506in}}%
\pgfpathlineto{\pgfqpoint{5.845574in}{2.649375in}}%
\pgfpathlineto{\pgfqpoint{5.838350in}{2.643411in}}%
\pgfpathlineto{\pgfqpoint{5.831125in}{2.637607in}}%
\pgfpathclose%
\pgfusepath{fill}%
\end{pgfscope}%
\begin{pgfscope}%
\pgfpathrectangle{\pgfqpoint{1.150000in}{0.150000in}}{\pgfqpoint{5.700000in}{5.700000in}}%
\pgfusepath{clip}%
\pgfsetbuttcap%
\pgfsetroundjoin%
\definecolor{currentfill}{rgb}{0.281446,0.084320,0.407414}%
\pgfsetfillcolor{currentfill}%
\pgfsetfillopacity{0.700000}%
\pgfsetlinewidth{0.000000pt}%
\definecolor{currentstroke}{rgb}{0.000000,0.000000,0.000000}%
\pgfsetstrokecolor{currentstroke}%
\pgfsetdash{}{0pt}%
\pgfpathmoveto{\pgfqpoint{2.651198in}{2.070528in}}%
\pgfpathlineto{\pgfqpoint{2.664778in}{2.061706in}}%
\pgfpathlineto{\pgfqpoint{2.678358in}{2.052994in}}%
\pgfpathlineto{\pgfqpoint{2.691939in}{2.044393in}}%
\pgfpathlineto{\pgfqpoint{2.705520in}{2.035900in}}%
\pgfpathlineto{\pgfqpoint{2.714029in}{2.042169in}}%
\pgfpathlineto{\pgfqpoint{2.722529in}{2.048537in}}%
\pgfpathlineto{\pgfqpoint{2.731019in}{2.055001in}}%
\pgfpathlineto{\pgfqpoint{2.739500in}{2.061560in}}%
\pgfpathlineto{\pgfqpoint{2.725940in}{2.069865in}}%
\pgfpathlineto{\pgfqpoint{2.712380in}{2.078279in}}%
\pgfpathlineto{\pgfqpoint{2.698822in}{2.086804in}}%
\pgfpathlineto{\pgfqpoint{2.685264in}{2.095439in}}%
\pgfpathlineto{\pgfqpoint{2.676763in}{2.089060in}}%
\pgfpathlineto{\pgfqpoint{2.668251in}{2.082780in}}%
\pgfpathlineto{\pgfqpoint{2.659729in}{2.076602in}}%
\pgfpathlineto{\pgfqpoint{2.651198in}{2.070528in}}%
\pgfpathclose%
\pgfusepath{fill}%
\end{pgfscope}%
\begin{pgfscope}%
\pgfpathrectangle{\pgfqpoint{1.150000in}{0.150000in}}{\pgfqpoint{5.700000in}{5.700000in}}%
\pgfusepath{clip}%
\pgfsetbuttcap%
\pgfsetroundjoin%
\definecolor{currentfill}{rgb}{0.283072,0.130895,0.449241}%
\pgfsetfillcolor{currentfill}%
\pgfsetfillopacity{0.700000}%
\pgfsetlinewidth{0.000000pt}%
\definecolor{currentstroke}{rgb}{0.000000,0.000000,0.000000}%
\pgfsetstrokecolor{currentstroke}%
\pgfsetdash{}{0pt}%
\pgfpathmoveto{\pgfqpoint{4.221277in}{2.147472in}}%
\pgfpathlineto{\pgfqpoint{4.235089in}{2.146913in}}%
\pgfpathlineto{\pgfqpoint{4.248909in}{2.146429in}}%
\pgfpathlineto{\pgfqpoint{4.262737in}{2.146020in}}%
\pgfpathlineto{\pgfqpoint{4.276574in}{2.145686in}}%
\pgfpathlineto{\pgfqpoint{4.284454in}{2.154002in}}%
\pgfpathlineto{\pgfqpoint{4.292329in}{2.162287in}}%
\pgfpathlineto{\pgfqpoint{4.300198in}{2.170544in}}%
\pgfpathlineto{\pgfqpoint{4.308061in}{2.178773in}}%
\pgfpathlineto{\pgfqpoint{4.294236in}{2.179190in}}%
\pgfpathlineto{\pgfqpoint{4.280419in}{2.179682in}}%
\pgfpathlineto{\pgfqpoint{4.266610in}{2.180249in}}%
\pgfpathlineto{\pgfqpoint{4.252810in}{2.180890in}}%
\pgfpathlineto{\pgfqpoint{4.244935in}{2.172571in}}%
\pgfpathlineto{\pgfqpoint{4.237055in}{2.164229in}}%
\pgfpathlineto{\pgfqpoint{4.229169in}{2.155864in}}%
\pgfpathlineto{\pgfqpoint{4.221277in}{2.147472in}}%
\pgfpathclose%
\pgfusepath{fill}%
\end{pgfscope}%
\begin{pgfscope}%
\pgfpathrectangle{\pgfqpoint{1.150000in}{0.150000in}}{\pgfqpoint{5.700000in}{5.700000in}}%
\pgfusepath{clip}%
\pgfsetbuttcap%
\pgfsetroundjoin%
\definecolor{currentfill}{rgb}{0.275191,0.194905,0.496005}%
\pgfsetfillcolor{currentfill}%
\pgfsetfillopacity{0.700000}%
\pgfsetlinewidth{0.000000pt}%
\definecolor{currentstroke}{rgb}{0.000000,0.000000,0.000000}%
\pgfsetstrokecolor{currentstroke}%
\pgfsetdash{}{0pt}%
\pgfpathmoveto{\pgfqpoint{4.623778in}{2.273591in}}%
\pgfpathlineto{\pgfqpoint{4.637719in}{2.274045in}}%
\pgfpathlineto{\pgfqpoint{4.651669in}{2.274572in}}%
\pgfpathlineto{\pgfqpoint{4.665628in}{2.275170in}}%
\pgfpathlineto{\pgfqpoint{4.679597in}{2.275840in}}%
\pgfpathlineto{\pgfqpoint{4.687320in}{2.283137in}}%
\pgfpathlineto{\pgfqpoint{4.695037in}{2.290416in}}%
\pgfpathlineto{\pgfqpoint{4.702747in}{2.297680in}}%
\pgfpathlineto{\pgfqpoint{4.710451in}{2.304934in}}%
\pgfpathlineto{\pgfqpoint{4.696496in}{2.304430in}}%
\pgfpathlineto{\pgfqpoint{4.682550in}{2.303998in}}%
\pgfpathlineto{\pgfqpoint{4.668614in}{2.303637in}}%
\pgfpathlineto{\pgfqpoint{4.654687in}{2.303348in}}%
\pgfpathlineto{\pgfqpoint{4.646969in}{2.295921in}}%
\pgfpathlineto{\pgfqpoint{4.639244in}{2.288488in}}%
\pgfpathlineto{\pgfqpoint{4.631514in}{2.281046in}}%
\pgfpathlineto{\pgfqpoint{4.623778in}{2.273591in}}%
\pgfpathclose%
\pgfusepath{fill}%
\end{pgfscope}%
\begin{pgfscope}%
\pgfpathrectangle{\pgfqpoint{1.150000in}{0.150000in}}{\pgfqpoint{5.700000in}{5.700000in}}%
\pgfusepath{clip}%
\pgfsetbuttcap%
\pgfsetroundjoin%
\definecolor{currentfill}{rgb}{0.237441,0.305202,0.541921}%
\pgfsetfillcolor{currentfill}%
\pgfsetfillopacity{0.700000}%
\pgfsetlinewidth{0.000000pt}%
\definecolor{currentstroke}{rgb}{0.000000,0.000000,0.000000}%
\pgfsetstrokecolor{currentstroke}%
\pgfsetdash{}{0pt}%
\pgfpathmoveto{\pgfqpoint{5.428827in}{2.521392in}}%
\pgfpathlineto{\pgfqpoint{5.443039in}{2.522722in}}%
\pgfpathlineto{\pgfqpoint{5.457262in}{2.524119in}}%
\pgfpathlineto{\pgfqpoint{5.471497in}{2.525584in}}%
\pgfpathlineto{\pgfqpoint{5.485743in}{2.527117in}}%
\pgfpathlineto{\pgfqpoint{5.493109in}{2.532598in}}%
\pgfpathlineto{\pgfqpoint{5.500472in}{2.538153in}}%
\pgfpathlineto{\pgfqpoint{5.507831in}{2.543789in}}%
\pgfpathlineto{\pgfqpoint{5.515186in}{2.549512in}}%
\pgfpathlineto{\pgfqpoint{5.500963in}{2.548313in}}%
\pgfpathlineto{\pgfqpoint{5.486751in}{2.547181in}}%
\pgfpathlineto{\pgfqpoint{5.472551in}{2.546116in}}%
\pgfpathlineto{\pgfqpoint{5.458361in}{2.545119in}}%
\pgfpathlineto{\pgfqpoint{5.450983in}{2.539056in}}%
\pgfpathlineto{\pgfqpoint{5.443601in}{2.533085in}}%
\pgfpathlineto{\pgfqpoint{5.436216in}{2.527199in}}%
\pgfpathlineto{\pgfqpoint{5.428827in}{2.521392in}}%
\pgfpathclose%
\pgfusepath{fill}%
\end{pgfscope}%
\begin{pgfscope}%
\pgfpathrectangle{\pgfqpoint{1.150000in}{0.150000in}}{\pgfqpoint{5.700000in}{5.700000in}}%
\pgfusepath{clip}%
\pgfsetbuttcap%
\pgfsetroundjoin%
\definecolor{currentfill}{rgb}{0.273809,0.031497,0.358853}%
\pgfsetfillcolor{currentfill}%
\pgfsetfillopacity{0.700000}%
\pgfsetlinewidth{0.000000pt}%
\definecolor{currentstroke}{rgb}{0.000000,0.000000,0.000000}%
\pgfsetstrokecolor{currentstroke}%
\pgfsetdash{}{0pt}%
\pgfpathmoveto{\pgfqpoint{2.990212in}{1.972703in}}%
\pgfpathlineto{\pgfqpoint{3.003787in}{1.966310in}}%
\pgfpathlineto{\pgfqpoint{3.017365in}{1.960014in}}%
\pgfpathlineto{\pgfqpoint{3.030946in}{1.953814in}}%
\pgfpathlineto{\pgfqpoint{3.044530in}{1.947710in}}%
\pgfpathlineto{\pgfqpoint{3.052875in}{1.955466in}}%
\pgfpathlineto{\pgfqpoint{3.061212in}{1.963278in}}%
\pgfpathlineto{\pgfqpoint{3.069541in}{1.971143in}}%
\pgfpathlineto{\pgfqpoint{3.077863in}{1.979059in}}%
\pgfpathlineto{\pgfqpoint{3.064296in}{1.985019in}}%
\pgfpathlineto{\pgfqpoint{3.050732in}{1.991074in}}%
\pgfpathlineto{\pgfqpoint{3.037171in}{1.997226in}}%
\pgfpathlineto{\pgfqpoint{3.023613in}{2.003475in}}%
\pgfpathlineto{\pgfqpoint{3.015274in}{1.995695in}}%
\pgfpathlineto{\pgfqpoint{3.006928in}{1.987972in}}%
\pgfpathlineto{\pgfqpoint{2.998574in}{1.980307in}}%
\pgfpathlineto{\pgfqpoint{2.990212in}{1.972703in}}%
\pgfpathclose%
\pgfusepath{fill}%
\end{pgfscope}%
\begin{pgfscope}%
\pgfpathrectangle{\pgfqpoint{1.150000in}{0.150000in}}{\pgfqpoint{5.700000in}{5.700000in}}%
\pgfusepath{clip}%
\pgfsetbuttcap%
\pgfsetroundjoin%
\definecolor{currentfill}{rgb}{0.280267,0.073417,0.397163}%
\pgfsetfillcolor{currentfill}%
\pgfsetfillopacity{0.700000}%
\pgfsetlinewidth{0.000000pt}%
\definecolor{currentstroke}{rgb}{0.000000,0.000000,0.000000}%
\pgfsetstrokecolor{currentstroke}%
\pgfsetdash{}{0pt}%
\pgfpathmoveto{\pgfqpoint{3.818771in}{2.033347in}}%
\pgfpathlineto{\pgfqpoint{3.832474in}{2.031367in}}%
\pgfpathlineto{\pgfqpoint{3.846183in}{2.029467in}}%
\pgfpathlineto{\pgfqpoint{3.859900in}{2.027646in}}%
\pgfpathlineto{\pgfqpoint{3.873623in}{2.025904in}}%
\pgfpathlineto{\pgfqpoint{3.881650in}{2.034852in}}%
\pgfpathlineto{\pgfqpoint{3.889671in}{2.043777in}}%
\pgfpathlineto{\pgfqpoint{3.897686in}{2.052681in}}%
\pgfpathlineto{\pgfqpoint{3.905696in}{2.061563in}}%
\pgfpathlineto{\pgfqpoint{3.891984in}{2.063305in}}%
\pgfpathlineto{\pgfqpoint{3.878278in}{2.065125in}}%
\pgfpathlineto{\pgfqpoint{3.864579in}{2.067025in}}%
\pgfpathlineto{\pgfqpoint{3.850888in}{2.069005in}}%
\pgfpathlineto{\pgfqpoint{3.842867in}{2.060115in}}%
\pgfpathlineto{\pgfqpoint{3.834841in}{2.051210in}}%
\pgfpathlineto{\pgfqpoint{3.826809in}{2.042287in}}%
\pgfpathlineto{\pgfqpoint{3.818771in}{2.033347in}}%
\pgfpathclose%
\pgfusepath{fill}%
\end{pgfscope}%
\begin{pgfscope}%
\pgfpathrectangle{\pgfqpoint{1.150000in}{0.150000in}}{\pgfqpoint{5.700000in}{5.700000in}}%
\pgfusepath{clip}%
\pgfsetbuttcap%
\pgfsetroundjoin%
\definecolor{currentfill}{rgb}{0.258965,0.251537,0.524736}%
\pgfsetfillcolor{currentfill}%
\pgfsetfillopacity{0.700000}%
\pgfsetlinewidth{0.000000pt}%
\definecolor{currentstroke}{rgb}{0.000000,0.000000,0.000000}%
\pgfsetstrokecolor{currentstroke}%
\pgfsetdash{}{0pt}%
\pgfpathmoveto{\pgfqpoint{5.026340in}{2.400128in}}%
\pgfpathlineto{\pgfqpoint{5.040418in}{2.401208in}}%
\pgfpathlineto{\pgfqpoint{5.054506in}{2.402358in}}%
\pgfpathlineto{\pgfqpoint{5.068604in}{2.403577in}}%
\pgfpathlineto{\pgfqpoint{5.082713in}{2.404865in}}%
\pgfpathlineto{\pgfqpoint{5.090262in}{2.411099in}}%
\pgfpathlineto{\pgfqpoint{5.097804in}{2.417350in}}%
\pgfpathlineto{\pgfqpoint{5.105342in}{2.423624in}}%
\pgfpathlineto{\pgfqpoint{5.112874in}{2.429924in}}%
\pgfpathlineto{\pgfqpoint{5.098783in}{2.428885in}}%
\pgfpathlineto{\pgfqpoint{5.084702in}{2.427916in}}%
\pgfpathlineto{\pgfqpoint{5.070632in}{2.427016in}}%
\pgfpathlineto{\pgfqpoint{5.056572in}{2.426186in}}%
\pgfpathlineto{\pgfqpoint{5.049022in}{2.419629in}}%
\pgfpathlineto{\pgfqpoint{5.041467in}{2.413103in}}%
\pgfpathlineto{\pgfqpoint{5.033906in}{2.406604in}}%
\pgfpathlineto{\pgfqpoint{5.026340in}{2.400128in}}%
\pgfpathclose%
\pgfusepath{fill}%
\end{pgfscope}%
\begin{pgfscope}%
\pgfpathrectangle{\pgfqpoint{1.150000in}{0.150000in}}{\pgfqpoint{5.700000in}{5.700000in}}%
\pgfusepath{clip}%
\pgfsetbuttcap%
\pgfsetroundjoin%
\definecolor{currentfill}{rgb}{0.283072,0.130895,0.449241}%
\pgfsetfillcolor{currentfill}%
\pgfsetfillopacity{0.700000}%
\pgfsetlinewidth{0.000000pt}%
\definecolor{currentstroke}{rgb}{0.000000,0.000000,0.000000}%
\pgfsetstrokecolor{currentstroke}%
\pgfsetdash{}{0pt}%
\pgfpathmoveto{\pgfqpoint{2.453734in}{2.164672in}}%
\pgfpathlineto{\pgfqpoint{2.467340in}{2.154241in}}%
\pgfpathlineto{\pgfqpoint{2.480945in}{2.143932in}}%
\pgfpathlineto{\pgfqpoint{2.494549in}{2.133744in}}%
\pgfpathlineto{\pgfqpoint{2.508153in}{2.123676in}}%
\pgfpathlineto{\pgfqpoint{2.516774in}{2.128870in}}%
\pgfpathlineto{\pgfqpoint{2.525384in}{2.134191in}}%
\pgfpathlineto{\pgfqpoint{2.533984in}{2.139636in}}%
\pgfpathlineto{\pgfqpoint{2.542572in}{2.145201in}}%
\pgfpathlineto{\pgfqpoint{2.528993in}{2.155059in}}%
\pgfpathlineto{\pgfqpoint{2.515414in}{2.165038in}}%
\pgfpathlineto{\pgfqpoint{2.501834in}{2.175137in}}%
\pgfpathlineto{\pgfqpoint{2.488254in}{2.185358in}}%
\pgfpathlineto{\pgfqpoint{2.479641in}{2.179995in}}%
\pgfpathlineto{\pgfqpoint{2.471017in}{2.174758in}}%
\pgfpathlineto{\pgfqpoint{2.462381in}{2.169649in}}%
\pgfpathlineto{\pgfqpoint{2.453734in}{2.164672in}}%
\pgfpathclose%
\pgfusepath{fill}%
\end{pgfscope}%
\begin{pgfscope}%
\pgfpathrectangle{\pgfqpoint{1.150000in}{0.150000in}}{\pgfqpoint{5.700000in}{5.700000in}}%
\pgfusepath{clip}%
\pgfsetbuttcap%
\pgfsetroundjoin%
\definecolor{currentfill}{rgb}{0.272594,0.025563,0.353093}%
\pgfsetfillcolor{currentfill}%
\pgfsetfillopacity{0.700000}%
\pgfsetlinewidth{0.000000pt}%
\definecolor{currentstroke}{rgb}{0.000000,0.000000,0.000000}%
\pgfsetstrokecolor{currentstroke}%
\pgfsetdash{}{0pt}%
\pgfpathmoveto{\pgfqpoint{3.132167in}{1.956168in}}%
\pgfpathlineto{\pgfqpoint{3.145752in}{1.950679in}}%
\pgfpathlineto{\pgfqpoint{3.159341in}{1.945283in}}%
\pgfpathlineto{\pgfqpoint{3.172933in}{1.939979in}}%
\pgfpathlineto{\pgfqpoint{3.186530in}{1.934767in}}%
\pgfpathlineto{\pgfqpoint{3.194813in}{1.942995in}}%
\pgfpathlineto{\pgfqpoint{3.203090in}{1.951261in}}%
\pgfpathlineto{\pgfqpoint{3.211359in}{1.959563in}}%
\pgfpathlineto{\pgfqpoint{3.219622in}{1.967900in}}%
\pgfpathlineto{\pgfqpoint{3.206041in}{1.972989in}}%
\pgfpathlineto{\pgfqpoint{3.192464in}{1.978170in}}%
\pgfpathlineto{\pgfqpoint{3.178890in}{1.983442in}}%
\pgfpathlineto{\pgfqpoint{3.165321in}{1.988808in}}%
\pgfpathlineto{\pgfqpoint{3.157043in}{1.980586in}}%
\pgfpathlineto{\pgfqpoint{3.148758in}{1.972405in}}%
\pgfpathlineto{\pgfqpoint{3.140466in}{1.964265in}}%
\pgfpathlineto{\pgfqpoint{3.132167in}{1.956168in}}%
\pgfpathclose%
\pgfusepath{fill}%
\end{pgfscope}%
\begin{pgfscope}%
\pgfpathrectangle{\pgfqpoint{1.150000in}{0.150000in}}{\pgfqpoint{5.700000in}{5.700000in}}%
\pgfusepath{clip}%
\pgfsetbuttcap%
\pgfsetroundjoin%
\definecolor{currentfill}{rgb}{0.276022,0.044167,0.370164}%
\pgfsetfillcolor{currentfill}%
\pgfsetfillopacity{0.700000}%
\pgfsetlinewidth{0.000000pt}%
\definecolor{currentstroke}{rgb}{0.000000,0.000000,0.000000}%
\pgfsetstrokecolor{currentstroke}%
\pgfsetdash{}{0pt}%
\pgfpathmoveto{\pgfqpoint{2.848031in}{1.998957in}}%
\pgfpathlineto{\pgfqpoint{2.861605in}{1.991601in}}%
\pgfpathlineto{\pgfqpoint{2.875181in}{1.984347in}}%
\pgfpathlineto{\pgfqpoint{2.888760in}{1.977194in}}%
\pgfpathlineto{\pgfqpoint{2.902340in}{1.970142in}}%
\pgfpathlineto{\pgfqpoint{2.910752in}{1.977312in}}%
\pgfpathlineto{\pgfqpoint{2.919156in}{1.984556in}}%
\pgfpathlineto{\pgfqpoint{2.927552in}{1.991872in}}%
\pgfpathlineto{\pgfqpoint{2.935939in}{1.999257in}}%
\pgfpathlineto{\pgfqpoint{2.922377in}{2.006144in}}%
\pgfpathlineto{\pgfqpoint{2.908818in}{2.013131in}}%
\pgfpathlineto{\pgfqpoint{2.895260in}{2.020220in}}%
\pgfpathlineto{\pgfqpoint{2.881705in}{2.027411in}}%
\pgfpathlineto{\pgfqpoint{2.873299in}{2.020183in}}%
\pgfpathlineto{\pgfqpoint{2.864885in}{2.013030in}}%
\pgfpathlineto{\pgfqpoint{2.856462in}{2.005954in}}%
\pgfpathlineto{\pgfqpoint{2.848031in}{1.998957in}}%
\pgfpathclose%
\pgfusepath{fill}%
\end{pgfscope}%
\begin{pgfscope}%
\pgfpathrectangle{\pgfqpoint{1.150000in}{0.150000in}}{\pgfqpoint{5.700000in}{5.700000in}}%
\pgfusepath{clip}%
\pgfsetbuttcap%
\pgfsetroundjoin%
\definecolor{currentfill}{rgb}{0.283229,0.120777,0.440584}%
\pgfsetfillcolor{currentfill}%
\pgfsetfillopacity{0.700000}%
\pgfsetlinewidth{0.000000pt}%
\definecolor{currentstroke}{rgb}{0.000000,0.000000,0.000000}%
\pgfsetstrokecolor{currentstroke}%
\pgfsetdash{}{0pt}%
\pgfpathmoveto{\pgfqpoint{4.134443in}{2.116360in}}%
\pgfpathlineto{\pgfqpoint{4.148233in}{2.115561in}}%
\pgfpathlineto{\pgfqpoint{4.162031in}{2.114837in}}%
\pgfpathlineto{\pgfqpoint{4.175837in}{2.114190in}}%
\pgfpathlineto{\pgfqpoint{4.189652in}{2.113617in}}%
\pgfpathlineto{\pgfqpoint{4.197567in}{2.122128in}}%
\pgfpathlineto{\pgfqpoint{4.205476in}{2.130606in}}%
\pgfpathlineto{\pgfqpoint{4.213380in}{2.139054in}}%
\pgfpathlineto{\pgfqpoint{4.221277in}{2.147472in}}%
\pgfpathlineto{\pgfqpoint{4.207474in}{2.148107in}}%
\pgfpathlineto{\pgfqpoint{4.193679in}{2.148817in}}%
\pgfpathlineto{\pgfqpoint{4.179892in}{2.149602in}}%
\pgfpathlineto{\pgfqpoint{4.166113in}{2.150463in}}%
\pgfpathlineto{\pgfqpoint{4.158204in}{2.141975in}}%
\pgfpathlineto{\pgfqpoint{4.150290in}{2.133462in}}%
\pgfpathlineto{\pgfqpoint{4.142370in}{2.124925in}}%
\pgfpathlineto{\pgfqpoint{4.134443in}{2.116360in}}%
\pgfpathclose%
\pgfusepath{fill}%
\end{pgfscope}%
\begin{pgfscope}%
\pgfpathrectangle{\pgfqpoint{1.150000in}{0.150000in}}{\pgfqpoint{5.700000in}{5.700000in}}%
\pgfusepath{clip}%
\pgfsetbuttcap%
\pgfsetroundjoin%
\definecolor{currentfill}{rgb}{0.274952,0.037752,0.364543}%
\pgfsetfillcolor{currentfill}%
\pgfsetfillopacity{0.700000}%
\pgfsetlinewidth{0.000000pt}%
\definecolor{currentstroke}{rgb}{0.000000,0.000000,0.000000}%
\pgfsetstrokecolor{currentstroke}%
\pgfsetdash{}{0pt}%
\pgfpathmoveto{\pgfqpoint{3.502975in}{1.969338in}}%
\pgfpathlineto{\pgfqpoint{3.516611in}{1.965925in}}%
\pgfpathlineto{\pgfqpoint{3.530254in}{1.962597in}}%
\pgfpathlineto{\pgfqpoint{3.543902in}{1.959353in}}%
\pgfpathlineto{\pgfqpoint{3.557556in}{1.956192in}}%
\pgfpathlineto{\pgfqpoint{3.565696in}{1.965137in}}%
\pgfpathlineto{\pgfqpoint{3.573831in}{1.974081in}}%
\pgfpathlineto{\pgfqpoint{3.581959in}{1.983024in}}%
\pgfpathlineto{\pgfqpoint{3.590082in}{1.991964in}}%
\pgfpathlineto{\pgfqpoint{3.576440in}{1.995064in}}%
\pgfpathlineto{\pgfqpoint{3.562804in}{1.998246in}}%
\pgfpathlineto{\pgfqpoint{3.549174in}{2.001513in}}%
\pgfpathlineto{\pgfqpoint{3.535550in}{2.004864in}}%
\pgfpathlineto{\pgfqpoint{3.527415in}{1.995977in}}%
\pgfpathlineto{\pgfqpoint{3.519274in}{1.987093in}}%
\pgfpathlineto{\pgfqpoint{3.511127in}{1.978214in}}%
\pgfpathlineto{\pgfqpoint{3.502975in}{1.969338in}}%
\pgfpathclose%
\pgfusepath{fill}%
\end{pgfscope}%
\begin{pgfscope}%
\pgfpathrectangle{\pgfqpoint{1.150000in}{0.150000in}}{\pgfqpoint{5.700000in}{5.700000in}}%
\pgfusepath{clip}%
\pgfsetbuttcap%
\pgfsetroundjoin%
\definecolor{currentfill}{rgb}{0.220057,0.343307,0.549413}%
\pgfsetfillcolor{currentfill}%
\pgfsetfillopacity{0.700000}%
\pgfsetlinewidth{0.000000pt}%
\definecolor{currentstroke}{rgb}{0.000000,0.000000,0.000000}%
\pgfsetstrokecolor{currentstroke}%
\pgfsetdash{}{0pt}%
\pgfpathmoveto{\pgfqpoint{5.744874in}{2.610066in}}%
\pgfpathlineto{\pgfqpoint{5.759193in}{2.611409in}}%
\pgfpathlineto{\pgfqpoint{5.773523in}{2.612819in}}%
\pgfpathlineto{\pgfqpoint{5.787864in}{2.614296in}}%
\pgfpathlineto{\pgfqpoint{5.802217in}{2.615839in}}%
\pgfpathlineto{\pgfqpoint{5.809446in}{2.621077in}}%
\pgfpathlineto{\pgfqpoint{5.816674in}{2.626446in}}%
\pgfpathlineto{\pgfqpoint{5.823900in}{2.631954in}}%
\pgfpathlineto{\pgfqpoint{5.831125in}{2.637607in}}%
\pgfpathlineto{\pgfqpoint{5.816800in}{2.636459in}}%
\pgfpathlineto{\pgfqpoint{5.802486in}{2.635377in}}%
\pgfpathlineto{\pgfqpoint{5.788183in}{2.634362in}}%
\pgfpathlineto{\pgfqpoint{5.773892in}{2.633414in}}%
\pgfpathlineto{\pgfqpoint{5.766639in}{2.627359in}}%
\pgfpathlineto{\pgfqpoint{5.759385in}{2.621454in}}%
\pgfpathlineto{\pgfqpoint{5.752131in}{2.615692in}}%
\pgfpathlineto{\pgfqpoint{5.744874in}{2.610066in}}%
\pgfpathclose%
\pgfusepath{fill}%
\end{pgfscope}%
\begin{pgfscope}%
\pgfpathrectangle{\pgfqpoint{1.150000in}{0.150000in}}{\pgfqpoint{5.700000in}{5.700000in}}%
\pgfusepath{clip}%
\pgfsetbuttcap%
\pgfsetroundjoin%
\definecolor{currentfill}{rgb}{0.278012,0.180367,0.486697}%
\pgfsetfillcolor{currentfill}%
\pgfsetfillopacity{0.700000}%
\pgfsetlinewidth{0.000000pt}%
\definecolor{currentstroke}{rgb}{0.000000,0.000000,0.000000}%
\pgfsetstrokecolor{currentstroke}%
\pgfsetdash{}{0pt}%
\pgfpathmoveto{\pgfqpoint{4.537052in}{2.241908in}}%
\pgfpathlineto{\pgfqpoint{4.550968in}{2.242219in}}%
\pgfpathlineto{\pgfqpoint{4.564894in}{2.242603in}}%
\pgfpathlineto{\pgfqpoint{4.578829in}{2.243058in}}%
\pgfpathlineto{\pgfqpoint{4.592773in}{2.243586in}}%
\pgfpathlineto{\pgfqpoint{4.600533in}{2.251121in}}%
\pgfpathlineto{\pgfqpoint{4.608288in}{2.258631in}}%
\pgfpathlineto{\pgfqpoint{4.616036in}{2.266120in}}%
\pgfpathlineto{\pgfqpoint{4.623778in}{2.273591in}}%
\pgfpathlineto{\pgfqpoint{4.609847in}{2.273208in}}%
\pgfpathlineto{\pgfqpoint{4.595925in}{2.272898in}}%
\pgfpathlineto{\pgfqpoint{4.582012in}{2.272660in}}%
\pgfpathlineto{\pgfqpoint{4.568109in}{2.272494in}}%
\pgfpathlineto{\pgfqpoint{4.560353in}{2.264871in}}%
\pgfpathlineto{\pgfqpoint{4.552592in}{2.257234in}}%
\pgfpathlineto{\pgfqpoint{4.544825in}{2.249581in}}%
\pgfpathlineto{\pgfqpoint{4.537052in}{2.241908in}}%
\pgfpathclose%
\pgfusepath{fill}%
\end{pgfscope}%
\begin{pgfscope}%
\pgfpathrectangle{\pgfqpoint{1.150000in}{0.150000in}}{\pgfqpoint{5.700000in}{5.700000in}}%
\pgfusepath{clip}%
\pgfsetbuttcap%
\pgfsetroundjoin%
\definecolor{currentfill}{rgb}{0.272594,0.025563,0.353093}%
\pgfsetfillcolor{currentfill}%
\pgfsetfillopacity{0.700000}%
\pgfsetlinewidth{0.000000pt}%
\definecolor{currentstroke}{rgb}{0.000000,0.000000,0.000000}%
\pgfsetstrokecolor{currentstroke}%
\pgfsetdash{}{0pt}%
\pgfpathmoveto{\pgfqpoint{3.273989in}{1.948452in}}%
\pgfpathlineto{\pgfqpoint{3.287592in}{1.943814in}}%
\pgfpathlineto{\pgfqpoint{3.301199in}{1.939266in}}%
\pgfpathlineto{\pgfqpoint{3.314811in}{1.934805in}}%
\pgfpathlineto{\pgfqpoint{3.328428in}{1.930433in}}%
\pgfpathlineto{\pgfqpoint{3.336656in}{1.939025in}}%
\pgfpathlineto{\pgfqpoint{3.344877in}{1.947638in}}%
\pgfpathlineto{\pgfqpoint{3.353092in}{1.956272in}}%
\pgfpathlineto{\pgfqpoint{3.361301in}{1.964925in}}%
\pgfpathlineto{\pgfqpoint{3.347698in}{1.969195in}}%
\pgfpathlineto{\pgfqpoint{3.334100in}{1.973552in}}%
\pgfpathlineto{\pgfqpoint{3.320507in}{1.977998in}}%
\pgfpathlineto{\pgfqpoint{3.306918in}{1.982533in}}%
\pgfpathlineto{\pgfqpoint{3.298696in}{1.973974in}}%
\pgfpathlineto{\pgfqpoint{3.290467in}{1.965441in}}%
\pgfpathlineto{\pgfqpoint{3.282231in}{1.956933in}}%
\pgfpathlineto{\pgfqpoint{3.273989in}{1.948452in}}%
\pgfpathclose%
\pgfusepath{fill}%
\end{pgfscope}%
\begin{pgfscope}%
\pgfpathrectangle{\pgfqpoint{1.150000in}{0.150000in}}{\pgfqpoint{5.700000in}{5.700000in}}%
\pgfusepath{clip}%
\pgfsetbuttcap%
\pgfsetroundjoin%
\definecolor{currentfill}{rgb}{0.241237,0.296485,0.539709}%
\pgfsetfillcolor{currentfill}%
\pgfsetfillopacity{0.700000}%
\pgfsetlinewidth{0.000000pt}%
\definecolor{currentstroke}{rgb}{0.000000,0.000000,0.000000}%
\pgfsetstrokecolor{currentstroke}%
\pgfsetdash{}{0pt}%
\pgfpathmoveto{\pgfqpoint{5.342400in}{2.492962in}}%
\pgfpathlineto{\pgfqpoint{5.356590in}{2.494332in}}%
\pgfpathlineto{\pgfqpoint{5.370792in}{2.495770in}}%
\pgfpathlineto{\pgfqpoint{5.385004in}{2.497276in}}%
\pgfpathlineto{\pgfqpoint{5.399227in}{2.498851in}}%
\pgfpathlineto{\pgfqpoint{5.406634in}{2.504395in}}%
\pgfpathlineto{\pgfqpoint{5.414036in}{2.509996in}}%
\pgfpathlineto{\pgfqpoint{5.421433in}{2.515660in}}%
\pgfpathlineto{\pgfqpoint{5.428827in}{2.521392in}}%
\pgfpathlineto{\pgfqpoint{5.414625in}{2.520131in}}%
\pgfpathlineto{\pgfqpoint{5.400434in}{2.518937in}}%
\pgfpathlineto{\pgfqpoint{5.386255in}{2.517811in}}%
\pgfpathlineto{\pgfqpoint{5.372086in}{2.516753in}}%
\pgfpathlineto{\pgfqpoint{5.364671in}{2.510701in}}%
\pgfpathlineto{\pgfqpoint{5.357252in}{2.504722in}}%
\pgfpathlineto{\pgfqpoint{5.349828in}{2.498811in}}%
\pgfpathlineto{\pgfqpoint{5.342400in}{2.492962in}}%
\pgfpathclose%
\pgfusepath{fill}%
\end{pgfscope}%
\begin{pgfscope}%
\pgfpathrectangle{\pgfqpoint{1.150000in}{0.150000in}}{\pgfqpoint{5.700000in}{5.700000in}}%
\pgfusepath{clip}%
\pgfsetbuttcap%
\pgfsetroundjoin%
\definecolor{currentfill}{rgb}{0.262138,0.242286,0.520837}%
\pgfsetfillcolor{currentfill}%
\pgfsetfillopacity{0.700000}%
\pgfsetlinewidth{0.000000pt}%
\definecolor{currentstroke}{rgb}{0.000000,0.000000,0.000000}%
\pgfsetstrokecolor{currentstroke}%
\pgfsetdash{}{0pt}%
\pgfpathmoveto{\pgfqpoint{4.939743in}{2.369826in}}%
\pgfpathlineto{\pgfqpoint{4.953796in}{2.370855in}}%
\pgfpathlineto{\pgfqpoint{4.967860in}{2.371955in}}%
\pgfpathlineto{\pgfqpoint{4.981934in}{2.373124in}}%
\pgfpathlineto{\pgfqpoint{4.996018in}{2.374364in}}%
\pgfpathlineto{\pgfqpoint{5.003607in}{2.380792in}}%
\pgfpathlineto{\pgfqpoint{5.011191in}{2.387227in}}%
\pgfpathlineto{\pgfqpoint{5.018768in}{2.393671in}}%
\pgfpathlineto{\pgfqpoint{5.026340in}{2.400128in}}%
\pgfpathlineto{\pgfqpoint{5.012273in}{2.399118in}}%
\pgfpathlineto{\pgfqpoint{4.998215in}{2.398178in}}%
\pgfpathlineto{\pgfqpoint{4.984168in}{2.397307in}}%
\pgfpathlineto{\pgfqpoint{4.970132in}{2.396507in}}%
\pgfpathlineto{\pgfqpoint{4.962543in}{2.389813in}}%
\pgfpathlineto{\pgfqpoint{4.954949in}{2.383137in}}%
\pgfpathlineto{\pgfqpoint{4.947349in}{2.376476in}}%
\pgfpathlineto{\pgfqpoint{4.939743in}{2.369826in}}%
\pgfpathclose%
\pgfusepath{fill}%
\end{pgfscope}%
\begin{pgfscope}%
\pgfpathrectangle{\pgfqpoint{1.150000in}{0.150000in}}{\pgfqpoint{5.700000in}{5.700000in}}%
\pgfusepath{clip}%
\pgfsetbuttcap%
\pgfsetroundjoin%
\definecolor{currentfill}{rgb}{0.271828,0.209303,0.504434}%
\pgfsetfillcolor{currentfill}%
\pgfsetfillopacity{0.700000}%
\pgfsetlinewidth{0.000000pt}%
\definecolor{currentstroke}{rgb}{0.000000,0.000000,0.000000}%
\pgfsetstrokecolor{currentstroke}%
\pgfsetdash{}{0pt}%
\pgfpathmoveto{\pgfqpoint{2.200641in}{2.334206in}}%
\pgfpathlineto{\pgfqpoint{2.214310in}{2.321442in}}%
\pgfpathlineto{\pgfqpoint{2.227976in}{2.308819in}}%
\pgfpathlineto{\pgfqpoint{2.241638in}{2.296334in}}%
\pgfpathlineto{\pgfqpoint{2.255298in}{2.283988in}}%
\pgfpathlineto{\pgfqpoint{2.264079in}{2.287686in}}%
\pgfpathlineto{\pgfqpoint{2.272847in}{2.291544in}}%
\pgfpathlineto{\pgfqpoint{2.281601in}{2.295559in}}%
\pgfpathlineto{\pgfqpoint{2.290342in}{2.299728in}}%
\pgfpathlineto{\pgfqpoint{2.276712in}{2.311841in}}%
\pgfpathlineto{\pgfqpoint{2.263079in}{2.324090in}}%
\pgfpathlineto{\pgfqpoint{2.249443in}{2.336479in}}%
\pgfpathlineto{\pgfqpoint{2.235804in}{2.349008in}}%
\pgfpathlineto{\pgfqpoint{2.227034in}{2.345065in}}%
\pgfpathlineto{\pgfqpoint{2.218250in}{2.341282in}}%
\pgfpathlineto{\pgfqpoint{2.209452in}{2.337661in}}%
\pgfpathlineto{\pgfqpoint{2.200641in}{2.334206in}}%
\pgfpathclose%
\pgfusepath{fill}%
\end{pgfscope}%
\begin{pgfscope}%
\pgfpathrectangle{\pgfqpoint{1.150000in}{0.150000in}}{\pgfqpoint{5.700000in}{5.700000in}}%
\pgfusepath{clip}%
\pgfsetbuttcap%
\pgfsetroundjoin%
\definecolor{currentfill}{rgb}{0.278791,0.062145,0.386592}%
\pgfsetfillcolor{currentfill}%
\pgfsetfillopacity{0.700000}%
\pgfsetlinewidth{0.000000pt}%
\definecolor{currentstroke}{rgb}{0.000000,0.000000,0.000000}%
\pgfsetstrokecolor{currentstroke}%
\pgfsetdash{}{0pt}%
\pgfpathmoveto{\pgfqpoint{3.731778in}{2.006207in}}%
\pgfpathlineto{\pgfqpoint{3.745465in}{2.003886in}}%
\pgfpathlineto{\pgfqpoint{3.759158in}{2.001646in}}%
\pgfpathlineto{\pgfqpoint{3.772858in}{1.999487in}}%
\pgfpathlineto{\pgfqpoint{3.786564in}{1.997407in}}%
\pgfpathlineto{\pgfqpoint{3.794625in}{2.006419in}}%
\pgfpathlineto{\pgfqpoint{3.802679in}{2.015413in}}%
\pgfpathlineto{\pgfqpoint{3.810728in}{2.024389in}}%
\pgfpathlineto{\pgfqpoint{3.818771in}{2.033347in}}%
\pgfpathlineto{\pgfqpoint{3.805076in}{2.035406in}}%
\pgfpathlineto{\pgfqpoint{3.791387in}{2.037545in}}%
\pgfpathlineto{\pgfqpoint{3.777705in}{2.039764in}}%
\pgfpathlineto{\pgfqpoint{3.764030in}{2.042064in}}%
\pgfpathlineto{\pgfqpoint{3.755975in}{2.033119in}}%
\pgfpathlineto{\pgfqpoint{3.747915in}{2.024161in}}%
\pgfpathlineto{\pgfqpoint{3.739850in}{2.015191in}}%
\pgfpathlineto{\pgfqpoint{3.731778in}{2.006207in}}%
\pgfpathclose%
\pgfusepath{fill}%
\end{pgfscope}%
\begin{pgfscope}%
\pgfpathrectangle{\pgfqpoint{1.150000in}{0.150000in}}{\pgfqpoint{5.700000in}{5.700000in}}%
\pgfusepath{clip}%
\pgfsetbuttcap%
\pgfsetroundjoin%
\definecolor{currentfill}{rgb}{0.279566,0.067836,0.391917}%
\pgfsetfillcolor{currentfill}%
\pgfsetfillopacity{0.700000}%
\pgfsetlinewidth{0.000000pt}%
\definecolor{currentstroke}{rgb}{0.000000,0.000000,0.000000}%
\pgfsetstrokecolor{currentstroke}%
\pgfsetdash{}{0pt}%
\pgfpathmoveto{\pgfqpoint{2.705520in}{2.035900in}}%
\pgfpathlineto{\pgfqpoint{2.719103in}{2.027517in}}%
\pgfpathlineto{\pgfqpoint{2.732687in}{2.019241in}}%
\pgfpathlineto{\pgfqpoint{2.746272in}{2.011072in}}%
\pgfpathlineto{\pgfqpoint{2.759858in}{2.003009in}}%
\pgfpathlineto{\pgfqpoint{2.768346in}{2.009472in}}%
\pgfpathlineto{\pgfqpoint{2.776824in}{2.016029in}}%
\pgfpathlineto{\pgfqpoint{2.785293in}{2.022677in}}%
\pgfpathlineto{\pgfqpoint{2.793753in}{2.029415in}}%
\pgfpathlineto{\pgfqpoint{2.780188in}{2.037291in}}%
\pgfpathlineto{\pgfqpoint{2.766624in}{2.045273in}}%
\pgfpathlineto{\pgfqpoint{2.753061in}{2.053363in}}%
\pgfpathlineto{\pgfqpoint{2.739500in}{2.061560in}}%
\pgfpathlineto{\pgfqpoint{2.731019in}{2.055001in}}%
\pgfpathlineto{\pgfqpoint{2.722529in}{2.048537in}}%
\pgfpathlineto{\pgfqpoint{2.714029in}{2.042169in}}%
\pgfpathlineto{\pgfqpoint{2.705520in}{2.035900in}}%
\pgfpathclose%
\pgfusepath{fill}%
\end{pgfscope}%
\begin{pgfscope}%
\pgfpathrectangle{\pgfqpoint{1.150000in}{0.150000in}}{\pgfqpoint{5.700000in}{5.700000in}}%
\pgfusepath{clip}%
\pgfsetbuttcap%
\pgfsetroundjoin%
\definecolor{currentfill}{rgb}{0.282910,0.105393,0.426902}%
\pgfsetfillcolor{currentfill}%
\pgfsetfillopacity{0.700000}%
\pgfsetlinewidth{0.000000pt}%
\definecolor{currentstroke}{rgb}{0.000000,0.000000,0.000000}%
\pgfsetstrokecolor{currentstroke}%
\pgfsetdash{}{0pt}%
\pgfpathmoveto{\pgfqpoint{4.047558in}{2.085600in}}%
\pgfpathlineto{\pgfqpoint{4.061327in}{2.084537in}}%
\pgfpathlineto{\pgfqpoint{4.075104in}{2.083550in}}%
\pgfpathlineto{\pgfqpoint{4.088889in}{2.082640in}}%
\pgfpathlineto{\pgfqpoint{4.102682in}{2.081807in}}%
\pgfpathlineto{\pgfqpoint{4.110631in}{2.090492in}}%
\pgfpathlineto{\pgfqpoint{4.118574in}{2.099145in}}%
\pgfpathlineto{\pgfqpoint{4.126512in}{2.107767in}}%
\pgfpathlineto{\pgfqpoint{4.134443in}{2.116360in}}%
\pgfpathlineto{\pgfqpoint{4.120662in}{2.117235in}}%
\pgfpathlineto{\pgfqpoint{4.106888in}{2.118187in}}%
\pgfpathlineto{\pgfqpoint{4.093122in}{2.119214in}}%
\pgfpathlineto{\pgfqpoint{4.079363in}{2.120319in}}%
\pgfpathlineto{\pgfqpoint{4.071421in}{2.111677in}}%
\pgfpathlineto{\pgfqpoint{4.063472in}{2.103011in}}%
\pgfpathlineto{\pgfqpoint{4.055518in}{2.094319in}}%
\pgfpathlineto{\pgfqpoint{4.047558in}{2.085600in}}%
\pgfpathclose%
\pgfusepath{fill}%
\end{pgfscope}%
\begin{pgfscope}%
\pgfpathrectangle{\pgfqpoint{1.150000in}{0.150000in}}{\pgfqpoint{5.700000in}{5.700000in}}%
\pgfusepath{clip}%
\pgfsetbuttcap%
\pgfsetroundjoin%
\definecolor{currentfill}{rgb}{0.283091,0.110553,0.431554}%
\pgfsetfillcolor{currentfill}%
\pgfsetfillopacity{0.700000}%
\pgfsetlinewidth{0.000000pt}%
\definecolor{currentstroke}{rgb}{0.000000,0.000000,0.000000}%
\pgfsetstrokecolor{currentstroke}%
\pgfsetdash{}{0pt}%
\pgfpathmoveto{\pgfqpoint{2.508153in}{2.123676in}}%
\pgfpathlineto{\pgfqpoint{2.521756in}{2.113726in}}%
\pgfpathlineto{\pgfqpoint{2.535358in}{2.103895in}}%
\pgfpathlineto{\pgfqpoint{2.548960in}{2.094180in}}%
\pgfpathlineto{\pgfqpoint{2.562562in}{2.084582in}}%
\pgfpathlineto{\pgfqpoint{2.571159in}{2.089994in}}%
\pgfpathlineto{\pgfqpoint{2.579745in}{2.095527in}}%
\pgfpathlineto{\pgfqpoint{2.588320in}{2.101177in}}%
\pgfpathlineto{\pgfqpoint{2.596884in}{2.106943in}}%
\pgfpathlineto{\pgfqpoint{2.583307in}{2.116333in}}%
\pgfpathlineto{\pgfqpoint{2.569729in}{2.125838in}}%
\pgfpathlineto{\pgfqpoint{2.556150in}{2.135460in}}%
\pgfpathlineto{\pgfqpoint{2.542572in}{2.145201in}}%
\pgfpathlineto{\pgfqpoint{2.533984in}{2.139636in}}%
\pgfpathlineto{\pgfqpoint{2.525384in}{2.134191in}}%
\pgfpathlineto{\pgfqpoint{2.516774in}{2.128870in}}%
\pgfpathlineto{\pgfqpoint{2.508153in}{2.123676in}}%
\pgfpathclose%
\pgfusepath{fill}%
\end{pgfscope}%
\begin{pgfscope}%
\pgfpathrectangle{\pgfqpoint{1.150000in}{0.150000in}}{\pgfqpoint{5.700000in}{5.700000in}}%
\pgfusepath{clip}%
\pgfsetbuttcap%
\pgfsetroundjoin%
\definecolor{currentfill}{rgb}{0.280255,0.165693,0.476498}%
\pgfsetfillcolor{currentfill}%
\pgfsetfillopacity{0.700000}%
\pgfsetlinewidth{0.000000pt}%
\definecolor{currentstroke}{rgb}{0.000000,0.000000,0.000000}%
\pgfsetstrokecolor{currentstroke}%
\pgfsetdash{}{0pt}%
\pgfpathmoveto{\pgfqpoint{4.450274in}{2.209962in}}%
\pgfpathlineto{\pgfqpoint{4.464167in}{2.210105in}}%
\pgfpathlineto{\pgfqpoint{4.478068in}{2.210322in}}%
\pgfpathlineto{\pgfqpoint{4.491978in}{2.210612in}}%
\pgfpathlineto{\pgfqpoint{4.505898in}{2.210975in}}%
\pgfpathlineto{\pgfqpoint{4.513696in}{2.218750in}}%
\pgfpathlineto{\pgfqpoint{4.521487in}{2.226496in}}%
\pgfpathlineto{\pgfqpoint{4.529272in}{2.234214in}}%
\pgfpathlineto{\pgfqpoint{4.537052in}{2.241908in}}%
\pgfpathlineto{\pgfqpoint{4.523144in}{2.241670in}}%
\pgfpathlineto{\pgfqpoint{4.509246in}{2.241505in}}%
\pgfpathlineto{\pgfqpoint{4.495357in}{2.241412in}}%
\pgfpathlineto{\pgfqpoint{4.481477in}{2.241393in}}%
\pgfpathlineto{\pgfqpoint{4.473686in}{2.233567in}}%
\pgfpathlineto{\pgfqpoint{4.465888in}{2.225721in}}%
\pgfpathlineto{\pgfqpoint{4.458084in}{2.217853in}}%
\pgfpathlineto{\pgfqpoint{4.450274in}{2.209962in}}%
\pgfpathclose%
\pgfusepath{fill}%
\end{pgfscope}%
\begin{pgfscope}%
\pgfpathrectangle{\pgfqpoint{1.150000in}{0.150000in}}{\pgfqpoint{5.700000in}{5.700000in}}%
\pgfusepath{clip}%
\pgfsetbuttcap%
\pgfsetroundjoin%
\definecolor{currentfill}{rgb}{0.223925,0.334994,0.548053}%
\pgfsetfillcolor{currentfill}%
\pgfsetfillopacity{0.700000}%
\pgfsetlinewidth{0.000000pt}%
\definecolor{currentstroke}{rgb}{0.000000,0.000000,0.000000}%
\pgfsetstrokecolor{currentstroke}%
\pgfsetdash{}{0pt}%
\pgfpathmoveto{\pgfqpoint{5.658563in}{2.582584in}}%
\pgfpathlineto{\pgfqpoint{5.672862in}{2.584035in}}%
\pgfpathlineto{\pgfqpoint{5.687173in}{2.585553in}}%
\pgfpathlineto{\pgfqpoint{5.701495in}{2.587138in}}%
\pgfpathlineto{\pgfqpoint{5.715828in}{2.588789in}}%
\pgfpathlineto{\pgfqpoint{5.723093in}{2.593938in}}%
\pgfpathlineto{\pgfqpoint{5.730356in}{2.599196in}}%
\pgfpathlineto{\pgfqpoint{5.737616in}{2.604570in}}%
\pgfpathlineto{\pgfqpoint{5.744874in}{2.610066in}}%
\pgfpathlineto{\pgfqpoint{5.730567in}{2.608790in}}%
\pgfpathlineto{\pgfqpoint{5.716271in}{2.607580in}}%
\pgfpathlineto{\pgfqpoint{5.701987in}{2.606436in}}%
\pgfpathlineto{\pgfqpoint{5.687714in}{2.605360in}}%
\pgfpathlineto{\pgfqpoint{5.680429in}{2.599481in}}%
\pgfpathlineto{\pgfqpoint{5.673143in}{2.593730in}}%
\pgfpathlineto{\pgfqpoint{5.665854in}{2.588100in}}%
\pgfpathlineto{\pgfqpoint{5.658563in}{2.582584in}}%
\pgfpathclose%
\pgfusepath{fill}%
\end{pgfscope}%
\begin{pgfscope}%
\pgfpathrectangle{\pgfqpoint{1.150000in}{0.150000in}}{\pgfqpoint{5.700000in}{5.700000in}}%
\pgfusepath{clip}%
\pgfsetbuttcap%
\pgfsetroundjoin%
\definecolor{currentfill}{rgb}{0.266580,0.228262,0.514349}%
\pgfsetfillcolor{currentfill}%
\pgfsetfillopacity{0.700000}%
\pgfsetlinewidth{0.000000pt}%
\definecolor{currentstroke}{rgb}{0.000000,0.000000,0.000000}%
\pgfsetstrokecolor{currentstroke}%
\pgfsetdash{}{0pt}%
\pgfpathmoveto{\pgfqpoint{4.853085in}{2.339002in}}%
\pgfpathlineto{\pgfqpoint{4.867114in}{2.339958in}}%
\pgfpathlineto{\pgfqpoint{4.881153in}{2.340985in}}%
\pgfpathlineto{\pgfqpoint{4.895202in}{2.342082in}}%
\pgfpathlineto{\pgfqpoint{4.909261in}{2.343250in}}%
\pgfpathlineto{\pgfqpoint{4.916891in}{2.349898in}}%
\pgfpathlineto{\pgfqpoint{4.924514in}{2.356541in}}%
\pgfpathlineto{\pgfqpoint{4.932131in}{2.363182in}}%
\pgfpathlineto{\pgfqpoint{4.939743in}{2.369826in}}%
\pgfpathlineto{\pgfqpoint{4.925700in}{2.368867in}}%
\pgfpathlineto{\pgfqpoint{4.911667in}{2.367978in}}%
\pgfpathlineto{\pgfqpoint{4.897644in}{2.367159in}}%
\pgfpathlineto{\pgfqpoint{4.883630in}{2.366411in}}%
\pgfpathlineto{\pgfqpoint{4.876003in}{2.359551in}}%
\pgfpathlineto{\pgfqpoint{4.868370in}{2.352699in}}%
\pgfpathlineto{\pgfqpoint{4.860730in}{2.345850in}}%
\pgfpathlineto{\pgfqpoint{4.853085in}{2.339002in}}%
\pgfpathclose%
\pgfusepath{fill}%
\end{pgfscope}%
\begin{pgfscope}%
\pgfpathrectangle{\pgfqpoint{1.150000in}{0.150000in}}{\pgfqpoint{5.700000in}{5.700000in}}%
\pgfusepath{clip}%
\pgfsetbuttcap%
\pgfsetroundjoin%
\definecolor{currentfill}{rgb}{0.276194,0.190074,0.493001}%
\pgfsetfillcolor{currentfill}%
\pgfsetfillopacity{0.700000}%
\pgfsetlinewidth{0.000000pt}%
\definecolor{currentstroke}{rgb}{0.000000,0.000000,0.000000}%
\pgfsetstrokecolor{currentstroke}%
\pgfsetdash{}{0pt}%
\pgfpathmoveto{\pgfqpoint{2.255298in}{2.283988in}}%
\pgfpathlineto{\pgfqpoint{2.268954in}{2.271778in}}%
\pgfpathlineto{\pgfqpoint{2.282608in}{2.259703in}}%
\pgfpathlineto{\pgfqpoint{2.296259in}{2.247763in}}%
\pgfpathlineto{\pgfqpoint{2.309908in}{2.235955in}}%
\pgfpathlineto{\pgfqpoint{2.318660in}{2.239894in}}%
\pgfpathlineto{\pgfqpoint{2.327399in}{2.243989in}}%
\pgfpathlineto{\pgfqpoint{2.336124in}{2.248235in}}%
\pgfpathlineto{\pgfqpoint{2.344837in}{2.252629in}}%
\pgfpathlineto{\pgfqpoint{2.331217in}{2.264204in}}%
\pgfpathlineto{\pgfqpoint{2.317595in}{2.275911in}}%
\pgfpathlineto{\pgfqpoint{2.303970in}{2.287752in}}%
\pgfpathlineto{\pgfqpoint{2.290342in}{2.299728in}}%
\pgfpathlineto{\pgfqpoint{2.281601in}{2.295559in}}%
\pgfpathlineto{\pgfqpoint{2.272847in}{2.291544in}}%
\pgfpathlineto{\pgfqpoint{2.264079in}{2.287686in}}%
\pgfpathlineto{\pgfqpoint{2.255298in}{2.283988in}}%
\pgfpathclose%
\pgfusepath{fill}%
\end{pgfscope}%
\begin{pgfscope}%
\pgfpathrectangle{\pgfqpoint{1.150000in}{0.150000in}}{\pgfqpoint{5.700000in}{5.700000in}}%
\pgfusepath{clip}%
\pgfsetbuttcap%
\pgfsetroundjoin%
\definecolor{currentfill}{rgb}{0.246811,0.283237,0.535941}%
\pgfsetfillcolor{currentfill}%
\pgfsetfillopacity{0.700000}%
\pgfsetlinewidth{0.000000pt}%
\definecolor{currentstroke}{rgb}{0.000000,0.000000,0.000000}%
\pgfsetstrokecolor{currentstroke}%
\pgfsetdash{}{0pt}%
\pgfpathmoveto{\pgfqpoint{5.255905in}{2.464115in}}%
\pgfpathlineto{\pgfqpoint{5.270073in}{2.465504in}}%
\pgfpathlineto{\pgfqpoint{5.284251in}{2.466961in}}%
\pgfpathlineto{\pgfqpoint{5.298440in}{2.468486in}}%
\pgfpathlineto{\pgfqpoint{5.312640in}{2.470080in}}%
\pgfpathlineto{\pgfqpoint{5.320088in}{2.475734in}}%
\pgfpathlineto{\pgfqpoint{5.327530in}{2.481429in}}%
\pgfpathlineto{\pgfqpoint{5.334967in}{2.487169in}}%
\pgfpathlineto{\pgfqpoint{5.342400in}{2.492962in}}%
\pgfpathlineto{\pgfqpoint{5.328221in}{2.491660in}}%
\pgfpathlineto{\pgfqpoint{5.314052in}{2.490427in}}%
\pgfpathlineto{\pgfqpoint{5.299894in}{2.489262in}}%
\pgfpathlineto{\pgfqpoint{5.285747in}{2.488165in}}%
\pgfpathlineto{\pgfqpoint{5.278294in}{2.482073in}}%
\pgfpathlineto{\pgfqpoint{5.270836in}{2.476038in}}%
\pgfpathlineto{\pgfqpoint{5.263373in}{2.470054in}}%
\pgfpathlineto{\pgfqpoint{5.255905in}{2.464115in}}%
\pgfpathclose%
\pgfusepath{fill}%
\end{pgfscope}%
\begin{pgfscope}%
\pgfpathrectangle{\pgfqpoint{1.150000in}{0.150000in}}{\pgfqpoint{5.700000in}{5.700000in}}%
\pgfusepath{clip}%
\pgfsetbuttcap%
\pgfsetroundjoin%
\definecolor{currentfill}{rgb}{0.273809,0.031497,0.358853}%
\pgfsetfillcolor{currentfill}%
\pgfsetfillopacity{0.700000}%
\pgfsetlinewidth{0.000000pt}%
\definecolor{currentstroke}{rgb}{0.000000,0.000000,0.000000}%
\pgfsetstrokecolor{currentstroke}%
\pgfsetdash{}{0pt}%
\pgfpathmoveto{\pgfqpoint{3.415761in}{1.948720in}}%
\pgfpathlineto{\pgfqpoint{3.429389in}{1.944885in}}%
\pgfpathlineto{\pgfqpoint{3.443022in}{1.941135in}}%
\pgfpathlineto{\pgfqpoint{3.456661in}{1.937471in}}%
\pgfpathlineto{\pgfqpoint{3.470304in}{1.933892in}}%
\pgfpathlineto{\pgfqpoint{3.478481in}{1.942743in}}%
\pgfpathlineto{\pgfqpoint{3.486652in}{1.951602in}}%
\pgfpathlineto{\pgfqpoint{3.494816in}{1.960467in}}%
\pgfpathlineto{\pgfqpoint{3.502975in}{1.969338in}}%
\pgfpathlineto{\pgfqpoint{3.489343in}{1.972835in}}%
\pgfpathlineto{\pgfqpoint{3.475718in}{1.976417in}}%
\pgfpathlineto{\pgfqpoint{3.462097in}{1.980084in}}%
\pgfpathlineto{\pgfqpoint{3.448482in}{1.983837in}}%
\pgfpathlineto{\pgfqpoint{3.440311in}{1.975041in}}%
\pgfpathlineto{\pgfqpoint{3.432134in}{1.966256in}}%
\pgfpathlineto{\pgfqpoint{3.423951in}{1.957481in}}%
\pgfpathlineto{\pgfqpoint{3.415761in}{1.948720in}}%
\pgfpathclose%
\pgfusepath{fill}%
\end{pgfscope}%
\begin{pgfscope}%
\pgfpathrectangle{\pgfqpoint{1.150000in}{0.150000in}}{\pgfqpoint{5.700000in}{5.700000in}}%
\pgfusepath{clip}%
\pgfsetbuttcap%
\pgfsetroundjoin%
\definecolor{currentfill}{rgb}{0.277018,0.050344,0.375715}%
\pgfsetfillcolor{currentfill}%
\pgfsetfillopacity{0.700000}%
\pgfsetlinewidth{0.000000pt}%
\definecolor{currentstroke}{rgb}{0.000000,0.000000,0.000000}%
\pgfsetstrokecolor{currentstroke}%
\pgfsetdash{}{0pt}%
\pgfpathmoveto{\pgfqpoint{3.644710in}{1.980396in}}%
\pgfpathlineto{\pgfqpoint{3.658382in}{1.977710in}}%
\pgfpathlineto{\pgfqpoint{3.672060in}{1.975105in}}%
\pgfpathlineto{\pgfqpoint{3.685745in}{1.972582in}}%
\pgfpathlineto{\pgfqpoint{3.699436in}{1.970140in}}%
\pgfpathlineto{\pgfqpoint{3.707531in}{1.979176in}}%
\pgfpathlineto{\pgfqpoint{3.715619in}{1.988199in}}%
\pgfpathlineto{\pgfqpoint{3.723702in}{1.997210in}}%
\pgfpathlineto{\pgfqpoint{3.731778in}{2.006207in}}%
\pgfpathlineto{\pgfqpoint{3.718099in}{2.008608in}}%
\pgfpathlineto{\pgfqpoint{3.704425in}{2.011090in}}%
\pgfpathlineto{\pgfqpoint{3.690758in}{2.013654in}}%
\pgfpathlineto{\pgfqpoint{3.677097in}{2.016299in}}%
\pgfpathlineto{\pgfqpoint{3.669009in}{2.007335in}}%
\pgfpathlineto{\pgfqpoint{3.660915in}{1.998363in}}%
\pgfpathlineto{\pgfqpoint{3.652815in}{1.989384in}}%
\pgfpathlineto{\pgfqpoint{3.644710in}{1.980396in}}%
\pgfpathclose%
\pgfusepath{fill}%
\end{pgfscope}%
\begin{pgfscope}%
\pgfpathrectangle{\pgfqpoint{1.150000in}{0.150000in}}{\pgfqpoint{5.700000in}{5.700000in}}%
\pgfusepath{clip}%
\pgfsetbuttcap%
\pgfsetroundjoin%
\definecolor{currentfill}{rgb}{0.272594,0.025563,0.353093}%
\pgfsetfillcolor{currentfill}%
\pgfsetfillopacity{0.700000}%
\pgfsetlinewidth{0.000000pt}%
\definecolor{currentstroke}{rgb}{0.000000,0.000000,0.000000}%
\pgfsetstrokecolor{currentstroke}%
\pgfsetdash{}{0pt}%
\pgfpathmoveto{\pgfqpoint{3.044530in}{1.947710in}}%
\pgfpathlineto{\pgfqpoint{3.058118in}{1.941702in}}%
\pgfpathlineto{\pgfqpoint{3.071708in}{1.935788in}}%
\pgfpathlineto{\pgfqpoint{3.085302in}{1.929968in}}%
\pgfpathlineto{\pgfqpoint{3.098900in}{1.924243in}}%
\pgfpathlineto{\pgfqpoint{3.107228in}{1.932151in}}%
\pgfpathlineto{\pgfqpoint{3.115548in}{1.940109in}}%
\pgfpathlineto{\pgfqpoint{3.123861in}{1.948115in}}%
\pgfpathlineto{\pgfqpoint{3.132167in}{1.956168in}}%
\pgfpathlineto{\pgfqpoint{3.118586in}{1.961749in}}%
\pgfpathlineto{\pgfqpoint{3.105008in}{1.967425in}}%
\pgfpathlineto{\pgfqpoint{3.091434in}{1.973195in}}%
\pgfpathlineto{\pgfqpoint{3.077863in}{1.979059in}}%
\pgfpathlineto{\pgfqpoint{3.069541in}{1.971143in}}%
\pgfpathlineto{\pgfqpoint{3.061212in}{1.963278in}}%
\pgfpathlineto{\pgfqpoint{3.052875in}{1.955466in}}%
\pgfpathlineto{\pgfqpoint{3.044530in}{1.947710in}}%
\pgfpathclose%
\pgfusepath{fill}%
\end{pgfscope}%
\begin{pgfscope}%
\pgfpathrectangle{\pgfqpoint{1.150000in}{0.150000in}}{\pgfqpoint{5.700000in}{5.700000in}}%
\pgfusepath{clip}%
\pgfsetbuttcap%
\pgfsetroundjoin%
\definecolor{currentfill}{rgb}{0.282327,0.094955,0.417331}%
\pgfsetfillcolor{currentfill}%
\pgfsetfillopacity{0.700000}%
\pgfsetlinewidth{0.000000pt}%
\definecolor{currentstroke}{rgb}{0.000000,0.000000,0.000000}%
\pgfsetstrokecolor{currentstroke}%
\pgfsetdash{}{0pt}%
\pgfpathmoveto{\pgfqpoint{3.960619in}{2.055379in}}%
\pgfpathlineto{\pgfqpoint{3.974369in}{2.054028in}}%
\pgfpathlineto{\pgfqpoint{3.988125in}{2.052754in}}%
\pgfpathlineto{\pgfqpoint{4.001890in}{2.051558in}}%
\pgfpathlineto{\pgfqpoint{4.015662in}{2.050439in}}%
\pgfpathlineto{\pgfqpoint{4.023644in}{2.059274in}}%
\pgfpathlineto{\pgfqpoint{4.031621in}{2.068078in}}%
\pgfpathlineto{\pgfqpoint{4.039593in}{2.076854in}}%
\pgfpathlineto{\pgfqpoint{4.047558in}{2.085600in}}%
\pgfpathlineto{\pgfqpoint{4.033797in}{2.086740in}}%
\pgfpathlineto{\pgfqpoint{4.020043in}{2.087957in}}%
\pgfpathlineto{\pgfqpoint{4.006297in}{2.089251in}}%
\pgfpathlineto{\pgfqpoint{3.992559in}{2.090623in}}%
\pgfpathlineto{\pgfqpoint{3.984582in}{2.081848in}}%
\pgfpathlineto{\pgfqpoint{3.976600in}{2.073050in}}%
\pgfpathlineto{\pgfqpoint{3.968613in}{2.064227in}}%
\pgfpathlineto{\pgfqpoint{3.960619in}{2.055379in}}%
\pgfpathclose%
\pgfusepath{fill}%
\end{pgfscope}%
\begin{pgfscope}%
\pgfpathrectangle{\pgfqpoint{1.150000in}{0.150000in}}{\pgfqpoint{5.700000in}{5.700000in}}%
\pgfusepath{clip}%
\pgfsetbuttcap%
\pgfsetroundjoin%
\definecolor{currentfill}{rgb}{0.281412,0.155834,0.469201}%
\pgfsetfillcolor{currentfill}%
\pgfsetfillopacity{0.700000}%
\pgfsetlinewidth{0.000000pt}%
\definecolor{currentstroke}{rgb}{0.000000,0.000000,0.000000}%
\pgfsetstrokecolor{currentstroke}%
\pgfsetdash{}{0pt}%
\pgfpathmoveto{\pgfqpoint{4.363448in}{2.177847in}}%
\pgfpathlineto{\pgfqpoint{4.377316in}{2.177801in}}%
\pgfpathlineto{\pgfqpoint{4.391194in}{2.177829in}}%
\pgfpathlineto{\pgfqpoint{4.405080in}{2.177930in}}%
\pgfpathlineto{\pgfqpoint{4.418975in}{2.178104in}}%
\pgfpathlineto{\pgfqpoint{4.426809in}{2.186116in}}%
\pgfpathlineto{\pgfqpoint{4.434637in}{2.194095in}}%
\pgfpathlineto{\pgfqpoint{4.442459in}{2.202043in}}%
\pgfpathlineto{\pgfqpoint{4.450274in}{2.209962in}}%
\pgfpathlineto{\pgfqpoint{4.436391in}{2.209891in}}%
\pgfpathlineto{\pgfqpoint{4.422517in}{2.209894in}}%
\pgfpathlineto{\pgfqpoint{4.408651in}{2.209970in}}%
\pgfpathlineto{\pgfqpoint{4.394794in}{2.210120in}}%
\pgfpathlineto{\pgfqpoint{4.386967in}{2.202090in}}%
\pgfpathlineto{\pgfqpoint{4.379133in}{2.194036in}}%
\pgfpathlineto{\pgfqpoint{4.371293in}{2.185956in}}%
\pgfpathlineto{\pgfqpoint{4.363448in}{2.177847in}}%
\pgfpathclose%
\pgfusepath{fill}%
\end{pgfscope}%
\begin{pgfscope}%
\pgfpathrectangle{\pgfqpoint{1.150000in}{0.150000in}}{\pgfqpoint{5.700000in}{5.700000in}}%
\pgfusepath{clip}%
\pgfsetbuttcap%
\pgfsetroundjoin%
\definecolor{currentfill}{rgb}{0.274952,0.037752,0.364543}%
\pgfsetfillcolor{currentfill}%
\pgfsetfillopacity{0.700000}%
\pgfsetlinewidth{0.000000pt}%
\definecolor{currentstroke}{rgb}{0.000000,0.000000,0.000000}%
\pgfsetstrokecolor{currentstroke}%
\pgfsetdash{}{0pt}%
\pgfpathmoveto{\pgfqpoint{2.902340in}{1.970142in}}%
\pgfpathlineto{\pgfqpoint{2.915923in}{1.963190in}}%
\pgfpathlineto{\pgfqpoint{2.929508in}{1.956338in}}%
\pgfpathlineto{\pgfqpoint{2.943096in}{1.949584in}}%
\pgfpathlineto{\pgfqpoint{2.956686in}{1.942929in}}%
\pgfpathlineto{\pgfqpoint{2.965080in}{1.950271in}}%
\pgfpathlineto{\pgfqpoint{2.973465in}{1.957683in}}%
\pgfpathlineto{\pgfqpoint{2.981843in}{1.965161in}}%
\pgfpathlineto{\pgfqpoint{2.990212in}{1.972703in}}%
\pgfpathlineto{\pgfqpoint{2.976640in}{1.979194in}}%
\pgfpathlineto{\pgfqpoint{2.963070in}{1.985782in}}%
\pgfpathlineto{\pgfqpoint{2.949503in}{1.992470in}}%
\pgfpathlineto{\pgfqpoint{2.935939in}{1.999257in}}%
\pgfpathlineto{\pgfqpoint{2.927552in}{1.991872in}}%
\pgfpathlineto{\pgfqpoint{2.919156in}{1.984556in}}%
\pgfpathlineto{\pgfqpoint{2.910752in}{1.977312in}}%
\pgfpathlineto{\pgfqpoint{2.902340in}{1.970142in}}%
\pgfpathclose%
\pgfusepath{fill}%
\end{pgfscope}%
\begin{pgfscope}%
\pgfpathrectangle{\pgfqpoint{1.150000in}{0.150000in}}{\pgfqpoint{5.700000in}{5.700000in}}%
\pgfusepath{clip}%
\pgfsetbuttcap%
\pgfsetroundjoin%
\definecolor{currentfill}{rgb}{0.271305,0.019942,0.347269}%
\pgfsetfillcolor{currentfill}%
\pgfsetfillopacity{0.700000}%
\pgfsetlinewidth{0.000000pt}%
\definecolor{currentstroke}{rgb}{0.000000,0.000000,0.000000}%
\pgfsetstrokecolor{currentstroke}%
\pgfsetdash{}{0pt}%
\pgfpathmoveto{\pgfqpoint{3.186530in}{1.934767in}}%
\pgfpathlineto{\pgfqpoint{3.200130in}{1.929646in}}%
\pgfpathlineto{\pgfqpoint{3.213734in}{1.924616in}}%
\pgfpathlineto{\pgfqpoint{3.227343in}{1.919676in}}%
\pgfpathlineto{\pgfqpoint{3.240956in}{1.914826in}}%
\pgfpathlineto{\pgfqpoint{3.249224in}{1.923185in}}%
\pgfpathlineto{\pgfqpoint{3.257486in}{1.931576in}}%
\pgfpathlineto{\pgfqpoint{3.265741in}{1.939999in}}%
\pgfpathlineto{\pgfqpoint{3.273989in}{1.948452in}}%
\pgfpathlineto{\pgfqpoint{3.260391in}{1.953179in}}%
\pgfpathlineto{\pgfqpoint{3.246797in}{1.957996in}}%
\pgfpathlineto{\pgfqpoint{3.233208in}{1.962903in}}%
\pgfpathlineto{\pgfqpoint{3.219622in}{1.967900in}}%
\pgfpathlineto{\pgfqpoint{3.211359in}{1.959563in}}%
\pgfpathlineto{\pgfqpoint{3.203090in}{1.951261in}}%
\pgfpathlineto{\pgfqpoint{3.194813in}{1.942995in}}%
\pgfpathlineto{\pgfqpoint{3.186530in}{1.934767in}}%
\pgfpathclose%
\pgfusepath{fill}%
\end{pgfscope}%
\begin{pgfscope}%
\pgfpathrectangle{\pgfqpoint{1.150000in}{0.150000in}}{\pgfqpoint{5.700000in}{5.700000in}}%
\pgfusepath{clip}%
\pgfsetbuttcap%
\pgfsetroundjoin%
\definecolor{currentfill}{rgb}{0.269308,0.218818,0.509577}%
\pgfsetfillcolor{currentfill}%
\pgfsetfillopacity{0.700000}%
\pgfsetlinewidth{0.000000pt}%
\definecolor{currentstroke}{rgb}{0.000000,0.000000,0.000000}%
\pgfsetstrokecolor{currentstroke}%
\pgfsetdash{}{0pt}%
\pgfpathmoveto{\pgfqpoint{4.766369in}{2.307663in}}%
\pgfpathlineto{\pgfqpoint{4.780373in}{2.308523in}}%
\pgfpathlineto{\pgfqpoint{4.794387in}{2.309455in}}%
\pgfpathlineto{\pgfqpoint{4.808411in}{2.310457in}}%
\pgfpathlineto{\pgfqpoint{4.822444in}{2.311531in}}%
\pgfpathlineto{\pgfqpoint{4.830114in}{2.318417in}}%
\pgfpathlineto{\pgfqpoint{4.837777in}{2.325288in}}%
\pgfpathlineto{\pgfqpoint{4.845434in}{2.332149in}}%
\pgfpathlineto{\pgfqpoint{4.853085in}{2.339002in}}%
\pgfpathlineto{\pgfqpoint{4.839067in}{2.338116in}}%
\pgfpathlineto{\pgfqpoint{4.825058in}{2.337301in}}%
\pgfpathlineto{\pgfqpoint{4.811059in}{2.336557in}}%
\pgfpathlineto{\pgfqpoint{4.797070in}{2.335884in}}%
\pgfpathlineto{\pgfqpoint{4.789403in}{2.328836in}}%
\pgfpathlineto{\pgfqpoint{4.781731in}{2.321786in}}%
\pgfpathlineto{\pgfqpoint{4.774053in}{2.314729in}}%
\pgfpathlineto{\pgfqpoint{4.766369in}{2.307663in}}%
\pgfpathclose%
\pgfusepath{fill}%
\end{pgfscope}%
\begin{pgfscope}%
\pgfpathrectangle{\pgfqpoint{1.150000in}{0.150000in}}{\pgfqpoint{5.700000in}{5.700000in}}%
\pgfusepath{clip}%
\pgfsetbuttcap%
\pgfsetroundjoin%
\definecolor{currentfill}{rgb}{0.227802,0.326594,0.546532}%
\pgfsetfillcolor{currentfill}%
\pgfsetfillopacity{0.700000}%
\pgfsetlinewidth{0.000000pt}%
\definecolor{currentstroke}{rgb}{0.000000,0.000000,0.000000}%
\pgfsetstrokecolor{currentstroke}%
\pgfsetdash{}{0pt}%
\pgfpathmoveto{\pgfqpoint{5.572187in}{2.554985in}}%
\pgfpathlineto{\pgfqpoint{5.586466in}{2.556522in}}%
\pgfpathlineto{\pgfqpoint{5.600756in}{2.558126in}}%
\pgfpathlineto{\pgfqpoint{5.615057in}{2.559797in}}%
\pgfpathlineto{\pgfqpoint{5.629370in}{2.561536in}}%
\pgfpathlineto{\pgfqpoint{5.636673in}{2.566658in}}%
\pgfpathlineto{\pgfqpoint{5.643973in}{2.571869in}}%
\pgfpathlineto{\pgfqpoint{5.651270in}{2.577176in}}%
\pgfpathlineto{\pgfqpoint{5.658563in}{2.582584in}}%
\pgfpathlineto{\pgfqpoint{5.644276in}{2.581200in}}%
\pgfpathlineto{\pgfqpoint{5.629999in}{2.579883in}}%
\pgfpathlineto{\pgfqpoint{5.615734in}{2.578634in}}%
\pgfpathlineto{\pgfqpoint{5.601480in}{2.577451in}}%
\pgfpathlineto{\pgfqpoint{5.594161in}{2.571681in}}%
\pgfpathlineto{\pgfqpoint{5.586840in}{2.566018in}}%
\pgfpathlineto{\pgfqpoint{5.579515in}{2.560454in}}%
\pgfpathlineto{\pgfqpoint{5.572187in}{2.554985in}}%
\pgfpathclose%
\pgfusepath{fill}%
\end{pgfscope}%
\begin{pgfscope}%
\pgfpathrectangle{\pgfqpoint{1.150000in}{0.150000in}}{\pgfqpoint{5.700000in}{5.700000in}}%
\pgfusepath{clip}%
\pgfsetbuttcap%
\pgfsetroundjoin%
\definecolor{currentfill}{rgb}{0.279574,0.170599,0.479997}%
\pgfsetfillcolor{currentfill}%
\pgfsetfillopacity{0.700000}%
\pgfsetlinewidth{0.000000pt}%
\definecolor{currentstroke}{rgb}{0.000000,0.000000,0.000000}%
\pgfsetstrokecolor{currentstroke}%
\pgfsetdash{}{0pt}%
\pgfpathmoveto{\pgfqpoint{2.309908in}{2.235955in}}%
\pgfpathlineto{\pgfqpoint{2.323554in}{2.224279in}}%
\pgfpathlineto{\pgfqpoint{2.337199in}{2.212733in}}%
\pgfpathlineto{\pgfqpoint{2.350841in}{2.201317in}}%
\pgfpathlineto{\pgfqpoint{2.364481in}{2.190030in}}%
\pgfpathlineto{\pgfqpoint{2.373204in}{2.194210in}}%
\pgfpathlineto{\pgfqpoint{2.381915in}{2.198539in}}%
\pgfpathlineto{\pgfqpoint{2.390613in}{2.203015in}}%
\pgfpathlineto{\pgfqpoint{2.399298in}{2.207634in}}%
\pgfpathlineto{\pgfqpoint{2.385686in}{2.218690in}}%
\pgfpathlineto{\pgfqpoint{2.372072in}{2.229873in}}%
\pgfpathlineto{\pgfqpoint{2.358455in}{2.241186in}}%
\pgfpathlineto{\pgfqpoint{2.344837in}{2.252629in}}%
\pgfpathlineto{\pgfqpoint{2.336124in}{2.248235in}}%
\pgfpathlineto{\pgfqpoint{2.327399in}{2.243989in}}%
\pgfpathlineto{\pgfqpoint{2.318660in}{2.239894in}}%
\pgfpathlineto{\pgfqpoint{2.309908in}{2.235955in}}%
\pgfpathclose%
\pgfusepath{fill}%
\end{pgfscope}%
\begin{pgfscope}%
\pgfpathrectangle{\pgfqpoint{1.150000in}{0.150000in}}{\pgfqpoint{5.700000in}{5.700000in}}%
\pgfusepath{clip}%
\pgfsetbuttcap%
\pgfsetroundjoin%
\definecolor{currentfill}{rgb}{0.282656,0.100196,0.422160}%
\pgfsetfillcolor{currentfill}%
\pgfsetfillopacity{0.700000}%
\pgfsetlinewidth{0.000000pt}%
\definecolor{currentstroke}{rgb}{0.000000,0.000000,0.000000}%
\pgfsetstrokecolor{currentstroke}%
\pgfsetdash{}{0pt}%
\pgfpathmoveto{\pgfqpoint{2.562562in}{2.084582in}}%
\pgfpathlineto{\pgfqpoint{2.576164in}{2.075099in}}%
\pgfpathlineto{\pgfqpoint{2.589766in}{2.065731in}}%
\pgfpathlineto{\pgfqpoint{2.603368in}{2.056476in}}%
\pgfpathlineto{\pgfqpoint{2.616970in}{2.047333in}}%
\pgfpathlineto{\pgfqpoint{2.625543in}{2.052961in}}%
\pgfpathlineto{\pgfqpoint{2.634105in}{2.058705in}}%
\pgfpathlineto{\pgfqpoint{2.642657in}{2.064562in}}%
\pgfpathlineto{\pgfqpoint{2.651198in}{2.070528in}}%
\pgfpathlineto{\pgfqpoint{2.637619in}{2.079462in}}%
\pgfpathlineto{\pgfqpoint{2.624041in}{2.088509in}}%
\pgfpathlineto{\pgfqpoint{2.610462in}{2.097669in}}%
\pgfpathlineto{\pgfqpoint{2.596884in}{2.106943in}}%
\pgfpathlineto{\pgfqpoint{2.588320in}{2.101177in}}%
\pgfpathlineto{\pgfqpoint{2.579745in}{2.095527in}}%
\pgfpathlineto{\pgfqpoint{2.571159in}{2.089994in}}%
\pgfpathlineto{\pgfqpoint{2.562562in}{2.084582in}}%
\pgfpathclose%
\pgfusepath{fill}%
\end{pgfscope}%
\begin{pgfscope}%
\pgfpathrectangle{\pgfqpoint{1.150000in}{0.150000in}}{\pgfqpoint{5.700000in}{5.700000in}}%
\pgfusepath{clip}%
\pgfsetbuttcap%
\pgfsetroundjoin%
\definecolor{currentfill}{rgb}{0.212395,0.359683,0.551710}%
\pgfsetfillcolor{currentfill}%
\pgfsetfillopacity{0.700000}%
\pgfsetlinewidth{0.000000pt}%
\definecolor{currentstroke}{rgb}{0.000000,0.000000,0.000000}%
\pgfsetstrokecolor{currentstroke}%
\pgfsetdash{}{0pt}%
\pgfpathmoveto{\pgfqpoint{5.888542in}{2.642860in}}%
\pgfpathlineto{\pgfqpoint{5.902925in}{2.644340in}}%
\pgfpathlineto{\pgfqpoint{5.917320in}{2.645885in}}%
\pgfpathlineto{\pgfqpoint{5.931727in}{2.647497in}}%
\pgfpathlineto{\pgfqpoint{5.938901in}{2.652587in}}%
\pgfpathlineto{\pgfqpoint{5.946075in}{2.657828in}}%
\pgfpathlineto{\pgfqpoint{5.953248in}{2.663227in}}%
\pgfpathlineto{\pgfqpoint{5.960421in}{2.668791in}}%
\pgfpathlineto{\pgfqpoint{5.946044in}{2.667596in}}%
\pgfpathlineto{\pgfqpoint{5.931678in}{2.666467in}}%
\pgfpathlineto{\pgfqpoint{5.917324in}{2.665404in}}%
\pgfpathlineto{\pgfqpoint{5.910129in}{2.659522in}}%
\pgfpathlineto{\pgfqpoint{5.902934in}{2.653809in}}%
\pgfpathlineto{\pgfqpoint{5.895738in}{2.648257in}}%
\pgfpathlineto{\pgfqpoint{5.888542in}{2.642860in}}%
\pgfpathclose%
\pgfusepath{fill}%
\end{pgfscope}%
\begin{pgfscope}%
\pgfpathrectangle{\pgfqpoint{1.150000in}{0.150000in}}{\pgfqpoint{5.700000in}{5.700000in}}%
\pgfusepath{clip}%
\pgfsetbuttcap%
\pgfsetroundjoin%
\definecolor{currentfill}{rgb}{0.250425,0.274290,0.533103}%
\pgfsetfillcolor{currentfill}%
\pgfsetfillopacity{0.700000}%
\pgfsetlinewidth{0.000000pt}%
\definecolor{currentstroke}{rgb}{0.000000,0.000000,0.000000}%
\pgfsetstrokecolor{currentstroke}%
\pgfsetdash{}{0pt}%
\pgfpathmoveto{\pgfqpoint{5.169343in}{2.434770in}}%
\pgfpathlineto{\pgfqpoint{5.183486in}{2.436155in}}%
\pgfpathlineto{\pgfqpoint{5.197641in}{2.437608in}}%
\pgfpathlineto{\pgfqpoint{5.211806in}{2.439130in}}%
\pgfpathlineto{\pgfqpoint{5.225982in}{2.440722in}}%
\pgfpathlineto{\pgfqpoint{5.233471in}{2.446526in}}%
\pgfpathlineto{\pgfqpoint{5.240955in}{2.452357in}}%
\pgfpathlineto{\pgfqpoint{5.248433in}{2.458218in}}%
\pgfpathlineto{\pgfqpoint{5.255905in}{2.464115in}}%
\pgfpathlineto{\pgfqpoint{5.241749in}{2.462796in}}%
\pgfpathlineto{\pgfqpoint{5.227603in}{2.461545in}}%
\pgfpathlineto{\pgfqpoint{5.213468in}{2.460363in}}%
\pgfpathlineto{\pgfqpoint{5.199343in}{2.459249in}}%
\pgfpathlineto{\pgfqpoint{5.191851in}{2.453073in}}%
\pgfpathlineto{\pgfqpoint{5.184353in}{2.446938in}}%
\pgfpathlineto{\pgfqpoint{5.176851in}{2.440838in}}%
\pgfpathlineto{\pgfqpoint{5.169343in}{2.434770in}}%
\pgfpathclose%
\pgfusepath{fill}%
\end{pgfscope}%
\begin{pgfscope}%
\pgfpathrectangle{\pgfqpoint{1.150000in}{0.150000in}}{\pgfqpoint{5.700000in}{5.700000in}}%
\pgfusepath{clip}%
\pgfsetbuttcap%
\pgfsetroundjoin%
\definecolor{currentfill}{rgb}{0.277941,0.056324,0.381191}%
\pgfsetfillcolor{currentfill}%
\pgfsetfillopacity{0.700000}%
\pgfsetlinewidth{0.000000pt}%
\definecolor{currentstroke}{rgb}{0.000000,0.000000,0.000000}%
\pgfsetstrokecolor{currentstroke}%
\pgfsetdash{}{0pt}%
\pgfpathmoveto{\pgfqpoint{2.759858in}{2.003009in}}%
\pgfpathlineto{\pgfqpoint{2.773446in}{1.995052in}}%
\pgfpathlineto{\pgfqpoint{2.787035in}{1.987200in}}%
\pgfpathlineto{\pgfqpoint{2.800625in}{1.979452in}}%
\pgfpathlineto{\pgfqpoint{2.814218in}{1.971807in}}%
\pgfpathlineto{\pgfqpoint{2.822685in}{1.978464in}}%
\pgfpathlineto{\pgfqpoint{2.831142in}{1.985209in}}%
\pgfpathlineto{\pgfqpoint{2.839591in}{1.992041in}}%
\pgfpathlineto{\pgfqpoint{2.848031in}{1.998957in}}%
\pgfpathlineto{\pgfqpoint{2.834459in}{2.006416in}}%
\pgfpathlineto{\pgfqpoint{2.820889in}{2.013978in}}%
\pgfpathlineto{\pgfqpoint{2.807320in}{2.021644in}}%
\pgfpathlineto{\pgfqpoint{2.793753in}{2.029415in}}%
\pgfpathlineto{\pgfqpoint{2.785293in}{2.022677in}}%
\pgfpathlineto{\pgfqpoint{2.776824in}{2.016029in}}%
\pgfpathlineto{\pgfqpoint{2.768346in}{2.009472in}}%
\pgfpathlineto{\pgfqpoint{2.759858in}{2.003009in}}%
\pgfpathclose%
\pgfusepath{fill}%
\end{pgfscope}%
\begin{pgfscope}%
\pgfpathrectangle{\pgfqpoint{1.150000in}{0.150000in}}{\pgfqpoint{5.700000in}{5.700000in}}%
\pgfusepath{clip}%
\pgfsetbuttcap%
\pgfsetroundjoin%
\definecolor{currentfill}{rgb}{0.282623,0.140926,0.457517}%
\pgfsetfillcolor{currentfill}%
\pgfsetfillopacity{0.700000}%
\pgfsetlinewidth{0.000000pt}%
\definecolor{currentstroke}{rgb}{0.000000,0.000000,0.000000}%
\pgfsetstrokecolor{currentstroke}%
\pgfsetdash{}{0pt}%
\pgfpathmoveto{\pgfqpoint{4.276574in}{2.145686in}}%
\pgfpathlineto{\pgfqpoint{4.290419in}{2.145426in}}%
\pgfpathlineto{\pgfqpoint{4.304273in}{2.145240in}}%
\pgfpathlineto{\pgfqpoint{4.318135in}{2.145129in}}%
\pgfpathlineto{\pgfqpoint{4.332006in}{2.145092in}}%
\pgfpathlineto{\pgfqpoint{4.339876in}{2.153333in}}%
\pgfpathlineto{\pgfqpoint{4.347739in}{2.161538in}}%
\pgfpathlineto{\pgfqpoint{4.355596in}{2.169709in}}%
\pgfpathlineto{\pgfqpoint{4.363448in}{2.177847in}}%
\pgfpathlineto{\pgfqpoint{4.349588in}{2.177968in}}%
\pgfpathlineto{\pgfqpoint{4.335737in}{2.178162in}}%
\pgfpathlineto{\pgfqpoint{4.321895in}{2.178430in}}%
\pgfpathlineto{\pgfqpoint{4.308061in}{2.178773in}}%
\pgfpathlineto{\pgfqpoint{4.300198in}{2.170544in}}%
\pgfpathlineto{\pgfqpoint{4.292329in}{2.162287in}}%
\pgfpathlineto{\pgfqpoint{4.284454in}{2.154002in}}%
\pgfpathlineto{\pgfqpoint{4.276574in}{2.145686in}}%
\pgfpathclose%
\pgfusepath{fill}%
\end{pgfscope}%
\begin{pgfscope}%
\pgfpathrectangle{\pgfqpoint{1.150000in}{0.150000in}}{\pgfqpoint{5.700000in}{5.700000in}}%
\pgfusepath{clip}%
\pgfsetbuttcap%
\pgfsetroundjoin%
\definecolor{currentfill}{rgb}{0.280894,0.078907,0.402329}%
\pgfsetfillcolor{currentfill}%
\pgfsetfillopacity{0.700000}%
\pgfsetlinewidth{0.000000pt}%
\definecolor{currentstroke}{rgb}{0.000000,0.000000,0.000000}%
\pgfsetstrokecolor{currentstroke}%
\pgfsetdash{}{0pt}%
\pgfpathmoveto{\pgfqpoint{3.873623in}{2.025904in}}%
\pgfpathlineto{\pgfqpoint{3.887354in}{2.024241in}}%
\pgfpathlineto{\pgfqpoint{3.901092in}{2.022656in}}%
\pgfpathlineto{\pgfqpoint{3.914837in}{2.021149in}}%
\pgfpathlineto{\pgfqpoint{3.928589in}{2.019721in}}%
\pgfpathlineto{\pgfqpoint{3.936605in}{2.028676in}}%
\pgfpathlineto{\pgfqpoint{3.944616in}{2.037604in}}%
\pgfpathlineto{\pgfqpoint{3.952620in}{2.046505in}}%
\pgfpathlineto{\pgfqpoint{3.960619in}{2.055379in}}%
\pgfpathlineto{\pgfqpoint{3.946878in}{2.056808in}}%
\pgfpathlineto{\pgfqpoint{3.933143in}{2.058314in}}%
\pgfpathlineto{\pgfqpoint{3.919416in}{2.059899in}}%
\pgfpathlineto{\pgfqpoint{3.905696in}{2.061563in}}%
\pgfpathlineto{\pgfqpoint{3.897686in}{2.052681in}}%
\pgfpathlineto{\pgfqpoint{3.889671in}{2.043777in}}%
\pgfpathlineto{\pgfqpoint{3.881650in}{2.034852in}}%
\pgfpathlineto{\pgfqpoint{3.873623in}{2.025904in}}%
\pgfpathclose%
\pgfusepath{fill}%
\end{pgfscope}%
\begin{pgfscope}%
\pgfpathrectangle{\pgfqpoint{1.150000in}{0.150000in}}{\pgfqpoint{5.700000in}{5.700000in}}%
\pgfusepath{clip}%
\pgfsetbuttcap%
\pgfsetroundjoin%
\definecolor{currentfill}{rgb}{0.274952,0.037752,0.364543}%
\pgfsetfillcolor{currentfill}%
\pgfsetfillopacity{0.700000}%
\pgfsetlinewidth{0.000000pt}%
\definecolor{currentstroke}{rgb}{0.000000,0.000000,0.000000}%
\pgfsetstrokecolor{currentstroke}%
\pgfsetdash{}{0pt}%
\pgfpathmoveto{\pgfqpoint{3.557556in}{1.956192in}}%
\pgfpathlineto{\pgfqpoint{3.571215in}{1.953114in}}%
\pgfpathlineto{\pgfqpoint{3.584881in}{1.950120in}}%
\pgfpathlineto{\pgfqpoint{3.598553in}{1.947208in}}%
\pgfpathlineto{\pgfqpoint{3.612230in}{1.944378in}}%
\pgfpathlineto{\pgfqpoint{3.620359in}{1.953392in}}%
\pgfpathlineto{\pgfqpoint{3.628482in}{1.962400in}}%
\pgfpathlineto{\pgfqpoint{3.636599in}{1.971402in}}%
\pgfpathlineto{\pgfqpoint{3.644710in}{1.980396in}}%
\pgfpathlineto{\pgfqpoint{3.631044in}{1.983165in}}%
\pgfpathlineto{\pgfqpoint{3.617384in}{1.986015in}}%
\pgfpathlineto{\pgfqpoint{3.603730in}{1.988948in}}%
\pgfpathlineto{\pgfqpoint{3.590082in}{1.991964in}}%
\pgfpathlineto{\pgfqpoint{3.581959in}{1.983024in}}%
\pgfpathlineto{\pgfqpoint{3.573831in}{1.974081in}}%
\pgfpathlineto{\pgfqpoint{3.565696in}{1.965137in}}%
\pgfpathlineto{\pgfqpoint{3.557556in}{1.956192in}}%
\pgfpathclose%
\pgfusepath{fill}%
\end{pgfscope}%
\begin{pgfscope}%
\pgfpathrectangle{\pgfqpoint{1.150000in}{0.150000in}}{\pgfqpoint{5.700000in}{5.700000in}}%
\pgfusepath{clip}%
\pgfsetbuttcap%
\pgfsetroundjoin%
\definecolor{currentfill}{rgb}{0.272594,0.025563,0.353093}%
\pgfsetfillcolor{currentfill}%
\pgfsetfillopacity{0.700000}%
\pgfsetlinewidth{0.000000pt}%
\definecolor{currentstroke}{rgb}{0.000000,0.000000,0.000000}%
\pgfsetstrokecolor{currentstroke}%
\pgfsetdash{}{0pt}%
\pgfpathmoveto{\pgfqpoint{3.328428in}{1.930433in}}%
\pgfpathlineto{\pgfqpoint{3.342049in}{1.926149in}}%
\pgfpathlineto{\pgfqpoint{3.355675in}{1.921952in}}%
\pgfpathlineto{\pgfqpoint{3.369306in}{1.917842in}}%
\pgfpathlineto{\pgfqpoint{3.382942in}{1.913818in}}%
\pgfpathlineto{\pgfqpoint{3.391156in}{1.922519in}}%
\pgfpathlineto{\pgfqpoint{3.399364in}{1.931238in}}%
\pgfpathlineto{\pgfqpoint{3.407566in}{1.939971in}}%
\pgfpathlineto{\pgfqpoint{3.415761in}{1.948720in}}%
\pgfpathlineto{\pgfqpoint{3.402139in}{1.952641in}}%
\pgfpathlineto{\pgfqpoint{3.388521in}{1.956649in}}%
\pgfpathlineto{\pgfqpoint{3.374909in}{1.960743in}}%
\pgfpathlineto{\pgfqpoint{3.361301in}{1.964925in}}%
\pgfpathlineto{\pgfqpoint{3.353092in}{1.956272in}}%
\pgfpathlineto{\pgfqpoint{3.344877in}{1.947638in}}%
\pgfpathlineto{\pgfqpoint{3.336656in}{1.939025in}}%
\pgfpathlineto{\pgfqpoint{3.328428in}{1.930433in}}%
\pgfpathclose%
\pgfusepath{fill}%
\end{pgfscope}%
\begin{pgfscope}%
\pgfpathrectangle{\pgfqpoint{1.150000in}{0.150000in}}{\pgfqpoint{5.700000in}{5.700000in}}%
\pgfusepath{clip}%
\pgfsetbuttcap%
\pgfsetroundjoin%
\definecolor{currentfill}{rgb}{0.273006,0.204520,0.501721}%
\pgfsetfillcolor{currentfill}%
\pgfsetfillopacity{0.700000}%
\pgfsetlinewidth{0.000000pt}%
\definecolor{currentstroke}{rgb}{0.000000,0.000000,0.000000}%
\pgfsetstrokecolor{currentstroke}%
\pgfsetdash{}{0pt}%
\pgfpathmoveto{\pgfqpoint{4.679597in}{2.275840in}}%
\pgfpathlineto{\pgfqpoint{4.693576in}{2.276582in}}%
\pgfpathlineto{\pgfqpoint{4.707565in}{2.277396in}}%
\pgfpathlineto{\pgfqpoint{4.721563in}{2.278280in}}%
\pgfpathlineto{\pgfqpoint{4.735571in}{2.279237in}}%
\pgfpathlineto{\pgfqpoint{4.743280in}{2.286374in}}%
\pgfpathlineto{\pgfqpoint{4.750983in}{2.293489in}}%
\pgfpathlineto{\pgfqpoint{4.758679in}{2.300584in}}%
\pgfpathlineto{\pgfqpoint{4.766369in}{2.307663in}}%
\pgfpathlineto{\pgfqpoint{4.752375in}{2.306874in}}%
\pgfpathlineto{\pgfqpoint{4.738391in}{2.306156in}}%
\pgfpathlineto{\pgfqpoint{4.724416in}{2.305509in}}%
\pgfpathlineto{\pgfqpoint{4.710451in}{2.304934in}}%
\pgfpathlineto{\pgfqpoint{4.702747in}{2.297680in}}%
\pgfpathlineto{\pgfqpoint{4.695037in}{2.290416in}}%
\pgfpathlineto{\pgfqpoint{4.687320in}{2.283137in}}%
\pgfpathlineto{\pgfqpoint{4.679597in}{2.275840in}}%
\pgfpathclose%
\pgfusepath{fill}%
\end{pgfscope}%
\begin{pgfscope}%
\pgfpathrectangle{\pgfqpoint{1.150000in}{0.150000in}}{\pgfqpoint{5.700000in}{5.700000in}}%
\pgfusepath{clip}%
\pgfsetbuttcap%
\pgfsetroundjoin%
\definecolor{currentfill}{rgb}{0.231674,0.318106,0.544834}%
\pgfsetfillcolor{currentfill}%
\pgfsetfillopacity{0.700000}%
\pgfsetlinewidth{0.000000pt}%
\definecolor{currentstroke}{rgb}{0.000000,0.000000,0.000000}%
\pgfsetstrokecolor{currentstroke}%
\pgfsetdash{}{0pt}%
\pgfpathmoveto{\pgfqpoint{5.485743in}{2.527117in}}%
\pgfpathlineto{\pgfqpoint{5.500000in}{2.528718in}}%
\pgfpathlineto{\pgfqpoint{5.514268in}{2.530386in}}%
\pgfpathlineto{\pgfqpoint{5.528547in}{2.532122in}}%
\pgfpathlineto{\pgfqpoint{5.542838in}{2.533926in}}%
\pgfpathlineto{\pgfqpoint{5.550182in}{2.539080in}}%
\pgfpathlineto{\pgfqpoint{5.557521in}{2.544304in}}%
\pgfpathlineto{\pgfqpoint{5.564856in}{2.549603in}}%
\pgfpathlineto{\pgfqpoint{5.572187in}{2.554985in}}%
\pgfpathlineto{\pgfqpoint{5.557920in}{2.553516in}}%
\pgfpathlineto{\pgfqpoint{5.543664in}{2.552114in}}%
\pgfpathlineto{\pgfqpoint{5.529419in}{2.550779in}}%
\pgfpathlineto{\pgfqpoint{5.515186in}{2.549512in}}%
\pgfpathlineto{\pgfqpoint{5.507831in}{2.543789in}}%
\pgfpathlineto{\pgfqpoint{5.500472in}{2.538153in}}%
\pgfpathlineto{\pgfqpoint{5.493109in}{2.532598in}}%
\pgfpathlineto{\pgfqpoint{5.485743in}{2.527117in}}%
\pgfpathclose%
\pgfusepath{fill}%
\end{pgfscope}%
\begin{pgfscope}%
\pgfpathrectangle{\pgfqpoint{1.150000in}{0.150000in}}{\pgfqpoint{5.700000in}{5.700000in}}%
\pgfusepath{clip}%
\pgfsetbuttcap%
\pgfsetroundjoin%
\definecolor{currentfill}{rgb}{0.253935,0.265254,0.529983}%
\pgfsetfillcolor{currentfill}%
\pgfsetfillopacity{0.700000}%
\pgfsetlinewidth{0.000000pt}%
\definecolor{currentstroke}{rgb}{0.000000,0.000000,0.000000}%
\pgfsetstrokecolor{currentstroke}%
\pgfsetdash{}{0pt}%
\pgfpathmoveto{\pgfqpoint{5.082713in}{2.404865in}}%
\pgfpathlineto{\pgfqpoint{5.096832in}{2.406224in}}%
\pgfpathlineto{\pgfqpoint{5.110962in}{2.407651in}}%
\pgfpathlineto{\pgfqpoint{5.125103in}{2.409148in}}%
\pgfpathlineto{\pgfqpoint{5.139255in}{2.410715in}}%
\pgfpathlineto{\pgfqpoint{5.146785in}{2.416705in}}%
\pgfpathlineto{\pgfqpoint{5.154310in}{2.422708in}}%
\pgfpathlineto{\pgfqpoint{5.161829in}{2.428728in}}%
\pgfpathlineto{\pgfqpoint{5.169343in}{2.434770in}}%
\pgfpathlineto{\pgfqpoint{5.155210in}{2.433455in}}%
\pgfpathlineto{\pgfqpoint{5.141087in}{2.432209in}}%
\pgfpathlineto{\pgfqpoint{5.126975in}{2.431032in}}%
\pgfpathlineto{\pgfqpoint{5.112874in}{2.429924in}}%
\pgfpathlineto{\pgfqpoint{5.105342in}{2.423624in}}%
\pgfpathlineto{\pgfqpoint{5.097804in}{2.417350in}}%
\pgfpathlineto{\pgfqpoint{5.090262in}{2.411099in}}%
\pgfpathlineto{\pgfqpoint{5.082713in}{2.404865in}}%
\pgfpathclose%
\pgfusepath{fill}%
\end{pgfscope}%
\begin{pgfscope}%
\pgfpathrectangle{\pgfqpoint{1.150000in}{0.150000in}}{\pgfqpoint{5.700000in}{5.700000in}}%
\pgfusepath{clip}%
\pgfsetbuttcap%
\pgfsetroundjoin%
\definecolor{currentfill}{rgb}{0.281887,0.150881,0.465405}%
\pgfsetfillcolor{currentfill}%
\pgfsetfillopacity{0.700000}%
\pgfsetlinewidth{0.000000pt}%
\definecolor{currentstroke}{rgb}{0.000000,0.000000,0.000000}%
\pgfsetstrokecolor{currentstroke}%
\pgfsetdash{}{0pt}%
\pgfpathmoveto{\pgfqpoint{2.364481in}{2.190030in}}%
\pgfpathlineto{\pgfqpoint{2.378120in}{2.178869in}}%
\pgfpathlineto{\pgfqpoint{2.391757in}{2.167835in}}%
\pgfpathlineto{\pgfqpoint{2.405392in}{2.156926in}}%
\pgfpathlineto{\pgfqpoint{2.419026in}{2.146141in}}%
\pgfpathlineto{\pgfqpoint{2.427721in}{2.150561in}}%
\pgfpathlineto{\pgfqpoint{2.436404in}{2.155125in}}%
\pgfpathlineto{\pgfqpoint{2.445075in}{2.159829in}}%
\pgfpathlineto{\pgfqpoint{2.453734in}{2.164672in}}%
\pgfpathlineto{\pgfqpoint{2.440127in}{2.175226in}}%
\pgfpathlineto{\pgfqpoint{2.426519in}{2.185903in}}%
\pgfpathlineto{\pgfqpoint{2.412909in}{2.196706in}}%
\pgfpathlineto{\pgfqpoint{2.399298in}{2.207634in}}%
\pgfpathlineto{\pgfqpoint{2.390613in}{2.203015in}}%
\pgfpathlineto{\pgfqpoint{2.381915in}{2.198539in}}%
\pgfpathlineto{\pgfqpoint{2.373204in}{2.194210in}}%
\pgfpathlineto{\pgfqpoint{2.364481in}{2.190030in}}%
\pgfpathclose%
\pgfusepath{fill}%
\end{pgfscope}%
\begin{pgfscope}%
\pgfpathrectangle{\pgfqpoint{1.150000in}{0.150000in}}{\pgfqpoint{5.700000in}{5.700000in}}%
\pgfusepath{clip}%
\pgfsetbuttcap%
\pgfsetroundjoin%
\definecolor{currentfill}{rgb}{0.283072,0.130895,0.449241}%
\pgfsetfillcolor{currentfill}%
\pgfsetfillopacity{0.700000}%
\pgfsetlinewidth{0.000000pt}%
\definecolor{currentstroke}{rgb}{0.000000,0.000000,0.000000}%
\pgfsetstrokecolor{currentstroke}%
\pgfsetdash{}{0pt}%
\pgfpathmoveto{\pgfqpoint{4.189652in}{2.113617in}}%
\pgfpathlineto{\pgfqpoint{4.203474in}{2.113121in}}%
\pgfpathlineto{\pgfqpoint{4.217305in}{2.112699in}}%
\pgfpathlineto{\pgfqpoint{4.231144in}{2.112352in}}%
\pgfpathlineto{\pgfqpoint{4.244992in}{2.112080in}}%
\pgfpathlineto{\pgfqpoint{4.252896in}{2.120535in}}%
\pgfpathlineto{\pgfqpoint{4.260795in}{2.128954in}}%
\pgfpathlineto{\pgfqpoint{4.268687in}{2.137337in}}%
\pgfpathlineto{\pgfqpoint{4.276574in}{2.145686in}}%
\pgfpathlineto{\pgfqpoint{4.262737in}{2.146020in}}%
\pgfpathlineto{\pgfqpoint{4.248909in}{2.146429in}}%
\pgfpathlineto{\pgfqpoint{4.235089in}{2.146913in}}%
\pgfpathlineto{\pgfqpoint{4.221277in}{2.147472in}}%
\pgfpathlineto{\pgfqpoint{4.213380in}{2.139054in}}%
\pgfpathlineto{\pgfqpoint{4.205476in}{2.130606in}}%
\pgfpathlineto{\pgfqpoint{4.197567in}{2.122128in}}%
\pgfpathlineto{\pgfqpoint{4.189652in}{2.113617in}}%
\pgfpathclose%
\pgfusepath{fill}%
\end{pgfscope}%
\begin{pgfscope}%
\pgfpathrectangle{\pgfqpoint{1.150000in}{0.150000in}}{\pgfqpoint{5.700000in}{5.700000in}}%
\pgfusepath{clip}%
\pgfsetbuttcap%
\pgfsetroundjoin%
\definecolor{currentfill}{rgb}{0.279566,0.067836,0.391917}%
\pgfsetfillcolor{currentfill}%
\pgfsetfillopacity{0.700000}%
\pgfsetlinewidth{0.000000pt}%
\definecolor{currentstroke}{rgb}{0.000000,0.000000,0.000000}%
\pgfsetstrokecolor{currentstroke}%
\pgfsetdash{}{0pt}%
\pgfpathmoveto{\pgfqpoint{3.786564in}{1.997407in}}%
\pgfpathlineto{\pgfqpoint{3.800278in}{1.995407in}}%
\pgfpathlineto{\pgfqpoint{3.813998in}{1.993487in}}%
\pgfpathlineto{\pgfqpoint{3.827725in}{1.991645in}}%
\pgfpathlineto{\pgfqpoint{3.841460in}{1.989883in}}%
\pgfpathlineto{\pgfqpoint{3.849509in}{1.998923in}}%
\pgfpathlineto{\pgfqpoint{3.857553in}{2.007940in}}%
\pgfpathlineto{\pgfqpoint{3.865591in}{2.016934in}}%
\pgfpathlineto{\pgfqpoint{3.873623in}{2.025904in}}%
\pgfpathlineto{\pgfqpoint{3.859900in}{2.027646in}}%
\pgfpathlineto{\pgfqpoint{3.846183in}{2.029467in}}%
\pgfpathlineto{\pgfqpoint{3.832474in}{2.031367in}}%
\pgfpathlineto{\pgfqpoint{3.818771in}{2.033347in}}%
\pgfpathlineto{\pgfqpoint{3.810728in}{2.024389in}}%
\pgfpathlineto{\pgfqpoint{3.802679in}{2.015413in}}%
\pgfpathlineto{\pgfqpoint{3.794625in}{2.006419in}}%
\pgfpathlineto{\pgfqpoint{3.786564in}{1.997407in}}%
\pgfpathclose%
\pgfusepath{fill}%
\end{pgfscope}%
\begin{pgfscope}%
\pgfpathrectangle{\pgfqpoint{1.150000in}{0.150000in}}{\pgfqpoint{5.700000in}{5.700000in}}%
\pgfusepath{clip}%
\pgfsetbuttcap%
\pgfsetroundjoin%
\definecolor{currentfill}{rgb}{0.276194,0.190074,0.493001}%
\pgfsetfillcolor{currentfill}%
\pgfsetfillopacity{0.700000}%
\pgfsetlinewidth{0.000000pt}%
\definecolor{currentstroke}{rgb}{0.000000,0.000000,0.000000}%
\pgfsetstrokecolor{currentstroke}%
\pgfsetdash{}{0pt}%
\pgfpathmoveto{\pgfqpoint{4.592773in}{2.243586in}}%
\pgfpathlineto{\pgfqpoint{4.606727in}{2.244187in}}%
\pgfpathlineto{\pgfqpoint{4.620690in}{2.244859in}}%
\pgfpathlineto{\pgfqpoint{4.634663in}{2.245603in}}%
\pgfpathlineto{\pgfqpoint{4.648646in}{2.246420in}}%
\pgfpathlineto{\pgfqpoint{4.656393in}{2.253816in}}%
\pgfpathlineto{\pgfqpoint{4.664134in}{2.261183in}}%
\pgfpathlineto{\pgfqpoint{4.671869in}{2.268523in}}%
\pgfpathlineto{\pgfqpoint{4.679597in}{2.275840in}}%
\pgfpathlineto{\pgfqpoint{4.665628in}{2.275170in}}%
\pgfpathlineto{\pgfqpoint{4.651669in}{2.274572in}}%
\pgfpathlineto{\pgfqpoint{4.637719in}{2.274045in}}%
\pgfpathlineto{\pgfqpoint{4.623778in}{2.273591in}}%
\pgfpathlineto{\pgfqpoint{4.616036in}{2.266120in}}%
\pgfpathlineto{\pgfqpoint{4.608288in}{2.258631in}}%
\pgfpathlineto{\pgfqpoint{4.600533in}{2.251121in}}%
\pgfpathlineto{\pgfqpoint{4.592773in}{2.243586in}}%
\pgfpathclose%
\pgfusepath{fill}%
\end{pgfscope}%
\begin{pgfscope}%
\pgfpathrectangle{\pgfqpoint{1.150000in}{0.150000in}}{\pgfqpoint{5.700000in}{5.700000in}}%
\pgfusepath{clip}%
\pgfsetbuttcap%
\pgfsetroundjoin%
\definecolor{currentfill}{rgb}{0.281446,0.084320,0.407414}%
\pgfsetfillcolor{currentfill}%
\pgfsetfillopacity{0.700000}%
\pgfsetlinewidth{0.000000pt}%
\definecolor{currentstroke}{rgb}{0.000000,0.000000,0.000000}%
\pgfsetstrokecolor{currentstroke}%
\pgfsetdash{}{0pt}%
\pgfpathmoveto{\pgfqpoint{2.616970in}{2.047333in}}%
\pgfpathlineto{\pgfqpoint{2.630573in}{2.038303in}}%
\pgfpathlineto{\pgfqpoint{2.644176in}{2.029383in}}%
\pgfpathlineto{\pgfqpoint{2.657780in}{2.020574in}}%
\pgfpathlineto{\pgfqpoint{2.671385in}{2.011874in}}%
\pgfpathlineto{\pgfqpoint{2.679934in}{2.017718in}}%
\pgfpathlineto{\pgfqpoint{2.688472in}{2.023672in}}%
\pgfpathlineto{\pgfqpoint{2.697001in}{2.029734in}}%
\pgfpathlineto{\pgfqpoint{2.705520in}{2.035900in}}%
\pgfpathlineto{\pgfqpoint{2.691939in}{2.044393in}}%
\pgfpathlineto{\pgfqpoint{2.678358in}{2.052994in}}%
\pgfpathlineto{\pgfqpoint{2.664778in}{2.061706in}}%
\pgfpathlineto{\pgfqpoint{2.651198in}{2.070528in}}%
\pgfpathlineto{\pgfqpoint{2.642657in}{2.064562in}}%
\pgfpathlineto{\pgfqpoint{2.634105in}{2.058705in}}%
\pgfpathlineto{\pgfqpoint{2.625543in}{2.052961in}}%
\pgfpathlineto{\pgfqpoint{2.616970in}{2.047333in}}%
\pgfpathclose%
\pgfusepath{fill}%
\end{pgfscope}%
\begin{pgfscope}%
\pgfpathrectangle{\pgfqpoint{1.150000in}{0.150000in}}{\pgfqpoint{5.700000in}{5.700000in}}%
\pgfusepath{clip}%
\pgfsetbuttcap%
\pgfsetroundjoin%
\definecolor{currentfill}{rgb}{0.214298,0.355619,0.551184}%
\pgfsetfillcolor{currentfill}%
\pgfsetfillopacity{0.700000}%
\pgfsetlinewidth{0.000000pt}%
\definecolor{currentstroke}{rgb}{0.000000,0.000000,0.000000}%
\pgfsetstrokecolor{currentstroke}%
\pgfsetdash{}{0pt}%
\pgfpathmoveto{\pgfqpoint{5.802217in}{2.615839in}}%
\pgfpathlineto{\pgfqpoint{5.816582in}{2.617449in}}%
\pgfpathlineto{\pgfqpoint{5.830958in}{2.619125in}}%
\pgfpathlineto{\pgfqpoint{5.845346in}{2.620869in}}%
\pgfpathlineto{\pgfqpoint{5.859746in}{2.622679in}}%
\pgfpathlineto{\pgfqpoint{5.866948in}{2.627527in}}%
\pgfpathlineto{\pgfqpoint{5.874147in}{2.632502in}}%
\pgfpathlineto{\pgfqpoint{5.881345in}{2.637611in}}%
\pgfpathlineto{\pgfqpoint{5.888542in}{2.642860in}}%
\pgfpathlineto{\pgfqpoint{5.874171in}{2.641448in}}%
\pgfpathlineto{\pgfqpoint{5.859811in}{2.640101in}}%
\pgfpathlineto{\pgfqpoint{5.845462in}{2.638821in}}%
\pgfpathlineto{\pgfqpoint{5.831125in}{2.637607in}}%
\pgfpathlineto{\pgfqpoint{5.823900in}{2.631954in}}%
\pgfpathlineto{\pgfqpoint{5.816674in}{2.626446in}}%
\pgfpathlineto{\pgfqpoint{5.809446in}{2.621077in}}%
\pgfpathlineto{\pgfqpoint{5.802217in}{2.615839in}}%
\pgfpathclose%
\pgfusepath{fill}%
\end{pgfscope}%
\begin{pgfscope}%
\pgfpathrectangle{\pgfqpoint{1.150000in}{0.150000in}}{\pgfqpoint{5.700000in}{5.700000in}}%
\pgfusepath{clip}%
\pgfsetbuttcap%
\pgfsetroundjoin%
\definecolor{currentfill}{rgb}{0.273809,0.031497,0.358853}%
\pgfsetfillcolor{currentfill}%
\pgfsetfillopacity{0.700000}%
\pgfsetlinewidth{0.000000pt}%
\definecolor{currentstroke}{rgb}{0.000000,0.000000,0.000000}%
\pgfsetstrokecolor{currentstroke}%
\pgfsetdash{}{0pt}%
\pgfpathmoveto{\pgfqpoint{2.956686in}{1.942929in}}%
\pgfpathlineto{\pgfqpoint{2.970279in}{1.936371in}}%
\pgfpathlineto{\pgfqpoint{2.983875in}{1.929910in}}%
\pgfpathlineto{\pgfqpoint{2.997474in}{1.923546in}}%
\pgfpathlineto{\pgfqpoint{3.011076in}{1.917277in}}%
\pgfpathlineto{\pgfqpoint{3.019451in}{1.924792in}}%
\pgfpathlineto{\pgfqpoint{3.027819in}{1.932371in}}%
\pgfpathlineto{\pgfqpoint{3.036178in}{1.940011in}}%
\pgfpathlineto{\pgfqpoint{3.044530in}{1.947710in}}%
\pgfpathlineto{\pgfqpoint{3.030946in}{1.953814in}}%
\pgfpathlineto{\pgfqpoint{3.017365in}{1.960014in}}%
\pgfpathlineto{\pgfqpoint{3.003787in}{1.966310in}}%
\pgfpathlineto{\pgfqpoint{2.990212in}{1.972703in}}%
\pgfpathlineto{\pgfqpoint{2.981843in}{1.965161in}}%
\pgfpathlineto{\pgfqpoint{2.973465in}{1.957683in}}%
\pgfpathlineto{\pgfqpoint{2.965080in}{1.950271in}}%
\pgfpathlineto{\pgfqpoint{2.956686in}{1.942929in}}%
\pgfpathclose%
\pgfusepath{fill}%
\end{pgfscope}%
\begin{pgfscope}%
\pgfpathrectangle{\pgfqpoint{1.150000in}{0.150000in}}{\pgfqpoint{5.700000in}{5.700000in}}%
\pgfusepath{clip}%
\pgfsetbuttcap%
\pgfsetroundjoin%
\definecolor{currentfill}{rgb}{0.271305,0.019942,0.347269}%
\pgfsetfillcolor{currentfill}%
\pgfsetfillopacity{0.700000}%
\pgfsetlinewidth{0.000000pt}%
\definecolor{currentstroke}{rgb}{0.000000,0.000000,0.000000}%
\pgfsetstrokecolor{currentstroke}%
\pgfsetdash{}{0pt}%
\pgfpathmoveto{\pgfqpoint{3.098900in}{1.924243in}}%
\pgfpathlineto{\pgfqpoint{3.112501in}{1.918610in}}%
\pgfpathlineto{\pgfqpoint{3.126106in}{1.913070in}}%
\pgfpathlineto{\pgfqpoint{3.139714in}{1.907623in}}%
\pgfpathlineto{\pgfqpoint{3.153326in}{1.902267in}}%
\pgfpathlineto{\pgfqpoint{3.161638in}{1.910326in}}%
\pgfpathlineto{\pgfqpoint{3.169942in}{1.918431in}}%
\pgfpathlineto{\pgfqpoint{3.178239in}{1.926578in}}%
\pgfpathlineto{\pgfqpoint{3.186530in}{1.934767in}}%
\pgfpathlineto{\pgfqpoint{3.172933in}{1.939979in}}%
\pgfpathlineto{\pgfqpoint{3.159341in}{1.945283in}}%
\pgfpathlineto{\pgfqpoint{3.145752in}{1.950679in}}%
\pgfpathlineto{\pgfqpoint{3.132167in}{1.956168in}}%
\pgfpathlineto{\pgfqpoint{3.123861in}{1.948115in}}%
\pgfpathlineto{\pgfqpoint{3.115548in}{1.940109in}}%
\pgfpathlineto{\pgfqpoint{3.107228in}{1.932151in}}%
\pgfpathlineto{\pgfqpoint{3.098900in}{1.924243in}}%
\pgfpathclose%
\pgfusepath{fill}%
\end{pgfscope}%
\begin{pgfscope}%
\pgfpathrectangle{\pgfqpoint{1.150000in}{0.150000in}}{\pgfqpoint{5.700000in}{5.700000in}}%
\pgfusepath{clip}%
\pgfsetbuttcap%
\pgfsetroundjoin%
\definecolor{currentfill}{rgb}{0.273809,0.031497,0.358853}%
\pgfsetfillcolor{currentfill}%
\pgfsetfillopacity{0.700000}%
\pgfsetlinewidth{0.000000pt}%
\definecolor{currentstroke}{rgb}{0.000000,0.000000,0.000000}%
\pgfsetstrokecolor{currentstroke}%
\pgfsetdash{}{0pt}%
\pgfpathmoveto{\pgfqpoint{3.470304in}{1.933892in}}%
\pgfpathlineto{\pgfqpoint{3.483954in}{1.930397in}}%
\pgfpathlineto{\pgfqpoint{3.497609in}{1.926987in}}%
\pgfpathlineto{\pgfqpoint{3.511269in}{1.923661in}}%
\pgfpathlineto{\pgfqpoint{3.524935in}{1.920418in}}%
\pgfpathlineto{\pgfqpoint{3.533099in}{1.929359in}}%
\pgfpathlineto{\pgfqpoint{3.541257in}{1.938302in}}%
\pgfpathlineto{\pgfqpoint{3.549409in}{1.947247in}}%
\pgfpathlineto{\pgfqpoint{3.557556in}{1.956192in}}%
\pgfpathlineto{\pgfqpoint{3.543902in}{1.959353in}}%
\pgfpathlineto{\pgfqpoint{3.530254in}{1.962597in}}%
\pgfpathlineto{\pgfqpoint{3.516611in}{1.965925in}}%
\pgfpathlineto{\pgfqpoint{3.502975in}{1.969338in}}%
\pgfpathlineto{\pgfqpoint{3.494816in}{1.960467in}}%
\pgfpathlineto{\pgfqpoint{3.486652in}{1.951602in}}%
\pgfpathlineto{\pgfqpoint{3.478481in}{1.942743in}}%
\pgfpathlineto{\pgfqpoint{3.470304in}{1.933892in}}%
\pgfpathclose%
\pgfusepath{fill}%
\end{pgfscope}%
\begin{pgfscope}%
\pgfpathrectangle{\pgfqpoint{1.150000in}{0.150000in}}{\pgfqpoint{5.700000in}{5.700000in}}%
\pgfusepath{clip}%
\pgfsetbuttcap%
\pgfsetroundjoin%
\definecolor{currentfill}{rgb}{0.258965,0.251537,0.524736}%
\pgfsetfillcolor{currentfill}%
\pgfsetfillopacity{0.700000}%
\pgfsetlinewidth{0.000000pt}%
\definecolor{currentstroke}{rgb}{0.000000,0.000000,0.000000}%
\pgfsetstrokecolor{currentstroke}%
\pgfsetdash{}{0pt}%
\pgfpathmoveto{\pgfqpoint{4.996018in}{2.374364in}}%
\pgfpathlineto{\pgfqpoint{5.010113in}{2.375673in}}%
\pgfpathlineto{\pgfqpoint{5.024218in}{2.377052in}}%
\pgfpathlineto{\pgfqpoint{5.038334in}{2.378502in}}%
\pgfpathlineto{\pgfqpoint{5.052460in}{2.380021in}}%
\pgfpathlineto{\pgfqpoint{5.060032in}{2.386227in}}%
\pgfpathlineto{\pgfqpoint{5.067598in}{2.392434in}}%
\pgfpathlineto{\pgfqpoint{5.075159in}{2.398645in}}%
\pgfpathlineto{\pgfqpoint{5.082713in}{2.404865in}}%
\pgfpathlineto{\pgfqpoint{5.068604in}{2.403577in}}%
\pgfpathlineto{\pgfqpoint{5.054506in}{2.402358in}}%
\pgfpathlineto{\pgfqpoint{5.040418in}{2.401208in}}%
\pgfpathlineto{\pgfqpoint{5.026340in}{2.400128in}}%
\pgfpathlineto{\pgfqpoint{5.018768in}{2.393671in}}%
\pgfpathlineto{\pgfqpoint{5.011191in}{2.387227in}}%
\pgfpathlineto{\pgfqpoint{5.003607in}{2.380792in}}%
\pgfpathlineto{\pgfqpoint{4.996018in}{2.374364in}}%
\pgfpathclose%
\pgfusepath{fill}%
\end{pgfscope}%
\begin{pgfscope}%
\pgfpathrectangle{\pgfqpoint{1.150000in}{0.150000in}}{\pgfqpoint{5.700000in}{5.700000in}}%
\pgfusepath{clip}%
\pgfsetbuttcap%
\pgfsetroundjoin%
\definecolor{currentfill}{rgb}{0.283197,0.115680,0.436115}%
\pgfsetfillcolor{currentfill}%
\pgfsetfillopacity{0.700000}%
\pgfsetlinewidth{0.000000pt}%
\definecolor{currentstroke}{rgb}{0.000000,0.000000,0.000000}%
\pgfsetstrokecolor{currentstroke}%
\pgfsetdash{}{0pt}%
\pgfpathmoveto{\pgfqpoint{4.102682in}{2.081807in}}%
\pgfpathlineto{\pgfqpoint{4.116482in}{2.081049in}}%
\pgfpathlineto{\pgfqpoint{4.130291in}{2.080367in}}%
\pgfpathlineto{\pgfqpoint{4.144108in}{2.079761in}}%
\pgfpathlineto{\pgfqpoint{4.157933in}{2.079231in}}%
\pgfpathlineto{\pgfqpoint{4.165871in}{2.087882in}}%
\pgfpathlineto{\pgfqpoint{4.173804in}{2.096496in}}%
\pgfpathlineto{\pgfqpoint{4.181731in}{2.105074in}}%
\pgfpathlineto{\pgfqpoint{4.189652in}{2.113617in}}%
\pgfpathlineto{\pgfqpoint{4.175837in}{2.114190in}}%
\pgfpathlineto{\pgfqpoint{4.162031in}{2.114837in}}%
\pgfpathlineto{\pgfqpoint{4.148233in}{2.115561in}}%
\pgfpathlineto{\pgfqpoint{4.134443in}{2.116360in}}%
\pgfpathlineto{\pgfqpoint{4.126512in}{2.107767in}}%
\pgfpathlineto{\pgfqpoint{4.118574in}{2.099145in}}%
\pgfpathlineto{\pgfqpoint{4.110631in}{2.090492in}}%
\pgfpathlineto{\pgfqpoint{4.102682in}{2.081807in}}%
\pgfpathclose%
\pgfusepath{fill}%
\end{pgfscope}%
\begin{pgfscope}%
\pgfpathrectangle{\pgfqpoint{1.150000in}{0.150000in}}{\pgfqpoint{5.700000in}{5.700000in}}%
\pgfusepath{clip}%
\pgfsetbuttcap%
\pgfsetroundjoin%
\definecolor{currentfill}{rgb}{0.235526,0.309527,0.542944}%
\pgfsetfillcolor{currentfill}%
\pgfsetfillopacity{0.700000}%
\pgfsetlinewidth{0.000000pt}%
\definecolor{currentstroke}{rgb}{0.000000,0.000000,0.000000}%
\pgfsetstrokecolor{currentstroke}%
\pgfsetdash{}{0pt}%
\pgfpathmoveto{\pgfqpoint{5.399227in}{2.498851in}}%
\pgfpathlineto{\pgfqpoint{5.413462in}{2.500494in}}%
\pgfpathlineto{\pgfqpoint{5.427707in}{2.502204in}}%
\pgfpathlineto{\pgfqpoint{5.441964in}{2.503983in}}%
\pgfpathlineto{\pgfqpoint{5.456233in}{2.505830in}}%
\pgfpathlineto{\pgfqpoint{5.463617in}{2.511068in}}%
\pgfpathlineto{\pgfqpoint{5.470997in}{2.516358in}}%
\pgfpathlineto{\pgfqpoint{5.478372in}{2.521706in}}%
\pgfpathlineto{\pgfqpoint{5.485743in}{2.527117in}}%
\pgfpathlineto{\pgfqpoint{5.471497in}{2.525584in}}%
\pgfpathlineto{\pgfqpoint{5.457262in}{2.524119in}}%
\pgfpathlineto{\pgfqpoint{5.443039in}{2.522722in}}%
\pgfpathlineto{\pgfqpoint{5.428827in}{2.521392in}}%
\pgfpathlineto{\pgfqpoint{5.421433in}{2.515660in}}%
\pgfpathlineto{\pgfqpoint{5.414036in}{2.509996in}}%
\pgfpathlineto{\pgfqpoint{5.406634in}{2.504395in}}%
\pgfpathlineto{\pgfqpoint{5.399227in}{2.498851in}}%
\pgfpathclose%
\pgfusepath{fill}%
\end{pgfscope}%
\begin{pgfscope}%
\pgfpathrectangle{\pgfqpoint{1.150000in}{0.150000in}}{\pgfqpoint{5.700000in}{5.700000in}}%
\pgfusepath{clip}%
\pgfsetbuttcap%
\pgfsetroundjoin%
\definecolor{currentfill}{rgb}{0.277018,0.050344,0.375715}%
\pgfsetfillcolor{currentfill}%
\pgfsetfillopacity{0.700000}%
\pgfsetlinewidth{0.000000pt}%
\definecolor{currentstroke}{rgb}{0.000000,0.000000,0.000000}%
\pgfsetstrokecolor{currentstroke}%
\pgfsetdash{}{0pt}%
\pgfpathmoveto{\pgfqpoint{2.814218in}{1.971807in}}%
\pgfpathlineto{\pgfqpoint{2.827812in}{1.964265in}}%
\pgfpathlineto{\pgfqpoint{2.841408in}{1.956825in}}%
\pgfpathlineto{\pgfqpoint{2.855006in}{1.949487in}}%
\pgfpathlineto{\pgfqpoint{2.868606in}{1.942249in}}%
\pgfpathlineto{\pgfqpoint{2.877053in}{1.949099in}}%
\pgfpathlineto{\pgfqpoint{2.885491in}{1.956033in}}%
\pgfpathlineto{\pgfqpoint{2.893920in}{1.963048in}}%
\pgfpathlineto{\pgfqpoint{2.902340in}{1.970142in}}%
\pgfpathlineto{\pgfqpoint{2.888760in}{1.977194in}}%
\pgfpathlineto{\pgfqpoint{2.875181in}{1.984347in}}%
\pgfpathlineto{\pgfqpoint{2.861605in}{1.991601in}}%
\pgfpathlineto{\pgfqpoint{2.848031in}{1.998957in}}%
\pgfpathlineto{\pgfqpoint{2.839591in}{1.992041in}}%
\pgfpathlineto{\pgfqpoint{2.831142in}{1.985209in}}%
\pgfpathlineto{\pgfqpoint{2.822685in}{1.978464in}}%
\pgfpathlineto{\pgfqpoint{2.814218in}{1.971807in}}%
\pgfpathclose%
\pgfusepath{fill}%
\end{pgfscope}%
\begin{pgfscope}%
\pgfpathrectangle{\pgfqpoint{1.150000in}{0.150000in}}{\pgfqpoint{5.700000in}{5.700000in}}%
\pgfusepath{clip}%
\pgfsetbuttcap%
\pgfsetroundjoin%
\definecolor{currentfill}{rgb}{0.282884,0.135920,0.453427}%
\pgfsetfillcolor{currentfill}%
\pgfsetfillopacity{0.700000}%
\pgfsetlinewidth{0.000000pt}%
\definecolor{currentstroke}{rgb}{0.000000,0.000000,0.000000}%
\pgfsetstrokecolor{currentstroke}%
\pgfsetdash{}{0pt}%
\pgfpathmoveto{\pgfqpoint{2.419026in}{2.146141in}}%
\pgfpathlineto{\pgfqpoint{2.432659in}{2.135479in}}%
\pgfpathlineto{\pgfqpoint{2.446290in}{2.124939in}}%
\pgfpathlineto{\pgfqpoint{2.459921in}{2.114520in}}%
\pgfpathlineto{\pgfqpoint{2.473551in}{2.104221in}}%
\pgfpathlineto{\pgfqpoint{2.482219in}{2.108880in}}%
\pgfpathlineto{\pgfqpoint{2.490875in}{2.113677in}}%
\pgfpathlineto{\pgfqpoint{2.499520in}{2.118610in}}%
\pgfpathlineto{\pgfqpoint{2.508153in}{2.123676in}}%
\pgfpathlineto{\pgfqpoint{2.494549in}{2.133744in}}%
\pgfpathlineto{\pgfqpoint{2.480945in}{2.143932in}}%
\pgfpathlineto{\pgfqpoint{2.467340in}{2.154241in}}%
\pgfpathlineto{\pgfqpoint{2.453734in}{2.164672in}}%
\pgfpathlineto{\pgfqpoint{2.445075in}{2.159829in}}%
\pgfpathlineto{\pgfqpoint{2.436404in}{2.155125in}}%
\pgfpathlineto{\pgfqpoint{2.427721in}{2.150561in}}%
\pgfpathlineto{\pgfqpoint{2.419026in}{2.146141in}}%
\pgfpathclose%
\pgfusepath{fill}%
\end{pgfscope}%
\begin{pgfscope}%
\pgfpathrectangle{\pgfqpoint{1.150000in}{0.150000in}}{\pgfqpoint{5.700000in}{5.700000in}}%
\pgfusepath{clip}%
\pgfsetbuttcap%
\pgfsetroundjoin%
\definecolor{currentfill}{rgb}{0.271305,0.019942,0.347269}%
\pgfsetfillcolor{currentfill}%
\pgfsetfillopacity{0.700000}%
\pgfsetlinewidth{0.000000pt}%
\definecolor{currentstroke}{rgb}{0.000000,0.000000,0.000000}%
\pgfsetstrokecolor{currentstroke}%
\pgfsetdash{}{0pt}%
\pgfpathmoveto{\pgfqpoint{3.240956in}{1.914826in}}%
\pgfpathlineto{\pgfqpoint{3.254573in}{1.910065in}}%
\pgfpathlineto{\pgfqpoint{3.268195in}{1.905393in}}%
\pgfpathlineto{\pgfqpoint{3.281821in}{1.900810in}}%
\pgfpathlineto{\pgfqpoint{3.295452in}{1.896315in}}%
\pgfpathlineto{\pgfqpoint{3.303706in}{1.904805in}}%
\pgfpathlineto{\pgfqpoint{3.311953in}{1.913322in}}%
\pgfpathlineto{\pgfqpoint{3.320193in}{1.921865in}}%
\pgfpathlineto{\pgfqpoint{3.328428in}{1.930433in}}%
\pgfpathlineto{\pgfqpoint{3.314811in}{1.934805in}}%
\pgfpathlineto{\pgfqpoint{3.301199in}{1.939266in}}%
\pgfpathlineto{\pgfqpoint{3.287592in}{1.943814in}}%
\pgfpathlineto{\pgfqpoint{3.273989in}{1.948452in}}%
\pgfpathlineto{\pgfqpoint{3.265741in}{1.939999in}}%
\pgfpathlineto{\pgfqpoint{3.257486in}{1.931576in}}%
\pgfpathlineto{\pgfqpoint{3.249224in}{1.923185in}}%
\pgfpathlineto{\pgfqpoint{3.240956in}{1.914826in}}%
\pgfpathclose%
\pgfusepath{fill}%
\end{pgfscope}%
\begin{pgfscope}%
\pgfpathrectangle{\pgfqpoint{1.150000in}{0.150000in}}{\pgfqpoint{5.700000in}{5.700000in}}%
\pgfusepath{clip}%
\pgfsetbuttcap%
\pgfsetroundjoin%
\definecolor{currentfill}{rgb}{0.278012,0.180367,0.486697}%
\pgfsetfillcolor{currentfill}%
\pgfsetfillopacity{0.700000}%
\pgfsetlinewidth{0.000000pt}%
\definecolor{currentstroke}{rgb}{0.000000,0.000000,0.000000}%
\pgfsetstrokecolor{currentstroke}%
\pgfsetdash{}{0pt}%
\pgfpathmoveto{\pgfqpoint{4.505898in}{2.210975in}}%
\pgfpathlineto{\pgfqpoint{4.519827in}{2.211411in}}%
\pgfpathlineto{\pgfqpoint{4.533765in}{2.211919in}}%
\pgfpathlineto{\pgfqpoint{4.547713in}{2.212500in}}%
\pgfpathlineto{\pgfqpoint{4.561670in}{2.213154in}}%
\pgfpathlineto{\pgfqpoint{4.569455in}{2.220811in}}%
\pgfpathlineto{\pgfqpoint{4.577234in}{2.228434in}}%
\pgfpathlineto{\pgfqpoint{4.585006in}{2.236025in}}%
\pgfpathlineto{\pgfqpoint{4.592773in}{2.243586in}}%
\pgfpathlineto{\pgfqpoint{4.578829in}{2.243058in}}%
\pgfpathlineto{\pgfqpoint{4.564894in}{2.242603in}}%
\pgfpathlineto{\pgfqpoint{4.550968in}{2.242219in}}%
\pgfpathlineto{\pgfqpoint{4.537052in}{2.241908in}}%
\pgfpathlineto{\pgfqpoint{4.529272in}{2.234214in}}%
\pgfpathlineto{\pgfqpoint{4.521487in}{2.226496in}}%
\pgfpathlineto{\pgfqpoint{4.513696in}{2.218750in}}%
\pgfpathlineto{\pgfqpoint{4.505898in}{2.210975in}}%
\pgfpathclose%
\pgfusepath{fill}%
\end{pgfscope}%
\begin{pgfscope}%
\pgfpathrectangle{\pgfqpoint{1.150000in}{0.150000in}}{\pgfqpoint{5.700000in}{5.700000in}}%
\pgfusepath{clip}%
\pgfsetbuttcap%
\pgfsetroundjoin%
\definecolor{currentfill}{rgb}{0.277941,0.056324,0.381191}%
\pgfsetfillcolor{currentfill}%
\pgfsetfillopacity{0.700000}%
\pgfsetlinewidth{0.000000pt}%
\definecolor{currentstroke}{rgb}{0.000000,0.000000,0.000000}%
\pgfsetstrokecolor{currentstroke}%
\pgfsetdash{}{0pt}%
\pgfpathmoveto{\pgfqpoint{3.699436in}{1.970140in}}%
\pgfpathlineto{\pgfqpoint{3.713134in}{1.967779in}}%
\pgfpathlineto{\pgfqpoint{3.726838in}{1.965498in}}%
\pgfpathlineto{\pgfqpoint{3.740549in}{1.963297in}}%
\pgfpathlineto{\pgfqpoint{3.754267in}{1.961177in}}%
\pgfpathlineto{\pgfqpoint{3.762350in}{1.970261in}}%
\pgfpathlineto{\pgfqpoint{3.770427in}{1.979328in}}%
\pgfpathlineto{\pgfqpoint{3.778499in}{1.988377in}}%
\pgfpathlineto{\pgfqpoint{3.786564in}{1.997407in}}%
\pgfpathlineto{\pgfqpoint{3.772858in}{1.999487in}}%
\pgfpathlineto{\pgfqpoint{3.759158in}{2.001646in}}%
\pgfpathlineto{\pgfqpoint{3.745465in}{2.003886in}}%
\pgfpathlineto{\pgfqpoint{3.731778in}{2.006207in}}%
\pgfpathlineto{\pgfqpoint{3.723702in}{1.997210in}}%
\pgfpathlineto{\pgfqpoint{3.715619in}{1.988199in}}%
\pgfpathlineto{\pgfqpoint{3.707531in}{1.979176in}}%
\pgfpathlineto{\pgfqpoint{3.699436in}{1.970140in}}%
\pgfpathclose%
\pgfusepath{fill}%
\end{pgfscope}%
\begin{pgfscope}%
\pgfpathrectangle{\pgfqpoint{1.150000in}{0.150000in}}{\pgfqpoint{5.700000in}{5.700000in}}%
\pgfusepath{clip}%
\pgfsetbuttcap%
\pgfsetroundjoin%
\definecolor{currentfill}{rgb}{0.218130,0.347432,0.550038}%
\pgfsetfillcolor{currentfill}%
\pgfsetfillopacity{0.700000}%
\pgfsetlinewidth{0.000000pt}%
\definecolor{currentstroke}{rgb}{0.000000,0.000000,0.000000}%
\pgfsetstrokecolor{currentstroke}%
\pgfsetdash{}{0pt}%
\pgfpathmoveto{\pgfqpoint{5.715828in}{2.588789in}}%
\pgfpathlineto{\pgfqpoint{5.730173in}{2.590508in}}%
\pgfpathlineto{\pgfqpoint{5.744530in}{2.592294in}}%
\pgfpathlineto{\pgfqpoint{5.758898in}{2.594147in}}%
\pgfpathlineto{\pgfqpoint{5.773278in}{2.596067in}}%
\pgfpathlineto{\pgfqpoint{5.780517in}{2.600846in}}%
\pgfpathlineto{\pgfqpoint{5.787753in}{2.605730in}}%
\pgfpathlineto{\pgfqpoint{5.794986in}{2.610726in}}%
\pgfpathlineto{\pgfqpoint{5.802217in}{2.615839in}}%
\pgfpathlineto{\pgfqpoint{5.787864in}{2.614296in}}%
\pgfpathlineto{\pgfqpoint{5.773523in}{2.612819in}}%
\pgfpathlineto{\pgfqpoint{5.759193in}{2.611409in}}%
\pgfpathlineto{\pgfqpoint{5.744874in}{2.610066in}}%
\pgfpathlineto{\pgfqpoint{5.737616in}{2.604570in}}%
\pgfpathlineto{\pgfqpoint{5.730356in}{2.599196in}}%
\pgfpathlineto{\pgfqpoint{5.723093in}{2.593938in}}%
\pgfpathlineto{\pgfqpoint{5.715828in}{2.588789in}}%
\pgfpathclose%
\pgfusepath{fill}%
\end{pgfscope}%
\begin{pgfscope}%
\pgfpathrectangle{\pgfqpoint{1.150000in}{0.150000in}}{\pgfqpoint{5.700000in}{5.700000in}}%
\pgfusepath{clip}%
\pgfsetbuttcap%
\pgfsetroundjoin%
\definecolor{currentfill}{rgb}{0.282656,0.100196,0.422160}%
\pgfsetfillcolor{currentfill}%
\pgfsetfillopacity{0.700000}%
\pgfsetlinewidth{0.000000pt}%
\definecolor{currentstroke}{rgb}{0.000000,0.000000,0.000000}%
\pgfsetstrokecolor{currentstroke}%
\pgfsetdash{}{0pt}%
\pgfpathmoveto{\pgfqpoint{4.015662in}{2.050439in}}%
\pgfpathlineto{\pgfqpoint{4.029441in}{2.049396in}}%
\pgfpathlineto{\pgfqpoint{4.043229in}{2.048431in}}%
\pgfpathlineto{\pgfqpoint{4.057024in}{2.047542in}}%
\pgfpathlineto{\pgfqpoint{4.070828in}{2.046729in}}%
\pgfpathlineto{\pgfqpoint{4.078800in}{2.055551in}}%
\pgfpathlineto{\pgfqpoint{4.086766in}{2.064337in}}%
\pgfpathlineto{\pgfqpoint{4.094727in}{2.073089in}}%
\pgfpathlineto{\pgfqpoint{4.102682in}{2.081807in}}%
\pgfpathlineto{\pgfqpoint{4.088889in}{2.082640in}}%
\pgfpathlineto{\pgfqpoint{4.075104in}{2.083550in}}%
\pgfpathlineto{\pgfqpoint{4.061327in}{2.084537in}}%
\pgfpathlineto{\pgfqpoint{4.047558in}{2.085600in}}%
\pgfpathlineto{\pgfqpoint{4.039593in}{2.076854in}}%
\pgfpathlineto{\pgfqpoint{4.031621in}{2.068078in}}%
\pgfpathlineto{\pgfqpoint{4.023644in}{2.059274in}}%
\pgfpathlineto{\pgfqpoint{4.015662in}{2.050439in}}%
\pgfpathclose%
\pgfusepath{fill}%
\end{pgfscope}%
\begin{pgfscope}%
\pgfpathrectangle{\pgfqpoint{1.150000in}{0.150000in}}{\pgfqpoint{5.700000in}{5.700000in}}%
\pgfusepath{clip}%
\pgfsetbuttcap%
\pgfsetroundjoin%
\definecolor{currentfill}{rgb}{0.262138,0.242286,0.520837}%
\pgfsetfillcolor{currentfill}%
\pgfsetfillopacity{0.700000}%
\pgfsetlinewidth{0.000000pt}%
\definecolor{currentstroke}{rgb}{0.000000,0.000000,0.000000}%
\pgfsetstrokecolor{currentstroke}%
\pgfsetdash{}{0pt}%
\pgfpathmoveto{\pgfqpoint{4.909261in}{2.343250in}}%
\pgfpathlineto{\pgfqpoint{4.923330in}{2.344488in}}%
\pgfpathlineto{\pgfqpoint{4.937410in}{2.345796in}}%
\pgfpathlineto{\pgfqpoint{4.951500in}{2.347175in}}%
\pgfpathlineto{\pgfqpoint{4.965601in}{2.348624in}}%
\pgfpathlineto{\pgfqpoint{4.973214in}{2.355071in}}%
\pgfpathlineto{\pgfqpoint{4.980822in}{2.361507in}}%
\pgfpathlineto{\pgfqpoint{4.988423in}{2.367936in}}%
\pgfpathlineto{\pgfqpoint{4.996018in}{2.374364in}}%
\pgfpathlineto{\pgfqpoint{4.981934in}{2.373124in}}%
\pgfpathlineto{\pgfqpoint{4.967860in}{2.371955in}}%
\pgfpathlineto{\pgfqpoint{4.953796in}{2.370855in}}%
\pgfpathlineto{\pgfqpoint{4.939743in}{2.369826in}}%
\pgfpathlineto{\pgfqpoint{4.932131in}{2.363182in}}%
\pgfpathlineto{\pgfqpoint{4.924514in}{2.356541in}}%
\pgfpathlineto{\pgfqpoint{4.916891in}{2.349898in}}%
\pgfpathlineto{\pgfqpoint{4.909261in}{2.343250in}}%
\pgfpathclose%
\pgfusepath{fill}%
\end{pgfscope}%
\begin{pgfscope}%
\pgfpathrectangle{\pgfqpoint{1.150000in}{0.150000in}}{\pgfqpoint{5.700000in}{5.700000in}}%
\pgfusepath{clip}%
\pgfsetbuttcap%
\pgfsetroundjoin%
\definecolor{currentfill}{rgb}{0.241237,0.296485,0.539709}%
\pgfsetfillcolor{currentfill}%
\pgfsetfillopacity{0.700000}%
\pgfsetlinewidth{0.000000pt}%
\definecolor{currentstroke}{rgb}{0.000000,0.000000,0.000000}%
\pgfsetstrokecolor{currentstroke}%
\pgfsetdash{}{0pt}%
\pgfpathmoveto{\pgfqpoint{5.312640in}{2.470080in}}%
\pgfpathlineto{\pgfqpoint{5.326851in}{2.471743in}}%
\pgfpathlineto{\pgfqpoint{5.341074in}{2.473474in}}%
\pgfpathlineto{\pgfqpoint{5.355307in}{2.475274in}}%
\pgfpathlineto{\pgfqpoint{5.369552in}{2.477142in}}%
\pgfpathlineto{\pgfqpoint{5.376978in}{2.482510in}}%
\pgfpathlineto{\pgfqpoint{5.384400in}{2.487914in}}%
\pgfpathlineto{\pgfqpoint{5.391816in}{2.493359in}}%
\pgfpathlineto{\pgfqpoint{5.399227in}{2.498851in}}%
\pgfpathlineto{\pgfqpoint{5.385004in}{2.497276in}}%
\pgfpathlineto{\pgfqpoint{5.370792in}{2.495770in}}%
\pgfpathlineto{\pgfqpoint{5.356590in}{2.494332in}}%
\pgfpathlineto{\pgfqpoint{5.342400in}{2.492962in}}%
\pgfpathlineto{\pgfqpoint{5.334967in}{2.487169in}}%
\pgfpathlineto{\pgfqpoint{5.327530in}{2.481429in}}%
\pgfpathlineto{\pgfqpoint{5.320088in}{2.475734in}}%
\pgfpathlineto{\pgfqpoint{5.312640in}{2.470080in}}%
\pgfpathclose%
\pgfusepath{fill}%
\end{pgfscope}%
\begin{pgfscope}%
\pgfpathrectangle{\pgfqpoint{1.150000in}{0.150000in}}{\pgfqpoint{5.700000in}{5.700000in}}%
\pgfusepath{clip}%
\pgfsetbuttcap%
\pgfsetroundjoin%
\definecolor{currentfill}{rgb}{0.280255,0.165693,0.476498}%
\pgfsetfillcolor{currentfill}%
\pgfsetfillopacity{0.700000}%
\pgfsetlinewidth{0.000000pt}%
\definecolor{currentstroke}{rgb}{0.000000,0.000000,0.000000}%
\pgfsetstrokecolor{currentstroke}%
\pgfsetdash{}{0pt}%
\pgfpathmoveto{\pgfqpoint{4.418975in}{2.178104in}}%
\pgfpathlineto{\pgfqpoint{4.432880in}{2.178352in}}%
\pgfpathlineto{\pgfqpoint{4.446793in}{2.178673in}}%
\pgfpathlineto{\pgfqpoint{4.460715in}{2.179068in}}%
\pgfpathlineto{\pgfqpoint{4.474647in}{2.179535in}}%
\pgfpathlineto{\pgfqpoint{4.482469in}{2.187451in}}%
\pgfpathlineto{\pgfqpoint{4.490285in}{2.195328in}}%
\pgfpathlineto{\pgfqpoint{4.498095in}{2.203169in}}%
\pgfpathlineto{\pgfqpoint{4.505898in}{2.210975in}}%
\pgfpathlineto{\pgfqpoint{4.491978in}{2.210612in}}%
\pgfpathlineto{\pgfqpoint{4.478068in}{2.210322in}}%
\pgfpathlineto{\pgfqpoint{4.464167in}{2.210105in}}%
\pgfpathlineto{\pgfqpoint{4.450274in}{2.209962in}}%
\pgfpathlineto{\pgfqpoint{4.442459in}{2.202043in}}%
\pgfpathlineto{\pgfqpoint{4.434637in}{2.194095in}}%
\pgfpathlineto{\pgfqpoint{4.426809in}{2.186116in}}%
\pgfpathlineto{\pgfqpoint{4.418975in}{2.178104in}}%
\pgfpathclose%
\pgfusepath{fill}%
\end{pgfscope}%
\begin{pgfscope}%
\pgfpathrectangle{\pgfqpoint{1.150000in}{0.150000in}}{\pgfqpoint{5.700000in}{5.700000in}}%
\pgfusepath{clip}%
\pgfsetbuttcap%
\pgfsetroundjoin%
\definecolor{currentfill}{rgb}{0.280267,0.073417,0.397163}%
\pgfsetfillcolor{currentfill}%
\pgfsetfillopacity{0.700000}%
\pgfsetlinewidth{0.000000pt}%
\definecolor{currentstroke}{rgb}{0.000000,0.000000,0.000000}%
\pgfsetstrokecolor{currentstroke}%
\pgfsetdash{}{0pt}%
\pgfpathmoveto{\pgfqpoint{2.671385in}{2.011874in}}%
\pgfpathlineto{\pgfqpoint{2.684990in}{2.003283in}}%
\pgfpathlineto{\pgfqpoint{2.698596in}{1.994800in}}%
\pgfpathlineto{\pgfqpoint{2.712203in}{1.986424in}}%
\pgfpathlineto{\pgfqpoint{2.725812in}{1.978154in}}%
\pgfpathlineto{\pgfqpoint{2.734338in}{1.984213in}}%
\pgfpathlineto{\pgfqpoint{2.742854in}{1.990377in}}%
\pgfpathlineto{\pgfqpoint{2.751361in}{1.996643in}}%
\pgfpathlineto{\pgfqpoint{2.759858in}{2.003009in}}%
\pgfpathlineto{\pgfqpoint{2.746272in}{2.011072in}}%
\pgfpathlineto{\pgfqpoint{2.732687in}{2.019241in}}%
\pgfpathlineto{\pgfqpoint{2.719103in}{2.027517in}}%
\pgfpathlineto{\pgfqpoint{2.705520in}{2.035900in}}%
\pgfpathlineto{\pgfqpoint{2.697001in}{2.029734in}}%
\pgfpathlineto{\pgfqpoint{2.688472in}{2.023672in}}%
\pgfpathlineto{\pgfqpoint{2.679934in}{2.017718in}}%
\pgfpathlineto{\pgfqpoint{2.671385in}{2.011874in}}%
\pgfpathclose%
\pgfusepath{fill}%
\end{pgfscope}%
\begin{pgfscope}%
\pgfpathrectangle{\pgfqpoint{1.150000in}{0.150000in}}{\pgfqpoint{5.700000in}{5.700000in}}%
\pgfusepath{clip}%
\pgfsetbuttcap%
\pgfsetroundjoin%
\definecolor{currentfill}{rgb}{0.272594,0.025563,0.353093}%
\pgfsetfillcolor{currentfill}%
\pgfsetfillopacity{0.700000}%
\pgfsetlinewidth{0.000000pt}%
\definecolor{currentstroke}{rgb}{0.000000,0.000000,0.000000}%
\pgfsetstrokecolor{currentstroke}%
\pgfsetdash{}{0pt}%
\pgfpathmoveto{\pgfqpoint{3.382942in}{1.913818in}}%
\pgfpathlineto{\pgfqpoint{3.396583in}{1.909880in}}%
\pgfpathlineto{\pgfqpoint{3.410229in}{1.906028in}}%
\pgfpathlineto{\pgfqpoint{3.423881in}{1.902262in}}%
\pgfpathlineto{\pgfqpoint{3.437538in}{1.898580in}}%
\pgfpathlineto{\pgfqpoint{3.445739in}{1.907392in}}%
\pgfpathlineto{\pgfqpoint{3.453933in}{1.916215in}}%
\pgfpathlineto{\pgfqpoint{3.462122in}{1.925049in}}%
\pgfpathlineto{\pgfqpoint{3.470304in}{1.933892in}}%
\pgfpathlineto{\pgfqpoint{3.456661in}{1.937471in}}%
\pgfpathlineto{\pgfqpoint{3.443022in}{1.941135in}}%
\pgfpathlineto{\pgfqpoint{3.429389in}{1.944885in}}%
\pgfpathlineto{\pgfqpoint{3.415761in}{1.948720in}}%
\pgfpathlineto{\pgfqpoint{3.407566in}{1.939971in}}%
\pgfpathlineto{\pgfqpoint{3.399364in}{1.931238in}}%
\pgfpathlineto{\pgfqpoint{3.391156in}{1.922519in}}%
\pgfpathlineto{\pgfqpoint{3.382942in}{1.913818in}}%
\pgfpathclose%
\pgfusepath{fill}%
\end{pgfscope}%
\begin{pgfscope}%
\pgfpathrectangle{\pgfqpoint{1.150000in}{0.150000in}}{\pgfqpoint{5.700000in}{5.700000in}}%
\pgfusepath{clip}%
\pgfsetbuttcap%
\pgfsetroundjoin%
\definecolor{currentfill}{rgb}{0.276022,0.044167,0.370164}%
\pgfsetfillcolor{currentfill}%
\pgfsetfillopacity{0.700000}%
\pgfsetlinewidth{0.000000pt}%
\definecolor{currentstroke}{rgb}{0.000000,0.000000,0.000000}%
\pgfsetstrokecolor{currentstroke}%
\pgfsetdash{}{0pt}%
\pgfpathmoveto{\pgfqpoint{3.612230in}{1.944378in}}%
\pgfpathlineto{\pgfqpoint{3.625914in}{1.941630in}}%
\pgfpathlineto{\pgfqpoint{3.639604in}{1.938964in}}%
\pgfpathlineto{\pgfqpoint{3.653300in}{1.936379in}}%
\pgfpathlineto{\pgfqpoint{3.667003in}{1.933876in}}%
\pgfpathlineto{\pgfqpoint{3.675120in}{1.942959in}}%
\pgfpathlineto{\pgfqpoint{3.683231in}{1.952031in}}%
\pgfpathlineto{\pgfqpoint{3.691337in}{1.961092in}}%
\pgfpathlineto{\pgfqpoint{3.699436in}{1.970140in}}%
\pgfpathlineto{\pgfqpoint{3.685745in}{1.972582in}}%
\pgfpathlineto{\pgfqpoint{3.672060in}{1.975105in}}%
\pgfpathlineto{\pgfqpoint{3.658382in}{1.977710in}}%
\pgfpathlineto{\pgfqpoint{3.644710in}{1.980396in}}%
\pgfpathlineto{\pgfqpoint{3.636599in}{1.971402in}}%
\pgfpathlineto{\pgfqpoint{3.628482in}{1.962400in}}%
\pgfpathlineto{\pgfqpoint{3.620359in}{1.953392in}}%
\pgfpathlineto{\pgfqpoint{3.612230in}{1.944378in}}%
\pgfpathclose%
\pgfusepath{fill}%
\end{pgfscope}%
\begin{pgfscope}%
\pgfpathrectangle{\pgfqpoint{1.150000in}{0.150000in}}{\pgfqpoint{5.700000in}{5.700000in}}%
\pgfusepath{clip}%
\pgfsetbuttcap%
\pgfsetroundjoin%
\definecolor{currentfill}{rgb}{0.283229,0.120777,0.440584}%
\pgfsetfillcolor{currentfill}%
\pgfsetfillopacity{0.700000}%
\pgfsetlinewidth{0.000000pt}%
\definecolor{currentstroke}{rgb}{0.000000,0.000000,0.000000}%
\pgfsetstrokecolor{currentstroke}%
\pgfsetdash{}{0pt}%
\pgfpathmoveto{\pgfqpoint{2.473551in}{2.104221in}}%
\pgfpathlineto{\pgfqpoint{2.487180in}{2.094042in}}%
\pgfpathlineto{\pgfqpoint{2.500808in}{2.083980in}}%
\pgfpathlineto{\pgfqpoint{2.514436in}{2.074036in}}%
\pgfpathlineto{\pgfqpoint{2.528064in}{2.064208in}}%
\pgfpathlineto{\pgfqpoint{2.536705in}{2.069104in}}%
\pgfpathlineto{\pgfqpoint{2.545336in}{2.074134in}}%
\pgfpathlineto{\pgfqpoint{2.553955in}{2.079294in}}%
\pgfpathlineto{\pgfqpoint{2.562562in}{2.084582in}}%
\pgfpathlineto{\pgfqpoint{2.548960in}{2.094180in}}%
\pgfpathlineto{\pgfqpoint{2.535358in}{2.103895in}}%
\pgfpathlineto{\pgfqpoint{2.521756in}{2.113726in}}%
\pgfpathlineto{\pgfqpoint{2.508153in}{2.123676in}}%
\pgfpathlineto{\pgfqpoint{2.499520in}{2.118610in}}%
\pgfpathlineto{\pgfqpoint{2.490875in}{2.113677in}}%
\pgfpathlineto{\pgfqpoint{2.482219in}{2.108880in}}%
\pgfpathlineto{\pgfqpoint{2.473551in}{2.104221in}}%
\pgfpathclose%
\pgfusepath{fill}%
\end{pgfscope}%
\begin{pgfscope}%
\pgfpathrectangle{\pgfqpoint{1.150000in}{0.150000in}}{\pgfqpoint{5.700000in}{5.700000in}}%
\pgfusepath{clip}%
\pgfsetbuttcap%
\pgfsetroundjoin%
\definecolor{currentfill}{rgb}{0.269308,0.218818,0.509577}%
\pgfsetfillcolor{currentfill}%
\pgfsetfillopacity{0.700000}%
\pgfsetlinewidth{0.000000pt}%
\definecolor{currentstroke}{rgb}{0.000000,0.000000,0.000000}%
\pgfsetstrokecolor{currentstroke}%
\pgfsetdash{}{0pt}%
\pgfpathmoveto{\pgfqpoint{2.165249in}{2.322113in}}%
\pgfpathlineto{\pgfqpoint{2.178950in}{2.309093in}}%
\pgfpathlineto{\pgfqpoint{2.192647in}{2.296213in}}%
\pgfpathlineto{\pgfqpoint{2.206341in}{2.283473in}}%
\pgfpathlineto{\pgfqpoint{2.220032in}{2.270870in}}%
\pgfpathlineto{\pgfqpoint{2.228870in}{2.273891in}}%
\pgfpathlineto{\pgfqpoint{2.237693in}{2.277087in}}%
\pgfpathlineto{\pgfqpoint{2.246502in}{2.280454in}}%
\pgfpathlineto{\pgfqpoint{2.255298in}{2.283988in}}%
\pgfpathlineto{\pgfqpoint{2.241638in}{2.296334in}}%
\pgfpathlineto{\pgfqpoint{2.227976in}{2.308819in}}%
\pgfpathlineto{\pgfqpoint{2.214310in}{2.321442in}}%
\pgfpathlineto{\pgfqpoint{2.200641in}{2.334206in}}%
\pgfpathlineto{\pgfqpoint{2.191815in}{2.330920in}}%
\pgfpathlineto{\pgfqpoint{2.182974in}{2.327806in}}%
\pgfpathlineto{\pgfqpoint{2.174119in}{2.324869in}}%
\pgfpathlineto{\pgfqpoint{2.165249in}{2.322113in}}%
\pgfpathclose%
\pgfusepath{fill}%
\end{pgfscope}%
\begin{pgfscope}%
\pgfpathrectangle{\pgfqpoint{1.150000in}{0.150000in}}{\pgfqpoint{5.700000in}{5.700000in}}%
\pgfusepath{clip}%
\pgfsetbuttcap%
\pgfsetroundjoin%
\definecolor{currentfill}{rgb}{0.281924,0.089666,0.412415}%
\pgfsetfillcolor{currentfill}%
\pgfsetfillopacity{0.700000}%
\pgfsetlinewidth{0.000000pt}%
\definecolor{currentstroke}{rgb}{0.000000,0.000000,0.000000}%
\pgfsetstrokecolor{currentstroke}%
\pgfsetdash{}{0pt}%
\pgfpathmoveto{\pgfqpoint{3.928589in}{2.019721in}}%
\pgfpathlineto{\pgfqpoint{3.942349in}{2.018370in}}%
\pgfpathlineto{\pgfqpoint{3.956117in}{2.017097in}}%
\pgfpathlineto{\pgfqpoint{3.969892in}{2.015901in}}%
\pgfpathlineto{\pgfqpoint{3.983674in}{2.014782in}}%
\pgfpathlineto{\pgfqpoint{3.991680in}{2.023745in}}%
\pgfpathlineto{\pgfqpoint{3.999679in}{2.032675in}}%
\pgfpathlineto{\pgfqpoint{4.007673in}{2.041572in}}%
\pgfpathlineto{\pgfqpoint{4.015662in}{2.050439in}}%
\pgfpathlineto{\pgfqpoint{4.001890in}{2.051558in}}%
\pgfpathlineto{\pgfqpoint{3.988125in}{2.052754in}}%
\pgfpathlineto{\pgfqpoint{3.974369in}{2.054028in}}%
\pgfpathlineto{\pgfqpoint{3.960619in}{2.055379in}}%
\pgfpathlineto{\pgfqpoint{3.952620in}{2.046505in}}%
\pgfpathlineto{\pgfqpoint{3.944616in}{2.037604in}}%
\pgfpathlineto{\pgfqpoint{3.936605in}{2.028676in}}%
\pgfpathlineto{\pgfqpoint{3.928589in}{2.019721in}}%
\pgfpathclose%
\pgfusepath{fill}%
\end{pgfscope}%
\begin{pgfscope}%
\pgfpathrectangle{\pgfqpoint{1.150000in}{0.150000in}}{\pgfqpoint{5.700000in}{5.700000in}}%
\pgfusepath{clip}%
\pgfsetbuttcap%
\pgfsetroundjoin%
\definecolor{currentfill}{rgb}{0.266580,0.228262,0.514349}%
\pgfsetfillcolor{currentfill}%
\pgfsetfillopacity{0.700000}%
\pgfsetlinewidth{0.000000pt}%
\definecolor{currentstroke}{rgb}{0.000000,0.000000,0.000000}%
\pgfsetstrokecolor{currentstroke}%
\pgfsetdash{}{0pt}%
\pgfpathmoveto{\pgfqpoint{4.822444in}{2.311531in}}%
\pgfpathlineto{\pgfqpoint{4.836488in}{2.312675in}}%
\pgfpathlineto{\pgfqpoint{4.850542in}{2.313890in}}%
\pgfpathlineto{\pgfqpoint{4.864606in}{2.315176in}}%
\pgfpathlineto{\pgfqpoint{4.878681in}{2.316532in}}%
\pgfpathlineto{\pgfqpoint{4.886335in}{2.323238in}}%
\pgfpathlineto{\pgfqpoint{4.893983in}{2.329923in}}%
\pgfpathlineto{\pgfqpoint{4.901625in}{2.336593in}}%
\pgfpathlineto{\pgfqpoint{4.909261in}{2.343250in}}%
\pgfpathlineto{\pgfqpoint{4.895202in}{2.342082in}}%
\pgfpathlineto{\pgfqpoint{4.881153in}{2.340985in}}%
\pgfpathlineto{\pgfqpoint{4.867114in}{2.339958in}}%
\pgfpathlineto{\pgfqpoint{4.853085in}{2.339002in}}%
\pgfpathlineto{\pgfqpoint{4.845434in}{2.332149in}}%
\pgfpathlineto{\pgfqpoint{4.837777in}{2.325288in}}%
\pgfpathlineto{\pgfqpoint{4.830114in}{2.318417in}}%
\pgfpathlineto{\pgfqpoint{4.822444in}{2.311531in}}%
\pgfpathclose%
\pgfusepath{fill}%
\end{pgfscope}%
\begin{pgfscope}%
\pgfpathrectangle{\pgfqpoint{1.150000in}{0.150000in}}{\pgfqpoint{5.700000in}{5.700000in}}%
\pgfusepath{clip}%
\pgfsetbuttcap%
\pgfsetroundjoin%
\definecolor{currentfill}{rgb}{0.221989,0.339161,0.548752}%
\pgfsetfillcolor{currentfill}%
\pgfsetfillopacity{0.700000}%
\pgfsetlinewidth{0.000000pt}%
\definecolor{currentstroke}{rgb}{0.000000,0.000000,0.000000}%
\pgfsetstrokecolor{currentstroke}%
\pgfsetdash{}{0pt}%
\pgfpathmoveto{\pgfqpoint{5.629370in}{2.561536in}}%
\pgfpathlineto{\pgfqpoint{5.643694in}{2.563342in}}%
\pgfpathlineto{\pgfqpoint{5.658030in}{2.565216in}}%
\pgfpathlineto{\pgfqpoint{5.672377in}{2.567157in}}%
\pgfpathlineto{\pgfqpoint{5.686737in}{2.569165in}}%
\pgfpathlineto{\pgfqpoint{5.694015in}{2.573939in}}%
\pgfpathlineto{\pgfqpoint{5.701289in}{2.578796in}}%
\pgfpathlineto{\pgfqpoint{5.708560in}{2.583744in}}%
\pgfpathlineto{\pgfqpoint{5.715828in}{2.588789in}}%
\pgfpathlineto{\pgfqpoint{5.701495in}{2.587138in}}%
\pgfpathlineto{\pgfqpoint{5.687173in}{2.585553in}}%
\pgfpathlineto{\pgfqpoint{5.672862in}{2.584035in}}%
\pgfpathlineto{\pgfqpoint{5.658563in}{2.582584in}}%
\pgfpathlineto{\pgfqpoint{5.651270in}{2.577176in}}%
\pgfpathlineto{\pgfqpoint{5.643973in}{2.571869in}}%
\pgfpathlineto{\pgfqpoint{5.636673in}{2.566658in}}%
\pgfpathlineto{\pgfqpoint{5.629370in}{2.561536in}}%
\pgfpathclose%
\pgfusepath{fill}%
\end{pgfscope}%
\begin{pgfscope}%
\pgfpathrectangle{\pgfqpoint{1.150000in}{0.150000in}}{\pgfqpoint{5.700000in}{5.700000in}}%
\pgfusepath{clip}%
\pgfsetbuttcap%
\pgfsetroundjoin%
\definecolor{currentfill}{rgb}{0.281887,0.150881,0.465405}%
\pgfsetfillcolor{currentfill}%
\pgfsetfillopacity{0.700000}%
\pgfsetlinewidth{0.000000pt}%
\definecolor{currentstroke}{rgb}{0.000000,0.000000,0.000000}%
\pgfsetstrokecolor{currentstroke}%
\pgfsetdash{}{0pt}%
\pgfpathmoveto{\pgfqpoint{4.332006in}{2.145092in}}%
\pgfpathlineto{\pgfqpoint{4.345886in}{2.145129in}}%
\pgfpathlineto{\pgfqpoint{4.359775in}{2.145240in}}%
\pgfpathlineto{\pgfqpoint{4.373673in}{2.145425in}}%
\pgfpathlineto{\pgfqpoint{4.387580in}{2.145683in}}%
\pgfpathlineto{\pgfqpoint{4.395438in}{2.153848in}}%
\pgfpathlineto{\pgfqpoint{4.403290in}{2.161972in}}%
\pgfpathlineto{\pgfqpoint{4.411136in}{2.170057in}}%
\pgfpathlineto{\pgfqpoint{4.418975in}{2.178104in}}%
\pgfpathlineto{\pgfqpoint{4.405080in}{2.177930in}}%
\pgfpathlineto{\pgfqpoint{4.391194in}{2.177829in}}%
\pgfpathlineto{\pgfqpoint{4.377316in}{2.177801in}}%
\pgfpathlineto{\pgfqpoint{4.363448in}{2.177847in}}%
\pgfpathlineto{\pgfqpoint{4.355596in}{2.169709in}}%
\pgfpathlineto{\pgfqpoint{4.347739in}{2.161538in}}%
\pgfpathlineto{\pgfqpoint{4.339876in}{2.153333in}}%
\pgfpathlineto{\pgfqpoint{4.332006in}{2.145092in}}%
\pgfpathclose%
\pgfusepath{fill}%
\end{pgfscope}%
\begin{pgfscope}%
\pgfpathrectangle{\pgfqpoint{1.150000in}{0.150000in}}{\pgfqpoint{5.700000in}{5.700000in}}%
\pgfusepath{clip}%
\pgfsetbuttcap%
\pgfsetroundjoin%
\definecolor{currentfill}{rgb}{0.272594,0.025563,0.353093}%
\pgfsetfillcolor{currentfill}%
\pgfsetfillopacity{0.700000}%
\pgfsetlinewidth{0.000000pt}%
\definecolor{currentstroke}{rgb}{0.000000,0.000000,0.000000}%
\pgfsetstrokecolor{currentstroke}%
\pgfsetdash{}{0pt}%
\pgfpathmoveto{\pgfqpoint{3.011076in}{1.917277in}}%
\pgfpathlineto{\pgfqpoint{3.024681in}{1.911104in}}%
\pgfpathlineto{\pgfqpoint{3.038289in}{1.905026in}}%
\pgfpathlineto{\pgfqpoint{3.051900in}{1.899043in}}%
\pgfpathlineto{\pgfqpoint{3.065514in}{1.893153in}}%
\pgfpathlineto{\pgfqpoint{3.073872in}{1.900839in}}%
\pgfpathlineto{\pgfqpoint{3.082222in}{1.908585in}}%
\pgfpathlineto{\pgfqpoint{3.090565in}{1.916387in}}%
\pgfpathlineto{\pgfqpoint{3.098900in}{1.924243in}}%
\pgfpathlineto{\pgfqpoint{3.085302in}{1.929968in}}%
\pgfpathlineto{\pgfqpoint{3.071708in}{1.935788in}}%
\pgfpathlineto{\pgfqpoint{3.058118in}{1.941702in}}%
\pgfpathlineto{\pgfqpoint{3.044530in}{1.947710in}}%
\pgfpathlineto{\pgfqpoint{3.036178in}{1.940011in}}%
\pgfpathlineto{\pgfqpoint{3.027819in}{1.932371in}}%
\pgfpathlineto{\pgfqpoint{3.019451in}{1.924792in}}%
\pgfpathlineto{\pgfqpoint{3.011076in}{1.917277in}}%
\pgfpathclose%
\pgfusepath{fill}%
\end{pgfscope}%
\begin{pgfscope}%
\pgfpathrectangle{\pgfqpoint{1.150000in}{0.150000in}}{\pgfqpoint{5.700000in}{5.700000in}}%
\pgfusepath{clip}%
\pgfsetbuttcap%
\pgfsetroundjoin%
\definecolor{currentfill}{rgb}{0.244972,0.287675,0.537260}%
\pgfsetfillcolor{currentfill}%
\pgfsetfillopacity{0.700000}%
\pgfsetlinewidth{0.000000pt}%
\definecolor{currentstroke}{rgb}{0.000000,0.000000,0.000000}%
\pgfsetstrokecolor{currentstroke}%
\pgfsetdash{}{0pt}%
\pgfpathmoveto{\pgfqpoint{5.225982in}{2.440722in}}%
\pgfpathlineto{\pgfqpoint{5.240169in}{2.442382in}}%
\pgfpathlineto{\pgfqpoint{5.254367in}{2.444111in}}%
\pgfpathlineto{\pgfqpoint{5.268576in}{2.445909in}}%
\pgfpathlineto{\pgfqpoint{5.282797in}{2.447776in}}%
\pgfpathlineto{\pgfqpoint{5.290266in}{2.453316in}}%
\pgfpathlineto{\pgfqpoint{5.297729in}{2.458876in}}%
\pgfpathlineto{\pgfqpoint{5.305188in}{2.464463in}}%
\pgfpathlineto{\pgfqpoint{5.312640in}{2.470080in}}%
\pgfpathlineto{\pgfqpoint{5.298440in}{2.468486in}}%
\pgfpathlineto{\pgfqpoint{5.284251in}{2.466961in}}%
\pgfpathlineto{\pgfqpoint{5.270073in}{2.465504in}}%
\pgfpathlineto{\pgfqpoint{5.255905in}{2.464115in}}%
\pgfpathlineto{\pgfqpoint{5.248433in}{2.458218in}}%
\pgfpathlineto{\pgfqpoint{5.240955in}{2.452357in}}%
\pgfpathlineto{\pgfqpoint{5.233471in}{2.446526in}}%
\pgfpathlineto{\pgfqpoint{5.225982in}{2.440722in}}%
\pgfpathclose%
\pgfusepath{fill}%
\end{pgfscope}%
\begin{pgfscope}%
\pgfpathrectangle{\pgfqpoint{1.150000in}{0.150000in}}{\pgfqpoint{5.700000in}{5.700000in}}%
\pgfusepath{clip}%
\pgfsetbuttcap%
\pgfsetroundjoin%
\definecolor{currentfill}{rgb}{0.274952,0.037752,0.364543}%
\pgfsetfillcolor{currentfill}%
\pgfsetfillopacity{0.700000}%
\pgfsetlinewidth{0.000000pt}%
\definecolor{currentstroke}{rgb}{0.000000,0.000000,0.000000}%
\pgfsetstrokecolor{currentstroke}%
\pgfsetdash{}{0pt}%
\pgfpathmoveto{\pgfqpoint{2.868606in}{1.942249in}}%
\pgfpathlineto{\pgfqpoint{2.882209in}{1.935112in}}%
\pgfpathlineto{\pgfqpoint{2.895813in}{1.928074in}}%
\pgfpathlineto{\pgfqpoint{2.909421in}{1.921135in}}%
\pgfpathlineto{\pgfqpoint{2.923030in}{1.914294in}}%
\pgfpathlineto{\pgfqpoint{2.931457in}{1.921337in}}%
\pgfpathlineto{\pgfqpoint{2.939875in}{1.928459in}}%
\pgfpathlineto{\pgfqpoint{2.948285in}{1.935657in}}%
\pgfpathlineto{\pgfqpoint{2.956686in}{1.942929in}}%
\pgfpathlineto{\pgfqpoint{2.943096in}{1.949584in}}%
\pgfpathlineto{\pgfqpoint{2.929508in}{1.956338in}}%
\pgfpathlineto{\pgfqpoint{2.915923in}{1.963190in}}%
\pgfpathlineto{\pgfqpoint{2.902340in}{1.970142in}}%
\pgfpathlineto{\pgfqpoint{2.893920in}{1.963048in}}%
\pgfpathlineto{\pgfqpoint{2.885491in}{1.956033in}}%
\pgfpathlineto{\pgfqpoint{2.877053in}{1.949099in}}%
\pgfpathlineto{\pgfqpoint{2.868606in}{1.942249in}}%
\pgfpathclose%
\pgfusepath{fill}%
\end{pgfscope}%
\begin{pgfscope}%
\pgfpathrectangle{\pgfqpoint{1.150000in}{0.150000in}}{\pgfqpoint{5.700000in}{5.700000in}}%
\pgfusepath{clip}%
\pgfsetbuttcap%
\pgfsetroundjoin%
\definecolor{currentfill}{rgb}{0.271305,0.019942,0.347269}%
\pgfsetfillcolor{currentfill}%
\pgfsetfillopacity{0.700000}%
\pgfsetlinewidth{0.000000pt}%
\definecolor{currentstroke}{rgb}{0.000000,0.000000,0.000000}%
\pgfsetstrokecolor{currentstroke}%
\pgfsetdash{}{0pt}%
\pgfpathmoveto{\pgfqpoint{3.153326in}{1.902267in}}%
\pgfpathlineto{\pgfqpoint{3.166942in}{1.897002in}}%
\pgfpathlineto{\pgfqpoint{3.180562in}{1.891829in}}%
\pgfpathlineto{\pgfqpoint{3.194186in}{1.886745in}}%
\pgfpathlineto{\pgfqpoint{3.207815in}{1.881752in}}%
\pgfpathlineto{\pgfqpoint{3.216110in}{1.889962in}}%
\pgfpathlineto{\pgfqpoint{3.224399in}{1.898213in}}%
\pgfpathlineto{\pgfqpoint{3.232681in}{1.906501in}}%
\pgfpathlineto{\pgfqpoint{3.240956in}{1.914826in}}%
\pgfpathlineto{\pgfqpoint{3.227343in}{1.919676in}}%
\pgfpathlineto{\pgfqpoint{3.213734in}{1.924616in}}%
\pgfpathlineto{\pgfqpoint{3.200130in}{1.929646in}}%
\pgfpathlineto{\pgfqpoint{3.186530in}{1.934767in}}%
\pgfpathlineto{\pgfqpoint{3.178239in}{1.926578in}}%
\pgfpathlineto{\pgfqpoint{3.169942in}{1.918431in}}%
\pgfpathlineto{\pgfqpoint{3.161638in}{1.910326in}}%
\pgfpathlineto{\pgfqpoint{3.153326in}{1.902267in}}%
\pgfpathclose%
\pgfusepath{fill}%
\end{pgfscope}%
\begin{pgfscope}%
\pgfpathrectangle{\pgfqpoint{1.150000in}{0.150000in}}{\pgfqpoint{5.700000in}{5.700000in}}%
\pgfusepath{clip}%
\pgfsetbuttcap%
\pgfsetroundjoin%
\definecolor{currentfill}{rgb}{0.274128,0.199721,0.498911}%
\pgfsetfillcolor{currentfill}%
\pgfsetfillopacity{0.700000}%
\pgfsetlinewidth{0.000000pt}%
\definecolor{currentstroke}{rgb}{0.000000,0.000000,0.000000}%
\pgfsetstrokecolor{currentstroke}%
\pgfsetdash{}{0pt}%
\pgfpathmoveto{\pgfqpoint{2.220032in}{2.270870in}}%
\pgfpathlineto{\pgfqpoint{2.233720in}{2.258405in}}%
\pgfpathlineto{\pgfqpoint{2.247404in}{2.246075in}}%
\pgfpathlineto{\pgfqpoint{2.261086in}{2.233880in}}%
\pgfpathlineto{\pgfqpoint{2.274765in}{2.221817in}}%
\pgfpathlineto{\pgfqpoint{2.283571in}{2.225101in}}%
\pgfpathlineto{\pgfqpoint{2.292364in}{2.228555in}}%
\pgfpathlineto{\pgfqpoint{2.301143in}{2.232174in}}%
\pgfpathlineto{\pgfqpoint{2.309908in}{2.235955in}}%
\pgfpathlineto{\pgfqpoint{2.296259in}{2.247763in}}%
\pgfpathlineto{\pgfqpoint{2.282608in}{2.259703in}}%
\pgfpathlineto{\pgfqpoint{2.268954in}{2.271778in}}%
\pgfpathlineto{\pgfqpoint{2.255298in}{2.283988in}}%
\pgfpathlineto{\pgfqpoint{2.246502in}{2.280454in}}%
\pgfpathlineto{\pgfqpoint{2.237693in}{2.277087in}}%
\pgfpathlineto{\pgfqpoint{2.228870in}{2.273891in}}%
\pgfpathlineto{\pgfqpoint{2.220032in}{2.270870in}}%
\pgfpathclose%
\pgfusepath{fill}%
\end{pgfscope}%
\begin{pgfscope}%
\pgfpathrectangle{\pgfqpoint{1.150000in}{0.150000in}}{\pgfqpoint{5.700000in}{5.700000in}}%
\pgfusepath{clip}%
\pgfsetbuttcap%
\pgfsetroundjoin%
\definecolor{currentfill}{rgb}{0.269308,0.218818,0.509577}%
\pgfsetfillcolor{currentfill}%
\pgfsetfillopacity{0.700000}%
\pgfsetlinewidth{0.000000pt}%
\definecolor{currentstroke}{rgb}{0.000000,0.000000,0.000000}%
\pgfsetstrokecolor{currentstroke}%
\pgfsetdash{}{0pt}%
\pgfpathmoveto{\pgfqpoint{4.735571in}{2.279237in}}%
\pgfpathlineto{\pgfqpoint{4.749590in}{2.280264in}}%
\pgfpathlineto{\pgfqpoint{4.763618in}{2.281363in}}%
\pgfpathlineto{\pgfqpoint{4.777656in}{2.282533in}}%
\pgfpathlineto{\pgfqpoint{4.791704in}{2.283775in}}%
\pgfpathlineto{\pgfqpoint{4.799399in}{2.290752in}}%
\pgfpathlineto{\pgfqpoint{4.807087in}{2.297702in}}%
\pgfpathlineto{\pgfqpoint{4.814769in}{2.304627in}}%
\pgfpathlineto{\pgfqpoint{4.822444in}{2.311531in}}%
\pgfpathlineto{\pgfqpoint{4.808411in}{2.310457in}}%
\pgfpathlineto{\pgfqpoint{4.794387in}{2.309455in}}%
\pgfpathlineto{\pgfqpoint{4.780373in}{2.308523in}}%
\pgfpathlineto{\pgfqpoint{4.766369in}{2.307663in}}%
\pgfpathlineto{\pgfqpoint{4.758679in}{2.300584in}}%
\pgfpathlineto{\pgfqpoint{4.750983in}{2.293489in}}%
\pgfpathlineto{\pgfqpoint{4.743280in}{2.286374in}}%
\pgfpathlineto{\pgfqpoint{4.735571in}{2.279237in}}%
\pgfpathclose%
\pgfusepath{fill}%
\end{pgfscope}%
\begin{pgfscope}%
\pgfpathrectangle{\pgfqpoint{1.150000in}{0.150000in}}{\pgfqpoint{5.700000in}{5.700000in}}%
\pgfusepath{clip}%
\pgfsetbuttcap%
\pgfsetroundjoin%
\definecolor{currentfill}{rgb}{0.280267,0.073417,0.397163}%
\pgfsetfillcolor{currentfill}%
\pgfsetfillopacity{0.700000}%
\pgfsetlinewidth{0.000000pt}%
\definecolor{currentstroke}{rgb}{0.000000,0.000000,0.000000}%
\pgfsetstrokecolor{currentstroke}%
\pgfsetdash{}{0pt}%
\pgfpathmoveto{\pgfqpoint{3.841460in}{1.989883in}}%
\pgfpathlineto{\pgfqpoint{3.855201in}{1.988199in}}%
\pgfpathlineto{\pgfqpoint{3.868950in}{1.986594in}}%
\pgfpathlineto{\pgfqpoint{3.882706in}{1.985068in}}%
\pgfpathlineto{\pgfqpoint{3.896469in}{1.983619in}}%
\pgfpathlineto{\pgfqpoint{3.904507in}{1.992687in}}%
\pgfpathlineto{\pgfqpoint{3.912540in}{2.001726in}}%
\pgfpathlineto{\pgfqpoint{3.920568in}{2.010738in}}%
\pgfpathlineto{\pgfqpoint{3.928589in}{2.019721in}}%
\pgfpathlineto{\pgfqpoint{3.914837in}{2.021149in}}%
\pgfpathlineto{\pgfqpoint{3.901092in}{2.022656in}}%
\pgfpathlineto{\pgfqpoint{3.887354in}{2.024241in}}%
\pgfpathlineto{\pgfqpoint{3.873623in}{2.025904in}}%
\pgfpathlineto{\pgfqpoint{3.865591in}{2.016934in}}%
\pgfpathlineto{\pgfqpoint{3.857553in}{2.007940in}}%
\pgfpathlineto{\pgfqpoint{3.849509in}{1.998923in}}%
\pgfpathlineto{\pgfqpoint{3.841460in}{1.989883in}}%
\pgfpathclose%
\pgfusepath{fill}%
\end{pgfscope}%
\begin{pgfscope}%
\pgfpathrectangle{\pgfqpoint{1.150000in}{0.150000in}}{\pgfqpoint{5.700000in}{5.700000in}}%
\pgfusepath{clip}%
\pgfsetbuttcap%
\pgfsetroundjoin%
\definecolor{currentfill}{rgb}{0.282623,0.140926,0.457517}%
\pgfsetfillcolor{currentfill}%
\pgfsetfillopacity{0.700000}%
\pgfsetlinewidth{0.000000pt}%
\definecolor{currentstroke}{rgb}{0.000000,0.000000,0.000000}%
\pgfsetstrokecolor{currentstroke}%
\pgfsetdash{}{0pt}%
\pgfpathmoveto{\pgfqpoint{4.244992in}{2.112080in}}%
\pgfpathlineto{\pgfqpoint{4.258848in}{2.111883in}}%
\pgfpathlineto{\pgfqpoint{4.272713in}{2.111760in}}%
\pgfpathlineto{\pgfqpoint{4.286587in}{2.111711in}}%
\pgfpathlineto{\pgfqpoint{4.300469in}{2.111737in}}%
\pgfpathlineto{\pgfqpoint{4.308363in}{2.120138in}}%
\pgfpathlineto{\pgfqpoint{4.316250in}{2.128496in}}%
\pgfpathlineto{\pgfqpoint{4.324131in}{2.136814in}}%
\pgfpathlineto{\pgfqpoint{4.332006in}{2.145092in}}%
\pgfpathlineto{\pgfqpoint{4.318135in}{2.145129in}}%
\pgfpathlineto{\pgfqpoint{4.304273in}{2.145240in}}%
\pgfpathlineto{\pgfqpoint{4.290419in}{2.145426in}}%
\pgfpathlineto{\pgfqpoint{4.276574in}{2.145686in}}%
\pgfpathlineto{\pgfqpoint{4.268687in}{2.137337in}}%
\pgfpathlineto{\pgfqpoint{4.260795in}{2.128954in}}%
\pgfpathlineto{\pgfqpoint{4.252896in}{2.120535in}}%
\pgfpathlineto{\pgfqpoint{4.244992in}{2.112080in}}%
\pgfpathclose%
\pgfusepath{fill}%
\end{pgfscope}%
\begin{pgfscope}%
\pgfpathrectangle{\pgfqpoint{1.150000in}{0.150000in}}{\pgfqpoint{5.700000in}{5.700000in}}%
\pgfusepath{clip}%
\pgfsetbuttcap%
\pgfsetroundjoin%
\definecolor{currentfill}{rgb}{0.274952,0.037752,0.364543}%
\pgfsetfillcolor{currentfill}%
\pgfsetfillopacity{0.700000}%
\pgfsetlinewidth{0.000000pt}%
\definecolor{currentstroke}{rgb}{0.000000,0.000000,0.000000}%
\pgfsetstrokecolor{currentstroke}%
\pgfsetdash{}{0pt}%
\pgfpathmoveto{\pgfqpoint{3.524935in}{1.920418in}}%
\pgfpathlineto{\pgfqpoint{3.538607in}{1.917258in}}%
\pgfpathlineto{\pgfqpoint{3.552285in}{1.914182in}}%
\pgfpathlineto{\pgfqpoint{3.565968in}{1.911188in}}%
\pgfpathlineto{\pgfqpoint{3.579658in}{1.908276in}}%
\pgfpathlineto{\pgfqpoint{3.587810in}{1.917307in}}%
\pgfpathlineto{\pgfqpoint{3.595956in}{1.926335in}}%
\pgfpathlineto{\pgfqpoint{3.604096in}{1.935358in}}%
\pgfpathlineto{\pgfqpoint{3.612230in}{1.944378in}}%
\pgfpathlineto{\pgfqpoint{3.598553in}{1.947208in}}%
\pgfpathlineto{\pgfqpoint{3.584881in}{1.950120in}}%
\pgfpathlineto{\pgfqpoint{3.571215in}{1.953114in}}%
\pgfpathlineto{\pgfqpoint{3.557556in}{1.956192in}}%
\pgfpathlineto{\pgfqpoint{3.549409in}{1.947247in}}%
\pgfpathlineto{\pgfqpoint{3.541257in}{1.938302in}}%
\pgfpathlineto{\pgfqpoint{3.533099in}{1.929359in}}%
\pgfpathlineto{\pgfqpoint{3.524935in}{1.920418in}}%
\pgfpathclose%
\pgfusepath{fill}%
\end{pgfscope}%
\begin{pgfscope}%
\pgfpathrectangle{\pgfqpoint{1.150000in}{0.150000in}}{\pgfqpoint{5.700000in}{5.700000in}}%
\pgfusepath{clip}%
\pgfsetbuttcap%
\pgfsetroundjoin%
\definecolor{currentfill}{rgb}{0.225863,0.330805,0.547314}%
\pgfsetfillcolor{currentfill}%
\pgfsetfillopacity{0.700000}%
\pgfsetlinewidth{0.000000pt}%
\definecolor{currentstroke}{rgb}{0.000000,0.000000,0.000000}%
\pgfsetstrokecolor{currentstroke}%
\pgfsetdash{}{0pt}%
\pgfpathmoveto{\pgfqpoint{5.542838in}{2.533926in}}%
\pgfpathlineto{\pgfqpoint{5.557141in}{2.535798in}}%
\pgfpathlineto{\pgfqpoint{5.571455in}{2.537737in}}%
\pgfpathlineto{\pgfqpoint{5.585780in}{2.539745in}}%
\pgfpathlineto{\pgfqpoint{5.600118in}{2.541820in}}%
\pgfpathlineto{\pgfqpoint{5.607437in}{2.546645in}}%
\pgfpathlineto{\pgfqpoint{5.614752in}{2.551535in}}%
\pgfpathlineto{\pgfqpoint{5.622063in}{2.556497in}}%
\pgfpathlineto{\pgfqpoint{5.629370in}{2.561536in}}%
\pgfpathlineto{\pgfqpoint{5.615057in}{2.559797in}}%
\pgfpathlineto{\pgfqpoint{5.600756in}{2.558126in}}%
\pgfpathlineto{\pgfqpoint{5.586466in}{2.556522in}}%
\pgfpathlineto{\pgfqpoint{5.572187in}{2.554985in}}%
\pgfpathlineto{\pgfqpoint{5.564856in}{2.549603in}}%
\pgfpathlineto{\pgfqpoint{5.557521in}{2.544304in}}%
\pgfpathlineto{\pgfqpoint{5.550182in}{2.539080in}}%
\pgfpathlineto{\pgfqpoint{5.542838in}{2.533926in}}%
\pgfpathclose%
\pgfusepath{fill}%
\end{pgfscope}%
\begin{pgfscope}%
\pgfpathrectangle{\pgfqpoint{1.150000in}{0.150000in}}{\pgfqpoint{5.700000in}{5.700000in}}%
\pgfusepath{clip}%
\pgfsetbuttcap%
\pgfsetroundjoin%
\definecolor{currentfill}{rgb}{0.248629,0.278775,0.534556}%
\pgfsetfillcolor{currentfill}%
\pgfsetfillopacity{0.700000}%
\pgfsetlinewidth{0.000000pt}%
\definecolor{currentstroke}{rgb}{0.000000,0.000000,0.000000}%
\pgfsetstrokecolor{currentstroke}%
\pgfsetdash{}{0pt}%
\pgfpathmoveto{\pgfqpoint{5.139255in}{2.410715in}}%
\pgfpathlineto{\pgfqpoint{5.153417in}{2.412350in}}%
\pgfpathlineto{\pgfqpoint{5.167590in}{2.414055in}}%
\pgfpathlineto{\pgfqpoint{5.181774in}{2.415830in}}%
\pgfpathlineto{\pgfqpoint{5.195969in}{2.417673in}}%
\pgfpathlineto{\pgfqpoint{5.203481in}{2.423419in}}%
\pgfpathlineto{\pgfqpoint{5.210987in}{2.429173in}}%
\pgfpathlineto{\pgfqpoint{5.218488in}{2.434939in}}%
\pgfpathlineto{\pgfqpoint{5.225982in}{2.440722in}}%
\pgfpathlineto{\pgfqpoint{5.211806in}{2.439130in}}%
\pgfpathlineto{\pgfqpoint{5.197641in}{2.437608in}}%
\pgfpathlineto{\pgfqpoint{5.183486in}{2.436155in}}%
\pgfpathlineto{\pgfqpoint{5.169343in}{2.434770in}}%
\pgfpathlineto{\pgfqpoint{5.161829in}{2.428728in}}%
\pgfpathlineto{\pgfqpoint{5.154310in}{2.422708in}}%
\pgfpathlineto{\pgfqpoint{5.146785in}{2.416705in}}%
\pgfpathlineto{\pgfqpoint{5.139255in}{2.410715in}}%
\pgfpathclose%
\pgfusepath{fill}%
\end{pgfscope}%
\begin{pgfscope}%
\pgfpathrectangle{\pgfqpoint{1.150000in}{0.150000in}}{\pgfqpoint{5.700000in}{5.700000in}}%
\pgfusepath{clip}%
\pgfsetbuttcap%
\pgfsetroundjoin%
\definecolor{currentfill}{rgb}{0.271305,0.019942,0.347269}%
\pgfsetfillcolor{currentfill}%
\pgfsetfillopacity{0.700000}%
\pgfsetlinewidth{0.000000pt}%
\definecolor{currentstroke}{rgb}{0.000000,0.000000,0.000000}%
\pgfsetstrokecolor{currentstroke}%
\pgfsetdash{}{0pt}%
\pgfpathmoveto{\pgfqpoint{3.295452in}{1.896315in}}%
\pgfpathlineto{\pgfqpoint{3.309087in}{1.891908in}}%
\pgfpathlineto{\pgfqpoint{3.322727in}{1.887588in}}%
\pgfpathlineto{\pgfqpoint{3.336372in}{1.883355in}}%
\pgfpathlineto{\pgfqpoint{3.350022in}{1.879208in}}%
\pgfpathlineto{\pgfqpoint{3.358262in}{1.887828in}}%
\pgfpathlineto{\pgfqpoint{3.366495in}{1.896471in}}%
\pgfpathlineto{\pgfqpoint{3.374722in}{1.905135in}}%
\pgfpathlineto{\pgfqpoint{3.382942in}{1.913818in}}%
\pgfpathlineto{\pgfqpoint{3.369306in}{1.917842in}}%
\pgfpathlineto{\pgfqpoint{3.355675in}{1.921952in}}%
\pgfpathlineto{\pgfqpoint{3.342049in}{1.926149in}}%
\pgfpathlineto{\pgfqpoint{3.328428in}{1.930433in}}%
\pgfpathlineto{\pgfqpoint{3.320193in}{1.921865in}}%
\pgfpathlineto{\pgfqpoint{3.311953in}{1.913322in}}%
\pgfpathlineto{\pgfqpoint{3.303706in}{1.904805in}}%
\pgfpathlineto{\pgfqpoint{3.295452in}{1.896315in}}%
\pgfpathclose%
\pgfusepath{fill}%
\end{pgfscope}%
\begin{pgfscope}%
\pgfpathrectangle{\pgfqpoint{1.150000in}{0.150000in}}{\pgfqpoint{5.700000in}{5.700000in}}%
\pgfusepath{clip}%
\pgfsetbuttcap%
\pgfsetroundjoin%
\definecolor{currentfill}{rgb}{0.282910,0.105393,0.426902}%
\pgfsetfillcolor{currentfill}%
\pgfsetfillopacity{0.700000}%
\pgfsetlinewidth{0.000000pt}%
\definecolor{currentstroke}{rgb}{0.000000,0.000000,0.000000}%
\pgfsetstrokecolor{currentstroke}%
\pgfsetdash{}{0pt}%
\pgfpathmoveto{\pgfqpoint{2.528064in}{2.064208in}}%
\pgfpathlineto{\pgfqpoint{2.541691in}{2.054495in}}%
\pgfpathlineto{\pgfqpoint{2.555318in}{2.044897in}}%
\pgfpathlineto{\pgfqpoint{2.568945in}{2.035413in}}%
\pgfpathlineto{\pgfqpoint{2.582572in}{2.026042in}}%
\pgfpathlineto{\pgfqpoint{2.591188in}{2.031175in}}%
\pgfpathlineto{\pgfqpoint{2.599793in}{2.036437in}}%
\pgfpathlineto{\pgfqpoint{2.608387in}{2.041824in}}%
\pgfpathlineto{\pgfqpoint{2.616970in}{2.047333in}}%
\pgfpathlineto{\pgfqpoint{2.603368in}{2.056476in}}%
\pgfpathlineto{\pgfqpoint{2.589766in}{2.065731in}}%
\pgfpathlineto{\pgfqpoint{2.576164in}{2.075099in}}%
\pgfpathlineto{\pgfqpoint{2.562562in}{2.084582in}}%
\pgfpathlineto{\pgfqpoint{2.553955in}{2.079294in}}%
\pgfpathlineto{\pgfqpoint{2.545336in}{2.074134in}}%
\pgfpathlineto{\pgfqpoint{2.536705in}{2.069104in}}%
\pgfpathlineto{\pgfqpoint{2.528064in}{2.064208in}}%
\pgfpathclose%
\pgfusepath{fill}%
\end{pgfscope}%
\begin{pgfscope}%
\pgfpathrectangle{\pgfqpoint{1.150000in}{0.150000in}}{\pgfqpoint{5.700000in}{5.700000in}}%
\pgfusepath{clip}%
\pgfsetbuttcap%
\pgfsetroundjoin%
\definecolor{currentfill}{rgb}{0.278791,0.062145,0.386592}%
\pgfsetfillcolor{currentfill}%
\pgfsetfillopacity{0.700000}%
\pgfsetlinewidth{0.000000pt}%
\definecolor{currentstroke}{rgb}{0.000000,0.000000,0.000000}%
\pgfsetstrokecolor{currentstroke}%
\pgfsetdash{}{0pt}%
\pgfpathmoveto{\pgfqpoint{2.725812in}{1.978154in}}%
\pgfpathlineto{\pgfqpoint{2.739421in}{1.969990in}}%
\pgfpathlineto{\pgfqpoint{2.753032in}{1.961931in}}%
\pgfpathlineto{\pgfqpoint{2.766644in}{1.953976in}}%
\pgfpathlineto{\pgfqpoint{2.780258in}{1.946125in}}%
\pgfpathlineto{\pgfqpoint{2.788762in}{1.952398in}}%
\pgfpathlineto{\pgfqpoint{2.797257in}{1.958771in}}%
\pgfpathlineto{\pgfqpoint{2.805742in}{1.965242in}}%
\pgfpathlineto{\pgfqpoint{2.814218in}{1.971807in}}%
\pgfpathlineto{\pgfqpoint{2.800625in}{1.979452in}}%
\pgfpathlineto{\pgfqpoint{2.787035in}{1.987200in}}%
\pgfpathlineto{\pgfqpoint{2.773446in}{1.995052in}}%
\pgfpathlineto{\pgfqpoint{2.759858in}{2.003009in}}%
\pgfpathlineto{\pgfqpoint{2.751361in}{1.996643in}}%
\pgfpathlineto{\pgfqpoint{2.742854in}{1.990377in}}%
\pgfpathlineto{\pgfqpoint{2.734338in}{1.984213in}}%
\pgfpathlineto{\pgfqpoint{2.725812in}{1.978154in}}%
\pgfpathclose%
\pgfusepath{fill}%
\end{pgfscope}%
\begin{pgfscope}%
\pgfpathrectangle{\pgfqpoint{1.150000in}{0.150000in}}{\pgfqpoint{5.700000in}{5.700000in}}%
\pgfusepath{clip}%
\pgfsetbuttcap%
\pgfsetroundjoin%
\definecolor{currentfill}{rgb}{0.210503,0.363727,0.552206}%
\pgfsetfillcolor{currentfill}%
\pgfsetfillopacity{0.700000}%
\pgfsetlinewidth{0.000000pt}%
\definecolor{currentstroke}{rgb}{0.000000,0.000000,0.000000}%
\pgfsetstrokecolor{currentstroke}%
\pgfsetdash{}{0pt}%
\pgfpathmoveto{\pgfqpoint{5.859746in}{2.622679in}}%
\pgfpathlineto{\pgfqpoint{5.874158in}{2.624555in}}%
\pgfpathlineto{\pgfqpoint{5.888582in}{2.626499in}}%
\pgfpathlineto{\pgfqpoint{5.903018in}{2.628509in}}%
\pgfpathlineto{\pgfqpoint{5.910198in}{2.633064in}}%
\pgfpathlineto{\pgfqpoint{5.917375in}{2.637743in}}%
\pgfpathlineto{\pgfqpoint{5.924552in}{2.642551in}}%
\pgfpathlineto{\pgfqpoint{5.931727in}{2.647497in}}%
\pgfpathlineto{\pgfqpoint{5.917320in}{2.645885in}}%
\pgfpathlineto{\pgfqpoint{5.902925in}{2.644340in}}%
\pgfpathlineto{\pgfqpoint{5.888542in}{2.642860in}}%
\pgfpathlineto{\pgfqpoint{5.881345in}{2.637611in}}%
\pgfpathlineto{\pgfqpoint{5.874147in}{2.632502in}}%
\pgfpathlineto{\pgfqpoint{5.866948in}{2.627527in}}%
\pgfpathlineto{\pgfqpoint{5.859746in}{2.622679in}}%
\pgfpathclose%
\pgfusepath{fill}%
\end{pgfscope}%
\begin{pgfscope}%
\pgfpathrectangle{\pgfqpoint{1.150000in}{0.150000in}}{\pgfqpoint{5.700000in}{5.700000in}}%
\pgfusepath{clip}%
\pgfsetbuttcap%
\pgfsetroundjoin%
\definecolor{currentfill}{rgb}{0.273006,0.204520,0.501721}%
\pgfsetfillcolor{currentfill}%
\pgfsetfillopacity{0.700000}%
\pgfsetlinewidth{0.000000pt}%
\definecolor{currentstroke}{rgb}{0.000000,0.000000,0.000000}%
\pgfsetstrokecolor{currentstroke}%
\pgfsetdash{}{0pt}%
\pgfpathmoveto{\pgfqpoint{4.648646in}{2.246420in}}%
\pgfpathlineto{\pgfqpoint{4.662638in}{2.247308in}}%
\pgfpathlineto{\pgfqpoint{4.676640in}{2.248268in}}%
\pgfpathlineto{\pgfqpoint{4.690652in}{2.249299in}}%
\pgfpathlineto{\pgfqpoint{4.704674in}{2.250402in}}%
\pgfpathlineto{\pgfqpoint{4.712408in}{2.257660in}}%
\pgfpathlineto{\pgfqpoint{4.720135in}{2.264882in}}%
\pgfpathlineto{\pgfqpoint{4.727856in}{2.272074in}}%
\pgfpathlineto{\pgfqpoint{4.735571in}{2.279237in}}%
\pgfpathlineto{\pgfqpoint{4.721563in}{2.278280in}}%
\pgfpathlineto{\pgfqpoint{4.707565in}{2.277396in}}%
\pgfpathlineto{\pgfqpoint{4.693576in}{2.276582in}}%
\pgfpathlineto{\pgfqpoint{4.679597in}{2.275840in}}%
\pgfpathlineto{\pgfqpoint{4.671869in}{2.268523in}}%
\pgfpathlineto{\pgfqpoint{4.664134in}{2.261183in}}%
\pgfpathlineto{\pgfqpoint{4.656393in}{2.253816in}}%
\pgfpathlineto{\pgfqpoint{4.648646in}{2.246420in}}%
\pgfpathclose%
\pgfusepath{fill}%
\end{pgfscope}%
\begin{pgfscope}%
\pgfpathrectangle{\pgfqpoint{1.150000in}{0.150000in}}{\pgfqpoint{5.700000in}{5.700000in}}%
\pgfusepath{clip}%
\pgfsetbuttcap%
\pgfsetroundjoin%
\definecolor{currentfill}{rgb}{0.278012,0.180367,0.486697}%
\pgfsetfillcolor{currentfill}%
\pgfsetfillopacity{0.700000}%
\pgfsetlinewidth{0.000000pt}%
\definecolor{currentstroke}{rgb}{0.000000,0.000000,0.000000}%
\pgfsetstrokecolor{currentstroke}%
\pgfsetdash{}{0pt}%
\pgfpathmoveto{\pgfqpoint{2.274765in}{2.221817in}}%
\pgfpathlineto{\pgfqpoint{2.288441in}{2.209887in}}%
\pgfpathlineto{\pgfqpoint{2.302116in}{2.198087in}}%
\pgfpathlineto{\pgfqpoint{2.315787in}{2.186418in}}%
\pgfpathlineto{\pgfqpoint{2.329457in}{2.174877in}}%
\pgfpathlineto{\pgfqpoint{2.338233in}{2.178423in}}%
\pgfpathlineto{\pgfqpoint{2.346996in}{2.182133in}}%
\pgfpathlineto{\pgfqpoint{2.355745in}{2.186003in}}%
\pgfpathlineto{\pgfqpoint{2.364481in}{2.190030in}}%
\pgfpathlineto{\pgfqpoint{2.350841in}{2.201317in}}%
\pgfpathlineto{\pgfqpoint{2.337199in}{2.212733in}}%
\pgfpathlineto{\pgfqpoint{2.323554in}{2.224279in}}%
\pgfpathlineto{\pgfqpoint{2.309908in}{2.235955in}}%
\pgfpathlineto{\pgfqpoint{2.301143in}{2.232174in}}%
\pgfpathlineto{\pgfqpoint{2.292364in}{2.228555in}}%
\pgfpathlineto{\pgfqpoint{2.283571in}{2.225101in}}%
\pgfpathlineto{\pgfqpoint{2.274765in}{2.221817in}}%
\pgfpathclose%
\pgfusepath{fill}%
\end{pgfscope}%
\begin{pgfscope}%
\pgfpathrectangle{\pgfqpoint{1.150000in}{0.150000in}}{\pgfqpoint{5.700000in}{5.700000in}}%
\pgfusepath{clip}%
\pgfsetbuttcap%
\pgfsetroundjoin%
\definecolor{currentfill}{rgb}{0.283187,0.125848,0.444960}%
\pgfsetfillcolor{currentfill}%
\pgfsetfillopacity{0.700000}%
\pgfsetlinewidth{0.000000pt}%
\definecolor{currentstroke}{rgb}{0.000000,0.000000,0.000000}%
\pgfsetstrokecolor{currentstroke}%
\pgfsetdash{}{0pt}%
\pgfpathmoveto{\pgfqpoint{4.157933in}{2.079231in}}%
\pgfpathlineto{\pgfqpoint{4.171766in}{2.078775in}}%
\pgfpathlineto{\pgfqpoint{4.185608in}{2.078396in}}%
\pgfpathlineto{\pgfqpoint{4.199458in}{2.078091in}}%
\pgfpathlineto{\pgfqpoint{4.213317in}{2.077861in}}%
\pgfpathlineto{\pgfqpoint{4.221244in}{2.086478in}}%
\pgfpathlineto{\pgfqpoint{4.229166in}{2.095052in}}%
\pgfpathlineto{\pgfqpoint{4.237082in}{2.103586in}}%
\pgfpathlineto{\pgfqpoint{4.244992in}{2.112080in}}%
\pgfpathlineto{\pgfqpoint{4.231144in}{2.112352in}}%
\pgfpathlineto{\pgfqpoint{4.217305in}{2.112699in}}%
\pgfpathlineto{\pgfqpoint{4.203474in}{2.113121in}}%
\pgfpathlineto{\pgfqpoint{4.189652in}{2.113617in}}%
\pgfpathlineto{\pgfqpoint{4.181731in}{2.105074in}}%
\pgfpathlineto{\pgfqpoint{4.173804in}{2.096496in}}%
\pgfpathlineto{\pgfqpoint{4.165871in}{2.087882in}}%
\pgfpathlineto{\pgfqpoint{4.157933in}{2.079231in}}%
\pgfpathclose%
\pgfusepath{fill}%
\end{pgfscope}%
\begin{pgfscope}%
\pgfpathrectangle{\pgfqpoint{1.150000in}{0.150000in}}{\pgfqpoint{5.700000in}{5.700000in}}%
\pgfusepath{clip}%
\pgfsetbuttcap%
\pgfsetroundjoin%
\definecolor{currentfill}{rgb}{0.278791,0.062145,0.386592}%
\pgfsetfillcolor{currentfill}%
\pgfsetfillopacity{0.700000}%
\pgfsetlinewidth{0.000000pt}%
\definecolor{currentstroke}{rgb}{0.000000,0.000000,0.000000}%
\pgfsetstrokecolor{currentstroke}%
\pgfsetdash{}{0pt}%
\pgfpathmoveto{\pgfqpoint{3.754267in}{1.961177in}}%
\pgfpathlineto{\pgfqpoint{3.767991in}{1.959136in}}%
\pgfpathlineto{\pgfqpoint{3.781723in}{1.957175in}}%
\pgfpathlineto{\pgfqpoint{3.795461in}{1.955293in}}%
\pgfpathlineto{\pgfqpoint{3.809206in}{1.953490in}}%
\pgfpathlineto{\pgfqpoint{3.817278in}{1.962623in}}%
\pgfpathlineto{\pgfqpoint{3.825344in}{1.971733in}}%
\pgfpathlineto{\pgfqpoint{3.833405in}{1.980819in}}%
\pgfpathlineto{\pgfqpoint{3.841460in}{1.989883in}}%
\pgfpathlineto{\pgfqpoint{3.827725in}{1.991645in}}%
\pgfpathlineto{\pgfqpoint{3.813998in}{1.993487in}}%
\pgfpathlineto{\pgfqpoint{3.800278in}{1.995407in}}%
\pgfpathlineto{\pgfqpoint{3.786564in}{1.997407in}}%
\pgfpathlineto{\pgfqpoint{3.778499in}{1.988377in}}%
\pgfpathlineto{\pgfqpoint{3.770427in}{1.979328in}}%
\pgfpathlineto{\pgfqpoint{3.762350in}{1.970261in}}%
\pgfpathlineto{\pgfqpoint{3.754267in}{1.961177in}}%
\pgfpathclose%
\pgfusepath{fill}%
\end{pgfscope}%
\begin{pgfscope}%
\pgfpathrectangle{\pgfqpoint{1.150000in}{0.150000in}}{\pgfqpoint{5.700000in}{5.700000in}}%
\pgfusepath{clip}%
\pgfsetbuttcap%
\pgfsetroundjoin%
\definecolor{currentfill}{rgb}{0.253935,0.265254,0.529983}%
\pgfsetfillcolor{currentfill}%
\pgfsetfillopacity{0.700000}%
\pgfsetlinewidth{0.000000pt}%
\definecolor{currentstroke}{rgb}{0.000000,0.000000,0.000000}%
\pgfsetstrokecolor{currentstroke}%
\pgfsetdash{}{0pt}%
\pgfpathmoveto{\pgfqpoint{5.052460in}{2.380021in}}%
\pgfpathlineto{\pgfqpoint{5.066597in}{2.381609in}}%
\pgfpathlineto{\pgfqpoint{5.080744in}{2.383268in}}%
\pgfpathlineto{\pgfqpoint{5.094903in}{2.384996in}}%
\pgfpathlineto{\pgfqpoint{5.109072in}{2.386794in}}%
\pgfpathlineto{\pgfqpoint{5.116627in}{2.392776in}}%
\pgfpathlineto{\pgfqpoint{5.124176in}{2.398754in}}%
\pgfpathlineto{\pgfqpoint{5.131718in}{2.404733in}}%
\pgfpathlineto{\pgfqpoint{5.139255in}{2.410715in}}%
\pgfpathlineto{\pgfqpoint{5.125103in}{2.409148in}}%
\pgfpathlineto{\pgfqpoint{5.110962in}{2.407651in}}%
\pgfpathlineto{\pgfqpoint{5.096832in}{2.406224in}}%
\pgfpathlineto{\pgfqpoint{5.082713in}{2.404865in}}%
\pgfpathlineto{\pgfqpoint{5.075159in}{2.398645in}}%
\pgfpathlineto{\pgfqpoint{5.067598in}{2.392434in}}%
\pgfpathlineto{\pgfqpoint{5.060032in}{2.386227in}}%
\pgfpathlineto{\pgfqpoint{5.052460in}{2.380021in}}%
\pgfpathclose%
\pgfusepath{fill}%
\end{pgfscope}%
\begin{pgfscope}%
\pgfpathrectangle{\pgfqpoint{1.150000in}{0.150000in}}{\pgfqpoint{5.700000in}{5.700000in}}%
\pgfusepath{clip}%
\pgfsetbuttcap%
\pgfsetroundjoin%
\definecolor{currentfill}{rgb}{0.229739,0.322361,0.545706}%
\pgfsetfillcolor{currentfill}%
\pgfsetfillopacity{0.700000}%
\pgfsetlinewidth{0.000000pt}%
\definecolor{currentstroke}{rgb}{0.000000,0.000000,0.000000}%
\pgfsetstrokecolor{currentstroke}%
\pgfsetdash{}{0pt}%
\pgfpathmoveto{\pgfqpoint{5.456233in}{2.505830in}}%
\pgfpathlineto{\pgfqpoint{5.470512in}{2.507746in}}%
\pgfpathlineto{\pgfqpoint{5.484803in}{2.509729in}}%
\pgfpathlineto{\pgfqpoint{5.499106in}{2.511780in}}%
\pgfpathlineto{\pgfqpoint{5.513420in}{2.513900in}}%
\pgfpathlineto{\pgfqpoint{5.520782in}{2.518829in}}%
\pgfpathlineto{\pgfqpoint{5.528139in}{2.523807in}}%
\pgfpathlineto{\pgfqpoint{5.535491in}{2.528837in}}%
\pgfpathlineto{\pgfqpoint{5.542838in}{2.533926in}}%
\pgfpathlineto{\pgfqpoint{5.528547in}{2.532122in}}%
\pgfpathlineto{\pgfqpoint{5.514268in}{2.530386in}}%
\pgfpathlineto{\pgfqpoint{5.500000in}{2.528718in}}%
\pgfpathlineto{\pgfqpoint{5.485743in}{2.527117in}}%
\pgfpathlineto{\pgfqpoint{5.478372in}{2.521706in}}%
\pgfpathlineto{\pgfqpoint{5.470997in}{2.516358in}}%
\pgfpathlineto{\pgfqpoint{5.463617in}{2.511068in}}%
\pgfpathlineto{\pgfqpoint{5.456233in}{2.505830in}}%
\pgfpathclose%
\pgfusepath{fill}%
\end{pgfscope}%
\begin{pgfscope}%
\pgfpathrectangle{\pgfqpoint{1.150000in}{0.150000in}}{\pgfqpoint{5.700000in}{5.700000in}}%
\pgfusepath{clip}%
\pgfsetbuttcap%
\pgfsetroundjoin%
\definecolor{currentfill}{rgb}{0.276194,0.190074,0.493001}%
\pgfsetfillcolor{currentfill}%
\pgfsetfillopacity{0.700000}%
\pgfsetlinewidth{0.000000pt}%
\definecolor{currentstroke}{rgb}{0.000000,0.000000,0.000000}%
\pgfsetstrokecolor{currentstroke}%
\pgfsetdash{}{0pt}%
\pgfpathmoveto{\pgfqpoint{4.561670in}{2.213154in}}%
\pgfpathlineto{\pgfqpoint{4.575636in}{2.213879in}}%
\pgfpathlineto{\pgfqpoint{4.589613in}{2.214677in}}%
\pgfpathlineto{\pgfqpoint{4.603598in}{2.215547in}}%
\pgfpathlineto{\pgfqpoint{4.617594in}{2.216490in}}%
\pgfpathlineto{\pgfqpoint{4.625366in}{2.224029in}}%
\pgfpathlineto{\pgfqpoint{4.633132in}{2.231529in}}%
\pgfpathlineto{\pgfqpoint{4.640892in}{2.238991in}}%
\pgfpathlineto{\pgfqpoint{4.648646in}{2.246420in}}%
\pgfpathlineto{\pgfqpoint{4.634663in}{2.245603in}}%
\pgfpathlineto{\pgfqpoint{4.620690in}{2.244859in}}%
\pgfpathlineto{\pgfqpoint{4.606727in}{2.244187in}}%
\pgfpathlineto{\pgfqpoint{4.592773in}{2.243586in}}%
\pgfpathlineto{\pgfqpoint{4.585006in}{2.236025in}}%
\pgfpathlineto{\pgfqpoint{4.577234in}{2.228434in}}%
\pgfpathlineto{\pgfqpoint{4.569455in}{2.220811in}}%
\pgfpathlineto{\pgfqpoint{4.561670in}{2.213154in}}%
\pgfpathclose%
\pgfusepath{fill}%
\end{pgfscope}%
\begin{pgfscope}%
\pgfpathrectangle{\pgfqpoint{1.150000in}{0.150000in}}{\pgfqpoint{5.700000in}{5.700000in}}%
\pgfusepath{clip}%
\pgfsetbuttcap%
\pgfsetroundjoin%
\definecolor{currentfill}{rgb}{0.272594,0.025563,0.353093}%
\pgfsetfillcolor{currentfill}%
\pgfsetfillopacity{0.700000}%
\pgfsetlinewidth{0.000000pt}%
\definecolor{currentstroke}{rgb}{0.000000,0.000000,0.000000}%
\pgfsetstrokecolor{currentstroke}%
\pgfsetdash{}{0pt}%
\pgfpathmoveto{\pgfqpoint{3.437538in}{1.898580in}}%
\pgfpathlineto{\pgfqpoint{3.451200in}{1.894983in}}%
\pgfpathlineto{\pgfqpoint{3.464867in}{1.891471in}}%
\pgfpathlineto{\pgfqpoint{3.478541in}{1.888042in}}%
\pgfpathlineto{\pgfqpoint{3.492219in}{1.884697in}}%
\pgfpathlineto{\pgfqpoint{3.500407in}{1.893618in}}%
\pgfpathlineto{\pgfqpoint{3.508589in}{1.902547in}}%
\pgfpathlineto{\pgfqpoint{3.516765in}{1.911480in}}%
\pgfpathlineto{\pgfqpoint{3.524935in}{1.920418in}}%
\pgfpathlineto{\pgfqpoint{3.511269in}{1.923661in}}%
\pgfpathlineto{\pgfqpoint{3.497609in}{1.926987in}}%
\pgfpathlineto{\pgfqpoint{3.483954in}{1.930397in}}%
\pgfpathlineto{\pgfqpoint{3.470304in}{1.933892in}}%
\pgfpathlineto{\pgfqpoint{3.462122in}{1.925049in}}%
\pgfpathlineto{\pgfqpoint{3.453933in}{1.916215in}}%
\pgfpathlineto{\pgfqpoint{3.445739in}{1.907392in}}%
\pgfpathlineto{\pgfqpoint{3.437538in}{1.898580in}}%
\pgfpathclose%
\pgfusepath{fill}%
\end{pgfscope}%
\begin{pgfscope}%
\pgfpathrectangle{\pgfqpoint{1.150000in}{0.150000in}}{\pgfqpoint{5.700000in}{5.700000in}}%
\pgfusepath{clip}%
\pgfsetbuttcap%
\pgfsetroundjoin%
\definecolor{currentfill}{rgb}{0.271305,0.019942,0.347269}%
\pgfsetfillcolor{currentfill}%
\pgfsetfillopacity{0.700000}%
\pgfsetlinewidth{0.000000pt}%
\definecolor{currentstroke}{rgb}{0.000000,0.000000,0.000000}%
\pgfsetstrokecolor{currentstroke}%
\pgfsetdash{}{0pt}%
\pgfpathmoveto{\pgfqpoint{3.065514in}{1.893153in}}%
\pgfpathlineto{\pgfqpoint{3.079132in}{1.887356in}}%
\pgfpathlineto{\pgfqpoint{3.092754in}{1.881652in}}%
\pgfpathlineto{\pgfqpoint{3.106379in}{1.876040in}}%
\pgfpathlineto{\pgfqpoint{3.120008in}{1.870520in}}%
\pgfpathlineto{\pgfqpoint{3.128348in}{1.878379in}}%
\pgfpathlineto{\pgfqpoint{3.136682in}{1.886291in}}%
\pgfpathlineto{\pgfqpoint{3.145007in}{1.894255in}}%
\pgfpathlineto{\pgfqpoint{3.153326in}{1.902267in}}%
\pgfpathlineto{\pgfqpoint{3.139714in}{1.907623in}}%
\pgfpathlineto{\pgfqpoint{3.126106in}{1.913070in}}%
\pgfpathlineto{\pgfqpoint{3.112501in}{1.918610in}}%
\pgfpathlineto{\pgfqpoint{3.098900in}{1.924243in}}%
\pgfpathlineto{\pgfqpoint{3.090565in}{1.916387in}}%
\pgfpathlineto{\pgfqpoint{3.082222in}{1.908585in}}%
\pgfpathlineto{\pgfqpoint{3.073872in}{1.900839in}}%
\pgfpathlineto{\pgfqpoint{3.065514in}{1.893153in}}%
\pgfpathclose%
\pgfusepath{fill}%
\end{pgfscope}%
\begin{pgfscope}%
\pgfpathrectangle{\pgfqpoint{1.150000in}{0.150000in}}{\pgfqpoint{5.700000in}{5.700000in}}%
\pgfusepath{clip}%
\pgfsetbuttcap%
\pgfsetroundjoin%
\definecolor{currentfill}{rgb}{0.283091,0.110553,0.431554}%
\pgfsetfillcolor{currentfill}%
\pgfsetfillopacity{0.700000}%
\pgfsetlinewidth{0.000000pt}%
\definecolor{currentstroke}{rgb}{0.000000,0.000000,0.000000}%
\pgfsetstrokecolor{currentstroke}%
\pgfsetdash{}{0pt}%
\pgfpathmoveto{\pgfqpoint{4.070828in}{2.046729in}}%
\pgfpathlineto{\pgfqpoint{4.084639in}{2.045992in}}%
\pgfpathlineto{\pgfqpoint{4.098458in}{2.045332in}}%
\pgfpathlineto{\pgfqpoint{4.112286in}{2.044747in}}%
\pgfpathlineto{\pgfqpoint{4.126122in}{2.044238in}}%
\pgfpathlineto{\pgfqpoint{4.134083in}{2.053046in}}%
\pgfpathlineto{\pgfqpoint{4.142039in}{2.061814in}}%
\pgfpathlineto{\pgfqpoint{4.149989in}{2.070542in}}%
\pgfpathlineto{\pgfqpoint{4.157933in}{2.079231in}}%
\pgfpathlineto{\pgfqpoint{4.144108in}{2.079761in}}%
\pgfpathlineto{\pgfqpoint{4.130291in}{2.080367in}}%
\pgfpathlineto{\pgfqpoint{4.116482in}{2.081049in}}%
\pgfpathlineto{\pgfqpoint{4.102682in}{2.081807in}}%
\pgfpathlineto{\pgfqpoint{4.094727in}{2.073089in}}%
\pgfpathlineto{\pgfqpoint{4.086766in}{2.064337in}}%
\pgfpathlineto{\pgfqpoint{4.078800in}{2.055551in}}%
\pgfpathlineto{\pgfqpoint{4.070828in}{2.046729in}}%
\pgfpathclose%
\pgfusepath{fill}%
\end{pgfscope}%
\begin{pgfscope}%
\pgfpathrectangle{\pgfqpoint{1.150000in}{0.150000in}}{\pgfqpoint{5.700000in}{5.700000in}}%
\pgfusepath{clip}%
\pgfsetbuttcap%
\pgfsetroundjoin%
\definecolor{currentfill}{rgb}{0.273809,0.031497,0.358853}%
\pgfsetfillcolor{currentfill}%
\pgfsetfillopacity{0.700000}%
\pgfsetlinewidth{0.000000pt}%
\definecolor{currentstroke}{rgb}{0.000000,0.000000,0.000000}%
\pgfsetstrokecolor{currentstroke}%
\pgfsetdash{}{0pt}%
\pgfpathmoveto{\pgfqpoint{2.923030in}{1.914294in}}%
\pgfpathlineto{\pgfqpoint{2.936642in}{1.907551in}}%
\pgfpathlineto{\pgfqpoint{2.950257in}{1.900905in}}%
\pgfpathlineto{\pgfqpoint{2.963874in}{1.894356in}}%
\pgfpathlineto{\pgfqpoint{2.977495in}{1.887902in}}%
\pgfpathlineto{\pgfqpoint{2.985902in}{1.895138in}}%
\pgfpathlineto{\pgfqpoint{2.994301in}{1.902448in}}%
\pgfpathlineto{\pgfqpoint{3.002693in}{1.909828in}}%
\pgfpathlineto{\pgfqpoint{3.011076in}{1.917277in}}%
\pgfpathlineto{\pgfqpoint{2.997474in}{1.923546in}}%
\pgfpathlineto{\pgfqpoint{2.983875in}{1.929910in}}%
\pgfpathlineto{\pgfqpoint{2.970279in}{1.936371in}}%
\pgfpathlineto{\pgfqpoint{2.956686in}{1.942929in}}%
\pgfpathlineto{\pgfqpoint{2.948285in}{1.935657in}}%
\pgfpathlineto{\pgfqpoint{2.939875in}{1.928459in}}%
\pgfpathlineto{\pgfqpoint{2.931457in}{1.921337in}}%
\pgfpathlineto{\pgfqpoint{2.923030in}{1.914294in}}%
\pgfpathclose%
\pgfusepath{fill}%
\end{pgfscope}%
\begin{pgfscope}%
\pgfpathrectangle{\pgfqpoint{1.150000in}{0.150000in}}{\pgfqpoint{5.700000in}{5.700000in}}%
\pgfusepath{clip}%
\pgfsetbuttcap%
\pgfsetroundjoin%
\definecolor{currentfill}{rgb}{0.280868,0.160771,0.472899}%
\pgfsetfillcolor{currentfill}%
\pgfsetfillopacity{0.700000}%
\pgfsetlinewidth{0.000000pt}%
\definecolor{currentstroke}{rgb}{0.000000,0.000000,0.000000}%
\pgfsetstrokecolor{currentstroke}%
\pgfsetdash{}{0pt}%
\pgfpathmoveto{\pgfqpoint{2.329457in}{2.174877in}}%
\pgfpathlineto{\pgfqpoint{2.343125in}{2.163463in}}%
\pgfpathlineto{\pgfqpoint{2.356791in}{2.152176in}}%
\pgfpathlineto{\pgfqpoint{2.370455in}{2.141014in}}%
\pgfpathlineto{\pgfqpoint{2.384118in}{2.129976in}}%
\pgfpathlineto{\pgfqpoint{2.392864in}{2.133783in}}%
\pgfpathlineto{\pgfqpoint{2.401598in}{2.137749in}}%
\pgfpathlineto{\pgfqpoint{2.410318in}{2.141869in}}%
\pgfpathlineto{\pgfqpoint{2.419026in}{2.146141in}}%
\pgfpathlineto{\pgfqpoint{2.405392in}{2.156926in}}%
\pgfpathlineto{\pgfqpoint{2.391757in}{2.167835in}}%
\pgfpathlineto{\pgfqpoint{2.378120in}{2.178869in}}%
\pgfpathlineto{\pgfqpoint{2.364481in}{2.190030in}}%
\pgfpathlineto{\pgfqpoint{2.355745in}{2.186003in}}%
\pgfpathlineto{\pgfqpoint{2.346996in}{2.182133in}}%
\pgfpathlineto{\pgfqpoint{2.338233in}{2.178423in}}%
\pgfpathlineto{\pgfqpoint{2.329457in}{2.174877in}}%
\pgfpathclose%
\pgfusepath{fill}%
\end{pgfscope}%
\begin{pgfscope}%
\pgfpathrectangle{\pgfqpoint{1.150000in}{0.150000in}}{\pgfqpoint{5.700000in}{5.700000in}}%
\pgfusepath{clip}%
\pgfsetbuttcap%
\pgfsetroundjoin%
\definecolor{currentfill}{rgb}{0.281924,0.089666,0.412415}%
\pgfsetfillcolor{currentfill}%
\pgfsetfillopacity{0.700000}%
\pgfsetlinewidth{0.000000pt}%
\definecolor{currentstroke}{rgb}{0.000000,0.000000,0.000000}%
\pgfsetstrokecolor{currentstroke}%
\pgfsetdash{}{0pt}%
\pgfpathmoveto{\pgfqpoint{2.582572in}{2.026042in}}%
\pgfpathlineto{\pgfqpoint{2.596199in}{2.016782in}}%
\pgfpathlineto{\pgfqpoint{2.609827in}{2.007634in}}%
\pgfpathlineto{\pgfqpoint{2.623455in}{1.998596in}}%
\pgfpathlineto{\pgfqpoint{2.637084in}{1.989668in}}%
\pgfpathlineto{\pgfqpoint{2.645675in}{1.995038in}}%
\pgfpathlineto{\pgfqpoint{2.654255in}{2.000531in}}%
\pgfpathlineto{\pgfqpoint{2.662825in}{2.006144in}}%
\pgfpathlineto{\pgfqpoint{2.671385in}{2.011874in}}%
\pgfpathlineto{\pgfqpoint{2.657780in}{2.020574in}}%
\pgfpathlineto{\pgfqpoint{2.644176in}{2.029383in}}%
\pgfpathlineto{\pgfqpoint{2.630573in}{2.038303in}}%
\pgfpathlineto{\pgfqpoint{2.616970in}{2.047333in}}%
\pgfpathlineto{\pgfqpoint{2.608387in}{2.041824in}}%
\pgfpathlineto{\pgfqpoint{2.599793in}{2.036437in}}%
\pgfpathlineto{\pgfqpoint{2.591188in}{2.031175in}}%
\pgfpathlineto{\pgfqpoint{2.582572in}{2.026042in}}%
\pgfpathclose%
\pgfusepath{fill}%
\end{pgfscope}%
\begin{pgfscope}%
\pgfpathrectangle{\pgfqpoint{1.150000in}{0.150000in}}{\pgfqpoint{5.700000in}{5.700000in}}%
\pgfusepath{clip}%
\pgfsetbuttcap%
\pgfsetroundjoin%
\definecolor{currentfill}{rgb}{0.277018,0.050344,0.375715}%
\pgfsetfillcolor{currentfill}%
\pgfsetfillopacity{0.700000}%
\pgfsetlinewidth{0.000000pt}%
\definecolor{currentstroke}{rgb}{0.000000,0.000000,0.000000}%
\pgfsetstrokecolor{currentstroke}%
\pgfsetdash{}{0pt}%
\pgfpathmoveto{\pgfqpoint{3.667003in}{1.933876in}}%
\pgfpathlineto{\pgfqpoint{3.680712in}{1.931453in}}%
\pgfpathlineto{\pgfqpoint{3.694428in}{1.929111in}}%
\pgfpathlineto{\pgfqpoint{3.708150in}{1.926849in}}%
\pgfpathlineto{\pgfqpoint{3.721879in}{1.924667in}}%
\pgfpathlineto{\pgfqpoint{3.729985in}{1.933819in}}%
\pgfpathlineto{\pgfqpoint{3.738084in}{1.942955in}}%
\pgfpathlineto{\pgfqpoint{3.746179in}{1.952075in}}%
\pgfpathlineto{\pgfqpoint{3.754267in}{1.961177in}}%
\pgfpathlineto{\pgfqpoint{3.740549in}{1.963297in}}%
\pgfpathlineto{\pgfqpoint{3.726838in}{1.965498in}}%
\pgfpathlineto{\pgfqpoint{3.713134in}{1.967779in}}%
\pgfpathlineto{\pgfqpoint{3.699436in}{1.970140in}}%
\pgfpathlineto{\pgfqpoint{3.691337in}{1.961092in}}%
\pgfpathlineto{\pgfqpoint{3.683231in}{1.952031in}}%
\pgfpathlineto{\pgfqpoint{3.675120in}{1.942959in}}%
\pgfpathlineto{\pgfqpoint{3.667003in}{1.933876in}}%
\pgfpathclose%
\pgfusepath{fill}%
\end{pgfscope}%
\begin{pgfscope}%
\pgfpathrectangle{\pgfqpoint{1.150000in}{0.150000in}}{\pgfqpoint{5.700000in}{5.700000in}}%
\pgfusepath{clip}%
\pgfsetbuttcap%
\pgfsetroundjoin%
\definecolor{currentfill}{rgb}{0.269944,0.014625,0.341379}%
\pgfsetfillcolor{currentfill}%
\pgfsetfillopacity{0.700000}%
\pgfsetlinewidth{0.000000pt}%
\definecolor{currentstroke}{rgb}{0.000000,0.000000,0.000000}%
\pgfsetstrokecolor{currentstroke}%
\pgfsetdash{}{0pt}%
\pgfpathmoveto{\pgfqpoint{3.207815in}{1.881752in}}%
\pgfpathlineto{\pgfqpoint{3.221447in}{1.876848in}}%
\pgfpathlineto{\pgfqpoint{3.235084in}{1.872032in}}%
\pgfpathlineto{\pgfqpoint{3.248725in}{1.867306in}}%
\pgfpathlineto{\pgfqpoint{3.262371in}{1.862668in}}%
\pgfpathlineto{\pgfqpoint{3.270651in}{1.871029in}}%
\pgfpathlineto{\pgfqpoint{3.278925in}{1.879426in}}%
\pgfpathlineto{\pgfqpoint{3.287192in}{1.887855in}}%
\pgfpathlineto{\pgfqpoint{3.295452in}{1.896315in}}%
\pgfpathlineto{\pgfqpoint{3.281821in}{1.900810in}}%
\pgfpathlineto{\pgfqpoint{3.268195in}{1.905393in}}%
\pgfpathlineto{\pgfqpoint{3.254573in}{1.910065in}}%
\pgfpathlineto{\pgfqpoint{3.240956in}{1.914826in}}%
\pgfpathlineto{\pgfqpoint{3.232681in}{1.906501in}}%
\pgfpathlineto{\pgfqpoint{3.224399in}{1.898213in}}%
\pgfpathlineto{\pgfqpoint{3.216110in}{1.889962in}}%
\pgfpathlineto{\pgfqpoint{3.207815in}{1.881752in}}%
\pgfpathclose%
\pgfusepath{fill}%
\end{pgfscope}%
\begin{pgfscope}%
\pgfpathrectangle{\pgfqpoint{1.150000in}{0.150000in}}{\pgfqpoint{5.700000in}{5.700000in}}%
\pgfusepath{clip}%
\pgfsetbuttcap%
\pgfsetroundjoin%
\definecolor{currentfill}{rgb}{0.212395,0.359683,0.551710}%
\pgfsetfillcolor{currentfill}%
\pgfsetfillopacity{0.700000}%
\pgfsetlinewidth{0.000000pt}%
\definecolor{currentstroke}{rgb}{0.000000,0.000000,0.000000}%
\pgfsetstrokecolor{currentstroke}%
\pgfsetdash{}{0pt}%
\pgfpathmoveto{\pgfqpoint{5.773278in}{2.596067in}}%
\pgfpathlineto{\pgfqpoint{5.787670in}{2.598054in}}%
\pgfpathlineto{\pgfqpoint{5.802074in}{2.600108in}}%
\pgfpathlineto{\pgfqpoint{5.816490in}{2.602230in}}%
\pgfpathlineto{\pgfqpoint{5.830918in}{2.604418in}}%
\pgfpathlineto{\pgfqpoint{5.838129in}{2.608826in}}%
\pgfpathlineto{\pgfqpoint{5.845338in}{2.613335in}}%
\pgfpathlineto{\pgfqpoint{5.852543in}{2.617950in}}%
\pgfpathlineto{\pgfqpoint{5.859746in}{2.622679in}}%
\pgfpathlineto{\pgfqpoint{5.845346in}{2.620869in}}%
\pgfpathlineto{\pgfqpoint{5.830958in}{2.619125in}}%
\pgfpathlineto{\pgfqpoint{5.816582in}{2.617449in}}%
\pgfpathlineto{\pgfqpoint{5.802217in}{2.615839in}}%
\pgfpathlineto{\pgfqpoint{5.794986in}{2.610726in}}%
\pgfpathlineto{\pgfqpoint{5.787753in}{2.605730in}}%
\pgfpathlineto{\pgfqpoint{5.780517in}{2.600846in}}%
\pgfpathlineto{\pgfqpoint{5.773278in}{2.596067in}}%
\pgfpathclose%
\pgfusepath{fill}%
\end{pgfscope}%
\begin{pgfscope}%
\pgfpathrectangle{\pgfqpoint{1.150000in}{0.150000in}}{\pgfqpoint{5.700000in}{5.700000in}}%
\pgfusepath{clip}%
\pgfsetbuttcap%
\pgfsetroundjoin%
\definecolor{currentfill}{rgb}{0.257322,0.256130,0.526563}%
\pgfsetfillcolor{currentfill}%
\pgfsetfillopacity{0.700000}%
\pgfsetlinewidth{0.000000pt}%
\definecolor{currentstroke}{rgb}{0.000000,0.000000,0.000000}%
\pgfsetstrokecolor{currentstroke}%
\pgfsetdash{}{0pt}%
\pgfpathmoveto{\pgfqpoint{4.965601in}{2.348624in}}%
\pgfpathlineto{\pgfqpoint{4.979712in}{2.350143in}}%
\pgfpathlineto{\pgfqpoint{4.993833in}{2.351733in}}%
\pgfpathlineto{\pgfqpoint{5.007966in}{2.353392in}}%
\pgfpathlineto{\pgfqpoint{5.022109in}{2.355121in}}%
\pgfpathlineto{\pgfqpoint{5.029706in}{2.361365in}}%
\pgfpathlineto{\pgfqpoint{5.037297in}{2.367594in}}%
\pgfpathlineto{\pgfqpoint{5.044881in}{2.373811in}}%
\pgfpathlineto{\pgfqpoint{5.052460in}{2.380021in}}%
\pgfpathlineto{\pgfqpoint{5.038334in}{2.378502in}}%
\pgfpathlineto{\pgfqpoint{5.024218in}{2.377052in}}%
\pgfpathlineto{\pgfqpoint{5.010113in}{2.375673in}}%
\pgfpathlineto{\pgfqpoint{4.996018in}{2.374364in}}%
\pgfpathlineto{\pgfqpoint{4.988423in}{2.367936in}}%
\pgfpathlineto{\pgfqpoint{4.980822in}{2.361507in}}%
\pgfpathlineto{\pgfqpoint{4.973214in}{2.355071in}}%
\pgfpathlineto{\pgfqpoint{4.965601in}{2.348624in}}%
\pgfpathclose%
\pgfusepath{fill}%
\end{pgfscope}%
\begin{pgfscope}%
\pgfpathrectangle{\pgfqpoint{1.150000in}{0.150000in}}{\pgfqpoint{5.700000in}{5.700000in}}%
\pgfusepath{clip}%
\pgfsetbuttcap%
\pgfsetroundjoin%
\definecolor{currentfill}{rgb}{0.278012,0.180367,0.486697}%
\pgfsetfillcolor{currentfill}%
\pgfsetfillopacity{0.700000}%
\pgfsetlinewidth{0.000000pt}%
\definecolor{currentstroke}{rgb}{0.000000,0.000000,0.000000}%
\pgfsetstrokecolor{currentstroke}%
\pgfsetdash{}{0pt}%
\pgfpathmoveto{\pgfqpoint{4.474647in}{2.179535in}}%
\pgfpathlineto{\pgfqpoint{4.488588in}{2.180075in}}%
\pgfpathlineto{\pgfqpoint{4.502539in}{2.180689in}}%
\pgfpathlineto{\pgfqpoint{4.516498in}{2.181374in}}%
\pgfpathlineto{\pgfqpoint{4.530468in}{2.182133in}}%
\pgfpathlineto{\pgfqpoint{4.538278in}{2.189951in}}%
\pgfpathlineto{\pgfqpoint{4.546081in}{2.197726in}}%
\pgfpathlineto{\pgfqpoint{4.553879in}{2.205459in}}%
\pgfpathlineto{\pgfqpoint{4.561670in}{2.213154in}}%
\pgfpathlineto{\pgfqpoint{4.547713in}{2.212500in}}%
\pgfpathlineto{\pgfqpoint{4.533765in}{2.211919in}}%
\pgfpathlineto{\pgfqpoint{4.519827in}{2.211411in}}%
\pgfpathlineto{\pgfqpoint{4.505898in}{2.210975in}}%
\pgfpathlineto{\pgfqpoint{4.498095in}{2.203169in}}%
\pgfpathlineto{\pgfqpoint{4.490285in}{2.195328in}}%
\pgfpathlineto{\pgfqpoint{4.482469in}{2.187451in}}%
\pgfpathlineto{\pgfqpoint{4.474647in}{2.179535in}}%
\pgfpathclose%
\pgfusepath{fill}%
\end{pgfscope}%
\begin{pgfscope}%
\pgfpathrectangle{\pgfqpoint{1.150000in}{0.150000in}}{\pgfqpoint{5.700000in}{5.700000in}}%
\pgfusepath{clip}%
\pgfsetbuttcap%
\pgfsetroundjoin%
\definecolor{currentfill}{rgb}{0.277018,0.050344,0.375715}%
\pgfsetfillcolor{currentfill}%
\pgfsetfillopacity{0.700000}%
\pgfsetlinewidth{0.000000pt}%
\definecolor{currentstroke}{rgb}{0.000000,0.000000,0.000000}%
\pgfsetstrokecolor{currentstroke}%
\pgfsetdash{}{0pt}%
\pgfpathmoveto{\pgfqpoint{2.780258in}{1.946125in}}%
\pgfpathlineto{\pgfqpoint{2.793874in}{1.938376in}}%
\pgfpathlineto{\pgfqpoint{2.807491in}{1.930730in}}%
\pgfpathlineto{\pgfqpoint{2.821110in}{1.923185in}}%
\pgfpathlineto{\pgfqpoint{2.834731in}{1.915741in}}%
\pgfpathlineto{\pgfqpoint{2.843214in}{1.922229in}}%
\pgfpathlineto{\pgfqpoint{2.851687in}{1.928811in}}%
\pgfpathlineto{\pgfqpoint{2.860151in}{1.935485in}}%
\pgfpathlineto{\pgfqpoint{2.868606in}{1.942249in}}%
\pgfpathlineto{\pgfqpoint{2.855006in}{1.949487in}}%
\pgfpathlineto{\pgfqpoint{2.841408in}{1.956825in}}%
\pgfpathlineto{\pgfqpoint{2.827812in}{1.964265in}}%
\pgfpathlineto{\pgfqpoint{2.814218in}{1.971807in}}%
\pgfpathlineto{\pgfqpoint{2.805742in}{1.965242in}}%
\pgfpathlineto{\pgfqpoint{2.797257in}{1.958771in}}%
\pgfpathlineto{\pgfqpoint{2.788762in}{1.952398in}}%
\pgfpathlineto{\pgfqpoint{2.780258in}{1.946125in}}%
\pgfpathclose%
\pgfusepath{fill}%
\end{pgfscope}%
\begin{pgfscope}%
\pgfpathrectangle{\pgfqpoint{1.150000in}{0.150000in}}{\pgfqpoint{5.700000in}{5.700000in}}%
\pgfusepath{clip}%
\pgfsetbuttcap%
\pgfsetroundjoin%
\definecolor{currentfill}{rgb}{0.235526,0.309527,0.542944}%
\pgfsetfillcolor{currentfill}%
\pgfsetfillopacity{0.700000}%
\pgfsetlinewidth{0.000000pt}%
\definecolor{currentstroke}{rgb}{0.000000,0.000000,0.000000}%
\pgfsetstrokecolor{currentstroke}%
\pgfsetdash{}{0pt}%
\pgfpathmoveto{\pgfqpoint{5.369552in}{2.477142in}}%
\pgfpathlineto{\pgfqpoint{5.383808in}{2.479078in}}%
\pgfpathlineto{\pgfqpoint{5.398075in}{2.481084in}}%
\pgfpathlineto{\pgfqpoint{5.412354in}{2.483157in}}%
\pgfpathlineto{\pgfqpoint{5.426644in}{2.485299in}}%
\pgfpathlineto{\pgfqpoint{5.434049in}{2.490380in}}%
\pgfpathlineto{\pgfqpoint{5.441449in}{2.495492in}}%
\pgfpathlineto{\pgfqpoint{5.448843in}{2.500640in}}%
\pgfpathlineto{\pgfqpoint{5.456233in}{2.505830in}}%
\pgfpathlineto{\pgfqpoint{5.441964in}{2.503983in}}%
\pgfpathlineto{\pgfqpoint{5.427707in}{2.502204in}}%
\pgfpathlineto{\pgfqpoint{5.413462in}{2.500494in}}%
\pgfpathlineto{\pgfqpoint{5.399227in}{2.498851in}}%
\pgfpathlineto{\pgfqpoint{5.391816in}{2.493359in}}%
\pgfpathlineto{\pgfqpoint{5.384400in}{2.487914in}}%
\pgfpathlineto{\pgfqpoint{5.376978in}{2.482510in}}%
\pgfpathlineto{\pgfqpoint{5.369552in}{2.477142in}}%
\pgfpathclose%
\pgfusepath{fill}%
\end{pgfscope}%
\begin{pgfscope}%
\pgfpathrectangle{\pgfqpoint{1.150000in}{0.150000in}}{\pgfqpoint{5.700000in}{5.700000in}}%
\pgfusepath{clip}%
\pgfsetbuttcap%
\pgfsetroundjoin%
\definecolor{currentfill}{rgb}{0.282656,0.100196,0.422160}%
\pgfsetfillcolor{currentfill}%
\pgfsetfillopacity{0.700000}%
\pgfsetlinewidth{0.000000pt}%
\definecolor{currentstroke}{rgb}{0.000000,0.000000,0.000000}%
\pgfsetstrokecolor{currentstroke}%
\pgfsetdash{}{0pt}%
\pgfpathmoveto{\pgfqpoint{3.983674in}{2.014782in}}%
\pgfpathlineto{\pgfqpoint{3.997465in}{2.013740in}}%
\pgfpathlineto{\pgfqpoint{4.011263in}{2.012775in}}%
\pgfpathlineto{\pgfqpoint{4.025069in}{2.011887in}}%
\pgfpathlineto{\pgfqpoint{4.038882in}{2.011075in}}%
\pgfpathlineto{\pgfqpoint{4.046877in}{2.020045in}}%
\pgfpathlineto{\pgfqpoint{4.054866in}{2.028977in}}%
\pgfpathlineto{\pgfqpoint{4.062850in}{2.037871in}}%
\pgfpathlineto{\pgfqpoint{4.070828in}{2.046729in}}%
\pgfpathlineto{\pgfqpoint{4.057024in}{2.047542in}}%
\pgfpathlineto{\pgfqpoint{4.043229in}{2.048431in}}%
\pgfpathlineto{\pgfqpoint{4.029441in}{2.049396in}}%
\pgfpathlineto{\pgfqpoint{4.015662in}{2.050439in}}%
\pgfpathlineto{\pgfqpoint{4.007673in}{2.041572in}}%
\pgfpathlineto{\pgfqpoint{3.999679in}{2.032675in}}%
\pgfpathlineto{\pgfqpoint{3.991680in}{2.023745in}}%
\pgfpathlineto{\pgfqpoint{3.983674in}{2.014782in}}%
\pgfpathclose%
\pgfusepath{fill}%
\end{pgfscope}%
\begin{pgfscope}%
\pgfpathrectangle{\pgfqpoint{1.150000in}{0.150000in}}{\pgfqpoint{5.700000in}{5.700000in}}%
\pgfusepath{clip}%
\pgfsetbuttcap%
\pgfsetroundjoin%
\definecolor{currentfill}{rgb}{0.282623,0.140926,0.457517}%
\pgfsetfillcolor{currentfill}%
\pgfsetfillopacity{0.700000}%
\pgfsetlinewidth{0.000000pt}%
\definecolor{currentstroke}{rgb}{0.000000,0.000000,0.000000}%
\pgfsetstrokecolor{currentstroke}%
\pgfsetdash{}{0pt}%
\pgfpathmoveto{\pgfqpoint{2.384118in}{2.129976in}}%
\pgfpathlineto{\pgfqpoint{2.397779in}{2.119062in}}%
\pgfpathlineto{\pgfqpoint{2.411439in}{2.108270in}}%
\pgfpathlineto{\pgfqpoint{2.425098in}{2.097599in}}%
\pgfpathlineto{\pgfqpoint{2.438755in}{2.087048in}}%
\pgfpathlineto{\pgfqpoint{2.447473in}{2.091115in}}%
\pgfpathlineto{\pgfqpoint{2.456178in}{2.095336in}}%
\pgfpathlineto{\pgfqpoint{2.464871in}{2.099705in}}%
\pgfpathlineto{\pgfqpoint{2.473551in}{2.104221in}}%
\pgfpathlineto{\pgfqpoint{2.459921in}{2.114520in}}%
\pgfpathlineto{\pgfqpoint{2.446290in}{2.124939in}}%
\pgfpathlineto{\pgfqpoint{2.432659in}{2.135479in}}%
\pgfpathlineto{\pgfqpoint{2.419026in}{2.146141in}}%
\pgfpathlineto{\pgfqpoint{2.410318in}{2.141869in}}%
\pgfpathlineto{\pgfqpoint{2.401598in}{2.137749in}}%
\pgfpathlineto{\pgfqpoint{2.392864in}{2.133783in}}%
\pgfpathlineto{\pgfqpoint{2.384118in}{2.129976in}}%
\pgfpathclose%
\pgfusepath{fill}%
\end{pgfscope}%
\begin{pgfscope}%
\pgfpathrectangle{\pgfqpoint{1.150000in}{0.150000in}}{\pgfqpoint{5.700000in}{5.700000in}}%
\pgfusepath{clip}%
\pgfsetbuttcap%
\pgfsetroundjoin%
\definecolor{currentfill}{rgb}{0.280255,0.165693,0.476498}%
\pgfsetfillcolor{currentfill}%
\pgfsetfillopacity{0.700000}%
\pgfsetlinewidth{0.000000pt}%
\definecolor{currentstroke}{rgb}{0.000000,0.000000,0.000000}%
\pgfsetstrokecolor{currentstroke}%
\pgfsetdash{}{0pt}%
\pgfpathmoveto{\pgfqpoint{4.387580in}{2.145683in}}%
\pgfpathlineto{\pgfqpoint{4.401496in}{2.146015in}}%
\pgfpathlineto{\pgfqpoint{4.415421in}{2.146420in}}%
\pgfpathlineto{\pgfqpoint{4.429355in}{2.146898in}}%
\pgfpathlineto{\pgfqpoint{4.443298in}{2.147450in}}%
\pgfpathlineto{\pgfqpoint{4.451145in}{2.155538in}}%
\pgfpathlineto{\pgfqpoint{4.458985in}{2.163581in}}%
\pgfpathlineto{\pgfqpoint{4.466819in}{2.171579in}}%
\pgfpathlineto{\pgfqpoint{4.474647in}{2.179535in}}%
\pgfpathlineto{\pgfqpoint{4.460715in}{2.179068in}}%
\pgfpathlineto{\pgfqpoint{4.446793in}{2.178673in}}%
\pgfpathlineto{\pgfqpoint{4.432880in}{2.178352in}}%
\pgfpathlineto{\pgfqpoint{4.418975in}{2.178104in}}%
\pgfpathlineto{\pgfqpoint{4.411136in}{2.170057in}}%
\pgfpathlineto{\pgfqpoint{4.403290in}{2.161972in}}%
\pgfpathlineto{\pgfqpoint{4.395438in}{2.153848in}}%
\pgfpathlineto{\pgfqpoint{4.387580in}{2.145683in}}%
\pgfpathclose%
\pgfusepath{fill}%
\end{pgfscope}%
\begin{pgfscope}%
\pgfpathrectangle{\pgfqpoint{1.150000in}{0.150000in}}{\pgfqpoint{5.700000in}{5.700000in}}%
\pgfusepath{clip}%
\pgfsetbuttcap%
\pgfsetroundjoin%
\definecolor{currentfill}{rgb}{0.262138,0.242286,0.520837}%
\pgfsetfillcolor{currentfill}%
\pgfsetfillopacity{0.700000}%
\pgfsetlinewidth{0.000000pt}%
\definecolor{currentstroke}{rgb}{0.000000,0.000000,0.000000}%
\pgfsetstrokecolor{currentstroke}%
\pgfsetdash{}{0pt}%
\pgfpathmoveto{\pgfqpoint{4.878681in}{2.316532in}}%
\pgfpathlineto{\pgfqpoint{4.892766in}{2.317959in}}%
\pgfpathlineto{\pgfqpoint{4.906861in}{2.319457in}}%
\pgfpathlineto{\pgfqpoint{4.920967in}{2.321025in}}%
\pgfpathlineto{\pgfqpoint{4.935083in}{2.322664in}}%
\pgfpathlineto{\pgfqpoint{4.942722in}{2.329187in}}%
\pgfpathlineto{\pgfqpoint{4.950355in}{2.335686in}}%
\pgfpathlineto{\pgfqpoint{4.957981in}{2.342164in}}%
\pgfpathlineto{\pgfqpoint{4.965601in}{2.348624in}}%
\pgfpathlineto{\pgfqpoint{4.951500in}{2.347175in}}%
\pgfpathlineto{\pgfqpoint{4.937410in}{2.345796in}}%
\pgfpathlineto{\pgfqpoint{4.923330in}{2.344488in}}%
\pgfpathlineto{\pgfqpoint{4.909261in}{2.343250in}}%
\pgfpathlineto{\pgfqpoint{4.901625in}{2.336593in}}%
\pgfpathlineto{\pgfqpoint{4.893983in}{2.329923in}}%
\pgfpathlineto{\pgfqpoint{4.886335in}{2.323238in}}%
\pgfpathlineto{\pgfqpoint{4.878681in}{2.316532in}}%
\pgfpathclose%
\pgfusepath{fill}%
\end{pgfscope}%
\begin{pgfscope}%
\pgfpathrectangle{\pgfqpoint{1.150000in}{0.150000in}}{\pgfqpoint{5.700000in}{5.700000in}}%
\pgfusepath{clip}%
\pgfsetbuttcap%
\pgfsetroundjoin%
\definecolor{currentfill}{rgb}{0.271305,0.019942,0.347269}%
\pgfsetfillcolor{currentfill}%
\pgfsetfillopacity{0.700000}%
\pgfsetlinewidth{0.000000pt}%
\definecolor{currentstroke}{rgb}{0.000000,0.000000,0.000000}%
\pgfsetstrokecolor{currentstroke}%
\pgfsetdash{}{0pt}%
\pgfpathmoveto{\pgfqpoint{3.350022in}{1.879208in}}%
\pgfpathlineto{\pgfqpoint{3.363677in}{1.875148in}}%
\pgfpathlineto{\pgfqpoint{3.377337in}{1.871173in}}%
\pgfpathlineto{\pgfqpoint{3.391002in}{1.867283in}}%
\pgfpathlineto{\pgfqpoint{3.404672in}{1.863479in}}%
\pgfpathlineto{\pgfqpoint{3.412898in}{1.872230in}}%
\pgfpathlineto{\pgfqpoint{3.421117in}{1.880998in}}%
\pgfpathlineto{\pgfqpoint{3.429331in}{1.889782in}}%
\pgfpathlineto{\pgfqpoint{3.437538in}{1.898580in}}%
\pgfpathlineto{\pgfqpoint{3.423881in}{1.902262in}}%
\pgfpathlineto{\pgfqpoint{3.410229in}{1.906028in}}%
\pgfpathlineto{\pgfqpoint{3.396583in}{1.909880in}}%
\pgfpathlineto{\pgfqpoint{3.382942in}{1.913818in}}%
\pgfpathlineto{\pgfqpoint{3.374722in}{1.905135in}}%
\pgfpathlineto{\pgfqpoint{3.366495in}{1.896471in}}%
\pgfpathlineto{\pgfqpoint{3.358262in}{1.887828in}}%
\pgfpathlineto{\pgfqpoint{3.350022in}{1.879208in}}%
\pgfpathclose%
\pgfusepath{fill}%
\end{pgfscope}%
\begin{pgfscope}%
\pgfpathrectangle{\pgfqpoint{1.150000in}{0.150000in}}{\pgfqpoint{5.700000in}{5.700000in}}%
\pgfusepath{clip}%
\pgfsetbuttcap%
\pgfsetroundjoin%
\definecolor{currentfill}{rgb}{0.274952,0.037752,0.364543}%
\pgfsetfillcolor{currentfill}%
\pgfsetfillopacity{0.700000}%
\pgfsetlinewidth{0.000000pt}%
\definecolor{currentstroke}{rgb}{0.000000,0.000000,0.000000}%
\pgfsetstrokecolor{currentstroke}%
\pgfsetdash{}{0pt}%
\pgfpathmoveto{\pgfqpoint{3.579658in}{1.908276in}}%
\pgfpathlineto{\pgfqpoint{3.593354in}{1.905447in}}%
\pgfpathlineto{\pgfqpoint{3.607056in}{1.902699in}}%
\pgfpathlineto{\pgfqpoint{3.620764in}{1.900033in}}%
\pgfpathlineto{\pgfqpoint{3.634478in}{1.897447in}}%
\pgfpathlineto{\pgfqpoint{3.642618in}{1.906567in}}%
\pgfpathlineto{\pgfqpoint{3.650752in}{1.915679in}}%
\pgfpathlineto{\pgfqpoint{3.658881in}{1.924782in}}%
\pgfpathlineto{\pgfqpoint{3.667003in}{1.933876in}}%
\pgfpathlineto{\pgfqpoint{3.653300in}{1.936379in}}%
\pgfpathlineto{\pgfqpoint{3.639604in}{1.938964in}}%
\pgfpathlineto{\pgfqpoint{3.625914in}{1.941630in}}%
\pgfpathlineto{\pgfqpoint{3.612230in}{1.944378in}}%
\pgfpathlineto{\pgfqpoint{3.604096in}{1.935358in}}%
\pgfpathlineto{\pgfqpoint{3.595956in}{1.926335in}}%
\pgfpathlineto{\pgfqpoint{3.587810in}{1.917307in}}%
\pgfpathlineto{\pgfqpoint{3.579658in}{1.908276in}}%
\pgfpathclose%
\pgfusepath{fill}%
\end{pgfscope}%
\begin{pgfscope}%
\pgfpathrectangle{\pgfqpoint{1.150000in}{0.150000in}}{\pgfqpoint{5.700000in}{5.700000in}}%
\pgfusepath{clip}%
\pgfsetbuttcap%
\pgfsetroundjoin%
\definecolor{currentfill}{rgb}{0.216210,0.351535,0.550627}%
\pgfsetfillcolor{currentfill}%
\pgfsetfillopacity{0.700000}%
\pgfsetlinewidth{0.000000pt}%
\definecolor{currentstroke}{rgb}{0.000000,0.000000,0.000000}%
\pgfsetstrokecolor{currentstroke}%
\pgfsetdash{}{0pt}%
\pgfpathmoveto{\pgfqpoint{5.686737in}{2.569165in}}%
\pgfpathlineto{\pgfqpoint{5.701108in}{2.571241in}}%
\pgfpathlineto{\pgfqpoint{5.715491in}{2.573384in}}%
\pgfpathlineto{\pgfqpoint{5.729885in}{2.575595in}}%
\pgfpathlineto{\pgfqpoint{5.744292in}{2.577873in}}%
\pgfpathlineto{\pgfqpoint{5.751544in}{2.582296in}}%
\pgfpathlineto{\pgfqpoint{5.758792in}{2.586798in}}%
\pgfpathlineto{\pgfqpoint{5.766037in}{2.591387in}}%
\pgfpathlineto{\pgfqpoint{5.773278in}{2.596067in}}%
\pgfpathlineto{\pgfqpoint{5.758898in}{2.594147in}}%
\pgfpathlineto{\pgfqpoint{5.744530in}{2.592294in}}%
\pgfpathlineto{\pgfqpoint{5.730173in}{2.590508in}}%
\pgfpathlineto{\pgfqpoint{5.715828in}{2.588789in}}%
\pgfpathlineto{\pgfqpoint{5.708560in}{2.583744in}}%
\pgfpathlineto{\pgfqpoint{5.701289in}{2.578796in}}%
\pgfpathlineto{\pgfqpoint{5.694015in}{2.573939in}}%
\pgfpathlineto{\pgfqpoint{5.686737in}{2.569165in}}%
\pgfpathclose%
\pgfusepath{fill}%
\end{pgfscope}%
\begin{pgfscope}%
\pgfpathrectangle{\pgfqpoint{1.150000in}{0.150000in}}{\pgfqpoint{5.700000in}{5.700000in}}%
\pgfusepath{clip}%
\pgfsetbuttcap%
\pgfsetroundjoin%
\definecolor{currentfill}{rgb}{0.281446,0.084320,0.407414}%
\pgfsetfillcolor{currentfill}%
\pgfsetfillopacity{0.700000}%
\pgfsetlinewidth{0.000000pt}%
\definecolor{currentstroke}{rgb}{0.000000,0.000000,0.000000}%
\pgfsetstrokecolor{currentstroke}%
\pgfsetdash{}{0pt}%
\pgfpathmoveto{\pgfqpoint{3.896469in}{1.983619in}}%
\pgfpathlineto{\pgfqpoint{3.910239in}{1.982248in}}%
\pgfpathlineto{\pgfqpoint{3.924017in}{1.980955in}}%
\pgfpathlineto{\pgfqpoint{3.937803in}{1.979739in}}%
\pgfpathlineto{\pgfqpoint{3.951596in}{1.978600in}}%
\pgfpathlineto{\pgfqpoint{3.959624in}{1.987696in}}%
\pgfpathlineto{\pgfqpoint{3.967647in}{1.996758in}}%
\pgfpathlineto{\pgfqpoint{3.975663in}{2.005787in}}%
\pgfpathlineto{\pgfqpoint{3.983674in}{2.014782in}}%
\pgfpathlineto{\pgfqpoint{3.969892in}{2.015901in}}%
\pgfpathlineto{\pgfqpoint{3.956117in}{2.017097in}}%
\pgfpathlineto{\pgfqpoint{3.942349in}{2.018370in}}%
\pgfpathlineto{\pgfqpoint{3.928589in}{2.019721in}}%
\pgfpathlineto{\pgfqpoint{3.920568in}{2.010738in}}%
\pgfpathlineto{\pgfqpoint{3.912540in}{2.001726in}}%
\pgfpathlineto{\pgfqpoint{3.904507in}{1.992687in}}%
\pgfpathlineto{\pgfqpoint{3.896469in}{1.983619in}}%
\pgfpathclose%
\pgfusepath{fill}%
\end{pgfscope}%
\begin{pgfscope}%
\pgfpathrectangle{\pgfqpoint{1.150000in}{0.150000in}}{\pgfqpoint{5.700000in}{5.700000in}}%
\pgfusepath{clip}%
\pgfsetbuttcap%
\pgfsetroundjoin%
\definecolor{currentfill}{rgb}{0.239346,0.300855,0.540844}%
\pgfsetfillcolor{currentfill}%
\pgfsetfillopacity{0.700000}%
\pgfsetlinewidth{0.000000pt}%
\definecolor{currentstroke}{rgb}{0.000000,0.000000,0.000000}%
\pgfsetstrokecolor{currentstroke}%
\pgfsetdash{}{0pt}%
\pgfpathmoveto{\pgfqpoint{5.282797in}{2.447776in}}%
\pgfpathlineto{\pgfqpoint{5.297028in}{2.449712in}}%
\pgfpathlineto{\pgfqpoint{5.311271in}{2.451717in}}%
\pgfpathlineto{\pgfqpoint{5.325525in}{2.453790in}}%
\pgfpathlineto{\pgfqpoint{5.339790in}{2.455933in}}%
\pgfpathlineto{\pgfqpoint{5.347239in}{2.461206in}}%
\pgfpathlineto{\pgfqpoint{5.354682in}{2.466495in}}%
\pgfpathlineto{\pgfqpoint{5.362120in}{2.471805in}}%
\pgfpathlineto{\pgfqpoint{5.369552in}{2.477142in}}%
\pgfpathlineto{\pgfqpoint{5.355307in}{2.475274in}}%
\pgfpathlineto{\pgfqpoint{5.341074in}{2.473474in}}%
\pgfpathlineto{\pgfqpoint{5.326851in}{2.471743in}}%
\pgfpathlineto{\pgfqpoint{5.312640in}{2.470080in}}%
\pgfpathlineto{\pgfqpoint{5.305188in}{2.464463in}}%
\pgfpathlineto{\pgfqpoint{5.297729in}{2.458876in}}%
\pgfpathlineto{\pgfqpoint{5.290266in}{2.453316in}}%
\pgfpathlineto{\pgfqpoint{5.282797in}{2.447776in}}%
\pgfpathclose%
\pgfusepath{fill}%
\end{pgfscope}%
\begin{pgfscope}%
\pgfpathrectangle{\pgfqpoint{1.150000in}{0.150000in}}{\pgfqpoint{5.700000in}{5.700000in}}%
\pgfusepath{clip}%
\pgfsetbuttcap%
\pgfsetroundjoin%
\definecolor{currentfill}{rgb}{0.280894,0.078907,0.402329}%
\pgfsetfillcolor{currentfill}%
\pgfsetfillopacity{0.700000}%
\pgfsetlinewidth{0.000000pt}%
\definecolor{currentstroke}{rgb}{0.000000,0.000000,0.000000}%
\pgfsetstrokecolor{currentstroke}%
\pgfsetdash{}{0pt}%
\pgfpathmoveto{\pgfqpoint{2.637084in}{1.989668in}}%
\pgfpathlineto{\pgfqpoint{2.650713in}{1.980849in}}%
\pgfpathlineto{\pgfqpoint{2.664343in}{1.972137in}}%
\pgfpathlineto{\pgfqpoint{2.677974in}{1.963533in}}%
\pgfpathlineto{\pgfqpoint{2.691605in}{1.955035in}}%
\pgfpathlineto{\pgfqpoint{2.700172in}{1.960641in}}%
\pgfpathlineto{\pgfqpoint{2.708729in}{1.966365in}}%
\pgfpathlineto{\pgfqpoint{2.717275in}{1.972204in}}%
\pgfpathlineto{\pgfqpoint{2.725812in}{1.978154in}}%
\pgfpathlineto{\pgfqpoint{2.712203in}{1.986424in}}%
\pgfpathlineto{\pgfqpoint{2.698596in}{1.994800in}}%
\pgfpathlineto{\pgfqpoint{2.684990in}{2.003283in}}%
\pgfpathlineto{\pgfqpoint{2.671385in}{2.011874in}}%
\pgfpathlineto{\pgfqpoint{2.662825in}{2.006144in}}%
\pgfpathlineto{\pgfqpoint{2.654255in}{2.000531in}}%
\pgfpathlineto{\pgfqpoint{2.645675in}{1.995038in}}%
\pgfpathlineto{\pgfqpoint{2.637084in}{1.989668in}}%
\pgfpathclose%
\pgfusepath{fill}%
\end{pgfscope}%
\begin{pgfscope}%
\pgfpathrectangle{\pgfqpoint{1.150000in}{0.150000in}}{\pgfqpoint{5.700000in}{5.700000in}}%
\pgfusepath{clip}%
\pgfsetbuttcap%
\pgfsetroundjoin%
\definecolor{currentfill}{rgb}{0.281887,0.150881,0.465405}%
\pgfsetfillcolor{currentfill}%
\pgfsetfillopacity{0.700000}%
\pgfsetlinewidth{0.000000pt}%
\definecolor{currentstroke}{rgb}{0.000000,0.000000,0.000000}%
\pgfsetstrokecolor{currentstroke}%
\pgfsetdash{}{0pt}%
\pgfpathmoveto{\pgfqpoint{4.300469in}{2.111737in}}%
\pgfpathlineto{\pgfqpoint{4.314361in}{2.111837in}}%
\pgfpathlineto{\pgfqpoint{4.328261in}{2.112011in}}%
\pgfpathlineto{\pgfqpoint{4.342170in}{2.112259in}}%
\pgfpathlineto{\pgfqpoint{4.356088in}{2.112580in}}%
\pgfpathlineto{\pgfqpoint{4.363970in}{2.120925in}}%
\pgfpathlineto{\pgfqpoint{4.371846in}{2.129223in}}%
\pgfpathlineto{\pgfqpoint{4.379716in}{2.137475in}}%
\pgfpathlineto{\pgfqpoint{4.387580in}{2.145683in}}%
\pgfpathlineto{\pgfqpoint{4.373673in}{2.145425in}}%
\pgfpathlineto{\pgfqpoint{4.359775in}{2.145240in}}%
\pgfpathlineto{\pgfqpoint{4.345886in}{2.145129in}}%
\pgfpathlineto{\pgfqpoint{4.332006in}{2.145092in}}%
\pgfpathlineto{\pgfqpoint{4.324131in}{2.136814in}}%
\pgfpathlineto{\pgfqpoint{4.316250in}{2.128496in}}%
\pgfpathlineto{\pgfqpoint{4.308363in}{2.120138in}}%
\pgfpathlineto{\pgfqpoint{4.300469in}{2.111737in}}%
\pgfpathclose%
\pgfusepath{fill}%
\end{pgfscope}%
\begin{pgfscope}%
\pgfpathrectangle{\pgfqpoint{1.150000in}{0.150000in}}{\pgfqpoint{5.700000in}{5.700000in}}%
\pgfusepath{clip}%
\pgfsetbuttcap%
\pgfsetroundjoin%
\definecolor{currentfill}{rgb}{0.265145,0.232956,0.516599}%
\pgfsetfillcolor{currentfill}%
\pgfsetfillopacity{0.700000}%
\pgfsetlinewidth{0.000000pt}%
\definecolor{currentstroke}{rgb}{0.000000,0.000000,0.000000}%
\pgfsetstrokecolor{currentstroke}%
\pgfsetdash{}{0pt}%
\pgfpathmoveto{\pgfqpoint{4.791704in}{2.283775in}}%
\pgfpathlineto{\pgfqpoint{4.805762in}{2.285087in}}%
\pgfpathlineto{\pgfqpoint{4.819831in}{2.286470in}}%
\pgfpathlineto{\pgfqpoint{4.833910in}{2.287924in}}%
\pgfpathlineto{\pgfqpoint{4.847999in}{2.289449in}}%
\pgfpathlineto{\pgfqpoint{4.855679in}{2.296266in}}%
\pgfpathlineto{\pgfqpoint{4.863353in}{2.303050in}}%
\pgfpathlineto{\pgfqpoint{4.871020in}{2.309804in}}%
\pgfpathlineto{\pgfqpoint{4.878681in}{2.316532in}}%
\pgfpathlineto{\pgfqpoint{4.864606in}{2.315176in}}%
\pgfpathlineto{\pgfqpoint{4.850542in}{2.313890in}}%
\pgfpathlineto{\pgfqpoint{4.836488in}{2.312675in}}%
\pgfpathlineto{\pgfqpoint{4.822444in}{2.311531in}}%
\pgfpathlineto{\pgfqpoint{4.814769in}{2.304627in}}%
\pgfpathlineto{\pgfqpoint{4.807087in}{2.297702in}}%
\pgfpathlineto{\pgfqpoint{4.799399in}{2.290752in}}%
\pgfpathlineto{\pgfqpoint{4.791704in}{2.283775in}}%
\pgfpathclose%
\pgfusepath{fill}%
\end{pgfscope}%
\begin{pgfscope}%
\pgfpathrectangle{\pgfqpoint{1.150000in}{0.150000in}}{\pgfqpoint{5.700000in}{5.700000in}}%
\pgfusepath{clip}%
\pgfsetbuttcap%
\pgfsetroundjoin%
\definecolor{currentfill}{rgb}{0.272594,0.025563,0.353093}%
\pgfsetfillcolor{currentfill}%
\pgfsetfillopacity{0.700000}%
\pgfsetlinewidth{0.000000pt}%
\definecolor{currentstroke}{rgb}{0.000000,0.000000,0.000000}%
\pgfsetstrokecolor{currentstroke}%
\pgfsetdash{}{0pt}%
\pgfpathmoveto{\pgfqpoint{2.977495in}{1.887902in}}%
\pgfpathlineto{\pgfqpoint{2.991118in}{1.881544in}}%
\pgfpathlineto{\pgfqpoint{3.004744in}{1.875281in}}%
\pgfpathlineto{\pgfqpoint{3.018373in}{1.869113in}}%
\pgfpathlineto{\pgfqpoint{3.032006in}{1.863038in}}%
\pgfpathlineto{\pgfqpoint{3.040395in}{1.870467in}}%
\pgfpathlineto{\pgfqpoint{3.048776in}{1.877964in}}%
\pgfpathlineto{\pgfqpoint{3.057149in}{1.885526in}}%
\pgfpathlineto{\pgfqpoint{3.065514in}{1.893153in}}%
\pgfpathlineto{\pgfqpoint{3.051900in}{1.899043in}}%
\pgfpathlineto{\pgfqpoint{3.038289in}{1.905026in}}%
\pgfpathlineto{\pgfqpoint{3.024681in}{1.911104in}}%
\pgfpathlineto{\pgfqpoint{3.011076in}{1.917277in}}%
\pgfpathlineto{\pgfqpoint{3.002693in}{1.909828in}}%
\pgfpathlineto{\pgfqpoint{2.994301in}{1.902448in}}%
\pgfpathlineto{\pgfqpoint{2.985902in}{1.895138in}}%
\pgfpathlineto{\pgfqpoint{2.977495in}{1.887902in}}%
\pgfpathclose%
\pgfusepath{fill}%
\end{pgfscope}%
\begin{pgfscope}%
\pgfpathrectangle{\pgfqpoint{1.150000in}{0.150000in}}{\pgfqpoint{5.700000in}{5.700000in}}%
\pgfusepath{clip}%
\pgfsetbuttcap%
\pgfsetroundjoin%
\definecolor{currentfill}{rgb}{0.269944,0.014625,0.341379}%
\pgfsetfillcolor{currentfill}%
\pgfsetfillopacity{0.700000}%
\pgfsetlinewidth{0.000000pt}%
\definecolor{currentstroke}{rgb}{0.000000,0.000000,0.000000}%
\pgfsetstrokecolor{currentstroke}%
\pgfsetdash{}{0pt}%
\pgfpathmoveto{\pgfqpoint{3.120008in}{1.870520in}}%
\pgfpathlineto{\pgfqpoint{3.133640in}{1.865092in}}%
\pgfpathlineto{\pgfqpoint{3.147277in}{1.859754in}}%
\pgfpathlineto{\pgfqpoint{3.160917in}{1.854507in}}%
\pgfpathlineto{\pgfqpoint{3.174561in}{1.849349in}}%
\pgfpathlineto{\pgfqpoint{3.182885in}{1.857380in}}%
\pgfpathlineto{\pgfqpoint{3.191202in}{1.865458in}}%
\pgfpathlineto{\pgfqpoint{3.199512in}{1.873583in}}%
\pgfpathlineto{\pgfqpoint{3.207815in}{1.881752in}}%
\pgfpathlineto{\pgfqpoint{3.194186in}{1.886745in}}%
\pgfpathlineto{\pgfqpoint{3.180562in}{1.891829in}}%
\pgfpathlineto{\pgfqpoint{3.166942in}{1.897002in}}%
\pgfpathlineto{\pgfqpoint{3.153326in}{1.902267in}}%
\pgfpathlineto{\pgfqpoint{3.145007in}{1.894255in}}%
\pgfpathlineto{\pgfqpoint{3.136682in}{1.886291in}}%
\pgfpathlineto{\pgfqpoint{3.128348in}{1.878379in}}%
\pgfpathlineto{\pgfqpoint{3.120008in}{1.870520in}}%
\pgfpathclose%
\pgfusepath{fill}%
\end{pgfscope}%
\begin{pgfscope}%
\pgfpathrectangle{\pgfqpoint{1.150000in}{0.150000in}}{\pgfqpoint{5.700000in}{5.700000in}}%
\pgfusepath{clip}%
\pgfsetbuttcap%
\pgfsetroundjoin%
\definecolor{currentfill}{rgb}{0.283187,0.125848,0.444960}%
\pgfsetfillcolor{currentfill}%
\pgfsetfillopacity{0.700000}%
\pgfsetlinewidth{0.000000pt}%
\definecolor{currentstroke}{rgb}{0.000000,0.000000,0.000000}%
\pgfsetstrokecolor{currentstroke}%
\pgfsetdash{}{0pt}%
\pgfpathmoveto{\pgfqpoint{2.438755in}{2.087048in}}%
\pgfpathlineto{\pgfqpoint{2.452412in}{2.076617in}}%
\pgfpathlineto{\pgfqpoint{2.466068in}{2.066304in}}%
\pgfpathlineto{\pgfqpoint{2.479723in}{2.056109in}}%
\pgfpathlineto{\pgfqpoint{2.493378in}{2.046030in}}%
\pgfpathlineto{\pgfqpoint{2.502067in}{2.050356in}}%
\pgfpathlineto{\pgfqpoint{2.510745in}{2.054830in}}%
\pgfpathlineto{\pgfqpoint{2.519410in}{2.059449in}}%
\pgfpathlineto{\pgfqpoint{2.528064in}{2.064208in}}%
\pgfpathlineto{\pgfqpoint{2.514436in}{2.074036in}}%
\pgfpathlineto{\pgfqpoint{2.500808in}{2.083980in}}%
\pgfpathlineto{\pgfqpoint{2.487180in}{2.094042in}}%
\pgfpathlineto{\pgfqpoint{2.473551in}{2.104221in}}%
\pgfpathlineto{\pgfqpoint{2.464871in}{2.099705in}}%
\pgfpathlineto{\pgfqpoint{2.456178in}{2.095336in}}%
\pgfpathlineto{\pgfqpoint{2.447473in}{2.091115in}}%
\pgfpathlineto{\pgfqpoint{2.438755in}{2.087048in}}%
\pgfpathclose%
\pgfusepath{fill}%
\end{pgfscope}%
\begin{pgfscope}%
\pgfpathrectangle{\pgfqpoint{1.150000in}{0.150000in}}{\pgfqpoint{5.700000in}{5.700000in}}%
\pgfusepath{clip}%
\pgfsetbuttcap%
\pgfsetroundjoin%
\definecolor{currentfill}{rgb}{0.280267,0.073417,0.397163}%
\pgfsetfillcolor{currentfill}%
\pgfsetfillopacity{0.700000}%
\pgfsetlinewidth{0.000000pt}%
\definecolor{currentstroke}{rgb}{0.000000,0.000000,0.000000}%
\pgfsetstrokecolor{currentstroke}%
\pgfsetdash{}{0pt}%
\pgfpathmoveto{\pgfqpoint{3.809206in}{1.953490in}}%
\pgfpathlineto{\pgfqpoint{3.822958in}{1.951765in}}%
\pgfpathlineto{\pgfqpoint{3.836718in}{1.950120in}}%
\pgfpathlineto{\pgfqpoint{3.850484in}{1.948552in}}%
\pgfpathlineto{\pgfqpoint{3.864258in}{1.947063in}}%
\pgfpathlineto{\pgfqpoint{3.872319in}{1.956244in}}%
\pgfpathlineto{\pgfqpoint{3.880375in}{1.965397in}}%
\pgfpathlineto{\pgfqpoint{3.888425in}{1.974522in}}%
\pgfpathlineto{\pgfqpoint{3.896469in}{1.983619in}}%
\pgfpathlineto{\pgfqpoint{3.882706in}{1.985068in}}%
\pgfpathlineto{\pgfqpoint{3.868950in}{1.986594in}}%
\pgfpathlineto{\pgfqpoint{3.855201in}{1.988199in}}%
\pgfpathlineto{\pgfqpoint{3.841460in}{1.989883in}}%
\pgfpathlineto{\pgfqpoint{3.833405in}{1.980819in}}%
\pgfpathlineto{\pgfqpoint{3.825344in}{1.971733in}}%
\pgfpathlineto{\pgfqpoint{3.817278in}{1.962623in}}%
\pgfpathlineto{\pgfqpoint{3.809206in}{1.953490in}}%
\pgfpathclose%
\pgfusepath{fill}%
\end{pgfscope}%
\begin{pgfscope}%
\pgfpathrectangle{\pgfqpoint{1.150000in}{0.150000in}}{\pgfqpoint{5.700000in}{5.700000in}}%
\pgfusepath{clip}%
\pgfsetbuttcap%
\pgfsetroundjoin%
\definecolor{currentfill}{rgb}{0.274952,0.037752,0.364543}%
\pgfsetfillcolor{currentfill}%
\pgfsetfillopacity{0.700000}%
\pgfsetlinewidth{0.000000pt}%
\definecolor{currentstroke}{rgb}{0.000000,0.000000,0.000000}%
\pgfsetstrokecolor{currentstroke}%
\pgfsetdash{}{0pt}%
\pgfpathmoveto{\pgfqpoint{2.834731in}{1.915741in}}%
\pgfpathlineto{\pgfqpoint{2.848354in}{1.908397in}}%
\pgfpathlineto{\pgfqpoint{2.861979in}{1.901153in}}%
\pgfpathlineto{\pgfqpoint{2.875607in}{1.894008in}}%
\pgfpathlineto{\pgfqpoint{2.889236in}{1.886962in}}%
\pgfpathlineto{\pgfqpoint{2.897698in}{1.893663in}}%
\pgfpathlineto{\pgfqpoint{2.906151in}{1.900454in}}%
\pgfpathlineto{\pgfqpoint{2.914595in}{1.907332in}}%
\pgfpathlineto{\pgfqpoint{2.923030in}{1.914294in}}%
\pgfpathlineto{\pgfqpoint{2.909421in}{1.921135in}}%
\pgfpathlineto{\pgfqpoint{2.895813in}{1.928074in}}%
\pgfpathlineto{\pgfqpoint{2.882209in}{1.935112in}}%
\pgfpathlineto{\pgfqpoint{2.868606in}{1.942249in}}%
\pgfpathlineto{\pgfqpoint{2.860151in}{1.935485in}}%
\pgfpathlineto{\pgfqpoint{2.851687in}{1.928811in}}%
\pgfpathlineto{\pgfqpoint{2.843214in}{1.922229in}}%
\pgfpathlineto{\pgfqpoint{2.834731in}{1.915741in}}%
\pgfpathclose%
\pgfusepath{fill}%
\end{pgfscope}%
\begin{pgfscope}%
\pgfpathrectangle{\pgfqpoint{1.150000in}{0.150000in}}{\pgfqpoint{5.700000in}{5.700000in}}%
\pgfusepath{clip}%
\pgfsetbuttcap%
\pgfsetroundjoin%
\definecolor{currentfill}{rgb}{0.220057,0.343307,0.549413}%
\pgfsetfillcolor{currentfill}%
\pgfsetfillopacity{0.700000}%
\pgfsetlinewidth{0.000000pt}%
\definecolor{currentstroke}{rgb}{0.000000,0.000000,0.000000}%
\pgfsetstrokecolor{currentstroke}%
\pgfsetdash{}{0pt}%
\pgfpathmoveto{\pgfqpoint{5.600118in}{2.541820in}}%
\pgfpathlineto{\pgfqpoint{5.614467in}{2.543962in}}%
\pgfpathlineto{\pgfqpoint{5.628827in}{2.546173in}}%
\pgfpathlineto{\pgfqpoint{5.643200in}{2.548451in}}%
\pgfpathlineto{\pgfqpoint{5.657584in}{2.550796in}}%
\pgfpathlineto{\pgfqpoint{5.664879in}{2.555292in}}%
\pgfpathlineto{\pgfqpoint{5.672169in}{2.559848in}}%
\pgfpathlineto{\pgfqpoint{5.679455in}{2.564471in}}%
\pgfpathlineto{\pgfqpoint{5.686737in}{2.569165in}}%
\pgfpathlineto{\pgfqpoint{5.672377in}{2.567157in}}%
\pgfpathlineto{\pgfqpoint{5.658030in}{2.565216in}}%
\pgfpathlineto{\pgfqpoint{5.643694in}{2.563342in}}%
\pgfpathlineto{\pgfqpoint{5.629370in}{2.561536in}}%
\pgfpathlineto{\pgfqpoint{5.622063in}{2.556497in}}%
\pgfpathlineto{\pgfqpoint{5.614752in}{2.551535in}}%
\pgfpathlineto{\pgfqpoint{5.607437in}{2.546645in}}%
\pgfpathlineto{\pgfqpoint{5.600118in}{2.541820in}}%
\pgfpathclose%
\pgfusepath{fill}%
\end{pgfscope}%
\begin{pgfscope}%
\pgfpathrectangle{\pgfqpoint{1.150000in}{0.150000in}}{\pgfqpoint{5.700000in}{5.700000in}}%
\pgfusepath{clip}%
\pgfsetbuttcap%
\pgfsetroundjoin%
\definecolor{currentfill}{rgb}{0.243113,0.292092,0.538516}%
\pgfsetfillcolor{currentfill}%
\pgfsetfillopacity{0.700000}%
\pgfsetlinewidth{0.000000pt}%
\definecolor{currentstroke}{rgb}{0.000000,0.000000,0.000000}%
\pgfsetstrokecolor{currentstroke}%
\pgfsetdash{}{0pt}%
\pgfpathmoveto{\pgfqpoint{5.195969in}{2.417673in}}%
\pgfpathlineto{\pgfqpoint{5.210175in}{2.419586in}}%
\pgfpathlineto{\pgfqpoint{5.224392in}{2.421568in}}%
\pgfpathlineto{\pgfqpoint{5.238621in}{2.423619in}}%
\pgfpathlineto{\pgfqpoint{5.252861in}{2.425739in}}%
\pgfpathlineto{\pgfqpoint{5.260354in}{2.431240in}}%
\pgfpathlineto{\pgfqpoint{5.267841in}{2.436743in}}%
\pgfpathlineto{\pgfqpoint{5.275322in}{2.442254in}}%
\pgfpathlineto{\pgfqpoint{5.282797in}{2.447776in}}%
\pgfpathlineto{\pgfqpoint{5.268576in}{2.445909in}}%
\pgfpathlineto{\pgfqpoint{5.254367in}{2.444111in}}%
\pgfpathlineto{\pgfqpoint{5.240169in}{2.442382in}}%
\pgfpathlineto{\pgfqpoint{5.225982in}{2.440722in}}%
\pgfpathlineto{\pgfqpoint{5.218488in}{2.434939in}}%
\pgfpathlineto{\pgfqpoint{5.210987in}{2.429173in}}%
\pgfpathlineto{\pgfqpoint{5.203481in}{2.423419in}}%
\pgfpathlineto{\pgfqpoint{5.195969in}{2.417673in}}%
\pgfpathclose%
\pgfusepath{fill}%
\end{pgfscope}%
\begin{pgfscope}%
\pgfpathrectangle{\pgfqpoint{1.150000in}{0.150000in}}{\pgfqpoint{5.700000in}{5.700000in}}%
\pgfusepath{clip}%
\pgfsetbuttcap%
\pgfsetroundjoin%
\definecolor{currentfill}{rgb}{0.282884,0.135920,0.453427}%
\pgfsetfillcolor{currentfill}%
\pgfsetfillopacity{0.700000}%
\pgfsetlinewidth{0.000000pt}%
\definecolor{currentstroke}{rgb}{0.000000,0.000000,0.000000}%
\pgfsetstrokecolor{currentstroke}%
\pgfsetdash{}{0pt}%
\pgfpathmoveto{\pgfqpoint{4.213317in}{2.077861in}}%
\pgfpathlineto{\pgfqpoint{4.227184in}{2.077706in}}%
\pgfpathlineto{\pgfqpoint{4.241060in}{2.077625in}}%
\pgfpathlineto{\pgfqpoint{4.254944in}{2.077619in}}%
\pgfpathlineto{\pgfqpoint{4.268837in}{2.077687in}}%
\pgfpathlineto{\pgfqpoint{4.276754in}{2.086269in}}%
\pgfpathlineto{\pgfqpoint{4.284665in}{2.094804in}}%
\pgfpathlineto{\pgfqpoint{4.292570in}{2.103293in}}%
\pgfpathlineto{\pgfqpoint{4.300469in}{2.111737in}}%
\pgfpathlineto{\pgfqpoint{4.286587in}{2.111711in}}%
\pgfpathlineto{\pgfqpoint{4.272713in}{2.111760in}}%
\pgfpathlineto{\pgfqpoint{4.258848in}{2.111883in}}%
\pgfpathlineto{\pgfqpoint{4.244992in}{2.112080in}}%
\pgfpathlineto{\pgfqpoint{4.237082in}{2.103586in}}%
\pgfpathlineto{\pgfqpoint{4.229166in}{2.095052in}}%
\pgfpathlineto{\pgfqpoint{4.221244in}{2.086478in}}%
\pgfpathlineto{\pgfqpoint{4.213317in}{2.077861in}}%
\pgfpathclose%
\pgfusepath{fill}%
\end{pgfscope}%
\begin{pgfscope}%
\pgfpathrectangle{\pgfqpoint{1.150000in}{0.150000in}}{\pgfqpoint{5.700000in}{5.700000in}}%
\pgfusepath{clip}%
\pgfsetbuttcap%
\pgfsetroundjoin%
\definecolor{currentfill}{rgb}{0.273809,0.031497,0.358853}%
\pgfsetfillcolor{currentfill}%
\pgfsetfillopacity{0.700000}%
\pgfsetlinewidth{0.000000pt}%
\definecolor{currentstroke}{rgb}{0.000000,0.000000,0.000000}%
\pgfsetstrokecolor{currentstroke}%
\pgfsetdash{}{0pt}%
\pgfpathmoveto{\pgfqpoint{3.492219in}{1.884697in}}%
\pgfpathlineto{\pgfqpoint{3.505904in}{1.881435in}}%
\pgfpathlineto{\pgfqpoint{3.519594in}{1.878256in}}%
\pgfpathlineto{\pgfqpoint{3.533290in}{1.875160in}}%
\pgfpathlineto{\pgfqpoint{3.546992in}{1.872147in}}%
\pgfpathlineto{\pgfqpoint{3.555167in}{1.881178in}}%
\pgfpathlineto{\pgfqpoint{3.563337in}{1.890211in}}%
\pgfpathlineto{\pgfqpoint{3.571500in}{1.899244in}}%
\pgfpathlineto{\pgfqpoint{3.579658in}{1.908276in}}%
\pgfpathlineto{\pgfqpoint{3.565968in}{1.911188in}}%
\pgfpathlineto{\pgfqpoint{3.552285in}{1.914182in}}%
\pgfpathlineto{\pgfqpoint{3.538607in}{1.917258in}}%
\pgfpathlineto{\pgfqpoint{3.524935in}{1.920418in}}%
\pgfpathlineto{\pgfqpoint{3.516765in}{1.911480in}}%
\pgfpathlineto{\pgfqpoint{3.508589in}{1.902547in}}%
\pgfpathlineto{\pgfqpoint{3.500407in}{1.893618in}}%
\pgfpathlineto{\pgfqpoint{3.492219in}{1.884697in}}%
\pgfpathclose%
\pgfusepath{fill}%
\end{pgfscope}%
\begin{pgfscope}%
\pgfpathrectangle{\pgfqpoint{1.150000in}{0.150000in}}{\pgfqpoint{5.700000in}{5.700000in}}%
\pgfusepath{clip}%
\pgfsetbuttcap%
\pgfsetroundjoin%
\definecolor{currentfill}{rgb}{0.269308,0.218818,0.509577}%
\pgfsetfillcolor{currentfill}%
\pgfsetfillopacity{0.700000}%
\pgfsetlinewidth{0.000000pt}%
\definecolor{currentstroke}{rgb}{0.000000,0.000000,0.000000}%
\pgfsetstrokecolor{currentstroke}%
\pgfsetdash{}{0pt}%
\pgfpathmoveto{\pgfqpoint{4.704674in}{2.250402in}}%
\pgfpathlineto{\pgfqpoint{4.718706in}{2.251577in}}%
\pgfpathlineto{\pgfqpoint{4.732747in}{2.252823in}}%
\pgfpathlineto{\pgfqpoint{4.746799in}{2.254141in}}%
\pgfpathlineto{\pgfqpoint{4.760862in}{2.255529in}}%
\pgfpathlineto{\pgfqpoint{4.768582in}{2.262647in}}%
\pgfpathlineto{\pgfqpoint{4.776296in}{2.269725in}}%
\pgfpathlineto{\pgfqpoint{4.784003in}{2.276767in}}%
\pgfpathlineto{\pgfqpoint{4.791704in}{2.283775in}}%
\pgfpathlineto{\pgfqpoint{4.777656in}{2.282533in}}%
\pgfpathlineto{\pgfqpoint{4.763618in}{2.281363in}}%
\pgfpathlineto{\pgfqpoint{4.749590in}{2.280264in}}%
\pgfpathlineto{\pgfqpoint{4.735571in}{2.279237in}}%
\pgfpathlineto{\pgfqpoint{4.727856in}{2.272074in}}%
\pgfpathlineto{\pgfqpoint{4.720135in}{2.264882in}}%
\pgfpathlineto{\pgfqpoint{4.712408in}{2.257660in}}%
\pgfpathlineto{\pgfqpoint{4.704674in}{2.250402in}}%
\pgfpathclose%
\pgfusepath{fill}%
\end{pgfscope}%
\begin{pgfscope}%
\pgfpathrectangle{\pgfqpoint{1.150000in}{0.150000in}}{\pgfqpoint{5.700000in}{5.700000in}}%
\pgfusepath{clip}%
\pgfsetbuttcap%
\pgfsetroundjoin%
\definecolor{currentfill}{rgb}{0.266580,0.228262,0.514349}%
\pgfsetfillcolor{currentfill}%
\pgfsetfillopacity{0.700000}%
\pgfsetlinewidth{0.000000pt}%
\definecolor{currentstroke}{rgb}{0.000000,0.000000,0.000000}%
\pgfsetstrokecolor{currentstroke}%
\pgfsetdash{}{0pt}%
\pgfpathmoveto{\pgfqpoint{2.129614in}{2.312967in}}%
\pgfpathlineto{\pgfqpoint{2.143349in}{2.299668in}}%
\pgfpathlineto{\pgfqpoint{2.157080in}{2.286510in}}%
\pgfpathlineto{\pgfqpoint{2.170807in}{2.273492in}}%
\pgfpathlineto{\pgfqpoint{2.184531in}{2.260612in}}%
\pgfpathlineto{\pgfqpoint{2.193429in}{2.262895in}}%
\pgfpathlineto{\pgfqpoint{2.202312in}{2.265368in}}%
\pgfpathlineto{\pgfqpoint{2.211179in}{2.268028in}}%
\pgfpathlineto{\pgfqpoint{2.220032in}{2.270870in}}%
\pgfpathlineto{\pgfqpoint{2.206341in}{2.283473in}}%
\pgfpathlineto{\pgfqpoint{2.192647in}{2.296213in}}%
\pgfpathlineto{\pgfqpoint{2.178950in}{2.309093in}}%
\pgfpathlineto{\pgfqpoint{2.165249in}{2.322113in}}%
\pgfpathlineto{\pgfqpoint{2.156364in}{2.319540in}}%
\pgfpathlineto{\pgfqpoint{2.147463in}{2.317156in}}%
\pgfpathlineto{\pgfqpoint{2.138547in}{2.314963in}}%
\pgfpathlineto{\pgfqpoint{2.129614in}{2.312967in}}%
\pgfpathclose%
\pgfusepath{fill}%
\end{pgfscope}%
\begin{pgfscope}%
\pgfpathrectangle{\pgfqpoint{1.150000in}{0.150000in}}{\pgfqpoint{5.700000in}{5.700000in}}%
\pgfusepath{clip}%
\pgfsetbuttcap%
\pgfsetroundjoin%
\definecolor{currentfill}{rgb}{0.269944,0.014625,0.341379}%
\pgfsetfillcolor{currentfill}%
\pgfsetfillopacity{0.700000}%
\pgfsetlinewidth{0.000000pt}%
\definecolor{currentstroke}{rgb}{0.000000,0.000000,0.000000}%
\pgfsetstrokecolor{currentstroke}%
\pgfsetdash{}{0pt}%
\pgfpathmoveto{\pgfqpoint{3.262371in}{1.862668in}}%
\pgfpathlineto{\pgfqpoint{3.276021in}{1.858117in}}%
\pgfpathlineto{\pgfqpoint{3.289676in}{1.853654in}}%
\pgfpathlineto{\pgfqpoint{3.303335in}{1.849277in}}%
\pgfpathlineto{\pgfqpoint{3.317000in}{1.844987in}}%
\pgfpathlineto{\pgfqpoint{3.325265in}{1.853500in}}%
\pgfpathlineto{\pgfqpoint{3.333524in}{1.862042in}}%
\pgfpathlineto{\pgfqpoint{3.341776in}{1.870612in}}%
\pgfpathlineto{\pgfqpoint{3.350022in}{1.879208in}}%
\pgfpathlineto{\pgfqpoint{3.336372in}{1.883355in}}%
\pgfpathlineto{\pgfqpoint{3.322727in}{1.887588in}}%
\pgfpathlineto{\pgfqpoint{3.309087in}{1.891908in}}%
\pgfpathlineto{\pgfqpoint{3.295452in}{1.896315in}}%
\pgfpathlineto{\pgfqpoint{3.287192in}{1.887855in}}%
\pgfpathlineto{\pgfqpoint{3.278925in}{1.879426in}}%
\pgfpathlineto{\pgfqpoint{3.270651in}{1.871029in}}%
\pgfpathlineto{\pgfqpoint{3.262371in}{1.862668in}}%
\pgfpathclose%
\pgfusepath{fill}%
\end{pgfscope}%
\begin{pgfscope}%
\pgfpathrectangle{\pgfqpoint{1.150000in}{0.150000in}}{\pgfqpoint{5.700000in}{5.700000in}}%
\pgfusepath{clip}%
\pgfsetbuttcap%
\pgfsetroundjoin%
\definecolor{currentfill}{rgb}{0.283229,0.120777,0.440584}%
\pgfsetfillcolor{currentfill}%
\pgfsetfillopacity{0.700000}%
\pgfsetlinewidth{0.000000pt}%
\definecolor{currentstroke}{rgb}{0.000000,0.000000,0.000000}%
\pgfsetstrokecolor{currentstroke}%
\pgfsetdash{}{0pt}%
\pgfpathmoveto{\pgfqpoint{4.126122in}{2.044238in}}%
\pgfpathlineto{\pgfqpoint{4.139966in}{2.043804in}}%
\pgfpathlineto{\pgfqpoint{4.153818in}{2.043446in}}%
\pgfpathlineto{\pgfqpoint{4.167679in}{2.043163in}}%
\pgfpathlineto{\pgfqpoint{4.181548in}{2.042954in}}%
\pgfpathlineto{\pgfqpoint{4.189499in}{2.051749in}}%
\pgfpathlineto{\pgfqpoint{4.197444in}{2.060497in}}%
\pgfpathlineto{\pgfqpoint{4.205383in}{2.069201in}}%
\pgfpathlineto{\pgfqpoint{4.213317in}{2.077861in}}%
\pgfpathlineto{\pgfqpoint{4.199458in}{2.078091in}}%
\pgfpathlineto{\pgfqpoint{4.185608in}{2.078396in}}%
\pgfpathlineto{\pgfqpoint{4.171766in}{2.078775in}}%
\pgfpathlineto{\pgfqpoint{4.157933in}{2.079231in}}%
\pgfpathlineto{\pgfqpoint{4.149989in}{2.070542in}}%
\pgfpathlineto{\pgfqpoint{4.142039in}{2.061814in}}%
\pgfpathlineto{\pgfqpoint{4.134083in}{2.053046in}}%
\pgfpathlineto{\pgfqpoint{4.126122in}{2.044238in}}%
\pgfpathclose%
\pgfusepath{fill}%
\end{pgfscope}%
\begin{pgfscope}%
\pgfpathrectangle{\pgfqpoint{1.150000in}{0.150000in}}{\pgfqpoint{5.700000in}{5.700000in}}%
\pgfusepath{clip}%
\pgfsetbuttcap%
\pgfsetroundjoin%
\definecolor{currentfill}{rgb}{0.277941,0.056324,0.381191}%
\pgfsetfillcolor{currentfill}%
\pgfsetfillopacity{0.700000}%
\pgfsetlinewidth{0.000000pt}%
\definecolor{currentstroke}{rgb}{0.000000,0.000000,0.000000}%
\pgfsetstrokecolor{currentstroke}%
\pgfsetdash{}{0pt}%
\pgfpathmoveto{\pgfqpoint{3.721879in}{1.924667in}}%
\pgfpathlineto{\pgfqpoint{3.735615in}{1.922565in}}%
\pgfpathlineto{\pgfqpoint{3.749357in}{1.920543in}}%
\pgfpathlineto{\pgfqpoint{3.763106in}{1.918600in}}%
\pgfpathlineto{\pgfqpoint{3.776862in}{1.916736in}}%
\pgfpathlineto{\pgfqpoint{3.784957in}{1.925957in}}%
\pgfpathlineto{\pgfqpoint{3.793046in}{1.935156in}}%
\pgfpathlineto{\pgfqpoint{3.801129in}{1.944334in}}%
\pgfpathlineto{\pgfqpoint{3.809206in}{1.953490in}}%
\pgfpathlineto{\pgfqpoint{3.795461in}{1.955293in}}%
\pgfpathlineto{\pgfqpoint{3.781723in}{1.957175in}}%
\pgfpathlineto{\pgfqpoint{3.767991in}{1.959136in}}%
\pgfpathlineto{\pgfqpoint{3.754267in}{1.961177in}}%
\pgfpathlineto{\pgfqpoint{3.746179in}{1.952075in}}%
\pgfpathlineto{\pgfqpoint{3.738084in}{1.942955in}}%
\pgfpathlineto{\pgfqpoint{3.729985in}{1.933819in}}%
\pgfpathlineto{\pgfqpoint{3.721879in}{1.924667in}}%
\pgfpathclose%
\pgfusepath{fill}%
\end{pgfscope}%
\begin{pgfscope}%
\pgfpathrectangle{\pgfqpoint{1.150000in}{0.150000in}}{\pgfqpoint{5.700000in}{5.700000in}}%
\pgfusepath{clip}%
\pgfsetbuttcap%
\pgfsetroundjoin%
\definecolor{currentfill}{rgb}{0.248629,0.278775,0.534556}%
\pgfsetfillcolor{currentfill}%
\pgfsetfillopacity{0.700000}%
\pgfsetlinewidth{0.000000pt}%
\definecolor{currentstroke}{rgb}{0.000000,0.000000,0.000000}%
\pgfsetstrokecolor{currentstroke}%
\pgfsetdash{}{0pt}%
\pgfpathmoveto{\pgfqpoint{5.109072in}{2.386794in}}%
\pgfpathlineto{\pgfqpoint{5.123252in}{2.388661in}}%
\pgfpathlineto{\pgfqpoint{5.137443in}{2.390598in}}%
\pgfpathlineto{\pgfqpoint{5.151646in}{2.392604in}}%
\pgfpathlineto{\pgfqpoint{5.165859in}{2.394680in}}%
\pgfpathlineto{\pgfqpoint{5.173396in}{2.400438in}}%
\pgfpathlineto{\pgfqpoint{5.180927in}{2.406187in}}%
\pgfpathlineto{\pgfqpoint{5.188451in}{2.411931in}}%
\pgfpathlineto{\pgfqpoint{5.195969in}{2.417673in}}%
\pgfpathlineto{\pgfqpoint{5.181774in}{2.415830in}}%
\pgfpathlineto{\pgfqpoint{5.167590in}{2.414055in}}%
\pgfpathlineto{\pgfqpoint{5.153417in}{2.412350in}}%
\pgfpathlineto{\pgfqpoint{5.139255in}{2.410715in}}%
\pgfpathlineto{\pgfqpoint{5.131718in}{2.404733in}}%
\pgfpathlineto{\pgfqpoint{5.124176in}{2.398754in}}%
\pgfpathlineto{\pgfqpoint{5.116627in}{2.392776in}}%
\pgfpathlineto{\pgfqpoint{5.109072in}{2.386794in}}%
\pgfpathclose%
\pgfusepath{fill}%
\end{pgfscope}%
\begin{pgfscope}%
\pgfpathrectangle{\pgfqpoint{1.150000in}{0.150000in}}{\pgfqpoint{5.700000in}{5.700000in}}%
\pgfusepath{clip}%
\pgfsetbuttcap%
\pgfsetroundjoin%
\definecolor{currentfill}{rgb}{0.273006,0.204520,0.501721}%
\pgfsetfillcolor{currentfill}%
\pgfsetfillopacity{0.700000}%
\pgfsetlinewidth{0.000000pt}%
\definecolor{currentstroke}{rgb}{0.000000,0.000000,0.000000}%
\pgfsetstrokecolor{currentstroke}%
\pgfsetdash{}{0pt}%
\pgfpathmoveto{\pgfqpoint{4.617594in}{2.216490in}}%
\pgfpathlineto{\pgfqpoint{4.631599in}{2.217504in}}%
\pgfpathlineto{\pgfqpoint{4.645614in}{2.218590in}}%
\pgfpathlineto{\pgfqpoint{4.659639in}{2.219748in}}%
\pgfpathlineto{\pgfqpoint{4.673674in}{2.220978in}}%
\pgfpathlineto{\pgfqpoint{4.681434in}{2.228398in}}%
\pgfpathlineto{\pgfqpoint{4.689187in}{2.235774in}}%
\pgfpathlineto{\pgfqpoint{4.696933in}{2.243108in}}%
\pgfpathlineto{\pgfqpoint{4.704674in}{2.250402in}}%
\pgfpathlineto{\pgfqpoint{4.690652in}{2.249299in}}%
\pgfpathlineto{\pgfqpoint{4.676640in}{2.248268in}}%
\pgfpathlineto{\pgfqpoint{4.662638in}{2.247308in}}%
\pgfpathlineto{\pgfqpoint{4.648646in}{2.246420in}}%
\pgfpathlineto{\pgfqpoint{4.640892in}{2.238991in}}%
\pgfpathlineto{\pgfqpoint{4.633132in}{2.231529in}}%
\pgfpathlineto{\pgfqpoint{4.625366in}{2.224029in}}%
\pgfpathlineto{\pgfqpoint{4.617594in}{2.216490in}}%
\pgfpathclose%
\pgfusepath{fill}%
\end{pgfscope}%
\begin{pgfscope}%
\pgfpathrectangle{\pgfqpoint{1.150000in}{0.150000in}}{\pgfqpoint{5.700000in}{5.700000in}}%
\pgfusepath{clip}%
\pgfsetbuttcap%
\pgfsetroundjoin%
\definecolor{currentfill}{rgb}{0.278791,0.062145,0.386592}%
\pgfsetfillcolor{currentfill}%
\pgfsetfillopacity{0.700000}%
\pgfsetlinewidth{0.000000pt}%
\definecolor{currentstroke}{rgb}{0.000000,0.000000,0.000000}%
\pgfsetstrokecolor{currentstroke}%
\pgfsetdash{}{0pt}%
\pgfpathmoveto{\pgfqpoint{2.691605in}{1.955035in}}%
\pgfpathlineto{\pgfqpoint{2.705238in}{1.946644in}}%
\pgfpathlineto{\pgfqpoint{2.718872in}{1.938357in}}%
\pgfpathlineto{\pgfqpoint{2.732507in}{1.930174in}}%
\pgfpathlineto{\pgfqpoint{2.746144in}{1.922096in}}%
\pgfpathlineto{\pgfqpoint{2.754687in}{1.927937in}}%
\pgfpathlineto{\pgfqpoint{2.763221in}{1.933891in}}%
\pgfpathlineto{\pgfqpoint{2.771744in}{1.939955in}}%
\pgfpathlineto{\pgfqpoint{2.780258in}{1.946125in}}%
\pgfpathlineto{\pgfqpoint{2.766644in}{1.953976in}}%
\pgfpathlineto{\pgfqpoint{2.753032in}{1.961931in}}%
\pgfpathlineto{\pgfqpoint{2.739421in}{1.969990in}}%
\pgfpathlineto{\pgfqpoint{2.725812in}{1.978154in}}%
\pgfpathlineto{\pgfqpoint{2.717275in}{1.972204in}}%
\pgfpathlineto{\pgfqpoint{2.708729in}{1.966365in}}%
\pgfpathlineto{\pgfqpoint{2.700172in}{1.960641in}}%
\pgfpathlineto{\pgfqpoint{2.691605in}{1.955035in}}%
\pgfpathclose%
\pgfusepath{fill}%
\end{pgfscope}%
\begin{pgfscope}%
\pgfpathrectangle{\pgfqpoint{1.150000in}{0.150000in}}{\pgfqpoint{5.700000in}{5.700000in}}%
\pgfusepath{clip}%
\pgfsetbuttcap%
\pgfsetroundjoin%
\definecolor{currentfill}{rgb}{0.271828,0.209303,0.504434}%
\pgfsetfillcolor{currentfill}%
\pgfsetfillopacity{0.700000}%
\pgfsetlinewidth{0.000000pt}%
\definecolor{currentstroke}{rgb}{0.000000,0.000000,0.000000}%
\pgfsetstrokecolor{currentstroke}%
\pgfsetdash{}{0pt}%
\pgfpathmoveto{\pgfqpoint{2.184531in}{2.260612in}}%
\pgfpathlineto{\pgfqpoint{2.198251in}{2.247870in}}%
\pgfpathlineto{\pgfqpoint{2.211968in}{2.235263in}}%
\pgfpathlineto{\pgfqpoint{2.225682in}{2.222791in}}%
\pgfpathlineto{\pgfqpoint{2.239394in}{2.210452in}}%
\pgfpathlineto{\pgfqpoint{2.248259in}{2.213020in}}%
\pgfpathlineto{\pgfqpoint{2.257109in}{2.215773in}}%
\pgfpathlineto{\pgfqpoint{2.265944in}{2.218706in}}%
\pgfpathlineto{\pgfqpoint{2.274765in}{2.221817in}}%
\pgfpathlineto{\pgfqpoint{2.261086in}{2.233880in}}%
\pgfpathlineto{\pgfqpoint{2.247404in}{2.246075in}}%
\pgfpathlineto{\pgfqpoint{2.233720in}{2.258405in}}%
\pgfpathlineto{\pgfqpoint{2.220032in}{2.270870in}}%
\pgfpathlineto{\pgfqpoint{2.211179in}{2.268028in}}%
\pgfpathlineto{\pgfqpoint{2.202312in}{2.265368in}}%
\pgfpathlineto{\pgfqpoint{2.193429in}{2.262895in}}%
\pgfpathlineto{\pgfqpoint{2.184531in}{2.260612in}}%
\pgfpathclose%
\pgfusepath{fill}%
\end{pgfscope}%
\begin{pgfscope}%
\pgfpathrectangle{\pgfqpoint{1.150000in}{0.150000in}}{\pgfqpoint{5.700000in}{5.700000in}}%
\pgfusepath{clip}%
\pgfsetbuttcap%
\pgfsetroundjoin%
\definecolor{currentfill}{rgb}{0.223925,0.334994,0.548053}%
\pgfsetfillcolor{currentfill}%
\pgfsetfillopacity{0.700000}%
\pgfsetlinewidth{0.000000pt}%
\definecolor{currentstroke}{rgb}{0.000000,0.000000,0.000000}%
\pgfsetstrokecolor{currentstroke}%
\pgfsetdash{}{0pt}%
\pgfpathmoveto{\pgfqpoint{5.513420in}{2.513900in}}%
\pgfpathlineto{\pgfqpoint{5.527746in}{2.516087in}}%
\pgfpathlineto{\pgfqpoint{5.542083in}{2.518343in}}%
\pgfpathlineto{\pgfqpoint{5.556433in}{2.520667in}}%
\pgfpathlineto{\pgfqpoint{5.570794in}{2.523058in}}%
\pgfpathlineto{\pgfqpoint{5.578132in}{2.527679in}}%
\pgfpathlineto{\pgfqpoint{5.585465in}{2.532342in}}%
\pgfpathlineto{\pgfqpoint{5.592794in}{2.537054in}}%
\pgfpathlineto{\pgfqpoint{5.600118in}{2.541820in}}%
\pgfpathlineto{\pgfqpoint{5.585780in}{2.539745in}}%
\pgfpathlineto{\pgfqpoint{5.571455in}{2.537737in}}%
\pgfpathlineto{\pgfqpoint{5.557141in}{2.535798in}}%
\pgfpathlineto{\pgfqpoint{5.542838in}{2.533926in}}%
\pgfpathlineto{\pgfqpoint{5.535491in}{2.528837in}}%
\pgfpathlineto{\pgfqpoint{5.528139in}{2.523807in}}%
\pgfpathlineto{\pgfqpoint{5.520782in}{2.518829in}}%
\pgfpathlineto{\pgfqpoint{5.513420in}{2.513900in}}%
\pgfpathclose%
\pgfusepath{fill}%
\end{pgfscope}%
\begin{pgfscope}%
\pgfpathrectangle{\pgfqpoint{1.150000in}{0.150000in}}{\pgfqpoint{5.700000in}{5.700000in}}%
\pgfusepath{clip}%
\pgfsetbuttcap%
\pgfsetroundjoin%
\definecolor{currentfill}{rgb}{0.283091,0.110553,0.431554}%
\pgfsetfillcolor{currentfill}%
\pgfsetfillopacity{0.700000}%
\pgfsetlinewidth{0.000000pt}%
\definecolor{currentstroke}{rgb}{0.000000,0.000000,0.000000}%
\pgfsetstrokecolor{currentstroke}%
\pgfsetdash{}{0pt}%
\pgfpathmoveto{\pgfqpoint{2.493378in}{2.046030in}}%
\pgfpathlineto{\pgfqpoint{2.507032in}{2.036067in}}%
\pgfpathlineto{\pgfqpoint{2.520685in}{2.026219in}}%
\pgfpathlineto{\pgfqpoint{2.534339in}{2.016484in}}%
\pgfpathlineto{\pgfqpoint{2.547992in}{2.006863in}}%
\pgfpathlineto{\pgfqpoint{2.556655in}{2.011447in}}%
\pgfpathlineto{\pgfqpoint{2.565306in}{2.016174in}}%
\pgfpathlineto{\pgfqpoint{2.573945in}{2.021040in}}%
\pgfpathlineto{\pgfqpoint{2.582572in}{2.026042in}}%
\pgfpathlineto{\pgfqpoint{2.568945in}{2.035413in}}%
\pgfpathlineto{\pgfqpoint{2.555318in}{2.044897in}}%
\pgfpathlineto{\pgfqpoint{2.541691in}{2.054495in}}%
\pgfpathlineto{\pgfqpoint{2.528064in}{2.064208in}}%
\pgfpathlineto{\pgfqpoint{2.519410in}{2.059449in}}%
\pgfpathlineto{\pgfqpoint{2.510745in}{2.054830in}}%
\pgfpathlineto{\pgfqpoint{2.502067in}{2.050356in}}%
\pgfpathlineto{\pgfqpoint{2.493378in}{2.046030in}}%
\pgfpathclose%
\pgfusepath{fill}%
\end{pgfscope}%
\begin{pgfscope}%
\pgfpathrectangle{\pgfqpoint{1.150000in}{0.150000in}}{\pgfqpoint{5.700000in}{5.700000in}}%
\pgfusepath{clip}%
\pgfsetbuttcap%
\pgfsetroundjoin%
\definecolor{currentfill}{rgb}{0.206756,0.371758,0.553117}%
\pgfsetfillcolor{currentfill}%
\pgfsetfillopacity{0.700000}%
\pgfsetlinewidth{0.000000pt}%
\definecolor{currentstroke}{rgb}{0.000000,0.000000,0.000000}%
\pgfsetstrokecolor{currentstroke}%
\pgfsetdash{}{0pt}%
\pgfpathmoveto{\pgfqpoint{5.830918in}{2.604418in}}%
\pgfpathlineto{\pgfqpoint{5.845358in}{2.606673in}}%
\pgfpathlineto{\pgfqpoint{5.859811in}{2.608995in}}%
\pgfpathlineto{\pgfqpoint{5.874275in}{2.611385in}}%
\pgfpathlineto{\pgfqpoint{5.881465in}{2.615514in}}%
\pgfpathlineto{\pgfqpoint{5.888652in}{2.619740in}}%
\pgfpathlineto{\pgfqpoint{5.895836in}{2.624070in}}%
\pgfpathlineto{\pgfqpoint{5.903018in}{2.628509in}}%
\pgfpathlineto{\pgfqpoint{5.888582in}{2.626499in}}%
\pgfpathlineto{\pgfqpoint{5.874158in}{2.624555in}}%
\pgfpathlineto{\pgfqpoint{5.859746in}{2.622679in}}%
\pgfpathlineto{\pgfqpoint{5.852543in}{2.617950in}}%
\pgfpathlineto{\pgfqpoint{5.845338in}{2.613335in}}%
\pgfpathlineto{\pgfqpoint{5.838129in}{2.608826in}}%
\pgfpathlineto{\pgfqpoint{5.830918in}{2.604418in}}%
\pgfpathclose%
\pgfusepath{fill}%
\end{pgfscope}%
\begin{pgfscope}%
\pgfpathrectangle{\pgfqpoint{1.150000in}{0.150000in}}{\pgfqpoint{5.700000in}{5.700000in}}%
\pgfusepath{clip}%
\pgfsetbuttcap%
\pgfsetroundjoin%
\definecolor{currentfill}{rgb}{0.271305,0.019942,0.347269}%
\pgfsetfillcolor{currentfill}%
\pgfsetfillopacity{0.700000}%
\pgfsetlinewidth{0.000000pt}%
\definecolor{currentstroke}{rgb}{0.000000,0.000000,0.000000}%
\pgfsetstrokecolor{currentstroke}%
\pgfsetdash{}{0pt}%
\pgfpathmoveto{\pgfqpoint{3.404672in}{1.863479in}}%
\pgfpathlineto{\pgfqpoint{3.418348in}{1.859760in}}%
\pgfpathlineto{\pgfqpoint{3.432029in}{1.856124in}}%
\pgfpathlineto{\pgfqpoint{3.445715in}{1.852573in}}%
\pgfpathlineto{\pgfqpoint{3.459407in}{1.849105in}}%
\pgfpathlineto{\pgfqpoint{3.467619in}{1.857986in}}%
\pgfpathlineto{\pgfqpoint{3.475825in}{1.866879in}}%
\pgfpathlineto{\pgfqpoint{3.484025in}{1.875784in}}%
\pgfpathlineto{\pgfqpoint{3.492219in}{1.884697in}}%
\pgfpathlineto{\pgfqpoint{3.478541in}{1.888042in}}%
\pgfpathlineto{\pgfqpoint{3.464867in}{1.891471in}}%
\pgfpathlineto{\pgfqpoint{3.451200in}{1.894983in}}%
\pgfpathlineto{\pgfqpoint{3.437538in}{1.898580in}}%
\pgfpathlineto{\pgfqpoint{3.429331in}{1.889782in}}%
\pgfpathlineto{\pgfqpoint{3.421117in}{1.880998in}}%
\pgfpathlineto{\pgfqpoint{3.412898in}{1.872230in}}%
\pgfpathlineto{\pgfqpoint{3.404672in}{1.863479in}}%
\pgfpathclose%
\pgfusepath{fill}%
\end{pgfscope}%
\begin{pgfscope}%
\pgfpathrectangle{\pgfqpoint{1.150000in}{0.150000in}}{\pgfqpoint{5.700000in}{5.700000in}}%
\pgfusepath{clip}%
\pgfsetbuttcap%
\pgfsetroundjoin%
\definecolor{currentfill}{rgb}{0.282910,0.105393,0.426902}%
\pgfsetfillcolor{currentfill}%
\pgfsetfillopacity{0.700000}%
\pgfsetlinewidth{0.000000pt}%
\definecolor{currentstroke}{rgb}{0.000000,0.000000,0.000000}%
\pgfsetstrokecolor{currentstroke}%
\pgfsetdash{}{0pt}%
\pgfpathmoveto{\pgfqpoint{4.038882in}{2.011075in}}%
\pgfpathlineto{\pgfqpoint{4.052704in}{2.010339in}}%
\pgfpathlineto{\pgfqpoint{4.066534in}{2.009679in}}%
\pgfpathlineto{\pgfqpoint{4.080372in}{2.009095in}}%
\pgfpathlineto{\pgfqpoint{4.094218in}{2.008587in}}%
\pgfpathlineto{\pgfqpoint{4.102203in}{2.017564in}}%
\pgfpathlineto{\pgfqpoint{4.110181in}{2.026497in}}%
\pgfpathlineto{\pgfqpoint{4.118154in}{2.035389in}}%
\pgfpathlineto{\pgfqpoint{4.126122in}{2.044238in}}%
\pgfpathlineto{\pgfqpoint{4.112286in}{2.044747in}}%
\pgfpathlineto{\pgfqpoint{4.098458in}{2.045332in}}%
\pgfpathlineto{\pgfqpoint{4.084639in}{2.045992in}}%
\pgfpathlineto{\pgfqpoint{4.070828in}{2.046729in}}%
\pgfpathlineto{\pgfqpoint{4.062850in}{2.037871in}}%
\pgfpathlineto{\pgfqpoint{4.054866in}{2.028977in}}%
\pgfpathlineto{\pgfqpoint{4.046877in}{2.020045in}}%
\pgfpathlineto{\pgfqpoint{4.038882in}{2.011075in}}%
\pgfpathclose%
\pgfusepath{fill}%
\end{pgfscope}%
\begin{pgfscope}%
\pgfpathrectangle{\pgfqpoint{1.150000in}{0.150000in}}{\pgfqpoint{5.700000in}{5.700000in}}%
\pgfusepath{clip}%
\pgfsetbuttcap%
\pgfsetroundjoin%
\definecolor{currentfill}{rgb}{0.276194,0.190074,0.493001}%
\pgfsetfillcolor{currentfill}%
\pgfsetfillopacity{0.700000}%
\pgfsetlinewidth{0.000000pt}%
\definecolor{currentstroke}{rgb}{0.000000,0.000000,0.000000}%
\pgfsetstrokecolor{currentstroke}%
\pgfsetdash{}{0pt}%
\pgfpathmoveto{\pgfqpoint{4.530468in}{2.182133in}}%
\pgfpathlineto{\pgfqpoint{4.544446in}{2.182964in}}%
\pgfpathlineto{\pgfqpoint{4.558435in}{2.183867in}}%
\pgfpathlineto{\pgfqpoint{4.572433in}{2.184843in}}%
\pgfpathlineto{\pgfqpoint{4.586441in}{2.185890in}}%
\pgfpathlineto{\pgfqpoint{4.594239in}{2.193611in}}%
\pgfpathlineto{\pgfqpoint{4.602030in}{2.201283in}}%
\pgfpathlineto{\pgfqpoint{4.609815in}{2.208908in}}%
\pgfpathlineto{\pgfqpoint{4.617594in}{2.216490in}}%
\pgfpathlineto{\pgfqpoint{4.603598in}{2.215547in}}%
\pgfpathlineto{\pgfqpoint{4.589613in}{2.214677in}}%
\pgfpathlineto{\pgfqpoint{4.575636in}{2.213879in}}%
\pgfpathlineto{\pgfqpoint{4.561670in}{2.213154in}}%
\pgfpathlineto{\pgfqpoint{4.553879in}{2.205459in}}%
\pgfpathlineto{\pgfqpoint{4.546081in}{2.197726in}}%
\pgfpathlineto{\pgfqpoint{4.538278in}{2.189951in}}%
\pgfpathlineto{\pgfqpoint{4.530468in}{2.182133in}}%
\pgfpathclose%
\pgfusepath{fill}%
\end{pgfscope}%
\begin{pgfscope}%
\pgfpathrectangle{\pgfqpoint{1.150000in}{0.150000in}}{\pgfqpoint{5.700000in}{5.700000in}}%
\pgfusepath{clip}%
\pgfsetbuttcap%
\pgfsetroundjoin%
\definecolor{currentfill}{rgb}{0.252194,0.269783,0.531579}%
\pgfsetfillcolor{currentfill}%
\pgfsetfillopacity{0.700000}%
\pgfsetlinewidth{0.000000pt}%
\definecolor{currentstroke}{rgb}{0.000000,0.000000,0.000000}%
\pgfsetstrokecolor{currentstroke}%
\pgfsetdash{}{0pt}%
\pgfpathmoveto{\pgfqpoint{5.022109in}{2.355121in}}%
\pgfpathlineto{\pgfqpoint{5.036263in}{2.356921in}}%
\pgfpathlineto{\pgfqpoint{5.050427in}{2.358790in}}%
\pgfpathlineto{\pgfqpoint{5.064603in}{2.360729in}}%
\pgfpathlineto{\pgfqpoint{5.078789in}{2.362739in}}%
\pgfpathlineto{\pgfqpoint{5.086369in}{2.368779in}}%
\pgfpathlineto{\pgfqpoint{5.093943in}{2.374799in}}%
\pgfpathlineto{\pgfqpoint{5.101511in}{2.380802in}}%
\pgfpathlineto{\pgfqpoint{5.109072in}{2.386794in}}%
\pgfpathlineto{\pgfqpoint{5.094903in}{2.384996in}}%
\pgfpathlineto{\pgfqpoint{5.080744in}{2.383268in}}%
\pgfpathlineto{\pgfqpoint{5.066597in}{2.381609in}}%
\pgfpathlineto{\pgfqpoint{5.052460in}{2.380021in}}%
\pgfpathlineto{\pgfqpoint{5.044881in}{2.373811in}}%
\pgfpathlineto{\pgfqpoint{5.037297in}{2.367594in}}%
\pgfpathlineto{\pgfqpoint{5.029706in}{2.361365in}}%
\pgfpathlineto{\pgfqpoint{5.022109in}{2.355121in}}%
\pgfpathclose%
\pgfusepath{fill}%
\end{pgfscope}%
\begin{pgfscope}%
\pgfpathrectangle{\pgfqpoint{1.150000in}{0.150000in}}{\pgfqpoint{5.700000in}{5.700000in}}%
\pgfusepath{clip}%
\pgfsetbuttcap%
\pgfsetroundjoin%
\definecolor{currentfill}{rgb}{0.271305,0.019942,0.347269}%
\pgfsetfillcolor{currentfill}%
\pgfsetfillopacity{0.700000}%
\pgfsetlinewidth{0.000000pt}%
\definecolor{currentstroke}{rgb}{0.000000,0.000000,0.000000}%
\pgfsetstrokecolor{currentstroke}%
\pgfsetdash{}{0pt}%
\pgfpathmoveto{\pgfqpoint{3.032006in}{1.863038in}}%
\pgfpathlineto{\pgfqpoint{3.045642in}{1.857057in}}%
\pgfpathlineto{\pgfqpoint{3.059281in}{1.851168in}}%
\pgfpathlineto{\pgfqpoint{3.072924in}{1.845372in}}%
\pgfpathlineto{\pgfqpoint{3.086570in}{1.839667in}}%
\pgfpathlineto{\pgfqpoint{3.094941in}{1.847288in}}%
\pgfpathlineto{\pgfqpoint{3.103304in}{1.854972in}}%
\pgfpathlineto{\pgfqpoint{3.111660in}{1.862717in}}%
\pgfpathlineto{\pgfqpoint{3.120008in}{1.870520in}}%
\pgfpathlineto{\pgfqpoint{3.106379in}{1.876040in}}%
\pgfpathlineto{\pgfqpoint{3.092754in}{1.881652in}}%
\pgfpathlineto{\pgfqpoint{3.079132in}{1.887356in}}%
\pgfpathlineto{\pgfqpoint{3.065514in}{1.893153in}}%
\pgfpathlineto{\pgfqpoint{3.057149in}{1.885526in}}%
\pgfpathlineto{\pgfqpoint{3.048776in}{1.877964in}}%
\pgfpathlineto{\pgfqpoint{3.040395in}{1.870467in}}%
\pgfpathlineto{\pgfqpoint{3.032006in}{1.863038in}}%
\pgfpathclose%
\pgfusepath{fill}%
\end{pgfscope}%
\begin{pgfscope}%
\pgfpathrectangle{\pgfqpoint{1.150000in}{0.150000in}}{\pgfqpoint{5.700000in}{5.700000in}}%
\pgfusepath{clip}%
\pgfsetbuttcap%
\pgfsetroundjoin%
\definecolor{currentfill}{rgb}{0.276194,0.190074,0.493001}%
\pgfsetfillcolor{currentfill}%
\pgfsetfillopacity{0.700000}%
\pgfsetlinewidth{0.000000pt}%
\definecolor{currentstroke}{rgb}{0.000000,0.000000,0.000000}%
\pgfsetstrokecolor{currentstroke}%
\pgfsetdash{}{0pt}%
\pgfpathmoveto{\pgfqpoint{2.239394in}{2.210452in}}%
\pgfpathlineto{\pgfqpoint{2.253102in}{2.198246in}}%
\pgfpathlineto{\pgfqpoint{2.266808in}{2.186171in}}%
\pgfpathlineto{\pgfqpoint{2.280511in}{2.174226in}}%
\pgfpathlineto{\pgfqpoint{2.294212in}{2.162410in}}%
\pgfpathlineto{\pgfqpoint{2.303045in}{2.165261in}}%
\pgfpathlineto{\pgfqpoint{2.311863in}{2.168292in}}%
\pgfpathlineto{\pgfqpoint{2.320667in}{2.171498in}}%
\pgfpathlineto{\pgfqpoint{2.329457in}{2.174877in}}%
\pgfpathlineto{\pgfqpoint{2.315787in}{2.186418in}}%
\pgfpathlineto{\pgfqpoint{2.302116in}{2.198087in}}%
\pgfpathlineto{\pgfqpoint{2.288441in}{2.209887in}}%
\pgfpathlineto{\pgfqpoint{2.274765in}{2.221817in}}%
\pgfpathlineto{\pgfqpoint{2.265944in}{2.218706in}}%
\pgfpathlineto{\pgfqpoint{2.257109in}{2.215773in}}%
\pgfpathlineto{\pgfqpoint{2.248259in}{2.213020in}}%
\pgfpathlineto{\pgfqpoint{2.239394in}{2.210452in}}%
\pgfpathclose%
\pgfusepath{fill}%
\end{pgfscope}%
\begin{pgfscope}%
\pgfpathrectangle{\pgfqpoint{1.150000in}{0.150000in}}{\pgfqpoint{5.700000in}{5.700000in}}%
\pgfusepath{clip}%
\pgfsetbuttcap%
\pgfsetroundjoin%
\definecolor{currentfill}{rgb}{0.276022,0.044167,0.370164}%
\pgfsetfillcolor{currentfill}%
\pgfsetfillopacity{0.700000}%
\pgfsetlinewidth{0.000000pt}%
\definecolor{currentstroke}{rgb}{0.000000,0.000000,0.000000}%
\pgfsetstrokecolor{currentstroke}%
\pgfsetdash{}{0pt}%
\pgfpathmoveto{\pgfqpoint{3.634478in}{1.897447in}}%
\pgfpathlineto{\pgfqpoint{3.648199in}{1.894943in}}%
\pgfpathlineto{\pgfqpoint{3.661926in}{1.892519in}}%
\pgfpathlineto{\pgfqpoint{3.675660in}{1.890176in}}%
\pgfpathlineto{\pgfqpoint{3.689400in}{1.887912in}}%
\pgfpathlineto{\pgfqpoint{3.697528in}{1.897122in}}%
\pgfpathlineto{\pgfqpoint{3.705651in}{1.906318in}}%
\pgfpathlineto{\pgfqpoint{3.713768in}{1.915500in}}%
\pgfpathlineto{\pgfqpoint{3.721879in}{1.924667in}}%
\pgfpathlineto{\pgfqpoint{3.708150in}{1.926849in}}%
\pgfpathlineto{\pgfqpoint{3.694428in}{1.929111in}}%
\pgfpathlineto{\pgfqpoint{3.680712in}{1.931453in}}%
\pgfpathlineto{\pgfqpoint{3.667003in}{1.933876in}}%
\pgfpathlineto{\pgfqpoint{3.658881in}{1.924782in}}%
\pgfpathlineto{\pgfqpoint{3.650752in}{1.915679in}}%
\pgfpathlineto{\pgfqpoint{3.642618in}{1.906567in}}%
\pgfpathlineto{\pgfqpoint{3.634478in}{1.897447in}}%
\pgfpathclose%
\pgfusepath{fill}%
\end{pgfscope}%
\begin{pgfscope}%
\pgfpathrectangle{\pgfqpoint{1.150000in}{0.150000in}}{\pgfqpoint{5.700000in}{5.700000in}}%
\pgfusepath{clip}%
\pgfsetbuttcap%
\pgfsetroundjoin%
\definecolor{currentfill}{rgb}{0.273809,0.031497,0.358853}%
\pgfsetfillcolor{currentfill}%
\pgfsetfillopacity{0.700000}%
\pgfsetlinewidth{0.000000pt}%
\definecolor{currentstroke}{rgb}{0.000000,0.000000,0.000000}%
\pgfsetstrokecolor{currentstroke}%
\pgfsetdash{}{0pt}%
\pgfpathmoveto{\pgfqpoint{2.889236in}{1.886962in}}%
\pgfpathlineto{\pgfqpoint{2.902868in}{1.880013in}}%
\pgfpathlineto{\pgfqpoint{2.916503in}{1.873161in}}%
\pgfpathlineto{\pgfqpoint{2.930140in}{1.866406in}}%
\pgfpathlineto{\pgfqpoint{2.943780in}{1.859747in}}%
\pgfpathlineto{\pgfqpoint{2.952222in}{1.866662in}}%
\pgfpathlineto{\pgfqpoint{2.960654in}{1.873661in}}%
\pgfpathlineto{\pgfqpoint{2.969079in}{1.880742in}}%
\pgfpathlineto{\pgfqpoint{2.977495in}{1.887902in}}%
\pgfpathlineto{\pgfqpoint{2.963874in}{1.894356in}}%
\pgfpathlineto{\pgfqpoint{2.950257in}{1.900905in}}%
\pgfpathlineto{\pgfqpoint{2.936642in}{1.907551in}}%
\pgfpathlineto{\pgfqpoint{2.923030in}{1.914294in}}%
\pgfpathlineto{\pgfqpoint{2.914595in}{1.907332in}}%
\pgfpathlineto{\pgfqpoint{2.906151in}{1.900454in}}%
\pgfpathlineto{\pgfqpoint{2.897698in}{1.893663in}}%
\pgfpathlineto{\pgfqpoint{2.889236in}{1.886962in}}%
\pgfpathclose%
\pgfusepath{fill}%
\end{pgfscope}%
\begin{pgfscope}%
\pgfpathrectangle{\pgfqpoint{1.150000in}{0.150000in}}{\pgfqpoint{5.700000in}{5.700000in}}%
\pgfusepath{clip}%
\pgfsetbuttcap%
\pgfsetroundjoin%
\definecolor{currentfill}{rgb}{0.227802,0.326594,0.546532}%
\pgfsetfillcolor{currentfill}%
\pgfsetfillopacity{0.700000}%
\pgfsetlinewidth{0.000000pt}%
\definecolor{currentstroke}{rgb}{0.000000,0.000000,0.000000}%
\pgfsetstrokecolor{currentstroke}%
\pgfsetdash{}{0pt}%
\pgfpathmoveto{\pgfqpoint{5.426644in}{2.485299in}}%
\pgfpathlineto{\pgfqpoint{5.440945in}{2.487509in}}%
\pgfpathlineto{\pgfqpoint{5.455259in}{2.489788in}}%
\pgfpathlineto{\pgfqpoint{5.469584in}{2.492135in}}%
\pgfpathlineto{\pgfqpoint{5.483920in}{2.494551in}}%
\pgfpathlineto{\pgfqpoint{5.491304in}{2.499343in}}%
\pgfpathlineto{\pgfqpoint{5.498681in}{2.504162in}}%
\pgfpathlineto{\pgfqpoint{5.506053in}{2.509012in}}%
\pgfpathlineto{\pgfqpoint{5.513420in}{2.513900in}}%
\pgfpathlineto{\pgfqpoint{5.499106in}{2.511780in}}%
\pgfpathlineto{\pgfqpoint{5.484803in}{2.509729in}}%
\pgfpathlineto{\pgfqpoint{5.470512in}{2.507746in}}%
\pgfpathlineto{\pgfqpoint{5.456233in}{2.505830in}}%
\pgfpathlineto{\pgfqpoint{5.448843in}{2.500640in}}%
\pgfpathlineto{\pgfqpoint{5.441449in}{2.495492in}}%
\pgfpathlineto{\pgfqpoint{5.434049in}{2.490380in}}%
\pgfpathlineto{\pgfqpoint{5.426644in}{2.485299in}}%
\pgfpathclose%
\pgfusepath{fill}%
\end{pgfscope}%
\begin{pgfscope}%
\pgfpathrectangle{\pgfqpoint{1.150000in}{0.150000in}}{\pgfqpoint{5.700000in}{5.700000in}}%
\pgfusepath{clip}%
\pgfsetbuttcap%
\pgfsetroundjoin%
\definecolor{currentfill}{rgb}{0.269944,0.014625,0.341379}%
\pgfsetfillcolor{currentfill}%
\pgfsetfillopacity{0.700000}%
\pgfsetlinewidth{0.000000pt}%
\definecolor{currentstroke}{rgb}{0.000000,0.000000,0.000000}%
\pgfsetstrokecolor{currentstroke}%
\pgfsetdash{}{0pt}%
\pgfpathmoveto{\pgfqpoint{3.174561in}{1.849349in}}%
\pgfpathlineto{\pgfqpoint{3.188210in}{1.844281in}}%
\pgfpathlineto{\pgfqpoint{3.201862in}{1.839302in}}%
\pgfpathlineto{\pgfqpoint{3.215519in}{1.834412in}}%
\pgfpathlineto{\pgfqpoint{3.229181in}{1.829610in}}%
\pgfpathlineto{\pgfqpoint{3.237489in}{1.837812in}}%
\pgfpathlineto{\pgfqpoint{3.245789in}{1.846057in}}%
\pgfpathlineto{\pgfqpoint{3.254084in}{1.854343in}}%
\pgfpathlineto{\pgfqpoint{3.262371in}{1.862668in}}%
\pgfpathlineto{\pgfqpoint{3.248725in}{1.867306in}}%
\pgfpathlineto{\pgfqpoint{3.235084in}{1.872032in}}%
\pgfpathlineto{\pgfqpoint{3.221447in}{1.876848in}}%
\pgfpathlineto{\pgfqpoint{3.207815in}{1.881752in}}%
\pgfpathlineto{\pgfqpoint{3.199512in}{1.873583in}}%
\pgfpathlineto{\pgfqpoint{3.191202in}{1.865458in}}%
\pgfpathlineto{\pgfqpoint{3.182885in}{1.857380in}}%
\pgfpathlineto{\pgfqpoint{3.174561in}{1.849349in}}%
\pgfpathclose%
\pgfusepath{fill}%
\end{pgfscope}%
\begin{pgfscope}%
\pgfpathrectangle{\pgfqpoint{1.150000in}{0.150000in}}{\pgfqpoint{5.700000in}{5.700000in}}%
\pgfusepath{clip}%
\pgfsetbuttcap%
\pgfsetroundjoin%
\definecolor{currentfill}{rgb}{0.282327,0.094955,0.417331}%
\pgfsetfillcolor{currentfill}%
\pgfsetfillopacity{0.700000}%
\pgfsetlinewidth{0.000000pt}%
\definecolor{currentstroke}{rgb}{0.000000,0.000000,0.000000}%
\pgfsetstrokecolor{currentstroke}%
\pgfsetdash{}{0pt}%
\pgfpathmoveto{\pgfqpoint{3.951596in}{1.978600in}}%
\pgfpathlineto{\pgfqpoint{3.965397in}{1.977538in}}%
\pgfpathlineto{\pgfqpoint{3.979206in}{1.976553in}}%
\pgfpathlineto{\pgfqpoint{3.993022in}{1.975645in}}%
\pgfpathlineto{\pgfqpoint{4.006846in}{1.974813in}}%
\pgfpathlineto{\pgfqpoint{4.014864in}{1.983937in}}%
\pgfpathlineto{\pgfqpoint{4.022876in}{1.993021in}}%
\pgfpathlineto{\pgfqpoint{4.030882in}{2.002067in}}%
\pgfpathlineto{\pgfqpoint{4.038882in}{2.011075in}}%
\pgfpathlineto{\pgfqpoint{4.025069in}{2.011887in}}%
\pgfpathlineto{\pgfqpoint{4.011263in}{2.012775in}}%
\pgfpathlineto{\pgfqpoint{3.997465in}{2.013740in}}%
\pgfpathlineto{\pgfqpoint{3.983674in}{2.014782in}}%
\pgfpathlineto{\pgfqpoint{3.975663in}{2.005787in}}%
\pgfpathlineto{\pgfqpoint{3.967647in}{1.996758in}}%
\pgfpathlineto{\pgfqpoint{3.959624in}{1.987696in}}%
\pgfpathlineto{\pgfqpoint{3.951596in}{1.978600in}}%
\pgfpathclose%
\pgfusepath{fill}%
\end{pgfscope}%
\begin{pgfscope}%
\pgfpathrectangle{\pgfqpoint{1.150000in}{0.150000in}}{\pgfqpoint{5.700000in}{5.700000in}}%
\pgfusepath{clip}%
\pgfsetbuttcap%
\pgfsetroundjoin%
\definecolor{currentfill}{rgb}{0.278826,0.175490,0.483397}%
\pgfsetfillcolor{currentfill}%
\pgfsetfillopacity{0.700000}%
\pgfsetlinewidth{0.000000pt}%
\definecolor{currentstroke}{rgb}{0.000000,0.000000,0.000000}%
\pgfsetstrokecolor{currentstroke}%
\pgfsetdash{}{0pt}%
\pgfpathmoveto{\pgfqpoint{4.443298in}{2.147450in}}%
\pgfpathlineto{\pgfqpoint{4.457251in}{2.148074in}}%
\pgfpathlineto{\pgfqpoint{4.471213in}{2.148772in}}%
\pgfpathlineto{\pgfqpoint{4.485184in}{2.149542in}}%
\pgfpathlineto{\pgfqpoint{4.499166in}{2.150385in}}%
\pgfpathlineto{\pgfqpoint{4.507000in}{2.158397in}}%
\pgfpathlineto{\pgfqpoint{4.514829in}{2.166358in}}%
\pgfpathlineto{\pgfqpoint{4.522651in}{2.174269in}}%
\pgfpathlineto{\pgfqpoint{4.530468in}{2.182133in}}%
\pgfpathlineto{\pgfqpoint{4.516498in}{2.181374in}}%
\pgfpathlineto{\pgfqpoint{4.502539in}{2.180689in}}%
\pgfpathlineto{\pgfqpoint{4.488588in}{2.180075in}}%
\pgfpathlineto{\pgfqpoint{4.474647in}{2.179535in}}%
\pgfpathlineto{\pgfqpoint{4.466819in}{2.171579in}}%
\pgfpathlineto{\pgfqpoint{4.458985in}{2.163581in}}%
\pgfpathlineto{\pgfqpoint{4.451145in}{2.155538in}}%
\pgfpathlineto{\pgfqpoint{4.443298in}{2.147450in}}%
\pgfpathclose%
\pgfusepath{fill}%
\end{pgfscope}%
\begin{pgfscope}%
\pgfpathrectangle{\pgfqpoint{1.150000in}{0.150000in}}{\pgfqpoint{5.700000in}{5.700000in}}%
\pgfusepath{clip}%
\pgfsetbuttcap%
\pgfsetroundjoin%
\definecolor{currentfill}{rgb}{0.257322,0.256130,0.526563}%
\pgfsetfillcolor{currentfill}%
\pgfsetfillopacity{0.700000}%
\pgfsetlinewidth{0.000000pt}%
\definecolor{currentstroke}{rgb}{0.000000,0.000000,0.000000}%
\pgfsetstrokecolor{currentstroke}%
\pgfsetdash{}{0pt}%
\pgfpathmoveto{\pgfqpoint{4.935083in}{2.322664in}}%
\pgfpathlineto{\pgfqpoint{4.949210in}{2.324373in}}%
\pgfpathlineto{\pgfqpoint{4.963348in}{2.326152in}}%
\pgfpathlineto{\pgfqpoint{4.977496in}{2.328001in}}%
\pgfpathlineto{\pgfqpoint{4.991655in}{2.329921in}}%
\pgfpathlineto{\pgfqpoint{4.999278in}{2.336262in}}%
\pgfpathlineto{\pgfqpoint{5.006895in}{2.342574in}}%
\pgfpathlineto{\pgfqpoint{5.014505in}{2.348859in}}%
\pgfpathlineto{\pgfqpoint{5.022109in}{2.355121in}}%
\pgfpathlineto{\pgfqpoint{5.007966in}{2.353392in}}%
\pgfpathlineto{\pgfqpoint{4.993833in}{2.351733in}}%
\pgfpathlineto{\pgfqpoint{4.979712in}{2.350143in}}%
\pgfpathlineto{\pgfqpoint{4.965601in}{2.348624in}}%
\pgfpathlineto{\pgfqpoint{4.957981in}{2.342164in}}%
\pgfpathlineto{\pgfqpoint{4.950355in}{2.335686in}}%
\pgfpathlineto{\pgfqpoint{4.942722in}{2.329187in}}%
\pgfpathlineto{\pgfqpoint{4.935083in}{2.322664in}}%
\pgfpathclose%
\pgfusepath{fill}%
\end{pgfscope}%
\begin{pgfscope}%
\pgfpathrectangle{\pgfqpoint{1.150000in}{0.150000in}}{\pgfqpoint{5.700000in}{5.700000in}}%
\pgfusepath{clip}%
\pgfsetbuttcap%
\pgfsetroundjoin%
\definecolor{currentfill}{rgb}{0.282327,0.094955,0.417331}%
\pgfsetfillcolor{currentfill}%
\pgfsetfillopacity{0.700000}%
\pgfsetlinewidth{0.000000pt}%
\definecolor{currentstroke}{rgb}{0.000000,0.000000,0.000000}%
\pgfsetstrokecolor{currentstroke}%
\pgfsetdash{}{0pt}%
\pgfpathmoveto{\pgfqpoint{2.547992in}{2.006863in}}%
\pgfpathlineto{\pgfqpoint{2.561646in}{1.997353in}}%
\pgfpathlineto{\pgfqpoint{2.575299in}{1.987955in}}%
\pgfpathlineto{\pgfqpoint{2.588953in}{1.978668in}}%
\pgfpathlineto{\pgfqpoint{2.602607in}{1.969490in}}%
\pgfpathlineto{\pgfqpoint{2.611243in}{1.974332in}}%
\pgfpathlineto{\pgfqpoint{2.619868in}{1.979312in}}%
\pgfpathlineto{\pgfqpoint{2.628481in}{1.984425in}}%
\pgfpathlineto{\pgfqpoint{2.637084in}{1.989668in}}%
\pgfpathlineto{\pgfqpoint{2.623455in}{1.998596in}}%
\pgfpathlineto{\pgfqpoint{2.609827in}{2.007634in}}%
\pgfpathlineto{\pgfqpoint{2.596199in}{2.016782in}}%
\pgfpathlineto{\pgfqpoint{2.582572in}{2.026042in}}%
\pgfpathlineto{\pgfqpoint{2.573945in}{2.021040in}}%
\pgfpathlineto{\pgfqpoint{2.565306in}{2.016174in}}%
\pgfpathlineto{\pgfqpoint{2.556655in}{2.011447in}}%
\pgfpathlineto{\pgfqpoint{2.547992in}{2.006863in}}%
\pgfpathclose%
\pgfusepath{fill}%
\end{pgfscope}%
\begin{pgfscope}%
\pgfpathrectangle{\pgfqpoint{1.150000in}{0.150000in}}{\pgfqpoint{5.700000in}{5.700000in}}%
\pgfusepath{clip}%
\pgfsetbuttcap%
\pgfsetroundjoin%
\definecolor{currentfill}{rgb}{0.277018,0.050344,0.375715}%
\pgfsetfillcolor{currentfill}%
\pgfsetfillopacity{0.700000}%
\pgfsetlinewidth{0.000000pt}%
\definecolor{currentstroke}{rgb}{0.000000,0.000000,0.000000}%
\pgfsetstrokecolor{currentstroke}%
\pgfsetdash{}{0pt}%
\pgfpathmoveto{\pgfqpoint{2.746144in}{1.922096in}}%
\pgfpathlineto{\pgfqpoint{2.759782in}{1.914120in}}%
\pgfpathlineto{\pgfqpoint{2.773421in}{1.906246in}}%
\pgfpathlineto{\pgfqpoint{2.787063in}{1.898474in}}%
\pgfpathlineto{\pgfqpoint{2.800706in}{1.890804in}}%
\pgfpathlineto{\pgfqpoint{2.809227in}{1.896880in}}%
\pgfpathlineto{\pgfqpoint{2.817738in}{1.903064in}}%
\pgfpathlineto{\pgfqpoint{2.826239in}{1.909352in}}%
\pgfpathlineto{\pgfqpoint{2.834731in}{1.915741in}}%
\pgfpathlineto{\pgfqpoint{2.821110in}{1.923185in}}%
\pgfpathlineto{\pgfqpoint{2.807491in}{1.930730in}}%
\pgfpathlineto{\pgfqpoint{2.793874in}{1.938376in}}%
\pgfpathlineto{\pgfqpoint{2.780258in}{1.946125in}}%
\pgfpathlineto{\pgfqpoint{2.771744in}{1.939955in}}%
\pgfpathlineto{\pgfqpoint{2.763221in}{1.933891in}}%
\pgfpathlineto{\pgfqpoint{2.754687in}{1.927937in}}%
\pgfpathlineto{\pgfqpoint{2.746144in}{1.922096in}}%
\pgfpathclose%
\pgfusepath{fill}%
\end{pgfscope}%
\begin{pgfscope}%
\pgfpathrectangle{\pgfqpoint{1.150000in}{0.150000in}}{\pgfqpoint{5.700000in}{5.700000in}}%
\pgfusepath{clip}%
\pgfsetbuttcap%
\pgfsetroundjoin%
\definecolor{currentfill}{rgb}{0.279574,0.170599,0.479997}%
\pgfsetfillcolor{currentfill}%
\pgfsetfillopacity{0.700000}%
\pgfsetlinewidth{0.000000pt}%
\definecolor{currentstroke}{rgb}{0.000000,0.000000,0.000000}%
\pgfsetstrokecolor{currentstroke}%
\pgfsetdash{}{0pt}%
\pgfpathmoveto{\pgfqpoint{2.294212in}{2.162410in}}%
\pgfpathlineto{\pgfqpoint{2.307911in}{2.150721in}}%
\pgfpathlineto{\pgfqpoint{2.321608in}{2.139159in}}%
\pgfpathlineto{\pgfqpoint{2.335303in}{2.127723in}}%
\pgfpathlineto{\pgfqpoint{2.348996in}{2.116411in}}%
\pgfpathlineto{\pgfqpoint{2.357797in}{2.119545in}}%
\pgfpathlineto{\pgfqpoint{2.366585in}{2.122853in}}%
\pgfpathlineto{\pgfqpoint{2.375358in}{2.126331in}}%
\pgfpathlineto{\pgfqpoint{2.384118in}{2.129976in}}%
\pgfpathlineto{\pgfqpoint{2.370455in}{2.141014in}}%
\pgfpathlineto{\pgfqpoint{2.356791in}{2.152176in}}%
\pgfpathlineto{\pgfqpoint{2.343125in}{2.163463in}}%
\pgfpathlineto{\pgfqpoint{2.329457in}{2.174877in}}%
\pgfpathlineto{\pgfqpoint{2.320667in}{2.171498in}}%
\pgfpathlineto{\pgfqpoint{2.311863in}{2.168292in}}%
\pgfpathlineto{\pgfqpoint{2.303045in}{2.165261in}}%
\pgfpathlineto{\pgfqpoint{2.294212in}{2.162410in}}%
\pgfpathclose%
\pgfusepath{fill}%
\end{pgfscope}%
\begin{pgfscope}%
\pgfpathrectangle{\pgfqpoint{1.150000in}{0.150000in}}{\pgfqpoint{5.700000in}{5.700000in}}%
\pgfusepath{clip}%
\pgfsetbuttcap%
\pgfsetroundjoin%
\definecolor{currentfill}{rgb}{0.210503,0.363727,0.552206}%
\pgfsetfillcolor{currentfill}%
\pgfsetfillopacity{0.700000}%
\pgfsetlinewidth{0.000000pt}%
\definecolor{currentstroke}{rgb}{0.000000,0.000000,0.000000}%
\pgfsetstrokecolor{currentstroke}%
\pgfsetdash{}{0pt}%
\pgfpathmoveto{\pgfqpoint{5.744292in}{2.577873in}}%
\pgfpathlineto{\pgfqpoint{5.758711in}{2.580218in}}%
\pgfpathlineto{\pgfqpoint{5.773142in}{2.582631in}}%
\pgfpathlineto{\pgfqpoint{5.787585in}{2.585111in}}%
\pgfpathlineto{\pgfqpoint{5.802040in}{2.587659in}}%
\pgfpathlineto{\pgfqpoint{5.809265in}{2.591730in}}%
\pgfpathlineto{\pgfqpoint{5.816487in}{2.595876in}}%
\pgfpathlineto{\pgfqpoint{5.823704in}{2.600103in}}%
\pgfpathlineto{\pgfqpoint{5.830918in}{2.604418in}}%
\pgfpathlineto{\pgfqpoint{5.816490in}{2.602230in}}%
\pgfpathlineto{\pgfqpoint{5.802074in}{2.600108in}}%
\pgfpathlineto{\pgfqpoint{5.787670in}{2.598054in}}%
\pgfpathlineto{\pgfqpoint{5.773278in}{2.596067in}}%
\pgfpathlineto{\pgfqpoint{5.766037in}{2.591387in}}%
\pgfpathlineto{\pgfqpoint{5.758792in}{2.586798in}}%
\pgfpathlineto{\pgfqpoint{5.751544in}{2.582296in}}%
\pgfpathlineto{\pgfqpoint{5.744292in}{2.577873in}}%
\pgfpathclose%
\pgfusepath{fill}%
\end{pgfscope}%
\begin{pgfscope}%
\pgfpathrectangle{\pgfqpoint{1.150000in}{0.150000in}}{\pgfqpoint{5.700000in}{5.700000in}}%
\pgfusepath{clip}%
\pgfsetbuttcap%
\pgfsetroundjoin%
\definecolor{currentfill}{rgb}{0.280868,0.160771,0.472899}%
\pgfsetfillcolor{currentfill}%
\pgfsetfillopacity{0.700000}%
\pgfsetlinewidth{0.000000pt}%
\definecolor{currentstroke}{rgb}{0.000000,0.000000,0.000000}%
\pgfsetstrokecolor{currentstroke}%
\pgfsetdash{}{0pt}%
\pgfpathmoveto{\pgfqpoint{4.356088in}{2.112580in}}%
\pgfpathlineto{\pgfqpoint{4.370015in}{2.112976in}}%
\pgfpathlineto{\pgfqpoint{4.383951in}{2.113444in}}%
\pgfpathlineto{\pgfqpoint{4.397896in}{2.113986in}}%
\pgfpathlineto{\pgfqpoint{4.411851in}{2.114601in}}%
\pgfpathlineto{\pgfqpoint{4.419722in}{2.122890in}}%
\pgfpathlineto{\pgfqpoint{4.427587in}{2.131127in}}%
\pgfpathlineto{\pgfqpoint{4.435446in}{2.139313in}}%
\pgfpathlineto{\pgfqpoint{4.443298in}{2.147450in}}%
\pgfpathlineto{\pgfqpoint{4.429355in}{2.146898in}}%
\pgfpathlineto{\pgfqpoint{4.415421in}{2.146420in}}%
\pgfpathlineto{\pgfqpoint{4.401496in}{2.146015in}}%
\pgfpathlineto{\pgfqpoint{4.387580in}{2.145683in}}%
\pgfpathlineto{\pgfqpoint{4.379716in}{2.137475in}}%
\pgfpathlineto{\pgfqpoint{4.371846in}{2.129223in}}%
\pgfpathlineto{\pgfqpoint{4.363970in}{2.120925in}}%
\pgfpathlineto{\pgfqpoint{4.356088in}{2.112580in}}%
\pgfpathclose%
\pgfusepath{fill}%
\end{pgfscope}%
\begin{pgfscope}%
\pgfpathrectangle{\pgfqpoint{1.150000in}{0.150000in}}{\pgfqpoint{5.700000in}{5.700000in}}%
\pgfusepath{clip}%
\pgfsetbuttcap%
\pgfsetroundjoin%
\definecolor{currentfill}{rgb}{0.269944,0.014625,0.341379}%
\pgfsetfillcolor{currentfill}%
\pgfsetfillopacity{0.700000}%
\pgfsetlinewidth{0.000000pt}%
\definecolor{currentstroke}{rgb}{0.000000,0.000000,0.000000}%
\pgfsetstrokecolor{currentstroke}%
\pgfsetdash{}{0pt}%
\pgfpathmoveto{\pgfqpoint{3.317000in}{1.844987in}}%
\pgfpathlineto{\pgfqpoint{3.330669in}{1.840784in}}%
\pgfpathlineto{\pgfqpoint{3.344343in}{1.836666in}}%
\pgfpathlineto{\pgfqpoint{3.358022in}{1.832633in}}%
\pgfpathlineto{\pgfqpoint{3.371706in}{1.828686in}}%
\pgfpathlineto{\pgfqpoint{3.379957in}{1.837349in}}%
\pgfpathlineto{\pgfqpoint{3.388202in}{1.846037in}}%
\pgfpathlineto{\pgfqpoint{3.396440in}{1.854748in}}%
\pgfpathlineto{\pgfqpoint{3.404672in}{1.863479in}}%
\pgfpathlineto{\pgfqpoint{3.391002in}{1.867283in}}%
\pgfpathlineto{\pgfqpoint{3.377337in}{1.871173in}}%
\pgfpathlineto{\pgfqpoint{3.363677in}{1.875148in}}%
\pgfpathlineto{\pgfqpoint{3.350022in}{1.879208in}}%
\pgfpathlineto{\pgfqpoint{3.341776in}{1.870612in}}%
\pgfpathlineto{\pgfqpoint{3.333524in}{1.862042in}}%
\pgfpathlineto{\pgfqpoint{3.325265in}{1.853500in}}%
\pgfpathlineto{\pgfqpoint{3.317000in}{1.844987in}}%
\pgfpathclose%
\pgfusepath{fill}%
\end{pgfscope}%
\begin{pgfscope}%
\pgfpathrectangle{\pgfqpoint{1.150000in}{0.150000in}}{\pgfqpoint{5.700000in}{5.700000in}}%
\pgfusepath{clip}%
\pgfsetbuttcap%
\pgfsetroundjoin%
\definecolor{currentfill}{rgb}{0.233603,0.313828,0.543914}%
\pgfsetfillcolor{currentfill}%
\pgfsetfillopacity{0.700000}%
\pgfsetlinewidth{0.000000pt}%
\definecolor{currentstroke}{rgb}{0.000000,0.000000,0.000000}%
\pgfsetstrokecolor{currentstroke}%
\pgfsetdash{}{0pt}%
\pgfpathmoveto{\pgfqpoint{5.339790in}{2.455933in}}%
\pgfpathlineto{\pgfqpoint{5.354067in}{2.458144in}}%
\pgfpathlineto{\pgfqpoint{5.368355in}{2.460423in}}%
\pgfpathlineto{\pgfqpoint{5.382655in}{2.462772in}}%
\pgfpathlineto{\pgfqpoint{5.396966in}{2.465189in}}%
\pgfpathlineto{\pgfqpoint{5.404394in}{2.470195in}}%
\pgfpathlineto{\pgfqpoint{5.411817in}{2.475211in}}%
\pgfpathlineto{\pgfqpoint{5.419233in}{2.480245in}}%
\pgfpathlineto{\pgfqpoint{5.426644in}{2.485299in}}%
\pgfpathlineto{\pgfqpoint{5.412354in}{2.483157in}}%
\pgfpathlineto{\pgfqpoint{5.398075in}{2.481084in}}%
\pgfpathlineto{\pgfqpoint{5.383808in}{2.479078in}}%
\pgfpathlineto{\pgfqpoint{5.369552in}{2.477142in}}%
\pgfpathlineto{\pgfqpoint{5.362120in}{2.471805in}}%
\pgfpathlineto{\pgfqpoint{5.354682in}{2.466495in}}%
\pgfpathlineto{\pgfqpoint{5.347239in}{2.461206in}}%
\pgfpathlineto{\pgfqpoint{5.339790in}{2.455933in}}%
\pgfpathclose%
\pgfusepath{fill}%
\end{pgfscope}%
\begin{pgfscope}%
\pgfpathrectangle{\pgfqpoint{1.150000in}{0.150000in}}{\pgfqpoint{5.700000in}{5.700000in}}%
\pgfusepath{clip}%
\pgfsetbuttcap%
\pgfsetroundjoin%
\definecolor{currentfill}{rgb}{0.273809,0.031497,0.358853}%
\pgfsetfillcolor{currentfill}%
\pgfsetfillopacity{0.700000}%
\pgfsetlinewidth{0.000000pt}%
\definecolor{currentstroke}{rgb}{0.000000,0.000000,0.000000}%
\pgfsetstrokecolor{currentstroke}%
\pgfsetdash{}{0pt}%
\pgfpathmoveto{\pgfqpoint{3.546992in}{1.872147in}}%
\pgfpathlineto{\pgfqpoint{3.560700in}{1.869215in}}%
\pgfpathlineto{\pgfqpoint{3.574414in}{1.866365in}}%
\pgfpathlineto{\pgfqpoint{3.588134in}{1.863596in}}%
\pgfpathlineto{\pgfqpoint{3.601861in}{1.860909in}}%
\pgfpathlineto{\pgfqpoint{3.610024in}{1.870050in}}%
\pgfpathlineto{\pgfqpoint{3.618181in}{1.879188in}}%
\pgfpathlineto{\pgfqpoint{3.626332in}{1.888320in}}%
\pgfpathlineto{\pgfqpoint{3.634478in}{1.897447in}}%
\pgfpathlineto{\pgfqpoint{3.620764in}{1.900033in}}%
\pgfpathlineto{\pgfqpoint{3.607056in}{1.902699in}}%
\pgfpathlineto{\pgfqpoint{3.593354in}{1.905447in}}%
\pgfpathlineto{\pgfqpoint{3.579658in}{1.908276in}}%
\pgfpathlineto{\pgfqpoint{3.571500in}{1.899244in}}%
\pgfpathlineto{\pgfqpoint{3.563337in}{1.890211in}}%
\pgfpathlineto{\pgfqpoint{3.555167in}{1.881178in}}%
\pgfpathlineto{\pgfqpoint{3.546992in}{1.872147in}}%
\pgfpathclose%
\pgfusepath{fill}%
\end{pgfscope}%
\begin{pgfscope}%
\pgfpathrectangle{\pgfqpoint{1.150000in}{0.150000in}}{\pgfqpoint{5.700000in}{5.700000in}}%
\pgfusepath{clip}%
\pgfsetbuttcap%
\pgfsetroundjoin%
\definecolor{currentfill}{rgb}{0.280894,0.078907,0.402329}%
\pgfsetfillcolor{currentfill}%
\pgfsetfillopacity{0.700000}%
\pgfsetlinewidth{0.000000pt}%
\definecolor{currentstroke}{rgb}{0.000000,0.000000,0.000000}%
\pgfsetstrokecolor{currentstroke}%
\pgfsetdash{}{0pt}%
\pgfpathmoveto{\pgfqpoint{3.864258in}{1.947063in}}%
\pgfpathlineto{\pgfqpoint{3.878039in}{1.945651in}}%
\pgfpathlineto{\pgfqpoint{3.891828in}{1.944318in}}%
\pgfpathlineto{\pgfqpoint{3.905624in}{1.943061in}}%
\pgfpathlineto{\pgfqpoint{3.919428in}{1.941882in}}%
\pgfpathlineto{\pgfqpoint{3.927478in}{1.951112in}}%
\pgfpathlineto{\pgfqpoint{3.935523in}{1.960308in}}%
\pgfpathlineto{\pgfqpoint{3.943563in}{1.969471in}}%
\pgfpathlineto{\pgfqpoint{3.951596in}{1.978600in}}%
\pgfpathlineto{\pgfqpoint{3.937803in}{1.979739in}}%
\pgfpathlineto{\pgfqpoint{3.924017in}{1.980955in}}%
\pgfpathlineto{\pgfqpoint{3.910239in}{1.982248in}}%
\pgfpathlineto{\pgfqpoint{3.896469in}{1.983619in}}%
\pgfpathlineto{\pgfqpoint{3.888425in}{1.974522in}}%
\pgfpathlineto{\pgfqpoint{3.880375in}{1.965397in}}%
\pgfpathlineto{\pgfqpoint{3.872319in}{1.956244in}}%
\pgfpathlineto{\pgfqpoint{3.864258in}{1.947063in}}%
\pgfpathclose%
\pgfusepath{fill}%
\end{pgfscope}%
\begin{pgfscope}%
\pgfpathrectangle{\pgfqpoint{1.150000in}{0.150000in}}{\pgfqpoint{5.700000in}{5.700000in}}%
\pgfusepath{clip}%
\pgfsetbuttcap%
\pgfsetroundjoin%
\definecolor{currentfill}{rgb}{0.260571,0.246922,0.522828}%
\pgfsetfillcolor{currentfill}%
\pgfsetfillopacity{0.700000}%
\pgfsetlinewidth{0.000000pt}%
\definecolor{currentstroke}{rgb}{0.000000,0.000000,0.000000}%
\pgfsetstrokecolor{currentstroke}%
\pgfsetdash{}{0pt}%
\pgfpathmoveto{\pgfqpoint{4.847999in}{2.289449in}}%
\pgfpathlineto{\pgfqpoint{4.862099in}{2.291045in}}%
\pgfpathlineto{\pgfqpoint{4.876209in}{2.292711in}}%
\pgfpathlineto{\pgfqpoint{4.890330in}{2.294449in}}%
\pgfpathlineto{\pgfqpoint{4.904461in}{2.296257in}}%
\pgfpathlineto{\pgfqpoint{4.912127in}{2.302912in}}%
\pgfpathlineto{\pgfqpoint{4.919785in}{2.309529in}}%
\pgfpathlineto{\pgfqpoint{4.927438in}{2.316112in}}%
\pgfpathlineto{\pgfqpoint{4.935083in}{2.322664in}}%
\pgfpathlineto{\pgfqpoint{4.920967in}{2.321025in}}%
\pgfpathlineto{\pgfqpoint{4.906861in}{2.319457in}}%
\pgfpathlineto{\pgfqpoint{4.892766in}{2.317959in}}%
\pgfpathlineto{\pgfqpoint{4.878681in}{2.316532in}}%
\pgfpathlineto{\pgfqpoint{4.871020in}{2.309804in}}%
\pgfpathlineto{\pgfqpoint{4.863353in}{2.303050in}}%
\pgfpathlineto{\pgfqpoint{4.855679in}{2.296266in}}%
\pgfpathlineto{\pgfqpoint{4.847999in}{2.289449in}}%
\pgfpathclose%
\pgfusepath{fill}%
\end{pgfscope}%
\begin{pgfscope}%
\pgfpathrectangle{\pgfqpoint{1.150000in}{0.150000in}}{\pgfqpoint{5.700000in}{5.700000in}}%
\pgfusepath{clip}%
\pgfsetbuttcap%
\pgfsetroundjoin%
\definecolor{currentfill}{rgb}{0.282290,0.145912,0.461510}%
\pgfsetfillcolor{currentfill}%
\pgfsetfillopacity{0.700000}%
\pgfsetlinewidth{0.000000pt}%
\definecolor{currentstroke}{rgb}{0.000000,0.000000,0.000000}%
\pgfsetstrokecolor{currentstroke}%
\pgfsetdash{}{0pt}%
\pgfpathmoveto{\pgfqpoint{4.268837in}{2.077687in}}%
\pgfpathlineto{\pgfqpoint{4.282739in}{2.077830in}}%
\pgfpathlineto{\pgfqpoint{4.296650in}{2.078046in}}%
\pgfpathlineto{\pgfqpoint{4.310570in}{2.078337in}}%
\pgfpathlineto{\pgfqpoint{4.324499in}{2.078701in}}%
\pgfpathlineto{\pgfqpoint{4.332405in}{2.087248in}}%
\pgfpathlineto{\pgfqpoint{4.340305in}{2.095743in}}%
\pgfpathlineto{\pgfqpoint{4.348199in}{2.104187in}}%
\pgfpathlineto{\pgfqpoint{4.356088in}{2.112580in}}%
\pgfpathlineto{\pgfqpoint{4.342170in}{2.112259in}}%
\pgfpathlineto{\pgfqpoint{4.328261in}{2.112011in}}%
\pgfpathlineto{\pgfqpoint{4.314361in}{2.111837in}}%
\pgfpathlineto{\pgfqpoint{4.300469in}{2.111737in}}%
\pgfpathlineto{\pgfqpoint{4.292570in}{2.103293in}}%
\pgfpathlineto{\pgfqpoint{4.284665in}{2.094804in}}%
\pgfpathlineto{\pgfqpoint{4.276754in}{2.086269in}}%
\pgfpathlineto{\pgfqpoint{4.268837in}{2.077687in}}%
\pgfpathclose%
\pgfusepath{fill}%
\end{pgfscope}%
\begin{pgfscope}%
\pgfpathrectangle{\pgfqpoint{1.150000in}{0.150000in}}{\pgfqpoint{5.700000in}{5.700000in}}%
\pgfusepath{clip}%
\pgfsetbuttcap%
\pgfsetroundjoin%
\definecolor{currentfill}{rgb}{0.281887,0.150881,0.465405}%
\pgfsetfillcolor{currentfill}%
\pgfsetfillopacity{0.700000}%
\pgfsetlinewidth{0.000000pt}%
\definecolor{currentstroke}{rgb}{0.000000,0.000000,0.000000}%
\pgfsetstrokecolor{currentstroke}%
\pgfsetdash{}{0pt}%
\pgfpathmoveto{\pgfqpoint{2.348996in}{2.116411in}}%
\pgfpathlineto{\pgfqpoint{2.362687in}{2.105223in}}%
\pgfpathlineto{\pgfqpoint{2.376377in}{2.094158in}}%
\pgfpathlineto{\pgfqpoint{2.390066in}{2.083214in}}%
\pgfpathlineto{\pgfqpoint{2.403753in}{2.072390in}}%
\pgfpathlineto{\pgfqpoint{2.412524in}{2.075805in}}%
\pgfpathlineto{\pgfqpoint{2.421281in}{2.079389in}}%
\pgfpathlineto{\pgfqpoint{2.430025in}{2.083138in}}%
\pgfpathlineto{\pgfqpoint{2.438755in}{2.087048in}}%
\pgfpathlineto{\pgfqpoint{2.425098in}{2.097599in}}%
\pgfpathlineto{\pgfqpoint{2.411439in}{2.108270in}}%
\pgfpathlineto{\pgfqpoint{2.397779in}{2.119062in}}%
\pgfpathlineto{\pgfqpoint{2.384118in}{2.129976in}}%
\pgfpathlineto{\pgfqpoint{2.375358in}{2.126331in}}%
\pgfpathlineto{\pgfqpoint{2.366585in}{2.122853in}}%
\pgfpathlineto{\pgfqpoint{2.357797in}{2.119545in}}%
\pgfpathlineto{\pgfqpoint{2.348996in}{2.116411in}}%
\pgfpathclose%
\pgfusepath{fill}%
\end{pgfscope}%
\begin{pgfscope}%
\pgfpathrectangle{\pgfqpoint{1.150000in}{0.150000in}}{\pgfqpoint{5.700000in}{5.700000in}}%
\pgfusepath{clip}%
\pgfsetbuttcap%
\pgfsetroundjoin%
\definecolor{currentfill}{rgb}{0.237441,0.305202,0.541921}%
\pgfsetfillcolor{currentfill}%
\pgfsetfillopacity{0.700000}%
\pgfsetlinewidth{0.000000pt}%
\definecolor{currentstroke}{rgb}{0.000000,0.000000,0.000000}%
\pgfsetstrokecolor{currentstroke}%
\pgfsetdash{}{0pt}%
\pgfpathmoveto{\pgfqpoint{5.252861in}{2.425739in}}%
\pgfpathlineto{\pgfqpoint{5.267111in}{2.427929in}}%
\pgfpathlineto{\pgfqpoint{5.281374in}{2.430187in}}%
\pgfpathlineto{\pgfqpoint{5.295648in}{2.432515in}}%
\pgfpathlineto{\pgfqpoint{5.309933in}{2.434911in}}%
\pgfpathlineto{\pgfqpoint{5.317406in}{2.440165in}}%
\pgfpathlineto{\pgfqpoint{5.324874in}{2.445417in}}%
\pgfpathlineto{\pgfqpoint{5.332335in}{2.450671in}}%
\pgfpathlineto{\pgfqpoint{5.339790in}{2.455933in}}%
\pgfpathlineto{\pgfqpoint{5.325525in}{2.453790in}}%
\pgfpathlineto{\pgfqpoint{5.311271in}{2.451717in}}%
\pgfpathlineto{\pgfqpoint{5.297028in}{2.449712in}}%
\pgfpathlineto{\pgfqpoint{5.282797in}{2.447776in}}%
\pgfpathlineto{\pgfqpoint{5.275322in}{2.442254in}}%
\pgfpathlineto{\pgfqpoint{5.267841in}{2.436743in}}%
\pgfpathlineto{\pgfqpoint{5.260354in}{2.431240in}}%
\pgfpathlineto{\pgfqpoint{5.252861in}{2.425739in}}%
\pgfpathclose%
\pgfusepath{fill}%
\end{pgfscope}%
\begin{pgfscope}%
\pgfpathrectangle{\pgfqpoint{1.150000in}{0.150000in}}{\pgfqpoint{5.700000in}{5.700000in}}%
\pgfusepath{clip}%
\pgfsetbuttcap%
\pgfsetroundjoin%
\definecolor{currentfill}{rgb}{0.214298,0.355619,0.551184}%
\pgfsetfillcolor{currentfill}%
\pgfsetfillopacity{0.700000}%
\pgfsetlinewidth{0.000000pt}%
\definecolor{currentstroke}{rgb}{0.000000,0.000000,0.000000}%
\pgfsetstrokecolor{currentstroke}%
\pgfsetdash{}{0pt}%
\pgfpathmoveto{\pgfqpoint{5.657584in}{2.550796in}}%
\pgfpathlineto{\pgfqpoint{5.671981in}{2.553210in}}%
\pgfpathlineto{\pgfqpoint{5.686389in}{2.555691in}}%
\pgfpathlineto{\pgfqpoint{5.700810in}{2.558240in}}%
\pgfpathlineto{\pgfqpoint{5.715243in}{2.560857in}}%
\pgfpathlineto{\pgfqpoint{5.722512in}{2.565021in}}%
\pgfpathlineto{\pgfqpoint{5.729776in}{2.569242in}}%
\pgfpathlineto{\pgfqpoint{5.737036in}{2.573523in}}%
\pgfpathlineto{\pgfqpoint{5.744292in}{2.577873in}}%
\pgfpathlineto{\pgfqpoint{5.729885in}{2.575595in}}%
\pgfpathlineto{\pgfqpoint{5.715491in}{2.573384in}}%
\pgfpathlineto{\pgfqpoint{5.701108in}{2.571241in}}%
\pgfpathlineto{\pgfqpoint{5.686737in}{2.569165in}}%
\pgfpathlineto{\pgfqpoint{5.679455in}{2.564471in}}%
\pgfpathlineto{\pgfqpoint{5.672169in}{2.559848in}}%
\pgfpathlineto{\pgfqpoint{5.664879in}{2.555292in}}%
\pgfpathlineto{\pgfqpoint{5.657584in}{2.550796in}}%
\pgfpathclose%
\pgfusepath{fill}%
\end{pgfscope}%
\begin{pgfscope}%
\pgfpathrectangle{\pgfqpoint{1.150000in}{0.150000in}}{\pgfqpoint{5.700000in}{5.700000in}}%
\pgfusepath{clip}%
\pgfsetbuttcap%
\pgfsetroundjoin%
\definecolor{currentfill}{rgb}{0.279566,0.067836,0.391917}%
\pgfsetfillcolor{currentfill}%
\pgfsetfillopacity{0.700000}%
\pgfsetlinewidth{0.000000pt}%
\definecolor{currentstroke}{rgb}{0.000000,0.000000,0.000000}%
\pgfsetstrokecolor{currentstroke}%
\pgfsetdash{}{0pt}%
\pgfpathmoveto{\pgfqpoint{3.776862in}{1.916736in}}%
\pgfpathlineto{\pgfqpoint{3.790626in}{1.914950in}}%
\pgfpathlineto{\pgfqpoint{3.804396in}{1.913244in}}%
\pgfpathlineto{\pgfqpoint{3.818173in}{1.911615in}}%
\pgfpathlineto{\pgfqpoint{3.831958in}{1.910065in}}%
\pgfpathlineto{\pgfqpoint{3.840041in}{1.919354in}}%
\pgfpathlineto{\pgfqpoint{3.848119in}{1.928618in}}%
\pgfpathlineto{\pgfqpoint{3.856192in}{1.937854in}}%
\pgfpathlineto{\pgfqpoint{3.864258in}{1.947063in}}%
\pgfpathlineto{\pgfqpoint{3.850484in}{1.948552in}}%
\pgfpathlineto{\pgfqpoint{3.836718in}{1.950120in}}%
\pgfpathlineto{\pgfqpoint{3.822958in}{1.951765in}}%
\pgfpathlineto{\pgfqpoint{3.809206in}{1.953490in}}%
\pgfpathlineto{\pgfqpoint{3.801129in}{1.944334in}}%
\pgfpathlineto{\pgfqpoint{3.793046in}{1.935156in}}%
\pgfpathlineto{\pgfqpoint{3.784957in}{1.925957in}}%
\pgfpathlineto{\pgfqpoint{3.776862in}{1.916736in}}%
\pgfpathclose%
\pgfusepath{fill}%
\end{pgfscope}%
\begin{pgfscope}%
\pgfpathrectangle{\pgfqpoint{1.150000in}{0.150000in}}{\pgfqpoint{5.700000in}{5.700000in}}%
\pgfusepath{clip}%
\pgfsetbuttcap%
\pgfsetroundjoin%
\definecolor{currentfill}{rgb}{0.265145,0.232956,0.516599}%
\pgfsetfillcolor{currentfill}%
\pgfsetfillopacity{0.700000}%
\pgfsetlinewidth{0.000000pt}%
\definecolor{currentstroke}{rgb}{0.000000,0.000000,0.000000}%
\pgfsetstrokecolor{currentstroke}%
\pgfsetdash{}{0pt}%
\pgfpathmoveto{\pgfqpoint{4.760862in}{2.255529in}}%
\pgfpathlineto{\pgfqpoint{4.774934in}{2.256989in}}%
\pgfpathlineto{\pgfqpoint{4.789017in}{2.258520in}}%
\pgfpathlineto{\pgfqpoint{4.803110in}{2.260123in}}%
\pgfpathlineto{\pgfqpoint{4.817213in}{2.261796in}}%
\pgfpathlineto{\pgfqpoint{4.824920in}{2.268773in}}%
\pgfpathlineto{\pgfqpoint{4.832619in}{2.275706in}}%
\pgfpathlineto{\pgfqpoint{4.840313in}{2.282597in}}%
\pgfpathlineto{\pgfqpoint{4.847999in}{2.289449in}}%
\pgfpathlineto{\pgfqpoint{4.833910in}{2.287924in}}%
\pgfpathlineto{\pgfqpoint{4.819831in}{2.286470in}}%
\pgfpathlineto{\pgfqpoint{4.805762in}{2.285087in}}%
\pgfpathlineto{\pgfqpoint{4.791704in}{2.283775in}}%
\pgfpathlineto{\pgfqpoint{4.784003in}{2.276767in}}%
\pgfpathlineto{\pgfqpoint{4.776296in}{2.269725in}}%
\pgfpathlineto{\pgfqpoint{4.768582in}{2.262647in}}%
\pgfpathlineto{\pgfqpoint{4.760862in}{2.255529in}}%
\pgfpathclose%
\pgfusepath{fill}%
\end{pgfscope}%
\begin{pgfscope}%
\pgfpathrectangle{\pgfqpoint{1.150000in}{0.150000in}}{\pgfqpoint{5.700000in}{5.700000in}}%
\pgfusepath{clip}%
\pgfsetbuttcap%
\pgfsetroundjoin%
\definecolor{currentfill}{rgb}{0.272594,0.025563,0.353093}%
\pgfsetfillcolor{currentfill}%
\pgfsetfillopacity{0.700000}%
\pgfsetlinewidth{0.000000pt}%
\definecolor{currentstroke}{rgb}{0.000000,0.000000,0.000000}%
\pgfsetstrokecolor{currentstroke}%
\pgfsetdash{}{0pt}%
\pgfpathmoveto{\pgfqpoint{2.943780in}{1.859747in}}%
\pgfpathlineto{\pgfqpoint{2.957423in}{1.853184in}}%
\pgfpathlineto{\pgfqpoint{2.971068in}{1.846716in}}%
\pgfpathlineto{\pgfqpoint{2.984717in}{1.840342in}}%
\pgfpathlineto{\pgfqpoint{2.998368in}{1.834062in}}%
\pgfpathlineto{\pgfqpoint{3.006790in}{1.841189in}}%
\pgfpathlineto{\pgfqpoint{3.015204in}{1.848397in}}%
\pgfpathlineto{\pgfqpoint{3.023609in}{1.855680in}}%
\pgfpathlineto{\pgfqpoint{3.032006in}{1.863038in}}%
\pgfpathlineto{\pgfqpoint{3.018373in}{1.869113in}}%
\pgfpathlineto{\pgfqpoint{3.004744in}{1.875281in}}%
\pgfpathlineto{\pgfqpoint{2.991118in}{1.881544in}}%
\pgfpathlineto{\pgfqpoint{2.977495in}{1.887902in}}%
\pgfpathlineto{\pgfqpoint{2.969079in}{1.880742in}}%
\pgfpathlineto{\pgfqpoint{2.960654in}{1.873661in}}%
\pgfpathlineto{\pgfqpoint{2.952222in}{1.866662in}}%
\pgfpathlineto{\pgfqpoint{2.943780in}{1.859747in}}%
\pgfpathclose%
\pgfusepath{fill}%
\end{pgfscope}%
\begin{pgfscope}%
\pgfpathrectangle{\pgfqpoint{1.150000in}{0.150000in}}{\pgfqpoint{5.700000in}{5.700000in}}%
\pgfusepath{clip}%
\pgfsetbuttcap%
\pgfsetroundjoin%
\definecolor{currentfill}{rgb}{0.269944,0.014625,0.341379}%
\pgfsetfillcolor{currentfill}%
\pgfsetfillopacity{0.700000}%
\pgfsetlinewidth{0.000000pt}%
\definecolor{currentstroke}{rgb}{0.000000,0.000000,0.000000}%
\pgfsetstrokecolor{currentstroke}%
\pgfsetdash{}{0pt}%
\pgfpathmoveto{\pgfqpoint{3.086570in}{1.839667in}}%
\pgfpathlineto{\pgfqpoint{3.100220in}{1.834054in}}%
\pgfpathlineto{\pgfqpoint{3.113873in}{1.828532in}}%
\pgfpathlineto{\pgfqpoint{3.127531in}{1.823100in}}%
\pgfpathlineto{\pgfqpoint{3.141192in}{1.817758in}}%
\pgfpathlineto{\pgfqpoint{3.149545in}{1.825572in}}%
\pgfpathlineto{\pgfqpoint{3.157891in}{1.833443in}}%
\pgfpathlineto{\pgfqpoint{3.166230in}{1.841369in}}%
\pgfpathlineto{\pgfqpoint{3.174561in}{1.849349in}}%
\pgfpathlineto{\pgfqpoint{3.160917in}{1.854507in}}%
\pgfpathlineto{\pgfqpoint{3.147277in}{1.859754in}}%
\pgfpathlineto{\pgfqpoint{3.133640in}{1.865092in}}%
\pgfpathlineto{\pgfqpoint{3.120008in}{1.870520in}}%
\pgfpathlineto{\pgfqpoint{3.111660in}{1.862717in}}%
\pgfpathlineto{\pgfqpoint{3.103304in}{1.854972in}}%
\pgfpathlineto{\pgfqpoint{3.094941in}{1.847288in}}%
\pgfpathlineto{\pgfqpoint{3.086570in}{1.839667in}}%
\pgfpathclose%
\pgfusepath{fill}%
\end{pgfscope}%
\begin{pgfscope}%
\pgfpathrectangle{\pgfqpoint{1.150000in}{0.150000in}}{\pgfqpoint{5.700000in}{5.700000in}}%
\pgfusepath{clip}%
\pgfsetbuttcap%
\pgfsetroundjoin%
\definecolor{currentfill}{rgb}{0.281446,0.084320,0.407414}%
\pgfsetfillcolor{currentfill}%
\pgfsetfillopacity{0.700000}%
\pgfsetlinewidth{0.000000pt}%
\definecolor{currentstroke}{rgb}{0.000000,0.000000,0.000000}%
\pgfsetstrokecolor{currentstroke}%
\pgfsetdash{}{0pt}%
\pgfpathmoveto{\pgfqpoint{2.602607in}{1.969490in}}%
\pgfpathlineto{\pgfqpoint{2.616262in}{1.960421in}}%
\pgfpathlineto{\pgfqpoint{2.629917in}{1.951461in}}%
\pgfpathlineto{\pgfqpoint{2.643573in}{1.942608in}}%
\pgfpathlineto{\pgfqpoint{2.657229in}{1.933861in}}%
\pgfpathlineto{\pgfqpoint{2.665840in}{1.938960in}}%
\pgfpathlineto{\pgfqpoint{2.674439in}{1.944191in}}%
\pgfpathlineto{\pgfqpoint{2.683028in}{1.949551in}}%
\pgfpathlineto{\pgfqpoint{2.691605in}{1.955035in}}%
\pgfpathlineto{\pgfqpoint{2.677974in}{1.963533in}}%
\pgfpathlineto{\pgfqpoint{2.664343in}{1.972137in}}%
\pgfpathlineto{\pgfqpoint{2.650713in}{1.980849in}}%
\pgfpathlineto{\pgfqpoint{2.637084in}{1.989668in}}%
\pgfpathlineto{\pgfqpoint{2.628481in}{1.984425in}}%
\pgfpathlineto{\pgfqpoint{2.619868in}{1.979312in}}%
\pgfpathlineto{\pgfqpoint{2.611243in}{1.974332in}}%
\pgfpathlineto{\pgfqpoint{2.602607in}{1.969490in}}%
\pgfpathclose%
\pgfusepath{fill}%
\end{pgfscope}%
\begin{pgfscope}%
\pgfpathrectangle{\pgfqpoint{1.150000in}{0.150000in}}{\pgfqpoint{5.700000in}{5.700000in}}%
\pgfusepath{clip}%
\pgfsetbuttcap%
\pgfsetroundjoin%
\definecolor{currentfill}{rgb}{0.272594,0.025563,0.353093}%
\pgfsetfillcolor{currentfill}%
\pgfsetfillopacity{0.700000}%
\pgfsetlinewidth{0.000000pt}%
\definecolor{currentstroke}{rgb}{0.000000,0.000000,0.000000}%
\pgfsetstrokecolor{currentstroke}%
\pgfsetdash{}{0pt}%
\pgfpathmoveto{\pgfqpoint{3.459407in}{1.849105in}}%
\pgfpathlineto{\pgfqpoint{3.473104in}{1.845721in}}%
\pgfpathlineto{\pgfqpoint{3.486808in}{1.842419in}}%
\pgfpathlineto{\pgfqpoint{3.500516in}{1.839201in}}%
\pgfpathlineto{\pgfqpoint{3.514231in}{1.836064in}}%
\pgfpathlineto{\pgfqpoint{3.522430in}{1.845076in}}%
\pgfpathlineto{\pgfqpoint{3.530624in}{1.854094in}}%
\pgfpathlineto{\pgfqpoint{3.538811in}{1.863118in}}%
\pgfpathlineto{\pgfqpoint{3.546992in}{1.872147in}}%
\pgfpathlineto{\pgfqpoint{3.533290in}{1.875160in}}%
\pgfpathlineto{\pgfqpoint{3.519594in}{1.878256in}}%
\pgfpathlineto{\pgfqpoint{3.505904in}{1.881435in}}%
\pgfpathlineto{\pgfqpoint{3.492219in}{1.884697in}}%
\pgfpathlineto{\pgfqpoint{3.484025in}{1.875784in}}%
\pgfpathlineto{\pgfqpoint{3.475825in}{1.866879in}}%
\pgfpathlineto{\pgfqpoint{3.467619in}{1.857986in}}%
\pgfpathlineto{\pgfqpoint{3.459407in}{1.849105in}}%
\pgfpathclose%
\pgfusepath{fill}%
\end{pgfscope}%
\begin{pgfscope}%
\pgfpathrectangle{\pgfqpoint{1.150000in}{0.150000in}}{\pgfqpoint{5.700000in}{5.700000in}}%
\pgfusepath{clip}%
\pgfsetbuttcap%
\pgfsetroundjoin%
\definecolor{currentfill}{rgb}{0.283072,0.130895,0.449241}%
\pgfsetfillcolor{currentfill}%
\pgfsetfillopacity{0.700000}%
\pgfsetlinewidth{0.000000pt}%
\definecolor{currentstroke}{rgb}{0.000000,0.000000,0.000000}%
\pgfsetstrokecolor{currentstroke}%
\pgfsetdash{}{0pt}%
\pgfpathmoveto{\pgfqpoint{4.181548in}{2.042954in}}%
\pgfpathlineto{\pgfqpoint{4.195425in}{2.042821in}}%
\pgfpathlineto{\pgfqpoint{4.209312in}{2.042762in}}%
\pgfpathlineto{\pgfqpoint{4.223207in}{2.042778in}}%
\pgfpathlineto{\pgfqpoint{4.237110in}{2.042868in}}%
\pgfpathlineto{\pgfqpoint{4.245051in}{2.051648in}}%
\pgfpathlineto{\pgfqpoint{4.252986in}{2.060378in}}%
\pgfpathlineto{\pgfqpoint{4.260914in}{2.069057in}}%
\pgfpathlineto{\pgfqpoint{4.268837in}{2.077687in}}%
\pgfpathlineto{\pgfqpoint{4.254944in}{2.077619in}}%
\pgfpathlineto{\pgfqpoint{4.241060in}{2.077625in}}%
\pgfpathlineto{\pgfqpoint{4.227184in}{2.077706in}}%
\pgfpathlineto{\pgfqpoint{4.213317in}{2.077861in}}%
\pgfpathlineto{\pgfqpoint{4.205383in}{2.069201in}}%
\pgfpathlineto{\pgfqpoint{4.197444in}{2.060497in}}%
\pgfpathlineto{\pgfqpoint{4.189499in}{2.051749in}}%
\pgfpathlineto{\pgfqpoint{4.181548in}{2.042954in}}%
\pgfpathclose%
\pgfusepath{fill}%
\end{pgfscope}%
\begin{pgfscope}%
\pgfpathrectangle{\pgfqpoint{1.150000in}{0.150000in}}{\pgfqpoint{5.700000in}{5.700000in}}%
\pgfusepath{clip}%
\pgfsetbuttcap%
\pgfsetroundjoin%
\definecolor{currentfill}{rgb}{0.269944,0.014625,0.341379}%
\pgfsetfillcolor{currentfill}%
\pgfsetfillopacity{0.700000}%
\pgfsetlinewidth{0.000000pt}%
\definecolor{currentstroke}{rgb}{0.000000,0.000000,0.000000}%
\pgfsetstrokecolor{currentstroke}%
\pgfsetdash{}{0pt}%
\pgfpathmoveto{\pgfqpoint{3.229181in}{1.829610in}}%
\pgfpathlineto{\pgfqpoint{3.242846in}{1.824895in}}%
\pgfpathlineto{\pgfqpoint{3.256516in}{1.820268in}}%
\pgfpathlineto{\pgfqpoint{3.270191in}{1.815728in}}%
\pgfpathlineto{\pgfqpoint{3.283871in}{1.811275in}}%
\pgfpathlineto{\pgfqpoint{3.292163in}{1.819648in}}%
\pgfpathlineto{\pgfqpoint{3.300449in}{1.828059in}}%
\pgfpathlineto{\pgfqpoint{3.308727in}{1.836507in}}%
\pgfpathlineto{\pgfqpoint{3.317000in}{1.844987in}}%
\pgfpathlineto{\pgfqpoint{3.303335in}{1.849277in}}%
\pgfpathlineto{\pgfqpoint{3.289676in}{1.853654in}}%
\pgfpathlineto{\pgfqpoint{3.276021in}{1.858117in}}%
\pgfpathlineto{\pgfqpoint{3.262371in}{1.862668in}}%
\pgfpathlineto{\pgfqpoint{3.254084in}{1.854343in}}%
\pgfpathlineto{\pgfqpoint{3.245789in}{1.846057in}}%
\pgfpathlineto{\pgfqpoint{3.237489in}{1.837812in}}%
\pgfpathlineto{\pgfqpoint{3.229181in}{1.829610in}}%
\pgfpathclose%
\pgfusepath{fill}%
\end{pgfscope}%
\begin{pgfscope}%
\pgfpathrectangle{\pgfqpoint{1.150000in}{0.150000in}}{\pgfqpoint{5.700000in}{5.700000in}}%
\pgfusepath{clip}%
\pgfsetbuttcap%
\pgfsetroundjoin%
\definecolor{currentfill}{rgb}{0.276022,0.044167,0.370164}%
\pgfsetfillcolor{currentfill}%
\pgfsetfillopacity{0.700000}%
\pgfsetlinewidth{0.000000pt}%
\definecolor{currentstroke}{rgb}{0.000000,0.000000,0.000000}%
\pgfsetstrokecolor{currentstroke}%
\pgfsetdash{}{0pt}%
\pgfpathmoveto{\pgfqpoint{2.800706in}{1.890804in}}%
\pgfpathlineto{\pgfqpoint{2.814351in}{1.883233in}}%
\pgfpathlineto{\pgfqpoint{2.827998in}{1.875762in}}%
\pgfpathlineto{\pgfqpoint{2.841647in}{1.868390in}}%
\pgfpathlineto{\pgfqpoint{2.855298in}{1.861117in}}%
\pgfpathlineto{\pgfqpoint{2.863796in}{1.867428in}}%
\pgfpathlineto{\pgfqpoint{2.872286in}{1.873841in}}%
\pgfpathlineto{\pgfqpoint{2.880766in}{1.880353in}}%
\pgfpathlineto{\pgfqpoint{2.889236in}{1.886962in}}%
\pgfpathlineto{\pgfqpoint{2.875607in}{1.894008in}}%
\pgfpathlineto{\pgfqpoint{2.861979in}{1.901153in}}%
\pgfpathlineto{\pgfqpoint{2.848354in}{1.908397in}}%
\pgfpathlineto{\pgfqpoint{2.834731in}{1.915741in}}%
\pgfpathlineto{\pgfqpoint{2.826239in}{1.909352in}}%
\pgfpathlineto{\pgfqpoint{2.817738in}{1.903064in}}%
\pgfpathlineto{\pgfqpoint{2.809227in}{1.896880in}}%
\pgfpathlineto{\pgfqpoint{2.800706in}{1.890804in}}%
\pgfpathclose%
\pgfusepath{fill}%
\end{pgfscope}%
\begin{pgfscope}%
\pgfpathrectangle{\pgfqpoint{1.150000in}{0.150000in}}{\pgfqpoint{5.700000in}{5.700000in}}%
\pgfusepath{clip}%
\pgfsetbuttcap%
\pgfsetroundjoin%
\definecolor{currentfill}{rgb}{0.269308,0.218818,0.509577}%
\pgfsetfillcolor{currentfill}%
\pgfsetfillopacity{0.700000}%
\pgfsetlinewidth{0.000000pt}%
\definecolor{currentstroke}{rgb}{0.000000,0.000000,0.000000}%
\pgfsetstrokecolor{currentstroke}%
\pgfsetdash{}{0pt}%
\pgfpathmoveto{\pgfqpoint{4.673674in}{2.220978in}}%
\pgfpathlineto{\pgfqpoint{4.687719in}{2.222279in}}%
\pgfpathlineto{\pgfqpoint{4.701774in}{2.223652in}}%
\pgfpathlineto{\pgfqpoint{4.715839in}{2.225097in}}%
\pgfpathlineto{\pgfqpoint{4.729915in}{2.226613in}}%
\pgfpathlineto{\pgfqpoint{4.737661in}{2.233914in}}%
\pgfpathlineto{\pgfqpoint{4.745401in}{2.241165in}}%
\pgfpathlineto{\pgfqpoint{4.753135in}{2.248370in}}%
\pgfpathlineto{\pgfqpoint{4.760862in}{2.255529in}}%
\pgfpathlineto{\pgfqpoint{4.746799in}{2.254141in}}%
\pgfpathlineto{\pgfqpoint{4.732747in}{2.252823in}}%
\pgfpathlineto{\pgfqpoint{4.718706in}{2.251577in}}%
\pgfpathlineto{\pgfqpoint{4.704674in}{2.250402in}}%
\pgfpathlineto{\pgfqpoint{4.696933in}{2.243108in}}%
\pgfpathlineto{\pgfqpoint{4.689187in}{2.235774in}}%
\pgfpathlineto{\pgfqpoint{4.681434in}{2.228398in}}%
\pgfpathlineto{\pgfqpoint{4.673674in}{2.220978in}}%
\pgfpathclose%
\pgfusepath{fill}%
\end{pgfscope}%
\begin{pgfscope}%
\pgfpathrectangle{\pgfqpoint{1.150000in}{0.150000in}}{\pgfqpoint{5.700000in}{5.700000in}}%
\pgfusepath{clip}%
\pgfsetbuttcap%
\pgfsetroundjoin%
\definecolor{currentfill}{rgb}{0.241237,0.296485,0.539709}%
\pgfsetfillcolor{currentfill}%
\pgfsetfillopacity{0.700000}%
\pgfsetlinewidth{0.000000pt}%
\definecolor{currentstroke}{rgb}{0.000000,0.000000,0.000000}%
\pgfsetstrokecolor{currentstroke}%
\pgfsetdash{}{0pt}%
\pgfpathmoveto{\pgfqpoint{5.165859in}{2.394680in}}%
\pgfpathlineto{\pgfqpoint{5.180083in}{2.396825in}}%
\pgfpathlineto{\pgfqpoint{5.194319in}{2.399040in}}%
\pgfpathlineto{\pgfqpoint{5.208566in}{2.401325in}}%
\pgfpathlineto{\pgfqpoint{5.222825in}{2.403678in}}%
\pgfpathlineto{\pgfqpoint{5.230343in}{2.409211in}}%
\pgfpathlineto{\pgfqpoint{5.237855in}{2.414729in}}%
\pgfpathlineto{\pgfqpoint{5.245361in}{2.420237in}}%
\pgfpathlineto{\pgfqpoint{5.252861in}{2.425739in}}%
\pgfpathlineto{\pgfqpoint{5.238621in}{2.423619in}}%
\pgfpathlineto{\pgfqpoint{5.224392in}{2.421568in}}%
\pgfpathlineto{\pgfqpoint{5.210175in}{2.419586in}}%
\pgfpathlineto{\pgfqpoint{5.195969in}{2.417673in}}%
\pgfpathlineto{\pgfqpoint{5.188451in}{2.411931in}}%
\pgfpathlineto{\pgfqpoint{5.180927in}{2.406187in}}%
\pgfpathlineto{\pgfqpoint{5.173396in}{2.400438in}}%
\pgfpathlineto{\pgfqpoint{5.165859in}{2.394680in}}%
\pgfpathclose%
\pgfusepath{fill}%
\end{pgfscope}%
\begin{pgfscope}%
\pgfpathrectangle{\pgfqpoint{1.150000in}{0.150000in}}{\pgfqpoint{5.700000in}{5.700000in}}%
\pgfusepath{clip}%
\pgfsetbuttcap%
\pgfsetroundjoin%
\definecolor{currentfill}{rgb}{0.277018,0.050344,0.375715}%
\pgfsetfillcolor{currentfill}%
\pgfsetfillopacity{0.700000}%
\pgfsetlinewidth{0.000000pt}%
\definecolor{currentstroke}{rgb}{0.000000,0.000000,0.000000}%
\pgfsetstrokecolor{currentstroke}%
\pgfsetdash{}{0pt}%
\pgfpathmoveto{\pgfqpoint{3.689400in}{1.887912in}}%
\pgfpathlineto{\pgfqpoint{3.703147in}{1.885729in}}%
\pgfpathlineto{\pgfqpoint{3.716901in}{1.883625in}}%
\pgfpathlineto{\pgfqpoint{3.730661in}{1.881600in}}%
\pgfpathlineto{\pgfqpoint{3.744428in}{1.879654in}}%
\pgfpathlineto{\pgfqpoint{3.752545in}{1.888953in}}%
\pgfpathlineto{\pgfqpoint{3.760657in}{1.898233in}}%
\pgfpathlineto{\pgfqpoint{3.768762in}{1.907494in}}%
\pgfpathlineto{\pgfqpoint{3.776862in}{1.916736in}}%
\pgfpathlineto{\pgfqpoint{3.763106in}{1.918600in}}%
\pgfpathlineto{\pgfqpoint{3.749357in}{1.920543in}}%
\pgfpathlineto{\pgfqpoint{3.735615in}{1.922565in}}%
\pgfpathlineto{\pgfqpoint{3.721879in}{1.924667in}}%
\pgfpathlineto{\pgfqpoint{3.713768in}{1.915500in}}%
\pgfpathlineto{\pgfqpoint{3.705651in}{1.906318in}}%
\pgfpathlineto{\pgfqpoint{3.697528in}{1.897122in}}%
\pgfpathlineto{\pgfqpoint{3.689400in}{1.887912in}}%
\pgfpathclose%
\pgfusepath{fill}%
\end{pgfscope}%
\begin{pgfscope}%
\pgfpathrectangle{\pgfqpoint{1.150000in}{0.150000in}}{\pgfqpoint{5.700000in}{5.700000in}}%
\pgfusepath{clip}%
\pgfsetbuttcap%
\pgfsetroundjoin%
\definecolor{currentfill}{rgb}{0.283197,0.115680,0.436115}%
\pgfsetfillcolor{currentfill}%
\pgfsetfillopacity{0.700000}%
\pgfsetlinewidth{0.000000pt}%
\definecolor{currentstroke}{rgb}{0.000000,0.000000,0.000000}%
\pgfsetstrokecolor{currentstroke}%
\pgfsetdash{}{0pt}%
\pgfpathmoveto{\pgfqpoint{4.094218in}{2.008587in}}%
\pgfpathlineto{\pgfqpoint{4.108072in}{2.008155in}}%
\pgfpathlineto{\pgfqpoint{4.121935in}{2.007797in}}%
\pgfpathlineto{\pgfqpoint{4.135806in}{2.007515in}}%
\pgfpathlineto{\pgfqpoint{4.149686in}{2.007308in}}%
\pgfpathlineto{\pgfqpoint{4.157660in}{2.016291in}}%
\pgfpathlineto{\pgfqpoint{4.165628in}{2.025226in}}%
\pgfpathlineto{\pgfqpoint{4.173591in}{2.034114in}}%
\pgfpathlineto{\pgfqpoint{4.181548in}{2.042954in}}%
\pgfpathlineto{\pgfqpoint{4.167679in}{2.043163in}}%
\pgfpathlineto{\pgfqpoint{4.153818in}{2.043446in}}%
\pgfpathlineto{\pgfqpoint{4.139966in}{2.043804in}}%
\pgfpathlineto{\pgfqpoint{4.126122in}{2.044238in}}%
\pgfpathlineto{\pgfqpoint{4.118154in}{2.035389in}}%
\pgfpathlineto{\pgfqpoint{4.110181in}{2.026497in}}%
\pgfpathlineto{\pgfqpoint{4.102203in}{2.017564in}}%
\pgfpathlineto{\pgfqpoint{4.094218in}{2.008587in}}%
\pgfpathclose%
\pgfusepath{fill}%
\end{pgfscope}%
\begin{pgfscope}%
\pgfpathrectangle{\pgfqpoint{1.150000in}{0.150000in}}{\pgfqpoint{5.700000in}{5.700000in}}%
\pgfusepath{clip}%
\pgfsetbuttcap%
\pgfsetroundjoin%
\definecolor{currentfill}{rgb}{0.218130,0.347432,0.550038}%
\pgfsetfillcolor{currentfill}%
\pgfsetfillopacity{0.700000}%
\pgfsetlinewidth{0.000000pt}%
\definecolor{currentstroke}{rgb}{0.000000,0.000000,0.000000}%
\pgfsetstrokecolor{currentstroke}%
\pgfsetdash{}{0pt}%
\pgfpathmoveto{\pgfqpoint{5.570794in}{2.523058in}}%
\pgfpathlineto{\pgfqpoint{5.585167in}{2.525518in}}%
\pgfpathlineto{\pgfqpoint{5.599552in}{2.528046in}}%
\pgfpathlineto{\pgfqpoint{5.613949in}{2.530642in}}%
\pgfpathlineto{\pgfqpoint{5.628358in}{2.533306in}}%
\pgfpathlineto{\pgfqpoint{5.635672in}{2.537616in}}%
\pgfpathlineto{\pgfqpoint{5.642981in}{2.541964in}}%
\pgfpathlineto{\pgfqpoint{5.650285in}{2.546356in}}%
\pgfpathlineto{\pgfqpoint{5.657584in}{2.550796in}}%
\pgfpathlineto{\pgfqpoint{5.643200in}{2.548451in}}%
\pgfpathlineto{\pgfqpoint{5.628827in}{2.546173in}}%
\pgfpathlineto{\pgfqpoint{5.614467in}{2.543962in}}%
\pgfpathlineto{\pgfqpoint{5.600118in}{2.541820in}}%
\pgfpathlineto{\pgfqpoint{5.592794in}{2.537054in}}%
\pgfpathlineto{\pgfqpoint{5.585465in}{2.532342in}}%
\pgfpathlineto{\pgfqpoint{5.578132in}{2.527679in}}%
\pgfpathlineto{\pgfqpoint{5.570794in}{2.523058in}}%
\pgfpathclose%
\pgfusepath{fill}%
\end{pgfscope}%
\begin{pgfscope}%
\pgfpathrectangle{\pgfqpoint{1.150000in}{0.150000in}}{\pgfqpoint{5.700000in}{5.700000in}}%
\pgfusepath{clip}%
\pgfsetbuttcap%
\pgfsetroundjoin%
\definecolor{currentfill}{rgb}{0.282884,0.135920,0.453427}%
\pgfsetfillcolor{currentfill}%
\pgfsetfillopacity{0.700000}%
\pgfsetlinewidth{0.000000pt}%
\definecolor{currentstroke}{rgb}{0.000000,0.000000,0.000000}%
\pgfsetstrokecolor{currentstroke}%
\pgfsetdash{}{0pt}%
\pgfpathmoveto{\pgfqpoint{2.403753in}{2.072390in}}%
\pgfpathlineto{\pgfqpoint{2.417439in}{2.061686in}}%
\pgfpathlineto{\pgfqpoint{2.431124in}{2.051101in}}%
\pgfpathlineto{\pgfqpoint{2.444808in}{2.040633in}}%
\pgfpathlineto{\pgfqpoint{2.458491in}{2.030282in}}%
\pgfpathlineto{\pgfqpoint{2.467232in}{2.033978in}}%
\pgfpathlineto{\pgfqpoint{2.475960in}{2.037837in}}%
\pgfpathlineto{\pgfqpoint{2.484675in}{2.041856in}}%
\pgfpathlineto{\pgfqpoint{2.493378in}{2.046030in}}%
\pgfpathlineto{\pgfqpoint{2.479723in}{2.056109in}}%
\pgfpathlineto{\pgfqpoint{2.466068in}{2.066304in}}%
\pgfpathlineto{\pgfqpoint{2.452412in}{2.076617in}}%
\pgfpathlineto{\pgfqpoint{2.438755in}{2.087048in}}%
\pgfpathlineto{\pgfqpoint{2.430025in}{2.083138in}}%
\pgfpathlineto{\pgfqpoint{2.421281in}{2.079389in}}%
\pgfpathlineto{\pgfqpoint{2.412524in}{2.075805in}}%
\pgfpathlineto{\pgfqpoint{2.403753in}{2.072390in}}%
\pgfpathclose%
\pgfusepath{fill}%
\end{pgfscope}%
\begin{pgfscope}%
\pgfpathrectangle{\pgfqpoint{1.150000in}{0.150000in}}{\pgfqpoint{5.700000in}{5.700000in}}%
\pgfusepath{clip}%
\pgfsetbuttcap%
\pgfsetroundjoin%
\definecolor{currentfill}{rgb}{0.273006,0.204520,0.501721}%
\pgfsetfillcolor{currentfill}%
\pgfsetfillopacity{0.700000}%
\pgfsetlinewidth{0.000000pt}%
\definecolor{currentstroke}{rgb}{0.000000,0.000000,0.000000}%
\pgfsetstrokecolor{currentstroke}%
\pgfsetdash{}{0pt}%
\pgfpathmoveto{\pgfqpoint{4.586441in}{2.185890in}}%
\pgfpathlineto{\pgfqpoint{4.600459in}{2.187010in}}%
\pgfpathlineto{\pgfqpoint{4.614486in}{2.188202in}}%
\pgfpathlineto{\pgfqpoint{4.628524in}{2.189466in}}%
\pgfpathlineto{\pgfqpoint{4.642571in}{2.190802in}}%
\pgfpathlineto{\pgfqpoint{4.650357in}{2.198425in}}%
\pgfpathlineto{\pgfqpoint{4.658136in}{2.205993in}}%
\pgfpathlineto{\pgfqpoint{4.665908in}{2.213510in}}%
\pgfpathlineto{\pgfqpoint{4.673674in}{2.220978in}}%
\pgfpathlineto{\pgfqpoint{4.659639in}{2.219748in}}%
\pgfpathlineto{\pgfqpoint{4.645614in}{2.218590in}}%
\pgfpathlineto{\pgfqpoint{4.631599in}{2.217504in}}%
\pgfpathlineto{\pgfqpoint{4.617594in}{2.216490in}}%
\pgfpathlineto{\pgfqpoint{4.609815in}{2.208908in}}%
\pgfpathlineto{\pgfqpoint{4.602030in}{2.201283in}}%
\pgfpathlineto{\pgfqpoint{4.594239in}{2.193611in}}%
\pgfpathlineto{\pgfqpoint{4.586441in}{2.185890in}}%
\pgfpathclose%
\pgfusepath{fill}%
\end{pgfscope}%
\begin{pgfscope}%
\pgfpathrectangle{\pgfqpoint{1.150000in}{0.150000in}}{\pgfqpoint{5.700000in}{5.700000in}}%
\pgfusepath{clip}%
\pgfsetbuttcap%
\pgfsetroundjoin%
\definecolor{currentfill}{rgb}{0.282656,0.100196,0.422160}%
\pgfsetfillcolor{currentfill}%
\pgfsetfillopacity{0.700000}%
\pgfsetlinewidth{0.000000pt}%
\definecolor{currentstroke}{rgb}{0.000000,0.000000,0.000000}%
\pgfsetstrokecolor{currentstroke}%
\pgfsetdash{}{0pt}%
\pgfpathmoveto{\pgfqpoint{4.006846in}{1.974813in}}%
\pgfpathlineto{\pgfqpoint{4.020678in}{1.974058in}}%
\pgfpathlineto{\pgfqpoint{4.034518in}{1.973378in}}%
\pgfpathlineto{\pgfqpoint{4.048367in}{1.972775in}}%
\pgfpathlineto{\pgfqpoint{4.062223in}{1.972247in}}%
\pgfpathlineto{\pgfqpoint{4.070230in}{1.981398in}}%
\pgfpathlineto{\pgfqpoint{4.078232in}{1.990504in}}%
\pgfpathlineto{\pgfqpoint{4.086228in}{1.999568in}}%
\pgfpathlineto{\pgfqpoint{4.094218in}{2.008587in}}%
\pgfpathlineto{\pgfqpoint{4.080372in}{2.009095in}}%
\pgfpathlineto{\pgfqpoint{4.066534in}{2.009679in}}%
\pgfpathlineto{\pgfqpoint{4.052704in}{2.010339in}}%
\pgfpathlineto{\pgfqpoint{4.038882in}{2.011075in}}%
\pgfpathlineto{\pgfqpoint{4.030882in}{2.002067in}}%
\pgfpathlineto{\pgfqpoint{4.022876in}{1.993021in}}%
\pgfpathlineto{\pgfqpoint{4.014864in}{1.983937in}}%
\pgfpathlineto{\pgfqpoint{4.006846in}{1.974813in}}%
\pgfpathclose%
\pgfusepath{fill}%
\end{pgfscope}%
\begin{pgfscope}%
\pgfpathrectangle{\pgfqpoint{1.150000in}{0.150000in}}{\pgfqpoint{5.700000in}{5.700000in}}%
\pgfusepath{clip}%
\pgfsetbuttcap%
\pgfsetroundjoin%
\definecolor{currentfill}{rgb}{0.271305,0.019942,0.347269}%
\pgfsetfillcolor{currentfill}%
\pgfsetfillopacity{0.700000}%
\pgfsetlinewidth{0.000000pt}%
\definecolor{currentstroke}{rgb}{0.000000,0.000000,0.000000}%
\pgfsetstrokecolor{currentstroke}%
\pgfsetdash{}{0pt}%
\pgfpathmoveto{\pgfqpoint{3.371706in}{1.828686in}}%
\pgfpathlineto{\pgfqpoint{3.385396in}{1.824823in}}%
\pgfpathlineto{\pgfqpoint{3.399090in}{1.821045in}}%
\pgfpathlineto{\pgfqpoint{3.412790in}{1.817350in}}%
\pgfpathlineto{\pgfqpoint{3.426496in}{1.813740in}}%
\pgfpathlineto{\pgfqpoint{3.434733in}{1.822554in}}%
\pgfpathlineto{\pgfqpoint{3.442964in}{1.831387in}}%
\pgfpathlineto{\pgfqpoint{3.451189in}{1.840238in}}%
\pgfpathlineto{\pgfqpoint{3.459407in}{1.849105in}}%
\pgfpathlineto{\pgfqpoint{3.445715in}{1.852573in}}%
\pgfpathlineto{\pgfqpoint{3.432029in}{1.856124in}}%
\pgfpathlineto{\pgfqpoint{3.418348in}{1.859760in}}%
\pgfpathlineto{\pgfqpoint{3.404672in}{1.863479in}}%
\pgfpathlineto{\pgfqpoint{3.396440in}{1.854748in}}%
\pgfpathlineto{\pgfqpoint{3.388202in}{1.846037in}}%
\pgfpathlineto{\pgfqpoint{3.379957in}{1.837349in}}%
\pgfpathlineto{\pgfqpoint{3.371706in}{1.828686in}}%
\pgfpathclose%
\pgfusepath{fill}%
\end{pgfscope}%
\begin{pgfscope}%
\pgfpathrectangle{\pgfqpoint{1.150000in}{0.150000in}}{\pgfqpoint{5.700000in}{5.700000in}}%
\pgfusepath{clip}%
\pgfsetbuttcap%
\pgfsetroundjoin%
\definecolor{currentfill}{rgb}{0.246811,0.283237,0.535941}%
\pgfsetfillcolor{currentfill}%
\pgfsetfillopacity{0.700000}%
\pgfsetlinewidth{0.000000pt}%
\definecolor{currentstroke}{rgb}{0.000000,0.000000,0.000000}%
\pgfsetstrokecolor{currentstroke}%
\pgfsetdash{}{0pt}%
\pgfpathmoveto{\pgfqpoint{5.078789in}{2.362739in}}%
\pgfpathlineto{\pgfqpoint{5.092986in}{2.364818in}}%
\pgfpathlineto{\pgfqpoint{5.107195in}{2.366966in}}%
\pgfpathlineto{\pgfqpoint{5.121415in}{2.369185in}}%
\pgfpathlineto{\pgfqpoint{5.135646in}{2.371473in}}%
\pgfpathlineto{\pgfqpoint{5.143209in}{2.377309in}}%
\pgfpathlineto{\pgfqpoint{5.150766in}{2.383119in}}%
\pgfpathlineto{\pgfqpoint{5.158316in}{2.388908in}}%
\pgfpathlineto{\pgfqpoint{5.165859in}{2.394680in}}%
\pgfpathlineto{\pgfqpoint{5.151646in}{2.392604in}}%
\pgfpathlineto{\pgfqpoint{5.137443in}{2.390598in}}%
\pgfpathlineto{\pgfqpoint{5.123252in}{2.388661in}}%
\pgfpathlineto{\pgfqpoint{5.109072in}{2.386794in}}%
\pgfpathlineto{\pgfqpoint{5.101511in}{2.380802in}}%
\pgfpathlineto{\pgfqpoint{5.093943in}{2.374799in}}%
\pgfpathlineto{\pgfqpoint{5.086369in}{2.368779in}}%
\pgfpathlineto{\pgfqpoint{5.078789in}{2.362739in}}%
\pgfpathclose%
\pgfusepath{fill}%
\end{pgfscope}%
\begin{pgfscope}%
\pgfpathrectangle{\pgfqpoint{1.150000in}{0.150000in}}{\pgfqpoint{5.700000in}{5.700000in}}%
\pgfusepath{clip}%
\pgfsetbuttcap%
\pgfsetroundjoin%
\definecolor{currentfill}{rgb}{0.279566,0.067836,0.391917}%
\pgfsetfillcolor{currentfill}%
\pgfsetfillopacity{0.700000}%
\pgfsetlinewidth{0.000000pt}%
\definecolor{currentstroke}{rgb}{0.000000,0.000000,0.000000}%
\pgfsetstrokecolor{currentstroke}%
\pgfsetdash{}{0pt}%
\pgfpathmoveto{\pgfqpoint{2.657229in}{1.933861in}}%
\pgfpathlineto{\pgfqpoint{2.670887in}{1.925221in}}%
\pgfpathlineto{\pgfqpoint{2.684545in}{1.916685in}}%
\pgfpathlineto{\pgfqpoint{2.698205in}{1.908255in}}%
\pgfpathlineto{\pgfqpoint{2.711865in}{1.899927in}}%
\pgfpathlineto{\pgfqpoint{2.720451in}{1.905283in}}%
\pgfpathlineto{\pgfqpoint{2.729026in}{1.910765in}}%
\pgfpathlineto{\pgfqpoint{2.737590in}{1.916370in}}%
\pgfpathlineto{\pgfqpoint{2.746144in}{1.922096in}}%
\pgfpathlineto{\pgfqpoint{2.732507in}{1.930174in}}%
\pgfpathlineto{\pgfqpoint{2.718872in}{1.938357in}}%
\pgfpathlineto{\pgfqpoint{2.705238in}{1.946644in}}%
\pgfpathlineto{\pgfqpoint{2.691605in}{1.955035in}}%
\pgfpathlineto{\pgfqpoint{2.683028in}{1.949551in}}%
\pgfpathlineto{\pgfqpoint{2.674439in}{1.944191in}}%
\pgfpathlineto{\pgfqpoint{2.665840in}{1.938960in}}%
\pgfpathlineto{\pgfqpoint{2.657229in}{1.933861in}}%
\pgfpathclose%
\pgfusepath{fill}%
\end{pgfscope}%
\begin{pgfscope}%
\pgfpathrectangle{\pgfqpoint{1.150000in}{0.150000in}}{\pgfqpoint{5.700000in}{5.700000in}}%
\pgfusepath{clip}%
\pgfsetbuttcap%
\pgfsetroundjoin%
\definecolor{currentfill}{rgb}{0.262138,0.242286,0.520837}%
\pgfsetfillcolor{currentfill}%
\pgfsetfillopacity{0.700000}%
\pgfsetlinewidth{0.000000pt}%
\definecolor{currentstroke}{rgb}{0.000000,0.000000,0.000000}%
\pgfsetstrokecolor{currentstroke}%
\pgfsetdash{}{0pt}%
\pgfpathmoveto{\pgfqpoint{2.093720in}{2.307028in}}%
\pgfpathlineto{\pgfqpoint{2.107490in}{2.293428in}}%
\pgfpathlineto{\pgfqpoint{2.121256in}{2.279970in}}%
\pgfpathlineto{\pgfqpoint{2.135019in}{2.266653in}}%
\pgfpathlineto{\pgfqpoint{2.148778in}{2.253473in}}%
\pgfpathlineto{\pgfqpoint{2.157741in}{2.254950in}}%
\pgfpathlineto{\pgfqpoint{2.166687in}{2.256636in}}%
\pgfpathlineto{\pgfqpoint{2.175617in}{2.258524in}}%
\pgfpathlineto{\pgfqpoint{2.184531in}{2.260612in}}%
\pgfpathlineto{\pgfqpoint{2.170807in}{2.273492in}}%
\pgfpathlineto{\pgfqpoint{2.157080in}{2.286510in}}%
\pgfpathlineto{\pgfqpoint{2.143349in}{2.299668in}}%
\pgfpathlineto{\pgfqpoint{2.129614in}{2.312967in}}%
\pgfpathlineto{\pgfqpoint{2.120666in}{2.311170in}}%
\pgfpathlineto{\pgfqpoint{2.111701in}{2.309579in}}%
\pgfpathlineto{\pgfqpoint{2.102719in}{2.308197in}}%
\pgfpathlineto{\pgfqpoint{2.093720in}{2.307028in}}%
\pgfpathclose%
\pgfusepath{fill}%
\end{pgfscope}%
\begin{pgfscope}%
\pgfpathrectangle{\pgfqpoint{1.150000in}{0.150000in}}{\pgfqpoint{5.700000in}{5.700000in}}%
\pgfusepath{clip}%
\pgfsetbuttcap%
\pgfsetroundjoin%
\definecolor{currentfill}{rgb}{0.274952,0.037752,0.364543}%
\pgfsetfillcolor{currentfill}%
\pgfsetfillopacity{0.700000}%
\pgfsetlinewidth{0.000000pt}%
\definecolor{currentstroke}{rgb}{0.000000,0.000000,0.000000}%
\pgfsetstrokecolor{currentstroke}%
\pgfsetdash{}{0pt}%
\pgfpathmoveto{\pgfqpoint{3.601861in}{1.860909in}}%
\pgfpathlineto{\pgfqpoint{3.615593in}{1.858303in}}%
\pgfpathlineto{\pgfqpoint{3.629332in}{1.855777in}}%
\pgfpathlineto{\pgfqpoint{3.643078in}{1.853331in}}%
\pgfpathlineto{\pgfqpoint{3.656830in}{1.850966in}}%
\pgfpathlineto{\pgfqpoint{3.664981in}{1.860217in}}%
\pgfpathlineto{\pgfqpoint{3.673126in}{1.869459in}}%
\pgfpathlineto{\pgfqpoint{3.681266in}{1.878691in}}%
\pgfpathlineto{\pgfqpoint{3.689400in}{1.887912in}}%
\pgfpathlineto{\pgfqpoint{3.675660in}{1.890176in}}%
\pgfpathlineto{\pgfqpoint{3.661926in}{1.892519in}}%
\pgfpathlineto{\pgfqpoint{3.648199in}{1.894943in}}%
\pgfpathlineto{\pgfqpoint{3.634478in}{1.897447in}}%
\pgfpathlineto{\pgfqpoint{3.626332in}{1.888320in}}%
\pgfpathlineto{\pgfqpoint{3.618181in}{1.879188in}}%
\pgfpathlineto{\pgfqpoint{3.610024in}{1.870050in}}%
\pgfpathlineto{\pgfqpoint{3.601861in}{1.860909in}}%
\pgfpathclose%
\pgfusepath{fill}%
\end{pgfscope}%
\begin{pgfscope}%
\pgfpathrectangle{\pgfqpoint{1.150000in}{0.150000in}}{\pgfqpoint{5.700000in}{5.700000in}}%
\pgfusepath{clip}%
\pgfsetbuttcap%
\pgfsetroundjoin%
\definecolor{currentfill}{rgb}{0.276194,0.190074,0.493001}%
\pgfsetfillcolor{currentfill}%
\pgfsetfillopacity{0.700000}%
\pgfsetlinewidth{0.000000pt}%
\definecolor{currentstroke}{rgb}{0.000000,0.000000,0.000000}%
\pgfsetstrokecolor{currentstroke}%
\pgfsetdash{}{0pt}%
\pgfpathmoveto{\pgfqpoint{4.499166in}{2.150385in}}%
\pgfpathlineto{\pgfqpoint{4.513156in}{2.151301in}}%
\pgfpathlineto{\pgfqpoint{4.527156in}{2.152289in}}%
\pgfpathlineto{\pgfqpoint{4.541166in}{2.153349in}}%
\pgfpathlineto{\pgfqpoint{4.555186in}{2.154482in}}%
\pgfpathlineto{\pgfqpoint{4.563009in}{2.162417in}}%
\pgfpathlineto{\pgfqpoint{4.570826in}{2.170295in}}%
\pgfpathlineto{\pgfqpoint{4.578637in}{2.178119in}}%
\pgfpathlineto{\pgfqpoint{4.586441in}{2.185890in}}%
\pgfpathlineto{\pgfqpoint{4.572433in}{2.184843in}}%
\pgfpathlineto{\pgfqpoint{4.558435in}{2.183867in}}%
\pgfpathlineto{\pgfqpoint{4.544446in}{2.182964in}}%
\pgfpathlineto{\pgfqpoint{4.530468in}{2.182133in}}%
\pgfpathlineto{\pgfqpoint{4.522651in}{2.174269in}}%
\pgfpathlineto{\pgfqpoint{4.514829in}{2.166358in}}%
\pgfpathlineto{\pgfqpoint{4.507000in}{2.158397in}}%
\pgfpathlineto{\pgfqpoint{4.499166in}{2.150385in}}%
\pgfpathclose%
\pgfusepath{fill}%
\end{pgfscope}%
\begin{pgfscope}%
\pgfpathrectangle{\pgfqpoint{1.150000in}{0.150000in}}{\pgfqpoint{5.700000in}{5.700000in}}%
\pgfusepath{clip}%
\pgfsetbuttcap%
\pgfsetroundjoin%
\definecolor{currentfill}{rgb}{0.221989,0.339161,0.548752}%
\pgfsetfillcolor{currentfill}%
\pgfsetfillopacity{0.700000}%
\pgfsetlinewidth{0.000000pt}%
\definecolor{currentstroke}{rgb}{0.000000,0.000000,0.000000}%
\pgfsetstrokecolor{currentstroke}%
\pgfsetdash{}{0pt}%
\pgfpathmoveto{\pgfqpoint{5.483920in}{2.494551in}}%
\pgfpathlineto{\pgfqpoint{5.498269in}{2.497035in}}%
\pgfpathlineto{\pgfqpoint{5.512629in}{2.499587in}}%
\pgfpathlineto{\pgfqpoint{5.527002in}{2.502208in}}%
\pgfpathlineto{\pgfqpoint{5.541386in}{2.504897in}}%
\pgfpathlineto{\pgfqpoint{5.548746in}{2.509400in}}%
\pgfpathlineto{\pgfqpoint{5.556101in}{2.513924in}}%
\pgfpathlineto{\pgfqpoint{5.563450in}{2.518475in}}%
\pgfpathlineto{\pgfqpoint{5.570794in}{2.523058in}}%
\pgfpathlineto{\pgfqpoint{5.556433in}{2.520667in}}%
\pgfpathlineto{\pgfqpoint{5.542083in}{2.518343in}}%
\pgfpathlineto{\pgfqpoint{5.527746in}{2.516087in}}%
\pgfpathlineto{\pgfqpoint{5.513420in}{2.513900in}}%
\pgfpathlineto{\pgfqpoint{5.506053in}{2.509012in}}%
\pgfpathlineto{\pgfqpoint{5.498681in}{2.504162in}}%
\pgfpathlineto{\pgfqpoint{5.491304in}{2.499343in}}%
\pgfpathlineto{\pgfqpoint{5.483920in}{2.494551in}}%
\pgfpathclose%
\pgfusepath{fill}%
\end{pgfscope}%
\begin{pgfscope}%
\pgfpathrectangle{\pgfqpoint{1.150000in}{0.150000in}}{\pgfqpoint{5.700000in}{5.700000in}}%
\pgfusepath{clip}%
\pgfsetbuttcap%
\pgfsetroundjoin%
\definecolor{currentfill}{rgb}{0.271305,0.019942,0.347269}%
\pgfsetfillcolor{currentfill}%
\pgfsetfillopacity{0.700000}%
\pgfsetlinewidth{0.000000pt}%
\definecolor{currentstroke}{rgb}{0.000000,0.000000,0.000000}%
\pgfsetstrokecolor{currentstroke}%
\pgfsetdash{}{0pt}%
\pgfpathmoveto{\pgfqpoint{2.998368in}{1.834062in}}%
\pgfpathlineto{\pgfqpoint{3.012023in}{1.827875in}}%
\pgfpathlineto{\pgfqpoint{3.025681in}{1.821781in}}%
\pgfpathlineto{\pgfqpoint{3.039342in}{1.815780in}}%
\pgfpathlineto{\pgfqpoint{3.053007in}{1.809870in}}%
\pgfpathlineto{\pgfqpoint{3.061410in}{1.817211in}}%
\pgfpathlineto{\pgfqpoint{3.069804in}{1.824626in}}%
\pgfpathlineto{\pgfqpoint{3.078191in}{1.832112in}}%
\pgfpathlineto{\pgfqpoint{3.086570in}{1.839667in}}%
\pgfpathlineto{\pgfqpoint{3.072924in}{1.845372in}}%
\pgfpathlineto{\pgfqpoint{3.059281in}{1.851168in}}%
\pgfpathlineto{\pgfqpoint{3.045642in}{1.857057in}}%
\pgfpathlineto{\pgfqpoint{3.032006in}{1.863038in}}%
\pgfpathlineto{\pgfqpoint{3.023609in}{1.855680in}}%
\pgfpathlineto{\pgfqpoint{3.015204in}{1.848397in}}%
\pgfpathlineto{\pgfqpoint{3.006790in}{1.841189in}}%
\pgfpathlineto{\pgfqpoint{2.998368in}{1.834062in}}%
\pgfpathclose%
\pgfusepath{fill}%
\end{pgfscope}%
\begin{pgfscope}%
\pgfpathrectangle{\pgfqpoint{1.150000in}{0.150000in}}{\pgfqpoint{5.700000in}{5.700000in}}%
\pgfusepath{clip}%
\pgfsetbuttcap%
\pgfsetroundjoin%
\definecolor{currentfill}{rgb}{0.283197,0.115680,0.436115}%
\pgfsetfillcolor{currentfill}%
\pgfsetfillopacity{0.700000}%
\pgfsetlinewidth{0.000000pt}%
\definecolor{currentstroke}{rgb}{0.000000,0.000000,0.000000}%
\pgfsetstrokecolor{currentstroke}%
\pgfsetdash{}{0pt}%
\pgfpathmoveto{\pgfqpoint{2.458491in}{2.030282in}}%
\pgfpathlineto{\pgfqpoint{2.472174in}{2.020047in}}%
\pgfpathlineto{\pgfqpoint{2.485856in}{2.009927in}}%
\pgfpathlineto{\pgfqpoint{2.499538in}{1.999921in}}%
\pgfpathlineto{\pgfqpoint{2.513219in}{1.990028in}}%
\pgfpathlineto{\pgfqpoint{2.521931in}{1.994003in}}%
\pgfpathlineto{\pgfqpoint{2.530631in}{1.998137in}}%
\pgfpathlineto{\pgfqpoint{2.539318in}{2.002425in}}%
\pgfpathlineto{\pgfqpoint{2.547992in}{2.006863in}}%
\pgfpathlineto{\pgfqpoint{2.534339in}{2.016484in}}%
\pgfpathlineto{\pgfqpoint{2.520685in}{2.026219in}}%
\pgfpathlineto{\pgfqpoint{2.507032in}{2.036067in}}%
\pgfpathlineto{\pgfqpoint{2.493378in}{2.046030in}}%
\pgfpathlineto{\pgfqpoint{2.484675in}{2.041856in}}%
\pgfpathlineto{\pgfqpoint{2.475960in}{2.037837in}}%
\pgfpathlineto{\pgfqpoint{2.467232in}{2.033978in}}%
\pgfpathlineto{\pgfqpoint{2.458491in}{2.030282in}}%
\pgfpathclose%
\pgfusepath{fill}%
\end{pgfscope}%
\begin{pgfscope}%
\pgfpathrectangle{\pgfqpoint{1.150000in}{0.150000in}}{\pgfqpoint{5.700000in}{5.700000in}}%
\pgfusepath{clip}%
\pgfsetbuttcap%
\pgfsetroundjoin%
\definecolor{currentfill}{rgb}{0.204903,0.375746,0.553533}%
\pgfsetfillcolor{currentfill}%
\pgfsetfillopacity{0.700000}%
\pgfsetlinewidth{0.000000pt}%
\definecolor{currentstroke}{rgb}{0.000000,0.000000,0.000000}%
\pgfsetstrokecolor{currentstroke}%
\pgfsetdash{}{0pt}%
\pgfpathmoveto{\pgfqpoint{5.802040in}{2.587659in}}%
\pgfpathlineto{\pgfqpoint{5.816508in}{2.590274in}}%
\pgfpathlineto{\pgfqpoint{5.830988in}{2.592956in}}%
\pgfpathlineto{\pgfqpoint{5.845481in}{2.595706in}}%
\pgfpathlineto{\pgfqpoint{5.852685in}{2.599512in}}%
\pgfpathlineto{\pgfqpoint{5.859885in}{2.603390in}}%
\pgfpathlineto{\pgfqpoint{5.867082in}{2.607346in}}%
\pgfpathlineto{\pgfqpoint{5.874275in}{2.611385in}}%
\pgfpathlineto{\pgfqpoint{5.859811in}{2.608995in}}%
\pgfpathlineto{\pgfqpoint{5.845358in}{2.606673in}}%
\pgfpathlineto{\pgfqpoint{5.830918in}{2.604418in}}%
\pgfpathlineto{\pgfqpoint{5.823704in}{2.600103in}}%
\pgfpathlineto{\pgfqpoint{5.816487in}{2.595876in}}%
\pgfpathlineto{\pgfqpoint{5.809265in}{2.591730in}}%
\pgfpathlineto{\pgfqpoint{5.802040in}{2.587659in}}%
\pgfpathclose%
\pgfusepath{fill}%
\end{pgfscope}%
\begin{pgfscope}%
\pgfpathrectangle{\pgfqpoint{1.150000in}{0.150000in}}{\pgfqpoint{5.700000in}{5.700000in}}%
\pgfusepath{clip}%
\pgfsetbuttcap%
\pgfsetroundjoin%
\definecolor{currentfill}{rgb}{0.281924,0.089666,0.412415}%
\pgfsetfillcolor{currentfill}%
\pgfsetfillopacity{0.700000}%
\pgfsetlinewidth{0.000000pt}%
\definecolor{currentstroke}{rgb}{0.000000,0.000000,0.000000}%
\pgfsetstrokecolor{currentstroke}%
\pgfsetdash{}{0pt}%
\pgfpathmoveto{\pgfqpoint{3.919428in}{1.941882in}}%
\pgfpathlineto{\pgfqpoint{3.933239in}{1.940780in}}%
\pgfpathlineto{\pgfqpoint{3.947058in}{1.939755in}}%
\pgfpathlineto{\pgfqpoint{3.960885in}{1.938806in}}%
\pgfpathlineto{\pgfqpoint{3.974719in}{1.937934in}}%
\pgfpathlineto{\pgfqpoint{3.982759in}{1.947212in}}%
\pgfpathlineto{\pgfqpoint{3.990794in}{1.956451in}}%
\pgfpathlineto{\pgfqpoint{3.998823in}{1.965651in}}%
\pgfpathlineto{\pgfqpoint{4.006846in}{1.974813in}}%
\pgfpathlineto{\pgfqpoint{3.993022in}{1.975645in}}%
\pgfpathlineto{\pgfqpoint{3.979206in}{1.976553in}}%
\pgfpathlineto{\pgfqpoint{3.965397in}{1.977538in}}%
\pgfpathlineto{\pgfqpoint{3.951596in}{1.978600in}}%
\pgfpathlineto{\pgfqpoint{3.943563in}{1.969471in}}%
\pgfpathlineto{\pgfqpoint{3.935523in}{1.960308in}}%
\pgfpathlineto{\pgfqpoint{3.927478in}{1.951112in}}%
\pgfpathlineto{\pgfqpoint{3.919428in}{1.941882in}}%
\pgfpathclose%
\pgfusepath{fill}%
\end{pgfscope}%
\begin{pgfscope}%
\pgfpathrectangle{\pgfqpoint{1.150000in}{0.150000in}}{\pgfqpoint{5.700000in}{5.700000in}}%
\pgfusepath{clip}%
\pgfsetbuttcap%
\pgfsetroundjoin%
\definecolor{currentfill}{rgb}{0.252194,0.269783,0.531579}%
\pgfsetfillcolor{currentfill}%
\pgfsetfillopacity{0.700000}%
\pgfsetlinewidth{0.000000pt}%
\definecolor{currentstroke}{rgb}{0.000000,0.000000,0.000000}%
\pgfsetstrokecolor{currentstroke}%
\pgfsetdash{}{0pt}%
\pgfpathmoveto{\pgfqpoint{4.991655in}{2.329921in}}%
\pgfpathlineto{\pgfqpoint{5.005825in}{2.331911in}}%
\pgfpathlineto{\pgfqpoint{5.020006in}{2.333971in}}%
\pgfpathlineto{\pgfqpoint{5.034198in}{2.336102in}}%
\pgfpathlineto{\pgfqpoint{5.048400in}{2.338302in}}%
\pgfpathlineto{\pgfqpoint{5.056008in}{2.344460in}}%
\pgfpathlineto{\pgfqpoint{5.063608in}{2.350583in}}%
\pgfpathlineto{\pgfqpoint{5.071202in}{2.356674in}}%
\pgfpathlineto{\pgfqpoint{5.078789in}{2.362739in}}%
\pgfpathlineto{\pgfqpoint{5.064603in}{2.360729in}}%
\pgfpathlineto{\pgfqpoint{5.050427in}{2.358790in}}%
\pgfpathlineto{\pgfqpoint{5.036263in}{2.356921in}}%
\pgfpathlineto{\pgfqpoint{5.022109in}{2.355121in}}%
\pgfpathlineto{\pgfqpoint{5.014505in}{2.348859in}}%
\pgfpathlineto{\pgfqpoint{5.006895in}{2.342574in}}%
\pgfpathlineto{\pgfqpoint{4.999278in}{2.336262in}}%
\pgfpathlineto{\pgfqpoint{4.991655in}{2.329921in}}%
\pgfpathclose%
\pgfusepath{fill}%
\end{pgfscope}%
\begin{pgfscope}%
\pgfpathrectangle{\pgfqpoint{1.150000in}{0.150000in}}{\pgfqpoint{5.700000in}{5.700000in}}%
\pgfusepath{clip}%
\pgfsetbuttcap%
\pgfsetroundjoin%
\definecolor{currentfill}{rgb}{0.269944,0.014625,0.341379}%
\pgfsetfillcolor{currentfill}%
\pgfsetfillopacity{0.700000}%
\pgfsetlinewidth{0.000000pt}%
\definecolor{currentstroke}{rgb}{0.000000,0.000000,0.000000}%
\pgfsetstrokecolor{currentstroke}%
\pgfsetdash{}{0pt}%
\pgfpathmoveto{\pgfqpoint{3.141192in}{1.817758in}}%
\pgfpathlineto{\pgfqpoint{3.154857in}{1.812506in}}%
\pgfpathlineto{\pgfqpoint{3.168526in}{1.807343in}}%
\pgfpathlineto{\pgfqpoint{3.182200in}{1.802268in}}%
\pgfpathlineto{\pgfqpoint{3.195877in}{1.797282in}}%
\pgfpathlineto{\pgfqpoint{3.204214in}{1.805287in}}%
\pgfpathlineto{\pgfqpoint{3.212543in}{1.813345in}}%
\pgfpathlineto{\pgfqpoint{3.220866in}{1.821453in}}%
\pgfpathlineto{\pgfqpoint{3.229181in}{1.829610in}}%
\pgfpathlineto{\pgfqpoint{3.215519in}{1.834412in}}%
\pgfpathlineto{\pgfqpoint{3.201862in}{1.839302in}}%
\pgfpathlineto{\pgfqpoint{3.188210in}{1.844281in}}%
\pgfpathlineto{\pgfqpoint{3.174561in}{1.849349in}}%
\pgfpathlineto{\pgfqpoint{3.166230in}{1.841369in}}%
\pgfpathlineto{\pgfqpoint{3.157891in}{1.833443in}}%
\pgfpathlineto{\pgfqpoint{3.149545in}{1.825572in}}%
\pgfpathlineto{\pgfqpoint{3.141192in}{1.817758in}}%
\pgfpathclose%
\pgfusepath{fill}%
\end{pgfscope}%
\begin{pgfscope}%
\pgfpathrectangle{\pgfqpoint{1.150000in}{0.150000in}}{\pgfqpoint{5.700000in}{5.700000in}}%
\pgfusepath{clip}%
\pgfsetbuttcap%
\pgfsetroundjoin%
\definecolor{currentfill}{rgb}{0.273809,0.031497,0.358853}%
\pgfsetfillcolor{currentfill}%
\pgfsetfillopacity{0.700000}%
\pgfsetlinewidth{0.000000pt}%
\definecolor{currentstroke}{rgb}{0.000000,0.000000,0.000000}%
\pgfsetstrokecolor{currentstroke}%
\pgfsetdash{}{0pt}%
\pgfpathmoveto{\pgfqpoint{2.855298in}{1.861117in}}%
\pgfpathlineto{\pgfqpoint{2.868951in}{1.853942in}}%
\pgfpathlineto{\pgfqpoint{2.882607in}{1.846864in}}%
\pgfpathlineto{\pgfqpoint{2.896265in}{1.839883in}}%
\pgfpathlineto{\pgfqpoint{2.909925in}{1.832998in}}%
\pgfpathlineto{\pgfqpoint{2.918403in}{1.839542in}}%
\pgfpathlineto{\pgfqpoint{2.926871in}{1.846184in}}%
\pgfpathlineto{\pgfqpoint{2.935330in}{1.852921in}}%
\pgfpathlineto{\pgfqpoint{2.943780in}{1.859747in}}%
\pgfpathlineto{\pgfqpoint{2.930140in}{1.866406in}}%
\pgfpathlineto{\pgfqpoint{2.916503in}{1.873161in}}%
\pgfpathlineto{\pgfqpoint{2.902868in}{1.880013in}}%
\pgfpathlineto{\pgfqpoint{2.889236in}{1.886962in}}%
\pgfpathlineto{\pgfqpoint{2.880766in}{1.880353in}}%
\pgfpathlineto{\pgfqpoint{2.872286in}{1.873841in}}%
\pgfpathlineto{\pgfqpoint{2.863796in}{1.867428in}}%
\pgfpathlineto{\pgfqpoint{2.855298in}{1.861117in}}%
\pgfpathclose%
\pgfusepath{fill}%
\end{pgfscope}%
\begin{pgfscope}%
\pgfpathrectangle{\pgfqpoint{1.150000in}{0.150000in}}{\pgfqpoint{5.700000in}{5.700000in}}%
\pgfusepath{clip}%
\pgfsetbuttcap%
\pgfsetroundjoin%
\definecolor{currentfill}{rgb}{0.269308,0.218818,0.509577}%
\pgfsetfillcolor{currentfill}%
\pgfsetfillopacity{0.700000}%
\pgfsetlinewidth{0.000000pt}%
\definecolor{currentstroke}{rgb}{0.000000,0.000000,0.000000}%
\pgfsetstrokecolor{currentstroke}%
\pgfsetdash{}{0pt}%
\pgfpathmoveto{\pgfqpoint{2.148778in}{2.253473in}}%
\pgfpathlineto{\pgfqpoint{2.162533in}{2.240432in}}%
\pgfpathlineto{\pgfqpoint{2.176285in}{2.227526in}}%
\pgfpathlineto{\pgfqpoint{2.190033in}{2.214756in}}%
\pgfpathlineto{\pgfqpoint{2.203779in}{2.202119in}}%
\pgfpathlineto{\pgfqpoint{2.212706in}{2.203903in}}%
\pgfpathlineto{\pgfqpoint{2.221618in}{2.205890in}}%
\pgfpathlineto{\pgfqpoint{2.230513in}{2.208074in}}%
\pgfpathlineto{\pgfqpoint{2.239394in}{2.210452in}}%
\pgfpathlineto{\pgfqpoint{2.225682in}{2.222791in}}%
\pgfpathlineto{\pgfqpoint{2.211968in}{2.235263in}}%
\pgfpathlineto{\pgfqpoint{2.198251in}{2.247870in}}%
\pgfpathlineto{\pgfqpoint{2.184531in}{2.260612in}}%
\pgfpathlineto{\pgfqpoint{2.175617in}{2.258524in}}%
\pgfpathlineto{\pgfqpoint{2.166687in}{2.256636in}}%
\pgfpathlineto{\pgfqpoint{2.157741in}{2.254950in}}%
\pgfpathlineto{\pgfqpoint{2.148778in}{2.253473in}}%
\pgfpathclose%
\pgfusepath{fill}%
\end{pgfscope}%
\begin{pgfscope}%
\pgfpathrectangle{\pgfqpoint{1.150000in}{0.150000in}}{\pgfqpoint{5.700000in}{5.700000in}}%
\pgfusepath{clip}%
\pgfsetbuttcap%
\pgfsetroundjoin%
\definecolor{currentfill}{rgb}{0.278826,0.175490,0.483397}%
\pgfsetfillcolor{currentfill}%
\pgfsetfillopacity{0.700000}%
\pgfsetlinewidth{0.000000pt}%
\definecolor{currentstroke}{rgb}{0.000000,0.000000,0.000000}%
\pgfsetstrokecolor{currentstroke}%
\pgfsetdash{}{0pt}%
\pgfpathmoveto{\pgfqpoint{4.411851in}{2.114601in}}%
\pgfpathlineto{\pgfqpoint{4.425815in}{2.115290in}}%
\pgfpathlineto{\pgfqpoint{4.439788in}{2.116051in}}%
\pgfpathlineto{\pgfqpoint{4.453771in}{2.116885in}}%
\pgfpathlineto{\pgfqpoint{4.467763in}{2.117792in}}%
\pgfpathlineto{\pgfqpoint{4.475623in}{2.126025in}}%
\pgfpathlineto{\pgfqpoint{4.483477in}{2.134201in}}%
\pgfpathlineto{\pgfqpoint{4.491324in}{2.142320in}}%
\pgfpathlineto{\pgfqpoint{4.499166in}{2.150385in}}%
\pgfpathlineto{\pgfqpoint{4.485184in}{2.149542in}}%
\pgfpathlineto{\pgfqpoint{4.471213in}{2.148772in}}%
\pgfpathlineto{\pgfqpoint{4.457251in}{2.148074in}}%
\pgfpathlineto{\pgfqpoint{4.443298in}{2.147450in}}%
\pgfpathlineto{\pgfqpoint{4.435446in}{2.139313in}}%
\pgfpathlineto{\pgfqpoint{4.427587in}{2.131127in}}%
\pgfpathlineto{\pgfqpoint{4.419722in}{2.122890in}}%
\pgfpathlineto{\pgfqpoint{4.411851in}{2.114601in}}%
\pgfpathclose%
\pgfusepath{fill}%
\end{pgfscope}%
\begin{pgfscope}%
\pgfpathrectangle{\pgfqpoint{1.150000in}{0.150000in}}{\pgfqpoint{5.700000in}{5.700000in}}%
\pgfusepath{clip}%
\pgfsetbuttcap%
\pgfsetroundjoin%
\definecolor{currentfill}{rgb}{0.225863,0.330805,0.547314}%
\pgfsetfillcolor{currentfill}%
\pgfsetfillopacity{0.700000}%
\pgfsetlinewidth{0.000000pt}%
\definecolor{currentstroke}{rgb}{0.000000,0.000000,0.000000}%
\pgfsetstrokecolor{currentstroke}%
\pgfsetdash{}{0pt}%
\pgfpathmoveto{\pgfqpoint{5.396966in}{2.465189in}}%
\pgfpathlineto{\pgfqpoint{5.411289in}{2.467675in}}%
\pgfpathlineto{\pgfqpoint{5.425624in}{2.470230in}}%
\pgfpathlineto{\pgfqpoint{5.439971in}{2.472853in}}%
\pgfpathlineto{\pgfqpoint{5.454329in}{2.475545in}}%
\pgfpathlineto{\pgfqpoint{5.461736in}{2.480282in}}%
\pgfpathlineto{\pgfqpoint{5.469137in}{2.485025in}}%
\pgfpathlineto{\pgfqpoint{5.476532in}{2.489780in}}%
\pgfpathlineto{\pgfqpoint{5.483920in}{2.494551in}}%
\pgfpathlineto{\pgfqpoint{5.469584in}{2.492135in}}%
\pgfpathlineto{\pgfqpoint{5.455259in}{2.489788in}}%
\pgfpathlineto{\pgfqpoint{5.440945in}{2.487509in}}%
\pgfpathlineto{\pgfqpoint{5.426644in}{2.485299in}}%
\pgfpathlineto{\pgfqpoint{5.419233in}{2.480245in}}%
\pgfpathlineto{\pgfqpoint{5.411817in}{2.475211in}}%
\pgfpathlineto{\pgfqpoint{5.404394in}{2.470195in}}%
\pgfpathlineto{\pgfqpoint{5.396966in}{2.465189in}}%
\pgfpathclose%
\pgfusepath{fill}%
\end{pgfscope}%
\begin{pgfscope}%
\pgfpathrectangle{\pgfqpoint{1.150000in}{0.150000in}}{\pgfqpoint{5.700000in}{5.700000in}}%
\pgfusepath{clip}%
\pgfsetbuttcap%
\pgfsetroundjoin%
\definecolor{currentfill}{rgb}{0.272594,0.025563,0.353093}%
\pgfsetfillcolor{currentfill}%
\pgfsetfillopacity{0.700000}%
\pgfsetlinewidth{0.000000pt}%
\definecolor{currentstroke}{rgb}{0.000000,0.000000,0.000000}%
\pgfsetstrokecolor{currentstroke}%
\pgfsetdash{}{0pt}%
\pgfpathmoveto{\pgfqpoint{3.514231in}{1.836064in}}%
\pgfpathlineto{\pgfqpoint{3.527952in}{1.833010in}}%
\pgfpathlineto{\pgfqpoint{3.541678in}{1.830037in}}%
\pgfpathlineto{\pgfqpoint{3.555411in}{1.827146in}}%
\pgfpathlineto{\pgfqpoint{3.569149in}{1.824336in}}%
\pgfpathlineto{\pgfqpoint{3.577336in}{1.833478in}}%
\pgfpathlineto{\pgfqpoint{3.585517in}{1.842621in}}%
\pgfpathlineto{\pgfqpoint{3.593692in}{1.851766in}}%
\pgfpathlineto{\pgfqpoint{3.601861in}{1.860909in}}%
\pgfpathlineto{\pgfqpoint{3.588134in}{1.863596in}}%
\pgfpathlineto{\pgfqpoint{3.574414in}{1.866365in}}%
\pgfpathlineto{\pgfqpoint{3.560700in}{1.869215in}}%
\pgfpathlineto{\pgfqpoint{3.546992in}{1.872147in}}%
\pgfpathlineto{\pgfqpoint{3.538811in}{1.863118in}}%
\pgfpathlineto{\pgfqpoint{3.530624in}{1.854094in}}%
\pgfpathlineto{\pgfqpoint{3.522430in}{1.845076in}}%
\pgfpathlineto{\pgfqpoint{3.514231in}{1.836064in}}%
\pgfpathclose%
\pgfusepath{fill}%
\end{pgfscope}%
\begin{pgfscope}%
\pgfpathrectangle{\pgfqpoint{1.150000in}{0.150000in}}{\pgfqpoint{5.700000in}{5.700000in}}%
\pgfusepath{clip}%
\pgfsetbuttcap%
\pgfsetroundjoin%
\definecolor{currentfill}{rgb}{0.280267,0.073417,0.397163}%
\pgfsetfillcolor{currentfill}%
\pgfsetfillopacity{0.700000}%
\pgfsetlinewidth{0.000000pt}%
\definecolor{currentstroke}{rgb}{0.000000,0.000000,0.000000}%
\pgfsetstrokecolor{currentstroke}%
\pgfsetdash{}{0pt}%
\pgfpathmoveto{\pgfqpoint{3.831958in}{1.910065in}}%
\pgfpathlineto{\pgfqpoint{3.845750in}{1.908592in}}%
\pgfpathlineto{\pgfqpoint{3.859549in}{1.907197in}}%
\pgfpathlineto{\pgfqpoint{3.873356in}{1.905880in}}%
\pgfpathlineto{\pgfqpoint{3.887170in}{1.904640in}}%
\pgfpathlineto{\pgfqpoint{3.895243in}{1.913998in}}%
\pgfpathlineto{\pgfqpoint{3.903310in}{1.923325in}}%
\pgfpathlineto{\pgfqpoint{3.911372in}{1.932620in}}%
\pgfpathlineto{\pgfqpoint{3.919428in}{1.941882in}}%
\pgfpathlineto{\pgfqpoint{3.905624in}{1.943061in}}%
\pgfpathlineto{\pgfqpoint{3.891828in}{1.944318in}}%
\pgfpathlineto{\pgfqpoint{3.878039in}{1.945651in}}%
\pgfpathlineto{\pgfqpoint{3.864258in}{1.947063in}}%
\pgfpathlineto{\pgfqpoint{3.856192in}{1.937854in}}%
\pgfpathlineto{\pgfqpoint{3.848119in}{1.928618in}}%
\pgfpathlineto{\pgfqpoint{3.840041in}{1.919354in}}%
\pgfpathlineto{\pgfqpoint{3.831958in}{1.910065in}}%
\pgfpathclose%
\pgfusepath{fill}%
\end{pgfscope}%
\begin{pgfscope}%
\pgfpathrectangle{\pgfqpoint{1.150000in}{0.150000in}}{\pgfqpoint{5.700000in}{5.700000in}}%
\pgfusepath{clip}%
\pgfsetbuttcap%
\pgfsetroundjoin%
\definecolor{currentfill}{rgb}{0.255645,0.260703,0.528312}%
\pgfsetfillcolor{currentfill}%
\pgfsetfillopacity{0.700000}%
\pgfsetlinewidth{0.000000pt}%
\definecolor{currentstroke}{rgb}{0.000000,0.000000,0.000000}%
\pgfsetstrokecolor{currentstroke}%
\pgfsetdash{}{0pt}%
\pgfpathmoveto{\pgfqpoint{4.904461in}{2.296257in}}%
\pgfpathlineto{\pgfqpoint{4.918603in}{2.298135in}}%
\pgfpathlineto{\pgfqpoint{4.932756in}{2.300084in}}%
\pgfpathlineto{\pgfqpoint{4.946920in}{2.302104in}}%
\pgfpathlineto{\pgfqpoint{4.961094in}{2.304194in}}%
\pgfpathlineto{\pgfqpoint{4.968745in}{2.310686in}}%
\pgfpathlineto{\pgfqpoint{4.976388in}{2.317137in}}%
\pgfpathlineto{\pgfqpoint{4.984025in}{2.323547in}}%
\pgfpathlineto{\pgfqpoint{4.991655in}{2.329921in}}%
\pgfpathlineto{\pgfqpoint{4.977496in}{2.328001in}}%
\pgfpathlineto{\pgfqpoint{4.963348in}{2.326152in}}%
\pgfpathlineto{\pgfqpoint{4.949210in}{2.324373in}}%
\pgfpathlineto{\pgfqpoint{4.935083in}{2.322664in}}%
\pgfpathlineto{\pgfqpoint{4.927438in}{2.316112in}}%
\pgfpathlineto{\pgfqpoint{4.919785in}{2.309529in}}%
\pgfpathlineto{\pgfqpoint{4.912127in}{2.302912in}}%
\pgfpathlineto{\pgfqpoint{4.904461in}{2.296257in}}%
\pgfpathclose%
\pgfusepath{fill}%
\end{pgfscope}%
\begin{pgfscope}%
\pgfpathrectangle{\pgfqpoint{1.150000in}{0.150000in}}{\pgfqpoint{5.700000in}{5.700000in}}%
\pgfusepath{clip}%
\pgfsetbuttcap%
\pgfsetroundjoin%
\definecolor{currentfill}{rgb}{0.269944,0.014625,0.341379}%
\pgfsetfillcolor{currentfill}%
\pgfsetfillopacity{0.700000}%
\pgfsetlinewidth{0.000000pt}%
\definecolor{currentstroke}{rgb}{0.000000,0.000000,0.000000}%
\pgfsetstrokecolor{currentstroke}%
\pgfsetdash{}{0pt}%
\pgfpathmoveto{\pgfqpoint{3.283871in}{1.811275in}}%
\pgfpathlineto{\pgfqpoint{3.297555in}{1.806908in}}%
\pgfpathlineto{\pgfqpoint{3.311244in}{1.802626in}}%
\pgfpathlineto{\pgfqpoint{3.324938in}{1.798430in}}%
\pgfpathlineto{\pgfqpoint{3.338636in}{1.794319in}}%
\pgfpathlineto{\pgfqpoint{3.346914in}{1.802863in}}%
\pgfpathlineto{\pgfqpoint{3.355185in}{1.811441in}}%
\pgfpathlineto{\pgfqpoint{3.363449in}{1.820049in}}%
\pgfpathlineto{\pgfqpoint{3.371706in}{1.828686in}}%
\pgfpathlineto{\pgfqpoint{3.358022in}{1.832633in}}%
\pgfpathlineto{\pgfqpoint{3.344343in}{1.836666in}}%
\pgfpathlineto{\pgfqpoint{3.330669in}{1.840784in}}%
\pgfpathlineto{\pgfqpoint{3.317000in}{1.844987in}}%
\pgfpathlineto{\pgfqpoint{3.308727in}{1.836507in}}%
\pgfpathlineto{\pgfqpoint{3.300449in}{1.828059in}}%
\pgfpathlineto{\pgfqpoint{3.292163in}{1.819648in}}%
\pgfpathlineto{\pgfqpoint{3.283871in}{1.811275in}}%
\pgfpathclose%
\pgfusepath{fill}%
\end{pgfscope}%
\begin{pgfscope}%
\pgfpathrectangle{\pgfqpoint{1.150000in}{0.150000in}}{\pgfqpoint{5.700000in}{5.700000in}}%
\pgfusepath{clip}%
\pgfsetbuttcap%
\pgfsetroundjoin%
\definecolor{currentfill}{rgb}{0.280868,0.160771,0.472899}%
\pgfsetfillcolor{currentfill}%
\pgfsetfillopacity{0.700000}%
\pgfsetlinewidth{0.000000pt}%
\definecolor{currentstroke}{rgb}{0.000000,0.000000,0.000000}%
\pgfsetstrokecolor{currentstroke}%
\pgfsetdash{}{0pt}%
\pgfpathmoveto{\pgfqpoint{4.324499in}{2.078701in}}%
\pgfpathlineto{\pgfqpoint{4.338437in}{2.079139in}}%
\pgfpathlineto{\pgfqpoint{4.352384in}{2.079651in}}%
\pgfpathlineto{\pgfqpoint{4.366340in}{2.080236in}}%
\pgfpathlineto{\pgfqpoint{4.380305in}{2.080894in}}%
\pgfpathlineto{\pgfqpoint{4.388201in}{2.089406in}}%
\pgfpathlineto{\pgfqpoint{4.396090in}{2.097860in}}%
\pgfpathlineto{\pgfqpoint{4.403974in}{2.106258in}}%
\pgfpathlineto{\pgfqpoint{4.411851in}{2.114601in}}%
\pgfpathlineto{\pgfqpoint{4.397896in}{2.113986in}}%
\pgfpathlineto{\pgfqpoint{4.383951in}{2.113444in}}%
\pgfpathlineto{\pgfqpoint{4.370015in}{2.112976in}}%
\pgfpathlineto{\pgfqpoint{4.356088in}{2.112580in}}%
\pgfpathlineto{\pgfqpoint{4.348199in}{2.104187in}}%
\pgfpathlineto{\pgfqpoint{4.340305in}{2.095743in}}%
\pgfpathlineto{\pgfqpoint{4.332405in}{2.087248in}}%
\pgfpathlineto{\pgfqpoint{4.324499in}{2.078701in}}%
\pgfpathclose%
\pgfusepath{fill}%
\end{pgfscope}%
\begin{pgfscope}%
\pgfpathrectangle{\pgfqpoint{1.150000in}{0.150000in}}{\pgfqpoint{5.700000in}{5.700000in}}%
\pgfusepath{clip}%
\pgfsetbuttcap%
\pgfsetroundjoin%
\definecolor{currentfill}{rgb}{0.274128,0.199721,0.498911}%
\pgfsetfillcolor{currentfill}%
\pgfsetfillopacity{0.700000}%
\pgfsetlinewidth{0.000000pt}%
\definecolor{currentstroke}{rgb}{0.000000,0.000000,0.000000}%
\pgfsetstrokecolor{currentstroke}%
\pgfsetdash{}{0pt}%
\pgfpathmoveto{\pgfqpoint{2.203779in}{2.202119in}}%
\pgfpathlineto{\pgfqpoint{2.217521in}{2.189615in}}%
\pgfpathlineto{\pgfqpoint{2.231261in}{2.177242in}}%
\pgfpathlineto{\pgfqpoint{2.244997in}{2.165000in}}%
\pgfpathlineto{\pgfqpoint{2.258732in}{2.152887in}}%
\pgfpathlineto{\pgfqpoint{2.267625in}{2.154976in}}%
\pgfpathlineto{\pgfqpoint{2.276503in}{2.157263in}}%
\pgfpathlineto{\pgfqpoint{2.285365in}{2.159742in}}%
\pgfpathlineto{\pgfqpoint{2.294212in}{2.162410in}}%
\pgfpathlineto{\pgfqpoint{2.280511in}{2.174226in}}%
\pgfpathlineto{\pgfqpoint{2.266808in}{2.186171in}}%
\pgfpathlineto{\pgfqpoint{2.253102in}{2.198246in}}%
\pgfpathlineto{\pgfqpoint{2.239394in}{2.210452in}}%
\pgfpathlineto{\pgfqpoint{2.230513in}{2.208074in}}%
\pgfpathlineto{\pgfqpoint{2.221618in}{2.205890in}}%
\pgfpathlineto{\pgfqpoint{2.212706in}{2.203903in}}%
\pgfpathlineto{\pgfqpoint{2.203779in}{2.202119in}}%
\pgfpathclose%
\pgfusepath{fill}%
\end{pgfscope}%
\begin{pgfscope}%
\pgfpathrectangle{\pgfqpoint{1.150000in}{0.150000in}}{\pgfqpoint{5.700000in}{5.700000in}}%
\pgfusepath{clip}%
\pgfsetbuttcap%
\pgfsetroundjoin%
\definecolor{currentfill}{rgb}{0.277941,0.056324,0.381191}%
\pgfsetfillcolor{currentfill}%
\pgfsetfillopacity{0.700000}%
\pgfsetlinewidth{0.000000pt}%
\definecolor{currentstroke}{rgb}{0.000000,0.000000,0.000000}%
\pgfsetstrokecolor{currentstroke}%
\pgfsetdash{}{0pt}%
\pgfpathmoveto{\pgfqpoint{2.711865in}{1.899927in}}%
\pgfpathlineto{\pgfqpoint{2.725527in}{1.891704in}}%
\pgfpathlineto{\pgfqpoint{2.739190in}{1.883582in}}%
\pgfpathlineto{\pgfqpoint{2.752855in}{1.875562in}}%
\pgfpathlineto{\pgfqpoint{2.766522in}{1.867644in}}%
\pgfpathlineto{\pgfqpoint{2.775083in}{1.873255in}}%
\pgfpathlineto{\pgfqpoint{2.783634in}{1.878988in}}%
\pgfpathlineto{\pgfqpoint{2.792175in}{1.884838in}}%
\pgfpathlineto{\pgfqpoint{2.800706in}{1.890804in}}%
\pgfpathlineto{\pgfqpoint{2.787063in}{1.898474in}}%
\pgfpathlineto{\pgfqpoint{2.773421in}{1.906246in}}%
\pgfpathlineto{\pgfqpoint{2.759782in}{1.914120in}}%
\pgfpathlineto{\pgfqpoint{2.746144in}{1.922096in}}%
\pgfpathlineto{\pgfqpoint{2.737590in}{1.916370in}}%
\pgfpathlineto{\pgfqpoint{2.729026in}{1.910765in}}%
\pgfpathlineto{\pgfqpoint{2.720451in}{1.905283in}}%
\pgfpathlineto{\pgfqpoint{2.711865in}{1.899927in}}%
\pgfpathclose%
\pgfusepath{fill}%
\end{pgfscope}%
\begin{pgfscope}%
\pgfpathrectangle{\pgfqpoint{1.150000in}{0.150000in}}{\pgfqpoint{5.700000in}{5.700000in}}%
\pgfusepath{clip}%
\pgfsetbuttcap%
\pgfsetroundjoin%
\definecolor{currentfill}{rgb}{0.282910,0.105393,0.426902}%
\pgfsetfillcolor{currentfill}%
\pgfsetfillopacity{0.700000}%
\pgfsetlinewidth{0.000000pt}%
\definecolor{currentstroke}{rgb}{0.000000,0.000000,0.000000}%
\pgfsetstrokecolor{currentstroke}%
\pgfsetdash{}{0pt}%
\pgfpathmoveto{\pgfqpoint{2.513219in}{1.990028in}}%
\pgfpathlineto{\pgfqpoint{2.526900in}{1.980247in}}%
\pgfpathlineto{\pgfqpoint{2.540581in}{1.970579in}}%
\pgfpathlineto{\pgfqpoint{2.554263in}{1.961020in}}%
\pgfpathlineto{\pgfqpoint{2.567944in}{1.951572in}}%
\pgfpathlineto{\pgfqpoint{2.576628in}{1.955826in}}%
\pgfpathlineto{\pgfqpoint{2.585300in}{1.960233in}}%
\pgfpathlineto{\pgfqpoint{2.593959in}{1.964789in}}%
\pgfpathlineto{\pgfqpoint{2.602607in}{1.969490in}}%
\pgfpathlineto{\pgfqpoint{2.588953in}{1.978668in}}%
\pgfpathlineto{\pgfqpoint{2.575299in}{1.987955in}}%
\pgfpathlineto{\pgfqpoint{2.561646in}{1.997353in}}%
\pgfpathlineto{\pgfqpoint{2.547992in}{2.006863in}}%
\pgfpathlineto{\pgfqpoint{2.539318in}{2.002425in}}%
\pgfpathlineto{\pgfqpoint{2.530631in}{1.998137in}}%
\pgfpathlineto{\pgfqpoint{2.521931in}{1.994003in}}%
\pgfpathlineto{\pgfqpoint{2.513219in}{1.990028in}}%
\pgfpathclose%
\pgfusepath{fill}%
\end{pgfscope}%
\begin{pgfscope}%
\pgfpathrectangle{\pgfqpoint{1.150000in}{0.150000in}}{\pgfqpoint{5.700000in}{5.700000in}}%
\pgfusepath{clip}%
\pgfsetbuttcap%
\pgfsetroundjoin%
\definecolor{currentfill}{rgb}{0.282290,0.145912,0.461510}%
\pgfsetfillcolor{currentfill}%
\pgfsetfillopacity{0.700000}%
\pgfsetlinewidth{0.000000pt}%
\definecolor{currentstroke}{rgb}{0.000000,0.000000,0.000000}%
\pgfsetstrokecolor{currentstroke}%
\pgfsetdash{}{0pt}%
\pgfpathmoveto{\pgfqpoint{4.237110in}{2.042868in}}%
\pgfpathlineto{\pgfqpoint{4.251023in}{2.043033in}}%
\pgfpathlineto{\pgfqpoint{4.264944in}{2.043271in}}%
\pgfpathlineto{\pgfqpoint{4.278875in}{2.043584in}}%
\pgfpathlineto{\pgfqpoint{4.292814in}{2.043971in}}%
\pgfpathlineto{\pgfqpoint{4.300744in}{2.052736in}}%
\pgfpathlineto{\pgfqpoint{4.308668in}{2.061446in}}%
\pgfpathlineto{\pgfqpoint{4.316587in}{2.070101in}}%
\pgfpathlineto{\pgfqpoint{4.324499in}{2.078701in}}%
\pgfpathlineto{\pgfqpoint{4.310570in}{2.078337in}}%
\pgfpathlineto{\pgfqpoint{4.296650in}{2.078046in}}%
\pgfpathlineto{\pgfqpoint{4.282739in}{2.077830in}}%
\pgfpathlineto{\pgfqpoint{4.268837in}{2.077687in}}%
\pgfpathlineto{\pgfqpoint{4.260914in}{2.069057in}}%
\pgfpathlineto{\pgfqpoint{4.252986in}{2.060378in}}%
\pgfpathlineto{\pgfqpoint{4.245051in}{2.051648in}}%
\pgfpathlineto{\pgfqpoint{4.237110in}{2.042868in}}%
\pgfpathclose%
\pgfusepath{fill}%
\end{pgfscope}%
\begin{pgfscope}%
\pgfpathrectangle{\pgfqpoint{1.150000in}{0.150000in}}{\pgfqpoint{5.700000in}{5.700000in}}%
\pgfusepath{clip}%
\pgfsetbuttcap%
\pgfsetroundjoin%
\definecolor{currentfill}{rgb}{0.260571,0.246922,0.522828}%
\pgfsetfillcolor{currentfill}%
\pgfsetfillopacity{0.700000}%
\pgfsetlinewidth{0.000000pt}%
\definecolor{currentstroke}{rgb}{0.000000,0.000000,0.000000}%
\pgfsetstrokecolor{currentstroke}%
\pgfsetdash{}{0pt}%
\pgfpathmoveto{\pgfqpoint{4.817213in}{2.261796in}}%
\pgfpathlineto{\pgfqpoint{4.831327in}{2.263540in}}%
\pgfpathlineto{\pgfqpoint{4.845452in}{2.265355in}}%
\pgfpathlineto{\pgfqpoint{4.859587in}{2.267241in}}%
\pgfpathlineto{\pgfqpoint{4.873732in}{2.269198in}}%
\pgfpathlineto{\pgfqpoint{4.881425in}{2.276034in}}%
\pgfpathlineto{\pgfqpoint{4.889110in}{2.282821in}}%
\pgfpathlineto{\pgfqpoint{4.896789in}{2.289561in}}%
\pgfpathlineto{\pgfqpoint{4.904461in}{2.296257in}}%
\pgfpathlineto{\pgfqpoint{4.890330in}{2.294449in}}%
\pgfpathlineto{\pgfqpoint{4.876209in}{2.292711in}}%
\pgfpathlineto{\pgfqpoint{4.862099in}{2.291045in}}%
\pgfpathlineto{\pgfqpoint{4.847999in}{2.289449in}}%
\pgfpathlineto{\pgfqpoint{4.840313in}{2.282597in}}%
\pgfpathlineto{\pgfqpoint{4.832619in}{2.275706in}}%
\pgfpathlineto{\pgfqpoint{4.824920in}{2.268773in}}%
\pgfpathlineto{\pgfqpoint{4.817213in}{2.261796in}}%
\pgfpathclose%
\pgfusepath{fill}%
\end{pgfscope}%
\begin{pgfscope}%
\pgfpathrectangle{\pgfqpoint{1.150000in}{0.150000in}}{\pgfqpoint{5.700000in}{5.700000in}}%
\pgfusepath{clip}%
\pgfsetbuttcap%
\pgfsetroundjoin%
\definecolor{currentfill}{rgb}{0.231674,0.318106,0.544834}%
\pgfsetfillcolor{currentfill}%
\pgfsetfillopacity{0.700000}%
\pgfsetlinewidth{0.000000pt}%
\definecolor{currentstroke}{rgb}{0.000000,0.000000,0.000000}%
\pgfsetstrokecolor{currentstroke}%
\pgfsetdash{}{0pt}%
\pgfpathmoveto{\pgfqpoint{5.309933in}{2.434911in}}%
\pgfpathlineto{\pgfqpoint{5.324230in}{2.437377in}}%
\pgfpathlineto{\pgfqpoint{5.338538in}{2.439912in}}%
\pgfpathlineto{\pgfqpoint{5.352858in}{2.442516in}}%
\pgfpathlineto{\pgfqpoint{5.367190in}{2.445188in}}%
\pgfpathlineto{\pgfqpoint{5.374644in}{2.450195in}}%
\pgfpathlineto{\pgfqpoint{5.382091in}{2.455194in}}%
\pgfpathlineto{\pgfqpoint{5.389532in}{2.460191in}}%
\pgfpathlineto{\pgfqpoint{5.396966in}{2.465189in}}%
\pgfpathlineto{\pgfqpoint{5.382655in}{2.462772in}}%
\pgfpathlineto{\pgfqpoint{5.368355in}{2.460423in}}%
\pgfpathlineto{\pgfqpoint{5.354067in}{2.458144in}}%
\pgfpathlineto{\pgfqpoint{5.339790in}{2.455933in}}%
\pgfpathlineto{\pgfqpoint{5.332335in}{2.450671in}}%
\pgfpathlineto{\pgfqpoint{5.324874in}{2.445417in}}%
\pgfpathlineto{\pgfqpoint{5.317406in}{2.440165in}}%
\pgfpathlineto{\pgfqpoint{5.309933in}{2.434911in}}%
\pgfpathclose%
\pgfusepath{fill}%
\end{pgfscope}%
\begin{pgfscope}%
\pgfpathrectangle{\pgfqpoint{1.150000in}{0.150000in}}{\pgfqpoint{5.700000in}{5.700000in}}%
\pgfusepath{clip}%
\pgfsetbuttcap%
\pgfsetroundjoin%
\definecolor{currentfill}{rgb}{0.206756,0.371758,0.553117}%
\pgfsetfillcolor{currentfill}%
\pgfsetfillopacity{0.700000}%
\pgfsetlinewidth{0.000000pt}%
\definecolor{currentstroke}{rgb}{0.000000,0.000000,0.000000}%
\pgfsetstrokecolor{currentstroke}%
\pgfsetdash{}{0pt}%
\pgfpathmoveto{\pgfqpoint{5.715243in}{2.560857in}}%
\pgfpathlineto{\pgfqpoint{5.729688in}{2.563542in}}%
\pgfpathlineto{\pgfqpoint{5.744145in}{2.566294in}}%
\pgfpathlineto{\pgfqpoint{5.758615in}{2.569114in}}%
\pgfpathlineto{\pgfqpoint{5.773097in}{2.572002in}}%
\pgfpathlineto{\pgfqpoint{5.780340in}{2.575834in}}%
\pgfpathlineto{\pgfqpoint{5.787578in}{2.579717in}}%
\pgfpathlineto{\pgfqpoint{5.794811in}{2.583656in}}%
\pgfpathlineto{\pgfqpoint{5.802040in}{2.587659in}}%
\pgfpathlineto{\pgfqpoint{5.787585in}{2.585111in}}%
\pgfpathlineto{\pgfqpoint{5.773142in}{2.582631in}}%
\pgfpathlineto{\pgfqpoint{5.758711in}{2.580218in}}%
\pgfpathlineto{\pgfqpoint{5.744292in}{2.577873in}}%
\pgfpathlineto{\pgfqpoint{5.737036in}{2.573523in}}%
\pgfpathlineto{\pgfqpoint{5.729776in}{2.569242in}}%
\pgfpathlineto{\pgfqpoint{5.722512in}{2.565021in}}%
\pgfpathlineto{\pgfqpoint{5.715243in}{2.560857in}}%
\pgfpathclose%
\pgfusepath{fill}%
\end{pgfscope}%
\begin{pgfscope}%
\pgfpathrectangle{\pgfqpoint{1.150000in}{0.150000in}}{\pgfqpoint{5.700000in}{5.700000in}}%
\pgfusepath{clip}%
\pgfsetbuttcap%
\pgfsetroundjoin%
\definecolor{currentfill}{rgb}{0.278791,0.062145,0.386592}%
\pgfsetfillcolor{currentfill}%
\pgfsetfillopacity{0.700000}%
\pgfsetlinewidth{0.000000pt}%
\definecolor{currentstroke}{rgb}{0.000000,0.000000,0.000000}%
\pgfsetstrokecolor{currentstroke}%
\pgfsetdash{}{0pt}%
\pgfpathmoveto{\pgfqpoint{3.744428in}{1.879654in}}%
\pgfpathlineto{\pgfqpoint{3.758203in}{1.877787in}}%
\pgfpathlineto{\pgfqpoint{3.771984in}{1.875999in}}%
\pgfpathlineto{\pgfqpoint{3.785772in}{1.874289in}}%
\pgfpathlineto{\pgfqpoint{3.799567in}{1.872657in}}%
\pgfpathlineto{\pgfqpoint{3.807673in}{1.882045in}}%
\pgfpathlineto{\pgfqpoint{3.815774in}{1.891409in}}%
\pgfpathlineto{\pgfqpoint{3.823869in}{1.900749in}}%
\pgfpathlineto{\pgfqpoint{3.831958in}{1.910065in}}%
\pgfpathlineto{\pgfqpoint{3.818173in}{1.911615in}}%
\pgfpathlineto{\pgfqpoint{3.804396in}{1.913244in}}%
\pgfpathlineto{\pgfqpoint{3.790626in}{1.914950in}}%
\pgfpathlineto{\pgfqpoint{3.776862in}{1.916736in}}%
\pgfpathlineto{\pgfqpoint{3.768762in}{1.907494in}}%
\pgfpathlineto{\pgfqpoint{3.760657in}{1.898233in}}%
\pgfpathlineto{\pgfqpoint{3.752545in}{1.888953in}}%
\pgfpathlineto{\pgfqpoint{3.744428in}{1.879654in}}%
\pgfpathclose%
\pgfusepath{fill}%
\end{pgfscope}%
\begin{pgfscope}%
\pgfpathrectangle{\pgfqpoint{1.150000in}{0.150000in}}{\pgfqpoint{5.700000in}{5.700000in}}%
\pgfusepath{clip}%
\pgfsetbuttcap%
\pgfsetroundjoin%
\definecolor{currentfill}{rgb}{0.278012,0.180367,0.486697}%
\pgfsetfillcolor{currentfill}%
\pgfsetfillopacity{0.700000}%
\pgfsetlinewidth{0.000000pt}%
\definecolor{currentstroke}{rgb}{0.000000,0.000000,0.000000}%
\pgfsetstrokecolor{currentstroke}%
\pgfsetdash{}{0pt}%
\pgfpathmoveto{\pgfqpoint{2.258732in}{2.152887in}}%
\pgfpathlineto{\pgfqpoint{2.272464in}{2.140902in}}%
\pgfpathlineto{\pgfqpoint{2.286193in}{2.129044in}}%
\pgfpathlineto{\pgfqpoint{2.299920in}{2.117312in}}%
\pgfpathlineto{\pgfqpoint{2.313646in}{2.105704in}}%
\pgfpathlineto{\pgfqpoint{2.322506in}{2.108098in}}%
\pgfpathlineto{\pgfqpoint{2.331350in}{2.110684in}}%
\pgfpathlineto{\pgfqpoint{2.340181in}{2.113456in}}%
\pgfpathlineto{\pgfqpoint{2.348996in}{2.116411in}}%
\pgfpathlineto{\pgfqpoint{2.335303in}{2.127723in}}%
\pgfpathlineto{\pgfqpoint{2.321608in}{2.139159in}}%
\pgfpathlineto{\pgfqpoint{2.307911in}{2.150721in}}%
\pgfpathlineto{\pgfqpoint{2.294212in}{2.162410in}}%
\pgfpathlineto{\pgfqpoint{2.285365in}{2.159742in}}%
\pgfpathlineto{\pgfqpoint{2.276503in}{2.157263in}}%
\pgfpathlineto{\pgfqpoint{2.267625in}{2.154976in}}%
\pgfpathlineto{\pgfqpoint{2.258732in}{2.152887in}}%
\pgfpathclose%
\pgfusepath{fill}%
\end{pgfscope}%
\begin{pgfscope}%
\pgfpathrectangle{\pgfqpoint{1.150000in}{0.150000in}}{\pgfqpoint{5.700000in}{5.700000in}}%
\pgfusepath{clip}%
\pgfsetbuttcap%
\pgfsetroundjoin%
\definecolor{currentfill}{rgb}{0.271305,0.019942,0.347269}%
\pgfsetfillcolor{currentfill}%
\pgfsetfillopacity{0.700000}%
\pgfsetlinewidth{0.000000pt}%
\definecolor{currentstroke}{rgb}{0.000000,0.000000,0.000000}%
\pgfsetstrokecolor{currentstroke}%
\pgfsetdash{}{0pt}%
\pgfpathmoveto{\pgfqpoint{3.426496in}{1.813740in}}%
\pgfpathlineto{\pgfqpoint{3.440207in}{1.810212in}}%
\pgfpathlineto{\pgfqpoint{3.453923in}{1.806768in}}%
\pgfpathlineto{\pgfqpoint{3.467645in}{1.803406in}}%
\pgfpathlineto{\pgfqpoint{3.481373in}{1.800126in}}%
\pgfpathlineto{\pgfqpoint{3.489597in}{1.809091in}}%
\pgfpathlineto{\pgfqpoint{3.497814in}{1.818070in}}%
\pgfpathlineto{\pgfqpoint{3.506026in}{1.827062in}}%
\pgfpathlineto{\pgfqpoint{3.514231in}{1.836064in}}%
\pgfpathlineto{\pgfqpoint{3.500516in}{1.839201in}}%
\pgfpathlineto{\pgfqpoint{3.486808in}{1.842419in}}%
\pgfpathlineto{\pgfqpoint{3.473104in}{1.845721in}}%
\pgfpathlineto{\pgfqpoint{3.459407in}{1.849105in}}%
\pgfpathlineto{\pgfqpoint{3.451189in}{1.840238in}}%
\pgfpathlineto{\pgfqpoint{3.442964in}{1.831387in}}%
\pgfpathlineto{\pgfqpoint{3.434733in}{1.822554in}}%
\pgfpathlineto{\pgfqpoint{3.426496in}{1.813740in}}%
\pgfpathclose%
\pgfusepath{fill}%
\end{pgfscope}%
\begin{pgfscope}%
\pgfpathrectangle{\pgfqpoint{1.150000in}{0.150000in}}{\pgfqpoint{5.700000in}{5.700000in}}%
\pgfusepath{clip}%
\pgfsetbuttcap%
\pgfsetroundjoin%
\definecolor{currentfill}{rgb}{0.283187,0.125848,0.444960}%
\pgfsetfillcolor{currentfill}%
\pgfsetfillopacity{0.700000}%
\pgfsetlinewidth{0.000000pt}%
\definecolor{currentstroke}{rgb}{0.000000,0.000000,0.000000}%
\pgfsetstrokecolor{currentstroke}%
\pgfsetdash{}{0pt}%
\pgfpathmoveto{\pgfqpoint{4.149686in}{2.007308in}}%
\pgfpathlineto{\pgfqpoint{4.163574in}{2.007176in}}%
\pgfpathlineto{\pgfqpoint{4.177470in}{2.007118in}}%
\pgfpathlineto{\pgfqpoint{4.191376in}{2.007135in}}%
\pgfpathlineto{\pgfqpoint{4.205290in}{2.007227in}}%
\pgfpathlineto{\pgfqpoint{4.213254in}{2.016216in}}%
\pgfpathlineto{\pgfqpoint{4.221212in}{2.025153in}}%
\pgfpathlineto{\pgfqpoint{4.229164in}{2.034037in}}%
\pgfpathlineto{\pgfqpoint{4.237110in}{2.042868in}}%
\pgfpathlineto{\pgfqpoint{4.223207in}{2.042778in}}%
\pgfpathlineto{\pgfqpoint{4.209312in}{2.042762in}}%
\pgfpathlineto{\pgfqpoint{4.195425in}{2.042821in}}%
\pgfpathlineto{\pgfqpoint{4.181548in}{2.042954in}}%
\pgfpathlineto{\pgfqpoint{4.173591in}{2.034114in}}%
\pgfpathlineto{\pgfqpoint{4.165628in}{2.025226in}}%
\pgfpathlineto{\pgfqpoint{4.157660in}{2.016291in}}%
\pgfpathlineto{\pgfqpoint{4.149686in}{2.007308in}}%
\pgfpathclose%
\pgfusepath{fill}%
\end{pgfscope}%
\begin{pgfscope}%
\pgfpathrectangle{\pgfqpoint{1.150000in}{0.150000in}}{\pgfqpoint{5.700000in}{5.700000in}}%
\pgfusepath{clip}%
\pgfsetbuttcap%
\pgfsetroundjoin%
\definecolor{currentfill}{rgb}{0.269944,0.014625,0.341379}%
\pgfsetfillcolor{currentfill}%
\pgfsetfillopacity{0.700000}%
\pgfsetlinewidth{0.000000pt}%
\definecolor{currentstroke}{rgb}{0.000000,0.000000,0.000000}%
\pgfsetstrokecolor{currentstroke}%
\pgfsetdash{}{0pt}%
\pgfpathmoveto{\pgfqpoint{3.053007in}{1.809870in}}%
\pgfpathlineto{\pgfqpoint{3.066675in}{1.804052in}}%
\pgfpathlineto{\pgfqpoint{3.080347in}{1.798325in}}%
\pgfpathlineto{\pgfqpoint{3.094022in}{1.792688in}}%
\pgfpathlineto{\pgfqpoint{3.107701in}{1.787142in}}%
\pgfpathlineto{\pgfqpoint{3.116085in}{1.794695in}}%
\pgfpathlineto{\pgfqpoint{3.124462in}{1.802317in}}%
\pgfpathlineto{\pgfqpoint{3.132831in}{1.810006in}}%
\pgfpathlineto{\pgfqpoint{3.141192in}{1.817758in}}%
\pgfpathlineto{\pgfqpoint{3.127531in}{1.823100in}}%
\pgfpathlineto{\pgfqpoint{3.113873in}{1.828532in}}%
\pgfpathlineto{\pgfqpoint{3.100220in}{1.834054in}}%
\pgfpathlineto{\pgfqpoint{3.086570in}{1.839667in}}%
\pgfpathlineto{\pgfqpoint{3.078191in}{1.832112in}}%
\pgfpathlineto{\pgfqpoint{3.069804in}{1.824626in}}%
\pgfpathlineto{\pgfqpoint{3.061410in}{1.817211in}}%
\pgfpathlineto{\pgfqpoint{3.053007in}{1.809870in}}%
\pgfpathclose%
\pgfusepath{fill}%
\end{pgfscope}%
\begin{pgfscope}%
\pgfpathrectangle{\pgfqpoint{1.150000in}{0.150000in}}{\pgfqpoint{5.700000in}{5.700000in}}%
\pgfusepath{clip}%
\pgfsetbuttcap%
\pgfsetroundjoin%
\definecolor{currentfill}{rgb}{0.265145,0.232956,0.516599}%
\pgfsetfillcolor{currentfill}%
\pgfsetfillopacity{0.700000}%
\pgfsetlinewidth{0.000000pt}%
\definecolor{currentstroke}{rgb}{0.000000,0.000000,0.000000}%
\pgfsetstrokecolor{currentstroke}%
\pgfsetdash{}{0pt}%
\pgfpathmoveto{\pgfqpoint{4.729915in}{2.226613in}}%
\pgfpathlineto{\pgfqpoint{4.744001in}{2.228200in}}%
\pgfpathlineto{\pgfqpoint{4.758097in}{2.229859in}}%
\pgfpathlineto{\pgfqpoint{4.772203in}{2.231589in}}%
\pgfpathlineto{\pgfqpoint{4.786320in}{2.233390in}}%
\pgfpathlineto{\pgfqpoint{4.794054in}{2.240571in}}%
\pgfpathlineto{\pgfqpoint{4.801780in}{2.247698in}}%
\pgfpathlineto{\pgfqpoint{4.809500in}{2.254772in}}%
\pgfpathlineto{\pgfqpoint{4.817213in}{2.261796in}}%
\pgfpathlineto{\pgfqpoint{4.803110in}{2.260123in}}%
\pgfpathlineto{\pgfqpoint{4.789017in}{2.258520in}}%
\pgfpathlineto{\pgfqpoint{4.774934in}{2.256989in}}%
\pgfpathlineto{\pgfqpoint{4.760862in}{2.255529in}}%
\pgfpathlineto{\pgfqpoint{4.753135in}{2.248370in}}%
\pgfpathlineto{\pgfqpoint{4.745401in}{2.241165in}}%
\pgfpathlineto{\pgfqpoint{4.737661in}{2.233914in}}%
\pgfpathlineto{\pgfqpoint{4.729915in}{2.226613in}}%
\pgfpathclose%
\pgfusepath{fill}%
\end{pgfscope}%
\begin{pgfscope}%
\pgfpathrectangle{\pgfqpoint{1.150000in}{0.150000in}}{\pgfqpoint{5.700000in}{5.700000in}}%
\pgfusepath{clip}%
\pgfsetbuttcap%
\pgfsetroundjoin%
\definecolor{currentfill}{rgb}{0.272594,0.025563,0.353093}%
\pgfsetfillcolor{currentfill}%
\pgfsetfillopacity{0.700000}%
\pgfsetlinewidth{0.000000pt}%
\definecolor{currentstroke}{rgb}{0.000000,0.000000,0.000000}%
\pgfsetstrokecolor{currentstroke}%
\pgfsetdash{}{0pt}%
\pgfpathmoveto{\pgfqpoint{2.909925in}{1.832998in}}%
\pgfpathlineto{\pgfqpoint{2.923589in}{1.826208in}}%
\pgfpathlineto{\pgfqpoint{2.937254in}{1.819513in}}%
\pgfpathlineto{\pgfqpoint{2.950923in}{1.812914in}}%
\pgfpathlineto{\pgfqpoint{2.964595in}{1.806408in}}%
\pgfpathlineto{\pgfqpoint{2.973051in}{1.813186in}}%
\pgfpathlineto{\pgfqpoint{2.981499in}{1.820057in}}%
\pgfpathlineto{\pgfqpoint{2.989938in}{1.827016in}}%
\pgfpathlineto{\pgfqpoint{2.998368in}{1.834062in}}%
\pgfpathlineto{\pgfqpoint{2.984717in}{1.840342in}}%
\pgfpathlineto{\pgfqpoint{2.971068in}{1.846716in}}%
\pgfpathlineto{\pgfqpoint{2.957423in}{1.853184in}}%
\pgfpathlineto{\pgfqpoint{2.943780in}{1.859747in}}%
\pgfpathlineto{\pgfqpoint{2.935330in}{1.852921in}}%
\pgfpathlineto{\pgfqpoint{2.926871in}{1.846184in}}%
\pgfpathlineto{\pgfqpoint{2.918403in}{1.839542in}}%
\pgfpathlineto{\pgfqpoint{2.909925in}{1.832998in}}%
\pgfpathclose%
\pgfusepath{fill}%
\end{pgfscope}%
\begin{pgfscope}%
\pgfpathrectangle{\pgfqpoint{1.150000in}{0.150000in}}{\pgfqpoint{5.700000in}{5.700000in}}%
\pgfusepath{clip}%
\pgfsetbuttcap%
\pgfsetroundjoin%
\definecolor{currentfill}{rgb}{0.235526,0.309527,0.542944}%
\pgfsetfillcolor{currentfill}%
\pgfsetfillopacity{0.700000}%
\pgfsetlinewidth{0.000000pt}%
\definecolor{currentstroke}{rgb}{0.000000,0.000000,0.000000}%
\pgfsetstrokecolor{currentstroke}%
\pgfsetdash{}{0pt}%
\pgfpathmoveto{\pgfqpoint{5.222825in}{2.403678in}}%
\pgfpathlineto{\pgfqpoint{5.237095in}{2.406101in}}%
\pgfpathlineto{\pgfqpoint{5.251376in}{2.408594in}}%
\pgfpathlineto{\pgfqpoint{5.265669in}{2.411156in}}%
\pgfpathlineto{\pgfqpoint{5.279973in}{2.413787in}}%
\pgfpathlineto{\pgfqpoint{5.287473in}{2.419093in}}%
\pgfpathlineto{\pgfqpoint{5.294966in}{2.424379in}}%
\pgfpathlineto{\pgfqpoint{5.302453in}{2.429651in}}%
\pgfpathlineto{\pgfqpoint{5.309933in}{2.434911in}}%
\pgfpathlineto{\pgfqpoint{5.295648in}{2.432515in}}%
\pgfpathlineto{\pgfqpoint{5.281374in}{2.430187in}}%
\pgfpathlineto{\pgfqpoint{5.267111in}{2.427929in}}%
\pgfpathlineto{\pgfqpoint{5.252861in}{2.425739in}}%
\pgfpathlineto{\pgfqpoint{5.245361in}{2.420237in}}%
\pgfpathlineto{\pgfqpoint{5.237855in}{2.414729in}}%
\pgfpathlineto{\pgfqpoint{5.230343in}{2.409211in}}%
\pgfpathlineto{\pgfqpoint{5.222825in}{2.403678in}}%
\pgfpathclose%
\pgfusepath{fill}%
\end{pgfscope}%
\begin{pgfscope}%
\pgfpathrectangle{\pgfqpoint{1.150000in}{0.150000in}}{\pgfqpoint{5.700000in}{5.700000in}}%
\pgfusepath{clip}%
\pgfsetbuttcap%
\pgfsetroundjoin%
\definecolor{currentfill}{rgb}{0.268510,0.009605,0.335427}%
\pgfsetfillcolor{currentfill}%
\pgfsetfillopacity{0.700000}%
\pgfsetlinewidth{0.000000pt}%
\definecolor{currentstroke}{rgb}{0.000000,0.000000,0.000000}%
\pgfsetstrokecolor{currentstroke}%
\pgfsetdash{}{0pt}%
\pgfpathmoveto{\pgfqpoint{3.195877in}{1.797282in}}%
\pgfpathlineto{\pgfqpoint{3.209559in}{1.792383in}}%
\pgfpathlineto{\pgfqpoint{3.223245in}{1.787572in}}%
\pgfpathlineto{\pgfqpoint{3.236936in}{1.782848in}}%
\pgfpathlineto{\pgfqpoint{3.250631in}{1.778210in}}%
\pgfpathlineto{\pgfqpoint{3.258952in}{1.786407in}}%
\pgfpathlineto{\pgfqpoint{3.267265in}{1.794652in}}%
\pgfpathlineto{\pgfqpoint{3.275571in}{1.802942in}}%
\pgfpathlineto{\pgfqpoint{3.283871in}{1.811275in}}%
\pgfpathlineto{\pgfqpoint{3.270191in}{1.815728in}}%
\pgfpathlineto{\pgfqpoint{3.256516in}{1.820268in}}%
\pgfpathlineto{\pgfqpoint{3.242846in}{1.824895in}}%
\pgfpathlineto{\pgfqpoint{3.229181in}{1.829610in}}%
\pgfpathlineto{\pgfqpoint{3.220866in}{1.821453in}}%
\pgfpathlineto{\pgfqpoint{3.212543in}{1.813345in}}%
\pgfpathlineto{\pgfqpoint{3.204214in}{1.805287in}}%
\pgfpathlineto{\pgfqpoint{3.195877in}{1.797282in}}%
\pgfpathclose%
\pgfusepath{fill}%
\end{pgfscope}%
\begin{pgfscope}%
\pgfpathrectangle{\pgfqpoint{1.150000in}{0.150000in}}{\pgfqpoint{5.700000in}{5.700000in}}%
\pgfusepath{clip}%
\pgfsetbuttcap%
\pgfsetroundjoin%
\definecolor{currentfill}{rgb}{0.276022,0.044167,0.370164}%
\pgfsetfillcolor{currentfill}%
\pgfsetfillopacity{0.700000}%
\pgfsetlinewidth{0.000000pt}%
\definecolor{currentstroke}{rgb}{0.000000,0.000000,0.000000}%
\pgfsetstrokecolor{currentstroke}%
\pgfsetdash{}{0pt}%
\pgfpathmoveto{\pgfqpoint{3.656830in}{1.850966in}}%
\pgfpathlineto{\pgfqpoint{3.670588in}{1.848680in}}%
\pgfpathlineto{\pgfqpoint{3.684353in}{1.846474in}}%
\pgfpathlineto{\pgfqpoint{3.698125in}{1.844347in}}%
\pgfpathlineto{\pgfqpoint{3.711904in}{1.842299in}}%
\pgfpathlineto{\pgfqpoint{3.720043in}{1.851660in}}%
\pgfpathlineto{\pgfqpoint{3.728177in}{1.861007in}}%
\pgfpathlineto{\pgfqpoint{3.736306in}{1.870338in}}%
\pgfpathlineto{\pgfqpoint{3.744428in}{1.879654in}}%
\pgfpathlineto{\pgfqpoint{3.730661in}{1.881600in}}%
\pgfpathlineto{\pgfqpoint{3.716901in}{1.883625in}}%
\pgfpathlineto{\pgfqpoint{3.703147in}{1.885729in}}%
\pgfpathlineto{\pgfqpoint{3.689400in}{1.887912in}}%
\pgfpathlineto{\pgfqpoint{3.681266in}{1.878691in}}%
\pgfpathlineto{\pgfqpoint{3.673126in}{1.869459in}}%
\pgfpathlineto{\pgfqpoint{3.664981in}{1.860217in}}%
\pgfpathlineto{\pgfqpoint{3.656830in}{1.850966in}}%
\pgfpathclose%
\pgfusepath{fill}%
\end{pgfscope}%
\begin{pgfscope}%
\pgfpathrectangle{\pgfqpoint{1.150000in}{0.150000in}}{\pgfqpoint{5.700000in}{5.700000in}}%
\pgfusepath{clip}%
\pgfsetbuttcap%
\pgfsetroundjoin%
\definecolor{currentfill}{rgb}{0.210503,0.363727,0.552206}%
\pgfsetfillcolor{currentfill}%
\pgfsetfillopacity{0.700000}%
\pgfsetlinewidth{0.000000pt}%
\definecolor{currentstroke}{rgb}{0.000000,0.000000,0.000000}%
\pgfsetstrokecolor{currentstroke}%
\pgfsetdash{}{0pt}%
\pgfpathmoveto{\pgfqpoint{5.628358in}{2.533306in}}%
\pgfpathlineto{\pgfqpoint{5.642779in}{2.536038in}}%
\pgfpathlineto{\pgfqpoint{5.657212in}{2.538838in}}%
\pgfpathlineto{\pgfqpoint{5.671658in}{2.541706in}}%
\pgfpathlineto{\pgfqpoint{5.686116in}{2.544643in}}%
\pgfpathlineto{\pgfqpoint{5.693406in}{2.548641in}}%
\pgfpathlineto{\pgfqpoint{5.700690in}{2.552673in}}%
\pgfpathlineto{\pgfqpoint{5.707969in}{2.556743in}}%
\pgfpathlineto{\pgfqpoint{5.715243in}{2.560857in}}%
\pgfpathlineto{\pgfqpoint{5.700810in}{2.558240in}}%
\pgfpathlineto{\pgfqpoint{5.686389in}{2.555691in}}%
\pgfpathlineto{\pgfqpoint{5.671981in}{2.553210in}}%
\pgfpathlineto{\pgfqpoint{5.657584in}{2.550796in}}%
\pgfpathlineto{\pgfqpoint{5.650285in}{2.546356in}}%
\pgfpathlineto{\pgfqpoint{5.642981in}{2.541964in}}%
\pgfpathlineto{\pgfqpoint{5.635672in}{2.537616in}}%
\pgfpathlineto{\pgfqpoint{5.628358in}{2.533306in}}%
\pgfpathclose%
\pgfusepath{fill}%
\end{pgfscope}%
\begin{pgfscope}%
\pgfpathrectangle{\pgfqpoint{1.150000in}{0.150000in}}{\pgfqpoint{5.700000in}{5.700000in}}%
\pgfusepath{clip}%
\pgfsetbuttcap%
\pgfsetroundjoin%
\definecolor{currentfill}{rgb}{0.283091,0.110553,0.431554}%
\pgfsetfillcolor{currentfill}%
\pgfsetfillopacity{0.700000}%
\pgfsetlinewidth{0.000000pt}%
\definecolor{currentstroke}{rgb}{0.000000,0.000000,0.000000}%
\pgfsetstrokecolor{currentstroke}%
\pgfsetdash{}{0pt}%
\pgfpathmoveto{\pgfqpoint{4.062223in}{1.972247in}}%
\pgfpathlineto{\pgfqpoint{4.076088in}{1.971795in}}%
\pgfpathlineto{\pgfqpoint{4.089961in}{1.971418in}}%
\pgfpathlineto{\pgfqpoint{4.103842in}{1.971116in}}%
\pgfpathlineto{\pgfqpoint{4.117731in}{1.970889in}}%
\pgfpathlineto{\pgfqpoint{4.125729in}{1.980067in}}%
\pgfpathlineto{\pgfqpoint{4.133720in}{1.989196in}}%
\pgfpathlineto{\pgfqpoint{4.141706in}{1.998276in}}%
\pgfpathlineto{\pgfqpoint{4.149686in}{2.007308in}}%
\pgfpathlineto{\pgfqpoint{4.135806in}{2.007515in}}%
\pgfpathlineto{\pgfqpoint{4.121935in}{2.007797in}}%
\pgfpathlineto{\pgfqpoint{4.108072in}{2.008155in}}%
\pgfpathlineto{\pgfqpoint{4.094218in}{2.008587in}}%
\pgfpathlineto{\pgfqpoint{4.086228in}{1.999568in}}%
\pgfpathlineto{\pgfqpoint{4.078232in}{1.990504in}}%
\pgfpathlineto{\pgfqpoint{4.070230in}{1.981398in}}%
\pgfpathlineto{\pgfqpoint{4.062223in}{1.972247in}}%
\pgfpathclose%
\pgfusepath{fill}%
\end{pgfscope}%
\begin{pgfscope}%
\pgfpathrectangle{\pgfqpoint{1.150000in}{0.150000in}}{\pgfqpoint{5.700000in}{5.700000in}}%
\pgfusepath{clip}%
\pgfsetbuttcap%
\pgfsetroundjoin%
\definecolor{currentfill}{rgb}{0.269308,0.218818,0.509577}%
\pgfsetfillcolor{currentfill}%
\pgfsetfillopacity{0.700000}%
\pgfsetlinewidth{0.000000pt}%
\definecolor{currentstroke}{rgb}{0.000000,0.000000,0.000000}%
\pgfsetstrokecolor{currentstroke}%
\pgfsetdash{}{0pt}%
\pgfpathmoveto{\pgfqpoint{4.642571in}{2.190802in}}%
\pgfpathlineto{\pgfqpoint{4.656629in}{2.192210in}}%
\pgfpathlineto{\pgfqpoint{4.670697in}{2.193689in}}%
\pgfpathlineto{\pgfqpoint{4.684774in}{2.195240in}}%
\pgfpathlineto{\pgfqpoint{4.698863in}{2.196863in}}%
\pgfpathlineto{\pgfqpoint{4.706636in}{2.204387in}}%
\pgfpathlineto{\pgfqpoint{4.714402in}{2.211851in}}%
\pgfpathlineto{\pgfqpoint{4.722162in}{2.219259in}}%
\pgfpathlineto{\pgfqpoint{4.729915in}{2.226613in}}%
\pgfpathlineto{\pgfqpoint{4.715839in}{2.225097in}}%
\pgfpathlineto{\pgfqpoint{4.701774in}{2.223652in}}%
\pgfpathlineto{\pgfqpoint{4.687719in}{2.222279in}}%
\pgfpathlineto{\pgfqpoint{4.673674in}{2.220978in}}%
\pgfpathlineto{\pgfqpoint{4.665908in}{2.213510in}}%
\pgfpathlineto{\pgfqpoint{4.658136in}{2.205993in}}%
\pgfpathlineto{\pgfqpoint{4.650357in}{2.198425in}}%
\pgfpathlineto{\pgfqpoint{4.642571in}{2.190802in}}%
\pgfpathclose%
\pgfusepath{fill}%
\end{pgfscope}%
\begin{pgfscope}%
\pgfpathrectangle{\pgfqpoint{1.150000in}{0.150000in}}{\pgfqpoint{5.700000in}{5.700000in}}%
\pgfusepath{clip}%
\pgfsetbuttcap%
\pgfsetroundjoin%
\definecolor{currentfill}{rgb}{0.281924,0.089666,0.412415}%
\pgfsetfillcolor{currentfill}%
\pgfsetfillopacity{0.700000}%
\pgfsetlinewidth{0.000000pt}%
\definecolor{currentstroke}{rgb}{0.000000,0.000000,0.000000}%
\pgfsetstrokecolor{currentstroke}%
\pgfsetdash{}{0pt}%
\pgfpathmoveto{\pgfqpoint{2.567944in}{1.951572in}}%
\pgfpathlineto{\pgfqpoint{2.581625in}{1.942233in}}%
\pgfpathlineto{\pgfqpoint{2.595307in}{1.933002in}}%
\pgfpathlineto{\pgfqpoint{2.608989in}{1.923879in}}%
\pgfpathlineto{\pgfqpoint{2.622672in}{1.914863in}}%
\pgfpathlineto{\pgfqpoint{2.631329in}{1.919395in}}%
\pgfpathlineto{\pgfqpoint{2.639974in}{1.924075in}}%
\pgfpathlineto{\pgfqpoint{2.648607in}{1.928898in}}%
\pgfpathlineto{\pgfqpoint{2.657229in}{1.933861in}}%
\pgfpathlineto{\pgfqpoint{2.643573in}{1.942608in}}%
\pgfpathlineto{\pgfqpoint{2.629917in}{1.951461in}}%
\pgfpathlineto{\pgfqpoint{2.616262in}{1.960421in}}%
\pgfpathlineto{\pgfqpoint{2.602607in}{1.969490in}}%
\pgfpathlineto{\pgfqpoint{2.593959in}{1.964789in}}%
\pgfpathlineto{\pgfqpoint{2.585300in}{1.960233in}}%
\pgfpathlineto{\pgfqpoint{2.576628in}{1.955826in}}%
\pgfpathlineto{\pgfqpoint{2.567944in}{1.951572in}}%
\pgfpathclose%
\pgfusepath{fill}%
\end{pgfscope}%
\begin{pgfscope}%
\pgfpathrectangle{\pgfqpoint{1.150000in}{0.150000in}}{\pgfqpoint{5.700000in}{5.700000in}}%
\pgfusepath{clip}%
\pgfsetbuttcap%
\pgfsetroundjoin%
\definecolor{currentfill}{rgb}{0.280868,0.160771,0.472899}%
\pgfsetfillcolor{currentfill}%
\pgfsetfillopacity{0.700000}%
\pgfsetlinewidth{0.000000pt}%
\definecolor{currentstroke}{rgb}{0.000000,0.000000,0.000000}%
\pgfsetstrokecolor{currentstroke}%
\pgfsetdash{}{0pt}%
\pgfpathmoveto{\pgfqpoint{2.313646in}{2.105704in}}%
\pgfpathlineto{\pgfqpoint{2.327369in}{2.094221in}}%
\pgfpathlineto{\pgfqpoint{2.341091in}{2.082860in}}%
\pgfpathlineto{\pgfqpoint{2.354811in}{2.071621in}}%
\pgfpathlineto{\pgfqpoint{2.368530in}{2.060503in}}%
\pgfpathlineto{\pgfqpoint{2.377357in}{2.063200in}}%
\pgfpathlineto{\pgfqpoint{2.386170in}{2.066083in}}%
\pgfpathlineto{\pgfqpoint{2.394969in}{2.069148in}}%
\pgfpathlineto{\pgfqpoint{2.403753in}{2.072390in}}%
\pgfpathlineto{\pgfqpoint{2.390066in}{2.083214in}}%
\pgfpathlineto{\pgfqpoint{2.376377in}{2.094158in}}%
\pgfpathlineto{\pgfqpoint{2.362687in}{2.105223in}}%
\pgfpathlineto{\pgfqpoint{2.348996in}{2.116411in}}%
\pgfpathlineto{\pgfqpoint{2.340181in}{2.113456in}}%
\pgfpathlineto{\pgfqpoint{2.331350in}{2.110684in}}%
\pgfpathlineto{\pgfqpoint{2.322506in}{2.108098in}}%
\pgfpathlineto{\pgfqpoint{2.313646in}{2.105704in}}%
\pgfpathclose%
\pgfusepath{fill}%
\end{pgfscope}%
\begin{pgfscope}%
\pgfpathrectangle{\pgfqpoint{1.150000in}{0.150000in}}{\pgfqpoint{5.700000in}{5.700000in}}%
\pgfusepath{clip}%
\pgfsetbuttcap%
\pgfsetroundjoin%
\definecolor{currentfill}{rgb}{0.276022,0.044167,0.370164}%
\pgfsetfillcolor{currentfill}%
\pgfsetfillopacity{0.700000}%
\pgfsetlinewidth{0.000000pt}%
\definecolor{currentstroke}{rgb}{0.000000,0.000000,0.000000}%
\pgfsetstrokecolor{currentstroke}%
\pgfsetdash{}{0pt}%
\pgfpathmoveto{\pgfqpoint{2.766522in}{1.867644in}}%
\pgfpathlineto{\pgfqpoint{2.780190in}{1.859825in}}%
\pgfpathlineto{\pgfqpoint{2.793860in}{1.852107in}}%
\pgfpathlineto{\pgfqpoint{2.807531in}{1.844488in}}%
\pgfpathlineto{\pgfqpoint{2.821205in}{1.836967in}}%
\pgfpathlineto{\pgfqpoint{2.829743in}{1.842834in}}%
\pgfpathlineto{\pgfqpoint{2.838271in}{1.848817in}}%
\pgfpathlineto{\pgfqpoint{2.846789in}{1.854912in}}%
\pgfpathlineto{\pgfqpoint{2.855298in}{1.861117in}}%
\pgfpathlineto{\pgfqpoint{2.841647in}{1.868390in}}%
\pgfpathlineto{\pgfqpoint{2.827998in}{1.875762in}}%
\pgfpathlineto{\pgfqpoint{2.814351in}{1.883233in}}%
\pgfpathlineto{\pgfqpoint{2.800706in}{1.890804in}}%
\pgfpathlineto{\pgfqpoint{2.792175in}{1.884838in}}%
\pgfpathlineto{\pgfqpoint{2.783634in}{1.878988in}}%
\pgfpathlineto{\pgfqpoint{2.775083in}{1.873255in}}%
\pgfpathlineto{\pgfqpoint{2.766522in}{1.867644in}}%
\pgfpathclose%
\pgfusepath{fill}%
\end{pgfscope}%
\begin{pgfscope}%
\pgfpathrectangle{\pgfqpoint{1.150000in}{0.150000in}}{\pgfqpoint{5.700000in}{5.700000in}}%
\pgfusepath{clip}%
\pgfsetbuttcap%
\pgfsetroundjoin%
\definecolor{currentfill}{rgb}{0.241237,0.296485,0.539709}%
\pgfsetfillcolor{currentfill}%
\pgfsetfillopacity{0.700000}%
\pgfsetlinewidth{0.000000pt}%
\definecolor{currentstroke}{rgb}{0.000000,0.000000,0.000000}%
\pgfsetstrokecolor{currentstroke}%
\pgfsetdash{}{0pt}%
\pgfpathmoveto{\pgfqpoint{5.135646in}{2.371473in}}%
\pgfpathlineto{\pgfqpoint{5.149888in}{2.373831in}}%
\pgfpathlineto{\pgfqpoint{5.164141in}{2.376259in}}%
\pgfpathlineto{\pgfqpoint{5.178406in}{2.378756in}}%
\pgfpathlineto{\pgfqpoint{5.192683in}{2.381323in}}%
\pgfpathlineto{\pgfqpoint{5.200229in}{2.386954in}}%
\pgfpathlineto{\pgfqpoint{5.207767in}{2.392554in}}%
\pgfpathlineto{\pgfqpoint{5.215299in}{2.398127in}}%
\pgfpathlineto{\pgfqpoint{5.222825in}{2.403678in}}%
\pgfpathlineto{\pgfqpoint{5.208566in}{2.401325in}}%
\pgfpathlineto{\pgfqpoint{5.194319in}{2.399040in}}%
\pgfpathlineto{\pgfqpoint{5.180083in}{2.396825in}}%
\pgfpathlineto{\pgfqpoint{5.165859in}{2.394680in}}%
\pgfpathlineto{\pgfqpoint{5.158316in}{2.388908in}}%
\pgfpathlineto{\pgfqpoint{5.150766in}{2.383119in}}%
\pgfpathlineto{\pgfqpoint{5.143209in}{2.377309in}}%
\pgfpathlineto{\pgfqpoint{5.135646in}{2.371473in}}%
\pgfpathclose%
\pgfusepath{fill}%
\end{pgfscope}%
\begin{pgfscope}%
\pgfpathrectangle{\pgfqpoint{1.150000in}{0.150000in}}{\pgfqpoint{5.700000in}{5.700000in}}%
\pgfusepath{clip}%
\pgfsetbuttcap%
\pgfsetroundjoin%
\definecolor{currentfill}{rgb}{0.282656,0.100196,0.422160}%
\pgfsetfillcolor{currentfill}%
\pgfsetfillopacity{0.700000}%
\pgfsetlinewidth{0.000000pt}%
\definecolor{currentstroke}{rgb}{0.000000,0.000000,0.000000}%
\pgfsetstrokecolor{currentstroke}%
\pgfsetdash{}{0pt}%
\pgfpathmoveto{\pgfqpoint{3.974719in}{1.937934in}}%
\pgfpathlineto{\pgfqpoint{3.988562in}{1.937138in}}%
\pgfpathlineto{\pgfqpoint{4.002412in}{1.936419in}}%
\pgfpathlineto{\pgfqpoint{4.016271in}{1.935775in}}%
\pgfpathlineto{\pgfqpoint{4.030137in}{1.935207in}}%
\pgfpathlineto{\pgfqpoint{4.038167in}{1.944532in}}%
\pgfpathlineto{\pgfqpoint{4.046191in}{1.953814in}}%
\pgfpathlineto{\pgfqpoint{4.054210in}{1.963052in}}%
\pgfpathlineto{\pgfqpoint{4.062223in}{1.972247in}}%
\pgfpathlineto{\pgfqpoint{4.048367in}{1.972775in}}%
\pgfpathlineto{\pgfqpoint{4.034518in}{1.973378in}}%
\pgfpathlineto{\pgfqpoint{4.020678in}{1.974058in}}%
\pgfpathlineto{\pgfqpoint{4.006846in}{1.974813in}}%
\pgfpathlineto{\pgfqpoint{3.998823in}{1.965651in}}%
\pgfpathlineto{\pgfqpoint{3.990794in}{1.956451in}}%
\pgfpathlineto{\pgfqpoint{3.982759in}{1.947212in}}%
\pgfpathlineto{\pgfqpoint{3.974719in}{1.937934in}}%
\pgfpathclose%
\pgfusepath{fill}%
\end{pgfscope}%
\begin{pgfscope}%
\pgfpathrectangle{\pgfqpoint{1.150000in}{0.150000in}}{\pgfqpoint{5.700000in}{5.700000in}}%
\pgfusepath{clip}%
\pgfsetbuttcap%
\pgfsetroundjoin%
\definecolor{currentfill}{rgb}{0.273006,0.204520,0.501721}%
\pgfsetfillcolor{currentfill}%
\pgfsetfillopacity{0.700000}%
\pgfsetlinewidth{0.000000pt}%
\definecolor{currentstroke}{rgb}{0.000000,0.000000,0.000000}%
\pgfsetstrokecolor{currentstroke}%
\pgfsetdash{}{0pt}%
\pgfpathmoveto{\pgfqpoint{4.555186in}{2.154482in}}%
\pgfpathlineto{\pgfqpoint{4.569216in}{2.155688in}}%
\pgfpathlineto{\pgfqpoint{4.583255in}{2.156965in}}%
\pgfpathlineto{\pgfqpoint{4.597305in}{2.158314in}}%
\pgfpathlineto{\pgfqpoint{4.611364in}{2.159736in}}%
\pgfpathlineto{\pgfqpoint{4.619176in}{2.167593in}}%
\pgfpathlineto{\pgfqpoint{4.626981in}{2.175388in}}%
\pgfpathlineto{\pgfqpoint{4.634779in}{2.183124in}}%
\pgfpathlineto{\pgfqpoint{4.642571in}{2.190802in}}%
\pgfpathlineto{\pgfqpoint{4.628524in}{2.189466in}}%
\pgfpathlineto{\pgfqpoint{4.614486in}{2.188202in}}%
\pgfpathlineto{\pgfqpoint{4.600459in}{2.187010in}}%
\pgfpathlineto{\pgfqpoint{4.586441in}{2.185890in}}%
\pgfpathlineto{\pgfqpoint{4.578637in}{2.178119in}}%
\pgfpathlineto{\pgfqpoint{4.570826in}{2.170295in}}%
\pgfpathlineto{\pgfqpoint{4.563009in}{2.162417in}}%
\pgfpathlineto{\pgfqpoint{4.555186in}{2.154482in}}%
\pgfpathclose%
\pgfusepath{fill}%
\end{pgfscope}%
\begin{pgfscope}%
\pgfpathrectangle{\pgfqpoint{1.150000in}{0.150000in}}{\pgfqpoint{5.700000in}{5.700000in}}%
\pgfusepath{clip}%
\pgfsetbuttcap%
\pgfsetroundjoin%
\definecolor{currentfill}{rgb}{0.269944,0.014625,0.341379}%
\pgfsetfillcolor{currentfill}%
\pgfsetfillopacity{0.700000}%
\pgfsetlinewidth{0.000000pt}%
\definecolor{currentstroke}{rgb}{0.000000,0.000000,0.000000}%
\pgfsetstrokecolor{currentstroke}%
\pgfsetdash{}{0pt}%
\pgfpathmoveto{\pgfqpoint{3.338636in}{1.794319in}}%
\pgfpathlineto{\pgfqpoint{3.352340in}{1.790293in}}%
\pgfpathlineto{\pgfqpoint{3.366049in}{1.786351in}}%
\pgfpathlineto{\pgfqpoint{3.379764in}{1.782493in}}%
\pgfpathlineto{\pgfqpoint{3.393483in}{1.778718in}}%
\pgfpathlineto{\pgfqpoint{3.401746in}{1.787434in}}%
\pgfpathlineto{\pgfqpoint{3.410002in}{1.796178in}}%
\pgfpathlineto{\pgfqpoint{3.418252in}{1.804947in}}%
\pgfpathlineto{\pgfqpoint{3.426496in}{1.813740in}}%
\pgfpathlineto{\pgfqpoint{3.412790in}{1.817350in}}%
\pgfpathlineto{\pgfqpoint{3.399090in}{1.821045in}}%
\pgfpathlineto{\pgfqpoint{3.385396in}{1.824823in}}%
\pgfpathlineto{\pgfqpoint{3.371706in}{1.828686in}}%
\pgfpathlineto{\pgfqpoint{3.363449in}{1.820049in}}%
\pgfpathlineto{\pgfqpoint{3.355185in}{1.811441in}}%
\pgfpathlineto{\pgfqpoint{3.346914in}{1.802863in}}%
\pgfpathlineto{\pgfqpoint{3.338636in}{1.794319in}}%
\pgfpathclose%
\pgfusepath{fill}%
\end{pgfscope}%
\begin{pgfscope}%
\pgfpathrectangle{\pgfqpoint{1.150000in}{0.150000in}}{\pgfqpoint{5.700000in}{5.700000in}}%
\pgfusepath{clip}%
\pgfsetbuttcap%
\pgfsetroundjoin%
\definecolor{currentfill}{rgb}{0.273809,0.031497,0.358853}%
\pgfsetfillcolor{currentfill}%
\pgfsetfillopacity{0.700000}%
\pgfsetlinewidth{0.000000pt}%
\definecolor{currentstroke}{rgb}{0.000000,0.000000,0.000000}%
\pgfsetstrokecolor{currentstroke}%
\pgfsetdash{}{0pt}%
\pgfpathmoveto{\pgfqpoint{3.569149in}{1.824336in}}%
\pgfpathlineto{\pgfqpoint{3.582894in}{1.821607in}}%
\pgfpathlineto{\pgfqpoint{3.596645in}{1.818959in}}%
\pgfpathlineto{\pgfqpoint{3.610403in}{1.816391in}}%
\pgfpathlineto{\pgfqpoint{3.624167in}{1.813903in}}%
\pgfpathlineto{\pgfqpoint{3.632341in}{1.823174in}}%
\pgfpathlineto{\pgfqpoint{3.640510in}{1.832443in}}%
\pgfpathlineto{\pgfqpoint{3.648673in}{1.841707in}}%
\pgfpathlineto{\pgfqpoint{3.656830in}{1.850966in}}%
\pgfpathlineto{\pgfqpoint{3.643078in}{1.853331in}}%
\pgfpathlineto{\pgfqpoint{3.629332in}{1.855777in}}%
\pgfpathlineto{\pgfqpoint{3.615593in}{1.858303in}}%
\pgfpathlineto{\pgfqpoint{3.601861in}{1.860909in}}%
\pgfpathlineto{\pgfqpoint{3.593692in}{1.851766in}}%
\pgfpathlineto{\pgfqpoint{3.585517in}{1.842621in}}%
\pgfpathlineto{\pgfqpoint{3.577336in}{1.833478in}}%
\pgfpathlineto{\pgfqpoint{3.569149in}{1.824336in}}%
\pgfpathclose%
\pgfusepath{fill}%
\end{pgfscope}%
\begin{pgfscope}%
\pgfpathrectangle{\pgfqpoint{1.150000in}{0.150000in}}{\pgfqpoint{5.700000in}{5.700000in}}%
\pgfusepath{clip}%
\pgfsetbuttcap%
\pgfsetroundjoin%
\definecolor{currentfill}{rgb}{0.214298,0.355619,0.551184}%
\pgfsetfillcolor{currentfill}%
\pgfsetfillopacity{0.700000}%
\pgfsetlinewidth{0.000000pt}%
\definecolor{currentstroke}{rgb}{0.000000,0.000000,0.000000}%
\pgfsetstrokecolor{currentstroke}%
\pgfsetdash{}{0pt}%
\pgfpathmoveto{\pgfqpoint{5.541386in}{2.504897in}}%
\pgfpathlineto{\pgfqpoint{5.555782in}{2.507655in}}%
\pgfpathlineto{\pgfqpoint{5.570191in}{2.510481in}}%
\pgfpathlineto{\pgfqpoint{5.584612in}{2.513375in}}%
\pgfpathlineto{\pgfqpoint{5.599045in}{2.516338in}}%
\pgfpathlineto{\pgfqpoint{5.606382in}{2.520550in}}%
\pgfpathlineto{\pgfqpoint{5.613713in}{2.524778in}}%
\pgfpathlineto{\pgfqpoint{5.621038in}{2.529028in}}%
\pgfpathlineto{\pgfqpoint{5.628358in}{2.533306in}}%
\pgfpathlineto{\pgfqpoint{5.613949in}{2.530642in}}%
\pgfpathlineto{\pgfqpoint{5.599552in}{2.528046in}}%
\pgfpathlineto{\pgfqpoint{5.585167in}{2.525518in}}%
\pgfpathlineto{\pgfqpoint{5.570794in}{2.523058in}}%
\pgfpathlineto{\pgfqpoint{5.563450in}{2.518475in}}%
\pgfpathlineto{\pgfqpoint{5.556101in}{2.513924in}}%
\pgfpathlineto{\pgfqpoint{5.548746in}{2.509400in}}%
\pgfpathlineto{\pgfqpoint{5.541386in}{2.504897in}}%
\pgfpathclose%
\pgfusepath{fill}%
\end{pgfscope}%
\begin{pgfscope}%
\pgfpathrectangle{\pgfqpoint{1.150000in}{0.150000in}}{\pgfqpoint{5.700000in}{5.700000in}}%
\pgfusepath{clip}%
\pgfsetbuttcap%
\pgfsetroundjoin%
\definecolor{currentfill}{rgb}{0.277134,0.185228,0.489898}%
\pgfsetfillcolor{currentfill}%
\pgfsetfillopacity{0.700000}%
\pgfsetlinewidth{0.000000pt}%
\definecolor{currentstroke}{rgb}{0.000000,0.000000,0.000000}%
\pgfsetstrokecolor{currentstroke}%
\pgfsetdash{}{0pt}%
\pgfpathmoveto{\pgfqpoint{4.467763in}{2.117792in}}%
\pgfpathlineto{\pgfqpoint{4.481765in}{2.118772in}}%
\pgfpathlineto{\pgfqpoint{4.495777in}{2.119825in}}%
\pgfpathlineto{\pgfqpoint{4.509798in}{2.120950in}}%
\pgfpathlineto{\pgfqpoint{4.523830in}{2.122147in}}%
\pgfpathlineto{\pgfqpoint{4.531678in}{2.130324in}}%
\pgfpathlineto{\pgfqpoint{4.539521in}{2.138437in}}%
\pgfpathlineto{\pgfqpoint{4.547357in}{2.146490in}}%
\pgfpathlineto{\pgfqpoint{4.555186in}{2.154482in}}%
\pgfpathlineto{\pgfqpoint{4.541166in}{2.153349in}}%
\pgfpathlineto{\pgfqpoint{4.527156in}{2.152289in}}%
\pgfpathlineto{\pgfqpoint{4.513156in}{2.151301in}}%
\pgfpathlineto{\pgfqpoint{4.499166in}{2.150385in}}%
\pgfpathlineto{\pgfqpoint{4.491324in}{2.142320in}}%
\pgfpathlineto{\pgfqpoint{4.483477in}{2.134201in}}%
\pgfpathlineto{\pgfqpoint{4.475623in}{2.126025in}}%
\pgfpathlineto{\pgfqpoint{4.467763in}{2.117792in}}%
\pgfpathclose%
\pgfusepath{fill}%
\end{pgfscope}%
\begin{pgfscope}%
\pgfpathrectangle{\pgfqpoint{1.150000in}{0.150000in}}{\pgfqpoint{5.700000in}{5.700000in}}%
\pgfusepath{clip}%
\pgfsetbuttcap%
\pgfsetroundjoin%
\definecolor{currentfill}{rgb}{0.282290,0.145912,0.461510}%
\pgfsetfillcolor{currentfill}%
\pgfsetfillopacity{0.700000}%
\pgfsetlinewidth{0.000000pt}%
\definecolor{currentstroke}{rgb}{0.000000,0.000000,0.000000}%
\pgfsetstrokecolor{currentstroke}%
\pgfsetdash{}{0pt}%
\pgfpathmoveto{\pgfqpoint{2.368530in}{2.060503in}}%
\pgfpathlineto{\pgfqpoint{2.382247in}{2.049504in}}%
\pgfpathlineto{\pgfqpoint{2.395963in}{2.038625in}}%
\pgfpathlineto{\pgfqpoint{2.409678in}{2.027863in}}%
\pgfpathlineto{\pgfqpoint{2.423392in}{2.017219in}}%
\pgfpathlineto{\pgfqpoint{2.432187in}{2.020218in}}%
\pgfpathlineto{\pgfqpoint{2.440969in}{2.023398in}}%
\pgfpathlineto{\pgfqpoint{2.449737in}{2.026754in}}%
\pgfpathlineto{\pgfqpoint{2.458491in}{2.030282in}}%
\pgfpathlineto{\pgfqpoint{2.444808in}{2.040633in}}%
\pgfpathlineto{\pgfqpoint{2.431124in}{2.051101in}}%
\pgfpathlineto{\pgfqpoint{2.417439in}{2.061686in}}%
\pgfpathlineto{\pgfqpoint{2.403753in}{2.072390in}}%
\pgfpathlineto{\pgfqpoint{2.394969in}{2.069148in}}%
\pgfpathlineto{\pgfqpoint{2.386170in}{2.066083in}}%
\pgfpathlineto{\pgfqpoint{2.377357in}{2.063200in}}%
\pgfpathlineto{\pgfqpoint{2.368530in}{2.060503in}}%
\pgfpathclose%
\pgfusepath{fill}%
\end{pgfscope}%
\begin{pgfscope}%
\pgfpathrectangle{\pgfqpoint{1.150000in}{0.150000in}}{\pgfqpoint{5.700000in}{5.700000in}}%
\pgfusepath{clip}%
\pgfsetbuttcap%
\pgfsetroundjoin%
\definecolor{currentfill}{rgb}{0.244972,0.287675,0.537260}%
\pgfsetfillcolor{currentfill}%
\pgfsetfillopacity{0.700000}%
\pgfsetlinewidth{0.000000pt}%
\definecolor{currentstroke}{rgb}{0.000000,0.000000,0.000000}%
\pgfsetstrokecolor{currentstroke}%
\pgfsetdash{}{0pt}%
\pgfpathmoveto{\pgfqpoint{5.048400in}{2.338302in}}%
\pgfpathlineto{\pgfqpoint{5.062614in}{2.340573in}}%
\pgfpathlineto{\pgfqpoint{5.076840in}{2.342913in}}%
\pgfpathlineto{\pgfqpoint{5.091076in}{2.345324in}}%
\pgfpathlineto{\pgfqpoint{5.105324in}{2.347804in}}%
\pgfpathlineto{\pgfqpoint{5.112915in}{2.353778in}}%
\pgfpathlineto{\pgfqpoint{5.120499in}{2.359711in}}%
\pgfpathlineto{\pgfqpoint{5.128076in}{2.365608in}}%
\pgfpathlineto{\pgfqpoint{5.135646in}{2.371473in}}%
\pgfpathlineto{\pgfqpoint{5.121415in}{2.369185in}}%
\pgfpathlineto{\pgfqpoint{5.107195in}{2.366966in}}%
\pgfpathlineto{\pgfqpoint{5.092986in}{2.364818in}}%
\pgfpathlineto{\pgfqpoint{5.078789in}{2.362739in}}%
\pgfpathlineto{\pgfqpoint{5.071202in}{2.356674in}}%
\pgfpathlineto{\pgfqpoint{5.063608in}{2.350583in}}%
\pgfpathlineto{\pgfqpoint{5.056008in}{2.344460in}}%
\pgfpathlineto{\pgfqpoint{5.048400in}{2.338302in}}%
\pgfpathclose%
\pgfusepath{fill}%
\end{pgfscope}%
\begin{pgfscope}%
\pgfpathrectangle{\pgfqpoint{1.150000in}{0.150000in}}{\pgfqpoint{5.700000in}{5.700000in}}%
\pgfusepath{clip}%
\pgfsetbuttcap%
\pgfsetroundjoin%
\definecolor{currentfill}{rgb}{0.281446,0.084320,0.407414}%
\pgfsetfillcolor{currentfill}%
\pgfsetfillopacity{0.700000}%
\pgfsetlinewidth{0.000000pt}%
\definecolor{currentstroke}{rgb}{0.000000,0.000000,0.000000}%
\pgfsetstrokecolor{currentstroke}%
\pgfsetdash{}{0pt}%
\pgfpathmoveto{\pgfqpoint{3.887170in}{1.904640in}}%
\pgfpathlineto{\pgfqpoint{3.900992in}{1.903477in}}%
\pgfpathlineto{\pgfqpoint{3.914821in}{1.902391in}}%
\pgfpathlineto{\pgfqpoint{3.928658in}{1.901382in}}%
\pgfpathlineto{\pgfqpoint{3.942503in}{1.900449in}}%
\pgfpathlineto{\pgfqpoint{3.950565in}{1.909875in}}%
\pgfpathlineto{\pgfqpoint{3.958622in}{1.919266in}}%
\pgfpathlineto{\pgfqpoint{3.966674in}{1.928619in}}%
\pgfpathlineto{\pgfqpoint{3.974719in}{1.937934in}}%
\pgfpathlineto{\pgfqpoint{3.960885in}{1.938806in}}%
\pgfpathlineto{\pgfqpoint{3.947058in}{1.939755in}}%
\pgfpathlineto{\pgfqpoint{3.933239in}{1.940780in}}%
\pgfpathlineto{\pgfqpoint{3.919428in}{1.941882in}}%
\pgfpathlineto{\pgfqpoint{3.911372in}{1.932620in}}%
\pgfpathlineto{\pgfqpoint{3.903310in}{1.923325in}}%
\pgfpathlineto{\pgfqpoint{3.895243in}{1.913998in}}%
\pgfpathlineto{\pgfqpoint{3.887170in}{1.904640in}}%
\pgfpathclose%
\pgfusepath{fill}%
\end{pgfscope}%
\begin{pgfscope}%
\pgfpathrectangle{\pgfqpoint{1.150000in}{0.150000in}}{\pgfqpoint{5.700000in}{5.700000in}}%
\pgfusepath{clip}%
\pgfsetbuttcap%
\pgfsetroundjoin%
\definecolor{currentfill}{rgb}{0.271305,0.019942,0.347269}%
\pgfsetfillcolor{currentfill}%
\pgfsetfillopacity{0.700000}%
\pgfsetlinewidth{0.000000pt}%
\definecolor{currentstroke}{rgb}{0.000000,0.000000,0.000000}%
\pgfsetstrokecolor{currentstroke}%
\pgfsetdash{}{0pt}%
\pgfpathmoveto{\pgfqpoint{2.964595in}{1.806408in}}%
\pgfpathlineto{\pgfqpoint{2.978269in}{1.799995in}}%
\pgfpathlineto{\pgfqpoint{2.991947in}{1.793675in}}%
\pgfpathlineto{\pgfqpoint{3.005628in}{1.787448in}}%
\pgfpathlineto{\pgfqpoint{3.019312in}{1.781313in}}%
\pgfpathlineto{\pgfqpoint{3.027748in}{1.788325in}}%
\pgfpathlineto{\pgfqpoint{3.036176in}{1.795424in}}%
\pgfpathlineto{\pgfqpoint{3.044596in}{1.802607in}}%
\pgfpathlineto{\pgfqpoint{3.053007in}{1.809870in}}%
\pgfpathlineto{\pgfqpoint{3.039342in}{1.815780in}}%
\pgfpathlineto{\pgfqpoint{3.025681in}{1.821781in}}%
\pgfpathlineto{\pgfqpoint{3.012023in}{1.827875in}}%
\pgfpathlineto{\pgfqpoint{2.998368in}{1.834062in}}%
\pgfpathlineto{\pgfqpoint{2.989938in}{1.827016in}}%
\pgfpathlineto{\pgfqpoint{2.981499in}{1.820057in}}%
\pgfpathlineto{\pgfqpoint{2.973051in}{1.813186in}}%
\pgfpathlineto{\pgfqpoint{2.964595in}{1.806408in}}%
\pgfpathclose%
\pgfusepath{fill}%
\end{pgfscope}%
\begin{pgfscope}%
\pgfpathrectangle{\pgfqpoint{1.150000in}{0.150000in}}{\pgfqpoint{5.700000in}{5.700000in}}%
\pgfusepath{clip}%
\pgfsetbuttcap%
\pgfsetroundjoin%
\definecolor{currentfill}{rgb}{0.280894,0.078907,0.402329}%
\pgfsetfillcolor{currentfill}%
\pgfsetfillopacity{0.700000}%
\pgfsetlinewidth{0.000000pt}%
\definecolor{currentstroke}{rgb}{0.000000,0.000000,0.000000}%
\pgfsetstrokecolor{currentstroke}%
\pgfsetdash{}{0pt}%
\pgfpathmoveto{\pgfqpoint{2.622672in}{1.914863in}}%
\pgfpathlineto{\pgfqpoint{2.636356in}{1.905952in}}%
\pgfpathlineto{\pgfqpoint{2.650040in}{1.897147in}}%
\pgfpathlineto{\pgfqpoint{2.663726in}{1.888447in}}%
\pgfpathlineto{\pgfqpoint{2.677412in}{1.879851in}}%
\pgfpathlineto{\pgfqpoint{2.686042in}{1.884660in}}%
\pgfpathlineto{\pgfqpoint{2.694661in}{1.889612in}}%
\pgfpathlineto{\pgfqpoint{2.703269in}{1.894703in}}%
\pgfpathlineto{\pgfqpoint{2.711865in}{1.899927in}}%
\pgfpathlineto{\pgfqpoint{2.698205in}{1.908255in}}%
\pgfpathlineto{\pgfqpoint{2.684545in}{1.916685in}}%
\pgfpathlineto{\pgfqpoint{2.670887in}{1.925221in}}%
\pgfpathlineto{\pgfqpoint{2.657229in}{1.933861in}}%
\pgfpathlineto{\pgfqpoint{2.648607in}{1.928898in}}%
\pgfpathlineto{\pgfqpoint{2.639974in}{1.924075in}}%
\pgfpathlineto{\pgfqpoint{2.631329in}{1.919395in}}%
\pgfpathlineto{\pgfqpoint{2.622672in}{1.914863in}}%
\pgfpathclose%
\pgfusepath{fill}%
\end{pgfscope}%
\begin{pgfscope}%
\pgfpathrectangle{\pgfqpoint{1.150000in}{0.150000in}}{\pgfqpoint{5.700000in}{5.700000in}}%
\pgfusepath{clip}%
\pgfsetbuttcap%
\pgfsetroundjoin%
\definecolor{currentfill}{rgb}{0.268510,0.009605,0.335427}%
\pgfsetfillcolor{currentfill}%
\pgfsetfillopacity{0.700000}%
\pgfsetlinewidth{0.000000pt}%
\definecolor{currentstroke}{rgb}{0.000000,0.000000,0.000000}%
\pgfsetstrokecolor{currentstroke}%
\pgfsetdash{}{0pt}%
\pgfpathmoveto{\pgfqpoint{3.107701in}{1.787142in}}%
\pgfpathlineto{\pgfqpoint{3.121384in}{1.781684in}}%
\pgfpathlineto{\pgfqpoint{3.135071in}{1.776316in}}%
\pgfpathlineto{\pgfqpoint{3.148761in}{1.771037in}}%
\pgfpathlineto{\pgfqpoint{3.162456in}{1.765846in}}%
\pgfpathlineto{\pgfqpoint{3.170823in}{1.773612in}}%
\pgfpathlineto{\pgfqpoint{3.179182in}{1.781441in}}%
\pgfpathlineto{\pgfqpoint{3.187533in}{1.789333in}}%
\pgfpathlineto{\pgfqpoint{3.195877in}{1.797282in}}%
\pgfpathlineto{\pgfqpoint{3.182200in}{1.802268in}}%
\pgfpathlineto{\pgfqpoint{3.168526in}{1.807343in}}%
\pgfpathlineto{\pgfqpoint{3.154857in}{1.812506in}}%
\pgfpathlineto{\pgfqpoint{3.141192in}{1.817758in}}%
\pgfpathlineto{\pgfqpoint{3.132831in}{1.810006in}}%
\pgfpathlineto{\pgfqpoint{3.124462in}{1.802317in}}%
\pgfpathlineto{\pgfqpoint{3.116085in}{1.794695in}}%
\pgfpathlineto{\pgfqpoint{3.107701in}{1.787142in}}%
\pgfpathclose%
\pgfusepath{fill}%
\end{pgfscope}%
\begin{pgfscope}%
\pgfpathrectangle{\pgfqpoint{1.150000in}{0.150000in}}{\pgfqpoint{5.700000in}{5.700000in}}%
\pgfusepath{clip}%
\pgfsetbuttcap%
\pgfsetroundjoin%
\definecolor{currentfill}{rgb}{0.279574,0.170599,0.479997}%
\pgfsetfillcolor{currentfill}%
\pgfsetfillopacity{0.700000}%
\pgfsetlinewidth{0.000000pt}%
\definecolor{currentstroke}{rgb}{0.000000,0.000000,0.000000}%
\pgfsetstrokecolor{currentstroke}%
\pgfsetdash{}{0pt}%
\pgfpathmoveto{\pgfqpoint{4.380305in}{2.080894in}}%
\pgfpathlineto{\pgfqpoint{4.394280in}{2.081626in}}%
\pgfpathlineto{\pgfqpoint{4.408264in}{2.082430in}}%
\pgfpathlineto{\pgfqpoint{4.422258in}{2.083308in}}%
\pgfpathlineto{\pgfqpoint{4.436261in}{2.084259in}}%
\pgfpathlineto{\pgfqpoint{4.444146in}{2.092735in}}%
\pgfpathlineto{\pgfqpoint{4.452025in}{2.101148in}}%
\pgfpathlineto{\pgfqpoint{4.459897in}{2.109501in}}%
\pgfpathlineto{\pgfqpoint{4.467763in}{2.117792in}}%
\pgfpathlineto{\pgfqpoint{4.453771in}{2.116885in}}%
\pgfpathlineto{\pgfqpoint{4.439788in}{2.116051in}}%
\pgfpathlineto{\pgfqpoint{4.425815in}{2.115290in}}%
\pgfpathlineto{\pgfqpoint{4.411851in}{2.114601in}}%
\pgfpathlineto{\pgfqpoint{4.403974in}{2.106258in}}%
\pgfpathlineto{\pgfqpoint{4.396090in}{2.097860in}}%
\pgfpathlineto{\pgfqpoint{4.388201in}{2.089406in}}%
\pgfpathlineto{\pgfqpoint{4.380305in}{2.080894in}}%
\pgfpathclose%
\pgfusepath{fill}%
\end{pgfscope}%
\begin{pgfscope}%
\pgfpathrectangle{\pgfqpoint{1.150000in}{0.150000in}}{\pgfqpoint{5.700000in}{5.700000in}}%
\pgfusepath{clip}%
\pgfsetbuttcap%
\pgfsetroundjoin%
\definecolor{currentfill}{rgb}{0.220057,0.343307,0.549413}%
\pgfsetfillcolor{currentfill}%
\pgfsetfillopacity{0.700000}%
\pgfsetlinewidth{0.000000pt}%
\definecolor{currentstroke}{rgb}{0.000000,0.000000,0.000000}%
\pgfsetstrokecolor{currentstroke}%
\pgfsetdash{}{0pt}%
\pgfpathmoveto{\pgfqpoint{5.454329in}{2.475545in}}%
\pgfpathlineto{\pgfqpoint{5.468700in}{2.478306in}}%
\pgfpathlineto{\pgfqpoint{5.483082in}{2.481136in}}%
\pgfpathlineto{\pgfqpoint{5.497477in}{2.484034in}}%
\pgfpathlineto{\pgfqpoint{5.511884in}{2.487001in}}%
\pgfpathlineto{\pgfqpoint{5.519269in}{2.491468in}}%
\pgfpathlineto{\pgfqpoint{5.526647in}{2.495936in}}%
\pgfpathlineto{\pgfqpoint{5.534020in}{2.500411in}}%
\pgfpathlineto{\pgfqpoint{5.541386in}{2.504897in}}%
\pgfpathlineto{\pgfqpoint{5.527002in}{2.502208in}}%
\pgfpathlineto{\pgfqpoint{5.512629in}{2.499587in}}%
\pgfpathlineto{\pgfqpoint{5.498269in}{2.497035in}}%
\pgfpathlineto{\pgfqpoint{5.483920in}{2.494551in}}%
\pgfpathlineto{\pgfqpoint{5.476532in}{2.489780in}}%
\pgfpathlineto{\pgfqpoint{5.469137in}{2.485025in}}%
\pgfpathlineto{\pgfqpoint{5.461736in}{2.480282in}}%
\pgfpathlineto{\pgfqpoint{5.454329in}{2.475545in}}%
\pgfpathclose%
\pgfusepath{fill}%
\end{pgfscope}%
\begin{pgfscope}%
\pgfpathrectangle{\pgfqpoint{1.150000in}{0.150000in}}{\pgfqpoint{5.700000in}{5.700000in}}%
\pgfusepath{clip}%
\pgfsetbuttcap%
\pgfsetroundjoin%
\definecolor{currentfill}{rgb}{0.250425,0.274290,0.533103}%
\pgfsetfillcolor{currentfill}%
\pgfsetfillopacity{0.700000}%
\pgfsetlinewidth{0.000000pt}%
\definecolor{currentstroke}{rgb}{0.000000,0.000000,0.000000}%
\pgfsetstrokecolor{currentstroke}%
\pgfsetdash{}{0pt}%
\pgfpathmoveto{\pgfqpoint{4.961094in}{2.304194in}}%
\pgfpathlineto{\pgfqpoint{4.975280in}{2.306354in}}%
\pgfpathlineto{\pgfqpoint{4.989476in}{2.308585in}}%
\pgfpathlineto{\pgfqpoint{5.003684in}{2.310886in}}%
\pgfpathlineto{\pgfqpoint{5.017902in}{2.313257in}}%
\pgfpathlineto{\pgfqpoint{5.025537in}{2.319587in}}%
\pgfpathlineto{\pgfqpoint{5.033165in}{2.325869in}}%
\pgfpathlineto{\pgfqpoint{5.040786in}{2.332106in}}%
\pgfpathlineto{\pgfqpoint{5.048400in}{2.338302in}}%
\pgfpathlineto{\pgfqpoint{5.034198in}{2.336102in}}%
\pgfpathlineto{\pgfqpoint{5.020006in}{2.333971in}}%
\pgfpathlineto{\pgfqpoint{5.005825in}{2.331911in}}%
\pgfpathlineto{\pgfqpoint{4.991655in}{2.329921in}}%
\pgfpathlineto{\pgfqpoint{4.984025in}{2.323547in}}%
\pgfpathlineto{\pgfqpoint{4.976388in}{2.317137in}}%
\pgfpathlineto{\pgfqpoint{4.968745in}{2.310686in}}%
\pgfpathlineto{\pgfqpoint{4.961094in}{2.304194in}}%
\pgfpathclose%
\pgfusepath{fill}%
\end{pgfscope}%
\begin{pgfscope}%
\pgfpathrectangle{\pgfqpoint{1.150000in}{0.150000in}}{\pgfqpoint{5.700000in}{5.700000in}}%
\pgfusepath{clip}%
\pgfsetbuttcap%
\pgfsetroundjoin%
\definecolor{currentfill}{rgb}{0.279566,0.067836,0.391917}%
\pgfsetfillcolor{currentfill}%
\pgfsetfillopacity{0.700000}%
\pgfsetlinewidth{0.000000pt}%
\definecolor{currentstroke}{rgb}{0.000000,0.000000,0.000000}%
\pgfsetstrokecolor{currentstroke}%
\pgfsetdash{}{0pt}%
\pgfpathmoveto{\pgfqpoint{3.799567in}{1.872657in}}%
\pgfpathlineto{\pgfqpoint{3.813370in}{1.871103in}}%
\pgfpathlineto{\pgfqpoint{3.827180in}{1.869627in}}%
\pgfpathlineto{\pgfqpoint{3.840997in}{1.868228in}}%
\pgfpathlineto{\pgfqpoint{3.854822in}{1.866907in}}%
\pgfpathlineto{\pgfqpoint{3.862917in}{1.876383in}}%
\pgfpathlineto{\pgfqpoint{3.871007in}{1.885832in}}%
\pgfpathlineto{\pgfqpoint{3.879091in}{1.895251in}}%
\pgfpathlineto{\pgfqpoint{3.887170in}{1.904640in}}%
\pgfpathlineto{\pgfqpoint{3.873356in}{1.905880in}}%
\pgfpathlineto{\pgfqpoint{3.859549in}{1.907197in}}%
\pgfpathlineto{\pgfqpoint{3.845750in}{1.908592in}}%
\pgfpathlineto{\pgfqpoint{3.831958in}{1.910065in}}%
\pgfpathlineto{\pgfqpoint{3.823869in}{1.900749in}}%
\pgfpathlineto{\pgfqpoint{3.815774in}{1.891409in}}%
\pgfpathlineto{\pgfqpoint{3.807673in}{1.882045in}}%
\pgfpathlineto{\pgfqpoint{3.799567in}{1.872657in}}%
\pgfpathclose%
\pgfusepath{fill}%
\end{pgfscope}%
\begin{pgfscope}%
\pgfpathrectangle{\pgfqpoint{1.150000in}{0.150000in}}{\pgfqpoint{5.700000in}{5.700000in}}%
\pgfusepath{clip}%
\pgfsetbuttcap%
\pgfsetroundjoin%
\definecolor{currentfill}{rgb}{0.272594,0.025563,0.353093}%
\pgfsetfillcolor{currentfill}%
\pgfsetfillopacity{0.700000}%
\pgfsetlinewidth{0.000000pt}%
\definecolor{currentstroke}{rgb}{0.000000,0.000000,0.000000}%
\pgfsetstrokecolor{currentstroke}%
\pgfsetdash{}{0pt}%
\pgfpathmoveto{\pgfqpoint{3.481373in}{1.800126in}}%
\pgfpathlineto{\pgfqpoint{3.495107in}{1.796929in}}%
\pgfpathlineto{\pgfqpoint{3.508846in}{1.793814in}}%
\pgfpathlineto{\pgfqpoint{3.522592in}{1.790779in}}%
\pgfpathlineto{\pgfqpoint{3.536343in}{1.787826in}}%
\pgfpathlineto{\pgfqpoint{3.544554in}{1.796942in}}%
\pgfpathlineto{\pgfqpoint{3.552758in}{1.806067in}}%
\pgfpathlineto{\pgfqpoint{3.560957in}{1.815199in}}%
\pgfpathlineto{\pgfqpoint{3.569149in}{1.824336in}}%
\pgfpathlineto{\pgfqpoint{3.555411in}{1.827146in}}%
\pgfpathlineto{\pgfqpoint{3.541678in}{1.830037in}}%
\pgfpathlineto{\pgfqpoint{3.527952in}{1.833010in}}%
\pgfpathlineto{\pgfqpoint{3.514231in}{1.836064in}}%
\pgfpathlineto{\pgfqpoint{3.506026in}{1.827062in}}%
\pgfpathlineto{\pgfqpoint{3.497814in}{1.818070in}}%
\pgfpathlineto{\pgfqpoint{3.489597in}{1.809091in}}%
\pgfpathlineto{\pgfqpoint{3.481373in}{1.800126in}}%
\pgfpathclose%
\pgfusepath{fill}%
\end{pgfscope}%
\begin{pgfscope}%
\pgfpathrectangle{\pgfqpoint{1.150000in}{0.150000in}}{\pgfqpoint{5.700000in}{5.700000in}}%
\pgfusepath{clip}%
\pgfsetbuttcap%
\pgfsetroundjoin%
\definecolor{currentfill}{rgb}{0.274952,0.037752,0.364543}%
\pgfsetfillcolor{currentfill}%
\pgfsetfillopacity{0.700000}%
\pgfsetlinewidth{0.000000pt}%
\definecolor{currentstroke}{rgb}{0.000000,0.000000,0.000000}%
\pgfsetstrokecolor{currentstroke}%
\pgfsetdash{}{0pt}%
\pgfpathmoveto{\pgfqpoint{2.821205in}{1.836967in}}%
\pgfpathlineto{\pgfqpoint{2.834881in}{1.829545in}}%
\pgfpathlineto{\pgfqpoint{2.848559in}{1.822219in}}%
\pgfpathlineto{\pgfqpoint{2.862239in}{1.814991in}}%
\pgfpathlineto{\pgfqpoint{2.875922in}{1.807859in}}%
\pgfpathlineto{\pgfqpoint{2.884437in}{1.813980in}}%
\pgfpathlineto{\pgfqpoint{2.892943in}{1.820213in}}%
\pgfpathlineto{\pgfqpoint{2.901439in}{1.826553in}}%
\pgfpathlineto{\pgfqpoint{2.909925in}{1.832998in}}%
\pgfpathlineto{\pgfqpoint{2.896265in}{1.839883in}}%
\pgfpathlineto{\pgfqpoint{2.882607in}{1.846864in}}%
\pgfpathlineto{\pgfqpoint{2.868951in}{1.853942in}}%
\pgfpathlineto{\pgfqpoint{2.855298in}{1.861117in}}%
\pgfpathlineto{\pgfqpoint{2.846789in}{1.854912in}}%
\pgfpathlineto{\pgfqpoint{2.838271in}{1.848817in}}%
\pgfpathlineto{\pgfqpoint{2.829743in}{1.842834in}}%
\pgfpathlineto{\pgfqpoint{2.821205in}{1.836967in}}%
\pgfpathclose%
\pgfusepath{fill}%
\end{pgfscope}%
\begin{pgfscope}%
\pgfpathrectangle{\pgfqpoint{1.150000in}{0.150000in}}{\pgfqpoint{5.700000in}{5.700000in}}%
\pgfusepath{clip}%
\pgfsetbuttcap%
\pgfsetroundjoin%
\definecolor{currentfill}{rgb}{0.281412,0.155834,0.469201}%
\pgfsetfillcolor{currentfill}%
\pgfsetfillopacity{0.700000}%
\pgfsetlinewidth{0.000000pt}%
\definecolor{currentstroke}{rgb}{0.000000,0.000000,0.000000}%
\pgfsetstrokecolor{currentstroke}%
\pgfsetdash{}{0pt}%
\pgfpathmoveto{\pgfqpoint{4.292814in}{2.043971in}}%
\pgfpathlineto{\pgfqpoint{4.306762in}{2.044431in}}%
\pgfpathlineto{\pgfqpoint{4.320720in}{2.044965in}}%
\pgfpathlineto{\pgfqpoint{4.334687in}{2.045572in}}%
\pgfpathlineto{\pgfqpoint{4.348662in}{2.046253in}}%
\pgfpathlineto{\pgfqpoint{4.356582in}{2.055004in}}%
\pgfpathlineto{\pgfqpoint{4.364496in}{2.063694in}}%
\pgfpathlineto{\pgfqpoint{4.372404in}{2.072324in}}%
\pgfpathlineto{\pgfqpoint{4.380305in}{2.080894in}}%
\pgfpathlineto{\pgfqpoint{4.366340in}{2.080236in}}%
\pgfpathlineto{\pgfqpoint{4.352384in}{2.079651in}}%
\pgfpathlineto{\pgfqpoint{4.338437in}{2.079139in}}%
\pgfpathlineto{\pgfqpoint{4.324499in}{2.078701in}}%
\pgfpathlineto{\pgfqpoint{4.316587in}{2.070101in}}%
\pgfpathlineto{\pgfqpoint{4.308668in}{2.061446in}}%
\pgfpathlineto{\pgfqpoint{4.300744in}{2.052736in}}%
\pgfpathlineto{\pgfqpoint{4.292814in}{2.043971in}}%
\pgfpathclose%
\pgfusepath{fill}%
\end{pgfscope}%
\begin{pgfscope}%
\pgfpathrectangle{\pgfqpoint{1.150000in}{0.150000in}}{\pgfqpoint{5.700000in}{5.700000in}}%
\pgfusepath{clip}%
\pgfsetbuttcap%
\pgfsetroundjoin%
\definecolor{currentfill}{rgb}{0.268510,0.009605,0.335427}%
\pgfsetfillcolor{currentfill}%
\pgfsetfillopacity{0.700000}%
\pgfsetlinewidth{0.000000pt}%
\definecolor{currentstroke}{rgb}{0.000000,0.000000,0.000000}%
\pgfsetstrokecolor{currentstroke}%
\pgfsetdash{}{0pt}%
\pgfpathmoveto{\pgfqpoint{3.250631in}{1.778210in}}%
\pgfpathlineto{\pgfqpoint{3.264331in}{1.773659in}}%
\pgfpathlineto{\pgfqpoint{3.278036in}{1.769193in}}%
\pgfpathlineto{\pgfqpoint{3.291745in}{1.764813in}}%
\pgfpathlineto{\pgfqpoint{3.305459in}{1.760518in}}%
\pgfpathlineto{\pgfqpoint{3.313764in}{1.768907in}}%
\pgfpathlineto{\pgfqpoint{3.322062in}{1.777338in}}%
\pgfpathlineto{\pgfqpoint{3.330352in}{1.785810in}}%
\pgfpathlineto{\pgfqpoint{3.338636in}{1.794319in}}%
\pgfpathlineto{\pgfqpoint{3.324938in}{1.798430in}}%
\pgfpathlineto{\pgfqpoint{3.311244in}{1.802626in}}%
\pgfpathlineto{\pgfqpoint{3.297555in}{1.806908in}}%
\pgfpathlineto{\pgfqpoint{3.283871in}{1.811275in}}%
\pgfpathlineto{\pgfqpoint{3.275571in}{1.802942in}}%
\pgfpathlineto{\pgfqpoint{3.267265in}{1.794652in}}%
\pgfpathlineto{\pgfqpoint{3.258952in}{1.786407in}}%
\pgfpathlineto{\pgfqpoint{3.250631in}{1.778210in}}%
\pgfpathclose%
\pgfusepath{fill}%
\end{pgfscope}%
\begin{pgfscope}%
\pgfpathrectangle{\pgfqpoint{1.150000in}{0.150000in}}{\pgfqpoint{5.700000in}{5.700000in}}%
\pgfusepath{clip}%
\pgfsetbuttcap%
\pgfsetroundjoin%
\definecolor{currentfill}{rgb}{0.201239,0.383670,0.554294}%
\pgfsetfillcolor{currentfill}%
\pgfsetfillopacity{0.700000}%
\pgfsetlinewidth{0.000000pt}%
\definecolor{currentstroke}{rgb}{0.000000,0.000000,0.000000}%
\pgfsetstrokecolor{currentstroke}%
\pgfsetdash{}{0pt}%
\pgfpathmoveto{\pgfqpoint{5.773097in}{2.572002in}}%
\pgfpathlineto{\pgfqpoint{5.787592in}{2.574958in}}%
\pgfpathlineto{\pgfqpoint{5.802099in}{2.577981in}}%
\pgfpathlineto{\pgfqpoint{5.816619in}{2.581073in}}%
\pgfpathlineto{\pgfqpoint{5.823841in}{2.584654in}}%
\pgfpathlineto{\pgfqpoint{5.831059in}{2.588283in}}%
\pgfpathlineto{\pgfqpoint{5.838272in}{2.591965in}}%
\pgfpathlineto{\pgfqpoint{5.845481in}{2.595706in}}%
\pgfpathlineto{\pgfqpoint{5.830988in}{2.592956in}}%
\pgfpathlineto{\pgfqpoint{5.816508in}{2.590274in}}%
\pgfpathlineto{\pgfqpoint{5.802040in}{2.587659in}}%
\pgfpathlineto{\pgfqpoint{5.794811in}{2.583656in}}%
\pgfpathlineto{\pgfqpoint{5.787578in}{2.579717in}}%
\pgfpathlineto{\pgfqpoint{5.780340in}{2.575834in}}%
\pgfpathlineto{\pgfqpoint{5.773097in}{2.572002in}}%
\pgfpathclose%
\pgfusepath{fill}%
\end{pgfscope}%
\begin{pgfscope}%
\pgfpathrectangle{\pgfqpoint{1.150000in}{0.150000in}}{\pgfqpoint{5.700000in}{5.700000in}}%
\pgfusepath{clip}%
\pgfsetbuttcap%
\pgfsetroundjoin%
\definecolor{currentfill}{rgb}{0.283187,0.125848,0.444960}%
\pgfsetfillcolor{currentfill}%
\pgfsetfillopacity{0.700000}%
\pgfsetlinewidth{0.000000pt}%
\definecolor{currentstroke}{rgb}{0.000000,0.000000,0.000000}%
\pgfsetstrokecolor{currentstroke}%
\pgfsetdash{}{0pt}%
\pgfpathmoveto{\pgfqpoint{2.423392in}{2.017219in}}%
\pgfpathlineto{\pgfqpoint{2.437105in}{2.006690in}}%
\pgfpathlineto{\pgfqpoint{2.450817in}{1.996277in}}%
\pgfpathlineto{\pgfqpoint{2.464528in}{1.985978in}}%
\pgfpathlineto{\pgfqpoint{2.478239in}{1.975792in}}%
\pgfpathlineto{\pgfqpoint{2.487005in}{1.979093in}}%
\pgfpathlineto{\pgfqpoint{2.495756in}{1.982568in}}%
\pgfpathlineto{\pgfqpoint{2.504494in}{1.986215in}}%
\pgfpathlineto{\pgfqpoint{2.513219in}{1.990028in}}%
\pgfpathlineto{\pgfqpoint{2.499538in}{1.999921in}}%
\pgfpathlineto{\pgfqpoint{2.485856in}{2.009927in}}%
\pgfpathlineto{\pgfqpoint{2.472174in}{2.020047in}}%
\pgfpathlineto{\pgfqpoint{2.458491in}{2.030282in}}%
\pgfpathlineto{\pgfqpoint{2.449737in}{2.026754in}}%
\pgfpathlineto{\pgfqpoint{2.440969in}{2.023398in}}%
\pgfpathlineto{\pgfqpoint{2.432187in}{2.020218in}}%
\pgfpathlineto{\pgfqpoint{2.423392in}{2.017219in}}%
\pgfpathclose%
\pgfusepath{fill}%
\end{pgfscope}%
\begin{pgfscope}%
\pgfpathrectangle{\pgfqpoint{1.150000in}{0.150000in}}{\pgfqpoint{5.700000in}{5.700000in}}%
\pgfusepath{clip}%
\pgfsetbuttcap%
\pgfsetroundjoin%
\definecolor{currentfill}{rgb}{0.257322,0.256130,0.526563}%
\pgfsetfillcolor{currentfill}%
\pgfsetfillopacity{0.700000}%
\pgfsetlinewidth{0.000000pt}%
\definecolor{currentstroke}{rgb}{0.000000,0.000000,0.000000}%
\pgfsetstrokecolor{currentstroke}%
\pgfsetdash{}{0pt}%
\pgfpathmoveto{\pgfqpoint{2.057546in}{2.304582in}}%
\pgfpathlineto{\pgfqpoint{2.071355in}{2.290660in}}%
\pgfpathlineto{\pgfqpoint{2.085159in}{2.276879in}}%
\pgfpathlineto{\pgfqpoint{2.098959in}{2.263239in}}%
\pgfpathlineto{\pgfqpoint{2.112755in}{2.249738in}}%
\pgfpathlineto{\pgfqpoint{2.121787in}{2.250336in}}%
\pgfpathlineto{\pgfqpoint{2.130801in}{2.251161in}}%
\pgfpathlineto{\pgfqpoint{2.139798in}{2.252209in}}%
\pgfpathlineto{\pgfqpoint{2.148778in}{2.253473in}}%
\pgfpathlineto{\pgfqpoint{2.135019in}{2.266653in}}%
\pgfpathlineto{\pgfqpoint{2.121256in}{2.279970in}}%
\pgfpathlineto{\pgfqpoint{2.107490in}{2.293428in}}%
\pgfpathlineto{\pgfqpoint{2.093720in}{2.307028in}}%
\pgfpathlineto{\pgfqpoint{2.084703in}{2.306077in}}%
\pgfpathlineto{\pgfqpoint{2.075669in}{2.305350in}}%
\pgfpathlineto{\pgfqpoint{2.066617in}{2.304849in}}%
\pgfpathlineto{\pgfqpoint{2.057546in}{2.304582in}}%
\pgfpathclose%
\pgfusepath{fill}%
\end{pgfscope}%
\begin{pgfscope}%
\pgfpathrectangle{\pgfqpoint{1.150000in}{0.150000in}}{\pgfqpoint{5.700000in}{5.700000in}}%
\pgfusepath{clip}%
\pgfsetbuttcap%
\pgfsetroundjoin%
\definecolor{currentfill}{rgb}{0.255645,0.260703,0.528312}%
\pgfsetfillcolor{currentfill}%
\pgfsetfillopacity{0.700000}%
\pgfsetlinewidth{0.000000pt}%
\definecolor{currentstroke}{rgb}{0.000000,0.000000,0.000000}%
\pgfsetstrokecolor{currentstroke}%
\pgfsetdash{}{0pt}%
\pgfpathmoveto{\pgfqpoint{4.873732in}{2.269198in}}%
\pgfpathlineto{\pgfqpoint{4.887889in}{2.271226in}}%
\pgfpathlineto{\pgfqpoint{4.902056in}{2.273325in}}%
\pgfpathlineto{\pgfqpoint{4.916235in}{2.275494in}}%
\pgfpathlineto{\pgfqpoint{4.930424in}{2.277733in}}%
\pgfpathlineto{\pgfqpoint{4.938102in}{2.284428in}}%
\pgfpathlineto{\pgfqpoint{4.945773in}{2.291067in}}%
\pgfpathlineto{\pgfqpoint{4.953437in}{2.297655in}}%
\pgfpathlineto{\pgfqpoint{4.961094in}{2.304194in}}%
\pgfpathlineto{\pgfqpoint{4.946920in}{2.302104in}}%
\pgfpathlineto{\pgfqpoint{4.932756in}{2.300084in}}%
\pgfpathlineto{\pgfqpoint{4.918603in}{2.298135in}}%
\pgfpathlineto{\pgfqpoint{4.904461in}{2.296257in}}%
\pgfpathlineto{\pgfqpoint{4.896789in}{2.289561in}}%
\pgfpathlineto{\pgfqpoint{4.889110in}{2.282821in}}%
\pgfpathlineto{\pgfqpoint{4.881425in}{2.276034in}}%
\pgfpathlineto{\pgfqpoint{4.873732in}{2.269198in}}%
\pgfpathclose%
\pgfusepath{fill}%
\end{pgfscope}%
\begin{pgfscope}%
\pgfpathrectangle{\pgfqpoint{1.150000in}{0.150000in}}{\pgfqpoint{5.700000in}{5.700000in}}%
\pgfusepath{clip}%
\pgfsetbuttcap%
\pgfsetroundjoin%
\definecolor{currentfill}{rgb}{0.282623,0.140926,0.457517}%
\pgfsetfillcolor{currentfill}%
\pgfsetfillopacity{0.700000}%
\pgfsetlinewidth{0.000000pt}%
\definecolor{currentstroke}{rgb}{0.000000,0.000000,0.000000}%
\pgfsetstrokecolor{currentstroke}%
\pgfsetdash{}{0pt}%
\pgfpathmoveto{\pgfqpoint{4.205290in}{2.007227in}}%
\pgfpathlineto{\pgfqpoint{4.219212in}{2.007393in}}%
\pgfpathlineto{\pgfqpoint{4.233144in}{2.007633in}}%
\pgfpathlineto{\pgfqpoint{4.247084in}{2.007947in}}%
\pgfpathlineto{\pgfqpoint{4.261034in}{2.008335in}}%
\pgfpathlineto{\pgfqpoint{4.268988in}{2.017331in}}%
\pgfpathlineto{\pgfqpoint{4.276936in}{2.026268in}}%
\pgfpathlineto{\pgfqpoint{4.284878in}{2.035148in}}%
\pgfpathlineto{\pgfqpoint{4.292814in}{2.043971in}}%
\pgfpathlineto{\pgfqpoint{4.278875in}{2.043584in}}%
\pgfpathlineto{\pgfqpoint{4.264944in}{2.043271in}}%
\pgfpathlineto{\pgfqpoint{4.251023in}{2.043033in}}%
\pgfpathlineto{\pgfqpoint{4.237110in}{2.042868in}}%
\pgfpathlineto{\pgfqpoint{4.229164in}{2.034037in}}%
\pgfpathlineto{\pgfqpoint{4.221212in}{2.025153in}}%
\pgfpathlineto{\pgfqpoint{4.213254in}{2.016216in}}%
\pgfpathlineto{\pgfqpoint{4.205290in}{2.007227in}}%
\pgfpathclose%
\pgfusepath{fill}%
\end{pgfscope}%
\begin{pgfscope}%
\pgfpathrectangle{\pgfqpoint{1.150000in}{0.150000in}}{\pgfqpoint{5.700000in}{5.700000in}}%
\pgfusepath{clip}%
\pgfsetbuttcap%
\pgfsetroundjoin%
\definecolor{currentfill}{rgb}{0.223925,0.334994,0.548053}%
\pgfsetfillcolor{currentfill}%
\pgfsetfillopacity{0.700000}%
\pgfsetlinewidth{0.000000pt}%
\definecolor{currentstroke}{rgb}{0.000000,0.000000,0.000000}%
\pgfsetstrokecolor{currentstroke}%
\pgfsetdash{}{0pt}%
\pgfpathmoveto{\pgfqpoint{5.367190in}{2.445188in}}%
\pgfpathlineto{\pgfqpoint{5.381534in}{2.447930in}}%
\pgfpathlineto{\pgfqpoint{5.395890in}{2.450741in}}%
\pgfpathlineto{\pgfqpoint{5.410257in}{2.453621in}}%
\pgfpathlineto{\pgfqpoint{5.424637in}{2.456570in}}%
\pgfpathlineto{\pgfqpoint{5.432070in}{2.461327in}}%
\pgfpathlineto{\pgfqpoint{5.439496in}{2.466073in}}%
\pgfpathlineto{\pgfqpoint{5.446916in}{2.470810in}}%
\pgfpathlineto{\pgfqpoint{5.454329in}{2.475545in}}%
\pgfpathlineto{\pgfqpoint{5.439971in}{2.472853in}}%
\pgfpathlineto{\pgfqpoint{5.425624in}{2.470230in}}%
\pgfpathlineto{\pgfqpoint{5.411289in}{2.467675in}}%
\pgfpathlineto{\pgfqpoint{5.396966in}{2.465189in}}%
\pgfpathlineto{\pgfqpoint{5.389532in}{2.460191in}}%
\pgfpathlineto{\pgfqpoint{5.382091in}{2.455194in}}%
\pgfpathlineto{\pgfqpoint{5.374644in}{2.450195in}}%
\pgfpathlineto{\pgfqpoint{5.367190in}{2.445188in}}%
\pgfpathclose%
\pgfusepath{fill}%
\end{pgfscope}%
\begin{pgfscope}%
\pgfpathrectangle{\pgfqpoint{1.150000in}{0.150000in}}{\pgfqpoint{5.700000in}{5.700000in}}%
\pgfusepath{clip}%
\pgfsetbuttcap%
\pgfsetroundjoin%
\definecolor{currentfill}{rgb}{0.277941,0.056324,0.381191}%
\pgfsetfillcolor{currentfill}%
\pgfsetfillopacity{0.700000}%
\pgfsetlinewidth{0.000000pt}%
\definecolor{currentstroke}{rgb}{0.000000,0.000000,0.000000}%
\pgfsetstrokecolor{currentstroke}%
\pgfsetdash{}{0pt}%
\pgfpathmoveto{\pgfqpoint{3.711904in}{1.842299in}}%
\pgfpathlineto{\pgfqpoint{3.725689in}{1.840330in}}%
\pgfpathlineto{\pgfqpoint{3.739481in}{1.838440in}}%
\pgfpathlineto{\pgfqpoint{3.753281in}{1.836628in}}%
\pgfpathlineto{\pgfqpoint{3.767087in}{1.834894in}}%
\pgfpathlineto{\pgfqpoint{3.775216in}{1.844364in}}%
\pgfpathlineto{\pgfqpoint{3.783339in}{1.853816in}}%
\pgfpathlineto{\pgfqpoint{3.791456in}{1.863247in}}%
\pgfpathlineto{\pgfqpoint{3.799567in}{1.872657in}}%
\pgfpathlineto{\pgfqpoint{3.785772in}{1.874289in}}%
\pgfpathlineto{\pgfqpoint{3.771984in}{1.875999in}}%
\pgfpathlineto{\pgfqpoint{3.758203in}{1.877787in}}%
\pgfpathlineto{\pgfqpoint{3.744428in}{1.879654in}}%
\pgfpathlineto{\pgfqpoint{3.736306in}{1.870338in}}%
\pgfpathlineto{\pgfqpoint{3.728177in}{1.861007in}}%
\pgfpathlineto{\pgfqpoint{3.720043in}{1.851660in}}%
\pgfpathlineto{\pgfqpoint{3.711904in}{1.842299in}}%
\pgfpathclose%
\pgfusepath{fill}%
\end{pgfscope}%
\begin{pgfscope}%
\pgfpathrectangle{\pgfqpoint{1.150000in}{0.150000in}}{\pgfqpoint{5.700000in}{5.700000in}}%
\pgfusepath{clip}%
\pgfsetbuttcap%
\pgfsetroundjoin%
\definecolor{currentfill}{rgb}{0.265145,0.232956,0.516599}%
\pgfsetfillcolor{currentfill}%
\pgfsetfillopacity{0.700000}%
\pgfsetlinewidth{0.000000pt}%
\definecolor{currentstroke}{rgb}{0.000000,0.000000,0.000000}%
\pgfsetstrokecolor{currentstroke}%
\pgfsetdash{}{0pt}%
\pgfpathmoveto{\pgfqpoint{2.112755in}{2.249738in}}%
\pgfpathlineto{\pgfqpoint{2.126547in}{2.236375in}}%
\pgfpathlineto{\pgfqpoint{2.140336in}{2.223148in}}%
\pgfpathlineto{\pgfqpoint{2.154121in}{2.210057in}}%
\pgfpathlineto{\pgfqpoint{2.167902in}{2.197100in}}%
\pgfpathlineto{\pgfqpoint{2.176897in}{2.198028in}}%
\pgfpathlineto{\pgfqpoint{2.185874in}{2.199177in}}%
\pgfpathlineto{\pgfqpoint{2.194835in}{2.200542in}}%
\pgfpathlineto{\pgfqpoint{2.203779in}{2.202119in}}%
\pgfpathlineto{\pgfqpoint{2.190033in}{2.214756in}}%
\pgfpathlineto{\pgfqpoint{2.176285in}{2.227526in}}%
\pgfpathlineto{\pgfqpoint{2.162533in}{2.240432in}}%
\pgfpathlineto{\pgfqpoint{2.148778in}{2.253473in}}%
\pgfpathlineto{\pgfqpoint{2.139798in}{2.252209in}}%
\pgfpathlineto{\pgfqpoint{2.130801in}{2.251161in}}%
\pgfpathlineto{\pgfqpoint{2.121787in}{2.250336in}}%
\pgfpathlineto{\pgfqpoint{2.112755in}{2.249738in}}%
\pgfpathclose%
\pgfusepath{fill}%
\end{pgfscope}%
\begin{pgfscope}%
\pgfpathrectangle{\pgfqpoint{1.150000in}{0.150000in}}{\pgfqpoint{5.700000in}{5.700000in}}%
\pgfusepath{clip}%
\pgfsetbuttcap%
\pgfsetroundjoin%
\definecolor{currentfill}{rgb}{0.260571,0.246922,0.522828}%
\pgfsetfillcolor{currentfill}%
\pgfsetfillopacity{0.700000}%
\pgfsetlinewidth{0.000000pt}%
\definecolor{currentstroke}{rgb}{0.000000,0.000000,0.000000}%
\pgfsetstrokecolor{currentstroke}%
\pgfsetdash{}{0pt}%
\pgfpathmoveto{\pgfqpoint{4.786320in}{2.233390in}}%
\pgfpathlineto{\pgfqpoint{4.800448in}{2.235262in}}%
\pgfpathlineto{\pgfqpoint{4.814586in}{2.237206in}}%
\pgfpathlineto{\pgfqpoint{4.828735in}{2.239220in}}%
\pgfpathlineto{\pgfqpoint{4.842894in}{2.241306in}}%
\pgfpathlineto{\pgfqpoint{4.850614in}{2.248366in}}%
\pgfpathlineto{\pgfqpoint{4.858327in}{2.255367in}}%
\pgfpathlineto{\pgfqpoint{4.866033in}{2.262310in}}%
\pgfpathlineto{\pgfqpoint{4.873732in}{2.269198in}}%
\pgfpathlineto{\pgfqpoint{4.859587in}{2.267241in}}%
\pgfpathlineto{\pgfqpoint{4.845452in}{2.265355in}}%
\pgfpathlineto{\pgfqpoint{4.831327in}{2.263540in}}%
\pgfpathlineto{\pgfqpoint{4.817213in}{2.261796in}}%
\pgfpathlineto{\pgfqpoint{4.809500in}{2.254772in}}%
\pgfpathlineto{\pgfqpoint{4.801780in}{2.247698in}}%
\pgfpathlineto{\pgfqpoint{4.794054in}{2.240571in}}%
\pgfpathlineto{\pgfqpoint{4.786320in}{2.233390in}}%
\pgfpathclose%
\pgfusepath{fill}%
\end{pgfscope}%
\begin{pgfscope}%
\pgfpathrectangle{\pgfqpoint{1.150000in}{0.150000in}}{\pgfqpoint{5.700000in}{5.700000in}}%
\pgfusepath{clip}%
\pgfsetbuttcap%
\pgfsetroundjoin%
\definecolor{currentfill}{rgb}{0.278791,0.062145,0.386592}%
\pgfsetfillcolor{currentfill}%
\pgfsetfillopacity{0.700000}%
\pgfsetlinewidth{0.000000pt}%
\definecolor{currentstroke}{rgb}{0.000000,0.000000,0.000000}%
\pgfsetstrokecolor{currentstroke}%
\pgfsetdash{}{0pt}%
\pgfpathmoveto{\pgfqpoint{2.677412in}{1.879851in}}%
\pgfpathlineto{\pgfqpoint{2.691099in}{1.871358in}}%
\pgfpathlineto{\pgfqpoint{2.704788in}{1.862967in}}%
\pgfpathlineto{\pgfqpoint{2.718477in}{1.854678in}}%
\pgfpathlineto{\pgfqpoint{2.732169in}{1.846491in}}%
\pgfpathlineto{\pgfqpoint{2.740773in}{1.851577in}}%
\pgfpathlineto{\pgfqpoint{2.749367in}{1.856801in}}%
\pgfpathlineto{\pgfqpoint{2.757950in}{1.862158in}}%
\pgfpathlineto{\pgfqpoint{2.766522in}{1.867644in}}%
\pgfpathlineto{\pgfqpoint{2.752855in}{1.875562in}}%
\pgfpathlineto{\pgfqpoint{2.739190in}{1.883582in}}%
\pgfpathlineto{\pgfqpoint{2.725527in}{1.891704in}}%
\pgfpathlineto{\pgfqpoint{2.711865in}{1.899927in}}%
\pgfpathlineto{\pgfqpoint{2.703269in}{1.894703in}}%
\pgfpathlineto{\pgfqpoint{2.694661in}{1.889612in}}%
\pgfpathlineto{\pgfqpoint{2.686042in}{1.884660in}}%
\pgfpathlineto{\pgfqpoint{2.677412in}{1.879851in}}%
\pgfpathclose%
\pgfusepath{fill}%
\end{pgfscope}%
\begin{pgfscope}%
\pgfpathrectangle{\pgfqpoint{1.150000in}{0.150000in}}{\pgfqpoint{5.700000in}{5.700000in}}%
\pgfusepath{clip}%
\pgfsetbuttcap%
\pgfsetroundjoin%
\definecolor{currentfill}{rgb}{0.283187,0.125848,0.444960}%
\pgfsetfillcolor{currentfill}%
\pgfsetfillopacity{0.700000}%
\pgfsetlinewidth{0.000000pt}%
\definecolor{currentstroke}{rgb}{0.000000,0.000000,0.000000}%
\pgfsetstrokecolor{currentstroke}%
\pgfsetdash{}{0pt}%
\pgfpathmoveto{\pgfqpoint{4.117731in}{1.970889in}}%
\pgfpathlineto{\pgfqpoint{4.131630in}{1.970738in}}%
\pgfpathlineto{\pgfqpoint{4.145536in}{1.970661in}}%
\pgfpathlineto{\pgfqpoint{4.159451in}{1.970659in}}%
\pgfpathlineto{\pgfqpoint{4.173375in}{1.970731in}}%
\pgfpathlineto{\pgfqpoint{4.181363in}{1.979936in}}%
\pgfpathlineto{\pgfqpoint{4.189344in}{1.989087in}}%
\pgfpathlineto{\pgfqpoint{4.197320in}{1.998184in}}%
\pgfpathlineto{\pgfqpoint{4.205290in}{2.007227in}}%
\pgfpathlineto{\pgfqpoint{4.191376in}{2.007135in}}%
\pgfpathlineto{\pgfqpoint{4.177470in}{2.007118in}}%
\pgfpathlineto{\pgfqpoint{4.163574in}{2.007176in}}%
\pgfpathlineto{\pgfqpoint{4.149686in}{2.007308in}}%
\pgfpathlineto{\pgfqpoint{4.141706in}{1.998276in}}%
\pgfpathlineto{\pgfqpoint{4.133720in}{1.989196in}}%
\pgfpathlineto{\pgfqpoint{4.125729in}{1.980067in}}%
\pgfpathlineto{\pgfqpoint{4.117731in}{1.970889in}}%
\pgfpathclose%
\pgfusepath{fill}%
\end{pgfscope}%
\begin{pgfscope}%
\pgfpathrectangle{\pgfqpoint{1.150000in}{0.150000in}}{\pgfqpoint{5.700000in}{5.700000in}}%
\pgfusepath{clip}%
\pgfsetbuttcap%
\pgfsetroundjoin%
\definecolor{currentfill}{rgb}{0.269944,0.014625,0.341379}%
\pgfsetfillcolor{currentfill}%
\pgfsetfillopacity{0.700000}%
\pgfsetlinewidth{0.000000pt}%
\definecolor{currentstroke}{rgb}{0.000000,0.000000,0.000000}%
\pgfsetstrokecolor{currentstroke}%
\pgfsetdash{}{0pt}%
\pgfpathmoveto{\pgfqpoint{3.393483in}{1.778718in}}%
\pgfpathlineto{\pgfqpoint{3.407208in}{1.775027in}}%
\pgfpathlineto{\pgfqpoint{3.420939in}{1.771419in}}%
\pgfpathlineto{\pgfqpoint{3.434674in}{1.767894in}}%
\pgfpathlineto{\pgfqpoint{3.448416in}{1.764451in}}%
\pgfpathlineto{\pgfqpoint{3.456665in}{1.773338in}}%
\pgfpathlineto{\pgfqpoint{3.464907in}{1.782248in}}%
\pgfpathlineto{\pgfqpoint{3.473143in}{1.791178in}}%
\pgfpathlineto{\pgfqpoint{3.481373in}{1.800126in}}%
\pgfpathlineto{\pgfqpoint{3.467645in}{1.803406in}}%
\pgfpathlineto{\pgfqpoint{3.453923in}{1.806768in}}%
\pgfpathlineto{\pgfqpoint{3.440207in}{1.810212in}}%
\pgfpathlineto{\pgfqpoint{3.426496in}{1.813740in}}%
\pgfpathlineto{\pgfqpoint{3.418252in}{1.804947in}}%
\pgfpathlineto{\pgfqpoint{3.410002in}{1.796178in}}%
\pgfpathlineto{\pgfqpoint{3.401746in}{1.787434in}}%
\pgfpathlineto{\pgfqpoint{3.393483in}{1.778718in}}%
\pgfpathclose%
\pgfusepath{fill}%
\end{pgfscope}%
\begin{pgfscope}%
\pgfpathrectangle{\pgfqpoint{1.150000in}{0.150000in}}{\pgfqpoint{5.700000in}{5.700000in}}%
\pgfusepath{clip}%
\pgfsetbuttcap%
\pgfsetroundjoin%
\definecolor{currentfill}{rgb}{0.227802,0.326594,0.546532}%
\pgfsetfillcolor{currentfill}%
\pgfsetfillopacity{0.700000}%
\pgfsetlinewidth{0.000000pt}%
\definecolor{currentstroke}{rgb}{0.000000,0.000000,0.000000}%
\pgfsetstrokecolor{currentstroke}%
\pgfsetdash{}{0pt}%
\pgfpathmoveto{\pgfqpoint{5.279973in}{2.413787in}}%
\pgfpathlineto{\pgfqpoint{5.294290in}{2.416487in}}%
\pgfpathlineto{\pgfqpoint{5.308618in}{2.419257in}}%
\pgfpathlineto{\pgfqpoint{5.322957in}{2.422096in}}%
\pgfpathlineto{\pgfqpoint{5.337309in}{2.425005in}}%
\pgfpathlineto{\pgfqpoint{5.344790in}{2.430083in}}%
\pgfpathlineto{\pgfqpoint{5.352263in}{2.435137in}}%
\pgfpathlineto{\pgfqpoint{5.359730in}{2.440171in}}%
\pgfpathlineto{\pgfqpoint{5.367190in}{2.445188in}}%
\pgfpathlineto{\pgfqpoint{5.352858in}{2.442516in}}%
\pgfpathlineto{\pgfqpoint{5.338538in}{2.439912in}}%
\pgfpathlineto{\pgfqpoint{5.324230in}{2.437377in}}%
\pgfpathlineto{\pgfqpoint{5.309933in}{2.434911in}}%
\pgfpathlineto{\pgfqpoint{5.302453in}{2.429651in}}%
\pgfpathlineto{\pgfqpoint{5.294966in}{2.424379in}}%
\pgfpathlineto{\pgfqpoint{5.287473in}{2.419093in}}%
\pgfpathlineto{\pgfqpoint{5.279973in}{2.413787in}}%
\pgfpathclose%
\pgfusepath{fill}%
\end{pgfscope}%
\begin{pgfscope}%
\pgfpathrectangle{\pgfqpoint{1.150000in}{0.150000in}}{\pgfqpoint{5.700000in}{5.700000in}}%
\pgfusepath{clip}%
\pgfsetbuttcap%
\pgfsetroundjoin%
\definecolor{currentfill}{rgb}{0.283091,0.110553,0.431554}%
\pgfsetfillcolor{currentfill}%
\pgfsetfillopacity{0.700000}%
\pgfsetlinewidth{0.000000pt}%
\definecolor{currentstroke}{rgb}{0.000000,0.000000,0.000000}%
\pgfsetstrokecolor{currentstroke}%
\pgfsetdash{}{0pt}%
\pgfpathmoveto{\pgfqpoint{2.478239in}{1.975792in}}%
\pgfpathlineto{\pgfqpoint{2.491950in}{1.965720in}}%
\pgfpathlineto{\pgfqpoint{2.505660in}{1.955758in}}%
\pgfpathlineto{\pgfqpoint{2.519370in}{1.945908in}}%
\pgfpathlineto{\pgfqpoint{2.533081in}{1.936168in}}%
\pgfpathlineto{\pgfqpoint{2.541816in}{1.939768in}}%
\pgfpathlineto{\pgfqpoint{2.550538in}{1.943539in}}%
\pgfpathlineto{\pgfqpoint{2.559247in}{1.947475in}}%
\pgfpathlineto{\pgfqpoint{2.567944in}{1.951572in}}%
\pgfpathlineto{\pgfqpoint{2.554263in}{1.961020in}}%
\pgfpathlineto{\pgfqpoint{2.540581in}{1.970579in}}%
\pgfpathlineto{\pgfqpoint{2.526900in}{1.980247in}}%
\pgfpathlineto{\pgfqpoint{2.513219in}{1.990028in}}%
\pgfpathlineto{\pgfqpoint{2.504494in}{1.986215in}}%
\pgfpathlineto{\pgfqpoint{2.495756in}{1.982568in}}%
\pgfpathlineto{\pgfqpoint{2.487005in}{1.979093in}}%
\pgfpathlineto{\pgfqpoint{2.478239in}{1.975792in}}%
\pgfpathclose%
\pgfusepath{fill}%
\end{pgfscope}%
\begin{pgfscope}%
\pgfpathrectangle{\pgfqpoint{1.150000in}{0.150000in}}{\pgfqpoint{5.700000in}{5.700000in}}%
\pgfusepath{clip}%
\pgfsetbuttcap%
\pgfsetroundjoin%
\definecolor{currentfill}{rgb}{0.269944,0.014625,0.341379}%
\pgfsetfillcolor{currentfill}%
\pgfsetfillopacity{0.700000}%
\pgfsetlinewidth{0.000000pt}%
\definecolor{currentstroke}{rgb}{0.000000,0.000000,0.000000}%
\pgfsetstrokecolor{currentstroke}%
\pgfsetdash{}{0pt}%
\pgfpathmoveto{\pgfqpoint{3.019312in}{1.781313in}}%
\pgfpathlineto{\pgfqpoint{3.032999in}{1.775269in}}%
\pgfpathlineto{\pgfqpoint{3.046690in}{1.769317in}}%
\pgfpathlineto{\pgfqpoint{3.060384in}{1.763454in}}%
\pgfpathlineto{\pgfqpoint{3.074082in}{1.757682in}}%
\pgfpathlineto{\pgfqpoint{3.082499in}{1.764928in}}%
\pgfpathlineto{\pgfqpoint{3.090908in}{1.772255in}}%
\pgfpathlineto{\pgfqpoint{3.099308in}{1.779661in}}%
\pgfpathlineto{\pgfqpoint{3.107701in}{1.787142in}}%
\pgfpathlineto{\pgfqpoint{3.094022in}{1.792688in}}%
\pgfpathlineto{\pgfqpoint{3.080347in}{1.798325in}}%
\pgfpathlineto{\pgfqpoint{3.066675in}{1.804052in}}%
\pgfpathlineto{\pgfqpoint{3.053007in}{1.809870in}}%
\pgfpathlineto{\pgfqpoint{3.044596in}{1.802607in}}%
\pgfpathlineto{\pgfqpoint{3.036176in}{1.795424in}}%
\pgfpathlineto{\pgfqpoint{3.027748in}{1.788325in}}%
\pgfpathlineto{\pgfqpoint{3.019312in}{1.781313in}}%
\pgfpathclose%
\pgfusepath{fill}%
\end{pgfscope}%
\begin{pgfscope}%
\pgfpathrectangle{\pgfqpoint{1.150000in}{0.150000in}}{\pgfqpoint{5.700000in}{5.700000in}}%
\pgfusepath{clip}%
\pgfsetbuttcap%
\pgfsetroundjoin%
\definecolor{currentfill}{rgb}{0.265145,0.232956,0.516599}%
\pgfsetfillcolor{currentfill}%
\pgfsetfillopacity{0.700000}%
\pgfsetlinewidth{0.000000pt}%
\definecolor{currentstroke}{rgb}{0.000000,0.000000,0.000000}%
\pgfsetstrokecolor{currentstroke}%
\pgfsetdash{}{0pt}%
\pgfpathmoveto{\pgfqpoint{4.698863in}{2.196863in}}%
\pgfpathlineto{\pgfqpoint{4.712961in}{2.198557in}}%
\pgfpathlineto{\pgfqpoint{4.727070in}{2.200323in}}%
\pgfpathlineto{\pgfqpoint{4.741190in}{2.202160in}}%
\pgfpathlineto{\pgfqpoint{4.755319in}{2.204069in}}%
\pgfpathlineto{\pgfqpoint{4.763080in}{2.211493in}}%
\pgfpathlineto{\pgfqpoint{4.770833in}{2.218853in}}%
\pgfpathlineto{\pgfqpoint{4.778580in}{2.226151in}}%
\pgfpathlineto{\pgfqpoint{4.786320in}{2.233390in}}%
\pgfpathlineto{\pgfqpoint{4.772203in}{2.231589in}}%
\pgfpathlineto{\pgfqpoint{4.758097in}{2.229859in}}%
\pgfpathlineto{\pgfqpoint{4.744001in}{2.228200in}}%
\pgfpathlineto{\pgfqpoint{4.729915in}{2.226613in}}%
\pgfpathlineto{\pgfqpoint{4.722162in}{2.219259in}}%
\pgfpathlineto{\pgfqpoint{4.714402in}{2.211851in}}%
\pgfpathlineto{\pgfqpoint{4.706636in}{2.204387in}}%
\pgfpathlineto{\pgfqpoint{4.698863in}{2.196863in}}%
\pgfpathclose%
\pgfusepath{fill}%
\end{pgfscope}%
\begin{pgfscope}%
\pgfpathrectangle{\pgfqpoint{1.150000in}{0.150000in}}{\pgfqpoint{5.700000in}{5.700000in}}%
\pgfusepath{clip}%
\pgfsetbuttcap%
\pgfsetroundjoin%
\definecolor{currentfill}{rgb}{0.270595,0.214069,0.507052}%
\pgfsetfillcolor{currentfill}%
\pgfsetfillopacity{0.700000}%
\pgfsetlinewidth{0.000000pt}%
\definecolor{currentstroke}{rgb}{0.000000,0.000000,0.000000}%
\pgfsetstrokecolor{currentstroke}%
\pgfsetdash{}{0pt}%
\pgfpathmoveto{\pgfqpoint{2.167902in}{2.197100in}}%
\pgfpathlineto{\pgfqpoint{2.181681in}{2.184276in}}%
\pgfpathlineto{\pgfqpoint{2.195456in}{2.171584in}}%
\pgfpathlineto{\pgfqpoint{2.209228in}{2.159022in}}%
\pgfpathlineto{\pgfqpoint{2.222998in}{2.146590in}}%
\pgfpathlineto{\pgfqpoint{2.231956in}{2.147846in}}%
\pgfpathlineto{\pgfqpoint{2.240898in}{2.149317in}}%
\pgfpathlineto{\pgfqpoint{2.249823in}{2.150999in}}%
\pgfpathlineto{\pgfqpoint{2.258732in}{2.152887in}}%
\pgfpathlineto{\pgfqpoint{2.244997in}{2.165000in}}%
\pgfpathlineto{\pgfqpoint{2.231261in}{2.177242in}}%
\pgfpathlineto{\pgfqpoint{2.217521in}{2.189615in}}%
\pgfpathlineto{\pgfqpoint{2.203779in}{2.202119in}}%
\pgfpathlineto{\pgfqpoint{2.194835in}{2.200542in}}%
\pgfpathlineto{\pgfqpoint{2.185874in}{2.199177in}}%
\pgfpathlineto{\pgfqpoint{2.176897in}{2.198028in}}%
\pgfpathlineto{\pgfqpoint{2.167902in}{2.197100in}}%
\pgfpathclose%
\pgfusepath{fill}%
\end{pgfscope}%
\begin{pgfscope}%
\pgfpathrectangle{\pgfqpoint{1.150000in}{0.150000in}}{\pgfqpoint{5.700000in}{5.700000in}}%
\pgfusepath{clip}%
\pgfsetbuttcap%
\pgfsetroundjoin%
\definecolor{currentfill}{rgb}{0.274952,0.037752,0.364543}%
\pgfsetfillcolor{currentfill}%
\pgfsetfillopacity{0.700000}%
\pgfsetlinewidth{0.000000pt}%
\definecolor{currentstroke}{rgb}{0.000000,0.000000,0.000000}%
\pgfsetstrokecolor{currentstroke}%
\pgfsetdash{}{0pt}%
\pgfpathmoveto{\pgfqpoint{3.624167in}{1.813903in}}%
\pgfpathlineto{\pgfqpoint{3.637937in}{1.811495in}}%
\pgfpathlineto{\pgfqpoint{3.651714in}{1.809166in}}%
\pgfpathlineto{\pgfqpoint{3.665497in}{1.806917in}}%
\pgfpathlineto{\pgfqpoint{3.679287in}{1.804746in}}%
\pgfpathlineto{\pgfqpoint{3.687450in}{1.814148in}}%
\pgfpathlineto{\pgfqpoint{3.695607in}{1.823542in}}%
\pgfpathlineto{\pgfqpoint{3.703758in}{1.832926in}}%
\pgfpathlineto{\pgfqpoint{3.711904in}{1.842299in}}%
\pgfpathlineto{\pgfqpoint{3.698125in}{1.844347in}}%
\pgfpathlineto{\pgfqpoint{3.684353in}{1.846474in}}%
\pgfpathlineto{\pgfqpoint{3.670588in}{1.848680in}}%
\pgfpathlineto{\pgfqpoint{3.656830in}{1.850966in}}%
\pgfpathlineto{\pgfqpoint{3.648673in}{1.841707in}}%
\pgfpathlineto{\pgfqpoint{3.640510in}{1.832443in}}%
\pgfpathlineto{\pgfqpoint{3.632341in}{1.823174in}}%
\pgfpathlineto{\pgfqpoint{3.624167in}{1.813903in}}%
\pgfpathclose%
\pgfusepath{fill}%
\end{pgfscope}%
\begin{pgfscope}%
\pgfpathrectangle{\pgfqpoint{1.150000in}{0.150000in}}{\pgfqpoint{5.700000in}{5.700000in}}%
\pgfusepath{clip}%
\pgfsetbuttcap%
\pgfsetroundjoin%
\definecolor{currentfill}{rgb}{0.204903,0.375746,0.553533}%
\pgfsetfillcolor{currentfill}%
\pgfsetfillopacity{0.700000}%
\pgfsetlinewidth{0.000000pt}%
\definecolor{currentstroke}{rgb}{0.000000,0.000000,0.000000}%
\pgfsetstrokecolor{currentstroke}%
\pgfsetdash{}{0pt}%
\pgfpathmoveto{\pgfqpoint{5.686116in}{2.544643in}}%
\pgfpathlineto{\pgfqpoint{5.700587in}{2.547647in}}%
\pgfpathlineto{\pgfqpoint{5.715070in}{2.550720in}}%
\pgfpathlineto{\pgfqpoint{5.729565in}{2.553861in}}%
\pgfpathlineto{\pgfqpoint{5.744073in}{2.557070in}}%
\pgfpathlineto{\pgfqpoint{5.751338in}{2.560755in}}%
\pgfpathlineto{\pgfqpoint{5.758596in}{2.564468in}}%
\pgfpathlineto{\pgfqpoint{5.765849in}{2.568215in}}%
\pgfpathlineto{\pgfqpoint{5.773097in}{2.572002in}}%
\pgfpathlineto{\pgfqpoint{5.758615in}{2.569114in}}%
\pgfpathlineto{\pgfqpoint{5.744145in}{2.566294in}}%
\pgfpathlineto{\pgfqpoint{5.729688in}{2.563542in}}%
\pgfpathlineto{\pgfqpoint{5.715243in}{2.560857in}}%
\pgfpathlineto{\pgfqpoint{5.707969in}{2.556743in}}%
\pgfpathlineto{\pgfqpoint{5.700690in}{2.552673in}}%
\pgfpathlineto{\pgfqpoint{5.693406in}{2.548641in}}%
\pgfpathlineto{\pgfqpoint{5.686116in}{2.544643in}}%
\pgfpathclose%
\pgfusepath{fill}%
\end{pgfscope}%
\begin{pgfscope}%
\pgfpathrectangle{\pgfqpoint{1.150000in}{0.150000in}}{\pgfqpoint{5.700000in}{5.700000in}}%
\pgfusepath{clip}%
\pgfsetbuttcap%
\pgfsetroundjoin%
\definecolor{currentfill}{rgb}{0.282910,0.105393,0.426902}%
\pgfsetfillcolor{currentfill}%
\pgfsetfillopacity{0.700000}%
\pgfsetlinewidth{0.000000pt}%
\definecolor{currentstroke}{rgb}{0.000000,0.000000,0.000000}%
\pgfsetstrokecolor{currentstroke}%
\pgfsetdash{}{0pt}%
\pgfpathmoveto{\pgfqpoint{4.030137in}{1.935207in}}%
\pgfpathlineto{\pgfqpoint{4.044012in}{1.934714in}}%
\pgfpathlineto{\pgfqpoint{4.057895in}{1.934297in}}%
\pgfpathlineto{\pgfqpoint{4.071786in}{1.933956in}}%
\pgfpathlineto{\pgfqpoint{4.085686in}{1.933689in}}%
\pgfpathlineto{\pgfqpoint{4.093706in}{1.943062in}}%
\pgfpathlineto{\pgfqpoint{4.101720in}{1.952387in}}%
\pgfpathlineto{\pgfqpoint{4.109728in}{1.961663in}}%
\pgfpathlineto{\pgfqpoint{4.117731in}{1.970889in}}%
\pgfpathlineto{\pgfqpoint{4.103842in}{1.971116in}}%
\pgfpathlineto{\pgfqpoint{4.089961in}{1.971418in}}%
\pgfpathlineto{\pgfqpoint{4.076088in}{1.971795in}}%
\pgfpathlineto{\pgfqpoint{4.062223in}{1.972247in}}%
\pgfpathlineto{\pgfqpoint{4.054210in}{1.963052in}}%
\pgfpathlineto{\pgfqpoint{4.046191in}{1.953814in}}%
\pgfpathlineto{\pgfqpoint{4.038167in}{1.944532in}}%
\pgfpathlineto{\pgfqpoint{4.030137in}{1.935207in}}%
\pgfpathclose%
\pgfusepath{fill}%
\end{pgfscope}%
\begin{pgfscope}%
\pgfpathrectangle{\pgfqpoint{1.150000in}{0.150000in}}{\pgfqpoint{5.700000in}{5.700000in}}%
\pgfusepath{clip}%
\pgfsetbuttcap%
\pgfsetroundjoin%
\definecolor{currentfill}{rgb}{0.268510,0.009605,0.335427}%
\pgfsetfillcolor{currentfill}%
\pgfsetfillopacity{0.700000}%
\pgfsetlinewidth{0.000000pt}%
\definecolor{currentstroke}{rgb}{0.000000,0.000000,0.000000}%
\pgfsetstrokecolor{currentstroke}%
\pgfsetdash{}{0pt}%
\pgfpathmoveto{\pgfqpoint{3.162456in}{1.765846in}}%
\pgfpathlineto{\pgfqpoint{3.176155in}{1.760743in}}%
\pgfpathlineto{\pgfqpoint{3.189858in}{1.755727in}}%
\pgfpathlineto{\pgfqpoint{3.203566in}{1.750798in}}%
\pgfpathlineto{\pgfqpoint{3.217278in}{1.745956in}}%
\pgfpathlineto{\pgfqpoint{3.225627in}{1.753934in}}%
\pgfpathlineto{\pgfqpoint{3.233969in}{1.761971in}}%
\pgfpathlineto{\pgfqpoint{3.242304in}{1.770064in}}%
\pgfpathlineto{\pgfqpoint{3.250631in}{1.778210in}}%
\pgfpathlineto{\pgfqpoint{3.236936in}{1.782848in}}%
\pgfpathlineto{\pgfqpoint{3.223245in}{1.787572in}}%
\pgfpathlineto{\pgfqpoint{3.209559in}{1.792383in}}%
\pgfpathlineto{\pgfqpoint{3.195877in}{1.797282in}}%
\pgfpathlineto{\pgfqpoint{3.187533in}{1.789333in}}%
\pgfpathlineto{\pgfqpoint{3.179182in}{1.781441in}}%
\pgfpathlineto{\pgfqpoint{3.170823in}{1.773612in}}%
\pgfpathlineto{\pgfqpoint{3.162456in}{1.765846in}}%
\pgfpathclose%
\pgfusepath{fill}%
\end{pgfscope}%
\begin{pgfscope}%
\pgfpathrectangle{\pgfqpoint{1.150000in}{0.150000in}}{\pgfqpoint{5.700000in}{5.700000in}}%
\pgfusepath{clip}%
\pgfsetbuttcap%
\pgfsetroundjoin%
\definecolor{currentfill}{rgb}{0.272594,0.025563,0.353093}%
\pgfsetfillcolor{currentfill}%
\pgfsetfillopacity{0.700000}%
\pgfsetlinewidth{0.000000pt}%
\definecolor{currentstroke}{rgb}{0.000000,0.000000,0.000000}%
\pgfsetstrokecolor{currentstroke}%
\pgfsetdash{}{0pt}%
\pgfpathmoveto{\pgfqpoint{2.875922in}{1.807859in}}%
\pgfpathlineto{\pgfqpoint{2.889607in}{1.800823in}}%
\pgfpathlineto{\pgfqpoint{2.903294in}{1.793881in}}%
\pgfpathlineto{\pgfqpoint{2.916985in}{1.787035in}}%
\pgfpathlineto{\pgfqpoint{2.930677in}{1.780282in}}%
\pgfpathlineto{\pgfqpoint{2.939171in}{1.786658in}}%
\pgfpathlineto{\pgfqpoint{2.947655in}{1.793140in}}%
\pgfpathlineto{\pgfqpoint{2.956129in}{1.799724in}}%
\pgfpathlineto{\pgfqpoint{2.964595in}{1.806408in}}%
\pgfpathlineto{\pgfqpoint{2.950923in}{1.812914in}}%
\pgfpathlineto{\pgfqpoint{2.937254in}{1.819513in}}%
\pgfpathlineto{\pgfqpoint{2.923589in}{1.826208in}}%
\pgfpathlineto{\pgfqpoint{2.909925in}{1.832998in}}%
\pgfpathlineto{\pgfqpoint{2.901439in}{1.826553in}}%
\pgfpathlineto{\pgfqpoint{2.892943in}{1.820213in}}%
\pgfpathlineto{\pgfqpoint{2.884437in}{1.813980in}}%
\pgfpathlineto{\pgfqpoint{2.875922in}{1.807859in}}%
\pgfpathclose%
\pgfusepath{fill}%
\end{pgfscope}%
\begin{pgfscope}%
\pgfpathrectangle{\pgfqpoint{1.150000in}{0.150000in}}{\pgfqpoint{5.700000in}{5.700000in}}%
\pgfusepath{clip}%
\pgfsetbuttcap%
\pgfsetroundjoin%
\definecolor{currentfill}{rgb}{0.269308,0.218818,0.509577}%
\pgfsetfillcolor{currentfill}%
\pgfsetfillopacity{0.700000}%
\pgfsetlinewidth{0.000000pt}%
\definecolor{currentstroke}{rgb}{0.000000,0.000000,0.000000}%
\pgfsetstrokecolor{currentstroke}%
\pgfsetdash{}{0pt}%
\pgfpathmoveto{\pgfqpoint{4.611364in}{2.159736in}}%
\pgfpathlineto{\pgfqpoint{4.625434in}{2.161229in}}%
\pgfpathlineto{\pgfqpoint{4.639514in}{2.162795in}}%
\pgfpathlineto{\pgfqpoint{4.653604in}{2.164432in}}%
\pgfpathlineto{\pgfqpoint{4.667704in}{2.166141in}}%
\pgfpathlineto{\pgfqpoint{4.675504in}{2.173919in}}%
\pgfpathlineto{\pgfqpoint{4.683297in}{2.181632in}}%
\pgfpathlineto{\pgfqpoint{4.691083in}{2.189279in}}%
\pgfpathlineto{\pgfqpoint{4.698863in}{2.196863in}}%
\pgfpathlineto{\pgfqpoint{4.684774in}{2.195240in}}%
\pgfpathlineto{\pgfqpoint{4.670697in}{2.193689in}}%
\pgfpathlineto{\pgfqpoint{4.656629in}{2.192210in}}%
\pgfpathlineto{\pgfqpoint{4.642571in}{2.190802in}}%
\pgfpathlineto{\pgfqpoint{4.634779in}{2.183124in}}%
\pgfpathlineto{\pgfqpoint{4.626981in}{2.175388in}}%
\pgfpathlineto{\pgfqpoint{4.619176in}{2.167593in}}%
\pgfpathlineto{\pgfqpoint{4.611364in}{2.159736in}}%
\pgfpathclose%
\pgfusepath{fill}%
\end{pgfscope}%
\begin{pgfscope}%
\pgfpathrectangle{\pgfqpoint{1.150000in}{0.150000in}}{\pgfqpoint{5.700000in}{5.700000in}}%
\pgfusepath{clip}%
\pgfsetbuttcap%
\pgfsetroundjoin%
\definecolor{currentfill}{rgb}{0.233603,0.313828,0.543914}%
\pgfsetfillcolor{currentfill}%
\pgfsetfillopacity{0.700000}%
\pgfsetlinewidth{0.000000pt}%
\definecolor{currentstroke}{rgb}{0.000000,0.000000,0.000000}%
\pgfsetstrokecolor{currentstroke}%
\pgfsetdash{}{0pt}%
\pgfpathmoveto{\pgfqpoint{5.192683in}{2.381323in}}%
\pgfpathlineto{\pgfqpoint{5.206971in}{2.383960in}}%
\pgfpathlineto{\pgfqpoint{5.221270in}{2.386667in}}%
\pgfpathlineto{\pgfqpoint{5.235582in}{2.389443in}}%
\pgfpathlineto{\pgfqpoint{5.249905in}{2.392288in}}%
\pgfpathlineto{\pgfqpoint{5.257432in}{2.397712in}}%
\pgfpathlineto{\pgfqpoint{5.264953in}{2.403100in}}%
\pgfpathlineto{\pgfqpoint{5.272467in}{2.408457in}}%
\pgfpathlineto{\pgfqpoint{5.279973in}{2.413787in}}%
\pgfpathlineto{\pgfqpoint{5.265669in}{2.411156in}}%
\pgfpathlineto{\pgfqpoint{5.251376in}{2.408594in}}%
\pgfpathlineto{\pgfqpoint{5.237095in}{2.406101in}}%
\pgfpathlineto{\pgfqpoint{5.222825in}{2.403678in}}%
\pgfpathlineto{\pgfqpoint{5.215299in}{2.398127in}}%
\pgfpathlineto{\pgfqpoint{5.207767in}{2.392554in}}%
\pgfpathlineto{\pgfqpoint{5.200229in}{2.386954in}}%
\pgfpathlineto{\pgfqpoint{5.192683in}{2.381323in}}%
\pgfpathclose%
\pgfusepath{fill}%
\end{pgfscope}%
\begin{pgfscope}%
\pgfpathrectangle{\pgfqpoint{1.150000in}{0.150000in}}{\pgfqpoint{5.700000in}{5.700000in}}%
\pgfusepath{clip}%
\pgfsetbuttcap%
\pgfsetroundjoin%
\definecolor{currentfill}{rgb}{0.281924,0.089666,0.412415}%
\pgfsetfillcolor{currentfill}%
\pgfsetfillopacity{0.700000}%
\pgfsetlinewidth{0.000000pt}%
\definecolor{currentstroke}{rgb}{0.000000,0.000000,0.000000}%
\pgfsetstrokecolor{currentstroke}%
\pgfsetdash{}{0pt}%
\pgfpathmoveto{\pgfqpoint{3.942503in}{1.900449in}}%
\pgfpathlineto{\pgfqpoint{3.956355in}{1.899592in}}%
\pgfpathlineto{\pgfqpoint{3.970216in}{1.898811in}}%
\pgfpathlineto{\pgfqpoint{3.984085in}{1.898107in}}%
\pgfpathlineto{\pgfqpoint{3.997961in}{1.897478in}}%
\pgfpathlineto{\pgfqpoint{4.006014in}{1.906973in}}%
\pgfpathlineto{\pgfqpoint{4.014060in}{1.916427in}}%
\pgfpathlineto{\pgfqpoint{4.022102in}{1.925838in}}%
\pgfpathlineto{\pgfqpoint{4.030137in}{1.935207in}}%
\pgfpathlineto{\pgfqpoint{4.016271in}{1.935775in}}%
\pgfpathlineto{\pgfqpoint{4.002412in}{1.936419in}}%
\pgfpathlineto{\pgfqpoint{3.988562in}{1.937138in}}%
\pgfpathlineto{\pgfqpoint{3.974719in}{1.937934in}}%
\pgfpathlineto{\pgfqpoint{3.966674in}{1.928619in}}%
\pgfpathlineto{\pgfqpoint{3.958622in}{1.919266in}}%
\pgfpathlineto{\pgfqpoint{3.950565in}{1.909875in}}%
\pgfpathlineto{\pgfqpoint{3.942503in}{1.900449in}}%
\pgfpathclose%
\pgfusepath{fill}%
\end{pgfscope}%
\begin{pgfscope}%
\pgfpathrectangle{\pgfqpoint{1.150000in}{0.150000in}}{\pgfqpoint{5.700000in}{5.700000in}}%
\pgfusepath{clip}%
\pgfsetbuttcap%
\pgfsetroundjoin%
\definecolor{currentfill}{rgb}{0.275191,0.194905,0.496005}%
\pgfsetfillcolor{currentfill}%
\pgfsetfillopacity{0.700000}%
\pgfsetlinewidth{0.000000pt}%
\definecolor{currentstroke}{rgb}{0.000000,0.000000,0.000000}%
\pgfsetstrokecolor{currentstroke}%
\pgfsetdash{}{0pt}%
\pgfpathmoveto{\pgfqpoint{2.222998in}{2.146590in}}%
\pgfpathlineto{\pgfqpoint{2.236765in}{2.134287in}}%
\pgfpathlineto{\pgfqpoint{2.250529in}{2.122110in}}%
\pgfpathlineto{\pgfqpoint{2.264291in}{2.110060in}}%
\pgfpathlineto{\pgfqpoint{2.278051in}{2.098135in}}%
\pgfpathlineto{\pgfqpoint{2.286974in}{2.099717in}}%
\pgfpathlineto{\pgfqpoint{2.295880in}{2.101509in}}%
\pgfpathlineto{\pgfqpoint{2.304771in}{2.103506in}}%
\pgfpathlineto{\pgfqpoint{2.313646in}{2.105704in}}%
\pgfpathlineto{\pgfqpoint{2.299920in}{2.117312in}}%
\pgfpathlineto{\pgfqpoint{2.286193in}{2.129044in}}%
\pgfpathlineto{\pgfqpoint{2.272464in}{2.140902in}}%
\pgfpathlineto{\pgfqpoint{2.258732in}{2.152887in}}%
\pgfpathlineto{\pgfqpoint{2.249823in}{2.150999in}}%
\pgfpathlineto{\pgfqpoint{2.240898in}{2.149317in}}%
\pgfpathlineto{\pgfqpoint{2.231956in}{2.147846in}}%
\pgfpathlineto{\pgfqpoint{2.222998in}{2.146590in}}%
\pgfpathclose%
\pgfusepath{fill}%
\end{pgfscope}%
\begin{pgfscope}%
\pgfpathrectangle{\pgfqpoint{1.150000in}{0.150000in}}{\pgfqpoint{5.700000in}{5.700000in}}%
\pgfusepath{clip}%
\pgfsetbuttcap%
\pgfsetroundjoin%
\definecolor{currentfill}{rgb}{0.268510,0.009605,0.335427}%
\pgfsetfillcolor{currentfill}%
\pgfsetfillopacity{0.700000}%
\pgfsetlinewidth{0.000000pt}%
\definecolor{currentstroke}{rgb}{0.000000,0.000000,0.000000}%
\pgfsetstrokecolor{currentstroke}%
\pgfsetdash{}{0pt}%
\pgfpathmoveto{\pgfqpoint{3.305459in}{1.760518in}}%
\pgfpathlineto{\pgfqpoint{3.319178in}{1.756308in}}%
\pgfpathlineto{\pgfqpoint{3.332903in}{1.752182in}}%
\pgfpathlineto{\pgfqpoint{3.346632in}{1.748140in}}%
\pgfpathlineto{\pgfqpoint{3.360366in}{1.744182in}}%
\pgfpathlineto{\pgfqpoint{3.368655in}{1.752762in}}%
\pgfpathlineto{\pgfqpoint{3.376938in}{1.761380in}}%
\pgfpathlineto{\pgfqpoint{3.385214in}{1.770033in}}%
\pgfpathlineto{\pgfqpoint{3.393483in}{1.778718in}}%
\pgfpathlineto{\pgfqpoint{3.379764in}{1.782493in}}%
\pgfpathlineto{\pgfqpoint{3.366049in}{1.786351in}}%
\pgfpathlineto{\pgfqpoint{3.352340in}{1.790293in}}%
\pgfpathlineto{\pgfqpoint{3.338636in}{1.794319in}}%
\pgfpathlineto{\pgfqpoint{3.330352in}{1.785810in}}%
\pgfpathlineto{\pgfqpoint{3.322062in}{1.777338in}}%
\pgfpathlineto{\pgfqpoint{3.313764in}{1.768907in}}%
\pgfpathlineto{\pgfqpoint{3.305459in}{1.760518in}}%
\pgfpathclose%
\pgfusepath{fill}%
\end{pgfscope}%
\begin{pgfscope}%
\pgfpathrectangle{\pgfqpoint{1.150000in}{0.150000in}}{\pgfqpoint{5.700000in}{5.700000in}}%
\pgfusepath{clip}%
\pgfsetbuttcap%
\pgfsetroundjoin%
\definecolor{currentfill}{rgb}{0.274128,0.199721,0.498911}%
\pgfsetfillcolor{currentfill}%
\pgfsetfillopacity{0.700000}%
\pgfsetlinewidth{0.000000pt}%
\definecolor{currentstroke}{rgb}{0.000000,0.000000,0.000000}%
\pgfsetstrokecolor{currentstroke}%
\pgfsetdash{}{0pt}%
\pgfpathmoveto{\pgfqpoint{4.523830in}{2.122147in}}%
\pgfpathlineto{\pgfqpoint{4.537871in}{2.123417in}}%
\pgfpathlineto{\pgfqpoint{4.551922in}{2.124759in}}%
\pgfpathlineto{\pgfqpoint{4.565982in}{2.126174in}}%
\pgfpathlineto{\pgfqpoint{4.580053in}{2.127660in}}%
\pgfpathlineto{\pgfqpoint{4.587891in}{2.135779in}}%
\pgfpathlineto{\pgfqpoint{4.595722in}{2.143831in}}%
\pgfpathlineto{\pgfqpoint{4.603546in}{2.151816in}}%
\pgfpathlineto{\pgfqpoint{4.611364in}{2.159736in}}%
\pgfpathlineto{\pgfqpoint{4.597305in}{2.158314in}}%
\pgfpathlineto{\pgfqpoint{4.583255in}{2.156965in}}%
\pgfpathlineto{\pgfqpoint{4.569216in}{2.155688in}}%
\pgfpathlineto{\pgfqpoint{4.555186in}{2.154482in}}%
\pgfpathlineto{\pgfqpoint{4.547357in}{2.146490in}}%
\pgfpathlineto{\pgfqpoint{4.539521in}{2.138437in}}%
\pgfpathlineto{\pgfqpoint{4.531678in}{2.130324in}}%
\pgfpathlineto{\pgfqpoint{4.523830in}{2.122147in}}%
\pgfpathclose%
\pgfusepath{fill}%
\end{pgfscope}%
\begin{pgfscope}%
\pgfpathrectangle{\pgfqpoint{1.150000in}{0.150000in}}{\pgfqpoint{5.700000in}{5.700000in}}%
\pgfusepath{clip}%
\pgfsetbuttcap%
\pgfsetroundjoin%
\definecolor{currentfill}{rgb}{0.272594,0.025563,0.353093}%
\pgfsetfillcolor{currentfill}%
\pgfsetfillopacity{0.700000}%
\pgfsetlinewidth{0.000000pt}%
\definecolor{currentstroke}{rgb}{0.000000,0.000000,0.000000}%
\pgfsetstrokecolor{currentstroke}%
\pgfsetdash{}{0pt}%
\pgfpathmoveto{\pgfqpoint{3.536343in}{1.787826in}}%
\pgfpathlineto{\pgfqpoint{3.550100in}{1.784954in}}%
\pgfpathlineto{\pgfqpoint{3.563864in}{1.782163in}}%
\pgfpathlineto{\pgfqpoint{3.577634in}{1.779452in}}%
\pgfpathlineto{\pgfqpoint{3.591410in}{1.776821in}}%
\pgfpathlineto{\pgfqpoint{3.599608in}{1.786087in}}%
\pgfpathlineto{\pgfqpoint{3.607800in}{1.795357in}}%
\pgfpathlineto{\pgfqpoint{3.615986in}{1.804630in}}%
\pgfpathlineto{\pgfqpoint{3.624167in}{1.813903in}}%
\pgfpathlineto{\pgfqpoint{3.610403in}{1.816391in}}%
\pgfpathlineto{\pgfqpoint{3.596645in}{1.818959in}}%
\pgfpathlineto{\pgfqpoint{3.582894in}{1.821607in}}%
\pgfpathlineto{\pgfqpoint{3.569149in}{1.824336in}}%
\pgfpathlineto{\pgfqpoint{3.560957in}{1.815199in}}%
\pgfpathlineto{\pgfqpoint{3.552758in}{1.806067in}}%
\pgfpathlineto{\pgfqpoint{3.544554in}{1.796942in}}%
\pgfpathlineto{\pgfqpoint{3.536343in}{1.787826in}}%
\pgfpathclose%
\pgfusepath{fill}%
\end{pgfscope}%
\begin{pgfscope}%
\pgfpathrectangle{\pgfqpoint{1.150000in}{0.150000in}}{\pgfqpoint{5.700000in}{5.700000in}}%
\pgfusepath{clip}%
\pgfsetbuttcap%
\pgfsetroundjoin%
\definecolor{currentfill}{rgb}{0.208623,0.367752,0.552675}%
\pgfsetfillcolor{currentfill}%
\pgfsetfillopacity{0.700000}%
\pgfsetlinewidth{0.000000pt}%
\definecolor{currentstroke}{rgb}{0.000000,0.000000,0.000000}%
\pgfsetstrokecolor{currentstroke}%
\pgfsetdash{}{0pt}%
\pgfpathmoveto{\pgfqpoint{5.599045in}{2.516338in}}%
\pgfpathlineto{\pgfqpoint{5.613490in}{2.519369in}}%
\pgfpathlineto{\pgfqpoint{5.627947in}{2.522469in}}%
\pgfpathlineto{\pgfqpoint{5.642418in}{2.525637in}}%
\pgfpathlineto{\pgfqpoint{5.656900in}{2.528873in}}%
\pgfpathlineto{\pgfqpoint{5.664213in}{2.532793in}}%
\pgfpathlineto{\pgfqpoint{5.671520in}{2.536724in}}%
\pgfpathlineto{\pgfqpoint{5.678821in}{2.540672in}}%
\pgfpathlineto{\pgfqpoint{5.686116in}{2.544643in}}%
\pgfpathlineto{\pgfqpoint{5.671658in}{2.541706in}}%
\pgfpathlineto{\pgfqpoint{5.657212in}{2.538838in}}%
\pgfpathlineto{\pgfqpoint{5.642779in}{2.536038in}}%
\pgfpathlineto{\pgfqpoint{5.628358in}{2.533306in}}%
\pgfpathlineto{\pgfqpoint{5.621038in}{2.529028in}}%
\pgfpathlineto{\pgfqpoint{5.613713in}{2.524778in}}%
\pgfpathlineto{\pgfqpoint{5.606382in}{2.520550in}}%
\pgfpathlineto{\pgfqpoint{5.599045in}{2.516338in}}%
\pgfpathclose%
\pgfusepath{fill}%
\end{pgfscope}%
\begin{pgfscope}%
\pgfpathrectangle{\pgfqpoint{1.150000in}{0.150000in}}{\pgfqpoint{5.700000in}{5.700000in}}%
\pgfusepath{clip}%
\pgfsetbuttcap%
\pgfsetroundjoin%
\definecolor{currentfill}{rgb}{0.277018,0.050344,0.375715}%
\pgfsetfillcolor{currentfill}%
\pgfsetfillopacity{0.700000}%
\pgfsetlinewidth{0.000000pt}%
\definecolor{currentstroke}{rgb}{0.000000,0.000000,0.000000}%
\pgfsetstrokecolor{currentstroke}%
\pgfsetdash{}{0pt}%
\pgfpathmoveto{\pgfqpoint{2.732169in}{1.846491in}}%
\pgfpathlineto{\pgfqpoint{2.745861in}{1.838404in}}%
\pgfpathlineto{\pgfqpoint{2.759556in}{1.830417in}}%
\pgfpathlineto{\pgfqpoint{2.773252in}{1.822530in}}%
\pgfpathlineto{\pgfqpoint{2.786949in}{1.814741in}}%
\pgfpathlineto{\pgfqpoint{2.795529in}{1.820104in}}%
\pgfpathlineto{\pgfqpoint{2.804098in}{1.825598in}}%
\pgfpathlineto{\pgfqpoint{2.812657in}{1.831221in}}%
\pgfpathlineto{\pgfqpoint{2.821205in}{1.836967in}}%
\pgfpathlineto{\pgfqpoint{2.807531in}{1.844488in}}%
\pgfpathlineto{\pgfqpoint{2.793860in}{1.852107in}}%
\pgfpathlineto{\pgfqpoint{2.780190in}{1.859825in}}%
\pgfpathlineto{\pgfqpoint{2.766522in}{1.867644in}}%
\pgfpathlineto{\pgfqpoint{2.757950in}{1.862158in}}%
\pgfpathlineto{\pgfqpoint{2.749367in}{1.856801in}}%
\pgfpathlineto{\pgfqpoint{2.740773in}{1.851577in}}%
\pgfpathlineto{\pgfqpoint{2.732169in}{1.846491in}}%
\pgfpathclose%
\pgfusepath{fill}%
\end{pgfscope}%
\begin{pgfscope}%
\pgfpathrectangle{\pgfqpoint{1.150000in}{0.150000in}}{\pgfqpoint{5.700000in}{5.700000in}}%
\pgfusepath{clip}%
\pgfsetbuttcap%
\pgfsetroundjoin%
\definecolor{currentfill}{rgb}{0.282656,0.100196,0.422160}%
\pgfsetfillcolor{currentfill}%
\pgfsetfillopacity{0.700000}%
\pgfsetlinewidth{0.000000pt}%
\definecolor{currentstroke}{rgb}{0.000000,0.000000,0.000000}%
\pgfsetstrokecolor{currentstroke}%
\pgfsetdash{}{0pt}%
\pgfpathmoveto{\pgfqpoint{2.533081in}{1.936168in}}%
\pgfpathlineto{\pgfqpoint{2.546791in}{1.926537in}}%
\pgfpathlineto{\pgfqpoint{2.560501in}{1.917015in}}%
\pgfpathlineto{\pgfqpoint{2.574211in}{1.907600in}}%
\pgfpathlineto{\pgfqpoint{2.587922in}{1.898293in}}%
\pgfpathlineto{\pgfqpoint{2.596629in}{1.902193in}}%
\pgfpathlineto{\pgfqpoint{2.605322in}{1.906257in}}%
\pgfpathlineto{\pgfqpoint{2.614003in}{1.910482in}}%
\pgfpathlineto{\pgfqpoint{2.622672in}{1.914863in}}%
\pgfpathlineto{\pgfqpoint{2.608989in}{1.923879in}}%
\pgfpathlineto{\pgfqpoint{2.595307in}{1.933002in}}%
\pgfpathlineto{\pgfqpoint{2.581625in}{1.942233in}}%
\pgfpathlineto{\pgfqpoint{2.567944in}{1.951572in}}%
\pgfpathlineto{\pgfqpoint{2.559247in}{1.947475in}}%
\pgfpathlineto{\pgfqpoint{2.550538in}{1.943539in}}%
\pgfpathlineto{\pgfqpoint{2.541816in}{1.939768in}}%
\pgfpathlineto{\pgfqpoint{2.533081in}{1.936168in}}%
\pgfpathclose%
\pgfusepath{fill}%
\end{pgfscope}%
\begin{pgfscope}%
\pgfpathrectangle{\pgfqpoint{1.150000in}{0.150000in}}{\pgfqpoint{5.700000in}{5.700000in}}%
\pgfusepath{clip}%
\pgfsetbuttcap%
\pgfsetroundjoin%
\definecolor{currentfill}{rgb}{0.239346,0.300855,0.540844}%
\pgfsetfillcolor{currentfill}%
\pgfsetfillopacity{0.700000}%
\pgfsetlinewidth{0.000000pt}%
\definecolor{currentstroke}{rgb}{0.000000,0.000000,0.000000}%
\pgfsetstrokecolor{currentstroke}%
\pgfsetdash{}{0pt}%
\pgfpathmoveto{\pgfqpoint{5.105324in}{2.347804in}}%
\pgfpathlineto{\pgfqpoint{5.119583in}{2.350355in}}%
\pgfpathlineto{\pgfqpoint{5.133854in}{2.352975in}}%
\pgfpathlineto{\pgfqpoint{5.148136in}{2.355665in}}%
\pgfpathlineto{\pgfqpoint{5.162430in}{2.358426in}}%
\pgfpathlineto{\pgfqpoint{5.170004in}{2.364214in}}%
\pgfpathlineto{\pgfqpoint{5.177570in}{2.369957in}}%
\pgfpathlineto{\pgfqpoint{5.185130in}{2.375659in}}%
\pgfpathlineto{\pgfqpoint{5.192683in}{2.381323in}}%
\pgfpathlineto{\pgfqpoint{5.178406in}{2.378756in}}%
\pgfpathlineto{\pgfqpoint{5.164141in}{2.376259in}}%
\pgfpathlineto{\pgfqpoint{5.149888in}{2.373831in}}%
\pgfpathlineto{\pgfqpoint{5.135646in}{2.371473in}}%
\pgfpathlineto{\pgfqpoint{5.128076in}{2.365608in}}%
\pgfpathlineto{\pgfqpoint{5.120499in}{2.359711in}}%
\pgfpathlineto{\pgfqpoint{5.112915in}{2.353778in}}%
\pgfpathlineto{\pgfqpoint{5.105324in}{2.347804in}}%
\pgfpathclose%
\pgfusepath{fill}%
\end{pgfscope}%
\begin{pgfscope}%
\pgfpathrectangle{\pgfqpoint{1.150000in}{0.150000in}}{\pgfqpoint{5.700000in}{5.700000in}}%
\pgfusepath{clip}%
\pgfsetbuttcap%
\pgfsetroundjoin%
\definecolor{currentfill}{rgb}{0.280894,0.078907,0.402329}%
\pgfsetfillcolor{currentfill}%
\pgfsetfillopacity{0.700000}%
\pgfsetlinewidth{0.000000pt}%
\definecolor{currentstroke}{rgb}{0.000000,0.000000,0.000000}%
\pgfsetstrokecolor{currentstroke}%
\pgfsetdash{}{0pt}%
\pgfpathmoveto{\pgfqpoint{3.854822in}{1.866907in}}%
\pgfpathlineto{\pgfqpoint{3.868654in}{1.865662in}}%
\pgfpathlineto{\pgfqpoint{3.882494in}{1.864495in}}%
\pgfpathlineto{\pgfqpoint{3.896342in}{1.863404in}}%
\pgfpathlineto{\pgfqpoint{3.910197in}{1.862389in}}%
\pgfpathlineto{\pgfqpoint{3.918282in}{1.871955in}}%
\pgfpathlineto{\pgfqpoint{3.926361in}{1.881488in}}%
\pgfpathlineto{\pgfqpoint{3.934435in}{1.890986in}}%
\pgfpathlineto{\pgfqpoint{3.942503in}{1.900449in}}%
\pgfpathlineto{\pgfqpoint{3.928658in}{1.901382in}}%
\pgfpathlineto{\pgfqpoint{3.914821in}{1.902391in}}%
\pgfpathlineto{\pgfqpoint{3.900992in}{1.903477in}}%
\pgfpathlineto{\pgfqpoint{3.887170in}{1.904640in}}%
\pgfpathlineto{\pgfqpoint{3.879091in}{1.895251in}}%
\pgfpathlineto{\pgfqpoint{3.871007in}{1.885832in}}%
\pgfpathlineto{\pgfqpoint{3.862917in}{1.876383in}}%
\pgfpathlineto{\pgfqpoint{3.854822in}{1.866907in}}%
\pgfpathclose%
\pgfusepath{fill}%
\end{pgfscope}%
\begin{pgfscope}%
\pgfpathrectangle{\pgfqpoint{1.150000in}{0.150000in}}{\pgfqpoint{5.700000in}{5.700000in}}%
\pgfusepath{clip}%
\pgfsetbuttcap%
\pgfsetroundjoin%
\definecolor{currentfill}{rgb}{0.277134,0.185228,0.489898}%
\pgfsetfillcolor{currentfill}%
\pgfsetfillopacity{0.700000}%
\pgfsetlinewidth{0.000000pt}%
\definecolor{currentstroke}{rgb}{0.000000,0.000000,0.000000}%
\pgfsetstrokecolor{currentstroke}%
\pgfsetdash{}{0pt}%
\pgfpathmoveto{\pgfqpoint{4.436261in}{2.084259in}}%
\pgfpathlineto{\pgfqpoint{4.450274in}{2.085282in}}%
\pgfpathlineto{\pgfqpoint{4.464297in}{2.086378in}}%
\pgfpathlineto{\pgfqpoint{4.478329in}{2.087547in}}%
\pgfpathlineto{\pgfqpoint{4.492371in}{2.088789in}}%
\pgfpathlineto{\pgfqpoint{4.500245in}{2.097229in}}%
\pgfpathlineto{\pgfqpoint{4.508113in}{2.105601in}}%
\pgfpathlineto{\pgfqpoint{4.515974in}{2.113907in}}%
\pgfpathlineto{\pgfqpoint{4.523830in}{2.122147in}}%
\pgfpathlineto{\pgfqpoint{4.509798in}{2.120950in}}%
\pgfpathlineto{\pgfqpoint{4.495777in}{2.119825in}}%
\pgfpathlineto{\pgfqpoint{4.481765in}{2.118772in}}%
\pgfpathlineto{\pgfqpoint{4.467763in}{2.117792in}}%
\pgfpathlineto{\pgfqpoint{4.459897in}{2.109501in}}%
\pgfpathlineto{\pgfqpoint{4.452025in}{2.101148in}}%
\pgfpathlineto{\pgfqpoint{4.444146in}{2.092735in}}%
\pgfpathlineto{\pgfqpoint{4.436261in}{2.084259in}}%
\pgfpathclose%
\pgfusepath{fill}%
\end{pgfscope}%
\begin{pgfscope}%
\pgfpathrectangle{\pgfqpoint{1.150000in}{0.150000in}}{\pgfqpoint{5.700000in}{5.700000in}}%
\pgfusepath{clip}%
\pgfsetbuttcap%
\pgfsetroundjoin%
\definecolor{currentfill}{rgb}{0.278826,0.175490,0.483397}%
\pgfsetfillcolor{currentfill}%
\pgfsetfillopacity{0.700000}%
\pgfsetlinewidth{0.000000pt}%
\definecolor{currentstroke}{rgb}{0.000000,0.000000,0.000000}%
\pgfsetstrokecolor{currentstroke}%
\pgfsetdash{}{0pt}%
\pgfpathmoveto{\pgfqpoint{2.278051in}{2.098135in}}%
\pgfpathlineto{\pgfqpoint{2.291808in}{2.086335in}}%
\pgfpathlineto{\pgfqpoint{2.305564in}{2.074657in}}%
\pgfpathlineto{\pgfqpoint{2.319318in}{2.063101in}}%
\pgfpathlineto{\pgfqpoint{2.333070in}{2.051667in}}%
\pgfpathlineto{\pgfqpoint{2.341958in}{2.053573in}}%
\pgfpathlineto{\pgfqpoint{2.350830in}{2.055685in}}%
\pgfpathlineto{\pgfqpoint{2.359688in}{2.057996in}}%
\pgfpathlineto{\pgfqpoint{2.368530in}{2.060503in}}%
\pgfpathlineto{\pgfqpoint{2.354811in}{2.071621in}}%
\pgfpathlineto{\pgfqpoint{2.341091in}{2.082860in}}%
\pgfpathlineto{\pgfqpoint{2.327369in}{2.094221in}}%
\pgfpathlineto{\pgfqpoint{2.313646in}{2.105704in}}%
\pgfpathlineto{\pgfqpoint{2.304771in}{2.103506in}}%
\pgfpathlineto{\pgfqpoint{2.295880in}{2.101509in}}%
\pgfpathlineto{\pgfqpoint{2.286974in}{2.099717in}}%
\pgfpathlineto{\pgfqpoint{2.278051in}{2.098135in}}%
\pgfpathclose%
\pgfusepath{fill}%
\end{pgfscope}%
\begin{pgfscope}%
\pgfpathrectangle{\pgfqpoint{1.150000in}{0.150000in}}{\pgfqpoint{5.700000in}{5.700000in}}%
\pgfusepath{clip}%
\pgfsetbuttcap%
\pgfsetroundjoin%
\definecolor{currentfill}{rgb}{0.279574,0.170599,0.479997}%
\pgfsetfillcolor{currentfill}%
\pgfsetfillopacity{0.700000}%
\pgfsetlinewidth{0.000000pt}%
\definecolor{currentstroke}{rgb}{0.000000,0.000000,0.000000}%
\pgfsetstrokecolor{currentstroke}%
\pgfsetdash{}{0pt}%
\pgfpathmoveto{\pgfqpoint{4.348662in}{2.046253in}}%
\pgfpathlineto{\pgfqpoint{4.362648in}{2.047007in}}%
\pgfpathlineto{\pgfqpoint{4.376642in}{2.047835in}}%
\pgfpathlineto{\pgfqpoint{4.390647in}{2.048735in}}%
\pgfpathlineto{\pgfqpoint{4.404660in}{2.049709in}}%
\pgfpathlineto{\pgfqpoint{4.412570in}{2.058445in}}%
\pgfpathlineto{\pgfqpoint{4.420473in}{2.067115in}}%
\pgfpathlineto{\pgfqpoint{4.428370in}{2.075719in}}%
\pgfpathlineto{\pgfqpoint{4.436261in}{2.084259in}}%
\pgfpathlineto{\pgfqpoint{4.422258in}{2.083308in}}%
\pgfpathlineto{\pgfqpoint{4.408264in}{2.082430in}}%
\pgfpathlineto{\pgfqpoint{4.394280in}{2.081626in}}%
\pgfpathlineto{\pgfqpoint{4.380305in}{2.080894in}}%
\pgfpathlineto{\pgfqpoint{4.372404in}{2.072324in}}%
\pgfpathlineto{\pgfqpoint{4.364496in}{2.063694in}}%
\pgfpathlineto{\pgfqpoint{4.356582in}{2.055004in}}%
\pgfpathlineto{\pgfqpoint{4.348662in}{2.046253in}}%
\pgfpathclose%
\pgfusepath{fill}%
\end{pgfscope}%
\begin{pgfscope}%
\pgfpathrectangle{\pgfqpoint{1.150000in}{0.150000in}}{\pgfqpoint{5.700000in}{5.700000in}}%
\pgfusepath{clip}%
\pgfsetbuttcap%
\pgfsetroundjoin%
\definecolor{currentfill}{rgb}{0.244972,0.287675,0.537260}%
\pgfsetfillcolor{currentfill}%
\pgfsetfillopacity{0.700000}%
\pgfsetlinewidth{0.000000pt}%
\definecolor{currentstroke}{rgb}{0.000000,0.000000,0.000000}%
\pgfsetstrokecolor{currentstroke}%
\pgfsetdash{}{0pt}%
\pgfpathmoveto{\pgfqpoint{5.017902in}{2.313257in}}%
\pgfpathlineto{\pgfqpoint{5.032132in}{2.315699in}}%
\pgfpathlineto{\pgfqpoint{5.046373in}{2.318211in}}%
\pgfpathlineto{\pgfqpoint{5.060626in}{2.320793in}}%
\pgfpathlineto{\pgfqpoint{5.074890in}{2.323446in}}%
\pgfpathlineto{\pgfqpoint{5.082509in}{2.329612in}}%
\pgfpathlineto{\pgfqpoint{5.090121in}{2.335725in}}%
\pgfpathlineto{\pgfqpoint{5.097726in}{2.341788in}}%
\pgfpathlineto{\pgfqpoint{5.105324in}{2.347804in}}%
\pgfpathlineto{\pgfqpoint{5.091076in}{2.345324in}}%
\pgfpathlineto{\pgfqpoint{5.076840in}{2.342913in}}%
\pgfpathlineto{\pgfqpoint{5.062614in}{2.340573in}}%
\pgfpathlineto{\pgfqpoint{5.048400in}{2.338302in}}%
\pgfpathlineto{\pgfqpoint{5.040786in}{2.332106in}}%
\pgfpathlineto{\pgfqpoint{5.033165in}{2.325869in}}%
\pgfpathlineto{\pgfqpoint{5.025537in}{2.319587in}}%
\pgfpathlineto{\pgfqpoint{5.017902in}{2.313257in}}%
\pgfpathclose%
\pgfusepath{fill}%
\end{pgfscope}%
\begin{pgfscope}%
\pgfpathrectangle{\pgfqpoint{1.150000in}{0.150000in}}{\pgfqpoint{5.700000in}{5.700000in}}%
\pgfusepath{clip}%
\pgfsetbuttcap%
\pgfsetroundjoin%
\definecolor{currentfill}{rgb}{0.212395,0.359683,0.551710}%
\pgfsetfillcolor{currentfill}%
\pgfsetfillopacity{0.700000}%
\pgfsetlinewidth{0.000000pt}%
\definecolor{currentstroke}{rgb}{0.000000,0.000000,0.000000}%
\pgfsetstrokecolor{currentstroke}%
\pgfsetdash{}{0pt}%
\pgfpathmoveto{\pgfqpoint{5.511884in}{2.487001in}}%
\pgfpathlineto{\pgfqpoint{5.526303in}{2.490037in}}%
\pgfpathlineto{\pgfqpoint{5.540734in}{2.493141in}}%
\pgfpathlineto{\pgfqpoint{5.555178in}{2.496314in}}%
\pgfpathlineto{\pgfqpoint{5.569634in}{2.499556in}}%
\pgfpathlineto{\pgfqpoint{5.576996in}{2.503752in}}%
\pgfpathlineto{\pgfqpoint{5.584352in}{2.507944in}}%
\pgfpathlineto{\pgfqpoint{5.591701in}{2.512137in}}%
\pgfpathlineto{\pgfqpoint{5.599045in}{2.516338in}}%
\pgfpathlineto{\pgfqpoint{5.584612in}{2.513375in}}%
\pgfpathlineto{\pgfqpoint{5.570191in}{2.510481in}}%
\pgfpathlineto{\pgfqpoint{5.555782in}{2.507655in}}%
\pgfpathlineto{\pgfqpoint{5.541386in}{2.504897in}}%
\pgfpathlineto{\pgfqpoint{5.534020in}{2.500411in}}%
\pgfpathlineto{\pgfqpoint{5.526647in}{2.495936in}}%
\pgfpathlineto{\pgfqpoint{5.519269in}{2.491468in}}%
\pgfpathlineto{\pgfqpoint{5.511884in}{2.487001in}}%
\pgfpathclose%
\pgfusepath{fill}%
\end{pgfscope}%
\begin{pgfscope}%
\pgfpathrectangle{\pgfqpoint{1.150000in}{0.150000in}}{\pgfqpoint{5.700000in}{5.700000in}}%
\pgfusepath{clip}%
\pgfsetbuttcap%
\pgfsetroundjoin%
\definecolor{currentfill}{rgb}{0.278791,0.062145,0.386592}%
\pgfsetfillcolor{currentfill}%
\pgfsetfillopacity{0.700000}%
\pgfsetlinewidth{0.000000pt}%
\definecolor{currentstroke}{rgb}{0.000000,0.000000,0.000000}%
\pgfsetstrokecolor{currentstroke}%
\pgfsetdash{}{0pt}%
\pgfpathmoveto{\pgfqpoint{3.767087in}{1.834894in}}%
\pgfpathlineto{\pgfqpoint{3.780901in}{1.833238in}}%
\pgfpathlineto{\pgfqpoint{3.794722in}{1.831660in}}%
\pgfpathlineto{\pgfqpoint{3.808550in}{1.830159in}}%
\pgfpathlineto{\pgfqpoint{3.822385in}{1.828735in}}%
\pgfpathlineto{\pgfqpoint{3.830503in}{1.838315in}}%
\pgfpathlineto{\pgfqpoint{3.838615in}{1.847871in}}%
\pgfpathlineto{\pgfqpoint{3.846721in}{1.857402in}}%
\pgfpathlineto{\pgfqpoint{3.854822in}{1.866907in}}%
\pgfpathlineto{\pgfqpoint{3.840997in}{1.868228in}}%
\pgfpathlineto{\pgfqpoint{3.827180in}{1.869627in}}%
\pgfpathlineto{\pgfqpoint{3.813370in}{1.871103in}}%
\pgfpathlineto{\pgfqpoint{3.799567in}{1.872657in}}%
\pgfpathlineto{\pgfqpoint{3.791456in}{1.863247in}}%
\pgfpathlineto{\pgfqpoint{3.783339in}{1.853816in}}%
\pgfpathlineto{\pgfqpoint{3.775216in}{1.844364in}}%
\pgfpathlineto{\pgfqpoint{3.767087in}{1.834894in}}%
\pgfpathclose%
\pgfusepath{fill}%
\end{pgfscope}%
\begin{pgfscope}%
\pgfpathrectangle{\pgfqpoint{1.150000in}{0.150000in}}{\pgfqpoint{5.700000in}{5.700000in}}%
\pgfusepath{clip}%
\pgfsetbuttcap%
\pgfsetroundjoin%
\definecolor{currentfill}{rgb}{0.268510,0.009605,0.335427}%
\pgfsetfillcolor{currentfill}%
\pgfsetfillopacity{0.700000}%
\pgfsetlinewidth{0.000000pt}%
\definecolor{currentstroke}{rgb}{0.000000,0.000000,0.000000}%
\pgfsetstrokecolor{currentstroke}%
\pgfsetdash{}{0pt}%
\pgfpathmoveto{\pgfqpoint{3.074082in}{1.757682in}}%
\pgfpathlineto{\pgfqpoint{3.087784in}{1.752000in}}%
\pgfpathlineto{\pgfqpoint{3.101489in}{1.746406in}}%
\pgfpathlineto{\pgfqpoint{3.115198in}{1.740901in}}%
\pgfpathlineto{\pgfqpoint{3.128911in}{1.735485in}}%
\pgfpathlineto{\pgfqpoint{3.137309in}{1.742964in}}%
\pgfpathlineto{\pgfqpoint{3.145699in}{1.750519in}}%
\pgfpathlineto{\pgfqpoint{3.154082in}{1.758147in}}%
\pgfpathlineto{\pgfqpoint{3.162456in}{1.765846in}}%
\pgfpathlineto{\pgfqpoint{3.148761in}{1.771037in}}%
\pgfpathlineto{\pgfqpoint{3.135071in}{1.776316in}}%
\pgfpathlineto{\pgfqpoint{3.121384in}{1.781684in}}%
\pgfpathlineto{\pgfqpoint{3.107701in}{1.787142in}}%
\pgfpathlineto{\pgfqpoint{3.099308in}{1.779661in}}%
\pgfpathlineto{\pgfqpoint{3.090908in}{1.772255in}}%
\pgfpathlineto{\pgfqpoint{3.082499in}{1.764928in}}%
\pgfpathlineto{\pgfqpoint{3.074082in}{1.757682in}}%
\pgfpathclose%
\pgfusepath{fill}%
\end{pgfscope}%
\begin{pgfscope}%
\pgfpathrectangle{\pgfqpoint{1.150000in}{0.150000in}}{\pgfqpoint{5.700000in}{5.700000in}}%
\pgfusepath{clip}%
\pgfsetbuttcap%
\pgfsetroundjoin%
\definecolor{currentfill}{rgb}{0.271305,0.019942,0.347269}%
\pgfsetfillcolor{currentfill}%
\pgfsetfillopacity{0.700000}%
\pgfsetlinewidth{0.000000pt}%
\definecolor{currentstroke}{rgb}{0.000000,0.000000,0.000000}%
\pgfsetstrokecolor{currentstroke}%
\pgfsetdash{}{0pt}%
\pgfpathmoveto{\pgfqpoint{2.930677in}{1.780282in}}%
\pgfpathlineto{\pgfqpoint{2.944373in}{1.773623in}}%
\pgfpathlineto{\pgfqpoint{2.958072in}{1.767057in}}%
\pgfpathlineto{\pgfqpoint{2.971773in}{1.760583in}}%
\pgfpathlineto{\pgfqpoint{2.985478in}{1.754202in}}%
\pgfpathlineto{\pgfqpoint{2.993950in}{1.760832in}}%
\pgfpathlineto{\pgfqpoint{3.002413in}{1.767563in}}%
\pgfpathlineto{\pgfqpoint{3.010867in}{1.774391in}}%
\pgfpathlineto{\pgfqpoint{3.019312in}{1.781313in}}%
\pgfpathlineto{\pgfqpoint{3.005628in}{1.787448in}}%
\pgfpathlineto{\pgfqpoint{2.991947in}{1.793675in}}%
\pgfpathlineto{\pgfqpoint{2.978269in}{1.799995in}}%
\pgfpathlineto{\pgfqpoint{2.964595in}{1.806408in}}%
\pgfpathlineto{\pgfqpoint{2.956129in}{1.799724in}}%
\pgfpathlineto{\pgfqpoint{2.947655in}{1.793140in}}%
\pgfpathlineto{\pgfqpoint{2.939171in}{1.786658in}}%
\pgfpathlineto{\pgfqpoint{2.930677in}{1.780282in}}%
\pgfpathclose%
\pgfusepath{fill}%
\end{pgfscope}%
\begin{pgfscope}%
\pgfpathrectangle{\pgfqpoint{1.150000in}{0.150000in}}{\pgfqpoint{5.700000in}{5.700000in}}%
\pgfusepath{clip}%
\pgfsetbuttcap%
\pgfsetroundjoin%
\definecolor{currentfill}{rgb}{0.271305,0.019942,0.347269}%
\pgfsetfillcolor{currentfill}%
\pgfsetfillopacity{0.700000}%
\pgfsetlinewidth{0.000000pt}%
\definecolor{currentstroke}{rgb}{0.000000,0.000000,0.000000}%
\pgfsetstrokecolor{currentstroke}%
\pgfsetdash{}{0pt}%
\pgfpathmoveto{\pgfqpoint{3.448416in}{1.764451in}}%
\pgfpathlineto{\pgfqpoint{3.462163in}{1.761090in}}%
\pgfpathlineto{\pgfqpoint{3.475916in}{1.757811in}}%
\pgfpathlineto{\pgfqpoint{3.489675in}{1.754614in}}%
\pgfpathlineto{\pgfqpoint{3.503439in}{1.751497in}}%
\pgfpathlineto{\pgfqpoint{3.511674in}{1.760555in}}%
\pgfpathlineto{\pgfqpoint{3.519903in}{1.769631in}}%
\pgfpathlineto{\pgfqpoint{3.528126in}{1.778722in}}%
\pgfpathlineto{\pgfqpoint{3.536343in}{1.787826in}}%
\pgfpathlineto{\pgfqpoint{3.522592in}{1.790779in}}%
\pgfpathlineto{\pgfqpoint{3.508846in}{1.793814in}}%
\pgfpathlineto{\pgfqpoint{3.495107in}{1.796929in}}%
\pgfpathlineto{\pgfqpoint{3.481373in}{1.800126in}}%
\pgfpathlineto{\pgfqpoint{3.473143in}{1.791178in}}%
\pgfpathlineto{\pgfqpoint{3.464907in}{1.782248in}}%
\pgfpathlineto{\pgfqpoint{3.456665in}{1.773338in}}%
\pgfpathlineto{\pgfqpoint{3.448416in}{1.764451in}}%
\pgfpathclose%
\pgfusepath{fill}%
\end{pgfscope}%
\begin{pgfscope}%
\pgfpathrectangle{\pgfqpoint{1.150000in}{0.150000in}}{\pgfqpoint{5.700000in}{5.700000in}}%
\pgfusepath{clip}%
\pgfsetbuttcap%
\pgfsetroundjoin%
\definecolor{currentfill}{rgb}{0.281887,0.150881,0.465405}%
\pgfsetfillcolor{currentfill}%
\pgfsetfillopacity{0.700000}%
\pgfsetlinewidth{0.000000pt}%
\definecolor{currentstroke}{rgb}{0.000000,0.000000,0.000000}%
\pgfsetstrokecolor{currentstroke}%
\pgfsetdash{}{0pt}%
\pgfpathmoveto{\pgfqpoint{4.261034in}{2.008335in}}%
\pgfpathlineto{\pgfqpoint{4.274992in}{2.008797in}}%
\pgfpathlineto{\pgfqpoint{4.288960in}{2.009333in}}%
\pgfpathlineto{\pgfqpoint{4.302937in}{2.009942in}}%
\pgfpathlineto{\pgfqpoint{4.316923in}{2.010625in}}%
\pgfpathlineto{\pgfqpoint{4.324867in}{2.019626in}}%
\pgfpathlineto{\pgfqpoint{4.332805in}{2.028565in}}%
\pgfpathlineto{\pgfqpoint{4.340737in}{2.037440in}}%
\pgfpathlineto{\pgfqpoint{4.348662in}{2.046253in}}%
\pgfpathlineto{\pgfqpoint{4.334687in}{2.045572in}}%
\pgfpathlineto{\pgfqpoint{4.320720in}{2.044965in}}%
\pgfpathlineto{\pgfqpoint{4.306762in}{2.044431in}}%
\pgfpathlineto{\pgfqpoint{4.292814in}{2.043971in}}%
\pgfpathlineto{\pgfqpoint{4.284878in}{2.035148in}}%
\pgfpathlineto{\pgfqpoint{4.276936in}{2.026268in}}%
\pgfpathlineto{\pgfqpoint{4.268988in}{2.017331in}}%
\pgfpathlineto{\pgfqpoint{4.261034in}{2.008335in}}%
\pgfpathclose%
\pgfusepath{fill}%
\end{pgfscope}%
\begin{pgfscope}%
\pgfpathrectangle{\pgfqpoint{1.150000in}{0.150000in}}{\pgfqpoint{5.700000in}{5.700000in}}%
\pgfusepath{clip}%
\pgfsetbuttcap%
\pgfsetroundjoin%
\definecolor{currentfill}{rgb}{0.268510,0.009605,0.335427}%
\pgfsetfillcolor{currentfill}%
\pgfsetfillopacity{0.700000}%
\pgfsetlinewidth{0.000000pt}%
\definecolor{currentstroke}{rgb}{0.000000,0.000000,0.000000}%
\pgfsetstrokecolor{currentstroke}%
\pgfsetdash{}{0pt}%
\pgfpathmoveto{\pgfqpoint{3.217278in}{1.745956in}}%
\pgfpathlineto{\pgfqpoint{3.230994in}{1.741200in}}%
\pgfpathlineto{\pgfqpoint{3.244715in}{1.736529in}}%
\pgfpathlineto{\pgfqpoint{3.258441in}{1.731945in}}%
\pgfpathlineto{\pgfqpoint{3.272171in}{1.727445in}}%
\pgfpathlineto{\pgfqpoint{3.280504in}{1.735635in}}%
\pgfpathlineto{\pgfqpoint{3.288829in}{1.743880in}}%
\pgfpathlineto{\pgfqpoint{3.297148in}{1.752175in}}%
\pgfpathlineto{\pgfqpoint{3.305459in}{1.760518in}}%
\pgfpathlineto{\pgfqpoint{3.291745in}{1.764813in}}%
\pgfpathlineto{\pgfqpoint{3.278036in}{1.769193in}}%
\pgfpathlineto{\pgfqpoint{3.264331in}{1.773659in}}%
\pgfpathlineto{\pgfqpoint{3.250631in}{1.778210in}}%
\pgfpathlineto{\pgfqpoint{3.242304in}{1.770064in}}%
\pgfpathlineto{\pgfqpoint{3.233969in}{1.761971in}}%
\pgfpathlineto{\pgfqpoint{3.225627in}{1.753934in}}%
\pgfpathlineto{\pgfqpoint{3.217278in}{1.745956in}}%
\pgfpathclose%
\pgfusepath{fill}%
\end{pgfscope}%
\begin{pgfscope}%
\pgfpathrectangle{\pgfqpoint{1.150000in}{0.150000in}}{\pgfqpoint{5.700000in}{5.700000in}}%
\pgfusepath{clip}%
\pgfsetbuttcap%
\pgfsetroundjoin%
\definecolor{currentfill}{rgb}{0.250425,0.274290,0.533103}%
\pgfsetfillcolor{currentfill}%
\pgfsetfillopacity{0.700000}%
\pgfsetlinewidth{0.000000pt}%
\definecolor{currentstroke}{rgb}{0.000000,0.000000,0.000000}%
\pgfsetstrokecolor{currentstroke}%
\pgfsetdash{}{0pt}%
\pgfpathmoveto{\pgfqpoint{4.930424in}{2.277733in}}%
\pgfpathlineto{\pgfqpoint{4.944624in}{2.280044in}}%
\pgfpathlineto{\pgfqpoint{4.958836in}{2.282425in}}%
\pgfpathlineto{\pgfqpoint{4.973058in}{2.284876in}}%
\pgfpathlineto{\pgfqpoint{4.987292in}{2.287398in}}%
\pgfpathlineto{\pgfqpoint{4.994955in}{2.293950in}}%
\pgfpathlineto{\pgfqpoint{5.002611in}{2.300442in}}%
\pgfpathlineto{\pgfqpoint{5.010260in}{2.306876in}}%
\pgfpathlineto{\pgfqpoint{5.017902in}{2.313257in}}%
\pgfpathlineto{\pgfqpoint{5.003684in}{2.310886in}}%
\pgfpathlineto{\pgfqpoint{4.989476in}{2.308585in}}%
\pgfpathlineto{\pgfqpoint{4.975280in}{2.306354in}}%
\pgfpathlineto{\pgfqpoint{4.961094in}{2.304194in}}%
\pgfpathlineto{\pgfqpoint{4.953437in}{2.297655in}}%
\pgfpathlineto{\pgfqpoint{4.945773in}{2.291067in}}%
\pgfpathlineto{\pgfqpoint{4.938102in}{2.284428in}}%
\pgfpathlineto{\pgfqpoint{4.930424in}{2.277733in}}%
\pgfpathclose%
\pgfusepath{fill}%
\end{pgfscope}%
\begin{pgfscope}%
\pgfpathrectangle{\pgfqpoint{1.150000in}{0.150000in}}{\pgfqpoint{5.700000in}{5.700000in}}%
\pgfusepath{clip}%
\pgfsetbuttcap%
\pgfsetroundjoin%
\definecolor{currentfill}{rgb}{0.281412,0.155834,0.469201}%
\pgfsetfillcolor{currentfill}%
\pgfsetfillopacity{0.700000}%
\pgfsetlinewidth{0.000000pt}%
\definecolor{currentstroke}{rgb}{0.000000,0.000000,0.000000}%
\pgfsetstrokecolor{currentstroke}%
\pgfsetdash{}{0pt}%
\pgfpathmoveto{\pgfqpoint{2.333070in}{2.051667in}}%
\pgfpathlineto{\pgfqpoint{2.346820in}{2.040352in}}%
\pgfpathlineto{\pgfqpoint{2.360569in}{2.029157in}}%
\pgfpathlineto{\pgfqpoint{2.374317in}{2.018080in}}%
\pgfpathlineto{\pgfqpoint{2.388063in}{2.007120in}}%
\pgfpathlineto{\pgfqpoint{2.396917in}{2.009350in}}%
\pgfpathlineto{\pgfqpoint{2.405757in}{2.011780in}}%
\pgfpathlineto{\pgfqpoint{2.414582in}{2.014405in}}%
\pgfpathlineto{\pgfqpoint{2.423392in}{2.017219in}}%
\pgfpathlineto{\pgfqpoint{2.409678in}{2.027863in}}%
\pgfpathlineto{\pgfqpoint{2.395963in}{2.038625in}}%
\pgfpathlineto{\pgfqpoint{2.382247in}{2.049504in}}%
\pgfpathlineto{\pgfqpoint{2.368530in}{2.060503in}}%
\pgfpathlineto{\pgfqpoint{2.359688in}{2.057996in}}%
\pgfpathlineto{\pgfqpoint{2.350830in}{2.055685in}}%
\pgfpathlineto{\pgfqpoint{2.341958in}{2.053573in}}%
\pgfpathlineto{\pgfqpoint{2.333070in}{2.051667in}}%
\pgfpathclose%
\pgfusepath{fill}%
\end{pgfscope}%
\begin{pgfscope}%
\pgfpathrectangle{\pgfqpoint{1.150000in}{0.150000in}}{\pgfqpoint{5.700000in}{5.700000in}}%
\pgfusepath{clip}%
\pgfsetbuttcap%
\pgfsetroundjoin%
\definecolor{currentfill}{rgb}{0.281446,0.084320,0.407414}%
\pgfsetfillcolor{currentfill}%
\pgfsetfillopacity{0.700000}%
\pgfsetlinewidth{0.000000pt}%
\definecolor{currentstroke}{rgb}{0.000000,0.000000,0.000000}%
\pgfsetstrokecolor{currentstroke}%
\pgfsetdash{}{0pt}%
\pgfpathmoveto{\pgfqpoint{2.587922in}{1.898293in}}%
\pgfpathlineto{\pgfqpoint{2.601634in}{1.889091in}}%
\pgfpathlineto{\pgfqpoint{2.615346in}{1.879996in}}%
\pgfpathlineto{\pgfqpoint{2.629058in}{1.871005in}}%
\pgfpathlineto{\pgfqpoint{2.642772in}{1.862118in}}%
\pgfpathlineto{\pgfqpoint{2.651450in}{1.866317in}}%
\pgfpathlineto{\pgfqpoint{2.660116in}{1.870675in}}%
\pgfpathlineto{\pgfqpoint{2.668770in}{1.875187in}}%
\pgfpathlineto{\pgfqpoint{2.677412in}{1.879851in}}%
\pgfpathlineto{\pgfqpoint{2.663726in}{1.888447in}}%
\pgfpathlineto{\pgfqpoint{2.650040in}{1.897147in}}%
\pgfpathlineto{\pgfqpoint{2.636356in}{1.905952in}}%
\pgfpathlineto{\pgfqpoint{2.622672in}{1.914863in}}%
\pgfpathlineto{\pgfqpoint{2.614003in}{1.910482in}}%
\pgfpathlineto{\pgfqpoint{2.605322in}{1.906257in}}%
\pgfpathlineto{\pgfqpoint{2.596629in}{1.902193in}}%
\pgfpathlineto{\pgfqpoint{2.587922in}{1.898293in}}%
\pgfpathclose%
\pgfusepath{fill}%
\end{pgfscope}%
\begin{pgfscope}%
\pgfpathrectangle{\pgfqpoint{1.150000in}{0.150000in}}{\pgfqpoint{5.700000in}{5.700000in}}%
\pgfusepath{clip}%
\pgfsetbuttcap%
\pgfsetroundjoin%
\definecolor{currentfill}{rgb}{0.282884,0.135920,0.453427}%
\pgfsetfillcolor{currentfill}%
\pgfsetfillopacity{0.700000}%
\pgfsetlinewidth{0.000000pt}%
\definecolor{currentstroke}{rgb}{0.000000,0.000000,0.000000}%
\pgfsetstrokecolor{currentstroke}%
\pgfsetdash{}{0pt}%
\pgfpathmoveto{\pgfqpoint{4.173375in}{1.970731in}}%
\pgfpathlineto{\pgfqpoint{4.187308in}{1.970878in}}%
\pgfpathlineto{\pgfqpoint{4.201250in}{1.971099in}}%
\pgfpathlineto{\pgfqpoint{4.215200in}{1.971394in}}%
\pgfpathlineto{\pgfqpoint{4.229159in}{1.971763in}}%
\pgfpathlineto{\pgfqpoint{4.237137in}{1.980995in}}%
\pgfpathlineto{\pgfqpoint{4.245108in}{1.990167in}}%
\pgfpathlineto{\pgfqpoint{4.253074in}{1.999281in}}%
\pgfpathlineto{\pgfqpoint{4.261034in}{2.008335in}}%
\pgfpathlineto{\pgfqpoint{4.247084in}{2.007947in}}%
\pgfpathlineto{\pgfqpoint{4.233144in}{2.007633in}}%
\pgfpathlineto{\pgfqpoint{4.219212in}{2.007393in}}%
\pgfpathlineto{\pgfqpoint{4.205290in}{2.007227in}}%
\pgfpathlineto{\pgfqpoint{4.197320in}{1.998184in}}%
\pgfpathlineto{\pgfqpoint{4.189344in}{1.989087in}}%
\pgfpathlineto{\pgfqpoint{4.181363in}{1.979936in}}%
\pgfpathlineto{\pgfqpoint{4.173375in}{1.970731in}}%
\pgfpathclose%
\pgfusepath{fill}%
\end{pgfscope}%
\begin{pgfscope}%
\pgfpathrectangle{\pgfqpoint{1.150000in}{0.150000in}}{\pgfqpoint{5.700000in}{5.700000in}}%
\pgfusepath{clip}%
\pgfsetbuttcap%
\pgfsetroundjoin%
\definecolor{currentfill}{rgb}{0.216210,0.351535,0.550627}%
\pgfsetfillcolor{currentfill}%
\pgfsetfillopacity{0.700000}%
\pgfsetlinewidth{0.000000pt}%
\definecolor{currentstroke}{rgb}{0.000000,0.000000,0.000000}%
\pgfsetstrokecolor{currentstroke}%
\pgfsetdash{}{0pt}%
\pgfpathmoveto{\pgfqpoint{5.424637in}{2.456570in}}%
\pgfpathlineto{\pgfqpoint{5.439029in}{2.459588in}}%
\pgfpathlineto{\pgfqpoint{5.453433in}{2.462675in}}%
\pgfpathlineto{\pgfqpoint{5.467849in}{2.465831in}}%
\pgfpathlineto{\pgfqpoint{5.482277in}{2.469056in}}%
\pgfpathlineto{\pgfqpoint{5.489689in}{2.473563in}}%
\pgfpathlineto{\pgfqpoint{5.497094in}{2.478053in}}%
\pgfpathlineto{\pgfqpoint{5.504492in}{2.482531in}}%
\pgfpathlineto{\pgfqpoint{5.511884in}{2.487001in}}%
\pgfpathlineto{\pgfqpoint{5.497477in}{2.484034in}}%
\pgfpathlineto{\pgfqpoint{5.483082in}{2.481136in}}%
\pgfpathlineto{\pgfqpoint{5.468700in}{2.478306in}}%
\pgfpathlineto{\pgfqpoint{5.454329in}{2.475545in}}%
\pgfpathlineto{\pgfqpoint{5.446916in}{2.470810in}}%
\pgfpathlineto{\pgfqpoint{5.439496in}{2.466073in}}%
\pgfpathlineto{\pgfqpoint{5.432070in}{2.461327in}}%
\pgfpathlineto{\pgfqpoint{5.424637in}{2.456570in}}%
\pgfpathclose%
\pgfusepath{fill}%
\end{pgfscope}%
\begin{pgfscope}%
\pgfpathrectangle{\pgfqpoint{1.150000in}{0.150000in}}{\pgfqpoint{5.700000in}{5.700000in}}%
\pgfusepath{clip}%
\pgfsetbuttcap%
\pgfsetroundjoin%
\definecolor{currentfill}{rgb}{0.276022,0.044167,0.370164}%
\pgfsetfillcolor{currentfill}%
\pgfsetfillopacity{0.700000}%
\pgfsetlinewidth{0.000000pt}%
\definecolor{currentstroke}{rgb}{0.000000,0.000000,0.000000}%
\pgfsetstrokecolor{currentstroke}%
\pgfsetdash{}{0pt}%
\pgfpathmoveto{\pgfqpoint{2.786949in}{1.814741in}}%
\pgfpathlineto{\pgfqpoint{2.800649in}{1.807050in}}%
\pgfpathlineto{\pgfqpoint{2.814351in}{1.799457in}}%
\pgfpathlineto{\pgfqpoint{2.828054in}{1.791960in}}%
\pgfpathlineto{\pgfqpoint{2.841760in}{1.784561in}}%
\pgfpathlineto{\pgfqpoint{2.850316in}{1.790199in}}%
\pgfpathlineto{\pgfqpoint{2.858861in}{1.795965in}}%
\pgfpathlineto{\pgfqpoint{2.867397in}{1.801852in}}%
\pgfpathlineto{\pgfqpoint{2.875922in}{1.807859in}}%
\pgfpathlineto{\pgfqpoint{2.862239in}{1.814991in}}%
\pgfpathlineto{\pgfqpoint{2.848559in}{1.822219in}}%
\pgfpathlineto{\pgfqpoint{2.834881in}{1.829545in}}%
\pgfpathlineto{\pgfqpoint{2.821205in}{1.836967in}}%
\pgfpathlineto{\pgfqpoint{2.812657in}{1.831221in}}%
\pgfpathlineto{\pgfqpoint{2.804098in}{1.825598in}}%
\pgfpathlineto{\pgfqpoint{2.795529in}{1.820104in}}%
\pgfpathlineto{\pgfqpoint{2.786949in}{1.814741in}}%
\pgfpathclose%
\pgfusepath{fill}%
\end{pgfscope}%
\begin{pgfscope}%
\pgfpathrectangle{\pgfqpoint{1.150000in}{0.150000in}}{\pgfqpoint{5.700000in}{5.700000in}}%
\pgfusepath{clip}%
\pgfsetbuttcap%
\pgfsetroundjoin%
\definecolor{currentfill}{rgb}{0.276022,0.044167,0.370164}%
\pgfsetfillcolor{currentfill}%
\pgfsetfillopacity{0.700000}%
\pgfsetlinewidth{0.000000pt}%
\definecolor{currentstroke}{rgb}{0.000000,0.000000,0.000000}%
\pgfsetstrokecolor{currentstroke}%
\pgfsetdash{}{0pt}%
\pgfpathmoveto{\pgfqpoint{3.679287in}{1.804746in}}%
\pgfpathlineto{\pgfqpoint{3.693084in}{1.802655in}}%
\pgfpathlineto{\pgfqpoint{3.706888in}{1.800642in}}%
\pgfpathlineto{\pgfqpoint{3.720699in}{1.798707in}}%
\pgfpathlineto{\pgfqpoint{3.734516in}{1.796851in}}%
\pgfpathlineto{\pgfqpoint{3.742668in}{1.806383in}}%
\pgfpathlineto{\pgfqpoint{3.750813in}{1.815902in}}%
\pgfpathlineto{\pgfqpoint{3.758953in}{1.825406in}}%
\pgfpathlineto{\pgfqpoint{3.767087in}{1.834894in}}%
\pgfpathlineto{\pgfqpoint{3.753281in}{1.836628in}}%
\pgfpathlineto{\pgfqpoint{3.739481in}{1.838440in}}%
\pgfpathlineto{\pgfqpoint{3.725689in}{1.840330in}}%
\pgfpathlineto{\pgfqpoint{3.711904in}{1.842299in}}%
\pgfpathlineto{\pgfqpoint{3.703758in}{1.832926in}}%
\pgfpathlineto{\pgfqpoint{3.695607in}{1.823542in}}%
\pgfpathlineto{\pgfqpoint{3.687450in}{1.814148in}}%
\pgfpathlineto{\pgfqpoint{3.679287in}{1.804746in}}%
\pgfpathclose%
\pgfusepath{fill}%
\end{pgfscope}%
\begin{pgfscope}%
\pgfpathrectangle{\pgfqpoint{1.150000in}{0.150000in}}{\pgfqpoint{5.700000in}{5.700000in}}%
\pgfusepath{clip}%
\pgfsetbuttcap%
\pgfsetroundjoin%
\definecolor{currentfill}{rgb}{0.255645,0.260703,0.528312}%
\pgfsetfillcolor{currentfill}%
\pgfsetfillopacity{0.700000}%
\pgfsetlinewidth{0.000000pt}%
\definecolor{currentstroke}{rgb}{0.000000,0.000000,0.000000}%
\pgfsetstrokecolor{currentstroke}%
\pgfsetdash{}{0pt}%
\pgfpathmoveto{\pgfqpoint{4.842894in}{2.241306in}}%
\pgfpathlineto{\pgfqpoint{4.857065in}{2.243462in}}%
\pgfpathlineto{\pgfqpoint{4.871246in}{2.245689in}}%
\pgfpathlineto{\pgfqpoint{4.885438in}{2.247988in}}%
\pgfpathlineto{\pgfqpoint{4.899642in}{2.250357in}}%
\pgfpathlineto{\pgfqpoint{4.907348in}{2.257296in}}%
\pgfpathlineto{\pgfqpoint{4.915047in}{2.264170in}}%
\pgfpathlineto{\pgfqpoint{4.922739in}{2.270982in}}%
\pgfpathlineto{\pgfqpoint{4.930424in}{2.277733in}}%
\pgfpathlineto{\pgfqpoint{4.916235in}{2.275494in}}%
\pgfpathlineto{\pgfqpoint{4.902056in}{2.273325in}}%
\pgfpathlineto{\pgfqpoint{4.887889in}{2.271226in}}%
\pgfpathlineto{\pgfqpoint{4.873732in}{2.269198in}}%
\pgfpathlineto{\pgfqpoint{4.866033in}{2.262310in}}%
\pgfpathlineto{\pgfqpoint{4.858327in}{2.255367in}}%
\pgfpathlineto{\pgfqpoint{4.850614in}{2.248366in}}%
\pgfpathlineto{\pgfqpoint{4.842894in}{2.241306in}}%
\pgfpathclose%
\pgfusepath{fill}%
\end{pgfscope}%
\begin{pgfscope}%
\pgfpathrectangle{\pgfqpoint{1.150000in}{0.150000in}}{\pgfqpoint{5.700000in}{5.700000in}}%
\pgfusepath{clip}%
\pgfsetbuttcap%
\pgfsetroundjoin%
\definecolor{currentfill}{rgb}{0.283229,0.120777,0.440584}%
\pgfsetfillcolor{currentfill}%
\pgfsetfillopacity{0.700000}%
\pgfsetlinewidth{0.000000pt}%
\definecolor{currentstroke}{rgb}{0.000000,0.000000,0.000000}%
\pgfsetstrokecolor{currentstroke}%
\pgfsetdash{}{0pt}%
\pgfpathmoveto{\pgfqpoint{4.085686in}{1.933689in}}%
\pgfpathlineto{\pgfqpoint{4.099594in}{1.933497in}}%
\pgfpathlineto{\pgfqpoint{4.113510in}{1.933380in}}%
\pgfpathlineto{\pgfqpoint{4.127435in}{1.933338in}}%
\pgfpathlineto{\pgfqpoint{4.141369in}{1.933371in}}%
\pgfpathlineto{\pgfqpoint{4.149379in}{1.942792in}}%
\pgfpathlineto{\pgfqpoint{4.157384in}{1.952159in}}%
\pgfpathlineto{\pgfqpoint{4.165382in}{1.961472in}}%
\pgfpathlineto{\pgfqpoint{4.173375in}{1.970731in}}%
\pgfpathlineto{\pgfqpoint{4.159451in}{1.970659in}}%
\pgfpathlineto{\pgfqpoint{4.145536in}{1.970661in}}%
\pgfpathlineto{\pgfqpoint{4.131630in}{1.970738in}}%
\pgfpathlineto{\pgfqpoint{4.117731in}{1.970889in}}%
\pgfpathlineto{\pgfqpoint{4.109728in}{1.961663in}}%
\pgfpathlineto{\pgfqpoint{4.101720in}{1.952387in}}%
\pgfpathlineto{\pgfqpoint{4.093706in}{1.943062in}}%
\pgfpathlineto{\pgfqpoint{4.085686in}{1.933689in}}%
\pgfpathclose%
\pgfusepath{fill}%
\end{pgfscope}%
\begin{pgfscope}%
\pgfpathrectangle{\pgfqpoint{1.150000in}{0.150000in}}{\pgfqpoint{5.700000in}{5.700000in}}%
\pgfusepath{clip}%
\pgfsetbuttcap%
\pgfsetroundjoin%
\definecolor{currentfill}{rgb}{0.269944,0.014625,0.341379}%
\pgfsetfillcolor{currentfill}%
\pgfsetfillopacity{0.700000}%
\pgfsetlinewidth{0.000000pt}%
\definecolor{currentstroke}{rgb}{0.000000,0.000000,0.000000}%
\pgfsetstrokecolor{currentstroke}%
\pgfsetdash{}{0pt}%
\pgfpathmoveto{\pgfqpoint{3.360366in}{1.744182in}}%
\pgfpathlineto{\pgfqpoint{3.374106in}{1.740307in}}%
\pgfpathlineto{\pgfqpoint{3.387851in}{1.736515in}}%
\pgfpathlineto{\pgfqpoint{3.401601in}{1.732805in}}%
\pgfpathlineto{\pgfqpoint{3.415357in}{1.729178in}}%
\pgfpathlineto{\pgfqpoint{3.423631in}{1.737950in}}%
\pgfpathlineto{\pgfqpoint{3.431899in}{1.746755in}}%
\pgfpathlineto{\pgfqpoint{3.440161in}{1.755589in}}%
\pgfpathlineto{\pgfqpoint{3.448416in}{1.764451in}}%
\pgfpathlineto{\pgfqpoint{3.434674in}{1.767894in}}%
\pgfpathlineto{\pgfqpoint{3.420939in}{1.771419in}}%
\pgfpathlineto{\pgfqpoint{3.407208in}{1.775027in}}%
\pgfpathlineto{\pgfqpoint{3.393483in}{1.778718in}}%
\pgfpathlineto{\pgfqpoint{3.385214in}{1.770033in}}%
\pgfpathlineto{\pgfqpoint{3.376938in}{1.761380in}}%
\pgfpathlineto{\pgfqpoint{3.368655in}{1.752762in}}%
\pgfpathlineto{\pgfqpoint{3.360366in}{1.744182in}}%
\pgfpathclose%
\pgfusepath{fill}%
\end{pgfscope}%
\begin{pgfscope}%
\pgfpathrectangle{\pgfqpoint{1.150000in}{0.150000in}}{\pgfqpoint{5.700000in}{5.700000in}}%
\pgfusepath{clip}%
\pgfsetbuttcap%
\pgfsetroundjoin%
\definecolor{currentfill}{rgb}{0.197636,0.391528,0.554969}%
\pgfsetfillcolor{currentfill}%
\pgfsetfillopacity{0.700000}%
\pgfsetlinewidth{0.000000pt}%
\definecolor{currentstroke}{rgb}{0.000000,0.000000,0.000000}%
\pgfsetstrokecolor{currentstroke}%
\pgfsetdash{}{0pt}%
\pgfpathmoveto{\pgfqpoint{5.744073in}{2.557070in}}%
\pgfpathlineto{\pgfqpoint{5.758594in}{2.560347in}}%
\pgfpathlineto{\pgfqpoint{5.773128in}{2.563693in}}%
\pgfpathlineto{\pgfqpoint{5.787674in}{2.567106in}}%
\pgfpathlineto{\pgfqpoint{5.794919in}{2.570556in}}%
\pgfpathlineto{\pgfqpoint{5.802158in}{2.574029in}}%
\pgfpathlineto{\pgfqpoint{5.809391in}{2.577533in}}%
\pgfpathlineto{\pgfqpoint{5.816619in}{2.581073in}}%
\pgfpathlineto{\pgfqpoint{5.802099in}{2.577981in}}%
\pgfpathlineto{\pgfqpoint{5.787592in}{2.574958in}}%
\pgfpathlineto{\pgfqpoint{5.773097in}{2.572002in}}%
\pgfpathlineto{\pgfqpoint{5.765849in}{2.568215in}}%
\pgfpathlineto{\pgfqpoint{5.758596in}{2.564468in}}%
\pgfpathlineto{\pgfqpoint{5.751338in}{2.560755in}}%
\pgfpathlineto{\pgfqpoint{5.744073in}{2.557070in}}%
\pgfpathclose%
\pgfusepath{fill}%
\end{pgfscope}%
\begin{pgfscope}%
\pgfpathrectangle{\pgfqpoint{1.150000in}{0.150000in}}{\pgfqpoint{5.700000in}{5.700000in}}%
\pgfusepath{clip}%
\pgfsetbuttcap%
\pgfsetroundjoin%
\definecolor{currentfill}{rgb}{0.260571,0.246922,0.522828}%
\pgfsetfillcolor{currentfill}%
\pgfsetfillopacity{0.700000}%
\pgfsetlinewidth{0.000000pt}%
\definecolor{currentstroke}{rgb}{0.000000,0.000000,0.000000}%
\pgfsetstrokecolor{currentstroke}%
\pgfsetdash{}{0pt}%
\pgfpathmoveto{\pgfqpoint{4.755319in}{2.204069in}}%
\pgfpathlineto{\pgfqpoint{4.769460in}{2.206049in}}%
\pgfpathlineto{\pgfqpoint{4.783611in}{2.208100in}}%
\pgfpathlineto{\pgfqpoint{4.797773in}{2.210222in}}%
\pgfpathlineto{\pgfqpoint{4.811946in}{2.212416in}}%
\pgfpathlineto{\pgfqpoint{4.819693in}{2.219740in}}%
\pgfpathlineto{\pgfqpoint{4.827434in}{2.226995in}}%
\pgfpathlineto{\pgfqpoint{4.835168in}{2.234182in}}%
\pgfpathlineto{\pgfqpoint{4.842894in}{2.241306in}}%
\pgfpathlineto{\pgfqpoint{4.828735in}{2.239220in}}%
\pgfpathlineto{\pgfqpoint{4.814586in}{2.237206in}}%
\pgfpathlineto{\pgfqpoint{4.800448in}{2.235262in}}%
\pgfpathlineto{\pgfqpoint{4.786320in}{2.233390in}}%
\pgfpathlineto{\pgfqpoint{4.778580in}{2.226151in}}%
\pgfpathlineto{\pgfqpoint{4.770833in}{2.218853in}}%
\pgfpathlineto{\pgfqpoint{4.763080in}{2.211493in}}%
\pgfpathlineto{\pgfqpoint{4.755319in}{2.204069in}}%
\pgfpathclose%
\pgfusepath{fill}%
\end{pgfscope}%
\begin{pgfscope}%
\pgfpathrectangle{\pgfqpoint{1.150000in}{0.150000in}}{\pgfqpoint{5.700000in}{5.700000in}}%
\pgfusepath{clip}%
\pgfsetbuttcap%
\pgfsetroundjoin%
\definecolor{currentfill}{rgb}{0.221989,0.339161,0.548752}%
\pgfsetfillcolor{currentfill}%
\pgfsetfillopacity{0.700000}%
\pgfsetlinewidth{0.000000pt}%
\definecolor{currentstroke}{rgb}{0.000000,0.000000,0.000000}%
\pgfsetstrokecolor{currentstroke}%
\pgfsetdash{}{0pt}%
\pgfpathmoveto{\pgfqpoint{5.337309in}{2.425005in}}%
\pgfpathlineto{\pgfqpoint{5.351673in}{2.427982in}}%
\pgfpathlineto{\pgfqpoint{5.366048in}{2.431030in}}%
\pgfpathlineto{\pgfqpoint{5.380436in}{2.434146in}}%
\pgfpathlineto{\pgfqpoint{5.394836in}{2.437332in}}%
\pgfpathlineto{\pgfqpoint{5.402297in}{2.442181in}}%
\pgfpathlineto{\pgfqpoint{5.409751in}{2.447001in}}%
\pgfpathlineto{\pgfqpoint{5.417197in}{2.451796in}}%
\pgfpathlineto{\pgfqpoint{5.424637in}{2.456570in}}%
\pgfpathlineto{\pgfqpoint{5.410257in}{2.453621in}}%
\pgfpathlineto{\pgfqpoint{5.395890in}{2.450741in}}%
\pgfpathlineto{\pgfqpoint{5.381534in}{2.447930in}}%
\pgfpathlineto{\pgfqpoint{5.367190in}{2.445188in}}%
\pgfpathlineto{\pgfqpoint{5.359730in}{2.440171in}}%
\pgfpathlineto{\pgfqpoint{5.352263in}{2.435137in}}%
\pgfpathlineto{\pgfqpoint{5.344790in}{2.430083in}}%
\pgfpathlineto{\pgfqpoint{5.337309in}{2.425005in}}%
\pgfpathclose%
\pgfusepath{fill}%
\end{pgfscope}%
\begin{pgfscope}%
\pgfpathrectangle{\pgfqpoint{1.150000in}{0.150000in}}{\pgfqpoint{5.700000in}{5.700000in}}%
\pgfusepath{clip}%
\pgfsetbuttcap%
\pgfsetroundjoin%
\definecolor{currentfill}{rgb}{0.282623,0.140926,0.457517}%
\pgfsetfillcolor{currentfill}%
\pgfsetfillopacity{0.700000}%
\pgfsetlinewidth{0.000000pt}%
\definecolor{currentstroke}{rgb}{0.000000,0.000000,0.000000}%
\pgfsetstrokecolor{currentstroke}%
\pgfsetdash{}{0pt}%
\pgfpathmoveto{\pgfqpoint{2.388063in}{2.007120in}}%
\pgfpathlineto{\pgfqpoint{2.401808in}{1.996276in}}%
\pgfpathlineto{\pgfqpoint{2.415552in}{1.985548in}}%
\pgfpathlineto{\pgfqpoint{2.429295in}{1.974934in}}%
\pgfpathlineto{\pgfqpoint{2.443038in}{1.964435in}}%
\pgfpathlineto{\pgfqpoint{2.451860in}{1.966988in}}%
\pgfpathlineto{\pgfqpoint{2.460667in}{1.969736in}}%
\pgfpathlineto{\pgfqpoint{2.469460in}{1.972672in}}%
\pgfpathlineto{\pgfqpoint{2.478239in}{1.975792in}}%
\pgfpathlineto{\pgfqpoint{2.464528in}{1.985978in}}%
\pgfpathlineto{\pgfqpoint{2.450817in}{1.996277in}}%
\pgfpathlineto{\pgfqpoint{2.437105in}{2.006690in}}%
\pgfpathlineto{\pgfqpoint{2.423392in}{2.017219in}}%
\pgfpathlineto{\pgfqpoint{2.414582in}{2.014405in}}%
\pgfpathlineto{\pgfqpoint{2.405757in}{2.011780in}}%
\pgfpathlineto{\pgfqpoint{2.396917in}{2.009350in}}%
\pgfpathlineto{\pgfqpoint{2.388063in}{2.007120in}}%
\pgfpathclose%
\pgfusepath{fill}%
\end{pgfscope}%
\begin{pgfscope}%
\pgfpathrectangle{\pgfqpoint{1.150000in}{0.150000in}}{\pgfqpoint{5.700000in}{5.700000in}}%
\pgfusepath{clip}%
\pgfsetbuttcap%
\pgfsetroundjoin%
\definecolor{currentfill}{rgb}{0.273809,0.031497,0.358853}%
\pgfsetfillcolor{currentfill}%
\pgfsetfillopacity{0.700000}%
\pgfsetlinewidth{0.000000pt}%
\definecolor{currentstroke}{rgb}{0.000000,0.000000,0.000000}%
\pgfsetstrokecolor{currentstroke}%
\pgfsetdash{}{0pt}%
\pgfpathmoveto{\pgfqpoint{3.591410in}{1.776821in}}%
\pgfpathlineto{\pgfqpoint{3.605193in}{1.774269in}}%
\pgfpathlineto{\pgfqpoint{3.618982in}{1.771798in}}%
\pgfpathlineto{\pgfqpoint{3.632777in}{1.769405in}}%
\pgfpathlineto{\pgfqpoint{3.646579in}{1.767092in}}%
\pgfpathlineto{\pgfqpoint{3.654765in}{1.776509in}}%
\pgfpathlineto{\pgfqpoint{3.662945in}{1.785925in}}%
\pgfpathlineto{\pgfqpoint{3.671119in}{1.795338in}}%
\pgfpathlineto{\pgfqpoint{3.679287in}{1.804746in}}%
\pgfpathlineto{\pgfqpoint{3.665497in}{1.806917in}}%
\pgfpathlineto{\pgfqpoint{3.651714in}{1.809166in}}%
\pgfpathlineto{\pgfqpoint{3.637937in}{1.811495in}}%
\pgfpathlineto{\pgfqpoint{3.624167in}{1.813903in}}%
\pgfpathlineto{\pgfqpoint{3.615986in}{1.804630in}}%
\pgfpathlineto{\pgfqpoint{3.607800in}{1.795357in}}%
\pgfpathlineto{\pgfqpoint{3.599608in}{1.786087in}}%
\pgfpathlineto{\pgfqpoint{3.591410in}{1.776821in}}%
\pgfpathclose%
\pgfusepath{fill}%
\end{pgfscope}%
\begin{pgfscope}%
\pgfpathrectangle{\pgfqpoint{1.150000in}{0.150000in}}{\pgfqpoint{5.700000in}{5.700000in}}%
\pgfusepath{clip}%
\pgfsetbuttcap%
\pgfsetroundjoin%
\definecolor{currentfill}{rgb}{0.282656,0.100196,0.422160}%
\pgfsetfillcolor{currentfill}%
\pgfsetfillopacity{0.700000}%
\pgfsetlinewidth{0.000000pt}%
\definecolor{currentstroke}{rgb}{0.000000,0.000000,0.000000}%
\pgfsetstrokecolor{currentstroke}%
\pgfsetdash{}{0pt}%
\pgfpathmoveto{\pgfqpoint{3.997961in}{1.897478in}}%
\pgfpathlineto{\pgfqpoint{4.011846in}{1.896925in}}%
\pgfpathlineto{\pgfqpoint{4.025739in}{1.896447in}}%
\pgfpathlineto{\pgfqpoint{4.039640in}{1.896045in}}%
\pgfpathlineto{\pgfqpoint{4.053549in}{1.895717in}}%
\pgfpathlineto{\pgfqpoint{4.061592in}{1.905281in}}%
\pgfpathlineto{\pgfqpoint{4.069629in}{1.914798in}}%
\pgfpathlineto{\pgfqpoint{4.077660in}{1.924267in}}%
\pgfpathlineto{\pgfqpoint{4.085686in}{1.933689in}}%
\pgfpathlineto{\pgfqpoint{4.071786in}{1.933956in}}%
\pgfpathlineto{\pgfqpoint{4.057895in}{1.934297in}}%
\pgfpathlineto{\pgfqpoint{4.044012in}{1.934714in}}%
\pgfpathlineto{\pgfqpoint{4.030137in}{1.935207in}}%
\pgfpathlineto{\pgfqpoint{4.022102in}{1.925838in}}%
\pgfpathlineto{\pgfqpoint{4.014060in}{1.916427in}}%
\pgfpathlineto{\pgfqpoint{4.006014in}{1.906973in}}%
\pgfpathlineto{\pgfqpoint{3.997961in}{1.897478in}}%
\pgfpathclose%
\pgfusepath{fill}%
\end{pgfscope}%
\begin{pgfscope}%
\pgfpathrectangle{\pgfqpoint{1.150000in}{0.150000in}}{\pgfqpoint{5.700000in}{5.700000in}}%
\pgfusepath{clip}%
\pgfsetbuttcap%
\pgfsetroundjoin%
\definecolor{currentfill}{rgb}{0.265145,0.232956,0.516599}%
\pgfsetfillcolor{currentfill}%
\pgfsetfillopacity{0.700000}%
\pgfsetlinewidth{0.000000pt}%
\definecolor{currentstroke}{rgb}{0.000000,0.000000,0.000000}%
\pgfsetstrokecolor{currentstroke}%
\pgfsetdash{}{0pt}%
\pgfpathmoveto{\pgfqpoint{4.667704in}{2.166141in}}%
\pgfpathlineto{\pgfqpoint{4.681815in}{2.167921in}}%
\pgfpathlineto{\pgfqpoint{4.695936in}{2.169773in}}%
\pgfpathlineto{\pgfqpoint{4.710068in}{2.171697in}}%
\pgfpathlineto{\pgfqpoint{4.724210in}{2.173692in}}%
\pgfpathlineto{\pgfqpoint{4.731998in}{2.181392in}}%
\pgfpathlineto{\pgfqpoint{4.739778in}{2.189021in}}%
\pgfpathlineto{\pgfqpoint{4.747552in}{2.196579in}}%
\pgfpathlineto{\pgfqpoint{4.755319in}{2.204069in}}%
\pgfpathlineto{\pgfqpoint{4.741190in}{2.202160in}}%
\pgfpathlineto{\pgfqpoint{4.727070in}{2.200323in}}%
\pgfpathlineto{\pgfqpoint{4.712961in}{2.198557in}}%
\pgfpathlineto{\pgfqpoint{4.698863in}{2.196863in}}%
\pgfpathlineto{\pgfqpoint{4.691083in}{2.189279in}}%
\pgfpathlineto{\pgfqpoint{4.683297in}{2.181632in}}%
\pgfpathlineto{\pgfqpoint{4.675504in}{2.173919in}}%
\pgfpathlineto{\pgfqpoint{4.667704in}{2.166141in}}%
\pgfpathclose%
\pgfusepath{fill}%
\end{pgfscope}%
\begin{pgfscope}%
\pgfpathrectangle{\pgfqpoint{1.150000in}{0.150000in}}{\pgfqpoint{5.700000in}{5.700000in}}%
\pgfusepath{clip}%
\pgfsetbuttcap%
\pgfsetroundjoin%
\definecolor{currentfill}{rgb}{0.280267,0.073417,0.397163}%
\pgfsetfillcolor{currentfill}%
\pgfsetfillopacity{0.700000}%
\pgfsetlinewidth{0.000000pt}%
\definecolor{currentstroke}{rgb}{0.000000,0.000000,0.000000}%
\pgfsetstrokecolor{currentstroke}%
\pgfsetdash{}{0pt}%
\pgfpathmoveto{\pgfqpoint{2.642772in}{1.862118in}}%
\pgfpathlineto{\pgfqpoint{2.656486in}{1.853334in}}%
\pgfpathlineto{\pgfqpoint{2.670201in}{1.844654in}}%
\pgfpathlineto{\pgfqpoint{2.683917in}{1.836075in}}%
\pgfpathlineto{\pgfqpoint{2.697635in}{1.827598in}}%
\pgfpathlineto{\pgfqpoint{2.706286in}{1.832095in}}%
\pgfpathlineto{\pgfqpoint{2.714925in}{1.836745in}}%
\pgfpathlineto{\pgfqpoint{2.723553in}{1.841545in}}%
\pgfpathlineto{\pgfqpoint{2.732169in}{1.846491in}}%
\pgfpathlineto{\pgfqpoint{2.718477in}{1.854678in}}%
\pgfpathlineto{\pgfqpoint{2.704788in}{1.862967in}}%
\pgfpathlineto{\pgfqpoint{2.691099in}{1.871358in}}%
\pgfpathlineto{\pgfqpoint{2.677412in}{1.879851in}}%
\pgfpathlineto{\pgfqpoint{2.668770in}{1.875187in}}%
\pgfpathlineto{\pgfqpoint{2.660116in}{1.870675in}}%
\pgfpathlineto{\pgfqpoint{2.651450in}{1.866317in}}%
\pgfpathlineto{\pgfqpoint{2.642772in}{1.862118in}}%
\pgfpathclose%
\pgfusepath{fill}%
\end{pgfscope}%
\begin{pgfscope}%
\pgfpathrectangle{\pgfqpoint{1.150000in}{0.150000in}}{\pgfqpoint{5.700000in}{5.700000in}}%
\pgfusepath{clip}%
\pgfsetbuttcap%
\pgfsetroundjoin%
\definecolor{currentfill}{rgb}{0.269944,0.014625,0.341379}%
\pgfsetfillcolor{currentfill}%
\pgfsetfillopacity{0.700000}%
\pgfsetlinewidth{0.000000pt}%
\definecolor{currentstroke}{rgb}{0.000000,0.000000,0.000000}%
\pgfsetstrokecolor{currentstroke}%
\pgfsetdash{}{0pt}%
\pgfpathmoveto{\pgfqpoint{2.985478in}{1.754202in}}%
\pgfpathlineto{\pgfqpoint{2.999186in}{1.747911in}}%
\pgfpathlineto{\pgfqpoint{3.012897in}{1.741712in}}%
\pgfpathlineto{\pgfqpoint{3.026611in}{1.735604in}}%
\pgfpathlineto{\pgfqpoint{3.040329in}{1.729586in}}%
\pgfpathlineto{\pgfqpoint{3.048780in}{1.736470in}}%
\pgfpathlineto{\pgfqpoint{3.057223in}{1.743450in}}%
\pgfpathlineto{\pgfqpoint{3.065657in}{1.750522in}}%
\pgfpathlineto{\pgfqpoint{3.074082in}{1.757682in}}%
\pgfpathlineto{\pgfqpoint{3.060384in}{1.763454in}}%
\pgfpathlineto{\pgfqpoint{3.046690in}{1.769317in}}%
\pgfpathlineto{\pgfqpoint{3.032999in}{1.775269in}}%
\pgfpathlineto{\pgfqpoint{3.019312in}{1.781313in}}%
\pgfpathlineto{\pgfqpoint{3.010867in}{1.774391in}}%
\pgfpathlineto{\pgfqpoint{3.002413in}{1.767563in}}%
\pgfpathlineto{\pgfqpoint{2.993950in}{1.760832in}}%
\pgfpathlineto{\pgfqpoint{2.985478in}{1.754202in}}%
\pgfpathclose%
\pgfusepath{fill}%
\end{pgfscope}%
\begin{pgfscope}%
\pgfpathrectangle{\pgfqpoint{1.150000in}{0.150000in}}{\pgfqpoint{5.700000in}{5.700000in}}%
\pgfusepath{clip}%
\pgfsetbuttcap%
\pgfsetroundjoin%
\definecolor{currentfill}{rgb}{0.268510,0.009605,0.335427}%
\pgfsetfillcolor{currentfill}%
\pgfsetfillopacity{0.700000}%
\pgfsetlinewidth{0.000000pt}%
\definecolor{currentstroke}{rgb}{0.000000,0.000000,0.000000}%
\pgfsetstrokecolor{currentstroke}%
\pgfsetdash{}{0pt}%
\pgfpathmoveto{\pgfqpoint{3.128911in}{1.735485in}}%
\pgfpathlineto{\pgfqpoint{3.142628in}{1.730156in}}%
\pgfpathlineto{\pgfqpoint{3.156349in}{1.724915in}}%
\pgfpathlineto{\pgfqpoint{3.170074in}{1.719761in}}%
\pgfpathlineto{\pgfqpoint{3.183804in}{1.714694in}}%
\pgfpathlineto{\pgfqpoint{3.192184in}{1.722405in}}%
\pgfpathlineto{\pgfqpoint{3.200556in}{1.730188in}}%
\pgfpathlineto{\pgfqpoint{3.208921in}{1.738039in}}%
\pgfpathlineto{\pgfqpoint{3.217278in}{1.745956in}}%
\pgfpathlineto{\pgfqpoint{3.203566in}{1.750798in}}%
\pgfpathlineto{\pgfqpoint{3.189858in}{1.755727in}}%
\pgfpathlineto{\pgfqpoint{3.176155in}{1.760743in}}%
\pgfpathlineto{\pgfqpoint{3.162456in}{1.765846in}}%
\pgfpathlineto{\pgfqpoint{3.154082in}{1.758147in}}%
\pgfpathlineto{\pgfqpoint{3.145699in}{1.750519in}}%
\pgfpathlineto{\pgfqpoint{3.137309in}{1.742964in}}%
\pgfpathlineto{\pgfqpoint{3.128911in}{1.735485in}}%
\pgfpathclose%
\pgfusepath{fill}%
\end{pgfscope}%
\begin{pgfscope}%
\pgfpathrectangle{\pgfqpoint{1.150000in}{0.150000in}}{\pgfqpoint{5.700000in}{5.700000in}}%
\pgfusepath{clip}%
\pgfsetbuttcap%
\pgfsetroundjoin%
\definecolor{currentfill}{rgb}{0.250425,0.274290,0.533103}%
\pgfsetfillcolor{currentfill}%
\pgfsetfillopacity{0.700000}%
\pgfsetlinewidth{0.000000pt}%
\definecolor{currentstroke}{rgb}{0.000000,0.000000,0.000000}%
\pgfsetstrokecolor{currentstroke}%
\pgfsetdash{}{0pt}%
\pgfpathmoveto{\pgfqpoint{2.021074in}{2.305939in}}%
\pgfpathlineto{\pgfqpoint{2.034923in}{2.291671in}}%
\pgfpathlineto{\pgfqpoint{2.048767in}{2.277545in}}%
\pgfpathlineto{\pgfqpoint{2.062607in}{2.263560in}}%
\pgfpathlineto{\pgfqpoint{2.076443in}{2.249714in}}%
\pgfpathlineto{\pgfqpoint{2.085549in}{2.249354in}}%
\pgfpathlineto{\pgfqpoint{2.094636in}{2.249242in}}%
\pgfpathlineto{\pgfqpoint{2.103705in}{2.249371in}}%
\pgfpathlineto{\pgfqpoint{2.112755in}{2.249738in}}%
\pgfpathlineto{\pgfqpoint{2.098959in}{2.263239in}}%
\pgfpathlineto{\pgfqpoint{2.085159in}{2.276879in}}%
\pgfpathlineto{\pgfqpoint{2.071355in}{2.290660in}}%
\pgfpathlineto{\pgfqpoint{2.057546in}{2.304582in}}%
\pgfpathlineto{\pgfqpoint{2.048457in}{2.304552in}}%
\pgfpathlineto{\pgfqpoint{2.039349in}{2.304765in}}%
\pgfpathlineto{\pgfqpoint{2.030221in}{2.305225in}}%
\pgfpathlineto{\pgfqpoint{2.021074in}{2.305939in}}%
\pgfpathclose%
\pgfusepath{fill}%
\end{pgfscope}%
\begin{pgfscope}%
\pgfpathrectangle{\pgfqpoint{1.150000in}{0.150000in}}{\pgfqpoint{5.700000in}{5.700000in}}%
\pgfusepath{clip}%
\pgfsetbuttcap%
\pgfsetroundjoin%
\definecolor{currentfill}{rgb}{0.225863,0.330805,0.547314}%
\pgfsetfillcolor{currentfill}%
\pgfsetfillopacity{0.700000}%
\pgfsetlinewidth{0.000000pt}%
\definecolor{currentstroke}{rgb}{0.000000,0.000000,0.000000}%
\pgfsetstrokecolor{currentstroke}%
\pgfsetdash{}{0pt}%
\pgfpathmoveto{\pgfqpoint{5.249905in}{2.392288in}}%
\pgfpathlineto{\pgfqpoint{5.264240in}{2.395203in}}%
\pgfpathlineto{\pgfqpoint{5.278586in}{2.398188in}}%
\pgfpathlineto{\pgfqpoint{5.292945in}{2.401243in}}%
\pgfpathlineto{\pgfqpoint{5.307315in}{2.404367in}}%
\pgfpathlineto{\pgfqpoint{5.314825in}{2.409583in}}%
\pgfpathlineto{\pgfqpoint{5.322327in}{2.414759in}}%
\pgfpathlineto{\pgfqpoint{5.329821in}{2.419898in}}%
\pgfpathlineto{\pgfqpoint{5.337309in}{2.425005in}}%
\pgfpathlineto{\pgfqpoint{5.322957in}{2.422096in}}%
\pgfpathlineto{\pgfqpoint{5.308618in}{2.419257in}}%
\pgfpathlineto{\pgfqpoint{5.294290in}{2.416487in}}%
\pgfpathlineto{\pgfqpoint{5.279973in}{2.413787in}}%
\pgfpathlineto{\pgfqpoint{5.272467in}{2.408457in}}%
\pgfpathlineto{\pgfqpoint{5.264953in}{2.403100in}}%
\pgfpathlineto{\pgfqpoint{5.257432in}{2.397712in}}%
\pgfpathlineto{\pgfqpoint{5.249905in}{2.392288in}}%
\pgfpathclose%
\pgfusepath{fill}%
\end{pgfscope}%
\begin{pgfscope}%
\pgfpathrectangle{\pgfqpoint{1.150000in}{0.150000in}}{\pgfqpoint{5.700000in}{5.700000in}}%
\pgfusepath{clip}%
\pgfsetbuttcap%
\pgfsetroundjoin%
\definecolor{currentfill}{rgb}{0.281446,0.084320,0.407414}%
\pgfsetfillcolor{currentfill}%
\pgfsetfillopacity{0.700000}%
\pgfsetlinewidth{0.000000pt}%
\definecolor{currentstroke}{rgb}{0.000000,0.000000,0.000000}%
\pgfsetstrokecolor{currentstroke}%
\pgfsetdash{}{0pt}%
\pgfpathmoveto{\pgfqpoint{3.910197in}{1.862389in}}%
\pgfpathlineto{\pgfqpoint{3.924060in}{1.861451in}}%
\pgfpathlineto{\pgfqpoint{3.937930in}{1.860589in}}%
\pgfpathlineto{\pgfqpoint{3.951809in}{1.859803in}}%
\pgfpathlineto{\pgfqpoint{3.965696in}{1.859093in}}%
\pgfpathlineto{\pgfqpoint{3.973770in}{1.868748in}}%
\pgfpathlineto{\pgfqpoint{3.981839in}{1.878365in}}%
\pgfpathlineto{\pgfqpoint{3.989903in}{1.887941in}}%
\pgfpathlineto{\pgfqpoint{3.997961in}{1.897478in}}%
\pgfpathlineto{\pgfqpoint{3.984085in}{1.898107in}}%
\pgfpathlineto{\pgfqpoint{3.970216in}{1.898811in}}%
\pgfpathlineto{\pgfqpoint{3.956355in}{1.899592in}}%
\pgfpathlineto{\pgfqpoint{3.942503in}{1.900449in}}%
\pgfpathlineto{\pgfqpoint{3.934435in}{1.890986in}}%
\pgfpathlineto{\pgfqpoint{3.926361in}{1.881488in}}%
\pgfpathlineto{\pgfqpoint{3.918282in}{1.871955in}}%
\pgfpathlineto{\pgfqpoint{3.910197in}{1.862389in}}%
\pgfpathclose%
\pgfusepath{fill}%
\end{pgfscope}%
\begin{pgfscope}%
\pgfpathrectangle{\pgfqpoint{1.150000in}{0.150000in}}{\pgfqpoint{5.700000in}{5.700000in}}%
\pgfusepath{clip}%
\pgfsetbuttcap%
\pgfsetroundjoin%
\definecolor{currentfill}{rgb}{0.270595,0.214069,0.507052}%
\pgfsetfillcolor{currentfill}%
\pgfsetfillopacity{0.700000}%
\pgfsetlinewidth{0.000000pt}%
\definecolor{currentstroke}{rgb}{0.000000,0.000000,0.000000}%
\pgfsetstrokecolor{currentstroke}%
\pgfsetdash{}{0pt}%
\pgfpathmoveto{\pgfqpoint{4.580053in}{2.127660in}}%
\pgfpathlineto{\pgfqpoint{4.594135in}{2.129219in}}%
\pgfpathlineto{\pgfqpoint{4.608226in}{2.130849in}}%
\pgfpathlineto{\pgfqpoint{4.622328in}{2.132552in}}%
\pgfpathlineto{\pgfqpoint{4.636439in}{2.134326in}}%
\pgfpathlineto{\pgfqpoint{4.644266in}{2.142388in}}%
\pgfpathlineto{\pgfqpoint{4.652085in}{2.150377in}}%
\pgfpathlineto{\pgfqpoint{4.659898in}{2.158294in}}%
\pgfpathlineto{\pgfqpoint{4.667704in}{2.166141in}}%
\pgfpathlineto{\pgfqpoint{4.653604in}{2.164432in}}%
\pgfpathlineto{\pgfqpoint{4.639514in}{2.162795in}}%
\pgfpathlineto{\pgfqpoint{4.625434in}{2.161229in}}%
\pgfpathlineto{\pgfqpoint{4.611364in}{2.159736in}}%
\pgfpathlineto{\pgfqpoint{4.603546in}{2.151816in}}%
\pgfpathlineto{\pgfqpoint{4.595722in}{2.143831in}}%
\pgfpathlineto{\pgfqpoint{4.587891in}{2.135779in}}%
\pgfpathlineto{\pgfqpoint{4.580053in}{2.127660in}}%
\pgfpathclose%
\pgfusepath{fill}%
\end{pgfscope}%
\begin{pgfscope}%
\pgfpathrectangle{\pgfqpoint{1.150000in}{0.150000in}}{\pgfqpoint{5.700000in}{5.700000in}}%
\pgfusepath{clip}%
\pgfsetbuttcap%
\pgfsetroundjoin%
\definecolor{currentfill}{rgb}{0.273809,0.031497,0.358853}%
\pgfsetfillcolor{currentfill}%
\pgfsetfillopacity{0.700000}%
\pgfsetlinewidth{0.000000pt}%
\definecolor{currentstroke}{rgb}{0.000000,0.000000,0.000000}%
\pgfsetstrokecolor{currentstroke}%
\pgfsetdash{}{0pt}%
\pgfpathmoveto{\pgfqpoint{2.841760in}{1.784561in}}%
\pgfpathlineto{\pgfqpoint{2.855469in}{1.777256in}}%
\pgfpathlineto{\pgfqpoint{2.869179in}{1.770047in}}%
\pgfpathlineto{\pgfqpoint{2.882892in}{1.762933in}}%
\pgfpathlineto{\pgfqpoint{2.896607in}{1.755913in}}%
\pgfpathlineto{\pgfqpoint{2.905140in}{1.761827in}}%
\pgfpathlineto{\pgfqpoint{2.913662in}{1.767863in}}%
\pgfpathlineto{\pgfqpoint{2.922175in}{1.774016in}}%
\pgfpathlineto{\pgfqpoint{2.930677in}{1.780282in}}%
\pgfpathlineto{\pgfqpoint{2.916985in}{1.787035in}}%
\pgfpathlineto{\pgfqpoint{2.903294in}{1.793881in}}%
\pgfpathlineto{\pgfqpoint{2.889607in}{1.800823in}}%
\pgfpathlineto{\pgfqpoint{2.875922in}{1.807859in}}%
\pgfpathlineto{\pgfqpoint{2.867397in}{1.801852in}}%
\pgfpathlineto{\pgfqpoint{2.858861in}{1.795965in}}%
\pgfpathlineto{\pgfqpoint{2.850316in}{1.790199in}}%
\pgfpathlineto{\pgfqpoint{2.841760in}{1.784561in}}%
\pgfpathclose%
\pgfusepath{fill}%
\end{pgfscope}%
\begin{pgfscope}%
\pgfpathrectangle{\pgfqpoint{1.150000in}{0.150000in}}{\pgfqpoint{5.700000in}{5.700000in}}%
\pgfusepath{clip}%
\pgfsetbuttcap%
\pgfsetroundjoin%
\definecolor{currentfill}{rgb}{0.271305,0.019942,0.347269}%
\pgfsetfillcolor{currentfill}%
\pgfsetfillopacity{0.700000}%
\pgfsetlinewidth{0.000000pt}%
\definecolor{currentstroke}{rgb}{0.000000,0.000000,0.000000}%
\pgfsetstrokecolor{currentstroke}%
\pgfsetdash{}{0pt}%
\pgfpathmoveto{\pgfqpoint{3.503439in}{1.751497in}}%
\pgfpathlineto{\pgfqpoint{3.517210in}{1.748462in}}%
\pgfpathlineto{\pgfqpoint{3.530986in}{1.745507in}}%
\pgfpathlineto{\pgfqpoint{3.544769in}{1.742632in}}%
\pgfpathlineto{\pgfqpoint{3.558558in}{1.739837in}}%
\pgfpathlineto{\pgfqpoint{3.566780in}{1.749067in}}%
\pgfpathlineto{\pgfqpoint{3.574996in}{1.758308in}}%
\pgfpathlineto{\pgfqpoint{3.583206in}{1.767561in}}%
\pgfpathlineto{\pgfqpoint{3.591410in}{1.776821in}}%
\pgfpathlineto{\pgfqpoint{3.577634in}{1.779452in}}%
\pgfpathlineto{\pgfqpoint{3.563864in}{1.782163in}}%
\pgfpathlineto{\pgfqpoint{3.550100in}{1.784954in}}%
\pgfpathlineto{\pgfqpoint{3.536343in}{1.787826in}}%
\pgfpathlineto{\pgfqpoint{3.528126in}{1.778722in}}%
\pgfpathlineto{\pgfqpoint{3.519903in}{1.769631in}}%
\pgfpathlineto{\pgfqpoint{3.511674in}{1.760555in}}%
\pgfpathlineto{\pgfqpoint{3.503439in}{1.751497in}}%
\pgfpathclose%
\pgfusepath{fill}%
\end{pgfscope}%
\begin{pgfscope}%
\pgfpathrectangle{\pgfqpoint{1.150000in}{0.150000in}}{\pgfqpoint{5.700000in}{5.700000in}}%
\pgfusepath{clip}%
\pgfsetbuttcap%
\pgfsetroundjoin%
\definecolor{currentfill}{rgb}{0.258965,0.251537,0.524736}%
\pgfsetfillcolor{currentfill}%
\pgfsetfillopacity{0.700000}%
\pgfsetlinewidth{0.000000pt}%
\definecolor{currentstroke}{rgb}{0.000000,0.000000,0.000000}%
\pgfsetstrokecolor{currentstroke}%
\pgfsetdash{}{0pt}%
\pgfpathmoveto{\pgfqpoint{2.076443in}{2.249714in}}%
\pgfpathlineto{\pgfqpoint{2.090274in}{2.236007in}}%
\pgfpathlineto{\pgfqpoint{2.104102in}{2.222437in}}%
\pgfpathlineto{\pgfqpoint{2.117926in}{2.209003in}}%
\pgfpathlineto{\pgfqpoint{2.131746in}{2.195703in}}%
\pgfpathlineto{\pgfqpoint{2.140812in}{2.195695in}}%
\pgfpathlineto{\pgfqpoint{2.149860in}{2.195929in}}%
\pgfpathlineto{\pgfqpoint{2.158890in}{2.196399in}}%
\pgfpathlineto{\pgfqpoint{2.167902in}{2.197100in}}%
\pgfpathlineto{\pgfqpoint{2.154121in}{2.210057in}}%
\pgfpathlineto{\pgfqpoint{2.140336in}{2.223148in}}%
\pgfpathlineto{\pgfqpoint{2.126547in}{2.236375in}}%
\pgfpathlineto{\pgfqpoint{2.112755in}{2.249738in}}%
\pgfpathlineto{\pgfqpoint{2.103705in}{2.249371in}}%
\pgfpathlineto{\pgfqpoint{2.094636in}{2.249242in}}%
\pgfpathlineto{\pgfqpoint{2.085549in}{2.249354in}}%
\pgfpathlineto{\pgfqpoint{2.076443in}{2.249714in}}%
\pgfpathclose%
\pgfusepath{fill}%
\end{pgfscope}%
\begin{pgfscope}%
\pgfpathrectangle{\pgfqpoint{1.150000in}{0.150000in}}{\pgfqpoint{5.700000in}{5.700000in}}%
\pgfusepath{clip}%
\pgfsetbuttcap%
\pgfsetroundjoin%
\definecolor{currentfill}{rgb}{0.283229,0.120777,0.440584}%
\pgfsetfillcolor{currentfill}%
\pgfsetfillopacity{0.700000}%
\pgfsetlinewidth{0.000000pt}%
\definecolor{currentstroke}{rgb}{0.000000,0.000000,0.000000}%
\pgfsetstrokecolor{currentstroke}%
\pgfsetdash{}{0pt}%
\pgfpathmoveto{\pgfqpoint{2.443038in}{1.964435in}}%
\pgfpathlineto{\pgfqpoint{2.456780in}{1.954048in}}%
\pgfpathlineto{\pgfqpoint{2.470521in}{1.943772in}}%
\pgfpathlineto{\pgfqpoint{2.484262in}{1.933609in}}%
\pgfpathlineto{\pgfqpoint{2.498003in}{1.923555in}}%
\pgfpathlineto{\pgfqpoint{2.506793in}{1.926430in}}%
\pgfpathlineto{\pgfqpoint{2.515570in}{1.929494in}}%
\pgfpathlineto{\pgfqpoint{2.524332in}{1.932742in}}%
\pgfpathlineto{\pgfqpoint{2.533081in}{1.936168in}}%
\pgfpathlineto{\pgfqpoint{2.519370in}{1.945908in}}%
\pgfpathlineto{\pgfqpoint{2.505660in}{1.955758in}}%
\pgfpathlineto{\pgfqpoint{2.491950in}{1.965720in}}%
\pgfpathlineto{\pgfqpoint{2.478239in}{1.975792in}}%
\pgfpathlineto{\pgfqpoint{2.469460in}{1.972672in}}%
\pgfpathlineto{\pgfqpoint{2.460667in}{1.969736in}}%
\pgfpathlineto{\pgfqpoint{2.451860in}{1.966988in}}%
\pgfpathlineto{\pgfqpoint{2.443038in}{1.964435in}}%
\pgfpathclose%
\pgfusepath{fill}%
\end{pgfscope}%
\begin{pgfscope}%
\pgfpathrectangle{\pgfqpoint{1.150000in}{0.150000in}}{\pgfqpoint{5.700000in}{5.700000in}}%
\pgfusepath{clip}%
\pgfsetbuttcap%
\pgfsetroundjoin%
\definecolor{currentfill}{rgb}{0.268510,0.009605,0.335427}%
\pgfsetfillcolor{currentfill}%
\pgfsetfillopacity{0.700000}%
\pgfsetlinewidth{0.000000pt}%
\definecolor{currentstroke}{rgb}{0.000000,0.000000,0.000000}%
\pgfsetstrokecolor{currentstroke}%
\pgfsetdash{}{0pt}%
\pgfpathmoveto{\pgfqpoint{3.272171in}{1.727445in}}%
\pgfpathlineto{\pgfqpoint{3.285906in}{1.723030in}}%
\pgfpathlineto{\pgfqpoint{3.299646in}{1.718700in}}%
\pgfpathlineto{\pgfqpoint{3.313391in}{1.714453in}}%
\pgfpathlineto{\pgfqpoint{3.327141in}{1.710290in}}%
\pgfpathlineto{\pgfqpoint{3.335457in}{1.718693in}}%
\pgfpathlineto{\pgfqpoint{3.343767in}{1.727144in}}%
\pgfpathlineto{\pgfqpoint{3.352070in}{1.735641in}}%
\pgfpathlineto{\pgfqpoint{3.360366in}{1.744182in}}%
\pgfpathlineto{\pgfqpoint{3.346632in}{1.748140in}}%
\pgfpathlineto{\pgfqpoint{3.332903in}{1.752182in}}%
\pgfpathlineto{\pgfqpoint{3.319178in}{1.756308in}}%
\pgfpathlineto{\pgfqpoint{3.305459in}{1.760518in}}%
\pgfpathlineto{\pgfqpoint{3.297148in}{1.752175in}}%
\pgfpathlineto{\pgfqpoint{3.288829in}{1.743880in}}%
\pgfpathlineto{\pgfqpoint{3.280504in}{1.735635in}}%
\pgfpathlineto{\pgfqpoint{3.272171in}{1.727445in}}%
\pgfpathclose%
\pgfusepath{fill}%
\end{pgfscope}%
\begin{pgfscope}%
\pgfpathrectangle{\pgfqpoint{1.150000in}{0.150000in}}{\pgfqpoint{5.700000in}{5.700000in}}%
\pgfusepath{clip}%
\pgfsetbuttcap%
\pgfsetroundjoin%
\definecolor{currentfill}{rgb}{0.201239,0.383670,0.554294}%
\pgfsetfillcolor{currentfill}%
\pgfsetfillopacity{0.700000}%
\pgfsetlinewidth{0.000000pt}%
\definecolor{currentstroke}{rgb}{0.000000,0.000000,0.000000}%
\pgfsetstrokecolor{currentstroke}%
\pgfsetdash{}{0pt}%
\pgfpathmoveto{\pgfqpoint{5.656900in}{2.528873in}}%
\pgfpathlineto{\pgfqpoint{5.671395in}{2.532178in}}%
\pgfpathlineto{\pgfqpoint{5.685903in}{2.535552in}}%
\pgfpathlineto{\pgfqpoint{5.700424in}{2.538994in}}%
\pgfpathlineto{\pgfqpoint{5.714957in}{2.542505in}}%
\pgfpathlineto{\pgfqpoint{5.722246in}{2.546131in}}%
\pgfpathlineto{\pgfqpoint{5.729528in}{2.549763in}}%
\pgfpathlineto{\pgfqpoint{5.736804in}{2.553408in}}%
\pgfpathlineto{\pgfqpoint{5.744073in}{2.557070in}}%
\pgfpathlineto{\pgfqpoint{5.729565in}{2.553861in}}%
\pgfpathlineto{\pgfqpoint{5.715070in}{2.550720in}}%
\pgfpathlineto{\pgfqpoint{5.700587in}{2.547647in}}%
\pgfpathlineto{\pgfqpoint{5.686116in}{2.544643in}}%
\pgfpathlineto{\pgfqpoint{5.678821in}{2.540672in}}%
\pgfpathlineto{\pgfqpoint{5.671520in}{2.536724in}}%
\pgfpathlineto{\pgfqpoint{5.664213in}{2.532793in}}%
\pgfpathlineto{\pgfqpoint{5.656900in}{2.528873in}}%
\pgfpathclose%
\pgfusepath{fill}%
\end{pgfscope}%
\begin{pgfscope}%
\pgfpathrectangle{\pgfqpoint{1.150000in}{0.150000in}}{\pgfqpoint{5.700000in}{5.700000in}}%
\pgfusepath{clip}%
\pgfsetbuttcap%
\pgfsetroundjoin%
\definecolor{currentfill}{rgb}{0.274128,0.199721,0.498911}%
\pgfsetfillcolor{currentfill}%
\pgfsetfillopacity{0.700000}%
\pgfsetlinewidth{0.000000pt}%
\definecolor{currentstroke}{rgb}{0.000000,0.000000,0.000000}%
\pgfsetstrokecolor{currentstroke}%
\pgfsetdash{}{0pt}%
\pgfpathmoveto{\pgfqpoint{4.492371in}{2.088789in}}%
\pgfpathlineto{\pgfqpoint{4.506423in}{2.090102in}}%
\pgfpathlineto{\pgfqpoint{4.520485in}{2.091489in}}%
\pgfpathlineto{\pgfqpoint{4.534557in}{2.092947in}}%
\pgfpathlineto{\pgfqpoint{4.548639in}{2.094478in}}%
\pgfpathlineto{\pgfqpoint{4.556502in}{2.102882in}}%
\pgfpathlineto{\pgfqpoint{4.564359in}{2.111213in}}%
\pgfpathlineto{\pgfqpoint{4.572210in}{2.119472in}}%
\pgfpathlineto{\pgfqpoint{4.580053in}{2.127660in}}%
\pgfpathlineto{\pgfqpoint{4.565982in}{2.126174in}}%
\pgfpathlineto{\pgfqpoint{4.551922in}{2.124759in}}%
\pgfpathlineto{\pgfqpoint{4.537871in}{2.123417in}}%
\pgfpathlineto{\pgfqpoint{4.523830in}{2.122147in}}%
\pgfpathlineto{\pgfqpoint{4.515974in}{2.113907in}}%
\pgfpathlineto{\pgfqpoint{4.508113in}{2.105601in}}%
\pgfpathlineto{\pgfqpoint{4.500245in}{2.097229in}}%
\pgfpathlineto{\pgfqpoint{4.492371in}{2.088789in}}%
\pgfpathclose%
\pgfusepath{fill}%
\end{pgfscope}%
\begin{pgfscope}%
\pgfpathrectangle{\pgfqpoint{1.150000in}{0.150000in}}{\pgfqpoint{5.700000in}{5.700000in}}%
\pgfusepath{clip}%
\pgfsetbuttcap%
\pgfsetroundjoin%
\definecolor{currentfill}{rgb}{0.279566,0.067836,0.391917}%
\pgfsetfillcolor{currentfill}%
\pgfsetfillopacity{0.700000}%
\pgfsetlinewidth{0.000000pt}%
\definecolor{currentstroke}{rgb}{0.000000,0.000000,0.000000}%
\pgfsetstrokecolor{currentstroke}%
\pgfsetdash{}{0pt}%
\pgfpathmoveto{\pgfqpoint{3.822385in}{1.828735in}}%
\pgfpathlineto{\pgfqpoint{3.836228in}{1.827389in}}%
\pgfpathlineto{\pgfqpoint{3.850078in}{1.826119in}}%
\pgfpathlineto{\pgfqpoint{3.863936in}{1.824926in}}%
\pgfpathlineto{\pgfqpoint{3.877801in}{1.823810in}}%
\pgfpathlineto{\pgfqpoint{3.885909in}{1.833500in}}%
\pgfpathlineto{\pgfqpoint{3.894010in}{1.843160in}}%
\pgfpathlineto{\pgfqpoint{3.902106in}{1.852791in}}%
\pgfpathlineto{\pgfqpoint{3.910197in}{1.862389in}}%
\pgfpathlineto{\pgfqpoint{3.896342in}{1.863404in}}%
\pgfpathlineto{\pgfqpoint{3.882494in}{1.864495in}}%
\pgfpathlineto{\pgfqpoint{3.868654in}{1.865662in}}%
\pgfpathlineto{\pgfqpoint{3.854822in}{1.866907in}}%
\pgfpathlineto{\pgfqpoint{3.846721in}{1.857402in}}%
\pgfpathlineto{\pgfqpoint{3.838615in}{1.847871in}}%
\pgfpathlineto{\pgfqpoint{3.830503in}{1.838315in}}%
\pgfpathlineto{\pgfqpoint{3.822385in}{1.828735in}}%
\pgfpathclose%
\pgfusepath{fill}%
\end{pgfscope}%
\begin{pgfscope}%
\pgfpathrectangle{\pgfqpoint{1.150000in}{0.150000in}}{\pgfqpoint{5.700000in}{5.700000in}}%
\pgfusepath{clip}%
\pgfsetbuttcap%
\pgfsetroundjoin%
\definecolor{currentfill}{rgb}{0.231674,0.318106,0.544834}%
\pgfsetfillcolor{currentfill}%
\pgfsetfillopacity{0.700000}%
\pgfsetlinewidth{0.000000pt}%
\definecolor{currentstroke}{rgb}{0.000000,0.000000,0.000000}%
\pgfsetstrokecolor{currentstroke}%
\pgfsetdash{}{0pt}%
\pgfpathmoveto{\pgfqpoint{5.162430in}{2.358426in}}%
\pgfpathlineto{\pgfqpoint{5.176735in}{2.361256in}}%
\pgfpathlineto{\pgfqpoint{5.191052in}{2.364156in}}%
\pgfpathlineto{\pgfqpoint{5.205381in}{2.367126in}}%
\pgfpathlineto{\pgfqpoint{5.219721in}{2.370166in}}%
\pgfpathlineto{\pgfqpoint{5.227278in}{2.375768in}}%
\pgfpathlineto{\pgfqpoint{5.234828in}{2.381320in}}%
\pgfpathlineto{\pgfqpoint{5.242370in}{2.386825in}}%
\pgfpathlineto{\pgfqpoint{5.249905in}{2.392288in}}%
\pgfpathlineto{\pgfqpoint{5.235582in}{2.389443in}}%
\pgfpathlineto{\pgfqpoint{5.221270in}{2.386667in}}%
\pgfpathlineto{\pgfqpoint{5.206971in}{2.383960in}}%
\pgfpathlineto{\pgfqpoint{5.192683in}{2.381323in}}%
\pgfpathlineto{\pgfqpoint{5.185130in}{2.375659in}}%
\pgfpathlineto{\pgfqpoint{5.177570in}{2.369957in}}%
\pgfpathlineto{\pgfqpoint{5.170004in}{2.364214in}}%
\pgfpathlineto{\pgfqpoint{5.162430in}{2.358426in}}%
\pgfpathclose%
\pgfusepath{fill}%
\end{pgfscope}%
\begin{pgfscope}%
\pgfpathrectangle{\pgfqpoint{1.150000in}{0.150000in}}{\pgfqpoint{5.700000in}{5.700000in}}%
\pgfusepath{clip}%
\pgfsetbuttcap%
\pgfsetroundjoin%
\definecolor{currentfill}{rgb}{0.278012,0.180367,0.486697}%
\pgfsetfillcolor{currentfill}%
\pgfsetfillopacity{0.700000}%
\pgfsetlinewidth{0.000000pt}%
\definecolor{currentstroke}{rgb}{0.000000,0.000000,0.000000}%
\pgfsetstrokecolor{currentstroke}%
\pgfsetdash{}{0pt}%
\pgfpathmoveto{\pgfqpoint{4.404660in}{2.049709in}}%
\pgfpathlineto{\pgfqpoint{4.418684in}{2.050755in}}%
\pgfpathlineto{\pgfqpoint{4.432717in}{2.051874in}}%
\pgfpathlineto{\pgfqpoint{4.446759in}{2.053066in}}%
\pgfpathlineto{\pgfqpoint{4.460812in}{2.054330in}}%
\pgfpathlineto{\pgfqpoint{4.468711in}{2.063051in}}%
\pgfpathlineto{\pgfqpoint{4.476604in}{2.071701in}}%
\pgfpathlineto{\pgfqpoint{4.484491in}{2.080280in}}%
\pgfpathlineto{\pgfqpoint{4.492371in}{2.088789in}}%
\pgfpathlineto{\pgfqpoint{4.478329in}{2.087547in}}%
\pgfpathlineto{\pgfqpoint{4.464297in}{2.086378in}}%
\pgfpathlineto{\pgfqpoint{4.450274in}{2.085282in}}%
\pgfpathlineto{\pgfqpoint{4.436261in}{2.084259in}}%
\pgfpathlineto{\pgfqpoint{4.428370in}{2.075719in}}%
\pgfpathlineto{\pgfqpoint{4.420473in}{2.067115in}}%
\pgfpathlineto{\pgfqpoint{4.412570in}{2.058445in}}%
\pgfpathlineto{\pgfqpoint{4.404660in}{2.049709in}}%
\pgfpathclose%
\pgfusepath{fill}%
\end{pgfscope}%
\begin{pgfscope}%
\pgfpathrectangle{\pgfqpoint{1.150000in}{0.150000in}}{\pgfqpoint{5.700000in}{5.700000in}}%
\pgfusepath{clip}%
\pgfsetbuttcap%
\pgfsetroundjoin%
\definecolor{currentfill}{rgb}{0.266580,0.228262,0.514349}%
\pgfsetfillcolor{currentfill}%
\pgfsetfillopacity{0.700000}%
\pgfsetlinewidth{0.000000pt}%
\definecolor{currentstroke}{rgb}{0.000000,0.000000,0.000000}%
\pgfsetstrokecolor{currentstroke}%
\pgfsetdash{}{0pt}%
\pgfpathmoveto{\pgfqpoint{2.131746in}{2.195703in}}%
\pgfpathlineto{\pgfqpoint{2.145562in}{2.182537in}}%
\pgfpathlineto{\pgfqpoint{2.159376in}{2.169503in}}%
\pgfpathlineto{\pgfqpoint{2.173186in}{2.156600in}}%
\pgfpathlineto{\pgfqpoint{2.186993in}{2.143826in}}%
\pgfpathlineto{\pgfqpoint{2.196021in}{2.144169in}}%
\pgfpathlineto{\pgfqpoint{2.205031in}{2.144747in}}%
\pgfpathlineto{\pgfqpoint{2.214023in}{2.145556in}}%
\pgfpathlineto{\pgfqpoint{2.222998in}{2.146590in}}%
\pgfpathlineto{\pgfqpoint{2.209228in}{2.159022in}}%
\pgfpathlineto{\pgfqpoint{2.195456in}{2.171584in}}%
\pgfpathlineto{\pgfqpoint{2.181681in}{2.184276in}}%
\pgfpathlineto{\pgfqpoint{2.167902in}{2.197100in}}%
\pgfpathlineto{\pgfqpoint{2.158890in}{2.196399in}}%
\pgfpathlineto{\pgfqpoint{2.149860in}{2.195929in}}%
\pgfpathlineto{\pgfqpoint{2.140812in}{2.195695in}}%
\pgfpathlineto{\pgfqpoint{2.131746in}{2.195703in}}%
\pgfpathclose%
\pgfusepath{fill}%
\end{pgfscope}%
\begin{pgfscope}%
\pgfpathrectangle{\pgfqpoint{1.150000in}{0.150000in}}{\pgfqpoint{5.700000in}{5.700000in}}%
\pgfusepath{clip}%
\pgfsetbuttcap%
\pgfsetroundjoin%
\definecolor{currentfill}{rgb}{0.237441,0.305202,0.541921}%
\pgfsetfillcolor{currentfill}%
\pgfsetfillopacity{0.700000}%
\pgfsetlinewidth{0.000000pt}%
\definecolor{currentstroke}{rgb}{0.000000,0.000000,0.000000}%
\pgfsetstrokecolor{currentstroke}%
\pgfsetdash{}{0pt}%
\pgfpathmoveto{\pgfqpoint{5.074890in}{2.323446in}}%
\pgfpathlineto{\pgfqpoint{5.089165in}{2.326168in}}%
\pgfpathlineto{\pgfqpoint{5.103452in}{2.328961in}}%
\pgfpathlineto{\pgfqpoint{5.117751in}{2.331824in}}%
\pgfpathlineto{\pgfqpoint{5.132061in}{2.334757in}}%
\pgfpathlineto{\pgfqpoint{5.139664in}{2.340758in}}%
\pgfpathlineto{\pgfqpoint{5.147260in}{2.346701in}}%
\pgfpathlineto{\pgfqpoint{5.154848in}{2.352589in}}%
\pgfpathlineto{\pgfqpoint{5.162430in}{2.358426in}}%
\pgfpathlineto{\pgfqpoint{5.148136in}{2.355665in}}%
\pgfpathlineto{\pgfqpoint{5.133854in}{2.352975in}}%
\pgfpathlineto{\pgfqpoint{5.119583in}{2.350355in}}%
\pgfpathlineto{\pgfqpoint{5.105324in}{2.347804in}}%
\pgfpathlineto{\pgfqpoint{5.097726in}{2.341788in}}%
\pgfpathlineto{\pgfqpoint{5.090121in}{2.335725in}}%
\pgfpathlineto{\pgfqpoint{5.082509in}{2.329612in}}%
\pgfpathlineto{\pgfqpoint{5.074890in}{2.323446in}}%
\pgfpathclose%
\pgfusepath{fill}%
\end{pgfscope}%
\begin{pgfscope}%
\pgfpathrectangle{\pgfqpoint{1.150000in}{0.150000in}}{\pgfqpoint{5.700000in}{5.700000in}}%
\pgfusepath{clip}%
\pgfsetbuttcap%
\pgfsetroundjoin%
\definecolor{currentfill}{rgb}{0.278791,0.062145,0.386592}%
\pgfsetfillcolor{currentfill}%
\pgfsetfillopacity{0.700000}%
\pgfsetlinewidth{0.000000pt}%
\definecolor{currentstroke}{rgb}{0.000000,0.000000,0.000000}%
\pgfsetstrokecolor{currentstroke}%
\pgfsetdash{}{0pt}%
\pgfpathmoveto{\pgfqpoint{2.697635in}{1.827598in}}%
\pgfpathlineto{\pgfqpoint{2.711354in}{1.819221in}}%
\pgfpathlineto{\pgfqpoint{2.725074in}{1.810945in}}%
\pgfpathlineto{\pgfqpoint{2.738796in}{1.802768in}}%
\pgfpathlineto{\pgfqpoint{2.752519in}{1.794690in}}%
\pgfpathlineto{\pgfqpoint{2.761144in}{1.799484in}}%
\pgfpathlineto{\pgfqpoint{2.769757in}{1.804427in}}%
\pgfpathlineto{\pgfqpoint{2.778359in}{1.809514in}}%
\pgfpathlineto{\pgfqpoint{2.786949in}{1.814741in}}%
\pgfpathlineto{\pgfqpoint{2.773252in}{1.822530in}}%
\pgfpathlineto{\pgfqpoint{2.759556in}{1.830417in}}%
\pgfpathlineto{\pgfqpoint{2.745861in}{1.838404in}}%
\pgfpathlineto{\pgfqpoint{2.732169in}{1.846491in}}%
\pgfpathlineto{\pgfqpoint{2.723553in}{1.841545in}}%
\pgfpathlineto{\pgfqpoint{2.714925in}{1.836745in}}%
\pgfpathlineto{\pgfqpoint{2.706286in}{1.832095in}}%
\pgfpathlineto{\pgfqpoint{2.697635in}{1.827598in}}%
\pgfpathclose%
\pgfusepath{fill}%
\end{pgfscope}%
\begin{pgfscope}%
\pgfpathrectangle{\pgfqpoint{1.150000in}{0.150000in}}{\pgfqpoint{5.700000in}{5.700000in}}%
\pgfusepath{clip}%
\pgfsetbuttcap%
\pgfsetroundjoin%
\definecolor{currentfill}{rgb}{0.280255,0.165693,0.476498}%
\pgfsetfillcolor{currentfill}%
\pgfsetfillopacity{0.700000}%
\pgfsetlinewidth{0.000000pt}%
\definecolor{currentstroke}{rgb}{0.000000,0.000000,0.000000}%
\pgfsetstrokecolor{currentstroke}%
\pgfsetdash{}{0pt}%
\pgfpathmoveto{\pgfqpoint{4.316923in}{2.010625in}}%
\pgfpathlineto{\pgfqpoint{4.330918in}{2.011381in}}%
\pgfpathlineto{\pgfqpoint{4.344923in}{2.012210in}}%
\pgfpathlineto{\pgfqpoint{4.358937in}{2.013112in}}%
\pgfpathlineto{\pgfqpoint{4.372961in}{2.014088in}}%
\pgfpathlineto{\pgfqpoint{4.380895in}{2.023096in}}%
\pgfpathlineto{\pgfqpoint{4.388823in}{2.032035in}}%
\pgfpathlineto{\pgfqpoint{4.396745in}{2.040905in}}%
\pgfpathlineto{\pgfqpoint{4.404660in}{2.049709in}}%
\pgfpathlineto{\pgfqpoint{4.390647in}{2.048735in}}%
\pgfpathlineto{\pgfqpoint{4.376642in}{2.047835in}}%
\pgfpathlineto{\pgfqpoint{4.362648in}{2.047007in}}%
\pgfpathlineto{\pgfqpoint{4.348662in}{2.046253in}}%
\pgfpathlineto{\pgfqpoint{4.340737in}{2.037440in}}%
\pgfpathlineto{\pgfqpoint{4.332805in}{2.028565in}}%
\pgfpathlineto{\pgfqpoint{4.324867in}{2.019626in}}%
\pgfpathlineto{\pgfqpoint{4.316923in}{2.010625in}}%
\pgfpathclose%
\pgfusepath{fill}%
\end{pgfscope}%
\begin{pgfscope}%
\pgfpathrectangle{\pgfqpoint{1.150000in}{0.150000in}}{\pgfqpoint{5.700000in}{5.700000in}}%
\pgfusepath{clip}%
\pgfsetbuttcap%
\pgfsetroundjoin%
\definecolor{currentfill}{rgb}{0.277941,0.056324,0.381191}%
\pgfsetfillcolor{currentfill}%
\pgfsetfillopacity{0.700000}%
\pgfsetlinewidth{0.000000pt}%
\definecolor{currentstroke}{rgb}{0.000000,0.000000,0.000000}%
\pgfsetstrokecolor{currentstroke}%
\pgfsetdash{}{0pt}%
\pgfpathmoveto{\pgfqpoint{3.734516in}{1.796851in}}%
\pgfpathlineto{\pgfqpoint{3.748341in}{1.795072in}}%
\pgfpathlineto{\pgfqpoint{3.762173in}{1.793371in}}%
\pgfpathlineto{\pgfqpoint{3.776012in}{1.791748in}}%
\pgfpathlineto{\pgfqpoint{3.789858in}{1.790202in}}%
\pgfpathlineto{\pgfqpoint{3.797998in}{1.799864in}}%
\pgfpathlineto{\pgfqpoint{3.806133in}{1.809508in}}%
\pgfpathlineto{\pgfqpoint{3.814262in}{1.819133in}}%
\pgfpathlineto{\pgfqpoint{3.822385in}{1.828735in}}%
\pgfpathlineto{\pgfqpoint{3.808550in}{1.830159in}}%
\pgfpathlineto{\pgfqpoint{3.794722in}{1.831660in}}%
\pgfpathlineto{\pgfqpoint{3.780901in}{1.833238in}}%
\pgfpathlineto{\pgfqpoint{3.767087in}{1.834894in}}%
\pgfpathlineto{\pgfqpoint{3.758953in}{1.825406in}}%
\pgfpathlineto{\pgfqpoint{3.750813in}{1.815902in}}%
\pgfpathlineto{\pgfqpoint{3.742668in}{1.806383in}}%
\pgfpathlineto{\pgfqpoint{3.734516in}{1.796851in}}%
\pgfpathclose%
\pgfusepath{fill}%
\end{pgfscope}%
\begin{pgfscope}%
\pgfpathrectangle{\pgfqpoint{1.150000in}{0.150000in}}{\pgfqpoint{5.700000in}{5.700000in}}%
\pgfusepath{clip}%
\pgfsetbuttcap%
\pgfsetroundjoin%
\definecolor{currentfill}{rgb}{0.204903,0.375746,0.553533}%
\pgfsetfillcolor{currentfill}%
\pgfsetfillopacity{0.700000}%
\pgfsetlinewidth{0.000000pt}%
\definecolor{currentstroke}{rgb}{0.000000,0.000000,0.000000}%
\pgfsetstrokecolor{currentstroke}%
\pgfsetdash{}{0pt}%
\pgfpathmoveto{\pgfqpoint{5.569634in}{2.499556in}}%
\pgfpathlineto{\pgfqpoint{5.584102in}{2.502867in}}%
\pgfpathlineto{\pgfqpoint{5.598583in}{2.506246in}}%
\pgfpathlineto{\pgfqpoint{5.613077in}{2.509695in}}%
\pgfpathlineto{\pgfqpoint{5.627583in}{2.513212in}}%
\pgfpathlineto{\pgfqpoint{5.634922in}{2.517135in}}%
\pgfpathlineto{\pgfqpoint{5.642255in}{2.521049in}}%
\pgfpathlineto{\pgfqpoint{5.649581in}{2.524960in}}%
\pgfpathlineto{\pgfqpoint{5.656900in}{2.528873in}}%
\pgfpathlineto{\pgfqpoint{5.642418in}{2.525637in}}%
\pgfpathlineto{\pgfqpoint{5.627947in}{2.522469in}}%
\pgfpathlineto{\pgfqpoint{5.613490in}{2.519369in}}%
\pgfpathlineto{\pgfqpoint{5.599045in}{2.516338in}}%
\pgfpathlineto{\pgfqpoint{5.591701in}{2.512137in}}%
\pgfpathlineto{\pgfqpoint{5.584352in}{2.507944in}}%
\pgfpathlineto{\pgfqpoint{5.576996in}{2.503752in}}%
\pgfpathlineto{\pgfqpoint{5.569634in}{2.499556in}}%
\pgfpathclose%
\pgfusepath{fill}%
\end{pgfscope}%
\begin{pgfscope}%
\pgfpathrectangle{\pgfqpoint{1.150000in}{0.150000in}}{\pgfqpoint{5.700000in}{5.700000in}}%
\pgfusepath{clip}%
\pgfsetbuttcap%
\pgfsetroundjoin%
\definecolor{currentfill}{rgb}{0.269944,0.014625,0.341379}%
\pgfsetfillcolor{currentfill}%
\pgfsetfillopacity{0.700000}%
\pgfsetlinewidth{0.000000pt}%
\definecolor{currentstroke}{rgb}{0.000000,0.000000,0.000000}%
\pgfsetstrokecolor{currentstroke}%
\pgfsetdash{}{0pt}%
\pgfpathmoveto{\pgfqpoint{3.415357in}{1.729178in}}%
\pgfpathlineto{\pgfqpoint{3.429118in}{1.725634in}}%
\pgfpathlineto{\pgfqpoint{3.442885in}{1.722170in}}%
\pgfpathlineto{\pgfqpoint{3.456658in}{1.718789in}}%
\pgfpathlineto{\pgfqpoint{3.470436in}{1.715488in}}%
\pgfpathlineto{\pgfqpoint{3.478696in}{1.724452in}}%
\pgfpathlineto{\pgfqpoint{3.486950in}{1.733443in}}%
\pgfpathlineto{\pgfqpoint{3.495198in}{1.742459in}}%
\pgfpathlineto{\pgfqpoint{3.503439in}{1.751497in}}%
\pgfpathlineto{\pgfqpoint{3.489675in}{1.754614in}}%
\pgfpathlineto{\pgfqpoint{3.475916in}{1.757811in}}%
\pgfpathlineto{\pgfqpoint{3.462163in}{1.761090in}}%
\pgfpathlineto{\pgfqpoint{3.448416in}{1.764451in}}%
\pgfpathlineto{\pgfqpoint{3.440161in}{1.755589in}}%
\pgfpathlineto{\pgfqpoint{3.431899in}{1.746755in}}%
\pgfpathlineto{\pgfqpoint{3.423631in}{1.737950in}}%
\pgfpathlineto{\pgfqpoint{3.415357in}{1.729178in}}%
\pgfpathclose%
\pgfusepath{fill}%
\end{pgfscope}%
\begin{pgfscope}%
\pgfpathrectangle{\pgfqpoint{1.150000in}{0.150000in}}{\pgfqpoint{5.700000in}{5.700000in}}%
\pgfusepath{clip}%
\pgfsetbuttcap%
\pgfsetroundjoin%
\definecolor{currentfill}{rgb}{0.271828,0.209303,0.504434}%
\pgfsetfillcolor{currentfill}%
\pgfsetfillopacity{0.700000}%
\pgfsetlinewidth{0.000000pt}%
\definecolor{currentstroke}{rgb}{0.000000,0.000000,0.000000}%
\pgfsetstrokecolor{currentstroke}%
\pgfsetdash{}{0pt}%
\pgfpathmoveto{\pgfqpoint{2.186993in}{2.143826in}}%
\pgfpathlineto{\pgfqpoint{2.200797in}{2.131182in}}%
\pgfpathlineto{\pgfqpoint{2.214598in}{2.118665in}}%
\pgfpathlineto{\pgfqpoint{2.228397in}{2.106275in}}%
\pgfpathlineto{\pgfqpoint{2.242193in}{2.094010in}}%
\pgfpathlineto{\pgfqpoint{2.251183in}{2.094701in}}%
\pgfpathlineto{\pgfqpoint{2.260156in}{2.095622in}}%
\pgfpathlineto{\pgfqpoint{2.269112in}{2.096769in}}%
\pgfpathlineto{\pgfqpoint{2.278051in}{2.098135in}}%
\pgfpathlineto{\pgfqpoint{2.264291in}{2.110060in}}%
\pgfpathlineto{\pgfqpoint{2.250529in}{2.122110in}}%
\pgfpathlineto{\pgfqpoint{2.236765in}{2.134287in}}%
\pgfpathlineto{\pgfqpoint{2.222998in}{2.146590in}}%
\pgfpathlineto{\pgfqpoint{2.214023in}{2.145556in}}%
\pgfpathlineto{\pgfqpoint{2.205031in}{2.144747in}}%
\pgfpathlineto{\pgfqpoint{2.196021in}{2.144169in}}%
\pgfpathlineto{\pgfqpoint{2.186993in}{2.143826in}}%
\pgfpathclose%
\pgfusepath{fill}%
\end{pgfscope}%
\begin{pgfscope}%
\pgfpathrectangle{\pgfqpoint{1.150000in}{0.150000in}}{\pgfqpoint{5.700000in}{5.700000in}}%
\pgfusepath{clip}%
\pgfsetbuttcap%
\pgfsetroundjoin%
\definecolor{currentfill}{rgb}{0.283091,0.110553,0.431554}%
\pgfsetfillcolor{currentfill}%
\pgfsetfillopacity{0.700000}%
\pgfsetlinewidth{0.000000pt}%
\definecolor{currentstroke}{rgb}{0.000000,0.000000,0.000000}%
\pgfsetstrokecolor{currentstroke}%
\pgfsetdash{}{0pt}%
\pgfpathmoveto{\pgfqpoint{2.498003in}{1.923555in}}%
\pgfpathlineto{\pgfqpoint{2.511744in}{1.913611in}}%
\pgfpathlineto{\pgfqpoint{2.525484in}{1.903776in}}%
\pgfpathlineto{\pgfqpoint{2.539224in}{1.894048in}}%
\pgfpathlineto{\pgfqpoint{2.552965in}{1.884428in}}%
\pgfpathlineto{\pgfqpoint{2.561725in}{1.887625in}}%
\pgfpathlineto{\pgfqpoint{2.570470in}{1.891004in}}%
\pgfpathlineto{\pgfqpoint{2.579203in}{1.894562in}}%
\pgfpathlineto{\pgfqpoint{2.587922in}{1.898293in}}%
\pgfpathlineto{\pgfqpoint{2.574211in}{1.907600in}}%
\pgfpathlineto{\pgfqpoint{2.560501in}{1.917015in}}%
\pgfpathlineto{\pgfqpoint{2.546791in}{1.926537in}}%
\pgfpathlineto{\pgfqpoint{2.533081in}{1.936168in}}%
\pgfpathlineto{\pgfqpoint{2.524332in}{1.932742in}}%
\pgfpathlineto{\pgfqpoint{2.515570in}{1.929494in}}%
\pgfpathlineto{\pgfqpoint{2.506793in}{1.926430in}}%
\pgfpathlineto{\pgfqpoint{2.498003in}{1.923555in}}%
\pgfpathclose%
\pgfusepath{fill}%
\end{pgfscope}%
\begin{pgfscope}%
\pgfpathrectangle{\pgfqpoint{1.150000in}{0.150000in}}{\pgfqpoint{5.700000in}{5.700000in}}%
\pgfusepath{clip}%
\pgfsetbuttcap%
\pgfsetroundjoin%
\definecolor{currentfill}{rgb}{0.282290,0.145912,0.461510}%
\pgfsetfillcolor{currentfill}%
\pgfsetfillopacity{0.700000}%
\pgfsetlinewidth{0.000000pt}%
\definecolor{currentstroke}{rgb}{0.000000,0.000000,0.000000}%
\pgfsetstrokecolor{currentstroke}%
\pgfsetdash{}{0pt}%
\pgfpathmoveto{\pgfqpoint{4.229159in}{1.971763in}}%
\pgfpathlineto{\pgfqpoint{4.243128in}{1.972206in}}%
\pgfpathlineto{\pgfqpoint{4.257105in}{1.972722in}}%
\pgfpathlineto{\pgfqpoint{4.271092in}{1.973312in}}%
\pgfpathlineto{\pgfqpoint{4.285088in}{1.973976in}}%
\pgfpathlineto{\pgfqpoint{4.293055in}{1.983235in}}%
\pgfpathlineto{\pgfqpoint{4.301017in}{1.992429in}}%
\pgfpathlineto{\pgfqpoint{4.308973in}{2.001559in}}%
\pgfpathlineto{\pgfqpoint{4.316923in}{2.010625in}}%
\pgfpathlineto{\pgfqpoint{4.302937in}{2.009942in}}%
\pgfpathlineto{\pgfqpoint{4.288960in}{2.009333in}}%
\pgfpathlineto{\pgfqpoint{4.274992in}{2.008797in}}%
\pgfpathlineto{\pgfqpoint{4.261034in}{2.008335in}}%
\pgfpathlineto{\pgfqpoint{4.253074in}{1.999281in}}%
\pgfpathlineto{\pgfqpoint{4.245108in}{1.990167in}}%
\pgfpathlineto{\pgfqpoint{4.237137in}{1.980995in}}%
\pgfpathlineto{\pgfqpoint{4.229159in}{1.971763in}}%
\pgfpathclose%
\pgfusepath{fill}%
\end{pgfscope}%
\begin{pgfscope}%
\pgfpathrectangle{\pgfqpoint{1.150000in}{0.150000in}}{\pgfqpoint{5.700000in}{5.700000in}}%
\pgfusepath{clip}%
\pgfsetbuttcap%
\pgfsetroundjoin%
\definecolor{currentfill}{rgb}{0.268510,0.009605,0.335427}%
\pgfsetfillcolor{currentfill}%
\pgfsetfillopacity{0.700000}%
\pgfsetlinewidth{0.000000pt}%
\definecolor{currentstroke}{rgb}{0.000000,0.000000,0.000000}%
\pgfsetstrokecolor{currentstroke}%
\pgfsetdash{}{0pt}%
\pgfpathmoveto{\pgfqpoint{3.040329in}{1.729586in}}%
\pgfpathlineto{\pgfqpoint{3.054050in}{1.723657in}}%
\pgfpathlineto{\pgfqpoint{3.067775in}{1.717817in}}%
\pgfpathlineto{\pgfqpoint{3.081503in}{1.712067in}}%
\pgfpathlineto{\pgfqpoint{3.095236in}{1.706404in}}%
\pgfpathlineto{\pgfqpoint{3.103667in}{1.713542in}}%
\pgfpathlineto{\pgfqpoint{3.112090in}{1.720771in}}%
\pgfpathlineto{\pgfqpoint{3.120505in}{1.728086in}}%
\pgfpathlineto{\pgfqpoint{3.128911in}{1.735485in}}%
\pgfpathlineto{\pgfqpoint{3.115198in}{1.740901in}}%
\pgfpathlineto{\pgfqpoint{3.101489in}{1.746406in}}%
\pgfpathlineto{\pgfqpoint{3.087784in}{1.752000in}}%
\pgfpathlineto{\pgfqpoint{3.074082in}{1.757682in}}%
\pgfpathlineto{\pgfqpoint{3.065657in}{1.750522in}}%
\pgfpathlineto{\pgfqpoint{3.057223in}{1.743450in}}%
\pgfpathlineto{\pgfqpoint{3.048780in}{1.736470in}}%
\pgfpathlineto{\pgfqpoint{3.040329in}{1.729586in}}%
\pgfpathclose%
\pgfusepath{fill}%
\end{pgfscope}%
\begin{pgfscope}%
\pgfpathrectangle{\pgfqpoint{1.150000in}{0.150000in}}{\pgfqpoint{5.700000in}{5.700000in}}%
\pgfusepath{clip}%
\pgfsetbuttcap%
\pgfsetroundjoin%
\definecolor{currentfill}{rgb}{0.243113,0.292092,0.538516}%
\pgfsetfillcolor{currentfill}%
\pgfsetfillopacity{0.700000}%
\pgfsetlinewidth{0.000000pt}%
\definecolor{currentstroke}{rgb}{0.000000,0.000000,0.000000}%
\pgfsetstrokecolor{currentstroke}%
\pgfsetdash{}{0pt}%
\pgfpathmoveto{\pgfqpoint{4.987292in}{2.287398in}}%
\pgfpathlineto{\pgfqpoint{5.001537in}{2.289991in}}%
\pgfpathlineto{\pgfqpoint{5.015793in}{2.292654in}}%
\pgfpathlineto{\pgfqpoint{5.030061in}{2.295387in}}%
\pgfpathlineto{\pgfqpoint{5.044340in}{2.298191in}}%
\pgfpathlineto{\pgfqpoint{5.051988in}{2.304599in}}%
\pgfpathlineto{\pgfqpoint{5.059630in}{2.310942in}}%
\pgfpathlineto{\pgfqpoint{5.067263in}{2.317224in}}%
\pgfpathlineto{\pgfqpoint{5.074890in}{2.323446in}}%
\pgfpathlineto{\pgfqpoint{5.060626in}{2.320793in}}%
\pgfpathlineto{\pgfqpoint{5.046373in}{2.318211in}}%
\pgfpathlineto{\pgfqpoint{5.032132in}{2.315699in}}%
\pgfpathlineto{\pgfqpoint{5.017902in}{2.313257in}}%
\pgfpathlineto{\pgfqpoint{5.010260in}{2.306876in}}%
\pgfpathlineto{\pgfqpoint{5.002611in}{2.300442in}}%
\pgfpathlineto{\pgfqpoint{4.994955in}{2.293950in}}%
\pgfpathlineto{\pgfqpoint{4.987292in}{2.287398in}}%
\pgfpathclose%
\pgfusepath{fill}%
\end{pgfscope}%
\begin{pgfscope}%
\pgfpathrectangle{\pgfqpoint{1.150000in}{0.150000in}}{\pgfqpoint{5.700000in}{5.700000in}}%
\pgfusepath{clip}%
\pgfsetbuttcap%
\pgfsetroundjoin%
\definecolor{currentfill}{rgb}{0.272594,0.025563,0.353093}%
\pgfsetfillcolor{currentfill}%
\pgfsetfillopacity{0.700000}%
\pgfsetlinewidth{0.000000pt}%
\definecolor{currentstroke}{rgb}{0.000000,0.000000,0.000000}%
\pgfsetstrokecolor{currentstroke}%
\pgfsetdash{}{0pt}%
\pgfpathmoveto{\pgfqpoint{2.896607in}{1.755913in}}%
\pgfpathlineto{\pgfqpoint{2.910325in}{1.748986in}}%
\pgfpathlineto{\pgfqpoint{2.924046in}{1.742153in}}%
\pgfpathlineto{\pgfqpoint{2.937770in}{1.735412in}}%
\pgfpathlineto{\pgfqpoint{2.951496in}{1.728763in}}%
\pgfpathlineto{\pgfqpoint{2.960006in}{1.734953in}}%
\pgfpathlineto{\pgfqpoint{2.968506in}{1.741258in}}%
\pgfpathlineto{\pgfqpoint{2.976997in}{1.747676in}}%
\pgfpathlineto{\pgfqpoint{2.985478in}{1.754202in}}%
\pgfpathlineto{\pgfqpoint{2.971773in}{1.760583in}}%
\pgfpathlineto{\pgfqpoint{2.958072in}{1.767057in}}%
\pgfpathlineto{\pgfqpoint{2.944373in}{1.773623in}}%
\pgfpathlineto{\pgfqpoint{2.930677in}{1.780282in}}%
\pgfpathlineto{\pgfqpoint{2.922175in}{1.774016in}}%
\pgfpathlineto{\pgfqpoint{2.913662in}{1.767863in}}%
\pgfpathlineto{\pgfqpoint{2.905140in}{1.761827in}}%
\pgfpathlineto{\pgfqpoint{2.896607in}{1.755913in}}%
\pgfpathclose%
\pgfusepath{fill}%
\end{pgfscope}%
\begin{pgfscope}%
\pgfpathrectangle{\pgfqpoint{1.150000in}{0.150000in}}{\pgfqpoint{5.700000in}{5.700000in}}%
\pgfusepath{clip}%
\pgfsetbuttcap%
\pgfsetroundjoin%
\definecolor{currentfill}{rgb}{0.267004,0.004874,0.329415}%
\pgfsetfillcolor{currentfill}%
\pgfsetfillopacity{0.700000}%
\pgfsetlinewidth{0.000000pt}%
\definecolor{currentstroke}{rgb}{0.000000,0.000000,0.000000}%
\pgfsetstrokecolor{currentstroke}%
\pgfsetdash{}{0pt}%
\pgfpathmoveto{\pgfqpoint{3.183804in}{1.714694in}}%
\pgfpathlineto{\pgfqpoint{3.197538in}{1.709712in}}%
\pgfpathlineto{\pgfqpoint{3.211276in}{1.704817in}}%
\pgfpathlineto{\pgfqpoint{3.225019in}{1.700007in}}%
\pgfpathlineto{\pgfqpoint{3.238766in}{1.695282in}}%
\pgfpathlineto{\pgfqpoint{3.247128in}{1.703227in}}%
\pgfpathlineto{\pgfqpoint{3.255483in}{1.711238in}}%
\pgfpathlineto{\pgfqpoint{3.263831in}{1.719311in}}%
\pgfpathlineto{\pgfqpoint{3.272171in}{1.727445in}}%
\pgfpathlineto{\pgfqpoint{3.258441in}{1.731945in}}%
\pgfpathlineto{\pgfqpoint{3.244715in}{1.736529in}}%
\pgfpathlineto{\pgfqpoint{3.230994in}{1.741200in}}%
\pgfpathlineto{\pgfqpoint{3.217278in}{1.745956in}}%
\pgfpathlineto{\pgfqpoint{3.208921in}{1.738039in}}%
\pgfpathlineto{\pgfqpoint{3.200556in}{1.730188in}}%
\pgfpathlineto{\pgfqpoint{3.192184in}{1.722405in}}%
\pgfpathlineto{\pgfqpoint{3.183804in}{1.714694in}}%
\pgfpathclose%
\pgfusepath{fill}%
\end{pgfscope}%
\begin{pgfscope}%
\pgfpathrectangle{\pgfqpoint{1.150000in}{0.150000in}}{\pgfqpoint{5.700000in}{5.700000in}}%
\pgfusepath{clip}%
\pgfsetbuttcap%
\pgfsetroundjoin%
\definecolor{currentfill}{rgb}{0.283072,0.130895,0.449241}%
\pgfsetfillcolor{currentfill}%
\pgfsetfillopacity{0.700000}%
\pgfsetlinewidth{0.000000pt}%
\definecolor{currentstroke}{rgb}{0.000000,0.000000,0.000000}%
\pgfsetstrokecolor{currentstroke}%
\pgfsetdash{}{0pt}%
\pgfpathmoveto{\pgfqpoint{4.141369in}{1.933371in}}%
\pgfpathlineto{\pgfqpoint{4.155312in}{1.933477in}}%
\pgfpathlineto{\pgfqpoint{4.169263in}{1.933658in}}%
\pgfpathlineto{\pgfqpoint{4.183223in}{1.933914in}}%
\pgfpathlineto{\pgfqpoint{4.197192in}{1.934243in}}%
\pgfpathlineto{\pgfqpoint{4.205192in}{1.943711in}}%
\pgfpathlineto{\pgfqpoint{4.213187in}{1.953121in}}%
\pgfpathlineto{\pgfqpoint{4.221176in}{1.962472in}}%
\pgfpathlineto{\pgfqpoint{4.229159in}{1.971763in}}%
\pgfpathlineto{\pgfqpoint{4.215200in}{1.971394in}}%
\pgfpathlineto{\pgfqpoint{4.201250in}{1.971099in}}%
\pgfpathlineto{\pgfqpoint{4.187308in}{1.970878in}}%
\pgfpathlineto{\pgfqpoint{4.173375in}{1.970731in}}%
\pgfpathlineto{\pgfqpoint{4.165382in}{1.961472in}}%
\pgfpathlineto{\pgfqpoint{4.157384in}{1.952159in}}%
\pgfpathlineto{\pgfqpoint{4.149379in}{1.942792in}}%
\pgfpathlineto{\pgfqpoint{4.141369in}{1.933371in}}%
\pgfpathclose%
\pgfusepath{fill}%
\end{pgfscope}%
\begin{pgfscope}%
\pgfpathrectangle{\pgfqpoint{1.150000in}{0.150000in}}{\pgfqpoint{5.700000in}{5.700000in}}%
\pgfusepath{clip}%
\pgfsetbuttcap%
\pgfsetroundjoin%
\definecolor{currentfill}{rgb}{0.274952,0.037752,0.364543}%
\pgfsetfillcolor{currentfill}%
\pgfsetfillopacity{0.700000}%
\pgfsetlinewidth{0.000000pt}%
\definecolor{currentstroke}{rgb}{0.000000,0.000000,0.000000}%
\pgfsetstrokecolor{currentstroke}%
\pgfsetdash{}{0pt}%
\pgfpathmoveto{\pgfqpoint{3.646579in}{1.767092in}}%
\pgfpathlineto{\pgfqpoint{3.660388in}{1.764857in}}%
\pgfpathlineto{\pgfqpoint{3.674203in}{1.762701in}}%
\pgfpathlineto{\pgfqpoint{3.688026in}{1.760624in}}%
\pgfpathlineto{\pgfqpoint{3.701855in}{1.758624in}}%
\pgfpathlineto{\pgfqpoint{3.710029in}{1.768192in}}%
\pgfpathlineto{\pgfqpoint{3.718197in}{1.777753in}}%
\pgfpathlineto{\pgfqpoint{3.726360in}{1.787307in}}%
\pgfpathlineto{\pgfqpoint{3.734516in}{1.796851in}}%
\pgfpathlineto{\pgfqpoint{3.720699in}{1.798707in}}%
\pgfpathlineto{\pgfqpoint{3.706888in}{1.800642in}}%
\pgfpathlineto{\pgfqpoint{3.693084in}{1.802655in}}%
\pgfpathlineto{\pgfqpoint{3.679287in}{1.804746in}}%
\pgfpathlineto{\pgfqpoint{3.671119in}{1.795338in}}%
\pgfpathlineto{\pgfqpoint{3.662945in}{1.785925in}}%
\pgfpathlineto{\pgfqpoint{3.654765in}{1.776509in}}%
\pgfpathlineto{\pgfqpoint{3.646579in}{1.767092in}}%
\pgfpathclose%
\pgfusepath{fill}%
\end{pgfscope}%
\begin{pgfscope}%
\pgfpathrectangle{\pgfqpoint{1.150000in}{0.150000in}}{\pgfqpoint{5.700000in}{5.700000in}}%
\pgfusepath{clip}%
\pgfsetbuttcap%
\pgfsetroundjoin%
\definecolor{currentfill}{rgb}{0.210503,0.363727,0.552206}%
\pgfsetfillcolor{currentfill}%
\pgfsetfillopacity{0.700000}%
\pgfsetlinewidth{0.000000pt}%
\definecolor{currentstroke}{rgb}{0.000000,0.000000,0.000000}%
\pgfsetstrokecolor{currentstroke}%
\pgfsetdash{}{0pt}%
\pgfpathmoveto{\pgfqpoint{5.482277in}{2.469056in}}%
\pgfpathlineto{\pgfqpoint{5.496718in}{2.472350in}}%
\pgfpathlineto{\pgfqpoint{5.511171in}{2.475713in}}%
\pgfpathlineto{\pgfqpoint{5.525637in}{2.479145in}}%
\pgfpathlineto{\pgfqpoint{5.540115in}{2.482646in}}%
\pgfpathlineto{\pgfqpoint{5.547505in}{2.486902in}}%
\pgfpathlineto{\pgfqpoint{5.554888in}{2.491136in}}%
\pgfpathlineto{\pgfqpoint{5.562264in}{2.495353in}}%
\pgfpathlineto{\pgfqpoint{5.569634in}{2.499556in}}%
\pgfpathlineto{\pgfqpoint{5.555178in}{2.496314in}}%
\pgfpathlineto{\pgfqpoint{5.540734in}{2.493141in}}%
\pgfpathlineto{\pgfqpoint{5.526303in}{2.490037in}}%
\pgfpathlineto{\pgfqpoint{5.511884in}{2.487001in}}%
\pgfpathlineto{\pgfqpoint{5.504492in}{2.482531in}}%
\pgfpathlineto{\pgfqpoint{5.497094in}{2.478053in}}%
\pgfpathlineto{\pgfqpoint{5.489689in}{2.473563in}}%
\pgfpathlineto{\pgfqpoint{5.482277in}{2.469056in}}%
\pgfpathclose%
\pgfusepath{fill}%
\end{pgfscope}%
\begin{pgfscope}%
\pgfpathrectangle{\pgfqpoint{1.150000in}{0.150000in}}{\pgfqpoint{5.700000in}{5.700000in}}%
\pgfusepath{clip}%
\pgfsetbuttcap%
\pgfsetroundjoin%
\definecolor{currentfill}{rgb}{0.248629,0.278775,0.534556}%
\pgfsetfillcolor{currentfill}%
\pgfsetfillopacity{0.700000}%
\pgfsetlinewidth{0.000000pt}%
\definecolor{currentstroke}{rgb}{0.000000,0.000000,0.000000}%
\pgfsetstrokecolor{currentstroke}%
\pgfsetdash{}{0pt}%
\pgfpathmoveto{\pgfqpoint{4.899642in}{2.250357in}}%
\pgfpathlineto{\pgfqpoint{4.913856in}{2.252796in}}%
\pgfpathlineto{\pgfqpoint{4.928081in}{2.255307in}}%
\pgfpathlineto{\pgfqpoint{4.942318in}{2.257888in}}%
\pgfpathlineto{\pgfqpoint{4.956566in}{2.260540in}}%
\pgfpathlineto{\pgfqpoint{4.964258in}{2.267358in}}%
\pgfpathlineto{\pgfqpoint{4.971943in}{2.274105in}}%
\pgfpathlineto{\pgfqpoint{4.979621in}{2.280784in}}%
\pgfpathlineto{\pgfqpoint{4.987292in}{2.287398in}}%
\pgfpathlineto{\pgfqpoint{4.973058in}{2.284876in}}%
\pgfpathlineto{\pgfqpoint{4.958836in}{2.282425in}}%
\pgfpathlineto{\pgfqpoint{4.944624in}{2.280044in}}%
\pgfpathlineto{\pgfqpoint{4.930424in}{2.277733in}}%
\pgfpathlineto{\pgfqpoint{4.922739in}{2.270982in}}%
\pgfpathlineto{\pgfqpoint{4.915047in}{2.264170in}}%
\pgfpathlineto{\pgfqpoint{4.907348in}{2.257296in}}%
\pgfpathlineto{\pgfqpoint{4.899642in}{2.250357in}}%
\pgfpathclose%
\pgfusepath{fill}%
\end{pgfscope}%
\begin{pgfscope}%
\pgfpathrectangle{\pgfqpoint{1.150000in}{0.150000in}}{\pgfqpoint{5.700000in}{5.700000in}}%
\pgfusepath{clip}%
\pgfsetbuttcap%
\pgfsetroundjoin%
\definecolor{currentfill}{rgb}{0.276194,0.190074,0.493001}%
\pgfsetfillcolor{currentfill}%
\pgfsetfillopacity{0.700000}%
\pgfsetlinewidth{0.000000pt}%
\definecolor{currentstroke}{rgb}{0.000000,0.000000,0.000000}%
\pgfsetstrokecolor{currentstroke}%
\pgfsetdash{}{0pt}%
\pgfpathmoveto{\pgfqpoint{2.242193in}{2.094010in}}%
\pgfpathlineto{\pgfqpoint{2.255987in}{2.081870in}}%
\pgfpathlineto{\pgfqpoint{2.269779in}{2.069853in}}%
\pgfpathlineto{\pgfqpoint{2.283568in}{2.057958in}}%
\pgfpathlineto{\pgfqpoint{2.297355in}{2.046185in}}%
\pgfpathlineto{\pgfqpoint{2.306309in}{2.047224in}}%
\pgfpathlineto{\pgfqpoint{2.315246in}{2.048487in}}%
\pgfpathlineto{\pgfqpoint{2.324166in}{2.049969in}}%
\pgfpathlineto{\pgfqpoint{2.333070in}{2.051667in}}%
\pgfpathlineto{\pgfqpoint{2.319318in}{2.063101in}}%
\pgfpathlineto{\pgfqpoint{2.305564in}{2.074657in}}%
\pgfpathlineto{\pgfqpoint{2.291808in}{2.086335in}}%
\pgfpathlineto{\pgfqpoint{2.278051in}{2.098135in}}%
\pgfpathlineto{\pgfqpoint{2.269112in}{2.096769in}}%
\pgfpathlineto{\pgfqpoint{2.260156in}{2.095622in}}%
\pgfpathlineto{\pgfqpoint{2.251183in}{2.094701in}}%
\pgfpathlineto{\pgfqpoint{2.242193in}{2.094010in}}%
\pgfpathclose%
\pgfusepath{fill}%
\end{pgfscope}%
\begin{pgfscope}%
\pgfpathrectangle{\pgfqpoint{1.150000in}{0.150000in}}{\pgfqpoint{5.700000in}{5.700000in}}%
\pgfusepath{clip}%
\pgfsetbuttcap%
\pgfsetroundjoin%
\definecolor{currentfill}{rgb}{0.283197,0.115680,0.436115}%
\pgfsetfillcolor{currentfill}%
\pgfsetfillopacity{0.700000}%
\pgfsetlinewidth{0.000000pt}%
\definecolor{currentstroke}{rgb}{0.000000,0.000000,0.000000}%
\pgfsetstrokecolor{currentstroke}%
\pgfsetdash{}{0pt}%
\pgfpathmoveto{\pgfqpoint{4.053549in}{1.895717in}}%
\pgfpathlineto{\pgfqpoint{4.067467in}{1.895465in}}%
\pgfpathlineto{\pgfqpoint{4.081393in}{1.895287in}}%
\pgfpathlineto{\pgfqpoint{4.095328in}{1.895184in}}%
\pgfpathlineto{\pgfqpoint{4.109272in}{1.895156in}}%
\pgfpathlineto{\pgfqpoint{4.117304in}{1.904788in}}%
\pgfpathlineto{\pgfqpoint{4.125332in}{1.914369in}}%
\pgfpathlineto{\pgfqpoint{4.133353in}{1.923896in}}%
\pgfpathlineto{\pgfqpoint{4.141369in}{1.933371in}}%
\pgfpathlineto{\pgfqpoint{4.127435in}{1.933338in}}%
\pgfpathlineto{\pgfqpoint{4.113510in}{1.933380in}}%
\pgfpathlineto{\pgfqpoint{4.099594in}{1.933497in}}%
\pgfpathlineto{\pgfqpoint{4.085686in}{1.933689in}}%
\pgfpathlineto{\pgfqpoint{4.077660in}{1.924267in}}%
\pgfpathlineto{\pgfqpoint{4.069629in}{1.914798in}}%
\pgfpathlineto{\pgfqpoint{4.061592in}{1.905281in}}%
\pgfpathlineto{\pgfqpoint{4.053549in}{1.895717in}}%
\pgfpathclose%
\pgfusepath{fill}%
\end{pgfscope}%
\begin{pgfscope}%
\pgfpathrectangle{\pgfqpoint{1.150000in}{0.150000in}}{\pgfqpoint{5.700000in}{5.700000in}}%
\pgfusepath{clip}%
\pgfsetbuttcap%
\pgfsetroundjoin%
\definecolor{currentfill}{rgb}{0.255645,0.260703,0.528312}%
\pgfsetfillcolor{currentfill}%
\pgfsetfillopacity{0.700000}%
\pgfsetlinewidth{0.000000pt}%
\definecolor{currentstroke}{rgb}{0.000000,0.000000,0.000000}%
\pgfsetstrokecolor{currentstroke}%
\pgfsetdash{}{0pt}%
\pgfpathmoveto{\pgfqpoint{4.811946in}{2.212416in}}%
\pgfpathlineto{\pgfqpoint{4.826129in}{2.214680in}}%
\pgfpathlineto{\pgfqpoint{4.840324in}{2.217016in}}%
\pgfpathlineto{\pgfqpoint{4.854529in}{2.219422in}}%
\pgfpathlineto{\pgfqpoint{4.868746in}{2.221900in}}%
\pgfpathlineto{\pgfqpoint{4.876481in}{2.229124in}}%
\pgfpathlineto{\pgfqpoint{4.884208in}{2.236273in}}%
\pgfpathlineto{\pgfqpoint{4.891928in}{2.243350in}}%
\pgfpathlineto{\pgfqpoint{4.899642in}{2.250357in}}%
\pgfpathlineto{\pgfqpoint{4.885438in}{2.247988in}}%
\pgfpathlineto{\pgfqpoint{4.871246in}{2.245689in}}%
\pgfpathlineto{\pgfqpoint{4.857065in}{2.243462in}}%
\pgfpathlineto{\pgfqpoint{4.842894in}{2.241306in}}%
\pgfpathlineto{\pgfqpoint{4.835168in}{2.234182in}}%
\pgfpathlineto{\pgfqpoint{4.827434in}{2.226995in}}%
\pgfpathlineto{\pgfqpoint{4.819693in}{2.219740in}}%
\pgfpathlineto{\pgfqpoint{4.811946in}{2.212416in}}%
\pgfpathclose%
\pgfusepath{fill}%
\end{pgfscope}%
\begin{pgfscope}%
\pgfpathrectangle{\pgfqpoint{1.150000in}{0.150000in}}{\pgfqpoint{5.700000in}{5.700000in}}%
\pgfusepath{clip}%
\pgfsetbuttcap%
\pgfsetroundjoin%
\definecolor{currentfill}{rgb}{0.268510,0.009605,0.335427}%
\pgfsetfillcolor{currentfill}%
\pgfsetfillopacity{0.700000}%
\pgfsetlinewidth{0.000000pt}%
\definecolor{currentstroke}{rgb}{0.000000,0.000000,0.000000}%
\pgfsetstrokecolor{currentstroke}%
\pgfsetdash{}{0pt}%
\pgfpathmoveto{\pgfqpoint{3.327141in}{1.710290in}}%
\pgfpathlineto{\pgfqpoint{3.340896in}{1.706211in}}%
\pgfpathlineto{\pgfqpoint{3.354656in}{1.702214in}}%
\pgfpathlineto{\pgfqpoint{3.368421in}{1.698300in}}%
\pgfpathlineto{\pgfqpoint{3.382192in}{1.694469in}}%
\pgfpathlineto{\pgfqpoint{3.390493in}{1.703084in}}%
\pgfpathlineto{\pgfqpoint{3.398788in}{1.711742in}}%
\pgfpathlineto{\pgfqpoint{3.407076in}{1.720441in}}%
\pgfpathlineto{\pgfqpoint{3.415357in}{1.729178in}}%
\pgfpathlineto{\pgfqpoint{3.401601in}{1.732805in}}%
\pgfpathlineto{\pgfqpoint{3.387851in}{1.736515in}}%
\pgfpathlineto{\pgfqpoint{3.374106in}{1.740307in}}%
\pgfpathlineto{\pgfqpoint{3.360366in}{1.744182in}}%
\pgfpathlineto{\pgfqpoint{3.352070in}{1.735641in}}%
\pgfpathlineto{\pgfqpoint{3.343767in}{1.727144in}}%
\pgfpathlineto{\pgfqpoint{3.335457in}{1.718693in}}%
\pgfpathlineto{\pgfqpoint{3.327141in}{1.710290in}}%
\pgfpathclose%
\pgfusepath{fill}%
\end{pgfscope}%
\begin{pgfscope}%
\pgfpathrectangle{\pgfqpoint{1.150000in}{0.150000in}}{\pgfqpoint{5.700000in}{5.700000in}}%
\pgfusepath{clip}%
\pgfsetbuttcap%
\pgfsetroundjoin%
\definecolor{currentfill}{rgb}{0.277018,0.050344,0.375715}%
\pgfsetfillcolor{currentfill}%
\pgfsetfillopacity{0.700000}%
\pgfsetlinewidth{0.000000pt}%
\definecolor{currentstroke}{rgb}{0.000000,0.000000,0.000000}%
\pgfsetstrokecolor{currentstroke}%
\pgfsetdash{}{0pt}%
\pgfpathmoveto{\pgfqpoint{2.752519in}{1.794690in}}%
\pgfpathlineto{\pgfqpoint{2.766244in}{1.786710in}}%
\pgfpathlineto{\pgfqpoint{2.779971in}{1.778827in}}%
\pgfpathlineto{\pgfqpoint{2.793700in}{1.771042in}}%
\pgfpathlineto{\pgfqpoint{2.807430in}{1.763353in}}%
\pgfpathlineto{\pgfqpoint{2.816029in}{1.768444in}}%
\pgfpathlineto{\pgfqpoint{2.824617in}{1.773679in}}%
\pgfpathlineto{\pgfqpoint{2.833194in}{1.779052in}}%
\pgfpathlineto{\pgfqpoint{2.841760in}{1.784561in}}%
\pgfpathlineto{\pgfqpoint{2.828054in}{1.791960in}}%
\pgfpathlineto{\pgfqpoint{2.814351in}{1.799457in}}%
\pgfpathlineto{\pgfqpoint{2.800649in}{1.807050in}}%
\pgfpathlineto{\pgfqpoint{2.786949in}{1.814741in}}%
\pgfpathlineto{\pgfqpoint{2.778359in}{1.809514in}}%
\pgfpathlineto{\pgfqpoint{2.769757in}{1.804427in}}%
\pgfpathlineto{\pgfqpoint{2.761144in}{1.799484in}}%
\pgfpathlineto{\pgfqpoint{2.752519in}{1.794690in}}%
\pgfpathclose%
\pgfusepath{fill}%
\end{pgfscope}%
\begin{pgfscope}%
\pgfpathrectangle{\pgfqpoint{1.150000in}{0.150000in}}{\pgfqpoint{5.700000in}{5.700000in}}%
\pgfusepath{clip}%
\pgfsetbuttcap%
\pgfsetroundjoin%
\definecolor{currentfill}{rgb}{0.214298,0.355619,0.551184}%
\pgfsetfillcolor{currentfill}%
\pgfsetfillopacity{0.700000}%
\pgfsetlinewidth{0.000000pt}%
\definecolor{currentstroke}{rgb}{0.000000,0.000000,0.000000}%
\pgfsetstrokecolor{currentstroke}%
\pgfsetdash{}{0pt}%
\pgfpathmoveto{\pgfqpoint{5.394836in}{2.437332in}}%
\pgfpathlineto{\pgfqpoint{5.409248in}{2.440587in}}%
\pgfpathlineto{\pgfqpoint{5.423673in}{2.443911in}}%
\pgfpathlineto{\pgfqpoint{5.438109in}{2.447305in}}%
\pgfpathlineto{\pgfqpoint{5.452559in}{2.450768in}}%
\pgfpathlineto{\pgfqpoint{5.459999in}{2.455387in}}%
\pgfpathlineto{\pgfqpoint{5.467432in}{2.459972in}}%
\pgfpathlineto{\pgfqpoint{5.474858in}{2.464527in}}%
\pgfpathlineto{\pgfqpoint{5.482277in}{2.469056in}}%
\pgfpathlineto{\pgfqpoint{5.467849in}{2.465831in}}%
\pgfpathlineto{\pgfqpoint{5.453433in}{2.462675in}}%
\pgfpathlineto{\pgfqpoint{5.439029in}{2.459588in}}%
\pgfpathlineto{\pgfqpoint{5.424637in}{2.456570in}}%
\pgfpathlineto{\pgfqpoint{5.417197in}{2.451796in}}%
\pgfpathlineto{\pgfqpoint{5.409751in}{2.447001in}}%
\pgfpathlineto{\pgfqpoint{5.402297in}{2.442181in}}%
\pgfpathlineto{\pgfqpoint{5.394836in}{2.437332in}}%
\pgfpathclose%
\pgfusepath{fill}%
\end{pgfscope}%
\begin{pgfscope}%
\pgfpathrectangle{\pgfqpoint{1.150000in}{0.150000in}}{\pgfqpoint{5.700000in}{5.700000in}}%
\pgfusepath{clip}%
\pgfsetbuttcap%
\pgfsetroundjoin%
\definecolor{currentfill}{rgb}{0.282327,0.094955,0.417331}%
\pgfsetfillcolor{currentfill}%
\pgfsetfillopacity{0.700000}%
\pgfsetlinewidth{0.000000pt}%
\definecolor{currentstroke}{rgb}{0.000000,0.000000,0.000000}%
\pgfsetstrokecolor{currentstroke}%
\pgfsetdash{}{0pt}%
\pgfpathmoveto{\pgfqpoint{2.552965in}{1.884428in}}%
\pgfpathlineto{\pgfqpoint{2.566706in}{1.874914in}}%
\pgfpathlineto{\pgfqpoint{2.580447in}{1.865507in}}%
\pgfpathlineto{\pgfqpoint{2.594189in}{1.856204in}}%
\pgfpathlineto{\pgfqpoint{2.607931in}{1.847005in}}%
\pgfpathlineto{\pgfqpoint{2.616661in}{1.850522in}}%
\pgfpathlineto{\pgfqpoint{2.625377in}{1.854216in}}%
\pgfpathlineto{\pgfqpoint{2.634081in}{1.858083in}}%
\pgfpathlineto{\pgfqpoint{2.642772in}{1.862118in}}%
\pgfpathlineto{\pgfqpoint{2.629058in}{1.871005in}}%
\pgfpathlineto{\pgfqpoint{2.615346in}{1.879996in}}%
\pgfpathlineto{\pgfqpoint{2.601634in}{1.889091in}}%
\pgfpathlineto{\pgfqpoint{2.587922in}{1.898293in}}%
\pgfpathlineto{\pgfqpoint{2.579203in}{1.894562in}}%
\pgfpathlineto{\pgfqpoint{2.570470in}{1.891004in}}%
\pgfpathlineto{\pgfqpoint{2.561725in}{1.887625in}}%
\pgfpathlineto{\pgfqpoint{2.552965in}{1.884428in}}%
\pgfpathclose%
\pgfusepath{fill}%
\end{pgfscope}%
\begin{pgfscope}%
\pgfpathrectangle{\pgfqpoint{1.150000in}{0.150000in}}{\pgfqpoint{5.700000in}{5.700000in}}%
\pgfusepath{clip}%
\pgfsetbuttcap%
\pgfsetroundjoin%
\definecolor{currentfill}{rgb}{0.272594,0.025563,0.353093}%
\pgfsetfillcolor{currentfill}%
\pgfsetfillopacity{0.700000}%
\pgfsetlinewidth{0.000000pt}%
\definecolor{currentstroke}{rgb}{0.000000,0.000000,0.000000}%
\pgfsetstrokecolor{currentstroke}%
\pgfsetdash{}{0pt}%
\pgfpathmoveto{\pgfqpoint{3.558558in}{1.739837in}}%
\pgfpathlineto{\pgfqpoint{3.572353in}{1.737123in}}%
\pgfpathlineto{\pgfqpoint{3.586155in}{1.734487in}}%
\pgfpathlineto{\pgfqpoint{3.599963in}{1.731931in}}%
\pgfpathlineto{\pgfqpoint{3.613777in}{1.729454in}}%
\pgfpathlineto{\pgfqpoint{3.621986in}{1.738855in}}%
\pgfpathlineto{\pgfqpoint{3.630190in}{1.748263in}}%
\pgfpathlineto{\pgfqpoint{3.638387in}{1.757676in}}%
\pgfpathlineto{\pgfqpoint{3.646579in}{1.767092in}}%
\pgfpathlineto{\pgfqpoint{3.632777in}{1.769405in}}%
\pgfpathlineto{\pgfqpoint{3.618982in}{1.771798in}}%
\pgfpathlineto{\pgfqpoint{3.605193in}{1.774269in}}%
\pgfpathlineto{\pgfqpoint{3.591410in}{1.776821in}}%
\pgfpathlineto{\pgfqpoint{3.583206in}{1.767561in}}%
\pgfpathlineto{\pgfqpoint{3.574996in}{1.758308in}}%
\pgfpathlineto{\pgfqpoint{3.566780in}{1.749067in}}%
\pgfpathlineto{\pgfqpoint{3.558558in}{1.739837in}}%
\pgfpathclose%
\pgfusepath{fill}%
\end{pgfscope}%
\begin{pgfscope}%
\pgfpathrectangle{\pgfqpoint{1.150000in}{0.150000in}}{\pgfqpoint{5.700000in}{5.700000in}}%
\pgfusepath{clip}%
\pgfsetbuttcap%
\pgfsetroundjoin%
\definecolor{currentfill}{rgb}{0.282327,0.094955,0.417331}%
\pgfsetfillcolor{currentfill}%
\pgfsetfillopacity{0.700000}%
\pgfsetlinewidth{0.000000pt}%
\definecolor{currentstroke}{rgb}{0.000000,0.000000,0.000000}%
\pgfsetstrokecolor{currentstroke}%
\pgfsetdash{}{0pt}%
\pgfpathmoveto{\pgfqpoint{3.965696in}{1.859093in}}%
\pgfpathlineto{\pgfqpoint{3.979590in}{1.858458in}}%
\pgfpathlineto{\pgfqpoint{3.993493in}{1.857899in}}%
\pgfpathlineto{\pgfqpoint{4.007404in}{1.857415in}}%
\pgfpathlineto{\pgfqpoint{4.021323in}{1.857007in}}%
\pgfpathlineto{\pgfqpoint{4.029388in}{1.866751in}}%
\pgfpathlineto{\pgfqpoint{4.037447in}{1.876451in}}%
\pgfpathlineto{\pgfqpoint{4.045501in}{1.886107in}}%
\pgfpathlineto{\pgfqpoint{4.053549in}{1.895717in}}%
\pgfpathlineto{\pgfqpoint{4.039640in}{1.896045in}}%
\pgfpathlineto{\pgfqpoint{4.025739in}{1.896447in}}%
\pgfpathlineto{\pgfqpoint{4.011846in}{1.896925in}}%
\pgfpathlineto{\pgfqpoint{3.997961in}{1.897478in}}%
\pgfpathlineto{\pgfqpoint{3.989903in}{1.887941in}}%
\pgfpathlineto{\pgfqpoint{3.981839in}{1.878365in}}%
\pgfpathlineto{\pgfqpoint{3.973770in}{1.868748in}}%
\pgfpathlineto{\pgfqpoint{3.965696in}{1.859093in}}%
\pgfpathclose%
\pgfusepath{fill}%
\end{pgfscope}%
\begin{pgfscope}%
\pgfpathrectangle{\pgfqpoint{1.150000in}{0.150000in}}{\pgfqpoint{5.700000in}{5.700000in}}%
\pgfusepath{clip}%
\pgfsetbuttcap%
\pgfsetroundjoin%
\definecolor{currentfill}{rgb}{0.260571,0.246922,0.522828}%
\pgfsetfillcolor{currentfill}%
\pgfsetfillopacity{0.700000}%
\pgfsetlinewidth{0.000000pt}%
\definecolor{currentstroke}{rgb}{0.000000,0.000000,0.000000}%
\pgfsetstrokecolor{currentstroke}%
\pgfsetdash{}{0pt}%
\pgfpathmoveto{\pgfqpoint{4.724210in}{2.173692in}}%
\pgfpathlineto{\pgfqpoint{4.738363in}{2.175759in}}%
\pgfpathlineto{\pgfqpoint{4.752526in}{2.177897in}}%
\pgfpathlineto{\pgfqpoint{4.766701in}{2.180106in}}%
\pgfpathlineto{\pgfqpoint{4.780886in}{2.182387in}}%
\pgfpathlineto{\pgfqpoint{4.788661in}{2.190008in}}%
\pgfpathlineto{\pgfqpoint{4.796430in}{2.197552in}}%
\pgfpathlineto{\pgfqpoint{4.804191in}{2.205020in}}%
\pgfpathlineto{\pgfqpoint{4.811946in}{2.212416in}}%
\pgfpathlineto{\pgfqpoint{4.797773in}{2.210222in}}%
\pgfpathlineto{\pgfqpoint{4.783611in}{2.208100in}}%
\pgfpathlineto{\pgfqpoint{4.769460in}{2.206049in}}%
\pgfpathlineto{\pgfqpoint{4.755319in}{2.204069in}}%
\pgfpathlineto{\pgfqpoint{4.747552in}{2.196579in}}%
\pgfpathlineto{\pgfqpoint{4.739778in}{2.189021in}}%
\pgfpathlineto{\pgfqpoint{4.731998in}{2.181392in}}%
\pgfpathlineto{\pgfqpoint{4.724210in}{2.173692in}}%
\pgfpathclose%
\pgfusepath{fill}%
\end{pgfscope}%
\begin{pgfscope}%
\pgfpathrectangle{\pgfqpoint{1.150000in}{0.150000in}}{\pgfqpoint{5.700000in}{5.700000in}}%
\pgfusepath{clip}%
\pgfsetbuttcap%
\pgfsetroundjoin%
\definecolor{currentfill}{rgb}{0.279574,0.170599,0.479997}%
\pgfsetfillcolor{currentfill}%
\pgfsetfillopacity{0.700000}%
\pgfsetlinewidth{0.000000pt}%
\definecolor{currentstroke}{rgb}{0.000000,0.000000,0.000000}%
\pgfsetstrokecolor{currentstroke}%
\pgfsetdash{}{0pt}%
\pgfpathmoveto{\pgfqpoint{2.297355in}{2.046185in}}%
\pgfpathlineto{\pgfqpoint{2.311141in}{2.034533in}}%
\pgfpathlineto{\pgfqpoint{2.324925in}{2.023000in}}%
\pgfpathlineto{\pgfqpoint{2.338707in}{2.011585in}}%
\pgfpathlineto{\pgfqpoint{2.352488in}{2.000288in}}%
\pgfpathlineto{\pgfqpoint{2.361406in}{2.001672in}}%
\pgfpathlineto{\pgfqpoint{2.370307in}{2.003275in}}%
\pgfpathlineto{\pgfqpoint{2.379193in}{2.005093in}}%
\pgfpathlineto{\pgfqpoint{2.388063in}{2.007120in}}%
\pgfpathlineto{\pgfqpoint{2.374317in}{2.018080in}}%
\pgfpathlineto{\pgfqpoint{2.360569in}{2.029157in}}%
\pgfpathlineto{\pgfqpoint{2.346820in}{2.040352in}}%
\pgfpathlineto{\pgfqpoint{2.333070in}{2.051667in}}%
\pgfpathlineto{\pgfqpoint{2.324166in}{2.049969in}}%
\pgfpathlineto{\pgfqpoint{2.315246in}{2.048487in}}%
\pgfpathlineto{\pgfqpoint{2.306309in}{2.047224in}}%
\pgfpathlineto{\pgfqpoint{2.297355in}{2.046185in}}%
\pgfpathclose%
\pgfusepath{fill}%
\end{pgfscope}%
\begin{pgfscope}%
\pgfpathrectangle{\pgfqpoint{1.150000in}{0.150000in}}{\pgfqpoint{5.700000in}{5.700000in}}%
\pgfusepath{clip}%
\pgfsetbuttcap%
\pgfsetroundjoin%
\definecolor{currentfill}{rgb}{0.280894,0.078907,0.402329}%
\pgfsetfillcolor{currentfill}%
\pgfsetfillopacity{0.700000}%
\pgfsetlinewidth{0.000000pt}%
\definecolor{currentstroke}{rgb}{0.000000,0.000000,0.000000}%
\pgfsetstrokecolor{currentstroke}%
\pgfsetdash{}{0pt}%
\pgfpathmoveto{\pgfqpoint{3.877801in}{1.823810in}}%
\pgfpathlineto{\pgfqpoint{3.891675in}{1.822770in}}%
\pgfpathlineto{\pgfqpoint{3.905556in}{1.821806in}}%
\pgfpathlineto{\pgfqpoint{3.919444in}{1.820918in}}%
\pgfpathlineto{\pgfqpoint{3.933341in}{1.820105in}}%
\pgfpathlineto{\pgfqpoint{3.941438in}{1.829905in}}%
\pgfpathlineto{\pgfqpoint{3.949529in}{1.839670in}}%
\pgfpathlineto{\pgfqpoint{3.957615in}{1.849400in}}%
\pgfpathlineto{\pgfqpoint{3.965696in}{1.859093in}}%
\pgfpathlineto{\pgfqpoint{3.951809in}{1.859803in}}%
\pgfpathlineto{\pgfqpoint{3.937930in}{1.860589in}}%
\pgfpathlineto{\pgfqpoint{3.924060in}{1.861451in}}%
\pgfpathlineto{\pgfqpoint{3.910197in}{1.862389in}}%
\pgfpathlineto{\pgfqpoint{3.902106in}{1.852791in}}%
\pgfpathlineto{\pgfqpoint{3.894010in}{1.843160in}}%
\pgfpathlineto{\pgfqpoint{3.885909in}{1.833500in}}%
\pgfpathlineto{\pgfqpoint{3.877801in}{1.823810in}}%
\pgfpathclose%
\pgfusepath{fill}%
\end{pgfscope}%
\begin{pgfscope}%
\pgfpathrectangle{\pgfqpoint{1.150000in}{0.150000in}}{\pgfqpoint{5.700000in}{5.700000in}}%
\pgfusepath{clip}%
\pgfsetbuttcap%
\pgfsetroundjoin%
\definecolor{currentfill}{rgb}{0.266580,0.228262,0.514349}%
\pgfsetfillcolor{currentfill}%
\pgfsetfillopacity{0.700000}%
\pgfsetlinewidth{0.000000pt}%
\definecolor{currentstroke}{rgb}{0.000000,0.000000,0.000000}%
\pgfsetstrokecolor{currentstroke}%
\pgfsetdash{}{0pt}%
\pgfpathmoveto{\pgfqpoint{4.636439in}{2.134326in}}%
\pgfpathlineto{\pgfqpoint{4.650562in}{2.136172in}}%
\pgfpathlineto{\pgfqpoint{4.664694in}{2.138090in}}%
\pgfpathlineto{\pgfqpoint{4.678838in}{2.140080in}}%
\pgfpathlineto{\pgfqpoint{4.692992in}{2.142141in}}%
\pgfpathlineto{\pgfqpoint{4.700806in}{2.150145in}}%
\pgfpathlineto{\pgfqpoint{4.708614in}{2.158070in}}%
\pgfpathlineto{\pgfqpoint{4.716416in}{2.165919in}}%
\pgfpathlineto{\pgfqpoint{4.724210in}{2.173692in}}%
\pgfpathlineto{\pgfqpoint{4.710068in}{2.171697in}}%
\pgfpathlineto{\pgfqpoint{4.695936in}{2.169773in}}%
\pgfpathlineto{\pgfqpoint{4.681815in}{2.167921in}}%
\pgfpathlineto{\pgfqpoint{4.667704in}{2.166141in}}%
\pgfpathlineto{\pgfqpoint{4.659898in}{2.158294in}}%
\pgfpathlineto{\pgfqpoint{4.652085in}{2.150377in}}%
\pgfpathlineto{\pgfqpoint{4.644266in}{2.142388in}}%
\pgfpathlineto{\pgfqpoint{4.636439in}{2.134326in}}%
\pgfpathclose%
\pgfusepath{fill}%
\end{pgfscope}%
\begin{pgfscope}%
\pgfpathrectangle{\pgfqpoint{1.150000in}{0.150000in}}{\pgfqpoint{5.700000in}{5.700000in}}%
\pgfusepath{clip}%
\pgfsetbuttcap%
\pgfsetroundjoin%
\definecolor{currentfill}{rgb}{0.220057,0.343307,0.549413}%
\pgfsetfillcolor{currentfill}%
\pgfsetfillopacity{0.700000}%
\pgfsetlinewidth{0.000000pt}%
\definecolor{currentstroke}{rgb}{0.000000,0.000000,0.000000}%
\pgfsetstrokecolor{currentstroke}%
\pgfsetdash{}{0pt}%
\pgfpathmoveto{\pgfqpoint{5.307315in}{2.404367in}}%
\pgfpathlineto{\pgfqpoint{5.321698in}{2.407560in}}%
\pgfpathlineto{\pgfqpoint{5.336093in}{2.410823in}}%
\pgfpathlineto{\pgfqpoint{5.350500in}{2.414156in}}%
\pgfpathlineto{\pgfqpoint{5.364919in}{2.417558in}}%
\pgfpathlineto{\pgfqpoint{5.372410in}{2.422566in}}%
\pgfpathlineto{\pgfqpoint{5.379893in}{2.427528in}}%
\pgfpathlineto{\pgfqpoint{5.387368in}{2.432449in}}%
\pgfpathlineto{\pgfqpoint{5.394836in}{2.437332in}}%
\pgfpathlineto{\pgfqpoint{5.380436in}{2.434146in}}%
\pgfpathlineto{\pgfqpoint{5.366048in}{2.431030in}}%
\pgfpathlineto{\pgfqpoint{5.351673in}{2.427982in}}%
\pgfpathlineto{\pgfqpoint{5.337309in}{2.425005in}}%
\pgfpathlineto{\pgfqpoint{5.329821in}{2.419898in}}%
\pgfpathlineto{\pgfqpoint{5.322327in}{2.414759in}}%
\pgfpathlineto{\pgfqpoint{5.314825in}{2.409583in}}%
\pgfpathlineto{\pgfqpoint{5.307315in}{2.404367in}}%
\pgfpathclose%
\pgfusepath{fill}%
\end{pgfscope}%
\begin{pgfscope}%
\pgfpathrectangle{\pgfqpoint{1.150000in}{0.150000in}}{\pgfqpoint{5.700000in}{5.700000in}}%
\pgfusepath{clip}%
\pgfsetbuttcap%
\pgfsetroundjoin%
\definecolor{currentfill}{rgb}{0.195860,0.395433,0.555276}%
\pgfsetfillcolor{currentfill}%
\pgfsetfillopacity{0.700000}%
\pgfsetlinewidth{0.000000pt}%
\definecolor{currentstroke}{rgb}{0.000000,0.000000,0.000000}%
\pgfsetstrokecolor{currentstroke}%
\pgfsetdash{}{0pt}%
\pgfpathmoveto{\pgfqpoint{5.714957in}{2.542505in}}%
\pgfpathlineto{\pgfqpoint{5.729503in}{2.546084in}}%
\pgfpathlineto{\pgfqpoint{5.744062in}{2.549732in}}%
\pgfpathlineto{\pgfqpoint{5.758635in}{2.553448in}}%
\pgfpathlineto{\pgfqpoint{5.765904in}{2.556853in}}%
\pgfpathlineto{\pgfqpoint{5.773167in}{2.560260in}}%
\pgfpathlineto{\pgfqpoint{5.780424in}{2.563677in}}%
\pgfpathlineto{\pgfqpoint{5.787674in}{2.567106in}}%
\pgfpathlineto{\pgfqpoint{5.773128in}{2.563693in}}%
\pgfpathlineto{\pgfqpoint{5.758594in}{2.560347in}}%
\pgfpathlineto{\pgfqpoint{5.744073in}{2.557070in}}%
\pgfpathlineto{\pgfqpoint{5.736804in}{2.553408in}}%
\pgfpathlineto{\pgfqpoint{5.729528in}{2.549763in}}%
\pgfpathlineto{\pgfqpoint{5.722246in}{2.546131in}}%
\pgfpathlineto{\pgfqpoint{5.714957in}{2.542505in}}%
\pgfpathclose%
\pgfusepath{fill}%
\end{pgfscope}%
\begin{pgfscope}%
\pgfpathrectangle{\pgfqpoint{1.150000in}{0.150000in}}{\pgfqpoint{5.700000in}{5.700000in}}%
\pgfusepath{clip}%
\pgfsetbuttcap%
\pgfsetroundjoin%
\definecolor{currentfill}{rgb}{0.268510,0.009605,0.335427}%
\pgfsetfillcolor{currentfill}%
\pgfsetfillopacity{0.700000}%
\pgfsetlinewidth{0.000000pt}%
\definecolor{currentstroke}{rgb}{0.000000,0.000000,0.000000}%
\pgfsetstrokecolor{currentstroke}%
\pgfsetdash{}{0pt}%
\pgfpathmoveto{\pgfqpoint{3.095236in}{1.706404in}}%
\pgfpathlineto{\pgfqpoint{3.108972in}{1.700830in}}%
\pgfpathlineto{\pgfqpoint{3.122712in}{1.695343in}}%
\pgfpathlineto{\pgfqpoint{3.136456in}{1.689943in}}%
\pgfpathlineto{\pgfqpoint{3.150204in}{1.684629in}}%
\pgfpathlineto{\pgfqpoint{3.158616in}{1.692021in}}%
\pgfpathlineto{\pgfqpoint{3.167020in}{1.699498in}}%
\pgfpathlineto{\pgfqpoint{3.175416in}{1.707057in}}%
\pgfpathlineto{\pgfqpoint{3.183804in}{1.714694in}}%
\pgfpathlineto{\pgfqpoint{3.170074in}{1.719761in}}%
\pgfpathlineto{\pgfqpoint{3.156349in}{1.724915in}}%
\pgfpathlineto{\pgfqpoint{3.142628in}{1.730156in}}%
\pgfpathlineto{\pgfqpoint{3.128911in}{1.735485in}}%
\pgfpathlineto{\pgfqpoint{3.120505in}{1.728086in}}%
\pgfpathlineto{\pgfqpoint{3.112090in}{1.720771in}}%
\pgfpathlineto{\pgfqpoint{3.103667in}{1.713542in}}%
\pgfpathlineto{\pgfqpoint{3.095236in}{1.706404in}}%
\pgfpathclose%
\pgfusepath{fill}%
\end{pgfscope}%
\begin{pgfscope}%
\pgfpathrectangle{\pgfqpoint{1.150000in}{0.150000in}}{\pgfqpoint{5.700000in}{5.700000in}}%
\pgfusepath{clip}%
\pgfsetbuttcap%
\pgfsetroundjoin%
\definecolor{currentfill}{rgb}{0.271305,0.019942,0.347269}%
\pgfsetfillcolor{currentfill}%
\pgfsetfillopacity{0.700000}%
\pgfsetlinewidth{0.000000pt}%
\definecolor{currentstroke}{rgb}{0.000000,0.000000,0.000000}%
\pgfsetstrokecolor{currentstroke}%
\pgfsetdash{}{0pt}%
\pgfpathmoveto{\pgfqpoint{2.951496in}{1.728763in}}%
\pgfpathlineto{\pgfqpoint{2.965226in}{1.722206in}}%
\pgfpathlineto{\pgfqpoint{2.978958in}{1.715740in}}%
\pgfpathlineto{\pgfqpoint{2.992694in}{1.709365in}}%
\pgfpathlineto{\pgfqpoint{3.006433in}{1.703079in}}%
\pgfpathlineto{\pgfqpoint{3.014921in}{1.709544in}}%
\pgfpathlineto{\pgfqpoint{3.023399in}{1.716119in}}%
\pgfpathlineto{\pgfqpoint{3.031869in}{1.722801in}}%
\pgfpathlineto{\pgfqpoint{3.040329in}{1.729586in}}%
\pgfpathlineto{\pgfqpoint{3.026611in}{1.735604in}}%
\pgfpathlineto{\pgfqpoint{3.012897in}{1.741712in}}%
\pgfpathlineto{\pgfqpoint{2.999186in}{1.747911in}}%
\pgfpathlineto{\pgfqpoint{2.985478in}{1.754202in}}%
\pgfpathlineto{\pgfqpoint{2.976997in}{1.747676in}}%
\pgfpathlineto{\pgfqpoint{2.968506in}{1.741258in}}%
\pgfpathlineto{\pgfqpoint{2.960006in}{1.734953in}}%
\pgfpathlineto{\pgfqpoint{2.951496in}{1.728763in}}%
\pgfpathclose%
\pgfusepath{fill}%
\end{pgfscope}%
\begin{pgfscope}%
\pgfpathrectangle{\pgfqpoint{1.150000in}{0.150000in}}{\pgfqpoint{5.700000in}{5.700000in}}%
\pgfusepath{clip}%
\pgfsetbuttcap%
\pgfsetroundjoin%
\definecolor{currentfill}{rgb}{0.270595,0.214069,0.507052}%
\pgfsetfillcolor{currentfill}%
\pgfsetfillopacity{0.700000}%
\pgfsetlinewidth{0.000000pt}%
\definecolor{currentstroke}{rgb}{0.000000,0.000000,0.000000}%
\pgfsetstrokecolor{currentstroke}%
\pgfsetdash{}{0pt}%
\pgfpathmoveto{\pgfqpoint{4.548639in}{2.094478in}}%
\pgfpathlineto{\pgfqpoint{4.562731in}{2.096081in}}%
\pgfpathlineto{\pgfqpoint{4.576833in}{2.097756in}}%
\pgfpathlineto{\pgfqpoint{4.590946in}{2.099503in}}%
\pgfpathlineto{\pgfqpoint{4.605069in}{2.101322in}}%
\pgfpathlineto{\pgfqpoint{4.612921in}{2.109689in}}%
\pgfpathlineto{\pgfqpoint{4.620767in}{2.117978in}}%
\pgfpathlineto{\pgfqpoint{4.628607in}{2.126190in}}%
\pgfpathlineto{\pgfqpoint{4.636439in}{2.134326in}}%
\pgfpathlineto{\pgfqpoint{4.622328in}{2.132552in}}%
\pgfpathlineto{\pgfqpoint{4.608226in}{2.130849in}}%
\pgfpathlineto{\pgfqpoint{4.594135in}{2.129219in}}%
\pgfpathlineto{\pgfqpoint{4.580053in}{2.127660in}}%
\pgfpathlineto{\pgfqpoint{4.572210in}{2.119472in}}%
\pgfpathlineto{\pgfqpoint{4.564359in}{2.111213in}}%
\pgfpathlineto{\pgfqpoint{4.556502in}{2.102882in}}%
\pgfpathlineto{\pgfqpoint{4.548639in}{2.094478in}}%
\pgfpathclose%
\pgfusepath{fill}%
\end{pgfscope}%
\begin{pgfscope}%
\pgfpathrectangle{\pgfqpoint{1.150000in}{0.150000in}}{\pgfqpoint{5.700000in}{5.700000in}}%
\pgfusepath{clip}%
\pgfsetbuttcap%
\pgfsetroundjoin%
\definecolor{currentfill}{rgb}{0.271305,0.019942,0.347269}%
\pgfsetfillcolor{currentfill}%
\pgfsetfillopacity{0.700000}%
\pgfsetlinewidth{0.000000pt}%
\definecolor{currentstroke}{rgb}{0.000000,0.000000,0.000000}%
\pgfsetstrokecolor{currentstroke}%
\pgfsetdash{}{0pt}%
\pgfpathmoveto{\pgfqpoint{3.470436in}{1.715488in}}%
\pgfpathlineto{\pgfqpoint{3.484220in}{1.712269in}}%
\pgfpathlineto{\pgfqpoint{3.498010in}{1.709130in}}%
\pgfpathlineto{\pgfqpoint{3.511806in}{1.706071in}}%
\pgfpathlineto{\pgfqpoint{3.525608in}{1.703092in}}%
\pgfpathlineto{\pgfqpoint{3.533855in}{1.712248in}}%
\pgfpathlineto{\pgfqpoint{3.542096in}{1.721425in}}%
\pgfpathlineto{\pgfqpoint{3.550330in}{1.730623in}}%
\pgfpathlineto{\pgfqpoint{3.558558in}{1.739837in}}%
\pgfpathlineto{\pgfqpoint{3.544769in}{1.742632in}}%
\pgfpathlineto{\pgfqpoint{3.530986in}{1.745507in}}%
\pgfpathlineto{\pgfqpoint{3.517210in}{1.748462in}}%
\pgfpathlineto{\pgfqpoint{3.503439in}{1.751497in}}%
\pgfpathlineto{\pgfqpoint{3.495198in}{1.742459in}}%
\pgfpathlineto{\pgfqpoint{3.486950in}{1.733443in}}%
\pgfpathlineto{\pgfqpoint{3.478696in}{1.724452in}}%
\pgfpathlineto{\pgfqpoint{3.470436in}{1.715488in}}%
\pgfpathclose%
\pgfusepath{fill}%
\end{pgfscope}%
\begin{pgfscope}%
\pgfpathrectangle{\pgfqpoint{1.150000in}{0.150000in}}{\pgfqpoint{5.700000in}{5.700000in}}%
\pgfusepath{clip}%
\pgfsetbuttcap%
\pgfsetroundjoin%
\definecolor{currentfill}{rgb}{0.278791,0.062145,0.386592}%
\pgfsetfillcolor{currentfill}%
\pgfsetfillopacity{0.700000}%
\pgfsetlinewidth{0.000000pt}%
\definecolor{currentstroke}{rgb}{0.000000,0.000000,0.000000}%
\pgfsetstrokecolor{currentstroke}%
\pgfsetdash{}{0pt}%
\pgfpathmoveto{\pgfqpoint{3.789858in}{1.790202in}}%
\pgfpathlineto{\pgfqpoint{3.803712in}{1.788733in}}%
\pgfpathlineto{\pgfqpoint{3.817573in}{1.787340in}}%
\pgfpathlineto{\pgfqpoint{3.831441in}{1.786025in}}%
\pgfpathlineto{\pgfqpoint{3.845317in}{1.784786in}}%
\pgfpathlineto{\pgfqpoint{3.853447in}{1.794579in}}%
\pgfpathlineto{\pgfqpoint{3.861570in}{1.804348in}}%
\pgfpathlineto{\pgfqpoint{3.869689in}{1.814092in}}%
\pgfpathlineto{\pgfqpoint{3.877801in}{1.823810in}}%
\pgfpathlineto{\pgfqpoint{3.863936in}{1.824926in}}%
\pgfpathlineto{\pgfqpoint{3.850078in}{1.826119in}}%
\pgfpathlineto{\pgfqpoint{3.836228in}{1.827389in}}%
\pgfpathlineto{\pgfqpoint{3.822385in}{1.828735in}}%
\pgfpathlineto{\pgfqpoint{3.814262in}{1.819133in}}%
\pgfpathlineto{\pgfqpoint{3.806133in}{1.809508in}}%
\pgfpathlineto{\pgfqpoint{3.797998in}{1.799864in}}%
\pgfpathlineto{\pgfqpoint{3.789858in}{1.790202in}}%
\pgfpathclose%
\pgfusepath{fill}%
\end{pgfscope}%
\begin{pgfscope}%
\pgfpathrectangle{\pgfqpoint{1.150000in}{0.150000in}}{\pgfqpoint{5.700000in}{5.700000in}}%
\pgfusepath{clip}%
\pgfsetbuttcap%
\pgfsetroundjoin%
\definecolor{currentfill}{rgb}{0.275191,0.194905,0.496005}%
\pgfsetfillcolor{currentfill}%
\pgfsetfillopacity{0.700000}%
\pgfsetlinewidth{0.000000pt}%
\definecolor{currentstroke}{rgb}{0.000000,0.000000,0.000000}%
\pgfsetstrokecolor{currentstroke}%
\pgfsetdash{}{0pt}%
\pgfpathmoveto{\pgfqpoint{4.460812in}{2.054330in}}%
\pgfpathlineto{\pgfqpoint{4.474874in}{2.055667in}}%
\pgfpathlineto{\pgfqpoint{4.488946in}{2.057077in}}%
\pgfpathlineto{\pgfqpoint{4.503029in}{2.058559in}}%
\pgfpathlineto{\pgfqpoint{4.517121in}{2.060113in}}%
\pgfpathlineto{\pgfqpoint{4.525010in}{2.068818in}}%
\pgfpathlineto{\pgfqpoint{4.532893in}{2.077447in}}%
\pgfpathlineto{\pgfqpoint{4.540769in}{2.086000in}}%
\pgfpathlineto{\pgfqpoint{4.548639in}{2.094478in}}%
\pgfpathlineto{\pgfqpoint{4.534557in}{2.092947in}}%
\pgfpathlineto{\pgfqpoint{4.520485in}{2.091489in}}%
\pgfpathlineto{\pgfqpoint{4.506423in}{2.090102in}}%
\pgfpathlineto{\pgfqpoint{4.492371in}{2.088789in}}%
\pgfpathlineto{\pgfqpoint{4.484491in}{2.080280in}}%
\pgfpathlineto{\pgfqpoint{4.476604in}{2.071701in}}%
\pgfpathlineto{\pgfqpoint{4.468711in}{2.063051in}}%
\pgfpathlineto{\pgfqpoint{4.460812in}{2.054330in}}%
\pgfpathclose%
\pgfusepath{fill}%
\end{pgfscope}%
\begin{pgfscope}%
\pgfpathrectangle{\pgfqpoint{1.150000in}{0.150000in}}{\pgfqpoint{5.700000in}{5.700000in}}%
\pgfusepath{clip}%
\pgfsetbuttcap%
\pgfsetroundjoin%
\definecolor{currentfill}{rgb}{0.267004,0.004874,0.329415}%
\pgfsetfillcolor{currentfill}%
\pgfsetfillopacity{0.700000}%
\pgfsetlinewidth{0.000000pt}%
\definecolor{currentstroke}{rgb}{0.000000,0.000000,0.000000}%
\pgfsetstrokecolor{currentstroke}%
\pgfsetdash{}{0pt}%
\pgfpathmoveto{\pgfqpoint{3.238766in}{1.695282in}}%
\pgfpathlineto{\pgfqpoint{3.252518in}{1.690642in}}%
\pgfpathlineto{\pgfqpoint{3.266274in}{1.686087in}}%
\pgfpathlineto{\pgfqpoint{3.280036in}{1.681615in}}%
\pgfpathlineto{\pgfqpoint{3.293802in}{1.677227in}}%
\pgfpathlineto{\pgfqpoint{3.302147in}{1.685404in}}%
\pgfpathlineto{\pgfqpoint{3.310486in}{1.693643in}}%
\pgfpathlineto{\pgfqpoint{3.318817in}{1.701939in}}%
\pgfpathlineto{\pgfqpoint{3.327141in}{1.710290in}}%
\pgfpathlineto{\pgfqpoint{3.313391in}{1.714453in}}%
\pgfpathlineto{\pgfqpoint{3.299646in}{1.718700in}}%
\pgfpathlineto{\pgfqpoint{3.285906in}{1.723030in}}%
\pgfpathlineto{\pgfqpoint{3.272171in}{1.727445in}}%
\pgfpathlineto{\pgfqpoint{3.263831in}{1.719311in}}%
\pgfpathlineto{\pgfqpoint{3.255483in}{1.711238in}}%
\pgfpathlineto{\pgfqpoint{3.247128in}{1.703227in}}%
\pgfpathlineto{\pgfqpoint{3.238766in}{1.695282in}}%
\pgfpathclose%
\pgfusepath{fill}%
\end{pgfscope}%
\begin{pgfscope}%
\pgfpathrectangle{\pgfqpoint{1.150000in}{0.150000in}}{\pgfqpoint{5.700000in}{5.700000in}}%
\pgfusepath{clip}%
\pgfsetbuttcap%
\pgfsetroundjoin%
\definecolor{currentfill}{rgb}{0.281887,0.150881,0.465405}%
\pgfsetfillcolor{currentfill}%
\pgfsetfillopacity{0.700000}%
\pgfsetlinewidth{0.000000pt}%
\definecolor{currentstroke}{rgb}{0.000000,0.000000,0.000000}%
\pgfsetstrokecolor{currentstroke}%
\pgfsetdash{}{0pt}%
\pgfpathmoveto{\pgfqpoint{2.352488in}{2.000288in}}%
\pgfpathlineto{\pgfqpoint{2.366267in}{1.989107in}}%
\pgfpathlineto{\pgfqpoint{2.380046in}{1.978042in}}%
\pgfpathlineto{\pgfqpoint{2.393823in}{1.967092in}}%
\pgfpathlineto{\pgfqpoint{2.407599in}{1.956256in}}%
\pgfpathlineto{\pgfqpoint{2.416482in}{1.957985in}}%
\pgfpathlineto{\pgfqpoint{2.425349in}{1.959928in}}%
\pgfpathlineto{\pgfqpoint{2.434201in}{1.962080in}}%
\pgfpathlineto{\pgfqpoint{2.443038in}{1.964435in}}%
\pgfpathlineto{\pgfqpoint{2.429295in}{1.974934in}}%
\pgfpathlineto{\pgfqpoint{2.415552in}{1.985548in}}%
\pgfpathlineto{\pgfqpoint{2.401808in}{1.996276in}}%
\pgfpathlineto{\pgfqpoint{2.388063in}{2.007120in}}%
\pgfpathlineto{\pgfqpoint{2.379193in}{2.005093in}}%
\pgfpathlineto{\pgfqpoint{2.370307in}{2.003275in}}%
\pgfpathlineto{\pgfqpoint{2.361406in}{2.001672in}}%
\pgfpathlineto{\pgfqpoint{2.352488in}{2.000288in}}%
\pgfpathclose%
\pgfusepath{fill}%
\end{pgfscope}%
\begin{pgfscope}%
\pgfpathrectangle{\pgfqpoint{1.150000in}{0.150000in}}{\pgfqpoint{5.700000in}{5.700000in}}%
\pgfusepath{clip}%
\pgfsetbuttcap%
\pgfsetroundjoin%
\definecolor{currentfill}{rgb}{0.223925,0.334994,0.548053}%
\pgfsetfillcolor{currentfill}%
\pgfsetfillopacity{0.700000}%
\pgfsetlinewidth{0.000000pt}%
\definecolor{currentstroke}{rgb}{0.000000,0.000000,0.000000}%
\pgfsetstrokecolor{currentstroke}%
\pgfsetdash{}{0pt}%
\pgfpathmoveto{\pgfqpoint{5.219721in}{2.370166in}}%
\pgfpathlineto{\pgfqpoint{5.234074in}{2.373275in}}%
\pgfpathlineto{\pgfqpoint{5.248439in}{2.376455in}}%
\pgfpathlineto{\pgfqpoint{5.262815in}{2.379704in}}%
\pgfpathlineto{\pgfqpoint{5.277204in}{2.383023in}}%
\pgfpathlineto{\pgfqpoint{5.284743in}{2.388438in}}%
\pgfpathlineto{\pgfqpoint{5.292275in}{2.393798in}}%
\pgfpathlineto{\pgfqpoint{5.299799in}{2.399106in}}%
\pgfpathlineto{\pgfqpoint{5.307315in}{2.404367in}}%
\pgfpathlineto{\pgfqpoint{5.292945in}{2.401243in}}%
\pgfpathlineto{\pgfqpoint{5.278586in}{2.398188in}}%
\pgfpathlineto{\pgfqpoint{5.264240in}{2.395203in}}%
\pgfpathlineto{\pgfqpoint{5.249905in}{2.392288in}}%
\pgfpathlineto{\pgfqpoint{5.242370in}{2.386825in}}%
\pgfpathlineto{\pgfqpoint{5.234828in}{2.381320in}}%
\pgfpathlineto{\pgfqpoint{5.227278in}{2.375768in}}%
\pgfpathlineto{\pgfqpoint{5.219721in}{2.370166in}}%
\pgfpathclose%
\pgfusepath{fill}%
\end{pgfscope}%
\begin{pgfscope}%
\pgfpathrectangle{\pgfqpoint{1.150000in}{0.150000in}}{\pgfqpoint{5.700000in}{5.700000in}}%
\pgfusepath{clip}%
\pgfsetbuttcap%
\pgfsetroundjoin%
\definecolor{currentfill}{rgb}{0.280894,0.078907,0.402329}%
\pgfsetfillcolor{currentfill}%
\pgfsetfillopacity{0.700000}%
\pgfsetlinewidth{0.000000pt}%
\definecolor{currentstroke}{rgb}{0.000000,0.000000,0.000000}%
\pgfsetstrokecolor{currentstroke}%
\pgfsetdash{}{0pt}%
\pgfpathmoveto{\pgfqpoint{2.607931in}{1.847005in}}%
\pgfpathlineto{\pgfqpoint{2.621674in}{1.837910in}}%
\pgfpathlineto{\pgfqpoint{2.635418in}{1.828918in}}%
\pgfpathlineto{\pgfqpoint{2.649162in}{1.820028in}}%
\pgfpathlineto{\pgfqpoint{2.662908in}{1.811239in}}%
\pgfpathlineto{\pgfqpoint{2.671609in}{1.815075in}}%
\pgfpathlineto{\pgfqpoint{2.680297in}{1.819084in}}%
\pgfpathlineto{\pgfqpoint{2.688972in}{1.823259in}}%
\pgfpathlineto{\pgfqpoint{2.697635in}{1.827598in}}%
\pgfpathlineto{\pgfqpoint{2.683917in}{1.836075in}}%
\pgfpathlineto{\pgfqpoint{2.670201in}{1.844654in}}%
\pgfpathlineto{\pgfqpoint{2.656486in}{1.853334in}}%
\pgfpathlineto{\pgfqpoint{2.642772in}{1.862118in}}%
\pgfpathlineto{\pgfqpoint{2.634081in}{1.858083in}}%
\pgfpathlineto{\pgfqpoint{2.625377in}{1.854216in}}%
\pgfpathlineto{\pgfqpoint{2.616661in}{1.850522in}}%
\pgfpathlineto{\pgfqpoint{2.607931in}{1.847005in}}%
\pgfpathclose%
\pgfusepath{fill}%
\end{pgfscope}%
\begin{pgfscope}%
\pgfpathrectangle{\pgfqpoint{1.150000in}{0.150000in}}{\pgfqpoint{5.700000in}{5.700000in}}%
\pgfusepath{clip}%
\pgfsetbuttcap%
\pgfsetroundjoin%
\definecolor{currentfill}{rgb}{0.274952,0.037752,0.364543}%
\pgfsetfillcolor{currentfill}%
\pgfsetfillopacity{0.700000}%
\pgfsetlinewidth{0.000000pt}%
\definecolor{currentstroke}{rgb}{0.000000,0.000000,0.000000}%
\pgfsetstrokecolor{currentstroke}%
\pgfsetdash{}{0pt}%
\pgfpathmoveto{\pgfqpoint{2.807430in}{1.763353in}}%
\pgfpathlineto{\pgfqpoint{2.821163in}{1.755760in}}%
\pgfpathlineto{\pgfqpoint{2.834898in}{1.748263in}}%
\pgfpathlineto{\pgfqpoint{2.848635in}{1.740860in}}%
\pgfpathlineto{\pgfqpoint{2.862374in}{1.733551in}}%
\pgfpathlineto{\pgfqpoint{2.870949in}{1.738939in}}%
\pgfpathlineto{\pgfqpoint{2.879512in}{1.744465in}}%
\pgfpathlineto{\pgfqpoint{2.888065in}{1.750124in}}%
\pgfpathlineto{\pgfqpoint{2.896607in}{1.755913in}}%
\pgfpathlineto{\pgfqpoint{2.882892in}{1.762933in}}%
\pgfpathlineto{\pgfqpoint{2.869179in}{1.770047in}}%
\pgfpathlineto{\pgfqpoint{2.855469in}{1.777256in}}%
\pgfpathlineto{\pgfqpoint{2.841760in}{1.784561in}}%
\pgfpathlineto{\pgfqpoint{2.833194in}{1.779052in}}%
\pgfpathlineto{\pgfqpoint{2.824617in}{1.773679in}}%
\pgfpathlineto{\pgfqpoint{2.816029in}{1.768444in}}%
\pgfpathlineto{\pgfqpoint{2.807430in}{1.763353in}}%
\pgfpathclose%
\pgfusepath{fill}%
\end{pgfscope}%
\begin{pgfscope}%
\pgfpathrectangle{\pgfqpoint{1.150000in}{0.150000in}}{\pgfqpoint{5.700000in}{5.700000in}}%
\pgfusepath{clip}%
\pgfsetbuttcap%
\pgfsetroundjoin%
\definecolor{currentfill}{rgb}{0.278012,0.180367,0.486697}%
\pgfsetfillcolor{currentfill}%
\pgfsetfillopacity{0.700000}%
\pgfsetlinewidth{0.000000pt}%
\definecolor{currentstroke}{rgb}{0.000000,0.000000,0.000000}%
\pgfsetstrokecolor{currentstroke}%
\pgfsetdash{}{0pt}%
\pgfpathmoveto{\pgfqpoint{4.372961in}{2.014088in}}%
\pgfpathlineto{\pgfqpoint{4.386994in}{2.015136in}}%
\pgfpathlineto{\pgfqpoint{4.401037in}{2.016258in}}%
\pgfpathlineto{\pgfqpoint{4.415090in}{2.017452in}}%
\pgfpathlineto{\pgfqpoint{4.429152in}{2.018718in}}%
\pgfpathlineto{\pgfqpoint{4.437077in}{2.027732in}}%
\pgfpathlineto{\pgfqpoint{4.444995in}{2.036671in}}%
\pgfpathlineto{\pgfqpoint{4.452906in}{2.045537in}}%
\pgfpathlineto{\pgfqpoint{4.460812in}{2.054330in}}%
\pgfpathlineto{\pgfqpoint{4.446759in}{2.053066in}}%
\pgfpathlineto{\pgfqpoint{4.432717in}{2.051874in}}%
\pgfpathlineto{\pgfqpoint{4.418684in}{2.050755in}}%
\pgfpathlineto{\pgfqpoint{4.404660in}{2.049709in}}%
\pgfpathlineto{\pgfqpoint{4.396745in}{2.040905in}}%
\pgfpathlineto{\pgfqpoint{4.388823in}{2.032035in}}%
\pgfpathlineto{\pgfqpoint{4.380895in}{2.023096in}}%
\pgfpathlineto{\pgfqpoint{4.372961in}{2.014088in}}%
\pgfpathclose%
\pgfusepath{fill}%
\end{pgfscope}%
\begin{pgfscope}%
\pgfpathrectangle{\pgfqpoint{1.150000in}{0.150000in}}{\pgfqpoint{5.700000in}{5.700000in}}%
\pgfusepath{clip}%
\pgfsetbuttcap%
\pgfsetroundjoin%
\definecolor{currentfill}{rgb}{0.277018,0.050344,0.375715}%
\pgfsetfillcolor{currentfill}%
\pgfsetfillopacity{0.700000}%
\pgfsetlinewidth{0.000000pt}%
\definecolor{currentstroke}{rgb}{0.000000,0.000000,0.000000}%
\pgfsetstrokecolor{currentstroke}%
\pgfsetdash{}{0pt}%
\pgfpathmoveto{\pgfqpoint{3.701855in}{1.758624in}}%
\pgfpathlineto{\pgfqpoint{3.715691in}{1.756702in}}%
\pgfpathlineto{\pgfqpoint{3.729534in}{1.754858in}}%
\pgfpathlineto{\pgfqpoint{3.743384in}{1.753092in}}%
\pgfpathlineto{\pgfqpoint{3.757241in}{1.751402in}}%
\pgfpathlineto{\pgfqpoint{3.765404in}{1.761121in}}%
\pgfpathlineto{\pgfqpoint{3.773561in}{1.770828in}}%
\pgfpathlineto{\pgfqpoint{3.781712in}{1.780523in}}%
\pgfpathlineto{\pgfqpoint{3.789858in}{1.790202in}}%
\pgfpathlineto{\pgfqpoint{3.776012in}{1.791748in}}%
\pgfpathlineto{\pgfqpoint{3.762173in}{1.793371in}}%
\pgfpathlineto{\pgfqpoint{3.748341in}{1.795072in}}%
\pgfpathlineto{\pgfqpoint{3.734516in}{1.796851in}}%
\pgfpathlineto{\pgfqpoint{3.726360in}{1.787307in}}%
\pgfpathlineto{\pgfqpoint{3.718197in}{1.777753in}}%
\pgfpathlineto{\pgfqpoint{3.710029in}{1.768192in}}%
\pgfpathlineto{\pgfqpoint{3.701855in}{1.758624in}}%
\pgfpathclose%
\pgfusepath{fill}%
\end{pgfscope}%
\begin{pgfscope}%
\pgfpathrectangle{\pgfqpoint{1.150000in}{0.150000in}}{\pgfqpoint{5.700000in}{5.700000in}}%
\pgfusepath{clip}%
\pgfsetbuttcap%
\pgfsetroundjoin%
\definecolor{currentfill}{rgb}{0.229739,0.322361,0.545706}%
\pgfsetfillcolor{currentfill}%
\pgfsetfillopacity{0.700000}%
\pgfsetlinewidth{0.000000pt}%
\definecolor{currentstroke}{rgb}{0.000000,0.000000,0.000000}%
\pgfsetstrokecolor{currentstroke}%
\pgfsetdash{}{0pt}%
\pgfpathmoveto{\pgfqpoint{5.132061in}{2.334757in}}%
\pgfpathlineto{\pgfqpoint{5.146383in}{2.337760in}}%
\pgfpathlineto{\pgfqpoint{5.160716in}{2.340834in}}%
\pgfpathlineto{\pgfqpoint{5.175062in}{2.343977in}}%
\pgfpathlineto{\pgfqpoint{5.189419in}{2.347191in}}%
\pgfpathlineto{\pgfqpoint{5.197006in}{2.353026in}}%
\pgfpathlineto{\pgfqpoint{5.204586in}{2.358798in}}%
\pgfpathlineto{\pgfqpoint{5.212157in}{2.364510in}}%
\pgfpathlineto{\pgfqpoint{5.219721in}{2.370166in}}%
\pgfpathlineto{\pgfqpoint{5.205381in}{2.367126in}}%
\pgfpathlineto{\pgfqpoint{5.191052in}{2.364156in}}%
\pgfpathlineto{\pgfqpoint{5.176735in}{2.361256in}}%
\pgfpathlineto{\pgfqpoint{5.162430in}{2.358426in}}%
\pgfpathlineto{\pgfqpoint{5.154848in}{2.352589in}}%
\pgfpathlineto{\pgfqpoint{5.147260in}{2.346701in}}%
\pgfpathlineto{\pgfqpoint{5.139664in}{2.340758in}}%
\pgfpathlineto{\pgfqpoint{5.132061in}{2.334757in}}%
\pgfpathclose%
\pgfusepath{fill}%
\end{pgfscope}%
\begin{pgfscope}%
\pgfpathrectangle{\pgfqpoint{1.150000in}{0.150000in}}{\pgfqpoint{5.700000in}{5.700000in}}%
\pgfusepath{clip}%
\pgfsetbuttcap%
\pgfsetroundjoin%
\definecolor{currentfill}{rgb}{0.280868,0.160771,0.472899}%
\pgfsetfillcolor{currentfill}%
\pgfsetfillopacity{0.700000}%
\pgfsetlinewidth{0.000000pt}%
\definecolor{currentstroke}{rgb}{0.000000,0.000000,0.000000}%
\pgfsetstrokecolor{currentstroke}%
\pgfsetdash{}{0pt}%
\pgfpathmoveto{\pgfqpoint{4.285088in}{1.973976in}}%
\pgfpathlineto{\pgfqpoint{4.299093in}{1.974713in}}%
\pgfpathlineto{\pgfqpoint{4.313107in}{1.975524in}}%
\pgfpathlineto{\pgfqpoint{4.327131in}{1.976407in}}%
\pgfpathlineto{\pgfqpoint{4.341165in}{1.977364in}}%
\pgfpathlineto{\pgfqpoint{4.349123in}{1.986649in}}%
\pgfpathlineto{\pgfqpoint{4.357075in}{1.995865in}}%
\pgfpathlineto{\pgfqpoint{4.365021in}{2.005011in}}%
\pgfpathlineto{\pgfqpoint{4.372961in}{2.014088in}}%
\pgfpathlineto{\pgfqpoint{4.358937in}{2.013112in}}%
\pgfpathlineto{\pgfqpoint{4.344923in}{2.012210in}}%
\pgfpathlineto{\pgfqpoint{4.330918in}{2.011381in}}%
\pgfpathlineto{\pgfqpoint{4.316923in}{2.010625in}}%
\pgfpathlineto{\pgfqpoint{4.308973in}{2.001559in}}%
\pgfpathlineto{\pgfqpoint{4.301017in}{1.992429in}}%
\pgfpathlineto{\pgfqpoint{4.293055in}{1.983235in}}%
\pgfpathlineto{\pgfqpoint{4.285088in}{1.973976in}}%
\pgfpathclose%
\pgfusepath{fill}%
\end{pgfscope}%
\begin{pgfscope}%
\pgfpathrectangle{\pgfqpoint{1.150000in}{0.150000in}}{\pgfqpoint{5.700000in}{5.700000in}}%
\pgfusepath{clip}%
\pgfsetbuttcap%
\pgfsetroundjoin%
\definecolor{currentfill}{rgb}{0.197636,0.391528,0.554969}%
\pgfsetfillcolor{currentfill}%
\pgfsetfillopacity{0.700000}%
\pgfsetlinewidth{0.000000pt}%
\definecolor{currentstroke}{rgb}{0.000000,0.000000,0.000000}%
\pgfsetstrokecolor{currentstroke}%
\pgfsetdash{}{0pt}%
\pgfpathmoveto{\pgfqpoint{5.627583in}{2.513212in}}%
\pgfpathlineto{\pgfqpoint{5.642102in}{2.516798in}}%
\pgfpathlineto{\pgfqpoint{5.656634in}{2.520453in}}%
\pgfpathlineto{\pgfqpoint{5.671179in}{2.524176in}}%
\pgfpathlineto{\pgfqpoint{5.685736in}{2.527969in}}%
\pgfpathlineto{\pgfqpoint{5.693052in}{2.531618in}}%
\pgfpathlineto{\pgfqpoint{5.700360in}{2.535253in}}%
\pgfpathlineto{\pgfqpoint{5.707662in}{2.538881in}}%
\pgfpathlineto{\pgfqpoint{5.714957in}{2.542505in}}%
\pgfpathlineto{\pgfqpoint{5.700424in}{2.538994in}}%
\pgfpathlineto{\pgfqpoint{5.685903in}{2.535552in}}%
\pgfpathlineto{\pgfqpoint{5.671395in}{2.532178in}}%
\pgfpathlineto{\pgfqpoint{5.656900in}{2.528873in}}%
\pgfpathlineto{\pgfqpoint{5.649581in}{2.524960in}}%
\pgfpathlineto{\pgfqpoint{5.642255in}{2.521049in}}%
\pgfpathlineto{\pgfqpoint{5.634922in}{2.517135in}}%
\pgfpathlineto{\pgfqpoint{5.627583in}{2.513212in}}%
\pgfpathclose%
\pgfusepath{fill}%
\end{pgfscope}%
\begin{pgfscope}%
\pgfpathrectangle{\pgfqpoint{1.150000in}{0.150000in}}{\pgfqpoint{5.700000in}{5.700000in}}%
\pgfusepath{clip}%
\pgfsetbuttcap%
\pgfsetroundjoin%
\definecolor{currentfill}{rgb}{0.268510,0.009605,0.335427}%
\pgfsetfillcolor{currentfill}%
\pgfsetfillopacity{0.700000}%
\pgfsetlinewidth{0.000000pt}%
\definecolor{currentstroke}{rgb}{0.000000,0.000000,0.000000}%
\pgfsetstrokecolor{currentstroke}%
\pgfsetdash{}{0pt}%
\pgfpathmoveto{\pgfqpoint{3.382192in}{1.694469in}}%
\pgfpathlineto{\pgfqpoint{3.395968in}{1.690720in}}%
\pgfpathlineto{\pgfqpoint{3.409749in}{1.687052in}}%
\pgfpathlineto{\pgfqpoint{3.423537in}{1.683466in}}%
\pgfpathlineto{\pgfqpoint{3.437329in}{1.679961in}}%
\pgfpathlineto{\pgfqpoint{3.445616in}{1.688788in}}%
\pgfpathlineto{\pgfqpoint{3.453896in}{1.697653in}}%
\pgfpathlineto{\pgfqpoint{3.462169in}{1.706554in}}%
\pgfpathlineto{\pgfqpoint{3.470436in}{1.715488in}}%
\pgfpathlineto{\pgfqpoint{3.456658in}{1.718789in}}%
\pgfpathlineto{\pgfqpoint{3.442885in}{1.722170in}}%
\pgfpathlineto{\pgfqpoint{3.429118in}{1.725634in}}%
\pgfpathlineto{\pgfqpoint{3.415357in}{1.729178in}}%
\pgfpathlineto{\pgfqpoint{3.407076in}{1.720441in}}%
\pgfpathlineto{\pgfqpoint{3.398788in}{1.711742in}}%
\pgfpathlineto{\pgfqpoint{3.390493in}{1.703084in}}%
\pgfpathlineto{\pgfqpoint{3.382192in}{1.694469in}}%
\pgfpathclose%
\pgfusepath{fill}%
\end{pgfscope}%
\begin{pgfscope}%
\pgfpathrectangle{\pgfqpoint{1.150000in}{0.150000in}}{\pgfqpoint{5.700000in}{5.700000in}}%
\pgfusepath{clip}%
\pgfsetbuttcap%
\pgfsetroundjoin%
\definecolor{currentfill}{rgb}{0.282623,0.140926,0.457517}%
\pgfsetfillcolor{currentfill}%
\pgfsetfillopacity{0.700000}%
\pgfsetlinewidth{0.000000pt}%
\definecolor{currentstroke}{rgb}{0.000000,0.000000,0.000000}%
\pgfsetstrokecolor{currentstroke}%
\pgfsetdash{}{0pt}%
\pgfpathmoveto{\pgfqpoint{4.197192in}{1.934243in}}%
\pgfpathlineto{\pgfqpoint{4.211170in}{1.934646in}}%
\pgfpathlineto{\pgfqpoint{4.225157in}{1.935123in}}%
\pgfpathlineto{\pgfqpoint{4.239153in}{1.935673in}}%
\pgfpathlineto{\pgfqpoint{4.253158in}{1.936297in}}%
\pgfpathlineto{\pgfqpoint{4.261149in}{1.945813in}}%
\pgfpathlineto{\pgfqpoint{4.269135in}{1.955265in}}%
\pgfpathlineto{\pgfqpoint{4.277114in}{1.964653in}}%
\pgfpathlineto{\pgfqpoint{4.285088in}{1.973976in}}%
\pgfpathlineto{\pgfqpoint{4.271092in}{1.973312in}}%
\pgfpathlineto{\pgfqpoint{4.257105in}{1.972722in}}%
\pgfpathlineto{\pgfqpoint{4.243128in}{1.972206in}}%
\pgfpathlineto{\pgfqpoint{4.229159in}{1.971763in}}%
\pgfpathlineto{\pgfqpoint{4.221176in}{1.962472in}}%
\pgfpathlineto{\pgfqpoint{4.213187in}{1.953121in}}%
\pgfpathlineto{\pgfqpoint{4.205192in}{1.943711in}}%
\pgfpathlineto{\pgfqpoint{4.197192in}{1.934243in}}%
\pgfpathclose%
\pgfusepath{fill}%
\end{pgfscope}%
\begin{pgfscope}%
\pgfpathrectangle{\pgfqpoint{1.150000in}{0.150000in}}{\pgfqpoint{5.700000in}{5.700000in}}%
\pgfusepath{clip}%
\pgfsetbuttcap%
\pgfsetroundjoin%
\definecolor{currentfill}{rgb}{0.282884,0.135920,0.453427}%
\pgfsetfillcolor{currentfill}%
\pgfsetfillopacity{0.700000}%
\pgfsetlinewidth{0.000000pt}%
\definecolor{currentstroke}{rgb}{0.000000,0.000000,0.000000}%
\pgfsetstrokecolor{currentstroke}%
\pgfsetdash{}{0pt}%
\pgfpathmoveto{\pgfqpoint{2.407599in}{1.956256in}}%
\pgfpathlineto{\pgfqpoint{2.421374in}{1.945533in}}%
\pgfpathlineto{\pgfqpoint{2.435148in}{1.934922in}}%
\pgfpathlineto{\pgfqpoint{2.448922in}{1.924423in}}%
\pgfpathlineto{\pgfqpoint{2.462695in}{1.914034in}}%
\pgfpathlineto{\pgfqpoint{2.471545in}{1.916107in}}%
\pgfpathlineto{\pgfqpoint{2.480379in}{1.918388in}}%
\pgfpathlineto{\pgfqpoint{2.489198in}{1.920873in}}%
\pgfpathlineto{\pgfqpoint{2.498003in}{1.923555in}}%
\pgfpathlineto{\pgfqpoint{2.484262in}{1.933609in}}%
\pgfpathlineto{\pgfqpoint{2.470521in}{1.943772in}}%
\pgfpathlineto{\pgfqpoint{2.456780in}{1.954048in}}%
\pgfpathlineto{\pgfqpoint{2.443038in}{1.964435in}}%
\pgfpathlineto{\pgfqpoint{2.434201in}{1.962080in}}%
\pgfpathlineto{\pgfqpoint{2.425349in}{1.959928in}}%
\pgfpathlineto{\pgfqpoint{2.416482in}{1.957985in}}%
\pgfpathlineto{\pgfqpoint{2.407599in}{1.956256in}}%
\pgfpathclose%
\pgfusepath{fill}%
\end{pgfscope}%
\begin{pgfscope}%
\pgfpathrectangle{\pgfqpoint{1.150000in}{0.150000in}}{\pgfqpoint{5.700000in}{5.700000in}}%
\pgfusepath{clip}%
\pgfsetbuttcap%
\pgfsetroundjoin%
\definecolor{currentfill}{rgb}{0.243113,0.292092,0.538516}%
\pgfsetfillcolor{currentfill}%
\pgfsetfillopacity{0.700000}%
\pgfsetlinewidth{0.000000pt}%
\definecolor{currentstroke}{rgb}{0.000000,0.000000,0.000000}%
\pgfsetstrokecolor{currentstroke}%
\pgfsetdash{}{0pt}%
\pgfpathmoveto{\pgfqpoint{1.984281in}{2.311433in}}%
\pgfpathlineto{\pgfqpoint{1.998172in}{2.296796in}}%
\pgfpathlineto{\pgfqpoint{2.012059in}{2.282302in}}%
\pgfpathlineto{\pgfqpoint{2.025942in}{2.267949in}}%
\pgfpathlineto{\pgfqpoint{2.039819in}{2.253737in}}%
\pgfpathlineto{\pgfqpoint{2.049006in}{2.252332in}}%
\pgfpathlineto{\pgfqpoint{2.058171in}{2.251198in}}%
\pgfpathlineto{\pgfqpoint{2.067317in}{2.250327in}}%
\pgfpathlineto{\pgfqpoint{2.076443in}{2.249714in}}%
\pgfpathlineto{\pgfqpoint{2.062607in}{2.263560in}}%
\pgfpathlineto{\pgfqpoint{2.048767in}{2.277545in}}%
\pgfpathlineto{\pgfqpoint{2.034923in}{2.291671in}}%
\pgfpathlineto{\pgfqpoint{2.021074in}{2.305939in}}%
\pgfpathlineto{\pgfqpoint{2.011907in}{2.306911in}}%
\pgfpathlineto{\pgfqpoint{2.002719in}{2.308147in}}%
\pgfpathlineto{\pgfqpoint{1.993510in}{2.309652in}}%
\pgfpathlineto{\pgfqpoint{1.984281in}{2.311433in}}%
\pgfpathclose%
\pgfusepath{fill}%
\end{pgfscope}%
\begin{pgfscope}%
\pgfpathrectangle{\pgfqpoint{1.150000in}{0.150000in}}{\pgfqpoint{5.700000in}{5.700000in}}%
\pgfusepath{clip}%
\pgfsetbuttcap%
\pgfsetroundjoin%
\definecolor{currentfill}{rgb}{0.235526,0.309527,0.542944}%
\pgfsetfillcolor{currentfill}%
\pgfsetfillopacity{0.700000}%
\pgfsetlinewidth{0.000000pt}%
\definecolor{currentstroke}{rgb}{0.000000,0.000000,0.000000}%
\pgfsetstrokecolor{currentstroke}%
\pgfsetdash{}{0pt}%
\pgfpathmoveto{\pgfqpoint{5.044340in}{2.298191in}}%
\pgfpathlineto{\pgfqpoint{5.058631in}{2.301065in}}%
\pgfpathlineto{\pgfqpoint{5.072933in}{2.304010in}}%
\pgfpathlineto{\pgfqpoint{5.087247in}{2.307025in}}%
\pgfpathlineto{\pgfqpoint{5.101573in}{2.310110in}}%
\pgfpathlineto{\pgfqpoint{5.109206in}{2.316374in}}%
\pgfpathlineto{\pgfqpoint{5.116832in}{2.322568in}}%
\pgfpathlineto{\pgfqpoint{5.124450in}{2.328695in}}%
\pgfpathlineto{\pgfqpoint{5.132061in}{2.334757in}}%
\pgfpathlineto{\pgfqpoint{5.117751in}{2.331824in}}%
\pgfpathlineto{\pgfqpoint{5.103452in}{2.328961in}}%
\pgfpathlineto{\pgfqpoint{5.089165in}{2.326168in}}%
\pgfpathlineto{\pgfqpoint{5.074890in}{2.323446in}}%
\pgfpathlineto{\pgfqpoint{5.067263in}{2.317224in}}%
\pgfpathlineto{\pgfqpoint{5.059630in}{2.310942in}}%
\pgfpathlineto{\pgfqpoint{5.051988in}{2.304599in}}%
\pgfpathlineto{\pgfqpoint{5.044340in}{2.298191in}}%
\pgfpathclose%
\pgfusepath{fill}%
\end{pgfscope}%
\begin{pgfscope}%
\pgfpathrectangle{\pgfqpoint{1.150000in}{0.150000in}}{\pgfqpoint{5.700000in}{5.700000in}}%
\pgfusepath{clip}%
\pgfsetbuttcap%
\pgfsetroundjoin%
\definecolor{currentfill}{rgb}{0.283187,0.125848,0.444960}%
\pgfsetfillcolor{currentfill}%
\pgfsetfillopacity{0.700000}%
\pgfsetlinewidth{0.000000pt}%
\definecolor{currentstroke}{rgb}{0.000000,0.000000,0.000000}%
\pgfsetstrokecolor{currentstroke}%
\pgfsetdash{}{0pt}%
\pgfpathmoveto{\pgfqpoint{4.109272in}{1.895156in}}%
\pgfpathlineto{\pgfqpoint{4.123224in}{1.895202in}}%
\pgfpathlineto{\pgfqpoint{4.137184in}{1.895323in}}%
\pgfpathlineto{\pgfqpoint{4.151154in}{1.895517in}}%
\pgfpathlineto{\pgfqpoint{4.165132in}{1.895786in}}%
\pgfpathlineto{\pgfqpoint{4.173156in}{1.905486in}}%
\pgfpathlineto{\pgfqpoint{4.181173in}{1.915130in}}%
\pgfpathlineto{\pgfqpoint{4.189186in}{1.924715in}}%
\pgfpathlineto{\pgfqpoint{4.197192in}{1.934243in}}%
\pgfpathlineto{\pgfqpoint{4.183223in}{1.933914in}}%
\pgfpathlineto{\pgfqpoint{4.169263in}{1.933658in}}%
\pgfpathlineto{\pgfqpoint{4.155312in}{1.933477in}}%
\pgfpathlineto{\pgfqpoint{4.141369in}{1.933371in}}%
\pgfpathlineto{\pgfqpoint{4.133353in}{1.923896in}}%
\pgfpathlineto{\pgfqpoint{4.125332in}{1.914369in}}%
\pgfpathlineto{\pgfqpoint{4.117304in}{1.904788in}}%
\pgfpathlineto{\pgfqpoint{4.109272in}{1.895156in}}%
\pgfpathclose%
\pgfusepath{fill}%
\end{pgfscope}%
\begin{pgfscope}%
\pgfpathrectangle{\pgfqpoint{1.150000in}{0.150000in}}{\pgfqpoint{5.700000in}{5.700000in}}%
\pgfusepath{clip}%
\pgfsetbuttcap%
\pgfsetroundjoin%
\definecolor{currentfill}{rgb}{0.273809,0.031497,0.358853}%
\pgfsetfillcolor{currentfill}%
\pgfsetfillopacity{0.700000}%
\pgfsetlinewidth{0.000000pt}%
\definecolor{currentstroke}{rgb}{0.000000,0.000000,0.000000}%
\pgfsetstrokecolor{currentstroke}%
\pgfsetdash{}{0pt}%
\pgfpathmoveto{\pgfqpoint{3.613777in}{1.729454in}}%
\pgfpathlineto{\pgfqpoint{3.627598in}{1.727056in}}%
\pgfpathlineto{\pgfqpoint{3.641425in}{1.724736in}}%
\pgfpathlineto{\pgfqpoint{3.655259in}{1.722495in}}%
\pgfpathlineto{\pgfqpoint{3.669100in}{1.720332in}}%
\pgfpathlineto{\pgfqpoint{3.677298in}{1.729904in}}%
\pgfpathlineto{\pgfqpoint{3.685489in}{1.739478in}}%
\pgfpathlineto{\pgfqpoint{3.693675in}{1.749052in}}%
\pgfpathlineto{\pgfqpoint{3.701855in}{1.758624in}}%
\pgfpathlineto{\pgfqpoint{3.688026in}{1.760624in}}%
\pgfpathlineto{\pgfqpoint{3.674203in}{1.762701in}}%
\pgfpathlineto{\pgfqpoint{3.660388in}{1.764857in}}%
\pgfpathlineto{\pgfqpoint{3.646579in}{1.767092in}}%
\pgfpathlineto{\pgfqpoint{3.638387in}{1.757676in}}%
\pgfpathlineto{\pgfqpoint{3.630190in}{1.748263in}}%
\pgfpathlineto{\pgfqpoint{3.621986in}{1.738855in}}%
\pgfpathlineto{\pgfqpoint{3.613777in}{1.729454in}}%
\pgfpathclose%
\pgfusepath{fill}%
\end{pgfscope}%
\begin{pgfscope}%
\pgfpathrectangle{\pgfqpoint{1.150000in}{0.150000in}}{\pgfqpoint{5.700000in}{5.700000in}}%
\pgfusepath{clip}%
\pgfsetbuttcap%
\pgfsetroundjoin%
\definecolor{currentfill}{rgb}{0.279566,0.067836,0.391917}%
\pgfsetfillcolor{currentfill}%
\pgfsetfillopacity{0.700000}%
\pgfsetlinewidth{0.000000pt}%
\definecolor{currentstroke}{rgb}{0.000000,0.000000,0.000000}%
\pgfsetstrokecolor{currentstroke}%
\pgfsetdash{}{0pt}%
\pgfpathmoveto{\pgfqpoint{2.662908in}{1.811239in}}%
\pgfpathlineto{\pgfqpoint{2.676655in}{1.802552in}}%
\pgfpathlineto{\pgfqpoint{2.690403in}{1.793964in}}%
\pgfpathlineto{\pgfqpoint{2.704152in}{1.785477in}}%
\pgfpathlineto{\pgfqpoint{2.717902in}{1.777088in}}%
\pgfpathlineto{\pgfqpoint{2.726575in}{1.781243in}}%
\pgfpathlineto{\pgfqpoint{2.735235in}{1.785564in}}%
\pgfpathlineto{\pgfqpoint{2.743883in}{1.790048in}}%
\pgfpathlineto{\pgfqpoint{2.752519in}{1.794690in}}%
\pgfpathlineto{\pgfqpoint{2.738796in}{1.802768in}}%
\pgfpathlineto{\pgfqpoint{2.725074in}{1.810945in}}%
\pgfpathlineto{\pgfqpoint{2.711354in}{1.819221in}}%
\pgfpathlineto{\pgfqpoint{2.697635in}{1.827598in}}%
\pgfpathlineto{\pgfqpoint{2.688972in}{1.823259in}}%
\pgfpathlineto{\pgfqpoint{2.680297in}{1.819084in}}%
\pgfpathlineto{\pgfqpoint{2.671609in}{1.815075in}}%
\pgfpathlineto{\pgfqpoint{2.662908in}{1.811239in}}%
\pgfpathclose%
\pgfusepath{fill}%
\end{pgfscope}%
\begin{pgfscope}%
\pgfpathrectangle{\pgfqpoint{1.150000in}{0.150000in}}{\pgfqpoint{5.700000in}{5.700000in}}%
\pgfusepath{clip}%
\pgfsetbuttcap%
\pgfsetroundjoin%
\definecolor{currentfill}{rgb}{0.269944,0.014625,0.341379}%
\pgfsetfillcolor{currentfill}%
\pgfsetfillopacity{0.700000}%
\pgfsetlinewidth{0.000000pt}%
\definecolor{currentstroke}{rgb}{0.000000,0.000000,0.000000}%
\pgfsetstrokecolor{currentstroke}%
\pgfsetdash{}{0pt}%
\pgfpathmoveto{\pgfqpoint{3.006433in}{1.703079in}}%
\pgfpathlineto{\pgfqpoint{3.020175in}{1.696884in}}%
\pgfpathlineto{\pgfqpoint{3.033921in}{1.690778in}}%
\pgfpathlineto{\pgfqpoint{3.047670in}{1.684760in}}%
\pgfpathlineto{\pgfqpoint{3.061422in}{1.678831in}}%
\pgfpathlineto{\pgfqpoint{3.069889in}{1.685570in}}%
\pgfpathlineto{\pgfqpoint{3.078347in}{1.692414in}}%
\pgfpathlineto{\pgfqpoint{3.086796in}{1.699360in}}%
\pgfpathlineto{\pgfqpoint{3.095236in}{1.706404in}}%
\pgfpathlineto{\pgfqpoint{3.081503in}{1.712067in}}%
\pgfpathlineto{\pgfqpoint{3.067775in}{1.717817in}}%
\pgfpathlineto{\pgfqpoint{3.054050in}{1.723657in}}%
\pgfpathlineto{\pgfqpoint{3.040329in}{1.729586in}}%
\pgfpathlineto{\pgfqpoint{3.031869in}{1.722801in}}%
\pgfpathlineto{\pgfqpoint{3.023399in}{1.716119in}}%
\pgfpathlineto{\pgfqpoint{3.014921in}{1.709544in}}%
\pgfpathlineto{\pgfqpoint{3.006433in}{1.703079in}}%
\pgfpathclose%
\pgfusepath{fill}%
\end{pgfscope}%
\begin{pgfscope}%
\pgfpathrectangle{\pgfqpoint{1.150000in}{0.150000in}}{\pgfqpoint{5.700000in}{5.700000in}}%
\pgfusepath{clip}%
\pgfsetbuttcap%
\pgfsetroundjoin%
\definecolor{currentfill}{rgb}{0.252194,0.269783,0.531579}%
\pgfsetfillcolor{currentfill}%
\pgfsetfillopacity{0.700000}%
\pgfsetlinewidth{0.000000pt}%
\definecolor{currentstroke}{rgb}{0.000000,0.000000,0.000000}%
\pgfsetstrokecolor{currentstroke}%
\pgfsetdash{}{0pt}%
\pgfpathmoveto{\pgfqpoint{2.039819in}{2.253737in}}%
\pgfpathlineto{\pgfqpoint{2.053693in}{2.239663in}}%
\pgfpathlineto{\pgfqpoint{2.067562in}{2.225726in}}%
\pgfpathlineto{\pgfqpoint{2.081427in}{2.211926in}}%
\pgfpathlineto{\pgfqpoint{2.095288in}{2.198261in}}%
\pgfpathlineto{\pgfqpoint{2.104432in}{2.197231in}}%
\pgfpathlineto{\pgfqpoint{2.113556in}{2.196466in}}%
\pgfpathlineto{\pgfqpoint{2.122660in}{2.195958in}}%
\pgfpathlineto{\pgfqpoint{2.131746in}{2.195703in}}%
\pgfpathlineto{\pgfqpoint{2.117926in}{2.209003in}}%
\pgfpathlineto{\pgfqpoint{2.104102in}{2.222437in}}%
\pgfpathlineto{\pgfqpoint{2.090274in}{2.236007in}}%
\pgfpathlineto{\pgfqpoint{2.076443in}{2.249714in}}%
\pgfpathlineto{\pgfqpoint{2.067317in}{2.250327in}}%
\pgfpathlineto{\pgfqpoint{2.058171in}{2.251198in}}%
\pgfpathlineto{\pgfqpoint{2.049006in}{2.252332in}}%
\pgfpathlineto{\pgfqpoint{2.039819in}{2.253737in}}%
\pgfpathclose%
\pgfusepath{fill}%
\end{pgfscope}%
\begin{pgfscope}%
\pgfpathrectangle{\pgfqpoint{1.150000in}{0.150000in}}{\pgfqpoint{5.700000in}{5.700000in}}%
\pgfusepath{clip}%
\pgfsetbuttcap%
\pgfsetroundjoin%
\definecolor{currentfill}{rgb}{0.243113,0.292092,0.538516}%
\pgfsetfillcolor{currentfill}%
\pgfsetfillopacity{0.700000}%
\pgfsetlinewidth{0.000000pt}%
\definecolor{currentstroke}{rgb}{0.000000,0.000000,0.000000}%
\pgfsetstrokecolor{currentstroke}%
\pgfsetdash{}{0pt}%
\pgfpathmoveto{\pgfqpoint{4.956566in}{2.260540in}}%
\pgfpathlineto{\pgfqpoint{4.970825in}{2.263263in}}%
\pgfpathlineto{\pgfqpoint{4.985096in}{2.266057in}}%
\pgfpathlineto{\pgfqpoint{4.999378in}{2.268921in}}%
\pgfpathlineto{\pgfqpoint{5.013672in}{2.271855in}}%
\pgfpathlineto{\pgfqpoint{5.021350in}{2.278550in}}%
\pgfpathlineto{\pgfqpoint{5.029021in}{2.285169in}}%
\pgfpathlineto{\pgfqpoint{5.036684in}{2.291715in}}%
\pgfpathlineto{\pgfqpoint{5.044340in}{2.298191in}}%
\pgfpathlineto{\pgfqpoint{5.030061in}{2.295387in}}%
\pgfpathlineto{\pgfqpoint{5.015793in}{2.292654in}}%
\pgfpathlineto{\pgfqpoint{5.001537in}{2.289991in}}%
\pgfpathlineto{\pgfqpoint{4.987292in}{2.287398in}}%
\pgfpathlineto{\pgfqpoint{4.979621in}{2.280784in}}%
\pgfpathlineto{\pgfqpoint{4.971943in}{2.274105in}}%
\pgfpathlineto{\pgfqpoint{4.964258in}{2.267358in}}%
\pgfpathlineto{\pgfqpoint{4.956566in}{2.260540in}}%
\pgfpathclose%
\pgfusepath{fill}%
\end{pgfscope}%
\begin{pgfscope}%
\pgfpathrectangle{\pgfqpoint{1.150000in}{0.150000in}}{\pgfqpoint{5.700000in}{5.700000in}}%
\pgfusepath{clip}%
\pgfsetbuttcap%
\pgfsetroundjoin%
\definecolor{currentfill}{rgb}{0.203063,0.379716,0.553925}%
\pgfsetfillcolor{currentfill}%
\pgfsetfillopacity{0.700000}%
\pgfsetlinewidth{0.000000pt}%
\definecolor{currentstroke}{rgb}{0.000000,0.000000,0.000000}%
\pgfsetstrokecolor{currentstroke}%
\pgfsetdash{}{0pt}%
\pgfpathmoveto{\pgfqpoint{5.540115in}{2.482646in}}%
\pgfpathlineto{\pgfqpoint{5.554606in}{2.486217in}}%
\pgfpathlineto{\pgfqpoint{5.569110in}{2.489856in}}%
\pgfpathlineto{\pgfqpoint{5.583626in}{2.493565in}}%
\pgfpathlineto{\pgfqpoint{5.598155in}{2.497342in}}%
\pgfpathlineto{\pgfqpoint{5.605523in}{2.501346in}}%
\pgfpathlineto{\pgfqpoint{5.612883in}{2.505322in}}%
\pgfpathlineto{\pgfqpoint{5.620237in}{2.509276in}}%
\pgfpathlineto{\pgfqpoint{5.627583in}{2.513212in}}%
\pgfpathlineto{\pgfqpoint{5.613077in}{2.509695in}}%
\pgfpathlineto{\pgfqpoint{5.598583in}{2.506246in}}%
\pgfpathlineto{\pgfqpoint{5.584102in}{2.502867in}}%
\pgfpathlineto{\pgfqpoint{5.569634in}{2.499556in}}%
\pgfpathlineto{\pgfqpoint{5.562264in}{2.495353in}}%
\pgfpathlineto{\pgfqpoint{5.554888in}{2.491136in}}%
\pgfpathlineto{\pgfqpoint{5.547505in}{2.486902in}}%
\pgfpathlineto{\pgfqpoint{5.540115in}{2.482646in}}%
\pgfpathclose%
\pgfusepath{fill}%
\end{pgfscope}%
\begin{pgfscope}%
\pgfpathrectangle{\pgfqpoint{1.150000in}{0.150000in}}{\pgfqpoint{5.700000in}{5.700000in}}%
\pgfusepath{clip}%
\pgfsetbuttcap%
\pgfsetroundjoin%
\definecolor{currentfill}{rgb}{0.267004,0.004874,0.329415}%
\pgfsetfillcolor{currentfill}%
\pgfsetfillopacity{0.700000}%
\pgfsetlinewidth{0.000000pt}%
\definecolor{currentstroke}{rgb}{0.000000,0.000000,0.000000}%
\pgfsetstrokecolor{currentstroke}%
\pgfsetdash{}{0pt}%
\pgfpathmoveto{\pgfqpoint{3.150204in}{1.684629in}}%
\pgfpathlineto{\pgfqpoint{3.163956in}{1.679402in}}%
\pgfpathlineto{\pgfqpoint{3.177712in}{1.674261in}}%
\pgfpathlineto{\pgfqpoint{3.191473in}{1.669206in}}%
\pgfpathlineto{\pgfqpoint{3.205238in}{1.664235in}}%
\pgfpathlineto{\pgfqpoint{3.213632in}{1.671880in}}%
\pgfpathlineto{\pgfqpoint{3.222018in}{1.679606in}}%
\pgfpathlineto{\pgfqpoint{3.230396in}{1.687408in}}%
\pgfpathlineto{\pgfqpoint{3.238766in}{1.695282in}}%
\pgfpathlineto{\pgfqpoint{3.225019in}{1.700007in}}%
\pgfpathlineto{\pgfqpoint{3.211276in}{1.704817in}}%
\pgfpathlineto{\pgfqpoint{3.197538in}{1.709712in}}%
\pgfpathlineto{\pgfqpoint{3.183804in}{1.714694in}}%
\pgfpathlineto{\pgfqpoint{3.175416in}{1.707057in}}%
\pgfpathlineto{\pgfqpoint{3.167020in}{1.699498in}}%
\pgfpathlineto{\pgfqpoint{3.158616in}{1.692021in}}%
\pgfpathlineto{\pgfqpoint{3.150204in}{1.684629in}}%
\pgfpathclose%
\pgfusepath{fill}%
\end{pgfscope}%
\begin{pgfscope}%
\pgfpathrectangle{\pgfqpoint{1.150000in}{0.150000in}}{\pgfqpoint{5.700000in}{5.700000in}}%
\pgfusepath{clip}%
\pgfsetbuttcap%
\pgfsetroundjoin%
\definecolor{currentfill}{rgb}{0.282910,0.105393,0.426902}%
\pgfsetfillcolor{currentfill}%
\pgfsetfillopacity{0.700000}%
\pgfsetlinewidth{0.000000pt}%
\definecolor{currentstroke}{rgb}{0.000000,0.000000,0.000000}%
\pgfsetstrokecolor{currentstroke}%
\pgfsetdash{}{0pt}%
\pgfpathmoveto{\pgfqpoint{4.021323in}{1.857007in}}%
\pgfpathlineto{\pgfqpoint{4.035251in}{1.856673in}}%
\pgfpathlineto{\pgfqpoint{4.049187in}{1.856414in}}%
\pgfpathlineto{\pgfqpoint{4.063131in}{1.856230in}}%
\pgfpathlineto{\pgfqpoint{4.077084in}{1.856120in}}%
\pgfpathlineto{\pgfqpoint{4.085139in}{1.865953in}}%
\pgfpathlineto{\pgfqpoint{4.093189in}{1.875738in}}%
\pgfpathlineto{\pgfqpoint{4.101233in}{1.885472in}}%
\pgfpathlineto{\pgfqpoint{4.109272in}{1.895156in}}%
\pgfpathlineto{\pgfqpoint{4.095328in}{1.895184in}}%
\pgfpathlineto{\pgfqpoint{4.081393in}{1.895287in}}%
\pgfpathlineto{\pgfqpoint{4.067467in}{1.895465in}}%
\pgfpathlineto{\pgfqpoint{4.053549in}{1.895717in}}%
\pgfpathlineto{\pgfqpoint{4.045501in}{1.886107in}}%
\pgfpathlineto{\pgfqpoint{4.037447in}{1.876451in}}%
\pgfpathlineto{\pgfqpoint{4.029388in}{1.866751in}}%
\pgfpathlineto{\pgfqpoint{4.021323in}{1.857007in}}%
\pgfpathclose%
\pgfusepath{fill}%
\end{pgfscope}%
\begin{pgfscope}%
\pgfpathrectangle{\pgfqpoint{1.150000in}{0.150000in}}{\pgfqpoint{5.700000in}{5.700000in}}%
\pgfusepath{clip}%
\pgfsetbuttcap%
\pgfsetroundjoin%
\definecolor{currentfill}{rgb}{0.273809,0.031497,0.358853}%
\pgfsetfillcolor{currentfill}%
\pgfsetfillopacity{0.700000}%
\pgfsetlinewidth{0.000000pt}%
\definecolor{currentstroke}{rgb}{0.000000,0.000000,0.000000}%
\pgfsetstrokecolor{currentstroke}%
\pgfsetdash{}{0pt}%
\pgfpathmoveto{\pgfqpoint{2.862374in}{1.733551in}}%
\pgfpathlineto{\pgfqpoint{2.876116in}{1.726336in}}%
\pgfpathlineto{\pgfqpoint{2.889861in}{1.719215in}}%
\pgfpathlineto{\pgfqpoint{2.903608in}{1.712186in}}%
\pgfpathlineto{\pgfqpoint{2.917357in}{1.705249in}}%
\pgfpathlineto{\pgfqpoint{2.925907in}{1.710933in}}%
\pgfpathlineto{\pgfqpoint{2.934447in}{1.716749in}}%
\pgfpathlineto{\pgfqpoint{2.942977in}{1.722694in}}%
\pgfpathlineto{\pgfqpoint{2.951496in}{1.728763in}}%
\pgfpathlineto{\pgfqpoint{2.937770in}{1.735412in}}%
\pgfpathlineto{\pgfqpoint{2.924046in}{1.742153in}}%
\pgfpathlineto{\pgfqpoint{2.910325in}{1.748986in}}%
\pgfpathlineto{\pgfqpoint{2.896607in}{1.755913in}}%
\pgfpathlineto{\pgfqpoint{2.888065in}{1.750124in}}%
\pgfpathlineto{\pgfqpoint{2.879512in}{1.744465in}}%
\pgfpathlineto{\pgfqpoint{2.870949in}{1.738939in}}%
\pgfpathlineto{\pgfqpoint{2.862374in}{1.733551in}}%
\pgfpathclose%
\pgfusepath{fill}%
\end{pgfscope}%
\begin{pgfscope}%
\pgfpathrectangle{\pgfqpoint{1.150000in}{0.150000in}}{\pgfqpoint{5.700000in}{5.700000in}}%
\pgfusepath{clip}%
\pgfsetbuttcap%
\pgfsetroundjoin%
\definecolor{currentfill}{rgb}{0.248629,0.278775,0.534556}%
\pgfsetfillcolor{currentfill}%
\pgfsetfillopacity{0.700000}%
\pgfsetlinewidth{0.000000pt}%
\definecolor{currentstroke}{rgb}{0.000000,0.000000,0.000000}%
\pgfsetstrokecolor{currentstroke}%
\pgfsetdash{}{0pt}%
\pgfpathmoveto{\pgfqpoint{4.868746in}{2.221900in}}%
\pgfpathlineto{\pgfqpoint{4.882974in}{2.224449in}}%
\pgfpathlineto{\pgfqpoint{4.897212in}{2.227068in}}%
\pgfpathlineto{\pgfqpoint{4.911463in}{2.229759in}}%
\pgfpathlineto{\pgfqpoint{4.925724in}{2.232520in}}%
\pgfpathlineto{\pgfqpoint{4.933446in}{2.239642in}}%
\pgfpathlineto{\pgfqpoint{4.941160in}{2.246685in}}%
\pgfpathlineto{\pgfqpoint{4.948867in}{2.253650in}}%
\pgfpathlineto{\pgfqpoint{4.956566in}{2.260540in}}%
\pgfpathlineto{\pgfqpoint{4.942318in}{2.257888in}}%
\pgfpathlineto{\pgfqpoint{4.928081in}{2.255307in}}%
\pgfpathlineto{\pgfqpoint{4.913856in}{2.252796in}}%
\pgfpathlineto{\pgfqpoint{4.899642in}{2.250357in}}%
\pgfpathlineto{\pgfqpoint{4.891928in}{2.243350in}}%
\pgfpathlineto{\pgfqpoint{4.884208in}{2.236273in}}%
\pgfpathlineto{\pgfqpoint{4.876481in}{2.229124in}}%
\pgfpathlineto{\pgfqpoint{4.868746in}{2.221900in}}%
\pgfpathclose%
\pgfusepath{fill}%
\end{pgfscope}%
\begin{pgfscope}%
\pgfpathrectangle{\pgfqpoint{1.150000in}{0.150000in}}{\pgfqpoint{5.700000in}{5.700000in}}%
\pgfusepath{clip}%
\pgfsetbuttcap%
\pgfsetroundjoin%
\definecolor{currentfill}{rgb}{0.260571,0.246922,0.522828}%
\pgfsetfillcolor{currentfill}%
\pgfsetfillopacity{0.700000}%
\pgfsetlinewidth{0.000000pt}%
\definecolor{currentstroke}{rgb}{0.000000,0.000000,0.000000}%
\pgfsetstrokecolor{currentstroke}%
\pgfsetdash{}{0pt}%
\pgfpathmoveto{\pgfqpoint{2.095288in}{2.198261in}}%
\pgfpathlineto{\pgfqpoint{2.109145in}{2.184729in}}%
\pgfpathlineto{\pgfqpoint{2.122999in}{2.171330in}}%
\pgfpathlineto{\pgfqpoint{2.136849in}{2.158063in}}%
\pgfpathlineto{\pgfqpoint{2.150696in}{2.144926in}}%
\pgfpathlineto{\pgfqpoint{2.159799in}{2.144270in}}%
\pgfpathlineto{\pgfqpoint{2.168882in}{2.143871in}}%
\pgfpathlineto{\pgfqpoint{2.177947in}{2.143726in}}%
\pgfpathlineto{\pgfqpoint{2.186993in}{2.143826in}}%
\pgfpathlineto{\pgfqpoint{2.173186in}{2.156600in}}%
\pgfpathlineto{\pgfqpoint{2.159376in}{2.169503in}}%
\pgfpathlineto{\pgfqpoint{2.145562in}{2.182537in}}%
\pgfpathlineto{\pgfqpoint{2.131746in}{2.195703in}}%
\pgfpathlineto{\pgfqpoint{2.122660in}{2.195958in}}%
\pgfpathlineto{\pgfqpoint{2.113556in}{2.196466in}}%
\pgfpathlineto{\pgfqpoint{2.104432in}{2.197231in}}%
\pgfpathlineto{\pgfqpoint{2.095288in}{2.198261in}}%
\pgfpathclose%
\pgfusepath{fill}%
\end{pgfscope}%
\begin{pgfscope}%
\pgfpathrectangle{\pgfqpoint{1.150000in}{0.150000in}}{\pgfqpoint{5.700000in}{5.700000in}}%
\pgfusepath{clip}%
\pgfsetbuttcap%
\pgfsetroundjoin%
\definecolor{currentfill}{rgb}{0.283229,0.120777,0.440584}%
\pgfsetfillcolor{currentfill}%
\pgfsetfillopacity{0.700000}%
\pgfsetlinewidth{0.000000pt}%
\definecolor{currentstroke}{rgb}{0.000000,0.000000,0.000000}%
\pgfsetstrokecolor{currentstroke}%
\pgfsetdash{}{0pt}%
\pgfpathmoveto{\pgfqpoint{2.462695in}{1.914034in}}%
\pgfpathlineto{\pgfqpoint{2.476468in}{1.903755in}}%
\pgfpathlineto{\pgfqpoint{2.490241in}{1.893585in}}%
\pgfpathlineto{\pgfqpoint{2.504013in}{1.883524in}}%
\pgfpathlineto{\pgfqpoint{2.517786in}{1.873569in}}%
\pgfpathlineto{\pgfqpoint{2.526602in}{1.875985in}}%
\pgfpathlineto{\pgfqpoint{2.535404in}{1.878603in}}%
\pgfpathlineto{\pgfqpoint{2.544192in}{1.881419in}}%
\pgfpathlineto{\pgfqpoint{2.552965in}{1.884428in}}%
\pgfpathlineto{\pgfqpoint{2.539224in}{1.894048in}}%
\pgfpathlineto{\pgfqpoint{2.525484in}{1.903776in}}%
\pgfpathlineto{\pgfqpoint{2.511744in}{1.913611in}}%
\pgfpathlineto{\pgfqpoint{2.498003in}{1.923555in}}%
\pgfpathlineto{\pgfqpoint{2.489198in}{1.920873in}}%
\pgfpathlineto{\pgfqpoint{2.480379in}{1.918388in}}%
\pgfpathlineto{\pgfqpoint{2.471545in}{1.916107in}}%
\pgfpathlineto{\pgfqpoint{2.462695in}{1.914034in}}%
\pgfpathclose%
\pgfusepath{fill}%
\end{pgfscope}%
\begin{pgfscope}%
\pgfpathrectangle{\pgfqpoint{1.150000in}{0.150000in}}{\pgfqpoint{5.700000in}{5.700000in}}%
\pgfusepath{clip}%
\pgfsetbuttcap%
\pgfsetroundjoin%
\definecolor{currentfill}{rgb}{0.281924,0.089666,0.412415}%
\pgfsetfillcolor{currentfill}%
\pgfsetfillopacity{0.700000}%
\pgfsetlinewidth{0.000000pt}%
\definecolor{currentstroke}{rgb}{0.000000,0.000000,0.000000}%
\pgfsetstrokecolor{currentstroke}%
\pgfsetdash{}{0pt}%
\pgfpathmoveto{\pgfqpoint{3.933341in}{1.820105in}}%
\pgfpathlineto{\pgfqpoint{3.947246in}{1.819369in}}%
\pgfpathlineto{\pgfqpoint{3.961158in}{1.818707in}}%
\pgfpathlineto{\pgfqpoint{3.975079in}{1.818121in}}%
\pgfpathlineto{\pgfqpoint{3.989008in}{1.817611in}}%
\pgfpathlineto{\pgfqpoint{3.997095in}{1.827520in}}%
\pgfpathlineto{\pgfqpoint{4.005177in}{1.837390in}}%
\pgfpathlineto{\pgfqpoint{4.013253in}{1.847219in}}%
\pgfpathlineto{\pgfqpoint{4.021323in}{1.857007in}}%
\pgfpathlineto{\pgfqpoint{4.007404in}{1.857415in}}%
\pgfpathlineto{\pgfqpoint{3.993493in}{1.857899in}}%
\pgfpathlineto{\pgfqpoint{3.979590in}{1.858458in}}%
\pgfpathlineto{\pgfqpoint{3.965696in}{1.859093in}}%
\pgfpathlineto{\pgfqpoint{3.957615in}{1.849400in}}%
\pgfpathlineto{\pgfqpoint{3.949529in}{1.839670in}}%
\pgfpathlineto{\pgfqpoint{3.941438in}{1.829905in}}%
\pgfpathlineto{\pgfqpoint{3.933341in}{1.820105in}}%
\pgfpathclose%
\pgfusepath{fill}%
\end{pgfscope}%
\begin{pgfscope}%
\pgfpathrectangle{\pgfqpoint{1.150000in}{0.150000in}}{\pgfqpoint{5.700000in}{5.700000in}}%
\pgfusepath{clip}%
\pgfsetbuttcap%
\pgfsetroundjoin%
\definecolor{currentfill}{rgb}{0.267004,0.004874,0.329415}%
\pgfsetfillcolor{currentfill}%
\pgfsetfillopacity{0.700000}%
\pgfsetlinewidth{0.000000pt}%
\definecolor{currentstroke}{rgb}{0.000000,0.000000,0.000000}%
\pgfsetstrokecolor{currentstroke}%
\pgfsetdash{}{0pt}%
\pgfpathmoveto{\pgfqpoint{3.293802in}{1.677227in}}%
\pgfpathlineto{\pgfqpoint{3.307573in}{1.672923in}}%
\pgfpathlineto{\pgfqpoint{3.321349in}{1.668701in}}%
\pgfpathlineto{\pgfqpoint{3.335130in}{1.664562in}}%
\pgfpathlineto{\pgfqpoint{3.348917in}{1.660506in}}%
\pgfpathlineto{\pgfqpoint{3.357246in}{1.668916in}}%
\pgfpathlineto{\pgfqpoint{3.365568in}{1.677382in}}%
\pgfpathlineto{\pgfqpoint{3.373884in}{1.685901in}}%
\pgfpathlineto{\pgfqpoint{3.382192in}{1.694469in}}%
\pgfpathlineto{\pgfqpoint{3.368421in}{1.698300in}}%
\pgfpathlineto{\pgfqpoint{3.354656in}{1.702214in}}%
\pgfpathlineto{\pgfqpoint{3.340896in}{1.706211in}}%
\pgfpathlineto{\pgfqpoint{3.327141in}{1.710290in}}%
\pgfpathlineto{\pgfqpoint{3.318817in}{1.701939in}}%
\pgfpathlineto{\pgfqpoint{3.310486in}{1.693643in}}%
\pgfpathlineto{\pgfqpoint{3.302147in}{1.685404in}}%
\pgfpathlineto{\pgfqpoint{3.293802in}{1.677227in}}%
\pgfpathclose%
\pgfusepath{fill}%
\end{pgfscope}%
\begin{pgfscope}%
\pgfpathrectangle{\pgfqpoint{1.150000in}{0.150000in}}{\pgfqpoint{5.700000in}{5.700000in}}%
\pgfusepath{clip}%
\pgfsetbuttcap%
\pgfsetroundjoin%
\definecolor{currentfill}{rgb}{0.271305,0.019942,0.347269}%
\pgfsetfillcolor{currentfill}%
\pgfsetfillopacity{0.700000}%
\pgfsetlinewidth{0.000000pt}%
\definecolor{currentstroke}{rgb}{0.000000,0.000000,0.000000}%
\pgfsetstrokecolor{currentstroke}%
\pgfsetdash{}{0pt}%
\pgfpathmoveto{\pgfqpoint{3.525608in}{1.703092in}}%
\pgfpathlineto{\pgfqpoint{3.539417in}{1.700193in}}%
\pgfpathlineto{\pgfqpoint{3.553231in}{1.697374in}}%
\pgfpathlineto{\pgfqpoint{3.567052in}{1.694634in}}%
\pgfpathlineto{\pgfqpoint{3.580879in}{1.691973in}}%
\pgfpathlineto{\pgfqpoint{3.589113in}{1.701320in}}%
\pgfpathlineto{\pgfqpoint{3.597340in}{1.710684in}}%
\pgfpathlineto{\pgfqpoint{3.605562in}{1.720063in}}%
\pgfpathlineto{\pgfqpoint{3.613777in}{1.729454in}}%
\pgfpathlineto{\pgfqpoint{3.599963in}{1.731931in}}%
\pgfpathlineto{\pgfqpoint{3.586155in}{1.734487in}}%
\pgfpathlineto{\pgfqpoint{3.572353in}{1.737123in}}%
\pgfpathlineto{\pgfqpoint{3.558558in}{1.739837in}}%
\pgfpathlineto{\pgfqpoint{3.550330in}{1.730623in}}%
\pgfpathlineto{\pgfqpoint{3.542096in}{1.721425in}}%
\pgfpathlineto{\pgfqpoint{3.533855in}{1.712248in}}%
\pgfpathlineto{\pgfqpoint{3.525608in}{1.703092in}}%
\pgfpathclose%
\pgfusepath{fill}%
\end{pgfscope}%
\begin{pgfscope}%
\pgfpathrectangle{\pgfqpoint{1.150000in}{0.150000in}}{\pgfqpoint{5.700000in}{5.700000in}}%
\pgfusepath{clip}%
\pgfsetbuttcap%
\pgfsetroundjoin%
\definecolor{currentfill}{rgb}{0.206756,0.371758,0.553117}%
\pgfsetfillcolor{currentfill}%
\pgfsetfillopacity{0.700000}%
\pgfsetlinewidth{0.000000pt}%
\definecolor{currentstroke}{rgb}{0.000000,0.000000,0.000000}%
\pgfsetstrokecolor{currentstroke}%
\pgfsetdash{}{0pt}%
\pgfpathmoveto{\pgfqpoint{5.452559in}{2.450768in}}%
\pgfpathlineto{\pgfqpoint{5.467020in}{2.454300in}}%
\pgfpathlineto{\pgfqpoint{5.481495in}{2.457902in}}%
\pgfpathlineto{\pgfqpoint{5.495981in}{2.461573in}}%
\pgfpathlineto{\pgfqpoint{5.510481in}{2.465313in}}%
\pgfpathlineto{\pgfqpoint{5.517901in}{2.469702in}}%
\pgfpathlineto{\pgfqpoint{5.525313in}{2.474050in}}%
\pgfpathlineto{\pgfqpoint{5.532718in}{2.478364in}}%
\pgfpathlineto{\pgfqpoint{5.540115in}{2.482646in}}%
\pgfpathlineto{\pgfqpoint{5.525637in}{2.479145in}}%
\pgfpathlineto{\pgfqpoint{5.511171in}{2.475713in}}%
\pgfpathlineto{\pgfqpoint{5.496718in}{2.472350in}}%
\pgfpathlineto{\pgfqpoint{5.482277in}{2.469056in}}%
\pgfpathlineto{\pgfqpoint{5.474858in}{2.464527in}}%
\pgfpathlineto{\pgfqpoint{5.467432in}{2.459972in}}%
\pgfpathlineto{\pgfqpoint{5.459999in}{2.455387in}}%
\pgfpathlineto{\pgfqpoint{5.452559in}{2.450768in}}%
\pgfpathclose%
\pgfusepath{fill}%
\end{pgfscope}%
\begin{pgfscope}%
\pgfpathrectangle{\pgfqpoint{1.150000in}{0.150000in}}{\pgfqpoint{5.700000in}{5.700000in}}%
\pgfusepath{clip}%
\pgfsetbuttcap%
\pgfsetroundjoin%
\definecolor{currentfill}{rgb}{0.255645,0.260703,0.528312}%
\pgfsetfillcolor{currentfill}%
\pgfsetfillopacity{0.700000}%
\pgfsetlinewidth{0.000000pt}%
\definecolor{currentstroke}{rgb}{0.000000,0.000000,0.000000}%
\pgfsetstrokecolor{currentstroke}%
\pgfsetdash{}{0pt}%
\pgfpathmoveto{\pgfqpoint{4.780886in}{2.182387in}}%
\pgfpathlineto{\pgfqpoint{4.795082in}{2.184739in}}%
\pgfpathlineto{\pgfqpoint{4.809289in}{2.187162in}}%
\pgfpathlineto{\pgfqpoint{4.823507in}{2.189657in}}%
\pgfpathlineto{\pgfqpoint{4.837736in}{2.192222in}}%
\pgfpathlineto{\pgfqpoint{4.845499in}{2.199763in}}%
\pgfpathlineto{\pgfqpoint{4.853255in}{2.207222in}}%
\pgfpathlineto{\pgfqpoint{4.861004in}{2.214600in}}%
\pgfpathlineto{\pgfqpoint{4.868746in}{2.221900in}}%
\pgfpathlineto{\pgfqpoint{4.854529in}{2.219422in}}%
\pgfpathlineto{\pgfqpoint{4.840324in}{2.217016in}}%
\pgfpathlineto{\pgfqpoint{4.826129in}{2.214680in}}%
\pgfpathlineto{\pgfqpoint{4.811946in}{2.212416in}}%
\pgfpathlineto{\pgfqpoint{4.804191in}{2.205020in}}%
\pgfpathlineto{\pgfqpoint{4.796430in}{2.197552in}}%
\pgfpathlineto{\pgfqpoint{4.788661in}{2.190008in}}%
\pgfpathlineto{\pgfqpoint{4.780886in}{2.182387in}}%
\pgfpathclose%
\pgfusepath{fill}%
\end{pgfscope}%
\begin{pgfscope}%
\pgfpathrectangle{\pgfqpoint{1.150000in}{0.150000in}}{\pgfqpoint{5.700000in}{5.700000in}}%
\pgfusepath{clip}%
\pgfsetbuttcap%
\pgfsetroundjoin%
\definecolor{currentfill}{rgb}{0.266580,0.228262,0.514349}%
\pgfsetfillcolor{currentfill}%
\pgfsetfillopacity{0.700000}%
\pgfsetlinewidth{0.000000pt}%
\definecolor{currentstroke}{rgb}{0.000000,0.000000,0.000000}%
\pgfsetstrokecolor{currentstroke}%
\pgfsetdash{}{0pt}%
\pgfpathmoveto{\pgfqpoint{2.150696in}{2.144926in}}%
\pgfpathlineto{\pgfqpoint{2.164540in}{2.131918in}}%
\pgfpathlineto{\pgfqpoint{2.178380in}{2.119038in}}%
\pgfpathlineto{\pgfqpoint{2.192218in}{2.106285in}}%
\pgfpathlineto{\pgfqpoint{2.206053in}{2.093658in}}%
\pgfpathlineto{\pgfqpoint{2.215116in}{2.093373in}}%
\pgfpathlineto{\pgfqpoint{2.224160in}{2.093341in}}%
\pgfpathlineto{\pgfqpoint{2.233185in}{2.093555in}}%
\pgfpathlineto{\pgfqpoint{2.242193in}{2.094010in}}%
\pgfpathlineto{\pgfqpoint{2.228397in}{2.106275in}}%
\pgfpathlineto{\pgfqpoint{2.214598in}{2.118665in}}%
\pgfpathlineto{\pgfqpoint{2.200797in}{2.131182in}}%
\pgfpathlineto{\pgfqpoint{2.186993in}{2.143826in}}%
\pgfpathlineto{\pgfqpoint{2.177947in}{2.143726in}}%
\pgfpathlineto{\pgfqpoint{2.168882in}{2.143871in}}%
\pgfpathlineto{\pgfqpoint{2.159799in}{2.144270in}}%
\pgfpathlineto{\pgfqpoint{2.150696in}{2.144926in}}%
\pgfpathclose%
\pgfusepath{fill}%
\end{pgfscope}%
\begin{pgfscope}%
\pgfpathrectangle{\pgfqpoint{1.150000in}{0.150000in}}{\pgfqpoint{5.700000in}{5.700000in}}%
\pgfusepath{clip}%
\pgfsetbuttcap%
\pgfsetroundjoin%
\definecolor{currentfill}{rgb}{0.280267,0.073417,0.397163}%
\pgfsetfillcolor{currentfill}%
\pgfsetfillopacity{0.700000}%
\pgfsetlinewidth{0.000000pt}%
\definecolor{currentstroke}{rgb}{0.000000,0.000000,0.000000}%
\pgfsetstrokecolor{currentstroke}%
\pgfsetdash{}{0pt}%
\pgfpathmoveto{\pgfqpoint{3.845317in}{1.784786in}}%
\pgfpathlineto{\pgfqpoint{3.859201in}{1.783623in}}%
\pgfpathlineto{\pgfqpoint{3.873092in}{1.782536in}}%
\pgfpathlineto{\pgfqpoint{3.886991in}{1.781525in}}%
\pgfpathlineto{\pgfqpoint{3.900898in}{1.780590in}}%
\pgfpathlineto{\pgfqpoint{3.909017in}{1.790514in}}%
\pgfpathlineto{\pgfqpoint{3.917130in}{1.800408in}}%
\pgfpathlineto{\pgfqpoint{3.925238in}{1.810273in}}%
\pgfpathlineto{\pgfqpoint{3.933341in}{1.820105in}}%
\pgfpathlineto{\pgfqpoint{3.919444in}{1.820918in}}%
\pgfpathlineto{\pgfqpoint{3.905556in}{1.821806in}}%
\pgfpathlineto{\pgfqpoint{3.891675in}{1.822770in}}%
\pgfpathlineto{\pgfqpoint{3.877801in}{1.823810in}}%
\pgfpathlineto{\pgfqpoint{3.869689in}{1.814092in}}%
\pgfpathlineto{\pgfqpoint{3.861570in}{1.804348in}}%
\pgfpathlineto{\pgfqpoint{3.853447in}{1.794579in}}%
\pgfpathlineto{\pgfqpoint{3.845317in}{1.784786in}}%
\pgfpathclose%
\pgfusepath{fill}%
\end{pgfscope}%
\begin{pgfscope}%
\pgfpathrectangle{\pgfqpoint{1.150000in}{0.150000in}}{\pgfqpoint{5.700000in}{5.700000in}}%
\pgfusepath{clip}%
\pgfsetbuttcap%
\pgfsetroundjoin%
\definecolor{currentfill}{rgb}{0.260571,0.246922,0.522828}%
\pgfsetfillcolor{currentfill}%
\pgfsetfillopacity{0.700000}%
\pgfsetlinewidth{0.000000pt}%
\definecolor{currentstroke}{rgb}{0.000000,0.000000,0.000000}%
\pgfsetstrokecolor{currentstroke}%
\pgfsetdash{}{0pt}%
\pgfpathmoveto{\pgfqpoint{4.692992in}{2.142141in}}%
\pgfpathlineto{\pgfqpoint{4.707156in}{2.144273in}}%
\pgfpathlineto{\pgfqpoint{4.721331in}{2.146478in}}%
\pgfpathlineto{\pgfqpoint{4.735517in}{2.148753in}}%
\pgfpathlineto{\pgfqpoint{4.749714in}{2.151100in}}%
\pgfpathlineto{\pgfqpoint{4.757517in}{2.159046in}}%
\pgfpathlineto{\pgfqpoint{4.765314in}{2.166908in}}%
\pgfpathlineto{\pgfqpoint{4.773103in}{2.174688in}}%
\pgfpathlineto{\pgfqpoint{4.780886in}{2.182387in}}%
\pgfpathlineto{\pgfqpoint{4.766701in}{2.180106in}}%
\pgfpathlineto{\pgfqpoint{4.752526in}{2.177897in}}%
\pgfpathlineto{\pgfqpoint{4.738363in}{2.175759in}}%
\pgfpathlineto{\pgfqpoint{4.724210in}{2.173692in}}%
\pgfpathlineto{\pgfqpoint{4.716416in}{2.165919in}}%
\pgfpathlineto{\pgfqpoint{4.708614in}{2.158070in}}%
\pgfpathlineto{\pgfqpoint{4.700806in}{2.150145in}}%
\pgfpathlineto{\pgfqpoint{4.692992in}{2.142141in}}%
\pgfpathclose%
\pgfusepath{fill}%
\end{pgfscope}%
\begin{pgfscope}%
\pgfpathrectangle{\pgfqpoint{1.150000in}{0.150000in}}{\pgfqpoint{5.700000in}{5.700000in}}%
\pgfusepath{clip}%
\pgfsetbuttcap%
\pgfsetroundjoin%
\definecolor{currentfill}{rgb}{0.277941,0.056324,0.381191}%
\pgfsetfillcolor{currentfill}%
\pgfsetfillopacity{0.700000}%
\pgfsetlinewidth{0.000000pt}%
\definecolor{currentstroke}{rgb}{0.000000,0.000000,0.000000}%
\pgfsetstrokecolor{currentstroke}%
\pgfsetdash{}{0pt}%
\pgfpathmoveto{\pgfqpoint{2.717902in}{1.777088in}}%
\pgfpathlineto{\pgfqpoint{2.731654in}{1.768798in}}%
\pgfpathlineto{\pgfqpoint{2.745408in}{1.760605in}}%
\pgfpathlineto{\pgfqpoint{2.759163in}{1.752510in}}%
\pgfpathlineto{\pgfqpoint{2.772920in}{1.744511in}}%
\pgfpathlineto{\pgfqpoint{2.781565in}{1.748984in}}%
\pgfpathlineto{\pgfqpoint{2.790199in}{1.753618in}}%
\pgfpathlineto{\pgfqpoint{2.798820in}{1.758410in}}%
\pgfpathlineto{\pgfqpoint{2.807430in}{1.763353in}}%
\pgfpathlineto{\pgfqpoint{2.793700in}{1.771042in}}%
\pgfpathlineto{\pgfqpoint{2.779971in}{1.778827in}}%
\pgfpathlineto{\pgfqpoint{2.766244in}{1.786710in}}%
\pgfpathlineto{\pgfqpoint{2.752519in}{1.794690in}}%
\pgfpathlineto{\pgfqpoint{2.743883in}{1.790048in}}%
\pgfpathlineto{\pgfqpoint{2.735235in}{1.785564in}}%
\pgfpathlineto{\pgfqpoint{2.726575in}{1.781243in}}%
\pgfpathlineto{\pgfqpoint{2.717902in}{1.777088in}}%
\pgfpathclose%
\pgfusepath{fill}%
\end{pgfscope}%
\begin{pgfscope}%
\pgfpathrectangle{\pgfqpoint{1.150000in}{0.150000in}}{\pgfqpoint{5.700000in}{5.700000in}}%
\pgfusepath{clip}%
\pgfsetbuttcap%
\pgfsetroundjoin%
\definecolor{currentfill}{rgb}{0.212395,0.359683,0.551710}%
\pgfsetfillcolor{currentfill}%
\pgfsetfillopacity{0.700000}%
\pgfsetlinewidth{0.000000pt}%
\definecolor{currentstroke}{rgb}{0.000000,0.000000,0.000000}%
\pgfsetstrokecolor{currentstroke}%
\pgfsetdash{}{0pt}%
\pgfpathmoveto{\pgfqpoint{5.364919in}{2.417558in}}%
\pgfpathlineto{\pgfqpoint{5.379351in}{2.421030in}}%
\pgfpathlineto{\pgfqpoint{5.393795in}{2.424572in}}%
\pgfpathlineto{\pgfqpoint{5.408252in}{2.428183in}}%
\pgfpathlineto{\pgfqpoint{5.422721in}{2.431864in}}%
\pgfpathlineto{\pgfqpoint{5.430192in}{2.436662in}}%
\pgfpathlineto{\pgfqpoint{5.437655in}{2.441409in}}%
\pgfpathlineto{\pgfqpoint{5.445111in}{2.446110in}}%
\pgfpathlineto{\pgfqpoint{5.452559in}{2.450768in}}%
\pgfpathlineto{\pgfqpoint{5.438109in}{2.447305in}}%
\pgfpathlineto{\pgfqpoint{5.423673in}{2.443911in}}%
\pgfpathlineto{\pgfqpoint{5.409248in}{2.440587in}}%
\pgfpathlineto{\pgfqpoint{5.394836in}{2.437332in}}%
\pgfpathlineto{\pgfqpoint{5.387368in}{2.432449in}}%
\pgfpathlineto{\pgfqpoint{5.379893in}{2.427528in}}%
\pgfpathlineto{\pgfqpoint{5.372410in}{2.422566in}}%
\pgfpathlineto{\pgfqpoint{5.364919in}{2.417558in}}%
\pgfpathclose%
\pgfusepath{fill}%
\end{pgfscope}%
\begin{pgfscope}%
\pgfpathrectangle{\pgfqpoint{1.150000in}{0.150000in}}{\pgfqpoint{5.700000in}{5.700000in}}%
\pgfusepath{clip}%
\pgfsetbuttcap%
\pgfsetroundjoin%
\definecolor{currentfill}{rgb}{0.266580,0.228262,0.514349}%
\pgfsetfillcolor{currentfill}%
\pgfsetfillopacity{0.700000}%
\pgfsetlinewidth{0.000000pt}%
\definecolor{currentstroke}{rgb}{0.000000,0.000000,0.000000}%
\pgfsetstrokecolor{currentstroke}%
\pgfsetdash{}{0pt}%
\pgfpathmoveto{\pgfqpoint{4.605069in}{2.101322in}}%
\pgfpathlineto{\pgfqpoint{4.619202in}{2.103213in}}%
\pgfpathlineto{\pgfqpoint{4.633346in}{2.105175in}}%
\pgfpathlineto{\pgfqpoint{4.647500in}{2.107210in}}%
\pgfpathlineto{\pgfqpoint{4.661665in}{2.109316in}}%
\pgfpathlineto{\pgfqpoint{4.669507in}{2.117646in}}%
\pgfpathlineto{\pgfqpoint{4.677342in}{2.125893in}}%
\pgfpathlineto{\pgfqpoint{4.685170in}{2.134057in}}%
\pgfpathlineto{\pgfqpoint{4.692992in}{2.142141in}}%
\pgfpathlineto{\pgfqpoint{4.678838in}{2.140080in}}%
\pgfpathlineto{\pgfqpoint{4.664694in}{2.138090in}}%
\pgfpathlineto{\pgfqpoint{4.650562in}{2.136172in}}%
\pgfpathlineto{\pgfqpoint{4.636439in}{2.134326in}}%
\pgfpathlineto{\pgfqpoint{4.628607in}{2.126190in}}%
\pgfpathlineto{\pgfqpoint{4.620767in}{2.117978in}}%
\pgfpathlineto{\pgfqpoint{4.612921in}{2.109689in}}%
\pgfpathlineto{\pgfqpoint{4.605069in}{2.101322in}}%
\pgfpathclose%
\pgfusepath{fill}%
\end{pgfscope}%
\begin{pgfscope}%
\pgfpathrectangle{\pgfqpoint{1.150000in}{0.150000in}}{\pgfqpoint{5.700000in}{5.700000in}}%
\pgfusepath{clip}%
\pgfsetbuttcap%
\pgfsetroundjoin%
\definecolor{currentfill}{rgb}{0.273006,0.204520,0.501721}%
\pgfsetfillcolor{currentfill}%
\pgfsetfillopacity{0.700000}%
\pgfsetlinewidth{0.000000pt}%
\definecolor{currentstroke}{rgb}{0.000000,0.000000,0.000000}%
\pgfsetstrokecolor{currentstroke}%
\pgfsetdash{}{0pt}%
\pgfpathmoveto{\pgfqpoint{2.206053in}{2.093658in}}%
\pgfpathlineto{\pgfqpoint{2.219885in}{2.081156in}}%
\pgfpathlineto{\pgfqpoint{2.233715in}{2.068777in}}%
\pgfpathlineto{\pgfqpoint{2.247542in}{2.056522in}}%
\pgfpathlineto{\pgfqpoint{2.261368in}{2.044388in}}%
\pgfpathlineto{\pgfqpoint{2.270391in}{2.044473in}}%
\pgfpathlineto{\pgfqpoint{2.279397in}{2.044805in}}%
\pgfpathlineto{\pgfqpoint{2.288385in}{2.045377in}}%
\pgfpathlineto{\pgfqpoint{2.297355in}{2.046185in}}%
\pgfpathlineto{\pgfqpoint{2.283568in}{2.057958in}}%
\pgfpathlineto{\pgfqpoint{2.269779in}{2.069853in}}%
\pgfpathlineto{\pgfqpoint{2.255987in}{2.081870in}}%
\pgfpathlineto{\pgfqpoint{2.242193in}{2.094010in}}%
\pgfpathlineto{\pgfqpoint{2.233185in}{2.093555in}}%
\pgfpathlineto{\pgfqpoint{2.224160in}{2.093341in}}%
\pgfpathlineto{\pgfqpoint{2.215116in}{2.093373in}}%
\pgfpathlineto{\pgfqpoint{2.206053in}{2.093658in}}%
\pgfpathclose%
\pgfusepath{fill}%
\end{pgfscope}%
\begin{pgfscope}%
\pgfpathrectangle{\pgfqpoint{1.150000in}{0.150000in}}{\pgfqpoint{5.700000in}{5.700000in}}%
\pgfusepath{clip}%
\pgfsetbuttcap%
\pgfsetroundjoin%
\definecolor{currentfill}{rgb}{0.271828,0.209303,0.504434}%
\pgfsetfillcolor{currentfill}%
\pgfsetfillopacity{0.700000}%
\pgfsetlinewidth{0.000000pt}%
\definecolor{currentstroke}{rgb}{0.000000,0.000000,0.000000}%
\pgfsetstrokecolor{currentstroke}%
\pgfsetdash{}{0pt}%
\pgfpathmoveto{\pgfqpoint{4.517121in}{2.060113in}}%
\pgfpathlineto{\pgfqpoint{4.531224in}{2.061739in}}%
\pgfpathlineto{\pgfqpoint{4.545336in}{2.063438in}}%
\pgfpathlineto{\pgfqpoint{4.559459in}{2.065209in}}%
\pgfpathlineto{\pgfqpoint{4.573592in}{2.067051in}}%
\pgfpathlineto{\pgfqpoint{4.581471in}{2.075741in}}%
\pgfpathlineto{\pgfqpoint{4.589344in}{2.084349in}}%
\pgfpathlineto{\pgfqpoint{4.597209in}{2.092875in}}%
\pgfpathlineto{\pgfqpoint{4.605069in}{2.101322in}}%
\pgfpathlineto{\pgfqpoint{4.590946in}{2.099503in}}%
\pgfpathlineto{\pgfqpoint{4.576833in}{2.097756in}}%
\pgfpathlineto{\pgfqpoint{4.562731in}{2.096081in}}%
\pgfpathlineto{\pgfqpoint{4.548639in}{2.094478in}}%
\pgfpathlineto{\pgfqpoint{4.540769in}{2.086000in}}%
\pgfpathlineto{\pgfqpoint{4.532893in}{2.077447in}}%
\pgfpathlineto{\pgfqpoint{4.525010in}{2.068818in}}%
\pgfpathlineto{\pgfqpoint{4.517121in}{2.060113in}}%
\pgfpathclose%
\pgfusepath{fill}%
\end{pgfscope}%
\begin{pgfscope}%
\pgfpathrectangle{\pgfqpoint{1.150000in}{0.150000in}}{\pgfqpoint{5.700000in}{5.700000in}}%
\pgfusepath{clip}%
\pgfsetbuttcap%
\pgfsetroundjoin%
\definecolor{currentfill}{rgb}{0.277941,0.056324,0.381191}%
\pgfsetfillcolor{currentfill}%
\pgfsetfillopacity{0.700000}%
\pgfsetlinewidth{0.000000pt}%
\definecolor{currentstroke}{rgb}{0.000000,0.000000,0.000000}%
\pgfsetstrokecolor{currentstroke}%
\pgfsetdash{}{0pt}%
\pgfpathmoveto{\pgfqpoint{3.757241in}{1.751402in}}%
\pgfpathlineto{\pgfqpoint{3.771106in}{1.749790in}}%
\pgfpathlineto{\pgfqpoint{3.784978in}{1.748255in}}%
\pgfpathlineto{\pgfqpoint{3.798857in}{1.746796in}}%
\pgfpathlineto{\pgfqpoint{3.812743in}{1.745413in}}%
\pgfpathlineto{\pgfqpoint{3.820895in}{1.755283in}}%
\pgfpathlineto{\pgfqpoint{3.829041in}{1.765136in}}%
\pgfpathlineto{\pgfqpoint{3.837182in}{1.774971in}}%
\pgfpathlineto{\pgfqpoint{3.845317in}{1.784786in}}%
\pgfpathlineto{\pgfqpoint{3.831441in}{1.786025in}}%
\pgfpathlineto{\pgfqpoint{3.817573in}{1.787340in}}%
\pgfpathlineto{\pgfqpoint{3.803712in}{1.788733in}}%
\pgfpathlineto{\pgfqpoint{3.789858in}{1.790202in}}%
\pgfpathlineto{\pgfqpoint{3.781712in}{1.780523in}}%
\pgfpathlineto{\pgfqpoint{3.773561in}{1.770828in}}%
\pgfpathlineto{\pgfqpoint{3.765404in}{1.761121in}}%
\pgfpathlineto{\pgfqpoint{3.757241in}{1.751402in}}%
\pgfpathclose%
\pgfusepath{fill}%
\end{pgfscope}%
\begin{pgfscope}%
\pgfpathrectangle{\pgfqpoint{1.150000in}{0.150000in}}{\pgfqpoint{5.700000in}{5.700000in}}%
\pgfusepath{clip}%
\pgfsetbuttcap%
\pgfsetroundjoin%
\definecolor{currentfill}{rgb}{0.269944,0.014625,0.341379}%
\pgfsetfillcolor{currentfill}%
\pgfsetfillopacity{0.700000}%
\pgfsetlinewidth{0.000000pt}%
\definecolor{currentstroke}{rgb}{0.000000,0.000000,0.000000}%
\pgfsetstrokecolor{currentstroke}%
\pgfsetdash{}{0pt}%
\pgfpathmoveto{\pgfqpoint{3.437329in}{1.679961in}}%
\pgfpathlineto{\pgfqpoint{3.451128in}{1.676536in}}%
\pgfpathlineto{\pgfqpoint{3.464932in}{1.673193in}}%
\pgfpathlineto{\pgfqpoint{3.478742in}{1.669929in}}%
\pgfpathlineto{\pgfqpoint{3.492558in}{1.666746in}}%
\pgfpathlineto{\pgfqpoint{3.500830in}{1.675786in}}%
\pgfpathlineto{\pgfqpoint{3.509096in}{1.684858in}}%
\pgfpathlineto{\pgfqpoint{3.517355in}{1.693961in}}%
\pgfpathlineto{\pgfqpoint{3.525608in}{1.703092in}}%
\pgfpathlineto{\pgfqpoint{3.511806in}{1.706071in}}%
\pgfpathlineto{\pgfqpoint{3.498010in}{1.709130in}}%
\pgfpathlineto{\pgfqpoint{3.484220in}{1.712269in}}%
\pgfpathlineto{\pgfqpoint{3.470436in}{1.715488in}}%
\pgfpathlineto{\pgfqpoint{3.462169in}{1.706554in}}%
\pgfpathlineto{\pgfqpoint{3.453896in}{1.697653in}}%
\pgfpathlineto{\pgfqpoint{3.445616in}{1.688788in}}%
\pgfpathlineto{\pgfqpoint{3.437329in}{1.679961in}}%
\pgfpathclose%
\pgfusepath{fill}%
\end{pgfscope}%
\begin{pgfscope}%
\pgfpathrectangle{\pgfqpoint{1.150000in}{0.150000in}}{\pgfqpoint{5.700000in}{5.700000in}}%
\pgfusepath{clip}%
\pgfsetbuttcap%
\pgfsetroundjoin%
\definecolor{currentfill}{rgb}{0.282910,0.105393,0.426902}%
\pgfsetfillcolor{currentfill}%
\pgfsetfillopacity{0.700000}%
\pgfsetlinewidth{0.000000pt}%
\definecolor{currentstroke}{rgb}{0.000000,0.000000,0.000000}%
\pgfsetstrokecolor{currentstroke}%
\pgfsetdash{}{0pt}%
\pgfpathmoveto{\pgfqpoint{2.517786in}{1.873569in}}%
\pgfpathlineto{\pgfqpoint{2.531558in}{1.863721in}}%
\pgfpathlineto{\pgfqpoint{2.545330in}{1.853980in}}%
\pgfpathlineto{\pgfqpoint{2.559103in}{1.844343in}}%
\pgfpathlineto{\pgfqpoint{2.572876in}{1.834811in}}%
\pgfpathlineto{\pgfqpoint{2.581661in}{1.837569in}}%
\pgfpathlineto{\pgfqpoint{2.590431in}{1.840524in}}%
\pgfpathlineto{\pgfqpoint{2.599188in}{1.843671in}}%
\pgfpathlineto{\pgfqpoint{2.607931in}{1.847005in}}%
\pgfpathlineto{\pgfqpoint{2.594189in}{1.856204in}}%
\pgfpathlineto{\pgfqpoint{2.580447in}{1.865507in}}%
\pgfpathlineto{\pgfqpoint{2.566706in}{1.874914in}}%
\pgfpathlineto{\pgfqpoint{2.552965in}{1.884428in}}%
\pgfpathlineto{\pgfqpoint{2.544192in}{1.881419in}}%
\pgfpathlineto{\pgfqpoint{2.535404in}{1.878603in}}%
\pgfpathlineto{\pgfqpoint{2.526602in}{1.875985in}}%
\pgfpathlineto{\pgfqpoint{2.517786in}{1.873569in}}%
\pgfpathclose%
\pgfusepath{fill}%
\end{pgfscope}%
\begin{pgfscope}%
\pgfpathrectangle{\pgfqpoint{1.150000in}{0.150000in}}{\pgfqpoint{5.700000in}{5.700000in}}%
\pgfusepath{clip}%
\pgfsetbuttcap%
\pgfsetroundjoin%
\definecolor{currentfill}{rgb}{0.268510,0.009605,0.335427}%
\pgfsetfillcolor{currentfill}%
\pgfsetfillopacity{0.700000}%
\pgfsetlinewidth{0.000000pt}%
\definecolor{currentstroke}{rgb}{0.000000,0.000000,0.000000}%
\pgfsetstrokecolor{currentstroke}%
\pgfsetdash{}{0pt}%
\pgfpathmoveto{\pgfqpoint{3.061422in}{1.678831in}}%
\pgfpathlineto{\pgfqpoint{3.075179in}{1.672990in}}%
\pgfpathlineto{\pgfqpoint{3.088939in}{1.667236in}}%
\pgfpathlineto{\pgfqpoint{3.102702in}{1.661569in}}%
\pgfpathlineto{\pgfqpoint{3.116470in}{1.655990in}}%
\pgfpathlineto{\pgfqpoint{3.124917in}{1.663003in}}%
\pgfpathlineto{\pgfqpoint{3.133354in}{1.670116in}}%
\pgfpathlineto{\pgfqpoint{3.141783in}{1.677327in}}%
\pgfpathlineto{\pgfqpoint{3.150204in}{1.684629in}}%
\pgfpathlineto{\pgfqpoint{3.136456in}{1.689943in}}%
\pgfpathlineto{\pgfqpoint{3.122712in}{1.695343in}}%
\pgfpathlineto{\pgfqpoint{3.108972in}{1.700830in}}%
\pgfpathlineto{\pgfqpoint{3.095236in}{1.706404in}}%
\pgfpathlineto{\pgfqpoint{3.086796in}{1.699360in}}%
\pgfpathlineto{\pgfqpoint{3.078347in}{1.692414in}}%
\pgfpathlineto{\pgfqpoint{3.069889in}{1.685570in}}%
\pgfpathlineto{\pgfqpoint{3.061422in}{1.678831in}}%
\pgfpathclose%
\pgfusepath{fill}%
\end{pgfscope}%
\begin{pgfscope}%
\pgfpathrectangle{\pgfqpoint{1.150000in}{0.150000in}}{\pgfqpoint{5.700000in}{5.700000in}}%
\pgfusepath{clip}%
\pgfsetbuttcap%
\pgfsetroundjoin%
\definecolor{currentfill}{rgb}{0.275191,0.194905,0.496005}%
\pgfsetfillcolor{currentfill}%
\pgfsetfillopacity{0.700000}%
\pgfsetlinewidth{0.000000pt}%
\definecolor{currentstroke}{rgb}{0.000000,0.000000,0.000000}%
\pgfsetstrokecolor{currentstroke}%
\pgfsetdash{}{0pt}%
\pgfpathmoveto{\pgfqpoint{4.429152in}{2.018718in}}%
\pgfpathlineto{\pgfqpoint{4.443225in}{2.020058in}}%
\pgfpathlineto{\pgfqpoint{4.457307in}{2.021470in}}%
\pgfpathlineto{\pgfqpoint{4.471399in}{2.022954in}}%
\pgfpathlineto{\pgfqpoint{4.485501in}{2.024511in}}%
\pgfpathlineto{\pgfqpoint{4.493416in}{2.033529in}}%
\pgfpathlineto{\pgfqpoint{4.501324in}{2.042469in}}%
\pgfpathlineto{\pgfqpoint{4.509226in}{2.051330in}}%
\pgfpathlineto{\pgfqpoint{4.517121in}{2.060113in}}%
\pgfpathlineto{\pgfqpoint{4.503029in}{2.058559in}}%
\pgfpathlineto{\pgfqpoint{4.488946in}{2.057077in}}%
\pgfpathlineto{\pgfqpoint{4.474874in}{2.055667in}}%
\pgfpathlineto{\pgfqpoint{4.460812in}{2.054330in}}%
\pgfpathlineto{\pgfqpoint{4.452906in}{2.045537in}}%
\pgfpathlineto{\pgfqpoint{4.444995in}{2.036671in}}%
\pgfpathlineto{\pgfqpoint{4.437077in}{2.027732in}}%
\pgfpathlineto{\pgfqpoint{4.429152in}{2.018718in}}%
\pgfpathclose%
\pgfusepath{fill}%
\end{pgfscope}%
\begin{pgfscope}%
\pgfpathrectangle{\pgfqpoint{1.150000in}{0.150000in}}{\pgfqpoint{5.700000in}{5.700000in}}%
\pgfusepath{clip}%
\pgfsetbuttcap%
\pgfsetroundjoin%
\definecolor{currentfill}{rgb}{0.218130,0.347432,0.550038}%
\pgfsetfillcolor{currentfill}%
\pgfsetfillopacity{0.700000}%
\pgfsetlinewidth{0.000000pt}%
\definecolor{currentstroke}{rgb}{0.000000,0.000000,0.000000}%
\pgfsetstrokecolor{currentstroke}%
\pgfsetdash{}{0pt}%
\pgfpathmoveto{\pgfqpoint{5.277204in}{2.383023in}}%
\pgfpathlineto{\pgfqpoint{5.291605in}{2.386412in}}%
\pgfpathlineto{\pgfqpoint{5.306018in}{2.389871in}}%
\pgfpathlineto{\pgfqpoint{5.320443in}{2.393400in}}%
\pgfpathlineto{\pgfqpoint{5.334881in}{2.396999in}}%
\pgfpathlineto{\pgfqpoint{5.342402in}{2.402225in}}%
\pgfpathlineto{\pgfqpoint{5.349916in}{2.407392in}}%
\pgfpathlineto{\pgfqpoint{5.357421in}{2.412501in}}%
\pgfpathlineto{\pgfqpoint{5.364919in}{2.417558in}}%
\pgfpathlineto{\pgfqpoint{5.350500in}{2.414156in}}%
\pgfpathlineto{\pgfqpoint{5.336093in}{2.410823in}}%
\pgfpathlineto{\pgfqpoint{5.321698in}{2.407560in}}%
\pgfpathlineto{\pgfqpoint{5.307315in}{2.404367in}}%
\pgfpathlineto{\pgfqpoint{5.299799in}{2.399106in}}%
\pgfpathlineto{\pgfqpoint{5.292275in}{2.393798in}}%
\pgfpathlineto{\pgfqpoint{5.284743in}{2.388438in}}%
\pgfpathlineto{\pgfqpoint{5.277204in}{2.383023in}}%
\pgfpathclose%
\pgfusepath{fill}%
\end{pgfscope}%
\begin{pgfscope}%
\pgfpathrectangle{\pgfqpoint{1.150000in}{0.150000in}}{\pgfqpoint{5.700000in}{5.700000in}}%
\pgfusepath{clip}%
\pgfsetbuttcap%
\pgfsetroundjoin%
\definecolor{currentfill}{rgb}{0.271305,0.019942,0.347269}%
\pgfsetfillcolor{currentfill}%
\pgfsetfillopacity{0.700000}%
\pgfsetlinewidth{0.000000pt}%
\definecolor{currentstroke}{rgb}{0.000000,0.000000,0.000000}%
\pgfsetstrokecolor{currentstroke}%
\pgfsetdash{}{0pt}%
\pgfpathmoveto{\pgfqpoint{2.917357in}{1.705249in}}%
\pgfpathlineto{\pgfqpoint{2.931110in}{1.698404in}}%
\pgfpathlineto{\pgfqpoint{2.944865in}{1.691650in}}%
\pgfpathlineto{\pgfqpoint{2.958623in}{1.684986in}}%
\pgfpathlineto{\pgfqpoint{2.972385in}{1.678413in}}%
\pgfpathlineto{\pgfqpoint{2.980911in}{1.684393in}}%
\pgfpathlineto{\pgfqpoint{2.989428in}{1.690500in}}%
\pgfpathlineto{\pgfqpoint{2.997935in}{1.696730in}}%
\pgfpathlineto{\pgfqpoint{3.006433in}{1.703079in}}%
\pgfpathlineto{\pgfqpoint{2.992694in}{1.709365in}}%
\pgfpathlineto{\pgfqpoint{2.978958in}{1.715740in}}%
\pgfpathlineto{\pgfqpoint{2.965226in}{1.722206in}}%
\pgfpathlineto{\pgfqpoint{2.951496in}{1.728763in}}%
\pgfpathlineto{\pgfqpoint{2.942977in}{1.722694in}}%
\pgfpathlineto{\pgfqpoint{2.934447in}{1.716749in}}%
\pgfpathlineto{\pgfqpoint{2.925907in}{1.710933in}}%
\pgfpathlineto{\pgfqpoint{2.917357in}{1.705249in}}%
\pgfpathclose%
\pgfusepath{fill}%
\end{pgfscope}%
\begin{pgfscope}%
\pgfpathrectangle{\pgfqpoint{1.150000in}{0.150000in}}{\pgfqpoint{5.700000in}{5.700000in}}%
\pgfusepath{clip}%
\pgfsetbuttcap%
\pgfsetroundjoin%
\definecolor{currentfill}{rgb}{0.267004,0.004874,0.329415}%
\pgfsetfillcolor{currentfill}%
\pgfsetfillopacity{0.700000}%
\pgfsetlinewidth{0.000000pt}%
\definecolor{currentstroke}{rgb}{0.000000,0.000000,0.000000}%
\pgfsetstrokecolor{currentstroke}%
\pgfsetdash{}{0pt}%
\pgfpathmoveto{\pgfqpoint{3.205238in}{1.664235in}}%
\pgfpathlineto{\pgfqpoint{3.219008in}{1.659349in}}%
\pgfpathlineto{\pgfqpoint{3.232782in}{1.654548in}}%
\pgfpathlineto{\pgfqpoint{3.246561in}{1.649830in}}%
\pgfpathlineto{\pgfqpoint{3.260344in}{1.645197in}}%
\pgfpathlineto{\pgfqpoint{3.268720in}{1.653096in}}%
\pgfpathlineto{\pgfqpoint{3.277088in}{1.661069in}}%
\pgfpathlineto{\pgfqpoint{3.285449in}{1.669114in}}%
\pgfpathlineto{\pgfqpoint{3.293802in}{1.677227in}}%
\pgfpathlineto{\pgfqpoint{3.280036in}{1.681615in}}%
\pgfpathlineto{\pgfqpoint{3.266274in}{1.686087in}}%
\pgfpathlineto{\pgfqpoint{3.252518in}{1.690642in}}%
\pgfpathlineto{\pgfqpoint{3.238766in}{1.695282in}}%
\pgfpathlineto{\pgfqpoint{3.230396in}{1.687408in}}%
\pgfpathlineto{\pgfqpoint{3.222018in}{1.679606in}}%
\pgfpathlineto{\pgfqpoint{3.213632in}{1.671880in}}%
\pgfpathlineto{\pgfqpoint{3.205238in}{1.664235in}}%
\pgfpathclose%
\pgfusepath{fill}%
\end{pgfscope}%
\begin{pgfscope}%
\pgfpathrectangle{\pgfqpoint{1.150000in}{0.150000in}}{\pgfqpoint{5.700000in}{5.700000in}}%
\pgfusepath{clip}%
\pgfsetbuttcap%
\pgfsetroundjoin%
\definecolor{currentfill}{rgb}{0.278826,0.175490,0.483397}%
\pgfsetfillcolor{currentfill}%
\pgfsetfillopacity{0.700000}%
\pgfsetlinewidth{0.000000pt}%
\definecolor{currentstroke}{rgb}{0.000000,0.000000,0.000000}%
\pgfsetstrokecolor{currentstroke}%
\pgfsetdash{}{0pt}%
\pgfpathmoveto{\pgfqpoint{4.341165in}{1.977364in}}%
\pgfpathlineto{\pgfqpoint{4.355208in}{1.978394in}}%
\pgfpathlineto{\pgfqpoint{4.369260in}{1.979496in}}%
\pgfpathlineto{\pgfqpoint{4.383322in}{1.980672in}}%
\pgfpathlineto{\pgfqpoint{4.397394in}{1.981920in}}%
\pgfpathlineto{\pgfqpoint{4.405343in}{1.991232in}}%
\pgfpathlineto{\pgfqpoint{4.413286in}{2.000469in}}%
\pgfpathlineto{\pgfqpoint{4.421222in}{2.009631in}}%
\pgfpathlineto{\pgfqpoint{4.429152in}{2.018718in}}%
\pgfpathlineto{\pgfqpoint{4.415090in}{2.017452in}}%
\pgfpathlineto{\pgfqpoint{4.401037in}{2.016258in}}%
\pgfpathlineto{\pgfqpoint{4.386994in}{2.015136in}}%
\pgfpathlineto{\pgfqpoint{4.372961in}{2.014088in}}%
\pgfpathlineto{\pgfqpoint{4.365021in}{2.005011in}}%
\pgfpathlineto{\pgfqpoint{4.357075in}{1.995865in}}%
\pgfpathlineto{\pgfqpoint{4.349123in}{1.986649in}}%
\pgfpathlineto{\pgfqpoint{4.341165in}{1.977364in}}%
\pgfpathclose%
\pgfusepath{fill}%
\end{pgfscope}%
\begin{pgfscope}%
\pgfpathrectangle{\pgfqpoint{1.150000in}{0.150000in}}{\pgfqpoint{5.700000in}{5.700000in}}%
\pgfusepath{clip}%
\pgfsetbuttcap%
\pgfsetroundjoin%
\definecolor{currentfill}{rgb}{0.192357,0.403199,0.555836}%
\pgfsetfillcolor{currentfill}%
\pgfsetfillopacity{0.700000}%
\pgfsetlinewidth{0.000000pt}%
\definecolor{currentstroke}{rgb}{0.000000,0.000000,0.000000}%
\pgfsetstrokecolor{currentstroke}%
\pgfsetdash{}{0pt}%
\pgfpathmoveto{\pgfqpoint{5.685736in}{2.527969in}}%
\pgfpathlineto{\pgfqpoint{5.700307in}{2.531830in}}%
\pgfpathlineto{\pgfqpoint{5.714891in}{2.535761in}}%
\pgfpathlineto{\pgfqpoint{5.729488in}{2.539760in}}%
\pgfpathlineto{\pgfqpoint{5.736785in}{2.543203in}}%
\pgfpathlineto{\pgfqpoint{5.744075in}{2.546628in}}%
\pgfpathlineto{\pgfqpoint{5.751358in}{2.550042in}}%
\pgfpathlineto{\pgfqpoint{5.758635in}{2.553448in}}%
\pgfpathlineto{\pgfqpoint{5.744062in}{2.549732in}}%
\pgfpathlineto{\pgfqpoint{5.729503in}{2.546084in}}%
\pgfpathlineto{\pgfqpoint{5.714957in}{2.542505in}}%
\pgfpathlineto{\pgfqpoint{5.707662in}{2.538881in}}%
\pgfpathlineto{\pgfqpoint{5.700360in}{2.535253in}}%
\pgfpathlineto{\pgfqpoint{5.693052in}{2.531618in}}%
\pgfpathlineto{\pgfqpoint{5.685736in}{2.527969in}}%
\pgfpathclose%
\pgfusepath{fill}%
\end{pgfscope}%
\begin{pgfscope}%
\pgfpathrectangle{\pgfqpoint{1.150000in}{0.150000in}}{\pgfqpoint{5.700000in}{5.700000in}}%
\pgfusepath{clip}%
\pgfsetbuttcap%
\pgfsetroundjoin%
\definecolor{currentfill}{rgb}{0.276022,0.044167,0.370164}%
\pgfsetfillcolor{currentfill}%
\pgfsetfillopacity{0.700000}%
\pgfsetlinewidth{0.000000pt}%
\definecolor{currentstroke}{rgb}{0.000000,0.000000,0.000000}%
\pgfsetstrokecolor{currentstroke}%
\pgfsetdash{}{0pt}%
\pgfpathmoveto{\pgfqpoint{3.669100in}{1.720332in}}%
\pgfpathlineto{\pgfqpoint{3.682948in}{1.718246in}}%
\pgfpathlineto{\pgfqpoint{3.696803in}{1.716239in}}%
\pgfpathlineto{\pgfqpoint{3.710665in}{1.714308in}}%
\pgfpathlineto{\pgfqpoint{3.724533in}{1.712455in}}%
\pgfpathlineto{\pgfqpoint{3.732719in}{1.722198in}}%
\pgfpathlineto{\pgfqpoint{3.740899in}{1.731939in}}%
\pgfpathlineto{\pgfqpoint{3.749073in}{1.741674in}}%
\pgfpathlineto{\pgfqpoint{3.757241in}{1.751402in}}%
\pgfpathlineto{\pgfqpoint{3.743384in}{1.753092in}}%
\pgfpathlineto{\pgfqpoint{3.729534in}{1.754858in}}%
\pgfpathlineto{\pgfqpoint{3.715691in}{1.756702in}}%
\pgfpathlineto{\pgfqpoint{3.701855in}{1.758624in}}%
\pgfpathlineto{\pgfqpoint{3.693675in}{1.749052in}}%
\pgfpathlineto{\pgfqpoint{3.685489in}{1.739478in}}%
\pgfpathlineto{\pgfqpoint{3.677298in}{1.729904in}}%
\pgfpathlineto{\pgfqpoint{3.669100in}{1.720332in}}%
\pgfpathclose%
\pgfusepath{fill}%
\end{pgfscope}%
\begin{pgfscope}%
\pgfpathrectangle{\pgfqpoint{1.150000in}{0.150000in}}{\pgfqpoint{5.700000in}{5.700000in}}%
\pgfusepath{clip}%
\pgfsetbuttcap%
\pgfsetroundjoin%
\definecolor{currentfill}{rgb}{0.277134,0.185228,0.489898}%
\pgfsetfillcolor{currentfill}%
\pgfsetfillopacity{0.700000}%
\pgfsetlinewidth{0.000000pt}%
\definecolor{currentstroke}{rgb}{0.000000,0.000000,0.000000}%
\pgfsetstrokecolor{currentstroke}%
\pgfsetdash{}{0pt}%
\pgfpathmoveto{\pgfqpoint{2.261368in}{2.044388in}}%
\pgfpathlineto{\pgfqpoint{2.275191in}{2.032374in}}%
\pgfpathlineto{\pgfqpoint{2.289012in}{2.020481in}}%
\pgfpathlineto{\pgfqpoint{2.302831in}{2.008706in}}%
\pgfpathlineto{\pgfqpoint{2.316648in}{1.997050in}}%
\pgfpathlineto{\pgfqpoint{2.325634in}{1.997503in}}%
\pgfpathlineto{\pgfqpoint{2.334602in}{1.998198in}}%
\pgfpathlineto{\pgfqpoint{2.343554in}{1.999128in}}%
\pgfpathlineto{\pgfqpoint{2.352488in}{2.000288in}}%
\pgfpathlineto{\pgfqpoint{2.338707in}{2.011585in}}%
\pgfpathlineto{\pgfqpoint{2.324925in}{2.023000in}}%
\pgfpathlineto{\pgfqpoint{2.311141in}{2.034533in}}%
\pgfpathlineto{\pgfqpoint{2.297355in}{2.046185in}}%
\pgfpathlineto{\pgfqpoint{2.288385in}{2.045377in}}%
\pgfpathlineto{\pgfqpoint{2.279397in}{2.044805in}}%
\pgfpathlineto{\pgfqpoint{2.270391in}{2.044473in}}%
\pgfpathlineto{\pgfqpoint{2.261368in}{2.044388in}}%
\pgfpathclose%
\pgfusepath{fill}%
\end{pgfscope}%
\begin{pgfscope}%
\pgfpathrectangle{\pgfqpoint{1.150000in}{0.150000in}}{\pgfqpoint{5.700000in}{5.700000in}}%
\pgfusepath{clip}%
\pgfsetbuttcap%
\pgfsetroundjoin%
\definecolor{currentfill}{rgb}{0.281412,0.155834,0.469201}%
\pgfsetfillcolor{currentfill}%
\pgfsetfillopacity{0.700000}%
\pgfsetlinewidth{0.000000pt}%
\definecolor{currentstroke}{rgb}{0.000000,0.000000,0.000000}%
\pgfsetstrokecolor{currentstroke}%
\pgfsetdash{}{0pt}%
\pgfpathmoveto{\pgfqpoint{4.253158in}{1.936297in}}%
\pgfpathlineto{\pgfqpoint{4.267173in}{1.936994in}}%
\pgfpathlineto{\pgfqpoint{4.281197in}{1.937765in}}%
\pgfpathlineto{\pgfqpoint{4.295230in}{1.938609in}}%
\pgfpathlineto{\pgfqpoint{4.309273in}{1.939526in}}%
\pgfpathlineto{\pgfqpoint{4.317255in}{1.949090in}}%
\pgfpathlineto{\pgfqpoint{4.325230in}{1.958584in}}%
\pgfpathlineto{\pgfqpoint{4.333201in}{1.968009in}}%
\pgfpathlineto{\pgfqpoint{4.341165in}{1.977364in}}%
\pgfpathlineto{\pgfqpoint{4.327131in}{1.976407in}}%
\pgfpathlineto{\pgfqpoint{4.313107in}{1.975524in}}%
\pgfpathlineto{\pgfqpoint{4.299093in}{1.974713in}}%
\pgfpathlineto{\pgfqpoint{4.285088in}{1.973976in}}%
\pgfpathlineto{\pgfqpoint{4.277114in}{1.964653in}}%
\pgfpathlineto{\pgfqpoint{4.269135in}{1.955265in}}%
\pgfpathlineto{\pgfqpoint{4.261149in}{1.945813in}}%
\pgfpathlineto{\pgfqpoint{4.253158in}{1.936297in}}%
\pgfpathclose%
\pgfusepath{fill}%
\end{pgfscope}%
\begin{pgfscope}%
\pgfpathrectangle{\pgfqpoint{1.150000in}{0.150000in}}{\pgfqpoint{5.700000in}{5.700000in}}%
\pgfusepath{clip}%
\pgfsetbuttcap%
\pgfsetroundjoin%
\definecolor{currentfill}{rgb}{0.223925,0.334994,0.548053}%
\pgfsetfillcolor{currentfill}%
\pgfsetfillopacity{0.700000}%
\pgfsetlinewidth{0.000000pt}%
\definecolor{currentstroke}{rgb}{0.000000,0.000000,0.000000}%
\pgfsetstrokecolor{currentstroke}%
\pgfsetdash{}{0pt}%
\pgfpathmoveto{\pgfqpoint{5.189419in}{2.347191in}}%
\pgfpathlineto{\pgfqpoint{5.203789in}{2.350474in}}%
\pgfpathlineto{\pgfqpoint{5.218170in}{2.353828in}}%
\pgfpathlineto{\pgfqpoint{5.232564in}{2.357252in}}%
\pgfpathlineto{\pgfqpoint{5.246970in}{2.360746in}}%
\pgfpathlineto{\pgfqpoint{5.254540in}{2.366415in}}%
\pgfpathlineto{\pgfqpoint{5.262102in}{2.372015in}}%
\pgfpathlineto{\pgfqpoint{5.269657in}{2.377550in}}%
\pgfpathlineto{\pgfqpoint{5.277204in}{2.383023in}}%
\pgfpathlineto{\pgfqpoint{5.262815in}{2.379704in}}%
\pgfpathlineto{\pgfqpoint{5.248439in}{2.376455in}}%
\pgfpathlineto{\pgfqpoint{5.234074in}{2.373275in}}%
\pgfpathlineto{\pgfqpoint{5.219721in}{2.370166in}}%
\pgfpathlineto{\pgfqpoint{5.212157in}{2.364510in}}%
\pgfpathlineto{\pgfqpoint{5.204586in}{2.358798in}}%
\pgfpathlineto{\pgfqpoint{5.197006in}{2.353026in}}%
\pgfpathlineto{\pgfqpoint{5.189419in}{2.347191in}}%
\pgfpathclose%
\pgfusepath{fill}%
\end{pgfscope}%
\begin{pgfscope}%
\pgfpathrectangle{\pgfqpoint{1.150000in}{0.150000in}}{\pgfqpoint{5.700000in}{5.700000in}}%
\pgfusepath{clip}%
\pgfsetbuttcap%
\pgfsetroundjoin%
\definecolor{currentfill}{rgb}{0.282884,0.135920,0.453427}%
\pgfsetfillcolor{currentfill}%
\pgfsetfillopacity{0.700000}%
\pgfsetlinewidth{0.000000pt}%
\definecolor{currentstroke}{rgb}{0.000000,0.000000,0.000000}%
\pgfsetstrokecolor{currentstroke}%
\pgfsetdash{}{0pt}%
\pgfpathmoveto{\pgfqpoint{4.165132in}{1.895786in}}%
\pgfpathlineto{\pgfqpoint{4.179120in}{1.896128in}}%
\pgfpathlineto{\pgfqpoint{4.193116in}{1.896544in}}%
\pgfpathlineto{\pgfqpoint{4.207121in}{1.897034in}}%
\pgfpathlineto{\pgfqpoint{4.221136in}{1.897598in}}%
\pgfpathlineto{\pgfqpoint{4.229150in}{1.907367in}}%
\pgfpathlineto{\pgfqpoint{4.237159in}{1.917073in}}%
\pgfpathlineto{\pgfqpoint{4.245161in}{1.926717in}}%
\pgfpathlineto{\pgfqpoint{4.253158in}{1.936297in}}%
\pgfpathlineto{\pgfqpoint{4.239153in}{1.935673in}}%
\pgfpathlineto{\pgfqpoint{4.225157in}{1.935123in}}%
\pgfpathlineto{\pgfqpoint{4.211170in}{1.934646in}}%
\pgfpathlineto{\pgfqpoint{4.197192in}{1.934243in}}%
\pgfpathlineto{\pgfqpoint{4.189186in}{1.924715in}}%
\pgfpathlineto{\pgfqpoint{4.181173in}{1.915130in}}%
\pgfpathlineto{\pgfqpoint{4.173156in}{1.905486in}}%
\pgfpathlineto{\pgfqpoint{4.165132in}{1.895786in}}%
\pgfpathclose%
\pgfusepath{fill}%
\end{pgfscope}%
\begin{pgfscope}%
\pgfpathrectangle{\pgfqpoint{1.150000in}{0.150000in}}{\pgfqpoint{5.700000in}{5.700000in}}%
\pgfusepath{clip}%
\pgfsetbuttcap%
\pgfsetroundjoin%
\definecolor{currentfill}{rgb}{0.276022,0.044167,0.370164}%
\pgfsetfillcolor{currentfill}%
\pgfsetfillopacity{0.700000}%
\pgfsetlinewidth{0.000000pt}%
\definecolor{currentstroke}{rgb}{0.000000,0.000000,0.000000}%
\pgfsetstrokecolor{currentstroke}%
\pgfsetdash{}{0pt}%
\pgfpathmoveto{\pgfqpoint{2.772920in}{1.744511in}}%
\pgfpathlineto{\pgfqpoint{2.786679in}{1.736608in}}%
\pgfpathlineto{\pgfqpoint{2.800440in}{1.728801in}}%
\pgfpathlineto{\pgfqpoint{2.814202in}{1.721088in}}%
\pgfpathlineto{\pgfqpoint{2.827967in}{1.713470in}}%
\pgfpathlineto{\pgfqpoint{2.836586in}{1.718261in}}%
\pgfpathlineto{\pgfqpoint{2.845193in}{1.723208in}}%
\pgfpathlineto{\pgfqpoint{2.853789in}{1.728306in}}%
\pgfpathlineto{\pgfqpoint{2.862374in}{1.733551in}}%
\pgfpathlineto{\pgfqpoint{2.848635in}{1.740860in}}%
\pgfpathlineto{\pgfqpoint{2.834898in}{1.748263in}}%
\pgfpathlineto{\pgfqpoint{2.821163in}{1.755760in}}%
\pgfpathlineto{\pgfqpoint{2.807430in}{1.763353in}}%
\pgfpathlineto{\pgfqpoint{2.798820in}{1.758410in}}%
\pgfpathlineto{\pgfqpoint{2.790199in}{1.753618in}}%
\pgfpathlineto{\pgfqpoint{2.781565in}{1.748984in}}%
\pgfpathlineto{\pgfqpoint{2.772920in}{1.744511in}}%
\pgfpathclose%
\pgfusepath{fill}%
\end{pgfscope}%
\begin{pgfscope}%
\pgfpathrectangle{\pgfqpoint{1.150000in}{0.150000in}}{\pgfqpoint{5.700000in}{5.700000in}}%
\pgfusepath{clip}%
\pgfsetbuttcap%
\pgfsetroundjoin%
\definecolor{currentfill}{rgb}{0.268510,0.009605,0.335427}%
\pgfsetfillcolor{currentfill}%
\pgfsetfillopacity{0.700000}%
\pgfsetlinewidth{0.000000pt}%
\definecolor{currentstroke}{rgb}{0.000000,0.000000,0.000000}%
\pgfsetstrokecolor{currentstroke}%
\pgfsetdash{}{0pt}%
\pgfpathmoveto{\pgfqpoint{3.348917in}{1.660506in}}%
\pgfpathlineto{\pgfqpoint{3.362709in}{1.656531in}}%
\pgfpathlineto{\pgfqpoint{3.376506in}{1.652638in}}%
\pgfpathlineto{\pgfqpoint{3.390308in}{1.648827in}}%
\pgfpathlineto{\pgfqpoint{3.404116in}{1.645097in}}%
\pgfpathlineto{\pgfqpoint{3.412430in}{1.653740in}}%
\pgfpathlineto{\pgfqpoint{3.420736in}{1.662433in}}%
\pgfpathlineto{\pgfqpoint{3.429036in}{1.671175in}}%
\pgfpathlineto{\pgfqpoint{3.437329in}{1.679961in}}%
\pgfpathlineto{\pgfqpoint{3.423537in}{1.683466in}}%
\pgfpathlineto{\pgfqpoint{3.409749in}{1.687052in}}%
\pgfpathlineto{\pgfqpoint{3.395968in}{1.690720in}}%
\pgfpathlineto{\pgfqpoint{3.382192in}{1.694469in}}%
\pgfpathlineto{\pgfqpoint{3.373884in}{1.685901in}}%
\pgfpathlineto{\pgfqpoint{3.365568in}{1.677382in}}%
\pgfpathlineto{\pgfqpoint{3.357246in}{1.668916in}}%
\pgfpathlineto{\pgfqpoint{3.348917in}{1.660506in}}%
\pgfpathclose%
\pgfusepath{fill}%
\end{pgfscope}%
\begin{pgfscope}%
\pgfpathrectangle{\pgfqpoint{1.150000in}{0.150000in}}{\pgfqpoint{5.700000in}{5.700000in}}%
\pgfusepath{clip}%
\pgfsetbuttcap%
\pgfsetroundjoin%
\definecolor{currentfill}{rgb}{0.281924,0.089666,0.412415}%
\pgfsetfillcolor{currentfill}%
\pgfsetfillopacity{0.700000}%
\pgfsetlinewidth{0.000000pt}%
\definecolor{currentstroke}{rgb}{0.000000,0.000000,0.000000}%
\pgfsetstrokecolor{currentstroke}%
\pgfsetdash{}{0pt}%
\pgfpathmoveto{\pgfqpoint{2.572876in}{1.834811in}}%
\pgfpathlineto{\pgfqpoint{2.586649in}{1.825383in}}%
\pgfpathlineto{\pgfqpoint{2.600423in}{1.816058in}}%
\pgfpathlineto{\pgfqpoint{2.614198in}{1.806835in}}%
\pgfpathlineto{\pgfqpoint{2.627974in}{1.797714in}}%
\pgfpathlineto{\pgfqpoint{2.636727in}{1.800812in}}%
\pgfpathlineto{\pgfqpoint{2.645468in}{1.804103in}}%
\pgfpathlineto{\pgfqpoint{2.654194in}{1.807580in}}%
\pgfpathlineto{\pgfqpoint{2.662908in}{1.811239in}}%
\pgfpathlineto{\pgfqpoint{2.649162in}{1.820028in}}%
\pgfpathlineto{\pgfqpoint{2.635418in}{1.828918in}}%
\pgfpathlineto{\pgfqpoint{2.621674in}{1.837910in}}%
\pgfpathlineto{\pgfqpoint{2.607931in}{1.847005in}}%
\pgfpathlineto{\pgfqpoint{2.599188in}{1.843671in}}%
\pgfpathlineto{\pgfqpoint{2.590431in}{1.840524in}}%
\pgfpathlineto{\pgfqpoint{2.581661in}{1.837569in}}%
\pgfpathlineto{\pgfqpoint{2.572876in}{1.834811in}}%
\pgfpathclose%
\pgfusepath{fill}%
\end{pgfscope}%
\begin{pgfscope}%
\pgfpathrectangle{\pgfqpoint{1.150000in}{0.150000in}}{\pgfqpoint{5.700000in}{5.700000in}}%
\pgfusepath{clip}%
\pgfsetbuttcap%
\pgfsetroundjoin%
\definecolor{currentfill}{rgb}{0.229739,0.322361,0.545706}%
\pgfsetfillcolor{currentfill}%
\pgfsetfillopacity{0.700000}%
\pgfsetlinewidth{0.000000pt}%
\definecolor{currentstroke}{rgb}{0.000000,0.000000,0.000000}%
\pgfsetstrokecolor{currentstroke}%
\pgfsetdash{}{0pt}%
\pgfpathmoveto{\pgfqpoint{5.101573in}{2.310110in}}%
\pgfpathlineto{\pgfqpoint{5.115910in}{2.313266in}}%
\pgfpathlineto{\pgfqpoint{5.130260in}{2.316492in}}%
\pgfpathlineto{\pgfqpoint{5.144621in}{2.319788in}}%
\pgfpathlineto{\pgfqpoint{5.158995in}{2.323155in}}%
\pgfpathlineto{\pgfqpoint{5.166613in}{2.329274in}}%
\pgfpathlineto{\pgfqpoint{5.174223in}{2.335318in}}%
\pgfpathlineto{\pgfqpoint{5.181825in}{2.341289in}}%
\pgfpathlineto{\pgfqpoint{5.189419in}{2.347191in}}%
\pgfpathlineto{\pgfqpoint{5.175062in}{2.343977in}}%
\pgfpathlineto{\pgfqpoint{5.160716in}{2.340834in}}%
\pgfpathlineto{\pgfqpoint{5.146383in}{2.337760in}}%
\pgfpathlineto{\pgfqpoint{5.132061in}{2.334757in}}%
\pgfpathlineto{\pgfqpoint{5.124450in}{2.328695in}}%
\pgfpathlineto{\pgfqpoint{5.116832in}{2.322568in}}%
\pgfpathlineto{\pgfqpoint{5.109206in}{2.316374in}}%
\pgfpathlineto{\pgfqpoint{5.101573in}{2.310110in}}%
\pgfpathclose%
\pgfusepath{fill}%
\end{pgfscope}%
\begin{pgfscope}%
\pgfpathrectangle{\pgfqpoint{1.150000in}{0.150000in}}{\pgfqpoint{5.700000in}{5.700000in}}%
\pgfusepath{clip}%
\pgfsetbuttcap%
\pgfsetroundjoin%
\definecolor{currentfill}{rgb}{0.283229,0.120777,0.440584}%
\pgfsetfillcolor{currentfill}%
\pgfsetfillopacity{0.700000}%
\pgfsetlinewidth{0.000000pt}%
\definecolor{currentstroke}{rgb}{0.000000,0.000000,0.000000}%
\pgfsetstrokecolor{currentstroke}%
\pgfsetdash{}{0pt}%
\pgfpathmoveto{\pgfqpoint{4.077084in}{1.856120in}}%
\pgfpathlineto{\pgfqpoint{4.091045in}{1.856085in}}%
\pgfpathlineto{\pgfqpoint{4.105015in}{1.856124in}}%
\pgfpathlineto{\pgfqpoint{4.118994in}{1.856237in}}%
\pgfpathlineto{\pgfqpoint{4.132982in}{1.856424in}}%
\pgfpathlineto{\pgfqpoint{4.141028in}{1.866347in}}%
\pgfpathlineto{\pgfqpoint{4.149068in}{1.876215in}}%
\pgfpathlineto{\pgfqpoint{4.157103in}{1.886028in}}%
\pgfpathlineto{\pgfqpoint{4.165132in}{1.895786in}}%
\pgfpathlineto{\pgfqpoint{4.151154in}{1.895517in}}%
\pgfpathlineto{\pgfqpoint{4.137184in}{1.895323in}}%
\pgfpathlineto{\pgfqpoint{4.123224in}{1.895202in}}%
\pgfpathlineto{\pgfqpoint{4.109272in}{1.895156in}}%
\pgfpathlineto{\pgfqpoint{4.101233in}{1.885472in}}%
\pgfpathlineto{\pgfqpoint{4.093189in}{1.875738in}}%
\pgfpathlineto{\pgfqpoint{4.085139in}{1.865953in}}%
\pgfpathlineto{\pgfqpoint{4.077084in}{1.856120in}}%
\pgfpathclose%
\pgfusepath{fill}%
\end{pgfscope}%
\begin{pgfscope}%
\pgfpathrectangle{\pgfqpoint{1.150000in}{0.150000in}}{\pgfqpoint{5.700000in}{5.700000in}}%
\pgfusepath{clip}%
\pgfsetbuttcap%
\pgfsetroundjoin%
\definecolor{currentfill}{rgb}{0.272594,0.025563,0.353093}%
\pgfsetfillcolor{currentfill}%
\pgfsetfillopacity{0.700000}%
\pgfsetlinewidth{0.000000pt}%
\definecolor{currentstroke}{rgb}{0.000000,0.000000,0.000000}%
\pgfsetstrokecolor{currentstroke}%
\pgfsetdash{}{0pt}%
\pgfpathmoveto{\pgfqpoint{3.580879in}{1.691973in}}%
\pgfpathlineto{\pgfqpoint{3.594713in}{1.689390in}}%
\pgfpathlineto{\pgfqpoint{3.608553in}{1.686886in}}%
\pgfpathlineto{\pgfqpoint{3.622399in}{1.684461in}}%
\pgfpathlineto{\pgfqpoint{3.636252in}{1.682113in}}%
\pgfpathlineto{\pgfqpoint{3.644473in}{1.691652in}}%
\pgfpathlineto{\pgfqpoint{3.652688in}{1.701204in}}%
\pgfpathlineto{\pgfqpoint{3.660897in}{1.710764in}}%
\pgfpathlineto{\pgfqpoint{3.669100in}{1.720332in}}%
\pgfpathlineto{\pgfqpoint{3.655259in}{1.722495in}}%
\pgfpathlineto{\pgfqpoint{3.641425in}{1.724736in}}%
\pgfpathlineto{\pgfqpoint{3.627598in}{1.727056in}}%
\pgfpathlineto{\pgfqpoint{3.613777in}{1.729454in}}%
\pgfpathlineto{\pgfqpoint{3.605562in}{1.720063in}}%
\pgfpathlineto{\pgfqpoint{3.597340in}{1.710684in}}%
\pgfpathlineto{\pgfqpoint{3.589113in}{1.701320in}}%
\pgfpathlineto{\pgfqpoint{3.580879in}{1.691973in}}%
\pgfpathclose%
\pgfusepath{fill}%
\end{pgfscope}%
\begin{pgfscope}%
\pgfpathrectangle{\pgfqpoint{1.150000in}{0.150000in}}{\pgfqpoint{5.700000in}{5.700000in}}%
\pgfusepath{clip}%
\pgfsetbuttcap%
\pgfsetroundjoin%
\definecolor{currentfill}{rgb}{0.280255,0.165693,0.476498}%
\pgfsetfillcolor{currentfill}%
\pgfsetfillopacity{0.700000}%
\pgfsetlinewidth{0.000000pt}%
\definecolor{currentstroke}{rgb}{0.000000,0.000000,0.000000}%
\pgfsetstrokecolor{currentstroke}%
\pgfsetdash{}{0pt}%
\pgfpathmoveto{\pgfqpoint{2.316648in}{1.997050in}}%
\pgfpathlineto{\pgfqpoint{2.330464in}{1.985510in}}%
\pgfpathlineto{\pgfqpoint{2.344279in}{1.974086in}}%
\pgfpathlineto{\pgfqpoint{2.358092in}{1.962778in}}%
\pgfpathlineto{\pgfqpoint{2.371903in}{1.951583in}}%
\pgfpathlineto{\pgfqpoint{2.380852in}{1.952404in}}%
\pgfpathlineto{\pgfqpoint{2.389784in}{1.953460in}}%
\pgfpathlineto{\pgfqpoint{2.398700in}{1.954746in}}%
\pgfpathlineto{\pgfqpoint{2.407599in}{1.956256in}}%
\pgfpathlineto{\pgfqpoint{2.393823in}{1.967092in}}%
\pgfpathlineto{\pgfqpoint{2.380046in}{1.978042in}}%
\pgfpathlineto{\pgfqpoint{2.366267in}{1.989107in}}%
\pgfpathlineto{\pgfqpoint{2.352488in}{2.000288in}}%
\pgfpathlineto{\pgfqpoint{2.343554in}{1.999128in}}%
\pgfpathlineto{\pgfqpoint{2.334602in}{1.998198in}}%
\pgfpathlineto{\pgfqpoint{2.325634in}{1.997503in}}%
\pgfpathlineto{\pgfqpoint{2.316648in}{1.997050in}}%
\pgfpathclose%
\pgfusepath{fill}%
\end{pgfscope}%
\begin{pgfscope}%
\pgfpathrectangle{\pgfqpoint{1.150000in}{0.150000in}}{\pgfqpoint{5.700000in}{5.700000in}}%
\pgfusepath{clip}%
\pgfsetbuttcap%
\pgfsetroundjoin%
\definecolor{currentfill}{rgb}{0.235526,0.309527,0.542944}%
\pgfsetfillcolor{currentfill}%
\pgfsetfillopacity{0.700000}%
\pgfsetlinewidth{0.000000pt}%
\definecolor{currentstroke}{rgb}{0.000000,0.000000,0.000000}%
\pgfsetstrokecolor{currentstroke}%
\pgfsetdash{}{0pt}%
\pgfpathmoveto{\pgfqpoint{5.013672in}{2.271855in}}%
\pgfpathlineto{\pgfqpoint{5.027977in}{2.274860in}}%
\pgfpathlineto{\pgfqpoint{5.042294in}{2.277936in}}%
\pgfpathlineto{\pgfqpoint{5.056623in}{2.281082in}}%
\pgfpathlineto{\pgfqpoint{5.070964in}{2.284299in}}%
\pgfpathlineto{\pgfqpoint{5.078627in}{2.290871in}}%
\pgfpathlineto{\pgfqpoint{5.086284in}{2.297361in}}%
\pgfpathlineto{\pgfqpoint{5.093932in}{2.303774in}}%
\pgfpathlineto{\pgfqpoint{5.101573in}{2.310110in}}%
\pgfpathlineto{\pgfqpoint{5.087247in}{2.307025in}}%
\pgfpathlineto{\pgfqpoint{5.072933in}{2.304010in}}%
\pgfpathlineto{\pgfqpoint{5.058631in}{2.301065in}}%
\pgfpathlineto{\pgfqpoint{5.044340in}{2.298191in}}%
\pgfpathlineto{\pgfqpoint{5.036684in}{2.291715in}}%
\pgfpathlineto{\pgfqpoint{5.029021in}{2.285169in}}%
\pgfpathlineto{\pgfqpoint{5.021350in}{2.278550in}}%
\pgfpathlineto{\pgfqpoint{5.013672in}{2.271855in}}%
\pgfpathclose%
\pgfusepath{fill}%
\end{pgfscope}%
\begin{pgfscope}%
\pgfpathrectangle{\pgfqpoint{1.150000in}{0.150000in}}{\pgfqpoint{5.700000in}{5.700000in}}%
\pgfusepath{clip}%
\pgfsetbuttcap%
\pgfsetroundjoin%
\definecolor{currentfill}{rgb}{0.282656,0.100196,0.422160}%
\pgfsetfillcolor{currentfill}%
\pgfsetfillopacity{0.700000}%
\pgfsetlinewidth{0.000000pt}%
\definecolor{currentstroke}{rgb}{0.000000,0.000000,0.000000}%
\pgfsetstrokecolor{currentstroke}%
\pgfsetdash{}{0pt}%
\pgfpathmoveto{\pgfqpoint{3.989008in}{1.817611in}}%
\pgfpathlineto{\pgfqpoint{4.002945in}{1.817175in}}%
\pgfpathlineto{\pgfqpoint{4.016891in}{1.816814in}}%
\pgfpathlineto{\pgfqpoint{4.030844in}{1.816527in}}%
\pgfpathlineto{\pgfqpoint{4.044807in}{1.816315in}}%
\pgfpathlineto{\pgfqpoint{4.052884in}{1.826335in}}%
\pgfpathlineto{\pgfqpoint{4.060956in}{1.836309in}}%
\pgfpathlineto{\pgfqpoint{4.069023in}{1.846238in}}%
\pgfpathlineto{\pgfqpoint{4.077084in}{1.856120in}}%
\pgfpathlineto{\pgfqpoint{4.063131in}{1.856230in}}%
\pgfpathlineto{\pgfqpoint{4.049187in}{1.856414in}}%
\pgfpathlineto{\pgfqpoint{4.035251in}{1.856673in}}%
\pgfpathlineto{\pgfqpoint{4.021323in}{1.857007in}}%
\pgfpathlineto{\pgfqpoint{4.013253in}{1.847219in}}%
\pgfpathlineto{\pgfqpoint{4.005177in}{1.837390in}}%
\pgfpathlineto{\pgfqpoint{3.997095in}{1.827520in}}%
\pgfpathlineto{\pgfqpoint{3.989008in}{1.817611in}}%
\pgfpathclose%
\pgfusepath{fill}%
\end{pgfscope}%
\begin{pgfscope}%
\pgfpathrectangle{\pgfqpoint{1.150000in}{0.150000in}}{\pgfqpoint{5.700000in}{5.700000in}}%
\pgfusepath{clip}%
\pgfsetbuttcap%
\pgfsetroundjoin%
\definecolor{currentfill}{rgb}{0.194100,0.399323,0.555565}%
\pgfsetfillcolor{currentfill}%
\pgfsetfillopacity{0.700000}%
\pgfsetlinewidth{0.000000pt}%
\definecolor{currentstroke}{rgb}{0.000000,0.000000,0.000000}%
\pgfsetstrokecolor{currentstroke}%
\pgfsetdash{}{0pt}%
\pgfpathmoveto{\pgfqpoint{5.598155in}{2.497342in}}%
\pgfpathlineto{\pgfqpoint{5.612697in}{2.501189in}}%
\pgfpathlineto{\pgfqpoint{5.627252in}{2.505105in}}%
\pgfpathlineto{\pgfqpoint{5.641820in}{2.509091in}}%
\pgfpathlineto{\pgfqpoint{5.656401in}{2.513145in}}%
\pgfpathlineto{\pgfqpoint{5.663746in}{2.516895in}}%
\pgfpathlineto{\pgfqpoint{5.671084in}{2.520612in}}%
\pgfpathlineto{\pgfqpoint{5.678414in}{2.524302in}}%
\pgfpathlineto{\pgfqpoint{5.685736in}{2.527969in}}%
\pgfpathlineto{\pgfqpoint{5.671179in}{2.524176in}}%
\pgfpathlineto{\pgfqpoint{5.656634in}{2.520453in}}%
\pgfpathlineto{\pgfqpoint{5.642102in}{2.516798in}}%
\pgfpathlineto{\pgfqpoint{5.627583in}{2.513212in}}%
\pgfpathlineto{\pgfqpoint{5.620237in}{2.509276in}}%
\pgfpathlineto{\pgfqpoint{5.612883in}{2.505322in}}%
\pgfpathlineto{\pgfqpoint{5.605523in}{2.501346in}}%
\pgfpathlineto{\pgfqpoint{5.598155in}{2.497342in}}%
\pgfpathclose%
\pgfusepath{fill}%
\end{pgfscope}%
\begin{pgfscope}%
\pgfpathrectangle{\pgfqpoint{1.150000in}{0.150000in}}{\pgfqpoint{5.700000in}{5.700000in}}%
\pgfusepath{clip}%
\pgfsetbuttcap%
\pgfsetroundjoin%
\definecolor{currentfill}{rgb}{0.267004,0.004874,0.329415}%
\pgfsetfillcolor{currentfill}%
\pgfsetfillopacity{0.700000}%
\pgfsetlinewidth{0.000000pt}%
\definecolor{currentstroke}{rgb}{0.000000,0.000000,0.000000}%
\pgfsetstrokecolor{currentstroke}%
\pgfsetdash{}{0pt}%
\pgfpathmoveto{\pgfqpoint{3.116470in}{1.655990in}}%
\pgfpathlineto{\pgfqpoint{3.130242in}{1.650496in}}%
\pgfpathlineto{\pgfqpoint{3.144018in}{1.645088in}}%
\pgfpathlineto{\pgfqpoint{3.157798in}{1.639766in}}%
\pgfpathlineto{\pgfqpoint{3.171582in}{1.634529in}}%
\pgfpathlineto{\pgfqpoint{3.180008in}{1.641817in}}%
\pgfpathlineto{\pgfqpoint{3.188427in}{1.649199in}}%
\pgfpathlineto{\pgfqpoint{3.196837in}{1.656673in}}%
\pgfpathlineto{\pgfqpoint{3.205238in}{1.664235in}}%
\pgfpathlineto{\pgfqpoint{3.191473in}{1.669206in}}%
\pgfpathlineto{\pgfqpoint{3.177712in}{1.674261in}}%
\pgfpathlineto{\pgfqpoint{3.163956in}{1.679402in}}%
\pgfpathlineto{\pgfqpoint{3.150204in}{1.684629in}}%
\pgfpathlineto{\pgfqpoint{3.141783in}{1.677327in}}%
\pgfpathlineto{\pgfqpoint{3.133354in}{1.670116in}}%
\pgfpathlineto{\pgfqpoint{3.124917in}{1.663003in}}%
\pgfpathlineto{\pgfqpoint{3.116470in}{1.655990in}}%
\pgfpathclose%
\pgfusepath{fill}%
\end{pgfscope}%
\begin{pgfscope}%
\pgfpathrectangle{\pgfqpoint{1.150000in}{0.150000in}}{\pgfqpoint{5.700000in}{5.700000in}}%
\pgfusepath{clip}%
\pgfsetbuttcap%
\pgfsetroundjoin%
\definecolor{currentfill}{rgb}{0.241237,0.296485,0.539709}%
\pgfsetfillcolor{currentfill}%
\pgfsetfillopacity{0.700000}%
\pgfsetlinewidth{0.000000pt}%
\definecolor{currentstroke}{rgb}{0.000000,0.000000,0.000000}%
\pgfsetstrokecolor{currentstroke}%
\pgfsetdash{}{0pt}%
\pgfpathmoveto{\pgfqpoint{4.925724in}{2.232520in}}%
\pgfpathlineto{\pgfqpoint{4.939997in}{2.235352in}}%
\pgfpathlineto{\pgfqpoint{4.954281in}{2.238255in}}%
\pgfpathlineto{\pgfqpoint{4.968577in}{2.241229in}}%
\pgfpathlineto{\pgfqpoint{4.982884in}{2.244273in}}%
\pgfpathlineto{\pgfqpoint{4.990593in}{2.251294in}}%
\pgfpathlineto{\pgfqpoint{4.998293in}{2.258229in}}%
\pgfpathlineto{\pgfqpoint{5.005986in}{2.265082in}}%
\pgfpathlineto{\pgfqpoint{5.013672in}{2.271855in}}%
\pgfpathlineto{\pgfqpoint{4.999378in}{2.268921in}}%
\pgfpathlineto{\pgfqpoint{4.985096in}{2.266057in}}%
\pgfpathlineto{\pgfqpoint{4.970825in}{2.263263in}}%
\pgfpathlineto{\pgfqpoint{4.956566in}{2.260540in}}%
\pgfpathlineto{\pgfqpoint{4.948867in}{2.253650in}}%
\pgfpathlineto{\pgfqpoint{4.941160in}{2.246685in}}%
\pgfpathlineto{\pgfqpoint{4.933446in}{2.239642in}}%
\pgfpathlineto{\pgfqpoint{4.925724in}{2.232520in}}%
\pgfpathclose%
\pgfusepath{fill}%
\end{pgfscope}%
\begin{pgfscope}%
\pgfpathrectangle{\pgfqpoint{1.150000in}{0.150000in}}{\pgfqpoint{5.700000in}{5.700000in}}%
\pgfusepath{clip}%
\pgfsetbuttcap%
\pgfsetroundjoin%
\definecolor{currentfill}{rgb}{0.269944,0.014625,0.341379}%
\pgfsetfillcolor{currentfill}%
\pgfsetfillopacity{0.700000}%
\pgfsetlinewidth{0.000000pt}%
\definecolor{currentstroke}{rgb}{0.000000,0.000000,0.000000}%
\pgfsetstrokecolor{currentstroke}%
\pgfsetdash{}{0pt}%
\pgfpathmoveto{\pgfqpoint{2.972385in}{1.678413in}}%
\pgfpathlineto{\pgfqpoint{2.986149in}{1.671930in}}%
\pgfpathlineto{\pgfqpoint{2.999917in}{1.665536in}}%
\pgfpathlineto{\pgfqpoint{3.013688in}{1.659231in}}%
\pgfpathlineto{\pgfqpoint{3.027462in}{1.653014in}}%
\pgfpathlineto{\pgfqpoint{3.035966in}{1.659289in}}%
\pgfpathlineto{\pgfqpoint{3.044461in}{1.665687in}}%
\pgfpathlineto{\pgfqpoint{3.052946in}{1.672202in}}%
\pgfpathlineto{\pgfqpoint{3.061422in}{1.678831in}}%
\pgfpathlineto{\pgfqpoint{3.047670in}{1.684760in}}%
\pgfpathlineto{\pgfqpoint{3.033921in}{1.690778in}}%
\pgfpathlineto{\pgfqpoint{3.020175in}{1.696884in}}%
\pgfpathlineto{\pgfqpoint{3.006433in}{1.703079in}}%
\pgfpathlineto{\pgfqpoint{2.997935in}{1.696730in}}%
\pgfpathlineto{\pgfqpoint{2.989428in}{1.690500in}}%
\pgfpathlineto{\pgfqpoint{2.980911in}{1.684393in}}%
\pgfpathlineto{\pgfqpoint{2.972385in}{1.678413in}}%
\pgfpathclose%
\pgfusepath{fill}%
\end{pgfscope}%
\begin{pgfscope}%
\pgfpathrectangle{\pgfqpoint{1.150000in}{0.150000in}}{\pgfqpoint{5.700000in}{5.700000in}}%
\pgfusepath{clip}%
\pgfsetbuttcap%
\pgfsetroundjoin%
\definecolor{currentfill}{rgb}{0.281446,0.084320,0.407414}%
\pgfsetfillcolor{currentfill}%
\pgfsetfillopacity{0.700000}%
\pgfsetlinewidth{0.000000pt}%
\definecolor{currentstroke}{rgb}{0.000000,0.000000,0.000000}%
\pgfsetstrokecolor{currentstroke}%
\pgfsetdash{}{0pt}%
\pgfpathmoveto{\pgfqpoint{3.900898in}{1.780590in}}%
\pgfpathlineto{\pgfqpoint{3.914812in}{1.779731in}}%
\pgfpathlineto{\pgfqpoint{3.928735in}{1.778947in}}%
\pgfpathlineto{\pgfqpoint{3.942665in}{1.778238in}}%
\pgfpathlineto{\pgfqpoint{3.956604in}{1.777604in}}%
\pgfpathlineto{\pgfqpoint{3.964713in}{1.787658in}}%
\pgfpathlineto{\pgfqpoint{3.972817in}{1.797678in}}%
\pgfpathlineto{\pgfqpoint{3.980915in}{1.807663in}}%
\pgfpathlineto{\pgfqpoint{3.989008in}{1.817611in}}%
\pgfpathlineto{\pgfqpoint{3.975079in}{1.818121in}}%
\pgfpathlineto{\pgfqpoint{3.961158in}{1.818707in}}%
\pgfpathlineto{\pgfqpoint{3.947246in}{1.819369in}}%
\pgfpathlineto{\pgfqpoint{3.933341in}{1.820105in}}%
\pgfpathlineto{\pgfqpoint{3.925238in}{1.810273in}}%
\pgfpathlineto{\pgfqpoint{3.917130in}{1.800408in}}%
\pgfpathlineto{\pgfqpoint{3.909017in}{1.790514in}}%
\pgfpathlineto{\pgfqpoint{3.900898in}{1.780590in}}%
\pgfpathclose%
\pgfusepath{fill}%
\end{pgfscope}%
\begin{pgfscope}%
\pgfpathrectangle{\pgfqpoint{1.150000in}{0.150000in}}{\pgfqpoint{5.700000in}{5.700000in}}%
\pgfusepath{clip}%
\pgfsetbuttcap%
\pgfsetroundjoin%
\definecolor{currentfill}{rgb}{0.269944,0.014625,0.341379}%
\pgfsetfillcolor{currentfill}%
\pgfsetfillopacity{0.700000}%
\pgfsetlinewidth{0.000000pt}%
\definecolor{currentstroke}{rgb}{0.000000,0.000000,0.000000}%
\pgfsetstrokecolor{currentstroke}%
\pgfsetdash{}{0pt}%
\pgfpathmoveto{\pgfqpoint{3.492558in}{1.666746in}}%
\pgfpathlineto{\pgfqpoint{3.506380in}{1.663643in}}%
\pgfpathlineto{\pgfqpoint{3.520208in}{1.660618in}}%
\pgfpathlineto{\pgfqpoint{3.534042in}{1.657674in}}%
\pgfpathlineto{\pgfqpoint{3.547883in}{1.654808in}}%
\pgfpathlineto{\pgfqpoint{3.556141in}{1.664060in}}%
\pgfpathlineto{\pgfqpoint{3.564393in}{1.673340in}}%
\pgfpathlineto{\pgfqpoint{3.572639in}{1.682645in}}%
\pgfpathlineto{\pgfqpoint{3.580879in}{1.691973in}}%
\pgfpathlineto{\pgfqpoint{3.567052in}{1.694634in}}%
\pgfpathlineto{\pgfqpoint{3.553231in}{1.697374in}}%
\pgfpathlineto{\pgfqpoint{3.539417in}{1.700193in}}%
\pgfpathlineto{\pgfqpoint{3.525608in}{1.703092in}}%
\pgfpathlineto{\pgfqpoint{3.517355in}{1.693961in}}%
\pgfpathlineto{\pgfqpoint{3.509096in}{1.684858in}}%
\pgfpathlineto{\pgfqpoint{3.500830in}{1.675786in}}%
\pgfpathlineto{\pgfqpoint{3.492558in}{1.666746in}}%
\pgfpathclose%
\pgfusepath{fill}%
\end{pgfscope}%
\begin{pgfscope}%
\pgfpathrectangle{\pgfqpoint{1.150000in}{0.150000in}}{\pgfqpoint{5.700000in}{5.700000in}}%
\pgfusepath{clip}%
\pgfsetbuttcap%
\pgfsetroundjoin%
\definecolor{currentfill}{rgb}{0.248629,0.278775,0.534556}%
\pgfsetfillcolor{currentfill}%
\pgfsetfillopacity{0.700000}%
\pgfsetlinewidth{0.000000pt}%
\definecolor{currentstroke}{rgb}{0.000000,0.000000,0.000000}%
\pgfsetstrokecolor{currentstroke}%
\pgfsetdash{}{0pt}%
\pgfpathmoveto{\pgfqpoint{4.837736in}{2.192222in}}%
\pgfpathlineto{\pgfqpoint{4.851976in}{2.194859in}}%
\pgfpathlineto{\pgfqpoint{4.866228in}{2.197566in}}%
\pgfpathlineto{\pgfqpoint{4.880490in}{2.200345in}}%
\pgfpathlineto{\pgfqpoint{4.894765in}{2.203195in}}%
\pgfpathlineto{\pgfqpoint{4.902515in}{2.210655in}}%
\pgfpathlineto{\pgfqpoint{4.910259in}{2.218028in}}%
\pgfpathlineto{\pgfqpoint{4.917995in}{2.225316in}}%
\pgfpathlineto{\pgfqpoint{4.925724in}{2.232520in}}%
\pgfpathlineto{\pgfqpoint{4.911463in}{2.229759in}}%
\pgfpathlineto{\pgfqpoint{4.897212in}{2.227068in}}%
\pgfpathlineto{\pgfqpoint{4.882974in}{2.224449in}}%
\pgfpathlineto{\pgfqpoint{4.868746in}{2.221900in}}%
\pgfpathlineto{\pgfqpoint{4.861004in}{2.214600in}}%
\pgfpathlineto{\pgfqpoint{4.853255in}{2.207222in}}%
\pgfpathlineto{\pgfqpoint{4.845499in}{2.199763in}}%
\pgfpathlineto{\pgfqpoint{4.837736in}{2.192222in}}%
\pgfpathclose%
\pgfusepath{fill}%
\end{pgfscope}%
\begin{pgfscope}%
\pgfpathrectangle{\pgfqpoint{1.150000in}{0.150000in}}{\pgfqpoint{5.700000in}{5.700000in}}%
\pgfusepath{clip}%
\pgfsetbuttcap%
\pgfsetroundjoin%
\definecolor{currentfill}{rgb}{0.281887,0.150881,0.465405}%
\pgfsetfillcolor{currentfill}%
\pgfsetfillopacity{0.700000}%
\pgfsetlinewidth{0.000000pt}%
\definecolor{currentstroke}{rgb}{0.000000,0.000000,0.000000}%
\pgfsetstrokecolor{currentstroke}%
\pgfsetdash{}{0pt}%
\pgfpathmoveto{\pgfqpoint{2.371903in}{1.951583in}}%
\pgfpathlineto{\pgfqpoint{2.385714in}{1.940502in}}%
\pgfpathlineto{\pgfqpoint{2.399524in}{1.929534in}}%
\pgfpathlineto{\pgfqpoint{2.413332in}{1.918677in}}%
\pgfpathlineto{\pgfqpoint{2.427140in}{1.907931in}}%
\pgfpathlineto{\pgfqpoint{2.436053in}{1.909118in}}%
\pgfpathlineto{\pgfqpoint{2.444950in}{1.910534in}}%
\pgfpathlineto{\pgfqpoint{2.453831in}{1.912175in}}%
\pgfpathlineto{\pgfqpoint{2.462695in}{1.914034in}}%
\pgfpathlineto{\pgfqpoint{2.448922in}{1.924423in}}%
\pgfpathlineto{\pgfqpoint{2.435148in}{1.934922in}}%
\pgfpathlineto{\pgfqpoint{2.421374in}{1.945533in}}%
\pgfpathlineto{\pgfqpoint{2.407599in}{1.956256in}}%
\pgfpathlineto{\pgfqpoint{2.398700in}{1.954746in}}%
\pgfpathlineto{\pgfqpoint{2.389784in}{1.953460in}}%
\pgfpathlineto{\pgfqpoint{2.380852in}{1.952404in}}%
\pgfpathlineto{\pgfqpoint{2.371903in}{1.951583in}}%
\pgfpathclose%
\pgfusepath{fill}%
\end{pgfscope}%
\begin{pgfscope}%
\pgfpathrectangle{\pgfqpoint{1.150000in}{0.150000in}}{\pgfqpoint{5.700000in}{5.700000in}}%
\pgfusepath{clip}%
\pgfsetbuttcap%
\pgfsetroundjoin%
\definecolor{currentfill}{rgb}{0.267004,0.004874,0.329415}%
\pgfsetfillcolor{currentfill}%
\pgfsetfillopacity{0.700000}%
\pgfsetlinewidth{0.000000pt}%
\definecolor{currentstroke}{rgb}{0.000000,0.000000,0.000000}%
\pgfsetstrokecolor{currentstroke}%
\pgfsetdash{}{0pt}%
\pgfpathmoveto{\pgfqpoint{3.260344in}{1.645197in}}%
\pgfpathlineto{\pgfqpoint{3.274133in}{1.640647in}}%
\pgfpathlineto{\pgfqpoint{3.287926in}{1.636179in}}%
\pgfpathlineto{\pgfqpoint{3.301724in}{1.631795in}}%
\pgfpathlineto{\pgfqpoint{3.315527in}{1.627492in}}%
\pgfpathlineto{\pgfqpoint{3.323886in}{1.635645in}}%
\pgfpathlineto{\pgfqpoint{3.332237in}{1.643867in}}%
\pgfpathlineto{\pgfqpoint{3.340580in}{1.652155in}}%
\pgfpathlineto{\pgfqpoint{3.348917in}{1.660506in}}%
\pgfpathlineto{\pgfqpoint{3.335130in}{1.664562in}}%
\pgfpathlineto{\pgfqpoint{3.321349in}{1.668701in}}%
\pgfpathlineto{\pgfqpoint{3.307573in}{1.672923in}}%
\pgfpathlineto{\pgfqpoint{3.293802in}{1.677227in}}%
\pgfpathlineto{\pgfqpoint{3.285449in}{1.669114in}}%
\pgfpathlineto{\pgfqpoint{3.277088in}{1.661069in}}%
\pgfpathlineto{\pgfqpoint{3.268720in}{1.653096in}}%
\pgfpathlineto{\pgfqpoint{3.260344in}{1.645197in}}%
\pgfpathclose%
\pgfusepath{fill}%
\end{pgfscope}%
\begin{pgfscope}%
\pgfpathrectangle{\pgfqpoint{1.150000in}{0.150000in}}{\pgfqpoint{5.700000in}{5.700000in}}%
\pgfusepath{clip}%
\pgfsetbuttcap%
\pgfsetroundjoin%
\definecolor{currentfill}{rgb}{0.280894,0.078907,0.402329}%
\pgfsetfillcolor{currentfill}%
\pgfsetfillopacity{0.700000}%
\pgfsetlinewidth{0.000000pt}%
\definecolor{currentstroke}{rgb}{0.000000,0.000000,0.000000}%
\pgfsetstrokecolor{currentstroke}%
\pgfsetdash{}{0pt}%
\pgfpathmoveto{\pgfqpoint{2.627974in}{1.797714in}}%
\pgfpathlineto{\pgfqpoint{2.641750in}{1.788694in}}%
\pgfpathlineto{\pgfqpoint{2.655527in}{1.779774in}}%
\pgfpathlineto{\pgfqpoint{2.669305in}{1.770954in}}%
\pgfpathlineto{\pgfqpoint{2.683085in}{1.762234in}}%
\pgfpathlineto{\pgfqpoint{2.691809in}{1.765673in}}%
\pgfpathlineto{\pgfqpoint{2.700519in}{1.769298in}}%
\pgfpathlineto{\pgfqpoint{2.709217in}{1.773105in}}%
\pgfpathlineto{\pgfqpoint{2.717902in}{1.777088in}}%
\pgfpathlineto{\pgfqpoint{2.704152in}{1.785477in}}%
\pgfpathlineto{\pgfqpoint{2.690403in}{1.793964in}}%
\pgfpathlineto{\pgfqpoint{2.676655in}{1.802552in}}%
\pgfpathlineto{\pgfqpoint{2.662908in}{1.811239in}}%
\pgfpathlineto{\pgfqpoint{2.654194in}{1.807580in}}%
\pgfpathlineto{\pgfqpoint{2.645468in}{1.804103in}}%
\pgfpathlineto{\pgfqpoint{2.636727in}{1.800812in}}%
\pgfpathlineto{\pgfqpoint{2.627974in}{1.797714in}}%
\pgfpathclose%
\pgfusepath{fill}%
\end{pgfscope}%
\begin{pgfscope}%
\pgfpathrectangle{\pgfqpoint{1.150000in}{0.150000in}}{\pgfqpoint{5.700000in}{5.700000in}}%
\pgfusepath{clip}%
\pgfsetbuttcap%
\pgfsetroundjoin%
\definecolor{currentfill}{rgb}{0.199430,0.387607,0.554642}%
\pgfsetfillcolor{currentfill}%
\pgfsetfillopacity{0.700000}%
\pgfsetlinewidth{0.000000pt}%
\definecolor{currentstroke}{rgb}{0.000000,0.000000,0.000000}%
\pgfsetstrokecolor{currentstroke}%
\pgfsetdash{}{0pt}%
\pgfpathmoveto{\pgfqpoint{5.510481in}{2.465313in}}%
\pgfpathlineto{\pgfqpoint{5.524993in}{2.469123in}}%
\pgfpathlineto{\pgfqpoint{5.539518in}{2.473003in}}%
\pgfpathlineto{\pgfqpoint{5.554056in}{2.476951in}}%
\pgfpathlineto{\pgfqpoint{5.568607in}{2.480970in}}%
\pgfpathlineto{\pgfqpoint{5.576006in}{2.485125in}}%
\pgfpathlineto{\pgfqpoint{5.583397in}{2.489237in}}%
\pgfpathlineto{\pgfqpoint{5.590780in}{2.493308in}}%
\pgfpathlineto{\pgfqpoint{5.598155in}{2.497342in}}%
\pgfpathlineto{\pgfqpoint{5.583626in}{2.493565in}}%
\pgfpathlineto{\pgfqpoint{5.569110in}{2.489856in}}%
\pgfpathlineto{\pgfqpoint{5.554606in}{2.486217in}}%
\pgfpathlineto{\pgfqpoint{5.540115in}{2.482646in}}%
\pgfpathlineto{\pgfqpoint{5.532718in}{2.478364in}}%
\pgfpathlineto{\pgfqpoint{5.525313in}{2.474050in}}%
\pgfpathlineto{\pgfqpoint{5.517901in}{2.469702in}}%
\pgfpathlineto{\pgfqpoint{5.510481in}{2.465313in}}%
\pgfpathclose%
\pgfusepath{fill}%
\end{pgfscope}%
\begin{pgfscope}%
\pgfpathrectangle{\pgfqpoint{1.150000in}{0.150000in}}{\pgfqpoint{5.700000in}{5.700000in}}%
\pgfusepath{clip}%
\pgfsetbuttcap%
\pgfsetroundjoin%
\definecolor{currentfill}{rgb}{0.279566,0.067836,0.391917}%
\pgfsetfillcolor{currentfill}%
\pgfsetfillopacity{0.700000}%
\pgfsetlinewidth{0.000000pt}%
\definecolor{currentstroke}{rgb}{0.000000,0.000000,0.000000}%
\pgfsetstrokecolor{currentstroke}%
\pgfsetdash{}{0pt}%
\pgfpathmoveto{\pgfqpoint{3.812743in}{1.745413in}}%
\pgfpathlineto{\pgfqpoint{3.826637in}{1.744107in}}%
\pgfpathlineto{\pgfqpoint{3.840539in}{1.742877in}}%
\pgfpathlineto{\pgfqpoint{3.854448in}{1.741723in}}%
\pgfpathlineto{\pgfqpoint{3.868365in}{1.740644in}}%
\pgfpathlineto{\pgfqpoint{3.876507in}{1.750665in}}%
\pgfpathlineto{\pgfqpoint{3.884643in}{1.760664in}}%
\pgfpathlineto{\pgfqpoint{3.892773in}{1.770640in}}%
\pgfpathlineto{\pgfqpoint{3.900898in}{1.780590in}}%
\pgfpathlineto{\pgfqpoint{3.886991in}{1.781525in}}%
\pgfpathlineto{\pgfqpoint{3.873092in}{1.782536in}}%
\pgfpathlineto{\pgfqpoint{3.859201in}{1.783623in}}%
\pgfpathlineto{\pgfqpoint{3.845317in}{1.784786in}}%
\pgfpathlineto{\pgfqpoint{3.837182in}{1.774971in}}%
\pgfpathlineto{\pgfqpoint{3.829041in}{1.765136in}}%
\pgfpathlineto{\pgfqpoint{3.820895in}{1.755283in}}%
\pgfpathlineto{\pgfqpoint{3.812743in}{1.745413in}}%
\pgfpathclose%
\pgfusepath{fill}%
\end{pgfscope}%
\begin{pgfscope}%
\pgfpathrectangle{\pgfqpoint{1.150000in}{0.150000in}}{\pgfqpoint{5.700000in}{5.700000in}}%
\pgfusepath{clip}%
\pgfsetbuttcap%
\pgfsetroundjoin%
\definecolor{currentfill}{rgb}{0.255645,0.260703,0.528312}%
\pgfsetfillcolor{currentfill}%
\pgfsetfillopacity{0.700000}%
\pgfsetlinewidth{0.000000pt}%
\definecolor{currentstroke}{rgb}{0.000000,0.000000,0.000000}%
\pgfsetstrokecolor{currentstroke}%
\pgfsetdash{}{0pt}%
\pgfpathmoveto{\pgfqpoint{4.749714in}{2.151100in}}%
\pgfpathlineto{\pgfqpoint{4.763922in}{2.153519in}}%
\pgfpathlineto{\pgfqpoint{4.778141in}{2.156009in}}%
\pgfpathlineto{\pgfqpoint{4.792371in}{2.158570in}}%
\pgfpathlineto{\pgfqpoint{4.806611in}{2.161202in}}%
\pgfpathlineto{\pgfqpoint{4.814403in}{2.169089in}}%
\pgfpathlineto{\pgfqpoint{4.822188in}{2.176886in}}%
\pgfpathlineto{\pgfqpoint{4.829965in}{2.184597in}}%
\pgfpathlineto{\pgfqpoint{4.837736in}{2.192222in}}%
\pgfpathlineto{\pgfqpoint{4.823507in}{2.189657in}}%
\pgfpathlineto{\pgfqpoint{4.809289in}{2.187162in}}%
\pgfpathlineto{\pgfqpoint{4.795082in}{2.184739in}}%
\pgfpathlineto{\pgfqpoint{4.780886in}{2.182387in}}%
\pgfpathlineto{\pgfqpoint{4.773103in}{2.174688in}}%
\pgfpathlineto{\pgfqpoint{4.765314in}{2.166908in}}%
\pgfpathlineto{\pgfqpoint{4.757517in}{2.159046in}}%
\pgfpathlineto{\pgfqpoint{4.749714in}{2.151100in}}%
\pgfpathclose%
\pgfusepath{fill}%
\end{pgfscope}%
\begin{pgfscope}%
\pgfpathrectangle{\pgfqpoint{1.150000in}{0.150000in}}{\pgfqpoint{5.700000in}{5.700000in}}%
\pgfusepath{clip}%
\pgfsetbuttcap%
\pgfsetroundjoin%
\definecolor{currentfill}{rgb}{0.274952,0.037752,0.364543}%
\pgfsetfillcolor{currentfill}%
\pgfsetfillopacity{0.700000}%
\pgfsetlinewidth{0.000000pt}%
\definecolor{currentstroke}{rgb}{0.000000,0.000000,0.000000}%
\pgfsetstrokecolor{currentstroke}%
\pgfsetdash{}{0pt}%
\pgfpathmoveto{\pgfqpoint{2.827967in}{1.713470in}}%
\pgfpathlineto{\pgfqpoint{2.841734in}{1.705946in}}%
\pgfpathlineto{\pgfqpoint{2.855504in}{1.698515in}}%
\pgfpathlineto{\pgfqpoint{2.869276in}{1.691177in}}%
\pgfpathlineto{\pgfqpoint{2.883050in}{1.683931in}}%
\pgfpathlineto{\pgfqpoint{2.891643in}{1.689038in}}%
\pgfpathlineto{\pgfqpoint{2.900226in}{1.694297in}}%
\pgfpathlineto{\pgfqpoint{2.908797in}{1.699702in}}%
\pgfpathlineto{\pgfqpoint{2.917357in}{1.705249in}}%
\pgfpathlineto{\pgfqpoint{2.903608in}{1.712186in}}%
\pgfpathlineto{\pgfqpoint{2.889861in}{1.719215in}}%
\pgfpathlineto{\pgfqpoint{2.876116in}{1.726336in}}%
\pgfpathlineto{\pgfqpoint{2.862374in}{1.733551in}}%
\pgfpathlineto{\pgfqpoint{2.853789in}{1.728306in}}%
\pgfpathlineto{\pgfqpoint{2.845193in}{1.723208in}}%
\pgfpathlineto{\pgfqpoint{2.836586in}{1.718261in}}%
\pgfpathlineto{\pgfqpoint{2.827967in}{1.713470in}}%
\pgfpathclose%
\pgfusepath{fill}%
\end{pgfscope}%
\begin{pgfscope}%
\pgfpathrectangle{\pgfqpoint{1.150000in}{0.150000in}}{\pgfqpoint{5.700000in}{5.700000in}}%
\pgfusepath{clip}%
\pgfsetbuttcap%
\pgfsetroundjoin%
\definecolor{currentfill}{rgb}{0.262138,0.242286,0.520837}%
\pgfsetfillcolor{currentfill}%
\pgfsetfillopacity{0.700000}%
\pgfsetlinewidth{0.000000pt}%
\definecolor{currentstroke}{rgb}{0.000000,0.000000,0.000000}%
\pgfsetstrokecolor{currentstroke}%
\pgfsetdash{}{0pt}%
\pgfpathmoveto{\pgfqpoint{4.661665in}{2.109316in}}%
\pgfpathlineto{\pgfqpoint{4.675840in}{2.111494in}}%
\pgfpathlineto{\pgfqpoint{4.690026in}{2.113743in}}%
\pgfpathlineto{\pgfqpoint{4.704223in}{2.116064in}}%
\pgfpathlineto{\pgfqpoint{4.718431in}{2.118457in}}%
\pgfpathlineto{\pgfqpoint{4.726262in}{2.126749in}}%
\pgfpathlineto{\pgfqpoint{4.734087in}{2.134953in}}%
\pgfpathlineto{\pgfqpoint{4.741904in}{2.143070in}}%
\pgfpathlineto{\pgfqpoint{4.749714in}{2.151100in}}%
\pgfpathlineto{\pgfqpoint{4.735517in}{2.148753in}}%
\pgfpathlineto{\pgfqpoint{4.721331in}{2.146478in}}%
\pgfpathlineto{\pgfqpoint{4.707156in}{2.144273in}}%
\pgfpathlineto{\pgfqpoint{4.692992in}{2.142141in}}%
\pgfpathlineto{\pgfqpoint{4.685170in}{2.134057in}}%
\pgfpathlineto{\pgfqpoint{4.677342in}{2.125893in}}%
\pgfpathlineto{\pgfqpoint{4.669507in}{2.117646in}}%
\pgfpathlineto{\pgfqpoint{4.661665in}{2.109316in}}%
\pgfpathclose%
\pgfusepath{fill}%
\end{pgfscope}%
\begin{pgfscope}%
\pgfpathrectangle{\pgfqpoint{1.150000in}{0.150000in}}{\pgfqpoint{5.700000in}{5.700000in}}%
\pgfusepath{clip}%
\pgfsetbuttcap%
\pgfsetroundjoin%
\definecolor{currentfill}{rgb}{0.233603,0.313828,0.543914}%
\pgfsetfillcolor{currentfill}%
\pgfsetfillopacity{0.700000}%
\pgfsetlinewidth{0.000000pt}%
\definecolor{currentstroke}{rgb}{0.000000,0.000000,0.000000}%
\pgfsetstrokecolor{currentstroke}%
\pgfsetdash{}{0pt}%
\pgfpathmoveto{\pgfqpoint{1.947142in}{2.321427in}}%
\pgfpathlineto{\pgfqpoint{1.961079in}{2.306398in}}%
\pgfpathlineto{\pgfqpoint{1.975012in}{2.291512in}}%
\pgfpathlineto{\pgfqpoint{1.988939in}{2.276768in}}%
\pgfpathlineto{\pgfqpoint{2.002861in}{2.262164in}}%
\pgfpathlineto{\pgfqpoint{2.012133in}{2.259624in}}%
\pgfpathlineto{\pgfqpoint{2.021384in}{2.257376in}}%
\pgfpathlineto{\pgfqpoint{2.030612in}{2.255416in}}%
\pgfpathlineto{\pgfqpoint{2.039819in}{2.253737in}}%
\pgfpathlineto{\pgfqpoint{2.025942in}{2.267949in}}%
\pgfpathlineto{\pgfqpoint{2.012059in}{2.282302in}}%
\pgfpathlineto{\pgfqpoint{1.998172in}{2.296796in}}%
\pgfpathlineto{\pgfqpoint{1.984281in}{2.311433in}}%
\pgfpathlineto{\pgfqpoint{1.975029in}{2.313495in}}%
\pgfpathlineto{\pgfqpoint{1.965756in}{2.315844in}}%
\pgfpathlineto{\pgfqpoint{1.956460in}{2.318486in}}%
\pgfpathlineto{\pgfqpoint{1.947142in}{2.321427in}}%
\pgfpathclose%
\pgfusepath{fill}%
\end{pgfscope}%
\begin{pgfscope}%
\pgfpathrectangle{\pgfqpoint{1.150000in}{0.150000in}}{\pgfqpoint{5.700000in}{5.700000in}}%
\pgfusepath{clip}%
\pgfsetbuttcap%
\pgfsetroundjoin%
\definecolor{currentfill}{rgb}{0.266580,0.228262,0.514349}%
\pgfsetfillcolor{currentfill}%
\pgfsetfillopacity{0.700000}%
\pgfsetlinewidth{0.000000pt}%
\definecolor{currentstroke}{rgb}{0.000000,0.000000,0.000000}%
\pgfsetstrokecolor{currentstroke}%
\pgfsetdash{}{0pt}%
\pgfpathmoveto{\pgfqpoint{4.573592in}{2.067051in}}%
\pgfpathlineto{\pgfqpoint{4.587736in}{2.068966in}}%
\pgfpathlineto{\pgfqpoint{4.601890in}{2.070952in}}%
\pgfpathlineto{\pgfqpoint{4.616055in}{2.073011in}}%
\pgfpathlineto{\pgfqpoint{4.630230in}{2.075141in}}%
\pgfpathlineto{\pgfqpoint{4.638099in}{2.083815in}}%
\pgfpathlineto{\pgfqpoint{4.645961in}{2.092401in}}%
\pgfpathlineto{\pgfqpoint{4.653816in}{2.100901in}}%
\pgfpathlineto{\pgfqpoint{4.661665in}{2.109316in}}%
\pgfpathlineto{\pgfqpoint{4.647500in}{2.107210in}}%
\pgfpathlineto{\pgfqpoint{4.633346in}{2.105175in}}%
\pgfpathlineto{\pgfqpoint{4.619202in}{2.103213in}}%
\pgfpathlineto{\pgfqpoint{4.605069in}{2.101322in}}%
\pgfpathlineto{\pgfqpoint{4.597209in}{2.092875in}}%
\pgfpathlineto{\pgfqpoint{4.589344in}{2.084349in}}%
\pgfpathlineto{\pgfqpoint{4.581471in}{2.075741in}}%
\pgfpathlineto{\pgfqpoint{4.573592in}{2.067051in}}%
\pgfpathclose%
\pgfusepath{fill}%
\end{pgfscope}%
\begin{pgfscope}%
\pgfpathrectangle{\pgfqpoint{1.150000in}{0.150000in}}{\pgfqpoint{5.700000in}{5.700000in}}%
\pgfusepath{clip}%
\pgfsetbuttcap%
\pgfsetroundjoin%
\definecolor{currentfill}{rgb}{0.204903,0.375746,0.553533}%
\pgfsetfillcolor{currentfill}%
\pgfsetfillopacity{0.700000}%
\pgfsetlinewidth{0.000000pt}%
\definecolor{currentstroke}{rgb}{0.000000,0.000000,0.000000}%
\pgfsetstrokecolor{currentstroke}%
\pgfsetdash{}{0pt}%
\pgfpathmoveto{\pgfqpoint{5.422721in}{2.431864in}}%
\pgfpathlineto{\pgfqpoint{5.437202in}{2.435614in}}%
\pgfpathlineto{\pgfqpoint{5.451697in}{2.439434in}}%
\pgfpathlineto{\pgfqpoint{5.466204in}{2.443324in}}%
\pgfpathlineto{\pgfqpoint{5.480724in}{2.447284in}}%
\pgfpathlineto{\pgfqpoint{5.488175in}{2.451870in}}%
\pgfpathlineto{\pgfqpoint{5.495618in}{2.456402in}}%
\pgfpathlineto{\pgfqpoint{5.503053in}{2.460881in}}%
\pgfpathlineto{\pgfqpoint{5.510481in}{2.465313in}}%
\pgfpathlineto{\pgfqpoint{5.495981in}{2.461573in}}%
\pgfpathlineto{\pgfqpoint{5.481495in}{2.457902in}}%
\pgfpathlineto{\pgfqpoint{5.467020in}{2.454300in}}%
\pgfpathlineto{\pgfqpoint{5.452559in}{2.450768in}}%
\pgfpathlineto{\pgfqpoint{5.445111in}{2.446110in}}%
\pgfpathlineto{\pgfqpoint{5.437655in}{2.441409in}}%
\pgfpathlineto{\pgfqpoint{5.430192in}{2.436662in}}%
\pgfpathlineto{\pgfqpoint{5.422721in}{2.431864in}}%
\pgfpathclose%
\pgfusepath{fill}%
\end{pgfscope}%
\begin{pgfscope}%
\pgfpathrectangle{\pgfqpoint{1.150000in}{0.150000in}}{\pgfqpoint{5.700000in}{5.700000in}}%
\pgfusepath{clip}%
\pgfsetbuttcap%
\pgfsetroundjoin%
\definecolor{currentfill}{rgb}{0.277018,0.050344,0.375715}%
\pgfsetfillcolor{currentfill}%
\pgfsetfillopacity{0.700000}%
\pgfsetlinewidth{0.000000pt}%
\definecolor{currentstroke}{rgb}{0.000000,0.000000,0.000000}%
\pgfsetstrokecolor{currentstroke}%
\pgfsetdash{}{0pt}%
\pgfpathmoveto{\pgfqpoint{3.724533in}{1.712455in}}%
\pgfpathlineto{\pgfqpoint{3.738409in}{1.710679in}}%
\pgfpathlineto{\pgfqpoint{3.752292in}{1.708979in}}%
\pgfpathlineto{\pgfqpoint{3.766182in}{1.707357in}}%
\pgfpathlineto{\pgfqpoint{3.780080in}{1.705810in}}%
\pgfpathlineto{\pgfqpoint{3.788254in}{1.715725in}}%
\pgfpathlineto{\pgfqpoint{3.796423in}{1.725632in}}%
\pgfpathlineto{\pgfqpoint{3.804586in}{1.735529in}}%
\pgfpathlineto{\pgfqpoint{3.812743in}{1.745413in}}%
\pgfpathlineto{\pgfqpoint{3.798857in}{1.746796in}}%
\pgfpathlineto{\pgfqpoint{3.784978in}{1.748255in}}%
\pgfpathlineto{\pgfqpoint{3.771106in}{1.749790in}}%
\pgfpathlineto{\pgfqpoint{3.757241in}{1.751402in}}%
\pgfpathlineto{\pgfqpoint{3.749073in}{1.741674in}}%
\pgfpathlineto{\pgfqpoint{3.740899in}{1.731939in}}%
\pgfpathlineto{\pgfqpoint{3.732719in}{1.722198in}}%
\pgfpathlineto{\pgfqpoint{3.724533in}{1.712455in}}%
\pgfpathclose%
\pgfusepath{fill}%
\end{pgfscope}%
\begin{pgfscope}%
\pgfpathrectangle{\pgfqpoint{1.150000in}{0.150000in}}{\pgfqpoint{5.700000in}{5.700000in}}%
\pgfusepath{clip}%
\pgfsetbuttcap%
\pgfsetroundjoin%
\definecolor{currentfill}{rgb}{0.271828,0.209303,0.504434}%
\pgfsetfillcolor{currentfill}%
\pgfsetfillopacity{0.700000}%
\pgfsetlinewidth{0.000000pt}%
\definecolor{currentstroke}{rgb}{0.000000,0.000000,0.000000}%
\pgfsetstrokecolor{currentstroke}%
\pgfsetdash{}{0pt}%
\pgfpathmoveto{\pgfqpoint{4.485501in}{2.024511in}}%
\pgfpathlineto{\pgfqpoint{4.499614in}{2.026140in}}%
\pgfpathlineto{\pgfqpoint{4.513736in}{2.027841in}}%
\pgfpathlineto{\pgfqpoint{4.527869in}{2.029614in}}%
\pgfpathlineto{\pgfqpoint{4.542012in}{2.031460in}}%
\pgfpathlineto{\pgfqpoint{4.549917in}{2.040484in}}%
\pgfpathlineto{\pgfqpoint{4.557815in}{2.049423in}}%
\pgfpathlineto{\pgfqpoint{4.565707in}{2.058279in}}%
\pgfpathlineto{\pgfqpoint{4.573592in}{2.067051in}}%
\pgfpathlineto{\pgfqpoint{4.559459in}{2.065209in}}%
\pgfpathlineto{\pgfqpoint{4.545336in}{2.063438in}}%
\pgfpathlineto{\pgfqpoint{4.531224in}{2.061739in}}%
\pgfpathlineto{\pgfqpoint{4.517121in}{2.060113in}}%
\pgfpathlineto{\pgfqpoint{4.509226in}{2.051330in}}%
\pgfpathlineto{\pgfqpoint{4.501324in}{2.042469in}}%
\pgfpathlineto{\pgfqpoint{4.493416in}{2.033529in}}%
\pgfpathlineto{\pgfqpoint{4.485501in}{2.024511in}}%
\pgfpathclose%
\pgfusepath{fill}%
\end{pgfscope}%
\begin{pgfscope}%
\pgfpathrectangle{\pgfqpoint{1.150000in}{0.150000in}}{\pgfqpoint{5.700000in}{5.700000in}}%
\pgfusepath{clip}%
\pgfsetbuttcap%
\pgfsetroundjoin%
\definecolor{currentfill}{rgb}{0.268510,0.009605,0.335427}%
\pgfsetfillcolor{currentfill}%
\pgfsetfillopacity{0.700000}%
\pgfsetlinewidth{0.000000pt}%
\definecolor{currentstroke}{rgb}{0.000000,0.000000,0.000000}%
\pgfsetstrokecolor{currentstroke}%
\pgfsetdash{}{0pt}%
\pgfpathmoveto{\pgfqpoint{3.404116in}{1.645097in}}%
\pgfpathlineto{\pgfqpoint{3.417929in}{1.641447in}}%
\pgfpathlineto{\pgfqpoint{3.431748in}{1.637879in}}%
\pgfpathlineto{\pgfqpoint{3.445573in}{1.634390in}}%
\pgfpathlineto{\pgfqpoint{3.459404in}{1.630981in}}%
\pgfpathlineto{\pgfqpoint{3.467702in}{1.639857in}}%
\pgfpathlineto{\pgfqpoint{3.475994in}{1.648779in}}%
\pgfpathlineto{\pgfqpoint{3.484280in}{1.657743in}}%
\pgfpathlineto{\pgfqpoint{3.492558in}{1.666746in}}%
\pgfpathlineto{\pgfqpoint{3.478742in}{1.669929in}}%
\pgfpathlineto{\pgfqpoint{3.464932in}{1.673193in}}%
\pgfpathlineto{\pgfqpoint{3.451128in}{1.676536in}}%
\pgfpathlineto{\pgfqpoint{3.437329in}{1.679961in}}%
\pgfpathlineto{\pgfqpoint{3.429036in}{1.671175in}}%
\pgfpathlineto{\pgfqpoint{3.420736in}{1.662433in}}%
\pgfpathlineto{\pgfqpoint{3.412430in}{1.653740in}}%
\pgfpathlineto{\pgfqpoint{3.404116in}{1.645097in}}%
\pgfpathclose%
\pgfusepath{fill}%
\end{pgfscope}%
\begin{pgfscope}%
\pgfpathrectangle{\pgfqpoint{1.150000in}{0.150000in}}{\pgfqpoint{5.700000in}{5.700000in}}%
\pgfusepath{clip}%
\pgfsetbuttcap%
\pgfsetroundjoin%
\definecolor{currentfill}{rgb}{0.244972,0.287675,0.537260}%
\pgfsetfillcolor{currentfill}%
\pgfsetfillopacity{0.700000}%
\pgfsetlinewidth{0.000000pt}%
\definecolor{currentstroke}{rgb}{0.000000,0.000000,0.000000}%
\pgfsetstrokecolor{currentstroke}%
\pgfsetdash{}{0pt}%
\pgfpathmoveto{\pgfqpoint{2.002861in}{2.262164in}}%
\pgfpathlineto{\pgfqpoint{2.016779in}{2.247700in}}%
\pgfpathlineto{\pgfqpoint{2.030692in}{2.233374in}}%
\pgfpathlineto{\pgfqpoint{2.044601in}{2.219184in}}%
\pgfpathlineto{\pgfqpoint{2.058505in}{2.205130in}}%
\pgfpathlineto{\pgfqpoint{2.067733in}{2.202988in}}%
\pgfpathlineto{\pgfqpoint{2.076939in}{2.201133in}}%
\pgfpathlineto{\pgfqpoint{2.086124in}{2.199559in}}%
\pgfpathlineto{\pgfqpoint{2.095288in}{2.198261in}}%
\pgfpathlineto{\pgfqpoint{2.081427in}{2.211926in}}%
\pgfpathlineto{\pgfqpoint{2.067562in}{2.225726in}}%
\pgfpathlineto{\pgfqpoint{2.053693in}{2.239663in}}%
\pgfpathlineto{\pgfqpoint{2.039819in}{2.253737in}}%
\pgfpathlineto{\pgfqpoint{2.030612in}{2.255416in}}%
\pgfpathlineto{\pgfqpoint{2.021384in}{2.257376in}}%
\pgfpathlineto{\pgfqpoint{2.012133in}{2.259624in}}%
\pgfpathlineto{\pgfqpoint{2.002861in}{2.262164in}}%
\pgfpathclose%
\pgfusepath{fill}%
\end{pgfscope}%
\begin{pgfscope}%
\pgfpathrectangle{\pgfqpoint{1.150000in}{0.150000in}}{\pgfqpoint{5.700000in}{5.700000in}}%
\pgfusepath{clip}%
\pgfsetbuttcap%
\pgfsetroundjoin%
\definecolor{currentfill}{rgb}{0.282884,0.135920,0.453427}%
\pgfsetfillcolor{currentfill}%
\pgfsetfillopacity{0.700000}%
\pgfsetlinewidth{0.000000pt}%
\definecolor{currentstroke}{rgb}{0.000000,0.000000,0.000000}%
\pgfsetstrokecolor{currentstroke}%
\pgfsetdash{}{0pt}%
\pgfpathmoveto{\pgfqpoint{2.427140in}{1.907931in}}%
\pgfpathlineto{\pgfqpoint{2.440948in}{1.897295in}}%
\pgfpathlineto{\pgfqpoint{2.454754in}{1.886768in}}%
\pgfpathlineto{\pgfqpoint{2.468561in}{1.876350in}}%
\pgfpathlineto{\pgfqpoint{2.482367in}{1.866040in}}%
\pgfpathlineto{\pgfqpoint{2.491245in}{1.867591in}}%
\pgfpathlineto{\pgfqpoint{2.500107in}{1.869367in}}%
\pgfpathlineto{\pgfqpoint{2.508954in}{1.871361in}}%
\pgfpathlineto{\pgfqpoint{2.517786in}{1.873569in}}%
\pgfpathlineto{\pgfqpoint{2.504013in}{1.883524in}}%
\pgfpathlineto{\pgfqpoint{2.490241in}{1.893585in}}%
\pgfpathlineto{\pgfqpoint{2.476468in}{1.903755in}}%
\pgfpathlineto{\pgfqpoint{2.462695in}{1.914034in}}%
\pgfpathlineto{\pgfqpoint{2.453831in}{1.912175in}}%
\pgfpathlineto{\pgfqpoint{2.444950in}{1.910534in}}%
\pgfpathlineto{\pgfqpoint{2.436053in}{1.909118in}}%
\pgfpathlineto{\pgfqpoint{2.427140in}{1.907931in}}%
\pgfpathclose%
\pgfusepath{fill}%
\end{pgfscope}%
\begin{pgfscope}%
\pgfpathrectangle{\pgfqpoint{1.150000in}{0.150000in}}{\pgfqpoint{5.700000in}{5.700000in}}%
\pgfusepath{clip}%
\pgfsetbuttcap%
\pgfsetroundjoin%
\definecolor{currentfill}{rgb}{0.276194,0.190074,0.493001}%
\pgfsetfillcolor{currentfill}%
\pgfsetfillopacity{0.700000}%
\pgfsetlinewidth{0.000000pt}%
\definecolor{currentstroke}{rgb}{0.000000,0.000000,0.000000}%
\pgfsetstrokecolor{currentstroke}%
\pgfsetdash{}{0pt}%
\pgfpathmoveto{\pgfqpoint{4.397394in}{1.981920in}}%
\pgfpathlineto{\pgfqpoint{4.411476in}{1.983241in}}%
\pgfpathlineto{\pgfqpoint{4.425568in}{1.984634in}}%
\pgfpathlineto{\pgfqpoint{4.439670in}{1.986100in}}%
\pgfpathlineto{\pgfqpoint{4.453781in}{1.987638in}}%
\pgfpathlineto{\pgfqpoint{4.461721in}{1.996977in}}%
\pgfpathlineto{\pgfqpoint{4.469654in}{2.006235in}}%
\pgfpathlineto{\pgfqpoint{4.477581in}{2.015413in}}%
\pgfpathlineto{\pgfqpoint{4.485501in}{2.024511in}}%
\pgfpathlineto{\pgfqpoint{4.471399in}{2.022954in}}%
\pgfpathlineto{\pgfqpoint{4.457307in}{2.021470in}}%
\pgfpathlineto{\pgfqpoint{4.443225in}{2.020058in}}%
\pgfpathlineto{\pgfqpoint{4.429152in}{2.018718in}}%
\pgfpathlineto{\pgfqpoint{4.421222in}{2.009631in}}%
\pgfpathlineto{\pgfqpoint{4.413286in}{2.000469in}}%
\pgfpathlineto{\pgfqpoint{4.405343in}{1.991232in}}%
\pgfpathlineto{\pgfqpoint{4.397394in}{1.981920in}}%
\pgfpathclose%
\pgfusepath{fill}%
\end{pgfscope}%
\begin{pgfscope}%
\pgfpathrectangle{\pgfqpoint{1.150000in}{0.150000in}}{\pgfqpoint{5.700000in}{5.700000in}}%
\pgfusepath{clip}%
\pgfsetbuttcap%
\pgfsetroundjoin%
\definecolor{currentfill}{rgb}{0.210503,0.363727,0.552206}%
\pgfsetfillcolor{currentfill}%
\pgfsetfillopacity{0.700000}%
\pgfsetlinewidth{0.000000pt}%
\definecolor{currentstroke}{rgb}{0.000000,0.000000,0.000000}%
\pgfsetstrokecolor{currentstroke}%
\pgfsetdash{}{0pt}%
\pgfpathmoveto{\pgfqpoint{5.334881in}{2.396999in}}%
\pgfpathlineto{\pgfqpoint{5.349331in}{2.400667in}}%
\pgfpathlineto{\pgfqpoint{5.363794in}{2.404405in}}%
\pgfpathlineto{\pgfqpoint{5.378269in}{2.408214in}}%
\pgfpathlineto{\pgfqpoint{5.392757in}{2.412092in}}%
\pgfpathlineto{\pgfqpoint{5.400260in}{2.417129in}}%
\pgfpathlineto{\pgfqpoint{5.407755in}{2.422101in}}%
\pgfpathlineto{\pgfqpoint{5.415242in}{2.427012in}}%
\pgfpathlineto{\pgfqpoint{5.422721in}{2.431864in}}%
\pgfpathlineto{\pgfqpoint{5.408252in}{2.428183in}}%
\pgfpathlineto{\pgfqpoint{5.393795in}{2.424572in}}%
\pgfpathlineto{\pgfqpoint{5.379351in}{2.421030in}}%
\pgfpathlineto{\pgfqpoint{5.364919in}{2.417558in}}%
\pgfpathlineto{\pgfqpoint{5.357421in}{2.412501in}}%
\pgfpathlineto{\pgfqpoint{5.349916in}{2.407392in}}%
\pgfpathlineto{\pgfqpoint{5.342402in}{2.402225in}}%
\pgfpathlineto{\pgfqpoint{5.334881in}{2.396999in}}%
\pgfpathclose%
\pgfusepath{fill}%
\end{pgfscope}%
\begin{pgfscope}%
\pgfpathrectangle{\pgfqpoint{1.150000in}{0.150000in}}{\pgfqpoint{5.700000in}{5.700000in}}%
\pgfusepath{clip}%
\pgfsetbuttcap%
\pgfsetroundjoin%
\definecolor{currentfill}{rgb}{0.279574,0.170599,0.479997}%
\pgfsetfillcolor{currentfill}%
\pgfsetfillopacity{0.700000}%
\pgfsetlinewidth{0.000000pt}%
\definecolor{currentstroke}{rgb}{0.000000,0.000000,0.000000}%
\pgfsetstrokecolor{currentstroke}%
\pgfsetdash{}{0pt}%
\pgfpathmoveto{\pgfqpoint{4.309273in}{1.939526in}}%
\pgfpathlineto{\pgfqpoint{4.323325in}{1.940516in}}%
\pgfpathlineto{\pgfqpoint{4.337387in}{1.941579in}}%
\pgfpathlineto{\pgfqpoint{4.351458in}{1.942715in}}%
\pgfpathlineto{\pgfqpoint{4.365539in}{1.943924in}}%
\pgfpathlineto{\pgfqpoint{4.373512in}{1.953535in}}%
\pgfpathlineto{\pgfqpoint{4.381479in}{1.963072in}}%
\pgfpathlineto{\pgfqpoint{4.389440in}{1.972533in}}%
\pgfpathlineto{\pgfqpoint{4.397394in}{1.981920in}}%
\pgfpathlineto{\pgfqpoint{4.383322in}{1.980672in}}%
\pgfpathlineto{\pgfqpoint{4.369260in}{1.979496in}}%
\pgfpathlineto{\pgfqpoint{4.355208in}{1.978394in}}%
\pgfpathlineto{\pgfqpoint{4.341165in}{1.977364in}}%
\pgfpathlineto{\pgfqpoint{4.333201in}{1.968009in}}%
\pgfpathlineto{\pgfqpoint{4.325230in}{1.958584in}}%
\pgfpathlineto{\pgfqpoint{4.317255in}{1.949090in}}%
\pgfpathlineto{\pgfqpoint{4.309273in}{1.939526in}}%
\pgfpathclose%
\pgfusepath{fill}%
\end{pgfscope}%
\begin{pgfscope}%
\pgfpathrectangle{\pgfqpoint{1.150000in}{0.150000in}}{\pgfqpoint{5.700000in}{5.700000in}}%
\pgfusepath{clip}%
\pgfsetbuttcap%
\pgfsetroundjoin%
\definecolor{currentfill}{rgb}{0.253935,0.265254,0.529983}%
\pgfsetfillcolor{currentfill}%
\pgfsetfillopacity{0.700000}%
\pgfsetlinewidth{0.000000pt}%
\definecolor{currentstroke}{rgb}{0.000000,0.000000,0.000000}%
\pgfsetstrokecolor{currentstroke}%
\pgfsetdash{}{0pt}%
\pgfpathmoveto{\pgfqpoint{2.058505in}{2.205130in}}%
\pgfpathlineto{\pgfqpoint{2.072406in}{2.191211in}}%
\pgfpathlineto{\pgfqpoint{2.086303in}{2.177424in}}%
\pgfpathlineto{\pgfqpoint{2.100195in}{2.163769in}}%
\pgfpathlineto{\pgfqpoint{2.114085in}{2.150245in}}%
\pgfpathlineto{\pgfqpoint{2.123268in}{2.148499in}}%
\pgfpathlineto{\pgfqpoint{2.132431in}{2.147035in}}%
\pgfpathlineto{\pgfqpoint{2.141573in}{2.145846in}}%
\pgfpathlineto{\pgfqpoint{2.150696in}{2.144926in}}%
\pgfpathlineto{\pgfqpoint{2.136849in}{2.158063in}}%
\pgfpathlineto{\pgfqpoint{2.122999in}{2.171330in}}%
\pgfpathlineto{\pgfqpoint{2.109145in}{2.184729in}}%
\pgfpathlineto{\pgfqpoint{2.095288in}{2.198261in}}%
\pgfpathlineto{\pgfqpoint{2.086124in}{2.199559in}}%
\pgfpathlineto{\pgfqpoint{2.076939in}{2.201133in}}%
\pgfpathlineto{\pgfqpoint{2.067733in}{2.202988in}}%
\pgfpathlineto{\pgfqpoint{2.058505in}{2.205130in}}%
\pgfpathclose%
\pgfusepath{fill}%
\end{pgfscope}%
\begin{pgfscope}%
\pgfpathrectangle{\pgfqpoint{1.150000in}{0.150000in}}{\pgfqpoint{5.700000in}{5.700000in}}%
\pgfusepath{clip}%
\pgfsetbuttcap%
\pgfsetroundjoin%
\definecolor{currentfill}{rgb}{0.279566,0.067836,0.391917}%
\pgfsetfillcolor{currentfill}%
\pgfsetfillopacity{0.700000}%
\pgfsetlinewidth{0.000000pt}%
\definecolor{currentstroke}{rgb}{0.000000,0.000000,0.000000}%
\pgfsetstrokecolor{currentstroke}%
\pgfsetdash{}{0pt}%
\pgfpathmoveto{\pgfqpoint{2.683085in}{1.762234in}}%
\pgfpathlineto{\pgfqpoint{2.696865in}{1.753612in}}%
\pgfpathlineto{\pgfqpoint{2.710647in}{1.745087in}}%
\pgfpathlineto{\pgfqpoint{2.724431in}{1.736660in}}%
\pgfpathlineto{\pgfqpoint{2.738216in}{1.728330in}}%
\pgfpathlineto{\pgfqpoint{2.746911in}{1.732109in}}%
\pgfpathlineto{\pgfqpoint{2.755593in}{1.736068in}}%
\pgfpathlineto{\pgfqpoint{2.764263in}{1.740204in}}%
\pgfpathlineto{\pgfqpoint{2.772920in}{1.744511in}}%
\pgfpathlineto{\pgfqpoint{2.759163in}{1.752510in}}%
\pgfpathlineto{\pgfqpoint{2.745408in}{1.760605in}}%
\pgfpathlineto{\pgfqpoint{2.731654in}{1.768798in}}%
\pgfpathlineto{\pgfqpoint{2.717902in}{1.777088in}}%
\pgfpathlineto{\pgfqpoint{2.709217in}{1.773105in}}%
\pgfpathlineto{\pgfqpoint{2.700519in}{1.769298in}}%
\pgfpathlineto{\pgfqpoint{2.691809in}{1.765673in}}%
\pgfpathlineto{\pgfqpoint{2.683085in}{1.762234in}}%
\pgfpathclose%
\pgfusepath{fill}%
\end{pgfscope}%
\begin{pgfscope}%
\pgfpathrectangle{\pgfqpoint{1.150000in}{0.150000in}}{\pgfqpoint{5.700000in}{5.700000in}}%
\pgfusepath{clip}%
\pgfsetbuttcap%
\pgfsetroundjoin%
\definecolor{currentfill}{rgb}{0.268510,0.009605,0.335427}%
\pgfsetfillcolor{currentfill}%
\pgfsetfillopacity{0.700000}%
\pgfsetlinewidth{0.000000pt}%
\definecolor{currentstroke}{rgb}{0.000000,0.000000,0.000000}%
\pgfsetstrokecolor{currentstroke}%
\pgfsetdash{}{0pt}%
\pgfpathmoveto{\pgfqpoint{3.027462in}{1.653014in}}%
\pgfpathlineto{\pgfqpoint{3.041240in}{1.646886in}}%
\pgfpathlineto{\pgfqpoint{3.055021in}{1.640844in}}%
\pgfpathlineto{\pgfqpoint{3.068806in}{1.634891in}}%
\pgfpathlineto{\pgfqpoint{3.082595in}{1.629023in}}%
\pgfpathlineto{\pgfqpoint{3.091077in}{1.635594in}}%
\pgfpathlineto{\pgfqpoint{3.099551in}{1.642281in}}%
\pgfpathlineto{\pgfqpoint{3.108015in}{1.649081in}}%
\pgfpathlineto{\pgfqpoint{3.116470in}{1.655990in}}%
\pgfpathlineto{\pgfqpoint{3.102702in}{1.661569in}}%
\pgfpathlineto{\pgfqpoint{3.088939in}{1.667236in}}%
\pgfpathlineto{\pgfqpoint{3.075179in}{1.672990in}}%
\pgfpathlineto{\pgfqpoint{3.061422in}{1.678831in}}%
\pgfpathlineto{\pgfqpoint{3.052946in}{1.672202in}}%
\pgfpathlineto{\pgfqpoint{3.044461in}{1.665687in}}%
\pgfpathlineto{\pgfqpoint{3.035966in}{1.659289in}}%
\pgfpathlineto{\pgfqpoint{3.027462in}{1.653014in}}%
\pgfpathclose%
\pgfusepath{fill}%
\end{pgfscope}%
\begin{pgfscope}%
\pgfpathrectangle{\pgfqpoint{1.150000in}{0.150000in}}{\pgfqpoint{5.700000in}{5.700000in}}%
\pgfusepath{clip}%
\pgfsetbuttcap%
\pgfsetroundjoin%
\definecolor{currentfill}{rgb}{0.273809,0.031497,0.358853}%
\pgfsetfillcolor{currentfill}%
\pgfsetfillopacity{0.700000}%
\pgfsetlinewidth{0.000000pt}%
\definecolor{currentstroke}{rgb}{0.000000,0.000000,0.000000}%
\pgfsetstrokecolor{currentstroke}%
\pgfsetdash{}{0pt}%
\pgfpathmoveto{\pgfqpoint{3.636252in}{1.682113in}}%
\pgfpathlineto{\pgfqpoint{3.650112in}{1.679843in}}%
\pgfpathlineto{\pgfqpoint{3.663979in}{1.677651in}}%
\pgfpathlineto{\pgfqpoint{3.677852in}{1.675537in}}%
\pgfpathlineto{\pgfqpoint{3.691733in}{1.673499in}}%
\pgfpathlineto{\pgfqpoint{3.699942in}{1.683230in}}%
\pgfpathlineto{\pgfqpoint{3.708145in}{1.692969in}}%
\pgfpathlineto{\pgfqpoint{3.716342in}{1.702711in}}%
\pgfpathlineto{\pgfqpoint{3.724533in}{1.712455in}}%
\pgfpathlineto{\pgfqpoint{3.710665in}{1.714308in}}%
\pgfpathlineto{\pgfqpoint{3.696803in}{1.716239in}}%
\pgfpathlineto{\pgfqpoint{3.682948in}{1.718246in}}%
\pgfpathlineto{\pgfqpoint{3.669100in}{1.720332in}}%
\pgfpathlineto{\pgfqpoint{3.660897in}{1.710764in}}%
\pgfpathlineto{\pgfqpoint{3.652688in}{1.701204in}}%
\pgfpathlineto{\pgfqpoint{3.644473in}{1.691652in}}%
\pgfpathlineto{\pgfqpoint{3.636252in}{1.682113in}}%
\pgfpathclose%
\pgfusepath{fill}%
\end{pgfscope}%
\begin{pgfscope}%
\pgfpathrectangle{\pgfqpoint{1.150000in}{0.150000in}}{\pgfqpoint{5.700000in}{5.700000in}}%
\pgfusepath{clip}%
\pgfsetbuttcap%
\pgfsetroundjoin%
\definecolor{currentfill}{rgb}{0.267004,0.004874,0.329415}%
\pgfsetfillcolor{currentfill}%
\pgfsetfillopacity{0.700000}%
\pgfsetlinewidth{0.000000pt}%
\definecolor{currentstroke}{rgb}{0.000000,0.000000,0.000000}%
\pgfsetstrokecolor{currentstroke}%
\pgfsetdash{}{0pt}%
\pgfpathmoveto{\pgfqpoint{3.171582in}{1.634529in}}%
\pgfpathlineto{\pgfqpoint{3.185370in}{1.629377in}}%
\pgfpathlineto{\pgfqpoint{3.199163in}{1.624309in}}%
\pgfpathlineto{\pgfqpoint{3.212960in}{1.619326in}}%
\pgfpathlineto{\pgfqpoint{3.226762in}{1.614426in}}%
\pgfpathlineto{\pgfqpoint{3.235170in}{1.621987in}}%
\pgfpathlineto{\pgfqpoint{3.243569in}{1.629639in}}%
\pgfpathlineto{\pgfqpoint{3.251961in}{1.637377in}}%
\pgfpathlineto{\pgfqpoint{3.260344in}{1.645197in}}%
\pgfpathlineto{\pgfqpoint{3.246561in}{1.649830in}}%
\pgfpathlineto{\pgfqpoint{3.232782in}{1.654548in}}%
\pgfpathlineto{\pgfqpoint{3.219008in}{1.659349in}}%
\pgfpathlineto{\pgfqpoint{3.205238in}{1.664235in}}%
\pgfpathlineto{\pgfqpoint{3.196837in}{1.656673in}}%
\pgfpathlineto{\pgfqpoint{3.188427in}{1.649199in}}%
\pgfpathlineto{\pgfqpoint{3.180008in}{1.641817in}}%
\pgfpathlineto{\pgfqpoint{3.171582in}{1.634529in}}%
\pgfpathclose%
\pgfusepath{fill}%
\end{pgfscope}%
\begin{pgfscope}%
\pgfpathrectangle{\pgfqpoint{1.150000in}{0.150000in}}{\pgfqpoint{5.700000in}{5.700000in}}%
\pgfusepath{clip}%
\pgfsetbuttcap%
\pgfsetroundjoin%
\definecolor{currentfill}{rgb}{0.281887,0.150881,0.465405}%
\pgfsetfillcolor{currentfill}%
\pgfsetfillopacity{0.700000}%
\pgfsetlinewidth{0.000000pt}%
\definecolor{currentstroke}{rgb}{0.000000,0.000000,0.000000}%
\pgfsetstrokecolor{currentstroke}%
\pgfsetdash{}{0pt}%
\pgfpathmoveto{\pgfqpoint{4.221136in}{1.897598in}}%
\pgfpathlineto{\pgfqpoint{4.235159in}{1.898235in}}%
\pgfpathlineto{\pgfqpoint{4.249192in}{1.898945in}}%
\pgfpathlineto{\pgfqpoint{4.263235in}{1.899728in}}%
\pgfpathlineto{\pgfqpoint{4.277286in}{1.900585in}}%
\pgfpathlineto{\pgfqpoint{4.285292in}{1.910422in}}%
\pgfpathlineto{\pgfqpoint{4.293291in}{1.920192in}}%
\pgfpathlineto{\pgfqpoint{4.301285in}{1.929893in}}%
\pgfpathlineto{\pgfqpoint{4.309273in}{1.939526in}}%
\pgfpathlineto{\pgfqpoint{4.295230in}{1.938609in}}%
\pgfpathlineto{\pgfqpoint{4.281197in}{1.937765in}}%
\pgfpathlineto{\pgfqpoint{4.267173in}{1.936994in}}%
\pgfpathlineto{\pgfqpoint{4.253158in}{1.936297in}}%
\pgfpathlineto{\pgfqpoint{4.245161in}{1.926717in}}%
\pgfpathlineto{\pgfqpoint{4.237159in}{1.917073in}}%
\pgfpathlineto{\pgfqpoint{4.229150in}{1.907367in}}%
\pgfpathlineto{\pgfqpoint{4.221136in}{1.897598in}}%
\pgfpathclose%
\pgfusepath{fill}%
\end{pgfscope}%
\begin{pgfscope}%
\pgfpathrectangle{\pgfqpoint{1.150000in}{0.150000in}}{\pgfqpoint{5.700000in}{5.700000in}}%
\pgfusepath{clip}%
\pgfsetbuttcap%
\pgfsetroundjoin%
\definecolor{currentfill}{rgb}{0.216210,0.351535,0.550627}%
\pgfsetfillcolor{currentfill}%
\pgfsetfillopacity{0.700000}%
\pgfsetlinewidth{0.000000pt}%
\definecolor{currentstroke}{rgb}{0.000000,0.000000,0.000000}%
\pgfsetstrokecolor{currentstroke}%
\pgfsetdash{}{0pt}%
\pgfpathmoveto{\pgfqpoint{5.246970in}{2.360746in}}%
\pgfpathlineto{\pgfqpoint{5.261388in}{2.364310in}}%
\pgfpathlineto{\pgfqpoint{5.275819in}{2.367944in}}%
\pgfpathlineto{\pgfqpoint{5.290261in}{2.371648in}}%
\pgfpathlineto{\pgfqpoint{5.304717in}{2.375422in}}%
\pgfpathlineto{\pgfqpoint{5.312270in}{2.380923in}}%
\pgfpathlineto{\pgfqpoint{5.319815in}{2.386351in}}%
\pgfpathlineto{\pgfqpoint{5.327352in}{2.391708in}}%
\pgfpathlineto{\pgfqpoint{5.334881in}{2.396999in}}%
\pgfpathlineto{\pgfqpoint{5.320443in}{2.393400in}}%
\pgfpathlineto{\pgfqpoint{5.306018in}{2.389871in}}%
\pgfpathlineto{\pgfqpoint{5.291605in}{2.386412in}}%
\pgfpathlineto{\pgfqpoint{5.277204in}{2.383023in}}%
\pgfpathlineto{\pgfqpoint{5.269657in}{2.377550in}}%
\pgfpathlineto{\pgfqpoint{5.262102in}{2.372015in}}%
\pgfpathlineto{\pgfqpoint{5.254540in}{2.366415in}}%
\pgfpathlineto{\pgfqpoint{5.246970in}{2.360746in}}%
\pgfpathclose%
\pgfusepath{fill}%
\end{pgfscope}%
\begin{pgfscope}%
\pgfpathrectangle{\pgfqpoint{1.150000in}{0.150000in}}{\pgfqpoint{5.700000in}{5.700000in}}%
\pgfusepath{clip}%
\pgfsetbuttcap%
\pgfsetroundjoin%
\definecolor{currentfill}{rgb}{0.283072,0.130895,0.449241}%
\pgfsetfillcolor{currentfill}%
\pgfsetfillopacity{0.700000}%
\pgfsetlinewidth{0.000000pt}%
\definecolor{currentstroke}{rgb}{0.000000,0.000000,0.000000}%
\pgfsetstrokecolor{currentstroke}%
\pgfsetdash{}{0pt}%
\pgfpathmoveto{\pgfqpoint{4.132982in}{1.856424in}}%
\pgfpathlineto{\pgfqpoint{4.146978in}{1.856685in}}%
\pgfpathlineto{\pgfqpoint{4.160984in}{1.857020in}}%
\pgfpathlineto{\pgfqpoint{4.174998in}{1.857428in}}%
\pgfpathlineto{\pgfqpoint{4.189022in}{1.857910in}}%
\pgfpathlineto{\pgfqpoint{4.197059in}{1.867922in}}%
\pgfpathlineto{\pgfqpoint{4.205090in}{1.877875in}}%
\pgfpathlineto{\pgfqpoint{4.213116in}{1.887767in}}%
\pgfpathlineto{\pgfqpoint{4.221136in}{1.897598in}}%
\pgfpathlineto{\pgfqpoint{4.207121in}{1.897034in}}%
\pgfpathlineto{\pgfqpoint{4.193116in}{1.896544in}}%
\pgfpathlineto{\pgfqpoint{4.179120in}{1.896128in}}%
\pgfpathlineto{\pgfqpoint{4.165132in}{1.895786in}}%
\pgfpathlineto{\pgfqpoint{4.157103in}{1.886028in}}%
\pgfpathlineto{\pgfqpoint{4.149068in}{1.876215in}}%
\pgfpathlineto{\pgfqpoint{4.141028in}{1.866347in}}%
\pgfpathlineto{\pgfqpoint{4.132982in}{1.856424in}}%
\pgfpathclose%
\pgfusepath{fill}%
\end{pgfscope}%
\begin{pgfscope}%
\pgfpathrectangle{\pgfqpoint{1.150000in}{0.150000in}}{\pgfqpoint{5.700000in}{5.700000in}}%
\pgfusepath{clip}%
\pgfsetbuttcap%
\pgfsetroundjoin%
\definecolor{currentfill}{rgb}{0.272594,0.025563,0.353093}%
\pgfsetfillcolor{currentfill}%
\pgfsetfillopacity{0.700000}%
\pgfsetlinewidth{0.000000pt}%
\definecolor{currentstroke}{rgb}{0.000000,0.000000,0.000000}%
\pgfsetstrokecolor{currentstroke}%
\pgfsetdash{}{0pt}%
\pgfpathmoveto{\pgfqpoint{2.883050in}{1.683931in}}%
\pgfpathlineto{\pgfqpoint{2.896827in}{1.676776in}}%
\pgfpathlineto{\pgfqpoint{2.910607in}{1.669713in}}%
\pgfpathlineto{\pgfqpoint{2.924389in}{1.662741in}}%
\pgfpathlineto{\pgfqpoint{2.938174in}{1.655859in}}%
\pgfpathlineto{\pgfqpoint{2.946743in}{1.661284in}}%
\pgfpathlineto{\pgfqpoint{2.955300in}{1.666854in}}%
\pgfpathlineto{\pgfqpoint{2.963848in}{1.672566in}}%
\pgfpathlineto{\pgfqpoint{2.972385in}{1.678413in}}%
\pgfpathlineto{\pgfqpoint{2.958623in}{1.684986in}}%
\pgfpathlineto{\pgfqpoint{2.944865in}{1.691650in}}%
\pgfpathlineto{\pgfqpoint{2.931110in}{1.698404in}}%
\pgfpathlineto{\pgfqpoint{2.917357in}{1.705249in}}%
\pgfpathlineto{\pgfqpoint{2.908797in}{1.699702in}}%
\pgfpathlineto{\pgfqpoint{2.900226in}{1.694297in}}%
\pgfpathlineto{\pgfqpoint{2.891643in}{1.689038in}}%
\pgfpathlineto{\pgfqpoint{2.883050in}{1.683931in}}%
\pgfpathclose%
\pgfusepath{fill}%
\end{pgfscope}%
\begin{pgfscope}%
\pgfpathrectangle{\pgfqpoint{1.150000in}{0.150000in}}{\pgfqpoint{5.700000in}{5.700000in}}%
\pgfusepath{clip}%
\pgfsetbuttcap%
\pgfsetroundjoin%
\definecolor{currentfill}{rgb}{0.260571,0.246922,0.522828}%
\pgfsetfillcolor{currentfill}%
\pgfsetfillopacity{0.700000}%
\pgfsetlinewidth{0.000000pt}%
\definecolor{currentstroke}{rgb}{0.000000,0.000000,0.000000}%
\pgfsetstrokecolor{currentstroke}%
\pgfsetdash{}{0pt}%
\pgfpathmoveto{\pgfqpoint{2.114085in}{2.150245in}}%
\pgfpathlineto{\pgfqpoint{2.127970in}{2.136851in}}%
\pgfpathlineto{\pgfqpoint{2.141853in}{2.123585in}}%
\pgfpathlineto{\pgfqpoint{2.155732in}{2.110446in}}%
\pgfpathlineto{\pgfqpoint{2.169608in}{2.097434in}}%
\pgfpathlineto{\pgfqpoint{2.178749in}{2.096083in}}%
\pgfpathlineto{\pgfqpoint{2.187870in}{2.095007in}}%
\pgfpathlineto{\pgfqpoint{2.196971in}{2.094201in}}%
\pgfpathlineto{\pgfqpoint{2.206053in}{2.093658in}}%
\pgfpathlineto{\pgfqpoint{2.192218in}{2.106285in}}%
\pgfpathlineto{\pgfqpoint{2.178380in}{2.119038in}}%
\pgfpathlineto{\pgfqpoint{2.164540in}{2.131918in}}%
\pgfpathlineto{\pgfqpoint{2.150696in}{2.144926in}}%
\pgfpathlineto{\pgfqpoint{2.141573in}{2.145846in}}%
\pgfpathlineto{\pgfqpoint{2.132431in}{2.147035in}}%
\pgfpathlineto{\pgfqpoint{2.123268in}{2.148499in}}%
\pgfpathlineto{\pgfqpoint{2.114085in}{2.150245in}}%
\pgfpathclose%
\pgfusepath{fill}%
\end{pgfscope}%
\begin{pgfscope}%
\pgfpathrectangle{\pgfqpoint{1.150000in}{0.150000in}}{\pgfqpoint{5.700000in}{5.700000in}}%
\pgfusepath{clip}%
\pgfsetbuttcap%
\pgfsetroundjoin%
\definecolor{currentfill}{rgb}{0.283229,0.120777,0.440584}%
\pgfsetfillcolor{currentfill}%
\pgfsetfillopacity{0.700000}%
\pgfsetlinewidth{0.000000pt}%
\definecolor{currentstroke}{rgb}{0.000000,0.000000,0.000000}%
\pgfsetstrokecolor{currentstroke}%
\pgfsetdash{}{0pt}%
\pgfpathmoveto{\pgfqpoint{2.482367in}{1.866040in}}%
\pgfpathlineto{\pgfqpoint{2.496172in}{1.855836in}}%
\pgfpathlineto{\pgfqpoint{2.509978in}{1.845739in}}%
\pgfpathlineto{\pgfqpoint{2.523784in}{1.835747in}}%
\pgfpathlineto{\pgfqpoint{2.537590in}{1.825860in}}%
\pgfpathlineto{\pgfqpoint{2.546434in}{1.827775in}}%
\pgfpathlineto{\pgfqpoint{2.555263in}{1.829909in}}%
\pgfpathlineto{\pgfqpoint{2.564077in}{1.832256in}}%
\pgfpathlineto{\pgfqpoint{2.572876in}{1.834811in}}%
\pgfpathlineto{\pgfqpoint{2.559103in}{1.844343in}}%
\pgfpathlineto{\pgfqpoint{2.545330in}{1.853980in}}%
\pgfpathlineto{\pgfqpoint{2.531558in}{1.863721in}}%
\pgfpathlineto{\pgfqpoint{2.517786in}{1.873569in}}%
\pgfpathlineto{\pgfqpoint{2.508954in}{1.871361in}}%
\pgfpathlineto{\pgfqpoint{2.500107in}{1.869367in}}%
\pgfpathlineto{\pgfqpoint{2.491245in}{1.867591in}}%
\pgfpathlineto{\pgfqpoint{2.482367in}{1.866040in}}%
\pgfpathclose%
\pgfusepath{fill}%
\end{pgfscope}%
\begin{pgfscope}%
\pgfpathrectangle{\pgfqpoint{1.150000in}{0.150000in}}{\pgfqpoint{5.700000in}{5.700000in}}%
\pgfusepath{clip}%
\pgfsetbuttcap%
\pgfsetroundjoin%
\definecolor{currentfill}{rgb}{0.283091,0.110553,0.431554}%
\pgfsetfillcolor{currentfill}%
\pgfsetfillopacity{0.700000}%
\pgfsetlinewidth{0.000000pt}%
\definecolor{currentstroke}{rgb}{0.000000,0.000000,0.000000}%
\pgfsetstrokecolor{currentstroke}%
\pgfsetdash{}{0pt}%
\pgfpathmoveto{\pgfqpoint{4.044807in}{1.816315in}}%
\pgfpathlineto{\pgfqpoint{4.058778in}{1.816178in}}%
\pgfpathlineto{\pgfqpoint{4.072757in}{1.816115in}}%
\pgfpathlineto{\pgfqpoint{4.086745in}{1.816125in}}%
\pgfpathlineto{\pgfqpoint{4.100742in}{1.816210in}}%
\pgfpathlineto{\pgfqpoint{4.108810in}{1.826340in}}%
\pgfpathlineto{\pgfqpoint{4.116873in}{1.836419in}}%
\pgfpathlineto{\pgfqpoint{4.124930in}{1.846448in}}%
\pgfpathlineto{\pgfqpoint{4.132982in}{1.856424in}}%
\pgfpathlineto{\pgfqpoint{4.118994in}{1.856237in}}%
\pgfpathlineto{\pgfqpoint{4.105015in}{1.856124in}}%
\pgfpathlineto{\pgfqpoint{4.091045in}{1.856085in}}%
\pgfpathlineto{\pgfqpoint{4.077084in}{1.856120in}}%
\pgfpathlineto{\pgfqpoint{4.069023in}{1.846238in}}%
\pgfpathlineto{\pgfqpoint{4.060956in}{1.836309in}}%
\pgfpathlineto{\pgfqpoint{4.052884in}{1.826335in}}%
\pgfpathlineto{\pgfqpoint{4.044807in}{1.816315in}}%
\pgfpathclose%
\pgfusepath{fill}%
\end{pgfscope}%
\begin{pgfscope}%
\pgfpathrectangle{\pgfqpoint{1.150000in}{0.150000in}}{\pgfqpoint{5.700000in}{5.700000in}}%
\pgfusepath{clip}%
\pgfsetbuttcap%
\pgfsetroundjoin%
\definecolor{currentfill}{rgb}{0.267004,0.004874,0.329415}%
\pgfsetfillcolor{currentfill}%
\pgfsetfillopacity{0.700000}%
\pgfsetlinewidth{0.000000pt}%
\definecolor{currentstroke}{rgb}{0.000000,0.000000,0.000000}%
\pgfsetstrokecolor{currentstroke}%
\pgfsetdash{}{0pt}%
\pgfpathmoveto{\pgfqpoint{3.315527in}{1.627492in}}%
\pgfpathlineto{\pgfqpoint{3.329335in}{1.623272in}}%
\pgfpathlineto{\pgfqpoint{3.343148in}{1.619134in}}%
\pgfpathlineto{\pgfqpoint{3.356967in}{1.615077in}}%
\pgfpathlineto{\pgfqpoint{3.370791in}{1.611101in}}%
\pgfpathlineto{\pgfqpoint{3.379133in}{1.619506in}}%
\pgfpathlineto{\pgfqpoint{3.387468in}{1.627977in}}%
\pgfpathlineto{\pgfqpoint{3.395795in}{1.636508in}}%
\pgfpathlineto{\pgfqpoint{3.404116in}{1.645097in}}%
\pgfpathlineto{\pgfqpoint{3.390308in}{1.648827in}}%
\pgfpathlineto{\pgfqpoint{3.376506in}{1.652638in}}%
\pgfpathlineto{\pgfqpoint{3.362709in}{1.656531in}}%
\pgfpathlineto{\pgfqpoint{3.348917in}{1.660506in}}%
\pgfpathlineto{\pgfqpoint{3.340580in}{1.652155in}}%
\pgfpathlineto{\pgfqpoint{3.332237in}{1.643867in}}%
\pgfpathlineto{\pgfqpoint{3.323886in}{1.635645in}}%
\pgfpathlineto{\pgfqpoint{3.315527in}{1.627492in}}%
\pgfpathclose%
\pgfusepath{fill}%
\end{pgfscope}%
\begin{pgfscope}%
\pgfpathrectangle{\pgfqpoint{1.150000in}{0.150000in}}{\pgfqpoint{5.700000in}{5.700000in}}%
\pgfusepath{clip}%
\pgfsetbuttcap%
\pgfsetroundjoin%
\definecolor{currentfill}{rgb}{0.221989,0.339161,0.548752}%
\pgfsetfillcolor{currentfill}%
\pgfsetfillopacity{0.700000}%
\pgfsetlinewidth{0.000000pt}%
\definecolor{currentstroke}{rgb}{0.000000,0.000000,0.000000}%
\pgfsetstrokecolor{currentstroke}%
\pgfsetdash{}{0pt}%
\pgfpathmoveto{\pgfqpoint{5.158995in}{2.323155in}}%
\pgfpathlineto{\pgfqpoint{5.173380in}{2.326592in}}%
\pgfpathlineto{\pgfqpoint{5.187778in}{2.330099in}}%
\pgfpathlineto{\pgfqpoint{5.202188in}{2.333677in}}%
\pgfpathlineto{\pgfqpoint{5.216610in}{2.337325in}}%
\pgfpathlineto{\pgfqpoint{5.224212in}{2.343298in}}%
\pgfpathlineto{\pgfqpoint{5.231806in}{2.349190in}}%
\pgfpathlineto{\pgfqpoint{5.239392in}{2.355005in}}%
\pgfpathlineto{\pgfqpoint{5.246970in}{2.360746in}}%
\pgfpathlineto{\pgfqpoint{5.232564in}{2.357252in}}%
\pgfpathlineto{\pgfqpoint{5.218170in}{2.353828in}}%
\pgfpathlineto{\pgfqpoint{5.203789in}{2.350474in}}%
\pgfpathlineto{\pgfqpoint{5.189419in}{2.347191in}}%
\pgfpathlineto{\pgfqpoint{5.181825in}{2.341289in}}%
\pgfpathlineto{\pgfqpoint{5.174223in}{2.335318in}}%
\pgfpathlineto{\pgfqpoint{5.166613in}{2.329274in}}%
\pgfpathlineto{\pgfqpoint{5.158995in}{2.323155in}}%
\pgfpathclose%
\pgfusepath{fill}%
\end{pgfscope}%
\begin{pgfscope}%
\pgfpathrectangle{\pgfqpoint{1.150000in}{0.150000in}}{\pgfqpoint{5.700000in}{5.700000in}}%
\pgfusepath{clip}%
\pgfsetbuttcap%
\pgfsetroundjoin%
\definecolor{currentfill}{rgb}{0.271305,0.019942,0.347269}%
\pgfsetfillcolor{currentfill}%
\pgfsetfillopacity{0.700000}%
\pgfsetlinewidth{0.000000pt}%
\definecolor{currentstroke}{rgb}{0.000000,0.000000,0.000000}%
\pgfsetstrokecolor{currentstroke}%
\pgfsetdash{}{0pt}%
\pgfpathmoveto{\pgfqpoint{3.547883in}{1.654808in}}%
\pgfpathlineto{\pgfqpoint{3.561729in}{1.652020in}}%
\pgfpathlineto{\pgfqpoint{3.575583in}{1.649312in}}%
\pgfpathlineto{\pgfqpoint{3.589442in}{1.646681in}}%
\pgfpathlineto{\pgfqpoint{3.603308in}{1.644129in}}%
\pgfpathlineto{\pgfqpoint{3.611553in}{1.653593in}}%
\pgfpathlineto{\pgfqpoint{3.619792in}{1.663081in}}%
\pgfpathlineto{\pgfqpoint{3.628025in}{1.672588in}}%
\pgfpathlineto{\pgfqpoint{3.636252in}{1.682113in}}%
\pgfpathlineto{\pgfqpoint{3.622399in}{1.684461in}}%
\pgfpathlineto{\pgfqpoint{3.608553in}{1.686886in}}%
\pgfpathlineto{\pgfqpoint{3.594713in}{1.689390in}}%
\pgfpathlineto{\pgfqpoint{3.580879in}{1.691973in}}%
\pgfpathlineto{\pgfqpoint{3.572639in}{1.682645in}}%
\pgfpathlineto{\pgfqpoint{3.564393in}{1.673340in}}%
\pgfpathlineto{\pgfqpoint{3.556141in}{1.664060in}}%
\pgfpathlineto{\pgfqpoint{3.547883in}{1.654808in}}%
\pgfpathclose%
\pgfusepath{fill}%
\end{pgfscope}%
\begin{pgfscope}%
\pgfpathrectangle{\pgfqpoint{1.150000in}{0.150000in}}{\pgfqpoint{5.700000in}{5.700000in}}%
\pgfusepath{clip}%
\pgfsetbuttcap%
\pgfsetroundjoin%
\definecolor{currentfill}{rgb}{0.282327,0.094955,0.417331}%
\pgfsetfillcolor{currentfill}%
\pgfsetfillopacity{0.700000}%
\pgfsetlinewidth{0.000000pt}%
\definecolor{currentstroke}{rgb}{0.000000,0.000000,0.000000}%
\pgfsetstrokecolor{currentstroke}%
\pgfsetdash{}{0pt}%
\pgfpathmoveto{\pgfqpoint{3.956604in}{1.777604in}}%
\pgfpathlineto{\pgfqpoint{3.970551in}{1.777046in}}%
\pgfpathlineto{\pgfqpoint{3.984506in}{1.776562in}}%
\pgfpathlineto{\pgfqpoint{3.998469in}{1.776152in}}%
\pgfpathlineto{\pgfqpoint{4.012441in}{1.775818in}}%
\pgfpathlineto{\pgfqpoint{4.020541in}{1.786002in}}%
\pgfpathlineto{\pgfqpoint{4.028635in}{1.796148in}}%
\pgfpathlineto{\pgfqpoint{4.036724in}{1.806252in}}%
\pgfpathlineto{\pgfqpoint{4.044807in}{1.816315in}}%
\pgfpathlineto{\pgfqpoint{4.030844in}{1.816527in}}%
\pgfpathlineto{\pgfqpoint{4.016891in}{1.816814in}}%
\pgfpathlineto{\pgfqpoint{4.002945in}{1.817175in}}%
\pgfpathlineto{\pgfqpoint{3.989008in}{1.817611in}}%
\pgfpathlineto{\pgfqpoint{3.980915in}{1.807663in}}%
\pgfpathlineto{\pgfqpoint{3.972817in}{1.797678in}}%
\pgfpathlineto{\pgfqpoint{3.964713in}{1.787658in}}%
\pgfpathlineto{\pgfqpoint{3.956604in}{1.777604in}}%
\pgfpathclose%
\pgfusepath{fill}%
\end{pgfscope}%
\begin{pgfscope}%
\pgfpathrectangle{\pgfqpoint{1.150000in}{0.150000in}}{\pgfqpoint{5.700000in}{5.700000in}}%
\pgfusepath{clip}%
\pgfsetbuttcap%
\pgfsetroundjoin%
\definecolor{currentfill}{rgb}{0.188923,0.410910,0.556326}%
\pgfsetfillcolor{currentfill}%
\pgfsetfillopacity{0.700000}%
\pgfsetlinewidth{0.000000pt}%
\definecolor{currentstroke}{rgb}{0.000000,0.000000,0.000000}%
\pgfsetstrokecolor{currentstroke}%
\pgfsetdash{}{0pt}%
\pgfpathmoveto{\pgfqpoint{5.656401in}{2.513145in}}%
\pgfpathlineto{\pgfqpoint{5.670995in}{2.517269in}}%
\pgfpathlineto{\pgfqpoint{5.685603in}{2.521462in}}%
\pgfpathlineto{\pgfqpoint{5.700223in}{2.525725in}}%
\pgfpathlineto{\pgfqpoint{5.707551in}{2.529283in}}%
\pgfpathlineto{\pgfqpoint{5.714871in}{2.532805in}}%
\pgfpathlineto{\pgfqpoint{5.722183in}{2.536296in}}%
\pgfpathlineto{\pgfqpoint{5.729488in}{2.539760in}}%
\pgfpathlineto{\pgfqpoint{5.714891in}{2.535761in}}%
\pgfpathlineto{\pgfqpoint{5.700307in}{2.531830in}}%
\pgfpathlineto{\pgfqpoint{5.685736in}{2.527969in}}%
\pgfpathlineto{\pgfqpoint{5.678414in}{2.524302in}}%
\pgfpathlineto{\pgfqpoint{5.671084in}{2.520612in}}%
\pgfpathlineto{\pgfqpoint{5.663746in}{2.516895in}}%
\pgfpathlineto{\pgfqpoint{5.656401in}{2.513145in}}%
\pgfpathclose%
\pgfusepath{fill}%
\end{pgfscope}%
\begin{pgfscope}%
\pgfpathrectangle{\pgfqpoint{1.150000in}{0.150000in}}{\pgfqpoint{5.700000in}{5.700000in}}%
\pgfusepath{clip}%
\pgfsetbuttcap%
\pgfsetroundjoin%
\definecolor{currentfill}{rgb}{0.267968,0.223549,0.512008}%
\pgfsetfillcolor{currentfill}%
\pgfsetfillopacity{0.700000}%
\pgfsetlinewidth{0.000000pt}%
\definecolor{currentstroke}{rgb}{0.000000,0.000000,0.000000}%
\pgfsetstrokecolor{currentstroke}%
\pgfsetdash{}{0pt}%
\pgfpathmoveto{\pgfqpoint{2.169608in}{2.097434in}}%
\pgfpathlineto{\pgfqpoint{2.183481in}{2.084547in}}%
\pgfpathlineto{\pgfqpoint{2.197352in}{2.071784in}}%
\pgfpathlineto{\pgfqpoint{2.211220in}{2.059145in}}%
\pgfpathlineto{\pgfqpoint{2.225085in}{2.046627in}}%
\pgfpathlineto{\pgfqpoint{2.234185in}{2.045669in}}%
\pgfpathlineto{\pgfqpoint{2.243265in}{2.044980in}}%
\pgfpathlineto{\pgfqpoint{2.252325in}{2.044555in}}%
\pgfpathlineto{\pgfqpoint{2.261368in}{2.044388in}}%
\pgfpathlineto{\pgfqpoint{2.247542in}{2.056522in}}%
\pgfpathlineto{\pgfqpoint{2.233715in}{2.068777in}}%
\pgfpathlineto{\pgfqpoint{2.219885in}{2.081156in}}%
\pgfpathlineto{\pgfqpoint{2.206053in}{2.093658in}}%
\pgfpathlineto{\pgfqpoint{2.196971in}{2.094201in}}%
\pgfpathlineto{\pgfqpoint{2.187870in}{2.095007in}}%
\pgfpathlineto{\pgfqpoint{2.178749in}{2.096083in}}%
\pgfpathlineto{\pgfqpoint{2.169608in}{2.097434in}}%
\pgfpathclose%
\pgfusepath{fill}%
\end{pgfscope}%
\begin{pgfscope}%
\pgfpathrectangle{\pgfqpoint{1.150000in}{0.150000in}}{\pgfqpoint{5.700000in}{5.700000in}}%
\pgfusepath{clip}%
\pgfsetbuttcap%
\pgfsetroundjoin%
\definecolor{currentfill}{rgb}{0.227802,0.326594,0.546532}%
\pgfsetfillcolor{currentfill}%
\pgfsetfillopacity{0.700000}%
\pgfsetlinewidth{0.000000pt}%
\definecolor{currentstroke}{rgb}{0.000000,0.000000,0.000000}%
\pgfsetstrokecolor{currentstroke}%
\pgfsetdash{}{0pt}%
\pgfpathmoveto{\pgfqpoint{5.070964in}{2.284299in}}%
\pgfpathlineto{\pgfqpoint{5.085316in}{2.287587in}}%
\pgfpathlineto{\pgfqpoint{5.099680in}{2.290945in}}%
\pgfpathlineto{\pgfqpoint{5.114057in}{2.294373in}}%
\pgfpathlineto{\pgfqpoint{5.128445in}{2.297872in}}%
\pgfpathlineto{\pgfqpoint{5.136094in}{2.304319in}}%
\pgfpathlineto{\pgfqpoint{5.143736in}{2.310680in}}%
\pgfpathlineto{\pgfqpoint{5.151369in}{2.316958in}}%
\pgfpathlineto{\pgfqpoint{5.158995in}{2.323155in}}%
\pgfpathlineto{\pgfqpoint{5.144621in}{2.319788in}}%
\pgfpathlineto{\pgfqpoint{5.130260in}{2.316492in}}%
\pgfpathlineto{\pgfqpoint{5.115910in}{2.313266in}}%
\pgfpathlineto{\pgfqpoint{5.101573in}{2.310110in}}%
\pgfpathlineto{\pgfqpoint{5.093932in}{2.303774in}}%
\pgfpathlineto{\pgfqpoint{5.086284in}{2.297361in}}%
\pgfpathlineto{\pgfqpoint{5.078627in}{2.290871in}}%
\pgfpathlineto{\pgfqpoint{5.070964in}{2.284299in}}%
\pgfpathclose%
\pgfusepath{fill}%
\end{pgfscope}%
\begin{pgfscope}%
\pgfpathrectangle{\pgfqpoint{1.150000in}{0.150000in}}{\pgfqpoint{5.700000in}{5.700000in}}%
\pgfusepath{clip}%
\pgfsetbuttcap%
\pgfsetroundjoin%
\definecolor{currentfill}{rgb}{0.277941,0.056324,0.381191}%
\pgfsetfillcolor{currentfill}%
\pgfsetfillopacity{0.700000}%
\pgfsetlinewidth{0.000000pt}%
\definecolor{currentstroke}{rgb}{0.000000,0.000000,0.000000}%
\pgfsetstrokecolor{currentstroke}%
\pgfsetdash{}{0pt}%
\pgfpathmoveto{\pgfqpoint{2.738216in}{1.728330in}}%
\pgfpathlineto{\pgfqpoint{2.752003in}{1.720096in}}%
\pgfpathlineto{\pgfqpoint{2.765791in}{1.711958in}}%
\pgfpathlineto{\pgfqpoint{2.779581in}{1.703915in}}%
\pgfpathlineto{\pgfqpoint{2.793373in}{1.695966in}}%
\pgfpathlineto{\pgfqpoint{2.802040in}{1.700083in}}%
\pgfpathlineto{\pgfqpoint{2.810695in}{1.704376in}}%
\pgfpathlineto{\pgfqpoint{2.819337in}{1.708840in}}%
\pgfpathlineto{\pgfqpoint{2.827967in}{1.713470in}}%
\pgfpathlineto{\pgfqpoint{2.814202in}{1.721088in}}%
\pgfpathlineto{\pgfqpoint{2.800440in}{1.728801in}}%
\pgfpathlineto{\pgfqpoint{2.786679in}{1.736608in}}%
\pgfpathlineto{\pgfqpoint{2.772920in}{1.744511in}}%
\pgfpathlineto{\pgfqpoint{2.764263in}{1.740204in}}%
\pgfpathlineto{\pgfqpoint{2.755593in}{1.736068in}}%
\pgfpathlineto{\pgfqpoint{2.746911in}{1.732109in}}%
\pgfpathlineto{\pgfqpoint{2.738216in}{1.728330in}}%
\pgfpathclose%
\pgfusepath{fill}%
\end{pgfscope}%
\begin{pgfscope}%
\pgfpathrectangle{\pgfqpoint{1.150000in}{0.150000in}}{\pgfqpoint{5.700000in}{5.700000in}}%
\pgfusepath{clip}%
\pgfsetbuttcap%
\pgfsetroundjoin%
\definecolor{currentfill}{rgb}{0.280894,0.078907,0.402329}%
\pgfsetfillcolor{currentfill}%
\pgfsetfillopacity{0.700000}%
\pgfsetlinewidth{0.000000pt}%
\definecolor{currentstroke}{rgb}{0.000000,0.000000,0.000000}%
\pgfsetstrokecolor{currentstroke}%
\pgfsetdash{}{0pt}%
\pgfpathmoveto{\pgfqpoint{3.868365in}{1.740644in}}%
\pgfpathlineto{\pgfqpoint{3.882290in}{1.739642in}}%
\pgfpathlineto{\pgfqpoint{3.896223in}{1.738714in}}%
\pgfpathlineto{\pgfqpoint{3.910164in}{1.737862in}}%
\pgfpathlineto{\pgfqpoint{3.924112in}{1.737084in}}%
\pgfpathlineto{\pgfqpoint{3.932243in}{1.747257in}}%
\pgfpathlineto{\pgfqpoint{3.940369in}{1.757402in}}%
\pgfpathlineto{\pgfqpoint{3.948489in}{1.767518in}}%
\pgfpathlineto{\pgfqpoint{3.956604in}{1.777604in}}%
\pgfpathlineto{\pgfqpoint{3.942665in}{1.778238in}}%
\pgfpathlineto{\pgfqpoint{3.928735in}{1.778947in}}%
\pgfpathlineto{\pgfqpoint{3.914812in}{1.779731in}}%
\pgfpathlineto{\pgfqpoint{3.900898in}{1.780590in}}%
\pgfpathlineto{\pgfqpoint{3.892773in}{1.770640in}}%
\pgfpathlineto{\pgfqpoint{3.884643in}{1.760664in}}%
\pgfpathlineto{\pgfqpoint{3.876507in}{1.750665in}}%
\pgfpathlineto{\pgfqpoint{3.868365in}{1.740644in}}%
\pgfpathclose%
\pgfusepath{fill}%
\end{pgfscope}%
\begin{pgfscope}%
\pgfpathrectangle{\pgfqpoint{1.150000in}{0.150000in}}{\pgfqpoint{5.700000in}{5.700000in}}%
\pgfusepath{clip}%
\pgfsetbuttcap%
\pgfsetroundjoin%
\definecolor{currentfill}{rgb}{0.235526,0.309527,0.542944}%
\pgfsetfillcolor{currentfill}%
\pgfsetfillopacity{0.700000}%
\pgfsetlinewidth{0.000000pt}%
\definecolor{currentstroke}{rgb}{0.000000,0.000000,0.000000}%
\pgfsetstrokecolor{currentstroke}%
\pgfsetdash{}{0pt}%
\pgfpathmoveto{\pgfqpoint{4.982884in}{2.244273in}}%
\pgfpathlineto{\pgfqpoint{4.997203in}{2.247389in}}%
\pgfpathlineto{\pgfqpoint{5.011534in}{2.250575in}}%
\pgfpathlineto{\pgfqpoint{5.025877in}{2.253831in}}%
\pgfpathlineto{\pgfqpoint{5.040231in}{2.257159in}}%
\pgfpathlineto{\pgfqpoint{5.047926in}{2.264077in}}%
\pgfpathlineto{\pgfqpoint{5.055613in}{2.270905in}}%
\pgfpathlineto{\pgfqpoint{5.063292in}{2.277645in}}%
\pgfpathlineto{\pgfqpoint{5.070964in}{2.284299in}}%
\pgfpathlineto{\pgfqpoint{5.056623in}{2.281082in}}%
\pgfpathlineto{\pgfqpoint{5.042294in}{2.277936in}}%
\pgfpathlineto{\pgfqpoint{5.027977in}{2.274860in}}%
\pgfpathlineto{\pgfqpoint{5.013672in}{2.271855in}}%
\pgfpathlineto{\pgfqpoint{5.005986in}{2.265082in}}%
\pgfpathlineto{\pgfqpoint{4.998293in}{2.258229in}}%
\pgfpathlineto{\pgfqpoint{4.990593in}{2.251294in}}%
\pgfpathlineto{\pgfqpoint{4.982884in}{2.244273in}}%
\pgfpathclose%
\pgfusepath{fill}%
\end{pgfscope}%
\begin{pgfscope}%
\pgfpathrectangle{\pgfqpoint{1.150000in}{0.150000in}}{\pgfqpoint{5.700000in}{5.700000in}}%
\pgfusepath{clip}%
\pgfsetbuttcap%
\pgfsetroundjoin%
\definecolor{currentfill}{rgb}{0.273006,0.204520,0.501721}%
\pgfsetfillcolor{currentfill}%
\pgfsetfillopacity{0.700000}%
\pgfsetlinewidth{0.000000pt}%
\definecolor{currentstroke}{rgb}{0.000000,0.000000,0.000000}%
\pgfsetstrokecolor{currentstroke}%
\pgfsetdash{}{0pt}%
\pgfpathmoveto{\pgfqpoint{2.225085in}{2.046627in}}%
\pgfpathlineto{\pgfqpoint{2.238948in}{2.034231in}}%
\pgfpathlineto{\pgfqpoint{2.252809in}{2.021955in}}%
\pgfpathlineto{\pgfqpoint{2.266667in}{2.009797in}}%
\pgfpathlineto{\pgfqpoint{2.280524in}{1.997759in}}%
\pgfpathlineto{\pgfqpoint{2.289583in}{1.997191in}}%
\pgfpathlineto{\pgfqpoint{2.298623in}{1.996888in}}%
\pgfpathlineto{\pgfqpoint{2.307645in}{1.996843in}}%
\pgfpathlineto{\pgfqpoint{2.316648in}{1.997050in}}%
\pgfpathlineto{\pgfqpoint{2.302831in}{2.008706in}}%
\pgfpathlineto{\pgfqpoint{2.289012in}{2.020481in}}%
\pgfpathlineto{\pgfqpoint{2.275191in}{2.032374in}}%
\pgfpathlineto{\pgfqpoint{2.261368in}{2.044388in}}%
\pgfpathlineto{\pgfqpoint{2.252325in}{2.044555in}}%
\pgfpathlineto{\pgfqpoint{2.243265in}{2.044980in}}%
\pgfpathlineto{\pgfqpoint{2.234185in}{2.045669in}}%
\pgfpathlineto{\pgfqpoint{2.225085in}{2.046627in}}%
\pgfpathclose%
\pgfusepath{fill}%
\end{pgfscope}%
\begin{pgfscope}%
\pgfpathrectangle{\pgfqpoint{1.150000in}{0.150000in}}{\pgfqpoint{5.700000in}{5.700000in}}%
\pgfusepath{clip}%
\pgfsetbuttcap%
\pgfsetroundjoin%
\definecolor{currentfill}{rgb}{0.269944,0.014625,0.341379}%
\pgfsetfillcolor{currentfill}%
\pgfsetfillopacity{0.700000}%
\pgfsetlinewidth{0.000000pt}%
\definecolor{currentstroke}{rgb}{0.000000,0.000000,0.000000}%
\pgfsetstrokecolor{currentstroke}%
\pgfsetdash{}{0pt}%
\pgfpathmoveto{\pgfqpoint{3.459404in}{1.630981in}}%
\pgfpathlineto{\pgfqpoint{3.473240in}{1.627653in}}%
\pgfpathlineto{\pgfqpoint{3.487082in}{1.624403in}}%
\pgfpathlineto{\pgfqpoint{3.500931in}{1.621233in}}%
\pgfpathlineto{\pgfqpoint{3.514785in}{1.618142in}}%
\pgfpathlineto{\pgfqpoint{3.523069in}{1.627251in}}%
\pgfpathlineto{\pgfqpoint{3.531347in}{1.636400in}}%
\pgfpathlineto{\pgfqpoint{3.539618in}{1.645587in}}%
\pgfpathlineto{\pgfqpoint{3.547883in}{1.654808in}}%
\pgfpathlineto{\pgfqpoint{3.534042in}{1.657674in}}%
\pgfpathlineto{\pgfqpoint{3.520208in}{1.660618in}}%
\pgfpathlineto{\pgfqpoint{3.506380in}{1.663643in}}%
\pgfpathlineto{\pgfqpoint{3.492558in}{1.666746in}}%
\pgfpathlineto{\pgfqpoint{3.484280in}{1.657743in}}%
\pgfpathlineto{\pgfqpoint{3.475994in}{1.648779in}}%
\pgfpathlineto{\pgfqpoint{3.467702in}{1.639857in}}%
\pgfpathlineto{\pgfqpoint{3.459404in}{1.630981in}}%
\pgfpathclose%
\pgfusepath{fill}%
\end{pgfscope}%
\begin{pgfscope}%
\pgfpathrectangle{\pgfqpoint{1.150000in}{0.150000in}}{\pgfqpoint{5.700000in}{5.700000in}}%
\pgfusepath{clip}%
\pgfsetbuttcap%
\pgfsetroundjoin%
\definecolor{currentfill}{rgb}{0.282910,0.105393,0.426902}%
\pgfsetfillcolor{currentfill}%
\pgfsetfillopacity{0.700000}%
\pgfsetlinewidth{0.000000pt}%
\definecolor{currentstroke}{rgb}{0.000000,0.000000,0.000000}%
\pgfsetstrokecolor{currentstroke}%
\pgfsetdash{}{0pt}%
\pgfpathmoveto{\pgfqpoint{2.537590in}{1.825860in}}%
\pgfpathlineto{\pgfqpoint{2.551395in}{1.816076in}}%
\pgfpathlineto{\pgfqpoint{2.565202in}{1.806396in}}%
\pgfpathlineto{\pgfqpoint{2.579008in}{1.796819in}}%
\pgfpathlineto{\pgfqpoint{2.592816in}{1.787344in}}%
\pgfpathlineto{\pgfqpoint{2.601627in}{1.789622in}}%
\pgfpathlineto{\pgfqpoint{2.610424in}{1.792113in}}%
\pgfpathlineto{\pgfqpoint{2.619206in}{1.794812in}}%
\pgfpathlineto{\pgfqpoint{2.627974in}{1.797714in}}%
\pgfpathlineto{\pgfqpoint{2.614198in}{1.806835in}}%
\pgfpathlineto{\pgfqpoint{2.600423in}{1.816058in}}%
\pgfpathlineto{\pgfqpoint{2.586649in}{1.825383in}}%
\pgfpathlineto{\pgfqpoint{2.572876in}{1.834811in}}%
\pgfpathlineto{\pgfqpoint{2.564077in}{1.832256in}}%
\pgfpathlineto{\pgfqpoint{2.555263in}{1.829909in}}%
\pgfpathlineto{\pgfqpoint{2.546434in}{1.827775in}}%
\pgfpathlineto{\pgfqpoint{2.537590in}{1.825860in}}%
\pgfpathclose%
\pgfusepath{fill}%
\end{pgfscope}%
\begin{pgfscope}%
\pgfpathrectangle{\pgfqpoint{1.150000in}{0.150000in}}{\pgfqpoint{5.700000in}{5.700000in}}%
\pgfusepath{clip}%
\pgfsetbuttcap%
\pgfsetroundjoin%
\definecolor{currentfill}{rgb}{0.241237,0.296485,0.539709}%
\pgfsetfillcolor{currentfill}%
\pgfsetfillopacity{0.700000}%
\pgfsetlinewidth{0.000000pt}%
\definecolor{currentstroke}{rgb}{0.000000,0.000000,0.000000}%
\pgfsetstrokecolor{currentstroke}%
\pgfsetdash{}{0pt}%
\pgfpathmoveto{\pgfqpoint{4.894765in}{2.203195in}}%
\pgfpathlineto{\pgfqpoint{4.909050in}{2.206115in}}%
\pgfpathlineto{\pgfqpoint{4.923347in}{2.209107in}}%
\pgfpathlineto{\pgfqpoint{4.937656in}{2.212170in}}%
\pgfpathlineto{\pgfqpoint{4.951976in}{2.215303in}}%
\pgfpathlineto{\pgfqpoint{4.959715in}{2.222683in}}%
\pgfpathlineto{\pgfqpoint{4.967445in}{2.229970in}}%
\pgfpathlineto{\pgfqpoint{4.975169in}{2.237166in}}%
\pgfpathlineto{\pgfqpoint{4.982884in}{2.244273in}}%
\pgfpathlineto{\pgfqpoint{4.968577in}{2.241229in}}%
\pgfpathlineto{\pgfqpoint{4.954281in}{2.238255in}}%
\pgfpathlineto{\pgfqpoint{4.939997in}{2.235352in}}%
\pgfpathlineto{\pgfqpoint{4.925724in}{2.232520in}}%
\pgfpathlineto{\pgfqpoint{4.917995in}{2.225316in}}%
\pgfpathlineto{\pgfqpoint{4.910259in}{2.218028in}}%
\pgfpathlineto{\pgfqpoint{4.902515in}{2.210655in}}%
\pgfpathlineto{\pgfqpoint{4.894765in}{2.203195in}}%
\pgfpathclose%
\pgfusepath{fill}%
\end{pgfscope}%
\begin{pgfscope}%
\pgfpathrectangle{\pgfqpoint{1.150000in}{0.150000in}}{\pgfqpoint{5.700000in}{5.700000in}}%
\pgfusepath{clip}%
\pgfsetbuttcap%
\pgfsetroundjoin%
\definecolor{currentfill}{rgb}{0.268510,0.009605,0.335427}%
\pgfsetfillcolor{currentfill}%
\pgfsetfillopacity{0.700000}%
\pgfsetlinewidth{0.000000pt}%
\definecolor{currentstroke}{rgb}{0.000000,0.000000,0.000000}%
\pgfsetstrokecolor{currentstroke}%
\pgfsetdash{}{0pt}%
\pgfpathmoveto{\pgfqpoint{3.082595in}{1.629023in}}%
\pgfpathlineto{\pgfqpoint{3.096387in}{1.623242in}}%
\pgfpathlineto{\pgfqpoint{3.110183in}{1.617547in}}%
\pgfpathlineto{\pgfqpoint{3.123984in}{1.611938in}}%
\pgfpathlineto{\pgfqpoint{3.137788in}{1.606414in}}%
\pgfpathlineto{\pgfqpoint{3.146250in}{1.613279in}}%
\pgfpathlineto{\pgfqpoint{3.154703in}{1.620256in}}%
\pgfpathlineto{\pgfqpoint{3.163146in}{1.627341in}}%
\pgfpathlineto{\pgfqpoint{3.171582in}{1.634529in}}%
\pgfpathlineto{\pgfqpoint{3.157798in}{1.639766in}}%
\pgfpathlineto{\pgfqpoint{3.144018in}{1.645088in}}%
\pgfpathlineto{\pgfqpoint{3.130242in}{1.650496in}}%
\pgfpathlineto{\pgfqpoint{3.116470in}{1.655990in}}%
\pgfpathlineto{\pgfqpoint{3.108015in}{1.649081in}}%
\pgfpathlineto{\pgfqpoint{3.099551in}{1.642281in}}%
\pgfpathlineto{\pgfqpoint{3.091077in}{1.635594in}}%
\pgfpathlineto{\pgfqpoint{3.082595in}{1.629023in}}%
\pgfpathclose%
\pgfusepath{fill}%
\end{pgfscope}%
\begin{pgfscope}%
\pgfpathrectangle{\pgfqpoint{1.150000in}{0.150000in}}{\pgfqpoint{5.700000in}{5.700000in}}%
\pgfusepath{clip}%
\pgfsetbuttcap%
\pgfsetroundjoin%
\definecolor{currentfill}{rgb}{0.277941,0.056324,0.381191}%
\pgfsetfillcolor{currentfill}%
\pgfsetfillopacity{0.700000}%
\pgfsetlinewidth{0.000000pt}%
\definecolor{currentstroke}{rgb}{0.000000,0.000000,0.000000}%
\pgfsetstrokecolor{currentstroke}%
\pgfsetdash{}{0pt}%
\pgfpathmoveto{\pgfqpoint{3.780080in}{1.705810in}}%
\pgfpathlineto{\pgfqpoint{3.793985in}{1.704340in}}%
\pgfpathlineto{\pgfqpoint{3.807897in}{1.702946in}}%
\pgfpathlineto{\pgfqpoint{3.821817in}{1.701628in}}%
\pgfpathlineto{\pgfqpoint{3.835744in}{1.700385in}}%
\pgfpathlineto{\pgfqpoint{3.843908in}{1.710472in}}%
\pgfpathlineto{\pgfqpoint{3.852066in}{1.720546in}}%
\pgfpathlineto{\pgfqpoint{3.860219in}{1.730604in}}%
\pgfpathlineto{\pgfqpoint{3.868365in}{1.740644in}}%
\pgfpathlineto{\pgfqpoint{3.854448in}{1.741723in}}%
\pgfpathlineto{\pgfqpoint{3.840539in}{1.742877in}}%
\pgfpathlineto{\pgfqpoint{3.826637in}{1.744107in}}%
\pgfpathlineto{\pgfqpoint{3.812743in}{1.745413in}}%
\pgfpathlineto{\pgfqpoint{3.804586in}{1.735529in}}%
\pgfpathlineto{\pgfqpoint{3.796423in}{1.725632in}}%
\pgfpathlineto{\pgfqpoint{3.788254in}{1.715725in}}%
\pgfpathlineto{\pgfqpoint{3.780080in}{1.705810in}}%
\pgfpathclose%
\pgfusepath{fill}%
\end{pgfscope}%
\begin{pgfscope}%
\pgfpathrectangle{\pgfqpoint{1.150000in}{0.150000in}}{\pgfqpoint{5.700000in}{5.700000in}}%
\pgfusepath{clip}%
\pgfsetbuttcap%
\pgfsetroundjoin%
\definecolor{currentfill}{rgb}{0.271305,0.019942,0.347269}%
\pgfsetfillcolor{currentfill}%
\pgfsetfillopacity{0.700000}%
\pgfsetlinewidth{0.000000pt}%
\definecolor{currentstroke}{rgb}{0.000000,0.000000,0.000000}%
\pgfsetstrokecolor{currentstroke}%
\pgfsetdash{}{0pt}%
\pgfpathmoveto{\pgfqpoint{2.938174in}{1.655859in}}%
\pgfpathlineto{\pgfqpoint{2.951962in}{1.649067in}}%
\pgfpathlineto{\pgfqpoint{2.965753in}{1.642365in}}%
\pgfpathlineto{\pgfqpoint{2.979548in}{1.635751in}}%
\pgfpathlineto{\pgfqpoint{2.993345in}{1.629226in}}%
\pgfpathlineto{\pgfqpoint{3.001890in}{1.634967in}}%
\pgfpathlineto{\pgfqpoint{3.010424in}{1.640849in}}%
\pgfpathlineto{\pgfqpoint{3.018948in}{1.646866in}}%
\pgfpathlineto{\pgfqpoint{3.027462in}{1.653014in}}%
\pgfpathlineto{\pgfqpoint{3.013688in}{1.659231in}}%
\pgfpathlineto{\pgfqpoint{2.999917in}{1.665536in}}%
\pgfpathlineto{\pgfqpoint{2.986149in}{1.671930in}}%
\pgfpathlineto{\pgfqpoint{2.972385in}{1.678413in}}%
\pgfpathlineto{\pgfqpoint{2.963848in}{1.672566in}}%
\pgfpathlineto{\pgfqpoint{2.955300in}{1.666854in}}%
\pgfpathlineto{\pgfqpoint{2.946743in}{1.661284in}}%
\pgfpathlineto{\pgfqpoint{2.938174in}{1.655859in}}%
\pgfpathclose%
\pgfusepath{fill}%
\end{pgfscope}%
\begin{pgfscope}%
\pgfpathrectangle{\pgfqpoint{1.150000in}{0.150000in}}{\pgfqpoint{5.700000in}{5.700000in}}%
\pgfusepath{clip}%
\pgfsetbuttcap%
\pgfsetroundjoin%
\definecolor{currentfill}{rgb}{0.248629,0.278775,0.534556}%
\pgfsetfillcolor{currentfill}%
\pgfsetfillopacity{0.700000}%
\pgfsetlinewidth{0.000000pt}%
\definecolor{currentstroke}{rgb}{0.000000,0.000000,0.000000}%
\pgfsetstrokecolor{currentstroke}%
\pgfsetdash{}{0pt}%
\pgfpathmoveto{\pgfqpoint{4.806611in}{2.161202in}}%
\pgfpathlineto{\pgfqpoint{4.820864in}{2.163906in}}%
\pgfpathlineto{\pgfqpoint{4.835127in}{2.166680in}}%
\pgfpathlineto{\pgfqpoint{4.849402in}{2.169526in}}%
\pgfpathlineto{\pgfqpoint{4.863688in}{2.172443in}}%
\pgfpathlineto{\pgfqpoint{4.871468in}{2.180271in}}%
\pgfpathlineto{\pgfqpoint{4.879241in}{2.188004in}}%
\pgfpathlineto{\pgfqpoint{4.887006in}{2.195645in}}%
\pgfpathlineto{\pgfqpoint{4.894765in}{2.203195in}}%
\pgfpathlineto{\pgfqpoint{4.880490in}{2.200345in}}%
\pgfpathlineto{\pgfqpoint{4.866228in}{2.197566in}}%
\pgfpathlineto{\pgfqpoint{4.851976in}{2.194859in}}%
\pgfpathlineto{\pgfqpoint{4.837736in}{2.192222in}}%
\pgfpathlineto{\pgfqpoint{4.829965in}{2.184597in}}%
\pgfpathlineto{\pgfqpoint{4.822188in}{2.176886in}}%
\pgfpathlineto{\pgfqpoint{4.814403in}{2.169089in}}%
\pgfpathlineto{\pgfqpoint{4.806611in}{2.161202in}}%
\pgfpathclose%
\pgfusepath{fill}%
\end{pgfscope}%
\begin{pgfscope}%
\pgfpathrectangle{\pgfqpoint{1.150000in}{0.150000in}}{\pgfqpoint{5.700000in}{5.700000in}}%
\pgfusepath{clip}%
\pgfsetbuttcap%
\pgfsetroundjoin%
\definecolor{currentfill}{rgb}{0.267004,0.004874,0.329415}%
\pgfsetfillcolor{currentfill}%
\pgfsetfillopacity{0.700000}%
\pgfsetlinewidth{0.000000pt}%
\definecolor{currentstroke}{rgb}{0.000000,0.000000,0.000000}%
\pgfsetstrokecolor{currentstroke}%
\pgfsetdash{}{0pt}%
\pgfpathmoveto{\pgfqpoint{3.226762in}{1.614426in}}%
\pgfpathlineto{\pgfqpoint{3.240568in}{1.609609in}}%
\pgfpathlineto{\pgfqpoint{3.254379in}{1.604875in}}%
\pgfpathlineto{\pgfqpoint{3.268195in}{1.600224in}}%
\pgfpathlineto{\pgfqpoint{3.282016in}{1.595656in}}%
\pgfpathlineto{\pgfqpoint{3.290405in}{1.603491in}}%
\pgfpathlineto{\pgfqpoint{3.298787in}{1.611412in}}%
\pgfpathlineto{\pgfqpoint{3.307161in}{1.619414in}}%
\pgfpathlineto{\pgfqpoint{3.315527in}{1.627492in}}%
\pgfpathlineto{\pgfqpoint{3.301724in}{1.631795in}}%
\pgfpathlineto{\pgfqpoint{3.287926in}{1.636179in}}%
\pgfpathlineto{\pgfqpoint{3.274133in}{1.640647in}}%
\pgfpathlineto{\pgfqpoint{3.260344in}{1.645197in}}%
\pgfpathlineto{\pgfqpoint{3.251961in}{1.637377in}}%
\pgfpathlineto{\pgfqpoint{3.243569in}{1.629639in}}%
\pgfpathlineto{\pgfqpoint{3.235170in}{1.621987in}}%
\pgfpathlineto{\pgfqpoint{3.226762in}{1.614426in}}%
\pgfpathclose%
\pgfusepath{fill}%
\end{pgfscope}%
\begin{pgfscope}%
\pgfpathrectangle{\pgfqpoint{1.150000in}{0.150000in}}{\pgfqpoint{5.700000in}{5.700000in}}%
\pgfusepath{clip}%
\pgfsetbuttcap%
\pgfsetroundjoin%
\definecolor{currentfill}{rgb}{0.192357,0.403199,0.555836}%
\pgfsetfillcolor{currentfill}%
\pgfsetfillopacity{0.700000}%
\pgfsetlinewidth{0.000000pt}%
\definecolor{currentstroke}{rgb}{0.000000,0.000000,0.000000}%
\pgfsetstrokecolor{currentstroke}%
\pgfsetdash{}{0pt}%
\pgfpathmoveto{\pgfqpoint{5.568607in}{2.480970in}}%
\pgfpathlineto{\pgfqpoint{5.583171in}{2.485057in}}%
\pgfpathlineto{\pgfqpoint{5.597748in}{2.489215in}}%
\pgfpathlineto{\pgfqpoint{5.612339in}{2.493442in}}%
\pgfpathlineto{\pgfqpoint{5.626942in}{2.497738in}}%
\pgfpathlineto{\pgfqpoint{5.634319in}{2.501660in}}%
\pgfpathlineto{\pgfqpoint{5.641687in}{2.505532in}}%
\pgfpathlineto{\pgfqpoint{5.649048in}{2.509359in}}%
\pgfpathlineto{\pgfqpoint{5.656401in}{2.513145in}}%
\pgfpathlineto{\pgfqpoint{5.641820in}{2.509091in}}%
\pgfpathlineto{\pgfqpoint{5.627252in}{2.505105in}}%
\pgfpathlineto{\pgfqpoint{5.612697in}{2.501189in}}%
\pgfpathlineto{\pgfqpoint{5.598155in}{2.497342in}}%
\pgfpathlineto{\pgfqpoint{5.590780in}{2.493308in}}%
\pgfpathlineto{\pgfqpoint{5.583397in}{2.489237in}}%
\pgfpathlineto{\pgfqpoint{5.576006in}{2.485125in}}%
\pgfpathlineto{\pgfqpoint{5.568607in}{2.480970in}}%
\pgfpathclose%
\pgfusepath{fill}%
\end{pgfscope}%
\begin{pgfscope}%
\pgfpathrectangle{\pgfqpoint{1.150000in}{0.150000in}}{\pgfqpoint{5.700000in}{5.700000in}}%
\pgfusepath{clip}%
\pgfsetbuttcap%
\pgfsetroundjoin%
\definecolor{currentfill}{rgb}{0.255645,0.260703,0.528312}%
\pgfsetfillcolor{currentfill}%
\pgfsetfillopacity{0.700000}%
\pgfsetlinewidth{0.000000pt}%
\definecolor{currentstroke}{rgb}{0.000000,0.000000,0.000000}%
\pgfsetstrokecolor{currentstroke}%
\pgfsetdash{}{0pt}%
\pgfpathmoveto{\pgfqpoint{4.718431in}{2.118457in}}%
\pgfpathlineto{\pgfqpoint{4.732650in}{2.120921in}}%
\pgfpathlineto{\pgfqpoint{4.746880in}{2.123456in}}%
\pgfpathlineto{\pgfqpoint{4.761121in}{2.126063in}}%
\pgfpathlineto{\pgfqpoint{4.775373in}{2.128741in}}%
\pgfpathlineto{\pgfqpoint{4.783193in}{2.136996in}}%
\pgfpathlineto{\pgfqpoint{4.791006in}{2.145157in}}%
\pgfpathlineto{\pgfqpoint{4.798812in}{2.153225in}}%
\pgfpathlineto{\pgfqpoint{4.806611in}{2.161202in}}%
\pgfpathlineto{\pgfqpoint{4.792371in}{2.158570in}}%
\pgfpathlineto{\pgfqpoint{4.778141in}{2.156009in}}%
\pgfpathlineto{\pgfqpoint{4.763922in}{2.153519in}}%
\pgfpathlineto{\pgfqpoint{4.749714in}{2.151100in}}%
\pgfpathlineto{\pgfqpoint{4.741904in}{2.143070in}}%
\pgfpathlineto{\pgfqpoint{4.734087in}{2.134953in}}%
\pgfpathlineto{\pgfqpoint{4.726262in}{2.126749in}}%
\pgfpathlineto{\pgfqpoint{4.718431in}{2.118457in}}%
\pgfpathclose%
\pgfusepath{fill}%
\end{pgfscope}%
\begin{pgfscope}%
\pgfpathrectangle{\pgfqpoint{1.150000in}{0.150000in}}{\pgfqpoint{5.700000in}{5.700000in}}%
\pgfusepath{clip}%
\pgfsetbuttcap%
\pgfsetroundjoin%
\definecolor{currentfill}{rgb}{0.262138,0.242286,0.520837}%
\pgfsetfillcolor{currentfill}%
\pgfsetfillopacity{0.700000}%
\pgfsetlinewidth{0.000000pt}%
\definecolor{currentstroke}{rgb}{0.000000,0.000000,0.000000}%
\pgfsetstrokecolor{currentstroke}%
\pgfsetdash{}{0pt}%
\pgfpathmoveto{\pgfqpoint{4.630230in}{2.075141in}}%
\pgfpathlineto{\pgfqpoint{4.644416in}{2.077343in}}%
\pgfpathlineto{\pgfqpoint{4.658613in}{2.079617in}}%
\pgfpathlineto{\pgfqpoint{4.672820in}{2.081962in}}%
\pgfpathlineto{\pgfqpoint{4.687038in}{2.084379in}}%
\pgfpathlineto{\pgfqpoint{4.694897in}{2.093036in}}%
\pgfpathlineto{\pgfqpoint{4.702749in}{2.101601in}}%
\pgfpathlineto{\pgfqpoint{4.710593in}{2.110074in}}%
\pgfpathlineto{\pgfqpoint{4.718431in}{2.118457in}}%
\pgfpathlineto{\pgfqpoint{4.704223in}{2.116064in}}%
\pgfpathlineto{\pgfqpoint{4.690026in}{2.113743in}}%
\pgfpathlineto{\pgfqpoint{4.675840in}{2.111494in}}%
\pgfpathlineto{\pgfqpoint{4.661665in}{2.109316in}}%
\pgfpathlineto{\pgfqpoint{4.653816in}{2.100901in}}%
\pgfpathlineto{\pgfqpoint{4.645961in}{2.092401in}}%
\pgfpathlineto{\pgfqpoint{4.638099in}{2.083815in}}%
\pgfpathlineto{\pgfqpoint{4.630230in}{2.075141in}}%
\pgfpathclose%
\pgfusepath{fill}%
\end{pgfscope}%
\begin{pgfscope}%
\pgfpathrectangle{\pgfqpoint{1.150000in}{0.150000in}}{\pgfqpoint{5.700000in}{5.700000in}}%
\pgfusepath{clip}%
\pgfsetbuttcap%
\pgfsetroundjoin%
\definecolor{currentfill}{rgb}{0.277134,0.185228,0.489898}%
\pgfsetfillcolor{currentfill}%
\pgfsetfillopacity{0.700000}%
\pgfsetlinewidth{0.000000pt}%
\definecolor{currentstroke}{rgb}{0.000000,0.000000,0.000000}%
\pgfsetstrokecolor{currentstroke}%
\pgfsetdash{}{0pt}%
\pgfpathmoveto{\pgfqpoint{2.280524in}{1.997759in}}%
\pgfpathlineto{\pgfqpoint{2.294378in}{1.985837in}}%
\pgfpathlineto{\pgfqpoint{2.308231in}{1.974032in}}%
\pgfpathlineto{\pgfqpoint{2.322082in}{1.962342in}}%
\pgfpathlineto{\pgfqpoint{2.335932in}{1.950767in}}%
\pgfpathlineto{\pgfqpoint{2.344952in}{1.950589in}}%
\pgfpathlineto{\pgfqpoint{2.353954in}{1.950670in}}%
\pgfpathlineto{\pgfqpoint{2.362937in}{1.951003in}}%
\pgfpathlineto{\pgfqpoint{2.371903in}{1.951583in}}%
\pgfpathlineto{\pgfqpoint{2.358092in}{1.962778in}}%
\pgfpathlineto{\pgfqpoint{2.344279in}{1.974086in}}%
\pgfpathlineto{\pgfqpoint{2.330464in}{1.985510in}}%
\pgfpathlineto{\pgfqpoint{2.316648in}{1.997050in}}%
\pgfpathlineto{\pgfqpoint{2.307645in}{1.996843in}}%
\pgfpathlineto{\pgfqpoint{2.298623in}{1.996888in}}%
\pgfpathlineto{\pgfqpoint{2.289583in}{1.997191in}}%
\pgfpathlineto{\pgfqpoint{2.280524in}{1.997759in}}%
\pgfpathclose%
\pgfusepath{fill}%
\end{pgfscope}%
\begin{pgfscope}%
\pgfpathrectangle{\pgfqpoint{1.150000in}{0.150000in}}{\pgfqpoint{5.700000in}{5.700000in}}%
\pgfusepath{clip}%
\pgfsetbuttcap%
\pgfsetroundjoin%
\definecolor{currentfill}{rgb}{0.276022,0.044167,0.370164}%
\pgfsetfillcolor{currentfill}%
\pgfsetfillopacity{0.700000}%
\pgfsetlinewidth{0.000000pt}%
\definecolor{currentstroke}{rgb}{0.000000,0.000000,0.000000}%
\pgfsetstrokecolor{currentstroke}%
\pgfsetdash{}{0pt}%
\pgfpathmoveto{\pgfqpoint{3.691733in}{1.673499in}}%
\pgfpathlineto{\pgfqpoint{3.705620in}{1.671538in}}%
\pgfpathlineto{\pgfqpoint{3.719515in}{1.669655in}}%
\pgfpathlineto{\pgfqpoint{3.733416in}{1.667847in}}%
\pgfpathlineto{\pgfqpoint{3.747325in}{1.666116in}}%
\pgfpathlineto{\pgfqpoint{3.755522in}{1.676040in}}%
\pgfpathlineto{\pgfqpoint{3.763714in}{1.685965in}}%
\pgfpathlineto{\pgfqpoint{3.771900in}{1.695890in}}%
\pgfpathlineto{\pgfqpoint{3.780080in}{1.705810in}}%
\pgfpathlineto{\pgfqpoint{3.766182in}{1.707357in}}%
\pgfpathlineto{\pgfqpoint{3.752292in}{1.708979in}}%
\pgfpathlineto{\pgfqpoint{3.738409in}{1.710679in}}%
\pgfpathlineto{\pgfqpoint{3.724533in}{1.712455in}}%
\pgfpathlineto{\pgfqpoint{3.716342in}{1.702711in}}%
\pgfpathlineto{\pgfqpoint{3.708145in}{1.692969in}}%
\pgfpathlineto{\pgfqpoint{3.699942in}{1.683230in}}%
\pgfpathlineto{\pgfqpoint{3.691733in}{1.673499in}}%
\pgfpathclose%
\pgfusepath{fill}%
\end{pgfscope}%
\begin{pgfscope}%
\pgfpathrectangle{\pgfqpoint{1.150000in}{0.150000in}}{\pgfqpoint{5.700000in}{5.700000in}}%
\pgfusepath{clip}%
\pgfsetbuttcap%
\pgfsetroundjoin%
\definecolor{currentfill}{rgb}{0.267968,0.223549,0.512008}%
\pgfsetfillcolor{currentfill}%
\pgfsetfillopacity{0.700000}%
\pgfsetlinewidth{0.000000pt}%
\definecolor{currentstroke}{rgb}{0.000000,0.000000,0.000000}%
\pgfsetstrokecolor{currentstroke}%
\pgfsetdash{}{0pt}%
\pgfpathmoveto{\pgfqpoint{4.542012in}{2.031460in}}%
\pgfpathlineto{\pgfqpoint{4.556166in}{2.033377in}}%
\pgfpathlineto{\pgfqpoint{4.570330in}{2.035367in}}%
\pgfpathlineto{\pgfqpoint{4.584504in}{2.037428in}}%
\pgfpathlineto{\pgfqpoint{4.598689in}{2.039561in}}%
\pgfpathlineto{\pgfqpoint{4.606584in}{2.048590in}}%
\pgfpathlineto{\pgfqpoint{4.614473in}{2.057529in}}%
\pgfpathlineto{\pgfqpoint{4.622355in}{2.066380in}}%
\pgfpathlineto{\pgfqpoint{4.630230in}{2.075141in}}%
\pgfpathlineto{\pgfqpoint{4.616055in}{2.073011in}}%
\pgfpathlineto{\pgfqpoint{4.601890in}{2.070952in}}%
\pgfpathlineto{\pgfqpoint{4.587736in}{2.068966in}}%
\pgfpathlineto{\pgfqpoint{4.573592in}{2.067051in}}%
\pgfpathlineto{\pgfqpoint{4.565707in}{2.058279in}}%
\pgfpathlineto{\pgfqpoint{4.557815in}{2.049423in}}%
\pgfpathlineto{\pgfqpoint{4.549917in}{2.040484in}}%
\pgfpathlineto{\pgfqpoint{4.542012in}{2.031460in}}%
\pgfpathclose%
\pgfusepath{fill}%
\end{pgfscope}%
\begin{pgfscope}%
\pgfpathrectangle{\pgfqpoint{1.150000in}{0.150000in}}{\pgfqpoint{5.700000in}{5.700000in}}%
\pgfusepath{clip}%
\pgfsetbuttcap%
\pgfsetroundjoin%
\definecolor{currentfill}{rgb}{0.195860,0.395433,0.555276}%
\pgfsetfillcolor{currentfill}%
\pgfsetfillopacity{0.700000}%
\pgfsetlinewidth{0.000000pt}%
\definecolor{currentstroke}{rgb}{0.000000,0.000000,0.000000}%
\pgfsetstrokecolor{currentstroke}%
\pgfsetdash{}{0pt}%
\pgfpathmoveto{\pgfqpoint{5.480724in}{2.447284in}}%
\pgfpathlineto{\pgfqpoint{5.495256in}{2.451313in}}%
\pgfpathlineto{\pgfqpoint{5.509802in}{2.455412in}}%
\pgfpathlineto{\pgfqpoint{5.524361in}{2.459580in}}%
\pgfpathlineto{\pgfqpoint{5.538933in}{2.463819in}}%
\pgfpathlineto{\pgfqpoint{5.546364in}{2.468194in}}%
\pgfpathlineto{\pgfqpoint{5.553786in}{2.472508in}}%
\pgfpathlineto{\pgfqpoint{5.561201in}{2.476765in}}%
\pgfpathlineto{\pgfqpoint{5.568607in}{2.480970in}}%
\pgfpathlineto{\pgfqpoint{5.554056in}{2.476951in}}%
\pgfpathlineto{\pgfqpoint{5.539518in}{2.473003in}}%
\pgfpathlineto{\pgfqpoint{5.524993in}{2.469123in}}%
\pgfpathlineto{\pgfqpoint{5.510481in}{2.465313in}}%
\pgfpathlineto{\pgfqpoint{5.503053in}{2.460881in}}%
\pgfpathlineto{\pgfqpoint{5.495618in}{2.456402in}}%
\pgfpathlineto{\pgfqpoint{5.488175in}{2.451870in}}%
\pgfpathlineto{\pgfqpoint{5.480724in}{2.447284in}}%
\pgfpathclose%
\pgfusepath{fill}%
\end{pgfscope}%
\begin{pgfscope}%
\pgfpathrectangle{\pgfqpoint{1.150000in}{0.150000in}}{\pgfqpoint{5.700000in}{5.700000in}}%
\pgfusepath{clip}%
\pgfsetbuttcap%
\pgfsetroundjoin%
\definecolor{currentfill}{rgb}{0.273006,0.204520,0.501721}%
\pgfsetfillcolor{currentfill}%
\pgfsetfillopacity{0.700000}%
\pgfsetlinewidth{0.000000pt}%
\definecolor{currentstroke}{rgb}{0.000000,0.000000,0.000000}%
\pgfsetstrokecolor{currentstroke}%
\pgfsetdash{}{0pt}%
\pgfpathmoveto{\pgfqpoint{4.453781in}{1.987638in}}%
\pgfpathlineto{\pgfqpoint{4.467903in}{1.989249in}}%
\pgfpathlineto{\pgfqpoint{4.482035in}{1.990932in}}%
\pgfpathlineto{\pgfqpoint{4.496177in}{1.992687in}}%
\pgfpathlineto{\pgfqpoint{4.510330in}{1.994514in}}%
\pgfpathlineto{\pgfqpoint{4.518260in}{2.003878in}}%
\pgfpathlineto{\pgfqpoint{4.526184in}{2.013157in}}%
\pgfpathlineto{\pgfqpoint{4.534101in}{2.022351in}}%
\pgfpathlineto{\pgfqpoint{4.542012in}{2.031460in}}%
\pgfpathlineto{\pgfqpoint{4.527869in}{2.029614in}}%
\pgfpathlineto{\pgfqpoint{4.513736in}{2.027841in}}%
\pgfpathlineto{\pgfqpoint{4.499614in}{2.026140in}}%
\pgfpathlineto{\pgfqpoint{4.485501in}{2.024511in}}%
\pgfpathlineto{\pgfqpoint{4.477581in}{2.015413in}}%
\pgfpathlineto{\pgfqpoint{4.469654in}{2.006235in}}%
\pgfpathlineto{\pgfqpoint{4.461721in}{1.996977in}}%
\pgfpathlineto{\pgfqpoint{4.453781in}{1.987638in}}%
\pgfpathclose%
\pgfusepath{fill}%
\end{pgfscope}%
\begin{pgfscope}%
\pgfpathrectangle{\pgfqpoint{1.150000in}{0.150000in}}{\pgfqpoint{5.700000in}{5.700000in}}%
\pgfusepath{clip}%
\pgfsetbuttcap%
\pgfsetroundjoin%
\definecolor{currentfill}{rgb}{0.276022,0.044167,0.370164}%
\pgfsetfillcolor{currentfill}%
\pgfsetfillopacity{0.700000}%
\pgfsetlinewidth{0.000000pt}%
\definecolor{currentstroke}{rgb}{0.000000,0.000000,0.000000}%
\pgfsetstrokecolor{currentstroke}%
\pgfsetdash{}{0pt}%
\pgfpathmoveto{\pgfqpoint{2.793373in}{1.695966in}}%
\pgfpathlineto{\pgfqpoint{2.807167in}{1.688111in}}%
\pgfpathlineto{\pgfqpoint{2.820964in}{1.680349in}}%
\pgfpathlineto{\pgfqpoint{2.834762in}{1.672680in}}%
\pgfpathlineto{\pgfqpoint{2.848563in}{1.665104in}}%
\pgfpathlineto{\pgfqpoint{2.857202in}{1.669560in}}%
\pgfpathlineto{\pgfqpoint{2.865830in}{1.674186in}}%
\pgfpathlineto{\pgfqpoint{2.874446in}{1.678978in}}%
\pgfpathlineto{\pgfqpoint{2.883050in}{1.683931in}}%
\pgfpathlineto{\pgfqpoint{2.869276in}{1.691177in}}%
\pgfpathlineto{\pgfqpoint{2.855504in}{1.698515in}}%
\pgfpathlineto{\pgfqpoint{2.841734in}{1.705946in}}%
\pgfpathlineto{\pgfqpoint{2.827967in}{1.713470in}}%
\pgfpathlineto{\pgfqpoint{2.819337in}{1.708840in}}%
\pgfpathlineto{\pgfqpoint{2.810695in}{1.704376in}}%
\pgfpathlineto{\pgfqpoint{2.802040in}{1.700083in}}%
\pgfpathlineto{\pgfqpoint{2.793373in}{1.695966in}}%
\pgfpathclose%
\pgfusepath{fill}%
\end{pgfscope}%
\begin{pgfscope}%
\pgfpathrectangle{\pgfqpoint{1.150000in}{0.150000in}}{\pgfqpoint{5.700000in}{5.700000in}}%
\pgfusepath{clip}%
\pgfsetbuttcap%
\pgfsetroundjoin%
\definecolor{currentfill}{rgb}{0.267004,0.004874,0.329415}%
\pgfsetfillcolor{currentfill}%
\pgfsetfillopacity{0.700000}%
\pgfsetlinewidth{0.000000pt}%
\definecolor{currentstroke}{rgb}{0.000000,0.000000,0.000000}%
\pgfsetstrokecolor{currentstroke}%
\pgfsetdash{}{0pt}%
\pgfpathmoveto{\pgfqpoint{3.370791in}{1.611101in}}%
\pgfpathlineto{\pgfqpoint{3.384620in}{1.607206in}}%
\pgfpathlineto{\pgfqpoint{3.398455in}{1.603391in}}%
\pgfpathlineto{\pgfqpoint{3.412295in}{1.599657in}}%
\pgfpathlineto{\pgfqpoint{3.426141in}{1.596002in}}%
\pgfpathlineto{\pgfqpoint{3.434467in}{1.604661in}}%
\pgfpathlineto{\pgfqpoint{3.442786in}{1.613380in}}%
\pgfpathlineto{\pgfqpoint{3.451098in}{1.622155in}}%
\pgfpathlineto{\pgfqpoint{3.459404in}{1.630981in}}%
\pgfpathlineto{\pgfqpoint{3.445573in}{1.634390in}}%
\pgfpathlineto{\pgfqpoint{3.431748in}{1.637879in}}%
\pgfpathlineto{\pgfqpoint{3.417929in}{1.641447in}}%
\pgfpathlineto{\pgfqpoint{3.404116in}{1.645097in}}%
\pgfpathlineto{\pgfqpoint{3.395795in}{1.636508in}}%
\pgfpathlineto{\pgfqpoint{3.387468in}{1.627977in}}%
\pgfpathlineto{\pgfqpoint{3.379133in}{1.619506in}}%
\pgfpathlineto{\pgfqpoint{3.370791in}{1.611101in}}%
\pgfpathclose%
\pgfusepath{fill}%
\end{pgfscope}%
\begin{pgfscope}%
\pgfpathrectangle{\pgfqpoint{1.150000in}{0.150000in}}{\pgfqpoint{5.700000in}{5.700000in}}%
\pgfusepath{clip}%
\pgfsetbuttcap%
\pgfsetroundjoin%
\definecolor{currentfill}{rgb}{0.277134,0.185228,0.489898}%
\pgfsetfillcolor{currentfill}%
\pgfsetfillopacity{0.700000}%
\pgfsetlinewidth{0.000000pt}%
\definecolor{currentstroke}{rgb}{0.000000,0.000000,0.000000}%
\pgfsetstrokecolor{currentstroke}%
\pgfsetdash{}{0pt}%
\pgfpathmoveto{\pgfqpoint{4.365539in}{1.943924in}}%
\pgfpathlineto{\pgfqpoint{4.379630in}{1.945205in}}%
\pgfpathlineto{\pgfqpoint{4.393731in}{1.946559in}}%
\pgfpathlineto{\pgfqpoint{4.407842in}{1.947985in}}%
\pgfpathlineto{\pgfqpoint{4.421962in}{1.949484in}}%
\pgfpathlineto{\pgfqpoint{4.429926in}{1.959143in}}%
\pgfpathlineto{\pgfqpoint{4.437884in}{1.968721in}}%
\pgfpathlineto{\pgfqpoint{4.445836in}{1.978220in}}%
\pgfpathlineto{\pgfqpoint{4.453781in}{1.987638in}}%
\pgfpathlineto{\pgfqpoint{4.439670in}{1.986100in}}%
\pgfpathlineto{\pgfqpoint{4.425568in}{1.984634in}}%
\pgfpathlineto{\pgfqpoint{4.411476in}{1.983241in}}%
\pgfpathlineto{\pgfqpoint{4.397394in}{1.981920in}}%
\pgfpathlineto{\pgfqpoint{4.389440in}{1.972533in}}%
\pgfpathlineto{\pgfqpoint{4.381479in}{1.963072in}}%
\pgfpathlineto{\pgfqpoint{4.373512in}{1.953535in}}%
\pgfpathlineto{\pgfqpoint{4.365539in}{1.943924in}}%
\pgfpathclose%
\pgfusepath{fill}%
\end{pgfscope}%
\begin{pgfscope}%
\pgfpathrectangle{\pgfqpoint{1.150000in}{0.150000in}}{\pgfqpoint{5.700000in}{5.700000in}}%
\pgfusepath{clip}%
\pgfsetbuttcap%
\pgfsetroundjoin%
\definecolor{currentfill}{rgb}{0.281924,0.089666,0.412415}%
\pgfsetfillcolor{currentfill}%
\pgfsetfillopacity{0.700000}%
\pgfsetlinewidth{0.000000pt}%
\definecolor{currentstroke}{rgb}{0.000000,0.000000,0.000000}%
\pgfsetstrokecolor{currentstroke}%
\pgfsetdash{}{0pt}%
\pgfpathmoveto{\pgfqpoint{2.592816in}{1.787344in}}%
\pgfpathlineto{\pgfqpoint{2.606623in}{1.777969in}}%
\pgfpathlineto{\pgfqpoint{2.620432in}{1.768696in}}%
\pgfpathlineto{\pgfqpoint{2.634241in}{1.759522in}}%
\pgfpathlineto{\pgfqpoint{2.648052in}{1.750448in}}%
\pgfpathlineto{\pgfqpoint{2.656831in}{1.753088in}}%
\pgfpathlineto{\pgfqpoint{2.665596in}{1.755936in}}%
\pgfpathlineto{\pgfqpoint{2.674347in}{1.758987in}}%
\pgfpathlineto{\pgfqpoint{2.683085in}{1.762234in}}%
\pgfpathlineto{\pgfqpoint{2.669305in}{1.770954in}}%
\pgfpathlineto{\pgfqpoint{2.655527in}{1.779774in}}%
\pgfpathlineto{\pgfqpoint{2.641750in}{1.788694in}}%
\pgfpathlineto{\pgfqpoint{2.627974in}{1.797714in}}%
\pgfpathlineto{\pgfqpoint{2.619206in}{1.794812in}}%
\pgfpathlineto{\pgfqpoint{2.610424in}{1.792113in}}%
\pgfpathlineto{\pgfqpoint{2.601627in}{1.789622in}}%
\pgfpathlineto{\pgfqpoint{2.592816in}{1.787344in}}%
\pgfpathclose%
\pgfusepath{fill}%
\end{pgfscope}%
\begin{pgfscope}%
\pgfpathrectangle{\pgfqpoint{1.150000in}{0.150000in}}{\pgfqpoint{5.700000in}{5.700000in}}%
\pgfusepath{clip}%
\pgfsetbuttcap%
\pgfsetroundjoin%
\definecolor{currentfill}{rgb}{0.280255,0.165693,0.476498}%
\pgfsetfillcolor{currentfill}%
\pgfsetfillopacity{0.700000}%
\pgfsetlinewidth{0.000000pt}%
\definecolor{currentstroke}{rgb}{0.000000,0.000000,0.000000}%
\pgfsetstrokecolor{currentstroke}%
\pgfsetdash{}{0pt}%
\pgfpathmoveto{\pgfqpoint{4.277286in}{1.900585in}}%
\pgfpathlineto{\pgfqpoint{4.291348in}{1.901514in}}%
\pgfpathlineto{\pgfqpoint{4.305418in}{1.902517in}}%
\pgfpathlineto{\pgfqpoint{4.319498in}{1.903592in}}%
\pgfpathlineto{\pgfqpoint{4.333588in}{1.904740in}}%
\pgfpathlineto{\pgfqpoint{4.341585in}{1.914646in}}%
\pgfpathlineto{\pgfqpoint{4.349576in}{1.924479in}}%
\pgfpathlineto{\pgfqpoint{4.357560in}{1.934238in}}%
\pgfpathlineto{\pgfqpoint{4.365539in}{1.943924in}}%
\pgfpathlineto{\pgfqpoint{4.351458in}{1.942715in}}%
\pgfpathlineto{\pgfqpoint{4.337387in}{1.941579in}}%
\pgfpathlineto{\pgfqpoint{4.323325in}{1.940516in}}%
\pgfpathlineto{\pgfqpoint{4.309273in}{1.939526in}}%
\pgfpathlineto{\pgfqpoint{4.301285in}{1.929893in}}%
\pgfpathlineto{\pgfqpoint{4.293291in}{1.920192in}}%
\pgfpathlineto{\pgfqpoint{4.285292in}{1.910422in}}%
\pgfpathlineto{\pgfqpoint{4.277286in}{1.900585in}}%
\pgfpathclose%
\pgfusepath{fill}%
\end{pgfscope}%
\begin{pgfscope}%
\pgfpathrectangle{\pgfqpoint{1.150000in}{0.150000in}}{\pgfqpoint{5.700000in}{5.700000in}}%
\pgfusepath{clip}%
\pgfsetbuttcap%
\pgfsetroundjoin%
\definecolor{currentfill}{rgb}{0.272594,0.025563,0.353093}%
\pgfsetfillcolor{currentfill}%
\pgfsetfillopacity{0.700000}%
\pgfsetlinewidth{0.000000pt}%
\definecolor{currentstroke}{rgb}{0.000000,0.000000,0.000000}%
\pgfsetstrokecolor{currentstroke}%
\pgfsetdash{}{0pt}%
\pgfpathmoveto{\pgfqpoint{3.603308in}{1.644129in}}%
\pgfpathlineto{\pgfqpoint{3.617180in}{1.641654in}}%
\pgfpathlineto{\pgfqpoint{3.631060in}{1.639257in}}%
\pgfpathlineto{\pgfqpoint{3.644945in}{1.636938in}}%
\pgfpathlineto{\pgfqpoint{3.658838in}{1.634695in}}%
\pgfpathlineto{\pgfqpoint{3.667071in}{1.644372in}}%
\pgfpathlineto{\pgfqpoint{3.675297in}{1.654067in}}%
\pgfpathlineto{\pgfqpoint{3.683518in}{1.663777in}}%
\pgfpathlineto{\pgfqpoint{3.691733in}{1.673499in}}%
\pgfpathlineto{\pgfqpoint{3.677852in}{1.675537in}}%
\pgfpathlineto{\pgfqpoint{3.663979in}{1.677651in}}%
\pgfpathlineto{\pgfqpoint{3.650112in}{1.679843in}}%
\pgfpathlineto{\pgfqpoint{3.636252in}{1.682113in}}%
\pgfpathlineto{\pgfqpoint{3.628025in}{1.672588in}}%
\pgfpathlineto{\pgfqpoint{3.619792in}{1.663081in}}%
\pgfpathlineto{\pgfqpoint{3.611553in}{1.653593in}}%
\pgfpathlineto{\pgfqpoint{3.603308in}{1.644129in}}%
\pgfpathclose%
\pgfusepath{fill}%
\end{pgfscope}%
\begin{pgfscope}%
\pgfpathrectangle{\pgfqpoint{1.150000in}{0.150000in}}{\pgfqpoint{5.700000in}{5.700000in}}%
\pgfusepath{clip}%
\pgfsetbuttcap%
\pgfsetroundjoin%
\definecolor{currentfill}{rgb}{0.282290,0.145912,0.461510}%
\pgfsetfillcolor{currentfill}%
\pgfsetfillopacity{0.700000}%
\pgfsetlinewidth{0.000000pt}%
\definecolor{currentstroke}{rgb}{0.000000,0.000000,0.000000}%
\pgfsetstrokecolor{currentstroke}%
\pgfsetdash{}{0pt}%
\pgfpathmoveto{\pgfqpoint{4.189022in}{1.857910in}}%
\pgfpathlineto{\pgfqpoint{4.203054in}{1.858465in}}%
\pgfpathlineto{\pgfqpoint{4.217096in}{1.859094in}}%
\pgfpathlineto{\pgfqpoint{4.231147in}{1.859796in}}%
\pgfpathlineto{\pgfqpoint{4.245208in}{1.860571in}}%
\pgfpathlineto{\pgfqpoint{4.253236in}{1.870672in}}%
\pgfpathlineto{\pgfqpoint{4.261258in}{1.880709in}}%
\pgfpathlineto{\pgfqpoint{4.269275in}{1.890680in}}%
\pgfpathlineto{\pgfqpoint{4.277286in}{1.900585in}}%
\pgfpathlineto{\pgfqpoint{4.263235in}{1.899728in}}%
\pgfpathlineto{\pgfqpoint{4.249192in}{1.898945in}}%
\pgfpathlineto{\pgfqpoint{4.235159in}{1.898235in}}%
\pgfpathlineto{\pgfqpoint{4.221136in}{1.897598in}}%
\pgfpathlineto{\pgfqpoint{4.213116in}{1.887767in}}%
\pgfpathlineto{\pgfqpoint{4.205090in}{1.877875in}}%
\pgfpathlineto{\pgfqpoint{4.197059in}{1.867922in}}%
\pgfpathlineto{\pgfqpoint{4.189022in}{1.857910in}}%
\pgfpathclose%
\pgfusepath{fill}%
\end{pgfscope}%
\begin{pgfscope}%
\pgfpathrectangle{\pgfqpoint{1.150000in}{0.150000in}}{\pgfqpoint{5.700000in}{5.700000in}}%
\pgfusepath{clip}%
\pgfsetbuttcap%
\pgfsetroundjoin%
\definecolor{currentfill}{rgb}{0.201239,0.383670,0.554294}%
\pgfsetfillcolor{currentfill}%
\pgfsetfillopacity{0.700000}%
\pgfsetlinewidth{0.000000pt}%
\definecolor{currentstroke}{rgb}{0.000000,0.000000,0.000000}%
\pgfsetstrokecolor{currentstroke}%
\pgfsetdash{}{0pt}%
\pgfpathmoveto{\pgfqpoint{5.392757in}{2.412092in}}%
\pgfpathlineto{\pgfqpoint{5.407258in}{2.416040in}}%
\pgfpathlineto{\pgfqpoint{5.421772in}{2.420058in}}%
\pgfpathlineto{\pgfqpoint{5.436298in}{2.424146in}}%
\pgfpathlineto{\pgfqpoint{5.450837in}{2.428304in}}%
\pgfpathlineto{\pgfqpoint{5.458321in}{2.433151in}}%
\pgfpathlineto{\pgfqpoint{5.465797in}{2.437927in}}%
\pgfpathlineto{\pgfqpoint{5.473264in}{2.442637in}}%
\pgfpathlineto{\pgfqpoint{5.480724in}{2.447284in}}%
\pgfpathlineto{\pgfqpoint{5.466204in}{2.443324in}}%
\pgfpathlineto{\pgfqpoint{5.451697in}{2.439434in}}%
\pgfpathlineto{\pgfqpoint{5.437202in}{2.435614in}}%
\pgfpathlineto{\pgfqpoint{5.422721in}{2.431864in}}%
\pgfpathlineto{\pgfqpoint{5.415242in}{2.427012in}}%
\pgfpathlineto{\pgfqpoint{5.407755in}{2.422101in}}%
\pgfpathlineto{\pgfqpoint{5.400260in}{2.417129in}}%
\pgfpathlineto{\pgfqpoint{5.392757in}{2.412092in}}%
\pgfpathclose%
\pgfusepath{fill}%
\end{pgfscope}%
\begin{pgfscope}%
\pgfpathrectangle{\pgfqpoint{1.150000in}{0.150000in}}{\pgfqpoint{5.700000in}{5.700000in}}%
\pgfusepath{clip}%
\pgfsetbuttcap%
\pgfsetroundjoin%
\definecolor{currentfill}{rgb}{0.280255,0.165693,0.476498}%
\pgfsetfillcolor{currentfill}%
\pgfsetfillopacity{0.700000}%
\pgfsetlinewidth{0.000000pt}%
\definecolor{currentstroke}{rgb}{0.000000,0.000000,0.000000}%
\pgfsetstrokecolor{currentstroke}%
\pgfsetdash{}{0pt}%
\pgfpathmoveto{\pgfqpoint{2.335932in}{1.950767in}}%
\pgfpathlineto{\pgfqpoint{2.349781in}{1.939306in}}%
\pgfpathlineto{\pgfqpoint{2.363628in}{1.927957in}}%
\pgfpathlineto{\pgfqpoint{2.377474in}{1.916720in}}%
\pgfpathlineto{\pgfqpoint{2.391319in}{1.905595in}}%
\pgfpathlineto{\pgfqpoint{2.400300in}{1.905806in}}%
\pgfpathlineto{\pgfqpoint{2.409264in}{1.906269in}}%
\pgfpathlineto{\pgfqpoint{2.418211in}{1.906980in}}%
\pgfpathlineto{\pgfqpoint{2.427140in}{1.907931in}}%
\pgfpathlineto{\pgfqpoint{2.413332in}{1.918677in}}%
\pgfpathlineto{\pgfqpoint{2.399524in}{1.929534in}}%
\pgfpathlineto{\pgfqpoint{2.385714in}{1.940502in}}%
\pgfpathlineto{\pgfqpoint{2.371903in}{1.951583in}}%
\pgfpathlineto{\pgfqpoint{2.362937in}{1.951003in}}%
\pgfpathlineto{\pgfqpoint{2.353954in}{1.950670in}}%
\pgfpathlineto{\pgfqpoint{2.344952in}{1.950589in}}%
\pgfpathlineto{\pgfqpoint{2.335932in}{1.950767in}}%
\pgfpathclose%
\pgfusepath{fill}%
\end{pgfscope}%
\begin{pgfscope}%
\pgfpathrectangle{\pgfqpoint{1.150000in}{0.150000in}}{\pgfqpoint{5.700000in}{5.700000in}}%
\pgfusepath{clip}%
\pgfsetbuttcap%
\pgfsetroundjoin%
\definecolor{currentfill}{rgb}{0.283187,0.125848,0.444960}%
\pgfsetfillcolor{currentfill}%
\pgfsetfillopacity{0.700000}%
\pgfsetlinewidth{0.000000pt}%
\definecolor{currentstroke}{rgb}{0.000000,0.000000,0.000000}%
\pgfsetstrokecolor{currentstroke}%
\pgfsetdash{}{0pt}%
\pgfpathmoveto{\pgfqpoint{4.100742in}{1.816210in}}%
\pgfpathlineto{\pgfqpoint{4.114747in}{1.816369in}}%
\pgfpathlineto{\pgfqpoint{4.128762in}{1.816602in}}%
\pgfpathlineto{\pgfqpoint{4.142785in}{1.816908in}}%
\pgfpathlineto{\pgfqpoint{4.156817in}{1.817287in}}%
\pgfpathlineto{\pgfqpoint{4.164877in}{1.827527in}}%
\pgfpathlineto{\pgfqpoint{4.172931in}{1.837711in}}%
\pgfpathlineto{\pgfqpoint{4.180979in}{1.847839in}}%
\pgfpathlineto{\pgfqpoint{4.189022in}{1.857910in}}%
\pgfpathlineto{\pgfqpoint{4.174998in}{1.857428in}}%
\pgfpathlineto{\pgfqpoint{4.160984in}{1.857020in}}%
\pgfpathlineto{\pgfqpoint{4.146978in}{1.856685in}}%
\pgfpathlineto{\pgfqpoint{4.132982in}{1.856424in}}%
\pgfpathlineto{\pgfqpoint{4.124930in}{1.846448in}}%
\pgfpathlineto{\pgfqpoint{4.116873in}{1.836419in}}%
\pgfpathlineto{\pgfqpoint{4.108810in}{1.826340in}}%
\pgfpathlineto{\pgfqpoint{4.100742in}{1.816210in}}%
\pgfpathclose%
\pgfusepath{fill}%
\end{pgfscope}%
\begin{pgfscope}%
\pgfpathrectangle{\pgfqpoint{1.150000in}{0.150000in}}{\pgfqpoint{5.700000in}{5.700000in}}%
\pgfusepath{clip}%
\pgfsetbuttcap%
\pgfsetroundjoin%
\definecolor{currentfill}{rgb}{0.267004,0.004874,0.329415}%
\pgfsetfillcolor{currentfill}%
\pgfsetfillopacity{0.700000}%
\pgfsetlinewidth{0.000000pt}%
\definecolor{currentstroke}{rgb}{0.000000,0.000000,0.000000}%
\pgfsetstrokecolor{currentstroke}%
\pgfsetdash{}{0pt}%
\pgfpathmoveto{\pgfqpoint{3.137788in}{1.606414in}}%
\pgfpathlineto{\pgfqpoint{3.151596in}{1.600975in}}%
\pgfpathlineto{\pgfqpoint{3.165409in}{1.595620in}}%
\pgfpathlineto{\pgfqpoint{3.179226in}{1.590349in}}%
\pgfpathlineto{\pgfqpoint{3.193047in}{1.585161in}}%
\pgfpathlineto{\pgfqpoint{3.201489in}{1.592322in}}%
\pgfpathlineto{\pgfqpoint{3.209922in}{1.599589in}}%
\pgfpathlineto{\pgfqpoint{3.218346in}{1.606958in}}%
\pgfpathlineto{\pgfqpoint{3.226762in}{1.614426in}}%
\pgfpathlineto{\pgfqpoint{3.212960in}{1.619326in}}%
\pgfpathlineto{\pgfqpoint{3.199163in}{1.624309in}}%
\pgfpathlineto{\pgfqpoint{3.185370in}{1.629377in}}%
\pgfpathlineto{\pgfqpoint{3.171582in}{1.634529in}}%
\pgfpathlineto{\pgfqpoint{3.163146in}{1.627341in}}%
\pgfpathlineto{\pgfqpoint{3.154703in}{1.620256in}}%
\pgfpathlineto{\pgfqpoint{3.146250in}{1.613279in}}%
\pgfpathlineto{\pgfqpoint{3.137788in}{1.606414in}}%
\pgfpathclose%
\pgfusepath{fill}%
\end{pgfscope}%
\begin{pgfscope}%
\pgfpathrectangle{\pgfqpoint{1.150000in}{0.150000in}}{\pgfqpoint{5.700000in}{5.700000in}}%
\pgfusepath{clip}%
\pgfsetbuttcap%
\pgfsetroundjoin%
\definecolor{currentfill}{rgb}{0.282910,0.105393,0.426902}%
\pgfsetfillcolor{currentfill}%
\pgfsetfillopacity{0.700000}%
\pgfsetlinewidth{0.000000pt}%
\definecolor{currentstroke}{rgb}{0.000000,0.000000,0.000000}%
\pgfsetstrokecolor{currentstroke}%
\pgfsetdash{}{0pt}%
\pgfpathmoveto{\pgfqpoint{4.012441in}{1.775818in}}%
\pgfpathlineto{\pgfqpoint{4.026421in}{1.775557in}}%
\pgfpathlineto{\pgfqpoint{4.040410in}{1.775371in}}%
\pgfpathlineto{\pgfqpoint{4.054407in}{1.775259in}}%
\pgfpathlineto{\pgfqpoint{4.068413in}{1.775221in}}%
\pgfpathlineto{\pgfqpoint{4.076503in}{1.785536in}}%
\pgfpathlineto{\pgfqpoint{4.084588in}{1.795807in}}%
\pgfpathlineto{\pgfqpoint{4.092668in}{1.806032in}}%
\pgfpathlineto{\pgfqpoint{4.100742in}{1.816210in}}%
\pgfpathlineto{\pgfqpoint{4.086745in}{1.816125in}}%
\pgfpathlineto{\pgfqpoint{4.072757in}{1.816115in}}%
\pgfpathlineto{\pgfqpoint{4.058778in}{1.816178in}}%
\pgfpathlineto{\pgfqpoint{4.044807in}{1.816315in}}%
\pgfpathlineto{\pgfqpoint{4.036724in}{1.806252in}}%
\pgfpathlineto{\pgfqpoint{4.028635in}{1.796148in}}%
\pgfpathlineto{\pgfqpoint{4.020541in}{1.786002in}}%
\pgfpathlineto{\pgfqpoint{4.012441in}{1.775818in}}%
\pgfpathclose%
\pgfusepath{fill}%
\end{pgfscope}%
\begin{pgfscope}%
\pgfpathrectangle{\pgfqpoint{1.150000in}{0.150000in}}{\pgfqpoint{5.700000in}{5.700000in}}%
\pgfusepath{clip}%
\pgfsetbuttcap%
\pgfsetroundjoin%
\definecolor{currentfill}{rgb}{0.269944,0.014625,0.341379}%
\pgfsetfillcolor{currentfill}%
\pgfsetfillopacity{0.700000}%
\pgfsetlinewidth{0.000000pt}%
\definecolor{currentstroke}{rgb}{0.000000,0.000000,0.000000}%
\pgfsetstrokecolor{currentstroke}%
\pgfsetdash{}{0pt}%
\pgfpathmoveto{\pgfqpoint{2.993345in}{1.629226in}}%
\pgfpathlineto{\pgfqpoint{3.007146in}{1.622789in}}%
\pgfpathlineto{\pgfqpoint{3.020950in}{1.616439in}}%
\pgfpathlineto{\pgfqpoint{3.034757in}{1.610177in}}%
\pgfpathlineto{\pgfqpoint{3.048568in}{1.604002in}}%
\pgfpathlineto{\pgfqpoint{3.057089in}{1.610059in}}%
\pgfpathlineto{\pgfqpoint{3.065601in}{1.616252in}}%
\pgfpathlineto{\pgfqpoint{3.074103in}{1.622574in}}%
\pgfpathlineto{\pgfqpoint{3.082595in}{1.629023in}}%
\pgfpathlineto{\pgfqpoint{3.068806in}{1.634891in}}%
\pgfpathlineto{\pgfqpoint{3.055021in}{1.640844in}}%
\pgfpathlineto{\pgfqpoint{3.041240in}{1.646886in}}%
\pgfpathlineto{\pgfqpoint{3.027462in}{1.653014in}}%
\pgfpathlineto{\pgfqpoint{3.018948in}{1.646866in}}%
\pgfpathlineto{\pgfqpoint{3.010424in}{1.640849in}}%
\pgfpathlineto{\pgfqpoint{3.001890in}{1.634967in}}%
\pgfpathlineto{\pgfqpoint{2.993345in}{1.629226in}}%
\pgfpathclose%
\pgfusepath{fill}%
\end{pgfscope}%
\begin{pgfscope}%
\pgfpathrectangle{\pgfqpoint{1.150000in}{0.150000in}}{\pgfqpoint{5.700000in}{5.700000in}}%
\pgfusepath{clip}%
\pgfsetbuttcap%
\pgfsetroundjoin%
\definecolor{currentfill}{rgb}{0.206756,0.371758,0.553117}%
\pgfsetfillcolor{currentfill}%
\pgfsetfillopacity{0.700000}%
\pgfsetlinewidth{0.000000pt}%
\definecolor{currentstroke}{rgb}{0.000000,0.000000,0.000000}%
\pgfsetstrokecolor{currentstroke}%
\pgfsetdash{}{0pt}%
\pgfpathmoveto{\pgfqpoint{5.304717in}{2.375422in}}%
\pgfpathlineto{\pgfqpoint{5.319185in}{2.379267in}}%
\pgfpathlineto{\pgfqpoint{5.333665in}{2.383181in}}%
\pgfpathlineto{\pgfqpoint{5.348158in}{2.387166in}}%
\pgfpathlineto{\pgfqpoint{5.362664in}{2.391221in}}%
\pgfpathlineto{\pgfqpoint{5.370200in}{2.396553in}}%
\pgfpathlineto{\pgfqpoint{5.377727in}{2.401807in}}%
\pgfpathlineto{\pgfqpoint{5.385246in}{2.406986in}}%
\pgfpathlineto{\pgfqpoint{5.392757in}{2.412092in}}%
\pgfpathlineto{\pgfqpoint{5.378269in}{2.408214in}}%
\pgfpathlineto{\pgfqpoint{5.363794in}{2.404405in}}%
\pgfpathlineto{\pgfqpoint{5.349331in}{2.400667in}}%
\pgfpathlineto{\pgfqpoint{5.334881in}{2.396999in}}%
\pgfpathlineto{\pgfqpoint{5.327352in}{2.391708in}}%
\pgfpathlineto{\pgfqpoint{5.319815in}{2.386351in}}%
\pgfpathlineto{\pgfqpoint{5.312270in}{2.380923in}}%
\pgfpathlineto{\pgfqpoint{5.304717in}{2.375422in}}%
\pgfpathclose%
\pgfusepath{fill}%
\end{pgfscope}%
\begin{pgfscope}%
\pgfpathrectangle{\pgfqpoint{1.150000in}{0.150000in}}{\pgfqpoint{5.700000in}{5.700000in}}%
\pgfusepath{clip}%
\pgfsetbuttcap%
\pgfsetroundjoin%
\definecolor{currentfill}{rgb}{0.269944,0.014625,0.341379}%
\pgfsetfillcolor{currentfill}%
\pgfsetfillopacity{0.700000}%
\pgfsetlinewidth{0.000000pt}%
\definecolor{currentstroke}{rgb}{0.000000,0.000000,0.000000}%
\pgfsetstrokecolor{currentstroke}%
\pgfsetdash{}{0pt}%
\pgfpathmoveto{\pgfqpoint{3.514785in}{1.618142in}}%
\pgfpathlineto{\pgfqpoint{3.528646in}{1.615129in}}%
\pgfpathlineto{\pgfqpoint{3.542512in}{1.612195in}}%
\pgfpathlineto{\pgfqpoint{3.556385in}{1.609339in}}%
\pgfpathlineto{\pgfqpoint{3.570265in}{1.606562in}}%
\pgfpathlineto{\pgfqpoint{3.578535in}{1.615904in}}%
\pgfpathlineto{\pgfqpoint{3.586799in}{1.625281in}}%
\pgfpathlineto{\pgfqpoint{3.595056in}{1.634690in}}%
\pgfpathlineto{\pgfqpoint{3.603308in}{1.644129in}}%
\pgfpathlineto{\pgfqpoint{3.589442in}{1.646681in}}%
\pgfpathlineto{\pgfqpoint{3.575583in}{1.649312in}}%
\pgfpathlineto{\pgfqpoint{3.561729in}{1.652020in}}%
\pgfpathlineto{\pgfqpoint{3.547883in}{1.654808in}}%
\pgfpathlineto{\pgfqpoint{3.539618in}{1.645587in}}%
\pgfpathlineto{\pgfqpoint{3.531347in}{1.636400in}}%
\pgfpathlineto{\pgfqpoint{3.523069in}{1.627251in}}%
\pgfpathlineto{\pgfqpoint{3.514785in}{1.618142in}}%
\pgfpathclose%
\pgfusepath{fill}%
\end{pgfscope}%
\begin{pgfscope}%
\pgfpathrectangle{\pgfqpoint{1.150000in}{0.150000in}}{\pgfqpoint{5.700000in}{5.700000in}}%
\pgfusepath{clip}%
\pgfsetbuttcap%
\pgfsetroundjoin%
\definecolor{currentfill}{rgb}{0.281446,0.084320,0.407414}%
\pgfsetfillcolor{currentfill}%
\pgfsetfillopacity{0.700000}%
\pgfsetlinewidth{0.000000pt}%
\definecolor{currentstroke}{rgb}{0.000000,0.000000,0.000000}%
\pgfsetstrokecolor{currentstroke}%
\pgfsetdash{}{0pt}%
\pgfpathmoveto{\pgfqpoint{3.924112in}{1.737084in}}%
\pgfpathlineto{\pgfqpoint{3.938069in}{1.736382in}}%
\pgfpathlineto{\pgfqpoint{3.952033in}{1.735755in}}%
\pgfpathlineto{\pgfqpoint{3.966006in}{1.735202in}}%
\pgfpathlineto{\pgfqpoint{3.979987in}{1.734723in}}%
\pgfpathlineto{\pgfqpoint{3.988109in}{1.745047in}}%
\pgfpathlineto{\pgfqpoint{3.996225in}{1.755338in}}%
\pgfpathlineto{\pgfqpoint{4.004336in}{1.765596in}}%
\pgfpathlineto{\pgfqpoint{4.012441in}{1.775818in}}%
\pgfpathlineto{\pgfqpoint{3.998469in}{1.776152in}}%
\pgfpathlineto{\pgfqpoint{3.984506in}{1.776562in}}%
\pgfpathlineto{\pgfqpoint{3.970551in}{1.777046in}}%
\pgfpathlineto{\pgfqpoint{3.956604in}{1.777604in}}%
\pgfpathlineto{\pgfqpoint{3.948489in}{1.767518in}}%
\pgfpathlineto{\pgfqpoint{3.940369in}{1.757402in}}%
\pgfpathlineto{\pgfqpoint{3.932243in}{1.747257in}}%
\pgfpathlineto{\pgfqpoint{3.924112in}{1.737084in}}%
\pgfpathclose%
\pgfusepath{fill}%
\end{pgfscope}%
\begin{pgfscope}%
\pgfpathrectangle{\pgfqpoint{1.150000in}{0.150000in}}{\pgfqpoint{5.700000in}{5.700000in}}%
\pgfusepath{clip}%
\pgfsetbuttcap%
\pgfsetroundjoin%
\definecolor{currentfill}{rgb}{0.267004,0.004874,0.329415}%
\pgfsetfillcolor{currentfill}%
\pgfsetfillopacity{0.700000}%
\pgfsetlinewidth{0.000000pt}%
\definecolor{currentstroke}{rgb}{0.000000,0.000000,0.000000}%
\pgfsetstrokecolor{currentstroke}%
\pgfsetdash{}{0pt}%
\pgfpathmoveto{\pgfqpoint{3.282016in}{1.595656in}}%
\pgfpathlineto{\pgfqpoint{3.295842in}{1.591169in}}%
\pgfpathlineto{\pgfqpoint{3.309672in}{1.586764in}}%
\pgfpathlineto{\pgfqpoint{3.323508in}{1.582441in}}%
\pgfpathlineto{\pgfqpoint{3.337349in}{1.578198in}}%
\pgfpathlineto{\pgfqpoint{3.345720in}{1.586308in}}%
\pgfpathlineto{\pgfqpoint{3.354085in}{1.594498in}}%
\pgfpathlineto{\pgfqpoint{3.362441in}{1.602763in}}%
\pgfpathlineto{\pgfqpoint{3.370791in}{1.611101in}}%
\pgfpathlineto{\pgfqpoint{3.356967in}{1.615077in}}%
\pgfpathlineto{\pgfqpoint{3.343148in}{1.619134in}}%
\pgfpathlineto{\pgfqpoint{3.329335in}{1.623272in}}%
\pgfpathlineto{\pgfqpoint{3.315527in}{1.627492in}}%
\pgfpathlineto{\pgfqpoint{3.307161in}{1.619414in}}%
\pgfpathlineto{\pgfqpoint{3.298787in}{1.611412in}}%
\pgfpathlineto{\pgfqpoint{3.290405in}{1.603491in}}%
\pgfpathlineto{\pgfqpoint{3.282016in}{1.595656in}}%
\pgfpathclose%
\pgfusepath{fill}%
\end{pgfscope}%
\begin{pgfscope}%
\pgfpathrectangle{\pgfqpoint{1.150000in}{0.150000in}}{\pgfqpoint{5.700000in}{5.700000in}}%
\pgfusepath{clip}%
\pgfsetbuttcap%
\pgfsetroundjoin%
\definecolor{currentfill}{rgb}{0.214298,0.355619,0.551184}%
\pgfsetfillcolor{currentfill}%
\pgfsetfillopacity{0.700000}%
\pgfsetlinewidth{0.000000pt}%
\definecolor{currentstroke}{rgb}{0.000000,0.000000,0.000000}%
\pgfsetstrokecolor{currentstroke}%
\pgfsetdash{}{0pt}%
\pgfpathmoveto{\pgfqpoint{5.216610in}{2.337325in}}%
\pgfpathlineto{\pgfqpoint{5.231044in}{2.341043in}}%
\pgfpathlineto{\pgfqpoint{5.245491in}{2.344831in}}%
\pgfpathlineto{\pgfqpoint{5.259951in}{2.348690in}}%
\pgfpathlineto{\pgfqpoint{5.274422in}{2.352619in}}%
\pgfpathlineto{\pgfqpoint{5.282008in}{2.358446in}}%
\pgfpathlineto{\pgfqpoint{5.289586in}{2.364186in}}%
\pgfpathlineto{\pgfqpoint{5.297155in}{2.369844in}}%
\pgfpathlineto{\pgfqpoint{5.304717in}{2.375422in}}%
\pgfpathlineto{\pgfqpoint{5.290261in}{2.371648in}}%
\pgfpathlineto{\pgfqpoint{5.275819in}{2.367944in}}%
\pgfpathlineto{\pgfqpoint{5.261388in}{2.364310in}}%
\pgfpathlineto{\pgfqpoint{5.246970in}{2.360746in}}%
\pgfpathlineto{\pgfqpoint{5.239392in}{2.355005in}}%
\pgfpathlineto{\pgfqpoint{5.231806in}{2.349190in}}%
\pgfpathlineto{\pgfqpoint{5.224212in}{2.343298in}}%
\pgfpathlineto{\pgfqpoint{5.216610in}{2.337325in}}%
\pgfpathclose%
\pgfusepath{fill}%
\end{pgfscope}%
\begin{pgfscope}%
\pgfpathrectangle{\pgfqpoint{1.150000in}{0.150000in}}{\pgfqpoint{5.700000in}{5.700000in}}%
\pgfusepath{clip}%
\pgfsetbuttcap%
\pgfsetroundjoin%
\definecolor{currentfill}{rgb}{0.281887,0.150881,0.465405}%
\pgfsetfillcolor{currentfill}%
\pgfsetfillopacity{0.700000}%
\pgfsetlinewidth{0.000000pt}%
\definecolor{currentstroke}{rgb}{0.000000,0.000000,0.000000}%
\pgfsetstrokecolor{currentstroke}%
\pgfsetdash{}{0pt}%
\pgfpathmoveto{\pgfqpoint{2.391319in}{1.905595in}}%
\pgfpathlineto{\pgfqpoint{2.405162in}{1.894580in}}%
\pgfpathlineto{\pgfqpoint{2.419006in}{1.883674in}}%
\pgfpathlineto{\pgfqpoint{2.432848in}{1.872878in}}%
\pgfpathlineto{\pgfqpoint{2.446690in}{1.862189in}}%
\pgfpathlineto{\pgfqpoint{2.455634in}{1.862787in}}%
\pgfpathlineto{\pgfqpoint{2.464562in}{1.863632in}}%
\pgfpathlineto{\pgfqpoint{2.473472in}{1.864718in}}%
\pgfpathlineto{\pgfqpoint{2.482367in}{1.866040in}}%
\pgfpathlineto{\pgfqpoint{2.468561in}{1.876350in}}%
\pgfpathlineto{\pgfqpoint{2.454754in}{1.886768in}}%
\pgfpathlineto{\pgfqpoint{2.440948in}{1.897295in}}%
\pgfpathlineto{\pgfqpoint{2.427140in}{1.907931in}}%
\pgfpathlineto{\pgfqpoint{2.418211in}{1.906980in}}%
\pgfpathlineto{\pgfqpoint{2.409264in}{1.906269in}}%
\pgfpathlineto{\pgfqpoint{2.400300in}{1.905806in}}%
\pgfpathlineto{\pgfqpoint{2.391319in}{1.905595in}}%
\pgfpathclose%
\pgfusepath{fill}%
\end{pgfscope}%
\begin{pgfscope}%
\pgfpathrectangle{\pgfqpoint{1.150000in}{0.150000in}}{\pgfqpoint{5.700000in}{5.700000in}}%
\pgfusepath{clip}%
\pgfsetbuttcap%
\pgfsetroundjoin%
\definecolor{currentfill}{rgb}{0.280894,0.078907,0.402329}%
\pgfsetfillcolor{currentfill}%
\pgfsetfillopacity{0.700000}%
\pgfsetlinewidth{0.000000pt}%
\definecolor{currentstroke}{rgb}{0.000000,0.000000,0.000000}%
\pgfsetstrokecolor{currentstroke}%
\pgfsetdash{}{0pt}%
\pgfpathmoveto{\pgfqpoint{2.648052in}{1.750448in}}%
\pgfpathlineto{\pgfqpoint{2.661863in}{1.741473in}}%
\pgfpathlineto{\pgfqpoint{2.675675in}{1.732595in}}%
\pgfpathlineto{\pgfqpoint{2.689489in}{1.723815in}}%
\pgfpathlineto{\pgfqpoint{2.703304in}{1.715132in}}%
\pgfpathlineto{\pgfqpoint{2.712052in}{1.718134in}}%
\pgfpathlineto{\pgfqpoint{2.720787in}{1.721338in}}%
\pgfpathlineto{\pgfqpoint{2.729508in}{1.724738in}}%
\pgfpathlineto{\pgfqpoint{2.738216in}{1.728330in}}%
\pgfpathlineto{\pgfqpoint{2.724431in}{1.736660in}}%
\pgfpathlineto{\pgfqpoint{2.710647in}{1.745087in}}%
\pgfpathlineto{\pgfqpoint{2.696865in}{1.753612in}}%
\pgfpathlineto{\pgfqpoint{2.683085in}{1.762234in}}%
\pgfpathlineto{\pgfqpoint{2.674347in}{1.758987in}}%
\pgfpathlineto{\pgfqpoint{2.665596in}{1.755936in}}%
\pgfpathlineto{\pgfqpoint{2.656831in}{1.753088in}}%
\pgfpathlineto{\pgfqpoint{2.648052in}{1.750448in}}%
\pgfpathclose%
\pgfusepath{fill}%
\end{pgfscope}%
\begin{pgfscope}%
\pgfpathrectangle{\pgfqpoint{1.150000in}{0.150000in}}{\pgfqpoint{5.700000in}{5.700000in}}%
\pgfusepath{clip}%
\pgfsetbuttcap%
\pgfsetroundjoin%
\definecolor{currentfill}{rgb}{0.273809,0.031497,0.358853}%
\pgfsetfillcolor{currentfill}%
\pgfsetfillopacity{0.700000}%
\pgfsetlinewidth{0.000000pt}%
\definecolor{currentstroke}{rgb}{0.000000,0.000000,0.000000}%
\pgfsetstrokecolor{currentstroke}%
\pgfsetdash{}{0pt}%
\pgfpathmoveto{\pgfqpoint{2.848563in}{1.665104in}}%
\pgfpathlineto{\pgfqpoint{2.862366in}{1.657620in}}%
\pgfpathlineto{\pgfqpoint{2.876171in}{1.650227in}}%
\pgfpathlineto{\pgfqpoint{2.889979in}{1.642925in}}%
\pgfpathlineto{\pgfqpoint{2.903790in}{1.635713in}}%
\pgfpathlineto{\pgfqpoint{2.912403in}{1.640507in}}%
\pgfpathlineto{\pgfqpoint{2.921005in}{1.645466in}}%
\pgfpathlineto{\pgfqpoint{2.929595in}{1.650585in}}%
\pgfpathlineto{\pgfqpoint{2.938174in}{1.655859in}}%
\pgfpathlineto{\pgfqpoint{2.924389in}{1.662741in}}%
\pgfpathlineto{\pgfqpoint{2.910607in}{1.669713in}}%
\pgfpathlineto{\pgfqpoint{2.896827in}{1.676776in}}%
\pgfpathlineto{\pgfqpoint{2.883050in}{1.683931in}}%
\pgfpathlineto{\pgfqpoint{2.874446in}{1.678978in}}%
\pgfpathlineto{\pgfqpoint{2.865830in}{1.674186in}}%
\pgfpathlineto{\pgfqpoint{2.857202in}{1.669560in}}%
\pgfpathlineto{\pgfqpoint{2.848563in}{1.665104in}}%
\pgfpathclose%
\pgfusepath{fill}%
\end{pgfscope}%
\begin{pgfscope}%
\pgfpathrectangle{\pgfqpoint{1.150000in}{0.150000in}}{\pgfqpoint{5.700000in}{5.700000in}}%
\pgfusepath{clip}%
\pgfsetbuttcap%
\pgfsetroundjoin%
\definecolor{currentfill}{rgb}{0.279566,0.067836,0.391917}%
\pgfsetfillcolor{currentfill}%
\pgfsetfillopacity{0.700000}%
\pgfsetlinewidth{0.000000pt}%
\definecolor{currentstroke}{rgb}{0.000000,0.000000,0.000000}%
\pgfsetstrokecolor{currentstroke}%
\pgfsetdash{}{0pt}%
\pgfpathmoveto{\pgfqpoint{3.835744in}{1.700385in}}%
\pgfpathlineto{\pgfqpoint{3.849680in}{1.699218in}}%
\pgfpathlineto{\pgfqpoint{3.863622in}{1.698127in}}%
\pgfpathlineto{\pgfqpoint{3.877573in}{1.697110in}}%
\pgfpathlineto{\pgfqpoint{3.891532in}{1.696169in}}%
\pgfpathlineto{\pgfqpoint{3.899685in}{1.706427in}}%
\pgfpathlineto{\pgfqpoint{3.907833in}{1.716668in}}%
\pgfpathlineto{\pgfqpoint{3.915975in}{1.726888in}}%
\pgfpathlineto{\pgfqpoint{3.924112in}{1.737084in}}%
\pgfpathlineto{\pgfqpoint{3.910164in}{1.737862in}}%
\pgfpathlineto{\pgfqpoint{3.896223in}{1.738714in}}%
\pgfpathlineto{\pgfqpoint{3.882290in}{1.739642in}}%
\pgfpathlineto{\pgfqpoint{3.868365in}{1.740644in}}%
\pgfpathlineto{\pgfqpoint{3.860219in}{1.730604in}}%
\pgfpathlineto{\pgfqpoint{3.852066in}{1.720546in}}%
\pgfpathlineto{\pgfqpoint{3.843908in}{1.710472in}}%
\pgfpathlineto{\pgfqpoint{3.835744in}{1.700385in}}%
\pgfpathclose%
\pgfusepath{fill}%
\end{pgfscope}%
\begin{pgfscope}%
\pgfpathrectangle{\pgfqpoint{1.150000in}{0.150000in}}{\pgfqpoint{5.700000in}{5.700000in}}%
\pgfusepath{clip}%
\pgfsetbuttcap%
\pgfsetroundjoin%
\definecolor{currentfill}{rgb}{0.220057,0.343307,0.549413}%
\pgfsetfillcolor{currentfill}%
\pgfsetfillopacity{0.700000}%
\pgfsetlinewidth{0.000000pt}%
\definecolor{currentstroke}{rgb}{0.000000,0.000000,0.000000}%
\pgfsetstrokecolor{currentstroke}%
\pgfsetdash{}{0pt}%
\pgfpathmoveto{\pgfqpoint{5.128445in}{2.297872in}}%
\pgfpathlineto{\pgfqpoint{5.142846in}{2.301441in}}%
\pgfpathlineto{\pgfqpoint{5.157259in}{2.305081in}}%
\pgfpathlineto{\pgfqpoint{5.171684in}{2.308792in}}%
\pgfpathlineto{\pgfqpoint{5.186121in}{2.312573in}}%
\pgfpathlineto{\pgfqpoint{5.193755in}{2.318895in}}%
\pgfpathlineto{\pgfqpoint{5.201382in}{2.325126in}}%
\pgfpathlineto{\pgfqpoint{5.209000in}{2.331268in}}%
\pgfpathlineto{\pgfqpoint{5.216610in}{2.337325in}}%
\pgfpathlineto{\pgfqpoint{5.202188in}{2.333677in}}%
\pgfpathlineto{\pgfqpoint{5.187778in}{2.330099in}}%
\pgfpathlineto{\pgfqpoint{5.173380in}{2.326592in}}%
\pgfpathlineto{\pgfqpoint{5.158995in}{2.323155in}}%
\pgfpathlineto{\pgfqpoint{5.151369in}{2.316958in}}%
\pgfpathlineto{\pgfqpoint{5.143736in}{2.310680in}}%
\pgfpathlineto{\pgfqpoint{5.136094in}{2.304319in}}%
\pgfpathlineto{\pgfqpoint{5.128445in}{2.297872in}}%
\pgfpathclose%
\pgfusepath{fill}%
\end{pgfscope}%
\begin{pgfscope}%
\pgfpathrectangle{\pgfqpoint{1.150000in}{0.150000in}}{\pgfqpoint{5.700000in}{5.700000in}}%
\pgfusepath{clip}%
\pgfsetbuttcap%
\pgfsetroundjoin%
\definecolor{currentfill}{rgb}{0.223925,0.334994,0.548053}%
\pgfsetfillcolor{currentfill}%
\pgfsetfillopacity{0.700000}%
\pgfsetlinewidth{0.000000pt}%
\definecolor{currentstroke}{rgb}{0.000000,0.000000,0.000000}%
\pgfsetstrokecolor{currentstroke}%
\pgfsetdash{}{0pt}%
\pgfpathmoveto{\pgfqpoint{1.909631in}{2.336308in}}%
\pgfpathlineto{\pgfqpoint{1.923617in}{2.320862in}}%
\pgfpathlineto{\pgfqpoint{1.937598in}{2.305561in}}%
\pgfpathlineto{\pgfqpoint{1.951573in}{2.290402in}}%
\pgfpathlineto{\pgfqpoint{1.965542in}{2.275384in}}%
\pgfpathlineto{\pgfqpoint{1.974907in}{2.271607in}}%
\pgfpathlineto{\pgfqpoint{1.984249in}{2.268150in}}%
\pgfpathlineto{\pgfqpoint{1.993566in}{2.265004in}}%
\pgfpathlineto{\pgfqpoint{2.002861in}{2.262164in}}%
\pgfpathlineto{\pgfqpoint{1.988939in}{2.276768in}}%
\pgfpathlineto{\pgfqpoint{1.975012in}{2.291512in}}%
\pgfpathlineto{\pgfqpoint{1.961079in}{2.306398in}}%
\pgfpathlineto{\pgfqpoint{1.947142in}{2.321427in}}%
\pgfpathlineto{\pgfqpoint{1.937800in}{2.324673in}}%
\pgfpathlineto{\pgfqpoint{1.928435in}{2.328231in}}%
\pgfpathlineto{\pgfqpoint{1.919045in}{2.332107in}}%
\pgfpathlineto{\pgfqpoint{1.909631in}{2.336308in}}%
\pgfpathclose%
\pgfusepath{fill}%
\end{pgfscope}%
\begin{pgfscope}%
\pgfpathrectangle{\pgfqpoint{1.150000in}{0.150000in}}{\pgfqpoint{5.700000in}{5.700000in}}%
\pgfusepath{clip}%
\pgfsetbuttcap%
\pgfsetroundjoin%
\definecolor{currentfill}{rgb}{0.233603,0.313828,0.543914}%
\pgfsetfillcolor{currentfill}%
\pgfsetfillopacity{0.700000}%
\pgfsetlinewidth{0.000000pt}%
\definecolor{currentstroke}{rgb}{0.000000,0.000000,0.000000}%
\pgfsetstrokecolor{currentstroke}%
\pgfsetdash{}{0pt}%
\pgfpathmoveto{\pgfqpoint{1.965542in}{2.275384in}}%
\pgfpathlineto{\pgfqpoint{1.979507in}{2.260505in}}%
\pgfpathlineto{\pgfqpoint{1.993467in}{2.245766in}}%
\pgfpathlineto{\pgfqpoint{2.007423in}{2.231163in}}%
\pgfpathlineto{\pgfqpoint{2.021374in}{2.216697in}}%
\pgfpathlineto{\pgfqpoint{2.030691in}{2.213343in}}%
\pgfpathlineto{\pgfqpoint{2.039985in}{2.210301in}}%
\pgfpathlineto{\pgfqpoint{2.049256in}{2.207566in}}%
\pgfpathlineto{\pgfqpoint{2.058505in}{2.205130in}}%
\pgfpathlineto{\pgfqpoint{2.044601in}{2.219184in}}%
\pgfpathlineto{\pgfqpoint{2.030692in}{2.233374in}}%
\pgfpathlineto{\pgfqpoint{2.016779in}{2.247700in}}%
\pgfpathlineto{\pgfqpoint{2.002861in}{2.262164in}}%
\pgfpathlineto{\pgfqpoint{1.993566in}{2.265004in}}%
\pgfpathlineto{\pgfqpoint{1.984249in}{2.268150in}}%
\pgfpathlineto{\pgfqpoint{1.974907in}{2.271607in}}%
\pgfpathlineto{\pgfqpoint{1.965542in}{2.275384in}}%
\pgfpathclose%
\pgfusepath{fill}%
\end{pgfscope}%
\begin{pgfscope}%
\pgfpathrectangle{\pgfqpoint{1.150000in}{0.150000in}}{\pgfqpoint{5.700000in}{5.700000in}}%
\pgfusepath{clip}%
\pgfsetbuttcap%
\pgfsetroundjoin%
\definecolor{currentfill}{rgb}{0.227802,0.326594,0.546532}%
\pgfsetfillcolor{currentfill}%
\pgfsetfillopacity{0.700000}%
\pgfsetlinewidth{0.000000pt}%
\definecolor{currentstroke}{rgb}{0.000000,0.000000,0.000000}%
\pgfsetstrokecolor{currentstroke}%
\pgfsetdash{}{0pt}%
\pgfpathmoveto{\pgfqpoint{5.040231in}{2.257159in}}%
\pgfpathlineto{\pgfqpoint{5.054598in}{2.260557in}}%
\pgfpathlineto{\pgfqpoint{5.068976in}{2.264026in}}%
\pgfpathlineto{\pgfqpoint{5.083366in}{2.267566in}}%
\pgfpathlineto{\pgfqpoint{5.097769in}{2.271176in}}%
\pgfpathlineto{\pgfqpoint{5.105450in}{2.277991in}}%
\pgfpathlineto{\pgfqpoint{5.113123in}{2.284710in}}%
\pgfpathlineto{\pgfqpoint{5.120788in}{2.291336in}}%
\pgfpathlineto{\pgfqpoint{5.128445in}{2.297872in}}%
\pgfpathlineto{\pgfqpoint{5.114057in}{2.294373in}}%
\pgfpathlineto{\pgfqpoint{5.099680in}{2.290945in}}%
\pgfpathlineto{\pgfqpoint{5.085316in}{2.287587in}}%
\pgfpathlineto{\pgfqpoint{5.070964in}{2.284299in}}%
\pgfpathlineto{\pgfqpoint{5.063292in}{2.277645in}}%
\pgfpathlineto{\pgfqpoint{5.055613in}{2.270905in}}%
\pgfpathlineto{\pgfqpoint{5.047926in}{2.264077in}}%
\pgfpathlineto{\pgfqpoint{5.040231in}{2.257159in}}%
\pgfpathclose%
\pgfusepath{fill}%
\end{pgfscope}%
\begin{pgfscope}%
\pgfpathrectangle{\pgfqpoint{1.150000in}{0.150000in}}{\pgfqpoint{5.700000in}{5.700000in}}%
\pgfusepath{clip}%
\pgfsetbuttcap%
\pgfsetroundjoin%
\definecolor{currentfill}{rgb}{0.277018,0.050344,0.375715}%
\pgfsetfillcolor{currentfill}%
\pgfsetfillopacity{0.700000}%
\pgfsetlinewidth{0.000000pt}%
\definecolor{currentstroke}{rgb}{0.000000,0.000000,0.000000}%
\pgfsetstrokecolor{currentstroke}%
\pgfsetdash{}{0pt}%
\pgfpathmoveto{\pgfqpoint{3.747325in}{1.666116in}}%
\pgfpathlineto{\pgfqpoint{3.761241in}{1.664462in}}%
\pgfpathlineto{\pgfqpoint{3.775164in}{1.662883in}}%
\pgfpathlineto{\pgfqpoint{3.789095in}{1.661380in}}%
\pgfpathlineto{\pgfqpoint{3.803034in}{1.659953in}}%
\pgfpathlineto{\pgfqpoint{3.811220in}{1.670069in}}%
\pgfpathlineto{\pgfqpoint{3.819400in}{1.680181in}}%
\pgfpathlineto{\pgfqpoint{3.827575in}{1.690288in}}%
\pgfpathlineto{\pgfqpoint{3.835744in}{1.700385in}}%
\pgfpathlineto{\pgfqpoint{3.821817in}{1.701628in}}%
\pgfpathlineto{\pgfqpoint{3.807897in}{1.702946in}}%
\pgfpathlineto{\pgfqpoint{3.793985in}{1.704340in}}%
\pgfpathlineto{\pgfqpoint{3.780080in}{1.705810in}}%
\pgfpathlineto{\pgfqpoint{3.771900in}{1.695890in}}%
\pgfpathlineto{\pgfqpoint{3.763714in}{1.685965in}}%
\pgfpathlineto{\pgfqpoint{3.755522in}{1.676040in}}%
\pgfpathlineto{\pgfqpoint{3.747325in}{1.666116in}}%
\pgfpathclose%
\pgfusepath{fill}%
\end{pgfscope}%
\begin{pgfscope}%
\pgfpathrectangle{\pgfqpoint{1.150000in}{0.150000in}}{\pgfqpoint{5.700000in}{5.700000in}}%
\pgfusepath{clip}%
\pgfsetbuttcap%
\pgfsetroundjoin%
\definecolor{currentfill}{rgb}{0.268510,0.009605,0.335427}%
\pgfsetfillcolor{currentfill}%
\pgfsetfillopacity{0.700000}%
\pgfsetlinewidth{0.000000pt}%
\definecolor{currentstroke}{rgb}{0.000000,0.000000,0.000000}%
\pgfsetstrokecolor{currentstroke}%
\pgfsetdash{}{0pt}%
\pgfpathmoveto{\pgfqpoint{3.426141in}{1.596002in}}%
\pgfpathlineto{\pgfqpoint{3.439993in}{1.592428in}}%
\pgfpathlineto{\pgfqpoint{3.453850in}{1.588932in}}%
\pgfpathlineto{\pgfqpoint{3.467713in}{1.585516in}}%
\pgfpathlineto{\pgfqpoint{3.481582in}{1.582179in}}%
\pgfpathlineto{\pgfqpoint{3.489893in}{1.591092in}}%
\pgfpathlineto{\pgfqpoint{3.498197in}{1.600059in}}%
\pgfpathlineto{\pgfqpoint{3.506494in}{1.609077in}}%
\pgfpathlineto{\pgfqpoint{3.514785in}{1.618142in}}%
\pgfpathlineto{\pgfqpoint{3.500931in}{1.621233in}}%
\pgfpathlineto{\pgfqpoint{3.487082in}{1.624403in}}%
\pgfpathlineto{\pgfqpoint{3.473240in}{1.627653in}}%
\pgfpathlineto{\pgfqpoint{3.459404in}{1.630981in}}%
\pgfpathlineto{\pgfqpoint{3.451098in}{1.622155in}}%
\pgfpathlineto{\pgfqpoint{3.442786in}{1.613380in}}%
\pgfpathlineto{\pgfqpoint{3.434467in}{1.604661in}}%
\pgfpathlineto{\pgfqpoint{3.426141in}{1.596002in}}%
\pgfpathclose%
\pgfusepath{fill}%
\end{pgfscope}%
\begin{pgfscope}%
\pgfpathrectangle{\pgfqpoint{1.150000in}{0.150000in}}{\pgfqpoint{5.700000in}{5.700000in}}%
\pgfusepath{clip}%
\pgfsetbuttcap%
\pgfsetroundjoin%
\definecolor{currentfill}{rgb}{0.235526,0.309527,0.542944}%
\pgfsetfillcolor{currentfill}%
\pgfsetfillopacity{0.700000}%
\pgfsetlinewidth{0.000000pt}%
\definecolor{currentstroke}{rgb}{0.000000,0.000000,0.000000}%
\pgfsetstrokecolor{currentstroke}%
\pgfsetdash{}{0pt}%
\pgfpathmoveto{\pgfqpoint{4.951976in}{2.215303in}}%
\pgfpathlineto{\pgfqpoint{4.966308in}{2.218507in}}%
\pgfpathlineto{\pgfqpoint{4.980652in}{2.221783in}}%
\pgfpathlineto{\pgfqpoint{4.995007in}{2.225129in}}%
\pgfpathlineto{\pgfqpoint{5.009375in}{2.228546in}}%
\pgfpathlineto{\pgfqpoint{5.017101in}{2.235844in}}%
\pgfpathlineto{\pgfqpoint{5.024819in}{2.243045in}}%
\pgfpathlineto{\pgfqpoint{5.032529in}{2.250149in}}%
\pgfpathlineto{\pgfqpoint{5.040231in}{2.257159in}}%
\pgfpathlineto{\pgfqpoint{5.025877in}{2.253831in}}%
\pgfpathlineto{\pgfqpoint{5.011534in}{2.250575in}}%
\pgfpathlineto{\pgfqpoint{4.997203in}{2.247389in}}%
\pgfpathlineto{\pgfqpoint{4.982884in}{2.244273in}}%
\pgfpathlineto{\pgfqpoint{4.975169in}{2.237166in}}%
\pgfpathlineto{\pgfqpoint{4.967445in}{2.229970in}}%
\pgfpathlineto{\pgfqpoint{4.959715in}{2.222683in}}%
\pgfpathlineto{\pgfqpoint{4.951976in}{2.215303in}}%
\pgfpathclose%
\pgfusepath{fill}%
\end{pgfscope}%
\begin{pgfscope}%
\pgfpathrectangle{\pgfqpoint{1.150000in}{0.150000in}}{\pgfqpoint{5.700000in}{5.700000in}}%
\pgfusepath{clip}%
\pgfsetbuttcap%
\pgfsetroundjoin%
\definecolor{currentfill}{rgb}{0.244972,0.287675,0.537260}%
\pgfsetfillcolor{currentfill}%
\pgfsetfillopacity{0.700000}%
\pgfsetlinewidth{0.000000pt}%
\definecolor{currentstroke}{rgb}{0.000000,0.000000,0.000000}%
\pgfsetstrokecolor{currentstroke}%
\pgfsetdash{}{0pt}%
\pgfpathmoveto{\pgfqpoint{2.021374in}{2.216697in}}%
\pgfpathlineto{\pgfqpoint{2.035320in}{2.202366in}}%
\pgfpathlineto{\pgfqpoint{2.049262in}{2.188168in}}%
\pgfpathlineto{\pgfqpoint{2.063200in}{2.174102in}}%
\pgfpathlineto{\pgfqpoint{2.077135in}{2.160168in}}%
\pgfpathlineto{\pgfqpoint{2.086405in}{2.157234in}}%
\pgfpathlineto{\pgfqpoint{2.095654in}{2.154606in}}%
\pgfpathlineto{\pgfqpoint{2.104880in}{2.152279in}}%
\pgfpathlineto{\pgfqpoint{2.114085in}{2.150245in}}%
\pgfpathlineto{\pgfqpoint{2.100195in}{2.163769in}}%
\pgfpathlineto{\pgfqpoint{2.086303in}{2.177424in}}%
\pgfpathlineto{\pgfqpoint{2.072406in}{2.191211in}}%
\pgfpathlineto{\pgfqpoint{2.058505in}{2.205130in}}%
\pgfpathlineto{\pgfqpoint{2.049256in}{2.207566in}}%
\pgfpathlineto{\pgfqpoint{2.039985in}{2.210301in}}%
\pgfpathlineto{\pgfqpoint{2.030691in}{2.213343in}}%
\pgfpathlineto{\pgfqpoint{2.021374in}{2.216697in}}%
\pgfpathclose%
\pgfusepath{fill}%
\end{pgfscope}%
\begin{pgfscope}%
\pgfpathrectangle{\pgfqpoint{1.150000in}{0.150000in}}{\pgfqpoint{5.700000in}{5.700000in}}%
\pgfusepath{clip}%
\pgfsetbuttcap%
\pgfsetroundjoin%
\definecolor{currentfill}{rgb}{0.185556,0.418570,0.556753}%
\pgfsetfillcolor{currentfill}%
\pgfsetfillopacity{0.700000}%
\pgfsetlinewidth{0.000000pt}%
\definecolor{currentstroke}{rgb}{0.000000,0.000000,0.000000}%
\pgfsetstrokecolor{currentstroke}%
\pgfsetdash{}{0pt}%
\pgfpathmoveto{\pgfqpoint{5.626942in}{2.497738in}}%
\pgfpathlineto{\pgfqpoint{5.641559in}{2.502104in}}%
\pgfpathlineto{\pgfqpoint{5.656189in}{2.506540in}}%
\pgfpathlineto{\pgfqpoint{5.670832in}{2.511045in}}%
\pgfpathlineto{\pgfqpoint{5.678192in}{2.514791in}}%
\pgfpathlineto{\pgfqpoint{5.685544in}{2.518483in}}%
\pgfpathlineto{\pgfqpoint{5.692888in}{2.522127in}}%
\pgfpathlineto{\pgfqpoint{5.700223in}{2.525725in}}%
\pgfpathlineto{\pgfqpoint{5.685603in}{2.521462in}}%
\pgfpathlineto{\pgfqpoint{5.670995in}{2.517269in}}%
\pgfpathlineto{\pgfqpoint{5.656401in}{2.513145in}}%
\pgfpathlineto{\pgfqpoint{5.649048in}{2.509359in}}%
\pgfpathlineto{\pgfqpoint{5.641687in}{2.505532in}}%
\pgfpathlineto{\pgfqpoint{5.634319in}{2.501660in}}%
\pgfpathlineto{\pgfqpoint{5.626942in}{2.497738in}}%
\pgfpathclose%
\pgfusepath{fill}%
\end{pgfscope}%
\begin{pgfscope}%
\pgfpathrectangle{\pgfqpoint{1.150000in}{0.150000in}}{\pgfqpoint{5.700000in}{5.700000in}}%
\pgfusepath{clip}%
\pgfsetbuttcap%
\pgfsetroundjoin%
\definecolor{currentfill}{rgb}{0.282884,0.135920,0.453427}%
\pgfsetfillcolor{currentfill}%
\pgfsetfillopacity{0.700000}%
\pgfsetlinewidth{0.000000pt}%
\definecolor{currentstroke}{rgb}{0.000000,0.000000,0.000000}%
\pgfsetstrokecolor{currentstroke}%
\pgfsetdash{}{0pt}%
\pgfpathmoveto{\pgfqpoint{2.446690in}{1.862189in}}%
\pgfpathlineto{\pgfqpoint{2.460531in}{1.851607in}}%
\pgfpathlineto{\pgfqpoint{2.474372in}{1.841132in}}%
\pgfpathlineto{\pgfqpoint{2.488213in}{1.830762in}}%
\pgfpathlineto{\pgfqpoint{2.502054in}{1.820498in}}%
\pgfpathlineto{\pgfqpoint{2.510962in}{1.821482in}}%
\pgfpathlineto{\pgfqpoint{2.519854in}{1.822707in}}%
\pgfpathlineto{\pgfqpoint{2.528730in}{1.824168in}}%
\pgfpathlineto{\pgfqpoint{2.537590in}{1.825860in}}%
\pgfpathlineto{\pgfqpoint{2.523784in}{1.835747in}}%
\pgfpathlineto{\pgfqpoint{2.509978in}{1.845739in}}%
\pgfpathlineto{\pgfqpoint{2.496172in}{1.855836in}}%
\pgfpathlineto{\pgfqpoint{2.482367in}{1.866040in}}%
\pgfpathlineto{\pgfqpoint{2.473472in}{1.864718in}}%
\pgfpathlineto{\pgfqpoint{2.464562in}{1.863632in}}%
\pgfpathlineto{\pgfqpoint{2.455634in}{1.862787in}}%
\pgfpathlineto{\pgfqpoint{2.446690in}{1.862189in}}%
\pgfpathclose%
\pgfusepath{fill}%
\end{pgfscope}%
\begin{pgfscope}%
\pgfpathrectangle{\pgfqpoint{1.150000in}{0.150000in}}{\pgfqpoint{5.700000in}{5.700000in}}%
\pgfusepath{clip}%
\pgfsetbuttcap%
\pgfsetroundjoin%
\definecolor{currentfill}{rgb}{0.241237,0.296485,0.539709}%
\pgfsetfillcolor{currentfill}%
\pgfsetfillopacity{0.700000}%
\pgfsetlinewidth{0.000000pt}%
\definecolor{currentstroke}{rgb}{0.000000,0.000000,0.000000}%
\pgfsetstrokecolor{currentstroke}%
\pgfsetdash{}{0pt}%
\pgfpathmoveto{\pgfqpoint{4.863688in}{2.172443in}}%
\pgfpathlineto{\pgfqpoint{4.877985in}{2.175431in}}%
\pgfpathlineto{\pgfqpoint{4.892294in}{2.178491in}}%
\pgfpathlineto{\pgfqpoint{4.906615in}{2.181621in}}%
\pgfpathlineto{\pgfqpoint{4.920947in}{2.184823in}}%
\pgfpathlineto{\pgfqpoint{4.928716in}{2.192590in}}%
\pgfpathlineto{\pgfqpoint{4.936477in}{2.200258in}}%
\pgfpathlineto{\pgfqpoint{4.944230in}{2.207829in}}%
\pgfpathlineto{\pgfqpoint{4.951976in}{2.215303in}}%
\pgfpathlineto{\pgfqpoint{4.937656in}{2.212170in}}%
\pgfpathlineto{\pgfqpoint{4.923347in}{2.209107in}}%
\pgfpathlineto{\pgfqpoint{4.909050in}{2.206115in}}%
\pgfpathlineto{\pgfqpoint{4.894765in}{2.203195in}}%
\pgfpathlineto{\pgfqpoint{4.887006in}{2.195645in}}%
\pgfpathlineto{\pgfqpoint{4.879241in}{2.188004in}}%
\pgfpathlineto{\pgfqpoint{4.871468in}{2.180271in}}%
\pgfpathlineto{\pgfqpoint{4.863688in}{2.172443in}}%
\pgfpathclose%
\pgfusepath{fill}%
\end{pgfscope}%
\begin{pgfscope}%
\pgfpathrectangle{\pgfqpoint{1.150000in}{0.150000in}}{\pgfqpoint{5.700000in}{5.700000in}}%
\pgfusepath{clip}%
\pgfsetbuttcap%
\pgfsetroundjoin%
\definecolor{currentfill}{rgb}{0.248629,0.278775,0.534556}%
\pgfsetfillcolor{currentfill}%
\pgfsetfillopacity{0.700000}%
\pgfsetlinewidth{0.000000pt}%
\definecolor{currentstroke}{rgb}{0.000000,0.000000,0.000000}%
\pgfsetstrokecolor{currentstroke}%
\pgfsetdash{}{0pt}%
\pgfpathmoveto{\pgfqpoint{4.775373in}{2.128741in}}%
\pgfpathlineto{\pgfqpoint{4.789636in}{2.131490in}}%
\pgfpathlineto{\pgfqpoint{4.803911in}{2.134311in}}%
\pgfpathlineto{\pgfqpoint{4.818197in}{2.137203in}}%
\pgfpathlineto{\pgfqpoint{4.832494in}{2.140167in}}%
\pgfpathlineto{\pgfqpoint{4.840303in}{2.148383in}}%
\pgfpathlineto{\pgfqpoint{4.848105in}{2.156501in}}%
\pgfpathlineto{\pgfqpoint{4.855900in}{2.164521in}}%
\pgfpathlineto{\pgfqpoint{4.863688in}{2.172443in}}%
\pgfpathlineto{\pgfqpoint{4.849402in}{2.169526in}}%
\pgfpathlineto{\pgfqpoint{4.835127in}{2.166680in}}%
\pgfpathlineto{\pgfqpoint{4.820864in}{2.163906in}}%
\pgfpathlineto{\pgfqpoint{4.806611in}{2.161202in}}%
\pgfpathlineto{\pgfqpoint{4.798812in}{2.153225in}}%
\pgfpathlineto{\pgfqpoint{4.791006in}{2.145157in}}%
\pgfpathlineto{\pgfqpoint{4.783193in}{2.136996in}}%
\pgfpathlineto{\pgfqpoint{4.775373in}{2.128741in}}%
\pgfpathclose%
\pgfusepath{fill}%
\end{pgfscope}%
\begin{pgfscope}%
\pgfpathrectangle{\pgfqpoint{1.150000in}{0.150000in}}{\pgfqpoint{5.700000in}{5.700000in}}%
\pgfusepath{clip}%
\pgfsetbuttcap%
\pgfsetroundjoin%
\definecolor{currentfill}{rgb}{0.268510,0.009605,0.335427}%
\pgfsetfillcolor{currentfill}%
\pgfsetfillopacity{0.700000}%
\pgfsetlinewidth{0.000000pt}%
\definecolor{currentstroke}{rgb}{0.000000,0.000000,0.000000}%
\pgfsetstrokecolor{currentstroke}%
\pgfsetdash{}{0pt}%
\pgfpathmoveto{\pgfqpoint{3.048568in}{1.604002in}}%
\pgfpathlineto{\pgfqpoint{3.062383in}{1.597913in}}%
\pgfpathlineto{\pgfqpoint{3.076201in}{1.591909in}}%
\pgfpathlineto{\pgfqpoint{3.090023in}{1.585992in}}%
\pgfpathlineto{\pgfqpoint{3.103848in}{1.580160in}}%
\pgfpathlineto{\pgfqpoint{3.112348in}{1.586533in}}%
\pgfpathlineto{\pgfqpoint{3.120837in}{1.593036in}}%
\pgfpathlineto{\pgfqpoint{3.129317in}{1.599665in}}%
\pgfpathlineto{\pgfqpoint{3.137788in}{1.606414in}}%
\pgfpathlineto{\pgfqpoint{3.123984in}{1.611938in}}%
\pgfpathlineto{\pgfqpoint{3.110183in}{1.617547in}}%
\pgfpathlineto{\pgfqpoint{3.096387in}{1.623242in}}%
\pgfpathlineto{\pgfqpoint{3.082595in}{1.629023in}}%
\pgfpathlineto{\pgfqpoint{3.074103in}{1.622574in}}%
\pgfpathlineto{\pgfqpoint{3.065601in}{1.616252in}}%
\pgfpathlineto{\pgfqpoint{3.057089in}{1.610059in}}%
\pgfpathlineto{\pgfqpoint{3.048568in}{1.604002in}}%
\pgfpathclose%
\pgfusepath{fill}%
\end{pgfscope}%
\begin{pgfscope}%
\pgfpathrectangle{\pgfqpoint{1.150000in}{0.150000in}}{\pgfqpoint{5.700000in}{5.700000in}}%
\pgfusepath{clip}%
\pgfsetbuttcap%
\pgfsetroundjoin%
\definecolor{currentfill}{rgb}{0.279566,0.067836,0.391917}%
\pgfsetfillcolor{currentfill}%
\pgfsetfillopacity{0.700000}%
\pgfsetlinewidth{0.000000pt}%
\definecolor{currentstroke}{rgb}{0.000000,0.000000,0.000000}%
\pgfsetstrokecolor{currentstroke}%
\pgfsetdash{}{0pt}%
\pgfpathmoveto{\pgfqpoint{2.703304in}{1.715132in}}%
\pgfpathlineto{\pgfqpoint{2.717120in}{1.706546in}}%
\pgfpathlineto{\pgfqpoint{2.730938in}{1.698055in}}%
\pgfpathlineto{\pgfqpoint{2.744757in}{1.689659in}}%
\pgfpathlineto{\pgfqpoint{2.758578in}{1.681358in}}%
\pgfpathlineto{\pgfqpoint{2.767297in}{1.684720in}}%
\pgfpathlineto{\pgfqpoint{2.776002in}{1.688279in}}%
\pgfpathlineto{\pgfqpoint{2.784694in}{1.692029in}}%
\pgfpathlineto{\pgfqpoint{2.793373in}{1.695966in}}%
\pgfpathlineto{\pgfqpoint{2.779581in}{1.703915in}}%
\pgfpathlineto{\pgfqpoint{2.765791in}{1.711958in}}%
\pgfpathlineto{\pgfqpoint{2.752003in}{1.720096in}}%
\pgfpathlineto{\pgfqpoint{2.738216in}{1.728330in}}%
\pgfpathlineto{\pgfqpoint{2.729508in}{1.724738in}}%
\pgfpathlineto{\pgfqpoint{2.720787in}{1.721338in}}%
\pgfpathlineto{\pgfqpoint{2.712052in}{1.718134in}}%
\pgfpathlineto{\pgfqpoint{2.703304in}{1.715132in}}%
\pgfpathclose%
\pgfusepath{fill}%
\end{pgfscope}%
\begin{pgfscope}%
\pgfpathrectangle{\pgfqpoint{1.150000in}{0.150000in}}{\pgfqpoint{5.700000in}{5.700000in}}%
\pgfusepath{clip}%
\pgfsetbuttcap%
\pgfsetroundjoin%
\definecolor{currentfill}{rgb}{0.253935,0.265254,0.529983}%
\pgfsetfillcolor{currentfill}%
\pgfsetfillopacity{0.700000}%
\pgfsetlinewidth{0.000000pt}%
\definecolor{currentstroke}{rgb}{0.000000,0.000000,0.000000}%
\pgfsetstrokecolor{currentstroke}%
\pgfsetdash{}{0pt}%
\pgfpathmoveto{\pgfqpoint{2.077135in}{2.160168in}}%
\pgfpathlineto{\pgfqpoint{2.091065in}{2.146363in}}%
\pgfpathlineto{\pgfqpoint{2.104992in}{2.132688in}}%
\pgfpathlineto{\pgfqpoint{2.118915in}{2.119141in}}%
\pgfpathlineto{\pgfqpoint{2.132835in}{2.105720in}}%
\pgfpathlineto{\pgfqpoint{2.142061in}{2.103204in}}%
\pgfpathlineto{\pgfqpoint{2.151264in}{2.100989in}}%
\pgfpathlineto{\pgfqpoint{2.160447in}{2.099067in}}%
\pgfpathlineto{\pgfqpoint{2.169608in}{2.097434in}}%
\pgfpathlineto{\pgfqpoint{2.155732in}{2.110446in}}%
\pgfpathlineto{\pgfqpoint{2.141853in}{2.123585in}}%
\pgfpathlineto{\pgfqpoint{2.127970in}{2.136851in}}%
\pgfpathlineto{\pgfqpoint{2.114085in}{2.150245in}}%
\pgfpathlineto{\pgfqpoint{2.104880in}{2.152279in}}%
\pgfpathlineto{\pgfqpoint{2.095654in}{2.154606in}}%
\pgfpathlineto{\pgfqpoint{2.086405in}{2.157234in}}%
\pgfpathlineto{\pgfqpoint{2.077135in}{2.160168in}}%
\pgfpathclose%
\pgfusepath{fill}%
\end{pgfscope}%
\begin{pgfscope}%
\pgfpathrectangle{\pgfqpoint{1.150000in}{0.150000in}}{\pgfqpoint{5.700000in}{5.700000in}}%
\pgfusepath{clip}%
\pgfsetbuttcap%
\pgfsetroundjoin%
\definecolor{currentfill}{rgb}{0.267004,0.004874,0.329415}%
\pgfsetfillcolor{currentfill}%
\pgfsetfillopacity{0.700000}%
\pgfsetlinewidth{0.000000pt}%
\definecolor{currentstroke}{rgb}{0.000000,0.000000,0.000000}%
\pgfsetstrokecolor{currentstroke}%
\pgfsetdash{}{0pt}%
\pgfpathmoveto{\pgfqpoint{3.193047in}{1.585161in}}%
\pgfpathlineto{\pgfqpoint{3.206873in}{1.580058in}}%
\pgfpathlineto{\pgfqpoint{3.220703in}{1.575037in}}%
\pgfpathlineto{\pgfqpoint{3.234538in}{1.570099in}}%
\pgfpathlineto{\pgfqpoint{3.248377in}{1.565243in}}%
\pgfpathlineto{\pgfqpoint{3.256799in}{1.572698in}}%
\pgfpathlineto{\pgfqpoint{3.265213in}{1.580255in}}%
\pgfpathlineto{\pgfqpoint{3.273618in}{1.587909in}}%
\pgfpathlineto{\pgfqpoint{3.282016in}{1.595656in}}%
\pgfpathlineto{\pgfqpoint{3.268195in}{1.600224in}}%
\pgfpathlineto{\pgfqpoint{3.254379in}{1.604875in}}%
\pgfpathlineto{\pgfqpoint{3.240568in}{1.609609in}}%
\pgfpathlineto{\pgfqpoint{3.226762in}{1.614426in}}%
\pgfpathlineto{\pgfqpoint{3.218346in}{1.606958in}}%
\pgfpathlineto{\pgfqpoint{3.209922in}{1.599589in}}%
\pgfpathlineto{\pgfqpoint{3.201489in}{1.592322in}}%
\pgfpathlineto{\pgfqpoint{3.193047in}{1.585161in}}%
\pgfpathclose%
\pgfusepath{fill}%
\end{pgfscope}%
\begin{pgfscope}%
\pgfpathrectangle{\pgfqpoint{1.150000in}{0.150000in}}{\pgfqpoint{5.700000in}{5.700000in}}%
\pgfusepath{clip}%
\pgfsetbuttcap%
\pgfsetroundjoin%
\definecolor{currentfill}{rgb}{0.274952,0.037752,0.364543}%
\pgfsetfillcolor{currentfill}%
\pgfsetfillopacity{0.700000}%
\pgfsetlinewidth{0.000000pt}%
\definecolor{currentstroke}{rgb}{0.000000,0.000000,0.000000}%
\pgfsetstrokecolor{currentstroke}%
\pgfsetdash{}{0pt}%
\pgfpathmoveto{\pgfqpoint{3.658838in}{1.634695in}}%
\pgfpathlineto{\pgfqpoint{3.672738in}{1.632529in}}%
\pgfpathlineto{\pgfqpoint{3.686644in}{1.630440in}}%
\pgfpathlineto{\pgfqpoint{3.700557in}{1.628428in}}%
\pgfpathlineto{\pgfqpoint{3.714478in}{1.626492in}}%
\pgfpathlineto{\pgfqpoint{3.722698in}{1.636382in}}%
\pgfpathlineto{\pgfqpoint{3.730913in}{1.646284in}}%
\pgfpathlineto{\pgfqpoint{3.739122in}{1.656197in}}%
\pgfpathlineto{\pgfqpoint{3.747325in}{1.666116in}}%
\pgfpathlineto{\pgfqpoint{3.733416in}{1.667847in}}%
\pgfpathlineto{\pgfqpoint{3.719515in}{1.669655in}}%
\pgfpathlineto{\pgfqpoint{3.705620in}{1.671538in}}%
\pgfpathlineto{\pgfqpoint{3.691733in}{1.673499in}}%
\pgfpathlineto{\pgfqpoint{3.683518in}{1.663777in}}%
\pgfpathlineto{\pgfqpoint{3.675297in}{1.654067in}}%
\pgfpathlineto{\pgfqpoint{3.667071in}{1.644372in}}%
\pgfpathlineto{\pgfqpoint{3.658838in}{1.634695in}}%
\pgfpathclose%
\pgfusepath{fill}%
\end{pgfscope}%
\begin{pgfscope}%
\pgfpathrectangle{\pgfqpoint{1.150000in}{0.150000in}}{\pgfqpoint{5.700000in}{5.700000in}}%
\pgfusepath{clip}%
\pgfsetbuttcap%
\pgfsetroundjoin%
\definecolor{currentfill}{rgb}{0.257322,0.256130,0.526563}%
\pgfsetfillcolor{currentfill}%
\pgfsetfillopacity{0.700000}%
\pgfsetlinewidth{0.000000pt}%
\definecolor{currentstroke}{rgb}{0.000000,0.000000,0.000000}%
\pgfsetstrokecolor{currentstroke}%
\pgfsetdash{}{0pt}%
\pgfpathmoveto{\pgfqpoint{4.687038in}{2.084379in}}%
\pgfpathlineto{\pgfqpoint{4.701268in}{2.086867in}}%
\pgfpathlineto{\pgfqpoint{4.715508in}{2.089427in}}%
\pgfpathlineto{\pgfqpoint{4.729759in}{2.092059in}}%
\pgfpathlineto{\pgfqpoint{4.744022in}{2.094762in}}%
\pgfpathlineto{\pgfqpoint{4.751870in}{2.103402in}}%
\pgfpathlineto{\pgfqpoint{4.759711in}{2.111945in}}%
\pgfpathlineto{\pgfqpoint{4.767546in}{2.120391in}}%
\pgfpathlineto{\pgfqpoint{4.775373in}{2.128741in}}%
\pgfpathlineto{\pgfqpoint{4.761121in}{2.126063in}}%
\pgfpathlineto{\pgfqpoint{4.746880in}{2.123456in}}%
\pgfpathlineto{\pgfqpoint{4.732650in}{2.120921in}}%
\pgfpathlineto{\pgfqpoint{4.718431in}{2.118457in}}%
\pgfpathlineto{\pgfqpoint{4.710593in}{2.110074in}}%
\pgfpathlineto{\pgfqpoint{4.702749in}{2.101601in}}%
\pgfpathlineto{\pgfqpoint{4.694897in}{2.093036in}}%
\pgfpathlineto{\pgfqpoint{4.687038in}{2.084379in}}%
\pgfpathclose%
\pgfusepath{fill}%
\end{pgfscope}%
\begin{pgfscope}%
\pgfpathrectangle{\pgfqpoint{1.150000in}{0.150000in}}{\pgfqpoint{5.700000in}{5.700000in}}%
\pgfusepath{clip}%
\pgfsetbuttcap%
\pgfsetroundjoin%
\definecolor{currentfill}{rgb}{0.263663,0.237631,0.518762}%
\pgfsetfillcolor{currentfill}%
\pgfsetfillopacity{0.700000}%
\pgfsetlinewidth{0.000000pt}%
\definecolor{currentstroke}{rgb}{0.000000,0.000000,0.000000}%
\pgfsetstrokecolor{currentstroke}%
\pgfsetdash{}{0pt}%
\pgfpathmoveto{\pgfqpoint{4.598689in}{2.039561in}}%
\pgfpathlineto{\pgfqpoint{4.612885in}{2.041766in}}%
\pgfpathlineto{\pgfqpoint{4.627091in}{2.044043in}}%
\pgfpathlineto{\pgfqpoint{4.641308in}{2.046391in}}%
\pgfpathlineto{\pgfqpoint{4.655537in}{2.048812in}}%
\pgfpathlineto{\pgfqpoint{4.663422in}{2.057845in}}%
\pgfpathlineto{\pgfqpoint{4.671301in}{2.066784in}}%
\pgfpathlineto{\pgfqpoint{4.679173in}{2.075628in}}%
\pgfpathlineto{\pgfqpoint{4.687038in}{2.084379in}}%
\pgfpathlineto{\pgfqpoint{4.672820in}{2.081962in}}%
\pgfpathlineto{\pgfqpoint{4.658613in}{2.079617in}}%
\pgfpathlineto{\pgfqpoint{4.644416in}{2.077343in}}%
\pgfpathlineto{\pgfqpoint{4.630230in}{2.075141in}}%
\pgfpathlineto{\pgfqpoint{4.622355in}{2.066380in}}%
\pgfpathlineto{\pgfqpoint{4.614473in}{2.057529in}}%
\pgfpathlineto{\pgfqpoint{4.606584in}{2.048590in}}%
\pgfpathlineto{\pgfqpoint{4.598689in}{2.039561in}}%
\pgfpathclose%
\pgfusepath{fill}%
\end{pgfscope}%
\begin{pgfscope}%
\pgfpathrectangle{\pgfqpoint{1.150000in}{0.150000in}}{\pgfqpoint{5.700000in}{5.700000in}}%
\pgfusepath{clip}%
\pgfsetbuttcap%
\pgfsetroundjoin%
\definecolor{currentfill}{rgb}{0.269308,0.218818,0.509577}%
\pgfsetfillcolor{currentfill}%
\pgfsetfillopacity{0.700000}%
\pgfsetlinewidth{0.000000pt}%
\definecolor{currentstroke}{rgb}{0.000000,0.000000,0.000000}%
\pgfsetstrokecolor{currentstroke}%
\pgfsetdash{}{0pt}%
\pgfpathmoveto{\pgfqpoint{4.510330in}{1.994514in}}%
\pgfpathlineto{\pgfqpoint{4.524492in}{1.996413in}}%
\pgfpathlineto{\pgfqpoint{4.538666in}{1.998384in}}%
\pgfpathlineto{\pgfqpoint{4.552849in}{2.000428in}}%
\pgfpathlineto{\pgfqpoint{4.567043in}{2.002543in}}%
\pgfpathlineto{\pgfqpoint{4.574965in}{2.011933in}}%
\pgfpathlineto{\pgfqpoint{4.582879in}{2.021233in}}%
\pgfpathlineto{\pgfqpoint{4.590788in}{2.030442in}}%
\pgfpathlineto{\pgfqpoint{4.598689in}{2.039561in}}%
\pgfpathlineto{\pgfqpoint{4.584504in}{2.037428in}}%
\pgfpathlineto{\pgfqpoint{4.570330in}{2.035367in}}%
\pgfpathlineto{\pgfqpoint{4.556166in}{2.033377in}}%
\pgfpathlineto{\pgfqpoint{4.542012in}{2.031460in}}%
\pgfpathlineto{\pgfqpoint{4.534101in}{2.022351in}}%
\pgfpathlineto{\pgfqpoint{4.526184in}{2.013157in}}%
\pgfpathlineto{\pgfqpoint{4.518260in}{2.003878in}}%
\pgfpathlineto{\pgfqpoint{4.510330in}{1.994514in}}%
\pgfpathclose%
\pgfusepath{fill}%
\end{pgfscope}%
\begin{pgfscope}%
\pgfpathrectangle{\pgfqpoint{1.150000in}{0.150000in}}{\pgfqpoint{5.700000in}{5.700000in}}%
\pgfusepath{clip}%
\pgfsetbuttcap%
\pgfsetroundjoin%
\definecolor{currentfill}{rgb}{0.272594,0.025563,0.353093}%
\pgfsetfillcolor{currentfill}%
\pgfsetfillopacity{0.700000}%
\pgfsetlinewidth{0.000000pt}%
\definecolor{currentstroke}{rgb}{0.000000,0.000000,0.000000}%
\pgfsetstrokecolor{currentstroke}%
\pgfsetdash{}{0pt}%
\pgfpathmoveto{\pgfqpoint{2.903790in}{1.635713in}}%
\pgfpathlineto{\pgfqpoint{2.917603in}{1.628591in}}%
\pgfpathlineto{\pgfqpoint{2.931419in}{1.621559in}}%
\pgfpathlineto{\pgfqpoint{2.945238in}{1.614615in}}%
\pgfpathlineto{\pgfqpoint{2.959060in}{1.607761in}}%
\pgfpathlineto{\pgfqpoint{2.967648in}{1.612892in}}%
\pgfpathlineto{\pgfqpoint{2.976224in}{1.618184in}}%
\pgfpathlineto{\pgfqpoint{2.984790in}{1.623630in}}%
\pgfpathlineto{\pgfqpoint{2.993345in}{1.629226in}}%
\pgfpathlineto{\pgfqpoint{2.979548in}{1.635751in}}%
\pgfpathlineto{\pgfqpoint{2.965753in}{1.642365in}}%
\pgfpathlineto{\pgfqpoint{2.951962in}{1.649067in}}%
\pgfpathlineto{\pgfqpoint{2.938174in}{1.655859in}}%
\pgfpathlineto{\pgfqpoint{2.929595in}{1.650585in}}%
\pgfpathlineto{\pgfqpoint{2.921005in}{1.645466in}}%
\pgfpathlineto{\pgfqpoint{2.912403in}{1.640507in}}%
\pgfpathlineto{\pgfqpoint{2.903790in}{1.635713in}}%
\pgfpathclose%
\pgfusepath{fill}%
\end{pgfscope}%
\begin{pgfscope}%
\pgfpathrectangle{\pgfqpoint{1.150000in}{0.150000in}}{\pgfqpoint{5.700000in}{5.700000in}}%
\pgfusepath{clip}%
\pgfsetbuttcap%
\pgfsetroundjoin%
\definecolor{currentfill}{rgb}{0.274128,0.199721,0.498911}%
\pgfsetfillcolor{currentfill}%
\pgfsetfillopacity{0.700000}%
\pgfsetlinewidth{0.000000pt}%
\definecolor{currentstroke}{rgb}{0.000000,0.000000,0.000000}%
\pgfsetstrokecolor{currentstroke}%
\pgfsetdash{}{0pt}%
\pgfpathmoveto{\pgfqpoint{4.421962in}{1.949484in}}%
\pgfpathlineto{\pgfqpoint{4.436093in}{1.951055in}}%
\pgfpathlineto{\pgfqpoint{4.450234in}{1.952699in}}%
\pgfpathlineto{\pgfqpoint{4.464385in}{1.954414in}}%
\pgfpathlineto{\pgfqpoint{4.478546in}{1.956202in}}%
\pgfpathlineto{\pgfqpoint{4.486501in}{1.965908in}}%
\pgfpathlineto{\pgfqpoint{4.494450in}{1.975529in}}%
\pgfpathlineto{\pgfqpoint{4.502393in}{1.985064in}}%
\pgfpathlineto{\pgfqpoint{4.510330in}{1.994514in}}%
\pgfpathlineto{\pgfqpoint{4.496177in}{1.992687in}}%
\pgfpathlineto{\pgfqpoint{4.482035in}{1.990932in}}%
\pgfpathlineto{\pgfqpoint{4.467903in}{1.989249in}}%
\pgfpathlineto{\pgfqpoint{4.453781in}{1.987638in}}%
\pgfpathlineto{\pgfqpoint{4.445836in}{1.978220in}}%
\pgfpathlineto{\pgfqpoint{4.437884in}{1.968721in}}%
\pgfpathlineto{\pgfqpoint{4.429926in}{1.959143in}}%
\pgfpathlineto{\pgfqpoint{4.421962in}{1.949484in}}%
\pgfpathclose%
\pgfusepath{fill}%
\end{pgfscope}%
\begin{pgfscope}%
\pgfpathrectangle{\pgfqpoint{1.150000in}{0.150000in}}{\pgfqpoint{5.700000in}{5.700000in}}%
\pgfusepath{clip}%
\pgfsetbuttcap%
\pgfsetroundjoin%
\definecolor{currentfill}{rgb}{0.278012,0.180367,0.486697}%
\pgfsetfillcolor{currentfill}%
\pgfsetfillopacity{0.700000}%
\pgfsetlinewidth{0.000000pt}%
\definecolor{currentstroke}{rgb}{0.000000,0.000000,0.000000}%
\pgfsetstrokecolor{currentstroke}%
\pgfsetdash{}{0pt}%
\pgfpathmoveto{\pgfqpoint{4.333588in}{1.904740in}}%
\pgfpathlineto{\pgfqpoint{4.347688in}{1.905961in}}%
\pgfpathlineto{\pgfqpoint{4.361797in}{1.907254in}}%
\pgfpathlineto{\pgfqpoint{4.375917in}{1.908620in}}%
\pgfpathlineto{\pgfqpoint{4.390046in}{1.910058in}}%
\pgfpathlineto{\pgfqpoint{4.398034in}{1.920033in}}%
\pgfpathlineto{\pgfqpoint{4.406016in}{1.929929in}}%
\pgfpathlineto{\pgfqpoint{4.413992in}{1.939746in}}%
\pgfpathlineto{\pgfqpoint{4.421962in}{1.949484in}}%
\pgfpathlineto{\pgfqpoint{4.407842in}{1.947985in}}%
\pgfpathlineto{\pgfqpoint{4.393731in}{1.946559in}}%
\pgfpathlineto{\pgfqpoint{4.379630in}{1.945205in}}%
\pgfpathlineto{\pgfqpoint{4.365539in}{1.943924in}}%
\pgfpathlineto{\pgfqpoint{4.357560in}{1.934238in}}%
\pgfpathlineto{\pgfqpoint{4.349576in}{1.924479in}}%
\pgfpathlineto{\pgfqpoint{4.341585in}{1.914646in}}%
\pgfpathlineto{\pgfqpoint{4.333588in}{1.904740in}}%
\pgfpathclose%
\pgfusepath{fill}%
\end{pgfscope}%
\begin{pgfscope}%
\pgfpathrectangle{\pgfqpoint{1.150000in}{0.150000in}}{\pgfqpoint{5.700000in}{5.700000in}}%
\pgfusepath{clip}%
\pgfsetbuttcap%
\pgfsetroundjoin%
\definecolor{currentfill}{rgb}{0.260571,0.246922,0.522828}%
\pgfsetfillcolor{currentfill}%
\pgfsetfillopacity{0.700000}%
\pgfsetlinewidth{0.000000pt}%
\definecolor{currentstroke}{rgb}{0.000000,0.000000,0.000000}%
\pgfsetstrokecolor{currentstroke}%
\pgfsetdash{}{0pt}%
\pgfpathmoveto{\pgfqpoint{2.132835in}{2.105720in}}%
\pgfpathlineto{\pgfqpoint{2.146752in}{2.092425in}}%
\pgfpathlineto{\pgfqpoint{2.160666in}{2.079255in}}%
\pgfpathlineto{\pgfqpoint{2.174577in}{2.066208in}}%
\pgfpathlineto{\pgfqpoint{2.188485in}{2.053284in}}%
\pgfpathlineto{\pgfqpoint{2.197666in}{2.051184in}}%
\pgfpathlineto{\pgfqpoint{2.206826in}{2.049379in}}%
\pgfpathlineto{\pgfqpoint{2.215966in}{2.047862in}}%
\pgfpathlineto{\pgfqpoint{2.225085in}{2.046627in}}%
\pgfpathlineto{\pgfqpoint{2.211220in}{2.059145in}}%
\pgfpathlineto{\pgfqpoint{2.197352in}{2.071784in}}%
\pgfpathlineto{\pgfqpoint{2.183481in}{2.084547in}}%
\pgfpathlineto{\pgfqpoint{2.169608in}{2.097434in}}%
\pgfpathlineto{\pgfqpoint{2.160447in}{2.099067in}}%
\pgfpathlineto{\pgfqpoint{2.151264in}{2.100989in}}%
\pgfpathlineto{\pgfqpoint{2.142061in}{2.103204in}}%
\pgfpathlineto{\pgfqpoint{2.132835in}{2.105720in}}%
\pgfpathclose%
\pgfusepath{fill}%
\end{pgfscope}%
\begin{pgfscope}%
\pgfpathrectangle{\pgfqpoint{1.150000in}{0.150000in}}{\pgfqpoint{5.700000in}{5.700000in}}%
\pgfusepath{clip}%
\pgfsetbuttcap%
\pgfsetroundjoin%
\definecolor{currentfill}{rgb}{0.188923,0.410910,0.556326}%
\pgfsetfillcolor{currentfill}%
\pgfsetfillopacity{0.700000}%
\pgfsetlinewidth{0.000000pt}%
\definecolor{currentstroke}{rgb}{0.000000,0.000000,0.000000}%
\pgfsetstrokecolor{currentstroke}%
\pgfsetdash{}{0pt}%
\pgfpathmoveto{\pgfqpoint{5.538933in}{2.463819in}}%
\pgfpathlineto{\pgfqpoint{5.553517in}{2.468127in}}%
\pgfpathlineto{\pgfqpoint{5.568116in}{2.472505in}}%
\pgfpathlineto{\pgfqpoint{5.582727in}{2.476953in}}%
\pgfpathlineto{\pgfqpoint{5.597352in}{2.481471in}}%
\pgfpathlineto{\pgfqpoint{5.604762in}{2.485632in}}%
\pgfpathlineto{\pgfqpoint{5.612164in}{2.489728in}}%
\pgfpathlineto{\pgfqpoint{5.619557in}{2.493762in}}%
\pgfpathlineto{\pgfqpoint{5.626942in}{2.497738in}}%
\pgfpathlineto{\pgfqpoint{5.612339in}{2.493442in}}%
\pgfpathlineto{\pgfqpoint{5.597748in}{2.489215in}}%
\pgfpathlineto{\pgfqpoint{5.583171in}{2.485057in}}%
\pgfpathlineto{\pgfqpoint{5.568607in}{2.480970in}}%
\pgfpathlineto{\pgfqpoint{5.561201in}{2.476765in}}%
\pgfpathlineto{\pgfqpoint{5.553786in}{2.472508in}}%
\pgfpathlineto{\pgfqpoint{5.546364in}{2.468194in}}%
\pgfpathlineto{\pgfqpoint{5.538933in}{2.463819in}}%
\pgfpathclose%
\pgfusepath{fill}%
\end{pgfscope}%
\begin{pgfscope}%
\pgfpathrectangle{\pgfqpoint{1.150000in}{0.150000in}}{\pgfqpoint{5.700000in}{5.700000in}}%
\pgfusepath{clip}%
\pgfsetbuttcap%
\pgfsetroundjoin%
\definecolor{currentfill}{rgb}{0.281412,0.155834,0.469201}%
\pgfsetfillcolor{currentfill}%
\pgfsetfillopacity{0.700000}%
\pgfsetlinewidth{0.000000pt}%
\definecolor{currentstroke}{rgb}{0.000000,0.000000,0.000000}%
\pgfsetstrokecolor{currentstroke}%
\pgfsetdash{}{0pt}%
\pgfpathmoveto{\pgfqpoint{4.245208in}{1.860571in}}%
\pgfpathlineto{\pgfqpoint{4.259277in}{1.861419in}}%
\pgfpathlineto{\pgfqpoint{4.273357in}{1.862340in}}%
\pgfpathlineto{\pgfqpoint{4.287445in}{1.863334in}}%
\pgfpathlineto{\pgfqpoint{4.301544in}{1.864401in}}%
\pgfpathlineto{\pgfqpoint{4.309564in}{1.874591in}}%
\pgfpathlineto{\pgfqpoint{4.317578in}{1.884712in}}%
\pgfpathlineto{\pgfqpoint{4.325586in}{1.894762in}}%
\pgfpathlineto{\pgfqpoint{4.333588in}{1.904740in}}%
\pgfpathlineto{\pgfqpoint{4.319498in}{1.903592in}}%
\pgfpathlineto{\pgfqpoint{4.305418in}{1.902517in}}%
\pgfpathlineto{\pgfqpoint{4.291348in}{1.901514in}}%
\pgfpathlineto{\pgfqpoint{4.277286in}{1.900585in}}%
\pgfpathlineto{\pgfqpoint{4.269275in}{1.890680in}}%
\pgfpathlineto{\pgfqpoint{4.261258in}{1.880709in}}%
\pgfpathlineto{\pgfqpoint{4.253236in}{1.870672in}}%
\pgfpathlineto{\pgfqpoint{4.245208in}{1.860571in}}%
\pgfpathclose%
\pgfusepath{fill}%
\end{pgfscope}%
\begin{pgfscope}%
\pgfpathrectangle{\pgfqpoint{1.150000in}{0.150000in}}{\pgfqpoint{5.700000in}{5.700000in}}%
\pgfusepath{clip}%
\pgfsetbuttcap%
\pgfsetroundjoin%
\definecolor{currentfill}{rgb}{0.267004,0.004874,0.329415}%
\pgfsetfillcolor{currentfill}%
\pgfsetfillopacity{0.700000}%
\pgfsetlinewidth{0.000000pt}%
\definecolor{currentstroke}{rgb}{0.000000,0.000000,0.000000}%
\pgfsetstrokecolor{currentstroke}%
\pgfsetdash{}{0pt}%
\pgfpathmoveto{\pgfqpoint{3.337349in}{1.578198in}}%
\pgfpathlineto{\pgfqpoint{3.351194in}{1.574037in}}%
\pgfpathlineto{\pgfqpoint{3.365046in}{1.569956in}}%
\pgfpathlineto{\pgfqpoint{3.378902in}{1.565955in}}%
\pgfpathlineto{\pgfqpoint{3.392765in}{1.562034in}}%
\pgfpathlineto{\pgfqpoint{3.401120in}{1.570418in}}%
\pgfpathlineto{\pgfqpoint{3.409467in}{1.578877in}}%
\pgfpathlineto{\pgfqpoint{3.417808in}{1.587406in}}%
\pgfpathlineto{\pgfqpoint{3.426141in}{1.596002in}}%
\pgfpathlineto{\pgfqpoint{3.412295in}{1.599657in}}%
\pgfpathlineto{\pgfqpoint{3.398455in}{1.603391in}}%
\pgfpathlineto{\pgfqpoint{3.384620in}{1.607206in}}%
\pgfpathlineto{\pgfqpoint{3.370791in}{1.611101in}}%
\pgfpathlineto{\pgfqpoint{3.362441in}{1.602763in}}%
\pgfpathlineto{\pgfqpoint{3.354085in}{1.594498in}}%
\pgfpathlineto{\pgfqpoint{3.345720in}{1.586308in}}%
\pgfpathlineto{\pgfqpoint{3.337349in}{1.578198in}}%
\pgfpathclose%
\pgfusepath{fill}%
\end{pgfscope}%
\begin{pgfscope}%
\pgfpathrectangle{\pgfqpoint{1.150000in}{0.150000in}}{\pgfqpoint{5.700000in}{5.700000in}}%
\pgfusepath{clip}%
\pgfsetbuttcap%
\pgfsetroundjoin%
\definecolor{currentfill}{rgb}{0.283229,0.120777,0.440584}%
\pgfsetfillcolor{currentfill}%
\pgfsetfillopacity{0.700000}%
\pgfsetlinewidth{0.000000pt}%
\definecolor{currentstroke}{rgb}{0.000000,0.000000,0.000000}%
\pgfsetstrokecolor{currentstroke}%
\pgfsetdash{}{0pt}%
\pgfpathmoveto{\pgfqpoint{2.502054in}{1.820498in}}%
\pgfpathlineto{\pgfqpoint{2.515894in}{1.810338in}}%
\pgfpathlineto{\pgfqpoint{2.529735in}{1.800281in}}%
\pgfpathlineto{\pgfqpoint{2.543576in}{1.790327in}}%
\pgfpathlineto{\pgfqpoint{2.557417in}{1.780475in}}%
\pgfpathlineto{\pgfqpoint{2.566290in}{1.781844in}}%
\pgfpathlineto{\pgfqpoint{2.575147in}{1.783449in}}%
\pgfpathlineto{\pgfqpoint{2.583989in}{1.785284in}}%
\pgfpathlineto{\pgfqpoint{2.592816in}{1.787344in}}%
\pgfpathlineto{\pgfqpoint{2.579008in}{1.796819in}}%
\pgfpathlineto{\pgfqpoint{2.565202in}{1.806396in}}%
\pgfpathlineto{\pgfqpoint{2.551395in}{1.816076in}}%
\pgfpathlineto{\pgfqpoint{2.537590in}{1.825860in}}%
\pgfpathlineto{\pgfqpoint{2.528730in}{1.824168in}}%
\pgfpathlineto{\pgfqpoint{2.519854in}{1.822707in}}%
\pgfpathlineto{\pgfqpoint{2.510962in}{1.821482in}}%
\pgfpathlineto{\pgfqpoint{2.502054in}{1.820498in}}%
\pgfpathclose%
\pgfusepath{fill}%
\end{pgfscope}%
\begin{pgfscope}%
\pgfpathrectangle{\pgfqpoint{1.150000in}{0.150000in}}{\pgfqpoint{5.700000in}{5.700000in}}%
\pgfusepath{clip}%
\pgfsetbuttcap%
\pgfsetroundjoin%
\definecolor{currentfill}{rgb}{0.282884,0.135920,0.453427}%
\pgfsetfillcolor{currentfill}%
\pgfsetfillopacity{0.700000}%
\pgfsetlinewidth{0.000000pt}%
\definecolor{currentstroke}{rgb}{0.000000,0.000000,0.000000}%
\pgfsetstrokecolor{currentstroke}%
\pgfsetdash{}{0pt}%
\pgfpathmoveto{\pgfqpoint{4.156817in}{1.817287in}}%
\pgfpathlineto{\pgfqpoint{4.170859in}{1.817740in}}%
\pgfpathlineto{\pgfqpoint{4.184909in}{1.818267in}}%
\pgfpathlineto{\pgfqpoint{4.198969in}{1.818866in}}%
\pgfpathlineto{\pgfqpoint{4.213038in}{1.819539in}}%
\pgfpathlineto{\pgfqpoint{4.221089in}{1.829889in}}%
\pgfpathlineto{\pgfqpoint{4.229134in}{1.840178in}}%
\pgfpathlineto{\pgfqpoint{4.237174in}{1.850406in}}%
\pgfpathlineto{\pgfqpoint{4.245208in}{1.860571in}}%
\pgfpathlineto{\pgfqpoint{4.231147in}{1.859796in}}%
\pgfpathlineto{\pgfqpoint{4.217096in}{1.859094in}}%
\pgfpathlineto{\pgfqpoint{4.203054in}{1.858465in}}%
\pgfpathlineto{\pgfqpoint{4.189022in}{1.857910in}}%
\pgfpathlineto{\pgfqpoint{4.180979in}{1.847839in}}%
\pgfpathlineto{\pgfqpoint{4.172931in}{1.837711in}}%
\pgfpathlineto{\pgfqpoint{4.164877in}{1.827527in}}%
\pgfpathlineto{\pgfqpoint{4.156817in}{1.817287in}}%
\pgfpathclose%
\pgfusepath{fill}%
\end{pgfscope}%
\begin{pgfscope}%
\pgfpathrectangle{\pgfqpoint{1.150000in}{0.150000in}}{\pgfqpoint{5.700000in}{5.700000in}}%
\pgfusepath{clip}%
\pgfsetbuttcap%
\pgfsetroundjoin%
\definecolor{currentfill}{rgb}{0.271305,0.019942,0.347269}%
\pgfsetfillcolor{currentfill}%
\pgfsetfillopacity{0.700000}%
\pgfsetlinewidth{0.000000pt}%
\definecolor{currentstroke}{rgb}{0.000000,0.000000,0.000000}%
\pgfsetstrokecolor{currentstroke}%
\pgfsetdash{}{0pt}%
\pgfpathmoveto{\pgfqpoint{3.570265in}{1.606562in}}%
\pgfpathlineto{\pgfqpoint{3.584150in}{1.603861in}}%
\pgfpathlineto{\pgfqpoint{3.598042in}{1.601239in}}%
\pgfpathlineto{\pgfqpoint{3.611941in}{1.598693in}}%
\pgfpathlineto{\pgfqpoint{3.625847in}{1.596225in}}%
\pgfpathlineto{\pgfqpoint{3.634104in}{1.605801in}}%
\pgfpathlineto{\pgfqpoint{3.642355in}{1.615406in}}%
\pgfpathlineto{\pgfqpoint{3.650599in}{1.625039in}}%
\pgfpathlineto{\pgfqpoint{3.658838in}{1.634695in}}%
\pgfpathlineto{\pgfqpoint{3.644945in}{1.636938in}}%
\pgfpathlineto{\pgfqpoint{3.631060in}{1.639257in}}%
\pgfpathlineto{\pgfqpoint{3.617180in}{1.641654in}}%
\pgfpathlineto{\pgfqpoint{3.603308in}{1.644129in}}%
\pgfpathlineto{\pgfqpoint{3.595056in}{1.634690in}}%
\pgfpathlineto{\pgfqpoint{3.586799in}{1.625281in}}%
\pgfpathlineto{\pgfqpoint{3.578535in}{1.615904in}}%
\pgfpathlineto{\pgfqpoint{3.570265in}{1.606562in}}%
\pgfpathclose%
\pgfusepath{fill}%
\end{pgfscope}%
\begin{pgfscope}%
\pgfpathrectangle{\pgfqpoint{1.150000in}{0.150000in}}{\pgfqpoint{5.700000in}{5.700000in}}%
\pgfusepath{clip}%
\pgfsetbuttcap%
\pgfsetroundjoin%
\definecolor{currentfill}{rgb}{0.283197,0.115680,0.436115}%
\pgfsetfillcolor{currentfill}%
\pgfsetfillopacity{0.700000}%
\pgfsetlinewidth{0.000000pt}%
\definecolor{currentstroke}{rgb}{0.000000,0.000000,0.000000}%
\pgfsetstrokecolor{currentstroke}%
\pgfsetdash{}{0pt}%
\pgfpathmoveto{\pgfqpoint{4.068413in}{1.775221in}}%
\pgfpathlineto{\pgfqpoint{4.082427in}{1.775256in}}%
\pgfpathlineto{\pgfqpoint{4.096451in}{1.775366in}}%
\pgfpathlineto{\pgfqpoint{4.110483in}{1.775549in}}%
\pgfpathlineto{\pgfqpoint{4.124524in}{1.775805in}}%
\pgfpathlineto{\pgfqpoint{4.132605in}{1.786251in}}%
\pgfpathlineto{\pgfqpoint{4.140682in}{1.796648in}}%
\pgfpathlineto{\pgfqpoint{4.148752in}{1.806994in}}%
\pgfpathlineto{\pgfqpoint{4.156817in}{1.817287in}}%
\pgfpathlineto{\pgfqpoint{4.142785in}{1.816908in}}%
\pgfpathlineto{\pgfqpoint{4.128762in}{1.816602in}}%
\pgfpathlineto{\pgfqpoint{4.114747in}{1.816369in}}%
\pgfpathlineto{\pgfqpoint{4.100742in}{1.816210in}}%
\pgfpathlineto{\pgfqpoint{4.092668in}{1.806032in}}%
\pgfpathlineto{\pgfqpoint{4.084588in}{1.795807in}}%
\pgfpathlineto{\pgfqpoint{4.076503in}{1.785536in}}%
\pgfpathlineto{\pgfqpoint{4.068413in}{1.775221in}}%
\pgfpathclose%
\pgfusepath{fill}%
\end{pgfscope}%
\begin{pgfscope}%
\pgfpathrectangle{\pgfqpoint{1.150000in}{0.150000in}}{\pgfqpoint{5.700000in}{5.700000in}}%
\pgfusepath{clip}%
\pgfsetbuttcap%
\pgfsetroundjoin%
\definecolor{currentfill}{rgb}{0.267968,0.223549,0.512008}%
\pgfsetfillcolor{currentfill}%
\pgfsetfillopacity{0.700000}%
\pgfsetlinewidth{0.000000pt}%
\definecolor{currentstroke}{rgb}{0.000000,0.000000,0.000000}%
\pgfsetstrokecolor{currentstroke}%
\pgfsetdash{}{0pt}%
\pgfpathmoveto{\pgfqpoint{2.188485in}{2.053284in}}%
\pgfpathlineto{\pgfqpoint{2.202390in}{2.040481in}}%
\pgfpathlineto{\pgfqpoint{2.216293in}{2.027799in}}%
\pgfpathlineto{\pgfqpoint{2.230193in}{2.015236in}}%
\pgfpathlineto{\pgfqpoint{2.244091in}{2.002792in}}%
\pgfpathlineto{\pgfqpoint{2.253230in}{2.001107in}}%
\pgfpathlineto{\pgfqpoint{2.262347in}{1.999710in}}%
\pgfpathlineto{\pgfqpoint{2.271445in}{1.998596in}}%
\pgfpathlineto{\pgfqpoint{2.280524in}{1.997759in}}%
\pgfpathlineto{\pgfqpoint{2.266667in}{2.009797in}}%
\pgfpathlineto{\pgfqpoint{2.252809in}{2.021955in}}%
\pgfpathlineto{\pgfqpoint{2.238948in}{2.034231in}}%
\pgfpathlineto{\pgfqpoint{2.225085in}{2.046627in}}%
\pgfpathlineto{\pgfqpoint{2.215966in}{2.047862in}}%
\pgfpathlineto{\pgfqpoint{2.206826in}{2.049379in}}%
\pgfpathlineto{\pgfqpoint{2.197666in}{2.051184in}}%
\pgfpathlineto{\pgfqpoint{2.188485in}{2.053284in}}%
\pgfpathclose%
\pgfusepath{fill}%
\end{pgfscope}%
\begin{pgfscope}%
\pgfpathrectangle{\pgfqpoint{1.150000in}{0.150000in}}{\pgfqpoint{5.700000in}{5.700000in}}%
\pgfusepath{clip}%
\pgfsetbuttcap%
\pgfsetroundjoin%
\definecolor{currentfill}{rgb}{0.194100,0.399323,0.555565}%
\pgfsetfillcolor{currentfill}%
\pgfsetfillopacity{0.700000}%
\pgfsetlinewidth{0.000000pt}%
\definecolor{currentstroke}{rgb}{0.000000,0.000000,0.000000}%
\pgfsetstrokecolor{currentstroke}%
\pgfsetdash{}{0pt}%
\pgfpathmoveto{\pgfqpoint{5.450837in}{2.428304in}}%
\pgfpathlineto{\pgfqpoint{5.465389in}{2.432532in}}%
\pgfpathlineto{\pgfqpoint{5.479955in}{2.436830in}}%
\pgfpathlineto{\pgfqpoint{5.494533in}{2.441198in}}%
\pgfpathlineto{\pgfqpoint{5.509125in}{2.445636in}}%
\pgfpathlineto{\pgfqpoint{5.516589in}{2.450291in}}%
\pgfpathlineto{\pgfqpoint{5.524046in}{2.454871in}}%
\pgfpathlineto{\pgfqpoint{5.531493in}{2.459379in}}%
\pgfpathlineto{\pgfqpoint{5.538933in}{2.463819in}}%
\pgfpathlineto{\pgfqpoint{5.524361in}{2.459580in}}%
\pgfpathlineto{\pgfqpoint{5.509802in}{2.455412in}}%
\pgfpathlineto{\pgfqpoint{5.495256in}{2.451313in}}%
\pgfpathlineto{\pgfqpoint{5.480724in}{2.447284in}}%
\pgfpathlineto{\pgfqpoint{5.473264in}{2.442637in}}%
\pgfpathlineto{\pgfqpoint{5.465797in}{2.437927in}}%
\pgfpathlineto{\pgfqpoint{5.458321in}{2.433151in}}%
\pgfpathlineto{\pgfqpoint{5.450837in}{2.428304in}}%
\pgfpathclose%
\pgfusepath{fill}%
\end{pgfscope}%
\begin{pgfscope}%
\pgfpathrectangle{\pgfqpoint{1.150000in}{0.150000in}}{\pgfqpoint{5.700000in}{5.700000in}}%
\pgfusepath{clip}%
\pgfsetbuttcap%
\pgfsetroundjoin%
\definecolor{currentfill}{rgb}{0.282656,0.100196,0.422160}%
\pgfsetfillcolor{currentfill}%
\pgfsetfillopacity{0.700000}%
\pgfsetlinewidth{0.000000pt}%
\definecolor{currentstroke}{rgb}{0.000000,0.000000,0.000000}%
\pgfsetstrokecolor{currentstroke}%
\pgfsetdash{}{0pt}%
\pgfpathmoveto{\pgfqpoint{3.979987in}{1.734723in}}%
\pgfpathlineto{\pgfqpoint{3.993977in}{1.734319in}}%
\pgfpathlineto{\pgfqpoint{4.007975in}{1.733989in}}%
\pgfpathlineto{\pgfqpoint{4.021981in}{1.733733in}}%
\pgfpathlineto{\pgfqpoint{4.035996in}{1.733551in}}%
\pgfpathlineto{\pgfqpoint{4.044108in}{1.744026in}}%
\pgfpathlineto{\pgfqpoint{4.052215in}{1.754464in}}%
\pgfpathlineto{\pgfqpoint{4.060317in}{1.764863in}}%
\pgfpathlineto{\pgfqpoint{4.068413in}{1.775221in}}%
\pgfpathlineto{\pgfqpoint{4.054407in}{1.775259in}}%
\pgfpathlineto{\pgfqpoint{4.040410in}{1.775371in}}%
\pgfpathlineto{\pgfqpoint{4.026421in}{1.775557in}}%
\pgfpathlineto{\pgfqpoint{4.012441in}{1.775818in}}%
\pgfpathlineto{\pgfqpoint{4.004336in}{1.765596in}}%
\pgfpathlineto{\pgfqpoint{3.996225in}{1.755338in}}%
\pgfpathlineto{\pgfqpoint{3.988109in}{1.745047in}}%
\pgfpathlineto{\pgfqpoint{3.979987in}{1.734723in}}%
\pgfpathclose%
\pgfusepath{fill}%
\end{pgfscope}%
\begin{pgfscope}%
\pgfpathrectangle{\pgfqpoint{1.150000in}{0.150000in}}{\pgfqpoint{5.700000in}{5.700000in}}%
\pgfusepath{clip}%
\pgfsetbuttcap%
\pgfsetroundjoin%
\definecolor{currentfill}{rgb}{0.277941,0.056324,0.381191}%
\pgfsetfillcolor{currentfill}%
\pgfsetfillopacity{0.700000}%
\pgfsetlinewidth{0.000000pt}%
\definecolor{currentstroke}{rgb}{0.000000,0.000000,0.000000}%
\pgfsetstrokecolor{currentstroke}%
\pgfsetdash{}{0pt}%
\pgfpathmoveto{\pgfqpoint{2.758578in}{1.681358in}}%
\pgfpathlineto{\pgfqpoint{2.772401in}{1.673151in}}%
\pgfpathlineto{\pgfqpoint{2.786226in}{1.665037in}}%
\pgfpathlineto{\pgfqpoint{2.800053in}{1.657017in}}%
\pgfpathlineto{\pgfqpoint{2.813881in}{1.649089in}}%
\pgfpathlineto{\pgfqpoint{2.822571in}{1.652811in}}%
\pgfpathlineto{\pgfqpoint{2.831247in}{1.656724in}}%
\pgfpathlineto{\pgfqpoint{2.839911in}{1.660824in}}%
\pgfpathlineto{\pgfqpoint{2.848563in}{1.665104in}}%
\pgfpathlineto{\pgfqpoint{2.834762in}{1.672680in}}%
\pgfpathlineto{\pgfqpoint{2.820964in}{1.680349in}}%
\pgfpathlineto{\pgfqpoint{2.807167in}{1.688111in}}%
\pgfpathlineto{\pgfqpoint{2.793373in}{1.695966in}}%
\pgfpathlineto{\pgfqpoint{2.784694in}{1.692029in}}%
\pgfpathlineto{\pgfqpoint{2.776002in}{1.688279in}}%
\pgfpathlineto{\pgfqpoint{2.767297in}{1.684720in}}%
\pgfpathlineto{\pgfqpoint{2.758578in}{1.681358in}}%
\pgfpathclose%
\pgfusepath{fill}%
\end{pgfscope}%
\begin{pgfscope}%
\pgfpathrectangle{\pgfqpoint{1.150000in}{0.150000in}}{\pgfqpoint{5.700000in}{5.700000in}}%
\pgfusepath{clip}%
\pgfsetbuttcap%
\pgfsetroundjoin%
\definecolor{currentfill}{rgb}{0.280894,0.078907,0.402329}%
\pgfsetfillcolor{currentfill}%
\pgfsetfillopacity{0.700000}%
\pgfsetlinewidth{0.000000pt}%
\definecolor{currentstroke}{rgb}{0.000000,0.000000,0.000000}%
\pgfsetstrokecolor{currentstroke}%
\pgfsetdash{}{0pt}%
\pgfpathmoveto{\pgfqpoint{3.891532in}{1.696169in}}%
\pgfpathlineto{\pgfqpoint{3.905498in}{1.695302in}}%
\pgfpathlineto{\pgfqpoint{3.919473in}{1.694510in}}%
\pgfpathlineto{\pgfqpoint{3.933455in}{1.693793in}}%
\pgfpathlineto{\pgfqpoint{3.947446in}{1.693150in}}%
\pgfpathlineto{\pgfqpoint{3.955590in}{1.703581in}}%
\pgfpathlineto{\pgfqpoint{3.963728in}{1.713988in}}%
\pgfpathlineto{\pgfqpoint{3.971860in}{1.724369in}}%
\pgfpathlineto{\pgfqpoint{3.979987in}{1.734723in}}%
\pgfpathlineto{\pgfqpoint{3.966006in}{1.735202in}}%
\pgfpathlineto{\pgfqpoint{3.952033in}{1.735755in}}%
\pgfpathlineto{\pgfqpoint{3.938069in}{1.736382in}}%
\pgfpathlineto{\pgfqpoint{3.924112in}{1.737084in}}%
\pgfpathlineto{\pgfqpoint{3.915975in}{1.726888in}}%
\pgfpathlineto{\pgfqpoint{3.907833in}{1.716668in}}%
\pgfpathlineto{\pgfqpoint{3.899685in}{1.706427in}}%
\pgfpathlineto{\pgfqpoint{3.891532in}{1.696169in}}%
\pgfpathclose%
\pgfusepath{fill}%
\end{pgfscope}%
\begin{pgfscope}%
\pgfpathrectangle{\pgfqpoint{1.150000in}{0.150000in}}{\pgfqpoint{5.700000in}{5.700000in}}%
\pgfusepath{clip}%
\pgfsetbuttcap%
\pgfsetroundjoin%
\definecolor{currentfill}{rgb}{0.199430,0.387607,0.554642}%
\pgfsetfillcolor{currentfill}%
\pgfsetfillopacity{0.700000}%
\pgfsetlinewidth{0.000000pt}%
\definecolor{currentstroke}{rgb}{0.000000,0.000000,0.000000}%
\pgfsetstrokecolor{currentstroke}%
\pgfsetdash{}{0pt}%
\pgfpathmoveto{\pgfqpoint{5.362664in}{2.391221in}}%
\pgfpathlineto{\pgfqpoint{5.377183in}{2.395346in}}%
\pgfpathlineto{\pgfqpoint{5.391714in}{2.399541in}}%
\pgfpathlineto{\pgfqpoint{5.406259in}{2.403806in}}%
\pgfpathlineto{\pgfqpoint{5.420816in}{2.408142in}}%
\pgfpathlineto{\pgfqpoint{5.428334in}{2.413305in}}%
\pgfpathlineto{\pgfqpoint{5.435844in}{2.418384in}}%
\pgfpathlineto{\pgfqpoint{5.443345in}{2.423383in}}%
\pgfpathlineto{\pgfqpoint{5.450837in}{2.428304in}}%
\pgfpathlineto{\pgfqpoint{5.436298in}{2.424146in}}%
\pgfpathlineto{\pgfqpoint{5.421772in}{2.420058in}}%
\pgfpathlineto{\pgfqpoint{5.407258in}{2.416040in}}%
\pgfpathlineto{\pgfqpoint{5.392757in}{2.412092in}}%
\pgfpathlineto{\pgfqpoint{5.385246in}{2.406986in}}%
\pgfpathlineto{\pgfqpoint{5.377727in}{2.401807in}}%
\pgfpathlineto{\pgfqpoint{5.370200in}{2.396553in}}%
\pgfpathlineto{\pgfqpoint{5.362664in}{2.391221in}}%
\pgfpathclose%
\pgfusepath{fill}%
\end{pgfscope}%
\begin{pgfscope}%
\pgfpathrectangle{\pgfqpoint{1.150000in}{0.150000in}}{\pgfqpoint{5.700000in}{5.700000in}}%
\pgfusepath{clip}%
\pgfsetbuttcap%
\pgfsetroundjoin%
\definecolor{currentfill}{rgb}{0.269944,0.014625,0.341379}%
\pgfsetfillcolor{currentfill}%
\pgfsetfillopacity{0.700000}%
\pgfsetlinewidth{0.000000pt}%
\definecolor{currentstroke}{rgb}{0.000000,0.000000,0.000000}%
\pgfsetstrokecolor{currentstroke}%
\pgfsetdash{}{0pt}%
\pgfpathmoveto{\pgfqpoint{3.481582in}{1.582179in}}%
\pgfpathlineto{\pgfqpoint{3.495457in}{1.578921in}}%
\pgfpathlineto{\pgfqpoint{3.509338in}{1.575741in}}%
\pgfpathlineto{\pgfqpoint{3.523225in}{1.572639in}}%
\pgfpathlineto{\pgfqpoint{3.537119in}{1.569615in}}%
\pgfpathlineto{\pgfqpoint{3.545415in}{1.578781in}}%
\pgfpathlineto{\pgfqpoint{3.553705in}{1.587997in}}%
\pgfpathlineto{\pgfqpoint{3.561988in}{1.597258in}}%
\pgfpathlineto{\pgfqpoint{3.570265in}{1.606562in}}%
\pgfpathlineto{\pgfqpoint{3.556385in}{1.609339in}}%
\pgfpathlineto{\pgfqpoint{3.542512in}{1.612195in}}%
\pgfpathlineto{\pgfqpoint{3.528646in}{1.615129in}}%
\pgfpathlineto{\pgfqpoint{3.514785in}{1.618142in}}%
\pgfpathlineto{\pgfqpoint{3.506494in}{1.609077in}}%
\pgfpathlineto{\pgfqpoint{3.498197in}{1.600059in}}%
\pgfpathlineto{\pgfqpoint{3.489893in}{1.591092in}}%
\pgfpathlineto{\pgfqpoint{3.481582in}{1.582179in}}%
\pgfpathclose%
\pgfusepath{fill}%
\end{pgfscope}%
\begin{pgfscope}%
\pgfpathrectangle{\pgfqpoint{1.150000in}{0.150000in}}{\pgfqpoint{5.700000in}{5.700000in}}%
\pgfusepath{clip}%
\pgfsetbuttcap%
\pgfsetroundjoin%
\definecolor{currentfill}{rgb}{0.267004,0.004874,0.329415}%
\pgfsetfillcolor{currentfill}%
\pgfsetfillopacity{0.700000}%
\pgfsetlinewidth{0.000000pt}%
\definecolor{currentstroke}{rgb}{0.000000,0.000000,0.000000}%
\pgfsetstrokecolor{currentstroke}%
\pgfsetdash{}{0pt}%
\pgfpathmoveto{\pgfqpoint{3.103848in}{1.580160in}}%
\pgfpathlineto{\pgfqpoint{3.117678in}{1.574412in}}%
\pgfpathlineto{\pgfqpoint{3.131512in}{1.568749in}}%
\pgfpathlineto{\pgfqpoint{3.145349in}{1.563170in}}%
\pgfpathlineto{\pgfqpoint{3.159191in}{1.557675in}}%
\pgfpathlineto{\pgfqpoint{3.167669in}{1.564364in}}%
\pgfpathlineto{\pgfqpoint{3.176137in}{1.571178in}}%
\pgfpathlineto{\pgfqpoint{3.184597in}{1.578112in}}%
\pgfpathlineto{\pgfqpoint{3.193047in}{1.585161in}}%
\pgfpathlineto{\pgfqpoint{3.179226in}{1.590349in}}%
\pgfpathlineto{\pgfqpoint{3.165409in}{1.595620in}}%
\pgfpathlineto{\pgfqpoint{3.151596in}{1.600975in}}%
\pgfpathlineto{\pgfqpoint{3.137788in}{1.606414in}}%
\pgfpathlineto{\pgfqpoint{3.129317in}{1.599665in}}%
\pgfpathlineto{\pgfqpoint{3.120837in}{1.593036in}}%
\pgfpathlineto{\pgfqpoint{3.112348in}{1.586533in}}%
\pgfpathlineto{\pgfqpoint{3.103848in}{1.580160in}}%
\pgfpathclose%
\pgfusepath{fill}%
\end{pgfscope}%
\begin{pgfscope}%
\pgfpathrectangle{\pgfqpoint{1.150000in}{0.150000in}}{\pgfqpoint{5.700000in}{5.700000in}}%
\pgfusepath{clip}%
\pgfsetbuttcap%
\pgfsetroundjoin%
\definecolor{currentfill}{rgb}{0.273006,0.204520,0.501721}%
\pgfsetfillcolor{currentfill}%
\pgfsetfillopacity{0.700000}%
\pgfsetlinewidth{0.000000pt}%
\definecolor{currentstroke}{rgb}{0.000000,0.000000,0.000000}%
\pgfsetstrokecolor{currentstroke}%
\pgfsetdash{}{0pt}%
\pgfpathmoveto{\pgfqpoint{2.244091in}{2.002792in}}%
\pgfpathlineto{\pgfqpoint{2.257987in}{1.990466in}}%
\pgfpathlineto{\pgfqpoint{2.271881in}{1.978257in}}%
\pgfpathlineto{\pgfqpoint{2.285773in}{1.966163in}}%
\pgfpathlineto{\pgfqpoint{2.299663in}{1.954184in}}%
\pgfpathlineto{\pgfqpoint{2.308760in}{1.952912in}}%
\pgfpathlineto{\pgfqpoint{2.317836in}{1.951922in}}%
\pgfpathlineto{\pgfqpoint{2.326894in}{1.951209in}}%
\pgfpathlineto{\pgfqpoint{2.335932in}{1.950767in}}%
\pgfpathlineto{\pgfqpoint{2.322082in}{1.962342in}}%
\pgfpathlineto{\pgfqpoint{2.308231in}{1.974032in}}%
\pgfpathlineto{\pgfqpoint{2.294378in}{1.985837in}}%
\pgfpathlineto{\pgfqpoint{2.280524in}{1.997759in}}%
\pgfpathlineto{\pgfqpoint{2.271445in}{1.998596in}}%
\pgfpathlineto{\pgfqpoint{2.262347in}{1.999710in}}%
\pgfpathlineto{\pgfqpoint{2.253230in}{2.001107in}}%
\pgfpathlineto{\pgfqpoint{2.244091in}{2.002792in}}%
\pgfpathclose%
\pgfusepath{fill}%
\end{pgfscope}%
\begin{pgfscope}%
\pgfpathrectangle{\pgfqpoint{1.150000in}{0.150000in}}{\pgfqpoint{5.700000in}{5.700000in}}%
\pgfusepath{clip}%
\pgfsetbuttcap%
\pgfsetroundjoin%
\definecolor{currentfill}{rgb}{0.282910,0.105393,0.426902}%
\pgfsetfillcolor{currentfill}%
\pgfsetfillopacity{0.700000}%
\pgfsetlinewidth{0.000000pt}%
\definecolor{currentstroke}{rgb}{0.000000,0.000000,0.000000}%
\pgfsetstrokecolor{currentstroke}%
\pgfsetdash{}{0pt}%
\pgfpathmoveto{\pgfqpoint{2.557417in}{1.780475in}}%
\pgfpathlineto{\pgfqpoint{2.571258in}{1.770725in}}%
\pgfpathlineto{\pgfqpoint{2.585100in}{1.761076in}}%
\pgfpathlineto{\pgfqpoint{2.598943in}{1.751526in}}%
\pgfpathlineto{\pgfqpoint{2.612786in}{1.742077in}}%
\pgfpathlineto{\pgfqpoint{2.621625in}{1.743830in}}%
\pgfpathlineto{\pgfqpoint{2.630449in}{1.745813in}}%
\pgfpathlineto{\pgfqpoint{2.639258in}{1.748021in}}%
\pgfpathlineto{\pgfqpoint{2.648052in}{1.750448in}}%
\pgfpathlineto{\pgfqpoint{2.634241in}{1.759522in}}%
\pgfpathlineto{\pgfqpoint{2.620432in}{1.768696in}}%
\pgfpathlineto{\pgfqpoint{2.606623in}{1.777969in}}%
\pgfpathlineto{\pgfqpoint{2.592816in}{1.787344in}}%
\pgfpathlineto{\pgfqpoint{2.583989in}{1.785284in}}%
\pgfpathlineto{\pgfqpoint{2.575147in}{1.783449in}}%
\pgfpathlineto{\pgfqpoint{2.566290in}{1.781844in}}%
\pgfpathlineto{\pgfqpoint{2.557417in}{1.780475in}}%
\pgfpathclose%
\pgfusepath{fill}%
\end{pgfscope}%
\begin{pgfscope}%
\pgfpathrectangle{\pgfqpoint{1.150000in}{0.150000in}}{\pgfqpoint{5.700000in}{5.700000in}}%
\pgfusepath{clip}%
\pgfsetbuttcap%
\pgfsetroundjoin%
\definecolor{currentfill}{rgb}{0.278791,0.062145,0.386592}%
\pgfsetfillcolor{currentfill}%
\pgfsetfillopacity{0.700000}%
\pgfsetlinewidth{0.000000pt}%
\definecolor{currentstroke}{rgb}{0.000000,0.000000,0.000000}%
\pgfsetstrokecolor{currentstroke}%
\pgfsetdash{}{0pt}%
\pgfpathmoveto{\pgfqpoint{3.803034in}{1.659953in}}%
\pgfpathlineto{\pgfqpoint{3.816979in}{1.658601in}}%
\pgfpathlineto{\pgfqpoint{3.830933in}{1.657324in}}%
\pgfpathlineto{\pgfqpoint{3.844894in}{1.656123in}}%
\pgfpathlineto{\pgfqpoint{3.858863in}{1.654996in}}%
\pgfpathlineto{\pgfqpoint{3.867038in}{1.665305in}}%
\pgfpathlineto{\pgfqpoint{3.875208in}{1.675605in}}%
\pgfpathlineto{\pgfqpoint{3.883373in}{1.685894in}}%
\pgfpathlineto{\pgfqpoint{3.891532in}{1.696169in}}%
\pgfpathlineto{\pgfqpoint{3.877573in}{1.697110in}}%
\pgfpathlineto{\pgfqpoint{3.863622in}{1.698127in}}%
\pgfpathlineto{\pgfqpoint{3.849680in}{1.699218in}}%
\pgfpathlineto{\pgfqpoint{3.835744in}{1.700385in}}%
\pgfpathlineto{\pgfqpoint{3.827575in}{1.690288in}}%
\pgfpathlineto{\pgfqpoint{3.819400in}{1.680181in}}%
\pgfpathlineto{\pgfqpoint{3.811220in}{1.670069in}}%
\pgfpathlineto{\pgfqpoint{3.803034in}{1.659953in}}%
\pgfpathclose%
\pgfusepath{fill}%
\end{pgfscope}%
\begin{pgfscope}%
\pgfpathrectangle{\pgfqpoint{1.150000in}{0.150000in}}{\pgfqpoint{5.700000in}{5.700000in}}%
\pgfusepath{clip}%
\pgfsetbuttcap%
\pgfsetroundjoin%
\definecolor{currentfill}{rgb}{0.271305,0.019942,0.347269}%
\pgfsetfillcolor{currentfill}%
\pgfsetfillopacity{0.700000}%
\pgfsetlinewidth{0.000000pt}%
\definecolor{currentstroke}{rgb}{0.000000,0.000000,0.000000}%
\pgfsetstrokecolor{currentstroke}%
\pgfsetdash{}{0pt}%
\pgfpathmoveto{\pgfqpoint{2.959060in}{1.607761in}}%
\pgfpathlineto{\pgfqpoint{2.972885in}{1.600994in}}%
\pgfpathlineto{\pgfqpoint{2.986713in}{1.594315in}}%
\pgfpathlineto{\pgfqpoint{3.000545in}{1.587723in}}%
\pgfpathlineto{\pgfqpoint{3.014379in}{1.581219in}}%
\pgfpathlineto{\pgfqpoint{3.022942in}{1.586688in}}%
\pgfpathlineto{\pgfqpoint{3.031495in}{1.592311in}}%
\pgfpathlineto{\pgfqpoint{3.040036in}{1.598084in}}%
\pgfpathlineto{\pgfqpoint{3.048568in}{1.604002in}}%
\pgfpathlineto{\pgfqpoint{3.034757in}{1.610177in}}%
\pgfpathlineto{\pgfqpoint{3.020950in}{1.616439in}}%
\pgfpathlineto{\pgfqpoint{3.007146in}{1.622789in}}%
\pgfpathlineto{\pgfqpoint{2.993345in}{1.629226in}}%
\pgfpathlineto{\pgfqpoint{2.984790in}{1.623630in}}%
\pgfpathlineto{\pgfqpoint{2.976224in}{1.618184in}}%
\pgfpathlineto{\pgfqpoint{2.967648in}{1.612892in}}%
\pgfpathlineto{\pgfqpoint{2.959060in}{1.607761in}}%
\pgfpathclose%
\pgfusepath{fill}%
\end{pgfscope}%
\begin{pgfscope}%
\pgfpathrectangle{\pgfqpoint{1.150000in}{0.150000in}}{\pgfqpoint{5.700000in}{5.700000in}}%
\pgfusepath{clip}%
\pgfsetbuttcap%
\pgfsetroundjoin%
\definecolor{currentfill}{rgb}{0.204903,0.375746,0.553533}%
\pgfsetfillcolor{currentfill}%
\pgfsetfillopacity{0.700000}%
\pgfsetlinewidth{0.000000pt}%
\definecolor{currentstroke}{rgb}{0.000000,0.000000,0.000000}%
\pgfsetstrokecolor{currentstroke}%
\pgfsetdash{}{0pt}%
\pgfpathmoveto{\pgfqpoint{5.274422in}{2.352619in}}%
\pgfpathlineto{\pgfqpoint{5.288907in}{2.356619in}}%
\pgfpathlineto{\pgfqpoint{5.303404in}{2.360689in}}%
\pgfpathlineto{\pgfqpoint{5.317914in}{2.364829in}}%
\pgfpathlineto{\pgfqpoint{5.332437in}{2.369040in}}%
\pgfpathlineto{\pgfqpoint{5.340006in}{2.374718in}}%
\pgfpathlineto{\pgfqpoint{5.347567in}{2.380306in}}%
\pgfpathlineto{\pgfqpoint{5.355120in}{2.385806in}}%
\pgfpathlineto{\pgfqpoint{5.362664in}{2.391221in}}%
\pgfpathlineto{\pgfqpoint{5.348158in}{2.387166in}}%
\pgfpathlineto{\pgfqpoint{5.333665in}{2.383181in}}%
\pgfpathlineto{\pgfqpoint{5.319185in}{2.379267in}}%
\pgfpathlineto{\pgfqpoint{5.304717in}{2.375422in}}%
\pgfpathlineto{\pgfqpoint{5.297155in}{2.369844in}}%
\pgfpathlineto{\pgfqpoint{5.289586in}{2.364186in}}%
\pgfpathlineto{\pgfqpoint{5.282008in}{2.358446in}}%
\pgfpathlineto{\pgfqpoint{5.274422in}{2.352619in}}%
\pgfpathclose%
\pgfusepath{fill}%
\end{pgfscope}%
\begin{pgfscope}%
\pgfpathrectangle{\pgfqpoint{1.150000in}{0.150000in}}{\pgfqpoint{5.700000in}{5.700000in}}%
\pgfusepath{clip}%
\pgfsetbuttcap%
\pgfsetroundjoin%
\definecolor{currentfill}{rgb}{0.267004,0.004874,0.329415}%
\pgfsetfillcolor{currentfill}%
\pgfsetfillopacity{0.700000}%
\pgfsetlinewidth{0.000000pt}%
\definecolor{currentstroke}{rgb}{0.000000,0.000000,0.000000}%
\pgfsetstrokecolor{currentstroke}%
\pgfsetdash{}{0pt}%
\pgfpathmoveto{\pgfqpoint{3.248377in}{1.565243in}}%
\pgfpathlineto{\pgfqpoint{3.262221in}{1.560469in}}%
\pgfpathlineto{\pgfqpoint{3.276070in}{1.555777in}}%
\pgfpathlineto{\pgfqpoint{3.289924in}{1.551167in}}%
\pgfpathlineto{\pgfqpoint{3.303782in}{1.546637in}}%
\pgfpathlineto{\pgfqpoint{3.312186in}{1.554388in}}%
\pgfpathlineto{\pgfqpoint{3.320581in}{1.562234in}}%
\pgfpathlineto{\pgfqpoint{3.328969in}{1.570172in}}%
\pgfpathlineto{\pgfqpoint{3.337349in}{1.578198in}}%
\pgfpathlineto{\pgfqpoint{3.323508in}{1.582441in}}%
\pgfpathlineto{\pgfqpoint{3.309672in}{1.586764in}}%
\pgfpathlineto{\pgfqpoint{3.295842in}{1.591169in}}%
\pgfpathlineto{\pgfqpoint{3.282016in}{1.595656in}}%
\pgfpathlineto{\pgfqpoint{3.273618in}{1.587909in}}%
\pgfpathlineto{\pgfqpoint{3.265213in}{1.580255in}}%
\pgfpathlineto{\pgfqpoint{3.256799in}{1.572698in}}%
\pgfpathlineto{\pgfqpoint{3.248377in}{1.565243in}}%
\pgfpathclose%
\pgfusepath{fill}%
\end{pgfscope}%
\begin{pgfscope}%
\pgfpathrectangle{\pgfqpoint{1.150000in}{0.150000in}}{\pgfqpoint{5.700000in}{5.700000in}}%
\pgfusepath{clip}%
\pgfsetbuttcap%
\pgfsetroundjoin%
\definecolor{currentfill}{rgb}{0.212395,0.359683,0.551710}%
\pgfsetfillcolor{currentfill}%
\pgfsetfillopacity{0.700000}%
\pgfsetlinewidth{0.000000pt}%
\definecolor{currentstroke}{rgb}{0.000000,0.000000,0.000000}%
\pgfsetstrokecolor{currentstroke}%
\pgfsetdash{}{0pt}%
\pgfpathmoveto{\pgfqpoint{5.186121in}{2.312573in}}%
\pgfpathlineto{\pgfqpoint{5.200571in}{2.316424in}}%
\pgfpathlineto{\pgfqpoint{5.215033in}{2.320346in}}%
\pgfpathlineto{\pgfqpoint{5.229508in}{2.324339in}}%
\pgfpathlineto{\pgfqpoint{5.243995in}{2.328402in}}%
\pgfpathlineto{\pgfqpoint{5.251615in}{2.334598in}}%
\pgfpathlineto{\pgfqpoint{5.259226in}{2.340698in}}%
\pgfpathlineto{\pgfqpoint{5.266828in}{2.346705in}}%
\pgfpathlineto{\pgfqpoint{5.274422in}{2.352619in}}%
\pgfpathlineto{\pgfqpoint{5.259951in}{2.348690in}}%
\pgfpathlineto{\pgfqpoint{5.245491in}{2.344831in}}%
\pgfpathlineto{\pgfqpoint{5.231044in}{2.341043in}}%
\pgfpathlineto{\pgfqpoint{5.216610in}{2.337325in}}%
\pgfpathlineto{\pgfqpoint{5.209000in}{2.331268in}}%
\pgfpathlineto{\pgfqpoint{5.201382in}{2.325126in}}%
\pgfpathlineto{\pgfqpoint{5.193755in}{2.318895in}}%
\pgfpathlineto{\pgfqpoint{5.186121in}{2.312573in}}%
\pgfpathclose%
\pgfusepath{fill}%
\end{pgfscope}%
\begin{pgfscope}%
\pgfpathrectangle{\pgfqpoint{1.150000in}{0.150000in}}{\pgfqpoint{5.700000in}{5.700000in}}%
\pgfusepath{clip}%
\pgfsetbuttcap%
\pgfsetroundjoin%
\definecolor{currentfill}{rgb}{0.276022,0.044167,0.370164}%
\pgfsetfillcolor{currentfill}%
\pgfsetfillopacity{0.700000}%
\pgfsetlinewidth{0.000000pt}%
\definecolor{currentstroke}{rgb}{0.000000,0.000000,0.000000}%
\pgfsetstrokecolor{currentstroke}%
\pgfsetdash{}{0pt}%
\pgfpathmoveto{\pgfqpoint{3.714478in}{1.626492in}}%
\pgfpathlineto{\pgfqpoint{3.728405in}{1.624632in}}%
\pgfpathlineto{\pgfqpoint{3.742340in}{1.622848in}}%
\pgfpathlineto{\pgfqpoint{3.756282in}{1.621139in}}%
\pgfpathlineto{\pgfqpoint{3.770232in}{1.619507in}}%
\pgfpathlineto{\pgfqpoint{3.778441in}{1.629610in}}%
\pgfpathlineto{\pgfqpoint{3.786644in}{1.639720in}}%
\pgfpathlineto{\pgfqpoint{3.794842in}{1.649836in}}%
\pgfpathlineto{\pgfqpoint{3.803034in}{1.659953in}}%
\pgfpathlineto{\pgfqpoint{3.789095in}{1.661380in}}%
\pgfpathlineto{\pgfqpoint{3.775164in}{1.662883in}}%
\pgfpathlineto{\pgfqpoint{3.761241in}{1.664462in}}%
\pgfpathlineto{\pgfqpoint{3.747325in}{1.666116in}}%
\pgfpathlineto{\pgfqpoint{3.739122in}{1.656197in}}%
\pgfpathlineto{\pgfqpoint{3.730913in}{1.646284in}}%
\pgfpathlineto{\pgfqpoint{3.722698in}{1.636382in}}%
\pgfpathlineto{\pgfqpoint{3.714478in}{1.626492in}}%
\pgfpathclose%
\pgfusepath{fill}%
\end{pgfscope}%
\begin{pgfscope}%
\pgfpathrectangle{\pgfqpoint{1.150000in}{0.150000in}}{\pgfqpoint{5.700000in}{5.700000in}}%
\pgfusepath{clip}%
\pgfsetbuttcap%
\pgfsetroundjoin%
\definecolor{currentfill}{rgb}{0.277134,0.185228,0.489898}%
\pgfsetfillcolor{currentfill}%
\pgfsetfillopacity{0.700000}%
\pgfsetlinewidth{0.000000pt}%
\definecolor{currentstroke}{rgb}{0.000000,0.000000,0.000000}%
\pgfsetstrokecolor{currentstroke}%
\pgfsetdash{}{0pt}%
\pgfpathmoveto{\pgfqpoint{2.299663in}{1.954184in}}%
\pgfpathlineto{\pgfqpoint{2.313552in}{1.942320in}}%
\pgfpathlineto{\pgfqpoint{2.327439in}{1.930568in}}%
\pgfpathlineto{\pgfqpoint{2.341324in}{1.918929in}}%
\pgfpathlineto{\pgfqpoint{2.355209in}{1.907402in}}%
\pgfpathlineto{\pgfqpoint{2.364264in}{1.906540in}}%
\pgfpathlineto{\pgfqpoint{2.373301in}{1.905956in}}%
\pgfpathlineto{\pgfqpoint{2.382319in}{1.905643in}}%
\pgfpathlineto{\pgfqpoint{2.391319in}{1.905595in}}%
\pgfpathlineto{\pgfqpoint{2.377474in}{1.916720in}}%
\pgfpathlineto{\pgfqpoint{2.363628in}{1.927957in}}%
\pgfpathlineto{\pgfqpoint{2.349781in}{1.939306in}}%
\pgfpathlineto{\pgfqpoint{2.335932in}{1.950767in}}%
\pgfpathlineto{\pgfqpoint{2.326894in}{1.951209in}}%
\pgfpathlineto{\pgfqpoint{2.317836in}{1.951922in}}%
\pgfpathlineto{\pgfqpoint{2.308760in}{1.952912in}}%
\pgfpathlineto{\pgfqpoint{2.299663in}{1.954184in}}%
\pgfpathclose%
\pgfusepath{fill}%
\end{pgfscope}%
\begin{pgfscope}%
\pgfpathrectangle{\pgfqpoint{1.150000in}{0.150000in}}{\pgfqpoint{5.700000in}{5.700000in}}%
\pgfusepath{clip}%
\pgfsetbuttcap%
\pgfsetroundjoin%
\definecolor{currentfill}{rgb}{0.220057,0.343307,0.549413}%
\pgfsetfillcolor{currentfill}%
\pgfsetfillopacity{0.700000}%
\pgfsetlinewidth{0.000000pt}%
\definecolor{currentstroke}{rgb}{0.000000,0.000000,0.000000}%
\pgfsetstrokecolor{currentstroke}%
\pgfsetdash{}{0pt}%
\pgfpathmoveto{\pgfqpoint{5.097769in}{2.271176in}}%
\pgfpathlineto{\pgfqpoint{5.112184in}{2.274857in}}%
\pgfpathlineto{\pgfqpoint{5.126611in}{2.278609in}}%
\pgfpathlineto{\pgfqpoint{5.141050in}{2.282431in}}%
\pgfpathlineto{\pgfqpoint{5.155502in}{2.286324in}}%
\pgfpathlineto{\pgfqpoint{5.163169in}{2.293035in}}%
\pgfpathlineto{\pgfqpoint{5.170828in}{2.299645in}}%
\pgfpathlineto{\pgfqpoint{5.178479in}{2.306157in}}%
\pgfpathlineto{\pgfqpoint{5.186121in}{2.312573in}}%
\pgfpathlineto{\pgfqpoint{5.171684in}{2.308792in}}%
\pgfpathlineto{\pgfqpoint{5.157259in}{2.305081in}}%
\pgfpathlineto{\pgfqpoint{5.142846in}{2.301441in}}%
\pgfpathlineto{\pgfqpoint{5.128445in}{2.297872in}}%
\pgfpathlineto{\pgfqpoint{5.120788in}{2.291336in}}%
\pgfpathlineto{\pgfqpoint{5.113123in}{2.284710in}}%
\pgfpathlineto{\pgfqpoint{5.105450in}{2.277991in}}%
\pgfpathlineto{\pgfqpoint{5.097769in}{2.271176in}}%
\pgfpathclose%
\pgfusepath{fill}%
\end{pgfscope}%
\begin{pgfscope}%
\pgfpathrectangle{\pgfqpoint{1.150000in}{0.150000in}}{\pgfqpoint{5.700000in}{5.700000in}}%
\pgfusepath{clip}%
\pgfsetbuttcap%
\pgfsetroundjoin%
\definecolor{currentfill}{rgb}{0.267004,0.004874,0.329415}%
\pgfsetfillcolor{currentfill}%
\pgfsetfillopacity{0.700000}%
\pgfsetlinewidth{0.000000pt}%
\definecolor{currentstroke}{rgb}{0.000000,0.000000,0.000000}%
\pgfsetstrokecolor{currentstroke}%
\pgfsetdash{}{0pt}%
\pgfpathmoveto{\pgfqpoint{3.392765in}{1.562034in}}%
\pgfpathlineto{\pgfqpoint{3.406632in}{1.558193in}}%
\pgfpathlineto{\pgfqpoint{3.420505in}{1.554431in}}%
\pgfpathlineto{\pgfqpoint{3.434384in}{1.550749in}}%
\pgfpathlineto{\pgfqpoint{3.448269in}{1.547145in}}%
\pgfpathlineto{\pgfqpoint{3.456608in}{1.555803in}}%
\pgfpathlineto{\pgfqpoint{3.464939in}{1.564531in}}%
\pgfpathlineto{\pgfqpoint{3.473264in}{1.573324in}}%
\pgfpathlineto{\pgfqpoint{3.481582in}{1.582179in}}%
\pgfpathlineto{\pgfqpoint{3.467713in}{1.585516in}}%
\pgfpathlineto{\pgfqpoint{3.453850in}{1.588932in}}%
\pgfpathlineto{\pgfqpoint{3.439993in}{1.592428in}}%
\pgfpathlineto{\pgfqpoint{3.426141in}{1.596002in}}%
\pgfpathlineto{\pgfqpoint{3.417808in}{1.587406in}}%
\pgfpathlineto{\pgfqpoint{3.409467in}{1.578877in}}%
\pgfpathlineto{\pgfqpoint{3.401120in}{1.570418in}}%
\pgfpathlineto{\pgfqpoint{3.392765in}{1.562034in}}%
\pgfpathclose%
\pgfusepath{fill}%
\end{pgfscope}%
\begin{pgfscope}%
\pgfpathrectangle{\pgfqpoint{1.150000in}{0.150000in}}{\pgfqpoint{5.700000in}{5.700000in}}%
\pgfusepath{clip}%
\pgfsetbuttcap%
\pgfsetroundjoin%
\definecolor{currentfill}{rgb}{0.276022,0.044167,0.370164}%
\pgfsetfillcolor{currentfill}%
\pgfsetfillopacity{0.700000}%
\pgfsetlinewidth{0.000000pt}%
\definecolor{currentstroke}{rgb}{0.000000,0.000000,0.000000}%
\pgfsetstrokecolor{currentstroke}%
\pgfsetdash{}{0pt}%
\pgfpathmoveto{\pgfqpoint{2.813881in}{1.649089in}}%
\pgfpathlineto{\pgfqpoint{2.827712in}{1.641253in}}%
\pgfpathlineto{\pgfqpoint{2.841545in}{1.633508in}}%
\pgfpathlineto{\pgfqpoint{2.855380in}{1.625855in}}%
\pgfpathlineto{\pgfqpoint{2.869218in}{1.618292in}}%
\pgfpathlineto{\pgfqpoint{2.877879in}{1.622373in}}%
\pgfpathlineto{\pgfqpoint{2.886528in}{1.626641in}}%
\pgfpathlineto{\pgfqpoint{2.895165in}{1.631089in}}%
\pgfpathlineto{\pgfqpoint{2.903790in}{1.635713in}}%
\pgfpathlineto{\pgfqpoint{2.889979in}{1.642925in}}%
\pgfpathlineto{\pgfqpoint{2.876171in}{1.650227in}}%
\pgfpathlineto{\pgfqpoint{2.862366in}{1.657620in}}%
\pgfpathlineto{\pgfqpoint{2.848563in}{1.665104in}}%
\pgfpathlineto{\pgfqpoint{2.839911in}{1.660824in}}%
\pgfpathlineto{\pgfqpoint{2.831247in}{1.656724in}}%
\pgfpathlineto{\pgfqpoint{2.822571in}{1.652811in}}%
\pgfpathlineto{\pgfqpoint{2.813881in}{1.649089in}}%
\pgfpathclose%
\pgfusepath{fill}%
\end{pgfscope}%
\begin{pgfscope}%
\pgfpathrectangle{\pgfqpoint{1.150000in}{0.150000in}}{\pgfqpoint{5.700000in}{5.700000in}}%
\pgfusepath{clip}%
\pgfsetbuttcap%
\pgfsetroundjoin%
\definecolor{currentfill}{rgb}{0.227802,0.326594,0.546532}%
\pgfsetfillcolor{currentfill}%
\pgfsetfillopacity{0.700000}%
\pgfsetlinewidth{0.000000pt}%
\definecolor{currentstroke}{rgb}{0.000000,0.000000,0.000000}%
\pgfsetstrokecolor{currentstroke}%
\pgfsetdash{}{0pt}%
\pgfpathmoveto{\pgfqpoint{5.009375in}{2.228546in}}%
\pgfpathlineto{\pgfqpoint{5.023754in}{2.232034in}}%
\pgfpathlineto{\pgfqpoint{5.038146in}{2.235593in}}%
\pgfpathlineto{\pgfqpoint{5.052549in}{2.239222in}}%
\pgfpathlineto{\pgfqpoint{5.066965in}{2.242923in}}%
\pgfpathlineto{\pgfqpoint{5.074678in}{2.250139in}}%
\pgfpathlineto{\pgfqpoint{5.082383in}{2.257252in}}%
\pgfpathlineto{\pgfqpoint{5.090080in}{2.264264in}}%
\pgfpathlineto{\pgfqpoint{5.097769in}{2.271176in}}%
\pgfpathlineto{\pgfqpoint{5.083366in}{2.267566in}}%
\pgfpathlineto{\pgfqpoint{5.068976in}{2.264026in}}%
\pgfpathlineto{\pgfqpoint{5.054598in}{2.260557in}}%
\pgfpathlineto{\pgfqpoint{5.040231in}{2.257159in}}%
\pgfpathlineto{\pgfqpoint{5.032529in}{2.250149in}}%
\pgfpathlineto{\pgfqpoint{5.024819in}{2.243045in}}%
\pgfpathlineto{\pgfqpoint{5.017101in}{2.235844in}}%
\pgfpathlineto{\pgfqpoint{5.009375in}{2.228546in}}%
\pgfpathclose%
\pgfusepath{fill}%
\end{pgfscope}%
\begin{pgfscope}%
\pgfpathrectangle{\pgfqpoint{1.150000in}{0.150000in}}{\pgfqpoint{5.700000in}{5.700000in}}%
\pgfusepath{clip}%
\pgfsetbuttcap%
\pgfsetroundjoin%
\definecolor{currentfill}{rgb}{0.281924,0.089666,0.412415}%
\pgfsetfillcolor{currentfill}%
\pgfsetfillopacity{0.700000}%
\pgfsetlinewidth{0.000000pt}%
\definecolor{currentstroke}{rgb}{0.000000,0.000000,0.000000}%
\pgfsetstrokecolor{currentstroke}%
\pgfsetdash{}{0pt}%
\pgfpathmoveto{\pgfqpoint{2.612786in}{1.742077in}}%
\pgfpathlineto{\pgfqpoint{2.626630in}{1.732726in}}%
\pgfpathlineto{\pgfqpoint{2.640474in}{1.723474in}}%
\pgfpathlineto{\pgfqpoint{2.654320in}{1.714319in}}%
\pgfpathlineto{\pgfqpoint{2.668167in}{1.705261in}}%
\pgfpathlineto{\pgfqpoint{2.676973in}{1.707397in}}%
\pgfpathlineto{\pgfqpoint{2.685764in}{1.709758in}}%
\pgfpathlineto{\pgfqpoint{2.694541in}{1.712339in}}%
\pgfpathlineto{\pgfqpoint{2.703304in}{1.715132in}}%
\pgfpathlineto{\pgfqpoint{2.689489in}{1.723815in}}%
\pgfpathlineto{\pgfqpoint{2.675675in}{1.732595in}}%
\pgfpathlineto{\pgfqpoint{2.661863in}{1.741473in}}%
\pgfpathlineto{\pgfqpoint{2.648052in}{1.750448in}}%
\pgfpathlineto{\pgfqpoint{2.639258in}{1.748021in}}%
\pgfpathlineto{\pgfqpoint{2.630449in}{1.745813in}}%
\pgfpathlineto{\pgfqpoint{2.621625in}{1.743830in}}%
\pgfpathlineto{\pgfqpoint{2.612786in}{1.742077in}}%
\pgfpathclose%
\pgfusepath{fill}%
\end{pgfscope}%
\begin{pgfscope}%
\pgfpathrectangle{\pgfqpoint{1.150000in}{0.150000in}}{\pgfqpoint{5.700000in}{5.700000in}}%
\pgfusepath{clip}%
\pgfsetbuttcap%
\pgfsetroundjoin%
\definecolor{currentfill}{rgb}{0.235526,0.309527,0.542944}%
\pgfsetfillcolor{currentfill}%
\pgfsetfillopacity{0.700000}%
\pgfsetlinewidth{0.000000pt}%
\definecolor{currentstroke}{rgb}{0.000000,0.000000,0.000000}%
\pgfsetstrokecolor{currentstroke}%
\pgfsetdash{}{0pt}%
\pgfpathmoveto{\pgfqpoint{4.920947in}{2.184823in}}%
\pgfpathlineto{\pgfqpoint{4.935291in}{2.188095in}}%
\pgfpathlineto{\pgfqpoint{4.949647in}{2.191438in}}%
\pgfpathlineto{\pgfqpoint{4.964015in}{2.194853in}}%
\pgfpathlineto{\pgfqpoint{4.978394in}{2.198338in}}%
\pgfpathlineto{\pgfqpoint{4.986151in}{2.206046in}}%
\pgfpathlineto{\pgfqpoint{4.993900in}{2.213648in}}%
\pgfpathlineto{\pgfqpoint{5.001641in}{2.221148in}}%
\pgfpathlineto{\pgfqpoint{5.009375in}{2.228546in}}%
\pgfpathlineto{\pgfqpoint{4.995007in}{2.225129in}}%
\pgfpathlineto{\pgfqpoint{4.980652in}{2.221783in}}%
\pgfpathlineto{\pgfqpoint{4.966308in}{2.218507in}}%
\pgfpathlineto{\pgfqpoint{4.951976in}{2.215303in}}%
\pgfpathlineto{\pgfqpoint{4.944230in}{2.207829in}}%
\pgfpathlineto{\pgfqpoint{4.936477in}{2.200258in}}%
\pgfpathlineto{\pgfqpoint{4.928716in}{2.192590in}}%
\pgfpathlineto{\pgfqpoint{4.920947in}{2.184823in}}%
\pgfpathclose%
\pgfusepath{fill}%
\end{pgfscope}%
\begin{pgfscope}%
\pgfpathrectangle{\pgfqpoint{1.150000in}{0.150000in}}{\pgfqpoint{5.700000in}{5.700000in}}%
\pgfusepath{clip}%
\pgfsetbuttcap%
\pgfsetroundjoin%
\definecolor{currentfill}{rgb}{0.243113,0.292092,0.538516}%
\pgfsetfillcolor{currentfill}%
\pgfsetfillopacity{0.700000}%
\pgfsetlinewidth{0.000000pt}%
\definecolor{currentstroke}{rgb}{0.000000,0.000000,0.000000}%
\pgfsetstrokecolor{currentstroke}%
\pgfsetdash{}{0pt}%
\pgfpathmoveto{\pgfqpoint{4.832494in}{2.140167in}}%
\pgfpathlineto{\pgfqpoint{4.846803in}{2.143201in}}%
\pgfpathlineto{\pgfqpoint{4.861123in}{2.146307in}}%
\pgfpathlineto{\pgfqpoint{4.875455in}{2.149484in}}%
\pgfpathlineto{\pgfqpoint{4.889798in}{2.152732in}}%
\pgfpathlineto{\pgfqpoint{4.897597in}{2.160910in}}%
\pgfpathlineto{\pgfqpoint{4.905388in}{2.168984in}}%
\pgfpathlineto{\pgfqpoint{4.913171in}{2.176954in}}%
\pgfpathlineto{\pgfqpoint{4.920947in}{2.184823in}}%
\pgfpathlineto{\pgfqpoint{4.906615in}{2.181621in}}%
\pgfpathlineto{\pgfqpoint{4.892294in}{2.178491in}}%
\pgfpathlineto{\pgfqpoint{4.877985in}{2.175431in}}%
\pgfpathlineto{\pgfqpoint{4.863688in}{2.172443in}}%
\pgfpathlineto{\pgfqpoint{4.855900in}{2.164521in}}%
\pgfpathlineto{\pgfqpoint{4.848105in}{2.156501in}}%
\pgfpathlineto{\pgfqpoint{4.840303in}{2.148383in}}%
\pgfpathlineto{\pgfqpoint{4.832494in}{2.140167in}}%
\pgfpathclose%
\pgfusepath{fill}%
\end{pgfscope}%
\begin{pgfscope}%
\pgfpathrectangle{\pgfqpoint{1.150000in}{0.150000in}}{\pgfqpoint{5.700000in}{5.700000in}}%
\pgfusepath{clip}%
\pgfsetbuttcap%
\pgfsetroundjoin%
\definecolor{currentfill}{rgb}{0.272594,0.025563,0.353093}%
\pgfsetfillcolor{currentfill}%
\pgfsetfillopacity{0.700000}%
\pgfsetlinewidth{0.000000pt}%
\definecolor{currentstroke}{rgb}{0.000000,0.000000,0.000000}%
\pgfsetstrokecolor{currentstroke}%
\pgfsetdash{}{0pt}%
\pgfpathmoveto{\pgfqpoint{3.625847in}{1.596225in}}%
\pgfpathlineto{\pgfqpoint{3.639759in}{1.593834in}}%
\pgfpathlineto{\pgfqpoint{3.653678in}{1.591519in}}%
\pgfpathlineto{\pgfqpoint{3.667603in}{1.589281in}}%
\pgfpathlineto{\pgfqpoint{3.681536in}{1.587119in}}%
\pgfpathlineto{\pgfqpoint{3.689781in}{1.596928in}}%
\pgfpathlineto{\pgfqpoint{3.698019in}{1.606762in}}%
\pgfpathlineto{\pgfqpoint{3.706251in}{1.616617in}}%
\pgfpathlineto{\pgfqpoint{3.714478in}{1.626492in}}%
\pgfpathlineto{\pgfqpoint{3.700557in}{1.628428in}}%
\pgfpathlineto{\pgfqpoint{3.686644in}{1.630440in}}%
\pgfpathlineto{\pgfqpoint{3.672738in}{1.632529in}}%
\pgfpathlineto{\pgfqpoint{3.658838in}{1.634695in}}%
\pgfpathlineto{\pgfqpoint{3.650599in}{1.625039in}}%
\pgfpathlineto{\pgfqpoint{3.642355in}{1.615406in}}%
\pgfpathlineto{\pgfqpoint{3.634104in}{1.605801in}}%
\pgfpathlineto{\pgfqpoint{3.625847in}{1.596225in}}%
\pgfpathclose%
\pgfusepath{fill}%
\end{pgfscope}%
\begin{pgfscope}%
\pgfpathrectangle{\pgfqpoint{1.150000in}{0.150000in}}{\pgfqpoint{5.700000in}{5.700000in}}%
\pgfusepath{clip}%
\pgfsetbuttcap%
\pgfsetroundjoin%
\definecolor{currentfill}{rgb}{0.250425,0.274290,0.533103}%
\pgfsetfillcolor{currentfill}%
\pgfsetfillopacity{0.700000}%
\pgfsetlinewidth{0.000000pt}%
\definecolor{currentstroke}{rgb}{0.000000,0.000000,0.000000}%
\pgfsetstrokecolor{currentstroke}%
\pgfsetdash{}{0pt}%
\pgfpathmoveto{\pgfqpoint{4.744022in}{2.094762in}}%
\pgfpathlineto{\pgfqpoint{4.758295in}{2.097536in}}%
\pgfpathlineto{\pgfqpoint{4.772580in}{2.100381in}}%
\pgfpathlineto{\pgfqpoint{4.786876in}{2.103298in}}%
\pgfpathlineto{\pgfqpoint{4.801184in}{2.106287in}}%
\pgfpathlineto{\pgfqpoint{4.809022in}{2.114911in}}%
\pgfpathlineto{\pgfqpoint{4.816853in}{2.123431in}}%
\pgfpathlineto{\pgfqpoint{4.824677in}{2.131850in}}%
\pgfpathlineto{\pgfqpoint{4.832494in}{2.140167in}}%
\pgfpathlineto{\pgfqpoint{4.818197in}{2.137203in}}%
\pgfpathlineto{\pgfqpoint{4.803911in}{2.134311in}}%
\pgfpathlineto{\pgfqpoint{4.789636in}{2.131490in}}%
\pgfpathlineto{\pgfqpoint{4.775373in}{2.128741in}}%
\pgfpathlineto{\pgfqpoint{4.767546in}{2.120391in}}%
\pgfpathlineto{\pgfqpoint{4.759711in}{2.111945in}}%
\pgfpathlineto{\pgfqpoint{4.751870in}{2.103402in}}%
\pgfpathlineto{\pgfqpoint{4.744022in}{2.094762in}}%
\pgfpathclose%
\pgfusepath{fill}%
\end{pgfscope}%
\begin{pgfscope}%
\pgfpathrectangle{\pgfqpoint{1.150000in}{0.150000in}}{\pgfqpoint{5.700000in}{5.700000in}}%
\pgfusepath{clip}%
\pgfsetbuttcap%
\pgfsetroundjoin%
\definecolor{currentfill}{rgb}{0.257322,0.256130,0.526563}%
\pgfsetfillcolor{currentfill}%
\pgfsetfillopacity{0.700000}%
\pgfsetlinewidth{0.000000pt}%
\definecolor{currentstroke}{rgb}{0.000000,0.000000,0.000000}%
\pgfsetstrokecolor{currentstroke}%
\pgfsetdash{}{0pt}%
\pgfpathmoveto{\pgfqpoint{4.655537in}{2.048812in}}%
\pgfpathlineto{\pgfqpoint{4.669775in}{2.051303in}}%
\pgfpathlineto{\pgfqpoint{4.684025in}{2.053867in}}%
\pgfpathlineto{\pgfqpoint{4.698286in}{2.056501in}}%
\pgfpathlineto{\pgfqpoint{4.712559in}{2.059208in}}%
\pgfpathlineto{\pgfqpoint{4.720435in}{2.068246in}}%
\pgfpathlineto{\pgfqpoint{4.728304in}{2.077184in}}%
\pgfpathlineto{\pgfqpoint{4.736166in}{2.086022in}}%
\pgfpathlineto{\pgfqpoint{4.744022in}{2.094762in}}%
\pgfpathlineto{\pgfqpoint{4.729759in}{2.092059in}}%
\pgfpathlineto{\pgfqpoint{4.715508in}{2.089427in}}%
\pgfpathlineto{\pgfqpoint{4.701268in}{2.086867in}}%
\pgfpathlineto{\pgfqpoint{4.687038in}{2.084379in}}%
\pgfpathlineto{\pgfqpoint{4.679173in}{2.075628in}}%
\pgfpathlineto{\pgfqpoint{4.671301in}{2.066784in}}%
\pgfpathlineto{\pgfqpoint{4.663422in}{2.057845in}}%
\pgfpathlineto{\pgfqpoint{4.655537in}{2.048812in}}%
\pgfpathclose%
\pgfusepath{fill}%
\end{pgfscope}%
\begin{pgfscope}%
\pgfpathrectangle{\pgfqpoint{1.150000in}{0.150000in}}{\pgfqpoint{5.700000in}{5.700000in}}%
\pgfusepath{clip}%
\pgfsetbuttcap%
\pgfsetroundjoin%
\definecolor{currentfill}{rgb}{0.279574,0.170599,0.479997}%
\pgfsetfillcolor{currentfill}%
\pgfsetfillopacity{0.700000}%
\pgfsetlinewidth{0.000000pt}%
\definecolor{currentstroke}{rgb}{0.000000,0.000000,0.000000}%
\pgfsetstrokecolor{currentstroke}%
\pgfsetdash{}{0pt}%
\pgfpathmoveto{\pgfqpoint{2.355209in}{1.907402in}}%
\pgfpathlineto{\pgfqpoint{2.369092in}{1.895985in}}%
\pgfpathlineto{\pgfqpoint{2.382974in}{1.884678in}}%
\pgfpathlineto{\pgfqpoint{2.396855in}{1.873480in}}%
\pgfpathlineto{\pgfqpoint{2.410735in}{1.862390in}}%
\pgfpathlineto{\pgfqpoint{2.419751in}{1.861939in}}%
\pgfpathlineto{\pgfqpoint{2.428749in}{1.861759in}}%
\pgfpathlineto{\pgfqpoint{2.437728in}{1.861844in}}%
\pgfpathlineto{\pgfqpoint{2.446690in}{1.862189in}}%
\pgfpathlineto{\pgfqpoint{2.432848in}{1.872878in}}%
\pgfpathlineto{\pgfqpoint{2.419006in}{1.883674in}}%
\pgfpathlineto{\pgfqpoint{2.405162in}{1.894580in}}%
\pgfpathlineto{\pgfqpoint{2.391319in}{1.905595in}}%
\pgfpathlineto{\pgfqpoint{2.382319in}{1.905643in}}%
\pgfpathlineto{\pgfqpoint{2.373301in}{1.905956in}}%
\pgfpathlineto{\pgfqpoint{2.364264in}{1.906540in}}%
\pgfpathlineto{\pgfqpoint{2.355209in}{1.907402in}}%
\pgfpathclose%
\pgfusepath{fill}%
\end{pgfscope}%
\begin{pgfscope}%
\pgfpathrectangle{\pgfqpoint{1.150000in}{0.150000in}}{\pgfqpoint{5.700000in}{5.700000in}}%
\pgfusepath{clip}%
\pgfsetbuttcap%
\pgfsetroundjoin%
\definecolor{currentfill}{rgb}{0.265145,0.232956,0.516599}%
\pgfsetfillcolor{currentfill}%
\pgfsetfillopacity{0.700000}%
\pgfsetlinewidth{0.000000pt}%
\definecolor{currentstroke}{rgb}{0.000000,0.000000,0.000000}%
\pgfsetstrokecolor{currentstroke}%
\pgfsetdash{}{0pt}%
\pgfpathmoveto{\pgfqpoint{4.567043in}{2.002543in}}%
\pgfpathlineto{\pgfqpoint{4.581248in}{2.004730in}}%
\pgfpathlineto{\pgfqpoint{4.595464in}{2.006988in}}%
\pgfpathlineto{\pgfqpoint{4.609690in}{2.009319in}}%
\pgfpathlineto{\pgfqpoint{4.623927in}{2.011721in}}%
\pgfpathlineto{\pgfqpoint{4.631840in}{2.021137in}}%
\pgfpathlineto{\pgfqpoint{4.639745in}{2.030458in}}%
\pgfpathlineto{\pgfqpoint{4.647644in}{2.039683in}}%
\pgfpathlineto{\pgfqpoint{4.655537in}{2.048812in}}%
\pgfpathlineto{\pgfqpoint{4.641308in}{2.046391in}}%
\pgfpathlineto{\pgfqpoint{4.627091in}{2.044043in}}%
\pgfpathlineto{\pgfqpoint{4.612885in}{2.041766in}}%
\pgfpathlineto{\pgfqpoint{4.598689in}{2.039561in}}%
\pgfpathlineto{\pgfqpoint{4.590788in}{2.030442in}}%
\pgfpathlineto{\pgfqpoint{4.582879in}{2.021233in}}%
\pgfpathlineto{\pgfqpoint{4.574965in}{2.011933in}}%
\pgfpathlineto{\pgfqpoint{4.567043in}{2.002543in}}%
\pgfpathclose%
\pgfusepath{fill}%
\end{pgfscope}%
\begin{pgfscope}%
\pgfpathrectangle{\pgfqpoint{1.150000in}{0.150000in}}{\pgfqpoint{5.700000in}{5.700000in}}%
\pgfusepath{clip}%
\pgfsetbuttcap%
\pgfsetroundjoin%
\definecolor{currentfill}{rgb}{0.270595,0.214069,0.507052}%
\pgfsetfillcolor{currentfill}%
\pgfsetfillopacity{0.700000}%
\pgfsetlinewidth{0.000000pt}%
\definecolor{currentstroke}{rgb}{0.000000,0.000000,0.000000}%
\pgfsetstrokecolor{currentstroke}%
\pgfsetdash{}{0pt}%
\pgfpathmoveto{\pgfqpoint{4.478546in}{1.956202in}}%
\pgfpathlineto{\pgfqpoint{4.492718in}{1.958062in}}%
\pgfpathlineto{\pgfqpoint{4.506900in}{1.959994in}}%
\pgfpathlineto{\pgfqpoint{4.521092in}{1.961998in}}%
\pgfpathlineto{\pgfqpoint{4.535295in}{1.964074in}}%
\pgfpathlineto{\pgfqpoint{4.543242in}{1.973827in}}%
\pgfpathlineto{\pgfqpoint{4.551182in}{1.983489in}}%
\pgfpathlineto{\pgfqpoint{4.559116in}{1.993061in}}%
\pgfpathlineto{\pgfqpoint{4.567043in}{2.002543in}}%
\pgfpathlineto{\pgfqpoint{4.552849in}{2.000428in}}%
\pgfpathlineto{\pgfqpoint{4.538666in}{1.998384in}}%
\pgfpathlineto{\pgfqpoint{4.524492in}{1.996413in}}%
\pgfpathlineto{\pgfqpoint{4.510330in}{1.994514in}}%
\pgfpathlineto{\pgfqpoint{4.502393in}{1.985064in}}%
\pgfpathlineto{\pgfqpoint{4.494450in}{1.975529in}}%
\pgfpathlineto{\pgfqpoint{4.486501in}{1.965908in}}%
\pgfpathlineto{\pgfqpoint{4.478546in}{1.956202in}}%
\pgfpathclose%
\pgfusepath{fill}%
\end{pgfscope}%
\begin{pgfscope}%
\pgfpathrectangle{\pgfqpoint{1.150000in}{0.150000in}}{\pgfqpoint{5.700000in}{5.700000in}}%
\pgfusepath{clip}%
\pgfsetbuttcap%
\pgfsetroundjoin%
\definecolor{currentfill}{rgb}{0.275191,0.194905,0.496005}%
\pgfsetfillcolor{currentfill}%
\pgfsetfillopacity{0.700000}%
\pgfsetlinewidth{0.000000pt}%
\definecolor{currentstroke}{rgb}{0.000000,0.000000,0.000000}%
\pgfsetstrokecolor{currentstroke}%
\pgfsetdash{}{0pt}%
\pgfpathmoveto{\pgfqpoint{4.390046in}{1.910058in}}%
\pgfpathlineto{\pgfqpoint{4.404185in}{1.911569in}}%
\pgfpathlineto{\pgfqpoint{4.418335in}{1.913152in}}%
\pgfpathlineto{\pgfqpoint{4.432494in}{1.914807in}}%
\pgfpathlineto{\pgfqpoint{4.446664in}{1.916534in}}%
\pgfpathlineto{\pgfqpoint{4.454643in}{1.926577in}}%
\pgfpathlineto{\pgfqpoint{4.462617in}{1.936536in}}%
\pgfpathlineto{\pgfqpoint{4.470585in}{1.946412in}}%
\pgfpathlineto{\pgfqpoint{4.478546in}{1.956202in}}%
\pgfpathlineto{\pgfqpoint{4.464385in}{1.954414in}}%
\pgfpathlineto{\pgfqpoint{4.450234in}{1.952699in}}%
\pgfpathlineto{\pgfqpoint{4.436093in}{1.951055in}}%
\pgfpathlineto{\pgfqpoint{4.421962in}{1.949484in}}%
\pgfpathlineto{\pgfqpoint{4.413992in}{1.939746in}}%
\pgfpathlineto{\pgfqpoint{4.406016in}{1.929929in}}%
\pgfpathlineto{\pgfqpoint{4.398034in}{1.920033in}}%
\pgfpathlineto{\pgfqpoint{4.390046in}{1.910058in}}%
\pgfpathclose%
\pgfusepath{fill}%
\end{pgfscope}%
\begin{pgfscope}%
\pgfpathrectangle{\pgfqpoint{1.150000in}{0.150000in}}{\pgfqpoint{5.700000in}{5.700000in}}%
\pgfusepath{clip}%
\pgfsetbuttcap%
\pgfsetroundjoin%
\definecolor{currentfill}{rgb}{0.279574,0.170599,0.479997}%
\pgfsetfillcolor{currentfill}%
\pgfsetfillopacity{0.700000}%
\pgfsetlinewidth{0.000000pt}%
\definecolor{currentstroke}{rgb}{0.000000,0.000000,0.000000}%
\pgfsetstrokecolor{currentstroke}%
\pgfsetdash{}{0pt}%
\pgfpathmoveto{\pgfqpoint{4.301544in}{1.864401in}}%
\pgfpathlineto{\pgfqpoint{4.315652in}{1.865540in}}%
\pgfpathlineto{\pgfqpoint{4.329770in}{1.866752in}}%
\pgfpathlineto{\pgfqpoint{4.343897in}{1.868036in}}%
\pgfpathlineto{\pgfqpoint{4.358035in}{1.869393in}}%
\pgfpathlineto{\pgfqpoint{4.366046in}{1.879673in}}%
\pgfpathlineto{\pgfqpoint{4.374052in}{1.889878in}}%
\pgfpathlineto{\pgfqpoint{4.382052in}{1.900006in}}%
\pgfpathlineto{\pgfqpoint{4.390046in}{1.910058in}}%
\pgfpathlineto{\pgfqpoint{4.375917in}{1.908620in}}%
\pgfpathlineto{\pgfqpoint{4.361797in}{1.907254in}}%
\pgfpathlineto{\pgfqpoint{4.347688in}{1.905961in}}%
\pgfpathlineto{\pgfqpoint{4.333588in}{1.904740in}}%
\pgfpathlineto{\pgfqpoint{4.325586in}{1.894762in}}%
\pgfpathlineto{\pgfqpoint{4.317578in}{1.884712in}}%
\pgfpathlineto{\pgfqpoint{4.309564in}{1.874591in}}%
\pgfpathlineto{\pgfqpoint{4.301544in}{1.864401in}}%
\pgfpathclose%
\pgfusepath{fill}%
\end{pgfscope}%
\begin{pgfscope}%
\pgfpathrectangle{\pgfqpoint{1.150000in}{0.150000in}}{\pgfqpoint{5.700000in}{5.700000in}}%
\pgfusepath{clip}%
\pgfsetbuttcap%
\pgfsetroundjoin%
\definecolor{currentfill}{rgb}{0.267004,0.004874,0.329415}%
\pgfsetfillcolor{currentfill}%
\pgfsetfillopacity{0.700000}%
\pgfsetlinewidth{0.000000pt}%
\definecolor{currentstroke}{rgb}{0.000000,0.000000,0.000000}%
\pgfsetstrokecolor{currentstroke}%
\pgfsetdash{}{0pt}%
\pgfpathmoveto{\pgfqpoint{3.159191in}{1.557675in}}%
\pgfpathlineto{\pgfqpoint{3.173037in}{1.552263in}}%
\pgfpathlineto{\pgfqpoint{3.186888in}{1.546935in}}%
\pgfpathlineto{\pgfqpoint{3.200743in}{1.541689in}}%
\pgfpathlineto{\pgfqpoint{3.214602in}{1.536525in}}%
\pgfpathlineto{\pgfqpoint{3.223059in}{1.543530in}}%
\pgfpathlineto{\pgfqpoint{3.231507in}{1.550655in}}%
\pgfpathlineto{\pgfqpoint{3.239946in}{1.557894in}}%
\pgfpathlineto{\pgfqpoint{3.248377in}{1.565243in}}%
\pgfpathlineto{\pgfqpoint{3.234538in}{1.570099in}}%
\pgfpathlineto{\pgfqpoint{3.220703in}{1.575037in}}%
\pgfpathlineto{\pgfqpoint{3.206873in}{1.580058in}}%
\pgfpathlineto{\pgfqpoint{3.193047in}{1.585161in}}%
\pgfpathlineto{\pgfqpoint{3.184597in}{1.578112in}}%
\pgfpathlineto{\pgfqpoint{3.176137in}{1.571178in}}%
\pgfpathlineto{\pgfqpoint{3.167669in}{1.564364in}}%
\pgfpathlineto{\pgfqpoint{3.159191in}{1.557675in}}%
\pgfpathclose%
\pgfusepath{fill}%
\end{pgfscope}%
\begin{pgfscope}%
\pgfpathrectangle{\pgfqpoint{1.150000in}{0.150000in}}{\pgfqpoint{5.700000in}{5.700000in}}%
\pgfusepath{clip}%
\pgfsetbuttcap%
\pgfsetroundjoin%
\definecolor{currentfill}{rgb}{0.182256,0.426184,0.557120}%
\pgfsetfillcolor{currentfill}%
\pgfsetfillopacity{0.700000}%
\pgfsetlinewidth{0.000000pt}%
\definecolor{currentstroke}{rgb}{0.000000,0.000000,0.000000}%
\pgfsetstrokecolor{currentstroke}%
\pgfsetdash{}{0pt}%
\pgfpathmoveto{\pgfqpoint{5.597352in}{2.481471in}}%
\pgfpathlineto{\pgfqpoint{5.611990in}{2.486059in}}%
\pgfpathlineto{\pgfqpoint{5.626642in}{2.490716in}}%
\pgfpathlineto{\pgfqpoint{5.641307in}{2.495444in}}%
\pgfpathlineto{\pgfqpoint{5.648701in}{2.499445in}}%
\pgfpathlineto{\pgfqpoint{5.656087in}{2.503376in}}%
\pgfpathlineto{\pgfqpoint{5.663464in}{2.507242in}}%
\pgfpathlineto{\pgfqpoint{5.670832in}{2.511045in}}%
\pgfpathlineto{\pgfqpoint{5.656189in}{2.506540in}}%
\pgfpathlineto{\pgfqpoint{5.641559in}{2.502104in}}%
\pgfpathlineto{\pgfqpoint{5.626942in}{2.497738in}}%
\pgfpathlineto{\pgfqpoint{5.619557in}{2.493762in}}%
\pgfpathlineto{\pgfqpoint{5.612164in}{2.489728in}}%
\pgfpathlineto{\pgfqpoint{5.604762in}{2.485632in}}%
\pgfpathlineto{\pgfqpoint{5.597352in}{2.481471in}}%
\pgfpathclose%
\pgfusepath{fill}%
\end{pgfscope}%
\begin{pgfscope}%
\pgfpathrectangle{\pgfqpoint{1.150000in}{0.150000in}}{\pgfqpoint{5.700000in}{5.700000in}}%
\pgfusepath{clip}%
\pgfsetbuttcap%
\pgfsetroundjoin%
\definecolor{currentfill}{rgb}{0.281887,0.150881,0.465405}%
\pgfsetfillcolor{currentfill}%
\pgfsetfillopacity{0.700000}%
\pgfsetlinewidth{0.000000pt}%
\definecolor{currentstroke}{rgb}{0.000000,0.000000,0.000000}%
\pgfsetstrokecolor{currentstroke}%
\pgfsetdash{}{0pt}%
\pgfpathmoveto{\pgfqpoint{4.213038in}{1.819539in}}%
\pgfpathlineto{\pgfqpoint{4.227116in}{1.820285in}}%
\pgfpathlineto{\pgfqpoint{4.241203in}{1.821103in}}%
\pgfpathlineto{\pgfqpoint{4.255301in}{1.821994in}}%
\pgfpathlineto{\pgfqpoint{4.269407in}{1.822958in}}%
\pgfpathlineto{\pgfqpoint{4.277450in}{1.833419in}}%
\pgfpathlineto{\pgfqpoint{4.285487in}{1.843813in}}%
\pgfpathlineto{\pgfqpoint{4.293518in}{1.854141in}}%
\pgfpathlineto{\pgfqpoint{4.301544in}{1.864401in}}%
\pgfpathlineto{\pgfqpoint{4.287445in}{1.863334in}}%
\pgfpathlineto{\pgfqpoint{4.273357in}{1.862340in}}%
\pgfpathlineto{\pgfqpoint{4.259277in}{1.861419in}}%
\pgfpathlineto{\pgfqpoint{4.245208in}{1.860571in}}%
\pgfpathlineto{\pgfqpoint{4.237174in}{1.850406in}}%
\pgfpathlineto{\pgfqpoint{4.229134in}{1.840178in}}%
\pgfpathlineto{\pgfqpoint{4.221089in}{1.829889in}}%
\pgfpathlineto{\pgfqpoint{4.213038in}{1.819539in}}%
\pgfpathclose%
\pgfusepath{fill}%
\end{pgfscope}%
\begin{pgfscope}%
\pgfpathrectangle{\pgfqpoint{1.150000in}{0.150000in}}{\pgfqpoint{5.700000in}{5.700000in}}%
\pgfusepath{clip}%
\pgfsetbuttcap%
\pgfsetroundjoin%
\definecolor{currentfill}{rgb}{0.269944,0.014625,0.341379}%
\pgfsetfillcolor{currentfill}%
\pgfsetfillopacity{0.700000}%
\pgfsetlinewidth{0.000000pt}%
\definecolor{currentstroke}{rgb}{0.000000,0.000000,0.000000}%
\pgfsetstrokecolor{currentstroke}%
\pgfsetdash{}{0pt}%
\pgfpathmoveto{\pgfqpoint{3.014379in}{1.581219in}}%
\pgfpathlineto{\pgfqpoint{3.028217in}{1.574800in}}%
\pgfpathlineto{\pgfqpoint{3.042059in}{1.568468in}}%
\pgfpathlineto{\pgfqpoint{3.055904in}{1.562221in}}%
\pgfpathlineto{\pgfqpoint{3.069753in}{1.556060in}}%
\pgfpathlineto{\pgfqpoint{3.078292in}{1.561866in}}%
\pgfpathlineto{\pgfqpoint{3.086821in}{1.567821in}}%
\pgfpathlineto{\pgfqpoint{3.095340in}{1.573921in}}%
\pgfpathlineto{\pgfqpoint{3.103848in}{1.580160in}}%
\pgfpathlineto{\pgfqpoint{3.090023in}{1.585992in}}%
\pgfpathlineto{\pgfqpoint{3.076201in}{1.591909in}}%
\pgfpathlineto{\pgfqpoint{3.062383in}{1.597913in}}%
\pgfpathlineto{\pgfqpoint{3.048568in}{1.604002in}}%
\pgfpathlineto{\pgfqpoint{3.040036in}{1.598084in}}%
\pgfpathlineto{\pgfqpoint{3.031495in}{1.592311in}}%
\pgfpathlineto{\pgfqpoint{3.022942in}{1.586688in}}%
\pgfpathlineto{\pgfqpoint{3.014379in}{1.581219in}}%
\pgfpathclose%
\pgfusepath{fill}%
\end{pgfscope}%
\begin{pgfscope}%
\pgfpathrectangle{\pgfqpoint{1.150000in}{0.150000in}}{\pgfqpoint{5.700000in}{5.700000in}}%
\pgfusepath{clip}%
\pgfsetbuttcap%
\pgfsetroundjoin%
\definecolor{currentfill}{rgb}{0.283072,0.130895,0.449241}%
\pgfsetfillcolor{currentfill}%
\pgfsetfillopacity{0.700000}%
\pgfsetlinewidth{0.000000pt}%
\definecolor{currentstroke}{rgb}{0.000000,0.000000,0.000000}%
\pgfsetstrokecolor{currentstroke}%
\pgfsetdash{}{0pt}%
\pgfpathmoveto{\pgfqpoint{4.124524in}{1.775805in}}%
\pgfpathlineto{\pgfqpoint{4.138574in}{1.776135in}}%
\pgfpathlineto{\pgfqpoint{4.152633in}{1.776538in}}%
\pgfpathlineto{\pgfqpoint{4.166701in}{1.777014in}}%
\pgfpathlineto{\pgfqpoint{4.180778in}{1.777563in}}%
\pgfpathlineto{\pgfqpoint{4.188851in}{1.788141in}}%
\pgfpathlineto{\pgfqpoint{4.196919in}{1.798664in}}%
\pgfpathlineto{\pgfqpoint{4.204981in}{1.809130in}}%
\pgfpathlineto{\pgfqpoint{4.213038in}{1.819539in}}%
\pgfpathlineto{\pgfqpoint{4.198969in}{1.818866in}}%
\pgfpathlineto{\pgfqpoint{4.184909in}{1.818267in}}%
\pgfpathlineto{\pgfqpoint{4.170859in}{1.817740in}}%
\pgfpathlineto{\pgfqpoint{4.156817in}{1.817287in}}%
\pgfpathlineto{\pgfqpoint{4.148752in}{1.806994in}}%
\pgfpathlineto{\pgfqpoint{4.140682in}{1.796648in}}%
\pgfpathlineto{\pgfqpoint{4.132605in}{1.786251in}}%
\pgfpathlineto{\pgfqpoint{4.124524in}{1.775805in}}%
\pgfpathclose%
\pgfusepath{fill}%
\end{pgfscope}%
\begin{pgfscope}%
\pgfpathrectangle{\pgfqpoint{1.150000in}{0.150000in}}{\pgfqpoint{5.700000in}{5.700000in}}%
\pgfusepath{clip}%
\pgfsetbuttcap%
\pgfsetroundjoin%
\definecolor{currentfill}{rgb}{0.283091,0.110553,0.431554}%
\pgfsetfillcolor{currentfill}%
\pgfsetfillopacity{0.700000}%
\pgfsetlinewidth{0.000000pt}%
\definecolor{currentstroke}{rgb}{0.000000,0.000000,0.000000}%
\pgfsetstrokecolor{currentstroke}%
\pgfsetdash{}{0pt}%
\pgfpathmoveto{\pgfqpoint{4.035996in}{1.733551in}}%
\pgfpathlineto{\pgfqpoint{4.050020in}{1.733442in}}%
\pgfpathlineto{\pgfqpoint{4.064052in}{1.733408in}}%
\pgfpathlineto{\pgfqpoint{4.078093in}{1.733447in}}%
\pgfpathlineto{\pgfqpoint{4.092142in}{1.733559in}}%
\pgfpathlineto{\pgfqpoint{4.100246in}{1.744186in}}%
\pgfpathlineto{\pgfqpoint{4.108344in}{1.754771in}}%
\pgfpathlineto{\pgfqpoint{4.116437in}{1.765311in}}%
\pgfpathlineto{\pgfqpoint{4.124524in}{1.775805in}}%
\pgfpathlineto{\pgfqpoint{4.110483in}{1.775549in}}%
\pgfpathlineto{\pgfqpoint{4.096451in}{1.775366in}}%
\pgfpathlineto{\pgfqpoint{4.082427in}{1.775256in}}%
\pgfpathlineto{\pgfqpoint{4.068413in}{1.775221in}}%
\pgfpathlineto{\pgfqpoint{4.060317in}{1.764863in}}%
\pgfpathlineto{\pgfqpoint{4.052215in}{1.754464in}}%
\pgfpathlineto{\pgfqpoint{4.044108in}{1.744026in}}%
\pgfpathlineto{\pgfqpoint{4.035996in}{1.733551in}}%
\pgfpathclose%
\pgfusepath{fill}%
\end{pgfscope}%
\begin{pgfscope}%
\pgfpathrectangle{\pgfqpoint{1.150000in}{0.150000in}}{\pgfqpoint{5.700000in}{5.700000in}}%
\pgfusepath{clip}%
\pgfsetbuttcap%
\pgfsetroundjoin%
\definecolor{currentfill}{rgb}{0.269944,0.014625,0.341379}%
\pgfsetfillcolor{currentfill}%
\pgfsetfillopacity{0.700000}%
\pgfsetlinewidth{0.000000pt}%
\definecolor{currentstroke}{rgb}{0.000000,0.000000,0.000000}%
\pgfsetstrokecolor{currentstroke}%
\pgfsetdash{}{0pt}%
\pgfpathmoveto{\pgfqpoint{3.537119in}{1.569615in}}%
\pgfpathlineto{\pgfqpoint{3.551018in}{1.566668in}}%
\pgfpathlineto{\pgfqpoint{3.564924in}{1.563800in}}%
\pgfpathlineto{\pgfqpoint{3.578837in}{1.561008in}}%
\pgfpathlineto{\pgfqpoint{3.592755in}{1.558293in}}%
\pgfpathlineto{\pgfqpoint{3.601038in}{1.567714in}}%
\pgfpathlineto{\pgfqpoint{3.609314in}{1.577179in}}%
\pgfpathlineto{\pgfqpoint{3.617583in}{1.586683in}}%
\pgfpathlineto{\pgfqpoint{3.625847in}{1.596225in}}%
\pgfpathlineto{\pgfqpoint{3.611941in}{1.598693in}}%
\pgfpathlineto{\pgfqpoint{3.598042in}{1.601239in}}%
\pgfpathlineto{\pgfqpoint{3.584150in}{1.603861in}}%
\pgfpathlineto{\pgfqpoint{3.570265in}{1.606562in}}%
\pgfpathlineto{\pgfqpoint{3.561988in}{1.597258in}}%
\pgfpathlineto{\pgfqpoint{3.553705in}{1.587997in}}%
\pgfpathlineto{\pgfqpoint{3.545415in}{1.578781in}}%
\pgfpathlineto{\pgfqpoint{3.537119in}{1.569615in}}%
\pgfpathclose%
\pgfusepath{fill}%
\end{pgfscope}%
\begin{pgfscope}%
\pgfpathrectangle{\pgfqpoint{1.150000in}{0.150000in}}{\pgfqpoint{5.700000in}{5.700000in}}%
\pgfusepath{clip}%
\pgfsetbuttcap%
\pgfsetroundjoin%
\definecolor{currentfill}{rgb}{0.267004,0.004874,0.329415}%
\pgfsetfillcolor{currentfill}%
\pgfsetfillopacity{0.700000}%
\pgfsetlinewidth{0.000000pt}%
\definecolor{currentstroke}{rgb}{0.000000,0.000000,0.000000}%
\pgfsetstrokecolor{currentstroke}%
\pgfsetdash{}{0pt}%
\pgfpathmoveto{\pgfqpoint{3.303782in}{1.546637in}}%
\pgfpathlineto{\pgfqpoint{3.317646in}{1.542189in}}%
\pgfpathlineto{\pgfqpoint{3.331515in}{1.537821in}}%
\pgfpathlineto{\pgfqpoint{3.345389in}{1.533533in}}%
\pgfpathlineto{\pgfqpoint{3.359268in}{1.529325in}}%
\pgfpathlineto{\pgfqpoint{3.367654in}{1.537370in}}%
\pgfpathlineto{\pgfqpoint{3.376032in}{1.545506in}}%
\pgfpathlineto{\pgfqpoint{3.384402in}{1.553729in}}%
\pgfpathlineto{\pgfqpoint{3.392765in}{1.562034in}}%
\pgfpathlineto{\pgfqpoint{3.378902in}{1.565955in}}%
\pgfpathlineto{\pgfqpoint{3.365046in}{1.569956in}}%
\pgfpathlineto{\pgfqpoint{3.351194in}{1.574037in}}%
\pgfpathlineto{\pgfqpoint{3.337349in}{1.578198in}}%
\pgfpathlineto{\pgfqpoint{3.328969in}{1.570172in}}%
\pgfpathlineto{\pgfqpoint{3.320581in}{1.562234in}}%
\pgfpathlineto{\pgfqpoint{3.312186in}{1.554388in}}%
\pgfpathlineto{\pgfqpoint{3.303782in}{1.546637in}}%
\pgfpathclose%
\pgfusepath{fill}%
\end{pgfscope}%
\begin{pgfscope}%
\pgfpathrectangle{\pgfqpoint{1.150000in}{0.150000in}}{\pgfqpoint{5.700000in}{5.700000in}}%
\pgfusepath{clip}%
\pgfsetbuttcap%
\pgfsetroundjoin%
\definecolor{currentfill}{rgb}{0.281924,0.089666,0.412415}%
\pgfsetfillcolor{currentfill}%
\pgfsetfillopacity{0.700000}%
\pgfsetlinewidth{0.000000pt}%
\definecolor{currentstroke}{rgb}{0.000000,0.000000,0.000000}%
\pgfsetstrokecolor{currentstroke}%
\pgfsetdash{}{0pt}%
\pgfpathmoveto{\pgfqpoint{3.947446in}{1.693150in}}%
\pgfpathlineto{\pgfqpoint{3.961445in}{1.692581in}}%
\pgfpathlineto{\pgfqpoint{3.975452in}{1.692086in}}%
\pgfpathlineto{\pgfqpoint{3.989468in}{1.691666in}}%
\pgfpathlineto{\pgfqpoint{4.003492in}{1.691319in}}%
\pgfpathlineto{\pgfqpoint{4.011626in}{1.701922in}}%
\pgfpathlineto{\pgfqpoint{4.019755in}{1.712497in}}%
\pgfpathlineto{\pgfqpoint{4.027878in}{1.723040in}}%
\pgfpathlineto{\pgfqpoint{4.035996in}{1.733551in}}%
\pgfpathlineto{\pgfqpoint{4.021981in}{1.733733in}}%
\pgfpathlineto{\pgfqpoint{4.007975in}{1.733989in}}%
\pgfpathlineto{\pgfqpoint{3.993977in}{1.734319in}}%
\pgfpathlineto{\pgfqpoint{3.979987in}{1.734723in}}%
\pgfpathlineto{\pgfqpoint{3.971860in}{1.724369in}}%
\pgfpathlineto{\pgfqpoint{3.963728in}{1.713988in}}%
\pgfpathlineto{\pgfqpoint{3.955590in}{1.703581in}}%
\pgfpathlineto{\pgfqpoint{3.947446in}{1.693150in}}%
\pgfpathclose%
\pgfusepath{fill}%
\end{pgfscope}%
\begin{pgfscope}%
\pgfpathrectangle{\pgfqpoint{1.150000in}{0.150000in}}{\pgfqpoint{5.700000in}{5.700000in}}%
\pgfusepath{clip}%
\pgfsetbuttcap%
\pgfsetroundjoin%
\definecolor{currentfill}{rgb}{0.280894,0.078907,0.402329}%
\pgfsetfillcolor{currentfill}%
\pgfsetfillopacity{0.700000}%
\pgfsetlinewidth{0.000000pt}%
\definecolor{currentstroke}{rgb}{0.000000,0.000000,0.000000}%
\pgfsetstrokecolor{currentstroke}%
\pgfsetdash{}{0pt}%
\pgfpathmoveto{\pgfqpoint{2.668167in}{1.705261in}}%
\pgfpathlineto{\pgfqpoint{2.682015in}{1.696300in}}%
\pgfpathlineto{\pgfqpoint{2.695864in}{1.687435in}}%
\pgfpathlineto{\pgfqpoint{2.709715in}{1.678665in}}%
\pgfpathlineto{\pgfqpoint{2.723567in}{1.669990in}}%
\pgfpathlineto{\pgfqpoint{2.732341in}{1.672508in}}%
\pgfpathlineto{\pgfqpoint{2.741101in}{1.675246in}}%
\pgfpathlineto{\pgfqpoint{2.749846in}{1.678198in}}%
\pgfpathlineto{\pgfqpoint{2.758578in}{1.681358in}}%
\pgfpathlineto{\pgfqpoint{2.744757in}{1.689659in}}%
\pgfpathlineto{\pgfqpoint{2.730938in}{1.698055in}}%
\pgfpathlineto{\pgfqpoint{2.717120in}{1.706546in}}%
\pgfpathlineto{\pgfqpoint{2.703304in}{1.715132in}}%
\pgfpathlineto{\pgfqpoint{2.694541in}{1.712339in}}%
\pgfpathlineto{\pgfqpoint{2.685764in}{1.709758in}}%
\pgfpathlineto{\pgfqpoint{2.676973in}{1.707397in}}%
\pgfpathlineto{\pgfqpoint{2.668167in}{1.705261in}}%
\pgfpathclose%
\pgfusepath{fill}%
\end{pgfscope}%
\begin{pgfscope}%
\pgfpathrectangle{\pgfqpoint{1.150000in}{0.150000in}}{\pgfqpoint{5.700000in}{5.700000in}}%
\pgfusepath{clip}%
\pgfsetbuttcap%
\pgfsetroundjoin%
\definecolor{currentfill}{rgb}{0.281887,0.150881,0.465405}%
\pgfsetfillcolor{currentfill}%
\pgfsetfillopacity{0.700000}%
\pgfsetlinewidth{0.000000pt}%
\definecolor{currentstroke}{rgb}{0.000000,0.000000,0.000000}%
\pgfsetstrokecolor{currentstroke}%
\pgfsetdash{}{0pt}%
\pgfpathmoveto{\pgfqpoint{2.410735in}{1.862390in}}%
\pgfpathlineto{\pgfqpoint{2.424614in}{1.851408in}}%
\pgfpathlineto{\pgfqpoint{2.438493in}{1.840533in}}%
\pgfpathlineto{\pgfqpoint{2.452371in}{1.829763in}}%
\pgfpathlineto{\pgfqpoint{2.466249in}{1.819099in}}%
\pgfpathlineto{\pgfqpoint{2.475226in}{1.819056in}}%
\pgfpathlineto{\pgfqpoint{2.484186in}{1.819279in}}%
\pgfpathlineto{\pgfqpoint{2.493129in}{1.819762in}}%
\pgfpathlineto{\pgfqpoint{2.502054in}{1.820498in}}%
\pgfpathlineto{\pgfqpoint{2.488213in}{1.830762in}}%
\pgfpathlineto{\pgfqpoint{2.474372in}{1.841132in}}%
\pgfpathlineto{\pgfqpoint{2.460531in}{1.851607in}}%
\pgfpathlineto{\pgfqpoint{2.446690in}{1.862189in}}%
\pgfpathlineto{\pgfqpoint{2.437728in}{1.861844in}}%
\pgfpathlineto{\pgfqpoint{2.428749in}{1.861759in}}%
\pgfpathlineto{\pgfqpoint{2.419751in}{1.861939in}}%
\pgfpathlineto{\pgfqpoint{2.410735in}{1.862390in}}%
\pgfpathclose%
\pgfusepath{fill}%
\end{pgfscope}%
\begin{pgfscope}%
\pgfpathrectangle{\pgfqpoint{1.150000in}{0.150000in}}{\pgfqpoint{5.700000in}{5.700000in}}%
\pgfusepath{clip}%
\pgfsetbuttcap%
\pgfsetroundjoin%
\definecolor{currentfill}{rgb}{0.273809,0.031497,0.358853}%
\pgfsetfillcolor{currentfill}%
\pgfsetfillopacity{0.700000}%
\pgfsetlinewidth{0.000000pt}%
\definecolor{currentstroke}{rgb}{0.000000,0.000000,0.000000}%
\pgfsetstrokecolor{currentstroke}%
\pgfsetdash{}{0pt}%
\pgfpathmoveto{\pgfqpoint{2.869218in}{1.618292in}}%
\pgfpathlineto{\pgfqpoint{2.883058in}{1.610819in}}%
\pgfpathlineto{\pgfqpoint{2.896901in}{1.603435in}}%
\pgfpathlineto{\pgfqpoint{2.910747in}{1.596141in}}%
\pgfpathlineto{\pgfqpoint{2.924595in}{1.588936in}}%
\pgfpathlineto{\pgfqpoint{2.933229in}{1.593376in}}%
\pgfpathlineto{\pgfqpoint{2.941851in}{1.597997in}}%
\pgfpathlineto{\pgfqpoint{2.950461in}{1.602794in}}%
\pgfpathlineto{\pgfqpoint{2.959060in}{1.607761in}}%
\pgfpathlineto{\pgfqpoint{2.945238in}{1.614615in}}%
\pgfpathlineto{\pgfqpoint{2.931419in}{1.621559in}}%
\pgfpathlineto{\pgfqpoint{2.917603in}{1.628591in}}%
\pgfpathlineto{\pgfqpoint{2.903790in}{1.635713in}}%
\pgfpathlineto{\pgfqpoint{2.895165in}{1.631089in}}%
\pgfpathlineto{\pgfqpoint{2.886528in}{1.626641in}}%
\pgfpathlineto{\pgfqpoint{2.877879in}{1.622373in}}%
\pgfpathlineto{\pgfqpoint{2.869218in}{1.618292in}}%
\pgfpathclose%
\pgfusepath{fill}%
\end{pgfscope}%
\begin{pgfscope}%
\pgfpathrectangle{\pgfqpoint{1.150000in}{0.150000in}}{\pgfqpoint{5.700000in}{5.700000in}}%
\pgfusepath{clip}%
\pgfsetbuttcap%
\pgfsetroundjoin%
\definecolor{currentfill}{rgb}{0.212395,0.359683,0.551710}%
\pgfsetfillcolor{currentfill}%
\pgfsetfillopacity{0.700000}%
\pgfsetlinewidth{0.000000pt}%
\definecolor{currentstroke}{rgb}{0.000000,0.000000,0.000000}%
\pgfsetstrokecolor{currentstroke}%
\pgfsetdash{}{0pt}%
\pgfpathmoveto{\pgfqpoint{1.871720in}{2.356493in}}%
\pgfpathlineto{\pgfqpoint{1.885758in}{2.340606in}}%
\pgfpathlineto{\pgfqpoint{1.899789in}{2.324865in}}%
\pgfpathlineto{\pgfqpoint{1.913815in}{2.309266in}}%
\pgfpathlineto{\pgfqpoint{1.927835in}{2.293809in}}%
\pgfpathlineto{\pgfqpoint{1.937300in}{2.288691in}}%
\pgfpathlineto{\pgfqpoint{1.946739in}{2.283919in}}%
\pgfpathlineto{\pgfqpoint{1.956153in}{2.279485in}}%
\pgfpathlineto{\pgfqpoint{1.965542in}{2.275384in}}%
\pgfpathlineto{\pgfqpoint{1.951573in}{2.290402in}}%
\pgfpathlineto{\pgfqpoint{1.937598in}{2.305561in}}%
\pgfpathlineto{\pgfqpoint{1.923617in}{2.320862in}}%
\pgfpathlineto{\pgfqpoint{1.909631in}{2.336308in}}%
\pgfpathlineto{\pgfqpoint{1.900192in}{2.340840in}}%
\pgfpathlineto{\pgfqpoint{1.890728in}{2.345710in}}%
\pgfpathlineto{\pgfqpoint{1.881237in}{2.350925in}}%
\pgfpathlineto{\pgfqpoint{1.871720in}{2.356493in}}%
\pgfpathclose%
\pgfusepath{fill}%
\end{pgfscope}%
\begin{pgfscope}%
\pgfpathrectangle{\pgfqpoint{1.150000in}{0.150000in}}{\pgfqpoint{5.700000in}{5.700000in}}%
\pgfusepath{clip}%
\pgfsetbuttcap%
\pgfsetroundjoin%
\definecolor{currentfill}{rgb}{0.280267,0.073417,0.397163}%
\pgfsetfillcolor{currentfill}%
\pgfsetfillopacity{0.700000}%
\pgfsetlinewidth{0.000000pt}%
\definecolor{currentstroke}{rgb}{0.000000,0.000000,0.000000}%
\pgfsetstrokecolor{currentstroke}%
\pgfsetdash{}{0pt}%
\pgfpathmoveto{\pgfqpoint{3.858863in}{1.654996in}}%
\pgfpathlineto{\pgfqpoint{3.872839in}{1.653944in}}%
\pgfpathlineto{\pgfqpoint{3.886824in}{1.652967in}}%
\pgfpathlineto{\pgfqpoint{3.900816in}{1.652065in}}%
\pgfpathlineto{\pgfqpoint{3.914817in}{1.651236in}}%
\pgfpathlineto{\pgfqpoint{3.922982in}{1.661738in}}%
\pgfpathlineto{\pgfqpoint{3.931142in}{1.672226in}}%
\pgfpathlineto{\pgfqpoint{3.939297in}{1.682697in}}%
\pgfpathlineto{\pgfqpoint{3.947446in}{1.693150in}}%
\pgfpathlineto{\pgfqpoint{3.933455in}{1.693793in}}%
\pgfpathlineto{\pgfqpoint{3.919473in}{1.694510in}}%
\pgfpathlineto{\pgfqpoint{3.905498in}{1.695302in}}%
\pgfpathlineto{\pgfqpoint{3.891532in}{1.696169in}}%
\pgfpathlineto{\pgfqpoint{3.883373in}{1.685894in}}%
\pgfpathlineto{\pgfqpoint{3.875208in}{1.675605in}}%
\pgfpathlineto{\pgfqpoint{3.867038in}{1.665305in}}%
\pgfpathlineto{\pgfqpoint{3.858863in}{1.654996in}}%
\pgfpathclose%
\pgfusepath{fill}%
\end{pgfscope}%
\begin{pgfscope}%
\pgfpathrectangle{\pgfqpoint{1.150000in}{0.150000in}}{\pgfqpoint{5.700000in}{5.700000in}}%
\pgfusepath{clip}%
\pgfsetbuttcap%
\pgfsetroundjoin%
\definecolor{currentfill}{rgb}{0.187231,0.414746,0.556547}%
\pgfsetfillcolor{currentfill}%
\pgfsetfillopacity{0.700000}%
\pgfsetlinewidth{0.000000pt}%
\definecolor{currentstroke}{rgb}{0.000000,0.000000,0.000000}%
\pgfsetstrokecolor{currentstroke}%
\pgfsetdash{}{0pt}%
\pgfpathmoveto{\pgfqpoint{5.509125in}{2.445636in}}%
\pgfpathlineto{\pgfqpoint{5.523729in}{2.450144in}}%
\pgfpathlineto{\pgfqpoint{5.538348in}{2.454722in}}%
\pgfpathlineto{\pgfqpoint{5.552979in}{2.459370in}}%
\pgfpathlineto{\pgfqpoint{5.567624in}{2.464089in}}%
\pgfpathlineto{\pgfqpoint{5.575069in}{2.468552in}}%
\pgfpathlineto{\pgfqpoint{5.582506in}{2.472934in}}%
\pgfpathlineto{\pgfqpoint{5.589933in}{2.477239in}}%
\pgfpathlineto{\pgfqpoint{5.597352in}{2.481471in}}%
\pgfpathlineto{\pgfqpoint{5.582727in}{2.476953in}}%
\pgfpathlineto{\pgfqpoint{5.568116in}{2.472505in}}%
\pgfpathlineto{\pgfqpoint{5.553517in}{2.468127in}}%
\pgfpathlineto{\pgfqpoint{5.538933in}{2.463819in}}%
\pgfpathlineto{\pgfqpoint{5.531493in}{2.459379in}}%
\pgfpathlineto{\pgfqpoint{5.524046in}{2.454871in}}%
\pgfpathlineto{\pgfqpoint{5.516589in}{2.450291in}}%
\pgfpathlineto{\pgfqpoint{5.509125in}{2.445636in}}%
\pgfpathclose%
\pgfusepath{fill}%
\end{pgfscope}%
\begin{pgfscope}%
\pgfpathrectangle{\pgfqpoint{1.150000in}{0.150000in}}{\pgfqpoint{5.700000in}{5.700000in}}%
\pgfusepath{clip}%
\pgfsetbuttcap%
\pgfsetroundjoin%
\definecolor{currentfill}{rgb}{0.223925,0.334994,0.548053}%
\pgfsetfillcolor{currentfill}%
\pgfsetfillopacity{0.700000}%
\pgfsetlinewidth{0.000000pt}%
\definecolor{currentstroke}{rgb}{0.000000,0.000000,0.000000}%
\pgfsetstrokecolor{currentstroke}%
\pgfsetdash{}{0pt}%
\pgfpathmoveto{\pgfqpoint{1.927835in}{2.293809in}}%
\pgfpathlineto{\pgfqpoint{1.941850in}{2.278493in}}%
\pgfpathlineto{\pgfqpoint{1.955860in}{2.263315in}}%
\pgfpathlineto{\pgfqpoint{1.969865in}{2.248276in}}%
\pgfpathlineto{\pgfqpoint{1.983865in}{2.233373in}}%
\pgfpathlineto{\pgfqpoint{1.993279in}{2.228701in}}%
\pgfpathlineto{\pgfqpoint{2.002668in}{2.224370in}}%
\pgfpathlineto{\pgfqpoint{2.012033in}{2.220370in}}%
\pgfpathlineto{\pgfqpoint{2.021374in}{2.216697in}}%
\pgfpathlineto{\pgfqpoint{2.007423in}{2.231163in}}%
\pgfpathlineto{\pgfqpoint{1.993467in}{2.245766in}}%
\pgfpathlineto{\pgfqpoint{1.979507in}{2.260505in}}%
\pgfpathlineto{\pgfqpoint{1.965542in}{2.275384in}}%
\pgfpathlineto{\pgfqpoint{1.956153in}{2.279485in}}%
\pgfpathlineto{\pgfqpoint{1.946739in}{2.283919in}}%
\pgfpathlineto{\pgfqpoint{1.937300in}{2.288691in}}%
\pgfpathlineto{\pgfqpoint{1.927835in}{2.293809in}}%
\pgfpathclose%
\pgfusepath{fill}%
\end{pgfscope}%
\begin{pgfscope}%
\pgfpathrectangle{\pgfqpoint{1.150000in}{0.150000in}}{\pgfqpoint{5.700000in}{5.700000in}}%
\pgfusepath{clip}%
\pgfsetbuttcap%
\pgfsetroundjoin%
\definecolor{currentfill}{rgb}{0.233603,0.313828,0.543914}%
\pgfsetfillcolor{currentfill}%
\pgfsetfillopacity{0.700000}%
\pgfsetlinewidth{0.000000pt}%
\definecolor{currentstroke}{rgb}{0.000000,0.000000,0.000000}%
\pgfsetstrokecolor{currentstroke}%
\pgfsetdash{}{0pt}%
\pgfpathmoveto{\pgfqpoint{1.983865in}{2.233373in}}%
\pgfpathlineto{\pgfqpoint{1.997860in}{2.218606in}}%
\pgfpathlineto{\pgfqpoint{2.011851in}{2.203972in}}%
\pgfpathlineto{\pgfqpoint{2.025837in}{2.189472in}}%
\pgfpathlineto{\pgfqpoint{2.039819in}{2.175103in}}%
\pgfpathlineto{\pgfqpoint{2.049184in}{2.170876in}}%
\pgfpathlineto{\pgfqpoint{2.058525in}{2.166982in}}%
\pgfpathlineto{\pgfqpoint{2.067841in}{2.163415in}}%
\pgfpathlineto{\pgfqpoint{2.077135in}{2.160168in}}%
\pgfpathlineto{\pgfqpoint{2.063200in}{2.174102in}}%
\pgfpathlineto{\pgfqpoint{2.049262in}{2.188168in}}%
\pgfpathlineto{\pgfqpoint{2.035320in}{2.202366in}}%
\pgfpathlineto{\pgfqpoint{2.021374in}{2.216697in}}%
\pgfpathlineto{\pgfqpoint{2.012033in}{2.220370in}}%
\pgfpathlineto{\pgfqpoint{2.002668in}{2.224370in}}%
\pgfpathlineto{\pgfqpoint{1.993279in}{2.228701in}}%
\pgfpathlineto{\pgfqpoint{1.983865in}{2.233373in}}%
\pgfpathclose%
\pgfusepath{fill}%
\end{pgfscope}%
\begin{pgfscope}%
\pgfpathrectangle{\pgfqpoint{1.150000in}{0.150000in}}{\pgfqpoint{5.700000in}{5.700000in}}%
\pgfusepath{clip}%
\pgfsetbuttcap%
\pgfsetroundjoin%
\definecolor{currentfill}{rgb}{0.268510,0.009605,0.335427}%
\pgfsetfillcolor{currentfill}%
\pgfsetfillopacity{0.700000}%
\pgfsetlinewidth{0.000000pt}%
\definecolor{currentstroke}{rgb}{0.000000,0.000000,0.000000}%
\pgfsetstrokecolor{currentstroke}%
\pgfsetdash{}{0pt}%
\pgfpathmoveto{\pgfqpoint{3.448269in}{1.547145in}}%
\pgfpathlineto{\pgfqpoint{3.462159in}{1.543620in}}%
\pgfpathlineto{\pgfqpoint{3.476055in}{1.540173in}}%
\pgfpathlineto{\pgfqpoint{3.489958in}{1.536804in}}%
\pgfpathlineto{\pgfqpoint{3.503866in}{1.533514in}}%
\pgfpathlineto{\pgfqpoint{3.512189in}{1.542446in}}%
\pgfpathlineto{\pgfqpoint{3.520506in}{1.551443in}}%
\pgfpathlineto{\pgfqpoint{3.528816in}{1.560501in}}%
\pgfpathlineto{\pgfqpoint{3.537119in}{1.569615in}}%
\pgfpathlineto{\pgfqpoint{3.523225in}{1.572639in}}%
\pgfpathlineto{\pgfqpoint{3.509338in}{1.575741in}}%
\pgfpathlineto{\pgfqpoint{3.495457in}{1.578921in}}%
\pgfpathlineto{\pgfqpoint{3.481582in}{1.582179in}}%
\pgfpathlineto{\pgfqpoint{3.473264in}{1.573324in}}%
\pgfpathlineto{\pgfqpoint{3.464939in}{1.564531in}}%
\pgfpathlineto{\pgfqpoint{3.456608in}{1.555803in}}%
\pgfpathlineto{\pgfqpoint{3.448269in}{1.547145in}}%
\pgfpathclose%
\pgfusepath{fill}%
\end{pgfscope}%
\begin{pgfscope}%
\pgfpathrectangle{\pgfqpoint{1.150000in}{0.150000in}}{\pgfqpoint{5.700000in}{5.700000in}}%
\pgfusepath{clip}%
\pgfsetbuttcap%
\pgfsetroundjoin%
\definecolor{currentfill}{rgb}{0.277018,0.050344,0.375715}%
\pgfsetfillcolor{currentfill}%
\pgfsetfillopacity{0.700000}%
\pgfsetlinewidth{0.000000pt}%
\definecolor{currentstroke}{rgb}{0.000000,0.000000,0.000000}%
\pgfsetstrokecolor{currentstroke}%
\pgfsetdash{}{0pt}%
\pgfpathmoveto{\pgfqpoint{3.770232in}{1.619507in}}%
\pgfpathlineto{\pgfqpoint{3.784188in}{1.617949in}}%
\pgfpathlineto{\pgfqpoint{3.798153in}{1.616467in}}%
\pgfpathlineto{\pgfqpoint{3.812125in}{1.615060in}}%
\pgfpathlineto{\pgfqpoint{3.826104in}{1.613728in}}%
\pgfpathlineto{\pgfqpoint{3.834302in}{1.624044in}}%
\pgfpathlineto{\pgfqpoint{3.842495in}{1.634363in}}%
\pgfpathlineto{\pgfqpoint{3.850681in}{1.644681in}}%
\pgfpathlineto{\pgfqpoint{3.858863in}{1.654996in}}%
\pgfpathlineto{\pgfqpoint{3.844894in}{1.656123in}}%
\pgfpathlineto{\pgfqpoint{3.830933in}{1.657324in}}%
\pgfpathlineto{\pgfqpoint{3.816979in}{1.658601in}}%
\pgfpathlineto{\pgfqpoint{3.803034in}{1.659953in}}%
\pgfpathlineto{\pgfqpoint{3.794842in}{1.649836in}}%
\pgfpathlineto{\pgfqpoint{3.786644in}{1.639720in}}%
\pgfpathlineto{\pgfqpoint{3.778441in}{1.629610in}}%
\pgfpathlineto{\pgfqpoint{3.770232in}{1.619507in}}%
\pgfpathclose%
\pgfusepath{fill}%
\end{pgfscope}%
\begin{pgfscope}%
\pgfpathrectangle{\pgfqpoint{1.150000in}{0.150000in}}{\pgfqpoint{5.700000in}{5.700000in}}%
\pgfusepath{clip}%
\pgfsetbuttcap%
\pgfsetroundjoin%
\definecolor{currentfill}{rgb}{0.192357,0.403199,0.555836}%
\pgfsetfillcolor{currentfill}%
\pgfsetfillopacity{0.700000}%
\pgfsetlinewidth{0.000000pt}%
\definecolor{currentstroke}{rgb}{0.000000,0.000000,0.000000}%
\pgfsetstrokecolor{currentstroke}%
\pgfsetdash{}{0pt}%
\pgfpathmoveto{\pgfqpoint{5.420816in}{2.408142in}}%
\pgfpathlineto{\pgfqpoint{5.435387in}{2.412548in}}%
\pgfpathlineto{\pgfqpoint{5.449971in}{2.417024in}}%
\pgfpathlineto{\pgfqpoint{5.464568in}{2.421570in}}%
\pgfpathlineto{\pgfqpoint{5.479178in}{2.426187in}}%
\pgfpathlineto{\pgfqpoint{5.486678in}{2.431180in}}%
\pgfpathlineto{\pgfqpoint{5.494169in}{2.436083in}}%
\pgfpathlineto{\pgfqpoint{5.501651in}{2.440901in}}%
\pgfpathlineto{\pgfqpoint{5.509125in}{2.445636in}}%
\pgfpathlineto{\pgfqpoint{5.494533in}{2.441198in}}%
\pgfpathlineto{\pgfqpoint{5.479955in}{2.436830in}}%
\pgfpathlineto{\pgfqpoint{5.465389in}{2.432532in}}%
\pgfpathlineto{\pgfqpoint{5.450837in}{2.428304in}}%
\pgfpathlineto{\pgfqpoint{5.443345in}{2.423383in}}%
\pgfpathlineto{\pgfqpoint{5.435844in}{2.418384in}}%
\pgfpathlineto{\pgfqpoint{5.428334in}{2.413305in}}%
\pgfpathlineto{\pgfqpoint{5.420816in}{2.408142in}}%
\pgfpathclose%
\pgfusepath{fill}%
\end{pgfscope}%
\begin{pgfscope}%
\pgfpathrectangle{\pgfqpoint{1.150000in}{0.150000in}}{\pgfqpoint{5.700000in}{5.700000in}}%
\pgfusepath{clip}%
\pgfsetbuttcap%
\pgfsetroundjoin%
\definecolor{currentfill}{rgb}{0.243113,0.292092,0.538516}%
\pgfsetfillcolor{currentfill}%
\pgfsetfillopacity{0.700000}%
\pgfsetlinewidth{0.000000pt}%
\definecolor{currentstroke}{rgb}{0.000000,0.000000,0.000000}%
\pgfsetstrokecolor{currentstroke}%
\pgfsetdash{}{0pt}%
\pgfpathmoveto{\pgfqpoint{2.039819in}{2.175103in}}%
\pgfpathlineto{\pgfqpoint{2.053798in}{2.160865in}}%
\pgfpathlineto{\pgfqpoint{2.067772in}{2.146757in}}%
\pgfpathlineto{\pgfqpoint{2.081742in}{2.132777in}}%
\pgfpathlineto{\pgfqpoint{2.095709in}{2.118924in}}%
\pgfpathlineto{\pgfqpoint{2.105025in}{2.115139in}}%
\pgfpathlineto{\pgfqpoint{2.114318in}{2.111681in}}%
\pgfpathlineto{\pgfqpoint{2.123588in}{2.108543in}}%
\pgfpathlineto{\pgfqpoint{2.132835in}{2.105720in}}%
\pgfpathlineto{\pgfqpoint{2.118915in}{2.119141in}}%
\pgfpathlineto{\pgfqpoint{2.104992in}{2.132688in}}%
\pgfpathlineto{\pgfqpoint{2.091065in}{2.146363in}}%
\pgfpathlineto{\pgfqpoint{2.077135in}{2.160168in}}%
\pgfpathlineto{\pgfqpoint{2.067841in}{2.163415in}}%
\pgfpathlineto{\pgfqpoint{2.058525in}{2.166982in}}%
\pgfpathlineto{\pgfqpoint{2.049184in}{2.170876in}}%
\pgfpathlineto{\pgfqpoint{2.039819in}{2.175103in}}%
\pgfpathclose%
\pgfusepath{fill}%
\end{pgfscope}%
\begin{pgfscope}%
\pgfpathrectangle{\pgfqpoint{1.150000in}{0.150000in}}{\pgfqpoint{5.700000in}{5.700000in}}%
\pgfusepath{clip}%
\pgfsetbuttcap%
\pgfsetroundjoin%
\definecolor{currentfill}{rgb}{0.282884,0.135920,0.453427}%
\pgfsetfillcolor{currentfill}%
\pgfsetfillopacity{0.700000}%
\pgfsetlinewidth{0.000000pt}%
\definecolor{currentstroke}{rgb}{0.000000,0.000000,0.000000}%
\pgfsetstrokecolor{currentstroke}%
\pgfsetdash{}{0pt}%
\pgfpathmoveto{\pgfqpoint{2.466249in}{1.819099in}}%
\pgfpathlineto{\pgfqpoint{2.480126in}{1.808539in}}%
\pgfpathlineto{\pgfqpoint{2.494004in}{1.798083in}}%
\pgfpathlineto{\pgfqpoint{2.507881in}{1.787731in}}%
\pgfpathlineto{\pgfqpoint{2.521758in}{1.777480in}}%
\pgfpathlineto{\pgfqpoint{2.530698in}{1.777845in}}%
\pgfpathlineto{\pgfqpoint{2.539621in}{1.778469in}}%
\pgfpathlineto{\pgfqpoint{2.548527in}{1.779348in}}%
\pgfpathlineto{\pgfqpoint{2.557417in}{1.780475in}}%
\pgfpathlineto{\pgfqpoint{2.543576in}{1.790327in}}%
\pgfpathlineto{\pgfqpoint{2.529735in}{1.800281in}}%
\pgfpathlineto{\pgfqpoint{2.515894in}{1.810338in}}%
\pgfpathlineto{\pgfqpoint{2.502054in}{1.820498in}}%
\pgfpathlineto{\pgfqpoint{2.493129in}{1.819762in}}%
\pgfpathlineto{\pgfqpoint{2.484186in}{1.819279in}}%
\pgfpathlineto{\pgfqpoint{2.475226in}{1.819056in}}%
\pgfpathlineto{\pgfqpoint{2.466249in}{1.819099in}}%
\pgfpathclose%
\pgfusepath{fill}%
\end{pgfscope}%
\begin{pgfscope}%
\pgfpathrectangle{\pgfqpoint{1.150000in}{0.150000in}}{\pgfqpoint{5.700000in}{5.700000in}}%
\pgfusepath{clip}%
\pgfsetbuttcap%
\pgfsetroundjoin%
\definecolor{currentfill}{rgb}{0.197636,0.391528,0.554969}%
\pgfsetfillcolor{currentfill}%
\pgfsetfillopacity{0.700000}%
\pgfsetlinewidth{0.000000pt}%
\definecolor{currentstroke}{rgb}{0.000000,0.000000,0.000000}%
\pgfsetstrokecolor{currentstroke}%
\pgfsetdash{}{0pt}%
\pgfpathmoveto{\pgfqpoint{5.332437in}{2.369040in}}%
\pgfpathlineto{\pgfqpoint{5.346973in}{2.373321in}}%
\pgfpathlineto{\pgfqpoint{5.361521in}{2.377672in}}%
\pgfpathlineto{\pgfqpoint{5.376083in}{2.382094in}}%
\pgfpathlineto{\pgfqpoint{5.390658in}{2.386587in}}%
\pgfpathlineto{\pgfqpoint{5.398210in}{2.392117in}}%
\pgfpathlineto{\pgfqpoint{5.405754in}{2.397551in}}%
\pgfpathlineto{\pgfqpoint{5.413290in}{2.402891in}}%
\pgfpathlineto{\pgfqpoint{5.420816in}{2.408142in}}%
\pgfpathlineto{\pgfqpoint{5.406259in}{2.403806in}}%
\pgfpathlineto{\pgfqpoint{5.391714in}{2.399541in}}%
\pgfpathlineto{\pgfqpoint{5.377183in}{2.395346in}}%
\pgfpathlineto{\pgfqpoint{5.362664in}{2.391221in}}%
\pgfpathlineto{\pgfqpoint{5.355120in}{2.385806in}}%
\pgfpathlineto{\pgfqpoint{5.347567in}{2.380306in}}%
\pgfpathlineto{\pgfqpoint{5.340006in}{2.374718in}}%
\pgfpathlineto{\pgfqpoint{5.332437in}{2.369040in}}%
\pgfpathclose%
\pgfusepath{fill}%
\end{pgfscope}%
\begin{pgfscope}%
\pgfpathrectangle{\pgfqpoint{1.150000in}{0.150000in}}{\pgfqpoint{5.700000in}{5.700000in}}%
\pgfusepath{clip}%
\pgfsetbuttcap%
\pgfsetroundjoin%
\definecolor{currentfill}{rgb}{0.268510,0.009605,0.335427}%
\pgfsetfillcolor{currentfill}%
\pgfsetfillopacity{0.700000}%
\pgfsetlinewidth{0.000000pt}%
\definecolor{currentstroke}{rgb}{0.000000,0.000000,0.000000}%
\pgfsetstrokecolor{currentstroke}%
\pgfsetdash{}{0pt}%
\pgfpathmoveto{\pgfqpoint{3.069753in}{1.556060in}}%
\pgfpathlineto{\pgfqpoint{3.083605in}{1.549984in}}%
\pgfpathlineto{\pgfqpoint{3.097461in}{1.543992in}}%
\pgfpathlineto{\pgfqpoint{3.111321in}{1.538084in}}%
\pgfpathlineto{\pgfqpoint{3.125185in}{1.532260in}}%
\pgfpathlineto{\pgfqpoint{3.133701in}{1.538403in}}%
\pgfpathlineto{\pgfqpoint{3.142208in}{1.544689in}}%
\pgfpathlineto{\pgfqpoint{3.150704in}{1.551115in}}%
\pgfpathlineto{\pgfqpoint{3.159191in}{1.557675in}}%
\pgfpathlineto{\pgfqpoint{3.145349in}{1.563170in}}%
\pgfpathlineto{\pgfqpoint{3.131512in}{1.568749in}}%
\pgfpathlineto{\pgfqpoint{3.117678in}{1.574412in}}%
\pgfpathlineto{\pgfqpoint{3.103848in}{1.580160in}}%
\pgfpathlineto{\pgfqpoint{3.095340in}{1.573921in}}%
\pgfpathlineto{\pgfqpoint{3.086821in}{1.567821in}}%
\pgfpathlineto{\pgfqpoint{3.078292in}{1.561866in}}%
\pgfpathlineto{\pgfqpoint{3.069753in}{1.556060in}}%
\pgfpathclose%
\pgfusepath{fill}%
\end{pgfscope}%
\begin{pgfscope}%
\pgfpathrectangle{\pgfqpoint{1.150000in}{0.150000in}}{\pgfqpoint{5.700000in}{5.700000in}}%
\pgfusepath{clip}%
\pgfsetbuttcap%
\pgfsetroundjoin%
\definecolor{currentfill}{rgb}{0.267004,0.004874,0.329415}%
\pgfsetfillcolor{currentfill}%
\pgfsetfillopacity{0.700000}%
\pgfsetlinewidth{0.000000pt}%
\definecolor{currentstroke}{rgb}{0.000000,0.000000,0.000000}%
\pgfsetstrokecolor{currentstroke}%
\pgfsetdash{}{0pt}%
\pgfpathmoveto{\pgfqpoint{3.214602in}{1.536525in}}%
\pgfpathlineto{\pgfqpoint{3.228466in}{1.531444in}}%
\pgfpathlineto{\pgfqpoint{3.242334in}{1.526444in}}%
\pgfpathlineto{\pgfqpoint{3.256207in}{1.521525in}}%
\pgfpathlineto{\pgfqpoint{3.270085in}{1.516688in}}%
\pgfpathlineto{\pgfqpoint{3.278522in}{1.524009in}}%
\pgfpathlineto{\pgfqpoint{3.286951in}{1.531444in}}%
\pgfpathlineto{\pgfqpoint{3.295371in}{1.538988in}}%
\pgfpathlineto{\pgfqpoint{3.303782in}{1.546637in}}%
\pgfpathlineto{\pgfqpoint{3.289924in}{1.551167in}}%
\pgfpathlineto{\pgfqpoint{3.276070in}{1.555777in}}%
\pgfpathlineto{\pgfqpoint{3.262221in}{1.560469in}}%
\pgfpathlineto{\pgfqpoint{3.248377in}{1.565243in}}%
\pgfpathlineto{\pgfqpoint{3.239946in}{1.557894in}}%
\pgfpathlineto{\pgfqpoint{3.231507in}{1.550655in}}%
\pgfpathlineto{\pgfqpoint{3.223059in}{1.543530in}}%
\pgfpathlineto{\pgfqpoint{3.214602in}{1.536525in}}%
\pgfpathclose%
\pgfusepath{fill}%
\end{pgfscope}%
\begin{pgfscope}%
\pgfpathrectangle{\pgfqpoint{1.150000in}{0.150000in}}{\pgfqpoint{5.700000in}{5.700000in}}%
\pgfusepath{clip}%
\pgfsetbuttcap%
\pgfsetroundjoin%
\definecolor{currentfill}{rgb}{0.279566,0.067836,0.391917}%
\pgfsetfillcolor{currentfill}%
\pgfsetfillopacity{0.700000}%
\pgfsetlinewidth{0.000000pt}%
\definecolor{currentstroke}{rgb}{0.000000,0.000000,0.000000}%
\pgfsetstrokecolor{currentstroke}%
\pgfsetdash{}{0pt}%
\pgfpathmoveto{\pgfqpoint{2.723567in}{1.669990in}}%
\pgfpathlineto{\pgfqpoint{2.737420in}{1.661409in}}%
\pgfpathlineto{\pgfqpoint{2.751275in}{1.652922in}}%
\pgfpathlineto{\pgfqpoint{2.765132in}{1.644528in}}%
\pgfpathlineto{\pgfqpoint{2.778991in}{1.636227in}}%
\pgfpathlineto{\pgfqpoint{2.787734in}{1.639127in}}%
\pgfpathlineto{\pgfqpoint{2.796463in}{1.642241in}}%
\pgfpathlineto{\pgfqpoint{2.805179in}{1.645564in}}%
\pgfpathlineto{\pgfqpoint{2.813881in}{1.649089in}}%
\pgfpathlineto{\pgfqpoint{2.800053in}{1.657017in}}%
\pgfpathlineto{\pgfqpoint{2.786226in}{1.665037in}}%
\pgfpathlineto{\pgfqpoint{2.772401in}{1.673151in}}%
\pgfpathlineto{\pgfqpoint{2.758578in}{1.681358in}}%
\pgfpathlineto{\pgfqpoint{2.749846in}{1.678198in}}%
\pgfpathlineto{\pgfqpoint{2.741101in}{1.675246in}}%
\pgfpathlineto{\pgfqpoint{2.732341in}{1.672508in}}%
\pgfpathlineto{\pgfqpoint{2.723567in}{1.669990in}}%
\pgfpathclose%
\pgfusepath{fill}%
\end{pgfscope}%
\begin{pgfscope}%
\pgfpathrectangle{\pgfqpoint{1.150000in}{0.150000in}}{\pgfqpoint{5.700000in}{5.700000in}}%
\pgfusepath{clip}%
\pgfsetbuttcap%
\pgfsetroundjoin%
\definecolor{currentfill}{rgb}{0.274952,0.037752,0.364543}%
\pgfsetfillcolor{currentfill}%
\pgfsetfillopacity{0.700000}%
\pgfsetlinewidth{0.000000pt}%
\definecolor{currentstroke}{rgb}{0.000000,0.000000,0.000000}%
\pgfsetstrokecolor{currentstroke}%
\pgfsetdash{}{0pt}%
\pgfpathmoveto{\pgfqpoint{3.681536in}{1.587119in}}%
\pgfpathlineto{\pgfqpoint{3.695476in}{1.585033in}}%
\pgfpathlineto{\pgfqpoint{3.709422in}{1.583023in}}%
\pgfpathlineto{\pgfqpoint{3.723376in}{1.581088in}}%
\pgfpathlineto{\pgfqpoint{3.737337in}{1.579230in}}%
\pgfpathlineto{\pgfqpoint{3.745570in}{1.589272in}}%
\pgfpathlineto{\pgfqpoint{3.753796in}{1.599335in}}%
\pgfpathlineto{\pgfqpoint{3.762017in}{1.609414in}}%
\pgfpathlineto{\pgfqpoint{3.770232in}{1.619507in}}%
\pgfpathlineto{\pgfqpoint{3.756282in}{1.621139in}}%
\pgfpathlineto{\pgfqpoint{3.742340in}{1.622848in}}%
\pgfpathlineto{\pgfqpoint{3.728405in}{1.624632in}}%
\pgfpathlineto{\pgfqpoint{3.714478in}{1.626492in}}%
\pgfpathlineto{\pgfqpoint{3.706251in}{1.616617in}}%
\pgfpathlineto{\pgfqpoint{3.698019in}{1.606762in}}%
\pgfpathlineto{\pgfqpoint{3.689781in}{1.596928in}}%
\pgfpathlineto{\pgfqpoint{3.681536in}{1.587119in}}%
\pgfpathclose%
\pgfusepath{fill}%
\end{pgfscope}%
\begin{pgfscope}%
\pgfpathrectangle{\pgfqpoint{1.150000in}{0.150000in}}{\pgfqpoint{5.700000in}{5.700000in}}%
\pgfusepath{clip}%
\pgfsetbuttcap%
\pgfsetroundjoin%
\definecolor{currentfill}{rgb}{0.252194,0.269783,0.531579}%
\pgfsetfillcolor{currentfill}%
\pgfsetfillopacity{0.700000}%
\pgfsetlinewidth{0.000000pt}%
\definecolor{currentstroke}{rgb}{0.000000,0.000000,0.000000}%
\pgfsetstrokecolor{currentstroke}%
\pgfsetdash{}{0pt}%
\pgfpathmoveto{\pgfqpoint{2.095709in}{2.118924in}}%
\pgfpathlineto{\pgfqpoint{2.109672in}{2.105197in}}%
\pgfpathlineto{\pgfqpoint{2.123632in}{2.091596in}}%
\pgfpathlineto{\pgfqpoint{2.137588in}{2.078118in}}%
\pgfpathlineto{\pgfqpoint{2.151542in}{2.064764in}}%
\pgfpathlineto{\pgfqpoint{2.160811in}{2.061418in}}%
\pgfpathlineto{\pgfqpoint{2.170058in}{2.058394in}}%
\pgfpathlineto{\pgfqpoint{2.179282in}{2.055685in}}%
\pgfpathlineto{\pgfqpoint{2.188485in}{2.053284in}}%
\pgfpathlineto{\pgfqpoint{2.174577in}{2.066208in}}%
\pgfpathlineto{\pgfqpoint{2.160666in}{2.079255in}}%
\pgfpathlineto{\pgfqpoint{2.146752in}{2.092425in}}%
\pgfpathlineto{\pgfqpoint{2.132835in}{2.105720in}}%
\pgfpathlineto{\pgfqpoint{2.123588in}{2.108543in}}%
\pgfpathlineto{\pgfqpoint{2.114318in}{2.111681in}}%
\pgfpathlineto{\pgfqpoint{2.105025in}{2.115139in}}%
\pgfpathlineto{\pgfqpoint{2.095709in}{2.118924in}}%
\pgfpathclose%
\pgfusepath{fill}%
\end{pgfscope}%
\begin{pgfscope}%
\pgfpathrectangle{\pgfqpoint{1.150000in}{0.150000in}}{\pgfqpoint{5.700000in}{5.700000in}}%
\pgfusepath{clip}%
\pgfsetbuttcap%
\pgfsetroundjoin%
\definecolor{currentfill}{rgb}{0.204903,0.375746,0.553533}%
\pgfsetfillcolor{currentfill}%
\pgfsetfillopacity{0.700000}%
\pgfsetlinewidth{0.000000pt}%
\definecolor{currentstroke}{rgb}{0.000000,0.000000,0.000000}%
\pgfsetstrokecolor{currentstroke}%
\pgfsetdash{}{0pt}%
\pgfpathmoveto{\pgfqpoint{5.243995in}{2.328402in}}%
\pgfpathlineto{\pgfqpoint{5.258496in}{2.332536in}}%
\pgfpathlineto{\pgfqpoint{5.273009in}{2.336740in}}%
\pgfpathlineto{\pgfqpoint{5.287534in}{2.341015in}}%
\pgfpathlineto{\pgfqpoint{5.302073in}{2.345360in}}%
\pgfpathlineto{\pgfqpoint{5.309677in}{2.351430in}}%
\pgfpathlineto{\pgfqpoint{5.317272in}{2.357398in}}%
\pgfpathlineto{\pgfqpoint{5.324859in}{2.363267in}}%
\pgfpathlineto{\pgfqpoint{5.332437in}{2.369040in}}%
\pgfpathlineto{\pgfqpoint{5.317914in}{2.364829in}}%
\pgfpathlineto{\pgfqpoint{5.303404in}{2.360689in}}%
\pgfpathlineto{\pgfqpoint{5.288907in}{2.356619in}}%
\pgfpathlineto{\pgfqpoint{5.274422in}{2.352619in}}%
\pgfpathlineto{\pgfqpoint{5.266828in}{2.346705in}}%
\pgfpathlineto{\pgfqpoint{5.259226in}{2.340698in}}%
\pgfpathlineto{\pgfqpoint{5.251615in}{2.334598in}}%
\pgfpathlineto{\pgfqpoint{5.243995in}{2.328402in}}%
\pgfpathclose%
\pgfusepath{fill}%
\end{pgfscope}%
\begin{pgfscope}%
\pgfpathrectangle{\pgfqpoint{1.150000in}{0.150000in}}{\pgfqpoint{5.700000in}{5.700000in}}%
\pgfusepath{clip}%
\pgfsetbuttcap%
\pgfsetroundjoin%
\definecolor{currentfill}{rgb}{0.272594,0.025563,0.353093}%
\pgfsetfillcolor{currentfill}%
\pgfsetfillopacity{0.700000}%
\pgfsetlinewidth{0.000000pt}%
\definecolor{currentstroke}{rgb}{0.000000,0.000000,0.000000}%
\pgfsetstrokecolor{currentstroke}%
\pgfsetdash{}{0pt}%
\pgfpathmoveto{\pgfqpoint{2.924595in}{1.588936in}}%
\pgfpathlineto{\pgfqpoint{2.938446in}{1.581818in}}%
\pgfpathlineto{\pgfqpoint{2.952300in}{1.574789in}}%
\pgfpathlineto{\pgfqpoint{2.966157in}{1.567847in}}%
\pgfpathlineto{\pgfqpoint{2.980017in}{1.560991in}}%
\pgfpathlineto{\pgfqpoint{2.988625in}{1.565790in}}%
\pgfpathlineto{\pgfqpoint{2.997221in}{1.570765in}}%
\pgfpathlineto{\pgfqpoint{3.005806in}{1.575909in}}%
\pgfpathlineto{\pgfqpoint{3.014379in}{1.581219in}}%
\pgfpathlineto{\pgfqpoint{3.000545in}{1.587723in}}%
\pgfpathlineto{\pgfqpoint{2.986713in}{1.594315in}}%
\pgfpathlineto{\pgfqpoint{2.972885in}{1.600994in}}%
\pgfpathlineto{\pgfqpoint{2.959060in}{1.607761in}}%
\pgfpathlineto{\pgfqpoint{2.950461in}{1.602794in}}%
\pgfpathlineto{\pgfqpoint{2.941851in}{1.597997in}}%
\pgfpathlineto{\pgfqpoint{2.933229in}{1.593376in}}%
\pgfpathlineto{\pgfqpoint{2.924595in}{1.588936in}}%
\pgfpathclose%
\pgfusepath{fill}%
\end{pgfscope}%
\begin{pgfscope}%
\pgfpathrectangle{\pgfqpoint{1.150000in}{0.150000in}}{\pgfqpoint{5.700000in}{5.700000in}}%
\pgfusepath{clip}%
\pgfsetbuttcap%
\pgfsetroundjoin%
\definecolor{currentfill}{rgb}{0.210503,0.363727,0.552206}%
\pgfsetfillcolor{currentfill}%
\pgfsetfillopacity{0.700000}%
\pgfsetlinewidth{0.000000pt}%
\definecolor{currentstroke}{rgb}{0.000000,0.000000,0.000000}%
\pgfsetstrokecolor{currentstroke}%
\pgfsetdash{}{0pt}%
\pgfpathmoveto{\pgfqpoint{5.155502in}{2.286324in}}%
\pgfpathlineto{\pgfqpoint{5.169966in}{2.290288in}}%
\pgfpathlineto{\pgfqpoint{5.184443in}{2.294322in}}%
\pgfpathlineto{\pgfqpoint{5.198932in}{2.298427in}}%
\pgfpathlineto{\pgfqpoint{5.213434in}{2.302603in}}%
\pgfpathlineto{\pgfqpoint{5.221087in}{2.309210in}}%
\pgfpathlineto{\pgfqpoint{5.228732in}{2.315710in}}%
\pgfpathlineto{\pgfqpoint{5.236368in}{2.322107in}}%
\pgfpathlineto{\pgfqpoint{5.243995in}{2.328402in}}%
\pgfpathlineto{\pgfqpoint{5.229508in}{2.324339in}}%
\pgfpathlineto{\pgfqpoint{5.215033in}{2.320346in}}%
\pgfpathlineto{\pgfqpoint{5.200571in}{2.316424in}}%
\pgfpathlineto{\pgfqpoint{5.186121in}{2.312573in}}%
\pgfpathlineto{\pgfqpoint{5.178479in}{2.306157in}}%
\pgfpathlineto{\pgfqpoint{5.170828in}{2.299645in}}%
\pgfpathlineto{\pgfqpoint{5.163169in}{2.293035in}}%
\pgfpathlineto{\pgfqpoint{5.155502in}{2.286324in}}%
\pgfpathclose%
\pgfusepath{fill}%
\end{pgfscope}%
\begin{pgfscope}%
\pgfpathrectangle{\pgfqpoint{1.150000in}{0.150000in}}{\pgfqpoint{5.700000in}{5.700000in}}%
\pgfusepath{clip}%
\pgfsetbuttcap%
\pgfsetroundjoin%
\definecolor{currentfill}{rgb}{0.267004,0.004874,0.329415}%
\pgfsetfillcolor{currentfill}%
\pgfsetfillopacity{0.700000}%
\pgfsetlinewidth{0.000000pt}%
\definecolor{currentstroke}{rgb}{0.000000,0.000000,0.000000}%
\pgfsetstrokecolor{currentstroke}%
\pgfsetdash{}{0pt}%
\pgfpathmoveto{\pgfqpoint{3.359268in}{1.529325in}}%
\pgfpathlineto{\pgfqpoint{3.373153in}{1.525196in}}%
\pgfpathlineto{\pgfqpoint{3.387043in}{1.521147in}}%
\pgfpathlineto{\pgfqpoint{3.400939in}{1.517178in}}%
\pgfpathlineto{\pgfqpoint{3.414840in}{1.513287in}}%
\pgfpathlineto{\pgfqpoint{3.423208in}{1.521626in}}%
\pgfpathlineto{\pgfqpoint{3.431569in}{1.530052in}}%
\pgfpathlineto{\pgfqpoint{3.439923in}{1.538560in}}%
\pgfpathlineto{\pgfqpoint{3.448269in}{1.547145in}}%
\pgfpathlineto{\pgfqpoint{3.434384in}{1.550749in}}%
\pgfpathlineto{\pgfqpoint{3.420505in}{1.554431in}}%
\pgfpathlineto{\pgfqpoint{3.406632in}{1.558193in}}%
\pgfpathlineto{\pgfqpoint{3.392765in}{1.562034in}}%
\pgfpathlineto{\pgfqpoint{3.384402in}{1.553729in}}%
\pgfpathlineto{\pgfqpoint{3.376032in}{1.545506in}}%
\pgfpathlineto{\pgfqpoint{3.367654in}{1.537370in}}%
\pgfpathlineto{\pgfqpoint{3.359268in}{1.529325in}}%
\pgfpathclose%
\pgfusepath{fill}%
\end{pgfscope}%
\begin{pgfscope}%
\pgfpathrectangle{\pgfqpoint{1.150000in}{0.150000in}}{\pgfqpoint{5.700000in}{5.700000in}}%
\pgfusepath{clip}%
\pgfsetbuttcap%
\pgfsetroundjoin%
\definecolor{currentfill}{rgb}{0.260571,0.246922,0.522828}%
\pgfsetfillcolor{currentfill}%
\pgfsetfillopacity{0.700000}%
\pgfsetlinewidth{0.000000pt}%
\definecolor{currentstroke}{rgb}{0.000000,0.000000,0.000000}%
\pgfsetstrokecolor{currentstroke}%
\pgfsetdash{}{0pt}%
\pgfpathmoveto{\pgfqpoint{2.151542in}{2.064764in}}%
\pgfpathlineto{\pgfqpoint{2.165492in}{2.051531in}}%
\pgfpathlineto{\pgfqpoint{2.179440in}{2.038420in}}%
\pgfpathlineto{\pgfqpoint{2.193385in}{2.025428in}}%
\pgfpathlineto{\pgfqpoint{2.207327in}{2.012556in}}%
\pgfpathlineto{\pgfqpoint{2.216551in}{2.009649in}}%
\pgfpathlineto{\pgfqpoint{2.225752in}{2.007057in}}%
\pgfpathlineto{\pgfqpoint{2.234933in}{2.004774in}}%
\pgfpathlineto{\pgfqpoint{2.244091in}{2.002792in}}%
\pgfpathlineto{\pgfqpoint{2.230193in}{2.015236in}}%
\pgfpathlineto{\pgfqpoint{2.216293in}{2.027799in}}%
\pgfpathlineto{\pgfqpoint{2.202390in}{2.040481in}}%
\pgfpathlineto{\pgfqpoint{2.188485in}{2.053284in}}%
\pgfpathlineto{\pgfqpoint{2.179282in}{2.055685in}}%
\pgfpathlineto{\pgfqpoint{2.170058in}{2.058394in}}%
\pgfpathlineto{\pgfqpoint{2.160811in}{2.061418in}}%
\pgfpathlineto{\pgfqpoint{2.151542in}{2.064764in}}%
\pgfpathclose%
\pgfusepath{fill}%
\end{pgfscope}%
\begin{pgfscope}%
\pgfpathrectangle{\pgfqpoint{1.150000in}{0.150000in}}{\pgfqpoint{5.700000in}{5.700000in}}%
\pgfusepath{clip}%
\pgfsetbuttcap%
\pgfsetroundjoin%
\definecolor{currentfill}{rgb}{0.218130,0.347432,0.550038}%
\pgfsetfillcolor{currentfill}%
\pgfsetfillopacity{0.700000}%
\pgfsetlinewidth{0.000000pt}%
\definecolor{currentstroke}{rgb}{0.000000,0.000000,0.000000}%
\pgfsetstrokecolor{currentstroke}%
\pgfsetdash{}{0pt}%
\pgfpathmoveto{\pgfqpoint{5.066965in}{2.242923in}}%
\pgfpathlineto{\pgfqpoint{5.081393in}{2.246694in}}%
\pgfpathlineto{\pgfqpoint{5.095833in}{2.250536in}}%
\pgfpathlineto{\pgfqpoint{5.110286in}{2.254449in}}%
\pgfpathlineto{\pgfqpoint{5.124751in}{2.258433in}}%
\pgfpathlineto{\pgfqpoint{5.132451in}{2.265567in}}%
\pgfpathlineto{\pgfqpoint{5.140143in}{2.272592in}}%
\pgfpathlineto{\pgfqpoint{5.147826in}{2.279510in}}%
\pgfpathlineto{\pgfqpoint{5.155502in}{2.286324in}}%
\pgfpathlineto{\pgfqpoint{5.141050in}{2.282431in}}%
\pgfpathlineto{\pgfqpoint{5.126611in}{2.278609in}}%
\pgfpathlineto{\pgfqpoint{5.112184in}{2.274857in}}%
\pgfpathlineto{\pgfqpoint{5.097769in}{2.271176in}}%
\pgfpathlineto{\pgfqpoint{5.090080in}{2.264264in}}%
\pgfpathlineto{\pgfqpoint{5.082383in}{2.257252in}}%
\pgfpathlineto{\pgfqpoint{5.074678in}{2.250139in}}%
\pgfpathlineto{\pgfqpoint{5.066965in}{2.242923in}}%
\pgfpathclose%
\pgfusepath{fill}%
\end{pgfscope}%
\begin{pgfscope}%
\pgfpathrectangle{\pgfqpoint{1.150000in}{0.150000in}}{\pgfqpoint{5.700000in}{5.700000in}}%
\pgfusepath{clip}%
\pgfsetbuttcap%
\pgfsetroundjoin%
\definecolor{currentfill}{rgb}{0.271305,0.019942,0.347269}%
\pgfsetfillcolor{currentfill}%
\pgfsetfillopacity{0.700000}%
\pgfsetlinewidth{0.000000pt}%
\definecolor{currentstroke}{rgb}{0.000000,0.000000,0.000000}%
\pgfsetstrokecolor{currentstroke}%
\pgfsetdash{}{0pt}%
\pgfpathmoveto{\pgfqpoint{3.592755in}{1.558293in}}%
\pgfpathlineto{\pgfqpoint{3.606681in}{1.555656in}}%
\pgfpathlineto{\pgfqpoint{3.620613in}{1.553095in}}%
\pgfpathlineto{\pgfqpoint{3.634551in}{1.550610in}}%
\pgfpathlineto{\pgfqpoint{3.648497in}{1.548201in}}%
\pgfpathlineto{\pgfqpoint{3.656766in}{1.557876in}}%
\pgfpathlineto{\pgfqpoint{3.665029in}{1.567590in}}%
\pgfpathlineto{\pgfqpoint{3.673285in}{1.577338in}}%
\pgfpathlineto{\pgfqpoint{3.681536in}{1.587119in}}%
\pgfpathlineto{\pgfqpoint{3.667603in}{1.589281in}}%
\pgfpathlineto{\pgfqpoint{3.653678in}{1.591519in}}%
\pgfpathlineto{\pgfqpoint{3.639759in}{1.593834in}}%
\pgfpathlineto{\pgfqpoint{3.625847in}{1.596225in}}%
\pgfpathlineto{\pgfqpoint{3.617583in}{1.586683in}}%
\pgfpathlineto{\pgfqpoint{3.609314in}{1.577179in}}%
\pgfpathlineto{\pgfqpoint{3.601038in}{1.567714in}}%
\pgfpathlineto{\pgfqpoint{3.592755in}{1.558293in}}%
\pgfpathclose%
\pgfusepath{fill}%
\end{pgfscope}%
\begin{pgfscope}%
\pgfpathrectangle{\pgfqpoint{1.150000in}{0.150000in}}{\pgfqpoint{5.700000in}{5.700000in}}%
\pgfusepath{clip}%
\pgfsetbuttcap%
\pgfsetroundjoin%
\definecolor{currentfill}{rgb}{0.283229,0.120777,0.440584}%
\pgfsetfillcolor{currentfill}%
\pgfsetfillopacity{0.700000}%
\pgfsetlinewidth{0.000000pt}%
\definecolor{currentstroke}{rgb}{0.000000,0.000000,0.000000}%
\pgfsetstrokecolor{currentstroke}%
\pgfsetdash{}{0pt}%
\pgfpathmoveto{\pgfqpoint{2.521758in}{1.777480in}}%
\pgfpathlineto{\pgfqpoint{2.535635in}{1.767332in}}%
\pgfpathlineto{\pgfqpoint{2.549513in}{1.757284in}}%
\pgfpathlineto{\pgfqpoint{2.563390in}{1.747337in}}%
\pgfpathlineto{\pgfqpoint{2.577269in}{1.737490in}}%
\pgfpathlineto{\pgfqpoint{2.586173in}{1.738261in}}%
\pgfpathlineto{\pgfqpoint{2.595060in}{1.739286in}}%
\pgfpathlineto{\pgfqpoint{2.603931in}{1.740560in}}%
\pgfpathlineto{\pgfqpoint{2.612786in}{1.742077in}}%
\pgfpathlineto{\pgfqpoint{2.598943in}{1.751526in}}%
\pgfpathlineto{\pgfqpoint{2.585100in}{1.761076in}}%
\pgfpathlineto{\pgfqpoint{2.571258in}{1.770725in}}%
\pgfpathlineto{\pgfqpoint{2.557417in}{1.780475in}}%
\pgfpathlineto{\pgfqpoint{2.548527in}{1.779348in}}%
\pgfpathlineto{\pgfqpoint{2.539621in}{1.778469in}}%
\pgfpathlineto{\pgfqpoint{2.530698in}{1.777845in}}%
\pgfpathlineto{\pgfqpoint{2.521758in}{1.777480in}}%
\pgfpathclose%
\pgfusepath{fill}%
\end{pgfscope}%
\begin{pgfscope}%
\pgfpathrectangle{\pgfqpoint{1.150000in}{0.150000in}}{\pgfqpoint{5.700000in}{5.700000in}}%
\pgfusepath{clip}%
\pgfsetbuttcap%
\pgfsetroundjoin%
\definecolor{currentfill}{rgb}{0.227802,0.326594,0.546532}%
\pgfsetfillcolor{currentfill}%
\pgfsetfillopacity{0.700000}%
\pgfsetlinewidth{0.000000pt}%
\definecolor{currentstroke}{rgb}{0.000000,0.000000,0.000000}%
\pgfsetstrokecolor{currentstroke}%
\pgfsetdash{}{0pt}%
\pgfpathmoveto{\pgfqpoint{4.978394in}{2.198338in}}%
\pgfpathlineto{\pgfqpoint{4.992786in}{2.201895in}}%
\pgfpathlineto{\pgfqpoint{5.007190in}{2.205522in}}%
\pgfpathlineto{\pgfqpoint{5.021605in}{2.209221in}}%
\pgfpathlineto{\pgfqpoint{5.036033in}{2.212990in}}%
\pgfpathlineto{\pgfqpoint{5.043778in}{2.220637in}}%
\pgfpathlineto{\pgfqpoint{5.051515in}{2.228173in}}%
\pgfpathlineto{\pgfqpoint{5.059244in}{2.235601in}}%
\pgfpathlineto{\pgfqpoint{5.066965in}{2.242923in}}%
\pgfpathlineto{\pgfqpoint{5.052549in}{2.239222in}}%
\pgfpathlineto{\pgfqpoint{5.038146in}{2.235593in}}%
\pgfpathlineto{\pgfqpoint{5.023754in}{2.232034in}}%
\pgfpathlineto{\pgfqpoint{5.009375in}{2.228546in}}%
\pgfpathlineto{\pgfqpoint{5.001641in}{2.221148in}}%
\pgfpathlineto{\pgfqpoint{4.993900in}{2.213648in}}%
\pgfpathlineto{\pgfqpoint{4.986151in}{2.206046in}}%
\pgfpathlineto{\pgfqpoint{4.978394in}{2.198338in}}%
\pgfpathclose%
\pgfusepath{fill}%
\end{pgfscope}%
\begin{pgfscope}%
\pgfpathrectangle{\pgfqpoint{1.150000in}{0.150000in}}{\pgfqpoint{5.700000in}{5.700000in}}%
\pgfusepath{clip}%
\pgfsetbuttcap%
\pgfsetroundjoin%
\definecolor{currentfill}{rgb}{0.235526,0.309527,0.542944}%
\pgfsetfillcolor{currentfill}%
\pgfsetfillopacity{0.700000}%
\pgfsetlinewidth{0.000000pt}%
\definecolor{currentstroke}{rgb}{0.000000,0.000000,0.000000}%
\pgfsetstrokecolor{currentstroke}%
\pgfsetdash{}{0pt}%
\pgfpathmoveto{\pgfqpoint{4.889798in}{2.152732in}}%
\pgfpathlineto{\pgfqpoint{4.904153in}{2.156051in}}%
\pgfpathlineto{\pgfqpoint{4.918520in}{2.159442in}}%
\pgfpathlineto{\pgfqpoint{4.932899in}{2.162903in}}%
\pgfpathlineto{\pgfqpoint{4.947290in}{2.166436in}}%
\pgfpathlineto{\pgfqpoint{4.955078in}{2.174575in}}%
\pgfpathlineto{\pgfqpoint{4.962858in}{2.182604in}}%
\pgfpathlineto{\pgfqpoint{4.970630in}{2.190525in}}%
\pgfpathlineto{\pgfqpoint{4.978394in}{2.198338in}}%
\pgfpathlineto{\pgfqpoint{4.964015in}{2.194853in}}%
\pgfpathlineto{\pgfqpoint{4.949647in}{2.191438in}}%
\pgfpathlineto{\pgfqpoint{4.935291in}{2.188095in}}%
\pgfpathlineto{\pgfqpoint{4.920947in}{2.184823in}}%
\pgfpathlineto{\pgfqpoint{4.913171in}{2.176954in}}%
\pgfpathlineto{\pgfqpoint{4.905388in}{2.168984in}}%
\pgfpathlineto{\pgfqpoint{4.897597in}{2.160910in}}%
\pgfpathlineto{\pgfqpoint{4.889798in}{2.152732in}}%
\pgfpathclose%
\pgfusepath{fill}%
\end{pgfscope}%
\begin{pgfscope}%
\pgfpathrectangle{\pgfqpoint{1.150000in}{0.150000in}}{\pgfqpoint{5.700000in}{5.700000in}}%
\pgfusepath{clip}%
\pgfsetbuttcap%
\pgfsetroundjoin%
\definecolor{currentfill}{rgb}{0.243113,0.292092,0.538516}%
\pgfsetfillcolor{currentfill}%
\pgfsetfillopacity{0.700000}%
\pgfsetlinewidth{0.000000pt}%
\definecolor{currentstroke}{rgb}{0.000000,0.000000,0.000000}%
\pgfsetstrokecolor{currentstroke}%
\pgfsetdash{}{0pt}%
\pgfpathmoveto{\pgfqpoint{4.801184in}{2.106287in}}%
\pgfpathlineto{\pgfqpoint{4.815503in}{2.109347in}}%
\pgfpathlineto{\pgfqpoint{4.829834in}{2.112477in}}%
\pgfpathlineto{\pgfqpoint{4.844176in}{2.115680in}}%
\pgfpathlineto{\pgfqpoint{4.858530in}{2.118953in}}%
\pgfpathlineto{\pgfqpoint{4.866358in}{2.127560in}}%
\pgfpathlineto{\pgfqpoint{4.874179in}{2.136058in}}%
\pgfpathlineto{\pgfqpoint{4.881992in}{2.144448in}}%
\pgfpathlineto{\pgfqpoint{4.889798in}{2.152732in}}%
\pgfpathlineto{\pgfqpoint{4.875455in}{2.149484in}}%
\pgfpathlineto{\pgfqpoint{4.861123in}{2.146307in}}%
\pgfpathlineto{\pgfqpoint{4.846803in}{2.143201in}}%
\pgfpathlineto{\pgfqpoint{4.832494in}{2.140167in}}%
\pgfpathlineto{\pgfqpoint{4.824677in}{2.131850in}}%
\pgfpathlineto{\pgfqpoint{4.816853in}{2.123431in}}%
\pgfpathlineto{\pgfqpoint{4.809022in}{2.114911in}}%
\pgfpathlineto{\pgfqpoint{4.801184in}{2.106287in}}%
\pgfpathclose%
\pgfusepath{fill}%
\end{pgfscope}%
\begin{pgfscope}%
\pgfpathrectangle{\pgfqpoint{1.150000in}{0.150000in}}{\pgfqpoint{5.700000in}{5.700000in}}%
\pgfusepath{clip}%
\pgfsetbuttcap%
\pgfsetroundjoin%
\definecolor{currentfill}{rgb}{0.271828,0.209303,0.504434}%
\pgfsetfillcolor{currentfill}%
\pgfsetfillopacity{0.700000}%
\pgfsetlinewidth{0.000000pt}%
\definecolor{currentstroke}{rgb}{0.000000,0.000000,0.000000}%
\pgfsetstrokecolor{currentstroke}%
\pgfsetdash{}{0pt}%
\pgfpathmoveto{\pgfqpoint{4.446664in}{1.916534in}}%
\pgfpathlineto{\pgfqpoint{4.460844in}{1.918334in}}%
\pgfpathlineto{\pgfqpoint{4.475034in}{1.920205in}}%
\pgfpathlineto{\pgfqpoint{4.489234in}{1.922149in}}%
\pgfpathlineto{\pgfqpoint{4.503446in}{1.924164in}}%
\pgfpathlineto{\pgfqpoint{4.511417in}{1.934275in}}%
\pgfpathlineto{\pgfqpoint{4.519383in}{1.944297in}}%
\pgfpathlineto{\pgfqpoint{4.527342in}{1.954230in}}%
\pgfpathlineto{\pgfqpoint{4.535295in}{1.964074in}}%
\pgfpathlineto{\pgfqpoint{4.521092in}{1.961998in}}%
\pgfpathlineto{\pgfqpoint{4.506900in}{1.959994in}}%
\pgfpathlineto{\pgfqpoint{4.492718in}{1.958062in}}%
\pgfpathlineto{\pgfqpoint{4.478546in}{1.956202in}}%
\pgfpathlineto{\pgfqpoint{4.470585in}{1.946412in}}%
\pgfpathlineto{\pgfqpoint{4.462617in}{1.936536in}}%
\pgfpathlineto{\pgfqpoint{4.454643in}{1.926577in}}%
\pgfpathlineto{\pgfqpoint{4.446664in}{1.916534in}}%
\pgfpathclose%
\pgfusepath{fill}%
\end{pgfscope}%
\begin{pgfscope}%
\pgfpathrectangle{\pgfqpoint{1.150000in}{0.150000in}}{\pgfqpoint{5.700000in}{5.700000in}}%
\pgfusepath{clip}%
\pgfsetbuttcap%
\pgfsetroundjoin%
\definecolor{currentfill}{rgb}{0.252194,0.269783,0.531579}%
\pgfsetfillcolor{currentfill}%
\pgfsetfillopacity{0.700000}%
\pgfsetlinewidth{0.000000pt}%
\definecolor{currentstroke}{rgb}{0.000000,0.000000,0.000000}%
\pgfsetstrokecolor{currentstroke}%
\pgfsetdash{}{0pt}%
\pgfpathmoveto{\pgfqpoint{4.712559in}{2.059208in}}%
\pgfpathlineto{\pgfqpoint{4.726842in}{2.061986in}}%
\pgfpathlineto{\pgfqpoint{4.741136in}{2.064835in}}%
\pgfpathlineto{\pgfqpoint{4.755442in}{2.067756in}}%
\pgfpathlineto{\pgfqpoint{4.769760in}{2.070748in}}%
\pgfpathlineto{\pgfqpoint{4.777626in}{2.079790in}}%
\pgfpathlineto{\pgfqpoint{4.785486in}{2.088727in}}%
\pgfpathlineto{\pgfqpoint{4.793339in}{2.097559in}}%
\pgfpathlineto{\pgfqpoint{4.801184in}{2.106287in}}%
\pgfpathlineto{\pgfqpoint{4.786876in}{2.103298in}}%
\pgfpathlineto{\pgfqpoint{4.772580in}{2.100381in}}%
\pgfpathlineto{\pgfqpoint{4.758295in}{2.097536in}}%
\pgfpathlineto{\pgfqpoint{4.744022in}{2.094762in}}%
\pgfpathlineto{\pgfqpoint{4.736166in}{2.086022in}}%
\pgfpathlineto{\pgfqpoint{4.728304in}{2.077184in}}%
\pgfpathlineto{\pgfqpoint{4.720435in}{2.068246in}}%
\pgfpathlineto{\pgfqpoint{4.712559in}{2.059208in}}%
\pgfpathclose%
\pgfusepath{fill}%
\end{pgfscope}%
\begin{pgfscope}%
\pgfpathrectangle{\pgfqpoint{1.150000in}{0.150000in}}{\pgfqpoint{5.700000in}{5.700000in}}%
\pgfusepath{clip}%
\pgfsetbuttcap%
\pgfsetroundjoin%
\definecolor{currentfill}{rgb}{0.266580,0.228262,0.514349}%
\pgfsetfillcolor{currentfill}%
\pgfsetfillopacity{0.700000}%
\pgfsetlinewidth{0.000000pt}%
\definecolor{currentstroke}{rgb}{0.000000,0.000000,0.000000}%
\pgfsetstrokecolor{currentstroke}%
\pgfsetdash{}{0pt}%
\pgfpathmoveto{\pgfqpoint{4.535295in}{1.964074in}}%
\pgfpathlineto{\pgfqpoint{4.549508in}{1.966221in}}%
\pgfpathlineto{\pgfqpoint{4.563733in}{1.968441in}}%
\pgfpathlineto{\pgfqpoint{4.577968in}{1.970732in}}%
\pgfpathlineto{\pgfqpoint{4.592213in}{1.973095in}}%
\pgfpathlineto{\pgfqpoint{4.600152in}{1.982895in}}%
\pgfpathlineto{\pgfqpoint{4.608083in}{1.992600in}}%
\pgfpathlineto{\pgfqpoint{4.616009in}{2.002208in}}%
\pgfpathlineto{\pgfqpoint{4.623927in}{2.011721in}}%
\pgfpathlineto{\pgfqpoint{4.609690in}{2.009319in}}%
\pgfpathlineto{\pgfqpoint{4.595464in}{2.006988in}}%
\pgfpathlineto{\pgfqpoint{4.581248in}{2.004730in}}%
\pgfpathlineto{\pgfqpoint{4.567043in}{2.002543in}}%
\pgfpathlineto{\pgfqpoint{4.559116in}{1.993061in}}%
\pgfpathlineto{\pgfqpoint{4.551182in}{1.983489in}}%
\pgfpathlineto{\pgfqpoint{4.543242in}{1.973827in}}%
\pgfpathlineto{\pgfqpoint{4.535295in}{1.964074in}}%
\pgfpathclose%
\pgfusepath{fill}%
\end{pgfscope}%
\begin{pgfscope}%
\pgfpathrectangle{\pgfqpoint{1.150000in}{0.150000in}}{\pgfqpoint{5.700000in}{5.700000in}}%
\pgfusepath{clip}%
\pgfsetbuttcap%
\pgfsetroundjoin%
\definecolor{currentfill}{rgb}{0.277134,0.185228,0.489898}%
\pgfsetfillcolor{currentfill}%
\pgfsetfillopacity{0.700000}%
\pgfsetlinewidth{0.000000pt}%
\definecolor{currentstroke}{rgb}{0.000000,0.000000,0.000000}%
\pgfsetstrokecolor{currentstroke}%
\pgfsetdash{}{0pt}%
\pgfpathmoveto{\pgfqpoint{4.358035in}{1.869393in}}%
\pgfpathlineto{\pgfqpoint{4.372182in}{1.870822in}}%
\pgfpathlineto{\pgfqpoint{4.386340in}{1.872323in}}%
\pgfpathlineto{\pgfqpoint{4.400507in}{1.873896in}}%
\pgfpathlineto{\pgfqpoint{4.414685in}{1.875542in}}%
\pgfpathlineto{\pgfqpoint{4.422688in}{1.885912in}}%
\pgfpathlineto{\pgfqpoint{4.430686in}{1.896201in}}%
\pgfpathlineto{\pgfqpoint{4.438678in}{1.906409in}}%
\pgfpathlineto{\pgfqpoint{4.446664in}{1.916534in}}%
\pgfpathlineto{\pgfqpoint{4.432494in}{1.914807in}}%
\pgfpathlineto{\pgfqpoint{4.418335in}{1.913152in}}%
\pgfpathlineto{\pgfqpoint{4.404185in}{1.911569in}}%
\pgfpathlineto{\pgfqpoint{4.390046in}{1.910058in}}%
\pgfpathlineto{\pgfqpoint{4.382052in}{1.900006in}}%
\pgfpathlineto{\pgfqpoint{4.374052in}{1.889878in}}%
\pgfpathlineto{\pgfqpoint{4.366046in}{1.879673in}}%
\pgfpathlineto{\pgfqpoint{4.358035in}{1.869393in}}%
\pgfpathclose%
\pgfusepath{fill}%
\end{pgfscope}%
\begin{pgfscope}%
\pgfpathrectangle{\pgfqpoint{1.150000in}{0.150000in}}{\pgfqpoint{5.700000in}{5.700000in}}%
\pgfusepath{clip}%
\pgfsetbuttcap%
\pgfsetroundjoin%
\definecolor{currentfill}{rgb}{0.258965,0.251537,0.524736}%
\pgfsetfillcolor{currentfill}%
\pgfsetfillopacity{0.700000}%
\pgfsetlinewidth{0.000000pt}%
\definecolor{currentstroke}{rgb}{0.000000,0.000000,0.000000}%
\pgfsetstrokecolor{currentstroke}%
\pgfsetdash{}{0pt}%
\pgfpathmoveto{\pgfqpoint{4.623927in}{2.011721in}}%
\pgfpathlineto{\pgfqpoint{4.638175in}{2.014195in}}%
\pgfpathlineto{\pgfqpoint{4.652435in}{2.016740in}}%
\pgfpathlineto{\pgfqpoint{4.666705in}{2.019357in}}%
\pgfpathlineto{\pgfqpoint{4.680986in}{2.022046in}}%
\pgfpathlineto{\pgfqpoint{4.688889in}{2.031488in}}%
\pgfpathlineto{\pgfqpoint{4.696786in}{2.040829in}}%
\pgfpathlineto{\pgfqpoint{4.704676in}{2.050069in}}%
\pgfpathlineto{\pgfqpoint{4.712559in}{2.059208in}}%
\pgfpathlineto{\pgfqpoint{4.698286in}{2.056501in}}%
\pgfpathlineto{\pgfqpoint{4.684025in}{2.053867in}}%
\pgfpathlineto{\pgfqpoint{4.669775in}{2.051303in}}%
\pgfpathlineto{\pgfqpoint{4.655537in}{2.048812in}}%
\pgfpathlineto{\pgfqpoint{4.647644in}{2.039683in}}%
\pgfpathlineto{\pgfqpoint{4.639745in}{2.030458in}}%
\pgfpathlineto{\pgfqpoint{4.631840in}{2.021137in}}%
\pgfpathlineto{\pgfqpoint{4.623927in}{2.011721in}}%
\pgfpathclose%
\pgfusepath{fill}%
\end{pgfscope}%
\begin{pgfscope}%
\pgfpathrectangle{\pgfqpoint{1.150000in}{0.150000in}}{\pgfqpoint{5.700000in}{5.700000in}}%
\pgfusepath{clip}%
\pgfsetbuttcap%
\pgfsetroundjoin%
\definecolor{currentfill}{rgb}{0.280255,0.165693,0.476498}%
\pgfsetfillcolor{currentfill}%
\pgfsetfillopacity{0.700000}%
\pgfsetlinewidth{0.000000pt}%
\definecolor{currentstroke}{rgb}{0.000000,0.000000,0.000000}%
\pgfsetstrokecolor{currentstroke}%
\pgfsetdash{}{0pt}%
\pgfpathmoveto{\pgfqpoint{4.269407in}{1.822958in}}%
\pgfpathlineto{\pgfqpoint{4.283523in}{1.823995in}}%
\pgfpathlineto{\pgfqpoint{4.297649in}{1.825104in}}%
\pgfpathlineto{\pgfqpoint{4.311785in}{1.826286in}}%
\pgfpathlineto{\pgfqpoint{4.325930in}{1.827540in}}%
\pgfpathlineto{\pgfqpoint{4.333965in}{1.838111in}}%
\pgfpathlineto{\pgfqpoint{4.341994in}{1.848610in}}%
\pgfpathlineto{\pgfqpoint{4.350017in}{1.859038in}}%
\pgfpathlineto{\pgfqpoint{4.358035in}{1.869393in}}%
\pgfpathlineto{\pgfqpoint{4.343897in}{1.868036in}}%
\pgfpathlineto{\pgfqpoint{4.329770in}{1.866752in}}%
\pgfpathlineto{\pgfqpoint{4.315652in}{1.865540in}}%
\pgfpathlineto{\pgfqpoint{4.301544in}{1.864401in}}%
\pgfpathlineto{\pgfqpoint{4.293518in}{1.854141in}}%
\pgfpathlineto{\pgfqpoint{4.285487in}{1.843813in}}%
\pgfpathlineto{\pgfqpoint{4.277450in}{1.833419in}}%
\pgfpathlineto{\pgfqpoint{4.269407in}{1.822958in}}%
\pgfpathclose%
\pgfusepath{fill}%
\end{pgfscope}%
\begin{pgfscope}%
\pgfpathrectangle{\pgfqpoint{1.150000in}{0.150000in}}{\pgfqpoint{5.700000in}{5.700000in}}%
\pgfusepath{clip}%
\pgfsetbuttcap%
\pgfsetroundjoin%
\definecolor{currentfill}{rgb}{0.282623,0.140926,0.457517}%
\pgfsetfillcolor{currentfill}%
\pgfsetfillopacity{0.700000}%
\pgfsetlinewidth{0.000000pt}%
\definecolor{currentstroke}{rgb}{0.000000,0.000000,0.000000}%
\pgfsetstrokecolor{currentstroke}%
\pgfsetdash{}{0pt}%
\pgfpathmoveto{\pgfqpoint{4.180778in}{1.777563in}}%
\pgfpathlineto{\pgfqpoint{4.194865in}{1.778185in}}%
\pgfpathlineto{\pgfqpoint{4.208960in}{1.778881in}}%
\pgfpathlineto{\pgfqpoint{4.223066in}{1.779648in}}%
\pgfpathlineto{\pgfqpoint{4.237180in}{1.780489in}}%
\pgfpathlineto{\pgfqpoint{4.245245in}{1.791198in}}%
\pgfpathlineto{\pgfqpoint{4.253305in}{1.801847in}}%
\pgfpathlineto{\pgfqpoint{4.261359in}{1.812434in}}%
\pgfpathlineto{\pgfqpoint{4.269407in}{1.822958in}}%
\pgfpathlineto{\pgfqpoint{4.255301in}{1.821994in}}%
\pgfpathlineto{\pgfqpoint{4.241203in}{1.821103in}}%
\pgfpathlineto{\pgfqpoint{4.227116in}{1.820285in}}%
\pgfpathlineto{\pgfqpoint{4.213038in}{1.819539in}}%
\pgfpathlineto{\pgfqpoint{4.204981in}{1.809130in}}%
\pgfpathlineto{\pgfqpoint{4.196919in}{1.798664in}}%
\pgfpathlineto{\pgfqpoint{4.188851in}{1.788141in}}%
\pgfpathlineto{\pgfqpoint{4.180778in}{1.777563in}}%
\pgfpathclose%
\pgfusepath{fill}%
\end{pgfscope}%
\begin{pgfscope}%
\pgfpathrectangle{\pgfqpoint{1.150000in}{0.150000in}}{\pgfqpoint{5.700000in}{5.700000in}}%
\pgfusepath{clip}%
\pgfsetbuttcap%
\pgfsetroundjoin%
\definecolor{currentfill}{rgb}{0.283229,0.120777,0.440584}%
\pgfsetfillcolor{currentfill}%
\pgfsetfillopacity{0.700000}%
\pgfsetlinewidth{0.000000pt}%
\definecolor{currentstroke}{rgb}{0.000000,0.000000,0.000000}%
\pgfsetstrokecolor{currentstroke}%
\pgfsetdash{}{0pt}%
\pgfpathmoveto{\pgfqpoint{4.092142in}{1.733559in}}%
\pgfpathlineto{\pgfqpoint{4.106201in}{1.733745in}}%
\pgfpathlineto{\pgfqpoint{4.120268in}{1.734004in}}%
\pgfpathlineto{\pgfqpoint{4.134345in}{1.734336in}}%
\pgfpathlineto{\pgfqpoint{4.148430in}{1.734740in}}%
\pgfpathlineto{\pgfqpoint{4.156525in}{1.745519in}}%
\pgfpathlineto{\pgfqpoint{4.164615in}{1.756251in}}%
\pgfpathlineto{\pgfqpoint{4.172699in}{1.766933in}}%
\pgfpathlineto{\pgfqpoint{4.180778in}{1.777563in}}%
\pgfpathlineto{\pgfqpoint{4.166701in}{1.777014in}}%
\pgfpathlineto{\pgfqpoint{4.152633in}{1.776538in}}%
\pgfpathlineto{\pgfqpoint{4.138574in}{1.776135in}}%
\pgfpathlineto{\pgfqpoint{4.124524in}{1.775805in}}%
\pgfpathlineto{\pgfqpoint{4.116437in}{1.765311in}}%
\pgfpathlineto{\pgfqpoint{4.108344in}{1.754771in}}%
\pgfpathlineto{\pgfqpoint{4.100246in}{1.744186in}}%
\pgfpathlineto{\pgfqpoint{4.092142in}{1.733559in}}%
\pgfpathclose%
\pgfusepath{fill}%
\end{pgfscope}%
\begin{pgfscope}%
\pgfpathrectangle{\pgfqpoint{1.150000in}{0.150000in}}{\pgfqpoint{5.700000in}{5.700000in}}%
\pgfusepath{clip}%
\pgfsetbuttcap%
\pgfsetroundjoin%
\definecolor{currentfill}{rgb}{0.266580,0.228262,0.514349}%
\pgfsetfillcolor{currentfill}%
\pgfsetfillopacity{0.700000}%
\pgfsetlinewidth{0.000000pt}%
\definecolor{currentstroke}{rgb}{0.000000,0.000000,0.000000}%
\pgfsetstrokecolor{currentstroke}%
\pgfsetdash{}{0pt}%
\pgfpathmoveto{\pgfqpoint{2.207327in}{2.012556in}}%
\pgfpathlineto{\pgfqpoint{2.221267in}{1.999802in}}%
\pgfpathlineto{\pgfqpoint{2.235205in}{1.987165in}}%
\pgfpathlineto{\pgfqpoint{2.249140in}{1.974644in}}%
\pgfpathlineto{\pgfqpoint{2.263073in}{1.962239in}}%
\pgfpathlineto{\pgfqpoint{2.272252in}{1.959767in}}%
\pgfpathlineto{\pgfqpoint{2.281410in}{1.957606in}}%
\pgfpathlineto{\pgfqpoint{2.290547in}{1.955747in}}%
\pgfpathlineto{\pgfqpoint{2.299663in}{1.954184in}}%
\pgfpathlineto{\pgfqpoint{2.285773in}{1.966163in}}%
\pgfpathlineto{\pgfqpoint{2.271881in}{1.978257in}}%
\pgfpathlineto{\pgfqpoint{2.257987in}{1.990466in}}%
\pgfpathlineto{\pgfqpoint{2.244091in}{2.002792in}}%
\pgfpathlineto{\pgfqpoint{2.234933in}{2.004774in}}%
\pgfpathlineto{\pgfqpoint{2.225752in}{2.007057in}}%
\pgfpathlineto{\pgfqpoint{2.216551in}{2.009649in}}%
\pgfpathlineto{\pgfqpoint{2.207327in}{2.012556in}}%
\pgfpathclose%
\pgfusepath{fill}%
\end{pgfscope}%
\begin{pgfscope}%
\pgfpathrectangle{\pgfqpoint{1.150000in}{0.150000in}}{\pgfqpoint{5.700000in}{5.700000in}}%
\pgfusepath{clip}%
\pgfsetbuttcap%
\pgfsetroundjoin%
\definecolor{currentfill}{rgb}{0.282656,0.100196,0.422160}%
\pgfsetfillcolor{currentfill}%
\pgfsetfillopacity{0.700000}%
\pgfsetlinewidth{0.000000pt}%
\definecolor{currentstroke}{rgb}{0.000000,0.000000,0.000000}%
\pgfsetstrokecolor{currentstroke}%
\pgfsetdash{}{0pt}%
\pgfpathmoveto{\pgfqpoint{4.003492in}{1.691319in}}%
\pgfpathlineto{\pgfqpoint{4.017524in}{1.691046in}}%
\pgfpathlineto{\pgfqpoint{4.031565in}{1.690846in}}%
\pgfpathlineto{\pgfqpoint{4.045615in}{1.690720in}}%
\pgfpathlineto{\pgfqpoint{4.059673in}{1.690668in}}%
\pgfpathlineto{\pgfqpoint{4.067799in}{1.701444in}}%
\pgfpathlineto{\pgfqpoint{4.075919in}{1.712186in}}%
\pgfpathlineto{\pgfqpoint{4.084033in}{1.722892in}}%
\pgfpathlineto{\pgfqpoint{4.092142in}{1.733559in}}%
\pgfpathlineto{\pgfqpoint{4.078093in}{1.733447in}}%
\pgfpathlineto{\pgfqpoint{4.064052in}{1.733408in}}%
\pgfpathlineto{\pgfqpoint{4.050020in}{1.733442in}}%
\pgfpathlineto{\pgfqpoint{4.035996in}{1.733551in}}%
\pgfpathlineto{\pgfqpoint{4.027878in}{1.723040in}}%
\pgfpathlineto{\pgfqpoint{4.019755in}{1.712497in}}%
\pgfpathlineto{\pgfqpoint{4.011626in}{1.701922in}}%
\pgfpathlineto{\pgfqpoint{4.003492in}{1.691319in}}%
\pgfpathclose%
\pgfusepath{fill}%
\end{pgfscope}%
\begin{pgfscope}%
\pgfpathrectangle{\pgfqpoint{1.150000in}{0.150000in}}{\pgfqpoint{5.700000in}{5.700000in}}%
\pgfusepath{clip}%
\pgfsetbuttcap%
\pgfsetroundjoin%
\definecolor{currentfill}{rgb}{0.277941,0.056324,0.381191}%
\pgfsetfillcolor{currentfill}%
\pgfsetfillopacity{0.700000}%
\pgfsetlinewidth{0.000000pt}%
\definecolor{currentstroke}{rgb}{0.000000,0.000000,0.000000}%
\pgfsetstrokecolor{currentstroke}%
\pgfsetdash{}{0pt}%
\pgfpathmoveto{\pgfqpoint{2.778991in}{1.636227in}}%
\pgfpathlineto{\pgfqpoint{2.792851in}{1.628018in}}%
\pgfpathlineto{\pgfqpoint{2.806714in}{1.619900in}}%
\pgfpathlineto{\pgfqpoint{2.820578in}{1.611874in}}%
\pgfpathlineto{\pgfqpoint{2.834445in}{1.603938in}}%
\pgfpathlineto{\pgfqpoint{2.843158in}{1.607219in}}%
\pgfpathlineto{\pgfqpoint{2.851858in}{1.610709in}}%
\pgfpathlineto{\pgfqpoint{2.860544in}{1.614402in}}%
\pgfpathlineto{\pgfqpoint{2.869218in}{1.618292in}}%
\pgfpathlineto{\pgfqpoint{2.855380in}{1.625855in}}%
\pgfpathlineto{\pgfqpoint{2.841545in}{1.633508in}}%
\pgfpathlineto{\pgfqpoint{2.827712in}{1.641253in}}%
\pgfpathlineto{\pgfqpoint{2.813881in}{1.649089in}}%
\pgfpathlineto{\pgfqpoint{2.805179in}{1.645564in}}%
\pgfpathlineto{\pgfqpoint{2.796463in}{1.642241in}}%
\pgfpathlineto{\pgfqpoint{2.787734in}{1.639127in}}%
\pgfpathlineto{\pgfqpoint{2.778991in}{1.636227in}}%
\pgfpathclose%
\pgfusepath{fill}%
\end{pgfscope}%
\begin{pgfscope}%
\pgfpathrectangle{\pgfqpoint{1.150000in}{0.150000in}}{\pgfqpoint{5.700000in}{5.700000in}}%
\pgfusepath{clip}%
\pgfsetbuttcap%
\pgfsetroundjoin%
\definecolor{currentfill}{rgb}{0.268510,0.009605,0.335427}%
\pgfsetfillcolor{currentfill}%
\pgfsetfillopacity{0.700000}%
\pgfsetlinewidth{0.000000pt}%
\definecolor{currentstroke}{rgb}{0.000000,0.000000,0.000000}%
\pgfsetstrokecolor{currentstroke}%
\pgfsetdash{}{0pt}%
\pgfpathmoveto{\pgfqpoint{3.503866in}{1.533514in}}%
\pgfpathlineto{\pgfqpoint{3.517780in}{1.530301in}}%
\pgfpathlineto{\pgfqpoint{3.531700in}{1.527165in}}%
\pgfpathlineto{\pgfqpoint{3.545627in}{1.524107in}}%
\pgfpathlineto{\pgfqpoint{3.559560in}{1.521125in}}%
\pgfpathlineto{\pgfqpoint{3.567869in}{1.530332in}}%
\pgfpathlineto{\pgfqpoint{3.576171in}{1.539599in}}%
\pgfpathlineto{\pgfqpoint{3.584466in}{1.548921in}}%
\pgfpathlineto{\pgfqpoint{3.592755in}{1.558293in}}%
\pgfpathlineto{\pgfqpoint{3.578837in}{1.561008in}}%
\pgfpathlineto{\pgfqpoint{3.564924in}{1.563800in}}%
\pgfpathlineto{\pgfqpoint{3.551018in}{1.566668in}}%
\pgfpathlineto{\pgfqpoint{3.537119in}{1.569615in}}%
\pgfpathlineto{\pgfqpoint{3.528816in}{1.560501in}}%
\pgfpathlineto{\pgfqpoint{3.520506in}{1.551443in}}%
\pgfpathlineto{\pgfqpoint{3.512189in}{1.542446in}}%
\pgfpathlineto{\pgfqpoint{3.503866in}{1.533514in}}%
\pgfpathclose%
\pgfusepath{fill}%
\end{pgfscope}%
\begin{pgfscope}%
\pgfpathrectangle{\pgfqpoint{1.150000in}{0.150000in}}{\pgfqpoint{5.700000in}{5.700000in}}%
\pgfusepath{clip}%
\pgfsetbuttcap%
\pgfsetroundjoin%
\definecolor{currentfill}{rgb}{0.280894,0.078907,0.402329}%
\pgfsetfillcolor{currentfill}%
\pgfsetfillopacity{0.700000}%
\pgfsetlinewidth{0.000000pt}%
\definecolor{currentstroke}{rgb}{0.000000,0.000000,0.000000}%
\pgfsetstrokecolor{currentstroke}%
\pgfsetdash{}{0pt}%
\pgfpathmoveto{\pgfqpoint{3.914817in}{1.651236in}}%
\pgfpathlineto{\pgfqpoint{3.928825in}{1.650482in}}%
\pgfpathlineto{\pgfqpoint{3.942842in}{1.649802in}}%
\pgfpathlineto{\pgfqpoint{3.956867in}{1.649196in}}%
\pgfpathlineto{\pgfqpoint{3.970901in}{1.648664in}}%
\pgfpathlineto{\pgfqpoint{3.979057in}{1.659359in}}%
\pgfpathlineto{\pgfqpoint{3.987207in}{1.670035in}}%
\pgfpathlineto{\pgfqpoint{3.995352in}{1.680689in}}%
\pgfpathlineto{\pgfqpoint{4.003492in}{1.691319in}}%
\pgfpathlineto{\pgfqpoint{3.989468in}{1.691666in}}%
\pgfpathlineto{\pgfqpoint{3.975452in}{1.692086in}}%
\pgfpathlineto{\pgfqpoint{3.961445in}{1.692581in}}%
\pgfpathlineto{\pgfqpoint{3.947446in}{1.693150in}}%
\pgfpathlineto{\pgfqpoint{3.939297in}{1.682697in}}%
\pgfpathlineto{\pgfqpoint{3.931142in}{1.672226in}}%
\pgfpathlineto{\pgfqpoint{3.922982in}{1.661738in}}%
\pgfpathlineto{\pgfqpoint{3.914817in}{1.651236in}}%
\pgfpathclose%
\pgfusepath{fill}%
\end{pgfscope}%
\begin{pgfscope}%
\pgfpathrectangle{\pgfqpoint{1.150000in}{0.150000in}}{\pgfqpoint{5.700000in}{5.700000in}}%
\pgfusepath{clip}%
\pgfsetbuttcap%
\pgfsetroundjoin%
\definecolor{currentfill}{rgb}{0.267004,0.004874,0.329415}%
\pgfsetfillcolor{currentfill}%
\pgfsetfillopacity{0.700000}%
\pgfsetlinewidth{0.000000pt}%
\definecolor{currentstroke}{rgb}{0.000000,0.000000,0.000000}%
\pgfsetstrokecolor{currentstroke}%
\pgfsetdash{}{0pt}%
\pgfpathmoveto{\pgfqpoint{3.125185in}{1.532260in}}%
\pgfpathlineto{\pgfqpoint{3.139053in}{1.526519in}}%
\pgfpathlineto{\pgfqpoint{3.152925in}{1.520862in}}%
\pgfpathlineto{\pgfqpoint{3.166801in}{1.515287in}}%
\pgfpathlineto{\pgfqpoint{3.180682in}{1.509794in}}%
\pgfpathlineto{\pgfqpoint{3.189176in}{1.516274in}}%
\pgfpathlineto{\pgfqpoint{3.197661in}{1.522892in}}%
\pgfpathlineto{\pgfqpoint{3.206136in}{1.529644in}}%
\pgfpathlineto{\pgfqpoint{3.214602in}{1.536525in}}%
\pgfpathlineto{\pgfqpoint{3.200743in}{1.541689in}}%
\pgfpathlineto{\pgfqpoint{3.186888in}{1.546935in}}%
\pgfpathlineto{\pgfqpoint{3.173037in}{1.552263in}}%
\pgfpathlineto{\pgfqpoint{3.159191in}{1.557675in}}%
\pgfpathlineto{\pgfqpoint{3.150704in}{1.551115in}}%
\pgfpathlineto{\pgfqpoint{3.142208in}{1.544689in}}%
\pgfpathlineto{\pgfqpoint{3.133701in}{1.538403in}}%
\pgfpathlineto{\pgfqpoint{3.125185in}{1.532260in}}%
\pgfpathclose%
\pgfusepath{fill}%
\end{pgfscope}%
\begin{pgfscope}%
\pgfpathrectangle{\pgfqpoint{1.150000in}{0.150000in}}{\pgfqpoint{5.700000in}{5.700000in}}%
\pgfusepath{clip}%
\pgfsetbuttcap%
\pgfsetroundjoin%
\definecolor{currentfill}{rgb}{0.282910,0.105393,0.426902}%
\pgfsetfillcolor{currentfill}%
\pgfsetfillopacity{0.700000}%
\pgfsetlinewidth{0.000000pt}%
\definecolor{currentstroke}{rgb}{0.000000,0.000000,0.000000}%
\pgfsetstrokecolor{currentstroke}%
\pgfsetdash{}{0pt}%
\pgfpathmoveto{\pgfqpoint{2.577269in}{1.737490in}}%
\pgfpathlineto{\pgfqpoint{2.591147in}{1.727742in}}%
\pgfpathlineto{\pgfqpoint{2.605027in}{1.718092in}}%
\pgfpathlineto{\pgfqpoint{2.618907in}{1.708541in}}%
\pgfpathlineto{\pgfqpoint{2.632788in}{1.699086in}}%
\pgfpathlineto{\pgfqpoint{2.641656in}{1.700262in}}%
\pgfpathlineto{\pgfqpoint{2.650509in}{1.701688in}}%
\pgfpathlineto{\pgfqpoint{2.659346in}{1.703356in}}%
\pgfpathlineto{\pgfqpoint{2.668167in}{1.705261in}}%
\pgfpathlineto{\pgfqpoint{2.654320in}{1.714319in}}%
\pgfpathlineto{\pgfqpoint{2.640474in}{1.723474in}}%
\pgfpathlineto{\pgfqpoint{2.626630in}{1.732726in}}%
\pgfpathlineto{\pgfqpoint{2.612786in}{1.742077in}}%
\pgfpathlineto{\pgfqpoint{2.603931in}{1.740560in}}%
\pgfpathlineto{\pgfqpoint{2.595060in}{1.739286in}}%
\pgfpathlineto{\pgfqpoint{2.586173in}{1.738261in}}%
\pgfpathlineto{\pgfqpoint{2.577269in}{1.737490in}}%
\pgfpathclose%
\pgfusepath{fill}%
\end{pgfscope}%
\begin{pgfscope}%
\pgfpathrectangle{\pgfqpoint{1.150000in}{0.150000in}}{\pgfqpoint{5.700000in}{5.700000in}}%
\pgfusepath{clip}%
\pgfsetbuttcap%
\pgfsetroundjoin%
\definecolor{currentfill}{rgb}{0.267004,0.004874,0.329415}%
\pgfsetfillcolor{currentfill}%
\pgfsetfillopacity{0.700000}%
\pgfsetlinewidth{0.000000pt}%
\definecolor{currentstroke}{rgb}{0.000000,0.000000,0.000000}%
\pgfsetstrokecolor{currentstroke}%
\pgfsetdash{}{0pt}%
\pgfpathmoveto{\pgfqpoint{3.270085in}{1.516688in}}%
\pgfpathlineto{\pgfqpoint{3.283968in}{1.511931in}}%
\pgfpathlineto{\pgfqpoint{3.297855in}{1.507255in}}%
\pgfpathlineto{\pgfqpoint{3.311748in}{1.502659in}}%
\pgfpathlineto{\pgfqpoint{3.325645in}{1.498144in}}%
\pgfpathlineto{\pgfqpoint{3.334064in}{1.505780in}}%
\pgfpathlineto{\pgfqpoint{3.342473in}{1.513525in}}%
\pgfpathlineto{\pgfqpoint{3.350875in}{1.521375in}}%
\pgfpathlineto{\pgfqpoint{3.359268in}{1.529325in}}%
\pgfpathlineto{\pgfqpoint{3.345389in}{1.533533in}}%
\pgfpathlineto{\pgfqpoint{3.331515in}{1.537821in}}%
\pgfpathlineto{\pgfqpoint{3.317646in}{1.542189in}}%
\pgfpathlineto{\pgfqpoint{3.303782in}{1.546637in}}%
\pgfpathlineto{\pgfqpoint{3.295371in}{1.538988in}}%
\pgfpathlineto{\pgfqpoint{3.286951in}{1.531444in}}%
\pgfpathlineto{\pgfqpoint{3.278522in}{1.524009in}}%
\pgfpathlineto{\pgfqpoint{3.270085in}{1.516688in}}%
\pgfpathclose%
\pgfusepath{fill}%
\end{pgfscope}%
\begin{pgfscope}%
\pgfpathrectangle{\pgfqpoint{1.150000in}{0.150000in}}{\pgfqpoint{5.700000in}{5.700000in}}%
\pgfusepath{clip}%
\pgfsetbuttcap%
\pgfsetroundjoin%
\definecolor{currentfill}{rgb}{0.271828,0.209303,0.504434}%
\pgfsetfillcolor{currentfill}%
\pgfsetfillopacity{0.700000}%
\pgfsetlinewidth{0.000000pt}%
\definecolor{currentstroke}{rgb}{0.000000,0.000000,0.000000}%
\pgfsetstrokecolor{currentstroke}%
\pgfsetdash{}{0pt}%
\pgfpathmoveto{\pgfqpoint{2.263073in}{1.962239in}}%
\pgfpathlineto{\pgfqpoint{2.277004in}{1.949948in}}%
\pgfpathlineto{\pgfqpoint{2.290934in}{1.937770in}}%
\pgfpathlineto{\pgfqpoint{2.304862in}{1.925706in}}%
\pgfpathlineto{\pgfqpoint{2.318788in}{1.913753in}}%
\pgfpathlineto{\pgfqpoint{2.327924in}{1.911716in}}%
\pgfpathlineto{\pgfqpoint{2.337039in}{1.909983in}}%
\pgfpathlineto{\pgfqpoint{2.346134in}{1.908547in}}%
\pgfpathlineto{\pgfqpoint{2.355209in}{1.907402in}}%
\pgfpathlineto{\pgfqpoint{2.341324in}{1.918929in}}%
\pgfpathlineto{\pgfqpoint{2.327439in}{1.930568in}}%
\pgfpathlineto{\pgfqpoint{2.313552in}{1.942320in}}%
\pgfpathlineto{\pgfqpoint{2.299663in}{1.954184in}}%
\pgfpathlineto{\pgfqpoint{2.290547in}{1.955747in}}%
\pgfpathlineto{\pgfqpoint{2.281410in}{1.957606in}}%
\pgfpathlineto{\pgfqpoint{2.272252in}{1.959767in}}%
\pgfpathlineto{\pgfqpoint{2.263073in}{1.962239in}}%
\pgfpathclose%
\pgfusepath{fill}%
\end{pgfscope}%
\begin{pgfscope}%
\pgfpathrectangle{\pgfqpoint{1.150000in}{0.150000in}}{\pgfqpoint{5.700000in}{5.700000in}}%
\pgfusepath{clip}%
\pgfsetbuttcap%
\pgfsetroundjoin%
\definecolor{currentfill}{rgb}{0.271305,0.019942,0.347269}%
\pgfsetfillcolor{currentfill}%
\pgfsetfillopacity{0.700000}%
\pgfsetlinewidth{0.000000pt}%
\definecolor{currentstroke}{rgb}{0.000000,0.000000,0.000000}%
\pgfsetstrokecolor{currentstroke}%
\pgfsetdash{}{0pt}%
\pgfpathmoveto{\pgfqpoint{2.980017in}{1.560991in}}%
\pgfpathlineto{\pgfqpoint{2.993880in}{1.554223in}}%
\pgfpathlineto{\pgfqpoint{3.007746in}{1.547540in}}%
\pgfpathlineto{\pgfqpoint{3.021616in}{1.540943in}}%
\pgfpathlineto{\pgfqpoint{3.035489in}{1.534431in}}%
\pgfpathlineto{\pgfqpoint{3.044071in}{1.539589in}}%
\pgfpathlineto{\pgfqpoint{3.052643in}{1.544916in}}%
\pgfpathlineto{\pgfqpoint{3.061203in}{1.550408in}}%
\pgfpathlineto{\pgfqpoint{3.069753in}{1.556060in}}%
\pgfpathlineto{\pgfqpoint{3.055904in}{1.562221in}}%
\pgfpathlineto{\pgfqpoint{3.042059in}{1.568468in}}%
\pgfpathlineto{\pgfqpoint{3.028217in}{1.574800in}}%
\pgfpathlineto{\pgfqpoint{3.014379in}{1.581219in}}%
\pgfpathlineto{\pgfqpoint{3.005806in}{1.575909in}}%
\pgfpathlineto{\pgfqpoint{2.997221in}{1.570765in}}%
\pgfpathlineto{\pgfqpoint{2.988625in}{1.565790in}}%
\pgfpathlineto{\pgfqpoint{2.980017in}{1.560991in}}%
\pgfpathclose%
\pgfusepath{fill}%
\end{pgfscope}%
\begin{pgfscope}%
\pgfpathrectangle{\pgfqpoint{1.150000in}{0.150000in}}{\pgfqpoint{5.700000in}{5.700000in}}%
\pgfusepath{clip}%
\pgfsetbuttcap%
\pgfsetroundjoin%
\definecolor{currentfill}{rgb}{0.278791,0.062145,0.386592}%
\pgfsetfillcolor{currentfill}%
\pgfsetfillopacity{0.700000}%
\pgfsetlinewidth{0.000000pt}%
\definecolor{currentstroke}{rgb}{0.000000,0.000000,0.000000}%
\pgfsetstrokecolor{currentstroke}%
\pgfsetdash{}{0pt}%
\pgfpathmoveto{\pgfqpoint{3.826104in}{1.613728in}}%
\pgfpathlineto{\pgfqpoint{3.840091in}{1.612470in}}%
\pgfpathlineto{\pgfqpoint{3.854086in}{1.611287in}}%
\pgfpathlineto{\pgfqpoint{3.868089in}{1.610179in}}%
\pgfpathlineto{\pgfqpoint{3.882099in}{1.609145in}}%
\pgfpathlineto{\pgfqpoint{3.890287in}{1.619675in}}%
\pgfpathlineto{\pgfqpoint{3.898469in}{1.630202in}}%
\pgfpathlineto{\pgfqpoint{3.906646in}{1.640723in}}%
\pgfpathlineto{\pgfqpoint{3.914817in}{1.651236in}}%
\pgfpathlineto{\pgfqpoint{3.900816in}{1.652065in}}%
\pgfpathlineto{\pgfqpoint{3.886824in}{1.652967in}}%
\pgfpathlineto{\pgfqpoint{3.872839in}{1.653944in}}%
\pgfpathlineto{\pgfqpoint{3.858863in}{1.654996in}}%
\pgfpathlineto{\pgfqpoint{3.850681in}{1.644681in}}%
\pgfpathlineto{\pgfqpoint{3.842495in}{1.634363in}}%
\pgfpathlineto{\pgfqpoint{3.834302in}{1.624044in}}%
\pgfpathlineto{\pgfqpoint{3.826104in}{1.613728in}}%
\pgfpathclose%
\pgfusepath{fill}%
\end{pgfscope}%
\begin{pgfscope}%
\pgfpathrectangle{\pgfqpoint{1.150000in}{0.150000in}}{\pgfqpoint{5.700000in}{5.700000in}}%
\pgfusepath{clip}%
\pgfsetbuttcap%
\pgfsetroundjoin%
\definecolor{currentfill}{rgb}{0.180629,0.429975,0.557282}%
\pgfsetfillcolor{currentfill}%
\pgfsetfillopacity{0.700000}%
\pgfsetlinewidth{0.000000pt}%
\definecolor{currentstroke}{rgb}{0.000000,0.000000,0.000000}%
\pgfsetstrokecolor{currentstroke}%
\pgfsetdash{}{0pt}%
\pgfpathmoveto{\pgfqpoint{5.567624in}{2.464089in}}%
\pgfpathlineto{\pgfqpoint{5.582283in}{2.468878in}}%
\pgfpathlineto{\pgfqpoint{5.596955in}{2.473737in}}%
\pgfpathlineto{\pgfqpoint{5.611641in}{2.478666in}}%
\pgfpathlineto{\pgfqpoint{5.619071in}{2.482984in}}%
\pgfpathlineto{\pgfqpoint{5.626492in}{2.487217in}}%
\pgfpathlineto{\pgfqpoint{5.633904in}{2.491369in}}%
\pgfpathlineto{\pgfqpoint{5.641307in}{2.495444in}}%
\pgfpathlineto{\pgfqpoint{5.626642in}{2.490716in}}%
\pgfpathlineto{\pgfqpoint{5.611990in}{2.486059in}}%
\pgfpathlineto{\pgfqpoint{5.597352in}{2.481471in}}%
\pgfpathlineto{\pgfqpoint{5.589933in}{2.477239in}}%
\pgfpathlineto{\pgfqpoint{5.582506in}{2.472934in}}%
\pgfpathlineto{\pgfqpoint{5.575069in}{2.468552in}}%
\pgfpathlineto{\pgfqpoint{5.567624in}{2.464089in}}%
\pgfpathclose%
\pgfusepath{fill}%
\end{pgfscope}%
\begin{pgfscope}%
\pgfpathrectangle{\pgfqpoint{1.150000in}{0.150000in}}{\pgfqpoint{5.700000in}{5.700000in}}%
\pgfusepath{clip}%
\pgfsetbuttcap%
\pgfsetroundjoin%
\definecolor{currentfill}{rgb}{0.276022,0.044167,0.370164}%
\pgfsetfillcolor{currentfill}%
\pgfsetfillopacity{0.700000}%
\pgfsetlinewidth{0.000000pt}%
\definecolor{currentstroke}{rgb}{0.000000,0.000000,0.000000}%
\pgfsetstrokecolor{currentstroke}%
\pgfsetdash{}{0pt}%
\pgfpathmoveto{\pgfqpoint{3.737337in}{1.579230in}}%
\pgfpathlineto{\pgfqpoint{3.751305in}{1.577446in}}%
\pgfpathlineto{\pgfqpoint{3.765281in}{1.575738in}}%
\pgfpathlineto{\pgfqpoint{3.779264in}{1.574104in}}%
\pgfpathlineto{\pgfqpoint{3.793254in}{1.572546in}}%
\pgfpathlineto{\pgfqpoint{3.801475in}{1.582822in}}%
\pgfpathlineto{\pgfqpoint{3.809691in}{1.593114in}}%
\pgfpathlineto{\pgfqpoint{3.817900in}{1.603416in}}%
\pgfpathlineto{\pgfqpoint{3.826104in}{1.613728in}}%
\pgfpathlineto{\pgfqpoint{3.812125in}{1.615060in}}%
\pgfpathlineto{\pgfqpoint{3.798153in}{1.616467in}}%
\pgfpathlineto{\pgfqpoint{3.784188in}{1.617949in}}%
\pgfpathlineto{\pgfqpoint{3.770232in}{1.619507in}}%
\pgfpathlineto{\pgfqpoint{3.762017in}{1.609414in}}%
\pgfpathlineto{\pgfqpoint{3.753796in}{1.599335in}}%
\pgfpathlineto{\pgfqpoint{3.745570in}{1.589272in}}%
\pgfpathlineto{\pgfqpoint{3.737337in}{1.579230in}}%
\pgfpathclose%
\pgfusepath{fill}%
\end{pgfscope}%
\begin{pgfscope}%
\pgfpathrectangle{\pgfqpoint{1.150000in}{0.150000in}}{\pgfqpoint{5.700000in}{5.700000in}}%
\pgfusepath{clip}%
\pgfsetbuttcap%
\pgfsetroundjoin%
\definecolor{currentfill}{rgb}{0.267004,0.004874,0.329415}%
\pgfsetfillcolor{currentfill}%
\pgfsetfillopacity{0.700000}%
\pgfsetlinewidth{0.000000pt}%
\definecolor{currentstroke}{rgb}{0.000000,0.000000,0.000000}%
\pgfsetstrokecolor{currentstroke}%
\pgfsetdash{}{0pt}%
\pgfpathmoveto{\pgfqpoint{3.414840in}{1.513287in}}%
\pgfpathlineto{\pgfqpoint{3.428746in}{1.509474in}}%
\pgfpathlineto{\pgfqpoint{3.442659in}{1.505740in}}%
\pgfpathlineto{\pgfqpoint{3.456577in}{1.502084in}}%
\pgfpathlineto{\pgfqpoint{3.470501in}{1.498506in}}%
\pgfpathlineto{\pgfqpoint{3.478853in}{1.507141in}}%
\pgfpathlineto{\pgfqpoint{3.487198in}{1.515857in}}%
\pgfpathlineto{\pgfqpoint{3.495535in}{1.524649in}}%
\pgfpathlineto{\pgfqpoint{3.503866in}{1.533514in}}%
\pgfpathlineto{\pgfqpoint{3.489958in}{1.536804in}}%
\pgfpathlineto{\pgfqpoint{3.476055in}{1.540173in}}%
\pgfpathlineto{\pgfqpoint{3.462159in}{1.543620in}}%
\pgfpathlineto{\pgfqpoint{3.448269in}{1.547145in}}%
\pgfpathlineto{\pgfqpoint{3.439923in}{1.538560in}}%
\pgfpathlineto{\pgfqpoint{3.431569in}{1.530052in}}%
\pgfpathlineto{\pgfqpoint{3.423208in}{1.521626in}}%
\pgfpathlineto{\pgfqpoint{3.414840in}{1.513287in}}%
\pgfpathclose%
\pgfusepath{fill}%
\end{pgfscope}%
\begin{pgfscope}%
\pgfpathrectangle{\pgfqpoint{1.150000in}{0.150000in}}{\pgfqpoint{5.700000in}{5.700000in}}%
\pgfusepath{clip}%
\pgfsetbuttcap%
\pgfsetroundjoin%
\definecolor{currentfill}{rgb}{0.276194,0.190074,0.493001}%
\pgfsetfillcolor{currentfill}%
\pgfsetfillopacity{0.700000}%
\pgfsetlinewidth{0.000000pt}%
\definecolor{currentstroke}{rgb}{0.000000,0.000000,0.000000}%
\pgfsetstrokecolor{currentstroke}%
\pgfsetdash{}{0pt}%
\pgfpathmoveto{\pgfqpoint{2.318788in}{1.913753in}}%
\pgfpathlineto{\pgfqpoint{2.332712in}{1.901911in}}%
\pgfpathlineto{\pgfqpoint{2.346636in}{1.890180in}}%
\pgfpathlineto{\pgfqpoint{2.360558in}{1.878558in}}%
\pgfpathlineto{\pgfqpoint{2.374479in}{1.867044in}}%
\pgfpathlineto{\pgfqpoint{2.383572in}{1.865440in}}%
\pgfpathlineto{\pgfqpoint{2.392646in}{1.864135in}}%
\pgfpathlineto{\pgfqpoint{2.401700in}{1.863120in}}%
\pgfpathlineto{\pgfqpoint{2.410735in}{1.862390in}}%
\pgfpathlineto{\pgfqpoint{2.396855in}{1.873480in}}%
\pgfpathlineto{\pgfqpoint{2.382974in}{1.884678in}}%
\pgfpathlineto{\pgfqpoint{2.369092in}{1.895985in}}%
\pgfpathlineto{\pgfqpoint{2.355209in}{1.907402in}}%
\pgfpathlineto{\pgfqpoint{2.346134in}{1.908547in}}%
\pgfpathlineto{\pgfqpoint{2.337039in}{1.909983in}}%
\pgfpathlineto{\pgfqpoint{2.327924in}{1.911716in}}%
\pgfpathlineto{\pgfqpoint{2.318788in}{1.913753in}}%
\pgfpathclose%
\pgfusepath{fill}%
\end{pgfscope}%
\begin{pgfscope}%
\pgfpathrectangle{\pgfqpoint{1.150000in}{0.150000in}}{\pgfqpoint{5.700000in}{5.700000in}}%
\pgfusepath{clip}%
\pgfsetbuttcap%
\pgfsetroundjoin%
\definecolor{currentfill}{rgb}{0.276022,0.044167,0.370164}%
\pgfsetfillcolor{currentfill}%
\pgfsetfillopacity{0.700000}%
\pgfsetlinewidth{0.000000pt}%
\definecolor{currentstroke}{rgb}{0.000000,0.000000,0.000000}%
\pgfsetstrokecolor{currentstroke}%
\pgfsetdash{}{0pt}%
\pgfpathmoveto{\pgfqpoint{2.834445in}{1.603938in}}%
\pgfpathlineto{\pgfqpoint{2.848314in}{1.596092in}}%
\pgfpathlineto{\pgfqpoint{2.862185in}{1.588337in}}%
\pgfpathlineto{\pgfqpoint{2.876059in}{1.580670in}}%
\pgfpathlineto{\pgfqpoint{2.889935in}{1.573092in}}%
\pgfpathlineto{\pgfqpoint{2.898619in}{1.576754in}}%
\pgfpathlineto{\pgfqpoint{2.907290in}{1.580619in}}%
\pgfpathlineto{\pgfqpoint{2.915949in}{1.584681in}}%
\pgfpathlineto{\pgfqpoint{2.924595in}{1.588936in}}%
\pgfpathlineto{\pgfqpoint{2.910747in}{1.596141in}}%
\pgfpathlineto{\pgfqpoint{2.896901in}{1.603435in}}%
\pgfpathlineto{\pgfqpoint{2.883058in}{1.610819in}}%
\pgfpathlineto{\pgfqpoint{2.869218in}{1.618292in}}%
\pgfpathlineto{\pgfqpoint{2.860544in}{1.614402in}}%
\pgfpathlineto{\pgfqpoint{2.851858in}{1.610709in}}%
\pgfpathlineto{\pgfqpoint{2.843158in}{1.607219in}}%
\pgfpathlineto{\pgfqpoint{2.834445in}{1.603938in}}%
\pgfpathclose%
\pgfusepath{fill}%
\end{pgfscope}%
\begin{pgfscope}%
\pgfpathrectangle{\pgfqpoint{1.150000in}{0.150000in}}{\pgfqpoint{5.700000in}{5.700000in}}%
\pgfusepath{clip}%
\pgfsetbuttcap%
\pgfsetroundjoin%
\definecolor{currentfill}{rgb}{0.183898,0.422383,0.556944}%
\pgfsetfillcolor{currentfill}%
\pgfsetfillopacity{0.700000}%
\pgfsetlinewidth{0.000000pt}%
\definecolor{currentstroke}{rgb}{0.000000,0.000000,0.000000}%
\pgfsetstrokecolor{currentstroke}%
\pgfsetdash{}{0pt}%
\pgfpathmoveto{\pgfqpoint{5.479178in}{2.426187in}}%
\pgfpathlineto{\pgfqpoint{5.493802in}{2.430874in}}%
\pgfpathlineto{\pgfqpoint{5.508439in}{2.435632in}}%
\pgfpathlineto{\pgfqpoint{5.523089in}{2.440460in}}%
\pgfpathlineto{\pgfqpoint{5.537753in}{2.445359in}}%
\pgfpathlineto{\pgfqpoint{5.545235in}{2.450180in}}%
\pgfpathlineto{\pgfqpoint{5.552707in}{2.454906in}}%
\pgfpathlineto{\pgfqpoint{5.560170in}{2.459541in}}%
\pgfpathlineto{\pgfqpoint{5.567624in}{2.464089in}}%
\pgfpathlineto{\pgfqpoint{5.552979in}{2.459370in}}%
\pgfpathlineto{\pgfqpoint{5.538348in}{2.454722in}}%
\pgfpathlineto{\pgfqpoint{5.523729in}{2.450144in}}%
\pgfpathlineto{\pgfqpoint{5.509125in}{2.445636in}}%
\pgfpathlineto{\pgfqpoint{5.501651in}{2.440901in}}%
\pgfpathlineto{\pgfqpoint{5.494169in}{2.436083in}}%
\pgfpathlineto{\pgfqpoint{5.486678in}{2.431180in}}%
\pgfpathlineto{\pgfqpoint{5.479178in}{2.426187in}}%
\pgfpathclose%
\pgfusepath{fill}%
\end{pgfscope}%
\begin{pgfscope}%
\pgfpathrectangle{\pgfqpoint{1.150000in}{0.150000in}}{\pgfqpoint{5.700000in}{5.700000in}}%
\pgfusepath{clip}%
\pgfsetbuttcap%
\pgfsetroundjoin%
\definecolor{currentfill}{rgb}{0.282327,0.094955,0.417331}%
\pgfsetfillcolor{currentfill}%
\pgfsetfillopacity{0.700000}%
\pgfsetlinewidth{0.000000pt}%
\definecolor{currentstroke}{rgb}{0.000000,0.000000,0.000000}%
\pgfsetstrokecolor{currentstroke}%
\pgfsetdash{}{0pt}%
\pgfpathmoveto{\pgfqpoint{2.632788in}{1.699086in}}%
\pgfpathlineto{\pgfqpoint{2.646669in}{1.689729in}}%
\pgfpathlineto{\pgfqpoint{2.660552in}{1.680467in}}%
\pgfpathlineto{\pgfqpoint{2.674436in}{1.671301in}}%
\pgfpathlineto{\pgfqpoint{2.688321in}{1.662230in}}%
\pgfpathlineto{\pgfqpoint{2.697155in}{1.663811in}}%
\pgfpathlineto{\pgfqpoint{2.705974in}{1.665635in}}%
\pgfpathlineto{\pgfqpoint{2.714778in}{1.667697in}}%
\pgfpathlineto{\pgfqpoint{2.723567in}{1.669990in}}%
\pgfpathlineto{\pgfqpoint{2.709715in}{1.678665in}}%
\pgfpathlineto{\pgfqpoint{2.695864in}{1.687435in}}%
\pgfpathlineto{\pgfqpoint{2.682015in}{1.696300in}}%
\pgfpathlineto{\pgfqpoint{2.668167in}{1.705261in}}%
\pgfpathlineto{\pgfqpoint{2.659346in}{1.703356in}}%
\pgfpathlineto{\pgfqpoint{2.650509in}{1.701688in}}%
\pgfpathlineto{\pgfqpoint{2.641656in}{1.700262in}}%
\pgfpathlineto{\pgfqpoint{2.632788in}{1.699086in}}%
\pgfpathclose%
\pgfusepath{fill}%
\end{pgfscope}%
\begin{pgfscope}%
\pgfpathrectangle{\pgfqpoint{1.150000in}{0.150000in}}{\pgfqpoint{5.700000in}{5.700000in}}%
\pgfusepath{clip}%
\pgfsetbuttcap%
\pgfsetroundjoin%
\definecolor{currentfill}{rgb}{0.272594,0.025563,0.353093}%
\pgfsetfillcolor{currentfill}%
\pgfsetfillopacity{0.700000}%
\pgfsetlinewidth{0.000000pt}%
\definecolor{currentstroke}{rgb}{0.000000,0.000000,0.000000}%
\pgfsetstrokecolor{currentstroke}%
\pgfsetdash{}{0pt}%
\pgfpathmoveto{\pgfqpoint{3.648497in}{1.548201in}}%
\pgfpathlineto{\pgfqpoint{3.662449in}{1.545869in}}%
\pgfpathlineto{\pgfqpoint{3.676408in}{1.543612in}}%
\pgfpathlineto{\pgfqpoint{3.690374in}{1.541431in}}%
\pgfpathlineto{\pgfqpoint{3.704347in}{1.539325in}}%
\pgfpathlineto{\pgfqpoint{3.712604in}{1.549255in}}%
\pgfpathlineto{\pgfqpoint{3.720854in}{1.559217in}}%
\pgfpathlineto{\pgfqpoint{3.729099in}{1.569210in}}%
\pgfpathlineto{\pgfqpoint{3.737337in}{1.579230in}}%
\pgfpathlineto{\pgfqpoint{3.723376in}{1.581088in}}%
\pgfpathlineto{\pgfqpoint{3.709422in}{1.583023in}}%
\pgfpathlineto{\pgfqpoint{3.695476in}{1.585033in}}%
\pgfpathlineto{\pgfqpoint{3.681536in}{1.587119in}}%
\pgfpathlineto{\pgfqpoint{3.673285in}{1.577338in}}%
\pgfpathlineto{\pgfqpoint{3.665029in}{1.567590in}}%
\pgfpathlineto{\pgfqpoint{3.656766in}{1.557876in}}%
\pgfpathlineto{\pgfqpoint{3.648497in}{1.548201in}}%
\pgfpathclose%
\pgfusepath{fill}%
\end{pgfscope}%
\begin{pgfscope}%
\pgfpathrectangle{\pgfqpoint{1.150000in}{0.150000in}}{\pgfqpoint{5.700000in}{5.700000in}}%
\pgfusepath{clip}%
\pgfsetbuttcap%
\pgfsetroundjoin%
\definecolor{currentfill}{rgb}{0.190631,0.407061,0.556089}%
\pgfsetfillcolor{currentfill}%
\pgfsetfillopacity{0.700000}%
\pgfsetlinewidth{0.000000pt}%
\definecolor{currentstroke}{rgb}{0.000000,0.000000,0.000000}%
\pgfsetstrokecolor{currentstroke}%
\pgfsetdash{}{0pt}%
\pgfpathmoveto{\pgfqpoint{5.390658in}{2.386587in}}%
\pgfpathlineto{\pgfqpoint{5.405246in}{2.391150in}}%
\pgfpathlineto{\pgfqpoint{5.419847in}{2.395783in}}%
\pgfpathlineto{\pgfqpoint{5.434461in}{2.400487in}}%
\pgfpathlineto{\pgfqpoint{5.449089in}{2.405262in}}%
\pgfpathlineto{\pgfqpoint{5.456625in}{2.410642in}}%
\pgfpathlineto{\pgfqpoint{5.464152in}{2.415922in}}%
\pgfpathlineto{\pgfqpoint{5.471669in}{2.421102in}}%
\pgfpathlineto{\pgfqpoint{5.479178in}{2.426187in}}%
\pgfpathlineto{\pgfqpoint{5.464568in}{2.421570in}}%
\pgfpathlineto{\pgfqpoint{5.449971in}{2.417024in}}%
\pgfpathlineto{\pgfqpoint{5.435387in}{2.412548in}}%
\pgfpathlineto{\pgfqpoint{5.420816in}{2.408142in}}%
\pgfpathlineto{\pgfqpoint{5.413290in}{2.402891in}}%
\pgfpathlineto{\pgfqpoint{5.405754in}{2.397551in}}%
\pgfpathlineto{\pgfqpoint{5.398210in}{2.392117in}}%
\pgfpathlineto{\pgfqpoint{5.390658in}{2.386587in}}%
\pgfpathclose%
\pgfusepath{fill}%
\end{pgfscope}%
\begin{pgfscope}%
\pgfpathrectangle{\pgfqpoint{1.150000in}{0.150000in}}{\pgfqpoint{5.700000in}{5.700000in}}%
\pgfusepath{clip}%
\pgfsetbuttcap%
\pgfsetroundjoin%
\definecolor{currentfill}{rgb}{0.267004,0.004874,0.329415}%
\pgfsetfillcolor{currentfill}%
\pgfsetfillopacity{0.700000}%
\pgfsetlinewidth{0.000000pt}%
\definecolor{currentstroke}{rgb}{0.000000,0.000000,0.000000}%
\pgfsetstrokecolor{currentstroke}%
\pgfsetdash{}{0pt}%
\pgfpathmoveto{\pgfqpoint{3.180682in}{1.509794in}}%
\pgfpathlineto{\pgfqpoint{3.194567in}{1.504384in}}%
\pgfpathlineto{\pgfqpoint{3.208456in}{1.499055in}}%
\pgfpathlineto{\pgfqpoint{3.222349in}{1.493808in}}%
\pgfpathlineto{\pgfqpoint{3.236248in}{1.488642in}}%
\pgfpathlineto{\pgfqpoint{3.244721in}{1.495458in}}%
\pgfpathlineto{\pgfqpoint{3.253184in}{1.502408in}}%
\pgfpathlineto{\pgfqpoint{3.261639in}{1.509486in}}%
\pgfpathlineto{\pgfqpoint{3.270085in}{1.516688in}}%
\pgfpathlineto{\pgfqpoint{3.256207in}{1.521525in}}%
\pgfpathlineto{\pgfqpoint{3.242334in}{1.526444in}}%
\pgfpathlineto{\pgfqpoint{3.228466in}{1.531444in}}%
\pgfpathlineto{\pgfqpoint{3.214602in}{1.536525in}}%
\pgfpathlineto{\pgfqpoint{3.206136in}{1.529644in}}%
\pgfpathlineto{\pgfqpoint{3.197661in}{1.522892in}}%
\pgfpathlineto{\pgfqpoint{3.189176in}{1.516274in}}%
\pgfpathlineto{\pgfqpoint{3.180682in}{1.509794in}}%
\pgfpathclose%
\pgfusepath{fill}%
\end{pgfscope}%
\begin{pgfscope}%
\pgfpathrectangle{\pgfqpoint{1.150000in}{0.150000in}}{\pgfqpoint{5.700000in}{5.700000in}}%
\pgfusepath{clip}%
\pgfsetbuttcap%
\pgfsetroundjoin%
\definecolor{currentfill}{rgb}{0.279574,0.170599,0.479997}%
\pgfsetfillcolor{currentfill}%
\pgfsetfillopacity{0.700000}%
\pgfsetlinewidth{0.000000pt}%
\definecolor{currentstroke}{rgb}{0.000000,0.000000,0.000000}%
\pgfsetstrokecolor{currentstroke}%
\pgfsetdash{}{0pt}%
\pgfpathmoveto{\pgfqpoint{2.374479in}{1.867044in}}%
\pgfpathlineto{\pgfqpoint{2.388399in}{1.855639in}}%
\pgfpathlineto{\pgfqpoint{2.402318in}{1.844340in}}%
\pgfpathlineto{\pgfqpoint{2.416236in}{1.833148in}}%
\pgfpathlineto{\pgfqpoint{2.430153in}{1.822061in}}%
\pgfpathlineto{\pgfqpoint{2.439206in}{1.820889in}}%
\pgfpathlineto{\pgfqpoint{2.448239in}{1.820009in}}%
\pgfpathlineto{\pgfqpoint{2.457253in}{1.819414in}}%
\pgfpathlineto{\pgfqpoint{2.466249in}{1.819099in}}%
\pgfpathlineto{\pgfqpoint{2.452371in}{1.829763in}}%
\pgfpathlineto{\pgfqpoint{2.438493in}{1.840533in}}%
\pgfpathlineto{\pgfqpoint{2.424614in}{1.851408in}}%
\pgfpathlineto{\pgfqpoint{2.410735in}{1.862390in}}%
\pgfpathlineto{\pgfqpoint{2.401700in}{1.863120in}}%
\pgfpathlineto{\pgfqpoint{2.392646in}{1.864135in}}%
\pgfpathlineto{\pgfqpoint{2.383572in}{1.865440in}}%
\pgfpathlineto{\pgfqpoint{2.374479in}{1.867044in}}%
\pgfpathclose%
\pgfusepath{fill}%
\end{pgfscope}%
\begin{pgfscope}%
\pgfpathrectangle{\pgfqpoint{1.150000in}{0.150000in}}{\pgfqpoint{5.700000in}{5.700000in}}%
\pgfusepath{clip}%
\pgfsetbuttcap%
\pgfsetroundjoin%
\definecolor{currentfill}{rgb}{0.195860,0.395433,0.555276}%
\pgfsetfillcolor{currentfill}%
\pgfsetfillopacity{0.700000}%
\pgfsetlinewidth{0.000000pt}%
\definecolor{currentstroke}{rgb}{0.000000,0.000000,0.000000}%
\pgfsetstrokecolor{currentstroke}%
\pgfsetdash{}{0pt}%
\pgfpathmoveto{\pgfqpoint{5.302073in}{2.345360in}}%
\pgfpathlineto{\pgfqpoint{5.316625in}{2.349776in}}%
\pgfpathlineto{\pgfqpoint{5.331189in}{2.354263in}}%
\pgfpathlineto{\pgfqpoint{5.345767in}{2.358821in}}%
\pgfpathlineto{\pgfqpoint{5.360358in}{2.363449in}}%
\pgfpathlineto{\pgfqpoint{5.367946in}{2.369391in}}%
\pgfpathlineto{\pgfqpoint{5.375525in}{2.375227in}}%
\pgfpathlineto{\pgfqpoint{5.383096in}{2.380958in}}%
\pgfpathlineto{\pgfqpoint{5.390658in}{2.386587in}}%
\pgfpathlineto{\pgfqpoint{5.376083in}{2.382094in}}%
\pgfpathlineto{\pgfqpoint{5.361521in}{2.377672in}}%
\pgfpathlineto{\pgfqpoint{5.346973in}{2.373321in}}%
\pgfpathlineto{\pgfqpoint{5.332437in}{2.369040in}}%
\pgfpathlineto{\pgfqpoint{5.324859in}{2.363267in}}%
\pgfpathlineto{\pgfqpoint{5.317272in}{2.357398in}}%
\pgfpathlineto{\pgfqpoint{5.309677in}{2.351430in}}%
\pgfpathlineto{\pgfqpoint{5.302073in}{2.345360in}}%
\pgfpathclose%
\pgfusepath{fill}%
\end{pgfscope}%
\begin{pgfscope}%
\pgfpathrectangle{\pgfqpoint{1.150000in}{0.150000in}}{\pgfqpoint{5.700000in}{5.700000in}}%
\pgfusepath{clip}%
\pgfsetbuttcap%
\pgfsetroundjoin%
\definecolor{currentfill}{rgb}{0.269944,0.014625,0.341379}%
\pgfsetfillcolor{currentfill}%
\pgfsetfillopacity{0.700000}%
\pgfsetlinewidth{0.000000pt}%
\definecolor{currentstroke}{rgb}{0.000000,0.000000,0.000000}%
\pgfsetstrokecolor{currentstroke}%
\pgfsetdash{}{0pt}%
\pgfpathmoveto{\pgfqpoint{3.035489in}{1.534431in}}%
\pgfpathlineto{\pgfqpoint{3.049366in}{1.528005in}}%
\pgfpathlineto{\pgfqpoint{3.063246in}{1.521663in}}%
\pgfpathlineto{\pgfqpoint{3.077130in}{1.515405in}}%
\pgfpathlineto{\pgfqpoint{3.091017in}{1.509231in}}%
\pgfpathlineto{\pgfqpoint{3.099575in}{1.514746in}}%
\pgfpathlineto{\pgfqpoint{3.108122in}{1.520426in}}%
\pgfpathlineto{\pgfqpoint{3.116659in}{1.526266in}}%
\pgfpathlineto{\pgfqpoint{3.125185in}{1.532260in}}%
\pgfpathlineto{\pgfqpoint{3.111321in}{1.538084in}}%
\pgfpathlineto{\pgfqpoint{3.097461in}{1.543992in}}%
\pgfpathlineto{\pgfqpoint{3.083605in}{1.549984in}}%
\pgfpathlineto{\pgfqpoint{3.069753in}{1.556060in}}%
\pgfpathlineto{\pgfqpoint{3.061203in}{1.550408in}}%
\pgfpathlineto{\pgfqpoint{3.052643in}{1.544916in}}%
\pgfpathlineto{\pgfqpoint{3.044071in}{1.539589in}}%
\pgfpathlineto{\pgfqpoint{3.035489in}{1.534431in}}%
\pgfpathclose%
\pgfusepath{fill}%
\end{pgfscope}%
\begin{pgfscope}%
\pgfpathrectangle{\pgfqpoint{1.150000in}{0.150000in}}{\pgfqpoint{5.700000in}{5.700000in}}%
\pgfusepath{clip}%
\pgfsetbuttcap%
\pgfsetroundjoin%
\definecolor{currentfill}{rgb}{0.203063,0.379716,0.553925}%
\pgfsetfillcolor{currentfill}%
\pgfsetfillopacity{0.700000}%
\pgfsetlinewidth{0.000000pt}%
\definecolor{currentstroke}{rgb}{0.000000,0.000000,0.000000}%
\pgfsetstrokecolor{currentstroke}%
\pgfsetdash{}{0pt}%
\pgfpathmoveto{\pgfqpoint{5.213434in}{2.302603in}}%
\pgfpathlineto{\pgfqpoint{5.227949in}{2.306850in}}%
\pgfpathlineto{\pgfqpoint{5.242476in}{2.311167in}}%
\pgfpathlineto{\pgfqpoint{5.257017in}{2.315556in}}%
\pgfpathlineto{\pgfqpoint{5.271570in}{2.320015in}}%
\pgfpathlineto{\pgfqpoint{5.279209in}{2.326516in}}%
\pgfpathlineto{\pgfqpoint{5.286839in}{2.332905in}}%
\pgfpathlineto{\pgfqpoint{5.294460in}{2.339186in}}%
\pgfpathlineto{\pgfqpoint{5.302073in}{2.345360in}}%
\pgfpathlineto{\pgfqpoint{5.287534in}{2.341015in}}%
\pgfpathlineto{\pgfqpoint{5.273009in}{2.336740in}}%
\pgfpathlineto{\pgfqpoint{5.258496in}{2.332536in}}%
\pgfpathlineto{\pgfqpoint{5.243995in}{2.328402in}}%
\pgfpathlineto{\pgfqpoint{5.236368in}{2.322107in}}%
\pgfpathlineto{\pgfqpoint{5.228732in}{2.315710in}}%
\pgfpathlineto{\pgfqpoint{5.221087in}{2.309210in}}%
\pgfpathlineto{\pgfqpoint{5.213434in}{2.302603in}}%
\pgfpathclose%
\pgfusepath{fill}%
\end{pgfscope}%
\begin{pgfscope}%
\pgfpathrectangle{\pgfqpoint{1.150000in}{0.150000in}}{\pgfqpoint{5.700000in}{5.700000in}}%
\pgfusepath{clip}%
\pgfsetbuttcap%
\pgfsetroundjoin%
\definecolor{currentfill}{rgb}{0.269944,0.014625,0.341379}%
\pgfsetfillcolor{currentfill}%
\pgfsetfillopacity{0.700000}%
\pgfsetlinewidth{0.000000pt}%
\definecolor{currentstroke}{rgb}{0.000000,0.000000,0.000000}%
\pgfsetstrokecolor{currentstroke}%
\pgfsetdash{}{0pt}%
\pgfpathmoveto{\pgfqpoint{3.559560in}{1.521125in}}%
\pgfpathlineto{\pgfqpoint{3.573500in}{1.518220in}}%
\pgfpathlineto{\pgfqpoint{3.587445in}{1.515392in}}%
\pgfpathlineto{\pgfqpoint{3.601398in}{1.512640in}}%
\pgfpathlineto{\pgfqpoint{3.615357in}{1.509965in}}%
\pgfpathlineto{\pgfqpoint{3.623651in}{1.519447in}}%
\pgfpathlineto{\pgfqpoint{3.631940in}{1.528983in}}%
\pgfpathlineto{\pgfqpoint{3.640221in}{1.538569in}}%
\pgfpathlineto{\pgfqpoint{3.648497in}{1.548201in}}%
\pgfpathlineto{\pgfqpoint{3.634551in}{1.550610in}}%
\pgfpathlineto{\pgfqpoint{3.620613in}{1.553095in}}%
\pgfpathlineto{\pgfqpoint{3.606681in}{1.555656in}}%
\pgfpathlineto{\pgfqpoint{3.592755in}{1.558293in}}%
\pgfpathlineto{\pgfqpoint{3.584466in}{1.548921in}}%
\pgfpathlineto{\pgfqpoint{3.576171in}{1.539599in}}%
\pgfpathlineto{\pgfqpoint{3.567869in}{1.530332in}}%
\pgfpathlineto{\pgfqpoint{3.559560in}{1.521125in}}%
\pgfpathclose%
\pgfusepath{fill}%
\end{pgfscope}%
\begin{pgfscope}%
\pgfpathrectangle{\pgfqpoint{1.150000in}{0.150000in}}{\pgfqpoint{5.700000in}{5.700000in}}%
\pgfusepath{clip}%
\pgfsetbuttcap%
\pgfsetroundjoin%
\definecolor{currentfill}{rgb}{0.267004,0.004874,0.329415}%
\pgfsetfillcolor{currentfill}%
\pgfsetfillopacity{0.700000}%
\pgfsetlinewidth{0.000000pt}%
\definecolor{currentstroke}{rgb}{0.000000,0.000000,0.000000}%
\pgfsetstrokecolor{currentstroke}%
\pgfsetdash{}{0pt}%
\pgfpathmoveto{\pgfqpoint{3.325645in}{1.498144in}}%
\pgfpathlineto{\pgfqpoint{3.339548in}{1.493707in}}%
\pgfpathlineto{\pgfqpoint{3.353456in}{1.489350in}}%
\pgfpathlineto{\pgfqpoint{3.367370in}{1.485072in}}%
\pgfpathlineto{\pgfqpoint{3.381288in}{1.480873in}}%
\pgfpathlineto{\pgfqpoint{3.389688in}{1.488826in}}%
\pgfpathlineto{\pgfqpoint{3.398080in}{1.496882in}}%
\pgfpathlineto{\pgfqpoint{3.406463in}{1.505037in}}%
\pgfpathlineto{\pgfqpoint{3.414840in}{1.513287in}}%
\pgfpathlineto{\pgfqpoint{3.400939in}{1.517178in}}%
\pgfpathlineto{\pgfqpoint{3.387043in}{1.521147in}}%
\pgfpathlineto{\pgfqpoint{3.373153in}{1.525196in}}%
\pgfpathlineto{\pgfqpoint{3.359268in}{1.529325in}}%
\pgfpathlineto{\pgfqpoint{3.350875in}{1.521375in}}%
\pgfpathlineto{\pgfqpoint{3.342473in}{1.513525in}}%
\pgfpathlineto{\pgfqpoint{3.334064in}{1.505780in}}%
\pgfpathlineto{\pgfqpoint{3.325645in}{1.498144in}}%
\pgfpathclose%
\pgfusepath{fill}%
\end{pgfscope}%
\begin{pgfscope}%
\pgfpathrectangle{\pgfqpoint{1.150000in}{0.150000in}}{\pgfqpoint{5.700000in}{5.700000in}}%
\pgfusepath{clip}%
\pgfsetbuttcap%
\pgfsetroundjoin%
\definecolor{currentfill}{rgb}{0.278012,0.180367,0.486697}%
\pgfsetfillcolor{currentfill}%
\pgfsetfillopacity{0.700000}%
\pgfsetlinewidth{0.000000pt}%
\definecolor{currentstroke}{rgb}{0.000000,0.000000,0.000000}%
\pgfsetstrokecolor{currentstroke}%
\pgfsetdash{}{0pt}%
\pgfpathmoveto{\pgfqpoint{4.325930in}{1.827540in}}%
\pgfpathlineto{\pgfqpoint{4.340085in}{1.828866in}}%
\pgfpathlineto{\pgfqpoint{4.354251in}{1.830265in}}%
\pgfpathlineto{\pgfqpoint{4.368426in}{1.831736in}}%
\pgfpathlineto{\pgfqpoint{4.382611in}{1.833279in}}%
\pgfpathlineto{\pgfqpoint{4.390638in}{1.843960in}}%
\pgfpathlineto{\pgfqpoint{4.398659in}{1.854565in}}%
\pgfpathlineto{\pgfqpoint{4.406675in}{1.865093in}}%
\pgfpathlineto{\pgfqpoint{4.414685in}{1.875542in}}%
\pgfpathlineto{\pgfqpoint{4.400507in}{1.873896in}}%
\pgfpathlineto{\pgfqpoint{4.386340in}{1.872323in}}%
\pgfpathlineto{\pgfqpoint{4.372182in}{1.870822in}}%
\pgfpathlineto{\pgfqpoint{4.358035in}{1.869393in}}%
\pgfpathlineto{\pgfqpoint{4.350017in}{1.859038in}}%
\pgfpathlineto{\pgfqpoint{4.341994in}{1.848610in}}%
\pgfpathlineto{\pgfqpoint{4.333965in}{1.838111in}}%
\pgfpathlineto{\pgfqpoint{4.325930in}{1.827540in}}%
\pgfpathclose%
\pgfusepath{fill}%
\end{pgfscope}%
\begin{pgfscope}%
\pgfpathrectangle{\pgfqpoint{1.150000in}{0.150000in}}{\pgfqpoint{5.700000in}{5.700000in}}%
\pgfusepath{clip}%
\pgfsetbuttcap%
\pgfsetroundjoin%
\definecolor{currentfill}{rgb}{0.281412,0.155834,0.469201}%
\pgfsetfillcolor{currentfill}%
\pgfsetfillopacity{0.700000}%
\pgfsetlinewidth{0.000000pt}%
\definecolor{currentstroke}{rgb}{0.000000,0.000000,0.000000}%
\pgfsetstrokecolor{currentstroke}%
\pgfsetdash{}{0pt}%
\pgfpathmoveto{\pgfqpoint{4.237180in}{1.780489in}}%
\pgfpathlineto{\pgfqpoint{4.251304in}{1.781402in}}%
\pgfpathlineto{\pgfqpoint{4.265438in}{1.782387in}}%
\pgfpathlineto{\pgfqpoint{4.279581in}{1.783445in}}%
\pgfpathlineto{\pgfqpoint{4.293734in}{1.784576in}}%
\pgfpathlineto{\pgfqpoint{4.301792in}{1.795416in}}%
\pgfpathlineto{\pgfqpoint{4.309844in}{1.806191in}}%
\pgfpathlineto{\pgfqpoint{4.317890in}{1.816900in}}%
\pgfpathlineto{\pgfqpoint{4.325930in}{1.827540in}}%
\pgfpathlineto{\pgfqpoint{4.311785in}{1.826286in}}%
\pgfpathlineto{\pgfqpoint{4.297649in}{1.825104in}}%
\pgfpathlineto{\pgfqpoint{4.283523in}{1.823995in}}%
\pgfpathlineto{\pgfqpoint{4.269407in}{1.822958in}}%
\pgfpathlineto{\pgfqpoint{4.261359in}{1.812434in}}%
\pgfpathlineto{\pgfqpoint{4.253305in}{1.801847in}}%
\pgfpathlineto{\pgfqpoint{4.245245in}{1.791198in}}%
\pgfpathlineto{\pgfqpoint{4.237180in}{1.780489in}}%
\pgfpathclose%
\pgfusepath{fill}%
\end{pgfscope}%
\begin{pgfscope}%
\pgfpathrectangle{\pgfqpoint{1.150000in}{0.150000in}}{\pgfqpoint{5.700000in}{5.700000in}}%
\pgfusepath{clip}%
\pgfsetbuttcap%
\pgfsetroundjoin%
\definecolor{currentfill}{rgb}{0.274128,0.199721,0.498911}%
\pgfsetfillcolor{currentfill}%
\pgfsetfillopacity{0.700000}%
\pgfsetlinewidth{0.000000pt}%
\definecolor{currentstroke}{rgb}{0.000000,0.000000,0.000000}%
\pgfsetstrokecolor{currentstroke}%
\pgfsetdash{}{0pt}%
\pgfpathmoveto{\pgfqpoint{4.414685in}{1.875542in}}%
\pgfpathlineto{\pgfqpoint{4.428872in}{1.877260in}}%
\pgfpathlineto{\pgfqpoint{4.443071in}{1.879050in}}%
\pgfpathlineto{\pgfqpoint{4.457279in}{1.880911in}}%
\pgfpathlineto{\pgfqpoint{4.471498in}{1.882845in}}%
\pgfpathlineto{\pgfqpoint{4.479494in}{1.893304in}}%
\pgfpathlineto{\pgfqpoint{4.487484in}{1.903678in}}%
\pgfpathlineto{\pgfqpoint{4.495468in}{1.913964in}}%
\pgfpathlineto{\pgfqpoint{4.503446in}{1.924164in}}%
\pgfpathlineto{\pgfqpoint{4.489234in}{1.922149in}}%
\pgfpathlineto{\pgfqpoint{4.475034in}{1.920205in}}%
\pgfpathlineto{\pgfqpoint{4.460844in}{1.918334in}}%
\pgfpathlineto{\pgfqpoint{4.446664in}{1.916534in}}%
\pgfpathlineto{\pgfqpoint{4.438678in}{1.906409in}}%
\pgfpathlineto{\pgfqpoint{4.430686in}{1.896201in}}%
\pgfpathlineto{\pgfqpoint{4.422688in}{1.885912in}}%
\pgfpathlineto{\pgfqpoint{4.414685in}{1.875542in}}%
\pgfpathclose%
\pgfusepath{fill}%
\end{pgfscope}%
\begin{pgfscope}%
\pgfpathrectangle{\pgfqpoint{1.150000in}{0.150000in}}{\pgfqpoint{5.700000in}{5.700000in}}%
\pgfusepath{clip}%
\pgfsetbuttcap%
\pgfsetroundjoin%
\definecolor{currentfill}{rgb}{0.210503,0.363727,0.552206}%
\pgfsetfillcolor{currentfill}%
\pgfsetfillopacity{0.700000}%
\pgfsetlinewidth{0.000000pt}%
\definecolor{currentstroke}{rgb}{0.000000,0.000000,0.000000}%
\pgfsetstrokecolor{currentstroke}%
\pgfsetdash{}{0pt}%
\pgfpathmoveto{\pgfqpoint{5.124751in}{2.258433in}}%
\pgfpathlineto{\pgfqpoint{5.139229in}{2.262488in}}%
\pgfpathlineto{\pgfqpoint{5.153719in}{2.266614in}}%
\pgfpathlineto{\pgfqpoint{5.168222in}{2.270810in}}%
\pgfpathlineto{\pgfqpoint{5.182737in}{2.275078in}}%
\pgfpathlineto{\pgfqpoint{5.190424in}{2.282128in}}%
\pgfpathlineto{\pgfqpoint{5.198103in}{2.289065in}}%
\pgfpathlineto{\pgfqpoint{5.205773in}{2.295889in}}%
\pgfpathlineto{\pgfqpoint{5.213434in}{2.302603in}}%
\pgfpathlineto{\pgfqpoint{5.198932in}{2.298427in}}%
\pgfpathlineto{\pgfqpoint{5.184443in}{2.294322in}}%
\pgfpathlineto{\pgfqpoint{5.169966in}{2.290288in}}%
\pgfpathlineto{\pgfqpoint{5.155502in}{2.286324in}}%
\pgfpathlineto{\pgfqpoint{5.147826in}{2.279510in}}%
\pgfpathlineto{\pgfqpoint{5.140143in}{2.272592in}}%
\pgfpathlineto{\pgfqpoint{5.132451in}{2.265567in}}%
\pgfpathlineto{\pgfqpoint{5.124751in}{2.258433in}}%
\pgfpathclose%
\pgfusepath{fill}%
\end{pgfscope}%
\begin{pgfscope}%
\pgfpathrectangle{\pgfqpoint{1.150000in}{0.150000in}}{\pgfqpoint{5.700000in}{5.700000in}}%
\pgfusepath{clip}%
\pgfsetbuttcap%
\pgfsetroundjoin%
\definecolor{currentfill}{rgb}{0.282884,0.135920,0.453427}%
\pgfsetfillcolor{currentfill}%
\pgfsetfillopacity{0.700000}%
\pgfsetlinewidth{0.000000pt}%
\definecolor{currentstroke}{rgb}{0.000000,0.000000,0.000000}%
\pgfsetstrokecolor{currentstroke}%
\pgfsetdash{}{0pt}%
\pgfpathmoveto{\pgfqpoint{4.148430in}{1.734740in}}%
\pgfpathlineto{\pgfqpoint{4.162525in}{1.735218in}}%
\pgfpathlineto{\pgfqpoint{4.176629in}{1.735769in}}%
\pgfpathlineto{\pgfqpoint{4.190742in}{1.736392in}}%
\pgfpathlineto{\pgfqpoint{4.204864in}{1.737088in}}%
\pgfpathlineto{\pgfqpoint{4.212952in}{1.748020in}}%
\pgfpathlineto{\pgfqpoint{4.221033in}{1.758898in}}%
\pgfpathlineto{\pgfqpoint{4.229109in}{1.769722in}}%
\pgfpathlineto{\pgfqpoint{4.237180in}{1.780489in}}%
\pgfpathlineto{\pgfqpoint{4.223066in}{1.779648in}}%
\pgfpathlineto{\pgfqpoint{4.208960in}{1.778881in}}%
\pgfpathlineto{\pgfqpoint{4.194865in}{1.778185in}}%
\pgfpathlineto{\pgfqpoint{4.180778in}{1.777563in}}%
\pgfpathlineto{\pgfqpoint{4.172699in}{1.766933in}}%
\pgfpathlineto{\pgfqpoint{4.164615in}{1.756251in}}%
\pgfpathlineto{\pgfqpoint{4.156525in}{1.745519in}}%
\pgfpathlineto{\pgfqpoint{4.148430in}{1.734740in}}%
\pgfpathclose%
\pgfusepath{fill}%
\end{pgfscope}%
\begin{pgfscope}%
\pgfpathrectangle{\pgfqpoint{1.150000in}{0.150000in}}{\pgfqpoint{5.700000in}{5.700000in}}%
\pgfusepath{clip}%
\pgfsetbuttcap%
\pgfsetroundjoin%
\definecolor{currentfill}{rgb}{0.267968,0.223549,0.512008}%
\pgfsetfillcolor{currentfill}%
\pgfsetfillopacity{0.700000}%
\pgfsetlinewidth{0.000000pt}%
\definecolor{currentstroke}{rgb}{0.000000,0.000000,0.000000}%
\pgfsetstrokecolor{currentstroke}%
\pgfsetdash{}{0pt}%
\pgfpathmoveto{\pgfqpoint{4.503446in}{1.924164in}}%
\pgfpathlineto{\pgfqpoint{4.517667in}{1.926251in}}%
\pgfpathlineto{\pgfqpoint{4.531900in}{1.928410in}}%
\pgfpathlineto{\pgfqpoint{4.546142in}{1.930641in}}%
\pgfpathlineto{\pgfqpoint{4.560396in}{1.932943in}}%
\pgfpathlineto{\pgfqpoint{4.568360in}{1.943123in}}%
\pgfpathlineto{\pgfqpoint{4.576317in}{1.953208in}}%
\pgfpathlineto{\pgfqpoint{4.584268in}{1.963199in}}%
\pgfpathlineto{\pgfqpoint{4.592213in}{1.973095in}}%
\pgfpathlineto{\pgfqpoint{4.577968in}{1.970732in}}%
\pgfpathlineto{\pgfqpoint{4.563733in}{1.968441in}}%
\pgfpathlineto{\pgfqpoint{4.549508in}{1.966221in}}%
\pgfpathlineto{\pgfqpoint{4.535295in}{1.964074in}}%
\pgfpathlineto{\pgfqpoint{4.527342in}{1.954230in}}%
\pgfpathlineto{\pgfqpoint{4.519383in}{1.944297in}}%
\pgfpathlineto{\pgfqpoint{4.511417in}{1.934275in}}%
\pgfpathlineto{\pgfqpoint{4.503446in}{1.924164in}}%
\pgfpathclose%
\pgfusepath{fill}%
\end{pgfscope}%
\begin{pgfscope}%
\pgfpathrectangle{\pgfqpoint{1.150000in}{0.150000in}}{\pgfqpoint{5.700000in}{5.700000in}}%
\pgfusepath{clip}%
\pgfsetbuttcap%
\pgfsetroundjoin%
\definecolor{currentfill}{rgb}{0.260571,0.246922,0.522828}%
\pgfsetfillcolor{currentfill}%
\pgfsetfillopacity{0.700000}%
\pgfsetlinewidth{0.000000pt}%
\definecolor{currentstroke}{rgb}{0.000000,0.000000,0.000000}%
\pgfsetstrokecolor{currentstroke}%
\pgfsetdash{}{0pt}%
\pgfpathmoveto{\pgfqpoint{4.592213in}{1.973095in}}%
\pgfpathlineto{\pgfqpoint{4.606470in}{1.975530in}}%
\pgfpathlineto{\pgfqpoint{4.620737in}{1.978036in}}%
\pgfpathlineto{\pgfqpoint{4.635016in}{1.980614in}}%
\pgfpathlineto{\pgfqpoint{4.649305in}{1.983263in}}%
\pgfpathlineto{\pgfqpoint{4.657235in}{1.993111in}}%
\pgfpathlineto{\pgfqpoint{4.665159in}{2.002857in}}%
\pgfpathlineto{\pgfqpoint{4.673076in}{2.012502in}}%
\pgfpathlineto{\pgfqpoint{4.680986in}{2.022046in}}%
\pgfpathlineto{\pgfqpoint{4.666705in}{2.019357in}}%
\pgfpathlineto{\pgfqpoint{4.652435in}{2.016740in}}%
\pgfpathlineto{\pgfqpoint{4.638175in}{2.014195in}}%
\pgfpathlineto{\pgfqpoint{4.623927in}{2.011721in}}%
\pgfpathlineto{\pgfqpoint{4.616009in}{2.002208in}}%
\pgfpathlineto{\pgfqpoint{4.608083in}{1.992600in}}%
\pgfpathlineto{\pgfqpoint{4.600152in}{1.982895in}}%
\pgfpathlineto{\pgfqpoint{4.592213in}{1.973095in}}%
\pgfpathclose%
\pgfusepath{fill}%
\end{pgfscope}%
\begin{pgfscope}%
\pgfpathrectangle{\pgfqpoint{1.150000in}{0.150000in}}{\pgfqpoint{5.700000in}{5.700000in}}%
\pgfusepath{clip}%
\pgfsetbuttcap%
\pgfsetroundjoin%
\definecolor{currentfill}{rgb}{0.283197,0.115680,0.436115}%
\pgfsetfillcolor{currentfill}%
\pgfsetfillopacity{0.700000}%
\pgfsetlinewidth{0.000000pt}%
\definecolor{currentstroke}{rgb}{0.000000,0.000000,0.000000}%
\pgfsetstrokecolor{currentstroke}%
\pgfsetdash{}{0pt}%
\pgfpathmoveto{\pgfqpoint{4.059673in}{1.690668in}}%
\pgfpathlineto{\pgfqpoint{4.073741in}{1.690688in}}%
\pgfpathlineto{\pgfqpoint{4.087816in}{1.690782in}}%
\pgfpathlineto{\pgfqpoint{4.101901in}{1.690949in}}%
\pgfpathlineto{\pgfqpoint{4.115995in}{1.691189in}}%
\pgfpathlineto{\pgfqpoint{4.124112in}{1.702138in}}%
\pgfpathlineto{\pgfqpoint{4.132223in}{1.713047in}}%
\pgfpathlineto{\pgfqpoint{4.140330in}{1.723916in}}%
\pgfpathlineto{\pgfqpoint{4.148430in}{1.734740in}}%
\pgfpathlineto{\pgfqpoint{4.134345in}{1.734336in}}%
\pgfpathlineto{\pgfqpoint{4.120268in}{1.734004in}}%
\pgfpathlineto{\pgfqpoint{4.106201in}{1.733745in}}%
\pgfpathlineto{\pgfqpoint{4.092142in}{1.733559in}}%
\pgfpathlineto{\pgfqpoint{4.084033in}{1.722892in}}%
\pgfpathlineto{\pgfqpoint{4.075919in}{1.712186in}}%
\pgfpathlineto{\pgfqpoint{4.067799in}{1.701444in}}%
\pgfpathlineto{\pgfqpoint{4.059673in}{1.690668in}}%
\pgfpathclose%
\pgfusepath{fill}%
\end{pgfscope}%
\begin{pgfscope}%
\pgfpathrectangle{\pgfqpoint{1.150000in}{0.150000in}}{\pgfqpoint{5.700000in}{5.700000in}}%
\pgfusepath{clip}%
\pgfsetbuttcap%
\pgfsetroundjoin%
\definecolor{currentfill}{rgb}{0.218130,0.347432,0.550038}%
\pgfsetfillcolor{currentfill}%
\pgfsetfillopacity{0.700000}%
\pgfsetlinewidth{0.000000pt}%
\definecolor{currentstroke}{rgb}{0.000000,0.000000,0.000000}%
\pgfsetstrokecolor{currentstroke}%
\pgfsetdash{}{0pt}%
\pgfpathmoveto{\pgfqpoint{5.036033in}{2.212990in}}%
\pgfpathlineto{\pgfqpoint{5.050474in}{2.216831in}}%
\pgfpathlineto{\pgfqpoint{5.064926in}{2.220742in}}%
\pgfpathlineto{\pgfqpoint{5.079391in}{2.224724in}}%
\pgfpathlineto{\pgfqpoint{5.093869in}{2.228778in}}%
\pgfpathlineto{\pgfqpoint{5.101602in}{2.236363in}}%
\pgfpathlineto{\pgfqpoint{5.109326in}{2.243833in}}%
\pgfpathlineto{\pgfqpoint{5.117043in}{2.251189in}}%
\pgfpathlineto{\pgfqpoint{5.124751in}{2.258433in}}%
\pgfpathlineto{\pgfqpoint{5.110286in}{2.254449in}}%
\pgfpathlineto{\pgfqpoint{5.095833in}{2.250536in}}%
\pgfpathlineto{\pgfqpoint{5.081393in}{2.246694in}}%
\pgfpathlineto{\pgfqpoint{5.066965in}{2.242923in}}%
\pgfpathlineto{\pgfqpoint{5.059244in}{2.235601in}}%
\pgfpathlineto{\pgfqpoint{5.051515in}{2.228173in}}%
\pgfpathlineto{\pgfqpoint{5.043778in}{2.220637in}}%
\pgfpathlineto{\pgfqpoint{5.036033in}{2.212990in}}%
\pgfpathclose%
\pgfusepath{fill}%
\end{pgfscope}%
\begin{pgfscope}%
\pgfpathrectangle{\pgfqpoint{1.150000in}{0.150000in}}{\pgfqpoint{5.700000in}{5.700000in}}%
\pgfusepath{clip}%
\pgfsetbuttcap%
\pgfsetroundjoin%
\definecolor{currentfill}{rgb}{0.252194,0.269783,0.531579}%
\pgfsetfillcolor{currentfill}%
\pgfsetfillopacity{0.700000}%
\pgfsetlinewidth{0.000000pt}%
\definecolor{currentstroke}{rgb}{0.000000,0.000000,0.000000}%
\pgfsetstrokecolor{currentstroke}%
\pgfsetdash{}{0pt}%
\pgfpathmoveto{\pgfqpoint{4.680986in}{2.022046in}}%
\pgfpathlineto{\pgfqpoint{4.695278in}{2.024806in}}%
\pgfpathlineto{\pgfqpoint{4.709582in}{2.027637in}}%
\pgfpathlineto{\pgfqpoint{4.723896in}{2.030540in}}%
\pgfpathlineto{\pgfqpoint{4.738223in}{2.033515in}}%
\pgfpathlineto{\pgfqpoint{4.746117in}{2.042983in}}%
\pgfpathlineto{\pgfqpoint{4.754005in}{2.052344in}}%
\pgfpathlineto{\pgfqpoint{4.761886in}{2.061599in}}%
\pgfpathlineto{\pgfqpoint{4.769760in}{2.070748in}}%
\pgfpathlineto{\pgfqpoint{4.755442in}{2.067756in}}%
\pgfpathlineto{\pgfqpoint{4.741136in}{2.064835in}}%
\pgfpathlineto{\pgfqpoint{4.726842in}{2.061986in}}%
\pgfpathlineto{\pgfqpoint{4.712559in}{2.059208in}}%
\pgfpathlineto{\pgfqpoint{4.704676in}{2.050069in}}%
\pgfpathlineto{\pgfqpoint{4.696786in}{2.040829in}}%
\pgfpathlineto{\pgfqpoint{4.688889in}{2.031488in}}%
\pgfpathlineto{\pgfqpoint{4.680986in}{2.022046in}}%
\pgfpathclose%
\pgfusepath{fill}%
\end{pgfscope}%
\begin{pgfscope}%
\pgfpathrectangle{\pgfqpoint{1.150000in}{0.150000in}}{\pgfqpoint{5.700000in}{5.700000in}}%
\pgfusepath{clip}%
\pgfsetbuttcap%
\pgfsetroundjoin%
\definecolor{currentfill}{rgb}{0.227802,0.326594,0.546532}%
\pgfsetfillcolor{currentfill}%
\pgfsetfillopacity{0.700000}%
\pgfsetlinewidth{0.000000pt}%
\definecolor{currentstroke}{rgb}{0.000000,0.000000,0.000000}%
\pgfsetstrokecolor{currentstroke}%
\pgfsetdash{}{0pt}%
\pgfpathmoveto{\pgfqpoint{4.947290in}{2.166436in}}%
\pgfpathlineto{\pgfqpoint{4.961693in}{2.170039in}}%
\pgfpathlineto{\pgfqpoint{4.976108in}{2.173714in}}%
\pgfpathlineto{\pgfqpoint{4.990535in}{2.177460in}}%
\pgfpathlineto{\pgfqpoint{5.004975in}{2.181277in}}%
\pgfpathlineto{\pgfqpoint{5.012751in}{2.189377in}}%
\pgfpathlineto{\pgfqpoint{5.020520in}{2.197362in}}%
\pgfpathlineto{\pgfqpoint{5.028281in}{2.205232in}}%
\pgfpathlineto{\pgfqpoint{5.036033in}{2.212990in}}%
\pgfpathlineto{\pgfqpoint{5.021605in}{2.209221in}}%
\pgfpathlineto{\pgfqpoint{5.007190in}{2.205522in}}%
\pgfpathlineto{\pgfqpoint{4.992786in}{2.201895in}}%
\pgfpathlineto{\pgfqpoint{4.978394in}{2.198338in}}%
\pgfpathlineto{\pgfqpoint{4.970630in}{2.190525in}}%
\pgfpathlineto{\pgfqpoint{4.962858in}{2.182604in}}%
\pgfpathlineto{\pgfqpoint{4.955078in}{2.174575in}}%
\pgfpathlineto{\pgfqpoint{4.947290in}{2.166436in}}%
\pgfpathclose%
\pgfusepath{fill}%
\end{pgfscope}%
\begin{pgfscope}%
\pgfpathrectangle{\pgfqpoint{1.150000in}{0.150000in}}{\pgfqpoint{5.700000in}{5.700000in}}%
\pgfusepath{clip}%
\pgfsetbuttcap%
\pgfsetroundjoin%
\definecolor{currentfill}{rgb}{0.244972,0.287675,0.537260}%
\pgfsetfillcolor{currentfill}%
\pgfsetfillopacity{0.700000}%
\pgfsetlinewidth{0.000000pt}%
\definecolor{currentstroke}{rgb}{0.000000,0.000000,0.000000}%
\pgfsetstrokecolor{currentstroke}%
\pgfsetdash{}{0pt}%
\pgfpathmoveto{\pgfqpoint{4.769760in}{2.070748in}}%
\pgfpathlineto{\pgfqpoint{4.784088in}{2.073811in}}%
\pgfpathlineto{\pgfqpoint{4.798429in}{2.076946in}}%
\pgfpathlineto{\pgfqpoint{4.812780in}{2.080152in}}%
\pgfpathlineto{\pgfqpoint{4.827144in}{2.083429in}}%
\pgfpathlineto{\pgfqpoint{4.835001in}{2.092476in}}%
\pgfpathlineto{\pgfqpoint{4.842851in}{2.101412in}}%
\pgfpathlineto{\pgfqpoint{4.850694in}{2.110238in}}%
\pgfpathlineto{\pgfqpoint{4.858530in}{2.118953in}}%
\pgfpathlineto{\pgfqpoint{4.844176in}{2.115680in}}%
\pgfpathlineto{\pgfqpoint{4.829834in}{2.112477in}}%
\pgfpathlineto{\pgfqpoint{4.815503in}{2.109347in}}%
\pgfpathlineto{\pgfqpoint{4.801184in}{2.106287in}}%
\pgfpathlineto{\pgfqpoint{4.793339in}{2.097559in}}%
\pgfpathlineto{\pgfqpoint{4.785486in}{2.088727in}}%
\pgfpathlineto{\pgfqpoint{4.777626in}{2.079790in}}%
\pgfpathlineto{\pgfqpoint{4.769760in}{2.070748in}}%
\pgfpathclose%
\pgfusepath{fill}%
\end{pgfscope}%
\begin{pgfscope}%
\pgfpathrectangle{\pgfqpoint{1.150000in}{0.150000in}}{\pgfqpoint{5.700000in}{5.700000in}}%
\pgfusepath{clip}%
\pgfsetbuttcap%
\pgfsetroundjoin%
\definecolor{currentfill}{rgb}{0.235526,0.309527,0.542944}%
\pgfsetfillcolor{currentfill}%
\pgfsetfillopacity{0.700000}%
\pgfsetlinewidth{0.000000pt}%
\definecolor{currentstroke}{rgb}{0.000000,0.000000,0.000000}%
\pgfsetstrokecolor{currentstroke}%
\pgfsetdash{}{0pt}%
\pgfpathmoveto{\pgfqpoint{4.858530in}{2.118953in}}%
\pgfpathlineto{\pgfqpoint{4.872895in}{2.122298in}}%
\pgfpathlineto{\pgfqpoint{4.887273in}{2.125714in}}%
\pgfpathlineto{\pgfqpoint{4.901662in}{2.129201in}}%
\pgfpathlineto{\pgfqpoint{4.916064in}{2.132759in}}%
\pgfpathlineto{\pgfqpoint{4.923882in}{2.141348in}}%
\pgfpathlineto{\pgfqpoint{4.931692in}{2.149823in}}%
\pgfpathlineto{\pgfqpoint{4.939495in}{2.158185in}}%
\pgfpathlineto{\pgfqpoint{4.947290in}{2.166436in}}%
\pgfpathlineto{\pgfqpoint{4.932899in}{2.162903in}}%
\pgfpathlineto{\pgfqpoint{4.918520in}{2.159442in}}%
\pgfpathlineto{\pgfqpoint{4.904153in}{2.156051in}}%
\pgfpathlineto{\pgfqpoint{4.889798in}{2.152732in}}%
\pgfpathlineto{\pgfqpoint{4.881992in}{2.144448in}}%
\pgfpathlineto{\pgfqpoint{4.874179in}{2.136058in}}%
\pgfpathlineto{\pgfqpoint{4.866358in}{2.127560in}}%
\pgfpathlineto{\pgfqpoint{4.858530in}{2.118953in}}%
\pgfpathclose%
\pgfusepath{fill}%
\end{pgfscope}%
\begin{pgfscope}%
\pgfpathrectangle{\pgfqpoint{1.150000in}{0.150000in}}{\pgfqpoint{5.700000in}{5.700000in}}%
\pgfusepath{clip}%
\pgfsetbuttcap%
\pgfsetroundjoin%
\definecolor{currentfill}{rgb}{0.282327,0.094955,0.417331}%
\pgfsetfillcolor{currentfill}%
\pgfsetfillopacity{0.700000}%
\pgfsetlinewidth{0.000000pt}%
\definecolor{currentstroke}{rgb}{0.000000,0.000000,0.000000}%
\pgfsetstrokecolor{currentstroke}%
\pgfsetdash{}{0pt}%
\pgfpathmoveto{\pgfqpoint{3.970901in}{1.648664in}}%
\pgfpathlineto{\pgfqpoint{3.984942in}{1.648205in}}%
\pgfpathlineto{\pgfqpoint{3.998992in}{1.647820in}}%
\pgfpathlineto{\pgfqpoint{4.013051in}{1.647509in}}%
\pgfpathlineto{\pgfqpoint{4.027118in}{1.647270in}}%
\pgfpathlineto{\pgfqpoint{4.035265in}{1.658158in}}%
\pgfpathlineto{\pgfqpoint{4.043407in}{1.669023in}}%
\pgfpathlineto{\pgfqpoint{4.051543in}{1.679860in}}%
\pgfpathlineto{\pgfqpoint{4.059673in}{1.690668in}}%
\pgfpathlineto{\pgfqpoint{4.045615in}{1.690720in}}%
\pgfpathlineto{\pgfqpoint{4.031565in}{1.690846in}}%
\pgfpathlineto{\pgfqpoint{4.017524in}{1.691046in}}%
\pgfpathlineto{\pgfqpoint{4.003492in}{1.691319in}}%
\pgfpathlineto{\pgfqpoint{3.995352in}{1.680689in}}%
\pgfpathlineto{\pgfqpoint{3.987207in}{1.670035in}}%
\pgfpathlineto{\pgfqpoint{3.979057in}{1.659359in}}%
\pgfpathlineto{\pgfqpoint{3.970901in}{1.648664in}}%
\pgfpathclose%
\pgfusepath{fill}%
\end{pgfscope}%
\begin{pgfscope}%
\pgfpathrectangle{\pgfqpoint{1.150000in}{0.150000in}}{\pgfqpoint{5.700000in}{5.700000in}}%
\pgfusepath{clip}%
\pgfsetbuttcap%
\pgfsetroundjoin%
\definecolor{currentfill}{rgb}{0.281446,0.084320,0.407414}%
\pgfsetfillcolor{currentfill}%
\pgfsetfillopacity{0.700000}%
\pgfsetlinewidth{0.000000pt}%
\definecolor{currentstroke}{rgb}{0.000000,0.000000,0.000000}%
\pgfsetstrokecolor{currentstroke}%
\pgfsetdash{}{0pt}%
\pgfpathmoveto{\pgfqpoint{2.688321in}{1.662230in}}%
\pgfpathlineto{\pgfqpoint{2.702207in}{1.653254in}}%
\pgfpathlineto{\pgfqpoint{2.716095in}{1.644371in}}%
\pgfpathlineto{\pgfqpoint{2.729984in}{1.635582in}}%
\pgfpathlineto{\pgfqpoint{2.743874in}{1.626885in}}%
\pgfpathlineto{\pgfqpoint{2.752676in}{1.628870in}}%
\pgfpathlineto{\pgfqpoint{2.761462in}{1.631092in}}%
\pgfpathlineto{\pgfqpoint{2.770233in}{1.633546in}}%
\pgfpathlineto{\pgfqpoint{2.778991in}{1.636227in}}%
\pgfpathlineto{\pgfqpoint{2.765132in}{1.644528in}}%
\pgfpathlineto{\pgfqpoint{2.751275in}{1.652922in}}%
\pgfpathlineto{\pgfqpoint{2.737420in}{1.661409in}}%
\pgfpathlineto{\pgfqpoint{2.723567in}{1.669990in}}%
\pgfpathlineto{\pgfqpoint{2.714778in}{1.667697in}}%
\pgfpathlineto{\pgfqpoint{2.705974in}{1.665635in}}%
\pgfpathlineto{\pgfqpoint{2.697155in}{1.663811in}}%
\pgfpathlineto{\pgfqpoint{2.688321in}{1.662230in}}%
\pgfpathclose%
\pgfusepath{fill}%
\end{pgfscope}%
\begin{pgfscope}%
\pgfpathrectangle{\pgfqpoint{1.150000in}{0.150000in}}{\pgfqpoint{5.700000in}{5.700000in}}%
\pgfusepath{clip}%
\pgfsetbuttcap%
\pgfsetroundjoin%
\definecolor{currentfill}{rgb}{0.274952,0.037752,0.364543}%
\pgfsetfillcolor{currentfill}%
\pgfsetfillopacity{0.700000}%
\pgfsetlinewidth{0.000000pt}%
\definecolor{currentstroke}{rgb}{0.000000,0.000000,0.000000}%
\pgfsetstrokecolor{currentstroke}%
\pgfsetdash{}{0pt}%
\pgfpathmoveto{\pgfqpoint{2.889935in}{1.573092in}}%
\pgfpathlineto{\pgfqpoint{2.903814in}{1.565603in}}%
\pgfpathlineto{\pgfqpoint{2.917696in}{1.558201in}}%
\pgfpathlineto{\pgfqpoint{2.931580in}{1.550887in}}%
\pgfpathlineto{\pgfqpoint{2.945467in}{1.543660in}}%
\pgfpathlineto{\pgfqpoint{2.954123in}{1.547702in}}%
\pgfpathlineto{\pgfqpoint{2.962766in}{1.551941in}}%
\pgfpathlineto{\pgfqpoint{2.971398in}{1.556373in}}%
\pgfpathlineto{\pgfqpoint{2.980017in}{1.560991in}}%
\pgfpathlineto{\pgfqpoint{2.966157in}{1.567847in}}%
\pgfpathlineto{\pgfqpoint{2.952300in}{1.574789in}}%
\pgfpathlineto{\pgfqpoint{2.938446in}{1.581818in}}%
\pgfpathlineto{\pgfqpoint{2.924595in}{1.588936in}}%
\pgfpathlineto{\pgfqpoint{2.915949in}{1.584681in}}%
\pgfpathlineto{\pgfqpoint{2.907290in}{1.580619in}}%
\pgfpathlineto{\pgfqpoint{2.898619in}{1.576754in}}%
\pgfpathlineto{\pgfqpoint{2.889935in}{1.573092in}}%
\pgfpathclose%
\pgfusepath{fill}%
\end{pgfscope}%
\begin{pgfscope}%
\pgfpathrectangle{\pgfqpoint{1.150000in}{0.150000in}}{\pgfqpoint{5.700000in}{5.700000in}}%
\pgfusepath{clip}%
\pgfsetbuttcap%
\pgfsetroundjoin%
\definecolor{currentfill}{rgb}{0.281412,0.155834,0.469201}%
\pgfsetfillcolor{currentfill}%
\pgfsetfillopacity{0.700000}%
\pgfsetlinewidth{0.000000pt}%
\definecolor{currentstroke}{rgb}{0.000000,0.000000,0.000000}%
\pgfsetstrokecolor{currentstroke}%
\pgfsetdash{}{0pt}%
\pgfpathmoveto{\pgfqpoint{2.430153in}{1.822061in}}%
\pgfpathlineto{\pgfqpoint{2.444070in}{1.811079in}}%
\pgfpathlineto{\pgfqpoint{2.457986in}{1.800202in}}%
\pgfpathlineto{\pgfqpoint{2.471902in}{1.789427in}}%
\pgfpathlineto{\pgfqpoint{2.485818in}{1.778756in}}%
\pgfpathlineto{\pgfqpoint{2.494831in}{1.778014in}}%
\pgfpathlineto{\pgfqpoint{2.503824in}{1.777558in}}%
\pgfpathlineto{\pgfqpoint{2.512800in}{1.777383in}}%
\pgfpathlineto{\pgfqpoint{2.521758in}{1.777480in}}%
\pgfpathlineto{\pgfqpoint{2.507881in}{1.787731in}}%
\pgfpathlineto{\pgfqpoint{2.494004in}{1.798083in}}%
\pgfpathlineto{\pgfqpoint{2.480126in}{1.808539in}}%
\pgfpathlineto{\pgfqpoint{2.466249in}{1.819099in}}%
\pgfpathlineto{\pgfqpoint{2.457253in}{1.819414in}}%
\pgfpathlineto{\pgfqpoint{2.448239in}{1.820009in}}%
\pgfpathlineto{\pgfqpoint{2.439206in}{1.820889in}}%
\pgfpathlineto{\pgfqpoint{2.430153in}{1.822061in}}%
\pgfpathclose%
\pgfusepath{fill}%
\end{pgfscope}%
\begin{pgfscope}%
\pgfpathrectangle{\pgfqpoint{1.150000in}{0.150000in}}{\pgfqpoint{5.700000in}{5.700000in}}%
\pgfusepath{clip}%
\pgfsetbuttcap%
\pgfsetroundjoin%
\definecolor{currentfill}{rgb}{0.280267,0.073417,0.397163}%
\pgfsetfillcolor{currentfill}%
\pgfsetfillopacity{0.700000}%
\pgfsetlinewidth{0.000000pt}%
\definecolor{currentstroke}{rgb}{0.000000,0.000000,0.000000}%
\pgfsetstrokecolor{currentstroke}%
\pgfsetdash{}{0pt}%
\pgfpathmoveto{\pgfqpoint{3.882099in}{1.609145in}}%
\pgfpathlineto{\pgfqpoint{3.896118in}{1.608184in}}%
\pgfpathlineto{\pgfqpoint{3.910144in}{1.607298in}}%
\pgfpathlineto{\pgfqpoint{3.924179in}{1.606486in}}%
\pgfpathlineto{\pgfqpoint{3.938222in}{1.605747in}}%
\pgfpathlineto{\pgfqpoint{3.946400in}{1.616492in}}%
\pgfpathlineto{\pgfqpoint{3.954572in}{1.627228in}}%
\pgfpathlineto{\pgfqpoint{3.962739in}{1.637953in}}%
\pgfpathlineto{\pgfqpoint{3.970901in}{1.648664in}}%
\pgfpathlineto{\pgfqpoint{3.956867in}{1.649196in}}%
\pgfpathlineto{\pgfqpoint{3.942842in}{1.649802in}}%
\pgfpathlineto{\pgfqpoint{3.928825in}{1.650482in}}%
\pgfpathlineto{\pgfqpoint{3.914817in}{1.651236in}}%
\pgfpathlineto{\pgfqpoint{3.906646in}{1.640723in}}%
\pgfpathlineto{\pgfqpoint{3.898469in}{1.630202in}}%
\pgfpathlineto{\pgfqpoint{3.890287in}{1.619675in}}%
\pgfpathlineto{\pgfqpoint{3.882099in}{1.609145in}}%
\pgfpathclose%
\pgfusepath{fill}%
\end{pgfscope}%
\begin{pgfscope}%
\pgfpathrectangle{\pgfqpoint{1.150000in}{0.150000in}}{\pgfqpoint{5.700000in}{5.700000in}}%
\pgfusepath{clip}%
\pgfsetbuttcap%
\pgfsetroundjoin%
\definecolor{currentfill}{rgb}{0.268510,0.009605,0.335427}%
\pgfsetfillcolor{currentfill}%
\pgfsetfillopacity{0.700000}%
\pgfsetlinewidth{0.000000pt}%
\definecolor{currentstroke}{rgb}{0.000000,0.000000,0.000000}%
\pgfsetstrokecolor{currentstroke}%
\pgfsetdash{}{0pt}%
\pgfpathmoveto{\pgfqpoint{3.470501in}{1.498506in}}%
\pgfpathlineto{\pgfqpoint{3.484431in}{1.495005in}}%
\pgfpathlineto{\pgfqpoint{3.498366in}{1.491582in}}%
\pgfpathlineto{\pgfqpoint{3.512308in}{1.488236in}}%
\pgfpathlineto{\pgfqpoint{3.526256in}{1.484967in}}%
\pgfpathlineto{\pgfqpoint{3.534593in}{1.493898in}}%
\pgfpathlineto{\pgfqpoint{3.542922in}{1.502903in}}%
\pgfpathlineto{\pgfqpoint{3.551245in}{1.511981in}}%
\pgfpathlineto{\pgfqpoint{3.559560in}{1.521125in}}%
\pgfpathlineto{\pgfqpoint{3.545627in}{1.524107in}}%
\pgfpathlineto{\pgfqpoint{3.531700in}{1.527165in}}%
\pgfpathlineto{\pgfqpoint{3.517780in}{1.530301in}}%
\pgfpathlineto{\pgfqpoint{3.503866in}{1.533514in}}%
\pgfpathlineto{\pgfqpoint{3.495535in}{1.524649in}}%
\pgfpathlineto{\pgfqpoint{3.487198in}{1.515857in}}%
\pgfpathlineto{\pgfqpoint{3.478853in}{1.507141in}}%
\pgfpathlineto{\pgfqpoint{3.470501in}{1.498506in}}%
\pgfpathclose%
\pgfusepath{fill}%
\end{pgfscope}%
\begin{pgfscope}%
\pgfpathrectangle{\pgfqpoint{1.150000in}{0.150000in}}{\pgfqpoint{5.700000in}{5.700000in}}%
\pgfusepath{clip}%
\pgfsetbuttcap%
\pgfsetroundjoin%
\definecolor{currentfill}{rgb}{0.277941,0.056324,0.381191}%
\pgfsetfillcolor{currentfill}%
\pgfsetfillopacity{0.700000}%
\pgfsetlinewidth{0.000000pt}%
\definecolor{currentstroke}{rgb}{0.000000,0.000000,0.000000}%
\pgfsetstrokecolor{currentstroke}%
\pgfsetdash{}{0pt}%
\pgfpathmoveto{\pgfqpoint{3.793254in}{1.572546in}}%
\pgfpathlineto{\pgfqpoint{3.807252in}{1.571062in}}%
\pgfpathlineto{\pgfqpoint{3.821258in}{1.569652in}}%
\pgfpathlineto{\pgfqpoint{3.835271in}{1.568317in}}%
\pgfpathlineto{\pgfqpoint{3.849292in}{1.567056in}}%
\pgfpathlineto{\pgfqpoint{3.857502in}{1.577567in}}%
\pgfpathlineto{\pgfqpoint{3.865707in}{1.588088in}}%
\pgfpathlineto{\pgfqpoint{3.873906in}{1.598615in}}%
\pgfpathlineto{\pgfqpoint{3.882099in}{1.609145in}}%
\pgfpathlineto{\pgfqpoint{3.868089in}{1.610179in}}%
\pgfpathlineto{\pgfqpoint{3.854086in}{1.611287in}}%
\pgfpathlineto{\pgfqpoint{3.840091in}{1.612470in}}%
\pgfpathlineto{\pgfqpoint{3.826104in}{1.613728in}}%
\pgfpathlineto{\pgfqpoint{3.817900in}{1.603416in}}%
\pgfpathlineto{\pgfqpoint{3.809691in}{1.593114in}}%
\pgfpathlineto{\pgfqpoint{3.801475in}{1.582822in}}%
\pgfpathlineto{\pgfqpoint{3.793254in}{1.572546in}}%
\pgfpathclose%
\pgfusepath{fill}%
\end{pgfscope}%
\begin{pgfscope}%
\pgfpathrectangle{\pgfqpoint{1.150000in}{0.150000in}}{\pgfqpoint{5.700000in}{5.700000in}}%
\pgfusepath{clip}%
\pgfsetbuttcap%
\pgfsetroundjoin%
\definecolor{currentfill}{rgb}{0.268510,0.009605,0.335427}%
\pgfsetfillcolor{currentfill}%
\pgfsetfillopacity{0.700000}%
\pgfsetlinewidth{0.000000pt}%
\definecolor{currentstroke}{rgb}{0.000000,0.000000,0.000000}%
\pgfsetstrokecolor{currentstroke}%
\pgfsetdash{}{0pt}%
\pgfpathmoveto{\pgfqpoint{3.091017in}{1.509231in}}%
\pgfpathlineto{\pgfqpoint{3.104909in}{1.503140in}}%
\pgfpathlineto{\pgfqpoint{3.118804in}{1.497133in}}%
\pgfpathlineto{\pgfqpoint{3.132703in}{1.491208in}}%
\pgfpathlineto{\pgfqpoint{3.146606in}{1.485366in}}%
\pgfpathlineto{\pgfqpoint{3.155140in}{1.491239in}}%
\pgfpathlineto{\pgfqpoint{3.163664in}{1.497272in}}%
\pgfpathlineto{\pgfqpoint{3.172178in}{1.503459in}}%
\pgfpathlineto{\pgfqpoint{3.180682in}{1.509794in}}%
\pgfpathlineto{\pgfqpoint{3.166801in}{1.515287in}}%
\pgfpathlineto{\pgfqpoint{3.152925in}{1.520862in}}%
\pgfpathlineto{\pgfqpoint{3.139053in}{1.526519in}}%
\pgfpathlineto{\pgfqpoint{3.125185in}{1.532260in}}%
\pgfpathlineto{\pgfqpoint{3.116659in}{1.526266in}}%
\pgfpathlineto{\pgfqpoint{3.108122in}{1.520426in}}%
\pgfpathlineto{\pgfqpoint{3.099575in}{1.514746in}}%
\pgfpathlineto{\pgfqpoint{3.091017in}{1.509231in}}%
\pgfpathclose%
\pgfusepath{fill}%
\end{pgfscope}%
\begin{pgfscope}%
\pgfpathrectangle{\pgfqpoint{1.150000in}{0.150000in}}{\pgfqpoint{5.700000in}{5.700000in}}%
\pgfusepath{clip}%
\pgfsetbuttcap%
\pgfsetroundjoin%
\definecolor{currentfill}{rgb}{0.282623,0.140926,0.457517}%
\pgfsetfillcolor{currentfill}%
\pgfsetfillopacity{0.700000}%
\pgfsetlinewidth{0.000000pt}%
\definecolor{currentstroke}{rgb}{0.000000,0.000000,0.000000}%
\pgfsetstrokecolor{currentstroke}%
\pgfsetdash{}{0pt}%
\pgfpathmoveto{\pgfqpoint{2.485818in}{1.778756in}}%
\pgfpathlineto{\pgfqpoint{2.499733in}{1.768186in}}%
\pgfpathlineto{\pgfqpoint{2.513649in}{1.757718in}}%
\pgfpathlineto{\pgfqpoint{2.527564in}{1.747351in}}%
\pgfpathlineto{\pgfqpoint{2.541480in}{1.737084in}}%
\pgfpathlineto{\pgfqpoint{2.550454in}{1.736771in}}%
\pgfpathlineto{\pgfqpoint{2.559410in}{1.736739in}}%
\pgfpathlineto{\pgfqpoint{2.568348in}{1.736980in}}%
\pgfpathlineto{\pgfqpoint{2.577269in}{1.737490in}}%
\pgfpathlineto{\pgfqpoint{2.563390in}{1.747337in}}%
\pgfpathlineto{\pgfqpoint{2.549513in}{1.757284in}}%
\pgfpathlineto{\pgfqpoint{2.535635in}{1.767332in}}%
\pgfpathlineto{\pgfqpoint{2.521758in}{1.777480in}}%
\pgfpathlineto{\pgfqpoint{2.512800in}{1.777383in}}%
\pgfpathlineto{\pgfqpoint{2.503824in}{1.777558in}}%
\pgfpathlineto{\pgfqpoint{2.494831in}{1.778014in}}%
\pgfpathlineto{\pgfqpoint{2.485818in}{1.778756in}}%
\pgfpathclose%
\pgfusepath{fill}%
\end{pgfscope}%
\begin{pgfscope}%
\pgfpathrectangle{\pgfqpoint{1.150000in}{0.150000in}}{\pgfqpoint{5.700000in}{5.700000in}}%
\pgfusepath{clip}%
\pgfsetbuttcap%
\pgfsetroundjoin%
\definecolor{currentfill}{rgb}{0.267004,0.004874,0.329415}%
\pgfsetfillcolor{currentfill}%
\pgfsetfillopacity{0.700000}%
\pgfsetlinewidth{0.000000pt}%
\definecolor{currentstroke}{rgb}{0.000000,0.000000,0.000000}%
\pgfsetstrokecolor{currentstroke}%
\pgfsetdash{}{0pt}%
\pgfpathmoveto{\pgfqpoint{3.236248in}{1.488642in}}%
\pgfpathlineto{\pgfqpoint{3.250151in}{1.483557in}}%
\pgfpathlineto{\pgfqpoint{3.264058in}{1.478552in}}%
\pgfpathlineto{\pgfqpoint{3.277971in}{1.473627in}}%
\pgfpathlineto{\pgfqpoint{3.291888in}{1.468783in}}%
\pgfpathlineto{\pgfqpoint{3.300340in}{1.475935in}}%
\pgfpathlineto{\pgfqpoint{3.308784in}{1.483216in}}%
\pgfpathlineto{\pgfqpoint{3.317219in}{1.490621in}}%
\pgfpathlineto{\pgfqpoint{3.325645in}{1.498144in}}%
\pgfpathlineto{\pgfqpoint{3.311748in}{1.502659in}}%
\pgfpathlineto{\pgfqpoint{3.297855in}{1.507255in}}%
\pgfpathlineto{\pgfqpoint{3.283968in}{1.511931in}}%
\pgfpathlineto{\pgfqpoint{3.270085in}{1.516688in}}%
\pgfpathlineto{\pgfqpoint{3.261639in}{1.509486in}}%
\pgfpathlineto{\pgfqpoint{3.253184in}{1.502408in}}%
\pgfpathlineto{\pgfqpoint{3.244721in}{1.495458in}}%
\pgfpathlineto{\pgfqpoint{3.236248in}{1.488642in}}%
\pgfpathclose%
\pgfusepath{fill}%
\end{pgfscope}%
\begin{pgfscope}%
\pgfpathrectangle{\pgfqpoint{1.150000in}{0.150000in}}{\pgfqpoint{5.700000in}{5.700000in}}%
\pgfusepath{clip}%
\pgfsetbuttcap%
\pgfsetroundjoin%
\definecolor{currentfill}{rgb}{0.274952,0.037752,0.364543}%
\pgfsetfillcolor{currentfill}%
\pgfsetfillopacity{0.700000}%
\pgfsetlinewidth{0.000000pt}%
\definecolor{currentstroke}{rgb}{0.000000,0.000000,0.000000}%
\pgfsetstrokecolor{currentstroke}%
\pgfsetdash{}{0pt}%
\pgfpathmoveto{\pgfqpoint{3.704347in}{1.539325in}}%
\pgfpathlineto{\pgfqpoint{3.718328in}{1.537295in}}%
\pgfpathlineto{\pgfqpoint{3.732315in}{1.535340in}}%
\pgfpathlineto{\pgfqpoint{3.746310in}{1.533459in}}%
\pgfpathlineto{\pgfqpoint{3.760312in}{1.531654in}}%
\pgfpathlineto{\pgfqpoint{3.768556in}{1.541837in}}%
\pgfpathlineto{\pgfqpoint{3.776795in}{1.552050in}}%
\pgfpathlineto{\pgfqpoint{3.785028in}{1.562287in}}%
\pgfpathlineto{\pgfqpoint{3.793254in}{1.572546in}}%
\pgfpathlineto{\pgfqpoint{3.779264in}{1.574104in}}%
\pgfpathlineto{\pgfqpoint{3.765281in}{1.575738in}}%
\pgfpathlineto{\pgfqpoint{3.751305in}{1.577446in}}%
\pgfpathlineto{\pgfqpoint{3.737337in}{1.579230in}}%
\pgfpathlineto{\pgfqpoint{3.729099in}{1.569210in}}%
\pgfpathlineto{\pgfqpoint{3.720854in}{1.559217in}}%
\pgfpathlineto{\pgfqpoint{3.712604in}{1.549255in}}%
\pgfpathlineto{\pgfqpoint{3.704347in}{1.539325in}}%
\pgfpathclose%
\pgfusepath{fill}%
\end{pgfscope}%
\begin{pgfscope}%
\pgfpathrectangle{\pgfqpoint{1.150000in}{0.150000in}}{\pgfqpoint{5.700000in}{5.700000in}}%
\pgfusepath{clip}%
\pgfsetbuttcap%
\pgfsetroundjoin%
\definecolor{currentfill}{rgb}{0.279566,0.067836,0.391917}%
\pgfsetfillcolor{currentfill}%
\pgfsetfillopacity{0.700000}%
\pgfsetlinewidth{0.000000pt}%
\definecolor{currentstroke}{rgb}{0.000000,0.000000,0.000000}%
\pgfsetstrokecolor{currentstroke}%
\pgfsetdash{}{0pt}%
\pgfpathmoveto{\pgfqpoint{2.743874in}{1.626885in}}%
\pgfpathlineto{\pgfqpoint{2.757767in}{1.618281in}}%
\pgfpathlineto{\pgfqpoint{2.771661in}{1.609769in}}%
\pgfpathlineto{\pgfqpoint{2.785557in}{1.601348in}}%
\pgfpathlineto{\pgfqpoint{2.799454in}{1.593018in}}%
\pgfpathlineto{\pgfqpoint{2.808223in}{1.595405in}}%
\pgfpathlineto{\pgfqpoint{2.816978in}{1.598025in}}%
\pgfpathlineto{\pgfqpoint{2.825718in}{1.600871in}}%
\pgfpathlineto{\pgfqpoint{2.834445in}{1.603938in}}%
\pgfpathlineto{\pgfqpoint{2.820578in}{1.611874in}}%
\pgfpathlineto{\pgfqpoint{2.806714in}{1.619900in}}%
\pgfpathlineto{\pgfqpoint{2.792851in}{1.628018in}}%
\pgfpathlineto{\pgfqpoint{2.778991in}{1.636227in}}%
\pgfpathlineto{\pgfqpoint{2.770233in}{1.633546in}}%
\pgfpathlineto{\pgfqpoint{2.761462in}{1.631092in}}%
\pgfpathlineto{\pgfqpoint{2.752676in}{1.628870in}}%
\pgfpathlineto{\pgfqpoint{2.743874in}{1.626885in}}%
\pgfpathclose%
\pgfusepath{fill}%
\end{pgfscope}%
\begin{pgfscope}%
\pgfpathrectangle{\pgfqpoint{1.150000in}{0.150000in}}{\pgfqpoint{5.700000in}{5.700000in}}%
\pgfusepath{clip}%
\pgfsetbuttcap%
\pgfsetroundjoin%
\definecolor{currentfill}{rgb}{0.272594,0.025563,0.353093}%
\pgfsetfillcolor{currentfill}%
\pgfsetfillopacity{0.700000}%
\pgfsetlinewidth{0.000000pt}%
\definecolor{currentstroke}{rgb}{0.000000,0.000000,0.000000}%
\pgfsetstrokecolor{currentstroke}%
\pgfsetdash{}{0pt}%
\pgfpathmoveto{\pgfqpoint{2.945467in}{1.543660in}}%
\pgfpathlineto{\pgfqpoint{2.959357in}{1.536519in}}%
\pgfpathlineto{\pgfqpoint{2.973250in}{1.529465in}}%
\pgfpathlineto{\pgfqpoint{2.987146in}{1.522497in}}%
\pgfpathlineto{\pgfqpoint{3.001046in}{1.515614in}}%
\pgfpathlineto{\pgfqpoint{3.009674in}{1.520035in}}%
\pgfpathlineto{\pgfqpoint{3.018291in}{1.524649in}}%
\pgfpathlineto{\pgfqpoint{3.026896in}{1.529450in}}%
\pgfpathlineto{\pgfqpoint{3.035489in}{1.534431in}}%
\pgfpathlineto{\pgfqpoint{3.021616in}{1.540943in}}%
\pgfpathlineto{\pgfqpoint{3.007746in}{1.547540in}}%
\pgfpathlineto{\pgfqpoint{2.993880in}{1.554223in}}%
\pgfpathlineto{\pgfqpoint{2.980017in}{1.560991in}}%
\pgfpathlineto{\pgfqpoint{2.971398in}{1.556373in}}%
\pgfpathlineto{\pgfqpoint{2.962766in}{1.551941in}}%
\pgfpathlineto{\pgfqpoint{2.954123in}{1.547702in}}%
\pgfpathlineto{\pgfqpoint{2.945467in}{1.543660in}}%
\pgfpathclose%
\pgfusepath{fill}%
\end{pgfscope}%
\begin{pgfscope}%
\pgfpathrectangle{\pgfqpoint{1.150000in}{0.150000in}}{\pgfqpoint{5.700000in}{5.700000in}}%
\pgfusepath{clip}%
\pgfsetbuttcap%
\pgfsetroundjoin%
\definecolor{currentfill}{rgb}{0.177423,0.437527,0.557565}%
\pgfsetfillcolor{currentfill}%
\pgfsetfillopacity{0.700000}%
\pgfsetlinewidth{0.000000pt}%
\definecolor{currentstroke}{rgb}{0.000000,0.000000,0.000000}%
\pgfsetstrokecolor{currentstroke}%
\pgfsetdash{}{0pt}%
\pgfpathmoveto{\pgfqpoint{5.537753in}{2.445359in}}%
\pgfpathlineto{\pgfqpoint{5.552431in}{2.450328in}}%
\pgfpathlineto{\pgfqpoint{5.567123in}{2.455367in}}%
\pgfpathlineto{\pgfqpoint{5.581828in}{2.460477in}}%
\pgfpathlineto{\pgfqpoint{5.589295in}{2.465169in}}%
\pgfpathlineto{\pgfqpoint{5.596753in}{2.469762in}}%
\pgfpathlineto{\pgfqpoint{5.604201in}{2.474260in}}%
\pgfpathlineto{\pgfqpoint{5.611641in}{2.478666in}}%
\pgfpathlineto{\pgfqpoint{5.596955in}{2.473737in}}%
\pgfpathlineto{\pgfqpoint{5.582283in}{2.468878in}}%
\pgfpathlineto{\pgfqpoint{5.567624in}{2.464089in}}%
\pgfpathlineto{\pgfqpoint{5.560170in}{2.459541in}}%
\pgfpathlineto{\pgfqpoint{5.552707in}{2.454906in}}%
\pgfpathlineto{\pgfqpoint{5.545235in}{2.450180in}}%
\pgfpathlineto{\pgfqpoint{5.537753in}{2.445359in}}%
\pgfpathclose%
\pgfusepath{fill}%
\end{pgfscope}%
\begin{pgfscope}%
\pgfpathrectangle{\pgfqpoint{1.150000in}{0.150000in}}{\pgfqpoint{5.700000in}{5.700000in}}%
\pgfusepath{clip}%
\pgfsetbuttcap%
\pgfsetroundjoin%
\definecolor{currentfill}{rgb}{0.267004,0.004874,0.329415}%
\pgfsetfillcolor{currentfill}%
\pgfsetfillopacity{0.700000}%
\pgfsetlinewidth{0.000000pt}%
\definecolor{currentstroke}{rgb}{0.000000,0.000000,0.000000}%
\pgfsetstrokecolor{currentstroke}%
\pgfsetdash{}{0pt}%
\pgfpathmoveto{\pgfqpoint{3.381288in}{1.480873in}}%
\pgfpathlineto{\pgfqpoint{3.395212in}{1.476753in}}%
\pgfpathlineto{\pgfqpoint{3.409142in}{1.472711in}}%
\pgfpathlineto{\pgfqpoint{3.423077in}{1.468747in}}%
\pgfpathlineto{\pgfqpoint{3.437018in}{1.464861in}}%
\pgfpathlineto{\pgfqpoint{3.445400in}{1.473129in}}%
\pgfpathlineto{\pgfqpoint{3.453774in}{1.481495in}}%
\pgfpathlineto{\pgfqpoint{3.462141in}{1.489956in}}%
\pgfpathlineto{\pgfqpoint{3.470501in}{1.498506in}}%
\pgfpathlineto{\pgfqpoint{3.456577in}{1.502084in}}%
\pgfpathlineto{\pgfqpoint{3.442659in}{1.505740in}}%
\pgfpathlineto{\pgfqpoint{3.428746in}{1.509474in}}%
\pgfpathlineto{\pgfqpoint{3.414840in}{1.513287in}}%
\pgfpathlineto{\pgfqpoint{3.406463in}{1.505037in}}%
\pgfpathlineto{\pgfqpoint{3.398080in}{1.496882in}}%
\pgfpathlineto{\pgfqpoint{3.389688in}{1.488826in}}%
\pgfpathlineto{\pgfqpoint{3.381288in}{1.480873in}}%
\pgfpathclose%
\pgfusepath{fill}%
\end{pgfscope}%
\begin{pgfscope}%
\pgfpathrectangle{\pgfqpoint{1.150000in}{0.150000in}}{\pgfqpoint{5.700000in}{5.700000in}}%
\pgfusepath{clip}%
\pgfsetbuttcap%
\pgfsetroundjoin%
\definecolor{currentfill}{rgb}{0.271305,0.019942,0.347269}%
\pgfsetfillcolor{currentfill}%
\pgfsetfillopacity{0.700000}%
\pgfsetlinewidth{0.000000pt}%
\definecolor{currentstroke}{rgb}{0.000000,0.000000,0.000000}%
\pgfsetstrokecolor{currentstroke}%
\pgfsetdash{}{0pt}%
\pgfpathmoveto{\pgfqpoint{3.615357in}{1.509965in}}%
\pgfpathlineto{\pgfqpoint{3.629322in}{1.507365in}}%
\pgfpathlineto{\pgfqpoint{3.643295in}{1.504841in}}%
\pgfpathlineto{\pgfqpoint{3.657274in}{1.502392in}}%
\pgfpathlineto{\pgfqpoint{3.671260in}{1.500019in}}%
\pgfpathlineto{\pgfqpoint{3.679541in}{1.509777in}}%
\pgfpathlineto{\pgfqpoint{3.687816in}{1.519583in}}%
\pgfpathlineto{\pgfqpoint{3.696085in}{1.529434in}}%
\pgfpathlineto{\pgfqpoint{3.704347in}{1.539325in}}%
\pgfpathlineto{\pgfqpoint{3.690374in}{1.541431in}}%
\pgfpathlineto{\pgfqpoint{3.676408in}{1.543612in}}%
\pgfpathlineto{\pgfqpoint{3.662449in}{1.545869in}}%
\pgfpathlineto{\pgfqpoint{3.648497in}{1.548201in}}%
\pgfpathlineto{\pgfqpoint{3.640221in}{1.538569in}}%
\pgfpathlineto{\pgfqpoint{3.631940in}{1.528983in}}%
\pgfpathlineto{\pgfqpoint{3.623651in}{1.519447in}}%
\pgfpathlineto{\pgfqpoint{3.615357in}{1.509965in}}%
\pgfpathclose%
\pgfusepath{fill}%
\end{pgfscope}%
\begin{pgfscope}%
\pgfpathrectangle{\pgfqpoint{1.150000in}{0.150000in}}{\pgfqpoint{5.700000in}{5.700000in}}%
\pgfusepath{clip}%
\pgfsetbuttcap%
\pgfsetroundjoin%
\definecolor{currentfill}{rgb}{0.283187,0.125848,0.444960}%
\pgfsetfillcolor{currentfill}%
\pgfsetfillopacity{0.700000}%
\pgfsetlinewidth{0.000000pt}%
\definecolor{currentstroke}{rgb}{0.000000,0.000000,0.000000}%
\pgfsetstrokecolor{currentstroke}%
\pgfsetdash{}{0pt}%
\pgfpathmoveto{\pgfqpoint{2.541480in}{1.737084in}}%
\pgfpathlineto{\pgfqpoint{2.555396in}{1.726916in}}%
\pgfpathlineto{\pgfqpoint{2.569312in}{1.716847in}}%
\pgfpathlineto{\pgfqpoint{2.583229in}{1.706876in}}%
\pgfpathlineto{\pgfqpoint{2.597146in}{1.697002in}}%
\pgfpathlineto{\pgfqpoint{2.606082in}{1.697117in}}%
\pgfpathlineto{\pgfqpoint{2.615001in}{1.697507in}}%
\pgfpathlineto{\pgfqpoint{2.623903in}{1.698166in}}%
\pgfpathlineto{\pgfqpoint{2.632788in}{1.699086in}}%
\pgfpathlineto{\pgfqpoint{2.618907in}{1.708541in}}%
\pgfpathlineto{\pgfqpoint{2.605027in}{1.718092in}}%
\pgfpathlineto{\pgfqpoint{2.591147in}{1.727742in}}%
\pgfpathlineto{\pgfqpoint{2.577269in}{1.737490in}}%
\pgfpathlineto{\pgfqpoint{2.568348in}{1.736980in}}%
\pgfpathlineto{\pgfqpoint{2.559410in}{1.736739in}}%
\pgfpathlineto{\pgfqpoint{2.550454in}{1.736771in}}%
\pgfpathlineto{\pgfqpoint{2.541480in}{1.737084in}}%
\pgfpathclose%
\pgfusepath{fill}%
\end{pgfscope}%
\begin{pgfscope}%
\pgfpathrectangle{\pgfqpoint{1.150000in}{0.150000in}}{\pgfqpoint{5.700000in}{5.700000in}}%
\pgfusepath{clip}%
\pgfsetbuttcap%
\pgfsetroundjoin%
\definecolor{currentfill}{rgb}{0.182256,0.426184,0.557120}%
\pgfsetfillcolor{currentfill}%
\pgfsetfillopacity{0.700000}%
\pgfsetlinewidth{0.000000pt}%
\definecolor{currentstroke}{rgb}{0.000000,0.000000,0.000000}%
\pgfsetstrokecolor{currentstroke}%
\pgfsetdash{}{0pt}%
\pgfpathmoveto{\pgfqpoint{5.449089in}{2.405262in}}%
\pgfpathlineto{\pgfqpoint{5.463730in}{2.410107in}}%
\pgfpathlineto{\pgfqpoint{5.478385in}{2.415023in}}%
\pgfpathlineto{\pgfqpoint{5.493053in}{2.420009in}}%
\pgfpathlineto{\pgfqpoint{5.507735in}{2.425067in}}%
\pgfpathlineto{\pgfqpoint{5.515254in}{2.430297in}}%
\pgfpathlineto{\pgfqpoint{5.522763in}{2.435420in}}%
\pgfpathlineto{\pgfqpoint{5.530263in}{2.440440in}}%
\pgfpathlineto{\pgfqpoint{5.537753in}{2.445359in}}%
\pgfpathlineto{\pgfqpoint{5.523089in}{2.440460in}}%
\pgfpathlineto{\pgfqpoint{5.508439in}{2.435632in}}%
\pgfpathlineto{\pgfqpoint{5.493802in}{2.430874in}}%
\pgfpathlineto{\pgfqpoint{5.479178in}{2.426187in}}%
\pgfpathlineto{\pgfqpoint{5.471669in}{2.421102in}}%
\pgfpathlineto{\pgfqpoint{5.464152in}{2.415922in}}%
\pgfpathlineto{\pgfqpoint{5.456625in}{2.410642in}}%
\pgfpathlineto{\pgfqpoint{5.449089in}{2.405262in}}%
\pgfpathclose%
\pgfusepath{fill}%
\end{pgfscope}%
\begin{pgfscope}%
\pgfpathrectangle{\pgfqpoint{1.150000in}{0.150000in}}{\pgfqpoint{5.700000in}{5.700000in}}%
\pgfusepath{clip}%
\pgfsetbuttcap%
\pgfsetroundjoin%
\definecolor{currentfill}{rgb}{0.282290,0.145912,0.461510}%
\pgfsetfillcolor{currentfill}%
\pgfsetfillopacity{0.700000}%
\pgfsetlinewidth{0.000000pt}%
\definecolor{currentstroke}{rgb}{0.000000,0.000000,0.000000}%
\pgfsetstrokecolor{currentstroke}%
\pgfsetdash{}{0pt}%
\pgfpathmoveto{\pgfqpoint{4.204864in}{1.737088in}}%
\pgfpathlineto{\pgfqpoint{4.218996in}{1.737857in}}%
\pgfpathlineto{\pgfqpoint{4.233138in}{1.738698in}}%
\pgfpathlineto{\pgfqpoint{4.247289in}{1.739611in}}%
\pgfpathlineto{\pgfqpoint{4.261449in}{1.740596in}}%
\pgfpathlineto{\pgfqpoint{4.269529in}{1.751680in}}%
\pgfpathlineto{\pgfqpoint{4.277603in}{1.762706in}}%
\pgfpathlineto{\pgfqpoint{4.285671in}{1.773672in}}%
\pgfpathlineto{\pgfqpoint{4.293734in}{1.784576in}}%
\pgfpathlineto{\pgfqpoint{4.279581in}{1.783445in}}%
\pgfpathlineto{\pgfqpoint{4.265438in}{1.782387in}}%
\pgfpathlineto{\pgfqpoint{4.251304in}{1.781402in}}%
\pgfpathlineto{\pgfqpoint{4.237180in}{1.780489in}}%
\pgfpathlineto{\pgfqpoint{4.229109in}{1.769722in}}%
\pgfpathlineto{\pgfqpoint{4.221033in}{1.758898in}}%
\pgfpathlineto{\pgfqpoint{4.212952in}{1.748020in}}%
\pgfpathlineto{\pgfqpoint{4.204864in}{1.737088in}}%
\pgfpathclose%
\pgfusepath{fill}%
\end{pgfscope}%
\begin{pgfscope}%
\pgfpathrectangle{\pgfqpoint{1.150000in}{0.150000in}}{\pgfqpoint{5.700000in}{5.700000in}}%
\pgfusepath{clip}%
\pgfsetbuttcap%
\pgfsetroundjoin%
\definecolor{currentfill}{rgb}{0.279574,0.170599,0.479997}%
\pgfsetfillcolor{currentfill}%
\pgfsetfillopacity{0.700000}%
\pgfsetlinewidth{0.000000pt}%
\definecolor{currentstroke}{rgb}{0.000000,0.000000,0.000000}%
\pgfsetstrokecolor{currentstroke}%
\pgfsetdash{}{0pt}%
\pgfpathmoveto{\pgfqpoint{4.293734in}{1.784576in}}%
\pgfpathlineto{\pgfqpoint{4.307897in}{1.785778in}}%
\pgfpathlineto{\pgfqpoint{4.322070in}{1.787053in}}%
\pgfpathlineto{\pgfqpoint{4.336252in}{1.788400in}}%
\pgfpathlineto{\pgfqpoint{4.350445in}{1.789819in}}%
\pgfpathlineto{\pgfqpoint{4.358495in}{1.800791in}}%
\pgfpathlineto{\pgfqpoint{4.366539in}{1.811693in}}%
\pgfpathlineto{\pgfqpoint{4.374578in}{1.822522in}}%
\pgfpathlineto{\pgfqpoint{4.382611in}{1.833279in}}%
\pgfpathlineto{\pgfqpoint{4.368426in}{1.831736in}}%
\pgfpathlineto{\pgfqpoint{4.354251in}{1.830265in}}%
\pgfpathlineto{\pgfqpoint{4.340085in}{1.828866in}}%
\pgfpathlineto{\pgfqpoint{4.325930in}{1.827540in}}%
\pgfpathlineto{\pgfqpoint{4.317890in}{1.816900in}}%
\pgfpathlineto{\pgfqpoint{4.309844in}{1.806191in}}%
\pgfpathlineto{\pgfqpoint{4.301792in}{1.795416in}}%
\pgfpathlineto{\pgfqpoint{4.293734in}{1.784576in}}%
\pgfpathclose%
\pgfusepath{fill}%
\end{pgfscope}%
\begin{pgfscope}%
\pgfpathrectangle{\pgfqpoint{1.150000in}{0.150000in}}{\pgfqpoint{5.700000in}{5.700000in}}%
\pgfusepath{clip}%
\pgfsetbuttcap%
\pgfsetroundjoin%
\definecolor{currentfill}{rgb}{0.283187,0.125848,0.444960}%
\pgfsetfillcolor{currentfill}%
\pgfsetfillopacity{0.700000}%
\pgfsetlinewidth{0.000000pt}%
\definecolor{currentstroke}{rgb}{0.000000,0.000000,0.000000}%
\pgfsetstrokecolor{currentstroke}%
\pgfsetdash{}{0pt}%
\pgfpathmoveto{\pgfqpoint{4.115995in}{1.691189in}}%
\pgfpathlineto{\pgfqpoint{4.130098in}{1.691501in}}%
\pgfpathlineto{\pgfqpoint{4.144210in}{1.691886in}}%
\pgfpathlineto{\pgfqpoint{4.158331in}{1.692344in}}%
\pgfpathlineto{\pgfqpoint{4.172461in}{1.692875in}}%
\pgfpathlineto{\pgfqpoint{4.180570in}{1.703997in}}%
\pgfpathlineto{\pgfqpoint{4.188674in}{1.715075in}}%
\pgfpathlineto{\pgfqpoint{4.196772in}{1.726106in}}%
\pgfpathlineto{\pgfqpoint{4.204864in}{1.737088in}}%
\pgfpathlineto{\pgfqpoint{4.190742in}{1.736392in}}%
\pgfpathlineto{\pgfqpoint{4.176629in}{1.735769in}}%
\pgfpathlineto{\pgfqpoint{4.162525in}{1.735218in}}%
\pgfpathlineto{\pgfqpoint{4.148430in}{1.734740in}}%
\pgfpathlineto{\pgfqpoint{4.140330in}{1.723916in}}%
\pgfpathlineto{\pgfqpoint{4.132223in}{1.713047in}}%
\pgfpathlineto{\pgfqpoint{4.124112in}{1.702138in}}%
\pgfpathlineto{\pgfqpoint{4.115995in}{1.691189in}}%
\pgfpathclose%
\pgfusepath{fill}%
\end{pgfscope}%
\begin{pgfscope}%
\pgfpathrectangle{\pgfqpoint{1.150000in}{0.150000in}}{\pgfqpoint{5.700000in}{5.700000in}}%
\pgfusepath{clip}%
\pgfsetbuttcap%
\pgfsetroundjoin%
\definecolor{currentfill}{rgb}{0.275191,0.194905,0.496005}%
\pgfsetfillcolor{currentfill}%
\pgfsetfillopacity{0.700000}%
\pgfsetlinewidth{0.000000pt}%
\definecolor{currentstroke}{rgb}{0.000000,0.000000,0.000000}%
\pgfsetstrokecolor{currentstroke}%
\pgfsetdash{}{0pt}%
\pgfpathmoveto{\pgfqpoint{4.382611in}{1.833279in}}%
\pgfpathlineto{\pgfqpoint{4.396806in}{1.834893in}}%
\pgfpathlineto{\pgfqpoint{4.411012in}{1.836580in}}%
\pgfpathlineto{\pgfqpoint{4.425228in}{1.838339in}}%
\pgfpathlineto{\pgfqpoint{4.439454in}{1.840170in}}%
\pgfpathlineto{\pgfqpoint{4.447474in}{1.850962in}}%
\pgfpathlineto{\pgfqpoint{4.455488in}{1.861673in}}%
\pgfpathlineto{\pgfqpoint{4.463496in}{1.872301in}}%
\pgfpathlineto{\pgfqpoint{4.471498in}{1.882845in}}%
\pgfpathlineto{\pgfqpoint{4.457279in}{1.880911in}}%
\pgfpathlineto{\pgfqpoint{4.443071in}{1.879050in}}%
\pgfpathlineto{\pgfqpoint{4.428872in}{1.877260in}}%
\pgfpathlineto{\pgfqpoint{4.414685in}{1.875542in}}%
\pgfpathlineto{\pgfqpoint{4.406675in}{1.865093in}}%
\pgfpathlineto{\pgfqpoint{4.398659in}{1.854565in}}%
\pgfpathlineto{\pgfqpoint{4.390638in}{1.843960in}}%
\pgfpathlineto{\pgfqpoint{4.382611in}{1.833279in}}%
\pgfpathclose%
\pgfusepath{fill}%
\end{pgfscope}%
\begin{pgfscope}%
\pgfpathrectangle{\pgfqpoint{1.150000in}{0.150000in}}{\pgfqpoint{5.700000in}{5.700000in}}%
\pgfusepath{clip}%
\pgfsetbuttcap%
\pgfsetroundjoin%
\definecolor{currentfill}{rgb}{0.278791,0.062145,0.386592}%
\pgfsetfillcolor{currentfill}%
\pgfsetfillopacity{0.700000}%
\pgfsetlinewidth{0.000000pt}%
\definecolor{currentstroke}{rgb}{0.000000,0.000000,0.000000}%
\pgfsetstrokecolor{currentstroke}%
\pgfsetdash{}{0pt}%
\pgfpathmoveto{\pgfqpoint{2.799454in}{1.593018in}}%
\pgfpathlineto{\pgfqpoint{2.813354in}{1.584778in}}%
\pgfpathlineto{\pgfqpoint{2.827256in}{1.576628in}}%
\pgfpathlineto{\pgfqpoint{2.841160in}{1.568567in}}%
\pgfpathlineto{\pgfqpoint{2.855066in}{1.560596in}}%
\pgfpathlineto{\pgfqpoint{2.863804in}{1.563385in}}%
\pgfpathlineto{\pgfqpoint{2.872528in}{1.566402in}}%
\pgfpathlineto{\pgfqpoint{2.881238in}{1.569639in}}%
\pgfpathlineto{\pgfqpoint{2.889935in}{1.573092in}}%
\pgfpathlineto{\pgfqpoint{2.876059in}{1.580670in}}%
\pgfpathlineto{\pgfqpoint{2.862185in}{1.588337in}}%
\pgfpathlineto{\pgfqpoint{2.848314in}{1.596092in}}%
\pgfpathlineto{\pgfqpoint{2.834445in}{1.603938in}}%
\pgfpathlineto{\pgfqpoint{2.825718in}{1.600871in}}%
\pgfpathlineto{\pgfqpoint{2.816978in}{1.598025in}}%
\pgfpathlineto{\pgfqpoint{2.808223in}{1.595405in}}%
\pgfpathlineto{\pgfqpoint{2.799454in}{1.593018in}}%
\pgfpathclose%
\pgfusepath{fill}%
\end{pgfscope}%
\begin{pgfscope}%
\pgfpathrectangle{\pgfqpoint{1.150000in}{0.150000in}}{\pgfqpoint{5.700000in}{5.700000in}}%
\pgfusepath{clip}%
\pgfsetbuttcap%
\pgfsetroundjoin%
\definecolor{currentfill}{rgb}{0.282910,0.105393,0.426902}%
\pgfsetfillcolor{currentfill}%
\pgfsetfillopacity{0.700000}%
\pgfsetlinewidth{0.000000pt}%
\definecolor{currentstroke}{rgb}{0.000000,0.000000,0.000000}%
\pgfsetstrokecolor{currentstroke}%
\pgfsetdash{}{0pt}%
\pgfpathmoveto{\pgfqpoint{4.027118in}{1.647270in}}%
\pgfpathlineto{\pgfqpoint{4.041194in}{1.647105in}}%
\pgfpathlineto{\pgfqpoint{4.055278in}{1.647013in}}%
\pgfpathlineto{\pgfqpoint{4.069372in}{1.646993in}}%
\pgfpathlineto{\pgfqpoint{4.083474in}{1.647047in}}%
\pgfpathlineto{\pgfqpoint{4.091612in}{1.658129in}}%
\pgfpathlineto{\pgfqpoint{4.099745in}{1.669182in}}%
\pgfpathlineto{\pgfqpoint{4.107873in}{1.680202in}}%
\pgfpathlineto{\pgfqpoint{4.115995in}{1.691189in}}%
\pgfpathlineto{\pgfqpoint{4.101901in}{1.690949in}}%
\pgfpathlineto{\pgfqpoint{4.087816in}{1.690782in}}%
\pgfpathlineto{\pgfqpoint{4.073741in}{1.690688in}}%
\pgfpathlineto{\pgfqpoint{4.059673in}{1.690668in}}%
\pgfpathlineto{\pgfqpoint{4.051543in}{1.679860in}}%
\pgfpathlineto{\pgfqpoint{4.043407in}{1.669023in}}%
\pgfpathlineto{\pgfqpoint{4.035265in}{1.658158in}}%
\pgfpathlineto{\pgfqpoint{4.027118in}{1.647270in}}%
\pgfpathclose%
\pgfusepath{fill}%
\end{pgfscope}%
\begin{pgfscope}%
\pgfpathrectangle{\pgfqpoint{1.150000in}{0.150000in}}{\pgfqpoint{5.700000in}{5.700000in}}%
\pgfusepath{clip}%
\pgfsetbuttcap%
\pgfsetroundjoin%
\definecolor{currentfill}{rgb}{0.269308,0.218818,0.509577}%
\pgfsetfillcolor{currentfill}%
\pgfsetfillopacity{0.700000}%
\pgfsetlinewidth{0.000000pt}%
\definecolor{currentstroke}{rgb}{0.000000,0.000000,0.000000}%
\pgfsetstrokecolor{currentstroke}%
\pgfsetdash{}{0pt}%
\pgfpathmoveto{\pgfqpoint{4.471498in}{1.882845in}}%
\pgfpathlineto{\pgfqpoint{4.485727in}{1.884850in}}%
\pgfpathlineto{\pgfqpoint{4.499967in}{1.886927in}}%
\pgfpathlineto{\pgfqpoint{4.514217in}{1.889076in}}%
\pgfpathlineto{\pgfqpoint{4.528479in}{1.891297in}}%
\pgfpathlineto{\pgfqpoint{4.536467in}{1.901846in}}%
\pgfpathlineto{\pgfqpoint{4.544450in}{1.912304in}}%
\pgfpathlineto{\pgfqpoint{4.552426in}{1.922670in}}%
\pgfpathlineto{\pgfqpoint{4.560396in}{1.932943in}}%
\pgfpathlineto{\pgfqpoint{4.546142in}{1.930641in}}%
\pgfpathlineto{\pgfqpoint{4.531900in}{1.928410in}}%
\pgfpathlineto{\pgfqpoint{4.517667in}{1.926251in}}%
\pgfpathlineto{\pgfqpoint{4.503446in}{1.924164in}}%
\pgfpathlineto{\pgfqpoint{4.495468in}{1.913964in}}%
\pgfpathlineto{\pgfqpoint{4.487484in}{1.903678in}}%
\pgfpathlineto{\pgfqpoint{4.479494in}{1.893304in}}%
\pgfpathlineto{\pgfqpoint{4.471498in}{1.882845in}}%
\pgfpathclose%
\pgfusepath{fill}%
\end{pgfscope}%
\begin{pgfscope}%
\pgfpathrectangle{\pgfqpoint{1.150000in}{0.150000in}}{\pgfqpoint{5.700000in}{5.700000in}}%
\pgfusepath{clip}%
\pgfsetbuttcap%
\pgfsetroundjoin%
\definecolor{currentfill}{rgb}{0.188923,0.410910,0.556326}%
\pgfsetfillcolor{currentfill}%
\pgfsetfillopacity{0.700000}%
\pgfsetlinewidth{0.000000pt}%
\definecolor{currentstroke}{rgb}{0.000000,0.000000,0.000000}%
\pgfsetstrokecolor{currentstroke}%
\pgfsetdash{}{0pt}%
\pgfpathmoveto{\pgfqpoint{5.360358in}{2.363449in}}%
\pgfpathlineto{\pgfqpoint{5.374962in}{2.368147in}}%
\pgfpathlineto{\pgfqpoint{5.389579in}{2.372917in}}%
\pgfpathlineto{\pgfqpoint{5.404210in}{2.377757in}}%
\pgfpathlineto{\pgfqpoint{5.418854in}{2.382669in}}%
\pgfpathlineto{\pgfqpoint{5.426427in}{2.388483in}}%
\pgfpathlineto{\pgfqpoint{5.433990in}{2.394185in}}%
\pgfpathlineto{\pgfqpoint{5.441544in}{2.399777in}}%
\pgfpathlineto{\pgfqpoint{5.449089in}{2.405262in}}%
\pgfpathlineto{\pgfqpoint{5.434461in}{2.400487in}}%
\pgfpathlineto{\pgfqpoint{5.419847in}{2.395783in}}%
\pgfpathlineto{\pgfqpoint{5.405246in}{2.391150in}}%
\pgfpathlineto{\pgfqpoint{5.390658in}{2.386587in}}%
\pgfpathlineto{\pgfqpoint{5.383096in}{2.380958in}}%
\pgfpathlineto{\pgfqpoint{5.375525in}{2.375227in}}%
\pgfpathlineto{\pgfqpoint{5.367946in}{2.369391in}}%
\pgfpathlineto{\pgfqpoint{5.360358in}{2.363449in}}%
\pgfpathclose%
\pgfusepath{fill}%
\end{pgfscope}%
\begin{pgfscope}%
\pgfpathrectangle{\pgfqpoint{1.150000in}{0.150000in}}{\pgfqpoint{5.700000in}{5.700000in}}%
\pgfusepath{clip}%
\pgfsetbuttcap%
\pgfsetroundjoin%
\definecolor{currentfill}{rgb}{0.268510,0.009605,0.335427}%
\pgfsetfillcolor{currentfill}%
\pgfsetfillopacity{0.700000}%
\pgfsetlinewidth{0.000000pt}%
\definecolor{currentstroke}{rgb}{0.000000,0.000000,0.000000}%
\pgfsetstrokecolor{currentstroke}%
\pgfsetdash{}{0pt}%
\pgfpathmoveto{\pgfqpoint{3.526256in}{1.484967in}}%
\pgfpathlineto{\pgfqpoint{3.540211in}{1.481775in}}%
\pgfpathlineto{\pgfqpoint{3.554171in}{1.478659in}}%
\pgfpathlineto{\pgfqpoint{3.568138in}{1.475619in}}%
\pgfpathlineto{\pgfqpoint{3.582111in}{1.472656in}}%
\pgfpathlineto{\pgfqpoint{3.590433in}{1.481882in}}%
\pgfpathlineto{\pgfqpoint{3.598747in}{1.491178in}}%
\pgfpathlineto{\pgfqpoint{3.607055in}{1.500540in}}%
\pgfpathlineto{\pgfqpoint{3.615357in}{1.509965in}}%
\pgfpathlineto{\pgfqpoint{3.601398in}{1.512640in}}%
\pgfpathlineto{\pgfqpoint{3.587445in}{1.515392in}}%
\pgfpathlineto{\pgfqpoint{3.573500in}{1.518220in}}%
\pgfpathlineto{\pgfqpoint{3.559560in}{1.521125in}}%
\pgfpathlineto{\pgfqpoint{3.551245in}{1.511981in}}%
\pgfpathlineto{\pgfqpoint{3.542922in}{1.502903in}}%
\pgfpathlineto{\pgfqpoint{3.534593in}{1.493898in}}%
\pgfpathlineto{\pgfqpoint{3.526256in}{1.484967in}}%
\pgfpathclose%
\pgfusepath{fill}%
\end{pgfscope}%
\begin{pgfscope}%
\pgfpathrectangle{\pgfqpoint{1.150000in}{0.150000in}}{\pgfqpoint{5.700000in}{5.700000in}}%
\pgfusepath{clip}%
\pgfsetbuttcap%
\pgfsetroundjoin%
\definecolor{currentfill}{rgb}{0.262138,0.242286,0.520837}%
\pgfsetfillcolor{currentfill}%
\pgfsetfillopacity{0.700000}%
\pgfsetlinewidth{0.000000pt}%
\definecolor{currentstroke}{rgb}{0.000000,0.000000,0.000000}%
\pgfsetstrokecolor{currentstroke}%
\pgfsetdash{}{0pt}%
\pgfpathmoveto{\pgfqpoint{4.560396in}{1.932943in}}%
\pgfpathlineto{\pgfqpoint{4.574661in}{1.935317in}}%
\pgfpathlineto{\pgfqpoint{4.588936in}{1.937763in}}%
\pgfpathlineto{\pgfqpoint{4.603222in}{1.940280in}}%
\pgfpathlineto{\pgfqpoint{4.617520in}{1.942869in}}%
\pgfpathlineto{\pgfqpoint{4.625476in}{1.953117in}}%
\pgfpathlineto{\pgfqpoint{4.633425in}{1.963266in}}%
\pgfpathlineto{\pgfqpoint{4.641369in}{1.973315in}}%
\pgfpathlineto{\pgfqpoint{4.649305in}{1.983263in}}%
\pgfpathlineto{\pgfqpoint{4.635016in}{1.980614in}}%
\pgfpathlineto{\pgfqpoint{4.620737in}{1.978036in}}%
\pgfpathlineto{\pgfqpoint{4.606470in}{1.975530in}}%
\pgfpathlineto{\pgfqpoint{4.592213in}{1.973095in}}%
\pgfpathlineto{\pgfqpoint{4.584268in}{1.963199in}}%
\pgfpathlineto{\pgfqpoint{4.576317in}{1.953208in}}%
\pgfpathlineto{\pgfqpoint{4.568360in}{1.943123in}}%
\pgfpathlineto{\pgfqpoint{4.560396in}{1.932943in}}%
\pgfpathclose%
\pgfusepath{fill}%
\end{pgfscope}%
\begin{pgfscope}%
\pgfpathrectangle{\pgfqpoint{1.150000in}{0.150000in}}{\pgfqpoint{5.700000in}{5.700000in}}%
\pgfusepath{clip}%
\pgfsetbuttcap%
\pgfsetroundjoin%
\definecolor{currentfill}{rgb}{0.267004,0.004874,0.329415}%
\pgfsetfillcolor{currentfill}%
\pgfsetfillopacity{0.700000}%
\pgfsetlinewidth{0.000000pt}%
\definecolor{currentstroke}{rgb}{0.000000,0.000000,0.000000}%
\pgfsetstrokecolor{currentstroke}%
\pgfsetdash{}{0pt}%
\pgfpathmoveto{\pgfqpoint{3.146606in}{1.485366in}}%
\pgfpathlineto{\pgfqpoint{3.160513in}{1.479606in}}%
\pgfpathlineto{\pgfqpoint{3.174425in}{1.473927in}}%
\pgfpathlineto{\pgfqpoint{3.188341in}{1.468330in}}%
\pgfpathlineto{\pgfqpoint{3.202261in}{1.462815in}}%
\pgfpathlineto{\pgfqpoint{3.210772in}{1.469045in}}%
\pgfpathlineto{\pgfqpoint{3.219274in}{1.475430in}}%
\pgfpathlineto{\pgfqpoint{3.227765in}{1.481964in}}%
\pgfpathlineto{\pgfqpoint{3.236248in}{1.488642in}}%
\pgfpathlineto{\pgfqpoint{3.222349in}{1.493808in}}%
\pgfpathlineto{\pgfqpoint{3.208456in}{1.499055in}}%
\pgfpathlineto{\pgfqpoint{3.194567in}{1.504384in}}%
\pgfpathlineto{\pgfqpoint{3.180682in}{1.509794in}}%
\pgfpathlineto{\pgfqpoint{3.172178in}{1.503459in}}%
\pgfpathlineto{\pgfqpoint{3.163664in}{1.497272in}}%
\pgfpathlineto{\pgfqpoint{3.155140in}{1.491239in}}%
\pgfpathlineto{\pgfqpoint{3.146606in}{1.485366in}}%
\pgfpathclose%
\pgfusepath{fill}%
\end{pgfscope}%
\begin{pgfscope}%
\pgfpathrectangle{\pgfqpoint{1.150000in}{0.150000in}}{\pgfqpoint{5.700000in}{5.700000in}}%
\pgfusepath{clip}%
\pgfsetbuttcap%
\pgfsetroundjoin%
\definecolor{currentfill}{rgb}{0.281446,0.084320,0.407414}%
\pgfsetfillcolor{currentfill}%
\pgfsetfillopacity{0.700000}%
\pgfsetlinewidth{0.000000pt}%
\definecolor{currentstroke}{rgb}{0.000000,0.000000,0.000000}%
\pgfsetstrokecolor{currentstroke}%
\pgfsetdash{}{0pt}%
\pgfpathmoveto{\pgfqpoint{3.938222in}{1.605747in}}%
\pgfpathlineto{\pgfqpoint{3.952273in}{1.605082in}}%
\pgfpathlineto{\pgfqpoint{3.966332in}{1.604491in}}%
\pgfpathlineto{\pgfqpoint{3.980400in}{1.603973in}}%
\pgfpathlineto{\pgfqpoint{3.994476in}{1.603528in}}%
\pgfpathlineto{\pgfqpoint{4.002645in}{1.614486in}}%
\pgfpathlineto{\pgfqpoint{4.010808in}{1.625431in}}%
\pgfpathlineto{\pgfqpoint{4.018966in}{1.636360in}}%
\pgfpathlineto{\pgfqpoint{4.027118in}{1.647270in}}%
\pgfpathlineto{\pgfqpoint{4.013051in}{1.647509in}}%
\pgfpathlineto{\pgfqpoint{3.998992in}{1.647820in}}%
\pgfpathlineto{\pgfqpoint{3.984942in}{1.648205in}}%
\pgfpathlineto{\pgfqpoint{3.970901in}{1.648664in}}%
\pgfpathlineto{\pgfqpoint{3.962739in}{1.637953in}}%
\pgfpathlineto{\pgfqpoint{3.954572in}{1.627228in}}%
\pgfpathlineto{\pgfqpoint{3.946400in}{1.616492in}}%
\pgfpathlineto{\pgfqpoint{3.938222in}{1.605747in}}%
\pgfpathclose%
\pgfusepath{fill}%
\end{pgfscope}%
\begin{pgfscope}%
\pgfpathrectangle{\pgfqpoint{1.150000in}{0.150000in}}{\pgfqpoint{5.700000in}{5.700000in}}%
\pgfusepath{clip}%
\pgfsetbuttcap%
\pgfsetroundjoin%
\definecolor{currentfill}{rgb}{0.255645,0.260703,0.528312}%
\pgfsetfillcolor{currentfill}%
\pgfsetfillopacity{0.700000}%
\pgfsetlinewidth{0.000000pt}%
\definecolor{currentstroke}{rgb}{0.000000,0.000000,0.000000}%
\pgfsetstrokecolor{currentstroke}%
\pgfsetdash{}{0pt}%
\pgfpathmoveto{\pgfqpoint{4.649305in}{1.983263in}}%
\pgfpathlineto{\pgfqpoint{4.663606in}{1.985984in}}%
\pgfpathlineto{\pgfqpoint{4.677918in}{1.988776in}}%
\pgfpathlineto{\pgfqpoint{4.692241in}{1.991640in}}%
\pgfpathlineto{\pgfqpoint{4.706575in}{1.994576in}}%
\pgfpathlineto{\pgfqpoint{4.714497in}{2.004470in}}%
\pgfpathlineto{\pgfqpoint{4.722413in}{2.014259in}}%
\pgfpathlineto{\pgfqpoint{4.730321in}{2.023940in}}%
\pgfpathlineto{\pgfqpoint{4.738223in}{2.033515in}}%
\pgfpathlineto{\pgfqpoint{4.723896in}{2.030540in}}%
\pgfpathlineto{\pgfqpoint{4.709582in}{2.027637in}}%
\pgfpathlineto{\pgfqpoint{4.695278in}{2.024806in}}%
\pgfpathlineto{\pgfqpoint{4.680986in}{2.022046in}}%
\pgfpathlineto{\pgfqpoint{4.673076in}{2.012502in}}%
\pgfpathlineto{\pgfqpoint{4.665159in}{2.002857in}}%
\pgfpathlineto{\pgfqpoint{4.657235in}{1.993111in}}%
\pgfpathlineto{\pgfqpoint{4.649305in}{1.983263in}}%
\pgfpathclose%
\pgfusepath{fill}%
\end{pgfscope}%
\begin{pgfscope}%
\pgfpathrectangle{\pgfqpoint{1.150000in}{0.150000in}}{\pgfqpoint{5.700000in}{5.700000in}}%
\pgfusepath{clip}%
\pgfsetbuttcap%
\pgfsetroundjoin%
\definecolor{currentfill}{rgb}{0.195860,0.395433,0.555276}%
\pgfsetfillcolor{currentfill}%
\pgfsetfillopacity{0.700000}%
\pgfsetlinewidth{0.000000pt}%
\definecolor{currentstroke}{rgb}{0.000000,0.000000,0.000000}%
\pgfsetstrokecolor{currentstroke}%
\pgfsetdash{}{0pt}%
\pgfpathmoveto{\pgfqpoint{5.271570in}{2.320015in}}%
\pgfpathlineto{\pgfqpoint{5.286137in}{2.324544in}}%
\pgfpathlineto{\pgfqpoint{5.300716in}{2.329145in}}%
\pgfpathlineto{\pgfqpoint{5.315309in}{2.333817in}}%
\pgfpathlineto{\pgfqpoint{5.329915in}{2.338559in}}%
\pgfpathlineto{\pgfqpoint{5.337539in}{2.344954in}}%
\pgfpathlineto{\pgfqpoint{5.345154in}{2.351232in}}%
\pgfpathlineto{\pgfqpoint{5.352760in}{2.357397in}}%
\pgfpathlineto{\pgfqpoint{5.360358in}{2.363449in}}%
\pgfpathlineto{\pgfqpoint{5.345767in}{2.358821in}}%
\pgfpathlineto{\pgfqpoint{5.331189in}{2.354263in}}%
\pgfpathlineto{\pgfqpoint{5.316625in}{2.349776in}}%
\pgfpathlineto{\pgfqpoint{5.302073in}{2.345360in}}%
\pgfpathlineto{\pgfqpoint{5.294460in}{2.339186in}}%
\pgfpathlineto{\pgfqpoint{5.286839in}{2.332905in}}%
\pgfpathlineto{\pgfqpoint{5.279209in}{2.326516in}}%
\pgfpathlineto{\pgfqpoint{5.271570in}{2.320015in}}%
\pgfpathclose%
\pgfusepath{fill}%
\end{pgfscope}%
\begin{pgfscope}%
\pgfpathrectangle{\pgfqpoint{1.150000in}{0.150000in}}{\pgfqpoint{5.700000in}{5.700000in}}%
\pgfusepath{clip}%
\pgfsetbuttcap%
\pgfsetroundjoin%
\definecolor{currentfill}{rgb}{0.246811,0.283237,0.535941}%
\pgfsetfillcolor{currentfill}%
\pgfsetfillopacity{0.700000}%
\pgfsetlinewidth{0.000000pt}%
\definecolor{currentstroke}{rgb}{0.000000,0.000000,0.000000}%
\pgfsetstrokecolor{currentstroke}%
\pgfsetdash{}{0pt}%
\pgfpathmoveto{\pgfqpoint{4.738223in}{2.033515in}}%
\pgfpathlineto{\pgfqpoint{4.752560in}{2.036561in}}%
\pgfpathlineto{\pgfqpoint{4.766909in}{2.039678in}}%
\pgfpathlineto{\pgfqpoint{4.781270in}{2.042867in}}%
\pgfpathlineto{\pgfqpoint{4.795642in}{2.046126in}}%
\pgfpathlineto{\pgfqpoint{4.803528in}{2.055620in}}%
\pgfpathlineto{\pgfqpoint{4.811407in}{2.065002in}}%
\pgfpathlineto{\pgfqpoint{4.819279in}{2.074271in}}%
\pgfpathlineto{\pgfqpoint{4.827144in}{2.083429in}}%
\pgfpathlineto{\pgfqpoint{4.812780in}{2.080152in}}%
\pgfpathlineto{\pgfqpoint{4.798429in}{2.076946in}}%
\pgfpathlineto{\pgfqpoint{4.784088in}{2.073811in}}%
\pgfpathlineto{\pgfqpoint{4.769760in}{2.070748in}}%
\pgfpathlineto{\pgfqpoint{4.761886in}{2.061599in}}%
\pgfpathlineto{\pgfqpoint{4.754005in}{2.052344in}}%
\pgfpathlineto{\pgfqpoint{4.746117in}{2.042983in}}%
\pgfpathlineto{\pgfqpoint{4.738223in}{2.033515in}}%
\pgfpathclose%
\pgfusepath{fill}%
\end{pgfscope}%
\begin{pgfscope}%
\pgfpathrectangle{\pgfqpoint{1.150000in}{0.150000in}}{\pgfqpoint{5.700000in}{5.700000in}}%
\pgfusepath{clip}%
\pgfsetbuttcap%
\pgfsetroundjoin%
\definecolor{currentfill}{rgb}{0.267004,0.004874,0.329415}%
\pgfsetfillcolor{currentfill}%
\pgfsetfillopacity{0.700000}%
\pgfsetlinewidth{0.000000pt}%
\definecolor{currentstroke}{rgb}{0.000000,0.000000,0.000000}%
\pgfsetstrokecolor{currentstroke}%
\pgfsetdash{}{0pt}%
\pgfpathmoveto{\pgfqpoint{3.291888in}{1.468783in}}%
\pgfpathlineto{\pgfqpoint{3.305810in}{1.464018in}}%
\pgfpathlineto{\pgfqpoint{3.319737in}{1.459332in}}%
\pgfpathlineto{\pgfqpoint{3.333669in}{1.454725in}}%
\pgfpathlineto{\pgfqpoint{3.347607in}{1.450197in}}%
\pgfpathlineto{\pgfqpoint{3.356040in}{1.457687in}}%
\pgfpathlineto{\pgfqpoint{3.364464in}{1.465299in}}%
\pgfpathlineto{\pgfqpoint{3.372880in}{1.473030in}}%
\pgfpathlineto{\pgfqpoint{3.381288in}{1.480873in}}%
\pgfpathlineto{\pgfqpoint{3.367370in}{1.485072in}}%
\pgfpathlineto{\pgfqpoint{3.353456in}{1.489350in}}%
\pgfpathlineto{\pgfqpoint{3.339548in}{1.493707in}}%
\pgfpathlineto{\pgfqpoint{3.325645in}{1.498144in}}%
\pgfpathlineto{\pgfqpoint{3.317219in}{1.490621in}}%
\pgfpathlineto{\pgfqpoint{3.308784in}{1.483216in}}%
\pgfpathlineto{\pgfqpoint{3.300340in}{1.475935in}}%
\pgfpathlineto{\pgfqpoint{3.291888in}{1.468783in}}%
\pgfpathclose%
\pgfusepath{fill}%
\end{pgfscope}%
\begin{pgfscope}%
\pgfpathrectangle{\pgfqpoint{1.150000in}{0.150000in}}{\pgfqpoint{5.700000in}{5.700000in}}%
\pgfusepath{clip}%
\pgfsetbuttcap%
\pgfsetroundjoin%
\definecolor{currentfill}{rgb}{0.203063,0.379716,0.553925}%
\pgfsetfillcolor{currentfill}%
\pgfsetfillopacity{0.700000}%
\pgfsetlinewidth{0.000000pt}%
\definecolor{currentstroke}{rgb}{0.000000,0.000000,0.000000}%
\pgfsetstrokecolor{currentstroke}%
\pgfsetdash{}{0pt}%
\pgfpathmoveto{\pgfqpoint{5.182737in}{2.275078in}}%
\pgfpathlineto{\pgfqpoint{5.197266in}{2.279416in}}%
\pgfpathlineto{\pgfqpoint{5.211807in}{2.283825in}}%
\pgfpathlineto{\pgfqpoint{5.226361in}{2.288306in}}%
\pgfpathlineto{\pgfqpoint{5.240928in}{2.292857in}}%
\pgfpathlineto{\pgfqpoint{5.248602in}{2.299823in}}%
\pgfpathlineto{\pgfqpoint{5.256267in}{2.306670in}}%
\pgfpathlineto{\pgfqpoint{5.263923in}{2.313400in}}%
\pgfpathlineto{\pgfqpoint{5.271570in}{2.320015in}}%
\pgfpathlineto{\pgfqpoint{5.257017in}{2.315556in}}%
\pgfpathlineto{\pgfqpoint{5.242476in}{2.311167in}}%
\pgfpathlineto{\pgfqpoint{5.227949in}{2.306850in}}%
\pgfpathlineto{\pgfqpoint{5.213434in}{2.302603in}}%
\pgfpathlineto{\pgfqpoint{5.205773in}{2.295889in}}%
\pgfpathlineto{\pgfqpoint{5.198103in}{2.289065in}}%
\pgfpathlineto{\pgfqpoint{5.190424in}{2.282128in}}%
\pgfpathlineto{\pgfqpoint{5.182737in}{2.275078in}}%
\pgfpathclose%
\pgfusepath{fill}%
\end{pgfscope}%
\begin{pgfscope}%
\pgfpathrectangle{\pgfqpoint{1.150000in}{0.150000in}}{\pgfqpoint{5.700000in}{5.700000in}}%
\pgfusepath{clip}%
\pgfsetbuttcap%
\pgfsetroundjoin%
\definecolor{currentfill}{rgb}{0.237441,0.305202,0.541921}%
\pgfsetfillcolor{currentfill}%
\pgfsetfillopacity{0.700000}%
\pgfsetlinewidth{0.000000pt}%
\definecolor{currentstroke}{rgb}{0.000000,0.000000,0.000000}%
\pgfsetstrokecolor{currentstroke}%
\pgfsetdash{}{0pt}%
\pgfpathmoveto{\pgfqpoint{4.827144in}{2.083429in}}%
\pgfpathlineto{\pgfqpoint{4.841519in}{2.086778in}}%
\pgfpathlineto{\pgfqpoint{4.855906in}{2.090198in}}%
\pgfpathlineto{\pgfqpoint{4.870305in}{2.093690in}}%
\pgfpathlineto{\pgfqpoint{4.884716in}{2.097252in}}%
\pgfpathlineto{\pgfqpoint{4.892564in}{2.106303in}}%
\pgfpathlineto{\pgfqpoint{4.900405in}{2.115237in}}%
\pgfpathlineto{\pgfqpoint{4.908238in}{2.124056in}}%
\pgfpathlineto{\pgfqpoint{4.916064in}{2.132759in}}%
\pgfpathlineto{\pgfqpoint{4.901662in}{2.129201in}}%
\pgfpathlineto{\pgfqpoint{4.887273in}{2.125714in}}%
\pgfpathlineto{\pgfqpoint{4.872895in}{2.122298in}}%
\pgfpathlineto{\pgfqpoint{4.858530in}{2.118953in}}%
\pgfpathlineto{\pgfqpoint{4.850694in}{2.110238in}}%
\pgfpathlineto{\pgfqpoint{4.842851in}{2.101412in}}%
\pgfpathlineto{\pgfqpoint{4.835001in}{2.092476in}}%
\pgfpathlineto{\pgfqpoint{4.827144in}{2.083429in}}%
\pgfpathclose%
\pgfusepath{fill}%
\end{pgfscope}%
\begin{pgfscope}%
\pgfpathrectangle{\pgfqpoint{1.150000in}{0.150000in}}{\pgfqpoint{5.700000in}{5.700000in}}%
\pgfusepath{clip}%
\pgfsetbuttcap%
\pgfsetroundjoin%
\definecolor{currentfill}{rgb}{0.271305,0.019942,0.347269}%
\pgfsetfillcolor{currentfill}%
\pgfsetfillopacity{0.700000}%
\pgfsetlinewidth{0.000000pt}%
\definecolor{currentstroke}{rgb}{0.000000,0.000000,0.000000}%
\pgfsetstrokecolor{currentstroke}%
\pgfsetdash{}{0pt}%
\pgfpathmoveto{\pgfqpoint{3.001046in}{1.515614in}}%
\pgfpathlineto{\pgfqpoint{3.014948in}{1.508816in}}%
\pgfpathlineto{\pgfqpoint{3.028854in}{1.502102in}}%
\pgfpathlineto{\pgfqpoint{3.042763in}{1.495473in}}%
\pgfpathlineto{\pgfqpoint{3.056676in}{1.488928in}}%
\pgfpathlineto{\pgfqpoint{3.065278in}{1.493729in}}%
\pgfpathlineto{\pgfqpoint{3.073869in}{1.498717in}}%
\pgfpathlineto{\pgfqpoint{3.082449in}{1.503886in}}%
\pgfpathlineto{\pgfqpoint{3.091017in}{1.509231in}}%
\pgfpathlineto{\pgfqpoint{3.077130in}{1.515405in}}%
\pgfpathlineto{\pgfqpoint{3.063246in}{1.521663in}}%
\pgfpathlineto{\pgfqpoint{3.049366in}{1.528005in}}%
\pgfpathlineto{\pgfqpoint{3.035489in}{1.534431in}}%
\pgfpathlineto{\pgfqpoint{3.026896in}{1.529450in}}%
\pgfpathlineto{\pgfqpoint{3.018291in}{1.524649in}}%
\pgfpathlineto{\pgfqpoint{3.009674in}{1.520035in}}%
\pgfpathlineto{\pgfqpoint{3.001046in}{1.515614in}}%
\pgfpathclose%
\pgfusepath{fill}%
\end{pgfscope}%
\begin{pgfscope}%
\pgfpathrectangle{\pgfqpoint{1.150000in}{0.150000in}}{\pgfqpoint{5.700000in}{5.700000in}}%
\pgfusepath{clip}%
\pgfsetbuttcap%
\pgfsetroundjoin%
\definecolor{currentfill}{rgb}{0.210503,0.363727,0.552206}%
\pgfsetfillcolor{currentfill}%
\pgfsetfillopacity{0.700000}%
\pgfsetlinewidth{0.000000pt}%
\definecolor{currentstroke}{rgb}{0.000000,0.000000,0.000000}%
\pgfsetstrokecolor{currentstroke}%
\pgfsetdash{}{0pt}%
\pgfpathmoveto{\pgfqpoint{5.093869in}{2.228778in}}%
\pgfpathlineto{\pgfqpoint{5.108359in}{2.232902in}}%
\pgfpathlineto{\pgfqpoint{5.122861in}{2.237098in}}%
\pgfpathlineto{\pgfqpoint{5.137377in}{2.241364in}}%
\pgfpathlineto{\pgfqpoint{5.151905in}{2.245702in}}%
\pgfpathlineto{\pgfqpoint{5.159626in}{2.253225in}}%
\pgfpathlineto{\pgfqpoint{5.167338in}{2.260628in}}%
\pgfpathlineto{\pgfqpoint{5.175042in}{2.267911in}}%
\pgfpathlineto{\pgfqpoint{5.182737in}{2.275078in}}%
\pgfpathlineto{\pgfqpoint{5.168222in}{2.270810in}}%
\pgfpathlineto{\pgfqpoint{5.153719in}{2.266614in}}%
\pgfpathlineto{\pgfqpoint{5.139229in}{2.262488in}}%
\pgfpathlineto{\pgfqpoint{5.124751in}{2.258433in}}%
\pgfpathlineto{\pgfqpoint{5.117043in}{2.251189in}}%
\pgfpathlineto{\pgfqpoint{5.109326in}{2.243833in}}%
\pgfpathlineto{\pgfqpoint{5.101602in}{2.236363in}}%
\pgfpathlineto{\pgfqpoint{5.093869in}{2.228778in}}%
\pgfpathclose%
\pgfusepath{fill}%
\end{pgfscope}%
\begin{pgfscope}%
\pgfpathrectangle{\pgfqpoint{1.150000in}{0.150000in}}{\pgfqpoint{5.700000in}{5.700000in}}%
\pgfusepath{clip}%
\pgfsetbuttcap%
\pgfsetroundjoin%
\definecolor{currentfill}{rgb}{0.227802,0.326594,0.546532}%
\pgfsetfillcolor{currentfill}%
\pgfsetfillopacity{0.700000}%
\pgfsetlinewidth{0.000000pt}%
\definecolor{currentstroke}{rgb}{0.000000,0.000000,0.000000}%
\pgfsetstrokecolor{currentstroke}%
\pgfsetdash{}{0pt}%
\pgfpathmoveto{\pgfqpoint{4.916064in}{2.132759in}}%
\pgfpathlineto{\pgfqpoint{4.930477in}{2.136389in}}%
\pgfpathlineto{\pgfqpoint{4.944902in}{2.140089in}}%
\pgfpathlineto{\pgfqpoint{4.959340in}{2.143861in}}%
\pgfpathlineto{\pgfqpoint{4.973789in}{2.147704in}}%
\pgfpathlineto{\pgfqpoint{4.981597in}{2.156275in}}%
\pgfpathlineto{\pgfqpoint{4.989398in}{2.164727in}}%
\pgfpathlineto{\pgfqpoint{4.997190in}{2.173061in}}%
\pgfpathlineto{\pgfqpoint{5.004975in}{2.181277in}}%
\pgfpathlineto{\pgfqpoint{4.990535in}{2.177460in}}%
\pgfpathlineto{\pgfqpoint{4.976108in}{2.173714in}}%
\pgfpathlineto{\pgfqpoint{4.961693in}{2.170039in}}%
\pgfpathlineto{\pgfqpoint{4.947290in}{2.166436in}}%
\pgfpathlineto{\pgfqpoint{4.939495in}{2.158185in}}%
\pgfpathlineto{\pgfqpoint{4.931692in}{2.149823in}}%
\pgfpathlineto{\pgfqpoint{4.923882in}{2.141348in}}%
\pgfpathlineto{\pgfqpoint{4.916064in}{2.132759in}}%
\pgfpathclose%
\pgfusepath{fill}%
\end{pgfscope}%
\begin{pgfscope}%
\pgfpathrectangle{\pgfqpoint{1.150000in}{0.150000in}}{\pgfqpoint{5.700000in}{5.700000in}}%
\pgfusepath{clip}%
\pgfsetbuttcap%
\pgfsetroundjoin%
\definecolor{currentfill}{rgb}{0.218130,0.347432,0.550038}%
\pgfsetfillcolor{currentfill}%
\pgfsetfillopacity{0.700000}%
\pgfsetlinewidth{0.000000pt}%
\definecolor{currentstroke}{rgb}{0.000000,0.000000,0.000000}%
\pgfsetstrokecolor{currentstroke}%
\pgfsetdash{}{0pt}%
\pgfpathmoveto{\pgfqpoint{5.004975in}{2.181277in}}%
\pgfpathlineto{\pgfqpoint{5.019426in}{2.185165in}}%
\pgfpathlineto{\pgfqpoint{5.033890in}{2.189124in}}%
\pgfpathlineto{\pgfqpoint{5.048366in}{2.193155in}}%
\pgfpathlineto{\pgfqpoint{5.062855in}{2.197256in}}%
\pgfpathlineto{\pgfqpoint{5.070621in}{2.205316in}}%
\pgfpathlineto{\pgfqpoint{5.078378in}{2.213256in}}%
\pgfpathlineto{\pgfqpoint{5.086128in}{2.221076in}}%
\pgfpathlineto{\pgfqpoint{5.093869in}{2.228778in}}%
\pgfpathlineto{\pgfqpoint{5.079391in}{2.224724in}}%
\pgfpathlineto{\pgfqpoint{5.064926in}{2.220742in}}%
\pgfpathlineto{\pgfqpoint{5.050474in}{2.216831in}}%
\pgfpathlineto{\pgfqpoint{5.036033in}{2.212990in}}%
\pgfpathlineto{\pgfqpoint{5.028281in}{2.205232in}}%
\pgfpathlineto{\pgfqpoint{5.020520in}{2.197362in}}%
\pgfpathlineto{\pgfqpoint{5.012751in}{2.189377in}}%
\pgfpathlineto{\pgfqpoint{5.004975in}{2.181277in}}%
\pgfpathclose%
\pgfusepath{fill}%
\end{pgfscope}%
\begin{pgfscope}%
\pgfpathrectangle{\pgfqpoint{1.150000in}{0.150000in}}{\pgfqpoint{5.700000in}{5.700000in}}%
\pgfusepath{clip}%
\pgfsetbuttcap%
\pgfsetroundjoin%
\definecolor{currentfill}{rgb}{0.278791,0.062145,0.386592}%
\pgfsetfillcolor{currentfill}%
\pgfsetfillopacity{0.700000}%
\pgfsetlinewidth{0.000000pt}%
\definecolor{currentstroke}{rgb}{0.000000,0.000000,0.000000}%
\pgfsetstrokecolor{currentstroke}%
\pgfsetdash{}{0pt}%
\pgfpathmoveto{\pgfqpoint{3.849292in}{1.567056in}}%
\pgfpathlineto{\pgfqpoint{3.863321in}{1.565869in}}%
\pgfpathlineto{\pgfqpoint{3.877358in}{1.564756in}}%
\pgfpathlineto{\pgfqpoint{3.891402in}{1.563717in}}%
\pgfpathlineto{\pgfqpoint{3.905455in}{1.562751in}}%
\pgfpathlineto{\pgfqpoint{3.913655in}{1.573497in}}%
\pgfpathlineto{\pgfqpoint{3.921850in}{1.584247in}}%
\pgfpathlineto{\pgfqpoint{3.930038in}{1.594998in}}%
\pgfpathlineto{\pgfqpoint{3.938222in}{1.605747in}}%
\pgfpathlineto{\pgfqpoint{3.924179in}{1.606486in}}%
\pgfpathlineto{\pgfqpoint{3.910144in}{1.607298in}}%
\pgfpathlineto{\pgfqpoint{3.896118in}{1.608184in}}%
\pgfpathlineto{\pgfqpoint{3.882099in}{1.609145in}}%
\pgfpathlineto{\pgfqpoint{3.873906in}{1.598615in}}%
\pgfpathlineto{\pgfqpoint{3.865707in}{1.588088in}}%
\pgfpathlineto{\pgfqpoint{3.857502in}{1.577567in}}%
\pgfpathlineto{\pgfqpoint{3.849292in}{1.567056in}}%
\pgfpathclose%
\pgfusepath{fill}%
\end{pgfscope}%
\begin{pgfscope}%
\pgfpathrectangle{\pgfqpoint{1.150000in}{0.150000in}}{\pgfqpoint{5.700000in}{5.700000in}}%
\pgfusepath{clip}%
\pgfsetbuttcap%
\pgfsetroundjoin%
\definecolor{currentfill}{rgb}{0.283091,0.110553,0.431554}%
\pgfsetfillcolor{currentfill}%
\pgfsetfillopacity{0.700000}%
\pgfsetlinewidth{0.000000pt}%
\definecolor{currentstroke}{rgb}{0.000000,0.000000,0.000000}%
\pgfsetstrokecolor{currentstroke}%
\pgfsetdash{}{0pt}%
\pgfpathmoveto{\pgfqpoint{2.597146in}{1.697002in}}%
\pgfpathlineto{\pgfqpoint{2.611064in}{1.687226in}}%
\pgfpathlineto{\pgfqpoint{2.624982in}{1.677546in}}%
\pgfpathlineto{\pgfqpoint{2.638902in}{1.667961in}}%
\pgfpathlineto{\pgfqpoint{2.652822in}{1.658472in}}%
\pgfpathlineto{\pgfqpoint{2.661721in}{1.659014in}}%
\pgfpathlineto{\pgfqpoint{2.670604in}{1.659826in}}%
\pgfpathlineto{\pgfqpoint{2.679471in}{1.660900in}}%
\pgfpathlineto{\pgfqpoint{2.688321in}{1.662230in}}%
\pgfpathlineto{\pgfqpoint{2.674436in}{1.671301in}}%
\pgfpathlineto{\pgfqpoint{2.660552in}{1.680467in}}%
\pgfpathlineto{\pgfqpoint{2.646669in}{1.689729in}}%
\pgfpathlineto{\pgfqpoint{2.632788in}{1.699086in}}%
\pgfpathlineto{\pgfqpoint{2.623903in}{1.698166in}}%
\pgfpathlineto{\pgfqpoint{2.615001in}{1.697507in}}%
\pgfpathlineto{\pgfqpoint{2.606082in}{1.697117in}}%
\pgfpathlineto{\pgfqpoint{2.597146in}{1.697002in}}%
\pgfpathclose%
\pgfusepath{fill}%
\end{pgfscope}%
\begin{pgfscope}%
\pgfpathrectangle{\pgfqpoint{1.150000in}{0.150000in}}{\pgfqpoint{5.700000in}{5.700000in}}%
\pgfusepath{clip}%
\pgfsetbuttcap%
\pgfsetroundjoin%
\definecolor{currentfill}{rgb}{0.276022,0.044167,0.370164}%
\pgfsetfillcolor{currentfill}%
\pgfsetfillopacity{0.700000}%
\pgfsetlinewidth{0.000000pt}%
\definecolor{currentstroke}{rgb}{0.000000,0.000000,0.000000}%
\pgfsetstrokecolor{currentstroke}%
\pgfsetdash{}{0pt}%
\pgfpathmoveto{\pgfqpoint{3.760312in}{1.531654in}}%
\pgfpathlineto{\pgfqpoint{3.774321in}{1.529922in}}%
\pgfpathlineto{\pgfqpoint{3.788338in}{1.528266in}}%
\pgfpathlineto{\pgfqpoint{3.802362in}{1.526683in}}%
\pgfpathlineto{\pgfqpoint{3.816394in}{1.525175in}}%
\pgfpathlineto{\pgfqpoint{3.824627in}{1.535614in}}%
\pgfpathlineto{\pgfqpoint{3.832855in}{1.546076in}}%
\pgfpathlineto{\pgfqpoint{3.841076in}{1.556558in}}%
\pgfpathlineto{\pgfqpoint{3.849292in}{1.567056in}}%
\pgfpathlineto{\pgfqpoint{3.835271in}{1.568317in}}%
\pgfpathlineto{\pgfqpoint{3.821258in}{1.569652in}}%
\pgfpathlineto{\pgfqpoint{3.807252in}{1.571062in}}%
\pgfpathlineto{\pgfqpoint{3.793254in}{1.572546in}}%
\pgfpathlineto{\pgfqpoint{3.785028in}{1.562287in}}%
\pgfpathlineto{\pgfqpoint{3.776795in}{1.552050in}}%
\pgfpathlineto{\pgfqpoint{3.768556in}{1.541837in}}%
\pgfpathlineto{\pgfqpoint{3.760312in}{1.531654in}}%
\pgfpathclose%
\pgfusepath{fill}%
\end{pgfscope}%
\begin{pgfscope}%
\pgfpathrectangle{\pgfqpoint{1.150000in}{0.150000in}}{\pgfqpoint{5.700000in}{5.700000in}}%
\pgfusepath{clip}%
\pgfsetbuttcap%
\pgfsetroundjoin%
\definecolor{currentfill}{rgb}{0.267004,0.004874,0.329415}%
\pgfsetfillcolor{currentfill}%
\pgfsetfillopacity{0.700000}%
\pgfsetlinewidth{0.000000pt}%
\definecolor{currentstroke}{rgb}{0.000000,0.000000,0.000000}%
\pgfsetstrokecolor{currentstroke}%
\pgfsetdash{}{0pt}%
\pgfpathmoveto{\pgfqpoint{3.437018in}{1.464861in}}%
\pgfpathlineto{\pgfqpoint{3.450964in}{1.461052in}}%
\pgfpathlineto{\pgfqpoint{3.464916in}{1.457320in}}%
\pgfpathlineto{\pgfqpoint{3.478874in}{1.453666in}}%
\pgfpathlineto{\pgfqpoint{3.492838in}{1.450089in}}%
\pgfpathlineto{\pgfqpoint{3.501204in}{1.458673in}}%
\pgfpathlineto{\pgfqpoint{3.509562in}{1.467350in}}%
\pgfpathlineto{\pgfqpoint{3.517913in}{1.476117in}}%
\pgfpathlineto{\pgfqpoint{3.526256in}{1.484967in}}%
\pgfpathlineto{\pgfqpoint{3.512308in}{1.488236in}}%
\pgfpathlineto{\pgfqpoint{3.498366in}{1.491582in}}%
\pgfpathlineto{\pgfqpoint{3.484431in}{1.495005in}}%
\pgfpathlineto{\pgfqpoint{3.470501in}{1.498506in}}%
\pgfpathlineto{\pgfqpoint{3.462141in}{1.489956in}}%
\pgfpathlineto{\pgfqpoint{3.453774in}{1.481495in}}%
\pgfpathlineto{\pgfqpoint{3.445400in}{1.473129in}}%
\pgfpathlineto{\pgfqpoint{3.437018in}{1.464861in}}%
\pgfpathclose%
\pgfusepath{fill}%
\end{pgfscope}%
\begin{pgfscope}%
\pgfpathrectangle{\pgfqpoint{1.150000in}{0.150000in}}{\pgfqpoint{5.700000in}{5.700000in}}%
\pgfusepath{clip}%
\pgfsetbuttcap%
\pgfsetroundjoin%
\definecolor{currentfill}{rgb}{0.277018,0.050344,0.375715}%
\pgfsetfillcolor{currentfill}%
\pgfsetfillopacity{0.700000}%
\pgfsetlinewidth{0.000000pt}%
\definecolor{currentstroke}{rgb}{0.000000,0.000000,0.000000}%
\pgfsetstrokecolor{currentstroke}%
\pgfsetdash{}{0pt}%
\pgfpathmoveto{\pgfqpoint{2.855066in}{1.560596in}}%
\pgfpathlineto{\pgfqpoint{2.868975in}{1.552712in}}%
\pgfpathlineto{\pgfqpoint{2.882886in}{1.544917in}}%
\pgfpathlineto{\pgfqpoint{2.896799in}{1.537209in}}%
\pgfpathlineto{\pgfqpoint{2.910715in}{1.529589in}}%
\pgfpathlineto{\pgfqpoint{2.919423in}{1.532780in}}%
\pgfpathlineto{\pgfqpoint{2.928117in}{1.536193in}}%
\pgfpathlineto{\pgfqpoint{2.936799in}{1.539822in}}%
\pgfpathlineto{\pgfqpoint{2.945467in}{1.543660in}}%
\pgfpathlineto{\pgfqpoint{2.931580in}{1.550887in}}%
\pgfpathlineto{\pgfqpoint{2.917696in}{1.558201in}}%
\pgfpathlineto{\pgfqpoint{2.903814in}{1.565603in}}%
\pgfpathlineto{\pgfqpoint{2.889935in}{1.573092in}}%
\pgfpathlineto{\pgfqpoint{2.881238in}{1.569639in}}%
\pgfpathlineto{\pgfqpoint{2.872528in}{1.566402in}}%
\pgfpathlineto{\pgfqpoint{2.863804in}{1.563385in}}%
\pgfpathlineto{\pgfqpoint{2.855066in}{1.560596in}}%
\pgfpathclose%
\pgfusepath{fill}%
\end{pgfscope}%
\begin{pgfscope}%
\pgfpathrectangle{\pgfqpoint{1.150000in}{0.150000in}}{\pgfqpoint{5.700000in}{5.700000in}}%
\pgfusepath{clip}%
\pgfsetbuttcap%
\pgfsetroundjoin%
\definecolor{currentfill}{rgb}{0.272594,0.025563,0.353093}%
\pgfsetfillcolor{currentfill}%
\pgfsetfillopacity{0.700000}%
\pgfsetlinewidth{0.000000pt}%
\definecolor{currentstroke}{rgb}{0.000000,0.000000,0.000000}%
\pgfsetstrokecolor{currentstroke}%
\pgfsetdash{}{0pt}%
\pgfpathmoveto{\pgfqpoint{3.671260in}{1.500019in}}%
\pgfpathlineto{\pgfqpoint{3.685252in}{1.497721in}}%
\pgfpathlineto{\pgfqpoint{3.699252in}{1.495498in}}%
\pgfpathlineto{\pgfqpoint{3.713259in}{1.493350in}}%
\pgfpathlineto{\pgfqpoint{3.727273in}{1.491277in}}%
\pgfpathlineto{\pgfqpoint{3.735542in}{1.501310in}}%
\pgfpathlineto{\pgfqpoint{3.743805in}{1.511386in}}%
\pgfpathlineto{\pgfqpoint{3.752061in}{1.521502in}}%
\pgfpathlineto{\pgfqpoint{3.760312in}{1.531654in}}%
\pgfpathlineto{\pgfqpoint{3.746310in}{1.533459in}}%
\pgfpathlineto{\pgfqpoint{3.732315in}{1.535340in}}%
\pgfpathlineto{\pgfqpoint{3.718328in}{1.537295in}}%
\pgfpathlineto{\pgfqpoint{3.704347in}{1.539325in}}%
\pgfpathlineto{\pgfqpoint{3.696085in}{1.529434in}}%
\pgfpathlineto{\pgfqpoint{3.687816in}{1.519583in}}%
\pgfpathlineto{\pgfqpoint{3.679541in}{1.509777in}}%
\pgfpathlineto{\pgfqpoint{3.671260in}{1.500019in}}%
\pgfpathclose%
\pgfusepath{fill}%
\end{pgfscope}%
\begin{pgfscope}%
\pgfpathrectangle{\pgfqpoint{1.150000in}{0.150000in}}{\pgfqpoint{5.700000in}{5.700000in}}%
\pgfusepath{clip}%
\pgfsetbuttcap%
\pgfsetroundjoin%
\definecolor{currentfill}{rgb}{0.282656,0.100196,0.422160}%
\pgfsetfillcolor{currentfill}%
\pgfsetfillopacity{0.700000}%
\pgfsetlinewidth{0.000000pt}%
\definecolor{currentstroke}{rgb}{0.000000,0.000000,0.000000}%
\pgfsetstrokecolor{currentstroke}%
\pgfsetdash{}{0pt}%
\pgfpathmoveto{\pgfqpoint{2.652822in}{1.658472in}}%
\pgfpathlineto{\pgfqpoint{2.666743in}{1.649078in}}%
\pgfpathlineto{\pgfqpoint{2.680666in}{1.639778in}}%
\pgfpathlineto{\pgfqpoint{2.694589in}{1.630571in}}%
\pgfpathlineto{\pgfqpoint{2.708514in}{1.621457in}}%
\pgfpathlineto{\pgfqpoint{2.717378in}{1.622425in}}%
\pgfpathlineto{\pgfqpoint{2.726226in}{1.623657in}}%
\pgfpathlineto{\pgfqpoint{2.735058in}{1.625146in}}%
\pgfpathlineto{\pgfqpoint{2.743874in}{1.626885in}}%
\pgfpathlineto{\pgfqpoint{2.729984in}{1.635582in}}%
\pgfpathlineto{\pgfqpoint{2.716095in}{1.644371in}}%
\pgfpathlineto{\pgfqpoint{2.702207in}{1.653254in}}%
\pgfpathlineto{\pgfqpoint{2.688321in}{1.662230in}}%
\pgfpathlineto{\pgfqpoint{2.679471in}{1.660900in}}%
\pgfpathlineto{\pgfqpoint{2.670604in}{1.659826in}}%
\pgfpathlineto{\pgfqpoint{2.661721in}{1.659014in}}%
\pgfpathlineto{\pgfqpoint{2.652822in}{1.658472in}}%
\pgfpathclose%
\pgfusepath{fill}%
\end{pgfscope}%
\begin{pgfscope}%
\pgfpathrectangle{\pgfqpoint{1.150000in}{0.150000in}}{\pgfqpoint{5.700000in}{5.700000in}}%
\pgfusepath{clip}%
\pgfsetbuttcap%
\pgfsetroundjoin%
\definecolor{currentfill}{rgb}{0.267004,0.004874,0.329415}%
\pgfsetfillcolor{currentfill}%
\pgfsetfillopacity{0.700000}%
\pgfsetlinewidth{0.000000pt}%
\definecolor{currentstroke}{rgb}{0.000000,0.000000,0.000000}%
\pgfsetstrokecolor{currentstroke}%
\pgfsetdash{}{0pt}%
\pgfpathmoveto{\pgfqpoint{3.202261in}{1.462815in}}%
\pgfpathlineto{\pgfqpoint{3.216185in}{1.457380in}}%
\pgfpathlineto{\pgfqpoint{3.230114in}{1.452025in}}%
\pgfpathlineto{\pgfqpoint{3.244048in}{1.446751in}}%
\pgfpathlineto{\pgfqpoint{3.257986in}{1.441556in}}%
\pgfpathlineto{\pgfqpoint{3.266475in}{1.448145in}}%
\pgfpathlineto{\pgfqpoint{3.274955in}{1.454882in}}%
\pgfpathlineto{\pgfqpoint{3.283426in}{1.461763in}}%
\pgfpathlineto{\pgfqpoint{3.291888in}{1.468783in}}%
\pgfpathlineto{\pgfqpoint{3.277971in}{1.473627in}}%
\pgfpathlineto{\pgfqpoint{3.264058in}{1.478552in}}%
\pgfpathlineto{\pgfqpoint{3.250151in}{1.483557in}}%
\pgfpathlineto{\pgfqpoint{3.236248in}{1.488642in}}%
\pgfpathlineto{\pgfqpoint{3.227765in}{1.481964in}}%
\pgfpathlineto{\pgfqpoint{3.219274in}{1.475430in}}%
\pgfpathlineto{\pgfqpoint{3.210772in}{1.469045in}}%
\pgfpathlineto{\pgfqpoint{3.202261in}{1.462815in}}%
\pgfpathclose%
\pgfusepath{fill}%
\end{pgfscope}%
\begin{pgfscope}%
\pgfpathrectangle{\pgfqpoint{1.150000in}{0.150000in}}{\pgfqpoint{5.700000in}{5.700000in}}%
\pgfusepath{clip}%
\pgfsetbuttcap%
\pgfsetroundjoin%
\definecolor{currentfill}{rgb}{0.269944,0.014625,0.341379}%
\pgfsetfillcolor{currentfill}%
\pgfsetfillopacity{0.700000}%
\pgfsetlinewidth{0.000000pt}%
\definecolor{currentstroke}{rgb}{0.000000,0.000000,0.000000}%
\pgfsetstrokecolor{currentstroke}%
\pgfsetdash{}{0pt}%
\pgfpathmoveto{\pgfqpoint{3.056676in}{1.488928in}}%
\pgfpathlineto{\pgfqpoint{3.070592in}{1.482466in}}%
\pgfpathlineto{\pgfqpoint{3.084512in}{1.476088in}}%
\pgfpathlineto{\pgfqpoint{3.098436in}{1.469792in}}%
\pgfpathlineto{\pgfqpoint{3.112363in}{1.463579in}}%
\pgfpathlineto{\pgfqpoint{3.120940in}{1.468758in}}%
\pgfpathlineto{\pgfqpoint{3.129506in}{1.474120in}}%
\pgfpathlineto{\pgfqpoint{3.138062in}{1.479658in}}%
\pgfpathlineto{\pgfqpoint{3.146606in}{1.485366in}}%
\pgfpathlineto{\pgfqpoint{3.132703in}{1.491208in}}%
\pgfpathlineto{\pgfqpoint{3.118804in}{1.497133in}}%
\pgfpathlineto{\pgfqpoint{3.104909in}{1.503140in}}%
\pgfpathlineto{\pgfqpoint{3.091017in}{1.509231in}}%
\pgfpathlineto{\pgfqpoint{3.082449in}{1.503886in}}%
\pgfpathlineto{\pgfqpoint{3.073869in}{1.498717in}}%
\pgfpathlineto{\pgfqpoint{3.065278in}{1.493729in}}%
\pgfpathlineto{\pgfqpoint{3.056676in}{1.488928in}}%
\pgfpathclose%
\pgfusepath{fill}%
\end{pgfscope}%
\begin{pgfscope}%
\pgfpathrectangle{\pgfqpoint{1.150000in}{0.150000in}}{\pgfqpoint{5.700000in}{5.700000in}}%
\pgfusepath{clip}%
\pgfsetbuttcap%
\pgfsetroundjoin%
\definecolor{currentfill}{rgb}{0.269944,0.014625,0.341379}%
\pgfsetfillcolor{currentfill}%
\pgfsetfillopacity{0.700000}%
\pgfsetlinewidth{0.000000pt}%
\definecolor{currentstroke}{rgb}{0.000000,0.000000,0.000000}%
\pgfsetstrokecolor{currentstroke}%
\pgfsetdash{}{0pt}%
\pgfpathmoveto{\pgfqpoint{3.582111in}{1.472656in}}%
\pgfpathlineto{\pgfqpoint{3.596091in}{1.469768in}}%
\pgfpathlineto{\pgfqpoint{3.610077in}{1.466956in}}%
\pgfpathlineto{\pgfqpoint{3.624070in}{1.464220in}}%
\pgfpathlineto{\pgfqpoint{3.638070in}{1.461558in}}%
\pgfpathlineto{\pgfqpoint{3.646377in}{1.471080in}}%
\pgfpathlineto{\pgfqpoint{3.654678in}{1.480667in}}%
\pgfpathlineto{\pgfqpoint{3.662972in}{1.490315in}}%
\pgfpathlineto{\pgfqpoint{3.671260in}{1.500019in}}%
\pgfpathlineto{\pgfqpoint{3.657274in}{1.502392in}}%
\pgfpathlineto{\pgfqpoint{3.643295in}{1.504841in}}%
\pgfpathlineto{\pgfqpoint{3.629322in}{1.507365in}}%
\pgfpathlineto{\pgfqpoint{3.615357in}{1.509965in}}%
\pgfpathlineto{\pgfqpoint{3.607055in}{1.500540in}}%
\pgfpathlineto{\pgfqpoint{3.598747in}{1.491178in}}%
\pgfpathlineto{\pgfqpoint{3.590433in}{1.481882in}}%
\pgfpathlineto{\pgfqpoint{3.582111in}{1.472656in}}%
\pgfpathclose%
\pgfusepath{fill}%
\end{pgfscope}%
\begin{pgfscope}%
\pgfpathrectangle{\pgfqpoint{1.150000in}{0.150000in}}{\pgfqpoint{5.700000in}{5.700000in}}%
\pgfusepath{clip}%
\pgfsetbuttcap%
\pgfsetroundjoin%
\definecolor{currentfill}{rgb}{0.267004,0.004874,0.329415}%
\pgfsetfillcolor{currentfill}%
\pgfsetfillopacity{0.700000}%
\pgfsetlinewidth{0.000000pt}%
\definecolor{currentstroke}{rgb}{0.000000,0.000000,0.000000}%
\pgfsetstrokecolor{currentstroke}%
\pgfsetdash{}{0pt}%
\pgfpathmoveto{\pgfqpoint{3.347607in}{1.450197in}}%
\pgfpathlineto{\pgfqpoint{3.361549in}{1.445748in}}%
\pgfpathlineto{\pgfqpoint{3.375497in}{1.441377in}}%
\pgfpathlineto{\pgfqpoint{3.389450in}{1.437084in}}%
\pgfpathlineto{\pgfqpoint{3.403409in}{1.432869in}}%
\pgfpathlineto{\pgfqpoint{3.411823in}{1.440695in}}%
\pgfpathlineto{\pgfqpoint{3.420229in}{1.448639in}}%
\pgfpathlineto{\pgfqpoint{3.428627in}{1.456696in}}%
\pgfpathlineto{\pgfqpoint{3.437018in}{1.464861in}}%
\pgfpathlineto{\pgfqpoint{3.423077in}{1.468747in}}%
\pgfpathlineto{\pgfqpoint{3.409142in}{1.472711in}}%
\pgfpathlineto{\pgfqpoint{3.395212in}{1.476753in}}%
\pgfpathlineto{\pgfqpoint{3.381288in}{1.480873in}}%
\pgfpathlineto{\pgfqpoint{3.372880in}{1.473030in}}%
\pgfpathlineto{\pgfqpoint{3.364464in}{1.465299in}}%
\pgfpathlineto{\pgfqpoint{3.356040in}{1.457687in}}%
\pgfpathlineto{\pgfqpoint{3.347607in}{1.450197in}}%
\pgfpathclose%
\pgfusepath{fill}%
\end{pgfscope}%
\begin{pgfscope}%
\pgfpathrectangle{\pgfqpoint{1.150000in}{0.150000in}}{\pgfqpoint{5.700000in}{5.700000in}}%
\pgfusepath{clip}%
\pgfsetbuttcap%
\pgfsetroundjoin%
\definecolor{currentfill}{rgb}{0.282623,0.140926,0.457517}%
\pgfsetfillcolor{currentfill}%
\pgfsetfillopacity{0.700000}%
\pgfsetlinewidth{0.000000pt}%
\definecolor{currentstroke}{rgb}{0.000000,0.000000,0.000000}%
\pgfsetstrokecolor{currentstroke}%
\pgfsetdash{}{0pt}%
\pgfpathmoveto{\pgfqpoint{4.172461in}{1.692875in}}%
\pgfpathlineto{\pgfqpoint{4.186601in}{1.693477in}}%
\pgfpathlineto{\pgfqpoint{4.200750in}{1.694153in}}%
\pgfpathlineto{\pgfqpoint{4.214908in}{1.694900in}}%
\pgfpathlineto{\pgfqpoint{4.229076in}{1.695720in}}%
\pgfpathlineto{\pgfqpoint{4.237177in}{1.707016in}}%
\pgfpathlineto{\pgfqpoint{4.245273in}{1.718262in}}%
\pgfpathlineto{\pgfqpoint{4.253364in}{1.729456in}}%
\pgfpathlineto{\pgfqpoint{4.261449in}{1.740596in}}%
\pgfpathlineto{\pgfqpoint{4.247289in}{1.739611in}}%
\pgfpathlineto{\pgfqpoint{4.233138in}{1.738698in}}%
\pgfpathlineto{\pgfqpoint{4.218996in}{1.737857in}}%
\pgfpathlineto{\pgfqpoint{4.204864in}{1.737088in}}%
\pgfpathlineto{\pgfqpoint{4.196772in}{1.726106in}}%
\pgfpathlineto{\pgfqpoint{4.188674in}{1.715075in}}%
\pgfpathlineto{\pgfqpoint{4.180570in}{1.703997in}}%
\pgfpathlineto{\pgfqpoint{4.172461in}{1.692875in}}%
\pgfpathclose%
\pgfusepath{fill}%
\end{pgfscope}%
\begin{pgfscope}%
\pgfpathrectangle{\pgfqpoint{1.150000in}{0.150000in}}{\pgfqpoint{5.700000in}{5.700000in}}%
\pgfusepath{clip}%
\pgfsetbuttcap%
\pgfsetroundjoin%
\definecolor{currentfill}{rgb}{0.280868,0.160771,0.472899}%
\pgfsetfillcolor{currentfill}%
\pgfsetfillopacity{0.700000}%
\pgfsetlinewidth{0.000000pt}%
\definecolor{currentstroke}{rgb}{0.000000,0.000000,0.000000}%
\pgfsetstrokecolor{currentstroke}%
\pgfsetdash{}{0pt}%
\pgfpathmoveto{\pgfqpoint{4.261449in}{1.740596in}}%
\pgfpathlineto{\pgfqpoint{4.275619in}{1.741654in}}%
\pgfpathlineto{\pgfqpoint{4.289799in}{1.742784in}}%
\pgfpathlineto{\pgfqpoint{4.303989in}{1.743986in}}%
\pgfpathlineto{\pgfqpoint{4.318189in}{1.745259in}}%
\pgfpathlineto{\pgfqpoint{4.326261in}{1.756496in}}%
\pgfpathlineto{\pgfqpoint{4.334328in}{1.767670in}}%
\pgfpathlineto{\pgfqpoint{4.342389in}{1.778778in}}%
\pgfpathlineto{\pgfqpoint{4.350445in}{1.789819in}}%
\pgfpathlineto{\pgfqpoint{4.336252in}{1.788400in}}%
\pgfpathlineto{\pgfqpoint{4.322070in}{1.787053in}}%
\pgfpathlineto{\pgfqpoint{4.307897in}{1.785778in}}%
\pgfpathlineto{\pgfqpoint{4.293734in}{1.784576in}}%
\pgfpathlineto{\pgfqpoint{4.285671in}{1.773672in}}%
\pgfpathlineto{\pgfqpoint{4.277603in}{1.762706in}}%
\pgfpathlineto{\pgfqpoint{4.269529in}{1.751680in}}%
\pgfpathlineto{\pgfqpoint{4.261449in}{1.740596in}}%
\pgfpathclose%
\pgfusepath{fill}%
\end{pgfscope}%
\begin{pgfscope}%
\pgfpathrectangle{\pgfqpoint{1.150000in}{0.150000in}}{\pgfqpoint{5.700000in}{5.700000in}}%
\pgfusepath{clip}%
\pgfsetbuttcap%
\pgfsetroundjoin%
\definecolor{currentfill}{rgb}{0.283197,0.115680,0.436115}%
\pgfsetfillcolor{currentfill}%
\pgfsetfillopacity{0.700000}%
\pgfsetlinewidth{0.000000pt}%
\definecolor{currentstroke}{rgb}{0.000000,0.000000,0.000000}%
\pgfsetstrokecolor{currentstroke}%
\pgfsetdash{}{0pt}%
\pgfpathmoveto{\pgfqpoint{4.083474in}{1.647047in}}%
\pgfpathlineto{\pgfqpoint{4.097585in}{1.647173in}}%
\pgfpathlineto{\pgfqpoint{4.111705in}{1.647372in}}%
\pgfpathlineto{\pgfqpoint{4.125834in}{1.647644in}}%
\pgfpathlineto{\pgfqpoint{4.139972in}{1.647988in}}%
\pgfpathlineto{\pgfqpoint{4.148102in}{1.659264in}}%
\pgfpathlineto{\pgfqpoint{4.156227in}{1.670506in}}%
\pgfpathlineto{\pgfqpoint{4.164347in}{1.681710in}}%
\pgfpathlineto{\pgfqpoint{4.172461in}{1.692875in}}%
\pgfpathlineto{\pgfqpoint{4.158331in}{1.692344in}}%
\pgfpathlineto{\pgfqpoint{4.144210in}{1.691886in}}%
\pgfpathlineto{\pgfqpoint{4.130098in}{1.691501in}}%
\pgfpathlineto{\pgfqpoint{4.115995in}{1.691189in}}%
\pgfpathlineto{\pgfqpoint{4.107873in}{1.680202in}}%
\pgfpathlineto{\pgfqpoint{4.099745in}{1.669182in}}%
\pgfpathlineto{\pgfqpoint{4.091612in}{1.658129in}}%
\pgfpathlineto{\pgfqpoint{4.083474in}{1.647047in}}%
\pgfpathclose%
\pgfusepath{fill}%
\end{pgfscope}%
\begin{pgfscope}%
\pgfpathrectangle{\pgfqpoint{1.150000in}{0.150000in}}{\pgfqpoint{5.700000in}{5.700000in}}%
\pgfusepath{clip}%
\pgfsetbuttcap%
\pgfsetroundjoin%
\definecolor{currentfill}{rgb}{0.175841,0.441290,0.557685}%
\pgfsetfillcolor{currentfill}%
\pgfsetfillopacity{0.700000}%
\pgfsetlinewidth{0.000000pt}%
\definecolor{currentstroke}{rgb}{0.000000,0.000000,0.000000}%
\pgfsetstrokecolor{currentstroke}%
\pgfsetdash{}{0pt}%
\pgfpathmoveto{\pgfqpoint{5.507735in}{2.425067in}}%
\pgfpathlineto{\pgfqpoint{5.522431in}{2.430195in}}%
\pgfpathlineto{\pgfqpoint{5.537141in}{2.435394in}}%
\pgfpathlineto{\pgfqpoint{5.551864in}{2.440663in}}%
\pgfpathlineto{\pgfqpoint{5.559369in}{2.445780in}}%
\pgfpathlineto{\pgfqpoint{5.566865in}{2.450786in}}%
\pgfpathlineto{\pgfqpoint{5.574351in}{2.455684in}}%
\pgfpathlineto{\pgfqpoint{5.581828in}{2.460477in}}%
\pgfpathlineto{\pgfqpoint{5.567123in}{2.455367in}}%
\pgfpathlineto{\pgfqpoint{5.552431in}{2.450328in}}%
\pgfpathlineto{\pgfqpoint{5.537753in}{2.445359in}}%
\pgfpathlineto{\pgfqpoint{5.530263in}{2.440440in}}%
\pgfpathlineto{\pgfqpoint{5.522763in}{2.435420in}}%
\pgfpathlineto{\pgfqpoint{5.515254in}{2.430297in}}%
\pgfpathlineto{\pgfqpoint{5.507735in}{2.425067in}}%
\pgfpathclose%
\pgfusepath{fill}%
\end{pgfscope}%
\begin{pgfscope}%
\pgfpathrectangle{\pgfqpoint{1.150000in}{0.150000in}}{\pgfqpoint{5.700000in}{5.700000in}}%
\pgfusepath{clip}%
\pgfsetbuttcap%
\pgfsetroundjoin%
\definecolor{currentfill}{rgb}{0.277134,0.185228,0.489898}%
\pgfsetfillcolor{currentfill}%
\pgfsetfillopacity{0.700000}%
\pgfsetlinewidth{0.000000pt}%
\definecolor{currentstroke}{rgb}{0.000000,0.000000,0.000000}%
\pgfsetstrokecolor{currentstroke}%
\pgfsetdash{}{0pt}%
\pgfpathmoveto{\pgfqpoint{4.350445in}{1.789819in}}%
\pgfpathlineto{\pgfqpoint{4.364648in}{1.791310in}}%
\pgfpathlineto{\pgfqpoint{4.378860in}{1.792872in}}%
\pgfpathlineto{\pgfqpoint{4.393083in}{1.794507in}}%
\pgfpathlineto{\pgfqpoint{4.407316in}{1.796214in}}%
\pgfpathlineto{\pgfqpoint{4.415359in}{1.807318in}}%
\pgfpathlineto{\pgfqpoint{4.423397in}{1.818346in}}%
\pgfpathlineto{\pgfqpoint{4.431428in}{1.829297in}}%
\pgfpathlineto{\pgfqpoint{4.439454in}{1.840170in}}%
\pgfpathlineto{\pgfqpoint{4.425228in}{1.838339in}}%
\pgfpathlineto{\pgfqpoint{4.411012in}{1.836580in}}%
\pgfpathlineto{\pgfqpoint{4.396806in}{1.834893in}}%
\pgfpathlineto{\pgfqpoint{4.382611in}{1.833279in}}%
\pgfpathlineto{\pgfqpoint{4.374578in}{1.822522in}}%
\pgfpathlineto{\pgfqpoint{4.366539in}{1.811693in}}%
\pgfpathlineto{\pgfqpoint{4.358495in}{1.800791in}}%
\pgfpathlineto{\pgfqpoint{4.350445in}{1.789819in}}%
\pgfpathclose%
\pgfusepath{fill}%
\end{pgfscope}%
\begin{pgfscope}%
\pgfpathrectangle{\pgfqpoint{1.150000in}{0.150000in}}{\pgfqpoint{5.700000in}{5.700000in}}%
\pgfusepath{clip}%
\pgfsetbuttcap%
\pgfsetroundjoin%
\definecolor{currentfill}{rgb}{0.282327,0.094955,0.417331}%
\pgfsetfillcolor{currentfill}%
\pgfsetfillopacity{0.700000}%
\pgfsetlinewidth{0.000000pt}%
\definecolor{currentstroke}{rgb}{0.000000,0.000000,0.000000}%
\pgfsetstrokecolor{currentstroke}%
\pgfsetdash{}{0pt}%
\pgfpathmoveto{\pgfqpoint{3.994476in}{1.603528in}}%
\pgfpathlineto{\pgfqpoint{4.008561in}{1.603156in}}%
\pgfpathlineto{\pgfqpoint{4.022654in}{1.602857in}}%
\pgfpathlineto{\pgfqpoint{4.036756in}{1.602630in}}%
\pgfpathlineto{\pgfqpoint{4.050866in}{1.602477in}}%
\pgfpathlineto{\pgfqpoint{4.059026in}{1.613650in}}%
\pgfpathlineto{\pgfqpoint{4.067181in}{1.624805in}}%
\pgfpathlineto{\pgfqpoint{4.075330in}{1.635938in}}%
\pgfpathlineto{\pgfqpoint{4.083474in}{1.647047in}}%
\pgfpathlineto{\pgfqpoint{4.069372in}{1.646993in}}%
\pgfpathlineto{\pgfqpoint{4.055278in}{1.647013in}}%
\pgfpathlineto{\pgfqpoint{4.041194in}{1.647105in}}%
\pgfpathlineto{\pgfqpoint{4.027118in}{1.647270in}}%
\pgfpathlineto{\pgfqpoint{4.018966in}{1.636360in}}%
\pgfpathlineto{\pgfqpoint{4.010808in}{1.625431in}}%
\pgfpathlineto{\pgfqpoint{4.002645in}{1.614486in}}%
\pgfpathlineto{\pgfqpoint{3.994476in}{1.603528in}}%
\pgfpathclose%
\pgfusepath{fill}%
\end{pgfscope}%
\begin{pgfscope}%
\pgfpathrectangle{\pgfqpoint{1.150000in}{0.150000in}}{\pgfqpoint{5.700000in}{5.700000in}}%
\pgfusepath{clip}%
\pgfsetbuttcap%
\pgfsetroundjoin%
\definecolor{currentfill}{rgb}{0.274952,0.037752,0.364543}%
\pgfsetfillcolor{currentfill}%
\pgfsetfillopacity{0.700000}%
\pgfsetlinewidth{0.000000pt}%
\definecolor{currentstroke}{rgb}{0.000000,0.000000,0.000000}%
\pgfsetstrokecolor{currentstroke}%
\pgfsetdash{}{0pt}%
\pgfpathmoveto{\pgfqpoint{2.910715in}{1.529589in}}%
\pgfpathlineto{\pgfqpoint{2.924634in}{1.522055in}}%
\pgfpathlineto{\pgfqpoint{2.938555in}{1.514608in}}%
\pgfpathlineto{\pgfqpoint{2.952480in}{1.507246in}}%
\pgfpathlineto{\pgfqpoint{2.966407in}{1.499970in}}%
\pgfpathlineto{\pgfqpoint{2.975086in}{1.503563in}}%
\pgfpathlineto{\pgfqpoint{2.983752in}{1.507371in}}%
\pgfpathlineto{\pgfqpoint{2.992405in}{1.511390in}}%
\pgfpathlineto{\pgfqpoint{3.001046in}{1.515614in}}%
\pgfpathlineto{\pgfqpoint{2.987146in}{1.522497in}}%
\pgfpathlineto{\pgfqpoint{2.973250in}{1.529465in}}%
\pgfpathlineto{\pgfqpoint{2.959357in}{1.536519in}}%
\pgfpathlineto{\pgfqpoint{2.945467in}{1.543660in}}%
\pgfpathlineto{\pgfqpoint{2.936799in}{1.539822in}}%
\pgfpathlineto{\pgfqpoint{2.928117in}{1.536193in}}%
\pgfpathlineto{\pgfqpoint{2.919423in}{1.532780in}}%
\pgfpathlineto{\pgfqpoint{2.910715in}{1.529589in}}%
\pgfpathclose%
\pgfusepath{fill}%
\end{pgfscope}%
\begin{pgfscope}%
\pgfpathrectangle{\pgfqpoint{1.150000in}{0.150000in}}{\pgfqpoint{5.700000in}{5.700000in}}%
\pgfusepath{clip}%
\pgfsetbuttcap%
\pgfsetroundjoin%
\definecolor{currentfill}{rgb}{0.271828,0.209303,0.504434}%
\pgfsetfillcolor{currentfill}%
\pgfsetfillopacity{0.700000}%
\pgfsetlinewidth{0.000000pt}%
\definecolor{currentstroke}{rgb}{0.000000,0.000000,0.000000}%
\pgfsetstrokecolor{currentstroke}%
\pgfsetdash{}{0pt}%
\pgfpathmoveto{\pgfqpoint{4.439454in}{1.840170in}}%
\pgfpathlineto{\pgfqpoint{4.453690in}{1.842072in}}%
\pgfpathlineto{\pgfqpoint{4.467937in}{1.844046in}}%
\pgfpathlineto{\pgfqpoint{4.482195in}{1.846092in}}%
\pgfpathlineto{\pgfqpoint{4.496463in}{1.848210in}}%
\pgfpathlineto{\pgfqpoint{4.504476in}{1.859113in}}%
\pgfpathlineto{\pgfqpoint{4.512483in}{1.869929in}}%
\pgfpathlineto{\pgfqpoint{4.520484in}{1.880658in}}%
\pgfpathlineto{\pgfqpoint{4.528479in}{1.891297in}}%
\pgfpathlineto{\pgfqpoint{4.514217in}{1.889076in}}%
\pgfpathlineto{\pgfqpoint{4.499967in}{1.886927in}}%
\pgfpathlineto{\pgfqpoint{4.485727in}{1.884850in}}%
\pgfpathlineto{\pgfqpoint{4.471498in}{1.882845in}}%
\pgfpathlineto{\pgfqpoint{4.463496in}{1.872301in}}%
\pgfpathlineto{\pgfqpoint{4.455488in}{1.861673in}}%
\pgfpathlineto{\pgfqpoint{4.447474in}{1.850962in}}%
\pgfpathlineto{\pgfqpoint{4.439454in}{1.840170in}}%
\pgfpathclose%
\pgfusepath{fill}%
\end{pgfscope}%
\begin{pgfscope}%
\pgfpathrectangle{\pgfqpoint{1.150000in}{0.150000in}}{\pgfqpoint{5.700000in}{5.700000in}}%
\pgfusepath{clip}%
\pgfsetbuttcap%
\pgfsetroundjoin%
\definecolor{currentfill}{rgb}{0.281446,0.084320,0.407414}%
\pgfsetfillcolor{currentfill}%
\pgfsetfillopacity{0.700000}%
\pgfsetlinewidth{0.000000pt}%
\definecolor{currentstroke}{rgb}{0.000000,0.000000,0.000000}%
\pgfsetstrokecolor{currentstroke}%
\pgfsetdash{}{0pt}%
\pgfpathmoveto{\pgfqpoint{2.708514in}{1.621457in}}%
\pgfpathlineto{\pgfqpoint{2.722440in}{1.612436in}}%
\pgfpathlineto{\pgfqpoint{2.736368in}{1.603507in}}%
\pgfpathlineto{\pgfqpoint{2.750297in}{1.594669in}}%
\pgfpathlineto{\pgfqpoint{2.764228in}{1.585922in}}%
\pgfpathlineto{\pgfqpoint{2.773058in}{1.587315in}}%
\pgfpathlineto{\pgfqpoint{2.781872in}{1.588967in}}%
\pgfpathlineto{\pgfqpoint{2.790671in}{1.590870in}}%
\pgfpathlineto{\pgfqpoint{2.799454in}{1.593018in}}%
\pgfpathlineto{\pgfqpoint{2.785557in}{1.601348in}}%
\pgfpathlineto{\pgfqpoint{2.771661in}{1.609769in}}%
\pgfpathlineto{\pgfqpoint{2.757767in}{1.618281in}}%
\pgfpathlineto{\pgfqpoint{2.743874in}{1.626885in}}%
\pgfpathlineto{\pgfqpoint{2.735058in}{1.625146in}}%
\pgfpathlineto{\pgfqpoint{2.726226in}{1.623657in}}%
\pgfpathlineto{\pgfqpoint{2.717378in}{1.622425in}}%
\pgfpathlineto{\pgfqpoint{2.708514in}{1.621457in}}%
\pgfpathclose%
\pgfusepath{fill}%
\end{pgfscope}%
\begin{pgfscope}%
\pgfpathrectangle{\pgfqpoint{1.150000in}{0.150000in}}{\pgfqpoint{5.700000in}{5.700000in}}%
\pgfusepath{clip}%
\pgfsetbuttcap%
\pgfsetroundjoin%
\definecolor{currentfill}{rgb}{0.280267,0.073417,0.397163}%
\pgfsetfillcolor{currentfill}%
\pgfsetfillopacity{0.700000}%
\pgfsetlinewidth{0.000000pt}%
\definecolor{currentstroke}{rgb}{0.000000,0.000000,0.000000}%
\pgfsetstrokecolor{currentstroke}%
\pgfsetdash{}{0pt}%
\pgfpathmoveto{\pgfqpoint{3.905455in}{1.562751in}}%
\pgfpathlineto{\pgfqpoint{3.919516in}{1.561859in}}%
\pgfpathlineto{\pgfqpoint{3.933585in}{1.561040in}}%
\pgfpathlineto{\pgfqpoint{3.947662in}{1.560295in}}%
\pgfpathlineto{\pgfqpoint{3.961747in}{1.559622in}}%
\pgfpathlineto{\pgfqpoint{3.969938in}{1.570603in}}%
\pgfpathlineto{\pgfqpoint{3.978123in}{1.581583in}}%
\pgfpathlineto{\pgfqpoint{3.986302in}{1.592559in}}%
\pgfpathlineto{\pgfqpoint{3.994476in}{1.603528in}}%
\pgfpathlineto{\pgfqpoint{3.980400in}{1.603973in}}%
\pgfpathlineto{\pgfqpoint{3.966332in}{1.604491in}}%
\pgfpathlineto{\pgfqpoint{3.952273in}{1.605082in}}%
\pgfpathlineto{\pgfqpoint{3.938222in}{1.605747in}}%
\pgfpathlineto{\pgfqpoint{3.930038in}{1.594998in}}%
\pgfpathlineto{\pgfqpoint{3.921850in}{1.584247in}}%
\pgfpathlineto{\pgfqpoint{3.913655in}{1.573497in}}%
\pgfpathlineto{\pgfqpoint{3.905455in}{1.562751in}}%
\pgfpathclose%
\pgfusepath{fill}%
\end{pgfscope}%
\begin{pgfscope}%
\pgfpathrectangle{\pgfqpoint{1.150000in}{0.150000in}}{\pgfqpoint{5.700000in}{5.700000in}}%
\pgfusepath{clip}%
\pgfsetbuttcap%
\pgfsetroundjoin%
\definecolor{currentfill}{rgb}{0.265145,0.232956,0.516599}%
\pgfsetfillcolor{currentfill}%
\pgfsetfillopacity{0.700000}%
\pgfsetlinewidth{0.000000pt}%
\definecolor{currentstroke}{rgb}{0.000000,0.000000,0.000000}%
\pgfsetstrokecolor{currentstroke}%
\pgfsetdash{}{0pt}%
\pgfpathmoveto{\pgfqpoint{4.528479in}{1.891297in}}%
\pgfpathlineto{\pgfqpoint{4.542750in}{1.893589in}}%
\pgfpathlineto{\pgfqpoint{4.557033in}{1.895953in}}%
\pgfpathlineto{\pgfqpoint{4.571327in}{1.898389in}}%
\pgfpathlineto{\pgfqpoint{4.585631in}{1.900896in}}%
\pgfpathlineto{\pgfqpoint{4.593613in}{1.911535in}}%
\pgfpathlineto{\pgfqpoint{4.601588in}{1.922077in}}%
\pgfpathlineto{\pgfqpoint{4.609557in}{1.932522in}}%
\pgfpathlineto{\pgfqpoint{4.617520in}{1.942869in}}%
\pgfpathlineto{\pgfqpoint{4.603222in}{1.940280in}}%
\pgfpathlineto{\pgfqpoint{4.588936in}{1.937763in}}%
\pgfpathlineto{\pgfqpoint{4.574661in}{1.935317in}}%
\pgfpathlineto{\pgfqpoint{4.560396in}{1.932943in}}%
\pgfpathlineto{\pgfqpoint{4.552426in}{1.922670in}}%
\pgfpathlineto{\pgfqpoint{4.544450in}{1.912304in}}%
\pgfpathlineto{\pgfqpoint{4.536467in}{1.901846in}}%
\pgfpathlineto{\pgfqpoint{4.528479in}{1.891297in}}%
\pgfpathclose%
\pgfusepath{fill}%
\end{pgfscope}%
\begin{pgfscope}%
\pgfpathrectangle{\pgfqpoint{1.150000in}{0.150000in}}{\pgfqpoint{5.700000in}{5.700000in}}%
\pgfusepath{clip}%
\pgfsetbuttcap%
\pgfsetroundjoin%
\definecolor{currentfill}{rgb}{0.268510,0.009605,0.335427}%
\pgfsetfillcolor{currentfill}%
\pgfsetfillopacity{0.700000}%
\pgfsetlinewidth{0.000000pt}%
\definecolor{currentstroke}{rgb}{0.000000,0.000000,0.000000}%
\pgfsetstrokecolor{currentstroke}%
\pgfsetdash{}{0pt}%
\pgfpathmoveto{\pgfqpoint{3.492838in}{1.450089in}}%
\pgfpathlineto{\pgfqpoint{3.506809in}{1.446588in}}%
\pgfpathlineto{\pgfqpoint{3.520785in}{1.443164in}}%
\pgfpathlineto{\pgfqpoint{3.534767in}{1.439816in}}%
\pgfpathlineto{\pgfqpoint{3.548755in}{1.436543in}}%
\pgfpathlineto{\pgfqpoint{3.557105in}{1.445444in}}%
\pgfpathlineto{\pgfqpoint{3.565447in}{1.454432in}}%
\pgfpathlineto{\pgfqpoint{3.573783in}{1.463504in}}%
\pgfpathlineto{\pgfqpoint{3.582111in}{1.472656in}}%
\pgfpathlineto{\pgfqpoint{3.568138in}{1.475619in}}%
\pgfpathlineto{\pgfqpoint{3.554171in}{1.478659in}}%
\pgfpathlineto{\pgfqpoint{3.540211in}{1.481775in}}%
\pgfpathlineto{\pgfqpoint{3.526256in}{1.484967in}}%
\pgfpathlineto{\pgfqpoint{3.517913in}{1.476117in}}%
\pgfpathlineto{\pgfqpoint{3.509562in}{1.467350in}}%
\pgfpathlineto{\pgfqpoint{3.501204in}{1.458673in}}%
\pgfpathlineto{\pgfqpoint{3.492838in}{1.450089in}}%
\pgfpathclose%
\pgfusepath{fill}%
\end{pgfscope}%
\begin{pgfscope}%
\pgfpathrectangle{\pgfqpoint{1.150000in}{0.150000in}}{\pgfqpoint{5.700000in}{5.700000in}}%
\pgfusepath{clip}%
\pgfsetbuttcap%
\pgfsetroundjoin%
\definecolor{currentfill}{rgb}{0.257322,0.256130,0.526563}%
\pgfsetfillcolor{currentfill}%
\pgfsetfillopacity{0.700000}%
\pgfsetlinewidth{0.000000pt}%
\definecolor{currentstroke}{rgb}{0.000000,0.000000,0.000000}%
\pgfsetstrokecolor{currentstroke}%
\pgfsetdash{}{0pt}%
\pgfpathmoveto{\pgfqpoint{4.617520in}{1.942869in}}%
\pgfpathlineto{\pgfqpoint{4.631828in}{1.945530in}}%
\pgfpathlineto{\pgfqpoint{4.646147in}{1.948262in}}%
\pgfpathlineto{\pgfqpoint{4.660478in}{1.951065in}}%
\pgfpathlineto{\pgfqpoint{4.674820in}{1.953940in}}%
\pgfpathlineto{\pgfqpoint{4.682769in}{1.964257in}}%
\pgfpathlineto{\pgfqpoint{4.690711in}{1.974468in}}%
\pgfpathlineto{\pgfqpoint{4.698647in}{1.984575in}}%
\pgfpathlineto{\pgfqpoint{4.706575in}{1.994576in}}%
\pgfpathlineto{\pgfqpoint{4.692241in}{1.991640in}}%
\pgfpathlineto{\pgfqpoint{4.677918in}{1.988776in}}%
\pgfpathlineto{\pgfqpoint{4.663606in}{1.985984in}}%
\pgfpathlineto{\pgfqpoint{4.649305in}{1.983263in}}%
\pgfpathlineto{\pgfqpoint{4.641369in}{1.973315in}}%
\pgfpathlineto{\pgfqpoint{4.633425in}{1.963266in}}%
\pgfpathlineto{\pgfqpoint{4.625476in}{1.953117in}}%
\pgfpathlineto{\pgfqpoint{4.617520in}{1.942869in}}%
\pgfpathclose%
\pgfusepath{fill}%
\end{pgfscope}%
\begin{pgfscope}%
\pgfpathrectangle{\pgfqpoint{1.150000in}{0.150000in}}{\pgfqpoint{5.700000in}{5.700000in}}%
\pgfusepath{clip}%
\pgfsetbuttcap%
\pgfsetroundjoin%
\definecolor{currentfill}{rgb}{0.248629,0.278775,0.534556}%
\pgfsetfillcolor{currentfill}%
\pgfsetfillopacity{0.700000}%
\pgfsetlinewidth{0.000000pt}%
\definecolor{currentstroke}{rgb}{0.000000,0.000000,0.000000}%
\pgfsetstrokecolor{currentstroke}%
\pgfsetdash{}{0pt}%
\pgfpathmoveto{\pgfqpoint{4.706575in}{1.994576in}}%
\pgfpathlineto{\pgfqpoint{4.720921in}{1.997583in}}%
\pgfpathlineto{\pgfqpoint{4.735278in}{2.000661in}}%
\pgfpathlineto{\pgfqpoint{4.749647in}{2.003810in}}%
\pgfpathlineto{\pgfqpoint{4.764028in}{2.007031in}}%
\pgfpathlineto{\pgfqpoint{4.771942in}{2.016973in}}%
\pgfpathlineto{\pgfqpoint{4.779849in}{2.026803in}}%
\pgfpathlineto{\pgfqpoint{4.787749in}{2.036521in}}%
\pgfpathlineto{\pgfqpoint{4.795642in}{2.046126in}}%
\pgfpathlineto{\pgfqpoint{4.781270in}{2.042867in}}%
\pgfpathlineto{\pgfqpoint{4.766909in}{2.039678in}}%
\pgfpathlineto{\pgfqpoint{4.752560in}{2.036561in}}%
\pgfpathlineto{\pgfqpoint{4.738223in}{2.033515in}}%
\pgfpathlineto{\pgfqpoint{4.730321in}{2.023940in}}%
\pgfpathlineto{\pgfqpoint{4.722413in}{2.014259in}}%
\pgfpathlineto{\pgfqpoint{4.714497in}{2.004470in}}%
\pgfpathlineto{\pgfqpoint{4.706575in}{1.994576in}}%
\pgfpathclose%
\pgfusepath{fill}%
\end{pgfscope}%
\begin{pgfscope}%
\pgfpathrectangle{\pgfqpoint{1.150000in}{0.150000in}}{\pgfqpoint{5.700000in}{5.700000in}}%
\pgfusepath{clip}%
\pgfsetbuttcap%
\pgfsetroundjoin%
\definecolor{currentfill}{rgb}{0.277941,0.056324,0.381191}%
\pgfsetfillcolor{currentfill}%
\pgfsetfillopacity{0.700000}%
\pgfsetlinewidth{0.000000pt}%
\definecolor{currentstroke}{rgb}{0.000000,0.000000,0.000000}%
\pgfsetstrokecolor{currentstroke}%
\pgfsetdash{}{0pt}%
\pgfpathmoveto{\pgfqpoint{3.816394in}{1.525175in}}%
\pgfpathlineto{\pgfqpoint{3.830434in}{1.523740in}}%
\pgfpathlineto{\pgfqpoint{3.844481in}{1.522380in}}%
\pgfpathlineto{\pgfqpoint{3.858536in}{1.521093in}}%
\pgfpathlineto{\pgfqpoint{3.872599in}{1.519880in}}%
\pgfpathlineto{\pgfqpoint{3.880821in}{1.530574in}}%
\pgfpathlineto{\pgfqpoint{3.889038in}{1.541286in}}%
\pgfpathlineto{\pgfqpoint{3.897249in}{1.552013in}}%
\pgfpathlineto{\pgfqpoint{3.905455in}{1.562751in}}%
\pgfpathlineto{\pgfqpoint{3.891402in}{1.563717in}}%
\pgfpathlineto{\pgfqpoint{3.877358in}{1.564756in}}%
\pgfpathlineto{\pgfqpoint{3.863321in}{1.565869in}}%
\pgfpathlineto{\pgfqpoint{3.849292in}{1.567056in}}%
\pgfpathlineto{\pgfqpoint{3.841076in}{1.556558in}}%
\pgfpathlineto{\pgfqpoint{3.832855in}{1.546076in}}%
\pgfpathlineto{\pgfqpoint{3.824627in}{1.535614in}}%
\pgfpathlineto{\pgfqpoint{3.816394in}{1.525175in}}%
\pgfpathclose%
\pgfusepath{fill}%
\end{pgfscope}%
\begin{pgfscope}%
\pgfpathrectangle{\pgfqpoint{1.150000in}{0.150000in}}{\pgfqpoint{5.700000in}{5.700000in}}%
\pgfusepath{clip}%
\pgfsetbuttcap%
\pgfsetroundjoin%
\definecolor{currentfill}{rgb}{0.180629,0.429975,0.557282}%
\pgfsetfillcolor{currentfill}%
\pgfsetfillopacity{0.700000}%
\pgfsetlinewidth{0.000000pt}%
\definecolor{currentstroke}{rgb}{0.000000,0.000000,0.000000}%
\pgfsetstrokecolor{currentstroke}%
\pgfsetdash{}{0pt}%
\pgfpathmoveto{\pgfqpoint{5.418854in}{2.382669in}}%
\pgfpathlineto{\pgfqpoint{5.433512in}{2.387651in}}%
\pgfpathlineto{\pgfqpoint{5.448184in}{2.392703in}}%
\pgfpathlineto{\pgfqpoint{5.462869in}{2.397827in}}%
\pgfpathlineto{\pgfqpoint{5.477567in}{2.403022in}}%
\pgfpathlineto{\pgfqpoint{5.485124in}{2.408707in}}%
\pgfpathlineto{\pgfqpoint{5.492670in}{2.414275in}}%
\pgfpathlineto{\pgfqpoint{5.500208in}{2.419727in}}%
\pgfpathlineto{\pgfqpoint{5.507735in}{2.425067in}}%
\pgfpathlineto{\pgfqpoint{5.493053in}{2.420009in}}%
\pgfpathlineto{\pgfqpoint{5.478385in}{2.415023in}}%
\pgfpathlineto{\pgfqpoint{5.463730in}{2.410107in}}%
\pgfpathlineto{\pgfqpoint{5.449089in}{2.405262in}}%
\pgfpathlineto{\pgfqpoint{5.441544in}{2.399777in}}%
\pgfpathlineto{\pgfqpoint{5.433990in}{2.394185in}}%
\pgfpathlineto{\pgfqpoint{5.426427in}{2.388483in}}%
\pgfpathlineto{\pgfqpoint{5.418854in}{2.382669in}}%
\pgfpathclose%
\pgfusepath{fill}%
\end{pgfscope}%
\begin{pgfscope}%
\pgfpathrectangle{\pgfqpoint{1.150000in}{0.150000in}}{\pgfqpoint{5.700000in}{5.700000in}}%
\pgfusepath{clip}%
\pgfsetbuttcap%
\pgfsetroundjoin%
\definecolor{currentfill}{rgb}{0.239346,0.300855,0.540844}%
\pgfsetfillcolor{currentfill}%
\pgfsetfillopacity{0.700000}%
\pgfsetlinewidth{0.000000pt}%
\definecolor{currentstroke}{rgb}{0.000000,0.000000,0.000000}%
\pgfsetstrokecolor{currentstroke}%
\pgfsetdash{}{0pt}%
\pgfpathmoveto{\pgfqpoint{4.795642in}{2.046126in}}%
\pgfpathlineto{\pgfqpoint{4.810026in}{2.049458in}}%
\pgfpathlineto{\pgfqpoint{4.824422in}{2.052860in}}%
\pgfpathlineto{\pgfqpoint{4.838830in}{2.056334in}}%
\pgfpathlineto{\pgfqpoint{4.853250in}{2.059879in}}%
\pgfpathlineto{\pgfqpoint{4.861127in}{2.069399in}}%
\pgfpathlineto{\pgfqpoint{4.868997in}{2.078800in}}%
\pgfpathlineto{\pgfqpoint{4.876860in}{2.088085in}}%
\pgfpathlineto{\pgfqpoint{4.884716in}{2.097252in}}%
\pgfpathlineto{\pgfqpoint{4.870305in}{2.093690in}}%
\pgfpathlineto{\pgfqpoint{4.855906in}{2.090198in}}%
\pgfpathlineto{\pgfqpoint{4.841519in}{2.086778in}}%
\pgfpathlineto{\pgfqpoint{4.827144in}{2.083429in}}%
\pgfpathlineto{\pgfqpoint{4.819279in}{2.074271in}}%
\pgfpathlineto{\pgfqpoint{4.811407in}{2.065002in}}%
\pgfpathlineto{\pgfqpoint{4.803528in}{2.055620in}}%
\pgfpathlineto{\pgfqpoint{4.795642in}{2.046126in}}%
\pgfpathclose%
\pgfusepath{fill}%
\end{pgfscope}%
\begin{pgfscope}%
\pgfpathrectangle{\pgfqpoint{1.150000in}{0.150000in}}{\pgfqpoint{5.700000in}{5.700000in}}%
\pgfusepath{clip}%
\pgfsetbuttcap%
\pgfsetroundjoin%
\definecolor{currentfill}{rgb}{0.187231,0.414746,0.556547}%
\pgfsetfillcolor{currentfill}%
\pgfsetfillopacity{0.700000}%
\pgfsetlinewidth{0.000000pt}%
\definecolor{currentstroke}{rgb}{0.000000,0.000000,0.000000}%
\pgfsetstrokecolor{currentstroke}%
\pgfsetdash{}{0pt}%
\pgfpathmoveto{\pgfqpoint{5.329915in}{2.338559in}}%
\pgfpathlineto{\pgfqpoint{5.344534in}{2.343373in}}%
\pgfpathlineto{\pgfqpoint{5.359167in}{2.348257in}}%
\pgfpathlineto{\pgfqpoint{5.373813in}{2.353212in}}%
\pgfpathlineto{\pgfqpoint{5.388472in}{2.358239in}}%
\pgfpathlineto{\pgfqpoint{5.396082in}{2.364527in}}%
\pgfpathlineto{\pgfqpoint{5.403682in}{2.370693in}}%
\pgfpathlineto{\pgfqpoint{5.411273in}{2.376739in}}%
\pgfpathlineto{\pgfqpoint{5.418854in}{2.382669in}}%
\pgfpathlineto{\pgfqpoint{5.404210in}{2.377757in}}%
\pgfpathlineto{\pgfqpoint{5.389579in}{2.372917in}}%
\pgfpathlineto{\pgfqpoint{5.374962in}{2.368147in}}%
\pgfpathlineto{\pgfqpoint{5.360358in}{2.363449in}}%
\pgfpathlineto{\pgfqpoint{5.352760in}{2.357397in}}%
\pgfpathlineto{\pgfqpoint{5.345154in}{2.351232in}}%
\pgfpathlineto{\pgfqpoint{5.337539in}{2.344954in}}%
\pgfpathlineto{\pgfqpoint{5.329915in}{2.338559in}}%
\pgfpathclose%
\pgfusepath{fill}%
\end{pgfscope}%
\begin{pgfscope}%
\pgfpathrectangle{\pgfqpoint{1.150000in}{0.150000in}}{\pgfqpoint{5.700000in}{5.700000in}}%
\pgfusepath{clip}%
\pgfsetbuttcap%
\pgfsetroundjoin%
\definecolor{currentfill}{rgb}{0.229739,0.322361,0.545706}%
\pgfsetfillcolor{currentfill}%
\pgfsetfillopacity{0.700000}%
\pgfsetlinewidth{0.000000pt}%
\definecolor{currentstroke}{rgb}{0.000000,0.000000,0.000000}%
\pgfsetstrokecolor{currentstroke}%
\pgfsetdash{}{0pt}%
\pgfpathmoveto{\pgfqpoint{4.884716in}{2.097252in}}%
\pgfpathlineto{\pgfqpoint{4.899139in}{2.100886in}}%
\pgfpathlineto{\pgfqpoint{4.913574in}{2.104591in}}%
\pgfpathlineto{\pgfqpoint{4.928021in}{2.108367in}}%
\pgfpathlineto{\pgfqpoint{4.942480in}{2.112215in}}%
\pgfpathlineto{\pgfqpoint{4.950319in}{2.121269in}}%
\pgfpathlineto{\pgfqpoint{4.958150in}{2.130202in}}%
\pgfpathlineto{\pgfqpoint{4.965974in}{2.139013in}}%
\pgfpathlineto{\pgfqpoint{4.973789in}{2.147704in}}%
\pgfpathlineto{\pgfqpoint{4.959340in}{2.143861in}}%
\pgfpathlineto{\pgfqpoint{4.944902in}{2.140089in}}%
\pgfpathlineto{\pgfqpoint{4.930477in}{2.136389in}}%
\pgfpathlineto{\pgfqpoint{4.916064in}{2.132759in}}%
\pgfpathlineto{\pgfqpoint{4.908238in}{2.124056in}}%
\pgfpathlineto{\pgfqpoint{4.900405in}{2.115237in}}%
\pgfpathlineto{\pgfqpoint{4.892564in}{2.106303in}}%
\pgfpathlineto{\pgfqpoint{4.884716in}{2.097252in}}%
\pgfpathclose%
\pgfusepath{fill}%
\end{pgfscope}%
\begin{pgfscope}%
\pgfpathrectangle{\pgfqpoint{1.150000in}{0.150000in}}{\pgfqpoint{5.700000in}{5.700000in}}%
\pgfusepath{clip}%
\pgfsetbuttcap%
\pgfsetroundjoin%
\definecolor{currentfill}{rgb}{0.220057,0.343307,0.549413}%
\pgfsetfillcolor{currentfill}%
\pgfsetfillopacity{0.700000}%
\pgfsetlinewidth{0.000000pt}%
\definecolor{currentstroke}{rgb}{0.000000,0.000000,0.000000}%
\pgfsetstrokecolor{currentstroke}%
\pgfsetdash{}{0pt}%
\pgfpathmoveto{\pgfqpoint{4.973789in}{2.147704in}}%
\pgfpathlineto{\pgfqpoint{4.988252in}{2.151619in}}%
\pgfpathlineto{\pgfqpoint{5.002726in}{2.155604in}}%
\pgfpathlineto{\pgfqpoint{5.017213in}{2.159661in}}%
\pgfpathlineto{\pgfqpoint{5.031712in}{2.163788in}}%
\pgfpathlineto{\pgfqpoint{5.039510in}{2.172341in}}%
\pgfpathlineto{\pgfqpoint{5.047300in}{2.180770in}}%
\pgfpathlineto{\pgfqpoint{5.055082in}{2.189074in}}%
\pgfpathlineto{\pgfqpoint{5.062855in}{2.197256in}}%
\pgfpathlineto{\pgfqpoint{5.048366in}{2.193155in}}%
\pgfpathlineto{\pgfqpoint{5.033890in}{2.189124in}}%
\pgfpathlineto{\pgfqpoint{5.019426in}{2.185165in}}%
\pgfpathlineto{\pgfqpoint{5.004975in}{2.181277in}}%
\pgfpathlineto{\pgfqpoint{4.997190in}{2.173061in}}%
\pgfpathlineto{\pgfqpoint{4.989398in}{2.164727in}}%
\pgfpathlineto{\pgfqpoint{4.981597in}{2.156275in}}%
\pgfpathlineto{\pgfqpoint{4.973789in}{2.147704in}}%
\pgfpathclose%
\pgfusepath{fill}%
\end{pgfscope}%
\begin{pgfscope}%
\pgfpathrectangle{\pgfqpoint{1.150000in}{0.150000in}}{\pgfqpoint{5.700000in}{5.700000in}}%
\pgfusepath{clip}%
\pgfsetbuttcap%
\pgfsetroundjoin%
\definecolor{currentfill}{rgb}{0.194100,0.399323,0.555565}%
\pgfsetfillcolor{currentfill}%
\pgfsetfillopacity{0.700000}%
\pgfsetlinewidth{0.000000pt}%
\definecolor{currentstroke}{rgb}{0.000000,0.000000,0.000000}%
\pgfsetstrokecolor{currentstroke}%
\pgfsetdash{}{0pt}%
\pgfpathmoveto{\pgfqpoint{5.240928in}{2.292857in}}%
\pgfpathlineto{\pgfqpoint{5.255508in}{2.297479in}}%
\pgfpathlineto{\pgfqpoint{5.270102in}{2.302172in}}%
\pgfpathlineto{\pgfqpoint{5.284708in}{2.306936in}}%
\pgfpathlineto{\pgfqpoint{5.299328in}{2.311772in}}%
\pgfpathlineto{\pgfqpoint{5.306988in}{2.318654in}}%
\pgfpathlineto{\pgfqpoint{5.314639in}{2.325411in}}%
\pgfpathlineto{\pgfqpoint{5.322282in}{2.332046in}}%
\pgfpathlineto{\pgfqpoint{5.329915in}{2.338559in}}%
\pgfpathlineto{\pgfqpoint{5.315309in}{2.333817in}}%
\pgfpathlineto{\pgfqpoint{5.300716in}{2.329145in}}%
\pgfpathlineto{\pgfqpoint{5.286137in}{2.324544in}}%
\pgfpathlineto{\pgfqpoint{5.271570in}{2.320015in}}%
\pgfpathlineto{\pgfqpoint{5.263923in}{2.313400in}}%
\pgfpathlineto{\pgfqpoint{5.256267in}{2.306670in}}%
\pgfpathlineto{\pgfqpoint{5.248602in}{2.299823in}}%
\pgfpathlineto{\pgfqpoint{5.240928in}{2.292857in}}%
\pgfpathclose%
\pgfusepath{fill}%
\end{pgfscope}%
\begin{pgfscope}%
\pgfpathrectangle{\pgfqpoint{1.150000in}{0.150000in}}{\pgfqpoint{5.700000in}{5.700000in}}%
\pgfusepath{clip}%
\pgfsetbuttcap%
\pgfsetroundjoin%
\definecolor{currentfill}{rgb}{0.268510,0.009605,0.335427}%
\pgfsetfillcolor{currentfill}%
\pgfsetfillopacity{0.700000}%
\pgfsetlinewidth{0.000000pt}%
\definecolor{currentstroke}{rgb}{0.000000,0.000000,0.000000}%
\pgfsetstrokecolor{currentstroke}%
\pgfsetdash{}{0pt}%
\pgfpathmoveto{\pgfqpoint{3.112363in}{1.463579in}}%
\pgfpathlineto{\pgfqpoint{3.126295in}{1.457447in}}%
\pgfpathlineto{\pgfqpoint{3.140230in}{1.451398in}}%
\pgfpathlineto{\pgfqpoint{3.154169in}{1.445430in}}%
\pgfpathlineto{\pgfqpoint{3.168113in}{1.439544in}}%
\pgfpathlineto{\pgfqpoint{3.176665in}{1.445102in}}%
\pgfpathlineto{\pgfqpoint{3.185208in}{1.450837in}}%
\pgfpathlineto{\pgfqpoint{3.193739in}{1.456743in}}%
\pgfpathlineto{\pgfqpoint{3.202261in}{1.462815in}}%
\pgfpathlineto{\pgfqpoint{3.188341in}{1.468330in}}%
\pgfpathlineto{\pgfqpoint{3.174425in}{1.473927in}}%
\pgfpathlineto{\pgfqpoint{3.160513in}{1.479606in}}%
\pgfpathlineto{\pgfqpoint{3.146606in}{1.485366in}}%
\pgfpathlineto{\pgfqpoint{3.138062in}{1.479658in}}%
\pgfpathlineto{\pgfqpoint{3.129506in}{1.474120in}}%
\pgfpathlineto{\pgfqpoint{3.120940in}{1.468758in}}%
\pgfpathlineto{\pgfqpoint{3.112363in}{1.463579in}}%
\pgfpathclose%
\pgfusepath{fill}%
\end{pgfscope}%
\begin{pgfscope}%
\pgfpathrectangle{\pgfqpoint{1.150000in}{0.150000in}}{\pgfqpoint{5.700000in}{5.700000in}}%
\pgfusepath{clip}%
\pgfsetbuttcap%
\pgfsetroundjoin%
\definecolor{currentfill}{rgb}{0.210503,0.363727,0.552206}%
\pgfsetfillcolor{currentfill}%
\pgfsetfillopacity{0.700000}%
\pgfsetlinewidth{0.000000pt}%
\definecolor{currentstroke}{rgb}{0.000000,0.000000,0.000000}%
\pgfsetstrokecolor{currentstroke}%
\pgfsetdash{}{0pt}%
\pgfpathmoveto{\pgfqpoint{5.062855in}{2.197256in}}%
\pgfpathlineto{\pgfqpoint{5.077357in}{2.201429in}}%
\pgfpathlineto{\pgfqpoint{5.091871in}{2.205672in}}%
\pgfpathlineto{\pgfqpoint{5.106398in}{2.209987in}}%
\pgfpathlineto{\pgfqpoint{5.120937in}{2.214373in}}%
\pgfpathlineto{\pgfqpoint{5.128692in}{2.222393in}}%
\pgfpathlineto{\pgfqpoint{5.136438in}{2.230287in}}%
\pgfpathlineto{\pgfqpoint{5.144175in}{2.238056in}}%
\pgfpathlineto{\pgfqpoint{5.151905in}{2.245702in}}%
\pgfpathlineto{\pgfqpoint{5.137377in}{2.241364in}}%
\pgfpathlineto{\pgfqpoint{5.122861in}{2.237098in}}%
\pgfpathlineto{\pgfqpoint{5.108359in}{2.232902in}}%
\pgfpathlineto{\pgfqpoint{5.093869in}{2.228778in}}%
\pgfpathlineto{\pgfqpoint{5.086128in}{2.221076in}}%
\pgfpathlineto{\pgfqpoint{5.078378in}{2.213256in}}%
\pgfpathlineto{\pgfqpoint{5.070621in}{2.205316in}}%
\pgfpathlineto{\pgfqpoint{5.062855in}{2.197256in}}%
\pgfpathclose%
\pgfusepath{fill}%
\end{pgfscope}%
\begin{pgfscope}%
\pgfpathrectangle{\pgfqpoint{1.150000in}{0.150000in}}{\pgfqpoint{5.700000in}{5.700000in}}%
\pgfusepath{clip}%
\pgfsetbuttcap%
\pgfsetroundjoin%
\definecolor{currentfill}{rgb}{0.267004,0.004874,0.329415}%
\pgfsetfillcolor{currentfill}%
\pgfsetfillopacity{0.700000}%
\pgfsetlinewidth{0.000000pt}%
\definecolor{currentstroke}{rgb}{0.000000,0.000000,0.000000}%
\pgfsetstrokecolor{currentstroke}%
\pgfsetdash{}{0pt}%
\pgfpathmoveto{\pgfqpoint{3.257986in}{1.441556in}}%
\pgfpathlineto{\pgfqpoint{3.271929in}{1.436442in}}%
\pgfpathlineto{\pgfqpoint{3.285876in}{1.431406in}}%
\pgfpathlineto{\pgfqpoint{3.299829in}{1.426450in}}%
\pgfpathlineto{\pgfqpoint{3.313786in}{1.421572in}}%
\pgfpathlineto{\pgfqpoint{3.322255in}{1.428518in}}%
\pgfpathlineto{\pgfqpoint{3.330714in}{1.435608in}}%
\pgfpathlineto{\pgfqpoint{3.339165in}{1.442836in}}%
\pgfpathlineto{\pgfqpoint{3.347607in}{1.450197in}}%
\pgfpathlineto{\pgfqpoint{3.333669in}{1.454725in}}%
\pgfpathlineto{\pgfqpoint{3.319737in}{1.459332in}}%
\pgfpathlineto{\pgfqpoint{3.305810in}{1.464018in}}%
\pgfpathlineto{\pgfqpoint{3.291888in}{1.468783in}}%
\pgfpathlineto{\pgfqpoint{3.283426in}{1.461763in}}%
\pgfpathlineto{\pgfqpoint{3.274955in}{1.454882in}}%
\pgfpathlineto{\pgfqpoint{3.266475in}{1.448145in}}%
\pgfpathlineto{\pgfqpoint{3.257986in}{1.441556in}}%
\pgfpathclose%
\pgfusepath{fill}%
\end{pgfscope}%
\begin{pgfscope}%
\pgfpathrectangle{\pgfqpoint{1.150000in}{0.150000in}}{\pgfqpoint{5.700000in}{5.700000in}}%
\pgfusepath{clip}%
\pgfsetbuttcap%
\pgfsetroundjoin%
\definecolor{currentfill}{rgb}{0.203063,0.379716,0.553925}%
\pgfsetfillcolor{currentfill}%
\pgfsetfillopacity{0.700000}%
\pgfsetlinewidth{0.000000pt}%
\definecolor{currentstroke}{rgb}{0.000000,0.000000,0.000000}%
\pgfsetstrokecolor{currentstroke}%
\pgfsetdash{}{0pt}%
\pgfpathmoveto{\pgfqpoint{5.151905in}{2.245702in}}%
\pgfpathlineto{\pgfqpoint{5.166446in}{2.250110in}}%
\pgfpathlineto{\pgfqpoint{5.180999in}{2.254590in}}%
\pgfpathlineto{\pgfqpoint{5.195566in}{2.259140in}}%
\pgfpathlineto{\pgfqpoint{5.210146in}{2.263762in}}%
\pgfpathlineto{\pgfqpoint{5.217854in}{2.271223in}}%
\pgfpathlineto{\pgfqpoint{5.225554in}{2.278558in}}%
\pgfpathlineto{\pgfqpoint{5.233246in}{2.285769in}}%
\pgfpathlineto{\pgfqpoint{5.240928in}{2.292857in}}%
\pgfpathlineto{\pgfqpoint{5.226361in}{2.288306in}}%
\pgfpathlineto{\pgfqpoint{5.211807in}{2.283825in}}%
\pgfpathlineto{\pgfqpoint{5.197266in}{2.279416in}}%
\pgfpathlineto{\pgfqpoint{5.182737in}{2.275078in}}%
\pgfpathlineto{\pgfqpoint{5.175042in}{2.267911in}}%
\pgfpathlineto{\pgfqpoint{5.167338in}{2.260628in}}%
\pgfpathlineto{\pgfqpoint{5.159626in}{2.253225in}}%
\pgfpathlineto{\pgfqpoint{5.151905in}{2.245702in}}%
\pgfpathclose%
\pgfusepath{fill}%
\end{pgfscope}%
\begin{pgfscope}%
\pgfpathrectangle{\pgfqpoint{1.150000in}{0.150000in}}{\pgfqpoint{5.700000in}{5.700000in}}%
\pgfusepath{clip}%
\pgfsetbuttcap%
\pgfsetroundjoin%
\definecolor{currentfill}{rgb}{0.274952,0.037752,0.364543}%
\pgfsetfillcolor{currentfill}%
\pgfsetfillopacity{0.700000}%
\pgfsetlinewidth{0.000000pt}%
\definecolor{currentstroke}{rgb}{0.000000,0.000000,0.000000}%
\pgfsetstrokecolor{currentstroke}%
\pgfsetdash{}{0pt}%
\pgfpathmoveto{\pgfqpoint{3.727273in}{1.491277in}}%
\pgfpathlineto{\pgfqpoint{3.741295in}{1.489278in}}%
\pgfpathlineto{\pgfqpoint{3.755323in}{1.487353in}}%
\pgfpathlineto{\pgfqpoint{3.769359in}{1.485503in}}%
\pgfpathlineto{\pgfqpoint{3.783403in}{1.483726in}}%
\pgfpathlineto{\pgfqpoint{3.791659in}{1.494035in}}%
\pgfpathlineto{\pgfqpoint{3.799910in}{1.504382in}}%
\pgfpathlineto{\pgfqpoint{3.808155in}{1.514763in}}%
\pgfpathlineto{\pgfqpoint{3.816394in}{1.525175in}}%
\pgfpathlineto{\pgfqpoint{3.802362in}{1.526683in}}%
\pgfpathlineto{\pgfqpoint{3.788338in}{1.528266in}}%
\pgfpathlineto{\pgfqpoint{3.774321in}{1.529922in}}%
\pgfpathlineto{\pgfqpoint{3.760312in}{1.531654in}}%
\pgfpathlineto{\pgfqpoint{3.752061in}{1.521502in}}%
\pgfpathlineto{\pgfqpoint{3.743805in}{1.511386in}}%
\pgfpathlineto{\pgfqpoint{3.735542in}{1.501310in}}%
\pgfpathlineto{\pgfqpoint{3.727273in}{1.491277in}}%
\pgfpathclose%
\pgfusepath{fill}%
\end{pgfscope}%
\begin{pgfscope}%
\pgfpathrectangle{\pgfqpoint{1.150000in}{0.150000in}}{\pgfqpoint{5.700000in}{5.700000in}}%
\pgfusepath{clip}%
\pgfsetbuttcap%
\pgfsetroundjoin%
\definecolor{currentfill}{rgb}{0.280267,0.073417,0.397163}%
\pgfsetfillcolor{currentfill}%
\pgfsetfillopacity{0.700000}%
\pgfsetlinewidth{0.000000pt}%
\definecolor{currentstroke}{rgb}{0.000000,0.000000,0.000000}%
\pgfsetstrokecolor{currentstroke}%
\pgfsetdash{}{0pt}%
\pgfpathmoveto{\pgfqpoint{2.764228in}{1.585922in}}%
\pgfpathlineto{\pgfqpoint{2.778161in}{1.577266in}}%
\pgfpathlineto{\pgfqpoint{2.792095in}{1.568700in}}%
\pgfpathlineto{\pgfqpoint{2.806031in}{1.560223in}}%
\pgfpathlineto{\pgfqpoint{2.819970in}{1.551836in}}%
\pgfpathlineto{\pgfqpoint{2.828766in}{1.553653in}}%
\pgfpathlineto{\pgfqpoint{2.837547in}{1.555723in}}%
\pgfpathlineto{\pgfqpoint{2.846314in}{1.558040in}}%
\pgfpathlineto{\pgfqpoint{2.855066in}{1.560596in}}%
\pgfpathlineto{\pgfqpoint{2.841160in}{1.568567in}}%
\pgfpathlineto{\pgfqpoint{2.827256in}{1.576628in}}%
\pgfpathlineto{\pgfqpoint{2.813354in}{1.584778in}}%
\pgfpathlineto{\pgfqpoint{2.799454in}{1.593018in}}%
\pgfpathlineto{\pgfqpoint{2.790671in}{1.590870in}}%
\pgfpathlineto{\pgfqpoint{2.781872in}{1.588967in}}%
\pgfpathlineto{\pgfqpoint{2.773058in}{1.587315in}}%
\pgfpathlineto{\pgfqpoint{2.764228in}{1.585922in}}%
\pgfpathclose%
\pgfusepath{fill}%
\end{pgfscope}%
\begin{pgfscope}%
\pgfpathrectangle{\pgfqpoint{1.150000in}{0.150000in}}{\pgfqpoint{5.700000in}{5.700000in}}%
\pgfusepath{clip}%
\pgfsetbuttcap%
\pgfsetroundjoin%
\definecolor{currentfill}{rgb}{0.267004,0.004874,0.329415}%
\pgfsetfillcolor{currentfill}%
\pgfsetfillopacity{0.700000}%
\pgfsetlinewidth{0.000000pt}%
\definecolor{currentstroke}{rgb}{0.000000,0.000000,0.000000}%
\pgfsetstrokecolor{currentstroke}%
\pgfsetdash{}{0pt}%
\pgfpathmoveto{\pgfqpoint{3.403409in}{1.432869in}}%
\pgfpathlineto{\pgfqpoint{3.417373in}{1.428731in}}%
\pgfpathlineto{\pgfqpoint{3.431343in}{1.424671in}}%
\pgfpathlineto{\pgfqpoint{3.445318in}{1.420688in}}%
\pgfpathlineto{\pgfqpoint{3.459300in}{1.416781in}}%
\pgfpathlineto{\pgfqpoint{3.467696in}{1.424944in}}%
\pgfpathlineto{\pgfqpoint{3.476085in}{1.433219in}}%
\pgfpathlineto{\pgfqpoint{3.484465in}{1.441603in}}%
\pgfpathlineto{\pgfqpoint{3.492838in}{1.450089in}}%
\pgfpathlineto{\pgfqpoint{3.478874in}{1.453666in}}%
\pgfpathlineto{\pgfqpoint{3.464916in}{1.457320in}}%
\pgfpathlineto{\pgfqpoint{3.450964in}{1.461052in}}%
\pgfpathlineto{\pgfqpoint{3.437018in}{1.464861in}}%
\pgfpathlineto{\pgfqpoint{3.428627in}{1.456696in}}%
\pgfpathlineto{\pgfqpoint{3.420229in}{1.448639in}}%
\pgfpathlineto{\pgfqpoint{3.411823in}{1.440695in}}%
\pgfpathlineto{\pgfqpoint{3.403409in}{1.432869in}}%
\pgfpathclose%
\pgfusepath{fill}%
\end{pgfscope}%
\begin{pgfscope}%
\pgfpathrectangle{\pgfqpoint{1.150000in}{0.150000in}}{\pgfqpoint{5.700000in}{5.700000in}}%
\pgfusepath{clip}%
\pgfsetbuttcap%
\pgfsetroundjoin%
\definecolor{currentfill}{rgb}{0.273809,0.031497,0.358853}%
\pgfsetfillcolor{currentfill}%
\pgfsetfillopacity{0.700000}%
\pgfsetlinewidth{0.000000pt}%
\definecolor{currentstroke}{rgb}{0.000000,0.000000,0.000000}%
\pgfsetstrokecolor{currentstroke}%
\pgfsetdash{}{0pt}%
\pgfpathmoveto{\pgfqpoint{2.966407in}{1.499970in}}%
\pgfpathlineto{\pgfqpoint{2.980337in}{1.492779in}}%
\pgfpathlineto{\pgfqpoint{2.994271in}{1.485673in}}%
\pgfpathlineto{\pgfqpoint{3.008207in}{1.478651in}}%
\pgfpathlineto{\pgfqpoint{3.022147in}{1.471713in}}%
\pgfpathlineto{\pgfqpoint{3.030798in}{1.475707in}}%
\pgfpathlineto{\pgfqpoint{3.039436in}{1.479911in}}%
\pgfpathlineto{\pgfqpoint{3.048062in}{1.484320in}}%
\pgfpathlineto{\pgfqpoint{3.056676in}{1.488928in}}%
\pgfpathlineto{\pgfqpoint{3.042763in}{1.495473in}}%
\pgfpathlineto{\pgfqpoint{3.028854in}{1.502102in}}%
\pgfpathlineto{\pgfqpoint{3.014948in}{1.508816in}}%
\pgfpathlineto{\pgfqpoint{3.001046in}{1.515614in}}%
\pgfpathlineto{\pgfqpoint{2.992405in}{1.511390in}}%
\pgfpathlineto{\pgfqpoint{2.983752in}{1.507371in}}%
\pgfpathlineto{\pgfqpoint{2.975086in}{1.503563in}}%
\pgfpathlineto{\pgfqpoint{2.966407in}{1.499970in}}%
\pgfpathclose%
\pgfusepath{fill}%
\end{pgfscope}%
\begin{pgfscope}%
\pgfpathrectangle{\pgfqpoint{1.150000in}{0.150000in}}{\pgfqpoint{5.700000in}{5.700000in}}%
\pgfusepath{clip}%
\pgfsetbuttcap%
\pgfsetroundjoin%
\definecolor{currentfill}{rgb}{0.271305,0.019942,0.347269}%
\pgfsetfillcolor{currentfill}%
\pgfsetfillopacity{0.700000}%
\pgfsetlinewidth{0.000000pt}%
\definecolor{currentstroke}{rgb}{0.000000,0.000000,0.000000}%
\pgfsetstrokecolor{currentstroke}%
\pgfsetdash{}{0pt}%
\pgfpathmoveto{\pgfqpoint{3.638070in}{1.461558in}}%
\pgfpathlineto{\pgfqpoint{3.652076in}{1.458972in}}%
\pgfpathlineto{\pgfqpoint{3.666089in}{1.456461in}}%
\pgfpathlineto{\pgfqpoint{3.680109in}{1.454024in}}%
\pgfpathlineto{\pgfqpoint{3.694136in}{1.451663in}}%
\pgfpathlineto{\pgfqpoint{3.702430in}{1.461480in}}%
\pgfpathlineto{\pgfqpoint{3.710717in}{1.471358in}}%
\pgfpathlineto{\pgfqpoint{3.718999in}{1.481292in}}%
\pgfpathlineto{\pgfqpoint{3.727273in}{1.491277in}}%
\pgfpathlineto{\pgfqpoint{3.713259in}{1.493350in}}%
\pgfpathlineto{\pgfqpoint{3.699252in}{1.495498in}}%
\pgfpathlineto{\pgfqpoint{3.685252in}{1.497721in}}%
\pgfpathlineto{\pgfqpoint{3.671260in}{1.500019in}}%
\pgfpathlineto{\pgfqpoint{3.662972in}{1.490315in}}%
\pgfpathlineto{\pgfqpoint{3.654678in}{1.480667in}}%
\pgfpathlineto{\pgfqpoint{3.646377in}{1.471080in}}%
\pgfpathlineto{\pgfqpoint{3.638070in}{1.461558in}}%
\pgfpathclose%
\pgfusepath{fill}%
\end{pgfscope}%
\begin{pgfscope}%
\pgfpathrectangle{\pgfqpoint{1.150000in}{0.150000in}}{\pgfqpoint{5.700000in}{5.700000in}}%
\pgfusepath{clip}%
\pgfsetbuttcap%
\pgfsetroundjoin%
\definecolor{currentfill}{rgb}{0.268510,0.009605,0.335427}%
\pgfsetfillcolor{currentfill}%
\pgfsetfillopacity{0.700000}%
\pgfsetlinewidth{0.000000pt}%
\definecolor{currentstroke}{rgb}{0.000000,0.000000,0.000000}%
\pgfsetstrokecolor{currentstroke}%
\pgfsetdash{}{0pt}%
\pgfpathmoveto{\pgfqpoint{3.548755in}{1.436543in}}%
\pgfpathlineto{\pgfqpoint{3.562750in}{1.433347in}}%
\pgfpathlineto{\pgfqpoint{3.576751in}{1.430227in}}%
\pgfpathlineto{\pgfqpoint{3.590759in}{1.427181in}}%
\pgfpathlineto{\pgfqpoint{3.604773in}{1.424211in}}%
\pgfpathlineto{\pgfqpoint{3.613107in}{1.433428in}}%
\pgfpathlineto{\pgfqpoint{3.621435in}{1.442728in}}%
\pgfpathlineto{\pgfqpoint{3.629756in}{1.452106in}}%
\pgfpathlineto{\pgfqpoint{3.638070in}{1.461558in}}%
\pgfpathlineto{\pgfqpoint{3.624070in}{1.464220in}}%
\pgfpathlineto{\pgfqpoint{3.610077in}{1.466956in}}%
\pgfpathlineto{\pgfqpoint{3.596091in}{1.469768in}}%
\pgfpathlineto{\pgfqpoint{3.582111in}{1.472656in}}%
\pgfpathlineto{\pgfqpoint{3.573783in}{1.463504in}}%
\pgfpathlineto{\pgfqpoint{3.565447in}{1.454432in}}%
\pgfpathlineto{\pgfqpoint{3.557105in}{1.445444in}}%
\pgfpathlineto{\pgfqpoint{3.548755in}{1.436543in}}%
\pgfpathclose%
\pgfusepath{fill}%
\end{pgfscope}%
\begin{pgfscope}%
\pgfpathrectangle{\pgfqpoint{1.150000in}{0.150000in}}{\pgfqpoint{5.700000in}{5.700000in}}%
\pgfusepath{clip}%
\pgfsetbuttcap%
\pgfsetroundjoin%
\definecolor{currentfill}{rgb}{0.283072,0.130895,0.449241}%
\pgfsetfillcolor{currentfill}%
\pgfsetfillopacity{0.700000}%
\pgfsetlinewidth{0.000000pt}%
\definecolor{currentstroke}{rgb}{0.000000,0.000000,0.000000}%
\pgfsetstrokecolor{currentstroke}%
\pgfsetdash{}{0pt}%
\pgfpathmoveto{\pgfqpoint{4.139972in}{1.647988in}}%
\pgfpathlineto{\pgfqpoint{4.154119in}{1.648404in}}%
\pgfpathlineto{\pgfqpoint{4.168276in}{1.648892in}}%
\pgfpathlineto{\pgfqpoint{4.182441in}{1.649453in}}%
\pgfpathlineto{\pgfqpoint{4.196617in}{1.650086in}}%
\pgfpathlineto{\pgfqpoint{4.204740in}{1.661557in}}%
\pgfpathlineto{\pgfqpoint{4.212857in}{1.672988in}}%
\pgfpathlineto{\pgfqpoint{4.220969in}{1.684376in}}%
\pgfpathlineto{\pgfqpoint{4.229076in}{1.695720in}}%
\pgfpathlineto{\pgfqpoint{4.214908in}{1.694900in}}%
\pgfpathlineto{\pgfqpoint{4.200750in}{1.694153in}}%
\pgfpathlineto{\pgfqpoint{4.186601in}{1.693477in}}%
\pgfpathlineto{\pgfqpoint{4.172461in}{1.692875in}}%
\pgfpathlineto{\pgfqpoint{4.164347in}{1.681710in}}%
\pgfpathlineto{\pgfqpoint{4.156227in}{1.670506in}}%
\pgfpathlineto{\pgfqpoint{4.148102in}{1.659264in}}%
\pgfpathlineto{\pgfqpoint{4.139972in}{1.647988in}}%
\pgfpathclose%
\pgfusepath{fill}%
\end{pgfscope}%
\begin{pgfscope}%
\pgfpathrectangle{\pgfqpoint{1.150000in}{0.150000in}}{\pgfqpoint{5.700000in}{5.700000in}}%
\pgfusepath{clip}%
\pgfsetbuttcap%
\pgfsetroundjoin%
\definecolor{currentfill}{rgb}{0.281887,0.150881,0.465405}%
\pgfsetfillcolor{currentfill}%
\pgfsetfillopacity{0.700000}%
\pgfsetlinewidth{0.000000pt}%
\definecolor{currentstroke}{rgb}{0.000000,0.000000,0.000000}%
\pgfsetstrokecolor{currentstroke}%
\pgfsetdash{}{0pt}%
\pgfpathmoveto{\pgfqpoint{4.229076in}{1.695720in}}%
\pgfpathlineto{\pgfqpoint{4.243253in}{1.696612in}}%
\pgfpathlineto{\pgfqpoint{4.257441in}{1.697575in}}%
\pgfpathlineto{\pgfqpoint{4.271637in}{1.698611in}}%
\pgfpathlineto{\pgfqpoint{4.285844in}{1.699719in}}%
\pgfpathlineto{\pgfqpoint{4.293938in}{1.711189in}}%
\pgfpathlineto{\pgfqpoint{4.302027in}{1.722603in}}%
\pgfpathlineto{\pgfqpoint{4.310111in}{1.733961in}}%
\pgfpathlineto{\pgfqpoint{4.318189in}{1.745259in}}%
\pgfpathlineto{\pgfqpoint{4.303989in}{1.743986in}}%
\pgfpathlineto{\pgfqpoint{4.289799in}{1.742784in}}%
\pgfpathlineto{\pgfqpoint{4.275619in}{1.741654in}}%
\pgfpathlineto{\pgfqpoint{4.261449in}{1.740596in}}%
\pgfpathlineto{\pgfqpoint{4.253364in}{1.729456in}}%
\pgfpathlineto{\pgfqpoint{4.245273in}{1.718262in}}%
\pgfpathlineto{\pgfqpoint{4.237177in}{1.707016in}}%
\pgfpathlineto{\pgfqpoint{4.229076in}{1.695720in}}%
\pgfpathclose%
\pgfusepath{fill}%
\end{pgfscope}%
\begin{pgfscope}%
\pgfpathrectangle{\pgfqpoint{1.150000in}{0.150000in}}{\pgfqpoint{5.700000in}{5.700000in}}%
\pgfusepath{clip}%
\pgfsetbuttcap%
\pgfsetroundjoin%
\definecolor{currentfill}{rgb}{0.282910,0.105393,0.426902}%
\pgfsetfillcolor{currentfill}%
\pgfsetfillopacity{0.700000}%
\pgfsetlinewidth{0.000000pt}%
\definecolor{currentstroke}{rgb}{0.000000,0.000000,0.000000}%
\pgfsetstrokecolor{currentstroke}%
\pgfsetdash{}{0pt}%
\pgfpathmoveto{\pgfqpoint{4.050866in}{1.602477in}}%
\pgfpathlineto{\pgfqpoint{4.064986in}{1.602396in}}%
\pgfpathlineto{\pgfqpoint{4.079114in}{1.602388in}}%
\pgfpathlineto{\pgfqpoint{4.093251in}{1.602452in}}%
\pgfpathlineto{\pgfqpoint{4.107397in}{1.602589in}}%
\pgfpathlineto{\pgfqpoint{4.115549in}{1.613977in}}%
\pgfpathlineto{\pgfqpoint{4.123695in}{1.625341in}}%
\pgfpathlineto{\pgfqpoint{4.131836in}{1.636679in}}%
\pgfpathlineto{\pgfqpoint{4.139972in}{1.647988in}}%
\pgfpathlineto{\pgfqpoint{4.125834in}{1.647644in}}%
\pgfpathlineto{\pgfqpoint{4.111705in}{1.647372in}}%
\pgfpathlineto{\pgfqpoint{4.097585in}{1.647173in}}%
\pgfpathlineto{\pgfqpoint{4.083474in}{1.647047in}}%
\pgfpathlineto{\pgfqpoint{4.075330in}{1.635938in}}%
\pgfpathlineto{\pgfqpoint{4.067181in}{1.624805in}}%
\pgfpathlineto{\pgfqpoint{4.059026in}{1.613650in}}%
\pgfpathlineto{\pgfqpoint{4.050866in}{1.602477in}}%
\pgfpathclose%
\pgfusepath{fill}%
\end{pgfscope}%
\begin{pgfscope}%
\pgfpathrectangle{\pgfqpoint{1.150000in}{0.150000in}}{\pgfqpoint{5.700000in}{5.700000in}}%
\pgfusepath{clip}%
\pgfsetbuttcap%
\pgfsetroundjoin%
\definecolor{currentfill}{rgb}{0.278791,0.062145,0.386592}%
\pgfsetfillcolor{currentfill}%
\pgfsetfillopacity{0.700000}%
\pgfsetlinewidth{0.000000pt}%
\definecolor{currentstroke}{rgb}{0.000000,0.000000,0.000000}%
\pgfsetstrokecolor{currentstroke}%
\pgfsetdash{}{0pt}%
\pgfpathmoveto{\pgfqpoint{2.819970in}{1.551836in}}%
\pgfpathlineto{\pgfqpoint{2.833910in}{1.543537in}}%
\pgfpathlineto{\pgfqpoint{2.847853in}{1.535326in}}%
\pgfpathlineto{\pgfqpoint{2.861797in}{1.527203in}}%
\pgfpathlineto{\pgfqpoint{2.875745in}{1.519167in}}%
\pgfpathlineto{\pgfqpoint{2.884509in}{1.521408in}}%
\pgfpathlineto{\pgfqpoint{2.893258in}{1.523897in}}%
\pgfpathlineto{\pgfqpoint{2.901994in}{1.526626in}}%
\pgfpathlineto{\pgfqpoint{2.910715in}{1.529589in}}%
\pgfpathlineto{\pgfqpoint{2.896799in}{1.537209in}}%
\pgfpathlineto{\pgfqpoint{2.882886in}{1.544917in}}%
\pgfpathlineto{\pgfqpoint{2.868975in}{1.552712in}}%
\pgfpathlineto{\pgfqpoint{2.855066in}{1.560596in}}%
\pgfpathlineto{\pgfqpoint{2.846314in}{1.558040in}}%
\pgfpathlineto{\pgfqpoint{2.837547in}{1.555723in}}%
\pgfpathlineto{\pgfqpoint{2.828766in}{1.553653in}}%
\pgfpathlineto{\pgfqpoint{2.819970in}{1.551836in}}%
\pgfpathclose%
\pgfusepath{fill}%
\end{pgfscope}%
\begin{pgfscope}%
\pgfpathrectangle{\pgfqpoint{1.150000in}{0.150000in}}{\pgfqpoint{5.700000in}{5.700000in}}%
\pgfusepath{clip}%
\pgfsetbuttcap%
\pgfsetroundjoin%
\definecolor{currentfill}{rgb}{0.268510,0.009605,0.335427}%
\pgfsetfillcolor{currentfill}%
\pgfsetfillopacity{0.700000}%
\pgfsetlinewidth{0.000000pt}%
\definecolor{currentstroke}{rgb}{0.000000,0.000000,0.000000}%
\pgfsetstrokecolor{currentstroke}%
\pgfsetdash{}{0pt}%
\pgfpathmoveto{\pgfqpoint{3.168113in}{1.439544in}}%
\pgfpathlineto{\pgfqpoint{3.182060in}{1.433738in}}%
\pgfpathlineto{\pgfqpoint{3.196012in}{1.428012in}}%
\pgfpathlineto{\pgfqpoint{3.209968in}{1.422367in}}%
\pgfpathlineto{\pgfqpoint{3.223929in}{1.416802in}}%
\pgfpathlineto{\pgfqpoint{3.232458in}{1.422740in}}%
\pgfpathlineto{\pgfqpoint{3.240977in}{1.428848in}}%
\pgfpathlineto{\pgfqpoint{3.249487in}{1.435122in}}%
\pgfpathlineto{\pgfqpoint{3.257986in}{1.441556in}}%
\pgfpathlineto{\pgfqpoint{3.244048in}{1.446751in}}%
\pgfpathlineto{\pgfqpoint{3.230114in}{1.452025in}}%
\pgfpathlineto{\pgfqpoint{3.216185in}{1.457380in}}%
\pgfpathlineto{\pgfqpoint{3.202261in}{1.462815in}}%
\pgfpathlineto{\pgfqpoint{3.193739in}{1.456743in}}%
\pgfpathlineto{\pgfqpoint{3.185208in}{1.450837in}}%
\pgfpathlineto{\pgfqpoint{3.176665in}{1.445102in}}%
\pgfpathlineto{\pgfqpoint{3.168113in}{1.439544in}}%
\pgfpathclose%
\pgfusepath{fill}%
\end{pgfscope}%
\begin{pgfscope}%
\pgfpathrectangle{\pgfqpoint{1.150000in}{0.150000in}}{\pgfqpoint{5.700000in}{5.700000in}}%
\pgfusepath{clip}%
\pgfsetbuttcap%
\pgfsetroundjoin%
\definecolor{currentfill}{rgb}{0.278826,0.175490,0.483397}%
\pgfsetfillcolor{currentfill}%
\pgfsetfillopacity{0.700000}%
\pgfsetlinewidth{0.000000pt}%
\definecolor{currentstroke}{rgb}{0.000000,0.000000,0.000000}%
\pgfsetstrokecolor{currentstroke}%
\pgfsetdash{}{0pt}%
\pgfpathmoveto{\pgfqpoint{4.318189in}{1.745259in}}%
\pgfpathlineto{\pgfqpoint{4.332398in}{1.746605in}}%
\pgfpathlineto{\pgfqpoint{4.346618in}{1.748023in}}%
\pgfpathlineto{\pgfqpoint{4.360847in}{1.749512in}}%
\pgfpathlineto{\pgfqpoint{4.375087in}{1.751073in}}%
\pgfpathlineto{\pgfqpoint{4.383153in}{1.762463in}}%
\pgfpathlineto{\pgfqpoint{4.391213in}{1.773784in}}%
\pgfpathlineto{\pgfqpoint{4.399268in}{1.785035in}}%
\pgfpathlineto{\pgfqpoint{4.407316in}{1.796214in}}%
\pgfpathlineto{\pgfqpoint{4.393083in}{1.794507in}}%
\pgfpathlineto{\pgfqpoint{4.378860in}{1.792872in}}%
\pgfpathlineto{\pgfqpoint{4.364648in}{1.791310in}}%
\pgfpathlineto{\pgfqpoint{4.350445in}{1.789819in}}%
\pgfpathlineto{\pgfqpoint{4.342389in}{1.778778in}}%
\pgfpathlineto{\pgfqpoint{4.334328in}{1.767670in}}%
\pgfpathlineto{\pgfqpoint{4.326261in}{1.756496in}}%
\pgfpathlineto{\pgfqpoint{4.318189in}{1.745259in}}%
\pgfpathclose%
\pgfusepath{fill}%
\end{pgfscope}%
\begin{pgfscope}%
\pgfpathrectangle{\pgfqpoint{1.150000in}{0.150000in}}{\pgfqpoint{5.700000in}{5.700000in}}%
\pgfusepath{clip}%
\pgfsetbuttcap%
\pgfsetroundjoin%
\definecolor{currentfill}{rgb}{0.281446,0.084320,0.407414}%
\pgfsetfillcolor{currentfill}%
\pgfsetfillopacity{0.700000}%
\pgfsetlinewidth{0.000000pt}%
\definecolor{currentstroke}{rgb}{0.000000,0.000000,0.000000}%
\pgfsetstrokecolor{currentstroke}%
\pgfsetdash{}{0pt}%
\pgfpathmoveto{\pgfqpoint{3.961747in}{1.559622in}}%
\pgfpathlineto{\pgfqpoint{3.975841in}{1.559023in}}%
\pgfpathlineto{\pgfqpoint{3.989943in}{1.558496in}}%
\pgfpathlineto{\pgfqpoint{4.004054in}{1.558042in}}%
\pgfpathlineto{\pgfqpoint{4.018173in}{1.557661in}}%
\pgfpathlineto{\pgfqpoint{4.026354in}{1.568877in}}%
\pgfpathlineto{\pgfqpoint{4.034530in}{1.580087in}}%
\pgfpathlineto{\pgfqpoint{4.042701in}{1.591288in}}%
\pgfpathlineto{\pgfqpoint{4.050866in}{1.602477in}}%
\pgfpathlineto{\pgfqpoint{4.036756in}{1.602630in}}%
\pgfpathlineto{\pgfqpoint{4.022654in}{1.602857in}}%
\pgfpathlineto{\pgfqpoint{4.008561in}{1.603156in}}%
\pgfpathlineto{\pgfqpoint{3.994476in}{1.603528in}}%
\pgfpathlineto{\pgfqpoint{3.986302in}{1.592559in}}%
\pgfpathlineto{\pgfqpoint{3.978123in}{1.581583in}}%
\pgfpathlineto{\pgfqpoint{3.969938in}{1.570603in}}%
\pgfpathlineto{\pgfqpoint{3.961747in}{1.559622in}}%
\pgfpathclose%
\pgfusepath{fill}%
\end{pgfscope}%
\begin{pgfscope}%
\pgfpathrectangle{\pgfqpoint{1.150000in}{0.150000in}}{\pgfqpoint{5.700000in}{5.700000in}}%
\pgfusepath{clip}%
\pgfsetbuttcap%
\pgfsetroundjoin%
\definecolor{currentfill}{rgb}{0.267004,0.004874,0.329415}%
\pgfsetfillcolor{currentfill}%
\pgfsetfillopacity{0.700000}%
\pgfsetlinewidth{0.000000pt}%
\definecolor{currentstroke}{rgb}{0.000000,0.000000,0.000000}%
\pgfsetstrokecolor{currentstroke}%
\pgfsetdash{}{0pt}%
\pgfpathmoveto{\pgfqpoint{3.313786in}{1.421572in}}%
\pgfpathlineto{\pgfqpoint{3.327749in}{1.416773in}}%
\pgfpathlineto{\pgfqpoint{3.341716in}{1.412053in}}%
\pgfpathlineto{\pgfqpoint{3.355689in}{1.407410in}}%
\pgfpathlineto{\pgfqpoint{3.369667in}{1.402845in}}%
\pgfpathlineto{\pgfqpoint{3.378115in}{1.410148in}}%
\pgfpathlineto{\pgfqpoint{3.386555in}{1.417590in}}%
\pgfpathlineto{\pgfqpoint{3.394986in}{1.425166in}}%
\pgfpathlineto{\pgfqpoint{3.403409in}{1.432869in}}%
\pgfpathlineto{\pgfqpoint{3.389450in}{1.437084in}}%
\pgfpathlineto{\pgfqpoint{3.375497in}{1.441377in}}%
\pgfpathlineto{\pgfqpoint{3.361549in}{1.445748in}}%
\pgfpathlineto{\pgfqpoint{3.347607in}{1.450197in}}%
\pgfpathlineto{\pgfqpoint{3.339165in}{1.442836in}}%
\pgfpathlineto{\pgfqpoint{3.330714in}{1.435608in}}%
\pgfpathlineto{\pgfqpoint{3.322255in}{1.428518in}}%
\pgfpathlineto{\pgfqpoint{3.313786in}{1.421572in}}%
\pgfpathclose%
\pgfusepath{fill}%
\end{pgfscope}%
\begin{pgfscope}%
\pgfpathrectangle{\pgfqpoint{1.150000in}{0.150000in}}{\pgfqpoint{5.700000in}{5.700000in}}%
\pgfusepath{clip}%
\pgfsetbuttcap%
\pgfsetroundjoin%
\definecolor{currentfill}{rgb}{0.274128,0.199721,0.498911}%
\pgfsetfillcolor{currentfill}%
\pgfsetfillopacity{0.700000}%
\pgfsetlinewidth{0.000000pt}%
\definecolor{currentstroke}{rgb}{0.000000,0.000000,0.000000}%
\pgfsetstrokecolor{currentstroke}%
\pgfsetdash{}{0pt}%
\pgfpathmoveto{\pgfqpoint{4.407316in}{1.796214in}}%
\pgfpathlineto{\pgfqpoint{4.421560in}{1.797992in}}%
\pgfpathlineto{\pgfqpoint{4.435814in}{1.799842in}}%
\pgfpathlineto{\pgfqpoint{4.450078in}{1.801763in}}%
\pgfpathlineto{\pgfqpoint{4.464353in}{1.803756in}}%
\pgfpathlineto{\pgfqpoint{4.472389in}{1.814993in}}%
\pgfpathlineto{\pgfqpoint{4.480420in}{1.826148in}}%
\pgfpathlineto{\pgfqpoint{4.488444in}{1.837221in}}%
\pgfpathlineto{\pgfqpoint{4.496463in}{1.848210in}}%
\pgfpathlineto{\pgfqpoint{4.482195in}{1.846092in}}%
\pgfpathlineto{\pgfqpoint{4.467937in}{1.844046in}}%
\pgfpathlineto{\pgfqpoint{4.453690in}{1.842072in}}%
\pgfpathlineto{\pgfqpoint{4.439454in}{1.840170in}}%
\pgfpathlineto{\pgfqpoint{4.431428in}{1.829297in}}%
\pgfpathlineto{\pgfqpoint{4.423397in}{1.818346in}}%
\pgfpathlineto{\pgfqpoint{4.415359in}{1.807318in}}%
\pgfpathlineto{\pgfqpoint{4.407316in}{1.796214in}}%
\pgfpathclose%
\pgfusepath{fill}%
\end{pgfscope}%
\begin{pgfscope}%
\pgfpathrectangle{\pgfqpoint{1.150000in}{0.150000in}}{\pgfqpoint{5.700000in}{5.700000in}}%
\pgfusepath{clip}%
\pgfsetbuttcap%
\pgfsetroundjoin%
\definecolor{currentfill}{rgb}{0.279566,0.067836,0.391917}%
\pgfsetfillcolor{currentfill}%
\pgfsetfillopacity{0.700000}%
\pgfsetlinewidth{0.000000pt}%
\definecolor{currentstroke}{rgb}{0.000000,0.000000,0.000000}%
\pgfsetstrokecolor{currentstroke}%
\pgfsetdash{}{0pt}%
\pgfpathmoveto{\pgfqpoint{3.872599in}{1.519880in}}%
\pgfpathlineto{\pgfqpoint{3.886670in}{1.518739in}}%
\pgfpathlineto{\pgfqpoint{3.900749in}{1.517673in}}%
\pgfpathlineto{\pgfqpoint{3.914836in}{1.516679in}}%
\pgfpathlineto{\pgfqpoint{3.928931in}{1.515758in}}%
\pgfpathlineto{\pgfqpoint{3.937143in}{1.526708in}}%
\pgfpathlineto{\pgfqpoint{3.945350in}{1.537672in}}%
\pgfpathlineto{\pgfqpoint{3.953551in}{1.548644in}}%
\pgfpathlineto{\pgfqpoint{3.961747in}{1.559622in}}%
\pgfpathlineto{\pgfqpoint{3.947662in}{1.560295in}}%
\pgfpathlineto{\pgfqpoint{3.933585in}{1.561040in}}%
\pgfpathlineto{\pgfqpoint{3.919516in}{1.561859in}}%
\pgfpathlineto{\pgfqpoint{3.905455in}{1.562751in}}%
\pgfpathlineto{\pgfqpoint{3.897249in}{1.552013in}}%
\pgfpathlineto{\pgfqpoint{3.889038in}{1.541286in}}%
\pgfpathlineto{\pgfqpoint{3.880821in}{1.530574in}}%
\pgfpathlineto{\pgfqpoint{3.872599in}{1.519880in}}%
\pgfpathclose%
\pgfusepath{fill}%
\end{pgfscope}%
\begin{pgfscope}%
\pgfpathrectangle{\pgfqpoint{1.150000in}{0.150000in}}{\pgfqpoint{5.700000in}{5.700000in}}%
\pgfusepath{clip}%
\pgfsetbuttcap%
\pgfsetroundjoin%
\definecolor{currentfill}{rgb}{0.266580,0.228262,0.514349}%
\pgfsetfillcolor{currentfill}%
\pgfsetfillopacity{0.700000}%
\pgfsetlinewidth{0.000000pt}%
\definecolor{currentstroke}{rgb}{0.000000,0.000000,0.000000}%
\pgfsetstrokecolor{currentstroke}%
\pgfsetdash{}{0pt}%
\pgfpathmoveto{\pgfqpoint{4.496463in}{1.848210in}}%
\pgfpathlineto{\pgfqpoint{4.510742in}{1.850399in}}%
\pgfpathlineto{\pgfqpoint{4.525032in}{1.852660in}}%
\pgfpathlineto{\pgfqpoint{4.539332in}{1.854992in}}%
\pgfpathlineto{\pgfqpoint{4.553643in}{1.857395in}}%
\pgfpathlineto{\pgfqpoint{4.561650in}{1.868410in}}%
\pgfpathlineto{\pgfqpoint{4.569650in}{1.879332in}}%
\pgfpathlineto{\pgfqpoint{4.577644in}{1.890161in}}%
\pgfpathlineto{\pgfqpoint{4.585631in}{1.900896in}}%
\pgfpathlineto{\pgfqpoint{4.571327in}{1.898389in}}%
\pgfpathlineto{\pgfqpoint{4.557033in}{1.895953in}}%
\pgfpathlineto{\pgfqpoint{4.542750in}{1.893589in}}%
\pgfpathlineto{\pgfqpoint{4.528479in}{1.891297in}}%
\pgfpathlineto{\pgfqpoint{4.520484in}{1.880658in}}%
\pgfpathlineto{\pgfqpoint{4.512483in}{1.869929in}}%
\pgfpathlineto{\pgfqpoint{4.504476in}{1.859113in}}%
\pgfpathlineto{\pgfqpoint{4.496463in}{1.848210in}}%
\pgfpathclose%
\pgfusepath{fill}%
\end{pgfscope}%
\begin{pgfscope}%
\pgfpathrectangle{\pgfqpoint{1.150000in}{0.150000in}}{\pgfqpoint{5.700000in}{5.700000in}}%
\pgfusepath{clip}%
\pgfsetbuttcap%
\pgfsetroundjoin%
\definecolor{currentfill}{rgb}{0.272594,0.025563,0.353093}%
\pgfsetfillcolor{currentfill}%
\pgfsetfillopacity{0.700000}%
\pgfsetlinewidth{0.000000pt}%
\definecolor{currentstroke}{rgb}{0.000000,0.000000,0.000000}%
\pgfsetstrokecolor{currentstroke}%
\pgfsetdash{}{0pt}%
\pgfpathmoveto{\pgfqpoint{3.022147in}{1.471713in}}%
\pgfpathlineto{\pgfqpoint{3.036090in}{1.464859in}}%
\pgfpathlineto{\pgfqpoint{3.050037in}{1.458088in}}%
\pgfpathlineto{\pgfqpoint{3.063987in}{1.451400in}}%
\pgfpathlineto{\pgfqpoint{3.077940in}{1.444794in}}%
\pgfpathlineto{\pgfqpoint{3.086564in}{1.449188in}}%
\pgfpathlineto{\pgfqpoint{3.095175in}{1.453787in}}%
\pgfpathlineto{\pgfqpoint{3.103775in}{1.458586in}}%
\pgfpathlineto{\pgfqpoint{3.112363in}{1.463579in}}%
\pgfpathlineto{\pgfqpoint{3.098436in}{1.469792in}}%
\pgfpathlineto{\pgfqpoint{3.084512in}{1.476088in}}%
\pgfpathlineto{\pgfqpoint{3.070592in}{1.482466in}}%
\pgfpathlineto{\pgfqpoint{3.056676in}{1.488928in}}%
\pgfpathlineto{\pgfqpoint{3.048062in}{1.484320in}}%
\pgfpathlineto{\pgfqpoint{3.039436in}{1.479911in}}%
\pgfpathlineto{\pgfqpoint{3.030798in}{1.475707in}}%
\pgfpathlineto{\pgfqpoint{3.022147in}{1.471713in}}%
\pgfpathclose%
\pgfusepath{fill}%
\end{pgfscope}%
\begin{pgfscope}%
\pgfpathrectangle{\pgfqpoint{1.150000in}{0.150000in}}{\pgfqpoint{5.700000in}{5.700000in}}%
\pgfusepath{clip}%
\pgfsetbuttcap%
\pgfsetroundjoin%
\definecolor{currentfill}{rgb}{0.258965,0.251537,0.524736}%
\pgfsetfillcolor{currentfill}%
\pgfsetfillopacity{0.700000}%
\pgfsetlinewidth{0.000000pt}%
\definecolor{currentstroke}{rgb}{0.000000,0.000000,0.000000}%
\pgfsetstrokecolor{currentstroke}%
\pgfsetdash{}{0pt}%
\pgfpathmoveto{\pgfqpoint{4.585631in}{1.900896in}}%
\pgfpathlineto{\pgfqpoint{4.599947in}{1.903474in}}%
\pgfpathlineto{\pgfqpoint{4.614274in}{1.906124in}}%
\pgfpathlineto{\pgfqpoint{4.628611in}{1.908846in}}%
\pgfpathlineto{\pgfqpoint{4.642960in}{1.911638in}}%
\pgfpathlineto{\pgfqpoint{4.650935in}{1.922367in}}%
\pgfpathlineto{\pgfqpoint{4.658903in}{1.932995in}}%
\pgfpathlineto{\pgfqpoint{4.666865in}{1.943519in}}%
\pgfpathlineto{\pgfqpoint{4.674820in}{1.953940in}}%
\pgfpathlineto{\pgfqpoint{4.660478in}{1.951065in}}%
\pgfpathlineto{\pgfqpoint{4.646147in}{1.948262in}}%
\pgfpathlineto{\pgfqpoint{4.631828in}{1.945530in}}%
\pgfpathlineto{\pgfqpoint{4.617520in}{1.942869in}}%
\pgfpathlineto{\pgfqpoint{4.609557in}{1.932522in}}%
\pgfpathlineto{\pgfqpoint{4.601588in}{1.922077in}}%
\pgfpathlineto{\pgfqpoint{4.593613in}{1.911535in}}%
\pgfpathlineto{\pgfqpoint{4.585631in}{1.900896in}}%
\pgfpathclose%
\pgfusepath{fill}%
\end{pgfscope}%
\begin{pgfscope}%
\pgfpathrectangle{\pgfqpoint{1.150000in}{0.150000in}}{\pgfqpoint{5.700000in}{5.700000in}}%
\pgfusepath{clip}%
\pgfsetbuttcap%
\pgfsetroundjoin%
\definecolor{currentfill}{rgb}{0.276022,0.044167,0.370164}%
\pgfsetfillcolor{currentfill}%
\pgfsetfillopacity{0.700000}%
\pgfsetlinewidth{0.000000pt}%
\definecolor{currentstroke}{rgb}{0.000000,0.000000,0.000000}%
\pgfsetstrokecolor{currentstroke}%
\pgfsetdash{}{0pt}%
\pgfpathmoveto{\pgfqpoint{3.783403in}{1.483726in}}%
\pgfpathlineto{\pgfqpoint{3.797454in}{1.482024in}}%
\pgfpathlineto{\pgfqpoint{3.811512in}{1.480395in}}%
\pgfpathlineto{\pgfqpoint{3.825578in}{1.478839in}}%
\pgfpathlineto{\pgfqpoint{3.839652in}{1.477357in}}%
\pgfpathlineto{\pgfqpoint{3.847897in}{1.487942in}}%
\pgfpathlineto{\pgfqpoint{3.856137in}{1.498560in}}%
\pgfpathlineto{\pgfqpoint{3.864371in}{1.509207in}}%
\pgfpathlineto{\pgfqpoint{3.872599in}{1.519880in}}%
\pgfpathlineto{\pgfqpoint{3.858536in}{1.521093in}}%
\pgfpathlineto{\pgfqpoint{3.844481in}{1.522380in}}%
\pgfpathlineto{\pgfqpoint{3.830434in}{1.523740in}}%
\pgfpathlineto{\pgfqpoint{3.816394in}{1.525175in}}%
\pgfpathlineto{\pgfqpoint{3.808155in}{1.514763in}}%
\pgfpathlineto{\pgfqpoint{3.799910in}{1.504382in}}%
\pgfpathlineto{\pgfqpoint{3.791659in}{1.494035in}}%
\pgfpathlineto{\pgfqpoint{3.783403in}{1.483726in}}%
\pgfpathclose%
\pgfusepath{fill}%
\end{pgfscope}%
\begin{pgfscope}%
\pgfpathrectangle{\pgfqpoint{1.150000in}{0.150000in}}{\pgfqpoint{5.700000in}{5.700000in}}%
\pgfusepath{clip}%
\pgfsetbuttcap%
\pgfsetroundjoin%
\definecolor{currentfill}{rgb}{0.250425,0.274290,0.533103}%
\pgfsetfillcolor{currentfill}%
\pgfsetfillopacity{0.700000}%
\pgfsetlinewidth{0.000000pt}%
\definecolor{currentstroke}{rgb}{0.000000,0.000000,0.000000}%
\pgfsetstrokecolor{currentstroke}%
\pgfsetdash{}{0pt}%
\pgfpathmoveto{\pgfqpoint{4.674820in}{1.953940in}}%
\pgfpathlineto{\pgfqpoint{4.689174in}{1.956886in}}%
\pgfpathlineto{\pgfqpoint{4.703538in}{1.959904in}}%
\pgfpathlineto{\pgfqpoint{4.717915in}{1.962993in}}%
\pgfpathlineto{\pgfqpoint{4.732303in}{1.966153in}}%
\pgfpathlineto{\pgfqpoint{4.740244in}{1.976539in}}%
\pgfpathlineto{\pgfqpoint{4.748179in}{1.986814in}}%
\pgfpathlineto{\pgfqpoint{4.756107in}{1.996978in}}%
\pgfpathlineto{\pgfqpoint{4.764028in}{2.007031in}}%
\pgfpathlineto{\pgfqpoint{4.749647in}{2.003810in}}%
\pgfpathlineto{\pgfqpoint{4.735278in}{2.000661in}}%
\pgfpathlineto{\pgfqpoint{4.720921in}{1.997583in}}%
\pgfpathlineto{\pgfqpoint{4.706575in}{1.994576in}}%
\pgfpathlineto{\pgfqpoint{4.698647in}{1.984575in}}%
\pgfpathlineto{\pgfqpoint{4.690711in}{1.974468in}}%
\pgfpathlineto{\pgfqpoint{4.682769in}{1.964257in}}%
\pgfpathlineto{\pgfqpoint{4.674820in}{1.953940in}}%
\pgfpathclose%
\pgfusepath{fill}%
\end{pgfscope}%
\begin{pgfscope}%
\pgfpathrectangle{\pgfqpoint{1.150000in}{0.150000in}}{\pgfqpoint{5.700000in}{5.700000in}}%
\pgfusepath{clip}%
\pgfsetbuttcap%
\pgfsetroundjoin%
\definecolor{currentfill}{rgb}{0.267004,0.004874,0.329415}%
\pgfsetfillcolor{currentfill}%
\pgfsetfillopacity{0.700000}%
\pgfsetlinewidth{0.000000pt}%
\definecolor{currentstroke}{rgb}{0.000000,0.000000,0.000000}%
\pgfsetstrokecolor{currentstroke}%
\pgfsetdash{}{0pt}%
\pgfpathmoveto{\pgfqpoint{3.459300in}{1.416781in}}%
\pgfpathlineto{\pgfqpoint{3.473287in}{1.412951in}}%
\pgfpathlineto{\pgfqpoint{3.487279in}{1.409198in}}%
\pgfpathlineto{\pgfqpoint{3.501278in}{1.405520in}}%
\pgfpathlineto{\pgfqpoint{3.515283in}{1.401919in}}%
\pgfpathlineto{\pgfqpoint{3.523662in}{1.410419in}}%
\pgfpathlineto{\pgfqpoint{3.532034in}{1.419026in}}%
\pgfpathlineto{\pgfqpoint{3.540398in}{1.427736in}}%
\pgfpathlineto{\pgfqpoint{3.548755in}{1.436543in}}%
\pgfpathlineto{\pgfqpoint{3.534767in}{1.439816in}}%
\pgfpathlineto{\pgfqpoint{3.520785in}{1.443164in}}%
\pgfpathlineto{\pgfqpoint{3.506809in}{1.446588in}}%
\pgfpathlineto{\pgfqpoint{3.492838in}{1.450089in}}%
\pgfpathlineto{\pgfqpoint{3.484465in}{1.441603in}}%
\pgfpathlineto{\pgfqpoint{3.476085in}{1.433219in}}%
\pgfpathlineto{\pgfqpoint{3.467696in}{1.424944in}}%
\pgfpathlineto{\pgfqpoint{3.459300in}{1.416781in}}%
\pgfpathclose%
\pgfusepath{fill}%
\end{pgfscope}%
\begin{pgfscope}%
\pgfpathrectangle{\pgfqpoint{1.150000in}{0.150000in}}{\pgfqpoint{5.700000in}{5.700000in}}%
\pgfusepath{clip}%
\pgfsetbuttcap%
\pgfsetroundjoin%
\definecolor{currentfill}{rgb}{0.174274,0.445044,0.557792}%
\pgfsetfillcolor{currentfill}%
\pgfsetfillopacity{0.700000}%
\pgfsetlinewidth{0.000000pt}%
\definecolor{currentstroke}{rgb}{0.000000,0.000000,0.000000}%
\pgfsetstrokecolor{currentstroke}%
\pgfsetdash{}{0pt}%
\pgfpathmoveto{\pgfqpoint{5.477567in}{2.403022in}}%
\pgfpathlineto{\pgfqpoint{5.492280in}{2.408288in}}%
\pgfpathlineto{\pgfqpoint{5.507006in}{2.413625in}}%
\pgfpathlineto{\pgfqpoint{5.521747in}{2.419033in}}%
\pgfpathlineto{\pgfqpoint{5.529291in}{2.424620in}}%
\pgfpathlineto{\pgfqpoint{5.536825in}{2.430086in}}%
\pgfpathlineto{\pgfqpoint{5.544349in}{2.435433in}}%
\pgfpathlineto{\pgfqpoint{5.551864in}{2.440663in}}%
\pgfpathlineto{\pgfqpoint{5.537141in}{2.435394in}}%
\pgfpathlineto{\pgfqpoint{5.522431in}{2.430195in}}%
\pgfpathlineto{\pgfqpoint{5.507735in}{2.425067in}}%
\pgfpathlineto{\pgfqpoint{5.500208in}{2.419727in}}%
\pgfpathlineto{\pgfqpoint{5.492670in}{2.414275in}}%
\pgfpathlineto{\pgfqpoint{5.485124in}{2.408707in}}%
\pgfpathlineto{\pgfqpoint{5.477567in}{2.403022in}}%
\pgfpathclose%
\pgfusepath{fill}%
\end{pgfscope}%
\begin{pgfscope}%
\pgfpathrectangle{\pgfqpoint{1.150000in}{0.150000in}}{\pgfqpoint{5.700000in}{5.700000in}}%
\pgfusepath{clip}%
\pgfsetbuttcap%
\pgfsetroundjoin%
\definecolor{currentfill}{rgb}{0.241237,0.296485,0.539709}%
\pgfsetfillcolor{currentfill}%
\pgfsetfillopacity{0.700000}%
\pgfsetlinewidth{0.000000pt}%
\definecolor{currentstroke}{rgb}{0.000000,0.000000,0.000000}%
\pgfsetstrokecolor{currentstroke}%
\pgfsetdash{}{0pt}%
\pgfpathmoveto{\pgfqpoint{4.764028in}{2.007031in}}%
\pgfpathlineto{\pgfqpoint{4.778420in}{2.010324in}}%
\pgfpathlineto{\pgfqpoint{4.792824in}{2.013687in}}%
\pgfpathlineto{\pgfqpoint{4.807239in}{2.017122in}}%
\pgfpathlineto{\pgfqpoint{4.821667in}{2.020628in}}%
\pgfpathlineto{\pgfqpoint{4.829573in}{2.030617in}}%
\pgfpathlineto{\pgfqpoint{4.837473in}{2.040489in}}%
\pgfpathlineto{\pgfqpoint{4.845365in}{2.050243in}}%
\pgfpathlineto{\pgfqpoint{4.853250in}{2.059879in}}%
\pgfpathlineto{\pgfqpoint{4.838830in}{2.056334in}}%
\pgfpathlineto{\pgfqpoint{4.824422in}{2.052860in}}%
\pgfpathlineto{\pgfqpoint{4.810026in}{2.049458in}}%
\pgfpathlineto{\pgfqpoint{4.795642in}{2.046126in}}%
\pgfpathlineto{\pgfqpoint{4.787749in}{2.036521in}}%
\pgfpathlineto{\pgfqpoint{4.779849in}{2.026803in}}%
\pgfpathlineto{\pgfqpoint{4.771942in}{2.016973in}}%
\pgfpathlineto{\pgfqpoint{4.764028in}{2.007031in}}%
\pgfpathclose%
\pgfusepath{fill}%
\end{pgfscope}%
\begin{pgfscope}%
\pgfpathrectangle{\pgfqpoint{1.150000in}{0.150000in}}{\pgfqpoint{5.700000in}{5.700000in}}%
\pgfusepath{clip}%
\pgfsetbuttcap%
\pgfsetroundjoin%
\definecolor{currentfill}{rgb}{0.231674,0.318106,0.544834}%
\pgfsetfillcolor{currentfill}%
\pgfsetfillopacity{0.700000}%
\pgfsetlinewidth{0.000000pt}%
\definecolor{currentstroke}{rgb}{0.000000,0.000000,0.000000}%
\pgfsetstrokecolor{currentstroke}%
\pgfsetdash{}{0pt}%
\pgfpathmoveto{\pgfqpoint{4.853250in}{2.059879in}}%
\pgfpathlineto{\pgfqpoint{4.867681in}{2.063496in}}%
\pgfpathlineto{\pgfqpoint{4.882125in}{2.067184in}}%
\pgfpathlineto{\pgfqpoint{4.896581in}{2.070943in}}%
\pgfpathlineto{\pgfqpoint{4.911048in}{2.074773in}}%
\pgfpathlineto{\pgfqpoint{4.918918in}{2.084318in}}%
\pgfpathlineto{\pgfqpoint{4.926779in}{2.093739in}}%
\pgfpathlineto{\pgfqpoint{4.934634in}{2.103038in}}%
\pgfpathlineto{\pgfqpoint{4.942480in}{2.112215in}}%
\pgfpathlineto{\pgfqpoint{4.928021in}{2.108367in}}%
\pgfpathlineto{\pgfqpoint{4.913574in}{2.104591in}}%
\pgfpathlineto{\pgfqpoint{4.899139in}{2.100886in}}%
\pgfpathlineto{\pgfqpoint{4.884716in}{2.097252in}}%
\pgfpathlineto{\pgfqpoint{4.876860in}{2.088085in}}%
\pgfpathlineto{\pgfqpoint{4.868997in}{2.078800in}}%
\pgfpathlineto{\pgfqpoint{4.861127in}{2.069399in}}%
\pgfpathlineto{\pgfqpoint{4.853250in}{2.059879in}}%
\pgfpathclose%
\pgfusepath{fill}%
\end{pgfscope}%
\begin{pgfscope}%
\pgfpathrectangle{\pgfqpoint{1.150000in}{0.150000in}}{\pgfqpoint{5.700000in}{5.700000in}}%
\pgfusepath{clip}%
\pgfsetbuttcap%
\pgfsetroundjoin%
\definecolor{currentfill}{rgb}{0.273809,0.031497,0.358853}%
\pgfsetfillcolor{currentfill}%
\pgfsetfillopacity{0.700000}%
\pgfsetlinewidth{0.000000pt}%
\definecolor{currentstroke}{rgb}{0.000000,0.000000,0.000000}%
\pgfsetstrokecolor{currentstroke}%
\pgfsetdash{}{0pt}%
\pgfpathmoveto{\pgfqpoint{3.694136in}{1.451663in}}%
\pgfpathlineto{\pgfqpoint{3.708170in}{1.449375in}}%
\pgfpathlineto{\pgfqpoint{3.722211in}{1.447162in}}%
\pgfpathlineto{\pgfqpoint{3.736259in}{1.445023in}}%
\pgfpathlineto{\pgfqpoint{3.750315in}{1.442957in}}%
\pgfpathlineto{\pgfqpoint{3.758596in}{1.453072in}}%
\pgfpathlineto{\pgfqpoint{3.766871in}{1.463241in}}%
\pgfpathlineto{\pgfqpoint{3.775140in}{1.473460in}}%
\pgfpathlineto{\pgfqpoint{3.783403in}{1.483726in}}%
\pgfpathlineto{\pgfqpoint{3.769359in}{1.485503in}}%
\pgfpathlineto{\pgfqpoint{3.755323in}{1.487353in}}%
\pgfpathlineto{\pgfqpoint{3.741295in}{1.489278in}}%
\pgfpathlineto{\pgfqpoint{3.727273in}{1.491277in}}%
\pgfpathlineto{\pgfqpoint{3.718999in}{1.481292in}}%
\pgfpathlineto{\pgfqpoint{3.710717in}{1.471358in}}%
\pgfpathlineto{\pgfqpoint{3.702430in}{1.461480in}}%
\pgfpathlineto{\pgfqpoint{3.694136in}{1.451663in}}%
\pgfpathclose%
\pgfusepath{fill}%
\end{pgfscope}%
\begin{pgfscope}%
\pgfpathrectangle{\pgfqpoint{1.150000in}{0.150000in}}{\pgfqpoint{5.700000in}{5.700000in}}%
\pgfusepath{clip}%
\pgfsetbuttcap%
\pgfsetroundjoin%
\definecolor{currentfill}{rgb}{0.221989,0.339161,0.548752}%
\pgfsetfillcolor{currentfill}%
\pgfsetfillopacity{0.700000}%
\pgfsetlinewidth{0.000000pt}%
\definecolor{currentstroke}{rgb}{0.000000,0.000000,0.000000}%
\pgfsetstrokecolor{currentstroke}%
\pgfsetdash{}{0pt}%
\pgfpathmoveto{\pgfqpoint{4.942480in}{2.112215in}}%
\pgfpathlineto{\pgfqpoint{4.956952in}{2.116134in}}%
\pgfpathlineto{\pgfqpoint{4.971436in}{2.120124in}}%
\pgfpathlineto{\pgfqpoint{4.985932in}{2.124185in}}%
\pgfpathlineto{\pgfqpoint{5.000441in}{2.128318in}}%
\pgfpathlineto{\pgfqpoint{5.008270in}{2.137375in}}%
\pgfpathlineto{\pgfqpoint{5.016092in}{2.146306in}}%
\pgfpathlineto{\pgfqpoint{5.023906in}{2.155110in}}%
\pgfpathlineto{\pgfqpoint{5.031712in}{2.163788in}}%
\pgfpathlineto{\pgfqpoint{5.017213in}{2.159661in}}%
\pgfpathlineto{\pgfqpoint{5.002726in}{2.155604in}}%
\pgfpathlineto{\pgfqpoint{4.988252in}{2.151619in}}%
\pgfpathlineto{\pgfqpoint{4.973789in}{2.147704in}}%
\pgfpathlineto{\pgfqpoint{4.965974in}{2.139013in}}%
\pgfpathlineto{\pgfqpoint{4.958150in}{2.130202in}}%
\pgfpathlineto{\pgfqpoint{4.950319in}{2.121269in}}%
\pgfpathlineto{\pgfqpoint{4.942480in}{2.112215in}}%
\pgfpathclose%
\pgfusepath{fill}%
\end{pgfscope}%
\begin{pgfscope}%
\pgfpathrectangle{\pgfqpoint{1.150000in}{0.150000in}}{\pgfqpoint{5.700000in}{5.700000in}}%
\pgfusepath{clip}%
\pgfsetbuttcap%
\pgfsetroundjoin%
\definecolor{currentfill}{rgb}{0.277941,0.056324,0.381191}%
\pgfsetfillcolor{currentfill}%
\pgfsetfillopacity{0.700000}%
\pgfsetlinewidth{0.000000pt}%
\definecolor{currentstroke}{rgb}{0.000000,0.000000,0.000000}%
\pgfsetstrokecolor{currentstroke}%
\pgfsetdash{}{0pt}%
\pgfpathmoveto{\pgfqpoint{2.875745in}{1.519167in}}%
\pgfpathlineto{\pgfqpoint{2.889694in}{1.511218in}}%
\pgfpathlineto{\pgfqpoint{2.903646in}{1.503356in}}%
\pgfpathlineto{\pgfqpoint{2.917601in}{1.495580in}}%
\pgfpathlineto{\pgfqpoint{2.931558in}{1.487889in}}%
\pgfpathlineto{\pgfqpoint{2.940291in}{1.490553in}}%
\pgfpathlineto{\pgfqpoint{2.949010in}{1.493459in}}%
\pgfpathlineto{\pgfqpoint{2.957715in}{1.496600in}}%
\pgfpathlineto{\pgfqpoint{2.966407in}{1.499970in}}%
\pgfpathlineto{\pgfqpoint{2.952480in}{1.507246in}}%
\pgfpathlineto{\pgfqpoint{2.938555in}{1.514608in}}%
\pgfpathlineto{\pgfqpoint{2.924634in}{1.522055in}}%
\pgfpathlineto{\pgfqpoint{2.910715in}{1.529589in}}%
\pgfpathlineto{\pgfqpoint{2.901994in}{1.526626in}}%
\pgfpathlineto{\pgfqpoint{2.893258in}{1.523897in}}%
\pgfpathlineto{\pgfqpoint{2.884509in}{1.521408in}}%
\pgfpathlineto{\pgfqpoint{2.875745in}{1.519167in}}%
\pgfpathclose%
\pgfusepath{fill}%
\end{pgfscope}%
\begin{pgfscope}%
\pgfpathrectangle{\pgfqpoint{1.150000in}{0.150000in}}{\pgfqpoint{5.700000in}{5.700000in}}%
\pgfusepath{clip}%
\pgfsetbuttcap%
\pgfsetroundjoin%
\definecolor{currentfill}{rgb}{0.212395,0.359683,0.551710}%
\pgfsetfillcolor{currentfill}%
\pgfsetfillopacity{0.700000}%
\pgfsetlinewidth{0.000000pt}%
\definecolor{currentstroke}{rgb}{0.000000,0.000000,0.000000}%
\pgfsetstrokecolor{currentstroke}%
\pgfsetdash{}{0pt}%
\pgfpathmoveto{\pgfqpoint{5.031712in}{2.163788in}}%
\pgfpathlineto{\pgfqpoint{5.046224in}{2.167987in}}%
\pgfpathlineto{\pgfqpoint{5.060748in}{2.172258in}}%
\pgfpathlineto{\pgfqpoint{5.075286in}{2.176599in}}%
\pgfpathlineto{\pgfqpoint{5.089836in}{2.181012in}}%
\pgfpathlineto{\pgfqpoint{5.097624in}{2.189546in}}%
\pgfpathlineto{\pgfqpoint{5.105403in}{2.197951in}}%
\pgfpathlineto{\pgfqpoint{5.113174in}{2.206226in}}%
\pgfpathlineto{\pgfqpoint{5.120937in}{2.214373in}}%
\pgfpathlineto{\pgfqpoint{5.106398in}{2.209987in}}%
\pgfpathlineto{\pgfqpoint{5.091871in}{2.205672in}}%
\pgfpathlineto{\pgfqpoint{5.077357in}{2.201429in}}%
\pgfpathlineto{\pgfqpoint{5.062855in}{2.197256in}}%
\pgfpathlineto{\pgfqpoint{5.055082in}{2.189074in}}%
\pgfpathlineto{\pgfqpoint{5.047300in}{2.180770in}}%
\pgfpathlineto{\pgfqpoint{5.039510in}{2.172341in}}%
\pgfpathlineto{\pgfqpoint{5.031712in}{2.163788in}}%
\pgfpathclose%
\pgfusepath{fill}%
\end{pgfscope}%
\begin{pgfscope}%
\pgfpathrectangle{\pgfqpoint{1.150000in}{0.150000in}}{\pgfqpoint{5.700000in}{5.700000in}}%
\pgfusepath{clip}%
\pgfsetbuttcap%
\pgfsetroundjoin%
\definecolor{currentfill}{rgb}{0.179019,0.433756,0.557430}%
\pgfsetfillcolor{currentfill}%
\pgfsetfillopacity{0.700000}%
\pgfsetlinewidth{0.000000pt}%
\definecolor{currentstroke}{rgb}{0.000000,0.000000,0.000000}%
\pgfsetstrokecolor{currentstroke}%
\pgfsetdash{}{0pt}%
\pgfpathmoveto{\pgfqpoint{5.388472in}{2.358239in}}%
\pgfpathlineto{\pgfqpoint{5.403145in}{2.363336in}}%
\pgfpathlineto{\pgfqpoint{5.417832in}{2.368504in}}%
\pgfpathlineto{\pgfqpoint{5.432533in}{2.373744in}}%
\pgfpathlineto{\pgfqpoint{5.447247in}{2.379055in}}%
\pgfpathlineto{\pgfqpoint{5.454841in}{2.385235in}}%
\pgfpathlineto{\pgfqpoint{5.462426in}{2.391289in}}%
\pgfpathlineto{\pgfqpoint{5.470002in}{2.397217in}}%
\pgfpathlineto{\pgfqpoint{5.477567in}{2.403022in}}%
\pgfpathlineto{\pgfqpoint{5.462869in}{2.397827in}}%
\pgfpathlineto{\pgfqpoint{5.448184in}{2.392703in}}%
\pgfpathlineto{\pgfqpoint{5.433512in}{2.387651in}}%
\pgfpathlineto{\pgfqpoint{5.418854in}{2.382669in}}%
\pgfpathlineto{\pgfqpoint{5.411273in}{2.376739in}}%
\pgfpathlineto{\pgfqpoint{5.403682in}{2.370693in}}%
\pgfpathlineto{\pgfqpoint{5.396082in}{2.364527in}}%
\pgfpathlineto{\pgfqpoint{5.388472in}{2.358239in}}%
\pgfpathclose%
\pgfusepath{fill}%
\end{pgfscope}%
\begin{pgfscope}%
\pgfpathrectangle{\pgfqpoint{1.150000in}{0.150000in}}{\pgfqpoint{5.700000in}{5.700000in}}%
\pgfusepath{clip}%
\pgfsetbuttcap%
\pgfsetroundjoin%
\definecolor{currentfill}{rgb}{0.203063,0.379716,0.553925}%
\pgfsetfillcolor{currentfill}%
\pgfsetfillopacity{0.700000}%
\pgfsetlinewidth{0.000000pt}%
\definecolor{currentstroke}{rgb}{0.000000,0.000000,0.000000}%
\pgfsetstrokecolor{currentstroke}%
\pgfsetdash{}{0pt}%
\pgfpathmoveto{\pgfqpoint{5.120937in}{2.214373in}}%
\pgfpathlineto{\pgfqpoint{5.135490in}{2.218830in}}%
\pgfpathlineto{\pgfqpoint{5.150055in}{2.223358in}}%
\pgfpathlineto{\pgfqpoint{5.164633in}{2.227958in}}%
\pgfpathlineto{\pgfqpoint{5.179224in}{2.232629in}}%
\pgfpathlineto{\pgfqpoint{5.186968in}{2.240608in}}%
\pgfpathlineto{\pgfqpoint{5.194702in}{2.248456in}}%
\pgfpathlineto{\pgfqpoint{5.202429in}{2.256174in}}%
\pgfpathlineto{\pgfqpoint{5.210146in}{2.263762in}}%
\pgfpathlineto{\pgfqpoint{5.195566in}{2.259140in}}%
\pgfpathlineto{\pgfqpoint{5.180999in}{2.254590in}}%
\pgfpathlineto{\pgfqpoint{5.166446in}{2.250110in}}%
\pgfpathlineto{\pgfqpoint{5.151905in}{2.245702in}}%
\pgfpathlineto{\pgfqpoint{5.144175in}{2.238056in}}%
\pgfpathlineto{\pgfqpoint{5.136438in}{2.230287in}}%
\pgfpathlineto{\pgfqpoint{5.128692in}{2.222393in}}%
\pgfpathlineto{\pgfqpoint{5.120937in}{2.214373in}}%
\pgfpathclose%
\pgfusepath{fill}%
\end{pgfscope}%
\begin{pgfscope}%
\pgfpathrectangle{\pgfqpoint{1.150000in}{0.150000in}}{\pgfqpoint{5.700000in}{5.700000in}}%
\pgfusepath{clip}%
\pgfsetbuttcap%
\pgfsetroundjoin%
\definecolor{currentfill}{rgb}{0.187231,0.414746,0.556547}%
\pgfsetfillcolor{currentfill}%
\pgfsetfillopacity{0.700000}%
\pgfsetlinewidth{0.000000pt}%
\definecolor{currentstroke}{rgb}{0.000000,0.000000,0.000000}%
\pgfsetstrokecolor{currentstroke}%
\pgfsetdash{}{0pt}%
\pgfpathmoveto{\pgfqpoint{5.299328in}{2.311772in}}%
\pgfpathlineto{\pgfqpoint{5.313961in}{2.316678in}}%
\pgfpathlineto{\pgfqpoint{5.328608in}{2.321656in}}%
\pgfpathlineto{\pgfqpoint{5.343268in}{2.326704in}}%
\pgfpathlineto{\pgfqpoint{5.357941in}{2.331824in}}%
\pgfpathlineto{\pgfqpoint{5.365588in}{2.338621in}}%
\pgfpathlineto{\pgfqpoint{5.373225in}{2.345288in}}%
\pgfpathlineto{\pgfqpoint{5.380853in}{2.351826in}}%
\pgfpathlineto{\pgfqpoint{5.388472in}{2.358239in}}%
\pgfpathlineto{\pgfqpoint{5.373813in}{2.353212in}}%
\pgfpathlineto{\pgfqpoint{5.359167in}{2.348257in}}%
\pgfpathlineto{\pgfqpoint{5.344534in}{2.343373in}}%
\pgfpathlineto{\pgfqpoint{5.329915in}{2.338559in}}%
\pgfpathlineto{\pgfqpoint{5.322282in}{2.332046in}}%
\pgfpathlineto{\pgfqpoint{5.314639in}{2.325411in}}%
\pgfpathlineto{\pgfqpoint{5.306988in}{2.318654in}}%
\pgfpathlineto{\pgfqpoint{5.299328in}{2.311772in}}%
\pgfpathclose%
\pgfusepath{fill}%
\end{pgfscope}%
\begin{pgfscope}%
\pgfpathrectangle{\pgfqpoint{1.150000in}{0.150000in}}{\pgfqpoint{5.700000in}{5.700000in}}%
\pgfusepath{clip}%
\pgfsetbuttcap%
\pgfsetroundjoin%
\definecolor{currentfill}{rgb}{0.194100,0.399323,0.555565}%
\pgfsetfillcolor{currentfill}%
\pgfsetfillopacity{0.700000}%
\pgfsetlinewidth{0.000000pt}%
\definecolor{currentstroke}{rgb}{0.000000,0.000000,0.000000}%
\pgfsetstrokecolor{currentstroke}%
\pgfsetdash{}{0pt}%
\pgfpathmoveto{\pgfqpoint{5.210146in}{2.263762in}}%
\pgfpathlineto{\pgfqpoint{5.224739in}{2.268455in}}%
\pgfpathlineto{\pgfqpoint{5.239345in}{2.273219in}}%
\pgfpathlineto{\pgfqpoint{5.253964in}{2.278055in}}%
\pgfpathlineto{\pgfqpoint{5.268597in}{2.282961in}}%
\pgfpathlineto{\pgfqpoint{5.276293in}{2.290360in}}%
\pgfpathlineto{\pgfqpoint{5.283980in}{2.297626in}}%
\pgfpathlineto{\pgfqpoint{5.291659in}{2.304763in}}%
\pgfpathlineto{\pgfqpoint{5.299328in}{2.311772in}}%
\pgfpathlineto{\pgfqpoint{5.284708in}{2.306936in}}%
\pgfpathlineto{\pgfqpoint{5.270102in}{2.302172in}}%
\pgfpathlineto{\pgfqpoint{5.255508in}{2.297479in}}%
\pgfpathlineto{\pgfqpoint{5.240928in}{2.292857in}}%
\pgfpathlineto{\pgfqpoint{5.233246in}{2.285769in}}%
\pgfpathlineto{\pgfqpoint{5.225554in}{2.278558in}}%
\pgfpathlineto{\pgfqpoint{5.217854in}{2.271223in}}%
\pgfpathlineto{\pgfqpoint{5.210146in}{2.263762in}}%
\pgfpathclose%
\pgfusepath{fill}%
\end{pgfscope}%
\begin{pgfscope}%
\pgfpathrectangle{\pgfqpoint{1.150000in}{0.150000in}}{\pgfqpoint{5.700000in}{5.700000in}}%
\pgfusepath{clip}%
\pgfsetbuttcap%
\pgfsetroundjoin%
\definecolor{currentfill}{rgb}{0.267004,0.004874,0.329415}%
\pgfsetfillcolor{currentfill}%
\pgfsetfillopacity{0.700000}%
\pgfsetlinewidth{0.000000pt}%
\definecolor{currentstroke}{rgb}{0.000000,0.000000,0.000000}%
\pgfsetstrokecolor{currentstroke}%
\pgfsetdash{}{0pt}%
\pgfpathmoveto{\pgfqpoint{3.223929in}{1.416802in}}%
\pgfpathlineto{\pgfqpoint{3.237894in}{1.411317in}}%
\pgfpathlineto{\pgfqpoint{3.251864in}{1.405911in}}%
\pgfpathlineto{\pgfqpoint{3.265838in}{1.400583in}}%
\pgfpathlineto{\pgfqpoint{3.279817in}{1.395335in}}%
\pgfpathlineto{\pgfqpoint{3.288324in}{1.401651in}}%
\pgfpathlineto{\pgfqpoint{3.296821in}{1.408133in}}%
\pgfpathlineto{\pgfqpoint{3.305308in}{1.414775in}}%
\pgfpathlineto{\pgfqpoint{3.313786in}{1.421572in}}%
\pgfpathlineto{\pgfqpoint{3.299829in}{1.426450in}}%
\pgfpathlineto{\pgfqpoint{3.285876in}{1.431406in}}%
\pgfpathlineto{\pgfqpoint{3.271929in}{1.436442in}}%
\pgfpathlineto{\pgfqpoint{3.257986in}{1.441556in}}%
\pgfpathlineto{\pgfqpoint{3.249487in}{1.435122in}}%
\pgfpathlineto{\pgfqpoint{3.240977in}{1.428848in}}%
\pgfpathlineto{\pgfqpoint{3.232458in}{1.422740in}}%
\pgfpathlineto{\pgfqpoint{3.223929in}{1.416802in}}%
\pgfpathclose%
\pgfusepath{fill}%
\end{pgfscope}%
\begin{pgfscope}%
\pgfpathrectangle{\pgfqpoint{1.150000in}{0.150000in}}{\pgfqpoint{5.700000in}{5.700000in}}%
\pgfusepath{clip}%
\pgfsetbuttcap%
\pgfsetroundjoin%
\definecolor{currentfill}{rgb}{0.271305,0.019942,0.347269}%
\pgfsetfillcolor{currentfill}%
\pgfsetfillopacity{0.700000}%
\pgfsetlinewidth{0.000000pt}%
\definecolor{currentstroke}{rgb}{0.000000,0.000000,0.000000}%
\pgfsetstrokecolor{currentstroke}%
\pgfsetdash{}{0pt}%
\pgfpathmoveto{\pgfqpoint{3.077940in}{1.444794in}}%
\pgfpathlineto{\pgfqpoint{3.091897in}{1.438271in}}%
\pgfpathlineto{\pgfqpoint{3.105858in}{1.431829in}}%
\pgfpathlineto{\pgfqpoint{3.119823in}{1.425469in}}%
\pgfpathlineto{\pgfqpoint{3.133791in}{1.419190in}}%
\pgfpathlineto{\pgfqpoint{3.142389in}{1.423984in}}%
\pgfpathlineto{\pgfqpoint{3.150975in}{1.428979in}}%
\pgfpathlineto{\pgfqpoint{3.159549in}{1.434167in}}%
\pgfpathlineto{\pgfqpoint{3.168113in}{1.439544in}}%
\pgfpathlineto{\pgfqpoint{3.154169in}{1.445430in}}%
\pgfpathlineto{\pgfqpoint{3.140230in}{1.451398in}}%
\pgfpathlineto{\pgfqpoint{3.126295in}{1.457447in}}%
\pgfpathlineto{\pgfqpoint{3.112363in}{1.463579in}}%
\pgfpathlineto{\pgfqpoint{3.103775in}{1.458586in}}%
\pgfpathlineto{\pgfqpoint{3.095175in}{1.453787in}}%
\pgfpathlineto{\pgfqpoint{3.086564in}{1.449188in}}%
\pgfpathlineto{\pgfqpoint{3.077940in}{1.444794in}}%
\pgfpathclose%
\pgfusepath{fill}%
\end{pgfscope}%
\begin{pgfscope}%
\pgfpathrectangle{\pgfqpoint{1.150000in}{0.150000in}}{\pgfqpoint{5.700000in}{5.700000in}}%
\pgfusepath{clip}%
\pgfsetbuttcap%
\pgfsetroundjoin%
\definecolor{currentfill}{rgb}{0.269944,0.014625,0.341379}%
\pgfsetfillcolor{currentfill}%
\pgfsetfillopacity{0.700000}%
\pgfsetlinewidth{0.000000pt}%
\definecolor{currentstroke}{rgb}{0.000000,0.000000,0.000000}%
\pgfsetstrokecolor{currentstroke}%
\pgfsetdash{}{0pt}%
\pgfpathmoveto{\pgfqpoint{3.604773in}{1.424211in}}%
\pgfpathlineto{\pgfqpoint{3.618793in}{1.421316in}}%
\pgfpathlineto{\pgfqpoint{3.632821in}{1.418496in}}%
\pgfpathlineto{\pgfqpoint{3.646855in}{1.415750in}}%
\pgfpathlineto{\pgfqpoint{3.660895in}{1.413079in}}%
\pgfpathlineto{\pgfqpoint{3.669215in}{1.422613in}}%
\pgfpathlineto{\pgfqpoint{3.677529in}{1.432224in}}%
\pgfpathlineto{\pgfqpoint{3.685836in}{1.441909in}}%
\pgfpathlineto{\pgfqpoint{3.694136in}{1.451663in}}%
\pgfpathlineto{\pgfqpoint{3.680109in}{1.454024in}}%
\pgfpathlineto{\pgfqpoint{3.666089in}{1.456461in}}%
\pgfpathlineto{\pgfqpoint{3.652076in}{1.458972in}}%
\pgfpathlineto{\pgfqpoint{3.638070in}{1.461558in}}%
\pgfpathlineto{\pgfqpoint{3.629756in}{1.452106in}}%
\pgfpathlineto{\pgfqpoint{3.621435in}{1.442728in}}%
\pgfpathlineto{\pgfqpoint{3.613107in}{1.433428in}}%
\pgfpathlineto{\pgfqpoint{3.604773in}{1.424211in}}%
\pgfpathclose%
\pgfusepath{fill}%
\end{pgfscope}%
\begin{pgfscope}%
\pgfpathrectangle{\pgfqpoint{1.150000in}{0.150000in}}{\pgfqpoint{5.700000in}{5.700000in}}%
\pgfusepath{clip}%
\pgfsetbuttcap%
\pgfsetroundjoin%
\definecolor{currentfill}{rgb}{0.267004,0.004874,0.329415}%
\pgfsetfillcolor{currentfill}%
\pgfsetfillopacity{0.700000}%
\pgfsetlinewidth{0.000000pt}%
\definecolor{currentstroke}{rgb}{0.000000,0.000000,0.000000}%
\pgfsetstrokecolor{currentstroke}%
\pgfsetdash{}{0pt}%
\pgfpathmoveto{\pgfqpoint{3.369667in}{1.402845in}}%
\pgfpathlineto{\pgfqpoint{3.383650in}{1.398357in}}%
\pgfpathlineto{\pgfqpoint{3.397639in}{1.393947in}}%
\pgfpathlineto{\pgfqpoint{3.411633in}{1.389614in}}%
\pgfpathlineto{\pgfqpoint{3.425632in}{1.385357in}}%
\pgfpathlineto{\pgfqpoint{3.434062in}{1.393019in}}%
\pgfpathlineto{\pgfqpoint{3.442483in}{1.400813in}}%
\pgfpathlineto{\pgfqpoint{3.450895in}{1.408736in}}%
\pgfpathlineto{\pgfqpoint{3.459300in}{1.416781in}}%
\pgfpathlineto{\pgfqpoint{3.445318in}{1.420688in}}%
\pgfpathlineto{\pgfqpoint{3.431343in}{1.424671in}}%
\pgfpathlineto{\pgfqpoint{3.417373in}{1.428731in}}%
\pgfpathlineto{\pgfqpoint{3.403409in}{1.432869in}}%
\pgfpathlineto{\pgfqpoint{3.394986in}{1.425166in}}%
\pgfpathlineto{\pgfqpoint{3.386555in}{1.417590in}}%
\pgfpathlineto{\pgfqpoint{3.378115in}{1.410148in}}%
\pgfpathlineto{\pgfqpoint{3.369667in}{1.402845in}}%
\pgfpathclose%
\pgfusepath{fill}%
\end{pgfscope}%
\begin{pgfscope}%
\pgfpathrectangle{\pgfqpoint{1.150000in}{0.150000in}}{\pgfqpoint{5.700000in}{5.700000in}}%
\pgfusepath{clip}%
\pgfsetbuttcap%
\pgfsetroundjoin%
\definecolor{currentfill}{rgb}{0.283229,0.120777,0.440584}%
\pgfsetfillcolor{currentfill}%
\pgfsetfillopacity{0.700000}%
\pgfsetlinewidth{0.000000pt}%
\definecolor{currentstroke}{rgb}{0.000000,0.000000,0.000000}%
\pgfsetstrokecolor{currentstroke}%
\pgfsetdash{}{0pt}%
\pgfpathmoveto{\pgfqpoint{4.107397in}{1.602589in}}%
\pgfpathlineto{\pgfqpoint{4.121552in}{1.602797in}}%
\pgfpathlineto{\pgfqpoint{4.135716in}{1.603078in}}%
\pgfpathlineto{\pgfqpoint{4.149889in}{1.603431in}}%
\pgfpathlineto{\pgfqpoint{4.164072in}{1.603856in}}%
\pgfpathlineto{\pgfqpoint{4.172216in}{1.615460in}}%
\pgfpathlineto{\pgfqpoint{4.180355in}{1.627035in}}%
\pgfpathlineto{\pgfqpoint{4.188488in}{1.638578in}}%
\pgfpathlineto{\pgfqpoint{4.196617in}{1.650086in}}%
\pgfpathlineto{\pgfqpoint{4.182441in}{1.649453in}}%
\pgfpathlineto{\pgfqpoint{4.168276in}{1.648892in}}%
\pgfpathlineto{\pgfqpoint{4.154119in}{1.648404in}}%
\pgfpathlineto{\pgfqpoint{4.139972in}{1.647988in}}%
\pgfpathlineto{\pgfqpoint{4.131836in}{1.636679in}}%
\pgfpathlineto{\pgfqpoint{4.123695in}{1.625341in}}%
\pgfpathlineto{\pgfqpoint{4.115549in}{1.613977in}}%
\pgfpathlineto{\pgfqpoint{4.107397in}{1.602589in}}%
\pgfpathclose%
\pgfusepath{fill}%
\end{pgfscope}%
\begin{pgfscope}%
\pgfpathrectangle{\pgfqpoint{1.150000in}{0.150000in}}{\pgfqpoint{5.700000in}{5.700000in}}%
\pgfusepath{clip}%
\pgfsetbuttcap%
\pgfsetroundjoin%
\definecolor{currentfill}{rgb}{0.282623,0.140926,0.457517}%
\pgfsetfillcolor{currentfill}%
\pgfsetfillopacity{0.700000}%
\pgfsetlinewidth{0.000000pt}%
\definecolor{currentstroke}{rgb}{0.000000,0.000000,0.000000}%
\pgfsetstrokecolor{currentstroke}%
\pgfsetdash{}{0pt}%
\pgfpathmoveto{\pgfqpoint{4.196617in}{1.650086in}}%
\pgfpathlineto{\pgfqpoint{4.210801in}{1.650791in}}%
\pgfpathlineto{\pgfqpoint{4.224995in}{1.651568in}}%
\pgfpathlineto{\pgfqpoint{4.239199in}{1.652416in}}%
\pgfpathlineto{\pgfqpoint{4.253413in}{1.653337in}}%
\pgfpathlineto{\pgfqpoint{4.261529in}{1.665002in}}%
\pgfpathlineto{\pgfqpoint{4.269639in}{1.676623in}}%
\pgfpathlineto{\pgfqpoint{4.277744in}{1.688196in}}%
\pgfpathlineto{\pgfqpoint{4.285844in}{1.699719in}}%
\pgfpathlineto{\pgfqpoint{4.271637in}{1.698611in}}%
\pgfpathlineto{\pgfqpoint{4.257441in}{1.697575in}}%
\pgfpathlineto{\pgfqpoint{4.243253in}{1.696612in}}%
\pgfpathlineto{\pgfqpoint{4.229076in}{1.695720in}}%
\pgfpathlineto{\pgfqpoint{4.220969in}{1.684376in}}%
\pgfpathlineto{\pgfqpoint{4.212857in}{1.672988in}}%
\pgfpathlineto{\pgfqpoint{4.204740in}{1.661557in}}%
\pgfpathlineto{\pgfqpoint{4.196617in}{1.650086in}}%
\pgfpathclose%
\pgfusepath{fill}%
\end{pgfscope}%
\begin{pgfscope}%
\pgfpathrectangle{\pgfqpoint{1.150000in}{0.150000in}}{\pgfqpoint{5.700000in}{5.700000in}}%
\pgfusepath{clip}%
\pgfsetbuttcap%
\pgfsetroundjoin%
\definecolor{currentfill}{rgb}{0.282327,0.094955,0.417331}%
\pgfsetfillcolor{currentfill}%
\pgfsetfillopacity{0.700000}%
\pgfsetlinewidth{0.000000pt}%
\definecolor{currentstroke}{rgb}{0.000000,0.000000,0.000000}%
\pgfsetstrokecolor{currentstroke}%
\pgfsetdash{}{0pt}%
\pgfpathmoveto{\pgfqpoint{4.018173in}{1.557661in}}%
\pgfpathlineto{\pgfqpoint{4.032301in}{1.557352in}}%
\pgfpathlineto{\pgfqpoint{4.046437in}{1.557116in}}%
\pgfpathlineto{\pgfqpoint{4.060583in}{1.556952in}}%
\pgfpathlineto{\pgfqpoint{4.074737in}{1.556860in}}%
\pgfpathlineto{\pgfqpoint{4.082910in}{1.568312in}}%
\pgfpathlineto{\pgfqpoint{4.091077in}{1.579753in}}%
\pgfpathlineto{\pgfqpoint{4.099240in}{1.591180in}}%
\pgfpathlineto{\pgfqpoint{4.107397in}{1.602589in}}%
\pgfpathlineto{\pgfqpoint{4.093251in}{1.602452in}}%
\pgfpathlineto{\pgfqpoint{4.079114in}{1.602388in}}%
\pgfpathlineto{\pgfqpoint{4.064986in}{1.602396in}}%
\pgfpathlineto{\pgfqpoint{4.050866in}{1.602477in}}%
\pgfpathlineto{\pgfqpoint{4.042701in}{1.591288in}}%
\pgfpathlineto{\pgfqpoint{4.034530in}{1.580087in}}%
\pgfpathlineto{\pgfqpoint{4.026354in}{1.568877in}}%
\pgfpathlineto{\pgfqpoint{4.018173in}{1.557661in}}%
\pgfpathclose%
\pgfusepath{fill}%
\end{pgfscope}%
\begin{pgfscope}%
\pgfpathrectangle{\pgfqpoint{1.150000in}{0.150000in}}{\pgfqpoint{5.700000in}{5.700000in}}%
\pgfusepath{clip}%
\pgfsetbuttcap%
\pgfsetroundjoin%
\definecolor{currentfill}{rgb}{0.280255,0.165693,0.476498}%
\pgfsetfillcolor{currentfill}%
\pgfsetfillopacity{0.700000}%
\pgfsetlinewidth{0.000000pt}%
\definecolor{currentstroke}{rgb}{0.000000,0.000000,0.000000}%
\pgfsetstrokecolor{currentstroke}%
\pgfsetdash{}{0pt}%
\pgfpathmoveto{\pgfqpoint{4.285844in}{1.699719in}}%
\pgfpathlineto{\pgfqpoint{4.300060in}{1.700898in}}%
\pgfpathlineto{\pgfqpoint{4.314287in}{1.702149in}}%
\pgfpathlineto{\pgfqpoint{4.328523in}{1.703472in}}%
\pgfpathlineto{\pgfqpoint{4.342769in}{1.704867in}}%
\pgfpathlineto{\pgfqpoint{4.350857in}{1.716511in}}%
\pgfpathlineto{\pgfqpoint{4.358939in}{1.728095in}}%
\pgfpathlineto{\pgfqpoint{4.367016in}{1.739616in}}%
\pgfpathlineto{\pgfqpoint{4.375087in}{1.751073in}}%
\pgfpathlineto{\pgfqpoint{4.360847in}{1.749512in}}%
\pgfpathlineto{\pgfqpoint{4.346618in}{1.748023in}}%
\pgfpathlineto{\pgfqpoint{4.332398in}{1.746605in}}%
\pgfpathlineto{\pgfqpoint{4.318189in}{1.745259in}}%
\pgfpathlineto{\pgfqpoint{4.310111in}{1.733961in}}%
\pgfpathlineto{\pgfqpoint{4.302027in}{1.722603in}}%
\pgfpathlineto{\pgfqpoint{4.293938in}{1.711189in}}%
\pgfpathlineto{\pgfqpoint{4.285844in}{1.699719in}}%
\pgfpathclose%
\pgfusepath{fill}%
\end{pgfscope}%
\begin{pgfscope}%
\pgfpathrectangle{\pgfqpoint{1.150000in}{0.150000in}}{\pgfqpoint{5.700000in}{5.700000in}}%
\pgfusepath{clip}%
\pgfsetbuttcap%
\pgfsetroundjoin%
\definecolor{currentfill}{rgb}{0.276022,0.044167,0.370164}%
\pgfsetfillcolor{currentfill}%
\pgfsetfillopacity{0.700000}%
\pgfsetlinewidth{0.000000pt}%
\definecolor{currentstroke}{rgb}{0.000000,0.000000,0.000000}%
\pgfsetstrokecolor{currentstroke}%
\pgfsetdash{}{0pt}%
\pgfpathmoveto{\pgfqpoint{2.931558in}{1.487889in}}%
\pgfpathlineto{\pgfqpoint{2.945518in}{1.480283in}}%
\pgfpathlineto{\pgfqpoint{2.959481in}{1.472763in}}%
\pgfpathlineto{\pgfqpoint{2.973447in}{1.465326in}}%
\pgfpathlineto{\pgfqpoint{2.987416in}{1.457974in}}%
\pgfpathlineto{\pgfqpoint{2.996118in}{1.461061in}}%
\pgfpathlineto{\pgfqpoint{3.004808in}{1.464384in}}%
\pgfpathlineto{\pgfqpoint{3.013484in}{1.467937in}}%
\pgfpathlineto{\pgfqpoint{3.022147in}{1.471713in}}%
\pgfpathlineto{\pgfqpoint{3.008207in}{1.478651in}}%
\pgfpathlineto{\pgfqpoint{2.994271in}{1.485673in}}%
\pgfpathlineto{\pgfqpoint{2.980337in}{1.492779in}}%
\pgfpathlineto{\pgfqpoint{2.966407in}{1.499970in}}%
\pgfpathlineto{\pgfqpoint{2.957715in}{1.496600in}}%
\pgfpathlineto{\pgfqpoint{2.949010in}{1.493459in}}%
\pgfpathlineto{\pgfqpoint{2.940291in}{1.490553in}}%
\pgfpathlineto{\pgfqpoint{2.931558in}{1.487889in}}%
\pgfpathclose%
\pgfusepath{fill}%
\end{pgfscope}%
\begin{pgfscope}%
\pgfpathrectangle{\pgfqpoint{1.150000in}{0.150000in}}{\pgfqpoint{5.700000in}{5.700000in}}%
\pgfusepath{clip}%
\pgfsetbuttcap%
\pgfsetroundjoin%
\definecolor{currentfill}{rgb}{0.280267,0.073417,0.397163}%
\pgfsetfillcolor{currentfill}%
\pgfsetfillopacity{0.700000}%
\pgfsetlinewidth{0.000000pt}%
\definecolor{currentstroke}{rgb}{0.000000,0.000000,0.000000}%
\pgfsetstrokecolor{currentstroke}%
\pgfsetdash{}{0pt}%
\pgfpathmoveto{\pgfqpoint{3.928931in}{1.515758in}}%
\pgfpathlineto{\pgfqpoint{3.943034in}{1.514911in}}%
\pgfpathlineto{\pgfqpoint{3.957145in}{1.514135in}}%
\pgfpathlineto{\pgfqpoint{3.971265in}{1.513433in}}%
\pgfpathlineto{\pgfqpoint{3.985393in}{1.512803in}}%
\pgfpathlineto{\pgfqpoint{3.993596in}{1.524010in}}%
\pgfpathlineto{\pgfqpoint{4.001794in}{1.535224in}}%
\pgfpathlineto{\pgfqpoint{4.009986in}{1.546442in}}%
\pgfpathlineto{\pgfqpoint{4.018173in}{1.557661in}}%
\pgfpathlineto{\pgfqpoint{4.004054in}{1.558042in}}%
\pgfpathlineto{\pgfqpoint{3.989943in}{1.558496in}}%
\pgfpathlineto{\pgfqpoint{3.975841in}{1.559023in}}%
\pgfpathlineto{\pgfqpoint{3.961747in}{1.559622in}}%
\pgfpathlineto{\pgfqpoint{3.953551in}{1.548644in}}%
\pgfpathlineto{\pgfqpoint{3.945350in}{1.537672in}}%
\pgfpathlineto{\pgfqpoint{3.937143in}{1.526708in}}%
\pgfpathlineto{\pgfqpoint{3.928931in}{1.515758in}}%
\pgfpathclose%
\pgfusepath{fill}%
\end{pgfscope}%
\begin{pgfscope}%
\pgfpathrectangle{\pgfqpoint{1.150000in}{0.150000in}}{\pgfqpoint{5.700000in}{5.700000in}}%
\pgfusepath{clip}%
\pgfsetbuttcap%
\pgfsetroundjoin%
\definecolor{currentfill}{rgb}{0.276194,0.190074,0.493001}%
\pgfsetfillcolor{currentfill}%
\pgfsetfillopacity{0.700000}%
\pgfsetlinewidth{0.000000pt}%
\definecolor{currentstroke}{rgb}{0.000000,0.000000,0.000000}%
\pgfsetstrokecolor{currentstroke}%
\pgfsetdash{}{0pt}%
\pgfpathmoveto{\pgfqpoint{4.375087in}{1.751073in}}%
\pgfpathlineto{\pgfqpoint{4.389337in}{1.752706in}}%
\pgfpathlineto{\pgfqpoint{4.403598in}{1.754410in}}%
\pgfpathlineto{\pgfqpoint{4.417868in}{1.756186in}}%
\pgfpathlineto{\pgfqpoint{4.432150in}{1.758034in}}%
\pgfpathlineto{\pgfqpoint{4.440209in}{1.769577in}}%
\pgfpathlineto{\pgfqpoint{4.448263in}{1.781047in}}%
\pgfpathlineto{\pgfqpoint{4.456310in}{1.792440in}}%
\pgfpathlineto{\pgfqpoint{4.464353in}{1.803756in}}%
\pgfpathlineto{\pgfqpoint{4.450078in}{1.801763in}}%
\pgfpathlineto{\pgfqpoint{4.435814in}{1.799842in}}%
\pgfpathlineto{\pgfqpoint{4.421560in}{1.797992in}}%
\pgfpathlineto{\pgfqpoint{4.407316in}{1.796214in}}%
\pgfpathlineto{\pgfqpoint{4.399268in}{1.785035in}}%
\pgfpathlineto{\pgfqpoint{4.391213in}{1.773784in}}%
\pgfpathlineto{\pgfqpoint{4.383153in}{1.762463in}}%
\pgfpathlineto{\pgfqpoint{4.375087in}{1.751073in}}%
\pgfpathclose%
\pgfusepath{fill}%
\end{pgfscope}%
\begin{pgfscope}%
\pgfpathrectangle{\pgfqpoint{1.150000in}{0.150000in}}{\pgfqpoint{5.700000in}{5.700000in}}%
\pgfusepath{clip}%
\pgfsetbuttcap%
\pgfsetroundjoin%
\definecolor{currentfill}{rgb}{0.268510,0.009605,0.335427}%
\pgfsetfillcolor{currentfill}%
\pgfsetfillopacity{0.700000}%
\pgfsetlinewidth{0.000000pt}%
\definecolor{currentstroke}{rgb}{0.000000,0.000000,0.000000}%
\pgfsetstrokecolor{currentstroke}%
\pgfsetdash{}{0pt}%
\pgfpathmoveto{\pgfqpoint{3.515283in}{1.401919in}}%
\pgfpathlineto{\pgfqpoint{3.529294in}{1.398393in}}%
\pgfpathlineto{\pgfqpoint{3.543311in}{1.394943in}}%
\pgfpathlineto{\pgfqpoint{3.557334in}{1.391568in}}%
\pgfpathlineto{\pgfqpoint{3.571364in}{1.388269in}}%
\pgfpathlineto{\pgfqpoint{3.579727in}{1.397106in}}%
\pgfpathlineto{\pgfqpoint{3.588083in}{1.406045in}}%
\pgfpathlineto{\pgfqpoint{3.596431in}{1.415082in}}%
\pgfpathlineto{\pgfqpoint{3.604773in}{1.424211in}}%
\pgfpathlineto{\pgfqpoint{3.590759in}{1.427181in}}%
\pgfpathlineto{\pgfqpoint{3.576751in}{1.430227in}}%
\pgfpathlineto{\pgfqpoint{3.562750in}{1.433347in}}%
\pgfpathlineto{\pgfqpoint{3.548755in}{1.436543in}}%
\pgfpathlineto{\pgfqpoint{3.540398in}{1.427736in}}%
\pgfpathlineto{\pgfqpoint{3.532034in}{1.419026in}}%
\pgfpathlineto{\pgfqpoint{3.523662in}{1.410419in}}%
\pgfpathlineto{\pgfqpoint{3.515283in}{1.401919in}}%
\pgfpathclose%
\pgfusepath{fill}%
\end{pgfscope}%
\begin{pgfscope}%
\pgfpathrectangle{\pgfqpoint{1.150000in}{0.150000in}}{\pgfqpoint{5.700000in}{5.700000in}}%
\pgfusepath{clip}%
\pgfsetbuttcap%
\pgfsetroundjoin%
\definecolor{currentfill}{rgb}{0.269308,0.218818,0.509577}%
\pgfsetfillcolor{currentfill}%
\pgfsetfillopacity{0.700000}%
\pgfsetlinewidth{0.000000pt}%
\definecolor{currentstroke}{rgb}{0.000000,0.000000,0.000000}%
\pgfsetstrokecolor{currentstroke}%
\pgfsetdash{}{0pt}%
\pgfpathmoveto{\pgfqpoint{4.464353in}{1.803756in}}%
\pgfpathlineto{\pgfqpoint{4.478638in}{1.805821in}}%
\pgfpathlineto{\pgfqpoint{4.492934in}{1.807957in}}%
\pgfpathlineto{\pgfqpoint{4.507241in}{1.810165in}}%
\pgfpathlineto{\pgfqpoint{4.521559in}{1.812444in}}%
\pgfpathlineto{\pgfqpoint{4.529589in}{1.823813in}}%
\pgfpathlineto{\pgfqpoint{4.537613in}{1.835095in}}%
\pgfpathlineto{\pgfqpoint{4.545631in}{1.846290in}}%
\pgfpathlineto{\pgfqpoint{4.553643in}{1.857395in}}%
\pgfpathlineto{\pgfqpoint{4.539332in}{1.854992in}}%
\pgfpathlineto{\pgfqpoint{4.525032in}{1.852660in}}%
\pgfpathlineto{\pgfqpoint{4.510742in}{1.850399in}}%
\pgfpathlineto{\pgfqpoint{4.496463in}{1.848210in}}%
\pgfpathlineto{\pgfqpoint{4.488444in}{1.837221in}}%
\pgfpathlineto{\pgfqpoint{4.480420in}{1.826148in}}%
\pgfpathlineto{\pgfqpoint{4.472389in}{1.814993in}}%
\pgfpathlineto{\pgfqpoint{4.464353in}{1.803756in}}%
\pgfpathclose%
\pgfusepath{fill}%
\end{pgfscope}%
\begin{pgfscope}%
\pgfpathrectangle{\pgfqpoint{1.150000in}{0.150000in}}{\pgfqpoint{5.700000in}{5.700000in}}%
\pgfusepath{clip}%
\pgfsetbuttcap%
\pgfsetroundjoin%
\definecolor{currentfill}{rgb}{0.277941,0.056324,0.381191}%
\pgfsetfillcolor{currentfill}%
\pgfsetfillopacity{0.700000}%
\pgfsetlinewidth{0.000000pt}%
\definecolor{currentstroke}{rgb}{0.000000,0.000000,0.000000}%
\pgfsetstrokecolor{currentstroke}%
\pgfsetdash{}{0pt}%
\pgfpathmoveto{\pgfqpoint{3.839652in}{1.477357in}}%
\pgfpathlineto{\pgfqpoint{3.853733in}{1.475949in}}%
\pgfpathlineto{\pgfqpoint{3.867823in}{1.474613in}}%
\pgfpathlineto{\pgfqpoint{3.881920in}{1.473351in}}%
\pgfpathlineto{\pgfqpoint{3.896025in}{1.472161in}}%
\pgfpathlineto{\pgfqpoint{3.904260in}{1.483022in}}%
\pgfpathlineto{\pgfqpoint{3.912489in}{1.493911in}}%
\pgfpathlineto{\pgfqpoint{3.920713in}{1.504825in}}%
\pgfpathlineto{\pgfqpoint{3.928931in}{1.515758in}}%
\pgfpathlineto{\pgfqpoint{3.914836in}{1.516679in}}%
\pgfpathlineto{\pgfqpoint{3.900749in}{1.517673in}}%
\pgfpathlineto{\pgfqpoint{3.886670in}{1.518739in}}%
\pgfpathlineto{\pgfqpoint{3.872599in}{1.519880in}}%
\pgfpathlineto{\pgfqpoint{3.864371in}{1.509207in}}%
\pgfpathlineto{\pgfqpoint{3.856137in}{1.498560in}}%
\pgfpathlineto{\pgfqpoint{3.847897in}{1.487942in}}%
\pgfpathlineto{\pgfqpoint{3.839652in}{1.477357in}}%
\pgfpathclose%
\pgfusepath{fill}%
\end{pgfscope}%
\begin{pgfscope}%
\pgfpathrectangle{\pgfqpoint{1.150000in}{0.150000in}}{\pgfqpoint{5.700000in}{5.700000in}}%
\pgfusepath{clip}%
\pgfsetbuttcap%
\pgfsetroundjoin%
\definecolor{currentfill}{rgb}{0.262138,0.242286,0.520837}%
\pgfsetfillcolor{currentfill}%
\pgfsetfillopacity{0.700000}%
\pgfsetlinewidth{0.000000pt}%
\definecolor{currentstroke}{rgb}{0.000000,0.000000,0.000000}%
\pgfsetstrokecolor{currentstroke}%
\pgfsetdash{}{0pt}%
\pgfpathmoveto{\pgfqpoint{4.553643in}{1.857395in}}%
\pgfpathlineto{\pgfqpoint{4.567966in}{1.859871in}}%
\pgfpathlineto{\pgfqpoint{4.582299in}{1.862417in}}%
\pgfpathlineto{\pgfqpoint{4.596643in}{1.865035in}}%
\pgfpathlineto{\pgfqpoint{4.610999in}{1.867725in}}%
\pgfpathlineto{\pgfqpoint{4.618998in}{1.878850in}}%
\pgfpathlineto{\pgfqpoint{4.626992in}{1.889879in}}%
\pgfpathlineto{\pgfqpoint{4.634979in}{1.900809in}}%
\pgfpathlineto{\pgfqpoint{4.642960in}{1.911638in}}%
\pgfpathlineto{\pgfqpoint{4.628611in}{1.908846in}}%
\pgfpathlineto{\pgfqpoint{4.614274in}{1.906124in}}%
\pgfpathlineto{\pgfqpoint{4.599947in}{1.903474in}}%
\pgfpathlineto{\pgfqpoint{4.585631in}{1.900896in}}%
\pgfpathlineto{\pgfqpoint{4.577644in}{1.890161in}}%
\pgfpathlineto{\pgfqpoint{4.569650in}{1.879332in}}%
\pgfpathlineto{\pgfqpoint{4.561650in}{1.868410in}}%
\pgfpathlineto{\pgfqpoint{4.553643in}{1.857395in}}%
\pgfpathclose%
\pgfusepath{fill}%
\end{pgfscope}%
\begin{pgfscope}%
\pgfpathrectangle{\pgfqpoint{1.150000in}{0.150000in}}{\pgfqpoint{5.700000in}{5.700000in}}%
\pgfusepath{clip}%
\pgfsetbuttcap%
\pgfsetroundjoin%
\definecolor{currentfill}{rgb}{0.267004,0.004874,0.329415}%
\pgfsetfillcolor{currentfill}%
\pgfsetfillopacity{0.700000}%
\pgfsetlinewidth{0.000000pt}%
\definecolor{currentstroke}{rgb}{0.000000,0.000000,0.000000}%
\pgfsetstrokecolor{currentstroke}%
\pgfsetdash{}{0pt}%
\pgfpathmoveto{\pgfqpoint{3.279817in}{1.395335in}}%
\pgfpathlineto{\pgfqpoint{3.293801in}{1.390165in}}%
\pgfpathlineto{\pgfqpoint{3.307790in}{1.385074in}}%
\pgfpathlineto{\pgfqpoint{3.321783in}{1.380060in}}%
\pgfpathlineto{\pgfqpoint{3.335782in}{1.375125in}}%
\pgfpathlineto{\pgfqpoint{3.344267in}{1.381819in}}%
\pgfpathlineto{\pgfqpoint{3.352743in}{1.388675in}}%
\pgfpathlineto{\pgfqpoint{3.361209in}{1.395685in}}%
\pgfpathlineto{\pgfqpoint{3.369667in}{1.402845in}}%
\pgfpathlineto{\pgfqpoint{3.355689in}{1.407410in}}%
\pgfpathlineto{\pgfqpoint{3.341716in}{1.412053in}}%
\pgfpathlineto{\pgfqpoint{3.327749in}{1.416773in}}%
\pgfpathlineto{\pgfqpoint{3.313786in}{1.421572in}}%
\pgfpathlineto{\pgfqpoint{3.305308in}{1.414775in}}%
\pgfpathlineto{\pgfqpoint{3.296821in}{1.408133in}}%
\pgfpathlineto{\pgfqpoint{3.288324in}{1.401651in}}%
\pgfpathlineto{\pgfqpoint{3.279817in}{1.395335in}}%
\pgfpathclose%
\pgfusepath{fill}%
\end{pgfscope}%
\begin{pgfscope}%
\pgfpathrectangle{\pgfqpoint{1.150000in}{0.150000in}}{\pgfqpoint{5.700000in}{5.700000in}}%
\pgfusepath{clip}%
\pgfsetbuttcap%
\pgfsetroundjoin%
\definecolor{currentfill}{rgb}{0.274952,0.037752,0.364543}%
\pgfsetfillcolor{currentfill}%
\pgfsetfillopacity{0.700000}%
\pgfsetlinewidth{0.000000pt}%
\definecolor{currentstroke}{rgb}{0.000000,0.000000,0.000000}%
\pgfsetstrokecolor{currentstroke}%
\pgfsetdash{}{0pt}%
\pgfpathmoveto{\pgfqpoint{3.750315in}{1.442957in}}%
\pgfpathlineto{\pgfqpoint{3.764378in}{1.440966in}}%
\pgfpathlineto{\pgfqpoint{3.778448in}{1.439048in}}%
\pgfpathlineto{\pgfqpoint{3.792526in}{1.437203in}}%
\pgfpathlineto{\pgfqpoint{3.806611in}{1.435432in}}%
\pgfpathlineto{\pgfqpoint{3.814880in}{1.445843in}}%
\pgfpathlineto{\pgfqpoint{3.823143in}{1.456304in}}%
\pgfpathlineto{\pgfqpoint{3.831400in}{1.466810in}}%
\pgfpathlineto{\pgfqpoint{3.839652in}{1.477357in}}%
\pgfpathlineto{\pgfqpoint{3.825578in}{1.478839in}}%
\pgfpathlineto{\pgfqpoint{3.811512in}{1.480395in}}%
\pgfpathlineto{\pgfqpoint{3.797454in}{1.482024in}}%
\pgfpathlineto{\pgfqpoint{3.783403in}{1.483726in}}%
\pgfpathlineto{\pgfqpoint{3.775140in}{1.473460in}}%
\pgfpathlineto{\pgfqpoint{3.766871in}{1.463241in}}%
\pgfpathlineto{\pgfqpoint{3.758596in}{1.453072in}}%
\pgfpathlineto{\pgfqpoint{3.750315in}{1.442957in}}%
\pgfpathclose%
\pgfusepath{fill}%
\end{pgfscope}%
\begin{pgfscope}%
\pgfpathrectangle{\pgfqpoint{1.150000in}{0.150000in}}{\pgfqpoint{5.700000in}{5.700000in}}%
\pgfusepath{clip}%
\pgfsetbuttcap%
\pgfsetroundjoin%
\definecolor{currentfill}{rgb}{0.253935,0.265254,0.529983}%
\pgfsetfillcolor{currentfill}%
\pgfsetfillopacity{0.700000}%
\pgfsetlinewidth{0.000000pt}%
\definecolor{currentstroke}{rgb}{0.000000,0.000000,0.000000}%
\pgfsetstrokecolor{currentstroke}%
\pgfsetdash{}{0pt}%
\pgfpathmoveto{\pgfqpoint{4.642960in}{1.911638in}}%
\pgfpathlineto{\pgfqpoint{4.657321in}{1.914503in}}%
\pgfpathlineto{\pgfqpoint{4.671692in}{1.917438in}}%
\pgfpathlineto{\pgfqpoint{4.686076in}{1.920445in}}%
\pgfpathlineto{\pgfqpoint{4.700470in}{1.923523in}}%
\pgfpathlineto{\pgfqpoint{4.708438in}{1.934343in}}%
\pgfpathlineto{\pgfqpoint{4.716400in}{1.945055in}}%
\pgfpathlineto{\pgfqpoint{4.724354in}{1.955658in}}%
\pgfpathlineto{\pgfqpoint{4.732303in}{1.966153in}}%
\pgfpathlineto{\pgfqpoint{4.717915in}{1.962993in}}%
\pgfpathlineto{\pgfqpoint{4.703538in}{1.959904in}}%
\pgfpathlineto{\pgfqpoint{4.689174in}{1.956886in}}%
\pgfpathlineto{\pgfqpoint{4.674820in}{1.953940in}}%
\pgfpathlineto{\pgfqpoint{4.666865in}{1.943519in}}%
\pgfpathlineto{\pgfqpoint{4.658903in}{1.932995in}}%
\pgfpathlineto{\pgfqpoint{4.650935in}{1.922367in}}%
\pgfpathlineto{\pgfqpoint{4.642960in}{1.911638in}}%
\pgfpathclose%
\pgfusepath{fill}%
\end{pgfscope}%
\begin{pgfscope}%
\pgfpathrectangle{\pgfqpoint{1.150000in}{0.150000in}}{\pgfqpoint{5.700000in}{5.700000in}}%
\pgfusepath{clip}%
\pgfsetbuttcap%
\pgfsetroundjoin%
\definecolor{currentfill}{rgb}{0.269944,0.014625,0.341379}%
\pgfsetfillcolor{currentfill}%
\pgfsetfillopacity{0.700000}%
\pgfsetlinewidth{0.000000pt}%
\definecolor{currentstroke}{rgb}{0.000000,0.000000,0.000000}%
\pgfsetstrokecolor{currentstroke}%
\pgfsetdash{}{0pt}%
\pgfpathmoveto{\pgfqpoint{3.133791in}{1.419190in}}%
\pgfpathlineto{\pgfqpoint{3.147764in}{1.412992in}}%
\pgfpathlineto{\pgfqpoint{3.161740in}{1.406875in}}%
\pgfpathlineto{\pgfqpoint{3.175721in}{1.400838in}}%
\pgfpathlineto{\pgfqpoint{3.189706in}{1.394881in}}%
\pgfpathlineto{\pgfqpoint{3.198278in}{1.400075in}}%
\pgfpathlineto{\pgfqpoint{3.206839in}{1.405464in}}%
\pgfpathlineto{\pgfqpoint{3.215389in}{1.411042in}}%
\pgfpathlineto{\pgfqpoint{3.223929in}{1.416802in}}%
\pgfpathlineto{\pgfqpoint{3.209968in}{1.422367in}}%
\pgfpathlineto{\pgfqpoint{3.196012in}{1.428012in}}%
\pgfpathlineto{\pgfqpoint{3.182060in}{1.433738in}}%
\pgfpathlineto{\pgfqpoint{3.168113in}{1.439544in}}%
\pgfpathlineto{\pgfqpoint{3.159549in}{1.434167in}}%
\pgfpathlineto{\pgfqpoint{3.150975in}{1.428979in}}%
\pgfpathlineto{\pgfqpoint{3.142389in}{1.423984in}}%
\pgfpathlineto{\pgfqpoint{3.133791in}{1.419190in}}%
\pgfpathclose%
\pgfusepath{fill}%
\end{pgfscope}%
\begin{pgfscope}%
\pgfpathrectangle{\pgfqpoint{1.150000in}{0.150000in}}{\pgfqpoint{5.700000in}{5.700000in}}%
\pgfusepath{clip}%
\pgfsetbuttcap%
\pgfsetroundjoin%
\definecolor{currentfill}{rgb}{0.243113,0.292092,0.538516}%
\pgfsetfillcolor{currentfill}%
\pgfsetfillopacity{0.700000}%
\pgfsetlinewidth{0.000000pt}%
\definecolor{currentstroke}{rgb}{0.000000,0.000000,0.000000}%
\pgfsetstrokecolor{currentstroke}%
\pgfsetdash{}{0pt}%
\pgfpathmoveto{\pgfqpoint{4.732303in}{1.966153in}}%
\pgfpathlineto{\pgfqpoint{4.746702in}{1.969385in}}%
\pgfpathlineto{\pgfqpoint{4.761113in}{1.972688in}}%
\pgfpathlineto{\pgfqpoint{4.775536in}{1.976062in}}%
\pgfpathlineto{\pgfqpoint{4.789971in}{1.979508in}}%
\pgfpathlineto{\pgfqpoint{4.797905in}{1.989962in}}%
\pgfpathlineto{\pgfqpoint{4.805833in}{2.000301in}}%
\pgfpathlineto{\pgfqpoint{4.813753in}{2.010523in}}%
\pgfpathlineto{\pgfqpoint{4.821667in}{2.020628in}}%
\pgfpathlineto{\pgfqpoint{4.807239in}{2.017122in}}%
\pgfpathlineto{\pgfqpoint{4.792824in}{2.013687in}}%
\pgfpathlineto{\pgfqpoint{4.778420in}{2.010324in}}%
\pgfpathlineto{\pgfqpoint{4.764028in}{2.007031in}}%
\pgfpathlineto{\pgfqpoint{4.756107in}{1.996978in}}%
\pgfpathlineto{\pgfqpoint{4.748179in}{1.986814in}}%
\pgfpathlineto{\pgfqpoint{4.740244in}{1.976539in}}%
\pgfpathlineto{\pgfqpoint{4.732303in}{1.966153in}}%
\pgfpathclose%
\pgfusepath{fill}%
\end{pgfscope}%
\begin{pgfscope}%
\pgfpathrectangle{\pgfqpoint{1.150000in}{0.150000in}}{\pgfqpoint{5.700000in}{5.700000in}}%
\pgfusepath{clip}%
\pgfsetbuttcap%
\pgfsetroundjoin%
\definecolor{currentfill}{rgb}{0.267004,0.004874,0.329415}%
\pgfsetfillcolor{currentfill}%
\pgfsetfillopacity{0.700000}%
\pgfsetlinewidth{0.000000pt}%
\definecolor{currentstroke}{rgb}{0.000000,0.000000,0.000000}%
\pgfsetstrokecolor{currentstroke}%
\pgfsetdash{}{0pt}%
\pgfpathmoveto{\pgfqpoint{3.425632in}{1.385357in}}%
\pgfpathlineto{\pgfqpoint{3.439637in}{1.381178in}}%
\pgfpathlineto{\pgfqpoint{3.453648in}{1.377074in}}%
\pgfpathlineto{\pgfqpoint{3.467665in}{1.373047in}}%
\pgfpathlineto{\pgfqpoint{3.481687in}{1.369095in}}%
\pgfpathlineto{\pgfqpoint{3.490098in}{1.377114in}}%
\pgfpathlineto{\pgfqpoint{3.498501in}{1.385261in}}%
\pgfpathlineto{\pgfqpoint{3.506896in}{1.393531in}}%
\pgfpathlineto{\pgfqpoint{3.515283in}{1.401919in}}%
\pgfpathlineto{\pgfqpoint{3.501278in}{1.405520in}}%
\pgfpathlineto{\pgfqpoint{3.487279in}{1.409198in}}%
\pgfpathlineto{\pgfqpoint{3.473287in}{1.412951in}}%
\pgfpathlineto{\pgfqpoint{3.459300in}{1.416781in}}%
\pgfpathlineto{\pgfqpoint{3.450895in}{1.408736in}}%
\pgfpathlineto{\pgfqpoint{3.442483in}{1.400813in}}%
\pgfpathlineto{\pgfqpoint{3.434062in}{1.393019in}}%
\pgfpathlineto{\pgfqpoint{3.425632in}{1.385357in}}%
\pgfpathclose%
\pgfusepath{fill}%
\end{pgfscope}%
\begin{pgfscope}%
\pgfpathrectangle{\pgfqpoint{1.150000in}{0.150000in}}{\pgfqpoint{5.700000in}{5.700000in}}%
\pgfusepath{clip}%
\pgfsetbuttcap%
\pgfsetroundjoin%
\definecolor{currentfill}{rgb}{0.233603,0.313828,0.543914}%
\pgfsetfillcolor{currentfill}%
\pgfsetfillopacity{0.700000}%
\pgfsetlinewidth{0.000000pt}%
\definecolor{currentstroke}{rgb}{0.000000,0.000000,0.000000}%
\pgfsetstrokecolor{currentstroke}%
\pgfsetdash{}{0pt}%
\pgfpathmoveto{\pgfqpoint{4.821667in}{2.020628in}}%
\pgfpathlineto{\pgfqpoint{4.836106in}{2.024206in}}%
\pgfpathlineto{\pgfqpoint{4.850558in}{2.027855in}}%
\pgfpathlineto{\pgfqpoint{4.865022in}{2.031575in}}%
\pgfpathlineto{\pgfqpoint{4.879497in}{2.035367in}}%
\pgfpathlineto{\pgfqpoint{4.887396in}{2.045402in}}%
\pgfpathlineto{\pgfqpoint{4.895288in}{2.055316in}}%
\pgfpathlineto{\pgfqpoint{4.903172in}{2.065106in}}%
\pgfpathlineto{\pgfqpoint{4.911048in}{2.074773in}}%
\pgfpathlineto{\pgfqpoint{4.896581in}{2.070943in}}%
\pgfpathlineto{\pgfqpoint{4.882125in}{2.067184in}}%
\pgfpathlineto{\pgfqpoint{4.867681in}{2.063496in}}%
\pgfpathlineto{\pgfqpoint{4.853250in}{2.059879in}}%
\pgfpathlineto{\pgfqpoint{4.845365in}{2.050243in}}%
\pgfpathlineto{\pgfqpoint{4.837473in}{2.040489in}}%
\pgfpathlineto{\pgfqpoint{4.829573in}{2.030617in}}%
\pgfpathlineto{\pgfqpoint{4.821667in}{2.020628in}}%
\pgfpathclose%
\pgfusepath{fill}%
\end{pgfscope}%
\begin{pgfscope}%
\pgfpathrectangle{\pgfqpoint{1.150000in}{0.150000in}}{\pgfqpoint{5.700000in}{5.700000in}}%
\pgfusepath{clip}%
\pgfsetbuttcap%
\pgfsetroundjoin%
\definecolor{currentfill}{rgb}{0.271305,0.019942,0.347269}%
\pgfsetfillcolor{currentfill}%
\pgfsetfillopacity{0.700000}%
\pgfsetlinewidth{0.000000pt}%
\definecolor{currentstroke}{rgb}{0.000000,0.000000,0.000000}%
\pgfsetstrokecolor{currentstroke}%
\pgfsetdash{}{0pt}%
\pgfpathmoveto{\pgfqpoint{3.660895in}{1.413079in}}%
\pgfpathlineto{\pgfqpoint{3.674943in}{1.410483in}}%
\pgfpathlineto{\pgfqpoint{3.688997in}{1.407960in}}%
\pgfpathlineto{\pgfqpoint{3.703059in}{1.405511in}}%
\pgfpathlineto{\pgfqpoint{3.717127in}{1.403137in}}%
\pgfpathlineto{\pgfqpoint{3.725434in}{1.412987in}}%
\pgfpathlineto{\pgfqpoint{3.733734in}{1.422911in}}%
\pgfpathlineto{\pgfqpoint{3.742028in}{1.432902in}}%
\pgfpathlineto{\pgfqpoint{3.750315in}{1.442957in}}%
\pgfpathlineto{\pgfqpoint{3.736259in}{1.445023in}}%
\pgfpathlineto{\pgfqpoint{3.722211in}{1.447162in}}%
\pgfpathlineto{\pgfqpoint{3.708170in}{1.449375in}}%
\pgfpathlineto{\pgfqpoint{3.694136in}{1.451663in}}%
\pgfpathlineto{\pgfqpoint{3.685836in}{1.441909in}}%
\pgfpathlineto{\pgfqpoint{3.677529in}{1.432224in}}%
\pgfpathlineto{\pgfqpoint{3.669215in}{1.422613in}}%
\pgfpathlineto{\pgfqpoint{3.660895in}{1.413079in}}%
\pgfpathclose%
\pgfusepath{fill}%
\end{pgfscope}%
\begin{pgfscope}%
\pgfpathrectangle{\pgfqpoint{1.150000in}{0.150000in}}{\pgfqpoint{5.700000in}{5.700000in}}%
\pgfusepath{clip}%
\pgfsetbuttcap%
\pgfsetroundjoin%
\definecolor{currentfill}{rgb}{0.274952,0.037752,0.364543}%
\pgfsetfillcolor{currentfill}%
\pgfsetfillopacity{0.700000}%
\pgfsetlinewidth{0.000000pt}%
\definecolor{currentstroke}{rgb}{0.000000,0.000000,0.000000}%
\pgfsetstrokecolor{currentstroke}%
\pgfsetdash{}{0pt}%
\pgfpathmoveto{\pgfqpoint{2.987416in}{1.457974in}}%
\pgfpathlineto{\pgfqpoint{3.001388in}{1.450706in}}%
\pgfpathlineto{\pgfqpoint{3.015363in}{1.443521in}}%
\pgfpathlineto{\pgfqpoint{3.029341in}{1.436418in}}%
\pgfpathlineto{\pgfqpoint{3.043322in}{1.429399in}}%
\pgfpathlineto{\pgfqpoint{3.051996in}{1.432908in}}%
\pgfpathlineto{\pgfqpoint{3.060657in}{1.436648in}}%
\pgfpathlineto{\pgfqpoint{3.069305in}{1.440612in}}%
\pgfpathlineto{\pgfqpoint{3.077940in}{1.444794in}}%
\pgfpathlineto{\pgfqpoint{3.063987in}{1.451400in}}%
\pgfpathlineto{\pgfqpoint{3.050037in}{1.458088in}}%
\pgfpathlineto{\pgfqpoint{3.036090in}{1.464859in}}%
\pgfpathlineto{\pgfqpoint{3.022147in}{1.471713in}}%
\pgfpathlineto{\pgfqpoint{3.013484in}{1.467937in}}%
\pgfpathlineto{\pgfqpoint{3.004808in}{1.464384in}}%
\pgfpathlineto{\pgfqpoint{2.996118in}{1.461061in}}%
\pgfpathlineto{\pgfqpoint{2.987416in}{1.457974in}}%
\pgfpathclose%
\pgfusepath{fill}%
\end{pgfscope}%
\begin{pgfscope}%
\pgfpathrectangle{\pgfqpoint{1.150000in}{0.150000in}}{\pgfqpoint{5.700000in}{5.700000in}}%
\pgfusepath{clip}%
\pgfsetbuttcap%
\pgfsetroundjoin%
\definecolor{currentfill}{rgb}{0.221989,0.339161,0.548752}%
\pgfsetfillcolor{currentfill}%
\pgfsetfillopacity{0.700000}%
\pgfsetlinewidth{0.000000pt}%
\definecolor{currentstroke}{rgb}{0.000000,0.000000,0.000000}%
\pgfsetstrokecolor{currentstroke}%
\pgfsetdash{}{0pt}%
\pgfpathmoveto{\pgfqpoint{4.911048in}{2.074773in}}%
\pgfpathlineto{\pgfqpoint{4.925529in}{2.078675in}}%
\pgfpathlineto{\pgfqpoint{4.940021in}{2.082648in}}%
\pgfpathlineto{\pgfqpoint{4.954526in}{2.086692in}}%
\pgfpathlineto{\pgfqpoint{4.969044in}{2.090808in}}%
\pgfpathlineto{\pgfqpoint{4.976904in}{2.100377in}}%
\pgfpathlineto{\pgfqpoint{4.984758in}{2.109819in}}%
\pgfpathlineto{\pgfqpoint{4.992603in}{2.119132in}}%
\pgfpathlineto{\pgfqpoint{5.000441in}{2.128318in}}%
\pgfpathlineto{\pgfqpoint{4.985932in}{2.124185in}}%
\pgfpathlineto{\pgfqpoint{4.971436in}{2.120124in}}%
\pgfpathlineto{\pgfqpoint{4.956952in}{2.116134in}}%
\pgfpathlineto{\pgfqpoint{4.942480in}{2.112215in}}%
\pgfpathlineto{\pgfqpoint{4.934634in}{2.103038in}}%
\pgfpathlineto{\pgfqpoint{4.926779in}{2.093739in}}%
\pgfpathlineto{\pgfqpoint{4.918918in}{2.084318in}}%
\pgfpathlineto{\pgfqpoint{4.911048in}{2.074773in}}%
\pgfpathclose%
\pgfusepath{fill}%
\end{pgfscope}%
\begin{pgfscope}%
\pgfpathrectangle{\pgfqpoint{1.150000in}{0.150000in}}{\pgfqpoint{5.700000in}{5.700000in}}%
\pgfusepath{clip}%
\pgfsetbuttcap%
\pgfsetroundjoin%
\definecolor{currentfill}{rgb}{0.172719,0.448791,0.557885}%
\pgfsetfillcolor{currentfill}%
\pgfsetfillopacity{0.700000}%
\pgfsetlinewidth{0.000000pt}%
\definecolor{currentstroke}{rgb}{0.000000,0.000000,0.000000}%
\pgfsetstrokecolor{currentstroke}%
\pgfsetdash{}{0pt}%
\pgfpathmoveto{\pgfqpoint{5.447247in}{2.379055in}}%
\pgfpathlineto{\pgfqpoint{5.461975in}{2.384437in}}%
\pgfpathlineto{\pgfqpoint{5.476717in}{2.389890in}}%
\pgfpathlineto{\pgfqpoint{5.491473in}{2.395414in}}%
\pgfpathlineto{\pgfqpoint{5.499056in}{2.401514in}}%
\pgfpathlineto{\pgfqpoint{5.506630in}{2.407482in}}%
\pgfpathlineto{\pgfqpoint{5.514193in}{2.413321in}}%
\pgfpathlineto{\pgfqpoint{5.521747in}{2.419033in}}%
\pgfpathlineto{\pgfqpoint{5.507006in}{2.413625in}}%
\pgfpathlineto{\pgfqpoint{5.492280in}{2.408288in}}%
\pgfpathlineto{\pgfqpoint{5.477567in}{2.403022in}}%
\pgfpathlineto{\pgfqpoint{5.470002in}{2.397217in}}%
\pgfpathlineto{\pgfqpoint{5.462426in}{2.391289in}}%
\pgfpathlineto{\pgfqpoint{5.454841in}{2.385235in}}%
\pgfpathlineto{\pgfqpoint{5.447247in}{2.379055in}}%
\pgfpathclose%
\pgfusepath{fill}%
\end{pgfscope}%
\begin{pgfscope}%
\pgfpathrectangle{\pgfqpoint{1.150000in}{0.150000in}}{\pgfqpoint{5.700000in}{5.700000in}}%
\pgfusepath{clip}%
\pgfsetbuttcap%
\pgfsetroundjoin%
\definecolor{currentfill}{rgb}{0.212395,0.359683,0.551710}%
\pgfsetfillcolor{currentfill}%
\pgfsetfillopacity{0.700000}%
\pgfsetlinewidth{0.000000pt}%
\definecolor{currentstroke}{rgb}{0.000000,0.000000,0.000000}%
\pgfsetstrokecolor{currentstroke}%
\pgfsetdash{}{0pt}%
\pgfpathmoveto{\pgfqpoint{5.000441in}{2.128318in}}%
\pgfpathlineto{\pgfqpoint{5.014962in}{2.132521in}}%
\pgfpathlineto{\pgfqpoint{5.029496in}{2.136797in}}%
\pgfpathlineto{\pgfqpoint{5.044043in}{2.141143in}}%
\pgfpathlineto{\pgfqpoint{5.058602in}{2.145561in}}%
\pgfpathlineto{\pgfqpoint{5.066423in}{2.154622in}}%
\pgfpathlineto{\pgfqpoint{5.074235in}{2.163550in}}%
\pgfpathlineto{\pgfqpoint{5.082040in}{2.172347in}}%
\pgfpathlineto{\pgfqpoint{5.089836in}{2.181012in}}%
\pgfpathlineto{\pgfqpoint{5.075286in}{2.176599in}}%
\pgfpathlineto{\pgfqpoint{5.060748in}{2.172258in}}%
\pgfpathlineto{\pgfqpoint{5.046224in}{2.167987in}}%
\pgfpathlineto{\pgfqpoint{5.031712in}{2.163788in}}%
\pgfpathlineto{\pgfqpoint{5.023906in}{2.155110in}}%
\pgfpathlineto{\pgfqpoint{5.016092in}{2.146306in}}%
\pgfpathlineto{\pgfqpoint{5.008270in}{2.137375in}}%
\pgfpathlineto{\pgfqpoint{5.000441in}{2.128318in}}%
\pgfpathclose%
\pgfusepath{fill}%
\end{pgfscope}%
\begin{pgfscope}%
\pgfpathrectangle{\pgfqpoint{1.150000in}{0.150000in}}{\pgfqpoint{5.700000in}{5.700000in}}%
\pgfusepath{clip}%
\pgfsetbuttcap%
\pgfsetroundjoin%
\definecolor{currentfill}{rgb}{0.203063,0.379716,0.553925}%
\pgfsetfillcolor{currentfill}%
\pgfsetfillopacity{0.700000}%
\pgfsetlinewidth{0.000000pt}%
\definecolor{currentstroke}{rgb}{0.000000,0.000000,0.000000}%
\pgfsetstrokecolor{currentstroke}%
\pgfsetdash{}{0pt}%
\pgfpathmoveto{\pgfqpoint{5.089836in}{2.181012in}}%
\pgfpathlineto{\pgfqpoint{5.104399in}{2.185496in}}%
\pgfpathlineto{\pgfqpoint{5.118974in}{2.190051in}}%
\pgfpathlineto{\pgfqpoint{5.133563in}{2.194678in}}%
\pgfpathlineto{\pgfqpoint{5.148165in}{2.199376in}}%
\pgfpathlineto{\pgfqpoint{5.155943in}{2.207891in}}%
\pgfpathlineto{\pgfqpoint{5.163712in}{2.216271in}}%
\pgfpathlineto{\pgfqpoint{5.171472in}{2.224517in}}%
\pgfpathlineto{\pgfqpoint{5.179224in}{2.232629in}}%
\pgfpathlineto{\pgfqpoint{5.164633in}{2.227958in}}%
\pgfpathlineto{\pgfqpoint{5.150055in}{2.223358in}}%
\pgfpathlineto{\pgfqpoint{5.135490in}{2.218830in}}%
\pgfpathlineto{\pgfqpoint{5.120937in}{2.214373in}}%
\pgfpathlineto{\pgfqpoint{5.113174in}{2.206226in}}%
\pgfpathlineto{\pgfqpoint{5.105403in}{2.197951in}}%
\pgfpathlineto{\pgfqpoint{5.097624in}{2.189546in}}%
\pgfpathlineto{\pgfqpoint{5.089836in}{2.181012in}}%
\pgfpathclose%
\pgfusepath{fill}%
\end{pgfscope}%
\begin{pgfscope}%
\pgfpathrectangle{\pgfqpoint{1.150000in}{0.150000in}}{\pgfqpoint{5.700000in}{5.700000in}}%
\pgfusepath{clip}%
\pgfsetbuttcap%
\pgfsetroundjoin%
\definecolor{currentfill}{rgb}{0.194100,0.399323,0.555565}%
\pgfsetfillcolor{currentfill}%
\pgfsetfillopacity{0.700000}%
\pgfsetlinewidth{0.000000pt}%
\definecolor{currentstroke}{rgb}{0.000000,0.000000,0.000000}%
\pgfsetstrokecolor{currentstroke}%
\pgfsetdash{}{0pt}%
\pgfpathmoveto{\pgfqpoint{5.179224in}{2.232629in}}%
\pgfpathlineto{\pgfqpoint{5.193829in}{2.237371in}}%
\pgfpathlineto{\pgfqpoint{5.208446in}{2.242184in}}%
\pgfpathlineto{\pgfqpoint{5.223077in}{2.247069in}}%
\pgfpathlineto{\pgfqpoint{5.237722in}{2.252025in}}%
\pgfpathlineto{\pgfqpoint{5.245454in}{2.259963in}}%
\pgfpathlineto{\pgfqpoint{5.253177in}{2.267764in}}%
\pgfpathlineto{\pgfqpoint{5.260891in}{2.275430in}}%
\pgfpathlineto{\pgfqpoint{5.268597in}{2.282961in}}%
\pgfpathlineto{\pgfqpoint{5.253964in}{2.278055in}}%
\pgfpathlineto{\pgfqpoint{5.239345in}{2.273219in}}%
\pgfpathlineto{\pgfqpoint{5.224739in}{2.268455in}}%
\pgfpathlineto{\pgfqpoint{5.210146in}{2.263762in}}%
\pgfpathlineto{\pgfqpoint{5.202429in}{2.256174in}}%
\pgfpathlineto{\pgfqpoint{5.194702in}{2.248456in}}%
\pgfpathlineto{\pgfqpoint{5.186968in}{2.240608in}}%
\pgfpathlineto{\pgfqpoint{5.179224in}{2.232629in}}%
\pgfpathclose%
\pgfusepath{fill}%
\end{pgfscope}%
\begin{pgfscope}%
\pgfpathrectangle{\pgfqpoint{1.150000in}{0.150000in}}{\pgfqpoint{5.700000in}{5.700000in}}%
\pgfusepath{clip}%
\pgfsetbuttcap%
\pgfsetroundjoin%
\definecolor{currentfill}{rgb}{0.185556,0.418570,0.556753}%
\pgfsetfillcolor{currentfill}%
\pgfsetfillopacity{0.700000}%
\pgfsetlinewidth{0.000000pt}%
\definecolor{currentstroke}{rgb}{0.000000,0.000000,0.000000}%
\pgfsetstrokecolor{currentstroke}%
\pgfsetdash{}{0pt}%
\pgfpathmoveto{\pgfqpoint{5.268597in}{2.282961in}}%
\pgfpathlineto{\pgfqpoint{5.283243in}{2.287939in}}%
\pgfpathlineto{\pgfqpoint{5.297902in}{2.292988in}}%
\pgfpathlineto{\pgfqpoint{5.312575in}{2.298108in}}%
\pgfpathlineto{\pgfqpoint{5.327261in}{2.303300in}}%
\pgfpathlineto{\pgfqpoint{5.334945in}{2.310635in}}%
\pgfpathlineto{\pgfqpoint{5.342620in}{2.317833in}}%
\pgfpathlineto{\pgfqpoint{5.350285in}{2.324895in}}%
\pgfpathlineto{\pgfqpoint{5.357941in}{2.331824in}}%
\pgfpathlineto{\pgfqpoint{5.343268in}{2.326704in}}%
\pgfpathlineto{\pgfqpoint{5.328608in}{2.321656in}}%
\pgfpathlineto{\pgfqpoint{5.313961in}{2.316678in}}%
\pgfpathlineto{\pgfqpoint{5.299328in}{2.311772in}}%
\pgfpathlineto{\pgfqpoint{5.291659in}{2.304763in}}%
\pgfpathlineto{\pgfqpoint{5.283980in}{2.297626in}}%
\pgfpathlineto{\pgfqpoint{5.276293in}{2.290360in}}%
\pgfpathlineto{\pgfqpoint{5.268597in}{2.282961in}}%
\pgfpathclose%
\pgfusepath{fill}%
\end{pgfscope}%
\begin{pgfscope}%
\pgfpathrectangle{\pgfqpoint{1.150000in}{0.150000in}}{\pgfqpoint{5.700000in}{5.700000in}}%
\pgfusepath{clip}%
\pgfsetbuttcap%
\pgfsetroundjoin%
\definecolor{currentfill}{rgb}{0.179019,0.433756,0.557430}%
\pgfsetfillcolor{currentfill}%
\pgfsetfillopacity{0.700000}%
\pgfsetlinewidth{0.000000pt}%
\definecolor{currentstroke}{rgb}{0.000000,0.000000,0.000000}%
\pgfsetstrokecolor{currentstroke}%
\pgfsetdash{}{0pt}%
\pgfpathmoveto{\pgfqpoint{5.357941in}{2.331824in}}%
\pgfpathlineto{\pgfqpoint{5.372628in}{2.337015in}}%
\pgfpathlineto{\pgfqpoint{5.387329in}{2.342278in}}%
\pgfpathlineto{\pgfqpoint{5.402044in}{2.347611in}}%
\pgfpathlineto{\pgfqpoint{5.416773in}{2.353017in}}%
\pgfpathlineto{\pgfqpoint{5.424406in}{2.359727in}}%
\pgfpathlineto{\pgfqpoint{5.432029in}{2.366303in}}%
\pgfpathlineto{\pgfqpoint{5.439643in}{2.372745in}}%
\pgfpathlineto{\pgfqpoint{5.447247in}{2.379055in}}%
\pgfpathlineto{\pgfqpoint{5.432533in}{2.373744in}}%
\pgfpathlineto{\pgfqpoint{5.417832in}{2.368504in}}%
\pgfpathlineto{\pgfqpoint{5.403145in}{2.363336in}}%
\pgfpathlineto{\pgfqpoint{5.388472in}{2.358239in}}%
\pgfpathlineto{\pgfqpoint{5.380853in}{2.351826in}}%
\pgfpathlineto{\pgfqpoint{5.373225in}{2.345288in}}%
\pgfpathlineto{\pgfqpoint{5.365588in}{2.338621in}}%
\pgfpathlineto{\pgfqpoint{5.357941in}{2.331824in}}%
\pgfpathclose%
\pgfusepath{fill}%
\end{pgfscope}%
\begin{pgfscope}%
\pgfpathrectangle{\pgfqpoint{1.150000in}{0.150000in}}{\pgfqpoint{5.700000in}{5.700000in}}%
\pgfusepath{clip}%
\pgfsetbuttcap%
\pgfsetroundjoin%
\definecolor{currentfill}{rgb}{0.268510,0.009605,0.335427}%
\pgfsetfillcolor{currentfill}%
\pgfsetfillopacity{0.700000}%
\pgfsetlinewidth{0.000000pt}%
\definecolor{currentstroke}{rgb}{0.000000,0.000000,0.000000}%
\pgfsetstrokecolor{currentstroke}%
\pgfsetdash{}{0pt}%
\pgfpathmoveto{\pgfqpoint{3.571364in}{1.388269in}}%
\pgfpathlineto{\pgfqpoint{3.585400in}{1.385044in}}%
\pgfpathlineto{\pgfqpoint{3.599442in}{1.381894in}}%
\pgfpathlineto{\pgfqpoint{3.613491in}{1.378819in}}%
\pgfpathlineto{\pgfqpoint{3.627546in}{1.375818in}}%
\pgfpathlineto{\pgfqpoint{3.635894in}{1.384993in}}%
\pgfpathlineto{\pgfqpoint{3.644235in}{1.394264in}}%
\pgfpathlineto{\pgfqpoint{3.652568in}{1.403628in}}%
\pgfpathlineto{\pgfqpoint{3.660895in}{1.413079in}}%
\pgfpathlineto{\pgfqpoint{3.646855in}{1.415750in}}%
\pgfpathlineto{\pgfqpoint{3.632821in}{1.418496in}}%
\pgfpathlineto{\pgfqpoint{3.618793in}{1.421316in}}%
\pgfpathlineto{\pgfqpoint{3.604773in}{1.424211in}}%
\pgfpathlineto{\pgfqpoint{3.596431in}{1.415082in}}%
\pgfpathlineto{\pgfqpoint{3.588083in}{1.406045in}}%
\pgfpathlineto{\pgfqpoint{3.579727in}{1.397106in}}%
\pgfpathlineto{\pgfqpoint{3.571364in}{1.388269in}}%
\pgfpathclose%
\pgfusepath{fill}%
\end{pgfscope}%
\begin{pgfscope}%
\pgfpathrectangle{\pgfqpoint{1.150000in}{0.150000in}}{\pgfqpoint{5.700000in}{5.700000in}}%
\pgfusepath{clip}%
\pgfsetbuttcap%
\pgfsetroundjoin%
\definecolor{currentfill}{rgb}{0.283091,0.110553,0.431554}%
\pgfsetfillcolor{currentfill}%
\pgfsetfillopacity{0.700000}%
\pgfsetlinewidth{0.000000pt}%
\definecolor{currentstroke}{rgb}{0.000000,0.000000,0.000000}%
\pgfsetstrokecolor{currentstroke}%
\pgfsetdash{}{0pt}%
\pgfpathmoveto{\pgfqpoint{4.074737in}{1.556860in}}%
\pgfpathlineto{\pgfqpoint{4.088900in}{1.556841in}}%
\pgfpathlineto{\pgfqpoint{4.103072in}{1.556893in}}%
\pgfpathlineto{\pgfqpoint{4.117253in}{1.557017in}}%
\pgfpathlineto{\pgfqpoint{4.131443in}{1.557214in}}%
\pgfpathlineto{\pgfqpoint{4.139608in}{1.568902in}}%
\pgfpathlineto{\pgfqpoint{4.147768in}{1.580574in}}%
\pgfpathlineto{\pgfqpoint{4.155923in}{1.592227in}}%
\pgfpathlineto{\pgfqpoint{4.164072in}{1.603856in}}%
\pgfpathlineto{\pgfqpoint{4.149889in}{1.603431in}}%
\pgfpathlineto{\pgfqpoint{4.135716in}{1.603078in}}%
\pgfpathlineto{\pgfqpoint{4.121552in}{1.602797in}}%
\pgfpathlineto{\pgfqpoint{4.107397in}{1.602589in}}%
\pgfpathlineto{\pgfqpoint{4.099240in}{1.591180in}}%
\pgfpathlineto{\pgfqpoint{4.091077in}{1.579753in}}%
\pgfpathlineto{\pgfqpoint{4.082910in}{1.568312in}}%
\pgfpathlineto{\pgfqpoint{4.074737in}{1.556860in}}%
\pgfpathclose%
\pgfusepath{fill}%
\end{pgfscope}%
\begin{pgfscope}%
\pgfpathrectangle{\pgfqpoint{1.150000in}{0.150000in}}{\pgfqpoint{5.700000in}{5.700000in}}%
\pgfusepath{clip}%
\pgfsetbuttcap%
\pgfsetroundjoin%
\definecolor{currentfill}{rgb}{0.283072,0.130895,0.449241}%
\pgfsetfillcolor{currentfill}%
\pgfsetfillopacity{0.700000}%
\pgfsetlinewidth{0.000000pt}%
\definecolor{currentstroke}{rgb}{0.000000,0.000000,0.000000}%
\pgfsetstrokecolor{currentstroke}%
\pgfsetdash{}{0pt}%
\pgfpathmoveto{\pgfqpoint{4.164072in}{1.603856in}}%
\pgfpathlineto{\pgfqpoint{4.178264in}{1.604353in}}%
\pgfpathlineto{\pgfqpoint{4.192465in}{1.604922in}}%
\pgfpathlineto{\pgfqpoint{4.206676in}{1.605562in}}%
\pgfpathlineto{\pgfqpoint{4.220896in}{1.606274in}}%
\pgfpathlineto{\pgfqpoint{4.229033in}{1.618094in}}%
\pgfpathlineto{\pgfqpoint{4.237165in}{1.629880in}}%
\pgfpathlineto{\pgfqpoint{4.245292in}{1.641628in}}%
\pgfpathlineto{\pgfqpoint{4.253413in}{1.653337in}}%
\pgfpathlineto{\pgfqpoint{4.239199in}{1.652416in}}%
\pgfpathlineto{\pgfqpoint{4.224995in}{1.651568in}}%
\pgfpathlineto{\pgfqpoint{4.210801in}{1.650791in}}%
\pgfpathlineto{\pgfqpoint{4.196617in}{1.650086in}}%
\pgfpathlineto{\pgfqpoint{4.188488in}{1.638578in}}%
\pgfpathlineto{\pgfqpoint{4.180355in}{1.627035in}}%
\pgfpathlineto{\pgfqpoint{4.172216in}{1.615460in}}%
\pgfpathlineto{\pgfqpoint{4.164072in}{1.603856in}}%
\pgfpathclose%
\pgfusepath{fill}%
\end{pgfscope}%
\begin{pgfscope}%
\pgfpathrectangle{\pgfqpoint{1.150000in}{0.150000in}}{\pgfqpoint{5.700000in}{5.700000in}}%
\pgfusepath{clip}%
\pgfsetbuttcap%
\pgfsetroundjoin%
\definecolor{currentfill}{rgb}{0.268510,0.009605,0.335427}%
\pgfsetfillcolor{currentfill}%
\pgfsetfillopacity{0.700000}%
\pgfsetlinewidth{0.000000pt}%
\definecolor{currentstroke}{rgb}{0.000000,0.000000,0.000000}%
\pgfsetstrokecolor{currentstroke}%
\pgfsetdash{}{0pt}%
\pgfpathmoveto{\pgfqpoint{3.189706in}{1.394881in}}%
\pgfpathlineto{\pgfqpoint{3.203695in}{1.389003in}}%
\pgfpathlineto{\pgfqpoint{3.217688in}{1.383205in}}%
\pgfpathlineto{\pgfqpoint{3.231686in}{1.377486in}}%
\pgfpathlineto{\pgfqpoint{3.245688in}{1.371846in}}%
\pgfpathlineto{\pgfqpoint{3.254236in}{1.377440in}}%
\pgfpathlineto{\pgfqpoint{3.262773in}{1.383224in}}%
\pgfpathlineto{\pgfqpoint{3.271300in}{1.389191in}}%
\pgfpathlineto{\pgfqpoint{3.279817in}{1.395335in}}%
\pgfpathlineto{\pgfqpoint{3.265838in}{1.400583in}}%
\pgfpathlineto{\pgfqpoint{3.251864in}{1.405911in}}%
\pgfpathlineto{\pgfqpoint{3.237894in}{1.411317in}}%
\pgfpathlineto{\pgfqpoint{3.223929in}{1.416802in}}%
\pgfpathlineto{\pgfqpoint{3.215389in}{1.411042in}}%
\pgfpathlineto{\pgfqpoint{3.206839in}{1.405464in}}%
\pgfpathlineto{\pgfqpoint{3.198278in}{1.400075in}}%
\pgfpathlineto{\pgfqpoint{3.189706in}{1.394881in}}%
\pgfpathclose%
\pgfusepath{fill}%
\end{pgfscope}%
\begin{pgfscope}%
\pgfpathrectangle{\pgfqpoint{1.150000in}{0.150000in}}{\pgfqpoint{5.700000in}{5.700000in}}%
\pgfusepath{clip}%
\pgfsetbuttcap%
\pgfsetroundjoin%
\definecolor{currentfill}{rgb}{0.281924,0.089666,0.412415}%
\pgfsetfillcolor{currentfill}%
\pgfsetfillopacity{0.700000}%
\pgfsetlinewidth{0.000000pt}%
\definecolor{currentstroke}{rgb}{0.000000,0.000000,0.000000}%
\pgfsetstrokecolor{currentstroke}%
\pgfsetdash{}{0pt}%
\pgfpathmoveto{\pgfqpoint{3.985393in}{1.512803in}}%
\pgfpathlineto{\pgfqpoint{3.999530in}{1.512246in}}%
\pgfpathlineto{\pgfqpoint{4.013675in}{1.511761in}}%
\pgfpathlineto{\pgfqpoint{4.027829in}{1.511348in}}%
\pgfpathlineto{\pgfqpoint{4.041992in}{1.511007in}}%
\pgfpathlineto{\pgfqpoint{4.050186in}{1.522470in}}%
\pgfpathlineto{\pgfqpoint{4.058375in}{1.533936in}}%
\pgfpathlineto{\pgfqpoint{4.066559in}{1.545400in}}%
\pgfpathlineto{\pgfqpoint{4.074737in}{1.556860in}}%
\pgfpathlineto{\pgfqpoint{4.060583in}{1.556952in}}%
\pgfpathlineto{\pgfqpoint{4.046437in}{1.557116in}}%
\pgfpathlineto{\pgfqpoint{4.032301in}{1.557352in}}%
\pgfpathlineto{\pgfqpoint{4.018173in}{1.557661in}}%
\pgfpathlineto{\pgfqpoint{4.009986in}{1.546442in}}%
\pgfpathlineto{\pgfqpoint{4.001794in}{1.535224in}}%
\pgfpathlineto{\pgfqpoint{3.993596in}{1.524010in}}%
\pgfpathlineto{\pgfqpoint{3.985393in}{1.512803in}}%
\pgfpathclose%
\pgfusepath{fill}%
\end{pgfscope}%
\begin{pgfscope}%
\pgfpathrectangle{\pgfqpoint{1.150000in}{0.150000in}}{\pgfqpoint{5.700000in}{5.700000in}}%
\pgfusepath{clip}%
\pgfsetbuttcap%
\pgfsetroundjoin%
\definecolor{currentfill}{rgb}{0.281412,0.155834,0.469201}%
\pgfsetfillcolor{currentfill}%
\pgfsetfillopacity{0.700000}%
\pgfsetlinewidth{0.000000pt}%
\definecolor{currentstroke}{rgb}{0.000000,0.000000,0.000000}%
\pgfsetstrokecolor{currentstroke}%
\pgfsetdash{}{0pt}%
\pgfpathmoveto{\pgfqpoint{4.253413in}{1.653337in}}%
\pgfpathlineto{\pgfqpoint{4.267636in}{1.654329in}}%
\pgfpathlineto{\pgfqpoint{4.281869in}{1.655392in}}%
\pgfpathlineto{\pgfqpoint{4.296111in}{1.656528in}}%
\pgfpathlineto{\pgfqpoint{4.310364in}{1.657735in}}%
\pgfpathlineto{\pgfqpoint{4.318474in}{1.669596in}}%
\pgfpathlineto{\pgfqpoint{4.326577in}{1.681407in}}%
\pgfpathlineto{\pgfqpoint{4.334676in}{1.693165in}}%
\pgfpathlineto{\pgfqpoint{4.342769in}{1.704867in}}%
\pgfpathlineto{\pgfqpoint{4.328523in}{1.703472in}}%
\pgfpathlineto{\pgfqpoint{4.314287in}{1.702149in}}%
\pgfpathlineto{\pgfqpoint{4.300060in}{1.700898in}}%
\pgfpathlineto{\pgfqpoint{4.285844in}{1.699719in}}%
\pgfpathlineto{\pgfqpoint{4.277744in}{1.688196in}}%
\pgfpathlineto{\pgfqpoint{4.269639in}{1.676623in}}%
\pgfpathlineto{\pgfqpoint{4.261529in}{1.665002in}}%
\pgfpathlineto{\pgfqpoint{4.253413in}{1.653337in}}%
\pgfpathclose%
\pgfusepath{fill}%
\end{pgfscope}%
\begin{pgfscope}%
\pgfpathrectangle{\pgfqpoint{1.150000in}{0.150000in}}{\pgfqpoint{5.700000in}{5.700000in}}%
\pgfusepath{clip}%
\pgfsetbuttcap%
\pgfsetroundjoin%
\definecolor{currentfill}{rgb}{0.267004,0.004874,0.329415}%
\pgfsetfillcolor{currentfill}%
\pgfsetfillopacity{0.700000}%
\pgfsetlinewidth{0.000000pt}%
\definecolor{currentstroke}{rgb}{0.000000,0.000000,0.000000}%
\pgfsetstrokecolor{currentstroke}%
\pgfsetdash{}{0pt}%
\pgfpathmoveto{\pgfqpoint{3.335782in}{1.375125in}}%
\pgfpathlineto{\pgfqpoint{3.349785in}{1.370266in}}%
\pgfpathlineto{\pgfqpoint{3.363794in}{1.365485in}}%
\pgfpathlineto{\pgfqpoint{3.377808in}{1.360781in}}%
\pgfpathlineto{\pgfqpoint{3.391827in}{1.356154in}}%
\pgfpathlineto{\pgfqpoint{3.400292in}{1.363228in}}%
\pgfpathlineto{\pgfqpoint{3.408748in}{1.370456in}}%
\pgfpathlineto{\pgfqpoint{3.417194in}{1.377835in}}%
\pgfpathlineto{\pgfqpoint{3.425632in}{1.385357in}}%
\pgfpathlineto{\pgfqpoint{3.411633in}{1.389614in}}%
\pgfpathlineto{\pgfqpoint{3.397639in}{1.393947in}}%
\pgfpathlineto{\pgfqpoint{3.383650in}{1.398357in}}%
\pgfpathlineto{\pgfqpoint{3.369667in}{1.402845in}}%
\pgfpathlineto{\pgfqpoint{3.361209in}{1.395685in}}%
\pgfpathlineto{\pgfqpoint{3.352743in}{1.388675in}}%
\pgfpathlineto{\pgfqpoint{3.344267in}{1.381819in}}%
\pgfpathlineto{\pgfqpoint{3.335782in}{1.375125in}}%
\pgfpathclose%
\pgfusepath{fill}%
\end{pgfscope}%
\begin{pgfscope}%
\pgfpathrectangle{\pgfqpoint{1.150000in}{0.150000in}}{\pgfqpoint{5.700000in}{5.700000in}}%
\pgfusepath{clip}%
\pgfsetbuttcap%
\pgfsetroundjoin%
\definecolor{currentfill}{rgb}{0.278012,0.180367,0.486697}%
\pgfsetfillcolor{currentfill}%
\pgfsetfillopacity{0.700000}%
\pgfsetlinewidth{0.000000pt}%
\definecolor{currentstroke}{rgb}{0.000000,0.000000,0.000000}%
\pgfsetstrokecolor{currentstroke}%
\pgfsetdash{}{0pt}%
\pgfpathmoveto{\pgfqpoint{4.342769in}{1.704867in}}%
\pgfpathlineto{\pgfqpoint{4.357026in}{1.706333in}}%
\pgfpathlineto{\pgfqpoint{4.371292in}{1.707870in}}%
\pgfpathlineto{\pgfqpoint{4.385569in}{1.709480in}}%
\pgfpathlineto{\pgfqpoint{4.399856in}{1.711160in}}%
\pgfpathlineto{\pgfqpoint{4.407938in}{1.722979in}}%
\pgfpathlineto{\pgfqpoint{4.416014in}{1.734732in}}%
\pgfpathlineto{\pgfqpoint{4.424085in}{1.746418in}}%
\pgfpathlineto{\pgfqpoint{4.432150in}{1.758034in}}%
\pgfpathlineto{\pgfqpoint{4.417868in}{1.756186in}}%
\pgfpathlineto{\pgfqpoint{4.403598in}{1.754410in}}%
\pgfpathlineto{\pgfqpoint{4.389337in}{1.752706in}}%
\pgfpathlineto{\pgfqpoint{4.375087in}{1.751073in}}%
\pgfpathlineto{\pgfqpoint{4.367016in}{1.739616in}}%
\pgfpathlineto{\pgfqpoint{4.358939in}{1.728095in}}%
\pgfpathlineto{\pgfqpoint{4.350857in}{1.716511in}}%
\pgfpathlineto{\pgfqpoint{4.342769in}{1.704867in}}%
\pgfpathclose%
\pgfusepath{fill}%
\end{pgfscope}%
\begin{pgfscope}%
\pgfpathrectangle{\pgfqpoint{1.150000in}{0.150000in}}{\pgfqpoint{5.700000in}{5.700000in}}%
\pgfusepath{clip}%
\pgfsetbuttcap%
\pgfsetroundjoin%
\definecolor{currentfill}{rgb}{0.279566,0.067836,0.391917}%
\pgfsetfillcolor{currentfill}%
\pgfsetfillopacity{0.700000}%
\pgfsetlinewidth{0.000000pt}%
\definecolor{currentstroke}{rgb}{0.000000,0.000000,0.000000}%
\pgfsetstrokecolor{currentstroke}%
\pgfsetdash{}{0pt}%
\pgfpathmoveto{\pgfqpoint{3.896025in}{1.472161in}}%
\pgfpathlineto{\pgfqpoint{3.910138in}{1.471044in}}%
\pgfpathlineto{\pgfqpoint{3.924259in}{1.470000in}}%
\pgfpathlineto{\pgfqpoint{3.938389in}{1.469028in}}%
\pgfpathlineto{\pgfqpoint{3.952527in}{1.468129in}}%
\pgfpathlineto{\pgfqpoint{3.960752in}{1.479267in}}%
\pgfpathlineto{\pgfqpoint{3.968971in}{1.490428in}}%
\pgfpathlineto{\pgfqpoint{3.977185in}{1.501608in}}%
\pgfpathlineto{\pgfqpoint{3.985393in}{1.512803in}}%
\pgfpathlineto{\pgfqpoint{3.971265in}{1.513433in}}%
\pgfpathlineto{\pgfqpoint{3.957145in}{1.514135in}}%
\pgfpathlineto{\pgfqpoint{3.943034in}{1.514911in}}%
\pgfpathlineto{\pgfqpoint{3.928931in}{1.515758in}}%
\pgfpathlineto{\pgfqpoint{3.920713in}{1.504825in}}%
\pgfpathlineto{\pgfqpoint{3.912489in}{1.493911in}}%
\pgfpathlineto{\pgfqpoint{3.904260in}{1.483022in}}%
\pgfpathlineto{\pgfqpoint{3.896025in}{1.472161in}}%
\pgfpathclose%
\pgfusepath{fill}%
\end{pgfscope}%
\begin{pgfscope}%
\pgfpathrectangle{\pgfqpoint{1.150000in}{0.150000in}}{\pgfqpoint{5.700000in}{5.700000in}}%
\pgfusepath{clip}%
\pgfsetbuttcap%
\pgfsetroundjoin%
\definecolor{currentfill}{rgb}{0.272594,0.025563,0.353093}%
\pgfsetfillcolor{currentfill}%
\pgfsetfillopacity{0.700000}%
\pgfsetlinewidth{0.000000pt}%
\definecolor{currentstroke}{rgb}{0.000000,0.000000,0.000000}%
\pgfsetstrokecolor{currentstroke}%
\pgfsetdash{}{0pt}%
\pgfpathmoveto{\pgfqpoint{3.043322in}{1.429399in}}%
\pgfpathlineto{\pgfqpoint{3.057307in}{1.422462in}}%
\pgfpathlineto{\pgfqpoint{3.071295in}{1.415606in}}%
\pgfpathlineto{\pgfqpoint{3.085287in}{1.408833in}}%
\pgfpathlineto{\pgfqpoint{3.099283in}{1.402140in}}%
\pgfpathlineto{\pgfqpoint{3.107928in}{1.406071in}}%
\pgfpathlineto{\pgfqpoint{3.116561in}{1.410227in}}%
\pgfpathlineto{\pgfqpoint{3.125182in}{1.414603in}}%
\pgfpathlineto{\pgfqpoint{3.133791in}{1.419190in}}%
\pgfpathlineto{\pgfqpoint{3.119823in}{1.425469in}}%
\pgfpathlineto{\pgfqpoint{3.105858in}{1.431829in}}%
\pgfpathlineto{\pgfqpoint{3.091897in}{1.438271in}}%
\pgfpathlineto{\pgfqpoint{3.077940in}{1.444794in}}%
\pgfpathlineto{\pgfqpoint{3.069305in}{1.440612in}}%
\pgfpathlineto{\pgfqpoint{3.060657in}{1.436648in}}%
\pgfpathlineto{\pgfqpoint{3.051996in}{1.432908in}}%
\pgfpathlineto{\pgfqpoint{3.043322in}{1.429399in}}%
\pgfpathclose%
\pgfusepath{fill}%
\end{pgfscope}%
\begin{pgfscope}%
\pgfpathrectangle{\pgfqpoint{1.150000in}{0.150000in}}{\pgfqpoint{5.700000in}{5.700000in}}%
\pgfusepath{clip}%
\pgfsetbuttcap%
\pgfsetroundjoin%
\definecolor{currentfill}{rgb}{0.271828,0.209303,0.504434}%
\pgfsetfillcolor{currentfill}%
\pgfsetfillopacity{0.700000}%
\pgfsetlinewidth{0.000000pt}%
\definecolor{currentstroke}{rgb}{0.000000,0.000000,0.000000}%
\pgfsetstrokecolor{currentstroke}%
\pgfsetdash{}{0pt}%
\pgfpathmoveto{\pgfqpoint{4.432150in}{1.758034in}}%
\pgfpathlineto{\pgfqpoint{4.446441in}{1.759952in}}%
\pgfpathlineto{\pgfqpoint{4.460743in}{1.761943in}}%
\pgfpathlineto{\pgfqpoint{4.475056in}{1.764004in}}%
\pgfpathlineto{\pgfqpoint{4.489380in}{1.766137in}}%
\pgfpathlineto{\pgfqpoint{4.497433in}{1.777835in}}%
\pgfpathlineto{\pgfqpoint{4.505481in}{1.789453in}}%
\pgfpathlineto{\pgfqpoint{4.513522in}{1.800990in}}%
\pgfpathlineto{\pgfqpoint{4.521559in}{1.812444in}}%
\pgfpathlineto{\pgfqpoint{4.507241in}{1.810165in}}%
\pgfpathlineto{\pgfqpoint{4.492934in}{1.807957in}}%
\pgfpathlineto{\pgfqpoint{4.478638in}{1.805821in}}%
\pgfpathlineto{\pgfqpoint{4.464353in}{1.803756in}}%
\pgfpathlineto{\pgfqpoint{4.456310in}{1.792440in}}%
\pgfpathlineto{\pgfqpoint{4.448263in}{1.781047in}}%
\pgfpathlineto{\pgfqpoint{4.440209in}{1.769577in}}%
\pgfpathlineto{\pgfqpoint{4.432150in}{1.758034in}}%
\pgfpathclose%
\pgfusepath{fill}%
\end{pgfscope}%
\begin{pgfscope}%
\pgfpathrectangle{\pgfqpoint{1.150000in}{0.150000in}}{\pgfqpoint{5.700000in}{5.700000in}}%
\pgfusepath{clip}%
\pgfsetbuttcap%
\pgfsetroundjoin%
\definecolor{currentfill}{rgb}{0.276022,0.044167,0.370164}%
\pgfsetfillcolor{currentfill}%
\pgfsetfillopacity{0.700000}%
\pgfsetlinewidth{0.000000pt}%
\definecolor{currentstroke}{rgb}{0.000000,0.000000,0.000000}%
\pgfsetstrokecolor{currentstroke}%
\pgfsetdash{}{0pt}%
\pgfpathmoveto{\pgfqpoint{3.806611in}{1.435432in}}%
\pgfpathlineto{\pgfqpoint{3.820704in}{1.433734in}}%
\pgfpathlineto{\pgfqpoint{3.834804in}{1.432109in}}%
\pgfpathlineto{\pgfqpoint{3.848912in}{1.430557in}}%
\pgfpathlineto{\pgfqpoint{3.863028in}{1.429078in}}%
\pgfpathlineto{\pgfqpoint{3.871286in}{1.439786in}}%
\pgfpathlineto{\pgfqpoint{3.879538in}{1.450539in}}%
\pgfpathlineto{\pgfqpoint{3.887784in}{1.461332in}}%
\pgfpathlineto{\pgfqpoint{3.896025in}{1.472161in}}%
\pgfpathlineto{\pgfqpoint{3.881920in}{1.473351in}}%
\pgfpathlineto{\pgfqpoint{3.867823in}{1.474613in}}%
\pgfpathlineto{\pgfqpoint{3.853733in}{1.475949in}}%
\pgfpathlineto{\pgfqpoint{3.839652in}{1.477357in}}%
\pgfpathlineto{\pgfqpoint{3.831400in}{1.466810in}}%
\pgfpathlineto{\pgfqpoint{3.823143in}{1.456304in}}%
\pgfpathlineto{\pgfqpoint{3.814880in}{1.445843in}}%
\pgfpathlineto{\pgfqpoint{3.806611in}{1.435432in}}%
\pgfpathclose%
\pgfusepath{fill}%
\end{pgfscope}%
\begin{pgfscope}%
\pgfpathrectangle{\pgfqpoint{1.150000in}{0.150000in}}{\pgfqpoint{5.700000in}{5.700000in}}%
\pgfusepath{clip}%
\pgfsetbuttcap%
\pgfsetroundjoin%
\definecolor{currentfill}{rgb}{0.267004,0.004874,0.329415}%
\pgfsetfillcolor{currentfill}%
\pgfsetfillopacity{0.700000}%
\pgfsetlinewidth{0.000000pt}%
\definecolor{currentstroke}{rgb}{0.000000,0.000000,0.000000}%
\pgfsetstrokecolor{currentstroke}%
\pgfsetdash{}{0pt}%
\pgfpathmoveto{\pgfqpoint{3.481687in}{1.369095in}}%
\pgfpathlineto{\pgfqpoint{3.495715in}{1.365219in}}%
\pgfpathlineto{\pgfqpoint{3.509749in}{1.361419in}}%
\pgfpathlineto{\pgfqpoint{3.523789in}{1.357694in}}%
\pgfpathlineto{\pgfqpoint{3.537835in}{1.354044in}}%
\pgfpathlineto{\pgfqpoint{3.546229in}{1.362421in}}%
\pgfpathlineto{\pgfqpoint{3.554615in}{1.370921in}}%
\pgfpathlineto{\pgfqpoint{3.562993in}{1.379539in}}%
\pgfpathlineto{\pgfqpoint{3.571364in}{1.388269in}}%
\pgfpathlineto{\pgfqpoint{3.557334in}{1.391568in}}%
\pgfpathlineto{\pgfqpoint{3.543311in}{1.394943in}}%
\pgfpathlineto{\pgfqpoint{3.529294in}{1.398393in}}%
\pgfpathlineto{\pgfqpoint{3.515283in}{1.401919in}}%
\pgfpathlineto{\pgfqpoint{3.506896in}{1.393531in}}%
\pgfpathlineto{\pgfqpoint{3.498501in}{1.385261in}}%
\pgfpathlineto{\pgfqpoint{3.490098in}{1.377114in}}%
\pgfpathlineto{\pgfqpoint{3.481687in}{1.369095in}}%
\pgfpathclose%
\pgfusepath{fill}%
\end{pgfscope}%
\begin{pgfscope}%
\pgfpathrectangle{\pgfqpoint{1.150000in}{0.150000in}}{\pgfqpoint{5.700000in}{5.700000in}}%
\pgfusepath{clip}%
\pgfsetbuttcap%
\pgfsetroundjoin%
\definecolor{currentfill}{rgb}{0.265145,0.232956,0.516599}%
\pgfsetfillcolor{currentfill}%
\pgfsetfillopacity{0.700000}%
\pgfsetlinewidth{0.000000pt}%
\definecolor{currentstroke}{rgb}{0.000000,0.000000,0.000000}%
\pgfsetstrokecolor{currentstroke}%
\pgfsetdash{}{0pt}%
\pgfpathmoveto{\pgfqpoint{4.521559in}{1.812444in}}%
\pgfpathlineto{\pgfqpoint{4.535887in}{1.814794in}}%
\pgfpathlineto{\pgfqpoint{4.550226in}{1.817216in}}%
\pgfpathlineto{\pgfqpoint{4.564577in}{1.819709in}}%
\pgfpathlineto{\pgfqpoint{4.578938in}{1.822274in}}%
\pgfpathlineto{\pgfqpoint{4.586962in}{1.833775in}}%
\pgfpathlineto{\pgfqpoint{4.594981in}{1.845186in}}%
\pgfpathlineto{\pgfqpoint{4.602993in}{1.856503in}}%
\pgfpathlineto{\pgfqpoint{4.610999in}{1.867725in}}%
\pgfpathlineto{\pgfqpoint{4.596643in}{1.865035in}}%
\pgfpathlineto{\pgfqpoint{4.582299in}{1.862417in}}%
\pgfpathlineto{\pgfqpoint{4.567966in}{1.859871in}}%
\pgfpathlineto{\pgfqpoint{4.553643in}{1.857395in}}%
\pgfpathlineto{\pgfqpoint{4.545631in}{1.846290in}}%
\pgfpathlineto{\pgfqpoint{4.537613in}{1.835095in}}%
\pgfpathlineto{\pgfqpoint{4.529589in}{1.823813in}}%
\pgfpathlineto{\pgfqpoint{4.521559in}{1.812444in}}%
\pgfpathclose%
\pgfusepath{fill}%
\end{pgfscope}%
\begin{pgfscope}%
\pgfpathrectangle{\pgfqpoint{1.150000in}{0.150000in}}{\pgfqpoint{5.700000in}{5.700000in}}%
\pgfusepath{clip}%
\pgfsetbuttcap%
\pgfsetroundjoin%
\definecolor{currentfill}{rgb}{0.255645,0.260703,0.528312}%
\pgfsetfillcolor{currentfill}%
\pgfsetfillopacity{0.700000}%
\pgfsetlinewidth{0.000000pt}%
\definecolor{currentstroke}{rgb}{0.000000,0.000000,0.000000}%
\pgfsetstrokecolor{currentstroke}%
\pgfsetdash{}{0pt}%
\pgfpathmoveto{\pgfqpoint{4.610999in}{1.867725in}}%
\pgfpathlineto{\pgfqpoint{4.625365in}{1.870485in}}%
\pgfpathlineto{\pgfqpoint{4.639743in}{1.873317in}}%
\pgfpathlineto{\pgfqpoint{4.654133in}{1.876221in}}%
\pgfpathlineto{\pgfqpoint{4.668533in}{1.879195in}}%
\pgfpathlineto{\pgfqpoint{4.676527in}{1.890433in}}%
\pgfpathlineto{\pgfqpoint{4.684515in}{1.901567in}}%
\pgfpathlineto{\pgfqpoint{4.692496in}{1.912598in}}%
\pgfpathlineto{\pgfqpoint{4.700470in}{1.923523in}}%
\pgfpathlineto{\pgfqpoint{4.686076in}{1.920445in}}%
\pgfpathlineto{\pgfqpoint{4.671692in}{1.917438in}}%
\pgfpathlineto{\pgfqpoint{4.657321in}{1.914503in}}%
\pgfpathlineto{\pgfqpoint{4.642960in}{1.911638in}}%
\pgfpathlineto{\pgfqpoint{4.634979in}{1.900809in}}%
\pgfpathlineto{\pgfqpoint{4.626992in}{1.889879in}}%
\pgfpathlineto{\pgfqpoint{4.618998in}{1.878850in}}%
\pgfpathlineto{\pgfqpoint{4.610999in}{1.867725in}}%
\pgfpathclose%
\pgfusepath{fill}%
\end{pgfscope}%
\begin{pgfscope}%
\pgfpathrectangle{\pgfqpoint{1.150000in}{0.150000in}}{\pgfqpoint{5.700000in}{5.700000in}}%
\pgfusepath{clip}%
\pgfsetbuttcap%
\pgfsetroundjoin%
\definecolor{currentfill}{rgb}{0.273809,0.031497,0.358853}%
\pgfsetfillcolor{currentfill}%
\pgfsetfillopacity{0.700000}%
\pgfsetlinewidth{0.000000pt}%
\definecolor{currentstroke}{rgb}{0.000000,0.000000,0.000000}%
\pgfsetstrokecolor{currentstroke}%
\pgfsetdash{}{0pt}%
\pgfpathmoveto{\pgfqpoint{3.717127in}{1.403137in}}%
\pgfpathlineto{\pgfqpoint{3.731203in}{1.400836in}}%
\pgfpathlineto{\pgfqpoint{3.745286in}{1.398608in}}%
\pgfpathlineto{\pgfqpoint{3.759376in}{1.396454in}}%
\pgfpathlineto{\pgfqpoint{3.773473in}{1.394373in}}%
\pgfpathlineto{\pgfqpoint{3.781767in}{1.404541in}}%
\pgfpathlineto{\pgfqpoint{3.790054in}{1.414777in}}%
\pgfpathlineto{\pgfqpoint{3.798336in}{1.425075in}}%
\pgfpathlineto{\pgfqpoint{3.806611in}{1.435432in}}%
\pgfpathlineto{\pgfqpoint{3.792526in}{1.437203in}}%
\pgfpathlineto{\pgfqpoint{3.778448in}{1.439048in}}%
\pgfpathlineto{\pgfqpoint{3.764378in}{1.440966in}}%
\pgfpathlineto{\pgfqpoint{3.750315in}{1.442957in}}%
\pgfpathlineto{\pgfqpoint{3.742028in}{1.432902in}}%
\pgfpathlineto{\pgfqpoint{3.733734in}{1.422911in}}%
\pgfpathlineto{\pgfqpoint{3.725434in}{1.412987in}}%
\pgfpathlineto{\pgfqpoint{3.717127in}{1.403137in}}%
\pgfpathclose%
\pgfusepath{fill}%
\end{pgfscope}%
\begin{pgfscope}%
\pgfpathrectangle{\pgfqpoint{1.150000in}{0.150000in}}{\pgfqpoint{5.700000in}{5.700000in}}%
\pgfusepath{clip}%
\pgfsetbuttcap%
\pgfsetroundjoin%
\definecolor{currentfill}{rgb}{0.246811,0.283237,0.535941}%
\pgfsetfillcolor{currentfill}%
\pgfsetfillopacity{0.700000}%
\pgfsetlinewidth{0.000000pt}%
\definecolor{currentstroke}{rgb}{0.000000,0.000000,0.000000}%
\pgfsetstrokecolor{currentstroke}%
\pgfsetdash{}{0pt}%
\pgfpathmoveto{\pgfqpoint{4.700470in}{1.923523in}}%
\pgfpathlineto{\pgfqpoint{4.714876in}{1.926673in}}%
\pgfpathlineto{\pgfqpoint{4.729294in}{1.929894in}}%
\pgfpathlineto{\pgfqpoint{4.743724in}{1.933186in}}%
\pgfpathlineto{\pgfqpoint{4.758165in}{1.936550in}}%
\pgfpathlineto{\pgfqpoint{4.766127in}{1.947459in}}%
\pgfpathlineto{\pgfqpoint{4.774082in}{1.958256in}}%
\pgfpathlineto{\pgfqpoint{4.782030in}{1.968939in}}%
\pgfpathlineto{\pgfqpoint{4.789971in}{1.979508in}}%
\pgfpathlineto{\pgfqpoint{4.775536in}{1.976062in}}%
\pgfpathlineto{\pgfqpoint{4.761113in}{1.972688in}}%
\pgfpathlineto{\pgfqpoint{4.746702in}{1.969385in}}%
\pgfpathlineto{\pgfqpoint{4.732303in}{1.966153in}}%
\pgfpathlineto{\pgfqpoint{4.724354in}{1.955658in}}%
\pgfpathlineto{\pgfqpoint{4.716400in}{1.945055in}}%
\pgfpathlineto{\pgfqpoint{4.708438in}{1.934343in}}%
\pgfpathlineto{\pgfqpoint{4.700470in}{1.923523in}}%
\pgfpathclose%
\pgfusepath{fill}%
\end{pgfscope}%
\begin{pgfscope}%
\pgfpathrectangle{\pgfqpoint{1.150000in}{0.150000in}}{\pgfqpoint{5.700000in}{5.700000in}}%
\pgfusepath{clip}%
\pgfsetbuttcap%
\pgfsetroundjoin%
\definecolor{currentfill}{rgb}{0.235526,0.309527,0.542944}%
\pgfsetfillcolor{currentfill}%
\pgfsetfillopacity{0.700000}%
\pgfsetlinewidth{0.000000pt}%
\definecolor{currentstroke}{rgb}{0.000000,0.000000,0.000000}%
\pgfsetstrokecolor{currentstroke}%
\pgfsetdash{}{0pt}%
\pgfpathmoveto{\pgfqpoint{4.789971in}{1.979508in}}%
\pgfpathlineto{\pgfqpoint{4.804418in}{1.983025in}}%
\pgfpathlineto{\pgfqpoint{4.818876in}{1.986613in}}%
\pgfpathlineto{\pgfqpoint{4.833347in}{1.990273in}}%
\pgfpathlineto{\pgfqpoint{4.847830in}{1.994004in}}%
\pgfpathlineto{\pgfqpoint{4.855757in}{2.004527in}}%
\pgfpathlineto{\pgfqpoint{4.863678in}{2.014928in}}%
\pgfpathlineto{\pgfqpoint{4.871591in}{2.025209in}}%
\pgfpathlineto{\pgfqpoint{4.879497in}{2.035367in}}%
\pgfpathlineto{\pgfqpoint{4.865022in}{2.031575in}}%
\pgfpathlineto{\pgfqpoint{4.850558in}{2.027855in}}%
\pgfpathlineto{\pgfqpoint{4.836106in}{2.024206in}}%
\pgfpathlineto{\pgfqpoint{4.821667in}{2.020628in}}%
\pgfpathlineto{\pgfqpoint{4.813753in}{2.010523in}}%
\pgfpathlineto{\pgfqpoint{4.805833in}{2.000301in}}%
\pgfpathlineto{\pgfqpoint{4.797905in}{1.989962in}}%
\pgfpathlineto{\pgfqpoint{4.789971in}{1.979508in}}%
\pgfpathclose%
\pgfusepath{fill}%
\end{pgfscope}%
\begin{pgfscope}%
\pgfpathrectangle{\pgfqpoint{1.150000in}{0.150000in}}{\pgfqpoint{5.700000in}{5.700000in}}%
\pgfusepath{clip}%
\pgfsetbuttcap%
\pgfsetroundjoin%
\definecolor{currentfill}{rgb}{0.268510,0.009605,0.335427}%
\pgfsetfillcolor{currentfill}%
\pgfsetfillopacity{0.700000}%
\pgfsetlinewidth{0.000000pt}%
\definecolor{currentstroke}{rgb}{0.000000,0.000000,0.000000}%
\pgfsetstrokecolor{currentstroke}%
\pgfsetdash{}{0pt}%
\pgfpathmoveto{\pgfqpoint{3.245688in}{1.371846in}}%
\pgfpathlineto{\pgfqpoint{3.259695in}{1.366284in}}%
\pgfpathlineto{\pgfqpoint{3.273706in}{1.360801in}}%
\pgfpathlineto{\pgfqpoint{3.287722in}{1.355396in}}%
\pgfpathlineto{\pgfqpoint{3.301743in}{1.350068in}}%
\pgfpathlineto{\pgfqpoint{3.310267in}{1.356062in}}%
\pgfpathlineto{\pgfqpoint{3.318782in}{1.362240in}}%
\pgfpathlineto{\pgfqpoint{3.327287in}{1.368596in}}%
\pgfpathlineto{\pgfqpoint{3.335782in}{1.375125in}}%
\pgfpathlineto{\pgfqpoint{3.321783in}{1.380060in}}%
\pgfpathlineto{\pgfqpoint{3.307790in}{1.385074in}}%
\pgfpathlineto{\pgfqpoint{3.293801in}{1.390165in}}%
\pgfpathlineto{\pgfqpoint{3.279817in}{1.395335in}}%
\pgfpathlineto{\pgfqpoint{3.271300in}{1.389191in}}%
\pgfpathlineto{\pgfqpoint{3.262773in}{1.383224in}}%
\pgfpathlineto{\pgfqpoint{3.254236in}{1.377440in}}%
\pgfpathlineto{\pgfqpoint{3.245688in}{1.371846in}}%
\pgfpathclose%
\pgfusepath{fill}%
\end{pgfscope}%
\begin{pgfscope}%
\pgfpathrectangle{\pgfqpoint{1.150000in}{0.150000in}}{\pgfqpoint{5.700000in}{5.700000in}}%
\pgfusepath{clip}%
\pgfsetbuttcap%
\pgfsetroundjoin%
\definecolor{currentfill}{rgb}{0.269944,0.014625,0.341379}%
\pgfsetfillcolor{currentfill}%
\pgfsetfillopacity{0.700000}%
\pgfsetlinewidth{0.000000pt}%
\definecolor{currentstroke}{rgb}{0.000000,0.000000,0.000000}%
\pgfsetstrokecolor{currentstroke}%
\pgfsetdash{}{0pt}%
\pgfpathmoveto{\pgfqpoint{3.627546in}{1.375818in}}%
\pgfpathlineto{\pgfqpoint{3.641608in}{1.372891in}}%
\pgfpathlineto{\pgfqpoint{3.655677in}{1.370039in}}%
\pgfpathlineto{\pgfqpoint{3.669752in}{1.367260in}}%
\pgfpathlineto{\pgfqpoint{3.683835in}{1.364555in}}%
\pgfpathlineto{\pgfqpoint{3.692168in}{1.374068in}}%
\pgfpathlineto{\pgfqpoint{3.700494in}{1.383672in}}%
\pgfpathlineto{\pgfqpoint{3.708814in}{1.393363in}}%
\pgfpathlineto{\pgfqpoint{3.717127in}{1.403137in}}%
\pgfpathlineto{\pgfqpoint{3.703059in}{1.405511in}}%
\pgfpathlineto{\pgfqpoint{3.688997in}{1.407960in}}%
\pgfpathlineto{\pgfqpoint{3.674943in}{1.410483in}}%
\pgfpathlineto{\pgfqpoint{3.660895in}{1.413079in}}%
\pgfpathlineto{\pgfqpoint{3.652568in}{1.403628in}}%
\pgfpathlineto{\pgfqpoint{3.644235in}{1.394264in}}%
\pgfpathlineto{\pgfqpoint{3.635894in}{1.384993in}}%
\pgfpathlineto{\pgfqpoint{3.627546in}{1.375818in}}%
\pgfpathclose%
\pgfusepath{fill}%
\end{pgfscope}%
\begin{pgfscope}%
\pgfpathrectangle{\pgfqpoint{1.150000in}{0.150000in}}{\pgfqpoint{5.700000in}{5.700000in}}%
\pgfusepath{clip}%
\pgfsetbuttcap%
\pgfsetroundjoin%
\definecolor{currentfill}{rgb}{0.223925,0.334994,0.548053}%
\pgfsetfillcolor{currentfill}%
\pgfsetfillopacity{0.700000}%
\pgfsetlinewidth{0.000000pt}%
\definecolor{currentstroke}{rgb}{0.000000,0.000000,0.000000}%
\pgfsetstrokecolor{currentstroke}%
\pgfsetdash{}{0pt}%
\pgfpathmoveto{\pgfqpoint{4.879497in}{2.035367in}}%
\pgfpathlineto{\pgfqpoint{4.893985in}{2.039229in}}%
\pgfpathlineto{\pgfqpoint{4.908486in}{2.043164in}}%
\pgfpathlineto{\pgfqpoint{4.922998in}{2.047169in}}%
\pgfpathlineto{\pgfqpoint{4.937523in}{2.051246in}}%
\pgfpathlineto{\pgfqpoint{4.945415in}{2.061329in}}%
\pgfpathlineto{\pgfqpoint{4.953299in}{2.071283in}}%
\pgfpathlineto{\pgfqpoint{4.961175in}{2.081110in}}%
\pgfpathlineto{\pgfqpoint{4.969044in}{2.090808in}}%
\pgfpathlineto{\pgfqpoint{4.954526in}{2.086692in}}%
\pgfpathlineto{\pgfqpoint{4.940021in}{2.082648in}}%
\pgfpathlineto{\pgfqpoint{4.925529in}{2.078675in}}%
\pgfpathlineto{\pgfqpoint{4.911048in}{2.074773in}}%
\pgfpathlineto{\pgfqpoint{4.903172in}{2.065106in}}%
\pgfpathlineto{\pgfqpoint{4.895288in}{2.055316in}}%
\pgfpathlineto{\pgfqpoint{4.887396in}{2.045402in}}%
\pgfpathlineto{\pgfqpoint{4.879497in}{2.035367in}}%
\pgfpathclose%
\pgfusepath{fill}%
\end{pgfscope}%
\begin{pgfscope}%
\pgfpathrectangle{\pgfqpoint{1.150000in}{0.150000in}}{\pgfqpoint{5.700000in}{5.700000in}}%
\pgfusepath{clip}%
\pgfsetbuttcap%
\pgfsetroundjoin%
\definecolor{currentfill}{rgb}{0.267004,0.004874,0.329415}%
\pgfsetfillcolor{currentfill}%
\pgfsetfillopacity{0.700000}%
\pgfsetlinewidth{0.000000pt}%
\definecolor{currentstroke}{rgb}{0.000000,0.000000,0.000000}%
\pgfsetstrokecolor{currentstroke}%
\pgfsetdash{}{0pt}%
\pgfpathmoveto{\pgfqpoint{3.391827in}{1.356154in}}%
\pgfpathlineto{\pgfqpoint{3.405852in}{1.351603in}}%
\pgfpathlineto{\pgfqpoint{3.419882in}{1.347129in}}%
\pgfpathlineto{\pgfqpoint{3.433917in}{1.342731in}}%
\pgfpathlineto{\pgfqpoint{3.447959in}{1.338408in}}%
\pgfpathlineto{\pgfqpoint{3.456403in}{1.345860in}}%
\pgfpathlineto{\pgfqpoint{3.464840in}{1.353463in}}%
\pgfpathlineto{\pgfqpoint{3.473267in}{1.361209in}}%
\pgfpathlineto{\pgfqpoint{3.481687in}{1.369095in}}%
\pgfpathlineto{\pgfqpoint{3.467665in}{1.373047in}}%
\pgfpathlineto{\pgfqpoint{3.453648in}{1.377074in}}%
\pgfpathlineto{\pgfqpoint{3.439637in}{1.381178in}}%
\pgfpathlineto{\pgfqpoint{3.425632in}{1.385357in}}%
\pgfpathlineto{\pgfqpoint{3.417194in}{1.377835in}}%
\pgfpathlineto{\pgfqpoint{3.408748in}{1.370456in}}%
\pgfpathlineto{\pgfqpoint{3.400292in}{1.363228in}}%
\pgfpathlineto{\pgfqpoint{3.391827in}{1.356154in}}%
\pgfpathclose%
\pgfusepath{fill}%
\end{pgfscope}%
\begin{pgfscope}%
\pgfpathrectangle{\pgfqpoint{1.150000in}{0.150000in}}{\pgfqpoint{5.700000in}{5.700000in}}%
\pgfusepath{clip}%
\pgfsetbuttcap%
\pgfsetroundjoin%
\definecolor{currentfill}{rgb}{0.271305,0.019942,0.347269}%
\pgfsetfillcolor{currentfill}%
\pgfsetfillopacity{0.700000}%
\pgfsetlinewidth{0.000000pt}%
\definecolor{currentstroke}{rgb}{0.000000,0.000000,0.000000}%
\pgfsetstrokecolor{currentstroke}%
\pgfsetdash{}{0pt}%
\pgfpathmoveto{\pgfqpoint{3.099283in}{1.402140in}}%
\pgfpathlineto{\pgfqpoint{3.113282in}{1.395529in}}%
\pgfpathlineto{\pgfqpoint{3.127285in}{1.388998in}}%
\pgfpathlineto{\pgfqpoint{3.141292in}{1.382547in}}%
\pgfpathlineto{\pgfqpoint{3.155302in}{1.376177in}}%
\pgfpathlineto{\pgfqpoint{3.163921in}{1.380529in}}%
\pgfpathlineto{\pgfqpoint{3.172527in}{1.385102in}}%
\pgfpathlineto{\pgfqpoint{3.181122in}{1.389888in}}%
\pgfpathlineto{\pgfqpoint{3.189706in}{1.394881in}}%
\pgfpathlineto{\pgfqpoint{3.175721in}{1.400838in}}%
\pgfpathlineto{\pgfqpoint{3.161740in}{1.406875in}}%
\pgfpathlineto{\pgfqpoint{3.147764in}{1.412992in}}%
\pgfpathlineto{\pgfqpoint{3.133791in}{1.419190in}}%
\pgfpathlineto{\pgfqpoint{3.125182in}{1.414603in}}%
\pgfpathlineto{\pgfqpoint{3.116561in}{1.410227in}}%
\pgfpathlineto{\pgfqpoint{3.107928in}{1.406071in}}%
\pgfpathlineto{\pgfqpoint{3.099283in}{1.402140in}}%
\pgfpathclose%
\pgfusepath{fill}%
\end{pgfscope}%
\begin{pgfscope}%
\pgfpathrectangle{\pgfqpoint{1.150000in}{0.150000in}}{\pgfqpoint{5.700000in}{5.700000in}}%
\pgfusepath{clip}%
\pgfsetbuttcap%
\pgfsetroundjoin%
\definecolor{currentfill}{rgb}{0.214298,0.355619,0.551184}%
\pgfsetfillcolor{currentfill}%
\pgfsetfillopacity{0.700000}%
\pgfsetlinewidth{0.000000pt}%
\definecolor{currentstroke}{rgb}{0.000000,0.000000,0.000000}%
\pgfsetstrokecolor{currentstroke}%
\pgfsetdash{}{0pt}%
\pgfpathmoveto{\pgfqpoint{4.969044in}{2.090808in}}%
\pgfpathlineto{\pgfqpoint{4.983573in}{2.094995in}}%
\pgfpathlineto{\pgfqpoint{4.998116in}{2.099253in}}%
\pgfpathlineto{\pgfqpoint{5.012671in}{2.103582in}}%
\pgfpathlineto{\pgfqpoint{5.027239in}{2.107983in}}%
\pgfpathlineto{\pgfqpoint{5.035092in}{2.117578in}}%
\pgfpathlineto{\pgfqpoint{5.042937in}{2.127039in}}%
\pgfpathlineto{\pgfqpoint{5.050774in}{2.136367in}}%
\pgfpathlineto{\pgfqpoint{5.058602in}{2.145561in}}%
\pgfpathlineto{\pgfqpoint{5.044043in}{2.141143in}}%
\pgfpathlineto{\pgfqpoint{5.029496in}{2.136797in}}%
\pgfpathlineto{\pgfqpoint{5.014962in}{2.132521in}}%
\pgfpathlineto{\pgfqpoint{5.000441in}{2.128318in}}%
\pgfpathlineto{\pgfqpoint{4.992603in}{2.119132in}}%
\pgfpathlineto{\pgfqpoint{4.984758in}{2.109819in}}%
\pgfpathlineto{\pgfqpoint{4.976904in}{2.100377in}}%
\pgfpathlineto{\pgfqpoint{4.969044in}{2.090808in}}%
\pgfpathclose%
\pgfusepath{fill}%
\end{pgfscope}%
\begin{pgfscope}%
\pgfpathrectangle{\pgfqpoint{1.150000in}{0.150000in}}{\pgfqpoint{5.700000in}{5.700000in}}%
\pgfusepath{clip}%
\pgfsetbuttcap%
\pgfsetroundjoin%
\definecolor{currentfill}{rgb}{0.204903,0.375746,0.553533}%
\pgfsetfillcolor{currentfill}%
\pgfsetfillopacity{0.700000}%
\pgfsetlinewidth{0.000000pt}%
\definecolor{currentstroke}{rgb}{0.000000,0.000000,0.000000}%
\pgfsetstrokecolor{currentstroke}%
\pgfsetdash{}{0pt}%
\pgfpathmoveto{\pgfqpoint{5.058602in}{2.145561in}}%
\pgfpathlineto{\pgfqpoint{5.073175in}{2.150050in}}%
\pgfpathlineto{\pgfqpoint{5.087760in}{2.154610in}}%
\pgfpathlineto{\pgfqpoint{5.102358in}{2.159242in}}%
\pgfpathlineto{\pgfqpoint{5.116969in}{2.163945in}}%
\pgfpathlineto{\pgfqpoint{5.124781in}{2.173009in}}%
\pgfpathlineto{\pgfqpoint{5.132584in}{2.181936in}}%
\pgfpathlineto{\pgfqpoint{5.140379in}{2.190724in}}%
\pgfpathlineto{\pgfqpoint{5.148165in}{2.199376in}}%
\pgfpathlineto{\pgfqpoint{5.133563in}{2.194678in}}%
\pgfpathlineto{\pgfqpoint{5.118974in}{2.190051in}}%
\pgfpathlineto{\pgfqpoint{5.104399in}{2.185496in}}%
\pgfpathlineto{\pgfqpoint{5.089836in}{2.181012in}}%
\pgfpathlineto{\pgfqpoint{5.082040in}{2.172347in}}%
\pgfpathlineto{\pgfqpoint{5.074235in}{2.163550in}}%
\pgfpathlineto{\pgfqpoint{5.066423in}{2.154622in}}%
\pgfpathlineto{\pgfqpoint{5.058602in}{2.145561in}}%
\pgfpathclose%
\pgfusepath{fill}%
\end{pgfscope}%
\begin{pgfscope}%
\pgfpathrectangle{\pgfqpoint{1.150000in}{0.150000in}}{\pgfqpoint{5.700000in}{5.700000in}}%
\pgfusepath{clip}%
\pgfsetbuttcap%
\pgfsetroundjoin%
\definecolor{currentfill}{rgb}{0.283229,0.120777,0.440584}%
\pgfsetfillcolor{currentfill}%
\pgfsetfillopacity{0.700000}%
\pgfsetlinewidth{0.000000pt}%
\definecolor{currentstroke}{rgb}{0.000000,0.000000,0.000000}%
\pgfsetstrokecolor{currentstroke}%
\pgfsetdash{}{0pt}%
\pgfpathmoveto{\pgfqpoint{4.131443in}{1.557214in}}%
\pgfpathlineto{\pgfqpoint{4.145642in}{1.557482in}}%
\pgfpathlineto{\pgfqpoint{4.159851in}{1.557821in}}%
\pgfpathlineto{\pgfqpoint{4.174069in}{1.558233in}}%
\pgfpathlineto{\pgfqpoint{4.188296in}{1.558716in}}%
\pgfpathlineto{\pgfqpoint{4.196454in}{1.570641in}}%
\pgfpathlineto{\pgfqpoint{4.204607in}{1.582545in}}%
\pgfpathlineto{\pgfqpoint{4.212754in}{1.594424in}}%
\pgfpathlineto{\pgfqpoint{4.220896in}{1.606274in}}%
\pgfpathlineto{\pgfqpoint{4.206676in}{1.605562in}}%
\pgfpathlineto{\pgfqpoint{4.192465in}{1.604922in}}%
\pgfpathlineto{\pgfqpoint{4.178264in}{1.604353in}}%
\pgfpathlineto{\pgfqpoint{4.164072in}{1.603856in}}%
\pgfpathlineto{\pgfqpoint{4.155923in}{1.592227in}}%
\pgfpathlineto{\pgfqpoint{4.147768in}{1.580574in}}%
\pgfpathlineto{\pgfqpoint{4.139608in}{1.568902in}}%
\pgfpathlineto{\pgfqpoint{4.131443in}{1.557214in}}%
\pgfpathclose%
\pgfusepath{fill}%
\end{pgfscope}%
\begin{pgfscope}%
\pgfpathrectangle{\pgfqpoint{1.150000in}{0.150000in}}{\pgfqpoint{5.700000in}{5.700000in}}%
\pgfusepath{clip}%
\pgfsetbuttcap%
\pgfsetroundjoin%
\definecolor{currentfill}{rgb}{0.171176,0.452530,0.557965}%
\pgfsetfillcolor{currentfill}%
\pgfsetfillopacity{0.700000}%
\pgfsetlinewidth{0.000000pt}%
\definecolor{currentstroke}{rgb}{0.000000,0.000000,0.000000}%
\pgfsetstrokecolor{currentstroke}%
\pgfsetdash{}{0pt}%
\pgfpathmoveto{\pgfqpoint{5.416773in}{2.353017in}}%
\pgfpathlineto{\pgfqpoint{5.431515in}{2.358493in}}%
\pgfpathlineto{\pgfqpoint{5.446272in}{2.364041in}}%
\pgfpathlineto{\pgfqpoint{5.461042in}{2.369660in}}%
\pgfpathlineto{\pgfqpoint{5.468665in}{2.376306in}}%
\pgfpathlineto{\pgfqpoint{5.476277in}{2.382813in}}%
\pgfpathlineto{\pgfqpoint{5.483880in}{2.389181in}}%
\pgfpathlineto{\pgfqpoint{5.491473in}{2.395414in}}%
\pgfpathlineto{\pgfqpoint{5.476717in}{2.389890in}}%
\pgfpathlineto{\pgfqpoint{5.461975in}{2.384437in}}%
\pgfpathlineto{\pgfqpoint{5.447247in}{2.379055in}}%
\pgfpathlineto{\pgfqpoint{5.439643in}{2.372745in}}%
\pgfpathlineto{\pgfqpoint{5.432029in}{2.366303in}}%
\pgfpathlineto{\pgfqpoint{5.424406in}{2.359727in}}%
\pgfpathlineto{\pgfqpoint{5.416773in}{2.353017in}}%
\pgfpathclose%
\pgfusepath{fill}%
\end{pgfscope}%
\begin{pgfscope}%
\pgfpathrectangle{\pgfqpoint{1.150000in}{0.150000in}}{\pgfqpoint{5.700000in}{5.700000in}}%
\pgfusepath{clip}%
\pgfsetbuttcap%
\pgfsetroundjoin%
\definecolor{currentfill}{rgb}{0.282656,0.100196,0.422160}%
\pgfsetfillcolor{currentfill}%
\pgfsetfillopacity{0.700000}%
\pgfsetlinewidth{0.000000pt}%
\definecolor{currentstroke}{rgb}{0.000000,0.000000,0.000000}%
\pgfsetstrokecolor{currentstroke}%
\pgfsetdash{}{0pt}%
\pgfpathmoveto{\pgfqpoint{4.041992in}{1.511007in}}%
\pgfpathlineto{\pgfqpoint{4.056163in}{1.510738in}}%
\pgfpathlineto{\pgfqpoint{4.070343in}{1.510541in}}%
\pgfpathlineto{\pgfqpoint{4.084532in}{1.510416in}}%
\pgfpathlineto{\pgfqpoint{4.098730in}{1.510362in}}%
\pgfpathlineto{\pgfqpoint{4.106916in}{1.522083in}}%
\pgfpathlineto{\pgfqpoint{4.115097in}{1.533801in}}%
\pgfpathlineto{\pgfqpoint{4.123273in}{1.545512in}}%
\pgfpathlineto{\pgfqpoint{4.131443in}{1.557214in}}%
\pgfpathlineto{\pgfqpoint{4.117253in}{1.557017in}}%
\pgfpathlineto{\pgfqpoint{4.103072in}{1.556893in}}%
\pgfpathlineto{\pgfqpoint{4.088900in}{1.556841in}}%
\pgfpathlineto{\pgfqpoint{4.074737in}{1.556860in}}%
\pgfpathlineto{\pgfqpoint{4.066559in}{1.545400in}}%
\pgfpathlineto{\pgfqpoint{4.058375in}{1.533936in}}%
\pgfpathlineto{\pgfqpoint{4.050186in}{1.522470in}}%
\pgfpathlineto{\pgfqpoint{4.041992in}{1.511007in}}%
\pgfpathclose%
\pgfusepath{fill}%
\end{pgfscope}%
\begin{pgfscope}%
\pgfpathrectangle{\pgfqpoint{1.150000in}{0.150000in}}{\pgfqpoint{5.700000in}{5.700000in}}%
\pgfusepath{clip}%
\pgfsetbuttcap%
\pgfsetroundjoin%
\definecolor{currentfill}{rgb}{0.282290,0.145912,0.461510}%
\pgfsetfillcolor{currentfill}%
\pgfsetfillopacity{0.700000}%
\pgfsetlinewidth{0.000000pt}%
\definecolor{currentstroke}{rgb}{0.000000,0.000000,0.000000}%
\pgfsetstrokecolor{currentstroke}%
\pgfsetdash{}{0pt}%
\pgfpathmoveto{\pgfqpoint{4.220896in}{1.606274in}}%
\pgfpathlineto{\pgfqpoint{4.235126in}{1.607058in}}%
\pgfpathlineto{\pgfqpoint{4.249366in}{1.607913in}}%
\pgfpathlineto{\pgfqpoint{4.263615in}{1.608840in}}%
\pgfpathlineto{\pgfqpoint{4.277874in}{1.609838in}}%
\pgfpathlineto{\pgfqpoint{4.286004in}{1.621874in}}%
\pgfpathlineto{\pgfqpoint{4.294130in}{1.633871in}}%
\pgfpathlineto{\pgfqpoint{4.302250in}{1.645825in}}%
\pgfpathlineto{\pgfqpoint{4.310364in}{1.657735in}}%
\pgfpathlineto{\pgfqpoint{4.296111in}{1.656528in}}%
\pgfpathlineto{\pgfqpoint{4.281869in}{1.655392in}}%
\pgfpathlineto{\pgfqpoint{4.267636in}{1.654329in}}%
\pgfpathlineto{\pgfqpoint{4.253413in}{1.653337in}}%
\pgfpathlineto{\pgfqpoint{4.245292in}{1.641628in}}%
\pgfpathlineto{\pgfqpoint{4.237165in}{1.629880in}}%
\pgfpathlineto{\pgfqpoint{4.229033in}{1.618094in}}%
\pgfpathlineto{\pgfqpoint{4.220896in}{1.606274in}}%
\pgfpathclose%
\pgfusepath{fill}%
\end{pgfscope}%
\begin{pgfscope}%
\pgfpathrectangle{\pgfqpoint{1.150000in}{0.150000in}}{\pgfqpoint{5.700000in}{5.700000in}}%
\pgfusepath{clip}%
\pgfsetbuttcap%
\pgfsetroundjoin%
\definecolor{currentfill}{rgb}{0.194100,0.399323,0.555565}%
\pgfsetfillcolor{currentfill}%
\pgfsetfillopacity{0.700000}%
\pgfsetlinewidth{0.000000pt}%
\definecolor{currentstroke}{rgb}{0.000000,0.000000,0.000000}%
\pgfsetstrokecolor{currentstroke}%
\pgfsetdash{}{0pt}%
\pgfpathmoveto{\pgfqpoint{5.148165in}{2.199376in}}%
\pgfpathlineto{\pgfqpoint{5.162780in}{2.204145in}}%
\pgfpathlineto{\pgfqpoint{5.177408in}{2.208986in}}%
\pgfpathlineto{\pgfqpoint{5.192049in}{2.213898in}}%
\pgfpathlineto{\pgfqpoint{5.206704in}{2.218881in}}%
\pgfpathlineto{\pgfqpoint{5.214472in}{2.227378in}}%
\pgfpathlineto{\pgfqpoint{5.222230in}{2.235733in}}%
\pgfpathlineto{\pgfqpoint{5.229980in}{2.243949in}}%
\pgfpathlineto{\pgfqpoint{5.237722in}{2.252025in}}%
\pgfpathlineto{\pgfqpoint{5.223077in}{2.247069in}}%
\pgfpathlineto{\pgfqpoint{5.208446in}{2.242184in}}%
\pgfpathlineto{\pgfqpoint{5.193829in}{2.237371in}}%
\pgfpathlineto{\pgfqpoint{5.179224in}{2.232629in}}%
\pgfpathlineto{\pgfqpoint{5.171472in}{2.224517in}}%
\pgfpathlineto{\pgfqpoint{5.163712in}{2.216271in}}%
\pgfpathlineto{\pgfqpoint{5.155943in}{2.207891in}}%
\pgfpathlineto{\pgfqpoint{5.148165in}{2.199376in}}%
\pgfpathclose%
\pgfusepath{fill}%
\end{pgfscope}%
\begin{pgfscope}%
\pgfpathrectangle{\pgfqpoint{1.150000in}{0.150000in}}{\pgfqpoint{5.700000in}{5.700000in}}%
\pgfusepath{clip}%
\pgfsetbuttcap%
\pgfsetroundjoin%
\definecolor{currentfill}{rgb}{0.280894,0.078907,0.402329}%
\pgfsetfillcolor{currentfill}%
\pgfsetfillopacity{0.700000}%
\pgfsetlinewidth{0.000000pt}%
\definecolor{currentstroke}{rgb}{0.000000,0.000000,0.000000}%
\pgfsetstrokecolor{currentstroke}%
\pgfsetdash{}{0pt}%
\pgfpathmoveto{\pgfqpoint{3.952527in}{1.468129in}}%
\pgfpathlineto{\pgfqpoint{3.966672in}{1.467302in}}%
\pgfpathlineto{\pgfqpoint{3.980827in}{1.466547in}}%
\pgfpathlineto{\pgfqpoint{3.994990in}{1.465865in}}%
\pgfpathlineto{\pgfqpoint{4.009161in}{1.465254in}}%
\pgfpathlineto{\pgfqpoint{4.017377in}{1.476669in}}%
\pgfpathlineto{\pgfqpoint{4.025587in}{1.488103in}}%
\pgfpathlineto{\pgfqpoint{4.033792in}{1.499550in}}%
\pgfpathlineto{\pgfqpoint{4.041992in}{1.511007in}}%
\pgfpathlineto{\pgfqpoint{4.027829in}{1.511348in}}%
\pgfpathlineto{\pgfqpoint{4.013675in}{1.511761in}}%
\pgfpathlineto{\pgfqpoint{3.999530in}{1.512246in}}%
\pgfpathlineto{\pgfqpoint{3.985393in}{1.512803in}}%
\pgfpathlineto{\pgfqpoint{3.977185in}{1.501608in}}%
\pgfpathlineto{\pgfqpoint{3.968971in}{1.490428in}}%
\pgfpathlineto{\pgfqpoint{3.960752in}{1.479267in}}%
\pgfpathlineto{\pgfqpoint{3.952527in}{1.468129in}}%
\pgfpathclose%
\pgfusepath{fill}%
\end{pgfscope}%
\begin{pgfscope}%
\pgfpathrectangle{\pgfqpoint{1.150000in}{0.150000in}}{\pgfqpoint{5.700000in}{5.700000in}}%
\pgfusepath{clip}%
\pgfsetbuttcap%
\pgfsetroundjoin%
\definecolor{currentfill}{rgb}{0.268510,0.009605,0.335427}%
\pgfsetfillcolor{currentfill}%
\pgfsetfillopacity{0.700000}%
\pgfsetlinewidth{0.000000pt}%
\definecolor{currentstroke}{rgb}{0.000000,0.000000,0.000000}%
\pgfsetstrokecolor{currentstroke}%
\pgfsetdash{}{0pt}%
\pgfpathmoveto{\pgfqpoint{3.537835in}{1.354044in}}%
\pgfpathlineto{\pgfqpoint{3.551888in}{1.350469in}}%
\pgfpathlineto{\pgfqpoint{3.565946in}{1.346969in}}%
\pgfpathlineto{\pgfqpoint{3.580011in}{1.343543in}}%
\pgfpathlineto{\pgfqpoint{3.594082in}{1.340191in}}%
\pgfpathlineto{\pgfqpoint{3.602459in}{1.348927in}}%
\pgfpathlineto{\pgfqpoint{3.610829in}{1.357780in}}%
\pgfpathlineto{\pgfqpoint{3.619191in}{1.366745in}}%
\pgfpathlineto{\pgfqpoint{3.627546in}{1.375818in}}%
\pgfpathlineto{\pgfqpoint{3.613491in}{1.378819in}}%
\pgfpathlineto{\pgfqpoint{3.599442in}{1.381894in}}%
\pgfpathlineto{\pgfqpoint{3.585400in}{1.385044in}}%
\pgfpathlineto{\pgfqpoint{3.571364in}{1.388269in}}%
\pgfpathlineto{\pgfqpoint{3.562993in}{1.379539in}}%
\pgfpathlineto{\pgfqpoint{3.554615in}{1.370921in}}%
\pgfpathlineto{\pgfqpoint{3.546229in}{1.362421in}}%
\pgfpathlineto{\pgfqpoint{3.537835in}{1.354044in}}%
\pgfpathclose%
\pgfusepath{fill}%
\end{pgfscope}%
\begin{pgfscope}%
\pgfpathrectangle{\pgfqpoint{1.150000in}{0.150000in}}{\pgfqpoint{5.700000in}{5.700000in}}%
\pgfusepath{clip}%
\pgfsetbuttcap%
\pgfsetroundjoin%
\definecolor{currentfill}{rgb}{0.279574,0.170599,0.479997}%
\pgfsetfillcolor{currentfill}%
\pgfsetfillopacity{0.700000}%
\pgfsetlinewidth{0.000000pt}%
\definecolor{currentstroke}{rgb}{0.000000,0.000000,0.000000}%
\pgfsetstrokecolor{currentstroke}%
\pgfsetdash{}{0pt}%
\pgfpathmoveto{\pgfqpoint{4.310364in}{1.657735in}}%
\pgfpathlineto{\pgfqpoint{4.324627in}{1.659013in}}%
\pgfpathlineto{\pgfqpoint{4.338900in}{1.660363in}}%
\pgfpathlineto{\pgfqpoint{4.353182in}{1.661784in}}%
\pgfpathlineto{\pgfqpoint{4.367475in}{1.663276in}}%
\pgfpathlineto{\pgfqpoint{4.375579in}{1.675333in}}%
\pgfpathlineto{\pgfqpoint{4.383677in}{1.687335in}}%
\pgfpathlineto{\pgfqpoint{4.391769in}{1.699278in}}%
\pgfpathlineto{\pgfqpoint{4.399856in}{1.711160in}}%
\pgfpathlineto{\pgfqpoint{4.385569in}{1.709480in}}%
\pgfpathlineto{\pgfqpoint{4.371292in}{1.707870in}}%
\pgfpathlineto{\pgfqpoint{4.357026in}{1.706333in}}%
\pgfpathlineto{\pgfqpoint{4.342769in}{1.704867in}}%
\pgfpathlineto{\pgfqpoint{4.334676in}{1.693165in}}%
\pgfpathlineto{\pgfqpoint{4.326577in}{1.681407in}}%
\pgfpathlineto{\pgfqpoint{4.318474in}{1.669596in}}%
\pgfpathlineto{\pgfqpoint{4.310364in}{1.657735in}}%
\pgfpathclose%
\pgfusepath{fill}%
\end{pgfscope}%
\begin{pgfscope}%
\pgfpathrectangle{\pgfqpoint{1.150000in}{0.150000in}}{\pgfqpoint{5.700000in}{5.700000in}}%
\pgfusepath{clip}%
\pgfsetbuttcap%
\pgfsetroundjoin%
\definecolor{currentfill}{rgb}{0.185556,0.418570,0.556753}%
\pgfsetfillcolor{currentfill}%
\pgfsetfillopacity{0.700000}%
\pgfsetlinewidth{0.000000pt}%
\definecolor{currentstroke}{rgb}{0.000000,0.000000,0.000000}%
\pgfsetstrokecolor{currentstroke}%
\pgfsetdash{}{0pt}%
\pgfpathmoveto{\pgfqpoint{5.237722in}{2.252025in}}%
\pgfpathlineto{\pgfqpoint{5.252379in}{2.257052in}}%
\pgfpathlineto{\pgfqpoint{5.267050in}{2.262151in}}%
\pgfpathlineto{\pgfqpoint{5.281735in}{2.267321in}}%
\pgfpathlineto{\pgfqpoint{5.296433in}{2.272563in}}%
\pgfpathlineto{\pgfqpoint{5.304154in}{2.280459in}}%
\pgfpathlineto{\pgfqpoint{5.311866in}{2.288214in}}%
\pgfpathlineto{\pgfqpoint{5.319568in}{2.295827in}}%
\pgfpathlineto{\pgfqpoint{5.327261in}{2.303300in}}%
\pgfpathlineto{\pgfqpoint{5.312575in}{2.298108in}}%
\pgfpathlineto{\pgfqpoint{5.297902in}{2.292988in}}%
\pgfpathlineto{\pgfqpoint{5.283243in}{2.287939in}}%
\pgfpathlineto{\pgfqpoint{5.268597in}{2.282961in}}%
\pgfpathlineto{\pgfqpoint{5.260891in}{2.275430in}}%
\pgfpathlineto{\pgfqpoint{5.253177in}{2.267764in}}%
\pgfpathlineto{\pgfqpoint{5.245454in}{2.259963in}}%
\pgfpathlineto{\pgfqpoint{5.237722in}{2.252025in}}%
\pgfpathclose%
\pgfusepath{fill}%
\end{pgfscope}%
\begin{pgfscope}%
\pgfpathrectangle{\pgfqpoint{1.150000in}{0.150000in}}{\pgfqpoint{5.700000in}{5.700000in}}%
\pgfusepath{clip}%
\pgfsetbuttcap%
\pgfsetroundjoin%
\definecolor{currentfill}{rgb}{0.177423,0.437527,0.557565}%
\pgfsetfillcolor{currentfill}%
\pgfsetfillopacity{0.700000}%
\pgfsetlinewidth{0.000000pt}%
\definecolor{currentstroke}{rgb}{0.000000,0.000000,0.000000}%
\pgfsetstrokecolor{currentstroke}%
\pgfsetdash{}{0pt}%
\pgfpathmoveto{\pgfqpoint{5.327261in}{2.303300in}}%
\pgfpathlineto{\pgfqpoint{5.341962in}{2.308563in}}%
\pgfpathlineto{\pgfqpoint{5.356675in}{2.313898in}}%
\pgfpathlineto{\pgfqpoint{5.371403in}{2.319304in}}%
\pgfpathlineto{\pgfqpoint{5.386145in}{2.324781in}}%
\pgfpathlineto{\pgfqpoint{5.393816in}{2.332052in}}%
\pgfpathlineto{\pgfqpoint{5.401478in}{2.339181in}}%
\pgfpathlineto{\pgfqpoint{5.409130in}{2.346168in}}%
\pgfpathlineto{\pgfqpoint{5.416773in}{2.353017in}}%
\pgfpathlineto{\pgfqpoint{5.402044in}{2.347611in}}%
\pgfpathlineto{\pgfqpoint{5.387329in}{2.342278in}}%
\pgfpathlineto{\pgfqpoint{5.372628in}{2.337015in}}%
\pgfpathlineto{\pgfqpoint{5.357941in}{2.331824in}}%
\pgfpathlineto{\pgfqpoint{5.350285in}{2.324895in}}%
\pgfpathlineto{\pgfqpoint{5.342620in}{2.317833in}}%
\pgfpathlineto{\pgfqpoint{5.334945in}{2.310635in}}%
\pgfpathlineto{\pgfqpoint{5.327261in}{2.303300in}}%
\pgfpathclose%
\pgfusepath{fill}%
\end{pgfscope}%
\begin{pgfscope}%
\pgfpathrectangle{\pgfqpoint{1.150000in}{0.150000in}}{\pgfqpoint{5.700000in}{5.700000in}}%
\pgfusepath{clip}%
\pgfsetbuttcap%
\pgfsetroundjoin%
\definecolor{currentfill}{rgb}{0.277941,0.056324,0.381191}%
\pgfsetfillcolor{currentfill}%
\pgfsetfillopacity{0.700000}%
\pgfsetlinewidth{0.000000pt}%
\definecolor{currentstroke}{rgb}{0.000000,0.000000,0.000000}%
\pgfsetstrokecolor{currentstroke}%
\pgfsetdash{}{0pt}%
\pgfpathmoveto{\pgfqpoint{3.863028in}{1.429078in}}%
\pgfpathlineto{\pgfqpoint{3.877152in}{1.427671in}}%
\pgfpathlineto{\pgfqpoint{3.891283in}{1.426337in}}%
\pgfpathlineto{\pgfqpoint{3.905423in}{1.425076in}}%
\pgfpathlineto{\pgfqpoint{3.919571in}{1.423886in}}%
\pgfpathlineto{\pgfqpoint{3.927818in}{1.434892in}}%
\pgfpathlineto{\pgfqpoint{3.936060in}{1.445937in}}%
\pgfpathlineto{\pgfqpoint{3.944296in}{1.457018in}}%
\pgfpathlineto{\pgfqpoint{3.952527in}{1.468129in}}%
\pgfpathlineto{\pgfqpoint{3.938389in}{1.469028in}}%
\pgfpathlineto{\pgfqpoint{3.924259in}{1.470000in}}%
\pgfpathlineto{\pgfqpoint{3.910138in}{1.471044in}}%
\pgfpathlineto{\pgfqpoint{3.896025in}{1.472161in}}%
\pgfpathlineto{\pgfqpoint{3.887784in}{1.461332in}}%
\pgfpathlineto{\pgfqpoint{3.879538in}{1.450539in}}%
\pgfpathlineto{\pgfqpoint{3.871286in}{1.439786in}}%
\pgfpathlineto{\pgfqpoint{3.863028in}{1.429078in}}%
\pgfpathclose%
\pgfusepath{fill}%
\end{pgfscope}%
\begin{pgfscope}%
\pgfpathrectangle{\pgfqpoint{1.150000in}{0.150000in}}{\pgfqpoint{5.700000in}{5.700000in}}%
\pgfusepath{clip}%
\pgfsetbuttcap%
\pgfsetroundjoin%
\definecolor{currentfill}{rgb}{0.274128,0.199721,0.498911}%
\pgfsetfillcolor{currentfill}%
\pgfsetfillopacity{0.700000}%
\pgfsetlinewidth{0.000000pt}%
\definecolor{currentstroke}{rgb}{0.000000,0.000000,0.000000}%
\pgfsetstrokecolor{currentstroke}%
\pgfsetdash{}{0pt}%
\pgfpathmoveto{\pgfqpoint{4.399856in}{1.711160in}}%
\pgfpathlineto{\pgfqpoint{4.414154in}{1.712912in}}%
\pgfpathlineto{\pgfqpoint{4.428462in}{1.714735in}}%
\pgfpathlineto{\pgfqpoint{4.442780in}{1.716630in}}%
\pgfpathlineto{\pgfqpoint{4.457109in}{1.718595in}}%
\pgfpathlineto{\pgfqpoint{4.465185in}{1.730589in}}%
\pgfpathlineto{\pgfqpoint{4.473256in}{1.742513in}}%
\pgfpathlineto{\pgfqpoint{4.481321in}{1.754363in}}%
\pgfpathlineto{\pgfqpoint{4.489380in}{1.766137in}}%
\pgfpathlineto{\pgfqpoint{4.475056in}{1.764004in}}%
\pgfpathlineto{\pgfqpoint{4.460743in}{1.761943in}}%
\pgfpathlineto{\pgfqpoint{4.446441in}{1.759952in}}%
\pgfpathlineto{\pgfqpoint{4.432150in}{1.758034in}}%
\pgfpathlineto{\pgfqpoint{4.424085in}{1.746418in}}%
\pgfpathlineto{\pgfqpoint{4.416014in}{1.734732in}}%
\pgfpathlineto{\pgfqpoint{4.407938in}{1.722979in}}%
\pgfpathlineto{\pgfqpoint{4.399856in}{1.711160in}}%
\pgfpathclose%
\pgfusepath{fill}%
\end{pgfscope}%
\begin{pgfscope}%
\pgfpathrectangle{\pgfqpoint{1.150000in}{0.150000in}}{\pgfqpoint{5.700000in}{5.700000in}}%
\pgfusepath{clip}%
\pgfsetbuttcap%
\pgfsetroundjoin%
\definecolor{currentfill}{rgb}{0.267968,0.223549,0.512008}%
\pgfsetfillcolor{currentfill}%
\pgfsetfillopacity{0.700000}%
\pgfsetlinewidth{0.000000pt}%
\definecolor{currentstroke}{rgb}{0.000000,0.000000,0.000000}%
\pgfsetstrokecolor{currentstroke}%
\pgfsetdash{}{0pt}%
\pgfpathmoveto{\pgfqpoint{4.489380in}{1.766137in}}%
\pgfpathlineto{\pgfqpoint{4.503714in}{1.768342in}}%
\pgfpathlineto{\pgfqpoint{4.518059in}{1.770617in}}%
\pgfpathlineto{\pgfqpoint{4.532415in}{1.772964in}}%
\pgfpathlineto{\pgfqpoint{4.546782in}{1.775382in}}%
\pgfpathlineto{\pgfqpoint{4.554830in}{1.787234in}}%
\pgfpathlineto{\pgfqpoint{4.562872in}{1.799001in}}%
\pgfpathlineto{\pgfqpoint{4.570908in}{1.810681in}}%
\pgfpathlineto{\pgfqpoint{4.578938in}{1.822274in}}%
\pgfpathlineto{\pgfqpoint{4.564577in}{1.819709in}}%
\pgfpathlineto{\pgfqpoint{4.550226in}{1.817216in}}%
\pgfpathlineto{\pgfqpoint{4.535887in}{1.814794in}}%
\pgfpathlineto{\pgfqpoint{4.521559in}{1.812444in}}%
\pgfpathlineto{\pgfqpoint{4.513522in}{1.800990in}}%
\pgfpathlineto{\pgfqpoint{4.505481in}{1.789453in}}%
\pgfpathlineto{\pgfqpoint{4.497433in}{1.777835in}}%
\pgfpathlineto{\pgfqpoint{4.489380in}{1.766137in}}%
\pgfpathclose%
\pgfusepath{fill}%
\end{pgfscope}%
\begin{pgfscope}%
\pgfpathrectangle{\pgfqpoint{1.150000in}{0.150000in}}{\pgfqpoint{5.700000in}{5.700000in}}%
\pgfusepath{clip}%
\pgfsetbuttcap%
\pgfsetroundjoin%
\definecolor{currentfill}{rgb}{0.274952,0.037752,0.364543}%
\pgfsetfillcolor{currentfill}%
\pgfsetfillopacity{0.700000}%
\pgfsetlinewidth{0.000000pt}%
\definecolor{currentstroke}{rgb}{0.000000,0.000000,0.000000}%
\pgfsetstrokecolor{currentstroke}%
\pgfsetdash{}{0pt}%
\pgfpathmoveto{\pgfqpoint{3.773473in}{1.394373in}}%
\pgfpathlineto{\pgfqpoint{3.787578in}{1.392365in}}%
\pgfpathlineto{\pgfqpoint{3.801690in}{1.390430in}}%
\pgfpathlineto{\pgfqpoint{3.815809in}{1.388567in}}%
\pgfpathlineto{\pgfqpoint{3.829937in}{1.386778in}}%
\pgfpathlineto{\pgfqpoint{3.838219in}{1.397264in}}%
\pgfpathlineto{\pgfqpoint{3.846494in}{1.407812in}}%
\pgfpathlineto{\pgfqpoint{3.854764in}{1.418419in}}%
\pgfpathlineto{\pgfqpoint{3.863028in}{1.429078in}}%
\pgfpathlineto{\pgfqpoint{3.848912in}{1.430557in}}%
\pgfpathlineto{\pgfqpoint{3.834804in}{1.432109in}}%
\pgfpathlineto{\pgfqpoint{3.820704in}{1.433734in}}%
\pgfpathlineto{\pgfqpoint{3.806611in}{1.435432in}}%
\pgfpathlineto{\pgfqpoint{3.798336in}{1.425075in}}%
\pgfpathlineto{\pgfqpoint{3.790054in}{1.414777in}}%
\pgfpathlineto{\pgfqpoint{3.781767in}{1.404541in}}%
\pgfpathlineto{\pgfqpoint{3.773473in}{1.394373in}}%
\pgfpathclose%
\pgfusepath{fill}%
\end{pgfscope}%
\begin{pgfscope}%
\pgfpathrectangle{\pgfqpoint{1.150000in}{0.150000in}}{\pgfqpoint{5.700000in}{5.700000in}}%
\pgfusepath{clip}%
\pgfsetbuttcap%
\pgfsetroundjoin%
\definecolor{currentfill}{rgb}{0.267004,0.004874,0.329415}%
\pgfsetfillcolor{currentfill}%
\pgfsetfillopacity{0.700000}%
\pgfsetlinewidth{0.000000pt}%
\definecolor{currentstroke}{rgb}{0.000000,0.000000,0.000000}%
\pgfsetstrokecolor{currentstroke}%
\pgfsetdash{}{0pt}%
\pgfpathmoveto{\pgfqpoint{3.301743in}{1.350068in}}%
\pgfpathlineto{\pgfqpoint{3.315768in}{1.344818in}}%
\pgfpathlineto{\pgfqpoint{3.329799in}{1.339645in}}%
\pgfpathlineto{\pgfqpoint{3.343834in}{1.334549in}}%
\pgfpathlineto{\pgfqpoint{3.357875in}{1.329530in}}%
\pgfpathlineto{\pgfqpoint{3.366377in}{1.335924in}}%
\pgfpathlineto{\pgfqpoint{3.374870in}{1.342496in}}%
\pgfpathlineto{\pgfqpoint{3.383353in}{1.349242in}}%
\pgfpathlineto{\pgfqpoint{3.391827in}{1.356154in}}%
\pgfpathlineto{\pgfqpoint{3.377808in}{1.360781in}}%
\pgfpathlineto{\pgfqpoint{3.363794in}{1.365485in}}%
\pgfpathlineto{\pgfqpoint{3.349785in}{1.370266in}}%
\pgfpathlineto{\pgfqpoint{3.335782in}{1.375125in}}%
\pgfpathlineto{\pgfqpoint{3.327287in}{1.368596in}}%
\pgfpathlineto{\pgfqpoint{3.318782in}{1.362240in}}%
\pgfpathlineto{\pgfqpoint{3.310267in}{1.356062in}}%
\pgfpathlineto{\pgfqpoint{3.301743in}{1.350068in}}%
\pgfpathclose%
\pgfusepath{fill}%
\end{pgfscope}%
\begin{pgfscope}%
\pgfpathrectangle{\pgfqpoint{1.150000in}{0.150000in}}{\pgfqpoint{5.700000in}{5.700000in}}%
\pgfusepath{clip}%
\pgfsetbuttcap%
\pgfsetroundjoin%
\definecolor{currentfill}{rgb}{0.269944,0.014625,0.341379}%
\pgfsetfillcolor{currentfill}%
\pgfsetfillopacity{0.700000}%
\pgfsetlinewidth{0.000000pt}%
\definecolor{currentstroke}{rgb}{0.000000,0.000000,0.000000}%
\pgfsetstrokecolor{currentstroke}%
\pgfsetdash{}{0pt}%
\pgfpathmoveto{\pgfqpoint{3.155302in}{1.376177in}}%
\pgfpathlineto{\pgfqpoint{3.169317in}{1.369886in}}%
\pgfpathlineto{\pgfqpoint{3.183336in}{1.363675in}}%
\pgfpathlineto{\pgfqpoint{3.197359in}{1.357542in}}%
\pgfpathlineto{\pgfqpoint{3.211386in}{1.351489in}}%
\pgfpathlineto{\pgfqpoint{3.219978in}{1.356263in}}%
\pgfpathlineto{\pgfqpoint{3.228559in}{1.361251in}}%
\pgfpathlineto{\pgfqpoint{3.237129in}{1.366448in}}%
\pgfpathlineto{\pgfqpoint{3.245688in}{1.371846in}}%
\pgfpathlineto{\pgfqpoint{3.231686in}{1.377486in}}%
\pgfpathlineto{\pgfqpoint{3.217688in}{1.383205in}}%
\pgfpathlineto{\pgfqpoint{3.203695in}{1.389003in}}%
\pgfpathlineto{\pgfqpoint{3.189706in}{1.394881in}}%
\pgfpathlineto{\pgfqpoint{3.181122in}{1.389888in}}%
\pgfpathlineto{\pgfqpoint{3.172527in}{1.385102in}}%
\pgfpathlineto{\pgfqpoint{3.163921in}{1.380529in}}%
\pgfpathlineto{\pgfqpoint{3.155302in}{1.376177in}}%
\pgfpathclose%
\pgfusepath{fill}%
\end{pgfscope}%
\begin{pgfscope}%
\pgfpathrectangle{\pgfqpoint{1.150000in}{0.150000in}}{\pgfqpoint{5.700000in}{5.700000in}}%
\pgfusepath{clip}%
\pgfsetbuttcap%
\pgfsetroundjoin%
\definecolor{currentfill}{rgb}{0.258965,0.251537,0.524736}%
\pgfsetfillcolor{currentfill}%
\pgfsetfillopacity{0.700000}%
\pgfsetlinewidth{0.000000pt}%
\definecolor{currentstroke}{rgb}{0.000000,0.000000,0.000000}%
\pgfsetstrokecolor{currentstroke}%
\pgfsetdash{}{0pt}%
\pgfpathmoveto{\pgfqpoint{4.578938in}{1.822274in}}%
\pgfpathlineto{\pgfqpoint{4.593311in}{1.824909in}}%
\pgfpathlineto{\pgfqpoint{4.607694in}{1.827616in}}%
\pgfpathlineto{\pgfqpoint{4.622089in}{1.830394in}}%
\pgfpathlineto{\pgfqpoint{4.636496in}{1.833244in}}%
\pgfpathlineto{\pgfqpoint{4.644514in}{1.844879in}}%
\pgfpathlineto{\pgfqpoint{4.652527in}{1.856417in}}%
\pgfpathlineto{\pgfqpoint{4.660533in}{1.867856in}}%
\pgfpathlineto{\pgfqpoint{4.668533in}{1.879195in}}%
\pgfpathlineto{\pgfqpoint{4.654133in}{1.876221in}}%
\pgfpathlineto{\pgfqpoint{4.639743in}{1.873317in}}%
\pgfpathlineto{\pgfqpoint{4.625365in}{1.870485in}}%
\pgfpathlineto{\pgfqpoint{4.610999in}{1.867725in}}%
\pgfpathlineto{\pgfqpoint{4.602993in}{1.856503in}}%
\pgfpathlineto{\pgfqpoint{4.594981in}{1.845186in}}%
\pgfpathlineto{\pgfqpoint{4.586962in}{1.833775in}}%
\pgfpathlineto{\pgfqpoint{4.578938in}{1.822274in}}%
\pgfpathclose%
\pgfusepath{fill}%
\end{pgfscope}%
\begin{pgfscope}%
\pgfpathrectangle{\pgfqpoint{1.150000in}{0.150000in}}{\pgfqpoint{5.700000in}{5.700000in}}%
\pgfusepath{clip}%
\pgfsetbuttcap%
\pgfsetroundjoin%
\definecolor{currentfill}{rgb}{0.267004,0.004874,0.329415}%
\pgfsetfillcolor{currentfill}%
\pgfsetfillopacity{0.700000}%
\pgfsetlinewidth{0.000000pt}%
\definecolor{currentstroke}{rgb}{0.000000,0.000000,0.000000}%
\pgfsetstrokecolor{currentstroke}%
\pgfsetdash{}{0pt}%
\pgfpathmoveto{\pgfqpoint{3.447959in}{1.338408in}}%
\pgfpathlineto{\pgfqpoint{3.462005in}{1.334161in}}%
\pgfpathlineto{\pgfqpoint{3.476058in}{1.329990in}}%
\pgfpathlineto{\pgfqpoint{3.490116in}{1.325894in}}%
\pgfpathlineto{\pgfqpoint{3.504180in}{1.321873in}}%
\pgfpathlineto{\pgfqpoint{3.512606in}{1.329704in}}%
\pgfpathlineto{\pgfqpoint{3.521024in}{1.337680in}}%
\pgfpathlineto{\pgfqpoint{3.529434in}{1.345795in}}%
\pgfpathlineto{\pgfqpoint{3.537835in}{1.354044in}}%
\pgfpathlineto{\pgfqpoint{3.523789in}{1.357694in}}%
\pgfpathlineto{\pgfqpoint{3.509749in}{1.361419in}}%
\pgfpathlineto{\pgfqpoint{3.495715in}{1.365219in}}%
\pgfpathlineto{\pgfqpoint{3.481687in}{1.369095in}}%
\pgfpathlineto{\pgfqpoint{3.473267in}{1.361209in}}%
\pgfpathlineto{\pgfqpoint{3.464840in}{1.353463in}}%
\pgfpathlineto{\pgfqpoint{3.456403in}{1.345860in}}%
\pgfpathlineto{\pgfqpoint{3.447959in}{1.338408in}}%
\pgfpathclose%
\pgfusepath{fill}%
\end{pgfscope}%
\begin{pgfscope}%
\pgfpathrectangle{\pgfqpoint{1.150000in}{0.150000in}}{\pgfqpoint{5.700000in}{5.700000in}}%
\pgfusepath{clip}%
\pgfsetbuttcap%
\pgfsetroundjoin%
\definecolor{currentfill}{rgb}{0.271305,0.019942,0.347269}%
\pgfsetfillcolor{currentfill}%
\pgfsetfillopacity{0.700000}%
\pgfsetlinewidth{0.000000pt}%
\definecolor{currentstroke}{rgb}{0.000000,0.000000,0.000000}%
\pgfsetstrokecolor{currentstroke}%
\pgfsetdash{}{0pt}%
\pgfpathmoveto{\pgfqpoint{3.683835in}{1.364555in}}%
\pgfpathlineto{\pgfqpoint{3.697924in}{1.361924in}}%
\pgfpathlineto{\pgfqpoint{3.712020in}{1.359366in}}%
\pgfpathlineto{\pgfqpoint{3.726123in}{1.356881in}}%
\pgfpathlineto{\pgfqpoint{3.740234in}{1.354469in}}%
\pgfpathlineto{\pgfqpoint{3.748553in}{1.364320in}}%
\pgfpathlineto{\pgfqpoint{3.756866in}{1.374257in}}%
\pgfpathlineto{\pgfqpoint{3.765173in}{1.384276in}}%
\pgfpathlineto{\pgfqpoint{3.773473in}{1.394373in}}%
\pgfpathlineto{\pgfqpoint{3.759376in}{1.396454in}}%
\pgfpathlineto{\pgfqpoint{3.745286in}{1.398608in}}%
\pgfpathlineto{\pgfqpoint{3.731203in}{1.400836in}}%
\pgfpathlineto{\pgfqpoint{3.717127in}{1.403137in}}%
\pgfpathlineto{\pgfqpoint{3.708814in}{1.393363in}}%
\pgfpathlineto{\pgfqpoint{3.700494in}{1.383672in}}%
\pgfpathlineto{\pgfqpoint{3.692168in}{1.374068in}}%
\pgfpathlineto{\pgfqpoint{3.683835in}{1.364555in}}%
\pgfpathclose%
\pgfusepath{fill}%
\end{pgfscope}%
\begin{pgfscope}%
\pgfpathrectangle{\pgfqpoint{1.150000in}{0.150000in}}{\pgfqpoint{5.700000in}{5.700000in}}%
\pgfusepath{clip}%
\pgfsetbuttcap%
\pgfsetroundjoin%
\definecolor{currentfill}{rgb}{0.248629,0.278775,0.534556}%
\pgfsetfillcolor{currentfill}%
\pgfsetfillopacity{0.700000}%
\pgfsetlinewidth{0.000000pt}%
\definecolor{currentstroke}{rgb}{0.000000,0.000000,0.000000}%
\pgfsetstrokecolor{currentstroke}%
\pgfsetdash{}{0pt}%
\pgfpathmoveto{\pgfqpoint{4.668533in}{1.879195in}}%
\pgfpathlineto{\pgfqpoint{4.682946in}{1.882241in}}%
\pgfpathlineto{\pgfqpoint{4.697369in}{1.885358in}}%
\pgfpathlineto{\pgfqpoint{4.711805in}{1.888547in}}%
\pgfpathlineto{\pgfqpoint{4.726252in}{1.891806in}}%
\pgfpathlineto{\pgfqpoint{4.734240in}{1.903156in}}%
\pgfpathlineto{\pgfqpoint{4.742222in}{1.914397in}}%
\pgfpathlineto{\pgfqpoint{4.750197in}{1.925529in}}%
\pgfpathlineto{\pgfqpoint{4.758165in}{1.936550in}}%
\pgfpathlineto{\pgfqpoint{4.743724in}{1.933186in}}%
\pgfpathlineto{\pgfqpoint{4.729294in}{1.929894in}}%
\pgfpathlineto{\pgfqpoint{4.714876in}{1.926673in}}%
\pgfpathlineto{\pgfqpoint{4.700470in}{1.923523in}}%
\pgfpathlineto{\pgfqpoint{4.692496in}{1.912598in}}%
\pgfpathlineto{\pgfqpoint{4.684515in}{1.901567in}}%
\pgfpathlineto{\pgfqpoint{4.676527in}{1.890433in}}%
\pgfpathlineto{\pgfqpoint{4.668533in}{1.879195in}}%
\pgfpathclose%
\pgfusepath{fill}%
\end{pgfscope}%
\begin{pgfscope}%
\pgfpathrectangle{\pgfqpoint{1.150000in}{0.150000in}}{\pgfqpoint{5.700000in}{5.700000in}}%
\pgfusepath{clip}%
\pgfsetbuttcap%
\pgfsetroundjoin%
\definecolor{currentfill}{rgb}{0.239346,0.300855,0.540844}%
\pgfsetfillcolor{currentfill}%
\pgfsetfillopacity{0.700000}%
\pgfsetlinewidth{0.000000pt}%
\definecolor{currentstroke}{rgb}{0.000000,0.000000,0.000000}%
\pgfsetstrokecolor{currentstroke}%
\pgfsetdash{}{0pt}%
\pgfpathmoveto{\pgfqpoint{4.758165in}{1.936550in}}%
\pgfpathlineto{\pgfqpoint{4.772618in}{1.939984in}}%
\pgfpathlineto{\pgfqpoint{4.787083in}{1.943490in}}%
\pgfpathlineto{\pgfqpoint{4.801560in}{1.947068in}}%
\pgfpathlineto{\pgfqpoint{4.816049in}{1.950716in}}%
\pgfpathlineto{\pgfqpoint{4.824005in}{1.961716in}}%
\pgfpathlineto{\pgfqpoint{4.831953in}{1.972598in}}%
\pgfpathlineto{\pgfqpoint{4.839895in}{1.983361in}}%
\pgfpathlineto{\pgfqpoint{4.847830in}{1.994004in}}%
\pgfpathlineto{\pgfqpoint{4.833347in}{1.990273in}}%
\pgfpathlineto{\pgfqpoint{4.818876in}{1.986613in}}%
\pgfpathlineto{\pgfqpoint{4.804418in}{1.983025in}}%
\pgfpathlineto{\pgfqpoint{4.789971in}{1.979508in}}%
\pgfpathlineto{\pgfqpoint{4.782030in}{1.968939in}}%
\pgfpathlineto{\pgfqpoint{4.774082in}{1.958256in}}%
\pgfpathlineto{\pgfqpoint{4.766127in}{1.947459in}}%
\pgfpathlineto{\pgfqpoint{4.758165in}{1.936550in}}%
\pgfpathclose%
\pgfusepath{fill}%
\end{pgfscope}%
\begin{pgfscope}%
\pgfpathrectangle{\pgfqpoint{1.150000in}{0.150000in}}{\pgfqpoint{5.700000in}{5.700000in}}%
\pgfusepath{clip}%
\pgfsetbuttcap%
\pgfsetroundjoin%
\definecolor{currentfill}{rgb}{0.227802,0.326594,0.546532}%
\pgfsetfillcolor{currentfill}%
\pgfsetfillopacity{0.700000}%
\pgfsetlinewidth{0.000000pt}%
\definecolor{currentstroke}{rgb}{0.000000,0.000000,0.000000}%
\pgfsetstrokecolor{currentstroke}%
\pgfsetdash{}{0pt}%
\pgfpathmoveto{\pgfqpoint{4.847830in}{1.994004in}}%
\pgfpathlineto{\pgfqpoint{4.862325in}{1.997806in}}%
\pgfpathlineto{\pgfqpoint{4.876832in}{2.001680in}}%
\pgfpathlineto{\pgfqpoint{4.891352in}{2.005624in}}%
\pgfpathlineto{\pgfqpoint{4.905883in}{2.009641in}}%
\pgfpathlineto{\pgfqpoint{4.913804in}{2.020232in}}%
\pgfpathlineto{\pgfqpoint{4.921718in}{2.030697in}}%
\pgfpathlineto{\pgfqpoint{4.929624in}{2.041035in}}%
\pgfpathlineto{\pgfqpoint{4.937523in}{2.051246in}}%
\pgfpathlineto{\pgfqpoint{4.922998in}{2.047169in}}%
\pgfpathlineto{\pgfqpoint{4.908486in}{2.043164in}}%
\pgfpathlineto{\pgfqpoint{4.893985in}{2.039229in}}%
\pgfpathlineto{\pgfqpoint{4.879497in}{2.035367in}}%
\pgfpathlineto{\pgfqpoint{4.871591in}{2.025209in}}%
\pgfpathlineto{\pgfqpoint{4.863678in}{2.014928in}}%
\pgfpathlineto{\pgfqpoint{4.855757in}{2.004527in}}%
\pgfpathlineto{\pgfqpoint{4.847830in}{1.994004in}}%
\pgfpathclose%
\pgfusepath{fill}%
\end{pgfscope}%
\begin{pgfscope}%
\pgfpathrectangle{\pgfqpoint{1.150000in}{0.150000in}}{\pgfqpoint{5.700000in}{5.700000in}}%
\pgfusepath{clip}%
\pgfsetbuttcap%
\pgfsetroundjoin%
\definecolor{currentfill}{rgb}{0.269944,0.014625,0.341379}%
\pgfsetfillcolor{currentfill}%
\pgfsetfillopacity{0.700000}%
\pgfsetlinewidth{0.000000pt}%
\definecolor{currentstroke}{rgb}{0.000000,0.000000,0.000000}%
\pgfsetstrokecolor{currentstroke}%
\pgfsetdash{}{0pt}%
\pgfpathmoveto{\pgfqpoint{3.594082in}{1.340191in}}%
\pgfpathlineto{\pgfqpoint{3.608160in}{1.336914in}}%
\pgfpathlineto{\pgfqpoint{3.622244in}{1.333711in}}%
\pgfpathlineto{\pgfqpoint{3.636334in}{1.330581in}}%
\pgfpathlineto{\pgfqpoint{3.650432in}{1.327525in}}%
\pgfpathlineto{\pgfqpoint{3.658793in}{1.336619in}}%
\pgfpathlineto{\pgfqpoint{3.667147in}{1.345826in}}%
\pgfpathlineto{\pgfqpoint{3.675494in}{1.355139in}}%
\pgfpathlineto{\pgfqpoint{3.683835in}{1.364555in}}%
\pgfpathlineto{\pgfqpoint{3.669752in}{1.367260in}}%
\pgfpathlineto{\pgfqpoint{3.655677in}{1.370039in}}%
\pgfpathlineto{\pgfqpoint{3.641608in}{1.372891in}}%
\pgfpathlineto{\pgfqpoint{3.627546in}{1.375818in}}%
\pgfpathlineto{\pgfqpoint{3.619191in}{1.366745in}}%
\pgfpathlineto{\pgfqpoint{3.610829in}{1.357780in}}%
\pgfpathlineto{\pgfqpoint{3.602459in}{1.348927in}}%
\pgfpathlineto{\pgfqpoint{3.594082in}{1.340191in}}%
\pgfpathclose%
\pgfusepath{fill}%
\end{pgfscope}%
\begin{pgfscope}%
\pgfpathrectangle{\pgfqpoint{1.150000in}{0.150000in}}{\pgfqpoint{5.700000in}{5.700000in}}%
\pgfusepath{clip}%
\pgfsetbuttcap%
\pgfsetroundjoin%
\definecolor{currentfill}{rgb}{0.216210,0.351535,0.550627}%
\pgfsetfillcolor{currentfill}%
\pgfsetfillopacity{0.700000}%
\pgfsetlinewidth{0.000000pt}%
\definecolor{currentstroke}{rgb}{0.000000,0.000000,0.000000}%
\pgfsetstrokecolor{currentstroke}%
\pgfsetdash{}{0pt}%
\pgfpathmoveto{\pgfqpoint{4.937523in}{2.051246in}}%
\pgfpathlineto{\pgfqpoint{4.952061in}{2.055394in}}%
\pgfpathlineto{\pgfqpoint{4.966611in}{2.059613in}}%
\pgfpathlineto{\pgfqpoint{4.981174in}{2.063904in}}%
\pgfpathlineto{\pgfqpoint{4.995749in}{2.068266in}}%
\pgfpathlineto{\pgfqpoint{5.003634in}{2.078396in}}%
\pgfpathlineto{\pgfqpoint{5.011510in}{2.088392in}}%
\pgfpathlineto{\pgfqpoint{5.019378in}{2.098255in}}%
\pgfpathlineto{\pgfqpoint{5.027239in}{2.107983in}}%
\pgfpathlineto{\pgfqpoint{5.012671in}{2.103582in}}%
\pgfpathlineto{\pgfqpoint{4.998116in}{2.099253in}}%
\pgfpathlineto{\pgfqpoint{4.983573in}{2.094995in}}%
\pgfpathlineto{\pgfqpoint{4.969044in}{2.090808in}}%
\pgfpathlineto{\pgfqpoint{4.961175in}{2.081110in}}%
\pgfpathlineto{\pgfqpoint{4.953299in}{2.071283in}}%
\pgfpathlineto{\pgfqpoint{4.945415in}{2.061329in}}%
\pgfpathlineto{\pgfqpoint{4.937523in}{2.051246in}}%
\pgfpathclose%
\pgfusepath{fill}%
\end{pgfscope}%
\begin{pgfscope}%
\pgfpathrectangle{\pgfqpoint{1.150000in}{0.150000in}}{\pgfqpoint{5.700000in}{5.700000in}}%
\pgfusepath{clip}%
\pgfsetbuttcap%
\pgfsetroundjoin%
\definecolor{currentfill}{rgb}{0.283091,0.110553,0.431554}%
\pgfsetfillcolor{currentfill}%
\pgfsetfillopacity{0.700000}%
\pgfsetlinewidth{0.000000pt}%
\definecolor{currentstroke}{rgb}{0.000000,0.000000,0.000000}%
\pgfsetstrokecolor{currentstroke}%
\pgfsetdash{}{0pt}%
\pgfpathmoveto{\pgfqpoint{4.098730in}{1.510362in}}%
\pgfpathlineto{\pgfqpoint{4.112937in}{1.510381in}}%
\pgfpathlineto{\pgfqpoint{4.127153in}{1.510471in}}%
\pgfpathlineto{\pgfqpoint{4.141378in}{1.510632in}}%
\pgfpathlineto{\pgfqpoint{4.155612in}{1.510865in}}%
\pgfpathlineto{\pgfqpoint{4.163791in}{1.522843in}}%
\pgfpathlineto{\pgfqpoint{4.171964in}{1.534813in}}%
\pgfpathlineto{\pgfqpoint{4.180133in}{1.546772in}}%
\pgfpathlineto{\pgfqpoint{4.188296in}{1.558716in}}%
\pgfpathlineto{\pgfqpoint{4.174069in}{1.558233in}}%
\pgfpathlineto{\pgfqpoint{4.159851in}{1.557821in}}%
\pgfpathlineto{\pgfqpoint{4.145642in}{1.557482in}}%
\pgfpathlineto{\pgfqpoint{4.131443in}{1.557214in}}%
\pgfpathlineto{\pgfqpoint{4.123273in}{1.545512in}}%
\pgfpathlineto{\pgfqpoint{4.115097in}{1.533801in}}%
\pgfpathlineto{\pgfqpoint{4.106916in}{1.522083in}}%
\pgfpathlineto{\pgfqpoint{4.098730in}{1.510362in}}%
\pgfpathclose%
\pgfusepath{fill}%
\end{pgfscope}%
\begin{pgfscope}%
\pgfpathrectangle{\pgfqpoint{1.150000in}{0.150000in}}{\pgfqpoint{5.700000in}{5.700000in}}%
\pgfusepath{clip}%
\pgfsetbuttcap%
\pgfsetroundjoin%
\definecolor{currentfill}{rgb}{0.281924,0.089666,0.412415}%
\pgfsetfillcolor{currentfill}%
\pgfsetfillopacity{0.700000}%
\pgfsetlinewidth{0.000000pt}%
\definecolor{currentstroke}{rgb}{0.000000,0.000000,0.000000}%
\pgfsetstrokecolor{currentstroke}%
\pgfsetdash{}{0pt}%
\pgfpathmoveto{\pgfqpoint{4.009161in}{1.465254in}}%
\pgfpathlineto{\pgfqpoint{4.023341in}{1.464715in}}%
\pgfpathlineto{\pgfqpoint{4.037529in}{1.464248in}}%
\pgfpathlineto{\pgfqpoint{4.051727in}{1.463852in}}%
\pgfpathlineto{\pgfqpoint{4.065932in}{1.463529in}}%
\pgfpathlineto{\pgfqpoint{4.074140in}{1.475222in}}%
\pgfpathlineto{\pgfqpoint{4.082342in}{1.486928in}}%
\pgfpathlineto{\pgfqpoint{4.090539in}{1.498643in}}%
\pgfpathlineto{\pgfqpoint{4.098730in}{1.510362in}}%
\pgfpathlineto{\pgfqpoint{4.084532in}{1.510416in}}%
\pgfpathlineto{\pgfqpoint{4.070343in}{1.510541in}}%
\pgfpathlineto{\pgfqpoint{4.056163in}{1.510738in}}%
\pgfpathlineto{\pgfqpoint{4.041992in}{1.511007in}}%
\pgfpathlineto{\pgfqpoint{4.033792in}{1.499550in}}%
\pgfpathlineto{\pgfqpoint{4.025587in}{1.488103in}}%
\pgfpathlineto{\pgfqpoint{4.017377in}{1.476669in}}%
\pgfpathlineto{\pgfqpoint{4.009161in}{1.465254in}}%
\pgfpathclose%
\pgfusepath{fill}%
\end{pgfscope}%
\begin{pgfscope}%
\pgfpathrectangle{\pgfqpoint{1.150000in}{0.150000in}}{\pgfqpoint{5.700000in}{5.700000in}}%
\pgfusepath{clip}%
\pgfsetbuttcap%
\pgfsetroundjoin%
\definecolor{currentfill}{rgb}{0.282884,0.135920,0.453427}%
\pgfsetfillcolor{currentfill}%
\pgfsetfillopacity{0.700000}%
\pgfsetlinewidth{0.000000pt}%
\definecolor{currentstroke}{rgb}{0.000000,0.000000,0.000000}%
\pgfsetstrokecolor{currentstroke}%
\pgfsetdash{}{0pt}%
\pgfpathmoveto{\pgfqpoint{4.188296in}{1.558716in}}%
\pgfpathlineto{\pgfqpoint{4.202533in}{1.559270in}}%
\pgfpathlineto{\pgfqpoint{4.216779in}{1.559896in}}%
\pgfpathlineto{\pgfqpoint{4.231034in}{1.560593in}}%
\pgfpathlineto{\pgfqpoint{4.245300in}{1.561361in}}%
\pgfpathlineto{\pgfqpoint{4.253451in}{1.573524in}}%
\pgfpathlineto{\pgfqpoint{4.261597in}{1.585660in}}%
\pgfpathlineto{\pgfqpoint{4.269738in}{1.597766in}}%
\pgfpathlineto{\pgfqpoint{4.277874in}{1.609838in}}%
\pgfpathlineto{\pgfqpoint{4.263615in}{1.608840in}}%
\pgfpathlineto{\pgfqpoint{4.249366in}{1.607913in}}%
\pgfpathlineto{\pgfqpoint{4.235126in}{1.607058in}}%
\pgfpathlineto{\pgfqpoint{4.220896in}{1.606274in}}%
\pgfpathlineto{\pgfqpoint{4.212754in}{1.594424in}}%
\pgfpathlineto{\pgfqpoint{4.204607in}{1.582545in}}%
\pgfpathlineto{\pgfqpoint{4.196454in}{1.570641in}}%
\pgfpathlineto{\pgfqpoint{4.188296in}{1.558716in}}%
\pgfpathclose%
\pgfusepath{fill}%
\end{pgfscope}%
\begin{pgfscope}%
\pgfpathrectangle{\pgfqpoint{1.150000in}{0.150000in}}{\pgfqpoint{5.700000in}{5.700000in}}%
\pgfusepath{clip}%
\pgfsetbuttcap%
\pgfsetroundjoin%
\definecolor{currentfill}{rgb}{0.269944,0.014625,0.341379}%
\pgfsetfillcolor{currentfill}%
\pgfsetfillopacity{0.700000}%
\pgfsetlinewidth{0.000000pt}%
\definecolor{currentstroke}{rgb}{0.000000,0.000000,0.000000}%
\pgfsetstrokecolor{currentstroke}%
\pgfsetdash{}{0pt}%
\pgfpathmoveto{\pgfqpoint{3.211386in}{1.351489in}}%
\pgfpathlineto{\pgfqpoint{3.225417in}{1.345514in}}%
\pgfpathlineto{\pgfqpoint{3.239453in}{1.339618in}}%
\pgfpathlineto{\pgfqpoint{3.253493in}{1.333799in}}%
\pgfpathlineto{\pgfqpoint{3.267538in}{1.328058in}}%
\pgfpathlineto{\pgfqpoint{3.276105in}{1.333253in}}%
\pgfpathlineto{\pgfqpoint{3.284662in}{1.338658in}}%
\pgfpathlineto{\pgfqpoint{3.293207in}{1.344265in}}%
\pgfpathlineto{\pgfqpoint{3.301743in}{1.350068in}}%
\pgfpathlineto{\pgfqpoint{3.287722in}{1.355396in}}%
\pgfpathlineto{\pgfqpoint{3.273706in}{1.360801in}}%
\pgfpathlineto{\pgfqpoint{3.259695in}{1.366284in}}%
\pgfpathlineto{\pgfqpoint{3.245688in}{1.371846in}}%
\pgfpathlineto{\pgfqpoint{3.237129in}{1.366448in}}%
\pgfpathlineto{\pgfqpoint{3.228559in}{1.361251in}}%
\pgfpathlineto{\pgfqpoint{3.219978in}{1.356263in}}%
\pgfpathlineto{\pgfqpoint{3.211386in}{1.351489in}}%
\pgfpathclose%
\pgfusepath{fill}%
\end{pgfscope}%
\begin{pgfscope}%
\pgfpathrectangle{\pgfqpoint{1.150000in}{0.150000in}}{\pgfqpoint{5.700000in}{5.700000in}}%
\pgfusepath{clip}%
\pgfsetbuttcap%
\pgfsetroundjoin%
\definecolor{currentfill}{rgb}{0.267004,0.004874,0.329415}%
\pgfsetfillcolor{currentfill}%
\pgfsetfillopacity{0.700000}%
\pgfsetlinewidth{0.000000pt}%
\definecolor{currentstroke}{rgb}{0.000000,0.000000,0.000000}%
\pgfsetstrokecolor{currentstroke}%
\pgfsetdash{}{0pt}%
\pgfpathmoveto{\pgfqpoint{3.357875in}{1.329530in}}%
\pgfpathlineto{\pgfqpoint{3.371920in}{1.324587in}}%
\pgfpathlineto{\pgfqpoint{3.385971in}{1.319721in}}%
\pgfpathlineto{\pgfqpoint{3.400027in}{1.314931in}}%
\pgfpathlineto{\pgfqpoint{3.414089in}{1.310216in}}%
\pgfpathlineto{\pgfqpoint{3.422570in}{1.317010in}}%
\pgfpathlineto{\pgfqpoint{3.431042in}{1.323977in}}%
\pgfpathlineto{\pgfqpoint{3.439505in}{1.331112in}}%
\pgfpathlineto{\pgfqpoint{3.447959in}{1.338408in}}%
\pgfpathlineto{\pgfqpoint{3.433917in}{1.342731in}}%
\pgfpathlineto{\pgfqpoint{3.419882in}{1.347129in}}%
\pgfpathlineto{\pgfqpoint{3.405852in}{1.351603in}}%
\pgfpathlineto{\pgfqpoint{3.391827in}{1.356154in}}%
\pgfpathlineto{\pgfqpoint{3.383353in}{1.349242in}}%
\pgfpathlineto{\pgfqpoint{3.374870in}{1.342496in}}%
\pgfpathlineto{\pgfqpoint{3.366377in}{1.335924in}}%
\pgfpathlineto{\pgfqpoint{3.357875in}{1.329530in}}%
\pgfpathclose%
\pgfusepath{fill}%
\end{pgfscope}%
\begin{pgfscope}%
\pgfpathrectangle{\pgfqpoint{1.150000in}{0.150000in}}{\pgfqpoint{5.700000in}{5.700000in}}%
\pgfusepath{clip}%
\pgfsetbuttcap%
\pgfsetroundjoin%
\definecolor{currentfill}{rgb}{0.279566,0.067836,0.391917}%
\pgfsetfillcolor{currentfill}%
\pgfsetfillopacity{0.700000}%
\pgfsetlinewidth{0.000000pt}%
\definecolor{currentstroke}{rgb}{0.000000,0.000000,0.000000}%
\pgfsetstrokecolor{currentstroke}%
\pgfsetdash{}{0pt}%
\pgfpathmoveto{\pgfqpoint{3.919571in}{1.423886in}}%
\pgfpathlineto{\pgfqpoint{3.933726in}{1.422769in}}%
\pgfpathlineto{\pgfqpoint{3.947890in}{1.421724in}}%
\pgfpathlineto{\pgfqpoint{3.962063in}{1.420750in}}%
\pgfpathlineto{\pgfqpoint{3.976243in}{1.419849in}}%
\pgfpathlineto{\pgfqpoint{3.984481in}{1.431153in}}%
\pgfpathlineto{\pgfqpoint{3.992713in}{1.442492in}}%
\pgfpathlineto{\pgfqpoint{4.000940in}{1.453860in}}%
\pgfpathlineto{\pgfqpoint{4.009161in}{1.465254in}}%
\pgfpathlineto{\pgfqpoint{3.994990in}{1.465865in}}%
\pgfpathlineto{\pgfqpoint{3.980827in}{1.466547in}}%
\pgfpathlineto{\pgfqpoint{3.966672in}{1.467302in}}%
\pgfpathlineto{\pgfqpoint{3.952527in}{1.468129in}}%
\pgfpathlineto{\pgfqpoint{3.944296in}{1.457018in}}%
\pgfpathlineto{\pgfqpoint{3.936060in}{1.445937in}}%
\pgfpathlineto{\pgfqpoint{3.927818in}{1.434892in}}%
\pgfpathlineto{\pgfqpoint{3.919571in}{1.423886in}}%
\pgfpathclose%
\pgfusepath{fill}%
\end{pgfscope}%
\begin{pgfscope}%
\pgfpathrectangle{\pgfqpoint{1.150000in}{0.150000in}}{\pgfqpoint{5.700000in}{5.700000in}}%
\pgfusepath{clip}%
\pgfsetbuttcap%
\pgfsetroundjoin%
\definecolor{currentfill}{rgb}{0.280868,0.160771,0.472899}%
\pgfsetfillcolor{currentfill}%
\pgfsetfillopacity{0.700000}%
\pgfsetlinewidth{0.000000pt}%
\definecolor{currentstroke}{rgb}{0.000000,0.000000,0.000000}%
\pgfsetstrokecolor{currentstroke}%
\pgfsetdash{}{0pt}%
\pgfpathmoveto{\pgfqpoint{4.277874in}{1.609838in}}%
\pgfpathlineto{\pgfqpoint{4.292143in}{1.610908in}}%
\pgfpathlineto{\pgfqpoint{4.306421in}{1.612048in}}%
\pgfpathlineto{\pgfqpoint{4.320710in}{1.613260in}}%
\pgfpathlineto{\pgfqpoint{4.335009in}{1.614543in}}%
\pgfpathlineto{\pgfqpoint{4.343134in}{1.626796in}}%
\pgfpathlineto{\pgfqpoint{4.351253in}{1.639005in}}%
\pgfpathlineto{\pgfqpoint{4.359367in}{1.651166in}}%
\pgfpathlineto{\pgfqpoint{4.367475in}{1.663276in}}%
\pgfpathlineto{\pgfqpoint{4.353182in}{1.661784in}}%
\pgfpathlineto{\pgfqpoint{4.338900in}{1.660363in}}%
\pgfpathlineto{\pgfqpoint{4.324627in}{1.659013in}}%
\pgfpathlineto{\pgfqpoint{4.310364in}{1.657735in}}%
\pgfpathlineto{\pgfqpoint{4.302250in}{1.645825in}}%
\pgfpathlineto{\pgfqpoint{4.294130in}{1.633871in}}%
\pgfpathlineto{\pgfqpoint{4.286004in}{1.621874in}}%
\pgfpathlineto{\pgfqpoint{4.277874in}{1.609838in}}%
\pgfpathclose%
\pgfusepath{fill}%
\end{pgfscope}%
\begin{pgfscope}%
\pgfpathrectangle{\pgfqpoint{1.150000in}{0.150000in}}{\pgfqpoint{5.700000in}{5.700000in}}%
\pgfusepath{clip}%
\pgfsetbuttcap%
\pgfsetroundjoin%
\definecolor{currentfill}{rgb}{0.204903,0.375746,0.553533}%
\pgfsetfillcolor{currentfill}%
\pgfsetfillopacity{0.700000}%
\pgfsetlinewidth{0.000000pt}%
\definecolor{currentstroke}{rgb}{0.000000,0.000000,0.000000}%
\pgfsetstrokecolor{currentstroke}%
\pgfsetdash{}{0pt}%
\pgfpathmoveto{\pgfqpoint{5.027239in}{2.107983in}}%
\pgfpathlineto{\pgfqpoint{5.041820in}{2.112456in}}%
\pgfpathlineto{\pgfqpoint{5.056414in}{2.116999in}}%
\pgfpathlineto{\pgfqpoint{5.071020in}{2.121614in}}%
\pgfpathlineto{\pgfqpoint{5.085640in}{2.126301in}}%
\pgfpathlineto{\pgfqpoint{5.093485in}{2.135921in}}%
\pgfpathlineto{\pgfqpoint{5.101321in}{2.145401in}}%
\pgfpathlineto{\pgfqpoint{5.109149in}{2.154743in}}%
\pgfpathlineto{\pgfqpoint{5.116969in}{2.163945in}}%
\pgfpathlineto{\pgfqpoint{5.102358in}{2.159242in}}%
\pgfpathlineto{\pgfqpoint{5.087760in}{2.154610in}}%
\pgfpathlineto{\pgfqpoint{5.073175in}{2.150050in}}%
\pgfpathlineto{\pgfqpoint{5.058602in}{2.145561in}}%
\pgfpathlineto{\pgfqpoint{5.050774in}{2.136367in}}%
\pgfpathlineto{\pgfqpoint{5.042937in}{2.127039in}}%
\pgfpathlineto{\pgfqpoint{5.035092in}{2.117578in}}%
\pgfpathlineto{\pgfqpoint{5.027239in}{2.107983in}}%
\pgfpathclose%
\pgfusepath{fill}%
\end{pgfscope}%
\begin{pgfscope}%
\pgfpathrectangle{\pgfqpoint{1.150000in}{0.150000in}}{\pgfqpoint{5.700000in}{5.700000in}}%
\pgfusepath{clip}%
\pgfsetbuttcap%
\pgfsetroundjoin%
\definecolor{currentfill}{rgb}{0.277134,0.185228,0.489898}%
\pgfsetfillcolor{currentfill}%
\pgfsetfillopacity{0.700000}%
\pgfsetlinewidth{0.000000pt}%
\definecolor{currentstroke}{rgb}{0.000000,0.000000,0.000000}%
\pgfsetstrokecolor{currentstroke}%
\pgfsetdash{}{0pt}%
\pgfpathmoveto{\pgfqpoint{4.367475in}{1.663276in}}%
\pgfpathlineto{\pgfqpoint{4.381779in}{1.664840in}}%
\pgfpathlineto{\pgfqpoint{4.396092in}{1.666474in}}%
\pgfpathlineto{\pgfqpoint{4.410416in}{1.668180in}}%
\pgfpathlineto{\pgfqpoint{4.424751in}{1.669958in}}%
\pgfpathlineto{\pgfqpoint{4.432849in}{1.682211in}}%
\pgfpathlineto{\pgfqpoint{4.440941in}{1.694404in}}%
\pgfpathlineto{\pgfqpoint{4.449028in}{1.706533in}}%
\pgfpathlineto{\pgfqpoint{4.457109in}{1.718595in}}%
\pgfpathlineto{\pgfqpoint{4.442780in}{1.716630in}}%
\pgfpathlineto{\pgfqpoint{4.428462in}{1.714735in}}%
\pgfpathlineto{\pgfqpoint{4.414154in}{1.712912in}}%
\pgfpathlineto{\pgfqpoint{4.399856in}{1.711160in}}%
\pgfpathlineto{\pgfqpoint{4.391769in}{1.699278in}}%
\pgfpathlineto{\pgfqpoint{4.383677in}{1.687335in}}%
\pgfpathlineto{\pgfqpoint{4.375579in}{1.675333in}}%
\pgfpathlineto{\pgfqpoint{4.367475in}{1.663276in}}%
\pgfpathclose%
\pgfusepath{fill}%
\end{pgfscope}%
\begin{pgfscope}%
\pgfpathrectangle{\pgfqpoint{1.150000in}{0.150000in}}{\pgfqpoint{5.700000in}{5.700000in}}%
\pgfusepath{clip}%
\pgfsetbuttcap%
\pgfsetroundjoin%
\definecolor{currentfill}{rgb}{0.276022,0.044167,0.370164}%
\pgfsetfillcolor{currentfill}%
\pgfsetfillopacity{0.700000}%
\pgfsetlinewidth{0.000000pt}%
\definecolor{currentstroke}{rgb}{0.000000,0.000000,0.000000}%
\pgfsetstrokecolor{currentstroke}%
\pgfsetdash{}{0pt}%
\pgfpathmoveto{\pgfqpoint{3.829937in}{1.386778in}}%
\pgfpathlineto{\pgfqpoint{3.844072in}{1.385060in}}%
\pgfpathlineto{\pgfqpoint{3.858214in}{1.383416in}}%
\pgfpathlineto{\pgfqpoint{3.872365in}{1.381843in}}%
\pgfpathlineto{\pgfqpoint{3.886523in}{1.380343in}}%
\pgfpathlineto{\pgfqpoint{3.894794in}{1.391148in}}%
\pgfpathlineto{\pgfqpoint{3.903058in}{1.402009in}}%
\pgfpathlineto{\pgfqpoint{3.911317in}{1.412924in}}%
\pgfpathlineto{\pgfqpoint{3.919571in}{1.423886in}}%
\pgfpathlineto{\pgfqpoint{3.905423in}{1.425076in}}%
\pgfpathlineto{\pgfqpoint{3.891283in}{1.426337in}}%
\pgfpathlineto{\pgfqpoint{3.877152in}{1.427671in}}%
\pgfpathlineto{\pgfqpoint{3.863028in}{1.429078in}}%
\pgfpathlineto{\pgfqpoint{3.854764in}{1.418419in}}%
\pgfpathlineto{\pgfqpoint{3.846494in}{1.407812in}}%
\pgfpathlineto{\pgfqpoint{3.838219in}{1.397264in}}%
\pgfpathlineto{\pgfqpoint{3.829937in}{1.386778in}}%
\pgfpathclose%
\pgfusepath{fill}%
\end{pgfscope}%
\begin{pgfscope}%
\pgfpathrectangle{\pgfqpoint{1.150000in}{0.150000in}}{\pgfqpoint{5.700000in}{5.700000in}}%
\pgfusepath{clip}%
\pgfsetbuttcap%
\pgfsetroundjoin%
\definecolor{currentfill}{rgb}{0.195860,0.395433,0.555276}%
\pgfsetfillcolor{currentfill}%
\pgfsetfillopacity{0.700000}%
\pgfsetlinewidth{0.000000pt}%
\definecolor{currentstroke}{rgb}{0.000000,0.000000,0.000000}%
\pgfsetstrokecolor{currentstroke}%
\pgfsetdash{}{0pt}%
\pgfpathmoveto{\pgfqpoint{5.116969in}{2.163945in}}%
\pgfpathlineto{\pgfqpoint{5.131594in}{2.168720in}}%
\pgfpathlineto{\pgfqpoint{5.146231in}{2.173566in}}%
\pgfpathlineto{\pgfqpoint{5.160882in}{2.178484in}}%
\pgfpathlineto{\pgfqpoint{5.175546in}{2.183473in}}%
\pgfpathlineto{\pgfqpoint{5.183349in}{2.192540in}}%
\pgfpathlineto{\pgfqpoint{5.191142in}{2.201463in}}%
\pgfpathlineto{\pgfqpoint{5.198928in}{2.210243in}}%
\pgfpathlineto{\pgfqpoint{5.206704in}{2.218881in}}%
\pgfpathlineto{\pgfqpoint{5.192049in}{2.213898in}}%
\pgfpathlineto{\pgfqpoint{5.177408in}{2.208986in}}%
\pgfpathlineto{\pgfqpoint{5.162780in}{2.204145in}}%
\pgfpathlineto{\pgfqpoint{5.148165in}{2.199376in}}%
\pgfpathlineto{\pgfqpoint{5.140379in}{2.190724in}}%
\pgfpathlineto{\pgfqpoint{5.132584in}{2.181936in}}%
\pgfpathlineto{\pgfqpoint{5.124781in}{2.173009in}}%
\pgfpathlineto{\pgfqpoint{5.116969in}{2.163945in}}%
\pgfpathclose%
\pgfusepath{fill}%
\end{pgfscope}%
\begin{pgfscope}%
\pgfpathrectangle{\pgfqpoint{1.150000in}{0.150000in}}{\pgfqpoint{5.700000in}{5.700000in}}%
\pgfusepath{clip}%
\pgfsetbuttcap%
\pgfsetroundjoin%
\definecolor{currentfill}{rgb}{0.267004,0.004874,0.329415}%
\pgfsetfillcolor{currentfill}%
\pgfsetfillopacity{0.700000}%
\pgfsetlinewidth{0.000000pt}%
\definecolor{currentstroke}{rgb}{0.000000,0.000000,0.000000}%
\pgfsetstrokecolor{currentstroke}%
\pgfsetdash{}{0pt}%
\pgfpathmoveto{\pgfqpoint{3.504180in}{1.321873in}}%
\pgfpathlineto{\pgfqpoint{3.518250in}{1.317926in}}%
\pgfpathlineto{\pgfqpoint{3.532326in}{1.314055in}}%
\pgfpathlineto{\pgfqpoint{3.546408in}{1.310257in}}%
\pgfpathlineto{\pgfqpoint{3.560496in}{1.306534in}}%
\pgfpathlineto{\pgfqpoint{3.568904in}{1.314745in}}%
\pgfpathlineto{\pgfqpoint{3.577305in}{1.323095in}}%
\pgfpathlineto{\pgfqpoint{3.585697in}{1.331579in}}%
\pgfpathlineto{\pgfqpoint{3.594082in}{1.340191in}}%
\pgfpathlineto{\pgfqpoint{3.580011in}{1.343543in}}%
\pgfpathlineto{\pgfqpoint{3.565946in}{1.346969in}}%
\pgfpathlineto{\pgfqpoint{3.551888in}{1.350469in}}%
\pgfpathlineto{\pgfqpoint{3.537835in}{1.354044in}}%
\pgfpathlineto{\pgfqpoint{3.529434in}{1.345795in}}%
\pgfpathlineto{\pgfqpoint{3.521024in}{1.337680in}}%
\pgfpathlineto{\pgfqpoint{3.512606in}{1.329704in}}%
\pgfpathlineto{\pgfqpoint{3.504180in}{1.321873in}}%
\pgfpathclose%
\pgfusepath{fill}%
\end{pgfscope}%
\begin{pgfscope}%
\pgfpathrectangle{\pgfqpoint{1.150000in}{0.150000in}}{\pgfqpoint{5.700000in}{5.700000in}}%
\pgfusepath{clip}%
\pgfsetbuttcap%
\pgfsetroundjoin%
\definecolor{currentfill}{rgb}{0.171176,0.452530,0.557965}%
\pgfsetfillcolor{currentfill}%
\pgfsetfillopacity{0.700000}%
\pgfsetlinewidth{0.000000pt}%
\definecolor{currentstroke}{rgb}{0.000000,0.000000,0.000000}%
\pgfsetstrokecolor{currentstroke}%
\pgfsetdash{}{0pt}%
\pgfpathmoveto{\pgfqpoint{5.386145in}{2.324781in}}%
\pgfpathlineto{\pgfqpoint{5.400900in}{2.330330in}}%
\pgfpathlineto{\pgfqpoint{5.415670in}{2.335951in}}%
\pgfpathlineto{\pgfqpoint{5.430454in}{2.341643in}}%
\pgfpathlineto{\pgfqpoint{5.438115in}{2.348866in}}%
\pgfpathlineto{\pgfqpoint{5.445768in}{2.355942in}}%
\pgfpathlineto{\pgfqpoint{5.453410in}{2.362872in}}%
\pgfpathlineto{\pgfqpoint{5.461042in}{2.369660in}}%
\pgfpathlineto{\pgfqpoint{5.446272in}{2.364041in}}%
\pgfpathlineto{\pgfqpoint{5.431515in}{2.358493in}}%
\pgfpathlineto{\pgfqpoint{5.416773in}{2.353017in}}%
\pgfpathlineto{\pgfqpoint{5.409130in}{2.346168in}}%
\pgfpathlineto{\pgfqpoint{5.401478in}{2.339181in}}%
\pgfpathlineto{\pgfqpoint{5.393816in}{2.332052in}}%
\pgfpathlineto{\pgfqpoint{5.386145in}{2.324781in}}%
\pgfpathclose%
\pgfusepath{fill}%
\end{pgfscope}%
\begin{pgfscope}%
\pgfpathrectangle{\pgfqpoint{1.150000in}{0.150000in}}{\pgfqpoint{5.700000in}{5.700000in}}%
\pgfusepath{clip}%
\pgfsetbuttcap%
\pgfsetroundjoin%
\definecolor{currentfill}{rgb}{0.270595,0.214069,0.507052}%
\pgfsetfillcolor{currentfill}%
\pgfsetfillopacity{0.700000}%
\pgfsetlinewidth{0.000000pt}%
\definecolor{currentstroke}{rgb}{0.000000,0.000000,0.000000}%
\pgfsetstrokecolor{currentstroke}%
\pgfsetdash{}{0pt}%
\pgfpathmoveto{\pgfqpoint{4.457109in}{1.718595in}}%
\pgfpathlineto{\pgfqpoint{4.471449in}{1.720632in}}%
\pgfpathlineto{\pgfqpoint{4.485800in}{1.722740in}}%
\pgfpathlineto{\pgfqpoint{4.500161in}{1.724919in}}%
\pgfpathlineto{\pgfqpoint{4.514533in}{1.727170in}}%
\pgfpathlineto{\pgfqpoint{4.522604in}{1.739339in}}%
\pgfpathlineto{\pgfqpoint{4.530669in}{1.751433in}}%
\pgfpathlineto{\pgfqpoint{4.538728in}{1.763448in}}%
\pgfpathlineto{\pgfqpoint{4.546782in}{1.775382in}}%
\pgfpathlineto{\pgfqpoint{4.532415in}{1.772964in}}%
\pgfpathlineto{\pgfqpoint{4.518059in}{1.770617in}}%
\pgfpathlineto{\pgfqpoint{4.503714in}{1.768342in}}%
\pgfpathlineto{\pgfqpoint{4.489380in}{1.766137in}}%
\pgfpathlineto{\pgfqpoint{4.481321in}{1.754363in}}%
\pgfpathlineto{\pgfqpoint{4.473256in}{1.742513in}}%
\pgfpathlineto{\pgfqpoint{4.465185in}{1.730589in}}%
\pgfpathlineto{\pgfqpoint{4.457109in}{1.718595in}}%
\pgfpathclose%
\pgfusepath{fill}%
\end{pgfscope}%
\begin{pgfscope}%
\pgfpathrectangle{\pgfqpoint{1.150000in}{0.150000in}}{\pgfqpoint{5.700000in}{5.700000in}}%
\pgfusepath{clip}%
\pgfsetbuttcap%
\pgfsetroundjoin%
\definecolor{currentfill}{rgb}{0.187231,0.414746,0.556547}%
\pgfsetfillcolor{currentfill}%
\pgfsetfillopacity{0.700000}%
\pgfsetlinewidth{0.000000pt}%
\definecolor{currentstroke}{rgb}{0.000000,0.000000,0.000000}%
\pgfsetstrokecolor{currentstroke}%
\pgfsetdash{}{0pt}%
\pgfpathmoveto{\pgfqpoint{5.206704in}{2.218881in}}%
\pgfpathlineto{\pgfqpoint{5.221372in}{2.223936in}}%
\pgfpathlineto{\pgfqpoint{5.236054in}{2.229063in}}%
\pgfpathlineto{\pgfqpoint{5.250749in}{2.234261in}}%
\pgfpathlineto{\pgfqpoint{5.265458in}{2.239531in}}%
\pgfpathlineto{\pgfqpoint{5.273215in}{2.248007in}}%
\pgfpathlineto{\pgfqpoint{5.280964in}{2.256338in}}%
\pgfpathlineto{\pgfqpoint{5.288703in}{2.264522in}}%
\pgfpathlineto{\pgfqpoint{5.296433in}{2.272563in}}%
\pgfpathlineto{\pgfqpoint{5.281735in}{2.267321in}}%
\pgfpathlineto{\pgfqpoint{5.267050in}{2.262151in}}%
\pgfpathlineto{\pgfqpoint{5.252379in}{2.257052in}}%
\pgfpathlineto{\pgfqpoint{5.237722in}{2.252025in}}%
\pgfpathlineto{\pgfqpoint{5.229980in}{2.243949in}}%
\pgfpathlineto{\pgfqpoint{5.222230in}{2.235733in}}%
\pgfpathlineto{\pgfqpoint{5.214472in}{2.227378in}}%
\pgfpathlineto{\pgfqpoint{5.206704in}{2.218881in}}%
\pgfpathclose%
\pgfusepath{fill}%
\end{pgfscope}%
\begin{pgfscope}%
\pgfpathrectangle{\pgfqpoint{1.150000in}{0.150000in}}{\pgfqpoint{5.700000in}{5.700000in}}%
\pgfusepath{clip}%
\pgfsetbuttcap%
\pgfsetroundjoin%
\definecolor{currentfill}{rgb}{0.177423,0.437527,0.557565}%
\pgfsetfillcolor{currentfill}%
\pgfsetfillopacity{0.700000}%
\pgfsetlinewidth{0.000000pt}%
\definecolor{currentstroke}{rgb}{0.000000,0.000000,0.000000}%
\pgfsetstrokecolor{currentstroke}%
\pgfsetdash{}{0pt}%
\pgfpathmoveto{\pgfqpoint{5.296433in}{2.272563in}}%
\pgfpathlineto{\pgfqpoint{5.311145in}{2.277876in}}%
\pgfpathlineto{\pgfqpoint{5.325871in}{2.283261in}}%
\pgfpathlineto{\pgfqpoint{5.340610in}{2.288717in}}%
\pgfpathlineto{\pgfqpoint{5.355364in}{2.294245in}}%
\pgfpathlineto{\pgfqpoint{5.363073in}{2.302099in}}%
\pgfpathlineto{\pgfqpoint{5.370773in}{2.309806in}}%
\pgfpathlineto{\pgfqpoint{5.378464in}{2.317366in}}%
\pgfpathlineto{\pgfqpoint{5.386145in}{2.324781in}}%
\pgfpathlineto{\pgfqpoint{5.371403in}{2.319304in}}%
\pgfpathlineto{\pgfqpoint{5.356675in}{2.313898in}}%
\pgfpathlineto{\pgfqpoint{5.341962in}{2.308563in}}%
\pgfpathlineto{\pgfqpoint{5.327261in}{2.303300in}}%
\pgfpathlineto{\pgfqpoint{5.319568in}{2.295827in}}%
\pgfpathlineto{\pgfqpoint{5.311866in}{2.288214in}}%
\pgfpathlineto{\pgfqpoint{5.304154in}{2.280459in}}%
\pgfpathlineto{\pgfqpoint{5.296433in}{2.272563in}}%
\pgfpathclose%
\pgfusepath{fill}%
\end{pgfscope}%
\begin{pgfscope}%
\pgfpathrectangle{\pgfqpoint{1.150000in}{0.150000in}}{\pgfqpoint{5.700000in}{5.700000in}}%
\pgfusepath{clip}%
\pgfsetbuttcap%
\pgfsetroundjoin%
\definecolor{currentfill}{rgb}{0.273809,0.031497,0.358853}%
\pgfsetfillcolor{currentfill}%
\pgfsetfillopacity{0.700000}%
\pgfsetlinewidth{0.000000pt}%
\definecolor{currentstroke}{rgb}{0.000000,0.000000,0.000000}%
\pgfsetstrokecolor{currentstroke}%
\pgfsetdash{}{0pt}%
\pgfpathmoveto{\pgfqpoint{3.740234in}{1.354469in}}%
\pgfpathlineto{\pgfqpoint{3.754351in}{1.352130in}}%
\pgfpathlineto{\pgfqpoint{3.768476in}{1.349864in}}%
\pgfpathlineto{\pgfqpoint{3.782608in}{1.347671in}}%
\pgfpathlineto{\pgfqpoint{3.796748in}{1.345550in}}%
\pgfpathlineto{\pgfqpoint{3.805054in}{1.355740in}}%
\pgfpathlineto{\pgfqpoint{3.813355in}{1.366011in}}%
\pgfpathlineto{\pgfqpoint{3.821649in}{1.376358in}}%
\pgfpathlineto{\pgfqpoint{3.829937in}{1.386778in}}%
\pgfpathlineto{\pgfqpoint{3.815809in}{1.388567in}}%
\pgfpathlineto{\pgfqpoint{3.801690in}{1.390430in}}%
\pgfpathlineto{\pgfqpoint{3.787578in}{1.392365in}}%
\pgfpathlineto{\pgfqpoint{3.773473in}{1.394373in}}%
\pgfpathlineto{\pgfqpoint{3.765173in}{1.384276in}}%
\pgfpathlineto{\pgfqpoint{3.756866in}{1.374257in}}%
\pgfpathlineto{\pgfqpoint{3.748553in}{1.364320in}}%
\pgfpathlineto{\pgfqpoint{3.740234in}{1.354469in}}%
\pgfpathclose%
\pgfusepath{fill}%
\end{pgfscope}%
\begin{pgfscope}%
\pgfpathrectangle{\pgfqpoint{1.150000in}{0.150000in}}{\pgfqpoint{5.700000in}{5.700000in}}%
\pgfusepath{clip}%
\pgfsetbuttcap%
\pgfsetroundjoin%
\definecolor{currentfill}{rgb}{0.262138,0.242286,0.520837}%
\pgfsetfillcolor{currentfill}%
\pgfsetfillopacity{0.700000}%
\pgfsetlinewidth{0.000000pt}%
\definecolor{currentstroke}{rgb}{0.000000,0.000000,0.000000}%
\pgfsetstrokecolor{currentstroke}%
\pgfsetdash{}{0pt}%
\pgfpathmoveto{\pgfqpoint{4.546782in}{1.775382in}}%
\pgfpathlineto{\pgfqpoint{4.561160in}{1.777872in}}%
\pgfpathlineto{\pgfqpoint{4.575549in}{1.780432in}}%
\pgfpathlineto{\pgfqpoint{4.589949in}{1.783063in}}%
\pgfpathlineto{\pgfqpoint{4.604361in}{1.785766in}}%
\pgfpathlineto{\pgfqpoint{4.612403in}{1.797772in}}%
\pgfpathlineto{\pgfqpoint{4.620440in}{1.809689in}}%
\pgfpathlineto{\pgfqpoint{4.628471in}{1.821513in}}%
\pgfpathlineto{\pgfqpoint{4.636496in}{1.833244in}}%
\pgfpathlineto{\pgfqpoint{4.622089in}{1.830394in}}%
\pgfpathlineto{\pgfqpoint{4.607694in}{1.827616in}}%
\pgfpathlineto{\pgfqpoint{4.593311in}{1.824909in}}%
\pgfpathlineto{\pgfqpoint{4.578938in}{1.822274in}}%
\pgfpathlineto{\pgfqpoint{4.570908in}{1.810681in}}%
\pgfpathlineto{\pgfqpoint{4.562872in}{1.799001in}}%
\pgfpathlineto{\pgfqpoint{4.554830in}{1.787234in}}%
\pgfpathlineto{\pgfqpoint{4.546782in}{1.775382in}}%
\pgfpathclose%
\pgfusepath{fill}%
\end{pgfscope}%
\begin{pgfscope}%
\pgfpathrectangle{\pgfqpoint{1.150000in}{0.150000in}}{\pgfqpoint{5.700000in}{5.700000in}}%
\pgfusepath{clip}%
\pgfsetbuttcap%
\pgfsetroundjoin%
\definecolor{currentfill}{rgb}{0.252194,0.269783,0.531579}%
\pgfsetfillcolor{currentfill}%
\pgfsetfillopacity{0.700000}%
\pgfsetlinewidth{0.000000pt}%
\definecolor{currentstroke}{rgb}{0.000000,0.000000,0.000000}%
\pgfsetstrokecolor{currentstroke}%
\pgfsetdash{}{0pt}%
\pgfpathmoveto{\pgfqpoint{4.636496in}{1.833244in}}%
\pgfpathlineto{\pgfqpoint{4.650914in}{1.836164in}}%
\pgfpathlineto{\pgfqpoint{4.665343in}{1.839156in}}%
\pgfpathlineto{\pgfqpoint{4.679784in}{1.842219in}}%
\pgfpathlineto{\pgfqpoint{4.694236in}{1.845353in}}%
\pgfpathlineto{\pgfqpoint{4.702249in}{1.857122in}}%
\pgfpathlineto{\pgfqpoint{4.710257in}{1.868788in}}%
\pgfpathlineto{\pgfqpoint{4.718258in}{1.880350in}}%
\pgfpathlineto{\pgfqpoint{4.726252in}{1.891806in}}%
\pgfpathlineto{\pgfqpoint{4.711805in}{1.888547in}}%
\pgfpathlineto{\pgfqpoint{4.697369in}{1.885358in}}%
\pgfpathlineto{\pgfqpoint{4.682946in}{1.882241in}}%
\pgfpathlineto{\pgfqpoint{4.668533in}{1.879195in}}%
\pgfpathlineto{\pgfqpoint{4.660533in}{1.867856in}}%
\pgfpathlineto{\pgfqpoint{4.652527in}{1.856417in}}%
\pgfpathlineto{\pgfqpoint{4.644514in}{1.844879in}}%
\pgfpathlineto{\pgfqpoint{4.636496in}{1.833244in}}%
\pgfpathclose%
\pgfusepath{fill}%
\end{pgfscope}%
\begin{pgfscope}%
\pgfpathrectangle{\pgfqpoint{1.150000in}{0.150000in}}{\pgfqpoint{5.700000in}{5.700000in}}%
\pgfusepath{clip}%
\pgfsetbuttcap%
\pgfsetroundjoin%
\definecolor{currentfill}{rgb}{0.269944,0.014625,0.341379}%
\pgfsetfillcolor{currentfill}%
\pgfsetfillopacity{0.700000}%
\pgfsetlinewidth{0.000000pt}%
\definecolor{currentstroke}{rgb}{0.000000,0.000000,0.000000}%
\pgfsetstrokecolor{currentstroke}%
\pgfsetdash{}{0pt}%
\pgfpathmoveto{\pgfqpoint{3.650432in}{1.327525in}}%
\pgfpathlineto{\pgfqpoint{3.664535in}{1.324542in}}%
\pgfpathlineto{\pgfqpoint{3.678646in}{1.321633in}}%
\pgfpathlineto{\pgfqpoint{3.692764in}{1.318797in}}%
\pgfpathlineto{\pgfqpoint{3.706888in}{1.316034in}}%
\pgfpathlineto{\pgfqpoint{3.715235in}{1.325487in}}%
\pgfpathlineto{\pgfqpoint{3.723575in}{1.335048in}}%
\pgfpathlineto{\pgfqpoint{3.731907in}{1.344710in}}%
\pgfpathlineto{\pgfqpoint{3.740234in}{1.354469in}}%
\pgfpathlineto{\pgfqpoint{3.726123in}{1.356881in}}%
\pgfpathlineto{\pgfqpoint{3.712020in}{1.359366in}}%
\pgfpathlineto{\pgfqpoint{3.697924in}{1.361924in}}%
\pgfpathlineto{\pgfqpoint{3.683835in}{1.364555in}}%
\pgfpathlineto{\pgfqpoint{3.675494in}{1.355139in}}%
\pgfpathlineto{\pgfqpoint{3.667147in}{1.345826in}}%
\pgfpathlineto{\pgfqpoint{3.658793in}{1.336619in}}%
\pgfpathlineto{\pgfqpoint{3.650432in}{1.327525in}}%
\pgfpathclose%
\pgfusepath{fill}%
\end{pgfscope}%
\begin{pgfscope}%
\pgfpathrectangle{\pgfqpoint{1.150000in}{0.150000in}}{\pgfqpoint{5.700000in}{5.700000in}}%
\pgfusepath{clip}%
\pgfsetbuttcap%
\pgfsetroundjoin%
\definecolor{currentfill}{rgb}{0.268510,0.009605,0.335427}%
\pgfsetfillcolor{currentfill}%
\pgfsetfillopacity{0.700000}%
\pgfsetlinewidth{0.000000pt}%
\definecolor{currentstroke}{rgb}{0.000000,0.000000,0.000000}%
\pgfsetstrokecolor{currentstroke}%
\pgfsetdash{}{0pt}%
\pgfpathmoveto{\pgfqpoint{3.267538in}{1.328058in}}%
\pgfpathlineto{\pgfqpoint{3.281587in}{1.322395in}}%
\pgfpathlineto{\pgfqpoint{3.295641in}{1.316809in}}%
\pgfpathlineto{\pgfqpoint{3.309700in}{1.311300in}}%
\pgfpathlineto{\pgfqpoint{3.323763in}{1.305868in}}%
\pgfpathlineto{\pgfqpoint{3.332306in}{1.311484in}}%
\pgfpathlineto{\pgfqpoint{3.340839in}{1.317304in}}%
\pgfpathlineto{\pgfqpoint{3.349362in}{1.323321in}}%
\pgfpathlineto{\pgfqpoint{3.357875in}{1.329530in}}%
\pgfpathlineto{\pgfqpoint{3.343834in}{1.334549in}}%
\pgfpathlineto{\pgfqpoint{3.329799in}{1.339645in}}%
\pgfpathlineto{\pgfqpoint{3.315768in}{1.344818in}}%
\pgfpathlineto{\pgfqpoint{3.301743in}{1.350068in}}%
\pgfpathlineto{\pgfqpoint{3.293207in}{1.344265in}}%
\pgfpathlineto{\pgfqpoint{3.284662in}{1.338658in}}%
\pgfpathlineto{\pgfqpoint{3.276105in}{1.333253in}}%
\pgfpathlineto{\pgfqpoint{3.267538in}{1.328058in}}%
\pgfpathclose%
\pgfusepath{fill}%
\end{pgfscope}%
\begin{pgfscope}%
\pgfpathrectangle{\pgfqpoint{1.150000in}{0.150000in}}{\pgfqpoint{5.700000in}{5.700000in}}%
\pgfusepath{clip}%
\pgfsetbuttcap%
\pgfsetroundjoin%
\definecolor{currentfill}{rgb}{0.241237,0.296485,0.539709}%
\pgfsetfillcolor{currentfill}%
\pgfsetfillopacity{0.700000}%
\pgfsetlinewidth{0.000000pt}%
\definecolor{currentstroke}{rgb}{0.000000,0.000000,0.000000}%
\pgfsetstrokecolor{currentstroke}%
\pgfsetdash{}{0pt}%
\pgfpathmoveto{\pgfqpoint{4.726252in}{1.891806in}}%
\pgfpathlineto{\pgfqpoint{4.740711in}{1.895137in}}%
\pgfpathlineto{\pgfqpoint{4.755182in}{1.898539in}}%
\pgfpathlineto{\pgfqpoint{4.769664in}{1.902012in}}%
\pgfpathlineto{\pgfqpoint{4.784159in}{1.905557in}}%
\pgfpathlineto{\pgfqpoint{4.792142in}{1.917018in}}%
\pgfpathlineto{\pgfqpoint{4.800117in}{1.928366in}}%
\pgfpathlineto{\pgfqpoint{4.808087in}{1.939599in}}%
\pgfpathlineto{\pgfqpoint{4.816049in}{1.950716in}}%
\pgfpathlineto{\pgfqpoint{4.801560in}{1.947068in}}%
\pgfpathlineto{\pgfqpoint{4.787083in}{1.943490in}}%
\pgfpathlineto{\pgfqpoint{4.772618in}{1.939984in}}%
\pgfpathlineto{\pgfqpoint{4.758165in}{1.936550in}}%
\pgfpathlineto{\pgfqpoint{4.750197in}{1.925529in}}%
\pgfpathlineto{\pgfqpoint{4.742222in}{1.914397in}}%
\pgfpathlineto{\pgfqpoint{4.734240in}{1.903156in}}%
\pgfpathlineto{\pgfqpoint{4.726252in}{1.891806in}}%
\pgfpathclose%
\pgfusepath{fill}%
\end{pgfscope}%
\begin{pgfscope}%
\pgfpathrectangle{\pgfqpoint{1.150000in}{0.150000in}}{\pgfqpoint{5.700000in}{5.700000in}}%
\pgfusepath{clip}%
\pgfsetbuttcap%
\pgfsetroundjoin%
\definecolor{currentfill}{rgb}{0.267004,0.004874,0.329415}%
\pgfsetfillcolor{currentfill}%
\pgfsetfillopacity{0.700000}%
\pgfsetlinewidth{0.000000pt}%
\definecolor{currentstroke}{rgb}{0.000000,0.000000,0.000000}%
\pgfsetstrokecolor{currentstroke}%
\pgfsetdash{}{0pt}%
\pgfpathmoveto{\pgfqpoint{3.414089in}{1.310216in}}%
\pgfpathlineto{\pgfqpoint{3.428155in}{1.305577in}}%
\pgfpathlineto{\pgfqpoint{3.442228in}{1.301014in}}%
\pgfpathlineto{\pgfqpoint{3.456305in}{1.296526in}}%
\pgfpathlineto{\pgfqpoint{3.470389in}{1.292112in}}%
\pgfpathlineto{\pgfqpoint{3.478850in}{1.299306in}}%
\pgfpathlineto{\pgfqpoint{3.487302in}{1.306668in}}%
\pgfpathlineto{\pgfqpoint{3.495745in}{1.314192in}}%
\pgfpathlineto{\pgfqpoint{3.504180in}{1.321873in}}%
\pgfpathlineto{\pgfqpoint{3.490116in}{1.325894in}}%
\pgfpathlineto{\pgfqpoint{3.476058in}{1.329990in}}%
\pgfpathlineto{\pgfqpoint{3.462005in}{1.334161in}}%
\pgfpathlineto{\pgfqpoint{3.447959in}{1.338408in}}%
\pgfpathlineto{\pgfqpoint{3.439505in}{1.331112in}}%
\pgfpathlineto{\pgfqpoint{3.431042in}{1.323977in}}%
\pgfpathlineto{\pgfqpoint{3.422570in}{1.317010in}}%
\pgfpathlineto{\pgfqpoint{3.414089in}{1.310216in}}%
\pgfpathclose%
\pgfusepath{fill}%
\end{pgfscope}%
\begin{pgfscope}%
\pgfpathrectangle{\pgfqpoint{1.150000in}{0.150000in}}{\pgfqpoint{5.700000in}{5.700000in}}%
\pgfusepath{clip}%
\pgfsetbuttcap%
\pgfsetroundjoin%
\definecolor{currentfill}{rgb}{0.229739,0.322361,0.545706}%
\pgfsetfillcolor{currentfill}%
\pgfsetfillopacity{0.700000}%
\pgfsetlinewidth{0.000000pt}%
\definecolor{currentstroke}{rgb}{0.000000,0.000000,0.000000}%
\pgfsetstrokecolor{currentstroke}%
\pgfsetdash{}{0pt}%
\pgfpathmoveto{\pgfqpoint{4.816049in}{1.950716in}}%
\pgfpathlineto{\pgfqpoint{4.830550in}{1.954436in}}%
\pgfpathlineto{\pgfqpoint{4.845064in}{1.958227in}}%
\pgfpathlineto{\pgfqpoint{4.859589in}{1.962089in}}%
\pgfpathlineto{\pgfqpoint{4.874127in}{1.966023in}}%
\pgfpathlineto{\pgfqpoint{4.882077in}{1.977113in}}%
\pgfpathlineto{\pgfqpoint{4.890020in}{1.988080in}}%
\pgfpathlineto{\pgfqpoint{4.897955in}{1.998923in}}%
\pgfpathlineto{\pgfqpoint{4.905883in}{2.009641in}}%
\pgfpathlineto{\pgfqpoint{4.891352in}{2.005624in}}%
\pgfpathlineto{\pgfqpoint{4.876832in}{2.001680in}}%
\pgfpathlineto{\pgfqpoint{4.862325in}{1.997806in}}%
\pgfpathlineto{\pgfqpoint{4.847830in}{1.994004in}}%
\pgfpathlineto{\pgfqpoint{4.839895in}{1.983361in}}%
\pgfpathlineto{\pgfqpoint{4.831953in}{1.972598in}}%
\pgfpathlineto{\pgfqpoint{4.824005in}{1.961716in}}%
\pgfpathlineto{\pgfqpoint{4.816049in}{1.950716in}}%
\pgfpathclose%
\pgfusepath{fill}%
\end{pgfscope}%
\begin{pgfscope}%
\pgfpathrectangle{\pgfqpoint{1.150000in}{0.150000in}}{\pgfqpoint{5.700000in}{5.700000in}}%
\pgfusepath{clip}%
\pgfsetbuttcap%
\pgfsetroundjoin%
\definecolor{currentfill}{rgb}{0.282656,0.100196,0.422160}%
\pgfsetfillcolor{currentfill}%
\pgfsetfillopacity{0.700000}%
\pgfsetlinewidth{0.000000pt}%
\definecolor{currentstroke}{rgb}{0.000000,0.000000,0.000000}%
\pgfsetstrokecolor{currentstroke}%
\pgfsetdash{}{0pt}%
\pgfpathmoveto{\pgfqpoint{4.065932in}{1.463529in}}%
\pgfpathlineto{\pgfqpoint{4.080147in}{1.463276in}}%
\pgfpathlineto{\pgfqpoint{4.094371in}{1.463095in}}%
\pgfpathlineto{\pgfqpoint{4.108604in}{1.462986in}}%
\pgfpathlineto{\pgfqpoint{4.122845in}{1.462947in}}%
\pgfpathlineto{\pgfqpoint{4.131045in}{1.474920in}}%
\pgfpathlineto{\pgfqpoint{4.139239in}{1.486899in}}%
\pgfpathlineto{\pgfqpoint{4.147428in}{1.498882in}}%
\pgfpathlineto{\pgfqpoint{4.155612in}{1.510865in}}%
\pgfpathlineto{\pgfqpoint{4.141378in}{1.510632in}}%
\pgfpathlineto{\pgfqpoint{4.127153in}{1.510471in}}%
\pgfpathlineto{\pgfqpoint{4.112937in}{1.510381in}}%
\pgfpathlineto{\pgfqpoint{4.098730in}{1.510362in}}%
\pgfpathlineto{\pgfqpoint{4.090539in}{1.498643in}}%
\pgfpathlineto{\pgfqpoint{4.082342in}{1.486928in}}%
\pgfpathlineto{\pgfqpoint{4.074140in}{1.475222in}}%
\pgfpathlineto{\pgfqpoint{4.065932in}{1.463529in}}%
\pgfpathclose%
\pgfusepath{fill}%
\end{pgfscope}%
\begin{pgfscope}%
\pgfpathrectangle{\pgfqpoint{1.150000in}{0.150000in}}{\pgfqpoint{5.700000in}{5.700000in}}%
\pgfusepath{clip}%
\pgfsetbuttcap%
\pgfsetroundjoin%
\definecolor{currentfill}{rgb}{0.283187,0.125848,0.444960}%
\pgfsetfillcolor{currentfill}%
\pgfsetfillopacity{0.700000}%
\pgfsetlinewidth{0.000000pt}%
\definecolor{currentstroke}{rgb}{0.000000,0.000000,0.000000}%
\pgfsetstrokecolor{currentstroke}%
\pgfsetdash{}{0pt}%
\pgfpathmoveto{\pgfqpoint{4.155612in}{1.510865in}}%
\pgfpathlineto{\pgfqpoint{4.169856in}{1.511169in}}%
\pgfpathlineto{\pgfqpoint{4.184109in}{1.511544in}}%
\pgfpathlineto{\pgfqpoint{4.198371in}{1.511990in}}%
\pgfpathlineto{\pgfqpoint{4.212643in}{1.512508in}}%
\pgfpathlineto{\pgfqpoint{4.220815in}{1.524744in}}%
\pgfpathlineto{\pgfqpoint{4.228982in}{1.536968in}}%
\pgfpathlineto{\pgfqpoint{4.237143in}{1.549174in}}%
\pgfpathlineto{\pgfqpoint{4.245300in}{1.561361in}}%
\pgfpathlineto{\pgfqpoint{4.231034in}{1.560593in}}%
\pgfpathlineto{\pgfqpoint{4.216779in}{1.559896in}}%
\pgfpathlineto{\pgfqpoint{4.202533in}{1.559270in}}%
\pgfpathlineto{\pgfqpoint{4.188296in}{1.558716in}}%
\pgfpathlineto{\pgfqpoint{4.180133in}{1.546772in}}%
\pgfpathlineto{\pgfqpoint{4.171964in}{1.534813in}}%
\pgfpathlineto{\pgfqpoint{4.163791in}{1.522843in}}%
\pgfpathlineto{\pgfqpoint{4.155612in}{1.510865in}}%
\pgfpathclose%
\pgfusepath{fill}%
\end{pgfscope}%
\begin{pgfscope}%
\pgfpathrectangle{\pgfqpoint{1.150000in}{0.150000in}}{\pgfqpoint{5.700000in}{5.700000in}}%
\pgfusepath{clip}%
\pgfsetbuttcap%
\pgfsetroundjoin%
\definecolor{currentfill}{rgb}{0.280894,0.078907,0.402329}%
\pgfsetfillcolor{currentfill}%
\pgfsetfillopacity{0.700000}%
\pgfsetlinewidth{0.000000pt}%
\definecolor{currentstroke}{rgb}{0.000000,0.000000,0.000000}%
\pgfsetstrokecolor{currentstroke}%
\pgfsetdash{}{0pt}%
\pgfpathmoveto{\pgfqpoint{3.976243in}{1.419849in}}%
\pgfpathlineto{\pgfqpoint{3.990432in}{1.419019in}}%
\pgfpathlineto{\pgfqpoint{4.004629in}{1.418261in}}%
\pgfpathlineto{\pgfqpoint{4.018835in}{1.417575in}}%
\pgfpathlineto{\pgfqpoint{4.033050in}{1.416960in}}%
\pgfpathlineto{\pgfqpoint{4.041279in}{1.428563in}}%
\pgfpathlineto{\pgfqpoint{4.049502in}{1.440195in}}%
\pgfpathlineto{\pgfqpoint{4.057720in}{1.451851in}}%
\pgfpathlineto{\pgfqpoint{4.065932in}{1.463529in}}%
\pgfpathlineto{\pgfqpoint{4.051727in}{1.463852in}}%
\pgfpathlineto{\pgfqpoint{4.037529in}{1.464248in}}%
\pgfpathlineto{\pgfqpoint{4.023341in}{1.464715in}}%
\pgfpathlineto{\pgfqpoint{4.009161in}{1.465254in}}%
\pgfpathlineto{\pgfqpoint{4.000940in}{1.453860in}}%
\pgfpathlineto{\pgfqpoint{3.992713in}{1.442492in}}%
\pgfpathlineto{\pgfqpoint{3.984481in}{1.431153in}}%
\pgfpathlineto{\pgfqpoint{3.976243in}{1.419849in}}%
\pgfpathclose%
\pgfusepath{fill}%
\end{pgfscope}%
\begin{pgfscope}%
\pgfpathrectangle{\pgfqpoint{1.150000in}{0.150000in}}{\pgfqpoint{5.700000in}{5.700000in}}%
\pgfusepath{clip}%
\pgfsetbuttcap%
\pgfsetroundjoin%
\definecolor{currentfill}{rgb}{0.268510,0.009605,0.335427}%
\pgfsetfillcolor{currentfill}%
\pgfsetfillopacity{0.700000}%
\pgfsetlinewidth{0.000000pt}%
\definecolor{currentstroke}{rgb}{0.000000,0.000000,0.000000}%
\pgfsetstrokecolor{currentstroke}%
\pgfsetdash{}{0pt}%
\pgfpathmoveto{\pgfqpoint{3.560496in}{1.306534in}}%
\pgfpathlineto{\pgfqpoint{3.574590in}{1.302886in}}%
\pgfpathlineto{\pgfqpoint{3.588691in}{1.299311in}}%
\pgfpathlineto{\pgfqpoint{3.602798in}{1.295809in}}%
\pgfpathlineto{\pgfqpoint{3.616911in}{1.292382in}}%
\pgfpathlineto{\pgfqpoint{3.625303in}{1.300971in}}%
\pgfpathlineto{\pgfqpoint{3.633687in}{1.309696in}}%
\pgfpathlineto{\pgfqpoint{3.642063in}{1.318549in}}%
\pgfpathlineto{\pgfqpoint{3.650432in}{1.327525in}}%
\pgfpathlineto{\pgfqpoint{3.636334in}{1.330581in}}%
\pgfpathlineto{\pgfqpoint{3.622244in}{1.333711in}}%
\pgfpathlineto{\pgfqpoint{3.608160in}{1.336914in}}%
\pgfpathlineto{\pgfqpoint{3.594082in}{1.340191in}}%
\pgfpathlineto{\pgfqpoint{3.585697in}{1.331579in}}%
\pgfpathlineto{\pgfqpoint{3.577305in}{1.323095in}}%
\pgfpathlineto{\pgfqpoint{3.568904in}{1.314745in}}%
\pgfpathlineto{\pgfqpoint{3.560496in}{1.306534in}}%
\pgfpathclose%
\pgfusepath{fill}%
\end{pgfscope}%
\begin{pgfscope}%
\pgfpathrectangle{\pgfqpoint{1.150000in}{0.150000in}}{\pgfqpoint{5.700000in}{5.700000in}}%
\pgfusepath{clip}%
\pgfsetbuttcap%
\pgfsetroundjoin%
\definecolor{currentfill}{rgb}{0.281887,0.150881,0.465405}%
\pgfsetfillcolor{currentfill}%
\pgfsetfillopacity{0.700000}%
\pgfsetlinewidth{0.000000pt}%
\definecolor{currentstroke}{rgb}{0.000000,0.000000,0.000000}%
\pgfsetstrokecolor{currentstroke}%
\pgfsetdash{}{0pt}%
\pgfpathmoveto{\pgfqpoint{4.245300in}{1.561361in}}%
\pgfpathlineto{\pgfqpoint{4.259575in}{1.562201in}}%
\pgfpathlineto{\pgfqpoint{4.273860in}{1.563111in}}%
\pgfpathlineto{\pgfqpoint{4.288154in}{1.564093in}}%
\pgfpathlineto{\pgfqpoint{4.302459in}{1.565146in}}%
\pgfpathlineto{\pgfqpoint{4.310604in}{1.577547in}}%
\pgfpathlineto{\pgfqpoint{4.318744in}{1.589915in}}%
\pgfpathlineto{\pgfqpoint{4.326879in}{1.602249in}}%
\pgfpathlineto{\pgfqpoint{4.335009in}{1.614543in}}%
\pgfpathlineto{\pgfqpoint{4.320710in}{1.613260in}}%
\pgfpathlineto{\pgfqpoint{4.306421in}{1.612048in}}%
\pgfpathlineto{\pgfqpoint{4.292143in}{1.610908in}}%
\pgfpathlineto{\pgfqpoint{4.277874in}{1.609838in}}%
\pgfpathlineto{\pgfqpoint{4.269738in}{1.597766in}}%
\pgfpathlineto{\pgfqpoint{4.261597in}{1.585660in}}%
\pgfpathlineto{\pgfqpoint{4.253451in}{1.573524in}}%
\pgfpathlineto{\pgfqpoint{4.245300in}{1.561361in}}%
\pgfpathclose%
\pgfusepath{fill}%
\end{pgfscope}%
\begin{pgfscope}%
\pgfpathrectangle{\pgfqpoint{1.150000in}{0.150000in}}{\pgfqpoint{5.700000in}{5.700000in}}%
\pgfusepath{clip}%
\pgfsetbuttcap%
\pgfsetroundjoin%
\definecolor{currentfill}{rgb}{0.218130,0.347432,0.550038}%
\pgfsetfillcolor{currentfill}%
\pgfsetfillopacity{0.700000}%
\pgfsetlinewidth{0.000000pt}%
\definecolor{currentstroke}{rgb}{0.000000,0.000000,0.000000}%
\pgfsetstrokecolor{currentstroke}%
\pgfsetdash{}{0pt}%
\pgfpathmoveto{\pgfqpoint{4.905883in}{2.009641in}}%
\pgfpathlineto{\pgfqpoint{4.920428in}{2.013728in}}%
\pgfpathlineto{\pgfqpoint{4.934985in}{2.017887in}}%
\pgfpathlineto{\pgfqpoint{4.949554in}{2.022117in}}%
\pgfpathlineto{\pgfqpoint{4.964136in}{2.026418in}}%
\pgfpathlineto{\pgfqpoint{4.972051in}{2.037079in}}%
\pgfpathlineto{\pgfqpoint{4.979958in}{2.047607in}}%
\pgfpathlineto{\pgfqpoint{4.987858in}{2.058003in}}%
\pgfpathlineto{\pgfqpoint{4.995749in}{2.068266in}}%
\pgfpathlineto{\pgfqpoint{4.981174in}{2.063904in}}%
\pgfpathlineto{\pgfqpoint{4.966611in}{2.059613in}}%
\pgfpathlineto{\pgfqpoint{4.952061in}{2.055394in}}%
\pgfpathlineto{\pgfqpoint{4.937523in}{2.051246in}}%
\pgfpathlineto{\pgfqpoint{4.929624in}{2.041035in}}%
\pgfpathlineto{\pgfqpoint{4.921718in}{2.030697in}}%
\pgfpathlineto{\pgfqpoint{4.913804in}{2.020232in}}%
\pgfpathlineto{\pgfqpoint{4.905883in}{2.009641in}}%
\pgfpathclose%
\pgfusepath{fill}%
\end{pgfscope}%
\begin{pgfscope}%
\pgfpathrectangle{\pgfqpoint{1.150000in}{0.150000in}}{\pgfqpoint{5.700000in}{5.700000in}}%
\pgfusepath{clip}%
\pgfsetbuttcap%
\pgfsetroundjoin%
\definecolor{currentfill}{rgb}{0.277941,0.056324,0.381191}%
\pgfsetfillcolor{currentfill}%
\pgfsetfillopacity{0.700000}%
\pgfsetlinewidth{0.000000pt}%
\definecolor{currentstroke}{rgb}{0.000000,0.000000,0.000000}%
\pgfsetstrokecolor{currentstroke}%
\pgfsetdash{}{0pt}%
\pgfpathmoveto{\pgfqpoint{3.886523in}{1.380343in}}%
\pgfpathlineto{\pgfqpoint{3.900689in}{1.378915in}}%
\pgfpathlineto{\pgfqpoint{3.914864in}{1.377558in}}%
\pgfpathlineto{\pgfqpoint{3.929046in}{1.376274in}}%
\pgfpathlineto{\pgfqpoint{3.943236in}{1.375061in}}%
\pgfpathlineto{\pgfqpoint{3.951497in}{1.386185in}}%
\pgfpathlineto{\pgfqpoint{3.959751in}{1.397360in}}%
\pgfpathlineto{\pgfqpoint{3.968000in}{1.408583in}}%
\pgfpathlineto{\pgfqpoint{3.976243in}{1.419849in}}%
\pgfpathlineto{\pgfqpoint{3.962063in}{1.420750in}}%
\pgfpathlineto{\pgfqpoint{3.947890in}{1.421724in}}%
\pgfpathlineto{\pgfqpoint{3.933726in}{1.422769in}}%
\pgfpathlineto{\pgfqpoint{3.919571in}{1.423886in}}%
\pgfpathlineto{\pgfqpoint{3.911317in}{1.412924in}}%
\pgfpathlineto{\pgfqpoint{3.903058in}{1.402009in}}%
\pgfpathlineto{\pgfqpoint{3.894794in}{1.391148in}}%
\pgfpathlineto{\pgfqpoint{3.886523in}{1.380343in}}%
\pgfpathclose%
\pgfusepath{fill}%
\end{pgfscope}%
\begin{pgfscope}%
\pgfpathrectangle{\pgfqpoint{1.150000in}{0.150000in}}{\pgfqpoint{5.700000in}{5.700000in}}%
\pgfusepath{clip}%
\pgfsetbuttcap%
\pgfsetroundjoin%
\definecolor{currentfill}{rgb}{0.278826,0.175490,0.483397}%
\pgfsetfillcolor{currentfill}%
\pgfsetfillopacity{0.700000}%
\pgfsetlinewidth{0.000000pt}%
\definecolor{currentstroke}{rgb}{0.000000,0.000000,0.000000}%
\pgfsetstrokecolor{currentstroke}%
\pgfsetdash{}{0pt}%
\pgfpathmoveto{\pgfqpoint{4.335009in}{1.614543in}}%
\pgfpathlineto{\pgfqpoint{4.349318in}{1.615898in}}%
\pgfpathlineto{\pgfqpoint{4.363637in}{1.617323in}}%
\pgfpathlineto{\pgfqpoint{4.377966in}{1.618819in}}%
\pgfpathlineto{\pgfqpoint{4.392306in}{1.620387in}}%
\pgfpathlineto{\pgfqpoint{4.400425in}{1.632857in}}%
\pgfpathlineto{\pgfqpoint{4.408539in}{1.645278in}}%
\pgfpathlineto{\pgfqpoint{4.416648in}{1.657645in}}%
\pgfpathlineto{\pgfqpoint{4.424751in}{1.669958in}}%
\pgfpathlineto{\pgfqpoint{4.410416in}{1.668180in}}%
\pgfpathlineto{\pgfqpoint{4.396092in}{1.666474in}}%
\pgfpathlineto{\pgfqpoint{4.381779in}{1.664840in}}%
\pgfpathlineto{\pgfqpoint{4.367475in}{1.663276in}}%
\pgfpathlineto{\pgfqpoint{4.359367in}{1.651166in}}%
\pgfpathlineto{\pgfqpoint{4.351253in}{1.639005in}}%
\pgfpathlineto{\pgfqpoint{4.343134in}{1.626796in}}%
\pgfpathlineto{\pgfqpoint{4.335009in}{1.614543in}}%
\pgfpathclose%
\pgfusepath{fill}%
\end{pgfscope}%
\begin{pgfscope}%
\pgfpathrectangle{\pgfqpoint{1.150000in}{0.150000in}}{\pgfqpoint{5.700000in}{5.700000in}}%
\pgfusepath{clip}%
\pgfsetbuttcap%
\pgfsetroundjoin%
\definecolor{currentfill}{rgb}{0.208623,0.367752,0.552675}%
\pgfsetfillcolor{currentfill}%
\pgfsetfillopacity{0.700000}%
\pgfsetlinewidth{0.000000pt}%
\definecolor{currentstroke}{rgb}{0.000000,0.000000,0.000000}%
\pgfsetstrokecolor{currentstroke}%
\pgfsetdash{}{0pt}%
\pgfpathmoveto{\pgfqpoint{4.995749in}{2.068266in}}%
\pgfpathlineto{\pgfqpoint{5.010338in}{2.072700in}}%
\pgfpathlineto{\pgfqpoint{5.024939in}{2.077205in}}%
\pgfpathlineto{\pgfqpoint{5.039553in}{2.081781in}}%
\pgfpathlineto{\pgfqpoint{5.054180in}{2.086429in}}%
\pgfpathlineto{\pgfqpoint{5.062057in}{2.096606in}}%
\pgfpathlineto{\pgfqpoint{5.069926in}{2.106643in}}%
\pgfpathlineto{\pgfqpoint{5.077787in}{2.116542in}}%
\pgfpathlineto{\pgfqpoint{5.085640in}{2.126301in}}%
\pgfpathlineto{\pgfqpoint{5.071020in}{2.121614in}}%
\pgfpathlineto{\pgfqpoint{5.056414in}{2.116999in}}%
\pgfpathlineto{\pgfqpoint{5.041820in}{2.112456in}}%
\pgfpathlineto{\pgfqpoint{5.027239in}{2.107983in}}%
\pgfpathlineto{\pgfqpoint{5.019378in}{2.098255in}}%
\pgfpathlineto{\pgfqpoint{5.011510in}{2.088392in}}%
\pgfpathlineto{\pgfqpoint{5.003634in}{2.078396in}}%
\pgfpathlineto{\pgfqpoint{4.995749in}{2.068266in}}%
\pgfpathclose%
\pgfusepath{fill}%
\end{pgfscope}%
\begin{pgfscope}%
\pgfpathrectangle{\pgfqpoint{1.150000in}{0.150000in}}{\pgfqpoint{5.700000in}{5.700000in}}%
\pgfusepath{clip}%
\pgfsetbuttcap%
\pgfsetroundjoin%
\definecolor{currentfill}{rgb}{0.273006,0.204520,0.501721}%
\pgfsetfillcolor{currentfill}%
\pgfsetfillopacity{0.700000}%
\pgfsetlinewidth{0.000000pt}%
\definecolor{currentstroke}{rgb}{0.000000,0.000000,0.000000}%
\pgfsetstrokecolor{currentstroke}%
\pgfsetdash{}{0pt}%
\pgfpathmoveto{\pgfqpoint{4.424751in}{1.669958in}}%
\pgfpathlineto{\pgfqpoint{4.439096in}{1.671806in}}%
\pgfpathlineto{\pgfqpoint{4.453451in}{1.673725in}}%
\pgfpathlineto{\pgfqpoint{4.467818in}{1.675715in}}%
\pgfpathlineto{\pgfqpoint{4.482195in}{1.677776in}}%
\pgfpathlineto{\pgfqpoint{4.490288in}{1.690227in}}%
\pgfpathlineto{\pgfqpoint{4.498375in}{1.702611in}}%
\pgfpathlineto{\pgfqpoint{4.506457in}{1.714926in}}%
\pgfpathlineto{\pgfqpoint{4.514533in}{1.727170in}}%
\pgfpathlineto{\pgfqpoint{4.500161in}{1.724919in}}%
\pgfpathlineto{\pgfqpoint{4.485800in}{1.722740in}}%
\pgfpathlineto{\pgfqpoint{4.471449in}{1.720632in}}%
\pgfpathlineto{\pgfqpoint{4.457109in}{1.718595in}}%
\pgfpathlineto{\pgfqpoint{4.449028in}{1.706533in}}%
\pgfpathlineto{\pgfqpoint{4.440941in}{1.694404in}}%
\pgfpathlineto{\pgfqpoint{4.432849in}{1.682211in}}%
\pgfpathlineto{\pgfqpoint{4.424751in}{1.669958in}}%
\pgfpathclose%
\pgfusepath{fill}%
\end{pgfscope}%
\begin{pgfscope}%
\pgfpathrectangle{\pgfqpoint{1.150000in}{0.150000in}}{\pgfqpoint{5.700000in}{5.700000in}}%
\pgfusepath{clip}%
\pgfsetbuttcap%
\pgfsetroundjoin%
\definecolor{currentfill}{rgb}{0.274952,0.037752,0.364543}%
\pgfsetfillcolor{currentfill}%
\pgfsetfillopacity{0.700000}%
\pgfsetlinewidth{0.000000pt}%
\definecolor{currentstroke}{rgb}{0.000000,0.000000,0.000000}%
\pgfsetstrokecolor{currentstroke}%
\pgfsetdash{}{0pt}%
\pgfpathmoveto{\pgfqpoint{3.796748in}{1.345550in}}%
\pgfpathlineto{\pgfqpoint{3.810895in}{1.343502in}}%
\pgfpathlineto{\pgfqpoint{3.825049in}{1.341526in}}%
\pgfpathlineto{\pgfqpoint{3.839211in}{1.339622in}}%
\pgfpathlineto{\pgfqpoint{3.853381in}{1.337790in}}%
\pgfpathlineto{\pgfqpoint{3.861676in}{1.348319in}}%
\pgfpathlineto{\pgfqpoint{3.869964in}{1.358924in}}%
\pgfpathlineto{\pgfqpoint{3.878247in}{1.369600in}}%
\pgfpathlineto{\pgfqpoint{3.886523in}{1.380343in}}%
\pgfpathlineto{\pgfqpoint{3.872365in}{1.381843in}}%
\pgfpathlineto{\pgfqpoint{3.858214in}{1.383416in}}%
\pgfpathlineto{\pgfqpoint{3.844072in}{1.385060in}}%
\pgfpathlineto{\pgfqpoint{3.829937in}{1.386778in}}%
\pgfpathlineto{\pgfqpoint{3.821649in}{1.376358in}}%
\pgfpathlineto{\pgfqpoint{3.813355in}{1.366011in}}%
\pgfpathlineto{\pgfqpoint{3.805054in}{1.355740in}}%
\pgfpathlineto{\pgfqpoint{3.796748in}{1.345550in}}%
\pgfpathclose%
\pgfusepath{fill}%
\end{pgfscope}%
\begin{pgfscope}%
\pgfpathrectangle{\pgfqpoint{1.150000in}{0.150000in}}{\pgfqpoint{5.700000in}{5.700000in}}%
\pgfusepath{clip}%
\pgfsetbuttcap%
\pgfsetroundjoin%
\definecolor{currentfill}{rgb}{0.197636,0.391528,0.554969}%
\pgfsetfillcolor{currentfill}%
\pgfsetfillopacity{0.700000}%
\pgfsetlinewidth{0.000000pt}%
\definecolor{currentstroke}{rgb}{0.000000,0.000000,0.000000}%
\pgfsetstrokecolor{currentstroke}%
\pgfsetdash{}{0pt}%
\pgfpathmoveto{\pgfqpoint{5.085640in}{2.126301in}}%
\pgfpathlineto{\pgfqpoint{5.100273in}{2.131059in}}%
\pgfpathlineto{\pgfqpoint{5.114919in}{2.135888in}}%
\pgfpathlineto{\pgfqpoint{5.129578in}{2.140789in}}%
\pgfpathlineto{\pgfqpoint{5.144250in}{2.145762in}}%
\pgfpathlineto{\pgfqpoint{5.152087in}{2.155407in}}%
\pgfpathlineto{\pgfqpoint{5.159915in}{2.164906in}}%
\pgfpathlineto{\pgfqpoint{5.167735in}{2.174262in}}%
\pgfpathlineto{\pgfqpoint{5.175546in}{2.183473in}}%
\pgfpathlineto{\pgfqpoint{5.160882in}{2.178484in}}%
\pgfpathlineto{\pgfqpoint{5.146231in}{2.173566in}}%
\pgfpathlineto{\pgfqpoint{5.131594in}{2.168720in}}%
\pgfpathlineto{\pgfqpoint{5.116969in}{2.163945in}}%
\pgfpathlineto{\pgfqpoint{5.109149in}{2.154743in}}%
\pgfpathlineto{\pgfqpoint{5.101321in}{2.145401in}}%
\pgfpathlineto{\pgfqpoint{5.093485in}{2.135921in}}%
\pgfpathlineto{\pgfqpoint{5.085640in}{2.126301in}}%
\pgfpathclose%
\pgfusepath{fill}%
\end{pgfscope}%
\begin{pgfscope}%
\pgfpathrectangle{\pgfqpoint{1.150000in}{0.150000in}}{\pgfqpoint{5.700000in}{5.700000in}}%
\pgfusepath{clip}%
\pgfsetbuttcap%
\pgfsetroundjoin%
\definecolor{currentfill}{rgb}{0.268510,0.009605,0.335427}%
\pgfsetfillcolor{currentfill}%
\pgfsetfillopacity{0.700000}%
\pgfsetlinewidth{0.000000pt}%
\definecolor{currentstroke}{rgb}{0.000000,0.000000,0.000000}%
\pgfsetstrokecolor{currentstroke}%
\pgfsetdash{}{0pt}%
\pgfpathmoveto{\pgfqpoint{3.323763in}{1.305868in}}%
\pgfpathlineto{\pgfqpoint{3.337831in}{1.300512in}}%
\pgfpathlineto{\pgfqpoint{3.351904in}{1.295233in}}%
\pgfpathlineto{\pgfqpoint{3.365983in}{1.290029in}}%
\pgfpathlineto{\pgfqpoint{3.380066in}{1.284902in}}%
\pgfpathlineto{\pgfqpoint{3.388587in}{1.290939in}}%
\pgfpathlineto{\pgfqpoint{3.397097in}{1.297175in}}%
\pgfpathlineto{\pgfqpoint{3.405598in}{1.303603in}}%
\pgfpathlineto{\pgfqpoint{3.414089in}{1.310216in}}%
\pgfpathlineto{\pgfqpoint{3.400027in}{1.314931in}}%
\pgfpathlineto{\pgfqpoint{3.385971in}{1.319721in}}%
\pgfpathlineto{\pgfqpoint{3.371920in}{1.324587in}}%
\pgfpathlineto{\pgfqpoint{3.357875in}{1.329530in}}%
\pgfpathlineto{\pgfqpoint{3.349362in}{1.323321in}}%
\pgfpathlineto{\pgfqpoint{3.340839in}{1.317304in}}%
\pgfpathlineto{\pgfqpoint{3.332306in}{1.311484in}}%
\pgfpathlineto{\pgfqpoint{3.323763in}{1.305868in}}%
\pgfpathclose%
\pgfusepath{fill}%
\end{pgfscope}%
\begin{pgfscope}%
\pgfpathrectangle{\pgfqpoint{1.150000in}{0.150000in}}{\pgfqpoint{5.700000in}{5.700000in}}%
\pgfusepath{clip}%
\pgfsetbuttcap%
\pgfsetroundjoin%
\definecolor{currentfill}{rgb}{0.265145,0.232956,0.516599}%
\pgfsetfillcolor{currentfill}%
\pgfsetfillopacity{0.700000}%
\pgfsetlinewidth{0.000000pt}%
\definecolor{currentstroke}{rgb}{0.000000,0.000000,0.000000}%
\pgfsetstrokecolor{currentstroke}%
\pgfsetdash{}{0pt}%
\pgfpathmoveto{\pgfqpoint{4.514533in}{1.727170in}}%
\pgfpathlineto{\pgfqpoint{4.528916in}{1.729491in}}%
\pgfpathlineto{\pgfqpoint{4.543310in}{1.731883in}}%
\pgfpathlineto{\pgfqpoint{4.557715in}{1.734347in}}%
\pgfpathlineto{\pgfqpoint{4.572131in}{1.736881in}}%
\pgfpathlineto{\pgfqpoint{4.580197in}{1.749227in}}%
\pgfpathlineto{\pgfqpoint{4.588258in}{1.761491in}}%
\pgfpathlineto{\pgfqpoint{4.596312in}{1.773672in}}%
\pgfpathlineto{\pgfqpoint{4.604361in}{1.785766in}}%
\pgfpathlineto{\pgfqpoint{4.589949in}{1.783063in}}%
\pgfpathlineto{\pgfqpoint{4.575549in}{1.780432in}}%
\pgfpathlineto{\pgfqpoint{4.561160in}{1.777872in}}%
\pgfpathlineto{\pgfqpoint{4.546782in}{1.775382in}}%
\pgfpathlineto{\pgfqpoint{4.538728in}{1.763448in}}%
\pgfpathlineto{\pgfqpoint{4.530669in}{1.751433in}}%
\pgfpathlineto{\pgfqpoint{4.522604in}{1.739339in}}%
\pgfpathlineto{\pgfqpoint{4.514533in}{1.727170in}}%
\pgfpathclose%
\pgfusepath{fill}%
\end{pgfscope}%
\begin{pgfscope}%
\pgfpathrectangle{\pgfqpoint{1.150000in}{0.150000in}}{\pgfqpoint{5.700000in}{5.700000in}}%
\pgfusepath{clip}%
\pgfsetbuttcap%
\pgfsetroundjoin%
\definecolor{currentfill}{rgb}{0.187231,0.414746,0.556547}%
\pgfsetfillcolor{currentfill}%
\pgfsetfillopacity{0.700000}%
\pgfsetlinewidth{0.000000pt}%
\definecolor{currentstroke}{rgb}{0.000000,0.000000,0.000000}%
\pgfsetstrokecolor{currentstroke}%
\pgfsetdash{}{0pt}%
\pgfpathmoveto{\pgfqpoint{5.175546in}{2.183473in}}%
\pgfpathlineto{\pgfqpoint{5.190224in}{2.188533in}}%
\pgfpathlineto{\pgfqpoint{5.204915in}{2.193665in}}%
\pgfpathlineto{\pgfqpoint{5.219619in}{2.198869in}}%
\pgfpathlineto{\pgfqpoint{5.234337in}{2.204145in}}%
\pgfpathlineto{\pgfqpoint{5.242131in}{2.213214in}}%
\pgfpathlineto{\pgfqpoint{5.249916in}{2.222135in}}%
\pgfpathlineto{\pgfqpoint{5.257691in}{2.230907in}}%
\pgfpathlineto{\pgfqpoint{5.265458in}{2.239531in}}%
\pgfpathlineto{\pgfqpoint{5.250749in}{2.234261in}}%
\pgfpathlineto{\pgfqpoint{5.236054in}{2.229063in}}%
\pgfpathlineto{\pgfqpoint{5.221372in}{2.223936in}}%
\pgfpathlineto{\pgfqpoint{5.206704in}{2.218881in}}%
\pgfpathlineto{\pgfqpoint{5.198928in}{2.210243in}}%
\pgfpathlineto{\pgfqpoint{5.191142in}{2.201463in}}%
\pgfpathlineto{\pgfqpoint{5.183349in}{2.192540in}}%
\pgfpathlineto{\pgfqpoint{5.175546in}{2.183473in}}%
\pgfpathclose%
\pgfusepath{fill}%
\end{pgfscope}%
\begin{pgfscope}%
\pgfpathrectangle{\pgfqpoint{1.150000in}{0.150000in}}{\pgfqpoint{5.700000in}{5.700000in}}%
\pgfusepath{clip}%
\pgfsetbuttcap%
\pgfsetroundjoin%
\definecolor{currentfill}{rgb}{0.267004,0.004874,0.329415}%
\pgfsetfillcolor{currentfill}%
\pgfsetfillopacity{0.700000}%
\pgfsetlinewidth{0.000000pt}%
\definecolor{currentstroke}{rgb}{0.000000,0.000000,0.000000}%
\pgfsetstrokecolor{currentstroke}%
\pgfsetdash{}{0pt}%
\pgfpathmoveto{\pgfqpoint{3.470389in}{1.292112in}}%
\pgfpathlineto{\pgfqpoint{3.484478in}{1.287774in}}%
\pgfpathlineto{\pgfqpoint{3.498573in}{1.283510in}}%
\pgfpathlineto{\pgfqpoint{3.512673in}{1.279321in}}%
\pgfpathlineto{\pgfqpoint{3.526780in}{1.275205in}}%
\pgfpathlineto{\pgfqpoint{3.535222in}{1.282799in}}%
\pgfpathlineto{\pgfqpoint{3.543655in}{1.290556in}}%
\pgfpathlineto{\pgfqpoint{3.552080in}{1.298469in}}%
\pgfpathlineto{\pgfqpoint{3.560496in}{1.306534in}}%
\pgfpathlineto{\pgfqpoint{3.546408in}{1.310257in}}%
\pgfpathlineto{\pgfqpoint{3.532326in}{1.314055in}}%
\pgfpathlineto{\pgfqpoint{3.518250in}{1.317926in}}%
\pgfpathlineto{\pgfqpoint{3.504180in}{1.321873in}}%
\pgfpathlineto{\pgfqpoint{3.495745in}{1.314192in}}%
\pgfpathlineto{\pgfqpoint{3.487302in}{1.306668in}}%
\pgfpathlineto{\pgfqpoint{3.478850in}{1.299306in}}%
\pgfpathlineto{\pgfqpoint{3.470389in}{1.292112in}}%
\pgfpathclose%
\pgfusepath{fill}%
\end{pgfscope}%
\begin{pgfscope}%
\pgfpathrectangle{\pgfqpoint{1.150000in}{0.150000in}}{\pgfqpoint{5.700000in}{5.700000in}}%
\pgfusepath{clip}%
\pgfsetbuttcap%
\pgfsetroundjoin%
\definecolor{currentfill}{rgb}{0.271305,0.019942,0.347269}%
\pgfsetfillcolor{currentfill}%
\pgfsetfillopacity{0.700000}%
\pgfsetlinewidth{0.000000pt}%
\definecolor{currentstroke}{rgb}{0.000000,0.000000,0.000000}%
\pgfsetstrokecolor{currentstroke}%
\pgfsetdash{}{0pt}%
\pgfpathmoveto{\pgfqpoint{3.706888in}{1.316034in}}%
\pgfpathlineto{\pgfqpoint{3.721020in}{1.313344in}}%
\pgfpathlineto{\pgfqpoint{3.735158in}{1.310727in}}%
\pgfpathlineto{\pgfqpoint{3.749304in}{1.308182in}}%
\pgfpathlineto{\pgfqpoint{3.763456in}{1.305709in}}%
\pgfpathlineto{\pgfqpoint{3.771789in}{1.315522in}}%
\pgfpathlineto{\pgfqpoint{3.780115in}{1.325436in}}%
\pgfpathlineto{\pgfqpoint{3.788435in}{1.335448in}}%
\pgfpathlineto{\pgfqpoint{3.796748in}{1.345550in}}%
\pgfpathlineto{\pgfqpoint{3.782608in}{1.347671in}}%
\pgfpathlineto{\pgfqpoint{3.768476in}{1.349864in}}%
\pgfpathlineto{\pgfqpoint{3.754351in}{1.352130in}}%
\pgfpathlineto{\pgfqpoint{3.740234in}{1.354469in}}%
\pgfpathlineto{\pgfqpoint{3.731907in}{1.344710in}}%
\pgfpathlineto{\pgfqpoint{3.723575in}{1.335048in}}%
\pgfpathlineto{\pgfqpoint{3.715235in}{1.325487in}}%
\pgfpathlineto{\pgfqpoint{3.706888in}{1.316034in}}%
\pgfpathclose%
\pgfusepath{fill}%
\end{pgfscope}%
\begin{pgfscope}%
\pgfpathrectangle{\pgfqpoint{1.150000in}{0.150000in}}{\pgfqpoint{5.700000in}{5.700000in}}%
\pgfusepath{clip}%
\pgfsetbuttcap%
\pgfsetroundjoin%
\definecolor{currentfill}{rgb}{0.171176,0.452530,0.557965}%
\pgfsetfillcolor{currentfill}%
\pgfsetfillopacity{0.700000}%
\pgfsetlinewidth{0.000000pt}%
\definecolor{currentstroke}{rgb}{0.000000,0.000000,0.000000}%
\pgfsetstrokecolor{currentstroke}%
\pgfsetdash{}{0pt}%
\pgfpathmoveto{\pgfqpoint{5.355364in}{2.294245in}}%
\pgfpathlineto{\pgfqpoint{5.370131in}{2.299844in}}%
\pgfpathlineto{\pgfqpoint{5.384912in}{2.305516in}}%
\pgfpathlineto{\pgfqpoint{5.399708in}{2.311259in}}%
\pgfpathlineto{\pgfqpoint{5.407409in}{2.319082in}}%
\pgfpathlineto{\pgfqpoint{5.415100in}{2.326752in}}%
\pgfpathlineto{\pgfqpoint{5.422782in}{2.334273in}}%
\pgfpathlineto{\pgfqpoint{5.430454in}{2.341643in}}%
\pgfpathlineto{\pgfqpoint{5.415670in}{2.335951in}}%
\pgfpathlineto{\pgfqpoint{5.400900in}{2.330330in}}%
\pgfpathlineto{\pgfqpoint{5.386145in}{2.324781in}}%
\pgfpathlineto{\pgfqpoint{5.378464in}{2.317366in}}%
\pgfpathlineto{\pgfqpoint{5.370773in}{2.309806in}}%
\pgfpathlineto{\pgfqpoint{5.363073in}{2.302099in}}%
\pgfpathlineto{\pgfqpoint{5.355364in}{2.294245in}}%
\pgfpathclose%
\pgfusepath{fill}%
\end{pgfscope}%
\begin{pgfscope}%
\pgfpathrectangle{\pgfqpoint{1.150000in}{0.150000in}}{\pgfqpoint{5.700000in}{5.700000in}}%
\pgfusepath{clip}%
\pgfsetbuttcap%
\pgfsetroundjoin%
\definecolor{currentfill}{rgb}{0.255645,0.260703,0.528312}%
\pgfsetfillcolor{currentfill}%
\pgfsetfillopacity{0.700000}%
\pgfsetlinewidth{0.000000pt}%
\definecolor{currentstroke}{rgb}{0.000000,0.000000,0.000000}%
\pgfsetstrokecolor{currentstroke}%
\pgfsetdash{}{0pt}%
\pgfpathmoveto{\pgfqpoint{4.604361in}{1.785766in}}%
\pgfpathlineto{\pgfqpoint{4.618783in}{1.788540in}}%
\pgfpathlineto{\pgfqpoint{4.633218in}{1.791385in}}%
\pgfpathlineto{\pgfqpoint{4.647663in}{1.794300in}}%
\pgfpathlineto{\pgfqpoint{4.662120in}{1.797287in}}%
\pgfpathlineto{\pgfqpoint{4.670158in}{1.809449in}}%
\pgfpathlineto{\pgfqpoint{4.678190in}{1.821515in}}%
\pgfpathlineto{\pgfqpoint{4.686216in}{1.833483in}}%
\pgfpathlineto{\pgfqpoint{4.694236in}{1.845353in}}%
\pgfpathlineto{\pgfqpoint{4.679784in}{1.842219in}}%
\pgfpathlineto{\pgfqpoint{4.665343in}{1.839156in}}%
\pgfpathlineto{\pgfqpoint{4.650914in}{1.836164in}}%
\pgfpathlineto{\pgfqpoint{4.636496in}{1.833244in}}%
\pgfpathlineto{\pgfqpoint{4.628471in}{1.821513in}}%
\pgfpathlineto{\pgfqpoint{4.620440in}{1.809689in}}%
\pgfpathlineto{\pgfqpoint{4.612403in}{1.797772in}}%
\pgfpathlineto{\pgfqpoint{4.604361in}{1.785766in}}%
\pgfpathclose%
\pgfusepath{fill}%
\end{pgfscope}%
\begin{pgfscope}%
\pgfpathrectangle{\pgfqpoint{1.150000in}{0.150000in}}{\pgfqpoint{5.700000in}{5.700000in}}%
\pgfusepath{clip}%
\pgfsetbuttcap%
\pgfsetroundjoin%
\definecolor{currentfill}{rgb}{0.179019,0.433756,0.557430}%
\pgfsetfillcolor{currentfill}%
\pgfsetfillopacity{0.700000}%
\pgfsetlinewidth{0.000000pt}%
\definecolor{currentstroke}{rgb}{0.000000,0.000000,0.000000}%
\pgfsetstrokecolor{currentstroke}%
\pgfsetdash{}{0pt}%
\pgfpathmoveto{\pgfqpoint{5.265458in}{2.239531in}}%
\pgfpathlineto{\pgfqpoint{5.280180in}{2.244872in}}%
\pgfpathlineto{\pgfqpoint{5.294916in}{2.250285in}}%
\pgfpathlineto{\pgfqpoint{5.309667in}{2.255769in}}%
\pgfpathlineto{\pgfqpoint{5.324431in}{2.261325in}}%
\pgfpathlineto{\pgfqpoint{5.332178in}{2.269782in}}%
\pgfpathlineto{\pgfqpoint{5.339916in}{2.278087in}}%
\pgfpathlineto{\pgfqpoint{5.347645in}{2.286241in}}%
\pgfpathlineto{\pgfqpoint{5.355364in}{2.294245in}}%
\pgfpathlineto{\pgfqpoint{5.340610in}{2.288717in}}%
\pgfpathlineto{\pgfqpoint{5.325871in}{2.283261in}}%
\pgfpathlineto{\pgfqpoint{5.311145in}{2.277876in}}%
\pgfpathlineto{\pgfqpoint{5.296433in}{2.272563in}}%
\pgfpathlineto{\pgfqpoint{5.288703in}{2.264522in}}%
\pgfpathlineto{\pgfqpoint{5.280964in}{2.256338in}}%
\pgfpathlineto{\pgfqpoint{5.273215in}{2.248007in}}%
\pgfpathlineto{\pgfqpoint{5.265458in}{2.239531in}}%
\pgfpathclose%
\pgfusepath{fill}%
\end{pgfscope}%
\begin{pgfscope}%
\pgfpathrectangle{\pgfqpoint{1.150000in}{0.150000in}}{\pgfqpoint{5.700000in}{5.700000in}}%
\pgfusepath{clip}%
\pgfsetbuttcap%
\pgfsetroundjoin%
\definecolor{currentfill}{rgb}{0.244972,0.287675,0.537260}%
\pgfsetfillcolor{currentfill}%
\pgfsetfillopacity{0.700000}%
\pgfsetlinewidth{0.000000pt}%
\definecolor{currentstroke}{rgb}{0.000000,0.000000,0.000000}%
\pgfsetstrokecolor{currentstroke}%
\pgfsetdash{}{0pt}%
\pgfpathmoveto{\pgfqpoint{4.694236in}{1.845353in}}%
\pgfpathlineto{\pgfqpoint{4.708700in}{1.848558in}}%
\pgfpathlineto{\pgfqpoint{4.723176in}{1.851834in}}%
\pgfpathlineto{\pgfqpoint{4.737664in}{1.855182in}}%
\pgfpathlineto{\pgfqpoint{4.752163in}{1.858600in}}%
\pgfpathlineto{\pgfqpoint{4.760172in}{1.870503in}}%
\pgfpathlineto{\pgfqpoint{4.768174in}{1.882297in}}%
\pgfpathlineto{\pgfqpoint{4.776170in}{1.893983in}}%
\pgfpathlineto{\pgfqpoint{4.784159in}{1.905557in}}%
\pgfpathlineto{\pgfqpoint{4.769664in}{1.902012in}}%
\pgfpathlineto{\pgfqpoint{4.755182in}{1.898539in}}%
\pgfpathlineto{\pgfqpoint{4.740711in}{1.895137in}}%
\pgfpathlineto{\pgfqpoint{4.726252in}{1.891806in}}%
\pgfpathlineto{\pgfqpoint{4.718258in}{1.880350in}}%
\pgfpathlineto{\pgfqpoint{4.710257in}{1.868788in}}%
\pgfpathlineto{\pgfqpoint{4.702249in}{1.857122in}}%
\pgfpathlineto{\pgfqpoint{4.694236in}{1.845353in}}%
\pgfpathclose%
\pgfusepath{fill}%
\end{pgfscope}%
\begin{pgfscope}%
\pgfpathrectangle{\pgfqpoint{1.150000in}{0.150000in}}{\pgfqpoint{5.700000in}{5.700000in}}%
\pgfusepath{clip}%
\pgfsetbuttcap%
\pgfsetroundjoin%
\definecolor{currentfill}{rgb}{0.269944,0.014625,0.341379}%
\pgfsetfillcolor{currentfill}%
\pgfsetfillopacity{0.700000}%
\pgfsetlinewidth{0.000000pt}%
\definecolor{currentstroke}{rgb}{0.000000,0.000000,0.000000}%
\pgfsetstrokecolor{currentstroke}%
\pgfsetdash{}{0pt}%
\pgfpathmoveto{\pgfqpoint{3.616911in}{1.292382in}}%
\pgfpathlineto{\pgfqpoint{3.631031in}{1.289027in}}%
\pgfpathlineto{\pgfqpoint{3.645158in}{1.285746in}}%
\pgfpathlineto{\pgfqpoint{3.659291in}{1.282538in}}%
\pgfpathlineto{\pgfqpoint{3.673430in}{1.279403in}}%
\pgfpathlineto{\pgfqpoint{3.681806in}{1.288373in}}%
\pgfpathlineto{\pgfqpoint{3.690174in}{1.297471in}}%
\pgfpathlineto{\pgfqpoint{3.698534in}{1.306694in}}%
\pgfpathlineto{\pgfqpoint{3.706888in}{1.316034in}}%
\pgfpathlineto{\pgfqpoint{3.692764in}{1.318797in}}%
\pgfpathlineto{\pgfqpoint{3.678646in}{1.321633in}}%
\pgfpathlineto{\pgfqpoint{3.664535in}{1.324542in}}%
\pgfpathlineto{\pgfqpoint{3.650432in}{1.327525in}}%
\pgfpathlineto{\pgfqpoint{3.642063in}{1.318549in}}%
\pgfpathlineto{\pgfqpoint{3.633687in}{1.309696in}}%
\pgfpathlineto{\pgfqpoint{3.625303in}{1.300971in}}%
\pgfpathlineto{\pgfqpoint{3.616911in}{1.292382in}}%
\pgfpathclose%
\pgfusepath{fill}%
\end{pgfscope}%
\begin{pgfscope}%
\pgfpathrectangle{\pgfqpoint{1.150000in}{0.150000in}}{\pgfqpoint{5.700000in}{5.700000in}}%
\pgfusepath{clip}%
\pgfsetbuttcap%
\pgfsetroundjoin%
\definecolor{currentfill}{rgb}{0.233603,0.313828,0.543914}%
\pgfsetfillcolor{currentfill}%
\pgfsetfillopacity{0.700000}%
\pgfsetlinewidth{0.000000pt}%
\definecolor{currentstroke}{rgb}{0.000000,0.000000,0.000000}%
\pgfsetstrokecolor{currentstroke}%
\pgfsetdash{}{0pt}%
\pgfpathmoveto{\pgfqpoint{4.784159in}{1.905557in}}%
\pgfpathlineto{\pgfqpoint{4.798666in}{1.909172in}}%
\pgfpathlineto{\pgfqpoint{4.813184in}{1.912859in}}%
\pgfpathlineto{\pgfqpoint{4.827715in}{1.916617in}}%
\pgfpathlineto{\pgfqpoint{4.842259in}{1.920446in}}%
\pgfpathlineto{\pgfqpoint{4.850236in}{1.932020in}}%
\pgfpathlineto{\pgfqpoint{4.858207in}{1.943475in}}%
\pgfpathlineto{\pgfqpoint{4.866170in}{1.954810in}}%
\pgfpathlineto{\pgfqpoint{4.874127in}{1.966023in}}%
\pgfpathlineto{\pgfqpoint{4.859589in}{1.962089in}}%
\pgfpathlineto{\pgfqpoint{4.845064in}{1.958227in}}%
\pgfpathlineto{\pgfqpoint{4.830550in}{1.954436in}}%
\pgfpathlineto{\pgfqpoint{4.816049in}{1.950716in}}%
\pgfpathlineto{\pgfqpoint{4.808087in}{1.939599in}}%
\pgfpathlineto{\pgfqpoint{4.800117in}{1.928366in}}%
\pgfpathlineto{\pgfqpoint{4.792142in}{1.917018in}}%
\pgfpathlineto{\pgfqpoint{4.784159in}{1.905557in}}%
\pgfpathclose%
\pgfusepath{fill}%
\end{pgfscope}%
\begin{pgfscope}%
\pgfpathrectangle{\pgfqpoint{1.150000in}{0.150000in}}{\pgfqpoint{5.700000in}{5.700000in}}%
\pgfusepath{clip}%
\pgfsetbuttcap%
\pgfsetroundjoin%
\definecolor{currentfill}{rgb}{0.283091,0.110553,0.431554}%
\pgfsetfillcolor{currentfill}%
\pgfsetfillopacity{0.700000}%
\pgfsetlinewidth{0.000000pt}%
\definecolor{currentstroke}{rgb}{0.000000,0.000000,0.000000}%
\pgfsetstrokecolor{currentstroke}%
\pgfsetdash{}{0pt}%
\pgfpathmoveto{\pgfqpoint{4.122845in}{1.462947in}}%
\pgfpathlineto{\pgfqpoint{4.137096in}{1.462980in}}%
\pgfpathlineto{\pgfqpoint{4.151356in}{1.463084in}}%
\pgfpathlineto{\pgfqpoint{4.165626in}{1.463259in}}%
\pgfpathlineto{\pgfqpoint{4.179904in}{1.463505in}}%
\pgfpathlineto{\pgfqpoint{4.188097in}{1.475757in}}%
\pgfpathlineto{\pgfqpoint{4.196284in}{1.488010in}}%
\pgfpathlineto{\pgfqpoint{4.204466in}{1.500262in}}%
\pgfpathlineto{\pgfqpoint{4.212643in}{1.512508in}}%
\pgfpathlineto{\pgfqpoint{4.198371in}{1.511990in}}%
\pgfpathlineto{\pgfqpoint{4.184109in}{1.511544in}}%
\pgfpathlineto{\pgfqpoint{4.169856in}{1.511169in}}%
\pgfpathlineto{\pgfqpoint{4.155612in}{1.510865in}}%
\pgfpathlineto{\pgfqpoint{4.147428in}{1.498882in}}%
\pgfpathlineto{\pgfqpoint{4.139239in}{1.486899in}}%
\pgfpathlineto{\pgfqpoint{4.131045in}{1.474920in}}%
\pgfpathlineto{\pgfqpoint{4.122845in}{1.462947in}}%
\pgfpathclose%
\pgfusepath{fill}%
\end{pgfscope}%
\begin{pgfscope}%
\pgfpathrectangle{\pgfqpoint{1.150000in}{0.150000in}}{\pgfqpoint{5.700000in}{5.700000in}}%
\pgfusepath{clip}%
\pgfsetbuttcap%
\pgfsetroundjoin%
\definecolor{currentfill}{rgb}{0.281924,0.089666,0.412415}%
\pgfsetfillcolor{currentfill}%
\pgfsetfillopacity{0.700000}%
\pgfsetlinewidth{0.000000pt}%
\definecolor{currentstroke}{rgb}{0.000000,0.000000,0.000000}%
\pgfsetstrokecolor{currentstroke}%
\pgfsetdash{}{0pt}%
\pgfpathmoveto{\pgfqpoint{4.033050in}{1.416960in}}%
\pgfpathlineto{\pgfqpoint{4.047273in}{1.416416in}}%
\pgfpathlineto{\pgfqpoint{4.061505in}{1.415944in}}%
\pgfpathlineto{\pgfqpoint{4.075746in}{1.415542in}}%
\pgfpathlineto{\pgfqpoint{4.089995in}{1.415212in}}%
\pgfpathlineto{\pgfqpoint{4.098216in}{1.427115in}}%
\pgfpathlineto{\pgfqpoint{4.106431in}{1.439041in}}%
\pgfpathlineto{\pgfqpoint{4.114641in}{1.450986in}}%
\pgfpathlineto{\pgfqpoint{4.122845in}{1.462947in}}%
\pgfpathlineto{\pgfqpoint{4.108604in}{1.462986in}}%
\pgfpathlineto{\pgfqpoint{4.094371in}{1.463095in}}%
\pgfpathlineto{\pgfqpoint{4.080147in}{1.463276in}}%
\pgfpathlineto{\pgfqpoint{4.065932in}{1.463529in}}%
\pgfpathlineto{\pgfqpoint{4.057720in}{1.451851in}}%
\pgfpathlineto{\pgfqpoint{4.049502in}{1.440195in}}%
\pgfpathlineto{\pgfqpoint{4.041279in}{1.428563in}}%
\pgfpathlineto{\pgfqpoint{4.033050in}{1.416960in}}%
\pgfpathclose%
\pgfusepath{fill}%
\end{pgfscope}%
\begin{pgfscope}%
\pgfpathrectangle{\pgfqpoint{1.150000in}{0.150000in}}{\pgfqpoint{5.700000in}{5.700000in}}%
\pgfusepath{clip}%
\pgfsetbuttcap%
\pgfsetroundjoin%
\definecolor{currentfill}{rgb}{0.282884,0.135920,0.453427}%
\pgfsetfillcolor{currentfill}%
\pgfsetfillopacity{0.700000}%
\pgfsetlinewidth{0.000000pt}%
\definecolor{currentstroke}{rgb}{0.000000,0.000000,0.000000}%
\pgfsetstrokecolor{currentstroke}%
\pgfsetdash{}{0pt}%
\pgfpathmoveto{\pgfqpoint{4.212643in}{1.512508in}}%
\pgfpathlineto{\pgfqpoint{4.226924in}{1.513096in}}%
\pgfpathlineto{\pgfqpoint{4.241215in}{1.513756in}}%
\pgfpathlineto{\pgfqpoint{4.255516in}{1.514487in}}%
\pgfpathlineto{\pgfqpoint{4.269827in}{1.515288in}}%
\pgfpathlineto{\pgfqpoint{4.277992in}{1.527784in}}%
\pgfpathlineto{\pgfqpoint{4.286153in}{1.540261in}}%
\pgfpathlineto{\pgfqpoint{4.294308in}{1.552716in}}%
\pgfpathlineto{\pgfqpoint{4.302459in}{1.565146in}}%
\pgfpathlineto{\pgfqpoint{4.288154in}{1.564093in}}%
\pgfpathlineto{\pgfqpoint{4.273860in}{1.563111in}}%
\pgfpathlineto{\pgfqpoint{4.259575in}{1.562201in}}%
\pgfpathlineto{\pgfqpoint{4.245300in}{1.561361in}}%
\pgfpathlineto{\pgfqpoint{4.237143in}{1.549174in}}%
\pgfpathlineto{\pgfqpoint{4.228982in}{1.536968in}}%
\pgfpathlineto{\pgfqpoint{4.220815in}{1.524744in}}%
\pgfpathlineto{\pgfqpoint{4.212643in}{1.512508in}}%
\pgfpathclose%
\pgfusepath{fill}%
\end{pgfscope}%
\begin{pgfscope}%
\pgfpathrectangle{\pgfqpoint{1.150000in}{0.150000in}}{\pgfqpoint{5.700000in}{5.700000in}}%
\pgfusepath{clip}%
\pgfsetbuttcap%
\pgfsetroundjoin%
\definecolor{currentfill}{rgb}{0.279566,0.067836,0.391917}%
\pgfsetfillcolor{currentfill}%
\pgfsetfillopacity{0.700000}%
\pgfsetlinewidth{0.000000pt}%
\definecolor{currentstroke}{rgb}{0.000000,0.000000,0.000000}%
\pgfsetstrokecolor{currentstroke}%
\pgfsetdash{}{0pt}%
\pgfpathmoveto{\pgfqpoint{3.943236in}{1.375061in}}%
\pgfpathlineto{\pgfqpoint{3.957435in}{1.373920in}}%
\pgfpathlineto{\pgfqpoint{3.971642in}{1.372850in}}%
\pgfpathlineto{\pgfqpoint{3.985857in}{1.371852in}}%
\pgfpathlineto{\pgfqpoint{4.000081in}{1.370925in}}%
\pgfpathlineto{\pgfqpoint{4.008331in}{1.382368in}}%
\pgfpathlineto{\pgfqpoint{4.016576in}{1.393858in}}%
\pgfpathlineto{\pgfqpoint{4.024816in}{1.405390in}}%
\pgfpathlineto{\pgfqpoint{4.033050in}{1.416960in}}%
\pgfpathlineto{\pgfqpoint{4.018835in}{1.417575in}}%
\pgfpathlineto{\pgfqpoint{4.004629in}{1.418261in}}%
\pgfpathlineto{\pgfqpoint{3.990432in}{1.419019in}}%
\pgfpathlineto{\pgfqpoint{3.976243in}{1.419849in}}%
\pgfpathlineto{\pgfqpoint{3.968000in}{1.408583in}}%
\pgfpathlineto{\pgfqpoint{3.959751in}{1.397360in}}%
\pgfpathlineto{\pgfqpoint{3.951497in}{1.386185in}}%
\pgfpathlineto{\pgfqpoint{3.943236in}{1.375061in}}%
\pgfpathclose%
\pgfusepath{fill}%
\end{pgfscope}%
\begin{pgfscope}%
\pgfpathrectangle{\pgfqpoint{1.150000in}{0.150000in}}{\pgfqpoint{5.700000in}{5.700000in}}%
\pgfusepath{clip}%
\pgfsetbuttcap%
\pgfsetroundjoin%
\definecolor{currentfill}{rgb}{0.268510,0.009605,0.335427}%
\pgfsetfillcolor{currentfill}%
\pgfsetfillopacity{0.700000}%
\pgfsetlinewidth{0.000000pt}%
\definecolor{currentstroke}{rgb}{0.000000,0.000000,0.000000}%
\pgfsetstrokecolor{currentstroke}%
\pgfsetdash{}{0pt}%
\pgfpathmoveto{\pgfqpoint{3.380066in}{1.284902in}}%
\pgfpathlineto{\pgfqpoint{3.394154in}{1.279850in}}%
\pgfpathlineto{\pgfqpoint{3.408248in}{1.274873in}}%
\pgfpathlineto{\pgfqpoint{3.422347in}{1.269972in}}%
\pgfpathlineto{\pgfqpoint{3.436451in}{1.265145in}}%
\pgfpathlineto{\pgfqpoint{3.444950in}{1.271604in}}%
\pgfpathlineto{\pgfqpoint{3.453439in}{1.278255in}}%
\pgfpathlineto{\pgfqpoint{3.461919in}{1.285093in}}%
\pgfpathlineto{\pgfqpoint{3.470389in}{1.292112in}}%
\pgfpathlineto{\pgfqpoint{3.456305in}{1.296526in}}%
\pgfpathlineto{\pgfqpoint{3.442228in}{1.301014in}}%
\pgfpathlineto{\pgfqpoint{3.428155in}{1.305577in}}%
\pgfpathlineto{\pgfqpoint{3.414089in}{1.310216in}}%
\pgfpathlineto{\pgfqpoint{3.405598in}{1.303603in}}%
\pgfpathlineto{\pgfqpoint{3.397097in}{1.297175in}}%
\pgfpathlineto{\pgfqpoint{3.388587in}{1.290939in}}%
\pgfpathlineto{\pgfqpoint{3.380066in}{1.284902in}}%
\pgfpathclose%
\pgfusepath{fill}%
\end{pgfscope}%
\begin{pgfscope}%
\pgfpathrectangle{\pgfqpoint{1.150000in}{0.150000in}}{\pgfqpoint{5.700000in}{5.700000in}}%
\pgfusepath{clip}%
\pgfsetbuttcap%
\pgfsetroundjoin%
\definecolor{currentfill}{rgb}{0.280255,0.165693,0.476498}%
\pgfsetfillcolor{currentfill}%
\pgfsetfillopacity{0.700000}%
\pgfsetlinewidth{0.000000pt}%
\definecolor{currentstroke}{rgb}{0.000000,0.000000,0.000000}%
\pgfsetstrokecolor{currentstroke}%
\pgfsetdash{}{0pt}%
\pgfpathmoveto{\pgfqpoint{4.302459in}{1.565146in}}%
\pgfpathlineto{\pgfqpoint{4.316773in}{1.566269in}}%
\pgfpathlineto{\pgfqpoint{4.331098in}{1.567464in}}%
\pgfpathlineto{\pgfqpoint{4.345433in}{1.568730in}}%
\pgfpathlineto{\pgfqpoint{4.359778in}{1.570066in}}%
\pgfpathlineto{\pgfqpoint{4.367918in}{1.582706in}}%
\pgfpathlineto{\pgfqpoint{4.376052in}{1.595308in}}%
\pgfpathlineto{\pgfqpoint{4.384182in}{1.607869in}}%
\pgfpathlineto{\pgfqpoint{4.392306in}{1.620387in}}%
\pgfpathlineto{\pgfqpoint{4.377966in}{1.618819in}}%
\pgfpathlineto{\pgfqpoint{4.363637in}{1.617323in}}%
\pgfpathlineto{\pgfqpoint{4.349318in}{1.615898in}}%
\pgfpathlineto{\pgfqpoint{4.335009in}{1.614543in}}%
\pgfpathlineto{\pgfqpoint{4.326879in}{1.602249in}}%
\pgfpathlineto{\pgfqpoint{4.318744in}{1.589915in}}%
\pgfpathlineto{\pgfqpoint{4.310604in}{1.577547in}}%
\pgfpathlineto{\pgfqpoint{4.302459in}{1.565146in}}%
\pgfpathclose%
\pgfusepath{fill}%
\end{pgfscope}%
\begin{pgfscope}%
\pgfpathrectangle{\pgfqpoint{1.150000in}{0.150000in}}{\pgfqpoint{5.700000in}{5.700000in}}%
\pgfusepath{clip}%
\pgfsetbuttcap%
\pgfsetroundjoin%
\definecolor{currentfill}{rgb}{0.221989,0.339161,0.548752}%
\pgfsetfillcolor{currentfill}%
\pgfsetfillopacity{0.700000}%
\pgfsetlinewidth{0.000000pt}%
\definecolor{currentstroke}{rgb}{0.000000,0.000000,0.000000}%
\pgfsetstrokecolor{currentstroke}%
\pgfsetdash{}{0pt}%
\pgfpathmoveto{\pgfqpoint{4.874127in}{1.966023in}}%
\pgfpathlineto{\pgfqpoint{4.888678in}{1.970028in}}%
\pgfpathlineto{\pgfqpoint{4.903240in}{1.974104in}}%
\pgfpathlineto{\pgfqpoint{4.917816in}{1.978251in}}%
\pgfpathlineto{\pgfqpoint{4.932404in}{1.982470in}}%
\pgfpathlineto{\pgfqpoint{4.940348in}{1.993652in}}%
\pgfpathlineto{\pgfqpoint{4.948285in}{2.004704in}}%
\pgfpathlineto{\pgfqpoint{4.956214in}{2.015626in}}%
\pgfpathlineto{\pgfqpoint{4.964136in}{2.026418in}}%
\pgfpathlineto{\pgfqpoint{4.949554in}{2.022117in}}%
\pgfpathlineto{\pgfqpoint{4.934985in}{2.017887in}}%
\pgfpathlineto{\pgfqpoint{4.920428in}{2.013728in}}%
\pgfpathlineto{\pgfqpoint{4.905883in}{2.009641in}}%
\pgfpathlineto{\pgfqpoint{4.897955in}{1.998923in}}%
\pgfpathlineto{\pgfqpoint{4.890020in}{1.988080in}}%
\pgfpathlineto{\pgfqpoint{4.882077in}{1.977113in}}%
\pgfpathlineto{\pgfqpoint{4.874127in}{1.966023in}}%
\pgfpathclose%
\pgfusepath{fill}%
\end{pgfscope}%
\begin{pgfscope}%
\pgfpathrectangle{\pgfqpoint{1.150000in}{0.150000in}}{\pgfqpoint{5.700000in}{5.700000in}}%
\pgfusepath{clip}%
\pgfsetbuttcap%
\pgfsetroundjoin%
\definecolor{currentfill}{rgb}{0.276022,0.044167,0.370164}%
\pgfsetfillcolor{currentfill}%
\pgfsetfillopacity{0.700000}%
\pgfsetlinewidth{0.000000pt}%
\definecolor{currentstroke}{rgb}{0.000000,0.000000,0.000000}%
\pgfsetstrokecolor{currentstroke}%
\pgfsetdash{}{0pt}%
\pgfpathmoveto{\pgfqpoint{3.853381in}{1.337790in}}%
\pgfpathlineto{\pgfqpoint{3.867559in}{1.336031in}}%
\pgfpathlineto{\pgfqpoint{3.881744in}{1.334342in}}%
\pgfpathlineto{\pgfqpoint{3.895937in}{1.332726in}}%
\pgfpathlineto{\pgfqpoint{3.910138in}{1.331181in}}%
\pgfpathlineto{\pgfqpoint{3.918421in}{1.342049in}}%
\pgfpathlineto{\pgfqpoint{3.926699in}{1.352989in}}%
\pgfpathlineto{\pgfqpoint{3.934970in}{1.363994in}}%
\pgfpathlineto{\pgfqpoint{3.943236in}{1.375061in}}%
\pgfpathlineto{\pgfqpoint{3.929046in}{1.376274in}}%
\pgfpathlineto{\pgfqpoint{3.914864in}{1.377558in}}%
\pgfpathlineto{\pgfqpoint{3.900689in}{1.378915in}}%
\pgfpathlineto{\pgfqpoint{3.886523in}{1.380343in}}%
\pgfpathlineto{\pgfqpoint{3.878247in}{1.369600in}}%
\pgfpathlineto{\pgfqpoint{3.869964in}{1.358924in}}%
\pgfpathlineto{\pgfqpoint{3.861676in}{1.348319in}}%
\pgfpathlineto{\pgfqpoint{3.853381in}{1.337790in}}%
\pgfpathclose%
\pgfusepath{fill}%
\end{pgfscope}%
\begin{pgfscope}%
\pgfpathrectangle{\pgfqpoint{1.150000in}{0.150000in}}{\pgfqpoint{5.700000in}{5.700000in}}%
\pgfusepath{clip}%
\pgfsetbuttcap%
\pgfsetroundjoin%
\definecolor{currentfill}{rgb}{0.276194,0.190074,0.493001}%
\pgfsetfillcolor{currentfill}%
\pgfsetfillopacity{0.700000}%
\pgfsetlinewidth{0.000000pt}%
\definecolor{currentstroke}{rgb}{0.000000,0.000000,0.000000}%
\pgfsetstrokecolor{currentstroke}%
\pgfsetdash{}{0pt}%
\pgfpathmoveto{\pgfqpoint{4.392306in}{1.620387in}}%
\pgfpathlineto{\pgfqpoint{4.406656in}{1.622025in}}%
\pgfpathlineto{\pgfqpoint{4.421017in}{1.623734in}}%
\pgfpathlineto{\pgfqpoint{4.435388in}{1.625514in}}%
\pgfpathlineto{\pgfqpoint{4.449770in}{1.627365in}}%
\pgfpathlineto{\pgfqpoint{4.457884in}{1.640053in}}%
\pgfpathlineto{\pgfqpoint{4.465993in}{1.652687in}}%
\pgfpathlineto{\pgfqpoint{4.474096in}{1.665262in}}%
\pgfpathlineto{\pgfqpoint{4.482195in}{1.677776in}}%
\pgfpathlineto{\pgfqpoint{4.467818in}{1.675715in}}%
\pgfpathlineto{\pgfqpoint{4.453451in}{1.673725in}}%
\pgfpathlineto{\pgfqpoint{4.439096in}{1.671806in}}%
\pgfpathlineto{\pgfqpoint{4.424751in}{1.669958in}}%
\pgfpathlineto{\pgfqpoint{4.416648in}{1.657645in}}%
\pgfpathlineto{\pgfqpoint{4.408539in}{1.645278in}}%
\pgfpathlineto{\pgfqpoint{4.400425in}{1.632857in}}%
\pgfpathlineto{\pgfqpoint{4.392306in}{1.620387in}}%
\pgfpathclose%
\pgfusepath{fill}%
\end{pgfscope}%
\begin{pgfscope}%
\pgfpathrectangle{\pgfqpoint{1.150000in}{0.150000in}}{\pgfqpoint{5.700000in}{5.700000in}}%
\pgfusepath{clip}%
\pgfsetbuttcap%
\pgfsetroundjoin%
\definecolor{currentfill}{rgb}{0.268510,0.009605,0.335427}%
\pgfsetfillcolor{currentfill}%
\pgfsetfillopacity{0.700000}%
\pgfsetlinewidth{0.000000pt}%
\definecolor{currentstroke}{rgb}{0.000000,0.000000,0.000000}%
\pgfsetstrokecolor{currentstroke}%
\pgfsetdash{}{0pt}%
\pgfpathmoveto{\pgfqpoint{3.526780in}{1.275205in}}%
\pgfpathlineto{\pgfqpoint{3.540892in}{1.271164in}}%
\pgfpathlineto{\pgfqpoint{3.555011in}{1.267196in}}%
\pgfpathlineto{\pgfqpoint{3.569136in}{1.263303in}}%
\pgfpathlineto{\pgfqpoint{3.583266in}{1.259482in}}%
\pgfpathlineto{\pgfqpoint{3.591690in}{1.267476in}}%
\pgfpathlineto{\pgfqpoint{3.600105in}{1.275628in}}%
\pgfpathlineto{\pgfqpoint{3.608512in}{1.283932in}}%
\pgfpathlineto{\pgfqpoint{3.616911in}{1.292382in}}%
\pgfpathlineto{\pgfqpoint{3.602798in}{1.295809in}}%
\pgfpathlineto{\pgfqpoint{3.588691in}{1.299311in}}%
\pgfpathlineto{\pgfqpoint{3.574590in}{1.302886in}}%
\pgfpathlineto{\pgfqpoint{3.560496in}{1.306534in}}%
\pgfpathlineto{\pgfqpoint{3.552080in}{1.298469in}}%
\pgfpathlineto{\pgfqpoint{3.543655in}{1.290556in}}%
\pgfpathlineto{\pgfqpoint{3.535222in}{1.282799in}}%
\pgfpathlineto{\pgfqpoint{3.526780in}{1.275205in}}%
\pgfpathclose%
\pgfusepath{fill}%
\end{pgfscope}%
\begin{pgfscope}%
\pgfpathrectangle{\pgfqpoint{1.150000in}{0.150000in}}{\pgfqpoint{5.700000in}{5.700000in}}%
\pgfusepath{clip}%
\pgfsetbuttcap%
\pgfsetroundjoin%
\definecolor{currentfill}{rgb}{0.210503,0.363727,0.552206}%
\pgfsetfillcolor{currentfill}%
\pgfsetfillopacity{0.700000}%
\pgfsetlinewidth{0.000000pt}%
\definecolor{currentstroke}{rgb}{0.000000,0.000000,0.000000}%
\pgfsetstrokecolor{currentstroke}%
\pgfsetdash{}{0pt}%
\pgfpathmoveto{\pgfqpoint{4.964136in}{2.026418in}}%
\pgfpathlineto{\pgfqpoint{4.978731in}{2.030791in}}%
\pgfpathlineto{\pgfqpoint{4.993339in}{2.035235in}}%
\pgfpathlineto{\pgfqpoint{5.007960in}{2.039751in}}%
\pgfpathlineto{\pgfqpoint{5.022593in}{2.044338in}}%
\pgfpathlineto{\pgfqpoint{5.030502in}{2.055067in}}%
\pgfpathlineto{\pgfqpoint{5.038402in}{2.065659in}}%
\pgfpathlineto{\pgfqpoint{5.046295in}{2.076113in}}%
\pgfpathlineto{\pgfqpoint{5.054180in}{2.086429in}}%
\pgfpathlineto{\pgfqpoint{5.039553in}{2.081781in}}%
\pgfpathlineto{\pgfqpoint{5.024939in}{2.077205in}}%
\pgfpathlineto{\pgfqpoint{5.010338in}{2.072700in}}%
\pgfpathlineto{\pgfqpoint{4.995749in}{2.068266in}}%
\pgfpathlineto{\pgfqpoint{4.987858in}{2.058003in}}%
\pgfpathlineto{\pgfqpoint{4.979958in}{2.047607in}}%
\pgfpathlineto{\pgfqpoint{4.972051in}{2.037079in}}%
\pgfpathlineto{\pgfqpoint{4.964136in}{2.026418in}}%
\pgfpathclose%
\pgfusepath{fill}%
\end{pgfscope}%
\begin{pgfscope}%
\pgfpathrectangle{\pgfqpoint{1.150000in}{0.150000in}}{\pgfqpoint{5.700000in}{5.700000in}}%
\pgfusepath{clip}%
\pgfsetbuttcap%
\pgfsetroundjoin%
\definecolor{currentfill}{rgb}{0.273809,0.031497,0.358853}%
\pgfsetfillcolor{currentfill}%
\pgfsetfillopacity{0.700000}%
\pgfsetlinewidth{0.000000pt}%
\definecolor{currentstroke}{rgb}{0.000000,0.000000,0.000000}%
\pgfsetstrokecolor{currentstroke}%
\pgfsetdash{}{0pt}%
\pgfpathmoveto{\pgfqpoint{3.763456in}{1.305709in}}%
\pgfpathlineto{\pgfqpoint{3.777616in}{1.303309in}}%
\pgfpathlineto{\pgfqpoint{3.791784in}{1.300981in}}%
\pgfpathlineto{\pgfqpoint{3.805958in}{1.298725in}}%
\pgfpathlineto{\pgfqpoint{3.820140in}{1.296541in}}%
\pgfpathlineto{\pgfqpoint{3.828460in}{1.306713in}}%
\pgfpathlineto{\pgfqpoint{3.836773in}{1.316982in}}%
\pgfpathlineto{\pgfqpoint{3.845080in}{1.327343in}}%
\pgfpathlineto{\pgfqpoint{3.853381in}{1.337790in}}%
\pgfpathlineto{\pgfqpoint{3.839211in}{1.339622in}}%
\pgfpathlineto{\pgfqpoint{3.825049in}{1.341526in}}%
\pgfpathlineto{\pgfqpoint{3.810895in}{1.343502in}}%
\pgfpathlineto{\pgfqpoint{3.796748in}{1.345550in}}%
\pgfpathlineto{\pgfqpoint{3.788435in}{1.335448in}}%
\pgfpathlineto{\pgfqpoint{3.780115in}{1.325436in}}%
\pgfpathlineto{\pgfqpoint{3.771789in}{1.315522in}}%
\pgfpathlineto{\pgfqpoint{3.763456in}{1.305709in}}%
\pgfpathclose%
\pgfusepath{fill}%
\end{pgfscope}%
\begin{pgfscope}%
\pgfpathrectangle{\pgfqpoint{1.150000in}{0.150000in}}{\pgfqpoint{5.700000in}{5.700000in}}%
\pgfusepath{clip}%
\pgfsetbuttcap%
\pgfsetroundjoin%
\definecolor{currentfill}{rgb}{0.269308,0.218818,0.509577}%
\pgfsetfillcolor{currentfill}%
\pgfsetfillopacity{0.700000}%
\pgfsetlinewidth{0.000000pt}%
\definecolor{currentstroke}{rgb}{0.000000,0.000000,0.000000}%
\pgfsetstrokecolor{currentstroke}%
\pgfsetdash{}{0pt}%
\pgfpathmoveto{\pgfqpoint{4.482195in}{1.677776in}}%
\pgfpathlineto{\pgfqpoint{4.496583in}{1.679908in}}%
\pgfpathlineto{\pgfqpoint{4.510981in}{1.682112in}}%
\pgfpathlineto{\pgfqpoint{4.525391in}{1.684385in}}%
\pgfpathlineto{\pgfqpoint{4.539811in}{1.686730in}}%
\pgfpathlineto{\pgfqpoint{4.547900in}{1.699378in}}%
\pgfpathlineto{\pgfqpoint{4.555983in}{1.711954in}}%
\pgfpathlineto{\pgfqpoint{4.564060in}{1.724456in}}%
\pgfpathlineto{\pgfqpoint{4.572131in}{1.736881in}}%
\pgfpathlineto{\pgfqpoint{4.557715in}{1.734347in}}%
\pgfpathlineto{\pgfqpoint{4.543310in}{1.731883in}}%
\pgfpathlineto{\pgfqpoint{4.528916in}{1.729491in}}%
\pgfpathlineto{\pgfqpoint{4.514533in}{1.727170in}}%
\pgfpathlineto{\pgfqpoint{4.506457in}{1.714926in}}%
\pgfpathlineto{\pgfqpoint{4.498375in}{1.702611in}}%
\pgfpathlineto{\pgfqpoint{4.490288in}{1.690227in}}%
\pgfpathlineto{\pgfqpoint{4.482195in}{1.677776in}}%
\pgfpathclose%
\pgfusepath{fill}%
\end{pgfscope}%
\begin{pgfscope}%
\pgfpathrectangle{\pgfqpoint{1.150000in}{0.150000in}}{\pgfqpoint{5.700000in}{5.700000in}}%
\pgfusepath{clip}%
\pgfsetbuttcap%
\pgfsetroundjoin%
\definecolor{currentfill}{rgb}{0.199430,0.387607,0.554642}%
\pgfsetfillcolor{currentfill}%
\pgfsetfillopacity{0.700000}%
\pgfsetlinewidth{0.000000pt}%
\definecolor{currentstroke}{rgb}{0.000000,0.000000,0.000000}%
\pgfsetstrokecolor{currentstroke}%
\pgfsetdash{}{0pt}%
\pgfpathmoveto{\pgfqpoint{5.054180in}{2.086429in}}%
\pgfpathlineto{\pgfqpoint{5.068820in}{2.091148in}}%
\pgfpathlineto{\pgfqpoint{5.083474in}{2.095939in}}%
\pgfpathlineto{\pgfqpoint{5.098140in}{2.100801in}}%
\pgfpathlineto{\pgfqpoint{5.112820in}{2.105735in}}%
\pgfpathlineto{\pgfqpoint{5.120690in}{2.115959in}}%
\pgfpathlineto{\pgfqpoint{5.128552in}{2.126038in}}%
\pgfpathlineto{\pgfqpoint{5.136405in}{2.135972in}}%
\pgfpathlineto{\pgfqpoint{5.144250in}{2.145762in}}%
\pgfpathlineto{\pgfqpoint{5.129578in}{2.140789in}}%
\pgfpathlineto{\pgfqpoint{5.114919in}{2.135888in}}%
\pgfpathlineto{\pgfqpoint{5.100273in}{2.131059in}}%
\pgfpathlineto{\pgfqpoint{5.085640in}{2.126301in}}%
\pgfpathlineto{\pgfqpoint{5.077787in}{2.116542in}}%
\pgfpathlineto{\pgfqpoint{5.069926in}{2.106643in}}%
\pgfpathlineto{\pgfqpoint{5.062057in}{2.096606in}}%
\pgfpathlineto{\pgfqpoint{5.054180in}{2.086429in}}%
\pgfpathclose%
\pgfusepath{fill}%
\end{pgfscope}%
\begin{pgfscope}%
\pgfpathrectangle{\pgfqpoint{1.150000in}{0.150000in}}{\pgfqpoint{5.700000in}{5.700000in}}%
\pgfusepath{clip}%
\pgfsetbuttcap%
\pgfsetroundjoin%
\definecolor{currentfill}{rgb}{0.260571,0.246922,0.522828}%
\pgfsetfillcolor{currentfill}%
\pgfsetfillopacity{0.700000}%
\pgfsetlinewidth{0.000000pt}%
\definecolor{currentstroke}{rgb}{0.000000,0.000000,0.000000}%
\pgfsetstrokecolor{currentstroke}%
\pgfsetdash{}{0pt}%
\pgfpathmoveto{\pgfqpoint{4.572131in}{1.736881in}}%
\pgfpathlineto{\pgfqpoint{4.586559in}{1.739487in}}%
\pgfpathlineto{\pgfqpoint{4.600998in}{1.742163in}}%
\pgfpathlineto{\pgfqpoint{4.615448in}{1.744910in}}%
\pgfpathlineto{\pgfqpoint{4.629909in}{1.747729in}}%
\pgfpathlineto{\pgfqpoint{4.637971in}{1.760251in}}%
\pgfpathlineto{\pgfqpoint{4.646026in}{1.772686in}}%
\pgfpathlineto{\pgfqpoint{4.654076in}{1.785033in}}%
\pgfpathlineto{\pgfqpoint{4.662120in}{1.797287in}}%
\pgfpathlineto{\pgfqpoint{4.647663in}{1.794300in}}%
\pgfpathlineto{\pgfqpoint{4.633218in}{1.791385in}}%
\pgfpathlineto{\pgfqpoint{4.618783in}{1.788540in}}%
\pgfpathlineto{\pgfqpoint{4.604361in}{1.785766in}}%
\pgfpathlineto{\pgfqpoint{4.596312in}{1.773672in}}%
\pgfpathlineto{\pgfqpoint{4.588258in}{1.761491in}}%
\pgfpathlineto{\pgfqpoint{4.580197in}{1.749227in}}%
\pgfpathlineto{\pgfqpoint{4.572131in}{1.736881in}}%
\pgfpathclose%
\pgfusepath{fill}%
\end{pgfscope}%
\begin{pgfscope}%
\pgfpathrectangle{\pgfqpoint{1.150000in}{0.150000in}}{\pgfqpoint{5.700000in}{5.700000in}}%
\pgfusepath{clip}%
\pgfsetbuttcap%
\pgfsetroundjoin%
\definecolor{currentfill}{rgb}{0.188923,0.410910,0.556326}%
\pgfsetfillcolor{currentfill}%
\pgfsetfillopacity{0.700000}%
\pgfsetlinewidth{0.000000pt}%
\definecolor{currentstroke}{rgb}{0.000000,0.000000,0.000000}%
\pgfsetstrokecolor{currentstroke}%
\pgfsetdash{}{0pt}%
\pgfpathmoveto{\pgfqpoint{5.144250in}{2.145762in}}%
\pgfpathlineto{\pgfqpoint{5.158936in}{2.150806in}}%
\pgfpathlineto{\pgfqpoint{5.173635in}{2.155922in}}%
\pgfpathlineto{\pgfqpoint{5.188348in}{2.161109in}}%
\pgfpathlineto{\pgfqpoint{5.203075in}{2.166368in}}%
\pgfpathlineto{\pgfqpoint{5.210904in}{2.176037in}}%
\pgfpathlineto{\pgfqpoint{5.218724in}{2.185557in}}%
\pgfpathlineto{\pgfqpoint{5.226535in}{2.194925in}}%
\pgfpathlineto{\pgfqpoint{5.234337in}{2.204145in}}%
\pgfpathlineto{\pgfqpoint{5.219619in}{2.198869in}}%
\pgfpathlineto{\pgfqpoint{5.204915in}{2.193665in}}%
\pgfpathlineto{\pgfqpoint{5.190224in}{2.188533in}}%
\pgfpathlineto{\pgfqpoint{5.175546in}{2.183473in}}%
\pgfpathlineto{\pgfqpoint{5.167735in}{2.174262in}}%
\pgfpathlineto{\pgfqpoint{5.159915in}{2.164906in}}%
\pgfpathlineto{\pgfqpoint{5.152087in}{2.155407in}}%
\pgfpathlineto{\pgfqpoint{5.144250in}{2.145762in}}%
\pgfpathclose%
\pgfusepath{fill}%
\end{pgfscope}%
\begin{pgfscope}%
\pgfpathrectangle{\pgfqpoint{1.150000in}{0.150000in}}{\pgfqpoint{5.700000in}{5.700000in}}%
\pgfusepath{clip}%
\pgfsetbuttcap%
\pgfsetroundjoin%
\definecolor{currentfill}{rgb}{0.271305,0.019942,0.347269}%
\pgfsetfillcolor{currentfill}%
\pgfsetfillopacity{0.700000}%
\pgfsetlinewidth{0.000000pt}%
\definecolor{currentstroke}{rgb}{0.000000,0.000000,0.000000}%
\pgfsetstrokecolor{currentstroke}%
\pgfsetdash{}{0pt}%
\pgfpathmoveto{\pgfqpoint{3.673430in}{1.279403in}}%
\pgfpathlineto{\pgfqpoint{3.687577in}{1.276341in}}%
\pgfpathlineto{\pgfqpoint{3.701730in}{1.273351in}}%
\pgfpathlineto{\pgfqpoint{3.715890in}{1.270433in}}%
\pgfpathlineto{\pgfqpoint{3.730057in}{1.267588in}}%
\pgfpathlineto{\pgfqpoint{3.738417in}{1.276938in}}%
\pgfpathlineto{\pgfqpoint{3.746770in}{1.286412in}}%
\pgfpathlineto{\pgfqpoint{3.755117in}{1.296004in}}%
\pgfpathlineto{\pgfqpoint{3.763456in}{1.305709in}}%
\pgfpathlineto{\pgfqpoint{3.749304in}{1.308182in}}%
\pgfpathlineto{\pgfqpoint{3.735158in}{1.310727in}}%
\pgfpathlineto{\pgfqpoint{3.721020in}{1.313344in}}%
\pgfpathlineto{\pgfqpoint{3.706888in}{1.316034in}}%
\pgfpathlineto{\pgfqpoint{3.698534in}{1.306694in}}%
\pgfpathlineto{\pgfqpoint{3.690174in}{1.297471in}}%
\pgfpathlineto{\pgfqpoint{3.681806in}{1.288373in}}%
\pgfpathlineto{\pgfqpoint{3.673430in}{1.279403in}}%
\pgfpathclose%
\pgfusepath{fill}%
\end{pgfscope}%
\begin{pgfscope}%
\pgfpathrectangle{\pgfqpoint{1.150000in}{0.150000in}}{\pgfqpoint{5.700000in}{5.700000in}}%
\pgfusepath{clip}%
\pgfsetbuttcap%
\pgfsetroundjoin%
\definecolor{currentfill}{rgb}{0.171176,0.452530,0.557965}%
\pgfsetfillcolor{currentfill}%
\pgfsetfillopacity{0.700000}%
\pgfsetlinewidth{0.000000pt}%
\definecolor{currentstroke}{rgb}{0.000000,0.000000,0.000000}%
\pgfsetstrokecolor{currentstroke}%
\pgfsetdash{}{0pt}%
\pgfpathmoveto{\pgfqpoint{5.324431in}{2.261325in}}%
\pgfpathlineto{\pgfqpoint{5.339209in}{2.266953in}}%
\pgfpathlineto{\pgfqpoint{5.354001in}{2.272653in}}%
\pgfpathlineto{\pgfqpoint{5.368807in}{2.278425in}}%
\pgfpathlineto{\pgfqpoint{5.376547in}{2.286867in}}%
\pgfpathlineto{\pgfqpoint{5.384277in}{2.295152in}}%
\pgfpathlineto{\pgfqpoint{5.391997in}{2.303283in}}%
\pgfpathlineto{\pgfqpoint{5.399708in}{2.311259in}}%
\pgfpathlineto{\pgfqpoint{5.384912in}{2.305516in}}%
\pgfpathlineto{\pgfqpoint{5.370131in}{2.299844in}}%
\pgfpathlineto{\pgfqpoint{5.355364in}{2.294245in}}%
\pgfpathlineto{\pgfqpoint{5.347645in}{2.286241in}}%
\pgfpathlineto{\pgfqpoint{5.339916in}{2.278087in}}%
\pgfpathlineto{\pgfqpoint{5.332178in}{2.269782in}}%
\pgfpathlineto{\pgfqpoint{5.324431in}{2.261325in}}%
\pgfpathclose%
\pgfusepath{fill}%
\end{pgfscope}%
\begin{pgfscope}%
\pgfpathrectangle{\pgfqpoint{1.150000in}{0.150000in}}{\pgfqpoint{5.700000in}{5.700000in}}%
\pgfusepath{clip}%
\pgfsetbuttcap%
\pgfsetroundjoin%
\definecolor{currentfill}{rgb}{0.250425,0.274290,0.533103}%
\pgfsetfillcolor{currentfill}%
\pgfsetfillopacity{0.700000}%
\pgfsetlinewidth{0.000000pt}%
\definecolor{currentstroke}{rgb}{0.000000,0.000000,0.000000}%
\pgfsetstrokecolor{currentstroke}%
\pgfsetdash{}{0pt}%
\pgfpathmoveto{\pgfqpoint{4.662120in}{1.797287in}}%
\pgfpathlineto{\pgfqpoint{4.676589in}{1.800345in}}%
\pgfpathlineto{\pgfqpoint{4.691070in}{1.803474in}}%
\pgfpathlineto{\pgfqpoint{4.705562in}{1.806674in}}%
\pgfpathlineto{\pgfqpoint{4.720066in}{1.809945in}}%
\pgfpathlineto{\pgfqpoint{4.728099in}{1.822262in}}%
\pgfpathlineto{\pgfqpoint{4.736127in}{1.834478in}}%
\pgfpathlineto{\pgfqpoint{4.744148in}{1.846592in}}%
\pgfpathlineto{\pgfqpoint{4.752163in}{1.858600in}}%
\pgfpathlineto{\pgfqpoint{4.737664in}{1.855182in}}%
\pgfpathlineto{\pgfqpoint{4.723176in}{1.851834in}}%
\pgfpathlineto{\pgfqpoint{4.708700in}{1.848558in}}%
\pgfpathlineto{\pgfqpoint{4.694236in}{1.845353in}}%
\pgfpathlineto{\pgfqpoint{4.686216in}{1.833483in}}%
\pgfpathlineto{\pgfqpoint{4.678190in}{1.821515in}}%
\pgfpathlineto{\pgfqpoint{4.670158in}{1.809449in}}%
\pgfpathlineto{\pgfqpoint{4.662120in}{1.797287in}}%
\pgfpathclose%
\pgfusepath{fill}%
\end{pgfscope}%
\begin{pgfscope}%
\pgfpathrectangle{\pgfqpoint{1.150000in}{0.150000in}}{\pgfqpoint{5.700000in}{5.700000in}}%
\pgfusepath{clip}%
\pgfsetbuttcap%
\pgfsetroundjoin%
\definecolor{currentfill}{rgb}{0.179019,0.433756,0.557430}%
\pgfsetfillcolor{currentfill}%
\pgfsetfillopacity{0.700000}%
\pgfsetlinewidth{0.000000pt}%
\definecolor{currentstroke}{rgb}{0.000000,0.000000,0.000000}%
\pgfsetstrokecolor{currentstroke}%
\pgfsetdash{}{0pt}%
\pgfpathmoveto{\pgfqpoint{5.234337in}{2.204145in}}%
\pgfpathlineto{\pgfqpoint{5.249069in}{2.209492in}}%
\pgfpathlineto{\pgfqpoint{5.263815in}{2.214910in}}%
\pgfpathlineto{\pgfqpoint{5.278574in}{2.220401in}}%
\pgfpathlineto{\pgfqpoint{5.293348in}{2.225963in}}%
\pgfpathlineto{\pgfqpoint{5.301132in}{2.235035in}}%
\pgfpathlineto{\pgfqpoint{5.308908in}{2.243952in}}%
\pgfpathlineto{\pgfqpoint{5.316674in}{2.252716in}}%
\pgfpathlineto{\pgfqpoint{5.324431in}{2.261325in}}%
\pgfpathlineto{\pgfqpoint{5.309667in}{2.255769in}}%
\pgfpathlineto{\pgfqpoint{5.294916in}{2.250285in}}%
\pgfpathlineto{\pgfqpoint{5.280180in}{2.244872in}}%
\pgfpathlineto{\pgfqpoint{5.265458in}{2.239531in}}%
\pgfpathlineto{\pgfqpoint{5.257691in}{2.230907in}}%
\pgfpathlineto{\pgfqpoint{5.249916in}{2.222135in}}%
\pgfpathlineto{\pgfqpoint{5.242131in}{2.213214in}}%
\pgfpathlineto{\pgfqpoint{5.234337in}{2.204145in}}%
\pgfpathclose%
\pgfusepath{fill}%
\end{pgfscope}%
\begin{pgfscope}%
\pgfpathrectangle{\pgfqpoint{1.150000in}{0.150000in}}{\pgfqpoint{5.700000in}{5.700000in}}%
\pgfusepath{clip}%
\pgfsetbuttcap%
\pgfsetroundjoin%
\definecolor{currentfill}{rgb}{0.268510,0.009605,0.335427}%
\pgfsetfillcolor{currentfill}%
\pgfsetfillopacity{0.700000}%
\pgfsetlinewidth{0.000000pt}%
\definecolor{currentstroke}{rgb}{0.000000,0.000000,0.000000}%
\pgfsetstrokecolor{currentstroke}%
\pgfsetdash{}{0pt}%
\pgfpathmoveto{\pgfqpoint{3.436451in}{1.265145in}}%
\pgfpathlineto{\pgfqpoint{3.450561in}{1.260394in}}%
\pgfpathlineto{\pgfqpoint{3.464676in}{1.255716in}}%
\pgfpathlineto{\pgfqpoint{3.478797in}{1.251113in}}%
\pgfpathlineto{\pgfqpoint{3.492924in}{1.246585in}}%
\pgfpathlineto{\pgfqpoint{3.501401in}{1.253464in}}%
\pgfpathlineto{\pgfqpoint{3.509870in}{1.260532in}}%
\pgfpathlineto{\pgfqpoint{3.518329in}{1.267781in}}%
\pgfpathlineto{\pgfqpoint{3.526780in}{1.275205in}}%
\pgfpathlineto{\pgfqpoint{3.512673in}{1.279321in}}%
\pgfpathlineto{\pgfqpoint{3.498573in}{1.283510in}}%
\pgfpathlineto{\pgfqpoint{3.484478in}{1.287774in}}%
\pgfpathlineto{\pgfqpoint{3.470389in}{1.292112in}}%
\pgfpathlineto{\pgfqpoint{3.461919in}{1.285093in}}%
\pgfpathlineto{\pgfqpoint{3.453439in}{1.278255in}}%
\pgfpathlineto{\pgfqpoint{3.444950in}{1.271604in}}%
\pgfpathlineto{\pgfqpoint{3.436451in}{1.265145in}}%
\pgfpathclose%
\pgfusepath{fill}%
\end{pgfscope}%
\begin{pgfscope}%
\pgfpathrectangle{\pgfqpoint{1.150000in}{0.150000in}}{\pgfqpoint{5.700000in}{5.700000in}}%
\pgfusepath{clip}%
\pgfsetbuttcap%
\pgfsetroundjoin%
\definecolor{currentfill}{rgb}{0.237441,0.305202,0.541921}%
\pgfsetfillcolor{currentfill}%
\pgfsetfillopacity{0.700000}%
\pgfsetlinewidth{0.000000pt}%
\definecolor{currentstroke}{rgb}{0.000000,0.000000,0.000000}%
\pgfsetstrokecolor{currentstroke}%
\pgfsetdash{}{0pt}%
\pgfpathmoveto{\pgfqpoint{4.752163in}{1.858600in}}%
\pgfpathlineto{\pgfqpoint{4.766675in}{1.862090in}}%
\pgfpathlineto{\pgfqpoint{4.781198in}{1.865651in}}%
\pgfpathlineto{\pgfqpoint{4.795734in}{1.869283in}}%
\pgfpathlineto{\pgfqpoint{4.810282in}{1.872986in}}%
\pgfpathlineto{\pgfqpoint{4.818286in}{1.885023in}}%
\pgfpathlineto{\pgfqpoint{4.826283in}{1.896946in}}%
\pgfpathlineto{\pgfqpoint{4.834274in}{1.908754in}}%
\pgfpathlineto{\pgfqpoint{4.842259in}{1.920446in}}%
\pgfpathlineto{\pgfqpoint{4.827715in}{1.916617in}}%
\pgfpathlineto{\pgfqpoint{4.813184in}{1.912859in}}%
\pgfpathlineto{\pgfqpoint{4.798666in}{1.909172in}}%
\pgfpathlineto{\pgfqpoint{4.784159in}{1.905557in}}%
\pgfpathlineto{\pgfqpoint{4.776170in}{1.893983in}}%
\pgfpathlineto{\pgfqpoint{4.768174in}{1.882297in}}%
\pgfpathlineto{\pgfqpoint{4.760172in}{1.870503in}}%
\pgfpathlineto{\pgfqpoint{4.752163in}{1.858600in}}%
\pgfpathclose%
\pgfusepath{fill}%
\end{pgfscope}%
\begin{pgfscope}%
\pgfpathrectangle{\pgfqpoint{1.150000in}{0.150000in}}{\pgfqpoint{5.700000in}{5.700000in}}%
\pgfusepath{clip}%
\pgfsetbuttcap%
\pgfsetroundjoin%
\definecolor{currentfill}{rgb}{0.282656,0.100196,0.422160}%
\pgfsetfillcolor{currentfill}%
\pgfsetfillopacity{0.700000}%
\pgfsetlinewidth{0.000000pt}%
\definecolor{currentstroke}{rgb}{0.000000,0.000000,0.000000}%
\pgfsetstrokecolor{currentstroke}%
\pgfsetdash{}{0pt}%
\pgfpathmoveto{\pgfqpoint{4.089995in}{1.415212in}}%
\pgfpathlineto{\pgfqpoint{4.104254in}{1.414953in}}%
\pgfpathlineto{\pgfqpoint{4.118521in}{1.414765in}}%
\pgfpathlineto{\pgfqpoint{4.132798in}{1.414647in}}%
\pgfpathlineto{\pgfqpoint{4.147083in}{1.414601in}}%
\pgfpathlineto{\pgfqpoint{4.155296in}{1.426803in}}%
\pgfpathlineto{\pgfqpoint{4.163504in}{1.439024in}}%
\pgfpathlineto{\pgfqpoint{4.171707in}{1.451260in}}%
\pgfpathlineto{\pgfqpoint{4.179904in}{1.463505in}}%
\pgfpathlineto{\pgfqpoint{4.165626in}{1.463259in}}%
\pgfpathlineto{\pgfqpoint{4.151356in}{1.463084in}}%
\pgfpathlineto{\pgfqpoint{4.137096in}{1.462980in}}%
\pgfpathlineto{\pgfqpoint{4.122845in}{1.462947in}}%
\pgfpathlineto{\pgfqpoint{4.114641in}{1.450986in}}%
\pgfpathlineto{\pgfqpoint{4.106431in}{1.439041in}}%
\pgfpathlineto{\pgfqpoint{4.098216in}{1.427115in}}%
\pgfpathlineto{\pgfqpoint{4.089995in}{1.415212in}}%
\pgfpathclose%
\pgfusepath{fill}%
\end{pgfscope}%
\begin{pgfscope}%
\pgfpathrectangle{\pgfqpoint{1.150000in}{0.150000in}}{\pgfqpoint{5.700000in}{5.700000in}}%
\pgfusepath{clip}%
\pgfsetbuttcap%
\pgfsetroundjoin%
\definecolor{currentfill}{rgb}{0.280894,0.078907,0.402329}%
\pgfsetfillcolor{currentfill}%
\pgfsetfillopacity{0.700000}%
\pgfsetlinewidth{0.000000pt}%
\definecolor{currentstroke}{rgb}{0.000000,0.000000,0.000000}%
\pgfsetstrokecolor{currentstroke}%
\pgfsetdash{}{0pt}%
\pgfpathmoveto{\pgfqpoint{4.000081in}{1.370925in}}%
\pgfpathlineto{\pgfqpoint{4.014313in}{1.370069in}}%
\pgfpathlineto{\pgfqpoint{4.028553in}{1.369284in}}%
\pgfpathlineto{\pgfqpoint{4.042803in}{1.368571in}}%
\pgfpathlineto{\pgfqpoint{4.057061in}{1.367928in}}%
\pgfpathlineto{\pgfqpoint{4.065302in}{1.379691in}}%
\pgfpathlineto{\pgfqpoint{4.073539in}{1.391496in}}%
\pgfpathlineto{\pgfqpoint{4.081770in}{1.403338in}}%
\pgfpathlineto{\pgfqpoint{4.089995in}{1.415212in}}%
\pgfpathlineto{\pgfqpoint{4.075746in}{1.415542in}}%
\pgfpathlineto{\pgfqpoint{4.061505in}{1.415944in}}%
\pgfpathlineto{\pgfqpoint{4.047273in}{1.416416in}}%
\pgfpathlineto{\pgfqpoint{4.033050in}{1.416960in}}%
\pgfpathlineto{\pgfqpoint{4.024816in}{1.405390in}}%
\pgfpathlineto{\pgfqpoint{4.016576in}{1.393858in}}%
\pgfpathlineto{\pgfqpoint{4.008331in}{1.382368in}}%
\pgfpathlineto{\pgfqpoint{4.000081in}{1.370925in}}%
\pgfpathclose%
\pgfusepath{fill}%
\end{pgfscope}%
\begin{pgfscope}%
\pgfpathrectangle{\pgfqpoint{1.150000in}{0.150000in}}{\pgfqpoint{5.700000in}{5.700000in}}%
\pgfusepath{clip}%
\pgfsetbuttcap%
\pgfsetroundjoin%
\definecolor{currentfill}{rgb}{0.283187,0.125848,0.444960}%
\pgfsetfillcolor{currentfill}%
\pgfsetfillopacity{0.700000}%
\pgfsetlinewidth{0.000000pt}%
\definecolor{currentstroke}{rgb}{0.000000,0.000000,0.000000}%
\pgfsetstrokecolor{currentstroke}%
\pgfsetdash{}{0pt}%
\pgfpathmoveto{\pgfqpoint{4.179904in}{1.463505in}}%
\pgfpathlineto{\pgfqpoint{4.194192in}{1.463822in}}%
\pgfpathlineto{\pgfqpoint{4.208490in}{1.464210in}}%
\pgfpathlineto{\pgfqpoint{4.222797in}{1.464668in}}%
\pgfpathlineto{\pgfqpoint{4.237113in}{1.465197in}}%
\pgfpathlineto{\pgfqpoint{4.245299in}{1.477729in}}%
\pgfpathlineto{\pgfqpoint{4.253480in}{1.490257in}}%
\pgfpathlineto{\pgfqpoint{4.261656in}{1.502778in}}%
\pgfpathlineto{\pgfqpoint{4.269827in}{1.515288in}}%
\pgfpathlineto{\pgfqpoint{4.255516in}{1.514487in}}%
\pgfpathlineto{\pgfqpoint{4.241215in}{1.513756in}}%
\pgfpathlineto{\pgfqpoint{4.226924in}{1.513096in}}%
\pgfpathlineto{\pgfqpoint{4.212643in}{1.512508in}}%
\pgfpathlineto{\pgfqpoint{4.204466in}{1.500262in}}%
\pgfpathlineto{\pgfqpoint{4.196284in}{1.488010in}}%
\pgfpathlineto{\pgfqpoint{4.188097in}{1.475757in}}%
\pgfpathlineto{\pgfqpoint{4.179904in}{1.463505in}}%
\pgfpathclose%
\pgfusepath{fill}%
\end{pgfscope}%
\begin{pgfscope}%
\pgfpathrectangle{\pgfqpoint{1.150000in}{0.150000in}}{\pgfqpoint{5.700000in}{5.700000in}}%
\pgfusepath{clip}%
\pgfsetbuttcap%
\pgfsetroundjoin%
\definecolor{currentfill}{rgb}{0.268510,0.009605,0.335427}%
\pgfsetfillcolor{currentfill}%
\pgfsetfillopacity{0.700000}%
\pgfsetlinewidth{0.000000pt}%
\definecolor{currentstroke}{rgb}{0.000000,0.000000,0.000000}%
\pgfsetstrokecolor{currentstroke}%
\pgfsetdash{}{0pt}%
\pgfpathmoveto{\pgfqpoint{3.583266in}{1.259482in}}%
\pgfpathlineto{\pgfqpoint{3.597403in}{1.255735in}}%
\pgfpathlineto{\pgfqpoint{3.611547in}{1.252061in}}%
\pgfpathlineto{\pgfqpoint{3.625696in}{1.248461in}}%
\pgfpathlineto{\pgfqpoint{3.639852in}{1.244932in}}%
\pgfpathlineto{\pgfqpoint{3.648259in}{1.253327in}}%
\pgfpathlineto{\pgfqpoint{3.656657in}{1.261874in}}%
\pgfpathlineto{\pgfqpoint{3.665047in}{1.270568in}}%
\pgfpathlineto{\pgfqpoint{3.673430in}{1.279403in}}%
\pgfpathlineto{\pgfqpoint{3.659291in}{1.282538in}}%
\pgfpathlineto{\pgfqpoint{3.645158in}{1.285746in}}%
\pgfpathlineto{\pgfqpoint{3.631031in}{1.289027in}}%
\pgfpathlineto{\pgfqpoint{3.616911in}{1.292382in}}%
\pgfpathlineto{\pgfqpoint{3.608512in}{1.283932in}}%
\pgfpathlineto{\pgfqpoint{3.600105in}{1.275628in}}%
\pgfpathlineto{\pgfqpoint{3.591690in}{1.267476in}}%
\pgfpathlineto{\pgfqpoint{3.583266in}{1.259482in}}%
\pgfpathclose%
\pgfusepath{fill}%
\end{pgfscope}%
\begin{pgfscope}%
\pgfpathrectangle{\pgfqpoint{1.150000in}{0.150000in}}{\pgfqpoint{5.700000in}{5.700000in}}%
\pgfusepath{clip}%
\pgfsetbuttcap%
\pgfsetroundjoin%
\definecolor{currentfill}{rgb}{0.281887,0.150881,0.465405}%
\pgfsetfillcolor{currentfill}%
\pgfsetfillopacity{0.700000}%
\pgfsetlinewidth{0.000000pt}%
\definecolor{currentstroke}{rgb}{0.000000,0.000000,0.000000}%
\pgfsetstrokecolor{currentstroke}%
\pgfsetdash{}{0pt}%
\pgfpathmoveto{\pgfqpoint{4.269827in}{1.515288in}}%
\pgfpathlineto{\pgfqpoint{4.284147in}{1.516160in}}%
\pgfpathlineto{\pgfqpoint{4.298477in}{1.517103in}}%
\pgfpathlineto{\pgfqpoint{4.312817in}{1.518117in}}%
\pgfpathlineto{\pgfqpoint{4.327167in}{1.519201in}}%
\pgfpathlineto{\pgfqpoint{4.335327in}{1.531957in}}%
\pgfpathlineto{\pgfqpoint{4.343483in}{1.544688in}}%
\pgfpathlineto{\pgfqpoint{4.351633in}{1.557392in}}%
\pgfpathlineto{\pgfqpoint{4.359778in}{1.570066in}}%
\pgfpathlineto{\pgfqpoint{4.345433in}{1.568730in}}%
\pgfpathlineto{\pgfqpoint{4.331098in}{1.567464in}}%
\pgfpathlineto{\pgfqpoint{4.316773in}{1.566269in}}%
\pgfpathlineto{\pgfqpoint{4.302459in}{1.565146in}}%
\pgfpathlineto{\pgfqpoint{4.294308in}{1.552716in}}%
\pgfpathlineto{\pgfqpoint{4.286153in}{1.540261in}}%
\pgfpathlineto{\pgfqpoint{4.277992in}{1.527784in}}%
\pgfpathlineto{\pgfqpoint{4.269827in}{1.515288in}}%
\pgfpathclose%
\pgfusepath{fill}%
\end{pgfscope}%
\begin{pgfscope}%
\pgfpathrectangle{\pgfqpoint{1.150000in}{0.150000in}}{\pgfqpoint{5.700000in}{5.700000in}}%
\pgfusepath{clip}%
\pgfsetbuttcap%
\pgfsetroundjoin%
\definecolor{currentfill}{rgb}{0.277941,0.056324,0.381191}%
\pgfsetfillcolor{currentfill}%
\pgfsetfillopacity{0.700000}%
\pgfsetlinewidth{0.000000pt}%
\definecolor{currentstroke}{rgb}{0.000000,0.000000,0.000000}%
\pgfsetstrokecolor{currentstroke}%
\pgfsetdash{}{0pt}%
\pgfpathmoveto{\pgfqpoint{3.910138in}{1.331181in}}%
\pgfpathlineto{\pgfqpoint{3.924347in}{1.329708in}}%
\pgfpathlineto{\pgfqpoint{3.938564in}{1.328305in}}%
\pgfpathlineto{\pgfqpoint{3.952789in}{1.326975in}}%
\pgfpathlineto{\pgfqpoint{3.967023in}{1.325715in}}%
\pgfpathlineto{\pgfqpoint{3.975296in}{1.336923in}}%
\pgfpathlineto{\pgfqpoint{3.983563in}{1.348198in}}%
\pgfpathlineto{\pgfqpoint{3.991825in}{1.359533in}}%
\pgfpathlineto{\pgfqpoint{4.000081in}{1.370925in}}%
\pgfpathlineto{\pgfqpoint{3.985857in}{1.371852in}}%
\pgfpathlineto{\pgfqpoint{3.971642in}{1.372850in}}%
\pgfpathlineto{\pgfqpoint{3.957435in}{1.373920in}}%
\pgfpathlineto{\pgfqpoint{3.943236in}{1.375061in}}%
\pgfpathlineto{\pgfqpoint{3.934970in}{1.363994in}}%
\pgfpathlineto{\pgfqpoint{3.926699in}{1.352989in}}%
\pgfpathlineto{\pgfqpoint{3.918421in}{1.342049in}}%
\pgfpathlineto{\pgfqpoint{3.910138in}{1.331181in}}%
\pgfpathclose%
\pgfusepath{fill}%
\end{pgfscope}%
\begin{pgfscope}%
\pgfpathrectangle{\pgfqpoint{1.150000in}{0.150000in}}{\pgfqpoint{5.700000in}{5.700000in}}%
\pgfusepath{clip}%
\pgfsetbuttcap%
\pgfsetroundjoin%
\definecolor{currentfill}{rgb}{0.225863,0.330805,0.547314}%
\pgfsetfillcolor{currentfill}%
\pgfsetfillopacity{0.700000}%
\pgfsetlinewidth{0.000000pt}%
\definecolor{currentstroke}{rgb}{0.000000,0.000000,0.000000}%
\pgfsetstrokecolor{currentstroke}%
\pgfsetdash{}{0pt}%
\pgfpathmoveto{\pgfqpoint{4.842259in}{1.920446in}}%
\pgfpathlineto{\pgfqpoint{4.856814in}{1.924347in}}%
\pgfpathlineto{\pgfqpoint{4.871382in}{1.928318in}}%
\pgfpathlineto{\pgfqpoint{4.885963in}{1.932361in}}%
\pgfpathlineto{\pgfqpoint{4.900556in}{1.936476in}}%
\pgfpathlineto{\pgfqpoint{4.908528in}{1.948162in}}%
\pgfpathlineto{\pgfqpoint{4.916494in}{1.959725in}}%
\pgfpathlineto{\pgfqpoint{4.924452in}{1.971161in}}%
\pgfpathlineto{\pgfqpoint{4.932404in}{1.982470in}}%
\pgfpathlineto{\pgfqpoint{4.917816in}{1.978251in}}%
\pgfpathlineto{\pgfqpoint{4.903240in}{1.974104in}}%
\pgfpathlineto{\pgfqpoint{4.888678in}{1.970028in}}%
\pgfpathlineto{\pgfqpoint{4.874127in}{1.966023in}}%
\pgfpathlineto{\pgfqpoint{4.866170in}{1.954810in}}%
\pgfpathlineto{\pgfqpoint{4.858207in}{1.943475in}}%
\pgfpathlineto{\pgfqpoint{4.850236in}{1.932020in}}%
\pgfpathlineto{\pgfqpoint{4.842259in}{1.920446in}}%
\pgfpathclose%
\pgfusepath{fill}%
\end{pgfscope}%
\begin{pgfscope}%
\pgfpathrectangle{\pgfqpoint{1.150000in}{0.150000in}}{\pgfqpoint{5.700000in}{5.700000in}}%
\pgfusepath{clip}%
\pgfsetbuttcap%
\pgfsetroundjoin%
\definecolor{currentfill}{rgb}{0.278012,0.180367,0.486697}%
\pgfsetfillcolor{currentfill}%
\pgfsetfillopacity{0.700000}%
\pgfsetlinewidth{0.000000pt}%
\definecolor{currentstroke}{rgb}{0.000000,0.000000,0.000000}%
\pgfsetstrokecolor{currentstroke}%
\pgfsetdash{}{0pt}%
\pgfpathmoveto{\pgfqpoint{4.359778in}{1.570066in}}%
\pgfpathlineto{\pgfqpoint{4.374133in}{1.571473in}}%
\pgfpathlineto{\pgfqpoint{4.388498in}{1.572951in}}%
\pgfpathlineto{\pgfqpoint{4.402874in}{1.574500in}}%
\pgfpathlineto{\pgfqpoint{4.417260in}{1.576119in}}%
\pgfpathlineto{\pgfqpoint{4.425395in}{1.588998in}}%
\pgfpathlineto{\pgfqpoint{4.433525in}{1.601834in}}%
\pgfpathlineto{\pgfqpoint{4.441650in}{1.614624in}}%
\pgfpathlineto{\pgfqpoint{4.449770in}{1.627365in}}%
\pgfpathlineto{\pgfqpoint{4.435388in}{1.625514in}}%
\pgfpathlineto{\pgfqpoint{4.421017in}{1.623734in}}%
\pgfpathlineto{\pgfqpoint{4.406656in}{1.622025in}}%
\pgfpathlineto{\pgfqpoint{4.392306in}{1.620387in}}%
\pgfpathlineto{\pgfqpoint{4.384182in}{1.607869in}}%
\pgfpathlineto{\pgfqpoint{4.376052in}{1.595308in}}%
\pgfpathlineto{\pgfqpoint{4.367918in}{1.582706in}}%
\pgfpathlineto{\pgfqpoint{4.359778in}{1.570066in}}%
\pgfpathclose%
\pgfusepath{fill}%
\end{pgfscope}%
\begin{pgfscope}%
\pgfpathrectangle{\pgfqpoint{1.150000in}{0.150000in}}{\pgfqpoint{5.700000in}{5.700000in}}%
\pgfusepath{clip}%
\pgfsetbuttcap%
\pgfsetroundjoin%
\definecolor{currentfill}{rgb}{0.274952,0.037752,0.364543}%
\pgfsetfillcolor{currentfill}%
\pgfsetfillopacity{0.700000}%
\pgfsetlinewidth{0.000000pt}%
\definecolor{currentstroke}{rgb}{0.000000,0.000000,0.000000}%
\pgfsetstrokecolor{currentstroke}%
\pgfsetdash{}{0pt}%
\pgfpathmoveto{\pgfqpoint{3.820140in}{1.296541in}}%
\pgfpathlineto{\pgfqpoint{3.834330in}{1.294429in}}%
\pgfpathlineto{\pgfqpoint{3.848527in}{1.292388in}}%
\pgfpathlineto{\pgfqpoint{3.862732in}{1.290419in}}%
\pgfpathlineto{\pgfqpoint{3.876944in}{1.288521in}}%
\pgfpathlineto{\pgfqpoint{3.885252in}{1.299054in}}%
\pgfpathlineto{\pgfqpoint{3.893553in}{1.309678in}}%
\pgfpathlineto{\pgfqpoint{3.901849in}{1.320389in}}%
\pgfpathlineto{\pgfqpoint{3.910138in}{1.331181in}}%
\pgfpathlineto{\pgfqpoint{3.895937in}{1.332726in}}%
\pgfpathlineto{\pgfqpoint{3.881744in}{1.334342in}}%
\pgfpathlineto{\pgfqpoint{3.867559in}{1.336031in}}%
\pgfpathlineto{\pgfqpoint{3.853381in}{1.337790in}}%
\pgfpathlineto{\pgfqpoint{3.845080in}{1.327343in}}%
\pgfpathlineto{\pgfqpoint{3.836773in}{1.316982in}}%
\pgfpathlineto{\pgfqpoint{3.828460in}{1.306713in}}%
\pgfpathlineto{\pgfqpoint{3.820140in}{1.296541in}}%
\pgfpathclose%
\pgfusepath{fill}%
\end{pgfscope}%
\begin{pgfscope}%
\pgfpathrectangle{\pgfqpoint{1.150000in}{0.150000in}}{\pgfqpoint{5.700000in}{5.700000in}}%
\pgfusepath{clip}%
\pgfsetbuttcap%
\pgfsetroundjoin%
\definecolor{currentfill}{rgb}{0.271828,0.209303,0.504434}%
\pgfsetfillcolor{currentfill}%
\pgfsetfillopacity{0.700000}%
\pgfsetlinewidth{0.000000pt}%
\definecolor{currentstroke}{rgb}{0.000000,0.000000,0.000000}%
\pgfsetstrokecolor{currentstroke}%
\pgfsetdash{}{0pt}%
\pgfpathmoveto{\pgfqpoint{4.449770in}{1.627365in}}%
\pgfpathlineto{\pgfqpoint{4.464162in}{1.629286in}}%
\pgfpathlineto{\pgfqpoint{4.478565in}{1.631279in}}%
\pgfpathlineto{\pgfqpoint{4.492979in}{1.633342in}}%
\pgfpathlineto{\pgfqpoint{4.507404in}{1.635476in}}%
\pgfpathlineto{\pgfqpoint{4.515514in}{1.648383in}}%
\pgfpathlineto{\pgfqpoint{4.523618in}{1.661230in}}%
\pgfpathlineto{\pgfqpoint{4.531718in}{1.674013in}}%
\pgfpathlineto{\pgfqpoint{4.539811in}{1.686730in}}%
\pgfpathlineto{\pgfqpoint{4.525391in}{1.684385in}}%
\pgfpathlineto{\pgfqpoint{4.510981in}{1.682112in}}%
\pgfpathlineto{\pgfqpoint{4.496583in}{1.679908in}}%
\pgfpathlineto{\pgfqpoint{4.482195in}{1.677776in}}%
\pgfpathlineto{\pgfqpoint{4.474096in}{1.665262in}}%
\pgfpathlineto{\pgfqpoint{4.465993in}{1.652687in}}%
\pgfpathlineto{\pgfqpoint{4.457884in}{1.640053in}}%
\pgfpathlineto{\pgfqpoint{4.449770in}{1.627365in}}%
\pgfpathclose%
\pgfusepath{fill}%
\end{pgfscope}%
\begin{pgfscope}%
\pgfpathrectangle{\pgfqpoint{1.150000in}{0.150000in}}{\pgfqpoint{5.700000in}{5.700000in}}%
\pgfusepath{clip}%
\pgfsetbuttcap%
\pgfsetroundjoin%
\definecolor{currentfill}{rgb}{0.214298,0.355619,0.551184}%
\pgfsetfillcolor{currentfill}%
\pgfsetfillopacity{0.700000}%
\pgfsetlinewidth{0.000000pt}%
\definecolor{currentstroke}{rgb}{0.000000,0.000000,0.000000}%
\pgfsetstrokecolor{currentstroke}%
\pgfsetdash{}{0pt}%
\pgfpathmoveto{\pgfqpoint{4.932404in}{1.982470in}}%
\pgfpathlineto{\pgfqpoint{4.947004in}{1.986760in}}%
\pgfpathlineto{\pgfqpoint{4.961618in}{1.991122in}}%
\pgfpathlineto{\pgfqpoint{4.976244in}{1.995555in}}%
\pgfpathlineto{\pgfqpoint{4.990883in}{2.000059in}}%
\pgfpathlineto{\pgfqpoint{4.998822in}{2.011331in}}%
\pgfpathlineto{\pgfqpoint{5.006753in}{2.022469in}}%
\pgfpathlineto{\pgfqpoint{5.014677in}{2.033472in}}%
\pgfpathlineto{\pgfqpoint{5.022593in}{2.044338in}}%
\pgfpathlineto{\pgfqpoint{5.007960in}{2.039751in}}%
\pgfpathlineto{\pgfqpoint{4.993339in}{2.035235in}}%
\pgfpathlineto{\pgfqpoint{4.978731in}{2.030791in}}%
\pgfpathlineto{\pgfqpoint{4.964136in}{2.026418in}}%
\pgfpathlineto{\pgfqpoint{4.956214in}{2.015626in}}%
\pgfpathlineto{\pgfqpoint{4.948285in}{2.004704in}}%
\pgfpathlineto{\pgfqpoint{4.940348in}{1.993652in}}%
\pgfpathlineto{\pgfqpoint{4.932404in}{1.982470in}}%
\pgfpathclose%
\pgfusepath{fill}%
\end{pgfscope}%
\begin{pgfscope}%
\pgfpathrectangle{\pgfqpoint{1.150000in}{0.150000in}}{\pgfqpoint{5.700000in}{5.700000in}}%
\pgfusepath{clip}%
\pgfsetbuttcap%
\pgfsetroundjoin%
\definecolor{currentfill}{rgb}{0.272594,0.025563,0.353093}%
\pgfsetfillcolor{currentfill}%
\pgfsetfillopacity{0.700000}%
\pgfsetlinewidth{0.000000pt}%
\definecolor{currentstroke}{rgb}{0.000000,0.000000,0.000000}%
\pgfsetstrokecolor{currentstroke}%
\pgfsetdash{}{0pt}%
\pgfpathmoveto{\pgfqpoint{3.730057in}{1.267588in}}%
\pgfpathlineto{\pgfqpoint{3.744231in}{1.264816in}}%
\pgfpathlineto{\pgfqpoint{3.758412in}{1.262115in}}%
\pgfpathlineto{\pgfqpoint{3.772600in}{1.259486in}}%
\pgfpathlineto{\pgfqpoint{3.786796in}{1.256929in}}%
\pgfpathlineto{\pgfqpoint{3.795142in}{1.266659in}}%
\pgfpathlineto{\pgfqpoint{3.803481in}{1.276508in}}%
\pgfpathlineto{\pgfqpoint{3.811814in}{1.286471in}}%
\pgfpathlineto{\pgfqpoint{3.820140in}{1.296541in}}%
\pgfpathlineto{\pgfqpoint{3.805958in}{1.298725in}}%
\pgfpathlineto{\pgfqpoint{3.791784in}{1.300981in}}%
\pgfpathlineto{\pgfqpoint{3.777616in}{1.303309in}}%
\pgfpathlineto{\pgfqpoint{3.763456in}{1.305709in}}%
\pgfpathlineto{\pgfqpoint{3.755117in}{1.296004in}}%
\pgfpathlineto{\pgfqpoint{3.746770in}{1.286412in}}%
\pgfpathlineto{\pgfqpoint{3.738417in}{1.276938in}}%
\pgfpathlineto{\pgfqpoint{3.730057in}{1.267588in}}%
\pgfpathclose%
\pgfusepath{fill}%
\end{pgfscope}%
\begin{pgfscope}%
\pgfpathrectangle{\pgfqpoint{1.150000in}{0.150000in}}{\pgfqpoint{5.700000in}{5.700000in}}%
\pgfusepath{clip}%
\pgfsetbuttcap%
\pgfsetroundjoin%
\definecolor{currentfill}{rgb}{0.268510,0.009605,0.335427}%
\pgfsetfillcolor{currentfill}%
\pgfsetfillopacity{0.700000}%
\pgfsetlinewidth{0.000000pt}%
\definecolor{currentstroke}{rgb}{0.000000,0.000000,0.000000}%
\pgfsetstrokecolor{currentstroke}%
\pgfsetdash{}{0pt}%
\pgfpathmoveto{\pgfqpoint{3.492924in}{1.246585in}}%
\pgfpathlineto{\pgfqpoint{3.507056in}{1.242130in}}%
\pgfpathlineto{\pgfqpoint{3.521194in}{1.237749in}}%
\pgfpathlineto{\pgfqpoint{3.535338in}{1.233442in}}%
\pgfpathlineto{\pgfqpoint{3.549487in}{1.229208in}}%
\pgfpathlineto{\pgfqpoint{3.557945in}{1.236509in}}%
\pgfpathlineto{\pgfqpoint{3.566394in}{1.243992in}}%
\pgfpathlineto{\pgfqpoint{3.574835in}{1.251652in}}%
\pgfpathlineto{\pgfqpoint{3.583266in}{1.259482in}}%
\pgfpathlineto{\pgfqpoint{3.569136in}{1.263303in}}%
\pgfpathlineto{\pgfqpoint{3.555011in}{1.267196in}}%
\pgfpathlineto{\pgfqpoint{3.540892in}{1.271164in}}%
\pgfpathlineto{\pgfqpoint{3.526780in}{1.275205in}}%
\pgfpathlineto{\pgfqpoint{3.518329in}{1.267781in}}%
\pgfpathlineto{\pgfqpoint{3.509870in}{1.260532in}}%
\pgfpathlineto{\pgfqpoint{3.501401in}{1.253464in}}%
\pgfpathlineto{\pgfqpoint{3.492924in}{1.246585in}}%
\pgfpathclose%
\pgfusepath{fill}%
\end{pgfscope}%
\begin{pgfscope}%
\pgfpathrectangle{\pgfqpoint{1.150000in}{0.150000in}}{\pgfqpoint{5.700000in}{5.700000in}}%
\pgfusepath{clip}%
\pgfsetbuttcap%
\pgfsetroundjoin%
\definecolor{currentfill}{rgb}{0.263663,0.237631,0.518762}%
\pgfsetfillcolor{currentfill}%
\pgfsetfillopacity{0.700000}%
\pgfsetlinewidth{0.000000pt}%
\definecolor{currentstroke}{rgb}{0.000000,0.000000,0.000000}%
\pgfsetstrokecolor{currentstroke}%
\pgfsetdash{}{0pt}%
\pgfpathmoveto{\pgfqpoint{4.539811in}{1.686730in}}%
\pgfpathlineto{\pgfqpoint{4.554243in}{1.689146in}}%
\pgfpathlineto{\pgfqpoint{4.568686in}{1.691632in}}%
\pgfpathlineto{\pgfqpoint{4.583140in}{1.694190in}}%
\pgfpathlineto{\pgfqpoint{4.597605in}{1.696818in}}%
\pgfpathlineto{\pgfqpoint{4.605690in}{1.709664in}}%
\pgfpathlineto{\pgfqpoint{4.613768in}{1.722433in}}%
\pgfpathlineto{\pgfqpoint{4.621842in}{1.735122in}}%
\pgfpathlineto{\pgfqpoint{4.629909in}{1.747729in}}%
\pgfpathlineto{\pgfqpoint{4.615448in}{1.744910in}}%
\pgfpathlineto{\pgfqpoint{4.600998in}{1.742163in}}%
\pgfpathlineto{\pgfqpoint{4.586559in}{1.739487in}}%
\pgfpathlineto{\pgfqpoint{4.572131in}{1.736881in}}%
\pgfpathlineto{\pgfqpoint{4.564060in}{1.724456in}}%
\pgfpathlineto{\pgfqpoint{4.555983in}{1.711954in}}%
\pgfpathlineto{\pgfqpoint{4.547900in}{1.699378in}}%
\pgfpathlineto{\pgfqpoint{4.539811in}{1.686730in}}%
\pgfpathclose%
\pgfusepath{fill}%
\end{pgfscope}%
\begin{pgfscope}%
\pgfpathrectangle{\pgfqpoint{1.150000in}{0.150000in}}{\pgfqpoint{5.700000in}{5.700000in}}%
\pgfusepath{clip}%
\pgfsetbuttcap%
\pgfsetroundjoin%
\definecolor{currentfill}{rgb}{0.201239,0.383670,0.554294}%
\pgfsetfillcolor{currentfill}%
\pgfsetfillopacity{0.700000}%
\pgfsetlinewidth{0.000000pt}%
\definecolor{currentstroke}{rgb}{0.000000,0.000000,0.000000}%
\pgfsetstrokecolor{currentstroke}%
\pgfsetdash{}{0pt}%
\pgfpathmoveto{\pgfqpoint{5.022593in}{2.044338in}}%
\pgfpathlineto{\pgfqpoint{5.037240in}{2.048997in}}%
\pgfpathlineto{\pgfqpoint{5.051899in}{2.053727in}}%
\pgfpathlineto{\pgfqpoint{5.066572in}{2.058528in}}%
\pgfpathlineto{\pgfqpoint{5.081258in}{2.063401in}}%
\pgfpathlineto{\pgfqpoint{5.089161in}{2.074200in}}%
\pgfpathlineto{\pgfqpoint{5.097055in}{2.084855in}}%
\pgfpathlineto{\pgfqpoint{5.104942in}{2.095367in}}%
\pgfpathlineto{\pgfqpoint{5.112820in}{2.105735in}}%
\pgfpathlineto{\pgfqpoint{5.098140in}{2.100801in}}%
\pgfpathlineto{\pgfqpoint{5.083474in}{2.095939in}}%
\pgfpathlineto{\pgfqpoint{5.068820in}{2.091148in}}%
\pgfpathlineto{\pgfqpoint{5.054180in}{2.086429in}}%
\pgfpathlineto{\pgfqpoint{5.046295in}{2.076113in}}%
\pgfpathlineto{\pgfqpoint{5.038402in}{2.065659in}}%
\pgfpathlineto{\pgfqpoint{5.030502in}{2.055067in}}%
\pgfpathlineto{\pgfqpoint{5.022593in}{2.044338in}}%
\pgfpathclose%
\pgfusepath{fill}%
\end{pgfscope}%
\begin{pgfscope}%
\pgfpathrectangle{\pgfqpoint{1.150000in}{0.150000in}}{\pgfqpoint{5.700000in}{5.700000in}}%
\pgfusepath{clip}%
\pgfsetbuttcap%
\pgfsetroundjoin%
\definecolor{currentfill}{rgb}{0.253935,0.265254,0.529983}%
\pgfsetfillcolor{currentfill}%
\pgfsetfillopacity{0.700000}%
\pgfsetlinewidth{0.000000pt}%
\definecolor{currentstroke}{rgb}{0.000000,0.000000,0.000000}%
\pgfsetstrokecolor{currentstroke}%
\pgfsetdash{}{0pt}%
\pgfpathmoveto{\pgfqpoint{4.629909in}{1.747729in}}%
\pgfpathlineto{\pgfqpoint{4.644382in}{1.750618in}}%
\pgfpathlineto{\pgfqpoint{4.658866in}{1.753578in}}%
\pgfpathlineto{\pgfqpoint{4.673362in}{1.756609in}}%
\pgfpathlineto{\pgfqpoint{4.687870in}{1.759711in}}%
\pgfpathlineto{\pgfqpoint{4.695928in}{1.772410in}}%
\pgfpathlineto{\pgfqpoint{4.703980in}{1.785017in}}%
\pgfpathlineto{\pgfqpoint{4.712026in}{1.797530in}}%
\pgfpathlineto{\pgfqpoint{4.720066in}{1.809945in}}%
\pgfpathlineto{\pgfqpoint{4.705562in}{1.806674in}}%
\pgfpathlineto{\pgfqpoint{4.691070in}{1.803474in}}%
\pgfpathlineto{\pgfqpoint{4.676589in}{1.800345in}}%
\pgfpathlineto{\pgfqpoint{4.662120in}{1.797287in}}%
\pgfpathlineto{\pgfqpoint{4.654076in}{1.785033in}}%
\pgfpathlineto{\pgfqpoint{4.646026in}{1.772686in}}%
\pgfpathlineto{\pgfqpoint{4.637971in}{1.760251in}}%
\pgfpathlineto{\pgfqpoint{4.629909in}{1.747729in}}%
\pgfpathclose%
\pgfusepath{fill}%
\end{pgfscope}%
\begin{pgfscope}%
\pgfpathrectangle{\pgfqpoint{1.150000in}{0.150000in}}{\pgfqpoint{5.700000in}{5.700000in}}%
\pgfusepath{clip}%
\pgfsetbuttcap%
\pgfsetroundjoin%
\definecolor{currentfill}{rgb}{0.190631,0.407061,0.556089}%
\pgfsetfillcolor{currentfill}%
\pgfsetfillopacity{0.700000}%
\pgfsetlinewidth{0.000000pt}%
\definecolor{currentstroke}{rgb}{0.000000,0.000000,0.000000}%
\pgfsetstrokecolor{currentstroke}%
\pgfsetdash{}{0pt}%
\pgfpathmoveto{\pgfqpoint{5.112820in}{2.105735in}}%
\pgfpathlineto{\pgfqpoint{5.127513in}{2.110741in}}%
\pgfpathlineto{\pgfqpoint{5.142220in}{2.115818in}}%
\pgfpathlineto{\pgfqpoint{5.156940in}{2.120966in}}%
\pgfpathlineto{\pgfqpoint{5.171673in}{2.126187in}}%
\pgfpathlineto{\pgfqpoint{5.179536in}{2.136458in}}%
\pgfpathlineto{\pgfqpoint{5.187391in}{2.146578in}}%
\pgfpathlineto{\pgfqpoint{5.195237in}{2.156548in}}%
\pgfpathlineto{\pgfqpoint{5.203075in}{2.166368in}}%
\pgfpathlineto{\pgfqpoint{5.188348in}{2.161109in}}%
\pgfpathlineto{\pgfqpoint{5.173635in}{2.155922in}}%
\pgfpathlineto{\pgfqpoint{5.158936in}{2.150806in}}%
\pgfpathlineto{\pgfqpoint{5.144250in}{2.145762in}}%
\pgfpathlineto{\pgfqpoint{5.136405in}{2.135972in}}%
\pgfpathlineto{\pgfqpoint{5.128552in}{2.126038in}}%
\pgfpathlineto{\pgfqpoint{5.120690in}{2.115959in}}%
\pgfpathlineto{\pgfqpoint{5.112820in}{2.105735in}}%
\pgfpathclose%
\pgfusepath{fill}%
\end{pgfscope}%
\begin{pgfscope}%
\pgfpathrectangle{\pgfqpoint{1.150000in}{0.150000in}}{\pgfqpoint{5.700000in}{5.700000in}}%
\pgfusepath{clip}%
\pgfsetbuttcap%
\pgfsetroundjoin%
\definecolor{currentfill}{rgb}{0.269944,0.014625,0.341379}%
\pgfsetfillcolor{currentfill}%
\pgfsetfillopacity{0.700000}%
\pgfsetlinewidth{0.000000pt}%
\definecolor{currentstroke}{rgb}{0.000000,0.000000,0.000000}%
\pgfsetstrokecolor{currentstroke}%
\pgfsetdash{}{0pt}%
\pgfpathmoveto{\pgfqpoint{3.639852in}{1.244932in}}%
\pgfpathlineto{\pgfqpoint{3.654015in}{1.241477in}}%
\pgfpathlineto{\pgfqpoint{3.668184in}{1.238094in}}%
\pgfpathlineto{\pgfqpoint{3.682360in}{1.234784in}}%
\pgfpathlineto{\pgfqpoint{3.696543in}{1.231545in}}%
\pgfpathlineto{\pgfqpoint{3.704932in}{1.240341in}}%
\pgfpathlineto{\pgfqpoint{3.713315in}{1.249284in}}%
\pgfpathlineto{\pgfqpoint{3.721689in}{1.258368in}}%
\pgfpathlineto{\pgfqpoint{3.730057in}{1.267588in}}%
\pgfpathlineto{\pgfqpoint{3.715890in}{1.270433in}}%
\pgfpathlineto{\pgfqpoint{3.701730in}{1.273351in}}%
\pgfpathlineto{\pgfqpoint{3.687577in}{1.276341in}}%
\pgfpathlineto{\pgfqpoint{3.673430in}{1.279403in}}%
\pgfpathlineto{\pgfqpoint{3.665047in}{1.270568in}}%
\pgfpathlineto{\pgfqpoint{3.656657in}{1.261874in}}%
\pgfpathlineto{\pgfqpoint{3.648259in}{1.253327in}}%
\pgfpathlineto{\pgfqpoint{3.639852in}{1.244932in}}%
\pgfpathclose%
\pgfusepath{fill}%
\end{pgfscope}%
\begin{pgfscope}%
\pgfpathrectangle{\pgfqpoint{1.150000in}{0.150000in}}{\pgfqpoint{5.700000in}{5.700000in}}%
\pgfusepath{clip}%
\pgfsetbuttcap%
\pgfsetroundjoin%
\definecolor{currentfill}{rgb}{0.172719,0.448791,0.557885}%
\pgfsetfillcolor{currentfill}%
\pgfsetfillopacity{0.700000}%
\pgfsetlinewidth{0.000000pt}%
\definecolor{currentstroke}{rgb}{0.000000,0.000000,0.000000}%
\pgfsetstrokecolor{currentstroke}%
\pgfsetdash{}{0pt}%
\pgfpathmoveto{\pgfqpoint{5.293348in}{2.225963in}}%
\pgfpathlineto{\pgfqpoint{5.308135in}{2.231597in}}%
\pgfpathlineto{\pgfqpoint{5.322937in}{2.237303in}}%
\pgfpathlineto{\pgfqpoint{5.337752in}{2.243081in}}%
\pgfpathlineto{\pgfqpoint{5.345530in}{2.252155in}}%
\pgfpathlineto{\pgfqpoint{5.353299in}{2.261070in}}%
\pgfpathlineto{\pgfqpoint{5.361057in}{2.269826in}}%
\pgfpathlineto{\pgfqpoint{5.368807in}{2.278425in}}%
\pgfpathlineto{\pgfqpoint{5.354001in}{2.272653in}}%
\pgfpathlineto{\pgfqpoint{5.339209in}{2.266953in}}%
\pgfpathlineto{\pgfqpoint{5.324431in}{2.261325in}}%
\pgfpathlineto{\pgfqpoint{5.316674in}{2.252716in}}%
\pgfpathlineto{\pgfqpoint{5.308908in}{2.243952in}}%
\pgfpathlineto{\pgfqpoint{5.301132in}{2.235035in}}%
\pgfpathlineto{\pgfqpoint{5.293348in}{2.225963in}}%
\pgfpathclose%
\pgfusepath{fill}%
\end{pgfscope}%
\begin{pgfscope}%
\pgfpathrectangle{\pgfqpoint{1.150000in}{0.150000in}}{\pgfqpoint{5.700000in}{5.700000in}}%
\pgfusepath{clip}%
\pgfsetbuttcap%
\pgfsetroundjoin%
\definecolor{currentfill}{rgb}{0.180629,0.429975,0.557282}%
\pgfsetfillcolor{currentfill}%
\pgfsetfillopacity{0.700000}%
\pgfsetlinewidth{0.000000pt}%
\definecolor{currentstroke}{rgb}{0.000000,0.000000,0.000000}%
\pgfsetstrokecolor{currentstroke}%
\pgfsetdash{}{0pt}%
\pgfpathmoveto{\pgfqpoint{5.203075in}{2.166368in}}%
\pgfpathlineto{\pgfqpoint{5.217815in}{2.171699in}}%
\pgfpathlineto{\pgfqpoint{5.232569in}{2.177101in}}%
\pgfpathlineto{\pgfqpoint{5.247337in}{2.182576in}}%
\pgfpathlineto{\pgfqpoint{5.262118in}{2.188122in}}%
\pgfpathlineto{\pgfqpoint{5.269939in}{2.197816in}}%
\pgfpathlineto{\pgfqpoint{5.277751in}{2.207354in}}%
\pgfpathlineto{\pgfqpoint{5.285554in}{2.216736in}}%
\pgfpathlineto{\pgfqpoint{5.293348in}{2.225963in}}%
\pgfpathlineto{\pgfqpoint{5.278574in}{2.220401in}}%
\pgfpathlineto{\pgfqpoint{5.263815in}{2.214910in}}%
\pgfpathlineto{\pgfqpoint{5.249069in}{2.209492in}}%
\pgfpathlineto{\pgfqpoint{5.234337in}{2.204145in}}%
\pgfpathlineto{\pgfqpoint{5.226535in}{2.194925in}}%
\pgfpathlineto{\pgfqpoint{5.218724in}{2.185557in}}%
\pgfpathlineto{\pgfqpoint{5.210904in}{2.176037in}}%
\pgfpathlineto{\pgfqpoint{5.203075in}{2.166368in}}%
\pgfpathclose%
\pgfusepath{fill}%
\end{pgfscope}%
\begin{pgfscope}%
\pgfpathrectangle{\pgfqpoint{1.150000in}{0.150000in}}{\pgfqpoint{5.700000in}{5.700000in}}%
\pgfusepath{clip}%
\pgfsetbuttcap%
\pgfsetroundjoin%
\definecolor{currentfill}{rgb}{0.243113,0.292092,0.538516}%
\pgfsetfillcolor{currentfill}%
\pgfsetfillopacity{0.700000}%
\pgfsetlinewidth{0.000000pt}%
\definecolor{currentstroke}{rgb}{0.000000,0.000000,0.000000}%
\pgfsetstrokecolor{currentstroke}%
\pgfsetdash{}{0pt}%
\pgfpathmoveto{\pgfqpoint{4.720066in}{1.809945in}}%
\pgfpathlineto{\pgfqpoint{4.734581in}{1.813287in}}%
\pgfpathlineto{\pgfqpoint{4.749109in}{1.816700in}}%
\pgfpathlineto{\pgfqpoint{4.763649in}{1.820185in}}%
\pgfpathlineto{\pgfqpoint{4.778200in}{1.823740in}}%
\pgfpathlineto{\pgfqpoint{4.786230in}{1.836212in}}%
\pgfpathlineto{\pgfqpoint{4.794254in}{1.848579in}}%
\pgfpathlineto{\pgfqpoint{4.802271in}{1.860838in}}%
\pgfpathlineto{\pgfqpoint{4.810282in}{1.872986in}}%
\pgfpathlineto{\pgfqpoint{4.795734in}{1.869283in}}%
\pgfpathlineto{\pgfqpoint{4.781198in}{1.865651in}}%
\pgfpathlineto{\pgfqpoint{4.766675in}{1.862090in}}%
\pgfpathlineto{\pgfqpoint{4.752163in}{1.858600in}}%
\pgfpathlineto{\pgfqpoint{4.744148in}{1.846592in}}%
\pgfpathlineto{\pgfqpoint{4.736127in}{1.834478in}}%
\pgfpathlineto{\pgfqpoint{4.728099in}{1.822262in}}%
\pgfpathlineto{\pgfqpoint{4.720066in}{1.809945in}}%
\pgfpathclose%
\pgfusepath{fill}%
\end{pgfscope}%
\begin{pgfscope}%
\pgfpathrectangle{\pgfqpoint{1.150000in}{0.150000in}}{\pgfqpoint{5.700000in}{5.700000in}}%
\pgfusepath{clip}%
\pgfsetbuttcap%
\pgfsetroundjoin%
\definecolor{currentfill}{rgb}{0.281924,0.089666,0.412415}%
\pgfsetfillcolor{currentfill}%
\pgfsetfillopacity{0.700000}%
\pgfsetlinewidth{0.000000pt}%
\definecolor{currentstroke}{rgb}{0.000000,0.000000,0.000000}%
\pgfsetstrokecolor{currentstroke}%
\pgfsetdash{}{0pt}%
\pgfpathmoveto{\pgfqpoint{4.057061in}{1.367928in}}%
\pgfpathlineto{\pgfqpoint{4.071327in}{1.367356in}}%
\pgfpathlineto{\pgfqpoint{4.085603in}{1.366855in}}%
\pgfpathlineto{\pgfqpoint{4.099887in}{1.366424in}}%
\pgfpathlineto{\pgfqpoint{4.114180in}{1.366065in}}%
\pgfpathlineto{\pgfqpoint{4.122414in}{1.378149in}}%
\pgfpathlineto{\pgfqpoint{4.130642in}{1.390269in}}%
\pgfpathlineto{\pgfqpoint{4.138865in}{1.402421in}}%
\pgfpathlineto{\pgfqpoint{4.147083in}{1.414601in}}%
\pgfpathlineto{\pgfqpoint{4.132798in}{1.414647in}}%
\pgfpathlineto{\pgfqpoint{4.118521in}{1.414765in}}%
\pgfpathlineto{\pgfqpoint{4.104254in}{1.414953in}}%
\pgfpathlineto{\pgfqpoint{4.089995in}{1.415212in}}%
\pgfpathlineto{\pgfqpoint{4.081770in}{1.403338in}}%
\pgfpathlineto{\pgfqpoint{4.073539in}{1.391496in}}%
\pgfpathlineto{\pgfqpoint{4.065302in}{1.379691in}}%
\pgfpathlineto{\pgfqpoint{4.057061in}{1.367928in}}%
\pgfpathclose%
\pgfusepath{fill}%
\end{pgfscope}%
\begin{pgfscope}%
\pgfpathrectangle{\pgfqpoint{1.150000in}{0.150000in}}{\pgfqpoint{5.700000in}{5.700000in}}%
\pgfusepath{clip}%
\pgfsetbuttcap%
\pgfsetroundjoin%
\definecolor{currentfill}{rgb}{0.283197,0.115680,0.436115}%
\pgfsetfillcolor{currentfill}%
\pgfsetfillopacity{0.700000}%
\pgfsetlinewidth{0.000000pt}%
\definecolor{currentstroke}{rgb}{0.000000,0.000000,0.000000}%
\pgfsetstrokecolor{currentstroke}%
\pgfsetdash{}{0pt}%
\pgfpathmoveto{\pgfqpoint{4.147083in}{1.414601in}}%
\pgfpathlineto{\pgfqpoint{4.161378in}{1.414625in}}%
\pgfpathlineto{\pgfqpoint{4.175682in}{1.414720in}}%
\pgfpathlineto{\pgfqpoint{4.189996in}{1.414885in}}%
\pgfpathlineto{\pgfqpoint{4.204319in}{1.415121in}}%
\pgfpathlineto{\pgfqpoint{4.212525in}{1.427625in}}%
\pgfpathlineto{\pgfqpoint{4.220726in}{1.440141in}}%
\pgfpathlineto{\pgfqpoint{4.228922in}{1.452667in}}%
\pgfpathlineto{\pgfqpoint{4.237113in}{1.465197in}}%
\pgfpathlineto{\pgfqpoint{4.222797in}{1.464668in}}%
\pgfpathlineto{\pgfqpoint{4.208490in}{1.464210in}}%
\pgfpathlineto{\pgfqpoint{4.194192in}{1.463822in}}%
\pgfpathlineto{\pgfqpoint{4.179904in}{1.463505in}}%
\pgfpathlineto{\pgfqpoint{4.171707in}{1.451260in}}%
\pgfpathlineto{\pgfqpoint{4.163504in}{1.439024in}}%
\pgfpathlineto{\pgfqpoint{4.155296in}{1.426803in}}%
\pgfpathlineto{\pgfqpoint{4.147083in}{1.414601in}}%
\pgfpathclose%
\pgfusepath{fill}%
\end{pgfscope}%
\begin{pgfscope}%
\pgfpathrectangle{\pgfqpoint{1.150000in}{0.150000in}}{\pgfqpoint{5.700000in}{5.700000in}}%
\pgfusepath{clip}%
\pgfsetbuttcap%
\pgfsetroundjoin%
\definecolor{currentfill}{rgb}{0.279566,0.067836,0.391917}%
\pgfsetfillcolor{currentfill}%
\pgfsetfillopacity{0.700000}%
\pgfsetlinewidth{0.000000pt}%
\definecolor{currentstroke}{rgb}{0.000000,0.000000,0.000000}%
\pgfsetstrokecolor{currentstroke}%
\pgfsetdash{}{0pt}%
\pgfpathmoveto{\pgfqpoint{3.967023in}{1.325715in}}%
\pgfpathlineto{\pgfqpoint{3.981264in}{1.324526in}}%
\pgfpathlineto{\pgfqpoint{3.995514in}{1.323408in}}%
\pgfpathlineto{\pgfqpoint{4.009773in}{1.322362in}}%
\pgfpathlineto{\pgfqpoint{4.024040in}{1.321386in}}%
\pgfpathlineto{\pgfqpoint{4.032303in}{1.332935in}}%
\pgfpathlineto{\pgfqpoint{4.040561in}{1.344545in}}%
\pgfpathlineto{\pgfqpoint{4.048814in}{1.356211in}}%
\pgfpathlineto{\pgfqpoint{4.057061in}{1.367928in}}%
\pgfpathlineto{\pgfqpoint{4.042803in}{1.368571in}}%
\pgfpathlineto{\pgfqpoint{4.028553in}{1.369284in}}%
\pgfpathlineto{\pgfqpoint{4.014313in}{1.370069in}}%
\pgfpathlineto{\pgfqpoint{4.000081in}{1.370925in}}%
\pgfpathlineto{\pgfqpoint{3.991825in}{1.359533in}}%
\pgfpathlineto{\pgfqpoint{3.983563in}{1.348198in}}%
\pgfpathlineto{\pgfqpoint{3.975296in}{1.336923in}}%
\pgfpathlineto{\pgfqpoint{3.967023in}{1.325715in}}%
\pgfpathclose%
\pgfusepath{fill}%
\end{pgfscope}%
\begin{pgfscope}%
\pgfpathrectangle{\pgfqpoint{1.150000in}{0.150000in}}{\pgfqpoint{5.700000in}{5.700000in}}%
\pgfusepath{clip}%
\pgfsetbuttcap%
\pgfsetroundjoin%
\definecolor{currentfill}{rgb}{0.282623,0.140926,0.457517}%
\pgfsetfillcolor{currentfill}%
\pgfsetfillopacity{0.700000}%
\pgfsetlinewidth{0.000000pt}%
\definecolor{currentstroke}{rgb}{0.000000,0.000000,0.000000}%
\pgfsetstrokecolor{currentstroke}%
\pgfsetdash{}{0pt}%
\pgfpathmoveto{\pgfqpoint{4.237113in}{1.465197in}}%
\pgfpathlineto{\pgfqpoint{4.251439in}{1.465797in}}%
\pgfpathlineto{\pgfqpoint{4.265775in}{1.466467in}}%
\pgfpathlineto{\pgfqpoint{4.280121in}{1.467208in}}%
\pgfpathlineto{\pgfqpoint{4.294476in}{1.468020in}}%
\pgfpathlineto{\pgfqpoint{4.302656in}{1.480832in}}%
\pgfpathlineto{\pgfqpoint{4.310832in}{1.493635in}}%
\pgfpathlineto{\pgfqpoint{4.319002in}{1.506426in}}%
\pgfpathlineto{\pgfqpoint{4.327167in}{1.519201in}}%
\pgfpathlineto{\pgfqpoint{4.312817in}{1.518117in}}%
\pgfpathlineto{\pgfqpoint{4.298477in}{1.517103in}}%
\pgfpathlineto{\pgfqpoint{4.284147in}{1.516160in}}%
\pgfpathlineto{\pgfqpoint{4.269827in}{1.515288in}}%
\pgfpathlineto{\pgfqpoint{4.261656in}{1.502778in}}%
\pgfpathlineto{\pgfqpoint{4.253480in}{1.490257in}}%
\pgfpathlineto{\pgfqpoint{4.245299in}{1.477729in}}%
\pgfpathlineto{\pgfqpoint{4.237113in}{1.465197in}}%
\pgfpathclose%
\pgfusepath{fill}%
\end{pgfscope}%
\begin{pgfscope}%
\pgfpathrectangle{\pgfqpoint{1.150000in}{0.150000in}}{\pgfqpoint{5.700000in}{5.700000in}}%
\pgfusepath{clip}%
\pgfsetbuttcap%
\pgfsetroundjoin%
\definecolor{currentfill}{rgb}{0.280255,0.165693,0.476498}%
\pgfsetfillcolor{currentfill}%
\pgfsetfillopacity{0.700000}%
\pgfsetlinewidth{0.000000pt}%
\definecolor{currentstroke}{rgb}{0.000000,0.000000,0.000000}%
\pgfsetstrokecolor{currentstroke}%
\pgfsetdash{}{0pt}%
\pgfpathmoveto{\pgfqpoint{4.327167in}{1.519201in}}%
\pgfpathlineto{\pgfqpoint{4.341527in}{1.520356in}}%
\pgfpathlineto{\pgfqpoint{4.355898in}{1.521582in}}%
\pgfpathlineto{\pgfqpoint{4.370278in}{1.522878in}}%
\pgfpathlineto{\pgfqpoint{4.384669in}{1.524245in}}%
\pgfpathlineto{\pgfqpoint{4.392825in}{1.537260in}}%
\pgfpathlineto{\pgfqpoint{4.400975in}{1.550247in}}%
\pgfpathlineto{\pgfqpoint{4.409120in}{1.563201in}}%
\pgfpathlineto{\pgfqpoint{4.417260in}{1.576119in}}%
\pgfpathlineto{\pgfqpoint{4.402874in}{1.574500in}}%
\pgfpathlineto{\pgfqpoint{4.388498in}{1.572951in}}%
\pgfpathlineto{\pgfqpoint{4.374133in}{1.571473in}}%
\pgfpathlineto{\pgfqpoint{4.359778in}{1.570066in}}%
\pgfpathlineto{\pgfqpoint{4.351633in}{1.557392in}}%
\pgfpathlineto{\pgfqpoint{4.343483in}{1.544688in}}%
\pgfpathlineto{\pgfqpoint{4.335327in}{1.531957in}}%
\pgfpathlineto{\pgfqpoint{4.327167in}{1.519201in}}%
\pgfpathclose%
\pgfusepath{fill}%
\end{pgfscope}%
\begin{pgfscope}%
\pgfpathrectangle{\pgfqpoint{1.150000in}{0.150000in}}{\pgfqpoint{5.700000in}{5.700000in}}%
\pgfusepath{clip}%
\pgfsetbuttcap%
\pgfsetroundjoin%
\definecolor{currentfill}{rgb}{0.276022,0.044167,0.370164}%
\pgfsetfillcolor{currentfill}%
\pgfsetfillopacity{0.700000}%
\pgfsetlinewidth{0.000000pt}%
\definecolor{currentstroke}{rgb}{0.000000,0.000000,0.000000}%
\pgfsetstrokecolor{currentstroke}%
\pgfsetdash{}{0pt}%
\pgfpathmoveto{\pgfqpoint{3.876944in}{1.288521in}}%
\pgfpathlineto{\pgfqpoint{3.891165in}{1.286695in}}%
\pgfpathlineto{\pgfqpoint{3.905393in}{1.284939in}}%
\pgfpathlineto{\pgfqpoint{3.919629in}{1.283255in}}%
\pgfpathlineto{\pgfqpoint{3.933873in}{1.281642in}}%
\pgfpathlineto{\pgfqpoint{3.942169in}{1.292536in}}%
\pgfpathlineto{\pgfqpoint{3.950460in}{1.303516in}}%
\pgfpathlineto{\pgfqpoint{3.958744in}{1.314577in}}%
\pgfpathlineto{\pgfqpoint{3.967023in}{1.325715in}}%
\pgfpathlineto{\pgfqpoint{3.952789in}{1.326975in}}%
\pgfpathlineto{\pgfqpoint{3.938564in}{1.328305in}}%
\pgfpathlineto{\pgfqpoint{3.924347in}{1.329708in}}%
\pgfpathlineto{\pgfqpoint{3.910138in}{1.331181in}}%
\pgfpathlineto{\pgfqpoint{3.901849in}{1.320389in}}%
\pgfpathlineto{\pgfqpoint{3.893553in}{1.309678in}}%
\pgfpathlineto{\pgfqpoint{3.885252in}{1.299054in}}%
\pgfpathlineto{\pgfqpoint{3.876944in}{1.288521in}}%
\pgfpathclose%
\pgfusepath{fill}%
\end{pgfscope}%
\begin{pgfscope}%
\pgfpathrectangle{\pgfqpoint{1.150000in}{0.150000in}}{\pgfqpoint{5.700000in}{5.700000in}}%
\pgfusepath{clip}%
\pgfsetbuttcap%
\pgfsetroundjoin%
\definecolor{currentfill}{rgb}{0.229739,0.322361,0.545706}%
\pgfsetfillcolor{currentfill}%
\pgfsetfillopacity{0.700000}%
\pgfsetlinewidth{0.000000pt}%
\definecolor{currentstroke}{rgb}{0.000000,0.000000,0.000000}%
\pgfsetstrokecolor{currentstroke}%
\pgfsetdash{}{0pt}%
\pgfpathmoveto{\pgfqpoint{4.810282in}{1.872986in}}%
\pgfpathlineto{\pgfqpoint{4.824842in}{1.876760in}}%
\pgfpathlineto{\pgfqpoint{4.839414in}{1.880605in}}%
\pgfpathlineto{\pgfqpoint{4.853999in}{1.884522in}}%
\pgfpathlineto{\pgfqpoint{4.868596in}{1.888510in}}%
\pgfpathlineto{\pgfqpoint{4.876596in}{1.900681in}}%
\pgfpathlineto{\pgfqpoint{4.884589in}{1.912733in}}%
\pgfpathlineto{\pgfqpoint{4.892576in}{1.924665in}}%
\pgfpathlineto{\pgfqpoint{4.900556in}{1.936476in}}%
\pgfpathlineto{\pgfqpoint{4.885963in}{1.932361in}}%
\pgfpathlineto{\pgfqpoint{4.871382in}{1.928318in}}%
\pgfpathlineto{\pgfqpoint{4.856814in}{1.924347in}}%
\pgfpathlineto{\pgfqpoint{4.842259in}{1.920446in}}%
\pgfpathlineto{\pgfqpoint{4.834274in}{1.908754in}}%
\pgfpathlineto{\pgfqpoint{4.826283in}{1.896946in}}%
\pgfpathlineto{\pgfqpoint{4.818286in}{1.885023in}}%
\pgfpathlineto{\pgfqpoint{4.810282in}{1.872986in}}%
\pgfpathclose%
\pgfusepath{fill}%
\end{pgfscope}%
\begin{pgfscope}%
\pgfpathrectangle{\pgfqpoint{1.150000in}{0.150000in}}{\pgfqpoint{5.700000in}{5.700000in}}%
\pgfusepath{clip}%
\pgfsetbuttcap%
\pgfsetroundjoin%
\definecolor{currentfill}{rgb}{0.268510,0.009605,0.335427}%
\pgfsetfillcolor{currentfill}%
\pgfsetfillopacity{0.700000}%
\pgfsetlinewidth{0.000000pt}%
\definecolor{currentstroke}{rgb}{0.000000,0.000000,0.000000}%
\pgfsetstrokecolor{currentstroke}%
\pgfsetdash{}{0pt}%
\pgfpathmoveto{\pgfqpoint{3.549487in}{1.229208in}}%
\pgfpathlineto{\pgfqpoint{3.563643in}{1.225047in}}%
\pgfpathlineto{\pgfqpoint{3.577805in}{1.220960in}}%
\pgfpathlineto{\pgfqpoint{3.591973in}{1.216945in}}%
\pgfpathlineto{\pgfqpoint{3.606147in}{1.213003in}}%
\pgfpathlineto{\pgfqpoint{3.614586in}{1.220725in}}%
\pgfpathlineto{\pgfqpoint{3.623016in}{1.228625in}}%
\pgfpathlineto{\pgfqpoint{3.631438in}{1.236696in}}%
\pgfpathlineto{\pgfqpoint{3.639852in}{1.244932in}}%
\pgfpathlineto{\pgfqpoint{3.625696in}{1.248461in}}%
\pgfpathlineto{\pgfqpoint{3.611547in}{1.252061in}}%
\pgfpathlineto{\pgfqpoint{3.597403in}{1.255735in}}%
\pgfpathlineto{\pgfqpoint{3.583266in}{1.259482in}}%
\pgfpathlineto{\pgfqpoint{3.574835in}{1.251652in}}%
\pgfpathlineto{\pgfqpoint{3.566394in}{1.243992in}}%
\pgfpathlineto{\pgfqpoint{3.557945in}{1.236509in}}%
\pgfpathlineto{\pgfqpoint{3.549487in}{1.229208in}}%
\pgfpathclose%
\pgfusepath{fill}%
\end{pgfscope}%
\begin{pgfscope}%
\pgfpathrectangle{\pgfqpoint{1.150000in}{0.150000in}}{\pgfqpoint{5.700000in}{5.700000in}}%
\pgfusepath{clip}%
\pgfsetbuttcap%
\pgfsetroundjoin%
\definecolor{currentfill}{rgb}{0.275191,0.194905,0.496005}%
\pgfsetfillcolor{currentfill}%
\pgfsetfillopacity{0.700000}%
\pgfsetlinewidth{0.000000pt}%
\definecolor{currentstroke}{rgb}{0.000000,0.000000,0.000000}%
\pgfsetstrokecolor{currentstroke}%
\pgfsetdash{}{0pt}%
\pgfpathmoveto{\pgfqpoint{4.417260in}{1.576119in}}%
\pgfpathlineto{\pgfqpoint{4.431657in}{1.577809in}}%
\pgfpathlineto{\pgfqpoint{4.446065in}{1.579569in}}%
\pgfpathlineto{\pgfqpoint{4.460483in}{1.581401in}}%
\pgfpathlineto{\pgfqpoint{4.474911in}{1.583302in}}%
\pgfpathlineto{\pgfqpoint{4.483042in}{1.596421in}}%
\pgfpathlineto{\pgfqpoint{4.491168in}{1.609492in}}%
\pgfpathlineto{\pgfqpoint{4.499289in}{1.622511in}}%
\pgfpathlineto{\pgfqpoint{4.507404in}{1.635476in}}%
\pgfpathlineto{\pgfqpoint{4.492979in}{1.633342in}}%
\pgfpathlineto{\pgfqpoint{4.478565in}{1.631279in}}%
\pgfpathlineto{\pgfqpoint{4.464162in}{1.629286in}}%
\pgfpathlineto{\pgfqpoint{4.449770in}{1.627365in}}%
\pgfpathlineto{\pgfqpoint{4.441650in}{1.614624in}}%
\pgfpathlineto{\pgfqpoint{4.433525in}{1.601834in}}%
\pgfpathlineto{\pgfqpoint{4.425395in}{1.588998in}}%
\pgfpathlineto{\pgfqpoint{4.417260in}{1.576119in}}%
\pgfpathclose%
\pgfusepath{fill}%
\end{pgfscope}%
\begin{pgfscope}%
\pgfpathrectangle{\pgfqpoint{1.150000in}{0.150000in}}{\pgfqpoint{5.700000in}{5.700000in}}%
\pgfusepath{clip}%
\pgfsetbuttcap%
\pgfsetroundjoin%
\definecolor{currentfill}{rgb}{0.273809,0.031497,0.358853}%
\pgfsetfillcolor{currentfill}%
\pgfsetfillopacity{0.700000}%
\pgfsetlinewidth{0.000000pt}%
\definecolor{currentstroke}{rgb}{0.000000,0.000000,0.000000}%
\pgfsetstrokecolor{currentstroke}%
\pgfsetdash{}{0pt}%
\pgfpathmoveto{\pgfqpoint{3.786796in}{1.256929in}}%
\pgfpathlineto{\pgfqpoint{3.800998in}{1.254443in}}%
\pgfpathlineto{\pgfqpoint{3.815209in}{1.252029in}}%
\pgfpathlineto{\pgfqpoint{3.829426in}{1.249687in}}%
\pgfpathlineto{\pgfqpoint{3.843651in}{1.247416in}}%
\pgfpathlineto{\pgfqpoint{3.851984in}{1.257527in}}%
\pgfpathlineto{\pgfqpoint{3.860310in}{1.267752in}}%
\pgfpathlineto{\pgfqpoint{3.868631in}{1.278085in}}%
\pgfpathlineto{\pgfqpoint{3.876944in}{1.288521in}}%
\pgfpathlineto{\pgfqpoint{3.862732in}{1.290419in}}%
\pgfpathlineto{\pgfqpoint{3.848527in}{1.292388in}}%
\pgfpathlineto{\pgfqpoint{3.834330in}{1.294429in}}%
\pgfpathlineto{\pgfqpoint{3.820140in}{1.296541in}}%
\pgfpathlineto{\pgfqpoint{3.811814in}{1.286471in}}%
\pgfpathlineto{\pgfqpoint{3.803481in}{1.276508in}}%
\pgfpathlineto{\pgfqpoint{3.795142in}{1.266659in}}%
\pgfpathlineto{\pgfqpoint{3.786796in}{1.256929in}}%
\pgfpathclose%
\pgfusepath{fill}%
\end{pgfscope}%
\begin{pgfscope}%
\pgfpathrectangle{\pgfqpoint{1.150000in}{0.150000in}}{\pgfqpoint{5.700000in}{5.700000in}}%
\pgfusepath{clip}%
\pgfsetbuttcap%
\pgfsetroundjoin%
\definecolor{currentfill}{rgb}{0.218130,0.347432,0.550038}%
\pgfsetfillcolor{currentfill}%
\pgfsetfillopacity{0.700000}%
\pgfsetlinewidth{0.000000pt}%
\definecolor{currentstroke}{rgb}{0.000000,0.000000,0.000000}%
\pgfsetstrokecolor{currentstroke}%
\pgfsetdash{}{0pt}%
\pgfpathmoveto{\pgfqpoint{4.900556in}{1.936476in}}%
\pgfpathlineto{\pgfqpoint{4.915161in}{1.940661in}}%
\pgfpathlineto{\pgfqpoint{4.929779in}{1.944918in}}%
\pgfpathlineto{\pgfqpoint{4.944410in}{1.949246in}}%
\pgfpathlineto{\pgfqpoint{4.959054in}{1.953645in}}%
\pgfpathlineto{\pgfqpoint{4.967022in}{1.965445in}}%
\pgfpathlineto{\pgfqpoint{4.974983in}{1.977115in}}%
\pgfpathlineto{\pgfqpoint{4.982937in}{1.988653in}}%
\pgfpathlineto{\pgfqpoint{4.990883in}{2.000059in}}%
\pgfpathlineto{\pgfqpoint{4.976244in}{1.995555in}}%
\pgfpathlineto{\pgfqpoint{4.961618in}{1.991122in}}%
\pgfpathlineto{\pgfqpoint{4.947004in}{1.986760in}}%
\pgfpathlineto{\pgfqpoint{4.932404in}{1.982470in}}%
\pgfpathlineto{\pgfqpoint{4.924452in}{1.971161in}}%
\pgfpathlineto{\pgfqpoint{4.916494in}{1.959725in}}%
\pgfpathlineto{\pgfqpoint{4.908528in}{1.948162in}}%
\pgfpathlineto{\pgfqpoint{4.900556in}{1.936476in}}%
\pgfpathclose%
\pgfusepath{fill}%
\end{pgfscope}%
\begin{pgfscope}%
\pgfpathrectangle{\pgfqpoint{1.150000in}{0.150000in}}{\pgfqpoint{5.700000in}{5.700000in}}%
\pgfusepath{clip}%
\pgfsetbuttcap%
\pgfsetroundjoin%
\definecolor{currentfill}{rgb}{0.267968,0.223549,0.512008}%
\pgfsetfillcolor{currentfill}%
\pgfsetfillopacity{0.700000}%
\pgfsetlinewidth{0.000000pt}%
\definecolor{currentstroke}{rgb}{0.000000,0.000000,0.000000}%
\pgfsetstrokecolor{currentstroke}%
\pgfsetdash{}{0pt}%
\pgfpathmoveto{\pgfqpoint{4.507404in}{1.635476in}}%
\pgfpathlineto{\pgfqpoint{4.521840in}{1.637680in}}%
\pgfpathlineto{\pgfqpoint{4.536286in}{1.639956in}}%
\pgfpathlineto{\pgfqpoint{4.550744in}{1.642301in}}%
\pgfpathlineto{\pgfqpoint{4.565213in}{1.644718in}}%
\pgfpathlineto{\pgfqpoint{4.573319in}{1.657845in}}%
\pgfpathlineto{\pgfqpoint{4.581420in}{1.670905in}}%
\pgfpathlineto{\pgfqpoint{4.589515in}{1.683897in}}%
\pgfpathlineto{\pgfqpoint{4.597605in}{1.696818in}}%
\pgfpathlineto{\pgfqpoint{4.583140in}{1.694190in}}%
\pgfpathlineto{\pgfqpoint{4.568686in}{1.691632in}}%
\pgfpathlineto{\pgfqpoint{4.554243in}{1.689146in}}%
\pgfpathlineto{\pgfqpoint{4.539811in}{1.686730in}}%
\pgfpathlineto{\pgfqpoint{4.531718in}{1.674013in}}%
\pgfpathlineto{\pgfqpoint{4.523618in}{1.661230in}}%
\pgfpathlineto{\pgfqpoint{4.515514in}{1.648383in}}%
\pgfpathlineto{\pgfqpoint{4.507404in}{1.635476in}}%
\pgfpathclose%
\pgfusepath{fill}%
\end{pgfscope}%
\begin{pgfscope}%
\pgfpathrectangle{\pgfqpoint{1.150000in}{0.150000in}}{\pgfqpoint{5.700000in}{5.700000in}}%
\pgfusepath{clip}%
\pgfsetbuttcap%
\pgfsetroundjoin%
\definecolor{currentfill}{rgb}{0.204903,0.375746,0.553533}%
\pgfsetfillcolor{currentfill}%
\pgfsetfillopacity{0.700000}%
\pgfsetlinewidth{0.000000pt}%
\definecolor{currentstroke}{rgb}{0.000000,0.000000,0.000000}%
\pgfsetstrokecolor{currentstroke}%
\pgfsetdash{}{0pt}%
\pgfpathmoveto{\pgfqpoint{4.990883in}{2.000059in}}%
\pgfpathlineto{\pgfqpoint{5.005535in}{2.004635in}}%
\pgfpathlineto{\pgfqpoint{5.020200in}{2.009282in}}%
\pgfpathlineto{\pgfqpoint{5.034879in}{2.014000in}}%
\pgfpathlineto{\pgfqpoint{5.049570in}{2.018790in}}%
\pgfpathlineto{\pgfqpoint{5.057504in}{2.030154in}}%
\pgfpathlineto{\pgfqpoint{5.065430in}{2.041377in}}%
\pgfpathlineto{\pgfqpoint{5.073348in}{2.052460in}}%
\pgfpathlineto{\pgfqpoint{5.081258in}{2.063401in}}%
\pgfpathlineto{\pgfqpoint{5.066572in}{2.058528in}}%
\pgfpathlineto{\pgfqpoint{5.051899in}{2.053727in}}%
\pgfpathlineto{\pgfqpoint{5.037240in}{2.048997in}}%
\pgfpathlineto{\pgfqpoint{5.022593in}{2.044338in}}%
\pgfpathlineto{\pgfqpoint{5.014677in}{2.033472in}}%
\pgfpathlineto{\pgfqpoint{5.006753in}{2.022469in}}%
\pgfpathlineto{\pgfqpoint{4.998822in}{2.011331in}}%
\pgfpathlineto{\pgfqpoint{4.990883in}{2.000059in}}%
\pgfpathclose%
\pgfusepath{fill}%
\end{pgfscope}%
\begin{pgfscope}%
\pgfpathrectangle{\pgfqpoint{1.150000in}{0.150000in}}{\pgfqpoint{5.700000in}{5.700000in}}%
\pgfusepath{clip}%
\pgfsetbuttcap%
\pgfsetroundjoin%
\definecolor{currentfill}{rgb}{0.271305,0.019942,0.347269}%
\pgfsetfillcolor{currentfill}%
\pgfsetfillopacity{0.700000}%
\pgfsetlinewidth{0.000000pt}%
\definecolor{currentstroke}{rgb}{0.000000,0.000000,0.000000}%
\pgfsetstrokecolor{currentstroke}%
\pgfsetdash{}{0pt}%
\pgfpathmoveto{\pgfqpoint{3.696543in}{1.231545in}}%
\pgfpathlineto{\pgfqpoint{3.710732in}{1.228379in}}%
\pgfpathlineto{\pgfqpoint{3.724928in}{1.225285in}}%
\pgfpathlineto{\pgfqpoint{3.739131in}{1.222262in}}%
\pgfpathlineto{\pgfqpoint{3.753341in}{1.219311in}}%
\pgfpathlineto{\pgfqpoint{3.761716in}{1.228508in}}%
\pgfpathlineto{\pgfqpoint{3.770083in}{1.237848in}}%
\pgfpathlineto{\pgfqpoint{3.778443in}{1.247323in}}%
\pgfpathlineto{\pgfqpoint{3.786796in}{1.256929in}}%
\pgfpathlineto{\pgfqpoint{3.772600in}{1.259486in}}%
\pgfpathlineto{\pgfqpoint{3.758412in}{1.262115in}}%
\pgfpathlineto{\pgfqpoint{3.744231in}{1.264816in}}%
\pgfpathlineto{\pgfqpoint{3.730057in}{1.267588in}}%
\pgfpathlineto{\pgfqpoint{3.721689in}{1.258368in}}%
\pgfpathlineto{\pgfqpoint{3.713315in}{1.249284in}}%
\pgfpathlineto{\pgfqpoint{3.704932in}{1.240341in}}%
\pgfpathlineto{\pgfqpoint{3.696543in}{1.231545in}}%
\pgfpathclose%
\pgfusepath{fill}%
\end{pgfscope}%
\begin{pgfscope}%
\pgfpathrectangle{\pgfqpoint{1.150000in}{0.150000in}}{\pgfqpoint{5.700000in}{5.700000in}}%
\pgfusepath{clip}%
\pgfsetbuttcap%
\pgfsetroundjoin%
\definecolor{currentfill}{rgb}{0.258965,0.251537,0.524736}%
\pgfsetfillcolor{currentfill}%
\pgfsetfillopacity{0.700000}%
\pgfsetlinewidth{0.000000pt}%
\definecolor{currentstroke}{rgb}{0.000000,0.000000,0.000000}%
\pgfsetstrokecolor{currentstroke}%
\pgfsetdash{}{0pt}%
\pgfpathmoveto{\pgfqpoint{4.597605in}{1.696818in}}%
\pgfpathlineto{\pgfqpoint{4.612082in}{1.699517in}}%
\pgfpathlineto{\pgfqpoint{4.626570in}{1.702286in}}%
\pgfpathlineto{\pgfqpoint{4.641070in}{1.705127in}}%
\pgfpathlineto{\pgfqpoint{4.655581in}{1.708038in}}%
\pgfpathlineto{\pgfqpoint{4.663662in}{1.721082in}}%
\pgfpathlineto{\pgfqpoint{4.671737in}{1.734044in}}%
\pgfpathlineto{\pgfqpoint{4.679807in}{1.746921in}}%
\pgfpathlineto{\pgfqpoint{4.687870in}{1.759711in}}%
\pgfpathlineto{\pgfqpoint{4.673362in}{1.756609in}}%
\pgfpathlineto{\pgfqpoint{4.658866in}{1.753578in}}%
\pgfpathlineto{\pgfqpoint{4.644382in}{1.750618in}}%
\pgfpathlineto{\pgfqpoint{4.629909in}{1.747729in}}%
\pgfpathlineto{\pgfqpoint{4.621842in}{1.735122in}}%
\pgfpathlineto{\pgfqpoint{4.613768in}{1.722433in}}%
\pgfpathlineto{\pgfqpoint{4.605690in}{1.709664in}}%
\pgfpathlineto{\pgfqpoint{4.597605in}{1.696818in}}%
\pgfpathclose%
\pgfusepath{fill}%
\end{pgfscope}%
\begin{pgfscope}%
\pgfpathrectangle{\pgfqpoint{1.150000in}{0.150000in}}{\pgfqpoint{5.700000in}{5.700000in}}%
\pgfusepath{clip}%
\pgfsetbuttcap%
\pgfsetroundjoin%
\definecolor{currentfill}{rgb}{0.194100,0.399323,0.555565}%
\pgfsetfillcolor{currentfill}%
\pgfsetfillopacity{0.700000}%
\pgfsetlinewidth{0.000000pt}%
\definecolor{currentstroke}{rgb}{0.000000,0.000000,0.000000}%
\pgfsetstrokecolor{currentstroke}%
\pgfsetdash{}{0pt}%
\pgfpathmoveto{\pgfqpoint{5.081258in}{2.063401in}}%
\pgfpathlineto{\pgfqpoint{5.095958in}{2.068346in}}%
\pgfpathlineto{\pgfqpoint{5.110670in}{2.073362in}}%
\pgfpathlineto{\pgfqpoint{5.125397in}{2.078450in}}%
\pgfpathlineto{\pgfqpoint{5.140136in}{2.083609in}}%
\pgfpathlineto{\pgfqpoint{5.148033in}{2.094477in}}%
\pgfpathlineto{\pgfqpoint{5.155921in}{2.105196in}}%
\pgfpathlineto{\pgfqpoint{5.163802in}{2.115766in}}%
\pgfpathlineto{\pgfqpoint{5.171673in}{2.126187in}}%
\pgfpathlineto{\pgfqpoint{5.156940in}{2.120966in}}%
\pgfpathlineto{\pgfqpoint{5.142220in}{2.115818in}}%
\pgfpathlineto{\pgfqpoint{5.127513in}{2.110741in}}%
\pgfpathlineto{\pgfqpoint{5.112820in}{2.105735in}}%
\pgfpathlineto{\pgfqpoint{5.104942in}{2.095367in}}%
\pgfpathlineto{\pgfqpoint{5.097055in}{2.084855in}}%
\pgfpathlineto{\pgfqpoint{5.089161in}{2.074200in}}%
\pgfpathlineto{\pgfqpoint{5.081258in}{2.063401in}}%
\pgfpathclose%
\pgfusepath{fill}%
\end{pgfscope}%
\begin{pgfscope}%
\pgfpathrectangle{\pgfqpoint{1.150000in}{0.150000in}}{\pgfqpoint{5.700000in}{5.700000in}}%
\pgfusepath{clip}%
\pgfsetbuttcap%
\pgfsetroundjoin%
\definecolor{currentfill}{rgb}{0.246811,0.283237,0.535941}%
\pgfsetfillcolor{currentfill}%
\pgfsetfillopacity{0.700000}%
\pgfsetlinewidth{0.000000pt}%
\definecolor{currentstroke}{rgb}{0.000000,0.000000,0.000000}%
\pgfsetstrokecolor{currentstroke}%
\pgfsetdash{}{0pt}%
\pgfpathmoveto{\pgfqpoint{4.687870in}{1.759711in}}%
\pgfpathlineto{\pgfqpoint{4.702390in}{1.762883in}}%
\pgfpathlineto{\pgfqpoint{4.716921in}{1.766127in}}%
\pgfpathlineto{\pgfqpoint{4.731464in}{1.769442in}}%
\pgfpathlineto{\pgfqpoint{4.746019in}{1.772827in}}%
\pgfpathlineto{\pgfqpoint{4.754074in}{1.785704in}}%
\pgfpathlineto{\pgfqpoint{4.762122in}{1.798483in}}%
\pgfpathlineto{\pgfqpoint{4.770164in}{1.811163in}}%
\pgfpathlineto{\pgfqpoint{4.778200in}{1.823740in}}%
\pgfpathlineto{\pgfqpoint{4.763649in}{1.820185in}}%
\pgfpathlineto{\pgfqpoint{4.749109in}{1.816700in}}%
\pgfpathlineto{\pgfqpoint{4.734581in}{1.813287in}}%
\pgfpathlineto{\pgfqpoint{4.720066in}{1.809945in}}%
\pgfpathlineto{\pgfqpoint{4.712026in}{1.797530in}}%
\pgfpathlineto{\pgfqpoint{4.703980in}{1.785017in}}%
\pgfpathlineto{\pgfqpoint{4.695928in}{1.772410in}}%
\pgfpathlineto{\pgfqpoint{4.687870in}{1.759711in}}%
\pgfpathclose%
\pgfusepath{fill}%
\end{pgfscope}%
\begin{pgfscope}%
\pgfpathrectangle{\pgfqpoint{1.150000in}{0.150000in}}{\pgfqpoint{5.700000in}{5.700000in}}%
\pgfusepath{clip}%
\pgfsetbuttcap%
\pgfsetroundjoin%
\definecolor{currentfill}{rgb}{0.282656,0.100196,0.422160}%
\pgfsetfillcolor{currentfill}%
\pgfsetfillopacity{0.700000}%
\pgfsetlinewidth{0.000000pt}%
\definecolor{currentstroke}{rgb}{0.000000,0.000000,0.000000}%
\pgfsetstrokecolor{currentstroke}%
\pgfsetdash{}{0pt}%
\pgfpathmoveto{\pgfqpoint{4.114180in}{1.366065in}}%
\pgfpathlineto{\pgfqpoint{4.128482in}{1.365775in}}%
\pgfpathlineto{\pgfqpoint{4.142794in}{1.365557in}}%
\pgfpathlineto{\pgfqpoint{4.157114in}{1.365408in}}%
\pgfpathlineto{\pgfqpoint{4.171444in}{1.365330in}}%
\pgfpathlineto{\pgfqpoint{4.179670in}{1.377736in}}%
\pgfpathlineto{\pgfqpoint{4.187891in}{1.390173in}}%
\pgfpathlineto{\pgfqpoint{4.196108in}{1.402636in}}%
\pgfpathlineto{\pgfqpoint{4.204319in}{1.415121in}}%
\pgfpathlineto{\pgfqpoint{4.189996in}{1.414885in}}%
\pgfpathlineto{\pgfqpoint{4.175682in}{1.414720in}}%
\pgfpathlineto{\pgfqpoint{4.161378in}{1.414625in}}%
\pgfpathlineto{\pgfqpoint{4.147083in}{1.414601in}}%
\pgfpathlineto{\pgfqpoint{4.138865in}{1.402421in}}%
\pgfpathlineto{\pgfqpoint{4.130642in}{1.390269in}}%
\pgfpathlineto{\pgfqpoint{4.122414in}{1.378149in}}%
\pgfpathlineto{\pgfqpoint{4.114180in}{1.366065in}}%
\pgfpathclose%
\pgfusepath{fill}%
\end{pgfscope}%
\begin{pgfscope}%
\pgfpathrectangle{\pgfqpoint{1.150000in}{0.150000in}}{\pgfqpoint{5.700000in}{5.700000in}}%
\pgfusepath{clip}%
\pgfsetbuttcap%
\pgfsetroundjoin%
\definecolor{currentfill}{rgb}{0.280894,0.078907,0.402329}%
\pgfsetfillcolor{currentfill}%
\pgfsetfillopacity{0.700000}%
\pgfsetlinewidth{0.000000pt}%
\definecolor{currentstroke}{rgb}{0.000000,0.000000,0.000000}%
\pgfsetstrokecolor{currentstroke}%
\pgfsetdash{}{0pt}%
\pgfpathmoveto{\pgfqpoint{4.024040in}{1.321386in}}%
\pgfpathlineto{\pgfqpoint{4.038315in}{1.320480in}}%
\pgfpathlineto{\pgfqpoint{4.052599in}{1.319645in}}%
\pgfpathlineto{\pgfqpoint{4.066892in}{1.318881in}}%
\pgfpathlineto{\pgfqpoint{4.081193in}{1.318187in}}%
\pgfpathlineto{\pgfqpoint{4.089448in}{1.330078in}}%
\pgfpathlineto{\pgfqpoint{4.097697in}{1.342025in}}%
\pgfpathlineto{\pgfqpoint{4.105941in}{1.354022in}}%
\pgfpathlineto{\pgfqpoint{4.114180in}{1.366065in}}%
\pgfpathlineto{\pgfqpoint{4.099887in}{1.366424in}}%
\pgfpathlineto{\pgfqpoint{4.085603in}{1.366855in}}%
\pgfpathlineto{\pgfqpoint{4.071327in}{1.367356in}}%
\pgfpathlineto{\pgfqpoint{4.057061in}{1.367928in}}%
\pgfpathlineto{\pgfqpoint{4.048814in}{1.356211in}}%
\pgfpathlineto{\pgfqpoint{4.040561in}{1.344545in}}%
\pgfpathlineto{\pgfqpoint{4.032303in}{1.332935in}}%
\pgfpathlineto{\pgfqpoint{4.024040in}{1.321386in}}%
\pgfpathclose%
\pgfusepath{fill}%
\end{pgfscope}%
\begin{pgfscope}%
\pgfpathrectangle{\pgfqpoint{1.150000in}{0.150000in}}{\pgfqpoint{5.700000in}{5.700000in}}%
\pgfusepath{clip}%
\pgfsetbuttcap%
\pgfsetroundjoin%
\definecolor{currentfill}{rgb}{0.283187,0.125848,0.444960}%
\pgfsetfillcolor{currentfill}%
\pgfsetfillopacity{0.700000}%
\pgfsetlinewidth{0.000000pt}%
\definecolor{currentstroke}{rgb}{0.000000,0.000000,0.000000}%
\pgfsetstrokecolor{currentstroke}%
\pgfsetdash{}{0pt}%
\pgfpathmoveto{\pgfqpoint{4.204319in}{1.415121in}}%
\pgfpathlineto{\pgfqpoint{4.218651in}{1.415428in}}%
\pgfpathlineto{\pgfqpoint{4.232993in}{1.415805in}}%
\pgfpathlineto{\pgfqpoint{4.247344in}{1.416252in}}%
\pgfpathlineto{\pgfqpoint{4.261706in}{1.416769in}}%
\pgfpathlineto{\pgfqpoint{4.269906in}{1.429574in}}%
\pgfpathlineto{\pgfqpoint{4.278101in}{1.442387in}}%
\pgfpathlineto{\pgfqpoint{4.286291in}{1.455203in}}%
\pgfpathlineto{\pgfqpoint{4.294476in}{1.468020in}}%
\pgfpathlineto{\pgfqpoint{4.280121in}{1.467208in}}%
\pgfpathlineto{\pgfqpoint{4.265775in}{1.466467in}}%
\pgfpathlineto{\pgfqpoint{4.251439in}{1.465797in}}%
\pgfpathlineto{\pgfqpoint{4.237113in}{1.465197in}}%
\pgfpathlineto{\pgfqpoint{4.228922in}{1.452667in}}%
\pgfpathlineto{\pgfqpoint{4.220726in}{1.440141in}}%
\pgfpathlineto{\pgfqpoint{4.212525in}{1.427625in}}%
\pgfpathlineto{\pgfqpoint{4.204319in}{1.415121in}}%
\pgfpathclose%
\pgfusepath{fill}%
\end{pgfscope}%
\begin{pgfscope}%
\pgfpathrectangle{\pgfqpoint{1.150000in}{0.150000in}}{\pgfqpoint{5.700000in}{5.700000in}}%
\pgfusepath{clip}%
\pgfsetbuttcap%
\pgfsetroundjoin%
\definecolor{currentfill}{rgb}{0.182256,0.426184,0.557120}%
\pgfsetfillcolor{currentfill}%
\pgfsetfillopacity{0.700000}%
\pgfsetlinewidth{0.000000pt}%
\definecolor{currentstroke}{rgb}{0.000000,0.000000,0.000000}%
\pgfsetstrokecolor{currentstroke}%
\pgfsetdash{}{0pt}%
\pgfpathmoveto{\pgfqpoint{5.171673in}{2.126187in}}%
\pgfpathlineto{\pgfqpoint{5.186420in}{2.131479in}}%
\pgfpathlineto{\pgfqpoint{5.201181in}{2.136843in}}%
\pgfpathlineto{\pgfqpoint{5.215956in}{2.142279in}}%
\pgfpathlineto{\pgfqpoint{5.230745in}{2.147786in}}%
\pgfpathlineto{\pgfqpoint{5.238601in}{2.158104in}}%
\pgfpathlineto{\pgfqpoint{5.246449in}{2.168266in}}%
\pgfpathlineto{\pgfqpoint{5.254288in}{2.178272in}}%
\pgfpathlineto{\pgfqpoint{5.262118in}{2.188122in}}%
\pgfpathlineto{\pgfqpoint{5.247337in}{2.182576in}}%
\pgfpathlineto{\pgfqpoint{5.232569in}{2.177101in}}%
\pgfpathlineto{\pgfqpoint{5.217815in}{2.171699in}}%
\pgfpathlineto{\pgfqpoint{5.203075in}{2.166368in}}%
\pgfpathlineto{\pgfqpoint{5.195237in}{2.156548in}}%
\pgfpathlineto{\pgfqpoint{5.187391in}{2.146578in}}%
\pgfpathlineto{\pgfqpoint{5.179536in}{2.136458in}}%
\pgfpathlineto{\pgfqpoint{5.171673in}{2.126187in}}%
\pgfpathclose%
\pgfusepath{fill}%
\end{pgfscope}%
\begin{pgfscope}%
\pgfpathrectangle{\pgfqpoint{1.150000in}{0.150000in}}{\pgfqpoint{5.700000in}{5.700000in}}%
\pgfusepath{clip}%
\pgfsetbuttcap%
\pgfsetroundjoin%
\definecolor{currentfill}{rgb}{0.269944,0.014625,0.341379}%
\pgfsetfillcolor{currentfill}%
\pgfsetfillopacity{0.700000}%
\pgfsetlinewidth{0.000000pt}%
\definecolor{currentstroke}{rgb}{0.000000,0.000000,0.000000}%
\pgfsetstrokecolor{currentstroke}%
\pgfsetdash{}{0pt}%
\pgfpathmoveto{\pgfqpoint{3.606147in}{1.213003in}}%
\pgfpathlineto{\pgfqpoint{3.620327in}{1.209134in}}%
\pgfpathlineto{\pgfqpoint{3.634514in}{1.205337in}}%
\pgfpathlineto{\pgfqpoint{3.648707in}{1.201612in}}%
\pgfpathlineto{\pgfqpoint{3.662906in}{1.197959in}}%
\pgfpathlineto{\pgfqpoint{3.671327in}{1.206104in}}%
\pgfpathlineto{\pgfqpoint{3.679740in}{1.214421in}}%
\pgfpathlineto{\pgfqpoint{3.688145in}{1.222903in}}%
\pgfpathlineto{\pgfqpoint{3.696543in}{1.231545in}}%
\pgfpathlineto{\pgfqpoint{3.682360in}{1.234784in}}%
\pgfpathlineto{\pgfqpoint{3.668184in}{1.238094in}}%
\pgfpathlineto{\pgfqpoint{3.654015in}{1.241477in}}%
\pgfpathlineto{\pgfqpoint{3.639852in}{1.244932in}}%
\pgfpathlineto{\pgfqpoint{3.631438in}{1.236696in}}%
\pgfpathlineto{\pgfqpoint{3.623016in}{1.228625in}}%
\pgfpathlineto{\pgfqpoint{3.614586in}{1.220725in}}%
\pgfpathlineto{\pgfqpoint{3.606147in}{1.213003in}}%
\pgfpathclose%
\pgfusepath{fill}%
\end{pgfscope}%
\begin{pgfscope}%
\pgfpathrectangle{\pgfqpoint{1.150000in}{0.150000in}}{\pgfqpoint{5.700000in}{5.700000in}}%
\pgfusepath{clip}%
\pgfsetbuttcap%
\pgfsetroundjoin%
\definecolor{currentfill}{rgb}{0.277941,0.056324,0.381191}%
\pgfsetfillcolor{currentfill}%
\pgfsetfillopacity{0.700000}%
\pgfsetlinewidth{0.000000pt}%
\definecolor{currentstroke}{rgb}{0.000000,0.000000,0.000000}%
\pgfsetstrokecolor{currentstroke}%
\pgfsetdash{}{0pt}%
\pgfpathmoveto{\pgfqpoint{3.933873in}{1.281642in}}%
\pgfpathlineto{\pgfqpoint{3.948125in}{1.280100in}}%
\pgfpathlineto{\pgfqpoint{3.962385in}{1.278629in}}%
\pgfpathlineto{\pgfqpoint{3.976654in}{1.277228in}}%
\pgfpathlineto{\pgfqpoint{3.990930in}{1.275898in}}%
\pgfpathlineto{\pgfqpoint{3.999216in}{1.287153in}}%
\pgfpathlineto{\pgfqpoint{4.007496in}{1.298489in}}%
\pgfpathlineto{\pgfqpoint{4.015771in}{1.309902in}}%
\pgfpathlineto{\pgfqpoint{4.024040in}{1.321386in}}%
\pgfpathlineto{\pgfqpoint{4.009773in}{1.322362in}}%
\pgfpathlineto{\pgfqpoint{3.995514in}{1.323408in}}%
\pgfpathlineto{\pgfqpoint{3.981264in}{1.324526in}}%
\pgfpathlineto{\pgfqpoint{3.967023in}{1.325715in}}%
\pgfpathlineto{\pgfqpoint{3.958744in}{1.314577in}}%
\pgfpathlineto{\pgfqpoint{3.950460in}{1.303516in}}%
\pgfpathlineto{\pgfqpoint{3.942169in}{1.292536in}}%
\pgfpathlineto{\pgfqpoint{3.933873in}{1.281642in}}%
\pgfpathclose%
\pgfusepath{fill}%
\end{pgfscope}%
\begin{pgfscope}%
\pgfpathrectangle{\pgfqpoint{1.150000in}{0.150000in}}{\pgfqpoint{5.700000in}{5.700000in}}%
\pgfusepath{clip}%
\pgfsetbuttcap%
\pgfsetroundjoin%
\definecolor{currentfill}{rgb}{0.174274,0.445044,0.557792}%
\pgfsetfillcolor{currentfill}%
\pgfsetfillopacity{0.700000}%
\pgfsetlinewidth{0.000000pt}%
\definecolor{currentstroke}{rgb}{0.000000,0.000000,0.000000}%
\pgfsetstrokecolor{currentstroke}%
\pgfsetdash{}{0pt}%
\pgfpathmoveto{\pgfqpoint{5.262118in}{2.188122in}}%
\pgfpathlineto{\pgfqpoint{5.276914in}{2.193740in}}%
\pgfpathlineto{\pgfqpoint{5.291723in}{2.199430in}}%
\pgfpathlineto{\pgfqpoint{5.306547in}{2.205192in}}%
\pgfpathlineto{\pgfqpoint{5.314362in}{2.214904in}}%
\pgfpathlineto{\pgfqpoint{5.322168in}{2.224456in}}%
\pgfpathlineto{\pgfqpoint{5.329965in}{2.233849in}}%
\pgfpathlineto{\pgfqpoint{5.337752in}{2.243081in}}%
\pgfpathlineto{\pgfqpoint{5.322937in}{2.237303in}}%
\pgfpathlineto{\pgfqpoint{5.308135in}{2.231597in}}%
\pgfpathlineto{\pgfqpoint{5.293348in}{2.225963in}}%
\pgfpathlineto{\pgfqpoint{5.285554in}{2.216736in}}%
\pgfpathlineto{\pgfqpoint{5.277751in}{2.207354in}}%
\pgfpathlineto{\pgfqpoint{5.269939in}{2.197816in}}%
\pgfpathlineto{\pgfqpoint{5.262118in}{2.188122in}}%
\pgfpathclose%
\pgfusepath{fill}%
\end{pgfscope}%
\begin{pgfscope}%
\pgfpathrectangle{\pgfqpoint{1.150000in}{0.150000in}}{\pgfqpoint{5.700000in}{5.700000in}}%
\pgfusepath{clip}%
\pgfsetbuttcap%
\pgfsetroundjoin%
\definecolor{currentfill}{rgb}{0.281887,0.150881,0.465405}%
\pgfsetfillcolor{currentfill}%
\pgfsetfillopacity{0.700000}%
\pgfsetlinewidth{0.000000pt}%
\definecolor{currentstroke}{rgb}{0.000000,0.000000,0.000000}%
\pgfsetstrokecolor{currentstroke}%
\pgfsetdash{}{0pt}%
\pgfpathmoveto{\pgfqpoint{4.294476in}{1.468020in}}%
\pgfpathlineto{\pgfqpoint{4.308842in}{1.468902in}}%
\pgfpathlineto{\pgfqpoint{4.323217in}{1.469854in}}%
\pgfpathlineto{\pgfqpoint{4.337602in}{1.470876in}}%
\pgfpathlineto{\pgfqpoint{4.351998in}{1.471969in}}%
\pgfpathlineto{\pgfqpoint{4.360173in}{1.485063in}}%
\pgfpathlineto{\pgfqpoint{4.368344in}{1.498142in}}%
\pgfpathlineto{\pgfqpoint{4.376509in}{1.511204in}}%
\pgfpathlineto{\pgfqpoint{4.384669in}{1.524245in}}%
\pgfpathlineto{\pgfqpoint{4.370278in}{1.522878in}}%
\pgfpathlineto{\pgfqpoint{4.355898in}{1.521582in}}%
\pgfpathlineto{\pgfqpoint{4.341527in}{1.520356in}}%
\pgfpathlineto{\pgfqpoint{4.327167in}{1.519201in}}%
\pgfpathlineto{\pgfqpoint{4.319002in}{1.506426in}}%
\pgfpathlineto{\pgfqpoint{4.310832in}{1.493635in}}%
\pgfpathlineto{\pgfqpoint{4.302656in}{1.480832in}}%
\pgfpathlineto{\pgfqpoint{4.294476in}{1.468020in}}%
\pgfpathclose%
\pgfusepath{fill}%
\end{pgfscope}%
\begin{pgfscope}%
\pgfpathrectangle{\pgfqpoint{1.150000in}{0.150000in}}{\pgfqpoint{5.700000in}{5.700000in}}%
\pgfusepath{clip}%
\pgfsetbuttcap%
\pgfsetroundjoin%
\definecolor{currentfill}{rgb}{0.233603,0.313828,0.543914}%
\pgfsetfillcolor{currentfill}%
\pgfsetfillopacity{0.700000}%
\pgfsetlinewidth{0.000000pt}%
\definecolor{currentstroke}{rgb}{0.000000,0.000000,0.000000}%
\pgfsetstrokecolor{currentstroke}%
\pgfsetdash{}{0pt}%
\pgfpathmoveto{\pgfqpoint{4.778200in}{1.823740in}}%
\pgfpathlineto{\pgfqpoint{4.792764in}{1.827366in}}%
\pgfpathlineto{\pgfqpoint{4.807341in}{1.831063in}}%
\pgfpathlineto{\pgfqpoint{4.821929in}{1.834831in}}%
\pgfpathlineto{\pgfqpoint{4.836530in}{1.838670in}}%
\pgfpathlineto{\pgfqpoint{4.844556in}{1.851300in}}%
\pgfpathlineto{\pgfqpoint{4.852576in}{1.863817in}}%
\pgfpathlineto{\pgfqpoint{4.860589in}{1.876221in}}%
\pgfpathlineto{\pgfqpoint{4.868596in}{1.888510in}}%
\pgfpathlineto{\pgfqpoint{4.853999in}{1.884522in}}%
\pgfpathlineto{\pgfqpoint{4.839414in}{1.880605in}}%
\pgfpathlineto{\pgfqpoint{4.824842in}{1.876760in}}%
\pgfpathlineto{\pgfqpoint{4.810282in}{1.872986in}}%
\pgfpathlineto{\pgfqpoint{4.802271in}{1.860838in}}%
\pgfpathlineto{\pgfqpoint{4.794254in}{1.848579in}}%
\pgfpathlineto{\pgfqpoint{4.786230in}{1.836212in}}%
\pgfpathlineto{\pgfqpoint{4.778200in}{1.823740in}}%
\pgfpathclose%
\pgfusepath{fill}%
\end{pgfscope}%
\begin{pgfscope}%
\pgfpathrectangle{\pgfqpoint{1.150000in}{0.150000in}}{\pgfqpoint{5.700000in}{5.700000in}}%
\pgfusepath{clip}%
\pgfsetbuttcap%
\pgfsetroundjoin%
\definecolor{currentfill}{rgb}{0.278012,0.180367,0.486697}%
\pgfsetfillcolor{currentfill}%
\pgfsetfillopacity{0.700000}%
\pgfsetlinewidth{0.000000pt}%
\definecolor{currentstroke}{rgb}{0.000000,0.000000,0.000000}%
\pgfsetstrokecolor{currentstroke}%
\pgfsetdash{}{0pt}%
\pgfpathmoveto{\pgfqpoint{4.384669in}{1.524245in}}%
\pgfpathlineto{\pgfqpoint{4.399070in}{1.525682in}}%
\pgfpathlineto{\pgfqpoint{4.413482in}{1.527189in}}%
\pgfpathlineto{\pgfqpoint{4.427904in}{1.528767in}}%
\pgfpathlineto{\pgfqpoint{4.442337in}{1.530416in}}%
\pgfpathlineto{\pgfqpoint{4.450488in}{1.543692in}}%
\pgfpathlineto{\pgfqpoint{4.458634in}{1.556934in}}%
\pgfpathlineto{\pgfqpoint{4.466775in}{1.570139in}}%
\pgfpathlineto{\pgfqpoint{4.474911in}{1.583302in}}%
\pgfpathlineto{\pgfqpoint{4.460483in}{1.581401in}}%
\pgfpathlineto{\pgfqpoint{4.446065in}{1.579569in}}%
\pgfpathlineto{\pgfqpoint{4.431657in}{1.577809in}}%
\pgfpathlineto{\pgfqpoint{4.417260in}{1.576119in}}%
\pgfpathlineto{\pgfqpoint{4.409120in}{1.563201in}}%
\pgfpathlineto{\pgfqpoint{4.400975in}{1.550247in}}%
\pgfpathlineto{\pgfqpoint{4.392825in}{1.537260in}}%
\pgfpathlineto{\pgfqpoint{4.384669in}{1.524245in}}%
\pgfpathclose%
\pgfusepath{fill}%
\end{pgfscope}%
\begin{pgfscope}%
\pgfpathrectangle{\pgfqpoint{1.150000in}{0.150000in}}{\pgfqpoint{5.700000in}{5.700000in}}%
\pgfusepath{clip}%
\pgfsetbuttcap%
\pgfsetroundjoin%
\definecolor{currentfill}{rgb}{0.274952,0.037752,0.364543}%
\pgfsetfillcolor{currentfill}%
\pgfsetfillopacity{0.700000}%
\pgfsetlinewidth{0.000000pt}%
\definecolor{currentstroke}{rgb}{0.000000,0.000000,0.000000}%
\pgfsetstrokecolor{currentstroke}%
\pgfsetdash{}{0pt}%
\pgfpathmoveto{\pgfqpoint{3.843651in}{1.247416in}}%
\pgfpathlineto{\pgfqpoint{3.857884in}{1.245216in}}%
\pgfpathlineto{\pgfqpoint{3.872124in}{1.243087in}}%
\pgfpathlineto{\pgfqpoint{3.886372in}{1.241029in}}%
\pgfpathlineto{\pgfqpoint{3.900627in}{1.239041in}}%
\pgfpathlineto{\pgfqpoint{3.908948in}{1.249535in}}%
\pgfpathlineto{\pgfqpoint{3.917262in}{1.260136in}}%
\pgfpathlineto{\pgfqpoint{3.925571in}{1.270840in}}%
\pgfpathlineto{\pgfqpoint{3.933873in}{1.281642in}}%
\pgfpathlineto{\pgfqpoint{3.919629in}{1.283255in}}%
\pgfpathlineto{\pgfqpoint{3.905393in}{1.284939in}}%
\pgfpathlineto{\pgfqpoint{3.891165in}{1.286695in}}%
\pgfpathlineto{\pgfqpoint{3.876944in}{1.288521in}}%
\pgfpathlineto{\pgfqpoint{3.868631in}{1.278085in}}%
\pgfpathlineto{\pgfqpoint{3.860310in}{1.267752in}}%
\pgfpathlineto{\pgfqpoint{3.851984in}{1.257527in}}%
\pgfpathlineto{\pgfqpoint{3.843651in}{1.247416in}}%
\pgfpathclose%
\pgfusepath{fill}%
\end{pgfscope}%
\begin{pgfscope}%
\pgfpathrectangle{\pgfqpoint{1.150000in}{0.150000in}}{\pgfqpoint{5.700000in}{5.700000in}}%
\pgfusepath{clip}%
\pgfsetbuttcap%
\pgfsetroundjoin%
\definecolor{currentfill}{rgb}{0.221989,0.339161,0.548752}%
\pgfsetfillcolor{currentfill}%
\pgfsetfillopacity{0.700000}%
\pgfsetlinewidth{0.000000pt}%
\definecolor{currentstroke}{rgb}{0.000000,0.000000,0.000000}%
\pgfsetstrokecolor{currentstroke}%
\pgfsetdash{}{0pt}%
\pgfpathmoveto{\pgfqpoint{4.868596in}{1.888510in}}%
\pgfpathlineto{\pgfqpoint{4.883206in}{1.892568in}}%
\pgfpathlineto{\pgfqpoint{4.897828in}{1.896699in}}%
\pgfpathlineto{\pgfqpoint{4.912463in}{1.900900in}}%
\pgfpathlineto{\pgfqpoint{4.927111in}{1.905172in}}%
\pgfpathlineto{\pgfqpoint{4.935107in}{1.917479in}}%
\pgfpathlineto{\pgfqpoint{4.943096in}{1.929660in}}%
\pgfpathlineto{\pgfqpoint{4.951079in}{1.941717in}}%
\pgfpathlineto{\pgfqpoint{4.959054in}{1.953645in}}%
\pgfpathlineto{\pgfqpoint{4.944410in}{1.949246in}}%
\pgfpathlineto{\pgfqpoint{4.929779in}{1.944918in}}%
\pgfpathlineto{\pgfqpoint{4.915161in}{1.940661in}}%
\pgfpathlineto{\pgfqpoint{4.900556in}{1.936476in}}%
\pgfpathlineto{\pgfqpoint{4.892576in}{1.924665in}}%
\pgfpathlineto{\pgfqpoint{4.884589in}{1.912733in}}%
\pgfpathlineto{\pgfqpoint{4.876596in}{1.900681in}}%
\pgfpathlineto{\pgfqpoint{4.868596in}{1.888510in}}%
\pgfpathclose%
\pgfusepath{fill}%
\end{pgfscope}%
\begin{pgfscope}%
\pgfpathrectangle{\pgfqpoint{1.150000in}{0.150000in}}{\pgfqpoint{5.700000in}{5.700000in}}%
\pgfusepath{clip}%
\pgfsetbuttcap%
\pgfsetroundjoin%
\definecolor{currentfill}{rgb}{0.271828,0.209303,0.504434}%
\pgfsetfillcolor{currentfill}%
\pgfsetfillopacity{0.700000}%
\pgfsetlinewidth{0.000000pt}%
\definecolor{currentstroke}{rgb}{0.000000,0.000000,0.000000}%
\pgfsetstrokecolor{currentstroke}%
\pgfsetdash{}{0pt}%
\pgfpathmoveto{\pgfqpoint{4.474911in}{1.583302in}}%
\pgfpathlineto{\pgfqpoint{4.489351in}{1.585274in}}%
\pgfpathlineto{\pgfqpoint{4.503801in}{1.587317in}}%
\pgfpathlineto{\pgfqpoint{4.518263in}{1.589430in}}%
\pgfpathlineto{\pgfqpoint{4.532735in}{1.591614in}}%
\pgfpathlineto{\pgfqpoint{4.540862in}{1.604973in}}%
\pgfpathlineto{\pgfqpoint{4.548985in}{1.618279in}}%
\pgfpathlineto{\pgfqpoint{4.557101in}{1.631528in}}%
\pgfpathlineto{\pgfqpoint{4.565213in}{1.644718in}}%
\pgfpathlineto{\pgfqpoint{4.550744in}{1.642301in}}%
\pgfpathlineto{\pgfqpoint{4.536286in}{1.639956in}}%
\pgfpathlineto{\pgfqpoint{4.521840in}{1.637680in}}%
\pgfpathlineto{\pgfqpoint{4.507404in}{1.635476in}}%
\pgfpathlineto{\pgfqpoint{4.499289in}{1.622511in}}%
\pgfpathlineto{\pgfqpoint{4.491168in}{1.609492in}}%
\pgfpathlineto{\pgfqpoint{4.483042in}{1.596421in}}%
\pgfpathlineto{\pgfqpoint{4.474911in}{1.583302in}}%
\pgfpathclose%
\pgfusepath{fill}%
\end{pgfscope}%
\begin{pgfscope}%
\pgfpathrectangle{\pgfqpoint{1.150000in}{0.150000in}}{\pgfqpoint{5.700000in}{5.700000in}}%
\pgfusepath{clip}%
\pgfsetbuttcap%
\pgfsetroundjoin%
\definecolor{currentfill}{rgb}{0.272594,0.025563,0.353093}%
\pgfsetfillcolor{currentfill}%
\pgfsetfillopacity{0.700000}%
\pgfsetlinewidth{0.000000pt}%
\definecolor{currentstroke}{rgb}{0.000000,0.000000,0.000000}%
\pgfsetstrokecolor{currentstroke}%
\pgfsetdash{}{0pt}%
\pgfpathmoveto{\pgfqpoint{3.753341in}{1.219311in}}%
\pgfpathlineto{\pgfqpoint{3.767558in}{1.216432in}}%
\pgfpathlineto{\pgfqpoint{3.781783in}{1.213624in}}%
\pgfpathlineto{\pgfqpoint{3.796014in}{1.210887in}}%
\pgfpathlineto{\pgfqpoint{3.810253in}{1.208222in}}%
\pgfpathlineto{\pgfqpoint{3.818612in}{1.217821in}}%
\pgfpathlineto{\pgfqpoint{3.826965in}{1.227557in}}%
\pgfpathlineto{\pgfqpoint{3.835312in}{1.237424in}}%
\pgfpathlineto{\pgfqpoint{3.843651in}{1.247416in}}%
\pgfpathlineto{\pgfqpoint{3.829426in}{1.249687in}}%
\pgfpathlineto{\pgfqpoint{3.815209in}{1.252029in}}%
\pgfpathlineto{\pgfqpoint{3.800998in}{1.254443in}}%
\pgfpathlineto{\pgfqpoint{3.786796in}{1.256929in}}%
\pgfpathlineto{\pgfqpoint{3.778443in}{1.247323in}}%
\pgfpathlineto{\pgfqpoint{3.770083in}{1.237848in}}%
\pgfpathlineto{\pgfqpoint{3.761716in}{1.228508in}}%
\pgfpathlineto{\pgfqpoint{3.753341in}{1.219311in}}%
\pgfpathclose%
\pgfusepath{fill}%
\end{pgfscope}%
\begin{pgfscope}%
\pgfpathrectangle{\pgfqpoint{1.150000in}{0.150000in}}{\pgfqpoint{5.700000in}{5.700000in}}%
\pgfusepath{clip}%
\pgfsetbuttcap%
\pgfsetroundjoin%
\definecolor{currentfill}{rgb}{0.262138,0.242286,0.520837}%
\pgfsetfillcolor{currentfill}%
\pgfsetfillopacity{0.700000}%
\pgfsetlinewidth{0.000000pt}%
\definecolor{currentstroke}{rgb}{0.000000,0.000000,0.000000}%
\pgfsetstrokecolor{currentstroke}%
\pgfsetdash{}{0pt}%
\pgfpathmoveto{\pgfqpoint{4.565213in}{1.644718in}}%
\pgfpathlineto{\pgfqpoint{4.579693in}{1.647205in}}%
\pgfpathlineto{\pgfqpoint{4.594185in}{1.649763in}}%
\pgfpathlineto{\pgfqpoint{4.608687in}{1.652391in}}%
\pgfpathlineto{\pgfqpoint{4.623202in}{1.655090in}}%
\pgfpathlineto{\pgfqpoint{4.631305in}{1.668437in}}%
\pgfpathlineto{\pgfqpoint{4.639402in}{1.681712in}}%
\pgfpathlineto{\pgfqpoint{4.647494in}{1.694914in}}%
\pgfpathlineto{\pgfqpoint{4.655581in}{1.708038in}}%
\pgfpathlineto{\pgfqpoint{4.641070in}{1.705127in}}%
\pgfpathlineto{\pgfqpoint{4.626570in}{1.702286in}}%
\pgfpathlineto{\pgfqpoint{4.612082in}{1.699517in}}%
\pgfpathlineto{\pgfqpoint{4.597605in}{1.696818in}}%
\pgfpathlineto{\pgfqpoint{4.589515in}{1.683897in}}%
\pgfpathlineto{\pgfqpoint{4.581420in}{1.670905in}}%
\pgfpathlineto{\pgfqpoint{4.573319in}{1.657845in}}%
\pgfpathlineto{\pgfqpoint{4.565213in}{1.644718in}}%
\pgfpathclose%
\pgfusepath{fill}%
\end{pgfscope}%
\begin{pgfscope}%
\pgfpathrectangle{\pgfqpoint{1.150000in}{0.150000in}}{\pgfqpoint{5.700000in}{5.700000in}}%
\pgfusepath{clip}%
\pgfsetbuttcap%
\pgfsetroundjoin%
\definecolor{currentfill}{rgb}{0.208623,0.367752,0.552675}%
\pgfsetfillcolor{currentfill}%
\pgfsetfillopacity{0.700000}%
\pgfsetlinewidth{0.000000pt}%
\definecolor{currentstroke}{rgb}{0.000000,0.000000,0.000000}%
\pgfsetstrokecolor{currentstroke}%
\pgfsetdash{}{0pt}%
\pgfpathmoveto{\pgfqpoint{4.959054in}{1.953645in}}%
\pgfpathlineto{\pgfqpoint{4.973711in}{1.958116in}}%
\pgfpathlineto{\pgfqpoint{4.988381in}{1.962658in}}%
\pgfpathlineto{\pgfqpoint{5.003063in}{1.967271in}}%
\pgfpathlineto{\pgfqpoint{5.017759in}{1.971956in}}%
\pgfpathlineto{\pgfqpoint{5.025723in}{1.983870in}}%
\pgfpathlineto{\pgfqpoint{5.033680in}{1.995647in}}%
\pgfpathlineto{\pgfqpoint{5.041629in}{2.007288in}}%
\pgfpathlineto{\pgfqpoint{5.049570in}{2.018790in}}%
\pgfpathlineto{\pgfqpoint{5.034879in}{2.014000in}}%
\pgfpathlineto{\pgfqpoint{5.020200in}{2.009282in}}%
\pgfpathlineto{\pgfqpoint{5.005535in}{2.004635in}}%
\pgfpathlineto{\pgfqpoint{4.990883in}{2.000059in}}%
\pgfpathlineto{\pgfqpoint{4.982937in}{1.988653in}}%
\pgfpathlineto{\pgfqpoint{4.974983in}{1.977115in}}%
\pgfpathlineto{\pgfqpoint{4.967022in}{1.965445in}}%
\pgfpathlineto{\pgfqpoint{4.959054in}{1.953645in}}%
\pgfpathclose%
\pgfusepath{fill}%
\end{pgfscope}%
\begin{pgfscope}%
\pgfpathrectangle{\pgfqpoint{1.150000in}{0.150000in}}{\pgfqpoint{5.700000in}{5.700000in}}%
\pgfusepath{clip}%
\pgfsetbuttcap%
\pgfsetroundjoin%
\definecolor{currentfill}{rgb}{0.269944,0.014625,0.341379}%
\pgfsetfillcolor{currentfill}%
\pgfsetfillopacity{0.700000}%
\pgfsetlinewidth{0.000000pt}%
\definecolor{currentstroke}{rgb}{0.000000,0.000000,0.000000}%
\pgfsetstrokecolor{currentstroke}%
\pgfsetdash{}{0pt}%
\pgfpathmoveto{\pgfqpoint{3.662906in}{1.197959in}}%
\pgfpathlineto{\pgfqpoint{3.677112in}{1.194379in}}%
\pgfpathlineto{\pgfqpoint{3.691325in}{1.190870in}}%
\pgfpathlineto{\pgfqpoint{3.705544in}{1.187433in}}%
\pgfpathlineto{\pgfqpoint{3.719770in}{1.184068in}}%
\pgfpathlineto{\pgfqpoint{3.728174in}{1.192634in}}%
\pgfpathlineto{\pgfqpoint{3.736571in}{1.201368in}}%
\pgfpathlineto{\pgfqpoint{3.744960in}{1.210263in}}%
\pgfpathlineto{\pgfqpoint{3.753341in}{1.219311in}}%
\pgfpathlineto{\pgfqpoint{3.739131in}{1.222262in}}%
\pgfpathlineto{\pgfqpoint{3.724928in}{1.225285in}}%
\pgfpathlineto{\pgfqpoint{3.710732in}{1.228379in}}%
\pgfpathlineto{\pgfqpoint{3.696543in}{1.231545in}}%
\pgfpathlineto{\pgfqpoint{3.688145in}{1.222903in}}%
\pgfpathlineto{\pgfqpoint{3.679740in}{1.214421in}}%
\pgfpathlineto{\pgfqpoint{3.671327in}{1.206104in}}%
\pgfpathlineto{\pgfqpoint{3.662906in}{1.197959in}}%
\pgfpathclose%
\pgfusepath{fill}%
\end{pgfscope}%
\begin{pgfscope}%
\pgfpathrectangle{\pgfqpoint{1.150000in}{0.150000in}}{\pgfqpoint{5.700000in}{5.700000in}}%
\pgfusepath{clip}%
\pgfsetbuttcap%
\pgfsetroundjoin%
\definecolor{currentfill}{rgb}{0.252194,0.269783,0.531579}%
\pgfsetfillcolor{currentfill}%
\pgfsetfillopacity{0.700000}%
\pgfsetlinewidth{0.000000pt}%
\definecolor{currentstroke}{rgb}{0.000000,0.000000,0.000000}%
\pgfsetstrokecolor{currentstroke}%
\pgfsetdash{}{0pt}%
\pgfpathmoveto{\pgfqpoint{4.655581in}{1.708038in}}%
\pgfpathlineto{\pgfqpoint{4.670104in}{1.711020in}}%
\pgfpathlineto{\pgfqpoint{4.684638in}{1.714072in}}%
\pgfpathlineto{\pgfqpoint{4.699184in}{1.717196in}}%
\pgfpathlineto{\pgfqpoint{4.713743in}{1.720390in}}%
\pgfpathlineto{\pgfqpoint{4.721820in}{1.733633in}}%
\pgfpathlineto{\pgfqpoint{4.729893in}{1.746789in}}%
\pgfpathlineto{\pgfqpoint{4.737959in}{1.759855in}}%
\pgfpathlineto{\pgfqpoint{4.746019in}{1.772827in}}%
\pgfpathlineto{\pgfqpoint{4.731464in}{1.769442in}}%
\pgfpathlineto{\pgfqpoint{4.716921in}{1.766127in}}%
\pgfpathlineto{\pgfqpoint{4.702390in}{1.762883in}}%
\pgfpathlineto{\pgfqpoint{4.687870in}{1.759711in}}%
\pgfpathlineto{\pgfqpoint{4.679807in}{1.746921in}}%
\pgfpathlineto{\pgfqpoint{4.671737in}{1.734044in}}%
\pgfpathlineto{\pgfqpoint{4.663662in}{1.721082in}}%
\pgfpathlineto{\pgfqpoint{4.655581in}{1.708038in}}%
\pgfpathclose%
\pgfusepath{fill}%
\end{pgfscope}%
\begin{pgfscope}%
\pgfpathrectangle{\pgfqpoint{1.150000in}{0.150000in}}{\pgfqpoint{5.700000in}{5.700000in}}%
\pgfusepath{clip}%
\pgfsetbuttcap%
\pgfsetroundjoin%
\definecolor{currentfill}{rgb}{0.195860,0.395433,0.555276}%
\pgfsetfillcolor{currentfill}%
\pgfsetfillopacity{0.700000}%
\pgfsetlinewidth{0.000000pt}%
\definecolor{currentstroke}{rgb}{0.000000,0.000000,0.000000}%
\pgfsetstrokecolor{currentstroke}%
\pgfsetdash{}{0pt}%
\pgfpathmoveto{\pgfqpoint{5.049570in}{2.018790in}}%
\pgfpathlineto{\pgfqpoint{5.064275in}{2.023652in}}%
\pgfpathlineto{\pgfqpoint{5.078993in}{2.028585in}}%
\pgfpathlineto{\pgfqpoint{5.093724in}{2.033590in}}%
\pgfpathlineto{\pgfqpoint{5.108469in}{2.038666in}}%
\pgfpathlineto{\pgfqpoint{5.116398in}{2.050121in}}%
\pgfpathlineto{\pgfqpoint{5.124319in}{2.061431in}}%
\pgfpathlineto{\pgfqpoint{5.132232in}{2.072594in}}%
\pgfpathlineto{\pgfqpoint{5.140136in}{2.083609in}}%
\pgfpathlineto{\pgfqpoint{5.125397in}{2.078450in}}%
\pgfpathlineto{\pgfqpoint{5.110670in}{2.073362in}}%
\pgfpathlineto{\pgfqpoint{5.095958in}{2.068346in}}%
\pgfpathlineto{\pgfqpoint{5.081258in}{2.063401in}}%
\pgfpathlineto{\pgfqpoint{5.073348in}{2.052460in}}%
\pgfpathlineto{\pgfqpoint{5.065430in}{2.041377in}}%
\pgfpathlineto{\pgfqpoint{5.057504in}{2.030154in}}%
\pgfpathlineto{\pgfqpoint{5.049570in}{2.018790in}}%
\pgfpathclose%
\pgfusepath{fill}%
\end{pgfscope}%
\begin{pgfscope}%
\pgfpathrectangle{\pgfqpoint{1.150000in}{0.150000in}}{\pgfqpoint{5.700000in}{5.700000in}}%
\pgfusepath{clip}%
\pgfsetbuttcap%
\pgfsetroundjoin%
\definecolor{currentfill}{rgb}{0.281924,0.089666,0.412415}%
\pgfsetfillcolor{currentfill}%
\pgfsetfillopacity{0.700000}%
\pgfsetlinewidth{0.000000pt}%
\definecolor{currentstroke}{rgb}{0.000000,0.000000,0.000000}%
\pgfsetstrokecolor{currentstroke}%
\pgfsetdash{}{0pt}%
\pgfpathmoveto{\pgfqpoint{4.081193in}{1.318187in}}%
\pgfpathlineto{\pgfqpoint{4.095503in}{1.317564in}}%
\pgfpathlineto{\pgfqpoint{4.109823in}{1.317011in}}%
\pgfpathlineto{\pgfqpoint{4.124151in}{1.316528in}}%
\pgfpathlineto{\pgfqpoint{4.138488in}{1.316115in}}%
\pgfpathlineto{\pgfqpoint{4.146734in}{1.328348in}}%
\pgfpathlineto{\pgfqpoint{4.154976in}{1.340632in}}%
\pgfpathlineto{\pgfqpoint{4.163212in}{1.352961in}}%
\pgfpathlineto{\pgfqpoint{4.171444in}{1.365330in}}%
\pgfpathlineto{\pgfqpoint{4.157114in}{1.365408in}}%
\pgfpathlineto{\pgfqpoint{4.142794in}{1.365557in}}%
\pgfpathlineto{\pgfqpoint{4.128482in}{1.365775in}}%
\pgfpathlineto{\pgfqpoint{4.114180in}{1.366065in}}%
\pgfpathlineto{\pgfqpoint{4.105941in}{1.354022in}}%
\pgfpathlineto{\pgfqpoint{4.097697in}{1.342025in}}%
\pgfpathlineto{\pgfqpoint{4.089448in}{1.330078in}}%
\pgfpathlineto{\pgfqpoint{4.081193in}{1.318187in}}%
\pgfpathclose%
\pgfusepath{fill}%
\end{pgfscope}%
\begin{pgfscope}%
\pgfpathrectangle{\pgfqpoint{1.150000in}{0.150000in}}{\pgfqpoint{5.700000in}{5.700000in}}%
\pgfusepath{clip}%
\pgfsetbuttcap%
\pgfsetroundjoin%
\definecolor{currentfill}{rgb}{0.283197,0.115680,0.436115}%
\pgfsetfillcolor{currentfill}%
\pgfsetfillopacity{0.700000}%
\pgfsetlinewidth{0.000000pt}%
\definecolor{currentstroke}{rgb}{0.000000,0.000000,0.000000}%
\pgfsetstrokecolor{currentstroke}%
\pgfsetdash{}{0pt}%
\pgfpathmoveto{\pgfqpoint{4.171444in}{1.365330in}}%
\pgfpathlineto{\pgfqpoint{4.185783in}{1.365323in}}%
\pgfpathlineto{\pgfqpoint{4.200131in}{1.365385in}}%
\pgfpathlineto{\pgfqpoint{4.214489in}{1.365518in}}%
\pgfpathlineto{\pgfqpoint{4.228856in}{1.365721in}}%
\pgfpathlineto{\pgfqpoint{4.237076in}{1.378448in}}%
\pgfpathlineto{\pgfqpoint{4.245291in}{1.391202in}}%
\pgfpathlineto{\pgfqpoint{4.253501in}{1.403977in}}%
\pgfpathlineto{\pgfqpoint{4.261706in}{1.416769in}}%
\pgfpathlineto{\pgfqpoint{4.247344in}{1.416252in}}%
\pgfpathlineto{\pgfqpoint{4.232993in}{1.415805in}}%
\pgfpathlineto{\pgfqpoint{4.218651in}{1.415428in}}%
\pgfpathlineto{\pgfqpoint{4.204319in}{1.415121in}}%
\pgfpathlineto{\pgfqpoint{4.196108in}{1.402636in}}%
\pgfpathlineto{\pgfqpoint{4.187891in}{1.390173in}}%
\pgfpathlineto{\pgfqpoint{4.179670in}{1.377736in}}%
\pgfpathlineto{\pgfqpoint{4.171444in}{1.365330in}}%
\pgfpathclose%
\pgfusepath{fill}%
\end{pgfscope}%
\begin{pgfscope}%
\pgfpathrectangle{\pgfqpoint{1.150000in}{0.150000in}}{\pgfqpoint{5.700000in}{5.700000in}}%
\pgfusepath{clip}%
\pgfsetbuttcap%
\pgfsetroundjoin%
\definecolor{currentfill}{rgb}{0.279566,0.067836,0.391917}%
\pgfsetfillcolor{currentfill}%
\pgfsetfillopacity{0.700000}%
\pgfsetlinewidth{0.000000pt}%
\definecolor{currentstroke}{rgb}{0.000000,0.000000,0.000000}%
\pgfsetstrokecolor{currentstroke}%
\pgfsetdash{}{0pt}%
\pgfpathmoveto{\pgfqpoint{3.990930in}{1.275898in}}%
\pgfpathlineto{\pgfqpoint{4.005215in}{1.274638in}}%
\pgfpathlineto{\pgfqpoint{4.019508in}{1.273449in}}%
\pgfpathlineto{\pgfqpoint{4.033810in}{1.272330in}}%
\pgfpathlineto{\pgfqpoint{4.048121in}{1.271282in}}%
\pgfpathlineto{\pgfqpoint{4.056397in}{1.282899in}}%
\pgfpathlineto{\pgfqpoint{4.064668in}{1.294593in}}%
\pgfpathlineto{\pgfqpoint{4.072933in}{1.306357in}}%
\pgfpathlineto{\pgfqpoint{4.081193in}{1.318187in}}%
\pgfpathlineto{\pgfqpoint{4.066892in}{1.318881in}}%
\pgfpathlineto{\pgfqpoint{4.052599in}{1.319645in}}%
\pgfpathlineto{\pgfqpoint{4.038315in}{1.320480in}}%
\pgfpathlineto{\pgfqpoint{4.024040in}{1.321386in}}%
\pgfpathlineto{\pgfqpoint{4.015771in}{1.309902in}}%
\pgfpathlineto{\pgfqpoint{4.007496in}{1.298489in}}%
\pgfpathlineto{\pgfqpoint{3.999216in}{1.287153in}}%
\pgfpathlineto{\pgfqpoint{3.990930in}{1.275898in}}%
\pgfpathclose%
\pgfusepath{fill}%
\end{pgfscope}%
\begin{pgfscope}%
\pgfpathrectangle{\pgfqpoint{1.150000in}{0.150000in}}{\pgfqpoint{5.700000in}{5.700000in}}%
\pgfusepath{clip}%
\pgfsetbuttcap%
\pgfsetroundjoin%
\definecolor{currentfill}{rgb}{0.282623,0.140926,0.457517}%
\pgfsetfillcolor{currentfill}%
\pgfsetfillopacity{0.700000}%
\pgfsetlinewidth{0.000000pt}%
\definecolor{currentstroke}{rgb}{0.000000,0.000000,0.000000}%
\pgfsetstrokecolor{currentstroke}%
\pgfsetdash{}{0pt}%
\pgfpathmoveto{\pgfqpoint{4.261706in}{1.416769in}}%
\pgfpathlineto{\pgfqpoint{4.276076in}{1.417357in}}%
\pgfpathlineto{\pgfqpoint{4.290457in}{1.418015in}}%
\pgfpathlineto{\pgfqpoint{4.304848in}{1.418743in}}%
\pgfpathlineto{\pgfqpoint{4.319248in}{1.419542in}}%
\pgfpathlineto{\pgfqpoint{4.327443in}{1.432648in}}%
\pgfpathlineto{\pgfqpoint{4.335633in}{1.445758in}}%
\pgfpathlineto{\pgfqpoint{4.343818in}{1.458867in}}%
\pgfpathlineto{\pgfqpoint{4.351998in}{1.471969in}}%
\pgfpathlineto{\pgfqpoint{4.337602in}{1.470876in}}%
\pgfpathlineto{\pgfqpoint{4.323217in}{1.469854in}}%
\pgfpathlineto{\pgfqpoint{4.308842in}{1.468902in}}%
\pgfpathlineto{\pgfqpoint{4.294476in}{1.468020in}}%
\pgfpathlineto{\pgfqpoint{4.286291in}{1.455203in}}%
\pgfpathlineto{\pgfqpoint{4.278101in}{1.442387in}}%
\pgfpathlineto{\pgfqpoint{4.269906in}{1.429574in}}%
\pgfpathlineto{\pgfqpoint{4.261706in}{1.416769in}}%
\pgfpathclose%
\pgfusepath{fill}%
\end{pgfscope}%
\begin{pgfscope}%
\pgfpathrectangle{\pgfqpoint{1.150000in}{0.150000in}}{\pgfqpoint{5.700000in}{5.700000in}}%
\pgfusepath{clip}%
\pgfsetbuttcap%
\pgfsetroundjoin%
\definecolor{currentfill}{rgb}{0.239346,0.300855,0.540844}%
\pgfsetfillcolor{currentfill}%
\pgfsetfillopacity{0.700000}%
\pgfsetlinewidth{0.000000pt}%
\definecolor{currentstroke}{rgb}{0.000000,0.000000,0.000000}%
\pgfsetstrokecolor{currentstroke}%
\pgfsetdash{}{0pt}%
\pgfpathmoveto{\pgfqpoint{4.746019in}{1.772827in}}%
\pgfpathlineto{\pgfqpoint{4.760587in}{1.776283in}}%
\pgfpathlineto{\pgfqpoint{4.775166in}{1.779811in}}%
\pgfpathlineto{\pgfqpoint{4.789757in}{1.783409in}}%
\pgfpathlineto{\pgfqpoint{4.804361in}{1.787078in}}%
\pgfpathlineto{\pgfqpoint{4.812413in}{1.800133in}}%
\pgfpathlineto{\pgfqpoint{4.820458in}{1.813085in}}%
\pgfpathlineto{\pgfqpoint{4.828497in}{1.825931in}}%
\pgfpathlineto{\pgfqpoint{4.836530in}{1.838670in}}%
\pgfpathlineto{\pgfqpoint{4.821929in}{1.834831in}}%
\pgfpathlineto{\pgfqpoint{4.807341in}{1.831063in}}%
\pgfpathlineto{\pgfqpoint{4.792764in}{1.827366in}}%
\pgfpathlineto{\pgfqpoint{4.778200in}{1.823740in}}%
\pgfpathlineto{\pgfqpoint{4.770164in}{1.811163in}}%
\pgfpathlineto{\pgfqpoint{4.762122in}{1.798483in}}%
\pgfpathlineto{\pgfqpoint{4.754074in}{1.785704in}}%
\pgfpathlineto{\pgfqpoint{4.746019in}{1.772827in}}%
\pgfpathclose%
\pgfusepath{fill}%
\end{pgfscope}%
\begin{pgfscope}%
\pgfpathrectangle{\pgfqpoint{1.150000in}{0.150000in}}{\pgfqpoint{5.700000in}{5.700000in}}%
\pgfusepath{clip}%
\pgfsetbuttcap%
\pgfsetroundjoin%
\definecolor{currentfill}{rgb}{0.276022,0.044167,0.370164}%
\pgfsetfillcolor{currentfill}%
\pgfsetfillopacity{0.700000}%
\pgfsetlinewidth{0.000000pt}%
\definecolor{currentstroke}{rgb}{0.000000,0.000000,0.000000}%
\pgfsetstrokecolor{currentstroke}%
\pgfsetdash{}{0pt}%
\pgfpathmoveto{\pgfqpoint{3.900627in}{1.239041in}}%
\pgfpathlineto{\pgfqpoint{3.914891in}{1.237125in}}%
\pgfpathlineto{\pgfqpoint{3.929162in}{1.235279in}}%
\pgfpathlineto{\pgfqpoint{3.943441in}{1.233504in}}%
\pgfpathlineto{\pgfqpoint{3.957729in}{1.231799in}}%
\pgfpathlineto{\pgfqpoint{3.966038in}{1.242675in}}%
\pgfpathlineto{\pgfqpoint{3.974341in}{1.253653in}}%
\pgfpathlineto{\pgfqpoint{3.982639in}{1.264729in}}%
\pgfpathlineto{\pgfqpoint{3.990930in}{1.275898in}}%
\pgfpathlineto{\pgfqpoint{3.976654in}{1.277228in}}%
\pgfpathlineto{\pgfqpoint{3.962385in}{1.278629in}}%
\pgfpathlineto{\pgfqpoint{3.948125in}{1.280100in}}%
\pgfpathlineto{\pgfqpoint{3.933873in}{1.281642in}}%
\pgfpathlineto{\pgfqpoint{3.925571in}{1.270840in}}%
\pgfpathlineto{\pgfqpoint{3.917262in}{1.260136in}}%
\pgfpathlineto{\pgfqpoint{3.908948in}{1.249535in}}%
\pgfpathlineto{\pgfqpoint{3.900627in}{1.239041in}}%
\pgfpathclose%
\pgfusepath{fill}%
\end{pgfscope}%
\begin{pgfscope}%
\pgfpathrectangle{\pgfqpoint{1.150000in}{0.150000in}}{\pgfqpoint{5.700000in}{5.700000in}}%
\pgfusepath{clip}%
\pgfsetbuttcap%
\pgfsetroundjoin%
\definecolor{currentfill}{rgb}{0.185556,0.418570,0.556753}%
\pgfsetfillcolor{currentfill}%
\pgfsetfillopacity{0.700000}%
\pgfsetlinewidth{0.000000pt}%
\definecolor{currentstroke}{rgb}{0.000000,0.000000,0.000000}%
\pgfsetstrokecolor{currentstroke}%
\pgfsetdash{}{0pt}%
\pgfpathmoveto{\pgfqpoint{5.140136in}{2.083609in}}%
\pgfpathlineto{\pgfqpoint{5.154890in}{2.088841in}}%
\pgfpathlineto{\pgfqpoint{5.169657in}{2.094144in}}%
\pgfpathlineto{\pgfqpoint{5.184438in}{2.099518in}}%
\pgfpathlineto{\pgfqpoint{5.199232in}{2.104965in}}%
\pgfpathlineto{\pgfqpoint{5.207123in}{2.115902in}}%
\pgfpathlineto{\pgfqpoint{5.215006in}{2.126685in}}%
\pgfpathlineto{\pgfqpoint{5.222880in}{2.137313in}}%
\pgfpathlineto{\pgfqpoint{5.230745in}{2.147786in}}%
\pgfpathlineto{\pgfqpoint{5.215956in}{2.142279in}}%
\pgfpathlineto{\pgfqpoint{5.201181in}{2.136843in}}%
\pgfpathlineto{\pgfqpoint{5.186420in}{2.131479in}}%
\pgfpathlineto{\pgfqpoint{5.171673in}{2.126187in}}%
\pgfpathlineto{\pgfqpoint{5.163802in}{2.115766in}}%
\pgfpathlineto{\pgfqpoint{5.155921in}{2.105196in}}%
\pgfpathlineto{\pgfqpoint{5.148033in}{2.094477in}}%
\pgfpathlineto{\pgfqpoint{5.140136in}{2.083609in}}%
\pgfpathclose%
\pgfusepath{fill}%
\end{pgfscope}%
\begin{pgfscope}%
\pgfpathrectangle{\pgfqpoint{1.150000in}{0.150000in}}{\pgfqpoint{5.700000in}{5.700000in}}%
\pgfusepath{clip}%
\pgfsetbuttcap%
\pgfsetroundjoin%
\definecolor{currentfill}{rgb}{0.280255,0.165693,0.476498}%
\pgfsetfillcolor{currentfill}%
\pgfsetfillopacity{0.700000}%
\pgfsetlinewidth{0.000000pt}%
\definecolor{currentstroke}{rgb}{0.000000,0.000000,0.000000}%
\pgfsetstrokecolor{currentstroke}%
\pgfsetdash{}{0pt}%
\pgfpathmoveto{\pgfqpoint{4.351998in}{1.471969in}}%
\pgfpathlineto{\pgfqpoint{4.366404in}{1.473133in}}%
\pgfpathlineto{\pgfqpoint{4.380820in}{1.474366in}}%
\pgfpathlineto{\pgfqpoint{4.395246in}{1.475670in}}%
\pgfpathlineto{\pgfqpoint{4.409683in}{1.477044in}}%
\pgfpathlineto{\pgfqpoint{4.417854in}{1.490419in}}%
\pgfpathlineto{\pgfqpoint{4.426020in}{1.503775in}}%
\pgfpathlineto{\pgfqpoint{4.434181in}{1.517109in}}%
\pgfpathlineto{\pgfqpoint{4.442337in}{1.530416in}}%
\pgfpathlineto{\pgfqpoint{4.427904in}{1.528767in}}%
\pgfpathlineto{\pgfqpoint{4.413482in}{1.527189in}}%
\pgfpathlineto{\pgfqpoint{4.399070in}{1.525682in}}%
\pgfpathlineto{\pgfqpoint{4.384669in}{1.524245in}}%
\pgfpathlineto{\pgfqpoint{4.376509in}{1.511204in}}%
\pgfpathlineto{\pgfqpoint{4.368344in}{1.498142in}}%
\pgfpathlineto{\pgfqpoint{4.360173in}{1.485063in}}%
\pgfpathlineto{\pgfqpoint{4.351998in}{1.471969in}}%
\pgfpathclose%
\pgfusepath{fill}%
\end{pgfscope}%
\begin{pgfscope}%
\pgfpathrectangle{\pgfqpoint{1.150000in}{0.150000in}}{\pgfqpoint{5.700000in}{5.700000in}}%
\pgfusepath{clip}%
\pgfsetbuttcap%
\pgfsetroundjoin%
\definecolor{currentfill}{rgb}{0.175841,0.441290,0.557685}%
\pgfsetfillcolor{currentfill}%
\pgfsetfillopacity{0.700000}%
\pgfsetlinewidth{0.000000pt}%
\definecolor{currentstroke}{rgb}{0.000000,0.000000,0.000000}%
\pgfsetstrokecolor{currentstroke}%
\pgfsetdash{}{0pt}%
\pgfpathmoveto{\pgfqpoint{5.230745in}{2.147786in}}%
\pgfpathlineto{\pgfqpoint{5.245547in}{2.153366in}}%
\pgfpathlineto{\pgfqpoint{5.260364in}{2.159017in}}%
\pgfpathlineto{\pgfqpoint{5.275195in}{2.164740in}}%
\pgfpathlineto{\pgfqpoint{5.283046in}{2.175093in}}%
\pgfpathlineto{\pgfqpoint{5.290889in}{2.185286in}}%
\pgfpathlineto{\pgfqpoint{5.298723in}{2.195319in}}%
\pgfpathlineto{\pgfqpoint{5.306547in}{2.205192in}}%
\pgfpathlineto{\pgfqpoint{5.291723in}{2.199430in}}%
\pgfpathlineto{\pgfqpoint{5.276914in}{2.193740in}}%
\pgfpathlineto{\pgfqpoint{5.262118in}{2.188122in}}%
\pgfpathlineto{\pgfqpoint{5.254288in}{2.178272in}}%
\pgfpathlineto{\pgfqpoint{5.246449in}{2.168266in}}%
\pgfpathlineto{\pgfqpoint{5.238601in}{2.158104in}}%
\pgfpathlineto{\pgfqpoint{5.230745in}{2.147786in}}%
\pgfpathclose%
\pgfusepath{fill}%
\end{pgfscope}%
\begin{pgfscope}%
\pgfpathrectangle{\pgfqpoint{1.150000in}{0.150000in}}{\pgfqpoint{5.700000in}{5.700000in}}%
\pgfusepath{clip}%
\pgfsetbuttcap%
\pgfsetroundjoin%
\definecolor{currentfill}{rgb}{0.275191,0.194905,0.496005}%
\pgfsetfillcolor{currentfill}%
\pgfsetfillopacity{0.700000}%
\pgfsetlinewidth{0.000000pt}%
\definecolor{currentstroke}{rgb}{0.000000,0.000000,0.000000}%
\pgfsetstrokecolor{currentstroke}%
\pgfsetdash{}{0pt}%
\pgfpathmoveto{\pgfqpoint{4.442337in}{1.530416in}}%
\pgfpathlineto{\pgfqpoint{4.456780in}{1.532134in}}%
\pgfpathlineto{\pgfqpoint{4.471235in}{1.533923in}}%
\pgfpathlineto{\pgfqpoint{4.485699in}{1.535782in}}%
\pgfpathlineto{\pgfqpoint{4.500175in}{1.537712in}}%
\pgfpathlineto{\pgfqpoint{4.508323in}{1.551250in}}%
\pgfpathlineto{\pgfqpoint{4.516465in}{1.564749in}}%
\pgfpathlineto{\pgfqpoint{4.524603in}{1.578205in}}%
\pgfpathlineto{\pgfqpoint{4.532735in}{1.591614in}}%
\pgfpathlineto{\pgfqpoint{4.518263in}{1.589430in}}%
\pgfpathlineto{\pgfqpoint{4.503801in}{1.587317in}}%
\pgfpathlineto{\pgfqpoint{4.489351in}{1.585274in}}%
\pgfpathlineto{\pgfqpoint{4.474911in}{1.583302in}}%
\pgfpathlineto{\pgfqpoint{4.466775in}{1.570139in}}%
\pgfpathlineto{\pgfqpoint{4.458634in}{1.556934in}}%
\pgfpathlineto{\pgfqpoint{4.450488in}{1.543692in}}%
\pgfpathlineto{\pgfqpoint{4.442337in}{1.530416in}}%
\pgfpathclose%
\pgfusepath{fill}%
\end{pgfscope}%
\begin{pgfscope}%
\pgfpathrectangle{\pgfqpoint{1.150000in}{0.150000in}}{\pgfqpoint{5.700000in}{5.700000in}}%
\pgfusepath{clip}%
\pgfsetbuttcap%
\pgfsetroundjoin%
\definecolor{currentfill}{rgb}{0.273809,0.031497,0.358853}%
\pgfsetfillcolor{currentfill}%
\pgfsetfillopacity{0.700000}%
\pgfsetlinewidth{0.000000pt}%
\definecolor{currentstroke}{rgb}{0.000000,0.000000,0.000000}%
\pgfsetstrokecolor{currentstroke}%
\pgfsetdash{}{0pt}%
\pgfpathmoveto{\pgfqpoint{3.810253in}{1.208222in}}%
\pgfpathlineto{\pgfqpoint{3.824499in}{1.205628in}}%
\pgfpathlineto{\pgfqpoint{3.838752in}{1.203104in}}%
\pgfpathlineto{\pgfqpoint{3.853013in}{1.200651in}}%
\pgfpathlineto{\pgfqpoint{3.867281in}{1.198269in}}%
\pgfpathlineto{\pgfqpoint{3.875627in}{1.208271in}}%
\pgfpathlineto{\pgfqpoint{3.883967in}{1.218404in}}%
\pgfpathlineto{\pgfqpoint{3.892300in}{1.228663in}}%
\pgfpathlineto{\pgfqpoint{3.900627in}{1.239041in}}%
\pgfpathlineto{\pgfqpoint{3.886372in}{1.241029in}}%
\pgfpathlineto{\pgfqpoint{3.872124in}{1.243087in}}%
\pgfpathlineto{\pgfqpoint{3.857884in}{1.245216in}}%
\pgfpathlineto{\pgfqpoint{3.843651in}{1.247416in}}%
\pgfpathlineto{\pgfqpoint{3.835312in}{1.237424in}}%
\pgfpathlineto{\pgfqpoint{3.826965in}{1.227557in}}%
\pgfpathlineto{\pgfqpoint{3.818612in}{1.217821in}}%
\pgfpathlineto{\pgfqpoint{3.810253in}{1.208222in}}%
\pgfpathclose%
\pgfusepath{fill}%
\end{pgfscope}%
\begin{pgfscope}%
\pgfpathrectangle{\pgfqpoint{1.150000in}{0.150000in}}{\pgfqpoint{5.700000in}{5.700000in}}%
\pgfusepath{clip}%
\pgfsetbuttcap%
\pgfsetroundjoin%
\definecolor{currentfill}{rgb}{0.225863,0.330805,0.547314}%
\pgfsetfillcolor{currentfill}%
\pgfsetfillopacity{0.700000}%
\pgfsetlinewidth{0.000000pt}%
\definecolor{currentstroke}{rgb}{0.000000,0.000000,0.000000}%
\pgfsetstrokecolor{currentstroke}%
\pgfsetdash{}{0pt}%
\pgfpathmoveto{\pgfqpoint{4.836530in}{1.838670in}}%
\pgfpathlineto{\pgfqpoint{4.851143in}{1.842581in}}%
\pgfpathlineto{\pgfqpoint{4.865769in}{1.846562in}}%
\pgfpathlineto{\pgfqpoint{4.880407in}{1.850615in}}%
\pgfpathlineto{\pgfqpoint{4.895058in}{1.854738in}}%
\pgfpathlineto{\pgfqpoint{4.903081in}{1.867524in}}%
\pgfpathlineto{\pgfqpoint{4.911098in}{1.880193in}}%
\pgfpathlineto{\pgfqpoint{4.919108in}{1.892743in}}%
\pgfpathlineto{\pgfqpoint{4.927111in}{1.905172in}}%
\pgfpathlineto{\pgfqpoint{4.912463in}{1.900900in}}%
\pgfpathlineto{\pgfqpoint{4.897828in}{1.896699in}}%
\pgfpathlineto{\pgfqpoint{4.883206in}{1.892568in}}%
\pgfpathlineto{\pgfqpoint{4.868596in}{1.888510in}}%
\pgfpathlineto{\pgfqpoint{4.860589in}{1.876221in}}%
\pgfpathlineto{\pgfqpoint{4.852576in}{1.863817in}}%
\pgfpathlineto{\pgfqpoint{4.844556in}{1.851300in}}%
\pgfpathlineto{\pgfqpoint{4.836530in}{1.838670in}}%
\pgfpathclose%
\pgfusepath{fill}%
\end{pgfscope}%
\begin{pgfscope}%
\pgfpathrectangle{\pgfqpoint{1.150000in}{0.150000in}}{\pgfqpoint{5.700000in}{5.700000in}}%
\pgfusepath{clip}%
\pgfsetbuttcap%
\pgfsetroundjoin%
\definecolor{currentfill}{rgb}{0.266580,0.228262,0.514349}%
\pgfsetfillcolor{currentfill}%
\pgfsetfillopacity{0.700000}%
\pgfsetlinewidth{0.000000pt}%
\definecolor{currentstroke}{rgb}{0.000000,0.000000,0.000000}%
\pgfsetstrokecolor{currentstroke}%
\pgfsetdash{}{0pt}%
\pgfpathmoveto{\pgfqpoint{4.532735in}{1.591614in}}%
\pgfpathlineto{\pgfqpoint{4.547219in}{1.593868in}}%
\pgfpathlineto{\pgfqpoint{4.561713in}{1.596192in}}%
\pgfpathlineto{\pgfqpoint{4.576219in}{1.598587in}}%
\pgfpathlineto{\pgfqpoint{4.590736in}{1.601053in}}%
\pgfpathlineto{\pgfqpoint{4.598860in}{1.614653in}}%
\pgfpathlineto{\pgfqpoint{4.606979in}{1.628195in}}%
\pgfpathlineto{\pgfqpoint{4.615093in}{1.641675in}}%
\pgfpathlineto{\pgfqpoint{4.623202in}{1.655090in}}%
\pgfpathlineto{\pgfqpoint{4.608687in}{1.652391in}}%
\pgfpathlineto{\pgfqpoint{4.594185in}{1.649763in}}%
\pgfpathlineto{\pgfqpoint{4.579693in}{1.647205in}}%
\pgfpathlineto{\pgfqpoint{4.565213in}{1.644718in}}%
\pgfpathlineto{\pgfqpoint{4.557101in}{1.631528in}}%
\pgfpathlineto{\pgfqpoint{4.548985in}{1.618279in}}%
\pgfpathlineto{\pgfqpoint{4.540862in}{1.604973in}}%
\pgfpathlineto{\pgfqpoint{4.532735in}{1.591614in}}%
\pgfpathclose%
\pgfusepath{fill}%
\end{pgfscope}%
\begin{pgfscope}%
\pgfpathrectangle{\pgfqpoint{1.150000in}{0.150000in}}{\pgfqpoint{5.700000in}{5.700000in}}%
\pgfusepath{clip}%
\pgfsetbuttcap%
\pgfsetroundjoin%
\definecolor{currentfill}{rgb}{0.212395,0.359683,0.551710}%
\pgfsetfillcolor{currentfill}%
\pgfsetfillopacity{0.700000}%
\pgfsetlinewidth{0.000000pt}%
\definecolor{currentstroke}{rgb}{0.000000,0.000000,0.000000}%
\pgfsetstrokecolor{currentstroke}%
\pgfsetdash{}{0pt}%
\pgfpathmoveto{\pgfqpoint{4.927111in}{1.905172in}}%
\pgfpathlineto{\pgfqpoint{4.941771in}{1.909516in}}%
\pgfpathlineto{\pgfqpoint{4.956445in}{1.913931in}}%
\pgfpathlineto{\pgfqpoint{4.971131in}{1.918417in}}%
\pgfpathlineto{\pgfqpoint{4.985830in}{1.922975in}}%
\pgfpathlineto{\pgfqpoint{4.993823in}{1.935417in}}%
\pgfpathlineto{\pgfqpoint{5.001809in}{1.947729in}}%
\pgfpathlineto{\pgfqpoint{5.009788in}{1.959909in}}%
\pgfpathlineto{\pgfqpoint{5.017759in}{1.971956in}}%
\pgfpathlineto{\pgfqpoint{5.003063in}{1.967271in}}%
\pgfpathlineto{\pgfqpoint{4.988381in}{1.962658in}}%
\pgfpathlineto{\pgfqpoint{4.973711in}{1.958116in}}%
\pgfpathlineto{\pgfqpoint{4.959054in}{1.953645in}}%
\pgfpathlineto{\pgfqpoint{4.951079in}{1.941717in}}%
\pgfpathlineto{\pgfqpoint{4.943096in}{1.929660in}}%
\pgfpathlineto{\pgfqpoint{4.935107in}{1.917479in}}%
\pgfpathlineto{\pgfqpoint{4.927111in}{1.905172in}}%
\pgfpathclose%
\pgfusepath{fill}%
\end{pgfscope}%
\begin{pgfscope}%
\pgfpathrectangle{\pgfqpoint{1.150000in}{0.150000in}}{\pgfqpoint{5.700000in}{5.700000in}}%
\pgfusepath{clip}%
\pgfsetbuttcap%
\pgfsetroundjoin%
\definecolor{currentfill}{rgb}{0.271305,0.019942,0.347269}%
\pgfsetfillcolor{currentfill}%
\pgfsetfillopacity{0.700000}%
\pgfsetlinewidth{0.000000pt}%
\definecolor{currentstroke}{rgb}{0.000000,0.000000,0.000000}%
\pgfsetstrokecolor{currentstroke}%
\pgfsetdash{}{0pt}%
\pgfpathmoveto{\pgfqpoint{3.719770in}{1.184068in}}%
\pgfpathlineto{\pgfqpoint{3.734003in}{1.180774in}}%
\pgfpathlineto{\pgfqpoint{3.748242in}{1.177551in}}%
\pgfpathlineto{\pgfqpoint{3.762489in}{1.174399in}}%
\pgfpathlineto{\pgfqpoint{3.776742in}{1.171319in}}%
\pgfpathlineto{\pgfqpoint{3.785131in}{1.180308in}}%
\pgfpathlineto{\pgfqpoint{3.793512in}{1.189459in}}%
\pgfpathlineto{\pgfqpoint{3.801886in}{1.198766in}}%
\pgfpathlineto{\pgfqpoint{3.810253in}{1.208222in}}%
\pgfpathlineto{\pgfqpoint{3.796014in}{1.210887in}}%
\pgfpathlineto{\pgfqpoint{3.781783in}{1.213624in}}%
\pgfpathlineto{\pgfqpoint{3.767558in}{1.216432in}}%
\pgfpathlineto{\pgfqpoint{3.753341in}{1.219311in}}%
\pgfpathlineto{\pgfqpoint{3.744960in}{1.210263in}}%
\pgfpathlineto{\pgfqpoint{3.736571in}{1.201368in}}%
\pgfpathlineto{\pgfqpoint{3.728174in}{1.192634in}}%
\pgfpathlineto{\pgfqpoint{3.719770in}{1.184068in}}%
\pgfpathclose%
\pgfusepath{fill}%
\end{pgfscope}%
\begin{pgfscope}%
\pgfpathrectangle{\pgfqpoint{1.150000in}{0.150000in}}{\pgfqpoint{5.700000in}{5.700000in}}%
\pgfusepath{clip}%
\pgfsetbuttcap%
\pgfsetroundjoin%
\definecolor{currentfill}{rgb}{0.257322,0.256130,0.526563}%
\pgfsetfillcolor{currentfill}%
\pgfsetfillopacity{0.700000}%
\pgfsetlinewidth{0.000000pt}%
\definecolor{currentstroke}{rgb}{0.000000,0.000000,0.000000}%
\pgfsetstrokecolor{currentstroke}%
\pgfsetdash{}{0pt}%
\pgfpathmoveto{\pgfqpoint{4.623202in}{1.655090in}}%
\pgfpathlineto{\pgfqpoint{4.637727in}{1.657860in}}%
\pgfpathlineto{\pgfqpoint{4.652265in}{1.660700in}}%
\pgfpathlineto{\pgfqpoint{4.666814in}{1.663610in}}%
\pgfpathlineto{\pgfqpoint{4.681374in}{1.666591in}}%
\pgfpathlineto{\pgfqpoint{4.689475in}{1.680159in}}%
\pgfpathlineto{\pgfqpoint{4.697570in}{1.693650in}}%
\pgfpathlineto{\pgfqpoint{4.705659in}{1.707061in}}%
\pgfpathlineto{\pgfqpoint{4.713743in}{1.720390in}}%
\pgfpathlineto{\pgfqpoint{4.699184in}{1.717196in}}%
\pgfpathlineto{\pgfqpoint{4.684638in}{1.714072in}}%
\pgfpathlineto{\pgfqpoint{4.670104in}{1.711020in}}%
\pgfpathlineto{\pgfqpoint{4.655581in}{1.708038in}}%
\pgfpathlineto{\pgfqpoint{4.647494in}{1.694914in}}%
\pgfpathlineto{\pgfqpoint{4.639402in}{1.681712in}}%
\pgfpathlineto{\pgfqpoint{4.631305in}{1.668437in}}%
\pgfpathlineto{\pgfqpoint{4.623202in}{1.655090in}}%
\pgfpathclose%
\pgfusepath{fill}%
\end{pgfscope}%
\begin{pgfscope}%
\pgfpathrectangle{\pgfqpoint{1.150000in}{0.150000in}}{\pgfqpoint{5.700000in}{5.700000in}}%
\pgfusepath{clip}%
\pgfsetbuttcap%
\pgfsetroundjoin%
\definecolor{currentfill}{rgb}{0.199430,0.387607,0.554642}%
\pgfsetfillcolor{currentfill}%
\pgfsetfillopacity{0.700000}%
\pgfsetlinewidth{0.000000pt}%
\definecolor{currentstroke}{rgb}{0.000000,0.000000,0.000000}%
\pgfsetstrokecolor{currentstroke}%
\pgfsetdash{}{0pt}%
\pgfpathmoveto{\pgfqpoint{5.017759in}{1.971956in}}%
\pgfpathlineto{\pgfqpoint{5.032468in}{1.976713in}}%
\pgfpathlineto{\pgfqpoint{5.047190in}{1.981541in}}%
\pgfpathlineto{\pgfqpoint{5.061926in}{1.986440in}}%
\pgfpathlineto{\pgfqpoint{5.076675in}{1.991411in}}%
\pgfpathlineto{\pgfqpoint{5.084635in}{2.003438in}}%
\pgfpathlineto{\pgfqpoint{5.092587in}{2.015323in}}%
\pgfpathlineto{\pgfqpoint{5.100532in}{2.027067in}}%
\pgfpathlineto{\pgfqpoint{5.108469in}{2.038666in}}%
\pgfpathlineto{\pgfqpoint{5.093724in}{2.033590in}}%
\pgfpathlineto{\pgfqpoint{5.078993in}{2.028585in}}%
\pgfpathlineto{\pgfqpoint{5.064275in}{2.023652in}}%
\pgfpathlineto{\pgfqpoint{5.049570in}{2.018790in}}%
\pgfpathlineto{\pgfqpoint{5.041629in}{2.007288in}}%
\pgfpathlineto{\pgfqpoint{5.033680in}{1.995647in}}%
\pgfpathlineto{\pgfqpoint{5.025723in}{1.983870in}}%
\pgfpathlineto{\pgfqpoint{5.017759in}{1.971956in}}%
\pgfpathclose%
\pgfusepath{fill}%
\end{pgfscope}%
\begin{pgfscope}%
\pgfpathrectangle{\pgfqpoint{1.150000in}{0.150000in}}{\pgfqpoint{5.700000in}{5.700000in}}%
\pgfusepath{clip}%
\pgfsetbuttcap%
\pgfsetroundjoin%
\definecolor{currentfill}{rgb}{0.282656,0.100196,0.422160}%
\pgfsetfillcolor{currentfill}%
\pgfsetfillopacity{0.700000}%
\pgfsetlinewidth{0.000000pt}%
\definecolor{currentstroke}{rgb}{0.000000,0.000000,0.000000}%
\pgfsetstrokecolor{currentstroke}%
\pgfsetdash{}{0pt}%
\pgfpathmoveto{\pgfqpoint{4.138488in}{1.316115in}}%
\pgfpathlineto{\pgfqpoint{4.152834in}{1.315773in}}%
\pgfpathlineto{\pgfqpoint{4.167189in}{1.315500in}}%
\pgfpathlineto{\pgfqpoint{4.181553in}{1.315298in}}%
\pgfpathlineto{\pgfqpoint{4.195927in}{1.315165in}}%
\pgfpathlineto{\pgfqpoint{4.204167in}{1.327741in}}%
\pgfpathlineto{\pgfqpoint{4.212401in}{1.340362in}}%
\pgfpathlineto{\pgfqpoint{4.220631in}{1.353024in}}%
\pgfpathlineto{\pgfqpoint{4.228856in}{1.365721in}}%
\pgfpathlineto{\pgfqpoint{4.214489in}{1.365518in}}%
\pgfpathlineto{\pgfqpoint{4.200131in}{1.365385in}}%
\pgfpathlineto{\pgfqpoint{4.185783in}{1.365323in}}%
\pgfpathlineto{\pgfqpoint{4.171444in}{1.365330in}}%
\pgfpathlineto{\pgfqpoint{4.163212in}{1.352961in}}%
\pgfpathlineto{\pgfqpoint{4.154976in}{1.340632in}}%
\pgfpathlineto{\pgfqpoint{4.146734in}{1.328348in}}%
\pgfpathlineto{\pgfqpoint{4.138488in}{1.316115in}}%
\pgfpathclose%
\pgfusepath{fill}%
\end{pgfscope}%
\begin{pgfscope}%
\pgfpathrectangle{\pgfqpoint{1.150000in}{0.150000in}}{\pgfqpoint{5.700000in}{5.700000in}}%
\pgfusepath{clip}%
\pgfsetbuttcap%
\pgfsetroundjoin%
\definecolor{currentfill}{rgb}{0.280894,0.078907,0.402329}%
\pgfsetfillcolor{currentfill}%
\pgfsetfillopacity{0.700000}%
\pgfsetlinewidth{0.000000pt}%
\definecolor{currentstroke}{rgb}{0.000000,0.000000,0.000000}%
\pgfsetstrokecolor{currentstroke}%
\pgfsetdash{}{0pt}%
\pgfpathmoveto{\pgfqpoint{4.048121in}{1.271282in}}%
\pgfpathlineto{\pgfqpoint{4.062439in}{1.270304in}}%
\pgfpathlineto{\pgfqpoint{4.076767in}{1.269395in}}%
\pgfpathlineto{\pgfqpoint{4.091103in}{1.268557in}}%
\pgfpathlineto{\pgfqpoint{4.105448in}{1.267789in}}%
\pgfpathlineto{\pgfqpoint{4.113716in}{1.279769in}}%
\pgfpathlineto{\pgfqpoint{4.121978in}{1.291821in}}%
\pgfpathlineto{\pgfqpoint{4.130236in}{1.303938in}}%
\pgfpathlineto{\pgfqpoint{4.138488in}{1.316115in}}%
\pgfpathlineto{\pgfqpoint{4.124151in}{1.316528in}}%
\pgfpathlineto{\pgfqpoint{4.109823in}{1.317011in}}%
\pgfpathlineto{\pgfqpoint{4.095503in}{1.317564in}}%
\pgfpathlineto{\pgfqpoint{4.081193in}{1.318187in}}%
\pgfpathlineto{\pgfqpoint{4.072933in}{1.306357in}}%
\pgfpathlineto{\pgfqpoint{4.064668in}{1.294593in}}%
\pgfpathlineto{\pgfqpoint{4.056397in}{1.282899in}}%
\pgfpathlineto{\pgfqpoint{4.048121in}{1.271282in}}%
\pgfpathclose%
\pgfusepath{fill}%
\end{pgfscope}%
\begin{pgfscope}%
\pgfpathrectangle{\pgfqpoint{1.150000in}{0.150000in}}{\pgfqpoint{5.700000in}{5.700000in}}%
\pgfusepath{clip}%
\pgfsetbuttcap%
\pgfsetroundjoin%
\definecolor{currentfill}{rgb}{0.283187,0.125848,0.444960}%
\pgfsetfillcolor{currentfill}%
\pgfsetfillopacity{0.700000}%
\pgfsetlinewidth{0.000000pt}%
\definecolor{currentstroke}{rgb}{0.000000,0.000000,0.000000}%
\pgfsetstrokecolor{currentstroke}%
\pgfsetdash{}{0pt}%
\pgfpathmoveto{\pgfqpoint{4.228856in}{1.365721in}}%
\pgfpathlineto{\pgfqpoint{4.243233in}{1.365994in}}%
\pgfpathlineto{\pgfqpoint{4.257619in}{1.366336in}}%
\pgfpathlineto{\pgfqpoint{4.272015in}{1.366749in}}%
\pgfpathlineto{\pgfqpoint{4.286421in}{1.367232in}}%
\pgfpathlineto{\pgfqpoint{4.294635in}{1.380283in}}%
\pgfpathlineto{\pgfqpoint{4.302844in}{1.393354in}}%
\pgfpathlineto{\pgfqpoint{4.311049in}{1.406442in}}%
\pgfpathlineto{\pgfqpoint{4.319248in}{1.419542in}}%
\pgfpathlineto{\pgfqpoint{4.304848in}{1.418743in}}%
\pgfpathlineto{\pgfqpoint{4.290457in}{1.418015in}}%
\pgfpathlineto{\pgfqpoint{4.276076in}{1.417357in}}%
\pgfpathlineto{\pgfqpoint{4.261706in}{1.416769in}}%
\pgfpathlineto{\pgfqpoint{4.253501in}{1.403977in}}%
\pgfpathlineto{\pgfqpoint{4.245291in}{1.391202in}}%
\pgfpathlineto{\pgfqpoint{4.237076in}{1.378448in}}%
\pgfpathlineto{\pgfqpoint{4.228856in}{1.365721in}}%
\pgfpathclose%
\pgfusepath{fill}%
\end{pgfscope}%
\begin{pgfscope}%
\pgfpathrectangle{\pgfqpoint{1.150000in}{0.150000in}}{\pgfqpoint{5.700000in}{5.700000in}}%
\pgfusepath{clip}%
\pgfsetbuttcap%
\pgfsetroundjoin%
\definecolor{currentfill}{rgb}{0.277941,0.056324,0.381191}%
\pgfsetfillcolor{currentfill}%
\pgfsetfillopacity{0.700000}%
\pgfsetlinewidth{0.000000pt}%
\definecolor{currentstroke}{rgb}{0.000000,0.000000,0.000000}%
\pgfsetstrokecolor{currentstroke}%
\pgfsetdash{}{0pt}%
\pgfpathmoveto{\pgfqpoint{3.957729in}{1.231799in}}%
\pgfpathlineto{\pgfqpoint{3.972024in}{1.230165in}}%
\pgfpathlineto{\pgfqpoint{3.986327in}{1.228601in}}%
\pgfpathlineto{\pgfqpoint{4.000639in}{1.227107in}}%
\pgfpathlineto{\pgfqpoint{4.014959in}{1.225683in}}%
\pgfpathlineto{\pgfqpoint{4.023258in}{1.236941in}}%
\pgfpathlineto{\pgfqpoint{4.031551in}{1.248297in}}%
\pgfpathlineto{\pgfqpoint{4.039839in}{1.259746in}}%
\pgfpathlineto{\pgfqpoint{4.048121in}{1.271282in}}%
\pgfpathlineto{\pgfqpoint{4.033810in}{1.272330in}}%
\pgfpathlineto{\pgfqpoint{4.019508in}{1.273449in}}%
\pgfpathlineto{\pgfqpoint{4.005215in}{1.274638in}}%
\pgfpathlineto{\pgfqpoint{3.990930in}{1.275898in}}%
\pgfpathlineto{\pgfqpoint{3.982639in}{1.264729in}}%
\pgfpathlineto{\pgfqpoint{3.974341in}{1.253653in}}%
\pgfpathlineto{\pgfqpoint{3.966038in}{1.242675in}}%
\pgfpathlineto{\pgfqpoint{3.957729in}{1.231799in}}%
\pgfpathclose%
\pgfusepath{fill}%
\end{pgfscope}%
\begin{pgfscope}%
\pgfpathrectangle{\pgfqpoint{1.150000in}{0.150000in}}{\pgfqpoint{5.700000in}{5.700000in}}%
\pgfusepath{clip}%
\pgfsetbuttcap%
\pgfsetroundjoin%
\definecolor{currentfill}{rgb}{0.244972,0.287675,0.537260}%
\pgfsetfillcolor{currentfill}%
\pgfsetfillopacity{0.700000}%
\pgfsetlinewidth{0.000000pt}%
\definecolor{currentstroke}{rgb}{0.000000,0.000000,0.000000}%
\pgfsetstrokecolor{currentstroke}%
\pgfsetdash{}{0pt}%
\pgfpathmoveto{\pgfqpoint{4.713743in}{1.720390in}}%
\pgfpathlineto{\pgfqpoint{4.728313in}{1.723655in}}%
\pgfpathlineto{\pgfqpoint{4.742894in}{1.726990in}}%
\pgfpathlineto{\pgfqpoint{4.757489in}{1.730396in}}%
\pgfpathlineto{\pgfqpoint{4.772095in}{1.733874in}}%
\pgfpathlineto{\pgfqpoint{4.780170in}{1.747317in}}%
\pgfpathlineto{\pgfqpoint{4.788240in}{1.760667in}}%
\pgfpathlineto{\pgfqpoint{4.796303in}{1.773922in}}%
\pgfpathlineto{\pgfqpoint{4.804361in}{1.787078in}}%
\pgfpathlineto{\pgfqpoint{4.789757in}{1.783409in}}%
\pgfpathlineto{\pgfqpoint{4.775166in}{1.779811in}}%
\pgfpathlineto{\pgfqpoint{4.760587in}{1.776283in}}%
\pgfpathlineto{\pgfqpoint{4.746019in}{1.772827in}}%
\pgfpathlineto{\pgfqpoint{4.737959in}{1.759855in}}%
\pgfpathlineto{\pgfqpoint{4.729893in}{1.746789in}}%
\pgfpathlineto{\pgfqpoint{4.721820in}{1.733633in}}%
\pgfpathlineto{\pgfqpoint{4.713743in}{1.720390in}}%
\pgfpathclose%
\pgfusepath{fill}%
\end{pgfscope}%
\begin{pgfscope}%
\pgfpathrectangle{\pgfqpoint{1.150000in}{0.150000in}}{\pgfqpoint{5.700000in}{5.700000in}}%
\pgfusepath{clip}%
\pgfsetbuttcap%
\pgfsetroundjoin%
\definecolor{currentfill}{rgb}{0.281412,0.155834,0.469201}%
\pgfsetfillcolor{currentfill}%
\pgfsetfillopacity{0.700000}%
\pgfsetlinewidth{0.000000pt}%
\definecolor{currentstroke}{rgb}{0.000000,0.000000,0.000000}%
\pgfsetstrokecolor{currentstroke}%
\pgfsetdash{}{0pt}%
\pgfpathmoveto{\pgfqpoint{4.319248in}{1.419542in}}%
\pgfpathlineto{\pgfqpoint{4.333659in}{1.420410in}}%
\pgfpathlineto{\pgfqpoint{4.348079in}{1.421348in}}%
\pgfpathlineto{\pgfqpoint{4.362510in}{1.422357in}}%
\pgfpathlineto{\pgfqpoint{4.376951in}{1.423435in}}%
\pgfpathlineto{\pgfqpoint{4.385141in}{1.436845in}}%
\pgfpathlineto{\pgfqpoint{4.393327in}{1.450253in}}%
\pgfpathlineto{\pgfqpoint{4.401507in}{1.463653in}}%
\pgfpathlineto{\pgfqpoint{4.409683in}{1.477044in}}%
\pgfpathlineto{\pgfqpoint{4.395246in}{1.475670in}}%
\pgfpathlineto{\pgfqpoint{4.380820in}{1.474366in}}%
\pgfpathlineto{\pgfqpoint{4.366404in}{1.473133in}}%
\pgfpathlineto{\pgfqpoint{4.351998in}{1.471969in}}%
\pgfpathlineto{\pgfqpoint{4.343818in}{1.458867in}}%
\pgfpathlineto{\pgfqpoint{4.335633in}{1.445758in}}%
\pgfpathlineto{\pgfqpoint{4.327443in}{1.432648in}}%
\pgfpathlineto{\pgfqpoint{4.319248in}{1.419542in}}%
\pgfpathclose%
\pgfusepath{fill}%
\end{pgfscope}%
\begin{pgfscope}%
\pgfpathrectangle{\pgfqpoint{1.150000in}{0.150000in}}{\pgfqpoint{5.700000in}{5.700000in}}%
\pgfusepath{clip}%
\pgfsetbuttcap%
\pgfsetroundjoin%
\definecolor{currentfill}{rgb}{0.187231,0.414746,0.556547}%
\pgfsetfillcolor{currentfill}%
\pgfsetfillopacity{0.700000}%
\pgfsetlinewidth{0.000000pt}%
\definecolor{currentstroke}{rgb}{0.000000,0.000000,0.000000}%
\pgfsetstrokecolor{currentstroke}%
\pgfsetdash{}{0pt}%
\pgfpathmoveto{\pgfqpoint{5.108469in}{2.038666in}}%
\pgfpathlineto{\pgfqpoint{5.123227in}{2.043814in}}%
\pgfpathlineto{\pgfqpoint{5.137999in}{2.049034in}}%
\pgfpathlineto{\pgfqpoint{5.152785in}{2.054325in}}%
\pgfpathlineto{\pgfqpoint{5.167584in}{2.059689in}}%
\pgfpathlineto{\pgfqpoint{5.175509in}{2.071235in}}%
\pgfpathlineto{\pgfqpoint{5.183425in}{2.082631in}}%
\pgfpathlineto{\pgfqpoint{5.191333in}{2.093874in}}%
\pgfpathlineto{\pgfqpoint{5.199232in}{2.104965in}}%
\pgfpathlineto{\pgfqpoint{5.184438in}{2.099518in}}%
\pgfpathlineto{\pgfqpoint{5.169657in}{2.094144in}}%
\pgfpathlineto{\pgfqpoint{5.154890in}{2.088841in}}%
\pgfpathlineto{\pgfqpoint{5.140136in}{2.083609in}}%
\pgfpathlineto{\pgfqpoint{5.132232in}{2.072594in}}%
\pgfpathlineto{\pgfqpoint{5.124319in}{2.061431in}}%
\pgfpathlineto{\pgfqpoint{5.116398in}{2.050121in}}%
\pgfpathlineto{\pgfqpoint{5.108469in}{2.038666in}}%
\pgfpathclose%
\pgfusepath{fill}%
\end{pgfscope}%
\begin{pgfscope}%
\pgfpathrectangle{\pgfqpoint{1.150000in}{0.150000in}}{\pgfqpoint{5.700000in}{5.700000in}}%
\pgfusepath{clip}%
\pgfsetbuttcap%
\pgfsetroundjoin%
\definecolor{currentfill}{rgb}{0.274952,0.037752,0.364543}%
\pgfsetfillcolor{currentfill}%
\pgfsetfillopacity{0.700000}%
\pgfsetlinewidth{0.000000pt}%
\definecolor{currentstroke}{rgb}{0.000000,0.000000,0.000000}%
\pgfsetstrokecolor{currentstroke}%
\pgfsetdash{}{0pt}%
\pgfpathmoveto{\pgfqpoint{3.867281in}{1.198269in}}%
\pgfpathlineto{\pgfqpoint{3.881557in}{1.195958in}}%
\pgfpathlineto{\pgfqpoint{3.895840in}{1.193717in}}%
\pgfpathlineto{\pgfqpoint{3.910131in}{1.191547in}}%
\pgfpathlineto{\pgfqpoint{3.924430in}{1.189446in}}%
\pgfpathlineto{\pgfqpoint{3.932764in}{1.199851in}}%
\pgfpathlineto{\pgfqpoint{3.941092in}{1.210382in}}%
\pgfpathlineto{\pgfqpoint{3.949413in}{1.221033in}}%
\pgfpathlineto{\pgfqpoint{3.957729in}{1.231799in}}%
\pgfpathlineto{\pgfqpoint{3.943441in}{1.233504in}}%
\pgfpathlineto{\pgfqpoint{3.929162in}{1.235279in}}%
\pgfpathlineto{\pgfqpoint{3.914891in}{1.237125in}}%
\pgfpathlineto{\pgfqpoint{3.900627in}{1.239041in}}%
\pgfpathlineto{\pgfqpoint{3.892300in}{1.228663in}}%
\pgfpathlineto{\pgfqpoint{3.883967in}{1.218404in}}%
\pgfpathlineto{\pgfqpoint{3.875627in}{1.208271in}}%
\pgfpathlineto{\pgfqpoint{3.867281in}{1.198269in}}%
\pgfpathclose%
\pgfusepath{fill}%
\end{pgfscope}%
\begin{pgfscope}%
\pgfpathrectangle{\pgfqpoint{1.150000in}{0.150000in}}{\pgfqpoint{5.700000in}{5.700000in}}%
\pgfusepath{clip}%
\pgfsetbuttcap%
\pgfsetroundjoin%
\definecolor{currentfill}{rgb}{0.277134,0.185228,0.489898}%
\pgfsetfillcolor{currentfill}%
\pgfsetfillopacity{0.700000}%
\pgfsetlinewidth{0.000000pt}%
\definecolor{currentstroke}{rgb}{0.000000,0.000000,0.000000}%
\pgfsetstrokecolor{currentstroke}%
\pgfsetdash{}{0pt}%
\pgfpathmoveto{\pgfqpoint{4.409683in}{1.477044in}}%
\pgfpathlineto{\pgfqpoint{4.424130in}{1.478487in}}%
\pgfpathlineto{\pgfqpoint{4.438588in}{1.480002in}}%
\pgfpathlineto{\pgfqpoint{4.453056in}{1.481586in}}%
\pgfpathlineto{\pgfqpoint{4.467535in}{1.483240in}}%
\pgfpathlineto{\pgfqpoint{4.475702in}{1.496898in}}%
\pgfpathlineto{\pgfqpoint{4.483865in}{1.510532in}}%
\pgfpathlineto{\pgfqpoint{4.492022in}{1.524138in}}%
\pgfpathlineto{\pgfqpoint{4.500175in}{1.537712in}}%
\pgfpathlineto{\pgfqpoint{4.485699in}{1.535782in}}%
\pgfpathlineto{\pgfqpoint{4.471235in}{1.533923in}}%
\pgfpathlineto{\pgfqpoint{4.456780in}{1.532134in}}%
\pgfpathlineto{\pgfqpoint{4.442337in}{1.530416in}}%
\pgfpathlineto{\pgfqpoint{4.434181in}{1.517109in}}%
\pgfpathlineto{\pgfqpoint{4.426020in}{1.503775in}}%
\pgfpathlineto{\pgfqpoint{4.417854in}{1.490419in}}%
\pgfpathlineto{\pgfqpoint{4.409683in}{1.477044in}}%
\pgfpathclose%
\pgfusepath{fill}%
\end{pgfscope}%
\begin{pgfscope}%
\pgfpathrectangle{\pgfqpoint{1.150000in}{0.150000in}}{\pgfqpoint{5.700000in}{5.700000in}}%
\pgfusepath{clip}%
\pgfsetbuttcap%
\pgfsetroundjoin%
\definecolor{currentfill}{rgb}{0.177423,0.437527,0.557565}%
\pgfsetfillcolor{currentfill}%
\pgfsetfillopacity{0.700000}%
\pgfsetlinewidth{0.000000pt}%
\definecolor{currentstroke}{rgb}{0.000000,0.000000,0.000000}%
\pgfsetstrokecolor{currentstroke}%
\pgfsetdash{}{0pt}%
\pgfpathmoveto{\pgfqpoint{5.199232in}{2.104965in}}%
\pgfpathlineto{\pgfqpoint{5.214041in}{2.110483in}}%
\pgfpathlineto{\pgfqpoint{5.228863in}{2.116074in}}%
\pgfpathlineto{\pgfqpoint{5.243699in}{2.121736in}}%
\pgfpathlineto{\pgfqpoint{5.251586in}{2.132725in}}%
\pgfpathlineto{\pgfqpoint{5.259465in}{2.143556in}}%
\pgfpathlineto{\pgfqpoint{5.267334in}{2.154228in}}%
\pgfpathlineto{\pgfqpoint{5.275195in}{2.164740in}}%
\pgfpathlineto{\pgfqpoint{5.260364in}{2.159017in}}%
\pgfpathlineto{\pgfqpoint{5.245547in}{2.153366in}}%
\pgfpathlineto{\pgfqpoint{5.230745in}{2.147786in}}%
\pgfpathlineto{\pgfqpoint{5.222880in}{2.137313in}}%
\pgfpathlineto{\pgfqpoint{5.215006in}{2.126685in}}%
\pgfpathlineto{\pgfqpoint{5.207123in}{2.115902in}}%
\pgfpathlineto{\pgfqpoint{5.199232in}{2.104965in}}%
\pgfpathclose%
\pgfusepath{fill}%
\end{pgfscope}%
\begin{pgfscope}%
\pgfpathrectangle{\pgfqpoint{1.150000in}{0.150000in}}{\pgfqpoint{5.700000in}{5.700000in}}%
\pgfusepath{clip}%
\pgfsetbuttcap%
\pgfsetroundjoin%
\definecolor{currentfill}{rgb}{0.231674,0.318106,0.544834}%
\pgfsetfillcolor{currentfill}%
\pgfsetfillopacity{0.700000}%
\pgfsetlinewidth{0.000000pt}%
\definecolor{currentstroke}{rgb}{0.000000,0.000000,0.000000}%
\pgfsetstrokecolor{currentstroke}%
\pgfsetdash{}{0pt}%
\pgfpathmoveto{\pgfqpoint{4.804361in}{1.787078in}}%
\pgfpathlineto{\pgfqpoint{4.818977in}{1.790818in}}%
\pgfpathlineto{\pgfqpoint{4.833605in}{1.794629in}}%
\pgfpathlineto{\pgfqpoint{4.848246in}{1.798510in}}%
\pgfpathlineto{\pgfqpoint{4.862900in}{1.802463in}}%
\pgfpathlineto{\pgfqpoint{4.870949in}{1.815697in}}%
\pgfpathlineto{\pgfqpoint{4.878992in}{1.828823in}}%
\pgfpathlineto{\pgfqpoint{4.887028in}{1.841837in}}%
\pgfpathlineto{\pgfqpoint{4.895058in}{1.854738in}}%
\pgfpathlineto{\pgfqpoint{4.880407in}{1.850615in}}%
\pgfpathlineto{\pgfqpoint{4.865769in}{1.846562in}}%
\pgfpathlineto{\pgfqpoint{4.851143in}{1.842581in}}%
\pgfpathlineto{\pgfqpoint{4.836530in}{1.838670in}}%
\pgfpathlineto{\pgfqpoint{4.828497in}{1.825931in}}%
\pgfpathlineto{\pgfqpoint{4.820458in}{1.813085in}}%
\pgfpathlineto{\pgfqpoint{4.812413in}{1.800133in}}%
\pgfpathlineto{\pgfqpoint{4.804361in}{1.787078in}}%
\pgfpathclose%
\pgfusepath{fill}%
\end{pgfscope}%
\begin{pgfscope}%
\pgfpathrectangle{\pgfqpoint{1.150000in}{0.150000in}}{\pgfqpoint{5.700000in}{5.700000in}}%
\pgfusepath{clip}%
\pgfsetbuttcap%
\pgfsetroundjoin%
\definecolor{currentfill}{rgb}{0.270595,0.214069,0.507052}%
\pgfsetfillcolor{currentfill}%
\pgfsetfillopacity{0.700000}%
\pgfsetlinewidth{0.000000pt}%
\definecolor{currentstroke}{rgb}{0.000000,0.000000,0.000000}%
\pgfsetstrokecolor{currentstroke}%
\pgfsetdash{}{0pt}%
\pgfpathmoveto{\pgfqpoint{4.500175in}{1.537712in}}%
\pgfpathlineto{\pgfqpoint{4.514662in}{1.539712in}}%
\pgfpathlineto{\pgfqpoint{4.529159in}{1.541782in}}%
\pgfpathlineto{\pgfqpoint{4.543668in}{1.543922in}}%
\pgfpathlineto{\pgfqpoint{4.558188in}{1.546132in}}%
\pgfpathlineto{\pgfqpoint{4.566332in}{1.559933in}}%
\pgfpathlineto{\pgfqpoint{4.574472in}{1.573689in}}%
\pgfpathlineto{\pgfqpoint{4.582607in}{1.587397in}}%
\pgfpathlineto{\pgfqpoint{4.590736in}{1.601053in}}%
\pgfpathlineto{\pgfqpoint{4.576219in}{1.598587in}}%
\pgfpathlineto{\pgfqpoint{4.561713in}{1.596192in}}%
\pgfpathlineto{\pgfqpoint{4.547219in}{1.593868in}}%
\pgfpathlineto{\pgfqpoint{4.532735in}{1.591614in}}%
\pgfpathlineto{\pgfqpoint{4.524603in}{1.578205in}}%
\pgfpathlineto{\pgfqpoint{4.516465in}{1.564749in}}%
\pgfpathlineto{\pgfqpoint{4.508323in}{1.551250in}}%
\pgfpathlineto{\pgfqpoint{4.500175in}{1.537712in}}%
\pgfpathclose%
\pgfusepath{fill}%
\end{pgfscope}%
\begin{pgfscope}%
\pgfpathrectangle{\pgfqpoint{1.150000in}{0.150000in}}{\pgfqpoint{5.700000in}{5.700000in}}%
\pgfusepath{clip}%
\pgfsetbuttcap%
\pgfsetroundjoin%
\definecolor{currentfill}{rgb}{0.272594,0.025563,0.353093}%
\pgfsetfillcolor{currentfill}%
\pgfsetfillopacity{0.700000}%
\pgfsetlinewidth{0.000000pt}%
\definecolor{currentstroke}{rgb}{0.000000,0.000000,0.000000}%
\pgfsetstrokecolor{currentstroke}%
\pgfsetdash{}{0pt}%
\pgfpathmoveto{\pgfqpoint{3.776742in}{1.171319in}}%
\pgfpathlineto{\pgfqpoint{3.791003in}{1.168309in}}%
\pgfpathlineto{\pgfqpoint{3.805271in}{1.165371in}}%
\pgfpathlineto{\pgfqpoint{3.819546in}{1.162502in}}%
\pgfpathlineto{\pgfqpoint{3.833828in}{1.159705in}}%
\pgfpathlineto{\pgfqpoint{3.842201in}{1.169117in}}%
\pgfpathlineto{\pgfqpoint{3.850568in}{1.178687in}}%
\pgfpathlineto{\pgfqpoint{3.858928in}{1.188406in}}%
\pgfpathlineto{\pgfqpoint{3.867281in}{1.198269in}}%
\pgfpathlineto{\pgfqpoint{3.853013in}{1.200651in}}%
\pgfpathlineto{\pgfqpoint{3.838752in}{1.203104in}}%
\pgfpathlineto{\pgfqpoint{3.824499in}{1.205628in}}%
\pgfpathlineto{\pgfqpoint{3.810253in}{1.208222in}}%
\pgfpathlineto{\pgfqpoint{3.801886in}{1.198766in}}%
\pgfpathlineto{\pgfqpoint{3.793512in}{1.189459in}}%
\pgfpathlineto{\pgfqpoint{3.785131in}{1.180308in}}%
\pgfpathlineto{\pgfqpoint{3.776742in}{1.171319in}}%
\pgfpathclose%
\pgfusepath{fill}%
\end{pgfscope}%
\begin{pgfscope}%
\pgfpathrectangle{\pgfqpoint{1.150000in}{0.150000in}}{\pgfqpoint{5.700000in}{5.700000in}}%
\pgfusepath{clip}%
\pgfsetbuttcap%
\pgfsetroundjoin%
\definecolor{currentfill}{rgb}{0.218130,0.347432,0.550038}%
\pgfsetfillcolor{currentfill}%
\pgfsetfillopacity{0.700000}%
\pgfsetlinewidth{0.000000pt}%
\definecolor{currentstroke}{rgb}{0.000000,0.000000,0.000000}%
\pgfsetstrokecolor{currentstroke}%
\pgfsetdash{}{0pt}%
\pgfpathmoveto{\pgfqpoint{4.895058in}{1.854738in}}%
\pgfpathlineto{\pgfqpoint{4.909721in}{1.858933in}}%
\pgfpathlineto{\pgfqpoint{4.924398in}{1.863199in}}%
\pgfpathlineto{\pgfqpoint{4.939087in}{1.867536in}}%
\pgfpathlineto{\pgfqpoint{4.953789in}{1.871944in}}%
\pgfpathlineto{\pgfqpoint{4.961810in}{1.884888in}}%
\pgfpathlineto{\pgfqpoint{4.969823in}{1.897709in}}%
\pgfpathlineto{\pgfqpoint{4.977830in}{1.910405in}}%
\pgfpathlineto{\pgfqpoint{4.985830in}{1.922975in}}%
\pgfpathlineto{\pgfqpoint{4.971131in}{1.918417in}}%
\pgfpathlineto{\pgfqpoint{4.956445in}{1.913931in}}%
\pgfpathlineto{\pgfqpoint{4.941771in}{1.909516in}}%
\pgfpathlineto{\pgfqpoint{4.927111in}{1.905172in}}%
\pgfpathlineto{\pgfqpoint{4.919108in}{1.892743in}}%
\pgfpathlineto{\pgfqpoint{4.911098in}{1.880193in}}%
\pgfpathlineto{\pgfqpoint{4.903081in}{1.867524in}}%
\pgfpathlineto{\pgfqpoint{4.895058in}{1.854738in}}%
\pgfpathclose%
\pgfusepath{fill}%
\end{pgfscope}%
\begin{pgfscope}%
\pgfpathrectangle{\pgfqpoint{1.150000in}{0.150000in}}{\pgfqpoint{5.700000in}{5.700000in}}%
\pgfusepath{clip}%
\pgfsetbuttcap%
\pgfsetroundjoin%
\definecolor{currentfill}{rgb}{0.260571,0.246922,0.522828}%
\pgfsetfillcolor{currentfill}%
\pgfsetfillopacity{0.700000}%
\pgfsetlinewidth{0.000000pt}%
\definecolor{currentstroke}{rgb}{0.000000,0.000000,0.000000}%
\pgfsetstrokecolor{currentstroke}%
\pgfsetdash{}{0pt}%
\pgfpathmoveto{\pgfqpoint{4.590736in}{1.601053in}}%
\pgfpathlineto{\pgfqpoint{4.605264in}{1.603588in}}%
\pgfpathlineto{\pgfqpoint{4.619804in}{1.606194in}}%
\pgfpathlineto{\pgfqpoint{4.634356in}{1.608871in}}%
\pgfpathlineto{\pgfqpoint{4.648918in}{1.611618in}}%
\pgfpathlineto{\pgfqpoint{4.657040in}{1.625460in}}%
\pgfpathlineto{\pgfqpoint{4.665157in}{1.639239in}}%
\pgfpathlineto{\pgfqpoint{4.673268in}{1.652950in}}%
\pgfpathlineto{\pgfqpoint{4.681374in}{1.666591in}}%
\pgfpathlineto{\pgfqpoint{4.666814in}{1.663610in}}%
\pgfpathlineto{\pgfqpoint{4.652265in}{1.660700in}}%
\pgfpathlineto{\pgfqpoint{4.637727in}{1.657860in}}%
\pgfpathlineto{\pgfqpoint{4.623202in}{1.655090in}}%
\pgfpathlineto{\pgfqpoint{4.615093in}{1.641675in}}%
\pgfpathlineto{\pgfqpoint{4.606979in}{1.628195in}}%
\pgfpathlineto{\pgfqpoint{4.598860in}{1.614653in}}%
\pgfpathlineto{\pgfqpoint{4.590736in}{1.601053in}}%
\pgfpathclose%
\pgfusepath{fill}%
\end{pgfscope}%
\begin{pgfscope}%
\pgfpathrectangle{\pgfqpoint{1.150000in}{0.150000in}}{\pgfqpoint{5.700000in}{5.700000in}}%
\pgfusepath{clip}%
\pgfsetbuttcap%
\pgfsetroundjoin%
\definecolor{currentfill}{rgb}{0.281924,0.089666,0.412415}%
\pgfsetfillcolor{currentfill}%
\pgfsetfillopacity{0.700000}%
\pgfsetlinewidth{0.000000pt}%
\definecolor{currentstroke}{rgb}{0.000000,0.000000,0.000000}%
\pgfsetstrokecolor{currentstroke}%
\pgfsetdash{}{0pt}%
\pgfpathmoveto{\pgfqpoint{4.105448in}{1.267789in}}%
\pgfpathlineto{\pgfqpoint{4.119802in}{1.267091in}}%
\pgfpathlineto{\pgfqpoint{4.134165in}{1.266463in}}%
\pgfpathlineto{\pgfqpoint{4.148537in}{1.265904in}}%
\pgfpathlineto{\pgfqpoint{4.162917in}{1.265415in}}%
\pgfpathlineto{\pgfqpoint{4.171177in}{1.277759in}}%
\pgfpathlineto{\pgfqpoint{4.179432in}{1.290169in}}%
\pgfpathlineto{\pgfqpoint{4.187682in}{1.302639in}}%
\pgfpathlineto{\pgfqpoint{4.195927in}{1.315165in}}%
\pgfpathlineto{\pgfqpoint{4.181553in}{1.315298in}}%
\pgfpathlineto{\pgfqpoint{4.167189in}{1.315500in}}%
\pgfpathlineto{\pgfqpoint{4.152834in}{1.315773in}}%
\pgfpathlineto{\pgfqpoint{4.138488in}{1.316115in}}%
\pgfpathlineto{\pgfqpoint{4.130236in}{1.303938in}}%
\pgfpathlineto{\pgfqpoint{4.121978in}{1.291821in}}%
\pgfpathlineto{\pgfqpoint{4.113716in}{1.279769in}}%
\pgfpathlineto{\pgfqpoint{4.105448in}{1.267789in}}%
\pgfpathclose%
\pgfusepath{fill}%
\end{pgfscope}%
\begin{pgfscope}%
\pgfpathrectangle{\pgfqpoint{1.150000in}{0.150000in}}{\pgfqpoint{5.700000in}{5.700000in}}%
\pgfusepath{clip}%
\pgfsetbuttcap%
\pgfsetroundjoin%
\definecolor{currentfill}{rgb}{0.204903,0.375746,0.553533}%
\pgfsetfillcolor{currentfill}%
\pgfsetfillopacity{0.700000}%
\pgfsetlinewidth{0.000000pt}%
\definecolor{currentstroke}{rgb}{0.000000,0.000000,0.000000}%
\pgfsetstrokecolor{currentstroke}%
\pgfsetdash{}{0pt}%
\pgfpathmoveto{\pgfqpoint{4.985830in}{1.922975in}}%
\pgfpathlineto{\pgfqpoint{5.000543in}{1.927604in}}%
\pgfpathlineto{\pgfqpoint{5.015268in}{1.932304in}}%
\pgfpathlineto{\pgfqpoint{5.030007in}{1.937076in}}%
\pgfpathlineto{\pgfqpoint{5.044759in}{1.941920in}}%
\pgfpathlineto{\pgfqpoint{5.052749in}{1.954497in}}%
\pgfpathlineto{\pgfqpoint{5.060732in}{1.966939in}}%
\pgfpathlineto{\pgfqpoint{5.068707in}{1.979244in}}%
\pgfpathlineto{\pgfqpoint{5.076675in}{1.991411in}}%
\pgfpathlineto{\pgfqpoint{5.061926in}{1.986440in}}%
\pgfpathlineto{\pgfqpoint{5.047190in}{1.981541in}}%
\pgfpathlineto{\pgfqpoint{5.032468in}{1.976713in}}%
\pgfpathlineto{\pgfqpoint{5.017759in}{1.971956in}}%
\pgfpathlineto{\pgfqpoint{5.009788in}{1.959909in}}%
\pgfpathlineto{\pgfqpoint{5.001809in}{1.947729in}}%
\pgfpathlineto{\pgfqpoint{4.993823in}{1.935417in}}%
\pgfpathlineto{\pgfqpoint{4.985830in}{1.922975in}}%
\pgfpathclose%
\pgfusepath{fill}%
\end{pgfscope}%
\begin{pgfscope}%
\pgfpathrectangle{\pgfqpoint{1.150000in}{0.150000in}}{\pgfqpoint{5.700000in}{5.700000in}}%
\pgfusepath{clip}%
\pgfsetbuttcap%
\pgfsetroundjoin%
\definecolor{currentfill}{rgb}{0.283197,0.115680,0.436115}%
\pgfsetfillcolor{currentfill}%
\pgfsetfillopacity{0.700000}%
\pgfsetlinewidth{0.000000pt}%
\definecolor{currentstroke}{rgb}{0.000000,0.000000,0.000000}%
\pgfsetstrokecolor{currentstroke}%
\pgfsetdash{}{0pt}%
\pgfpathmoveto{\pgfqpoint{4.195927in}{1.315165in}}%
\pgfpathlineto{\pgfqpoint{4.210310in}{1.315102in}}%
\pgfpathlineto{\pgfqpoint{4.224702in}{1.315109in}}%
\pgfpathlineto{\pgfqpoint{4.239104in}{1.315186in}}%
\pgfpathlineto{\pgfqpoint{4.253515in}{1.315332in}}%
\pgfpathlineto{\pgfqpoint{4.261749in}{1.328252in}}%
\pgfpathlineto{\pgfqpoint{4.269978in}{1.341212in}}%
\pgfpathlineto{\pgfqpoint{4.278202in}{1.354207in}}%
\pgfpathlineto{\pgfqpoint{4.286421in}{1.367232in}}%
\pgfpathlineto{\pgfqpoint{4.272015in}{1.366749in}}%
\pgfpathlineto{\pgfqpoint{4.257619in}{1.366336in}}%
\pgfpathlineto{\pgfqpoint{4.243233in}{1.365994in}}%
\pgfpathlineto{\pgfqpoint{4.228856in}{1.365721in}}%
\pgfpathlineto{\pgfqpoint{4.220631in}{1.353024in}}%
\pgfpathlineto{\pgfqpoint{4.212401in}{1.340362in}}%
\pgfpathlineto{\pgfqpoint{4.204167in}{1.327741in}}%
\pgfpathlineto{\pgfqpoint{4.195927in}{1.315165in}}%
\pgfpathclose%
\pgfusepath{fill}%
\end{pgfscope}%
\begin{pgfscope}%
\pgfpathrectangle{\pgfqpoint{1.150000in}{0.150000in}}{\pgfqpoint{5.700000in}{5.700000in}}%
\pgfusepath{clip}%
\pgfsetbuttcap%
\pgfsetroundjoin%
\definecolor{currentfill}{rgb}{0.250425,0.274290,0.533103}%
\pgfsetfillcolor{currentfill}%
\pgfsetfillopacity{0.700000}%
\pgfsetlinewidth{0.000000pt}%
\definecolor{currentstroke}{rgb}{0.000000,0.000000,0.000000}%
\pgfsetstrokecolor{currentstroke}%
\pgfsetdash{}{0pt}%
\pgfpathmoveto{\pgfqpoint{4.681374in}{1.666591in}}%
\pgfpathlineto{\pgfqpoint{4.695947in}{1.669643in}}%
\pgfpathlineto{\pgfqpoint{4.710531in}{1.672766in}}%
\pgfpathlineto{\pgfqpoint{4.725127in}{1.675958in}}%
\pgfpathlineto{\pgfqpoint{4.739735in}{1.679222in}}%
\pgfpathlineto{\pgfqpoint{4.747833in}{1.693011in}}%
\pgfpathlineto{\pgfqpoint{4.755926in}{1.706718in}}%
\pgfpathlineto{\pgfqpoint{4.764013in}{1.720339in}}%
\pgfpathlineto{\pgfqpoint{4.772095in}{1.733874in}}%
\pgfpathlineto{\pgfqpoint{4.757489in}{1.730396in}}%
\pgfpathlineto{\pgfqpoint{4.742894in}{1.726990in}}%
\pgfpathlineto{\pgfqpoint{4.728313in}{1.723655in}}%
\pgfpathlineto{\pgfqpoint{4.713743in}{1.720390in}}%
\pgfpathlineto{\pgfqpoint{4.705659in}{1.707061in}}%
\pgfpathlineto{\pgfqpoint{4.697570in}{1.693650in}}%
\pgfpathlineto{\pgfqpoint{4.689475in}{1.680159in}}%
\pgfpathlineto{\pgfqpoint{4.681374in}{1.666591in}}%
\pgfpathclose%
\pgfusepath{fill}%
\end{pgfscope}%
\begin{pgfscope}%
\pgfpathrectangle{\pgfqpoint{1.150000in}{0.150000in}}{\pgfqpoint{5.700000in}{5.700000in}}%
\pgfusepath{clip}%
\pgfsetbuttcap%
\pgfsetroundjoin%
\definecolor{currentfill}{rgb}{0.279566,0.067836,0.391917}%
\pgfsetfillcolor{currentfill}%
\pgfsetfillopacity{0.700000}%
\pgfsetlinewidth{0.000000pt}%
\definecolor{currentstroke}{rgb}{0.000000,0.000000,0.000000}%
\pgfsetstrokecolor{currentstroke}%
\pgfsetdash{}{0pt}%
\pgfpathmoveto{\pgfqpoint{4.014959in}{1.225683in}}%
\pgfpathlineto{\pgfqpoint{4.029288in}{1.224329in}}%
\pgfpathlineto{\pgfqpoint{4.043624in}{1.223045in}}%
\pgfpathlineto{\pgfqpoint{4.057970in}{1.221831in}}%
\pgfpathlineto{\pgfqpoint{4.072324in}{1.220687in}}%
\pgfpathlineto{\pgfqpoint{4.080613in}{1.232329in}}%
\pgfpathlineto{\pgfqpoint{4.088897in}{1.244063in}}%
\pgfpathlineto{\pgfqpoint{4.097175in}{1.255885in}}%
\pgfpathlineto{\pgfqpoint{4.105448in}{1.267789in}}%
\pgfpathlineto{\pgfqpoint{4.091103in}{1.268557in}}%
\pgfpathlineto{\pgfqpoint{4.076767in}{1.269395in}}%
\pgfpathlineto{\pgfqpoint{4.062439in}{1.270304in}}%
\pgfpathlineto{\pgfqpoint{4.048121in}{1.271282in}}%
\pgfpathlineto{\pgfqpoint{4.039839in}{1.259746in}}%
\pgfpathlineto{\pgfqpoint{4.031551in}{1.248297in}}%
\pgfpathlineto{\pgfqpoint{4.023258in}{1.236941in}}%
\pgfpathlineto{\pgfqpoint{4.014959in}{1.225683in}}%
\pgfpathclose%
\pgfusepath{fill}%
\end{pgfscope}%
\begin{pgfscope}%
\pgfpathrectangle{\pgfqpoint{1.150000in}{0.150000in}}{\pgfqpoint{5.700000in}{5.700000in}}%
\pgfusepath{clip}%
\pgfsetbuttcap%
\pgfsetroundjoin%
\definecolor{currentfill}{rgb}{0.282623,0.140926,0.457517}%
\pgfsetfillcolor{currentfill}%
\pgfsetfillopacity{0.700000}%
\pgfsetlinewidth{0.000000pt}%
\definecolor{currentstroke}{rgb}{0.000000,0.000000,0.000000}%
\pgfsetstrokecolor{currentstroke}%
\pgfsetdash{}{0pt}%
\pgfpathmoveto{\pgfqpoint{4.286421in}{1.367232in}}%
\pgfpathlineto{\pgfqpoint{4.300836in}{1.367785in}}%
\pgfpathlineto{\pgfqpoint{4.315262in}{1.368407in}}%
\pgfpathlineto{\pgfqpoint{4.329697in}{1.369100in}}%
\pgfpathlineto{\pgfqpoint{4.344142in}{1.369862in}}%
\pgfpathlineto{\pgfqpoint{4.352352in}{1.383236in}}%
\pgfpathlineto{\pgfqpoint{4.360556in}{1.396626in}}%
\pgfpathlineto{\pgfqpoint{4.368756in}{1.410028in}}%
\pgfpathlineto{\pgfqpoint{4.376951in}{1.423435in}}%
\pgfpathlineto{\pgfqpoint{4.362510in}{1.422357in}}%
\pgfpathlineto{\pgfqpoint{4.348079in}{1.421348in}}%
\pgfpathlineto{\pgfqpoint{4.333659in}{1.420410in}}%
\pgfpathlineto{\pgfqpoint{4.319248in}{1.419542in}}%
\pgfpathlineto{\pgfqpoint{4.311049in}{1.406442in}}%
\pgfpathlineto{\pgfqpoint{4.302844in}{1.393354in}}%
\pgfpathlineto{\pgfqpoint{4.294635in}{1.380283in}}%
\pgfpathlineto{\pgfqpoint{4.286421in}{1.367232in}}%
\pgfpathclose%
\pgfusepath{fill}%
\end{pgfscope}%
\begin{pgfscope}%
\pgfpathrectangle{\pgfqpoint{1.150000in}{0.150000in}}{\pgfqpoint{5.700000in}{5.700000in}}%
\pgfusepath{clip}%
\pgfsetbuttcap%
\pgfsetroundjoin%
\definecolor{currentfill}{rgb}{0.276022,0.044167,0.370164}%
\pgfsetfillcolor{currentfill}%
\pgfsetfillopacity{0.700000}%
\pgfsetlinewidth{0.000000pt}%
\definecolor{currentstroke}{rgb}{0.000000,0.000000,0.000000}%
\pgfsetstrokecolor{currentstroke}%
\pgfsetdash{}{0pt}%
\pgfpathmoveto{\pgfqpoint{3.924430in}{1.189446in}}%
\pgfpathlineto{\pgfqpoint{3.938737in}{1.187417in}}%
\pgfpathlineto{\pgfqpoint{3.953052in}{1.185457in}}%
\pgfpathlineto{\pgfqpoint{3.967375in}{1.183567in}}%
\pgfpathlineto{\pgfqpoint{3.981706in}{1.181747in}}%
\pgfpathlineto{\pgfqpoint{3.990028in}{1.192555in}}%
\pgfpathlineto{\pgfqpoint{3.998344in}{1.203484in}}%
\pgfpathlineto{\pgfqpoint{4.006655in}{1.214529in}}%
\pgfpathlineto{\pgfqpoint{4.014959in}{1.225683in}}%
\pgfpathlineto{\pgfqpoint{4.000639in}{1.227107in}}%
\pgfpathlineto{\pgfqpoint{3.986327in}{1.228601in}}%
\pgfpathlineto{\pgfqpoint{3.972024in}{1.230165in}}%
\pgfpathlineto{\pgfqpoint{3.957729in}{1.231799in}}%
\pgfpathlineto{\pgfqpoint{3.949413in}{1.221033in}}%
\pgfpathlineto{\pgfqpoint{3.941092in}{1.210382in}}%
\pgfpathlineto{\pgfqpoint{3.932764in}{1.199851in}}%
\pgfpathlineto{\pgfqpoint{3.924430in}{1.189446in}}%
\pgfpathclose%
\pgfusepath{fill}%
\end{pgfscope}%
\begin{pgfscope}%
\pgfpathrectangle{\pgfqpoint{1.150000in}{0.150000in}}{\pgfqpoint{5.700000in}{5.700000in}}%
\pgfusepath{clip}%
\pgfsetbuttcap%
\pgfsetroundjoin%
\definecolor{currentfill}{rgb}{0.279574,0.170599,0.479997}%
\pgfsetfillcolor{currentfill}%
\pgfsetfillopacity{0.700000}%
\pgfsetlinewidth{0.000000pt}%
\definecolor{currentstroke}{rgb}{0.000000,0.000000,0.000000}%
\pgfsetstrokecolor{currentstroke}%
\pgfsetdash{}{0pt}%
\pgfpathmoveto{\pgfqpoint{4.376951in}{1.423435in}}%
\pgfpathlineto{\pgfqpoint{4.391402in}{1.424584in}}%
\pgfpathlineto{\pgfqpoint{4.405864in}{1.425802in}}%
\pgfpathlineto{\pgfqpoint{4.420336in}{1.427090in}}%
\pgfpathlineto{\pgfqpoint{4.434818in}{1.428448in}}%
\pgfpathlineto{\pgfqpoint{4.443004in}{1.442162in}}%
\pgfpathlineto{\pgfqpoint{4.451186in}{1.455868in}}%
\pgfpathlineto{\pgfqpoint{4.459363in}{1.469562in}}%
\pgfpathlineto{\pgfqpoint{4.467535in}{1.483240in}}%
\pgfpathlineto{\pgfqpoint{4.453056in}{1.481586in}}%
\pgfpathlineto{\pgfqpoint{4.438588in}{1.480002in}}%
\pgfpathlineto{\pgfqpoint{4.424130in}{1.478487in}}%
\pgfpathlineto{\pgfqpoint{4.409683in}{1.477044in}}%
\pgfpathlineto{\pgfqpoint{4.401507in}{1.463653in}}%
\pgfpathlineto{\pgfqpoint{4.393327in}{1.450253in}}%
\pgfpathlineto{\pgfqpoint{4.385141in}{1.436845in}}%
\pgfpathlineto{\pgfqpoint{4.376951in}{1.423435in}}%
\pgfpathclose%
\pgfusepath{fill}%
\end{pgfscope}%
\begin{pgfscope}%
\pgfpathrectangle{\pgfqpoint{1.150000in}{0.150000in}}{\pgfqpoint{5.700000in}{5.700000in}}%
\pgfusepath{clip}%
\pgfsetbuttcap%
\pgfsetroundjoin%
\definecolor{currentfill}{rgb}{0.190631,0.407061,0.556089}%
\pgfsetfillcolor{currentfill}%
\pgfsetfillopacity{0.700000}%
\pgfsetlinewidth{0.000000pt}%
\definecolor{currentstroke}{rgb}{0.000000,0.000000,0.000000}%
\pgfsetstrokecolor{currentstroke}%
\pgfsetdash{}{0pt}%
\pgfpathmoveto{\pgfqpoint{5.076675in}{1.991411in}}%
\pgfpathlineto{\pgfqpoint{5.091437in}{1.996454in}}%
\pgfpathlineto{\pgfqpoint{5.106213in}{2.001568in}}%
\pgfpathlineto{\pgfqpoint{5.121003in}{2.006754in}}%
\pgfpathlineto{\pgfqpoint{5.135806in}{2.012011in}}%
\pgfpathlineto{\pgfqpoint{5.143762in}{2.024152in}}%
\pgfpathlineto{\pgfqpoint{5.151711in}{2.036146in}}%
\pgfpathlineto{\pgfqpoint{5.159652in}{2.047992in}}%
\pgfpathlineto{\pgfqpoint{5.167584in}{2.059689in}}%
\pgfpathlineto{\pgfqpoint{5.152785in}{2.054325in}}%
\pgfpathlineto{\pgfqpoint{5.137999in}{2.049034in}}%
\pgfpathlineto{\pgfqpoint{5.123227in}{2.043814in}}%
\pgfpathlineto{\pgfqpoint{5.108469in}{2.038666in}}%
\pgfpathlineto{\pgfqpoint{5.100532in}{2.027067in}}%
\pgfpathlineto{\pgfqpoint{5.092587in}{2.015323in}}%
\pgfpathlineto{\pgfqpoint{5.084635in}{2.003438in}}%
\pgfpathlineto{\pgfqpoint{5.076675in}{1.991411in}}%
\pgfpathclose%
\pgfusepath{fill}%
\end{pgfscope}%
\begin{pgfscope}%
\pgfpathrectangle{\pgfqpoint{1.150000in}{0.150000in}}{\pgfqpoint{5.700000in}{5.700000in}}%
\pgfusepath{clip}%
\pgfsetbuttcap%
\pgfsetroundjoin%
\definecolor{currentfill}{rgb}{0.237441,0.305202,0.541921}%
\pgfsetfillcolor{currentfill}%
\pgfsetfillopacity{0.700000}%
\pgfsetlinewidth{0.000000pt}%
\definecolor{currentstroke}{rgb}{0.000000,0.000000,0.000000}%
\pgfsetstrokecolor{currentstroke}%
\pgfsetdash{}{0pt}%
\pgfpathmoveto{\pgfqpoint{4.772095in}{1.733874in}}%
\pgfpathlineto{\pgfqpoint{4.786713in}{1.737421in}}%
\pgfpathlineto{\pgfqpoint{4.801344in}{1.741040in}}%
\pgfpathlineto{\pgfqpoint{4.815986in}{1.744730in}}%
\pgfpathlineto{\pgfqpoint{4.830642in}{1.748490in}}%
\pgfpathlineto{\pgfqpoint{4.838715in}{1.762134in}}%
\pgfpathlineto{\pgfqpoint{4.846783in}{1.775679in}}%
\pgfpathlineto{\pgfqpoint{4.854844in}{1.789123in}}%
\pgfpathlineto{\pgfqpoint{4.862900in}{1.802463in}}%
\pgfpathlineto{\pgfqpoint{4.848246in}{1.798510in}}%
\pgfpathlineto{\pgfqpoint{4.833605in}{1.794629in}}%
\pgfpathlineto{\pgfqpoint{4.818977in}{1.790818in}}%
\pgfpathlineto{\pgfqpoint{4.804361in}{1.787078in}}%
\pgfpathlineto{\pgfqpoint{4.796303in}{1.773922in}}%
\pgfpathlineto{\pgfqpoint{4.788240in}{1.760667in}}%
\pgfpathlineto{\pgfqpoint{4.780170in}{1.747317in}}%
\pgfpathlineto{\pgfqpoint{4.772095in}{1.733874in}}%
\pgfpathclose%
\pgfusepath{fill}%
\end{pgfscope}%
\begin{pgfscope}%
\pgfpathrectangle{\pgfqpoint{1.150000in}{0.150000in}}{\pgfqpoint{5.700000in}{5.700000in}}%
\pgfusepath{clip}%
\pgfsetbuttcap%
\pgfsetroundjoin%
\definecolor{currentfill}{rgb}{0.180629,0.429975,0.557282}%
\pgfsetfillcolor{currentfill}%
\pgfsetfillopacity{0.700000}%
\pgfsetlinewidth{0.000000pt}%
\definecolor{currentstroke}{rgb}{0.000000,0.000000,0.000000}%
\pgfsetstrokecolor{currentstroke}%
\pgfsetdash{}{0pt}%
\pgfpathmoveto{\pgfqpoint{5.167584in}{2.059689in}}%
\pgfpathlineto{\pgfqpoint{5.182397in}{2.065124in}}%
\pgfpathlineto{\pgfqpoint{5.197224in}{2.070631in}}%
\pgfpathlineto{\pgfqpoint{5.212066in}{2.076210in}}%
\pgfpathlineto{\pgfqpoint{5.219987in}{2.087825in}}%
\pgfpathlineto{\pgfqpoint{5.227899in}{2.099285in}}%
\pgfpathlineto{\pgfqpoint{5.235804in}{2.110589in}}%
\pgfpathlineto{\pgfqpoint{5.243699in}{2.121736in}}%
\pgfpathlineto{\pgfqpoint{5.228863in}{2.116074in}}%
\pgfpathlineto{\pgfqpoint{5.214041in}{2.110483in}}%
\pgfpathlineto{\pgfqpoint{5.199232in}{2.104965in}}%
\pgfpathlineto{\pgfqpoint{5.191333in}{2.093874in}}%
\pgfpathlineto{\pgfqpoint{5.183425in}{2.082631in}}%
\pgfpathlineto{\pgfqpoint{5.175509in}{2.071235in}}%
\pgfpathlineto{\pgfqpoint{5.167584in}{2.059689in}}%
\pgfpathclose%
\pgfusepath{fill}%
\end{pgfscope}%
\begin{pgfscope}%
\pgfpathrectangle{\pgfqpoint{1.150000in}{0.150000in}}{\pgfqpoint{5.700000in}{5.700000in}}%
\pgfusepath{clip}%
\pgfsetbuttcap%
\pgfsetroundjoin%
\definecolor{currentfill}{rgb}{0.273809,0.031497,0.358853}%
\pgfsetfillcolor{currentfill}%
\pgfsetfillopacity{0.700000}%
\pgfsetlinewidth{0.000000pt}%
\definecolor{currentstroke}{rgb}{0.000000,0.000000,0.000000}%
\pgfsetstrokecolor{currentstroke}%
\pgfsetdash{}{0pt}%
\pgfpathmoveto{\pgfqpoint{3.833828in}{1.159705in}}%
\pgfpathlineto{\pgfqpoint{3.848117in}{1.156978in}}%
\pgfpathlineto{\pgfqpoint{3.862414in}{1.154321in}}%
\pgfpathlineto{\pgfqpoint{3.876719in}{1.151735in}}%
\pgfpathlineto{\pgfqpoint{3.891031in}{1.149218in}}%
\pgfpathlineto{\pgfqpoint{3.899390in}{1.159055in}}%
\pgfpathlineto{\pgfqpoint{3.907744in}{1.169042in}}%
\pgfpathlineto{\pgfqpoint{3.916090in}{1.179175in}}%
\pgfpathlineto{\pgfqpoint{3.924430in}{1.189446in}}%
\pgfpathlineto{\pgfqpoint{3.910131in}{1.191547in}}%
\pgfpathlineto{\pgfqpoint{3.895840in}{1.193717in}}%
\pgfpathlineto{\pgfqpoint{3.881557in}{1.195958in}}%
\pgfpathlineto{\pgfqpoint{3.867281in}{1.198269in}}%
\pgfpathlineto{\pgfqpoint{3.858928in}{1.188406in}}%
\pgfpathlineto{\pgfqpoint{3.850568in}{1.178687in}}%
\pgfpathlineto{\pgfqpoint{3.842201in}{1.169117in}}%
\pgfpathlineto{\pgfqpoint{3.833828in}{1.159705in}}%
\pgfpathclose%
\pgfusepath{fill}%
\end{pgfscope}%
\begin{pgfscope}%
\pgfpathrectangle{\pgfqpoint{1.150000in}{0.150000in}}{\pgfqpoint{5.700000in}{5.700000in}}%
\pgfusepath{clip}%
\pgfsetbuttcap%
\pgfsetroundjoin%
\definecolor{currentfill}{rgb}{0.274128,0.199721,0.498911}%
\pgfsetfillcolor{currentfill}%
\pgfsetfillopacity{0.700000}%
\pgfsetlinewidth{0.000000pt}%
\definecolor{currentstroke}{rgb}{0.000000,0.000000,0.000000}%
\pgfsetstrokecolor{currentstroke}%
\pgfsetdash{}{0pt}%
\pgfpathmoveto{\pgfqpoint{4.467535in}{1.483240in}}%
\pgfpathlineto{\pgfqpoint{4.482025in}{1.484964in}}%
\pgfpathlineto{\pgfqpoint{4.496525in}{1.486759in}}%
\pgfpathlineto{\pgfqpoint{4.511037in}{1.488623in}}%
\pgfpathlineto{\pgfqpoint{4.525559in}{1.490557in}}%
\pgfpathlineto{\pgfqpoint{4.533724in}{1.504499in}}%
\pgfpathlineto{\pgfqpoint{4.541883in}{1.518411in}}%
\pgfpathlineto{\pgfqpoint{4.550038in}{1.532290in}}%
\pgfpathlineto{\pgfqpoint{4.558188in}{1.546132in}}%
\pgfpathlineto{\pgfqpoint{4.543668in}{1.543922in}}%
\pgfpathlineto{\pgfqpoint{4.529159in}{1.541782in}}%
\pgfpathlineto{\pgfqpoint{4.514662in}{1.539712in}}%
\pgfpathlineto{\pgfqpoint{4.500175in}{1.537712in}}%
\pgfpathlineto{\pgfqpoint{4.492022in}{1.524138in}}%
\pgfpathlineto{\pgfqpoint{4.483865in}{1.510532in}}%
\pgfpathlineto{\pgfqpoint{4.475702in}{1.496898in}}%
\pgfpathlineto{\pgfqpoint{4.467535in}{1.483240in}}%
\pgfpathclose%
\pgfusepath{fill}%
\end{pgfscope}%
\begin{pgfscope}%
\pgfpathrectangle{\pgfqpoint{1.150000in}{0.150000in}}{\pgfqpoint{5.700000in}{5.700000in}}%
\pgfusepath{clip}%
\pgfsetbuttcap%
\pgfsetroundjoin%
\definecolor{currentfill}{rgb}{0.221989,0.339161,0.548752}%
\pgfsetfillcolor{currentfill}%
\pgfsetfillopacity{0.700000}%
\pgfsetlinewidth{0.000000pt}%
\definecolor{currentstroke}{rgb}{0.000000,0.000000,0.000000}%
\pgfsetstrokecolor{currentstroke}%
\pgfsetdash{}{0pt}%
\pgfpathmoveto{\pgfqpoint{4.862900in}{1.802463in}}%
\pgfpathlineto{\pgfqpoint{4.877566in}{1.806487in}}%
\pgfpathlineto{\pgfqpoint{4.892244in}{1.810582in}}%
\pgfpathlineto{\pgfqpoint{4.906936in}{1.814748in}}%
\pgfpathlineto{\pgfqpoint{4.921640in}{1.818985in}}%
\pgfpathlineto{\pgfqpoint{4.929687in}{1.832398in}}%
\pgfpathlineto{\pgfqpoint{4.937728in}{1.845697in}}%
\pgfpathlineto{\pgfqpoint{4.945762in}{1.858880in}}%
\pgfpathlineto{\pgfqpoint{4.953789in}{1.871944in}}%
\pgfpathlineto{\pgfqpoint{4.939087in}{1.867536in}}%
\pgfpathlineto{\pgfqpoint{4.924398in}{1.863199in}}%
\pgfpathlineto{\pgfqpoint{4.909721in}{1.858933in}}%
\pgfpathlineto{\pgfqpoint{4.895058in}{1.854738in}}%
\pgfpathlineto{\pgfqpoint{4.887028in}{1.841837in}}%
\pgfpathlineto{\pgfqpoint{4.878992in}{1.828823in}}%
\pgfpathlineto{\pgfqpoint{4.870949in}{1.815697in}}%
\pgfpathlineto{\pgfqpoint{4.862900in}{1.802463in}}%
\pgfpathclose%
\pgfusepath{fill}%
\end{pgfscope}%
\begin{pgfscope}%
\pgfpathrectangle{\pgfqpoint{1.150000in}{0.150000in}}{\pgfqpoint{5.700000in}{5.700000in}}%
\pgfusepath{clip}%
\pgfsetbuttcap%
\pgfsetroundjoin%
\definecolor{currentfill}{rgb}{0.266580,0.228262,0.514349}%
\pgfsetfillcolor{currentfill}%
\pgfsetfillopacity{0.700000}%
\pgfsetlinewidth{0.000000pt}%
\definecolor{currentstroke}{rgb}{0.000000,0.000000,0.000000}%
\pgfsetstrokecolor{currentstroke}%
\pgfsetdash{}{0pt}%
\pgfpathmoveto{\pgfqpoint{4.558188in}{1.546132in}}%
\pgfpathlineto{\pgfqpoint{4.572718in}{1.548413in}}%
\pgfpathlineto{\pgfqpoint{4.587261in}{1.550764in}}%
\pgfpathlineto{\pgfqpoint{4.601814in}{1.553185in}}%
\pgfpathlineto{\pgfqpoint{4.616379in}{1.555676in}}%
\pgfpathlineto{\pgfqpoint{4.624521in}{1.569740in}}%
\pgfpathlineto{\pgfqpoint{4.632659in}{1.583754in}}%
\pgfpathlineto{\pgfqpoint{4.640791in}{1.597714in}}%
\pgfpathlineto{\pgfqpoint{4.648918in}{1.611618in}}%
\pgfpathlineto{\pgfqpoint{4.634356in}{1.608871in}}%
\pgfpathlineto{\pgfqpoint{4.619804in}{1.606194in}}%
\pgfpathlineto{\pgfqpoint{4.605264in}{1.603588in}}%
\pgfpathlineto{\pgfqpoint{4.590736in}{1.601053in}}%
\pgfpathlineto{\pgfqpoint{4.582607in}{1.587397in}}%
\pgfpathlineto{\pgfqpoint{4.574472in}{1.573689in}}%
\pgfpathlineto{\pgfqpoint{4.566332in}{1.559933in}}%
\pgfpathlineto{\pgfqpoint{4.558188in}{1.546132in}}%
\pgfpathclose%
\pgfusepath{fill}%
\end{pgfscope}%
\begin{pgfscope}%
\pgfpathrectangle{\pgfqpoint{1.150000in}{0.150000in}}{\pgfqpoint{5.700000in}{5.700000in}}%
\pgfusepath{clip}%
\pgfsetbuttcap%
\pgfsetroundjoin%
\definecolor{currentfill}{rgb}{0.255645,0.260703,0.528312}%
\pgfsetfillcolor{currentfill}%
\pgfsetfillopacity{0.700000}%
\pgfsetlinewidth{0.000000pt}%
\definecolor{currentstroke}{rgb}{0.000000,0.000000,0.000000}%
\pgfsetstrokecolor{currentstroke}%
\pgfsetdash{}{0pt}%
\pgfpathmoveto{\pgfqpoint{4.648918in}{1.611618in}}%
\pgfpathlineto{\pgfqpoint{4.663493in}{1.614435in}}%
\pgfpathlineto{\pgfqpoint{4.678079in}{1.617323in}}%
\pgfpathlineto{\pgfqpoint{4.692677in}{1.620281in}}%
\pgfpathlineto{\pgfqpoint{4.707287in}{1.623309in}}%
\pgfpathlineto{\pgfqpoint{4.715407in}{1.637394in}}%
\pgfpathlineto{\pgfqpoint{4.723522in}{1.651411in}}%
\pgfpathlineto{\pgfqpoint{4.731631in}{1.665354in}}%
\pgfpathlineto{\pgfqpoint{4.739735in}{1.679222in}}%
\pgfpathlineto{\pgfqpoint{4.725127in}{1.675958in}}%
\pgfpathlineto{\pgfqpoint{4.710531in}{1.672766in}}%
\pgfpathlineto{\pgfqpoint{4.695947in}{1.669643in}}%
\pgfpathlineto{\pgfqpoint{4.681374in}{1.666591in}}%
\pgfpathlineto{\pgfqpoint{4.673268in}{1.652950in}}%
\pgfpathlineto{\pgfqpoint{4.665157in}{1.639239in}}%
\pgfpathlineto{\pgfqpoint{4.657040in}{1.625460in}}%
\pgfpathlineto{\pgfqpoint{4.648918in}{1.611618in}}%
\pgfpathclose%
\pgfusepath{fill}%
\end{pgfscope}%
\begin{pgfscope}%
\pgfpathrectangle{\pgfqpoint{1.150000in}{0.150000in}}{\pgfqpoint{5.700000in}{5.700000in}}%
\pgfusepath{clip}%
\pgfsetbuttcap%
\pgfsetroundjoin%
\definecolor{currentfill}{rgb}{0.208623,0.367752,0.552675}%
\pgfsetfillcolor{currentfill}%
\pgfsetfillopacity{0.700000}%
\pgfsetlinewidth{0.000000pt}%
\definecolor{currentstroke}{rgb}{0.000000,0.000000,0.000000}%
\pgfsetstrokecolor{currentstroke}%
\pgfsetdash{}{0pt}%
\pgfpathmoveto{\pgfqpoint{4.953789in}{1.871944in}}%
\pgfpathlineto{\pgfqpoint{4.968504in}{1.876424in}}%
\pgfpathlineto{\pgfqpoint{4.983232in}{1.880975in}}%
\pgfpathlineto{\pgfqpoint{4.997974in}{1.885597in}}%
\pgfpathlineto{\pgfqpoint{5.012728in}{1.890290in}}%
\pgfpathlineto{\pgfqpoint{5.020747in}{1.903392in}}%
\pgfpathlineto{\pgfqpoint{5.028758in}{1.916365in}}%
\pgfpathlineto{\pgfqpoint{5.036762in}{1.929208in}}%
\pgfpathlineto{\pgfqpoint{5.044759in}{1.941920in}}%
\pgfpathlineto{\pgfqpoint{5.030007in}{1.937076in}}%
\pgfpathlineto{\pgfqpoint{5.015268in}{1.932304in}}%
\pgfpathlineto{\pgfqpoint{5.000543in}{1.927604in}}%
\pgfpathlineto{\pgfqpoint{4.985830in}{1.922975in}}%
\pgfpathlineto{\pgfqpoint{4.977830in}{1.910405in}}%
\pgfpathlineto{\pgfqpoint{4.969823in}{1.897709in}}%
\pgfpathlineto{\pgfqpoint{4.961810in}{1.884888in}}%
\pgfpathlineto{\pgfqpoint{4.953789in}{1.871944in}}%
\pgfpathclose%
\pgfusepath{fill}%
\end{pgfscope}%
\begin{pgfscope}%
\pgfpathrectangle{\pgfqpoint{1.150000in}{0.150000in}}{\pgfqpoint{5.700000in}{5.700000in}}%
\pgfusepath{clip}%
\pgfsetbuttcap%
\pgfsetroundjoin%
\definecolor{currentfill}{rgb}{0.282656,0.100196,0.422160}%
\pgfsetfillcolor{currentfill}%
\pgfsetfillopacity{0.700000}%
\pgfsetlinewidth{0.000000pt}%
\definecolor{currentstroke}{rgb}{0.000000,0.000000,0.000000}%
\pgfsetstrokecolor{currentstroke}%
\pgfsetdash{}{0pt}%
\pgfpathmoveto{\pgfqpoint{4.162917in}{1.265415in}}%
\pgfpathlineto{\pgfqpoint{4.177307in}{1.264996in}}%
\pgfpathlineto{\pgfqpoint{4.191706in}{1.264647in}}%
\pgfpathlineto{\pgfqpoint{4.206115in}{1.264366in}}%
\pgfpathlineto{\pgfqpoint{4.220532in}{1.264156in}}%
\pgfpathlineto{\pgfqpoint{4.228786in}{1.276864in}}%
\pgfpathlineto{\pgfqpoint{4.237034in}{1.289633in}}%
\pgfpathlineto{\pgfqpoint{4.245277in}{1.302458in}}%
\pgfpathlineto{\pgfqpoint{4.253515in}{1.315332in}}%
\pgfpathlineto{\pgfqpoint{4.239104in}{1.315186in}}%
\pgfpathlineto{\pgfqpoint{4.224702in}{1.315109in}}%
\pgfpathlineto{\pgfqpoint{4.210310in}{1.315102in}}%
\pgfpathlineto{\pgfqpoint{4.195927in}{1.315165in}}%
\pgfpathlineto{\pgfqpoint{4.187682in}{1.302639in}}%
\pgfpathlineto{\pgfqpoint{4.179432in}{1.290169in}}%
\pgfpathlineto{\pgfqpoint{4.171177in}{1.277759in}}%
\pgfpathlineto{\pgfqpoint{4.162917in}{1.265415in}}%
\pgfpathclose%
\pgfusepath{fill}%
\end{pgfscope}%
\begin{pgfscope}%
\pgfpathrectangle{\pgfqpoint{1.150000in}{0.150000in}}{\pgfqpoint{5.700000in}{5.700000in}}%
\pgfusepath{clip}%
\pgfsetbuttcap%
\pgfsetroundjoin%
\definecolor{currentfill}{rgb}{0.280894,0.078907,0.402329}%
\pgfsetfillcolor{currentfill}%
\pgfsetfillopacity{0.700000}%
\pgfsetlinewidth{0.000000pt}%
\definecolor{currentstroke}{rgb}{0.000000,0.000000,0.000000}%
\pgfsetstrokecolor{currentstroke}%
\pgfsetdash{}{0pt}%
\pgfpathmoveto{\pgfqpoint{4.072324in}{1.220687in}}%
\pgfpathlineto{\pgfqpoint{4.086686in}{1.219613in}}%
\pgfpathlineto{\pgfqpoint{4.101057in}{1.218608in}}%
\pgfpathlineto{\pgfqpoint{4.115437in}{1.217672in}}%
\pgfpathlineto{\pgfqpoint{4.129826in}{1.216807in}}%
\pgfpathlineto{\pgfqpoint{4.138106in}{1.228833in}}%
\pgfpathlineto{\pgfqpoint{4.146382in}{1.240947in}}%
\pgfpathlineto{\pgfqpoint{4.154652in}{1.253143in}}%
\pgfpathlineto{\pgfqpoint{4.162917in}{1.265415in}}%
\pgfpathlineto{\pgfqpoint{4.148537in}{1.265904in}}%
\pgfpathlineto{\pgfqpoint{4.134165in}{1.266463in}}%
\pgfpathlineto{\pgfqpoint{4.119802in}{1.267091in}}%
\pgfpathlineto{\pgfqpoint{4.105448in}{1.267789in}}%
\pgfpathlineto{\pgfqpoint{4.097175in}{1.255885in}}%
\pgfpathlineto{\pgfqpoint{4.088897in}{1.244063in}}%
\pgfpathlineto{\pgfqpoint{4.080613in}{1.232329in}}%
\pgfpathlineto{\pgfqpoint{4.072324in}{1.220687in}}%
\pgfpathclose%
\pgfusepath{fill}%
\end{pgfscope}%
\begin{pgfscope}%
\pgfpathrectangle{\pgfqpoint{1.150000in}{0.150000in}}{\pgfqpoint{5.700000in}{5.700000in}}%
\pgfusepath{clip}%
\pgfsetbuttcap%
\pgfsetroundjoin%
\definecolor{currentfill}{rgb}{0.283187,0.125848,0.444960}%
\pgfsetfillcolor{currentfill}%
\pgfsetfillopacity{0.700000}%
\pgfsetlinewidth{0.000000pt}%
\definecolor{currentstroke}{rgb}{0.000000,0.000000,0.000000}%
\pgfsetstrokecolor{currentstroke}%
\pgfsetdash{}{0pt}%
\pgfpathmoveto{\pgfqpoint{4.253515in}{1.315332in}}%
\pgfpathlineto{\pgfqpoint{4.267936in}{1.315549in}}%
\pgfpathlineto{\pgfqpoint{4.282367in}{1.315834in}}%
\pgfpathlineto{\pgfqpoint{4.296808in}{1.316190in}}%
\pgfpathlineto{\pgfqpoint{4.311258in}{1.316615in}}%
\pgfpathlineto{\pgfqpoint{4.319486in}{1.329879in}}%
\pgfpathlineto{\pgfqpoint{4.327709in}{1.343178in}}%
\pgfpathlineto{\pgfqpoint{4.335928in}{1.356507in}}%
\pgfpathlineto{\pgfqpoint{4.344142in}{1.369862in}}%
\pgfpathlineto{\pgfqpoint{4.329697in}{1.369100in}}%
\pgfpathlineto{\pgfqpoint{4.315262in}{1.368407in}}%
\pgfpathlineto{\pgfqpoint{4.300836in}{1.367785in}}%
\pgfpathlineto{\pgfqpoint{4.286421in}{1.367232in}}%
\pgfpathlineto{\pgfqpoint{4.278202in}{1.354207in}}%
\pgfpathlineto{\pgfqpoint{4.269978in}{1.341212in}}%
\pgfpathlineto{\pgfqpoint{4.261749in}{1.328252in}}%
\pgfpathlineto{\pgfqpoint{4.253515in}{1.315332in}}%
\pgfpathclose%
\pgfusepath{fill}%
\end{pgfscope}%
\begin{pgfscope}%
\pgfpathrectangle{\pgfqpoint{1.150000in}{0.150000in}}{\pgfqpoint{5.700000in}{5.700000in}}%
\pgfusepath{clip}%
\pgfsetbuttcap%
\pgfsetroundjoin%
\definecolor{currentfill}{rgb}{0.277941,0.056324,0.381191}%
\pgfsetfillcolor{currentfill}%
\pgfsetfillopacity{0.700000}%
\pgfsetlinewidth{0.000000pt}%
\definecolor{currentstroke}{rgb}{0.000000,0.000000,0.000000}%
\pgfsetstrokecolor{currentstroke}%
\pgfsetdash{}{0pt}%
\pgfpathmoveto{\pgfqpoint{3.981706in}{1.181747in}}%
\pgfpathlineto{\pgfqpoint{3.996044in}{1.179997in}}%
\pgfpathlineto{\pgfqpoint{4.010392in}{1.178317in}}%
\pgfpathlineto{\pgfqpoint{4.024747in}{1.176706in}}%
\pgfpathlineto{\pgfqpoint{4.039111in}{1.175165in}}%
\pgfpathlineto{\pgfqpoint{4.047422in}{1.186377in}}%
\pgfpathlineto{\pgfqpoint{4.055728in}{1.197705in}}%
\pgfpathlineto{\pgfqpoint{4.064029in}{1.209144in}}%
\pgfpathlineto{\pgfqpoint{4.072324in}{1.220687in}}%
\pgfpathlineto{\pgfqpoint{4.057970in}{1.221831in}}%
\pgfpathlineto{\pgfqpoint{4.043624in}{1.223045in}}%
\pgfpathlineto{\pgfqpoint{4.029288in}{1.224329in}}%
\pgfpathlineto{\pgfqpoint{4.014959in}{1.225683in}}%
\pgfpathlineto{\pgfqpoint{4.006655in}{1.214529in}}%
\pgfpathlineto{\pgfqpoint{3.998344in}{1.203484in}}%
\pgfpathlineto{\pgfqpoint{3.990028in}{1.192555in}}%
\pgfpathlineto{\pgfqpoint{3.981706in}{1.181747in}}%
\pgfpathclose%
\pgfusepath{fill}%
\end{pgfscope}%
\begin{pgfscope}%
\pgfpathrectangle{\pgfqpoint{1.150000in}{0.150000in}}{\pgfqpoint{5.700000in}{5.700000in}}%
\pgfusepath{clip}%
\pgfsetbuttcap%
\pgfsetroundjoin%
\definecolor{currentfill}{rgb}{0.281412,0.155834,0.469201}%
\pgfsetfillcolor{currentfill}%
\pgfsetfillopacity{0.700000}%
\pgfsetlinewidth{0.000000pt}%
\definecolor{currentstroke}{rgb}{0.000000,0.000000,0.000000}%
\pgfsetstrokecolor{currentstroke}%
\pgfsetdash{}{0pt}%
\pgfpathmoveto{\pgfqpoint{4.344142in}{1.369862in}}%
\pgfpathlineto{\pgfqpoint{4.358598in}{1.370693in}}%
\pgfpathlineto{\pgfqpoint{4.373063in}{1.371595in}}%
\pgfpathlineto{\pgfqpoint{4.387539in}{1.372566in}}%
\pgfpathlineto{\pgfqpoint{4.402025in}{1.373607in}}%
\pgfpathlineto{\pgfqpoint{4.410230in}{1.387306in}}%
\pgfpathlineto{\pgfqpoint{4.418431in}{1.401016in}}%
\pgfpathlineto{\pgfqpoint{4.426627in}{1.414731in}}%
\pgfpathlineto{\pgfqpoint{4.434818in}{1.428448in}}%
\pgfpathlineto{\pgfqpoint{4.420336in}{1.427090in}}%
\pgfpathlineto{\pgfqpoint{4.405864in}{1.425802in}}%
\pgfpathlineto{\pgfqpoint{4.391402in}{1.424584in}}%
\pgfpathlineto{\pgfqpoint{4.376951in}{1.423435in}}%
\pgfpathlineto{\pgfqpoint{4.368756in}{1.410028in}}%
\pgfpathlineto{\pgfqpoint{4.360556in}{1.396626in}}%
\pgfpathlineto{\pgfqpoint{4.352352in}{1.383236in}}%
\pgfpathlineto{\pgfqpoint{4.344142in}{1.369862in}}%
\pgfpathclose%
\pgfusepath{fill}%
\end{pgfscope}%
\begin{pgfscope}%
\pgfpathrectangle{\pgfqpoint{1.150000in}{0.150000in}}{\pgfqpoint{5.700000in}{5.700000in}}%
\pgfusepath{clip}%
\pgfsetbuttcap%
\pgfsetroundjoin%
\definecolor{currentfill}{rgb}{0.243113,0.292092,0.538516}%
\pgfsetfillcolor{currentfill}%
\pgfsetfillopacity{0.700000}%
\pgfsetlinewidth{0.000000pt}%
\definecolor{currentstroke}{rgb}{0.000000,0.000000,0.000000}%
\pgfsetstrokecolor{currentstroke}%
\pgfsetdash{}{0pt}%
\pgfpathmoveto{\pgfqpoint{4.739735in}{1.679222in}}%
\pgfpathlineto{\pgfqpoint{4.754355in}{1.682556in}}%
\pgfpathlineto{\pgfqpoint{4.768987in}{1.685961in}}%
\pgfpathlineto{\pgfqpoint{4.783632in}{1.689436in}}%
\pgfpathlineto{\pgfqpoint{4.798289in}{1.692982in}}%
\pgfpathlineto{\pgfqpoint{4.806386in}{1.706993in}}%
\pgfpathlineto{\pgfqpoint{4.814477in}{1.720917in}}%
\pgfpathlineto{\pgfqpoint{4.822562in}{1.734750in}}%
\pgfpathlineto{\pgfqpoint{4.830642in}{1.748490in}}%
\pgfpathlineto{\pgfqpoint{4.815986in}{1.744730in}}%
\pgfpathlineto{\pgfqpoint{4.801344in}{1.741040in}}%
\pgfpathlineto{\pgfqpoint{4.786713in}{1.737421in}}%
\pgfpathlineto{\pgfqpoint{4.772095in}{1.733874in}}%
\pgfpathlineto{\pgfqpoint{4.764013in}{1.720339in}}%
\pgfpathlineto{\pgfqpoint{4.755926in}{1.706718in}}%
\pgfpathlineto{\pgfqpoint{4.747833in}{1.693011in}}%
\pgfpathlineto{\pgfqpoint{4.739735in}{1.679222in}}%
\pgfpathclose%
\pgfusepath{fill}%
\end{pgfscope}%
\begin{pgfscope}%
\pgfpathrectangle{\pgfqpoint{1.150000in}{0.150000in}}{\pgfqpoint{5.700000in}{5.700000in}}%
\pgfusepath{clip}%
\pgfsetbuttcap%
\pgfsetroundjoin%
\definecolor{currentfill}{rgb}{0.195860,0.395433,0.555276}%
\pgfsetfillcolor{currentfill}%
\pgfsetfillopacity{0.700000}%
\pgfsetlinewidth{0.000000pt}%
\definecolor{currentstroke}{rgb}{0.000000,0.000000,0.000000}%
\pgfsetstrokecolor{currentstroke}%
\pgfsetdash{}{0pt}%
\pgfpathmoveto{\pgfqpoint{5.044759in}{1.941920in}}%
\pgfpathlineto{\pgfqpoint{5.059525in}{1.946834in}}%
\pgfpathlineto{\pgfqpoint{5.074304in}{1.951821in}}%
\pgfpathlineto{\pgfqpoint{5.089097in}{1.956878in}}%
\pgfpathlineto{\pgfqpoint{5.103903in}{1.962008in}}%
\pgfpathlineto{\pgfqpoint{5.111890in}{1.974722in}}%
\pgfpathlineto{\pgfqpoint{5.119870in}{1.987295in}}%
\pgfpathlineto{\pgfqpoint{5.127842in}{1.999725in}}%
\pgfpathlineto{\pgfqpoint{5.135806in}{2.012011in}}%
\pgfpathlineto{\pgfqpoint{5.121003in}{2.006754in}}%
\pgfpathlineto{\pgfqpoint{5.106213in}{2.001568in}}%
\pgfpathlineto{\pgfqpoint{5.091437in}{1.996454in}}%
\pgfpathlineto{\pgfqpoint{5.076675in}{1.991411in}}%
\pgfpathlineto{\pgfqpoint{5.068707in}{1.979244in}}%
\pgfpathlineto{\pgfqpoint{5.060732in}{1.966939in}}%
\pgfpathlineto{\pgfqpoint{5.052749in}{1.954497in}}%
\pgfpathlineto{\pgfqpoint{5.044759in}{1.941920in}}%
\pgfpathclose%
\pgfusepath{fill}%
\end{pgfscope}%
\begin{pgfscope}%
\pgfpathrectangle{\pgfqpoint{1.150000in}{0.150000in}}{\pgfqpoint{5.700000in}{5.700000in}}%
\pgfusepath{clip}%
\pgfsetbuttcap%
\pgfsetroundjoin%
\definecolor{currentfill}{rgb}{0.274952,0.037752,0.364543}%
\pgfsetfillcolor{currentfill}%
\pgfsetfillopacity{0.700000}%
\pgfsetlinewidth{0.000000pt}%
\definecolor{currentstroke}{rgb}{0.000000,0.000000,0.000000}%
\pgfsetstrokecolor{currentstroke}%
\pgfsetdash{}{0pt}%
\pgfpathmoveto{\pgfqpoint{3.891031in}{1.149218in}}%
\pgfpathlineto{\pgfqpoint{3.905350in}{1.146772in}}%
\pgfpathlineto{\pgfqpoint{3.919677in}{1.144396in}}%
\pgfpathlineto{\pgfqpoint{3.934012in}{1.142090in}}%
\pgfpathlineto{\pgfqpoint{3.948355in}{1.139853in}}%
\pgfpathlineto{\pgfqpoint{3.956702in}{1.150113in}}%
\pgfpathlineto{\pgfqpoint{3.965043in}{1.160520in}}%
\pgfpathlineto{\pgfqpoint{3.973377in}{1.171067in}}%
\pgfpathlineto{\pgfqpoint{3.981706in}{1.181747in}}%
\pgfpathlineto{\pgfqpoint{3.967375in}{1.183567in}}%
\pgfpathlineto{\pgfqpoint{3.953052in}{1.185457in}}%
\pgfpathlineto{\pgfqpoint{3.938737in}{1.187417in}}%
\pgfpathlineto{\pgfqpoint{3.924430in}{1.189446in}}%
\pgfpathlineto{\pgfqpoint{3.916090in}{1.179175in}}%
\pgfpathlineto{\pgfqpoint{3.907744in}{1.169042in}}%
\pgfpathlineto{\pgfqpoint{3.899390in}{1.159055in}}%
\pgfpathlineto{\pgfqpoint{3.891031in}{1.149218in}}%
\pgfpathclose%
\pgfusepath{fill}%
\end{pgfscope}%
\begin{pgfscope}%
\pgfpathrectangle{\pgfqpoint{1.150000in}{0.150000in}}{\pgfqpoint{5.700000in}{5.700000in}}%
\pgfusepath{clip}%
\pgfsetbuttcap%
\pgfsetroundjoin%
\definecolor{currentfill}{rgb}{0.277134,0.185228,0.489898}%
\pgfsetfillcolor{currentfill}%
\pgfsetfillopacity{0.700000}%
\pgfsetlinewidth{0.000000pt}%
\definecolor{currentstroke}{rgb}{0.000000,0.000000,0.000000}%
\pgfsetstrokecolor{currentstroke}%
\pgfsetdash{}{0pt}%
\pgfpathmoveto{\pgfqpoint{4.434818in}{1.428448in}}%
\pgfpathlineto{\pgfqpoint{4.449311in}{1.429876in}}%
\pgfpathlineto{\pgfqpoint{4.463815in}{1.431373in}}%
\pgfpathlineto{\pgfqpoint{4.478329in}{1.432941in}}%
\pgfpathlineto{\pgfqpoint{4.492854in}{1.434578in}}%
\pgfpathlineto{\pgfqpoint{4.501037in}{1.448596in}}%
\pgfpathlineto{\pgfqpoint{4.509216in}{1.462602in}}%
\pgfpathlineto{\pgfqpoint{4.517390in}{1.476590in}}%
\pgfpathlineto{\pgfqpoint{4.525559in}{1.490557in}}%
\pgfpathlineto{\pgfqpoint{4.511037in}{1.488623in}}%
\pgfpathlineto{\pgfqpoint{4.496525in}{1.486759in}}%
\pgfpathlineto{\pgfqpoint{4.482025in}{1.484964in}}%
\pgfpathlineto{\pgfqpoint{4.467535in}{1.483240in}}%
\pgfpathlineto{\pgfqpoint{4.459363in}{1.469562in}}%
\pgfpathlineto{\pgfqpoint{4.451186in}{1.455868in}}%
\pgfpathlineto{\pgfqpoint{4.443004in}{1.442162in}}%
\pgfpathlineto{\pgfqpoint{4.434818in}{1.428448in}}%
\pgfpathclose%
\pgfusepath{fill}%
\end{pgfscope}%
\begin{pgfscope}%
\pgfpathrectangle{\pgfqpoint{1.150000in}{0.150000in}}{\pgfqpoint{5.700000in}{5.700000in}}%
\pgfusepath{clip}%
\pgfsetbuttcap%
\pgfsetroundjoin%
\definecolor{currentfill}{rgb}{0.183898,0.422383,0.556944}%
\pgfsetfillcolor{currentfill}%
\pgfsetfillopacity{0.700000}%
\pgfsetlinewidth{0.000000pt}%
\definecolor{currentstroke}{rgb}{0.000000,0.000000,0.000000}%
\pgfsetstrokecolor{currentstroke}%
\pgfsetdash{}{0pt}%
\pgfpathmoveto{\pgfqpoint{5.135806in}{2.012011in}}%
\pgfpathlineto{\pgfqpoint{5.150623in}{2.017340in}}%
\pgfpathlineto{\pgfqpoint{5.165454in}{2.022741in}}%
\pgfpathlineto{\pgfqpoint{5.180298in}{2.028214in}}%
\pgfpathlineto{\pgfqpoint{5.188252in}{2.040441in}}%
\pgfpathlineto{\pgfqpoint{5.196198in}{2.052516in}}%
\pgfpathlineto{\pgfqpoint{5.204136in}{2.064440in}}%
\pgfpathlineto{\pgfqpoint{5.212066in}{2.076210in}}%
\pgfpathlineto{\pgfqpoint{5.197224in}{2.070631in}}%
\pgfpathlineto{\pgfqpoint{5.182397in}{2.065124in}}%
\pgfpathlineto{\pgfqpoint{5.167584in}{2.059689in}}%
\pgfpathlineto{\pgfqpoint{5.159652in}{2.047992in}}%
\pgfpathlineto{\pgfqpoint{5.151711in}{2.036146in}}%
\pgfpathlineto{\pgfqpoint{5.143762in}{2.024152in}}%
\pgfpathlineto{\pgfqpoint{5.135806in}{2.012011in}}%
\pgfpathclose%
\pgfusepath{fill}%
\end{pgfscope}%
\begin{pgfscope}%
\pgfpathrectangle{\pgfqpoint{1.150000in}{0.150000in}}{\pgfqpoint{5.700000in}{5.700000in}}%
\pgfusepath{clip}%
\pgfsetbuttcap%
\pgfsetroundjoin%
\definecolor{currentfill}{rgb}{0.270595,0.214069,0.507052}%
\pgfsetfillcolor{currentfill}%
\pgfsetfillopacity{0.700000}%
\pgfsetlinewidth{0.000000pt}%
\definecolor{currentstroke}{rgb}{0.000000,0.000000,0.000000}%
\pgfsetstrokecolor{currentstroke}%
\pgfsetdash{}{0pt}%
\pgfpathmoveto{\pgfqpoint{4.525559in}{1.490557in}}%
\pgfpathlineto{\pgfqpoint{4.540093in}{1.492562in}}%
\pgfpathlineto{\pgfqpoint{4.554637in}{1.494636in}}%
\pgfpathlineto{\pgfqpoint{4.569193in}{1.496780in}}%
\pgfpathlineto{\pgfqpoint{4.583759in}{1.498994in}}%
\pgfpathlineto{\pgfqpoint{4.591922in}{1.513221in}}%
\pgfpathlineto{\pgfqpoint{4.600079in}{1.527412in}}%
\pgfpathlineto{\pgfqpoint{4.608231in}{1.541565in}}%
\pgfpathlineto{\pgfqpoint{4.616379in}{1.555676in}}%
\pgfpathlineto{\pgfqpoint{4.601814in}{1.553185in}}%
\pgfpathlineto{\pgfqpoint{4.587261in}{1.550764in}}%
\pgfpathlineto{\pgfqpoint{4.572718in}{1.548413in}}%
\pgfpathlineto{\pgfqpoint{4.558188in}{1.546132in}}%
\pgfpathlineto{\pgfqpoint{4.550038in}{1.532290in}}%
\pgfpathlineto{\pgfqpoint{4.541883in}{1.518411in}}%
\pgfpathlineto{\pgfqpoint{4.533724in}{1.504499in}}%
\pgfpathlineto{\pgfqpoint{4.525559in}{1.490557in}}%
\pgfpathclose%
\pgfusepath{fill}%
\end{pgfscope}%
\begin{pgfscope}%
\pgfpathrectangle{\pgfqpoint{1.150000in}{0.150000in}}{\pgfqpoint{5.700000in}{5.700000in}}%
\pgfusepath{clip}%
\pgfsetbuttcap%
\pgfsetroundjoin%
\definecolor{currentfill}{rgb}{0.227802,0.326594,0.546532}%
\pgfsetfillcolor{currentfill}%
\pgfsetfillopacity{0.700000}%
\pgfsetlinewidth{0.000000pt}%
\definecolor{currentstroke}{rgb}{0.000000,0.000000,0.000000}%
\pgfsetstrokecolor{currentstroke}%
\pgfsetdash{}{0pt}%
\pgfpathmoveto{\pgfqpoint{4.830642in}{1.748490in}}%
\pgfpathlineto{\pgfqpoint{4.845310in}{1.752321in}}%
\pgfpathlineto{\pgfqpoint{4.859990in}{1.756223in}}%
\pgfpathlineto{\pgfqpoint{4.874683in}{1.760196in}}%
\pgfpathlineto{\pgfqpoint{4.889388in}{1.764239in}}%
\pgfpathlineto{\pgfqpoint{4.897461in}{1.778085in}}%
\pgfpathlineto{\pgfqpoint{4.905527in}{1.791826in}}%
\pgfpathlineto{\pgfqpoint{4.913586in}{1.805460in}}%
\pgfpathlineto{\pgfqpoint{4.921640in}{1.818985in}}%
\pgfpathlineto{\pgfqpoint{4.906936in}{1.814748in}}%
\pgfpathlineto{\pgfqpoint{4.892244in}{1.810582in}}%
\pgfpathlineto{\pgfqpoint{4.877566in}{1.806487in}}%
\pgfpathlineto{\pgfqpoint{4.862900in}{1.802463in}}%
\pgfpathlineto{\pgfqpoint{4.854844in}{1.789123in}}%
\pgfpathlineto{\pgfqpoint{4.846783in}{1.775679in}}%
\pgfpathlineto{\pgfqpoint{4.838715in}{1.762134in}}%
\pgfpathlineto{\pgfqpoint{4.830642in}{1.748490in}}%
\pgfpathclose%
\pgfusepath{fill}%
\end{pgfscope}%
\begin{pgfscope}%
\pgfpathrectangle{\pgfqpoint{1.150000in}{0.150000in}}{\pgfqpoint{5.700000in}{5.700000in}}%
\pgfusepath{clip}%
\pgfsetbuttcap%
\pgfsetroundjoin%
\definecolor{currentfill}{rgb}{0.260571,0.246922,0.522828}%
\pgfsetfillcolor{currentfill}%
\pgfsetfillopacity{0.700000}%
\pgfsetlinewidth{0.000000pt}%
\definecolor{currentstroke}{rgb}{0.000000,0.000000,0.000000}%
\pgfsetstrokecolor{currentstroke}%
\pgfsetdash{}{0pt}%
\pgfpathmoveto{\pgfqpoint{4.616379in}{1.555676in}}%
\pgfpathlineto{\pgfqpoint{4.630955in}{1.558237in}}%
\pgfpathlineto{\pgfqpoint{4.645543in}{1.560868in}}%
\pgfpathlineto{\pgfqpoint{4.660143in}{1.563570in}}%
\pgfpathlineto{\pgfqpoint{4.674754in}{1.566342in}}%
\pgfpathlineto{\pgfqpoint{4.682895in}{1.580670in}}%
\pgfpathlineto{\pgfqpoint{4.691030in}{1.594943in}}%
\pgfpathlineto{\pgfqpoint{4.699161in}{1.609157in}}%
\pgfpathlineto{\pgfqpoint{4.707287in}{1.623309in}}%
\pgfpathlineto{\pgfqpoint{4.692677in}{1.620281in}}%
\pgfpathlineto{\pgfqpoint{4.678079in}{1.617323in}}%
\pgfpathlineto{\pgfqpoint{4.663493in}{1.614435in}}%
\pgfpathlineto{\pgfqpoint{4.648918in}{1.611618in}}%
\pgfpathlineto{\pgfqpoint{4.640791in}{1.597714in}}%
\pgfpathlineto{\pgfqpoint{4.632659in}{1.583754in}}%
\pgfpathlineto{\pgfqpoint{4.624521in}{1.569740in}}%
\pgfpathlineto{\pgfqpoint{4.616379in}{1.555676in}}%
\pgfpathclose%
\pgfusepath{fill}%
\end{pgfscope}%
\begin{pgfscope}%
\pgfpathrectangle{\pgfqpoint{1.150000in}{0.150000in}}{\pgfqpoint{5.700000in}{5.700000in}}%
\pgfusepath{clip}%
\pgfsetbuttcap%
\pgfsetroundjoin%
\definecolor{currentfill}{rgb}{0.214298,0.355619,0.551184}%
\pgfsetfillcolor{currentfill}%
\pgfsetfillopacity{0.700000}%
\pgfsetlinewidth{0.000000pt}%
\definecolor{currentstroke}{rgb}{0.000000,0.000000,0.000000}%
\pgfsetstrokecolor{currentstroke}%
\pgfsetdash{}{0pt}%
\pgfpathmoveto{\pgfqpoint{4.921640in}{1.818985in}}%
\pgfpathlineto{\pgfqpoint{4.936357in}{1.823293in}}%
\pgfpathlineto{\pgfqpoint{4.951087in}{1.827672in}}%
\pgfpathlineto{\pgfqpoint{4.965830in}{1.832122in}}%
\pgfpathlineto{\pgfqpoint{4.980586in}{1.836644in}}%
\pgfpathlineto{\pgfqpoint{4.988632in}{1.850237in}}%
\pgfpathlineto{\pgfqpoint{4.996671in}{1.863711in}}%
\pgfpathlineto{\pgfqpoint{5.004703in}{1.877063in}}%
\pgfpathlineto{\pgfqpoint{5.012728in}{1.890290in}}%
\pgfpathlineto{\pgfqpoint{4.997974in}{1.885597in}}%
\pgfpathlineto{\pgfqpoint{4.983232in}{1.880975in}}%
\pgfpathlineto{\pgfqpoint{4.968504in}{1.876424in}}%
\pgfpathlineto{\pgfqpoint{4.953789in}{1.871944in}}%
\pgfpathlineto{\pgfqpoint{4.945762in}{1.858880in}}%
\pgfpathlineto{\pgfqpoint{4.937728in}{1.845697in}}%
\pgfpathlineto{\pgfqpoint{4.929687in}{1.832398in}}%
\pgfpathlineto{\pgfqpoint{4.921640in}{1.818985in}}%
\pgfpathclose%
\pgfusepath{fill}%
\end{pgfscope}%
\begin{pgfscope}%
\pgfpathrectangle{\pgfqpoint{1.150000in}{0.150000in}}{\pgfqpoint{5.700000in}{5.700000in}}%
\pgfusepath{clip}%
\pgfsetbuttcap%
\pgfsetroundjoin%
\definecolor{currentfill}{rgb}{0.281924,0.089666,0.412415}%
\pgfsetfillcolor{currentfill}%
\pgfsetfillopacity{0.700000}%
\pgfsetlinewidth{0.000000pt}%
\definecolor{currentstroke}{rgb}{0.000000,0.000000,0.000000}%
\pgfsetstrokecolor{currentstroke}%
\pgfsetdash{}{0pt}%
\pgfpathmoveto{\pgfqpoint{4.129826in}{1.216807in}}%
\pgfpathlineto{\pgfqpoint{4.144223in}{1.216011in}}%
\pgfpathlineto{\pgfqpoint{4.158630in}{1.215284in}}%
\pgfpathlineto{\pgfqpoint{4.173046in}{1.214626in}}%
\pgfpathlineto{\pgfqpoint{4.187470in}{1.214038in}}%
\pgfpathlineto{\pgfqpoint{4.195743in}{1.226449in}}%
\pgfpathlineto{\pgfqpoint{4.204011in}{1.238943in}}%
\pgfpathlineto{\pgfqpoint{4.212274in}{1.251514in}}%
\pgfpathlineto{\pgfqpoint{4.220532in}{1.264156in}}%
\pgfpathlineto{\pgfqpoint{4.206115in}{1.264366in}}%
\pgfpathlineto{\pgfqpoint{4.191706in}{1.264647in}}%
\pgfpathlineto{\pgfqpoint{4.177307in}{1.264996in}}%
\pgfpathlineto{\pgfqpoint{4.162917in}{1.265415in}}%
\pgfpathlineto{\pgfqpoint{4.154652in}{1.253143in}}%
\pgfpathlineto{\pgfqpoint{4.146382in}{1.240947in}}%
\pgfpathlineto{\pgfqpoint{4.138106in}{1.228833in}}%
\pgfpathlineto{\pgfqpoint{4.129826in}{1.216807in}}%
\pgfpathclose%
\pgfusepath{fill}%
\end{pgfscope}%
\begin{pgfscope}%
\pgfpathrectangle{\pgfqpoint{1.150000in}{0.150000in}}{\pgfqpoint{5.700000in}{5.700000in}}%
\pgfusepath{clip}%
\pgfsetbuttcap%
\pgfsetroundjoin%
\definecolor{currentfill}{rgb}{0.283091,0.110553,0.431554}%
\pgfsetfillcolor{currentfill}%
\pgfsetfillopacity{0.700000}%
\pgfsetlinewidth{0.000000pt}%
\definecolor{currentstroke}{rgb}{0.000000,0.000000,0.000000}%
\pgfsetstrokecolor{currentstroke}%
\pgfsetdash{}{0pt}%
\pgfpathmoveto{\pgfqpoint{4.220532in}{1.264156in}}%
\pgfpathlineto{\pgfqpoint{4.234959in}{1.264015in}}%
\pgfpathlineto{\pgfqpoint{4.249396in}{1.263943in}}%
\pgfpathlineto{\pgfqpoint{4.263842in}{1.263941in}}%
\pgfpathlineto{\pgfqpoint{4.278298in}{1.264008in}}%
\pgfpathlineto{\pgfqpoint{4.286545in}{1.277082in}}%
\pgfpathlineto{\pgfqpoint{4.294787in}{1.290211in}}%
\pgfpathlineto{\pgfqpoint{4.303025in}{1.303390in}}%
\pgfpathlineto{\pgfqpoint{4.311258in}{1.316615in}}%
\pgfpathlineto{\pgfqpoint{4.296808in}{1.316190in}}%
\pgfpathlineto{\pgfqpoint{4.282367in}{1.315834in}}%
\pgfpathlineto{\pgfqpoint{4.267936in}{1.315549in}}%
\pgfpathlineto{\pgfqpoint{4.253515in}{1.315332in}}%
\pgfpathlineto{\pgfqpoint{4.245277in}{1.302458in}}%
\pgfpathlineto{\pgfqpoint{4.237034in}{1.289633in}}%
\pgfpathlineto{\pgfqpoint{4.228786in}{1.276864in}}%
\pgfpathlineto{\pgfqpoint{4.220532in}{1.264156in}}%
\pgfpathclose%
\pgfusepath{fill}%
\end{pgfscope}%
\begin{pgfscope}%
\pgfpathrectangle{\pgfqpoint{1.150000in}{0.150000in}}{\pgfqpoint{5.700000in}{5.700000in}}%
\pgfusepath{clip}%
\pgfsetbuttcap%
\pgfsetroundjoin%
\definecolor{currentfill}{rgb}{0.279566,0.067836,0.391917}%
\pgfsetfillcolor{currentfill}%
\pgfsetfillopacity{0.700000}%
\pgfsetlinewidth{0.000000pt}%
\definecolor{currentstroke}{rgb}{0.000000,0.000000,0.000000}%
\pgfsetstrokecolor{currentstroke}%
\pgfsetdash{}{0pt}%
\pgfpathmoveto{\pgfqpoint{4.039111in}{1.175165in}}%
\pgfpathlineto{\pgfqpoint{4.053483in}{1.173694in}}%
\pgfpathlineto{\pgfqpoint{4.067863in}{1.172292in}}%
\pgfpathlineto{\pgfqpoint{4.082252in}{1.170959in}}%
\pgfpathlineto{\pgfqpoint{4.096649in}{1.169696in}}%
\pgfpathlineto{\pgfqpoint{4.104952in}{1.181313in}}%
\pgfpathlineto{\pgfqpoint{4.113248in}{1.193041in}}%
\pgfpathlineto{\pgfqpoint{4.121540in}{1.204874in}}%
\pgfpathlineto{\pgfqpoint{4.129826in}{1.216807in}}%
\pgfpathlineto{\pgfqpoint{4.115437in}{1.217672in}}%
\pgfpathlineto{\pgfqpoint{4.101057in}{1.218608in}}%
\pgfpathlineto{\pgfqpoint{4.086686in}{1.219613in}}%
\pgfpathlineto{\pgfqpoint{4.072324in}{1.220687in}}%
\pgfpathlineto{\pgfqpoint{4.064029in}{1.209144in}}%
\pgfpathlineto{\pgfqpoint{4.055728in}{1.197705in}}%
\pgfpathlineto{\pgfqpoint{4.047422in}{1.186377in}}%
\pgfpathlineto{\pgfqpoint{4.039111in}{1.175165in}}%
\pgfpathclose%
\pgfusepath{fill}%
\end{pgfscope}%
\begin{pgfscope}%
\pgfpathrectangle{\pgfqpoint{1.150000in}{0.150000in}}{\pgfqpoint{5.700000in}{5.700000in}}%
\pgfusepath{clip}%
\pgfsetbuttcap%
\pgfsetroundjoin%
\definecolor{currentfill}{rgb}{0.282623,0.140926,0.457517}%
\pgfsetfillcolor{currentfill}%
\pgfsetfillopacity{0.700000}%
\pgfsetlinewidth{0.000000pt}%
\definecolor{currentstroke}{rgb}{0.000000,0.000000,0.000000}%
\pgfsetstrokecolor{currentstroke}%
\pgfsetdash{}{0pt}%
\pgfpathmoveto{\pgfqpoint{4.311258in}{1.316615in}}%
\pgfpathlineto{\pgfqpoint{4.325718in}{1.317109in}}%
\pgfpathlineto{\pgfqpoint{4.340188in}{1.317673in}}%
\pgfpathlineto{\pgfqpoint{4.354668in}{1.318306in}}%
\pgfpathlineto{\pgfqpoint{4.369158in}{1.319009in}}%
\pgfpathlineto{\pgfqpoint{4.377382in}{1.332619in}}%
\pgfpathlineto{\pgfqpoint{4.385601in}{1.346258in}}%
\pgfpathlineto{\pgfqpoint{4.393815in}{1.359923in}}%
\pgfpathlineto{\pgfqpoint{4.402025in}{1.373607in}}%
\pgfpathlineto{\pgfqpoint{4.387539in}{1.372566in}}%
\pgfpathlineto{\pgfqpoint{4.373063in}{1.371595in}}%
\pgfpathlineto{\pgfqpoint{4.358598in}{1.370693in}}%
\pgfpathlineto{\pgfqpoint{4.344142in}{1.369862in}}%
\pgfpathlineto{\pgfqpoint{4.335928in}{1.356507in}}%
\pgfpathlineto{\pgfqpoint{4.327709in}{1.343178in}}%
\pgfpathlineto{\pgfqpoint{4.319486in}{1.329879in}}%
\pgfpathlineto{\pgfqpoint{4.311258in}{1.316615in}}%
\pgfpathclose%
\pgfusepath{fill}%
\end{pgfscope}%
\begin{pgfscope}%
\pgfpathrectangle{\pgfqpoint{1.150000in}{0.150000in}}{\pgfqpoint{5.700000in}{5.700000in}}%
\pgfusepath{clip}%
\pgfsetbuttcap%
\pgfsetroundjoin%
\definecolor{currentfill}{rgb}{0.248629,0.278775,0.534556}%
\pgfsetfillcolor{currentfill}%
\pgfsetfillopacity{0.700000}%
\pgfsetlinewidth{0.000000pt}%
\definecolor{currentstroke}{rgb}{0.000000,0.000000,0.000000}%
\pgfsetstrokecolor{currentstroke}%
\pgfsetdash{}{0pt}%
\pgfpathmoveto{\pgfqpoint{4.707287in}{1.623309in}}%
\pgfpathlineto{\pgfqpoint{4.721908in}{1.626408in}}%
\pgfpathlineto{\pgfqpoint{4.736542in}{1.629577in}}%
\pgfpathlineto{\pgfqpoint{4.751188in}{1.632816in}}%
\pgfpathlineto{\pgfqpoint{4.765845in}{1.636126in}}%
\pgfpathlineto{\pgfqpoint{4.773964in}{1.650456in}}%
\pgfpathlineto{\pgfqpoint{4.782078in}{1.664710in}}%
\pgfpathlineto{\pgfqpoint{4.790186in}{1.678887in}}%
\pgfpathlineto{\pgfqpoint{4.798289in}{1.692982in}}%
\pgfpathlineto{\pgfqpoint{4.783632in}{1.689436in}}%
\pgfpathlineto{\pgfqpoint{4.768987in}{1.685961in}}%
\pgfpathlineto{\pgfqpoint{4.754355in}{1.682556in}}%
\pgfpathlineto{\pgfqpoint{4.739735in}{1.679222in}}%
\pgfpathlineto{\pgfqpoint{4.731631in}{1.665354in}}%
\pgfpathlineto{\pgfqpoint{4.723522in}{1.651411in}}%
\pgfpathlineto{\pgfqpoint{4.715407in}{1.637394in}}%
\pgfpathlineto{\pgfqpoint{4.707287in}{1.623309in}}%
\pgfpathclose%
\pgfusepath{fill}%
\end{pgfscope}%
\begin{pgfscope}%
\pgfpathrectangle{\pgfqpoint{1.150000in}{0.150000in}}{\pgfqpoint{5.700000in}{5.700000in}}%
\pgfusepath{clip}%
\pgfsetbuttcap%
\pgfsetroundjoin%
\definecolor{currentfill}{rgb}{0.276022,0.044167,0.370164}%
\pgfsetfillcolor{currentfill}%
\pgfsetfillopacity{0.700000}%
\pgfsetlinewidth{0.000000pt}%
\definecolor{currentstroke}{rgb}{0.000000,0.000000,0.000000}%
\pgfsetstrokecolor{currentstroke}%
\pgfsetdash{}{0pt}%
\pgfpathmoveto{\pgfqpoint{3.948355in}{1.139853in}}%
\pgfpathlineto{\pgfqpoint{3.962705in}{1.137686in}}%
\pgfpathlineto{\pgfqpoint{3.977064in}{1.135589in}}%
\pgfpathlineto{\pgfqpoint{3.991430in}{1.133561in}}%
\pgfpathlineto{\pgfqpoint{4.005805in}{1.131602in}}%
\pgfpathlineto{\pgfqpoint{4.014140in}{1.142288in}}%
\pgfpathlineto{\pgfqpoint{4.022470in}{1.153114in}}%
\pgfpathlineto{\pgfqpoint{4.030793in}{1.164076in}}%
\pgfpathlineto{\pgfqpoint{4.039111in}{1.175165in}}%
\pgfpathlineto{\pgfqpoint{4.024747in}{1.176706in}}%
\pgfpathlineto{\pgfqpoint{4.010392in}{1.178317in}}%
\pgfpathlineto{\pgfqpoint{3.996044in}{1.179997in}}%
\pgfpathlineto{\pgfqpoint{3.981706in}{1.181747in}}%
\pgfpathlineto{\pgfqpoint{3.973377in}{1.171067in}}%
\pgfpathlineto{\pgfqpoint{3.965043in}{1.160520in}}%
\pgfpathlineto{\pgfqpoint{3.956702in}{1.150113in}}%
\pgfpathlineto{\pgfqpoint{3.948355in}{1.139853in}}%
\pgfpathclose%
\pgfusepath{fill}%
\end{pgfscope}%
\begin{pgfscope}%
\pgfpathrectangle{\pgfqpoint{1.150000in}{0.150000in}}{\pgfqpoint{5.700000in}{5.700000in}}%
\pgfusepath{clip}%
\pgfsetbuttcap%
\pgfsetroundjoin%
\definecolor{currentfill}{rgb}{0.279574,0.170599,0.479997}%
\pgfsetfillcolor{currentfill}%
\pgfsetfillopacity{0.700000}%
\pgfsetlinewidth{0.000000pt}%
\definecolor{currentstroke}{rgb}{0.000000,0.000000,0.000000}%
\pgfsetstrokecolor{currentstroke}%
\pgfsetdash{}{0pt}%
\pgfpathmoveto{\pgfqpoint{4.402025in}{1.373607in}}%
\pgfpathlineto{\pgfqpoint{4.416522in}{1.374717in}}%
\pgfpathlineto{\pgfqpoint{4.431028in}{1.375897in}}%
\pgfpathlineto{\pgfqpoint{4.445546in}{1.377147in}}%
\pgfpathlineto{\pgfqpoint{4.460074in}{1.378466in}}%
\pgfpathlineto{\pgfqpoint{4.468276in}{1.392490in}}%
\pgfpathlineto{\pgfqpoint{4.476473in}{1.406521in}}%
\pgfpathlineto{\pgfqpoint{4.484666in}{1.420551in}}%
\pgfpathlineto{\pgfqpoint{4.492854in}{1.434578in}}%
\pgfpathlineto{\pgfqpoint{4.478329in}{1.432941in}}%
\pgfpathlineto{\pgfqpoint{4.463815in}{1.431373in}}%
\pgfpathlineto{\pgfqpoint{4.449311in}{1.429876in}}%
\pgfpathlineto{\pgfqpoint{4.434818in}{1.428448in}}%
\pgfpathlineto{\pgfqpoint{4.426627in}{1.414731in}}%
\pgfpathlineto{\pgfqpoint{4.418431in}{1.401016in}}%
\pgfpathlineto{\pgfqpoint{4.410230in}{1.387306in}}%
\pgfpathlineto{\pgfqpoint{4.402025in}{1.373607in}}%
\pgfpathclose%
\pgfusepath{fill}%
\end{pgfscope}%
\begin{pgfscope}%
\pgfpathrectangle{\pgfqpoint{1.150000in}{0.150000in}}{\pgfqpoint{5.700000in}{5.700000in}}%
\pgfusepath{clip}%
\pgfsetbuttcap%
\pgfsetroundjoin%
\definecolor{currentfill}{rgb}{0.199430,0.387607,0.554642}%
\pgfsetfillcolor{currentfill}%
\pgfsetfillopacity{0.700000}%
\pgfsetlinewidth{0.000000pt}%
\definecolor{currentstroke}{rgb}{0.000000,0.000000,0.000000}%
\pgfsetstrokecolor{currentstroke}%
\pgfsetdash{}{0pt}%
\pgfpathmoveto{\pgfqpoint{5.012728in}{1.890290in}}%
\pgfpathlineto{\pgfqpoint{5.027496in}{1.895055in}}%
\pgfpathlineto{\pgfqpoint{5.042277in}{1.899891in}}%
\pgfpathlineto{\pgfqpoint{5.057072in}{1.904799in}}%
\pgfpathlineto{\pgfqpoint{5.071880in}{1.909778in}}%
\pgfpathlineto{\pgfqpoint{5.079896in}{1.923038in}}%
\pgfpathlineto{\pgfqpoint{5.087906in}{1.936164in}}%
\pgfpathlineto{\pgfqpoint{5.095908in}{1.949155in}}%
\pgfpathlineto{\pgfqpoint{5.103903in}{1.962008in}}%
\pgfpathlineto{\pgfqpoint{5.089097in}{1.956878in}}%
\pgfpathlineto{\pgfqpoint{5.074304in}{1.951821in}}%
\pgfpathlineto{\pgfqpoint{5.059525in}{1.946834in}}%
\pgfpathlineto{\pgfqpoint{5.044759in}{1.941920in}}%
\pgfpathlineto{\pgfqpoint{5.036762in}{1.929208in}}%
\pgfpathlineto{\pgfqpoint{5.028758in}{1.916365in}}%
\pgfpathlineto{\pgfqpoint{5.020747in}{1.903392in}}%
\pgfpathlineto{\pgfqpoint{5.012728in}{1.890290in}}%
\pgfpathclose%
\pgfusepath{fill}%
\end{pgfscope}%
\begin{pgfscope}%
\pgfpathrectangle{\pgfqpoint{1.150000in}{0.150000in}}{\pgfqpoint{5.700000in}{5.700000in}}%
\pgfusepath{clip}%
\pgfsetbuttcap%
\pgfsetroundjoin%
\definecolor{currentfill}{rgb}{0.188923,0.410910,0.556326}%
\pgfsetfillcolor{currentfill}%
\pgfsetfillopacity{0.700000}%
\pgfsetlinewidth{0.000000pt}%
\definecolor{currentstroke}{rgb}{0.000000,0.000000,0.000000}%
\pgfsetstrokecolor{currentstroke}%
\pgfsetdash{}{0pt}%
\pgfpathmoveto{\pgfqpoint{5.103903in}{1.962008in}}%
\pgfpathlineto{\pgfqpoint{5.118722in}{1.967209in}}%
\pgfpathlineto{\pgfqpoint{5.133556in}{1.972482in}}%
\pgfpathlineto{\pgfqpoint{5.148403in}{1.977826in}}%
\pgfpathlineto{\pgfqpoint{5.156389in}{1.990643in}}%
\pgfpathlineto{\pgfqpoint{5.164366in}{2.003314in}}%
\pgfpathlineto{\pgfqpoint{5.172336in}{2.015838in}}%
\pgfpathlineto{\pgfqpoint{5.180298in}{2.028214in}}%
\pgfpathlineto{\pgfqpoint{5.165454in}{2.022741in}}%
\pgfpathlineto{\pgfqpoint{5.150623in}{2.017340in}}%
\pgfpathlineto{\pgfqpoint{5.135806in}{2.012011in}}%
\pgfpathlineto{\pgfqpoint{5.127842in}{1.999725in}}%
\pgfpathlineto{\pgfqpoint{5.119870in}{1.987295in}}%
\pgfpathlineto{\pgfqpoint{5.111890in}{1.974722in}}%
\pgfpathlineto{\pgfqpoint{5.103903in}{1.962008in}}%
\pgfpathclose%
\pgfusepath{fill}%
\end{pgfscope}%
\begin{pgfscope}%
\pgfpathrectangle{\pgfqpoint{1.150000in}{0.150000in}}{\pgfqpoint{5.700000in}{5.700000in}}%
\pgfusepath{clip}%
\pgfsetbuttcap%
\pgfsetroundjoin%
\definecolor{currentfill}{rgb}{0.274128,0.199721,0.498911}%
\pgfsetfillcolor{currentfill}%
\pgfsetfillopacity{0.700000}%
\pgfsetlinewidth{0.000000pt}%
\definecolor{currentstroke}{rgb}{0.000000,0.000000,0.000000}%
\pgfsetstrokecolor{currentstroke}%
\pgfsetdash{}{0pt}%
\pgfpathmoveto{\pgfqpoint{4.492854in}{1.434578in}}%
\pgfpathlineto{\pgfqpoint{4.507390in}{1.436285in}}%
\pgfpathlineto{\pgfqpoint{4.521936in}{1.438062in}}%
\pgfpathlineto{\pgfqpoint{4.536494in}{1.439908in}}%
\pgfpathlineto{\pgfqpoint{4.551063in}{1.441824in}}%
\pgfpathlineto{\pgfqpoint{4.559244in}{1.456148in}}%
\pgfpathlineto{\pgfqpoint{4.567421in}{1.470454in}}%
\pgfpathlineto{\pgfqpoint{4.575592in}{1.484737in}}%
\pgfpathlineto{\pgfqpoint{4.583759in}{1.498994in}}%
\pgfpathlineto{\pgfqpoint{4.569193in}{1.496780in}}%
\pgfpathlineto{\pgfqpoint{4.554637in}{1.494636in}}%
\pgfpathlineto{\pgfqpoint{4.540093in}{1.492562in}}%
\pgfpathlineto{\pgfqpoint{4.525559in}{1.490557in}}%
\pgfpathlineto{\pgfqpoint{4.517390in}{1.476590in}}%
\pgfpathlineto{\pgfqpoint{4.509216in}{1.462602in}}%
\pgfpathlineto{\pgfqpoint{4.501037in}{1.448596in}}%
\pgfpathlineto{\pgfqpoint{4.492854in}{1.434578in}}%
\pgfpathclose%
\pgfusepath{fill}%
\end{pgfscope}%
\begin{pgfscope}%
\pgfpathrectangle{\pgfqpoint{1.150000in}{0.150000in}}{\pgfqpoint{5.700000in}{5.700000in}}%
\pgfusepath{clip}%
\pgfsetbuttcap%
\pgfsetroundjoin%
\definecolor{currentfill}{rgb}{0.233603,0.313828,0.543914}%
\pgfsetfillcolor{currentfill}%
\pgfsetfillopacity{0.700000}%
\pgfsetlinewidth{0.000000pt}%
\definecolor{currentstroke}{rgb}{0.000000,0.000000,0.000000}%
\pgfsetstrokecolor{currentstroke}%
\pgfsetdash{}{0pt}%
\pgfpathmoveto{\pgfqpoint{4.798289in}{1.692982in}}%
\pgfpathlineto{\pgfqpoint{4.812958in}{1.696599in}}%
\pgfpathlineto{\pgfqpoint{4.827639in}{1.700286in}}%
\pgfpathlineto{\pgfqpoint{4.842333in}{1.704044in}}%
\pgfpathlineto{\pgfqpoint{4.857040in}{1.707872in}}%
\pgfpathlineto{\pgfqpoint{4.865136in}{1.722106in}}%
\pgfpathlineto{\pgfqpoint{4.873226in}{1.736247in}}%
\pgfpathlineto{\pgfqpoint{4.881310in}{1.750293in}}%
\pgfpathlineto{\pgfqpoint{4.889388in}{1.764239in}}%
\pgfpathlineto{\pgfqpoint{4.874683in}{1.760196in}}%
\pgfpathlineto{\pgfqpoint{4.859990in}{1.756223in}}%
\pgfpathlineto{\pgfqpoint{4.845310in}{1.752321in}}%
\pgfpathlineto{\pgfqpoint{4.830642in}{1.748490in}}%
\pgfpathlineto{\pgfqpoint{4.822562in}{1.734750in}}%
\pgfpathlineto{\pgfqpoint{4.814477in}{1.720917in}}%
\pgfpathlineto{\pgfqpoint{4.806386in}{1.706993in}}%
\pgfpathlineto{\pgfqpoint{4.798289in}{1.692982in}}%
\pgfpathclose%
\pgfusepath{fill}%
\end{pgfscope}%
\begin{pgfscope}%
\pgfpathrectangle{\pgfqpoint{1.150000in}{0.150000in}}{\pgfqpoint{5.700000in}{5.700000in}}%
\pgfusepath{clip}%
\pgfsetbuttcap%
\pgfsetroundjoin%
\definecolor{currentfill}{rgb}{0.265145,0.232956,0.516599}%
\pgfsetfillcolor{currentfill}%
\pgfsetfillopacity{0.700000}%
\pgfsetlinewidth{0.000000pt}%
\definecolor{currentstroke}{rgb}{0.000000,0.000000,0.000000}%
\pgfsetstrokecolor{currentstroke}%
\pgfsetdash{}{0pt}%
\pgfpathmoveto{\pgfqpoint{4.583759in}{1.498994in}}%
\pgfpathlineto{\pgfqpoint{4.598337in}{1.501278in}}%
\pgfpathlineto{\pgfqpoint{4.612927in}{1.503633in}}%
\pgfpathlineto{\pgfqpoint{4.627528in}{1.506057in}}%
\pgfpathlineto{\pgfqpoint{4.642140in}{1.508551in}}%
\pgfpathlineto{\pgfqpoint{4.650301in}{1.523062in}}%
\pgfpathlineto{\pgfqpoint{4.658457in}{1.537534in}}%
\pgfpathlineto{\pgfqpoint{4.666608in}{1.551962in}}%
\pgfpathlineto{\pgfqpoint{4.674754in}{1.566342in}}%
\pgfpathlineto{\pgfqpoint{4.660143in}{1.563570in}}%
\pgfpathlineto{\pgfqpoint{4.645543in}{1.560868in}}%
\pgfpathlineto{\pgfqpoint{4.630955in}{1.558237in}}%
\pgfpathlineto{\pgfqpoint{4.616379in}{1.555676in}}%
\pgfpathlineto{\pgfqpoint{4.608231in}{1.541565in}}%
\pgfpathlineto{\pgfqpoint{4.600079in}{1.527412in}}%
\pgfpathlineto{\pgfqpoint{4.591922in}{1.513221in}}%
\pgfpathlineto{\pgfqpoint{4.583759in}{1.498994in}}%
\pgfpathclose%
\pgfusepath{fill}%
\end{pgfscope}%
\begin{pgfscope}%
\pgfpathrectangle{\pgfqpoint{1.150000in}{0.150000in}}{\pgfqpoint{5.700000in}{5.700000in}}%
\pgfusepath{clip}%
\pgfsetbuttcap%
\pgfsetroundjoin%
\definecolor{currentfill}{rgb}{0.220057,0.343307,0.549413}%
\pgfsetfillcolor{currentfill}%
\pgfsetfillopacity{0.700000}%
\pgfsetlinewidth{0.000000pt}%
\definecolor{currentstroke}{rgb}{0.000000,0.000000,0.000000}%
\pgfsetstrokecolor{currentstroke}%
\pgfsetdash{}{0pt}%
\pgfpathmoveto{\pgfqpoint{4.889388in}{1.764239in}}%
\pgfpathlineto{\pgfqpoint{4.904107in}{1.768354in}}%
\pgfpathlineto{\pgfqpoint{4.918838in}{1.772539in}}%
\pgfpathlineto{\pgfqpoint{4.933582in}{1.776796in}}%
\pgfpathlineto{\pgfqpoint{4.948339in}{1.781124in}}%
\pgfpathlineto{\pgfqpoint{4.956410in}{1.795171in}}%
\pgfpathlineto{\pgfqpoint{4.964475in}{1.809108in}}%
\pgfpathlineto{\pgfqpoint{4.972534in}{1.822933in}}%
\pgfpathlineto{\pgfqpoint{4.980586in}{1.836644in}}%
\pgfpathlineto{\pgfqpoint{4.965830in}{1.832122in}}%
\pgfpathlineto{\pgfqpoint{4.951087in}{1.827672in}}%
\pgfpathlineto{\pgfqpoint{4.936357in}{1.823293in}}%
\pgfpathlineto{\pgfqpoint{4.921640in}{1.818985in}}%
\pgfpathlineto{\pgfqpoint{4.913586in}{1.805460in}}%
\pgfpathlineto{\pgfqpoint{4.905527in}{1.791826in}}%
\pgfpathlineto{\pgfqpoint{4.897461in}{1.778085in}}%
\pgfpathlineto{\pgfqpoint{4.889388in}{1.764239in}}%
\pgfpathclose%
\pgfusepath{fill}%
\end{pgfscope}%
\begin{pgfscope}%
\pgfpathrectangle{\pgfqpoint{1.150000in}{0.150000in}}{\pgfqpoint{5.700000in}{5.700000in}}%
\pgfusepath{clip}%
\pgfsetbuttcap%
\pgfsetroundjoin%
\definecolor{currentfill}{rgb}{0.282656,0.100196,0.422160}%
\pgfsetfillcolor{currentfill}%
\pgfsetfillopacity{0.700000}%
\pgfsetlinewidth{0.000000pt}%
\definecolor{currentstroke}{rgb}{0.000000,0.000000,0.000000}%
\pgfsetstrokecolor{currentstroke}%
\pgfsetdash{}{0pt}%
\pgfpathmoveto{\pgfqpoint{4.187470in}{1.214038in}}%
\pgfpathlineto{\pgfqpoint{4.201904in}{1.213519in}}%
\pgfpathlineto{\pgfqpoint{4.216347in}{1.213069in}}%
\pgfpathlineto{\pgfqpoint{4.230799in}{1.212689in}}%
\pgfpathlineto{\pgfqpoint{4.245261in}{1.212377in}}%
\pgfpathlineto{\pgfqpoint{4.253527in}{1.225174in}}%
\pgfpathlineto{\pgfqpoint{4.261789in}{1.238049in}}%
\pgfpathlineto{\pgfqpoint{4.270046in}{1.250995in}}%
\pgfpathlineto{\pgfqpoint{4.278298in}{1.264008in}}%
\pgfpathlineto{\pgfqpoint{4.263842in}{1.263941in}}%
\pgfpathlineto{\pgfqpoint{4.249396in}{1.263943in}}%
\pgfpathlineto{\pgfqpoint{4.234959in}{1.264015in}}%
\pgfpathlineto{\pgfqpoint{4.220532in}{1.264156in}}%
\pgfpathlineto{\pgfqpoint{4.212274in}{1.251514in}}%
\pgfpathlineto{\pgfqpoint{4.204011in}{1.238943in}}%
\pgfpathlineto{\pgfqpoint{4.195743in}{1.226449in}}%
\pgfpathlineto{\pgfqpoint{4.187470in}{1.214038in}}%
\pgfpathclose%
\pgfusepath{fill}%
\end{pgfscope}%
\begin{pgfscope}%
\pgfpathrectangle{\pgfqpoint{1.150000in}{0.150000in}}{\pgfqpoint{5.700000in}{5.700000in}}%
\pgfusepath{clip}%
\pgfsetbuttcap%
\pgfsetroundjoin%
\definecolor{currentfill}{rgb}{0.280894,0.078907,0.402329}%
\pgfsetfillcolor{currentfill}%
\pgfsetfillopacity{0.700000}%
\pgfsetlinewidth{0.000000pt}%
\definecolor{currentstroke}{rgb}{0.000000,0.000000,0.000000}%
\pgfsetstrokecolor{currentstroke}%
\pgfsetdash{}{0pt}%
\pgfpathmoveto{\pgfqpoint{4.096649in}{1.169696in}}%
\pgfpathlineto{\pgfqpoint{4.111056in}{1.168502in}}%
\pgfpathlineto{\pgfqpoint{4.125471in}{1.167377in}}%
\pgfpathlineto{\pgfqpoint{4.139894in}{1.166322in}}%
\pgfpathlineto{\pgfqpoint{4.154327in}{1.165335in}}%
\pgfpathlineto{\pgfqpoint{4.162620in}{1.177358in}}%
\pgfpathlineto{\pgfqpoint{4.170909in}{1.189487in}}%
\pgfpathlineto{\pgfqpoint{4.179192in}{1.201715in}}%
\pgfpathlineto{\pgfqpoint{4.187470in}{1.214038in}}%
\pgfpathlineto{\pgfqpoint{4.173046in}{1.214626in}}%
\pgfpathlineto{\pgfqpoint{4.158630in}{1.215284in}}%
\pgfpathlineto{\pgfqpoint{4.144223in}{1.216011in}}%
\pgfpathlineto{\pgfqpoint{4.129826in}{1.216807in}}%
\pgfpathlineto{\pgfqpoint{4.121540in}{1.204874in}}%
\pgfpathlineto{\pgfqpoint{4.113248in}{1.193041in}}%
\pgfpathlineto{\pgfqpoint{4.104952in}{1.181313in}}%
\pgfpathlineto{\pgfqpoint{4.096649in}{1.169696in}}%
\pgfpathclose%
\pgfusepath{fill}%
\end{pgfscope}%
\begin{pgfscope}%
\pgfpathrectangle{\pgfqpoint{1.150000in}{0.150000in}}{\pgfqpoint{5.700000in}{5.700000in}}%
\pgfusepath{clip}%
\pgfsetbuttcap%
\pgfsetroundjoin%
\definecolor{currentfill}{rgb}{0.283187,0.125848,0.444960}%
\pgfsetfillcolor{currentfill}%
\pgfsetfillopacity{0.700000}%
\pgfsetlinewidth{0.000000pt}%
\definecolor{currentstroke}{rgb}{0.000000,0.000000,0.000000}%
\pgfsetstrokecolor{currentstroke}%
\pgfsetdash{}{0pt}%
\pgfpathmoveto{\pgfqpoint{4.278298in}{1.264008in}}%
\pgfpathlineto{\pgfqpoint{4.292763in}{1.264144in}}%
\pgfpathlineto{\pgfqpoint{4.307238in}{1.264350in}}%
\pgfpathlineto{\pgfqpoint{4.321723in}{1.264625in}}%
\pgfpathlineto{\pgfqpoint{4.336217in}{1.264968in}}%
\pgfpathlineto{\pgfqpoint{4.344459in}{1.278408in}}%
\pgfpathlineto{\pgfqpoint{4.352697in}{1.291899in}}%
\pgfpathlineto{\pgfqpoint{4.360930in}{1.305434in}}%
\pgfpathlineto{\pgfqpoint{4.369158in}{1.319009in}}%
\pgfpathlineto{\pgfqpoint{4.354668in}{1.318306in}}%
\pgfpathlineto{\pgfqpoint{4.340188in}{1.317673in}}%
\pgfpathlineto{\pgfqpoint{4.325718in}{1.317109in}}%
\pgfpathlineto{\pgfqpoint{4.311258in}{1.316615in}}%
\pgfpathlineto{\pgfqpoint{4.303025in}{1.303390in}}%
\pgfpathlineto{\pgfqpoint{4.294787in}{1.290211in}}%
\pgfpathlineto{\pgfqpoint{4.286545in}{1.277082in}}%
\pgfpathlineto{\pgfqpoint{4.278298in}{1.264008in}}%
\pgfpathclose%
\pgfusepath{fill}%
\end{pgfscope}%
\begin{pgfscope}%
\pgfpathrectangle{\pgfqpoint{1.150000in}{0.150000in}}{\pgfqpoint{5.700000in}{5.700000in}}%
\pgfusepath{clip}%
\pgfsetbuttcap%
\pgfsetroundjoin%
\definecolor{currentfill}{rgb}{0.253935,0.265254,0.529983}%
\pgfsetfillcolor{currentfill}%
\pgfsetfillopacity{0.700000}%
\pgfsetlinewidth{0.000000pt}%
\definecolor{currentstroke}{rgb}{0.000000,0.000000,0.000000}%
\pgfsetstrokecolor{currentstroke}%
\pgfsetdash{}{0pt}%
\pgfpathmoveto{\pgfqpoint{4.674754in}{1.566342in}}%
\pgfpathlineto{\pgfqpoint{4.689376in}{1.569184in}}%
\pgfpathlineto{\pgfqpoint{4.704011in}{1.572096in}}%
\pgfpathlineto{\pgfqpoint{4.718658in}{1.575079in}}%
\pgfpathlineto{\pgfqpoint{4.733316in}{1.578131in}}%
\pgfpathlineto{\pgfqpoint{4.741456in}{1.592725in}}%
\pgfpathlineto{\pgfqpoint{4.749591in}{1.607257in}}%
\pgfpathlineto{\pgfqpoint{4.757721in}{1.621726in}}%
\pgfpathlineto{\pgfqpoint{4.765845in}{1.636126in}}%
\pgfpathlineto{\pgfqpoint{4.751188in}{1.632816in}}%
\pgfpathlineto{\pgfqpoint{4.736542in}{1.629577in}}%
\pgfpathlineto{\pgfqpoint{4.721908in}{1.626408in}}%
\pgfpathlineto{\pgfqpoint{4.707287in}{1.623309in}}%
\pgfpathlineto{\pgfqpoint{4.699161in}{1.609157in}}%
\pgfpathlineto{\pgfqpoint{4.691030in}{1.594943in}}%
\pgfpathlineto{\pgfqpoint{4.682895in}{1.580670in}}%
\pgfpathlineto{\pgfqpoint{4.674754in}{1.566342in}}%
\pgfpathclose%
\pgfusepath{fill}%
\end{pgfscope}%
\begin{pgfscope}%
\pgfpathrectangle{\pgfqpoint{1.150000in}{0.150000in}}{\pgfqpoint{5.700000in}{5.700000in}}%
\pgfusepath{clip}%
\pgfsetbuttcap%
\pgfsetroundjoin%
\definecolor{currentfill}{rgb}{0.277941,0.056324,0.381191}%
\pgfsetfillcolor{currentfill}%
\pgfsetfillopacity{0.700000}%
\pgfsetlinewidth{0.000000pt}%
\definecolor{currentstroke}{rgb}{0.000000,0.000000,0.000000}%
\pgfsetstrokecolor{currentstroke}%
\pgfsetdash{}{0pt}%
\pgfpathmoveto{\pgfqpoint{4.005805in}{1.131602in}}%
\pgfpathlineto{\pgfqpoint{4.020187in}{1.129713in}}%
\pgfpathlineto{\pgfqpoint{4.034578in}{1.127894in}}%
\pgfpathlineto{\pgfqpoint{4.048977in}{1.126143in}}%
\pgfpathlineto{\pgfqpoint{4.063385in}{1.124462in}}%
\pgfpathlineto{\pgfqpoint{4.071709in}{1.135573in}}%
\pgfpathlineto{\pgfqpoint{4.080028in}{1.146820in}}%
\pgfpathlineto{\pgfqpoint{4.088342in}{1.158196in}}%
\pgfpathlineto{\pgfqpoint{4.096649in}{1.169696in}}%
\pgfpathlineto{\pgfqpoint{4.082252in}{1.170959in}}%
\pgfpathlineto{\pgfqpoint{4.067863in}{1.172292in}}%
\pgfpathlineto{\pgfqpoint{4.053483in}{1.173694in}}%
\pgfpathlineto{\pgfqpoint{4.039111in}{1.175165in}}%
\pgfpathlineto{\pgfqpoint{4.030793in}{1.164076in}}%
\pgfpathlineto{\pgfqpoint{4.022470in}{1.153114in}}%
\pgfpathlineto{\pgfqpoint{4.014140in}{1.142288in}}%
\pgfpathlineto{\pgfqpoint{4.005805in}{1.131602in}}%
\pgfpathclose%
\pgfusepath{fill}%
\end{pgfscope}%
\begin{pgfscope}%
\pgfpathrectangle{\pgfqpoint{1.150000in}{0.150000in}}{\pgfqpoint{5.700000in}{5.700000in}}%
\pgfusepath{clip}%
\pgfsetbuttcap%
\pgfsetroundjoin%
\definecolor{currentfill}{rgb}{0.281412,0.155834,0.469201}%
\pgfsetfillcolor{currentfill}%
\pgfsetfillopacity{0.700000}%
\pgfsetlinewidth{0.000000pt}%
\definecolor{currentstroke}{rgb}{0.000000,0.000000,0.000000}%
\pgfsetstrokecolor{currentstroke}%
\pgfsetdash{}{0pt}%
\pgfpathmoveto{\pgfqpoint{4.369158in}{1.319009in}}%
\pgfpathlineto{\pgfqpoint{4.383658in}{1.319781in}}%
\pgfpathlineto{\pgfqpoint{4.398169in}{1.320622in}}%
\pgfpathlineto{\pgfqpoint{4.412690in}{1.321533in}}%
\pgfpathlineto{\pgfqpoint{4.427221in}{1.322513in}}%
\pgfpathlineto{\pgfqpoint{4.435441in}{1.336469in}}%
\pgfpathlineto{\pgfqpoint{4.443656in}{1.350450in}}%
\pgfpathlineto{\pgfqpoint{4.451867in}{1.364451in}}%
\pgfpathlineto{\pgfqpoint{4.460074in}{1.378466in}}%
\pgfpathlineto{\pgfqpoint{4.445546in}{1.377147in}}%
\pgfpathlineto{\pgfqpoint{4.431028in}{1.375897in}}%
\pgfpathlineto{\pgfqpoint{4.416522in}{1.374717in}}%
\pgfpathlineto{\pgfqpoint{4.402025in}{1.373607in}}%
\pgfpathlineto{\pgfqpoint{4.393815in}{1.359923in}}%
\pgfpathlineto{\pgfqpoint{4.385601in}{1.346258in}}%
\pgfpathlineto{\pgfqpoint{4.377382in}{1.332619in}}%
\pgfpathlineto{\pgfqpoint{4.369158in}{1.319009in}}%
\pgfpathclose%
\pgfusepath{fill}%
\end{pgfscope}%
\begin{pgfscope}%
\pgfpathrectangle{\pgfqpoint{1.150000in}{0.150000in}}{\pgfqpoint{5.700000in}{5.700000in}}%
\pgfusepath{clip}%
\pgfsetbuttcap%
\pgfsetroundjoin%
\definecolor{currentfill}{rgb}{0.204903,0.375746,0.553533}%
\pgfsetfillcolor{currentfill}%
\pgfsetfillopacity{0.700000}%
\pgfsetlinewidth{0.000000pt}%
\definecolor{currentstroke}{rgb}{0.000000,0.000000,0.000000}%
\pgfsetstrokecolor{currentstroke}%
\pgfsetdash{}{0pt}%
\pgfpathmoveto{\pgfqpoint{4.980586in}{1.836644in}}%
\pgfpathlineto{\pgfqpoint{4.995355in}{1.841237in}}%
\pgfpathlineto{\pgfqpoint{5.010138in}{1.845901in}}%
\pgfpathlineto{\pgfqpoint{5.024934in}{1.850636in}}%
\pgfpathlineto{\pgfqpoint{5.039743in}{1.855442in}}%
\pgfpathlineto{\pgfqpoint{5.047788in}{1.869216in}}%
\pgfpathlineto{\pgfqpoint{5.055825in}{1.882865in}}%
\pgfpathlineto{\pgfqpoint{5.063856in}{1.896386in}}%
\pgfpathlineto{\pgfqpoint{5.071880in}{1.909778in}}%
\pgfpathlineto{\pgfqpoint{5.057072in}{1.904799in}}%
\pgfpathlineto{\pgfqpoint{5.042277in}{1.899891in}}%
\pgfpathlineto{\pgfqpoint{5.027496in}{1.895055in}}%
\pgfpathlineto{\pgfqpoint{5.012728in}{1.890290in}}%
\pgfpathlineto{\pgfqpoint{5.004703in}{1.877063in}}%
\pgfpathlineto{\pgfqpoint{4.996671in}{1.863711in}}%
\pgfpathlineto{\pgfqpoint{4.988632in}{1.850237in}}%
\pgfpathlineto{\pgfqpoint{4.980586in}{1.836644in}}%
\pgfpathclose%
\pgfusepath{fill}%
\end{pgfscope}%
\begin{pgfscope}%
\pgfpathrectangle{\pgfqpoint{1.150000in}{0.150000in}}{\pgfqpoint{5.700000in}{5.700000in}}%
\pgfusepath{clip}%
\pgfsetbuttcap%
\pgfsetroundjoin%
\definecolor{currentfill}{rgb}{0.277134,0.185228,0.489898}%
\pgfsetfillcolor{currentfill}%
\pgfsetfillopacity{0.700000}%
\pgfsetlinewidth{0.000000pt}%
\definecolor{currentstroke}{rgb}{0.000000,0.000000,0.000000}%
\pgfsetstrokecolor{currentstroke}%
\pgfsetdash{}{0pt}%
\pgfpathmoveto{\pgfqpoint{4.460074in}{1.378466in}}%
\pgfpathlineto{\pgfqpoint{4.474612in}{1.379854in}}%
\pgfpathlineto{\pgfqpoint{4.489161in}{1.381312in}}%
\pgfpathlineto{\pgfqpoint{4.503721in}{1.382840in}}%
\pgfpathlineto{\pgfqpoint{4.518292in}{1.384437in}}%
\pgfpathlineto{\pgfqpoint{4.526492in}{1.398788in}}%
\pgfpathlineto{\pgfqpoint{4.534687in}{1.413140in}}%
\pgfpathlineto{\pgfqpoint{4.542877in}{1.427486in}}%
\pgfpathlineto{\pgfqpoint{4.551063in}{1.441824in}}%
\pgfpathlineto{\pgfqpoint{4.536494in}{1.439908in}}%
\pgfpathlineto{\pgfqpoint{4.521936in}{1.438062in}}%
\pgfpathlineto{\pgfqpoint{4.507390in}{1.436285in}}%
\pgfpathlineto{\pgfqpoint{4.492854in}{1.434578in}}%
\pgfpathlineto{\pgfqpoint{4.484666in}{1.420551in}}%
\pgfpathlineto{\pgfqpoint{4.476473in}{1.406521in}}%
\pgfpathlineto{\pgfqpoint{4.468276in}{1.392490in}}%
\pgfpathlineto{\pgfqpoint{4.460074in}{1.378466in}}%
\pgfpathclose%
\pgfusepath{fill}%
\end{pgfscope}%
\begin{pgfscope}%
\pgfpathrectangle{\pgfqpoint{1.150000in}{0.150000in}}{\pgfqpoint{5.700000in}{5.700000in}}%
\pgfusepath{clip}%
\pgfsetbuttcap%
\pgfsetroundjoin%
\definecolor{currentfill}{rgb}{0.241237,0.296485,0.539709}%
\pgfsetfillcolor{currentfill}%
\pgfsetfillopacity{0.700000}%
\pgfsetlinewidth{0.000000pt}%
\definecolor{currentstroke}{rgb}{0.000000,0.000000,0.000000}%
\pgfsetstrokecolor{currentstroke}%
\pgfsetdash{}{0pt}%
\pgfpathmoveto{\pgfqpoint{4.765845in}{1.636126in}}%
\pgfpathlineto{\pgfqpoint{4.780515in}{1.639507in}}%
\pgfpathlineto{\pgfqpoint{4.795197in}{1.642958in}}%
\pgfpathlineto{\pgfqpoint{4.809892in}{1.646479in}}%
\pgfpathlineto{\pgfqpoint{4.824599in}{1.650071in}}%
\pgfpathlineto{\pgfqpoint{4.832717in}{1.664645in}}%
\pgfpathlineto{\pgfqpoint{4.840830in}{1.679138in}}%
\pgfpathlineto{\pgfqpoint{4.848938in}{1.693549in}}%
\pgfpathlineto{\pgfqpoint{4.857040in}{1.707872in}}%
\pgfpathlineto{\pgfqpoint{4.842333in}{1.704044in}}%
\pgfpathlineto{\pgfqpoint{4.827639in}{1.700286in}}%
\pgfpathlineto{\pgfqpoint{4.812958in}{1.696599in}}%
\pgfpathlineto{\pgfqpoint{4.798289in}{1.692982in}}%
\pgfpathlineto{\pgfqpoint{4.790186in}{1.678887in}}%
\pgfpathlineto{\pgfqpoint{4.782078in}{1.664710in}}%
\pgfpathlineto{\pgfqpoint{4.773964in}{1.650456in}}%
\pgfpathlineto{\pgfqpoint{4.765845in}{1.636126in}}%
\pgfpathclose%
\pgfusepath{fill}%
\end{pgfscope}%
\begin{pgfscope}%
\pgfpathrectangle{\pgfqpoint{1.150000in}{0.150000in}}{\pgfqpoint{5.700000in}{5.700000in}}%
\pgfusepath{clip}%
\pgfsetbuttcap%
\pgfsetroundjoin%
\definecolor{currentfill}{rgb}{0.192357,0.403199,0.555836}%
\pgfsetfillcolor{currentfill}%
\pgfsetfillopacity{0.700000}%
\pgfsetlinewidth{0.000000pt}%
\definecolor{currentstroke}{rgb}{0.000000,0.000000,0.000000}%
\pgfsetstrokecolor{currentstroke}%
\pgfsetdash{}{0pt}%
\pgfpathmoveto{\pgfqpoint{5.071880in}{1.909778in}}%
\pgfpathlineto{\pgfqpoint{5.086701in}{1.914829in}}%
\pgfpathlineto{\pgfqpoint{5.101537in}{1.919951in}}%
\pgfpathlineto{\pgfqpoint{5.116386in}{1.925144in}}%
\pgfpathlineto{\pgfqpoint{5.124401in}{1.938524in}}%
\pgfpathlineto{\pgfqpoint{5.132409in}{1.951765in}}%
\pgfpathlineto{\pgfqpoint{5.140410in}{1.964867in}}%
\pgfpathlineto{\pgfqpoint{5.148403in}{1.977826in}}%
\pgfpathlineto{\pgfqpoint{5.133556in}{1.972482in}}%
\pgfpathlineto{\pgfqpoint{5.118722in}{1.967209in}}%
\pgfpathlineto{\pgfqpoint{5.103903in}{1.962008in}}%
\pgfpathlineto{\pgfqpoint{5.095908in}{1.949155in}}%
\pgfpathlineto{\pgfqpoint{5.087906in}{1.936164in}}%
\pgfpathlineto{\pgfqpoint{5.079896in}{1.923038in}}%
\pgfpathlineto{\pgfqpoint{5.071880in}{1.909778in}}%
\pgfpathclose%
\pgfusepath{fill}%
\end{pgfscope}%
\begin{pgfscope}%
\pgfpathrectangle{\pgfqpoint{1.150000in}{0.150000in}}{\pgfqpoint{5.700000in}{5.700000in}}%
\pgfusepath{clip}%
\pgfsetbuttcap%
\pgfsetroundjoin%
\definecolor{currentfill}{rgb}{0.270595,0.214069,0.507052}%
\pgfsetfillcolor{currentfill}%
\pgfsetfillopacity{0.700000}%
\pgfsetlinewidth{0.000000pt}%
\definecolor{currentstroke}{rgb}{0.000000,0.000000,0.000000}%
\pgfsetstrokecolor{currentstroke}%
\pgfsetdash{}{0pt}%
\pgfpathmoveto{\pgfqpoint{4.551063in}{1.441824in}}%
\pgfpathlineto{\pgfqpoint{4.565643in}{1.443810in}}%
\pgfpathlineto{\pgfqpoint{4.580234in}{1.445865in}}%
\pgfpathlineto{\pgfqpoint{4.594836in}{1.447991in}}%
\pgfpathlineto{\pgfqpoint{4.609449in}{1.450186in}}%
\pgfpathlineto{\pgfqpoint{4.617629in}{1.464816in}}%
\pgfpathlineto{\pgfqpoint{4.625804in}{1.479423in}}%
\pgfpathlineto{\pgfqpoint{4.633974in}{1.494003in}}%
\pgfpathlineto{\pgfqpoint{4.642140in}{1.508551in}}%
\pgfpathlineto{\pgfqpoint{4.627528in}{1.506057in}}%
\pgfpathlineto{\pgfqpoint{4.612927in}{1.503633in}}%
\pgfpathlineto{\pgfqpoint{4.598337in}{1.501278in}}%
\pgfpathlineto{\pgfqpoint{4.583759in}{1.498994in}}%
\pgfpathlineto{\pgfqpoint{4.575592in}{1.484737in}}%
\pgfpathlineto{\pgfqpoint{4.567421in}{1.470454in}}%
\pgfpathlineto{\pgfqpoint{4.559244in}{1.456148in}}%
\pgfpathlineto{\pgfqpoint{4.551063in}{1.441824in}}%
\pgfpathclose%
\pgfusepath{fill}%
\end{pgfscope}%
\begin{pgfscope}%
\pgfpathrectangle{\pgfqpoint{1.150000in}{0.150000in}}{\pgfqpoint{5.700000in}{5.700000in}}%
\pgfusepath{clip}%
\pgfsetbuttcap%
\pgfsetroundjoin%
\definecolor{currentfill}{rgb}{0.225863,0.330805,0.547314}%
\pgfsetfillcolor{currentfill}%
\pgfsetfillopacity{0.700000}%
\pgfsetlinewidth{0.000000pt}%
\definecolor{currentstroke}{rgb}{0.000000,0.000000,0.000000}%
\pgfsetstrokecolor{currentstroke}%
\pgfsetdash{}{0pt}%
\pgfpathmoveto{\pgfqpoint{4.857040in}{1.707872in}}%
\pgfpathlineto{\pgfqpoint{4.871759in}{1.711772in}}%
\pgfpathlineto{\pgfqpoint{4.886491in}{1.715742in}}%
\pgfpathlineto{\pgfqpoint{4.901235in}{1.719783in}}%
\pgfpathlineto{\pgfqpoint{4.915992in}{1.723894in}}%
\pgfpathlineto{\pgfqpoint{4.924088in}{1.738352in}}%
\pgfpathlineto{\pgfqpoint{4.932178in}{1.752711in}}%
\pgfpathlineto{\pgfqpoint{4.940262in}{1.766970in}}%
\pgfpathlineto{\pgfqpoint{4.948339in}{1.781124in}}%
\pgfpathlineto{\pgfqpoint{4.933582in}{1.776796in}}%
\pgfpathlineto{\pgfqpoint{4.918838in}{1.772539in}}%
\pgfpathlineto{\pgfqpoint{4.904107in}{1.768354in}}%
\pgfpathlineto{\pgfqpoint{4.889388in}{1.764239in}}%
\pgfpathlineto{\pgfqpoint{4.881310in}{1.750293in}}%
\pgfpathlineto{\pgfqpoint{4.873226in}{1.736247in}}%
\pgfpathlineto{\pgfqpoint{4.865136in}{1.722106in}}%
\pgfpathlineto{\pgfqpoint{4.857040in}{1.707872in}}%
\pgfpathclose%
\pgfusepath{fill}%
\end{pgfscope}%
\begin{pgfscope}%
\pgfpathrectangle{\pgfqpoint{1.150000in}{0.150000in}}{\pgfqpoint{5.700000in}{5.700000in}}%
\pgfusepath{clip}%
\pgfsetbuttcap%
\pgfsetroundjoin%
\definecolor{currentfill}{rgb}{0.258965,0.251537,0.524736}%
\pgfsetfillcolor{currentfill}%
\pgfsetfillopacity{0.700000}%
\pgfsetlinewidth{0.000000pt}%
\definecolor{currentstroke}{rgb}{0.000000,0.000000,0.000000}%
\pgfsetstrokecolor{currentstroke}%
\pgfsetdash{}{0pt}%
\pgfpathmoveto{\pgfqpoint{4.642140in}{1.508551in}}%
\pgfpathlineto{\pgfqpoint{4.656764in}{1.511114in}}%
\pgfpathlineto{\pgfqpoint{4.671399in}{1.513748in}}%
\pgfpathlineto{\pgfqpoint{4.686047in}{1.516452in}}%
\pgfpathlineto{\pgfqpoint{4.700706in}{1.519226in}}%
\pgfpathlineto{\pgfqpoint{4.708866in}{1.534024in}}%
\pgfpathlineto{\pgfqpoint{4.717021in}{1.548777in}}%
\pgfpathlineto{\pgfqpoint{4.725171in}{1.563481in}}%
\pgfpathlineto{\pgfqpoint{4.733316in}{1.578131in}}%
\pgfpathlineto{\pgfqpoint{4.718658in}{1.575079in}}%
\pgfpathlineto{\pgfqpoint{4.704011in}{1.572096in}}%
\pgfpathlineto{\pgfqpoint{4.689376in}{1.569184in}}%
\pgfpathlineto{\pgfqpoint{4.674754in}{1.566342in}}%
\pgfpathlineto{\pgfqpoint{4.666608in}{1.551962in}}%
\pgfpathlineto{\pgfqpoint{4.658457in}{1.537534in}}%
\pgfpathlineto{\pgfqpoint{4.650301in}{1.523062in}}%
\pgfpathlineto{\pgfqpoint{4.642140in}{1.508551in}}%
\pgfpathclose%
\pgfusepath{fill}%
\end{pgfscope}%
\begin{pgfscope}%
\pgfpathrectangle{\pgfqpoint{1.150000in}{0.150000in}}{\pgfqpoint{5.700000in}{5.700000in}}%
\pgfusepath{clip}%
\pgfsetbuttcap%
\pgfsetroundjoin%
\definecolor{currentfill}{rgb}{0.281924,0.089666,0.412415}%
\pgfsetfillcolor{currentfill}%
\pgfsetfillopacity{0.700000}%
\pgfsetlinewidth{0.000000pt}%
\definecolor{currentstroke}{rgb}{0.000000,0.000000,0.000000}%
\pgfsetstrokecolor{currentstroke}%
\pgfsetdash{}{0pt}%
\pgfpathmoveto{\pgfqpoint{4.154327in}{1.165335in}}%
\pgfpathlineto{\pgfqpoint{4.168768in}{1.164418in}}%
\pgfpathlineto{\pgfqpoint{4.183219in}{1.163569in}}%
\pgfpathlineto{\pgfqpoint{4.197678in}{1.162790in}}%
\pgfpathlineto{\pgfqpoint{4.212147in}{1.162079in}}%
\pgfpathlineto{\pgfqpoint{4.220433in}{1.174508in}}%
\pgfpathlineto{\pgfqpoint{4.228714in}{1.187038in}}%
\pgfpathlineto{\pgfqpoint{4.236990in}{1.199663in}}%
\pgfpathlineto{\pgfqpoint{4.245261in}{1.212377in}}%
\pgfpathlineto{\pgfqpoint{4.230799in}{1.212689in}}%
\pgfpathlineto{\pgfqpoint{4.216347in}{1.213069in}}%
\pgfpathlineto{\pgfqpoint{4.201904in}{1.213519in}}%
\pgfpathlineto{\pgfqpoint{4.187470in}{1.214038in}}%
\pgfpathlineto{\pgfqpoint{4.179192in}{1.201715in}}%
\pgfpathlineto{\pgfqpoint{4.170909in}{1.189487in}}%
\pgfpathlineto{\pgfqpoint{4.162620in}{1.177358in}}%
\pgfpathlineto{\pgfqpoint{4.154327in}{1.165335in}}%
\pgfpathclose%
\pgfusepath{fill}%
\end{pgfscope}%
\begin{pgfscope}%
\pgfpathrectangle{\pgfqpoint{1.150000in}{0.150000in}}{\pgfqpoint{5.700000in}{5.700000in}}%
\pgfusepath{clip}%
\pgfsetbuttcap%
\pgfsetroundjoin%
\definecolor{currentfill}{rgb}{0.283091,0.110553,0.431554}%
\pgfsetfillcolor{currentfill}%
\pgfsetfillopacity{0.700000}%
\pgfsetlinewidth{0.000000pt}%
\definecolor{currentstroke}{rgb}{0.000000,0.000000,0.000000}%
\pgfsetstrokecolor{currentstroke}%
\pgfsetdash{}{0pt}%
\pgfpathmoveto{\pgfqpoint{4.245261in}{1.212377in}}%
\pgfpathlineto{\pgfqpoint{4.259732in}{1.212135in}}%
\pgfpathlineto{\pgfqpoint{4.274213in}{1.211961in}}%
\pgfpathlineto{\pgfqpoint{4.288703in}{1.211857in}}%
\pgfpathlineto{\pgfqpoint{4.303203in}{1.211821in}}%
\pgfpathlineto{\pgfqpoint{4.311463in}{1.225005in}}%
\pgfpathlineto{\pgfqpoint{4.319719in}{1.238261in}}%
\pgfpathlineto{\pgfqpoint{4.327970in}{1.251584in}}%
\pgfpathlineto{\pgfqpoint{4.336217in}{1.264968in}}%
\pgfpathlineto{\pgfqpoint{4.321723in}{1.264625in}}%
\pgfpathlineto{\pgfqpoint{4.307238in}{1.264350in}}%
\pgfpathlineto{\pgfqpoint{4.292763in}{1.264144in}}%
\pgfpathlineto{\pgfqpoint{4.278298in}{1.264008in}}%
\pgfpathlineto{\pgfqpoint{4.270046in}{1.250995in}}%
\pgfpathlineto{\pgfqpoint{4.261789in}{1.238049in}}%
\pgfpathlineto{\pgfqpoint{4.253527in}{1.225174in}}%
\pgfpathlineto{\pgfqpoint{4.245261in}{1.212377in}}%
\pgfpathclose%
\pgfusepath{fill}%
\end{pgfscope}%
\begin{pgfscope}%
\pgfpathrectangle{\pgfqpoint{1.150000in}{0.150000in}}{\pgfqpoint{5.700000in}{5.700000in}}%
\pgfusepath{clip}%
\pgfsetbuttcap%
\pgfsetroundjoin%
\definecolor{currentfill}{rgb}{0.279566,0.067836,0.391917}%
\pgfsetfillcolor{currentfill}%
\pgfsetfillopacity{0.700000}%
\pgfsetlinewidth{0.000000pt}%
\definecolor{currentstroke}{rgb}{0.000000,0.000000,0.000000}%
\pgfsetstrokecolor{currentstroke}%
\pgfsetdash{}{0pt}%
\pgfpathmoveto{\pgfqpoint{4.063385in}{1.124462in}}%
\pgfpathlineto{\pgfqpoint{4.077801in}{1.122850in}}%
\pgfpathlineto{\pgfqpoint{4.092225in}{1.121306in}}%
\pgfpathlineto{\pgfqpoint{4.106658in}{1.119832in}}%
\pgfpathlineto{\pgfqpoint{4.121099in}{1.118427in}}%
\pgfpathlineto{\pgfqpoint{4.129414in}{1.129964in}}%
\pgfpathlineto{\pgfqpoint{4.137724in}{1.141632in}}%
\pgfpathlineto{\pgfqpoint{4.146028in}{1.153425in}}%
\pgfpathlineto{\pgfqpoint{4.154327in}{1.165335in}}%
\pgfpathlineto{\pgfqpoint{4.139894in}{1.166322in}}%
\pgfpathlineto{\pgfqpoint{4.125471in}{1.167377in}}%
\pgfpathlineto{\pgfqpoint{4.111056in}{1.168502in}}%
\pgfpathlineto{\pgfqpoint{4.096649in}{1.169696in}}%
\pgfpathlineto{\pgfqpoint{4.088342in}{1.158196in}}%
\pgfpathlineto{\pgfqpoint{4.080028in}{1.146820in}}%
\pgfpathlineto{\pgfqpoint{4.071709in}{1.135573in}}%
\pgfpathlineto{\pgfqpoint{4.063385in}{1.124462in}}%
\pgfpathclose%
\pgfusepath{fill}%
\end{pgfscope}%
\begin{pgfscope}%
\pgfpathrectangle{\pgfqpoint{1.150000in}{0.150000in}}{\pgfqpoint{5.700000in}{5.700000in}}%
\pgfusepath{clip}%
\pgfsetbuttcap%
\pgfsetroundjoin%
\definecolor{currentfill}{rgb}{0.282623,0.140926,0.457517}%
\pgfsetfillcolor{currentfill}%
\pgfsetfillopacity{0.700000}%
\pgfsetlinewidth{0.000000pt}%
\definecolor{currentstroke}{rgb}{0.000000,0.000000,0.000000}%
\pgfsetstrokecolor{currentstroke}%
\pgfsetdash{}{0pt}%
\pgfpathmoveto{\pgfqpoint{4.336217in}{1.264968in}}%
\pgfpathlineto{\pgfqpoint{4.350722in}{1.265382in}}%
\pgfpathlineto{\pgfqpoint{4.365236in}{1.265864in}}%
\pgfpathlineto{\pgfqpoint{4.379761in}{1.266415in}}%
\pgfpathlineto{\pgfqpoint{4.394296in}{1.267035in}}%
\pgfpathlineto{\pgfqpoint{4.402534in}{1.280842in}}%
\pgfpathlineto{\pgfqpoint{4.410767in}{1.294694in}}%
\pgfpathlineto{\pgfqpoint{4.418996in}{1.308586in}}%
\pgfpathlineto{\pgfqpoint{4.427221in}{1.322513in}}%
\pgfpathlineto{\pgfqpoint{4.412690in}{1.321533in}}%
\pgfpathlineto{\pgfqpoint{4.398169in}{1.320622in}}%
\pgfpathlineto{\pgfqpoint{4.383658in}{1.319781in}}%
\pgfpathlineto{\pgfqpoint{4.369158in}{1.319009in}}%
\pgfpathlineto{\pgfqpoint{4.360930in}{1.305434in}}%
\pgfpathlineto{\pgfqpoint{4.352697in}{1.291899in}}%
\pgfpathlineto{\pgfqpoint{4.344459in}{1.278408in}}%
\pgfpathlineto{\pgfqpoint{4.336217in}{1.264968in}}%
\pgfpathclose%
\pgfusepath{fill}%
\end{pgfscope}%
\begin{pgfscope}%
\pgfpathrectangle{\pgfqpoint{1.150000in}{0.150000in}}{\pgfqpoint{5.700000in}{5.700000in}}%
\pgfusepath{clip}%
\pgfsetbuttcap%
\pgfsetroundjoin%
\definecolor{currentfill}{rgb}{0.210503,0.363727,0.552206}%
\pgfsetfillcolor{currentfill}%
\pgfsetfillopacity{0.700000}%
\pgfsetlinewidth{0.000000pt}%
\definecolor{currentstroke}{rgb}{0.000000,0.000000,0.000000}%
\pgfsetstrokecolor{currentstroke}%
\pgfsetdash{}{0pt}%
\pgfpathmoveto{\pgfqpoint{4.948339in}{1.781124in}}%
\pgfpathlineto{\pgfqpoint{4.963109in}{1.785522in}}%
\pgfpathlineto{\pgfqpoint{4.977892in}{1.789992in}}%
\pgfpathlineto{\pgfqpoint{4.992689in}{1.794532in}}%
\pgfpathlineto{\pgfqpoint{5.007498in}{1.799144in}}%
\pgfpathlineto{\pgfqpoint{5.015569in}{1.813394in}}%
\pgfpathlineto{\pgfqpoint{5.023634in}{1.827529in}}%
\pgfpathlineto{\pgfqpoint{5.031692in}{1.841546in}}%
\pgfpathlineto{\pgfqpoint{5.039743in}{1.855442in}}%
\pgfpathlineto{\pgfqpoint{5.024934in}{1.850636in}}%
\pgfpathlineto{\pgfqpoint{5.010138in}{1.845901in}}%
\pgfpathlineto{\pgfqpoint{4.995355in}{1.841237in}}%
\pgfpathlineto{\pgfqpoint{4.980586in}{1.836644in}}%
\pgfpathlineto{\pgfqpoint{4.972534in}{1.822933in}}%
\pgfpathlineto{\pgfqpoint{4.964475in}{1.809108in}}%
\pgfpathlineto{\pgfqpoint{4.956410in}{1.795171in}}%
\pgfpathlineto{\pgfqpoint{4.948339in}{1.781124in}}%
\pgfpathclose%
\pgfusepath{fill}%
\end{pgfscope}%
\begin{pgfscope}%
\pgfpathrectangle{\pgfqpoint{1.150000in}{0.150000in}}{\pgfqpoint{5.700000in}{5.700000in}}%
\pgfusepath{clip}%
\pgfsetbuttcap%
\pgfsetroundjoin%
\definecolor{currentfill}{rgb}{0.246811,0.283237,0.535941}%
\pgfsetfillcolor{currentfill}%
\pgfsetfillopacity{0.700000}%
\pgfsetlinewidth{0.000000pt}%
\definecolor{currentstroke}{rgb}{0.000000,0.000000,0.000000}%
\pgfsetstrokecolor{currentstroke}%
\pgfsetdash{}{0pt}%
\pgfpathmoveto{\pgfqpoint{4.733316in}{1.578131in}}%
\pgfpathlineto{\pgfqpoint{4.747987in}{1.581254in}}%
\pgfpathlineto{\pgfqpoint{4.762669in}{1.584447in}}%
\pgfpathlineto{\pgfqpoint{4.777364in}{1.587710in}}%
\pgfpathlineto{\pgfqpoint{4.792071in}{1.591044in}}%
\pgfpathlineto{\pgfqpoint{4.800211in}{1.605903in}}%
\pgfpathlineto{\pgfqpoint{4.808345in}{1.620697in}}%
\pgfpathlineto{\pgfqpoint{4.816475in}{1.635420in}}%
\pgfpathlineto{\pgfqpoint{4.824599in}{1.650071in}}%
\pgfpathlineto{\pgfqpoint{4.809892in}{1.646479in}}%
\pgfpathlineto{\pgfqpoint{4.795197in}{1.642958in}}%
\pgfpathlineto{\pgfqpoint{4.780515in}{1.639507in}}%
\pgfpathlineto{\pgfqpoint{4.765845in}{1.636126in}}%
\pgfpathlineto{\pgfqpoint{4.757721in}{1.621726in}}%
\pgfpathlineto{\pgfqpoint{4.749591in}{1.607257in}}%
\pgfpathlineto{\pgfqpoint{4.741456in}{1.592725in}}%
\pgfpathlineto{\pgfqpoint{4.733316in}{1.578131in}}%
\pgfpathclose%
\pgfusepath{fill}%
\end{pgfscope}%
\begin{pgfscope}%
\pgfpathrectangle{\pgfqpoint{1.150000in}{0.150000in}}{\pgfqpoint{5.700000in}{5.700000in}}%
\pgfusepath{clip}%
\pgfsetbuttcap%
\pgfsetroundjoin%
\definecolor{currentfill}{rgb}{0.279574,0.170599,0.479997}%
\pgfsetfillcolor{currentfill}%
\pgfsetfillopacity{0.700000}%
\pgfsetlinewidth{0.000000pt}%
\definecolor{currentstroke}{rgb}{0.000000,0.000000,0.000000}%
\pgfsetstrokecolor{currentstroke}%
\pgfsetdash{}{0pt}%
\pgfpathmoveto{\pgfqpoint{4.427221in}{1.322513in}}%
\pgfpathlineto{\pgfqpoint{4.441762in}{1.323563in}}%
\pgfpathlineto{\pgfqpoint{4.456314in}{1.324681in}}%
\pgfpathlineto{\pgfqpoint{4.470877in}{1.325869in}}%
\pgfpathlineto{\pgfqpoint{4.485450in}{1.327126in}}%
\pgfpathlineto{\pgfqpoint{4.493667in}{1.341430in}}%
\pgfpathlineto{\pgfqpoint{4.501880in}{1.355753in}}%
\pgfpathlineto{\pgfqpoint{4.510088in}{1.370090in}}%
\pgfpathlineto{\pgfqpoint{4.518292in}{1.384437in}}%
\pgfpathlineto{\pgfqpoint{4.503721in}{1.382840in}}%
\pgfpathlineto{\pgfqpoint{4.489161in}{1.381312in}}%
\pgfpathlineto{\pgfqpoint{4.474612in}{1.379854in}}%
\pgfpathlineto{\pgfqpoint{4.460074in}{1.378466in}}%
\pgfpathlineto{\pgfqpoint{4.451867in}{1.364451in}}%
\pgfpathlineto{\pgfqpoint{4.443656in}{1.350450in}}%
\pgfpathlineto{\pgfqpoint{4.435441in}{1.336469in}}%
\pgfpathlineto{\pgfqpoint{4.427221in}{1.322513in}}%
\pgfpathclose%
\pgfusepath{fill}%
\end{pgfscope}%
\begin{pgfscope}%
\pgfpathrectangle{\pgfqpoint{1.150000in}{0.150000in}}{\pgfqpoint{5.700000in}{5.700000in}}%
\pgfusepath{clip}%
\pgfsetbuttcap%
\pgfsetroundjoin%
\definecolor{currentfill}{rgb}{0.197636,0.391528,0.554969}%
\pgfsetfillcolor{currentfill}%
\pgfsetfillopacity{0.700000}%
\pgfsetlinewidth{0.000000pt}%
\definecolor{currentstroke}{rgb}{0.000000,0.000000,0.000000}%
\pgfsetstrokecolor{currentstroke}%
\pgfsetdash{}{0pt}%
\pgfpathmoveto{\pgfqpoint{5.039743in}{1.855442in}}%
\pgfpathlineto{\pgfqpoint{5.054566in}{1.860320in}}%
\pgfpathlineto{\pgfqpoint{5.069402in}{1.865269in}}%
\pgfpathlineto{\pgfqpoint{5.084252in}{1.870290in}}%
\pgfpathlineto{\pgfqpoint{5.092296in}{1.884200in}}%
\pgfpathlineto{\pgfqpoint{5.100333in}{1.897980in}}%
\pgfpathlineto{\pgfqpoint{5.108363in}{1.911629in}}%
\pgfpathlineto{\pgfqpoint{5.116386in}{1.925144in}}%
\pgfpathlineto{\pgfqpoint{5.101537in}{1.919951in}}%
\pgfpathlineto{\pgfqpoint{5.086701in}{1.914829in}}%
\pgfpathlineto{\pgfqpoint{5.071880in}{1.909778in}}%
\pgfpathlineto{\pgfqpoint{5.063856in}{1.896386in}}%
\pgfpathlineto{\pgfqpoint{5.055825in}{1.882865in}}%
\pgfpathlineto{\pgfqpoint{5.047788in}{1.869216in}}%
\pgfpathlineto{\pgfqpoint{5.039743in}{1.855442in}}%
\pgfpathclose%
\pgfusepath{fill}%
\end{pgfscope}%
\begin{pgfscope}%
\pgfpathrectangle{\pgfqpoint{1.150000in}{0.150000in}}{\pgfqpoint{5.700000in}{5.700000in}}%
\pgfusepath{clip}%
\pgfsetbuttcap%
\pgfsetroundjoin%
\definecolor{currentfill}{rgb}{0.274128,0.199721,0.498911}%
\pgfsetfillcolor{currentfill}%
\pgfsetfillopacity{0.700000}%
\pgfsetlinewidth{0.000000pt}%
\definecolor{currentstroke}{rgb}{0.000000,0.000000,0.000000}%
\pgfsetstrokecolor{currentstroke}%
\pgfsetdash{}{0pt}%
\pgfpathmoveto{\pgfqpoint{4.518292in}{1.384437in}}%
\pgfpathlineto{\pgfqpoint{4.532874in}{1.386103in}}%
\pgfpathlineto{\pgfqpoint{4.547467in}{1.387839in}}%
\pgfpathlineto{\pgfqpoint{4.562070in}{1.389645in}}%
\pgfpathlineto{\pgfqpoint{4.576685in}{1.391520in}}%
\pgfpathlineto{\pgfqpoint{4.584883in}{1.406199in}}%
\pgfpathlineto{\pgfqpoint{4.593076in}{1.420872in}}%
\pgfpathlineto{\pgfqpoint{4.601265in}{1.435536in}}%
\pgfpathlineto{\pgfqpoint{4.609449in}{1.450186in}}%
\pgfpathlineto{\pgfqpoint{4.594836in}{1.447991in}}%
\pgfpathlineto{\pgfqpoint{4.580234in}{1.445865in}}%
\pgfpathlineto{\pgfqpoint{4.565643in}{1.443810in}}%
\pgfpathlineto{\pgfqpoint{4.551063in}{1.441824in}}%
\pgfpathlineto{\pgfqpoint{4.542877in}{1.427486in}}%
\pgfpathlineto{\pgfqpoint{4.534687in}{1.413140in}}%
\pgfpathlineto{\pgfqpoint{4.526492in}{1.398788in}}%
\pgfpathlineto{\pgfqpoint{4.518292in}{1.384437in}}%
\pgfpathclose%
\pgfusepath{fill}%
\end{pgfscope}%
\begin{pgfscope}%
\pgfpathrectangle{\pgfqpoint{1.150000in}{0.150000in}}{\pgfqpoint{5.700000in}{5.700000in}}%
\pgfusepath{clip}%
\pgfsetbuttcap%
\pgfsetroundjoin%
\definecolor{currentfill}{rgb}{0.231674,0.318106,0.544834}%
\pgfsetfillcolor{currentfill}%
\pgfsetfillopacity{0.700000}%
\pgfsetlinewidth{0.000000pt}%
\definecolor{currentstroke}{rgb}{0.000000,0.000000,0.000000}%
\pgfsetstrokecolor{currentstroke}%
\pgfsetdash{}{0pt}%
\pgfpathmoveto{\pgfqpoint{4.824599in}{1.650071in}}%
\pgfpathlineto{\pgfqpoint{4.839318in}{1.653733in}}%
\pgfpathlineto{\pgfqpoint{4.854050in}{1.657466in}}%
\pgfpathlineto{\pgfqpoint{4.868795in}{1.661270in}}%
\pgfpathlineto{\pgfqpoint{4.883552in}{1.665144in}}%
\pgfpathlineto{\pgfqpoint{4.891670in}{1.679963in}}%
\pgfpathlineto{\pgfqpoint{4.899783in}{1.694697in}}%
\pgfpathlineto{\pgfqpoint{4.907891in}{1.709342in}}%
\pgfpathlineto{\pgfqpoint{4.915992in}{1.723894in}}%
\pgfpathlineto{\pgfqpoint{4.901235in}{1.719783in}}%
\pgfpathlineto{\pgfqpoint{4.886491in}{1.715742in}}%
\pgfpathlineto{\pgfqpoint{4.871759in}{1.711772in}}%
\pgfpathlineto{\pgfqpoint{4.857040in}{1.707872in}}%
\pgfpathlineto{\pgfqpoint{4.848938in}{1.693549in}}%
\pgfpathlineto{\pgfqpoint{4.840830in}{1.679138in}}%
\pgfpathlineto{\pgfqpoint{4.832717in}{1.664645in}}%
\pgfpathlineto{\pgfqpoint{4.824599in}{1.650071in}}%
\pgfpathclose%
\pgfusepath{fill}%
\end{pgfscope}%
\begin{pgfscope}%
\pgfpathrectangle{\pgfqpoint{1.150000in}{0.150000in}}{\pgfqpoint{5.700000in}{5.700000in}}%
\pgfusepath{clip}%
\pgfsetbuttcap%
\pgfsetroundjoin%
\definecolor{currentfill}{rgb}{0.265145,0.232956,0.516599}%
\pgfsetfillcolor{currentfill}%
\pgfsetfillopacity{0.700000}%
\pgfsetlinewidth{0.000000pt}%
\definecolor{currentstroke}{rgb}{0.000000,0.000000,0.000000}%
\pgfsetstrokecolor{currentstroke}%
\pgfsetdash{}{0pt}%
\pgfpathmoveto{\pgfqpoint{4.609449in}{1.450186in}}%
\pgfpathlineto{\pgfqpoint{4.624074in}{1.452450in}}%
\pgfpathlineto{\pgfqpoint{4.638711in}{1.454785in}}%
\pgfpathlineto{\pgfqpoint{4.653359in}{1.457189in}}%
\pgfpathlineto{\pgfqpoint{4.668018in}{1.459662in}}%
\pgfpathlineto{\pgfqpoint{4.676197in}{1.474600in}}%
\pgfpathlineto{\pgfqpoint{4.684371in}{1.489510in}}%
\pgfpathlineto{\pgfqpoint{4.692541in}{1.504386in}}%
\pgfpathlineto{\pgfqpoint{4.700706in}{1.519226in}}%
\pgfpathlineto{\pgfqpoint{4.686047in}{1.516452in}}%
\pgfpathlineto{\pgfqpoint{4.671399in}{1.513748in}}%
\pgfpathlineto{\pgfqpoint{4.656764in}{1.511114in}}%
\pgfpathlineto{\pgfqpoint{4.642140in}{1.508551in}}%
\pgfpathlineto{\pgfqpoint{4.633974in}{1.494003in}}%
\pgfpathlineto{\pgfqpoint{4.625804in}{1.479423in}}%
\pgfpathlineto{\pgfqpoint{4.617629in}{1.464816in}}%
\pgfpathlineto{\pgfqpoint{4.609449in}{1.450186in}}%
\pgfpathclose%
\pgfusepath{fill}%
\end{pgfscope}%
\begin{pgfscope}%
\pgfpathrectangle{\pgfqpoint{1.150000in}{0.150000in}}{\pgfqpoint{5.700000in}{5.700000in}}%
\pgfusepath{clip}%
\pgfsetbuttcap%
\pgfsetroundjoin%
\definecolor{currentfill}{rgb}{0.282656,0.100196,0.422160}%
\pgfsetfillcolor{currentfill}%
\pgfsetfillopacity{0.700000}%
\pgfsetlinewidth{0.000000pt}%
\definecolor{currentstroke}{rgb}{0.000000,0.000000,0.000000}%
\pgfsetstrokecolor{currentstroke}%
\pgfsetdash{}{0pt}%
\pgfpathmoveto{\pgfqpoint{4.212147in}{1.162079in}}%
\pgfpathlineto{\pgfqpoint{4.226624in}{1.161437in}}%
\pgfpathlineto{\pgfqpoint{4.241111in}{1.160864in}}%
\pgfpathlineto{\pgfqpoint{4.255607in}{1.160360in}}%
\pgfpathlineto{\pgfqpoint{4.270113in}{1.159924in}}%
\pgfpathlineto{\pgfqpoint{4.278393in}{1.172761in}}%
\pgfpathlineto{\pgfqpoint{4.286667in}{1.185693in}}%
\pgfpathlineto{\pgfqpoint{4.294937in}{1.198715in}}%
\pgfpathlineto{\pgfqpoint{4.303203in}{1.211821in}}%
\pgfpathlineto{\pgfqpoint{4.288703in}{1.211857in}}%
\pgfpathlineto{\pgfqpoint{4.274213in}{1.211961in}}%
\pgfpathlineto{\pgfqpoint{4.259732in}{1.212135in}}%
\pgfpathlineto{\pgfqpoint{4.245261in}{1.212377in}}%
\pgfpathlineto{\pgfqpoint{4.236990in}{1.199663in}}%
\pgfpathlineto{\pgfqpoint{4.228714in}{1.187038in}}%
\pgfpathlineto{\pgfqpoint{4.220433in}{1.174508in}}%
\pgfpathlineto{\pgfqpoint{4.212147in}{1.162079in}}%
\pgfpathclose%
\pgfusepath{fill}%
\end{pgfscope}%
\begin{pgfscope}%
\pgfpathrectangle{\pgfqpoint{1.150000in}{0.150000in}}{\pgfqpoint{5.700000in}{5.700000in}}%
\pgfusepath{clip}%
\pgfsetbuttcap%
\pgfsetroundjoin%
\definecolor{currentfill}{rgb}{0.280267,0.073417,0.397163}%
\pgfsetfillcolor{currentfill}%
\pgfsetfillopacity{0.700000}%
\pgfsetlinewidth{0.000000pt}%
\definecolor{currentstroke}{rgb}{0.000000,0.000000,0.000000}%
\pgfsetstrokecolor{currentstroke}%
\pgfsetdash{}{0pt}%
\pgfpathmoveto{\pgfqpoint{4.121099in}{1.118427in}}%
\pgfpathlineto{\pgfqpoint{4.135549in}{1.117090in}}%
\pgfpathlineto{\pgfqpoint{4.150008in}{1.115822in}}%
\pgfpathlineto{\pgfqpoint{4.164475in}{1.114623in}}%
\pgfpathlineto{\pgfqpoint{4.178952in}{1.113492in}}%
\pgfpathlineto{\pgfqpoint{4.187258in}{1.125457in}}%
\pgfpathlineto{\pgfqpoint{4.195559in}{1.137547in}}%
\pgfpathlineto{\pgfqpoint{4.203855in}{1.149757in}}%
\pgfpathlineto{\pgfqpoint{4.212147in}{1.162079in}}%
\pgfpathlineto{\pgfqpoint{4.197678in}{1.162790in}}%
\pgfpathlineto{\pgfqpoint{4.183219in}{1.163569in}}%
\pgfpathlineto{\pgfqpoint{4.168768in}{1.164418in}}%
\pgfpathlineto{\pgfqpoint{4.154327in}{1.165335in}}%
\pgfpathlineto{\pgfqpoint{4.146028in}{1.153425in}}%
\pgfpathlineto{\pgfqpoint{4.137724in}{1.141632in}}%
\pgfpathlineto{\pgfqpoint{4.129414in}{1.129964in}}%
\pgfpathlineto{\pgfqpoint{4.121099in}{1.118427in}}%
\pgfpathclose%
\pgfusepath{fill}%
\end{pgfscope}%
\begin{pgfscope}%
\pgfpathrectangle{\pgfqpoint{1.150000in}{0.150000in}}{\pgfqpoint{5.700000in}{5.700000in}}%
\pgfusepath{clip}%
\pgfsetbuttcap%
\pgfsetroundjoin%
\definecolor{currentfill}{rgb}{0.218130,0.347432,0.550038}%
\pgfsetfillcolor{currentfill}%
\pgfsetfillopacity{0.700000}%
\pgfsetlinewidth{0.000000pt}%
\definecolor{currentstroke}{rgb}{0.000000,0.000000,0.000000}%
\pgfsetstrokecolor{currentstroke}%
\pgfsetdash{}{0pt}%
\pgfpathmoveto{\pgfqpoint{4.915992in}{1.723894in}}%
\pgfpathlineto{\pgfqpoint{4.930763in}{1.728077in}}%
\pgfpathlineto{\pgfqpoint{4.945546in}{1.732330in}}%
\pgfpathlineto{\pgfqpoint{4.960342in}{1.736654in}}%
\pgfpathlineto{\pgfqpoint{4.975152in}{1.741050in}}%
\pgfpathlineto{\pgfqpoint{4.983248in}{1.755732in}}%
\pgfpathlineto{\pgfqpoint{4.991337in}{1.770310in}}%
\pgfpathlineto{\pgfqpoint{4.999421in}{1.784782in}}%
\pgfpathlineto{\pgfqpoint{5.007498in}{1.799144in}}%
\pgfpathlineto{\pgfqpoint{4.992689in}{1.794532in}}%
\pgfpathlineto{\pgfqpoint{4.977892in}{1.789992in}}%
\pgfpathlineto{\pgfqpoint{4.963109in}{1.785522in}}%
\pgfpathlineto{\pgfqpoint{4.948339in}{1.781124in}}%
\pgfpathlineto{\pgfqpoint{4.940262in}{1.766970in}}%
\pgfpathlineto{\pgfqpoint{4.932178in}{1.752711in}}%
\pgfpathlineto{\pgfqpoint{4.924088in}{1.738352in}}%
\pgfpathlineto{\pgfqpoint{4.915992in}{1.723894in}}%
\pgfpathclose%
\pgfusepath{fill}%
\end{pgfscope}%
\begin{pgfscope}%
\pgfpathrectangle{\pgfqpoint{1.150000in}{0.150000in}}{\pgfqpoint{5.700000in}{5.700000in}}%
\pgfusepath{clip}%
\pgfsetbuttcap%
\pgfsetroundjoin%
\definecolor{currentfill}{rgb}{0.283187,0.125848,0.444960}%
\pgfsetfillcolor{currentfill}%
\pgfsetfillopacity{0.700000}%
\pgfsetlinewidth{0.000000pt}%
\definecolor{currentstroke}{rgb}{0.000000,0.000000,0.000000}%
\pgfsetstrokecolor{currentstroke}%
\pgfsetdash{}{0pt}%
\pgfpathmoveto{\pgfqpoint{4.303203in}{1.211821in}}%
\pgfpathlineto{\pgfqpoint{4.317712in}{1.211854in}}%
\pgfpathlineto{\pgfqpoint{4.332231in}{1.211957in}}%
\pgfpathlineto{\pgfqpoint{4.346760in}{1.212128in}}%
\pgfpathlineto{\pgfqpoint{4.361299in}{1.212368in}}%
\pgfpathlineto{\pgfqpoint{4.369555in}{1.225939in}}%
\pgfpathlineto{\pgfqpoint{4.377806in}{1.239578in}}%
\pgfpathlineto{\pgfqpoint{4.386053in}{1.253279in}}%
\pgfpathlineto{\pgfqpoint{4.394296in}{1.267035in}}%
\pgfpathlineto{\pgfqpoint{4.379761in}{1.266415in}}%
\pgfpathlineto{\pgfqpoint{4.365236in}{1.265864in}}%
\pgfpathlineto{\pgfqpoint{4.350722in}{1.265382in}}%
\pgfpathlineto{\pgfqpoint{4.336217in}{1.264968in}}%
\pgfpathlineto{\pgfqpoint{4.327970in}{1.251584in}}%
\pgfpathlineto{\pgfqpoint{4.319719in}{1.238261in}}%
\pgfpathlineto{\pgfqpoint{4.311463in}{1.225005in}}%
\pgfpathlineto{\pgfqpoint{4.303203in}{1.211821in}}%
\pgfpathclose%
\pgfusepath{fill}%
\end{pgfscope}%
\begin{pgfscope}%
\pgfpathrectangle{\pgfqpoint{1.150000in}{0.150000in}}{\pgfqpoint{5.700000in}{5.700000in}}%
\pgfusepath{clip}%
\pgfsetbuttcap%
\pgfsetroundjoin%
\definecolor{currentfill}{rgb}{0.253935,0.265254,0.529983}%
\pgfsetfillcolor{currentfill}%
\pgfsetfillopacity{0.700000}%
\pgfsetlinewidth{0.000000pt}%
\definecolor{currentstroke}{rgb}{0.000000,0.000000,0.000000}%
\pgfsetstrokecolor{currentstroke}%
\pgfsetdash{}{0pt}%
\pgfpathmoveto{\pgfqpoint{4.700706in}{1.519226in}}%
\pgfpathlineto{\pgfqpoint{4.715377in}{1.522070in}}%
\pgfpathlineto{\pgfqpoint{4.730059in}{1.524983in}}%
\pgfpathlineto{\pgfqpoint{4.744754in}{1.527967in}}%
\pgfpathlineto{\pgfqpoint{4.759461in}{1.531021in}}%
\pgfpathlineto{\pgfqpoint{4.767621in}{1.546107in}}%
\pgfpathlineto{\pgfqpoint{4.775776in}{1.561141in}}%
\pgfpathlineto{\pgfqpoint{4.783926in}{1.576122in}}%
\pgfpathlineto{\pgfqpoint{4.792071in}{1.591044in}}%
\pgfpathlineto{\pgfqpoint{4.777364in}{1.587710in}}%
\pgfpathlineto{\pgfqpoint{4.762669in}{1.584447in}}%
\pgfpathlineto{\pgfqpoint{4.747987in}{1.581254in}}%
\pgfpathlineto{\pgfqpoint{4.733316in}{1.578131in}}%
\pgfpathlineto{\pgfqpoint{4.725171in}{1.563481in}}%
\pgfpathlineto{\pgfqpoint{4.717021in}{1.548777in}}%
\pgfpathlineto{\pgfqpoint{4.708866in}{1.534024in}}%
\pgfpathlineto{\pgfqpoint{4.700706in}{1.519226in}}%
\pgfpathclose%
\pgfusepath{fill}%
\end{pgfscope}%
\begin{pgfscope}%
\pgfpathrectangle{\pgfqpoint{1.150000in}{0.150000in}}{\pgfqpoint{5.700000in}{5.700000in}}%
\pgfusepath{clip}%
\pgfsetbuttcap%
\pgfsetroundjoin%
\definecolor{currentfill}{rgb}{0.281887,0.150881,0.465405}%
\pgfsetfillcolor{currentfill}%
\pgfsetfillopacity{0.700000}%
\pgfsetlinewidth{0.000000pt}%
\definecolor{currentstroke}{rgb}{0.000000,0.000000,0.000000}%
\pgfsetstrokecolor{currentstroke}%
\pgfsetdash{}{0pt}%
\pgfpathmoveto{\pgfqpoint{4.394296in}{1.267035in}}%
\pgfpathlineto{\pgfqpoint{4.408840in}{1.267725in}}%
\pgfpathlineto{\pgfqpoint{4.423396in}{1.268483in}}%
\pgfpathlineto{\pgfqpoint{4.437961in}{1.269310in}}%
\pgfpathlineto{\pgfqpoint{4.452537in}{1.270206in}}%
\pgfpathlineto{\pgfqpoint{4.460772in}{1.284382in}}%
\pgfpathlineto{\pgfqpoint{4.469002in}{1.298597in}}%
\pgfpathlineto{\pgfqpoint{4.477228in}{1.312847in}}%
\pgfpathlineto{\pgfqpoint{4.485450in}{1.327126in}}%
\pgfpathlineto{\pgfqpoint{4.470877in}{1.325869in}}%
\pgfpathlineto{\pgfqpoint{4.456314in}{1.324681in}}%
\pgfpathlineto{\pgfqpoint{4.441762in}{1.323563in}}%
\pgfpathlineto{\pgfqpoint{4.427221in}{1.322513in}}%
\pgfpathlineto{\pgfqpoint{4.418996in}{1.308586in}}%
\pgfpathlineto{\pgfqpoint{4.410767in}{1.294694in}}%
\pgfpathlineto{\pgfqpoint{4.402534in}{1.280842in}}%
\pgfpathlineto{\pgfqpoint{4.394296in}{1.267035in}}%
\pgfpathclose%
\pgfusepath{fill}%
\end{pgfscope}%
\begin{pgfscope}%
\pgfpathrectangle{\pgfqpoint{1.150000in}{0.150000in}}{\pgfqpoint{5.700000in}{5.700000in}}%
\pgfusepath{clip}%
\pgfsetbuttcap%
\pgfsetroundjoin%
\definecolor{currentfill}{rgb}{0.203063,0.379716,0.553925}%
\pgfsetfillcolor{currentfill}%
\pgfsetfillopacity{0.700000}%
\pgfsetlinewidth{0.000000pt}%
\definecolor{currentstroke}{rgb}{0.000000,0.000000,0.000000}%
\pgfsetstrokecolor{currentstroke}%
\pgfsetdash{}{0pt}%
\pgfpathmoveto{\pgfqpoint{5.007498in}{1.799144in}}%
\pgfpathlineto{\pgfqpoint{5.022321in}{1.803827in}}%
\pgfpathlineto{\pgfqpoint{5.037158in}{1.808582in}}%
\pgfpathlineto{\pgfqpoint{5.052008in}{1.813407in}}%
\pgfpathlineto{\pgfqpoint{5.060079in}{1.827809in}}%
\pgfpathlineto{\pgfqpoint{5.068143in}{1.842092in}}%
\pgfpathlineto{\pgfqpoint{5.076201in}{1.856253in}}%
\pgfpathlineto{\pgfqpoint{5.084252in}{1.870290in}}%
\pgfpathlineto{\pgfqpoint{5.069402in}{1.865269in}}%
\pgfpathlineto{\pgfqpoint{5.054566in}{1.860320in}}%
\pgfpathlineto{\pgfqpoint{5.039743in}{1.855442in}}%
\pgfpathlineto{\pgfqpoint{5.031692in}{1.841546in}}%
\pgfpathlineto{\pgfqpoint{5.023634in}{1.827529in}}%
\pgfpathlineto{\pgfqpoint{5.015569in}{1.813394in}}%
\pgfpathlineto{\pgfqpoint{5.007498in}{1.799144in}}%
\pgfpathclose%
\pgfusepath{fill}%
\end{pgfscope}%
\begin{pgfscope}%
\pgfpathrectangle{\pgfqpoint{1.150000in}{0.150000in}}{\pgfqpoint{5.700000in}{5.700000in}}%
\pgfusepath{clip}%
\pgfsetbuttcap%
\pgfsetroundjoin%
\definecolor{currentfill}{rgb}{0.277134,0.185228,0.489898}%
\pgfsetfillcolor{currentfill}%
\pgfsetfillopacity{0.700000}%
\pgfsetlinewidth{0.000000pt}%
\definecolor{currentstroke}{rgb}{0.000000,0.000000,0.000000}%
\pgfsetstrokecolor{currentstroke}%
\pgfsetdash{}{0pt}%
\pgfpathmoveto{\pgfqpoint{4.485450in}{1.327126in}}%
\pgfpathlineto{\pgfqpoint{4.500034in}{1.328452in}}%
\pgfpathlineto{\pgfqpoint{4.514628in}{1.329848in}}%
\pgfpathlineto{\pgfqpoint{4.529234in}{1.331312in}}%
\pgfpathlineto{\pgfqpoint{4.543850in}{1.332846in}}%
\pgfpathlineto{\pgfqpoint{4.552065in}{1.347498in}}%
\pgfpathlineto{\pgfqpoint{4.560276in}{1.362165in}}%
\pgfpathlineto{\pgfqpoint{4.568483in}{1.376840in}}%
\pgfpathlineto{\pgfqpoint{4.576685in}{1.391520in}}%
\pgfpathlineto{\pgfqpoint{4.562070in}{1.389645in}}%
\pgfpathlineto{\pgfqpoint{4.547467in}{1.387839in}}%
\pgfpathlineto{\pgfqpoint{4.532874in}{1.386103in}}%
\pgfpathlineto{\pgfqpoint{4.518292in}{1.384437in}}%
\pgfpathlineto{\pgfqpoint{4.510088in}{1.370090in}}%
\pgfpathlineto{\pgfqpoint{4.501880in}{1.355753in}}%
\pgfpathlineto{\pgfqpoint{4.493667in}{1.341430in}}%
\pgfpathlineto{\pgfqpoint{4.485450in}{1.327126in}}%
\pgfpathclose%
\pgfusepath{fill}%
\end{pgfscope}%
\begin{pgfscope}%
\pgfpathrectangle{\pgfqpoint{1.150000in}{0.150000in}}{\pgfqpoint{5.700000in}{5.700000in}}%
\pgfusepath{clip}%
\pgfsetbuttcap%
\pgfsetroundjoin%
\definecolor{currentfill}{rgb}{0.239346,0.300855,0.540844}%
\pgfsetfillcolor{currentfill}%
\pgfsetfillopacity{0.700000}%
\pgfsetlinewidth{0.000000pt}%
\definecolor{currentstroke}{rgb}{0.000000,0.000000,0.000000}%
\pgfsetstrokecolor{currentstroke}%
\pgfsetdash{}{0pt}%
\pgfpathmoveto{\pgfqpoint{4.792071in}{1.591044in}}%
\pgfpathlineto{\pgfqpoint{4.806790in}{1.594447in}}%
\pgfpathlineto{\pgfqpoint{4.821522in}{1.597922in}}%
\pgfpathlineto{\pgfqpoint{4.836266in}{1.601466in}}%
\pgfpathlineto{\pgfqpoint{4.851023in}{1.605081in}}%
\pgfpathlineto{\pgfqpoint{4.859163in}{1.620207in}}%
\pgfpathlineto{\pgfqpoint{4.867298in}{1.635262in}}%
\pgfpathlineto{\pgfqpoint{4.875428in}{1.650242in}}%
\pgfpathlineto{\pgfqpoint{4.883552in}{1.665144in}}%
\pgfpathlineto{\pgfqpoint{4.868795in}{1.661270in}}%
\pgfpathlineto{\pgfqpoint{4.854050in}{1.657466in}}%
\pgfpathlineto{\pgfqpoint{4.839318in}{1.653733in}}%
\pgfpathlineto{\pgfqpoint{4.824599in}{1.650071in}}%
\pgfpathlineto{\pgfqpoint{4.816475in}{1.635420in}}%
\pgfpathlineto{\pgfqpoint{4.808345in}{1.620697in}}%
\pgfpathlineto{\pgfqpoint{4.800211in}{1.605903in}}%
\pgfpathlineto{\pgfqpoint{4.792071in}{1.591044in}}%
\pgfpathclose%
\pgfusepath{fill}%
\end{pgfscope}%
\begin{pgfscope}%
\pgfpathrectangle{\pgfqpoint{1.150000in}{0.150000in}}{\pgfqpoint{5.700000in}{5.700000in}}%
\pgfusepath{clip}%
\pgfsetbuttcap%
\pgfsetroundjoin%
\definecolor{currentfill}{rgb}{0.270595,0.214069,0.507052}%
\pgfsetfillcolor{currentfill}%
\pgfsetfillopacity{0.700000}%
\pgfsetlinewidth{0.000000pt}%
\definecolor{currentstroke}{rgb}{0.000000,0.000000,0.000000}%
\pgfsetstrokecolor{currentstroke}%
\pgfsetdash{}{0pt}%
\pgfpathmoveto{\pgfqpoint{4.576685in}{1.391520in}}%
\pgfpathlineto{\pgfqpoint{4.591311in}{1.393464in}}%
\pgfpathlineto{\pgfqpoint{4.605948in}{1.395478in}}%
\pgfpathlineto{\pgfqpoint{4.620597in}{1.397561in}}%
\pgfpathlineto{\pgfqpoint{4.635257in}{1.399713in}}%
\pgfpathlineto{\pgfqpoint{4.643454in}{1.414721in}}%
\pgfpathlineto{\pgfqpoint{4.651646in}{1.429718in}}%
\pgfpathlineto{\pgfqpoint{4.659835in}{1.444700in}}%
\pgfpathlineto{\pgfqpoint{4.668018in}{1.459662in}}%
\pgfpathlineto{\pgfqpoint{4.653359in}{1.457189in}}%
\pgfpathlineto{\pgfqpoint{4.638711in}{1.454785in}}%
\pgfpathlineto{\pgfqpoint{4.624074in}{1.452450in}}%
\pgfpathlineto{\pgfqpoint{4.609449in}{1.450186in}}%
\pgfpathlineto{\pgfqpoint{4.601265in}{1.435536in}}%
\pgfpathlineto{\pgfqpoint{4.593076in}{1.420872in}}%
\pgfpathlineto{\pgfqpoint{4.584883in}{1.406199in}}%
\pgfpathlineto{\pgfqpoint{4.576685in}{1.391520in}}%
\pgfpathclose%
\pgfusepath{fill}%
\end{pgfscope}%
\begin{pgfscope}%
\pgfpathrectangle{\pgfqpoint{1.150000in}{0.150000in}}{\pgfqpoint{5.700000in}{5.700000in}}%
\pgfusepath{clip}%
\pgfsetbuttcap%
\pgfsetroundjoin%
\definecolor{currentfill}{rgb}{0.223925,0.334994,0.548053}%
\pgfsetfillcolor{currentfill}%
\pgfsetfillopacity{0.700000}%
\pgfsetlinewidth{0.000000pt}%
\definecolor{currentstroke}{rgb}{0.000000,0.000000,0.000000}%
\pgfsetstrokecolor{currentstroke}%
\pgfsetdash{}{0pt}%
\pgfpathmoveto{\pgfqpoint{4.883552in}{1.665144in}}%
\pgfpathlineto{\pgfqpoint{4.898322in}{1.669088in}}%
\pgfpathlineto{\pgfqpoint{4.913104in}{1.673104in}}%
\pgfpathlineto{\pgfqpoint{4.927900in}{1.677190in}}%
\pgfpathlineto{\pgfqpoint{4.942709in}{1.681346in}}%
\pgfpathlineto{\pgfqpoint{4.950828in}{1.696412in}}%
\pgfpathlineto{\pgfqpoint{4.958942in}{1.711386in}}%
\pgfpathlineto{\pgfqpoint{4.967050in}{1.726267in}}%
\pgfpathlineto{\pgfqpoint{4.975152in}{1.741050in}}%
\pgfpathlineto{\pgfqpoint{4.960342in}{1.736654in}}%
\pgfpathlineto{\pgfqpoint{4.945546in}{1.732330in}}%
\pgfpathlineto{\pgfqpoint{4.930763in}{1.728077in}}%
\pgfpathlineto{\pgfqpoint{4.915992in}{1.723894in}}%
\pgfpathlineto{\pgfqpoint{4.907891in}{1.709342in}}%
\pgfpathlineto{\pgfqpoint{4.899783in}{1.694697in}}%
\pgfpathlineto{\pgfqpoint{4.891670in}{1.679963in}}%
\pgfpathlineto{\pgfqpoint{4.883552in}{1.665144in}}%
\pgfpathclose%
\pgfusepath{fill}%
\end{pgfscope}%
\begin{pgfscope}%
\pgfpathrectangle{\pgfqpoint{1.150000in}{0.150000in}}{\pgfqpoint{5.700000in}{5.700000in}}%
\pgfusepath{clip}%
\pgfsetbuttcap%
\pgfsetroundjoin%
\definecolor{currentfill}{rgb}{0.258965,0.251537,0.524736}%
\pgfsetfillcolor{currentfill}%
\pgfsetfillopacity{0.700000}%
\pgfsetlinewidth{0.000000pt}%
\definecolor{currentstroke}{rgb}{0.000000,0.000000,0.000000}%
\pgfsetstrokecolor{currentstroke}%
\pgfsetdash{}{0pt}%
\pgfpathmoveto{\pgfqpoint{4.668018in}{1.459662in}}%
\pgfpathlineto{\pgfqpoint{4.682689in}{1.462206in}}%
\pgfpathlineto{\pgfqpoint{4.697372in}{1.464819in}}%
\pgfpathlineto{\pgfqpoint{4.712067in}{1.467502in}}%
\pgfpathlineto{\pgfqpoint{4.726773in}{1.470255in}}%
\pgfpathlineto{\pgfqpoint{4.734952in}{1.485501in}}%
\pgfpathlineto{\pgfqpoint{4.743126in}{1.500714in}}%
\pgfpathlineto{\pgfqpoint{4.751296in}{1.515889in}}%
\pgfpathlineto{\pgfqpoint{4.759461in}{1.531021in}}%
\pgfpathlineto{\pgfqpoint{4.744754in}{1.527967in}}%
\pgfpathlineto{\pgfqpoint{4.730059in}{1.524983in}}%
\pgfpathlineto{\pgfqpoint{4.715377in}{1.522070in}}%
\pgfpathlineto{\pgfqpoint{4.700706in}{1.519226in}}%
\pgfpathlineto{\pgfqpoint{4.692541in}{1.504386in}}%
\pgfpathlineto{\pgfqpoint{4.684371in}{1.489510in}}%
\pgfpathlineto{\pgfqpoint{4.676197in}{1.474600in}}%
\pgfpathlineto{\pgfqpoint{4.668018in}{1.459662in}}%
\pgfpathclose%
\pgfusepath{fill}%
\end{pgfscope}%
\begin{pgfscope}%
\pgfpathrectangle{\pgfqpoint{1.150000in}{0.150000in}}{\pgfqpoint{5.700000in}{5.700000in}}%
\pgfusepath{clip}%
\pgfsetbuttcap%
\pgfsetroundjoin%
\definecolor{currentfill}{rgb}{0.281446,0.084320,0.407414}%
\pgfsetfillcolor{currentfill}%
\pgfsetfillopacity{0.700000}%
\pgfsetlinewidth{0.000000pt}%
\definecolor{currentstroke}{rgb}{0.000000,0.000000,0.000000}%
\pgfsetstrokecolor{currentstroke}%
\pgfsetdash{}{0pt}%
\pgfpathmoveto{\pgfqpoint{4.178952in}{1.113492in}}%
\pgfpathlineto{\pgfqpoint{4.193437in}{1.112431in}}%
\pgfpathlineto{\pgfqpoint{4.207931in}{1.111437in}}%
\pgfpathlineto{\pgfqpoint{4.222435in}{1.110512in}}%
\pgfpathlineto{\pgfqpoint{4.236947in}{1.109656in}}%
\pgfpathlineto{\pgfqpoint{4.245246in}{1.122049in}}%
\pgfpathlineto{\pgfqpoint{4.253540in}{1.134562in}}%
\pgfpathlineto{\pgfqpoint{4.261829in}{1.147189in}}%
\pgfpathlineto{\pgfqpoint{4.270113in}{1.159924in}}%
\pgfpathlineto{\pgfqpoint{4.255607in}{1.160360in}}%
\pgfpathlineto{\pgfqpoint{4.241111in}{1.160864in}}%
\pgfpathlineto{\pgfqpoint{4.226624in}{1.161437in}}%
\pgfpathlineto{\pgfqpoint{4.212147in}{1.162079in}}%
\pgfpathlineto{\pgfqpoint{4.203855in}{1.149757in}}%
\pgfpathlineto{\pgfqpoint{4.195559in}{1.137547in}}%
\pgfpathlineto{\pgfqpoint{4.187258in}{1.125457in}}%
\pgfpathlineto{\pgfqpoint{4.178952in}{1.113492in}}%
\pgfpathclose%
\pgfusepath{fill}%
\end{pgfscope}%
\begin{pgfscope}%
\pgfpathrectangle{\pgfqpoint{1.150000in}{0.150000in}}{\pgfqpoint{5.700000in}{5.700000in}}%
\pgfusepath{clip}%
\pgfsetbuttcap%
\pgfsetroundjoin%
\definecolor{currentfill}{rgb}{0.283091,0.110553,0.431554}%
\pgfsetfillcolor{currentfill}%
\pgfsetfillopacity{0.700000}%
\pgfsetlinewidth{0.000000pt}%
\definecolor{currentstroke}{rgb}{0.000000,0.000000,0.000000}%
\pgfsetstrokecolor{currentstroke}%
\pgfsetdash{}{0pt}%
\pgfpathmoveto{\pgfqpoint{4.270113in}{1.159924in}}%
\pgfpathlineto{\pgfqpoint{4.284628in}{1.159557in}}%
\pgfpathlineto{\pgfqpoint{4.299153in}{1.159259in}}%
\pgfpathlineto{\pgfqpoint{4.313687in}{1.159029in}}%
\pgfpathlineto{\pgfqpoint{4.328231in}{1.158868in}}%
\pgfpathlineto{\pgfqpoint{4.336505in}{1.172113in}}%
\pgfpathlineto{\pgfqpoint{4.344774in}{1.185448in}}%
\pgfpathlineto{\pgfqpoint{4.353039in}{1.198869in}}%
\pgfpathlineto{\pgfqpoint{4.361299in}{1.212368in}}%
\pgfpathlineto{\pgfqpoint{4.346760in}{1.212128in}}%
\pgfpathlineto{\pgfqpoint{4.332231in}{1.211957in}}%
\pgfpathlineto{\pgfqpoint{4.317712in}{1.211854in}}%
\pgfpathlineto{\pgfqpoint{4.303203in}{1.211821in}}%
\pgfpathlineto{\pgfqpoint{4.294937in}{1.198715in}}%
\pgfpathlineto{\pgfqpoint{4.286667in}{1.185693in}}%
\pgfpathlineto{\pgfqpoint{4.278393in}{1.172761in}}%
\pgfpathlineto{\pgfqpoint{4.270113in}{1.159924in}}%
\pgfpathclose%
\pgfusepath{fill}%
\end{pgfscope}%
\begin{pgfscope}%
\pgfpathrectangle{\pgfqpoint{1.150000in}{0.150000in}}{\pgfqpoint{5.700000in}{5.700000in}}%
\pgfusepath{clip}%
\pgfsetbuttcap%
\pgfsetroundjoin%
\definecolor{currentfill}{rgb}{0.282884,0.135920,0.453427}%
\pgfsetfillcolor{currentfill}%
\pgfsetfillopacity{0.700000}%
\pgfsetlinewidth{0.000000pt}%
\definecolor{currentstroke}{rgb}{0.000000,0.000000,0.000000}%
\pgfsetstrokecolor{currentstroke}%
\pgfsetdash{}{0pt}%
\pgfpathmoveto{\pgfqpoint{4.361299in}{1.212368in}}%
\pgfpathlineto{\pgfqpoint{4.375848in}{1.212676in}}%
\pgfpathlineto{\pgfqpoint{4.390407in}{1.213053in}}%
\pgfpathlineto{\pgfqpoint{4.404976in}{1.213500in}}%
\pgfpathlineto{\pgfqpoint{4.419555in}{1.214014in}}%
\pgfpathlineto{\pgfqpoint{4.427807in}{1.227975in}}%
\pgfpathlineto{\pgfqpoint{4.436055in}{1.241998in}}%
\pgfpathlineto{\pgfqpoint{4.444298in}{1.256077in}}%
\pgfpathlineto{\pgfqpoint{4.452537in}{1.270206in}}%
\pgfpathlineto{\pgfqpoint{4.437961in}{1.269310in}}%
\pgfpathlineto{\pgfqpoint{4.423396in}{1.268483in}}%
\pgfpathlineto{\pgfqpoint{4.408840in}{1.267725in}}%
\pgfpathlineto{\pgfqpoint{4.394296in}{1.267035in}}%
\pgfpathlineto{\pgfqpoint{4.386053in}{1.253279in}}%
\pgfpathlineto{\pgfqpoint{4.377806in}{1.239578in}}%
\pgfpathlineto{\pgfqpoint{4.369555in}{1.225939in}}%
\pgfpathlineto{\pgfqpoint{4.361299in}{1.212368in}}%
\pgfpathclose%
\pgfusepath{fill}%
\end{pgfscope}%
\begin{pgfscope}%
\pgfpathrectangle{\pgfqpoint{1.150000in}{0.150000in}}{\pgfqpoint{5.700000in}{5.700000in}}%
\pgfusepath{clip}%
\pgfsetbuttcap%
\pgfsetroundjoin%
\definecolor{currentfill}{rgb}{0.210503,0.363727,0.552206}%
\pgfsetfillcolor{currentfill}%
\pgfsetfillopacity{0.700000}%
\pgfsetlinewidth{0.000000pt}%
\definecolor{currentstroke}{rgb}{0.000000,0.000000,0.000000}%
\pgfsetstrokecolor{currentstroke}%
\pgfsetdash{}{0pt}%
\pgfpathmoveto{\pgfqpoint{4.975152in}{1.741050in}}%
\pgfpathlineto{\pgfqpoint{4.989974in}{1.745516in}}%
\pgfpathlineto{\pgfqpoint{5.004810in}{1.750053in}}%
\pgfpathlineto{\pgfqpoint{5.019659in}{1.754661in}}%
\pgfpathlineto{\pgfqpoint{5.027756in}{1.769512in}}%
\pgfpathlineto{\pgfqpoint{5.035846in}{1.784256in}}%
\pgfpathlineto{\pgfqpoint{5.043930in}{1.798888in}}%
\pgfpathlineto{\pgfqpoint{5.052008in}{1.813407in}}%
\pgfpathlineto{\pgfqpoint{5.037158in}{1.808582in}}%
\pgfpathlineto{\pgfqpoint{5.022321in}{1.803827in}}%
\pgfpathlineto{\pgfqpoint{5.007498in}{1.799144in}}%
\pgfpathlineto{\pgfqpoint{4.999421in}{1.784782in}}%
\pgfpathlineto{\pgfqpoint{4.991337in}{1.770310in}}%
\pgfpathlineto{\pgfqpoint{4.983248in}{1.755732in}}%
\pgfpathlineto{\pgfqpoint{4.975152in}{1.741050in}}%
\pgfpathclose%
\pgfusepath{fill}%
\end{pgfscope}%
\begin{pgfscope}%
\pgfpathrectangle{\pgfqpoint{1.150000in}{0.150000in}}{\pgfqpoint{5.700000in}{5.700000in}}%
\pgfusepath{clip}%
\pgfsetbuttcap%
\pgfsetroundjoin%
\definecolor{currentfill}{rgb}{0.280255,0.165693,0.476498}%
\pgfsetfillcolor{currentfill}%
\pgfsetfillopacity{0.700000}%
\pgfsetlinewidth{0.000000pt}%
\definecolor{currentstroke}{rgb}{0.000000,0.000000,0.000000}%
\pgfsetstrokecolor{currentstroke}%
\pgfsetdash{}{0pt}%
\pgfpathmoveto{\pgfqpoint{4.452537in}{1.270206in}}%
\pgfpathlineto{\pgfqpoint{4.467123in}{1.271172in}}%
\pgfpathlineto{\pgfqpoint{4.481720in}{1.272206in}}%
\pgfpathlineto{\pgfqpoint{4.496328in}{1.273309in}}%
\pgfpathlineto{\pgfqpoint{4.510946in}{1.274481in}}%
\pgfpathlineto{\pgfqpoint{4.519178in}{1.289025in}}%
\pgfpathlineto{\pgfqpoint{4.527406in}{1.303605in}}%
\pgfpathlineto{\pgfqpoint{4.535630in}{1.318213in}}%
\pgfpathlineto{\pgfqpoint{4.543850in}{1.332846in}}%
\pgfpathlineto{\pgfqpoint{4.529234in}{1.331312in}}%
\pgfpathlineto{\pgfqpoint{4.514628in}{1.329848in}}%
\pgfpathlineto{\pgfqpoint{4.500034in}{1.328452in}}%
\pgfpathlineto{\pgfqpoint{4.485450in}{1.327126in}}%
\pgfpathlineto{\pgfqpoint{4.477228in}{1.312847in}}%
\pgfpathlineto{\pgfqpoint{4.469002in}{1.298597in}}%
\pgfpathlineto{\pgfqpoint{4.460772in}{1.284382in}}%
\pgfpathlineto{\pgfqpoint{4.452537in}{1.270206in}}%
\pgfpathclose%
\pgfusepath{fill}%
\end{pgfscope}%
\begin{pgfscope}%
\pgfpathrectangle{\pgfqpoint{1.150000in}{0.150000in}}{\pgfqpoint{5.700000in}{5.700000in}}%
\pgfusepath{clip}%
\pgfsetbuttcap%
\pgfsetroundjoin%
\definecolor{currentfill}{rgb}{0.246811,0.283237,0.535941}%
\pgfsetfillcolor{currentfill}%
\pgfsetfillopacity{0.700000}%
\pgfsetlinewidth{0.000000pt}%
\definecolor{currentstroke}{rgb}{0.000000,0.000000,0.000000}%
\pgfsetstrokecolor{currentstroke}%
\pgfsetdash{}{0pt}%
\pgfpathmoveto{\pgfqpoint{4.759461in}{1.531021in}}%
\pgfpathlineto{\pgfqpoint{4.774180in}{1.534145in}}%
\pgfpathlineto{\pgfqpoint{4.788911in}{1.537338in}}%
\pgfpathlineto{\pgfqpoint{4.803654in}{1.540602in}}%
\pgfpathlineto{\pgfqpoint{4.818410in}{1.543936in}}%
\pgfpathlineto{\pgfqpoint{4.826571in}{1.559310in}}%
\pgfpathlineto{\pgfqpoint{4.834726in}{1.574628in}}%
\pgfpathlineto{\pgfqpoint{4.842877in}{1.589886in}}%
\pgfpathlineto{\pgfqpoint{4.851023in}{1.605081in}}%
\pgfpathlineto{\pgfqpoint{4.836266in}{1.601466in}}%
\pgfpathlineto{\pgfqpoint{4.821522in}{1.597922in}}%
\pgfpathlineto{\pgfqpoint{4.806790in}{1.594447in}}%
\pgfpathlineto{\pgfqpoint{4.792071in}{1.591044in}}%
\pgfpathlineto{\pgfqpoint{4.783926in}{1.576122in}}%
\pgfpathlineto{\pgfqpoint{4.775776in}{1.561141in}}%
\pgfpathlineto{\pgfqpoint{4.767621in}{1.546107in}}%
\pgfpathlineto{\pgfqpoint{4.759461in}{1.531021in}}%
\pgfpathclose%
\pgfusepath{fill}%
\end{pgfscope}%
\begin{pgfscope}%
\pgfpathrectangle{\pgfqpoint{1.150000in}{0.150000in}}{\pgfqpoint{5.700000in}{5.700000in}}%
\pgfusepath{clip}%
\pgfsetbuttcap%
\pgfsetroundjoin%
\definecolor{currentfill}{rgb}{0.274128,0.199721,0.498911}%
\pgfsetfillcolor{currentfill}%
\pgfsetfillopacity{0.700000}%
\pgfsetlinewidth{0.000000pt}%
\definecolor{currentstroke}{rgb}{0.000000,0.000000,0.000000}%
\pgfsetstrokecolor{currentstroke}%
\pgfsetdash{}{0pt}%
\pgfpathmoveto{\pgfqpoint{4.543850in}{1.332846in}}%
\pgfpathlineto{\pgfqpoint{4.558477in}{1.334449in}}%
\pgfpathlineto{\pgfqpoint{4.573116in}{1.336121in}}%
\pgfpathlineto{\pgfqpoint{4.587765in}{1.337863in}}%
\pgfpathlineto{\pgfqpoint{4.602425in}{1.339673in}}%
\pgfpathlineto{\pgfqpoint{4.610640in}{1.354675in}}%
\pgfpathlineto{\pgfqpoint{4.618850in}{1.369685in}}%
\pgfpathlineto{\pgfqpoint{4.627056in}{1.384700in}}%
\pgfpathlineto{\pgfqpoint{4.635257in}{1.399713in}}%
\pgfpathlineto{\pgfqpoint{4.620597in}{1.397561in}}%
\pgfpathlineto{\pgfqpoint{4.605948in}{1.395478in}}%
\pgfpathlineto{\pgfqpoint{4.591311in}{1.393464in}}%
\pgfpathlineto{\pgfqpoint{4.576685in}{1.391520in}}%
\pgfpathlineto{\pgfqpoint{4.568483in}{1.376840in}}%
\pgfpathlineto{\pgfqpoint{4.560276in}{1.362165in}}%
\pgfpathlineto{\pgfqpoint{4.552065in}{1.347498in}}%
\pgfpathlineto{\pgfqpoint{4.543850in}{1.332846in}}%
\pgfpathclose%
\pgfusepath{fill}%
\end{pgfscope}%
\begin{pgfscope}%
\pgfpathrectangle{\pgfqpoint{1.150000in}{0.150000in}}{\pgfqpoint{5.700000in}{5.700000in}}%
\pgfusepath{clip}%
\pgfsetbuttcap%
\pgfsetroundjoin%
\definecolor{currentfill}{rgb}{0.231674,0.318106,0.544834}%
\pgfsetfillcolor{currentfill}%
\pgfsetfillopacity{0.700000}%
\pgfsetlinewidth{0.000000pt}%
\definecolor{currentstroke}{rgb}{0.000000,0.000000,0.000000}%
\pgfsetstrokecolor{currentstroke}%
\pgfsetdash{}{0pt}%
\pgfpathmoveto{\pgfqpoint{4.851023in}{1.605081in}}%
\pgfpathlineto{\pgfqpoint{4.865792in}{1.608766in}}%
\pgfpathlineto{\pgfqpoint{4.880573in}{1.612521in}}%
\pgfpathlineto{\pgfqpoint{4.895368in}{1.616347in}}%
\pgfpathlineto{\pgfqpoint{4.910175in}{1.620244in}}%
\pgfpathlineto{\pgfqpoint{4.918317in}{1.635638in}}%
\pgfpathlineto{\pgfqpoint{4.926453in}{1.650956in}}%
\pgfpathlineto{\pgfqpoint{4.934584in}{1.666193in}}%
\pgfpathlineto{\pgfqpoint{4.942709in}{1.681346in}}%
\pgfpathlineto{\pgfqpoint{4.927900in}{1.677190in}}%
\pgfpathlineto{\pgfqpoint{4.913104in}{1.673104in}}%
\pgfpathlineto{\pgfqpoint{4.898322in}{1.669088in}}%
\pgfpathlineto{\pgfqpoint{4.883552in}{1.665144in}}%
\pgfpathlineto{\pgfqpoint{4.875428in}{1.650242in}}%
\pgfpathlineto{\pgfqpoint{4.867298in}{1.635262in}}%
\pgfpathlineto{\pgfqpoint{4.859163in}{1.620207in}}%
\pgfpathlineto{\pgfqpoint{4.851023in}{1.605081in}}%
\pgfpathclose%
\pgfusepath{fill}%
\end{pgfscope}%
\begin{pgfscope}%
\pgfpathrectangle{\pgfqpoint{1.150000in}{0.150000in}}{\pgfqpoint{5.700000in}{5.700000in}}%
\pgfusepath{clip}%
\pgfsetbuttcap%
\pgfsetroundjoin%
\definecolor{currentfill}{rgb}{0.265145,0.232956,0.516599}%
\pgfsetfillcolor{currentfill}%
\pgfsetfillopacity{0.700000}%
\pgfsetlinewidth{0.000000pt}%
\definecolor{currentstroke}{rgb}{0.000000,0.000000,0.000000}%
\pgfsetstrokecolor{currentstroke}%
\pgfsetdash{}{0pt}%
\pgfpathmoveto{\pgfqpoint{4.635257in}{1.399713in}}%
\pgfpathlineto{\pgfqpoint{4.649928in}{1.401935in}}%
\pgfpathlineto{\pgfqpoint{4.664611in}{1.404227in}}%
\pgfpathlineto{\pgfqpoint{4.679306in}{1.406588in}}%
\pgfpathlineto{\pgfqpoint{4.694012in}{1.409018in}}%
\pgfpathlineto{\pgfqpoint{4.702209in}{1.424356in}}%
\pgfpathlineto{\pgfqpoint{4.710401in}{1.439677in}}%
\pgfpathlineto{\pgfqpoint{4.718589in}{1.454978in}}%
\pgfpathlineto{\pgfqpoint{4.726773in}{1.470255in}}%
\pgfpathlineto{\pgfqpoint{4.712067in}{1.467502in}}%
\pgfpathlineto{\pgfqpoint{4.697372in}{1.464819in}}%
\pgfpathlineto{\pgfqpoint{4.682689in}{1.462206in}}%
\pgfpathlineto{\pgfqpoint{4.668018in}{1.459662in}}%
\pgfpathlineto{\pgfqpoint{4.659835in}{1.444700in}}%
\pgfpathlineto{\pgfqpoint{4.651646in}{1.429718in}}%
\pgfpathlineto{\pgfqpoint{4.643454in}{1.414721in}}%
\pgfpathlineto{\pgfqpoint{4.635257in}{1.399713in}}%
\pgfpathclose%
\pgfusepath{fill}%
\end{pgfscope}%
\begin{pgfscope}%
\pgfpathrectangle{\pgfqpoint{1.150000in}{0.150000in}}{\pgfqpoint{5.700000in}{5.700000in}}%
\pgfusepath{clip}%
\pgfsetbuttcap%
\pgfsetroundjoin%
\definecolor{currentfill}{rgb}{0.282327,0.094955,0.417331}%
\pgfsetfillcolor{currentfill}%
\pgfsetfillopacity{0.700000}%
\pgfsetlinewidth{0.000000pt}%
\definecolor{currentstroke}{rgb}{0.000000,0.000000,0.000000}%
\pgfsetstrokecolor{currentstroke}%
\pgfsetdash{}{0pt}%
\pgfpathmoveto{\pgfqpoint{4.236947in}{1.109656in}}%
\pgfpathlineto{\pgfqpoint{4.251469in}{1.108868in}}%
\pgfpathlineto{\pgfqpoint{4.266000in}{1.108149in}}%
\pgfpathlineto{\pgfqpoint{4.280540in}{1.107497in}}%
\pgfpathlineto{\pgfqpoint{4.295090in}{1.106915in}}%
\pgfpathlineto{\pgfqpoint{4.303382in}{1.119736in}}%
\pgfpathlineto{\pgfqpoint{4.311670in}{1.132673in}}%
\pgfpathlineto{\pgfqpoint{4.319952in}{1.145719in}}%
\pgfpathlineto{\pgfqpoint{4.328231in}{1.158868in}}%
\pgfpathlineto{\pgfqpoint{4.313687in}{1.159029in}}%
\pgfpathlineto{\pgfqpoint{4.299153in}{1.159259in}}%
\pgfpathlineto{\pgfqpoint{4.284628in}{1.159557in}}%
\pgfpathlineto{\pgfqpoint{4.270113in}{1.159924in}}%
\pgfpathlineto{\pgfqpoint{4.261829in}{1.147189in}}%
\pgfpathlineto{\pgfqpoint{4.253540in}{1.134562in}}%
\pgfpathlineto{\pgfqpoint{4.245246in}{1.122049in}}%
\pgfpathlineto{\pgfqpoint{4.236947in}{1.109656in}}%
\pgfpathclose%
\pgfusepath{fill}%
\end{pgfscope}%
\begin{pgfscope}%
\pgfpathrectangle{\pgfqpoint{1.150000in}{0.150000in}}{\pgfqpoint{5.700000in}{5.700000in}}%
\pgfusepath{clip}%
\pgfsetbuttcap%
\pgfsetroundjoin%
\definecolor{currentfill}{rgb}{0.216210,0.351535,0.550627}%
\pgfsetfillcolor{currentfill}%
\pgfsetfillopacity{0.700000}%
\pgfsetlinewidth{0.000000pt}%
\definecolor{currentstroke}{rgb}{0.000000,0.000000,0.000000}%
\pgfsetstrokecolor{currentstroke}%
\pgfsetdash{}{0pt}%
\pgfpathmoveto{\pgfqpoint{4.942709in}{1.681346in}}%
\pgfpathlineto{\pgfqpoint{4.957530in}{1.685574in}}%
\pgfpathlineto{\pgfqpoint{4.972365in}{1.689872in}}%
\pgfpathlineto{\pgfqpoint{4.987213in}{1.694241in}}%
\pgfpathlineto{\pgfqpoint{4.995333in}{1.709492in}}%
\pgfpathlineto{\pgfqpoint{5.003448in}{1.724648in}}%
\pgfpathlineto{\pgfqpoint{5.011557in}{1.739705in}}%
\pgfpathlineto{\pgfqpoint{5.019659in}{1.754661in}}%
\pgfpathlineto{\pgfqpoint{5.004810in}{1.750053in}}%
\pgfpathlineto{\pgfqpoint{4.989974in}{1.745516in}}%
\pgfpathlineto{\pgfqpoint{4.975152in}{1.741050in}}%
\pgfpathlineto{\pgfqpoint{4.967050in}{1.726267in}}%
\pgfpathlineto{\pgfqpoint{4.958942in}{1.711386in}}%
\pgfpathlineto{\pgfqpoint{4.950828in}{1.696412in}}%
\pgfpathlineto{\pgfqpoint{4.942709in}{1.681346in}}%
\pgfpathclose%
\pgfusepath{fill}%
\end{pgfscope}%
\begin{pgfscope}%
\pgfpathrectangle{\pgfqpoint{1.150000in}{0.150000in}}{\pgfqpoint{5.700000in}{5.700000in}}%
\pgfusepath{clip}%
\pgfsetbuttcap%
\pgfsetroundjoin%
\definecolor{currentfill}{rgb}{0.283229,0.120777,0.440584}%
\pgfsetfillcolor{currentfill}%
\pgfsetfillopacity{0.700000}%
\pgfsetlinewidth{0.000000pt}%
\definecolor{currentstroke}{rgb}{0.000000,0.000000,0.000000}%
\pgfsetstrokecolor{currentstroke}%
\pgfsetdash{}{0pt}%
\pgfpathmoveto{\pgfqpoint{4.328231in}{1.158868in}}%
\pgfpathlineto{\pgfqpoint{4.342784in}{1.158775in}}%
\pgfpathlineto{\pgfqpoint{4.357348in}{1.158751in}}%
\pgfpathlineto{\pgfqpoint{4.371921in}{1.158795in}}%
\pgfpathlineto{\pgfqpoint{4.386504in}{1.158908in}}%
\pgfpathlineto{\pgfqpoint{4.394773in}{1.172562in}}%
\pgfpathlineto{\pgfqpoint{4.403038in}{1.186302in}}%
\pgfpathlineto{\pgfqpoint{4.411299in}{1.200122in}}%
\pgfpathlineto{\pgfqpoint{4.419555in}{1.214014in}}%
\pgfpathlineto{\pgfqpoint{4.404976in}{1.213500in}}%
\pgfpathlineto{\pgfqpoint{4.390407in}{1.213053in}}%
\pgfpathlineto{\pgfqpoint{4.375848in}{1.212676in}}%
\pgfpathlineto{\pgfqpoint{4.361299in}{1.212368in}}%
\pgfpathlineto{\pgfqpoint{4.353039in}{1.198869in}}%
\pgfpathlineto{\pgfqpoint{4.344774in}{1.185448in}}%
\pgfpathlineto{\pgfqpoint{4.336505in}{1.172113in}}%
\pgfpathlineto{\pgfqpoint{4.328231in}{1.158868in}}%
\pgfpathclose%
\pgfusepath{fill}%
\end{pgfscope}%
\begin{pgfscope}%
\pgfpathrectangle{\pgfqpoint{1.150000in}{0.150000in}}{\pgfqpoint{5.700000in}{5.700000in}}%
\pgfusepath{clip}%
\pgfsetbuttcap%
\pgfsetroundjoin%
\definecolor{currentfill}{rgb}{0.281887,0.150881,0.465405}%
\pgfsetfillcolor{currentfill}%
\pgfsetfillopacity{0.700000}%
\pgfsetlinewidth{0.000000pt}%
\definecolor{currentstroke}{rgb}{0.000000,0.000000,0.000000}%
\pgfsetstrokecolor{currentstroke}%
\pgfsetdash{}{0pt}%
\pgfpathmoveto{\pgfqpoint{4.419555in}{1.214014in}}%
\pgfpathlineto{\pgfqpoint{4.434144in}{1.214598in}}%
\pgfpathlineto{\pgfqpoint{4.448744in}{1.215250in}}%
\pgfpathlineto{\pgfqpoint{4.463354in}{1.215971in}}%
\pgfpathlineto{\pgfqpoint{4.477975in}{1.216760in}}%
\pgfpathlineto{\pgfqpoint{4.486224in}{1.231111in}}%
\pgfpathlineto{\pgfqpoint{4.494469in}{1.245518in}}%
\pgfpathlineto{\pgfqpoint{4.502709in}{1.259977in}}%
\pgfpathlineto{\pgfqpoint{4.510946in}{1.274481in}}%
\pgfpathlineto{\pgfqpoint{4.496328in}{1.273309in}}%
\pgfpathlineto{\pgfqpoint{4.481720in}{1.272206in}}%
\pgfpathlineto{\pgfqpoint{4.467123in}{1.271172in}}%
\pgfpathlineto{\pgfqpoint{4.452537in}{1.270206in}}%
\pgfpathlineto{\pgfqpoint{4.444298in}{1.256077in}}%
\pgfpathlineto{\pgfqpoint{4.436055in}{1.241998in}}%
\pgfpathlineto{\pgfqpoint{4.427807in}{1.227975in}}%
\pgfpathlineto{\pgfqpoint{4.419555in}{1.214014in}}%
\pgfpathclose%
\pgfusepath{fill}%
\end{pgfscope}%
\begin{pgfscope}%
\pgfpathrectangle{\pgfqpoint{1.150000in}{0.150000in}}{\pgfqpoint{5.700000in}{5.700000in}}%
\pgfusepath{clip}%
\pgfsetbuttcap%
\pgfsetroundjoin%
\definecolor{currentfill}{rgb}{0.252194,0.269783,0.531579}%
\pgfsetfillcolor{currentfill}%
\pgfsetfillopacity{0.700000}%
\pgfsetlinewidth{0.000000pt}%
\definecolor{currentstroke}{rgb}{0.000000,0.000000,0.000000}%
\pgfsetstrokecolor{currentstroke}%
\pgfsetdash{}{0pt}%
\pgfpathmoveto{\pgfqpoint{4.726773in}{1.470255in}}%
\pgfpathlineto{\pgfqpoint{4.741491in}{1.473077in}}%
\pgfpathlineto{\pgfqpoint{4.756222in}{1.475969in}}%
\pgfpathlineto{\pgfqpoint{4.770964in}{1.478931in}}%
\pgfpathlineto{\pgfqpoint{4.785719in}{1.481963in}}%
\pgfpathlineto{\pgfqpoint{4.793899in}{1.497519in}}%
\pgfpathlineto{\pgfqpoint{4.802074in}{1.513037in}}%
\pgfpathlineto{\pgfqpoint{4.810244in}{1.528510in}}%
\pgfpathlineto{\pgfqpoint{4.818410in}{1.543936in}}%
\pgfpathlineto{\pgfqpoint{4.803654in}{1.540602in}}%
\pgfpathlineto{\pgfqpoint{4.788911in}{1.537338in}}%
\pgfpathlineto{\pgfqpoint{4.774180in}{1.534145in}}%
\pgfpathlineto{\pgfqpoint{4.759461in}{1.531021in}}%
\pgfpathlineto{\pgfqpoint{4.751296in}{1.515889in}}%
\pgfpathlineto{\pgfqpoint{4.743126in}{1.500714in}}%
\pgfpathlineto{\pgfqpoint{4.734952in}{1.485501in}}%
\pgfpathlineto{\pgfqpoint{4.726773in}{1.470255in}}%
\pgfpathclose%
\pgfusepath{fill}%
\end{pgfscope}%
\begin{pgfscope}%
\pgfpathrectangle{\pgfqpoint{1.150000in}{0.150000in}}{\pgfqpoint{5.700000in}{5.700000in}}%
\pgfusepath{clip}%
\pgfsetbuttcap%
\pgfsetroundjoin%
\definecolor{currentfill}{rgb}{0.278012,0.180367,0.486697}%
\pgfsetfillcolor{currentfill}%
\pgfsetfillopacity{0.700000}%
\pgfsetlinewidth{0.000000pt}%
\definecolor{currentstroke}{rgb}{0.000000,0.000000,0.000000}%
\pgfsetstrokecolor{currentstroke}%
\pgfsetdash{}{0pt}%
\pgfpathmoveto{\pgfqpoint{4.510946in}{1.274481in}}%
\pgfpathlineto{\pgfqpoint{4.525575in}{1.275722in}}%
\pgfpathlineto{\pgfqpoint{4.540214in}{1.277032in}}%
\pgfpathlineto{\pgfqpoint{4.554865in}{1.278410in}}%
\pgfpathlineto{\pgfqpoint{4.569526in}{1.279858in}}%
\pgfpathlineto{\pgfqpoint{4.577757in}{1.294773in}}%
\pgfpathlineto{\pgfqpoint{4.585984in}{1.309717in}}%
\pgfpathlineto{\pgfqpoint{4.594207in}{1.324686in}}%
\pgfpathlineto{\pgfqpoint{4.602425in}{1.339673in}}%
\pgfpathlineto{\pgfqpoint{4.587765in}{1.337863in}}%
\pgfpathlineto{\pgfqpoint{4.573116in}{1.336121in}}%
\pgfpathlineto{\pgfqpoint{4.558477in}{1.334449in}}%
\pgfpathlineto{\pgfqpoint{4.543850in}{1.332846in}}%
\pgfpathlineto{\pgfqpoint{4.535630in}{1.318213in}}%
\pgfpathlineto{\pgfqpoint{4.527406in}{1.303605in}}%
\pgfpathlineto{\pgfqpoint{4.519178in}{1.289025in}}%
\pgfpathlineto{\pgfqpoint{4.510946in}{1.274481in}}%
\pgfpathclose%
\pgfusepath{fill}%
\end{pgfscope}%
\begin{pgfscope}%
\pgfpathrectangle{\pgfqpoint{1.150000in}{0.150000in}}{\pgfqpoint{5.700000in}{5.700000in}}%
\pgfusepath{clip}%
\pgfsetbuttcap%
\pgfsetroundjoin%
\definecolor{currentfill}{rgb}{0.239346,0.300855,0.540844}%
\pgfsetfillcolor{currentfill}%
\pgfsetfillopacity{0.700000}%
\pgfsetlinewidth{0.000000pt}%
\definecolor{currentstroke}{rgb}{0.000000,0.000000,0.000000}%
\pgfsetstrokecolor{currentstroke}%
\pgfsetdash{}{0pt}%
\pgfpathmoveto{\pgfqpoint{4.818410in}{1.543936in}}%
\pgfpathlineto{\pgfqpoint{4.833178in}{1.547340in}}%
\pgfpathlineto{\pgfqpoint{4.847959in}{1.550814in}}%
\pgfpathlineto{\pgfqpoint{4.862752in}{1.554359in}}%
\pgfpathlineto{\pgfqpoint{4.877557in}{1.557973in}}%
\pgfpathlineto{\pgfqpoint{4.885720in}{1.573637in}}%
\pgfpathlineto{\pgfqpoint{4.893877in}{1.589239in}}%
\pgfpathlineto{\pgfqpoint{4.902029in}{1.604776in}}%
\pgfpathlineto{\pgfqpoint{4.910175in}{1.620244in}}%
\pgfpathlineto{\pgfqpoint{4.895368in}{1.616347in}}%
\pgfpathlineto{\pgfqpoint{4.880573in}{1.612521in}}%
\pgfpathlineto{\pgfqpoint{4.865792in}{1.608766in}}%
\pgfpathlineto{\pgfqpoint{4.851023in}{1.605081in}}%
\pgfpathlineto{\pgfqpoint{4.842877in}{1.589886in}}%
\pgfpathlineto{\pgfqpoint{4.834726in}{1.574628in}}%
\pgfpathlineto{\pgfqpoint{4.826571in}{1.559310in}}%
\pgfpathlineto{\pgfqpoint{4.818410in}{1.543936in}}%
\pgfpathclose%
\pgfusepath{fill}%
\end{pgfscope}%
\begin{pgfscope}%
\pgfpathrectangle{\pgfqpoint{1.150000in}{0.150000in}}{\pgfqpoint{5.700000in}{5.700000in}}%
\pgfusepath{clip}%
\pgfsetbuttcap%
\pgfsetroundjoin%
\definecolor{currentfill}{rgb}{0.270595,0.214069,0.507052}%
\pgfsetfillcolor{currentfill}%
\pgfsetfillopacity{0.700000}%
\pgfsetlinewidth{0.000000pt}%
\definecolor{currentstroke}{rgb}{0.000000,0.000000,0.000000}%
\pgfsetstrokecolor{currentstroke}%
\pgfsetdash{}{0pt}%
\pgfpathmoveto{\pgfqpoint{4.602425in}{1.339673in}}%
\pgfpathlineto{\pgfqpoint{4.617097in}{1.341553in}}%
\pgfpathlineto{\pgfqpoint{4.631780in}{1.343502in}}%
\pgfpathlineto{\pgfqpoint{4.646475in}{1.345520in}}%
\pgfpathlineto{\pgfqpoint{4.661181in}{1.347607in}}%
\pgfpathlineto{\pgfqpoint{4.669395in}{1.362959in}}%
\pgfpathlineto{\pgfqpoint{4.677605in}{1.378315in}}%
\pgfpathlineto{\pgfqpoint{4.685810in}{1.393670in}}%
\pgfpathlineto{\pgfqpoint{4.694012in}{1.409018in}}%
\pgfpathlineto{\pgfqpoint{4.679306in}{1.406588in}}%
\pgfpathlineto{\pgfqpoint{4.664611in}{1.404227in}}%
\pgfpathlineto{\pgfqpoint{4.649928in}{1.401935in}}%
\pgfpathlineto{\pgfqpoint{4.635257in}{1.399713in}}%
\pgfpathlineto{\pgfqpoint{4.627056in}{1.384700in}}%
\pgfpathlineto{\pgfqpoint{4.618850in}{1.369685in}}%
\pgfpathlineto{\pgfqpoint{4.610640in}{1.354675in}}%
\pgfpathlineto{\pgfqpoint{4.602425in}{1.339673in}}%
\pgfpathclose%
\pgfusepath{fill}%
\end{pgfscope}%
\begin{pgfscope}%
\pgfpathrectangle{\pgfqpoint{1.150000in}{0.150000in}}{\pgfqpoint{5.700000in}{5.700000in}}%
\pgfusepath{clip}%
\pgfsetbuttcap%
\pgfsetroundjoin%
\definecolor{currentfill}{rgb}{0.223925,0.334994,0.548053}%
\pgfsetfillcolor{currentfill}%
\pgfsetfillopacity{0.700000}%
\pgfsetlinewidth{0.000000pt}%
\definecolor{currentstroke}{rgb}{0.000000,0.000000,0.000000}%
\pgfsetstrokecolor{currentstroke}%
\pgfsetdash{}{0pt}%
\pgfpathmoveto{\pgfqpoint{4.910175in}{1.620244in}}%
\pgfpathlineto{\pgfqpoint{4.924996in}{1.624210in}}%
\pgfpathlineto{\pgfqpoint{4.939829in}{1.628248in}}%
\pgfpathlineto{\pgfqpoint{4.954675in}{1.632356in}}%
\pgfpathlineto{\pgfqpoint{4.962818in}{1.647952in}}%
\pgfpathlineto{\pgfqpoint{4.970955in}{1.663467in}}%
\pgfpathlineto{\pgfqpoint{4.979087in}{1.678898in}}%
\pgfpathlineto{\pgfqpoint{4.987213in}{1.694241in}}%
\pgfpathlineto{\pgfqpoint{4.972365in}{1.689872in}}%
\pgfpathlineto{\pgfqpoint{4.957530in}{1.685574in}}%
\pgfpathlineto{\pgfqpoint{4.942709in}{1.681346in}}%
\pgfpathlineto{\pgfqpoint{4.934584in}{1.666193in}}%
\pgfpathlineto{\pgfqpoint{4.926453in}{1.650956in}}%
\pgfpathlineto{\pgfqpoint{4.918317in}{1.635638in}}%
\pgfpathlineto{\pgfqpoint{4.910175in}{1.620244in}}%
\pgfpathclose%
\pgfusepath{fill}%
\end{pgfscope}%
\begin{pgfscope}%
\pgfpathrectangle{\pgfqpoint{1.150000in}{0.150000in}}{\pgfqpoint{5.700000in}{5.700000in}}%
\pgfusepath{clip}%
\pgfsetbuttcap%
\pgfsetroundjoin%
\definecolor{currentfill}{rgb}{0.283091,0.110553,0.431554}%
\pgfsetfillcolor{currentfill}%
\pgfsetfillopacity{0.700000}%
\pgfsetlinewidth{0.000000pt}%
\definecolor{currentstroke}{rgb}{0.000000,0.000000,0.000000}%
\pgfsetstrokecolor{currentstroke}%
\pgfsetdash{}{0pt}%
\pgfpathmoveto{\pgfqpoint{4.295090in}{1.106915in}}%
\pgfpathlineto{\pgfqpoint{4.309649in}{1.106400in}}%
\pgfpathlineto{\pgfqpoint{4.324217in}{1.105954in}}%
\pgfpathlineto{\pgfqpoint{4.338796in}{1.105575in}}%
\pgfpathlineto{\pgfqpoint{4.353383in}{1.105266in}}%
\pgfpathlineto{\pgfqpoint{4.361670in}{1.118517in}}%
\pgfpathlineto{\pgfqpoint{4.369952in}{1.131879in}}%
\pgfpathlineto{\pgfqpoint{4.378230in}{1.145344in}}%
\pgfpathlineto{\pgfqpoint{4.386504in}{1.158908in}}%
\pgfpathlineto{\pgfqpoint{4.371921in}{1.158795in}}%
\pgfpathlineto{\pgfqpoint{4.357348in}{1.158751in}}%
\pgfpathlineto{\pgfqpoint{4.342784in}{1.158775in}}%
\pgfpathlineto{\pgfqpoint{4.328231in}{1.158868in}}%
\pgfpathlineto{\pgfqpoint{4.319952in}{1.145719in}}%
\pgfpathlineto{\pgfqpoint{4.311670in}{1.132673in}}%
\pgfpathlineto{\pgfqpoint{4.303382in}{1.119736in}}%
\pgfpathlineto{\pgfqpoint{4.295090in}{1.106915in}}%
\pgfpathclose%
\pgfusepath{fill}%
\end{pgfscope}%
\begin{pgfscope}%
\pgfpathrectangle{\pgfqpoint{1.150000in}{0.150000in}}{\pgfqpoint{5.700000in}{5.700000in}}%
\pgfusepath{clip}%
\pgfsetbuttcap%
\pgfsetroundjoin%
\definecolor{currentfill}{rgb}{0.258965,0.251537,0.524736}%
\pgfsetfillcolor{currentfill}%
\pgfsetfillopacity{0.700000}%
\pgfsetlinewidth{0.000000pt}%
\definecolor{currentstroke}{rgb}{0.000000,0.000000,0.000000}%
\pgfsetstrokecolor{currentstroke}%
\pgfsetdash{}{0pt}%
\pgfpathmoveto{\pgfqpoint{4.694012in}{1.409018in}}%
\pgfpathlineto{\pgfqpoint{4.708730in}{1.411518in}}%
\pgfpathlineto{\pgfqpoint{4.723459in}{1.414088in}}%
\pgfpathlineto{\pgfqpoint{4.738201in}{1.416727in}}%
\pgfpathlineto{\pgfqpoint{4.752954in}{1.419435in}}%
\pgfpathlineto{\pgfqpoint{4.761152in}{1.435103in}}%
\pgfpathlineto{\pgfqpoint{4.769346in}{1.450750in}}%
\pgfpathlineto{\pgfqpoint{4.777535in}{1.466372in}}%
\pgfpathlineto{\pgfqpoint{4.785719in}{1.481963in}}%
\pgfpathlineto{\pgfqpoint{4.770964in}{1.478931in}}%
\pgfpathlineto{\pgfqpoint{4.756222in}{1.475969in}}%
\pgfpathlineto{\pgfqpoint{4.741491in}{1.473077in}}%
\pgfpathlineto{\pgfqpoint{4.726773in}{1.470255in}}%
\pgfpathlineto{\pgfqpoint{4.718589in}{1.454978in}}%
\pgfpathlineto{\pgfqpoint{4.710401in}{1.439677in}}%
\pgfpathlineto{\pgfqpoint{4.702209in}{1.424356in}}%
\pgfpathlineto{\pgfqpoint{4.694012in}{1.409018in}}%
\pgfpathclose%
\pgfusepath{fill}%
\end{pgfscope}%
\begin{pgfscope}%
\pgfpathrectangle{\pgfqpoint{1.150000in}{0.150000in}}{\pgfqpoint{5.700000in}{5.700000in}}%
\pgfusepath{clip}%
\pgfsetbuttcap%
\pgfsetroundjoin%
\definecolor{currentfill}{rgb}{0.282884,0.135920,0.453427}%
\pgfsetfillcolor{currentfill}%
\pgfsetfillopacity{0.700000}%
\pgfsetlinewidth{0.000000pt}%
\definecolor{currentstroke}{rgb}{0.000000,0.000000,0.000000}%
\pgfsetstrokecolor{currentstroke}%
\pgfsetdash{}{0pt}%
\pgfpathmoveto{\pgfqpoint{4.386504in}{1.158908in}}%
\pgfpathlineto{\pgfqpoint{4.401097in}{1.159089in}}%
\pgfpathlineto{\pgfqpoint{4.415700in}{1.159338in}}%
\pgfpathlineto{\pgfqpoint{4.430313in}{1.159656in}}%
\pgfpathlineto{\pgfqpoint{4.444937in}{1.160043in}}%
\pgfpathlineto{\pgfqpoint{4.453202in}{1.174108in}}%
\pgfpathlineto{\pgfqpoint{4.461464in}{1.188253in}}%
\pgfpathlineto{\pgfqpoint{4.469721in}{1.202472in}}%
\pgfpathlineto{\pgfqpoint{4.477975in}{1.216760in}}%
\pgfpathlineto{\pgfqpoint{4.463354in}{1.215971in}}%
\pgfpathlineto{\pgfqpoint{4.448744in}{1.215250in}}%
\pgfpathlineto{\pgfqpoint{4.434144in}{1.214598in}}%
\pgfpathlineto{\pgfqpoint{4.419555in}{1.214014in}}%
\pgfpathlineto{\pgfqpoint{4.411299in}{1.200122in}}%
\pgfpathlineto{\pgfqpoint{4.403038in}{1.186302in}}%
\pgfpathlineto{\pgfqpoint{4.394773in}{1.172562in}}%
\pgfpathlineto{\pgfqpoint{4.386504in}{1.158908in}}%
\pgfpathclose%
\pgfusepath{fill}%
\end{pgfscope}%
\begin{pgfscope}%
\pgfpathrectangle{\pgfqpoint{1.150000in}{0.150000in}}{\pgfqpoint{5.700000in}{5.700000in}}%
\pgfusepath{clip}%
\pgfsetbuttcap%
\pgfsetroundjoin%
\definecolor{currentfill}{rgb}{0.280255,0.165693,0.476498}%
\pgfsetfillcolor{currentfill}%
\pgfsetfillopacity{0.700000}%
\pgfsetlinewidth{0.000000pt}%
\definecolor{currentstroke}{rgb}{0.000000,0.000000,0.000000}%
\pgfsetstrokecolor{currentstroke}%
\pgfsetdash{}{0pt}%
\pgfpathmoveto{\pgfqpoint{4.477975in}{1.216760in}}%
\pgfpathlineto{\pgfqpoint{4.492605in}{1.217618in}}%
\pgfpathlineto{\pgfqpoint{4.507247in}{1.218545in}}%
\pgfpathlineto{\pgfqpoint{4.521899in}{1.219540in}}%
\pgfpathlineto{\pgfqpoint{4.536562in}{1.220604in}}%
\pgfpathlineto{\pgfqpoint{4.544809in}{1.235346in}}%
\pgfpathlineto{\pgfqpoint{4.553052in}{1.250139in}}%
\pgfpathlineto{\pgfqpoint{4.561291in}{1.264978in}}%
\pgfpathlineto{\pgfqpoint{4.569526in}{1.279858in}}%
\pgfpathlineto{\pgfqpoint{4.554865in}{1.278410in}}%
\pgfpathlineto{\pgfqpoint{4.540214in}{1.277032in}}%
\pgfpathlineto{\pgfqpoint{4.525575in}{1.275722in}}%
\pgfpathlineto{\pgfqpoint{4.510946in}{1.274481in}}%
\pgfpathlineto{\pgfqpoint{4.502709in}{1.259977in}}%
\pgfpathlineto{\pgfqpoint{4.494469in}{1.245518in}}%
\pgfpathlineto{\pgfqpoint{4.486224in}{1.231111in}}%
\pgfpathlineto{\pgfqpoint{4.477975in}{1.216760in}}%
\pgfpathclose%
\pgfusepath{fill}%
\end{pgfscope}%
\begin{pgfscope}%
\pgfpathrectangle{\pgfqpoint{1.150000in}{0.150000in}}{\pgfqpoint{5.700000in}{5.700000in}}%
\pgfusepath{clip}%
\pgfsetbuttcap%
\pgfsetroundjoin%
\definecolor{currentfill}{rgb}{0.246811,0.283237,0.535941}%
\pgfsetfillcolor{currentfill}%
\pgfsetfillopacity{0.700000}%
\pgfsetlinewidth{0.000000pt}%
\definecolor{currentstroke}{rgb}{0.000000,0.000000,0.000000}%
\pgfsetstrokecolor{currentstroke}%
\pgfsetdash{}{0pt}%
\pgfpathmoveto{\pgfqpoint{4.785719in}{1.481963in}}%
\pgfpathlineto{\pgfqpoint{4.800486in}{1.485065in}}%
\pgfpathlineto{\pgfqpoint{4.815265in}{1.488236in}}%
\pgfpathlineto{\pgfqpoint{4.830056in}{1.491478in}}%
\pgfpathlineto{\pgfqpoint{4.844860in}{1.494789in}}%
\pgfpathlineto{\pgfqpoint{4.853042in}{1.510656in}}%
\pgfpathlineto{\pgfqpoint{4.861218in}{1.526479in}}%
\pgfpathlineto{\pgfqpoint{4.869390in}{1.542253in}}%
\pgfpathlineto{\pgfqpoint{4.877557in}{1.557973in}}%
\pgfpathlineto{\pgfqpoint{4.862752in}{1.554359in}}%
\pgfpathlineto{\pgfqpoint{4.847959in}{1.550814in}}%
\pgfpathlineto{\pgfqpoint{4.833178in}{1.547340in}}%
\pgfpathlineto{\pgfqpoint{4.818410in}{1.543936in}}%
\pgfpathlineto{\pgfqpoint{4.810244in}{1.528510in}}%
\pgfpathlineto{\pgfqpoint{4.802074in}{1.513037in}}%
\pgfpathlineto{\pgfqpoint{4.793899in}{1.497519in}}%
\pgfpathlineto{\pgfqpoint{4.785719in}{1.481963in}}%
\pgfpathclose%
\pgfusepath{fill}%
\end{pgfscope}%
\begin{pgfscope}%
\pgfpathrectangle{\pgfqpoint{1.150000in}{0.150000in}}{\pgfqpoint{5.700000in}{5.700000in}}%
\pgfusepath{clip}%
\pgfsetbuttcap%
\pgfsetroundjoin%
\definecolor{currentfill}{rgb}{0.274128,0.199721,0.498911}%
\pgfsetfillcolor{currentfill}%
\pgfsetfillopacity{0.700000}%
\pgfsetlinewidth{0.000000pt}%
\definecolor{currentstroke}{rgb}{0.000000,0.000000,0.000000}%
\pgfsetstrokecolor{currentstroke}%
\pgfsetdash{}{0pt}%
\pgfpathmoveto{\pgfqpoint{4.569526in}{1.279858in}}%
\pgfpathlineto{\pgfqpoint{4.584199in}{1.281374in}}%
\pgfpathlineto{\pgfqpoint{4.598883in}{1.282960in}}%
\pgfpathlineto{\pgfqpoint{4.613577in}{1.284614in}}%
\pgfpathlineto{\pgfqpoint{4.628283in}{1.286337in}}%
\pgfpathlineto{\pgfqpoint{4.636514in}{1.301623in}}%
\pgfpathlineto{\pgfqpoint{4.644740in}{1.316934in}}%
\pgfpathlineto{\pgfqpoint{4.652963in}{1.332264in}}%
\pgfpathlineto{\pgfqpoint{4.661181in}{1.347607in}}%
\pgfpathlineto{\pgfqpoint{4.646475in}{1.345520in}}%
\pgfpathlineto{\pgfqpoint{4.631780in}{1.343502in}}%
\pgfpathlineto{\pgfqpoint{4.617097in}{1.341553in}}%
\pgfpathlineto{\pgfqpoint{4.602425in}{1.339673in}}%
\pgfpathlineto{\pgfqpoint{4.594207in}{1.324686in}}%
\pgfpathlineto{\pgfqpoint{4.585984in}{1.309717in}}%
\pgfpathlineto{\pgfqpoint{4.577757in}{1.294773in}}%
\pgfpathlineto{\pgfqpoint{4.569526in}{1.279858in}}%
\pgfpathclose%
\pgfusepath{fill}%
\end{pgfscope}%
\begin{pgfscope}%
\pgfpathrectangle{\pgfqpoint{1.150000in}{0.150000in}}{\pgfqpoint{5.700000in}{5.700000in}}%
\pgfusepath{clip}%
\pgfsetbuttcap%
\pgfsetroundjoin%
\definecolor{currentfill}{rgb}{0.231674,0.318106,0.544834}%
\pgfsetfillcolor{currentfill}%
\pgfsetfillopacity{0.700000}%
\pgfsetlinewidth{0.000000pt}%
\definecolor{currentstroke}{rgb}{0.000000,0.000000,0.000000}%
\pgfsetstrokecolor{currentstroke}%
\pgfsetdash{}{0pt}%
\pgfpathmoveto{\pgfqpoint{4.877557in}{1.557973in}}%
\pgfpathlineto{\pgfqpoint{4.892376in}{1.561658in}}%
\pgfpathlineto{\pgfqpoint{4.907207in}{1.565413in}}%
\pgfpathlineto{\pgfqpoint{4.922051in}{1.569238in}}%
\pgfpathlineto{\pgfqpoint{4.930215in}{1.585120in}}%
\pgfpathlineto{\pgfqpoint{4.938373in}{1.600936in}}%
\pgfpathlineto{\pgfqpoint{4.946527in}{1.616682in}}%
\pgfpathlineto{\pgfqpoint{4.954675in}{1.632356in}}%
\pgfpathlineto{\pgfqpoint{4.939829in}{1.628248in}}%
\pgfpathlineto{\pgfqpoint{4.924996in}{1.624210in}}%
\pgfpathlineto{\pgfqpoint{4.910175in}{1.620244in}}%
\pgfpathlineto{\pgfqpoint{4.902029in}{1.604776in}}%
\pgfpathlineto{\pgfqpoint{4.893877in}{1.589239in}}%
\pgfpathlineto{\pgfqpoint{4.885720in}{1.573637in}}%
\pgfpathlineto{\pgfqpoint{4.877557in}{1.557973in}}%
\pgfpathclose%
\pgfusepath{fill}%
\end{pgfscope}%
\begin{pgfscope}%
\pgfpathrectangle{\pgfqpoint{1.150000in}{0.150000in}}{\pgfqpoint{5.700000in}{5.700000in}}%
\pgfusepath{clip}%
\pgfsetbuttcap%
\pgfsetroundjoin%
\definecolor{currentfill}{rgb}{0.265145,0.232956,0.516599}%
\pgfsetfillcolor{currentfill}%
\pgfsetfillopacity{0.700000}%
\pgfsetlinewidth{0.000000pt}%
\definecolor{currentstroke}{rgb}{0.000000,0.000000,0.000000}%
\pgfsetstrokecolor{currentstroke}%
\pgfsetdash{}{0pt}%
\pgfpathmoveto{\pgfqpoint{4.661181in}{1.347607in}}%
\pgfpathlineto{\pgfqpoint{4.675898in}{1.349764in}}%
\pgfpathlineto{\pgfqpoint{4.690627in}{1.351989in}}%
\pgfpathlineto{\pgfqpoint{4.705368in}{1.354284in}}%
\pgfpathlineto{\pgfqpoint{4.720120in}{1.356648in}}%
\pgfpathlineto{\pgfqpoint{4.728335in}{1.372352in}}%
\pgfpathlineto{\pgfqpoint{4.736546in}{1.388055in}}%
\pgfpathlineto{\pgfqpoint{4.744752in}{1.403751in}}%
\pgfpathlineto{\pgfqpoint{4.752954in}{1.419435in}}%
\pgfpathlineto{\pgfqpoint{4.738201in}{1.416727in}}%
\pgfpathlineto{\pgfqpoint{4.723459in}{1.414088in}}%
\pgfpathlineto{\pgfqpoint{4.708730in}{1.411518in}}%
\pgfpathlineto{\pgfqpoint{4.694012in}{1.409018in}}%
\pgfpathlineto{\pgfqpoint{4.685810in}{1.393670in}}%
\pgfpathlineto{\pgfqpoint{4.677605in}{1.378315in}}%
\pgfpathlineto{\pgfqpoint{4.669395in}{1.362959in}}%
\pgfpathlineto{\pgfqpoint{4.661181in}{1.347607in}}%
\pgfpathclose%
\pgfusepath{fill}%
\end{pgfscope}%
\begin{pgfscope}%
\pgfpathrectangle{\pgfqpoint{1.150000in}{0.150000in}}{\pgfqpoint{5.700000in}{5.700000in}}%
\pgfusepath{clip}%
\pgfsetbuttcap%
\pgfsetroundjoin%
\definecolor{currentfill}{rgb}{0.283229,0.120777,0.440584}%
\pgfsetfillcolor{currentfill}%
\pgfsetfillopacity{0.700000}%
\pgfsetlinewidth{0.000000pt}%
\definecolor{currentstroke}{rgb}{0.000000,0.000000,0.000000}%
\pgfsetstrokecolor{currentstroke}%
\pgfsetdash{}{0pt}%
\pgfpathmoveto{\pgfqpoint{4.353383in}{1.105266in}}%
\pgfpathlineto{\pgfqpoint{4.367981in}{1.105024in}}%
\pgfpathlineto{\pgfqpoint{4.382588in}{1.104850in}}%
\pgfpathlineto{\pgfqpoint{4.397206in}{1.104745in}}%
\pgfpathlineto{\pgfqpoint{4.411833in}{1.104707in}}%
\pgfpathlineto{\pgfqpoint{4.420115in}{1.118390in}}%
\pgfpathlineto{\pgfqpoint{4.428393in}{1.132177in}}%
\pgfpathlineto{\pgfqpoint{4.436667in}{1.146064in}}%
\pgfpathlineto{\pgfqpoint{4.444937in}{1.160043in}}%
\pgfpathlineto{\pgfqpoint{4.430313in}{1.159656in}}%
\pgfpathlineto{\pgfqpoint{4.415700in}{1.159338in}}%
\pgfpathlineto{\pgfqpoint{4.401097in}{1.159089in}}%
\pgfpathlineto{\pgfqpoint{4.386504in}{1.158908in}}%
\pgfpathlineto{\pgfqpoint{4.378230in}{1.145344in}}%
\pgfpathlineto{\pgfqpoint{4.369952in}{1.131879in}}%
\pgfpathlineto{\pgfqpoint{4.361670in}{1.118517in}}%
\pgfpathlineto{\pgfqpoint{4.353383in}{1.105266in}}%
\pgfpathclose%
\pgfusepath{fill}%
\end{pgfscope}%
\begin{pgfscope}%
\pgfpathrectangle{\pgfqpoint{1.150000in}{0.150000in}}{\pgfqpoint{5.700000in}{5.700000in}}%
\pgfusepath{clip}%
\pgfsetbuttcap%
\pgfsetroundjoin%
\definecolor{currentfill}{rgb}{0.281887,0.150881,0.465405}%
\pgfsetfillcolor{currentfill}%
\pgfsetfillopacity{0.700000}%
\pgfsetlinewidth{0.000000pt}%
\definecolor{currentstroke}{rgb}{0.000000,0.000000,0.000000}%
\pgfsetstrokecolor{currentstroke}%
\pgfsetdash{}{0pt}%
\pgfpathmoveto{\pgfqpoint{4.444937in}{1.160043in}}%
\pgfpathlineto{\pgfqpoint{4.459570in}{1.160497in}}%
\pgfpathlineto{\pgfqpoint{4.474214in}{1.161020in}}%
\pgfpathlineto{\pgfqpoint{4.488869in}{1.161611in}}%
\pgfpathlineto{\pgfqpoint{4.503534in}{1.162271in}}%
\pgfpathlineto{\pgfqpoint{4.511797in}{1.176748in}}%
\pgfpathlineto{\pgfqpoint{4.520056in}{1.191299in}}%
\pgfpathlineto{\pgfqpoint{4.528311in}{1.205920in}}%
\pgfpathlineto{\pgfqpoint{4.536562in}{1.220604in}}%
\pgfpathlineto{\pgfqpoint{4.521899in}{1.219540in}}%
\pgfpathlineto{\pgfqpoint{4.507247in}{1.218545in}}%
\pgfpathlineto{\pgfqpoint{4.492605in}{1.217618in}}%
\pgfpathlineto{\pgfqpoint{4.477975in}{1.216760in}}%
\pgfpathlineto{\pgfqpoint{4.469721in}{1.202472in}}%
\pgfpathlineto{\pgfqpoint{4.461464in}{1.188253in}}%
\pgfpathlineto{\pgfqpoint{4.453202in}{1.174108in}}%
\pgfpathlineto{\pgfqpoint{4.444937in}{1.160043in}}%
\pgfpathclose%
\pgfusepath{fill}%
\end{pgfscope}%
\begin{pgfscope}%
\pgfpathrectangle{\pgfqpoint{1.150000in}{0.150000in}}{\pgfqpoint{5.700000in}{5.700000in}}%
\pgfusepath{clip}%
\pgfsetbuttcap%
\pgfsetroundjoin%
\definecolor{currentfill}{rgb}{0.252194,0.269783,0.531579}%
\pgfsetfillcolor{currentfill}%
\pgfsetfillopacity{0.700000}%
\pgfsetlinewidth{0.000000pt}%
\definecolor{currentstroke}{rgb}{0.000000,0.000000,0.000000}%
\pgfsetstrokecolor{currentstroke}%
\pgfsetdash{}{0pt}%
\pgfpathmoveto{\pgfqpoint{4.752954in}{1.419435in}}%
\pgfpathlineto{\pgfqpoint{4.767720in}{1.422213in}}%
\pgfpathlineto{\pgfqpoint{4.782497in}{1.425061in}}%
\pgfpathlineto{\pgfqpoint{4.797287in}{1.427978in}}%
\pgfpathlineto{\pgfqpoint{4.812089in}{1.430965in}}%
\pgfpathlineto{\pgfqpoint{4.820289in}{1.446965in}}%
\pgfpathlineto{\pgfqpoint{4.828484in}{1.462939in}}%
\pgfpathlineto{\pgfqpoint{4.836674in}{1.478881in}}%
\pgfpathlineto{\pgfqpoint{4.844860in}{1.494789in}}%
\pgfpathlineto{\pgfqpoint{4.830056in}{1.491478in}}%
\pgfpathlineto{\pgfqpoint{4.815265in}{1.488236in}}%
\pgfpathlineto{\pgfqpoint{4.800486in}{1.485065in}}%
\pgfpathlineto{\pgfqpoint{4.785719in}{1.481963in}}%
\pgfpathlineto{\pgfqpoint{4.777535in}{1.466372in}}%
\pgfpathlineto{\pgfqpoint{4.769346in}{1.450750in}}%
\pgfpathlineto{\pgfqpoint{4.761152in}{1.435103in}}%
\pgfpathlineto{\pgfqpoint{4.752954in}{1.419435in}}%
\pgfpathclose%
\pgfusepath{fill}%
\end{pgfscope}%
\begin{pgfscope}%
\pgfpathrectangle{\pgfqpoint{1.150000in}{0.150000in}}{\pgfqpoint{5.700000in}{5.700000in}}%
\pgfusepath{clip}%
\pgfsetbuttcap%
\pgfsetroundjoin%
\definecolor{currentfill}{rgb}{0.278012,0.180367,0.486697}%
\pgfsetfillcolor{currentfill}%
\pgfsetfillopacity{0.700000}%
\pgfsetlinewidth{0.000000pt}%
\definecolor{currentstroke}{rgb}{0.000000,0.000000,0.000000}%
\pgfsetstrokecolor{currentstroke}%
\pgfsetdash{}{0pt}%
\pgfpathmoveto{\pgfqpoint{4.536562in}{1.220604in}}%
\pgfpathlineto{\pgfqpoint{4.551236in}{1.221737in}}%
\pgfpathlineto{\pgfqpoint{4.565920in}{1.222938in}}%
\pgfpathlineto{\pgfqpoint{4.580616in}{1.224208in}}%
\pgfpathlineto{\pgfqpoint{4.595322in}{1.225546in}}%
\pgfpathlineto{\pgfqpoint{4.603568in}{1.240680in}}%
\pgfpathlineto{\pgfqpoint{4.611810in}{1.255860in}}%
\pgfpathlineto{\pgfqpoint{4.620049in}{1.271081in}}%
\pgfpathlineto{\pgfqpoint{4.628283in}{1.286337in}}%
\pgfpathlineto{\pgfqpoint{4.613577in}{1.284614in}}%
\pgfpathlineto{\pgfqpoint{4.598883in}{1.282960in}}%
\pgfpathlineto{\pgfqpoint{4.584199in}{1.281374in}}%
\pgfpathlineto{\pgfqpoint{4.569526in}{1.279858in}}%
\pgfpathlineto{\pgfqpoint{4.561291in}{1.264978in}}%
\pgfpathlineto{\pgfqpoint{4.553052in}{1.250139in}}%
\pgfpathlineto{\pgfqpoint{4.544809in}{1.235346in}}%
\pgfpathlineto{\pgfqpoint{4.536562in}{1.220604in}}%
\pgfpathclose%
\pgfusepath{fill}%
\end{pgfscope}%
\begin{pgfscope}%
\pgfpathrectangle{\pgfqpoint{1.150000in}{0.150000in}}{\pgfqpoint{5.700000in}{5.700000in}}%
\pgfusepath{clip}%
\pgfsetbuttcap%
\pgfsetroundjoin%
\definecolor{currentfill}{rgb}{0.239346,0.300855,0.540844}%
\pgfsetfillcolor{currentfill}%
\pgfsetfillopacity{0.700000}%
\pgfsetlinewidth{0.000000pt}%
\definecolor{currentstroke}{rgb}{0.000000,0.000000,0.000000}%
\pgfsetstrokecolor{currentstroke}%
\pgfsetdash{}{0pt}%
\pgfpathmoveto{\pgfqpoint{4.844860in}{1.494789in}}%
\pgfpathlineto{\pgfqpoint{4.859677in}{1.498170in}}%
\pgfpathlineto{\pgfqpoint{4.874506in}{1.501621in}}%
\pgfpathlineto{\pgfqpoint{4.889347in}{1.505142in}}%
\pgfpathlineto{\pgfqpoint{4.897530in}{1.521243in}}%
\pgfpathlineto{\pgfqpoint{4.905709in}{1.537296in}}%
\pgfpathlineto{\pgfqpoint{4.913882in}{1.553296in}}%
\pgfpathlineto{\pgfqpoint{4.922051in}{1.569238in}}%
\pgfpathlineto{\pgfqpoint{4.907207in}{1.565413in}}%
\pgfpathlineto{\pgfqpoint{4.892376in}{1.561658in}}%
\pgfpathlineto{\pgfqpoint{4.877557in}{1.557973in}}%
\pgfpathlineto{\pgfqpoint{4.869390in}{1.542253in}}%
\pgfpathlineto{\pgfqpoint{4.861218in}{1.526479in}}%
\pgfpathlineto{\pgfqpoint{4.853042in}{1.510656in}}%
\pgfpathlineto{\pgfqpoint{4.844860in}{1.494789in}}%
\pgfpathclose%
\pgfusepath{fill}%
\end{pgfscope}%
\begin{pgfscope}%
\pgfpathrectangle{\pgfqpoint{1.150000in}{0.150000in}}{\pgfqpoint{5.700000in}{5.700000in}}%
\pgfusepath{clip}%
\pgfsetbuttcap%
\pgfsetroundjoin%
\definecolor{currentfill}{rgb}{0.270595,0.214069,0.507052}%
\pgfsetfillcolor{currentfill}%
\pgfsetfillopacity{0.700000}%
\pgfsetlinewidth{0.000000pt}%
\definecolor{currentstroke}{rgb}{0.000000,0.000000,0.000000}%
\pgfsetstrokecolor{currentstroke}%
\pgfsetdash{}{0pt}%
\pgfpathmoveto{\pgfqpoint{4.628283in}{1.286337in}}%
\pgfpathlineto{\pgfqpoint{4.643001in}{1.288129in}}%
\pgfpathlineto{\pgfqpoint{4.657729in}{1.289990in}}%
\pgfpathlineto{\pgfqpoint{4.672469in}{1.291920in}}%
\pgfpathlineto{\pgfqpoint{4.687221in}{1.293919in}}%
\pgfpathlineto{\pgfqpoint{4.695452in}{1.309577in}}%
\pgfpathlineto{\pgfqpoint{4.703678in}{1.325255in}}%
\pgfpathlineto{\pgfqpoint{4.711901in}{1.340947in}}%
\pgfpathlineto{\pgfqpoint{4.720120in}{1.356648in}}%
\pgfpathlineto{\pgfqpoint{4.705368in}{1.354284in}}%
\pgfpathlineto{\pgfqpoint{4.690627in}{1.351989in}}%
\pgfpathlineto{\pgfqpoint{4.675898in}{1.349764in}}%
\pgfpathlineto{\pgfqpoint{4.661181in}{1.347607in}}%
\pgfpathlineto{\pgfqpoint{4.652963in}{1.332264in}}%
\pgfpathlineto{\pgfqpoint{4.644740in}{1.316934in}}%
\pgfpathlineto{\pgfqpoint{4.636514in}{1.301623in}}%
\pgfpathlineto{\pgfqpoint{4.628283in}{1.286337in}}%
\pgfpathclose%
\pgfusepath{fill}%
\end{pgfscope}%
\begin{pgfscope}%
\pgfpathrectangle{\pgfqpoint{1.150000in}{0.150000in}}{\pgfqpoint{5.700000in}{5.700000in}}%
\pgfusepath{clip}%
\pgfsetbuttcap%
\pgfsetroundjoin%
\definecolor{currentfill}{rgb}{0.260571,0.246922,0.522828}%
\pgfsetfillcolor{currentfill}%
\pgfsetfillopacity{0.700000}%
\pgfsetlinewidth{0.000000pt}%
\definecolor{currentstroke}{rgb}{0.000000,0.000000,0.000000}%
\pgfsetstrokecolor{currentstroke}%
\pgfsetdash{}{0pt}%
\pgfpathmoveto{\pgfqpoint{4.720120in}{1.356648in}}%
\pgfpathlineto{\pgfqpoint{4.734885in}{1.359081in}}%
\pgfpathlineto{\pgfqpoint{4.749661in}{1.361584in}}%
\pgfpathlineto{\pgfqpoint{4.764449in}{1.364156in}}%
\pgfpathlineto{\pgfqpoint{4.779249in}{1.366797in}}%
\pgfpathlineto{\pgfqpoint{4.787465in}{1.382854in}}%
\pgfpathlineto{\pgfqpoint{4.795677in}{1.398904in}}%
\pgfpathlineto{\pgfqpoint{4.803885in}{1.414943in}}%
\pgfpathlineto{\pgfqpoint{4.812089in}{1.430965in}}%
\pgfpathlineto{\pgfqpoint{4.797287in}{1.427978in}}%
\pgfpathlineto{\pgfqpoint{4.782497in}{1.425061in}}%
\pgfpathlineto{\pgfqpoint{4.767720in}{1.422213in}}%
\pgfpathlineto{\pgfqpoint{4.752954in}{1.419435in}}%
\pgfpathlineto{\pgfqpoint{4.744752in}{1.403751in}}%
\pgfpathlineto{\pgfqpoint{4.736546in}{1.388055in}}%
\pgfpathlineto{\pgfqpoint{4.728335in}{1.372352in}}%
\pgfpathlineto{\pgfqpoint{4.720120in}{1.356648in}}%
\pgfpathclose%
\pgfusepath{fill}%
\end{pgfscope}%
\begin{pgfscope}%
\pgfpathrectangle{\pgfqpoint{1.150000in}{0.150000in}}{\pgfqpoint{5.700000in}{5.700000in}}%
\pgfusepath{clip}%
\pgfsetbuttcap%
\pgfsetroundjoin%
\definecolor{currentfill}{rgb}{0.283072,0.130895,0.449241}%
\pgfsetfillcolor{currentfill}%
\pgfsetfillopacity{0.700000}%
\pgfsetlinewidth{0.000000pt}%
\definecolor{currentstroke}{rgb}{0.000000,0.000000,0.000000}%
\pgfsetstrokecolor{currentstroke}%
\pgfsetdash{}{0pt}%
\pgfpathmoveto{\pgfqpoint{4.411833in}{1.104707in}}%
\pgfpathlineto{\pgfqpoint{4.426470in}{1.104738in}}%
\pgfpathlineto{\pgfqpoint{4.441117in}{1.104837in}}%
\pgfpathlineto{\pgfqpoint{4.455774in}{1.105003in}}%
\pgfpathlineto{\pgfqpoint{4.470442in}{1.105238in}}%
\pgfpathlineto{\pgfqpoint{4.478721in}{1.119353in}}%
\pgfpathlineto{\pgfqpoint{4.486996in}{1.133567in}}%
\pgfpathlineto{\pgfqpoint{4.495267in}{1.147876in}}%
\pgfpathlineto{\pgfqpoint{4.503534in}{1.162271in}}%
\pgfpathlineto{\pgfqpoint{4.488869in}{1.161611in}}%
\pgfpathlineto{\pgfqpoint{4.474214in}{1.161020in}}%
\pgfpathlineto{\pgfqpoint{4.459570in}{1.160497in}}%
\pgfpathlineto{\pgfqpoint{4.444937in}{1.160043in}}%
\pgfpathlineto{\pgfqpoint{4.436667in}{1.146064in}}%
\pgfpathlineto{\pgfqpoint{4.428393in}{1.132177in}}%
\pgfpathlineto{\pgfqpoint{4.420115in}{1.118390in}}%
\pgfpathlineto{\pgfqpoint{4.411833in}{1.104707in}}%
\pgfpathclose%
\pgfusepath{fill}%
\end{pgfscope}%
\begin{pgfscope}%
\pgfpathrectangle{\pgfqpoint{1.150000in}{0.150000in}}{\pgfqpoint{5.700000in}{5.700000in}}%
\pgfusepath{clip}%
\pgfsetbuttcap%
\pgfsetroundjoin%
\definecolor{currentfill}{rgb}{0.280868,0.160771,0.472899}%
\pgfsetfillcolor{currentfill}%
\pgfsetfillopacity{0.700000}%
\pgfsetlinewidth{0.000000pt}%
\definecolor{currentstroke}{rgb}{0.000000,0.000000,0.000000}%
\pgfsetstrokecolor{currentstroke}%
\pgfsetdash{}{0pt}%
\pgfpathmoveto{\pgfqpoint{4.503534in}{1.162271in}}%
\pgfpathlineto{\pgfqpoint{4.518209in}{1.162999in}}%
\pgfpathlineto{\pgfqpoint{4.532895in}{1.163795in}}%
\pgfpathlineto{\pgfqpoint{4.547591in}{1.164660in}}%
\pgfpathlineto{\pgfqpoint{4.562299in}{1.165592in}}%
\pgfpathlineto{\pgfqpoint{4.570560in}{1.180482in}}%
\pgfpathlineto{\pgfqpoint{4.578818in}{1.195441in}}%
\pgfpathlineto{\pgfqpoint{4.587072in}{1.210464in}}%
\pgfpathlineto{\pgfqpoint{4.595322in}{1.225546in}}%
\pgfpathlineto{\pgfqpoint{4.580616in}{1.224208in}}%
\pgfpathlineto{\pgfqpoint{4.565920in}{1.222938in}}%
\pgfpathlineto{\pgfqpoint{4.551236in}{1.221737in}}%
\pgfpathlineto{\pgfqpoint{4.536562in}{1.220604in}}%
\pgfpathlineto{\pgfqpoint{4.528311in}{1.205920in}}%
\pgfpathlineto{\pgfqpoint{4.520056in}{1.191299in}}%
\pgfpathlineto{\pgfqpoint{4.511797in}{1.176748in}}%
\pgfpathlineto{\pgfqpoint{4.503534in}{1.162271in}}%
\pgfpathclose%
\pgfusepath{fill}%
\end{pgfscope}%
\begin{pgfscope}%
\pgfpathrectangle{\pgfqpoint{1.150000in}{0.150000in}}{\pgfqpoint{5.700000in}{5.700000in}}%
\pgfusepath{clip}%
\pgfsetbuttcap%
\pgfsetroundjoin%
\definecolor{currentfill}{rgb}{0.246811,0.283237,0.535941}%
\pgfsetfillcolor{currentfill}%
\pgfsetfillopacity{0.700000}%
\pgfsetlinewidth{0.000000pt}%
\definecolor{currentstroke}{rgb}{0.000000,0.000000,0.000000}%
\pgfsetstrokecolor{currentstroke}%
\pgfsetdash{}{0pt}%
\pgfpathmoveto{\pgfqpoint{4.812089in}{1.430965in}}%
\pgfpathlineto{\pgfqpoint{4.826903in}{1.434021in}}%
\pgfpathlineto{\pgfqpoint{4.841730in}{1.437147in}}%
\pgfpathlineto{\pgfqpoint{4.856569in}{1.440343in}}%
\pgfpathlineto{\pgfqpoint{4.864770in}{1.456593in}}%
\pgfpathlineto{\pgfqpoint{4.872967in}{1.472812in}}%
\pgfpathlineto{\pgfqpoint{4.881159in}{1.488997in}}%
\pgfpathlineto{\pgfqpoint{4.889347in}{1.505142in}}%
\pgfpathlineto{\pgfqpoint{4.874506in}{1.501621in}}%
\pgfpathlineto{\pgfqpoint{4.859677in}{1.498170in}}%
\pgfpathlineto{\pgfqpoint{4.844860in}{1.494789in}}%
\pgfpathlineto{\pgfqpoint{4.836674in}{1.478881in}}%
\pgfpathlineto{\pgfqpoint{4.828484in}{1.462939in}}%
\pgfpathlineto{\pgfqpoint{4.820289in}{1.446965in}}%
\pgfpathlineto{\pgfqpoint{4.812089in}{1.430965in}}%
\pgfpathclose%
\pgfusepath{fill}%
\end{pgfscope}%
\begin{pgfscope}%
\pgfpathrectangle{\pgfqpoint{1.150000in}{0.150000in}}{\pgfqpoint{5.700000in}{5.700000in}}%
\pgfusepath{clip}%
\pgfsetbuttcap%
\pgfsetroundjoin%
\definecolor{currentfill}{rgb}{0.275191,0.194905,0.496005}%
\pgfsetfillcolor{currentfill}%
\pgfsetfillopacity{0.700000}%
\pgfsetlinewidth{0.000000pt}%
\definecolor{currentstroke}{rgb}{0.000000,0.000000,0.000000}%
\pgfsetstrokecolor{currentstroke}%
\pgfsetdash{}{0pt}%
\pgfpathmoveto{\pgfqpoint{4.595322in}{1.225546in}}%
\pgfpathlineto{\pgfqpoint{4.610039in}{1.226953in}}%
\pgfpathlineto{\pgfqpoint{4.624768in}{1.228428in}}%
\pgfpathlineto{\pgfqpoint{4.639508in}{1.229972in}}%
\pgfpathlineto{\pgfqpoint{4.654259in}{1.231585in}}%
\pgfpathlineto{\pgfqpoint{4.662505in}{1.247112in}}%
\pgfpathlineto{\pgfqpoint{4.670747in}{1.262680in}}%
\pgfpathlineto{\pgfqpoint{4.678986in}{1.278284in}}%
\pgfpathlineto{\pgfqpoint{4.687221in}{1.293919in}}%
\pgfpathlineto{\pgfqpoint{4.672469in}{1.291920in}}%
\pgfpathlineto{\pgfqpoint{4.657729in}{1.289990in}}%
\pgfpathlineto{\pgfqpoint{4.643001in}{1.288129in}}%
\pgfpathlineto{\pgfqpoint{4.628283in}{1.286337in}}%
\pgfpathlineto{\pgfqpoint{4.620049in}{1.271081in}}%
\pgfpathlineto{\pgfqpoint{4.611810in}{1.255860in}}%
\pgfpathlineto{\pgfqpoint{4.603568in}{1.240680in}}%
\pgfpathlineto{\pgfqpoint{4.595322in}{1.225546in}}%
\pgfpathclose%
\pgfusepath{fill}%
\end{pgfscope}%
\begin{pgfscope}%
\pgfpathrectangle{\pgfqpoint{1.150000in}{0.150000in}}{\pgfqpoint{5.700000in}{5.700000in}}%
\pgfusepath{clip}%
\pgfsetbuttcap%
\pgfsetroundjoin%
\definecolor{currentfill}{rgb}{0.266580,0.228262,0.514349}%
\pgfsetfillcolor{currentfill}%
\pgfsetfillopacity{0.700000}%
\pgfsetlinewidth{0.000000pt}%
\definecolor{currentstroke}{rgb}{0.000000,0.000000,0.000000}%
\pgfsetstrokecolor{currentstroke}%
\pgfsetdash{}{0pt}%
\pgfpathmoveto{\pgfqpoint{4.687221in}{1.293919in}}%
\pgfpathlineto{\pgfqpoint{4.701984in}{1.295986in}}%
\pgfpathlineto{\pgfqpoint{4.716758in}{1.298123in}}%
\pgfpathlineto{\pgfqpoint{4.731545in}{1.300328in}}%
\pgfpathlineto{\pgfqpoint{4.746343in}{1.302603in}}%
\pgfpathlineto{\pgfqpoint{4.754575in}{1.318636in}}%
\pgfpathlineto{\pgfqpoint{4.762804in}{1.334682in}}%
\pgfpathlineto{\pgfqpoint{4.771028in}{1.350738in}}%
\pgfpathlineto{\pgfqpoint{4.779249in}{1.366797in}}%
\pgfpathlineto{\pgfqpoint{4.764449in}{1.364156in}}%
\pgfpathlineto{\pgfqpoint{4.749661in}{1.361584in}}%
\pgfpathlineto{\pgfqpoint{4.734885in}{1.359081in}}%
\pgfpathlineto{\pgfqpoint{4.720120in}{1.356648in}}%
\pgfpathlineto{\pgfqpoint{4.711901in}{1.340947in}}%
\pgfpathlineto{\pgfqpoint{4.703678in}{1.325255in}}%
\pgfpathlineto{\pgfqpoint{4.695452in}{1.309577in}}%
\pgfpathlineto{\pgfqpoint{4.687221in}{1.293919in}}%
\pgfpathclose%
\pgfusepath{fill}%
\end{pgfscope}%
\begin{pgfscope}%
\pgfpathrectangle{\pgfqpoint{1.150000in}{0.150000in}}{\pgfqpoint{5.700000in}{5.700000in}}%
\pgfusepath{clip}%
\pgfsetbuttcap%
\pgfsetroundjoin%
\definecolor{currentfill}{rgb}{0.253935,0.265254,0.529983}%
\pgfsetfillcolor{currentfill}%
\pgfsetfillopacity{0.700000}%
\pgfsetlinewidth{0.000000pt}%
\definecolor{currentstroke}{rgb}{0.000000,0.000000,0.000000}%
\pgfsetstrokecolor{currentstroke}%
\pgfsetdash{}{0pt}%
\pgfpathmoveto{\pgfqpoint{4.779249in}{1.366797in}}%
\pgfpathlineto{\pgfqpoint{4.794061in}{1.369507in}}%
\pgfpathlineto{\pgfqpoint{4.808885in}{1.372287in}}%
\pgfpathlineto{\pgfqpoint{4.823721in}{1.375136in}}%
\pgfpathlineto{\pgfqpoint{4.831939in}{1.391459in}}%
\pgfpathlineto{\pgfqpoint{4.840153in}{1.407771in}}%
\pgfpathlineto{\pgfqpoint{4.848363in}{1.424067in}}%
\pgfpathlineto{\pgfqpoint{4.856569in}{1.440343in}}%
\pgfpathlineto{\pgfqpoint{4.841730in}{1.437147in}}%
\pgfpathlineto{\pgfqpoint{4.826903in}{1.434021in}}%
\pgfpathlineto{\pgfqpoint{4.812089in}{1.430965in}}%
\pgfpathlineto{\pgfqpoint{4.803885in}{1.414943in}}%
\pgfpathlineto{\pgfqpoint{4.795677in}{1.398904in}}%
\pgfpathlineto{\pgfqpoint{4.787465in}{1.382854in}}%
\pgfpathlineto{\pgfqpoint{4.779249in}{1.366797in}}%
\pgfpathclose%
\pgfusepath{fill}%
\end{pgfscope}%
\begin{pgfscope}%
\pgfpathrectangle{\pgfqpoint{1.150000in}{0.150000in}}{\pgfqpoint{5.700000in}{5.700000in}}%
\pgfusepath{clip}%
\pgfsetbuttcap%
\pgfsetroundjoin%
\definecolor{currentfill}{rgb}{0.282290,0.145912,0.461510}%
\pgfsetfillcolor{currentfill}%
\pgfsetfillopacity{0.700000}%
\pgfsetlinewidth{0.000000pt}%
\definecolor{currentstroke}{rgb}{0.000000,0.000000,0.000000}%
\pgfsetstrokecolor{currentstroke}%
\pgfsetdash{}{0pt}%
\pgfpathmoveto{\pgfqpoint{4.470442in}{1.105238in}}%
\pgfpathlineto{\pgfqpoint{4.485120in}{1.105541in}}%
\pgfpathlineto{\pgfqpoint{4.499808in}{1.105912in}}%
\pgfpathlineto{\pgfqpoint{4.514506in}{1.106350in}}%
\pgfpathlineto{\pgfqpoint{4.529215in}{1.106857in}}%
\pgfpathlineto{\pgfqpoint{4.537492in}{1.121405in}}%
\pgfpathlineto{\pgfqpoint{4.545764in}{1.136048in}}%
\pgfpathlineto{\pgfqpoint{4.554034in}{1.150779in}}%
\pgfpathlineto{\pgfqpoint{4.562299in}{1.165592in}}%
\pgfpathlineto{\pgfqpoint{4.547591in}{1.164660in}}%
\pgfpathlineto{\pgfqpoint{4.532895in}{1.163795in}}%
\pgfpathlineto{\pgfqpoint{4.518209in}{1.162999in}}%
\pgfpathlineto{\pgfqpoint{4.503534in}{1.162271in}}%
\pgfpathlineto{\pgfqpoint{4.495267in}{1.147876in}}%
\pgfpathlineto{\pgfqpoint{4.486996in}{1.133567in}}%
\pgfpathlineto{\pgfqpoint{4.478721in}{1.119353in}}%
\pgfpathlineto{\pgfqpoint{4.470442in}{1.105238in}}%
\pgfpathclose%
\pgfusepath{fill}%
\end{pgfscope}%
\begin{pgfscope}%
\pgfpathrectangle{\pgfqpoint{1.150000in}{0.150000in}}{\pgfqpoint{5.700000in}{5.700000in}}%
\pgfusepath{clip}%
\pgfsetbuttcap%
\pgfsetroundjoin%
\definecolor{currentfill}{rgb}{0.278826,0.175490,0.483397}%
\pgfsetfillcolor{currentfill}%
\pgfsetfillopacity{0.700000}%
\pgfsetlinewidth{0.000000pt}%
\definecolor{currentstroke}{rgb}{0.000000,0.000000,0.000000}%
\pgfsetstrokecolor{currentstroke}%
\pgfsetdash{}{0pt}%
\pgfpathmoveto{\pgfqpoint{4.562299in}{1.165592in}}%
\pgfpathlineto{\pgfqpoint{4.577017in}{1.166594in}}%
\pgfpathlineto{\pgfqpoint{4.591746in}{1.167663in}}%
\pgfpathlineto{\pgfqpoint{4.606486in}{1.168801in}}%
\pgfpathlineto{\pgfqpoint{4.621237in}{1.170007in}}%
\pgfpathlineto{\pgfqpoint{4.629498in}{1.185310in}}%
\pgfpathlineto{\pgfqpoint{4.637755in}{1.200678in}}%
\pgfpathlineto{\pgfqpoint{4.646009in}{1.216105in}}%
\pgfpathlineto{\pgfqpoint{4.654259in}{1.231585in}}%
\pgfpathlineto{\pgfqpoint{4.639508in}{1.229972in}}%
\pgfpathlineto{\pgfqpoint{4.624768in}{1.228428in}}%
\pgfpathlineto{\pgfqpoint{4.610039in}{1.226953in}}%
\pgfpathlineto{\pgfqpoint{4.595322in}{1.225546in}}%
\pgfpathlineto{\pgfqpoint{4.587072in}{1.210464in}}%
\pgfpathlineto{\pgfqpoint{4.578818in}{1.195441in}}%
\pgfpathlineto{\pgfqpoint{4.570560in}{1.180482in}}%
\pgfpathlineto{\pgfqpoint{4.562299in}{1.165592in}}%
\pgfpathclose%
\pgfusepath{fill}%
\end{pgfscope}%
\begin{pgfscope}%
\pgfpathrectangle{\pgfqpoint{1.150000in}{0.150000in}}{\pgfqpoint{5.700000in}{5.700000in}}%
\pgfusepath{clip}%
\pgfsetbuttcap%
\pgfsetroundjoin%
\definecolor{currentfill}{rgb}{0.271828,0.209303,0.504434}%
\pgfsetfillcolor{currentfill}%
\pgfsetfillopacity{0.700000}%
\pgfsetlinewidth{0.000000pt}%
\definecolor{currentstroke}{rgb}{0.000000,0.000000,0.000000}%
\pgfsetstrokecolor{currentstroke}%
\pgfsetdash{}{0pt}%
\pgfpathmoveto{\pgfqpoint{4.654259in}{1.231585in}}%
\pgfpathlineto{\pgfqpoint{4.669021in}{1.233266in}}%
\pgfpathlineto{\pgfqpoint{4.683795in}{1.235016in}}%
\pgfpathlineto{\pgfqpoint{4.698580in}{1.236835in}}%
\pgfpathlineto{\pgfqpoint{4.713376in}{1.238722in}}%
\pgfpathlineto{\pgfqpoint{4.721624in}{1.254644in}}%
\pgfpathlineto{\pgfqpoint{4.729867in}{1.270601in}}%
\pgfpathlineto{\pgfqpoint{4.738107in}{1.286590in}}%
\pgfpathlineto{\pgfqpoint{4.746343in}{1.302603in}}%
\pgfpathlineto{\pgfqpoint{4.731545in}{1.300328in}}%
\pgfpathlineto{\pgfqpoint{4.716758in}{1.298123in}}%
\pgfpathlineto{\pgfqpoint{4.701984in}{1.295986in}}%
\pgfpathlineto{\pgfqpoint{4.687221in}{1.293919in}}%
\pgfpathlineto{\pgfqpoint{4.678986in}{1.278284in}}%
\pgfpathlineto{\pgfqpoint{4.670747in}{1.262680in}}%
\pgfpathlineto{\pgfqpoint{4.662505in}{1.247112in}}%
\pgfpathlineto{\pgfqpoint{4.654259in}{1.231585in}}%
\pgfpathclose%
\pgfusepath{fill}%
\end{pgfscope}%
\begin{pgfscope}%
\pgfpathrectangle{\pgfqpoint{1.150000in}{0.150000in}}{\pgfqpoint{5.700000in}{5.700000in}}%
\pgfusepath{clip}%
\pgfsetbuttcap%
\pgfsetroundjoin%
\definecolor{currentfill}{rgb}{0.260571,0.246922,0.522828}%
\pgfsetfillcolor{currentfill}%
\pgfsetfillopacity{0.700000}%
\pgfsetlinewidth{0.000000pt}%
\definecolor{currentstroke}{rgb}{0.000000,0.000000,0.000000}%
\pgfsetstrokecolor{currentstroke}%
\pgfsetdash{}{0pt}%
\pgfpathmoveto{\pgfqpoint{4.746343in}{1.302603in}}%
\pgfpathlineto{\pgfqpoint{4.761153in}{1.304947in}}%
\pgfpathlineto{\pgfqpoint{4.775975in}{1.307359in}}%
\pgfpathlineto{\pgfqpoint{4.790809in}{1.309841in}}%
\pgfpathlineto{\pgfqpoint{4.799043in}{1.326155in}}%
\pgfpathlineto{\pgfqpoint{4.807273in}{1.342479in}}%
\pgfpathlineto{\pgfqpoint{4.815499in}{1.358808in}}%
\pgfpathlineto{\pgfqpoint{4.823721in}{1.375136in}}%
\pgfpathlineto{\pgfqpoint{4.808885in}{1.372287in}}%
\pgfpathlineto{\pgfqpoint{4.794061in}{1.369507in}}%
\pgfpathlineto{\pgfqpoint{4.779249in}{1.366797in}}%
\pgfpathlineto{\pgfqpoint{4.771028in}{1.350738in}}%
\pgfpathlineto{\pgfqpoint{4.762804in}{1.334682in}}%
\pgfpathlineto{\pgfqpoint{4.754575in}{1.318636in}}%
\pgfpathlineto{\pgfqpoint{4.746343in}{1.302603in}}%
\pgfpathclose%
\pgfusepath{fill}%
\end{pgfscope}%
\begin{pgfscope}%
\pgfpathrectangle{\pgfqpoint{1.150000in}{0.150000in}}{\pgfqpoint{5.700000in}{5.700000in}}%
\pgfusepath{clip}%
\pgfsetbuttcap%
\pgfsetroundjoin%
\definecolor{currentfill}{rgb}{0.280868,0.160771,0.472899}%
\pgfsetfillcolor{currentfill}%
\pgfsetfillopacity{0.700000}%
\pgfsetlinewidth{0.000000pt}%
\definecolor{currentstroke}{rgb}{0.000000,0.000000,0.000000}%
\pgfsetstrokecolor{currentstroke}%
\pgfsetdash{}{0pt}%
\pgfpathmoveto{\pgfqpoint{4.529215in}{1.106857in}}%
\pgfpathlineto{\pgfqpoint{4.543935in}{1.107432in}}%
\pgfpathlineto{\pgfqpoint{4.558665in}{1.108075in}}%
\pgfpathlineto{\pgfqpoint{4.573405in}{1.108785in}}%
\pgfpathlineto{\pgfqpoint{4.588157in}{1.109564in}}%
\pgfpathlineto{\pgfqpoint{4.596432in}{1.124547in}}%
\pgfpathlineto{\pgfqpoint{4.604704in}{1.139619in}}%
\pgfpathlineto{\pgfqpoint{4.612972in}{1.154774in}}%
\pgfpathlineto{\pgfqpoint{4.621237in}{1.170007in}}%
\pgfpathlineto{\pgfqpoint{4.606486in}{1.168801in}}%
\pgfpathlineto{\pgfqpoint{4.591746in}{1.167663in}}%
\pgfpathlineto{\pgfqpoint{4.577017in}{1.166594in}}%
\pgfpathlineto{\pgfqpoint{4.562299in}{1.165592in}}%
\pgfpathlineto{\pgfqpoint{4.554034in}{1.150779in}}%
\pgfpathlineto{\pgfqpoint{4.545764in}{1.136048in}}%
\pgfpathlineto{\pgfqpoint{4.537492in}{1.121405in}}%
\pgfpathlineto{\pgfqpoint{4.529215in}{1.106857in}}%
\pgfpathclose%
\pgfusepath{fill}%
\end{pgfscope}%
\begin{pgfscope}%
\pgfpathrectangle{\pgfqpoint{1.150000in}{0.150000in}}{\pgfqpoint{5.700000in}{5.700000in}}%
\pgfusepath{clip}%
\pgfsetbuttcap%
\pgfsetroundjoin%
\definecolor{currentfill}{rgb}{0.275191,0.194905,0.496005}%
\pgfsetfillcolor{currentfill}%
\pgfsetfillopacity{0.700000}%
\pgfsetlinewidth{0.000000pt}%
\definecolor{currentstroke}{rgb}{0.000000,0.000000,0.000000}%
\pgfsetstrokecolor{currentstroke}%
\pgfsetdash{}{0pt}%
\pgfpathmoveto{\pgfqpoint{4.621237in}{1.170007in}}%
\pgfpathlineto{\pgfqpoint{4.635999in}{1.171281in}}%
\pgfpathlineto{\pgfqpoint{4.650772in}{1.172623in}}%
\pgfpathlineto{\pgfqpoint{4.665556in}{1.174034in}}%
\pgfpathlineto{\pgfqpoint{4.680352in}{1.175513in}}%
\pgfpathlineto{\pgfqpoint{4.688613in}{1.191232in}}%
\pgfpathlineto{\pgfqpoint{4.696871in}{1.207010in}}%
\pgfpathlineto{\pgfqpoint{4.705126in}{1.222842in}}%
\pgfpathlineto{\pgfqpoint{4.713376in}{1.238722in}}%
\pgfpathlineto{\pgfqpoint{4.698580in}{1.236835in}}%
\pgfpathlineto{\pgfqpoint{4.683795in}{1.235016in}}%
\pgfpathlineto{\pgfqpoint{4.669021in}{1.233266in}}%
\pgfpathlineto{\pgfqpoint{4.654259in}{1.231585in}}%
\pgfpathlineto{\pgfqpoint{4.646009in}{1.216105in}}%
\pgfpathlineto{\pgfqpoint{4.637755in}{1.200678in}}%
\pgfpathlineto{\pgfqpoint{4.629498in}{1.185310in}}%
\pgfpathlineto{\pgfqpoint{4.621237in}{1.170007in}}%
\pgfpathclose%
\pgfusepath{fill}%
\end{pgfscope}%
\begin{pgfscope}%
\pgfpathrectangle{\pgfqpoint{1.150000in}{0.150000in}}{\pgfqpoint{5.700000in}{5.700000in}}%
\pgfusepath{clip}%
\pgfsetbuttcap%
\pgfsetroundjoin%
\definecolor{currentfill}{rgb}{0.267968,0.223549,0.512008}%
\pgfsetfillcolor{currentfill}%
\pgfsetfillopacity{0.700000}%
\pgfsetlinewidth{0.000000pt}%
\definecolor{currentstroke}{rgb}{0.000000,0.000000,0.000000}%
\pgfsetstrokecolor{currentstroke}%
\pgfsetdash{}{0pt}%
\pgfpathmoveto{\pgfqpoint{4.713376in}{1.238722in}}%
\pgfpathlineto{\pgfqpoint{4.728185in}{1.240678in}}%
\pgfpathlineto{\pgfqpoint{4.743005in}{1.242702in}}%
\pgfpathlineto{\pgfqpoint{4.757836in}{1.244796in}}%
\pgfpathlineto{\pgfqpoint{4.766085in}{1.261014in}}%
\pgfpathlineto{\pgfqpoint{4.774330in}{1.277265in}}%
\pgfpathlineto{\pgfqpoint{4.782571in}{1.293542in}}%
\pgfpathlineto{\pgfqpoint{4.790809in}{1.309841in}}%
\pgfpathlineto{\pgfqpoint{4.775975in}{1.307359in}}%
\pgfpathlineto{\pgfqpoint{4.761153in}{1.304947in}}%
\pgfpathlineto{\pgfqpoint{4.746343in}{1.302603in}}%
\pgfpathlineto{\pgfqpoint{4.738107in}{1.286590in}}%
\pgfpathlineto{\pgfqpoint{4.729867in}{1.270601in}}%
\pgfpathlineto{\pgfqpoint{4.721624in}{1.254644in}}%
\pgfpathlineto{\pgfqpoint{4.713376in}{1.238722in}}%
\pgfpathclose%
\pgfusepath{fill}%
\end{pgfscope}%
\begin{pgfscope}%
\pgfpathrectangle{\pgfqpoint{1.150000in}{0.150000in}}{\pgfqpoint{5.700000in}{5.700000in}}%
\pgfusepath{clip}%
\pgfsetbuttcap%
\pgfsetroundjoin%
\definecolor{currentfill}{rgb}{0.278826,0.175490,0.483397}%
\pgfsetfillcolor{currentfill}%
\pgfsetfillopacity{0.700000}%
\pgfsetlinewidth{0.000000pt}%
\definecolor{currentstroke}{rgb}{0.000000,0.000000,0.000000}%
\pgfsetstrokecolor{currentstroke}%
\pgfsetdash{}{0pt}%
\pgfpathmoveto{\pgfqpoint{4.588157in}{1.109564in}}%
\pgfpathlineto{\pgfqpoint{4.602919in}{1.110411in}}%
\pgfpathlineto{\pgfqpoint{4.617692in}{1.111325in}}%
\pgfpathlineto{\pgfqpoint{4.632477in}{1.112308in}}%
\pgfpathlineto{\pgfqpoint{4.647272in}{1.113359in}}%
\pgfpathlineto{\pgfqpoint{4.655547in}{1.128777in}}%
\pgfpathlineto{\pgfqpoint{4.663818in}{1.144280in}}%
\pgfpathlineto{\pgfqpoint{4.672087in}{1.159861in}}%
\pgfpathlineto{\pgfqpoint{4.680352in}{1.175513in}}%
\pgfpathlineto{\pgfqpoint{4.665556in}{1.174034in}}%
\pgfpathlineto{\pgfqpoint{4.650772in}{1.172623in}}%
\pgfpathlineto{\pgfqpoint{4.635999in}{1.171281in}}%
\pgfpathlineto{\pgfqpoint{4.621237in}{1.170007in}}%
\pgfpathlineto{\pgfqpoint{4.612972in}{1.154774in}}%
\pgfpathlineto{\pgfqpoint{4.604704in}{1.139619in}}%
\pgfpathlineto{\pgfqpoint{4.596432in}{1.124547in}}%
\pgfpathlineto{\pgfqpoint{4.588157in}{1.109564in}}%
\pgfpathclose%
\pgfusepath{fill}%
\end{pgfscope}%
\begin{pgfscope}%
\pgfpathrectangle{\pgfqpoint{1.150000in}{0.150000in}}{\pgfqpoint{5.700000in}{5.700000in}}%
\pgfusepath{clip}%
\pgfsetbuttcap%
\pgfsetroundjoin%
\definecolor{currentfill}{rgb}{0.273006,0.204520,0.501721}%
\pgfsetfillcolor{currentfill}%
\pgfsetfillopacity{0.700000}%
\pgfsetlinewidth{0.000000pt}%
\definecolor{currentstroke}{rgb}{0.000000,0.000000,0.000000}%
\pgfsetstrokecolor{currentstroke}%
\pgfsetdash{}{0pt}%
\pgfpathmoveto{\pgfqpoint{4.680352in}{1.175513in}}%
\pgfpathlineto{\pgfqpoint{4.695159in}{1.177061in}}%
\pgfpathlineto{\pgfqpoint{4.709977in}{1.178677in}}%
\pgfpathlineto{\pgfqpoint{4.724807in}{1.180361in}}%
\pgfpathlineto{\pgfqpoint{4.733069in}{1.196392in}}%
\pgfpathlineto{\pgfqpoint{4.741329in}{1.212479in}}%
\pgfpathlineto{\pgfqpoint{4.749584in}{1.228615in}}%
\pgfpathlineto{\pgfqpoint{4.757836in}{1.244796in}}%
\pgfpathlineto{\pgfqpoint{4.743005in}{1.242702in}}%
\pgfpathlineto{\pgfqpoint{4.728185in}{1.240678in}}%
\pgfpathlineto{\pgfqpoint{4.713376in}{1.238722in}}%
\pgfpathlineto{\pgfqpoint{4.705126in}{1.222842in}}%
\pgfpathlineto{\pgfqpoint{4.696871in}{1.207010in}}%
\pgfpathlineto{\pgfqpoint{4.688613in}{1.191232in}}%
\pgfpathlineto{\pgfqpoint{4.680352in}{1.175513in}}%
\pgfpathclose%
\pgfusepath{fill}%
\end{pgfscope}%
\begin{pgfscope}%
\pgfpathrectangle{\pgfqpoint{1.150000in}{0.150000in}}{\pgfqpoint{5.700000in}{5.700000in}}%
\pgfusepath{clip}%
\pgfsetbuttcap%
\pgfsetroundjoin%
\definecolor{currentfill}{rgb}{0.277134,0.185228,0.489898}%
\pgfsetfillcolor{currentfill}%
\pgfsetfillopacity{0.700000}%
\pgfsetlinewidth{0.000000pt}%
\definecolor{currentstroke}{rgb}{0.000000,0.000000,0.000000}%
\pgfsetstrokecolor{currentstroke}%
\pgfsetdash{}{0pt}%
\pgfpathmoveto{\pgfqpoint{4.647272in}{1.113359in}}%
\pgfpathlineto{\pgfqpoint{4.662078in}{1.114477in}}%
\pgfpathlineto{\pgfqpoint{4.676895in}{1.115664in}}%
\pgfpathlineto{\pgfqpoint{4.691724in}{1.116919in}}%
\pgfpathlineto{\pgfqpoint{4.699999in}{1.132665in}}%
\pgfpathlineto{\pgfqpoint{4.708272in}{1.148491in}}%
\pgfpathlineto{\pgfqpoint{4.716541in}{1.164392in}}%
\pgfpathlineto{\pgfqpoint{4.724807in}{1.180361in}}%
\pgfpathlineto{\pgfqpoint{4.709977in}{1.178677in}}%
\pgfpathlineto{\pgfqpoint{4.695159in}{1.177061in}}%
\pgfpathlineto{\pgfqpoint{4.680352in}{1.175513in}}%
\pgfpathlineto{\pgfqpoint{4.672087in}{1.159861in}}%
\pgfpathlineto{\pgfqpoint{4.663818in}{1.144280in}}%
\pgfpathlineto{\pgfqpoint{4.655547in}{1.128777in}}%
\pgfpathlineto{\pgfqpoint{4.647272in}{1.113359in}}%
\pgfpathclose%
\pgfusepath{fill}%
\end{pgfscope}%
\end{pgfpicture}%
\makeatother%
\endgroup%
}
        \caption{3D graf funkcie}
        \label{fig:newton_vpravo}
    \end{subfigure}
    
    \label{fig:newton_komplet}
\end{figure}


\newpage
\noindent \textbf{Počiatočný bod} $x^{[0]} = [2; -2]$
\vspace{0.5cm}


\begin{table}[h!]
    \centering
    \begin{tabular}{ccccc}
        \toprule
        \textbf{Iterácia} &
        $\boldsymbol{x}$ &
        $\boldsymbol{y}$ &
        $\boldsymbol{f(x,y)}$ &
        $\boldsymbol{|f_k - f_{k-1}|}$ \\
        \midrule
        0 & $ 2{,}000000$ & $-2{,}000000$ & $6{,}270671$ & $3{,}623404$ \\
        1 & $ 1{,}950255$ & $-1{,}594805$ & $2{,}647267$ & $0{,}699685$ \\
        2 & $ 0{,}129873$ & $-0{,}931993$ & $1{,}947582$ & $0{,}370219$ \\
        3 & $-0{,}457773$ & $-0{,}821499$ & $1{,}577363$ & $0{,}008275$ \\
        4 & $ 0{,}392100$ & $-0{,}748534$ & $1{,}569088$ & $0{,}000092$ \\
        \bottomrule
    \end{tabular}
    \caption{Priebeh Newtonovej metódy pre $x^{[0]} = [2;\,-2]$.}
    \label{tab:newton_priebeh}
\end{table}

Aj napriek voľbe počiatočného bodu s veľkými súradnicami metóda skonvergovala už po štyroch iteráciách. Výrazný posun v prvej iterácii smerom k optimu potvrdzuje kvadratický charakter konvergencie, pričom ďalšie iterácie zabezpečili postupné spresňovanie riešenia na požadovanú presnosť.


\begin{figure}[H]
    \centering
    
    % --- ĽAVÝ OBRÁZOK ---
    \begin{subfigure}[b]{0.48\textwidth}
        \centering
        \resizebox{\linewidth}{!}{%% Creator: Matplotlib, PGF backend
%%
%% To include the figure in your LaTeX document, write
%%   \input{<filename>.pgf}
%%
%% Make sure the required packages are loaded in your preamble
%%   \usepackage{pgf}
%%
%% Also ensure that all the required font packages are loaded; for instance,
%% the lmodern package is sometimes necessary when using math font.
%%   \usepackage{lmodern}
%%
%% Figures using additional raster images can only be included by \input if
%% they are in the same directory as the main LaTeX file. For loading figures
%% from other directories you can use the `import` package
%%   \usepackage{import}
%%
%% and then include the figures with
%%   \import{<path to file>}{<filename>.pgf}
%%
%% Matplotlib used the following preamble
%%   
%%   \usepackage{fontspec}
%%   \setmainfont{DejaVuSerif.ttf}[Path=\detokenize{/home/radimek/Documents/projekt_mat_prog/mat_prog_kernel/lib/python3.12/site-packages/matplotlib/mpl-data/fonts/ttf/}]
%%   \setsansfont{DejaVuSans.ttf}[Path=\detokenize{/home/radimek/Documents/projekt_mat_prog/mat_prog_kernel/lib/python3.12/site-packages/matplotlib/mpl-data/fonts/ttf/}]
%%   \setmonofont{DejaVuSansMono.ttf}[Path=\detokenize{/home/radimek/Documents/projekt_mat_prog/mat_prog_kernel/lib/python3.12/site-packages/matplotlib/mpl-data/fonts/ttf/}]
%%   \makeatletter\@ifpackageloaded{underscore}{}{\usepackage[strings]{underscore}}\makeatother
%%
\begingroup%
\makeatletter%
\begin{pgfpicture}%
\pgfpathrectangle{\pgfpointorigin}{\pgfqpoint{8.000000in}{6.000000in}}%
\pgfusepath{use as bounding box, clip}%
\begin{pgfscope}%
\pgfsetbuttcap%
\pgfsetmiterjoin%
\definecolor{currentfill}{rgb}{1.000000,1.000000,1.000000}%
\pgfsetfillcolor{currentfill}%
\pgfsetlinewidth{0.000000pt}%
\definecolor{currentstroke}{rgb}{1.000000,1.000000,1.000000}%
\pgfsetstrokecolor{currentstroke}%
\pgfsetdash{}{0pt}%
\pgfpathmoveto{\pgfqpoint{0.000000in}{0.000000in}}%
\pgfpathlineto{\pgfqpoint{8.000000in}{0.000000in}}%
\pgfpathlineto{\pgfqpoint{8.000000in}{6.000000in}}%
\pgfpathlineto{\pgfqpoint{0.000000in}{6.000000in}}%
\pgfpathlineto{\pgfqpoint{0.000000in}{0.000000in}}%
\pgfpathclose%
\pgfusepath{fill}%
\end{pgfscope}%
\begin{pgfscope}%
\pgfsetbuttcap%
\pgfsetmiterjoin%
\definecolor{currentfill}{rgb}{1.000000,1.000000,1.000000}%
\pgfsetfillcolor{currentfill}%
\pgfsetlinewidth{0.000000pt}%
\definecolor{currentstroke}{rgb}{0.000000,0.000000,0.000000}%
\pgfsetstrokecolor{currentstroke}%
\pgfsetstrokeopacity{0.000000}%
\pgfsetdash{}{0pt}%
\pgfpathmoveto{\pgfqpoint{0.766095in}{0.571603in}}%
\pgfpathlineto{\pgfqpoint{7.739560in}{0.571603in}}%
\pgfpathlineto{\pgfqpoint{7.739560in}{5.797238in}}%
\pgfpathlineto{\pgfqpoint{0.766095in}{5.797238in}}%
\pgfpathlineto{\pgfqpoint{0.766095in}{0.571603in}}%
\pgfpathclose%
\pgfusepath{fill}%
\end{pgfscope}%
\begin{pgfscope}%
\pgfpathrectangle{\pgfqpoint{0.766095in}{0.571603in}}{\pgfqpoint{6.973465in}{5.225635in}}%
\pgfusepath{clip}%
\pgfsetbuttcap%
\pgfsetroundjoin%
\definecolor{currentfill}{rgb}{1.000000,0.000000,0.000000}%
\pgfsetfillcolor{currentfill}%
\pgfsetlinewidth{1.003750pt}%
\definecolor{currentstroke}{rgb}{1.000000,0.000000,0.000000}%
\pgfsetstrokecolor{currentstroke}%
\pgfsetdash{}{0pt}%
\pgfsys@defobject{currentmarker}{\pgfqpoint{-0.041667in}{-0.041667in}}{\pgfqpoint{0.041667in}{0.041667in}}{%
\pgfpathmoveto{\pgfqpoint{0.000000in}{-0.041667in}}%
\pgfpathcurveto{\pgfqpoint{0.011050in}{-0.041667in}}{\pgfqpoint{0.021649in}{-0.037276in}}{\pgfqpoint{0.029463in}{-0.029463in}}%
\pgfpathcurveto{\pgfqpoint{0.037276in}{-0.021649in}}{\pgfqpoint{0.041667in}{-0.011050in}}{\pgfqpoint{0.041667in}{0.000000in}}%
\pgfpathcurveto{\pgfqpoint{0.041667in}{0.011050in}}{\pgfqpoint{0.037276in}{0.021649in}}{\pgfqpoint{0.029463in}{0.029463in}}%
\pgfpathcurveto{\pgfqpoint{0.021649in}{0.037276in}}{\pgfqpoint{0.011050in}{0.041667in}}{\pgfqpoint{0.000000in}{0.041667in}}%
\pgfpathcurveto{\pgfqpoint{-0.011050in}{0.041667in}}{\pgfqpoint{-0.021649in}{0.037276in}}{\pgfqpoint{-0.029463in}{0.029463in}}%
\pgfpathcurveto{\pgfqpoint{-0.037276in}{0.021649in}}{\pgfqpoint{-0.041667in}{0.011050in}}{\pgfqpoint{-0.041667in}{0.000000in}}%
\pgfpathcurveto{\pgfqpoint{-0.041667in}{-0.011050in}}{\pgfqpoint{-0.037276in}{-0.021649in}}{\pgfqpoint{-0.029463in}{-0.029463in}}%
\pgfpathcurveto{\pgfqpoint{-0.021649in}{-0.037276in}}{\pgfqpoint{-0.011050in}{-0.041667in}}{\pgfqpoint{0.000000in}{-0.041667in}}%
\pgfpathlineto{\pgfqpoint{0.000000in}{-0.041667in}}%
\pgfpathclose%
\pgfusepath{stroke,fill}%
}%
\begin{pgfscope}%
\pgfsys@transformshift{6.743351in}{1.318123in}%
\pgfsys@useobject{currentmarker}{}%
\end{pgfscope}%
\begin{pgfscope}%
\pgfsys@transformshift{6.644238in}{1.923094in}%
\pgfsys@useobject{currentmarker}{}%
\end{pgfscope}%
\begin{pgfscope}%
\pgfsys@transformshift{3.017275in}{2.912699in}%
\pgfsys@useobject{currentmarker}{}%
\end{pgfscope}%
\begin{pgfscope}%
\pgfsys@transformshift{3.670589in}{3.077670in}%
\pgfsys@useobject{currentmarker}{}%
\end{pgfscope}%
\begin{pgfscope}%
\pgfsys@transformshift{3.539740in}{3.186610in}%
\pgfsys@useobject{currentmarker}{}%
\end{pgfscope}%
\end{pgfscope}%
\begin{pgfscope}%
\pgfsetbuttcap%
\pgfsetroundjoin%
\definecolor{currentfill}{rgb}{0.000000,0.000000,0.000000}%
\pgfsetfillcolor{currentfill}%
\pgfsetlinewidth{0.803000pt}%
\definecolor{currentstroke}{rgb}{0.000000,0.000000,0.000000}%
\pgfsetstrokecolor{currentstroke}%
\pgfsetdash{}{0pt}%
\pgfsys@defobject{currentmarker}{\pgfqpoint{0.000000in}{-0.048611in}}{\pgfqpoint{0.000000in}{0.000000in}}{%
\pgfpathmoveto{\pgfqpoint{0.000000in}{0.000000in}}%
\pgfpathlineto{\pgfqpoint{0.000000in}{-0.048611in}}%
\pgfusepath{stroke,fill}%
}%
\begin{pgfscope}%
\pgfsys@transformshift{0.766095in}{0.571603in}%
\pgfsys@useobject{currentmarker}{}%
\end{pgfscope}%
\end{pgfscope}%
\begin{pgfscope}%
\definecolor{textcolor}{rgb}{0.000000,0.000000,0.000000}%
\pgfsetstrokecolor{textcolor}%
\pgfsetfillcolor{textcolor}%
\pgftext[x=0.766095in,y=0.474381in,,top]{\color{textcolor}\sffamily\fontsize{10.000000}{12.000000}\selectfont \ensuremath{-}1.0}%
\end{pgfscope}%
\begin{pgfscope}%
\pgfsetbuttcap%
\pgfsetroundjoin%
\definecolor{currentfill}{rgb}{0.000000,0.000000,0.000000}%
\pgfsetfillcolor{currentfill}%
\pgfsetlinewidth{0.803000pt}%
\definecolor{currentstroke}{rgb}{0.000000,0.000000,0.000000}%
\pgfsetstrokecolor{currentstroke}%
\pgfsetdash{}{0pt}%
\pgfsys@defobject{currentmarker}{\pgfqpoint{0.000000in}{-0.048611in}}{\pgfqpoint{0.000000in}{0.000000in}}{%
\pgfpathmoveto{\pgfqpoint{0.000000in}{0.000000in}}%
\pgfpathlineto{\pgfqpoint{0.000000in}{-0.048611in}}%
\pgfusepath{stroke,fill}%
}%
\begin{pgfscope}%
\pgfsys@transformshift{1.762304in}{0.571603in}%
\pgfsys@useobject{currentmarker}{}%
\end{pgfscope}%
\end{pgfscope}%
\begin{pgfscope}%
\definecolor{textcolor}{rgb}{0.000000,0.000000,0.000000}%
\pgfsetstrokecolor{textcolor}%
\pgfsetfillcolor{textcolor}%
\pgftext[x=1.762304in,y=0.474381in,,top]{\color{textcolor}\sffamily\fontsize{10.000000}{12.000000}\selectfont \ensuremath{-}0.5}%
\end{pgfscope}%
\begin{pgfscope}%
\pgfsetbuttcap%
\pgfsetroundjoin%
\definecolor{currentfill}{rgb}{0.000000,0.000000,0.000000}%
\pgfsetfillcolor{currentfill}%
\pgfsetlinewidth{0.803000pt}%
\definecolor{currentstroke}{rgb}{0.000000,0.000000,0.000000}%
\pgfsetstrokecolor{currentstroke}%
\pgfsetdash{}{0pt}%
\pgfsys@defobject{currentmarker}{\pgfqpoint{0.000000in}{-0.048611in}}{\pgfqpoint{0.000000in}{0.000000in}}{%
\pgfpathmoveto{\pgfqpoint{0.000000in}{0.000000in}}%
\pgfpathlineto{\pgfqpoint{0.000000in}{-0.048611in}}%
\pgfusepath{stroke,fill}%
}%
\begin{pgfscope}%
\pgfsys@transformshift{2.758514in}{0.571603in}%
\pgfsys@useobject{currentmarker}{}%
\end{pgfscope}%
\end{pgfscope}%
\begin{pgfscope}%
\definecolor{textcolor}{rgb}{0.000000,0.000000,0.000000}%
\pgfsetstrokecolor{textcolor}%
\pgfsetfillcolor{textcolor}%
\pgftext[x=2.758514in,y=0.474381in,,top]{\color{textcolor}\sffamily\fontsize{10.000000}{12.000000}\selectfont 0.0}%
\end{pgfscope}%
\begin{pgfscope}%
\pgfsetbuttcap%
\pgfsetroundjoin%
\definecolor{currentfill}{rgb}{0.000000,0.000000,0.000000}%
\pgfsetfillcolor{currentfill}%
\pgfsetlinewidth{0.803000pt}%
\definecolor{currentstroke}{rgb}{0.000000,0.000000,0.000000}%
\pgfsetstrokecolor{currentstroke}%
\pgfsetdash{}{0pt}%
\pgfsys@defobject{currentmarker}{\pgfqpoint{0.000000in}{-0.048611in}}{\pgfqpoint{0.000000in}{0.000000in}}{%
\pgfpathmoveto{\pgfqpoint{0.000000in}{0.000000in}}%
\pgfpathlineto{\pgfqpoint{0.000000in}{-0.048611in}}%
\pgfusepath{stroke,fill}%
}%
\begin{pgfscope}%
\pgfsys@transformshift{3.754723in}{0.571603in}%
\pgfsys@useobject{currentmarker}{}%
\end{pgfscope}%
\end{pgfscope}%
\begin{pgfscope}%
\definecolor{textcolor}{rgb}{0.000000,0.000000,0.000000}%
\pgfsetstrokecolor{textcolor}%
\pgfsetfillcolor{textcolor}%
\pgftext[x=3.754723in,y=0.474381in,,top]{\color{textcolor}\sffamily\fontsize{10.000000}{12.000000}\selectfont 0.5}%
\end{pgfscope}%
\begin{pgfscope}%
\pgfsetbuttcap%
\pgfsetroundjoin%
\definecolor{currentfill}{rgb}{0.000000,0.000000,0.000000}%
\pgfsetfillcolor{currentfill}%
\pgfsetlinewidth{0.803000pt}%
\definecolor{currentstroke}{rgb}{0.000000,0.000000,0.000000}%
\pgfsetstrokecolor{currentstroke}%
\pgfsetdash{}{0pt}%
\pgfsys@defobject{currentmarker}{\pgfqpoint{0.000000in}{-0.048611in}}{\pgfqpoint{0.000000in}{0.000000in}}{%
\pgfpathmoveto{\pgfqpoint{0.000000in}{0.000000in}}%
\pgfpathlineto{\pgfqpoint{0.000000in}{-0.048611in}}%
\pgfusepath{stroke,fill}%
}%
\begin{pgfscope}%
\pgfsys@transformshift{4.750932in}{0.571603in}%
\pgfsys@useobject{currentmarker}{}%
\end{pgfscope}%
\end{pgfscope}%
\begin{pgfscope}%
\definecolor{textcolor}{rgb}{0.000000,0.000000,0.000000}%
\pgfsetstrokecolor{textcolor}%
\pgfsetfillcolor{textcolor}%
\pgftext[x=4.750932in,y=0.474381in,,top]{\color{textcolor}\sffamily\fontsize{10.000000}{12.000000}\selectfont 1.0}%
\end{pgfscope}%
\begin{pgfscope}%
\pgfsetbuttcap%
\pgfsetroundjoin%
\definecolor{currentfill}{rgb}{0.000000,0.000000,0.000000}%
\pgfsetfillcolor{currentfill}%
\pgfsetlinewidth{0.803000pt}%
\definecolor{currentstroke}{rgb}{0.000000,0.000000,0.000000}%
\pgfsetstrokecolor{currentstroke}%
\pgfsetdash{}{0pt}%
\pgfsys@defobject{currentmarker}{\pgfqpoint{0.000000in}{-0.048611in}}{\pgfqpoint{0.000000in}{0.000000in}}{%
\pgfpathmoveto{\pgfqpoint{0.000000in}{0.000000in}}%
\pgfpathlineto{\pgfqpoint{0.000000in}{-0.048611in}}%
\pgfusepath{stroke,fill}%
}%
\begin{pgfscope}%
\pgfsys@transformshift{5.747142in}{0.571603in}%
\pgfsys@useobject{currentmarker}{}%
\end{pgfscope}%
\end{pgfscope}%
\begin{pgfscope}%
\definecolor{textcolor}{rgb}{0.000000,0.000000,0.000000}%
\pgfsetstrokecolor{textcolor}%
\pgfsetfillcolor{textcolor}%
\pgftext[x=5.747142in,y=0.474381in,,top]{\color{textcolor}\sffamily\fontsize{10.000000}{12.000000}\selectfont 1.5}%
\end{pgfscope}%
\begin{pgfscope}%
\pgfsetbuttcap%
\pgfsetroundjoin%
\definecolor{currentfill}{rgb}{0.000000,0.000000,0.000000}%
\pgfsetfillcolor{currentfill}%
\pgfsetlinewidth{0.803000pt}%
\definecolor{currentstroke}{rgb}{0.000000,0.000000,0.000000}%
\pgfsetstrokecolor{currentstroke}%
\pgfsetdash{}{0pt}%
\pgfsys@defobject{currentmarker}{\pgfqpoint{0.000000in}{-0.048611in}}{\pgfqpoint{0.000000in}{0.000000in}}{%
\pgfpathmoveto{\pgfqpoint{0.000000in}{0.000000in}}%
\pgfpathlineto{\pgfqpoint{0.000000in}{-0.048611in}}%
\pgfusepath{stroke,fill}%
}%
\begin{pgfscope}%
\pgfsys@transformshift{6.743351in}{0.571603in}%
\pgfsys@useobject{currentmarker}{}%
\end{pgfscope}%
\end{pgfscope}%
\begin{pgfscope}%
\definecolor{textcolor}{rgb}{0.000000,0.000000,0.000000}%
\pgfsetstrokecolor{textcolor}%
\pgfsetfillcolor{textcolor}%
\pgftext[x=6.743351in,y=0.474381in,,top]{\color{textcolor}\sffamily\fontsize{10.000000}{12.000000}\selectfont 2.0}%
\end{pgfscope}%
\begin{pgfscope}%
\pgfsetbuttcap%
\pgfsetroundjoin%
\definecolor{currentfill}{rgb}{0.000000,0.000000,0.000000}%
\pgfsetfillcolor{currentfill}%
\pgfsetlinewidth{0.803000pt}%
\definecolor{currentstroke}{rgb}{0.000000,0.000000,0.000000}%
\pgfsetstrokecolor{currentstroke}%
\pgfsetdash{}{0pt}%
\pgfsys@defobject{currentmarker}{\pgfqpoint{0.000000in}{-0.048611in}}{\pgfqpoint{0.000000in}{0.000000in}}{%
\pgfpathmoveto{\pgfqpoint{0.000000in}{0.000000in}}%
\pgfpathlineto{\pgfqpoint{0.000000in}{-0.048611in}}%
\pgfusepath{stroke,fill}%
}%
\begin{pgfscope}%
\pgfsys@transformshift{7.739560in}{0.571603in}%
\pgfsys@useobject{currentmarker}{}%
\end{pgfscope}%
\end{pgfscope}%
\begin{pgfscope}%
\definecolor{textcolor}{rgb}{0.000000,0.000000,0.000000}%
\pgfsetstrokecolor{textcolor}%
\pgfsetfillcolor{textcolor}%
\pgftext[x=7.739560in,y=0.474381in,,top]{\color{textcolor}\sffamily\fontsize{10.000000}{12.000000}\selectfont 2.5}%
\end{pgfscope}%
\begin{pgfscope}%
\definecolor{textcolor}{rgb}{0.000000,0.000000,0.000000}%
\pgfsetstrokecolor{textcolor}%
\pgfsetfillcolor{textcolor}%
\pgftext[x=4.252828in,y=0.284413in,,top]{\color{textcolor}\sffamily\fontsize{10.000000}{12.000000}\selectfont x}%
\end{pgfscope}%
\begin{pgfscope}%
\pgfsetbuttcap%
\pgfsetroundjoin%
\definecolor{currentfill}{rgb}{0.000000,0.000000,0.000000}%
\pgfsetfillcolor{currentfill}%
\pgfsetlinewidth{0.803000pt}%
\definecolor{currentstroke}{rgb}{0.000000,0.000000,0.000000}%
\pgfsetstrokecolor{currentstroke}%
\pgfsetdash{}{0pt}%
\pgfsys@defobject{currentmarker}{\pgfqpoint{-0.048611in}{0.000000in}}{\pgfqpoint{-0.000000in}{0.000000in}}{%
\pgfpathmoveto{\pgfqpoint{-0.000000in}{0.000000in}}%
\pgfpathlineto{\pgfqpoint{-0.048611in}{0.000000in}}%
\pgfusepath{stroke,fill}%
}%
\begin{pgfscope}%
\pgfsys@transformshift{0.766095in}{0.571603in}%
\pgfsys@useobject{currentmarker}{}%
\end{pgfscope}%
\end{pgfscope}%
\begin{pgfscope}%
\definecolor{textcolor}{rgb}{0.000000,0.000000,0.000000}%
\pgfsetstrokecolor{textcolor}%
\pgfsetfillcolor{textcolor}%
\pgftext[x=0.339968in, y=0.518842in, left, base]{\color{textcolor}\sffamily\fontsize{10.000000}{12.000000}\selectfont \ensuremath{-}2.5}%
\end{pgfscope}%
\begin{pgfscope}%
\pgfsetbuttcap%
\pgfsetroundjoin%
\definecolor{currentfill}{rgb}{0.000000,0.000000,0.000000}%
\pgfsetfillcolor{currentfill}%
\pgfsetlinewidth{0.803000pt}%
\definecolor{currentstroke}{rgb}{0.000000,0.000000,0.000000}%
\pgfsetstrokecolor{currentstroke}%
\pgfsetdash{}{0pt}%
\pgfsys@defobject{currentmarker}{\pgfqpoint{-0.048611in}{0.000000in}}{\pgfqpoint{-0.000000in}{0.000000in}}{%
\pgfpathmoveto{\pgfqpoint{-0.000000in}{0.000000in}}%
\pgfpathlineto{\pgfqpoint{-0.048611in}{0.000000in}}%
\pgfusepath{stroke,fill}%
}%
\begin{pgfscope}%
\pgfsys@transformshift{0.766095in}{1.318123in}%
\pgfsys@useobject{currentmarker}{}%
\end{pgfscope}%
\end{pgfscope}%
\begin{pgfscope}%
\definecolor{textcolor}{rgb}{0.000000,0.000000,0.000000}%
\pgfsetstrokecolor{textcolor}%
\pgfsetfillcolor{textcolor}%
\pgftext[x=0.339968in, y=1.265361in, left, base]{\color{textcolor}\sffamily\fontsize{10.000000}{12.000000}\selectfont \ensuremath{-}2.0}%
\end{pgfscope}%
\begin{pgfscope}%
\pgfsetbuttcap%
\pgfsetroundjoin%
\definecolor{currentfill}{rgb}{0.000000,0.000000,0.000000}%
\pgfsetfillcolor{currentfill}%
\pgfsetlinewidth{0.803000pt}%
\definecolor{currentstroke}{rgb}{0.000000,0.000000,0.000000}%
\pgfsetstrokecolor{currentstroke}%
\pgfsetdash{}{0pt}%
\pgfsys@defobject{currentmarker}{\pgfqpoint{-0.048611in}{0.000000in}}{\pgfqpoint{-0.000000in}{0.000000in}}{%
\pgfpathmoveto{\pgfqpoint{-0.000000in}{0.000000in}}%
\pgfpathlineto{\pgfqpoint{-0.048611in}{0.000000in}}%
\pgfusepath{stroke,fill}%
}%
\begin{pgfscope}%
\pgfsys@transformshift{0.766095in}{2.064642in}%
\pgfsys@useobject{currentmarker}{}%
\end{pgfscope}%
\end{pgfscope}%
\begin{pgfscope}%
\definecolor{textcolor}{rgb}{0.000000,0.000000,0.000000}%
\pgfsetstrokecolor{textcolor}%
\pgfsetfillcolor{textcolor}%
\pgftext[x=0.339968in, y=2.011880in, left, base]{\color{textcolor}\sffamily\fontsize{10.000000}{12.000000}\selectfont \ensuremath{-}1.5}%
\end{pgfscope}%
\begin{pgfscope}%
\pgfsetbuttcap%
\pgfsetroundjoin%
\definecolor{currentfill}{rgb}{0.000000,0.000000,0.000000}%
\pgfsetfillcolor{currentfill}%
\pgfsetlinewidth{0.803000pt}%
\definecolor{currentstroke}{rgb}{0.000000,0.000000,0.000000}%
\pgfsetstrokecolor{currentstroke}%
\pgfsetdash{}{0pt}%
\pgfsys@defobject{currentmarker}{\pgfqpoint{-0.048611in}{0.000000in}}{\pgfqpoint{-0.000000in}{0.000000in}}{%
\pgfpathmoveto{\pgfqpoint{-0.000000in}{0.000000in}}%
\pgfpathlineto{\pgfqpoint{-0.048611in}{0.000000in}}%
\pgfusepath{stroke,fill}%
}%
\begin{pgfscope}%
\pgfsys@transformshift{0.766095in}{2.811161in}%
\pgfsys@useobject{currentmarker}{}%
\end{pgfscope}%
\end{pgfscope}%
\begin{pgfscope}%
\definecolor{textcolor}{rgb}{0.000000,0.000000,0.000000}%
\pgfsetstrokecolor{textcolor}%
\pgfsetfillcolor{textcolor}%
\pgftext[x=0.339968in, y=2.758400in, left, base]{\color{textcolor}\sffamily\fontsize{10.000000}{12.000000}\selectfont \ensuremath{-}1.0}%
\end{pgfscope}%
\begin{pgfscope}%
\pgfsetbuttcap%
\pgfsetroundjoin%
\definecolor{currentfill}{rgb}{0.000000,0.000000,0.000000}%
\pgfsetfillcolor{currentfill}%
\pgfsetlinewidth{0.803000pt}%
\definecolor{currentstroke}{rgb}{0.000000,0.000000,0.000000}%
\pgfsetstrokecolor{currentstroke}%
\pgfsetdash{}{0pt}%
\pgfsys@defobject{currentmarker}{\pgfqpoint{-0.048611in}{0.000000in}}{\pgfqpoint{-0.000000in}{0.000000in}}{%
\pgfpathmoveto{\pgfqpoint{-0.000000in}{0.000000in}}%
\pgfpathlineto{\pgfqpoint{-0.048611in}{0.000000in}}%
\pgfusepath{stroke,fill}%
}%
\begin{pgfscope}%
\pgfsys@transformshift{0.766095in}{3.557681in}%
\pgfsys@useobject{currentmarker}{}%
\end{pgfscope}%
\end{pgfscope}%
\begin{pgfscope}%
\definecolor{textcolor}{rgb}{0.000000,0.000000,0.000000}%
\pgfsetstrokecolor{textcolor}%
\pgfsetfillcolor{textcolor}%
\pgftext[x=0.339968in, y=3.504919in, left, base]{\color{textcolor}\sffamily\fontsize{10.000000}{12.000000}\selectfont \ensuremath{-}0.5}%
\end{pgfscope}%
\begin{pgfscope}%
\pgfsetbuttcap%
\pgfsetroundjoin%
\definecolor{currentfill}{rgb}{0.000000,0.000000,0.000000}%
\pgfsetfillcolor{currentfill}%
\pgfsetlinewidth{0.803000pt}%
\definecolor{currentstroke}{rgb}{0.000000,0.000000,0.000000}%
\pgfsetstrokecolor{currentstroke}%
\pgfsetdash{}{0pt}%
\pgfsys@defobject{currentmarker}{\pgfqpoint{-0.048611in}{0.000000in}}{\pgfqpoint{-0.000000in}{0.000000in}}{%
\pgfpathmoveto{\pgfqpoint{-0.000000in}{0.000000in}}%
\pgfpathlineto{\pgfqpoint{-0.048611in}{0.000000in}}%
\pgfusepath{stroke,fill}%
}%
\begin{pgfscope}%
\pgfsys@transformshift{0.766095in}{4.304200in}%
\pgfsys@useobject{currentmarker}{}%
\end{pgfscope}%
\end{pgfscope}%
\begin{pgfscope}%
\definecolor{textcolor}{rgb}{0.000000,0.000000,0.000000}%
\pgfsetstrokecolor{textcolor}%
\pgfsetfillcolor{textcolor}%
\pgftext[x=0.447993in, y=4.251438in, left, base]{\color{textcolor}\sffamily\fontsize{10.000000}{12.000000}\selectfont 0.0}%
\end{pgfscope}%
\begin{pgfscope}%
\pgfsetbuttcap%
\pgfsetroundjoin%
\definecolor{currentfill}{rgb}{0.000000,0.000000,0.000000}%
\pgfsetfillcolor{currentfill}%
\pgfsetlinewidth{0.803000pt}%
\definecolor{currentstroke}{rgb}{0.000000,0.000000,0.000000}%
\pgfsetstrokecolor{currentstroke}%
\pgfsetdash{}{0pt}%
\pgfsys@defobject{currentmarker}{\pgfqpoint{-0.048611in}{0.000000in}}{\pgfqpoint{-0.000000in}{0.000000in}}{%
\pgfpathmoveto{\pgfqpoint{-0.000000in}{0.000000in}}%
\pgfpathlineto{\pgfqpoint{-0.048611in}{0.000000in}}%
\pgfusepath{stroke,fill}%
}%
\begin{pgfscope}%
\pgfsys@transformshift{0.766095in}{5.050719in}%
\pgfsys@useobject{currentmarker}{}%
\end{pgfscope}%
\end{pgfscope}%
\begin{pgfscope}%
\definecolor{textcolor}{rgb}{0.000000,0.000000,0.000000}%
\pgfsetstrokecolor{textcolor}%
\pgfsetfillcolor{textcolor}%
\pgftext[x=0.447993in, y=4.997958in, left, base]{\color{textcolor}\sffamily\fontsize{10.000000}{12.000000}\selectfont 0.5}%
\end{pgfscope}%
\begin{pgfscope}%
\pgfsetbuttcap%
\pgfsetroundjoin%
\definecolor{currentfill}{rgb}{0.000000,0.000000,0.000000}%
\pgfsetfillcolor{currentfill}%
\pgfsetlinewidth{0.803000pt}%
\definecolor{currentstroke}{rgb}{0.000000,0.000000,0.000000}%
\pgfsetstrokecolor{currentstroke}%
\pgfsetdash{}{0pt}%
\pgfsys@defobject{currentmarker}{\pgfqpoint{-0.048611in}{0.000000in}}{\pgfqpoint{-0.000000in}{0.000000in}}{%
\pgfpathmoveto{\pgfqpoint{-0.000000in}{0.000000in}}%
\pgfpathlineto{\pgfqpoint{-0.048611in}{0.000000in}}%
\pgfusepath{stroke,fill}%
}%
\begin{pgfscope}%
\pgfsys@transformshift{0.766095in}{5.797238in}%
\pgfsys@useobject{currentmarker}{}%
\end{pgfscope}%
\end{pgfscope}%
\begin{pgfscope}%
\definecolor{textcolor}{rgb}{0.000000,0.000000,0.000000}%
\pgfsetstrokecolor{textcolor}%
\pgfsetfillcolor{textcolor}%
\pgftext[x=0.447993in, y=5.744477in, left, base]{\color{textcolor}\sffamily\fontsize{10.000000}{12.000000}\selectfont 1.0}%
\end{pgfscope}%
\begin{pgfscope}%
\definecolor{textcolor}{rgb}{0.000000,0.000000,0.000000}%
\pgfsetstrokecolor{textcolor}%
\pgfsetfillcolor{textcolor}%
\pgftext[x=0.284413in,y=3.184421in,,bottom,rotate=90.000000]{\color{textcolor}\sffamily\fontsize{10.000000}{12.000000}\selectfont y}%
\end{pgfscope}%
\begin{pgfscope}%
\pgfpathrectangle{\pgfqpoint{0.766095in}{0.571603in}}{\pgfqpoint{6.973465in}{5.225635in}}%
\pgfusepath{clip}%
\pgfsetbuttcap%
\pgfsetroundjoin%
\pgfsetlinewidth{1.505625pt}%
\definecolor{currentstroke}{rgb}{0.276022,0.044167,0.370164}%
\pgfsetstrokecolor{currentstroke}%
\pgfsetdash{}{0pt}%
\pgfpathmoveto{\pgfqpoint{7.739560in}{2.153527in}}%
\pgfpathlineto{\pgfqpoint{7.704518in}{2.170846in}}%
\pgfpathlineto{\pgfqpoint{7.699403in}{2.173431in}}%
\pgfpathlineto{\pgfqpoint{7.669475in}{2.187701in}}%
\pgfpathlineto{\pgfqpoint{7.645186in}{2.199691in}}%
\pgfpathlineto{\pgfqpoint{7.634433in}{2.204727in}}%
\pgfpathlineto{\pgfqpoint{7.599390in}{2.221541in}}%
\pgfpathlineto{\pgfqpoint{7.590411in}{2.225950in}}%
\pgfpathlineto{\pgfqpoint{7.564348in}{2.238132in}}%
\pgfpathlineto{\pgfqpoint{7.535167in}{2.252210in}}%
\pgfpathlineto{\pgfqpoint{7.529305in}{2.254914in}}%
\pgfpathlineto{\pgfqpoint{7.494263in}{2.271361in}}%
\pgfpathlineto{\pgfqpoint{7.479497in}{2.278469in}}%
\pgfpathlineto{\pgfqpoint{7.459220in}{2.287830in}}%
\pgfpathlineto{\pgfqpoint{7.424177in}{2.304477in}}%
\pgfpathlineto{\pgfqpoint{7.423655in}{2.304729in}}%
\pgfpathlineto{\pgfqpoint{7.389135in}{2.320685in}}%
\pgfpathlineto{\pgfqpoint{7.367466in}{2.330988in}}%
\pgfpathlineto{\pgfqpoint{7.354092in}{2.337123in}}%
\pgfpathlineto{\pgfqpoint{7.339737in}{2.343844in}}%
\pgfusepath{stroke}%
\end{pgfscope}%
\begin{pgfscope}%
\pgfpathrectangle{\pgfqpoint{0.766095in}{0.571603in}}{\pgfqpoint{6.973465in}{5.225635in}}%
\pgfusepath{clip}%
\pgfsetbuttcap%
\pgfsetroundjoin%
\pgfsetlinewidth{1.505625pt}%
\definecolor{currentstroke}{rgb}{0.276022,0.044167,0.370164}%
\pgfsetstrokecolor{currentstroke}%
\pgfsetdash{}{0pt}%
\pgfpathmoveto{\pgfqpoint{7.057477in}{2.475683in}}%
\pgfpathlineto{\pgfqpoint{6.793412in}{2.600675in}}%
\pgfpathlineto{\pgfqpoint{6.548114in}{2.720020in}}%
\pgfpathlineto{\pgfqpoint{6.330017in}{2.829918in}}%
\pgfpathlineto{\pgfqpoint{6.197689in}{2.898754in}}%
\pgfpathlineto{\pgfqpoint{6.057518in}{2.973918in}}%
\pgfpathlineto{\pgfqpoint{5.917348in}{3.051726in}}%
\pgfpathlineto{\pgfqpoint{5.800766in}{3.118772in}}%
\pgfpathlineto{\pgfqpoint{5.707093in}{3.174420in}}%
\pgfpathlineto{\pgfqpoint{5.584220in}{3.250070in}}%
\pgfpathlineto{\pgfqpoint{5.496838in}{3.306068in}}%
\pgfpathlineto{\pgfqpoint{5.383961in}{3.381367in}}%
\pgfpathlineto{\pgfqpoint{5.286583in}{3.449578in}}%
\pgfpathlineto{\pgfqpoint{5.216497in}{3.500756in}}%
\pgfpathlineto{\pgfqpoint{5.146412in}{3.553946in}}%
\pgfpathlineto{\pgfqpoint{5.076327in}{3.609415in}}%
\pgfpathlineto{\pgfqpoint{5.034151in}{3.643962in}}%
\pgfpathlineto{\pgfqpoint{4.971200in}{3.697614in}}%
\pgfpathlineto{\pgfqpoint{4.901115in}{3.760456in}}%
\pgfpathlineto{\pgfqpoint{4.857299in}{3.801519in}}%
\pgfpathlineto{\pgfqpoint{4.795987in}{3.862101in}}%
\pgfpathlineto{\pgfqpoint{4.753096in}{3.906556in}}%
\pgfpathlineto{\pgfqpoint{4.690859in}{3.975282in}}%
\pgfpathlineto{\pgfqpoint{4.655817in}{4.016238in}}%
\pgfpathlineto{\pgfqpoint{4.616803in}{4.064113in}}%
\pgfpathlineto{\pgfqpoint{4.576482in}{4.116632in}}%
\pgfpathlineto{\pgfqpoint{4.538646in}{4.169151in}}%
\pgfpathlineto{\pgfqpoint{4.503199in}{4.221670in}}%
\pgfpathlineto{\pgfqpoint{4.470043in}{4.274189in}}%
\pgfpathlineto{\pgfqpoint{4.439068in}{4.326708in}}%
\pgfpathlineto{\pgfqpoint{4.410152in}{4.379227in}}%
\pgfpathlineto{\pgfqpoint{4.375477in}{4.447233in}}%
\pgfpathlineto{\pgfqpoint{4.357569in}{4.484265in}}%
\pgfpathlineto{\pgfqpoint{4.322119in}{4.563043in}}%
\pgfpathlineto{\pgfqpoint{4.278347in}{4.668081in}}%
\pgfpathlineto{\pgfqpoint{4.224638in}{4.799378in}}%
\pgfpathlineto{\pgfqpoint{4.200264in}{4.855106in}}%
\pgfpathlineto{\pgfqpoint{4.176498in}{4.904416in}}%
\pgfpathlineto{\pgfqpoint{4.162840in}{4.930676in}}%
\pgfpathlineto{\pgfqpoint{4.132090in}{4.983195in}}%
\pgfpathlineto{\pgfqpoint{4.114456in}{5.009454in}}%
\pgfpathlineto{\pgfqpoint{4.095136in}{5.035803in}}%
\pgfpathlineto{\pgfqpoint{4.060094in}{5.076672in}}%
\pgfpathlineto{\pgfqpoint{4.048929in}{5.088233in}}%
\pgfpathlineto{\pgfqpoint{4.020996in}{5.114492in}}%
\pgfpathlineto{\pgfqpoint{3.988539in}{5.140752in}}%
\pgfpathlineto{\pgfqpoint{3.950082in}{5.167011in}}%
\pgfpathlineto{\pgfqpoint{3.903520in}{5.193271in}}%
\pgfpathlineto{\pgfqpoint{3.884881in}{5.202357in}}%
\pgfpathlineto{\pgfqpoint{3.845487in}{5.219530in}}%
\pgfpathlineto{\pgfqpoint{3.814796in}{5.230876in}}%
\pgfpathlineto{\pgfqpoint{3.768127in}{5.245790in}}%
\pgfpathlineto{\pgfqpoint{3.709668in}{5.260566in}}%
\pgfpathlineto{\pgfqpoint{3.639583in}{5.274025in}}%
\pgfpathlineto{\pgfqpoint{3.569498in}{5.283277in}}%
\pgfpathlineto{\pgfqpoint{3.499413in}{5.289348in}}%
\pgfpathlineto{\pgfqpoint{3.429328in}{5.292625in}}%
\pgfpathlineto{\pgfqpoint{3.359243in}{5.293427in}}%
\pgfpathlineto{\pgfqpoint{3.289158in}{5.292025in}}%
\pgfpathlineto{\pgfqpoint{3.219073in}{5.288649in}}%
\pgfpathlineto{\pgfqpoint{3.148988in}{5.283492in}}%
\pgfpathlineto{\pgfqpoint{3.078903in}{5.276722in}}%
\pgfpathlineto{\pgfqpoint{3.008818in}{5.268329in}}%
\pgfpathlineto{\pgfqpoint{2.903690in}{5.252855in}}%
\pgfpathlineto{\pgfqpoint{2.833605in}{5.240779in}}%
\pgfpathlineto{\pgfqpoint{2.726101in}{5.219530in}}%
\pgfpathlineto{\pgfqpoint{2.612141in}{5.193271in}}%
\pgfpathlineto{\pgfqpoint{2.511619in}{5.167011in}}%
\pgfpathlineto{\pgfqpoint{2.413094in}{5.138289in}}%
\pgfpathlineto{\pgfqpoint{2.307967in}{5.104152in}}%
\pgfpathlineto{\pgfqpoint{2.237882in}{5.079367in}}%
\pgfpathlineto{\pgfqpoint{2.167797in}{5.052782in}}%
\pgfpathlineto{\pgfqpoint{2.097711in}{5.024273in}}%
\pgfpathlineto{\pgfqpoint{2.027626in}{4.993702in}}%
\pgfpathlineto{\pgfqpoint{1.949368in}{4.956935in}}%
\pgfpathlineto{\pgfqpoint{1.887456in}{4.925559in}}%
\pgfpathlineto{\pgfqpoint{1.817371in}{4.887378in}}%
\pgfpathlineto{\pgfqpoint{1.756779in}{4.851897in}}%
\pgfpathlineto{\pgfqpoint{1.712244in}{4.824144in}}%
\pgfpathlineto{\pgfqpoint{1.642158in}{4.777060in}}%
\pgfpathlineto{\pgfqpoint{1.600417in}{4.746860in}}%
\pgfpathlineto{\pgfqpoint{1.537031in}{4.697343in}}%
\pgfpathlineto{\pgfqpoint{1.501988in}{4.667814in}}%
\pgfpathlineto{\pgfqpoint{1.444977in}{4.615562in}}%
\pgfpathlineto{\pgfqpoint{1.418451in}{4.589303in}}%
\pgfpathlineto{\pgfqpoint{1.393242in}{4.563043in}}%
\pgfpathlineto{\pgfqpoint{1.347034in}{4.510524in}}%
\pgfpathlineto{\pgfqpoint{1.325699in}{4.484265in}}%
\pgfpathlineto{\pgfqpoint{1.286932in}{4.431746in}}%
\pgfpathlineto{\pgfqpoint{1.252772in}{4.379227in}}%
\pgfpathlineto{\pgfqpoint{1.221648in}{4.323981in}}%
\pgfpathlineto{\pgfqpoint{1.197719in}{4.274189in}}%
\pgfpathlineto{\pgfqpoint{1.186477in}{4.247930in}}%
\pgfpathlineto{\pgfqpoint{1.167358in}{4.195411in}}%
\pgfpathlineto{\pgfqpoint{1.151563in}{4.140987in}}%
\pgfpathlineto{\pgfqpoint{1.140799in}{4.090373in}}%
\pgfpathlineto{\pgfqpoint{1.133387in}{4.037854in}}%
\pgfpathlineto{\pgfqpoint{1.129807in}{3.985335in}}%
\pgfpathlineto{\pgfqpoint{1.130083in}{3.932816in}}%
\pgfpathlineto{\pgfqpoint{1.134226in}{3.880297in}}%
\pgfpathlineto{\pgfqpoint{1.142232in}{3.827778in}}%
\pgfpathlineto{\pgfqpoint{1.154148in}{3.775259in}}%
\pgfpathlineto{\pgfqpoint{1.170199in}{3.722740in}}%
\pgfpathlineto{\pgfqpoint{1.190194in}{3.670221in}}%
\pgfpathlineto{\pgfqpoint{1.214467in}{3.617702in}}%
\pgfpathlineto{\pgfqpoint{1.228187in}{3.591443in}}%
\pgfpathlineto{\pgfqpoint{1.258882in}{3.538924in}}%
\pgfpathlineto{\pgfqpoint{1.294111in}{3.486405in}}%
\pgfpathlineto{\pgfqpoint{1.334022in}{3.433886in}}%
\pgfpathlineto{\pgfqpoint{1.378739in}{3.381367in}}%
\pgfpathlineto{\pgfqpoint{1.428355in}{3.328848in}}%
\pgfpathlineto{\pgfqpoint{1.466946in}{3.291554in}}%
\pgfpathlineto{\pgfqpoint{1.512726in}{3.250070in}}%
\pgfpathlineto{\pgfqpoint{1.575754in}{3.197551in}}%
\pgfpathlineto{\pgfqpoint{1.644579in}{3.145032in}}%
\pgfpathlineto{\pgfqpoint{1.719473in}{3.092513in}}%
\pgfpathlineto{\pgfqpoint{1.800654in}{3.039994in}}%
\pgfpathlineto{\pgfqpoint{1.888304in}{2.987475in}}%
\pgfpathlineto{\pgfqpoint{1.982992in}{2.934956in}}%
\pgfpathlineto{\pgfqpoint{2.084843in}{2.882437in}}%
\pgfpathlineto{\pgfqpoint{2.194222in}{2.829918in}}%
\pgfpathlineto{\pgfqpoint{2.311501in}{2.777399in}}%
\pgfpathlineto{\pgfqpoint{2.448137in}{2.720512in}}%
\pgfpathlineto{\pgfqpoint{2.571464in}{2.672361in}}%
\pgfpathlineto{\pgfqpoint{2.728477in}{2.615097in}}%
\pgfpathlineto{\pgfqpoint{2.868647in}{2.567025in}}%
\pgfpathlineto{\pgfqpoint{3.043860in}{2.510784in}}%
\pgfpathlineto{\pgfqpoint{3.219073in}{2.458002in}}%
\pgfpathlineto{\pgfqpoint{3.394285in}{2.408236in}}%
\pgfpathlineto{\pgfqpoint{3.604541in}{2.352202in}}%
\pgfpathlineto{\pgfqpoint{3.814796in}{2.299507in}}%
\pgfpathlineto{\pgfqpoint{4.025051in}{2.249759in}}%
\pgfpathlineto{\pgfqpoint{4.270349in}{2.195198in}}%
\pgfpathlineto{\pgfqpoint{4.515647in}{2.143829in}}%
\pgfpathlineto{\pgfqpoint{4.795987in}{2.088787in}}%
\pgfpathlineto{\pgfqpoint{5.076327in}{2.037181in}}%
\pgfpathlineto{\pgfqpoint{5.356668in}{1.988744in}}%
\pgfpathlineto{\pgfqpoint{5.707093in}{1.932757in}}%
\pgfpathlineto{\pgfqpoint{6.034188in}{1.884577in}}%
\pgfpathlineto{\pgfqpoint{6.337859in}{1.843956in}}%
\pgfpathlineto{\pgfqpoint{6.548114in}{1.817954in}}%
\pgfpathlineto{\pgfqpoint{6.723327in}{1.797794in}}%
\pgfpathlineto{\pgfqpoint{6.968624in}{1.772266in}}%
\pgfpathlineto{\pgfqpoint{7.178880in}{1.753039in}}%
\pgfpathlineto{\pgfqpoint{7.354092in}{1.740073in}}%
\pgfpathlineto{\pgfqpoint{7.529305in}{1.729468in}}%
\pgfpathlineto{\pgfqpoint{7.634433in}{1.725048in}}%
\pgfpathlineto{\pgfqpoint{7.739560in}{1.722582in}}%
\pgfpathlineto{\pgfqpoint{7.739560in}{1.722582in}}%
\pgfusepath{stroke}%
\end{pgfscope}%
\begin{pgfscope}%
\pgfpathrectangle{\pgfqpoint{0.766095in}{0.571603in}}{\pgfqpoint{6.973465in}{5.225635in}}%
\pgfusepath{clip}%
\pgfsetbuttcap%
\pgfsetroundjoin%
\pgfsetlinewidth{1.505625pt}%
\definecolor{currentstroke}{rgb}{0.281446,0.084320,0.407414}%
\pgfsetstrokecolor{currentstroke}%
\pgfsetdash{}{0pt}%
\pgfpathmoveto{\pgfqpoint{2.997242in}{5.797238in}}%
\pgfpathlineto{\pgfqpoint{2.973775in}{5.792512in}}%
\pgfpathlineto{\pgfqpoint{2.938732in}{5.785372in}}%
\pgfpathlineto{\pgfqpoint{2.903690in}{5.778176in}}%
\pgfpathlineto{\pgfqpoint{2.868911in}{5.770979in}}%
\pgfpathlineto{\pgfqpoint{2.868647in}{5.770923in}}%
\pgfpathlineto{\pgfqpoint{2.833605in}{5.763325in}}%
\pgfpathlineto{\pgfqpoint{2.798562in}{5.755675in}}%
\pgfpathlineto{\pgfqpoint{2.763520in}{5.747973in}}%
\pgfpathlineto{\pgfqpoint{2.748906in}{5.744720in}}%
\pgfpathlineto{\pgfqpoint{2.728477in}{5.740042in}}%
\pgfpathlineto{\pgfqpoint{2.693435in}{5.731931in}}%
\pgfpathlineto{\pgfqpoint{2.658392in}{5.723771in}}%
\pgfpathlineto{\pgfqpoint{2.635805in}{5.718460in}}%
\pgfpathlineto{\pgfqpoint{2.623350in}{5.715449in}}%
\pgfpathlineto{\pgfqpoint{2.588307in}{5.706870in}}%
\pgfpathlineto{\pgfqpoint{2.553264in}{5.698244in}}%
\pgfpathlineto{\pgfqpoint{2.528922in}{5.692201in}}%
\pgfpathlineto{\pgfqpoint{2.518222in}{5.689470in}}%
\pgfpathlineto{\pgfqpoint{2.483179in}{5.680415in}}%
\pgfpathlineto{\pgfqpoint{2.481725in}{5.680037in}}%
\pgfusepath{stroke}%
\end{pgfscope}%
\begin{pgfscope}%
\pgfpathrectangle{\pgfqpoint{0.766095in}{0.571603in}}{\pgfqpoint{6.973465in}{5.225635in}}%
\pgfusepath{clip}%
\pgfsetbuttcap%
\pgfsetroundjoin%
\pgfsetlinewidth{1.505625pt}%
\definecolor{currentstroke}{rgb}{0.281446,0.084320,0.407414}%
\pgfsetstrokecolor{currentstroke}%
\pgfsetdash{}{0pt}%
\pgfpathmoveto{\pgfqpoint{2.181928in}{5.596096in}}%
\pgfpathlineto{\pgfqpoint{2.167797in}{5.591844in}}%
\pgfpathlineto{\pgfqpoint{2.152389in}{5.587163in}}%
\pgfpathlineto{\pgfqpoint{2.132754in}{5.581036in}}%
\pgfpathlineto{\pgfqpoint{2.097711in}{5.570012in}}%
\pgfpathlineto{\pgfqpoint{2.068917in}{5.560903in}}%
\pgfpathlineto{\pgfqpoint{2.062669in}{5.558874in}}%
\pgfpathlineto{\pgfqpoint{2.027626in}{5.547355in}}%
\pgfpathlineto{\pgfqpoint{1.992584in}{5.535795in}}%
\pgfpathlineto{\pgfqpoint{1.989137in}{5.534644in}}%
\pgfpathlineto{\pgfqpoint{1.957541in}{5.523808in}}%
\pgfpathlineto{\pgfqpoint{1.922499in}{5.511738in}}%
\pgfpathlineto{\pgfqpoint{1.912863in}{5.508384in}}%
\pgfpathlineto{\pgfqpoint{1.887456in}{5.499306in}}%
\pgfpathlineto{\pgfqpoint{1.852414in}{5.486709in}}%
\pgfpathlineto{\pgfqpoint{1.839783in}{5.482125in}}%
\pgfpathlineto{\pgfqpoint{1.817371in}{5.473776in}}%
\pgfpathlineto{\pgfqpoint{1.782329in}{5.460634in}}%
\pgfpathlineto{\pgfqpoint{1.769731in}{5.455865in}}%
\pgfpathlineto{\pgfqpoint{1.747286in}{5.447144in}}%
\pgfpathlineto{\pgfqpoint{1.712244in}{5.433439in}}%
\pgfpathlineto{\pgfqpoint{1.702540in}{5.429606in}}%
\pgfpathlineto{\pgfqpoint{1.677201in}{5.419332in}}%
\pgfpathlineto{\pgfqpoint{1.642158in}{5.405043in}}%
\pgfpathlineto{\pgfqpoint{1.638046in}{5.403346in}}%
\pgfpathlineto{\pgfqpoint{1.607116in}{5.390253in}}%
\pgfpathlineto{\pgfqpoint{1.576173in}{5.377087in}}%
\pgfpathlineto{\pgfqpoint{1.572073in}{5.375297in}}%
\pgfpathlineto{\pgfqpoint{1.537031in}{5.359819in}}%
\pgfpathlineto{\pgfqpoint{1.516813in}{5.350827in}}%
\pgfpathlineto{\pgfqpoint{1.501988in}{5.344061in}}%
\pgfpathlineto{\pgfqpoint{1.466946in}{5.327933in}}%
\pgfpathlineto{\pgfqpoint{1.459711in}{5.324568in}}%
\pgfpathlineto{\pgfqpoint{1.431903in}{5.311296in}}%
\pgfpathlineto{\pgfqpoint{1.404880in}{5.298308in}}%
\pgfpathlineto{\pgfqpoint{1.396861in}{5.294353in}}%
\pgfpathlineto{\pgfqpoint{1.361818in}{5.276892in}}%
\pgfpathlineto{\pgfqpoint{1.352191in}{5.272049in}}%
\pgfpathlineto{\pgfqpoint{1.326776in}{5.258925in}}%
\pgfpathlineto{\pgfqpoint{1.301541in}{5.245790in}}%
\pgfpathlineto{\pgfqpoint{1.291733in}{5.240549in}}%
\pgfpathlineto{\pgfqpoint{1.256691in}{5.221645in}}%
\pgfpathlineto{\pgfqpoint{1.252813in}{5.219530in}}%
\pgfpathlineto{\pgfqpoint{1.221648in}{5.202077in}}%
\pgfpathlineto{\pgfqpoint{1.206062in}{5.193271in}}%
\pgfpathlineto{\pgfqpoint{1.186605in}{5.181981in}}%
\pgfpathlineto{\pgfqpoint{1.161046in}{5.167011in}}%
\pgfpathlineto{\pgfqpoint{1.151563in}{5.161307in}}%
\pgfpathlineto{\pgfqpoint{1.117727in}{5.140752in}}%
\pgfpathlineto{\pgfqpoint{1.116520in}{5.139998in}}%
\pgfpathlineto{\pgfqpoint{1.081478in}{5.117873in}}%
\pgfpathlineto{\pgfqpoint{1.076180in}{5.114492in}}%
\pgfpathlineto{\pgfqpoint{1.046435in}{5.094990in}}%
\pgfpathlineto{\pgfqpoint{1.036240in}{5.088233in}}%
\pgfpathlineto{\pgfqpoint{1.011393in}{5.071309in}}%
\pgfpathlineto{\pgfqpoint{0.997839in}{5.061973in}}%
\pgfpathlineto{\pgfqpoint{0.976350in}{5.046757in}}%
\pgfpathlineto{\pgfqpoint{0.960933in}{5.035714in}}%
\pgfpathlineto{\pgfqpoint{0.941308in}{5.021257in}}%
\pgfpathlineto{\pgfqpoint{0.925476in}{5.009454in}}%
\pgfpathlineto{\pgfqpoint{0.906265in}{4.994720in}}%
\pgfpathlineto{\pgfqpoint{0.891423in}{4.983195in}}%
\pgfpathlineto{\pgfqpoint{0.871223in}{4.967052in}}%
\pgfpathlineto{\pgfqpoint{0.858724in}{4.956935in}}%
\pgfpathlineto{\pgfqpoint{0.836180in}{4.938146in}}%
\pgfpathlineto{\pgfqpoint{0.827333in}{4.930676in}}%
\pgfpathlineto{\pgfqpoint{0.801138in}{4.907887in}}%
\pgfpathlineto{\pgfqpoint{0.797201in}{4.904416in}}%
\pgfpathlineto{\pgfqpoint{0.768325in}{4.878157in}}%
\pgfpathlineto{\pgfqpoint{0.766095in}{4.876063in}}%
\pgfusepath{stroke}%
\end{pgfscope}%
\begin{pgfscope}%
\pgfpathrectangle{\pgfqpoint{0.766095in}{0.571603in}}{\pgfqpoint{6.973465in}{5.225635in}}%
\pgfusepath{clip}%
\pgfsetbuttcap%
\pgfsetroundjoin%
\pgfsetlinewidth{1.505625pt}%
\definecolor{currentstroke}{rgb}{0.281446,0.084320,0.407414}%
\pgfsetstrokecolor{currentstroke}%
\pgfsetdash{}{0pt}%
\pgfpathmoveto{\pgfqpoint{0.766095in}{3.233437in}}%
\pgfpathlineto{\pgfqpoint{0.805987in}{3.197551in}}%
\pgfpathlineto{\pgfqpoint{0.871223in}{3.142994in}}%
\pgfpathlineto{\pgfqpoint{0.941308in}{3.088940in}}%
\pgfpathlineto{\pgfqpoint{1.011393in}{3.038719in}}%
\pgfpathlineto{\pgfqpoint{1.088222in}{2.987475in}}%
\pgfpathlineto{\pgfqpoint{1.172732in}{2.934956in}}%
\pgfpathlineto{\pgfqpoint{1.263281in}{2.882437in}}%
\pgfpathlineto{\pgfqpoint{1.361818in}{2.829076in}}%
\pgfpathlineto{\pgfqpoint{1.466946in}{2.775874in}}%
\pgfpathlineto{\pgfqpoint{1.574341in}{2.724880in}}%
\pgfpathlineto{\pgfqpoint{1.692201in}{2.672361in}}%
\pgfpathlineto{\pgfqpoint{1.817516in}{2.619842in}}%
\pgfpathlineto{\pgfqpoint{1.957541in}{2.564790in}}%
\pgfpathlineto{\pgfqpoint{2.097711in}{2.512879in}}%
\pgfpathlineto{\pgfqpoint{2.242297in}{2.462285in}}%
\pgfpathlineto{\pgfqpoint{2.413094in}{2.405991in}}%
\pgfpathlineto{\pgfqpoint{2.588307in}{2.351524in}}%
\pgfpathlineto{\pgfqpoint{2.763520in}{2.299948in}}%
\pgfpathlineto{\pgfqpoint{2.938732in}{2.250900in}}%
\pgfpathlineto{\pgfqpoint{3.148988in}{2.195147in}}%
\pgfpathlineto{\pgfqpoint{3.359243in}{2.142215in}}%
\pgfpathlineto{\pgfqpoint{3.604541in}{2.083675in}}%
\pgfpathlineto{\pgfqpoint{3.814796in}{2.035824in}}%
\pgfpathlineto{\pgfqpoint{4.060094in}{1.982523in}}%
\pgfpathlineto{\pgfqpoint{4.340434in}{1.924533in}}%
\pgfpathlineto{\pgfqpoint{4.585732in}{1.876012in}}%
\pgfpathlineto{\pgfqpoint{4.866072in}{1.822899in}}%
\pgfpathlineto{\pgfqpoint{5.146412in}{1.771998in}}%
\pgfpathlineto{\pgfqpoint{5.496838in}{1.711236in}}%
\pgfpathlineto{\pgfqpoint{5.742136in}{1.670328in}}%
\pgfpathlineto{\pgfqpoint{6.197689in}{1.597837in}}%
\pgfpathlineto{\pgfqpoint{6.394189in}{1.567851in}}%
\pgfpathlineto{\pgfqpoint{6.394189in}{1.567851in}}%
\pgfusepath{stroke}%
\end{pgfscope}%
\begin{pgfscope}%
\pgfpathrectangle{\pgfqpoint{0.766095in}{0.571603in}}{\pgfqpoint{6.973465in}{5.225635in}}%
\pgfusepath{clip}%
\pgfsetbuttcap%
\pgfsetroundjoin%
\pgfsetlinewidth{1.505625pt}%
\definecolor{currentstroke}{rgb}{0.281446,0.084320,0.407414}%
\pgfsetstrokecolor{currentstroke}%
\pgfsetdash{}{0pt}%
\pgfpathmoveto{\pgfqpoint{6.702111in}{1.522393in}}%
\pgfpathlineto{\pgfqpoint{6.723327in}{1.519265in}}%
\pgfpathlineto{\pgfqpoint{6.739061in}{1.516944in}}%
\pgfpathlineto{\pgfqpoint{6.758369in}{1.514216in}}%
\pgfpathlineto{\pgfqpoint{6.793412in}{1.509262in}}%
\pgfpathlineto{\pgfqpoint{6.828454in}{1.504278in}}%
\pgfpathlineto{\pgfqpoint{6.863497in}{1.499263in}}%
\pgfpathlineto{\pgfqpoint{6.898539in}{1.494216in}}%
\pgfpathlineto{\pgfqpoint{6.922991in}{1.490685in}}%
\pgfpathlineto{\pgfqpoint{6.933582in}{1.489220in}}%
\pgfpathlineto{\pgfqpoint{6.968624in}{1.484387in}}%
\pgfpathlineto{\pgfqpoint{7.003667in}{1.479526in}}%
\pgfpathlineto{\pgfqpoint{7.038709in}{1.474635in}}%
\pgfpathlineto{\pgfqpoint{7.073752in}{1.469713in}}%
\pgfpathlineto{\pgfqpoint{7.108795in}{1.464760in}}%
\pgfpathlineto{\pgfqpoint{7.111173in}{1.464425in}}%
\pgfpathlineto{\pgfqpoint{7.143837in}{1.460033in}}%
\pgfpathlineto{\pgfqpoint{7.178880in}{1.455298in}}%
\pgfpathlineto{\pgfqpoint{7.213922in}{1.450535in}}%
\pgfpathlineto{\pgfqpoint{7.248965in}{1.445742in}}%
\pgfpathlineto{\pgfqpoint{7.284007in}{1.440919in}}%
\pgfpathlineto{\pgfqpoint{7.303993in}{1.438166in}}%
\pgfpathlineto{\pgfqpoint{7.319050in}{1.436184in}}%
\pgfpathlineto{\pgfqpoint{7.354092in}{1.431579in}}%
\pgfpathlineto{\pgfqpoint{7.389135in}{1.426947in}}%
\pgfpathlineto{\pgfqpoint{7.424177in}{1.422288in}}%
\pgfpathlineto{\pgfqpoint{7.459220in}{1.417600in}}%
\pgfpathlineto{\pgfqpoint{7.494263in}{1.412883in}}%
\pgfpathlineto{\pgfqpoint{7.501542in}{1.411906in}}%
\pgfpathlineto{\pgfqpoint{7.529305in}{1.408355in}}%
\pgfpathlineto{\pgfqpoint{7.564348in}{1.403859in}}%
\pgfpathlineto{\pgfqpoint{7.599390in}{1.399337in}}%
\pgfpathlineto{\pgfqpoint{7.634433in}{1.394789in}}%
\pgfpathlineto{\pgfqpoint{7.669475in}{1.390214in}}%
\pgfpathlineto{\pgfqpoint{7.704232in}{1.385647in}}%
\pgfpathlineto{\pgfqpoint{7.704518in}{1.385611in}}%
\pgfpathlineto{\pgfqpoint{7.739560in}{1.381256in}}%
\pgfusepath{stroke}%
\end{pgfscope}%
\begin{pgfscope}%
\pgfpathrectangle{\pgfqpoint{0.766095in}{0.571603in}}{\pgfqpoint{6.973465in}{5.225635in}}%
\pgfusepath{clip}%
\pgfsetbuttcap%
\pgfsetroundjoin%
\pgfsetlinewidth{1.505625pt}%
\definecolor{currentstroke}{rgb}{0.281446,0.084320,0.407414}%
\pgfsetstrokecolor{currentstroke}%
\pgfsetdash{}{0pt}%
\pgfpathmoveto{\pgfqpoint{7.739560in}{2.642079in}}%
\pgfpathlineto{\pgfqpoint{7.730320in}{2.646102in}}%
\pgfpathlineto{\pgfqpoint{7.706754in}{2.656338in}}%
\pgfusepath{stroke}%
\end{pgfscope}%
\begin{pgfscope}%
\pgfpathrectangle{\pgfqpoint{0.766095in}{0.571603in}}{\pgfqpoint{6.973465in}{5.225635in}}%
\pgfusepath{clip}%
\pgfsetbuttcap%
\pgfsetroundjoin%
\pgfsetlinewidth{1.505625pt}%
\definecolor{currentstroke}{rgb}{0.281446,0.084320,0.407414}%
\pgfsetstrokecolor{currentstroke}%
\pgfsetdash{}{0pt}%
\pgfpathmoveto{\pgfqpoint{7.422660in}{2.784123in}}%
\pgfpathlineto{\pgfqpoint{7.270182in}{2.856177in}}%
\pgfpathlineto{\pgfqpoint{7.143837in}{2.918040in}}%
\pgfpathlineto{\pgfqpoint{7.003667in}{2.989281in}}%
\pgfpathlineto{\pgfqpoint{6.898539in}{3.044721in}}%
\pgfpathlineto{\pgfqpoint{6.763601in}{3.118772in}}%
\pgfpathlineto{\pgfqpoint{6.653242in}{3.182128in}}%
\pgfpathlineto{\pgfqpoint{6.548114in}{3.245129in}}%
\pgfpathlineto{\pgfqpoint{6.478029in}{3.288780in}}%
\pgfpathlineto{\pgfqpoint{6.407944in}{3.333890in}}%
\pgfpathlineto{\pgfqpoint{6.336768in}{3.381367in}}%
\pgfpathlineto{\pgfqpoint{6.261169in}{3.433886in}}%
\pgfpathlineto{\pgfqpoint{6.188882in}{3.486405in}}%
\pgfpathlineto{\pgfqpoint{6.119908in}{3.538924in}}%
\pgfpathlineto{\pgfqpoint{6.054257in}{3.591443in}}%
\pgfpathlineto{\pgfqpoint{5.987433in}{3.647885in}}%
\pgfpathlineto{\pgfqpoint{5.932836in}{3.696481in}}%
\pgfpathlineto{\pgfqpoint{5.877133in}{3.749000in}}%
\pgfpathlineto{\pgfqpoint{5.824698in}{3.801519in}}%
\pgfpathlineto{\pgfqpoint{5.775618in}{3.854037in}}%
\pgfpathlineto{\pgfqpoint{5.729760in}{3.906556in}}%
\pgfpathlineto{\pgfqpoint{5.687210in}{3.959075in}}%
\pgfpathlineto{\pgfqpoint{5.647929in}{4.011594in}}%
\pgfpathlineto{\pgfqpoint{5.611883in}{4.064113in}}%
\pgfpathlineto{\pgfqpoint{5.579043in}{4.116632in}}%
\pgfpathlineto{\pgfqpoint{5.549386in}{4.169151in}}%
\pgfpathlineto{\pgfqpoint{5.522895in}{4.221670in}}%
\pgfpathlineto{\pgfqpoint{5.499526in}{4.274189in}}%
\pgfpathlineto{\pgfqpoint{5.479162in}{4.326708in}}%
\pgfpathlineto{\pgfqpoint{5.461795in}{4.379583in}}%
\pgfpathlineto{\pgfqpoint{5.447490in}{4.431746in}}%
\pgfpathlineto{\pgfqpoint{5.436034in}{4.484265in}}%
\pgfpathlineto{\pgfqpoint{5.426753in}{4.542193in}}%
\pgfpathlineto{\pgfqpoint{5.421496in}{4.589303in}}%
\pgfpathlineto{\pgfqpoint{5.418229in}{4.641822in}}%
\pgfpathlineto{\pgfqpoint{5.417522in}{4.694341in}}%
\pgfpathlineto{\pgfqpoint{5.419262in}{4.746860in}}%
\pgfpathlineto{\pgfqpoint{5.423329in}{4.799378in}}%
\pgfpathlineto{\pgfqpoint{5.429558in}{4.851897in}}%
\pgfpathlineto{\pgfqpoint{5.437773in}{4.904416in}}%
\pgfpathlineto{\pgfqpoint{5.447859in}{4.956935in}}%
\pgfpathlineto{\pgfqpoint{5.466053in}{5.035714in}}%
\pgfpathlineto{\pgfqpoint{5.487119in}{5.114492in}}%
\pgfpathlineto{\pgfqpoint{5.518104in}{5.219530in}}%
\pgfpathlineto{\pgfqpoint{5.557195in}{5.350827in}}%
\pgfpathlineto{\pgfqpoint{5.571326in}{5.403346in}}%
\pgfpathlineto{\pgfqpoint{5.583493in}{5.455865in}}%
\pgfpathlineto{\pgfqpoint{5.592906in}{5.508384in}}%
\pgfpathlineto{\pgfqpoint{5.596217in}{5.534644in}}%
\pgfpathlineto{\pgfqpoint{5.598365in}{5.560903in}}%
\pgfpathlineto{\pgfqpoint{5.599143in}{5.587163in}}%
\pgfpathlineto{\pgfqpoint{5.598313in}{5.613422in}}%
\pgfpathlineto{\pgfqpoint{5.595598in}{5.639682in}}%
\pgfpathlineto{\pgfqpoint{5.590679in}{5.665941in}}%
\pgfpathlineto{\pgfqpoint{5.583180in}{5.692201in}}%
\pgfpathlineto{\pgfqpoint{5.572660in}{5.718460in}}%
\pgfpathlineto{\pgfqpoint{5.558398in}{5.744720in}}%
\pgfpathlineto{\pgfqpoint{5.539705in}{5.770979in}}%
\pgfpathlineto{\pgfqpoint{5.515497in}{5.797238in}}%
\pgfpathlineto{\pgfqpoint{5.515497in}{5.797238in}}%
\pgfusepath{stroke}%
\end{pgfscope}%
\begin{pgfscope}%
\pgfpathrectangle{\pgfqpoint{0.766095in}{0.571603in}}{\pgfqpoint{6.973465in}{5.225635in}}%
\pgfusepath{clip}%
\pgfsetbuttcap%
\pgfsetroundjoin%
\pgfsetlinewidth{1.505625pt}%
\definecolor{currentstroke}{rgb}{0.283229,0.120777,0.440584}%
\pgfsetstrokecolor{currentstroke}%
\pgfsetdash{}{0pt}%
\pgfpathmoveto{\pgfqpoint{1.827312in}{5.797238in}}%
\pgfpathlineto{\pgfqpoint{1.817371in}{5.794143in}}%
\pgfpathlineto{\pgfqpoint{1.790276in}{5.785643in}}%
\pgfusepath{stroke}%
\end{pgfscope}%
\begin{pgfscope}%
\pgfpathrectangle{\pgfqpoint{0.766095in}{0.571603in}}{\pgfqpoint{6.973465in}{5.225635in}}%
\pgfusepath{clip}%
\pgfsetbuttcap%
\pgfsetroundjoin%
\pgfsetlinewidth{1.505625pt}%
\definecolor{currentstroke}{rgb}{0.283229,0.120777,0.440584}%
\pgfsetstrokecolor{currentstroke}%
\pgfsetdash{}{0pt}%
\pgfpathmoveto{\pgfqpoint{1.495078in}{5.686586in}}%
\pgfpathlineto{\pgfqpoint{1.466946in}{5.676351in}}%
\pgfpathlineto{\pgfqpoint{1.438418in}{5.665941in}}%
\pgfpathlineto{\pgfqpoint{1.431903in}{5.663502in}}%
\pgfpathlineto{\pgfqpoint{1.396861in}{5.650280in}}%
\pgfpathlineto{\pgfqpoint{1.368860in}{5.639682in}}%
\pgfpathlineto{\pgfqpoint{1.361818in}{5.636948in}}%
\pgfpathlineto{\pgfqpoint{1.326776in}{5.623235in}}%
\pgfpathlineto{\pgfqpoint{1.301794in}{5.613422in}}%
\pgfpathlineto{\pgfqpoint{1.291733in}{5.609369in}}%
\pgfpathlineto{\pgfqpoint{1.256691in}{5.595148in}}%
\pgfpathlineto{\pgfqpoint{1.237112in}{5.587163in}}%
\pgfpathlineto{\pgfqpoint{1.221648in}{5.580693in}}%
\pgfpathlineto{\pgfqpoint{1.186605in}{5.565946in}}%
\pgfpathlineto{\pgfqpoint{1.174705in}{5.560903in}}%
\pgfpathlineto{\pgfqpoint{1.151563in}{5.550846in}}%
\pgfpathlineto{\pgfqpoint{1.116520in}{5.535552in}}%
\pgfpathlineto{\pgfqpoint{1.114459in}{5.534644in}}%
\pgfpathlineto{\pgfqpoint{1.081478in}{5.519745in}}%
\pgfpathlineto{\pgfqpoint{1.056437in}{5.508384in}}%
\pgfpathlineto{\pgfqpoint{1.046435in}{5.503730in}}%
\pgfpathlineto{\pgfqpoint{1.011393in}{5.487305in}}%
\pgfpathlineto{\pgfqpoint{1.000420in}{5.482125in}}%
\pgfpathlineto{\pgfqpoint{0.976350in}{5.470470in}}%
\pgfpathlineto{\pgfqpoint{0.946346in}{5.455865in}}%
\pgfpathlineto{\pgfqpoint{0.941308in}{5.453350in}}%
\pgfpathlineto{\pgfqpoint{0.906265in}{5.435704in}}%
\pgfpathlineto{\pgfqpoint{0.894237in}{5.429606in}}%
\pgfpathlineto{\pgfqpoint{0.871223in}{5.417636in}}%
\pgfpathlineto{\pgfqpoint{0.843917in}{5.403346in}}%
\pgfpathlineto{\pgfqpoint{0.836180in}{5.399192in}}%
\pgfpathlineto{\pgfqpoint{0.801138in}{5.380229in}}%
\pgfpathlineto{\pgfqpoint{0.795379in}{5.377087in}}%
\pgfpathlineto{\pgfqpoint{0.766095in}{5.360691in}}%
\pgfusepath{stroke}%
\end{pgfscope}%
\begin{pgfscope}%
\pgfpathrectangle{\pgfqpoint{0.766095in}{0.571603in}}{\pgfqpoint{6.973465in}{5.225635in}}%
\pgfusepath{clip}%
\pgfsetbuttcap%
\pgfsetroundjoin%
\pgfsetlinewidth{1.505625pt}%
\definecolor{currentstroke}{rgb}{0.283229,0.120777,0.440584}%
\pgfsetstrokecolor{currentstroke}%
\pgfsetdash{}{0pt}%
\pgfpathmoveto{\pgfqpoint{7.739560in}{2.996182in}}%
\pgfpathlineto{\pgfqpoint{7.705076in}{3.013734in}}%
\pgfpathlineto{\pgfqpoint{7.704518in}{3.014022in}}%
\pgfpathlineto{\pgfqpoint{7.702302in}{3.015163in}}%
\pgfusepath{stroke}%
\end{pgfscope}%
\begin{pgfscope}%
\pgfpathrectangle{\pgfqpoint{0.766095in}{0.571603in}}{\pgfqpoint{6.973465in}{5.225635in}}%
\pgfusepath{clip}%
\pgfsetbuttcap%
\pgfsetroundjoin%
\pgfsetlinewidth{1.505625pt}%
\definecolor{currentstroke}{rgb}{0.283229,0.120777,0.440584}%
\pgfsetstrokecolor{currentstroke}%
\pgfsetdash{}{0pt}%
\pgfpathmoveto{\pgfqpoint{7.427945in}{3.162870in}}%
\pgfpathlineto{\pgfqpoint{7.354092in}{3.205392in}}%
\pgfpathlineto{\pgfqpoint{7.248965in}{3.268386in}}%
\pgfpathlineto{\pgfqpoint{7.178880in}{3.312184in}}%
\pgfpathlineto{\pgfqpoint{7.108795in}{3.357585in}}%
\pgfpathlineto{\pgfqpoint{7.034619in}{3.407626in}}%
\pgfpathlineto{\pgfqpoint{6.960271in}{3.460145in}}%
\pgfpathlineto{\pgfqpoint{6.889496in}{3.512664in}}%
\pgfpathlineto{\pgfqpoint{6.822294in}{3.565183in}}%
\pgfpathlineto{\pgfqpoint{6.758369in}{3.617962in}}%
\pgfpathlineto{\pgfqpoint{6.688284in}{3.679633in}}%
\pgfpathlineto{\pgfqpoint{6.642043in}{3.722740in}}%
\pgfpathlineto{\pgfqpoint{6.583156in}{3.781470in}}%
\pgfpathlineto{\pgfqpoint{6.539680in}{3.827778in}}%
\pgfpathlineto{\pgfqpoint{6.493805in}{3.880297in}}%
\pgfpathlineto{\pgfqpoint{6.451482in}{3.932816in}}%
\pgfpathlineto{\pgfqpoint{6.407944in}{3.992230in}}%
\pgfpathlineto{\pgfqpoint{6.377374in}{4.037854in}}%
\pgfpathlineto{\pgfqpoint{6.345545in}{4.090373in}}%
\pgfpathlineto{\pgfqpoint{6.317183in}{4.142892in}}%
\pgfpathlineto{\pgfqpoint{6.292281in}{4.195411in}}%
\pgfpathlineto{\pgfqpoint{6.270813in}{4.247930in}}%
\pgfpathlineto{\pgfqpoint{6.252689in}{4.300449in}}%
\pgfpathlineto{\pgfqpoint{6.237966in}{4.352967in}}%
\pgfpathlineto{\pgfqpoint{6.226543in}{4.405486in}}%
\pgfpathlineto{\pgfqpoint{6.218389in}{4.458005in}}%
\pgfpathlineto{\pgfqpoint{6.213486in}{4.510524in}}%
\pgfpathlineto{\pgfqpoint{6.211773in}{4.563043in}}%
\pgfpathlineto{\pgfqpoint{6.213195in}{4.615562in}}%
\pgfpathlineto{\pgfqpoint{6.217694in}{4.668081in}}%
\pgfpathlineto{\pgfqpoint{6.225220in}{4.720600in}}%
\pgfpathlineto{\pgfqpoint{6.235692in}{4.773119in}}%
\pgfpathlineto{\pgfqpoint{6.248989in}{4.825638in}}%
\pgfpathlineto{\pgfqpoint{6.267774in}{4.885812in}}%
\pgfpathlineto{\pgfqpoint{6.283903in}{4.930676in}}%
\pgfpathlineto{\pgfqpoint{6.305332in}{4.983195in}}%
\pgfpathlineto{\pgfqpoint{6.329196in}{5.035714in}}%
\pgfpathlineto{\pgfqpoint{6.355436in}{5.088233in}}%
\pgfpathlineto{\pgfqpoint{6.383928in}{5.140752in}}%
\pgfpathlineto{\pgfqpoint{6.414515in}{5.193271in}}%
\pgfpathlineto{\pgfqpoint{6.447031in}{5.245790in}}%
\pgfpathlineto{\pgfqpoint{6.498992in}{5.324568in}}%
\pgfpathlineto{\pgfqpoint{6.554245in}{5.403346in}}%
\pgfpathlineto{\pgfqpoint{6.631491in}{5.508384in}}%
\pgfpathlineto{\pgfqpoint{6.788429in}{5.718460in}}%
\pgfpathlineto{\pgfqpoint{6.843242in}{5.797238in}}%
\pgfpathlineto{\pgfqpoint{6.843242in}{5.797238in}}%
\pgfusepath{stroke}%
\end{pgfscope}%
\begin{pgfscope}%
\pgfpathrectangle{\pgfqpoint{0.766095in}{0.571603in}}{\pgfqpoint{6.973465in}{5.225635in}}%
\pgfusepath{clip}%
\pgfsetbuttcap%
\pgfsetroundjoin%
\pgfsetlinewidth{1.505625pt}%
\definecolor{currentstroke}{rgb}{0.283229,0.120777,0.440584}%
\pgfsetstrokecolor{currentstroke}%
\pgfsetdash{}{0pt}%
\pgfpathmoveto{\pgfqpoint{0.766095in}{2.828254in}}%
\pgfpathlineto{\pgfqpoint{0.858047in}{2.777399in}}%
\pgfpathlineto{\pgfqpoint{0.959193in}{2.724880in}}%
\pgfpathlineto{\pgfqpoint{1.066906in}{2.672361in}}%
\pgfpathlineto{\pgfqpoint{1.186605in}{2.617581in}}%
\pgfpathlineto{\pgfqpoint{1.303223in}{2.567323in}}%
\pgfpathlineto{\pgfqpoint{1.432380in}{2.514804in}}%
\pgfpathlineto{\pgfqpoint{1.572073in}{2.461288in}}%
\pgfpathlineto{\pgfqpoint{1.714393in}{2.409766in}}%
\pgfpathlineto{\pgfqpoint{1.887456in}{2.350800in}}%
\pgfpathlineto{\pgfqpoint{2.029811in}{2.304729in}}%
\pgfpathlineto{\pgfqpoint{2.237882in}{2.241271in}}%
\pgfpathlineto{\pgfqpoint{2.380874in}{2.199691in}}%
\pgfpathlineto{\pgfqpoint{2.588307in}{2.142411in}}%
\pgfpathlineto{\pgfqpoint{2.798562in}{2.087307in}}%
\pgfpathlineto{\pgfqpoint{3.008818in}{2.034823in}}%
\pgfpathlineto{\pgfqpoint{3.219073in}{1.984650in}}%
\pgfpathlineto{\pgfqpoint{3.464371in}{1.928778in}}%
\pgfpathlineto{\pgfqpoint{3.709668in}{1.875364in}}%
\pgfpathlineto{\pgfqpoint{3.954966in}{1.824133in}}%
\pgfpathlineto{\pgfqpoint{4.235306in}{1.767999in}}%
\pgfpathlineto{\pgfqpoint{4.515647in}{1.714115in}}%
\pgfpathlineto{\pgfqpoint{4.795987in}{1.662250in}}%
\pgfpathlineto{\pgfqpoint{5.076327in}{1.612195in}}%
\pgfpathlineto{\pgfqpoint{5.426753in}{1.551926in}}%
\pgfpathlineto{\pgfqpoint{5.672050in}{1.511095in}}%
\pgfpathlineto{\pgfqpoint{6.127603in}{1.437935in}}%
\pgfpathlineto{\pgfqpoint{6.723327in}{1.347281in}}%
\pgfpathlineto{\pgfqpoint{7.108795in}{1.291095in}}%
\pgfpathlineto{\pgfqpoint{7.164950in}{1.282966in}}%
\pgfpathlineto{\pgfqpoint{7.164950in}{1.282966in}}%
\pgfusepath{stroke}%
\end{pgfscope}%
\begin{pgfscope}%
\pgfpathrectangle{\pgfqpoint{0.766095in}{0.571603in}}{\pgfqpoint{6.973465in}{5.225635in}}%
\pgfusepath{clip}%
\pgfsetbuttcap%
\pgfsetroundjoin%
\pgfsetlinewidth{1.505625pt}%
\definecolor{currentstroke}{rgb}{0.283229,0.120777,0.440584}%
\pgfsetstrokecolor{currentstroke}%
\pgfsetdash{}{0pt}%
\pgfpathmoveto{\pgfqpoint{7.473182in}{1.239685in}}%
\pgfpathlineto{\pgfqpoint{7.494263in}{1.236746in}}%
\pgfpathlineto{\pgfqpoint{7.529305in}{1.231831in}}%
\pgfpathlineto{\pgfqpoint{7.555868in}{1.228090in}}%
\pgfpathlineto{\pgfqpoint{7.564348in}{1.226942in}}%
\pgfpathlineto{\pgfqpoint{7.599390in}{1.222202in}}%
\pgfpathlineto{\pgfqpoint{7.634433in}{1.217435in}}%
\pgfpathlineto{\pgfqpoint{7.669475in}{1.212639in}}%
\pgfpathlineto{\pgfqpoint{7.704518in}{1.207814in}}%
\pgfpathlineto{\pgfqpoint{7.739560in}{1.202958in}}%
\pgfusepath{stroke}%
\end{pgfscope}%
\begin{pgfscope}%
\pgfpathrectangle{\pgfqpoint{0.766095in}{0.571603in}}{\pgfqpoint{6.973465in}{5.225635in}}%
\pgfusepath{clip}%
\pgfsetbuttcap%
\pgfsetroundjoin%
\pgfsetlinewidth{1.505625pt}%
\definecolor{currentstroke}{rgb}{0.281412,0.155834,0.469201}%
\pgfsetstrokecolor{currentstroke}%
\pgfsetdash{}{0pt}%
\pgfpathmoveto{\pgfqpoint{1.117112in}{5.797238in}}%
\pgfpathlineto{\pgfqpoint{1.116520in}{5.797019in}}%
\pgfpathlineto{\pgfqpoint{1.086482in}{5.785760in}}%
\pgfusepath{stroke}%
\end{pgfscope}%
\begin{pgfscope}%
\pgfpathrectangle{\pgfqpoint{0.766095in}{0.571603in}}{\pgfqpoint{6.973465in}{5.225635in}}%
\pgfusepath{clip}%
\pgfsetbuttcap%
\pgfsetroundjoin%
\pgfsetlinewidth{1.505625pt}%
\definecolor{currentstroke}{rgb}{0.281412,0.155834,0.469201}%
\pgfsetstrokecolor{currentstroke}%
\pgfsetdash{}{0pt}%
\pgfpathmoveto{\pgfqpoint{0.797628in}{5.669347in}}%
\pgfpathlineto{\pgfqpoint{0.789754in}{5.665941in}}%
\pgfpathlineto{\pgfqpoint{0.766095in}{5.655454in}}%
\pgfusepath{stroke}%
\end{pgfscope}%
\begin{pgfscope}%
\pgfpathrectangle{\pgfqpoint{0.766095in}{0.571603in}}{\pgfqpoint{6.973465in}{5.225635in}}%
\pgfusepath{clip}%
\pgfsetbuttcap%
\pgfsetroundjoin%
\pgfsetlinewidth{1.505625pt}%
\definecolor{currentstroke}{rgb}{0.281412,0.155834,0.469201}%
\pgfsetstrokecolor{currentstroke}%
\pgfsetdash{}{0pt}%
\pgfpathmoveto{\pgfqpoint{7.739560in}{3.353965in}}%
\pgfpathlineto{\pgfqpoint{7.737794in}{3.355107in}}%
\pgfpathlineto{\pgfqpoint{7.704518in}{3.377098in}}%
\pgfpathlineto{\pgfqpoint{7.698068in}{3.381367in}}%
\pgfpathlineto{\pgfqpoint{7.669475in}{3.400717in}}%
\pgfpathlineto{\pgfqpoint{7.659277in}{3.407626in}}%
\pgfpathlineto{\pgfqpoint{7.634433in}{3.424855in}}%
\pgfpathlineto{\pgfqpoint{7.621422in}{3.433886in}}%
\pgfpathlineto{\pgfqpoint{7.599390in}{3.449550in}}%
\pgfpathlineto{\pgfqpoint{7.584499in}{3.460145in}}%
\pgfpathlineto{\pgfqpoint{7.564348in}{3.474845in}}%
\pgfpathlineto{\pgfqpoint{7.548509in}{3.486405in}}%
\pgfpathlineto{\pgfqpoint{7.529305in}{3.500787in}}%
\pgfpathlineto{\pgfqpoint{7.513450in}{3.512664in}}%
\pgfpathlineto{\pgfqpoint{7.494263in}{3.527428in}}%
\pgfpathlineto{\pgfqpoint{7.479322in}{3.538924in}}%
\pgfpathlineto{\pgfqpoint{7.459220in}{3.554827in}}%
\pgfpathlineto{\pgfqpoint{7.446126in}{3.565183in}}%
\pgfpathlineto{\pgfqpoint{7.424177in}{3.583051in}}%
\pgfpathlineto{\pgfqpoint{7.413864in}{3.591443in}}%
\pgfpathlineto{\pgfqpoint{7.389135in}{3.612173in}}%
\pgfpathlineto{\pgfqpoint{7.382535in}{3.617702in}}%
\pgfpathlineto{\pgfqpoint{7.354092in}{3.642278in}}%
\pgfpathlineto{\pgfqpoint{7.352141in}{3.643962in}}%
\pgfpathlineto{\pgfqpoint{7.322660in}{3.670221in}}%
\pgfpathlineto{\pgfqpoint{7.319050in}{3.673545in}}%
\pgfpathlineto{\pgfqpoint{7.294098in}{3.696481in}}%
\pgfpathlineto{\pgfqpoint{7.284007in}{3.706084in}}%
\pgfpathlineto{\pgfqpoint{7.266474in}{3.722740in}}%
\pgfpathlineto{\pgfqpoint{7.248965in}{3.739983in}}%
\pgfpathlineto{\pgfqpoint{7.239790in}{3.749000in}}%
\pgfpathlineto{\pgfqpoint{7.214050in}{3.775259in}}%
\pgfpathlineto{\pgfqpoint{7.213922in}{3.775394in}}%
\pgfpathlineto{\pgfqpoint{7.189186in}{3.801519in}}%
\pgfpathlineto{\pgfqpoint{7.178880in}{3.812852in}}%
\pgfpathlineto{\pgfqpoint{7.165269in}{3.827778in}}%
\pgfpathlineto{\pgfqpoint{7.143837in}{3.852291in}}%
\pgfpathlineto{\pgfqpoint{7.142305in}{3.854037in}}%
\pgfpathlineto{\pgfqpoint{7.120217in}{3.880297in}}%
\pgfpathlineto{\pgfqpoint{7.108795in}{3.894507in}}%
\pgfpathlineto{\pgfqpoint{7.099075in}{3.906556in}}%
\pgfpathlineto{\pgfqpoint{7.078861in}{3.932816in}}%
\pgfpathlineto{\pgfqpoint{7.073752in}{3.939780in}}%
\pgfpathlineto{\pgfqpoint{7.059540in}{3.959075in}}%
\pgfpathlineto{\pgfqpoint{7.041175in}{3.985335in}}%
\pgfpathlineto{\pgfqpoint{7.038709in}{3.989049in}}%
\pgfpathlineto{\pgfqpoint{7.023679in}{4.011594in}}%
\pgfpathlineto{\pgfqpoint{7.007141in}{4.037854in}}%
\pgfpathlineto{\pgfqpoint{7.003667in}{4.043699in}}%
\pgfpathlineto{\pgfqpoint{6.991476in}{4.064113in}}%
\pgfpathlineto{\pgfqpoint{6.976745in}{4.090373in}}%
\pgfpathlineto{\pgfqpoint{6.968624in}{4.105833in}}%
\pgfpathlineto{\pgfqpoint{6.962921in}{4.116632in}}%
\pgfpathlineto{\pgfqpoint{6.949982in}{4.142892in}}%
\pgfpathlineto{\pgfqpoint{6.937979in}{4.169151in}}%
\pgfpathlineto{\pgfqpoint{6.933582in}{4.179568in}}%
\pgfpathlineto{\pgfqpoint{6.926852in}{4.195411in}}%
\pgfpathlineto{\pgfqpoint{6.916617in}{4.221670in}}%
\pgfpathlineto{\pgfqpoint{6.907296in}{4.247930in}}%
\pgfpathlineto{\pgfqpoint{6.898883in}{4.274189in}}%
\pgfpathlineto{\pgfqpoint{6.898539in}{4.275390in}}%
\pgfpathlineto{\pgfqpoint{6.891315in}{4.300449in}}%
\pgfpathlineto{\pgfqpoint{6.884643in}{4.326708in}}%
\pgfpathlineto{\pgfqpoint{6.878861in}{4.352967in}}%
\pgfpathlineto{\pgfqpoint{6.873964in}{4.379227in}}%
\pgfpathlineto{\pgfqpoint{6.869945in}{4.405486in}}%
\pgfpathlineto{\pgfqpoint{6.866797in}{4.431746in}}%
\pgfpathlineto{\pgfqpoint{6.864514in}{4.458005in}}%
\pgfpathlineto{\pgfqpoint{6.863497in}{4.476793in}}%
\pgfpathlineto{\pgfqpoint{6.863089in}{4.484265in}}%
\pgfpathlineto{\pgfqpoint{6.862516in}{4.510524in}}%
\pgfpathlineto{\pgfqpoint{6.862801in}{4.536784in}}%
\pgfpathlineto{\pgfqpoint{6.863497in}{4.552895in}}%
\pgfpathlineto{\pgfqpoint{6.863933in}{4.563043in}}%
\pgfpathlineto{\pgfqpoint{6.865901in}{4.589303in}}%
\pgfpathlineto{\pgfqpoint{6.868705in}{4.615562in}}%
\pgfpathlineto{\pgfqpoint{6.872341in}{4.641822in}}%
\pgfpathlineto{\pgfqpoint{6.876806in}{4.668081in}}%
\pgfpathlineto{\pgfqpoint{6.882096in}{4.694341in}}%
\pgfpathlineto{\pgfqpoint{6.888208in}{4.720600in}}%
\pgfpathlineto{\pgfqpoint{6.895139in}{4.746860in}}%
\pgfpathlineto{\pgfqpoint{6.898539in}{4.758415in}}%
\pgfpathlineto{\pgfqpoint{6.902852in}{4.773119in}}%
\pgfpathlineto{\pgfqpoint{6.911346in}{4.799378in}}%
\pgfpathlineto{\pgfqpoint{6.920643in}{4.825638in}}%
\pgfpathlineto{\pgfqpoint{6.930741in}{4.851897in}}%
\pgfpathlineto{\pgfqpoint{6.933582in}{4.858771in}}%
\pgfpathlineto{\pgfqpoint{6.941576in}{4.878157in}}%
\pgfpathlineto{\pgfqpoint{6.953178in}{4.904416in}}%
\pgfpathlineto{\pgfqpoint{6.965568in}{4.930676in}}%
\pgfpathlineto{\pgfqpoint{6.968624in}{4.936797in}}%
\pgfpathlineto{\pgfqpoint{6.978665in}{4.956935in}}%
\pgfpathlineto{\pgfqpoint{6.992514in}{4.983195in}}%
\pgfpathlineto{\pgfqpoint{7.003667in}{5.003255in}}%
\pgfpathlineto{\pgfqpoint{7.007110in}{5.009454in}}%
\pgfpathlineto{\pgfqpoint{7.022384in}{5.035714in}}%
\pgfpathlineto{\pgfqpoint{7.038422in}{5.061973in}}%
\pgfpathlineto{\pgfqpoint{7.038709in}{5.062427in}}%
\pgfpathlineto{\pgfqpoint{7.055089in}{5.088233in}}%
\pgfpathlineto{\pgfqpoint{7.072505in}{5.114492in}}%
\pgfpathlineto{\pgfqpoint{7.073752in}{5.116308in}}%
\pgfpathlineto{\pgfqpoint{7.090536in}{5.140752in}}%
\pgfpathlineto{\pgfqpoint{7.108795in}{5.166316in}}%
\pgfpathlineto{\pgfqpoint{7.109292in}{5.167011in}}%
\pgfpathlineto{\pgfqpoint{7.128631in}{5.193271in}}%
\pgfpathlineto{\pgfqpoint{7.143837in}{5.213215in}}%
\pgfpathlineto{\pgfqpoint{7.148657in}{5.219530in}}%
\pgfpathlineto{\pgfqpoint{7.169277in}{5.245790in}}%
\pgfpathlineto{\pgfqpoint{7.178880in}{5.257675in}}%
\pgfpathlineto{\pgfqpoint{7.190513in}{5.272049in}}%
\pgfpathlineto{\pgfqpoint{7.212372in}{5.298308in}}%
\pgfpathlineto{\pgfqpoint{7.213922in}{5.300131in}}%
\pgfpathlineto{\pgfqpoint{7.234754in}{5.324568in}}%
\pgfpathlineto{\pgfqpoint{7.248965in}{5.340823in}}%
\pgfpathlineto{\pgfqpoint{7.257735in}{5.350827in}}%
\pgfpathlineto{\pgfqpoint{7.281269in}{5.377087in}}%
\pgfpathlineto{\pgfqpoint{7.284007in}{5.380089in}}%
\pgfpathlineto{\pgfqpoint{7.305288in}{5.403346in}}%
\pgfpathlineto{\pgfqpoint{7.319050in}{5.418086in}}%
\pgfpathlineto{\pgfqpoint{7.329844in}{5.429606in}}%
\pgfpathlineto{\pgfqpoint{7.349227in}{5.449911in}}%
\pgfusepath{stroke}%
\end{pgfscope}%
\begin{pgfscope}%
\pgfpathrectangle{\pgfqpoint{0.766095in}{0.571603in}}{\pgfqpoint{6.973465in}{5.225635in}}%
\pgfusepath{clip}%
\pgfsetbuttcap%
\pgfsetroundjoin%
\pgfsetlinewidth{1.505625pt}%
\definecolor{currentstroke}{rgb}{0.281412,0.155834,0.469201}%
\pgfsetstrokecolor{currentstroke}%
\pgfsetdash{}{0pt}%
\pgfpathmoveto{\pgfqpoint{7.572360in}{5.668029in}}%
\pgfpathlineto{\pgfqpoint{7.598490in}{5.692201in}}%
\pgfpathlineto{\pgfqpoint{7.599390in}{5.693034in}}%
\pgfpathlineto{\pgfqpoint{7.627062in}{5.718460in}}%
\pgfpathlineto{\pgfqpoint{7.634433in}{5.725225in}}%
\pgfpathlineto{\pgfqpoint{7.655835in}{5.744720in}}%
\pgfpathlineto{\pgfqpoint{7.669475in}{5.757151in}}%
\pgfpathlineto{\pgfqpoint{7.684770in}{5.770979in}}%
\pgfpathlineto{\pgfqpoint{7.704518in}{5.788874in}}%
\pgfpathlineto{\pgfqpoint{7.713826in}{5.797238in}}%
\pgfusepath{stroke}%
\end{pgfscope}%
\begin{pgfscope}%
\pgfpathrectangle{\pgfqpoint{0.766095in}{0.571603in}}{\pgfqpoint{6.973465in}{5.225635in}}%
\pgfusepath{clip}%
\pgfsetbuttcap%
\pgfsetroundjoin%
\pgfsetlinewidth{1.505625pt}%
\definecolor{currentstroke}{rgb}{0.281412,0.155834,0.469201}%
\pgfsetstrokecolor{currentstroke}%
\pgfsetdash{}{0pt}%
\pgfpathmoveto{\pgfqpoint{0.766095in}{2.577141in}}%
\pgfpathlineto{\pgfqpoint{0.846638in}{2.541064in}}%
\pgfpathlineto{\pgfqpoint{0.976350in}{2.486017in}}%
\pgfpathlineto{\pgfqpoint{1.116520in}{2.430099in}}%
\pgfpathlineto{\pgfqpoint{1.256691in}{2.377353in}}%
\pgfpathlineto{\pgfqpoint{1.396861in}{2.327386in}}%
\pgfpathlineto{\pgfqpoint{1.541303in}{2.278469in}}%
\pgfpathlineto{\pgfqpoint{1.712244in}{2.223651in}}%
\pgfpathlineto{\pgfqpoint{1.887456in}{2.170401in}}%
\pgfpathlineto{\pgfqpoint{2.062669in}{2.119745in}}%
\pgfpathlineto{\pgfqpoint{2.272924in}{2.062096in}}%
\pgfpathlineto{\pgfqpoint{2.483179in}{2.007316in}}%
\pgfpathlineto{\pgfqpoint{2.693435in}{1.955073in}}%
\pgfpathlineto{\pgfqpoint{2.903690in}{1.905067in}}%
\pgfpathlineto{\pgfqpoint{3.148988in}{1.849292in}}%
\pgfpathlineto{\pgfqpoint{3.394285in}{1.795895in}}%
\pgfpathlineto{\pgfqpoint{3.639583in}{1.744609in}}%
\pgfpathlineto{\pgfqpoint{3.919924in}{1.688316in}}%
\pgfpathlineto{\pgfqpoint{4.200264in}{1.634188in}}%
\pgfpathlineto{\pgfqpoint{4.480604in}{1.581998in}}%
\pgfpathlineto{\pgfqpoint{4.760944in}{1.531540in}}%
\pgfpathlineto{\pgfqpoint{5.076327in}{1.476639in}}%
\pgfpathlineto{\pgfqpoint{5.391710in}{1.423510in}}%
\pgfpathlineto{\pgfqpoint{5.707093in}{1.371988in}}%
\pgfpathlineto{\pgfqpoint{6.057518in}{1.316434in}}%
\pgfpathlineto{\pgfqpoint{6.372901in}{1.267856in}}%
\pgfpathlineto{\pgfqpoint{6.779590in}{1.206938in}}%
\pgfpathlineto{\pgfqpoint{6.779590in}{1.206938in}}%
\pgfusepath{stroke}%
\end{pgfscope}%
\begin{pgfscope}%
\pgfpathrectangle{\pgfqpoint{0.766095in}{0.571603in}}{\pgfqpoint{6.973465in}{5.225635in}}%
\pgfusepath{clip}%
\pgfsetbuttcap%
\pgfsetroundjoin%
\pgfsetlinewidth{1.505625pt}%
\definecolor{currentstroke}{rgb}{0.281412,0.155834,0.469201}%
\pgfsetstrokecolor{currentstroke}%
\pgfsetdash{}{0pt}%
\pgfpathmoveto{\pgfqpoint{7.087606in}{1.162131in}}%
\pgfpathlineto{\pgfqpoint{7.108795in}{1.159076in}}%
\pgfpathlineto{\pgfqpoint{7.143837in}{1.153997in}}%
\pgfpathlineto{\pgfqpoint{7.176001in}{1.149312in}}%
\pgfpathlineto{\pgfqpoint{7.178880in}{1.148906in}}%
\pgfpathlineto{\pgfqpoint{7.213922in}{1.143981in}}%
\pgfpathlineto{\pgfqpoint{7.248965in}{1.139029in}}%
\pgfpathlineto{\pgfqpoint{7.284007in}{1.134050in}}%
\pgfpathlineto{\pgfqpoint{7.319050in}{1.129043in}}%
\pgfpathlineto{\pgfqpoint{7.354092in}{1.124009in}}%
\pgfpathlineto{\pgfqpoint{7.360754in}{1.123052in}}%
\pgfpathlineto{\pgfqpoint{7.389135in}{1.119116in}}%
\pgfpathlineto{\pgfqpoint{7.424177in}{1.114236in}}%
\pgfpathlineto{\pgfqpoint{7.459220in}{1.109330in}}%
\pgfpathlineto{\pgfqpoint{7.494263in}{1.104397in}}%
\pgfpathlineto{\pgfqpoint{7.529305in}{1.099436in}}%
\pgfpathlineto{\pgfqpoint{7.547938in}{1.096793in}}%
\pgfpathlineto{\pgfqpoint{7.564348in}{1.094545in}}%
\pgfpathlineto{\pgfqpoint{7.599390in}{1.089738in}}%
\pgfpathlineto{\pgfqpoint{7.634433in}{1.084905in}}%
\pgfpathlineto{\pgfqpoint{7.669475in}{1.080046in}}%
\pgfpathlineto{\pgfqpoint{7.704518in}{1.075159in}}%
\pgfpathlineto{\pgfqpoint{7.737508in}{1.070533in}}%
\pgfpathlineto{\pgfqpoint{7.739560in}{1.070256in}}%
\pgfusepath{stroke}%
\end{pgfscope}%
\begin{pgfscope}%
\pgfpathrectangle{\pgfqpoint{0.766095in}{0.571603in}}{\pgfqpoint{6.973465in}{5.225635in}}%
\pgfusepath{clip}%
\pgfsetbuttcap%
\pgfsetroundjoin%
\pgfsetlinewidth{1.505625pt}%
\definecolor{currentstroke}{rgb}{0.276194,0.190074,0.493001}%
\pgfsetstrokecolor{currentstroke}%
\pgfsetdash{}{0pt}%
\pgfpathmoveto{\pgfqpoint{7.739560in}{3.804502in}}%
\pgfpathlineto{\pgfqpoint{7.718520in}{3.827778in}}%
\pgfpathlineto{\pgfqpoint{7.704518in}{3.843954in}}%
\pgfpathlineto{\pgfqpoint{7.695766in}{3.854037in}}%
\pgfpathlineto{\pgfqpoint{7.673965in}{3.880297in}}%
\pgfpathlineto{\pgfqpoint{7.669475in}{3.885958in}}%
\pgfpathlineto{\pgfqpoint{7.653089in}{3.906556in}}%
\pgfpathlineto{\pgfqpoint{7.634433in}{3.931194in}}%
\pgfpathlineto{\pgfqpoint{7.633201in}{3.932816in}}%
\pgfpathlineto{\pgfqpoint{7.614210in}{3.959075in}}%
\pgfpathlineto{\pgfqpoint{7.599390in}{3.980706in}}%
\pgfpathlineto{\pgfqpoint{7.596207in}{3.985335in}}%
\pgfpathlineto{\pgfqpoint{7.579110in}{4.011594in}}%
\pgfpathlineto{\pgfqpoint{7.564348in}{4.035661in}}%
\pgfpathlineto{\pgfqpoint{7.562997in}{4.037854in}}%
\pgfpathlineto{\pgfqpoint{7.547776in}{4.064113in}}%
\pgfpathlineto{\pgfqpoint{7.533534in}{4.090373in}}%
\pgfpathlineto{\pgfqpoint{7.529305in}{4.098731in}}%
\pgfpathlineto{\pgfqpoint{7.520203in}{4.116632in}}%
\pgfpathlineto{\pgfqpoint{7.507810in}{4.142892in}}%
\pgfpathlineto{\pgfqpoint{7.496375in}{4.169151in}}%
\pgfpathlineto{\pgfqpoint{7.494263in}{4.174438in}}%
\pgfpathlineto{\pgfqpoint{7.485836in}{4.195411in}}%
\pgfpathlineto{\pgfqpoint{7.476231in}{4.221670in}}%
\pgfpathlineto{\pgfqpoint{7.467566in}{4.247930in}}%
\pgfpathlineto{\pgfqpoint{7.459836in}{4.274189in}}%
\pgfpathlineto{\pgfqpoint{7.459220in}{4.276565in}}%
\pgfpathlineto{\pgfqpoint{7.452991in}{4.300449in}}%
\pgfpathlineto{\pgfqpoint{7.447070in}{4.326708in}}%
\pgfpathlineto{\pgfqpoint{7.442070in}{4.352967in}}%
\pgfpathlineto{\pgfqpoint{7.437986in}{4.379227in}}%
\pgfpathlineto{\pgfqpoint{7.434811in}{4.405486in}}%
\pgfpathlineto{\pgfqpoint{7.432542in}{4.431746in}}%
\pgfpathlineto{\pgfqpoint{7.431174in}{4.458005in}}%
\pgfpathlineto{\pgfqpoint{7.430702in}{4.484265in}}%
\pgfpathlineto{\pgfqpoint{7.431123in}{4.510524in}}%
\pgfpathlineto{\pgfqpoint{7.432433in}{4.536784in}}%
\pgfpathlineto{\pgfqpoint{7.434628in}{4.563043in}}%
\pgfpathlineto{\pgfqpoint{7.437707in}{4.589303in}}%
\pgfpathlineto{\pgfqpoint{7.441665in}{4.615562in}}%
\pgfpathlineto{\pgfqpoint{7.446501in}{4.641822in}}%
\pgfpathlineto{\pgfqpoint{7.452212in}{4.668081in}}%
\pgfpathlineto{\pgfqpoint{7.458797in}{4.694341in}}%
\pgfpathlineto{\pgfqpoint{7.459220in}{4.695834in}}%
\pgfpathlineto{\pgfqpoint{7.466207in}{4.720600in}}%
\pgfpathlineto{\pgfqpoint{7.474478in}{4.746860in}}%
\pgfpathlineto{\pgfqpoint{7.483612in}{4.773119in}}%
\pgfpathlineto{\pgfqpoint{7.493609in}{4.799378in}}%
\pgfpathlineto{\pgfqpoint{7.494263in}{4.800965in}}%
\pgfpathlineto{\pgfqpoint{7.504399in}{4.825638in}}%
\pgfpathlineto{\pgfqpoint{7.516039in}{4.851897in}}%
\pgfpathlineto{\pgfqpoint{7.526286in}{4.873432in}}%
\pgfusepath{stroke}%
\end{pgfscope}%
\begin{pgfscope}%
\pgfpathrectangle{\pgfqpoint{0.766095in}{0.571603in}}{\pgfqpoint{6.973465in}{5.225635in}}%
\pgfusepath{clip}%
\pgfsetbuttcap%
\pgfsetroundjoin%
\pgfsetlinewidth{1.505625pt}%
\definecolor{currentstroke}{rgb}{0.276194,0.190074,0.493001}%
\pgfsetstrokecolor{currentstroke}%
\pgfsetdash{}{0pt}%
\pgfpathmoveto{\pgfqpoint{7.693814in}{5.136482in}}%
\pgfpathlineto{\pgfqpoint{7.697121in}{5.140752in}}%
\pgfpathlineto{\pgfqpoint{7.704518in}{5.149973in}}%
\pgfpathlineto{\pgfqpoint{7.718194in}{5.167011in}}%
\pgfpathlineto{\pgfqpoint{7.739560in}{5.192699in}}%
\pgfusepath{stroke}%
\end{pgfscope}%
\begin{pgfscope}%
\pgfpathrectangle{\pgfqpoint{0.766095in}{0.571603in}}{\pgfqpoint{6.973465in}{5.225635in}}%
\pgfusepath{clip}%
\pgfsetbuttcap%
\pgfsetroundjoin%
\pgfsetlinewidth{1.505625pt}%
\definecolor{currentstroke}{rgb}{0.276194,0.190074,0.493001}%
\pgfsetstrokecolor{currentstroke}%
\pgfsetdash{}{0pt}%
\pgfpathmoveto{\pgfqpoint{0.766095in}{2.388388in}}%
\pgfpathlineto{\pgfqpoint{0.846186in}{2.357248in}}%
\pgfpathlineto{\pgfqpoint{0.987669in}{2.304729in}}%
\pgfpathlineto{\pgfqpoint{1.151563in}{2.247360in}}%
\pgfpathlineto{\pgfqpoint{1.295038in}{2.199691in}}%
\pgfpathlineto{\pgfqpoint{1.466946in}{2.145523in}}%
\pgfpathlineto{\pgfqpoint{1.642158in}{2.093122in}}%
\pgfpathlineto{\pgfqpoint{1.852414in}{2.033608in}}%
\pgfpathlineto{\pgfqpoint{2.027626in}{1.986327in}}%
\pgfpathlineto{\pgfqpoint{2.237882in}{1.932210in}}%
\pgfpathlineto{\pgfqpoint{2.483179in}{1.872182in}}%
\pgfpathlineto{\pgfqpoint{2.693435in}{1.823011in}}%
\pgfpathlineto{\pgfqpoint{2.938732in}{1.768051in}}%
\pgfpathlineto{\pgfqpoint{3.184030in}{1.715356in}}%
\pgfpathlineto{\pgfqpoint{3.429328in}{1.664669in}}%
\pgfpathlineto{\pgfqpoint{3.709668in}{1.608940in}}%
\pgfpathlineto{\pgfqpoint{3.990009in}{1.555272in}}%
\pgfpathlineto{\pgfqpoint{4.270349in}{1.503445in}}%
\pgfpathlineto{\pgfqpoint{4.550689in}{1.453262in}}%
\pgfpathlineto{\pgfqpoint{4.901115in}{1.392581in}}%
\pgfpathlineto{\pgfqpoint{5.146412in}{1.351363in}}%
\pgfpathlineto{\pgfqpoint{5.580074in}{1.280609in}}%
\pgfpathlineto{\pgfqpoint{6.197689in}{1.184386in}}%
\pgfpathlineto{\pgfqpoint{6.513071in}{1.136936in}}%
\pgfpathlineto{\pgfqpoint{6.968624in}{1.070097in}}%
\pgfpathlineto{\pgfqpoint{7.367358in}{1.013386in}}%
\pgfpathlineto{\pgfqpoint{7.367358in}{1.013386in}}%
\pgfusepath{stroke}%
\end{pgfscope}%
\begin{pgfscope}%
\pgfpathrectangle{\pgfqpoint{0.766095in}{0.571603in}}{\pgfqpoint{6.973465in}{5.225635in}}%
\pgfusepath{clip}%
\pgfsetbuttcap%
\pgfsetroundjoin%
\pgfsetlinewidth{1.505625pt}%
\definecolor{currentstroke}{rgb}{0.276194,0.190074,0.493001}%
\pgfsetstrokecolor{currentstroke}%
\pgfsetdash{}{0pt}%
\pgfpathmoveto{\pgfqpoint{7.675640in}{0.970460in}}%
\pgfpathlineto{\pgfqpoint{7.704518in}{0.966420in}}%
\pgfpathlineto{\pgfqpoint{7.711130in}{0.965495in}}%
\pgfpathlineto{\pgfqpoint{7.739560in}{0.961647in}}%
\pgfusepath{stroke}%
\end{pgfscope}%
\begin{pgfscope}%
\pgfpathrectangle{\pgfqpoint{0.766095in}{0.571603in}}{\pgfqpoint{6.973465in}{5.225635in}}%
\pgfusepath{clip}%
\pgfsetbuttcap%
\pgfsetroundjoin%
\pgfsetlinewidth{1.505625pt}%
\definecolor{currentstroke}{rgb}{0.267968,0.223549,0.512008}%
\pgfsetstrokecolor{currentstroke}%
\pgfsetdash{}{0pt}%
\pgfpathmoveto{\pgfqpoint{0.766095in}{2.234991in}}%
\pgfpathlineto{\pgfqpoint{0.871223in}{2.198274in}}%
\pgfpathlineto{\pgfqpoint{1.046435in}{2.140166in}}%
\pgfpathlineto{\pgfqpoint{1.190665in}{2.094653in}}%
\pgfpathlineto{\pgfqpoint{1.396861in}{2.033118in}}%
\pgfpathlineto{\pgfqpoint{1.572073in}{1.983467in}}%
\pgfpathlineto{\pgfqpoint{1.782329in}{1.926819in}}%
\pgfpathlineto{\pgfqpoint{1.957541in}{1.881670in}}%
\pgfpathlineto{\pgfqpoint{2.202839in}{1.821454in}}%
\pgfpathlineto{\pgfqpoint{2.413094in}{1.772126in}}%
\pgfpathlineto{\pgfqpoint{2.658392in}{1.717011in}}%
\pgfpathlineto{\pgfqpoint{2.903690in}{1.664174in}}%
\pgfpathlineto{\pgfqpoint{3.148988in}{1.613352in}}%
\pgfpathlineto{\pgfqpoint{3.429328in}{1.557481in}}%
\pgfpathlineto{\pgfqpoint{3.709668in}{1.503671in}}%
\pgfpathlineto{\pgfqpoint{3.990009in}{1.451698in}}%
\pgfpathlineto{\pgfqpoint{4.305391in}{1.395173in}}%
\pgfpathlineto{\pgfqpoint{4.585732in}{1.346496in}}%
\pgfpathlineto{\pgfqpoint{4.936157in}{1.287450in}}%
\pgfpathlineto{\pgfqpoint{5.216497in}{1.241583in}}%
\pgfpathlineto{\pgfqpoint{5.601965in}{1.180165in}}%
\pgfpathlineto{\pgfqpoint{5.882306in}{1.136723in}}%
\pgfpathlineto{\pgfqpoint{6.144615in}{1.096726in}}%
\pgfpathlineto{\pgfqpoint{6.144615in}{1.096726in}}%
\pgfusepath{stroke}%
\end{pgfscope}%
\begin{pgfscope}%
\pgfpathrectangle{\pgfqpoint{0.766095in}{0.571603in}}{\pgfqpoint{6.973465in}{5.225635in}}%
\pgfusepath{clip}%
\pgfsetbuttcap%
\pgfsetroundjoin%
\pgfsetlinewidth{1.505625pt}%
\definecolor{currentstroke}{rgb}{0.267968,0.223549,0.512008}%
\pgfsetstrokecolor{currentstroke}%
\pgfsetdash{}{0pt}%
\pgfpathmoveto{\pgfqpoint{6.531155in}{1.039309in}}%
\pgfpathlineto{\pgfqpoint{6.548114in}{1.036846in}}%
\pgfpathlineto{\pgfqpoint{6.583156in}{1.031735in}}%
\pgfpathlineto{\pgfqpoint{6.618199in}{1.026600in}}%
\pgfpathlineto{\pgfqpoint{6.653242in}{1.021442in}}%
\pgfpathlineto{\pgfqpoint{6.676464in}{1.018014in}}%
\pgfpathlineto{\pgfqpoint{6.688284in}{1.016320in}}%
\pgfpathlineto{\pgfqpoint{6.723327in}{1.011293in}}%
\pgfpathlineto{\pgfqpoint{6.758369in}{1.006244in}}%
\pgfpathlineto{\pgfqpoint{6.793412in}{1.001171in}}%
\pgfpathlineto{\pgfqpoint{6.828454in}{0.996075in}}%
\pgfpathlineto{\pgfqpoint{6.858048in}{0.991755in}}%
\pgfpathlineto{\pgfqpoint{6.863497in}{0.990982in}}%
\pgfpathlineto{\pgfqpoint{6.898539in}{0.986017in}}%
\pgfpathlineto{\pgfqpoint{6.933582in}{0.981030in}}%
\pgfpathlineto{\pgfqpoint{6.968624in}{0.976019in}}%
\pgfpathlineto{\pgfqpoint{7.003667in}{0.970985in}}%
\pgfpathlineto{\pgfqpoint{7.038709in}{0.965927in}}%
\pgfpathlineto{\pgfqpoint{7.041699in}{0.965495in}}%
\pgfpathlineto{\pgfqpoint{7.073752in}{0.961008in}}%
\pgfpathlineto{\pgfqpoint{7.108795in}{0.956081in}}%
\pgfpathlineto{\pgfqpoint{7.143837in}{0.951132in}}%
\pgfpathlineto{\pgfqpoint{7.178880in}{0.946160in}}%
\pgfpathlineto{\pgfqpoint{7.213922in}{0.941163in}}%
\pgfpathlineto{\pgfqpoint{7.227427in}{0.939236in}}%
\pgfpathlineto{\pgfqpoint{7.248965in}{0.936252in}}%
\pgfpathlineto{\pgfqpoint{7.284007in}{0.931387in}}%
\pgfpathlineto{\pgfqpoint{7.319050in}{0.926499in}}%
\pgfpathlineto{\pgfqpoint{7.354092in}{0.921588in}}%
\pgfpathlineto{\pgfqpoint{7.389135in}{0.916654in}}%
\pgfpathlineto{\pgfqpoint{7.415167in}{0.912976in}}%
\pgfpathlineto{\pgfqpoint{7.424177in}{0.911741in}}%
\pgfpathlineto{\pgfqpoint{7.459220in}{0.906937in}}%
\pgfpathlineto{\pgfqpoint{7.494263in}{0.902110in}}%
\pgfpathlineto{\pgfqpoint{7.529305in}{0.897261in}}%
\pgfpathlineto{\pgfqpoint{7.564348in}{0.892388in}}%
\pgfpathlineto{\pgfqpoint{7.599390in}{0.887492in}}%
\pgfpathlineto{\pgfqpoint{7.604936in}{0.886717in}}%
\pgfpathlineto{\pgfqpoint{7.634433in}{0.882720in}}%
\pgfpathlineto{\pgfqpoint{7.669475in}{0.877955in}}%
\pgfpathlineto{\pgfqpoint{7.704518in}{0.873167in}}%
\pgfpathlineto{\pgfqpoint{7.739560in}{0.868356in}}%
\pgfusepath{stroke}%
\end{pgfscope}%
\begin{pgfscope}%
\pgfpathrectangle{\pgfqpoint{0.766095in}{0.571603in}}{\pgfqpoint{6.973465in}{5.225635in}}%
\pgfusepath{clip}%
\pgfsetbuttcap%
\pgfsetroundjoin%
\pgfsetlinewidth{1.505625pt}%
\definecolor{currentstroke}{rgb}{0.257322,0.256130,0.526563}%
\pgfsetstrokecolor{currentstroke}%
\pgfsetdash{}{0pt}%
\pgfpathmoveto{\pgfqpoint{0.766095in}{2.104614in}}%
\pgfpathlineto{\pgfqpoint{0.878582in}{2.068393in}}%
\pgfpathlineto{\pgfqpoint{1.049445in}{2.015874in}}%
\pgfpathlineto{\pgfqpoint{1.256691in}{1.955674in}}%
\pgfpathlineto{\pgfqpoint{1.431903in}{1.907250in}}%
\pgfpathlineto{\pgfqpoint{1.642158in}{1.851930in}}%
\pgfpathlineto{\pgfqpoint{1.852414in}{1.799206in}}%
\pgfpathlineto{\pgfqpoint{2.062669in}{1.748769in}}%
\pgfpathlineto{\pgfqpoint{2.307967in}{1.692542in}}%
\pgfpathlineto{\pgfqpoint{2.553264in}{1.638728in}}%
\pgfpathlineto{\pgfqpoint{2.798562in}{1.587047in}}%
\pgfpathlineto{\pgfqpoint{3.078903in}{1.530311in}}%
\pgfpathlineto{\pgfqpoint{3.359243in}{1.475738in}}%
\pgfpathlineto{\pgfqpoint{3.639583in}{1.423089in}}%
\pgfpathlineto{\pgfqpoint{3.919924in}{1.372150in}}%
\pgfpathlineto{\pgfqpoint{4.235306in}{1.316653in}}%
\pgfpathlineto{\pgfqpoint{4.515647in}{1.268783in}}%
\pgfpathlineto{\pgfqpoint{4.901115in}{1.204879in}}%
\pgfpathlineto{\pgfqpoint{5.181455in}{1.159803in}}%
\pgfpathlineto{\pgfqpoint{5.531880in}{1.104760in}}%
\pgfpathlineto{\pgfqpoint{5.847263in}{1.056450in}}%
\pgfpathlineto{\pgfqpoint{6.279729in}{0.991755in}}%
\pgfpathlineto{\pgfqpoint{6.915318in}{0.900066in}}%
\pgfpathlineto{\pgfqpoint{6.915318in}{0.900066in}}%
\pgfusepath{stroke}%
\end{pgfscope}%
\begin{pgfscope}%
\pgfpathrectangle{\pgfqpoint{0.766095in}{0.571603in}}{\pgfqpoint{6.973465in}{5.225635in}}%
\pgfusepath{clip}%
\pgfsetbuttcap%
\pgfsetroundjoin%
\pgfsetlinewidth{1.505625pt}%
\definecolor{currentstroke}{rgb}{0.257322,0.256130,0.526563}%
\pgfsetstrokecolor{currentstroke}%
\pgfsetdash{}{0pt}%
\pgfpathmoveto{\pgfqpoint{7.302310in}{0.845815in}}%
\pgfpathlineto{\pgfqpoint{7.319050in}{0.843489in}}%
\pgfpathlineto{\pgfqpoint{7.354092in}{0.838597in}}%
\pgfpathlineto{\pgfqpoint{7.385487in}{0.834198in}}%
\pgfpathlineto{\pgfqpoint{7.389135in}{0.833701in}}%
\pgfpathlineto{\pgfqpoint{7.424177in}{0.828932in}}%
\pgfpathlineto{\pgfqpoint{7.459220in}{0.824143in}}%
\pgfpathlineto{\pgfqpoint{7.494263in}{0.819332in}}%
\pgfpathlineto{\pgfqpoint{7.529305in}{0.814499in}}%
\pgfpathlineto{\pgfqpoint{7.564348in}{0.809645in}}%
\pgfpathlineto{\pgfqpoint{7.576653in}{0.807939in}}%
\pgfpathlineto{\pgfqpoint{7.599390in}{0.804876in}}%
\pgfpathlineto{\pgfqpoint{7.634433in}{0.800144in}}%
\pgfpathlineto{\pgfqpoint{7.669475in}{0.795391in}}%
\pgfpathlineto{\pgfqpoint{7.704518in}{0.790617in}}%
\pgfpathlineto{\pgfqpoint{7.739560in}{0.785822in}}%
\pgfusepath{stroke}%
\end{pgfscope}%
\begin{pgfscope}%
\pgfpathrectangle{\pgfqpoint{0.766095in}{0.571603in}}{\pgfqpoint{6.973465in}{5.225635in}}%
\pgfusepath{clip}%
\pgfsetbuttcap%
\pgfsetroundjoin%
\pgfsetlinewidth{1.505625pt}%
\definecolor{currentstroke}{rgb}{0.244972,0.287675,0.537260}%
\pgfsetstrokecolor{currentstroke}%
\pgfsetdash{}{0pt}%
\pgfpathmoveto{\pgfqpoint{0.766095in}{1.990541in}}%
\pgfpathlineto{\pgfqpoint{0.871223in}{1.958919in}}%
\pgfpathlineto{\pgfqpoint{1.046435in}{1.908215in}}%
\pgfpathlineto{\pgfqpoint{1.256691in}{1.850507in}}%
\pgfpathlineto{\pgfqpoint{1.466946in}{1.795702in}}%
\pgfpathlineto{\pgfqpoint{1.677201in}{1.743457in}}%
\pgfpathlineto{\pgfqpoint{1.887456in}{1.693468in}}%
\pgfpathlineto{\pgfqpoint{2.132754in}{1.637699in}}%
\pgfpathlineto{\pgfqpoint{2.378052in}{1.584312in}}%
\pgfpathlineto{\pgfqpoint{2.623350in}{1.533026in}}%
\pgfpathlineto{\pgfqpoint{2.903690in}{1.476696in}}%
\pgfpathlineto{\pgfqpoint{3.148988in}{1.429158in}}%
\pgfpathlineto{\pgfqpoint{3.464371in}{1.370190in}}%
\pgfpathlineto{\pgfqpoint{3.744711in}{1.319562in}}%
\pgfpathlineto{\pgfqpoint{4.060094in}{1.264381in}}%
\pgfpathlineto{\pgfqpoint{4.375477in}{1.210887in}}%
\pgfpathlineto{\pgfqpoint{4.690859in}{1.158905in}}%
\pgfpathlineto{\pgfqpoint{5.041285in}{1.102706in}}%
\pgfpathlineto{\pgfqpoint{5.356668in}{1.053454in}}%
\pgfpathlineto{\pgfqpoint{5.548948in}{1.023939in}}%
\pgfpathlineto{\pgfqpoint{5.548948in}{1.023939in}}%
\pgfusepath{stroke}%
\end{pgfscope}%
\begin{pgfscope}%
\pgfpathrectangle{\pgfqpoint{0.766095in}{0.571603in}}{\pgfqpoint{6.973465in}{5.225635in}}%
\pgfusepath{clip}%
\pgfsetbuttcap%
\pgfsetroundjoin%
\pgfsetlinewidth{1.505625pt}%
\definecolor{currentstroke}{rgb}{0.244972,0.287675,0.537260}%
\pgfsetstrokecolor{currentstroke}%
\pgfsetdash{}{0pt}%
\pgfpathmoveto{\pgfqpoint{5.935377in}{0.965768in}}%
\pgfpathlineto{\pgfqpoint{5.937191in}{0.965495in}}%
\pgfpathlineto{\pgfqpoint{5.952391in}{0.963268in}}%
\pgfpathlineto{\pgfqpoint{5.987433in}{0.958127in}}%
\pgfpathlineto{\pgfqpoint{6.022476in}{0.952966in}}%
\pgfpathlineto{\pgfqpoint{6.057518in}{0.947785in}}%
\pgfpathlineto{\pgfqpoint{6.092561in}{0.942584in}}%
\pgfpathlineto{\pgfqpoint{6.115065in}{0.939236in}}%
\pgfpathlineto{\pgfqpoint{6.127603in}{0.937419in}}%
\pgfpathlineto{\pgfqpoint{6.162646in}{0.932337in}}%
\pgfpathlineto{\pgfqpoint{6.197689in}{0.927236in}}%
\pgfpathlineto{\pgfqpoint{6.232731in}{0.922114in}}%
\pgfpathlineto{\pgfqpoint{6.267774in}{0.916971in}}%
\pgfpathlineto{\pgfqpoint{6.294918in}{0.912976in}}%
\pgfpathlineto{\pgfqpoint{6.302816in}{0.911844in}}%
\pgfpathlineto{\pgfqpoint{6.337859in}{0.906821in}}%
\pgfpathlineto{\pgfqpoint{6.372901in}{0.901778in}}%
\pgfpathlineto{\pgfqpoint{6.407944in}{0.896715in}}%
\pgfpathlineto{\pgfqpoint{6.442986in}{0.891631in}}%
\pgfpathlineto{\pgfqpoint{6.476726in}{0.886717in}}%
\pgfpathlineto{\pgfqpoint{6.478029in}{0.886532in}}%
\pgfpathlineto{\pgfqpoint{6.513071in}{0.881567in}}%
\pgfpathlineto{\pgfqpoint{6.548114in}{0.876582in}}%
\pgfpathlineto{\pgfqpoint{6.583156in}{0.871576in}}%
\pgfpathlineto{\pgfqpoint{6.618199in}{0.866551in}}%
\pgfpathlineto{\pgfqpoint{6.653242in}{0.861504in}}%
\pgfpathlineto{\pgfqpoint{6.660508in}{0.860458in}}%
\pgfpathlineto{\pgfqpoint{6.688284in}{0.856563in}}%
\pgfpathlineto{\pgfqpoint{6.723327in}{0.851635in}}%
\pgfpathlineto{\pgfqpoint{6.758369in}{0.846687in}}%
\pgfpathlineto{\pgfqpoint{6.793412in}{0.841719in}}%
\pgfpathlineto{\pgfqpoint{6.828454in}{0.836730in}}%
\pgfpathlineto{\pgfqpoint{6.846213in}{0.834198in}}%
\pgfpathlineto{\pgfqpoint{6.863497in}{0.831799in}}%
\pgfpathlineto{\pgfqpoint{6.898539in}{0.826928in}}%
\pgfpathlineto{\pgfqpoint{6.933582in}{0.822037in}}%
\pgfpathlineto{\pgfqpoint{6.968624in}{0.817126in}}%
\pgfpathlineto{\pgfqpoint{7.003667in}{0.812195in}}%
\pgfpathlineto{\pgfqpoint{7.033804in}{0.807939in}}%
\pgfpathlineto{\pgfqpoint{7.038709in}{0.807264in}}%
\pgfpathlineto{\pgfqpoint{7.073752in}{0.802450in}}%
\pgfpathlineto{\pgfqpoint{7.108795in}{0.797615in}}%
\pgfpathlineto{\pgfqpoint{7.143837in}{0.792761in}}%
\pgfpathlineto{\pgfqpoint{7.178880in}{0.787887in}}%
\pgfpathlineto{\pgfqpoint{7.213922in}{0.782991in}}%
\pgfpathlineto{\pgfqpoint{7.223312in}{0.781679in}}%
\pgfpathlineto{\pgfqpoint{7.248965in}{0.778191in}}%
\pgfpathlineto{\pgfqpoint{7.284007in}{0.773413in}}%
\pgfpathlineto{\pgfqpoint{7.319050in}{0.768615in}}%
\pgfpathlineto{\pgfqpoint{7.354092in}{0.763797in}}%
\pgfpathlineto{\pgfqpoint{7.389135in}{0.758959in}}%
\pgfpathlineto{\pgfqpoint{7.414693in}{0.755420in}}%
\pgfpathlineto{\pgfqpoint{7.424177in}{0.754142in}}%
\pgfpathlineto{\pgfqpoint{7.459220in}{0.749420in}}%
\pgfpathlineto{\pgfqpoint{7.494263in}{0.744678in}}%
\pgfpathlineto{\pgfqpoint{7.529305in}{0.739916in}}%
\pgfpathlineto{\pgfqpoint{7.564348in}{0.735135in}}%
\pgfpathlineto{\pgfqpoint{7.599390in}{0.730332in}}%
\pgfpathlineto{\pgfqpoint{7.607939in}{0.729160in}}%
\pgfpathlineto{\pgfqpoint{7.634433in}{0.725628in}}%
\pgfpathlineto{\pgfqpoint{7.669475in}{0.720942in}}%
\pgfpathlineto{\pgfqpoint{7.704518in}{0.716236in}}%
\pgfpathlineto{\pgfqpoint{7.739560in}{0.711511in}}%
\pgfusepath{stroke}%
\end{pgfscope}%
\begin{pgfscope}%
\pgfpathrectangle{\pgfqpoint{0.766095in}{0.571603in}}{\pgfqpoint{6.973465in}{5.225635in}}%
\pgfusepath{clip}%
\pgfsetbuttcap%
\pgfsetroundjoin%
\pgfsetlinewidth{1.505625pt}%
\definecolor{currentstroke}{rgb}{0.229739,0.322361,0.545706}%
\pgfsetstrokecolor{currentstroke}%
\pgfsetdash{}{0pt}%
\pgfpathmoveto{\pgfqpoint{0.766095in}{1.888895in}}%
\pgfpathlineto{\pgfqpoint{0.906265in}{1.849183in}}%
\pgfpathlineto{\pgfqpoint{1.081478in}{1.801500in}}%
\pgfpathlineto{\pgfqpoint{1.291733in}{1.746990in}}%
\pgfpathlineto{\pgfqpoint{1.501988in}{1.695016in}}%
\pgfpathlineto{\pgfqpoint{1.747286in}{1.637240in}}%
\pgfpathlineto{\pgfqpoint{1.957541in}{1.589807in}}%
\pgfpathlineto{\pgfqpoint{2.237882in}{1.529308in}}%
\pgfpathlineto{\pgfqpoint{2.483179in}{1.478566in}}%
\pgfpathlineto{\pgfqpoint{2.728477in}{1.429636in}}%
\pgfpathlineto{\pgfqpoint{3.043860in}{1.369106in}}%
\pgfpathlineto{\pgfqpoint{3.289158in}{1.323653in}}%
\pgfpathlineto{\pgfqpoint{3.639583in}{1.260879in}}%
\pgfpathlineto{\pgfqpoint{3.884881in}{1.218329in}}%
\pgfpathlineto{\pgfqpoint{4.294804in}{1.149312in}}%
\pgfpathlineto{\pgfqpoint{4.831030in}{1.062839in}}%
\pgfpathlineto{\pgfqpoint{4.883278in}{1.054612in}}%
\pgfpathlineto{\pgfqpoint{4.883278in}{1.054612in}}%
\pgfusepath{stroke}%
\end{pgfscope}%
\begin{pgfscope}%
\pgfpathrectangle{\pgfqpoint{0.766095in}{0.571603in}}{\pgfqpoint{6.973465in}{5.225635in}}%
\pgfusepath{clip}%
\pgfsetbuttcap%
\pgfsetroundjoin%
\pgfsetlinewidth{1.505625pt}%
\definecolor{currentstroke}{rgb}{0.229739,0.322361,0.545706}%
\pgfsetstrokecolor{currentstroke}%
\pgfsetdash{}{0pt}%
\pgfpathmoveto{\pgfqpoint{5.269437in}{0.994663in}}%
\pgfpathlineto{\pgfqpoint{5.286583in}{0.992010in}}%
\pgfpathlineto{\pgfqpoint{5.288233in}{0.991755in}}%
\pgfpathlineto{\pgfqpoint{5.321625in}{0.986719in}}%
\pgfpathlineto{\pgfqpoint{5.356668in}{0.981417in}}%
\pgfpathlineto{\pgfqpoint{5.391710in}{0.976095in}}%
\pgfpathlineto{\pgfqpoint{5.426753in}{0.970755in}}%
\pgfpathlineto{\pgfqpoint{5.461140in}{0.965495in}}%
\pgfpathlineto{\pgfqpoint{5.461795in}{0.965398in}}%
\pgfpathlineto{\pgfqpoint{5.496838in}{0.960175in}}%
\pgfpathlineto{\pgfqpoint{5.531880in}{0.954934in}}%
\pgfpathlineto{\pgfqpoint{5.566923in}{0.949673in}}%
\pgfpathlineto{\pgfqpoint{5.601965in}{0.944394in}}%
\pgfpathlineto{\pgfqpoint{5.636076in}{0.939236in}}%
\pgfpathlineto{\pgfqpoint{5.637008in}{0.939099in}}%
\pgfpathlineto{\pgfqpoint{5.672050in}{0.933936in}}%
\pgfpathlineto{\pgfqpoint{5.707093in}{0.928754in}}%
\pgfpathlineto{\pgfqpoint{5.742136in}{0.923554in}}%
\pgfpathlineto{\pgfqpoint{5.777178in}{0.918334in}}%
\pgfpathlineto{\pgfqpoint{5.812221in}{0.913095in}}%
\pgfpathlineto{\pgfqpoint{5.813011in}{0.912976in}}%
\pgfpathlineto{\pgfqpoint{5.847263in}{0.907987in}}%
\pgfpathlineto{\pgfqpoint{5.882306in}{0.902865in}}%
\pgfpathlineto{\pgfqpoint{5.917348in}{0.897723in}}%
\pgfpathlineto{\pgfqpoint{5.952391in}{0.892562in}}%
\pgfpathlineto{\pgfqpoint{5.987433in}{0.887382in}}%
\pgfpathlineto{\pgfqpoint{5.991934in}{0.886717in}}%
\pgfpathlineto{\pgfqpoint{6.022476in}{0.882316in}}%
\pgfpathlineto{\pgfqpoint{6.057518in}{0.877252in}}%
\pgfpathlineto{\pgfqpoint{6.092561in}{0.872169in}}%
\pgfpathlineto{\pgfqpoint{6.127603in}{0.867066in}}%
\pgfpathlineto{\pgfqpoint{6.162646in}{0.861944in}}%
\pgfpathlineto{\pgfqpoint{6.172810in}{0.860458in}}%
\pgfpathlineto{\pgfqpoint{6.197689in}{0.856911in}}%
\pgfpathlineto{\pgfqpoint{6.232731in}{0.851904in}}%
\pgfpathlineto{\pgfqpoint{6.267774in}{0.846878in}}%
\pgfpathlineto{\pgfqpoint{6.302816in}{0.841833in}}%
\pgfpathlineto{\pgfqpoint{6.337859in}{0.836768in}}%
\pgfpathlineto{\pgfqpoint{6.355614in}{0.834198in}}%
\pgfpathlineto{\pgfqpoint{6.372901in}{0.831759in}}%
\pgfpathlineto{\pgfqpoint{6.407944in}{0.826808in}}%
\pgfpathlineto{\pgfqpoint{6.442986in}{0.821839in}}%
\pgfpathlineto{\pgfqpoint{6.478029in}{0.816851in}}%
\pgfpathlineto{\pgfqpoint{6.513071in}{0.811843in}}%
\pgfpathlineto{\pgfqpoint{6.540316in}{0.807939in}}%
\pgfpathlineto{\pgfqpoint{6.548114in}{0.806849in}}%
\pgfpathlineto{\pgfqpoint{6.583156in}{0.801955in}}%
\pgfpathlineto{\pgfqpoint{6.618199in}{0.797042in}}%
\pgfpathlineto{\pgfqpoint{6.653242in}{0.792110in}}%
\pgfpathlineto{\pgfqpoint{6.688284in}{0.787158in}}%
\pgfpathlineto{\pgfqpoint{6.723327in}{0.782187in}}%
\pgfpathlineto{\pgfqpoint{6.726909in}{0.781679in}}%
\pgfpathlineto{\pgfqpoint{6.758369in}{0.777333in}}%
\pgfpathlineto{\pgfqpoint{6.793412in}{0.772475in}}%
\pgfpathlineto{\pgfqpoint{6.828454in}{0.767599in}}%
\pgfpathlineto{\pgfqpoint{6.863497in}{0.762703in}}%
\pgfpathlineto{\pgfqpoint{6.898539in}{0.757788in}}%
\pgfpathlineto{\pgfqpoint{6.915401in}{0.755420in}}%
\pgfpathlineto{\pgfqpoint{6.933582in}{0.752932in}}%
\pgfpathlineto{\pgfqpoint{6.968624in}{0.748129in}}%
\pgfpathlineto{\pgfqpoint{7.003667in}{0.743308in}}%
\pgfpathlineto{\pgfqpoint{7.038709in}{0.738468in}}%
\pgfpathlineto{\pgfqpoint{7.073752in}{0.733608in}}%
\pgfpathlineto{\pgfqpoint{7.105715in}{0.729160in}}%
\pgfpathlineto{\pgfqpoint{7.108795in}{0.728743in}}%
\pgfpathlineto{\pgfqpoint{7.143837in}{0.723995in}}%
\pgfpathlineto{\pgfqpoint{7.178880in}{0.719228in}}%
\pgfpathlineto{\pgfqpoint{7.213922in}{0.714443in}}%
\pgfpathlineto{\pgfqpoint{7.248965in}{0.709639in}}%
\pgfpathlineto{\pgfqpoint{7.284007in}{0.704815in}}%
\pgfpathlineto{\pgfqpoint{7.297902in}{0.702901in}}%
\pgfpathlineto{\pgfqpoint{7.319050in}{0.700063in}}%
\pgfpathlineto{\pgfqpoint{7.354092in}{0.695351in}}%
\pgfpathlineto{\pgfqpoint{7.389135in}{0.690620in}}%
\pgfpathlineto{\pgfqpoint{7.424177in}{0.685871in}}%
\pgfpathlineto{\pgfqpoint{7.459220in}{0.681102in}}%
\pgfpathlineto{\pgfqpoint{7.491879in}{0.676641in}}%
\pgfpathlineto{\pgfqpoint{7.494263in}{0.676324in}}%
\pgfpathlineto{\pgfqpoint{7.529305in}{0.671666in}}%
\pgfpathlineto{\pgfqpoint{7.564348in}{0.666990in}}%
\pgfpathlineto{\pgfqpoint{7.599390in}{0.662295in}}%
\pgfpathlineto{\pgfqpoint{7.634433in}{0.657581in}}%
\pgfpathlineto{\pgfqpoint{7.669475in}{0.652848in}}%
\pgfpathlineto{\pgfqpoint{7.687705in}{0.650382in}}%
\pgfpathlineto{\pgfqpoint{7.704518in}{0.648167in}}%
\pgfpathlineto{\pgfqpoint{7.739560in}{0.643545in}}%
\pgfusepath{stroke}%
\end{pgfscope}%
\begin{pgfscope}%
\pgfpathrectangle{\pgfqpoint{0.766095in}{0.571603in}}{\pgfqpoint{6.973465in}{5.225635in}}%
\pgfusepath{clip}%
\pgfsetbuttcap%
\pgfsetroundjoin%
\pgfsetlinewidth{1.505625pt}%
\definecolor{currentstroke}{rgb}{0.216210,0.351535,0.550627}%
\pgfsetstrokecolor{currentstroke}%
\pgfsetdash{}{0pt}%
\pgfpathmoveto{\pgfqpoint{0.766095in}{1.796874in}}%
\pgfpathlineto{\pgfqpoint{0.941308in}{1.749678in}}%
\pgfpathlineto{\pgfqpoint{1.151563in}{1.695705in}}%
\pgfpathlineto{\pgfqpoint{1.361818in}{1.644224in}}%
\pgfpathlineto{\pgfqpoint{1.607116in}{1.586987in}}%
\pgfpathlineto{\pgfqpoint{1.852414in}{1.532350in}}%
\pgfpathlineto{\pgfqpoint{2.097711in}{1.480004in}}%
\pgfpathlineto{\pgfqpoint{2.343009in}{1.429676in}}%
\pgfpathlineto{\pgfqpoint{2.623350in}{1.374361in}}%
\pgfpathlineto{\pgfqpoint{2.903690in}{1.321094in}}%
\pgfpathlineto{\pgfqpoint{3.219073in}{1.263316in}}%
\pgfpathlineto{\pgfqpoint{3.499413in}{1.213670in}}%
\pgfpathlineto{\pgfqpoint{3.814796in}{1.159494in}}%
\pgfpathlineto{\pgfqpoint{4.130179in}{1.106917in}}%
\pgfpathlineto{\pgfqpoint{4.480604in}{1.050140in}}%
\pgfpathlineto{\pgfqpoint{4.795987in}{1.000422in}}%
\pgfpathlineto{\pgfqpoint{5.146412in}{0.946493in}}%
\pgfpathlineto{\pgfqpoint{5.496838in}{0.893850in}}%
\pgfpathlineto{\pgfqpoint{5.882306in}{0.837243in}}%
\pgfpathlineto{\pgfqpoint{6.232731in}{0.786945in}}%
\pgfpathlineto{\pgfqpoint{6.319590in}{0.774637in}}%
\pgfpathlineto{\pgfqpoint{6.319590in}{0.774637in}}%
\pgfusepath{stroke}%
\end{pgfscope}%
\begin{pgfscope}%
\pgfpathrectangle{\pgfqpoint{0.766095in}{0.571603in}}{\pgfqpoint{6.973465in}{5.225635in}}%
\pgfusepath{clip}%
\pgfsetbuttcap%
\pgfsetroundjoin%
\pgfsetlinewidth{1.505625pt}%
\definecolor{currentstroke}{rgb}{0.216210,0.351535,0.550627}%
\pgfsetstrokecolor{currentstroke}%
\pgfsetdash{}{0pt}%
\pgfpathmoveto{\pgfqpoint{6.706590in}{0.720448in}}%
\pgfpathlineto{\pgfqpoint{6.723327in}{0.718139in}}%
\pgfpathlineto{\pgfqpoint{6.758369in}{0.713287in}}%
\pgfpathlineto{\pgfqpoint{6.793412in}{0.708418in}}%
\pgfpathlineto{\pgfqpoint{6.828454in}{0.703529in}}%
\pgfpathlineto{\pgfqpoint{6.832962in}{0.702901in}}%
\pgfpathlineto{\pgfqpoint{6.863497in}{0.698749in}}%
\pgfpathlineto{\pgfqpoint{6.898539in}{0.693969in}}%
\pgfpathlineto{\pgfqpoint{6.933582in}{0.689171in}}%
\pgfpathlineto{\pgfqpoint{6.968624in}{0.684356in}}%
\pgfpathlineto{\pgfqpoint{7.003667in}{0.679522in}}%
\pgfpathlineto{\pgfqpoint{7.024511in}{0.676641in}}%
\pgfpathlineto{\pgfqpoint{7.038709in}{0.674728in}}%
\pgfpathlineto{\pgfqpoint{7.073752in}{0.670002in}}%
\pgfpathlineto{\pgfqpoint{7.108795in}{0.665258in}}%
\pgfpathlineto{\pgfqpoint{7.143837in}{0.660496in}}%
\pgfpathlineto{\pgfqpoint{7.178880in}{0.655716in}}%
\pgfpathlineto{\pgfqpoint{7.213922in}{0.650918in}}%
\pgfpathlineto{\pgfqpoint{7.217840in}{0.650382in}}%
\pgfpathlineto{\pgfqpoint{7.248965in}{0.646229in}}%
\pgfpathlineto{\pgfqpoint{7.284007in}{0.641538in}}%
\pgfpathlineto{\pgfqpoint{7.319050in}{0.636830in}}%
\pgfpathlineto{\pgfqpoint{7.354092in}{0.632104in}}%
\pgfpathlineto{\pgfqpoint{7.389135in}{0.627359in}}%
\pgfpathlineto{\pgfqpoint{7.412987in}{0.624122in}}%
\pgfpathlineto{\pgfqpoint{7.424177in}{0.622642in}}%
\pgfpathlineto{\pgfqpoint{7.459220in}{0.618004in}}%
\pgfpathlineto{\pgfqpoint{7.494263in}{0.613349in}}%
\pgfpathlineto{\pgfqpoint{7.529305in}{0.608676in}}%
\pgfpathlineto{\pgfqpoint{7.564348in}{0.603985in}}%
\pgfpathlineto{\pgfqpoint{7.599390in}{0.599276in}}%
\pgfpathlineto{\pgfqpoint{7.609898in}{0.597863in}}%
\pgfpathlineto{\pgfqpoint{7.634433in}{0.594648in}}%
\pgfpathlineto{\pgfqpoint{7.669475in}{0.590046in}}%
\pgfpathlineto{\pgfqpoint{7.704518in}{0.585425in}}%
\pgfpathlineto{\pgfqpoint{7.739560in}{0.580788in}}%
\pgfusepath{stroke}%
\end{pgfscope}%
\begin{pgfscope}%
\pgfpathrectangle{\pgfqpoint{0.766095in}{0.571603in}}{\pgfqpoint{6.973465in}{5.225635in}}%
\pgfusepath{clip}%
\pgfsetbuttcap%
\pgfsetroundjoin%
\pgfsetlinewidth{1.505625pt}%
\definecolor{currentstroke}{rgb}{0.203063,0.379716,0.553925}%
\pgfsetstrokecolor{currentstroke}%
\pgfsetdash{}{0pt}%
\pgfpathmoveto{\pgfqpoint{0.766095in}{1.712598in}}%
\pgfpathlineto{\pgfqpoint{0.801138in}{1.703343in}}%
\pgfpathlineto{\pgfqpoint{0.810941in}{1.700761in}}%
\pgfpathlineto{\pgfqpoint{0.836180in}{1.694253in}}%
\pgfpathlineto{\pgfqpoint{0.871223in}{1.685225in}}%
\pgfpathlineto{\pgfqpoint{0.906265in}{1.676194in}}%
\pgfpathlineto{\pgfqpoint{0.912852in}{1.674501in}}%
\pgfpathlineto{\pgfqpoint{0.941308in}{1.667343in}}%
\pgfpathlineto{\pgfqpoint{0.976350in}{1.658532in}}%
\pgfpathlineto{\pgfqpoint{1.011393in}{1.649715in}}%
\pgfpathlineto{\pgfqpoint{1.017266in}{1.648242in}}%
\pgfpathlineto{\pgfqpoint{1.046435in}{1.641079in}}%
\pgfpathlineto{\pgfqpoint{1.081478in}{1.632475in}}%
\pgfpathlineto{\pgfqpoint{1.116520in}{1.623865in}}%
\pgfpathlineto{\pgfqpoint{1.124205in}{1.621982in}}%
\pgfpathlineto{\pgfqpoint{1.151563in}{1.615420in}}%
\pgfpathlineto{\pgfqpoint{1.186605in}{1.607016in}}%
\pgfpathlineto{\pgfqpoint{1.221648in}{1.598605in}}%
\pgfpathlineto{\pgfqpoint{1.233680in}{1.595723in}}%
\pgfpathlineto{\pgfqpoint{1.256691in}{1.590327in}}%
\pgfpathlineto{\pgfqpoint{1.291733in}{1.582116in}}%
\pgfpathlineto{\pgfqpoint{1.326776in}{1.573897in}}%
\pgfpathlineto{\pgfqpoint{1.345700in}{1.569463in}}%
\pgfpathlineto{\pgfqpoint{1.361818in}{1.565767in}}%
\pgfpathlineto{\pgfqpoint{1.396861in}{1.557741in}}%
\pgfpathlineto{\pgfqpoint{1.431903in}{1.549707in}}%
\pgfpathlineto{\pgfqpoint{1.460262in}{1.543204in}}%
\pgfpathlineto{\pgfqpoint{1.466946in}{1.541703in}}%
\pgfpathlineto{\pgfqpoint{1.501988in}{1.533857in}}%
\pgfpathlineto{\pgfqpoint{1.537031in}{1.526001in}}%
\pgfpathlineto{\pgfqpoint{1.572073in}{1.518137in}}%
\pgfpathlineto{\pgfqpoint{1.577398in}{1.516944in}}%
\pgfpathlineto{\pgfqpoint{1.607116in}{1.510432in}}%
\pgfpathlineto{\pgfqpoint{1.642158in}{1.502749in}}%
\pgfpathlineto{\pgfqpoint{1.677201in}{1.495056in}}%
\pgfpathlineto{\pgfqpoint{1.697123in}{1.490685in}}%
\pgfpathlineto{\pgfqpoint{1.712244in}{1.487438in}}%
\pgfpathlineto{\pgfqpoint{1.747286in}{1.479921in}}%
\pgfpathlineto{\pgfqpoint{1.782329in}{1.472394in}}%
\pgfpathlineto{\pgfqpoint{1.817371in}{1.464856in}}%
\pgfpathlineto{\pgfqpoint{1.819378in}{1.464425in}}%
\pgfpathlineto{\pgfqpoint{1.852414in}{1.457489in}}%
\pgfpathlineto{\pgfqpoint{1.887456in}{1.450122in}}%
\pgfpathlineto{\pgfqpoint{1.922499in}{1.442745in}}%
\pgfpathlineto{\pgfqpoint{1.944248in}{1.438166in}}%
\pgfpathlineto{\pgfqpoint{1.957541in}{1.435427in}}%
\pgfpathlineto{\pgfqpoint{1.992584in}{1.428216in}}%
\pgfpathlineto{\pgfqpoint{2.027626in}{1.420993in}}%
\pgfpathlineto{\pgfqpoint{2.062669in}{1.413758in}}%
\pgfpathlineto{\pgfqpoint{2.071652in}{1.411906in}}%
\pgfpathlineto{\pgfqpoint{2.097711in}{1.406649in}}%
\pgfpathlineto{\pgfqpoint{2.132754in}{1.399576in}}%
\pgfpathlineto{\pgfqpoint{2.167797in}{1.392491in}}%
\pgfpathlineto{\pgfqpoint{2.201595in}{1.385647in}}%
\pgfpathlineto{\pgfqpoint{2.202839in}{1.385400in}}%
\pgfpathlineto{\pgfqpoint{2.237882in}{1.378473in}}%
\pgfpathlineto{\pgfqpoint{2.272924in}{1.371533in}}%
\pgfpathlineto{\pgfqpoint{2.307967in}{1.364580in}}%
\pgfpathlineto{\pgfqpoint{2.334116in}{1.359388in}}%
\pgfpathlineto{\pgfqpoint{2.343009in}{1.357660in}}%
\pgfpathlineto{\pgfqpoint{2.378052in}{1.350860in}}%
\pgfpathlineto{\pgfqpoint{2.413094in}{1.344048in}}%
\pgfpathlineto{\pgfqpoint{2.448137in}{1.337222in}}%
\pgfpathlineto{\pgfqpoint{2.469152in}{1.333128in}}%
\pgfpathlineto{\pgfqpoint{2.483179in}{1.330454in}}%
\pgfpathlineto{\pgfqpoint{2.518222in}{1.323778in}}%
\pgfpathlineto{\pgfqpoint{2.553264in}{1.317088in}}%
\pgfpathlineto{\pgfqpoint{2.588307in}{1.310385in}}%
\pgfpathlineto{\pgfqpoint{2.606691in}{1.306869in}}%
\pgfpathlineto{\pgfqpoint{2.623350in}{1.303751in}}%
\pgfpathlineto{\pgfqpoint{2.658392in}{1.297194in}}%
\pgfpathlineto{\pgfqpoint{2.693435in}{1.290623in}}%
\pgfpathlineto{\pgfqpoint{2.728477in}{1.284037in}}%
\pgfpathlineto{\pgfqpoint{2.746719in}{1.280609in}}%
\pgfpathlineto{\pgfqpoint{2.763520in}{1.277520in}}%
\pgfpathlineto{\pgfqpoint{2.798562in}{1.271077in}}%
\pgfpathlineto{\pgfqpoint{2.833605in}{1.264621in}}%
\pgfpathlineto{\pgfqpoint{2.868647in}{1.258149in}}%
\pgfpathlineto{\pgfqpoint{2.889213in}{1.254350in}}%
\pgfpathlineto{\pgfqpoint{2.903690in}{1.251733in}}%
\pgfpathlineto{\pgfqpoint{2.938732in}{1.245401in}}%
\pgfpathlineto{\pgfqpoint{2.973775in}{1.239055in}}%
\pgfpathlineto{\pgfqpoint{3.008818in}{1.232694in}}%
\pgfpathlineto{\pgfqpoint{3.034147in}{1.228090in}}%
\pgfpathlineto{\pgfqpoint{3.043860in}{1.226363in}}%
\pgfpathlineto{\pgfqpoint{3.078903in}{1.220138in}}%
\pgfpathlineto{\pgfqpoint{3.113945in}{1.213899in}}%
\pgfpathlineto{\pgfqpoint{3.148988in}{1.207644in}}%
\pgfpathlineto{\pgfqpoint{3.181488in}{1.201831in}}%
\pgfpathlineto{\pgfqpoint{3.184030in}{1.201386in}}%
\pgfpathlineto{\pgfqpoint{3.219073in}{1.195264in}}%
\pgfpathlineto{\pgfqpoint{3.254115in}{1.189128in}}%
\pgfpathlineto{\pgfqpoint{3.289158in}{1.182977in}}%
\pgfpathlineto{\pgfqpoint{3.324200in}{1.176810in}}%
\pgfpathlineto{\pgfqpoint{3.331243in}{1.175571in}}%
\pgfpathlineto{\pgfqpoint{3.359243in}{1.170756in}}%
\pgfpathlineto{\pgfqpoint{3.394285in}{1.164720in}}%
\pgfpathlineto{\pgfqpoint{3.429328in}{1.158669in}}%
\pgfpathlineto{\pgfqpoint{3.464371in}{1.152601in}}%
\pgfpathlineto{\pgfqpoint{3.483359in}{1.149312in}}%
\pgfpathlineto{\pgfqpoint{3.499413in}{1.146592in}}%
\pgfpathlineto{\pgfqpoint{3.534456in}{1.140653in}}%
\pgfpathlineto{\pgfqpoint{3.569498in}{1.134698in}}%
\pgfpathlineto{\pgfqpoint{3.604541in}{1.128728in}}%
\pgfpathlineto{\pgfqpoint{3.637774in}{1.123052in}}%
\pgfpathlineto{\pgfqpoint{3.639583in}{1.122750in}}%
\pgfpathlineto{\pgfqpoint{3.674626in}{1.116906in}}%
\pgfpathlineto{\pgfqpoint{3.709668in}{1.111046in}}%
\pgfpathlineto{\pgfqpoint{3.744711in}{1.105170in}}%
\pgfpathlineto{\pgfqpoint{3.779753in}{1.099278in}}%
\pgfpathlineto{\pgfqpoint{3.794531in}{1.096793in}}%
\pgfpathlineto{\pgfqpoint{3.814796in}{1.093460in}}%
\pgfpathlineto{\pgfqpoint{3.849838in}{1.087692in}}%
\pgfpathlineto{\pgfqpoint{3.884881in}{1.081908in}}%
\pgfpathlineto{\pgfqpoint{3.919924in}{1.076108in}}%
\pgfpathlineto{\pgfqpoint{3.953516in}{1.070533in}}%
\pgfpathlineto{\pgfqpoint{3.954966in}{1.070298in}}%
\pgfpathlineto{\pgfqpoint{3.990009in}{1.064620in}}%
\pgfpathlineto{\pgfqpoint{4.025051in}{1.058925in}}%
\pgfpathlineto{\pgfqpoint{4.060094in}{1.053215in}}%
\pgfpathlineto{\pgfqpoint{4.095136in}{1.047487in}}%
\pgfpathlineto{\pgfqpoint{4.114778in}{1.044274in}}%
\pgfpathlineto{\pgfqpoint{4.130179in}{1.041811in}}%
\pgfpathlineto{\pgfqpoint{4.165221in}{1.036204in}}%
\pgfpathlineto{\pgfqpoint{4.200264in}{1.030580in}}%
\pgfpathlineto{\pgfqpoint{4.235306in}{1.024941in}}%
\pgfpathlineto{\pgfqpoint{4.250852in}{1.022431in}}%
\pgfusepath{stroke}%
\end{pgfscope}%
\begin{pgfscope}%
\pgfpathrectangle{\pgfqpoint{0.766095in}{0.571603in}}{\pgfqpoint{6.973465in}{5.225635in}}%
\pgfusepath{clip}%
\pgfsetbuttcap%
\pgfsetroundjoin%
\pgfsetlinewidth{1.505625pt}%
\definecolor{currentstroke}{rgb}{0.203063,0.379716,0.553925}%
\pgfsetstrokecolor{currentstroke}%
\pgfsetdash{}{0pt}%
\pgfpathmoveto{\pgfqpoint{4.636876in}{0.961607in}}%
\pgfpathlineto{\pgfqpoint{4.655817in}{0.958694in}}%
\pgfpathlineto{\pgfqpoint{4.690859in}{0.953288in}}%
\pgfpathlineto{\pgfqpoint{4.725902in}{0.947866in}}%
\pgfpathlineto{\pgfqpoint{4.760944in}{0.942426in}}%
\pgfpathlineto{\pgfqpoint{4.781468in}{0.939236in}}%
\pgfpathlineto{\pgfqpoint{4.795987in}{0.937031in}}%
\pgfpathlineto{\pgfqpoint{4.831030in}{0.931705in}}%
\pgfpathlineto{\pgfqpoint{4.866072in}{0.926363in}}%
\pgfpathlineto{\pgfqpoint{4.901115in}{0.921003in}}%
\pgfpathlineto{\pgfqpoint{4.936157in}{0.915627in}}%
\pgfpathlineto{\pgfqpoint{4.953415in}{0.912976in}}%
\pgfpathlineto{\pgfqpoint{4.971200in}{0.910308in}}%
\pgfpathlineto{\pgfqpoint{5.006242in}{0.905044in}}%
\pgfpathlineto{\pgfqpoint{5.041285in}{0.899764in}}%
\pgfpathlineto{\pgfqpoint{5.076327in}{0.894466in}}%
\pgfpathlineto{\pgfqpoint{5.111370in}{0.889152in}}%
\pgfpathlineto{\pgfqpoint{5.127407in}{0.886717in}}%
\pgfpathlineto{\pgfqpoint{5.146412in}{0.883899in}}%
\pgfpathlineto{\pgfqpoint{5.181455in}{0.878695in}}%
\pgfpathlineto{\pgfqpoint{5.216497in}{0.873476in}}%
\pgfpathlineto{\pgfqpoint{5.251540in}{0.868239in}}%
\pgfpathlineto{\pgfqpoint{5.286583in}{0.862985in}}%
\pgfpathlineto{\pgfqpoint{5.303419in}{0.860458in}}%
\pgfpathlineto{\pgfqpoint{5.321625in}{0.857788in}}%
\pgfpathlineto{\pgfqpoint{5.356668in}{0.852644in}}%
\pgfpathlineto{\pgfqpoint{5.391710in}{0.847484in}}%
\pgfpathlineto{\pgfqpoint{5.426753in}{0.842306in}}%
\pgfpathlineto{\pgfqpoint{5.461795in}{0.837111in}}%
\pgfpathlineto{\pgfqpoint{5.481419in}{0.834198in}}%
\pgfpathlineto{\pgfqpoint{5.496838in}{0.831962in}}%
\pgfpathlineto{\pgfqpoint{5.531880in}{0.826877in}}%
\pgfpathlineto{\pgfqpoint{5.566923in}{0.821774in}}%
\pgfpathlineto{\pgfqpoint{5.601965in}{0.816655in}}%
\pgfpathlineto{\pgfqpoint{5.637008in}{0.811518in}}%
\pgfpathlineto{\pgfqpoint{5.661377in}{0.807939in}}%
\pgfpathlineto{\pgfqpoint{5.672050in}{0.806407in}}%
\pgfpathlineto{\pgfqpoint{5.707093in}{0.801379in}}%
\pgfpathlineto{\pgfqpoint{5.742136in}{0.796334in}}%
\pgfpathlineto{\pgfqpoint{5.777178in}{0.791272in}}%
\pgfpathlineto{\pgfqpoint{5.812221in}{0.786192in}}%
\pgfpathlineto{\pgfqpoint{5.843264in}{0.781679in}}%
\pgfpathlineto{\pgfqpoint{5.847263in}{0.781111in}}%
\pgfpathlineto{\pgfqpoint{5.882306in}{0.776139in}}%
\pgfpathlineto{\pgfqpoint{5.917348in}{0.771150in}}%
\pgfpathlineto{\pgfqpoint{5.952391in}{0.766144in}}%
\pgfpathlineto{\pgfqpoint{5.987433in}{0.761121in}}%
\pgfpathlineto{\pgfqpoint{6.022476in}{0.756080in}}%
\pgfpathlineto{\pgfqpoint{6.027071in}{0.755420in}}%
\pgfpathlineto{\pgfqpoint{6.057518in}{0.751145in}}%
\pgfpathlineto{\pgfqpoint{6.092561in}{0.746212in}}%
\pgfpathlineto{\pgfqpoint{6.127603in}{0.741261in}}%
\pgfpathlineto{\pgfqpoint{6.162646in}{0.736294in}}%
\pgfpathlineto{\pgfqpoint{6.197689in}{0.731308in}}%
\pgfpathlineto{\pgfqpoint{6.212774in}{0.729160in}}%
\pgfpathlineto{\pgfqpoint{6.232731in}{0.726386in}}%
\pgfpathlineto{\pgfqpoint{6.267774in}{0.721507in}}%
\pgfpathlineto{\pgfqpoint{6.302816in}{0.716611in}}%
\pgfpathlineto{\pgfqpoint{6.337859in}{0.711699in}}%
\pgfpathlineto{\pgfqpoint{6.372901in}{0.706768in}}%
\pgfpathlineto{\pgfqpoint{6.400320in}{0.702901in}}%
\pgfpathlineto{\pgfqpoint{6.407944in}{0.701851in}}%
\pgfpathlineto{\pgfqpoint{6.442986in}{0.697026in}}%
\pgfpathlineto{\pgfqpoint{6.478029in}{0.692185in}}%
\pgfpathlineto{\pgfqpoint{6.513071in}{0.687326in}}%
\pgfpathlineto{\pgfqpoint{6.548114in}{0.682450in}}%
\pgfpathlineto{\pgfqpoint{6.583156in}{0.677557in}}%
\pgfpathlineto{\pgfqpoint{6.589711in}{0.676641in}}%
\pgfpathlineto{\pgfqpoint{6.618199in}{0.672759in}}%
\pgfpathlineto{\pgfqpoint{6.653242in}{0.667971in}}%
\pgfpathlineto{\pgfqpoint{6.688284in}{0.663166in}}%
\pgfpathlineto{\pgfqpoint{6.723327in}{0.658344in}}%
\pgfpathlineto{\pgfqpoint{6.758369in}{0.653504in}}%
\pgfpathlineto{\pgfqpoint{6.780930in}{0.650382in}}%
\pgfpathlineto{\pgfqpoint{6.793412in}{0.648696in}}%
\pgfpathlineto{\pgfqpoint{6.828454in}{0.643961in}}%
\pgfpathlineto{\pgfqpoint{6.863497in}{0.639209in}}%
\pgfpathlineto{\pgfqpoint{6.898539in}{0.634440in}}%
\pgfpathlineto{\pgfqpoint{6.933582in}{0.629654in}}%
\pgfpathlineto{\pgfqpoint{6.968624in}{0.624850in}}%
\pgfpathlineto{\pgfqpoint{6.973932in}{0.624122in}}%
\pgfpathlineto{\pgfqpoint{7.003667in}{0.620146in}}%
\pgfpathlineto{\pgfqpoint{7.038709in}{0.615447in}}%
\pgfpathlineto{\pgfqpoint{7.073752in}{0.610730in}}%
\pgfpathlineto{\pgfqpoint{7.108795in}{0.605997in}}%
\pgfpathlineto{\pgfqpoint{7.143837in}{0.601246in}}%
\pgfpathlineto{\pgfqpoint{7.168732in}{0.597863in}}%
\pgfpathlineto{\pgfqpoint{7.178880in}{0.596518in}}%
\pgfpathlineto{\pgfqpoint{7.213922in}{0.591870in}}%
\pgfpathlineto{\pgfqpoint{7.248965in}{0.587206in}}%
\pgfpathlineto{\pgfqpoint{7.284007in}{0.582525in}}%
\pgfpathlineto{\pgfqpoint{7.319050in}{0.577827in}}%
\pgfpathlineto{\pgfqpoint{7.354092in}{0.573111in}}%
\pgfpathlineto{\pgfqpoint{7.365289in}{0.571603in}}%
\pgfusepath{stroke}%
\end{pgfscope}%
\begin{pgfscope}%
\pgfpathrectangle{\pgfqpoint{0.766095in}{0.571603in}}{\pgfqpoint{6.973465in}{5.225635in}}%
\pgfusepath{clip}%
\pgfsetbuttcap%
\pgfsetroundjoin%
\pgfsetlinewidth{1.505625pt}%
\definecolor{currentstroke}{rgb}{0.190631,0.407061,0.556089}%
\pgfsetstrokecolor{currentstroke}%
\pgfsetdash{}{0pt}%
\pgfpathmoveto{\pgfqpoint{0.766095in}{1.634725in}}%
\pgfpathlineto{\pgfqpoint{0.801138in}{1.625824in}}%
\pgfpathlineto{\pgfqpoint{0.816294in}{1.621982in}}%
\pgfpathlineto{\pgfqpoint{0.836180in}{1.617046in}}%
\pgfpathlineto{\pgfqpoint{0.871223in}{1.608361in}}%
\pgfpathlineto{\pgfqpoint{0.906265in}{1.599672in}}%
\pgfpathlineto{\pgfqpoint{0.922224in}{1.595723in}}%
\pgfpathlineto{\pgfqpoint{0.941308in}{1.591098in}}%
\pgfpathlineto{\pgfqpoint{0.976350in}{1.582617in}}%
\pgfpathlineto{\pgfqpoint{1.011393in}{1.574133in}}%
\pgfpathlineto{\pgfqpoint{1.030706in}{1.569463in}}%
\pgfpathlineto{\pgfqpoint{1.046435in}{1.565739in}}%
\pgfpathlineto{\pgfqpoint{1.081478in}{1.557456in}}%
\pgfpathlineto{\pgfqpoint{1.116520in}{1.549168in}}%
\pgfpathlineto{\pgfqpoint{1.141753in}{1.543204in}}%
\pgfpathlineto{\pgfqpoint{1.151563in}{1.540933in}}%
\pgfpathlineto{\pgfqpoint{1.186605in}{1.532840in}}%
\pgfpathlineto{\pgfqpoint{1.221648in}{1.524741in}}%
\pgfpathlineto{\pgfqpoint{1.255366in}{1.516944in}}%
\pgfpathlineto{\pgfqpoint{1.256691in}{1.516644in}}%
\pgfpathlineto{\pgfqpoint{1.291733in}{1.508734in}}%
\pgfpathlineto{\pgfqpoint{1.326776in}{1.500818in}}%
\pgfpathlineto{\pgfqpoint{1.361818in}{1.492895in}}%
\pgfpathlineto{\pgfqpoint{1.371614in}{1.490685in}}%
\pgfpathlineto{\pgfqpoint{1.396861in}{1.485107in}}%
\pgfpathlineto{\pgfqpoint{1.431903in}{1.477366in}}%
\pgfpathlineto{\pgfqpoint{1.466946in}{1.469619in}}%
\pgfpathlineto{\pgfqpoint{1.490441in}{1.464425in}}%
\pgfpathlineto{\pgfqpoint{1.501988in}{1.461926in}}%
\pgfpathlineto{\pgfqpoint{1.537031in}{1.454355in}}%
\pgfpathlineto{\pgfqpoint{1.572073in}{1.446777in}}%
\pgfpathlineto{\pgfqpoint{1.607116in}{1.439189in}}%
\pgfpathlineto{\pgfqpoint{1.611853in}{1.438166in}}%
\pgfpathlineto{\pgfqpoint{1.642158in}{1.431757in}}%
\pgfpathlineto{\pgfqpoint{1.677201in}{1.424341in}}%
\pgfpathlineto{\pgfqpoint{1.712244in}{1.416916in}}%
\pgfpathlineto{\pgfqpoint{1.735891in}{1.411906in}}%
\pgfpathlineto{\pgfqpoint{1.747286in}{1.409543in}}%
\pgfpathlineto{\pgfqpoint{1.782329in}{1.402285in}}%
\pgfpathlineto{\pgfqpoint{1.817371in}{1.395017in}}%
\pgfpathlineto{\pgfqpoint{1.852414in}{1.387740in}}%
\pgfpathlineto{\pgfqpoint{1.862508in}{1.385647in}}%
\pgfpathlineto{\pgfqpoint{1.887456in}{1.380582in}}%
\pgfpathlineto{\pgfqpoint{1.922499in}{1.373467in}}%
\pgfpathlineto{\pgfqpoint{1.957541in}{1.366342in}}%
\pgfpathlineto{\pgfqpoint{1.991694in}{1.359388in}}%
\pgfpathlineto{\pgfqpoint{1.992584in}{1.359210in}}%
\pgfpathlineto{\pgfqpoint{2.027626in}{1.352242in}}%
\pgfpathlineto{\pgfqpoint{2.062669in}{1.345264in}}%
\pgfpathlineto{\pgfqpoint{2.097711in}{1.338274in}}%
\pgfpathlineto{\pgfqpoint{2.123499in}{1.333128in}}%
\pgfpathlineto{\pgfqpoint{2.132754in}{1.331320in}}%
\pgfpathlineto{\pgfqpoint{2.167797in}{1.324484in}}%
\pgfpathlineto{\pgfqpoint{2.202839in}{1.317636in}}%
\pgfpathlineto{\pgfqpoint{2.237882in}{1.310777in}}%
\pgfpathlineto{\pgfqpoint{2.257852in}{1.306869in}}%
\pgfpathlineto{\pgfqpoint{2.272924in}{1.303981in}}%
\pgfpathlineto{\pgfqpoint{2.307967in}{1.297271in}}%
\pgfpathlineto{\pgfqpoint{2.343009in}{1.290549in}}%
\pgfpathlineto{\pgfqpoint{2.378052in}{1.283816in}}%
\pgfpathlineto{\pgfqpoint{2.394746in}{1.280609in}}%
\pgfpathlineto{\pgfqpoint{2.413094in}{1.277159in}}%
\pgfpathlineto{\pgfqpoint{2.448137in}{1.270571in}}%
\pgfpathlineto{\pgfqpoint{2.483179in}{1.263970in}}%
\pgfpathlineto{\pgfqpoint{2.518222in}{1.257358in}}%
\pgfpathlineto{\pgfqpoint{2.534166in}{1.254350in}}%
\pgfpathlineto{\pgfqpoint{2.553264in}{1.250823in}}%
\pgfpathlineto{\pgfqpoint{2.588307in}{1.244352in}}%
\pgfpathlineto{\pgfqpoint{2.623350in}{1.237868in}}%
\pgfpathlineto{\pgfqpoint{2.658392in}{1.231371in}}%
\pgfpathlineto{\pgfqpoint{2.676091in}{1.228090in}}%
\pgfpathlineto{\pgfqpoint{2.693435in}{1.224943in}}%
\pgfpathlineto{\pgfqpoint{2.728477in}{1.218585in}}%
\pgfpathlineto{\pgfqpoint{2.763520in}{1.212214in}}%
\pgfpathlineto{\pgfqpoint{2.798562in}{1.205829in}}%
\pgfpathlineto{\pgfqpoint{2.820495in}{1.201831in}}%
\pgfpathlineto{\pgfqpoint{2.833605in}{1.199492in}}%
\pgfpathlineto{\pgfqpoint{2.868647in}{1.193242in}}%
\pgfpathlineto{\pgfqpoint{2.903690in}{1.186980in}}%
\pgfpathlineto{\pgfqpoint{2.938732in}{1.180704in}}%
\pgfpathlineto{\pgfqpoint{2.967346in}{1.175571in}}%
\pgfpathlineto{\pgfqpoint{2.973775in}{1.174443in}}%
\pgfpathlineto{\pgfqpoint{3.008818in}{1.168299in}}%
\pgfpathlineto{\pgfqpoint{3.043860in}{1.162141in}}%
\pgfpathlineto{\pgfqpoint{3.078903in}{1.155970in}}%
\pgfpathlineto{\pgfqpoint{3.113945in}{1.149784in}}%
\pgfpathlineto{\pgfqpoint{3.116623in}{1.149312in}}%
\pgfpathlineto{\pgfqpoint{3.148988in}{1.143730in}}%
\pgfpathlineto{\pgfqpoint{3.184030in}{1.137674in}}%
\pgfpathlineto{\pgfqpoint{3.219073in}{1.131603in}}%
\pgfpathlineto{\pgfqpoint{3.254115in}{1.125519in}}%
\pgfpathlineto{\pgfqpoint{3.268322in}{1.123052in}}%
\pgfpathlineto{\pgfqpoint{3.289158in}{1.119512in}}%
\pgfpathlineto{\pgfqpoint{3.324200in}{1.113555in}}%
\pgfpathlineto{\pgfqpoint{3.359243in}{1.107583in}}%
\pgfpathlineto{\pgfqpoint{3.394285in}{1.101597in}}%
\pgfpathlineto{\pgfqpoint{3.422359in}{1.096793in}}%
\pgfpathlineto{\pgfqpoint{3.429328in}{1.095626in}}%
\pgfpathlineto{\pgfqpoint{3.464371in}{1.089764in}}%
\pgfpathlineto{\pgfqpoint{3.499413in}{1.083888in}}%
\pgfpathlineto{\pgfqpoint{3.534456in}{1.077997in}}%
\pgfpathlineto{\pgfqpoint{3.551946in}{1.075049in}}%
\pgfusepath{stroke}%
\end{pgfscope}%
\begin{pgfscope}%
\pgfpathrectangle{\pgfqpoint{0.766095in}{0.571603in}}{\pgfqpoint{6.973465in}{5.225635in}}%
\pgfusepath{clip}%
\pgfsetbuttcap%
\pgfsetroundjoin%
\pgfsetlinewidth{1.505625pt}%
\definecolor{currentstroke}{rgb}{0.190631,0.407061,0.556089}%
\pgfsetstrokecolor{currentstroke}%
\pgfsetdash{}{0pt}%
\pgfpathmoveto{\pgfqpoint{3.937573in}{1.011730in}}%
\pgfpathlineto{\pgfqpoint{3.954966in}{1.008940in}}%
\pgfpathlineto{\pgfqpoint{3.990009in}{1.003305in}}%
\pgfpathlineto{\pgfqpoint{4.025051in}{0.997653in}}%
\pgfpathlineto{\pgfqpoint{4.060094in}{0.991986in}}%
\pgfpathlineto{\pgfqpoint{4.061527in}{0.991755in}}%
\pgfpathlineto{\pgfqpoint{4.095136in}{0.986445in}}%
\pgfpathlineto{\pgfqpoint{4.130179in}{0.980895in}}%
\pgfpathlineto{\pgfqpoint{4.165221in}{0.975330in}}%
\pgfpathlineto{\pgfqpoint{4.200264in}{0.969748in}}%
\pgfpathlineto{\pgfqpoint{4.226916in}{0.965495in}}%
\pgfpathlineto{\pgfqpoint{4.235306in}{0.964186in}}%
\pgfpathlineto{\pgfqpoint{4.270349in}{0.958719in}}%
\pgfpathlineto{\pgfqpoint{4.305391in}{0.953237in}}%
\pgfpathlineto{\pgfqpoint{4.340434in}{0.947740in}}%
\pgfpathlineto{\pgfqpoint{4.375477in}{0.942226in}}%
\pgfpathlineto{\pgfqpoint{4.394462in}{0.939236in}}%
\pgfpathlineto{\pgfqpoint{4.410519in}{0.936763in}}%
\pgfpathlineto{\pgfqpoint{4.445562in}{0.931363in}}%
\pgfpathlineto{\pgfqpoint{4.480604in}{0.925947in}}%
\pgfpathlineto{\pgfqpoint{4.515647in}{0.920515in}}%
\pgfpathlineto{\pgfqpoint{4.550689in}{0.915067in}}%
\pgfpathlineto{\pgfqpoint{4.564131in}{0.912976in}}%
\pgfpathlineto{\pgfqpoint{4.585732in}{0.909692in}}%
\pgfpathlineto{\pgfqpoint{4.620774in}{0.904356in}}%
\pgfpathlineto{\pgfqpoint{4.655817in}{0.899004in}}%
\pgfpathlineto{\pgfqpoint{4.690859in}{0.893637in}}%
\pgfpathlineto{\pgfqpoint{4.725902in}{0.888253in}}%
\pgfpathlineto{\pgfqpoint{4.735897in}{0.886717in}}%
\pgfpathlineto{\pgfqpoint{4.760944in}{0.882955in}}%
\pgfpathlineto{\pgfqpoint{4.795987in}{0.877681in}}%
\pgfpathlineto{\pgfqpoint{4.831030in}{0.872392in}}%
\pgfpathlineto{\pgfqpoint{4.866072in}{0.867087in}}%
\pgfpathlineto{\pgfqpoint{4.901115in}{0.861766in}}%
\pgfpathlineto{\pgfqpoint{4.909729in}{0.860458in}}%
\pgfpathlineto{\pgfqpoint{4.936157in}{0.856535in}}%
\pgfpathlineto{\pgfqpoint{4.971200in}{0.851322in}}%
\pgfpathlineto{\pgfqpoint{5.006242in}{0.846094in}}%
\pgfpathlineto{\pgfqpoint{5.041285in}{0.840850in}}%
\pgfpathlineto{\pgfqpoint{5.076327in}{0.835590in}}%
\pgfpathlineto{\pgfqpoint{5.085597in}{0.834198in}}%
\pgfpathlineto{\pgfqpoint{5.111370in}{0.830416in}}%
\pgfpathlineto{\pgfqpoint{5.146412in}{0.825264in}}%
\pgfpathlineto{\pgfqpoint{5.181455in}{0.820096in}}%
\pgfpathlineto{\pgfqpoint{5.216497in}{0.814911in}}%
\pgfpathlineto{\pgfqpoint{5.251540in}{0.809710in}}%
\pgfpathlineto{\pgfqpoint{5.263470in}{0.807939in}}%
\pgfpathlineto{\pgfqpoint{5.286583in}{0.804585in}}%
\pgfpathlineto{\pgfqpoint{5.321625in}{0.799491in}}%
\pgfpathlineto{\pgfqpoint{5.356668in}{0.794381in}}%
\pgfpathlineto{\pgfqpoint{5.391710in}{0.789255in}}%
\pgfpathlineto{\pgfqpoint{5.426753in}{0.784112in}}%
\pgfpathlineto{\pgfqpoint{5.443315in}{0.781679in}}%
\pgfpathlineto{\pgfqpoint{5.461795in}{0.779026in}}%
\pgfpathlineto{\pgfqpoint{5.496838in}{0.773990in}}%
\pgfpathlineto{\pgfqpoint{5.531880in}{0.768937in}}%
\pgfpathlineto{\pgfqpoint{5.566923in}{0.763868in}}%
\pgfpathlineto{\pgfqpoint{5.601965in}{0.758783in}}%
\pgfpathlineto{\pgfqpoint{5.625100in}{0.755420in}}%
\pgfpathlineto{\pgfqpoint{5.637008in}{0.753728in}}%
\pgfpathlineto{\pgfqpoint{5.672050in}{0.748748in}}%
\pgfpathlineto{\pgfqpoint{5.707093in}{0.743752in}}%
\pgfpathlineto{\pgfqpoint{5.742136in}{0.738739in}}%
\pgfpathlineto{\pgfqpoint{5.777178in}{0.733710in}}%
\pgfpathlineto{\pgfqpoint{5.808791in}{0.729160in}}%
\pgfpathlineto{\pgfqpoint{5.812221in}{0.728678in}}%
\pgfpathlineto{\pgfqpoint{5.847263in}{0.723753in}}%
\pgfpathlineto{\pgfqpoint{5.882306in}{0.718812in}}%
\pgfpathlineto{\pgfqpoint{5.917348in}{0.713855in}}%
\pgfpathlineto{\pgfqpoint{5.952391in}{0.708882in}}%
\pgfpathlineto{\pgfqpoint{5.987433in}{0.703891in}}%
\pgfpathlineto{\pgfqpoint{5.994391in}{0.702901in}}%
\pgfpathlineto{\pgfqpoint{6.022476in}{0.698994in}}%
\pgfpathlineto{\pgfqpoint{6.057518in}{0.694108in}}%
\pgfpathlineto{\pgfqpoint{6.092561in}{0.689205in}}%
\pgfpathlineto{\pgfqpoint{6.127603in}{0.684286in}}%
\pgfpathlineto{\pgfqpoint{6.162646in}{0.679351in}}%
\pgfpathlineto{\pgfqpoint{6.181857in}{0.676641in}}%
\pgfpathlineto{\pgfqpoint{6.197689in}{0.674460in}}%
\pgfpathlineto{\pgfqpoint{6.232731in}{0.669628in}}%
\pgfpathlineto{\pgfqpoint{6.267774in}{0.664779in}}%
\pgfpathlineto{\pgfqpoint{6.302816in}{0.659914in}}%
\pgfpathlineto{\pgfqpoint{6.337859in}{0.655033in}}%
\pgfpathlineto{\pgfqpoint{6.371139in}{0.650382in}}%
\pgfpathlineto{\pgfqpoint{6.372901in}{0.650141in}}%
\pgfpathlineto{\pgfqpoint{6.407944in}{0.645362in}}%
\pgfpathlineto{\pgfqpoint{6.442986in}{0.640566in}}%
\pgfpathlineto{\pgfqpoint{6.478029in}{0.635754in}}%
\pgfpathlineto{\pgfqpoint{6.513071in}{0.630926in}}%
\pgfpathlineto{\pgfqpoint{6.548114in}{0.626082in}}%
\pgfpathlineto{\pgfqpoint{6.562273in}{0.624122in}}%
\pgfpathlineto{\pgfqpoint{6.583156in}{0.621300in}}%
\pgfpathlineto{\pgfqpoint{6.618199in}{0.616557in}}%
\pgfpathlineto{\pgfqpoint{6.653242in}{0.611798in}}%
\pgfpathlineto{\pgfqpoint{6.688284in}{0.607023in}}%
\pgfpathlineto{\pgfqpoint{6.723327in}{0.602231in}}%
\pgfpathlineto{\pgfqpoint{6.755173in}{0.597863in}}%
\pgfpathlineto{\pgfqpoint{6.758369in}{0.597435in}}%
\pgfpathlineto{\pgfqpoint{6.793412in}{0.592744in}}%
\pgfpathlineto{\pgfqpoint{6.828454in}{0.588037in}}%
\pgfpathlineto{\pgfqpoint{6.863497in}{0.583313in}}%
\pgfpathlineto{\pgfqpoint{6.898539in}{0.578574in}}%
\pgfpathlineto{\pgfqpoint{6.933582in}{0.573818in}}%
\pgfpathlineto{\pgfqpoint{6.949877in}{0.571603in}}%
\pgfusepath{stroke}%
\end{pgfscope}%
\begin{pgfscope}%
\pgfpathrectangle{\pgfqpoint{0.766095in}{0.571603in}}{\pgfqpoint{6.973465in}{5.225635in}}%
\pgfusepath{clip}%
\pgfsetbuttcap%
\pgfsetroundjoin%
\pgfsetlinewidth{1.505625pt}%
\definecolor{currentstroke}{rgb}{0.179019,0.433756,0.557430}%
\pgfsetstrokecolor{currentstroke}%
\pgfsetdash{}{0pt}%
\pgfpathmoveto{\pgfqpoint{0.766095in}{1.562226in}}%
\pgfpathlineto{\pgfqpoint{0.976350in}{1.511805in}}%
\pgfpathlineto{\pgfqpoint{1.221648in}{1.455727in}}%
\pgfpathlineto{\pgfqpoint{1.466946in}{1.402189in}}%
\pgfpathlineto{\pgfqpoint{1.712244in}{1.350882in}}%
\pgfpathlineto{\pgfqpoint{1.992584in}{1.294661in}}%
\pgfpathlineto{\pgfqpoint{2.272924in}{1.240674in}}%
\pgfpathlineto{\pgfqpoint{2.553264in}{1.188654in}}%
\pgfpathlineto{\pgfqpoint{2.868647in}{1.132186in}}%
\pgfpathlineto{\pgfqpoint{3.148988in}{1.083627in}}%
\pgfpathlineto{\pgfqpoint{3.499413in}{1.024786in}}%
\pgfpathlineto{\pgfqpoint{3.814796in}{0.973444in}}%
\pgfpathlineto{\pgfqpoint{4.130179in}{0.923444in}}%
\pgfpathlineto{\pgfqpoint{4.515647in}{0.863901in}}%
\pgfpathlineto{\pgfqpoint{4.831030in}{0.816446in}}%
\pgfpathlineto{\pgfqpoint{5.245513in}{0.755420in}}%
\pgfpathlineto{\pgfqpoint{5.688826in}{0.691959in}}%
\pgfpathlineto{\pgfqpoint{5.688826in}{0.691959in}}%
\pgfusepath{stroke}%
\end{pgfscope}%
\begin{pgfscope}%
\pgfpathrectangle{\pgfqpoint{0.766095in}{0.571603in}}{\pgfqpoint{6.973465in}{5.225635in}}%
\pgfusepath{clip}%
\pgfsetbuttcap%
\pgfsetroundjoin%
\pgfsetlinewidth{1.505625pt}%
\definecolor{currentstroke}{rgb}{0.179019,0.433756,0.557430}%
\pgfsetstrokecolor{currentstroke}%
\pgfsetdash{}{0pt}%
\pgfpathmoveto{\pgfqpoint{6.075823in}{0.637745in}}%
\pgfpathlineto{\pgfqpoint{6.092561in}{0.635425in}}%
\pgfpathlineto{\pgfqpoint{6.127603in}{0.630553in}}%
\pgfpathlineto{\pgfqpoint{6.162646in}{0.625664in}}%
\pgfpathlineto{\pgfqpoint{6.173692in}{0.624122in}}%
\pgfpathlineto{\pgfqpoint{6.197689in}{0.620849in}}%
\pgfpathlineto{\pgfqpoint{6.232731in}{0.616061in}}%
\pgfpathlineto{\pgfqpoint{6.267774in}{0.611256in}}%
\pgfpathlineto{\pgfqpoint{6.302816in}{0.606437in}}%
\pgfpathlineto{\pgfqpoint{6.337859in}{0.601601in}}%
\pgfpathlineto{\pgfqpoint{6.364887in}{0.597863in}}%
\pgfpathlineto{\pgfqpoint{6.372901in}{0.596780in}}%
\pgfpathlineto{\pgfqpoint{6.407944in}{0.592043in}}%
\pgfpathlineto{\pgfqpoint{6.442986in}{0.587291in}}%
\pgfpathlineto{\pgfqpoint{6.478029in}{0.582524in}}%
\pgfpathlineto{\pgfqpoint{6.513071in}{0.577741in}}%
\pgfpathlineto{\pgfqpoint{6.548114in}{0.572942in}}%
\pgfpathlineto{\pgfqpoint{6.557885in}{0.571603in}}%
\pgfusepath{stroke}%
\end{pgfscope}%
\begin{pgfscope}%
\pgfpathrectangle{\pgfqpoint{0.766095in}{0.571603in}}{\pgfqpoint{6.973465in}{5.225635in}}%
\pgfusepath{clip}%
\pgfsetbuttcap%
\pgfsetroundjoin%
\pgfsetlinewidth{1.505625pt}%
\definecolor{currentstroke}{rgb}{0.168126,0.459988,0.558082}%
\pgfsetstrokecolor{currentstroke}%
\pgfsetdash{}{0pt}%
\pgfpathmoveto{\pgfqpoint{0.766095in}{1.494323in}}%
\pgfpathlineto{\pgfqpoint{0.781294in}{1.490685in}}%
\pgfpathlineto{\pgfqpoint{0.801138in}{1.486030in}}%
\pgfpathlineto{\pgfqpoint{0.836180in}{1.477822in}}%
\pgfpathlineto{\pgfqpoint{0.871223in}{1.469613in}}%
\pgfpathlineto{\pgfqpoint{0.893395in}{1.464425in}}%
\pgfpathlineto{\pgfqpoint{0.906265in}{1.461474in}}%
\pgfpathlineto{\pgfqpoint{0.941308in}{1.453457in}}%
\pgfpathlineto{\pgfqpoint{0.976350in}{1.445437in}}%
\pgfpathlineto{\pgfqpoint{1.008123in}{1.438166in}}%
\pgfpathlineto{\pgfqpoint{1.011393in}{1.437433in}}%
\pgfpathlineto{\pgfqpoint{1.046435in}{1.429598in}}%
\pgfpathlineto{\pgfqpoint{1.081478in}{1.421761in}}%
\pgfpathlineto{\pgfqpoint{1.116520in}{1.413920in}}%
\pgfpathlineto{\pgfqpoint{1.125538in}{1.411906in}}%
\pgfpathlineto{\pgfqpoint{1.151563in}{1.406213in}}%
\pgfpathlineto{\pgfqpoint{1.186605in}{1.398551in}}%
\pgfpathlineto{\pgfqpoint{1.221648in}{1.390885in}}%
\pgfpathlineto{\pgfqpoint{1.245603in}{1.385647in}}%
\pgfpathlineto{\pgfqpoint{1.256691in}{1.383271in}}%
\pgfpathlineto{\pgfqpoint{1.291733in}{1.375778in}}%
\pgfpathlineto{\pgfqpoint{1.326776in}{1.368280in}}%
\pgfpathlineto{\pgfqpoint{1.361818in}{1.360777in}}%
\pgfpathlineto{\pgfqpoint{1.368320in}{1.359388in}}%
\pgfpathlineto{\pgfqpoint{1.396861in}{1.353413in}}%
\pgfpathlineto{\pgfqpoint{1.431903in}{1.346078in}}%
\pgfpathlineto{\pgfqpoint{1.466946in}{1.338736in}}%
\pgfpathlineto{\pgfqpoint{1.493711in}{1.333128in}}%
\pgfpathlineto{\pgfqpoint{1.501988in}{1.331429in}}%
\pgfpathlineto{\pgfqpoint{1.537031in}{1.324250in}}%
\pgfpathlineto{\pgfqpoint{1.572073in}{1.317064in}}%
\pgfpathlineto{\pgfqpoint{1.607116in}{1.309871in}}%
\pgfpathlineto{\pgfqpoint{1.621763in}{1.306869in}}%
\pgfpathlineto{\pgfqpoint{1.642158in}{1.302772in}}%
\pgfpathlineto{\pgfqpoint{1.677201in}{1.295737in}}%
\pgfpathlineto{\pgfqpoint{1.712244in}{1.288695in}}%
\pgfpathlineto{\pgfqpoint{1.747286in}{1.281645in}}%
\pgfpathlineto{\pgfqpoint{1.752444in}{1.280609in}}%
\pgfpathlineto{\pgfqpoint{1.782329in}{1.274731in}}%
\pgfpathlineto{\pgfqpoint{1.817371in}{1.267834in}}%
\pgfpathlineto{\pgfqpoint{1.852414in}{1.260929in}}%
\pgfpathlineto{\pgfqpoint{1.885766in}{1.254350in}}%
\pgfpathlineto{\pgfqpoint{1.887456in}{1.254023in}}%
\pgfpathlineto{\pgfqpoint{1.922499in}{1.247267in}}%
\pgfpathlineto{\pgfqpoint{1.957541in}{1.240502in}}%
\pgfpathlineto{\pgfqpoint{1.992584in}{1.233728in}}%
\pgfpathlineto{\pgfqpoint{2.021729in}{1.228090in}}%
\pgfpathlineto{\pgfqpoint{2.027626in}{1.226973in}}%
\pgfpathlineto{\pgfqpoint{2.062669in}{1.220344in}}%
\pgfpathlineto{\pgfqpoint{2.097711in}{1.213706in}}%
\pgfpathlineto{\pgfqpoint{2.132754in}{1.207058in}}%
\pgfpathlineto{\pgfqpoint{2.160289in}{1.201831in}}%
\pgfpathlineto{\pgfqpoint{2.167797in}{1.200434in}}%
\pgfpathlineto{\pgfqpoint{2.202839in}{1.193928in}}%
\pgfpathlineto{\pgfqpoint{2.237882in}{1.187411in}}%
\pgfpathlineto{\pgfqpoint{2.272924in}{1.180884in}}%
\pgfpathlineto{\pgfqpoint{2.301424in}{1.175571in}}%
\pgfpathlineto{\pgfqpoint{2.307967in}{1.174377in}}%
\pgfpathlineto{\pgfqpoint{2.343009in}{1.167987in}}%
\pgfpathlineto{\pgfqpoint{2.378052in}{1.161587in}}%
\pgfpathlineto{\pgfqpoint{2.413094in}{1.155177in}}%
\pgfpathlineto{\pgfqpoint{2.445112in}{1.149312in}}%
\pgfpathlineto{\pgfqpoint{2.448137in}{1.148769in}}%
\pgfpathlineto{\pgfqpoint{2.483179in}{1.142492in}}%
\pgfpathlineto{\pgfqpoint{2.518222in}{1.136205in}}%
\pgfpathlineto{\pgfqpoint{2.553264in}{1.129907in}}%
\pgfpathlineto{\pgfqpoint{2.588307in}{1.123597in}}%
\pgfpathlineto{\pgfqpoint{2.591337in}{1.123052in}}%
\pgfpathlineto{\pgfqpoint{2.623350in}{1.117416in}}%
\pgfpathlineto{\pgfqpoint{2.658392in}{1.111237in}}%
\pgfpathlineto{\pgfqpoint{2.693435in}{1.105047in}}%
\pgfpathlineto{\pgfqpoint{2.728477in}{1.098845in}}%
\pgfpathlineto{\pgfqpoint{2.740078in}{1.096793in}}%
\pgfpathlineto{\pgfqpoint{2.763520in}{1.092732in}}%
\pgfpathlineto{\pgfqpoint{2.798562in}{1.086658in}}%
\pgfpathlineto{\pgfqpoint{2.833605in}{1.080572in}}%
\pgfpathlineto{\pgfqpoint{2.868647in}{1.074473in}}%
\pgfpathlineto{\pgfqpoint{2.886283in}{1.071402in}}%
\pgfusepath{stroke}%
\end{pgfscope}%
\begin{pgfscope}%
\pgfpathrectangle{\pgfqpoint{0.766095in}{0.571603in}}{\pgfqpoint{6.973465in}{5.225635in}}%
\pgfusepath{clip}%
\pgfsetbuttcap%
\pgfsetroundjoin%
\pgfsetlinewidth{1.505625pt}%
\definecolor{currentstroke}{rgb}{0.168126,0.459988,0.558082}%
\pgfsetstrokecolor{currentstroke}%
\pgfsetdash{}{0pt}%
\pgfpathmoveto{\pgfqpoint{3.271620in}{1.006325in}}%
\pgfpathlineto{\pgfqpoint{3.289158in}{1.003419in}}%
\pgfpathlineto{\pgfqpoint{3.324200in}{0.997600in}}%
\pgfpathlineto{\pgfqpoint{3.359243in}{0.991768in}}%
\pgfpathlineto{\pgfqpoint{3.359321in}{0.991755in}}%
\pgfpathlineto{\pgfqpoint{3.394285in}{0.986066in}}%
\pgfpathlineto{\pgfqpoint{3.429328in}{0.980352in}}%
\pgfpathlineto{\pgfqpoint{3.464371in}{0.974624in}}%
\pgfpathlineto{\pgfqpoint{3.499413in}{0.968883in}}%
\pgfpathlineto{\pgfqpoint{3.520076in}{0.965495in}}%
\pgfpathlineto{\pgfqpoint{3.534456in}{0.963187in}}%
\pgfpathlineto{\pgfqpoint{3.569498in}{0.957562in}}%
\pgfpathlineto{\pgfqpoint{3.604541in}{0.951923in}}%
\pgfpathlineto{\pgfqpoint{3.639583in}{0.946271in}}%
\pgfpathlineto{\pgfqpoint{3.674626in}{0.940605in}}%
\pgfpathlineto{\pgfqpoint{3.683095in}{0.939236in}}%
\pgfpathlineto{\pgfqpoint{3.709668in}{0.935032in}}%
\pgfpathlineto{\pgfqpoint{3.744711in}{0.929479in}}%
\pgfpathlineto{\pgfqpoint{3.779753in}{0.923914in}}%
\pgfpathlineto{\pgfqpoint{3.814796in}{0.918334in}}%
\pgfpathlineto{\pgfqpoint{3.848363in}{0.912976in}}%
\pgfpathlineto{\pgfqpoint{3.849838in}{0.912746in}}%
\pgfpathlineto{\pgfqpoint{3.884881in}{0.907278in}}%
\pgfpathlineto{\pgfqpoint{3.919924in}{0.901796in}}%
\pgfpathlineto{\pgfqpoint{3.954966in}{0.896300in}}%
\pgfpathlineto{\pgfqpoint{3.990009in}{0.890791in}}%
\pgfpathlineto{\pgfqpoint{4.015876in}{0.886717in}}%
\pgfpathlineto{\pgfqpoint{4.025051in}{0.885303in}}%
\pgfpathlineto{\pgfqpoint{4.060094in}{0.879903in}}%
\pgfpathlineto{\pgfqpoint{4.095136in}{0.874489in}}%
\pgfpathlineto{\pgfqpoint{4.130179in}{0.869061in}}%
\pgfpathlineto{\pgfqpoint{4.165221in}{0.863619in}}%
\pgfpathlineto{\pgfqpoint{4.185561in}{0.860458in}}%
\pgfpathlineto{\pgfqpoint{4.200264in}{0.858220in}}%
\pgfpathlineto{\pgfqpoint{4.235306in}{0.852887in}}%
\pgfpathlineto{\pgfqpoint{4.270349in}{0.847539in}}%
\pgfpathlineto{\pgfqpoint{4.305391in}{0.842177in}}%
\pgfpathlineto{\pgfqpoint{4.340434in}{0.836801in}}%
\pgfpathlineto{\pgfqpoint{4.357385in}{0.834198in}}%
\pgfpathlineto{\pgfqpoint{4.375477in}{0.831480in}}%
\pgfpathlineto{\pgfqpoint{4.410519in}{0.826210in}}%
\pgfpathlineto{\pgfqpoint{4.445562in}{0.820927in}}%
\pgfpathlineto{\pgfqpoint{4.480604in}{0.815629in}}%
\pgfpathlineto{\pgfqpoint{4.515647in}{0.810316in}}%
\pgfpathlineto{\pgfqpoint{4.531317in}{0.807939in}}%
\pgfpathlineto{\pgfqpoint{4.550689in}{0.805063in}}%
\pgfpathlineto{\pgfqpoint{4.585732in}{0.799856in}}%
\pgfpathlineto{\pgfqpoint{4.620774in}{0.794634in}}%
\pgfpathlineto{\pgfqpoint{4.655817in}{0.789398in}}%
\pgfpathlineto{\pgfqpoint{4.690859in}{0.784148in}}%
\pgfpathlineto{\pgfqpoint{4.707321in}{0.781679in}}%
\pgfpathlineto{\pgfqpoint{4.725902in}{0.778953in}}%
\pgfpathlineto{\pgfqpoint{4.760944in}{0.773806in}}%
\pgfpathlineto{\pgfqpoint{4.795987in}{0.768646in}}%
\pgfpathlineto{\pgfqpoint{4.831030in}{0.763470in}}%
\pgfpathlineto{\pgfqpoint{4.866072in}{0.758280in}}%
\pgfpathlineto{\pgfqpoint{4.885362in}{0.755420in}}%
\pgfpathlineto{\pgfqpoint{4.901115in}{0.753134in}}%
\pgfpathlineto{\pgfqpoint{4.936157in}{0.748047in}}%
\pgfpathlineto{\pgfqpoint{4.971200in}{0.742945in}}%
\pgfpathlineto{\pgfqpoint{5.006242in}{0.737829in}}%
\pgfpathlineto{\pgfqpoint{5.041285in}{0.732698in}}%
\pgfpathlineto{\pgfqpoint{5.065403in}{0.729160in}}%
\pgfpathlineto{\pgfqpoint{5.076327in}{0.727593in}}%
\pgfpathlineto{\pgfqpoint{5.111370in}{0.722563in}}%
\pgfpathlineto{\pgfqpoint{5.146412in}{0.717519in}}%
\pgfpathlineto{\pgfqpoint{5.181455in}{0.712461in}}%
\pgfpathlineto{\pgfqpoint{5.216497in}{0.707387in}}%
\pgfpathlineto{\pgfqpoint{5.247408in}{0.702901in}}%
\pgfpathlineto{\pgfqpoint{5.251540in}{0.702314in}}%
\pgfpathlineto{\pgfqpoint{5.286583in}{0.697341in}}%
\pgfpathlineto{\pgfqpoint{5.321625in}{0.692354in}}%
\pgfpathlineto{\pgfqpoint{5.356668in}{0.687352in}}%
\pgfpathlineto{\pgfqpoint{5.391710in}{0.682335in}}%
\pgfpathlineto{\pgfqpoint{5.426753in}{0.677303in}}%
\pgfpathlineto{\pgfqpoint{5.431363in}{0.676641in}}%
\pgfpathlineto{\pgfqpoint{5.461795in}{0.672369in}}%
\pgfpathlineto{\pgfqpoint{5.496838in}{0.667437in}}%
\pgfpathlineto{\pgfqpoint{5.531880in}{0.662490in}}%
\pgfpathlineto{\pgfqpoint{5.566923in}{0.657529in}}%
\pgfpathlineto{\pgfqpoint{5.601965in}{0.652552in}}%
\pgfpathlineto{\pgfqpoint{5.617236in}{0.650382in}}%
\pgfpathlineto{\pgfqpoint{5.637008in}{0.647633in}}%
\pgfpathlineto{\pgfqpoint{5.672050in}{0.642756in}}%
\pgfpathlineto{\pgfqpoint{5.707093in}{0.637864in}}%
\pgfpathlineto{\pgfqpoint{5.742136in}{0.632957in}}%
\pgfpathlineto{\pgfqpoint{5.777178in}{0.628035in}}%
\pgfpathlineto{\pgfqpoint{5.804970in}{0.624122in}}%
\pgfpathlineto{\pgfqpoint{5.812221in}{0.623124in}}%
\pgfpathlineto{\pgfqpoint{5.847263in}{0.618300in}}%
\pgfpathlineto{\pgfqpoint{5.882306in}{0.613462in}}%
\pgfpathlineto{\pgfqpoint{5.917348in}{0.608609in}}%
\pgfpathlineto{\pgfqpoint{5.952391in}{0.603740in}}%
\pgfpathlineto{\pgfqpoint{5.987433in}{0.598857in}}%
\pgfpathlineto{\pgfqpoint{5.994564in}{0.597863in}}%
\pgfpathlineto{\pgfqpoint{6.022476in}{0.594059in}}%
\pgfpathlineto{\pgfqpoint{6.057518in}{0.589274in}}%
\pgfpathlineto{\pgfqpoint{6.092561in}{0.584473in}}%
\pgfpathlineto{\pgfqpoint{6.127603in}{0.579658in}}%
\pgfpathlineto{\pgfqpoint{6.162646in}{0.574827in}}%
\pgfpathlineto{\pgfqpoint{6.185988in}{0.571603in}}%
\pgfusepath{stroke}%
\end{pgfscope}%
\begin{pgfscope}%
\pgfpathrectangle{\pgfqpoint{0.766095in}{0.571603in}}{\pgfqpoint{6.973465in}{5.225635in}}%
\pgfusepath{clip}%
\pgfsetbuttcap%
\pgfsetroundjoin%
\pgfsetlinewidth{1.505625pt}%
\definecolor{currentstroke}{rgb}{0.157729,0.485932,0.558013}%
\pgfsetstrokecolor{currentstroke}%
\pgfsetdash{}{0pt}%
\pgfpathmoveto{\pgfqpoint{0.766095in}{1.430458in}}%
\pgfpathlineto{\pgfqpoint{1.011393in}{1.375146in}}%
\pgfpathlineto{\pgfqpoint{1.256691in}{1.322338in}}%
\pgfpathlineto{\pgfqpoint{1.501988in}{1.271725in}}%
\pgfpathlineto{\pgfqpoint{1.782329in}{1.216254in}}%
\pgfpathlineto{\pgfqpoint{2.062669in}{1.162981in}}%
\pgfpathlineto{\pgfqpoint{2.343009in}{1.111637in}}%
\pgfpathlineto{\pgfqpoint{2.658392in}{1.055900in}}%
\pgfpathlineto{\pgfqpoint{2.973775in}{1.002043in}}%
\pgfpathlineto{\pgfqpoint{3.289158in}{0.949845in}}%
\pgfpathlineto{\pgfqpoint{3.639583in}{0.893545in}}%
\pgfpathlineto{\pgfqpoint{3.954966in}{0.844273in}}%
\pgfpathlineto{\pgfqpoint{4.340434in}{0.785563in}}%
\pgfpathlineto{\pgfqpoint{4.690859in}{0.733562in}}%
\pgfpathlineto{\pgfqpoint{5.023076in}{0.685340in}}%
\pgfpathlineto{\pgfqpoint{5.023076in}{0.685340in}}%
\pgfusepath{stroke}%
\end{pgfscope}%
\begin{pgfscope}%
\pgfpathrectangle{\pgfqpoint{0.766095in}{0.571603in}}{\pgfqpoint{6.973465in}{5.225635in}}%
\pgfusepath{clip}%
\pgfsetbuttcap%
\pgfsetroundjoin%
\pgfsetlinewidth{1.505625pt}%
\definecolor{currentstroke}{rgb}{0.157729,0.485932,0.558013}%
\pgfsetstrokecolor{currentstroke}%
\pgfsetdash{}{0pt}%
\pgfpathmoveto{\pgfqpoint{5.409960in}{0.630313in}}%
\pgfpathlineto{\pgfqpoint{5.426753in}{0.627935in}}%
\pgfpathlineto{\pgfqpoint{5.453618in}{0.624122in}}%
\pgfpathlineto{\pgfqpoint{5.461795in}{0.622987in}}%
\pgfpathlineto{\pgfqpoint{5.496838in}{0.618122in}}%
\pgfpathlineto{\pgfqpoint{5.531880in}{0.613242in}}%
\pgfpathlineto{\pgfqpoint{5.566923in}{0.608349in}}%
\pgfpathlineto{\pgfqpoint{5.601965in}{0.603441in}}%
\pgfpathlineto{\pgfqpoint{5.637008in}{0.598518in}}%
\pgfpathlineto{\pgfqpoint{5.641674in}{0.597863in}}%
\pgfpathlineto{\pgfqpoint{5.672050in}{0.593690in}}%
\pgfpathlineto{\pgfqpoint{5.707093in}{0.588864in}}%
\pgfpathlineto{\pgfqpoint{5.742136in}{0.584024in}}%
\pgfpathlineto{\pgfqpoint{5.777178in}{0.579169in}}%
\pgfpathlineto{\pgfqpoint{5.812221in}{0.574300in}}%
\pgfpathlineto{\pgfqpoint{5.831605in}{0.571603in}}%
\pgfusepath{stroke}%
\end{pgfscope}%
\begin{pgfscope}%
\pgfpathrectangle{\pgfqpoint{0.766095in}{0.571603in}}{\pgfqpoint{6.973465in}{5.225635in}}%
\pgfusepath{clip}%
\pgfsetbuttcap%
\pgfsetroundjoin%
\pgfsetlinewidth{1.505625pt}%
\definecolor{currentstroke}{rgb}{0.147607,0.511733,0.557049}%
\pgfsetstrokecolor{currentstroke}%
\pgfsetdash{}{0pt}%
\pgfpathmoveto{\pgfqpoint{0.766095in}{1.370059in}}%
\pgfpathlineto{\pgfqpoint{0.801138in}{1.362151in}}%
\pgfpathlineto{\pgfqpoint{0.813409in}{1.359388in}}%
\pgfpathlineto{\pgfqpoint{0.836180in}{1.354361in}}%
\pgfpathlineto{\pgfqpoint{0.871223in}{1.346634in}}%
\pgfpathlineto{\pgfqpoint{0.906265in}{1.338906in}}%
\pgfpathlineto{\pgfqpoint{0.932486in}{1.333128in}}%
\pgfpathlineto{\pgfqpoint{0.941308in}{1.331222in}}%
\pgfpathlineto{\pgfqpoint{0.976350in}{1.323669in}}%
\pgfpathlineto{\pgfqpoint{1.011393in}{1.316115in}}%
\pgfpathlineto{\pgfqpoint{1.046435in}{1.308558in}}%
\pgfpathlineto{\pgfqpoint{1.054291in}{1.306869in}}%
\pgfpathlineto{\pgfqpoint{1.081478in}{1.301135in}}%
\pgfpathlineto{\pgfqpoint{1.116520in}{1.293747in}}%
\pgfpathlineto{\pgfqpoint{1.151563in}{1.286357in}}%
\pgfpathlineto{\pgfqpoint{1.178825in}{1.280609in}}%
\pgfpathlineto{\pgfqpoint{1.186605in}{1.279001in}}%
\pgfpathlineto{\pgfqpoint{1.221648in}{1.271774in}}%
\pgfpathlineto{\pgfqpoint{1.256691in}{1.264544in}}%
\pgfpathlineto{\pgfqpoint{1.291733in}{1.257310in}}%
\pgfpathlineto{\pgfqpoint{1.306093in}{1.254350in}}%
\pgfpathlineto{\pgfqpoint{1.326776in}{1.250170in}}%
\pgfpathlineto{\pgfqpoint{1.361818in}{1.243094in}}%
\pgfpathlineto{\pgfqpoint{1.396861in}{1.236014in}}%
\pgfpathlineto{\pgfqpoint{1.431903in}{1.228928in}}%
\pgfpathlineto{\pgfqpoint{1.436059in}{1.228090in}}%
\pgfpathlineto{\pgfqpoint{1.466946in}{1.221982in}}%
\pgfpathlineto{\pgfqpoint{1.501988in}{1.215050in}}%
\pgfpathlineto{\pgfqpoint{1.537031in}{1.208113in}}%
\pgfpathlineto{\pgfqpoint{1.568748in}{1.201831in}}%
\pgfpathlineto{\pgfqpoint{1.572073in}{1.201185in}}%
\pgfpathlineto{\pgfqpoint{1.607116in}{1.194397in}}%
\pgfpathlineto{\pgfqpoint{1.642158in}{1.187602in}}%
\pgfpathlineto{\pgfqpoint{1.677201in}{1.180801in}}%
\pgfpathlineto{\pgfqpoint{1.704146in}{1.175571in}}%
\pgfpathlineto{\pgfqpoint{1.712244in}{1.174030in}}%
\pgfpathlineto{\pgfqpoint{1.747286in}{1.167374in}}%
\pgfpathlineto{\pgfqpoint{1.782329in}{1.160711in}}%
\pgfpathlineto{\pgfqpoint{1.817371in}{1.154041in}}%
\pgfpathlineto{\pgfqpoint{1.842216in}{1.149312in}}%
\pgfpathlineto{\pgfqpoint{1.852414in}{1.147409in}}%
\pgfpathlineto{\pgfqpoint{1.887456in}{1.140879in}}%
\pgfpathlineto{\pgfqpoint{1.922499in}{1.134342in}}%
\pgfpathlineto{\pgfqpoint{1.957541in}{1.127797in}}%
\pgfpathlineto{\pgfqpoint{1.982940in}{1.123052in}}%
\pgfpathlineto{\pgfqpoint{1.992584in}{1.121286in}}%
\pgfpathlineto{\pgfqpoint{2.027626in}{1.114878in}}%
\pgfpathlineto{\pgfqpoint{2.062669in}{1.108462in}}%
\pgfpathlineto{\pgfqpoint{2.097711in}{1.102037in}}%
\pgfpathlineto{\pgfqpoint{2.126295in}{1.096793in}}%
\pgfpathlineto{\pgfqpoint{2.132754in}{1.095631in}}%
\pgfpathlineto{\pgfqpoint{2.167797in}{1.089339in}}%
\pgfpathlineto{\pgfqpoint{2.202839in}{1.083039in}}%
\pgfpathlineto{\pgfqpoint{2.228417in}{1.078434in}}%
\pgfusepath{stroke}%
\end{pgfscope}%
\begin{pgfscope}%
\pgfpathrectangle{\pgfqpoint{0.766095in}{0.571603in}}{\pgfqpoint{6.973465in}{5.225635in}}%
\pgfusepath{clip}%
\pgfsetbuttcap%
\pgfsetroundjoin%
\pgfsetlinewidth{1.505625pt}%
\definecolor{currentstroke}{rgb}{0.147607,0.511733,0.557049}%
\pgfsetstrokecolor{currentstroke}%
\pgfsetdash{}{0pt}%
\pgfpathmoveto{\pgfqpoint{2.613343in}{1.010944in}}%
\pgfpathlineto{\pgfqpoint{2.623350in}{1.009237in}}%
\pgfpathlineto{\pgfqpoint{2.658392in}{1.003249in}}%
\pgfpathlineto{\pgfqpoint{2.693435in}{0.997250in}}%
\pgfpathlineto{\pgfqpoint{2.725487in}{0.991755in}}%
\pgfpathlineto{\pgfqpoint{2.728477in}{0.991253in}}%
\pgfpathlineto{\pgfqpoint{2.763520in}{0.985374in}}%
\pgfpathlineto{\pgfqpoint{2.798562in}{0.979485in}}%
\pgfpathlineto{\pgfqpoint{2.833605in}{0.973586in}}%
\pgfpathlineto{\pgfqpoint{2.868647in}{0.967675in}}%
\pgfpathlineto{\pgfqpoint{2.881574in}{0.965495in}}%
\pgfpathlineto{\pgfqpoint{2.903690in}{0.961842in}}%
\pgfpathlineto{\pgfqpoint{2.938732in}{0.956049in}}%
\pgfpathlineto{\pgfqpoint{2.973775in}{0.950245in}}%
\pgfpathlineto{\pgfqpoint{3.008818in}{0.944430in}}%
\pgfpathlineto{\pgfqpoint{3.040070in}{0.939236in}}%
\pgfpathlineto{\pgfqpoint{3.043860in}{0.938619in}}%
\pgfpathlineto{\pgfqpoint{3.078903in}{0.932919in}}%
\pgfpathlineto{\pgfqpoint{3.113945in}{0.927208in}}%
\pgfpathlineto{\pgfqpoint{3.148988in}{0.921486in}}%
\pgfpathlineto{\pgfqpoint{3.184030in}{0.915752in}}%
\pgfpathlineto{\pgfqpoint{3.200990in}{0.912976in}}%
\pgfpathlineto{\pgfqpoint{3.219073in}{0.910077in}}%
\pgfpathlineto{\pgfqpoint{3.254115in}{0.904456in}}%
\pgfpathlineto{\pgfqpoint{3.289158in}{0.898824in}}%
\pgfpathlineto{\pgfqpoint{3.324200in}{0.893179in}}%
\pgfpathlineto{\pgfqpoint{3.359243in}{0.887523in}}%
\pgfpathlineto{\pgfqpoint{3.364241in}{0.886717in}}%
\pgfpathlineto{\pgfqpoint{3.394285in}{0.881970in}}%
\pgfpathlineto{\pgfqpoint{3.429328in}{0.876425in}}%
\pgfpathlineto{\pgfqpoint{3.464371in}{0.870868in}}%
\pgfpathlineto{\pgfqpoint{3.499413in}{0.865299in}}%
\pgfpathlineto{\pgfqpoint{3.529820in}{0.860458in}}%
\pgfpathlineto{\pgfqpoint{3.534456in}{0.859735in}}%
\pgfpathlineto{\pgfqpoint{3.569498in}{0.854274in}}%
\pgfpathlineto{\pgfqpoint{3.604541in}{0.848801in}}%
\pgfpathlineto{\pgfqpoint{3.639583in}{0.843317in}}%
\pgfpathlineto{\pgfqpoint{3.674626in}{0.837819in}}%
\pgfpathlineto{\pgfqpoint{3.697684in}{0.834198in}}%
\pgfpathlineto{\pgfqpoint{3.709668in}{0.832354in}}%
\pgfpathlineto{\pgfqpoint{3.744711in}{0.826964in}}%
\pgfpathlineto{\pgfqpoint{3.779753in}{0.821562in}}%
\pgfpathlineto{\pgfqpoint{3.814796in}{0.816147in}}%
\pgfpathlineto{\pgfqpoint{3.849838in}{0.810719in}}%
\pgfpathlineto{\pgfqpoint{3.867776in}{0.807939in}}%
\pgfpathlineto{\pgfqpoint{3.884881in}{0.805342in}}%
\pgfpathlineto{\pgfqpoint{3.919924in}{0.800019in}}%
\pgfpathlineto{\pgfqpoint{3.954966in}{0.794684in}}%
\pgfpathlineto{\pgfqpoint{3.990009in}{0.789337in}}%
\pgfpathlineto{\pgfqpoint{4.025051in}{0.783976in}}%
\pgfpathlineto{\pgfqpoint{4.040060in}{0.781679in}}%
\pgfpathlineto{\pgfqpoint{4.060094in}{0.778676in}}%
\pgfpathlineto{\pgfqpoint{4.095136in}{0.773419in}}%
\pgfpathlineto{\pgfqpoint{4.130179in}{0.768150in}}%
\pgfpathlineto{\pgfqpoint{4.165221in}{0.762867in}}%
\pgfpathlineto{\pgfqpoint{4.200264in}{0.757572in}}%
\pgfpathlineto{\pgfqpoint{4.214498in}{0.755420in}}%
\pgfpathlineto{\pgfqpoint{4.235306in}{0.752339in}}%
\pgfpathlineto{\pgfqpoint{4.270349in}{0.747146in}}%
\pgfpathlineto{\pgfqpoint{4.305391in}{0.741939in}}%
\pgfpathlineto{\pgfqpoint{4.340434in}{0.736720in}}%
\pgfpathlineto{\pgfqpoint{4.375477in}{0.731487in}}%
\pgfpathlineto{\pgfqpoint{4.391052in}{0.729160in}}%
\pgfpathlineto{\pgfqpoint{4.410519in}{0.726312in}}%
\pgfpathlineto{\pgfqpoint{4.445562in}{0.721180in}}%
\pgfpathlineto{\pgfqpoint{4.480604in}{0.716036in}}%
\pgfpathlineto{\pgfqpoint{4.515647in}{0.710878in}}%
\pgfpathlineto{\pgfqpoint{4.550689in}{0.705706in}}%
\pgfpathlineto{\pgfqpoint{4.569681in}{0.702901in}}%
\pgfpathlineto{\pgfqpoint{4.585732in}{0.700579in}}%
\pgfpathlineto{\pgfqpoint{4.620774in}{0.695507in}}%
\pgfpathlineto{\pgfqpoint{4.655817in}{0.690423in}}%
\pgfpathlineto{\pgfqpoint{4.690859in}{0.685324in}}%
\pgfpathlineto{\pgfqpoint{4.725902in}{0.680213in}}%
\pgfpathlineto{\pgfqpoint{4.750346in}{0.676641in}}%
\pgfpathlineto{\pgfqpoint{4.760944in}{0.675125in}}%
\pgfpathlineto{\pgfqpoint{4.795987in}{0.670112in}}%
\pgfpathlineto{\pgfqpoint{4.831030in}{0.665085in}}%
\pgfpathlineto{\pgfqpoint{4.866072in}{0.660046in}}%
\pgfpathlineto{\pgfqpoint{4.901115in}{0.654992in}}%
\pgfpathlineto{\pgfqpoint{4.933007in}{0.650382in}}%
\pgfpathlineto{\pgfqpoint{4.936157in}{0.649936in}}%
\pgfpathlineto{\pgfqpoint{4.971200in}{0.644980in}}%
\pgfpathlineto{\pgfqpoint{5.006242in}{0.640010in}}%
\pgfpathlineto{\pgfqpoint{5.041285in}{0.635027in}}%
\pgfpathlineto{\pgfqpoint{5.076327in}{0.630030in}}%
\pgfpathlineto{\pgfqpoint{5.111370in}{0.625020in}}%
\pgfpathlineto{\pgfqpoint{5.117649in}{0.624122in}}%
\pgfpathlineto{\pgfqpoint{5.146412in}{0.620097in}}%
\pgfpathlineto{\pgfqpoint{5.181455in}{0.615183in}}%
\pgfpathlineto{\pgfqpoint{5.216497in}{0.610256in}}%
\pgfpathlineto{\pgfqpoint{5.251540in}{0.605315in}}%
\pgfpathlineto{\pgfqpoint{5.286583in}{0.600360in}}%
\pgfpathlineto{\pgfqpoint{5.304228in}{0.597863in}}%
\pgfpathlineto{\pgfqpoint{5.321625in}{0.595453in}}%
\pgfpathlineto{\pgfqpoint{5.356668in}{0.590593in}}%
\pgfpathlineto{\pgfqpoint{5.391710in}{0.585721in}}%
\pgfpathlineto{\pgfqpoint{5.426753in}{0.580834in}}%
\pgfpathlineto{\pgfqpoint{5.461795in}{0.575934in}}%
\pgfpathlineto{\pgfqpoint{5.492688in}{0.571603in}}%
\pgfusepath{stroke}%
\end{pgfscope}%
\begin{pgfscope}%
\pgfpathrectangle{\pgfqpoint{0.766095in}{0.571603in}}{\pgfqpoint{6.973465in}{5.225635in}}%
\pgfusepath{clip}%
\pgfsetbuttcap%
\pgfsetroundjoin%
\pgfsetlinewidth{1.505625pt}%
\definecolor{currentstroke}{rgb}{0.137770,0.537492,0.554906}%
\pgfsetstrokecolor{currentstroke}%
\pgfsetdash{}{0pt}%
\pgfpathmoveto{\pgfqpoint{0.766095in}{1.312705in}}%
\pgfpathlineto{\pgfqpoint{0.792506in}{1.306869in}}%
\pgfpathlineto{\pgfqpoint{0.801138in}{1.304998in}}%
\pgfpathlineto{\pgfqpoint{0.836180in}{1.297423in}}%
\pgfpathlineto{\pgfqpoint{0.871223in}{1.289848in}}%
\pgfpathlineto{\pgfqpoint{0.906265in}{1.282272in}}%
\pgfpathlineto{\pgfqpoint{0.913975in}{1.280609in}}%
\pgfpathlineto{\pgfqpoint{0.941308in}{1.274829in}}%
\pgfpathlineto{\pgfqpoint{0.976350in}{1.267423in}}%
\pgfpathlineto{\pgfqpoint{1.011393in}{1.260015in}}%
\pgfpathlineto{\pgfqpoint{1.038206in}{1.254350in}}%
\pgfpathlineto{\pgfqpoint{1.046435in}{1.252645in}}%
\pgfpathlineto{\pgfqpoint{1.081478in}{1.245401in}}%
\pgfpathlineto{\pgfqpoint{1.116520in}{1.238155in}}%
\pgfpathlineto{\pgfqpoint{1.151563in}{1.230907in}}%
\pgfpathlineto{\pgfqpoint{1.165203in}{1.228090in}}%
\pgfpathlineto{\pgfqpoint{1.186605in}{1.223756in}}%
\pgfpathlineto{\pgfqpoint{1.221648in}{1.216666in}}%
\pgfpathlineto{\pgfqpoint{1.256691in}{1.209573in}}%
\pgfpathlineto{\pgfqpoint{1.291733in}{1.202477in}}%
\pgfpathlineto{\pgfqpoint{1.294931in}{1.201831in}}%
\pgfpathlineto{\pgfqpoint{1.326776in}{1.195523in}}%
\pgfpathlineto{\pgfqpoint{1.361818in}{1.188580in}}%
\pgfpathlineto{\pgfqpoint{1.396861in}{1.181632in}}%
\pgfpathlineto{\pgfqpoint{1.427425in}{1.175571in}}%
\pgfpathlineto{\pgfqpoint{1.431903in}{1.174700in}}%
\pgfpathlineto{\pgfqpoint{1.466946in}{1.167902in}}%
\pgfpathlineto{\pgfqpoint{1.501988in}{1.161099in}}%
\pgfpathlineto{\pgfqpoint{1.537031in}{1.154291in}}%
\pgfpathlineto{\pgfqpoint{1.555301in}{1.150741in}}%
\pgfusepath{stroke}%
\end{pgfscope}%
\begin{pgfscope}%
\pgfpathrectangle{\pgfqpoint{0.766095in}{0.571603in}}{\pgfqpoint{6.973465in}{5.225635in}}%
\pgfusepath{clip}%
\pgfsetbuttcap%
\pgfsetroundjoin%
\pgfsetlinewidth{1.505625pt}%
\definecolor{currentstroke}{rgb}{0.137770,0.537492,0.554906}%
\pgfsetstrokecolor{currentstroke}%
\pgfsetdash{}{0pt}%
\pgfpathmoveto{\pgfqpoint{1.939431in}{1.078809in}}%
\pgfpathlineto{\pgfqpoint{1.957541in}{1.075488in}}%
\pgfpathlineto{\pgfqpoint{1.984546in}{1.070533in}}%
\pgfpathlineto{\pgfqpoint{1.992584in}{1.069088in}}%
\pgfpathlineto{\pgfqpoint{2.027626in}{1.062794in}}%
\pgfpathlineto{\pgfqpoint{2.062669in}{1.056492in}}%
\pgfpathlineto{\pgfqpoint{2.097711in}{1.050182in}}%
\pgfpathlineto{\pgfqpoint{2.130490in}{1.044274in}}%
\pgfpathlineto{\pgfqpoint{2.132754in}{1.043874in}}%
\pgfpathlineto{\pgfqpoint{2.167797in}{1.037693in}}%
\pgfpathlineto{\pgfqpoint{2.202839in}{1.031504in}}%
\pgfpathlineto{\pgfqpoint{2.237882in}{1.025307in}}%
\pgfpathlineto{\pgfqpoint{2.272924in}{1.019101in}}%
\pgfpathlineto{\pgfqpoint{2.279068in}{1.018014in}}%
\pgfpathlineto{\pgfqpoint{2.307967in}{1.013004in}}%
\pgfpathlineto{\pgfqpoint{2.343009in}{1.006924in}}%
\pgfpathlineto{\pgfqpoint{2.378052in}{1.000835in}}%
\pgfpathlineto{\pgfqpoint{2.413094in}{0.994736in}}%
\pgfpathlineto{\pgfqpoint{2.430231in}{0.991755in}}%
\pgfpathlineto{\pgfqpoint{2.448137in}{0.988701in}}%
\pgfpathlineto{\pgfqpoint{2.483179in}{0.982725in}}%
\pgfpathlineto{\pgfqpoint{2.518222in}{0.976740in}}%
\pgfpathlineto{\pgfqpoint{2.553264in}{0.970745in}}%
\pgfpathlineto{\pgfqpoint{2.583916in}{0.965495in}}%
\pgfpathlineto{\pgfqpoint{2.588307in}{0.964758in}}%
\pgfpathlineto{\pgfqpoint{2.623350in}{0.958883in}}%
\pgfpathlineto{\pgfqpoint{2.658392in}{0.952998in}}%
\pgfpathlineto{\pgfqpoint{2.693435in}{0.947104in}}%
\pgfpathlineto{\pgfqpoint{2.728477in}{0.941199in}}%
\pgfpathlineto{\pgfqpoint{2.740135in}{0.939236in}}%
\pgfpathlineto{\pgfqpoint{2.763520in}{0.935376in}}%
\pgfpathlineto{\pgfqpoint{2.798562in}{0.929588in}}%
\pgfpathlineto{\pgfqpoint{2.833605in}{0.923790in}}%
\pgfpathlineto{\pgfqpoint{2.868647in}{0.917982in}}%
\pgfpathlineto{\pgfqpoint{2.898805in}{0.912976in}}%
\pgfpathlineto{\pgfqpoint{2.903690in}{0.912182in}}%
\pgfpathlineto{\pgfqpoint{2.938732in}{0.906488in}}%
\pgfpathlineto{\pgfqpoint{2.973775in}{0.900783in}}%
\pgfpathlineto{\pgfqpoint{3.008818in}{0.895068in}}%
\pgfpathlineto{\pgfqpoint{3.043860in}{0.889342in}}%
\pgfpathlineto{\pgfqpoint{3.059925in}{0.886717in}}%
\pgfpathlineto{\pgfqpoint{3.078903in}{0.883678in}}%
\pgfpathlineto{\pgfqpoint{3.113945in}{0.878064in}}%
\pgfpathlineto{\pgfqpoint{3.148988in}{0.872439in}}%
\pgfpathlineto{\pgfqpoint{3.184030in}{0.866803in}}%
\pgfpathlineto{\pgfqpoint{3.219073in}{0.861156in}}%
\pgfpathlineto{\pgfqpoint{3.223411in}{0.860458in}}%
\pgfpathlineto{\pgfqpoint{3.254115in}{0.855613in}}%
\pgfpathlineto{\pgfqpoint{3.289158in}{0.850076in}}%
\pgfpathlineto{\pgfqpoint{3.324200in}{0.844528in}}%
\pgfpathlineto{\pgfqpoint{3.359243in}{0.838968in}}%
\pgfpathlineto{\pgfqpoint{3.389258in}{0.834198in}}%
\pgfpathlineto{\pgfqpoint{3.394285in}{0.833415in}}%
\pgfpathlineto{\pgfqpoint{3.429328in}{0.827963in}}%
\pgfpathlineto{\pgfqpoint{3.464371in}{0.822499in}}%
\pgfpathlineto{\pgfqpoint{3.499413in}{0.817024in}}%
\pgfpathlineto{\pgfqpoint{3.534456in}{0.811537in}}%
\pgfpathlineto{\pgfqpoint{3.557416in}{0.807939in}}%
\pgfpathlineto{\pgfqpoint{3.569498in}{0.806083in}}%
\pgfpathlineto{\pgfqpoint{3.604541in}{0.800702in}}%
\pgfpathlineto{\pgfqpoint{3.639583in}{0.795309in}}%
\pgfpathlineto{\pgfqpoint{3.674626in}{0.789905in}}%
\pgfpathlineto{\pgfqpoint{3.709668in}{0.784488in}}%
\pgfpathlineto{\pgfqpoint{3.727831in}{0.781679in}}%
\pgfpathlineto{\pgfqpoint{3.744711in}{0.779121in}}%
\pgfpathlineto{\pgfqpoint{3.779753in}{0.773809in}}%
\pgfpathlineto{\pgfqpoint{3.814796in}{0.768484in}}%
\pgfpathlineto{\pgfqpoint{3.849838in}{0.763148in}}%
\pgfpathlineto{\pgfqpoint{3.884881in}{0.757799in}}%
\pgfpathlineto{\pgfqpoint{3.900462in}{0.755420in}}%
\pgfpathlineto{\pgfqpoint{3.919924in}{0.752508in}}%
\pgfpathlineto{\pgfqpoint{3.954966in}{0.747262in}}%
\pgfpathlineto{\pgfqpoint{3.990009in}{0.742004in}}%
\pgfpathlineto{\pgfqpoint{4.025051in}{0.736733in}}%
\pgfpathlineto{\pgfqpoint{4.060094in}{0.731450in}}%
\pgfpathlineto{\pgfqpoint{4.075273in}{0.729160in}}%
\pgfpathlineto{\pgfqpoint{4.095136in}{0.726225in}}%
\pgfpathlineto{\pgfqpoint{4.130179in}{0.721043in}}%
\pgfpathlineto{\pgfqpoint{4.165221in}{0.715848in}}%
\pgfpathlineto{\pgfqpoint{4.200264in}{0.710641in}}%
\pgfpathlineto{\pgfqpoint{4.235306in}{0.705422in}}%
\pgfpathlineto{\pgfqpoint{4.252220in}{0.702901in}}%
\pgfpathlineto{\pgfqpoint{4.270349in}{0.700254in}}%
\pgfpathlineto{\pgfqpoint{4.305391in}{0.695134in}}%
\pgfpathlineto{\pgfqpoint{4.340434in}{0.690001in}}%
\pgfpathlineto{\pgfqpoint{4.375477in}{0.684856in}}%
\pgfpathlineto{\pgfqpoint{4.410519in}{0.679698in}}%
\pgfpathlineto{\pgfqpoint{4.431263in}{0.676641in}}%
\pgfpathlineto{\pgfqpoint{4.445562in}{0.674577in}}%
\pgfpathlineto{\pgfqpoint{4.480604in}{0.669518in}}%
\pgfpathlineto{\pgfqpoint{4.515647in}{0.664445in}}%
\pgfpathlineto{\pgfqpoint{4.550689in}{0.659360in}}%
\pgfpathlineto{\pgfqpoint{4.585732in}{0.654262in}}%
\pgfpathlineto{\pgfqpoint{4.612359in}{0.650382in}}%
\pgfpathlineto{\pgfqpoint{4.620774in}{0.649181in}}%
\pgfpathlineto{\pgfqpoint{4.655817in}{0.644180in}}%
\pgfpathlineto{\pgfqpoint{4.690859in}{0.639166in}}%
\pgfpathlineto{\pgfqpoint{4.725902in}{0.634139in}}%
\pgfpathlineto{\pgfqpoint{4.760944in}{0.629100in}}%
\pgfpathlineto{\pgfqpoint{4.795466in}{0.624122in}}%
\pgfpathlineto{\pgfqpoint{4.795987in}{0.624049in}}%
\pgfpathlineto{\pgfqpoint{4.831030in}{0.619105in}}%
\pgfpathlineto{\pgfqpoint{4.866072in}{0.614148in}}%
\pgfpathlineto{\pgfqpoint{4.901115in}{0.609179in}}%
\pgfpathlineto{\pgfqpoint{4.936157in}{0.604196in}}%
\pgfpathlineto{\pgfqpoint{4.971200in}{0.599201in}}%
\pgfpathlineto{\pgfqpoint{4.980582in}{0.597863in}}%
\pgfpathlineto{\pgfqpoint{5.006242in}{0.594280in}}%
\pgfpathlineto{\pgfqpoint{5.041285in}{0.589379in}}%
\pgfpathlineto{\pgfqpoint{5.076327in}{0.584466in}}%
\pgfpathlineto{\pgfqpoint{5.111370in}{0.579539in}}%
\pgfpathlineto{\pgfqpoint{5.146412in}{0.574599in}}%
\pgfpathlineto{\pgfqpoint{5.167634in}{0.571603in}}%
\pgfusepath{stroke}%
\end{pgfscope}%
\begin{pgfscope}%
\pgfpathrectangle{\pgfqpoint{0.766095in}{0.571603in}}{\pgfqpoint{6.973465in}{5.225635in}}%
\pgfusepath{clip}%
\pgfsetbuttcap%
\pgfsetroundjoin%
\pgfsetlinewidth{1.505625pt}%
\definecolor{currentstroke}{rgb}{0.127568,0.566949,0.550556}%
\pgfsetstrokecolor{currentstroke}%
\pgfsetdash{}{0pt}%
\pgfpathmoveto{\pgfqpoint{0.766095in}{1.258097in}}%
\pgfpathlineto{\pgfqpoint{0.783424in}{1.254350in}}%
\pgfpathlineto{\pgfqpoint{0.801138in}{1.250592in}}%
\pgfpathlineto{\pgfqpoint{0.836180in}{1.243172in}}%
\pgfpathlineto{\pgfqpoint{0.871223in}{1.235751in}}%
\pgfpathlineto{\pgfqpoint{0.906265in}{1.228329in}}%
\pgfpathlineto{\pgfqpoint{0.907399in}{1.228090in}}%
\pgfpathlineto{\pgfqpoint{0.941308in}{1.221068in}}%
\pgfpathlineto{\pgfqpoint{0.976350in}{1.213811in}}%
\pgfpathlineto{\pgfqpoint{1.011393in}{1.206553in}}%
\pgfpathlineto{\pgfqpoint{1.034215in}{1.201831in}}%
\pgfpathlineto{\pgfqpoint{1.046435in}{1.199350in}}%
\pgfpathlineto{\pgfqpoint{1.081478in}{1.192251in}}%
\pgfpathlineto{\pgfqpoint{1.116520in}{1.185150in}}%
\pgfpathlineto{\pgfqpoint{1.151563in}{1.178047in}}%
\pgfpathlineto{\pgfqpoint{1.163797in}{1.175571in}}%
\pgfpathlineto{\pgfqpoint{1.186605in}{1.171044in}}%
\pgfpathlineto{\pgfqpoint{1.221648in}{1.164095in}}%
\pgfpathlineto{\pgfqpoint{1.256691in}{1.157142in}}%
\pgfpathlineto{\pgfqpoint{1.291733in}{1.150186in}}%
\pgfpathlineto{\pgfqpoint{1.296148in}{1.149312in}}%
\pgfpathlineto{\pgfqpoint{1.326776in}{1.143363in}}%
\pgfpathlineto{\pgfqpoint{1.361818in}{1.136556in}}%
\pgfpathlineto{\pgfqpoint{1.396861in}{1.129745in}}%
\pgfpathlineto{\pgfqpoint{1.431273in}{1.123052in}}%
\pgfpathlineto{\pgfqpoint{1.431903in}{1.122932in}}%
\pgfpathlineto{\pgfqpoint{1.466946in}{1.116265in}}%
\pgfpathlineto{\pgfqpoint{1.501988in}{1.109594in}}%
\pgfpathlineto{\pgfqpoint{1.537031in}{1.102918in}}%
\pgfpathlineto{\pgfqpoint{1.569167in}{1.096793in}}%
\pgfpathlineto{\pgfqpoint{1.572073in}{1.096249in}}%
\pgfpathlineto{\pgfqpoint{1.607116in}{1.089713in}}%
\pgfpathlineto{\pgfqpoint{1.642158in}{1.083172in}}%
\pgfpathlineto{\pgfqpoint{1.677201in}{1.076625in}}%
\pgfpathlineto{\pgfqpoint{1.709787in}{1.070533in}}%
\pgfpathlineto{\pgfqpoint{1.712244in}{1.070083in}}%
\pgfpathlineto{\pgfqpoint{1.747286in}{1.063672in}}%
\pgfpathlineto{\pgfqpoint{1.782329in}{1.057255in}}%
\pgfpathlineto{\pgfqpoint{1.817371in}{1.050832in}}%
\pgfpathlineto{\pgfqpoint{1.852414in}{1.044402in}}%
\pgfpathlineto{\pgfqpoint{1.853114in}{1.044274in}}%
\pgfpathlineto{\pgfqpoint{1.887456in}{1.038108in}}%
\pgfpathlineto{\pgfqpoint{1.922499in}{1.031810in}}%
\pgfpathlineto{\pgfqpoint{1.957541in}{1.025506in}}%
\pgfpathlineto{\pgfqpoint{1.992584in}{1.019194in}}%
\pgfpathlineto{\pgfqpoint{1.999141in}{1.018014in}}%
\pgfpathlineto{\pgfqpoint{2.027626in}{1.012991in}}%
\pgfpathlineto{\pgfqpoint{2.062669in}{1.006807in}}%
\pgfpathlineto{\pgfqpoint{2.097711in}{1.000616in}}%
\pgfpathlineto{\pgfqpoint{2.132754in}{0.994417in}}%
\pgfpathlineto{\pgfqpoint{2.147815in}{0.991755in}}%
\pgfpathlineto{\pgfqpoint{2.167797in}{0.988291in}}%
\pgfpathlineto{\pgfqpoint{2.202839in}{0.982217in}}%
\pgfpathlineto{\pgfqpoint{2.237882in}{0.976136in}}%
\pgfpathlineto{\pgfqpoint{2.272924in}{0.970046in}}%
\pgfpathlineto{\pgfqpoint{2.299095in}{0.965495in}}%
\pgfpathlineto{\pgfqpoint{2.307967in}{0.963983in}}%
\pgfpathlineto{\pgfqpoint{2.343009in}{0.958015in}}%
\pgfpathlineto{\pgfqpoint{2.378052in}{0.952038in}}%
\pgfpathlineto{\pgfqpoint{2.413094in}{0.946053in}}%
\pgfpathlineto{\pgfqpoint{2.448137in}{0.940059in}}%
\pgfpathlineto{\pgfqpoint{2.452956in}{0.939236in}}%
\pgfpathlineto{\pgfqpoint{2.483179in}{0.934174in}}%
\pgfpathlineto{\pgfqpoint{2.518222in}{0.928299in}}%
\pgfpathlineto{\pgfqpoint{2.553264in}{0.922415in}}%
\pgfpathlineto{\pgfqpoint{2.588307in}{0.916521in}}%
\pgfpathlineto{\pgfqpoint{2.609379in}{0.912976in}}%
\pgfpathlineto{\pgfqpoint{2.623350in}{0.910672in}}%
\pgfpathlineto{\pgfqpoint{2.658392in}{0.904895in}}%
\pgfpathlineto{\pgfqpoint{2.693435in}{0.899108in}}%
\pgfpathlineto{\pgfqpoint{2.728477in}{0.893312in}}%
\pgfpathlineto{\pgfqpoint{2.763520in}{0.887506in}}%
\pgfpathlineto{\pgfqpoint{2.768286in}{0.886717in}}%
\pgfpathlineto{\pgfqpoint{2.798562in}{0.881805in}}%
\pgfpathlineto{\pgfqpoint{2.833605in}{0.876112in}}%
\pgfpathlineto{\pgfqpoint{2.868647in}{0.870410in}}%
\pgfpathlineto{\pgfqpoint{2.903690in}{0.864697in}}%
\pgfpathlineto{\pgfqpoint{2.929673in}{0.860458in}}%
\pgfpathlineto{\pgfqpoint{2.938732in}{0.859008in}}%
\pgfpathlineto{\pgfqpoint{2.973775in}{0.853407in}}%
\pgfpathlineto{\pgfqpoint{3.008818in}{0.847795in}}%
\pgfpathlineto{\pgfqpoint{3.043860in}{0.842173in}}%
\pgfpathlineto{\pgfqpoint{3.078903in}{0.836541in}}%
\pgfpathlineto{\pgfqpoint{3.093479in}{0.834198in}}%
\pgfpathlineto{\pgfqpoint{3.113945in}{0.830973in}}%
\pgfpathlineto{\pgfqpoint{3.148988in}{0.825450in}}%
\pgfpathlineto{\pgfqpoint{3.184030in}{0.819916in}}%
\pgfpathlineto{\pgfqpoint{3.219073in}{0.814371in}}%
\pgfpathlineto{\pgfqpoint{3.254115in}{0.808815in}}%
\pgfpathlineto{\pgfqpoint{3.259648in}{0.807939in}}%
\pgfpathlineto{\pgfqpoint{3.289158in}{0.803356in}}%
\pgfpathlineto{\pgfqpoint{3.324200in}{0.797907in}}%
\pgfpathlineto{\pgfqpoint{3.359243in}{0.792448in}}%
\pgfpathlineto{\pgfqpoint{3.394285in}{0.786977in}}%
\pgfpathlineto{\pgfqpoint{3.428153in}{0.781679in}}%
\pgfpathlineto{\pgfqpoint{3.429328in}{0.781499in}}%
\pgfpathlineto{\pgfqpoint{3.464371in}{0.776133in}}%
\pgfpathlineto{\pgfqpoint{3.499413in}{0.770755in}}%
\pgfpathlineto{\pgfqpoint{3.534456in}{0.765367in}}%
\pgfpathlineto{\pgfqpoint{3.569498in}{0.759967in}}%
\pgfpathlineto{\pgfqpoint{3.598963in}{0.755420in}}%
\pgfpathlineto{\pgfqpoint{3.604541in}{0.754576in}}%
\pgfpathlineto{\pgfqpoint{3.639583in}{0.749279in}}%
\pgfpathlineto{\pgfqpoint{3.674626in}{0.743971in}}%
\pgfpathlineto{\pgfqpoint{3.709668in}{0.738651in}}%
\pgfpathlineto{\pgfqpoint{3.744711in}{0.733320in}}%
\pgfpathlineto{\pgfqpoint{3.772013in}{0.729160in}}%
\pgfpathlineto{\pgfqpoint{3.779753in}{0.728004in}}%
\pgfpathlineto{\pgfqpoint{3.814796in}{0.722774in}}%
\pgfpathlineto{\pgfqpoint{3.849838in}{0.717533in}}%
\pgfpathlineto{\pgfqpoint{3.884881in}{0.712280in}}%
\pgfpathlineto{\pgfqpoint{3.919924in}{0.707015in}}%
\pgfpathlineto{\pgfqpoint{3.947263in}{0.702901in}}%
\pgfpathlineto{\pgfqpoint{3.954966in}{0.701765in}}%
\pgfpathlineto{\pgfqpoint{3.990009in}{0.696599in}}%
\pgfpathlineto{\pgfqpoint{4.025051in}{0.691422in}}%
\pgfpathlineto{\pgfqpoint{4.060094in}{0.686233in}}%
\pgfpathlineto{\pgfqpoint{4.095136in}{0.681032in}}%
\pgfpathlineto{\pgfqpoint{4.124668in}{0.676641in}}%
\pgfpathlineto{\pgfqpoint{4.130179in}{0.675838in}}%
\pgfpathlineto{\pgfqpoint{4.165221in}{0.670735in}}%
\pgfpathlineto{\pgfqpoint{4.200264in}{0.665621in}}%
\pgfpathlineto{\pgfqpoint{4.235306in}{0.660494in}}%
\pgfpathlineto{\pgfqpoint{4.270349in}{0.655355in}}%
\pgfpathlineto{\pgfqpoint{4.304185in}{0.650382in}}%
\pgfpathlineto{\pgfqpoint{4.305391in}{0.650208in}}%
\pgfpathlineto{\pgfqpoint{4.322260in}{0.647781in}}%
\pgfusepath{stroke}%
\end{pgfscope}%
\begin{pgfscope}%
\pgfpathrectangle{\pgfqpoint{0.766095in}{0.571603in}}{\pgfqpoint{6.973465in}{5.225635in}}%
\pgfusepath{clip}%
\pgfsetbuttcap%
\pgfsetroundjoin%
\pgfsetlinewidth{1.505625pt}%
\definecolor{currentstroke}{rgb}{0.127568,0.566949,0.550556}%
\pgfsetstrokecolor{currentstroke}%
\pgfsetdash{}{0pt}%
\pgfpathmoveto{\pgfqpoint{4.709077in}{0.592281in}}%
\pgfpathlineto{\pgfqpoint{4.725902in}{0.589910in}}%
\pgfpathlineto{\pgfqpoint{4.760944in}{0.584960in}}%
\pgfpathlineto{\pgfqpoint{4.795987in}{0.579997in}}%
\pgfpathlineto{\pgfqpoint{4.831030in}{0.575022in}}%
\pgfpathlineto{\pgfqpoint{4.855071in}{0.571603in}}%
\pgfusepath{stroke}%
\end{pgfscope}%
\begin{pgfscope}%
\pgfpathrectangle{\pgfqpoint{0.766095in}{0.571603in}}{\pgfqpoint{6.973465in}{5.225635in}}%
\pgfusepath{clip}%
\pgfsetbuttcap%
\pgfsetroundjoin%
\pgfsetlinewidth{1.505625pt}%
\definecolor{currentstroke}{rgb}{0.121148,0.592739,0.544641}%
\pgfsetstrokecolor{currentstroke}%
\pgfsetdash{}{0pt}%
\pgfpathmoveto{\pgfqpoint{0.766095in}{1.205962in}}%
\pgfpathlineto{\pgfqpoint{0.785612in}{1.201831in}}%
\pgfpathlineto{\pgfqpoint{0.801138in}{1.198606in}}%
\pgfpathlineto{\pgfqpoint{0.836180in}{1.191341in}}%
\pgfpathlineto{\pgfqpoint{0.871223in}{1.184077in}}%
\pgfpathlineto{\pgfqpoint{0.906265in}{1.176812in}}%
\pgfpathlineto{\pgfqpoint{0.912266in}{1.175571in}}%
\pgfpathlineto{\pgfqpoint{0.941308in}{1.169680in}}%
\pgfpathlineto{\pgfqpoint{0.976350in}{1.162574in}}%
\pgfpathlineto{\pgfqpoint{1.011393in}{1.155467in}}%
\pgfpathlineto{\pgfqpoint{1.041753in}{1.149312in}}%
\pgfpathlineto{\pgfqpoint{1.046435in}{1.148380in}}%
\pgfpathlineto{\pgfqpoint{1.081478in}{1.141428in}}%
\pgfpathlineto{\pgfqpoint{1.116520in}{1.134473in}}%
\pgfpathlineto{\pgfqpoint{1.151563in}{1.127517in}}%
\pgfpathlineto{\pgfqpoint{1.174071in}{1.123052in}}%
\pgfpathlineto{\pgfqpoint{1.186605in}{1.120613in}}%
\pgfpathlineto{\pgfqpoint{1.221648in}{1.113806in}}%
\pgfpathlineto{\pgfqpoint{1.256691in}{1.106995in}}%
\pgfpathlineto{\pgfqpoint{1.291733in}{1.100182in}}%
\pgfpathlineto{\pgfqpoint{1.309184in}{1.096793in}}%
\pgfpathlineto{\pgfqpoint{1.326776in}{1.093441in}}%
\pgfpathlineto{\pgfqpoint{1.361818in}{1.086772in}}%
\pgfpathlineto{\pgfqpoint{1.396861in}{1.080099in}}%
\pgfpathlineto{\pgfqpoint{1.431903in}{1.073422in}}%
\pgfpathlineto{\pgfqpoint{1.447084in}{1.070533in}}%
\pgfpathlineto{\pgfqpoint{1.466946in}{1.066825in}}%
\pgfpathlineto{\pgfqpoint{1.501988in}{1.060288in}}%
\pgfpathlineto{\pgfqpoint{1.537031in}{1.053747in}}%
\pgfpathlineto{\pgfqpoint{1.572073in}{1.047201in}}%
\pgfpathlineto{\pgfqpoint{1.587756in}{1.044274in}}%
\pgfpathlineto{\pgfqpoint{1.607116in}{1.040730in}}%
\pgfpathlineto{\pgfqpoint{1.642158in}{1.034319in}}%
\pgfpathlineto{\pgfqpoint{1.677201in}{1.027903in}}%
\pgfpathlineto{\pgfqpoint{1.712244in}{1.021481in}}%
\pgfpathlineto{\pgfqpoint{1.731174in}{1.018014in}}%
\pgfpathlineto{\pgfqpoint{1.747286in}{1.015120in}}%
\pgfpathlineto{\pgfqpoint{1.782329in}{1.008830in}}%
\pgfpathlineto{\pgfqpoint{1.817371in}{1.002534in}}%
\pgfpathlineto{\pgfqpoint{1.852414in}{0.996232in}}%
\pgfpathlineto{\pgfqpoint{1.877308in}{0.991755in}}%
\pgfpathlineto{\pgfqpoint{1.887456in}{0.989965in}}%
\pgfpathlineto{\pgfqpoint{1.922499in}{0.983790in}}%
\pgfpathlineto{\pgfqpoint{1.957541in}{0.977610in}}%
\pgfpathlineto{\pgfqpoint{1.992584in}{0.971423in}}%
\pgfpathlineto{\pgfqpoint{2.026119in}{0.965495in}}%
\pgfpathlineto{\pgfqpoint{2.027626in}{0.965234in}}%
\pgfpathlineto{\pgfqpoint{2.062669in}{0.959171in}}%
\pgfpathlineto{\pgfqpoint{2.097711in}{0.953101in}}%
\pgfpathlineto{\pgfqpoint{2.132754in}{0.947024in}}%
\pgfpathlineto{\pgfqpoint{2.167797in}{0.940939in}}%
\pgfpathlineto{\pgfqpoint{2.177616in}{0.939236in}}%
\pgfpathlineto{\pgfqpoint{2.202839in}{0.934945in}}%
\pgfpathlineto{\pgfqpoint{2.237882in}{0.928981in}}%
\pgfpathlineto{\pgfqpoint{2.272924in}{0.923010in}}%
\pgfpathlineto{\pgfqpoint{2.307967in}{0.917031in}}%
\pgfpathlineto{\pgfqpoint{2.331718in}{0.912976in}}%
\pgfpathlineto{\pgfqpoint{2.343009in}{0.911086in}}%
\pgfpathlineto{\pgfqpoint{2.378052in}{0.905225in}}%
\pgfpathlineto{\pgfqpoint{2.413094in}{0.899356in}}%
\pgfpathlineto{\pgfqpoint{2.448137in}{0.893478in}}%
\pgfpathlineto{\pgfqpoint{2.483179in}{0.887591in}}%
\pgfpathlineto{\pgfqpoint{2.488390in}{0.886717in}}%
\pgfpathlineto{\pgfqpoint{2.518222in}{0.881809in}}%
\pgfpathlineto{\pgfqpoint{2.553264in}{0.876038in}}%
\pgfpathlineto{\pgfqpoint{2.588307in}{0.870258in}}%
\pgfpathlineto{\pgfqpoint{2.623350in}{0.864469in}}%
\pgfpathlineto{\pgfqpoint{2.647616in}{0.860458in}}%
\pgfpathlineto{\pgfqpoint{2.658392in}{0.858711in}}%
\pgfpathlineto{\pgfqpoint{2.693435in}{0.853034in}}%
\pgfpathlineto{\pgfqpoint{2.728477in}{0.847349in}}%
\pgfpathlineto{\pgfqpoint{2.763520in}{0.841654in}}%
\pgfpathlineto{\pgfqpoint{2.798562in}{0.835950in}}%
\pgfpathlineto{\pgfqpoint{2.809331in}{0.834198in}}%
\pgfpathlineto{\pgfqpoint{2.833605in}{0.830326in}}%
\pgfpathlineto{\pgfqpoint{2.868647in}{0.824732in}}%
\pgfpathlineto{\pgfqpoint{2.903690in}{0.819128in}}%
\pgfpathlineto{\pgfqpoint{2.938732in}{0.813515in}}%
\pgfpathlineto{\pgfqpoint{2.973481in}{0.807939in}}%
\pgfpathlineto{\pgfqpoint{2.973775in}{0.807892in}}%
\pgfpathlineto{\pgfqpoint{3.008818in}{0.802387in}}%
\pgfpathlineto{\pgfqpoint{3.043860in}{0.796871in}}%
\pgfpathlineto{\pgfqpoint{3.078903in}{0.791346in}}%
\pgfpathlineto{\pgfqpoint{3.113945in}{0.785810in}}%
\pgfpathlineto{\pgfqpoint{3.140071in}{0.781679in}}%
\pgfpathlineto{\pgfqpoint{3.148988in}{0.780297in}}%
\pgfpathlineto{\pgfqpoint{3.184030in}{0.774867in}}%
\pgfpathlineto{\pgfqpoint{3.219073in}{0.769427in}}%
\pgfpathlineto{\pgfqpoint{3.254115in}{0.763976in}}%
\pgfpathlineto{\pgfqpoint{3.289158in}{0.758515in}}%
\pgfpathlineto{\pgfqpoint{3.309012in}{0.755420in}}%
\pgfpathlineto{\pgfqpoint{3.324200in}{0.753098in}}%
\pgfpathlineto{\pgfqpoint{3.359243in}{0.747740in}}%
\pgfpathlineto{\pgfqpoint{3.394285in}{0.742373in}}%
\pgfpathlineto{\pgfqpoint{3.429328in}{0.736994in}}%
\pgfpathlineto{\pgfqpoint{3.464371in}{0.731605in}}%
\pgfpathlineto{\pgfqpoint{3.480265in}{0.729160in}}%
\pgfpathlineto{\pgfqpoint{3.499413in}{0.726272in}}%
\pgfpathlineto{\pgfqpoint{3.534456in}{0.720985in}}%
\pgfpathlineto{\pgfqpoint{3.569498in}{0.715687in}}%
\pgfpathlineto{\pgfqpoint{3.604541in}{0.710378in}}%
\pgfpathlineto{\pgfqpoint{3.636297in}{0.705556in}}%
\pgfusepath{stroke}%
\end{pgfscope}%
\begin{pgfscope}%
\pgfpathrectangle{\pgfqpoint{0.766095in}{0.571603in}}{\pgfqpoint{6.973465in}{5.225635in}}%
\pgfusepath{clip}%
\pgfsetbuttcap%
\pgfsetroundjoin%
\pgfsetlinewidth{1.505625pt}%
\definecolor{currentstroke}{rgb}{0.121148,0.592739,0.544641}%
\pgfsetstrokecolor{currentstroke}%
\pgfsetdash{}{0pt}%
\pgfpathmoveto{\pgfqpoint{4.022838in}{0.648149in}}%
\pgfpathlineto{\pgfqpoint{4.025051in}{0.647828in}}%
\pgfpathlineto{\pgfqpoint{4.060094in}{0.642736in}}%
\pgfpathlineto{\pgfqpoint{4.095136in}{0.637632in}}%
\pgfpathlineto{\pgfqpoint{4.130179in}{0.632517in}}%
\pgfpathlineto{\pgfqpoint{4.165221in}{0.627390in}}%
\pgfpathlineto{\pgfqpoint{4.187532in}{0.624122in}}%
\pgfpathlineto{\pgfqpoint{4.200264in}{0.622295in}}%
\pgfpathlineto{\pgfqpoint{4.235306in}{0.617263in}}%
\pgfpathlineto{\pgfqpoint{4.270349in}{0.612220in}}%
\pgfpathlineto{\pgfqpoint{4.305391in}{0.607166in}}%
\pgfpathlineto{\pgfqpoint{4.340434in}{0.602100in}}%
\pgfpathlineto{\pgfqpoint{4.369686in}{0.597863in}}%
\pgfpathlineto{\pgfqpoint{4.375477in}{0.597041in}}%
\pgfpathlineto{\pgfqpoint{4.410519in}{0.592069in}}%
\pgfpathlineto{\pgfqpoint{4.445562in}{0.587085in}}%
\pgfpathlineto{\pgfqpoint{4.480604in}{0.582089in}}%
\pgfpathlineto{\pgfqpoint{4.515647in}{0.577082in}}%
\pgfpathlineto{\pgfqpoint{4.550689in}{0.572062in}}%
\pgfpathlineto{\pgfqpoint{4.553896in}{0.571603in}}%
\pgfusepath{stroke}%
\end{pgfscope}%
\begin{pgfscope}%
\pgfpathrectangle{\pgfqpoint{0.766095in}{0.571603in}}{\pgfqpoint{6.973465in}{5.225635in}}%
\pgfusepath{clip}%
\pgfsetbuttcap%
\pgfsetroundjoin%
\pgfsetlinewidth{1.505625pt}%
\definecolor{currentstroke}{rgb}{0.119699,0.618490,0.536347}%
\pgfsetstrokecolor{currentstroke}%
\pgfsetdash{}{0pt}%
\pgfpathmoveto{\pgfqpoint{0.766095in}{1.156057in}}%
\pgfpathlineto{\pgfqpoint{0.798630in}{1.149312in}}%
\pgfpathlineto{\pgfqpoint{0.801138in}{1.148802in}}%
\pgfpathlineto{\pgfqpoint{0.836180in}{1.141694in}}%
\pgfpathlineto{\pgfqpoint{0.871223in}{1.134586in}}%
\pgfpathlineto{\pgfqpoint{0.889823in}{1.130813in}}%
\pgfusepath{stroke}%
\end{pgfscope}%
\begin{pgfscope}%
\pgfpathrectangle{\pgfqpoint{0.766095in}{0.571603in}}{\pgfqpoint{6.973465in}{5.225635in}}%
\pgfusepath{clip}%
\pgfsetbuttcap%
\pgfsetroundjoin%
\pgfsetlinewidth{1.505625pt}%
\definecolor{currentstroke}{rgb}{0.119699,0.618490,0.536347}%
\pgfsetstrokecolor{currentstroke}%
\pgfsetdash{}{0pt}%
\pgfpathmoveto{\pgfqpoint{1.273355in}{1.055731in}}%
\pgfpathlineto{\pgfqpoint{1.291733in}{1.052234in}}%
\pgfpathlineto{\pgfqpoint{1.326776in}{1.045561in}}%
\pgfpathlineto{\pgfqpoint{1.333547in}{1.044274in}}%
\pgfpathlineto{\pgfqpoint{1.361818in}{1.039002in}}%
\pgfpathlineto{\pgfqpoint{1.396861in}{1.032469in}}%
\pgfpathlineto{\pgfqpoint{1.431903in}{1.025932in}}%
\pgfpathlineto{\pgfqpoint{1.466946in}{1.019391in}}%
\pgfpathlineto{\pgfqpoint{1.474333in}{1.018014in}}%
\pgfpathlineto{\pgfqpoint{1.501988in}{1.012959in}}%
\pgfpathlineto{\pgfqpoint{1.537031in}{1.006553in}}%
\pgfpathlineto{\pgfqpoint{1.572073in}{1.000143in}}%
\pgfpathlineto{\pgfqpoint{1.607116in}{0.993727in}}%
\pgfpathlineto{\pgfqpoint{1.617907in}{0.991755in}}%
\pgfpathlineto{\pgfqpoint{1.642158in}{0.987405in}}%
\pgfpathlineto{\pgfqpoint{1.677201in}{0.981121in}}%
\pgfpathlineto{\pgfqpoint{1.712244in}{0.974832in}}%
\pgfpathlineto{\pgfqpoint{1.747286in}{0.968538in}}%
\pgfpathlineto{\pgfqpoint{1.764234in}{0.965495in}}%
\pgfpathlineto{\pgfqpoint{1.782329in}{0.962309in}}%
\pgfpathlineto{\pgfqpoint{1.817371in}{0.956142in}}%
\pgfpathlineto{\pgfqpoint{1.852414in}{0.949969in}}%
\pgfpathlineto{\pgfqpoint{1.887456in}{0.943790in}}%
\pgfpathlineto{\pgfqpoint{1.913278in}{0.939236in}}%
\pgfpathlineto{\pgfqpoint{1.922499in}{0.937641in}}%
\pgfpathlineto{\pgfqpoint{1.957541in}{0.931585in}}%
\pgfpathlineto{\pgfqpoint{1.992584in}{0.925524in}}%
\pgfpathlineto{\pgfqpoint{2.027626in}{0.919456in}}%
\pgfpathlineto{\pgfqpoint{2.062669in}{0.913381in}}%
\pgfpathlineto{\pgfqpoint{2.065006in}{0.912976in}}%
\pgfpathlineto{\pgfqpoint{2.097711in}{0.907424in}}%
\pgfpathlineto{\pgfqpoint{2.132754in}{0.901470in}}%
\pgfpathlineto{\pgfqpoint{2.167797in}{0.895509in}}%
\pgfpathlineto{\pgfqpoint{2.202839in}{0.889540in}}%
\pgfpathlineto{\pgfqpoint{2.219417in}{0.886717in}}%
\pgfpathlineto{\pgfqpoint{2.237882in}{0.883633in}}%
\pgfpathlineto{\pgfqpoint{2.272924in}{0.877781in}}%
\pgfpathlineto{\pgfqpoint{2.307967in}{0.871922in}}%
\pgfpathlineto{\pgfqpoint{2.343009in}{0.866056in}}%
\pgfpathlineto{\pgfqpoint{2.376407in}{0.860458in}}%
\pgfpathlineto{\pgfqpoint{2.378052in}{0.860187in}}%
\pgfpathlineto{\pgfqpoint{2.413094in}{0.854435in}}%
\pgfpathlineto{\pgfqpoint{2.448137in}{0.848674in}}%
\pgfpathlineto{\pgfqpoint{2.483179in}{0.842906in}}%
\pgfpathlineto{\pgfqpoint{2.518222in}{0.837129in}}%
\pgfpathlineto{\pgfqpoint{2.536004in}{0.834198in}}%
\pgfpathlineto{\pgfqpoint{2.553264in}{0.831407in}}%
\pgfpathlineto{\pgfqpoint{2.588307in}{0.825742in}}%
\pgfpathlineto{\pgfqpoint{2.623350in}{0.820069in}}%
\pgfpathlineto{\pgfqpoint{2.658392in}{0.814387in}}%
\pgfpathlineto{\pgfqpoint{2.693435in}{0.808695in}}%
\pgfpathlineto{\pgfqpoint{2.698100in}{0.807939in}}%
\pgfpathlineto{\pgfqpoint{2.728477in}{0.803105in}}%
\pgfpathlineto{\pgfqpoint{2.763520in}{0.797524in}}%
\pgfpathlineto{\pgfqpoint{2.798562in}{0.791933in}}%
\pgfpathlineto{\pgfqpoint{2.833605in}{0.786333in}}%
\pgfpathlineto{\pgfqpoint{2.862693in}{0.781679in}}%
\pgfpathlineto{\pgfqpoint{2.868647in}{0.780745in}}%
\pgfpathlineto{\pgfqpoint{2.903690in}{0.775252in}}%
\pgfpathlineto{\pgfqpoint{2.938732in}{0.769749in}}%
\pgfpathlineto{\pgfqpoint{2.973775in}{0.764237in}}%
\pgfpathlineto{\pgfqpoint{3.008818in}{0.758716in}}%
\pgfpathlineto{\pgfqpoint{3.029729in}{0.755420in}}%
\pgfpathlineto{\pgfqpoint{3.043860in}{0.753235in}}%
\pgfpathlineto{\pgfqpoint{3.078903in}{0.747818in}}%
\pgfpathlineto{\pgfqpoint{3.113945in}{0.742392in}}%
\pgfpathlineto{\pgfqpoint{3.148988in}{0.736956in}}%
\pgfpathlineto{\pgfqpoint{3.184030in}{0.731510in}}%
\pgfpathlineto{\pgfqpoint{3.199147in}{0.729160in}}%
\pgfpathlineto{\pgfqpoint{3.219073in}{0.726123in}}%
\pgfpathlineto{\pgfqpoint{3.254115in}{0.720780in}}%
\pgfpathlineto{\pgfqpoint{3.289158in}{0.715426in}}%
\pgfpathlineto{\pgfqpoint{3.324200in}{0.710063in}}%
\pgfpathlineto{\pgfqpoint{3.359243in}{0.704689in}}%
\pgfpathlineto{\pgfqpoint{3.370905in}{0.702901in}}%
\pgfpathlineto{\pgfqpoint{3.394285in}{0.699385in}}%
\pgfpathlineto{\pgfqpoint{3.429328in}{0.694112in}}%
\pgfpathlineto{\pgfqpoint{3.464371in}{0.688829in}}%
\pgfpathlineto{\pgfqpoint{3.499413in}{0.683535in}}%
\pgfpathlineto{\pgfqpoint{3.534456in}{0.678231in}}%
\pgfpathlineto{\pgfqpoint{3.544960in}{0.676641in}}%
\pgfpathlineto{\pgfqpoint{3.569498in}{0.673000in}}%
\pgfpathlineto{\pgfqpoint{3.604541in}{0.667794in}}%
\pgfpathlineto{\pgfqpoint{3.639583in}{0.662579in}}%
\pgfpathlineto{\pgfqpoint{3.674626in}{0.657352in}}%
\pgfpathlineto{\pgfqpoint{3.709668in}{0.652115in}}%
\pgfpathlineto{\pgfqpoint{3.721265in}{0.650382in}}%
\pgfpathlineto{\pgfqpoint{3.744711in}{0.646946in}}%
\pgfpathlineto{\pgfqpoint{3.779753in}{0.641806in}}%
\pgfpathlineto{\pgfqpoint{3.814796in}{0.636655in}}%
\pgfpathlineto{\pgfqpoint{3.849838in}{0.631494in}}%
\pgfpathlineto{\pgfqpoint{3.884881in}{0.626321in}}%
\pgfpathlineto{\pgfqpoint{3.899773in}{0.624122in}}%
\pgfpathlineto{\pgfqpoint{3.919924in}{0.621205in}}%
\pgfpathlineto{\pgfqpoint{3.954966in}{0.616128in}}%
\pgfpathlineto{\pgfqpoint{3.990009in}{0.611041in}}%
\pgfpathlineto{\pgfqpoint{4.025051in}{0.605942in}}%
\pgfpathlineto{\pgfqpoint{4.060094in}{0.600832in}}%
\pgfpathlineto{\pgfqpoint{4.080436in}{0.597863in}}%
\pgfpathlineto{\pgfqpoint{4.095136in}{0.595759in}}%
\pgfpathlineto{\pgfqpoint{4.130179in}{0.590744in}}%
\pgfpathlineto{\pgfqpoint{4.165221in}{0.585717in}}%
\pgfpathlineto{\pgfqpoint{4.200264in}{0.580679in}}%
\pgfpathlineto{\pgfqpoint{4.235306in}{0.575630in}}%
\pgfpathlineto{\pgfqpoint{4.263205in}{0.571603in}}%
\pgfusepath{stroke}%
\end{pgfscope}%
\begin{pgfscope}%
\pgfpathrectangle{\pgfqpoint{0.766095in}{0.571603in}}{\pgfqpoint{6.973465in}{5.225635in}}%
\pgfusepath{clip}%
\pgfsetbuttcap%
\pgfsetroundjoin%
\pgfsetlinewidth{1.505625pt}%
\definecolor{currentstroke}{rgb}{0.126326,0.644107,0.525311}%
\pgfsetstrokecolor{currentstroke}%
\pgfsetdash{}{0pt}%
\pgfpathmoveto{\pgfqpoint{0.766095in}{1.108161in}}%
\pgfpathlineto{\pgfqpoint{0.801138in}{1.101056in}}%
\pgfpathlineto{\pgfqpoint{0.822192in}{1.096793in}}%
\pgfpathlineto{\pgfqpoint{0.836180in}{1.094013in}}%
\pgfpathlineto{\pgfqpoint{0.871223in}{1.087062in}}%
\pgfpathlineto{\pgfqpoint{0.906265in}{1.080111in}}%
\pgfpathlineto{\pgfqpoint{0.941308in}{1.073160in}}%
\pgfpathlineto{\pgfqpoint{0.954573in}{1.070533in}}%
\pgfpathlineto{\pgfqpoint{0.976350in}{1.066301in}}%
\pgfpathlineto{\pgfqpoint{1.011393in}{1.059499in}}%
\pgfpathlineto{\pgfqpoint{1.046435in}{1.052696in}}%
\pgfpathlineto{\pgfqpoint{1.081478in}{1.045891in}}%
\pgfpathlineto{\pgfqpoint{1.089824in}{1.044274in}}%
\pgfpathlineto{\pgfqpoint{1.116520in}{1.039197in}}%
\pgfpathlineto{\pgfqpoint{1.151563in}{1.032536in}}%
\pgfpathlineto{\pgfqpoint{1.186605in}{1.025873in}}%
\pgfpathlineto{\pgfqpoint{1.221648in}{1.019208in}}%
\pgfpathlineto{\pgfqpoint{1.227939in}{1.018014in}}%
\pgfpathlineto{\pgfqpoint{1.256691in}{1.012659in}}%
\pgfpathlineto{\pgfqpoint{1.291733in}{1.006133in}}%
\pgfpathlineto{\pgfqpoint{1.326776in}{0.999604in}}%
\pgfpathlineto{\pgfqpoint{1.361818in}{0.993072in}}%
\pgfpathlineto{\pgfqpoint{1.368900in}{0.991755in}}%
\pgfpathlineto{\pgfqpoint{1.396861in}{0.986650in}}%
\pgfpathlineto{\pgfqpoint{1.431903in}{0.980253in}}%
\pgfpathlineto{\pgfqpoint{1.466946in}{0.973852in}}%
\pgfpathlineto{\pgfqpoint{1.501988in}{0.967447in}}%
\pgfpathlineto{\pgfqpoint{1.512684in}{0.965495in}}%
\pgfpathlineto{\pgfqpoint{1.537031in}{0.961135in}}%
\pgfpathlineto{\pgfqpoint{1.572073in}{0.954861in}}%
\pgfpathlineto{\pgfqpoint{1.607116in}{0.948583in}}%
\pgfpathlineto{\pgfqpoint{1.642158in}{0.942299in}}%
\pgfpathlineto{\pgfqpoint{1.659258in}{0.939236in}}%
\pgfpathlineto{\pgfqpoint{1.677201in}{0.936081in}}%
\pgfpathlineto{\pgfqpoint{1.712244in}{0.929925in}}%
\pgfpathlineto{\pgfqpoint{1.747286in}{0.923764in}}%
\pgfpathlineto{\pgfqpoint{1.782329in}{0.917597in}}%
\pgfpathlineto{\pgfqpoint{1.808583in}{0.912976in}}%
\pgfpathlineto{\pgfqpoint{1.817371in}{0.911459in}}%
\pgfpathlineto{\pgfqpoint{1.852414in}{0.905415in}}%
\pgfpathlineto{\pgfqpoint{1.887456in}{0.899366in}}%
\pgfpathlineto{\pgfqpoint{1.922499in}{0.893311in}}%
\pgfpathlineto{\pgfqpoint{1.957541in}{0.887250in}}%
\pgfpathlineto{\pgfqpoint{1.960630in}{0.886717in}}%
\pgfpathlineto{\pgfqpoint{1.992584in}{0.881303in}}%
\pgfpathlineto{\pgfqpoint{2.027626in}{0.875362in}}%
\pgfpathlineto{\pgfqpoint{2.062669in}{0.869415in}}%
\pgfpathlineto{\pgfqpoint{2.097711in}{0.863460in}}%
\pgfpathlineto{\pgfqpoint{2.115389in}{0.860458in}}%
\pgfpathlineto{\pgfqpoint{2.132754in}{0.857563in}}%
\pgfpathlineto{\pgfqpoint{2.167797in}{0.851726in}}%
\pgfpathlineto{\pgfqpoint{2.202839in}{0.845881in}}%
\pgfpathlineto{\pgfqpoint{2.237882in}{0.840030in}}%
\pgfpathlineto{\pgfqpoint{2.272760in}{0.834198in}}%
\pgfpathlineto{\pgfqpoint{2.272924in}{0.834171in}}%
\pgfpathlineto{\pgfqpoint{2.307967in}{0.828433in}}%
\pgfpathlineto{\pgfqpoint{2.343009in}{0.822688in}}%
\pgfpathlineto{\pgfqpoint{2.378052in}{0.816935in}}%
\pgfpathlineto{\pgfqpoint{2.413094in}{0.811174in}}%
\pgfpathlineto{\pgfqpoint{2.432777in}{0.807939in}}%
\pgfpathlineto{\pgfqpoint{2.448137in}{0.805461in}}%
\pgfpathlineto{\pgfqpoint{2.483179in}{0.799811in}}%
\pgfpathlineto{\pgfqpoint{2.518222in}{0.794154in}}%
\pgfpathlineto{\pgfqpoint{2.553264in}{0.788488in}}%
\pgfpathlineto{\pgfqpoint{2.588307in}{0.782814in}}%
\pgfpathlineto{\pgfqpoint{2.595325in}{0.781679in}}%
\pgfpathlineto{\pgfqpoint{2.623350in}{0.777232in}}%
\pgfpathlineto{\pgfqpoint{2.658392in}{0.771666in}}%
\pgfpathlineto{\pgfqpoint{2.693435in}{0.766092in}}%
\pgfpathlineto{\pgfqpoint{2.728477in}{0.760510in}}%
\pgfpathlineto{\pgfqpoint{2.760388in}{0.755420in}}%
\pgfpathlineto{\pgfqpoint{2.763520in}{0.754930in}}%
\pgfpathlineto{\pgfqpoint{2.798562in}{0.749453in}}%
\pgfpathlineto{\pgfqpoint{2.833605in}{0.743967in}}%
\pgfpathlineto{\pgfqpoint{2.868647in}{0.738473in}}%
\pgfpathlineto{\pgfqpoint{2.903690in}{0.732970in}}%
\pgfpathlineto{\pgfqpoint{2.927933in}{0.729160in}}%
\pgfpathlineto{\pgfqpoint{2.938732in}{0.727495in}}%
\pgfpathlineto{\pgfqpoint{2.955869in}{0.724855in}}%
\pgfusepath{stroke}%
\end{pgfscope}%
\begin{pgfscope}%
\pgfpathrectangle{\pgfqpoint{0.766095in}{0.571603in}}{\pgfqpoint{6.973465in}{5.225635in}}%
\pgfusepath{clip}%
\pgfsetbuttcap%
\pgfsetroundjoin%
\pgfsetlinewidth{1.505625pt}%
\definecolor{currentstroke}{rgb}{0.126326,0.644107,0.525311}%
\pgfsetstrokecolor{currentstroke}%
\pgfsetdash{}{0pt}%
\pgfpathmoveto{\pgfqpoint{3.342172in}{0.665846in}}%
\pgfpathlineto{\pgfqpoint{3.359243in}{0.663281in}}%
\pgfpathlineto{\pgfqpoint{3.394285in}{0.658007in}}%
\pgfpathlineto{\pgfqpoint{3.429328in}{0.652722in}}%
\pgfpathlineto{\pgfqpoint{3.444847in}{0.650382in}}%
\pgfpathlineto{\pgfqpoint{3.464371in}{0.647493in}}%
\pgfpathlineto{\pgfqpoint{3.499413in}{0.642307in}}%
\pgfpathlineto{\pgfqpoint{3.534456in}{0.637110in}}%
\pgfpathlineto{\pgfqpoint{3.569498in}{0.631904in}}%
\pgfpathlineto{\pgfqpoint{3.604541in}{0.626687in}}%
\pgfpathlineto{\pgfqpoint{3.621758in}{0.624122in}}%
\pgfpathlineto{\pgfqpoint{3.639583in}{0.621518in}}%
\pgfpathlineto{\pgfqpoint{3.674626in}{0.616398in}}%
\pgfpathlineto{\pgfqpoint{3.709668in}{0.611266in}}%
\pgfpathlineto{\pgfqpoint{3.744711in}{0.606125in}}%
\pgfpathlineto{\pgfqpoint{3.779753in}{0.600973in}}%
\pgfpathlineto{\pgfqpoint{3.800891in}{0.597863in}}%
\pgfpathlineto{\pgfqpoint{3.814796in}{0.595856in}}%
\pgfpathlineto{\pgfqpoint{3.849838in}{0.590799in}}%
\pgfpathlineto{\pgfqpoint{3.884881in}{0.585731in}}%
\pgfpathlineto{\pgfqpoint{3.919924in}{0.580653in}}%
\pgfpathlineto{\pgfqpoint{3.954966in}{0.575564in}}%
\pgfpathlineto{\pgfqpoint{3.982196in}{0.571603in}}%
\pgfusepath{stroke}%
\end{pgfscope}%
\begin{pgfscope}%
\pgfpathrectangle{\pgfqpoint{0.766095in}{0.571603in}}{\pgfqpoint{6.973465in}{5.225635in}}%
\pgfusepath{clip}%
\pgfsetbuttcap%
\pgfsetroundjoin%
\pgfsetlinewidth{1.505625pt}%
\definecolor{currentstroke}{rgb}{0.143303,0.669459,0.511215}%
\pgfsetstrokecolor{currentstroke}%
\pgfsetdash{}{0pt}%
\pgfpathmoveto{\pgfqpoint{0.766095in}{1.062075in}}%
\pgfpathlineto{\pgfqpoint{0.801138in}{1.055131in}}%
\pgfpathlineto{\pgfqpoint{0.836180in}{1.048187in}}%
\pgfpathlineto{\pgfqpoint{0.855958in}{1.044274in}}%
\pgfpathlineto{\pgfqpoint{0.871223in}{1.041309in}}%
\pgfpathlineto{\pgfqpoint{0.906265in}{1.034514in}}%
\pgfpathlineto{\pgfqpoint{0.941308in}{1.027719in}}%
\pgfpathlineto{\pgfqpoint{0.976350in}{1.020924in}}%
\pgfpathlineto{\pgfqpoint{0.991378in}{1.018014in}}%
\pgfpathlineto{\pgfqpoint{1.011393in}{1.014211in}}%
\pgfpathlineto{\pgfqpoint{1.046435in}{1.007559in}}%
\pgfpathlineto{\pgfqpoint{1.081478in}{1.000906in}}%
\pgfpathlineto{\pgfqpoint{1.116520in}{0.994252in}}%
\pgfpathlineto{\pgfqpoint{1.129693in}{0.991755in}}%
\pgfpathlineto{\pgfqpoint{1.151563in}{0.987685in}}%
\pgfpathlineto{\pgfqpoint{1.186605in}{0.981170in}}%
\pgfpathlineto{\pgfqpoint{1.221648in}{0.974653in}}%
\pgfpathlineto{\pgfqpoint{1.256691in}{0.968133in}}%
\pgfpathlineto{\pgfqpoint{1.270887in}{0.965495in}}%
\pgfpathlineto{\pgfqpoint{1.291733in}{0.961694in}}%
\pgfpathlineto{\pgfqpoint{1.326776in}{0.955309in}}%
\pgfpathlineto{\pgfqpoint{1.361818in}{0.948920in}}%
\pgfpathlineto{\pgfqpoint{1.396861in}{0.942529in}}%
\pgfpathlineto{\pgfqpoint{1.414935in}{0.939236in}}%
\pgfpathlineto{\pgfqpoint{1.431903in}{0.936201in}}%
\pgfpathlineto{\pgfqpoint{1.466946in}{0.929940in}}%
\pgfpathlineto{\pgfqpoint{1.501988in}{0.923675in}}%
\pgfpathlineto{\pgfqpoint{1.537031in}{0.917407in}}%
\pgfpathlineto{\pgfqpoint{1.561804in}{0.912976in}}%
\pgfpathlineto{\pgfqpoint{1.572073in}{0.911174in}}%
\pgfpathlineto{\pgfqpoint{1.607116in}{0.905032in}}%
\pgfpathlineto{\pgfqpoint{1.642158in}{0.898885in}}%
\pgfpathlineto{\pgfqpoint{1.677201in}{0.892734in}}%
\pgfpathlineto{\pgfqpoint{1.711455in}{0.886717in}}%
\pgfpathlineto{\pgfqpoint{1.712244in}{0.886581in}}%
\pgfpathlineto{\pgfqpoint{1.747286in}{0.880553in}}%
\pgfpathlineto{\pgfqpoint{1.782329in}{0.874519in}}%
\pgfpathlineto{\pgfqpoint{1.817371in}{0.868481in}}%
\pgfpathlineto{\pgfqpoint{1.852414in}{0.862437in}}%
\pgfpathlineto{\pgfqpoint{1.863902in}{0.860458in}}%
\pgfpathlineto{\pgfqpoint{1.887456in}{0.856475in}}%
\pgfpathlineto{\pgfqpoint{1.922499in}{0.850550in}}%
\pgfpathlineto{\pgfqpoint{1.957541in}{0.844619in}}%
\pgfpathlineto{\pgfqpoint{1.992584in}{0.838683in}}%
\pgfpathlineto{\pgfqpoint{2.019049in}{0.834198in}}%
\pgfpathlineto{\pgfqpoint{2.027626in}{0.832771in}}%
\pgfpathlineto{\pgfqpoint{2.062669in}{0.826951in}}%
\pgfpathlineto{\pgfqpoint{2.097711in}{0.821124in}}%
\pgfpathlineto{\pgfqpoint{2.132754in}{0.815291in}}%
\pgfpathlineto{\pgfqpoint{2.167797in}{0.809451in}}%
\pgfpathlineto{\pgfqpoint{2.176880in}{0.807939in}}%
\pgfpathlineto{\pgfqpoint{2.202839in}{0.803697in}}%
\pgfpathlineto{\pgfqpoint{2.237882in}{0.797970in}}%
\pgfpathlineto{\pgfqpoint{2.272924in}{0.792236in}}%
\pgfpathlineto{\pgfqpoint{2.307967in}{0.786495in}}%
\pgfpathlineto{\pgfqpoint{2.337341in}{0.781679in}}%
\pgfpathlineto{\pgfqpoint{2.343009in}{0.780767in}}%
\pgfpathlineto{\pgfqpoint{2.378052in}{0.775136in}}%
\pgfpathlineto{\pgfqpoint{2.413094in}{0.769498in}}%
\pgfpathlineto{\pgfqpoint{2.448137in}{0.763852in}}%
\pgfpathlineto{\pgfqpoint{2.483179in}{0.758199in}}%
\pgfpathlineto{\pgfqpoint{2.500407in}{0.755420in}}%
\pgfpathlineto{\pgfqpoint{2.518222in}{0.752600in}}%
\pgfpathlineto{\pgfqpoint{2.553264in}{0.747054in}}%
\pgfpathlineto{\pgfqpoint{2.588307in}{0.741500in}}%
\pgfpathlineto{\pgfqpoint{2.623350in}{0.735938in}}%
\pgfpathlineto{\pgfqpoint{2.658392in}{0.730368in}}%
\pgfpathlineto{\pgfqpoint{2.665999in}{0.729160in}}%
\pgfpathlineto{\pgfqpoint{2.693435in}{0.724885in}}%
\pgfpathlineto{\pgfqpoint{2.728477in}{0.719420in}}%
\pgfpathlineto{\pgfqpoint{2.763520in}{0.713947in}}%
\pgfpathlineto{\pgfqpoint{2.798562in}{0.708465in}}%
\pgfpathlineto{\pgfqpoint{2.833605in}{0.702975in}}%
\pgfpathlineto{\pgfqpoint{2.834079in}{0.702901in}}%
\pgfpathlineto{\pgfqpoint{2.868647in}{0.697594in}}%
\pgfpathlineto{\pgfqpoint{2.903690in}{0.692207in}}%
\pgfpathlineto{\pgfqpoint{2.938732in}{0.686810in}}%
\pgfpathlineto{\pgfqpoint{2.973775in}{0.681405in}}%
\pgfpathlineto{\pgfqpoint{2.984386in}{0.679767in}}%
\pgfusepath{stroke}%
\end{pgfscope}%
\begin{pgfscope}%
\pgfpathrectangle{\pgfqpoint{0.766095in}{0.571603in}}{\pgfqpoint{6.973465in}{5.225635in}}%
\pgfusepath{clip}%
\pgfsetbuttcap%
\pgfsetroundjoin%
\pgfsetlinewidth{1.505625pt}%
\definecolor{currentstroke}{rgb}{0.143303,0.669459,0.511215}%
\pgfsetstrokecolor{currentstroke}%
\pgfsetdash{}{0pt}%
\pgfpathmoveto{\pgfqpoint{3.370796in}{0.621474in}}%
\pgfpathlineto{\pgfqpoint{3.394285in}{0.618012in}}%
\pgfpathlineto{\pgfqpoint{3.429328in}{0.612838in}}%
\pgfpathlineto{\pgfqpoint{3.464371in}{0.607654in}}%
\pgfpathlineto{\pgfqpoint{3.499413in}{0.602461in}}%
\pgfpathlineto{\pgfqpoint{3.530390in}{0.597863in}}%
\pgfpathlineto{\pgfqpoint{3.534456in}{0.597271in}}%
\pgfpathlineto{\pgfqpoint{3.569498in}{0.592172in}}%
\pgfpathlineto{\pgfqpoint{3.604541in}{0.587064in}}%
\pgfpathlineto{\pgfqpoint{3.639583in}{0.581946in}}%
\pgfpathlineto{\pgfqpoint{3.674626in}{0.576818in}}%
\pgfpathlineto{\pgfqpoint{3.709668in}{0.571680in}}%
\pgfpathlineto{\pgfqpoint{3.710193in}{0.571603in}}%
\pgfusepath{stroke}%
\end{pgfscope}%
\begin{pgfscope}%
\pgfpathrectangle{\pgfqpoint{0.766095in}{0.571603in}}{\pgfqpoint{6.973465in}{5.225635in}}%
\pgfusepath{clip}%
\pgfsetbuttcap%
\pgfsetroundjoin%
\pgfsetlinewidth{1.505625pt}%
\definecolor{currentstroke}{rgb}{0.170948,0.694384,0.493803}%
\pgfsetstrokecolor{currentstroke}%
\pgfsetdash{}{0pt}%
\pgfpathmoveto{\pgfqpoint{0.766095in}{1.017619in}}%
\pgfpathlineto{\pgfqpoint{0.801138in}{1.010834in}}%
\pgfpathlineto{\pgfqpoint{0.836180in}{1.004050in}}%
\pgfpathlineto{\pgfqpoint{0.871223in}{0.997267in}}%
\pgfpathlineto{\pgfqpoint{0.899712in}{0.991755in}}%
\pgfpathlineto{\pgfqpoint{0.906265in}{0.990510in}}%
\pgfpathlineto{\pgfqpoint{0.941308in}{0.983870in}}%
\pgfpathlineto{\pgfqpoint{0.976350in}{0.977230in}}%
\pgfpathlineto{\pgfqpoint{1.011393in}{0.970589in}}%
\pgfpathlineto{\pgfqpoint{1.038286in}{0.965495in}}%
\pgfpathlineto{\pgfqpoint{1.046435in}{0.963980in}}%
\pgfpathlineto{\pgfqpoint{1.081478in}{0.957478in}}%
\pgfpathlineto{\pgfqpoint{1.116520in}{0.950975in}}%
\pgfpathlineto{\pgfqpoint{1.151563in}{0.944470in}}%
\pgfpathlineto{\pgfqpoint{1.179767in}{0.939236in}}%
\pgfpathlineto{\pgfqpoint{1.186605in}{0.937990in}}%
\pgfpathlineto{\pgfqpoint{1.221648in}{0.931619in}}%
\pgfpathlineto{\pgfqpoint{1.256691in}{0.925247in}}%
\pgfpathlineto{\pgfqpoint{1.291733in}{0.918871in}}%
\pgfpathlineto{\pgfqpoint{1.324130in}{0.912976in}}%
\pgfpathlineto{\pgfqpoint{1.326776in}{0.912504in}}%
\pgfpathlineto{\pgfqpoint{1.361818in}{0.906258in}}%
\pgfpathlineto{\pgfqpoint{1.396861in}{0.900010in}}%
\pgfpathlineto{\pgfqpoint{1.431903in}{0.893759in}}%
\pgfpathlineto{\pgfqpoint{1.466946in}{0.887504in}}%
\pgfpathlineto{\pgfqpoint{1.471366in}{0.886717in}}%
\pgfpathlineto{\pgfqpoint{1.501988in}{0.881362in}}%
\pgfpathlineto{\pgfqpoint{1.537031in}{0.875233in}}%
\pgfpathlineto{\pgfqpoint{1.572073in}{0.869100in}}%
\pgfpathlineto{\pgfqpoint{1.607116in}{0.862963in}}%
\pgfpathlineto{\pgfqpoint{1.621439in}{0.860458in}}%
\pgfpathlineto{\pgfqpoint{1.642158in}{0.856899in}}%
\pgfpathlineto{\pgfqpoint{1.677201in}{0.850884in}}%
\pgfpathlineto{\pgfqpoint{1.712244in}{0.844865in}}%
\pgfpathlineto{\pgfqpoint{1.747286in}{0.838841in}}%
\pgfpathlineto{\pgfqpoint{1.774288in}{0.834198in}}%
\pgfpathlineto{\pgfqpoint{1.782329in}{0.832841in}}%
\pgfpathlineto{\pgfqpoint{1.817371in}{0.826935in}}%
\pgfpathlineto{\pgfqpoint{1.852414in}{0.821024in}}%
\pgfpathlineto{\pgfqpoint{1.887456in}{0.815108in}}%
\pgfpathlineto{\pgfqpoint{1.922499in}{0.809187in}}%
\pgfpathlineto{\pgfqpoint{1.929895in}{0.807939in}}%
\pgfpathlineto{\pgfqpoint{1.957541in}{0.803359in}}%
\pgfpathlineto{\pgfqpoint{1.992584in}{0.797553in}}%
\pgfpathlineto{\pgfqpoint{2.027626in}{0.791740in}}%
\pgfpathlineto{\pgfqpoint{2.062669in}{0.785922in}}%
\pgfpathlineto{\pgfqpoint{2.088218in}{0.781679in}}%
\pgfpathlineto{\pgfqpoint{2.097711in}{0.780132in}}%
\pgfpathlineto{\pgfqpoint{2.132754in}{0.774425in}}%
\pgfpathlineto{\pgfqpoint{2.167797in}{0.768713in}}%
\pgfpathlineto{\pgfqpoint{2.202839in}{0.762994in}}%
\pgfpathlineto{\pgfqpoint{2.237882in}{0.757269in}}%
\pgfpathlineto{\pgfqpoint{2.249207in}{0.755420in}}%
\pgfpathlineto{\pgfqpoint{2.272924in}{0.751619in}}%
\pgfpathlineto{\pgfqpoint{2.290226in}{0.748846in}}%
\pgfusepath{stroke}%
\end{pgfscope}%
\begin{pgfscope}%
\pgfpathrectangle{\pgfqpoint{0.766095in}{0.571603in}}{\pgfqpoint{6.973465in}{5.225635in}}%
\pgfusepath{clip}%
\pgfsetbuttcap%
\pgfsetroundjoin%
\pgfsetlinewidth{1.505625pt}%
\definecolor{currentstroke}{rgb}{0.170948,0.694384,0.493803}%
\pgfsetstrokecolor{currentstroke}%
\pgfsetdash{}{0pt}%
\pgfpathmoveto{\pgfqpoint{2.676213in}{0.687786in}}%
\pgfpathlineto{\pgfqpoint{2.693435in}{0.685104in}}%
\pgfpathlineto{\pgfqpoint{2.728477in}{0.679639in}}%
\pgfpathlineto{\pgfqpoint{2.747694in}{0.676641in}}%
\pgfpathlineto{\pgfqpoint{2.763520in}{0.674219in}}%
\pgfpathlineto{\pgfqpoint{2.798562in}{0.668855in}}%
\pgfpathlineto{\pgfqpoint{2.833605in}{0.663483in}}%
\pgfpathlineto{\pgfqpoint{2.868647in}{0.658103in}}%
\pgfpathlineto{\pgfqpoint{2.903690in}{0.652715in}}%
\pgfpathlineto{\pgfqpoint{2.918862in}{0.650382in}}%
\pgfpathlineto{\pgfqpoint{2.938732in}{0.647384in}}%
\pgfpathlineto{\pgfqpoint{2.973775in}{0.642095in}}%
\pgfpathlineto{\pgfqpoint{3.008818in}{0.636797in}}%
\pgfpathlineto{\pgfqpoint{3.043860in}{0.631491in}}%
\pgfpathlineto{\pgfqpoint{3.078903in}{0.626176in}}%
\pgfpathlineto{\pgfqpoint{3.092447in}{0.624122in}}%
\pgfpathlineto{\pgfqpoint{3.113945in}{0.620923in}}%
\pgfpathlineto{\pgfqpoint{3.148988in}{0.615706in}}%
\pgfpathlineto{\pgfqpoint{3.184030in}{0.610480in}}%
\pgfpathlineto{\pgfqpoint{3.219073in}{0.605245in}}%
\pgfpathlineto{\pgfqpoint{3.254115in}{0.600000in}}%
\pgfpathlineto{\pgfqpoint{3.268399in}{0.597863in}}%
\pgfpathlineto{\pgfqpoint{3.289158in}{0.594814in}}%
\pgfpathlineto{\pgfqpoint{3.324200in}{0.589666in}}%
\pgfpathlineto{\pgfqpoint{3.359243in}{0.584508in}}%
\pgfpathlineto{\pgfqpoint{3.394285in}{0.579341in}}%
\pgfpathlineto{\pgfqpoint{3.429328in}{0.574165in}}%
\pgfpathlineto{\pgfqpoint{3.446665in}{0.571603in}}%
\pgfusepath{stroke}%
\end{pgfscope}%
\begin{pgfscope}%
\pgfpathrectangle{\pgfqpoint{0.766095in}{0.571603in}}{\pgfqpoint{6.973465in}{5.225635in}}%
\pgfusepath{clip}%
\pgfsetbuttcap%
\pgfsetroundjoin%
\pgfsetlinewidth{1.505625pt}%
\definecolor{currentstroke}{rgb}{0.208030,0.718701,0.472873}%
\pgfsetstrokecolor{currentstroke}%
\pgfsetdash{}{0pt}%
\pgfpathmoveto{\pgfqpoint{0.766095in}{0.974825in}}%
\pgfpathlineto{\pgfqpoint{0.801138in}{0.968056in}}%
\pgfpathlineto{\pgfqpoint{0.814421in}{0.965495in}}%
\pgfpathlineto{\pgfqpoint{0.836180in}{0.961376in}}%
\pgfpathlineto{\pgfqpoint{0.871223in}{0.954750in}}%
\pgfpathlineto{\pgfqpoint{0.906265in}{0.948125in}}%
\pgfpathlineto{\pgfqpoint{0.941308in}{0.941500in}}%
\pgfpathlineto{\pgfqpoint{0.953303in}{0.939236in}}%
\pgfpathlineto{\pgfqpoint{0.976350in}{0.934965in}}%
\pgfpathlineto{\pgfqpoint{1.011393in}{0.928478in}}%
\pgfpathlineto{\pgfqpoint{1.046435in}{0.921990in}}%
\pgfpathlineto{\pgfqpoint{1.081478in}{0.915501in}}%
\pgfpathlineto{\pgfqpoint{1.095132in}{0.912976in}}%
\pgfpathlineto{\pgfqpoint{1.116520in}{0.909094in}}%
\pgfpathlineto{\pgfqpoint{1.151563in}{0.902738in}}%
\pgfpathlineto{\pgfqpoint{1.186605in}{0.896381in}}%
\pgfpathlineto{\pgfqpoint{1.221648in}{0.890022in}}%
\pgfpathlineto{\pgfqpoint{1.239883in}{0.886717in}}%
\pgfpathlineto{\pgfqpoint{1.256691in}{0.883725in}}%
\pgfpathlineto{\pgfqpoint{1.291733in}{0.877496in}}%
\pgfpathlineto{\pgfqpoint{1.326776in}{0.871264in}}%
\pgfpathlineto{\pgfqpoint{1.361818in}{0.865029in}}%
\pgfpathlineto{\pgfqpoint{1.387521in}{0.860458in}}%
\pgfpathlineto{\pgfqpoint{1.396861in}{0.858826in}}%
\pgfpathlineto{\pgfqpoint{1.431903in}{0.852717in}}%
\pgfpathlineto{\pgfqpoint{1.466946in}{0.846604in}}%
\pgfpathlineto{\pgfqpoint{1.501988in}{0.840488in}}%
\pgfpathlineto{\pgfqpoint{1.537031in}{0.834369in}}%
\pgfpathlineto{\pgfqpoint{1.538012in}{0.834198in}}%
\pgfpathlineto{\pgfqpoint{1.572073in}{0.828371in}}%
\pgfpathlineto{\pgfqpoint{1.607116in}{0.822372in}}%
\pgfpathlineto{\pgfqpoint{1.642158in}{0.816370in}}%
\pgfpathlineto{\pgfqpoint{1.677201in}{0.810364in}}%
\pgfpathlineto{\pgfqpoint{1.691363in}{0.807939in}}%
\pgfpathlineto{\pgfqpoint{1.712244in}{0.804428in}}%
\pgfpathlineto{\pgfqpoint{1.747286in}{0.798539in}}%
\pgfpathlineto{\pgfqpoint{1.782329in}{0.792646in}}%
\pgfpathlineto{\pgfqpoint{1.817371in}{0.786748in}}%
\pgfpathlineto{\pgfqpoint{1.847471in}{0.781679in}}%
\pgfpathlineto{\pgfqpoint{1.852414in}{0.780862in}}%
\pgfpathlineto{\pgfqpoint{1.887456in}{0.775078in}}%
\pgfpathlineto{\pgfqpoint{1.922499in}{0.769289in}}%
\pgfpathlineto{\pgfqpoint{1.957541in}{0.763495in}}%
\pgfpathlineto{\pgfqpoint{1.992584in}{0.757695in}}%
\pgfpathlineto{\pgfqpoint{2.006342in}{0.755420in}}%
\pgfpathlineto{\pgfqpoint{2.027626in}{0.751964in}}%
\pgfpathlineto{\pgfqpoint{2.062669in}{0.746275in}}%
\pgfpathlineto{\pgfqpoint{2.097711in}{0.740581in}}%
\pgfpathlineto{\pgfqpoint{2.132754in}{0.734880in}}%
\pgfpathlineto{\pgfqpoint{2.167797in}{0.729174in}}%
\pgfpathlineto{\pgfqpoint{2.167880in}{0.729160in}}%
\pgfpathlineto{\pgfqpoint{2.202839in}{0.723581in}}%
\pgfpathlineto{\pgfqpoint{2.237882in}{0.717983in}}%
\pgfpathlineto{\pgfqpoint{2.272924in}{0.712378in}}%
\pgfpathlineto{\pgfqpoint{2.290180in}{0.709615in}}%
\pgfusepath{stroke}%
\end{pgfscope}%
\begin{pgfscope}%
\pgfpathrectangle{\pgfqpoint{0.766095in}{0.571603in}}{\pgfqpoint{6.973465in}{5.225635in}}%
\pgfusepath{clip}%
\pgfsetbuttcap%
\pgfsetroundjoin%
\pgfsetlinewidth{1.505625pt}%
\definecolor{currentstroke}{rgb}{0.208030,0.718701,0.472873}%
\pgfsetstrokecolor{currentstroke}%
\pgfsetdash{}{0pt}%
\pgfpathmoveto{\pgfqpoint{2.676262in}{0.649162in}}%
\pgfpathlineto{\pgfqpoint{2.693435in}{0.646544in}}%
\pgfpathlineto{\pgfqpoint{2.728477in}{0.641199in}}%
\pgfpathlineto{\pgfqpoint{2.763520in}{0.635846in}}%
\pgfpathlineto{\pgfqpoint{2.798562in}{0.630486in}}%
\pgfpathlineto{\pgfqpoint{2.833605in}{0.625117in}}%
\pgfpathlineto{\pgfqpoint{2.840103in}{0.624122in}}%
\pgfpathlineto{\pgfqpoint{2.868647in}{0.619834in}}%
\pgfpathlineto{\pgfqpoint{2.903690in}{0.614563in}}%
\pgfpathlineto{\pgfqpoint{2.938732in}{0.609285in}}%
\pgfpathlineto{\pgfqpoint{2.973775in}{0.603999in}}%
\pgfpathlineto{\pgfqpoint{3.008818in}{0.598704in}}%
\pgfpathlineto{\pgfqpoint{3.014389in}{0.597863in}}%
\pgfpathlineto{\pgfqpoint{3.043860in}{0.593496in}}%
\pgfpathlineto{\pgfqpoint{3.078903in}{0.588298in}}%
\pgfpathlineto{\pgfqpoint{3.113945in}{0.583091in}}%
\pgfpathlineto{\pgfqpoint{3.148988in}{0.577876in}}%
\pgfpathlineto{\pgfqpoint{3.184030in}{0.572651in}}%
\pgfpathlineto{\pgfqpoint{3.191065in}{0.571603in}}%
\pgfusepath{stroke}%
\end{pgfscope}%
\begin{pgfscope}%
\pgfpathrectangle{\pgfqpoint{0.766095in}{0.571603in}}{\pgfqpoint{6.973465in}{5.225635in}}%
\pgfusepath{clip}%
\pgfsetbuttcap%
\pgfsetroundjoin%
\pgfsetlinewidth{1.505625pt}%
\definecolor{currentstroke}{rgb}{0.252899,0.742211,0.448284}%
\pgfsetstrokecolor{currentstroke}%
\pgfsetdash{}{0pt}%
\pgfpathmoveto{\pgfqpoint{0.766095in}{0.933383in}}%
\pgfpathlineto{\pgfqpoint{0.801138in}{0.926774in}}%
\pgfpathlineto{\pgfqpoint{0.836180in}{0.920166in}}%
\pgfpathlineto{\pgfqpoint{0.871223in}{0.913559in}}%
\pgfpathlineto{\pgfqpoint{0.874321in}{0.912976in}}%
\pgfpathlineto{\pgfqpoint{0.906265in}{0.907077in}}%
\pgfpathlineto{\pgfqpoint{0.941308in}{0.900607in}}%
\pgfpathlineto{\pgfqpoint{0.976350in}{0.894138in}}%
\pgfpathlineto{\pgfqpoint{1.011393in}{0.887667in}}%
\pgfpathlineto{\pgfqpoint{1.016552in}{0.886717in}}%
\pgfpathlineto{\pgfqpoint{1.046435in}{0.881311in}}%
\pgfpathlineto{\pgfqpoint{1.081478in}{0.874974in}}%
\pgfpathlineto{\pgfqpoint{1.116520in}{0.868635in}}%
\pgfpathlineto{\pgfqpoint{1.151563in}{0.862296in}}%
\pgfpathlineto{\pgfqpoint{1.161741in}{0.860458in}}%
\pgfpathlineto{\pgfqpoint{1.186605in}{0.856048in}}%
\pgfpathlineto{\pgfqpoint{1.221648in}{0.849837in}}%
\pgfpathlineto{\pgfqpoint{1.256691in}{0.843624in}}%
\pgfpathlineto{\pgfqpoint{1.291733in}{0.837409in}}%
\pgfpathlineto{\pgfqpoint{1.309854in}{0.834198in}}%
\pgfpathlineto{\pgfqpoint{1.326776in}{0.831254in}}%
\pgfpathlineto{\pgfqpoint{1.361818in}{0.825163in}}%
\pgfpathlineto{\pgfqpoint{1.396861in}{0.819070in}}%
\pgfpathlineto{\pgfqpoint{1.431903in}{0.812974in}}%
\pgfpathlineto{\pgfqpoint{1.460846in}{0.807939in}}%
\pgfpathlineto{\pgfqpoint{1.466946in}{0.806897in}}%
\pgfpathlineto{\pgfqpoint{1.501988in}{0.800921in}}%
\pgfpathlineto{\pgfqpoint{1.537031in}{0.794942in}}%
\pgfpathlineto{\pgfqpoint{1.572073in}{0.788960in}}%
\pgfpathlineto{\pgfqpoint{1.589627in}{0.785962in}}%
\pgfusepath{stroke}%
\end{pgfscope}%
\begin{pgfscope}%
\pgfpathrectangle{\pgfqpoint{0.766095in}{0.571603in}}{\pgfqpoint{6.973465in}{5.225635in}}%
\pgfusepath{clip}%
\pgfsetbuttcap%
\pgfsetroundjoin%
\pgfsetlinewidth{1.505625pt}%
\definecolor{currentstroke}{rgb}{0.252899,0.742211,0.448284}%
\pgfsetstrokecolor{currentstroke}%
\pgfsetdash{}{0pt}%
\pgfpathmoveto{\pgfqpoint{1.975148in}{0.721994in}}%
\pgfpathlineto{\pgfqpoint{1.992584in}{0.719174in}}%
\pgfpathlineto{\pgfqpoint{2.027626in}{0.713500in}}%
\pgfpathlineto{\pgfqpoint{2.062669in}{0.707821in}}%
\pgfpathlineto{\pgfqpoint{2.093009in}{0.702901in}}%
\pgfpathlineto{\pgfqpoint{2.097711in}{0.702152in}}%
\pgfpathlineto{\pgfqpoint{2.132754in}{0.696580in}}%
\pgfpathlineto{\pgfqpoint{2.167797in}{0.691003in}}%
\pgfpathlineto{\pgfqpoint{2.202839in}{0.685419in}}%
\pgfpathlineto{\pgfqpoint{2.237882in}{0.679830in}}%
\pgfpathlineto{\pgfqpoint{2.257873in}{0.676641in}}%
\pgfpathlineto{\pgfqpoint{2.272924in}{0.674284in}}%
\pgfpathlineto{\pgfqpoint{2.307967in}{0.668799in}}%
\pgfpathlineto{\pgfqpoint{2.343009in}{0.663308in}}%
\pgfpathlineto{\pgfqpoint{2.378052in}{0.657810in}}%
\pgfpathlineto{\pgfqpoint{2.413094in}{0.652306in}}%
\pgfpathlineto{\pgfqpoint{2.425350in}{0.650382in}}%
\pgfpathlineto{\pgfqpoint{2.448137in}{0.646870in}}%
\pgfpathlineto{\pgfqpoint{2.483179in}{0.641467in}}%
\pgfpathlineto{\pgfqpoint{2.518222in}{0.636058in}}%
\pgfpathlineto{\pgfqpoint{2.553264in}{0.630642in}}%
\pgfpathlineto{\pgfqpoint{2.588307in}{0.625218in}}%
\pgfpathlineto{\pgfqpoint{2.595395in}{0.624122in}}%
\pgfpathlineto{\pgfqpoint{2.623350in}{0.619879in}}%
\pgfpathlineto{\pgfqpoint{2.658392in}{0.614555in}}%
\pgfpathlineto{\pgfqpoint{2.693435in}{0.609224in}}%
\pgfpathlineto{\pgfqpoint{2.728477in}{0.603885in}}%
\pgfpathlineto{\pgfqpoint{2.763520in}{0.598539in}}%
\pgfpathlineto{\pgfqpoint{2.767959in}{0.597863in}}%
\pgfpathlineto{\pgfqpoint{2.798562in}{0.593284in}}%
\pgfpathlineto{\pgfqpoint{2.833605in}{0.588036in}}%
\pgfpathlineto{\pgfqpoint{2.868647in}{0.582779in}}%
\pgfpathlineto{\pgfqpoint{2.903690in}{0.577515in}}%
\pgfpathlineto{\pgfqpoint{2.938732in}{0.572243in}}%
\pgfpathlineto{\pgfqpoint{2.942990in}{0.571603in}}%
\pgfusepath{stroke}%
\end{pgfscope}%
\begin{pgfscope}%
\pgfpathrectangle{\pgfqpoint{0.766095in}{0.571603in}}{\pgfqpoint{6.973465in}{5.225635in}}%
\pgfusepath{clip}%
\pgfsetbuttcap%
\pgfsetroundjoin%
\pgfsetlinewidth{1.505625pt}%
\definecolor{currentstroke}{rgb}{0.304148,0.764704,0.419943}%
\pgfsetstrokecolor{currentstroke}%
\pgfsetdash{}{0pt}%
\pgfpathmoveto{\pgfqpoint{0.766095in}{0.893281in}}%
\pgfpathlineto{\pgfqpoint{0.801019in}{0.886717in}}%
\pgfpathlineto{\pgfqpoint{0.801138in}{0.886695in}}%
\pgfpathlineto{\pgfqpoint{0.836180in}{0.880245in}}%
\pgfpathlineto{\pgfqpoint{0.871223in}{0.873795in}}%
\pgfpathlineto{\pgfqpoint{0.906265in}{0.867346in}}%
\pgfpathlineto{\pgfqpoint{0.941308in}{0.860897in}}%
\pgfpathlineto{\pgfqpoint{0.943703in}{0.860458in}}%
\pgfpathlineto{\pgfqpoint{0.976350in}{0.854571in}}%
\pgfpathlineto{\pgfqpoint{1.011393in}{0.848255in}}%
\pgfpathlineto{\pgfqpoint{1.046435in}{0.841938in}}%
\pgfpathlineto{\pgfqpoint{1.081478in}{0.835620in}}%
\pgfpathlineto{\pgfqpoint{1.089378in}{0.834198in}}%
\pgfpathlineto{\pgfqpoint{1.116520in}{0.829401in}}%
\pgfpathlineto{\pgfqpoint{1.151563in}{0.823211in}}%
\pgfpathlineto{\pgfqpoint{1.186605in}{0.817020in}}%
\pgfpathlineto{\pgfqpoint{1.221648in}{0.810827in}}%
\pgfpathlineto{\pgfqpoint{1.238009in}{0.807939in}}%
\pgfpathlineto{\pgfqpoint{1.256691in}{0.804700in}}%
\pgfpathlineto{\pgfqpoint{1.291733in}{0.798630in}}%
\pgfpathlineto{\pgfqpoint{1.326776in}{0.792559in}}%
\pgfpathlineto{\pgfqpoint{1.361818in}{0.786485in}}%
\pgfpathlineto{\pgfqpoint{1.389551in}{0.781679in}}%
\pgfpathlineto{\pgfqpoint{1.396861in}{0.780435in}}%
\pgfpathlineto{\pgfqpoint{1.431903in}{0.774481in}}%
\pgfpathlineto{\pgfqpoint{1.466946in}{0.768524in}}%
\pgfpathlineto{\pgfqpoint{1.501988in}{0.762565in}}%
\pgfpathlineto{\pgfqpoint{1.537031in}{0.756602in}}%
\pgfpathlineto{\pgfqpoint{1.543989in}{0.755420in}}%
\pgfpathlineto{\pgfqpoint{1.572073in}{0.750734in}}%
\pgfpathlineto{\pgfqpoint{1.589561in}{0.747816in}}%
\pgfusepath{stroke}%
\end{pgfscope}%
\begin{pgfscope}%
\pgfpathrectangle{\pgfqpoint{0.766095in}{0.571603in}}{\pgfqpoint{6.973465in}{5.225635in}}%
\pgfusepath{clip}%
\pgfsetbuttcap%
\pgfsetroundjoin%
\pgfsetlinewidth{1.505625pt}%
\definecolor{currentstroke}{rgb}{0.304148,0.764704,0.419943}%
\pgfsetstrokecolor{currentstroke}%
\pgfsetdash{}{0pt}%
\pgfpathmoveto{\pgfqpoint{1.975197in}{0.684553in}}%
\pgfpathlineto{\pgfqpoint{1.992584in}{0.681747in}}%
\pgfpathlineto{\pgfqpoint{2.024197in}{0.676641in}}%
\pgfpathlineto{\pgfqpoint{2.027626in}{0.676097in}}%
\pgfpathlineto{\pgfqpoint{2.062669in}{0.670549in}}%
\pgfpathlineto{\pgfqpoint{2.097711in}{0.664994in}}%
\pgfpathlineto{\pgfqpoint{2.132754in}{0.659435in}}%
\pgfpathlineto{\pgfqpoint{2.167797in}{0.653869in}}%
\pgfpathlineto{\pgfqpoint{2.189754in}{0.650382in}}%
\pgfpathlineto{\pgfqpoint{2.202839in}{0.648341in}}%
\pgfpathlineto{\pgfqpoint{2.237882in}{0.642879in}}%
\pgfpathlineto{\pgfqpoint{2.272924in}{0.637412in}}%
\pgfpathlineto{\pgfqpoint{2.307967in}{0.631938in}}%
\pgfpathlineto{\pgfqpoint{2.343009in}{0.626458in}}%
\pgfpathlineto{\pgfqpoint{2.357950in}{0.624122in}}%
\pgfpathlineto{\pgfqpoint{2.378052in}{0.621037in}}%
\pgfpathlineto{\pgfqpoint{2.413094in}{0.615658in}}%
\pgfpathlineto{\pgfqpoint{2.448137in}{0.610273in}}%
\pgfpathlineto{\pgfqpoint{2.483179in}{0.604881in}}%
\pgfpathlineto{\pgfqpoint{2.518222in}{0.599482in}}%
\pgfpathlineto{\pgfqpoint{2.528739in}{0.597863in}}%
\pgfpathlineto{\pgfqpoint{2.553264in}{0.594155in}}%
\pgfpathlineto{\pgfqpoint{2.588307in}{0.588855in}}%
\pgfpathlineto{\pgfqpoint{2.623350in}{0.583549in}}%
\pgfpathlineto{\pgfqpoint{2.658392in}{0.578235in}}%
\pgfpathlineto{\pgfqpoint{2.693435in}{0.572914in}}%
\pgfpathlineto{\pgfqpoint{2.702069in}{0.571603in}}%
\pgfusepath{stroke}%
\end{pgfscope}%
\begin{pgfscope}%
\pgfpathrectangle{\pgfqpoint{0.766095in}{0.571603in}}{\pgfqpoint{6.973465in}{5.225635in}}%
\pgfusepath{clip}%
\pgfsetbuttcap%
\pgfsetroundjoin%
\pgfsetlinewidth{1.505625pt}%
\definecolor{currentstroke}{rgb}{0.369214,0.788888,0.382914}%
\pgfsetstrokecolor{currentstroke}%
\pgfsetdash{}{0pt}%
\pgfpathmoveto{\pgfqpoint{0.958346in}{0.819455in}}%
\pgfpathlineto{\pgfqpoint{0.976350in}{0.816222in}}%
\pgfpathlineto{\pgfqpoint{1.011393in}{0.809927in}}%
\pgfpathlineto{\pgfqpoint{1.022486in}{0.807939in}}%
\pgfpathlineto{\pgfqpoint{1.046435in}{0.803720in}}%
\pgfpathlineto{\pgfqpoint{1.081478in}{0.797553in}}%
\pgfpathlineto{\pgfqpoint{1.116520in}{0.791386in}}%
\pgfpathlineto{\pgfqpoint{1.151563in}{0.785217in}}%
\pgfpathlineto{\pgfqpoint{1.171678in}{0.781679in}}%
\pgfpathlineto{\pgfqpoint{1.186605in}{0.779100in}}%
\pgfpathlineto{\pgfqpoint{1.221648in}{0.773054in}}%
\pgfpathlineto{\pgfqpoint{1.256691in}{0.767007in}}%
\pgfpathlineto{\pgfqpoint{1.291733in}{0.760957in}}%
\pgfpathlineto{\pgfqpoint{1.323808in}{0.755420in}}%
\pgfpathlineto{\pgfqpoint{1.326776in}{0.754916in}}%
\pgfpathlineto{\pgfqpoint{1.361818in}{0.748986in}}%
\pgfpathlineto{\pgfqpoint{1.396861in}{0.743054in}}%
\pgfpathlineto{\pgfqpoint{1.431903in}{0.737119in}}%
\pgfpathlineto{\pgfqpoint{1.466946in}{0.731181in}}%
\pgfpathlineto{\pgfqpoint{1.478886in}{0.729160in}}%
\pgfpathlineto{\pgfqpoint{1.501988in}{0.725320in}}%
\pgfpathlineto{\pgfqpoint{1.537031in}{0.719497in}}%
\pgfpathlineto{\pgfqpoint{1.572073in}{0.713672in}}%
\pgfpathlineto{\pgfqpoint{1.607116in}{0.707842in}}%
\pgfpathlineto{\pgfqpoint{1.636818in}{0.702901in}}%
\pgfpathlineto{\pgfqpoint{1.642158in}{0.702028in}}%
\pgfpathlineto{\pgfqpoint{1.677201in}{0.696311in}}%
\pgfpathlineto{\pgfqpoint{1.712244in}{0.690590in}}%
\pgfpathlineto{\pgfqpoint{1.747286in}{0.684865in}}%
\pgfpathlineto{\pgfqpoint{1.782329in}{0.679136in}}%
\pgfpathlineto{\pgfqpoint{1.797599in}{0.676641in}}%
\pgfpathlineto{\pgfqpoint{1.817371in}{0.673469in}}%
\pgfpathlineto{\pgfqpoint{1.852414in}{0.667848in}}%
\pgfpathlineto{\pgfqpoint{1.887456in}{0.662223in}}%
\pgfpathlineto{\pgfqpoint{1.922499in}{0.656594in}}%
\pgfpathlineto{\pgfqpoint{1.957541in}{0.650959in}}%
\pgfpathlineto{\pgfqpoint{1.961138in}{0.650382in}}%
\pgfpathlineto{\pgfqpoint{1.992584in}{0.645423in}}%
\pgfpathlineto{\pgfqpoint{2.027626in}{0.639895in}}%
\pgfpathlineto{\pgfqpoint{2.062669in}{0.634361in}}%
\pgfpathlineto{\pgfqpoint{2.097711in}{0.628821in}}%
\pgfpathlineto{\pgfqpoint{2.127422in}{0.624122in}}%
\pgfpathlineto{\pgfqpoint{2.132754in}{0.623294in}}%
\pgfpathlineto{\pgfqpoint{2.167797in}{0.617858in}}%
\pgfpathlineto{\pgfqpoint{2.202839in}{0.612416in}}%
\pgfpathlineto{\pgfqpoint{2.237882in}{0.606968in}}%
\pgfpathlineto{\pgfqpoint{2.272924in}{0.601515in}}%
\pgfpathlineto{\pgfqpoint{2.296385in}{0.597863in}}%
\pgfpathlineto{\pgfqpoint{2.307967in}{0.596092in}}%
\pgfpathlineto{\pgfqpoint{2.343009in}{0.590739in}}%
\pgfpathlineto{\pgfqpoint{2.378052in}{0.585380in}}%
\pgfpathlineto{\pgfqpoint{2.413094in}{0.580014in}}%
\pgfpathlineto{\pgfqpoint{2.448137in}{0.574642in}}%
\pgfpathlineto{\pgfqpoint{2.467962in}{0.571603in}}%
\pgfusepath{stroke}%
\end{pgfscope}%
\begin{pgfscope}%
\pgfpathrectangle{\pgfqpoint{0.766095in}{0.571603in}}{\pgfqpoint{6.973465in}{5.225635in}}%
\pgfusepath{clip}%
\pgfsetbuttcap%
\pgfsetroundjoin%
\pgfsetlinewidth{1.505625pt}%
\definecolor{currentstroke}{rgb}{0.430983,0.808473,0.346476}%
\pgfsetstrokecolor{currentstroke}%
\pgfsetdash{}{0pt}%
\pgfpathmoveto{\pgfqpoint{0.766095in}{0.816682in}}%
\pgfpathlineto{\pgfqpoint{0.801138in}{0.810282in}}%
\pgfpathlineto{\pgfqpoint{0.813991in}{0.807939in}}%
\pgfpathlineto{\pgfqpoint{0.836180in}{0.803964in}}%
\pgfpathlineto{\pgfqpoint{0.871223in}{0.797695in}}%
\pgfpathlineto{\pgfqpoint{0.906265in}{0.791426in}}%
\pgfpathlineto{\pgfqpoint{0.918129in}{0.789303in}}%
\pgfusepath{stroke}%
\end{pgfscope}%
\begin{pgfscope}%
\pgfpathrectangle{\pgfqpoint{0.766095in}{0.571603in}}{\pgfqpoint{6.973465in}{5.225635in}}%
\pgfusepath{clip}%
\pgfsetbuttcap%
\pgfsetroundjoin%
\pgfsetlinewidth{1.505625pt}%
\definecolor{currentstroke}{rgb}{0.430983,0.808473,0.346476}%
\pgfsetstrokecolor{currentstroke}%
\pgfsetdash{}{0pt}%
\pgfpathmoveto{\pgfqpoint{1.303165in}{0.722450in}}%
\pgfpathlineto{\pgfqpoint{1.326776in}{0.718470in}}%
\pgfpathlineto{\pgfqpoint{1.361818in}{0.712562in}}%
\pgfpathlineto{\pgfqpoint{1.396861in}{0.706651in}}%
\pgfpathlineto{\pgfqpoint{1.419104in}{0.702901in}}%
\pgfpathlineto{\pgfqpoint{1.431903in}{0.700781in}}%
\pgfpathlineto{\pgfqpoint{1.466946in}{0.694984in}}%
\pgfpathlineto{\pgfqpoint{1.501988in}{0.689185in}}%
\pgfpathlineto{\pgfqpoint{1.537031in}{0.683383in}}%
\pgfpathlineto{\pgfqpoint{1.572073in}{0.677578in}}%
\pgfpathlineto{\pgfqpoint{1.577738in}{0.676641in}}%
\pgfpathlineto{\pgfqpoint{1.607116in}{0.671868in}}%
\pgfpathlineto{\pgfqpoint{1.642158in}{0.666174in}}%
\pgfpathlineto{\pgfqpoint{1.677201in}{0.660476in}}%
\pgfpathlineto{\pgfqpoint{1.712244in}{0.654775in}}%
\pgfpathlineto{\pgfqpoint{1.739241in}{0.650382in}}%
\pgfpathlineto{\pgfqpoint{1.747286in}{0.649096in}}%
\pgfpathlineto{\pgfqpoint{1.782329in}{0.643502in}}%
\pgfpathlineto{\pgfqpoint{1.817371in}{0.637904in}}%
\pgfpathlineto{\pgfqpoint{1.852414in}{0.632302in}}%
\pgfpathlineto{\pgfqpoint{1.887456in}{0.626696in}}%
\pgfpathlineto{\pgfqpoint{1.903553in}{0.624122in}}%
\pgfpathlineto{\pgfqpoint{1.922499in}{0.621147in}}%
\pgfpathlineto{\pgfqpoint{1.957541in}{0.615645in}}%
\pgfpathlineto{\pgfqpoint{1.992584in}{0.610139in}}%
\pgfpathlineto{\pgfqpoint{2.027626in}{0.604627in}}%
\pgfpathlineto{\pgfqpoint{2.062669in}{0.599111in}}%
\pgfpathlineto{\pgfqpoint{2.070607in}{0.597863in}}%
\pgfpathlineto{\pgfqpoint{2.097711in}{0.593676in}}%
\pgfpathlineto{\pgfqpoint{2.132754in}{0.588262in}}%
\pgfpathlineto{\pgfqpoint{2.167797in}{0.582842in}}%
\pgfpathlineto{\pgfqpoint{2.202839in}{0.577417in}}%
\pgfpathlineto{\pgfqpoint{2.237882in}{0.571986in}}%
\pgfpathlineto{\pgfqpoint{2.240357in}{0.571603in}}%
\pgfusepath{stroke}%
\end{pgfscope}%
\begin{pgfscope}%
\pgfpathrectangle{\pgfqpoint{0.766095in}{0.571603in}}{\pgfqpoint{6.973465in}{5.225635in}}%
\pgfusepath{clip}%
\pgfsetbuttcap%
\pgfsetroundjoin%
\pgfsetlinewidth{1.505625pt}%
\definecolor{currentstroke}{rgb}{0.496615,0.826376,0.306377}%
\pgfsetstrokecolor{currentstroke}%
\pgfsetdash{}{0pt}%
\pgfpathmoveto{\pgfqpoint{0.766095in}{0.779978in}}%
\pgfpathlineto{\pgfqpoint{0.801138in}{0.773735in}}%
\pgfpathlineto{\pgfqpoint{0.836180in}{0.767492in}}%
\pgfpathlineto{\pgfqpoint{0.871223in}{0.761251in}}%
\pgfpathlineto{\pgfqpoint{0.903972in}{0.755420in}}%
\pgfpathlineto{\pgfqpoint{0.906265in}{0.755018in}}%
\pgfpathlineto{\pgfqpoint{0.923995in}{0.751924in}}%
\pgfusepath{stroke}%
\end{pgfscope}%
\begin{pgfscope}%
\pgfpathrectangle{\pgfqpoint{0.766095in}{0.571603in}}{\pgfqpoint{6.973465in}{5.225635in}}%
\pgfusepath{clip}%
\pgfsetbuttcap%
\pgfsetroundjoin%
\pgfsetlinewidth{1.505625pt}%
\definecolor{currentstroke}{rgb}{0.496615,0.826376,0.306377}%
\pgfsetstrokecolor{currentstroke}%
\pgfsetdash{}{0pt}%
\pgfpathmoveto{\pgfqpoint{1.309174in}{0.685910in}}%
\pgfpathlineto{\pgfqpoint{1.326776in}{0.682955in}}%
\pgfpathlineto{\pgfqpoint{1.361818in}{0.677070in}}%
\pgfpathlineto{\pgfqpoint{1.364379in}{0.676641in}}%
\pgfpathlineto{\pgfqpoint{1.396861in}{0.671293in}}%
\pgfpathlineto{\pgfqpoint{1.431903in}{0.665522in}}%
\pgfpathlineto{\pgfqpoint{1.466946in}{0.659749in}}%
\pgfpathlineto{\pgfqpoint{1.501988in}{0.653972in}}%
\pgfpathlineto{\pgfqpoint{1.523783in}{0.650382in}}%
\pgfpathlineto{\pgfqpoint{1.537031in}{0.648237in}}%
\pgfpathlineto{\pgfqpoint{1.572073in}{0.642571in}}%
\pgfpathlineto{\pgfqpoint{1.607116in}{0.636902in}}%
\pgfpathlineto{\pgfqpoint{1.642158in}{0.631230in}}%
\pgfpathlineto{\pgfqpoint{1.677201in}{0.625554in}}%
\pgfpathlineto{\pgfqpoint{1.686051in}{0.624122in}}%
\pgfpathlineto{\pgfqpoint{1.712244in}{0.619960in}}%
\pgfpathlineto{\pgfqpoint{1.747286in}{0.614391in}}%
\pgfpathlineto{\pgfqpoint{1.782329in}{0.608818in}}%
\pgfpathlineto{\pgfqpoint{1.817371in}{0.603241in}}%
\pgfpathlineto{\pgfqpoint{1.851143in}{0.597863in}}%
\pgfpathlineto{\pgfqpoint{1.852414in}{0.597664in}}%
\pgfpathlineto{\pgfqpoint{1.887456in}{0.592191in}}%
\pgfpathlineto{\pgfqpoint{1.922499in}{0.586714in}}%
\pgfpathlineto{\pgfqpoint{1.957541in}{0.581232in}}%
\pgfpathlineto{\pgfqpoint{1.992584in}{0.575746in}}%
\pgfpathlineto{\pgfqpoint{2.019036in}{0.571603in}}%
\pgfusepath{stroke}%
\end{pgfscope}%
\begin{pgfscope}%
\pgfpathrectangle{\pgfqpoint{0.766095in}{0.571603in}}{\pgfqpoint{6.973465in}{5.225635in}}%
\pgfusepath{clip}%
\pgfsetbuttcap%
\pgfsetroundjoin%
\pgfsetlinewidth{1.505625pt}%
\definecolor{currentstroke}{rgb}{0.565498,0.842430,0.262877}%
\pgfsetstrokecolor{currentstroke}%
\pgfsetdash{}{0pt}%
\pgfpathmoveto{\pgfqpoint{0.766095in}{0.744368in}}%
\pgfpathlineto{\pgfqpoint{0.801138in}{0.738155in}}%
\pgfpathlineto{\pgfqpoint{0.836180in}{0.731943in}}%
\pgfpathlineto{\pgfqpoint{0.851906in}{0.729160in}}%
\pgfpathlineto{\pgfqpoint{0.871223in}{0.725800in}}%
\pgfpathlineto{\pgfqpoint{0.906265in}{0.719713in}}%
\pgfpathlineto{\pgfqpoint{0.923939in}{0.716643in}}%
\pgfusepath{stroke}%
\end{pgfscope}%
\begin{pgfscope}%
\pgfpathrectangle{\pgfqpoint{0.766095in}{0.571603in}}{\pgfqpoint{6.973465in}{5.225635in}}%
\pgfusepath{clip}%
\pgfsetbuttcap%
\pgfsetroundjoin%
\pgfsetlinewidth{1.505625pt}%
\definecolor{currentstroke}{rgb}{0.565498,0.842430,0.262877}%
\pgfsetstrokecolor{currentstroke}%
\pgfsetdash{}{0pt}%
\pgfpathmoveto{\pgfqpoint{1.309226in}{0.651266in}}%
\pgfpathlineto{\pgfqpoint{1.314523in}{0.650382in}}%
\pgfpathlineto{\pgfqpoint{1.326776in}{0.648371in}}%
\pgfpathlineto{\pgfqpoint{1.361818in}{0.642630in}}%
\pgfpathlineto{\pgfqpoint{1.396861in}{0.636886in}}%
\pgfpathlineto{\pgfqpoint{1.431903in}{0.631140in}}%
\pgfpathlineto{\pgfqpoint{1.466946in}{0.625392in}}%
\pgfpathlineto{\pgfqpoint{1.474699in}{0.624122in}}%
\pgfpathlineto{\pgfqpoint{1.501988in}{0.619730in}}%
\pgfpathlineto{\pgfqpoint{1.537031in}{0.614091in}}%
\pgfpathlineto{\pgfqpoint{1.572073in}{0.608449in}}%
\pgfpathlineto{\pgfqpoint{1.607116in}{0.602804in}}%
\pgfpathlineto{\pgfqpoint{1.637783in}{0.597863in}}%
\pgfpathlineto{\pgfqpoint{1.642158in}{0.597170in}}%
\pgfpathlineto{\pgfqpoint{1.677201in}{0.591631in}}%
\pgfpathlineto{\pgfqpoint{1.712244in}{0.586088in}}%
\pgfpathlineto{\pgfqpoint{1.747286in}{0.580543in}}%
\pgfpathlineto{\pgfqpoint{1.782329in}{0.574993in}}%
\pgfpathlineto{\pgfqpoint{1.803737in}{0.571603in}}%
\pgfusepath{stroke}%
\end{pgfscope}%
\begin{pgfscope}%
\pgfpathrectangle{\pgfqpoint{0.766095in}{0.571603in}}{\pgfqpoint{6.973465in}{5.225635in}}%
\pgfusepath{clip}%
\pgfsetbuttcap%
\pgfsetroundjoin%
\pgfsetlinewidth{1.505625pt}%
\definecolor{currentstroke}{rgb}{0.636902,0.856542,0.216620}%
\pgfsetstrokecolor{currentstroke}%
\pgfsetdash{}{0pt}%
\pgfpathmoveto{\pgfqpoint{0.766095in}{0.709638in}}%
\pgfpathlineto{\pgfqpoint{0.801138in}{0.703457in}}%
\pgfpathlineto{\pgfqpoint{0.804302in}{0.702901in}}%
\pgfpathlineto{\pgfqpoint{0.836180in}{0.697388in}}%
\pgfpathlineto{\pgfqpoint{0.871223in}{0.691331in}}%
\pgfpathlineto{\pgfqpoint{0.906265in}{0.685275in}}%
\pgfpathlineto{\pgfqpoint{0.918612in}{0.683142in}}%
\pgfusepath{stroke}%
\end{pgfscope}%
\begin{pgfscope}%
\pgfpathrectangle{\pgfqpoint{0.766095in}{0.571603in}}{\pgfqpoint{6.973465in}{5.225635in}}%
\pgfusepath{clip}%
\pgfsetbuttcap%
\pgfsetroundjoin%
\pgfsetlinewidth{1.505625pt}%
\definecolor{currentstroke}{rgb}{0.636902,0.856542,0.216620}%
\pgfsetstrokecolor{currentstroke}%
\pgfsetdash{}{0pt}%
\pgfpathmoveto{\pgfqpoint{1.304014in}{0.618458in}}%
\pgfpathlineto{\pgfqpoint{1.326776in}{0.614748in}}%
\pgfpathlineto{\pgfqpoint{1.361818in}{0.609033in}}%
\pgfpathlineto{\pgfqpoint{1.396861in}{0.603317in}}%
\pgfpathlineto{\pgfqpoint{1.430285in}{0.597863in}}%
\pgfpathlineto{\pgfqpoint{1.431903in}{0.597603in}}%
\pgfpathlineto{\pgfqpoint{1.466946in}{0.591995in}}%
\pgfpathlineto{\pgfqpoint{1.501988in}{0.586385in}}%
\pgfpathlineto{\pgfqpoint{1.537031in}{0.580772in}}%
\pgfpathlineto{\pgfqpoint{1.572073in}{0.575156in}}%
\pgfpathlineto{\pgfqpoint{1.594249in}{0.571603in}}%
\pgfusepath{stroke}%
\end{pgfscope}%
\begin{pgfscope}%
\pgfpathrectangle{\pgfqpoint{0.766095in}{0.571603in}}{\pgfqpoint{6.973465in}{5.225635in}}%
\pgfusepath{clip}%
\pgfsetbuttcap%
\pgfsetroundjoin%
\pgfsetlinewidth{1.505625pt}%
\definecolor{currentstroke}{rgb}{0.709898,0.868751,0.169257}%
\pgfsetstrokecolor{currentstroke}%
\pgfsetdash{}{0pt}%
\pgfpathmoveto{\pgfqpoint{0.766095in}{0.675767in}}%
\pgfpathlineto{\pgfqpoint{0.801138in}{0.669741in}}%
\pgfpathlineto{\pgfqpoint{0.836180in}{0.663716in}}%
\pgfpathlineto{\pgfqpoint{0.871223in}{0.657692in}}%
\pgfpathlineto{\pgfqpoint{0.888805in}{0.654669in}}%
\pgfusepath{stroke}%
\end{pgfscope}%
\begin{pgfscope}%
\pgfpathrectangle{\pgfqpoint{0.766095in}{0.571603in}}{\pgfqpoint{6.973465in}{5.225635in}}%
\pgfusepath{clip}%
\pgfsetbuttcap%
\pgfsetroundjoin%
\pgfsetlinewidth{1.505625pt}%
\definecolor{currentstroke}{rgb}{0.709898,0.868751,0.169257}%
\pgfsetstrokecolor{currentstroke}%
\pgfsetdash{}{0pt}%
\pgfpathmoveto{\pgfqpoint{1.274282in}{0.590435in}}%
\pgfpathlineto{\pgfqpoint{1.291733in}{0.587606in}}%
\pgfpathlineto{\pgfqpoint{1.326776in}{0.581922in}}%
\pgfpathlineto{\pgfqpoint{1.361818in}{0.576236in}}%
\pgfpathlineto{\pgfqpoint{1.390378in}{0.571603in}}%
\pgfusepath{stroke}%
\end{pgfscope}%
\begin{pgfscope}%
\pgfpathrectangle{\pgfqpoint{0.766095in}{0.571603in}}{\pgfqpoint{6.973465in}{5.225635in}}%
\pgfusepath{clip}%
\pgfsetbuttcap%
\pgfsetroundjoin%
\pgfsetlinewidth{1.505625pt}%
\definecolor{currentstroke}{rgb}{0.783315,0.879285,0.125405}%
\pgfsetstrokecolor{currentstroke}%
\pgfsetdash{}{0pt}%
\pgfpathmoveto{\pgfqpoint{0.766095in}{0.642818in}}%
\pgfpathlineto{\pgfqpoint{0.801138in}{0.636826in}}%
\pgfpathlineto{\pgfqpoint{0.818714in}{0.633821in}}%
\pgfusepath{stroke}%
\end{pgfscope}%
\begin{pgfscope}%
\pgfpathrectangle{\pgfqpoint{0.766095in}{0.571603in}}{\pgfqpoint{6.973465in}{5.225635in}}%
\pgfusepath{clip}%
\pgfsetbuttcap%
\pgfsetroundjoin%
\pgfsetlinewidth{1.505625pt}%
\definecolor{currentstroke}{rgb}{0.855810,0.888601,0.097452}%
\pgfsetstrokecolor{currentstroke}%
\pgfsetdash{}{0pt}%
\pgfpathmoveto{\pgfqpoint{0.766095in}{0.610625in}}%
\pgfpathlineto{\pgfqpoint{0.801138in}{0.604669in}}%
\pgfpathlineto{\pgfqpoint{0.836180in}{0.598715in}}%
\pgfpathlineto{\pgfqpoint{0.841204in}{0.597863in}}%
\pgfpathlineto{\pgfqpoint{0.871223in}{0.592859in}}%
\pgfpathlineto{\pgfqpoint{0.906265in}{0.587021in}}%
\pgfpathlineto{\pgfqpoint{0.941308in}{0.581183in}}%
\pgfpathlineto{\pgfqpoint{0.976350in}{0.575346in}}%
\pgfpathlineto{\pgfqpoint{0.998836in}{0.571603in}}%
\pgfusepath{stroke}%
\end{pgfscope}%
\begin{pgfscope}%
\pgfpathrectangle{\pgfqpoint{0.766095in}{0.571603in}}{\pgfqpoint{6.973465in}{5.225635in}}%
\pgfusepath{clip}%
\pgfsetbuttcap%
\pgfsetroundjoin%
\pgfsetlinewidth{1.505625pt}%
\definecolor{currentstroke}{rgb}{0.926106,0.897330,0.104071}%
\pgfsetstrokecolor{currentstroke}%
\pgfsetdash{}{0pt}%
\pgfpathmoveto{\pgfqpoint{0.766095in}{0.579157in}}%
\pgfpathlineto{\pgfqpoint{0.801138in}{0.573238in}}%
\pgfpathlineto{\pgfqpoint{0.810836in}{0.571603in}}%
\pgfusepath{stroke}%
\end{pgfscope}%
\begin{pgfscope}%
\pgfpathrectangle{\pgfqpoint{0.766095in}{0.571603in}}{\pgfqpoint{6.973465in}{5.225635in}}%
\pgfusepath{clip}%
\pgfsetrectcap%
\pgfsetroundjoin%
\pgfsetlinewidth{1.505625pt}%
\definecolor{currentstroke}{rgb}{0.000000,0.000000,0.000000}%
\pgfsetstrokecolor{currentstroke}%
\pgfsetdash{}{0pt}%
\pgfpathmoveto{\pgfqpoint{6.743351in}{1.318123in}}%
\pgfpathlineto{\pgfqpoint{6.644238in}{1.923094in}}%
\pgfpathlineto{\pgfqpoint{3.017275in}{2.912699in}}%
\pgfpathlineto{\pgfqpoint{3.670589in}{3.077670in}}%
\pgfpathlineto{\pgfqpoint{3.539740in}{3.186610in}}%
\pgfusepath{stroke}%
\end{pgfscope}%
\begin{pgfscope}%
\pgfsetrectcap%
\pgfsetmiterjoin%
\pgfsetlinewidth{0.803000pt}%
\definecolor{currentstroke}{rgb}{0.000000,0.000000,0.000000}%
\pgfsetstrokecolor{currentstroke}%
\pgfsetdash{}{0pt}%
\pgfpathmoveto{\pgfqpoint{0.766095in}{0.571603in}}%
\pgfpathlineto{\pgfqpoint{0.766095in}{5.797238in}}%
\pgfusepath{stroke}%
\end{pgfscope}%
\begin{pgfscope}%
\pgfsetrectcap%
\pgfsetmiterjoin%
\pgfsetlinewidth{0.803000pt}%
\definecolor{currentstroke}{rgb}{0.000000,0.000000,0.000000}%
\pgfsetstrokecolor{currentstroke}%
\pgfsetdash{}{0pt}%
\pgfpathmoveto{\pgfqpoint{7.739560in}{0.571603in}}%
\pgfpathlineto{\pgfqpoint{7.739560in}{5.797238in}}%
\pgfusepath{stroke}%
\end{pgfscope}%
\begin{pgfscope}%
\pgfsetrectcap%
\pgfsetmiterjoin%
\pgfsetlinewidth{0.803000pt}%
\definecolor{currentstroke}{rgb}{0.000000,0.000000,0.000000}%
\pgfsetstrokecolor{currentstroke}%
\pgfsetdash{}{0pt}%
\pgfpathmoveto{\pgfqpoint{0.766095in}{0.571603in}}%
\pgfpathlineto{\pgfqpoint{7.739560in}{0.571603in}}%
\pgfusepath{stroke}%
\end{pgfscope}%
\begin{pgfscope}%
\pgfsetrectcap%
\pgfsetmiterjoin%
\pgfsetlinewidth{0.803000pt}%
\definecolor{currentstroke}{rgb}{0.000000,0.000000,0.000000}%
\pgfsetstrokecolor{currentstroke}%
\pgfsetdash{}{0pt}%
\pgfpathmoveto{\pgfqpoint{0.766095in}{5.797238in}}%
\pgfpathlineto{\pgfqpoint{7.739560in}{5.797238in}}%
\pgfusepath{stroke}%
\end{pgfscope}%
\begin{pgfscope}%
\definecolor{textcolor}{rgb}{0.276022,0.044167,0.370164}%
\pgfsetstrokecolor{textcolor}%
\pgfsetfillcolor{textcolor}%
\pgftext[x=7.105562in, y=2.419156in, left, base,rotate=335.103584]{\color{textcolor}\sffamily\fontsize{8.000000}{9.600000}\selectfont 3.0}%
\end{pgfscope}%
\begin{pgfscope}%
\definecolor{textcolor}{rgb}{0.281446,0.084320,0.407414}%
\pgfsetstrokecolor{textcolor}%
\pgfsetfillcolor{textcolor}%
\pgftext[x=2.254477in, y=5.586467in, left, base,rotate=15.548559]{\color{textcolor}\sffamily\fontsize{8.000000}{9.600000}\selectfont 4.5}%
\end{pgfscope}%
\begin{pgfscope}%
\definecolor{textcolor}{rgb}{0.281446,0.084320,0.407414}%
\pgfsetstrokecolor{textcolor}%
\pgfsetfillcolor{textcolor}%
\pgftext[x=6.456244in, y=1.527391in, left, base,rotate=351.640412]{\color{textcolor}\sffamily\fontsize{8.000000}{9.600000}\selectfont 4.5}%
\end{pgfscope}%
\begin{pgfscope}%
\definecolor{textcolor}{rgb}{0.281446,0.084320,0.407414}%
\pgfsetstrokecolor{textcolor}%
\pgfsetfillcolor{textcolor}%
\pgftext[x=7.471179in, y=2.727499in, left, base,rotate=335.911742]{\color{textcolor}\sffamily\fontsize{8.000000}{9.600000}\selectfont 4.5}%
\end{pgfscope}%
\begin{pgfscope}%
\definecolor{textcolor}{rgb}{0.283229,0.120777,0.440584}%
\pgfsetstrokecolor{textcolor}%
\pgfsetfillcolor{textcolor}%
\pgftext[x=1.568064in, y=5.680618in, left, base,rotate=18.465302]{\color{textcolor}\sffamily\fontsize{8.000000}{9.600000}\selectfont 6.0}%
\end{pgfscope}%
\begin{pgfscope}%
\definecolor{textcolor}{rgb}{0.283229,0.120777,0.440584}%
\pgfsetstrokecolor{textcolor}%
\pgfsetfillcolor{textcolor}%
\pgftext[x=7.471987in, y=3.102229in, left, base,rotate=331.841919]{\color{textcolor}\sffamily\fontsize{8.000000}{9.600000}\selectfont 6.0}%
\end{pgfscope}%
\begin{pgfscope}%
\definecolor{textcolor}{rgb}{0.283229,0.120777,0.440584}%
\pgfsetstrokecolor{textcolor}%
\pgfsetfillcolor{textcolor}%
\pgftext[x=7.227304in, y=1.243088in, left, base,rotate=352.037733]{\color{textcolor}\sffamily\fontsize{8.000000}{9.600000}\selectfont 6.0}%
\end{pgfscope}%
\begin{pgfscope}%
\definecolor{textcolor}{rgb}{0.281412,0.155834,0.469201}%
\pgfsetstrokecolor{textcolor}%
\pgfsetfillcolor{textcolor}%
\pgftext[x=0.870681in, y=5.668130in, left, base,rotate=21.821149]{\color{textcolor}\sffamily\fontsize{8.000000}{9.600000}\selectfont 7.5}%
\end{pgfscope}%
\begin{pgfscope}%
\definecolor{textcolor}{rgb}{0.281412,0.155834,0.469201}%
\pgfsetstrokecolor{textcolor}%
\pgfsetfillcolor{textcolor}%
\pgftext[x=7.417143in, y=5.477077in, left, base,rotate=44.126658]{\color{textcolor}\sffamily\fontsize{8.000000}{9.600000}\selectfont 7.5}%
\end{pgfscope}%
\begin{pgfscope}%
\definecolor{textcolor}{rgb}{0.281412,0.155834,0.469201}%
\pgfsetstrokecolor{textcolor}%
\pgfsetfillcolor{textcolor}%
\pgftext[x=6.841752in, y=1.166734in, left, base,rotate=351.769151]{\color{textcolor}\sffamily\fontsize{8.000000}{9.600000}\selectfont 7.5}%
\end{pgfscope}%
\begin{pgfscope}%
\definecolor{textcolor}{rgb}{0.276194,0.190074,0.493001}%
\pgfsetstrokecolor{textcolor}%
\pgfsetfillcolor{textcolor}%
\pgftext[x=7.580962in, y=4.918577in, left, base,rotate=57.216316]{\color{textcolor}\sffamily\fontsize{8.000000}{9.600000}\selectfont 9.0}%
\end{pgfscope}%
\begin{pgfscope}%
\definecolor{textcolor}{rgb}{0.276194,0.190074,0.493001}%
\pgfsetstrokecolor{textcolor}%
\pgfsetfillcolor{textcolor}%
\pgftext[x=7.429754in, y=0.973511in, left, base,rotate=352.115611]{\color{textcolor}\sffamily\fontsize{8.000000}{9.600000}\selectfont 9.0}%
\end{pgfscope}%
\begin{pgfscope}%
\definecolor{textcolor}{rgb}{0.267968,0.223549,0.512008}%
\pgfsetstrokecolor{textcolor}%
\pgfsetfillcolor{textcolor}%
\pgftext[x=6.211005in, y=1.055632in, left, base,rotate=351.579062]{\color{textcolor}\sffamily\fontsize{8.000000}{9.600000}\selectfont 10.5}%
\end{pgfscope}%
\begin{pgfscope}%
\definecolor{textcolor}{rgb}{0.257322,0.256130,0.526563}%
\pgfsetstrokecolor{textcolor}%
\pgfsetfillcolor{textcolor}%
\pgftext[x=6.982050in, y=0.859511in, left, base,rotate=352.067333]{\color{textcolor}\sffamily\fontsize{8.000000}{9.600000}\selectfont 12.0}%
\end{pgfscope}%
\begin{pgfscope}%
\definecolor{textcolor}{rgb}{0.244972,0.287675,0.537260}%
\pgfsetstrokecolor{textcolor}%
\pgfsetfillcolor{textcolor}%
\pgftext[x=5.615267in, y=0.982626in, left, base,rotate=351.506550]{\color{textcolor}\sffamily\fontsize{8.000000}{9.600000}\selectfont 13.5}%
\end{pgfscope}%
\begin{pgfscope}%
\definecolor{textcolor}{rgb}{0.229739,0.322361,0.545706}%
\pgfsetstrokecolor{textcolor}%
\pgfsetfillcolor{textcolor}%
\pgftext[x=4.949405in, y=1.012968in, left, base,rotate=351.247117]{\color{textcolor}\sffamily\fontsize{8.000000}{9.600000}\selectfont 15.0}%
\end{pgfscope}%
\begin{pgfscope}%
\definecolor{textcolor}{rgb}{0.216210,0.351535,0.550627}%
\pgfsetstrokecolor{textcolor}%
\pgfsetfillcolor{textcolor}%
\pgftext[x=6.386324in, y=0.734147in, left, base,rotate=352.055858]{\color{textcolor}\sffamily\fontsize{8.000000}{9.600000}\selectfont 16.5}%
\end{pgfscope}%
\begin{pgfscope}%
\definecolor{textcolor}{rgb}{0.203063,0.379716,0.553925}%
\pgfsetstrokecolor{textcolor}%
\pgfsetfillcolor{textcolor}%
\pgftext[x=4.316871in, y=0.980615in, left, base,rotate=351.096828]{\color{textcolor}\sffamily\fontsize{8.000000}{9.600000}\selectfont 18.0}%
\end{pgfscope}%
\begin{pgfscope}%
\definecolor{textcolor}{rgb}{0.190631,0.407061,0.556089}%
\pgfsetstrokecolor{textcolor}%
\pgfsetfillcolor{textcolor}%
\pgftext[x=3.617689in, y=1.032780in, left, base,rotate=350.721616]{\color{textcolor}\sffamily\fontsize{8.000000}{9.600000}\selectfont 19.5}%
\end{pgfscope}%
\begin{pgfscope}%
\definecolor{textcolor}{rgb}{0.179019,0.433756,0.557430}%
\pgfsetstrokecolor{textcolor}%
\pgfsetfillcolor{textcolor}%
\pgftext[x=5.755561in, y=0.651433in, left, base,rotate=352.065998]{\color{textcolor}\sffamily\fontsize{8.000000}{9.600000}\selectfont 21.0}%
\end{pgfscope}%
\begin{pgfscope}%
\definecolor{textcolor}{rgb}{0.168126,0.459988,0.558082}%
\pgfsetstrokecolor{textcolor}%
\pgfsetfillcolor{textcolor}%
\pgftext[x=2.951833in, y=1.028864in, left, base,rotate=350.449769]{\color{textcolor}\sffamily\fontsize{8.000000}{9.600000}\selectfont 22.5}%
\end{pgfscope}%
\begin{pgfscope}%
\definecolor{textcolor}{rgb}{0.157729,0.485932,0.558013}%
\pgfsetstrokecolor{textcolor}%
\pgfsetfillcolor{textcolor}%
\pgftext[x=5.089730in, y=0.644614in, left, base,rotate=351.967133]{\color{textcolor}\sffamily\fontsize{8.000000}{9.600000}\selectfont 24.0}%
\end{pgfscope}%
\begin{pgfscope}%
\definecolor{textcolor}{rgb}{0.147607,0.511733,0.557049}%
\pgfsetstrokecolor{textcolor}%
\pgfsetfillcolor{textcolor}%
\pgftext[x=2.293684in, y=1.035315in, left, base,rotate=350.113002]{\color{textcolor}\sffamily\fontsize{8.000000}{9.600000}\selectfont 25.5}%
\end{pgfscope}%
\begin{pgfscope}%
\definecolor{textcolor}{rgb}{0.137770,0.537492,0.554906}%
\pgfsetstrokecolor{textcolor}%
\pgfsetfillcolor{textcolor}%
\pgftext[x=1.620066in, y=1.106865in, left, base,rotate=349.446024]{\color{textcolor}\sffamily\fontsize{8.000000}{9.600000}\selectfont 27.0}%
\end{pgfscope}%
\begin{pgfscope}%
\definecolor{textcolor}{rgb}{0.127568,0.566949,0.550556}%
\pgfsetstrokecolor{textcolor}%
\pgfsetfillcolor{textcolor}%
\pgftext[x=4.388856in, y=0.607030in, left, base,rotate=351.866769]{\color{textcolor}\sffamily\fontsize{8.000000}{9.600000}\selectfont 28.5}%
\end{pgfscope}%
\begin{pgfscope}%
\definecolor{textcolor}{rgb}{0.121148,0.592739,0.544641}%
\pgfsetstrokecolor{textcolor}%
\pgfsetfillcolor{textcolor}%
\pgftext[x=3.702688in, y=0.664386in, left, base,rotate=351.600490]{\color{textcolor}\sffamily\fontsize{8.000000}{9.600000}\selectfont 30.0}%
\end{pgfscope}%
\begin{pgfscope}%
\definecolor{textcolor}{rgb}{0.119699,0.618490,0.536347}%
\pgfsetstrokecolor{textcolor}%
\pgfsetfillcolor{textcolor}%
\pgftext[x=0.954202in, y=1.086241in, left, base,rotate=348.986684]{\color{textcolor}\sffamily\fontsize{8.000000}{9.600000}\selectfont 31.5}%
\end{pgfscope}%
\begin{pgfscope}%
\definecolor{textcolor}{rgb}{0.126326,0.644107,0.525311}%
\pgfsetstrokecolor{textcolor}%
\pgfsetfillcolor{textcolor}%
\pgftext[x=3.022086in, y=0.683435in, left, base,rotate=351.349558]{\color{textcolor}\sffamily\fontsize{8.000000}{9.600000}\selectfont 33.0}%
\end{pgfscope}%
\begin{pgfscope}%
\definecolor{textcolor}{rgb}{0.143303,0.669459,0.511215}%
\pgfsetstrokecolor{textcolor}%
\pgfsetfillcolor{textcolor}%
\pgftext[x=3.050679in, y=0.638415in, left, base,rotate=351.470137]{\color{textcolor}\sffamily\fontsize{8.000000}{9.600000}\selectfont 34.5}%
\end{pgfscope}%
\begin{pgfscope}%
\definecolor{textcolor}{rgb}{0.170948,0.694384,0.493803}%
\pgfsetstrokecolor{textcolor}%
\pgfsetfillcolor{textcolor}%
\pgftext[x=2.356220in, y=0.707010in, left, base,rotate=351.055780]{\color{textcolor}\sffamily\fontsize{8.000000}{9.600000}\selectfont 36.0}%
\end{pgfscope}%
\begin{pgfscope}%
\definecolor{textcolor}{rgb}{0.208030,0.718701,0.472873}%
\pgfsetstrokecolor{textcolor}%
\pgfsetfillcolor{textcolor}%
\pgftext[x=2.356241in, y=0.667834in, left, base,rotate=351.163645]{\color{textcolor}\sffamily\fontsize{8.000000}{9.600000}\selectfont 37.5}%
\end{pgfscope}%
\begin{pgfscope}%
\definecolor{textcolor}{rgb}{0.252899,0.742211,0.448284}%
\pgfsetstrokecolor{textcolor}%
\pgfsetfillcolor{textcolor}%
\pgftext[x=1.655290in, y=0.743525in, left, base,rotate=350.624570]{\color{textcolor}\sffamily\fontsize{8.000000}{9.600000}\selectfont 39.0}%
\end{pgfscope}%
\begin{pgfscope}%
\definecolor{textcolor}{rgb}{0.304148,0.764704,0.419943}%
\pgfsetstrokecolor{textcolor}%
\pgfsetfillcolor{textcolor}%
\pgftext[x=1.655309in, y=0.705533in, left, base,rotate=350.736134]{\color{textcolor}\sffamily\fontsize{8.000000}{9.600000}\selectfont 40.5}%
\end{pgfscope}%
\begin{pgfscope}%
\definecolor{textcolor}{rgb}{0.369214,0.788888,0.382914}%
\pgfsetstrokecolor{textcolor}%
\pgfsetfillcolor{textcolor}%
\pgftext[x=0.638910in, y=0.846347in, left, base,rotate=349.705845]{\color{textcolor}\sffamily\fontsize{8.000000}{9.600000}\selectfont 42.0}%
\end{pgfscope}%
\begin{pgfscope}%
\definecolor{textcolor}{rgb}{0.430983,0.808473,0.346476}%
\pgfsetstrokecolor{textcolor}%
\pgfsetfillcolor{textcolor}%
\pgftext[x=0.983459in, y=0.746253in, left, base,rotate=350.206953]{\color{textcolor}\sffamily\fontsize{8.000000}{9.600000}\selectfont 43.5}%
\end{pgfscope}%
\begin{pgfscope}%
\definecolor{textcolor}{rgb}{0.496615,0.826376,0.306377}%
\pgfsetstrokecolor{textcolor}%
\pgfsetfillcolor{textcolor}%
\pgftext[x=0.989430in, y=0.709114in, left, base,rotate=350.323393]{\color{textcolor}\sffamily\fontsize{8.000000}{9.600000}\selectfont 45.0}%
\end{pgfscope}%
\begin{pgfscope}%
\definecolor{textcolor}{rgb}{0.565498,0.842430,0.262877}%
\pgfsetstrokecolor{textcolor}%
\pgfsetfillcolor{textcolor}%
\pgftext[x=0.989449in, y=0.673904in, left, base,rotate=350.437532]{\color{textcolor}\sffamily\fontsize{8.000000}{9.600000}\selectfont 46.5}%
\end{pgfscope}%
\begin{pgfscope}%
\definecolor{textcolor}{rgb}{0.636902,0.856542,0.216620}%
\pgfsetstrokecolor{textcolor}%
\pgfsetfillcolor{textcolor}%
\pgftext[x=0.984187in, y=0.640503in, left, base,rotate=350.527794]{\color{textcolor}\sffamily\fontsize{8.000000}{9.600000}\selectfont 48.0}%
\end{pgfscope}%
\begin{pgfscope}%
\definecolor{textcolor}{rgb}{0.709898,0.868751,0.169257}%
\pgfsetstrokecolor{textcolor}%
\pgfsetfillcolor{textcolor}%
\pgftext[x=0.954432in, y=0.612151in, left, base,rotate=350.585313]{\color{textcolor}\sffamily\fontsize{8.000000}{9.600000}\selectfont 49.5}%
\end{pgfscope}%
\begin{pgfscope}%
\definecolor{textcolor}{rgb}{0.783315,0.879285,0.125405}%
\pgfsetstrokecolor{textcolor}%
\pgfsetfillcolor{textcolor}%
\pgftext[x=0.884352in, y=0.591272in, left, base,rotate=350.614754]{\color{textcolor}\sffamily\fontsize{8.000000}{9.600000}\selectfont 51.0}%
\end{pgfscope}%
\end{pgfpicture}%
\makeatother%
\endgroup%
}
        \caption{Pohľad zhora (Vrstevnice)}
        \label{fig:newton_vlavo}
    \end{subfigure}
    \hfill
    % --- PRAVÝ OBRÁZOK ---
    \begin{subfigure}[b]{0.48\textwidth}
        \centering
        \resizebox{\linewidth}{!}{%% Creator: Matplotlib, PGF backend
%%
%% To include the figure in your LaTeX document, write
%%   \input{<filename>.pgf}
%%
%% Make sure the required packages are loaded in your preamble
%%   \usepackage{pgf}
%%
%% Also ensure that all the required font packages are loaded; for instance,
%% the lmodern package is sometimes necessary when using math font.
%%   \usepackage{lmodern}
%%
%% Figures using additional raster images can only be included by \input if
%% they are in the same directory as the main LaTeX file. For loading figures
%% from other directories you can use the `import` package
%%   \usepackage{import}
%%
%% and then include the figures with
%%   \import{<path to file>}{<filename>.pgf}
%%
%% Matplotlib used the following preamble
%%   
%%   \usepackage{fontspec}
%%   \setmainfont{DejaVuSerif.ttf}[Path=\detokenize{/home/radimek/Documents/projekt_mat_prog/mat_prog_kernel/lib/python3.12/site-packages/matplotlib/mpl-data/fonts/ttf/}]
%%   \setsansfont{DejaVuSans.ttf}[Path=\detokenize{/home/radimek/Documents/projekt_mat_prog/mat_prog_kernel/lib/python3.12/site-packages/matplotlib/mpl-data/fonts/ttf/}]
%%   \setmonofont{DejaVuSansMono.ttf}[Path=\detokenize{/home/radimek/Documents/projekt_mat_prog/mat_prog_kernel/lib/python3.12/site-packages/matplotlib/mpl-data/fonts/ttf/}]
%%   \makeatletter\@ifpackageloaded{underscore}{}{\usepackage[strings]{underscore}}\makeatother
%%
\begingroup%
\makeatletter%
\begin{pgfpicture}%
\pgfpathrectangle{\pgfpointorigin}{\pgfqpoint{8.000000in}{6.000000in}}%
\pgfusepath{use as bounding box, clip}%
\begin{pgfscope}%
\pgfsetbuttcap%
\pgfsetmiterjoin%
\definecolor{currentfill}{rgb}{1.000000,1.000000,1.000000}%
\pgfsetfillcolor{currentfill}%
\pgfsetlinewidth{0.000000pt}%
\definecolor{currentstroke}{rgb}{1.000000,1.000000,1.000000}%
\pgfsetstrokecolor{currentstroke}%
\pgfsetdash{}{0pt}%
\pgfpathmoveto{\pgfqpoint{0.000000in}{0.000000in}}%
\pgfpathlineto{\pgfqpoint{8.000000in}{0.000000in}}%
\pgfpathlineto{\pgfqpoint{8.000000in}{6.000000in}}%
\pgfpathlineto{\pgfqpoint{0.000000in}{6.000000in}}%
\pgfpathlineto{\pgfqpoint{0.000000in}{0.000000in}}%
\pgfpathclose%
\pgfusepath{fill}%
\end{pgfscope}%
\begin{pgfscope}%
\pgfsetbuttcap%
\pgfsetmiterjoin%
\definecolor{currentfill}{rgb}{1.000000,1.000000,1.000000}%
\pgfsetfillcolor{currentfill}%
\pgfsetlinewidth{0.000000pt}%
\definecolor{currentstroke}{rgb}{0.000000,0.000000,0.000000}%
\pgfsetstrokecolor{currentstroke}%
\pgfsetstrokeopacity{0.000000}%
\pgfsetdash{}{0pt}%
\pgfpathmoveto{\pgfqpoint{1.150000in}{0.150000in}}%
\pgfpathlineto{\pgfqpoint{6.850000in}{0.150000in}}%
\pgfpathlineto{\pgfqpoint{6.850000in}{5.850000in}}%
\pgfpathlineto{\pgfqpoint{1.150000in}{5.850000in}}%
\pgfpathlineto{\pgfqpoint{1.150000in}{0.150000in}}%
\pgfpathclose%
\pgfusepath{fill}%
\end{pgfscope}%
\begin{pgfscope}%
\pgfsetbuttcap%
\pgfsetmiterjoin%
\definecolor{currentfill}{rgb}{0.950000,0.950000,0.950000}%
\pgfsetfillcolor{currentfill}%
\pgfsetfillopacity{0.500000}%
\pgfsetlinewidth{1.003750pt}%
\definecolor{currentstroke}{rgb}{0.950000,0.950000,0.950000}%
\pgfsetstrokecolor{currentstroke}%
\pgfsetstrokeopacity{0.500000}%
\pgfsetdash{}{0pt}%
\pgfpathmoveto{\pgfqpoint{1.580389in}{1.555437in}}%
\pgfpathlineto{\pgfqpoint{3.462715in}{3.133240in}}%
\pgfpathlineto{\pgfqpoint{3.436549in}{5.408715in}}%
\pgfpathlineto{\pgfqpoint{1.464144in}{3.969343in}}%
\pgfusepath{stroke,fill}%
\end{pgfscope}%
\begin{pgfscope}%
\pgfsetbuttcap%
\pgfsetmiterjoin%
\definecolor{currentfill}{rgb}{0.900000,0.900000,0.900000}%
\pgfsetfillcolor{currentfill}%
\pgfsetfillopacity{0.500000}%
\pgfsetlinewidth{1.003750pt}%
\definecolor{currentstroke}{rgb}{0.900000,0.900000,0.900000}%
\pgfsetstrokecolor{currentstroke}%
\pgfsetstrokeopacity{0.500000}%
\pgfsetdash{}{0pt}%
\pgfpathmoveto{\pgfqpoint{3.462715in}{3.133240in}}%
\pgfpathlineto{\pgfqpoint{6.483177in}{2.255311in}}%
\pgfpathlineto{\pgfqpoint{6.590967in}{4.609162in}}%
\pgfpathlineto{\pgfqpoint{3.436549in}{5.408715in}}%
\pgfusepath{stroke,fill}%
\end{pgfscope}%
\begin{pgfscope}%
\pgfsetbuttcap%
\pgfsetmiterjoin%
\definecolor{currentfill}{rgb}{0.925000,0.925000,0.925000}%
\pgfsetfillcolor{currentfill}%
\pgfsetfillopacity{0.500000}%
\pgfsetlinewidth{1.003750pt}%
\definecolor{currentstroke}{rgb}{0.925000,0.925000,0.925000}%
\pgfsetstrokecolor{currentstroke}%
\pgfsetstrokeopacity{0.500000}%
\pgfsetdash{}{0pt}%
\pgfpathmoveto{\pgfqpoint{1.580389in}{1.555437in}}%
\pgfpathlineto{\pgfqpoint{4.782226in}{0.509717in}}%
\pgfpathlineto{\pgfqpoint{6.483177in}{2.255311in}}%
\pgfpathlineto{\pgfqpoint{3.462715in}{3.133240in}}%
\pgfusepath{stroke,fill}%
\end{pgfscope}%
\begin{pgfscope}%
\pgfsetrectcap%
\pgfsetroundjoin%
\pgfsetlinewidth{0.803000pt}%
\definecolor{currentstroke}{rgb}{0.000000,0.000000,0.000000}%
\pgfsetstrokecolor{currentstroke}%
\pgfsetdash{}{0pt}%
\pgfpathmoveto{\pgfqpoint{1.580389in}{1.555437in}}%
\pgfpathlineto{\pgfqpoint{4.782226in}{0.509717in}}%
\pgfusepath{stroke}%
\end{pgfscope}%
\begin{pgfscope}%
\definecolor{textcolor}{rgb}{0.000000,0.000000,0.000000}%
\pgfsetstrokecolor{textcolor}%
\pgfsetfillcolor{textcolor}%
\pgftext[x=2.913491in,y=0.557898in,,]{\color{textcolor}\sffamily\fontsize{10.000000}{12.000000}\selectfont x}%
\end{pgfscope}%
\begin{pgfscope}%
\pgfsetbuttcap%
\pgfsetroundjoin%
\pgfsetlinewidth{0.803000pt}%
\definecolor{currentstroke}{rgb}{0.690196,0.690196,0.690196}%
\pgfsetstrokecolor{currentstroke}%
\pgfsetdash{}{0pt}%
\pgfpathmoveto{\pgfqpoint{1.774309in}{1.492103in}}%
\pgfpathlineto{\pgfqpoint{3.646411in}{3.079847in}}%
\pgfpathlineto{\pgfqpoint{3.628011in}{5.360185in}}%
\pgfusepath{stroke}%
\end{pgfscope}%
\begin{pgfscope}%
\pgfsetbuttcap%
\pgfsetroundjoin%
\pgfsetlinewidth{0.803000pt}%
\definecolor{currentstroke}{rgb}{0.690196,0.690196,0.690196}%
\pgfsetstrokecolor{currentstroke}%
\pgfsetdash{}{0pt}%
\pgfpathmoveto{\pgfqpoint{2.157946in}{1.366807in}}%
\pgfpathlineto{\pgfqpoint{4.009529in}{2.974303in}}%
\pgfpathlineto{\pgfqpoint{4.006627in}{5.264216in}}%
\pgfusepath{stroke}%
\end{pgfscope}%
\begin{pgfscope}%
\pgfsetbuttcap%
\pgfsetroundjoin%
\pgfsetlinewidth{0.803000pt}%
\definecolor{currentstroke}{rgb}{0.690196,0.690196,0.690196}%
\pgfsetstrokecolor{currentstroke}%
\pgfsetdash{}{0pt}%
\pgfpathmoveto{\pgfqpoint{2.546576in}{1.239881in}}%
\pgfpathlineto{\pgfqpoint{4.376983in}{2.867499in}}%
\pgfpathlineto{\pgfqpoint{4.389959in}{5.167053in}}%
\pgfusepath{stroke}%
\end{pgfscope}%
\begin{pgfscope}%
\pgfsetbuttcap%
\pgfsetroundjoin%
\pgfsetlinewidth{0.803000pt}%
\definecolor{currentstroke}{rgb}{0.690196,0.690196,0.690196}%
\pgfsetstrokecolor{currentstroke}%
\pgfsetdash{}{0pt}%
\pgfpathmoveto{\pgfqpoint{2.940300in}{1.111291in}}%
\pgfpathlineto{\pgfqpoint{4.748850in}{2.759412in}}%
\pgfpathlineto{\pgfqpoint{4.778096in}{5.068672in}}%
\pgfusepath{stroke}%
\end{pgfscope}%
\begin{pgfscope}%
\pgfsetbuttcap%
\pgfsetroundjoin%
\pgfsetlinewidth{0.803000pt}%
\definecolor{currentstroke}{rgb}{0.690196,0.690196,0.690196}%
\pgfsetstrokecolor{currentstroke}%
\pgfsetdash{}{0pt}%
\pgfpathmoveto{\pgfqpoint{3.339217in}{0.981004in}}%
\pgfpathlineto{\pgfqpoint{5.125211in}{2.650019in}}%
\pgfpathlineto{\pgfqpoint{5.171128in}{4.969050in}}%
\pgfusepath{stroke}%
\end{pgfscope}%
\begin{pgfscope}%
\pgfsetbuttcap%
\pgfsetroundjoin%
\pgfsetlinewidth{0.803000pt}%
\definecolor{currentstroke}{rgb}{0.690196,0.690196,0.690196}%
\pgfsetstrokecolor{currentstroke}%
\pgfsetdash{}{0pt}%
\pgfpathmoveto{\pgfqpoint{3.743431in}{0.848988in}}%
\pgfpathlineto{\pgfqpoint{5.506147in}{2.539296in}}%
\pgfpathlineto{\pgfqpoint{5.569148in}{4.868163in}}%
\pgfusepath{stroke}%
\end{pgfscope}%
\begin{pgfscope}%
\pgfsetbuttcap%
\pgfsetroundjoin%
\pgfsetlinewidth{0.803000pt}%
\definecolor{currentstroke}{rgb}{0.690196,0.690196,0.690196}%
\pgfsetstrokecolor{currentstroke}%
\pgfsetdash{}{0pt}%
\pgfpathmoveto{\pgfqpoint{4.153048in}{0.715207in}}%
\pgfpathlineto{\pgfqpoint{5.891743in}{2.427218in}}%
\pgfpathlineto{\pgfqpoint{5.972253in}{4.765988in}}%
\pgfusepath{stroke}%
\end{pgfscope}%
\begin{pgfscope}%
\pgfsetbuttcap%
\pgfsetroundjoin%
\pgfsetlinewidth{0.803000pt}%
\definecolor{currentstroke}{rgb}{0.690196,0.690196,0.690196}%
\pgfsetstrokecolor{currentstroke}%
\pgfsetdash{}{0pt}%
\pgfpathmoveto{\pgfqpoint{4.568177in}{0.579626in}}%
\pgfpathlineto{\pgfqpoint{6.282083in}{2.313762in}}%
\pgfpathlineto{\pgfqpoint{6.380540in}{4.662499in}}%
\pgfusepath{stroke}%
\end{pgfscope}%
\begin{pgfscope}%
\pgfsetrectcap%
\pgfsetroundjoin%
\pgfsetlinewidth{0.803000pt}%
\definecolor{currentstroke}{rgb}{0.000000,0.000000,0.000000}%
\pgfsetstrokecolor{currentstroke}%
\pgfsetdash{}{0pt}%
\pgfpathmoveto{\pgfqpoint{1.790612in}{1.505929in}}%
\pgfpathlineto{\pgfqpoint{1.741635in}{1.464392in}}%
\pgfusepath{stroke}%
\end{pgfscope}%
\begin{pgfscope}%
\definecolor{textcolor}{rgb}{0.000000,0.000000,0.000000}%
\pgfsetstrokecolor{textcolor}%
\pgfsetfillcolor{textcolor}%
\pgftext[x=1.669876in,y=1.274184in,,top]{\color{textcolor}\sffamily\fontsize{10.000000}{12.000000}\selectfont \ensuremath{-}1.0}%
\end{pgfscope}%
\begin{pgfscope}%
\pgfsetrectcap%
\pgfsetroundjoin%
\pgfsetlinewidth{0.803000pt}%
\definecolor{currentstroke}{rgb}{0.000000,0.000000,0.000000}%
\pgfsetstrokecolor{currentstroke}%
\pgfsetdash{}{0pt}%
\pgfpathmoveto{\pgfqpoint{2.174078in}{1.380813in}}%
\pgfpathlineto{\pgfqpoint{2.125612in}{1.338736in}}%
\pgfusepath{stroke}%
\end{pgfscope}%
\begin{pgfscope}%
\definecolor{textcolor}{rgb}{0.000000,0.000000,0.000000}%
\pgfsetstrokecolor{textcolor}%
\pgfsetfillcolor{textcolor}%
\pgftext[x=2.053764in,y=1.147172in,,top]{\color{textcolor}\sffamily\fontsize{10.000000}{12.000000}\selectfont \ensuremath{-}0.5}%
\end{pgfscope}%
\begin{pgfscope}%
\pgfsetrectcap%
\pgfsetroundjoin%
\pgfsetlinewidth{0.803000pt}%
\definecolor{currentstroke}{rgb}{0.000000,0.000000,0.000000}%
\pgfsetstrokecolor{currentstroke}%
\pgfsetdash{}{0pt}%
\pgfpathmoveto{\pgfqpoint{2.562532in}{1.254069in}}%
\pgfpathlineto{\pgfqpoint{2.514595in}{1.211443in}}%
\pgfusepath{stroke}%
\end{pgfscope}%
\begin{pgfscope}%
\definecolor{textcolor}{rgb}{0.000000,0.000000,0.000000}%
\pgfsetstrokecolor{textcolor}%
\pgfsetfillcolor{textcolor}%
\pgftext[x=2.442663in,y=1.018502in,,top]{\color{textcolor}\sffamily\fontsize{10.000000}{12.000000}\selectfont 0.0}%
\end{pgfscope}%
\begin{pgfscope}%
\pgfsetrectcap%
\pgfsetroundjoin%
\pgfsetlinewidth{0.803000pt}%
\definecolor{currentstroke}{rgb}{0.000000,0.000000,0.000000}%
\pgfsetstrokecolor{currentstroke}%
\pgfsetdash{}{0pt}%
\pgfpathmoveto{\pgfqpoint{2.956074in}{1.125665in}}%
\pgfpathlineto{\pgfqpoint{2.908683in}{1.082478in}}%
\pgfusepath{stroke}%
\end{pgfscope}%
\begin{pgfscope}%
\definecolor{textcolor}{rgb}{0.000000,0.000000,0.000000}%
\pgfsetstrokecolor{textcolor}%
\pgfsetfillcolor{textcolor}%
\pgftext[x=2.836671in,y=0.888142in,,top]{\color{textcolor}\sffamily\fontsize{10.000000}{12.000000}\selectfont 0.5}%
\end{pgfscope}%
\begin{pgfscope}%
\pgfsetrectcap%
\pgfsetroundjoin%
\pgfsetlinewidth{0.803000pt}%
\definecolor{currentstroke}{rgb}{0.000000,0.000000,0.000000}%
\pgfsetstrokecolor{currentstroke}%
\pgfsetdash{}{0pt}%
\pgfpathmoveto{\pgfqpoint{3.354803in}{0.995569in}}%
\pgfpathlineto{\pgfqpoint{3.307977in}{0.951810in}}%
\pgfusepath{stroke}%
\end{pgfscope}%
\begin{pgfscope}%
\definecolor{textcolor}{rgb}{0.000000,0.000000,0.000000}%
\pgfsetstrokecolor{textcolor}%
\pgfsetfillcolor{textcolor}%
\pgftext[x=3.235888in,y=0.756059in,,top]{\color{textcolor}\sffamily\fontsize{10.000000}{12.000000}\selectfont 1.0}%
\end{pgfscope}%
\begin{pgfscope}%
\pgfsetrectcap%
\pgfsetroundjoin%
\pgfsetlinewidth{0.803000pt}%
\definecolor{currentstroke}{rgb}{0.000000,0.000000,0.000000}%
\pgfsetstrokecolor{currentstroke}%
\pgfsetdash{}{0pt}%
\pgfpathmoveto{\pgfqpoint{3.758822in}{0.863747in}}%
\pgfpathlineto{\pgfqpoint{3.712580in}{0.819405in}}%
\pgfusepath{stroke}%
\end{pgfscope}%
\begin{pgfscope}%
\definecolor{textcolor}{rgb}{0.000000,0.000000,0.000000}%
\pgfsetstrokecolor{textcolor}%
\pgfsetfillcolor{textcolor}%
\pgftext[x=3.640420in,y=0.622217in,,top]{\color{textcolor}\sffamily\fontsize{10.000000}{12.000000}\selectfont 1.5}%
\end{pgfscope}%
\begin{pgfscope}%
\pgfsetrectcap%
\pgfsetroundjoin%
\pgfsetlinewidth{0.803000pt}%
\definecolor{currentstroke}{rgb}{0.000000,0.000000,0.000000}%
\pgfsetstrokecolor{currentstroke}%
\pgfsetdash{}{0pt}%
\pgfpathmoveto{\pgfqpoint{4.168237in}{0.730164in}}%
\pgfpathlineto{\pgfqpoint{4.122600in}{0.685227in}}%
\pgfusepath{stroke}%
\end{pgfscope}%
\begin{pgfscope}%
\definecolor{textcolor}{rgb}{0.000000,0.000000,0.000000}%
\pgfsetstrokecolor{textcolor}%
\pgfsetfillcolor{textcolor}%
\pgftext[x=4.050372in,y=0.486582in,,top]{\color{textcolor}\sffamily\fontsize{10.000000}{12.000000}\selectfont 2.0}%
\end{pgfscope}%
\begin{pgfscope}%
\pgfsetrectcap%
\pgfsetroundjoin%
\pgfsetlinewidth{0.803000pt}%
\definecolor{currentstroke}{rgb}{0.000000,0.000000,0.000000}%
\pgfsetstrokecolor{currentstroke}%
\pgfsetdash{}{0pt}%
\pgfpathmoveto{\pgfqpoint{4.583158in}{0.594784in}}%
\pgfpathlineto{\pgfqpoint{4.538146in}{0.549241in}}%
\pgfusepath{stroke}%
\end{pgfscope}%
\begin{pgfscope}%
\definecolor{textcolor}{rgb}{0.000000,0.000000,0.000000}%
\pgfsetstrokecolor{textcolor}%
\pgfsetfillcolor{textcolor}%
\pgftext[x=4.465855in,y=0.349116in,,top]{\color{textcolor}\sffamily\fontsize{10.000000}{12.000000}\selectfont 2.5}%
\end{pgfscope}%
\begin{pgfscope}%
\pgfsetrectcap%
\pgfsetroundjoin%
\pgfsetlinewidth{0.803000pt}%
\definecolor{currentstroke}{rgb}{0.000000,0.000000,0.000000}%
\pgfsetstrokecolor{currentstroke}%
\pgfsetdash{}{0pt}%
\pgfpathmoveto{\pgfqpoint{6.483177in}{2.255311in}}%
\pgfpathlineto{\pgfqpoint{4.782226in}{0.509717in}}%
\pgfusepath{stroke}%
\end{pgfscope}%
\begin{pgfscope}%
\definecolor{textcolor}{rgb}{0.000000,0.000000,0.000000}%
\pgfsetstrokecolor{textcolor}%
\pgfsetfillcolor{textcolor}%
\pgftext[x=6.045209in,y=1.032725in,,]{\color{textcolor}\sffamily\fontsize{10.000000}{12.000000}\selectfont y}%
\end{pgfscope}%
\begin{pgfscope}%
\pgfsetbuttcap%
\pgfsetroundjoin%
\pgfsetlinewidth{0.803000pt}%
\definecolor{currentstroke}{rgb}{0.690196,0.690196,0.690196}%
\pgfsetstrokecolor{currentstroke}%
\pgfsetdash{}{0pt}%
\pgfpathmoveto{\pgfqpoint{1.600541in}{4.068879in}}%
\pgfpathlineto{\pgfqpoint{1.710097in}{1.664161in}}%
\pgfpathlineto{\pgfqpoint{4.899919in}{0.630499in}}%
\pgfusepath{stroke}%
\end{pgfscope}%
\begin{pgfscope}%
\pgfsetbuttcap%
\pgfsetroundjoin%
\pgfsetlinewidth{0.803000pt}%
\definecolor{currentstroke}{rgb}{0.690196,0.690196,0.690196}%
\pgfsetstrokecolor{currentstroke}%
\pgfsetdash{}{0pt}%
\pgfpathmoveto{\pgfqpoint{1.863162in}{4.260528in}}%
\pgfpathlineto{\pgfqpoint{1.960032in}{1.873661in}}%
\pgfpathlineto{\pgfqpoint{5.126498in}{0.863025in}}%
\pgfusepath{stroke}%
\end{pgfscope}%
\begin{pgfscope}%
\pgfsetbuttcap%
\pgfsetroundjoin%
\pgfsetlinewidth{0.803000pt}%
\definecolor{currentstroke}{rgb}{0.690196,0.690196,0.690196}%
\pgfsetstrokecolor{currentstroke}%
\pgfsetdash{}{0pt}%
\pgfpathmoveto{\pgfqpoint{2.119857in}{4.447853in}}%
\pgfpathlineto{\pgfqpoint{2.204571in}{2.078639in}}%
\pgfpathlineto{\pgfqpoint{5.347930in}{1.090268in}}%
\pgfusepath{stroke}%
\end{pgfscope}%
\begin{pgfscope}%
\pgfsetbuttcap%
\pgfsetroundjoin%
\pgfsetlinewidth{0.803000pt}%
\definecolor{currentstroke}{rgb}{0.690196,0.690196,0.690196}%
\pgfsetstrokecolor{currentstroke}%
\pgfsetdash{}{0pt}%
\pgfpathmoveto{\pgfqpoint{2.370826in}{4.630998in}}%
\pgfpathlineto{\pgfqpoint{2.443888in}{2.279239in}}%
\pgfpathlineto{\pgfqpoint{5.564386in}{1.312406in}}%
\pgfusepath{stroke}%
\end{pgfscope}%
\begin{pgfscope}%
\pgfsetbuttcap%
\pgfsetroundjoin%
\pgfsetlinewidth{0.803000pt}%
\definecolor{currentstroke}{rgb}{0.690196,0.690196,0.690196}%
\pgfsetstrokecolor{currentstroke}%
\pgfsetdash{}{0pt}%
\pgfpathmoveto{\pgfqpoint{2.616257in}{4.810103in}}%
\pgfpathlineto{\pgfqpoint{2.678149in}{2.475601in}}%
\pgfpathlineto{\pgfqpoint{5.776034in}{1.529608in}}%
\pgfusepath{stroke}%
\end{pgfscope}%
\begin{pgfscope}%
\pgfsetbuttcap%
\pgfsetroundjoin%
\pgfsetlinewidth{0.803000pt}%
\definecolor{currentstroke}{rgb}{0.690196,0.690196,0.690196}%
\pgfsetstrokecolor{currentstroke}%
\pgfsetdash{}{0pt}%
\pgfpathmoveto{\pgfqpoint{2.856332in}{4.985298in}}%
\pgfpathlineto{\pgfqpoint{2.907511in}{2.667857in}}%
\pgfpathlineto{\pgfqpoint{5.983031in}{1.742038in}}%
\pgfusepath{stroke}%
\end{pgfscope}%
\begin{pgfscope}%
\pgfsetbuttcap%
\pgfsetroundjoin%
\pgfsetlinewidth{0.803000pt}%
\definecolor{currentstroke}{rgb}{0.690196,0.690196,0.690196}%
\pgfsetstrokecolor{currentstroke}%
\pgfsetdash{}{0pt}%
\pgfpathmoveto{\pgfqpoint{3.091224in}{5.156712in}}%
\pgfpathlineto{\pgfqpoint{3.132127in}{2.856135in}}%
\pgfpathlineto{\pgfqpoint{6.185529in}{1.949852in}}%
\pgfusepath{stroke}%
\end{pgfscope}%
\begin{pgfscope}%
\pgfsetbuttcap%
\pgfsetroundjoin%
\pgfsetlinewidth{0.803000pt}%
\definecolor{currentstroke}{rgb}{0.690196,0.690196,0.690196}%
\pgfsetstrokecolor{currentstroke}%
\pgfsetdash{}{0pt}%
\pgfpathmoveto{\pgfqpoint{3.321099in}{5.324464in}}%
\pgfpathlineto{\pgfqpoint{3.352144in}{3.040557in}}%
\pgfpathlineto{\pgfqpoint{6.383674in}{2.153197in}}%
\pgfusepath{stroke}%
\end{pgfscope}%
\begin{pgfscope}%
\pgfsetrectcap%
\pgfsetroundjoin%
\pgfsetlinewidth{0.803000pt}%
\definecolor{currentstroke}{rgb}{0.000000,0.000000,0.000000}%
\pgfsetstrokecolor{currentstroke}%
\pgfsetdash{}{0pt}%
\pgfpathmoveto{\pgfqpoint{4.873038in}{0.639210in}}%
\pgfpathlineto{\pgfqpoint{4.953750in}{0.613055in}}%
\pgfusepath{stroke}%
\end{pgfscope}%
\begin{pgfscope}%
\definecolor{textcolor}{rgb}{0.000000,0.000000,0.000000}%
\pgfsetstrokecolor{textcolor}%
\pgfsetfillcolor{textcolor}%
\pgftext[x=5.078779in,y=0.444104in,,top]{\color{textcolor}\sffamily\fontsize{10.000000}{12.000000}\selectfont \ensuremath{-}2.5}%
\end{pgfscope}%
\begin{pgfscope}%
\pgfsetrectcap%
\pgfsetroundjoin%
\pgfsetlinewidth{0.803000pt}%
\definecolor{currentstroke}{rgb}{0.000000,0.000000,0.000000}%
\pgfsetstrokecolor{currentstroke}%
\pgfsetdash{}{0pt}%
\pgfpathmoveto{\pgfqpoint{5.099830in}{0.871537in}}%
\pgfpathlineto{\pgfqpoint{5.179903in}{0.845980in}}%
\pgfusepath{stroke}%
\end{pgfscope}%
\begin{pgfscope}%
\definecolor{textcolor}{rgb}{0.000000,0.000000,0.000000}%
\pgfsetstrokecolor{textcolor}%
\pgfsetfillcolor{textcolor}%
\pgftext[x=5.303389in,y=0.678862in,,top]{\color{textcolor}\sffamily\fontsize{10.000000}{12.000000}\selectfont \ensuremath{-}2.0}%
\end{pgfscope}%
\begin{pgfscope}%
\pgfsetrectcap%
\pgfsetroundjoin%
\pgfsetlinewidth{0.803000pt}%
\definecolor{currentstroke}{rgb}{0.000000,0.000000,0.000000}%
\pgfsetstrokecolor{currentstroke}%
\pgfsetdash{}{0pt}%
\pgfpathmoveto{\pgfqpoint{5.321471in}{1.098588in}}%
\pgfpathlineto{\pgfqpoint{5.400914in}{1.073608in}}%
\pgfusepath{stroke}%
\end{pgfscope}%
\begin{pgfscope}%
\definecolor{textcolor}{rgb}{0.000000,0.000000,0.000000}%
\pgfsetstrokecolor{textcolor}%
\pgfsetfillcolor{textcolor}%
\pgftext[x=5.522894in,y=0.908284in,,top]{\color{textcolor}\sffamily\fontsize{10.000000}{12.000000}\selectfont \ensuremath{-}1.5}%
\end{pgfscope}%
\begin{pgfscope}%
\pgfsetrectcap%
\pgfsetroundjoin%
\pgfsetlinewidth{0.803000pt}%
\definecolor{currentstroke}{rgb}{0.000000,0.000000,0.000000}%
\pgfsetstrokecolor{currentstroke}%
\pgfsetdash{}{0pt}%
\pgfpathmoveto{\pgfqpoint{5.538135in}{1.320539in}}%
\pgfpathlineto{\pgfqpoint{5.616955in}{1.296118in}}%
\pgfusepath{stroke}%
\end{pgfscope}%
\begin{pgfscope}%
\definecolor{textcolor}{rgb}{0.000000,0.000000,0.000000}%
\pgfsetstrokecolor{textcolor}%
\pgfsetfillcolor{textcolor}%
\pgftext[x=5.737466in,y=1.132550in,,top]{\color{textcolor}\sffamily\fontsize{10.000000}{12.000000}\selectfont \ensuremath{-}1.0}%
\end{pgfscope}%
\begin{pgfscope}%
\pgfsetrectcap%
\pgfsetroundjoin%
\pgfsetlinewidth{0.803000pt}%
\definecolor{currentstroke}{rgb}{0.000000,0.000000,0.000000}%
\pgfsetstrokecolor{currentstroke}%
\pgfsetdash{}{0pt}%
\pgfpathmoveto{\pgfqpoint{5.749987in}{1.537562in}}%
\pgfpathlineto{\pgfqpoint{5.828193in}{1.513681in}}%
\pgfusepath{stroke}%
\end{pgfscope}%
\begin{pgfscope}%
\definecolor{textcolor}{rgb}{0.000000,0.000000,0.000000}%
\pgfsetstrokecolor{textcolor}%
\pgfsetfillcolor{textcolor}%
\pgftext[x=5.947269in,y=1.351832in,,top]{\color{textcolor}\sffamily\fontsize{10.000000}{12.000000}\selectfont \ensuremath{-}0.5}%
\end{pgfscope}%
\begin{pgfscope}%
\pgfsetrectcap%
\pgfsetroundjoin%
\pgfsetlinewidth{0.803000pt}%
\definecolor{currentstroke}{rgb}{0.000000,0.000000,0.000000}%
\pgfsetstrokecolor{currentstroke}%
\pgfsetdash{}{0pt}%
\pgfpathmoveto{\pgfqpoint{5.957186in}{1.749819in}}%
\pgfpathlineto{\pgfqpoint{6.034785in}{1.726459in}}%
\pgfusepath{stroke}%
\end{pgfscope}%
\begin{pgfscope}%
\definecolor{textcolor}{rgb}{0.000000,0.000000,0.000000}%
\pgfsetstrokecolor{textcolor}%
\pgfsetfillcolor{textcolor}%
\pgftext[x=6.152460in,y=1.566293in,,top]{\color{textcolor}\sffamily\fontsize{10.000000}{12.000000}\selectfont 0.0}%
\end{pgfscope}%
\begin{pgfscope}%
\pgfsetrectcap%
\pgfsetroundjoin%
\pgfsetlinewidth{0.803000pt}%
\definecolor{currentstroke}{rgb}{0.000000,0.000000,0.000000}%
\pgfsetstrokecolor{currentstroke}%
\pgfsetdash{}{0pt}%
\pgfpathmoveto{\pgfqpoint{6.159883in}{1.957464in}}%
\pgfpathlineto{\pgfqpoint{6.236884in}{1.934609in}}%
\pgfusepath{stroke}%
\end{pgfscope}%
\begin{pgfscope}%
\definecolor{textcolor}{rgb}{0.000000,0.000000,0.000000}%
\pgfsetstrokecolor{textcolor}%
\pgfsetfillcolor{textcolor}%
\pgftext[x=6.353190in,y=1.776092in,,top]{\color{textcolor}\sffamily\fontsize{10.000000}{12.000000}\selectfont 0.5}%
\end{pgfscope}%
\begin{pgfscope}%
\pgfsetrectcap%
\pgfsetroundjoin%
\pgfsetlinewidth{0.803000pt}%
\definecolor{currentstroke}{rgb}{0.000000,0.000000,0.000000}%
\pgfsetstrokecolor{currentstroke}%
\pgfsetdash{}{0pt}%
\pgfpathmoveto{\pgfqpoint{6.358225in}{2.160646in}}%
\pgfpathlineto{\pgfqpoint{6.434634in}{2.138280in}}%
\pgfusepath{stroke}%
\end{pgfscope}%
\begin{pgfscope}%
\definecolor{textcolor}{rgb}{0.000000,0.000000,0.000000}%
\pgfsetstrokecolor{textcolor}%
\pgfsetfillcolor{textcolor}%
\pgftext[x=6.549603in,y=1.981379in,,top]{\color{textcolor}\sffamily\fontsize{10.000000}{12.000000}\selectfont 1.0}%
\end{pgfscope}%
\begin{pgfscope}%
\pgfsetrectcap%
\pgfsetroundjoin%
\pgfsetlinewidth{0.803000pt}%
\definecolor{currentstroke}{rgb}{0.000000,0.000000,0.000000}%
\pgfsetstrokecolor{currentstroke}%
\pgfsetdash{}{0pt}%
\pgfpathmoveto{\pgfqpoint{6.483177in}{2.255311in}}%
\pgfpathlineto{\pgfqpoint{6.590967in}{4.609162in}}%
\pgfusepath{stroke}%
\end{pgfscope}%
\begin{pgfscope}%
\definecolor{textcolor}{rgb}{0.000000,0.000000,0.000000}%
\pgfsetstrokecolor{textcolor}%
\pgfsetfillcolor{textcolor}%
\pgftext[x=7.097978in,y=3.481758in,,,rotate=87.378092]{\color{textcolor}\sffamily\fontsize{10.000000}{12.000000}\selectfont f(x,y)}%
\end{pgfscope}%
\begin{pgfscope}%
\pgfsetbuttcap%
\pgfsetroundjoin%
\pgfsetlinewidth{0.803000pt}%
\definecolor{currentstroke}{rgb}{0.690196,0.690196,0.690196}%
\pgfsetstrokecolor{currentstroke}%
\pgfsetdash{}{0pt}%
\pgfpathmoveto{\pgfqpoint{6.487073in}{2.340383in}}%
\pgfpathlineto{\pgfqpoint{3.461768in}{3.215647in}}%
\pgfpathlineto{\pgfqpoint{1.576195in}{1.642538in}}%
\pgfusepath{stroke}%
\end{pgfscope}%
\begin{pgfscope}%
\pgfsetbuttcap%
\pgfsetroundjoin%
\pgfsetlinewidth{0.803000pt}%
\definecolor{currentstroke}{rgb}{0.690196,0.690196,0.690196}%
\pgfsetstrokecolor{currentstroke}%
\pgfsetdash{}{0pt}%
\pgfpathmoveto{\pgfqpoint{6.504305in}{2.716689in}}%
\pgfpathlineto{\pgfqpoint{3.457578in}{3.580013in}}%
\pgfpathlineto{\pgfqpoint{1.557635in}{2.027945in}}%
\pgfusepath{stroke}%
\end{pgfscope}%
\begin{pgfscope}%
\pgfsetbuttcap%
\pgfsetroundjoin%
\pgfsetlinewidth{0.803000pt}%
\definecolor{currentstroke}{rgb}{0.690196,0.690196,0.690196}%
\pgfsetstrokecolor{currentstroke}%
\pgfsetdash{}{0pt}%
\pgfpathmoveto{\pgfqpoint{6.521785in}{3.098415in}}%
\pgfpathlineto{\pgfqpoint{3.453331in}{3.949376in}}%
\pgfpathlineto{\pgfqpoint{1.538798in}{2.419116in}}%
\pgfusepath{stroke}%
\end{pgfscope}%
\begin{pgfscope}%
\pgfsetbuttcap%
\pgfsetroundjoin%
\pgfsetlinewidth{0.803000pt}%
\definecolor{currentstroke}{rgb}{0.690196,0.690196,0.690196}%
\pgfsetstrokecolor{currentstroke}%
\pgfsetdash{}{0pt}%
\pgfpathmoveto{\pgfqpoint{6.539519in}{3.485679in}}%
\pgfpathlineto{\pgfqpoint{3.449025in}{4.323839in}}%
\pgfpathlineto{\pgfqpoint{1.519676in}{2.816181in}}%
\pgfusepath{stroke}%
\end{pgfscope}%
\begin{pgfscope}%
\pgfsetbuttcap%
\pgfsetroundjoin%
\pgfsetlinewidth{0.803000pt}%
\definecolor{currentstroke}{rgb}{0.690196,0.690196,0.690196}%
\pgfsetstrokecolor{currentstroke}%
\pgfsetdash{}{0pt}%
\pgfpathmoveto{\pgfqpoint{6.557513in}{3.878602in}}%
\pgfpathlineto{\pgfqpoint{3.444659in}{4.703508in}}%
\pgfpathlineto{\pgfqpoint{1.500265in}{3.219273in}}%
\pgfusepath{stroke}%
\end{pgfscope}%
\begin{pgfscope}%
\pgfsetbuttcap%
\pgfsetroundjoin%
\pgfsetlinewidth{0.803000pt}%
\definecolor{currentstroke}{rgb}{0.690196,0.690196,0.690196}%
\pgfsetstrokecolor{currentstroke}%
\pgfsetdash{}{0pt}%
\pgfpathmoveto{\pgfqpoint{6.575771in}{4.277310in}}%
\pgfpathlineto{\pgfqpoint{3.440232in}{5.088494in}}%
\pgfpathlineto{\pgfqpoint{1.480556in}{3.628531in}}%
\pgfusepath{stroke}%
\end{pgfscope}%
\begin{pgfscope}%
\pgfsetrectcap%
\pgfsetroundjoin%
\pgfsetlinewidth{0.803000pt}%
\definecolor{currentstroke}{rgb}{0.000000,0.000000,0.000000}%
\pgfsetstrokecolor{currentstroke}%
\pgfsetdash{}{0pt}%
\pgfpathmoveto{\pgfqpoint{6.461680in}{2.347730in}}%
\pgfpathlineto{\pgfqpoint{6.537919in}{2.325673in}}%
\pgfusepath{stroke}%
\end{pgfscope}%
\begin{pgfscope}%
\definecolor{textcolor}{rgb}{0.000000,0.000000,0.000000}%
\pgfsetstrokecolor{textcolor}%
\pgfsetfillcolor{textcolor}%
\pgftext[x=6.741196in,y=2.376494in,,top]{\color{textcolor}\sffamily\fontsize{10.000000}{12.000000}\selectfont 0}%
\end{pgfscope}%
\begin{pgfscope}%
\pgfsetrectcap%
\pgfsetroundjoin%
\pgfsetlinewidth{0.803000pt}%
\definecolor{currentstroke}{rgb}{0.000000,0.000000,0.000000}%
\pgfsetstrokecolor{currentstroke}%
\pgfsetdash{}{0pt}%
\pgfpathmoveto{\pgfqpoint{6.478724in}{2.723938in}}%
\pgfpathlineto{\pgfqpoint{6.555528in}{2.702174in}}%
\pgfusepath{stroke}%
\end{pgfscope}%
\begin{pgfscope}%
\definecolor{textcolor}{rgb}{0.000000,0.000000,0.000000}%
\pgfsetstrokecolor{textcolor}%
\pgfsetfillcolor{textcolor}%
\pgftext[x=6.760208in,y=2.752319in,,top]{\color{textcolor}\sffamily\fontsize{10.000000}{12.000000}\selectfont 10}%
\end{pgfscope}%
\begin{pgfscope}%
\pgfsetrectcap%
\pgfsetroundjoin%
\pgfsetlinewidth{0.803000pt}%
\definecolor{currentstroke}{rgb}{0.000000,0.000000,0.000000}%
\pgfsetstrokecolor{currentstroke}%
\pgfsetdash{}{0pt}%
\pgfpathmoveto{\pgfqpoint{6.496013in}{3.105562in}}%
\pgfpathlineto{\pgfqpoint{6.573392in}{3.084103in}}%
\pgfusepath{stroke}%
\end{pgfscope}%
\begin{pgfscope}%
\definecolor{textcolor}{rgb}{0.000000,0.000000,0.000000}%
\pgfsetstrokecolor{textcolor}%
\pgfsetfillcolor{textcolor}%
\pgftext[x=6.779493in,y=3.133547in,,top]{\color{textcolor}\sffamily\fontsize{10.000000}{12.000000}\selectfont 20}%
\end{pgfscope}%
\begin{pgfscope}%
\pgfsetrectcap%
\pgfsetroundjoin%
\pgfsetlinewidth{0.803000pt}%
\definecolor{currentstroke}{rgb}{0.000000,0.000000,0.000000}%
\pgfsetstrokecolor{currentstroke}%
\pgfsetdash{}{0pt}%
\pgfpathmoveto{\pgfqpoint{6.513553in}{3.492721in}}%
\pgfpathlineto{\pgfqpoint{6.591515in}{3.471578in}}%
\pgfusepath{stroke}%
\end{pgfscope}%
\begin{pgfscope}%
\definecolor{textcolor}{rgb}{0.000000,0.000000,0.000000}%
\pgfsetstrokecolor{textcolor}%
\pgfsetfillcolor{textcolor}%
\pgftext[x=6.799057in,y=3.520294in,,top]{\color{textcolor}\sffamily\fontsize{10.000000}{12.000000}\selectfont 30}%
\end{pgfscope}%
\begin{pgfscope}%
\pgfsetrectcap%
\pgfsetroundjoin%
\pgfsetlinewidth{0.803000pt}%
\definecolor{currentstroke}{rgb}{0.000000,0.000000,0.000000}%
\pgfsetstrokecolor{currentstroke}%
\pgfsetdash{}{0pt}%
\pgfpathmoveto{\pgfqpoint{6.531349in}{3.885536in}}%
\pgfpathlineto{\pgfqpoint{6.609903in}{3.864719in}}%
\pgfusepath{stroke}%
\end{pgfscope}%
\begin{pgfscope}%
\definecolor{textcolor}{rgb}{0.000000,0.000000,0.000000}%
\pgfsetstrokecolor{textcolor}%
\pgfsetfillcolor{textcolor}%
\pgftext[x=6.818907in,y=3.912681in,,top]{\color{textcolor}\sffamily\fontsize{10.000000}{12.000000}\selectfont 40}%
\end{pgfscope}%
\begin{pgfscope}%
\pgfsetrectcap%
\pgfsetroundjoin%
\pgfsetlinewidth{0.803000pt}%
\definecolor{currentstroke}{rgb}{0.000000,0.000000,0.000000}%
\pgfsetstrokecolor{currentstroke}%
\pgfsetdash{}{0pt}%
\pgfpathmoveto{\pgfqpoint{6.549407in}{4.284130in}}%
\pgfpathlineto{\pgfqpoint{6.628562in}{4.263653in}}%
\pgfusepath{stroke}%
\end{pgfscope}%
\begin{pgfscope}%
\definecolor{textcolor}{rgb}{0.000000,0.000000,0.000000}%
\pgfsetstrokecolor{textcolor}%
\pgfsetfillcolor{textcolor}%
\pgftext[x=6.839048in,y=4.310833in,,top]{\color{textcolor}\sffamily\fontsize{10.000000}{12.000000}\selectfont 50}%
\end{pgfscope}%
\begin{pgfscope}%
\pgfpathrectangle{\pgfqpoint{1.150000in}{0.150000in}}{\pgfqpoint{5.700000in}{5.700000in}}%
\pgfusepath{clip}%
\pgfsetrectcap%
\pgfsetroundjoin%
\pgfsetlinewidth{2.007500pt}%
\definecolor{currentstroke}{rgb}{0.000000,0.000000,0.000000}%
\pgfsetstrokecolor{currentstroke}%
\pgfsetdash{}{0pt}%
\pgfpathmoveto{\pgfqpoint{4.507492in}{1.394829in}}%
\pgfpathlineto{\pgfqpoint{4.650299in}{1.445511in}}%
\pgfpathlineto{\pgfqpoint{3.515596in}{2.143639in}}%
\pgfpathlineto{\pgfqpoint{3.818460in}{2.097460in}}%
\pgfpathlineto{\pgfqpoint{3.800874in}{2.142388in}}%
\pgfusepath{stroke}%
\end{pgfscope}%
\begin{pgfscope}%
\pgfpathrectangle{\pgfqpoint{1.150000in}{0.150000in}}{\pgfqpoint{5.700000in}{5.700000in}}%
\pgfusepath{clip}%
\pgfsetbuttcap%
\pgfsetroundjoin%
\definecolor{currentfill}{rgb}{1.000000,0.000000,0.000000}%
\pgfsetfillcolor{currentfill}%
\pgfsetfillopacity{0.300000}%
\pgfsetlinewidth{1.003750pt}%
\definecolor{currentstroke}{rgb}{1.000000,0.000000,0.000000}%
\pgfsetstrokecolor{currentstroke}%
\pgfsetstrokeopacity{0.300000}%
\pgfsetdash{}{0pt}%
\pgfpathmoveto{\pgfqpoint{3.800874in}{2.093283in}}%
\pgfpathcurveto{\pgfqpoint{3.813896in}{2.093283in}}{\pgfqpoint{3.826387in}{2.098457in}}{\pgfqpoint{3.835596in}{2.107666in}}%
\pgfpathcurveto{\pgfqpoint{3.844804in}{2.116874in}}{\pgfqpoint{3.849978in}{2.129365in}}{\pgfqpoint{3.849978in}{2.142388in}}%
\pgfpathcurveto{\pgfqpoint{3.849978in}{2.155411in}}{\pgfqpoint{3.844804in}{2.167902in}}{\pgfqpoint{3.835596in}{2.177110in}}%
\pgfpathcurveto{\pgfqpoint{3.826387in}{2.186319in}}{\pgfqpoint{3.813896in}{2.191493in}}{\pgfqpoint{3.800874in}{2.191493in}}%
\pgfpathcurveto{\pgfqpoint{3.787851in}{2.191493in}}{\pgfqpoint{3.775360in}{2.186319in}}{\pgfqpoint{3.766151in}{2.177110in}}%
\pgfpathcurveto{\pgfqpoint{3.756943in}{2.167902in}}{\pgfqpoint{3.751769in}{2.155411in}}{\pgfqpoint{3.751769in}{2.142388in}}%
\pgfpathcurveto{\pgfqpoint{3.751769in}{2.129365in}}{\pgfqpoint{3.756943in}{2.116874in}}{\pgfqpoint{3.766151in}{2.107666in}}%
\pgfpathcurveto{\pgfqpoint{3.775360in}{2.098457in}}{\pgfqpoint{3.787851in}{2.093283in}}{\pgfqpoint{3.800874in}{2.093283in}}%
\pgfpathlineto{\pgfqpoint{3.800874in}{2.093283in}}%
\pgfpathclose%
\pgfusepath{stroke,fill}%
\end{pgfscope}%
\begin{pgfscope}%
\pgfpathrectangle{\pgfqpoint{1.150000in}{0.150000in}}{\pgfqpoint{5.700000in}{5.700000in}}%
\pgfusepath{clip}%
\pgfsetbuttcap%
\pgfsetroundjoin%
\definecolor{currentfill}{rgb}{1.000000,0.000000,0.000000}%
\pgfsetfillcolor{currentfill}%
\pgfsetfillopacity{0.312703}%
\pgfsetlinewidth{1.003750pt}%
\definecolor{currentstroke}{rgb}{1.000000,0.000000,0.000000}%
\pgfsetstrokecolor{currentstroke}%
\pgfsetstrokeopacity{0.312703}%
\pgfsetdash{}{0pt}%
\pgfpathmoveto{\pgfqpoint{3.515596in}{2.094535in}}%
\pgfpathcurveto{\pgfqpoint{3.528618in}{2.094535in}}{\pgfqpoint{3.541110in}{2.099709in}}{\pgfqpoint{3.550318in}{2.108917in}}%
\pgfpathcurveto{\pgfqpoint{3.559526in}{2.118125in}}{\pgfqpoint{3.564700in}{2.130617in}}{\pgfqpoint{3.564700in}{2.143639in}}%
\pgfpathcurveto{\pgfqpoint{3.564700in}{2.156662in}}{\pgfqpoint{3.559526in}{2.169153in}}{\pgfqpoint{3.550318in}{2.178361in}}%
\pgfpathcurveto{\pgfqpoint{3.541110in}{2.187570in}}{\pgfqpoint{3.528618in}{2.192744in}}{\pgfqpoint{3.515596in}{2.192744in}}%
\pgfpathcurveto{\pgfqpoint{3.502573in}{2.192744in}}{\pgfqpoint{3.490082in}{2.187570in}}{\pgfqpoint{3.480874in}{2.178361in}}%
\pgfpathcurveto{\pgfqpoint{3.471665in}{2.169153in}}{\pgfqpoint{3.466491in}{2.156662in}}{\pgfqpoint{3.466491in}{2.143639in}}%
\pgfpathcurveto{\pgfqpoint{3.466491in}{2.130617in}}{\pgfqpoint{3.471665in}{2.118125in}}{\pgfqpoint{3.480874in}{2.108917in}}%
\pgfpathcurveto{\pgfqpoint{3.490082in}{2.099709in}}{\pgfqpoint{3.502573in}{2.094535in}}{\pgfqpoint{3.515596in}{2.094535in}}%
\pgfpathlineto{\pgfqpoint{3.515596in}{2.094535in}}%
\pgfpathclose%
\pgfusepath{stroke,fill}%
\end{pgfscope}%
\begin{pgfscope}%
\pgfpathrectangle{\pgfqpoint{1.150000in}{0.150000in}}{\pgfqpoint{5.700000in}{5.700000in}}%
\pgfusepath{clip}%
\pgfsetbuttcap%
\pgfsetroundjoin%
\definecolor{currentfill}{rgb}{1.000000,0.000000,0.000000}%
\pgfsetfillcolor{currentfill}%
\pgfsetfillopacity{0.331699}%
\pgfsetlinewidth{1.003750pt}%
\definecolor{currentstroke}{rgb}{1.000000,0.000000,0.000000}%
\pgfsetstrokecolor{currentstroke}%
\pgfsetstrokeopacity{0.331699}%
\pgfsetdash{}{0pt}%
\pgfpathmoveto{\pgfqpoint{3.818460in}{2.048355in}}%
\pgfpathcurveto{\pgfqpoint{3.831483in}{2.048355in}}{\pgfqpoint{3.843974in}{2.053529in}}{\pgfqpoint{3.853183in}{2.062738in}}%
\pgfpathcurveto{\pgfqpoint{3.862391in}{2.071946in}}{\pgfqpoint{3.867565in}{2.084437in}}{\pgfqpoint{3.867565in}{2.097460in}}%
\pgfpathcurveto{\pgfqpoint{3.867565in}{2.110483in}}{\pgfqpoint{3.862391in}{2.122974in}}{\pgfqpoint{3.853183in}{2.132182in}}%
\pgfpathcurveto{\pgfqpoint{3.843974in}{2.141391in}}{\pgfqpoint{3.831483in}{2.146565in}}{\pgfqpoint{3.818460in}{2.146565in}}%
\pgfpathcurveto{\pgfqpoint{3.805438in}{2.146565in}}{\pgfqpoint{3.792947in}{2.141391in}}{\pgfqpoint{3.783738in}{2.132182in}}%
\pgfpathcurveto{\pgfqpoint{3.774530in}{2.122974in}}{\pgfqpoint{3.769356in}{2.110483in}}{\pgfqpoint{3.769356in}{2.097460in}}%
\pgfpathcurveto{\pgfqpoint{3.769356in}{2.084437in}}{\pgfqpoint{3.774530in}{2.071946in}}{\pgfqpoint{3.783738in}{2.062738in}}%
\pgfpathcurveto{\pgfqpoint{3.792947in}{2.053529in}}{\pgfqpoint{3.805438in}{2.048355in}}{\pgfqpoint{3.818460in}{2.048355in}}%
\pgfpathlineto{\pgfqpoint{3.818460in}{2.048355in}}%
\pgfpathclose%
\pgfusepath{stroke,fill}%
\end{pgfscope}%
\begin{pgfscope}%
\pgfpathrectangle{\pgfqpoint{1.150000in}{0.150000in}}{\pgfqpoint{5.700000in}{5.700000in}}%
\pgfusepath{clip}%
\pgfsetbuttcap%
\pgfsetroundjoin%
\definecolor{currentfill}{rgb}{1.000000,0.000000,0.000000}%
\pgfsetfillcolor{currentfill}%
\pgfsetfillopacity{0.827267}%
\pgfsetlinewidth{1.003750pt}%
\definecolor{currentstroke}{rgb}{1.000000,0.000000,0.000000}%
\pgfsetstrokecolor{currentstroke}%
\pgfsetstrokeopacity{0.827267}%
\pgfsetdash{}{0pt}%
\pgfpathmoveto{\pgfqpoint{4.650299in}{1.396407in}}%
\pgfpathcurveto{\pgfqpoint{4.663322in}{1.396407in}}{\pgfqpoint{4.675813in}{1.401581in}}{\pgfqpoint{4.685021in}{1.410789in}}%
\pgfpathcurveto{\pgfqpoint{4.694230in}{1.419997in}}{\pgfqpoint{4.699404in}{1.432489in}}{\pgfqpoint{4.699404in}{1.445511in}}%
\pgfpathcurveto{\pgfqpoint{4.699404in}{1.458534in}}{\pgfqpoint{4.694230in}{1.471025in}}{\pgfqpoint{4.685021in}{1.480233in}}%
\pgfpathcurveto{\pgfqpoint{4.675813in}{1.489442in}}{\pgfqpoint{4.663322in}{1.494616in}}{\pgfqpoint{4.650299in}{1.494616in}}%
\pgfpathcurveto{\pgfqpoint{4.637276in}{1.494616in}}{\pgfqpoint{4.624785in}{1.489442in}}{\pgfqpoint{4.615577in}{1.480233in}}%
\pgfpathcurveto{\pgfqpoint{4.606368in}{1.471025in}}{\pgfqpoint{4.601194in}{1.458534in}}{\pgfqpoint{4.601194in}{1.445511in}}%
\pgfpathcurveto{\pgfqpoint{4.601194in}{1.432489in}}{\pgfqpoint{4.606368in}{1.419997in}}{\pgfqpoint{4.615577in}{1.410789in}}%
\pgfpathcurveto{\pgfqpoint{4.624785in}{1.401581in}}{\pgfqpoint{4.637276in}{1.396407in}}{\pgfqpoint{4.650299in}{1.396407in}}%
\pgfpathlineto{\pgfqpoint{4.650299in}{1.396407in}}%
\pgfpathclose%
\pgfusepath{stroke,fill}%
\end{pgfscope}%
\begin{pgfscope}%
\pgfpathrectangle{\pgfqpoint{1.150000in}{0.150000in}}{\pgfqpoint{5.700000in}{5.700000in}}%
\pgfusepath{clip}%
\pgfsetbuttcap%
\pgfsetroundjoin%
\definecolor{currentfill}{rgb}{1.000000,0.000000,0.000000}%
\pgfsetfillcolor{currentfill}%
\pgfsetlinewidth{1.003750pt}%
\definecolor{currentstroke}{rgb}{1.000000,0.000000,0.000000}%
\pgfsetstrokecolor{currentstroke}%
\pgfsetdash{}{0pt}%
\pgfpathmoveto{\pgfqpoint{4.507492in}{1.345725in}}%
\pgfpathcurveto{\pgfqpoint{4.520515in}{1.345725in}}{\pgfqpoint{4.533006in}{1.350899in}}{\pgfqpoint{4.542215in}{1.360107in}}%
\pgfpathcurveto{\pgfqpoint{4.551423in}{1.369315in}}{\pgfqpoint{4.556597in}{1.381807in}}{\pgfqpoint{4.556597in}{1.394829in}}%
\pgfpathcurveto{\pgfqpoint{4.556597in}{1.407852in}}{\pgfqpoint{4.551423in}{1.420343in}}{\pgfqpoint{4.542215in}{1.429551in}}%
\pgfpathcurveto{\pgfqpoint{4.533006in}{1.438760in}}{\pgfqpoint{4.520515in}{1.443934in}}{\pgfqpoint{4.507492in}{1.443934in}}%
\pgfpathcurveto{\pgfqpoint{4.494470in}{1.443934in}}{\pgfqpoint{4.481979in}{1.438760in}}{\pgfqpoint{4.472770in}{1.429551in}}%
\pgfpathcurveto{\pgfqpoint{4.463562in}{1.420343in}}{\pgfqpoint{4.458388in}{1.407852in}}{\pgfqpoint{4.458388in}{1.394829in}}%
\pgfpathcurveto{\pgfqpoint{4.458388in}{1.381807in}}{\pgfqpoint{4.463562in}{1.369315in}}{\pgfqpoint{4.472770in}{1.360107in}}%
\pgfpathcurveto{\pgfqpoint{4.481979in}{1.350899in}}{\pgfqpoint{4.494470in}{1.345725in}}{\pgfqpoint{4.507492in}{1.345725in}}%
\pgfpathlineto{\pgfqpoint{4.507492in}{1.345725in}}%
\pgfpathclose%
\pgfusepath{stroke,fill}%
\end{pgfscope}%
\begin{pgfscope}%
\pgfpathrectangle{\pgfqpoint{1.150000in}{0.150000in}}{\pgfqpoint{5.700000in}{5.700000in}}%
\pgfusepath{clip}%
\pgfsetbuttcap%
\pgfsetroundjoin%
\definecolor{currentfill}{rgb}{0.278826,0.175490,0.483397}%
\pgfsetfillcolor{currentfill}%
\pgfsetfillopacity{0.700000}%
\pgfsetlinewidth{0.000000pt}%
\definecolor{currentstroke}{rgb}{0.000000,0.000000,0.000000}%
\pgfsetstrokecolor{currentstroke}%
\pgfsetdash{}{0pt}%
\pgfpathmoveto{\pgfqpoint{3.532493in}{3.378249in}}%
\pgfpathlineto{\pgfqpoint{3.545367in}{3.370898in}}%
\pgfpathlineto{\pgfqpoint{3.558243in}{3.363598in}}%
\pgfpathlineto{\pgfqpoint{3.571122in}{3.356348in}}%
\pgfpathlineto{\pgfqpoint{3.584005in}{3.349149in}}%
\pgfpathlineto{\pgfqpoint{3.576405in}{3.336113in}}%
\pgfpathlineto{\pgfqpoint{3.568799in}{3.323276in}}%
\pgfpathlineto{\pgfqpoint{3.561187in}{3.310632in}}%
\pgfpathlineto{\pgfqpoint{3.553570in}{3.298176in}}%
\pgfpathlineto{\pgfqpoint{3.540681in}{3.305214in}}%
\pgfpathlineto{\pgfqpoint{3.527795in}{3.312303in}}%
\pgfpathlineto{\pgfqpoint{3.514912in}{3.319441in}}%
\pgfpathlineto{\pgfqpoint{3.502033in}{3.326631in}}%
\pgfpathlineto{\pgfqpoint{3.509656in}{3.339244in}}%
\pgfpathlineto{\pgfqpoint{3.517275in}{3.352047in}}%
\pgfpathlineto{\pgfqpoint{3.524887in}{3.365048in}}%
\pgfpathlineto{\pgfqpoint{3.532493in}{3.378249in}}%
\pgfpathclose%
\pgfusepath{fill}%
\end{pgfscope}%
\begin{pgfscope}%
\pgfpathrectangle{\pgfqpoint{1.150000in}{0.150000in}}{\pgfqpoint{5.700000in}{5.700000in}}%
\pgfusepath{clip}%
\pgfsetbuttcap%
\pgfsetroundjoin%
\definecolor{currentfill}{rgb}{0.280255,0.165693,0.476498}%
\pgfsetfillcolor{currentfill}%
\pgfsetfillopacity{0.700000}%
\pgfsetlinewidth{0.000000pt}%
\definecolor{currentstroke}{rgb}{0.000000,0.000000,0.000000}%
\pgfsetstrokecolor{currentstroke}%
\pgfsetdash{}{0pt}%
\pgfpathmoveto{\pgfqpoint{3.584005in}{3.349149in}}%
\pgfpathlineto{\pgfqpoint{3.596891in}{3.341999in}}%
\pgfpathlineto{\pgfqpoint{3.609781in}{3.334898in}}%
\pgfpathlineto{\pgfqpoint{3.622673in}{3.327846in}}%
\pgfpathlineto{\pgfqpoint{3.635569in}{3.320841in}}%
\pgfpathlineto{\pgfqpoint{3.627975in}{3.307972in}}%
\pgfpathlineto{\pgfqpoint{3.620376in}{3.295298in}}%
\pgfpathlineto{\pgfqpoint{3.612771in}{3.282813in}}%
\pgfpathlineto{\pgfqpoint{3.605161in}{3.270514in}}%
\pgfpathlineto{\pgfqpoint{3.592258in}{3.277357in}}%
\pgfpathlineto{\pgfqpoint{3.579359in}{3.284248in}}%
\pgfpathlineto{\pgfqpoint{3.566463in}{3.291188in}}%
\pgfpathlineto{\pgfqpoint{3.553570in}{3.298176in}}%
\pgfpathlineto{\pgfqpoint{3.561187in}{3.310632in}}%
\pgfpathlineto{\pgfqpoint{3.568799in}{3.323276in}}%
\pgfpathlineto{\pgfqpoint{3.576405in}{3.336113in}}%
\pgfpathlineto{\pgfqpoint{3.584005in}{3.349149in}}%
\pgfpathclose%
\pgfusepath{fill}%
\end{pgfscope}%
\begin{pgfscope}%
\pgfpathrectangle{\pgfqpoint{1.150000in}{0.150000in}}{\pgfqpoint{5.700000in}{5.700000in}}%
\pgfusepath{clip}%
\pgfsetbuttcap%
\pgfsetroundjoin%
\definecolor{currentfill}{rgb}{0.281412,0.155834,0.469201}%
\pgfsetfillcolor{currentfill}%
\pgfsetfillopacity{0.700000}%
\pgfsetlinewidth{0.000000pt}%
\definecolor{currentstroke}{rgb}{0.000000,0.000000,0.000000}%
\pgfsetstrokecolor{currentstroke}%
\pgfsetdash{}{0pt}%
\pgfpathmoveto{\pgfqpoint{3.502033in}{3.326631in}}%
\pgfpathlineto{\pgfqpoint{3.514912in}{3.319441in}}%
\pgfpathlineto{\pgfqpoint{3.527795in}{3.312303in}}%
\pgfpathlineto{\pgfqpoint{3.540681in}{3.305214in}}%
\pgfpathlineto{\pgfqpoint{3.553570in}{3.298176in}}%
\pgfpathlineto{\pgfqpoint{3.545947in}{3.285903in}}%
\pgfpathlineto{\pgfqpoint{3.538319in}{3.273809in}}%
\pgfpathlineto{\pgfqpoint{3.530684in}{3.261887in}}%
\pgfpathlineto{\pgfqpoint{3.523044in}{3.250134in}}%
\pgfpathlineto{\pgfqpoint{3.510147in}{3.257024in}}%
\pgfpathlineto{\pgfqpoint{3.497254in}{3.263964in}}%
\pgfpathlineto{\pgfqpoint{3.484364in}{3.270954in}}%
\pgfpathlineto{\pgfqpoint{3.471478in}{3.277995in}}%
\pgfpathlineto{\pgfqpoint{3.479125in}{3.289892in}}%
\pgfpathlineto{\pgfqpoint{3.486767in}{3.301960in}}%
\pgfpathlineto{\pgfqpoint{3.494403in}{3.314205in}}%
\pgfpathlineto{\pgfqpoint{3.502033in}{3.326631in}}%
\pgfpathclose%
\pgfusepath{fill}%
\end{pgfscope}%
\begin{pgfscope}%
\pgfpathrectangle{\pgfqpoint{1.150000in}{0.150000in}}{\pgfqpoint{5.700000in}{5.700000in}}%
\pgfusepath{clip}%
\pgfsetbuttcap%
\pgfsetroundjoin%
\definecolor{currentfill}{rgb}{0.281412,0.155834,0.469201}%
\pgfsetfillcolor{currentfill}%
\pgfsetfillopacity{0.700000}%
\pgfsetlinewidth{0.000000pt}%
\definecolor{currentstroke}{rgb}{0.000000,0.000000,0.000000}%
\pgfsetstrokecolor{currentstroke}%
\pgfsetdash{}{0pt}%
\pgfpathmoveto{\pgfqpoint{3.635569in}{3.320841in}}%
\pgfpathlineto{\pgfqpoint{3.648469in}{3.313885in}}%
\pgfpathlineto{\pgfqpoint{3.661372in}{3.306975in}}%
\pgfpathlineto{\pgfqpoint{3.674278in}{3.300112in}}%
\pgfpathlineto{\pgfqpoint{3.687189in}{3.293295in}}%
\pgfpathlineto{\pgfqpoint{3.679601in}{3.280592in}}%
\pgfpathlineto{\pgfqpoint{3.672009in}{3.268081in}}%
\pgfpathlineto{\pgfqpoint{3.664411in}{3.255756in}}%
\pgfpathlineto{\pgfqpoint{3.656808in}{3.243614in}}%
\pgfpathlineto{\pgfqpoint{3.643891in}{3.250269in}}%
\pgfpathlineto{\pgfqpoint{3.630977in}{3.256970in}}%
\pgfpathlineto{\pgfqpoint{3.618067in}{3.263719in}}%
\pgfpathlineto{\pgfqpoint{3.605161in}{3.270514in}}%
\pgfpathlineto{\pgfqpoint{3.612771in}{3.282813in}}%
\pgfpathlineto{\pgfqpoint{3.620376in}{3.295298in}}%
\pgfpathlineto{\pgfqpoint{3.627975in}{3.307972in}}%
\pgfpathlineto{\pgfqpoint{3.635569in}{3.320841in}}%
\pgfpathclose%
\pgfusepath{fill}%
\end{pgfscope}%
\begin{pgfscope}%
\pgfpathrectangle{\pgfqpoint{1.150000in}{0.150000in}}{\pgfqpoint{5.700000in}{5.700000in}}%
\pgfusepath{clip}%
\pgfsetbuttcap%
\pgfsetroundjoin%
\definecolor{currentfill}{rgb}{0.282290,0.145912,0.461510}%
\pgfsetfillcolor{currentfill}%
\pgfsetfillopacity{0.700000}%
\pgfsetlinewidth{0.000000pt}%
\definecolor{currentstroke}{rgb}{0.000000,0.000000,0.000000}%
\pgfsetstrokecolor{currentstroke}%
\pgfsetdash{}{0pt}%
\pgfpathmoveto{\pgfqpoint{3.553570in}{3.298176in}}%
\pgfpathlineto{\pgfqpoint{3.566463in}{3.291188in}}%
\pgfpathlineto{\pgfqpoint{3.579359in}{3.284248in}}%
\pgfpathlineto{\pgfqpoint{3.592258in}{3.277357in}}%
\pgfpathlineto{\pgfqpoint{3.605161in}{3.270514in}}%
\pgfpathlineto{\pgfqpoint{3.597545in}{3.258395in}}%
\pgfpathlineto{\pgfqpoint{3.589924in}{3.246450in}}%
\pgfpathlineto{\pgfqpoint{3.582297in}{3.234676in}}%
\pgfpathlineto{\pgfqpoint{3.574664in}{3.223066in}}%
\pgfpathlineto{\pgfqpoint{3.561754in}{3.229761in}}%
\pgfpathlineto{\pgfqpoint{3.548847in}{3.236503in}}%
\pgfpathlineto{\pgfqpoint{3.535944in}{3.243294in}}%
\pgfpathlineto{\pgfqpoint{3.523044in}{3.250134in}}%
\pgfpathlineto{\pgfqpoint{3.530684in}{3.261887in}}%
\pgfpathlineto{\pgfqpoint{3.538319in}{3.273809in}}%
\pgfpathlineto{\pgfqpoint{3.545947in}{3.285903in}}%
\pgfpathlineto{\pgfqpoint{3.553570in}{3.298176in}}%
\pgfpathclose%
\pgfusepath{fill}%
\end{pgfscope}%
\begin{pgfscope}%
\pgfpathrectangle{\pgfqpoint{1.150000in}{0.150000in}}{\pgfqpoint{5.700000in}{5.700000in}}%
\pgfusepath{clip}%
\pgfsetbuttcap%
\pgfsetroundjoin%
\definecolor{currentfill}{rgb}{0.282290,0.145912,0.461510}%
\pgfsetfillcolor{currentfill}%
\pgfsetfillopacity{0.700000}%
\pgfsetlinewidth{0.000000pt}%
\definecolor{currentstroke}{rgb}{0.000000,0.000000,0.000000}%
\pgfsetstrokecolor{currentstroke}%
\pgfsetdash{}{0pt}%
\pgfpathmoveto{\pgfqpoint{3.687189in}{3.293295in}}%
\pgfpathlineto{\pgfqpoint{3.700102in}{3.286525in}}%
\pgfpathlineto{\pgfqpoint{3.713020in}{3.279799in}}%
\pgfpathlineto{\pgfqpoint{3.725941in}{3.273118in}}%
\pgfpathlineto{\pgfqpoint{3.738865in}{3.266482in}}%
\pgfpathlineto{\pgfqpoint{3.731285in}{3.253945in}}%
\pgfpathlineto{\pgfqpoint{3.723699in}{3.241597in}}%
\pgfpathlineto{\pgfqpoint{3.716109in}{3.229432in}}%
\pgfpathlineto{\pgfqpoint{3.708513in}{3.217446in}}%
\pgfpathlineto{\pgfqpoint{3.695581in}{3.223921in}}%
\pgfpathlineto{\pgfqpoint{3.682653in}{3.230440in}}%
\pgfpathlineto{\pgfqpoint{3.669729in}{3.237004in}}%
\pgfpathlineto{\pgfqpoint{3.656808in}{3.243614in}}%
\pgfpathlineto{\pgfqpoint{3.664411in}{3.255756in}}%
\pgfpathlineto{\pgfqpoint{3.672009in}{3.268081in}}%
\pgfpathlineto{\pgfqpoint{3.679601in}{3.280592in}}%
\pgfpathlineto{\pgfqpoint{3.687189in}{3.293295in}}%
\pgfpathclose%
\pgfusepath{fill}%
\end{pgfscope}%
\begin{pgfscope}%
\pgfpathrectangle{\pgfqpoint{1.150000in}{0.150000in}}{\pgfqpoint{5.700000in}{5.700000in}}%
\pgfusepath{clip}%
\pgfsetbuttcap%
\pgfsetroundjoin%
\definecolor{currentfill}{rgb}{0.282623,0.140926,0.457517}%
\pgfsetfillcolor{currentfill}%
\pgfsetfillopacity{0.700000}%
\pgfsetlinewidth{0.000000pt}%
\definecolor{currentstroke}{rgb}{0.000000,0.000000,0.000000}%
\pgfsetstrokecolor{currentstroke}%
\pgfsetdash{}{0pt}%
\pgfpathmoveto{\pgfqpoint{3.471478in}{3.277995in}}%
\pgfpathlineto{\pgfqpoint{3.484364in}{3.270954in}}%
\pgfpathlineto{\pgfqpoint{3.497254in}{3.263964in}}%
\pgfpathlineto{\pgfqpoint{3.510147in}{3.257024in}}%
\pgfpathlineto{\pgfqpoint{3.523044in}{3.250134in}}%
\pgfpathlineto{\pgfqpoint{3.515398in}{3.238545in}}%
\pgfpathlineto{\pgfqpoint{3.507746in}{3.227116in}}%
\pgfpathlineto{\pgfqpoint{3.500088in}{3.215841in}}%
\pgfpathlineto{\pgfqpoint{3.492424in}{3.204718in}}%
\pgfpathlineto{\pgfqpoint{3.479520in}{3.211471in}}%
\pgfpathlineto{\pgfqpoint{3.466619in}{3.218275in}}%
\pgfpathlineto{\pgfqpoint{3.453721in}{3.225130in}}%
\pgfpathlineto{\pgfqpoint{3.440827in}{3.232035in}}%
\pgfpathlineto{\pgfqpoint{3.448499in}{3.243290in}}%
\pgfpathlineto{\pgfqpoint{3.456165in}{3.254699in}}%
\pgfpathlineto{\pgfqpoint{3.463824in}{3.266266in}}%
\pgfpathlineto{\pgfqpoint{3.471478in}{3.277995in}}%
\pgfpathclose%
\pgfusepath{fill}%
\end{pgfscope}%
\begin{pgfscope}%
\pgfpathrectangle{\pgfqpoint{1.150000in}{0.150000in}}{\pgfqpoint{5.700000in}{5.700000in}}%
\pgfusepath{clip}%
\pgfsetbuttcap%
\pgfsetroundjoin%
\definecolor{currentfill}{rgb}{0.282623,0.140926,0.457517}%
\pgfsetfillcolor{currentfill}%
\pgfsetfillopacity{0.700000}%
\pgfsetlinewidth{0.000000pt}%
\definecolor{currentstroke}{rgb}{0.000000,0.000000,0.000000}%
\pgfsetstrokecolor{currentstroke}%
\pgfsetdash{}{0pt}%
\pgfpathmoveto{\pgfqpoint{3.605161in}{3.270514in}}%
\pgfpathlineto{\pgfqpoint{3.618067in}{3.263719in}}%
\pgfpathlineto{\pgfqpoint{3.630977in}{3.256970in}}%
\pgfpathlineto{\pgfqpoint{3.643891in}{3.250269in}}%
\pgfpathlineto{\pgfqpoint{3.656808in}{3.243614in}}%
\pgfpathlineto{\pgfqpoint{3.649199in}{3.231647in}}%
\pgfpathlineto{\pgfqpoint{3.641585in}{3.219853in}}%
\pgfpathlineto{\pgfqpoint{3.633966in}{3.208226in}}%
\pgfpathlineto{\pgfqpoint{3.626341in}{3.196760in}}%
\pgfpathlineto{\pgfqpoint{3.613416in}{3.203267in}}%
\pgfpathlineto{\pgfqpoint{3.600495in}{3.209820in}}%
\pgfpathlineto{\pgfqpoint{3.587578in}{3.216420in}}%
\pgfpathlineto{\pgfqpoint{3.574664in}{3.223066in}}%
\pgfpathlineto{\pgfqpoint{3.582297in}{3.234676in}}%
\pgfpathlineto{\pgfqpoint{3.589924in}{3.246450in}}%
\pgfpathlineto{\pgfqpoint{3.597545in}{3.258395in}}%
\pgfpathlineto{\pgfqpoint{3.605161in}{3.270514in}}%
\pgfpathclose%
\pgfusepath{fill}%
\end{pgfscope}%
\begin{pgfscope}%
\pgfpathrectangle{\pgfqpoint{1.150000in}{0.150000in}}{\pgfqpoint{5.700000in}{5.700000in}}%
\pgfusepath{clip}%
\pgfsetbuttcap%
\pgfsetroundjoin%
\definecolor{currentfill}{rgb}{0.282623,0.140926,0.457517}%
\pgfsetfillcolor{currentfill}%
\pgfsetfillopacity{0.700000}%
\pgfsetlinewidth{0.000000pt}%
\definecolor{currentstroke}{rgb}{0.000000,0.000000,0.000000}%
\pgfsetstrokecolor{currentstroke}%
\pgfsetdash{}{0pt}%
\pgfpathmoveto{\pgfqpoint{3.738865in}{3.266482in}}%
\pgfpathlineto{\pgfqpoint{3.751794in}{3.259890in}}%
\pgfpathlineto{\pgfqpoint{3.764726in}{3.253342in}}%
\pgfpathlineto{\pgfqpoint{3.777662in}{3.246836in}}%
\pgfpathlineto{\pgfqpoint{3.790602in}{3.240374in}}%
\pgfpathlineto{\pgfqpoint{3.783029in}{3.228003in}}%
\pgfpathlineto{\pgfqpoint{3.775451in}{3.215818in}}%
\pgfpathlineto{\pgfqpoint{3.767868in}{3.203813in}}%
\pgfpathlineto{\pgfqpoint{3.760279in}{3.191984in}}%
\pgfpathlineto{\pgfqpoint{3.747332in}{3.198284in}}%
\pgfpathlineto{\pgfqpoint{3.734389in}{3.204628in}}%
\pgfpathlineto{\pgfqpoint{3.721449in}{3.211015in}}%
\pgfpathlineto{\pgfqpoint{3.708513in}{3.217446in}}%
\pgfpathlineto{\pgfqpoint{3.716109in}{3.229432in}}%
\pgfpathlineto{\pgfqpoint{3.723699in}{3.241597in}}%
\pgfpathlineto{\pgfqpoint{3.731285in}{3.253945in}}%
\pgfpathlineto{\pgfqpoint{3.738865in}{3.266482in}}%
\pgfpathclose%
\pgfusepath{fill}%
\end{pgfscope}%
\begin{pgfscope}%
\pgfpathrectangle{\pgfqpoint{1.150000in}{0.150000in}}{\pgfqpoint{5.700000in}{5.700000in}}%
\pgfusepath{clip}%
\pgfsetbuttcap%
\pgfsetroundjoin%
\definecolor{currentfill}{rgb}{0.282884,0.135920,0.453427}%
\pgfsetfillcolor{currentfill}%
\pgfsetfillopacity{0.700000}%
\pgfsetlinewidth{0.000000pt}%
\definecolor{currentstroke}{rgb}{0.000000,0.000000,0.000000}%
\pgfsetstrokecolor{currentstroke}%
\pgfsetdash{}{0pt}%
\pgfpathmoveto{\pgfqpoint{3.523044in}{3.250134in}}%
\pgfpathlineto{\pgfqpoint{3.535944in}{3.243294in}}%
\pgfpathlineto{\pgfqpoint{3.548847in}{3.236503in}}%
\pgfpathlineto{\pgfqpoint{3.561754in}{3.229761in}}%
\pgfpathlineto{\pgfqpoint{3.574664in}{3.223066in}}%
\pgfpathlineto{\pgfqpoint{3.567025in}{3.211618in}}%
\pgfpathlineto{\pgfqpoint{3.559381in}{3.200326in}}%
\pgfpathlineto{\pgfqpoint{3.551731in}{3.189186in}}%
\pgfpathlineto{\pgfqpoint{3.544076in}{3.178193in}}%
\pgfpathlineto{\pgfqpoint{3.531157in}{3.184752in}}%
\pgfpathlineto{\pgfqpoint{3.518243in}{3.191358in}}%
\pgfpathlineto{\pgfqpoint{3.505332in}{3.198013in}}%
\pgfpathlineto{\pgfqpoint{3.492424in}{3.204718in}}%
\pgfpathlineto{\pgfqpoint{3.500088in}{3.215841in}}%
\pgfpathlineto{\pgfqpoint{3.507746in}{3.227116in}}%
\pgfpathlineto{\pgfqpoint{3.515398in}{3.238545in}}%
\pgfpathlineto{\pgfqpoint{3.523044in}{3.250134in}}%
\pgfpathclose%
\pgfusepath{fill}%
\end{pgfscope}%
\begin{pgfscope}%
\pgfpathrectangle{\pgfqpoint{1.150000in}{0.150000in}}{\pgfqpoint{5.700000in}{5.700000in}}%
\pgfusepath{clip}%
\pgfsetbuttcap%
\pgfsetroundjoin%
\definecolor{currentfill}{rgb}{0.283072,0.130895,0.449241}%
\pgfsetfillcolor{currentfill}%
\pgfsetfillopacity{0.700000}%
\pgfsetlinewidth{0.000000pt}%
\definecolor{currentstroke}{rgb}{0.000000,0.000000,0.000000}%
\pgfsetstrokecolor{currentstroke}%
\pgfsetdash{}{0pt}%
\pgfpathmoveto{\pgfqpoint{3.656808in}{3.243614in}}%
\pgfpathlineto{\pgfqpoint{3.669729in}{3.237004in}}%
\pgfpathlineto{\pgfqpoint{3.682653in}{3.230440in}}%
\pgfpathlineto{\pgfqpoint{3.695581in}{3.223921in}}%
\pgfpathlineto{\pgfqpoint{3.708513in}{3.217446in}}%
\pgfpathlineto{\pgfqpoint{3.700912in}{3.205633in}}%
\pgfpathlineto{\pgfqpoint{3.693306in}{3.193989in}}%
\pgfpathlineto{\pgfqpoint{3.685694in}{3.182509in}}%
\pgfpathlineto{\pgfqpoint{3.678077in}{3.171187in}}%
\pgfpathlineto{\pgfqpoint{3.665137in}{3.177513in}}%
\pgfpathlineto{\pgfqpoint{3.652201in}{3.183884in}}%
\pgfpathlineto{\pgfqpoint{3.639269in}{3.190300in}}%
\pgfpathlineto{\pgfqpoint{3.626341in}{3.196760in}}%
\pgfpathlineto{\pgfqpoint{3.633966in}{3.208226in}}%
\pgfpathlineto{\pgfqpoint{3.641585in}{3.219853in}}%
\pgfpathlineto{\pgfqpoint{3.649199in}{3.231647in}}%
\pgfpathlineto{\pgfqpoint{3.656808in}{3.243614in}}%
\pgfpathclose%
\pgfusepath{fill}%
\end{pgfscope}%
\begin{pgfscope}%
\pgfpathrectangle{\pgfqpoint{1.150000in}{0.150000in}}{\pgfqpoint{5.700000in}{5.700000in}}%
\pgfusepath{clip}%
\pgfsetbuttcap%
\pgfsetroundjoin%
\definecolor{currentfill}{rgb}{0.283072,0.130895,0.449241}%
\pgfsetfillcolor{currentfill}%
\pgfsetfillopacity{0.700000}%
\pgfsetlinewidth{0.000000pt}%
\definecolor{currentstroke}{rgb}{0.000000,0.000000,0.000000}%
\pgfsetstrokecolor{currentstroke}%
\pgfsetdash{}{0pt}%
\pgfpathmoveto{\pgfqpoint{3.790602in}{3.240374in}}%
\pgfpathlineto{\pgfqpoint{3.803546in}{3.233954in}}%
\pgfpathlineto{\pgfqpoint{3.816494in}{3.227577in}}%
\pgfpathlineto{\pgfqpoint{3.829445in}{3.221241in}}%
\pgfpathlineto{\pgfqpoint{3.842401in}{3.214947in}}%
\pgfpathlineto{\pgfqpoint{3.834835in}{3.202742in}}%
\pgfpathlineto{\pgfqpoint{3.827265in}{3.190720in}}%
\pgfpathlineto{\pgfqpoint{3.819689in}{3.178875in}}%
\pgfpathlineto{\pgfqpoint{3.812109in}{3.167202in}}%
\pgfpathlineto{\pgfqpoint{3.799146in}{3.173335in}}%
\pgfpathlineto{\pgfqpoint{3.786186in}{3.179509in}}%
\pgfpathlineto{\pgfqpoint{3.773231in}{3.185725in}}%
\pgfpathlineto{\pgfqpoint{3.760279in}{3.191984in}}%
\pgfpathlineto{\pgfqpoint{3.767868in}{3.203813in}}%
\pgfpathlineto{\pgfqpoint{3.775451in}{3.215818in}}%
\pgfpathlineto{\pgfqpoint{3.783029in}{3.228003in}}%
\pgfpathlineto{\pgfqpoint{3.790602in}{3.240374in}}%
\pgfpathclose%
\pgfusepath{fill}%
\end{pgfscope}%
\begin{pgfscope}%
\pgfpathrectangle{\pgfqpoint{1.150000in}{0.150000in}}{\pgfqpoint{5.700000in}{5.700000in}}%
\pgfusepath{clip}%
\pgfsetbuttcap%
\pgfsetroundjoin%
\definecolor{currentfill}{rgb}{0.283072,0.130895,0.449241}%
\pgfsetfillcolor{currentfill}%
\pgfsetfillopacity{0.700000}%
\pgfsetlinewidth{0.000000pt}%
\definecolor{currentstroke}{rgb}{0.000000,0.000000,0.000000}%
\pgfsetstrokecolor{currentstroke}%
\pgfsetdash{}{0pt}%
\pgfpathmoveto{\pgfqpoint{3.440827in}{3.232035in}}%
\pgfpathlineto{\pgfqpoint{3.453721in}{3.225130in}}%
\pgfpathlineto{\pgfqpoint{3.466619in}{3.218275in}}%
\pgfpathlineto{\pgfqpoint{3.479520in}{3.211471in}}%
\pgfpathlineto{\pgfqpoint{3.492424in}{3.204718in}}%
\pgfpathlineto{\pgfqpoint{3.484754in}{3.193740in}}%
\pgfpathlineto{\pgfqpoint{3.477078in}{3.182905in}}%
\pgfpathlineto{\pgfqpoint{3.469397in}{3.172208in}}%
\pgfpathlineto{\pgfqpoint{3.461709in}{3.161644in}}%
\pgfpathlineto{\pgfqpoint{3.448796in}{3.168275in}}%
\pgfpathlineto{\pgfqpoint{3.435887in}{3.174956in}}%
\pgfpathlineto{\pgfqpoint{3.422981in}{3.181687in}}%
\pgfpathlineto{\pgfqpoint{3.410079in}{3.188470in}}%
\pgfpathlineto{\pgfqpoint{3.417775in}{3.199151in}}%
\pgfpathlineto{\pgfqpoint{3.425465in}{3.209970in}}%
\pgfpathlineto{\pgfqpoint{3.433149in}{3.220930in}}%
\pgfpathlineto{\pgfqpoint{3.440827in}{3.232035in}}%
\pgfpathclose%
\pgfusepath{fill}%
\end{pgfscope}%
\begin{pgfscope}%
\pgfpathrectangle{\pgfqpoint{1.150000in}{0.150000in}}{\pgfqpoint{5.700000in}{5.700000in}}%
\pgfusepath{clip}%
\pgfsetbuttcap%
\pgfsetroundjoin%
\definecolor{currentfill}{rgb}{0.283187,0.125848,0.444960}%
\pgfsetfillcolor{currentfill}%
\pgfsetfillopacity{0.700000}%
\pgfsetlinewidth{0.000000pt}%
\definecolor{currentstroke}{rgb}{0.000000,0.000000,0.000000}%
\pgfsetstrokecolor{currentstroke}%
\pgfsetdash{}{0pt}%
\pgfpathmoveto{\pgfqpoint{3.574664in}{3.223066in}}%
\pgfpathlineto{\pgfqpoint{3.587578in}{3.216420in}}%
\pgfpathlineto{\pgfqpoint{3.600495in}{3.209820in}}%
\pgfpathlineto{\pgfqpoint{3.613416in}{3.203267in}}%
\pgfpathlineto{\pgfqpoint{3.626341in}{3.196760in}}%
\pgfpathlineto{\pgfqpoint{3.618710in}{3.185453in}}%
\pgfpathlineto{\pgfqpoint{3.611074in}{3.174298in}}%
\pgfpathlineto{\pgfqpoint{3.603432in}{3.163292in}}%
\pgfpathlineto{\pgfqpoint{3.595785in}{3.152431in}}%
\pgfpathlineto{\pgfqpoint{3.582852in}{3.158801in}}%
\pgfpathlineto{\pgfqpoint{3.569923in}{3.165218in}}%
\pgfpathlineto{\pgfqpoint{3.556998in}{3.171682in}}%
\pgfpathlineto{\pgfqpoint{3.544076in}{3.178193in}}%
\pgfpathlineto{\pgfqpoint{3.551731in}{3.189186in}}%
\pgfpathlineto{\pgfqpoint{3.559381in}{3.200326in}}%
\pgfpathlineto{\pgfqpoint{3.567025in}{3.211618in}}%
\pgfpathlineto{\pgfqpoint{3.574664in}{3.223066in}}%
\pgfpathclose%
\pgfusepath{fill}%
\end{pgfscope}%
\begin{pgfscope}%
\pgfpathrectangle{\pgfqpoint{1.150000in}{0.150000in}}{\pgfqpoint{5.700000in}{5.700000in}}%
\pgfusepath{clip}%
\pgfsetbuttcap%
\pgfsetroundjoin%
\definecolor{currentfill}{rgb}{0.283187,0.125848,0.444960}%
\pgfsetfillcolor{currentfill}%
\pgfsetfillopacity{0.700000}%
\pgfsetlinewidth{0.000000pt}%
\definecolor{currentstroke}{rgb}{0.000000,0.000000,0.000000}%
\pgfsetstrokecolor{currentstroke}%
\pgfsetdash{}{0pt}%
\pgfpathmoveto{\pgfqpoint{3.708513in}{3.217446in}}%
\pgfpathlineto{\pgfqpoint{3.721449in}{3.211015in}}%
\pgfpathlineto{\pgfqpoint{3.734389in}{3.204628in}}%
\pgfpathlineto{\pgfqpoint{3.747332in}{3.198284in}}%
\pgfpathlineto{\pgfqpoint{3.760279in}{3.191984in}}%
\pgfpathlineto{\pgfqpoint{3.752686in}{3.180324in}}%
\pgfpathlineto{\pgfqpoint{3.745088in}{3.168831in}}%
\pgfpathlineto{\pgfqpoint{3.737484in}{3.157497in}}%
\pgfpathlineto{\pgfqpoint{3.729876in}{3.146320in}}%
\pgfpathlineto{\pgfqpoint{3.716920in}{3.152472in}}%
\pgfpathlineto{\pgfqpoint{3.703968in}{3.158667in}}%
\pgfpathlineto{\pgfqpoint{3.691021in}{3.164905in}}%
\pgfpathlineto{\pgfqpoint{3.678077in}{3.171187in}}%
\pgfpathlineto{\pgfqpoint{3.685694in}{3.182509in}}%
\pgfpathlineto{\pgfqpoint{3.693306in}{3.193989in}}%
\pgfpathlineto{\pgfqpoint{3.700912in}{3.205633in}}%
\pgfpathlineto{\pgfqpoint{3.708513in}{3.217446in}}%
\pgfpathclose%
\pgfusepath{fill}%
\end{pgfscope}%
\begin{pgfscope}%
\pgfpathrectangle{\pgfqpoint{1.150000in}{0.150000in}}{\pgfqpoint{5.700000in}{5.700000in}}%
\pgfusepath{clip}%
\pgfsetbuttcap%
\pgfsetroundjoin%
\definecolor{currentfill}{rgb}{0.283187,0.125848,0.444960}%
\pgfsetfillcolor{currentfill}%
\pgfsetfillopacity{0.700000}%
\pgfsetlinewidth{0.000000pt}%
\definecolor{currentstroke}{rgb}{0.000000,0.000000,0.000000}%
\pgfsetstrokecolor{currentstroke}%
\pgfsetdash{}{0pt}%
\pgfpathmoveto{\pgfqpoint{3.842401in}{3.214947in}}%
\pgfpathlineto{\pgfqpoint{3.855361in}{3.208693in}}%
\pgfpathlineto{\pgfqpoint{3.868325in}{3.202481in}}%
\pgfpathlineto{\pgfqpoint{3.881293in}{3.196308in}}%
\pgfpathlineto{\pgfqpoint{3.894265in}{3.190176in}}%
\pgfpathlineto{\pgfqpoint{3.886707in}{3.178138in}}%
\pgfpathlineto{\pgfqpoint{3.879144in}{3.166279in}}%
\pgfpathlineto{\pgfqpoint{3.871577in}{3.154594in}}%
\pgfpathlineto{\pgfqpoint{3.864005in}{3.143078in}}%
\pgfpathlineto{\pgfqpoint{3.851025in}{3.149048in}}%
\pgfpathlineto{\pgfqpoint{3.838049in}{3.155059in}}%
\pgfpathlineto{\pgfqpoint{3.825077in}{3.161110in}}%
\pgfpathlineto{\pgfqpoint{3.812109in}{3.167202in}}%
\pgfpathlineto{\pgfqpoint{3.819689in}{3.178875in}}%
\pgfpathlineto{\pgfqpoint{3.827265in}{3.190720in}}%
\pgfpathlineto{\pgfqpoint{3.834835in}{3.202742in}}%
\pgfpathlineto{\pgfqpoint{3.842401in}{3.214947in}}%
\pgfpathclose%
\pgfusepath{fill}%
\end{pgfscope}%
\begin{pgfscope}%
\pgfpathrectangle{\pgfqpoint{1.150000in}{0.150000in}}{\pgfqpoint{5.700000in}{5.700000in}}%
\pgfusepath{clip}%
\pgfsetbuttcap%
\pgfsetroundjoin%
\definecolor{currentfill}{rgb}{0.283229,0.120777,0.440584}%
\pgfsetfillcolor{currentfill}%
\pgfsetfillopacity{0.700000}%
\pgfsetlinewidth{0.000000pt}%
\definecolor{currentstroke}{rgb}{0.000000,0.000000,0.000000}%
\pgfsetstrokecolor{currentstroke}%
\pgfsetdash{}{0pt}%
\pgfpathmoveto{\pgfqpoint{3.492424in}{3.204718in}}%
\pgfpathlineto{\pgfqpoint{3.505332in}{3.198013in}}%
\pgfpathlineto{\pgfqpoint{3.518243in}{3.191358in}}%
\pgfpathlineto{\pgfqpoint{3.531157in}{3.184752in}}%
\pgfpathlineto{\pgfqpoint{3.544076in}{3.178193in}}%
\pgfpathlineto{\pgfqpoint{3.536414in}{3.167343in}}%
\pgfpathlineto{\pgfqpoint{3.528747in}{3.156633in}}%
\pgfpathlineto{\pgfqpoint{3.521074in}{3.146057in}}%
\pgfpathlineto{\pgfqpoint{3.513395in}{3.135613in}}%
\pgfpathlineto{\pgfqpoint{3.500468in}{3.142048in}}%
\pgfpathlineto{\pgfqpoint{3.487545in}{3.148531in}}%
\pgfpathlineto{\pgfqpoint{3.474625in}{3.155063in}}%
\pgfpathlineto{\pgfqpoint{3.461709in}{3.161644in}}%
\pgfpathlineto{\pgfqpoint{3.469397in}{3.172208in}}%
\pgfpathlineto{\pgfqpoint{3.477078in}{3.182905in}}%
\pgfpathlineto{\pgfqpoint{3.484754in}{3.193740in}}%
\pgfpathlineto{\pgfqpoint{3.492424in}{3.204718in}}%
\pgfpathclose%
\pgfusepath{fill}%
\end{pgfscope}%
\begin{pgfscope}%
\pgfpathrectangle{\pgfqpoint{1.150000in}{0.150000in}}{\pgfqpoint{5.700000in}{5.700000in}}%
\pgfusepath{clip}%
\pgfsetbuttcap%
\pgfsetroundjoin%
\definecolor{currentfill}{rgb}{0.283229,0.120777,0.440584}%
\pgfsetfillcolor{currentfill}%
\pgfsetfillopacity{0.700000}%
\pgfsetlinewidth{0.000000pt}%
\definecolor{currentstroke}{rgb}{0.000000,0.000000,0.000000}%
\pgfsetstrokecolor{currentstroke}%
\pgfsetdash{}{0pt}%
\pgfpathmoveto{\pgfqpoint{3.626341in}{3.196760in}}%
\pgfpathlineto{\pgfqpoint{3.639269in}{3.190300in}}%
\pgfpathlineto{\pgfqpoint{3.652201in}{3.183884in}}%
\pgfpathlineto{\pgfqpoint{3.665137in}{3.177513in}}%
\pgfpathlineto{\pgfqpoint{3.678077in}{3.171187in}}%
\pgfpathlineto{\pgfqpoint{3.670455in}{3.160020in}}%
\pgfpathlineto{\pgfqpoint{3.662827in}{3.149003in}}%
\pgfpathlineto{\pgfqpoint{3.655194in}{3.138132in}}%
\pgfpathlineto{\pgfqpoint{3.647555in}{3.127402in}}%
\pgfpathlineto{\pgfqpoint{3.634607in}{3.133592in}}%
\pgfpathlineto{\pgfqpoint{3.621662in}{3.139826in}}%
\pgfpathlineto{\pgfqpoint{3.608722in}{3.146106in}}%
\pgfpathlineto{\pgfqpoint{3.595785in}{3.152431in}}%
\pgfpathlineto{\pgfqpoint{3.603432in}{3.163292in}}%
\pgfpathlineto{\pgfqpoint{3.611074in}{3.174298in}}%
\pgfpathlineto{\pgfqpoint{3.618710in}{3.185453in}}%
\pgfpathlineto{\pgfqpoint{3.626341in}{3.196760in}}%
\pgfpathclose%
\pgfusepath{fill}%
\end{pgfscope}%
\begin{pgfscope}%
\pgfpathrectangle{\pgfqpoint{1.150000in}{0.150000in}}{\pgfqpoint{5.700000in}{5.700000in}}%
\pgfusepath{clip}%
\pgfsetbuttcap%
\pgfsetroundjoin%
\definecolor{currentfill}{rgb}{0.283197,0.115680,0.436115}%
\pgfsetfillcolor{currentfill}%
\pgfsetfillopacity{0.700000}%
\pgfsetlinewidth{0.000000pt}%
\definecolor{currentstroke}{rgb}{0.000000,0.000000,0.000000}%
\pgfsetstrokecolor{currentstroke}%
\pgfsetdash{}{0pt}%
\pgfpathmoveto{\pgfqpoint{3.760279in}{3.191984in}}%
\pgfpathlineto{\pgfqpoint{3.773231in}{3.185725in}}%
\pgfpathlineto{\pgfqpoint{3.786186in}{3.179509in}}%
\pgfpathlineto{\pgfqpoint{3.799146in}{3.173335in}}%
\pgfpathlineto{\pgfqpoint{3.812109in}{3.167202in}}%
\pgfpathlineto{\pgfqpoint{3.804524in}{3.155696in}}%
\pgfpathlineto{\pgfqpoint{3.796934in}{3.144353in}}%
\pgfpathlineto{\pgfqpoint{3.789339in}{3.133167in}}%
\pgfpathlineto{\pgfqpoint{3.781738in}{3.122134in}}%
\pgfpathlineto{\pgfqpoint{3.768766in}{3.128118in}}%
\pgfpathlineto{\pgfqpoint{3.755799in}{3.134143in}}%
\pgfpathlineto{\pgfqpoint{3.742835in}{3.140211in}}%
\pgfpathlineto{\pgfqpoint{3.729876in}{3.146320in}}%
\pgfpathlineto{\pgfqpoint{3.737484in}{3.157497in}}%
\pgfpathlineto{\pgfqpoint{3.745088in}{3.168831in}}%
\pgfpathlineto{\pgfqpoint{3.752686in}{3.180324in}}%
\pgfpathlineto{\pgfqpoint{3.760279in}{3.191984in}}%
\pgfpathclose%
\pgfusepath{fill}%
\end{pgfscope}%
\begin{pgfscope}%
\pgfpathrectangle{\pgfqpoint{1.150000in}{0.150000in}}{\pgfqpoint{5.700000in}{5.700000in}}%
\pgfusepath{clip}%
\pgfsetbuttcap%
\pgfsetroundjoin%
\definecolor{currentfill}{rgb}{0.283197,0.115680,0.436115}%
\pgfsetfillcolor{currentfill}%
\pgfsetfillopacity{0.700000}%
\pgfsetlinewidth{0.000000pt}%
\definecolor{currentstroke}{rgb}{0.000000,0.000000,0.000000}%
\pgfsetstrokecolor{currentstroke}%
\pgfsetdash{}{0pt}%
\pgfpathmoveto{\pgfqpoint{3.410079in}{3.188470in}}%
\pgfpathlineto{\pgfqpoint{3.422981in}{3.181687in}}%
\pgfpathlineto{\pgfqpoint{3.435887in}{3.174956in}}%
\pgfpathlineto{\pgfqpoint{3.448796in}{3.168275in}}%
\pgfpathlineto{\pgfqpoint{3.461709in}{3.161644in}}%
\pgfpathlineto{\pgfqpoint{3.454015in}{3.151212in}}%
\pgfpathlineto{\pgfqpoint{3.446315in}{3.140906in}}%
\pgfpathlineto{\pgfqpoint{3.438609in}{3.130722in}}%
\pgfpathlineto{\pgfqpoint{3.430897in}{3.120658in}}%
\pgfpathlineto{\pgfqpoint{3.417976in}{3.127178in}}%
\pgfpathlineto{\pgfqpoint{3.405058in}{3.133749in}}%
\pgfpathlineto{\pgfqpoint{3.392143in}{3.140370in}}%
\pgfpathlineto{\pgfqpoint{3.379232in}{3.147042in}}%
\pgfpathlineto{\pgfqpoint{3.386953in}{3.157212in}}%
\pgfpathlineto{\pgfqpoint{3.394668in}{3.167504in}}%
\pgfpathlineto{\pgfqpoint{3.402376in}{3.177922in}}%
\pgfpathlineto{\pgfqpoint{3.410079in}{3.188470in}}%
\pgfpathclose%
\pgfusepath{fill}%
\end{pgfscope}%
\begin{pgfscope}%
\pgfpathrectangle{\pgfqpoint{1.150000in}{0.150000in}}{\pgfqpoint{5.700000in}{5.700000in}}%
\pgfusepath{clip}%
\pgfsetbuttcap%
\pgfsetroundjoin%
\definecolor{currentfill}{rgb}{0.283229,0.120777,0.440584}%
\pgfsetfillcolor{currentfill}%
\pgfsetfillopacity{0.700000}%
\pgfsetlinewidth{0.000000pt}%
\definecolor{currentstroke}{rgb}{0.000000,0.000000,0.000000}%
\pgfsetstrokecolor{currentstroke}%
\pgfsetdash{}{0pt}%
\pgfpathmoveto{\pgfqpoint{3.894265in}{3.190176in}}%
\pgfpathlineto{\pgfqpoint{3.907241in}{3.184083in}}%
\pgfpathlineto{\pgfqpoint{3.920221in}{3.178030in}}%
\pgfpathlineto{\pgfqpoint{3.933206in}{3.172016in}}%
\pgfpathlineto{\pgfqpoint{3.946195in}{3.166041in}}%
\pgfpathlineto{\pgfqpoint{3.938645in}{3.154169in}}%
\pgfpathlineto{\pgfqpoint{3.931091in}{3.142473in}}%
\pgfpathlineto{\pgfqpoint{3.923532in}{3.130948in}}%
\pgfpathlineto{\pgfqpoint{3.915968in}{3.119589in}}%
\pgfpathlineto{\pgfqpoint{3.902971in}{3.125402in}}%
\pgfpathlineto{\pgfqpoint{3.889978in}{3.131255in}}%
\pgfpathlineto{\pgfqpoint{3.876989in}{3.137147in}}%
\pgfpathlineto{\pgfqpoint{3.864005in}{3.143078in}}%
\pgfpathlineto{\pgfqpoint{3.871577in}{3.154594in}}%
\pgfpathlineto{\pgfqpoint{3.879144in}{3.166279in}}%
\pgfpathlineto{\pgfqpoint{3.886707in}{3.178138in}}%
\pgfpathlineto{\pgfqpoint{3.894265in}{3.190176in}}%
\pgfpathclose%
\pgfusepath{fill}%
\end{pgfscope}%
\begin{pgfscope}%
\pgfpathrectangle{\pgfqpoint{1.150000in}{0.150000in}}{\pgfqpoint{5.700000in}{5.700000in}}%
\pgfusepath{clip}%
\pgfsetbuttcap%
\pgfsetroundjoin%
\definecolor{currentfill}{rgb}{0.283197,0.115680,0.436115}%
\pgfsetfillcolor{currentfill}%
\pgfsetfillopacity{0.700000}%
\pgfsetlinewidth{0.000000pt}%
\definecolor{currentstroke}{rgb}{0.000000,0.000000,0.000000}%
\pgfsetstrokecolor{currentstroke}%
\pgfsetdash{}{0pt}%
\pgfpathmoveto{\pgfqpoint{3.544076in}{3.178193in}}%
\pgfpathlineto{\pgfqpoint{3.556998in}{3.171682in}}%
\pgfpathlineto{\pgfqpoint{3.569923in}{3.165218in}}%
\pgfpathlineto{\pgfqpoint{3.582852in}{3.158801in}}%
\pgfpathlineto{\pgfqpoint{3.595785in}{3.152431in}}%
\pgfpathlineto{\pgfqpoint{3.588132in}{3.141709in}}%
\pgfpathlineto{\pgfqpoint{3.580473in}{3.131123in}}%
\pgfpathlineto{\pgfqpoint{3.572809in}{3.120669in}}%
\pgfpathlineto{\pgfqpoint{3.565138in}{3.110343in}}%
\pgfpathlineto{\pgfqpoint{3.552197in}{3.116591in}}%
\pgfpathlineto{\pgfqpoint{3.539259in}{3.122884in}}%
\pgfpathlineto{\pgfqpoint{3.526325in}{3.129225in}}%
\pgfpathlineto{\pgfqpoint{3.513395in}{3.135613in}}%
\pgfpathlineto{\pgfqpoint{3.521074in}{3.146057in}}%
\pgfpathlineto{\pgfqpoint{3.528747in}{3.156633in}}%
\pgfpathlineto{\pgfqpoint{3.536414in}{3.167343in}}%
\pgfpathlineto{\pgfqpoint{3.544076in}{3.178193in}}%
\pgfpathclose%
\pgfusepath{fill}%
\end{pgfscope}%
\begin{pgfscope}%
\pgfpathrectangle{\pgfqpoint{1.150000in}{0.150000in}}{\pgfqpoint{5.700000in}{5.700000in}}%
\pgfusepath{clip}%
\pgfsetbuttcap%
\pgfsetroundjoin%
\definecolor{currentfill}{rgb}{0.283091,0.110553,0.431554}%
\pgfsetfillcolor{currentfill}%
\pgfsetfillopacity{0.700000}%
\pgfsetlinewidth{0.000000pt}%
\definecolor{currentstroke}{rgb}{0.000000,0.000000,0.000000}%
\pgfsetstrokecolor{currentstroke}%
\pgfsetdash{}{0pt}%
\pgfpathmoveto{\pgfqpoint{3.678077in}{3.171187in}}%
\pgfpathlineto{\pgfqpoint{3.691021in}{3.164905in}}%
\pgfpathlineto{\pgfqpoint{3.703968in}{3.158667in}}%
\pgfpathlineto{\pgfqpoint{3.716920in}{3.152472in}}%
\pgfpathlineto{\pgfqpoint{3.729876in}{3.146320in}}%
\pgfpathlineto{\pgfqpoint{3.722261in}{3.135294in}}%
\pgfpathlineto{\pgfqpoint{3.714642in}{3.124415in}}%
\pgfpathlineto{\pgfqpoint{3.707017in}{3.113678in}}%
\pgfpathlineto{\pgfqpoint{3.699387in}{3.103079in}}%
\pgfpathlineto{\pgfqpoint{3.686423in}{3.109095in}}%
\pgfpathlineto{\pgfqpoint{3.673463in}{3.115153in}}%
\pgfpathlineto{\pgfqpoint{3.660507in}{3.121256in}}%
\pgfpathlineto{\pgfqpoint{3.647555in}{3.127402in}}%
\pgfpathlineto{\pgfqpoint{3.655194in}{3.138132in}}%
\pgfpathlineto{\pgfqpoint{3.662827in}{3.149003in}}%
\pgfpathlineto{\pgfqpoint{3.670455in}{3.160020in}}%
\pgfpathlineto{\pgfqpoint{3.678077in}{3.171187in}}%
\pgfpathclose%
\pgfusepath{fill}%
\end{pgfscope}%
\begin{pgfscope}%
\pgfpathrectangle{\pgfqpoint{1.150000in}{0.150000in}}{\pgfqpoint{5.700000in}{5.700000in}}%
\pgfusepath{clip}%
\pgfsetbuttcap%
\pgfsetroundjoin%
\definecolor{currentfill}{rgb}{0.283091,0.110553,0.431554}%
\pgfsetfillcolor{currentfill}%
\pgfsetfillopacity{0.700000}%
\pgfsetlinewidth{0.000000pt}%
\definecolor{currentstroke}{rgb}{0.000000,0.000000,0.000000}%
\pgfsetstrokecolor{currentstroke}%
\pgfsetdash{}{0pt}%
\pgfpathmoveto{\pgfqpoint{3.812109in}{3.167202in}}%
\pgfpathlineto{\pgfqpoint{3.825077in}{3.161110in}}%
\pgfpathlineto{\pgfqpoint{3.838049in}{3.155059in}}%
\pgfpathlineto{\pgfqpoint{3.851025in}{3.149048in}}%
\pgfpathlineto{\pgfqpoint{3.864005in}{3.143078in}}%
\pgfpathlineto{\pgfqpoint{3.856428in}{3.131726in}}%
\pgfpathlineto{\pgfqpoint{3.848846in}{3.120533in}}%
\pgfpathlineto{\pgfqpoint{3.841260in}{3.109494in}}%
\pgfpathlineto{\pgfqpoint{3.833668in}{3.098605in}}%
\pgfpathlineto{\pgfqpoint{3.820679in}{3.104427in}}%
\pgfpathlineto{\pgfqpoint{3.807695in}{3.110289in}}%
\pgfpathlineto{\pgfqpoint{3.794714in}{3.116191in}}%
\pgfpathlineto{\pgfqpoint{3.781738in}{3.122134in}}%
\pgfpathlineto{\pgfqpoint{3.789339in}{3.133167in}}%
\pgfpathlineto{\pgfqpoint{3.796934in}{3.144353in}}%
\pgfpathlineto{\pgfqpoint{3.804524in}{3.155696in}}%
\pgfpathlineto{\pgfqpoint{3.812109in}{3.167202in}}%
\pgfpathclose%
\pgfusepath{fill}%
\end{pgfscope}%
\begin{pgfscope}%
\pgfpathrectangle{\pgfqpoint{1.150000in}{0.150000in}}{\pgfqpoint{5.700000in}{5.700000in}}%
\pgfusepath{clip}%
\pgfsetbuttcap%
\pgfsetroundjoin%
\definecolor{currentfill}{rgb}{0.283091,0.110553,0.431554}%
\pgfsetfillcolor{currentfill}%
\pgfsetfillopacity{0.700000}%
\pgfsetlinewidth{0.000000pt}%
\definecolor{currentstroke}{rgb}{0.000000,0.000000,0.000000}%
\pgfsetstrokecolor{currentstroke}%
\pgfsetdash{}{0pt}%
\pgfpathmoveto{\pgfqpoint{3.461709in}{3.161644in}}%
\pgfpathlineto{\pgfqpoint{3.474625in}{3.155063in}}%
\pgfpathlineto{\pgfqpoint{3.487545in}{3.148531in}}%
\pgfpathlineto{\pgfqpoint{3.500468in}{3.142048in}}%
\pgfpathlineto{\pgfqpoint{3.513395in}{3.135613in}}%
\pgfpathlineto{\pgfqpoint{3.505710in}{3.125295in}}%
\pgfpathlineto{\pgfqpoint{3.498019in}{3.115101in}}%
\pgfpathlineto{\pgfqpoint{3.490322in}{3.105026in}}%
\pgfpathlineto{\pgfqpoint{3.482619in}{3.095068in}}%
\pgfpathlineto{\pgfqpoint{3.469683in}{3.101393in}}%
\pgfpathlineto{\pgfqpoint{3.456751in}{3.107766in}}%
\pgfpathlineto{\pgfqpoint{3.443822in}{3.114187in}}%
\pgfpathlineto{\pgfqpoint{3.430897in}{3.120658in}}%
\pgfpathlineto{\pgfqpoint{3.438609in}{3.130722in}}%
\pgfpathlineto{\pgfqpoint{3.446315in}{3.140906in}}%
\pgfpathlineto{\pgfqpoint{3.454015in}{3.151212in}}%
\pgfpathlineto{\pgfqpoint{3.461709in}{3.161644in}}%
\pgfpathclose%
\pgfusepath{fill}%
\end{pgfscope}%
\begin{pgfscope}%
\pgfpathrectangle{\pgfqpoint{1.150000in}{0.150000in}}{\pgfqpoint{5.700000in}{5.700000in}}%
\pgfusepath{clip}%
\pgfsetbuttcap%
\pgfsetroundjoin%
\definecolor{currentfill}{rgb}{0.282910,0.105393,0.426902}%
\pgfsetfillcolor{currentfill}%
\pgfsetfillopacity{0.700000}%
\pgfsetlinewidth{0.000000pt}%
\definecolor{currentstroke}{rgb}{0.000000,0.000000,0.000000}%
\pgfsetstrokecolor{currentstroke}%
\pgfsetdash{}{0pt}%
\pgfpathmoveto{\pgfqpoint{3.595785in}{3.152431in}}%
\pgfpathlineto{\pgfqpoint{3.608722in}{3.146106in}}%
\pgfpathlineto{\pgfqpoint{3.621662in}{3.139826in}}%
\pgfpathlineto{\pgfqpoint{3.634607in}{3.133592in}}%
\pgfpathlineto{\pgfqpoint{3.647555in}{3.127402in}}%
\pgfpathlineto{\pgfqpoint{3.639910in}{3.116808in}}%
\pgfpathlineto{\pgfqpoint{3.632260in}{3.106347in}}%
\pgfpathlineto{\pgfqpoint{3.624605in}{3.096015in}}%
\pgfpathlineto{\pgfqpoint{3.616944in}{3.085807in}}%
\pgfpathlineto{\pgfqpoint{3.603986in}{3.091874in}}%
\pgfpathlineto{\pgfqpoint{3.591033in}{3.097985in}}%
\pgfpathlineto{\pgfqpoint{3.578084in}{3.104141in}}%
\pgfpathlineto{\pgfqpoint{3.565138in}{3.110343in}}%
\pgfpathlineto{\pgfqpoint{3.572809in}{3.120669in}}%
\pgfpathlineto{\pgfqpoint{3.580473in}{3.131123in}}%
\pgfpathlineto{\pgfqpoint{3.588132in}{3.141709in}}%
\pgfpathlineto{\pgfqpoint{3.595785in}{3.152431in}}%
\pgfpathclose%
\pgfusepath{fill}%
\end{pgfscope}%
\begin{pgfscope}%
\pgfpathrectangle{\pgfqpoint{1.150000in}{0.150000in}}{\pgfqpoint{5.700000in}{5.700000in}}%
\pgfusepath{clip}%
\pgfsetbuttcap%
\pgfsetroundjoin%
\definecolor{currentfill}{rgb}{0.283091,0.110553,0.431554}%
\pgfsetfillcolor{currentfill}%
\pgfsetfillopacity{0.700000}%
\pgfsetlinewidth{0.000000pt}%
\definecolor{currentstroke}{rgb}{0.000000,0.000000,0.000000}%
\pgfsetstrokecolor{currentstroke}%
\pgfsetdash{}{0pt}%
\pgfpathmoveto{\pgfqpoint{3.946195in}{3.166041in}}%
\pgfpathlineto{\pgfqpoint{3.959189in}{3.160104in}}%
\pgfpathlineto{\pgfqpoint{3.972186in}{3.154205in}}%
\pgfpathlineto{\pgfqpoint{3.985188in}{3.148344in}}%
\pgfpathlineto{\pgfqpoint{3.998195in}{3.142520in}}%
\pgfpathlineto{\pgfqpoint{3.990653in}{3.130814in}}%
\pgfpathlineto{\pgfqpoint{3.983107in}{3.119282in}}%
\pgfpathlineto{\pgfqpoint{3.975557in}{3.107917in}}%
\pgfpathlineto{\pgfqpoint{3.968002in}{3.096715in}}%
\pgfpathlineto{\pgfqpoint{3.954987in}{3.102377in}}%
\pgfpathlineto{\pgfqpoint{3.941976in}{3.108076in}}%
\pgfpathlineto{\pgfqpoint{3.928970in}{3.113813in}}%
\pgfpathlineto{\pgfqpoint{3.915968in}{3.119589in}}%
\pgfpathlineto{\pgfqpoint{3.923532in}{3.130948in}}%
\pgfpathlineto{\pgfqpoint{3.931091in}{3.142473in}}%
\pgfpathlineto{\pgfqpoint{3.938645in}{3.154169in}}%
\pgfpathlineto{\pgfqpoint{3.946195in}{3.166041in}}%
\pgfpathclose%
\pgfusepath{fill}%
\end{pgfscope}%
\begin{pgfscope}%
\pgfpathrectangle{\pgfqpoint{1.150000in}{0.150000in}}{\pgfqpoint{5.700000in}{5.700000in}}%
\pgfusepath{clip}%
\pgfsetbuttcap%
\pgfsetroundjoin%
\definecolor{currentfill}{rgb}{0.282910,0.105393,0.426902}%
\pgfsetfillcolor{currentfill}%
\pgfsetfillopacity{0.700000}%
\pgfsetlinewidth{0.000000pt}%
\definecolor{currentstroke}{rgb}{0.000000,0.000000,0.000000}%
\pgfsetstrokecolor{currentstroke}%
\pgfsetdash{}{0pt}%
\pgfpathmoveto{\pgfqpoint{3.729876in}{3.146320in}}%
\pgfpathlineto{\pgfqpoint{3.742835in}{3.140211in}}%
\pgfpathlineto{\pgfqpoint{3.755799in}{3.134143in}}%
\pgfpathlineto{\pgfqpoint{3.768766in}{3.128118in}}%
\pgfpathlineto{\pgfqpoint{3.781738in}{3.122134in}}%
\pgfpathlineto{\pgfqpoint{3.774133in}{3.111249in}}%
\pgfpathlineto{\pgfqpoint{3.766522in}{3.100507in}}%
\pgfpathlineto{\pgfqpoint{3.758906in}{3.089905in}}%
\pgfpathlineto{\pgfqpoint{3.751285in}{3.079437in}}%
\pgfpathlineto{\pgfqpoint{3.738304in}{3.085285in}}%
\pgfpathlineto{\pgfqpoint{3.725328in}{3.091174in}}%
\pgfpathlineto{\pgfqpoint{3.712356in}{3.097105in}}%
\pgfpathlineto{\pgfqpoint{3.699387in}{3.103079in}}%
\pgfpathlineto{\pgfqpoint{3.707017in}{3.113678in}}%
\pgfpathlineto{\pgfqpoint{3.714642in}{3.124415in}}%
\pgfpathlineto{\pgfqpoint{3.722261in}{3.135294in}}%
\pgfpathlineto{\pgfqpoint{3.729876in}{3.146320in}}%
\pgfpathclose%
\pgfusepath{fill}%
\end{pgfscope}%
\begin{pgfscope}%
\pgfpathrectangle{\pgfqpoint{1.150000in}{0.150000in}}{\pgfqpoint{5.700000in}{5.700000in}}%
\pgfusepath{clip}%
\pgfsetbuttcap%
\pgfsetroundjoin%
\definecolor{currentfill}{rgb}{0.282910,0.105393,0.426902}%
\pgfsetfillcolor{currentfill}%
\pgfsetfillopacity{0.700000}%
\pgfsetlinewidth{0.000000pt}%
\definecolor{currentstroke}{rgb}{0.000000,0.000000,0.000000}%
\pgfsetstrokecolor{currentstroke}%
\pgfsetdash{}{0pt}%
\pgfpathmoveto{\pgfqpoint{3.864005in}{3.143078in}}%
\pgfpathlineto{\pgfqpoint{3.876989in}{3.137147in}}%
\pgfpathlineto{\pgfqpoint{3.889978in}{3.131255in}}%
\pgfpathlineto{\pgfqpoint{3.902971in}{3.125402in}}%
\pgfpathlineto{\pgfqpoint{3.915968in}{3.119589in}}%
\pgfpathlineto{\pgfqpoint{3.908400in}{3.108390in}}%
\pgfpathlineto{\pgfqpoint{3.900827in}{3.097348in}}%
\pgfpathlineto{\pgfqpoint{3.893249in}{3.086457in}}%
\pgfpathlineto{\pgfqpoint{3.885666in}{3.075712in}}%
\pgfpathlineto{\pgfqpoint{3.872660in}{3.081377in}}%
\pgfpathlineto{\pgfqpoint{3.859658in}{3.087080in}}%
\pgfpathlineto{\pgfqpoint{3.846661in}{3.092823in}}%
\pgfpathlineto{\pgfqpoint{3.833668in}{3.098605in}}%
\pgfpathlineto{\pgfqpoint{3.841260in}{3.109494in}}%
\pgfpathlineto{\pgfqpoint{3.848846in}{3.120533in}}%
\pgfpathlineto{\pgfqpoint{3.856428in}{3.131726in}}%
\pgfpathlineto{\pgfqpoint{3.864005in}{3.143078in}}%
\pgfpathclose%
\pgfusepath{fill}%
\end{pgfscope}%
\begin{pgfscope}%
\pgfpathrectangle{\pgfqpoint{1.150000in}{0.150000in}}{\pgfqpoint{5.700000in}{5.700000in}}%
\pgfusepath{clip}%
\pgfsetbuttcap%
\pgfsetroundjoin%
\definecolor{currentfill}{rgb}{0.282910,0.105393,0.426902}%
\pgfsetfillcolor{currentfill}%
\pgfsetfillopacity{0.700000}%
\pgfsetlinewidth{0.000000pt}%
\definecolor{currentstroke}{rgb}{0.000000,0.000000,0.000000}%
\pgfsetstrokecolor{currentstroke}%
\pgfsetdash{}{0pt}%
\pgfpathmoveto{\pgfqpoint{3.379232in}{3.147042in}}%
\pgfpathlineto{\pgfqpoint{3.392143in}{3.140370in}}%
\pgfpathlineto{\pgfqpoint{3.405058in}{3.133749in}}%
\pgfpathlineto{\pgfqpoint{3.417976in}{3.127178in}}%
\pgfpathlineto{\pgfqpoint{3.430897in}{3.120658in}}%
\pgfpathlineto{\pgfqpoint{3.423179in}{3.110709in}}%
\pgfpathlineto{\pgfqpoint{3.415455in}{3.100873in}}%
\pgfpathlineto{\pgfqpoint{3.407725in}{3.091146in}}%
\pgfpathlineto{\pgfqpoint{3.399988in}{3.081524in}}%
\pgfpathlineto{\pgfqpoint{3.387057in}{3.087947in}}%
\pgfpathlineto{\pgfqpoint{3.374130in}{3.094420in}}%
\pgfpathlineto{\pgfqpoint{3.361206in}{3.100943in}}%
\pgfpathlineto{\pgfqpoint{3.348285in}{3.107518in}}%
\pgfpathlineto{\pgfqpoint{3.356031in}{3.117232in}}%
\pgfpathlineto{\pgfqpoint{3.363771in}{3.127056in}}%
\pgfpathlineto{\pgfqpoint{3.371505in}{3.136991in}}%
\pgfpathlineto{\pgfqpoint{3.379232in}{3.147042in}}%
\pgfpathclose%
\pgfusepath{fill}%
\end{pgfscope}%
\begin{pgfscope}%
\pgfpathrectangle{\pgfqpoint{1.150000in}{0.150000in}}{\pgfqpoint{5.700000in}{5.700000in}}%
\pgfusepath{clip}%
\pgfsetbuttcap%
\pgfsetroundjoin%
\definecolor{currentfill}{rgb}{0.282656,0.100196,0.422160}%
\pgfsetfillcolor{currentfill}%
\pgfsetfillopacity{0.700000}%
\pgfsetlinewidth{0.000000pt}%
\definecolor{currentstroke}{rgb}{0.000000,0.000000,0.000000}%
\pgfsetstrokecolor{currentstroke}%
\pgfsetdash{}{0pt}%
\pgfpathmoveto{\pgfqpoint{3.513395in}{3.135613in}}%
\pgfpathlineto{\pgfqpoint{3.526325in}{3.129225in}}%
\pgfpathlineto{\pgfqpoint{3.539259in}{3.122884in}}%
\pgfpathlineto{\pgfqpoint{3.552197in}{3.116591in}}%
\pgfpathlineto{\pgfqpoint{3.565138in}{3.110343in}}%
\pgfpathlineto{\pgfqpoint{3.557462in}{3.100141in}}%
\pgfpathlineto{\pgfqpoint{3.549781in}{3.090059in}}%
\pgfpathlineto{\pgfqpoint{3.542093in}{3.080093in}}%
\pgfpathlineto{\pgfqpoint{3.534399in}{3.070240in}}%
\pgfpathlineto{\pgfqpoint{3.521449in}{3.076377in}}%
\pgfpathlineto{\pgfqpoint{3.508502in}{3.082561in}}%
\pgfpathlineto{\pgfqpoint{3.495558in}{3.088791in}}%
\pgfpathlineto{\pgfqpoint{3.482619in}{3.095068in}}%
\pgfpathlineto{\pgfqpoint{3.490322in}{3.105026in}}%
\pgfpathlineto{\pgfqpoint{3.498019in}{3.115101in}}%
\pgfpathlineto{\pgfqpoint{3.505710in}{3.125295in}}%
\pgfpathlineto{\pgfqpoint{3.513395in}{3.135613in}}%
\pgfpathclose%
\pgfusepath{fill}%
\end{pgfscope}%
\begin{pgfscope}%
\pgfpathrectangle{\pgfqpoint{1.150000in}{0.150000in}}{\pgfqpoint{5.700000in}{5.700000in}}%
\pgfusepath{clip}%
\pgfsetbuttcap%
\pgfsetroundjoin%
\definecolor{currentfill}{rgb}{0.282656,0.100196,0.422160}%
\pgfsetfillcolor{currentfill}%
\pgfsetfillopacity{0.700000}%
\pgfsetlinewidth{0.000000pt}%
\definecolor{currentstroke}{rgb}{0.000000,0.000000,0.000000}%
\pgfsetstrokecolor{currentstroke}%
\pgfsetdash{}{0pt}%
\pgfpathmoveto{\pgfqpoint{3.647555in}{3.127402in}}%
\pgfpathlineto{\pgfqpoint{3.660507in}{3.121256in}}%
\pgfpathlineto{\pgfqpoint{3.673463in}{3.115153in}}%
\pgfpathlineto{\pgfqpoint{3.686423in}{3.109095in}}%
\pgfpathlineto{\pgfqpoint{3.699387in}{3.103079in}}%
\pgfpathlineto{\pgfqpoint{3.691752in}{3.092613in}}%
\pgfpathlineto{\pgfqpoint{3.684111in}{3.082277in}}%
\pgfpathlineto{\pgfqpoint{3.676464in}{3.072067in}}%
\pgfpathlineto{\pgfqpoint{3.668812in}{3.061978in}}%
\pgfpathlineto{\pgfqpoint{3.655839in}{3.067870in}}%
\pgfpathlineto{\pgfqpoint{3.642870in}{3.073806in}}%
\pgfpathlineto{\pgfqpoint{3.629905in}{3.079785in}}%
\pgfpathlineto{\pgfqpoint{3.616944in}{3.085807in}}%
\pgfpathlineto{\pgfqpoint{3.624605in}{3.096015in}}%
\pgfpathlineto{\pgfqpoint{3.632260in}{3.106347in}}%
\pgfpathlineto{\pgfqpoint{3.639910in}{3.116808in}}%
\pgfpathlineto{\pgfqpoint{3.647555in}{3.127402in}}%
\pgfpathclose%
\pgfusepath{fill}%
\end{pgfscope}%
\begin{pgfscope}%
\pgfpathrectangle{\pgfqpoint{1.150000in}{0.150000in}}{\pgfqpoint{5.700000in}{5.700000in}}%
\pgfusepath{clip}%
\pgfsetbuttcap%
\pgfsetroundjoin%
\definecolor{currentfill}{rgb}{0.282910,0.105393,0.426902}%
\pgfsetfillcolor{currentfill}%
\pgfsetfillopacity{0.700000}%
\pgfsetlinewidth{0.000000pt}%
\definecolor{currentstroke}{rgb}{0.000000,0.000000,0.000000}%
\pgfsetstrokecolor{currentstroke}%
\pgfsetdash{}{0pt}%
\pgfpathmoveto{\pgfqpoint{3.998195in}{3.142520in}}%
\pgfpathlineto{\pgfqpoint{4.011206in}{3.136734in}}%
\pgfpathlineto{\pgfqpoint{4.024221in}{3.130984in}}%
\pgfpathlineto{\pgfqpoint{4.037241in}{3.125272in}}%
\pgfpathlineto{\pgfqpoint{4.050265in}{3.119596in}}%
\pgfpathlineto{\pgfqpoint{4.042732in}{3.108057in}}%
\pgfpathlineto{\pgfqpoint{4.035195in}{3.096688in}}%
\pgfpathlineto{\pgfqpoint{4.027653in}{3.085483in}}%
\pgfpathlineto{\pgfqpoint{4.020108in}{3.074438in}}%
\pgfpathlineto{\pgfqpoint{4.007074in}{3.079952in}}%
\pgfpathlineto{\pgfqpoint{3.994046in}{3.085503in}}%
\pgfpathlineto{\pgfqpoint{3.981021in}{3.091091in}}%
\pgfpathlineto{\pgfqpoint{3.968002in}{3.096715in}}%
\pgfpathlineto{\pgfqpoint{3.975557in}{3.107917in}}%
\pgfpathlineto{\pgfqpoint{3.983107in}{3.119282in}}%
\pgfpathlineto{\pgfqpoint{3.990653in}{3.130814in}}%
\pgfpathlineto{\pgfqpoint{3.998195in}{3.142520in}}%
\pgfpathclose%
\pgfusepath{fill}%
\end{pgfscope}%
\begin{pgfscope}%
\pgfpathrectangle{\pgfqpoint{1.150000in}{0.150000in}}{\pgfqpoint{5.700000in}{5.700000in}}%
\pgfusepath{clip}%
\pgfsetbuttcap%
\pgfsetroundjoin%
\definecolor{currentfill}{rgb}{0.282656,0.100196,0.422160}%
\pgfsetfillcolor{currentfill}%
\pgfsetfillopacity{0.700000}%
\pgfsetlinewidth{0.000000pt}%
\definecolor{currentstroke}{rgb}{0.000000,0.000000,0.000000}%
\pgfsetstrokecolor{currentstroke}%
\pgfsetdash{}{0pt}%
\pgfpathmoveto{\pgfqpoint{3.781738in}{3.122134in}}%
\pgfpathlineto{\pgfqpoint{3.794714in}{3.116191in}}%
\pgfpathlineto{\pgfqpoint{3.807695in}{3.110289in}}%
\pgfpathlineto{\pgfqpoint{3.820679in}{3.104427in}}%
\pgfpathlineto{\pgfqpoint{3.833668in}{3.098605in}}%
\pgfpathlineto{\pgfqpoint{3.826071in}{3.087861in}}%
\pgfpathlineto{\pgfqpoint{3.818469in}{3.077257in}}%
\pgfpathlineto{\pgfqpoint{3.810863in}{3.066789in}}%
\pgfpathlineto{\pgfqpoint{3.803251in}{3.056453in}}%
\pgfpathlineto{\pgfqpoint{3.790253in}{3.062139in}}%
\pgfpathlineto{\pgfqpoint{3.777259in}{3.067864in}}%
\pgfpathlineto{\pgfqpoint{3.764270in}{3.073630in}}%
\pgfpathlineto{\pgfqpoint{3.751285in}{3.079437in}}%
\pgfpathlineto{\pgfqpoint{3.758906in}{3.089905in}}%
\pgfpathlineto{\pgfqpoint{3.766522in}{3.100507in}}%
\pgfpathlineto{\pgfqpoint{3.774133in}{3.111249in}}%
\pgfpathlineto{\pgfqpoint{3.781738in}{3.122134in}}%
\pgfpathclose%
\pgfusepath{fill}%
\end{pgfscope}%
\begin{pgfscope}%
\pgfpathrectangle{\pgfqpoint{1.150000in}{0.150000in}}{\pgfqpoint{5.700000in}{5.700000in}}%
\pgfusepath{clip}%
\pgfsetbuttcap%
\pgfsetroundjoin%
\definecolor{currentfill}{rgb}{0.282656,0.100196,0.422160}%
\pgfsetfillcolor{currentfill}%
\pgfsetfillopacity{0.700000}%
\pgfsetlinewidth{0.000000pt}%
\definecolor{currentstroke}{rgb}{0.000000,0.000000,0.000000}%
\pgfsetstrokecolor{currentstroke}%
\pgfsetdash{}{0pt}%
\pgfpathmoveto{\pgfqpoint{3.430897in}{3.120658in}}%
\pgfpathlineto{\pgfqpoint{3.443822in}{3.114187in}}%
\pgfpathlineto{\pgfqpoint{3.456751in}{3.107766in}}%
\pgfpathlineto{\pgfqpoint{3.469683in}{3.101393in}}%
\pgfpathlineto{\pgfqpoint{3.482619in}{3.095068in}}%
\pgfpathlineto{\pgfqpoint{3.474910in}{3.085222in}}%
\pgfpathlineto{\pgfqpoint{3.467195in}{3.075485in}}%
\pgfpathlineto{\pgfqpoint{3.459474in}{3.065854in}}%
\pgfpathlineto{\pgfqpoint{3.451747in}{3.056325in}}%
\pgfpathlineto{\pgfqpoint{3.438802in}{3.062552in}}%
\pgfpathlineto{\pgfqpoint{3.425860in}{3.068827in}}%
\pgfpathlineto{\pgfqpoint{3.412922in}{3.075151in}}%
\pgfpathlineto{\pgfqpoint{3.399988in}{3.081524in}}%
\pgfpathlineto{\pgfqpoint{3.407725in}{3.091146in}}%
\pgfpathlineto{\pgfqpoint{3.415455in}{3.100873in}}%
\pgfpathlineto{\pgfqpoint{3.423179in}{3.110709in}}%
\pgfpathlineto{\pgfqpoint{3.430897in}{3.120658in}}%
\pgfpathclose%
\pgfusepath{fill}%
\end{pgfscope}%
\begin{pgfscope}%
\pgfpathrectangle{\pgfqpoint{1.150000in}{0.150000in}}{\pgfqpoint{5.700000in}{5.700000in}}%
\pgfusepath{clip}%
\pgfsetbuttcap%
\pgfsetroundjoin%
\definecolor{currentfill}{rgb}{0.282656,0.100196,0.422160}%
\pgfsetfillcolor{currentfill}%
\pgfsetfillopacity{0.700000}%
\pgfsetlinewidth{0.000000pt}%
\definecolor{currentstroke}{rgb}{0.000000,0.000000,0.000000}%
\pgfsetstrokecolor{currentstroke}%
\pgfsetdash{}{0pt}%
\pgfpathmoveto{\pgfqpoint{3.915968in}{3.119589in}}%
\pgfpathlineto{\pgfqpoint{3.928970in}{3.113813in}}%
\pgfpathlineto{\pgfqpoint{3.941976in}{3.108076in}}%
\pgfpathlineto{\pgfqpoint{3.954987in}{3.102377in}}%
\pgfpathlineto{\pgfqpoint{3.968002in}{3.096715in}}%
\pgfpathlineto{\pgfqpoint{3.960442in}{3.085671in}}%
\pgfpathlineto{\pgfqpoint{3.952878in}{3.074779in}}%
\pgfpathlineto{\pgfqpoint{3.945309in}{3.064035in}}%
\pgfpathlineto{\pgfqpoint{3.937736in}{3.053435in}}%
\pgfpathlineto{\pgfqpoint{3.924712in}{3.058948in}}%
\pgfpathlineto{\pgfqpoint{3.911692in}{3.064498in}}%
\pgfpathlineto{\pgfqpoint{3.898677in}{3.070086in}}%
\pgfpathlineto{\pgfqpoint{3.885666in}{3.075712in}}%
\pgfpathlineto{\pgfqpoint{3.893249in}{3.086457in}}%
\pgfpathlineto{\pgfqpoint{3.900827in}{3.097348in}}%
\pgfpathlineto{\pgfqpoint{3.908400in}{3.108390in}}%
\pgfpathlineto{\pgfqpoint{3.915968in}{3.119589in}}%
\pgfpathclose%
\pgfusepath{fill}%
\end{pgfscope}%
\begin{pgfscope}%
\pgfpathrectangle{\pgfqpoint{1.150000in}{0.150000in}}{\pgfqpoint{5.700000in}{5.700000in}}%
\pgfusepath{clip}%
\pgfsetbuttcap%
\pgfsetroundjoin%
\definecolor{currentfill}{rgb}{0.282327,0.094955,0.417331}%
\pgfsetfillcolor{currentfill}%
\pgfsetfillopacity{0.700000}%
\pgfsetlinewidth{0.000000pt}%
\definecolor{currentstroke}{rgb}{0.000000,0.000000,0.000000}%
\pgfsetstrokecolor{currentstroke}%
\pgfsetdash{}{0pt}%
\pgfpathmoveto{\pgfqpoint{3.565138in}{3.110343in}}%
\pgfpathlineto{\pgfqpoint{3.578084in}{3.104141in}}%
\pgfpathlineto{\pgfqpoint{3.591033in}{3.097985in}}%
\pgfpathlineto{\pgfqpoint{3.603986in}{3.091874in}}%
\pgfpathlineto{\pgfqpoint{3.616944in}{3.085807in}}%
\pgfpathlineto{\pgfqpoint{3.609277in}{3.075720in}}%
\pgfpathlineto{\pgfqpoint{3.601604in}{3.065750in}}%
\pgfpathlineto{\pgfqpoint{3.593926in}{3.055894in}}%
\pgfpathlineto{\pgfqpoint{3.586242in}{3.046147in}}%
\pgfpathlineto{\pgfqpoint{3.573275in}{3.052103in}}%
\pgfpathlineto{\pgfqpoint{3.560313in}{3.058104in}}%
\pgfpathlineto{\pgfqpoint{3.547354in}{3.064149in}}%
\pgfpathlineto{\pgfqpoint{3.534399in}{3.070240in}}%
\pgfpathlineto{\pgfqpoint{3.542093in}{3.080093in}}%
\pgfpathlineto{\pgfqpoint{3.549781in}{3.090059in}}%
\pgfpathlineto{\pgfqpoint{3.557462in}{3.100141in}}%
\pgfpathlineto{\pgfqpoint{3.565138in}{3.110343in}}%
\pgfpathclose%
\pgfusepath{fill}%
\end{pgfscope}%
\begin{pgfscope}%
\pgfpathrectangle{\pgfqpoint{1.150000in}{0.150000in}}{\pgfqpoint{5.700000in}{5.700000in}}%
\pgfusepath{clip}%
\pgfsetbuttcap%
\pgfsetroundjoin%
\definecolor{currentfill}{rgb}{0.282327,0.094955,0.417331}%
\pgfsetfillcolor{currentfill}%
\pgfsetfillopacity{0.700000}%
\pgfsetlinewidth{0.000000pt}%
\definecolor{currentstroke}{rgb}{0.000000,0.000000,0.000000}%
\pgfsetstrokecolor{currentstroke}%
\pgfsetdash{}{0pt}%
\pgfpathmoveto{\pgfqpoint{3.699387in}{3.103079in}}%
\pgfpathlineto{\pgfqpoint{3.712356in}{3.097105in}}%
\pgfpathlineto{\pgfqpoint{3.725328in}{3.091174in}}%
\pgfpathlineto{\pgfqpoint{3.738304in}{3.085285in}}%
\pgfpathlineto{\pgfqpoint{3.751285in}{3.079437in}}%
\pgfpathlineto{\pgfqpoint{3.743659in}{3.069100in}}%
\pgfpathlineto{\pgfqpoint{3.736027in}{3.058889in}}%
\pgfpathlineto{\pgfqpoint{3.728390in}{3.048801in}}%
\pgfpathlineto{\pgfqpoint{3.720747in}{3.038830in}}%
\pgfpathlineto{\pgfqpoint{3.707757in}{3.044555in}}%
\pgfpathlineto{\pgfqpoint{3.694771in}{3.050320in}}%
\pgfpathlineto{\pgfqpoint{3.681790in}{3.056128in}}%
\pgfpathlineto{\pgfqpoint{3.668812in}{3.061978in}}%
\pgfpathlineto{\pgfqpoint{3.676464in}{3.072067in}}%
\pgfpathlineto{\pgfqpoint{3.684111in}{3.082277in}}%
\pgfpathlineto{\pgfqpoint{3.691752in}{3.092613in}}%
\pgfpathlineto{\pgfqpoint{3.699387in}{3.103079in}}%
\pgfpathclose%
\pgfusepath{fill}%
\end{pgfscope}%
\begin{pgfscope}%
\pgfpathrectangle{\pgfqpoint{1.150000in}{0.150000in}}{\pgfqpoint{5.700000in}{5.700000in}}%
\pgfusepath{clip}%
\pgfsetbuttcap%
\pgfsetroundjoin%
\definecolor{currentfill}{rgb}{0.282656,0.100196,0.422160}%
\pgfsetfillcolor{currentfill}%
\pgfsetfillopacity{0.700000}%
\pgfsetlinewidth{0.000000pt}%
\definecolor{currentstroke}{rgb}{0.000000,0.000000,0.000000}%
\pgfsetstrokecolor{currentstroke}%
\pgfsetdash{}{0pt}%
\pgfpathmoveto{\pgfqpoint{4.050265in}{3.119596in}}%
\pgfpathlineto{\pgfqpoint{4.063294in}{3.113956in}}%
\pgfpathlineto{\pgfqpoint{4.076328in}{3.108352in}}%
\pgfpathlineto{\pgfqpoint{4.089366in}{3.102783in}}%
\pgfpathlineto{\pgfqpoint{4.102409in}{3.097250in}}%
\pgfpathlineto{\pgfqpoint{4.094885in}{3.085878in}}%
\pgfpathlineto{\pgfqpoint{4.087356in}{3.074672in}}%
\pgfpathlineto{\pgfqpoint{4.079824in}{3.063628in}}%
\pgfpathlineto{\pgfqpoint{4.072287in}{3.052740in}}%
\pgfpathlineto{\pgfqpoint{4.059235in}{3.058111in}}%
\pgfpathlineto{\pgfqpoint{4.046188in}{3.063518in}}%
\pgfpathlineto{\pgfqpoint{4.033145in}{3.068960in}}%
\pgfpathlineto{\pgfqpoint{4.020108in}{3.074438in}}%
\pgfpathlineto{\pgfqpoint{4.027653in}{3.085483in}}%
\pgfpathlineto{\pgfqpoint{4.035195in}{3.096688in}}%
\pgfpathlineto{\pgfqpoint{4.042732in}{3.108057in}}%
\pgfpathlineto{\pgfqpoint{4.050265in}{3.119596in}}%
\pgfpathclose%
\pgfusepath{fill}%
\end{pgfscope}%
\begin{pgfscope}%
\pgfpathrectangle{\pgfqpoint{1.150000in}{0.150000in}}{\pgfqpoint{5.700000in}{5.700000in}}%
\pgfusepath{clip}%
\pgfsetbuttcap%
\pgfsetroundjoin%
\definecolor{currentfill}{rgb}{0.282327,0.094955,0.417331}%
\pgfsetfillcolor{currentfill}%
\pgfsetfillopacity{0.700000}%
\pgfsetlinewidth{0.000000pt}%
\definecolor{currentstroke}{rgb}{0.000000,0.000000,0.000000}%
\pgfsetstrokecolor{currentstroke}%
\pgfsetdash{}{0pt}%
\pgfpathmoveto{\pgfqpoint{3.833668in}{3.098605in}}%
\pgfpathlineto{\pgfqpoint{3.846661in}{3.092823in}}%
\pgfpathlineto{\pgfqpoint{3.859658in}{3.087080in}}%
\pgfpathlineto{\pgfqpoint{3.872660in}{3.081377in}}%
\pgfpathlineto{\pgfqpoint{3.885666in}{3.075712in}}%
\pgfpathlineto{\pgfqpoint{3.878079in}{3.065109in}}%
\pgfpathlineto{\pgfqpoint{3.870486in}{3.054643in}}%
\pgfpathlineto{\pgfqpoint{3.862889in}{3.044310in}}%
\pgfpathlineto{\pgfqpoint{3.855286in}{3.034106in}}%
\pgfpathlineto{\pgfqpoint{3.842270in}{3.039634in}}%
\pgfpathlineto{\pgfqpoint{3.829259in}{3.045201in}}%
\pgfpathlineto{\pgfqpoint{3.816253in}{3.050808in}}%
\pgfpathlineto{\pgfqpoint{3.803251in}{3.056453in}}%
\pgfpathlineto{\pgfqpoint{3.810863in}{3.066789in}}%
\pgfpathlineto{\pgfqpoint{3.818469in}{3.077257in}}%
\pgfpathlineto{\pgfqpoint{3.826071in}{3.087861in}}%
\pgfpathlineto{\pgfqpoint{3.833668in}{3.098605in}}%
\pgfpathclose%
\pgfusepath{fill}%
\end{pgfscope}%
\begin{pgfscope}%
\pgfpathrectangle{\pgfqpoint{1.150000in}{0.150000in}}{\pgfqpoint{5.700000in}{5.700000in}}%
\pgfusepath{clip}%
\pgfsetbuttcap%
\pgfsetroundjoin%
\definecolor{currentfill}{rgb}{0.282656,0.100196,0.422160}%
\pgfsetfillcolor{currentfill}%
\pgfsetfillopacity{0.700000}%
\pgfsetlinewidth{0.000000pt}%
\definecolor{currentstroke}{rgb}{0.000000,0.000000,0.000000}%
\pgfsetstrokecolor{currentstroke}%
\pgfsetdash{}{0pt}%
\pgfpathmoveto{\pgfqpoint{3.348285in}{3.107518in}}%
\pgfpathlineto{\pgfqpoint{3.361206in}{3.100943in}}%
\pgfpathlineto{\pgfqpoint{3.374130in}{3.094420in}}%
\pgfpathlineto{\pgfqpoint{3.387057in}{3.087947in}}%
\pgfpathlineto{\pgfqpoint{3.399988in}{3.081524in}}%
\pgfpathlineto{\pgfqpoint{3.392245in}{3.072005in}}%
\pgfpathlineto{\pgfqpoint{3.384496in}{3.062585in}}%
\pgfpathlineto{\pgfqpoint{3.376741in}{3.053262in}}%
\pgfpathlineto{\pgfqpoint{3.368980in}{3.044033in}}%
\pgfpathlineto{\pgfqpoint{3.356039in}{3.050370in}}%
\pgfpathlineto{\pgfqpoint{3.343102in}{3.056758in}}%
\pgfpathlineto{\pgfqpoint{3.330168in}{3.063197in}}%
\pgfpathlineto{\pgfqpoint{3.317238in}{3.069687in}}%
\pgfpathlineto{\pgfqpoint{3.325009in}{3.078996in}}%
\pgfpathlineto{\pgfqpoint{3.332774in}{3.088403in}}%
\pgfpathlineto{\pgfqpoint{3.340533in}{3.097909in}}%
\pgfpathlineto{\pgfqpoint{3.348285in}{3.107518in}}%
\pgfpathclose%
\pgfusepath{fill}%
\end{pgfscope}%
\begin{pgfscope}%
\pgfpathrectangle{\pgfqpoint{1.150000in}{0.150000in}}{\pgfqpoint{5.700000in}{5.700000in}}%
\pgfusepath{clip}%
\pgfsetbuttcap%
\pgfsetroundjoin%
\definecolor{currentfill}{rgb}{0.282327,0.094955,0.417331}%
\pgfsetfillcolor{currentfill}%
\pgfsetfillopacity{0.700000}%
\pgfsetlinewidth{0.000000pt}%
\definecolor{currentstroke}{rgb}{0.000000,0.000000,0.000000}%
\pgfsetstrokecolor{currentstroke}%
\pgfsetdash{}{0pt}%
\pgfpathmoveto{\pgfqpoint{3.482619in}{3.095068in}}%
\pgfpathlineto{\pgfqpoint{3.495558in}{3.088791in}}%
\pgfpathlineto{\pgfqpoint{3.508502in}{3.082561in}}%
\pgfpathlineto{\pgfqpoint{3.521449in}{3.076377in}}%
\pgfpathlineto{\pgfqpoint{3.534399in}{3.070240in}}%
\pgfpathlineto{\pgfqpoint{3.526700in}{3.060497in}}%
\pgfpathlineto{\pgfqpoint{3.518995in}{3.050859in}}%
\pgfpathlineto{\pgfqpoint{3.511284in}{3.041325in}}%
\pgfpathlineto{\pgfqpoint{3.503566in}{3.031889in}}%
\pgfpathlineto{\pgfqpoint{3.490606in}{3.037928in}}%
\pgfpathlineto{\pgfqpoint{3.477649in}{3.044013in}}%
\pgfpathlineto{\pgfqpoint{3.464697in}{3.050145in}}%
\pgfpathlineto{\pgfqpoint{3.451747in}{3.056325in}}%
\pgfpathlineto{\pgfqpoint{3.459474in}{3.065854in}}%
\pgfpathlineto{\pgfqpoint{3.467195in}{3.075485in}}%
\pgfpathlineto{\pgfqpoint{3.474910in}{3.085222in}}%
\pgfpathlineto{\pgfqpoint{3.482619in}{3.095068in}}%
\pgfpathclose%
\pgfusepath{fill}%
\end{pgfscope}%
\begin{pgfscope}%
\pgfpathrectangle{\pgfqpoint{1.150000in}{0.150000in}}{\pgfqpoint{5.700000in}{5.700000in}}%
\pgfusepath{clip}%
\pgfsetbuttcap%
\pgfsetroundjoin%
\definecolor{currentfill}{rgb}{0.282327,0.094955,0.417331}%
\pgfsetfillcolor{currentfill}%
\pgfsetfillopacity{0.700000}%
\pgfsetlinewidth{0.000000pt}%
\definecolor{currentstroke}{rgb}{0.000000,0.000000,0.000000}%
\pgfsetstrokecolor{currentstroke}%
\pgfsetdash{}{0pt}%
\pgfpathmoveto{\pgfqpoint{3.968002in}{3.096715in}}%
\pgfpathlineto{\pgfqpoint{3.981021in}{3.091091in}}%
\pgfpathlineto{\pgfqpoint{3.994046in}{3.085503in}}%
\pgfpathlineto{\pgfqpoint{4.007074in}{3.079952in}}%
\pgfpathlineto{\pgfqpoint{4.020108in}{3.074438in}}%
\pgfpathlineto{\pgfqpoint{4.012557in}{3.063548in}}%
\pgfpathlineto{\pgfqpoint{4.005002in}{3.052807in}}%
\pgfpathlineto{\pgfqpoint{3.997443in}{3.042211in}}%
\pgfpathlineto{\pgfqpoint{3.989879in}{3.031755in}}%
\pgfpathlineto{\pgfqpoint{3.976836in}{3.037120in}}%
\pgfpathlineto{\pgfqpoint{3.963798in}{3.042521in}}%
\pgfpathlineto{\pgfqpoint{3.950765in}{3.047960in}}%
\pgfpathlineto{\pgfqpoint{3.937736in}{3.053435in}}%
\pgfpathlineto{\pgfqpoint{3.945309in}{3.064035in}}%
\pgfpathlineto{\pgfqpoint{3.952878in}{3.074779in}}%
\pgfpathlineto{\pgfqpoint{3.960442in}{3.085671in}}%
\pgfpathlineto{\pgfqpoint{3.968002in}{3.096715in}}%
\pgfpathclose%
\pgfusepath{fill}%
\end{pgfscope}%
\begin{pgfscope}%
\pgfpathrectangle{\pgfqpoint{1.150000in}{0.150000in}}{\pgfqpoint{5.700000in}{5.700000in}}%
\pgfusepath{clip}%
\pgfsetbuttcap%
\pgfsetroundjoin%
\definecolor{currentfill}{rgb}{0.281924,0.089666,0.412415}%
\pgfsetfillcolor{currentfill}%
\pgfsetfillopacity{0.700000}%
\pgfsetlinewidth{0.000000pt}%
\definecolor{currentstroke}{rgb}{0.000000,0.000000,0.000000}%
\pgfsetstrokecolor{currentstroke}%
\pgfsetdash{}{0pt}%
\pgfpathmoveto{\pgfqpoint{3.616944in}{3.085807in}}%
\pgfpathlineto{\pgfqpoint{3.629905in}{3.079785in}}%
\pgfpathlineto{\pgfqpoint{3.642870in}{3.073806in}}%
\pgfpathlineto{\pgfqpoint{3.655839in}{3.067870in}}%
\pgfpathlineto{\pgfqpoint{3.668812in}{3.061978in}}%
\pgfpathlineto{\pgfqpoint{3.661155in}{3.052007in}}%
\pgfpathlineto{\pgfqpoint{3.653491in}{3.042149in}}%
\pgfpathlineto{\pgfqpoint{3.645823in}{3.032402in}}%
\pgfpathlineto{\pgfqpoint{3.638148in}{3.022761in}}%
\pgfpathlineto{\pgfqpoint{3.625165in}{3.028543in}}%
\pgfpathlineto{\pgfqpoint{3.612187in}{3.034367in}}%
\pgfpathlineto{\pgfqpoint{3.599212in}{3.040235in}}%
\pgfpathlineto{\pgfqpoint{3.586242in}{3.046147in}}%
\pgfpathlineto{\pgfqpoint{3.593926in}{3.055894in}}%
\pgfpathlineto{\pgfqpoint{3.601604in}{3.065750in}}%
\pgfpathlineto{\pgfqpoint{3.609277in}{3.075720in}}%
\pgfpathlineto{\pgfqpoint{3.616944in}{3.085807in}}%
\pgfpathclose%
\pgfusepath{fill}%
\end{pgfscope}%
\begin{pgfscope}%
\pgfpathrectangle{\pgfqpoint{1.150000in}{0.150000in}}{\pgfqpoint{5.700000in}{5.700000in}}%
\pgfusepath{clip}%
\pgfsetbuttcap%
\pgfsetroundjoin%
\definecolor{currentfill}{rgb}{0.281924,0.089666,0.412415}%
\pgfsetfillcolor{currentfill}%
\pgfsetfillopacity{0.700000}%
\pgfsetlinewidth{0.000000pt}%
\definecolor{currentstroke}{rgb}{0.000000,0.000000,0.000000}%
\pgfsetstrokecolor{currentstroke}%
\pgfsetdash{}{0pt}%
\pgfpathmoveto{\pgfqpoint{3.751285in}{3.079437in}}%
\pgfpathlineto{\pgfqpoint{3.764270in}{3.073630in}}%
\pgfpathlineto{\pgfqpoint{3.777259in}{3.067864in}}%
\pgfpathlineto{\pgfqpoint{3.790253in}{3.062139in}}%
\pgfpathlineto{\pgfqpoint{3.803251in}{3.056453in}}%
\pgfpathlineto{\pgfqpoint{3.795633in}{3.046244in}}%
\pgfpathlineto{\pgfqpoint{3.788011in}{3.036159in}}%
\pgfpathlineto{\pgfqpoint{3.780383in}{3.026192in}}%
\pgfpathlineto{\pgfqpoint{3.772750in}{3.016341in}}%
\pgfpathlineto{\pgfqpoint{3.759743in}{3.021903in}}%
\pgfpathlineto{\pgfqpoint{3.746740in}{3.027505in}}%
\pgfpathlineto{\pgfqpoint{3.733741in}{3.033147in}}%
\pgfpathlineto{\pgfqpoint{3.720747in}{3.038830in}}%
\pgfpathlineto{\pgfqpoint{3.728390in}{3.048801in}}%
\pgfpathlineto{\pgfqpoint{3.736027in}{3.058889in}}%
\pgfpathlineto{\pgfqpoint{3.743659in}{3.069100in}}%
\pgfpathlineto{\pgfqpoint{3.751285in}{3.079437in}}%
\pgfpathclose%
\pgfusepath{fill}%
\end{pgfscope}%
\begin{pgfscope}%
\pgfpathrectangle{\pgfqpoint{1.150000in}{0.150000in}}{\pgfqpoint{5.700000in}{5.700000in}}%
\pgfusepath{clip}%
\pgfsetbuttcap%
\pgfsetroundjoin%
\definecolor{currentfill}{rgb}{0.282327,0.094955,0.417331}%
\pgfsetfillcolor{currentfill}%
\pgfsetfillopacity{0.700000}%
\pgfsetlinewidth{0.000000pt}%
\definecolor{currentstroke}{rgb}{0.000000,0.000000,0.000000}%
\pgfsetstrokecolor{currentstroke}%
\pgfsetdash{}{0pt}%
\pgfpathmoveto{\pgfqpoint{4.102409in}{3.097250in}}%
\pgfpathlineto{\pgfqpoint{4.115456in}{3.091752in}}%
\pgfpathlineto{\pgfqpoint{4.128508in}{3.086289in}}%
\pgfpathlineto{\pgfqpoint{4.141565in}{3.080861in}}%
\pgfpathlineto{\pgfqpoint{4.154627in}{3.075467in}}%
\pgfpathlineto{\pgfqpoint{4.147112in}{3.064261in}}%
\pgfpathlineto{\pgfqpoint{4.139593in}{3.053219in}}%
\pgfpathlineto{\pgfqpoint{4.132070in}{3.042336in}}%
\pgfpathlineto{\pgfqpoint{4.124543in}{3.031605in}}%
\pgfpathlineto{\pgfqpoint{4.111472in}{3.036837in}}%
\pgfpathlineto{\pgfqpoint{4.098406in}{3.042103in}}%
\pgfpathlineto{\pgfqpoint{4.085344in}{3.047404in}}%
\pgfpathlineto{\pgfqpoint{4.072287in}{3.052740in}}%
\pgfpathlineto{\pgfqpoint{4.079824in}{3.063628in}}%
\pgfpathlineto{\pgfqpoint{4.087356in}{3.074672in}}%
\pgfpathlineto{\pgfqpoint{4.094885in}{3.085878in}}%
\pgfpathlineto{\pgfqpoint{4.102409in}{3.097250in}}%
\pgfpathclose%
\pgfusepath{fill}%
\end{pgfscope}%
\begin{pgfscope}%
\pgfpathrectangle{\pgfqpoint{1.150000in}{0.150000in}}{\pgfqpoint{5.700000in}{5.700000in}}%
\pgfusepath{clip}%
\pgfsetbuttcap%
\pgfsetroundjoin%
\definecolor{currentfill}{rgb}{0.281924,0.089666,0.412415}%
\pgfsetfillcolor{currentfill}%
\pgfsetfillopacity{0.700000}%
\pgfsetlinewidth{0.000000pt}%
\definecolor{currentstroke}{rgb}{0.000000,0.000000,0.000000}%
\pgfsetstrokecolor{currentstroke}%
\pgfsetdash{}{0pt}%
\pgfpathmoveto{\pgfqpoint{3.885666in}{3.075712in}}%
\pgfpathlineto{\pgfqpoint{3.898677in}{3.070086in}}%
\pgfpathlineto{\pgfqpoint{3.911692in}{3.064498in}}%
\pgfpathlineto{\pgfqpoint{3.924712in}{3.058948in}}%
\pgfpathlineto{\pgfqpoint{3.937736in}{3.053435in}}%
\pgfpathlineto{\pgfqpoint{3.930158in}{3.042973in}}%
\pgfpathlineto{\pgfqpoint{3.922574in}{3.032646in}}%
\pgfpathlineto{\pgfqpoint{3.914986in}{3.022447in}}%
\pgfpathlineto{\pgfqpoint{3.907393in}{3.012375in}}%
\pgfpathlineto{\pgfqpoint{3.894360in}{3.017751in}}%
\pgfpathlineto{\pgfqpoint{3.881330in}{3.023165in}}%
\pgfpathlineto{\pgfqpoint{3.868306in}{3.028616in}}%
\pgfpathlineto{\pgfqpoint{3.855286in}{3.034106in}}%
\pgfpathlineto{\pgfqpoint{3.862889in}{3.044310in}}%
\pgfpathlineto{\pgfqpoint{3.870486in}{3.054643in}}%
\pgfpathlineto{\pgfqpoint{3.878079in}{3.065109in}}%
\pgfpathlineto{\pgfqpoint{3.885666in}{3.075712in}}%
\pgfpathclose%
\pgfusepath{fill}%
\end{pgfscope}%
\begin{pgfscope}%
\pgfpathrectangle{\pgfqpoint{1.150000in}{0.150000in}}{\pgfqpoint{5.700000in}{5.700000in}}%
\pgfusepath{clip}%
\pgfsetbuttcap%
\pgfsetroundjoin%
\definecolor{currentfill}{rgb}{0.281924,0.089666,0.412415}%
\pgfsetfillcolor{currentfill}%
\pgfsetfillopacity{0.700000}%
\pgfsetlinewidth{0.000000pt}%
\definecolor{currentstroke}{rgb}{0.000000,0.000000,0.000000}%
\pgfsetstrokecolor{currentstroke}%
\pgfsetdash{}{0pt}%
\pgfpathmoveto{\pgfqpoint{3.399988in}{3.081524in}}%
\pgfpathlineto{\pgfqpoint{3.412922in}{3.075151in}}%
\pgfpathlineto{\pgfqpoint{3.425860in}{3.068827in}}%
\pgfpathlineto{\pgfqpoint{3.438802in}{3.062552in}}%
\pgfpathlineto{\pgfqpoint{3.451747in}{3.056325in}}%
\pgfpathlineto{\pgfqpoint{3.444015in}{3.046896in}}%
\pgfpathlineto{\pgfqpoint{3.436276in}{3.037563in}}%
\pgfpathlineto{\pgfqpoint{3.428530in}{3.028323in}}%
\pgfpathlineto{\pgfqpoint{3.420779in}{3.019174in}}%
\pgfpathlineto{\pgfqpoint{3.407824in}{3.025316in}}%
\pgfpathlineto{\pgfqpoint{3.394872in}{3.031506in}}%
\pgfpathlineto{\pgfqpoint{3.381924in}{3.037745in}}%
\pgfpathlineto{\pgfqpoint{3.368980in}{3.044033in}}%
\pgfpathlineto{\pgfqpoint{3.376741in}{3.053262in}}%
\pgfpathlineto{\pgfqpoint{3.384496in}{3.062585in}}%
\pgfpathlineto{\pgfqpoint{3.392245in}{3.072005in}}%
\pgfpathlineto{\pgfqpoint{3.399988in}{3.081524in}}%
\pgfpathclose%
\pgfusepath{fill}%
\end{pgfscope}%
\begin{pgfscope}%
\pgfpathrectangle{\pgfqpoint{1.150000in}{0.150000in}}{\pgfqpoint{5.700000in}{5.700000in}}%
\pgfusepath{clip}%
\pgfsetbuttcap%
\pgfsetroundjoin%
\definecolor{currentfill}{rgb}{0.281446,0.084320,0.407414}%
\pgfsetfillcolor{currentfill}%
\pgfsetfillopacity{0.700000}%
\pgfsetlinewidth{0.000000pt}%
\definecolor{currentstroke}{rgb}{0.000000,0.000000,0.000000}%
\pgfsetstrokecolor{currentstroke}%
\pgfsetdash{}{0pt}%
\pgfpathmoveto{\pgfqpoint{3.534399in}{3.070240in}}%
\pgfpathlineto{\pgfqpoint{3.547354in}{3.064149in}}%
\pgfpathlineto{\pgfqpoint{3.560313in}{3.058104in}}%
\pgfpathlineto{\pgfqpoint{3.573275in}{3.052103in}}%
\pgfpathlineto{\pgfqpoint{3.586242in}{3.046147in}}%
\pgfpathlineto{\pgfqpoint{3.578552in}{3.036506in}}%
\pgfpathlineto{\pgfqpoint{3.570856in}{3.026968in}}%
\pgfpathlineto{\pgfqpoint{3.563155in}{3.017530in}}%
\pgfpathlineto{\pgfqpoint{3.555448in}{3.008188in}}%
\pgfpathlineto{\pgfqpoint{3.542471in}{3.014046in}}%
\pgfpathlineto{\pgfqpoint{3.529499in}{3.019948in}}%
\pgfpathlineto{\pgfqpoint{3.516531in}{3.025896in}}%
\pgfpathlineto{\pgfqpoint{3.503566in}{3.031889in}}%
\pgfpathlineto{\pgfqpoint{3.511284in}{3.041325in}}%
\pgfpathlineto{\pgfqpoint{3.518995in}{3.050859in}}%
\pgfpathlineto{\pgfqpoint{3.526700in}{3.060497in}}%
\pgfpathlineto{\pgfqpoint{3.534399in}{3.070240in}}%
\pgfpathclose%
\pgfusepath{fill}%
\end{pgfscope}%
\begin{pgfscope}%
\pgfpathrectangle{\pgfqpoint{1.150000in}{0.150000in}}{\pgfqpoint{5.700000in}{5.700000in}}%
\pgfusepath{clip}%
\pgfsetbuttcap%
\pgfsetroundjoin%
\definecolor{currentfill}{rgb}{0.281924,0.089666,0.412415}%
\pgfsetfillcolor{currentfill}%
\pgfsetfillopacity{0.700000}%
\pgfsetlinewidth{0.000000pt}%
\definecolor{currentstroke}{rgb}{0.000000,0.000000,0.000000}%
\pgfsetstrokecolor{currentstroke}%
\pgfsetdash{}{0pt}%
\pgfpathmoveto{\pgfqpoint{4.020108in}{3.074438in}}%
\pgfpathlineto{\pgfqpoint{4.033145in}{3.068960in}}%
\pgfpathlineto{\pgfqpoint{4.046188in}{3.063518in}}%
\pgfpathlineto{\pgfqpoint{4.059235in}{3.058111in}}%
\pgfpathlineto{\pgfqpoint{4.072287in}{3.052740in}}%
\pgfpathlineto{\pgfqpoint{4.064746in}{3.042004in}}%
\pgfpathlineto{\pgfqpoint{4.057201in}{3.031414in}}%
\pgfpathlineto{\pgfqpoint{4.049651in}{3.020966in}}%
\pgfpathlineto{\pgfqpoint{4.042097in}{3.010654in}}%
\pgfpathlineto{\pgfqpoint{4.029035in}{3.015876in}}%
\pgfpathlineto{\pgfqpoint{4.015978in}{3.021133in}}%
\pgfpathlineto{\pgfqpoint{4.002926in}{3.026426in}}%
\pgfpathlineto{\pgfqpoint{3.989879in}{3.031755in}}%
\pgfpathlineto{\pgfqpoint{3.997443in}{3.042211in}}%
\pgfpathlineto{\pgfqpoint{4.005002in}{3.052807in}}%
\pgfpathlineto{\pgfqpoint{4.012557in}{3.063548in}}%
\pgfpathlineto{\pgfqpoint{4.020108in}{3.074438in}}%
\pgfpathclose%
\pgfusepath{fill}%
\end{pgfscope}%
\begin{pgfscope}%
\pgfpathrectangle{\pgfqpoint{1.150000in}{0.150000in}}{\pgfqpoint{5.700000in}{5.700000in}}%
\pgfusepath{clip}%
\pgfsetbuttcap%
\pgfsetroundjoin%
\definecolor{currentfill}{rgb}{0.281446,0.084320,0.407414}%
\pgfsetfillcolor{currentfill}%
\pgfsetfillopacity{0.700000}%
\pgfsetlinewidth{0.000000pt}%
\definecolor{currentstroke}{rgb}{0.000000,0.000000,0.000000}%
\pgfsetstrokecolor{currentstroke}%
\pgfsetdash{}{0pt}%
\pgfpathmoveto{\pgfqpoint{3.668812in}{3.061978in}}%
\pgfpathlineto{\pgfqpoint{3.681790in}{3.056128in}}%
\pgfpathlineto{\pgfqpoint{3.694771in}{3.050320in}}%
\pgfpathlineto{\pgfqpoint{3.707757in}{3.044555in}}%
\pgfpathlineto{\pgfqpoint{3.720747in}{3.038830in}}%
\pgfpathlineto{\pgfqpoint{3.713099in}{3.028975in}}%
\pgfpathlineto{\pgfqpoint{3.705445in}{3.019230in}}%
\pgfpathlineto{\pgfqpoint{3.697786in}{3.009592in}}%
\pgfpathlineto{\pgfqpoint{3.690121in}{3.000057in}}%
\pgfpathlineto{\pgfqpoint{3.677122in}{3.005670in}}%
\pgfpathlineto{\pgfqpoint{3.664126in}{3.011325in}}%
\pgfpathlineto{\pgfqpoint{3.651135in}{3.017022in}}%
\pgfpathlineto{\pgfqpoint{3.638148in}{3.022761in}}%
\pgfpathlineto{\pgfqpoint{3.645823in}{3.032402in}}%
\pgfpathlineto{\pgfqpoint{3.653491in}{3.042149in}}%
\pgfpathlineto{\pgfqpoint{3.661155in}{3.052007in}}%
\pgfpathlineto{\pgfqpoint{3.668812in}{3.061978in}}%
\pgfpathclose%
\pgfusepath{fill}%
\end{pgfscope}%
\begin{pgfscope}%
\pgfpathrectangle{\pgfqpoint{1.150000in}{0.150000in}}{\pgfqpoint{5.700000in}{5.700000in}}%
\pgfusepath{clip}%
\pgfsetbuttcap%
\pgfsetroundjoin%
\definecolor{currentfill}{rgb}{0.281446,0.084320,0.407414}%
\pgfsetfillcolor{currentfill}%
\pgfsetfillopacity{0.700000}%
\pgfsetlinewidth{0.000000pt}%
\definecolor{currentstroke}{rgb}{0.000000,0.000000,0.000000}%
\pgfsetstrokecolor{currentstroke}%
\pgfsetdash{}{0pt}%
\pgfpathmoveto{\pgfqpoint{3.803251in}{3.056453in}}%
\pgfpathlineto{\pgfqpoint{3.816253in}{3.050808in}}%
\pgfpathlineto{\pgfqpoint{3.829259in}{3.045201in}}%
\pgfpathlineto{\pgfqpoint{3.842270in}{3.039634in}}%
\pgfpathlineto{\pgfqpoint{3.855286in}{3.034106in}}%
\pgfpathlineto{\pgfqpoint{3.847678in}{3.024026in}}%
\pgfpathlineto{\pgfqpoint{3.840065in}{3.014065in}}%
\pgfpathlineto{\pgfqpoint{3.832447in}{3.004221in}}%
\pgfpathlineto{\pgfqpoint{3.824824in}{2.994489in}}%
\pgfpathlineto{\pgfqpoint{3.811799in}{2.999893in}}%
\pgfpathlineto{\pgfqpoint{3.798778in}{3.005337in}}%
\pgfpathlineto{\pgfqpoint{3.785762in}{3.010819in}}%
\pgfpathlineto{\pgfqpoint{3.772750in}{3.016341in}}%
\pgfpathlineto{\pgfqpoint{3.780383in}{3.026192in}}%
\pgfpathlineto{\pgfqpoint{3.788011in}{3.036159in}}%
\pgfpathlineto{\pgfqpoint{3.795633in}{3.046244in}}%
\pgfpathlineto{\pgfqpoint{3.803251in}{3.056453in}}%
\pgfpathclose%
\pgfusepath{fill}%
\end{pgfscope}%
\begin{pgfscope}%
\pgfpathrectangle{\pgfqpoint{1.150000in}{0.150000in}}{\pgfqpoint{5.700000in}{5.700000in}}%
\pgfusepath{clip}%
\pgfsetbuttcap%
\pgfsetroundjoin%
\definecolor{currentfill}{rgb}{0.282327,0.094955,0.417331}%
\pgfsetfillcolor{currentfill}%
\pgfsetfillopacity{0.700000}%
\pgfsetlinewidth{0.000000pt}%
\definecolor{currentstroke}{rgb}{0.000000,0.000000,0.000000}%
\pgfsetstrokecolor{currentstroke}%
\pgfsetdash{}{0pt}%
\pgfpathmoveto{\pgfqpoint{4.154627in}{3.075467in}}%
\pgfpathlineto{\pgfqpoint{4.167693in}{3.070107in}}%
\pgfpathlineto{\pgfqpoint{4.180765in}{3.064781in}}%
\pgfpathlineto{\pgfqpoint{4.193841in}{3.059489in}}%
\pgfpathlineto{\pgfqpoint{4.206922in}{3.054231in}}%
\pgfpathlineto{\pgfqpoint{4.199417in}{3.043192in}}%
\pgfpathlineto{\pgfqpoint{4.191908in}{3.032314in}}%
\pgfpathlineto{\pgfqpoint{4.184394in}{3.021591in}}%
\pgfpathlineto{\pgfqpoint{4.176877in}{3.011018in}}%
\pgfpathlineto{\pgfqpoint{4.163786in}{3.016114in}}%
\pgfpathlineto{\pgfqpoint{4.150700in}{3.021244in}}%
\pgfpathlineto{\pgfqpoint{4.137619in}{3.026408in}}%
\pgfpathlineto{\pgfqpoint{4.124543in}{3.031605in}}%
\pgfpathlineto{\pgfqpoint{4.132070in}{3.042336in}}%
\pgfpathlineto{\pgfqpoint{4.139593in}{3.053219in}}%
\pgfpathlineto{\pgfqpoint{4.147112in}{3.064261in}}%
\pgfpathlineto{\pgfqpoint{4.154627in}{3.075467in}}%
\pgfpathclose%
\pgfusepath{fill}%
\end{pgfscope}%
\begin{pgfscope}%
\pgfpathrectangle{\pgfqpoint{1.150000in}{0.150000in}}{\pgfqpoint{5.700000in}{5.700000in}}%
\pgfusepath{clip}%
\pgfsetbuttcap%
\pgfsetroundjoin%
\definecolor{currentfill}{rgb}{0.281924,0.089666,0.412415}%
\pgfsetfillcolor{currentfill}%
\pgfsetfillopacity{0.700000}%
\pgfsetlinewidth{0.000000pt}%
\definecolor{currentstroke}{rgb}{0.000000,0.000000,0.000000}%
\pgfsetstrokecolor{currentstroke}%
\pgfsetdash{}{0pt}%
\pgfpathmoveto{\pgfqpoint{3.317238in}{3.069687in}}%
\pgfpathlineto{\pgfqpoint{3.330168in}{3.063197in}}%
\pgfpathlineto{\pgfqpoint{3.343102in}{3.056758in}}%
\pgfpathlineto{\pgfqpoint{3.356039in}{3.050370in}}%
\pgfpathlineto{\pgfqpoint{3.368980in}{3.044033in}}%
\pgfpathlineto{\pgfqpoint{3.361213in}{3.034894in}}%
\pgfpathlineto{\pgfqpoint{3.353439in}{3.025844in}}%
\pgfpathlineto{\pgfqpoint{3.345659in}{3.016879in}}%
\pgfpathlineto{\pgfqpoint{3.337873in}{3.007997in}}%
\pgfpathlineto{\pgfqpoint{3.324922in}{3.014262in}}%
\pgfpathlineto{\pgfqpoint{3.311974in}{3.020578in}}%
\pgfpathlineto{\pgfqpoint{3.299030in}{3.026944in}}%
\pgfpathlineto{\pgfqpoint{3.286090in}{3.033362in}}%
\pgfpathlineto{\pgfqpoint{3.293886in}{3.042311in}}%
\pgfpathlineto{\pgfqpoint{3.301676in}{3.051347in}}%
\pgfpathlineto{\pgfqpoint{3.309460in}{3.060471in}}%
\pgfpathlineto{\pgfqpoint{3.317238in}{3.069687in}}%
\pgfpathclose%
\pgfusepath{fill}%
\end{pgfscope}%
\begin{pgfscope}%
\pgfpathrectangle{\pgfqpoint{1.150000in}{0.150000in}}{\pgfqpoint{5.700000in}{5.700000in}}%
\pgfusepath{clip}%
\pgfsetbuttcap%
\pgfsetroundjoin%
\definecolor{currentfill}{rgb}{0.281446,0.084320,0.407414}%
\pgfsetfillcolor{currentfill}%
\pgfsetfillopacity{0.700000}%
\pgfsetlinewidth{0.000000pt}%
\definecolor{currentstroke}{rgb}{0.000000,0.000000,0.000000}%
\pgfsetstrokecolor{currentstroke}%
\pgfsetdash{}{0pt}%
\pgfpathmoveto{\pgfqpoint{3.451747in}{3.056325in}}%
\pgfpathlineto{\pgfqpoint{3.464697in}{3.050145in}}%
\pgfpathlineto{\pgfqpoint{3.477649in}{3.044013in}}%
\pgfpathlineto{\pgfqpoint{3.490606in}{3.037928in}}%
\pgfpathlineto{\pgfqpoint{3.503566in}{3.031889in}}%
\pgfpathlineto{\pgfqpoint{3.495843in}{3.022550in}}%
\pgfpathlineto{\pgfqpoint{3.488114in}{3.013303in}}%
\pgfpathlineto{\pgfqpoint{3.480379in}{3.004147in}}%
\pgfpathlineto{\pgfqpoint{3.472638in}{2.995078in}}%
\pgfpathlineto{\pgfqpoint{3.459668in}{3.001032in}}%
\pgfpathlineto{\pgfqpoint{3.446701in}{3.007032in}}%
\pgfpathlineto{\pgfqpoint{3.433738in}{3.013079in}}%
\pgfpathlineto{\pgfqpoint{3.420779in}{3.019174in}}%
\pgfpathlineto{\pgfqpoint{3.428530in}{3.028323in}}%
\pgfpathlineto{\pgfqpoint{3.436276in}{3.037563in}}%
\pgfpathlineto{\pgfqpoint{3.444015in}{3.046896in}}%
\pgfpathlineto{\pgfqpoint{3.451747in}{3.056325in}}%
\pgfpathclose%
\pgfusepath{fill}%
\end{pgfscope}%
\begin{pgfscope}%
\pgfpathrectangle{\pgfqpoint{1.150000in}{0.150000in}}{\pgfqpoint{5.700000in}{5.700000in}}%
\pgfusepath{clip}%
\pgfsetbuttcap%
\pgfsetroundjoin%
\definecolor{currentfill}{rgb}{0.281446,0.084320,0.407414}%
\pgfsetfillcolor{currentfill}%
\pgfsetfillopacity{0.700000}%
\pgfsetlinewidth{0.000000pt}%
\definecolor{currentstroke}{rgb}{0.000000,0.000000,0.000000}%
\pgfsetstrokecolor{currentstroke}%
\pgfsetdash{}{0pt}%
\pgfpathmoveto{\pgfqpoint{3.937736in}{3.053435in}}%
\pgfpathlineto{\pgfqpoint{3.950765in}{3.047960in}}%
\pgfpathlineto{\pgfqpoint{3.963798in}{3.042521in}}%
\pgfpathlineto{\pgfqpoint{3.976836in}{3.037120in}}%
\pgfpathlineto{\pgfqpoint{3.989879in}{3.031755in}}%
\pgfpathlineto{\pgfqpoint{3.982310in}{3.021434in}}%
\pgfpathlineto{\pgfqpoint{3.974736in}{3.011245in}}%
\pgfpathlineto{\pgfqpoint{3.967158in}{3.001182in}}%
\pgfpathlineto{\pgfqpoint{3.959575in}{2.991241in}}%
\pgfpathlineto{\pgfqpoint{3.946522in}{2.996470in}}%
\pgfpathlineto{\pgfqpoint{3.933475in}{3.001734in}}%
\pgfpathlineto{\pgfqpoint{3.920432in}{3.007036in}}%
\pgfpathlineto{\pgfqpoint{3.907393in}{3.012375in}}%
\pgfpathlineto{\pgfqpoint{3.914986in}{3.022447in}}%
\pgfpathlineto{\pgfqpoint{3.922574in}{3.032646in}}%
\pgfpathlineto{\pgfqpoint{3.930158in}{3.042973in}}%
\pgfpathlineto{\pgfqpoint{3.937736in}{3.053435in}}%
\pgfpathclose%
\pgfusepath{fill}%
\end{pgfscope}%
\begin{pgfscope}%
\pgfpathrectangle{\pgfqpoint{1.150000in}{0.150000in}}{\pgfqpoint{5.700000in}{5.700000in}}%
\pgfusepath{clip}%
\pgfsetbuttcap%
\pgfsetroundjoin%
\definecolor{currentfill}{rgb}{0.280894,0.078907,0.402329}%
\pgfsetfillcolor{currentfill}%
\pgfsetfillopacity{0.700000}%
\pgfsetlinewidth{0.000000pt}%
\definecolor{currentstroke}{rgb}{0.000000,0.000000,0.000000}%
\pgfsetstrokecolor{currentstroke}%
\pgfsetdash{}{0pt}%
\pgfpathmoveto{\pgfqpoint{3.586242in}{3.046147in}}%
\pgfpathlineto{\pgfqpoint{3.599212in}{3.040235in}}%
\pgfpathlineto{\pgfqpoint{3.612187in}{3.034367in}}%
\pgfpathlineto{\pgfqpoint{3.625165in}{3.028543in}}%
\pgfpathlineto{\pgfqpoint{3.638148in}{3.022761in}}%
\pgfpathlineto{\pgfqpoint{3.630468in}{3.013223in}}%
\pgfpathlineto{\pgfqpoint{3.622782in}{3.003785in}}%
\pgfpathlineto{\pgfqpoint{3.615091in}{2.994443in}}%
\pgfpathlineto{\pgfqpoint{3.607394in}{2.985194in}}%
\pgfpathlineto{\pgfqpoint{3.594401in}{2.990877in}}%
\pgfpathlineto{\pgfqpoint{3.581412in}{2.996604in}}%
\pgfpathlineto{\pgfqpoint{3.568428in}{3.002374in}}%
\pgfpathlineto{\pgfqpoint{3.555448in}{3.008188in}}%
\pgfpathlineto{\pgfqpoint{3.563155in}{3.017530in}}%
\pgfpathlineto{\pgfqpoint{3.570856in}{3.026968in}}%
\pgfpathlineto{\pgfqpoint{3.578552in}{3.036506in}}%
\pgfpathlineto{\pgfqpoint{3.586242in}{3.046147in}}%
\pgfpathclose%
\pgfusepath{fill}%
\end{pgfscope}%
\begin{pgfscope}%
\pgfpathrectangle{\pgfqpoint{1.150000in}{0.150000in}}{\pgfqpoint{5.700000in}{5.700000in}}%
\pgfusepath{clip}%
\pgfsetbuttcap%
\pgfsetroundjoin%
\definecolor{currentfill}{rgb}{0.280894,0.078907,0.402329}%
\pgfsetfillcolor{currentfill}%
\pgfsetfillopacity{0.700000}%
\pgfsetlinewidth{0.000000pt}%
\definecolor{currentstroke}{rgb}{0.000000,0.000000,0.000000}%
\pgfsetstrokecolor{currentstroke}%
\pgfsetdash{}{0pt}%
\pgfpathmoveto{\pgfqpoint{3.720747in}{3.038830in}}%
\pgfpathlineto{\pgfqpoint{3.733741in}{3.033147in}}%
\pgfpathlineto{\pgfqpoint{3.746740in}{3.027505in}}%
\pgfpathlineto{\pgfqpoint{3.759743in}{3.021903in}}%
\pgfpathlineto{\pgfqpoint{3.772750in}{3.016341in}}%
\pgfpathlineto{\pgfqpoint{3.765112in}{3.006601in}}%
\pgfpathlineto{\pgfqpoint{3.757468in}{2.996969in}}%
\pgfpathlineto{\pgfqpoint{3.749819in}{2.987440in}}%
\pgfpathlineto{\pgfqpoint{3.742164in}{2.978012in}}%
\pgfpathlineto{\pgfqpoint{3.729147in}{2.983463in}}%
\pgfpathlineto{\pgfqpoint{3.716134in}{2.988953in}}%
\pgfpathlineto{\pgfqpoint{3.703126in}{2.994485in}}%
\pgfpathlineto{\pgfqpoint{3.690121in}{3.000057in}}%
\pgfpathlineto{\pgfqpoint{3.697786in}{3.009592in}}%
\pgfpathlineto{\pgfqpoint{3.705445in}{3.019230in}}%
\pgfpathlineto{\pgfqpoint{3.713099in}{3.028975in}}%
\pgfpathlineto{\pgfqpoint{3.720747in}{3.038830in}}%
\pgfpathclose%
\pgfusepath{fill}%
\end{pgfscope}%
\begin{pgfscope}%
\pgfpathrectangle{\pgfqpoint{1.150000in}{0.150000in}}{\pgfqpoint{5.700000in}{5.700000in}}%
\pgfusepath{clip}%
\pgfsetbuttcap%
\pgfsetroundjoin%
\definecolor{currentfill}{rgb}{0.281446,0.084320,0.407414}%
\pgfsetfillcolor{currentfill}%
\pgfsetfillopacity{0.700000}%
\pgfsetlinewidth{0.000000pt}%
\definecolor{currentstroke}{rgb}{0.000000,0.000000,0.000000}%
\pgfsetstrokecolor{currentstroke}%
\pgfsetdash{}{0pt}%
\pgfpathmoveto{\pgfqpoint{4.072287in}{3.052740in}}%
\pgfpathlineto{\pgfqpoint{4.085344in}{3.047404in}}%
\pgfpathlineto{\pgfqpoint{4.098406in}{3.042103in}}%
\pgfpathlineto{\pgfqpoint{4.111472in}{3.036837in}}%
\pgfpathlineto{\pgfqpoint{4.124543in}{3.031605in}}%
\pgfpathlineto{\pgfqpoint{4.117012in}{3.021023in}}%
\pgfpathlineto{\pgfqpoint{4.109476in}{3.010584in}}%
\pgfpathlineto{\pgfqpoint{4.101936in}{3.000284in}}%
\pgfpathlineto{\pgfqpoint{4.094392in}{2.990118in}}%
\pgfpathlineto{\pgfqpoint{4.081311in}{2.995200in}}%
\pgfpathlineto{\pgfqpoint{4.068234in}{3.000317in}}%
\pgfpathlineto{\pgfqpoint{4.055163in}{3.005468in}}%
\pgfpathlineto{\pgfqpoint{4.042097in}{3.010654in}}%
\pgfpathlineto{\pgfqpoint{4.049651in}{3.020966in}}%
\pgfpathlineto{\pgfqpoint{4.057201in}{3.031414in}}%
\pgfpathlineto{\pgfqpoint{4.064746in}{3.042004in}}%
\pgfpathlineto{\pgfqpoint{4.072287in}{3.052740in}}%
\pgfpathclose%
\pgfusepath{fill}%
\end{pgfscope}%
\begin{pgfscope}%
\pgfpathrectangle{\pgfqpoint{1.150000in}{0.150000in}}{\pgfqpoint{5.700000in}{5.700000in}}%
\pgfusepath{clip}%
\pgfsetbuttcap%
\pgfsetroundjoin%
\definecolor{currentfill}{rgb}{0.280894,0.078907,0.402329}%
\pgfsetfillcolor{currentfill}%
\pgfsetfillopacity{0.700000}%
\pgfsetlinewidth{0.000000pt}%
\definecolor{currentstroke}{rgb}{0.000000,0.000000,0.000000}%
\pgfsetstrokecolor{currentstroke}%
\pgfsetdash{}{0pt}%
\pgfpathmoveto{\pgfqpoint{3.855286in}{3.034106in}}%
\pgfpathlineto{\pgfqpoint{3.868306in}{3.028616in}}%
\pgfpathlineto{\pgfqpoint{3.881330in}{3.023165in}}%
\pgfpathlineto{\pgfqpoint{3.894360in}{3.017751in}}%
\pgfpathlineto{\pgfqpoint{3.907393in}{3.012375in}}%
\pgfpathlineto{\pgfqpoint{3.899795in}{3.002423in}}%
\pgfpathlineto{\pgfqpoint{3.892192in}{2.992588in}}%
\pgfpathlineto{\pgfqpoint{3.884584in}{2.982867in}}%
\pgfpathlineto{\pgfqpoint{3.876971in}{2.973254in}}%
\pgfpathlineto{\pgfqpoint{3.863927in}{2.978506in}}%
\pgfpathlineto{\pgfqpoint{3.850888in}{2.983795in}}%
\pgfpathlineto{\pgfqpoint{3.837854in}{2.989123in}}%
\pgfpathlineto{\pgfqpoint{3.824824in}{2.994489in}}%
\pgfpathlineto{\pgfqpoint{3.832447in}{3.004221in}}%
\pgfpathlineto{\pgfqpoint{3.840065in}{3.014065in}}%
\pgfpathlineto{\pgfqpoint{3.847678in}{3.024026in}}%
\pgfpathlineto{\pgfqpoint{3.855286in}{3.034106in}}%
\pgfpathclose%
\pgfusepath{fill}%
\end{pgfscope}%
\begin{pgfscope}%
\pgfpathrectangle{\pgfqpoint{1.150000in}{0.150000in}}{\pgfqpoint{5.700000in}{5.700000in}}%
\pgfusepath{clip}%
\pgfsetbuttcap%
\pgfsetroundjoin%
\definecolor{currentfill}{rgb}{0.281446,0.084320,0.407414}%
\pgfsetfillcolor{currentfill}%
\pgfsetfillopacity{0.700000}%
\pgfsetlinewidth{0.000000pt}%
\definecolor{currentstroke}{rgb}{0.000000,0.000000,0.000000}%
\pgfsetstrokecolor{currentstroke}%
\pgfsetdash{}{0pt}%
\pgfpathmoveto{\pgfqpoint{3.368980in}{3.044033in}}%
\pgfpathlineto{\pgfqpoint{3.381924in}{3.037745in}}%
\pgfpathlineto{\pgfqpoint{3.394872in}{3.031506in}}%
\pgfpathlineto{\pgfqpoint{3.407824in}{3.025316in}}%
\pgfpathlineto{\pgfqpoint{3.420779in}{3.019174in}}%
\pgfpathlineto{\pgfqpoint{3.413022in}{3.010112in}}%
\pgfpathlineto{\pgfqpoint{3.405259in}{3.001136in}}%
\pgfpathlineto{\pgfqpoint{3.397489in}{2.992241in}}%
\pgfpathlineto{\pgfqpoint{3.389713in}{2.983427in}}%
\pgfpathlineto{\pgfqpoint{3.376748in}{2.989497in}}%
\pgfpathlineto{\pgfqpoint{3.363786in}{2.995615in}}%
\pgfpathlineto{\pgfqpoint{3.350827in}{3.001781in}}%
\pgfpathlineto{\pgfqpoint{3.337873in}{3.007997in}}%
\pgfpathlineto{\pgfqpoint{3.345659in}{3.016879in}}%
\pgfpathlineto{\pgfqpoint{3.353439in}{3.025844in}}%
\pgfpathlineto{\pgfqpoint{3.361213in}{3.034894in}}%
\pgfpathlineto{\pgfqpoint{3.368980in}{3.044033in}}%
\pgfpathclose%
\pgfusepath{fill}%
\end{pgfscope}%
\begin{pgfscope}%
\pgfpathrectangle{\pgfqpoint{1.150000in}{0.150000in}}{\pgfqpoint{5.700000in}{5.700000in}}%
\pgfusepath{clip}%
\pgfsetbuttcap%
\pgfsetroundjoin%
\definecolor{currentfill}{rgb}{0.281924,0.089666,0.412415}%
\pgfsetfillcolor{currentfill}%
\pgfsetfillopacity{0.700000}%
\pgfsetlinewidth{0.000000pt}%
\definecolor{currentstroke}{rgb}{0.000000,0.000000,0.000000}%
\pgfsetstrokecolor{currentstroke}%
\pgfsetdash{}{0pt}%
\pgfpathmoveto{\pgfqpoint{4.206922in}{3.054231in}}%
\pgfpathlineto{\pgfqpoint{4.220008in}{3.049006in}}%
\pgfpathlineto{\pgfqpoint{4.233099in}{3.043814in}}%
\pgfpathlineto{\pgfqpoint{4.246195in}{3.038655in}}%
\pgfpathlineto{\pgfqpoint{4.259296in}{3.033529in}}%
\pgfpathlineto{\pgfqpoint{4.251800in}{3.022657in}}%
\pgfpathlineto{\pgfqpoint{4.244301in}{3.011943in}}%
\pgfpathlineto{\pgfqpoint{4.236798in}{3.001380in}}%
\pgfpathlineto{\pgfqpoint{4.229291in}{2.990965in}}%
\pgfpathlineto{\pgfqpoint{4.216180in}{2.995929in}}%
\pgfpathlineto{\pgfqpoint{4.203074in}{3.000926in}}%
\pgfpathlineto{\pgfqpoint{4.189973in}{3.005955in}}%
\pgfpathlineto{\pgfqpoint{4.176877in}{3.011018in}}%
\pgfpathlineto{\pgfqpoint{4.184394in}{3.021591in}}%
\pgfpathlineto{\pgfqpoint{4.191908in}{3.032314in}}%
\pgfpathlineto{\pgfqpoint{4.199417in}{3.043192in}}%
\pgfpathlineto{\pgfqpoint{4.206922in}{3.054231in}}%
\pgfpathclose%
\pgfusepath{fill}%
\end{pgfscope}%
\begin{pgfscope}%
\pgfpathrectangle{\pgfqpoint{1.150000in}{0.150000in}}{\pgfqpoint{5.700000in}{5.700000in}}%
\pgfusepath{clip}%
\pgfsetbuttcap%
\pgfsetroundjoin%
\definecolor{currentfill}{rgb}{0.280894,0.078907,0.402329}%
\pgfsetfillcolor{currentfill}%
\pgfsetfillopacity{0.700000}%
\pgfsetlinewidth{0.000000pt}%
\definecolor{currentstroke}{rgb}{0.000000,0.000000,0.000000}%
\pgfsetstrokecolor{currentstroke}%
\pgfsetdash{}{0pt}%
\pgfpathmoveto{\pgfqpoint{3.503566in}{3.031889in}}%
\pgfpathlineto{\pgfqpoint{3.516531in}{3.025896in}}%
\pgfpathlineto{\pgfqpoint{3.529499in}{3.019948in}}%
\pgfpathlineto{\pgfqpoint{3.542471in}{3.014046in}}%
\pgfpathlineto{\pgfqpoint{3.555448in}{3.008188in}}%
\pgfpathlineto{\pgfqpoint{3.547735in}{2.998938in}}%
\pgfpathlineto{\pgfqpoint{3.540016in}{2.989779in}}%
\pgfpathlineto{\pgfqpoint{3.532291in}{2.980707in}}%
\pgfpathlineto{\pgfqpoint{3.524560in}{2.971718in}}%
\pgfpathlineto{\pgfqpoint{3.511574in}{2.977491in}}%
\pgfpathlineto{\pgfqpoint{3.498591in}{2.983308in}}%
\pgfpathlineto{\pgfqpoint{3.485613in}{2.989171in}}%
\pgfpathlineto{\pgfqpoint{3.472638in}{2.995078in}}%
\pgfpathlineto{\pgfqpoint{3.480379in}{3.004147in}}%
\pgfpathlineto{\pgfqpoint{3.488114in}{3.013303in}}%
\pgfpathlineto{\pgfqpoint{3.495843in}{3.022550in}}%
\pgfpathlineto{\pgfqpoint{3.503566in}{3.031889in}}%
\pgfpathclose%
\pgfusepath{fill}%
\end{pgfscope}%
\begin{pgfscope}%
\pgfpathrectangle{\pgfqpoint{1.150000in}{0.150000in}}{\pgfqpoint{5.700000in}{5.700000in}}%
\pgfusepath{clip}%
\pgfsetbuttcap%
\pgfsetroundjoin%
\definecolor{currentfill}{rgb}{0.280894,0.078907,0.402329}%
\pgfsetfillcolor{currentfill}%
\pgfsetfillopacity{0.700000}%
\pgfsetlinewidth{0.000000pt}%
\definecolor{currentstroke}{rgb}{0.000000,0.000000,0.000000}%
\pgfsetstrokecolor{currentstroke}%
\pgfsetdash{}{0pt}%
\pgfpathmoveto{\pgfqpoint{3.989879in}{3.031755in}}%
\pgfpathlineto{\pgfqpoint{4.002926in}{3.026426in}}%
\pgfpathlineto{\pgfqpoint{4.015978in}{3.021133in}}%
\pgfpathlineto{\pgfqpoint{4.029035in}{3.015876in}}%
\pgfpathlineto{\pgfqpoint{4.042097in}{3.010654in}}%
\pgfpathlineto{\pgfqpoint{4.034538in}{3.000476in}}%
\pgfpathlineto{\pgfqpoint{4.026974in}{2.990424in}}%
\pgfpathlineto{\pgfqpoint{4.019406in}{2.980497in}}%
\pgfpathlineto{\pgfqpoint{4.011832in}{2.970688in}}%
\pgfpathlineto{\pgfqpoint{3.998761in}{2.975773in}}%
\pgfpathlineto{\pgfqpoint{3.985694in}{2.980893in}}%
\pgfpathlineto{\pgfqpoint{3.972632in}{2.986049in}}%
\pgfpathlineto{\pgfqpoint{3.959575in}{2.991241in}}%
\pgfpathlineto{\pgfqpoint{3.967158in}{3.001182in}}%
\pgfpathlineto{\pgfqpoint{3.974736in}{3.011245in}}%
\pgfpathlineto{\pgfqpoint{3.982310in}{3.021434in}}%
\pgfpathlineto{\pgfqpoint{3.989879in}{3.031755in}}%
\pgfpathclose%
\pgfusepath{fill}%
\end{pgfscope}%
\begin{pgfscope}%
\pgfpathrectangle{\pgfqpoint{1.150000in}{0.150000in}}{\pgfqpoint{5.700000in}{5.700000in}}%
\pgfusepath{clip}%
\pgfsetbuttcap%
\pgfsetroundjoin%
\definecolor{currentfill}{rgb}{0.280267,0.073417,0.397163}%
\pgfsetfillcolor{currentfill}%
\pgfsetfillopacity{0.700000}%
\pgfsetlinewidth{0.000000pt}%
\definecolor{currentstroke}{rgb}{0.000000,0.000000,0.000000}%
\pgfsetstrokecolor{currentstroke}%
\pgfsetdash{}{0pt}%
\pgfpathmoveto{\pgfqpoint{3.638148in}{3.022761in}}%
\pgfpathlineto{\pgfqpoint{3.651135in}{3.017022in}}%
\pgfpathlineto{\pgfqpoint{3.664126in}{3.011325in}}%
\pgfpathlineto{\pgfqpoint{3.677122in}{3.005670in}}%
\pgfpathlineto{\pgfqpoint{3.690121in}{3.000057in}}%
\pgfpathlineto{\pgfqpoint{3.682451in}{2.990622in}}%
\pgfpathlineto{\pgfqpoint{3.674776in}{2.981284in}}%
\pgfpathlineto{\pgfqpoint{3.667094in}{2.972038in}}%
\pgfpathlineto{\pgfqpoint{3.659407in}{2.962883in}}%
\pgfpathlineto{\pgfqpoint{3.646397in}{2.968398in}}%
\pgfpathlineto{\pgfqpoint{3.633392in}{2.973954in}}%
\pgfpathlineto{\pgfqpoint{3.620391in}{2.979553in}}%
\pgfpathlineto{\pgfqpoint{3.607394in}{2.985194in}}%
\pgfpathlineto{\pgfqpoint{3.615091in}{2.994443in}}%
\pgfpathlineto{\pgfqpoint{3.622782in}{3.003785in}}%
\pgfpathlineto{\pgfqpoint{3.630468in}{3.013223in}}%
\pgfpathlineto{\pgfqpoint{3.638148in}{3.022761in}}%
\pgfpathclose%
\pgfusepath{fill}%
\end{pgfscope}%
\begin{pgfscope}%
\pgfpathrectangle{\pgfqpoint{1.150000in}{0.150000in}}{\pgfqpoint{5.700000in}{5.700000in}}%
\pgfusepath{clip}%
\pgfsetbuttcap%
\pgfsetroundjoin%
\definecolor{currentfill}{rgb}{0.280267,0.073417,0.397163}%
\pgfsetfillcolor{currentfill}%
\pgfsetfillopacity{0.700000}%
\pgfsetlinewidth{0.000000pt}%
\definecolor{currentstroke}{rgb}{0.000000,0.000000,0.000000}%
\pgfsetstrokecolor{currentstroke}%
\pgfsetdash{}{0pt}%
\pgfpathmoveto{\pgfqpoint{3.772750in}{3.016341in}}%
\pgfpathlineto{\pgfqpoint{3.785762in}{3.010819in}}%
\pgfpathlineto{\pgfqpoint{3.798778in}{3.005337in}}%
\pgfpathlineto{\pgfqpoint{3.811799in}{2.999893in}}%
\pgfpathlineto{\pgfqpoint{3.824824in}{2.994489in}}%
\pgfpathlineto{\pgfqpoint{3.817195in}{2.984865in}}%
\pgfpathlineto{\pgfqpoint{3.809561in}{2.975345in}}%
\pgfpathlineto{\pgfqpoint{3.801922in}{2.965926in}}%
\pgfpathlineto{\pgfqpoint{3.794278in}{2.956604in}}%
\pgfpathlineto{\pgfqpoint{3.781242in}{2.961897in}}%
\pgfpathlineto{\pgfqpoint{3.768212in}{2.967230in}}%
\pgfpathlineto{\pgfqpoint{3.755186in}{2.972601in}}%
\pgfpathlineto{\pgfqpoint{3.742164in}{2.978012in}}%
\pgfpathlineto{\pgfqpoint{3.749819in}{2.987440in}}%
\pgfpathlineto{\pgfqpoint{3.757468in}{2.996969in}}%
\pgfpathlineto{\pgfqpoint{3.765112in}{3.006601in}}%
\pgfpathlineto{\pgfqpoint{3.772750in}{3.016341in}}%
\pgfpathclose%
\pgfusepath{fill}%
\end{pgfscope}%
\begin{pgfscope}%
\pgfpathrectangle{\pgfqpoint{1.150000in}{0.150000in}}{\pgfqpoint{5.700000in}{5.700000in}}%
\pgfusepath{clip}%
\pgfsetbuttcap%
\pgfsetroundjoin%
\definecolor{currentfill}{rgb}{0.280894,0.078907,0.402329}%
\pgfsetfillcolor{currentfill}%
\pgfsetfillopacity{0.700000}%
\pgfsetlinewidth{0.000000pt}%
\definecolor{currentstroke}{rgb}{0.000000,0.000000,0.000000}%
\pgfsetstrokecolor{currentstroke}%
\pgfsetdash{}{0pt}%
\pgfpathmoveto{\pgfqpoint{4.124543in}{3.031605in}}%
\pgfpathlineto{\pgfqpoint{4.137619in}{3.026408in}}%
\pgfpathlineto{\pgfqpoint{4.150700in}{3.021244in}}%
\pgfpathlineto{\pgfqpoint{4.163786in}{3.016114in}}%
\pgfpathlineto{\pgfqpoint{4.176877in}{3.011018in}}%
\pgfpathlineto{\pgfqpoint{4.169356in}{3.000590in}}%
\pgfpathlineto{\pgfqpoint{4.161830in}{2.990303in}}%
\pgfpathlineto{\pgfqpoint{4.154300in}{2.980151in}}%
\pgfpathlineto{\pgfqpoint{4.146766in}{2.970129in}}%
\pgfpathlineto{\pgfqpoint{4.133665in}{2.975076in}}%
\pgfpathlineto{\pgfqpoint{4.120569in}{2.980056in}}%
\pgfpathlineto{\pgfqpoint{4.107478in}{2.985070in}}%
\pgfpathlineto{\pgfqpoint{4.094392in}{2.990118in}}%
\pgfpathlineto{\pgfqpoint{4.101936in}{3.000284in}}%
\pgfpathlineto{\pgfqpoint{4.109476in}{3.010584in}}%
\pgfpathlineto{\pgfqpoint{4.117012in}{3.021023in}}%
\pgfpathlineto{\pgfqpoint{4.124543in}{3.031605in}}%
\pgfpathclose%
\pgfusepath{fill}%
\end{pgfscope}%
\begin{pgfscope}%
\pgfpathrectangle{\pgfqpoint{1.150000in}{0.150000in}}{\pgfqpoint{5.700000in}{5.700000in}}%
\pgfusepath{clip}%
\pgfsetbuttcap%
\pgfsetroundjoin%
\definecolor{currentfill}{rgb}{0.280267,0.073417,0.397163}%
\pgfsetfillcolor{currentfill}%
\pgfsetfillopacity{0.700000}%
\pgfsetlinewidth{0.000000pt}%
\definecolor{currentstroke}{rgb}{0.000000,0.000000,0.000000}%
\pgfsetstrokecolor{currentstroke}%
\pgfsetdash{}{0pt}%
\pgfpathmoveto{\pgfqpoint{3.907393in}{3.012375in}}%
\pgfpathlineto{\pgfqpoint{3.920432in}{3.007036in}}%
\pgfpathlineto{\pgfqpoint{3.933475in}{3.001734in}}%
\pgfpathlineto{\pgfqpoint{3.946522in}{2.996470in}}%
\pgfpathlineto{\pgfqpoint{3.959575in}{2.991241in}}%
\pgfpathlineto{\pgfqpoint{3.951987in}{2.981418in}}%
\pgfpathlineto{\pgfqpoint{3.944394in}{2.971709in}}%
\pgfpathlineto{\pgfqpoint{3.936795in}{2.962110in}}%
\pgfpathlineto{\pgfqpoint{3.929192in}{2.952616in}}%
\pgfpathlineto{\pgfqpoint{3.916130in}{2.957720in}}%
\pgfpathlineto{\pgfqpoint{3.903072in}{2.962861in}}%
\pgfpathlineto{\pgfqpoint{3.890019in}{2.968039in}}%
\pgfpathlineto{\pgfqpoint{3.876971in}{2.973254in}}%
\pgfpathlineto{\pgfqpoint{3.884584in}{2.982867in}}%
\pgfpathlineto{\pgfqpoint{3.892192in}{2.992588in}}%
\pgfpathlineto{\pgfqpoint{3.899795in}{3.002423in}}%
\pgfpathlineto{\pgfqpoint{3.907393in}{3.012375in}}%
\pgfpathclose%
\pgfusepath{fill}%
\end{pgfscope}%
\begin{pgfscope}%
\pgfpathrectangle{\pgfqpoint{1.150000in}{0.150000in}}{\pgfqpoint{5.700000in}{5.700000in}}%
\pgfusepath{clip}%
\pgfsetbuttcap%
\pgfsetroundjoin%
\definecolor{currentfill}{rgb}{0.281446,0.084320,0.407414}%
\pgfsetfillcolor{currentfill}%
\pgfsetfillopacity{0.700000}%
\pgfsetlinewidth{0.000000pt}%
\definecolor{currentstroke}{rgb}{0.000000,0.000000,0.000000}%
\pgfsetstrokecolor{currentstroke}%
\pgfsetdash{}{0pt}%
\pgfpathmoveto{\pgfqpoint{3.286090in}{3.033362in}}%
\pgfpathlineto{\pgfqpoint{3.299030in}{3.026944in}}%
\pgfpathlineto{\pgfqpoint{3.311974in}{3.020578in}}%
\pgfpathlineto{\pgfqpoint{3.324922in}{3.014262in}}%
\pgfpathlineto{\pgfqpoint{3.337873in}{3.007997in}}%
\pgfpathlineto{\pgfqpoint{3.330080in}{2.999196in}}%
\pgfpathlineto{\pgfqpoint{3.322282in}{2.990473in}}%
\pgfpathlineto{\pgfqpoint{3.314477in}{2.981826in}}%
\pgfpathlineto{\pgfqpoint{3.306665in}{2.973253in}}%
\pgfpathlineto{\pgfqpoint{3.293703in}{2.979458in}}%
\pgfpathlineto{\pgfqpoint{3.280745in}{2.985715in}}%
\pgfpathlineto{\pgfqpoint{3.267790in}{2.992022in}}%
\pgfpathlineto{\pgfqpoint{3.254839in}{2.998380in}}%
\pgfpathlineto{\pgfqpoint{3.262661in}{3.007008in}}%
\pgfpathlineto{\pgfqpoint{3.270477in}{3.015712in}}%
\pgfpathlineto{\pgfqpoint{3.278287in}{3.024496in}}%
\pgfpathlineto{\pgfqpoint{3.286090in}{3.033362in}}%
\pgfpathclose%
\pgfusepath{fill}%
\end{pgfscope}%
\begin{pgfscope}%
\pgfpathrectangle{\pgfqpoint{1.150000in}{0.150000in}}{\pgfqpoint{5.700000in}{5.700000in}}%
\pgfusepath{clip}%
\pgfsetbuttcap%
\pgfsetroundjoin%
\definecolor{currentfill}{rgb}{0.280894,0.078907,0.402329}%
\pgfsetfillcolor{currentfill}%
\pgfsetfillopacity{0.700000}%
\pgfsetlinewidth{0.000000pt}%
\definecolor{currentstroke}{rgb}{0.000000,0.000000,0.000000}%
\pgfsetstrokecolor{currentstroke}%
\pgfsetdash{}{0pt}%
\pgfpathmoveto{\pgfqpoint{3.420779in}{3.019174in}}%
\pgfpathlineto{\pgfqpoint{3.433738in}{3.013079in}}%
\pgfpathlineto{\pgfqpoint{3.446701in}{3.007032in}}%
\pgfpathlineto{\pgfqpoint{3.459668in}{3.001032in}}%
\pgfpathlineto{\pgfqpoint{3.472638in}{2.995078in}}%
\pgfpathlineto{\pgfqpoint{3.464891in}{2.986094in}}%
\pgfpathlineto{\pgfqpoint{3.457138in}{2.977192in}}%
\pgfpathlineto{\pgfqpoint{3.449379in}{2.968368in}}%
\pgfpathlineto{\pgfqpoint{3.441614in}{2.959622in}}%
\pgfpathlineto{\pgfqpoint{3.428633in}{2.965503in}}%
\pgfpathlineto{\pgfqpoint{3.415656in}{2.971431in}}%
\pgfpathlineto{\pgfqpoint{3.402683in}{2.977405in}}%
\pgfpathlineto{\pgfqpoint{3.389713in}{2.983427in}}%
\pgfpathlineto{\pgfqpoint{3.397489in}{2.992241in}}%
\pgfpathlineto{\pgfqpoint{3.405259in}{3.001136in}}%
\pgfpathlineto{\pgfqpoint{3.413022in}{3.010112in}}%
\pgfpathlineto{\pgfqpoint{3.420779in}{3.019174in}}%
\pgfpathclose%
\pgfusepath{fill}%
\end{pgfscope}%
\begin{pgfscope}%
\pgfpathrectangle{\pgfqpoint{1.150000in}{0.150000in}}{\pgfqpoint{5.700000in}{5.700000in}}%
\pgfusepath{clip}%
\pgfsetbuttcap%
\pgfsetroundjoin%
\definecolor{currentfill}{rgb}{0.280267,0.073417,0.397163}%
\pgfsetfillcolor{currentfill}%
\pgfsetfillopacity{0.700000}%
\pgfsetlinewidth{0.000000pt}%
\definecolor{currentstroke}{rgb}{0.000000,0.000000,0.000000}%
\pgfsetstrokecolor{currentstroke}%
\pgfsetdash{}{0pt}%
\pgfpathmoveto{\pgfqpoint{3.555448in}{3.008188in}}%
\pgfpathlineto{\pgfqpoint{3.568428in}{3.002374in}}%
\pgfpathlineto{\pgfqpoint{3.581412in}{2.996604in}}%
\pgfpathlineto{\pgfqpoint{3.594401in}{2.990877in}}%
\pgfpathlineto{\pgfqpoint{3.607394in}{2.985194in}}%
\pgfpathlineto{\pgfqpoint{3.599691in}{2.976035in}}%
\pgfpathlineto{\pgfqpoint{3.591982in}{2.966962in}}%
\pgfpathlineto{\pgfqpoint{3.584268in}{2.957974in}}%
\pgfpathlineto{\pgfqpoint{3.576547in}{2.949066in}}%
\pgfpathlineto{\pgfqpoint{3.563544in}{2.954664in}}%
\pgfpathlineto{\pgfqpoint{3.550545in}{2.960305in}}%
\pgfpathlineto{\pgfqpoint{3.537551in}{2.965990in}}%
\pgfpathlineto{\pgfqpoint{3.524560in}{2.971718in}}%
\pgfpathlineto{\pgfqpoint{3.532291in}{2.980707in}}%
\pgfpathlineto{\pgfqpoint{3.540016in}{2.989779in}}%
\pgfpathlineto{\pgfqpoint{3.547735in}{2.998938in}}%
\pgfpathlineto{\pgfqpoint{3.555448in}{3.008188in}}%
\pgfpathclose%
\pgfusepath{fill}%
\end{pgfscope}%
\begin{pgfscope}%
\pgfpathrectangle{\pgfqpoint{1.150000in}{0.150000in}}{\pgfqpoint{5.700000in}{5.700000in}}%
\pgfusepath{clip}%
\pgfsetbuttcap%
\pgfsetroundjoin%
\definecolor{currentfill}{rgb}{0.281446,0.084320,0.407414}%
\pgfsetfillcolor{currentfill}%
\pgfsetfillopacity{0.700000}%
\pgfsetlinewidth{0.000000pt}%
\definecolor{currentstroke}{rgb}{0.000000,0.000000,0.000000}%
\pgfsetstrokecolor{currentstroke}%
\pgfsetdash{}{0pt}%
\pgfpathmoveto{\pgfqpoint{4.259296in}{3.033529in}}%
\pgfpathlineto{\pgfqpoint{4.272402in}{3.028435in}}%
\pgfpathlineto{\pgfqpoint{4.285513in}{3.023373in}}%
\pgfpathlineto{\pgfqpoint{4.298629in}{3.018344in}}%
\pgfpathlineto{\pgfqpoint{4.311750in}{3.013347in}}%
\pgfpathlineto{\pgfqpoint{4.304265in}{3.002642in}}%
\pgfpathlineto{\pgfqpoint{4.296775in}{2.992092in}}%
\pgfpathlineto{\pgfqpoint{4.289283in}{2.981691in}}%
\pgfpathlineto{\pgfqpoint{4.281786in}{2.971433in}}%
\pgfpathlineto{\pgfqpoint{4.268654in}{2.976268in}}%
\pgfpathlineto{\pgfqpoint{4.255528in}{2.981135in}}%
\pgfpathlineto{\pgfqpoint{4.242407in}{2.986034in}}%
\pgfpathlineto{\pgfqpoint{4.229291in}{2.990965in}}%
\pgfpathlineto{\pgfqpoint{4.236798in}{3.001380in}}%
\pgfpathlineto{\pgfqpoint{4.244301in}{3.011943in}}%
\pgfpathlineto{\pgfqpoint{4.251800in}{3.022657in}}%
\pgfpathlineto{\pgfqpoint{4.259296in}{3.033529in}}%
\pgfpathclose%
\pgfusepath{fill}%
\end{pgfscope}%
\begin{pgfscope}%
\pgfpathrectangle{\pgfqpoint{1.150000in}{0.150000in}}{\pgfqpoint{5.700000in}{5.700000in}}%
\pgfusepath{clip}%
\pgfsetbuttcap%
\pgfsetroundjoin%
\definecolor{currentfill}{rgb}{0.280267,0.073417,0.397163}%
\pgfsetfillcolor{currentfill}%
\pgfsetfillopacity{0.700000}%
\pgfsetlinewidth{0.000000pt}%
\definecolor{currentstroke}{rgb}{0.000000,0.000000,0.000000}%
\pgfsetstrokecolor{currentstroke}%
\pgfsetdash{}{0pt}%
\pgfpathmoveto{\pgfqpoint{4.042097in}{3.010654in}}%
\pgfpathlineto{\pgfqpoint{4.055163in}{3.005468in}}%
\pgfpathlineto{\pgfqpoint{4.068234in}{3.000317in}}%
\pgfpathlineto{\pgfqpoint{4.081311in}{2.995200in}}%
\pgfpathlineto{\pgfqpoint{4.094392in}{2.990118in}}%
\pgfpathlineto{\pgfqpoint{4.086843in}{2.980080in}}%
\pgfpathlineto{\pgfqpoint{4.079289in}{2.970168in}}%
\pgfpathlineto{\pgfqpoint{4.071731in}{2.960375in}}%
\pgfpathlineto{\pgfqpoint{4.064168in}{2.950699in}}%
\pgfpathlineto{\pgfqpoint{4.051077in}{2.955644in}}%
\pgfpathlineto{\pgfqpoint{4.037991in}{2.960624in}}%
\pgfpathlineto{\pgfqpoint{4.024909in}{2.965638in}}%
\pgfpathlineto{\pgfqpoint{4.011832in}{2.970688in}}%
\pgfpathlineto{\pgfqpoint{4.019406in}{2.980497in}}%
\pgfpathlineto{\pgfqpoint{4.026974in}{2.990424in}}%
\pgfpathlineto{\pgfqpoint{4.034538in}{3.000476in}}%
\pgfpathlineto{\pgfqpoint{4.042097in}{3.010654in}}%
\pgfpathclose%
\pgfusepath{fill}%
\end{pgfscope}%
\begin{pgfscope}%
\pgfpathrectangle{\pgfqpoint{1.150000in}{0.150000in}}{\pgfqpoint{5.700000in}{5.700000in}}%
\pgfusepath{clip}%
\pgfsetbuttcap%
\pgfsetroundjoin%
\definecolor{currentfill}{rgb}{0.279566,0.067836,0.391917}%
\pgfsetfillcolor{currentfill}%
\pgfsetfillopacity{0.700000}%
\pgfsetlinewidth{0.000000pt}%
\definecolor{currentstroke}{rgb}{0.000000,0.000000,0.000000}%
\pgfsetstrokecolor{currentstroke}%
\pgfsetdash{}{0pt}%
\pgfpathmoveto{\pgfqpoint{3.690121in}{3.000057in}}%
\pgfpathlineto{\pgfqpoint{3.703126in}{2.994485in}}%
\pgfpathlineto{\pgfqpoint{3.716134in}{2.988953in}}%
\pgfpathlineto{\pgfqpoint{3.729147in}{2.983463in}}%
\pgfpathlineto{\pgfqpoint{3.742164in}{2.978012in}}%
\pgfpathlineto{\pgfqpoint{3.734504in}{2.968680in}}%
\pgfpathlineto{\pgfqpoint{3.726838in}{2.959442in}}%
\pgfpathlineto{\pgfqpoint{3.719167in}{2.950293in}}%
\pgfpathlineto{\pgfqpoint{3.711490in}{2.941231in}}%
\pgfpathlineto{\pgfqpoint{3.698463in}{2.946583in}}%
\pgfpathlineto{\pgfqpoint{3.685440in}{2.951976in}}%
\pgfpathlineto{\pgfqpoint{3.672421in}{2.957409in}}%
\pgfpathlineto{\pgfqpoint{3.659407in}{2.962883in}}%
\pgfpathlineto{\pgfqpoint{3.667094in}{2.972038in}}%
\pgfpathlineto{\pgfqpoint{3.674776in}{2.981284in}}%
\pgfpathlineto{\pgfqpoint{3.682451in}{2.990622in}}%
\pgfpathlineto{\pgfqpoint{3.690121in}{3.000057in}}%
\pgfpathclose%
\pgfusepath{fill}%
\end{pgfscope}%
\begin{pgfscope}%
\pgfpathrectangle{\pgfqpoint{1.150000in}{0.150000in}}{\pgfqpoint{5.700000in}{5.700000in}}%
\pgfusepath{clip}%
\pgfsetbuttcap%
\pgfsetroundjoin%
\definecolor{currentfill}{rgb}{0.279566,0.067836,0.391917}%
\pgfsetfillcolor{currentfill}%
\pgfsetfillopacity{0.700000}%
\pgfsetlinewidth{0.000000pt}%
\definecolor{currentstroke}{rgb}{0.000000,0.000000,0.000000}%
\pgfsetstrokecolor{currentstroke}%
\pgfsetdash{}{0pt}%
\pgfpathmoveto{\pgfqpoint{3.824824in}{2.994489in}}%
\pgfpathlineto{\pgfqpoint{3.837854in}{2.989123in}}%
\pgfpathlineto{\pgfqpoint{3.850888in}{2.983795in}}%
\pgfpathlineto{\pgfqpoint{3.863927in}{2.978506in}}%
\pgfpathlineto{\pgfqpoint{3.876971in}{2.973254in}}%
\pgfpathlineto{\pgfqpoint{3.869352in}{2.963746in}}%
\pgfpathlineto{\pgfqpoint{3.861728in}{2.954339in}}%
\pgfpathlineto{\pgfqpoint{3.854100in}{2.945029in}}%
\pgfpathlineto{\pgfqpoint{3.846465in}{2.935814in}}%
\pgfpathlineto{\pgfqpoint{3.833411in}{2.940955in}}%
\pgfpathlineto{\pgfqpoint{3.820362in}{2.946133in}}%
\pgfpathlineto{\pgfqpoint{3.807318in}{2.951349in}}%
\pgfpathlineto{\pgfqpoint{3.794278in}{2.956604in}}%
\pgfpathlineto{\pgfqpoint{3.801922in}{2.965926in}}%
\pgfpathlineto{\pgfqpoint{3.809561in}{2.975345in}}%
\pgfpathlineto{\pgfqpoint{3.817195in}{2.984865in}}%
\pgfpathlineto{\pgfqpoint{3.824824in}{2.994489in}}%
\pgfpathclose%
\pgfusepath{fill}%
\end{pgfscope}%
\begin{pgfscope}%
\pgfpathrectangle{\pgfqpoint{1.150000in}{0.150000in}}{\pgfqpoint{5.700000in}{5.700000in}}%
\pgfusepath{clip}%
\pgfsetbuttcap%
\pgfsetroundjoin%
\definecolor{currentfill}{rgb}{0.280894,0.078907,0.402329}%
\pgfsetfillcolor{currentfill}%
\pgfsetfillopacity{0.700000}%
\pgfsetlinewidth{0.000000pt}%
\definecolor{currentstroke}{rgb}{0.000000,0.000000,0.000000}%
\pgfsetstrokecolor{currentstroke}%
\pgfsetdash{}{0pt}%
\pgfpathmoveto{\pgfqpoint{4.176877in}{3.011018in}}%
\pgfpathlineto{\pgfqpoint{4.189973in}{3.005955in}}%
\pgfpathlineto{\pgfqpoint{4.203074in}{3.000926in}}%
\pgfpathlineto{\pgfqpoint{4.216180in}{2.995929in}}%
\pgfpathlineto{\pgfqpoint{4.229291in}{2.990965in}}%
\pgfpathlineto{\pgfqpoint{4.221780in}{2.980692in}}%
\pgfpathlineto{\pgfqpoint{4.214264in}{2.970556in}}%
\pgfpathlineto{\pgfqpoint{4.206745in}{2.960552in}}%
\pgfpathlineto{\pgfqpoint{4.199221in}{2.950675in}}%
\pgfpathlineto{\pgfqpoint{4.186100in}{2.955489in}}%
\pgfpathlineto{\pgfqpoint{4.172983in}{2.960336in}}%
\pgfpathlineto{\pgfqpoint{4.159872in}{2.965216in}}%
\pgfpathlineto{\pgfqpoint{4.146766in}{2.970129in}}%
\pgfpathlineto{\pgfqpoint{4.154300in}{2.980151in}}%
\pgfpathlineto{\pgfqpoint{4.161830in}{2.990303in}}%
\pgfpathlineto{\pgfqpoint{4.169356in}{3.000590in}}%
\pgfpathlineto{\pgfqpoint{4.176877in}{3.011018in}}%
\pgfpathclose%
\pgfusepath{fill}%
\end{pgfscope}%
\begin{pgfscope}%
\pgfpathrectangle{\pgfqpoint{1.150000in}{0.150000in}}{\pgfqpoint{5.700000in}{5.700000in}}%
\pgfusepath{clip}%
\pgfsetbuttcap%
\pgfsetroundjoin%
\definecolor{currentfill}{rgb}{0.280894,0.078907,0.402329}%
\pgfsetfillcolor{currentfill}%
\pgfsetfillopacity{0.700000}%
\pgfsetlinewidth{0.000000pt}%
\definecolor{currentstroke}{rgb}{0.000000,0.000000,0.000000}%
\pgfsetstrokecolor{currentstroke}%
\pgfsetdash{}{0pt}%
\pgfpathmoveto{\pgfqpoint{3.337873in}{3.007997in}}%
\pgfpathlineto{\pgfqpoint{3.350827in}{3.001781in}}%
\pgfpathlineto{\pgfqpoint{3.363786in}{2.995615in}}%
\pgfpathlineto{\pgfqpoint{3.376748in}{2.989497in}}%
\pgfpathlineto{\pgfqpoint{3.389713in}{2.983427in}}%
\pgfpathlineto{\pgfqpoint{3.381932in}{2.974690in}}%
\pgfpathlineto{\pgfqpoint{3.374144in}{2.966028in}}%
\pgfpathlineto{\pgfqpoint{3.366349in}{2.957439in}}%
\pgfpathlineto{\pgfqpoint{3.358549in}{2.948921in}}%
\pgfpathlineto{\pgfqpoint{3.345572in}{2.954931in}}%
\pgfpathlineto{\pgfqpoint{3.332600in}{2.960989in}}%
\pgfpathlineto{\pgfqpoint{3.319631in}{2.967096in}}%
\pgfpathlineto{\pgfqpoint{3.306665in}{2.973253in}}%
\pgfpathlineto{\pgfqpoint{3.314477in}{2.981826in}}%
\pgfpathlineto{\pgfqpoint{3.322282in}{2.990473in}}%
\pgfpathlineto{\pgfqpoint{3.330080in}{2.999196in}}%
\pgfpathlineto{\pgfqpoint{3.337873in}{3.007997in}}%
\pgfpathclose%
\pgfusepath{fill}%
\end{pgfscope}%
\begin{pgfscope}%
\pgfpathrectangle{\pgfqpoint{1.150000in}{0.150000in}}{\pgfqpoint{5.700000in}{5.700000in}}%
\pgfusepath{clip}%
\pgfsetbuttcap%
\pgfsetroundjoin%
\definecolor{currentfill}{rgb}{0.280267,0.073417,0.397163}%
\pgfsetfillcolor{currentfill}%
\pgfsetfillopacity{0.700000}%
\pgfsetlinewidth{0.000000pt}%
\definecolor{currentstroke}{rgb}{0.000000,0.000000,0.000000}%
\pgfsetstrokecolor{currentstroke}%
\pgfsetdash{}{0pt}%
\pgfpathmoveto{\pgfqpoint{3.472638in}{2.995078in}}%
\pgfpathlineto{\pgfqpoint{3.485613in}{2.989171in}}%
\pgfpathlineto{\pgfqpoint{3.498591in}{2.983308in}}%
\pgfpathlineto{\pgfqpoint{3.511574in}{2.977491in}}%
\pgfpathlineto{\pgfqpoint{3.524560in}{2.971718in}}%
\pgfpathlineto{\pgfqpoint{3.516824in}{2.962811in}}%
\pgfpathlineto{\pgfqpoint{3.509081in}{2.953983in}}%
\pgfpathlineto{\pgfqpoint{3.501333in}{2.945231in}}%
\pgfpathlineto{\pgfqpoint{3.493578in}{2.936552in}}%
\pgfpathlineto{\pgfqpoint{3.480581in}{2.942252in}}%
\pgfpathlineto{\pgfqpoint{3.467588in}{2.947997in}}%
\pgfpathlineto{\pgfqpoint{3.454599in}{2.953786in}}%
\pgfpathlineto{\pgfqpoint{3.441614in}{2.959622in}}%
\pgfpathlineto{\pgfqpoint{3.449379in}{2.968368in}}%
\pgfpathlineto{\pgfqpoint{3.457138in}{2.977192in}}%
\pgfpathlineto{\pgfqpoint{3.464891in}{2.986094in}}%
\pgfpathlineto{\pgfqpoint{3.472638in}{2.995078in}}%
\pgfpathclose%
\pgfusepath{fill}%
\end{pgfscope}%
\begin{pgfscope}%
\pgfpathrectangle{\pgfqpoint{1.150000in}{0.150000in}}{\pgfqpoint{5.700000in}{5.700000in}}%
\pgfusepath{clip}%
\pgfsetbuttcap%
\pgfsetroundjoin%
\definecolor{currentfill}{rgb}{0.279566,0.067836,0.391917}%
\pgfsetfillcolor{currentfill}%
\pgfsetfillopacity{0.700000}%
\pgfsetlinewidth{0.000000pt}%
\definecolor{currentstroke}{rgb}{0.000000,0.000000,0.000000}%
\pgfsetstrokecolor{currentstroke}%
\pgfsetdash{}{0pt}%
\pgfpathmoveto{\pgfqpoint{3.959575in}{2.991241in}}%
\pgfpathlineto{\pgfqpoint{3.972632in}{2.986049in}}%
\pgfpathlineto{\pgfqpoint{3.985694in}{2.980893in}}%
\pgfpathlineto{\pgfqpoint{3.998761in}{2.975773in}}%
\pgfpathlineto{\pgfqpoint{4.011832in}{2.970688in}}%
\pgfpathlineto{\pgfqpoint{4.004255in}{2.960994in}}%
\pgfpathlineto{\pgfqpoint{3.996672in}{2.951411in}}%
\pgfpathlineto{\pgfqpoint{3.989084in}{2.941934in}}%
\pgfpathlineto{\pgfqpoint{3.981491in}{2.932560in}}%
\pgfpathlineto{\pgfqpoint{3.968409in}{2.937520in}}%
\pgfpathlineto{\pgfqpoint{3.955332in}{2.942516in}}%
\pgfpathlineto{\pgfqpoint{3.942260in}{2.947548in}}%
\pgfpathlineto{\pgfqpoint{3.929192in}{2.952616in}}%
\pgfpathlineto{\pgfqpoint{3.936795in}{2.962110in}}%
\pgfpathlineto{\pgfqpoint{3.944394in}{2.971709in}}%
\pgfpathlineto{\pgfqpoint{3.951987in}{2.981418in}}%
\pgfpathlineto{\pgfqpoint{3.959575in}{2.991241in}}%
\pgfpathclose%
\pgfusepath{fill}%
\end{pgfscope}%
\begin{pgfscope}%
\pgfpathrectangle{\pgfqpoint{1.150000in}{0.150000in}}{\pgfqpoint{5.700000in}{5.700000in}}%
\pgfusepath{clip}%
\pgfsetbuttcap%
\pgfsetroundjoin%
\definecolor{currentfill}{rgb}{0.279566,0.067836,0.391917}%
\pgfsetfillcolor{currentfill}%
\pgfsetfillopacity{0.700000}%
\pgfsetlinewidth{0.000000pt}%
\definecolor{currentstroke}{rgb}{0.000000,0.000000,0.000000}%
\pgfsetstrokecolor{currentstroke}%
\pgfsetdash{}{0pt}%
\pgfpathmoveto{\pgfqpoint{3.607394in}{2.985194in}}%
\pgfpathlineto{\pgfqpoint{3.620391in}{2.979553in}}%
\pgfpathlineto{\pgfqpoint{3.633392in}{2.973954in}}%
\pgfpathlineto{\pgfqpoint{3.646397in}{2.968398in}}%
\pgfpathlineto{\pgfqpoint{3.659407in}{2.962883in}}%
\pgfpathlineto{\pgfqpoint{3.651714in}{2.953814in}}%
\pgfpathlineto{\pgfqpoint{3.644016in}{2.944829in}}%
\pgfpathlineto{\pgfqpoint{3.636312in}{2.935924in}}%
\pgfpathlineto{\pgfqpoint{3.628602in}{2.927097in}}%
\pgfpathlineto{\pgfqpoint{3.615582in}{2.932527in}}%
\pgfpathlineto{\pgfqpoint{3.602566in}{2.937998in}}%
\pgfpathlineto{\pgfqpoint{3.589555in}{2.943511in}}%
\pgfpathlineto{\pgfqpoint{3.576547in}{2.949066in}}%
\pgfpathlineto{\pgfqpoint{3.584268in}{2.957974in}}%
\pgfpathlineto{\pgfqpoint{3.591982in}{2.966962in}}%
\pgfpathlineto{\pgfqpoint{3.599691in}{2.976035in}}%
\pgfpathlineto{\pgfqpoint{3.607394in}{2.985194in}}%
\pgfpathclose%
\pgfusepath{fill}%
\end{pgfscope}%
\begin{pgfscope}%
\pgfpathrectangle{\pgfqpoint{1.150000in}{0.150000in}}{\pgfqpoint{5.700000in}{5.700000in}}%
\pgfusepath{clip}%
\pgfsetbuttcap%
\pgfsetroundjoin%
\definecolor{currentfill}{rgb}{0.281446,0.084320,0.407414}%
\pgfsetfillcolor{currentfill}%
\pgfsetfillopacity{0.700000}%
\pgfsetlinewidth{0.000000pt}%
\definecolor{currentstroke}{rgb}{0.000000,0.000000,0.000000}%
\pgfsetstrokecolor{currentstroke}%
\pgfsetdash{}{0pt}%
\pgfpathmoveto{\pgfqpoint{4.311750in}{3.013347in}}%
\pgfpathlineto{\pgfqpoint{4.324876in}{3.008381in}}%
\pgfpathlineto{\pgfqpoint{4.338008in}{3.003447in}}%
\pgfpathlineto{\pgfqpoint{4.351144in}{2.998544in}}%
\pgfpathlineto{\pgfqpoint{4.364286in}{2.993673in}}%
\pgfpathlineto{\pgfqpoint{4.356811in}{2.983136in}}%
\pgfpathlineto{\pgfqpoint{4.349333in}{2.972750in}}%
\pgfpathlineto{\pgfqpoint{4.341851in}{2.962510in}}%
\pgfpathlineto{\pgfqpoint{4.334365in}{2.952410in}}%
\pgfpathlineto{\pgfqpoint{4.321212in}{2.957119in}}%
\pgfpathlineto{\pgfqpoint{4.308065in}{2.961859in}}%
\pgfpathlineto{\pgfqpoint{4.294923in}{2.966630in}}%
\pgfpathlineto{\pgfqpoint{4.281786in}{2.971433in}}%
\pgfpathlineto{\pgfqpoint{4.289283in}{2.981691in}}%
\pgfpathlineto{\pgfqpoint{4.296775in}{2.992092in}}%
\pgfpathlineto{\pgfqpoint{4.304265in}{3.002642in}}%
\pgfpathlineto{\pgfqpoint{4.311750in}{3.013347in}}%
\pgfpathclose%
\pgfusepath{fill}%
\end{pgfscope}%
\begin{pgfscope}%
\pgfpathrectangle{\pgfqpoint{1.150000in}{0.150000in}}{\pgfqpoint{5.700000in}{5.700000in}}%
\pgfusepath{clip}%
\pgfsetbuttcap%
\pgfsetroundjoin%
\definecolor{currentfill}{rgb}{0.279566,0.067836,0.391917}%
\pgfsetfillcolor{currentfill}%
\pgfsetfillopacity{0.700000}%
\pgfsetlinewidth{0.000000pt}%
\definecolor{currentstroke}{rgb}{0.000000,0.000000,0.000000}%
\pgfsetstrokecolor{currentstroke}%
\pgfsetdash{}{0pt}%
\pgfpathmoveto{\pgfqpoint{4.094392in}{2.990118in}}%
\pgfpathlineto{\pgfqpoint{4.107478in}{2.985070in}}%
\pgfpathlineto{\pgfqpoint{4.120569in}{2.980056in}}%
\pgfpathlineto{\pgfqpoint{4.133665in}{2.975076in}}%
\pgfpathlineto{\pgfqpoint{4.146766in}{2.970129in}}%
\pgfpathlineto{\pgfqpoint{4.139227in}{2.960234in}}%
\pgfpathlineto{\pgfqpoint{4.131684in}{2.950460in}}%
\pgfpathlineto{\pgfqpoint{4.124137in}{2.940803in}}%
\pgfpathlineto{\pgfqpoint{4.116584in}{2.931259in}}%
\pgfpathlineto{\pgfqpoint{4.103473in}{2.936068in}}%
\pgfpathlineto{\pgfqpoint{4.090366in}{2.940911in}}%
\pgfpathlineto{\pgfqpoint{4.077265in}{2.945788in}}%
\pgfpathlineto{\pgfqpoint{4.064168in}{2.950699in}}%
\pgfpathlineto{\pgfqpoint{4.071731in}{2.960375in}}%
\pgfpathlineto{\pgfqpoint{4.079289in}{2.970168in}}%
\pgfpathlineto{\pgfqpoint{4.086843in}{2.980080in}}%
\pgfpathlineto{\pgfqpoint{4.094392in}{2.990118in}}%
\pgfpathclose%
\pgfusepath{fill}%
\end{pgfscope}%
\begin{pgfscope}%
\pgfpathrectangle{\pgfqpoint{1.150000in}{0.150000in}}{\pgfqpoint{5.700000in}{5.700000in}}%
\pgfusepath{clip}%
\pgfsetbuttcap%
\pgfsetroundjoin%
\definecolor{currentfill}{rgb}{0.278791,0.062145,0.386592}%
\pgfsetfillcolor{currentfill}%
\pgfsetfillopacity{0.700000}%
\pgfsetlinewidth{0.000000pt}%
\definecolor{currentstroke}{rgb}{0.000000,0.000000,0.000000}%
\pgfsetstrokecolor{currentstroke}%
\pgfsetdash{}{0pt}%
\pgfpathmoveto{\pgfqpoint{3.742164in}{2.978012in}}%
\pgfpathlineto{\pgfqpoint{3.755186in}{2.972601in}}%
\pgfpathlineto{\pgfqpoint{3.768212in}{2.967230in}}%
\pgfpathlineto{\pgfqpoint{3.781242in}{2.961897in}}%
\pgfpathlineto{\pgfqpoint{3.794278in}{2.956604in}}%
\pgfpathlineto{\pgfqpoint{3.786628in}{2.947376in}}%
\pgfpathlineto{\pgfqpoint{3.778973in}{2.938237in}}%
\pgfpathlineto{\pgfqpoint{3.771312in}{2.929186in}}%
\pgfpathlineto{\pgfqpoint{3.763646in}{2.920218in}}%
\pgfpathlineto{\pgfqpoint{3.750600in}{2.925412in}}%
\pgfpathlineto{\pgfqpoint{3.737559in}{2.930646in}}%
\pgfpathlineto{\pgfqpoint{3.724522in}{2.935919in}}%
\pgfpathlineto{\pgfqpoint{3.711490in}{2.941231in}}%
\pgfpathlineto{\pgfqpoint{3.719167in}{2.950293in}}%
\pgfpathlineto{\pgfqpoint{3.726838in}{2.959442in}}%
\pgfpathlineto{\pgfqpoint{3.734504in}{2.968680in}}%
\pgfpathlineto{\pgfqpoint{3.742164in}{2.978012in}}%
\pgfpathclose%
\pgfusepath{fill}%
\end{pgfscope}%
\begin{pgfscope}%
\pgfpathrectangle{\pgfqpoint{1.150000in}{0.150000in}}{\pgfqpoint{5.700000in}{5.700000in}}%
\pgfusepath{clip}%
\pgfsetbuttcap%
\pgfsetroundjoin%
\definecolor{currentfill}{rgb}{0.278791,0.062145,0.386592}%
\pgfsetfillcolor{currentfill}%
\pgfsetfillopacity{0.700000}%
\pgfsetlinewidth{0.000000pt}%
\definecolor{currentstroke}{rgb}{0.000000,0.000000,0.000000}%
\pgfsetstrokecolor{currentstroke}%
\pgfsetdash{}{0pt}%
\pgfpathmoveto{\pgfqpoint{3.876971in}{2.973254in}}%
\pgfpathlineto{\pgfqpoint{3.890019in}{2.968039in}}%
\pgfpathlineto{\pgfqpoint{3.903072in}{2.962861in}}%
\pgfpathlineto{\pgfqpoint{3.916130in}{2.957720in}}%
\pgfpathlineto{\pgfqpoint{3.929192in}{2.952616in}}%
\pgfpathlineto{\pgfqpoint{3.921584in}{2.943224in}}%
\pgfpathlineto{\pgfqpoint{3.913971in}{2.933931in}}%
\pgfpathlineto{\pgfqpoint{3.906352in}{2.924731in}}%
\pgfpathlineto{\pgfqpoint{3.898729in}{2.915622in}}%
\pgfpathlineto{\pgfqpoint{3.885656in}{2.920615in}}%
\pgfpathlineto{\pgfqpoint{3.872587in}{2.925644in}}%
\pgfpathlineto{\pgfqpoint{3.859524in}{2.930711in}}%
\pgfpathlineto{\pgfqpoint{3.846465in}{2.935814in}}%
\pgfpathlineto{\pgfqpoint{3.854100in}{2.945029in}}%
\pgfpathlineto{\pgfqpoint{3.861728in}{2.954339in}}%
\pgfpathlineto{\pgfqpoint{3.869352in}{2.963746in}}%
\pgfpathlineto{\pgfqpoint{3.876971in}{2.973254in}}%
\pgfpathclose%
\pgfusepath{fill}%
\end{pgfscope}%
\begin{pgfscope}%
\pgfpathrectangle{\pgfqpoint{1.150000in}{0.150000in}}{\pgfqpoint{5.700000in}{5.700000in}}%
\pgfusepath{clip}%
\pgfsetbuttcap%
\pgfsetroundjoin%
\definecolor{currentfill}{rgb}{0.280894,0.078907,0.402329}%
\pgfsetfillcolor{currentfill}%
\pgfsetfillopacity{0.700000}%
\pgfsetlinewidth{0.000000pt}%
\definecolor{currentstroke}{rgb}{0.000000,0.000000,0.000000}%
\pgfsetstrokecolor{currentstroke}%
\pgfsetdash{}{0pt}%
\pgfpathmoveto{\pgfqpoint{3.254839in}{2.998380in}}%
\pgfpathlineto{\pgfqpoint{3.267790in}{2.992022in}}%
\pgfpathlineto{\pgfqpoint{3.280745in}{2.985715in}}%
\pgfpathlineto{\pgfqpoint{3.293703in}{2.979458in}}%
\pgfpathlineto{\pgfqpoint{3.306665in}{2.973253in}}%
\pgfpathlineto{\pgfqpoint{3.298848in}{2.964751in}}%
\pgfpathlineto{\pgfqpoint{3.291024in}{2.956320in}}%
\pgfpathlineto{\pgfqpoint{3.283194in}{2.947956in}}%
\pgfpathlineto{\pgfqpoint{3.275357in}{2.939659in}}%
\pgfpathlineto{\pgfqpoint{3.262384in}{2.945818in}}%
\pgfpathlineto{\pgfqpoint{3.249415in}{2.952027in}}%
\pgfpathlineto{\pgfqpoint{3.236449in}{2.958287in}}%
\pgfpathlineto{\pgfqpoint{3.223486in}{2.964599in}}%
\pgfpathlineto{\pgfqpoint{3.231334in}{2.972938in}}%
\pgfpathlineto{\pgfqpoint{3.239176in}{2.981347in}}%
\pgfpathlineto{\pgfqpoint{3.247011in}{2.989827in}}%
\pgfpathlineto{\pgfqpoint{3.254839in}{2.998380in}}%
\pgfpathclose%
\pgfusepath{fill}%
\end{pgfscope}%
\begin{pgfscope}%
\pgfpathrectangle{\pgfqpoint{1.150000in}{0.150000in}}{\pgfqpoint{5.700000in}{5.700000in}}%
\pgfusepath{clip}%
\pgfsetbuttcap%
\pgfsetroundjoin%
\definecolor{currentfill}{rgb}{0.280267,0.073417,0.397163}%
\pgfsetfillcolor{currentfill}%
\pgfsetfillopacity{0.700000}%
\pgfsetlinewidth{0.000000pt}%
\definecolor{currentstroke}{rgb}{0.000000,0.000000,0.000000}%
\pgfsetstrokecolor{currentstroke}%
\pgfsetdash{}{0pt}%
\pgfpathmoveto{\pgfqpoint{4.229291in}{2.990965in}}%
\pgfpathlineto{\pgfqpoint{4.242407in}{2.986034in}}%
\pgfpathlineto{\pgfqpoint{4.255528in}{2.981135in}}%
\pgfpathlineto{\pgfqpoint{4.268654in}{2.976268in}}%
\pgfpathlineto{\pgfqpoint{4.281786in}{2.971433in}}%
\pgfpathlineto{\pgfqpoint{4.274285in}{2.961315in}}%
\pgfpathlineto{\pgfqpoint{4.266781in}{2.951330in}}%
\pgfpathlineto{\pgfqpoint{4.259272in}{2.941474in}}%
\pgfpathlineto{\pgfqpoint{4.251759in}{2.931743in}}%
\pgfpathlineto{\pgfqpoint{4.238617in}{2.936428in}}%
\pgfpathlineto{\pgfqpoint{4.225480in}{2.941145in}}%
\pgfpathlineto{\pgfqpoint{4.212348in}{2.945894in}}%
\pgfpathlineto{\pgfqpoint{4.199221in}{2.950675in}}%
\pgfpathlineto{\pgfqpoint{4.206745in}{2.960552in}}%
\pgfpathlineto{\pgfqpoint{4.214264in}{2.970556in}}%
\pgfpathlineto{\pgfqpoint{4.221780in}{2.980692in}}%
\pgfpathlineto{\pgfqpoint{4.229291in}{2.990965in}}%
\pgfpathclose%
\pgfusepath{fill}%
\end{pgfscope}%
\begin{pgfscope}%
\pgfpathrectangle{\pgfqpoint{1.150000in}{0.150000in}}{\pgfqpoint{5.700000in}{5.700000in}}%
\pgfusepath{clip}%
\pgfsetbuttcap%
\pgfsetroundjoin%
\definecolor{currentfill}{rgb}{0.279566,0.067836,0.391917}%
\pgfsetfillcolor{currentfill}%
\pgfsetfillopacity{0.700000}%
\pgfsetlinewidth{0.000000pt}%
\definecolor{currentstroke}{rgb}{0.000000,0.000000,0.000000}%
\pgfsetstrokecolor{currentstroke}%
\pgfsetdash{}{0pt}%
\pgfpathmoveto{\pgfqpoint{3.389713in}{2.983427in}}%
\pgfpathlineto{\pgfqpoint{3.402683in}{2.977405in}}%
\pgfpathlineto{\pgfqpoint{3.415656in}{2.971431in}}%
\pgfpathlineto{\pgfqpoint{3.428633in}{2.965503in}}%
\pgfpathlineto{\pgfqpoint{3.441614in}{2.959622in}}%
\pgfpathlineto{\pgfqpoint{3.433843in}{2.950949in}}%
\pgfpathlineto{\pgfqpoint{3.426066in}{2.942349in}}%
\pgfpathlineto{\pgfqpoint{3.418283in}{2.933818in}}%
\pgfpathlineto{\pgfqpoint{3.410493in}{2.925355in}}%
\pgfpathlineto{\pgfqpoint{3.397501in}{2.931176in}}%
\pgfpathlineto{\pgfqpoint{3.384513in}{2.937044in}}%
\pgfpathlineto{\pgfqpoint{3.371529in}{2.942959in}}%
\pgfpathlineto{\pgfqpoint{3.358549in}{2.948921in}}%
\pgfpathlineto{\pgfqpoint{3.366349in}{2.957439in}}%
\pgfpathlineto{\pgfqpoint{3.374144in}{2.966028in}}%
\pgfpathlineto{\pgfqpoint{3.381932in}{2.974690in}}%
\pgfpathlineto{\pgfqpoint{3.389713in}{2.983427in}}%
\pgfpathclose%
\pgfusepath{fill}%
\end{pgfscope}%
\begin{pgfscope}%
\pgfpathrectangle{\pgfqpoint{1.150000in}{0.150000in}}{\pgfqpoint{5.700000in}{5.700000in}}%
\pgfusepath{clip}%
\pgfsetbuttcap%
\pgfsetroundjoin%
\definecolor{currentfill}{rgb}{0.278791,0.062145,0.386592}%
\pgfsetfillcolor{currentfill}%
\pgfsetfillopacity{0.700000}%
\pgfsetlinewidth{0.000000pt}%
\definecolor{currentstroke}{rgb}{0.000000,0.000000,0.000000}%
\pgfsetstrokecolor{currentstroke}%
\pgfsetdash{}{0pt}%
\pgfpathmoveto{\pgfqpoint{3.524560in}{2.971718in}}%
\pgfpathlineto{\pgfqpoint{3.537551in}{2.965990in}}%
\pgfpathlineto{\pgfqpoint{3.550545in}{2.960305in}}%
\pgfpathlineto{\pgfqpoint{3.563544in}{2.954664in}}%
\pgfpathlineto{\pgfqpoint{3.576547in}{2.949066in}}%
\pgfpathlineto{\pgfqpoint{3.568821in}{2.940237in}}%
\pgfpathlineto{\pgfqpoint{3.561089in}{2.931483in}}%
\pgfpathlineto{\pgfqpoint{3.553352in}{2.922802in}}%
\pgfpathlineto{\pgfqpoint{3.545608in}{2.914191in}}%
\pgfpathlineto{\pgfqpoint{3.532594in}{2.919716in}}%
\pgfpathlineto{\pgfqpoint{3.519585in}{2.925284in}}%
\pgfpathlineto{\pgfqpoint{3.506579in}{2.930896in}}%
\pgfpathlineto{\pgfqpoint{3.493578in}{2.936552in}}%
\pgfpathlineto{\pgfqpoint{3.501333in}{2.945231in}}%
\pgfpathlineto{\pgfqpoint{3.509081in}{2.953983in}}%
\pgfpathlineto{\pgfqpoint{3.516824in}{2.962811in}}%
\pgfpathlineto{\pgfqpoint{3.524560in}{2.971718in}}%
\pgfpathclose%
\pgfusepath{fill}%
\end{pgfscope}%
\begin{pgfscope}%
\pgfpathrectangle{\pgfqpoint{1.150000in}{0.150000in}}{\pgfqpoint{5.700000in}{5.700000in}}%
\pgfusepath{clip}%
\pgfsetbuttcap%
\pgfsetroundjoin%
\definecolor{currentfill}{rgb}{0.278791,0.062145,0.386592}%
\pgfsetfillcolor{currentfill}%
\pgfsetfillopacity{0.700000}%
\pgfsetlinewidth{0.000000pt}%
\definecolor{currentstroke}{rgb}{0.000000,0.000000,0.000000}%
\pgfsetstrokecolor{currentstroke}%
\pgfsetdash{}{0pt}%
\pgfpathmoveto{\pgfqpoint{4.011832in}{2.970688in}}%
\pgfpathlineto{\pgfqpoint{4.024909in}{2.965638in}}%
\pgfpathlineto{\pgfqpoint{4.037991in}{2.960624in}}%
\pgfpathlineto{\pgfqpoint{4.051077in}{2.955644in}}%
\pgfpathlineto{\pgfqpoint{4.064168in}{2.950699in}}%
\pgfpathlineto{\pgfqpoint{4.056601in}{2.941134in}}%
\pgfpathlineto{\pgfqpoint{4.049028in}{2.931677in}}%
\pgfpathlineto{\pgfqpoint{4.041451in}{2.922323in}}%
\pgfpathlineto{\pgfqpoint{4.033869in}{2.913068in}}%
\pgfpathlineto{\pgfqpoint{4.020767in}{2.917889in}}%
\pgfpathlineto{\pgfqpoint{4.007670in}{2.922744in}}%
\pgfpathlineto{\pgfqpoint{3.994578in}{2.927634in}}%
\pgfpathlineto{\pgfqpoint{3.981491in}{2.932560in}}%
\pgfpathlineto{\pgfqpoint{3.989084in}{2.941934in}}%
\pgfpathlineto{\pgfqpoint{3.996672in}{2.951411in}}%
\pgfpathlineto{\pgfqpoint{4.004255in}{2.960994in}}%
\pgfpathlineto{\pgfqpoint{4.011832in}{2.970688in}}%
\pgfpathclose%
\pgfusepath{fill}%
\end{pgfscope}%
\begin{pgfscope}%
\pgfpathrectangle{\pgfqpoint{1.150000in}{0.150000in}}{\pgfqpoint{5.700000in}{5.700000in}}%
\pgfusepath{clip}%
\pgfsetbuttcap%
\pgfsetroundjoin%
\definecolor{currentfill}{rgb}{0.278791,0.062145,0.386592}%
\pgfsetfillcolor{currentfill}%
\pgfsetfillopacity{0.700000}%
\pgfsetlinewidth{0.000000pt}%
\definecolor{currentstroke}{rgb}{0.000000,0.000000,0.000000}%
\pgfsetstrokecolor{currentstroke}%
\pgfsetdash{}{0pt}%
\pgfpathmoveto{\pgfqpoint{3.659407in}{2.962883in}}%
\pgfpathlineto{\pgfqpoint{3.672421in}{2.957409in}}%
\pgfpathlineto{\pgfqpoint{3.685440in}{2.951976in}}%
\pgfpathlineto{\pgfqpoint{3.698463in}{2.946583in}}%
\pgfpathlineto{\pgfqpoint{3.711490in}{2.941231in}}%
\pgfpathlineto{\pgfqpoint{3.703808in}{2.932253in}}%
\pgfpathlineto{\pgfqpoint{3.696120in}{2.923355in}}%
\pgfpathlineto{\pgfqpoint{3.688427in}{2.914534in}}%
\pgfpathlineto{\pgfqpoint{3.680728in}{2.905789in}}%
\pgfpathlineto{\pgfqpoint{3.667690in}{2.911055in}}%
\pgfpathlineto{\pgfqpoint{3.654656in}{2.916362in}}%
\pgfpathlineto{\pgfqpoint{3.641627in}{2.921709in}}%
\pgfpathlineto{\pgfqpoint{3.628602in}{2.927097in}}%
\pgfpathlineto{\pgfqpoint{3.636312in}{2.935924in}}%
\pgfpathlineto{\pgfqpoint{3.644016in}{2.944829in}}%
\pgfpathlineto{\pgfqpoint{3.651714in}{2.953814in}}%
\pgfpathlineto{\pgfqpoint{3.659407in}{2.962883in}}%
\pgfpathclose%
\pgfusepath{fill}%
\end{pgfscope}%
\begin{pgfscope}%
\pgfpathrectangle{\pgfqpoint{1.150000in}{0.150000in}}{\pgfqpoint{5.700000in}{5.700000in}}%
\pgfusepath{clip}%
\pgfsetbuttcap%
\pgfsetroundjoin%
\definecolor{currentfill}{rgb}{0.280894,0.078907,0.402329}%
\pgfsetfillcolor{currentfill}%
\pgfsetfillopacity{0.700000}%
\pgfsetlinewidth{0.000000pt}%
\definecolor{currentstroke}{rgb}{0.000000,0.000000,0.000000}%
\pgfsetstrokecolor{currentstroke}%
\pgfsetdash{}{0pt}%
\pgfpathmoveto{\pgfqpoint{4.364286in}{2.993673in}}%
\pgfpathlineto{\pgfqpoint{4.377433in}{2.988832in}}%
\pgfpathlineto{\pgfqpoint{4.390585in}{2.984023in}}%
\pgfpathlineto{\pgfqpoint{4.403743in}{2.979244in}}%
\pgfpathlineto{\pgfqpoint{4.416906in}{2.974496in}}%
\pgfpathlineto{\pgfqpoint{4.409442in}{2.964127in}}%
\pgfpathlineto{\pgfqpoint{4.401974in}{2.953905in}}%
\pgfpathlineto{\pgfqpoint{4.394503in}{2.943826in}}%
\pgfpathlineto{\pgfqpoint{4.387028in}{2.933885in}}%
\pgfpathlineto{\pgfqpoint{4.373854in}{2.938470in}}%
\pgfpathlineto{\pgfqpoint{4.360686in}{2.943086in}}%
\pgfpathlineto{\pgfqpoint{4.347523in}{2.947733in}}%
\pgfpathlineto{\pgfqpoint{4.334365in}{2.952410in}}%
\pgfpathlineto{\pgfqpoint{4.341851in}{2.962510in}}%
\pgfpathlineto{\pgfqpoint{4.349333in}{2.972750in}}%
\pgfpathlineto{\pgfqpoint{4.356811in}{2.983136in}}%
\pgfpathlineto{\pgfqpoint{4.364286in}{2.993673in}}%
\pgfpathclose%
\pgfusepath{fill}%
\end{pgfscope}%
\begin{pgfscope}%
\pgfpathrectangle{\pgfqpoint{1.150000in}{0.150000in}}{\pgfqpoint{5.700000in}{5.700000in}}%
\pgfusepath{clip}%
\pgfsetbuttcap%
\pgfsetroundjoin%
\definecolor{currentfill}{rgb}{0.278791,0.062145,0.386592}%
\pgfsetfillcolor{currentfill}%
\pgfsetfillopacity{0.700000}%
\pgfsetlinewidth{0.000000pt}%
\definecolor{currentstroke}{rgb}{0.000000,0.000000,0.000000}%
\pgfsetstrokecolor{currentstroke}%
\pgfsetdash{}{0pt}%
\pgfpathmoveto{\pgfqpoint{3.794278in}{2.956604in}}%
\pgfpathlineto{\pgfqpoint{3.807318in}{2.951349in}}%
\pgfpathlineto{\pgfqpoint{3.820362in}{2.946133in}}%
\pgfpathlineto{\pgfqpoint{3.833411in}{2.940955in}}%
\pgfpathlineto{\pgfqpoint{3.846465in}{2.935814in}}%
\pgfpathlineto{\pgfqpoint{3.838826in}{2.926689in}}%
\pgfpathlineto{\pgfqpoint{3.831181in}{2.917651in}}%
\pgfpathlineto{\pgfqpoint{3.823531in}{2.908696in}}%
\pgfpathlineto{\pgfqpoint{3.815875in}{2.899822in}}%
\pgfpathlineto{\pgfqpoint{3.802811in}{2.904864in}}%
\pgfpathlineto{\pgfqpoint{3.789751in}{2.909944in}}%
\pgfpathlineto{\pgfqpoint{3.776696in}{2.915061in}}%
\pgfpathlineto{\pgfqpoint{3.763646in}{2.920218in}}%
\pgfpathlineto{\pgfqpoint{3.771312in}{2.929186in}}%
\pgfpathlineto{\pgfqpoint{3.778973in}{2.938237in}}%
\pgfpathlineto{\pgfqpoint{3.786628in}{2.947376in}}%
\pgfpathlineto{\pgfqpoint{3.794278in}{2.956604in}}%
\pgfpathclose%
\pgfusepath{fill}%
\end{pgfscope}%
\begin{pgfscope}%
\pgfpathrectangle{\pgfqpoint{1.150000in}{0.150000in}}{\pgfqpoint{5.700000in}{5.700000in}}%
\pgfusepath{clip}%
\pgfsetbuttcap%
\pgfsetroundjoin%
\definecolor{currentfill}{rgb}{0.279566,0.067836,0.391917}%
\pgfsetfillcolor{currentfill}%
\pgfsetfillopacity{0.700000}%
\pgfsetlinewidth{0.000000pt}%
\definecolor{currentstroke}{rgb}{0.000000,0.000000,0.000000}%
\pgfsetstrokecolor{currentstroke}%
\pgfsetdash{}{0pt}%
\pgfpathmoveto{\pgfqpoint{4.146766in}{2.970129in}}%
\pgfpathlineto{\pgfqpoint{4.159872in}{2.965216in}}%
\pgfpathlineto{\pgfqpoint{4.172983in}{2.960336in}}%
\pgfpathlineto{\pgfqpoint{4.186100in}{2.955489in}}%
\pgfpathlineto{\pgfqpoint{4.199221in}{2.950675in}}%
\pgfpathlineto{\pgfqpoint{4.191693in}{2.940922in}}%
\pgfpathlineto{\pgfqpoint{4.184161in}{2.931287in}}%
\pgfpathlineto{\pgfqpoint{4.176624in}{2.921766in}}%
\pgfpathlineto{\pgfqpoint{4.169082in}{2.912354in}}%
\pgfpathlineto{\pgfqpoint{4.155950in}{2.917031in}}%
\pgfpathlineto{\pgfqpoint{4.142823in}{2.921740in}}%
\pgfpathlineto{\pgfqpoint{4.129701in}{2.926483in}}%
\pgfpathlineto{\pgfqpoint{4.116584in}{2.931259in}}%
\pgfpathlineto{\pgfqpoint{4.124137in}{2.940803in}}%
\pgfpathlineto{\pgfqpoint{4.131684in}{2.950460in}}%
\pgfpathlineto{\pgfqpoint{4.139227in}{2.960234in}}%
\pgfpathlineto{\pgfqpoint{4.146766in}{2.970129in}}%
\pgfpathclose%
\pgfusepath{fill}%
\end{pgfscope}%
\begin{pgfscope}%
\pgfpathrectangle{\pgfqpoint{1.150000in}{0.150000in}}{\pgfqpoint{5.700000in}{5.700000in}}%
\pgfusepath{clip}%
\pgfsetbuttcap%
\pgfsetroundjoin%
\definecolor{currentfill}{rgb}{0.280267,0.073417,0.397163}%
\pgfsetfillcolor{currentfill}%
\pgfsetfillopacity{0.700000}%
\pgfsetlinewidth{0.000000pt}%
\definecolor{currentstroke}{rgb}{0.000000,0.000000,0.000000}%
\pgfsetstrokecolor{currentstroke}%
\pgfsetdash{}{0pt}%
\pgfpathmoveto{\pgfqpoint{3.306665in}{2.973253in}}%
\pgfpathlineto{\pgfqpoint{3.319631in}{2.967096in}}%
\pgfpathlineto{\pgfqpoint{3.332600in}{2.960989in}}%
\pgfpathlineto{\pgfqpoint{3.345572in}{2.954931in}}%
\pgfpathlineto{\pgfqpoint{3.358549in}{2.948921in}}%
\pgfpathlineto{\pgfqpoint{3.350742in}{2.940471in}}%
\pgfpathlineto{\pgfqpoint{3.342930in}{2.932088in}}%
\pgfpathlineto{\pgfqpoint{3.335111in}{2.923770in}}%
\pgfpathlineto{\pgfqpoint{3.327285in}{2.915515in}}%
\pgfpathlineto{\pgfqpoint{3.314298in}{2.921478in}}%
\pgfpathlineto{\pgfqpoint{3.301314in}{2.927489in}}%
\pgfpathlineto{\pgfqpoint{3.288334in}{2.933549in}}%
\pgfpathlineto{\pgfqpoint{3.275357in}{2.939659in}}%
\pgfpathlineto{\pgfqpoint{3.283194in}{2.947956in}}%
\pgfpathlineto{\pgfqpoint{3.291024in}{2.956320in}}%
\pgfpathlineto{\pgfqpoint{3.298848in}{2.964751in}}%
\pgfpathlineto{\pgfqpoint{3.306665in}{2.973253in}}%
\pgfpathclose%
\pgfusepath{fill}%
\end{pgfscope}%
\begin{pgfscope}%
\pgfpathrectangle{\pgfqpoint{1.150000in}{0.150000in}}{\pgfqpoint{5.700000in}{5.700000in}}%
\pgfusepath{clip}%
\pgfsetbuttcap%
\pgfsetroundjoin%
\definecolor{currentfill}{rgb}{0.278791,0.062145,0.386592}%
\pgfsetfillcolor{currentfill}%
\pgfsetfillopacity{0.700000}%
\pgfsetlinewidth{0.000000pt}%
\definecolor{currentstroke}{rgb}{0.000000,0.000000,0.000000}%
\pgfsetstrokecolor{currentstroke}%
\pgfsetdash{}{0pt}%
\pgfpathmoveto{\pgfqpoint{3.929192in}{2.952616in}}%
\pgfpathlineto{\pgfqpoint{3.942260in}{2.947548in}}%
\pgfpathlineto{\pgfqpoint{3.955332in}{2.942516in}}%
\pgfpathlineto{\pgfqpoint{3.968409in}{2.937520in}}%
\pgfpathlineto{\pgfqpoint{3.981491in}{2.932560in}}%
\pgfpathlineto{\pgfqpoint{3.973893in}{2.923284in}}%
\pgfpathlineto{\pgfqpoint{3.966291in}{2.914104in}}%
\pgfpathlineto{\pgfqpoint{3.958683in}{2.905014in}}%
\pgfpathlineto{\pgfqpoint{3.951070in}{2.896012in}}%
\pgfpathlineto{\pgfqpoint{3.937977in}{2.900861in}}%
\pgfpathlineto{\pgfqpoint{3.924889in}{2.905746in}}%
\pgfpathlineto{\pgfqpoint{3.911807in}{2.910666in}}%
\pgfpathlineto{\pgfqpoint{3.898729in}{2.915622in}}%
\pgfpathlineto{\pgfqpoint{3.906352in}{2.924731in}}%
\pgfpathlineto{\pgfqpoint{3.913971in}{2.933931in}}%
\pgfpathlineto{\pgfqpoint{3.921584in}{2.943224in}}%
\pgfpathlineto{\pgfqpoint{3.929192in}{2.952616in}}%
\pgfpathclose%
\pgfusepath{fill}%
\end{pgfscope}%
\begin{pgfscope}%
\pgfpathrectangle{\pgfqpoint{1.150000in}{0.150000in}}{\pgfqpoint{5.700000in}{5.700000in}}%
\pgfusepath{clip}%
\pgfsetbuttcap%
\pgfsetroundjoin%
\definecolor{currentfill}{rgb}{0.278791,0.062145,0.386592}%
\pgfsetfillcolor{currentfill}%
\pgfsetfillopacity{0.700000}%
\pgfsetlinewidth{0.000000pt}%
\definecolor{currentstroke}{rgb}{0.000000,0.000000,0.000000}%
\pgfsetstrokecolor{currentstroke}%
\pgfsetdash{}{0pt}%
\pgfpathmoveto{\pgfqpoint{3.441614in}{2.959622in}}%
\pgfpathlineto{\pgfqpoint{3.454599in}{2.953786in}}%
\pgfpathlineto{\pgfqpoint{3.467588in}{2.947997in}}%
\pgfpathlineto{\pgfqpoint{3.480581in}{2.942252in}}%
\pgfpathlineto{\pgfqpoint{3.493578in}{2.936552in}}%
\pgfpathlineto{\pgfqpoint{3.485818in}{2.927944in}}%
\pgfpathlineto{\pgfqpoint{3.478052in}{2.919405in}}%
\pgfpathlineto{\pgfqpoint{3.470279in}{2.910933in}}%
\pgfpathlineto{\pgfqpoint{3.462501in}{2.902524in}}%
\pgfpathlineto{\pgfqpoint{3.449493in}{2.908164in}}%
\pgfpathlineto{\pgfqpoint{3.436489in}{2.913849in}}%
\pgfpathlineto{\pgfqpoint{3.423489in}{2.919579in}}%
\pgfpathlineto{\pgfqpoint{3.410493in}{2.925355in}}%
\pgfpathlineto{\pgfqpoint{3.418283in}{2.933818in}}%
\pgfpathlineto{\pgfqpoint{3.426066in}{2.942349in}}%
\pgfpathlineto{\pgfqpoint{3.433843in}{2.950949in}}%
\pgfpathlineto{\pgfqpoint{3.441614in}{2.959622in}}%
\pgfpathclose%
\pgfusepath{fill}%
\end{pgfscope}%
\begin{pgfscope}%
\pgfpathrectangle{\pgfqpoint{1.150000in}{0.150000in}}{\pgfqpoint{5.700000in}{5.700000in}}%
\pgfusepath{clip}%
\pgfsetbuttcap%
\pgfsetroundjoin%
\definecolor{currentfill}{rgb}{0.280267,0.073417,0.397163}%
\pgfsetfillcolor{currentfill}%
\pgfsetfillopacity{0.700000}%
\pgfsetlinewidth{0.000000pt}%
\definecolor{currentstroke}{rgb}{0.000000,0.000000,0.000000}%
\pgfsetstrokecolor{currentstroke}%
\pgfsetdash{}{0pt}%
\pgfpathmoveto{\pgfqpoint{4.281786in}{2.971433in}}%
\pgfpathlineto{\pgfqpoint{4.294923in}{2.966630in}}%
\pgfpathlineto{\pgfqpoint{4.308065in}{2.961859in}}%
\pgfpathlineto{\pgfqpoint{4.321212in}{2.957119in}}%
\pgfpathlineto{\pgfqpoint{4.334365in}{2.952410in}}%
\pgfpathlineto{\pgfqpoint{4.326875in}{2.942446in}}%
\pgfpathlineto{\pgfqpoint{4.319381in}{2.932613in}}%
\pgfpathlineto{\pgfqpoint{4.311883in}{2.922906in}}%
\pgfpathlineto{\pgfqpoint{4.304381in}{2.913321in}}%
\pgfpathlineto{\pgfqpoint{4.291218in}{2.917879in}}%
\pgfpathlineto{\pgfqpoint{4.278060in}{2.922469in}}%
\pgfpathlineto{\pgfqpoint{4.264907in}{2.927090in}}%
\pgfpathlineto{\pgfqpoint{4.251759in}{2.931743in}}%
\pgfpathlineto{\pgfqpoint{4.259272in}{2.941474in}}%
\pgfpathlineto{\pgfqpoint{4.266781in}{2.951330in}}%
\pgfpathlineto{\pgfqpoint{4.274285in}{2.961315in}}%
\pgfpathlineto{\pgfqpoint{4.281786in}{2.971433in}}%
\pgfpathclose%
\pgfusepath{fill}%
\end{pgfscope}%
\begin{pgfscope}%
\pgfpathrectangle{\pgfqpoint{1.150000in}{0.150000in}}{\pgfqpoint{5.700000in}{5.700000in}}%
\pgfusepath{clip}%
\pgfsetbuttcap%
\pgfsetroundjoin%
\definecolor{currentfill}{rgb}{0.278791,0.062145,0.386592}%
\pgfsetfillcolor{currentfill}%
\pgfsetfillopacity{0.700000}%
\pgfsetlinewidth{0.000000pt}%
\definecolor{currentstroke}{rgb}{0.000000,0.000000,0.000000}%
\pgfsetstrokecolor{currentstroke}%
\pgfsetdash{}{0pt}%
\pgfpathmoveto{\pgfqpoint{3.576547in}{2.949066in}}%
\pgfpathlineto{\pgfqpoint{3.589555in}{2.943511in}}%
\pgfpathlineto{\pgfqpoint{3.602566in}{2.937998in}}%
\pgfpathlineto{\pgfqpoint{3.615582in}{2.932527in}}%
\pgfpathlineto{\pgfqpoint{3.628602in}{2.927097in}}%
\pgfpathlineto{\pgfqpoint{3.620887in}{2.918346in}}%
\pgfpathlineto{\pgfqpoint{3.613166in}{2.909666in}}%
\pgfpathlineto{\pgfqpoint{3.605439in}{2.901056in}}%
\pgfpathlineto{\pgfqpoint{3.597706in}{2.892513in}}%
\pgfpathlineto{\pgfqpoint{3.584675in}{2.897870in}}%
\pgfpathlineto{\pgfqpoint{3.571648in}{2.903268in}}%
\pgfpathlineto{\pgfqpoint{3.558626in}{2.908708in}}%
\pgfpathlineto{\pgfqpoint{3.545608in}{2.914191in}}%
\pgfpathlineto{\pgfqpoint{3.553352in}{2.922802in}}%
\pgfpathlineto{\pgfqpoint{3.561089in}{2.931483in}}%
\pgfpathlineto{\pgfqpoint{3.568821in}{2.940237in}}%
\pgfpathlineto{\pgfqpoint{3.576547in}{2.949066in}}%
\pgfpathclose%
\pgfusepath{fill}%
\end{pgfscope}%
\begin{pgfscope}%
\pgfpathrectangle{\pgfqpoint{1.150000in}{0.150000in}}{\pgfqpoint{5.700000in}{5.700000in}}%
\pgfusepath{clip}%
\pgfsetbuttcap%
\pgfsetroundjoin%
\definecolor{currentfill}{rgb}{0.278791,0.062145,0.386592}%
\pgfsetfillcolor{currentfill}%
\pgfsetfillopacity{0.700000}%
\pgfsetlinewidth{0.000000pt}%
\definecolor{currentstroke}{rgb}{0.000000,0.000000,0.000000}%
\pgfsetstrokecolor{currentstroke}%
\pgfsetdash{}{0pt}%
\pgfpathmoveto{\pgfqpoint{4.064168in}{2.950699in}}%
\pgfpathlineto{\pgfqpoint{4.077265in}{2.945788in}}%
\pgfpathlineto{\pgfqpoint{4.090366in}{2.940911in}}%
\pgfpathlineto{\pgfqpoint{4.103473in}{2.936068in}}%
\pgfpathlineto{\pgfqpoint{4.116584in}{2.931259in}}%
\pgfpathlineto{\pgfqpoint{4.109027in}{2.921823in}}%
\pgfpathlineto{\pgfqpoint{4.101466in}{2.912492in}}%
\pgfpathlineto{\pgfqpoint{4.093899in}{2.903261in}}%
\pgfpathlineto{\pgfqpoint{4.086328in}{2.894126in}}%
\pgfpathlineto{\pgfqpoint{4.073206in}{2.898811in}}%
\pgfpathlineto{\pgfqpoint{4.060088in}{2.903529in}}%
\pgfpathlineto{\pgfqpoint{4.046976in}{2.908281in}}%
\pgfpathlineto{\pgfqpoint{4.033869in}{2.913068in}}%
\pgfpathlineto{\pgfqpoint{4.041451in}{2.922323in}}%
\pgfpathlineto{\pgfqpoint{4.049028in}{2.931677in}}%
\pgfpathlineto{\pgfqpoint{4.056601in}{2.941134in}}%
\pgfpathlineto{\pgfqpoint{4.064168in}{2.950699in}}%
\pgfpathclose%
\pgfusepath{fill}%
\end{pgfscope}%
\begin{pgfscope}%
\pgfpathrectangle{\pgfqpoint{1.150000in}{0.150000in}}{\pgfqpoint{5.700000in}{5.700000in}}%
\pgfusepath{clip}%
\pgfsetbuttcap%
\pgfsetroundjoin%
\definecolor{currentfill}{rgb}{0.277941,0.056324,0.381191}%
\pgfsetfillcolor{currentfill}%
\pgfsetfillopacity{0.700000}%
\pgfsetlinewidth{0.000000pt}%
\definecolor{currentstroke}{rgb}{0.000000,0.000000,0.000000}%
\pgfsetstrokecolor{currentstroke}%
\pgfsetdash{}{0pt}%
\pgfpathmoveto{\pgfqpoint{3.711490in}{2.941231in}}%
\pgfpathlineto{\pgfqpoint{3.724522in}{2.935919in}}%
\pgfpathlineto{\pgfqpoint{3.737559in}{2.930646in}}%
\pgfpathlineto{\pgfqpoint{3.750600in}{2.925412in}}%
\pgfpathlineto{\pgfqpoint{3.763646in}{2.920218in}}%
\pgfpathlineto{\pgfqpoint{3.755974in}{2.911330in}}%
\pgfpathlineto{\pgfqpoint{3.748297in}{2.902519in}}%
\pgfpathlineto{\pgfqpoint{3.740614in}{2.893783in}}%
\pgfpathlineto{\pgfqpoint{3.732926in}{2.885118in}}%
\pgfpathlineto{\pgfqpoint{3.719869in}{2.890227in}}%
\pgfpathlineto{\pgfqpoint{3.706817in}{2.895375in}}%
\pgfpathlineto{\pgfqpoint{3.693770in}{2.900562in}}%
\pgfpathlineto{\pgfqpoint{3.680728in}{2.905789in}}%
\pgfpathlineto{\pgfqpoint{3.688427in}{2.914534in}}%
\pgfpathlineto{\pgfqpoint{3.696120in}{2.923355in}}%
\pgfpathlineto{\pgfqpoint{3.703808in}{2.932253in}}%
\pgfpathlineto{\pgfqpoint{3.711490in}{2.941231in}}%
\pgfpathclose%
\pgfusepath{fill}%
\end{pgfscope}%
\begin{pgfscope}%
\pgfpathrectangle{\pgfqpoint{1.150000in}{0.150000in}}{\pgfqpoint{5.700000in}{5.700000in}}%
\pgfusepath{clip}%
\pgfsetbuttcap%
\pgfsetroundjoin%
\definecolor{currentfill}{rgb}{0.280894,0.078907,0.402329}%
\pgfsetfillcolor{currentfill}%
\pgfsetfillopacity{0.700000}%
\pgfsetlinewidth{0.000000pt}%
\definecolor{currentstroke}{rgb}{0.000000,0.000000,0.000000}%
\pgfsetstrokecolor{currentstroke}%
\pgfsetdash{}{0pt}%
\pgfpathmoveto{\pgfqpoint{4.416906in}{2.974496in}}%
\pgfpathlineto{\pgfqpoint{4.430074in}{2.969779in}}%
\pgfpathlineto{\pgfqpoint{4.443248in}{2.965091in}}%
\pgfpathlineto{\pgfqpoint{4.456427in}{2.960434in}}%
\pgfpathlineto{\pgfqpoint{4.469612in}{2.955806in}}%
\pgfpathlineto{\pgfqpoint{4.462159in}{2.945605in}}%
\pgfpathlineto{\pgfqpoint{4.454702in}{2.935548in}}%
\pgfpathlineto{\pgfqpoint{4.447242in}{2.925630in}}%
\pgfpathlineto{\pgfqpoint{4.439779in}{2.915847in}}%
\pgfpathlineto{\pgfqpoint{4.426583in}{2.920312in}}%
\pgfpathlineto{\pgfqpoint{4.413393in}{2.924806in}}%
\pgfpathlineto{\pgfqpoint{4.400208in}{2.929330in}}%
\pgfpathlineto{\pgfqpoint{4.387028in}{2.933885in}}%
\pgfpathlineto{\pgfqpoint{4.394503in}{2.943826in}}%
\pgfpathlineto{\pgfqpoint{4.401974in}{2.953905in}}%
\pgfpathlineto{\pgfqpoint{4.409442in}{2.964127in}}%
\pgfpathlineto{\pgfqpoint{4.416906in}{2.974496in}}%
\pgfpathclose%
\pgfusepath{fill}%
\end{pgfscope}%
\begin{pgfscope}%
\pgfpathrectangle{\pgfqpoint{1.150000in}{0.150000in}}{\pgfqpoint{5.700000in}{5.700000in}}%
\pgfusepath{clip}%
\pgfsetbuttcap%
\pgfsetroundjoin%
\definecolor{currentfill}{rgb}{0.277941,0.056324,0.381191}%
\pgfsetfillcolor{currentfill}%
\pgfsetfillopacity{0.700000}%
\pgfsetlinewidth{0.000000pt}%
\definecolor{currentstroke}{rgb}{0.000000,0.000000,0.000000}%
\pgfsetstrokecolor{currentstroke}%
\pgfsetdash{}{0pt}%
\pgfpathmoveto{\pgfqpoint{3.846465in}{2.935814in}}%
\pgfpathlineto{\pgfqpoint{3.859524in}{2.930711in}}%
\pgfpathlineto{\pgfqpoint{3.872587in}{2.925644in}}%
\pgfpathlineto{\pgfqpoint{3.885656in}{2.920615in}}%
\pgfpathlineto{\pgfqpoint{3.898729in}{2.915622in}}%
\pgfpathlineto{\pgfqpoint{3.891100in}{2.906601in}}%
\pgfpathlineto{\pgfqpoint{3.883466in}{2.897663in}}%
\pgfpathlineto{\pgfqpoint{3.875826in}{2.888806in}}%
\pgfpathlineto{\pgfqpoint{3.868181in}{2.880026in}}%
\pgfpathlineto{\pgfqpoint{3.855098in}{2.884919in}}%
\pgfpathlineto{\pgfqpoint{3.842019in}{2.889850in}}%
\pgfpathlineto{\pgfqpoint{3.828945in}{2.894817in}}%
\pgfpathlineto{\pgfqpoint{3.815875in}{2.899822in}}%
\pgfpathlineto{\pgfqpoint{3.823531in}{2.908696in}}%
\pgfpathlineto{\pgfqpoint{3.831181in}{2.917651in}}%
\pgfpathlineto{\pgfqpoint{3.838826in}{2.926689in}}%
\pgfpathlineto{\pgfqpoint{3.846465in}{2.935814in}}%
\pgfpathclose%
\pgfusepath{fill}%
\end{pgfscope}%
\begin{pgfscope}%
\pgfpathrectangle{\pgfqpoint{1.150000in}{0.150000in}}{\pgfqpoint{5.700000in}{5.700000in}}%
\pgfusepath{clip}%
\pgfsetbuttcap%
\pgfsetroundjoin%
\definecolor{currentfill}{rgb}{0.278791,0.062145,0.386592}%
\pgfsetfillcolor{currentfill}%
\pgfsetfillopacity{0.700000}%
\pgfsetlinewidth{0.000000pt}%
\definecolor{currentstroke}{rgb}{0.000000,0.000000,0.000000}%
\pgfsetstrokecolor{currentstroke}%
\pgfsetdash{}{0pt}%
\pgfpathmoveto{\pgfqpoint{4.199221in}{2.950675in}}%
\pgfpathlineto{\pgfqpoint{4.212348in}{2.945894in}}%
\pgfpathlineto{\pgfqpoint{4.225480in}{2.941145in}}%
\pgfpathlineto{\pgfqpoint{4.238617in}{2.936428in}}%
\pgfpathlineto{\pgfqpoint{4.251759in}{2.931743in}}%
\pgfpathlineto{\pgfqpoint{4.244242in}{2.922132in}}%
\pgfpathlineto{\pgfqpoint{4.236720in}{2.912636in}}%
\pgfpathlineto{\pgfqpoint{4.229194in}{2.903250in}}%
\pgfpathlineto{\pgfqpoint{4.221664in}{2.893972in}}%
\pgfpathlineto{\pgfqpoint{4.208511in}{2.898519in}}%
\pgfpathlineto{\pgfqpoint{4.195363in}{2.903098in}}%
\pgfpathlineto{\pgfqpoint{4.182220in}{2.907710in}}%
\pgfpathlineto{\pgfqpoint{4.169082in}{2.912354in}}%
\pgfpathlineto{\pgfqpoint{4.176624in}{2.921766in}}%
\pgfpathlineto{\pgfqpoint{4.184161in}{2.931287in}}%
\pgfpathlineto{\pgfqpoint{4.191693in}{2.940922in}}%
\pgfpathlineto{\pgfqpoint{4.199221in}{2.950675in}}%
\pgfpathclose%
\pgfusepath{fill}%
\end{pgfscope}%
\begin{pgfscope}%
\pgfpathrectangle{\pgfqpoint{1.150000in}{0.150000in}}{\pgfqpoint{5.700000in}{5.700000in}}%
\pgfusepath{clip}%
\pgfsetbuttcap%
\pgfsetroundjoin%
\definecolor{currentfill}{rgb}{0.280267,0.073417,0.397163}%
\pgfsetfillcolor{currentfill}%
\pgfsetfillopacity{0.700000}%
\pgfsetlinewidth{0.000000pt}%
\definecolor{currentstroke}{rgb}{0.000000,0.000000,0.000000}%
\pgfsetstrokecolor{currentstroke}%
\pgfsetdash{}{0pt}%
\pgfpathmoveto{\pgfqpoint{3.223486in}{2.964599in}}%
\pgfpathlineto{\pgfqpoint{3.236449in}{2.958287in}}%
\pgfpathlineto{\pgfqpoint{3.249415in}{2.952027in}}%
\pgfpathlineto{\pgfqpoint{3.262384in}{2.945818in}}%
\pgfpathlineto{\pgfqpoint{3.275357in}{2.939659in}}%
\pgfpathlineto{\pgfqpoint{3.267514in}{2.931425in}}%
\pgfpathlineto{\pgfqpoint{3.259665in}{2.923255in}}%
\pgfpathlineto{\pgfqpoint{3.251809in}{2.915146in}}%
\pgfpathlineto{\pgfqpoint{3.243947in}{2.907097in}}%
\pgfpathlineto{\pgfqpoint{3.230963in}{2.913222in}}%
\pgfpathlineto{\pgfqpoint{3.217982in}{2.919398in}}%
\pgfpathlineto{\pgfqpoint{3.205004in}{2.925624in}}%
\pgfpathlineto{\pgfqpoint{3.192030in}{2.931902in}}%
\pgfpathlineto{\pgfqpoint{3.199904in}{2.939980in}}%
\pgfpathlineto{\pgfqpoint{3.207771in}{2.948121in}}%
\pgfpathlineto{\pgfqpoint{3.215632in}{2.956327in}}%
\pgfpathlineto{\pgfqpoint{3.223486in}{2.964599in}}%
\pgfpathclose%
\pgfusepath{fill}%
\end{pgfscope}%
\begin{pgfscope}%
\pgfpathrectangle{\pgfqpoint{1.150000in}{0.150000in}}{\pgfqpoint{5.700000in}{5.700000in}}%
\pgfusepath{clip}%
\pgfsetbuttcap%
\pgfsetroundjoin%
\definecolor{currentfill}{rgb}{0.278791,0.062145,0.386592}%
\pgfsetfillcolor{currentfill}%
\pgfsetfillopacity{0.700000}%
\pgfsetlinewidth{0.000000pt}%
\definecolor{currentstroke}{rgb}{0.000000,0.000000,0.000000}%
\pgfsetstrokecolor{currentstroke}%
\pgfsetdash{}{0pt}%
\pgfpathmoveto{\pgfqpoint{3.358549in}{2.948921in}}%
\pgfpathlineto{\pgfqpoint{3.371529in}{2.942959in}}%
\pgfpathlineto{\pgfqpoint{3.384513in}{2.937044in}}%
\pgfpathlineto{\pgfqpoint{3.397501in}{2.931176in}}%
\pgfpathlineto{\pgfqpoint{3.410493in}{2.925355in}}%
\pgfpathlineto{\pgfqpoint{3.402698in}{2.916957in}}%
\pgfpathlineto{\pgfqpoint{3.394896in}{2.908622in}}%
\pgfpathlineto{\pgfqpoint{3.387089in}{2.900349in}}%
\pgfpathlineto{\pgfqpoint{3.379275in}{2.892136in}}%
\pgfpathlineto{\pgfqpoint{3.366272in}{2.897910in}}%
\pgfpathlineto{\pgfqpoint{3.353272in}{2.903731in}}%
\pgfpathlineto{\pgfqpoint{3.340277in}{2.909599in}}%
\pgfpathlineto{\pgfqpoint{3.327285in}{2.915515in}}%
\pgfpathlineto{\pgfqpoint{3.335111in}{2.923770in}}%
\pgfpathlineto{\pgfqpoint{3.342930in}{2.932088in}}%
\pgfpathlineto{\pgfqpoint{3.350742in}{2.940471in}}%
\pgfpathlineto{\pgfqpoint{3.358549in}{2.948921in}}%
\pgfpathclose%
\pgfusepath{fill}%
\end{pgfscope}%
\begin{pgfscope}%
\pgfpathrectangle{\pgfqpoint{1.150000in}{0.150000in}}{\pgfqpoint{5.700000in}{5.700000in}}%
\pgfusepath{clip}%
\pgfsetbuttcap%
\pgfsetroundjoin%
\definecolor{currentfill}{rgb}{0.277941,0.056324,0.381191}%
\pgfsetfillcolor{currentfill}%
\pgfsetfillopacity{0.700000}%
\pgfsetlinewidth{0.000000pt}%
\definecolor{currentstroke}{rgb}{0.000000,0.000000,0.000000}%
\pgfsetstrokecolor{currentstroke}%
\pgfsetdash{}{0pt}%
\pgfpathmoveto{\pgfqpoint{3.493578in}{2.936552in}}%
\pgfpathlineto{\pgfqpoint{3.506579in}{2.930896in}}%
\pgfpathlineto{\pgfqpoint{3.519585in}{2.925284in}}%
\pgfpathlineto{\pgfqpoint{3.532594in}{2.919716in}}%
\pgfpathlineto{\pgfqpoint{3.545608in}{2.914191in}}%
\pgfpathlineto{\pgfqpoint{3.537858in}{2.905648in}}%
\pgfpathlineto{\pgfqpoint{3.530103in}{2.897170in}}%
\pgfpathlineto{\pgfqpoint{3.522342in}{2.888756in}}%
\pgfpathlineto{\pgfqpoint{3.514575in}{2.880403in}}%
\pgfpathlineto{\pgfqpoint{3.501550in}{2.885868in}}%
\pgfpathlineto{\pgfqpoint{3.488529in}{2.891377in}}%
\pgfpathlineto{\pgfqpoint{3.475513in}{2.896929in}}%
\pgfpathlineto{\pgfqpoint{3.462501in}{2.902524in}}%
\pgfpathlineto{\pgfqpoint{3.470279in}{2.910933in}}%
\pgfpathlineto{\pgfqpoint{3.478052in}{2.919405in}}%
\pgfpathlineto{\pgfqpoint{3.485818in}{2.927944in}}%
\pgfpathlineto{\pgfqpoint{3.493578in}{2.936552in}}%
\pgfpathclose%
\pgfusepath{fill}%
\end{pgfscope}%
\begin{pgfscope}%
\pgfpathrectangle{\pgfqpoint{1.150000in}{0.150000in}}{\pgfqpoint{5.700000in}{5.700000in}}%
\pgfusepath{clip}%
\pgfsetbuttcap%
\pgfsetroundjoin%
\definecolor{currentfill}{rgb}{0.277941,0.056324,0.381191}%
\pgfsetfillcolor{currentfill}%
\pgfsetfillopacity{0.700000}%
\pgfsetlinewidth{0.000000pt}%
\definecolor{currentstroke}{rgb}{0.000000,0.000000,0.000000}%
\pgfsetstrokecolor{currentstroke}%
\pgfsetdash{}{0pt}%
\pgfpathmoveto{\pgfqpoint{3.981491in}{2.932560in}}%
\pgfpathlineto{\pgfqpoint{3.994578in}{2.927634in}}%
\pgfpathlineto{\pgfqpoint{4.007670in}{2.922744in}}%
\pgfpathlineto{\pgfqpoint{4.020767in}{2.917889in}}%
\pgfpathlineto{\pgfqpoint{4.033869in}{2.913068in}}%
\pgfpathlineto{\pgfqpoint{4.026282in}{2.903909in}}%
\pgfpathlineto{\pgfqpoint{4.018690in}{2.894842in}}%
\pgfpathlineto{\pgfqpoint{4.011093in}{2.885862in}}%
\pgfpathlineto{\pgfqpoint{4.003491in}{2.876967in}}%
\pgfpathlineto{\pgfqpoint{3.990378in}{2.881676in}}%
\pgfpathlineto{\pgfqpoint{3.977270in}{2.886420in}}%
\pgfpathlineto{\pgfqpoint{3.964168in}{2.891199in}}%
\pgfpathlineto{\pgfqpoint{3.951070in}{2.896012in}}%
\pgfpathlineto{\pgfqpoint{3.958683in}{2.905014in}}%
\pgfpathlineto{\pgfqpoint{3.966291in}{2.914104in}}%
\pgfpathlineto{\pgfqpoint{3.973893in}{2.923284in}}%
\pgfpathlineto{\pgfqpoint{3.981491in}{2.932560in}}%
\pgfpathclose%
\pgfusepath{fill}%
\end{pgfscope}%
\begin{pgfscope}%
\pgfpathrectangle{\pgfqpoint{1.150000in}{0.150000in}}{\pgfqpoint{5.700000in}{5.700000in}}%
\pgfusepath{clip}%
\pgfsetbuttcap%
\pgfsetroundjoin%
\definecolor{currentfill}{rgb}{0.277941,0.056324,0.381191}%
\pgfsetfillcolor{currentfill}%
\pgfsetfillopacity{0.700000}%
\pgfsetlinewidth{0.000000pt}%
\definecolor{currentstroke}{rgb}{0.000000,0.000000,0.000000}%
\pgfsetstrokecolor{currentstroke}%
\pgfsetdash{}{0pt}%
\pgfpathmoveto{\pgfqpoint{3.628602in}{2.927097in}}%
\pgfpathlineto{\pgfqpoint{3.641627in}{2.921709in}}%
\pgfpathlineto{\pgfqpoint{3.654656in}{2.916362in}}%
\pgfpathlineto{\pgfqpoint{3.667690in}{2.911055in}}%
\pgfpathlineto{\pgfqpoint{3.680728in}{2.905789in}}%
\pgfpathlineto{\pgfqpoint{3.673023in}{2.897115in}}%
\pgfpathlineto{\pgfqpoint{3.665313in}{2.888510in}}%
\pgfpathlineto{\pgfqpoint{3.657596in}{2.879971in}}%
\pgfpathlineto{\pgfqpoint{3.649875in}{2.871497in}}%
\pgfpathlineto{\pgfqpoint{3.636826in}{2.876690in}}%
\pgfpathlineto{\pgfqpoint{3.623781in}{2.881924in}}%
\pgfpathlineto{\pgfqpoint{3.610741in}{2.887198in}}%
\pgfpathlineto{\pgfqpoint{3.597706in}{2.892513in}}%
\pgfpathlineto{\pgfqpoint{3.605439in}{2.901056in}}%
\pgfpathlineto{\pgfqpoint{3.613166in}{2.909666in}}%
\pgfpathlineto{\pgfqpoint{3.620887in}{2.918346in}}%
\pgfpathlineto{\pgfqpoint{3.628602in}{2.927097in}}%
\pgfpathclose%
\pgfusepath{fill}%
\end{pgfscope}%
\begin{pgfscope}%
\pgfpathrectangle{\pgfqpoint{1.150000in}{0.150000in}}{\pgfqpoint{5.700000in}{5.700000in}}%
\pgfusepath{clip}%
\pgfsetbuttcap%
\pgfsetroundjoin%
\definecolor{currentfill}{rgb}{0.279566,0.067836,0.391917}%
\pgfsetfillcolor{currentfill}%
\pgfsetfillopacity{0.700000}%
\pgfsetlinewidth{0.000000pt}%
\definecolor{currentstroke}{rgb}{0.000000,0.000000,0.000000}%
\pgfsetstrokecolor{currentstroke}%
\pgfsetdash{}{0pt}%
\pgfpathmoveto{\pgfqpoint{4.334365in}{2.952410in}}%
\pgfpathlineto{\pgfqpoint{4.347523in}{2.947733in}}%
\pgfpathlineto{\pgfqpoint{4.360686in}{2.943086in}}%
\pgfpathlineto{\pgfqpoint{4.373854in}{2.938470in}}%
\pgfpathlineto{\pgfqpoint{4.387028in}{2.933885in}}%
\pgfpathlineto{\pgfqpoint{4.379550in}{2.924076in}}%
\pgfpathlineto{\pgfqpoint{4.372067in}{2.914395in}}%
\pgfpathlineto{\pgfqpoint{4.364581in}{2.904837in}}%
\pgfpathlineto{\pgfqpoint{4.357090in}{2.895397in}}%
\pgfpathlineto{\pgfqpoint{4.343905in}{2.899832in}}%
\pgfpathlineto{\pgfqpoint{4.330725in}{2.904297in}}%
\pgfpathlineto{\pgfqpoint{4.317550in}{2.908793in}}%
\pgfpathlineto{\pgfqpoint{4.304381in}{2.913321in}}%
\pgfpathlineto{\pgfqpoint{4.311883in}{2.922906in}}%
\pgfpathlineto{\pgfqpoint{4.319381in}{2.932613in}}%
\pgfpathlineto{\pgfqpoint{4.326875in}{2.942446in}}%
\pgfpathlineto{\pgfqpoint{4.334365in}{2.952410in}}%
\pgfpathclose%
\pgfusepath{fill}%
\end{pgfscope}%
\begin{pgfscope}%
\pgfpathrectangle{\pgfqpoint{1.150000in}{0.150000in}}{\pgfqpoint{5.700000in}{5.700000in}}%
\pgfusepath{clip}%
\pgfsetbuttcap%
\pgfsetroundjoin%
\definecolor{currentfill}{rgb}{0.277941,0.056324,0.381191}%
\pgfsetfillcolor{currentfill}%
\pgfsetfillopacity{0.700000}%
\pgfsetlinewidth{0.000000pt}%
\definecolor{currentstroke}{rgb}{0.000000,0.000000,0.000000}%
\pgfsetstrokecolor{currentstroke}%
\pgfsetdash{}{0pt}%
\pgfpathmoveto{\pgfqpoint{4.116584in}{2.931259in}}%
\pgfpathlineto{\pgfqpoint{4.129701in}{2.926483in}}%
\pgfpathlineto{\pgfqpoint{4.142823in}{2.921740in}}%
\pgfpathlineto{\pgfqpoint{4.155950in}{2.917031in}}%
\pgfpathlineto{\pgfqpoint{4.169082in}{2.912354in}}%
\pgfpathlineto{\pgfqpoint{4.161536in}{2.903048in}}%
\pgfpathlineto{\pgfqpoint{4.153985in}{2.893843in}}%
\pgfpathlineto{\pgfqpoint{4.146430in}{2.884735in}}%
\pgfpathlineto{\pgfqpoint{4.138870in}{2.875720in}}%
\pgfpathlineto{\pgfqpoint{4.125727in}{2.880272in}}%
\pgfpathlineto{\pgfqpoint{4.112588in}{2.884857in}}%
\pgfpathlineto{\pgfqpoint{4.099456in}{2.889475in}}%
\pgfpathlineto{\pgfqpoint{4.086328in}{2.894126in}}%
\pgfpathlineto{\pgfqpoint{4.093899in}{2.903261in}}%
\pgfpathlineto{\pgfqpoint{4.101466in}{2.912492in}}%
\pgfpathlineto{\pgfqpoint{4.109027in}{2.921823in}}%
\pgfpathlineto{\pgfqpoint{4.116584in}{2.931259in}}%
\pgfpathclose%
\pgfusepath{fill}%
\end{pgfscope}%
\begin{pgfscope}%
\pgfpathrectangle{\pgfqpoint{1.150000in}{0.150000in}}{\pgfqpoint{5.700000in}{5.700000in}}%
\pgfusepath{clip}%
\pgfsetbuttcap%
\pgfsetroundjoin%
\definecolor{currentfill}{rgb}{0.277018,0.050344,0.375715}%
\pgfsetfillcolor{currentfill}%
\pgfsetfillopacity{0.700000}%
\pgfsetlinewidth{0.000000pt}%
\definecolor{currentstroke}{rgb}{0.000000,0.000000,0.000000}%
\pgfsetstrokecolor{currentstroke}%
\pgfsetdash{}{0pt}%
\pgfpathmoveto{\pgfqpoint{3.763646in}{2.920218in}}%
\pgfpathlineto{\pgfqpoint{3.776696in}{2.915061in}}%
\pgfpathlineto{\pgfqpoint{3.789751in}{2.909944in}}%
\pgfpathlineto{\pgfqpoint{3.802811in}{2.904864in}}%
\pgfpathlineto{\pgfqpoint{3.815875in}{2.899822in}}%
\pgfpathlineto{\pgfqpoint{3.808214in}{2.891025in}}%
\pgfpathlineto{\pgfqpoint{3.800548in}{2.882302in}}%
\pgfpathlineto{\pgfqpoint{3.792876in}{2.873650in}}%
\pgfpathlineto{\pgfqpoint{3.785199in}{2.865067in}}%
\pgfpathlineto{\pgfqpoint{3.772123in}{2.870023in}}%
\pgfpathlineto{\pgfqpoint{3.759053in}{2.875017in}}%
\pgfpathlineto{\pgfqpoint{3.745987in}{2.880048in}}%
\pgfpathlineto{\pgfqpoint{3.732926in}{2.885118in}}%
\pgfpathlineto{\pgfqpoint{3.740614in}{2.893783in}}%
\pgfpathlineto{\pgfqpoint{3.748297in}{2.902519in}}%
\pgfpathlineto{\pgfqpoint{3.755974in}{2.911330in}}%
\pgfpathlineto{\pgfqpoint{3.763646in}{2.920218in}}%
\pgfpathclose%
\pgfusepath{fill}%
\end{pgfscope}%
\begin{pgfscope}%
\pgfpathrectangle{\pgfqpoint{1.150000in}{0.150000in}}{\pgfqpoint{5.700000in}{5.700000in}}%
\pgfusepath{clip}%
\pgfsetbuttcap%
\pgfsetroundjoin%
\definecolor{currentfill}{rgb}{0.280267,0.073417,0.397163}%
\pgfsetfillcolor{currentfill}%
\pgfsetfillopacity{0.700000}%
\pgfsetlinewidth{0.000000pt}%
\definecolor{currentstroke}{rgb}{0.000000,0.000000,0.000000}%
\pgfsetstrokecolor{currentstroke}%
\pgfsetdash{}{0pt}%
\pgfpathmoveto{\pgfqpoint{4.469612in}{2.955806in}}%
\pgfpathlineto{\pgfqpoint{4.482801in}{2.951209in}}%
\pgfpathlineto{\pgfqpoint{4.495997in}{2.946641in}}%
\pgfpathlineto{\pgfqpoint{4.509198in}{2.942103in}}%
\pgfpathlineto{\pgfqpoint{4.522404in}{2.937594in}}%
\pgfpathlineto{\pgfqpoint{4.514963in}{2.927560in}}%
\pgfpathlineto{\pgfqpoint{4.507518in}{2.917668in}}%
\pgfpathlineto{\pgfqpoint{4.500070in}{2.907912in}}%
\pgfpathlineto{\pgfqpoint{4.492618in}{2.898287in}}%
\pgfpathlineto{\pgfqpoint{4.479400in}{2.902633in}}%
\pgfpathlineto{\pgfqpoint{4.466187in}{2.907008in}}%
\pgfpathlineto{\pgfqpoint{4.452980in}{2.911413in}}%
\pgfpathlineto{\pgfqpoint{4.439779in}{2.915847in}}%
\pgfpathlineto{\pgfqpoint{4.447242in}{2.925630in}}%
\pgfpathlineto{\pgfqpoint{4.454702in}{2.935548in}}%
\pgfpathlineto{\pgfqpoint{4.462159in}{2.945605in}}%
\pgfpathlineto{\pgfqpoint{4.469612in}{2.955806in}}%
\pgfpathclose%
\pgfusepath{fill}%
\end{pgfscope}%
\begin{pgfscope}%
\pgfpathrectangle{\pgfqpoint{1.150000in}{0.150000in}}{\pgfqpoint{5.700000in}{5.700000in}}%
\pgfusepath{clip}%
\pgfsetbuttcap%
\pgfsetroundjoin%
\definecolor{currentfill}{rgb}{0.277018,0.050344,0.375715}%
\pgfsetfillcolor{currentfill}%
\pgfsetfillopacity{0.700000}%
\pgfsetlinewidth{0.000000pt}%
\definecolor{currentstroke}{rgb}{0.000000,0.000000,0.000000}%
\pgfsetstrokecolor{currentstroke}%
\pgfsetdash{}{0pt}%
\pgfpathmoveto{\pgfqpoint{3.898729in}{2.915622in}}%
\pgfpathlineto{\pgfqpoint{3.911807in}{2.910666in}}%
\pgfpathlineto{\pgfqpoint{3.924889in}{2.905746in}}%
\pgfpathlineto{\pgfqpoint{3.937977in}{2.900861in}}%
\pgfpathlineto{\pgfqpoint{3.951070in}{2.896012in}}%
\pgfpathlineto{\pgfqpoint{3.943452in}{2.887094in}}%
\pgfpathlineto{\pgfqpoint{3.935828in}{2.878257in}}%
\pgfpathlineto{\pgfqpoint{3.928200in}{2.869497in}}%
\pgfpathlineto{\pgfqpoint{3.920566in}{2.860811in}}%
\pgfpathlineto{\pgfqpoint{3.907462in}{2.865561in}}%
\pgfpathlineto{\pgfqpoint{3.894364in}{2.870347in}}%
\pgfpathlineto{\pgfqpoint{3.881270in}{2.875168in}}%
\pgfpathlineto{\pgfqpoint{3.868181in}{2.880026in}}%
\pgfpathlineto{\pgfqpoint{3.875826in}{2.888806in}}%
\pgfpathlineto{\pgfqpoint{3.883466in}{2.897663in}}%
\pgfpathlineto{\pgfqpoint{3.891100in}{2.906601in}}%
\pgfpathlineto{\pgfqpoint{3.898729in}{2.915622in}}%
\pgfpathclose%
\pgfusepath{fill}%
\end{pgfscope}%
\begin{pgfscope}%
\pgfpathrectangle{\pgfqpoint{1.150000in}{0.150000in}}{\pgfqpoint{5.700000in}{5.700000in}}%
\pgfusepath{clip}%
\pgfsetbuttcap%
\pgfsetroundjoin%
\definecolor{currentfill}{rgb}{0.279566,0.067836,0.391917}%
\pgfsetfillcolor{currentfill}%
\pgfsetfillopacity{0.700000}%
\pgfsetlinewidth{0.000000pt}%
\definecolor{currentstroke}{rgb}{0.000000,0.000000,0.000000}%
\pgfsetstrokecolor{currentstroke}%
\pgfsetdash{}{0pt}%
\pgfpathmoveto{\pgfqpoint{3.275357in}{2.939659in}}%
\pgfpathlineto{\pgfqpoint{3.288334in}{2.933549in}}%
\pgfpathlineto{\pgfqpoint{3.301314in}{2.927489in}}%
\pgfpathlineto{\pgfqpoint{3.314298in}{2.921478in}}%
\pgfpathlineto{\pgfqpoint{3.327285in}{2.915515in}}%
\pgfpathlineto{\pgfqpoint{3.319454in}{2.907320in}}%
\pgfpathlineto{\pgfqpoint{3.311616in}{2.899186in}}%
\pgfpathlineto{\pgfqpoint{3.303772in}{2.891109in}}%
\pgfpathlineto{\pgfqpoint{3.295922in}{2.883089in}}%
\pgfpathlineto{\pgfqpoint{3.282923in}{2.889018in}}%
\pgfpathlineto{\pgfqpoint{3.269927in}{2.894995in}}%
\pgfpathlineto{\pgfqpoint{3.256935in}{2.901021in}}%
\pgfpathlineto{\pgfqpoint{3.243947in}{2.907097in}}%
\pgfpathlineto{\pgfqpoint{3.251809in}{2.915146in}}%
\pgfpathlineto{\pgfqpoint{3.259665in}{2.923255in}}%
\pgfpathlineto{\pgfqpoint{3.267514in}{2.931425in}}%
\pgfpathlineto{\pgfqpoint{3.275357in}{2.939659in}}%
\pgfpathclose%
\pgfusepath{fill}%
\end{pgfscope}%
\begin{pgfscope}%
\pgfpathrectangle{\pgfqpoint{1.150000in}{0.150000in}}{\pgfqpoint{5.700000in}{5.700000in}}%
\pgfusepath{clip}%
\pgfsetbuttcap%
\pgfsetroundjoin%
\definecolor{currentfill}{rgb}{0.278791,0.062145,0.386592}%
\pgfsetfillcolor{currentfill}%
\pgfsetfillopacity{0.700000}%
\pgfsetlinewidth{0.000000pt}%
\definecolor{currentstroke}{rgb}{0.000000,0.000000,0.000000}%
\pgfsetstrokecolor{currentstroke}%
\pgfsetdash{}{0pt}%
\pgfpathmoveto{\pgfqpoint{4.251759in}{2.931743in}}%
\pgfpathlineto{\pgfqpoint{4.264907in}{2.927090in}}%
\pgfpathlineto{\pgfqpoint{4.278060in}{2.922469in}}%
\pgfpathlineto{\pgfqpoint{4.291218in}{2.917879in}}%
\pgfpathlineto{\pgfqpoint{4.304381in}{2.913321in}}%
\pgfpathlineto{\pgfqpoint{4.296875in}{2.903852in}}%
\pgfpathlineto{\pgfqpoint{4.289365in}{2.894495in}}%
\pgfpathlineto{\pgfqpoint{4.281850in}{2.885245in}}%
\pgfpathlineto{\pgfqpoint{4.274331in}{2.876099in}}%
\pgfpathlineto{\pgfqpoint{4.261156in}{2.880520in}}%
\pgfpathlineto{\pgfqpoint{4.247987in}{2.884972in}}%
\pgfpathlineto{\pgfqpoint{4.234823in}{2.889456in}}%
\pgfpathlineto{\pgfqpoint{4.221664in}{2.893972in}}%
\pgfpathlineto{\pgfqpoint{4.229194in}{2.903250in}}%
\pgfpathlineto{\pgfqpoint{4.236720in}{2.912636in}}%
\pgfpathlineto{\pgfqpoint{4.244242in}{2.922132in}}%
\pgfpathlineto{\pgfqpoint{4.251759in}{2.931743in}}%
\pgfpathclose%
\pgfusepath{fill}%
\end{pgfscope}%
\begin{pgfscope}%
\pgfpathrectangle{\pgfqpoint{1.150000in}{0.150000in}}{\pgfqpoint{5.700000in}{5.700000in}}%
\pgfusepath{clip}%
\pgfsetbuttcap%
\pgfsetroundjoin%
\definecolor{currentfill}{rgb}{0.277941,0.056324,0.381191}%
\pgfsetfillcolor{currentfill}%
\pgfsetfillopacity{0.700000}%
\pgfsetlinewidth{0.000000pt}%
\definecolor{currentstroke}{rgb}{0.000000,0.000000,0.000000}%
\pgfsetstrokecolor{currentstroke}%
\pgfsetdash{}{0pt}%
\pgfpathmoveto{\pgfqpoint{3.410493in}{2.925355in}}%
\pgfpathlineto{\pgfqpoint{3.423489in}{2.919579in}}%
\pgfpathlineto{\pgfqpoint{3.436489in}{2.913849in}}%
\pgfpathlineto{\pgfqpoint{3.449493in}{2.908164in}}%
\pgfpathlineto{\pgfqpoint{3.462501in}{2.902524in}}%
\pgfpathlineto{\pgfqpoint{3.454717in}{2.894178in}}%
\pgfpathlineto{\pgfqpoint{3.446926in}{2.885892in}}%
\pgfpathlineto{\pgfqpoint{3.439130in}{2.877665in}}%
\pgfpathlineto{\pgfqpoint{3.431327in}{2.869494in}}%
\pgfpathlineto{\pgfqpoint{3.418308in}{2.875087in}}%
\pgfpathlineto{\pgfqpoint{3.405293in}{2.880725in}}%
\pgfpathlineto{\pgfqpoint{3.392282in}{2.886408in}}%
\pgfpathlineto{\pgfqpoint{3.379275in}{2.892136in}}%
\pgfpathlineto{\pgfqpoint{3.387089in}{2.900349in}}%
\pgfpathlineto{\pgfqpoint{3.394896in}{2.908622in}}%
\pgfpathlineto{\pgfqpoint{3.402698in}{2.916957in}}%
\pgfpathlineto{\pgfqpoint{3.410493in}{2.925355in}}%
\pgfpathclose%
\pgfusepath{fill}%
\end{pgfscope}%
\begin{pgfscope}%
\pgfpathrectangle{\pgfqpoint{1.150000in}{0.150000in}}{\pgfqpoint{5.700000in}{5.700000in}}%
\pgfusepath{clip}%
\pgfsetbuttcap%
\pgfsetroundjoin%
\definecolor{currentfill}{rgb}{0.277018,0.050344,0.375715}%
\pgfsetfillcolor{currentfill}%
\pgfsetfillopacity{0.700000}%
\pgfsetlinewidth{0.000000pt}%
\definecolor{currentstroke}{rgb}{0.000000,0.000000,0.000000}%
\pgfsetstrokecolor{currentstroke}%
\pgfsetdash{}{0pt}%
\pgfpathmoveto{\pgfqpoint{3.545608in}{2.914191in}}%
\pgfpathlineto{\pgfqpoint{3.558626in}{2.908708in}}%
\pgfpathlineto{\pgfqpoint{3.571648in}{2.903268in}}%
\pgfpathlineto{\pgfqpoint{3.584675in}{2.897870in}}%
\pgfpathlineto{\pgfqpoint{3.597706in}{2.892513in}}%
\pgfpathlineto{\pgfqpoint{3.589967in}{2.884035in}}%
\pgfpathlineto{\pgfqpoint{3.582223in}{2.875620in}}%
\pgfpathlineto{\pgfqpoint{3.574473in}{2.867264in}}%
\pgfpathlineto{\pgfqpoint{3.566717in}{2.858967in}}%
\pgfpathlineto{\pgfqpoint{3.553675in}{2.864263in}}%
\pgfpathlineto{\pgfqpoint{3.540637in}{2.869601in}}%
\pgfpathlineto{\pgfqpoint{3.527604in}{2.874981in}}%
\pgfpathlineto{\pgfqpoint{3.514575in}{2.880403in}}%
\pgfpathlineto{\pgfqpoint{3.522342in}{2.888756in}}%
\pgfpathlineto{\pgfqpoint{3.530103in}{2.897170in}}%
\pgfpathlineto{\pgfqpoint{3.537858in}{2.905648in}}%
\pgfpathlineto{\pgfqpoint{3.545608in}{2.914191in}}%
\pgfpathclose%
\pgfusepath{fill}%
\end{pgfscope}%
\begin{pgfscope}%
\pgfpathrectangle{\pgfqpoint{1.150000in}{0.150000in}}{\pgfqpoint{5.700000in}{5.700000in}}%
\pgfusepath{clip}%
\pgfsetbuttcap%
\pgfsetroundjoin%
\definecolor{currentfill}{rgb}{0.277018,0.050344,0.375715}%
\pgfsetfillcolor{currentfill}%
\pgfsetfillopacity{0.700000}%
\pgfsetlinewidth{0.000000pt}%
\definecolor{currentstroke}{rgb}{0.000000,0.000000,0.000000}%
\pgfsetstrokecolor{currentstroke}%
\pgfsetdash{}{0pt}%
\pgfpathmoveto{\pgfqpoint{4.033869in}{2.913068in}}%
\pgfpathlineto{\pgfqpoint{4.046976in}{2.908281in}}%
\pgfpathlineto{\pgfqpoint{4.060088in}{2.903529in}}%
\pgfpathlineto{\pgfqpoint{4.073206in}{2.898811in}}%
\pgfpathlineto{\pgfqpoint{4.086328in}{2.894126in}}%
\pgfpathlineto{\pgfqpoint{4.078752in}{2.885084in}}%
\pgfpathlineto{\pgfqpoint{4.071171in}{2.876130in}}%
\pgfpathlineto{\pgfqpoint{4.063585in}{2.867261in}}%
\pgfpathlineto{\pgfqpoint{4.055994in}{2.858473in}}%
\pgfpathlineto{\pgfqpoint{4.042860in}{2.863046in}}%
\pgfpathlineto{\pgfqpoint{4.029732in}{2.867652in}}%
\pgfpathlineto{\pgfqpoint{4.016609in}{2.872293in}}%
\pgfpathlineto{\pgfqpoint{4.003491in}{2.876967in}}%
\pgfpathlineto{\pgfqpoint{4.011093in}{2.885862in}}%
\pgfpathlineto{\pgfqpoint{4.018690in}{2.894842in}}%
\pgfpathlineto{\pgfqpoint{4.026282in}{2.903909in}}%
\pgfpathlineto{\pgfqpoint{4.033869in}{2.913068in}}%
\pgfpathclose%
\pgfusepath{fill}%
\end{pgfscope}%
\begin{pgfscope}%
\pgfpathrectangle{\pgfqpoint{1.150000in}{0.150000in}}{\pgfqpoint{5.700000in}{5.700000in}}%
\pgfusepath{clip}%
\pgfsetbuttcap%
\pgfsetroundjoin%
\definecolor{currentfill}{rgb}{0.277018,0.050344,0.375715}%
\pgfsetfillcolor{currentfill}%
\pgfsetfillopacity{0.700000}%
\pgfsetlinewidth{0.000000pt}%
\definecolor{currentstroke}{rgb}{0.000000,0.000000,0.000000}%
\pgfsetstrokecolor{currentstroke}%
\pgfsetdash{}{0pt}%
\pgfpathmoveto{\pgfqpoint{3.680728in}{2.905789in}}%
\pgfpathlineto{\pgfqpoint{3.693770in}{2.900562in}}%
\pgfpathlineto{\pgfqpoint{3.706817in}{2.895375in}}%
\pgfpathlineto{\pgfqpoint{3.719869in}{2.890227in}}%
\pgfpathlineto{\pgfqpoint{3.732926in}{2.885118in}}%
\pgfpathlineto{\pgfqpoint{3.725232in}{2.876522in}}%
\pgfpathlineto{\pgfqpoint{3.717532in}{2.867992in}}%
\pgfpathlineto{\pgfqpoint{3.709827in}{2.859525in}}%
\pgfpathlineto{\pgfqpoint{3.702116in}{2.851119in}}%
\pgfpathlineto{\pgfqpoint{3.689049in}{2.856155in}}%
\pgfpathlineto{\pgfqpoint{3.675986in}{2.861229in}}%
\pgfpathlineto{\pgfqpoint{3.662928in}{2.866343in}}%
\pgfpathlineto{\pgfqpoint{3.649875in}{2.871497in}}%
\pgfpathlineto{\pgfqpoint{3.657596in}{2.879971in}}%
\pgfpathlineto{\pgfqpoint{3.665313in}{2.888510in}}%
\pgfpathlineto{\pgfqpoint{3.673023in}{2.897115in}}%
\pgfpathlineto{\pgfqpoint{3.680728in}{2.905789in}}%
\pgfpathclose%
\pgfusepath{fill}%
\end{pgfscope}%
\begin{pgfscope}%
\pgfpathrectangle{\pgfqpoint{1.150000in}{0.150000in}}{\pgfqpoint{5.700000in}{5.700000in}}%
\pgfusepath{clip}%
\pgfsetbuttcap%
\pgfsetroundjoin%
\definecolor{currentfill}{rgb}{0.279566,0.067836,0.391917}%
\pgfsetfillcolor{currentfill}%
\pgfsetfillopacity{0.700000}%
\pgfsetlinewidth{0.000000pt}%
\definecolor{currentstroke}{rgb}{0.000000,0.000000,0.000000}%
\pgfsetstrokecolor{currentstroke}%
\pgfsetdash{}{0pt}%
\pgfpathmoveto{\pgfqpoint{4.387028in}{2.933885in}}%
\pgfpathlineto{\pgfqpoint{4.400208in}{2.929330in}}%
\pgfpathlineto{\pgfqpoint{4.413393in}{2.924806in}}%
\pgfpathlineto{\pgfqpoint{4.426583in}{2.920312in}}%
\pgfpathlineto{\pgfqpoint{4.439779in}{2.915847in}}%
\pgfpathlineto{\pgfqpoint{4.432312in}{2.906194in}}%
\pgfpathlineto{\pgfqpoint{4.424841in}{2.896665in}}%
\pgfpathlineto{\pgfqpoint{4.417366in}{2.887255in}}%
\pgfpathlineto{\pgfqpoint{4.409887in}{2.877961in}}%
\pgfpathlineto{\pgfqpoint{4.396679in}{2.882275in}}%
\pgfpathlineto{\pgfqpoint{4.383477in}{2.886619in}}%
\pgfpathlineto{\pgfqpoint{4.370281in}{2.890993in}}%
\pgfpathlineto{\pgfqpoint{4.357090in}{2.895397in}}%
\pgfpathlineto{\pgfqpoint{4.364581in}{2.904837in}}%
\pgfpathlineto{\pgfqpoint{4.372067in}{2.914395in}}%
\pgfpathlineto{\pgfqpoint{4.379550in}{2.924076in}}%
\pgfpathlineto{\pgfqpoint{4.387028in}{2.933885in}}%
\pgfpathclose%
\pgfusepath{fill}%
\end{pgfscope}%
\begin{pgfscope}%
\pgfpathrectangle{\pgfqpoint{1.150000in}{0.150000in}}{\pgfqpoint{5.700000in}{5.700000in}}%
\pgfusepath{clip}%
\pgfsetbuttcap%
\pgfsetroundjoin%
\definecolor{currentfill}{rgb}{0.277018,0.050344,0.375715}%
\pgfsetfillcolor{currentfill}%
\pgfsetfillopacity{0.700000}%
\pgfsetlinewidth{0.000000pt}%
\definecolor{currentstroke}{rgb}{0.000000,0.000000,0.000000}%
\pgfsetstrokecolor{currentstroke}%
\pgfsetdash{}{0pt}%
\pgfpathmoveto{\pgfqpoint{3.815875in}{2.899822in}}%
\pgfpathlineto{\pgfqpoint{3.828945in}{2.894817in}}%
\pgfpathlineto{\pgfqpoint{3.842019in}{2.889850in}}%
\pgfpathlineto{\pgfqpoint{3.855098in}{2.884919in}}%
\pgfpathlineto{\pgfqpoint{3.868181in}{2.880026in}}%
\pgfpathlineto{\pgfqpoint{3.860531in}{2.871319in}}%
\pgfpathlineto{\pgfqpoint{3.852876in}{2.862684in}}%
\pgfpathlineto{\pgfqpoint{3.845215in}{2.854117in}}%
\pgfpathlineto{\pgfqpoint{3.837548in}{2.845615in}}%
\pgfpathlineto{\pgfqpoint{3.824454in}{2.850423in}}%
\pgfpathlineto{\pgfqpoint{3.811364in}{2.855267in}}%
\pgfpathlineto{\pgfqpoint{3.798279in}{2.860148in}}%
\pgfpathlineto{\pgfqpoint{3.785199in}{2.865067in}}%
\pgfpathlineto{\pgfqpoint{3.792876in}{2.873650in}}%
\pgfpathlineto{\pgfqpoint{3.800548in}{2.882302in}}%
\pgfpathlineto{\pgfqpoint{3.808214in}{2.891025in}}%
\pgfpathlineto{\pgfqpoint{3.815875in}{2.899822in}}%
\pgfpathclose%
\pgfusepath{fill}%
\end{pgfscope}%
\begin{pgfscope}%
\pgfpathrectangle{\pgfqpoint{1.150000in}{0.150000in}}{\pgfqpoint{5.700000in}{5.700000in}}%
\pgfusepath{clip}%
\pgfsetbuttcap%
\pgfsetroundjoin%
\definecolor{currentfill}{rgb}{0.277941,0.056324,0.381191}%
\pgfsetfillcolor{currentfill}%
\pgfsetfillopacity{0.700000}%
\pgfsetlinewidth{0.000000pt}%
\definecolor{currentstroke}{rgb}{0.000000,0.000000,0.000000}%
\pgfsetstrokecolor{currentstroke}%
\pgfsetdash{}{0pt}%
\pgfpathmoveto{\pgfqpoint{4.169082in}{2.912354in}}%
\pgfpathlineto{\pgfqpoint{4.182220in}{2.907710in}}%
\pgfpathlineto{\pgfqpoint{4.195363in}{2.903098in}}%
\pgfpathlineto{\pgfqpoint{4.208511in}{2.898519in}}%
\pgfpathlineto{\pgfqpoint{4.221664in}{2.893972in}}%
\pgfpathlineto{\pgfqpoint{4.214129in}{2.884795in}}%
\pgfpathlineto{\pgfqpoint{4.206590in}{2.875716in}}%
\pgfpathlineto{\pgfqpoint{4.199045in}{2.866732in}}%
\pgfpathlineto{\pgfqpoint{4.191496in}{2.857837in}}%
\pgfpathlineto{\pgfqpoint{4.178332in}{2.862259in}}%
\pgfpathlineto{\pgfqpoint{4.165172in}{2.866714in}}%
\pgfpathlineto{\pgfqpoint{4.152019in}{2.871201in}}%
\pgfpathlineto{\pgfqpoint{4.138870in}{2.875720in}}%
\pgfpathlineto{\pgfqpoint{4.146430in}{2.884735in}}%
\pgfpathlineto{\pgfqpoint{4.153985in}{2.893843in}}%
\pgfpathlineto{\pgfqpoint{4.161536in}{2.903048in}}%
\pgfpathlineto{\pgfqpoint{4.169082in}{2.912354in}}%
\pgfpathclose%
\pgfusepath{fill}%
\end{pgfscope}%
\begin{pgfscope}%
\pgfpathrectangle{\pgfqpoint{1.150000in}{0.150000in}}{\pgfqpoint{5.700000in}{5.700000in}}%
\pgfusepath{clip}%
\pgfsetbuttcap%
\pgfsetroundjoin%
\definecolor{currentfill}{rgb}{0.279566,0.067836,0.391917}%
\pgfsetfillcolor{currentfill}%
\pgfsetfillopacity{0.700000}%
\pgfsetlinewidth{0.000000pt}%
\definecolor{currentstroke}{rgb}{0.000000,0.000000,0.000000}%
\pgfsetstrokecolor{currentstroke}%
\pgfsetdash{}{0pt}%
\pgfpathmoveto{\pgfqpoint{3.192030in}{2.931902in}}%
\pgfpathlineto{\pgfqpoint{3.205004in}{2.925624in}}%
\pgfpathlineto{\pgfqpoint{3.217982in}{2.919398in}}%
\pgfpathlineto{\pgfqpoint{3.230963in}{2.913222in}}%
\pgfpathlineto{\pgfqpoint{3.243947in}{2.907097in}}%
\pgfpathlineto{\pgfqpoint{3.236079in}{2.899106in}}%
\pgfpathlineto{\pgfqpoint{3.228204in}{2.891172in}}%
\pgfpathlineto{\pgfqpoint{3.220323in}{2.883294in}}%
\pgfpathlineto{\pgfqpoint{3.212435in}{2.875472in}}%
\pgfpathlineto{\pgfqpoint{3.199439in}{2.881576in}}%
\pgfpathlineto{\pgfqpoint{3.186446in}{2.887730in}}%
\pgfpathlineto{\pgfqpoint{3.173456in}{2.893935in}}%
\pgfpathlineto{\pgfqpoint{3.160470in}{2.900192in}}%
\pgfpathlineto{\pgfqpoint{3.168370in}{2.908031in}}%
\pgfpathlineto{\pgfqpoint{3.176263in}{2.915928in}}%
\pgfpathlineto{\pgfqpoint{3.184150in}{2.923885in}}%
\pgfpathlineto{\pgfqpoint{3.192030in}{2.931902in}}%
\pgfpathclose%
\pgfusepath{fill}%
\end{pgfscope}%
\begin{pgfscope}%
\pgfpathrectangle{\pgfqpoint{1.150000in}{0.150000in}}{\pgfqpoint{5.700000in}{5.700000in}}%
\pgfusepath{clip}%
\pgfsetbuttcap%
\pgfsetroundjoin%
\definecolor{currentfill}{rgb}{0.278791,0.062145,0.386592}%
\pgfsetfillcolor{currentfill}%
\pgfsetfillopacity{0.700000}%
\pgfsetlinewidth{0.000000pt}%
\definecolor{currentstroke}{rgb}{0.000000,0.000000,0.000000}%
\pgfsetstrokecolor{currentstroke}%
\pgfsetdash{}{0pt}%
\pgfpathmoveto{\pgfqpoint{3.327285in}{2.915515in}}%
\pgfpathlineto{\pgfqpoint{3.340277in}{2.909599in}}%
\pgfpathlineto{\pgfqpoint{3.353272in}{2.903731in}}%
\pgfpathlineto{\pgfqpoint{3.366272in}{2.897910in}}%
\pgfpathlineto{\pgfqpoint{3.379275in}{2.892136in}}%
\pgfpathlineto{\pgfqpoint{3.371455in}{2.883981in}}%
\pgfpathlineto{\pgfqpoint{3.363629in}{2.875882in}}%
\pgfpathlineto{\pgfqpoint{3.355796in}{2.867838in}}%
\pgfpathlineto{\pgfqpoint{3.347958in}{2.859847in}}%
\pgfpathlineto{\pgfqpoint{3.334943in}{2.865587in}}%
\pgfpathlineto{\pgfqpoint{3.321932in}{2.871374in}}%
\pgfpathlineto{\pgfqpoint{3.308925in}{2.877208in}}%
\pgfpathlineto{\pgfqpoint{3.295922in}{2.883089in}}%
\pgfpathlineto{\pgfqpoint{3.303772in}{2.891109in}}%
\pgfpathlineto{\pgfqpoint{3.311616in}{2.899186in}}%
\pgfpathlineto{\pgfqpoint{3.319454in}{2.907320in}}%
\pgfpathlineto{\pgfqpoint{3.327285in}{2.915515in}}%
\pgfpathclose%
\pgfusepath{fill}%
\end{pgfscope}%
\begin{pgfscope}%
\pgfpathrectangle{\pgfqpoint{1.150000in}{0.150000in}}{\pgfqpoint{5.700000in}{5.700000in}}%
\pgfusepath{clip}%
\pgfsetbuttcap%
\pgfsetroundjoin%
\definecolor{currentfill}{rgb}{0.277018,0.050344,0.375715}%
\pgfsetfillcolor{currentfill}%
\pgfsetfillopacity{0.700000}%
\pgfsetlinewidth{0.000000pt}%
\definecolor{currentstroke}{rgb}{0.000000,0.000000,0.000000}%
\pgfsetstrokecolor{currentstroke}%
\pgfsetdash{}{0pt}%
\pgfpathmoveto{\pgfqpoint{3.951070in}{2.896012in}}%
\pgfpathlineto{\pgfqpoint{3.964168in}{2.891199in}}%
\pgfpathlineto{\pgfqpoint{3.977270in}{2.886420in}}%
\pgfpathlineto{\pgfqpoint{3.990378in}{2.881676in}}%
\pgfpathlineto{\pgfqpoint{4.003491in}{2.876967in}}%
\pgfpathlineto{\pgfqpoint{3.995884in}{2.868153in}}%
\pgfpathlineto{\pgfqpoint{3.988271in}{2.859417in}}%
\pgfpathlineto{\pgfqpoint{3.980654in}{2.850755in}}%
\pgfpathlineto{\pgfqpoint{3.973031in}{2.842163in}}%
\pgfpathlineto{\pgfqpoint{3.959907in}{2.846773in}}%
\pgfpathlineto{\pgfqpoint{3.946788in}{2.851417in}}%
\pgfpathlineto{\pgfqpoint{3.933675in}{2.856097in}}%
\pgfpathlineto{\pgfqpoint{3.920566in}{2.860811in}}%
\pgfpathlineto{\pgfqpoint{3.928200in}{2.869497in}}%
\pgfpathlineto{\pgfqpoint{3.935828in}{2.878257in}}%
\pgfpathlineto{\pgfqpoint{3.943452in}{2.887094in}}%
\pgfpathlineto{\pgfqpoint{3.951070in}{2.896012in}}%
\pgfpathclose%
\pgfusepath{fill}%
\end{pgfscope}%
\begin{pgfscope}%
\pgfpathrectangle{\pgfqpoint{1.150000in}{0.150000in}}{\pgfqpoint{5.700000in}{5.700000in}}%
\pgfusepath{clip}%
\pgfsetbuttcap%
\pgfsetroundjoin%
\definecolor{currentfill}{rgb}{0.280267,0.073417,0.397163}%
\pgfsetfillcolor{currentfill}%
\pgfsetfillopacity{0.700000}%
\pgfsetlinewidth{0.000000pt}%
\definecolor{currentstroke}{rgb}{0.000000,0.000000,0.000000}%
\pgfsetstrokecolor{currentstroke}%
\pgfsetdash{}{0pt}%
\pgfpathmoveto{\pgfqpoint{4.522404in}{2.937594in}}%
\pgfpathlineto{\pgfqpoint{4.535616in}{2.933114in}}%
\pgfpathlineto{\pgfqpoint{4.548834in}{2.928664in}}%
\pgfpathlineto{\pgfqpoint{4.562057in}{2.924242in}}%
\pgfpathlineto{\pgfqpoint{4.575286in}{2.919849in}}%
\pgfpathlineto{\pgfqpoint{4.567856in}{2.909984in}}%
\pgfpathlineto{\pgfqpoint{4.560423in}{2.900256in}}%
\pgfpathlineto{\pgfqpoint{4.552987in}{2.890662in}}%
\pgfpathlineto{\pgfqpoint{4.545547in}{2.881197in}}%
\pgfpathlineto{\pgfqpoint{4.532306in}{2.885426in}}%
\pgfpathlineto{\pgfqpoint{4.519071in}{2.889684in}}%
\pgfpathlineto{\pgfqpoint{4.505841in}{2.893971in}}%
\pgfpathlineto{\pgfqpoint{4.492618in}{2.898287in}}%
\pgfpathlineto{\pgfqpoint{4.500070in}{2.907912in}}%
\pgfpathlineto{\pgfqpoint{4.507518in}{2.917668in}}%
\pgfpathlineto{\pgfqpoint{4.514963in}{2.927560in}}%
\pgfpathlineto{\pgfqpoint{4.522404in}{2.937594in}}%
\pgfpathclose%
\pgfusepath{fill}%
\end{pgfscope}%
\begin{pgfscope}%
\pgfpathrectangle{\pgfqpoint{1.150000in}{0.150000in}}{\pgfqpoint{5.700000in}{5.700000in}}%
\pgfusepath{clip}%
\pgfsetbuttcap%
\pgfsetroundjoin%
\definecolor{currentfill}{rgb}{0.277018,0.050344,0.375715}%
\pgfsetfillcolor{currentfill}%
\pgfsetfillopacity{0.700000}%
\pgfsetlinewidth{0.000000pt}%
\definecolor{currentstroke}{rgb}{0.000000,0.000000,0.000000}%
\pgfsetstrokecolor{currentstroke}%
\pgfsetdash{}{0pt}%
\pgfpathmoveto{\pgfqpoint{3.462501in}{2.902524in}}%
\pgfpathlineto{\pgfqpoint{3.475513in}{2.896929in}}%
\pgfpathlineto{\pgfqpoint{3.488529in}{2.891377in}}%
\pgfpathlineto{\pgfqpoint{3.501550in}{2.885868in}}%
\pgfpathlineto{\pgfqpoint{3.514575in}{2.880403in}}%
\pgfpathlineto{\pgfqpoint{3.506801in}{2.872109in}}%
\pgfpathlineto{\pgfqpoint{3.499022in}{2.863872in}}%
\pgfpathlineto{\pgfqpoint{3.491237in}{2.855691in}}%
\pgfpathlineto{\pgfqpoint{3.483446in}{2.847562in}}%
\pgfpathlineto{\pgfqpoint{3.470410in}{2.852980in}}%
\pgfpathlineto{\pgfqpoint{3.457378in}{2.858441in}}%
\pgfpathlineto{\pgfqpoint{3.444351in}{2.863946in}}%
\pgfpathlineto{\pgfqpoint{3.431327in}{2.869494in}}%
\pgfpathlineto{\pgfqpoint{3.439130in}{2.877665in}}%
\pgfpathlineto{\pgfqpoint{3.446926in}{2.885892in}}%
\pgfpathlineto{\pgfqpoint{3.454717in}{2.894178in}}%
\pgfpathlineto{\pgfqpoint{3.462501in}{2.902524in}}%
\pgfpathclose%
\pgfusepath{fill}%
\end{pgfscope}%
\begin{pgfscope}%
\pgfpathrectangle{\pgfqpoint{1.150000in}{0.150000in}}{\pgfqpoint{5.700000in}{5.700000in}}%
\pgfusepath{clip}%
\pgfsetbuttcap%
\pgfsetroundjoin%
\definecolor{currentfill}{rgb}{0.278791,0.062145,0.386592}%
\pgfsetfillcolor{currentfill}%
\pgfsetfillopacity{0.700000}%
\pgfsetlinewidth{0.000000pt}%
\definecolor{currentstroke}{rgb}{0.000000,0.000000,0.000000}%
\pgfsetstrokecolor{currentstroke}%
\pgfsetdash{}{0pt}%
\pgfpathmoveto{\pgfqpoint{4.304381in}{2.913321in}}%
\pgfpathlineto{\pgfqpoint{4.317550in}{2.908793in}}%
\pgfpathlineto{\pgfqpoint{4.330725in}{2.904297in}}%
\pgfpathlineto{\pgfqpoint{4.343905in}{2.899832in}}%
\pgfpathlineto{\pgfqpoint{4.357090in}{2.895397in}}%
\pgfpathlineto{\pgfqpoint{4.349595in}{2.886070in}}%
\pgfpathlineto{\pgfqpoint{4.342097in}{2.876853in}}%
\pgfpathlineto{\pgfqpoint{4.334593in}{2.867740in}}%
\pgfpathlineto{\pgfqpoint{4.327086in}{2.858727in}}%
\pgfpathlineto{\pgfqpoint{4.313889in}{2.863024in}}%
\pgfpathlineto{\pgfqpoint{4.300698in}{2.867351in}}%
\pgfpathlineto{\pgfqpoint{4.287512in}{2.871710in}}%
\pgfpathlineto{\pgfqpoint{4.274331in}{2.876099in}}%
\pgfpathlineto{\pgfqpoint{4.281850in}{2.885245in}}%
\pgfpathlineto{\pgfqpoint{4.289365in}{2.894495in}}%
\pgfpathlineto{\pgfqpoint{4.296875in}{2.903852in}}%
\pgfpathlineto{\pgfqpoint{4.304381in}{2.913321in}}%
\pgfpathclose%
\pgfusepath{fill}%
\end{pgfscope}%
\begin{pgfscope}%
\pgfpathrectangle{\pgfqpoint{1.150000in}{0.150000in}}{\pgfqpoint{5.700000in}{5.700000in}}%
\pgfusepath{clip}%
\pgfsetbuttcap%
\pgfsetroundjoin%
\definecolor{currentfill}{rgb}{0.277018,0.050344,0.375715}%
\pgfsetfillcolor{currentfill}%
\pgfsetfillopacity{0.700000}%
\pgfsetlinewidth{0.000000pt}%
\definecolor{currentstroke}{rgb}{0.000000,0.000000,0.000000}%
\pgfsetstrokecolor{currentstroke}%
\pgfsetdash{}{0pt}%
\pgfpathmoveto{\pgfqpoint{3.597706in}{2.892513in}}%
\pgfpathlineto{\pgfqpoint{3.610741in}{2.887198in}}%
\pgfpathlineto{\pgfqpoint{3.623781in}{2.881924in}}%
\pgfpathlineto{\pgfqpoint{3.636826in}{2.876690in}}%
\pgfpathlineto{\pgfqpoint{3.649875in}{2.871497in}}%
\pgfpathlineto{\pgfqpoint{3.642147in}{2.863084in}}%
\pgfpathlineto{\pgfqpoint{3.634414in}{2.854730in}}%
\pgfpathlineto{\pgfqpoint{3.626675in}{2.846433in}}%
\pgfpathlineto{\pgfqpoint{3.618930in}{2.838191in}}%
\pgfpathlineto{\pgfqpoint{3.605870in}{2.843324in}}%
\pgfpathlineto{\pgfqpoint{3.592814in}{2.848498in}}%
\pgfpathlineto{\pgfqpoint{3.579763in}{2.853712in}}%
\pgfpathlineto{\pgfqpoint{3.566717in}{2.858967in}}%
\pgfpathlineto{\pgfqpoint{3.574473in}{2.867264in}}%
\pgfpathlineto{\pgfqpoint{3.582223in}{2.875620in}}%
\pgfpathlineto{\pgfqpoint{3.589967in}{2.884035in}}%
\pgfpathlineto{\pgfqpoint{3.597706in}{2.892513in}}%
\pgfpathclose%
\pgfusepath{fill}%
\end{pgfscope}%
\begin{pgfscope}%
\pgfpathrectangle{\pgfqpoint{1.150000in}{0.150000in}}{\pgfqpoint{5.700000in}{5.700000in}}%
\pgfusepath{clip}%
\pgfsetbuttcap%
\pgfsetroundjoin%
\definecolor{currentfill}{rgb}{0.277018,0.050344,0.375715}%
\pgfsetfillcolor{currentfill}%
\pgfsetfillopacity{0.700000}%
\pgfsetlinewidth{0.000000pt}%
\definecolor{currentstroke}{rgb}{0.000000,0.000000,0.000000}%
\pgfsetstrokecolor{currentstroke}%
\pgfsetdash{}{0pt}%
\pgfpathmoveto{\pgfqpoint{4.086328in}{2.894126in}}%
\pgfpathlineto{\pgfqpoint{4.099456in}{2.889475in}}%
\pgfpathlineto{\pgfqpoint{4.112588in}{2.884857in}}%
\pgfpathlineto{\pgfqpoint{4.125727in}{2.880272in}}%
\pgfpathlineto{\pgfqpoint{4.138870in}{2.875720in}}%
\pgfpathlineto{\pgfqpoint{4.131305in}{2.866794in}}%
\pgfpathlineto{\pgfqpoint{4.123735in}{2.857954in}}%
\pgfpathlineto{\pgfqpoint{4.116160in}{2.849196in}}%
\pgfpathlineto{\pgfqpoint{4.108581in}{2.840515in}}%
\pgfpathlineto{\pgfqpoint{4.095426in}{2.844955in}}%
\pgfpathlineto{\pgfqpoint{4.082277in}{2.849428in}}%
\pgfpathlineto{\pgfqpoint{4.069133in}{2.853934in}}%
\pgfpathlineto{\pgfqpoint{4.055994in}{2.858473in}}%
\pgfpathlineto{\pgfqpoint{4.063585in}{2.867261in}}%
\pgfpathlineto{\pgfqpoint{4.071171in}{2.876130in}}%
\pgfpathlineto{\pgfqpoint{4.078752in}{2.885084in}}%
\pgfpathlineto{\pgfqpoint{4.086328in}{2.894126in}}%
\pgfpathclose%
\pgfusepath{fill}%
\end{pgfscope}%
\begin{pgfscope}%
\pgfpathrectangle{\pgfqpoint{1.150000in}{0.150000in}}{\pgfqpoint{5.700000in}{5.700000in}}%
\pgfusepath{clip}%
\pgfsetbuttcap%
\pgfsetroundjoin%
\definecolor{currentfill}{rgb}{0.276022,0.044167,0.370164}%
\pgfsetfillcolor{currentfill}%
\pgfsetfillopacity{0.700000}%
\pgfsetlinewidth{0.000000pt}%
\definecolor{currentstroke}{rgb}{0.000000,0.000000,0.000000}%
\pgfsetstrokecolor{currentstroke}%
\pgfsetdash{}{0pt}%
\pgfpathmoveto{\pgfqpoint{3.732926in}{2.885118in}}%
\pgfpathlineto{\pgfqpoint{3.745987in}{2.880048in}}%
\pgfpathlineto{\pgfqpoint{3.759053in}{2.875017in}}%
\pgfpathlineto{\pgfqpoint{3.772123in}{2.870023in}}%
\pgfpathlineto{\pgfqpoint{3.785199in}{2.865067in}}%
\pgfpathlineto{\pgfqpoint{3.777516in}{2.856549in}}%
\pgfpathlineto{\pgfqpoint{3.769827in}{2.848094in}}%
\pgfpathlineto{\pgfqpoint{3.762133in}{2.839698in}}%
\pgfpathlineto{\pgfqpoint{3.754434in}{2.831361in}}%
\pgfpathlineto{\pgfqpoint{3.741347in}{2.836243in}}%
\pgfpathlineto{\pgfqpoint{3.728265in}{2.841164in}}%
\pgfpathlineto{\pgfqpoint{3.715189in}{2.846122in}}%
\pgfpathlineto{\pgfqpoint{3.702116in}{2.851119in}}%
\pgfpathlineto{\pgfqpoint{3.709827in}{2.859525in}}%
\pgfpathlineto{\pgfqpoint{3.717532in}{2.867992in}}%
\pgfpathlineto{\pgfqpoint{3.725232in}{2.876522in}}%
\pgfpathlineto{\pgfqpoint{3.732926in}{2.885118in}}%
\pgfpathclose%
\pgfusepath{fill}%
\end{pgfscope}%
\begin{pgfscope}%
\pgfpathrectangle{\pgfqpoint{1.150000in}{0.150000in}}{\pgfqpoint{5.700000in}{5.700000in}}%
\pgfusepath{clip}%
\pgfsetbuttcap%
\pgfsetroundjoin%
\definecolor{currentfill}{rgb}{0.279566,0.067836,0.391917}%
\pgfsetfillcolor{currentfill}%
\pgfsetfillopacity{0.700000}%
\pgfsetlinewidth{0.000000pt}%
\definecolor{currentstroke}{rgb}{0.000000,0.000000,0.000000}%
\pgfsetstrokecolor{currentstroke}%
\pgfsetdash{}{0pt}%
\pgfpathmoveto{\pgfqpoint{4.439779in}{2.915847in}}%
\pgfpathlineto{\pgfqpoint{4.452980in}{2.911413in}}%
\pgfpathlineto{\pgfqpoint{4.466187in}{2.907008in}}%
\pgfpathlineto{\pgfqpoint{4.479400in}{2.902633in}}%
\pgfpathlineto{\pgfqpoint{4.492618in}{2.898287in}}%
\pgfpathlineto{\pgfqpoint{4.485162in}{2.888789in}}%
\pgfpathlineto{\pgfqpoint{4.477703in}{2.879413in}}%
\pgfpathlineto{\pgfqpoint{4.470240in}{2.870152in}}%
\pgfpathlineto{\pgfqpoint{4.462773in}{2.861004in}}%
\pgfpathlineto{\pgfqpoint{4.449543in}{2.865199in}}%
\pgfpathlineto{\pgfqpoint{4.436319in}{2.869423in}}%
\pgfpathlineto{\pgfqpoint{4.423100in}{2.873677in}}%
\pgfpathlineto{\pgfqpoint{4.409887in}{2.877961in}}%
\pgfpathlineto{\pgfqpoint{4.417366in}{2.887255in}}%
\pgfpathlineto{\pgfqpoint{4.424841in}{2.896665in}}%
\pgfpathlineto{\pgfqpoint{4.432312in}{2.906194in}}%
\pgfpathlineto{\pgfqpoint{4.439779in}{2.915847in}}%
\pgfpathclose%
\pgfusepath{fill}%
\end{pgfscope}%
\begin{pgfscope}%
\pgfpathrectangle{\pgfqpoint{1.150000in}{0.150000in}}{\pgfqpoint{5.700000in}{5.700000in}}%
\pgfusepath{clip}%
\pgfsetbuttcap%
\pgfsetroundjoin%
\definecolor{currentfill}{rgb}{0.276022,0.044167,0.370164}%
\pgfsetfillcolor{currentfill}%
\pgfsetfillopacity{0.700000}%
\pgfsetlinewidth{0.000000pt}%
\definecolor{currentstroke}{rgb}{0.000000,0.000000,0.000000}%
\pgfsetstrokecolor{currentstroke}%
\pgfsetdash{}{0pt}%
\pgfpathmoveto{\pgfqpoint{3.868181in}{2.880026in}}%
\pgfpathlineto{\pgfqpoint{3.881270in}{2.875168in}}%
\pgfpathlineto{\pgfqpoint{3.894364in}{2.870347in}}%
\pgfpathlineto{\pgfqpoint{3.907462in}{2.865561in}}%
\pgfpathlineto{\pgfqpoint{3.920566in}{2.860811in}}%
\pgfpathlineto{\pgfqpoint{3.912927in}{2.852196in}}%
\pgfpathlineto{\pgfqpoint{3.905282in}{2.843649in}}%
\pgfpathlineto{\pgfqpoint{3.897633in}{2.835167in}}%
\pgfpathlineto{\pgfqpoint{3.889978in}{2.826746in}}%
\pgfpathlineto{\pgfqpoint{3.876863in}{2.831410in}}%
\pgfpathlineto{\pgfqpoint{3.863753in}{2.836109in}}%
\pgfpathlineto{\pgfqpoint{3.850648in}{2.840844in}}%
\pgfpathlineto{\pgfqpoint{3.837548in}{2.845615in}}%
\pgfpathlineto{\pgfqpoint{3.845215in}{2.854117in}}%
\pgfpathlineto{\pgfqpoint{3.852876in}{2.862684in}}%
\pgfpathlineto{\pgfqpoint{3.860531in}{2.871319in}}%
\pgfpathlineto{\pgfqpoint{3.868181in}{2.880026in}}%
\pgfpathclose%
\pgfusepath{fill}%
\end{pgfscope}%
\begin{pgfscope}%
\pgfpathrectangle{\pgfqpoint{1.150000in}{0.150000in}}{\pgfqpoint{5.700000in}{5.700000in}}%
\pgfusepath{clip}%
\pgfsetbuttcap%
\pgfsetroundjoin%
\definecolor{currentfill}{rgb}{0.277941,0.056324,0.381191}%
\pgfsetfillcolor{currentfill}%
\pgfsetfillopacity{0.700000}%
\pgfsetlinewidth{0.000000pt}%
\definecolor{currentstroke}{rgb}{0.000000,0.000000,0.000000}%
\pgfsetstrokecolor{currentstroke}%
\pgfsetdash{}{0pt}%
\pgfpathmoveto{\pgfqpoint{4.221664in}{2.893972in}}%
\pgfpathlineto{\pgfqpoint{4.234823in}{2.889456in}}%
\pgfpathlineto{\pgfqpoint{4.247987in}{2.884972in}}%
\pgfpathlineto{\pgfqpoint{4.261156in}{2.880520in}}%
\pgfpathlineto{\pgfqpoint{4.274331in}{2.876099in}}%
\pgfpathlineto{\pgfqpoint{4.266808in}{2.867053in}}%
\pgfpathlineto{\pgfqpoint{4.259280in}{2.858101in}}%
\pgfpathlineto{\pgfqpoint{4.251747in}{2.849239in}}%
\pgfpathlineto{\pgfqpoint{4.244210in}{2.840465in}}%
\pgfpathlineto{\pgfqpoint{4.231023in}{2.844761in}}%
\pgfpathlineto{\pgfqpoint{4.217842in}{2.849088in}}%
\pgfpathlineto{\pgfqpoint{4.204667in}{2.853446in}}%
\pgfpathlineto{\pgfqpoint{4.191496in}{2.857837in}}%
\pgfpathlineto{\pgfqpoint{4.199045in}{2.866732in}}%
\pgfpathlineto{\pgfqpoint{4.206590in}{2.875716in}}%
\pgfpathlineto{\pgfqpoint{4.214129in}{2.884795in}}%
\pgfpathlineto{\pgfqpoint{4.221664in}{2.893972in}}%
\pgfpathclose%
\pgfusepath{fill}%
\end{pgfscope}%
\begin{pgfscope}%
\pgfpathrectangle{\pgfqpoint{1.150000in}{0.150000in}}{\pgfqpoint{5.700000in}{5.700000in}}%
\pgfusepath{clip}%
\pgfsetbuttcap%
\pgfsetroundjoin%
\definecolor{currentfill}{rgb}{0.278791,0.062145,0.386592}%
\pgfsetfillcolor{currentfill}%
\pgfsetfillopacity{0.700000}%
\pgfsetlinewidth{0.000000pt}%
\definecolor{currentstroke}{rgb}{0.000000,0.000000,0.000000}%
\pgfsetstrokecolor{currentstroke}%
\pgfsetdash{}{0pt}%
\pgfpathmoveto{\pgfqpoint{3.243947in}{2.907097in}}%
\pgfpathlineto{\pgfqpoint{3.256935in}{2.901021in}}%
\pgfpathlineto{\pgfqpoint{3.269927in}{2.894995in}}%
\pgfpathlineto{\pgfqpoint{3.282923in}{2.889018in}}%
\pgfpathlineto{\pgfqpoint{3.295922in}{2.883089in}}%
\pgfpathlineto{\pgfqpoint{3.288066in}{2.875124in}}%
\pgfpathlineto{\pgfqpoint{3.280203in}{2.867213in}}%
\pgfpathlineto{\pgfqpoint{3.272334in}{2.859355in}}%
\pgfpathlineto{\pgfqpoint{3.264458in}{2.851549in}}%
\pgfpathlineto{\pgfqpoint{3.251447in}{2.857456in}}%
\pgfpathlineto{\pgfqpoint{3.238439in}{2.863413in}}%
\pgfpathlineto{\pgfqpoint{3.225435in}{2.869417in}}%
\pgfpathlineto{\pgfqpoint{3.212435in}{2.875472in}}%
\pgfpathlineto{\pgfqpoint{3.220323in}{2.883294in}}%
\pgfpathlineto{\pgfqpoint{3.228204in}{2.891172in}}%
\pgfpathlineto{\pgfqpoint{3.236079in}{2.899106in}}%
\pgfpathlineto{\pgfqpoint{3.243947in}{2.907097in}}%
\pgfpathclose%
\pgfusepath{fill}%
\end{pgfscope}%
\begin{pgfscope}%
\pgfpathrectangle{\pgfqpoint{1.150000in}{0.150000in}}{\pgfqpoint{5.700000in}{5.700000in}}%
\pgfusepath{clip}%
\pgfsetbuttcap%
\pgfsetroundjoin%
\definecolor{currentfill}{rgb}{0.277941,0.056324,0.381191}%
\pgfsetfillcolor{currentfill}%
\pgfsetfillopacity{0.700000}%
\pgfsetlinewidth{0.000000pt}%
\definecolor{currentstroke}{rgb}{0.000000,0.000000,0.000000}%
\pgfsetstrokecolor{currentstroke}%
\pgfsetdash{}{0pt}%
\pgfpathmoveto{\pgfqpoint{3.379275in}{2.892136in}}%
\pgfpathlineto{\pgfqpoint{3.392282in}{2.886408in}}%
\pgfpathlineto{\pgfqpoint{3.405293in}{2.880725in}}%
\pgfpathlineto{\pgfqpoint{3.418308in}{2.875087in}}%
\pgfpathlineto{\pgfqpoint{3.431327in}{2.869494in}}%
\pgfpathlineto{\pgfqpoint{3.423519in}{2.861378in}}%
\pgfpathlineto{\pgfqpoint{3.415704in}{2.853315in}}%
\pgfpathlineto{\pgfqpoint{3.407884in}{2.845304in}}%
\pgfpathlineto{\pgfqpoint{3.400057in}{2.837343in}}%
\pgfpathlineto{\pgfqpoint{3.387026in}{2.842901in}}%
\pgfpathlineto{\pgfqpoint{3.373999in}{2.848504in}}%
\pgfpathlineto{\pgfqpoint{3.360977in}{2.854153in}}%
\pgfpathlineto{\pgfqpoint{3.347958in}{2.859847in}}%
\pgfpathlineto{\pgfqpoint{3.355796in}{2.867838in}}%
\pgfpathlineto{\pgfqpoint{3.363629in}{2.875882in}}%
\pgfpathlineto{\pgfqpoint{3.371455in}{2.883981in}}%
\pgfpathlineto{\pgfqpoint{3.379275in}{2.892136in}}%
\pgfpathclose%
\pgfusepath{fill}%
\end{pgfscope}%
\begin{pgfscope}%
\pgfpathrectangle{\pgfqpoint{1.150000in}{0.150000in}}{\pgfqpoint{5.700000in}{5.700000in}}%
\pgfusepath{clip}%
\pgfsetbuttcap%
\pgfsetroundjoin%
\definecolor{currentfill}{rgb}{0.277018,0.050344,0.375715}%
\pgfsetfillcolor{currentfill}%
\pgfsetfillopacity{0.700000}%
\pgfsetlinewidth{0.000000pt}%
\definecolor{currentstroke}{rgb}{0.000000,0.000000,0.000000}%
\pgfsetstrokecolor{currentstroke}%
\pgfsetdash{}{0pt}%
\pgfpathmoveto{\pgfqpoint{3.514575in}{2.880403in}}%
\pgfpathlineto{\pgfqpoint{3.527604in}{2.874981in}}%
\pgfpathlineto{\pgfqpoint{3.540637in}{2.869601in}}%
\pgfpathlineto{\pgfqpoint{3.553675in}{2.864263in}}%
\pgfpathlineto{\pgfqpoint{3.566717in}{2.858967in}}%
\pgfpathlineto{\pgfqpoint{3.558955in}{2.850725in}}%
\pgfpathlineto{\pgfqpoint{3.551187in}{2.842537in}}%
\pgfpathlineto{\pgfqpoint{3.543414in}{2.834401in}}%
\pgfpathlineto{\pgfqpoint{3.535634in}{2.826315in}}%
\pgfpathlineto{\pgfqpoint{3.522581in}{2.831564in}}%
\pgfpathlineto{\pgfqpoint{3.509531in}{2.836854in}}%
\pgfpathlineto{\pgfqpoint{3.496487in}{2.842187in}}%
\pgfpathlineto{\pgfqpoint{3.483446in}{2.847562in}}%
\pgfpathlineto{\pgfqpoint{3.491237in}{2.855691in}}%
\pgfpathlineto{\pgfqpoint{3.499022in}{2.863872in}}%
\pgfpathlineto{\pgfqpoint{3.506801in}{2.872109in}}%
\pgfpathlineto{\pgfqpoint{3.514575in}{2.880403in}}%
\pgfpathclose%
\pgfusepath{fill}%
\end{pgfscope}%
\begin{pgfscope}%
\pgfpathrectangle{\pgfqpoint{1.150000in}{0.150000in}}{\pgfqpoint{5.700000in}{5.700000in}}%
\pgfusepath{clip}%
\pgfsetbuttcap%
\pgfsetroundjoin%
\definecolor{currentfill}{rgb}{0.276022,0.044167,0.370164}%
\pgfsetfillcolor{currentfill}%
\pgfsetfillopacity{0.700000}%
\pgfsetlinewidth{0.000000pt}%
\definecolor{currentstroke}{rgb}{0.000000,0.000000,0.000000}%
\pgfsetstrokecolor{currentstroke}%
\pgfsetdash{}{0pt}%
\pgfpathmoveto{\pgfqpoint{4.003491in}{2.876967in}}%
\pgfpathlineto{\pgfqpoint{4.016609in}{2.872293in}}%
\pgfpathlineto{\pgfqpoint{4.029732in}{2.867652in}}%
\pgfpathlineto{\pgfqpoint{4.042860in}{2.863046in}}%
\pgfpathlineto{\pgfqpoint{4.055994in}{2.858473in}}%
\pgfpathlineto{\pgfqpoint{4.048398in}{2.849763in}}%
\pgfpathlineto{\pgfqpoint{4.040797in}{2.841128in}}%
\pgfpathlineto{\pgfqpoint{4.033191in}{2.832563in}}%
\pgfpathlineto{\pgfqpoint{4.025579in}{2.824066in}}%
\pgfpathlineto{\pgfqpoint{4.012434in}{2.828539in}}%
\pgfpathlineto{\pgfqpoint{3.999295in}{2.833046in}}%
\pgfpathlineto{\pgfqpoint{3.986160in}{2.837588in}}%
\pgfpathlineto{\pgfqpoint{3.973031in}{2.842163in}}%
\pgfpathlineto{\pgfqpoint{3.980654in}{2.850755in}}%
\pgfpathlineto{\pgfqpoint{3.988271in}{2.859417in}}%
\pgfpathlineto{\pgfqpoint{3.995884in}{2.868153in}}%
\pgfpathlineto{\pgfqpoint{4.003491in}{2.876967in}}%
\pgfpathclose%
\pgfusepath{fill}%
\end{pgfscope}%
\begin{pgfscope}%
\pgfpathrectangle{\pgfqpoint{1.150000in}{0.150000in}}{\pgfqpoint{5.700000in}{5.700000in}}%
\pgfusepath{clip}%
\pgfsetbuttcap%
\pgfsetroundjoin%
\definecolor{currentfill}{rgb}{0.280267,0.073417,0.397163}%
\pgfsetfillcolor{currentfill}%
\pgfsetfillopacity{0.700000}%
\pgfsetlinewidth{0.000000pt}%
\definecolor{currentstroke}{rgb}{0.000000,0.000000,0.000000}%
\pgfsetstrokecolor{currentstroke}%
\pgfsetdash{}{0pt}%
\pgfpathmoveto{\pgfqpoint{4.575286in}{2.919849in}}%
\pgfpathlineto{\pgfqpoint{4.588520in}{2.915485in}}%
\pgfpathlineto{\pgfqpoint{4.601760in}{2.911150in}}%
\pgfpathlineto{\pgfqpoint{4.615006in}{2.906843in}}%
\pgfpathlineto{\pgfqpoint{4.628257in}{2.902565in}}%
\pgfpathlineto{\pgfqpoint{4.620840in}{2.892867in}}%
\pgfpathlineto{\pgfqpoint{4.613419in}{2.883305in}}%
\pgfpathlineto{\pgfqpoint{4.605995in}{2.873873in}}%
\pgfpathlineto{\pgfqpoint{4.598568in}{2.864567in}}%
\pgfpathlineto{\pgfqpoint{4.585304in}{2.868681in}}%
\pgfpathlineto{\pgfqpoint{4.572046in}{2.872824in}}%
\pgfpathlineto{\pgfqpoint{4.558793in}{2.876996in}}%
\pgfpathlineto{\pgfqpoint{4.545547in}{2.881197in}}%
\pgfpathlineto{\pgfqpoint{4.552987in}{2.890662in}}%
\pgfpathlineto{\pgfqpoint{4.560423in}{2.900256in}}%
\pgfpathlineto{\pgfqpoint{4.567856in}{2.909984in}}%
\pgfpathlineto{\pgfqpoint{4.575286in}{2.919849in}}%
\pgfpathclose%
\pgfusepath{fill}%
\end{pgfscope}%
\begin{pgfscope}%
\pgfpathrectangle{\pgfqpoint{1.150000in}{0.150000in}}{\pgfqpoint{5.700000in}{5.700000in}}%
\pgfusepath{clip}%
\pgfsetbuttcap%
\pgfsetroundjoin%
\definecolor{currentfill}{rgb}{0.277941,0.056324,0.381191}%
\pgfsetfillcolor{currentfill}%
\pgfsetfillopacity{0.700000}%
\pgfsetlinewidth{0.000000pt}%
\definecolor{currentstroke}{rgb}{0.000000,0.000000,0.000000}%
\pgfsetstrokecolor{currentstroke}%
\pgfsetdash{}{0pt}%
\pgfpathmoveto{\pgfqpoint{4.357090in}{2.895397in}}%
\pgfpathlineto{\pgfqpoint{4.370281in}{2.890993in}}%
\pgfpathlineto{\pgfqpoint{4.383477in}{2.886619in}}%
\pgfpathlineto{\pgfqpoint{4.396679in}{2.882275in}}%
\pgfpathlineto{\pgfqpoint{4.409887in}{2.877961in}}%
\pgfpathlineto{\pgfqpoint{4.402404in}{2.868777in}}%
\pgfpathlineto{\pgfqpoint{4.394917in}{2.859699in}}%
\pgfpathlineto{\pgfqpoint{4.387425in}{2.850723in}}%
\pgfpathlineto{\pgfqpoint{4.379930in}{2.841843in}}%
\pgfpathlineto{\pgfqpoint{4.366710in}{2.846019in}}%
\pgfpathlineto{\pgfqpoint{4.353497in}{2.850224in}}%
\pgfpathlineto{\pgfqpoint{4.340288in}{2.854460in}}%
\pgfpathlineto{\pgfqpoint{4.327086in}{2.858727in}}%
\pgfpathlineto{\pgfqpoint{4.334593in}{2.867740in}}%
\pgfpathlineto{\pgfqpoint{4.342097in}{2.876853in}}%
\pgfpathlineto{\pgfqpoint{4.349595in}{2.886070in}}%
\pgfpathlineto{\pgfqpoint{4.357090in}{2.895397in}}%
\pgfpathclose%
\pgfusepath{fill}%
\end{pgfscope}%
\begin{pgfscope}%
\pgfpathrectangle{\pgfqpoint{1.150000in}{0.150000in}}{\pgfqpoint{5.700000in}{5.700000in}}%
\pgfusepath{clip}%
\pgfsetbuttcap%
\pgfsetroundjoin%
\definecolor{currentfill}{rgb}{0.276022,0.044167,0.370164}%
\pgfsetfillcolor{currentfill}%
\pgfsetfillopacity{0.700000}%
\pgfsetlinewidth{0.000000pt}%
\definecolor{currentstroke}{rgb}{0.000000,0.000000,0.000000}%
\pgfsetstrokecolor{currentstroke}%
\pgfsetdash{}{0pt}%
\pgfpathmoveto{\pgfqpoint{3.649875in}{2.871497in}}%
\pgfpathlineto{\pgfqpoint{3.662928in}{2.866343in}}%
\pgfpathlineto{\pgfqpoint{3.675986in}{2.861229in}}%
\pgfpathlineto{\pgfqpoint{3.689049in}{2.856155in}}%
\pgfpathlineto{\pgfqpoint{3.702116in}{2.851119in}}%
\pgfpathlineto{\pgfqpoint{3.694400in}{2.842772in}}%
\pgfpathlineto{\pgfqpoint{3.686678in}{2.834480in}}%
\pgfpathlineto{\pgfqpoint{3.678950in}{2.826242in}}%
\pgfpathlineto{\pgfqpoint{3.671217in}{2.818055in}}%
\pgfpathlineto{\pgfqpoint{3.658138in}{2.823030in}}%
\pgfpathlineto{\pgfqpoint{3.645064in}{2.828045in}}%
\pgfpathlineto{\pgfqpoint{3.631995in}{2.833098in}}%
\pgfpathlineto{\pgfqpoint{3.618930in}{2.838191in}}%
\pgfpathlineto{\pgfqpoint{3.626675in}{2.846433in}}%
\pgfpathlineto{\pgfqpoint{3.634414in}{2.854730in}}%
\pgfpathlineto{\pgfqpoint{3.642147in}{2.863084in}}%
\pgfpathlineto{\pgfqpoint{3.649875in}{2.871497in}}%
\pgfpathclose%
\pgfusepath{fill}%
\end{pgfscope}%
\begin{pgfscope}%
\pgfpathrectangle{\pgfqpoint{1.150000in}{0.150000in}}{\pgfqpoint{5.700000in}{5.700000in}}%
\pgfusepath{clip}%
\pgfsetbuttcap%
\pgfsetroundjoin%
\definecolor{currentfill}{rgb}{0.277018,0.050344,0.375715}%
\pgfsetfillcolor{currentfill}%
\pgfsetfillopacity{0.700000}%
\pgfsetlinewidth{0.000000pt}%
\definecolor{currentstroke}{rgb}{0.000000,0.000000,0.000000}%
\pgfsetstrokecolor{currentstroke}%
\pgfsetdash{}{0pt}%
\pgfpathmoveto{\pgfqpoint{4.138870in}{2.875720in}}%
\pgfpathlineto{\pgfqpoint{4.152019in}{2.871201in}}%
\pgfpathlineto{\pgfqpoint{4.165172in}{2.866714in}}%
\pgfpathlineto{\pgfqpoint{4.178332in}{2.862259in}}%
\pgfpathlineto{\pgfqpoint{4.191496in}{2.857837in}}%
\pgfpathlineto{\pgfqpoint{4.183943in}{2.849028in}}%
\pgfpathlineto{\pgfqpoint{4.176384in}{2.840302in}}%
\pgfpathlineto{\pgfqpoint{4.168821in}{2.831654in}}%
\pgfpathlineto{\pgfqpoint{4.161253in}{2.823081in}}%
\pgfpathlineto{\pgfqpoint{4.148077in}{2.827392in}}%
\pgfpathlineto{\pgfqpoint{4.134906in}{2.831734in}}%
\pgfpathlineto{\pgfqpoint{4.121741in}{2.836108in}}%
\pgfpathlineto{\pgfqpoint{4.108581in}{2.840515in}}%
\pgfpathlineto{\pgfqpoint{4.116160in}{2.849196in}}%
\pgfpathlineto{\pgfqpoint{4.123735in}{2.857954in}}%
\pgfpathlineto{\pgfqpoint{4.131305in}{2.866794in}}%
\pgfpathlineto{\pgfqpoint{4.138870in}{2.875720in}}%
\pgfpathclose%
\pgfusepath{fill}%
\end{pgfscope}%
\begin{pgfscope}%
\pgfpathrectangle{\pgfqpoint{1.150000in}{0.150000in}}{\pgfqpoint{5.700000in}{5.700000in}}%
\pgfusepath{clip}%
\pgfsetbuttcap%
\pgfsetroundjoin%
\definecolor{currentfill}{rgb}{0.276022,0.044167,0.370164}%
\pgfsetfillcolor{currentfill}%
\pgfsetfillopacity{0.700000}%
\pgfsetlinewidth{0.000000pt}%
\definecolor{currentstroke}{rgb}{0.000000,0.000000,0.000000}%
\pgfsetstrokecolor{currentstroke}%
\pgfsetdash{}{0pt}%
\pgfpathmoveto{\pgfqpoint{3.785199in}{2.865067in}}%
\pgfpathlineto{\pgfqpoint{3.798279in}{2.860148in}}%
\pgfpathlineto{\pgfqpoint{3.811364in}{2.855267in}}%
\pgfpathlineto{\pgfqpoint{3.824454in}{2.850423in}}%
\pgfpathlineto{\pgfqpoint{3.837548in}{2.845615in}}%
\pgfpathlineto{\pgfqpoint{3.829877in}{2.837175in}}%
\pgfpathlineto{\pgfqpoint{3.822199in}{2.828795in}}%
\pgfpathlineto{\pgfqpoint{3.814517in}{2.820472in}}%
\pgfpathlineto{\pgfqpoint{3.806828in}{2.812203in}}%
\pgfpathlineto{\pgfqpoint{3.793722in}{2.816937in}}%
\pgfpathlineto{\pgfqpoint{3.780621in}{2.821708in}}%
\pgfpathlineto{\pgfqpoint{3.767525in}{2.826516in}}%
\pgfpathlineto{\pgfqpoint{3.754434in}{2.831361in}}%
\pgfpathlineto{\pgfqpoint{3.762133in}{2.839698in}}%
\pgfpathlineto{\pgfqpoint{3.769827in}{2.848094in}}%
\pgfpathlineto{\pgfqpoint{3.777516in}{2.856549in}}%
\pgfpathlineto{\pgfqpoint{3.785199in}{2.865067in}}%
\pgfpathclose%
\pgfusepath{fill}%
\end{pgfscope}%
\begin{pgfscope}%
\pgfpathrectangle{\pgfqpoint{1.150000in}{0.150000in}}{\pgfqpoint{5.700000in}{5.700000in}}%
\pgfusepath{clip}%
\pgfsetbuttcap%
\pgfsetroundjoin%
\definecolor{currentfill}{rgb}{0.278791,0.062145,0.386592}%
\pgfsetfillcolor{currentfill}%
\pgfsetfillopacity{0.700000}%
\pgfsetlinewidth{0.000000pt}%
\definecolor{currentstroke}{rgb}{0.000000,0.000000,0.000000}%
\pgfsetstrokecolor{currentstroke}%
\pgfsetdash{}{0pt}%
\pgfpathmoveto{\pgfqpoint{4.492618in}{2.898287in}}%
\pgfpathlineto{\pgfqpoint{4.505841in}{2.893971in}}%
\pgfpathlineto{\pgfqpoint{4.519071in}{2.889684in}}%
\pgfpathlineto{\pgfqpoint{4.532306in}{2.885426in}}%
\pgfpathlineto{\pgfqpoint{4.545547in}{2.881197in}}%
\pgfpathlineto{\pgfqpoint{4.538103in}{2.871854in}}%
\pgfpathlineto{\pgfqpoint{4.530656in}{2.862630in}}%
\pgfpathlineto{\pgfqpoint{4.523205in}{2.853519in}}%
\pgfpathlineto{\pgfqpoint{4.515751in}{2.844517in}}%
\pgfpathlineto{\pgfqpoint{4.502498in}{2.848595in}}%
\pgfpathlineto{\pgfqpoint{4.489250in}{2.852702in}}%
\pgfpathlineto{\pgfqpoint{4.476009in}{2.856839in}}%
\pgfpathlineto{\pgfqpoint{4.462773in}{2.861004in}}%
\pgfpathlineto{\pgfqpoint{4.470240in}{2.870152in}}%
\pgfpathlineto{\pgfqpoint{4.477703in}{2.879413in}}%
\pgfpathlineto{\pgfqpoint{4.485162in}{2.888789in}}%
\pgfpathlineto{\pgfqpoint{4.492618in}{2.898287in}}%
\pgfpathclose%
\pgfusepath{fill}%
\end{pgfscope}%
\begin{pgfscope}%
\pgfpathrectangle{\pgfqpoint{1.150000in}{0.150000in}}{\pgfqpoint{5.700000in}{5.700000in}}%
\pgfusepath{clip}%
\pgfsetbuttcap%
\pgfsetroundjoin%
\definecolor{currentfill}{rgb}{0.276022,0.044167,0.370164}%
\pgfsetfillcolor{currentfill}%
\pgfsetfillopacity{0.700000}%
\pgfsetlinewidth{0.000000pt}%
\definecolor{currentstroke}{rgb}{0.000000,0.000000,0.000000}%
\pgfsetstrokecolor{currentstroke}%
\pgfsetdash{}{0pt}%
\pgfpathmoveto{\pgfqpoint{3.920566in}{2.860811in}}%
\pgfpathlineto{\pgfqpoint{3.933675in}{2.856097in}}%
\pgfpathlineto{\pgfqpoint{3.946788in}{2.851417in}}%
\pgfpathlineto{\pgfqpoint{3.959907in}{2.846773in}}%
\pgfpathlineto{\pgfqpoint{3.973031in}{2.842163in}}%
\pgfpathlineto{\pgfqpoint{3.965403in}{2.833639in}}%
\pgfpathlineto{\pgfqpoint{3.957770in}{2.825180in}}%
\pgfpathlineto{\pgfqpoint{3.950132in}{2.816782in}}%
\pgfpathlineto{\pgfqpoint{3.942488in}{2.808444in}}%
\pgfpathlineto{\pgfqpoint{3.929353in}{2.812967in}}%
\pgfpathlineto{\pgfqpoint{3.916222in}{2.817525in}}%
\pgfpathlineto{\pgfqpoint{3.903097in}{2.822118in}}%
\pgfpathlineto{\pgfqpoint{3.889978in}{2.826746in}}%
\pgfpathlineto{\pgfqpoint{3.897633in}{2.835167in}}%
\pgfpathlineto{\pgfqpoint{3.905282in}{2.843649in}}%
\pgfpathlineto{\pgfqpoint{3.912927in}{2.852196in}}%
\pgfpathlineto{\pgfqpoint{3.920566in}{2.860811in}}%
\pgfpathclose%
\pgfusepath{fill}%
\end{pgfscope}%
\begin{pgfscope}%
\pgfpathrectangle{\pgfqpoint{1.150000in}{0.150000in}}{\pgfqpoint{5.700000in}{5.700000in}}%
\pgfusepath{clip}%
\pgfsetbuttcap%
\pgfsetroundjoin%
\definecolor{currentfill}{rgb}{0.277941,0.056324,0.381191}%
\pgfsetfillcolor{currentfill}%
\pgfsetfillopacity{0.700000}%
\pgfsetlinewidth{0.000000pt}%
\definecolor{currentstroke}{rgb}{0.000000,0.000000,0.000000}%
\pgfsetstrokecolor{currentstroke}%
\pgfsetdash{}{0pt}%
\pgfpathmoveto{\pgfqpoint{3.295922in}{2.883089in}}%
\pgfpathlineto{\pgfqpoint{3.308925in}{2.877208in}}%
\pgfpathlineto{\pgfqpoint{3.321932in}{2.871374in}}%
\pgfpathlineto{\pgfqpoint{3.334943in}{2.865587in}}%
\pgfpathlineto{\pgfqpoint{3.347958in}{2.859847in}}%
\pgfpathlineto{\pgfqpoint{3.340113in}{2.851909in}}%
\pgfpathlineto{\pgfqpoint{3.332262in}{2.844021in}}%
\pgfpathlineto{\pgfqpoint{3.324405in}{2.836182in}}%
\pgfpathlineto{\pgfqpoint{3.316542in}{2.828393in}}%
\pgfpathlineto{\pgfqpoint{3.303515in}{2.834111in}}%
\pgfpathlineto{\pgfqpoint{3.290492in}{2.839877in}}%
\pgfpathlineto{\pgfqpoint{3.277473in}{2.845689in}}%
\pgfpathlineto{\pgfqpoint{3.264458in}{2.851549in}}%
\pgfpathlineto{\pgfqpoint{3.272334in}{2.859355in}}%
\pgfpathlineto{\pgfqpoint{3.280203in}{2.867213in}}%
\pgfpathlineto{\pgfqpoint{3.288066in}{2.875124in}}%
\pgfpathlineto{\pgfqpoint{3.295922in}{2.883089in}}%
\pgfpathclose%
\pgfusepath{fill}%
\end{pgfscope}%
\begin{pgfscope}%
\pgfpathrectangle{\pgfqpoint{1.150000in}{0.150000in}}{\pgfqpoint{5.700000in}{5.700000in}}%
\pgfusepath{clip}%
\pgfsetbuttcap%
\pgfsetroundjoin%
\definecolor{currentfill}{rgb}{0.279566,0.067836,0.391917}%
\pgfsetfillcolor{currentfill}%
\pgfsetfillopacity{0.700000}%
\pgfsetlinewidth{0.000000pt}%
\definecolor{currentstroke}{rgb}{0.000000,0.000000,0.000000}%
\pgfsetstrokecolor{currentstroke}%
\pgfsetdash{}{0pt}%
\pgfpathmoveto{\pgfqpoint{3.160470in}{2.900192in}}%
\pgfpathlineto{\pgfqpoint{3.173456in}{2.893935in}}%
\pgfpathlineto{\pgfqpoint{3.186446in}{2.887730in}}%
\pgfpathlineto{\pgfqpoint{3.199439in}{2.881576in}}%
\pgfpathlineto{\pgfqpoint{3.212435in}{2.875472in}}%
\pgfpathlineto{\pgfqpoint{3.204541in}{2.867703in}}%
\pgfpathlineto{\pgfqpoint{3.196640in}{2.859987in}}%
\pgfpathlineto{\pgfqpoint{3.188733in}{2.852323in}}%
\pgfpathlineto{\pgfqpoint{3.180820in}{2.844711in}}%
\pgfpathlineto{\pgfqpoint{3.167811in}{2.850806in}}%
\pgfpathlineto{\pgfqpoint{3.154805in}{2.856953in}}%
\pgfpathlineto{\pgfqpoint{3.141803in}{2.863150in}}%
\pgfpathlineto{\pgfqpoint{3.128805in}{2.869398in}}%
\pgfpathlineto{\pgfqpoint{3.136731in}{2.877014in}}%
\pgfpathlineto{\pgfqpoint{3.144651in}{2.884684in}}%
\pgfpathlineto{\pgfqpoint{3.152564in}{2.892410in}}%
\pgfpathlineto{\pgfqpoint{3.160470in}{2.900192in}}%
\pgfpathclose%
\pgfusepath{fill}%
\end{pgfscope}%
\begin{pgfscope}%
\pgfpathrectangle{\pgfqpoint{1.150000in}{0.150000in}}{\pgfqpoint{5.700000in}{5.700000in}}%
\pgfusepath{clip}%
\pgfsetbuttcap%
\pgfsetroundjoin%
\definecolor{currentfill}{rgb}{0.277018,0.050344,0.375715}%
\pgfsetfillcolor{currentfill}%
\pgfsetfillopacity{0.700000}%
\pgfsetlinewidth{0.000000pt}%
\definecolor{currentstroke}{rgb}{0.000000,0.000000,0.000000}%
\pgfsetstrokecolor{currentstroke}%
\pgfsetdash{}{0pt}%
\pgfpathmoveto{\pgfqpoint{4.274331in}{2.876099in}}%
\pgfpathlineto{\pgfqpoint{4.287512in}{2.871710in}}%
\pgfpathlineto{\pgfqpoint{4.300698in}{2.867351in}}%
\pgfpathlineto{\pgfqpoint{4.313889in}{2.863024in}}%
\pgfpathlineto{\pgfqpoint{4.327086in}{2.858727in}}%
\pgfpathlineto{\pgfqpoint{4.319574in}{2.849810in}}%
\pgfpathlineto{\pgfqpoint{4.312058in}{2.840985in}}%
\pgfpathlineto{\pgfqpoint{4.304537in}{2.832247in}}%
\pgfpathlineto{\pgfqpoint{4.297011in}{2.823593in}}%
\pgfpathlineto{\pgfqpoint{4.283802in}{2.827765in}}%
\pgfpathlineto{\pgfqpoint{4.270599in}{2.831967in}}%
\pgfpathlineto{\pgfqpoint{4.257402in}{2.836201in}}%
\pgfpathlineto{\pgfqpoint{4.244210in}{2.840465in}}%
\pgfpathlineto{\pgfqpoint{4.251747in}{2.849239in}}%
\pgfpathlineto{\pgfqpoint{4.259280in}{2.858101in}}%
\pgfpathlineto{\pgfqpoint{4.266808in}{2.867053in}}%
\pgfpathlineto{\pgfqpoint{4.274331in}{2.876099in}}%
\pgfpathclose%
\pgfusepath{fill}%
\end{pgfscope}%
\begin{pgfscope}%
\pgfpathrectangle{\pgfqpoint{1.150000in}{0.150000in}}{\pgfqpoint{5.700000in}{5.700000in}}%
\pgfusepath{clip}%
\pgfsetbuttcap%
\pgfsetroundjoin%
\definecolor{currentfill}{rgb}{0.277018,0.050344,0.375715}%
\pgfsetfillcolor{currentfill}%
\pgfsetfillopacity{0.700000}%
\pgfsetlinewidth{0.000000pt}%
\definecolor{currentstroke}{rgb}{0.000000,0.000000,0.000000}%
\pgfsetstrokecolor{currentstroke}%
\pgfsetdash{}{0pt}%
\pgfpathmoveto{\pgfqpoint{3.431327in}{2.869494in}}%
\pgfpathlineto{\pgfqpoint{3.444351in}{2.863946in}}%
\pgfpathlineto{\pgfqpoint{3.457378in}{2.858441in}}%
\pgfpathlineto{\pgfqpoint{3.470410in}{2.852980in}}%
\pgfpathlineto{\pgfqpoint{3.483446in}{2.847562in}}%
\pgfpathlineto{\pgfqpoint{3.475649in}{2.839485in}}%
\pgfpathlineto{\pgfqpoint{3.467846in}{2.831458in}}%
\pgfpathlineto{\pgfqpoint{3.460038in}{2.823480in}}%
\pgfpathlineto{\pgfqpoint{3.452223in}{2.815549in}}%
\pgfpathlineto{\pgfqpoint{3.439175in}{2.820932in}}%
\pgfpathlineto{\pgfqpoint{3.426131in}{2.826358in}}%
\pgfpathlineto{\pgfqpoint{3.413092in}{2.831829in}}%
\pgfpathlineto{\pgfqpoint{3.400057in}{2.837343in}}%
\pgfpathlineto{\pgfqpoint{3.407884in}{2.845304in}}%
\pgfpathlineto{\pgfqpoint{3.415704in}{2.853315in}}%
\pgfpathlineto{\pgfqpoint{3.423519in}{2.861378in}}%
\pgfpathlineto{\pgfqpoint{3.431327in}{2.869494in}}%
\pgfpathclose%
\pgfusepath{fill}%
\end{pgfscope}%
\begin{pgfscope}%
\pgfpathrectangle{\pgfqpoint{1.150000in}{0.150000in}}{\pgfqpoint{5.700000in}{5.700000in}}%
\pgfusepath{clip}%
\pgfsetbuttcap%
\pgfsetroundjoin%
\definecolor{currentfill}{rgb}{0.276022,0.044167,0.370164}%
\pgfsetfillcolor{currentfill}%
\pgfsetfillopacity{0.700000}%
\pgfsetlinewidth{0.000000pt}%
\definecolor{currentstroke}{rgb}{0.000000,0.000000,0.000000}%
\pgfsetstrokecolor{currentstroke}%
\pgfsetdash{}{0pt}%
\pgfpathmoveto{\pgfqpoint{3.566717in}{2.858967in}}%
\pgfpathlineto{\pgfqpoint{3.579763in}{2.853712in}}%
\pgfpathlineto{\pgfqpoint{3.592814in}{2.848498in}}%
\pgfpathlineto{\pgfqpoint{3.605870in}{2.843324in}}%
\pgfpathlineto{\pgfqpoint{3.618930in}{2.838191in}}%
\pgfpathlineto{\pgfqpoint{3.611180in}{2.830002in}}%
\pgfpathlineto{\pgfqpoint{3.603423in}{2.821863in}}%
\pgfpathlineto{\pgfqpoint{3.595661in}{2.813773in}}%
\pgfpathlineto{\pgfqpoint{3.587893in}{2.805730in}}%
\pgfpathlineto{\pgfqpoint{3.574822in}{2.810815in}}%
\pgfpathlineto{\pgfqpoint{3.561755in}{2.815941in}}%
\pgfpathlineto{\pgfqpoint{3.548692in}{2.821107in}}%
\pgfpathlineto{\pgfqpoint{3.535634in}{2.826315in}}%
\pgfpathlineto{\pgfqpoint{3.543414in}{2.834401in}}%
\pgfpathlineto{\pgfqpoint{3.551187in}{2.842537in}}%
\pgfpathlineto{\pgfqpoint{3.558955in}{2.850725in}}%
\pgfpathlineto{\pgfqpoint{3.566717in}{2.858967in}}%
\pgfpathclose%
\pgfusepath{fill}%
\end{pgfscope}%
\begin{pgfscope}%
\pgfpathrectangle{\pgfqpoint{1.150000in}{0.150000in}}{\pgfqpoint{5.700000in}{5.700000in}}%
\pgfusepath{clip}%
\pgfsetbuttcap%
\pgfsetroundjoin%
\definecolor{currentfill}{rgb}{0.276022,0.044167,0.370164}%
\pgfsetfillcolor{currentfill}%
\pgfsetfillopacity{0.700000}%
\pgfsetlinewidth{0.000000pt}%
\definecolor{currentstroke}{rgb}{0.000000,0.000000,0.000000}%
\pgfsetstrokecolor{currentstroke}%
\pgfsetdash{}{0pt}%
\pgfpathmoveto{\pgfqpoint{4.055994in}{2.858473in}}%
\pgfpathlineto{\pgfqpoint{4.069133in}{2.853934in}}%
\pgfpathlineto{\pgfqpoint{4.082277in}{2.849428in}}%
\pgfpathlineto{\pgfqpoint{4.095426in}{2.844955in}}%
\pgfpathlineto{\pgfqpoint{4.108581in}{2.840515in}}%
\pgfpathlineto{\pgfqpoint{4.100996in}{2.831910in}}%
\pgfpathlineto{\pgfqpoint{4.093406in}{2.823375in}}%
\pgfpathlineto{\pgfqpoint{4.085811in}{2.814908in}}%
\pgfpathlineto{\pgfqpoint{4.078211in}{2.806506in}}%
\pgfpathlineto{\pgfqpoint{4.065045in}{2.810846in}}%
\pgfpathlineto{\pgfqpoint{4.051885in}{2.815220in}}%
\pgfpathlineto{\pgfqpoint{4.038729in}{2.819626in}}%
\pgfpathlineto{\pgfqpoint{4.025579in}{2.824066in}}%
\pgfpathlineto{\pgfqpoint{4.033191in}{2.832563in}}%
\pgfpathlineto{\pgfqpoint{4.040797in}{2.841128in}}%
\pgfpathlineto{\pgfqpoint{4.048398in}{2.849763in}}%
\pgfpathlineto{\pgfqpoint{4.055994in}{2.858473in}}%
\pgfpathclose%
\pgfusepath{fill}%
\end{pgfscope}%
\begin{pgfscope}%
\pgfpathrectangle{\pgfqpoint{1.150000in}{0.150000in}}{\pgfqpoint{5.700000in}{5.700000in}}%
\pgfusepath{clip}%
\pgfsetbuttcap%
\pgfsetroundjoin%
\definecolor{currentfill}{rgb}{0.279566,0.067836,0.391917}%
\pgfsetfillcolor{currentfill}%
\pgfsetfillopacity{0.700000}%
\pgfsetlinewidth{0.000000pt}%
\definecolor{currentstroke}{rgb}{0.000000,0.000000,0.000000}%
\pgfsetstrokecolor{currentstroke}%
\pgfsetdash{}{0pt}%
\pgfpathmoveto{\pgfqpoint{4.628257in}{2.902565in}}%
\pgfpathlineto{\pgfqpoint{4.641515in}{2.898315in}}%
\pgfpathlineto{\pgfqpoint{4.654778in}{2.894093in}}%
\pgfpathlineto{\pgfqpoint{4.668046in}{2.889899in}}%
\pgfpathlineto{\pgfqpoint{4.681321in}{2.885733in}}%
\pgfpathlineto{\pgfqpoint{4.673916in}{2.876204in}}%
\pgfpathlineto{\pgfqpoint{4.666508in}{2.866807in}}%
\pgfpathlineto{\pgfqpoint{4.659096in}{2.857537in}}%
\pgfpathlineto{\pgfqpoint{4.651682in}{2.848390in}}%
\pgfpathlineto{\pgfqpoint{4.638394in}{2.852392in}}%
\pgfpathlineto{\pgfqpoint{4.625113in}{2.856422in}}%
\pgfpathlineto{\pgfqpoint{4.611837in}{2.860480in}}%
\pgfpathlineto{\pgfqpoint{4.598568in}{2.864567in}}%
\pgfpathlineto{\pgfqpoint{4.605995in}{2.873873in}}%
\pgfpathlineto{\pgfqpoint{4.613419in}{2.883305in}}%
\pgfpathlineto{\pgfqpoint{4.620840in}{2.892867in}}%
\pgfpathlineto{\pgfqpoint{4.628257in}{2.902565in}}%
\pgfpathclose%
\pgfusepath{fill}%
\end{pgfscope}%
\begin{pgfscope}%
\pgfpathrectangle{\pgfqpoint{1.150000in}{0.150000in}}{\pgfqpoint{5.700000in}{5.700000in}}%
\pgfusepath{clip}%
\pgfsetbuttcap%
\pgfsetroundjoin%
\definecolor{currentfill}{rgb}{0.274952,0.037752,0.364543}%
\pgfsetfillcolor{currentfill}%
\pgfsetfillopacity{0.700000}%
\pgfsetlinewidth{0.000000pt}%
\definecolor{currentstroke}{rgb}{0.000000,0.000000,0.000000}%
\pgfsetstrokecolor{currentstroke}%
\pgfsetdash{}{0pt}%
\pgfpathmoveto{\pgfqpoint{3.702116in}{2.851119in}}%
\pgfpathlineto{\pgfqpoint{3.715189in}{2.846122in}}%
\pgfpathlineto{\pgfqpoint{3.728265in}{2.841164in}}%
\pgfpathlineto{\pgfqpoint{3.741347in}{2.836243in}}%
\pgfpathlineto{\pgfqpoint{3.754434in}{2.831361in}}%
\pgfpathlineto{\pgfqpoint{3.746729in}{2.823078in}}%
\pgfpathlineto{\pgfqpoint{3.739018in}{2.814849in}}%
\pgfpathlineto{\pgfqpoint{3.731301in}{2.806670in}}%
\pgfpathlineto{\pgfqpoint{3.723579in}{2.798539in}}%
\pgfpathlineto{\pgfqpoint{3.710482in}{2.803361in}}%
\pgfpathlineto{\pgfqpoint{3.697389in}{2.808221in}}%
\pgfpathlineto{\pgfqpoint{3.684300in}{2.813119in}}%
\pgfpathlineto{\pgfqpoint{3.671217in}{2.818055in}}%
\pgfpathlineto{\pgfqpoint{3.678950in}{2.826242in}}%
\pgfpathlineto{\pgfqpoint{3.686678in}{2.834480in}}%
\pgfpathlineto{\pgfqpoint{3.694400in}{2.842772in}}%
\pgfpathlineto{\pgfqpoint{3.702116in}{2.851119in}}%
\pgfpathclose%
\pgfusepath{fill}%
\end{pgfscope}%
\begin{pgfscope}%
\pgfpathrectangle{\pgfqpoint{1.150000in}{0.150000in}}{\pgfqpoint{5.700000in}{5.700000in}}%
\pgfusepath{clip}%
\pgfsetbuttcap%
\pgfsetroundjoin%
\definecolor{currentfill}{rgb}{0.277941,0.056324,0.381191}%
\pgfsetfillcolor{currentfill}%
\pgfsetfillopacity{0.700000}%
\pgfsetlinewidth{0.000000pt}%
\definecolor{currentstroke}{rgb}{0.000000,0.000000,0.000000}%
\pgfsetstrokecolor{currentstroke}%
\pgfsetdash{}{0pt}%
\pgfpathmoveto{\pgfqpoint{4.409887in}{2.877961in}}%
\pgfpathlineto{\pgfqpoint{4.423100in}{2.873677in}}%
\pgfpathlineto{\pgfqpoint{4.436319in}{2.869423in}}%
\pgfpathlineto{\pgfqpoint{4.449543in}{2.865199in}}%
\pgfpathlineto{\pgfqpoint{4.462773in}{2.861004in}}%
\pgfpathlineto{\pgfqpoint{4.455302in}{2.851963in}}%
\pgfpathlineto{\pgfqpoint{4.447827in}{2.843025in}}%
\pgfpathlineto{\pgfqpoint{4.440348in}{2.834185in}}%
\pgfpathlineto{\pgfqpoint{4.432864in}{2.825439in}}%
\pgfpathlineto{\pgfqpoint{4.419622in}{2.829495in}}%
\pgfpathlineto{\pgfqpoint{4.406385in}{2.833581in}}%
\pgfpathlineto{\pgfqpoint{4.393155in}{2.837697in}}%
\pgfpathlineto{\pgfqpoint{4.379930in}{2.841843in}}%
\pgfpathlineto{\pgfqpoint{4.387425in}{2.850723in}}%
\pgfpathlineto{\pgfqpoint{4.394917in}{2.859699in}}%
\pgfpathlineto{\pgfqpoint{4.402404in}{2.868777in}}%
\pgfpathlineto{\pgfqpoint{4.409887in}{2.877961in}}%
\pgfpathclose%
\pgfusepath{fill}%
\end{pgfscope}%
\begin{pgfscope}%
\pgfpathrectangle{\pgfqpoint{1.150000in}{0.150000in}}{\pgfqpoint{5.700000in}{5.700000in}}%
\pgfusepath{clip}%
\pgfsetbuttcap%
\pgfsetroundjoin%
\definecolor{currentfill}{rgb}{0.274952,0.037752,0.364543}%
\pgfsetfillcolor{currentfill}%
\pgfsetfillopacity{0.700000}%
\pgfsetlinewidth{0.000000pt}%
\definecolor{currentstroke}{rgb}{0.000000,0.000000,0.000000}%
\pgfsetstrokecolor{currentstroke}%
\pgfsetdash{}{0pt}%
\pgfpathmoveto{\pgfqpoint{3.837548in}{2.845615in}}%
\pgfpathlineto{\pgfqpoint{3.850648in}{2.840844in}}%
\pgfpathlineto{\pgfqpoint{3.863753in}{2.836109in}}%
\pgfpathlineto{\pgfqpoint{3.876863in}{2.831410in}}%
\pgfpathlineto{\pgfqpoint{3.889978in}{2.826746in}}%
\pgfpathlineto{\pgfqpoint{3.882317in}{2.818385in}}%
\pgfpathlineto{\pgfqpoint{3.874651in}{2.810080in}}%
\pgfpathlineto{\pgfqpoint{3.866980in}{2.801828in}}%
\pgfpathlineto{\pgfqpoint{3.859303in}{2.793628in}}%
\pgfpathlineto{\pgfqpoint{3.846177in}{2.798218in}}%
\pgfpathlineto{\pgfqpoint{3.833056in}{2.802844in}}%
\pgfpathlineto{\pgfqpoint{3.819939in}{2.807505in}}%
\pgfpathlineto{\pgfqpoint{3.806828in}{2.812203in}}%
\pgfpathlineto{\pgfqpoint{3.814517in}{2.820472in}}%
\pgfpathlineto{\pgfqpoint{3.822199in}{2.828795in}}%
\pgfpathlineto{\pgfqpoint{3.829877in}{2.837175in}}%
\pgfpathlineto{\pgfqpoint{3.837548in}{2.845615in}}%
\pgfpathclose%
\pgfusepath{fill}%
\end{pgfscope}%
\begin{pgfscope}%
\pgfpathrectangle{\pgfqpoint{1.150000in}{0.150000in}}{\pgfqpoint{5.700000in}{5.700000in}}%
\pgfusepath{clip}%
\pgfsetbuttcap%
\pgfsetroundjoin%
\definecolor{currentfill}{rgb}{0.276022,0.044167,0.370164}%
\pgfsetfillcolor{currentfill}%
\pgfsetfillopacity{0.700000}%
\pgfsetlinewidth{0.000000pt}%
\definecolor{currentstroke}{rgb}{0.000000,0.000000,0.000000}%
\pgfsetstrokecolor{currentstroke}%
\pgfsetdash{}{0pt}%
\pgfpathmoveto{\pgfqpoint{4.191496in}{2.857837in}}%
\pgfpathlineto{\pgfqpoint{4.204667in}{2.853446in}}%
\pgfpathlineto{\pgfqpoint{4.217842in}{2.849088in}}%
\pgfpathlineto{\pgfqpoint{4.231023in}{2.844761in}}%
\pgfpathlineto{\pgfqpoint{4.244210in}{2.840465in}}%
\pgfpathlineto{\pgfqpoint{4.236668in}{2.831774in}}%
\pgfpathlineto{\pgfqpoint{4.229121in}{2.823161in}}%
\pgfpathlineto{\pgfqpoint{4.221569in}{2.814624in}}%
\pgfpathlineto{\pgfqpoint{4.214013in}{2.806159in}}%
\pgfpathlineto{\pgfqpoint{4.200814in}{2.810343in}}%
\pgfpathlineto{\pgfqpoint{4.187622in}{2.814557in}}%
\pgfpathlineto{\pgfqpoint{4.174435in}{2.818803in}}%
\pgfpathlineto{\pgfqpoint{4.161253in}{2.823081in}}%
\pgfpathlineto{\pgfqpoint{4.168821in}{2.831654in}}%
\pgfpathlineto{\pgfqpoint{4.176384in}{2.840302in}}%
\pgfpathlineto{\pgfqpoint{4.183943in}{2.849028in}}%
\pgfpathlineto{\pgfqpoint{4.191496in}{2.857837in}}%
\pgfpathclose%
\pgfusepath{fill}%
\end{pgfscope}%
\begin{pgfscope}%
\pgfpathrectangle{\pgfqpoint{1.150000in}{0.150000in}}{\pgfqpoint{5.700000in}{5.700000in}}%
\pgfusepath{clip}%
\pgfsetbuttcap%
\pgfsetroundjoin%
\definecolor{currentfill}{rgb}{0.277941,0.056324,0.381191}%
\pgfsetfillcolor{currentfill}%
\pgfsetfillopacity{0.700000}%
\pgfsetlinewidth{0.000000pt}%
\definecolor{currentstroke}{rgb}{0.000000,0.000000,0.000000}%
\pgfsetstrokecolor{currentstroke}%
\pgfsetdash{}{0pt}%
\pgfpathmoveto{\pgfqpoint{3.212435in}{2.875472in}}%
\pgfpathlineto{\pgfqpoint{3.225435in}{2.869417in}}%
\pgfpathlineto{\pgfqpoint{3.238439in}{2.863413in}}%
\pgfpathlineto{\pgfqpoint{3.251447in}{2.857456in}}%
\pgfpathlineto{\pgfqpoint{3.264458in}{2.851549in}}%
\pgfpathlineto{\pgfqpoint{3.256576in}{2.843793in}}%
\pgfpathlineto{\pgfqpoint{3.248688in}{2.836087in}}%
\pgfpathlineto{\pgfqpoint{3.240793in}{2.828430in}}%
\pgfpathlineto{\pgfqpoint{3.232892in}{2.820821in}}%
\pgfpathlineto{\pgfqpoint{3.219869in}{2.826721in}}%
\pgfpathlineto{\pgfqpoint{3.206849in}{2.832668in}}%
\pgfpathlineto{\pgfqpoint{3.193832in}{2.838665in}}%
\pgfpathlineto{\pgfqpoint{3.180820in}{2.844711in}}%
\pgfpathlineto{\pgfqpoint{3.188733in}{2.852323in}}%
\pgfpathlineto{\pgfqpoint{3.196640in}{2.859987in}}%
\pgfpathlineto{\pgfqpoint{3.204541in}{2.867703in}}%
\pgfpathlineto{\pgfqpoint{3.212435in}{2.875472in}}%
\pgfpathclose%
\pgfusepath{fill}%
\end{pgfscope}%
\begin{pgfscope}%
\pgfpathrectangle{\pgfqpoint{1.150000in}{0.150000in}}{\pgfqpoint{5.700000in}{5.700000in}}%
\pgfusepath{clip}%
\pgfsetbuttcap%
\pgfsetroundjoin%
\definecolor{currentfill}{rgb}{0.277018,0.050344,0.375715}%
\pgfsetfillcolor{currentfill}%
\pgfsetfillopacity{0.700000}%
\pgfsetlinewidth{0.000000pt}%
\definecolor{currentstroke}{rgb}{0.000000,0.000000,0.000000}%
\pgfsetstrokecolor{currentstroke}%
\pgfsetdash{}{0pt}%
\pgfpathmoveto{\pgfqpoint{3.347958in}{2.859847in}}%
\pgfpathlineto{\pgfqpoint{3.360977in}{2.854153in}}%
\pgfpathlineto{\pgfqpoint{3.373999in}{2.848504in}}%
\pgfpathlineto{\pgfqpoint{3.387026in}{2.842901in}}%
\pgfpathlineto{\pgfqpoint{3.400057in}{2.837343in}}%
\pgfpathlineto{\pgfqpoint{3.392224in}{2.829430in}}%
\pgfpathlineto{\pgfqpoint{3.384385in}{2.821566in}}%
\pgfpathlineto{\pgfqpoint{3.376540in}{2.813747in}}%
\pgfpathlineto{\pgfqpoint{3.368689in}{2.805974in}}%
\pgfpathlineto{\pgfqpoint{3.355646in}{2.811511in}}%
\pgfpathlineto{\pgfqpoint{3.342607in}{2.817093in}}%
\pgfpathlineto{\pgfqpoint{3.329572in}{2.822720in}}%
\pgfpathlineto{\pgfqpoint{3.316542in}{2.828393in}}%
\pgfpathlineto{\pgfqpoint{3.324405in}{2.836182in}}%
\pgfpathlineto{\pgfqpoint{3.332262in}{2.844021in}}%
\pgfpathlineto{\pgfqpoint{3.340113in}{2.851909in}}%
\pgfpathlineto{\pgfqpoint{3.347958in}{2.859847in}}%
\pgfpathclose%
\pgfusepath{fill}%
\end{pgfscope}%
\begin{pgfscope}%
\pgfpathrectangle{\pgfqpoint{1.150000in}{0.150000in}}{\pgfqpoint{5.700000in}{5.700000in}}%
\pgfusepath{clip}%
\pgfsetbuttcap%
\pgfsetroundjoin%
\definecolor{currentfill}{rgb}{0.278791,0.062145,0.386592}%
\pgfsetfillcolor{currentfill}%
\pgfsetfillopacity{0.700000}%
\pgfsetlinewidth{0.000000pt}%
\definecolor{currentstroke}{rgb}{0.000000,0.000000,0.000000}%
\pgfsetstrokecolor{currentstroke}%
\pgfsetdash{}{0pt}%
\pgfpathmoveto{\pgfqpoint{4.545547in}{2.881197in}}%
\pgfpathlineto{\pgfqpoint{4.558793in}{2.876996in}}%
\pgfpathlineto{\pgfqpoint{4.572046in}{2.872824in}}%
\pgfpathlineto{\pgfqpoint{4.585304in}{2.868681in}}%
\pgfpathlineto{\pgfqpoint{4.598568in}{2.864567in}}%
\pgfpathlineto{\pgfqpoint{4.591136in}{2.855380in}}%
\pgfpathlineto{\pgfqpoint{4.583702in}{2.846308in}}%
\pgfpathlineto{\pgfqpoint{4.576263in}{2.837347in}}%
\pgfpathlineto{\pgfqpoint{4.568821in}{2.828491in}}%
\pgfpathlineto{\pgfqpoint{4.555545in}{2.832455in}}%
\pgfpathlineto{\pgfqpoint{4.542274in}{2.836447in}}%
\pgfpathlineto{\pgfqpoint{4.529009in}{2.840468in}}%
\pgfpathlineto{\pgfqpoint{4.515751in}{2.844517in}}%
\pgfpathlineto{\pgfqpoint{4.523205in}{2.853519in}}%
\pgfpathlineto{\pgfqpoint{4.530656in}{2.862630in}}%
\pgfpathlineto{\pgfqpoint{4.538103in}{2.871854in}}%
\pgfpathlineto{\pgfqpoint{4.545547in}{2.881197in}}%
\pgfpathclose%
\pgfusepath{fill}%
\end{pgfscope}%
\begin{pgfscope}%
\pgfpathrectangle{\pgfqpoint{1.150000in}{0.150000in}}{\pgfqpoint{5.700000in}{5.700000in}}%
\pgfusepath{clip}%
\pgfsetbuttcap%
\pgfsetroundjoin%
\definecolor{currentfill}{rgb}{0.274952,0.037752,0.364543}%
\pgfsetfillcolor{currentfill}%
\pgfsetfillopacity{0.700000}%
\pgfsetlinewidth{0.000000pt}%
\definecolor{currentstroke}{rgb}{0.000000,0.000000,0.000000}%
\pgfsetstrokecolor{currentstroke}%
\pgfsetdash{}{0pt}%
\pgfpathmoveto{\pgfqpoint{3.973031in}{2.842163in}}%
\pgfpathlineto{\pgfqpoint{3.986160in}{2.837588in}}%
\pgfpathlineto{\pgfqpoint{3.999295in}{2.833046in}}%
\pgfpathlineto{\pgfqpoint{4.012434in}{2.828539in}}%
\pgfpathlineto{\pgfqpoint{4.025579in}{2.824066in}}%
\pgfpathlineto{\pgfqpoint{4.017963in}{2.815633in}}%
\pgfpathlineto{\pgfqpoint{4.010341in}{2.807262in}}%
\pgfpathlineto{\pgfqpoint{4.002714in}{2.798950in}}%
\pgfpathlineto{\pgfqpoint{3.995081in}{2.790693in}}%
\pgfpathlineto{\pgfqpoint{3.981925in}{2.795080in}}%
\pgfpathlineto{\pgfqpoint{3.968774in}{2.799500in}}%
\pgfpathlineto{\pgfqpoint{3.955628in}{2.803955in}}%
\pgfpathlineto{\pgfqpoint{3.942488in}{2.808444in}}%
\pgfpathlineto{\pgfqpoint{3.950132in}{2.816782in}}%
\pgfpathlineto{\pgfqpoint{3.957770in}{2.825180in}}%
\pgfpathlineto{\pgfqpoint{3.965403in}{2.833639in}}%
\pgfpathlineto{\pgfqpoint{3.973031in}{2.842163in}}%
\pgfpathclose%
\pgfusepath{fill}%
\end{pgfscope}%
\begin{pgfscope}%
\pgfpathrectangle{\pgfqpoint{1.150000in}{0.150000in}}{\pgfqpoint{5.700000in}{5.700000in}}%
\pgfusepath{clip}%
\pgfsetbuttcap%
\pgfsetroundjoin%
\definecolor{currentfill}{rgb}{0.276022,0.044167,0.370164}%
\pgfsetfillcolor{currentfill}%
\pgfsetfillopacity{0.700000}%
\pgfsetlinewidth{0.000000pt}%
\definecolor{currentstroke}{rgb}{0.000000,0.000000,0.000000}%
\pgfsetstrokecolor{currentstroke}%
\pgfsetdash{}{0pt}%
\pgfpathmoveto{\pgfqpoint{3.483446in}{2.847562in}}%
\pgfpathlineto{\pgfqpoint{3.496487in}{2.842187in}}%
\pgfpathlineto{\pgfqpoint{3.509531in}{2.836854in}}%
\pgfpathlineto{\pgfqpoint{3.522581in}{2.831564in}}%
\pgfpathlineto{\pgfqpoint{3.535634in}{2.826315in}}%
\pgfpathlineto{\pgfqpoint{3.527849in}{2.818278in}}%
\pgfpathlineto{\pgfqpoint{3.520057in}{2.810287in}}%
\pgfpathlineto{\pgfqpoint{3.512260in}{2.802342in}}%
\pgfpathlineto{\pgfqpoint{3.504457in}{2.794440in}}%
\pgfpathlineto{\pgfqpoint{3.491392in}{2.799654in}}%
\pgfpathlineto{\pgfqpoint{3.478331in}{2.804910in}}%
\pgfpathlineto{\pgfqpoint{3.465275in}{2.810208in}}%
\pgfpathlineto{\pgfqpoint{3.452223in}{2.815549in}}%
\pgfpathlineto{\pgfqpoint{3.460038in}{2.823480in}}%
\pgfpathlineto{\pgfqpoint{3.467846in}{2.831458in}}%
\pgfpathlineto{\pgfqpoint{3.475649in}{2.839485in}}%
\pgfpathlineto{\pgfqpoint{3.483446in}{2.847562in}}%
\pgfpathclose%
\pgfusepath{fill}%
\end{pgfscope}%
\begin{pgfscope}%
\pgfpathrectangle{\pgfqpoint{1.150000in}{0.150000in}}{\pgfqpoint{5.700000in}{5.700000in}}%
\pgfusepath{clip}%
\pgfsetbuttcap%
\pgfsetroundjoin%
\definecolor{currentfill}{rgb}{0.277018,0.050344,0.375715}%
\pgfsetfillcolor{currentfill}%
\pgfsetfillopacity{0.700000}%
\pgfsetlinewidth{0.000000pt}%
\definecolor{currentstroke}{rgb}{0.000000,0.000000,0.000000}%
\pgfsetstrokecolor{currentstroke}%
\pgfsetdash{}{0pt}%
\pgfpathmoveto{\pgfqpoint{4.327086in}{2.858727in}}%
\pgfpathlineto{\pgfqpoint{4.340288in}{2.854460in}}%
\pgfpathlineto{\pgfqpoint{4.353497in}{2.850224in}}%
\pgfpathlineto{\pgfqpoint{4.366710in}{2.846019in}}%
\pgfpathlineto{\pgfqpoint{4.379930in}{2.841843in}}%
\pgfpathlineto{\pgfqpoint{4.372430in}{2.833056in}}%
\pgfpathlineto{\pgfqpoint{4.364925in}{2.824358in}}%
\pgfpathlineto{\pgfqpoint{4.357416in}{2.815744in}}%
\pgfpathlineto{\pgfqpoint{4.349903in}{2.807211in}}%
\pgfpathlineto{\pgfqpoint{4.336671in}{2.811261in}}%
\pgfpathlineto{\pgfqpoint{4.323445in}{2.815342in}}%
\pgfpathlineto{\pgfqpoint{4.310225in}{2.819452in}}%
\pgfpathlineto{\pgfqpoint{4.297011in}{2.823593in}}%
\pgfpathlineto{\pgfqpoint{4.304537in}{2.832247in}}%
\pgfpathlineto{\pgfqpoint{4.312058in}{2.840985in}}%
\pgfpathlineto{\pgfqpoint{4.319574in}{2.849810in}}%
\pgfpathlineto{\pgfqpoint{4.327086in}{2.858727in}}%
\pgfpathclose%
\pgfusepath{fill}%
\end{pgfscope}%
\begin{pgfscope}%
\pgfpathrectangle{\pgfqpoint{1.150000in}{0.150000in}}{\pgfqpoint{5.700000in}{5.700000in}}%
\pgfusepath{clip}%
\pgfsetbuttcap%
\pgfsetroundjoin%
\definecolor{currentfill}{rgb}{0.274952,0.037752,0.364543}%
\pgfsetfillcolor{currentfill}%
\pgfsetfillopacity{0.700000}%
\pgfsetlinewidth{0.000000pt}%
\definecolor{currentstroke}{rgb}{0.000000,0.000000,0.000000}%
\pgfsetstrokecolor{currentstroke}%
\pgfsetdash{}{0pt}%
\pgfpathmoveto{\pgfqpoint{3.618930in}{2.838191in}}%
\pgfpathlineto{\pgfqpoint{3.631995in}{2.833098in}}%
\pgfpathlineto{\pgfqpoint{3.645064in}{2.828045in}}%
\pgfpathlineto{\pgfqpoint{3.658138in}{2.823030in}}%
\pgfpathlineto{\pgfqpoint{3.671217in}{2.818055in}}%
\pgfpathlineto{\pgfqpoint{3.663478in}{2.809918in}}%
\pgfpathlineto{\pgfqpoint{3.655733in}{2.801829in}}%
\pgfpathlineto{\pgfqpoint{3.647982in}{2.793785in}}%
\pgfpathlineto{\pgfqpoint{3.640226in}{2.785785in}}%
\pgfpathlineto{\pgfqpoint{3.627136in}{2.790712in}}%
\pgfpathlineto{\pgfqpoint{3.614050in}{2.795678in}}%
\pgfpathlineto{\pgfqpoint{3.600969in}{2.800684in}}%
\pgfpathlineto{\pgfqpoint{3.587893in}{2.805730in}}%
\pgfpathlineto{\pgfqpoint{3.595661in}{2.813773in}}%
\pgfpathlineto{\pgfqpoint{3.603423in}{2.821863in}}%
\pgfpathlineto{\pgfqpoint{3.611180in}{2.830002in}}%
\pgfpathlineto{\pgfqpoint{3.618930in}{2.838191in}}%
\pgfpathclose%
\pgfusepath{fill}%
\end{pgfscope}%
\begin{pgfscope}%
\pgfpathrectangle{\pgfqpoint{1.150000in}{0.150000in}}{\pgfqpoint{5.700000in}{5.700000in}}%
\pgfusepath{clip}%
\pgfsetbuttcap%
\pgfsetroundjoin%
\definecolor{currentfill}{rgb}{0.276022,0.044167,0.370164}%
\pgfsetfillcolor{currentfill}%
\pgfsetfillopacity{0.700000}%
\pgfsetlinewidth{0.000000pt}%
\definecolor{currentstroke}{rgb}{0.000000,0.000000,0.000000}%
\pgfsetstrokecolor{currentstroke}%
\pgfsetdash{}{0pt}%
\pgfpathmoveto{\pgfqpoint{4.108581in}{2.840515in}}%
\pgfpathlineto{\pgfqpoint{4.121741in}{2.836108in}}%
\pgfpathlineto{\pgfqpoint{4.134906in}{2.831734in}}%
\pgfpathlineto{\pgfqpoint{4.148077in}{2.827392in}}%
\pgfpathlineto{\pgfqpoint{4.161253in}{2.823081in}}%
\pgfpathlineto{\pgfqpoint{4.153680in}{2.814580in}}%
\pgfpathlineto{\pgfqpoint{4.146102in}{2.806147in}}%
\pgfpathlineto{\pgfqpoint{4.138518in}{2.797778in}}%
\pgfpathlineto{\pgfqpoint{4.130930in}{2.789471in}}%
\pgfpathlineto{\pgfqpoint{4.117742in}{2.793681in}}%
\pgfpathlineto{\pgfqpoint{4.104560in}{2.797924in}}%
\pgfpathlineto{\pgfqpoint{4.091383in}{2.802198in}}%
\pgfpathlineto{\pgfqpoint{4.078211in}{2.806506in}}%
\pgfpathlineto{\pgfqpoint{4.085811in}{2.814908in}}%
\pgfpathlineto{\pgfqpoint{4.093406in}{2.823375in}}%
\pgfpathlineto{\pgfqpoint{4.100996in}{2.831910in}}%
\pgfpathlineto{\pgfqpoint{4.108581in}{2.840515in}}%
\pgfpathclose%
\pgfusepath{fill}%
\end{pgfscope}%
\begin{pgfscope}%
\pgfpathrectangle{\pgfqpoint{1.150000in}{0.150000in}}{\pgfqpoint{5.700000in}{5.700000in}}%
\pgfusepath{clip}%
\pgfsetbuttcap%
\pgfsetroundjoin%
\definecolor{currentfill}{rgb}{0.274952,0.037752,0.364543}%
\pgfsetfillcolor{currentfill}%
\pgfsetfillopacity{0.700000}%
\pgfsetlinewidth{0.000000pt}%
\definecolor{currentstroke}{rgb}{0.000000,0.000000,0.000000}%
\pgfsetstrokecolor{currentstroke}%
\pgfsetdash{}{0pt}%
\pgfpathmoveto{\pgfqpoint{3.754434in}{2.831361in}}%
\pgfpathlineto{\pgfqpoint{3.767525in}{2.826516in}}%
\pgfpathlineto{\pgfqpoint{3.780621in}{2.821708in}}%
\pgfpathlineto{\pgfqpoint{3.793722in}{2.816937in}}%
\pgfpathlineto{\pgfqpoint{3.806828in}{2.812203in}}%
\pgfpathlineto{\pgfqpoint{3.799135in}{2.803986in}}%
\pgfpathlineto{\pgfqpoint{3.791435in}{2.795819in}}%
\pgfpathlineto{\pgfqpoint{3.783730in}{2.787699in}}%
\pgfpathlineto{\pgfqpoint{3.776020in}{2.779624in}}%
\pgfpathlineto{\pgfqpoint{3.762902in}{2.784298in}}%
\pgfpathlineto{\pgfqpoint{3.749790in}{2.789008in}}%
\pgfpathlineto{\pgfqpoint{3.736682in}{2.793755in}}%
\pgfpathlineto{\pgfqpoint{3.723579in}{2.798539in}}%
\pgfpathlineto{\pgfqpoint{3.731301in}{2.806670in}}%
\pgfpathlineto{\pgfqpoint{3.739018in}{2.814849in}}%
\pgfpathlineto{\pgfqpoint{3.746729in}{2.823078in}}%
\pgfpathlineto{\pgfqpoint{3.754434in}{2.831361in}}%
\pgfpathclose%
\pgfusepath{fill}%
\end{pgfscope}%
\begin{pgfscope}%
\pgfpathrectangle{\pgfqpoint{1.150000in}{0.150000in}}{\pgfqpoint{5.700000in}{5.700000in}}%
\pgfusepath{clip}%
\pgfsetbuttcap%
\pgfsetroundjoin%
\definecolor{currentfill}{rgb}{0.279566,0.067836,0.391917}%
\pgfsetfillcolor{currentfill}%
\pgfsetfillopacity{0.700000}%
\pgfsetlinewidth{0.000000pt}%
\definecolor{currentstroke}{rgb}{0.000000,0.000000,0.000000}%
\pgfsetstrokecolor{currentstroke}%
\pgfsetdash{}{0pt}%
\pgfpathmoveto{\pgfqpoint{4.681321in}{2.885733in}}%
\pgfpathlineto{\pgfqpoint{4.694602in}{2.881594in}}%
\pgfpathlineto{\pgfqpoint{4.707888in}{2.877484in}}%
\pgfpathlineto{\pgfqpoint{4.721180in}{2.873401in}}%
\pgfpathlineto{\pgfqpoint{4.734478in}{2.869346in}}%
\pgfpathlineto{\pgfqpoint{4.727086in}{2.859985in}}%
\pgfpathlineto{\pgfqpoint{4.719691in}{2.850754in}}%
\pgfpathlineto{\pgfqpoint{4.712292in}{2.841647in}}%
\pgfpathlineto{\pgfqpoint{4.704890in}{2.832659in}}%
\pgfpathlineto{\pgfqpoint{4.691579in}{2.836550in}}%
\pgfpathlineto{\pgfqpoint{4.678274in}{2.840469in}}%
\pgfpathlineto{\pgfqpoint{4.664975in}{2.844415in}}%
\pgfpathlineto{\pgfqpoint{4.651682in}{2.848390in}}%
\pgfpathlineto{\pgfqpoint{4.659096in}{2.857537in}}%
\pgfpathlineto{\pgfqpoint{4.666508in}{2.866807in}}%
\pgfpathlineto{\pgfqpoint{4.673916in}{2.876204in}}%
\pgfpathlineto{\pgfqpoint{4.681321in}{2.885733in}}%
\pgfpathclose%
\pgfusepath{fill}%
\end{pgfscope}%
\begin{pgfscope}%
\pgfpathrectangle{\pgfqpoint{1.150000in}{0.150000in}}{\pgfqpoint{5.700000in}{5.700000in}}%
\pgfusepath{clip}%
\pgfsetbuttcap%
\pgfsetroundjoin%
\definecolor{currentfill}{rgb}{0.277941,0.056324,0.381191}%
\pgfsetfillcolor{currentfill}%
\pgfsetfillopacity{0.700000}%
\pgfsetlinewidth{0.000000pt}%
\definecolor{currentstroke}{rgb}{0.000000,0.000000,0.000000}%
\pgfsetstrokecolor{currentstroke}%
\pgfsetdash{}{0pt}%
\pgfpathmoveto{\pgfqpoint{4.462773in}{2.861004in}}%
\pgfpathlineto{\pgfqpoint{4.476009in}{2.856839in}}%
\pgfpathlineto{\pgfqpoint{4.489250in}{2.852702in}}%
\pgfpathlineto{\pgfqpoint{4.502498in}{2.848595in}}%
\pgfpathlineto{\pgfqpoint{4.515751in}{2.844517in}}%
\pgfpathlineto{\pgfqpoint{4.508292in}{2.835619in}}%
\pgfpathlineto{\pgfqpoint{4.500829in}{2.826821in}}%
\pgfpathlineto{\pgfqpoint{4.493362in}{2.818118in}}%
\pgfpathlineto{\pgfqpoint{4.485891in}{2.809505in}}%
\pgfpathlineto{\pgfqpoint{4.472626in}{2.813445in}}%
\pgfpathlineto{\pgfqpoint{4.459366in}{2.817414in}}%
\pgfpathlineto{\pgfqpoint{4.446112in}{2.821412in}}%
\pgfpathlineto{\pgfqpoint{4.432864in}{2.825439in}}%
\pgfpathlineto{\pgfqpoint{4.440348in}{2.834185in}}%
\pgfpathlineto{\pgfqpoint{4.447827in}{2.843025in}}%
\pgfpathlineto{\pgfqpoint{4.455302in}{2.851963in}}%
\pgfpathlineto{\pgfqpoint{4.462773in}{2.861004in}}%
\pgfpathclose%
\pgfusepath{fill}%
\end{pgfscope}%
\begin{pgfscope}%
\pgfpathrectangle{\pgfqpoint{1.150000in}{0.150000in}}{\pgfqpoint{5.700000in}{5.700000in}}%
\pgfusepath{clip}%
\pgfsetbuttcap%
\pgfsetroundjoin%
\definecolor{currentfill}{rgb}{0.274952,0.037752,0.364543}%
\pgfsetfillcolor{currentfill}%
\pgfsetfillopacity{0.700000}%
\pgfsetlinewidth{0.000000pt}%
\definecolor{currentstroke}{rgb}{0.000000,0.000000,0.000000}%
\pgfsetstrokecolor{currentstroke}%
\pgfsetdash{}{0pt}%
\pgfpathmoveto{\pgfqpoint{3.889978in}{2.826746in}}%
\pgfpathlineto{\pgfqpoint{3.903097in}{2.822118in}}%
\pgfpathlineto{\pgfqpoint{3.916222in}{2.817525in}}%
\pgfpathlineto{\pgfqpoint{3.929353in}{2.812967in}}%
\pgfpathlineto{\pgfqpoint{3.942488in}{2.808444in}}%
\pgfpathlineto{\pgfqpoint{3.934839in}{2.800161in}}%
\pgfpathlineto{\pgfqpoint{3.927184in}{2.791931in}}%
\pgfpathlineto{\pgfqpoint{3.919524in}{2.783752in}}%
\pgfpathlineto{\pgfqpoint{3.911859in}{2.775621in}}%
\pgfpathlineto{\pgfqpoint{3.898712in}{2.780071in}}%
\pgfpathlineto{\pgfqpoint{3.885571in}{2.784555in}}%
\pgfpathlineto{\pgfqpoint{3.872434in}{2.789074in}}%
\pgfpathlineto{\pgfqpoint{3.859303in}{2.793628in}}%
\pgfpathlineto{\pgfqpoint{3.866980in}{2.801828in}}%
\pgfpathlineto{\pgfqpoint{3.874651in}{2.810080in}}%
\pgfpathlineto{\pgfqpoint{3.882317in}{2.818385in}}%
\pgfpathlineto{\pgfqpoint{3.889978in}{2.826746in}}%
\pgfpathclose%
\pgfusepath{fill}%
\end{pgfscope}%
\begin{pgfscope}%
\pgfpathrectangle{\pgfqpoint{1.150000in}{0.150000in}}{\pgfqpoint{5.700000in}{5.700000in}}%
\pgfusepath{clip}%
\pgfsetbuttcap%
\pgfsetroundjoin%
\definecolor{currentfill}{rgb}{0.276022,0.044167,0.370164}%
\pgfsetfillcolor{currentfill}%
\pgfsetfillopacity{0.700000}%
\pgfsetlinewidth{0.000000pt}%
\definecolor{currentstroke}{rgb}{0.000000,0.000000,0.000000}%
\pgfsetstrokecolor{currentstroke}%
\pgfsetdash{}{0pt}%
\pgfpathmoveto{\pgfqpoint{4.244210in}{2.840465in}}%
\pgfpathlineto{\pgfqpoint{4.257402in}{2.836201in}}%
\pgfpathlineto{\pgfqpoint{4.270599in}{2.831967in}}%
\pgfpathlineto{\pgfqpoint{4.283802in}{2.827765in}}%
\pgfpathlineto{\pgfqpoint{4.297011in}{2.823593in}}%
\pgfpathlineto{\pgfqpoint{4.289481in}{2.815019in}}%
\pgfpathlineto{\pgfqpoint{4.281946in}{2.806521in}}%
\pgfpathlineto{\pgfqpoint{4.274406in}{2.798095in}}%
\pgfpathlineto{\pgfqpoint{4.266862in}{2.789738in}}%
\pgfpathlineto{\pgfqpoint{4.253641in}{2.793797in}}%
\pgfpathlineto{\pgfqpoint{4.240426in}{2.797887in}}%
\pgfpathlineto{\pgfqpoint{4.227216in}{2.802008in}}%
\pgfpathlineto{\pgfqpoint{4.214013in}{2.806159in}}%
\pgfpathlineto{\pgfqpoint{4.221569in}{2.814624in}}%
\pgfpathlineto{\pgfqpoint{4.229121in}{2.823161in}}%
\pgfpathlineto{\pgfqpoint{4.236668in}{2.831774in}}%
\pgfpathlineto{\pgfqpoint{4.244210in}{2.840465in}}%
\pgfpathclose%
\pgfusepath{fill}%
\end{pgfscope}%
\begin{pgfscope}%
\pgfpathrectangle{\pgfqpoint{1.150000in}{0.150000in}}{\pgfqpoint{5.700000in}{5.700000in}}%
\pgfusepath{clip}%
\pgfsetbuttcap%
\pgfsetroundjoin%
\definecolor{currentfill}{rgb}{0.277018,0.050344,0.375715}%
\pgfsetfillcolor{currentfill}%
\pgfsetfillopacity{0.700000}%
\pgfsetlinewidth{0.000000pt}%
\definecolor{currentstroke}{rgb}{0.000000,0.000000,0.000000}%
\pgfsetstrokecolor{currentstroke}%
\pgfsetdash{}{0pt}%
\pgfpathmoveto{\pgfqpoint{3.264458in}{2.851549in}}%
\pgfpathlineto{\pgfqpoint{3.277473in}{2.845689in}}%
\pgfpathlineto{\pgfqpoint{3.290492in}{2.839877in}}%
\pgfpathlineto{\pgfqpoint{3.303515in}{2.834111in}}%
\pgfpathlineto{\pgfqpoint{3.316542in}{2.828393in}}%
\pgfpathlineto{\pgfqpoint{3.308672in}{2.820650in}}%
\pgfpathlineto{\pgfqpoint{3.300796in}{2.812954in}}%
\pgfpathlineto{\pgfqpoint{3.292914in}{2.805304in}}%
\pgfpathlineto{\pgfqpoint{3.285025in}{2.797699in}}%
\pgfpathlineto{\pgfqpoint{3.271986in}{2.803410in}}%
\pgfpathlineto{\pgfqpoint{3.258951in}{2.809166in}}%
\pgfpathlineto{\pgfqpoint{3.245920in}{2.814970in}}%
\pgfpathlineto{\pgfqpoint{3.232892in}{2.820821in}}%
\pgfpathlineto{\pgfqpoint{3.240793in}{2.828430in}}%
\pgfpathlineto{\pgfqpoint{3.248688in}{2.836087in}}%
\pgfpathlineto{\pgfqpoint{3.256576in}{2.843793in}}%
\pgfpathlineto{\pgfqpoint{3.264458in}{2.851549in}}%
\pgfpathclose%
\pgfusepath{fill}%
\end{pgfscope}%
\begin{pgfscope}%
\pgfpathrectangle{\pgfqpoint{1.150000in}{0.150000in}}{\pgfqpoint{5.700000in}{5.700000in}}%
\pgfusepath{clip}%
\pgfsetbuttcap%
\pgfsetroundjoin%
\definecolor{currentfill}{rgb}{0.278791,0.062145,0.386592}%
\pgfsetfillcolor{currentfill}%
\pgfsetfillopacity{0.700000}%
\pgfsetlinewidth{0.000000pt}%
\definecolor{currentstroke}{rgb}{0.000000,0.000000,0.000000}%
\pgfsetstrokecolor{currentstroke}%
\pgfsetdash{}{0pt}%
\pgfpathmoveto{\pgfqpoint{3.128805in}{2.869398in}}%
\pgfpathlineto{\pgfqpoint{3.141803in}{2.863150in}}%
\pgfpathlineto{\pgfqpoint{3.154805in}{2.856953in}}%
\pgfpathlineto{\pgfqpoint{3.167811in}{2.850806in}}%
\pgfpathlineto{\pgfqpoint{3.180820in}{2.844711in}}%
\pgfpathlineto{\pgfqpoint{3.172900in}{2.837149in}}%
\pgfpathlineto{\pgfqpoint{3.164973in}{2.829638in}}%
\pgfpathlineto{\pgfqpoint{3.157040in}{2.822176in}}%
\pgfpathlineto{\pgfqpoint{3.149100in}{2.814764in}}%
\pgfpathlineto{\pgfqpoint{3.136078in}{2.820864in}}%
\pgfpathlineto{\pgfqpoint{3.123060in}{2.827015in}}%
\pgfpathlineto{\pgfqpoint{3.110045in}{2.833217in}}%
\pgfpathlineto{\pgfqpoint{3.097033in}{2.839470in}}%
\pgfpathlineto{\pgfqpoint{3.104986in}{2.846873in}}%
\pgfpathlineto{\pgfqpoint{3.112932in}{2.854328in}}%
\pgfpathlineto{\pgfqpoint{3.120872in}{2.861836in}}%
\pgfpathlineto{\pgfqpoint{3.128805in}{2.869398in}}%
\pgfpathclose%
\pgfusepath{fill}%
\end{pgfscope}%
\begin{pgfscope}%
\pgfpathrectangle{\pgfqpoint{1.150000in}{0.150000in}}{\pgfqpoint{5.700000in}{5.700000in}}%
\pgfusepath{clip}%
\pgfsetbuttcap%
\pgfsetroundjoin%
\definecolor{currentfill}{rgb}{0.276022,0.044167,0.370164}%
\pgfsetfillcolor{currentfill}%
\pgfsetfillopacity{0.700000}%
\pgfsetlinewidth{0.000000pt}%
\definecolor{currentstroke}{rgb}{0.000000,0.000000,0.000000}%
\pgfsetstrokecolor{currentstroke}%
\pgfsetdash{}{0pt}%
\pgfpathmoveto{\pgfqpoint{3.400057in}{2.837343in}}%
\pgfpathlineto{\pgfqpoint{3.413092in}{2.831829in}}%
\pgfpathlineto{\pgfqpoint{3.426131in}{2.826358in}}%
\pgfpathlineto{\pgfqpoint{3.439175in}{2.820932in}}%
\pgfpathlineto{\pgfqpoint{3.452223in}{2.815549in}}%
\pgfpathlineto{\pgfqpoint{3.444402in}{2.807663in}}%
\pgfpathlineto{\pgfqpoint{3.436575in}{2.799821in}}%
\pgfpathlineto{\pgfqpoint{3.428741in}{2.792023in}}%
\pgfpathlineto{\pgfqpoint{3.420902in}{2.784267in}}%
\pgfpathlineto{\pgfqpoint{3.407843in}{2.789628in}}%
\pgfpathlineto{\pgfqpoint{3.394787in}{2.795033in}}%
\pgfpathlineto{\pgfqpoint{3.381736in}{2.800482in}}%
\pgfpathlineto{\pgfqpoint{3.368689in}{2.805974in}}%
\pgfpathlineto{\pgfqpoint{3.376540in}{2.813747in}}%
\pgfpathlineto{\pgfqpoint{3.384385in}{2.821566in}}%
\pgfpathlineto{\pgfqpoint{3.392224in}{2.829430in}}%
\pgfpathlineto{\pgfqpoint{3.400057in}{2.837343in}}%
\pgfpathclose%
\pgfusepath{fill}%
\end{pgfscope}%
\begin{pgfscope}%
\pgfpathrectangle{\pgfqpoint{1.150000in}{0.150000in}}{\pgfqpoint{5.700000in}{5.700000in}}%
\pgfusepath{clip}%
\pgfsetbuttcap%
\pgfsetroundjoin%
\definecolor{currentfill}{rgb}{0.274952,0.037752,0.364543}%
\pgfsetfillcolor{currentfill}%
\pgfsetfillopacity{0.700000}%
\pgfsetlinewidth{0.000000pt}%
\definecolor{currentstroke}{rgb}{0.000000,0.000000,0.000000}%
\pgfsetstrokecolor{currentstroke}%
\pgfsetdash{}{0pt}%
\pgfpathmoveto{\pgfqpoint{3.535634in}{2.826315in}}%
\pgfpathlineto{\pgfqpoint{3.548692in}{2.821107in}}%
\pgfpathlineto{\pgfqpoint{3.561755in}{2.815941in}}%
\pgfpathlineto{\pgfqpoint{3.574822in}{2.810815in}}%
\pgfpathlineto{\pgfqpoint{3.587893in}{2.805730in}}%
\pgfpathlineto{\pgfqpoint{3.580119in}{2.797732in}}%
\pgfpathlineto{\pgfqpoint{3.572340in}{2.789778in}}%
\pgfpathlineto{\pgfqpoint{3.564554in}{2.781865in}}%
\pgfpathlineto{\pgfqpoint{3.556763in}{2.773994in}}%
\pgfpathlineto{\pgfqpoint{3.543680in}{2.779044in}}%
\pgfpathlineto{\pgfqpoint{3.530601in}{2.784135in}}%
\pgfpathlineto{\pgfqpoint{3.517527in}{2.789267in}}%
\pgfpathlineto{\pgfqpoint{3.504457in}{2.794440in}}%
\pgfpathlineto{\pgfqpoint{3.512260in}{2.802342in}}%
\pgfpathlineto{\pgfqpoint{3.520057in}{2.810287in}}%
\pgfpathlineto{\pgfqpoint{3.527849in}{2.818278in}}%
\pgfpathlineto{\pgfqpoint{3.535634in}{2.826315in}}%
\pgfpathclose%
\pgfusepath{fill}%
\end{pgfscope}%
\begin{pgfscope}%
\pgfpathrectangle{\pgfqpoint{1.150000in}{0.150000in}}{\pgfqpoint{5.700000in}{5.700000in}}%
\pgfusepath{clip}%
\pgfsetbuttcap%
\pgfsetroundjoin%
\definecolor{currentfill}{rgb}{0.278791,0.062145,0.386592}%
\pgfsetfillcolor{currentfill}%
\pgfsetfillopacity{0.700000}%
\pgfsetlinewidth{0.000000pt}%
\definecolor{currentstroke}{rgb}{0.000000,0.000000,0.000000}%
\pgfsetstrokecolor{currentstroke}%
\pgfsetdash{}{0pt}%
\pgfpathmoveto{\pgfqpoint{4.598568in}{2.864567in}}%
\pgfpathlineto{\pgfqpoint{4.611837in}{2.860480in}}%
\pgfpathlineto{\pgfqpoint{4.625113in}{2.856422in}}%
\pgfpathlineto{\pgfqpoint{4.638394in}{2.852392in}}%
\pgfpathlineto{\pgfqpoint{4.651682in}{2.848390in}}%
\pgfpathlineto{\pgfqpoint{4.644263in}{2.839359in}}%
\pgfpathlineto{\pgfqpoint{4.636841in}{2.830440in}}%
\pgfpathlineto{\pgfqpoint{4.629416in}{2.821629in}}%
\pgfpathlineto{\pgfqpoint{4.621986in}{2.812920in}}%
\pgfpathlineto{\pgfqpoint{4.608686in}{2.816771in}}%
\pgfpathlineto{\pgfqpoint{4.595392in}{2.820649in}}%
\pgfpathlineto{\pgfqpoint{4.582103in}{2.824556in}}%
\pgfpathlineto{\pgfqpoint{4.568821in}{2.828491in}}%
\pgfpathlineto{\pgfqpoint{4.576263in}{2.837347in}}%
\pgfpathlineto{\pgfqpoint{4.583702in}{2.846308in}}%
\pgfpathlineto{\pgfqpoint{4.591136in}{2.855380in}}%
\pgfpathlineto{\pgfqpoint{4.598568in}{2.864567in}}%
\pgfpathclose%
\pgfusepath{fill}%
\end{pgfscope}%
\begin{pgfscope}%
\pgfpathrectangle{\pgfqpoint{1.150000in}{0.150000in}}{\pgfqpoint{5.700000in}{5.700000in}}%
\pgfusepath{clip}%
\pgfsetbuttcap%
\pgfsetroundjoin%
\definecolor{currentfill}{rgb}{0.274952,0.037752,0.364543}%
\pgfsetfillcolor{currentfill}%
\pgfsetfillopacity{0.700000}%
\pgfsetlinewidth{0.000000pt}%
\definecolor{currentstroke}{rgb}{0.000000,0.000000,0.000000}%
\pgfsetstrokecolor{currentstroke}%
\pgfsetdash{}{0pt}%
\pgfpathmoveto{\pgfqpoint{4.025579in}{2.824066in}}%
\pgfpathlineto{\pgfqpoint{4.038729in}{2.819626in}}%
\pgfpathlineto{\pgfqpoint{4.051885in}{2.815220in}}%
\pgfpathlineto{\pgfqpoint{4.065045in}{2.810846in}}%
\pgfpathlineto{\pgfqpoint{4.078211in}{2.806506in}}%
\pgfpathlineto{\pgfqpoint{4.070606in}{2.798165in}}%
\pgfpathlineto{\pgfqpoint{4.062996in}{2.789883in}}%
\pgfpathlineto{\pgfqpoint{4.055381in}{2.781655in}}%
\pgfpathlineto{\pgfqpoint{4.047760in}{2.773480in}}%
\pgfpathlineto{\pgfqpoint{4.034582in}{2.777734in}}%
\pgfpathlineto{\pgfqpoint{4.021410in}{2.782020in}}%
\pgfpathlineto{\pgfqpoint{4.008243in}{2.786340in}}%
\pgfpathlineto{\pgfqpoint{3.995081in}{2.790693in}}%
\pgfpathlineto{\pgfqpoint{4.002714in}{2.798950in}}%
\pgfpathlineto{\pgfqpoint{4.010341in}{2.807262in}}%
\pgfpathlineto{\pgfqpoint{4.017963in}{2.815633in}}%
\pgfpathlineto{\pgfqpoint{4.025579in}{2.824066in}}%
\pgfpathclose%
\pgfusepath{fill}%
\end{pgfscope}%
\begin{pgfscope}%
\pgfpathrectangle{\pgfqpoint{1.150000in}{0.150000in}}{\pgfqpoint{5.700000in}{5.700000in}}%
\pgfusepath{clip}%
\pgfsetbuttcap%
\pgfsetroundjoin%
\definecolor{currentfill}{rgb}{0.273809,0.031497,0.358853}%
\pgfsetfillcolor{currentfill}%
\pgfsetfillopacity{0.700000}%
\pgfsetlinewidth{0.000000pt}%
\definecolor{currentstroke}{rgb}{0.000000,0.000000,0.000000}%
\pgfsetstrokecolor{currentstroke}%
\pgfsetdash{}{0pt}%
\pgfpathmoveto{\pgfqpoint{3.671217in}{2.818055in}}%
\pgfpathlineto{\pgfqpoint{3.684300in}{2.813119in}}%
\pgfpathlineto{\pgfqpoint{3.697389in}{2.808221in}}%
\pgfpathlineto{\pgfqpoint{3.710482in}{2.803361in}}%
\pgfpathlineto{\pgfqpoint{3.723579in}{2.798539in}}%
\pgfpathlineto{\pgfqpoint{3.715852in}{2.790455in}}%
\pgfpathlineto{\pgfqpoint{3.708118in}{2.782415in}}%
\pgfpathlineto{\pgfqpoint{3.700379in}{2.774417in}}%
\pgfpathlineto{\pgfqpoint{3.692635in}{2.766460in}}%
\pgfpathlineto{\pgfqpoint{3.679525in}{2.771234in}}%
\pgfpathlineto{\pgfqpoint{3.666421in}{2.776046in}}%
\pgfpathlineto{\pgfqpoint{3.653321in}{2.780896in}}%
\pgfpathlineto{\pgfqpoint{3.640226in}{2.785785in}}%
\pgfpathlineto{\pgfqpoint{3.647982in}{2.793785in}}%
\pgfpathlineto{\pgfqpoint{3.655733in}{2.801829in}}%
\pgfpathlineto{\pgfqpoint{3.663478in}{2.809918in}}%
\pgfpathlineto{\pgfqpoint{3.671217in}{2.818055in}}%
\pgfpathclose%
\pgfusepath{fill}%
\end{pgfscope}%
\begin{pgfscope}%
\pgfpathrectangle{\pgfqpoint{1.150000in}{0.150000in}}{\pgfqpoint{5.700000in}{5.700000in}}%
\pgfusepath{clip}%
\pgfsetbuttcap%
\pgfsetroundjoin%
\definecolor{currentfill}{rgb}{0.277018,0.050344,0.375715}%
\pgfsetfillcolor{currentfill}%
\pgfsetfillopacity{0.700000}%
\pgfsetlinewidth{0.000000pt}%
\definecolor{currentstroke}{rgb}{0.000000,0.000000,0.000000}%
\pgfsetstrokecolor{currentstroke}%
\pgfsetdash{}{0pt}%
\pgfpathmoveto{\pgfqpoint{4.379930in}{2.841843in}}%
\pgfpathlineto{\pgfqpoint{4.393155in}{2.837697in}}%
\pgfpathlineto{\pgfqpoint{4.406385in}{2.833581in}}%
\pgfpathlineto{\pgfqpoint{4.419622in}{2.829495in}}%
\pgfpathlineto{\pgfqpoint{4.432864in}{2.825439in}}%
\pgfpathlineto{\pgfqpoint{4.425376in}{2.816782in}}%
\pgfpathlineto{\pgfqpoint{4.417884in}{2.808211in}}%
\pgfpathlineto{\pgfqpoint{4.410387in}{2.799722in}}%
\pgfpathlineto{\pgfqpoint{4.402886in}{2.791309in}}%
\pgfpathlineto{\pgfqpoint{4.389631in}{2.795240in}}%
\pgfpathlineto{\pgfqpoint{4.376383in}{2.799201in}}%
\pgfpathlineto{\pgfqpoint{4.363140in}{2.803191in}}%
\pgfpathlineto{\pgfqpoint{4.349903in}{2.807211in}}%
\pgfpathlineto{\pgfqpoint{4.357416in}{2.815744in}}%
\pgfpathlineto{\pgfqpoint{4.364925in}{2.824358in}}%
\pgfpathlineto{\pgfqpoint{4.372430in}{2.833056in}}%
\pgfpathlineto{\pgfqpoint{4.379930in}{2.841843in}}%
\pgfpathclose%
\pgfusepath{fill}%
\end{pgfscope}%
\begin{pgfscope}%
\pgfpathrectangle{\pgfqpoint{1.150000in}{0.150000in}}{\pgfqpoint{5.700000in}{5.700000in}}%
\pgfusepath{clip}%
\pgfsetbuttcap%
\pgfsetroundjoin%
\definecolor{currentfill}{rgb}{0.274952,0.037752,0.364543}%
\pgfsetfillcolor{currentfill}%
\pgfsetfillopacity{0.700000}%
\pgfsetlinewidth{0.000000pt}%
\definecolor{currentstroke}{rgb}{0.000000,0.000000,0.000000}%
\pgfsetstrokecolor{currentstroke}%
\pgfsetdash{}{0pt}%
\pgfpathmoveto{\pgfqpoint{4.161253in}{2.823081in}}%
\pgfpathlineto{\pgfqpoint{4.174435in}{2.818803in}}%
\pgfpathlineto{\pgfqpoint{4.187622in}{2.814557in}}%
\pgfpathlineto{\pgfqpoint{4.200814in}{2.810343in}}%
\pgfpathlineto{\pgfqpoint{4.214013in}{2.806159in}}%
\pgfpathlineto{\pgfqpoint{4.206451in}{2.797763in}}%
\pgfpathlineto{\pgfqpoint{4.198885in}{2.789431in}}%
\pgfpathlineto{\pgfqpoint{4.191314in}{2.781160in}}%
\pgfpathlineto{\pgfqpoint{4.183737in}{2.772948in}}%
\pgfpathlineto{\pgfqpoint{4.170527in}{2.777031in}}%
\pgfpathlineto{\pgfqpoint{4.157323in}{2.781146in}}%
\pgfpathlineto{\pgfqpoint{4.144124in}{2.785292in}}%
\pgfpathlineto{\pgfqpoint{4.130930in}{2.789471in}}%
\pgfpathlineto{\pgfqpoint{4.138518in}{2.797778in}}%
\pgfpathlineto{\pgfqpoint{4.146102in}{2.806147in}}%
\pgfpathlineto{\pgfqpoint{4.153680in}{2.814580in}}%
\pgfpathlineto{\pgfqpoint{4.161253in}{2.823081in}}%
\pgfpathclose%
\pgfusepath{fill}%
\end{pgfscope}%
\begin{pgfscope}%
\pgfpathrectangle{\pgfqpoint{1.150000in}{0.150000in}}{\pgfqpoint{5.700000in}{5.700000in}}%
\pgfusepath{clip}%
\pgfsetbuttcap%
\pgfsetroundjoin%
\definecolor{currentfill}{rgb}{0.273809,0.031497,0.358853}%
\pgfsetfillcolor{currentfill}%
\pgfsetfillopacity{0.700000}%
\pgfsetlinewidth{0.000000pt}%
\definecolor{currentstroke}{rgb}{0.000000,0.000000,0.000000}%
\pgfsetstrokecolor{currentstroke}%
\pgfsetdash{}{0pt}%
\pgfpathmoveto{\pgfqpoint{3.806828in}{2.812203in}}%
\pgfpathlineto{\pgfqpoint{3.819939in}{2.807505in}}%
\pgfpathlineto{\pgfqpoint{3.833056in}{2.802844in}}%
\pgfpathlineto{\pgfqpoint{3.846177in}{2.798218in}}%
\pgfpathlineto{\pgfqpoint{3.859303in}{2.793628in}}%
\pgfpathlineto{\pgfqpoint{3.851620in}{2.785477in}}%
\pgfpathlineto{\pgfqpoint{3.843933in}{2.777372in}}%
\pgfpathlineto{\pgfqpoint{3.836239in}{2.769312in}}%
\pgfpathlineto{\pgfqpoint{3.828540in}{2.761293in}}%
\pgfpathlineto{\pgfqpoint{3.815403in}{2.765822in}}%
\pgfpathlineto{\pgfqpoint{3.802270in}{2.770387in}}%
\pgfpathlineto{\pgfqpoint{3.789142in}{2.774987in}}%
\pgfpathlineto{\pgfqpoint{3.776020in}{2.779624in}}%
\pgfpathlineto{\pgfqpoint{3.783730in}{2.787699in}}%
\pgfpathlineto{\pgfqpoint{3.791435in}{2.795819in}}%
\pgfpathlineto{\pgfqpoint{3.799135in}{2.803986in}}%
\pgfpathlineto{\pgfqpoint{3.806828in}{2.812203in}}%
\pgfpathclose%
\pgfusepath{fill}%
\end{pgfscope}%
\begin{pgfscope}%
\pgfpathrectangle{\pgfqpoint{1.150000in}{0.150000in}}{\pgfqpoint{5.700000in}{5.700000in}}%
\pgfusepath{clip}%
\pgfsetbuttcap%
\pgfsetroundjoin%
\definecolor{currentfill}{rgb}{0.279566,0.067836,0.391917}%
\pgfsetfillcolor{currentfill}%
\pgfsetfillopacity{0.700000}%
\pgfsetlinewidth{0.000000pt}%
\definecolor{currentstroke}{rgb}{0.000000,0.000000,0.000000}%
\pgfsetstrokecolor{currentstroke}%
\pgfsetdash{}{0pt}%
\pgfpathmoveto{\pgfqpoint{4.734478in}{2.869346in}}%
\pgfpathlineto{\pgfqpoint{4.747783in}{2.865318in}}%
\pgfpathlineto{\pgfqpoint{4.761093in}{2.861317in}}%
\pgfpathlineto{\pgfqpoint{4.774409in}{2.857344in}}%
\pgfpathlineto{\pgfqpoint{4.787731in}{2.853397in}}%
\pgfpathlineto{\pgfqpoint{4.780352in}{2.844206in}}%
\pgfpathlineto{\pgfqpoint{4.772970in}{2.835141in}}%
\pgfpathlineto{\pgfqpoint{4.765584in}{2.826197in}}%
\pgfpathlineto{\pgfqpoint{4.758196in}{2.817368in}}%
\pgfpathlineto{\pgfqpoint{4.744860in}{2.821150in}}%
\pgfpathlineto{\pgfqpoint{4.731531in}{2.824959in}}%
\pgfpathlineto{\pgfqpoint{4.718208in}{2.828795in}}%
\pgfpathlineto{\pgfqpoint{4.704890in}{2.832659in}}%
\pgfpathlineto{\pgfqpoint{4.712292in}{2.841647in}}%
\pgfpathlineto{\pgfqpoint{4.719691in}{2.850754in}}%
\pgfpathlineto{\pgfqpoint{4.727086in}{2.859985in}}%
\pgfpathlineto{\pgfqpoint{4.734478in}{2.869346in}}%
\pgfpathclose%
\pgfusepath{fill}%
\end{pgfscope}%
\begin{pgfscope}%
\pgfpathrectangle{\pgfqpoint{1.150000in}{0.150000in}}{\pgfqpoint{5.700000in}{5.700000in}}%
\pgfusepath{clip}%
\pgfsetbuttcap%
\pgfsetroundjoin%
\definecolor{currentfill}{rgb}{0.277941,0.056324,0.381191}%
\pgfsetfillcolor{currentfill}%
\pgfsetfillopacity{0.700000}%
\pgfsetlinewidth{0.000000pt}%
\definecolor{currentstroke}{rgb}{0.000000,0.000000,0.000000}%
\pgfsetstrokecolor{currentstroke}%
\pgfsetdash{}{0pt}%
\pgfpathmoveto{\pgfqpoint{4.515751in}{2.844517in}}%
\pgfpathlineto{\pgfqpoint{4.529009in}{2.840468in}}%
\pgfpathlineto{\pgfqpoint{4.542274in}{2.836447in}}%
\pgfpathlineto{\pgfqpoint{4.555545in}{2.832455in}}%
\pgfpathlineto{\pgfqpoint{4.568821in}{2.828491in}}%
\pgfpathlineto{\pgfqpoint{4.561375in}{2.819737in}}%
\pgfpathlineto{\pgfqpoint{4.553925in}{2.811079in}}%
\pgfpathlineto{\pgfqpoint{4.546470in}{2.802513in}}%
\pgfpathlineto{\pgfqpoint{4.539012in}{2.794034in}}%
\pgfpathlineto{\pgfqpoint{4.525723in}{2.797859in}}%
\pgfpathlineto{\pgfqpoint{4.512440in}{2.801712in}}%
\pgfpathlineto{\pgfqpoint{4.499162in}{2.805594in}}%
\pgfpathlineto{\pgfqpoint{4.485891in}{2.809505in}}%
\pgfpathlineto{\pgfqpoint{4.493362in}{2.818118in}}%
\pgfpathlineto{\pgfqpoint{4.500829in}{2.826821in}}%
\pgfpathlineto{\pgfqpoint{4.508292in}{2.835619in}}%
\pgfpathlineto{\pgfqpoint{4.515751in}{2.844517in}}%
\pgfpathclose%
\pgfusepath{fill}%
\end{pgfscope}%
\begin{pgfscope}%
\pgfpathrectangle{\pgfqpoint{1.150000in}{0.150000in}}{\pgfqpoint{5.700000in}{5.700000in}}%
\pgfusepath{clip}%
\pgfsetbuttcap%
\pgfsetroundjoin%
\definecolor{currentfill}{rgb}{0.277941,0.056324,0.381191}%
\pgfsetfillcolor{currentfill}%
\pgfsetfillopacity{0.700000}%
\pgfsetlinewidth{0.000000pt}%
\definecolor{currentstroke}{rgb}{0.000000,0.000000,0.000000}%
\pgfsetstrokecolor{currentstroke}%
\pgfsetdash{}{0pt}%
\pgfpathmoveto{\pgfqpoint{3.180820in}{2.844711in}}%
\pgfpathlineto{\pgfqpoint{3.193832in}{2.838665in}}%
\pgfpathlineto{\pgfqpoint{3.206849in}{2.832668in}}%
\pgfpathlineto{\pgfqpoint{3.219869in}{2.826721in}}%
\pgfpathlineto{\pgfqpoint{3.232892in}{2.820821in}}%
\pgfpathlineto{\pgfqpoint{3.224985in}{2.813260in}}%
\pgfpathlineto{\pgfqpoint{3.217071in}{2.805746in}}%
\pgfpathlineto{\pgfqpoint{3.209151in}{2.798278in}}%
\pgfpathlineto{\pgfqpoint{3.201224in}{2.790856in}}%
\pgfpathlineto{\pgfqpoint{3.188187in}{2.796760in}}%
\pgfpathlineto{\pgfqpoint{3.175154in}{2.802712in}}%
\pgfpathlineto{\pgfqpoint{3.162125in}{2.808713in}}%
\pgfpathlineto{\pgfqpoint{3.149100in}{2.814764in}}%
\pgfpathlineto{\pgfqpoint{3.157040in}{2.822176in}}%
\pgfpathlineto{\pgfqpoint{3.164973in}{2.829638in}}%
\pgfpathlineto{\pgfqpoint{3.172900in}{2.837149in}}%
\pgfpathlineto{\pgfqpoint{3.180820in}{2.844711in}}%
\pgfpathclose%
\pgfusepath{fill}%
\end{pgfscope}%
\begin{pgfscope}%
\pgfpathrectangle{\pgfqpoint{1.150000in}{0.150000in}}{\pgfqpoint{5.700000in}{5.700000in}}%
\pgfusepath{clip}%
\pgfsetbuttcap%
\pgfsetroundjoin%
\definecolor{currentfill}{rgb}{0.276022,0.044167,0.370164}%
\pgfsetfillcolor{currentfill}%
\pgfsetfillopacity{0.700000}%
\pgfsetlinewidth{0.000000pt}%
\definecolor{currentstroke}{rgb}{0.000000,0.000000,0.000000}%
\pgfsetstrokecolor{currentstroke}%
\pgfsetdash{}{0pt}%
\pgfpathmoveto{\pgfqpoint{3.316542in}{2.828393in}}%
\pgfpathlineto{\pgfqpoint{3.329572in}{2.822720in}}%
\pgfpathlineto{\pgfqpoint{3.342607in}{2.817093in}}%
\pgfpathlineto{\pgfqpoint{3.355646in}{2.811511in}}%
\pgfpathlineto{\pgfqpoint{3.368689in}{2.805974in}}%
\pgfpathlineto{\pgfqpoint{3.360831in}{2.798245in}}%
\pgfpathlineto{\pgfqpoint{3.352968in}{2.790560in}}%
\pgfpathlineto{\pgfqpoint{3.345098in}{2.782917in}}%
\pgfpathlineto{\pgfqpoint{3.337222in}{2.775316in}}%
\pgfpathlineto{\pgfqpoint{3.324167in}{2.780844in}}%
\pgfpathlineto{\pgfqpoint{3.311116in}{2.786417in}}%
\pgfpathlineto{\pgfqpoint{3.298069in}{2.792035in}}%
\pgfpathlineto{\pgfqpoint{3.285025in}{2.797699in}}%
\pgfpathlineto{\pgfqpoint{3.292914in}{2.805304in}}%
\pgfpathlineto{\pgfqpoint{3.300796in}{2.812954in}}%
\pgfpathlineto{\pgfqpoint{3.308672in}{2.820650in}}%
\pgfpathlineto{\pgfqpoint{3.316542in}{2.828393in}}%
\pgfpathclose%
\pgfusepath{fill}%
\end{pgfscope}%
\begin{pgfscope}%
\pgfpathrectangle{\pgfqpoint{1.150000in}{0.150000in}}{\pgfqpoint{5.700000in}{5.700000in}}%
\pgfusepath{clip}%
\pgfsetbuttcap%
\pgfsetroundjoin%
\definecolor{currentfill}{rgb}{0.273809,0.031497,0.358853}%
\pgfsetfillcolor{currentfill}%
\pgfsetfillopacity{0.700000}%
\pgfsetlinewidth{0.000000pt}%
\definecolor{currentstroke}{rgb}{0.000000,0.000000,0.000000}%
\pgfsetstrokecolor{currentstroke}%
\pgfsetdash{}{0pt}%
\pgfpathmoveto{\pgfqpoint{3.942488in}{2.808444in}}%
\pgfpathlineto{\pgfqpoint{3.955628in}{2.803955in}}%
\pgfpathlineto{\pgfqpoint{3.968774in}{2.799500in}}%
\pgfpathlineto{\pgfqpoint{3.981925in}{2.795080in}}%
\pgfpathlineto{\pgfqpoint{3.995081in}{2.790693in}}%
\pgfpathlineto{\pgfqpoint{3.987444in}{2.782489in}}%
\pgfpathlineto{\pgfqpoint{3.979801in}{2.774335in}}%
\pgfpathlineto{\pgfqpoint{3.972153in}{2.766228in}}%
\pgfpathlineto{\pgfqpoint{3.964499in}{2.758166in}}%
\pgfpathlineto{\pgfqpoint{3.951331in}{2.762479in}}%
\pgfpathlineto{\pgfqpoint{3.938168in}{2.766825in}}%
\pgfpathlineto{\pgfqpoint{3.925011in}{2.771206in}}%
\pgfpathlineto{\pgfqpoint{3.911859in}{2.775621in}}%
\pgfpathlineto{\pgfqpoint{3.919524in}{2.783752in}}%
\pgfpathlineto{\pgfqpoint{3.927184in}{2.791931in}}%
\pgfpathlineto{\pgfqpoint{3.934839in}{2.800161in}}%
\pgfpathlineto{\pgfqpoint{3.942488in}{2.808444in}}%
\pgfpathclose%
\pgfusepath{fill}%
\end{pgfscope}%
\begin{pgfscope}%
\pgfpathrectangle{\pgfqpoint{1.150000in}{0.150000in}}{\pgfqpoint{5.700000in}{5.700000in}}%
\pgfusepath{clip}%
\pgfsetbuttcap%
\pgfsetroundjoin%
\definecolor{currentfill}{rgb}{0.274952,0.037752,0.364543}%
\pgfsetfillcolor{currentfill}%
\pgfsetfillopacity{0.700000}%
\pgfsetlinewidth{0.000000pt}%
\definecolor{currentstroke}{rgb}{0.000000,0.000000,0.000000}%
\pgfsetstrokecolor{currentstroke}%
\pgfsetdash{}{0pt}%
\pgfpathmoveto{\pgfqpoint{3.452223in}{2.815549in}}%
\pgfpathlineto{\pgfqpoint{3.465275in}{2.810208in}}%
\pgfpathlineto{\pgfqpoint{3.478331in}{2.804910in}}%
\pgfpathlineto{\pgfqpoint{3.491392in}{2.799654in}}%
\pgfpathlineto{\pgfqpoint{3.504457in}{2.794440in}}%
\pgfpathlineto{\pgfqpoint{3.496648in}{2.786581in}}%
\pgfpathlineto{\pgfqpoint{3.488833in}{2.778763in}}%
\pgfpathlineto{\pgfqpoint{3.481012in}{2.770985in}}%
\pgfpathlineto{\pgfqpoint{3.473185in}{2.763245in}}%
\pgfpathlineto{\pgfqpoint{3.460108in}{2.768438in}}%
\pgfpathlineto{\pgfqpoint{3.447035in}{2.773672in}}%
\pgfpathlineto{\pgfqpoint{3.433966in}{2.778948in}}%
\pgfpathlineto{\pgfqpoint{3.420902in}{2.784267in}}%
\pgfpathlineto{\pgfqpoint{3.428741in}{2.792023in}}%
\pgfpathlineto{\pgfqpoint{3.436575in}{2.799821in}}%
\pgfpathlineto{\pgfqpoint{3.444402in}{2.807663in}}%
\pgfpathlineto{\pgfqpoint{3.452223in}{2.815549in}}%
\pgfpathclose%
\pgfusepath{fill}%
\end{pgfscope}%
\begin{pgfscope}%
\pgfpathrectangle{\pgfqpoint{1.150000in}{0.150000in}}{\pgfqpoint{5.700000in}{5.700000in}}%
\pgfusepath{clip}%
\pgfsetbuttcap%
\pgfsetroundjoin%
\definecolor{currentfill}{rgb}{0.276022,0.044167,0.370164}%
\pgfsetfillcolor{currentfill}%
\pgfsetfillopacity{0.700000}%
\pgfsetlinewidth{0.000000pt}%
\definecolor{currentstroke}{rgb}{0.000000,0.000000,0.000000}%
\pgfsetstrokecolor{currentstroke}%
\pgfsetdash{}{0pt}%
\pgfpathmoveto{\pgfqpoint{4.297011in}{2.823593in}}%
\pgfpathlineto{\pgfqpoint{4.310225in}{2.819452in}}%
\pgfpathlineto{\pgfqpoint{4.323445in}{2.815342in}}%
\pgfpathlineto{\pgfqpoint{4.336671in}{2.811261in}}%
\pgfpathlineto{\pgfqpoint{4.349903in}{2.807211in}}%
\pgfpathlineto{\pgfqpoint{4.342384in}{2.798755in}}%
\pgfpathlineto{\pgfqpoint{4.334862in}{2.790371in}}%
\pgfpathlineto{\pgfqpoint{4.327334in}{2.782056in}}%
\pgfpathlineto{\pgfqpoint{4.319801in}{2.773807in}}%
\pgfpathlineto{\pgfqpoint{4.306558in}{2.777745in}}%
\pgfpathlineto{\pgfqpoint{4.293320in}{2.781712in}}%
\pgfpathlineto{\pgfqpoint{4.280088in}{2.785710in}}%
\pgfpathlineto{\pgfqpoint{4.266862in}{2.789738in}}%
\pgfpathlineto{\pgfqpoint{4.274406in}{2.798095in}}%
\pgfpathlineto{\pgfqpoint{4.281946in}{2.806521in}}%
\pgfpathlineto{\pgfqpoint{4.289481in}{2.815019in}}%
\pgfpathlineto{\pgfqpoint{4.297011in}{2.823593in}}%
\pgfpathclose%
\pgfusepath{fill}%
\end{pgfscope}%
\begin{pgfscope}%
\pgfpathrectangle{\pgfqpoint{1.150000in}{0.150000in}}{\pgfqpoint{5.700000in}{5.700000in}}%
\pgfusepath{clip}%
\pgfsetbuttcap%
\pgfsetroundjoin%
\definecolor{currentfill}{rgb}{0.273809,0.031497,0.358853}%
\pgfsetfillcolor{currentfill}%
\pgfsetfillopacity{0.700000}%
\pgfsetlinewidth{0.000000pt}%
\definecolor{currentstroke}{rgb}{0.000000,0.000000,0.000000}%
\pgfsetstrokecolor{currentstroke}%
\pgfsetdash{}{0pt}%
\pgfpathmoveto{\pgfqpoint{3.587893in}{2.805730in}}%
\pgfpathlineto{\pgfqpoint{3.600969in}{2.800684in}}%
\pgfpathlineto{\pgfqpoint{3.614050in}{2.795678in}}%
\pgfpathlineto{\pgfqpoint{3.627136in}{2.790712in}}%
\pgfpathlineto{\pgfqpoint{3.640226in}{2.785785in}}%
\pgfpathlineto{\pgfqpoint{3.632464in}{2.777827in}}%
\pgfpathlineto{\pgfqpoint{3.624696in}{2.769909in}}%
\pgfpathlineto{\pgfqpoint{3.616922in}{2.762030in}}%
\pgfpathlineto{\pgfqpoint{3.609143in}{2.754188in}}%
\pgfpathlineto{\pgfqpoint{3.596041in}{2.759081in}}%
\pgfpathlineto{\pgfqpoint{3.582944in}{2.764012in}}%
\pgfpathlineto{\pgfqpoint{3.569851in}{2.768983in}}%
\pgfpathlineto{\pgfqpoint{3.556763in}{2.773994in}}%
\pgfpathlineto{\pgfqpoint{3.564554in}{2.781865in}}%
\pgfpathlineto{\pgfqpoint{3.572340in}{2.789778in}}%
\pgfpathlineto{\pgfqpoint{3.580119in}{2.797732in}}%
\pgfpathlineto{\pgfqpoint{3.587893in}{2.805730in}}%
\pgfpathclose%
\pgfusepath{fill}%
\end{pgfscope}%
\begin{pgfscope}%
\pgfpathrectangle{\pgfqpoint{1.150000in}{0.150000in}}{\pgfqpoint{5.700000in}{5.700000in}}%
\pgfusepath{clip}%
\pgfsetbuttcap%
\pgfsetroundjoin%
\definecolor{currentfill}{rgb}{0.274952,0.037752,0.364543}%
\pgfsetfillcolor{currentfill}%
\pgfsetfillopacity{0.700000}%
\pgfsetlinewidth{0.000000pt}%
\definecolor{currentstroke}{rgb}{0.000000,0.000000,0.000000}%
\pgfsetstrokecolor{currentstroke}%
\pgfsetdash{}{0pt}%
\pgfpathmoveto{\pgfqpoint{4.078211in}{2.806506in}}%
\pgfpathlineto{\pgfqpoint{4.091383in}{2.802198in}}%
\pgfpathlineto{\pgfqpoint{4.104560in}{2.797924in}}%
\pgfpathlineto{\pgfqpoint{4.117742in}{2.793681in}}%
\pgfpathlineto{\pgfqpoint{4.130930in}{2.789471in}}%
\pgfpathlineto{\pgfqpoint{4.123337in}{2.781221in}}%
\pgfpathlineto{\pgfqpoint{4.115738in}{2.773027in}}%
\pgfpathlineto{\pgfqpoint{4.108135in}{2.764886in}}%
\pgfpathlineto{\pgfqpoint{4.100526in}{2.756793in}}%
\pgfpathlineto{\pgfqpoint{4.087326in}{2.760916in}}%
\pgfpathlineto{\pgfqpoint{4.074132in}{2.765072in}}%
\pgfpathlineto{\pgfqpoint{4.060943in}{2.769260in}}%
\pgfpathlineto{\pgfqpoint{4.047760in}{2.773480in}}%
\pgfpathlineto{\pgfqpoint{4.055381in}{2.781655in}}%
\pgfpathlineto{\pgfqpoint{4.062996in}{2.789883in}}%
\pgfpathlineto{\pgfqpoint{4.070606in}{2.798165in}}%
\pgfpathlineto{\pgfqpoint{4.078211in}{2.806506in}}%
\pgfpathclose%
\pgfusepath{fill}%
\end{pgfscope}%
\begin{pgfscope}%
\pgfpathrectangle{\pgfqpoint{1.150000in}{0.150000in}}{\pgfqpoint{5.700000in}{5.700000in}}%
\pgfusepath{clip}%
\pgfsetbuttcap%
\pgfsetroundjoin%
\definecolor{currentfill}{rgb}{0.278791,0.062145,0.386592}%
\pgfsetfillcolor{currentfill}%
\pgfsetfillopacity{0.700000}%
\pgfsetlinewidth{0.000000pt}%
\definecolor{currentstroke}{rgb}{0.000000,0.000000,0.000000}%
\pgfsetstrokecolor{currentstroke}%
\pgfsetdash{}{0pt}%
\pgfpathmoveto{\pgfqpoint{4.651682in}{2.848390in}}%
\pgfpathlineto{\pgfqpoint{4.664975in}{2.844415in}}%
\pgfpathlineto{\pgfqpoint{4.678274in}{2.840469in}}%
\pgfpathlineto{\pgfqpoint{4.691579in}{2.836550in}}%
\pgfpathlineto{\pgfqpoint{4.704890in}{2.832659in}}%
\pgfpathlineto{\pgfqpoint{4.697485in}{2.823785in}}%
\pgfpathlineto{\pgfqpoint{4.690076in}{2.815019in}}%
\pgfpathlineto{\pgfqpoint{4.682664in}{2.806358in}}%
\pgfpathlineto{\pgfqpoint{4.675247in}{2.797796in}}%
\pgfpathlineto{\pgfqpoint{4.661923in}{2.801535in}}%
\pgfpathlineto{\pgfqpoint{4.648605in}{2.805302in}}%
\pgfpathlineto{\pgfqpoint{4.635292in}{2.809097in}}%
\pgfpathlineto{\pgfqpoint{4.621986in}{2.812920in}}%
\pgfpathlineto{\pgfqpoint{4.629416in}{2.821629in}}%
\pgfpathlineto{\pgfqpoint{4.636841in}{2.830440in}}%
\pgfpathlineto{\pgfqpoint{4.644263in}{2.839359in}}%
\pgfpathlineto{\pgfqpoint{4.651682in}{2.848390in}}%
\pgfpathclose%
\pgfusepath{fill}%
\end{pgfscope}%
\begin{pgfscope}%
\pgfpathrectangle{\pgfqpoint{1.150000in}{0.150000in}}{\pgfqpoint{5.700000in}{5.700000in}}%
\pgfusepath{clip}%
\pgfsetbuttcap%
\pgfsetroundjoin%
\definecolor{currentfill}{rgb}{0.273809,0.031497,0.358853}%
\pgfsetfillcolor{currentfill}%
\pgfsetfillopacity{0.700000}%
\pgfsetlinewidth{0.000000pt}%
\definecolor{currentstroke}{rgb}{0.000000,0.000000,0.000000}%
\pgfsetstrokecolor{currentstroke}%
\pgfsetdash{}{0pt}%
\pgfpathmoveto{\pgfqpoint{3.723579in}{2.798539in}}%
\pgfpathlineto{\pgfqpoint{3.736682in}{2.793755in}}%
\pgfpathlineto{\pgfqpoint{3.749790in}{2.789008in}}%
\pgfpathlineto{\pgfqpoint{3.762902in}{2.784298in}}%
\pgfpathlineto{\pgfqpoint{3.776020in}{2.779624in}}%
\pgfpathlineto{\pgfqpoint{3.768304in}{2.771593in}}%
\pgfpathlineto{\pgfqpoint{3.760582in}{2.763602in}}%
\pgfpathlineto{\pgfqpoint{3.752855in}{2.755651in}}%
\pgfpathlineto{\pgfqpoint{3.745122in}{2.747737in}}%
\pgfpathlineto{\pgfqpoint{3.731993in}{2.752362in}}%
\pgfpathlineto{\pgfqpoint{3.718868in}{2.757024in}}%
\pgfpathlineto{\pgfqpoint{3.705749in}{2.761723in}}%
\pgfpathlineto{\pgfqpoint{3.692635in}{2.766460in}}%
\pgfpathlineto{\pgfqpoint{3.700379in}{2.774417in}}%
\pgfpathlineto{\pgfqpoint{3.708118in}{2.782415in}}%
\pgfpathlineto{\pgfqpoint{3.715852in}{2.790455in}}%
\pgfpathlineto{\pgfqpoint{3.723579in}{2.798539in}}%
\pgfpathclose%
\pgfusepath{fill}%
\end{pgfscope}%
\begin{pgfscope}%
\pgfpathrectangle{\pgfqpoint{1.150000in}{0.150000in}}{\pgfqpoint{5.700000in}{5.700000in}}%
\pgfusepath{clip}%
\pgfsetbuttcap%
\pgfsetroundjoin%
\definecolor{currentfill}{rgb}{0.277018,0.050344,0.375715}%
\pgfsetfillcolor{currentfill}%
\pgfsetfillopacity{0.700000}%
\pgfsetlinewidth{0.000000pt}%
\definecolor{currentstroke}{rgb}{0.000000,0.000000,0.000000}%
\pgfsetstrokecolor{currentstroke}%
\pgfsetdash{}{0pt}%
\pgfpathmoveto{\pgfqpoint{4.432864in}{2.825439in}}%
\pgfpathlineto{\pgfqpoint{4.446112in}{2.821412in}}%
\pgfpathlineto{\pgfqpoint{4.459366in}{2.817414in}}%
\pgfpathlineto{\pgfqpoint{4.472626in}{2.813445in}}%
\pgfpathlineto{\pgfqpoint{4.485891in}{2.809505in}}%
\pgfpathlineto{\pgfqpoint{4.478416in}{2.800979in}}%
\pgfpathlineto{\pgfqpoint{4.470936in}{2.792536in}}%
\pgfpathlineto{\pgfqpoint{4.463451in}{2.784170in}}%
\pgfpathlineto{\pgfqpoint{4.455963in}{2.775879in}}%
\pgfpathlineto{\pgfqpoint{4.442685in}{2.779693in}}%
\pgfpathlineto{\pgfqpoint{4.429412in}{2.783536in}}%
\pgfpathlineto{\pgfqpoint{4.416146in}{2.787408in}}%
\pgfpathlineto{\pgfqpoint{4.402886in}{2.791309in}}%
\pgfpathlineto{\pgfqpoint{4.410387in}{2.799722in}}%
\pgfpathlineto{\pgfqpoint{4.417884in}{2.808211in}}%
\pgfpathlineto{\pgfqpoint{4.425376in}{2.816782in}}%
\pgfpathlineto{\pgfqpoint{4.432864in}{2.825439in}}%
\pgfpathclose%
\pgfusepath{fill}%
\end{pgfscope}%
\begin{pgfscope}%
\pgfpathrectangle{\pgfqpoint{1.150000in}{0.150000in}}{\pgfqpoint{5.700000in}{5.700000in}}%
\pgfusepath{clip}%
\pgfsetbuttcap%
\pgfsetroundjoin%
\definecolor{currentfill}{rgb}{0.273809,0.031497,0.358853}%
\pgfsetfillcolor{currentfill}%
\pgfsetfillopacity{0.700000}%
\pgfsetlinewidth{0.000000pt}%
\definecolor{currentstroke}{rgb}{0.000000,0.000000,0.000000}%
\pgfsetstrokecolor{currentstroke}%
\pgfsetdash{}{0pt}%
\pgfpathmoveto{\pgfqpoint{3.859303in}{2.793628in}}%
\pgfpathlineto{\pgfqpoint{3.872434in}{2.789074in}}%
\pgfpathlineto{\pgfqpoint{3.885571in}{2.784555in}}%
\pgfpathlineto{\pgfqpoint{3.898712in}{2.780071in}}%
\pgfpathlineto{\pgfqpoint{3.911859in}{2.775621in}}%
\pgfpathlineto{\pgfqpoint{3.904188in}{2.767536in}}%
\pgfpathlineto{\pgfqpoint{3.896512in}{2.759494in}}%
\pgfpathlineto{\pgfqpoint{3.888830in}{2.751492in}}%
\pgfpathlineto{\pgfqpoint{3.881143in}{2.743530in}}%
\pgfpathlineto{\pgfqpoint{3.867985in}{2.747919in}}%
\pgfpathlineto{\pgfqpoint{3.854831in}{2.752342in}}%
\pgfpathlineto{\pgfqpoint{3.841683in}{2.756800in}}%
\pgfpathlineto{\pgfqpoint{3.828540in}{2.761293in}}%
\pgfpathlineto{\pgfqpoint{3.836239in}{2.769312in}}%
\pgfpathlineto{\pgfqpoint{3.843933in}{2.777372in}}%
\pgfpathlineto{\pgfqpoint{3.851620in}{2.785477in}}%
\pgfpathlineto{\pgfqpoint{3.859303in}{2.793628in}}%
\pgfpathclose%
\pgfusepath{fill}%
\end{pgfscope}%
\begin{pgfscope}%
\pgfpathrectangle{\pgfqpoint{1.150000in}{0.150000in}}{\pgfqpoint{5.700000in}{5.700000in}}%
\pgfusepath{clip}%
\pgfsetbuttcap%
\pgfsetroundjoin%
\definecolor{currentfill}{rgb}{0.274952,0.037752,0.364543}%
\pgfsetfillcolor{currentfill}%
\pgfsetfillopacity{0.700000}%
\pgfsetlinewidth{0.000000pt}%
\definecolor{currentstroke}{rgb}{0.000000,0.000000,0.000000}%
\pgfsetstrokecolor{currentstroke}%
\pgfsetdash{}{0pt}%
\pgfpathmoveto{\pgfqpoint{4.214013in}{2.806159in}}%
\pgfpathlineto{\pgfqpoint{4.227216in}{2.802008in}}%
\pgfpathlineto{\pgfqpoint{4.240426in}{2.797887in}}%
\pgfpathlineto{\pgfqpoint{4.253641in}{2.793797in}}%
\pgfpathlineto{\pgfqpoint{4.266862in}{2.789738in}}%
\pgfpathlineto{\pgfqpoint{4.259312in}{2.781446in}}%
\pgfpathlineto{\pgfqpoint{4.251758in}{2.773216in}}%
\pgfpathlineto{\pgfqpoint{4.244198in}{2.765044in}}%
\pgfpathlineto{\pgfqpoint{4.236634in}{2.756927in}}%
\pgfpathlineto{\pgfqpoint{4.223401in}{2.760885in}}%
\pgfpathlineto{\pgfqpoint{4.210174in}{2.764875in}}%
\pgfpathlineto{\pgfqpoint{4.196953in}{2.768896in}}%
\pgfpathlineto{\pgfqpoint{4.183737in}{2.772948in}}%
\pgfpathlineto{\pgfqpoint{4.191314in}{2.781160in}}%
\pgfpathlineto{\pgfqpoint{4.198885in}{2.789431in}}%
\pgfpathlineto{\pgfqpoint{4.206451in}{2.797763in}}%
\pgfpathlineto{\pgfqpoint{4.214013in}{2.806159in}}%
\pgfpathclose%
\pgfusepath{fill}%
\end{pgfscope}%
\begin{pgfscope}%
\pgfpathrectangle{\pgfqpoint{1.150000in}{0.150000in}}{\pgfqpoint{5.700000in}{5.700000in}}%
\pgfusepath{clip}%
\pgfsetbuttcap%
\pgfsetroundjoin%
\definecolor{currentfill}{rgb}{0.277018,0.050344,0.375715}%
\pgfsetfillcolor{currentfill}%
\pgfsetfillopacity{0.700000}%
\pgfsetlinewidth{0.000000pt}%
\definecolor{currentstroke}{rgb}{0.000000,0.000000,0.000000}%
\pgfsetstrokecolor{currentstroke}%
\pgfsetdash{}{0pt}%
\pgfpathmoveto{\pgfqpoint{3.232892in}{2.820821in}}%
\pgfpathlineto{\pgfqpoint{3.245920in}{2.814970in}}%
\pgfpathlineto{\pgfqpoint{3.258951in}{2.809166in}}%
\pgfpathlineto{\pgfqpoint{3.271986in}{2.803410in}}%
\pgfpathlineto{\pgfqpoint{3.285025in}{2.797699in}}%
\pgfpathlineto{\pgfqpoint{3.277131in}{2.790139in}}%
\pgfpathlineto{\pgfqpoint{3.269229in}{2.782622in}}%
\pgfpathlineto{\pgfqpoint{3.261322in}{2.775148in}}%
\pgfpathlineto{\pgfqpoint{3.253408in}{2.767717in}}%
\pgfpathlineto{\pgfqpoint{3.240356in}{2.773432in}}%
\pgfpathlineto{\pgfqpoint{3.227308in}{2.779193in}}%
\pgfpathlineto{\pgfqpoint{3.214264in}{2.785001in}}%
\pgfpathlineto{\pgfqpoint{3.201224in}{2.790856in}}%
\pgfpathlineto{\pgfqpoint{3.209151in}{2.798278in}}%
\pgfpathlineto{\pgfqpoint{3.217071in}{2.805746in}}%
\pgfpathlineto{\pgfqpoint{3.224985in}{2.813260in}}%
\pgfpathlineto{\pgfqpoint{3.232892in}{2.820821in}}%
\pgfpathclose%
\pgfusepath{fill}%
\end{pgfscope}%
\begin{pgfscope}%
\pgfpathrectangle{\pgfqpoint{1.150000in}{0.150000in}}{\pgfqpoint{5.700000in}{5.700000in}}%
\pgfusepath{clip}%
\pgfsetbuttcap%
\pgfsetroundjoin%
\definecolor{currentfill}{rgb}{0.274952,0.037752,0.364543}%
\pgfsetfillcolor{currentfill}%
\pgfsetfillopacity{0.700000}%
\pgfsetlinewidth{0.000000pt}%
\definecolor{currentstroke}{rgb}{0.000000,0.000000,0.000000}%
\pgfsetstrokecolor{currentstroke}%
\pgfsetdash{}{0pt}%
\pgfpathmoveto{\pgfqpoint{3.368689in}{2.805974in}}%
\pgfpathlineto{\pgfqpoint{3.381736in}{2.800482in}}%
\pgfpathlineto{\pgfqpoint{3.394787in}{2.795033in}}%
\pgfpathlineto{\pgfqpoint{3.407843in}{2.789628in}}%
\pgfpathlineto{\pgfqpoint{3.420902in}{2.784267in}}%
\pgfpathlineto{\pgfqpoint{3.413057in}{2.776552in}}%
\pgfpathlineto{\pgfqpoint{3.405206in}{2.768877in}}%
\pgfpathlineto{\pgfqpoint{3.397348in}{2.761241in}}%
\pgfpathlineto{\pgfqpoint{3.389485in}{2.753643in}}%
\pgfpathlineto{\pgfqpoint{3.376413in}{2.758996in}}%
\pgfpathlineto{\pgfqpoint{3.363345in}{2.764392in}}%
\pgfpathlineto{\pgfqpoint{3.350281in}{2.769832in}}%
\pgfpathlineto{\pgfqpoint{3.337222in}{2.775316in}}%
\pgfpathlineto{\pgfqpoint{3.345098in}{2.782917in}}%
\pgfpathlineto{\pgfqpoint{3.352968in}{2.790560in}}%
\pgfpathlineto{\pgfqpoint{3.360831in}{2.798245in}}%
\pgfpathlineto{\pgfqpoint{3.368689in}{2.805974in}}%
\pgfpathclose%
\pgfusepath{fill}%
\end{pgfscope}%
\begin{pgfscope}%
\pgfpathrectangle{\pgfqpoint{1.150000in}{0.150000in}}{\pgfqpoint{5.700000in}{5.700000in}}%
\pgfusepath{clip}%
\pgfsetbuttcap%
\pgfsetroundjoin%
\definecolor{currentfill}{rgb}{0.278791,0.062145,0.386592}%
\pgfsetfillcolor{currentfill}%
\pgfsetfillopacity{0.700000}%
\pgfsetlinewidth{0.000000pt}%
\definecolor{currentstroke}{rgb}{0.000000,0.000000,0.000000}%
\pgfsetstrokecolor{currentstroke}%
\pgfsetdash{}{0pt}%
\pgfpathmoveto{\pgfqpoint{3.097033in}{2.839470in}}%
\pgfpathlineto{\pgfqpoint{3.110045in}{2.833217in}}%
\pgfpathlineto{\pgfqpoint{3.123060in}{2.827015in}}%
\pgfpathlineto{\pgfqpoint{3.136078in}{2.820864in}}%
\pgfpathlineto{\pgfqpoint{3.149100in}{2.814764in}}%
\pgfpathlineto{\pgfqpoint{3.141153in}{2.807400in}}%
\pgfpathlineto{\pgfqpoint{3.133200in}{2.800086in}}%
\pgfpathlineto{\pgfqpoint{3.125240in}{2.792820in}}%
\pgfpathlineto{\pgfqpoint{3.117274in}{2.785604in}}%
\pgfpathlineto{\pgfqpoint{3.104239in}{2.791721in}}%
\pgfpathlineto{\pgfqpoint{3.091207in}{2.797890in}}%
\pgfpathlineto{\pgfqpoint{3.078179in}{2.804109in}}%
\pgfpathlineto{\pgfqpoint{3.065154in}{2.810380in}}%
\pgfpathlineto{\pgfqpoint{3.073134in}{2.817574in}}%
\pgfpathlineto{\pgfqpoint{3.081107in}{2.824821in}}%
\pgfpathlineto{\pgfqpoint{3.089074in}{2.832119in}}%
\pgfpathlineto{\pgfqpoint{3.097033in}{2.839470in}}%
\pgfpathclose%
\pgfusepath{fill}%
\end{pgfscope}%
\begin{pgfscope}%
\pgfpathrectangle{\pgfqpoint{1.150000in}{0.150000in}}{\pgfqpoint{5.700000in}{5.700000in}}%
\pgfusepath{clip}%
\pgfsetbuttcap%
\pgfsetroundjoin%
\definecolor{currentfill}{rgb}{0.279566,0.067836,0.391917}%
\pgfsetfillcolor{currentfill}%
\pgfsetfillopacity{0.700000}%
\pgfsetlinewidth{0.000000pt}%
\definecolor{currentstroke}{rgb}{0.000000,0.000000,0.000000}%
\pgfsetstrokecolor{currentstroke}%
\pgfsetdash{}{0pt}%
\pgfpathmoveto{\pgfqpoint{4.787731in}{2.853397in}}%
\pgfpathlineto{\pgfqpoint{4.801059in}{2.849478in}}%
\pgfpathlineto{\pgfqpoint{4.814393in}{2.845586in}}%
\pgfpathlineto{\pgfqpoint{4.827734in}{2.841720in}}%
\pgfpathlineto{\pgfqpoint{4.841080in}{2.837882in}}%
\pgfpathlineto{\pgfqpoint{4.833714in}{2.828860in}}%
\pgfpathlineto{\pgfqpoint{4.826346in}{2.819961in}}%
\pgfpathlineto{\pgfqpoint{4.818974in}{2.811179in}}%
\pgfpathlineto{\pgfqpoint{4.811599in}{2.802510in}}%
\pgfpathlineto{\pgfqpoint{4.798239in}{2.806184in}}%
\pgfpathlineto{\pgfqpoint{4.784885in}{2.809885in}}%
\pgfpathlineto{\pgfqpoint{4.771538in}{2.813613in}}%
\pgfpathlineto{\pgfqpoint{4.758196in}{2.817368in}}%
\pgfpathlineto{\pgfqpoint{4.765584in}{2.826197in}}%
\pgfpathlineto{\pgfqpoint{4.772970in}{2.835141in}}%
\pgfpathlineto{\pgfqpoint{4.780352in}{2.844206in}}%
\pgfpathlineto{\pgfqpoint{4.787731in}{2.853397in}}%
\pgfpathclose%
\pgfusepath{fill}%
\end{pgfscope}%
\begin{pgfscope}%
\pgfpathrectangle{\pgfqpoint{1.150000in}{0.150000in}}{\pgfqpoint{5.700000in}{5.700000in}}%
\pgfusepath{clip}%
\pgfsetbuttcap%
\pgfsetroundjoin%
\definecolor{currentfill}{rgb}{0.277941,0.056324,0.381191}%
\pgfsetfillcolor{currentfill}%
\pgfsetfillopacity{0.700000}%
\pgfsetlinewidth{0.000000pt}%
\definecolor{currentstroke}{rgb}{0.000000,0.000000,0.000000}%
\pgfsetstrokecolor{currentstroke}%
\pgfsetdash{}{0pt}%
\pgfpathmoveto{\pgfqpoint{4.568821in}{2.828491in}}%
\pgfpathlineto{\pgfqpoint{4.582103in}{2.824556in}}%
\pgfpathlineto{\pgfqpoint{4.595392in}{2.820649in}}%
\pgfpathlineto{\pgfqpoint{4.608686in}{2.816771in}}%
\pgfpathlineto{\pgfqpoint{4.621986in}{2.812920in}}%
\pgfpathlineto{\pgfqpoint{4.614553in}{2.804309in}}%
\pgfpathlineto{\pgfqpoint{4.607116in}{2.795791in}}%
\pgfpathlineto{\pgfqpoint{4.599674in}{2.787362in}}%
\pgfpathlineto{\pgfqpoint{4.592229in}{2.779018in}}%
\pgfpathlineto{\pgfqpoint{4.578915in}{2.782730in}}%
\pgfpathlineto{\pgfqpoint{4.565608in}{2.786470in}}%
\pgfpathlineto{\pgfqpoint{4.552307in}{2.790238in}}%
\pgfpathlineto{\pgfqpoint{4.539012in}{2.794034in}}%
\pgfpathlineto{\pgfqpoint{4.546470in}{2.802513in}}%
\pgfpathlineto{\pgfqpoint{4.553925in}{2.811079in}}%
\pgfpathlineto{\pgfqpoint{4.561375in}{2.819737in}}%
\pgfpathlineto{\pgfqpoint{4.568821in}{2.828491in}}%
\pgfpathclose%
\pgfusepath{fill}%
\end{pgfscope}%
\begin{pgfscope}%
\pgfpathrectangle{\pgfqpoint{1.150000in}{0.150000in}}{\pgfqpoint{5.700000in}{5.700000in}}%
\pgfusepath{clip}%
\pgfsetbuttcap%
\pgfsetroundjoin%
\definecolor{currentfill}{rgb}{0.273809,0.031497,0.358853}%
\pgfsetfillcolor{currentfill}%
\pgfsetfillopacity{0.700000}%
\pgfsetlinewidth{0.000000pt}%
\definecolor{currentstroke}{rgb}{0.000000,0.000000,0.000000}%
\pgfsetstrokecolor{currentstroke}%
\pgfsetdash{}{0pt}%
\pgfpathmoveto{\pgfqpoint{3.504457in}{2.794440in}}%
\pgfpathlineto{\pgfqpoint{3.517527in}{2.789267in}}%
\pgfpathlineto{\pgfqpoint{3.530601in}{2.784135in}}%
\pgfpathlineto{\pgfqpoint{3.543680in}{2.779044in}}%
\pgfpathlineto{\pgfqpoint{3.556763in}{2.773994in}}%
\pgfpathlineto{\pgfqpoint{3.548966in}{2.766161in}}%
\pgfpathlineto{\pgfqpoint{3.541163in}{2.758367in}}%
\pgfpathlineto{\pgfqpoint{3.533354in}{2.750609in}}%
\pgfpathlineto{\pgfqpoint{3.525539in}{2.742887in}}%
\pgfpathlineto{\pgfqpoint{3.512443in}{2.747916in}}%
\pgfpathlineto{\pgfqpoint{3.499353in}{2.752985in}}%
\pgfpathlineto{\pgfqpoint{3.486266in}{2.758094in}}%
\pgfpathlineto{\pgfqpoint{3.473185in}{2.763245in}}%
\pgfpathlineto{\pgfqpoint{3.481012in}{2.770985in}}%
\pgfpathlineto{\pgfqpoint{3.488833in}{2.778763in}}%
\pgfpathlineto{\pgfqpoint{3.496648in}{2.786581in}}%
\pgfpathlineto{\pgfqpoint{3.504457in}{2.794440in}}%
\pgfpathclose%
\pgfusepath{fill}%
\end{pgfscope}%
\begin{pgfscope}%
\pgfpathrectangle{\pgfqpoint{1.150000in}{0.150000in}}{\pgfqpoint{5.700000in}{5.700000in}}%
\pgfusepath{clip}%
\pgfsetbuttcap%
\pgfsetroundjoin%
\definecolor{currentfill}{rgb}{0.273809,0.031497,0.358853}%
\pgfsetfillcolor{currentfill}%
\pgfsetfillopacity{0.700000}%
\pgfsetlinewidth{0.000000pt}%
\definecolor{currentstroke}{rgb}{0.000000,0.000000,0.000000}%
\pgfsetstrokecolor{currentstroke}%
\pgfsetdash{}{0pt}%
\pgfpathmoveto{\pgfqpoint{3.995081in}{2.790693in}}%
\pgfpathlineto{\pgfqpoint{4.008243in}{2.786340in}}%
\pgfpathlineto{\pgfqpoint{4.021410in}{2.782020in}}%
\pgfpathlineto{\pgfqpoint{4.034582in}{2.777734in}}%
\pgfpathlineto{\pgfqpoint{4.047760in}{2.773480in}}%
\pgfpathlineto{\pgfqpoint{4.040134in}{2.765355in}}%
\pgfpathlineto{\pgfqpoint{4.032503in}{2.757277in}}%
\pgfpathlineto{\pgfqpoint{4.024866in}{2.749242in}}%
\pgfpathlineto{\pgfqpoint{4.017224in}{2.741250in}}%
\pgfpathlineto{\pgfqpoint{4.004035in}{2.745429in}}%
\pgfpathlineto{\pgfqpoint{3.990851in}{2.749642in}}%
\pgfpathlineto{\pgfqpoint{3.977672in}{2.753887in}}%
\pgfpathlineto{\pgfqpoint{3.964499in}{2.758166in}}%
\pgfpathlineto{\pgfqpoint{3.972153in}{2.766228in}}%
\pgfpathlineto{\pgfqpoint{3.979801in}{2.774335in}}%
\pgfpathlineto{\pgfqpoint{3.987444in}{2.782489in}}%
\pgfpathlineto{\pgfqpoint{3.995081in}{2.790693in}}%
\pgfpathclose%
\pgfusepath{fill}%
\end{pgfscope}%
\begin{pgfscope}%
\pgfpathrectangle{\pgfqpoint{1.150000in}{0.150000in}}{\pgfqpoint{5.700000in}{5.700000in}}%
\pgfusepath{clip}%
\pgfsetbuttcap%
\pgfsetroundjoin%
\definecolor{currentfill}{rgb}{0.276022,0.044167,0.370164}%
\pgfsetfillcolor{currentfill}%
\pgfsetfillopacity{0.700000}%
\pgfsetlinewidth{0.000000pt}%
\definecolor{currentstroke}{rgb}{0.000000,0.000000,0.000000}%
\pgfsetstrokecolor{currentstroke}%
\pgfsetdash{}{0pt}%
\pgfpathmoveto{\pgfqpoint{4.349903in}{2.807211in}}%
\pgfpathlineto{\pgfqpoint{4.363140in}{2.803191in}}%
\pgfpathlineto{\pgfqpoint{4.376383in}{2.799201in}}%
\pgfpathlineto{\pgfqpoint{4.389631in}{2.795240in}}%
\pgfpathlineto{\pgfqpoint{4.402886in}{2.791309in}}%
\pgfpathlineto{\pgfqpoint{4.395380in}{2.782971in}}%
\pgfpathlineto{\pgfqpoint{4.387869in}{2.774702in}}%
\pgfpathlineto{\pgfqpoint{4.380354in}{2.766499in}}%
\pgfpathlineto{\pgfqpoint{4.372834in}{2.758358in}}%
\pgfpathlineto{\pgfqpoint{4.359567in}{2.762176in}}%
\pgfpathlineto{\pgfqpoint{4.346306in}{2.766023in}}%
\pgfpathlineto{\pgfqpoint{4.333051in}{2.769900in}}%
\pgfpathlineto{\pgfqpoint{4.319801in}{2.773807in}}%
\pgfpathlineto{\pgfqpoint{4.327334in}{2.782056in}}%
\pgfpathlineto{\pgfqpoint{4.334862in}{2.790371in}}%
\pgfpathlineto{\pgfqpoint{4.342384in}{2.798755in}}%
\pgfpathlineto{\pgfqpoint{4.349903in}{2.807211in}}%
\pgfpathclose%
\pgfusepath{fill}%
\end{pgfscope}%
\begin{pgfscope}%
\pgfpathrectangle{\pgfqpoint{1.150000in}{0.150000in}}{\pgfqpoint{5.700000in}{5.700000in}}%
\pgfusepath{clip}%
\pgfsetbuttcap%
\pgfsetroundjoin%
\definecolor{currentfill}{rgb}{0.273809,0.031497,0.358853}%
\pgfsetfillcolor{currentfill}%
\pgfsetfillopacity{0.700000}%
\pgfsetlinewidth{0.000000pt}%
\definecolor{currentstroke}{rgb}{0.000000,0.000000,0.000000}%
\pgfsetstrokecolor{currentstroke}%
\pgfsetdash{}{0pt}%
\pgfpathmoveto{\pgfqpoint{3.640226in}{2.785785in}}%
\pgfpathlineto{\pgfqpoint{3.653321in}{2.780896in}}%
\pgfpathlineto{\pgfqpoint{3.666421in}{2.776046in}}%
\pgfpathlineto{\pgfqpoint{3.679525in}{2.771234in}}%
\pgfpathlineto{\pgfqpoint{3.692635in}{2.766460in}}%
\pgfpathlineto{\pgfqpoint{3.684884in}{2.758542in}}%
\pgfpathlineto{\pgfqpoint{3.677128in}{2.750660in}}%
\pgfpathlineto{\pgfqpoint{3.669366in}{2.742815in}}%
\pgfpathlineto{\pgfqpoint{3.661599in}{2.735004in}}%
\pgfpathlineto{\pgfqpoint{3.648478in}{2.739743in}}%
\pgfpathlineto{\pgfqpoint{3.635361in}{2.744520in}}%
\pgfpathlineto{\pgfqpoint{3.622250in}{2.749335in}}%
\pgfpathlineto{\pgfqpoint{3.609143in}{2.754188in}}%
\pgfpathlineto{\pgfqpoint{3.616922in}{2.762030in}}%
\pgfpathlineto{\pgfqpoint{3.624696in}{2.769909in}}%
\pgfpathlineto{\pgfqpoint{3.632464in}{2.777827in}}%
\pgfpathlineto{\pgfqpoint{3.640226in}{2.785785in}}%
\pgfpathclose%
\pgfusepath{fill}%
\end{pgfscope}%
\begin{pgfscope}%
\pgfpathrectangle{\pgfqpoint{1.150000in}{0.150000in}}{\pgfqpoint{5.700000in}{5.700000in}}%
\pgfusepath{clip}%
\pgfsetbuttcap%
\pgfsetroundjoin%
\definecolor{currentfill}{rgb}{0.274952,0.037752,0.364543}%
\pgfsetfillcolor{currentfill}%
\pgfsetfillopacity{0.700000}%
\pgfsetlinewidth{0.000000pt}%
\definecolor{currentstroke}{rgb}{0.000000,0.000000,0.000000}%
\pgfsetstrokecolor{currentstroke}%
\pgfsetdash{}{0pt}%
\pgfpathmoveto{\pgfqpoint{4.130930in}{2.789471in}}%
\pgfpathlineto{\pgfqpoint{4.144124in}{2.785292in}}%
\pgfpathlineto{\pgfqpoint{4.157323in}{2.781146in}}%
\pgfpathlineto{\pgfqpoint{4.170527in}{2.777031in}}%
\pgfpathlineto{\pgfqpoint{4.183737in}{2.772948in}}%
\pgfpathlineto{\pgfqpoint{4.176156in}{2.764791in}}%
\pgfpathlineto{\pgfqpoint{4.168569in}{2.756685in}}%
\pgfpathlineto{\pgfqpoint{4.160977in}{2.748629in}}%
\pgfpathlineto{\pgfqpoint{4.153380in}{2.740619in}}%
\pgfpathlineto{\pgfqpoint{4.140158in}{2.744615in}}%
\pgfpathlineto{\pgfqpoint{4.126942in}{2.748642in}}%
\pgfpathlineto{\pgfqpoint{4.113731in}{2.752702in}}%
\pgfpathlineto{\pgfqpoint{4.100526in}{2.756793in}}%
\pgfpathlineto{\pgfqpoint{4.108135in}{2.764886in}}%
\pgfpathlineto{\pgfqpoint{4.115738in}{2.773027in}}%
\pgfpathlineto{\pgfqpoint{4.123337in}{2.781221in}}%
\pgfpathlineto{\pgfqpoint{4.130930in}{2.789471in}}%
\pgfpathclose%
\pgfusepath{fill}%
\end{pgfscope}%
\begin{pgfscope}%
\pgfpathrectangle{\pgfqpoint{1.150000in}{0.150000in}}{\pgfqpoint{5.700000in}{5.700000in}}%
\pgfusepath{clip}%
\pgfsetbuttcap%
\pgfsetroundjoin%
\definecolor{currentfill}{rgb}{0.273809,0.031497,0.358853}%
\pgfsetfillcolor{currentfill}%
\pgfsetfillopacity{0.700000}%
\pgfsetlinewidth{0.000000pt}%
\definecolor{currentstroke}{rgb}{0.000000,0.000000,0.000000}%
\pgfsetstrokecolor{currentstroke}%
\pgfsetdash{}{0pt}%
\pgfpathmoveto{\pgfqpoint{3.776020in}{2.779624in}}%
\pgfpathlineto{\pgfqpoint{3.789142in}{2.774987in}}%
\pgfpathlineto{\pgfqpoint{3.802270in}{2.770387in}}%
\pgfpathlineto{\pgfqpoint{3.815403in}{2.765822in}}%
\pgfpathlineto{\pgfqpoint{3.828540in}{2.761293in}}%
\pgfpathlineto{\pgfqpoint{3.820836in}{2.753315in}}%
\pgfpathlineto{\pgfqpoint{3.813126in}{2.745374in}}%
\pgfpathlineto{\pgfqpoint{3.805410in}{2.737469in}}%
\pgfpathlineto{\pgfqpoint{3.797689in}{2.729598in}}%
\pgfpathlineto{\pgfqpoint{3.784540in}{2.734079in}}%
\pgfpathlineto{\pgfqpoint{3.771395in}{2.738596in}}%
\pgfpathlineto{\pgfqpoint{3.758256in}{2.743148in}}%
\pgfpathlineto{\pgfqpoint{3.745122in}{2.747737in}}%
\pgfpathlineto{\pgfqpoint{3.752855in}{2.755651in}}%
\pgfpathlineto{\pgfqpoint{3.760582in}{2.763602in}}%
\pgfpathlineto{\pgfqpoint{3.768304in}{2.771593in}}%
\pgfpathlineto{\pgfqpoint{3.776020in}{2.779624in}}%
\pgfpathclose%
\pgfusepath{fill}%
\end{pgfscope}%
\begin{pgfscope}%
\pgfpathrectangle{\pgfqpoint{1.150000in}{0.150000in}}{\pgfqpoint{5.700000in}{5.700000in}}%
\pgfusepath{clip}%
\pgfsetbuttcap%
\pgfsetroundjoin%
\definecolor{currentfill}{rgb}{0.278791,0.062145,0.386592}%
\pgfsetfillcolor{currentfill}%
\pgfsetfillopacity{0.700000}%
\pgfsetlinewidth{0.000000pt}%
\definecolor{currentstroke}{rgb}{0.000000,0.000000,0.000000}%
\pgfsetstrokecolor{currentstroke}%
\pgfsetdash{}{0pt}%
\pgfpathmoveto{\pgfqpoint{4.704890in}{2.832659in}}%
\pgfpathlineto{\pgfqpoint{4.718208in}{2.828795in}}%
\pgfpathlineto{\pgfqpoint{4.731531in}{2.824959in}}%
\pgfpathlineto{\pgfqpoint{4.744860in}{2.821150in}}%
\pgfpathlineto{\pgfqpoint{4.758196in}{2.817368in}}%
\pgfpathlineto{\pgfqpoint{4.750804in}{2.808650in}}%
\pgfpathlineto{\pgfqpoint{4.743408in}{2.800038in}}%
\pgfpathlineto{\pgfqpoint{4.736009in}{2.791527in}}%
\pgfpathlineto{\pgfqpoint{4.728607in}{2.783112in}}%
\pgfpathlineto{\pgfqpoint{4.715258in}{2.786742in}}%
\pgfpathlineto{\pgfqpoint{4.701915in}{2.790399in}}%
\pgfpathlineto{\pgfqpoint{4.688578in}{2.794084in}}%
\pgfpathlineto{\pgfqpoint{4.675247in}{2.797796in}}%
\pgfpathlineto{\pgfqpoint{4.682664in}{2.806358in}}%
\pgfpathlineto{\pgfqpoint{4.690076in}{2.815019in}}%
\pgfpathlineto{\pgfqpoint{4.697485in}{2.823785in}}%
\pgfpathlineto{\pgfqpoint{4.704890in}{2.832659in}}%
\pgfpathclose%
\pgfusepath{fill}%
\end{pgfscope}%
\begin{pgfscope}%
\pgfpathrectangle{\pgfqpoint{1.150000in}{0.150000in}}{\pgfqpoint{5.700000in}{5.700000in}}%
\pgfusepath{clip}%
\pgfsetbuttcap%
\pgfsetroundjoin%
\definecolor{currentfill}{rgb}{0.277018,0.050344,0.375715}%
\pgfsetfillcolor{currentfill}%
\pgfsetfillopacity{0.700000}%
\pgfsetlinewidth{0.000000pt}%
\definecolor{currentstroke}{rgb}{0.000000,0.000000,0.000000}%
\pgfsetstrokecolor{currentstroke}%
\pgfsetdash{}{0pt}%
\pgfpathmoveto{\pgfqpoint{4.485891in}{2.809505in}}%
\pgfpathlineto{\pgfqpoint{4.499162in}{2.805594in}}%
\pgfpathlineto{\pgfqpoint{4.512440in}{2.801712in}}%
\pgfpathlineto{\pgfqpoint{4.525723in}{2.797859in}}%
\pgfpathlineto{\pgfqpoint{4.539012in}{2.794034in}}%
\pgfpathlineto{\pgfqpoint{4.531549in}{2.785639in}}%
\pgfpathlineto{\pgfqpoint{4.524082in}{2.777323in}}%
\pgfpathlineto{\pgfqpoint{4.516611in}{2.769082in}}%
\pgfpathlineto{\pgfqpoint{4.509135in}{2.760912in}}%
\pgfpathlineto{\pgfqpoint{4.495833in}{2.764611in}}%
\pgfpathlineto{\pgfqpoint{4.482537in}{2.768338in}}%
\pgfpathlineto{\pgfqpoint{4.469247in}{2.772094in}}%
\pgfpathlineto{\pgfqpoint{4.455963in}{2.775879in}}%
\pgfpathlineto{\pgfqpoint{4.463451in}{2.784170in}}%
\pgfpathlineto{\pgfqpoint{4.470936in}{2.792536in}}%
\pgfpathlineto{\pgfqpoint{4.478416in}{2.800979in}}%
\pgfpathlineto{\pgfqpoint{4.485891in}{2.809505in}}%
\pgfpathclose%
\pgfusepath{fill}%
\end{pgfscope}%
\begin{pgfscope}%
\pgfpathrectangle{\pgfqpoint{1.150000in}{0.150000in}}{\pgfqpoint{5.700000in}{5.700000in}}%
\pgfusepath{clip}%
\pgfsetbuttcap%
\pgfsetroundjoin%
\definecolor{currentfill}{rgb}{0.276022,0.044167,0.370164}%
\pgfsetfillcolor{currentfill}%
\pgfsetfillopacity{0.700000}%
\pgfsetlinewidth{0.000000pt}%
\definecolor{currentstroke}{rgb}{0.000000,0.000000,0.000000}%
\pgfsetstrokecolor{currentstroke}%
\pgfsetdash{}{0pt}%
\pgfpathmoveto{\pgfqpoint{3.285025in}{2.797699in}}%
\pgfpathlineto{\pgfqpoint{3.298069in}{2.792035in}}%
\pgfpathlineto{\pgfqpoint{3.311116in}{2.786417in}}%
\pgfpathlineto{\pgfqpoint{3.324167in}{2.780844in}}%
\pgfpathlineto{\pgfqpoint{3.337222in}{2.775316in}}%
\pgfpathlineto{\pgfqpoint{3.329340in}{2.767756in}}%
\pgfpathlineto{\pgfqpoint{3.321451in}{2.760236in}}%
\pgfpathlineto{\pgfqpoint{3.313557in}{2.752756in}}%
\pgfpathlineto{\pgfqpoint{3.305655in}{2.745317in}}%
\pgfpathlineto{\pgfqpoint{3.292588in}{2.750849in}}%
\pgfpathlineto{\pgfqpoint{3.279524in}{2.756426in}}%
\pgfpathlineto{\pgfqpoint{3.266464in}{2.762049in}}%
\pgfpathlineto{\pgfqpoint{3.253408in}{2.767717in}}%
\pgfpathlineto{\pgfqpoint{3.261322in}{2.775148in}}%
\pgfpathlineto{\pgfqpoint{3.269229in}{2.782622in}}%
\pgfpathlineto{\pgfqpoint{3.277131in}{2.790139in}}%
\pgfpathlineto{\pgfqpoint{3.285025in}{2.797699in}}%
\pgfpathclose%
\pgfusepath{fill}%
\end{pgfscope}%
\begin{pgfscope}%
\pgfpathrectangle{\pgfqpoint{1.150000in}{0.150000in}}{\pgfqpoint{5.700000in}{5.700000in}}%
\pgfusepath{clip}%
\pgfsetbuttcap%
\pgfsetroundjoin%
\definecolor{currentfill}{rgb}{0.273809,0.031497,0.358853}%
\pgfsetfillcolor{currentfill}%
\pgfsetfillopacity{0.700000}%
\pgfsetlinewidth{0.000000pt}%
\definecolor{currentstroke}{rgb}{0.000000,0.000000,0.000000}%
\pgfsetstrokecolor{currentstroke}%
\pgfsetdash{}{0pt}%
\pgfpathmoveto{\pgfqpoint{3.911859in}{2.775621in}}%
\pgfpathlineto{\pgfqpoint{3.925011in}{2.771206in}}%
\pgfpathlineto{\pgfqpoint{3.938168in}{2.766825in}}%
\pgfpathlineto{\pgfqpoint{3.951331in}{2.762479in}}%
\pgfpathlineto{\pgfqpoint{3.964499in}{2.758166in}}%
\pgfpathlineto{\pgfqpoint{3.956840in}{2.750147in}}%
\pgfpathlineto{\pgfqpoint{3.949175in}{2.742167in}}%
\pgfpathlineto{\pgfqpoint{3.941505in}{2.734226in}}%
\pgfpathlineto{\pgfqpoint{3.933830in}{2.726320in}}%
\pgfpathlineto{\pgfqpoint{3.920650in}{2.730571in}}%
\pgfpathlineto{\pgfqpoint{3.907476in}{2.734857in}}%
\pgfpathlineto{\pgfqpoint{3.894307in}{2.739176in}}%
\pgfpathlineto{\pgfqpoint{3.881143in}{2.743530in}}%
\pgfpathlineto{\pgfqpoint{3.888830in}{2.751492in}}%
\pgfpathlineto{\pgfqpoint{3.896512in}{2.759494in}}%
\pgfpathlineto{\pgfqpoint{3.904188in}{2.767536in}}%
\pgfpathlineto{\pgfqpoint{3.911859in}{2.775621in}}%
\pgfpathclose%
\pgfusepath{fill}%
\end{pgfscope}%
\begin{pgfscope}%
\pgfpathrectangle{\pgfqpoint{1.150000in}{0.150000in}}{\pgfqpoint{5.700000in}{5.700000in}}%
\pgfusepath{clip}%
\pgfsetbuttcap%
\pgfsetroundjoin%
\definecolor{currentfill}{rgb}{0.277018,0.050344,0.375715}%
\pgfsetfillcolor{currentfill}%
\pgfsetfillopacity{0.700000}%
\pgfsetlinewidth{0.000000pt}%
\definecolor{currentstroke}{rgb}{0.000000,0.000000,0.000000}%
\pgfsetstrokecolor{currentstroke}%
\pgfsetdash{}{0pt}%
\pgfpathmoveto{\pgfqpoint{3.149100in}{2.814764in}}%
\pgfpathlineto{\pgfqpoint{3.162125in}{2.808713in}}%
\pgfpathlineto{\pgfqpoint{3.175154in}{2.802712in}}%
\pgfpathlineto{\pgfqpoint{3.188187in}{2.796760in}}%
\pgfpathlineto{\pgfqpoint{3.201224in}{2.790856in}}%
\pgfpathlineto{\pgfqpoint{3.193290in}{2.783481in}}%
\pgfpathlineto{\pgfqpoint{3.185351in}{2.776151in}}%
\pgfpathlineto{\pgfqpoint{3.177404in}{2.768866in}}%
\pgfpathlineto{\pgfqpoint{3.169451in}{2.761627in}}%
\pgfpathlineto{\pgfqpoint{3.156401in}{2.767548in}}%
\pgfpathlineto{\pgfqpoint{3.143355in}{2.773517in}}%
\pgfpathlineto{\pgfqpoint{3.130313in}{2.779536in}}%
\pgfpathlineto{\pgfqpoint{3.117274in}{2.785604in}}%
\pgfpathlineto{\pgfqpoint{3.125240in}{2.792820in}}%
\pgfpathlineto{\pgfqpoint{3.133200in}{2.800086in}}%
\pgfpathlineto{\pgfqpoint{3.141153in}{2.807400in}}%
\pgfpathlineto{\pgfqpoint{3.149100in}{2.814764in}}%
\pgfpathclose%
\pgfusepath{fill}%
\end{pgfscope}%
\begin{pgfscope}%
\pgfpathrectangle{\pgfqpoint{1.150000in}{0.150000in}}{\pgfqpoint{5.700000in}{5.700000in}}%
\pgfusepath{clip}%
\pgfsetbuttcap%
\pgfsetroundjoin%
\definecolor{currentfill}{rgb}{0.274952,0.037752,0.364543}%
\pgfsetfillcolor{currentfill}%
\pgfsetfillopacity{0.700000}%
\pgfsetlinewidth{0.000000pt}%
\definecolor{currentstroke}{rgb}{0.000000,0.000000,0.000000}%
\pgfsetstrokecolor{currentstroke}%
\pgfsetdash{}{0pt}%
\pgfpathmoveto{\pgfqpoint{4.266862in}{2.789738in}}%
\pgfpathlineto{\pgfqpoint{4.280088in}{2.785710in}}%
\pgfpathlineto{\pgfqpoint{4.293320in}{2.781712in}}%
\pgfpathlineto{\pgfqpoint{4.306558in}{2.777745in}}%
\pgfpathlineto{\pgfqpoint{4.319801in}{2.773807in}}%
\pgfpathlineto{\pgfqpoint{4.312264in}{2.765620in}}%
\pgfpathlineto{\pgfqpoint{4.304722in}{2.757492in}}%
\pgfpathlineto{\pgfqpoint{4.297175in}{2.749418in}}%
\pgfpathlineto{\pgfqpoint{4.289623in}{2.741397in}}%
\pgfpathlineto{\pgfqpoint{4.276367in}{2.745234in}}%
\pgfpathlineto{\pgfqpoint{4.263117in}{2.749101in}}%
\pgfpathlineto{\pgfqpoint{4.249873in}{2.752998in}}%
\pgfpathlineto{\pgfqpoint{4.236634in}{2.756927in}}%
\pgfpathlineto{\pgfqpoint{4.244198in}{2.765044in}}%
\pgfpathlineto{\pgfqpoint{4.251758in}{2.773216in}}%
\pgfpathlineto{\pgfqpoint{4.259312in}{2.781446in}}%
\pgfpathlineto{\pgfqpoint{4.266862in}{2.789738in}}%
\pgfpathclose%
\pgfusepath{fill}%
\end{pgfscope}%
\begin{pgfscope}%
\pgfpathrectangle{\pgfqpoint{1.150000in}{0.150000in}}{\pgfqpoint{5.700000in}{5.700000in}}%
\pgfusepath{clip}%
\pgfsetbuttcap%
\pgfsetroundjoin%
\definecolor{currentfill}{rgb}{0.274952,0.037752,0.364543}%
\pgfsetfillcolor{currentfill}%
\pgfsetfillopacity{0.700000}%
\pgfsetlinewidth{0.000000pt}%
\definecolor{currentstroke}{rgb}{0.000000,0.000000,0.000000}%
\pgfsetstrokecolor{currentstroke}%
\pgfsetdash{}{0pt}%
\pgfpathmoveto{\pgfqpoint{3.420902in}{2.784267in}}%
\pgfpathlineto{\pgfqpoint{3.433966in}{2.778948in}}%
\pgfpathlineto{\pgfqpoint{3.447035in}{2.773672in}}%
\pgfpathlineto{\pgfqpoint{3.460108in}{2.768438in}}%
\pgfpathlineto{\pgfqpoint{3.473185in}{2.763245in}}%
\pgfpathlineto{\pgfqpoint{3.465352in}{2.755544in}}%
\pgfpathlineto{\pgfqpoint{3.457513in}{2.747879in}}%
\pgfpathlineto{\pgfqpoint{3.449668in}{2.740251in}}%
\pgfpathlineto{\pgfqpoint{3.441816in}{2.732658in}}%
\pgfpathlineto{\pgfqpoint{3.428727in}{2.737841in}}%
\pgfpathlineto{\pgfqpoint{3.415642in}{2.743066in}}%
\pgfpathlineto{\pgfqpoint{3.402561in}{2.748334in}}%
\pgfpathlineto{\pgfqpoint{3.389485in}{2.753643in}}%
\pgfpathlineto{\pgfqpoint{3.397348in}{2.761241in}}%
\pgfpathlineto{\pgfqpoint{3.405206in}{2.768877in}}%
\pgfpathlineto{\pgfqpoint{3.413057in}{2.776552in}}%
\pgfpathlineto{\pgfqpoint{3.420902in}{2.784267in}}%
\pgfpathclose%
\pgfusepath{fill}%
\end{pgfscope}%
\begin{pgfscope}%
\pgfpathrectangle{\pgfqpoint{1.150000in}{0.150000in}}{\pgfqpoint{5.700000in}{5.700000in}}%
\pgfusepath{clip}%
\pgfsetbuttcap%
\pgfsetroundjoin%
\definecolor{currentfill}{rgb}{0.273809,0.031497,0.358853}%
\pgfsetfillcolor{currentfill}%
\pgfsetfillopacity{0.700000}%
\pgfsetlinewidth{0.000000pt}%
\definecolor{currentstroke}{rgb}{0.000000,0.000000,0.000000}%
\pgfsetstrokecolor{currentstroke}%
\pgfsetdash{}{0pt}%
\pgfpathmoveto{\pgfqpoint{3.556763in}{2.773994in}}%
\pgfpathlineto{\pgfqpoint{3.569851in}{2.768983in}}%
\pgfpathlineto{\pgfqpoint{3.582944in}{2.764012in}}%
\pgfpathlineto{\pgfqpoint{3.596041in}{2.759081in}}%
\pgfpathlineto{\pgfqpoint{3.609143in}{2.754188in}}%
\pgfpathlineto{\pgfqpoint{3.601357in}{2.746383in}}%
\pgfpathlineto{\pgfqpoint{3.593566in}{2.738612in}}%
\pgfpathlineto{\pgfqpoint{3.585769in}{2.730875in}}%
\pgfpathlineto{\pgfqpoint{3.577967in}{2.723170in}}%
\pgfpathlineto{\pgfqpoint{3.564853in}{2.728040in}}%
\pgfpathlineto{\pgfqpoint{3.551743in}{2.732949in}}%
\pgfpathlineto{\pgfqpoint{3.538639in}{2.737898in}}%
\pgfpathlineto{\pgfqpoint{3.525539in}{2.742887in}}%
\pgfpathlineto{\pgfqpoint{3.533354in}{2.750609in}}%
\pgfpathlineto{\pgfqpoint{3.541163in}{2.758367in}}%
\pgfpathlineto{\pgfqpoint{3.548966in}{2.766161in}}%
\pgfpathlineto{\pgfqpoint{3.556763in}{2.773994in}}%
\pgfpathclose%
\pgfusepath{fill}%
\end{pgfscope}%
\begin{pgfscope}%
\pgfpathrectangle{\pgfqpoint{1.150000in}{0.150000in}}{\pgfqpoint{5.700000in}{5.700000in}}%
\pgfusepath{clip}%
\pgfsetbuttcap%
\pgfsetroundjoin%
\definecolor{currentfill}{rgb}{0.279566,0.067836,0.391917}%
\pgfsetfillcolor{currentfill}%
\pgfsetfillopacity{0.700000}%
\pgfsetlinewidth{0.000000pt}%
\definecolor{currentstroke}{rgb}{0.000000,0.000000,0.000000}%
\pgfsetstrokecolor{currentstroke}%
\pgfsetdash{}{0pt}%
\pgfpathmoveto{\pgfqpoint{4.841080in}{2.837882in}}%
\pgfpathlineto{\pgfqpoint{4.854432in}{2.834070in}}%
\pgfpathlineto{\pgfqpoint{4.867791in}{2.830284in}}%
\pgfpathlineto{\pgfqpoint{4.881156in}{2.826526in}}%
\pgfpathlineto{\pgfqpoint{4.894527in}{2.822793in}}%
\pgfpathlineto{\pgfqpoint{4.887176in}{2.813941in}}%
\pgfpathlineto{\pgfqpoint{4.879821in}{2.805208in}}%
\pgfpathlineto{\pgfqpoint{4.872463in}{2.796590in}}%
\pgfpathlineto{\pgfqpoint{4.865102in}{2.788081in}}%
\pgfpathlineto{\pgfqpoint{4.851717in}{2.791649in}}%
\pgfpathlineto{\pgfqpoint{4.838338in}{2.795243in}}%
\pgfpathlineto{\pgfqpoint{4.824966in}{2.798863in}}%
\pgfpathlineto{\pgfqpoint{4.811599in}{2.802510in}}%
\pgfpathlineto{\pgfqpoint{4.818974in}{2.811179in}}%
\pgfpathlineto{\pgfqpoint{4.826346in}{2.819961in}}%
\pgfpathlineto{\pgfqpoint{4.833714in}{2.828860in}}%
\pgfpathlineto{\pgfqpoint{4.841080in}{2.837882in}}%
\pgfpathclose%
\pgfusepath{fill}%
\end{pgfscope}%
\begin{pgfscope}%
\pgfpathrectangle{\pgfqpoint{1.150000in}{0.150000in}}{\pgfqpoint{5.700000in}{5.700000in}}%
\pgfusepath{clip}%
\pgfsetbuttcap%
\pgfsetroundjoin%
\definecolor{currentfill}{rgb}{0.277941,0.056324,0.381191}%
\pgfsetfillcolor{currentfill}%
\pgfsetfillopacity{0.700000}%
\pgfsetlinewidth{0.000000pt}%
\definecolor{currentstroke}{rgb}{0.000000,0.000000,0.000000}%
\pgfsetstrokecolor{currentstroke}%
\pgfsetdash{}{0pt}%
\pgfpathmoveto{\pgfqpoint{4.621986in}{2.812920in}}%
\pgfpathlineto{\pgfqpoint{4.635292in}{2.809097in}}%
\pgfpathlineto{\pgfqpoint{4.648605in}{2.805302in}}%
\pgfpathlineto{\pgfqpoint{4.661923in}{2.801535in}}%
\pgfpathlineto{\pgfqpoint{4.675247in}{2.797796in}}%
\pgfpathlineto{\pgfqpoint{4.667827in}{2.789328in}}%
\pgfpathlineto{\pgfqpoint{4.660403in}{2.780951in}}%
\pgfpathlineto{\pgfqpoint{4.652975in}{2.772660in}}%
\pgfpathlineto{\pgfqpoint{4.645543in}{2.764450in}}%
\pgfpathlineto{\pgfqpoint{4.632205in}{2.768050in}}%
\pgfpathlineto{\pgfqpoint{4.618873in}{2.771678in}}%
\pgfpathlineto{\pgfqpoint{4.605548in}{2.775334in}}%
\pgfpathlineto{\pgfqpoint{4.592229in}{2.779018in}}%
\pgfpathlineto{\pgfqpoint{4.599674in}{2.787362in}}%
\pgfpathlineto{\pgfqpoint{4.607116in}{2.795791in}}%
\pgfpathlineto{\pgfqpoint{4.614553in}{2.804309in}}%
\pgfpathlineto{\pgfqpoint{4.621986in}{2.812920in}}%
\pgfpathclose%
\pgfusepath{fill}%
\end{pgfscope}%
\begin{pgfscope}%
\pgfpathrectangle{\pgfqpoint{1.150000in}{0.150000in}}{\pgfqpoint{5.700000in}{5.700000in}}%
\pgfusepath{clip}%
\pgfsetbuttcap%
\pgfsetroundjoin%
\definecolor{currentfill}{rgb}{0.273809,0.031497,0.358853}%
\pgfsetfillcolor{currentfill}%
\pgfsetfillopacity{0.700000}%
\pgfsetlinewidth{0.000000pt}%
\definecolor{currentstroke}{rgb}{0.000000,0.000000,0.000000}%
\pgfsetstrokecolor{currentstroke}%
\pgfsetdash{}{0pt}%
\pgfpathmoveto{\pgfqpoint{4.047760in}{2.773480in}}%
\pgfpathlineto{\pgfqpoint{4.060943in}{2.769260in}}%
\pgfpathlineto{\pgfqpoint{4.074132in}{2.765072in}}%
\pgfpathlineto{\pgfqpoint{4.087326in}{2.760916in}}%
\pgfpathlineto{\pgfqpoint{4.100526in}{2.756793in}}%
\pgfpathlineto{\pgfqpoint{4.092912in}{2.748747in}}%
\pgfpathlineto{\pgfqpoint{4.085292in}{2.740744in}}%
\pgfpathlineto{\pgfqpoint{4.077668in}{2.732782in}}%
\pgfpathlineto{\pgfqpoint{4.070038in}{2.724859in}}%
\pgfpathlineto{\pgfqpoint{4.056826in}{2.728908in}}%
\pgfpathlineto{\pgfqpoint{4.043620in}{2.732990in}}%
\pgfpathlineto{\pgfqpoint{4.030420in}{2.737104in}}%
\pgfpathlineto{\pgfqpoint{4.017224in}{2.741250in}}%
\pgfpathlineto{\pgfqpoint{4.024866in}{2.749242in}}%
\pgfpathlineto{\pgfqpoint{4.032503in}{2.757277in}}%
\pgfpathlineto{\pgfqpoint{4.040134in}{2.765355in}}%
\pgfpathlineto{\pgfqpoint{4.047760in}{2.773480in}}%
\pgfpathclose%
\pgfusepath{fill}%
\end{pgfscope}%
\begin{pgfscope}%
\pgfpathrectangle{\pgfqpoint{1.150000in}{0.150000in}}{\pgfqpoint{5.700000in}{5.700000in}}%
\pgfusepath{clip}%
\pgfsetbuttcap%
\pgfsetroundjoin%
\definecolor{currentfill}{rgb}{0.272594,0.025563,0.353093}%
\pgfsetfillcolor{currentfill}%
\pgfsetfillopacity{0.700000}%
\pgfsetlinewidth{0.000000pt}%
\definecolor{currentstroke}{rgb}{0.000000,0.000000,0.000000}%
\pgfsetstrokecolor{currentstroke}%
\pgfsetdash{}{0pt}%
\pgfpathmoveto{\pgfqpoint{3.692635in}{2.766460in}}%
\pgfpathlineto{\pgfqpoint{3.705749in}{2.761723in}}%
\pgfpathlineto{\pgfqpoint{3.718868in}{2.757024in}}%
\pgfpathlineto{\pgfqpoint{3.731993in}{2.752362in}}%
\pgfpathlineto{\pgfqpoint{3.745122in}{2.747737in}}%
\pgfpathlineto{\pgfqpoint{3.737383in}{2.739859in}}%
\pgfpathlineto{\pgfqpoint{3.729639in}{2.732014in}}%
\pgfpathlineto{\pgfqpoint{3.721889in}{2.724202in}}%
\pgfpathlineto{\pgfqpoint{3.714133in}{2.716421in}}%
\pgfpathlineto{\pgfqpoint{3.700992in}{2.721011in}}%
\pgfpathlineto{\pgfqpoint{3.687856in}{2.725638in}}%
\pgfpathlineto{\pgfqpoint{3.674725in}{2.730302in}}%
\pgfpathlineto{\pgfqpoint{3.661599in}{2.735004in}}%
\pgfpathlineto{\pgfqpoint{3.669366in}{2.742815in}}%
\pgfpathlineto{\pgfqpoint{3.677128in}{2.750660in}}%
\pgfpathlineto{\pgfqpoint{3.684884in}{2.758542in}}%
\pgfpathlineto{\pgfqpoint{3.692635in}{2.766460in}}%
\pgfpathclose%
\pgfusepath{fill}%
\end{pgfscope}%
\begin{pgfscope}%
\pgfpathrectangle{\pgfqpoint{1.150000in}{0.150000in}}{\pgfqpoint{5.700000in}{5.700000in}}%
\pgfusepath{clip}%
\pgfsetbuttcap%
\pgfsetroundjoin%
\definecolor{currentfill}{rgb}{0.276022,0.044167,0.370164}%
\pgfsetfillcolor{currentfill}%
\pgfsetfillopacity{0.700000}%
\pgfsetlinewidth{0.000000pt}%
\definecolor{currentstroke}{rgb}{0.000000,0.000000,0.000000}%
\pgfsetstrokecolor{currentstroke}%
\pgfsetdash{}{0pt}%
\pgfpathmoveto{\pgfqpoint{4.402886in}{2.791309in}}%
\pgfpathlineto{\pgfqpoint{4.416146in}{2.787408in}}%
\pgfpathlineto{\pgfqpoint{4.429412in}{2.783536in}}%
\pgfpathlineto{\pgfqpoint{4.442685in}{2.779693in}}%
\pgfpathlineto{\pgfqpoint{4.455963in}{2.775879in}}%
\pgfpathlineto{\pgfqpoint{4.448469in}{2.767658in}}%
\pgfpathlineto{\pgfqpoint{4.440971in}{2.759504in}}%
\pgfpathlineto{\pgfqpoint{4.433469in}{2.751413in}}%
\pgfpathlineto{\pgfqpoint{4.425961in}{2.743380in}}%
\pgfpathlineto{\pgfqpoint{4.412670in}{2.747081in}}%
\pgfpathlineto{\pgfqpoint{4.399386in}{2.750811in}}%
\pgfpathlineto{\pgfqpoint{4.386107in}{2.754569in}}%
\pgfpathlineto{\pgfqpoint{4.372834in}{2.758358in}}%
\pgfpathlineto{\pgfqpoint{4.380354in}{2.766499in}}%
\pgfpathlineto{\pgfqpoint{4.387869in}{2.774702in}}%
\pgfpathlineto{\pgfqpoint{4.395380in}{2.782971in}}%
\pgfpathlineto{\pgfqpoint{4.402886in}{2.791309in}}%
\pgfpathclose%
\pgfusepath{fill}%
\end{pgfscope}%
\begin{pgfscope}%
\pgfpathrectangle{\pgfqpoint{1.150000in}{0.150000in}}{\pgfqpoint{5.700000in}{5.700000in}}%
\pgfusepath{clip}%
\pgfsetbuttcap%
\pgfsetroundjoin%
\definecolor{currentfill}{rgb}{0.272594,0.025563,0.353093}%
\pgfsetfillcolor{currentfill}%
\pgfsetfillopacity{0.700000}%
\pgfsetlinewidth{0.000000pt}%
\definecolor{currentstroke}{rgb}{0.000000,0.000000,0.000000}%
\pgfsetstrokecolor{currentstroke}%
\pgfsetdash{}{0pt}%
\pgfpathmoveto{\pgfqpoint{3.828540in}{2.761293in}}%
\pgfpathlineto{\pgfqpoint{3.841683in}{2.756800in}}%
\pgfpathlineto{\pgfqpoint{3.854831in}{2.752342in}}%
\pgfpathlineto{\pgfqpoint{3.867985in}{2.747919in}}%
\pgfpathlineto{\pgfqpoint{3.881143in}{2.743530in}}%
\pgfpathlineto{\pgfqpoint{3.873450in}{2.735605in}}%
\pgfpathlineto{\pgfqpoint{3.865752in}{2.727714in}}%
\pgfpathlineto{\pgfqpoint{3.858048in}{2.719855in}}%
\pgfpathlineto{\pgfqpoint{3.850339in}{2.712028in}}%
\pgfpathlineto{\pgfqpoint{3.837169in}{2.716368in}}%
\pgfpathlineto{\pgfqpoint{3.824004in}{2.720743in}}%
\pgfpathlineto{\pgfqpoint{3.810844in}{2.725153in}}%
\pgfpathlineto{\pgfqpoint{3.797689in}{2.729598in}}%
\pgfpathlineto{\pgfqpoint{3.805410in}{2.737469in}}%
\pgfpathlineto{\pgfqpoint{3.813126in}{2.745374in}}%
\pgfpathlineto{\pgfqpoint{3.820836in}{2.753315in}}%
\pgfpathlineto{\pgfqpoint{3.828540in}{2.761293in}}%
\pgfpathclose%
\pgfusepath{fill}%
\end{pgfscope}%
\begin{pgfscope}%
\pgfpathrectangle{\pgfqpoint{1.150000in}{0.150000in}}{\pgfqpoint{5.700000in}{5.700000in}}%
\pgfusepath{clip}%
\pgfsetbuttcap%
\pgfsetroundjoin%
\definecolor{currentfill}{rgb}{0.274952,0.037752,0.364543}%
\pgfsetfillcolor{currentfill}%
\pgfsetfillopacity{0.700000}%
\pgfsetlinewidth{0.000000pt}%
\definecolor{currentstroke}{rgb}{0.000000,0.000000,0.000000}%
\pgfsetstrokecolor{currentstroke}%
\pgfsetdash{}{0pt}%
\pgfpathmoveto{\pgfqpoint{4.183737in}{2.772948in}}%
\pgfpathlineto{\pgfqpoint{4.196953in}{2.768896in}}%
\pgfpathlineto{\pgfqpoint{4.210174in}{2.764875in}}%
\pgfpathlineto{\pgfqpoint{4.223401in}{2.760885in}}%
\pgfpathlineto{\pgfqpoint{4.236634in}{2.756927in}}%
\pgfpathlineto{\pgfqpoint{4.229065in}{2.748861in}}%
\pgfpathlineto{\pgfqpoint{4.221490in}{2.740845in}}%
\pgfpathlineto{\pgfqpoint{4.213911in}{2.732874in}}%
\pgfpathlineto{\pgfqpoint{4.206326in}{2.724947in}}%
\pgfpathlineto{\pgfqpoint{4.193081in}{2.728818in}}%
\pgfpathlineto{\pgfqpoint{4.179842in}{2.732721in}}%
\pgfpathlineto{\pgfqpoint{4.166608in}{2.736654in}}%
\pgfpathlineto{\pgfqpoint{4.153380in}{2.740619in}}%
\pgfpathlineto{\pgfqpoint{4.160977in}{2.748629in}}%
\pgfpathlineto{\pgfqpoint{4.168569in}{2.756685in}}%
\pgfpathlineto{\pgfqpoint{4.176156in}{2.764791in}}%
\pgfpathlineto{\pgfqpoint{4.183737in}{2.772948in}}%
\pgfpathclose%
\pgfusepath{fill}%
\end{pgfscope}%
\begin{pgfscope}%
\pgfpathrectangle{\pgfqpoint{1.150000in}{0.150000in}}{\pgfqpoint{5.700000in}{5.700000in}}%
\pgfusepath{clip}%
\pgfsetbuttcap%
\pgfsetroundjoin%
\definecolor{currentfill}{rgb}{0.278791,0.062145,0.386592}%
\pgfsetfillcolor{currentfill}%
\pgfsetfillopacity{0.700000}%
\pgfsetlinewidth{0.000000pt}%
\definecolor{currentstroke}{rgb}{0.000000,0.000000,0.000000}%
\pgfsetstrokecolor{currentstroke}%
\pgfsetdash{}{0pt}%
\pgfpathmoveto{\pgfqpoint{4.758196in}{2.817368in}}%
\pgfpathlineto{\pgfqpoint{4.771538in}{2.813613in}}%
\pgfpathlineto{\pgfqpoint{4.784885in}{2.809885in}}%
\pgfpathlineto{\pgfqpoint{4.798239in}{2.806184in}}%
\pgfpathlineto{\pgfqpoint{4.811599in}{2.802510in}}%
\pgfpathlineto{\pgfqpoint{4.804221in}{2.793949in}}%
\pgfpathlineto{\pgfqpoint{4.796839in}{2.785491in}}%
\pgfpathlineto{\pgfqpoint{4.789454in}{2.777130in}}%
\pgfpathlineto{\pgfqpoint{4.782065in}{2.768863in}}%
\pgfpathlineto{\pgfqpoint{4.768691in}{2.772384in}}%
\pgfpathlineto{\pgfqpoint{4.755323in}{2.775933in}}%
\pgfpathlineto{\pgfqpoint{4.741962in}{2.779509in}}%
\pgfpathlineto{\pgfqpoint{4.728607in}{2.783112in}}%
\pgfpathlineto{\pgfqpoint{4.736009in}{2.791527in}}%
\pgfpathlineto{\pgfqpoint{4.743408in}{2.800038in}}%
\pgfpathlineto{\pgfqpoint{4.750804in}{2.808650in}}%
\pgfpathlineto{\pgfqpoint{4.758196in}{2.817368in}}%
\pgfpathclose%
\pgfusepath{fill}%
\end{pgfscope}%
\begin{pgfscope}%
\pgfpathrectangle{\pgfqpoint{1.150000in}{0.150000in}}{\pgfqpoint{5.700000in}{5.700000in}}%
\pgfusepath{clip}%
\pgfsetbuttcap%
\pgfsetroundjoin%
\definecolor{currentfill}{rgb}{0.276022,0.044167,0.370164}%
\pgfsetfillcolor{currentfill}%
\pgfsetfillopacity{0.700000}%
\pgfsetlinewidth{0.000000pt}%
\definecolor{currentstroke}{rgb}{0.000000,0.000000,0.000000}%
\pgfsetstrokecolor{currentstroke}%
\pgfsetdash{}{0pt}%
\pgfpathmoveto{\pgfqpoint{3.201224in}{2.790856in}}%
\pgfpathlineto{\pgfqpoint{3.214264in}{2.785001in}}%
\pgfpathlineto{\pgfqpoint{3.227308in}{2.779193in}}%
\pgfpathlineto{\pgfqpoint{3.240356in}{2.773432in}}%
\pgfpathlineto{\pgfqpoint{3.253408in}{2.767717in}}%
\pgfpathlineto{\pgfqpoint{3.245488in}{2.760329in}}%
\pgfpathlineto{\pgfqpoint{3.237561in}{2.752983in}}%
\pgfpathlineto{\pgfqpoint{3.229628in}{2.745680in}}%
\pgfpathlineto{\pgfqpoint{3.221688in}{2.738418in}}%
\pgfpathlineto{\pgfqpoint{3.208623in}{2.744150in}}%
\pgfpathlineto{\pgfqpoint{3.195562in}{2.749928in}}%
\pgfpathlineto{\pgfqpoint{3.182504in}{2.755754in}}%
\pgfpathlineto{\pgfqpoint{3.169451in}{2.761627in}}%
\pgfpathlineto{\pgfqpoint{3.177404in}{2.768866in}}%
\pgfpathlineto{\pgfqpoint{3.185351in}{2.776151in}}%
\pgfpathlineto{\pgfqpoint{3.193290in}{2.783481in}}%
\pgfpathlineto{\pgfqpoint{3.201224in}{2.790856in}}%
\pgfpathclose%
\pgfusepath{fill}%
\end{pgfscope}%
\begin{pgfscope}%
\pgfpathrectangle{\pgfqpoint{1.150000in}{0.150000in}}{\pgfqpoint{5.700000in}{5.700000in}}%
\pgfusepath{clip}%
\pgfsetbuttcap%
\pgfsetroundjoin%
\definecolor{currentfill}{rgb}{0.274952,0.037752,0.364543}%
\pgfsetfillcolor{currentfill}%
\pgfsetfillopacity{0.700000}%
\pgfsetlinewidth{0.000000pt}%
\definecolor{currentstroke}{rgb}{0.000000,0.000000,0.000000}%
\pgfsetstrokecolor{currentstroke}%
\pgfsetdash{}{0pt}%
\pgfpathmoveto{\pgfqpoint{3.337222in}{2.775316in}}%
\pgfpathlineto{\pgfqpoint{3.350281in}{2.769832in}}%
\pgfpathlineto{\pgfqpoint{3.363345in}{2.764392in}}%
\pgfpathlineto{\pgfqpoint{3.376413in}{2.758996in}}%
\pgfpathlineto{\pgfqpoint{3.389485in}{2.753643in}}%
\pgfpathlineto{\pgfqpoint{3.381615in}{2.746084in}}%
\pgfpathlineto{\pgfqpoint{3.373739in}{2.738562in}}%
\pgfpathlineto{\pgfqpoint{3.365857in}{2.731077in}}%
\pgfpathlineto{\pgfqpoint{3.357969in}{2.723628in}}%
\pgfpathlineto{\pgfqpoint{3.344884in}{2.728985in}}%
\pgfpathlineto{\pgfqpoint{3.331804in}{2.734385in}}%
\pgfpathlineto{\pgfqpoint{3.318728in}{2.739829in}}%
\pgfpathlineto{\pgfqpoint{3.305655in}{2.745317in}}%
\pgfpathlineto{\pgfqpoint{3.313557in}{2.752756in}}%
\pgfpathlineto{\pgfqpoint{3.321451in}{2.760236in}}%
\pgfpathlineto{\pgfqpoint{3.329340in}{2.767756in}}%
\pgfpathlineto{\pgfqpoint{3.337222in}{2.775316in}}%
\pgfpathclose%
\pgfusepath{fill}%
\end{pgfscope}%
\begin{pgfscope}%
\pgfpathrectangle{\pgfqpoint{1.150000in}{0.150000in}}{\pgfqpoint{5.700000in}{5.700000in}}%
\pgfusepath{clip}%
\pgfsetbuttcap%
\pgfsetroundjoin%
\definecolor{currentfill}{rgb}{0.277018,0.050344,0.375715}%
\pgfsetfillcolor{currentfill}%
\pgfsetfillopacity{0.700000}%
\pgfsetlinewidth{0.000000pt}%
\definecolor{currentstroke}{rgb}{0.000000,0.000000,0.000000}%
\pgfsetstrokecolor{currentstroke}%
\pgfsetdash{}{0pt}%
\pgfpathmoveto{\pgfqpoint{4.539012in}{2.794034in}}%
\pgfpathlineto{\pgfqpoint{4.552307in}{2.790238in}}%
\pgfpathlineto{\pgfqpoint{4.565608in}{2.786470in}}%
\pgfpathlineto{\pgfqpoint{4.578915in}{2.782730in}}%
\pgfpathlineto{\pgfqpoint{4.592229in}{2.779018in}}%
\pgfpathlineto{\pgfqpoint{4.584779in}{2.770754in}}%
\pgfpathlineto{\pgfqpoint{4.577325in}{2.762566in}}%
\pgfpathlineto{\pgfqpoint{4.569866in}{2.754449in}}%
\pgfpathlineto{\pgfqpoint{4.562403in}{2.746401in}}%
\pgfpathlineto{\pgfqpoint{4.549077in}{2.749987in}}%
\pgfpathlineto{\pgfqpoint{4.535757in}{2.753600in}}%
\pgfpathlineto{\pgfqpoint{4.522443in}{2.757242in}}%
\pgfpathlineto{\pgfqpoint{4.509135in}{2.760912in}}%
\pgfpathlineto{\pgfqpoint{4.516611in}{2.769082in}}%
\pgfpathlineto{\pgfqpoint{4.524082in}{2.777323in}}%
\pgfpathlineto{\pgfqpoint{4.531549in}{2.785639in}}%
\pgfpathlineto{\pgfqpoint{4.539012in}{2.794034in}}%
\pgfpathclose%
\pgfusepath{fill}%
\end{pgfscope}%
\begin{pgfscope}%
\pgfpathrectangle{\pgfqpoint{1.150000in}{0.150000in}}{\pgfqpoint{5.700000in}{5.700000in}}%
\pgfusepath{clip}%
\pgfsetbuttcap%
\pgfsetroundjoin%
\definecolor{currentfill}{rgb}{0.277941,0.056324,0.381191}%
\pgfsetfillcolor{currentfill}%
\pgfsetfillopacity{0.700000}%
\pgfsetlinewidth{0.000000pt}%
\definecolor{currentstroke}{rgb}{0.000000,0.000000,0.000000}%
\pgfsetstrokecolor{currentstroke}%
\pgfsetdash{}{0pt}%
\pgfpathmoveto{\pgfqpoint{3.065154in}{2.810380in}}%
\pgfpathlineto{\pgfqpoint{3.078179in}{2.804109in}}%
\pgfpathlineto{\pgfqpoint{3.091207in}{2.797890in}}%
\pgfpathlineto{\pgfqpoint{3.104239in}{2.791721in}}%
\pgfpathlineto{\pgfqpoint{3.117274in}{2.785604in}}%
\pgfpathlineto{\pgfqpoint{3.109301in}{2.778435in}}%
\pgfpathlineto{\pgfqpoint{3.101321in}{2.771316in}}%
\pgfpathlineto{\pgfqpoint{3.093334in}{2.764246in}}%
\pgfpathlineto{\pgfqpoint{3.085340in}{2.757226in}}%
\pgfpathlineto{\pgfqpoint{3.072291in}{2.763374in}}%
\pgfpathlineto{\pgfqpoint{3.059245in}{2.769573in}}%
\pgfpathlineto{\pgfqpoint{3.046203in}{2.775823in}}%
\pgfpathlineto{\pgfqpoint{3.033164in}{2.782124in}}%
\pgfpathlineto{\pgfqpoint{3.041172in}{2.789109in}}%
\pgfpathlineto{\pgfqpoint{3.049173in}{2.796147in}}%
\pgfpathlineto{\pgfqpoint{3.057167in}{2.803237in}}%
\pgfpathlineto{\pgfqpoint{3.065154in}{2.810380in}}%
\pgfpathclose%
\pgfusepath{fill}%
\end{pgfscope}%
\begin{pgfscope}%
\pgfpathrectangle{\pgfqpoint{1.150000in}{0.150000in}}{\pgfqpoint{5.700000in}{5.700000in}}%
\pgfusepath{clip}%
\pgfsetbuttcap%
\pgfsetroundjoin%
\definecolor{currentfill}{rgb}{0.273809,0.031497,0.358853}%
\pgfsetfillcolor{currentfill}%
\pgfsetfillopacity{0.700000}%
\pgfsetlinewidth{0.000000pt}%
\definecolor{currentstroke}{rgb}{0.000000,0.000000,0.000000}%
\pgfsetstrokecolor{currentstroke}%
\pgfsetdash{}{0pt}%
\pgfpathmoveto{\pgfqpoint{3.473185in}{2.763245in}}%
\pgfpathlineto{\pgfqpoint{3.486266in}{2.758094in}}%
\pgfpathlineto{\pgfqpoint{3.499353in}{2.752985in}}%
\pgfpathlineto{\pgfqpoint{3.512443in}{2.747916in}}%
\pgfpathlineto{\pgfqpoint{3.525539in}{2.742887in}}%
\pgfpathlineto{\pgfqpoint{3.517718in}{2.735199in}}%
\pgfpathlineto{\pgfqpoint{3.509891in}{2.727546in}}%
\pgfpathlineto{\pgfqpoint{3.502058in}{2.719925in}}%
\pgfpathlineto{\pgfqpoint{3.494220in}{2.712336in}}%
\pgfpathlineto{\pgfqpoint{3.481112in}{2.717355in}}%
\pgfpathlineto{\pgfqpoint{3.468009in}{2.722415in}}%
\pgfpathlineto{\pgfqpoint{3.454910in}{2.727516in}}%
\pgfpathlineto{\pgfqpoint{3.441816in}{2.732658in}}%
\pgfpathlineto{\pgfqpoint{3.449668in}{2.740251in}}%
\pgfpathlineto{\pgfqpoint{3.457513in}{2.747879in}}%
\pgfpathlineto{\pgfqpoint{3.465352in}{2.755544in}}%
\pgfpathlineto{\pgfqpoint{3.473185in}{2.763245in}}%
\pgfpathclose%
\pgfusepath{fill}%
\end{pgfscope}%
\begin{pgfscope}%
\pgfpathrectangle{\pgfqpoint{1.150000in}{0.150000in}}{\pgfqpoint{5.700000in}{5.700000in}}%
\pgfusepath{clip}%
\pgfsetbuttcap%
\pgfsetroundjoin%
\definecolor{currentfill}{rgb}{0.272594,0.025563,0.353093}%
\pgfsetfillcolor{currentfill}%
\pgfsetfillopacity{0.700000}%
\pgfsetlinewidth{0.000000pt}%
\definecolor{currentstroke}{rgb}{0.000000,0.000000,0.000000}%
\pgfsetstrokecolor{currentstroke}%
\pgfsetdash{}{0pt}%
\pgfpathmoveto{\pgfqpoint{3.964499in}{2.758166in}}%
\pgfpathlineto{\pgfqpoint{3.977672in}{2.753887in}}%
\pgfpathlineto{\pgfqpoint{3.990851in}{2.749642in}}%
\pgfpathlineto{\pgfqpoint{4.004035in}{2.745429in}}%
\pgfpathlineto{\pgfqpoint{4.017224in}{2.741250in}}%
\pgfpathlineto{\pgfqpoint{4.009577in}{2.733297in}}%
\pgfpathlineto{\pgfqpoint{4.001925in}{2.725380in}}%
\pgfpathlineto{\pgfqpoint{3.994267in}{2.717499in}}%
\pgfpathlineto{\pgfqpoint{3.986603in}{2.709649in}}%
\pgfpathlineto{\pgfqpoint{3.973402in}{2.713767in}}%
\pgfpathlineto{\pgfqpoint{3.960206in}{2.717918in}}%
\pgfpathlineto{\pgfqpoint{3.947015in}{2.722102in}}%
\pgfpathlineto{\pgfqpoint{3.933830in}{2.726320in}}%
\pgfpathlineto{\pgfqpoint{3.941505in}{2.734226in}}%
\pgfpathlineto{\pgfqpoint{3.949175in}{2.742167in}}%
\pgfpathlineto{\pgfqpoint{3.956840in}{2.750147in}}%
\pgfpathlineto{\pgfqpoint{3.964499in}{2.758166in}}%
\pgfpathclose%
\pgfusepath{fill}%
\end{pgfscope}%
\begin{pgfscope}%
\pgfpathrectangle{\pgfqpoint{1.150000in}{0.150000in}}{\pgfqpoint{5.700000in}{5.700000in}}%
\pgfusepath{clip}%
\pgfsetbuttcap%
\pgfsetroundjoin%
\definecolor{currentfill}{rgb}{0.274952,0.037752,0.364543}%
\pgfsetfillcolor{currentfill}%
\pgfsetfillopacity{0.700000}%
\pgfsetlinewidth{0.000000pt}%
\definecolor{currentstroke}{rgb}{0.000000,0.000000,0.000000}%
\pgfsetstrokecolor{currentstroke}%
\pgfsetdash{}{0pt}%
\pgfpathmoveto{\pgfqpoint{4.319801in}{2.773807in}}%
\pgfpathlineto{\pgfqpoint{4.333051in}{2.769900in}}%
\pgfpathlineto{\pgfqpoint{4.346306in}{2.766023in}}%
\pgfpathlineto{\pgfqpoint{4.359567in}{2.762176in}}%
\pgfpathlineto{\pgfqpoint{4.372834in}{2.758358in}}%
\pgfpathlineto{\pgfqpoint{4.365309in}{2.750276in}}%
\pgfpathlineto{\pgfqpoint{4.357779in}{2.742249in}}%
\pgfpathlineto{\pgfqpoint{4.350245in}{2.734275in}}%
\pgfpathlineto{\pgfqpoint{4.342705in}{2.726349in}}%
\pgfpathlineto{\pgfqpoint{4.329426in}{2.730066in}}%
\pgfpathlineto{\pgfqpoint{4.316152in}{2.733813in}}%
\pgfpathlineto{\pgfqpoint{4.302885in}{2.737590in}}%
\pgfpathlineto{\pgfqpoint{4.289623in}{2.741397in}}%
\pgfpathlineto{\pgfqpoint{4.297175in}{2.749418in}}%
\pgfpathlineto{\pgfqpoint{4.304722in}{2.757492in}}%
\pgfpathlineto{\pgfqpoint{4.312264in}{2.765620in}}%
\pgfpathlineto{\pgfqpoint{4.319801in}{2.773807in}}%
\pgfpathclose%
\pgfusepath{fill}%
\end{pgfscope}%
\begin{pgfscope}%
\pgfpathrectangle{\pgfqpoint{1.150000in}{0.150000in}}{\pgfqpoint{5.700000in}{5.700000in}}%
\pgfusepath{clip}%
\pgfsetbuttcap%
\pgfsetroundjoin%
\definecolor{currentfill}{rgb}{0.272594,0.025563,0.353093}%
\pgfsetfillcolor{currentfill}%
\pgfsetfillopacity{0.700000}%
\pgfsetlinewidth{0.000000pt}%
\definecolor{currentstroke}{rgb}{0.000000,0.000000,0.000000}%
\pgfsetstrokecolor{currentstroke}%
\pgfsetdash{}{0pt}%
\pgfpathmoveto{\pgfqpoint{3.609143in}{2.754188in}}%
\pgfpathlineto{\pgfqpoint{3.622250in}{2.749335in}}%
\pgfpathlineto{\pgfqpoint{3.635361in}{2.744520in}}%
\pgfpathlineto{\pgfqpoint{3.648478in}{2.739743in}}%
\pgfpathlineto{\pgfqpoint{3.661599in}{2.735004in}}%
\pgfpathlineto{\pgfqpoint{3.653825in}{2.727225in}}%
\pgfpathlineto{\pgfqpoint{3.646046in}{2.719478in}}%
\pgfpathlineto{\pgfqpoint{3.638261in}{2.711761in}}%
\pgfpathlineto{\pgfqpoint{3.630471in}{2.704074in}}%
\pgfpathlineto{\pgfqpoint{3.617337in}{2.708791in}}%
\pgfpathlineto{\pgfqpoint{3.604209in}{2.713545in}}%
\pgfpathlineto{\pgfqpoint{3.591085in}{2.718338in}}%
\pgfpathlineto{\pgfqpoint{3.577967in}{2.723170in}}%
\pgfpathlineto{\pgfqpoint{3.585769in}{2.730875in}}%
\pgfpathlineto{\pgfqpoint{3.593566in}{2.738612in}}%
\pgfpathlineto{\pgfqpoint{3.601357in}{2.746383in}}%
\pgfpathlineto{\pgfqpoint{3.609143in}{2.754188in}}%
\pgfpathclose%
\pgfusepath{fill}%
\end{pgfscope}%
\begin{pgfscope}%
\pgfpathrectangle{\pgfqpoint{1.150000in}{0.150000in}}{\pgfqpoint{5.700000in}{5.700000in}}%
\pgfusepath{clip}%
\pgfsetbuttcap%
\pgfsetroundjoin%
\definecolor{currentfill}{rgb}{0.279566,0.067836,0.391917}%
\pgfsetfillcolor{currentfill}%
\pgfsetfillopacity{0.700000}%
\pgfsetlinewidth{0.000000pt}%
\definecolor{currentstroke}{rgb}{0.000000,0.000000,0.000000}%
\pgfsetstrokecolor{currentstroke}%
\pgfsetdash{}{0pt}%
\pgfpathmoveto{\pgfqpoint{4.894527in}{2.822793in}}%
\pgfpathlineto{\pgfqpoint{4.907904in}{2.819087in}}%
\pgfpathlineto{\pgfqpoint{4.921288in}{2.815408in}}%
\pgfpathlineto{\pgfqpoint{4.934678in}{2.811754in}}%
\pgfpathlineto{\pgfqpoint{4.948074in}{2.808127in}}%
\pgfpathlineto{\pgfqpoint{4.940737in}{2.799444in}}%
\pgfpathlineto{\pgfqpoint{4.933396in}{2.790878in}}%
\pgfpathlineto{\pgfqpoint{4.926053in}{2.782423in}}%
\pgfpathlineto{\pgfqpoint{4.918707in}{2.774075in}}%
\pgfpathlineto{\pgfqpoint{4.905296in}{2.777537in}}%
\pgfpathlineto{\pgfqpoint{4.891892in}{2.781025in}}%
\pgfpathlineto{\pgfqpoint{4.878494in}{2.784540in}}%
\pgfpathlineto{\pgfqpoint{4.865102in}{2.788081in}}%
\pgfpathlineto{\pgfqpoint{4.872463in}{2.796590in}}%
\pgfpathlineto{\pgfqpoint{4.879821in}{2.805208in}}%
\pgfpathlineto{\pgfqpoint{4.887176in}{2.813941in}}%
\pgfpathlineto{\pgfqpoint{4.894527in}{2.822793in}}%
\pgfpathclose%
\pgfusepath{fill}%
\end{pgfscope}%
\begin{pgfscope}%
\pgfpathrectangle{\pgfqpoint{1.150000in}{0.150000in}}{\pgfqpoint{5.700000in}{5.700000in}}%
\pgfusepath{clip}%
\pgfsetbuttcap%
\pgfsetroundjoin%
\definecolor{currentfill}{rgb}{0.273809,0.031497,0.358853}%
\pgfsetfillcolor{currentfill}%
\pgfsetfillopacity{0.700000}%
\pgfsetlinewidth{0.000000pt}%
\definecolor{currentstroke}{rgb}{0.000000,0.000000,0.000000}%
\pgfsetstrokecolor{currentstroke}%
\pgfsetdash{}{0pt}%
\pgfpathmoveto{\pgfqpoint{4.100526in}{2.756793in}}%
\pgfpathlineto{\pgfqpoint{4.113731in}{2.752702in}}%
\pgfpathlineto{\pgfqpoint{4.126942in}{2.748642in}}%
\pgfpathlineto{\pgfqpoint{4.140158in}{2.744615in}}%
\pgfpathlineto{\pgfqpoint{4.153380in}{2.740619in}}%
\pgfpathlineto{\pgfqpoint{4.145778in}{2.732652in}}%
\pgfpathlineto{\pgfqpoint{4.138171in}{2.724725in}}%
\pgfpathlineto{\pgfqpoint{4.130558in}{2.716836in}}%
\pgfpathlineto{\pgfqpoint{4.122940in}{2.708983in}}%
\pgfpathlineto{\pgfqpoint{4.109706in}{2.712905in}}%
\pgfpathlineto{\pgfqpoint{4.096478in}{2.716858in}}%
\pgfpathlineto{\pgfqpoint{4.083255in}{2.720843in}}%
\pgfpathlineto{\pgfqpoint{4.070038in}{2.724859in}}%
\pgfpathlineto{\pgfqpoint{4.077668in}{2.732782in}}%
\pgfpathlineto{\pgfqpoint{4.085292in}{2.740744in}}%
\pgfpathlineto{\pgfqpoint{4.092912in}{2.748747in}}%
\pgfpathlineto{\pgfqpoint{4.100526in}{2.756793in}}%
\pgfpathclose%
\pgfusepath{fill}%
\end{pgfscope}%
\begin{pgfscope}%
\pgfpathrectangle{\pgfqpoint{1.150000in}{0.150000in}}{\pgfqpoint{5.700000in}{5.700000in}}%
\pgfusepath{clip}%
\pgfsetbuttcap%
\pgfsetroundjoin%
\definecolor{currentfill}{rgb}{0.277941,0.056324,0.381191}%
\pgfsetfillcolor{currentfill}%
\pgfsetfillopacity{0.700000}%
\pgfsetlinewidth{0.000000pt}%
\definecolor{currentstroke}{rgb}{0.000000,0.000000,0.000000}%
\pgfsetstrokecolor{currentstroke}%
\pgfsetdash{}{0pt}%
\pgfpathmoveto{\pgfqpoint{4.675247in}{2.797796in}}%
\pgfpathlineto{\pgfqpoint{4.688578in}{2.794084in}}%
\pgfpathlineto{\pgfqpoint{4.701915in}{2.790399in}}%
\pgfpathlineto{\pgfqpoint{4.715258in}{2.786742in}}%
\pgfpathlineto{\pgfqpoint{4.728607in}{2.783112in}}%
\pgfpathlineto{\pgfqpoint{4.721200in}{2.774789in}}%
\pgfpathlineto{\pgfqpoint{4.713789in}{2.766552in}}%
\pgfpathlineto{\pgfqpoint{4.706375in}{2.758399in}}%
\pgfpathlineto{\pgfqpoint{4.698956in}{2.750323in}}%
\pgfpathlineto{\pgfqpoint{4.685593in}{2.753814in}}%
\pgfpathlineto{\pgfqpoint{4.672237in}{2.757332in}}%
\pgfpathlineto{\pgfqpoint{4.658887in}{2.760877in}}%
\pgfpathlineto{\pgfqpoint{4.645543in}{2.764450in}}%
\pgfpathlineto{\pgfqpoint{4.652975in}{2.772660in}}%
\pgfpathlineto{\pgfqpoint{4.660403in}{2.780951in}}%
\pgfpathlineto{\pgfqpoint{4.667827in}{2.789328in}}%
\pgfpathlineto{\pgfqpoint{4.675247in}{2.797796in}}%
\pgfpathclose%
\pgfusepath{fill}%
\end{pgfscope}%
\begin{pgfscope}%
\pgfpathrectangle{\pgfqpoint{1.150000in}{0.150000in}}{\pgfqpoint{5.700000in}{5.700000in}}%
\pgfusepath{clip}%
\pgfsetbuttcap%
\pgfsetroundjoin%
\definecolor{currentfill}{rgb}{0.272594,0.025563,0.353093}%
\pgfsetfillcolor{currentfill}%
\pgfsetfillopacity{0.700000}%
\pgfsetlinewidth{0.000000pt}%
\definecolor{currentstroke}{rgb}{0.000000,0.000000,0.000000}%
\pgfsetstrokecolor{currentstroke}%
\pgfsetdash{}{0pt}%
\pgfpathmoveto{\pgfqpoint{3.745122in}{2.747737in}}%
\pgfpathlineto{\pgfqpoint{3.758256in}{2.743148in}}%
\pgfpathlineto{\pgfqpoint{3.771395in}{2.738596in}}%
\pgfpathlineto{\pgfqpoint{3.784540in}{2.734079in}}%
\pgfpathlineto{\pgfqpoint{3.797689in}{2.729598in}}%
\pgfpathlineto{\pgfqpoint{3.789962in}{2.721760in}}%
\pgfpathlineto{\pgfqpoint{3.782230in}{2.713953in}}%
\pgfpathlineto{\pgfqpoint{3.774492in}{2.706175in}}%
\pgfpathlineto{\pgfqpoint{3.766748in}{2.698424in}}%
\pgfpathlineto{\pgfqpoint{3.753587in}{2.702869in}}%
\pgfpathlineto{\pgfqpoint{3.740430in}{2.707351in}}%
\pgfpathlineto{\pgfqpoint{3.727279in}{2.711868in}}%
\pgfpathlineto{\pgfqpoint{3.714133in}{2.716421in}}%
\pgfpathlineto{\pgfqpoint{3.721889in}{2.724202in}}%
\pgfpathlineto{\pgfqpoint{3.729639in}{2.732014in}}%
\pgfpathlineto{\pgfqpoint{3.737383in}{2.739859in}}%
\pgfpathlineto{\pgfqpoint{3.745122in}{2.747737in}}%
\pgfpathclose%
\pgfusepath{fill}%
\end{pgfscope}%
\begin{pgfscope}%
\pgfpathrectangle{\pgfqpoint{1.150000in}{0.150000in}}{\pgfqpoint{5.700000in}{5.700000in}}%
\pgfusepath{clip}%
\pgfsetbuttcap%
\pgfsetroundjoin%
\definecolor{currentfill}{rgb}{0.276022,0.044167,0.370164}%
\pgfsetfillcolor{currentfill}%
\pgfsetfillopacity{0.700000}%
\pgfsetlinewidth{0.000000pt}%
\definecolor{currentstroke}{rgb}{0.000000,0.000000,0.000000}%
\pgfsetstrokecolor{currentstroke}%
\pgfsetdash{}{0pt}%
\pgfpathmoveto{\pgfqpoint{4.455963in}{2.775879in}}%
\pgfpathlineto{\pgfqpoint{4.469247in}{2.772094in}}%
\pgfpathlineto{\pgfqpoint{4.482537in}{2.768338in}}%
\pgfpathlineto{\pgfqpoint{4.495833in}{2.764611in}}%
\pgfpathlineto{\pgfqpoint{4.509135in}{2.760912in}}%
\pgfpathlineto{\pgfqpoint{4.501654in}{2.752810in}}%
\pgfpathlineto{\pgfqpoint{4.494169in}{2.744770in}}%
\pgfpathlineto{\pgfqpoint{4.486679in}{2.736791in}}%
\pgfpathlineto{\pgfqpoint{4.479184in}{2.728867in}}%
\pgfpathlineto{\pgfqpoint{4.465869in}{2.732453in}}%
\pgfpathlineto{\pgfqpoint{4.452561in}{2.736066in}}%
\pgfpathlineto{\pgfqpoint{4.439258in}{2.739709in}}%
\pgfpathlineto{\pgfqpoint{4.425961in}{2.743380in}}%
\pgfpathlineto{\pgfqpoint{4.433469in}{2.751413in}}%
\pgfpathlineto{\pgfqpoint{4.440971in}{2.759504in}}%
\pgfpathlineto{\pgfqpoint{4.448469in}{2.767658in}}%
\pgfpathlineto{\pgfqpoint{4.455963in}{2.775879in}}%
\pgfpathclose%
\pgfusepath{fill}%
\end{pgfscope}%
\begin{pgfscope}%
\pgfpathrectangle{\pgfqpoint{1.150000in}{0.150000in}}{\pgfqpoint{5.700000in}{5.700000in}}%
\pgfusepath{clip}%
\pgfsetbuttcap%
\pgfsetroundjoin%
\definecolor{currentfill}{rgb}{0.272594,0.025563,0.353093}%
\pgfsetfillcolor{currentfill}%
\pgfsetfillopacity{0.700000}%
\pgfsetlinewidth{0.000000pt}%
\definecolor{currentstroke}{rgb}{0.000000,0.000000,0.000000}%
\pgfsetstrokecolor{currentstroke}%
\pgfsetdash{}{0pt}%
\pgfpathmoveto{\pgfqpoint{3.881143in}{2.743530in}}%
\pgfpathlineto{\pgfqpoint{3.894307in}{2.739176in}}%
\pgfpathlineto{\pgfqpoint{3.907476in}{2.734857in}}%
\pgfpathlineto{\pgfqpoint{3.920650in}{2.730571in}}%
\pgfpathlineto{\pgfqpoint{3.933830in}{2.726320in}}%
\pgfpathlineto{\pgfqpoint{3.926149in}{2.718448in}}%
\pgfpathlineto{\pgfqpoint{3.918463in}{2.710607in}}%
\pgfpathlineto{\pgfqpoint{3.910771in}{2.702795in}}%
\pgfpathlineto{\pgfqpoint{3.903074in}{2.695012in}}%
\pgfpathlineto{\pgfqpoint{3.889882in}{2.699215in}}%
\pgfpathlineto{\pgfqpoint{3.876696in}{2.703452in}}%
\pgfpathlineto{\pgfqpoint{3.863515in}{2.707723in}}%
\pgfpathlineto{\pgfqpoint{3.850339in}{2.712028in}}%
\pgfpathlineto{\pgfqpoint{3.858048in}{2.719855in}}%
\pgfpathlineto{\pgfqpoint{3.865752in}{2.727714in}}%
\pgfpathlineto{\pgfqpoint{3.873450in}{2.735605in}}%
\pgfpathlineto{\pgfqpoint{3.881143in}{2.743530in}}%
\pgfpathclose%
\pgfusepath{fill}%
\end{pgfscope}%
\begin{pgfscope}%
\pgfpathrectangle{\pgfqpoint{1.150000in}{0.150000in}}{\pgfqpoint{5.700000in}{5.700000in}}%
\pgfusepath{clip}%
\pgfsetbuttcap%
\pgfsetroundjoin%
\definecolor{currentfill}{rgb}{0.274952,0.037752,0.364543}%
\pgfsetfillcolor{currentfill}%
\pgfsetfillopacity{0.700000}%
\pgfsetlinewidth{0.000000pt}%
\definecolor{currentstroke}{rgb}{0.000000,0.000000,0.000000}%
\pgfsetstrokecolor{currentstroke}%
\pgfsetdash{}{0pt}%
\pgfpathmoveto{\pgfqpoint{3.253408in}{2.767717in}}%
\pgfpathlineto{\pgfqpoint{3.266464in}{2.762049in}}%
\pgfpathlineto{\pgfqpoint{3.279524in}{2.756426in}}%
\pgfpathlineto{\pgfqpoint{3.292588in}{2.750849in}}%
\pgfpathlineto{\pgfqpoint{3.305655in}{2.745317in}}%
\pgfpathlineto{\pgfqpoint{3.297748in}{2.737916in}}%
\pgfpathlineto{\pgfqpoint{3.289834in}{2.730555in}}%
\pgfpathlineto{\pgfqpoint{3.281914in}{2.723233in}}%
\pgfpathlineto{\pgfqpoint{3.273988in}{2.715949in}}%
\pgfpathlineto{\pgfqpoint{3.260907in}{2.721499in}}%
\pgfpathlineto{\pgfqpoint{3.247830in}{2.727093in}}%
\pgfpathlineto{\pgfqpoint{3.234757in}{2.732733in}}%
\pgfpathlineto{\pgfqpoint{3.221688in}{2.738418in}}%
\pgfpathlineto{\pgfqpoint{3.229628in}{2.745680in}}%
\pgfpathlineto{\pgfqpoint{3.237561in}{2.752983in}}%
\pgfpathlineto{\pgfqpoint{3.245488in}{2.760329in}}%
\pgfpathlineto{\pgfqpoint{3.253408in}{2.767717in}}%
\pgfpathclose%
\pgfusepath{fill}%
\end{pgfscope}%
\begin{pgfscope}%
\pgfpathrectangle{\pgfqpoint{1.150000in}{0.150000in}}{\pgfqpoint{5.700000in}{5.700000in}}%
\pgfusepath{clip}%
\pgfsetbuttcap%
\pgfsetroundjoin%
\definecolor{currentfill}{rgb}{0.273809,0.031497,0.358853}%
\pgfsetfillcolor{currentfill}%
\pgfsetfillopacity{0.700000}%
\pgfsetlinewidth{0.000000pt}%
\definecolor{currentstroke}{rgb}{0.000000,0.000000,0.000000}%
\pgfsetstrokecolor{currentstroke}%
\pgfsetdash{}{0pt}%
\pgfpathmoveto{\pgfqpoint{4.236634in}{2.756927in}}%
\pgfpathlineto{\pgfqpoint{4.249873in}{2.752998in}}%
\pgfpathlineto{\pgfqpoint{4.263117in}{2.749101in}}%
\pgfpathlineto{\pgfqpoint{4.276367in}{2.745234in}}%
\pgfpathlineto{\pgfqpoint{4.289623in}{2.741397in}}%
\pgfpathlineto{\pgfqpoint{4.282066in}{2.733424in}}%
\pgfpathlineto{\pgfqpoint{4.274503in}{2.725496in}}%
\pgfpathlineto{\pgfqpoint{4.266936in}{2.717612in}}%
\pgfpathlineto{\pgfqpoint{4.259364in}{2.709767in}}%
\pgfpathlineto{\pgfqpoint{4.246095in}{2.713516in}}%
\pgfpathlineto{\pgfqpoint{4.232833in}{2.717296in}}%
\pgfpathlineto{\pgfqpoint{4.219576in}{2.721106in}}%
\pgfpathlineto{\pgfqpoint{4.206326in}{2.724947in}}%
\pgfpathlineto{\pgfqpoint{4.213911in}{2.732874in}}%
\pgfpathlineto{\pgfqpoint{4.221490in}{2.740845in}}%
\pgfpathlineto{\pgfqpoint{4.229065in}{2.748861in}}%
\pgfpathlineto{\pgfqpoint{4.236634in}{2.756927in}}%
\pgfpathclose%
\pgfusepath{fill}%
\end{pgfscope}%
\begin{pgfscope}%
\pgfpathrectangle{\pgfqpoint{1.150000in}{0.150000in}}{\pgfqpoint{5.700000in}{5.700000in}}%
\pgfusepath{clip}%
\pgfsetbuttcap%
\pgfsetroundjoin%
\definecolor{currentfill}{rgb}{0.277018,0.050344,0.375715}%
\pgfsetfillcolor{currentfill}%
\pgfsetfillopacity{0.700000}%
\pgfsetlinewidth{0.000000pt}%
\definecolor{currentstroke}{rgb}{0.000000,0.000000,0.000000}%
\pgfsetstrokecolor{currentstroke}%
\pgfsetdash{}{0pt}%
\pgfpathmoveto{\pgfqpoint{3.117274in}{2.785604in}}%
\pgfpathlineto{\pgfqpoint{3.130313in}{2.779536in}}%
\pgfpathlineto{\pgfqpoint{3.143355in}{2.773517in}}%
\pgfpathlineto{\pgfqpoint{3.156401in}{2.767548in}}%
\pgfpathlineto{\pgfqpoint{3.169451in}{2.761627in}}%
\pgfpathlineto{\pgfqpoint{3.161491in}{2.754433in}}%
\pgfpathlineto{\pgfqpoint{3.153525in}{2.747285in}}%
\pgfpathlineto{\pgfqpoint{3.145552in}{2.740183in}}%
\pgfpathlineto{\pgfqpoint{3.137572in}{2.733128in}}%
\pgfpathlineto{\pgfqpoint{3.124509in}{2.739079in}}%
\pgfpathlineto{\pgfqpoint{3.111449in}{2.745079in}}%
\pgfpathlineto{\pgfqpoint{3.098393in}{2.751127in}}%
\pgfpathlineto{\pgfqpoint{3.085340in}{2.757226in}}%
\pgfpathlineto{\pgfqpoint{3.093334in}{2.764246in}}%
\pgfpathlineto{\pgfqpoint{3.101321in}{2.771316in}}%
\pgfpathlineto{\pgfqpoint{3.109301in}{2.778435in}}%
\pgfpathlineto{\pgfqpoint{3.117274in}{2.785604in}}%
\pgfpathclose%
\pgfusepath{fill}%
\end{pgfscope}%
\begin{pgfscope}%
\pgfpathrectangle{\pgfqpoint{1.150000in}{0.150000in}}{\pgfqpoint{5.700000in}{5.700000in}}%
\pgfusepath{clip}%
\pgfsetbuttcap%
\pgfsetroundjoin%
\definecolor{currentfill}{rgb}{0.273809,0.031497,0.358853}%
\pgfsetfillcolor{currentfill}%
\pgfsetfillopacity{0.700000}%
\pgfsetlinewidth{0.000000pt}%
\definecolor{currentstroke}{rgb}{0.000000,0.000000,0.000000}%
\pgfsetstrokecolor{currentstroke}%
\pgfsetdash{}{0pt}%
\pgfpathmoveto{\pgfqpoint{3.389485in}{2.753643in}}%
\pgfpathlineto{\pgfqpoint{3.402561in}{2.748334in}}%
\pgfpathlineto{\pgfqpoint{3.415642in}{2.743066in}}%
\pgfpathlineto{\pgfqpoint{3.428727in}{2.737841in}}%
\pgfpathlineto{\pgfqpoint{3.441816in}{2.732658in}}%
\pgfpathlineto{\pgfqpoint{3.433959in}{2.725099in}}%
\pgfpathlineto{\pgfqpoint{3.426096in}{2.717575in}}%
\pgfpathlineto{\pgfqpoint{3.418227in}{2.710084in}}%
\pgfpathlineto{\pgfqpoint{3.410351in}{2.702627in}}%
\pgfpathlineto{\pgfqpoint{3.397249in}{2.707814in}}%
\pgfpathlineto{\pgfqpoint{3.384151in}{2.713043in}}%
\pgfpathlineto{\pgfqpoint{3.371058in}{2.718314in}}%
\pgfpathlineto{\pgfqpoint{3.357969in}{2.723628in}}%
\pgfpathlineto{\pgfqpoint{3.365857in}{2.731077in}}%
\pgfpathlineto{\pgfqpoint{3.373739in}{2.738562in}}%
\pgfpathlineto{\pgfqpoint{3.381615in}{2.746084in}}%
\pgfpathlineto{\pgfqpoint{3.389485in}{2.753643in}}%
\pgfpathclose%
\pgfusepath{fill}%
\end{pgfscope}%
\begin{pgfscope}%
\pgfpathrectangle{\pgfqpoint{1.150000in}{0.150000in}}{\pgfqpoint{5.700000in}{5.700000in}}%
\pgfusepath{clip}%
\pgfsetbuttcap%
\pgfsetroundjoin%
\definecolor{currentfill}{rgb}{0.278791,0.062145,0.386592}%
\pgfsetfillcolor{currentfill}%
\pgfsetfillopacity{0.700000}%
\pgfsetlinewidth{0.000000pt}%
\definecolor{currentstroke}{rgb}{0.000000,0.000000,0.000000}%
\pgfsetstrokecolor{currentstroke}%
\pgfsetdash{}{0pt}%
\pgfpathmoveto{\pgfqpoint{4.811599in}{2.802510in}}%
\pgfpathlineto{\pgfqpoint{4.824966in}{2.798863in}}%
\pgfpathlineto{\pgfqpoint{4.838338in}{2.795243in}}%
\pgfpathlineto{\pgfqpoint{4.851717in}{2.791649in}}%
\pgfpathlineto{\pgfqpoint{4.865102in}{2.788081in}}%
\pgfpathlineto{\pgfqpoint{4.857738in}{2.779677in}}%
\pgfpathlineto{\pgfqpoint{4.850370in}{2.771372in}}%
\pgfpathlineto{\pgfqpoint{4.842999in}{2.763162in}}%
\pgfpathlineto{\pgfqpoint{4.835624in}{2.755043in}}%
\pgfpathlineto{\pgfqpoint{4.822225in}{2.758458in}}%
\pgfpathlineto{\pgfqpoint{4.808832in}{2.761899in}}%
\pgfpathlineto{\pgfqpoint{4.795445in}{2.765368in}}%
\pgfpathlineto{\pgfqpoint{4.782065in}{2.768863in}}%
\pgfpathlineto{\pgfqpoint{4.789454in}{2.777130in}}%
\pgfpathlineto{\pgfqpoint{4.796839in}{2.785491in}}%
\pgfpathlineto{\pgfqpoint{4.804221in}{2.793949in}}%
\pgfpathlineto{\pgfqpoint{4.811599in}{2.802510in}}%
\pgfpathclose%
\pgfusepath{fill}%
\end{pgfscope}%
\begin{pgfscope}%
\pgfpathrectangle{\pgfqpoint{1.150000in}{0.150000in}}{\pgfqpoint{5.700000in}{5.700000in}}%
\pgfusepath{clip}%
\pgfsetbuttcap%
\pgfsetroundjoin%
\definecolor{currentfill}{rgb}{0.272594,0.025563,0.353093}%
\pgfsetfillcolor{currentfill}%
\pgfsetfillopacity{0.700000}%
\pgfsetlinewidth{0.000000pt}%
\definecolor{currentstroke}{rgb}{0.000000,0.000000,0.000000}%
\pgfsetstrokecolor{currentstroke}%
\pgfsetdash{}{0pt}%
\pgfpathmoveto{\pgfqpoint{3.525539in}{2.742887in}}%
\pgfpathlineto{\pgfqpoint{3.538639in}{2.737898in}}%
\pgfpathlineto{\pgfqpoint{3.551743in}{2.732949in}}%
\pgfpathlineto{\pgfqpoint{3.564853in}{2.728040in}}%
\pgfpathlineto{\pgfqpoint{3.577967in}{2.723170in}}%
\pgfpathlineto{\pgfqpoint{3.570158in}{2.715496in}}%
\pgfpathlineto{\pgfqpoint{3.562343in}{2.707853in}}%
\pgfpathlineto{\pgfqpoint{3.554523in}{2.700240in}}%
\pgfpathlineto{\pgfqpoint{3.546697in}{2.692656in}}%
\pgfpathlineto{\pgfqpoint{3.533570in}{2.697517in}}%
\pgfpathlineto{\pgfqpoint{3.520449in}{2.702417in}}%
\pgfpathlineto{\pgfqpoint{3.507332in}{2.707356in}}%
\pgfpathlineto{\pgfqpoint{3.494220in}{2.712336in}}%
\pgfpathlineto{\pgfqpoint{3.502058in}{2.719925in}}%
\pgfpathlineto{\pgfqpoint{3.509891in}{2.727546in}}%
\pgfpathlineto{\pgfqpoint{3.517718in}{2.735199in}}%
\pgfpathlineto{\pgfqpoint{3.525539in}{2.742887in}}%
\pgfpathclose%
\pgfusepath{fill}%
\end{pgfscope}%
\begin{pgfscope}%
\pgfpathrectangle{\pgfqpoint{1.150000in}{0.150000in}}{\pgfqpoint{5.700000in}{5.700000in}}%
\pgfusepath{clip}%
\pgfsetbuttcap%
\pgfsetroundjoin%
\definecolor{currentfill}{rgb}{0.277018,0.050344,0.375715}%
\pgfsetfillcolor{currentfill}%
\pgfsetfillopacity{0.700000}%
\pgfsetlinewidth{0.000000pt}%
\definecolor{currentstroke}{rgb}{0.000000,0.000000,0.000000}%
\pgfsetstrokecolor{currentstroke}%
\pgfsetdash{}{0pt}%
\pgfpathmoveto{\pgfqpoint{4.592229in}{2.779018in}}%
\pgfpathlineto{\pgfqpoint{4.605548in}{2.775334in}}%
\pgfpathlineto{\pgfqpoint{4.618873in}{2.771678in}}%
\pgfpathlineto{\pgfqpoint{4.632205in}{2.768050in}}%
\pgfpathlineto{\pgfqpoint{4.645543in}{2.764450in}}%
\pgfpathlineto{\pgfqpoint{4.638106in}{2.756317in}}%
\pgfpathlineto{\pgfqpoint{4.630665in}{2.748257in}}%
\pgfpathlineto{\pgfqpoint{4.623220in}{2.740265in}}%
\pgfpathlineto{\pgfqpoint{4.615770in}{2.732339in}}%
\pgfpathlineto{\pgfqpoint{4.602419in}{2.735813in}}%
\pgfpathlineto{\pgfqpoint{4.589074in}{2.739314in}}%
\pgfpathlineto{\pgfqpoint{4.575736in}{2.742844in}}%
\pgfpathlineto{\pgfqpoint{4.562403in}{2.746401in}}%
\pgfpathlineto{\pgfqpoint{4.569866in}{2.754449in}}%
\pgfpathlineto{\pgfqpoint{4.577325in}{2.762566in}}%
\pgfpathlineto{\pgfqpoint{4.584779in}{2.770754in}}%
\pgfpathlineto{\pgfqpoint{4.592229in}{2.779018in}}%
\pgfpathclose%
\pgfusepath{fill}%
\end{pgfscope}%
\begin{pgfscope}%
\pgfpathrectangle{\pgfqpoint{1.150000in}{0.150000in}}{\pgfqpoint{5.700000in}{5.700000in}}%
\pgfusepath{clip}%
\pgfsetbuttcap%
\pgfsetroundjoin%
\definecolor{currentfill}{rgb}{0.272594,0.025563,0.353093}%
\pgfsetfillcolor{currentfill}%
\pgfsetfillopacity{0.700000}%
\pgfsetlinewidth{0.000000pt}%
\definecolor{currentstroke}{rgb}{0.000000,0.000000,0.000000}%
\pgfsetstrokecolor{currentstroke}%
\pgfsetdash{}{0pt}%
\pgfpathmoveto{\pgfqpoint{4.017224in}{2.741250in}}%
\pgfpathlineto{\pgfqpoint{4.030420in}{2.737104in}}%
\pgfpathlineto{\pgfqpoint{4.043620in}{2.732990in}}%
\pgfpathlineto{\pgfqpoint{4.056826in}{2.728908in}}%
\pgfpathlineto{\pgfqpoint{4.070038in}{2.724859in}}%
\pgfpathlineto{\pgfqpoint{4.062402in}{2.716973in}}%
\pgfpathlineto{\pgfqpoint{4.054762in}{2.709119in}}%
\pgfpathlineto{\pgfqpoint{4.047116in}{2.701297in}}%
\pgfpathlineto{\pgfqpoint{4.039464in}{2.693505in}}%
\pgfpathlineto{\pgfqpoint{4.026241in}{2.697492in}}%
\pgfpathlineto{\pgfqpoint{4.013022in}{2.701512in}}%
\pgfpathlineto{\pgfqpoint{3.999810in}{2.705564in}}%
\pgfpathlineto{\pgfqpoint{3.986603in}{2.709649in}}%
\pgfpathlineto{\pgfqpoint{3.994267in}{2.717499in}}%
\pgfpathlineto{\pgfqpoint{4.001925in}{2.725380in}}%
\pgfpathlineto{\pgfqpoint{4.009577in}{2.733297in}}%
\pgfpathlineto{\pgfqpoint{4.017224in}{2.741250in}}%
\pgfpathclose%
\pgfusepath{fill}%
\end{pgfscope}%
\begin{pgfscope}%
\pgfpathrectangle{\pgfqpoint{1.150000in}{0.150000in}}{\pgfqpoint{5.700000in}{5.700000in}}%
\pgfusepath{clip}%
\pgfsetbuttcap%
\pgfsetroundjoin%
\definecolor{currentfill}{rgb}{0.272594,0.025563,0.353093}%
\pgfsetfillcolor{currentfill}%
\pgfsetfillopacity{0.700000}%
\pgfsetlinewidth{0.000000pt}%
\definecolor{currentstroke}{rgb}{0.000000,0.000000,0.000000}%
\pgfsetstrokecolor{currentstroke}%
\pgfsetdash{}{0pt}%
\pgfpathmoveto{\pgfqpoint{3.661599in}{2.735004in}}%
\pgfpathlineto{\pgfqpoint{3.674725in}{2.730302in}}%
\pgfpathlineto{\pgfqpoint{3.687856in}{2.725638in}}%
\pgfpathlineto{\pgfqpoint{3.700992in}{2.721011in}}%
\pgfpathlineto{\pgfqpoint{3.714133in}{2.716421in}}%
\pgfpathlineto{\pgfqpoint{3.706372in}{2.708670in}}%
\pgfpathlineto{\pgfqpoint{3.698605in}{2.700947in}}%
\pgfpathlineto{\pgfqpoint{3.690832in}{2.693251in}}%
\pgfpathlineto{\pgfqpoint{3.683053in}{2.685581in}}%
\pgfpathlineto{\pgfqpoint{3.669900in}{2.690149in}}%
\pgfpathlineto{\pgfqpoint{3.656752in}{2.694753in}}%
\pgfpathlineto{\pgfqpoint{3.643609in}{2.699395in}}%
\pgfpathlineto{\pgfqpoint{3.630471in}{2.704074in}}%
\pgfpathlineto{\pgfqpoint{3.638261in}{2.711761in}}%
\pgfpathlineto{\pgfqpoint{3.646046in}{2.719478in}}%
\pgfpathlineto{\pgfqpoint{3.653825in}{2.727225in}}%
\pgfpathlineto{\pgfqpoint{3.661599in}{2.735004in}}%
\pgfpathclose%
\pgfusepath{fill}%
\end{pgfscope}%
\begin{pgfscope}%
\pgfpathrectangle{\pgfqpoint{1.150000in}{0.150000in}}{\pgfqpoint{5.700000in}{5.700000in}}%
\pgfusepath{clip}%
\pgfsetbuttcap%
\pgfsetroundjoin%
\definecolor{currentfill}{rgb}{0.274952,0.037752,0.364543}%
\pgfsetfillcolor{currentfill}%
\pgfsetfillopacity{0.700000}%
\pgfsetlinewidth{0.000000pt}%
\definecolor{currentstroke}{rgb}{0.000000,0.000000,0.000000}%
\pgfsetstrokecolor{currentstroke}%
\pgfsetdash{}{0pt}%
\pgfpathmoveto{\pgfqpoint{4.372834in}{2.758358in}}%
\pgfpathlineto{\pgfqpoint{4.386107in}{2.754569in}}%
\pgfpathlineto{\pgfqpoint{4.399386in}{2.750811in}}%
\pgfpathlineto{\pgfqpoint{4.412670in}{2.747081in}}%
\pgfpathlineto{\pgfqpoint{4.425961in}{2.743380in}}%
\pgfpathlineto{\pgfqpoint{4.418449in}{2.735404in}}%
\pgfpathlineto{\pgfqpoint{4.410932in}{2.727479in}}%
\pgfpathlineto{\pgfqpoint{4.403410in}{2.719604in}}%
\pgfpathlineto{\pgfqpoint{4.395883in}{2.711774in}}%
\pgfpathlineto{\pgfqpoint{4.382579in}{2.715373in}}%
\pgfpathlineto{\pgfqpoint{4.369282in}{2.719002in}}%
\pgfpathlineto{\pgfqpoint{4.355990in}{2.722661in}}%
\pgfpathlineto{\pgfqpoint{4.342705in}{2.726349in}}%
\pgfpathlineto{\pgfqpoint{4.350245in}{2.734275in}}%
\pgfpathlineto{\pgfqpoint{4.357779in}{2.742249in}}%
\pgfpathlineto{\pgfqpoint{4.365309in}{2.750276in}}%
\pgfpathlineto{\pgfqpoint{4.372834in}{2.758358in}}%
\pgfpathclose%
\pgfusepath{fill}%
\end{pgfscope}%
\begin{pgfscope}%
\pgfpathrectangle{\pgfqpoint{1.150000in}{0.150000in}}{\pgfqpoint{5.700000in}{5.700000in}}%
\pgfusepath{clip}%
\pgfsetbuttcap%
\pgfsetroundjoin%
\definecolor{currentfill}{rgb}{0.280267,0.073417,0.397163}%
\pgfsetfillcolor{currentfill}%
\pgfsetfillopacity{0.700000}%
\pgfsetlinewidth{0.000000pt}%
\definecolor{currentstroke}{rgb}{0.000000,0.000000,0.000000}%
\pgfsetstrokecolor{currentstroke}%
\pgfsetdash{}{0pt}%
\pgfpathmoveto{\pgfqpoint{4.948074in}{2.808127in}}%
\pgfpathlineto{\pgfqpoint{4.961477in}{2.804525in}}%
\pgfpathlineto{\pgfqpoint{4.974885in}{2.800950in}}%
\pgfpathlineto{\pgfqpoint{4.988301in}{2.797401in}}%
\pgfpathlineto{\pgfqpoint{5.001722in}{2.793877in}}%
\pgfpathlineto{\pgfqpoint{4.994400in}{2.785365in}}%
\pgfpathlineto{\pgfqpoint{4.987074in}{2.776965in}}%
\pgfpathlineto{\pgfqpoint{4.979745in}{2.768674in}}%
\pgfpathlineto{\pgfqpoint{4.972413in}{2.760487in}}%
\pgfpathlineto{\pgfqpoint{4.958977in}{2.763845in}}%
\pgfpathlineto{\pgfqpoint{4.945547in}{2.767229in}}%
\pgfpathlineto{\pgfqpoint{4.932124in}{2.770639in}}%
\pgfpathlineto{\pgfqpoint{4.918707in}{2.774075in}}%
\pgfpathlineto{\pgfqpoint{4.926053in}{2.782423in}}%
\pgfpathlineto{\pgfqpoint{4.933396in}{2.790878in}}%
\pgfpathlineto{\pgfqpoint{4.940737in}{2.799444in}}%
\pgfpathlineto{\pgfqpoint{4.948074in}{2.808127in}}%
\pgfpathclose%
\pgfusepath{fill}%
\end{pgfscope}%
\begin{pgfscope}%
\pgfpathrectangle{\pgfqpoint{1.150000in}{0.150000in}}{\pgfqpoint{5.700000in}{5.700000in}}%
\pgfusepath{clip}%
\pgfsetbuttcap%
\pgfsetroundjoin%
\definecolor{currentfill}{rgb}{0.273809,0.031497,0.358853}%
\pgfsetfillcolor{currentfill}%
\pgfsetfillopacity{0.700000}%
\pgfsetlinewidth{0.000000pt}%
\definecolor{currentstroke}{rgb}{0.000000,0.000000,0.000000}%
\pgfsetstrokecolor{currentstroke}%
\pgfsetdash{}{0pt}%
\pgfpathmoveto{\pgfqpoint{4.153380in}{2.740619in}}%
\pgfpathlineto{\pgfqpoint{4.166608in}{2.736654in}}%
\pgfpathlineto{\pgfqpoint{4.179842in}{2.732721in}}%
\pgfpathlineto{\pgfqpoint{4.193081in}{2.728818in}}%
\pgfpathlineto{\pgfqpoint{4.206326in}{2.724947in}}%
\pgfpathlineto{\pgfqpoint{4.198736in}{2.717059in}}%
\pgfpathlineto{\pgfqpoint{4.191141in}{2.709209in}}%
\pgfpathlineto{\pgfqpoint{4.183540in}{2.701393in}}%
\pgfpathlineto{\pgfqpoint{4.175934in}{2.693609in}}%
\pgfpathlineto{\pgfqpoint{4.162677in}{2.697406in}}%
\pgfpathlineto{\pgfqpoint{4.149426in}{2.701234in}}%
\pgfpathlineto{\pgfqpoint{4.136180in}{2.705093in}}%
\pgfpathlineto{\pgfqpoint{4.122940in}{2.708983in}}%
\pgfpathlineto{\pgfqpoint{4.130558in}{2.716836in}}%
\pgfpathlineto{\pgfqpoint{4.138171in}{2.724725in}}%
\pgfpathlineto{\pgfqpoint{4.145778in}{2.732652in}}%
\pgfpathlineto{\pgfqpoint{4.153380in}{2.740619in}}%
\pgfpathclose%
\pgfusepath{fill}%
\end{pgfscope}%
\begin{pgfscope}%
\pgfpathrectangle{\pgfqpoint{1.150000in}{0.150000in}}{\pgfqpoint{5.700000in}{5.700000in}}%
\pgfusepath{clip}%
\pgfsetbuttcap%
\pgfsetroundjoin%
\definecolor{currentfill}{rgb}{0.272594,0.025563,0.353093}%
\pgfsetfillcolor{currentfill}%
\pgfsetfillopacity{0.700000}%
\pgfsetlinewidth{0.000000pt}%
\definecolor{currentstroke}{rgb}{0.000000,0.000000,0.000000}%
\pgfsetstrokecolor{currentstroke}%
\pgfsetdash{}{0pt}%
\pgfpathmoveto{\pgfqpoint{3.797689in}{2.729598in}}%
\pgfpathlineto{\pgfqpoint{3.810844in}{2.725153in}}%
\pgfpathlineto{\pgfqpoint{3.824004in}{2.720743in}}%
\pgfpathlineto{\pgfqpoint{3.837169in}{2.716368in}}%
\pgfpathlineto{\pgfqpoint{3.850339in}{2.712028in}}%
\pgfpathlineto{\pgfqpoint{3.842624in}{2.704230in}}%
\pgfpathlineto{\pgfqpoint{3.834904in}{2.696460in}}%
\pgfpathlineto{\pgfqpoint{3.827178in}{2.688716in}}%
\pgfpathlineto{\pgfqpoint{3.819446in}{2.680996in}}%
\pgfpathlineto{\pgfqpoint{3.806264in}{2.685300in}}%
\pgfpathlineto{\pgfqpoint{3.793087in}{2.689640in}}%
\pgfpathlineto{\pgfqpoint{3.779915in}{2.694014in}}%
\pgfpathlineto{\pgfqpoint{3.766748in}{2.698424in}}%
\pgfpathlineto{\pgfqpoint{3.774492in}{2.706175in}}%
\pgfpathlineto{\pgfqpoint{3.782230in}{2.713953in}}%
\pgfpathlineto{\pgfqpoint{3.789962in}{2.721760in}}%
\pgfpathlineto{\pgfqpoint{3.797689in}{2.729598in}}%
\pgfpathclose%
\pgfusepath{fill}%
\end{pgfscope}%
\begin{pgfscope}%
\pgfpathrectangle{\pgfqpoint{1.150000in}{0.150000in}}{\pgfqpoint{5.700000in}{5.700000in}}%
\pgfusepath{clip}%
\pgfsetbuttcap%
\pgfsetroundjoin%
\definecolor{currentfill}{rgb}{0.277941,0.056324,0.381191}%
\pgfsetfillcolor{currentfill}%
\pgfsetfillopacity{0.700000}%
\pgfsetlinewidth{0.000000pt}%
\definecolor{currentstroke}{rgb}{0.000000,0.000000,0.000000}%
\pgfsetstrokecolor{currentstroke}%
\pgfsetdash{}{0pt}%
\pgfpathmoveto{\pgfqpoint{4.728607in}{2.783112in}}%
\pgfpathlineto{\pgfqpoint{4.741962in}{2.779509in}}%
\pgfpathlineto{\pgfqpoint{4.755323in}{2.775933in}}%
\pgfpathlineto{\pgfqpoint{4.768691in}{2.772384in}}%
\pgfpathlineto{\pgfqpoint{4.782065in}{2.768863in}}%
\pgfpathlineto{\pgfqpoint{4.774672in}{2.760684in}}%
\pgfpathlineto{\pgfqpoint{4.767275in}{2.752588in}}%
\pgfpathlineto{\pgfqpoint{4.759874in}{2.744573in}}%
\pgfpathlineto{\pgfqpoint{4.752469in}{2.736632in}}%
\pgfpathlineto{\pgfqpoint{4.739081in}{2.740014in}}%
\pgfpathlineto{\pgfqpoint{4.725700in}{2.743424in}}%
\pgfpathlineto{\pgfqpoint{4.712325in}{2.746860in}}%
\pgfpathlineto{\pgfqpoint{4.698956in}{2.750323in}}%
\pgfpathlineto{\pgfqpoint{4.706375in}{2.758399in}}%
\pgfpathlineto{\pgfqpoint{4.713789in}{2.766552in}}%
\pgfpathlineto{\pgfqpoint{4.721200in}{2.774789in}}%
\pgfpathlineto{\pgfqpoint{4.728607in}{2.783112in}}%
\pgfpathclose%
\pgfusepath{fill}%
\end{pgfscope}%
\begin{pgfscope}%
\pgfpathrectangle{\pgfqpoint{1.150000in}{0.150000in}}{\pgfqpoint{5.700000in}{5.700000in}}%
\pgfusepath{clip}%
\pgfsetbuttcap%
\pgfsetroundjoin%
\definecolor{currentfill}{rgb}{0.276022,0.044167,0.370164}%
\pgfsetfillcolor{currentfill}%
\pgfsetfillopacity{0.700000}%
\pgfsetlinewidth{0.000000pt}%
\definecolor{currentstroke}{rgb}{0.000000,0.000000,0.000000}%
\pgfsetstrokecolor{currentstroke}%
\pgfsetdash{}{0pt}%
\pgfpathmoveto{\pgfqpoint{4.509135in}{2.760912in}}%
\pgfpathlineto{\pgfqpoint{4.522443in}{2.757242in}}%
\pgfpathlineto{\pgfqpoint{4.535757in}{2.753600in}}%
\pgfpathlineto{\pgfqpoint{4.549077in}{2.749987in}}%
\pgfpathlineto{\pgfqpoint{4.562403in}{2.746401in}}%
\pgfpathlineto{\pgfqpoint{4.554936in}{2.738417in}}%
\pgfpathlineto{\pgfqpoint{4.547464in}{2.730493in}}%
\pgfpathlineto{\pgfqpoint{4.539987in}{2.722626in}}%
\pgfpathlineto{\pgfqpoint{4.532505in}{2.714811in}}%
\pgfpathlineto{\pgfqpoint{4.519166in}{2.718283in}}%
\pgfpathlineto{\pgfqpoint{4.505832in}{2.721783in}}%
\pgfpathlineto{\pgfqpoint{4.492505in}{2.725311in}}%
\pgfpathlineto{\pgfqpoint{4.479184in}{2.728867in}}%
\pgfpathlineto{\pgfqpoint{4.486679in}{2.736791in}}%
\pgfpathlineto{\pgfqpoint{4.494169in}{2.744770in}}%
\pgfpathlineto{\pgfqpoint{4.501654in}{2.752810in}}%
\pgfpathlineto{\pgfqpoint{4.509135in}{2.760912in}}%
\pgfpathclose%
\pgfusepath{fill}%
\end{pgfscope}%
\begin{pgfscope}%
\pgfpathrectangle{\pgfqpoint{1.150000in}{0.150000in}}{\pgfqpoint{5.700000in}{5.700000in}}%
\pgfusepath{clip}%
\pgfsetbuttcap%
\pgfsetroundjoin%
\definecolor{currentfill}{rgb}{0.276022,0.044167,0.370164}%
\pgfsetfillcolor{currentfill}%
\pgfsetfillopacity{0.700000}%
\pgfsetlinewidth{0.000000pt}%
\definecolor{currentstroke}{rgb}{0.000000,0.000000,0.000000}%
\pgfsetstrokecolor{currentstroke}%
\pgfsetdash{}{0pt}%
\pgfpathmoveto{\pgfqpoint{3.169451in}{2.761627in}}%
\pgfpathlineto{\pgfqpoint{3.182504in}{2.755754in}}%
\pgfpathlineto{\pgfqpoint{3.195562in}{2.749928in}}%
\pgfpathlineto{\pgfqpoint{3.208623in}{2.744150in}}%
\pgfpathlineto{\pgfqpoint{3.221688in}{2.738418in}}%
\pgfpathlineto{\pgfqpoint{3.213742in}{2.731199in}}%
\pgfpathlineto{\pgfqpoint{3.205789in}{2.724023in}}%
\pgfpathlineto{\pgfqpoint{3.197830in}{2.716889in}}%
\pgfpathlineto{\pgfqpoint{3.189864in}{2.709798in}}%
\pgfpathlineto{\pgfqpoint{3.176785in}{2.715560in}}%
\pgfpathlineto{\pgfqpoint{3.163710in}{2.721369in}}%
\pgfpathlineto{\pgfqpoint{3.150639in}{2.727224in}}%
\pgfpathlineto{\pgfqpoint{3.137572in}{2.733128in}}%
\pgfpathlineto{\pgfqpoint{3.145552in}{2.740183in}}%
\pgfpathlineto{\pgfqpoint{3.153525in}{2.747285in}}%
\pgfpathlineto{\pgfqpoint{3.161491in}{2.754433in}}%
\pgfpathlineto{\pgfqpoint{3.169451in}{2.761627in}}%
\pgfpathclose%
\pgfusepath{fill}%
\end{pgfscope}%
\begin{pgfscope}%
\pgfpathrectangle{\pgfqpoint{1.150000in}{0.150000in}}{\pgfqpoint{5.700000in}{5.700000in}}%
\pgfusepath{clip}%
\pgfsetbuttcap%
\pgfsetroundjoin%
\definecolor{currentfill}{rgb}{0.273809,0.031497,0.358853}%
\pgfsetfillcolor{currentfill}%
\pgfsetfillopacity{0.700000}%
\pgfsetlinewidth{0.000000pt}%
\definecolor{currentstroke}{rgb}{0.000000,0.000000,0.000000}%
\pgfsetstrokecolor{currentstroke}%
\pgfsetdash{}{0pt}%
\pgfpathmoveto{\pgfqpoint{3.305655in}{2.745317in}}%
\pgfpathlineto{\pgfqpoint{3.318728in}{2.739829in}}%
\pgfpathlineto{\pgfqpoint{3.331804in}{2.734385in}}%
\pgfpathlineto{\pgfqpoint{3.344884in}{2.728985in}}%
\pgfpathlineto{\pgfqpoint{3.357969in}{2.723628in}}%
\pgfpathlineto{\pgfqpoint{3.350075in}{2.716216in}}%
\pgfpathlineto{\pgfqpoint{3.342174in}{2.708839in}}%
\pgfpathlineto{\pgfqpoint{3.334267in}{2.701498in}}%
\pgfpathlineto{\pgfqpoint{3.326354in}{2.694193in}}%
\pgfpathlineto{\pgfqpoint{3.313256in}{2.699567in}}%
\pgfpathlineto{\pgfqpoint{3.300162in}{2.704984in}}%
\pgfpathlineto{\pgfqpoint{3.287073in}{2.710445in}}%
\pgfpathlineto{\pgfqpoint{3.273988in}{2.715949in}}%
\pgfpathlineto{\pgfqpoint{3.281914in}{2.723233in}}%
\pgfpathlineto{\pgfqpoint{3.289834in}{2.730555in}}%
\pgfpathlineto{\pgfqpoint{3.297748in}{2.737916in}}%
\pgfpathlineto{\pgfqpoint{3.305655in}{2.745317in}}%
\pgfpathclose%
\pgfusepath{fill}%
\end{pgfscope}%
\begin{pgfscope}%
\pgfpathrectangle{\pgfqpoint{1.150000in}{0.150000in}}{\pgfqpoint{5.700000in}{5.700000in}}%
\pgfusepath{clip}%
\pgfsetbuttcap%
\pgfsetroundjoin%
\definecolor{currentfill}{rgb}{0.272594,0.025563,0.353093}%
\pgfsetfillcolor{currentfill}%
\pgfsetfillopacity{0.700000}%
\pgfsetlinewidth{0.000000pt}%
\definecolor{currentstroke}{rgb}{0.000000,0.000000,0.000000}%
\pgfsetstrokecolor{currentstroke}%
\pgfsetdash{}{0pt}%
\pgfpathmoveto{\pgfqpoint{3.933830in}{2.726320in}}%
\pgfpathlineto{\pgfqpoint{3.947015in}{2.722102in}}%
\pgfpathlineto{\pgfqpoint{3.960206in}{2.717918in}}%
\pgfpathlineto{\pgfqpoint{3.973402in}{2.713767in}}%
\pgfpathlineto{\pgfqpoint{3.986603in}{2.709649in}}%
\pgfpathlineto{\pgfqpoint{3.978934in}{2.701830in}}%
\pgfpathlineto{\pgfqpoint{3.971260in}{2.694039in}}%
\pgfpathlineto{\pgfqpoint{3.963580in}{2.686275in}}%
\pgfpathlineto{\pgfqpoint{3.955894in}{2.678535in}}%
\pgfpathlineto{\pgfqpoint{3.942681in}{2.682604in}}%
\pgfpathlineto{\pgfqpoint{3.929473in}{2.686707in}}%
\pgfpathlineto{\pgfqpoint{3.916271in}{2.690843in}}%
\pgfpathlineto{\pgfqpoint{3.903074in}{2.695012in}}%
\pgfpathlineto{\pgfqpoint{3.910771in}{2.702795in}}%
\pgfpathlineto{\pgfqpoint{3.918463in}{2.710607in}}%
\pgfpathlineto{\pgfqpoint{3.926149in}{2.718448in}}%
\pgfpathlineto{\pgfqpoint{3.933830in}{2.726320in}}%
\pgfpathclose%
\pgfusepath{fill}%
\end{pgfscope}%
\begin{pgfscope}%
\pgfpathrectangle{\pgfqpoint{1.150000in}{0.150000in}}{\pgfqpoint{5.700000in}{5.700000in}}%
\pgfusepath{clip}%
\pgfsetbuttcap%
\pgfsetroundjoin%
\definecolor{currentfill}{rgb}{0.277941,0.056324,0.381191}%
\pgfsetfillcolor{currentfill}%
\pgfsetfillopacity{0.700000}%
\pgfsetlinewidth{0.000000pt}%
\definecolor{currentstroke}{rgb}{0.000000,0.000000,0.000000}%
\pgfsetstrokecolor{currentstroke}%
\pgfsetdash{}{0pt}%
\pgfpathmoveto{\pgfqpoint{3.033164in}{2.782124in}}%
\pgfpathlineto{\pgfqpoint{3.046203in}{2.775823in}}%
\pgfpathlineto{\pgfqpoint{3.059245in}{2.769573in}}%
\pgfpathlineto{\pgfqpoint{3.072291in}{2.763374in}}%
\pgfpathlineto{\pgfqpoint{3.085340in}{2.757226in}}%
\pgfpathlineto{\pgfqpoint{3.077339in}{2.750255in}}%
\pgfpathlineto{\pgfqpoint{3.069332in}{2.743335in}}%
\pgfpathlineto{\pgfqpoint{3.061317in}{2.736466in}}%
\pgfpathlineto{\pgfqpoint{3.053296in}{2.729649in}}%
\pgfpathlineto{\pgfqpoint{3.040232in}{2.735840in}}%
\pgfpathlineto{\pgfqpoint{3.027172in}{2.742083in}}%
\pgfpathlineto{\pgfqpoint{3.014116in}{2.748376in}}%
\pgfpathlineto{\pgfqpoint{3.001062in}{2.754721in}}%
\pgfpathlineto{\pgfqpoint{3.009099in}{2.761490in}}%
\pgfpathlineto{\pgfqpoint{3.017128in}{2.768314in}}%
\pgfpathlineto{\pgfqpoint{3.025149in}{2.775192in}}%
\pgfpathlineto{\pgfqpoint{3.033164in}{2.782124in}}%
\pgfpathclose%
\pgfusepath{fill}%
\end{pgfscope}%
\begin{pgfscope}%
\pgfpathrectangle{\pgfqpoint{1.150000in}{0.150000in}}{\pgfqpoint{5.700000in}{5.700000in}}%
\pgfusepath{clip}%
\pgfsetbuttcap%
\pgfsetroundjoin%
\definecolor{currentfill}{rgb}{0.272594,0.025563,0.353093}%
\pgfsetfillcolor{currentfill}%
\pgfsetfillopacity{0.700000}%
\pgfsetlinewidth{0.000000pt}%
\definecolor{currentstroke}{rgb}{0.000000,0.000000,0.000000}%
\pgfsetstrokecolor{currentstroke}%
\pgfsetdash{}{0pt}%
\pgfpathmoveto{\pgfqpoint{3.441816in}{2.732658in}}%
\pgfpathlineto{\pgfqpoint{3.454910in}{2.727516in}}%
\pgfpathlineto{\pgfqpoint{3.468009in}{2.722415in}}%
\pgfpathlineto{\pgfqpoint{3.481112in}{2.717355in}}%
\pgfpathlineto{\pgfqpoint{3.494220in}{2.712336in}}%
\pgfpathlineto{\pgfqpoint{3.486375in}{2.704778in}}%
\pgfpathlineto{\pgfqpoint{3.478524in}{2.697252in}}%
\pgfpathlineto{\pgfqpoint{3.470667in}{2.689756in}}%
\pgfpathlineto{\pgfqpoint{3.462805in}{2.682290in}}%
\pgfpathlineto{\pgfqpoint{3.449684in}{2.687313in}}%
\pgfpathlineto{\pgfqpoint{3.436569in}{2.692377in}}%
\pgfpathlineto{\pgfqpoint{3.423458in}{2.697481in}}%
\pgfpathlineto{\pgfqpoint{3.410351in}{2.702627in}}%
\pgfpathlineto{\pgfqpoint{3.418227in}{2.710084in}}%
\pgfpathlineto{\pgfqpoint{3.426096in}{2.717575in}}%
\pgfpathlineto{\pgfqpoint{3.433959in}{2.725099in}}%
\pgfpathlineto{\pgfqpoint{3.441816in}{2.732658in}}%
\pgfpathclose%
\pgfusepath{fill}%
\end{pgfscope}%
\begin{pgfscope}%
\pgfpathrectangle{\pgfqpoint{1.150000in}{0.150000in}}{\pgfqpoint{5.700000in}{5.700000in}}%
\pgfusepath{clip}%
\pgfsetbuttcap%
\pgfsetroundjoin%
\definecolor{currentfill}{rgb}{0.274952,0.037752,0.364543}%
\pgfsetfillcolor{currentfill}%
\pgfsetfillopacity{0.700000}%
\pgfsetlinewidth{0.000000pt}%
\definecolor{currentstroke}{rgb}{0.000000,0.000000,0.000000}%
\pgfsetstrokecolor{currentstroke}%
\pgfsetdash{}{0pt}%
\pgfpathmoveto{\pgfqpoint{4.289623in}{2.741397in}}%
\pgfpathlineto{\pgfqpoint{4.302885in}{2.737590in}}%
\pgfpathlineto{\pgfqpoint{4.316152in}{2.733813in}}%
\pgfpathlineto{\pgfqpoint{4.329426in}{2.730066in}}%
\pgfpathlineto{\pgfqpoint{4.342705in}{2.726349in}}%
\pgfpathlineto{\pgfqpoint{4.335160in}{2.718468in}}%
\pgfpathlineto{\pgfqpoint{4.327611in}{2.710630in}}%
\pgfpathlineto{\pgfqpoint{4.320056in}{2.702832in}}%
\pgfpathlineto{\pgfqpoint{4.312496in}{2.695070in}}%
\pgfpathlineto{\pgfqpoint{4.299204in}{2.698699in}}%
\pgfpathlineto{\pgfqpoint{4.285918in}{2.702358in}}%
\pgfpathlineto{\pgfqpoint{4.272638in}{2.706048in}}%
\pgfpathlineto{\pgfqpoint{4.259364in}{2.709767in}}%
\pgfpathlineto{\pgfqpoint{4.266936in}{2.717612in}}%
\pgfpathlineto{\pgfqpoint{4.274503in}{2.725496in}}%
\pgfpathlineto{\pgfqpoint{4.282066in}{2.733424in}}%
\pgfpathlineto{\pgfqpoint{4.289623in}{2.741397in}}%
\pgfpathclose%
\pgfusepath{fill}%
\end{pgfscope}%
\begin{pgfscope}%
\pgfpathrectangle{\pgfqpoint{1.150000in}{0.150000in}}{\pgfqpoint{5.700000in}{5.700000in}}%
\pgfusepath{clip}%
\pgfsetbuttcap%
\pgfsetroundjoin%
\definecolor{currentfill}{rgb}{0.272594,0.025563,0.353093}%
\pgfsetfillcolor{currentfill}%
\pgfsetfillopacity{0.700000}%
\pgfsetlinewidth{0.000000pt}%
\definecolor{currentstroke}{rgb}{0.000000,0.000000,0.000000}%
\pgfsetstrokecolor{currentstroke}%
\pgfsetdash{}{0pt}%
\pgfpathmoveto{\pgfqpoint{3.577967in}{2.723170in}}%
\pgfpathlineto{\pgfqpoint{3.591085in}{2.718338in}}%
\pgfpathlineto{\pgfqpoint{3.604209in}{2.713545in}}%
\pgfpathlineto{\pgfqpoint{3.617337in}{2.708791in}}%
\pgfpathlineto{\pgfqpoint{3.630471in}{2.704074in}}%
\pgfpathlineto{\pgfqpoint{3.622674in}{2.696415in}}%
\pgfpathlineto{\pgfqpoint{3.614872in}{2.688783in}}%
\pgfpathlineto{\pgfqpoint{3.607064in}{2.681177in}}%
\pgfpathlineto{\pgfqpoint{3.599250in}{2.673597in}}%
\pgfpathlineto{\pgfqpoint{3.586104in}{2.678304in}}%
\pgfpathlineto{\pgfqpoint{3.572963in}{2.683050in}}%
\pgfpathlineto{\pgfqpoint{3.559828in}{2.687833in}}%
\pgfpathlineto{\pgfqpoint{3.546697in}{2.692656in}}%
\pgfpathlineto{\pgfqpoint{3.554523in}{2.700240in}}%
\pgfpathlineto{\pgfqpoint{3.562343in}{2.707853in}}%
\pgfpathlineto{\pgfqpoint{3.570158in}{2.715496in}}%
\pgfpathlineto{\pgfqpoint{3.577967in}{2.723170in}}%
\pgfpathclose%
\pgfusepath{fill}%
\end{pgfscope}%
\begin{pgfscope}%
\pgfpathrectangle{\pgfqpoint{1.150000in}{0.150000in}}{\pgfqpoint{5.700000in}{5.700000in}}%
\pgfusepath{clip}%
\pgfsetbuttcap%
\pgfsetroundjoin%
\definecolor{currentfill}{rgb}{0.279566,0.067836,0.391917}%
\pgfsetfillcolor{currentfill}%
\pgfsetfillopacity{0.700000}%
\pgfsetlinewidth{0.000000pt}%
\definecolor{currentstroke}{rgb}{0.000000,0.000000,0.000000}%
\pgfsetstrokecolor{currentstroke}%
\pgfsetdash{}{0pt}%
\pgfpathmoveto{\pgfqpoint{4.865102in}{2.788081in}}%
\pgfpathlineto{\pgfqpoint{4.878494in}{2.784540in}}%
\pgfpathlineto{\pgfqpoint{4.891892in}{2.781025in}}%
\pgfpathlineto{\pgfqpoint{4.905296in}{2.777537in}}%
\pgfpathlineto{\pgfqpoint{4.918707in}{2.774075in}}%
\pgfpathlineto{\pgfqpoint{4.911357in}{2.765828in}}%
\pgfpathlineto{\pgfqpoint{4.904003in}{2.757677in}}%
\pgfpathlineto{\pgfqpoint{4.896646in}{2.749619in}}%
\pgfpathlineto{\pgfqpoint{4.889286in}{2.741646in}}%
\pgfpathlineto{\pgfqpoint{4.875861in}{2.744956in}}%
\pgfpathlineto{\pgfqpoint{4.862442in}{2.748292in}}%
\pgfpathlineto{\pgfqpoint{4.849030in}{2.751654in}}%
\pgfpathlineto{\pgfqpoint{4.835624in}{2.755043in}}%
\pgfpathlineto{\pgfqpoint{4.842999in}{2.763162in}}%
\pgfpathlineto{\pgfqpoint{4.850370in}{2.771372in}}%
\pgfpathlineto{\pgfqpoint{4.857738in}{2.779677in}}%
\pgfpathlineto{\pgfqpoint{4.865102in}{2.788081in}}%
\pgfpathclose%
\pgfusepath{fill}%
\end{pgfscope}%
\begin{pgfscope}%
\pgfpathrectangle{\pgfqpoint{1.150000in}{0.150000in}}{\pgfqpoint{5.700000in}{5.700000in}}%
\pgfusepath{clip}%
\pgfsetbuttcap%
\pgfsetroundjoin%
\definecolor{currentfill}{rgb}{0.277018,0.050344,0.375715}%
\pgfsetfillcolor{currentfill}%
\pgfsetfillopacity{0.700000}%
\pgfsetlinewidth{0.000000pt}%
\definecolor{currentstroke}{rgb}{0.000000,0.000000,0.000000}%
\pgfsetstrokecolor{currentstroke}%
\pgfsetdash{}{0pt}%
\pgfpathmoveto{\pgfqpoint{4.645543in}{2.764450in}}%
\pgfpathlineto{\pgfqpoint{4.658887in}{2.760877in}}%
\pgfpathlineto{\pgfqpoint{4.672237in}{2.757332in}}%
\pgfpathlineto{\pgfqpoint{4.685593in}{2.753814in}}%
\pgfpathlineto{\pgfqpoint{4.698956in}{2.750323in}}%
\pgfpathlineto{\pgfqpoint{4.691533in}{2.742322in}}%
\pgfpathlineto{\pgfqpoint{4.684105in}{2.734390in}}%
\pgfpathlineto{\pgfqpoint{4.676674in}{2.726524in}}%
\pgfpathlineto{\pgfqpoint{4.669237in}{2.718719in}}%
\pgfpathlineto{\pgfqpoint{4.655861in}{2.722083in}}%
\pgfpathlineto{\pgfqpoint{4.642491in}{2.725474in}}%
\pgfpathlineto{\pgfqpoint{4.629128in}{2.728893in}}%
\pgfpathlineto{\pgfqpoint{4.615770in}{2.732339in}}%
\pgfpathlineto{\pgfqpoint{4.623220in}{2.740265in}}%
\pgfpathlineto{\pgfqpoint{4.630665in}{2.748257in}}%
\pgfpathlineto{\pgfqpoint{4.638106in}{2.756317in}}%
\pgfpathlineto{\pgfqpoint{4.645543in}{2.764450in}}%
\pgfpathclose%
\pgfusepath{fill}%
\end{pgfscope}%
\begin{pgfscope}%
\pgfpathrectangle{\pgfqpoint{1.150000in}{0.150000in}}{\pgfqpoint{5.700000in}{5.700000in}}%
\pgfusepath{clip}%
\pgfsetbuttcap%
\pgfsetroundjoin%
\definecolor{currentfill}{rgb}{0.272594,0.025563,0.353093}%
\pgfsetfillcolor{currentfill}%
\pgfsetfillopacity{0.700000}%
\pgfsetlinewidth{0.000000pt}%
\definecolor{currentstroke}{rgb}{0.000000,0.000000,0.000000}%
\pgfsetstrokecolor{currentstroke}%
\pgfsetdash{}{0pt}%
\pgfpathmoveto{\pgfqpoint{4.070038in}{2.724859in}}%
\pgfpathlineto{\pgfqpoint{4.083255in}{2.720843in}}%
\pgfpathlineto{\pgfqpoint{4.096478in}{2.716858in}}%
\pgfpathlineto{\pgfqpoint{4.109706in}{2.712905in}}%
\pgfpathlineto{\pgfqpoint{4.122940in}{2.708983in}}%
\pgfpathlineto{\pgfqpoint{4.115317in}{2.701163in}}%
\pgfpathlineto{\pgfqpoint{4.107689in}{2.693373in}}%
\pgfpathlineto{\pgfqpoint{4.100055in}{2.685611in}}%
\pgfpathlineto{\pgfqpoint{4.092415in}{2.677875in}}%
\pgfpathlineto{\pgfqpoint{4.079169in}{2.681735in}}%
\pgfpathlineto{\pgfqpoint{4.065928in}{2.685626in}}%
\pgfpathlineto{\pgfqpoint{4.052693in}{2.689549in}}%
\pgfpathlineto{\pgfqpoint{4.039464in}{2.693505in}}%
\pgfpathlineto{\pgfqpoint{4.047116in}{2.701297in}}%
\pgfpathlineto{\pgfqpoint{4.054762in}{2.709119in}}%
\pgfpathlineto{\pgfqpoint{4.062402in}{2.716973in}}%
\pgfpathlineto{\pgfqpoint{4.070038in}{2.724859in}}%
\pgfpathclose%
\pgfusepath{fill}%
\end{pgfscope}%
\begin{pgfscope}%
\pgfpathrectangle{\pgfqpoint{1.150000in}{0.150000in}}{\pgfqpoint{5.700000in}{5.700000in}}%
\pgfusepath{clip}%
\pgfsetbuttcap%
\pgfsetroundjoin%
\definecolor{currentfill}{rgb}{0.271305,0.019942,0.347269}%
\pgfsetfillcolor{currentfill}%
\pgfsetfillopacity{0.700000}%
\pgfsetlinewidth{0.000000pt}%
\definecolor{currentstroke}{rgb}{0.000000,0.000000,0.000000}%
\pgfsetstrokecolor{currentstroke}%
\pgfsetdash{}{0pt}%
\pgfpathmoveto{\pgfqpoint{3.714133in}{2.716421in}}%
\pgfpathlineto{\pgfqpoint{3.727279in}{2.711868in}}%
\pgfpathlineto{\pgfqpoint{3.740430in}{2.707351in}}%
\pgfpathlineto{\pgfqpoint{3.753587in}{2.702869in}}%
\pgfpathlineto{\pgfqpoint{3.766748in}{2.698424in}}%
\pgfpathlineto{\pgfqpoint{3.758999in}{2.690700in}}%
\pgfpathlineto{\pgfqpoint{3.751244in}{2.683001in}}%
\pgfpathlineto{\pgfqpoint{3.743483in}{2.675326in}}%
\pgfpathlineto{\pgfqpoint{3.735717in}{2.667674in}}%
\pgfpathlineto{\pgfqpoint{3.722543in}{2.672096in}}%
\pgfpathlineto{\pgfqpoint{3.709375in}{2.676555in}}%
\pgfpathlineto{\pgfqpoint{3.696212in}{2.681050in}}%
\pgfpathlineto{\pgfqpoint{3.683053in}{2.685581in}}%
\pgfpathlineto{\pgfqpoint{3.690832in}{2.693251in}}%
\pgfpathlineto{\pgfqpoint{3.698605in}{2.700947in}}%
\pgfpathlineto{\pgfqpoint{3.706372in}{2.708670in}}%
\pgfpathlineto{\pgfqpoint{3.714133in}{2.716421in}}%
\pgfpathclose%
\pgfusepath{fill}%
\end{pgfscope}%
\begin{pgfscope}%
\pgfpathrectangle{\pgfqpoint{1.150000in}{0.150000in}}{\pgfqpoint{5.700000in}{5.700000in}}%
\pgfusepath{clip}%
\pgfsetbuttcap%
\pgfsetroundjoin%
\definecolor{currentfill}{rgb}{0.274952,0.037752,0.364543}%
\pgfsetfillcolor{currentfill}%
\pgfsetfillopacity{0.700000}%
\pgfsetlinewidth{0.000000pt}%
\definecolor{currentstroke}{rgb}{0.000000,0.000000,0.000000}%
\pgfsetstrokecolor{currentstroke}%
\pgfsetdash{}{0pt}%
\pgfpathmoveto{\pgfqpoint{4.425961in}{2.743380in}}%
\pgfpathlineto{\pgfqpoint{4.439258in}{2.739709in}}%
\pgfpathlineto{\pgfqpoint{4.452561in}{2.736066in}}%
\pgfpathlineto{\pgfqpoint{4.465869in}{2.732453in}}%
\pgfpathlineto{\pgfqpoint{4.479184in}{2.728867in}}%
\pgfpathlineto{\pgfqpoint{4.471685in}{2.720996in}}%
\pgfpathlineto{\pgfqpoint{4.464180in}{2.713174in}}%
\pgfpathlineto{\pgfqpoint{4.456671in}{2.705398in}}%
\pgfpathlineto{\pgfqpoint{4.449157in}{2.697664in}}%
\pgfpathlineto{\pgfqpoint{4.435829in}{2.701148in}}%
\pgfpathlineto{\pgfqpoint{4.422508in}{2.704661in}}%
\pgfpathlineto{\pgfqpoint{4.409192in}{2.708203in}}%
\pgfpathlineto{\pgfqpoint{4.395883in}{2.711774in}}%
\pgfpathlineto{\pgfqpoint{4.403410in}{2.719604in}}%
\pgfpathlineto{\pgfqpoint{4.410932in}{2.727479in}}%
\pgfpathlineto{\pgfqpoint{4.418449in}{2.735404in}}%
\pgfpathlineto{\pgfqpoint{4.425961in}{2.743380in}}%
\pgfpathclose%
\pgfusepath{fill}%
\end{pgfscope}%
\begin{pgfscope}%
\pgfpathrectangle{\pgfqpoint{1.150000in}{0.150000in}}{\pgfqpoint{5.700000in}{5.700000in}}%
\pgfusepath{clip}%
\pgfsetbuttcap%
\pgfsetroundjoin%
\definecolor{currentfill}{rgb}{0.271305,0.019942,0.347269}%
\pgfsetfillcolor{currentfill}%
\pgfsetfillopacity{0.700000}%
\pgfsetlinewidth{0.000000pt}%
\definecolor{currentstroke}{rgb}{0.000000,0.000000,0.000000}%
\pgfsetstrokecolor{currentstroke}%
\pgfsetdash{}{0pt}%
\pgfpathmoveto{\pgfqpoint{3.850339in}{2.712028in}}%
\pgfpathlineto{\pgfqpoint{3.863515in}{2.707723in}}%
\pgfpathlineto{\pgfqpoint{3.876696in}{2.703452in}}%
\pgfpathlineto{\pgfqpoint{3.889882in}{2.699215in}}%
\pgfpathlineto{\pgfqpoint{3.903074in}{2.695012in}}%
\pgfpathlineto{\pgfqpoint{3.895371in}{2.687254in}}%
\pgfpathlineto{\pgfqpoint{3.887662in}{2.679521in}}%
\pgfpathlineto{\pgfqpoint{3.879948in}{2.671811in}}%
\pgfpathlineto{\pgfqpoint{3.872228in}{2.664122in}}%
\pgfpathlineto{\pgfqpoint{3.859025in}{2.668289in}}%
\pgfpathlineto{\pgfqpoint{3.845827in}{2.672490in}}%
\pgfpathlineto{\pgfqpoint{3.832634in}{2.676726in}}%
\pgfpathlineto{\pgfqpoint{3.819446in}{2.680996in}}%
\pgfpathlineto{\pgfqpoint{3.827178in}{2.688716in}}%
\pgfpathlineto{\pgfqpoint{3.834904in}{2.696460in}}%
\pgfpathlineto{\pgfqpoint{3.842624in}{2.704230in}}%
\pgfpathlineto{\pgfqpoint{3.850339in}{2.712028in}}%
\pgfpathclose%
\pgfusepath{fill}%
\end{pgfscope}%
\begin{pgfscope}%
\pgfpathrectangle{\pgfqpoint{1.150000in}{0.150000in}}{\pgfqpoint{5.700000in}{5.700000in}}%
\pgfusepath{clip}%
\pgfsetbuttcap%
\pgfsetroundjoin%
\definecolor{currentfill}{rgb}{0.280267,0.073417,0.397163}%
\pgfsetfillcolor{currentfill}%
\pgfsetfillopacity{0.700000}%
\pgfsetlinewidth{0.000000pt}%
\definecolor{currentstroke}{rgb}{0.000000,0.000000,0.000000}%
\pgfsetstrokecolor{currentstroke}%
\pgfsetdash{}{0pt}%
\pgfpathmoveto{\pgfqpoint{5.001722in}{2.793877in}}%
\pgfpathlineto{\pgfqpoint{5.015150in}{2.790380in}}%
\pgfpathlineto{\pgfqpoint{5.028585in}{2.786908in}}%
\pgfpathlineto{\pgfqpoint{5.042026in}{2.783462in}}%
\pgfpathlineto{\pgfqpoint{5.055473in}{2.780041in}}%
\pgfpathlineto{\pgfqpoint{5.048166in}{2.771698in}}%
\pgfpathlineto{\pgfqpoint{5.040855in}{2.763466in}}%
\pgfpathlineto{\pgfqpoint{5.033541in}{2.755339in}}%
\pgfpathlineto{\pgfqpoint{5.026224in}{2.747313in}}%
\pgfpathlineto{\pgfqpoint{5.012762in}{2.750568in}}%
\pgfpathlineto{\pgfqpoint{4.999306in}{2.753848in}}%
\pgfpathlineto{\pgfqpoint{4.985856in}{2.757155in}}%
\pgfpathlineto{\pgfqpoint{4.972413in}{2.760487in}}%
\pgfpathlineto{\pgfqpoint{4.979745in}{2.768674in}}%
\pgfpathlineto{\pgfqpoint{4.987074in}{2.776965in}}%
\pgfpathlineto{\pgfqpoint{4.994400in}{2.785365in}}%
\pgfpathlineto{\pgfqpoint{5.001722in}{2.793877in}}%
\pgfpathclose%
\pgfusepath{fill}%
\end{pgfscope}%
\begin{pgfscope}%
\pgfpathrectangle{\pgfqpoint{1.150000in}{0.150000in}}{\pgfqpoint{5.700000in}{5.700000in}}%
\pgfusepath{clip}%
\pgfsetbuttcap%
\pgfsetroundjoin%
\definecolor{currentfill}{rgb}{0.273809,0.031497,0.358853}%
\pgfsetfillcolor{currentfill}%
\pgfsetfillopacity{0.700000}%
\pgfsetlinewidth{0.000000pt}%
\definecolor{currentstroke}{rgb}{0.000000,0.000000,0.000000}%
\pgfsetstrokecolor{currentstroke}%
\pgfsetdash{}{0pt}%
\pgfpathmoveto{\pgfqpoint{4.206326in}{2.724947in}}%
\pgfpathlineto{\pgfqpoint{4.219576in}{2.721106in}}%
\pgfpathlineto{\pgfqpoint{4.232833in}{2.717296in}}%
\pgfpathlineto{\pgfqpoint{4.246095in}{2.713516in}}%
\pgfpathlineto{\pgfqpoint{4.259364in}{2.709767in}}%
\pgfpathlineto{\pgfqpoint{4.251786in}{2.701959in}}%
\pgfpathlineto{\pgfqpoint{4.244203in}{2.694185in}}%
\pgfpathlineto{\pgfqpoint{4.236615in}{2.686442in}}%
\pgfpathlineto{\pgfqpoint{4.229022in}{2.678729in}}%
\pgfpathlineto{\pgfqpoint{4.215741in}{2.682403in}}%
\pgfpathlineto{\pgfqpoint{4.202466in}{2.686108in}}%
\pgfpathlineto{\pgfqpoint{4.189197in}{2.689843in}}%
\pgfpathlineto{\pgfqpoint{4.175934in}{2.693609in}}%
\pgfpathlineto{\pgfqpoint{4.183540in}{2.701393in}}%
\pgfpathlineto{\pgfqpoint{4.191141in}{2.709209in}}%
\pgfpathlineto{\pgfqpoint{4.198736in}{2.717059in}}%
\pgfpathlineto{\pgfqpoint{4.206326in}{2.724947in}}%
\pgfpathclose%
\pgfusepath{fill}%
\end{pgfscope}%
\begin{pgfscope}%
\pgfpathrectangle{\pgfqpoint{1.150000in}{0.150000in}}{\pgfqpoint{5.700000in}{5.700000in}}%
\pgfusepath{clip}%
\pgfsetbuttcap%
\pgfsetroundjoin%
\definecolor{currentfill}{rgb}{0.274952,0.037752,0.364543}%
\pgfsetfillcolor{currentfill}%
\pgfsetfillopacity{0.700000}%
\pgfsetlinewidth{0.000000pt}%
\definecolor{currentstroke}{rgb}{0.000000,0.000000,0.000000}%
\pgfsetstrokecolor{currentstroke}%
\pgfsetdash{}{0pt}%
\pgfpathmoveto{\pgfqpoint{3.221688in}{2.738418in}}%
\pgfpathlineto{\pgfqpoint{3.234757in}{2.732733in}}%
\pgfpathlineto{\pgfqpoint{3.247830in}{2.727093in}}%
\pgfpathlineto{\pgfqpoint{3.260907in}{2.721499in}}%
\pgfpathlineto{\pgfqpoint{3.273988in}{2.715949in}}%
\pgfpathlineto{\pgfqpoint{3.266055in}{2.708705in}}%
\pgfpathlineto{\pgfqpoint{3.258116in}{2.701500in}}%
\pgfpathlineto{\pgfqpoint{3.250170in}{2.694335in}}%
\pgfpathlineto{\pgfqpoint{3.242218in}{2.687209in}}%
\pgfpathlineto{\pgfqpoint{3.229123in}{2.692789in}}%
\pgfpathlineto{\pgfqpoint{3.216033in}{2.698413in}}%
\pgfpathlineto{\pgfqpoint{3.202946in}{2.704083in}}%
\pgfpathlineto{\pgfqpoint{3.189864in}{2.709798in}}%
\pgfpathlineto{\pgfqpoint{3.197830in}{2.716889in}}%
\pgfpathlineto{\pgfqpoint{3.205789in}{2.724023in}}%
\pgfpathlineto{\pgfqpoint{3.213742in}{2.731199in}}%
\pgfpathlineto{\pgfqpoint{3.221688in}{2.738418in}}%
\pgfpathclose%
\pgfusepath{fill}%
\end{pgfscope}%
\begin{pgfscope}%
\pgfpathrectangle{\pgfqpoint{1.150000in}{0.150000in}}{\pgfqpoint{5.700000in}{5.700000in}}%
\pgfusepath{clip}%
\pgfsetbuttcap%
\pgfsetroundjoin%
\definecolor{currentfill}{rgb}{0.278791,0.062145,0.386592}%
\pgfsetfillcolor{currentfill}%
\pgfsetfillopacity{0.700000}%
\pgfsetlinewidth{0.000000pt}%
\definecolor{currentstroke}{rgb}{0.000000,0.000000,0.000000}%
\pgfsetstrokecolor{currentstroke}%
\pgfsetdash{}{0pt}%
\pgfpathmoveto{\pgfqpoint{4.782065in}{2.768863in}}%
\pgfpathlineto{\pgfqpoint{4.795445in}{2.765368in}}%
\pgfpathlineto{\pgfqpoint{4.808832in}{2.761899in}}%
\pgfpathlineto{\pgfqpoint{4.822225in}{2.758458in}}%
\pgfpathlineto{\pgfqpoint{4.835624in}{2.755043in}}%
\pgfpathlineto{\pgfqpoint{4.828245in}{2.747008in}}%
\pgfpathlineto{\pgfqpoint{4.820862in}{2.739054in}}%
\pgfpathlineto{\pgfqpoint{4.813476in}{2.731177in}}%
\pgfpathlineto{\pgfqpoint{4.806085in}{2.723371in}}%
\pgfpathlineto{\pgfqpoint{4.792671in}{2.726646in}}%
\pgfpathlineto{\pgfqpoint{4.779264in}{2.729948in}}%
\pgfpathlineto{\pgfqpoint{4.765863in}{2.733277in}}%
\pgfpathlineto{\pgfqpoint{4.752469in}{2.736632in}}%
\pgfpathlineto{\pgfqpoint{4.759874in}{2.744573in}}%
\pgfpathlineto{\pgfqpoint{4.767275in}{2.752588in}}%
\pgfpathlineto{\pgfqpoint{4.774672in}{2.760684in}}%
\pgfpathlineto{\pgfqpoint{4.782065in}{2.768863in}}%
\pgfpathclose%
\pgfusepath{fill}%
\end{pgfscope}%
\begin{pgfscope}%
\pgfpathrectangle{\pgfqpoint{1.150000in}{0.150000in}}{\pgfqpoint{5.700000in}{5.700000in}}%
\pgfusepath{clip}%
\pgfsetbuttcap%
\pgfsetroundjoin%
\definecolor{currentfill}{rgb}{0.273809,0.031497,0.358853}%
\pgfsetfillcolor{currentfill}%
\pgfsetfillopacity{0.700000}%
\pgfsetlinewidth{0.000000pt}%
\definecolor{currentstroke}{rgb}{0.000000,0.000000,0.000000}%
\pgfsetstrokecolor{currentstroke}%
\pgfsetdash{}{0pt}%
\pgfpathmoveto{\pgfqpoint{3.357969in}{2.723628in}}%
\pgfpathlineto{\pgfqpoint{3.371058in}{2.718314in}}%
\pgfpathlineto{\pgfqpoint{3.384151in}{2.713043in}}%
\pgfpathlineto{\pgfqpoint{3.397249in}{2.707814in}}%
\pgfpathlineto{\pgfqpoint{3.410351in}{2.702627in}}%
\pgfpathlineto{\pgfqpoint{3.402470in}{2.695202in}}%
\pgfpathlineto{\pgfqpoint{3.394582in}{2.687810in}}%
\pgfpathlineto{\pgfqpoint{3.386688in}{2.680451in}}%
\pgfpathlineto{\pgfqpoint{3.378788in}{2.673124in}}%
\pgfpathlineto{\pgfqpoint{3.365673in}{2.678328in}}%
\pgfpathlineto{\pgfqpoint{3.352562in}{2.683574in}}%
\pgfpathlineto{\pgfqpoint{3.339456in}{2.688862in}}%
\pgfpathlineto{\pgfqpoint{3.326354in}{2.694193in}}%
\pgfpathlineto{\pgfqpoint{3.334267in}{2.701498in}}%
\pgfpathlineto{\pgfqpoint{3.342174in}{2.708839in}}%
\pgfpathlineto{\pgfqpoint{3.350075in}{2.716216in}}%
\pgfpathlineto{\pgfqpoint{3.357969in}{2.723628in}}%
\pgfpathclose%
\pgfusepath{fill}%
\end{pgfscope}%
\begin{pgfscope}%
\pgfpathrectangle{\pgfqpoint{1.150000in}{0.150000in}}{\pgfqpoint{5.700000in}{5.700000in}}%
\pgfusepath{clip}%
\pgfsetbuttcap%
\pgfsetroundjoin%
\definecolor{currentfill}{rgb}{0.277018,0.050344,0.375715}%
\pgfsetfillcolor{currentfill}%
\pgfsetfillopacity{0.700000}%
\pgfsetlinewidth{0.000000pt}%
\definecolor{currentstroke}{rgb}{0.000000,0.000000,0.000000}%
\pgfsetstrokecolor{currentstroke}%
\pgfsetdash{}{0pt}%
\pgfpathmoveto{\pgfqpoint{3.085340in}{2.757226in}}%
\pgfpathlineto{\pgfqpoint{3.098393in}{2.751127in}}%
\pgfpathlineto{\pgfqpoint{3.111449in}{2.745079in}}%
\pgfpathlineto{\pgfqpoint{3.124509in}{2.739079in}}%
\pgfpathlineto{\pgfqpoint{3.137572in}{2.733128in}}%
\pgfpathlineto{\pgfqpoint{3.129585in}{2.726119in}}%
\pgfpathlineto{\pgfqpoint{3.121592in}{2.719157in}}%
\pgfpathlineto{\pgfqpoint{3.113592in}{2.712243in}}%
\pgfpathlineto{\pgfqpoint{3.105585in}{2.705377in}}%
\pgfpathlineto{\pgfqpoint{3.092507in}{2.711372in}}%
\pgfpathlineto{\pgfqpoint{3.079433in}{2.717415in}}%
\pgfpathlineto{\pgfqpoint{3.066363in}{2.723507in}}%
\pgfpathlineto{\pgfqpoint{3.053296in}{2.729649in}}%
\pgfpathlineto{\pgfqpoint{3.061317in}{2.736466in}}%
\pgfpathlineto{\pgfqpoint{3.069332in}{2.743335in}}%
\pgfpathlineto{\pgfqpoint{3.077339in}{2.750255in}}%
\pgfpathlineto{\pgfqpoint{3.085340in}{2.757226in}}%
\pgfpathclose%
\pgfusepath{fill}%
\end{pgfscope}%
\begin{pgfscope}%
\pgfpathrectangle{\pgfqpoint{1.150000in}{0.150000in}}{\pgfqpoint{5.700000in}{5.700000in}}%
\pgfusepath{clip}%
\pgfsetbuttcap%
\pgfsetroundjoin%
\definecolor{currentfill}{rgb}{0.272594,0.025563,0.353093}%
\pgfsetfillcolor{currentfill}%
\pgfsetfillopacity{0.700000}%
\pgfsetlinewidth{0.000000pt}%
\definecolor{currentstroke}{rgb}{0.000000,0.000000,0.000000}%
\pgfsetstrokecolor{currentstroke}%
\pgfsetdash{}{0pt}%
\pgfpathmoveto{\pgfqpoint{3.494220in}{2.712336in}}%
\pgfpathlineto{\pgfqpoint{3.507332in}{2.707356in}}%
\pgfpathlineto{\pgfqpoint{3.520449in}{2.702417in}}%
\pgfpathlineto{\pgfqpoint{3.533570in}{2.697517in}}%
\pgfpathlineto{\pgfqpoint{3.546697in}{2.692656in}}%
\pgfpathlineto{\pgfqpoint{3.538864in}{2.685099in}}%
\pgfpathlineto{\pgfqpoint{3.531026in}{2.677571in}}%
\pgfpathlineto{\pgfqpoint{3.523182in}{2.670069in}}%
\pgfpathlineto{\pgfqpoint{3.515332in}{2.662594in}}%
\pgfpathlineto{\pgfqpoint{3.502193in}{2.667459in}}%
\pgfpathlineto{\pgfqpoint{3.489059in}{2.672363in}}%
\pgfpathlineto{\pgfqpoint{3.475929in}{2.677306in}}%
\pgfpathlineto{\pgfqpoint{3.462805in}{2.682290in}}%
\pgfpathlineto{\pgfqpoint{3.470667in}{2.689756in}}%
\pgfpathlineto{\pgfqpoint{3.478524in}{2.697252in}}%
\pgfpathlineto{\pgfqpoint{3.486375in}{2.704778in}}%
\pgfpathlineto{\pgfqpoint{3.494220in}{2.712336in}}%
\pgfpathclose%
\pgfusepath{fill}%
\end{pgfscope}%
\begin{pgfscope}%
\pgfpathrectangle{\pgfqpoint{1.150000in}{0.150000in}}{\pgfqpoint{5.700000in}{5.700000in}}%
\pgfusepath{clip}%
\pgfsetbuttcap%
\pgfsetroundjoin%
\definecolor{currentfill}{rgb}{0.276022,0.044167,0.370164}%
\pgfsetfillcolor{currentfill}%
\pgfsetfillopacity{0.700000}%
\pgfsetlinewidth{0.000000pt}%
\definecolor{currentstroke}{rgb}{0.000000,0.000000,0.000000}%
\pgfsetstrokecolor{currentstroke}%
\pgfsetdash{}{0pt}%
\pgfpathmoveto{\pgfqpoint{4.562403in}{2.746401in}}%
\pgfpathlineto{\pgfqpoint{4.575736in}{2.742844in}}%
\pgfpathlineto{\pgfqpoint{4.589074in}{2.739314in}}%
\pgfpathlineto{\pgfqpoint{4.602419in}{2.735813in}}%
\pgfpathlineto{\pgfqpoint{4.615770in}{2.732339in}}%
\pgfpathlineto{\pgfqpoint{4.608316in}{2.724473in}}%
\pgfpathlineto{\pgfqpoint{4.600857in}{2.716665in}}%
\pgfpathlineto{\pgfqpoint{4.593394in}{2.708910in}}%
\pgfpathlineto{\pgfqpoint{4.585925in}{2.701204in}}%
\pgfpathlineto{\pgfqpoint{4.572561in}{2.704564in}}%
\pgfpathlineto{\pgfqpoint{4.559203in}{2.707952in}}%
\pgfpathlineto{\pgfqpoint{4.545851in}{2.711367in}}%
\pgfpathlineto{\pgfqpoint{4.532505in}{2.714811in}}%
\pgfpathlineto{\pgfqpoint{4.539987in}{2.722626in}}%
\pgfpathlineto{\pgfqpoint{4.547464in}{2.730493in}}%
\pgfpathlineto{\pgfqpoint{4.554936in}{2.738417in}}%
\pgfpathlineto{\pgfqpoint{4.562403in}{2.746401in}}%
\pgfpathclose%
\pgfusepath{fill}%
\end{pgfscope}%
\begin{pgfscope}%
\pgfpathrectangle{\pgfqpoint{1.150000in}{0.150000in}}{\pgfqpoint{5.700000in}{5.700000in}}%
\pgfusepath{clip}%
\pgfsetbuttcap%
\pgfsetroundjoin%
\definecolor{currentfill}{rgb}{0.272594,0.025563,0.353093}%
\pgfsetfillcolor{currentfill}%
\pgfsetfillopacity{0.700000}%
\pgfsetlinewidth{0.000000pt}%
\definecolor{currentstroke}{rgb}{0.000000,0.000000,0.000000}%
\pgfsetstrokecolor{currentstroke}%
\pgfsetdash{}{0pt}%
\pgfpathmoveto{\pgfqpoint{3.986603in}{2.709649in}}%
\pgfpathlineto{\pgfqpoint{3.999810in}{2.705564in}}%
\pgfpathlineto{\pgfqpoint{4.013022in}{2.701512in}}%
\pgfpathlineto{\pgfqpoint{4.026241in}{2.697492in}}%
\pgfpathlineto{\pgfqpoint{4.039464in}{2.693505in}}%
\pgfpathlineto{\pgfqpoint{4.031807in}{2.685739in}}%
\pgfpathlineto{\pgfqpoint{4.024145in}{2.677999in}}%
\pgfpathlineto{\pgfqpoint{4.016477in}{2.670282in}}%
\pgfpathlineto{\pgfqpoint{4.008804in}{2.662586in}}%
\pgfpathlineto{\pgfqpoint{3.995568in}{2.666524in}}%
\pgfpathlineto{\pgfqpoint{3.982338in}{2.670495in}}%
\pgfpathlineto{\pgfqpoint{3.969113in}{2.674499in}}%
\pgfpathlineto{\pgfqpoint{3.955894in}{2.678535in}}%
\pgfpathlineto{\pgfqpoint{3.963580in}{2.686275in}}%
\pgfpathlineto{\pgfqpoint{3.971260in}{2.694039in}}%
\pgfpathlineto{\pgfqpoint{3.978934in}{2.701830in}}%
\pgfpathlineto{\pgfqpoint{3.986603in}{2.709649in}}%
\pgfpathclose%
\pgfusepath{fill}%
\end{pgfscope}%
\begin{pgfscope}%
\pgfpathrectangle{\pgfqpoint{1.150000in}{0.150000in}}{\pgfqpoint{5.700000in}{5.700000in}}%
\pgfusepath{clip}%
\pgfsetbuttcap%
\pgfsetroundjoin%
\definecolor{currentfill}{rgb}{0.274952,0.037752,0.364543}%
\pgfsetfillcolor{currentfill}%
\pgfsetfillopacity{0.700000}%
\pgfsetlinewidth{0.000000pt}%
\definecolor{currentstroke}{rgb}{0.000000,0.000000,0.000000}%
\pgfsetstrokecolor{currentstroke}%
\pgfsetdash{}{0pt}%
\pgfpathmoveto{\pgfqpoint{4.342705in}{2.726349in}}%
\pgfpathlineto{\pgfqpoint{4.355990in}{2.722661in}}%
\pgfpathlineto{\pgfqpoint{4.369282in}{2.719002in}}%
\pgfpathlineto{\pgfqpoint{4.382579in}{2.715373in}}%
\pgfpathlineto{\pgfqpoint{4.395883in}{2.711774in}}%
\pgfpathlineto{\pgfqpoint{4.388350in}{2.703986in}}%
\pgfpathlineto{\pgfqpoint{4.380813in}{2.696237in}}%
\pgfpathlineto{\pgfqpoint{4.373271in}{2.688525in}}%
\pgfpathlineto{\pgfqpoint{4.365724in}{2.680846in}}%
\pgfpathlineto{\pgfqpoint{4.352408in}{2.684358in}}%
\pgfpathlineto{\pgfqpoint{4.339098in}{2.687899in}}%
\pgfpathlineto{\pgfqpoint{4.325794in}{2.691470in}}%
\pgfpathlineto{\pgfqpoint{4.312496in}{2.695070in}}%
\pgfpathlineto{\pgfqpoint{4.320056in}{2.702832in}}%
\pgfpathlineto{\pgfqpoint{4.327611in}{2.710630in}}%
\pgfpathlineto{\pgfqpoint{4.335160in}{2.718468in}}%
\pgfpathlineto{\pgfqpoint{4.342705in}{2.726349in}}%
\pgfpathclose%
\pgfusepath{fill}%
\end{pgfscope}%
\begin{pgfscope}%
\pgfpathrectangle{\pgfqpoint{1.150000in}{0.150000in}}{\pgfqpoint{5.700000in}{5.700000in}}%
\pgfusepath{clip}%
\pgfsetbuttcap%
\pgfsetroundjoin%
\definecolor{currentfill}{rgb}{0.271305,0.019942,0.347269}%
\pgfsetfillcolor{currentfill}%
\pgfsetfillopacity{0.700000}%
\pgfsetlinewidth{0.000000pt}%
\definecolor{currentstroke}{rgb}{0.000000,0.000000,0.000000}%
\pgfsetstrokecolor{currentstroke}%
\pgfsetdash{}{0pt}%
\pgfpathmoveto{\pgfqpoint{3.630471in}{2.704074in}}%
\pgfpathlineto{\pgfqpoint{3.643609in}{2.699395in}}%
\pgfpathlineto{\pgfqpoint{3.656752in}{2.694753in}}%
\pgfpathlineto{\pgfqpoint{3.669900in}{2.690149in}}%
\pgfpathlineto{\pgfqpoint{3.683053in}{2.685581in}}%
\pgfpathlineto{\pgfqpoint{3.675269in}{2.677936in}}%
\pgfpathlineto{\pgfqpoint{3.667479in}{2.670315in}}%
\pgfpathlineto{\pgfqpoint{3.659683in}{2.662717in}}%
\pgfpathlineto{\pgfqpoint{3.651881in}{2.655142in}}%
\pgfpathlineto{\pgfqpoint{3.638716in}{2.659700in}}%
\pgfpathlineto{\pgfqpoint{3.625556in}{2.664295in}}%
\pgfpathlineto{\pgfqpoint{3.612400in}{2.668927in}}%
\pgfpathlineto{\pgfqpoint{3.599250in}{2.673597in}}%
\pgfpathlineto{\pgfqpoint{3.607064in}{2.681177in}}%
\pgfpathlineto{\pgfqpoint{3.614872in}{2.688783in}}%
\pgfpathlineto{\pgfqpoint{3.622674in}{2.696415in}}%
\pgfpathlineto{\pgfqpoint{3.630471in}{2.704074in}}%
\pgfpathclose%
\pgfusepath{fill}%
\end{pgfscope}%
\begin{pgfscope}%
\pgfpathrectangle{\pgfqpoint{1.150000in}{0.150000in}}{\pgfqpoint{5.700000in}{5.700000in}}%
\pgfusepath{clip}%
\pgfsetbuttcap%
\pgfsetroundjoin%
\definecolor{currentfill}{rgb}{0.279566,0.067836,0.391917}%
\pgfsetfillcolor{currentfill}%
\pgfsetfillopacity{0.700000}%
\pgfsetlinewidth{0.000000pt}%
\definecolor{currentstroke}{rgb}{0.000000,0.000000,0.000000}%
\pgfsetstrokecolor{currentstroke}%
\pgfsetdash{}{0pt}%
\pgfpathmoveto{\pgfqpoint{4.918707in}{2.774075in}}%
\pgfpathlineto{\pgfqpoint{4.932124in}{2.770639in}}%
\pgfpathlineto{\pgfqpoint{4.945547in}{2.767229in}}%
\pgfpathlineto{\pgfqpoint{4.958977in}{2.763845in}}%
\pgfpathlineto{\pgfqpoint{4.972413in}{2.760487in}}%
\pgfpathlineto{\pgfqpoint{4.965078in}{2.752397in}}%
\pgfpathlineto{\pgfqpoint{4.957740in}{2.744401in}}%
\pgfpathlineto{\pgfqpoint{4.950397in}{2.736494in}}%
\pgfpathlineto{\pgfqpoint{4.943051in}{2.728670in}}%
\pgfpathlineto{\pgfqpoint{4.929600in}{2.731875in}}%
\pgfpathlineto{\pgfqpoint{4.916156in}{2.735106in}}%
\pgfpathlineto{\pgfqpoint{4.902717in}{2.738363in}}%
\pgfpathlineto{\pgfqpoint{4.889286in}{2.741646in}}%
\pgfpathlineto{\pgfqpoint{4.896646in}{2.749619in}}%
\pgfpathlineto{\pgfqpoint{4.904003in}{2.757677in}}%
\pgfpathlineto{\pgfqpoint{4.911357in}{2.765828in}}%
\pgfpathlineto{\pgfqpoint{4.918707in}{2.774075in}}%
\pgfpathclose%
\pgfusepath{fill}%
\end{pgfscope}%
\begin{pgfscope}%
\pgfpathrectangle{\pgfqpoint{1.150000in}{0.150000in}}{\pgfqpoint{5.700000in}{5.700000in}}%
\pgfusepath{clip}%
\pgfsetbuttcap%
\pgfsetroundjoin%
\definecolor{currentfill}{rgb}{0.272594,0.025563,0.353093}%
\pgfsetfillcolor{currentfill}%
\pgfsetfillopacity{0.700000}%
\pgfsetlinewidth{0.000000pt}%
\definecolor{currentstroke}{rgb}{0.000000,0.000000,0.000000}%
\pgfsetstrokecolor{currentstroke}%
\pgfsetdash{}{0pt}%
\pgfpathmoveto{\pgfqpoint{4.122940in}{2.708983in}}%
\pgfpathlineto{\pgfqpoint{4.136180in}{2.705093in}}%
\pgfpathlineto{\pgfqpoint{4.149426in}{2.701234in}}%
\pgfpathlineto{\pgfqpoint{4.162677in}{2.697406in}}%
\pgfpathlineto{\pgfqpoint{4.175934in}{2.693609in}}%
\pgfpathlineto{\pgfqpoint{4.168323in}{2.685856in}}%
\pgfpathlineto{\pgfqpoint{4.160707in}{2.678129in}}%
\pgfpathlineto{\pgfqpoint{4.153085in}{2.670427in}}%
\pgfpathlineto{\pgfqpoint{4.145458in}{2.662749in}}%
\pgfpathlineto{\pgfqpoint{4.132189in}{2.666484in}}%
\pgfpathlineto{\pgfqpoint{4.118925in}{2.670250in}}%
\pgfpathlineto{\pgfqpoint{4.105667in}{2.674047in}}%
\pgfpathlineto{\pgfqpoint{4.092415in}{2.677875in}}%
\pgfpathlineto{\pgfqpoint{4.100055in}{2.685611in}}%
\pgfpathlineto{\pgfqpoint{4.107689in}{2.693373in}}%
\pgfpathlineto{\pgfqpoint{4.115317in}{2.701163in}}%
\pgfpathlineto{\pgfqpoint{4.122940in}{2.708983in}}%
\pgfpathclose%
\pgfusepath{fill}%
\end{pgfscope}%
\begin{pgfscope}%
\pgfpathrectangle{\pgfqpoint{1.150000in}{0.150000in}}{\pgfqpoint{5.700000in}{5.700000in}}%
\pgfusepath{clip}%
\pgfsetbuttcap%
\pgfsetroundjoin%
\definecolor{currentfill}{rgb}{0.271305,0.019942,0.347269}%
\pgfsetfillcolor{currentfill}%
\pgfsetfillopacity{0.700000}%
\pgfsetlinewidth{0.000000pt}%
\definecolor{currentstroke}{rgb}{0.000000,0.000000,0.000000}%
\pgfsetstrokecolor{currentstroke}%
\pgfsetdash{}{0pt}%
\pgfpathmoveto{\pgfqpoint{3.766748in}{2.698424in}}%
\pgfpathlineto{\pgfqpoint{3.779915in}{2.694014in}}%
\pgfpathlineto{\pgfqpoint{3.793087in}{2.689640in}}%
\pgfpathlineto{\pgfqpoint{3.806264in}{2.685300in}}%
\pgfpathlineto{\pgfqpoint{3.819446in}{2.680996in}}%
\pgfpathlineto{\pgfqpoint{3.811709in}{2.673299in}}%
\pgfpathlineto{\pgfqpoint{3.803966in}{2.665624in}}%
\pgfpathlineto{\pgfqpoint{3.796217in}{2.657970in}}%
\pgfpathlineto{\pgfqpoint{3.788463in}{2.650336in}}%
\pgfpathlineto{\pgfqpoint{3.775269in}{2.654618in}}%
\pgfpathlineto{\pgfqpoint{3.762079in}{2.658934in}}%
\pgfpathlineto{\pgfqpoint{3.748895in}{2.663286in}}%
\pgfpathlineto{\pgfqpoint{3.735717in}{2.667674in}}%
\pgfpathlineto{\pgfqpoint{3.743483in}{2.675326in}}%
\pgfpathlineto{\pgfqpoint{3.751244in}{2.683001in}}%
\pgfpathlineto{\pgfqpoint{3.758999in}{2.690700in}}%
\pgfpathlineto{\pgfqpoint{3.766748in}{2.698424in}}%
\pgfpathclose%
\pgfusepath{fill}%
\end{pgfscope}%
\begin{pgfscope}%
\pgfpathrectangle{\pgfqpoint{1.150000in}{0.150000in}}{\pgfqpoint{5.700000in}{5.700000in}}%
\pgfusepath{clip}%
\pgfsetbuttcap%
\pgfsetroundjoin%
\definecolor{currentfill}{rgb}{0.277941,0.056324,0.381191}%
\pgfsetfillcolor{currentfill}%
\pgfsetfillopacity{0.700000}%
\pgfsetlinewidth{0.000000pt}%
\definecolor{currentstroke}{rgb}{0.000000,0.000000,0.000000}%
\pgfsetstrokecolor{currentstroke}%
\pgfsetdash{}{0pt}%
\pgfpathmoveto{\pgfqpoint{4.698956in}{2.750323in}}%
\pgfpathlineto{\pgfqpoint{4.712325in}{2.746860in}}%
\pgfpathlineto{\pgfqpoint{4.725700in}{2.743424in}}%
\pgfpathlineto{\pgfqpoint{4.739081in}{2.740014in}}%
\pgfpathlineto{\pgfqpoint{4.752469in}{2.736632in}}%
\pgfpathlineto{\pgfqpoint{4.745060in}{2.728762in}}%
\pgfpathlineto{\pgfqpoint{4.737646in}{2.720959in}}%
\pgfpathlineto{\pgfqpoint{4.730229in}{2.713218in}}%
\pgfpathlineto{\pgfqpoint{4.722806in}{2.705536in}}%
\pgfpathlineto{\pgfqpoint{4.709404in}{2.708791in}}%
\pgfpathlineto{\pgfqpoint{4.696009in}{2.712074in}}%
\pgfpathlineto{\pgfqpoint{4.682620in}{2.715383in}}%
\pgfpathlineto{\pgfqpoint{4.669237in}{2.718719in}}%
\pgfpathlineto{\pgfqpoint{4.676674in}{2.726524in}}%
\pgfpathlineto{\pgfqpoint{4.684105in}{2.734390in}}%
\pgfpathlineto{\pgfqpoint{4.691533in}{2.742322in}}%
\pgfpathlineto{\pgfqpoint{4.698956in}{2.750323in}}%
\pgfpathclose%
\pgfusepath{fill}%
\end{pgfscope}%
\begin{pgfscope}%
\pgfpathrectangle{\pgfqpoint{1.150000in}{0.150000in}}{\pgfqpoint{5.700000in}{5.700000in}}%
\pgfusepath{clip}%
\pgfsetbuttcap%
\pgfsetroundjoin%
\definecolor{currentfill}{rgb}{0.276022,0.044167,0.370164}%
\pgfsetfillcolor{currentfill}%
\pgfsetfillopacity{0.700000}%
\pgfsetlinewidth{0.000000pt}%
\definecolor{currentstroke}{rgb}{0.000000,0.000000,0.000000}%
\pgfsetstrokecolor{currentstroke}%
\pgfsetdash{}{0pt}%
\pgfpathmoveto{\pgfqpoint{4.479184in}{2.728867in}}%
\pgfpathlineto{\pgfqpoint{4.492505in}{2.725311in}}%
\pgfpathlineto{\pgfqpoint{4.505832in}{2.721783in}}%
\pgfpathlineto{\pgfqpoint{4.519166in}{2.718283in}}%
\pgfpathlineto{\pgfqpoint{4.532505in}{2.714811in}}%
\pgfpathlineto{\pgfqpoint{4.525019in}{2.707046in}}%
\pgfpathlineto{\pgfqpoint{4.517527in}{2.699326in}}%
\pgfpathlineto{\pgfqpoint{4.510031in}{2.691649in}}%
\pgfpathlineto{\pgfqpoint{4.502530in}{2.684011in}}%
\pgfpathlineto{\pgfqpoint{4.489177in}{2.687382in}}%
\pgfpathlineto{\pgfqpoint{4.475831in}{2.690781in}}%
\pgfpathlineto{\pgfqpoint{4.462491in}{2.694208in}}%
\pgfpathlineto{\pgfqpoint{4.449157in}{2.697664in}}%
\pgfpathlineto{\pgfqpoint{4.456671in}{2.705398in}}%
\pgfpathlineto{\pgfqpoint{4.464180in}{2.713174in}}%
\pgfpathlineto{\pgfqpoint{4.471685in}{2.720996in}}%
\pgfpathlineto{\pgfqpoint{4.479184in}{2.728867in}}%
\pgfpathclose%
\pgfusepath{fill}%
\end{pgfscope}%
\begin{pgfscope}%
\pgfpathrectangle{\pgfqpoint{1.150000in}{0.150000in}}{\pgfqpoint{5.700000in}{5.700000in}}%
\pgfusepath{clip}%
\pgfsetbuttcap%
\pgfsetroundjoin%
\definecolor{currentfill}{rgb}{0.273809,0.031497,0.358853}%
\pgfsetfillcolor{currentfill}%
\pgfsetfillopacity{0.700000}%
\pgfsetlinewidth{0.000000pt}%
\definecolor{currentstroke}{rgb}{0.000000,0.000000,0.000000}%
\pgfsetstrokecolor{currentstroke}%
\pgfsetdash{}{0pt}%
\pgfpathmoveto{\pgfqpoint{3.273988in}{2.715949in}}%
\pgfpathlineto{\pgfqpoint{3.287073in}{2.710445in}}%
\pgfpathlineto{\pgfqpoint{3.300162in}{2.704984in}}%
\pgfpathlineto{\pgfqpoint{3.313256in}{2.699567in}}%
\pgfpathlineto{\pgfqpoint{3.326354in}{2.694193in}}%
\pgfpathlineto{\pgfqpoint{3.318434in}{2.686924in}}%
\pgfpathlineto{\pgfqpoint{3.310508in}{2.679691in}}%
\pgfpathlineto{\pgfqpoint{3.302576in}{2.672494in}}%
\pgfpathlineto{\pgfqpoint{3.294637in}{2.665333in}}%
\pgfpathlineto{\pgfqpoint{3.281526in}{2.670737in}}%
\pgfpathlineto{\pgfqpoint{3.268419in}{2.676184in}}%
\pgfpathlineto{\pgfqpoint{3.255316in}{2.681674in}}%
\pgfpathlineto{\pgfqpoint{3.242218in}{2.687209in}}%
\pgfpathlineto{\pgfqpoint{3.250170in}{2.694335in}}%
\pgfpathlineto{\pgfqpoint{3.258116in}{2.701500in}}%
\pgfpathlineto{\pgfqpoint{3.266055in}{2.708705in}}%
\pgfpathlineto{\pgfqpoint{3.273988in}{2.715949in}}%
\pgfpathclose%
\pgfusepath{fill}%
\end{pgfscope}%
\begin{pgfscope}%
\pgfpathrectangle{\pgfqpoint{1.150000in}{0.150000in}}{\pgfqpoint{5.700000in}{5.700000in}}%
\pgfusepath{clip}%
\pgfsetbuttcap%
\pgfsetroundjoin%
\definecolor{currentfill}{rgb}{0.276022,0.044167,0.370164}%
\pgfsetfillcolor{currentfill}%
\pgfsetfillopacity{0.700000}%
\pgfsetlinewidth{0.000000pt}%
\definecolor{currentstroke}{rgb}{0.000000,0.000000,0.000000}%
\pgfsetstrokecolor{currentstroke}%
\pgfsetdash{}{0pt}%
\pgfpathmoveto{\pgfqpoint{3.137572in}{2.733128in}}%
\pgfpathlineto{\pgfqpoint{3.150639in}{2.727224in}}%
\pgfpathlineto{\pgfqpoint{3.163710in}{2.721369in}}%
\pgfpathlineto{\pgfqpoint{3.176785in}{2.715560in}}%
\pgfpathlineto{\pgfqpoint{3.189864in}{2.709798in}}%
\pgfpathlineto{\pgfqpoint{3.181891in}{2.702751in}}%
\pgfpathlineto{\pgfqpoint{3.173912in}{2.695748in}}%
\pgfpathlineto{\pgfqpoint{3.165926in}{2.688789in}}%
\pgfpathlineto{\pgfqpoint{3.157933in}{2.681875in}}%
\pgfpathlineto{\pgfqpoint{3.144840in}{2.687680in}}%
\pgfpathlineto{\pgfqpoint{3.131751in}{2.693532in}}%
\pgfpathlineto{\pgfqpoint{3.118666in}{2.699431in}}%
\pgfpathlineto{\pgfqpoint{3.105585in}{2.705377in}}%
\pgfpathlineto{\pgfqpoint{3.113592in}{2.712243in}}%
\pgfpathlineto{\pgfqpoint{3.121592in}{2.719157in}}%
\pgfpathlineto{\pgfqpoint{3.129585in}{2.726119in}}%
\pgfpathlineto{\pgfqpoint{3.137572in}{2.733128in}}%
\pgfpathclose%
\pgfusepath{fill}%
\end{pgfscope}%
\begin{pgfscope}%
\pgfpathrectangle{\pgfqpoint{1.150000in}{0.150000in}}{\pgfqpoint{5.700000in}{5.700000in}}%
\pgfusepath{clip}%
\pgfsetbuttcap%
\pgfsetroundjoin%
\definecolor{currentfill}{rgb}{0.271305,0.019942,0.347269}%
\pgfsetfillcolor{currentfill}%
\pgfsetfillopacity{0.700000}%
\pgfsetlinewidth{0.000000pt}%
\definecolor{currentstroke}{rgb}{0.000000,0.000000,0.000000}%
\pgfsetstrokecolor{currentstroke}%
\pgfsetdash{}{0pt}%
\pgfpathmoveto{\pgfqpoint{3.903074in}{2.695012in}}%
\pgfpathlineto{\pgfqpoint{3.916271in}{2.690843in}}%
\pgfpathlineto{\pgfqpoint{3.929473in}{2.686707in}}%
\pgfpathlineto{\pgfqpoint{3.942681in}{2.682604in}}%
\pgfpathlineto{\pgfqpoint{3.955894in}{2.678535in}}%
\pgfpathlineto{\pgfqpoint{3.948204in}{2.670818in}}%
\pgfpathlineto{\pgfqpoint{3.940507in}{2.663122in}}%
\pgfpathlineto{\pgfqpoint{3.932805in}{2.655446in}}%
\pgfpathlineto{\pgfqpoint{3.925098in}{2.647788in}}%
\pgfpathlineto{\pgfqpoint{3.911872in}{2.651822in}}%
\pgfpathlineto{\pgfqpoint{3.898652in}{2.655888in}}%
\pgfpathlineto{\pgfqpoint{3.885438in}{2.659988in}}%
\pgfpathlineto{\pgfqpoint{3.872228in}{2.664122in}}%
\pgfpathlineto{\pgfqpoint{3.879948in}{2.671811in}}%
\pgfpathlineto{\pgfqpoint{3.887662in}{2.679521in}}%
\pgfpathlineto{\pgfqpoint{3.895371in}{2.687254in}}%
\pgfpathlineto{\pgfqpoint{3.903074in}{2.695012in}}%
\pgfpathclose%
\pgfusepath{fill}%
\end{pgfscope}%
\begin{pgfscope}%
\pgfpathrectangle{\pgfqpoint{1.150000in}{0.150000in}}{\pgfqpoint{5.700000in}{5.700000in}}%
\pgfusepath{clip}%
\pgfsetbuttcap%
\pgfsetroundjoin%
\definecolor{currentfill}{rgb}{0.272594,0.025563,0.353093}%
\pgfsetfillcolor{currentfill}%
\pgfsetfillopacity{0.700000}%
\pgfsetlinewidth{0.000000pt}%
\definecolor{currentstroke}{rgb}{0.000000,0.000000,0.000000}%
\pgfsetstrokecolor{currentstroke}%
\pgfsetdash{}{0pt}%
\pgfpathmoveto{\pgfqpoint{3.410351in}{2.702627in}}%
\pgfpathlineto{\pgfqpoint{3.423458in}{2.697481in}}%
\pgfpathlineto{\pgfqpoint{3.436569in}{2.692377in}}%
\pgfpathlineto{\pgfqpoint{3.449684in}{2.687313in}}%
\pgfpathlineto{\pgfqpoint{3.462805in}{2.682290in}}%
\pgfpathlineto{\pgfqpoint{3.454936in}{2.674853in}}%
\pgfpathlineto{\pgfqpoint{3.447061in}{2.667446in}}%
\pgfpathlineto{\pgfqpoint{3.439180in}{2.660068in}}%
\pgfpathlineto{\pgfqpoint{3.431293in}{2.652720in}}%
\pgfpathlineto{\pgfqpoint{3.418160in}{2.657760in}}%
\pgfpathlineto{\pgfqpoint{3.405031in}{2.662841in}}%
\pgfpathlineto{\pgfqpoint{3.391907in}{2.667962in}}%
\pgfpathlineto{\pgfqpoint{3.378788in}{2.673124in}}%
\pgfpathlineto{\pgfqpoint{3.386688in}{2.680451in}}%
\pgfpathlineto{\pgfqpoint{3.394582in}{2.687810in}}%
\pgfpathlineto{\pgfqpoint{3.402470in}{2.695202in}}%
\pgfpathlineto{\pgfqpoint{3.410351in}{2.702627in}}%
\pgfpathclose%
\pgfusepath{fill}%
\end{pgfscope}%
\begin{pgfscope}%
\pgfpathrectangle{\pgfqpoint{1.150000in}{0.150000in}}{\pgfqpoint{5.700000in}{5.700000in}}%
\pgfusepath{clip}%
\pgfsetbuttcap%
\pgfsetroundjoin%
\definecolor{currentfill}{rgb}{0.273809,0.031497,0.358853}%
\pgfsetfillcolor{currentfill}%
\pgfsetfillopacity{0.700000}%
\pgfsetlinewidth{0.000000pt}%
\definecolor{currentstroke}{rgb}{0.000000,0.000000,0.000000}%
\pgfsetstrokecolor{currentstroke}%
\pgfsetdash{}{0pt}%
\pgfpathmoveto{\pgfqpoint{4.259364in}{2.709767in}}%
\pgfpathlineto{\pgfqpoint{4.272638in}{2.706048in}}%
\pgfpathlineto{\pgfqpoint{4.285918in}{2.702358in}}%
\pgfpathlineto{\pgfqpoint{4.299204in}{2.698699in}}%
\pgfpathlineto{\pgfqpoint{4.312496in}{2.695070in}}%
\pgfpathlineto{\pgfqpoint{4.304931in}{2.687341in}}%
\pgfpathlineto{\pgfqpoint{4.297360in}{2.679644in}}%
\pgfpathlineto{\pgfqpoint{4.289784in}{2.671975in}}%
\pgfpathlineto{\pgfqpoint{4.282204in}{2.664331in}}%
\pgfpathlineto{\pgfqpoint{4.268899in}{2.667886in}}%
\pgfpathlineto{\pgfqpoint{4.255601in}{2.671470in}}%
\pgfpathlineto{\pgfqpoint{4.242308in}{2.675084in}}%
\pgfpathlineto{\pgfqpoint{4.229022in}{2.678729in}}%
\pgfpathlineto{\pgfqpoint{4.236615in}{2.686442in}}%
\pgfpathlineto{\pgfqpoint{4.244203in}{2.694185in}}%
\pgfpathlineto{\pgfqpoint{4.251786in}{2.701959in}}%
\pgfpathlineto{\pgfqpoint{4.259364in}{2.709767in}}%
\pgfpathclose%
\pgfusepath{fill}%
\end{pgfscope}%
\begin{pgfscope}%
\pgfpathrectangle{\pgfqpoint{1.150000in}{0.150000in}}{\pgfqpoint{5.700000in}{5.700000in}}%
\pgfusepath{clip}%
\pgfsetbuttcap%
\pgfsetroundjoin%
\definecolor{currentfill}{rgb}{0.277941,0.056324,0.381191}%
\pgfsetfillcolor{currentfill}%
\pgfsetfillopacity{0.700000}%
\pgfsetlinewidth{0.000000pt}%
\definecolor{currentstroke}{rgb}{0.000000,0.000000,0.000000}%
\pgfsetstrokecolor{currentstroke}%
\pgfsetdash{}{0pt}%
\pgfpathmoveto{\pgfqpoint{3.001062in}{2.754721in}}%
\pgfpathlineto{\pgfqpoint{3.014116in}{2.748376in}}%
\pgfpathlineto{\pgfqpoint{3.027172in}{2.742083in}}%
\pgfpathlineto{\pgfqpoint{3.040232in}{2.735840in}}%
\pgfpathlineto{\pgfqpoint{3.053296in}{2.729649in}}%
\pgfpathlineto{\pgfqpoint{3.045267in}{2.722884in}}%
\pgfpathlineto{\pgfqpoint{3.037231in}{2.716172in}}%
\pgfpathlineto{\pgfqpoint{3.029188in}{2.709515in}}%
\pgfpathlineto{\pgfqpoint{3.021138in}{2.702913in}}%
\pgfpathlineto{\pgfqpoint{3.008060in}{2.709161in}}%
\pgfpathlineto{\pgfqpoint{2.994985in}{2.715460in}}%
\pgfpathlineto{\pgfqpoint{2.981913in}{2.721810in}}%
\pgfpathlineto{\pgfqpoint{2.968845in}{2.728212in}}%
\pgfpathlineto{\pgfqpoint{2.976910in}{2.734752in}}%
\pgfpathlineto{\pgfqpoint{2.984968in}{2.741351in}}%
\pgfpathlineto{\pgfqpoint{2.993019in}{2.748008in}}%
\pgfpathlineto{\pgfqpoint{3.001062in}{2.754721in}}%
\pgfpathclose%
\pgfusepath{fill}%
\end{pgfscope}%
\begin{pgfscope}%
\pgfpathrectangle{\pgfqpoint{1.150000in}{0.150000in}}{\pgfqpoint{5.700000in}{5.700000in}}%
\pgfusepath{clip}%
\pgfsetbuttcap%
\pgfsetroundjoin%
\definecolor{currentfill}{rgb}{0.280267,0.073417,0.397163}%
\pgfsetfillcolor{currentfill}%
\pgfsetfillopacity{0.700000}%
\pgfsetlinewidth{0.000000pt}%
\definecolor{currentstroke}{rgb}{0.000000,0.000000,0.000000}%
\pgfsetstrokecolor{currentstroke}%
\pgfsetdash{}{0pt}%
\pgfpathmoveto{\pgfqpoint{5.055473in}{2.780041in}}%
\pgfpathlineto{\pgfqpoint{5.068927in}{2.776646in}}%
\pgfpathlineto{\pgfqpoint{5.082388in}{2.773276in}}%
\pgfpathlineto{\pgfqpoint{5.095855in}{2.769932in}}%
\pgfpathlineto{\pgfqpoint{5.109328in}{2.766614in}}%
\pgfpathlineto{\pgfqpoint{5.102036in}{2.758442in}}%
\pgfpathlineto{\pgfqpoint{5.094741in}{2.750377in}}%
\pgfpathlineto{\pgfqpoint{5.087443in}{2.742414in}}%
\pgfpathlineto{\pgfqpoint{5.080141in}{2.734549in}}%
\pgfpathlineto{\pgfqpoint{5.066652in}{2.737702in}}%
\pgfpathlineto{\pgfqpoint{5.053169in}{2.740880in}}%
\pgfpathlineto{\pgfqpoint{5.039694in}{2.744083in}}%
\pgfpathlineto{\pgfqpoint{5.026224in}{2.747313in}}%
\pgfpathlineto{\pgfqpoint{5.033541in}{2.755339in}}%
\pgfpathlineto{\pgfqpoint{5.040855in}{2.763466in}}%
\pgfpathlineto{\pgfqpoint{5.048166in}{2.771698in}}%
\pgfpathlineto{\pgfqpoint{5.055473in}{2.780041in}}%
\pgfpathclose%
\pgfusepath{fill}%
\end{pgfscope}%
\begin{pgfscope}%
\pgfpathrectangle{\pgfqpoint{1.150000in}{0.150000in}}{\pgfqpoint{5.700000in}{5.700000in}}%
\pgfusepath{clip}%
\pgfsetbuttcap%
\pgfsetroundjoin%
\definecolor{currentfill}{rgb}{0.271305,0.019942,0.347269}%
\pgfsetfillcolor{currentfill}%
\pgfsetfillopacity{0.700000}%
\pgfsetlinewidth{0.000000pt}%
\definecolor{currentstroke}{rgb}{0.000000,0.000000,0.000000}%
\pgfsetstrokecolor{currentstroke}%
\pgfsetdash{}{0pt}%
\pgfpathmoveto{\pgfqpoint{3.546697in}{2.692656in}}%
\pgfpathlineto{\pgfqpoint{3.559828in}{2.687833in}}%
\pgfpathlineto{\pgfqpoint{3.572963in}{2.683050in}}%
\pgfpathlineto{\pgfqpoint{3.586104in}{2.678304in}}%
\pgfpathlineto{\pgfqpoint{3.599250in}{2.673597in}}%
\pgfpathlineto{\pgfqpoint{3.591430in}{2.666042in}}%
\pgfpathlineto{\pgfqpoint{3.583604in}{2.658512in}}%
\pgfpathlineto{\pgfqpoint{3.575773in}{2.651005in}}%
\pgfpathlineto{\pgfqpoint{3.567935in}{2.643522in}}%
\pgfpathlineto{\pgfqpoint{3.554777in}{2.648233in}}%
\pgfpathlineto{\pgfqpoint{3.541624in}{2.652982in}}%
\pgfpathlineto{\pgfqpoint{3.528475in}{2.657769in}}%
\pgfpathlineto{\pgfqpoint{3.515332in}{2.662594in}}%
\pgfpathlineto{\pgfqpoint{3.523182in}{2.670069in}}%
\pgfpathlineto{\pgfqpoint{3.531026in}{2.677571in}}%
\pgfpathlineto{\pgfqpoint{3.538864in}{2.685099in}}%
\pgfpathlineto{\pgfqpoint{3.546697in}{2.692656in}}%
\pgfpathclose%
\pgfusepath{fill}%
\end{pgfscope}%
\begin{pgfscope}%
\pgfpathrectangle{\pgfqpoint{1.150000in}{0.150000in}}{\pgfqpoint{5.700000in}{5.700000in}}%
\pgfusepath{clip}%
\pgfsetbuttcap%
\pgfsetroundjoin%
\definecolor{currentfill}{rgb}{0.278791,0.062145,0.386592}%
\pgfsetfillcolor{currentfill}%
\pgfsetfillopacity{0.700000}%
\pgfsetlinewidth{0.000000pt}%
\definecolor{currentstroke}{rgb}{0.000000,0.000000,0.000000}%
\pgfsetstrokecolor{currentstroke}%
\pgfsetdash{}{0pt}%
\pgfpathmoveto{\pgfqpoint{4.835624in}{2.755043in}}%
\pgfpathlineto{\pgfqpoint{4.849030in}{2.751654in}}%
\pgfpathlineto{\pgfqpoint{4.862442in}{2.748292in}}%
\pgfpathlineto{\pgfqpoint{4.875861in}{2.744956in}}%
\pgfpathlineto{\pgfqpoint{4.889286in}{2.741646in}}%
\pgfpathlineto{\pgfqpoint{4.881921in}{2.733757in}}%
\pgfpathlineto{\pgfqpoint{4.874553in}{2.725945in}}%
\pgfpathlineto{\pgfqpoint{4.867181in}{2.718206in}}%
\pgfpathlineto{\pgfqpoint{4.859804in}{2.710535in}}%
\pgfpathlineto{\pgfqpoint{4.846364in}{2.713705in}}%
\pgfpathlineto{\pgfqpoint{4.832931in}{2.716900in}}%
\pgfpathlineto{\pgfqpoint{4.819505in}{2.720123in}}%
\pgfpathlineto{\pgfqpoint{4.806085in}{2.723371in}}%
\pgfpathlineto{\pgfqpoint{4.813476in}{2.731177in}}%
\pgfpathlineto{\pgfqpoint{4.820862in}{2.739054in}}%
\pgfpathlineto{\pgfqpoint{4.828245in}{2.747008in}}%
\pgfpathlineto{\pgfqpoint{4.835624in}{2.755043in}}%
\pgfpathclose%
\pgfusepath{fill}%
\end{pgfscope}%
\begin{pgfscope}%
\pgfpathrectangle{\pgfqpoint{1.150000in}{0.150000in}}{\pgfqpoint{5.700000in}{5.700000in}}%
\pgfusepath{clip}%
\pgfsetbuttcap%
\pgfsetroundjoin%
\definecolor{currentfill}{rgb}{0.277018,0.050344,0.375715}%
\pgfsetfillcolor{currentfill}%
\pgfsetfillopacity{0.700000}%
\pgfsetlinewidth{0.000000pt}%
\definecolor{currentstroke}{rgb}{0.000000,0.000000,0.000000}%
\pgfsetstrokecolor{currentstroke}%
\pgfsetdash{}{0pt}%
\pgfpathmoveto{\pgfqpoint{4.615770in}{2.732339in}}%
\pgfpathlineto{\pgfqpoint{4.629128in}{2.728893in}}%
\pgfpathlineto{\pgfqpoint{4.642491in}{2.725474in}}%
\pgfpathlineto{\pgfqpoint{4.655861in}{2.722083in}}%
\pgfpathlineto{\pgfqpoint{4.669237in}{2.718719in}}%
\pgfpathlineto{\pgfqpoint{4.661797in}{2.710973in}}%
\pgfpathlineto{\pgfqpoint{4.654351in}{2.703280in}}%
\pgfpathlineto{\pgfqpoint{4.646901in}{2.695637in}}%
\pgfpathlineto{\pgfqpoint{4.639447in}{2.688041in}}%
\pgfpathlineto{\pgfqpoint{4.626057in}{2.691291in}}%
\pgfpathlineto{\pgfqpoint{4.612673in}{2.694568in}}%
\pgfpathlineto{\pgfqpoint{4.599296in}{2.697872in}}%
\pgfpathlineto{\pgfqpoint{4.585925in}{2.701204in}}%
\pgfpathlineto{\pgfqpoint{4.593394in}{2.708910in}}%
\pgfpathlineto{\pgfqpoint{4.600857in}{2.716665in}}%
\pgfpathlineto{\pgfqpoint{4.608316in}{2.724473in}}%
\pgfpathlineto{\pgfqpoint{4.615770in}{2.732339in}}%
\pgfpathclose%
\pgfusepath{fill}%
\end{pgfscope}%
\begin{pgfscope}%
\pgfpathrectangle{\pgfqpoint{1.150000in}{0.150000in}}{\pgfqpoint{5.700000in}{5.700000in}}%
\pgfusepath{clip}%
\pgfsetbuttcap%
\pgfsetroundjoin%
\definecolor{currentfill}{rgb}{0.272594,0.025563,0.353093}%
\pgfsetfillcolor{currentfill}%
\pgfsetfillopacity{0.700000}%
\pgfsetlinewidth{0.000000pt}%
\definecolor{currentstroke}{rgb}{0.000000,0.000000,0.000000}%
\pgfsetstrokecolor{currentstroke}%
\pgfsetdash{}{0pt}%
\pgfpathmoveto{\pgfqpoint{4.039464in}{2.693505in}}%
\pgfpathlineto{\pgfqpoint{4.052693in}{2.689549in}}%
\pgfpathlineto{\pgfqpoint{4.065928in}{2.685626in}}%
\pgfpathlineto{\pgfqpoint{4.079169in}{2.681735in}}%
\pgfpathlineto{\pgfqpoint{4.092415in}{2.677875in}}%
\pgfpathlineto{\pgfqpoint{4.084771in}{2.670163in}}%
\pgfpathlineto{\pgfqpoint{4.077120in}{2.662473in}}%
\pgfpathlineto{\pgfqpoint{4.069465in}{2.654803in}}%
\pgfpathlineto{\pgfqpoint{4.061804in}{2.647151in}}%
\pgfpathlineto{\pgfqpoint{4.048545in}{2.650962in}}%
\pgfpathlineto{\pgfqpoint{4.035292in}{2.654805in}}%
\pgfpathlineto{\pgfqpoint{4.022045in}{2.658679in}}%
\pgfpathlineto{\pgfqpoint{4.008804in}{2.662586in}}%
\pgfpathlineto{\pgfqpoint{4.016477in}{2.670282in}}%
\pgfpathlineto{\pgfqpoint{4.024145in}{2.677999in}}%
\pgfpathlineto{\pgfqpoint{4.031807in}{2.685739in}}%
\pgfpathlineto{\pgfqpoint{4.039464in}{2.693505in}}%
\pgfpathclose%
\pgfusepath{fill}%
\end{pgfscope}%
\begin{pgfscope}%
\pgfpathrectangle{\pgfqpoint{1.150000in}{0.150000in}}{\pgfqpoint{5.700000in}{5.700000in}}%
\pgfusepath{clip}%
\pgfsetbuttcap%
\pgfsetroundjoin%
\definecolor{currentfill}{rgb}{0.271305,0.019942,0.347269}%
\pgfsetfillcolor{currentfill}%
\pgfsetfillopacity{0.700000}%
\pgfsetlinewidth{0.000000pt}%
\definecolor{currentstroke}{rgb}{0.000000,0.000000,0.000000}%
\pgfsetstrokecolor{currentstroke}%
\pgfsetdash{}{0pt}%
\pgfpathmoveto{\pgfqpoint{3.683053in}{2.685581in}}%
\pgfpathlineto{\pgfqpoint{3.696212in}{2.681050in}}%
\pgfpathlineto{\pgfqpoint{3.709375in}{2.676555in}}%
\pgfpathlineto{\pgfqpoint{3.722543in}{2.672096in}}%
\pgfpathlineto{\pgfqpoint{3.735717in}{2.667674in}}%
\pgfpathlineto{\pgfqpoint{3.727945in}{2.660043in}}%
\pgfpathlineto{\pgfqpoint{3.720167in}{2.652433in}}%
\pgfpathlineto{\pgfqpoint{3.712383in}{2.644843in}}%
\pgfpathlineto{\pgfqpoint{3.704594in}{2.637273in}}%
\pgfpathlineto{\pgfqpoint{3.691408in}{2.641686in}}%
\pgfpathlineto{\pgfqpoint{3.678227in}{2.646135in}}%
\pgfpathlineto{\pgfqpoint{3.665052in}{2.650620in}}%
\pgfpathlineto{\pgfqpoint{3.651881in}{2.655142in}}%
\pgfpathlineto{\pgfqpoint{3.659683in}{2.662717in}}%
\pgfpathlineto{\pgfqpoint{3.667479in}{2.670315in}}%
\pgfpathlineto{\pgfqpoint{3.675269in}{2.677936in}}%
\pgfpathlineto{\pgfqpoint{3.683053in}{2.685581in}}%
\pgfpathclose%
\pgfusepath{fill}%
\end{pgfscope}%
\begin{pgfscope}%
\pgfpathrectangle{\pgfqpoint{1.150000in}{0.150000in}}{\pgfqpoint{5.700000in}{5.700000in}}%
\pgfusepath{clip}%
\pgfsetbuttcap%
\pgfsetroundjoin%
\definecolor{currentfill}{rgb}{0.274952,0.037752,0.364543}%
\pgfsetfillcolor{currentfill}%
\pgfsetfillopacity{0.700000}%
\pgfsetlinewidth{0.000000pt}%
\definecolor{currentstroke}{rgb}{0.000000,0.000000,0.000000}%
\pgfsetstrokecolor{currentstroke}%
\pgfsetdash{}{0pt}%
\pgfpathmoveto{\pgfqpoint{4.395883in}{2.711774in}}%
\pgfpathlineto{\pgfqpoint{4.409192in}{2.708203in}}%
\pgfpathlineto{\pgfqpoint{4.422508in}{2.704661in}}%
\pgfpathlineto{\pgfqpoint{4.435829in}{2.701148in}}%
\pgfpathlineto{\pgfqpoint{4.449157in}{2.697664in}}%
\pgfpathlineto{\pgfqpoint{4.441638in}{2.689969in}}%
\pgfpathlineto{\pgfqpoint{4.434113in}{2.682310in}}%
\pgfpathlineto{\pgfqpoint{4.426584in}{2.674684in}}%
\pgfpathlineto{\pgfqpoint{4.419049in}{2.667089in}}%
\pgfpathlineto{\pgfqpoint{4.405709in}{2.670485in}}%
\pgfpathlineto{\pgfqpoint{4.392374in}{2.673910in}}%
\pgfpathlineto{\pgfqpoint{4.379046in}{2.677363in}}%
\pgfpathlineto{\pgfqpoint{4.365724in}{2.680846in}}%
\pgfpathlineto{\pgfqpoint{4.373271in}{2.688525in}}%
\pgfpathlineto{\pgfqpoint{4.380813in}{2.696237in}}%
\pgfpathlineto{\pgfqpoint{4.388350in}{2.703986in}}%
\pgfpathlineto{\pgfqpoint{4.395883in}{2.711774in}}%
\pgfpathclose%
\pgfusepath{fill}%
\end{pgfscope}%
\begin{pgfscope}%
\pgfpathrectangle{\pgfqpoint{1.150000in}{0.150000in}}{\pgfqpoint{5.700000in}{5.700000in}}%
\pgfusepath{clip}%
\pgfsetbuttcap%
\pgfsetroundjoin%
\definecolor{currentfill}{rgb}{0.271305,0.019942,0.347269}%
\pgfsetfillcolor{currentfill}%
\pgfsetfillopacity{0.700000}%
\pgfsetlinewidth{0.000000pt}%
\definecolor{currentstroke}{rgb}{0.000000,0.000000,0.000000}%
\pgfsetstrokecolor{currentstroke}%
\pgfsetdash{}{0pt}%
\pgfpathmoveto{\pgfqpoint{3.819446in}{2.680996in}}%
\pgfpathlineto{\pgfqpoint{3.832634in}{2.676726in}}%
\pgfpathlineto{\pgfqpoint{3.845827in}{2.672490in}}%
\pgfpathlineto{\pgfqpoint{3.859025in}{2.668289in}}%
\pgfpathlineto{\pgfqpoint{3.872228in}{2.664122in}}%
\pgfpathlineto{\pgfqpoint{3.864503in}{2.656453in}}%
\pgfpathlineto{\pgfqpoint{3.856773in}{2.648802in}}%
\pgfpathlineto{\pgfqpoint{3.849036in}{2.641169in}}%
\pgfpathlineto{\pgfqpoint{3.841294in}{2.633553in}}%
\pgfpathlineto{\pgfqpoint{3.828078in}{2.637697in}}%
\pgfpathlineto{\pgfqpoint{3.814868in}{2.641876in}}%
\pgfpathlineto{\pgfqpoint{3.801663in}{2.646089in}}%
\pgfpathlineto{\pgfqpoint{3.788463in}{2.650336in}}%
\pgfpathlineto{\pgfqpoint{3.796217in}{2.657970in}}%
\pgfpathlineto{\pgfqpoint{3.803966in}{2.665624in}}%
\pgfpathlineto{\pgfqpoint{3.811709in}{2.673299in}}%
\pgfpathlineto{\pgfqpoint{3.819446in}{2.680996in}}%
\pgfpathclose%
\pgfusepath{fill}%
\end{pgfscope}%
\begin{pgfscope}%
\pgfpathrectangle{\pgfqpoint{1.150000in}{0.150000in}}{\pgfqpoint{5.700000in}{5.700000in}}%
\pgfusepath{clip}%
\pgfsetbuttcap%
\pgfsetroundjoin%
\definecolor{currentfill}{rgb}{0.279566,0.067836,0.391917}%
\pgfsetfillcolor{currentfill}%
\pgfsetfillopacity{0.700000}%
\pgfsetlinewidth{0.000000pt}%
\definecolor{currentstroke}{rgb}{0.000000,0.000000,0.000000}%
\pgfsetstrokecolor{currentstroke}%
\pgfsetdash{}{0pt}%
\pgfpathmoveto{\pgfqpoint{4.972413in}{2.760487in}}%
\pgfpathlineto{\pgfqpoint{4.985856in}{2.757155in}}%
\pgfpathlineto{\pgfqpoint{4.999306in}{2.753848in}}%
\pgfpathlineto{\pgfqpoint{5.012762in}{2.750568in}}%
\pgfpathlineto{\pgfqpoint{5.026224in}{2.747313in}}%
\pgfpathlineto{\pgfqpoint{5.018904in}{2.739381in}}%
\pgfpathlineto{\pgfqpoint{5.011581in}{2.731540in}}%
\pgfpathlineto{\pgfqpoint{5.004254in}{2.723784in}}%
\pgfpathlineto{\pgfqpoint{4.996923in}{2.716108in}}%
\pgfpathlineto{\pgfqpoint{4.983445in}{2.719210in}}%
\pgfpathlineto{\pgfqpoint{4.969974in}{2.722337in}}%
\pgfpathlineto{\pgfqpoint{4.956509in}{2.725491in}}%
\pgfpathlineto{\pgfqpoint{4.943051in}{2.728670in}}%
\pgfpathlineto{\pgfqpoint{4.950397in}{2.736494in}}%
\pgfpathlineto{\pgfqpoint{4.957740in}{2.744401in}}%
\pgfpathlineto{\pgfqpoint{4.965078in}{2.752397in}}%
\pgfpathlineto{\pgfqpoint{4.972413in}{2.760487in}}%
\pgfpathclose%
\pgfusepath{fill}%
\end{pgfscope}%
\begin{pgfscope}%
\pgfpathrectangle{\pgfqpoint{1.150000in}{0.150000in}}{\pgfqpoint{5.700000in}{5.700000in}}%
\pgfusepath{clip}%
\pgfsetbuttcap%
\pgfsetroundjoin%
\definecolor{currentfill}{rgb}{0.272594,0.025563,0.353093}%
\pgfsetfillcolor{currentfill}%
\pgfsetfillopacity{0.700000}%
\pgfsetlinewidth{0.000000pt}%
\definecolor{currentstroke}{rgb}{0.000000,0.000000,0.000000}%
\pgfsetstrokecolor{currentstroke}%
\pgfsetdash{}{0pt}%
\pgfpathmoveto{\pgfqpoint{4.175934in}{2.693609in}}%
\pgfpathlineto{\pgfqpoint{4.189197in}{2.689843in}}%
\pgfpathlineto{\pgfqpoint{4.202466in}{2.686108in}}%
\pgfpathlineto{\pgfqpoint{4.215741in}{2.682403in}}%
\pgfpathlineto{\pgfqpoint{4.229022in}{2.678729in}}%
\pgfpathlineto{\pgfqpoint{4.221423in}{2.671042in}}%
\pgfpathlineto{\pgfqpoint{4.213819in}{2.663379in}}%
\pgfpathlineto{\pgfqpoint{4.206209in}{2.655737in}}%
\pgfpathlineto{\pgfqpoint{4.198595in}{2.648116in}}%
\pgfpathlineto{\pgfqpoint{4.185302in}{2.651728in}}%
\pgfpathlineto{\pgfqpoint{4.172015in}{2.655371in}}%
\pgfpathlineto{\pgfqpoint{4.158733in}{2.659045in}}%
\pgfpathlineto{\pgfqpoint{4.145458in}{2.662749in}}%
\pgfpathlineto{\pgfqpoint{4.153085in}{2.670427in}}%
\pgfpathlineto{\pgfqpoint{4.160707in}{2.678129in}}%
\pgfpathlineto{\pgfqpoint{4.168323in}{2.685856in}}%
\pgfpathlineto{\pgfqpoint{4.175934in}{2.693609in}}%
\pgfpathclose%
\pgfusepath{fill}%
\end{pgfscope}%
\begin{pgfscope}%
\pgfpathrectangle{\pgfqpoint{1.150000in}{0.150000in}}{\pgfqpoint{5.700000in}{5.700000in}}%
\pgfusepath{clip}%
\pgfsetbuttcap%
\pgfsetroundjoin%
\definecolor{currentfill}{rgb}{0.277941,0.056324,0.381191}%
\pgfsetfillcolor{currentfill}%
\pgfsetfillopacity{0.700000}%
\pgfsetlinewidth{0.000000pt}%
\definecolor{currentstroke}{rgb}{0.000000,0.000000,0.000000}%
\pgfsetstrokecolor{currentstroke}%
\pgfsetdash{}{0pt}%
\pgfpathmoveto{\pgfqpoint{4.752469in}{2.736632in}}%
\pgfpathlineto{\pgfqpoint{4.765863in}{2.733277in}}%
\pgfpathlineto{\pgfqpoint{4.779264in}{2.729948in}}%
\pgfpathlineto{\pgfqpoint{4.792671in}{2.726646in}}%
\pgfpathlineto{\pgfqpoint{4.806085in}{2.723371in}}%
\pgfpathlineto{\pgfqpoint{4.798690in}{2.715633in}}%
\pgfpathlineto{\pgfqpoint{4.791290in}{2.707959in}}%
\pgfpathlineto{\pgfqpoint{4.783886in}{2.700344in}}%
\pgfpathlineto{\pgfqpoint{4.776478in}{2.692784in}}%
\pgfpathlineto{\pgfqpoint{4.763050in}{2.695932in}}%
\pgfpathlineto{\pgfqpoint{4.749629in}{2.699107in}}%
\pgfpathlineto{\pgfqpoint{4.736214in}{2.702308in}}%
\pgfpathlineto{\pgfqpoint{4.722806in}{2.705536in}}%
\pgfpathlineto{\pgfqpoint{4.730229in}{2.713218in}}%
\pgfpathlineto{\pgfqpoint{4.737646in}{2.720959in}}%
\pgfpathlineto{\pgfqpoint{4.745060in}{2.728762in}}%
\pgfpathlineto{\pgfqpoint{4.752469in}{2.736632in}}%
\pgfpathclose%
\pgfusepath{fill}%
\end{pgfscope}%
\begin{pgfscope}%
\pgfpathrectangle{\pgfqpoint{1.150000in}{0.150000in}}{\pgfqpoint{5.700000in}{5.700000in}}%
\pgfusepath{clip}%
\pgfsetbuttcap%
\pgfsetroundjoin%
\definecolor{currentfill}{rgb}{0.274952,0.037752,0.364543}%
\pgfsetfillcolor{currentfill}%
\pgfsetfillopacity{0.700000}%
\pgfsetlinewidth{0.000000pt}%
\definecolor{currentstroke}{rgb}{0.000000,0.000000,0.000000}%
\pgfsetstrokecolor{currentstroke}%
\pgfsetdash{}{0pt}%
\pgfpathmoveto{\pgfqpoint{3.189864in}{2.709798in}}%
\pgfpathlineto{\pgfqpoint{3.202946in}{2.704083in}}%
\pgfpathlineto{\pgfqpoint{3.216033in}{2.698413in}}%
\pgfpathlineto{\pgfqpoint{3.229123in}{2.692789in}}%
\pgfpathlineto{\pgfqpoint{3.242218in}{2.687209in}}%
\pgfpathlineto{\pgfqpoint{3.234259in}{2.680124in}}%
\pgfpathlineto{\pgfqpoint{3.226293in}{2.673079in}}%
\pgfpathlineto{\pgfqpoint{3.218321in}{2.666075in}}%
\pgfpathlineto{\pgfqpoint{3.210343in}{2.659114in}}%
\pgfpathlineto{\pgfqpoint{3.197234in}{2.664736in}}%
\pgfpathlineto{\pgfqpoint{3.184130in}{2.670404in}}%
\pgfpathlineto{\pgfqpoint{3.171029in}{2.676117in}}%
\pgfpathlineto{\pgfqpoint{3.157933in}{2.681875in}}%
\pgfpathlineto{\pgfqpoint{3.165926in}{2.688789in}}%
\pgfpathlineto{\pgfqpoint{3.173912in}{2.695748in}}%
\pgfpathlineto{\pgfqpoint{3.181891in}{2.702751in}}%
\pgfpathlineto{\pgfqpoint{3.189864in}{2.709798in}}%
\pgfpathclose%
\pgfusepath{fill}%
\end{pgfscope}%
\begin{pgfscope}%
\pgfpathrectangle{\pgfqpoint{1.150000in}{0.150000in}}{\pgfqpoint{5.700000in}{5.700000in}}%
\pgfusepath{clip}%
\pgfsetbuttcap%
\pgfsetroundjoin%
\definecolor{currentfill}{rgb}{0.272594,0.025563,0.353093}%
\pgfsetfillcolor{currentfill}%
\pgfsetfillopacity{0.700000}%
\pgfsetlinewidth{0.000000pt}%
\definecolor{currentstroke}{rgb}{0.000000,0.000000,0.000000}%
\pgfsetstrokecolor{currentstroke}%
\pgfsetdash{}{0pt}%
\pgfpathmoveto{\pgfqpoint{3.326354in}{2.694193in}}%
\pgfpathlineto{\pgfqpoint{3.339456in}{2.688862in}}%
\pgfpathlineto{\pgfqpoint{3.352562in}{2.683574in}}%
\pgfpathlineto{\pgfqpoint{3.365673in}{2.678328in}}%
\pgfpathlineto{\pgfqpoint{3.378788in}{2.673124in}}%
\pgfpathlineto{\pgfqpoint{3.370881in}{2.665830in}}%
\pgfpathlineto{\pgfqpoint{3.362969in}{2.658569in}}%
\pgfpathlineto{\pgfqpoint{3.355050in}{2.651340in}}%
\pgfpathlineto{\pgfqpoint{3.347124in}{2.644145in}}%
\pgfpathlineto{\pgfqpoint{3.333996in}{2.649379in}}%
\pgfpathlineto{\pgfqpoint{3.320872in}{2.654655in}}%
\pgfpathlineto{\pgfqpoint{3.307752in}{2.659973in}}%
\pgfpathlineto{\pgfqpoint{3.294637in}{2.665333in}}%
\pgfpathlineto{\pgfqpoint{3.302576in}{2.672494in}}%
\pgfpathlineto{\pgfqpoint{3.310508in}{2.679691in}}%
\pgfpathlineto{\pgfqpoint{3.318434in}{2.686924in}}%
\pgfpathlineto{\pgfqpoint{3.326354in}{2.694193in}}%
\pgfpathclose%
\pgfusepath{fill}%
\end{pgfscope}%
\begin{pgfscope}%
\pgfpathrectangle{\pgfqpoint{1.150000in}{0.150000in}}{\pgfqpoint{5.700000in}{5.700000in}}%
\pgfusepath{clip}%
\pgfsetbuttcap%
\pgfsetroundjoin%
\definecolor{currentfill}{rgb}{0.277018,0.050344,0.375715}%
\pgfsetfillcolor{currentfill}%
\pgfsetfillopacity{0.700000}%
\pgfsetlinewidth{0.000000pt}%
\definecolor{currentstroke}{rgb}{0.000000,0.000000,0.000000}%
\pgfsetstrokecolor{currentstroke}%
\pgfsetdash{}{0pt}%
\pgfpathmoveto{\pgfqpoint{3.053296in}{2.729649in}}%
\pgfpathlineto{\pgfqpoint{3.066363in}{2.723507in}}%
\pgfpathlineto{\pgfqpoint{3.079433in}{2.717415in}}%
\pgfpathlineto{\pgfqpoint{3.092507in}{2.711372in}}%
\pgfpathlineto{\pgfqpoint{3.105585in}{2.705377in}}%
\pgfpathlineto{\pgfqpoint{3.097571in}{2.698561in}}%
\pgfpathlineto{\pgfqpoint{3.089550in}{2.691795in}}%
\pgfpathlineto{\pgfqpoint{3.081521in}{2.685079in}}%
\pgfpathlineto{\pgfqpoint{3.073486in}{2.678416in}}%
\pgfpathlineto{\pgfqpoint{3.060394in}{2.684467in}}%
\pgfpathlineto{\pgfqpoint{3.047305in}{2.690567in}}%
\pgfpathlineto{\pgfqpoint{3.034220in}{2.696715in}}%
\pgfpathlineto{\pgfqpoint{3.021138in}{2.702913in}}%
\pgfpathlineto{\pgfqpoint{3.029188in}{2.709515in}}%
\pgfpathlineto{\pgfqpoint{3.037231in}{2.716172in}}%
\pgfpathlineto{\pgfqpoint{3.045267in}{2.722884in}}%
\pgfpathlineto{\pgfqpoint{3.053296in}{2.729649in}}%
\pgfpathclose%
\pgfusepath{fill}%
\end{pgfscope}%
\begin{pgfscope}%
\pgfpathrectangle{\pgfqpoint{1.150000in}{0.150000in}}{\pgfqpoint{5.700000in}{5.700000in}}%
\pgfusepath{clip}%
\pgfsetbuttcap%
\pgfsetroundjoin%
\definecolor{currentfill}{rgb}{0.276022,0.044167,0.370164}%
\pgfsetfillcolor{currentfill}%
\pgfsetfillopacity{0.700000}%
\pgfsetlinewidth{0.000000pt}%
\definecolor{currentstroke}{rgb}{0.000000,0.000000,0.000000}%
\pgfsetstrokecolor{currentstroke}%
\pgfsetdash{}{0pt}%
\pgfpathmoveto{\pgfqpoint{4.532505in}{2.714811in}}%
\pgfpathlineto{\pgfqpoint{4.545851in}{2.711367in}}%
\pgfpathlineto{\pgfqpoint{4.559203in}{2.707952in}}%
\pgfpathlineto{\pgfqpoint{4.572561in}{2.704564in}}%
\pgfpathlineto{\pgfqpoint{4.585925in}{2.701204in}}%
\pgfpathlineto{\pgfqpoint{4.578452in}{2.693545in}}%
\pgfpathlineto{\pgfqpoint{4.570974in}{2.685928in}}%
\pgfpathlineto{\pgfqpoint{4.563491in}{2.678351in}}%
\pgfpathlineto{\pgfqpoint{4.556003in}{2.670810in}}%
\pgfpathlineto{\pgfqpoint{4.542625in}{2.674068in}}%
\pgfpathlineto{\pgfqpoint{4.529254in}{2.677354in}}%
\pgfpathlineto{\pgfqpoint{4.515889in}{2.680669in}}%
\pgfpathlineto{\pgfqpoint{4.502530in}{2.684011in}}%
\pgfpathlineto{\pgfqpoint{4.510031in}{2.691649in}}%
\pgfpathlineto{\pgfqpoint{4.517527in}{2.699326in}}%
\pgfpathlineto{\pgfqpoint{4.525019in}{2.707046in}}%
\pgfpathlineto{\pgfqpoint{4.532505in}{2.714811in}}%
\pgfpathclose%
\pgfusepath{fill}%
\end{pgfscope}%
\begin{pgfscope}%
\pgfpathrectangle{\pgfqpoint{1.150000in}{0.150000in}}{\pgfqpoint{5.700000in}{5.700000in}}%
\pgfusepath{clip}%
\pgfsetbuttcap%
\pgfsetroundjoin%
\definecolor{currentfill}{rgb}{0.271305,0.019942,0.347269}%
\pgfsetfillcolor{currentfill}%
\pgfsetfillopacity{0.700000}%
\pgfsetlinewidth{0.000000pt}%
\definecolor{currentstroke}{rgb}{0.000000,0.000000,0.000000}%
\pgfsetstrokecolor{currentstroke}%
\pgfsetdash{}{0pt}%
\pgfpathmoveto{\pgfqpoint{3.462805in}{2.682290in}}%
\pgfpathlineto{\pgfqpoint{3.475929in}{2.677306in}}%
\pgfpathlineto{\pgfqpoint{3.489059in}{2.672363in}}%
\pgfpathlineto{\pgfqpoint{3.502193in}{2.667459in}}%
\pgfpathlineto{\pgfqpoint{3.515332in}{2.662594in}}%
\pgfpathlineto{\pgfqpoint{3.507476in}{2.655146in}}%
\pgfpathlineto{\pgfqpoint{3.499614in}{2.647724in}}%
\pgfpathlineto{\pgfqpoint{3.491745in}{2.640328in}}%
\pgfpathlineto{\pgfqpoint{3.483871in}{2.632959in}}%
\pgfpathlineto{\pgfqpoint{3.470720in}{2.637840in}}%
\pgfpathlineto{\pgfqpoint{3.457573in}{2.642760in}}%
\pgfpathlineto{\pgfqpoint{3.444430in}{2.647720in}}%
\pgfpathlineto{\pgfqpoint{3.431293in}{2.652720in}}%
\pgfpathlineto{\pgfqpoint{3.439180in}{2.660068in}}%
\pgfpathlineto{\pgfqpoint{3.447061in}{2.667446in}}%
\pgfpathlineto{\pgfqpoint{3.454936in}{2.674853in}}%
\pgfpathlineto{\pgfqpoint{3.462805in}{2.682290in}}%
\pgfpathclose%
\pgfusepath{fill}%
\end{pgfscope}%
\begin{pgfscope}%
\pgfpathrectangle{\pgfqpoint{1.150000in}{0.150000in}}{\pgfqpoint{5.700000in}{5.700000in}}%
\pgfusepath{clip}%
\pgfsetbuttcap%
\pgfsetroundjoin%
\definecolor{currentfill}{rgb}{0.271305,0.019942,0.347269}%
\pgfsetfillcolor{currentfill}%
\pgfsetfillopacity{0.700000}%
\pgfsetlinewidth{0.000000pt}%
\definecolor{currentstroke}{rgb}{0.000000,0.000000,0.000000}%
\pgfsetstrokecolor{currentstroke}%
\pgfsetdash{}{0pt}%
\pgfpathmoveto{\pgfqpoint{3.955894in}{2.678535in}}%
\pgfpathlineto{\pgfqpoint{3.969113in}{2.674499in}}%
\pgfpathlineto{\pgfqpoint{3.982338in}{2.670495in}}%
\pgfpathlineto{\pgfqpoint{3.995568in}{2.666524in}}%
\pgfpathlineto{\pgfqpoint{4.008804in}{2.662586in}}%
\pgfpathlineto{\pgfqpoint{4.001125in}{2.654909in}}%
\pgfpathlineto{\pgfqpoint{3.993441in}{2.647251in}}%
\pgfpathlineto{\pgfqpoint{3.985751in}{2.639609in}}%
\pgfpathlineto{\pgfqpoint{3.978056in}{2.631982in}}%
\pgfpathlineto{\pgfqpoint{3.964808in}{2.635885in}}%
\pgfpathlineto{\pgfqpoint{3.951566in}{2.639820in}}%
\pgfpathlineto{\pgfqpoint{3.938329in}{2.643788in}}%
\pgfpathlineto{\pgfqpoint{3.925098in}{2.647788in}}%
\pgfpathlineto{\pgfqpoint{3.932805in}{2.655446in}}%
\pgfpathlineto{\pgfqpoint{3.940507in}{2.663122in}}%
\pgfpathlineto{\pgfqpoint{3.948204in}{2.670818in}}%
\pgfpathlineto{\pgfqpoint{3.955894in}{2.678535in}}%
\pgfpathclose%
\pgfusepath{fill}%
\end{pgfscope}%
\begin{pgfscope}%
\pgfpathrectangle{\pgfqpoint{1.150000in}{0.150000in}}{\pgfqpoint{5.700000in}{5.700000in}}%
\pgfusepath{clip}%
\pgfsetbuttcap%
\pgfsetroundjoin%
\definecolor{currentfill}{rgb}{0.273809,0.031497,0.358853}%
\pgfsetfillcolor{currentfill}%
\pgfsetfillopacity{0.700000}%
\pgfsetlinewidth{0.000000pt}%
\definecolor{currentstroke}{rgb}{0.000000,0.000000,0.000000}%
\pgfsetstrokecolor{currentstroke}%
\pgfsetdash{}{0pt}%
\pgfpathmoveto{\pgfqpoint{4.312496in}{2.695070in}}%
\pgfpathlineto{\pgfqpoint{4.325794in}{2.691470in}}%
\pgfpathlineto{\pgfqpoint{4.339098in}{2.687899in}}%
\pgfpathlineto{\pgfqpoint{4.352408in}{2.684358in}}%
\pgfpathlineto{\pgfqpoint{4.365724in}{2.680846in}}%
\pgfpathlineto{\pgfqpoint{4.358171in}{2.673198in}}%
\pgfpathlineto{\pgfqpoint{4.350613in}{2.665577in}}%
\pgfpathlineto{\pgfqpoint{4.343050in}{2.657981in}}%
\pgfpathlineto{\pgfqpoint{4.335482in}{2.650408in}}%
\pgfpathlineto{\pgfqpoint{4.322153in}{2.653845in}}%
\pgfpathlineto{\pgfqpoint{4.308831in}{2.657311in}}%
\pgfpathlineto{\pgfqpoint{4.295514in}{2.660806in}}%
\pgfpathlineto{\pgfqpoint{4.282204in}{2.664331in}}%
\pgfpathlineto{\pgfqpoint{4.289784in}{2.671975in}}%
\pgfpathlineto{\pgfqpoint{4.297360in}{2.679644in}}%
\pgfpathlineto{\pgfqpoint{4.304931in}{2.687341in}}%
\pgfpathlineto{\pgfqpoint{4.312496in}{2.695070in}}%
\pgfpathclose%
\pgfusepath{fill}%
\end{pgfscope}%
\begin{pgfscope}%
\pgfpathrectangle{\pgfqpoint{1.150000in}{0.150000in}}{\pgfqpoint{5.700000in}{5.700000in}}%
\pgfusepath{clip}%
\pgfsetbuttcap%
\pgfsetroundjoin%
\definecolor{currentfill}{rgb}{0.271305,0.019942,0.347269}%
\pgfsetfillcolor{currentfill}%
\pgfsetfillopacity{0.700000}%
\pgfsetlinewidth{0.000000pt}%
\definecolor{currentstroke}{rgb}{0.000000,0.000000,0.000000}%
\pgfsetstrokecolor{currentstroke}%
\pgfsetdash{}{0pt}%
\pgfpathmoveto{\pgfqpoint{3.599250in}{2.673597in}}%
\pgfpathlineto{\pgfqpoint{3.612400in}{2.668927in}}%
\pgfpathlineto{\pgfqpoint{3.625556in}{2.664295in}}%
\pgfpathlineto{\pgfqpoint{3.638716in}{2.659700in}}%
\pgfpathlineto{\pgfqpoint{3.651881in}{2.655142in}}%
\pgfpathlineto{\pgfqpoint{3.644074in}{2.647588in}}%
\pgfpathlineto{\pgfqpoint{3.636261in}{2.640056in}}%
\pgfpathlineto{\pgfqpoint{3.628442in}{2.632544in}}%
\pgfpathlineto{\pgfqpoint{3.620617in}{2.625053in}}%
\pgfpathlineto{\pgfqpoint{3.607439in}{2.629614in}}%
\pgfpathlineto{\pgfqpoint{3.594266in}{2.634213in}}%
\pgfpathlineto{\pgfqpoint{3.581098in}{2.638849in}}%
\pgfpathlineto{\pgfqpoint{3.567935in}{2.643522in}}%
\pgfpathlineto{\pgfqpoint{3.575773in}{2.651005in}}%
\pgfpathlineto{\pgfqpoint{3.583604in}{2.658512in}}%
\pgfpathlineto{\pgfqpoint{3.591430in}{2.666042in}}%
\pgfpathlineto{\pgfqpoint{3.599250in}{2.673597in}}%
\pgfpathclose%
\pgfusepath{fill}%
\end{pgfscope}%
\begin{pgfscope}%
\pgfpathrectangle{\pgfqpoint{1.150000in}{0.150000in}}{\pgfqpoint{5.700000in}{5.700000in}}%
\pgfusepath{clip}%
\pgfsetbuttcap%
\pgfsetroundjoin%
\definecolor{currentfill}{rgb}{0.280894,0.078907,0.402329}%
\pgfsetfillcolor{currentfill}%
\pgfsetfillopacity{0.700000}%
\pgfsetlinewidth{0.000000pt}%
\definecolor{currentstroke}{rgb}{0.000000,0.000000,0.000000}%
\pgfsetstrokecolor{currentstroke}%
\pgfsetdash{}{0pt}%
\pgfpathmoveto{\pgfqpoint{5.109328in}{2.766614in}}%
\pgfpathlineto{\pgfqpoint{5.122809in}{2.763320in}}%
\pgfpathlineto{\pgfqpoint{5.136296in}{2.760052in}}%
\pgfpathlineto{\pgfqpoint{5.149789in}{2.756809in}}%
\pgfpathlineto{\pgfqpoint{5.163289in}{2.753591in}}%
\pgfpathlineto{\pgfqpoint{5.156013in}{2.745590in}}%
\pgfpathlineto{\pgfqpoint{5.148733in}{2.737693in}}%
\pgfpathlineto{\pgfqpoint{5.141451in}{2.729895in}}%
\pgfpathlineto{\pgfqpoint{5.134165in}{2.722191in}}%
\pgfpathlineto{\pgfqpoint{5.120649in}{2.725243in}}%
\pgfpathlineto{\pgfqpoint{5.107140in}{2.728319in}}%
\pgfpathlineto{\pgfqpoint{5.093637in}{2.731421in}}%
\pgfpathlineto{\pgfqpoint{5.080141in}{2.734549in}}%
\pgfpathlineto{\pgfqpoint{5.087443in}{2.742414in}}%
\pgfpathlineto{\pgfqpoint{5.094741in}{2.750377in}}%
\pgfpathlineto{\pgfqpoint{5.102036in}{2.758442in}}%
\pgfpathlineto{\pgfqpoint{5.109328in}{2.766614in}}%
\pgfpathclose%
\pgfusepath{fill}%
\end{pgfscope}%
\begin{pgfscope}%
\pgfpathrectangle{\pgfqpoint{1.150000in}{0.150000in}}{\pgfqpoint{5.700000in}{5.700000in}}%
\pgfusepath{clip}%
\pgfsetbuttcap%
\pgfsetroundjoin%
\definecolor{currentfill}{rgb}{0.278791,0.062145,0.386592}%
\pgfsetfillcolor{currentfill}%
\pgfsetfillopacity{0.700000}%
\pgfsetlinewidth{0.000000pt}%
\definecolor{currentstroke}{rgb}{0.000000,0.000000,0.000000}%
\pgfsetstrokecolor{currentstroke}%
\pgfsetdash{}{0pt}%
\pgfpathmoveto{\pgfqpoint{4.889286in}{2.741646in}}%
\pgfpathlineto{\pgfqpoint{4.902717in}{2.738363in}}%
\pgfpathlineto{\pgfqpoint{4.916156in}{2.735106in}}%
\pgfpathlineto{\pgfqpoint{4.929600in}{2.731875in}}%
\pgfpathlineto{\pgfqpoint{4.943051in}{2.728670in}}%
\pgfpathlineto{\pgfqpoint{4.935702in}{2.720925in}}%
\pgfpathlineto{\pgfqpoint{4.928348in}{2.713255in}}%
\pgfpathlineto{\pgfqpoint{4.920990in}{2.705655in}}%
\pgfpathlineto{\pgfqpoint{4.913629in}{2.698120in}}%
\pgfpathlineto{\pgfqpoint{4.900163in}{2.701185in}}%
\pgfpathlineto{\pgfqpoint{4.886703in}{2.704275in}}%
\pgfpathlineto{\pgfqpoint{4.873250in}{2.707392in}}%
\pgfpathlineto{\pgfqpoint{4.859804in}{2.710535in}}%
\pgfpathlineto{\pgfqpoint{4.867181in}{2.718206in}}%
\pgfpathlineto{\pgfqpoint{4.874553in}{2.725945in}}%
\pgfpathlineto{\pgfqpoint{4.881921in}{2.733757in}}%
\pgfpathlineto{\pgfqpoint{4.889286in}{2.741646in}}%
\pgfpathclose%
\pgfusepath{fill}%
\end{pgfscope}%
\begin{pgfscope}%
\pgfpathrectangle{\pgfqpoint{1.150000in}{0.150000in}}{\pgfqpoint{5.700000in}{5.700000in}}%
\pgfusepath{clip}%
\pgfsetbuttcap%
\pgfsetroundjoin%
\definecolor{currentfill}{rgb}{0.272594,0.025563,0.353093}%
\pgfsetfillcolor{currentfill}%
\pgfsetfillopacity{0.700000}%
\pgfsetlinewidth{0.000000pt}%
\definecolor{currentstroke}{rgb}{0.000000,0.000000,0.000000}%
\pgfsetstrokecolor{currentstroke}%
\pgfsetdash{}{0pt}%
\pgfpathmoveto{\pgfqpoint{4.092415in}{2.677875in}}%
\pgfpathlineto{\pgfqpoint{4.105667in}{2.674047in}}%
\pgfpathlineto{\pgfqpoint{4.118925in}{2.670250in}}%
\pgfpathlineto{\pgfqpoint{4.132189in}{2.666484in}}%
\pgfpathlineto{\pgfqpoint{4.145458in}{2.662749in}}%
\pgfpathlineto{\pgfqpoint{4.137826in}{2.655091in}}%
\pgfpathlineto{\pgfqpoint{4.130188in}{2.647451in}}%
\pgfpathlineto{\pgfqpoint{4.122544in}{2.639829in}}%
\pgfpathlineto{\pgfqpoint{4.114896in}{2.632221in}}%
\pgfpathlineto{\pgfqpoint{4.101614in}{2.635907in}}%
\pgfpathlineto{\pgfqpoint{4.088338in}{2.639624in}}%
\pgfpathlineto{\pgfqpoint{4.075068in}{2.643372in}}%
\pgfpathlineto{\pgfqpoint{4.061804in}{2.647151in}}%
\pgfpathlineto{\pgfqpoint{4.069465in}{2.654803in}}%
\pgfpathlineto{\pgfqpoint{4.077120in}{2.662473in}}%
\pgfpathlineto{\pgfqpoint{4.084771in}{2.670163in}}%
\pgfpathlineto{\pgfqpoint{4.092415in}{2.677875in}}%
\pgfpathclose%
\pgfusepath{fill}%
\end{pgfscope}%
\begin{pgfscope}%
\pgfpathrectangle{\pgfqpoint{1.150000in}{0.150000in}}{\pgfqpoint{5.700000in}{5.700000in}}%
\pgfusepath{clip}%
\pgfsetbuttcap%
\pgfsetroundjoin%
\definecolor{currentfill}{rgb}{0.271305,0.019942,0.347269}%
\pgfsetfillcolor{currentfill}%
\pgfsetfillopacity{0.700000}%
\pgfsetlinewidth{0.000000pt}%
\definecolor{currentstroke}{rgb}{0.000000,0.000000,0.000000}%
\pgfsetstrokecolor{currentstroke}%
\pgfsetdash{}{0pt}%
\pgfpathmoveto{\pgfqpoint{3.735717in}{2.667674in}}%
\pgfpathlineto{\pgfqpoint{3.748895in}{2.663286in}}%
\pgfpathlineto{\pgfqpoint{3.762079in}{2.658934in}}%
\pgfpathlineto{\pgfqpoint{3.775269in}{2.654618in}}%
\pgfpathlineto{\pgfqpoint{3.788463in}{2.650336in}}%
\pgfpathlineto{\pgfqpoint{3.780703in}{2.642720in}}%
\pgfpathlineto{\pgfqpoint{3.772937in}{2.635121in}}%
\pgfpathlineto{\pgfqpoint{3.765166in}{2.627540in}}%
\pgfpathlineto{\pgfqpoint{3.757389in}{2.619974in}}%
\pgfpathlineto{\pgfqpoint{3.744183in}{2.624246in}}%
\pgfpathlineto{\pgfqpoint{3.730981in}{2.628553in}}%
\pgfpathlineto{\pgfqpoint{3.717785in}{2.632895in}}%
\pgfpathlineto{\pgfqpoint{3.704594in}{2.637273in}}%
\pgfpathlineto{\pgfqpoint{3.712383in}{2.644843in}}%
\pgfpathlineto{\pgfqpoint{3.720167in}{2.652433in}}%
\pgfpathlineto{\pgfqpoint{3.727945in}{2.660043in}}%
\pgfpathlineto{\pgfqpoint{3.735717in}{2.667674in}}%
\pgfpathclose%
\pgfusepath{fill}%
\end{pgfscope}%
\begin{pgfscope}%
\pgfpathrectangle{\pgfqpoint{1.150000in}{0.150000in}}{\pgfqpoint{5.700000in}{5.700000in}}%
\pgfusepath{clip}%
\pgfsetbuttcap%
\pgfsetroundjoin%
\definecolor{currentfill}{rgb}{0.277018,0.050344,0.375715}%
\pgfsetfillcolor{currentfill}%
\pgfsetfillopacity{0.700000}%
\pgfsetlinewidth{0.000000pt}%
\definecolor{currentstroke}{rgb}{0.000000,0.000000,0.000000}%
\pgfsetstrokecolor{currentstroke}%
\pgfsetdash{}{0pt}%
\pgfpathmoveto{\pgfqpoint{4.669237in}{2.718719in}}%
\pgfpathlineto{\pgfqpoint{4.682620in}{2.715383in}}%
\pgfpathlineto{\pgfqpoint{4.696009in}{2.712074in}}%
\pgfpathlineto{\pgfqpoint{4.709404in}{2.708791in}}%
\pgfpathlineto{\pgfqpoint{4.722806in}{2.705536in}}%
\pgfpathlineto{\pgfqpoint{4.715379in}{2.697909in}}%
\pgfpathlineto{\pgfqpoint{4.707948in}{2.690332in}}%
\pgfpathlineto{\pgfqpoint{4.700511in}{2.682802in}}%
\pgfpathlineto{\pgfqpoint{4.693070in}{2.675315in}}%
\pgfpathlineto{\pgfqpoint{4.679655in}{2.678456in}}%
\pgfpathlineto{\pgfqpoint{4.666246in}{2.681624in}}%
\pgfpathlineto{\pgfqpoint{4.652843in}{2.684819in}}%
\pgfpathlineto{\pgfqpoint{4.639447in}{2.688041in}}%
\pgfpathlineto{\pgfqpoint{4.646901in}{2.695637in}}%
\pgfpathlineto{\pgfqpoint{4.654351in}{2.703280in}}%
\pgfpathlineto{\pgfqpoint{4.661797in}{2.710973in}}%
\pgfpathlineto{\pgfqpoint{4.669237in}{2.718719in}}%
\pgfpathclose%
\pgfusepath{fill}%
\end{pgfscope}%
\begin{pgfscope}%
\pgfpathrectangle{\pgfqpoint{1.150000in}{0.150000in}}{\pgfqpoint{5.700000in}{5.700000in}}%
\pgfusepath{clip}%
\pgfsetbuttcap%
\pgfsetroundjoin%
\definecolor{currentfill}{rgb}{0.274952,0.037752,0.364543}%
\pgfsetfillcolor{currentfill}%
\pgfsetfillopacity{0.700000}%
\pgfsetlinewidth{0.000000pt}%
\definecolor{currentstroke}{rgb}{0.000000,0.000000,0.000000}%
\pgfsetstrokecolor{currentstroke}%
\pgfsetdash{}{0pt}%
\pgfpathmoveto{\pgfqpoint{4.449157in}{2.697664in}}%
\pgfpathlineto{\pgfqpoint{4.462491in}{2.694208in}}%
\pgfpathlineto{\pgfqpoint{4.475831in}{2.690781in}}%
\pgfpathlineto{\pgfqpoint{4.489177in}{2.687382in}}%
\pgfpathlineto{\pgfqpoint{4.502530in}{2.684011in}}%
\pgfpathlineto{\pgfqpoint{4.495024in}{2.676409in}}%
\pgfpathlineto{\pgfqpoint{4.487513in}{2.668841in}}%
\pgfpathlineto{\pgfqpoint{4.479996in}{2.661301in}}%
\pgfpathlineto{\pgfqpoint{4.472475in}{2.653789in}}%
\pgfpathlineto{\pgfqpoint{4.459109in}{2.657072in}}%
\pgfpathlineto{\pgfqpoint{4.445750in}{2.660382in}}%
\pgfpathlineto{\pgfqpoint{4.432396in}{2.663721in}}%
\pgfpathlineto{\pgfqpoint{4.419049in}{2.667089in}}%
\pgfpathlineto{\pgfqpoint{4.426584in}{2.674684in}}%
\pgfpathlineto{\pgfqpoint{4.434113in}{2.682310in}}%
\pgfpathlineto{\pgfqpoint{4.441638in}{2.689969in}}%
\pgfpathlineto{\pgfqpoint{4.449157in}{2.697664in}}%
\pgfpathclose%
\pgfusepath{fill}%
\end{pgfscope}%
\begin{pgfscope}%
\pgfpathrectangle{\pgfqpoint{1.150000in}{0.150000in}}{\pgfqpoint{5.700000in}{5.700000in}}%
\pgfusepath{clip}%
\pgfsetbuttcap%
\pgfsetroundjoin%
\definecolor{currentfill}{rgb}{0.271305,0.019942,0.347269}%
\pgfsetfillcolor{currentfill}%
\pgfsetfillopacity{0.700000}%
\pgfsetlinewidth{0.000000pt}%
\definecolor{currentstroke}{rgb}{0.000000,0.000000,0.000000}%
\pgfsetstrokecolor{currentstroke}%
\pgfsetdash{}{0pt}%
\pgfpathmoveto{\pgfqpoint{3.872228in}{2.664122in}}%
\pgfpathlineto{\pgfqpoint{3.885438in}{2.659988in}}%
\pgfpathlineto{\pgfqpoint{3.898652in}{2.655888in}}%
\pgfpathlineto{\pgfqpoint{3.911872in}{2.651822in}}%
\pgfpathlineto{\pgfqpoint{3.925098in}{2.647788in}}%
\pgfpathlineto{\pgfqpoint{3.917385in}{2.640147in}}%
\pgfpathlineto{\pgfqpoint{3.909666in}{2.632521in}}%
\pgfpathlineto{\pgfqpoint{3.901942in}{2.624909in}}%
\pgfpathlineto{\pgfqpoint{3.894212in}{2.617310in}}%
\pgfpathlineto{\pgfqpoint{3.880974in}{2.621321in}}%
\pgfpathlineto{\pgfqpoint{3.867742in}{2.625365in}}%
\pgfpathlineto{\pgfqpoint{3.854515in}{2.629442in}}%
\pgfpathlineto{\pgfqpoint{3.841294in}{2.633553in}}%
\pgfpathlineto{\pgfqpoint{3.849036in}{2.641169in}}%
\pgfpathlineto{\pgfqpoint{3.856773in}{2.648802in}}%
\pgfpathlineto{\pgfqpoint{3.864503in}{2.656453in}}%
\pgfpathlineto{\pgfqpoint{3.872228in}{2.664122in}}%
\pgfpathclose%
\pgfusepath{fill}%
\end{pgfscope}%
\begin{pgfscope}%
\pgfpathrectangle{\pgfqpoint{1.150000in}{0.150000in}}{\pgfqpoint{5.700000in}{5.700000in}}%
\pgfusepath{clip}%
\pgfsetbuttcap%
\pgfsetroundjoin%
\definecolor{currentfill}{rgb}{0.273809,0.031497,0.358853}%
\pgfsetfillcolor{currentfill}%
\pgfsetfillopacity{0.700000}%
\pgfsetlinewidth{0.000000pt}%
\definecolor{currentstroke}{rgb}{0.000000,0.000000,0.000000}%
\pgfsetstrokecolor{currentstroke}%
\pgfsetdash{}{0pt}%
\pgfpathmoveto{\pgfqpoint{3.242218in}{2.687209in}}%
\pgfpathlineto{\pgfqpoint{3.255316in}{2.681674in}}%
\pgfpathlineto{\pgfqpoint{3.268419in}{2.676184in}}%
\pgfpathlineto{\pgfqpoint{3.281526in}{2.670737in}}%
\pgfpathlineto{\pgfqpoint{3.294637in}{2.665333in}}%
\pgfpathlineto{\pgfqpoint{3.286692in}{2.658210in}}%
\pgfpathlineto{\pgfqpoint{3.278740in}{2.651123in}}%
\pgfpathlineto{\pgfqpoint{3.270782in}{2.644075in}}%
\pgfpathlineto{\pgfqpoint{3.262817in}{2.637066in}}%
\pgfpathlineto{\pgfqpoint{3.249692in}{2.642513in}}%
\pgfpathlineto{\pgfqpoint{3.236572in}{2.648002in}}%
\pgfpathlineto{\pgfqpoint{3.223455in}{2.653536in}}%
\pgfpathlineto{\pgfqpoint{3.210343in}{2.659114in}}%
\pgfpathlineto{\pgfqpoint{3.218321in}{2.666075in}}%
\pgfpathlineto{\pgfqpoint{3.226293in}{2.673079in}}%
\pgfpathlineto{\pgfqpoint{3.234259in}{2.680124in}}%
\pgfpathlineto{\pgfqpoint{3.242218in}{2.687209in}}%
\pgfpathclose%
\pgfusepath{fill}%
\end{pgfscope}%
\begin{pgfscope}%
\pgfpathrectangle{\pgfqpoint{1.150000in}{0.150000in}}{\pgfqpoint{5.700000in}{5.700000in}}%
\pgfusepath{clip}%
\pgfsetbuttcap%
\pgfsetroundjoin%
\definecolor{currentfill}{rgb}{0.276022,0.044167,0.370164}%
\pgfsetfillcolor{currentfill}%
\pgfsetfillopacity{0.700000}%
\pgfsetlinewidth{0.000000pt}%
\definecolor{currentstroke}{rgb}{0.000000,0.000000,0.000000}%
\pgfsetstrokecolor{currentstroke}%
\pgfsetdash{}{0pt}%
\pgfpathmoveto{\pgfqpoint{3.105585in}{2.705377in}}%
\pgfpathlineto{\pgfqpoint{3.118666in}{2.699431in}}%
\pgfpathlineto{\pgfqpoint{3.131751in}{2.693532in}}%
\pgfpathlineto{\pgfqpoint{3.144840in}{2.687680in}}%
\pgfpathlineto{\pgfqpoint{3.157933in}{2.681875in}}%
\pgfpathlineto{\pgfqpoint{3.149933in}{2.675008in}}%
\pgfpathlineto{\pgfqpoint{3.141927in}{2.668187in}}%
\pgfpathlineto{\pgfqpoint{3.133913in}{2.661414in}}%
\pgfpathlineto{\pgfqpoint{3.125893in}{2.654690in}}%
\pgfpathlineto{\pgfqpoint{3.112786in}{2.660551in}}%
\pgfpathlineto{\pgfqpoint{3.099682in}{2.666459in}}%
\pgfpathlineto{\pgfqpoint{3.086582in}{2.672414in}}%
\pgfpathlineto{\pgfqpoint{3.073486in}{2.678416in}}%
\pgfpathlineto{\pgfqpoint{3.081521in}{2.685079in}}%
\pgfpathlineto{\pgfqpoint{3.089550in}{2.691795in}}%
\pgfpathlineto{\pgfqpoint{3.097571in}{2.698561in}}%
\pgfpathlineto{\pgfqpoint{3.105585in}{2.705377in}}%
\pgfpathclose%
\pgfusepath{fill}%
\end{pgfscope}%
\begin{pgfscope}%
\pgfpathrectangle{\pgfqpoint{1.150000in}{0.150000in}}{\pgfqpoint{5.700000in}{5.700000in}}%
\pgfusepath{clip}%
\pgfsetbuttcap%
\pgfsetroundjoin%
\definecolor{currentfill}{rgb}{0.272594,0.025563,0.353093}%
\pgfsetfillcolor{currentfill}%
\pgfsetfillopacity{0.700000}%
\pgfsetlinewidth{0.000000pt}%
\definecolor{currentstroke}{rgb}{0.000000,0.000000,0.000000}%
\pgfsetstrokecolor{currentstroke}%
\pgfsetdash{}{0pt}%
\pgfpathmoveto{\pgfqpoint{3.378788in}{2.673124in}}%
\pgfpathlineto{\pgfqpoint{3.391907in}{2.667962in}}%
\pgfpathlineto{\pgfqpoint{3.405031in}{2.662841in}}%
\pgfpathlineto{\pgfqpoint{3.418160in}{2.657760in}}%
\pgfpathlineto{\pgfqpoint{3.431293in}{2.652720in}}%
\pgfpathlineto{\pgfqpoint{3.423399in}{2.645401in}}%
\pgfpathlineto{\pgfqpoint{3.415500in}{2.638112in}}%
\pgfpathlineto{\pgfqpoint{3.407594in}{2.630852in}}%
\pgfpathlineto{\pgfqpoint{3.399682in}{2.623622in}}%
\pgfpathlineto{\pgfqpoint{3.386536in}{2.628692in}}%
\pgfpathlineto{\pgfqpoint{3.373394in}{2.633802in}}%
\pgfpathlineto{\pgfqpoint{3.360257in}{2.638953in}}%
\pgfpathlineto{\pgfqpoint{3.347124in}{2.644145in}}%
\pgfpathlineto{\pgfqpoint{3.355050in}{2.651340in}}%
\pgfpathlineto{\pgfqpoint{3.362969in}{2.658569in}}%
\pgfpathlineto{\pgfqpoint{3.370881in}{2.665830in}}%
\pgfpathlineto{\pgfqpoint{3.378788in}{2.673124in}}%
\pgfpathclose%
\pgfusepath{fill}%
\end{pgfscope}%
\begin{pgfscope}%
\pgfpathrectangle{\pgfqpoint{1.150000in}{0.150000in}}{\pgfqpoint{5.700000in}{5.700000in}}%
\pgfusepath{clip}%
\pgfsetbuttcap%
\pgfsetroundjoin%
\definecolor{currentfill}{rgb}{0.273809,0.031497,0.358853}%
\pgfsetfillcolor{currentfill}%
\pgfsetfillopacity{0.700000}%
\pgfsetlinewidth{0.000000pt}%
\definecolor{currentstroke}{rgb}{0.000000,0.000000,0.000000}%
\pgfsetstrokecolor{currentstroke}%
\pgfsetdash{}{0pt}%
\pgfpathmoveto{\pgfqpoint{4.229022in}{2.678729in}}%
\pgfpathlineto{\pgfqpoint{4.242308in}{2.675084in}}%
\pgfpathlineto{\pgfqpoint{4.255601in}{2.671470in}}%
\pgfpathlineto{\pgfqpoint{4.268899in}{2.667886in}}%
\pgfpathlineto{\pgfqpoint{4.282204in}{2.664331in}}%
\pgfpathlineto{\pgfqpoint{4.274617in}{2.656711in}}%
\pgfpathlineto{\pgfqpoint{4.267026in}{2.649112in}}%
\pgfpathlineto{\pgfqpoint{4.259429in}{2.641531in}}%
\pgfpathlineto{\pgfqpoint{4.251827in}{2.633967in}}%
\pgfpathlineto{\pgfqpoint{4.238510in}{2.637459in}}%
\pgfpathlineto{\pgfqpoint{4.225199in}{2.640982in}}%
\pgfpathlineto{\pgfqpoint{4.211894in}{2.644534in}}%
\pgfpathlineto{\pgfqpoint{4.198595in}{2.648116in}}%
\pgfpathlineto{\pgfqpoint{4.206209in}{2.655737in}}%
\pgfpathlineto{\pgfqpoint{4.213819in}{2.663379in}}%
\pgfpathlineto{\pgfqpoint{4.221423in}{2.671042in}}%
\pgfpathlineto{\pgfqpoint{4.229022in}{2.678729in}}%
\pgfpathclose%
\pgfusepath{fill}%
\end{pgfscope}%
\begin{pgfscope}%
\pgfpathrectangle{\pgfqpoint{1.150000in}{0.150000in}}{\pgfqpoint{5.700000in}{5.700000in}}%
\pgfusepath{clip}%
\pgfsetbuttcap%
\pgfsetroundjoin%
\definecolor{currentfill}{rgb}{0.280267,0.073417,0.397163}%
\pgfsetfillcolor{currentfill}%
\pgfsetfillopacity{0.700000}%
\pgfsetlinewidth{0.000000pt}%
\definecolor{currentstroke}{rgb}{0.000000,0.000000,0.000000}%
\pgfsetstrokecolor{currentstroke}%
\pgfsetdash{}{0pt}%
\pgfpathmoveto{\pgfqpoint{5.026224in}{2.747313in}}%
\pgfpathlineto{\pgfqpoint{5.039694in}{2.744083in}}%
\pgfpathlineto{\pgfqpoint{5.053169in}{2.740880in}}%
\pgfpathlineto{\pgfqpoint{5.066652in}{2.737702in}}%
\pgfpathlineto{\pgfqpoint{5.080141in}{2.734549in}}%
\pgfpathlineto{\pgfqpoint{5.072836in}{2.726775in}}%
\pgfpathlineto{\pgfqpoint{5.065528in}{2.719089in}}%
\pgfpathlineto{\pgfqpoint{5.058216in}{2.711485in}}%
\pgfpathlineto{\pgfqpoint{5.050901in}{2.703958in}}%
\pgfpathlineto{\pgfqpoint{5.037396in}{2.706957in}}%
\pgfpathlineto{\pgfqpoint{5.023898in}{2.709982in}}%
\pgfpathlineto{\pgfqpoint{5.010407in}{2.713032in}}%
\pgfpathlineto{\pgfqpoint{4.996923in}{2.716108in}}%
\pgfpathlineto{\pgfqpoint{5.004254in}{2.723784in}}%
\pgfpathlineto{\pgfqpoint{5.011581in}{2.731540in}}%
\pgfpathlineto{\pgfqpoint{5.018904in}{2.739381in}}%
\pgfpathlineto{\pgfqpoint{5.026224in}{2.747313in}}%
\pgfpathclose%
\pgfusepath{fill}%
\end{pgfscope}%
\begin{pgfscope}%
\pgfpathrectangle{\pgfqpoint{1.150000in}{0.150000in}}{\pgfqpoint{5.700000in}{5.700000in}}%
\pgfusepath{clip}%
\pgfsetbuttcap%
\pgfsetroundjoin%
\definecolor{currentfill}{rgb}{0.277941,0.056324,0.381191}%
\pgfsetfillcolor{currentfill}%
\pgfsetfillopacity{0.700000}%
\pgfsetlinewidth{0.000000pt}%
\definecolor{currentstroke}{rgb}{0.000000,0.000000,0.000000}%
\pgfsetstrokecolor{currentstroke}%
\pgfsetdash{}{0pt}%
\pgfpathmoveto{\pgfqpoint{2.968845in}{2.728212in}}%
\pgfpathlineto{\pgfqpoint{2.981913in}{2.721810in}}%
\pgfpathlineto{\pgfqpoint{2.994985in}{2.715460in}}%
\pgfpathlineto{\pgfqpoint{3.008060in}{2.709161in}}%
\pgfpathlineto{\pgfqpoint{3.021138in}{2.702913in}}%
\pgfpathlineto{\pgfqpoint{3.013081in}{2.696368in}}%
\pgfpathlineto{\pgfqpoint{3.005016in}{2.689880in}}%
\pgfpathlineto{\pgfqpoint{2.996944in}{2.683452in}}%
\pgfpathlineto{\pgfqpoint{2.988864in}{2.677084in}}%
\pgfpathlineto{\pgfqpoint{2.975770in}{2.683402in}}%
\pgfpathlineto{\pgfqpoint{2.962680in}{2.689770in}}%
\pgfpathlineto{\pgfqpoint{2.949593in}{2.696189in}}%
\pgfpathlineto{\pgfqpoint{2.936509in}{2.702660in}}%
\pgfpathlineto{\pgfqpoint{2.944604in}{2.708954in}}%
\pgfpathlineto{\pgfqpoint{2.952692in}{2.715311in}}%
\pgfpathlineto{\pgfqpoint{2.960772in}{2.721731in}}%
\pgfpathlineto{\pgfqpoint{2.968845in}{2.728212in}}%
\pgfpathclose%
\pgfusepath{fill}%
\end{pgfscope}%
\begin{pgfscope}%
\pgfpathrectangle{\pgfqpoint{1.150000in}{0.150000in}}{\pgfqpoint{5.700000in}{5.700000in}}%
\pgfusepath{clip}%
\pgfsetbuttcap%
\pgfsetroundjoin%
\definecolor{currentfill}{rgb}{0.277941,0.056324,0.381191}%
\pgfsetfillcolor{currentfill}%
\pgfsetfillopacity{0.700000}%
\pgfsetlinewidth{0.000000pt}%
\definecolor{currentstroke}{rgb}{0.000000,0.000000,0.000000}%
\pgfsetstrokecolor{currentstroke}%
\pgfsetdash{}{0pt}%
\pgfpathmoveto{\pgfqpoint{4.806085in}{2.723371in}}%
\pgfpathlineto{\pgfqpoint{4.819505in}{2.720123in}}%
\pgfpathlineto{\pgfqpoint{4.832931in}{2.716900in}}%
\pgfpathlineto{\pgfqpoint{4.846364in}{2.713705in}}%
\pgfpathlineto{\pgfqpoint{4.859804in}{2.710535in}}%
\pgfpathlineto{\pgfqpoint{4.852423in}{2.702930in}}%
\pgfpathlineto{\pgfqpoint{4.845038in}{2.695384in}}%
\pgfpathlineto{\pgfqpoint{4.837649in}{2.687895in}}%
\pgfpathlineto{\pgfqpoint{4.830255in}{2.680458in}}%
\pgfpathlineto{\pgfqpoint{4.816801in}{2.683500in}}%
\pgfpathlineto{\pgfqpoint{4.803353in}{2.686568in}}%
\pgfpathlineto{\pgfqpoint{4.789912in}{2.689663in}}%
\pgfpathlineto{\pgfqpoint{4.776478in}{2.692784in}}%
\pgfpathlineto{\pgfqpoint{4.783886in}{2.700344in}}%
\pgfpathlineto{\pgfqpoint{4.791290in}{2.707959in}}%
\pgfpathlineto{\pgfqpoint{4.798690in}{2.715633in}}%
\pgfpathlineto{\pgfqpoint{4.806085in}{2.723371in}}%
\pgfpathclose%
\pgfusepath{fill}%
\end{pgfscope}%
\begin{pgfscope}%
\pgfpathrectangle{\pgfqpoint{1.150000in}{0.150000in}}{\pgfqpoint{5.700000in}{5.700000in}}%
\pgfusepath{clip}%
\pgfsetbuttcap%
\pgfsetroundjoin%
\definecolor{currentfill}{rgb}{0.271305,0.019942,0.347269}%
\pgfsetfillcolor{currentfill}%
\pgfsetfillopacity{0.700000}%
\pgfsetlinewidth{0.000000pt}%
\definecolor{currentstroke}{rgb}{0.000000,0.000000,0.000000}%
\pgfsetstrokecolor{currentstroke}%
\pgfsetdash{}{0pt}%
\pgfpathmoveto{\pgfqpoint{3.515332in}{2.662594in}}%
\pgfpathlineto{\pgfqpoint{3.528475in}{2.657769in}}%
\pgfpathlineto{\pgfqpoint{3.541624in}{2.652982in}}%
\pgfpathlineto{\pgfqpoint{3.554777in}{2.648233in}}%
\pgfpathlineto{\pgfqpoint{3.567935in}{2.643522in}}%
\pgfpathlineto{\pgfqpoint{3.560092in}{2.636062in}}%
\pgfpathlineto{\pgfqpoint{3.552242in}{2.628625in}}%
\pgfpathlineto{\pgfqpoint{3.544387in}{2.621211in}}%
\pgfpathlineto{\pgfqpoint{3.536526in}{2.613820in}}%
\pgfpathlineto{\pgfqpoint{3.523355in}{2.618547in}}%
\pgfpathlineto{\pgfqpoint{3.510189in}{2.623313in}}%
\pgfpathlineto{\pgfqpoint{3.497028in}{2.628116in}}%
\pgfpathlineto{\pgfqpoint{3.483871in}{2.632959in}}%
\pgfpathlineto{\pgfqpoint{3.491745in}{2.640328in}}%
\pgfpathlineto{\pgfqpoint{3.499614in}{2.647724in}}%
\pgfpathlineto{\pgfqpoint{3.507476in}{2.655146in}}%
\pgfpathlineto{\pgfqpoint{3.515332in}{2.662594in}}%
\pgfpathclose%
\pgfusepath{fill}%
\end{pgfscope}%
\begin{pgfscope}%
\pgfpathrectangle{\pgfqpoint{1.150000in}{0.150000in}}{\pgfqpoint{5.700000in}{5.700000in}}%
\pgfusepath{clip}%
\pgfsetbuttcap%
\pgfsetroundjoin%
\definecolor{currentfill}{rgb}{0.276022,0.044167,0.370164}%
\pgfsetfillcolor{currentfill}%
\pgfsetfillopacity{0.700000}%
\pgfsetlinewidth{0.000000pt}%
\definecolor{currentstroke}{rgb}{0.000000,0.000000,0.000000}%
\pgfsetstrokecolor{currentstroke}%
\pgfsetdash{}{0pt}%
\pgfpathmoveto{\pgfqpoint{4.585925in}{2.701204in}}%
\pgfpathlineto{\pgfqpoint{4.599296in}{2.697872in}}%
\pgfpathlineto{\pgfqpoint{4.612673in}{2.694568in}}%
\pgfpathlineto{\pgfqpoint{4.626057in}{2.691291in}}%
\pgfpathlineto{\pgfqpoint{4.639447in}{2.688041in}}%
\pgfpathlineto{\pgfqpoint{4.631987in}{2.680488in}}%
\pgfpathlineto{\pgfqpoint{4.624522in}{2.672974in}}%
\pgfpathlineto{\pgfqpoint{4.617053in}{2.665497in}}%
\pgfpathlineto{\pgfqpoint{4.609578in}{2.658052in}}%
\pgfpathlineto{\pgfqpoint{4.596175in}{2.661200in}}%
\pgfpathlineto{\pgfqpoint{4.582778in}{2.664376in}}%
\pgfpathlineto{\pgfqpoint{4.569387in}{2.667579in}}%
\pgfpathlineto{\pgfqpoint{4.556003in}{2.670810in}}%
\pgfpathlineto{\pgfqpoint{4.563491in}{2.678351in}}%
\pgfpathlineto{\pgfqpoint{4.570974in}{2.685928in}}%
\pgfpathlineto{\pgfqpoint{4.578452in}{2.693545in}}%
\pgfpathlineto{\pgfqpoint{4.585925in}{2.701204in}}%
\pgfpathclose%
\pgfusepath{fill}%
\end{pgfscope}%
\begin{pgfscope}%
\pgfpathrectangle{\pgfqpoint{1.150000in}{0.150000in}}{\pgfqpoint{5.700000in}{5.700000in}}%
\pgfusepath{clip}%
\pgfsetbuttcap%
\pgfsetroundjoin%
\definecolor{currentfill}{rgb}{0.271305,0.019942,0.347269}%
\pgfsetfillcolor{currentfill}%
\pgfsetfillopacity{0.700000}%
\pgfsetlinewidth{0.000000pt}%
\definecolor{currentstroke}{rgb}{0.000000,0.000000,0.000000}%
\pgfsetstrokecolor{currentstroke}%
\pgfsetdash{}{0pt}%
\pgfpathmoveto{\pgfqpoint{4.008804in}{2.662586in}}%
\pgfpathlineto{\pgfqpoint{4.022045in}{2.658679in}}%
\pgfpathlineto{\pgfqpoint{4.035292in}{2.654805in}}%
\pgfpathlineto{\pgfqpoint{4.048545in}{2.650962in}}%
\pgfpathlineto{\pgfqpoint{4.061804in}{2.647151in}}%
\pgfpathlineto{\pgfqpoint{4.054137in}{2.639516in}}%
\pgfpathlineto{\pgfqpoint{4.046465in}{2.631895in}}%
\pgfpathlineto{\pgfqpoint{4.038788in}{2.624287in}}%
\pgfpathlineto{\pgfqpoint{4.031104in}{2.616692in}}%
\pgfpathlineto{\pgfqpoint{4.017834in}{2.620466in}}%
\pgfpathlineto{\pgfqpoint{4.004569in}{2.624273in}}%
\pgfpathlineto{\pgfqpoint{3.991309in}{2.628111in}}%
\pgfpathlineto{\pgfqpoint{3.978056in}{2.631982in}}%
\pgfpathlineto{\pgfqpoint{3.985751in}{2.639609in}}%
\pgfpathlineto{\pgfqpoint{3.993441in}{2.647251in}}%
\pgfpathlineto{\pgfqpoint{4.001125in}{2.654909in}}%
\pgfpathlineto{\pgfqpoint{4.008804in}{2.662586in}}%
\pgfpathclose%
\pgfusepath{fill}%
\end{pgfscope}%
\begin{pgfscope}%
\pgfpathrectangle{\pgfqpoint{1.150000in}{0.150000in}}{\pgfqpoint{5.700000in}{5.700000in}}%
\pgfusepath{clip}%
\pgfsetbuttcap%
\pgfsetroundjoin%
\definecolor{currentfill}{rgb}{0.269944,0.014625,0.341379}%
\pgfsetfillcolor{currentfill}%
\pgfsetfillopacity{0.700000}%
\pgfsetlinewidth{0.000000pt}%
\definecolor{currentstroke}{rgb}{0.000000,0.000000,0.000000}%
\pgfsetstrokecolor{currentstroke}%
\pgfsetdash{}{0pt}%
\pgfpathmoveto{\pgfqpoint{3.651881in}{2.655142in}}%
\pgfpathlineto{\pgfqpoint{3.665052in}{2.650620in}}%
\pgfpathlineto{\pgfqpoint{3.678227in}{2.646135in}}%
\pgfpathlineto{\pgfqpoint{3.691408in}{2.641686in}}%
\pgfpathlineto{\pgfqpoint{3.704594in}{2.637273in}}%
\pgfpathlineto{\pgfqpoint{3.696799in}{2.629721in}}%
\pgfpathlineto{\pgfqpoint{3.688998in}{2.622187in}}%
\pgfpathlineto{\pgfqpoint{3.681191in}{2.614670in}}%
\pgfpathlineto{\pgfqpoint{3.673379in}{2.607170in}}%
\pgfpathlineto{\pgfqpoint{3.660181in}{2.611587in}}%
\pgfpathlineto{\pgfqpoint{3.646988in}{2.616039in}}%
\pgfpathlineto{\pgfqpoint{3.633800in}{2.620528in}}%
\pgfpathlineto{\pgfqpoint{3.620617in}{2.625053in}}%
\pgfpathlineto{\pgfqpoint{3.628442in}{2.632544in}}%
\pgfpathlineto{\pgfqpoint{3.636261in}{2.640056in}}%
\pgfpathlineto{\pgfqpoint{3.644074in}{2.647588in}}%
\pgfpathlineto{\pgfqpoint{3.651881in}{2.655142in}}%
\pgfpathclose%
\pgfusepath{fill}%
\end{pgfscope}%
\begin{pgfscope}%
\pgfpathrectangle{\pgfqpoint{1.150000in}{0.150000in}}{\pgfqpoint{5.700000in}{5.700000in}}%
\pgfusepath{clip}%
\pgfsetbuttcap%
\pgfsetroundjoin%
\definecolor{currentfill}{rgb}{0.273809,0.031497,0.358853}%
\pgfsetfillcolor{currentfill}%
\pgfsetfillopacity{0.700000}%
\pgfsetlinewidth{0.000000pt}%
\definecolor{currentstroke}{rgb}{0.000000,0.000000,0.000000}%
\pgfsetstrokecolor{currentstroke}%
\pgfsetdash{}{0pt}%
\pgfpathmoveto{\pgfqpoint{4.365724in}{2.680846in}}%
\pgfpathlineto{\pgfqpoint{4.379046in}{2.677363in}}%
\pgfpathlineto{\pgfqpoint{4.392374in}{2.673910in}}%
\pgfpathlineto{\pgfqpoint{4.405709in}{2.670485in}}%
\pgfpathlineto{\pgfqpoint{4.419049in}{2.667089in}}%
\pgfpathlineto{\pgfqpoint{4.411510in}{2.659520in}}%
\pgfpathlineto{\pgfqpoint{4.403965in}{2.651976in}}%
\pgfpathlineto{\pgfqpoint{4.396415in}{2.644455in}}%
\pgfpathlineto{\pgfqpoint{4.388859in}{2.636952in}}%
\pgfpathlineto{\pgfqpoint{4.375506in}{2.640273in}}%
\pgfpathlineto{\pgfqpoint{4.362158in}{2.643622in}}%
\pgfpathlineto{\pgfqpoint{4.348817in}{2.647001in}}%
\pgfpathlineto{\pgfqpoint{4.335482in}{2.650408in}}%
\pgfpathlineto{\pgfqpoint{4.343050in}{2.657981in}}%
\pgfpathlineto{\pgfqpoint{4.350613in}{2.665577in}}%
\pgfpathlineto{\pgfqpoint{4.358171in}{2.673198in}}%
\pgfpathlineto{\pgfqpoint{4.365724in}{2.680846in}}%
\pgfpathclose%
\pgfusepath{fill}%
\end{pgfscope}%
\begin{pgfscope}%
\pgfpathrectangle{\pgfqpoint{1.150000in}{0.150000in}}{\pgfqpoint{5.700000in}{5.700000in}}%
\pgfusepath{clip}%
\pgfsetbuttcap%
\pgfsetroundjoin%
\definecolor{currentfill}{rgb}{0.280894,0.078907,0.402329}%
\pgfsetfillcolor{currentfill}%
\pgfsetfillopacity{0.700000}%
\pgfsetlinewidth{0.000000pt}%
\definecolor{currentstroke}{rgb}{0.000000,0.000000,0.000000}%
\pgfsetstrokecolor{currentstroke}%
\pgfsetdash{}{0pt}%
\pgfpathmoveto{\pgfqpoint{5.163289in}{2.753591in}}%
\pgfpathlineto{\pgfqpoint{5.176796in}{2.750399in}}%
\pgfpathlineto{\pgfqpoint{5.190310in}{2.747231in}}%
\pgfpathlineto{\pgfqpoint{5.203831in}{2.744089in}}%
\pgfpathlineto{\pgfqpoint{5.217358in}{2.740971in}}%
\pgfpathlineto{\pgfqpoint{5.210097in}{2.733141in}}%
\pgfpathlineto{\pgfqpoint{5.202834in}{2.725412in}}%
\pgfpathlineto{\pgfqpoint{5.195568in}{2.717779in}}%
\pgfpathlineto{\pgfqpoint{5.188298in}{2.710237in}}%
\pgfpathlineto{\pgfqpoint{5.174754in}{2.713188in}}%
\pgfpathlineto{\pgfqpoint{5.161218in}{2.716164in}}%
\pgfpathlineto{\pgfqpoint{5.147688in}{2.719165in}}%
\pgfpathlineto{\pgfqpoint{5.134165in}{2.722191in}}%
\pgfpathlineto{\pgfqpoint{5.141451in}{2.729895in}}%
\pgfpathlineto{\pgfqpoint{5.148733in}{2.737693in}}%
\pgfpathlineto{\pgfqpoint{5.156013in}{2.745590in}}%
\pgfpathlineto{\pgfqpoint{5.163289in}{2.753591in}}%
\pgfpathclose%
\pgfusepath{fill}%
\end{pgfscope}%
\begin{pgfscope}%
\pgfpathrectangle{\pgfqpoint{1.150000in}{0.150000in}}{\pgfqpoint{5.700000in}{5.700000in}}%
\pgfusepath{clip}%
\pgfsetbuttcap%
\pgfsetroundjoin%
\definecolor{currentfill}{rgb}{0.279566,0.067836,0.391917}%
\pgfsetfillcolor{currentfill}%
\pgfsetfillopacity{0.700000}%
\pgfsetlinewidth{0.000000pt}%
\definecolor{currentstroke}{rgb}{0.000000,0.000000,0.000000}%
\pgfsetstrokecolor{currentstroke}%
\pgfsetdash{}{0pt}%
\pgfpathmoveto{\pgfqpoint{4.943051in}{2.728670in}}%
\pgfpathlineto{\pgfqpoint{4.956509in}{2.725491in}}%
\pgfpathlineto{\pgfqpoint{4.969974in}{2.722337in}}%
\pgfpathlineto{\pgfqpoint{4.983445in}{2.719210in}}%
\pgfpathlineto{\pgfqpoint{4.996923in}{2.716108in}}%
\pgfpathlineto{\pgfqpoint{4.989588in}{2.708509in}}%
\pgfpathlineto{\pgfqpoint{4.982249in}{2.700981in}}%
\pgfpathlineto{\pgfqpoint{4.974907in}{2.693519in}}%
\pgfpathlineto{\pgfqpoint{4.967560in}{2.686121in}}%
\pgfpathlineto{\pgfqpoint{4.954067in}{2.689082in}}%
\pgfpathlineto{\pgfqpoint{4.940581in}{2.692069in}}%
\pgfpathlineto{\pgfqpoint{4.927101in}{2.695081in}}%
\pgfpathlineto{\pgfqpoint{4.913629in}{2.698120in}}%
\pgfpathlineto{\pgfqpoint{4.920990in}{2.705655in}}%
\pgfpathlineto{\pgfqpoint{4.928348in}{2.713255in}}%
\pgfpathlineto{\pgfqpoint{4.935702in}{2.720925in}}%
\pgfpathlineto{\pgfqpoint{4.943051in}{2.728670in}}%
\pgfpathclose%
\pgfusepath{fill}%
\end{pgfscope}%
\begin{pgfscope}%
\pgfpathrectangle{\pgfqpoint{1.150000in}{0.150000in}}{\pgfqpoint{5.700000in}{5.700000in}}%
\pgfusepath{clip}%
\pgfsetbuttcap%
\pgfsetroundjoin%
\definecolor{currentfill}{rgb}{0.269944,0.014625,0.341379}%
\pgfsetfillcolor{currentfill}%
\pgfsetfillopacity{0.700000}%
\pgfsetlinewidth{0.000000pt}%
\definecolor{currentstroke}{rgb}{0.000000,0.000000,0.000000}%
\pgfsetstrokecolor{currentstroke}%
\pgfsetdash{}{0pt}%
\pgfpathmoveto{\pgfqpoint{3.788463in}{2.650336in}}%
\pgfpathlineto{\pgfqpoint{3.801663in}{2.646089in}}%
\pgfpathlineto{\pgfqpoint{3.814868in}{2.641876in}}%
\pgfpathlineto{\pgfqpoint{3.828078in}{2.637697in}}%
\pgfpathlineto{\pgfqpoint{3.841294in}{2.633553in}}%
\pgfpathlineto{\pgfqpoint{3.833546in}{2.625951in}}%
\pgfpathlineto{\pgfqpoint{3.825793in}{2.618364in}}%
\pgfpathlineto{\pgfqpoint{3.818034in}{2.610791in}}%
\pgfpathlineto{\pgfqpoint{3.810270in}{2.603230in}}%
\pgfpathlineto{\pgfqpoint{3.797041in}{2.607365in}}%
\pgfpathlineto{\pgfqpoint{3.783819in}{2.611533in}}%
\pgfpathlineto{\pgfqpoint{3.770601in}{2.615736in}}%
\pgfpathlineto{\pgfqpoint{3.757389in}{2.619974in}}%
\pgfpathlineto{\pgfqpoint{3.765166in}{2.627540in}}%
\pgfpathlineto{\pgfqpoint{3.772937in}{2.635121in}}%
\pgfpathlineto{\pgfqpoint{3.780703in}{2.642720in}}%
\pgfpathlineto{\pgfqpoint{3.788463in}{2.650336in}}%
\pgfpathclose%
\pgfusepath{fill}%
\end{pgfscope}%
\begin{pgfscope}%
\pgfpathrectangle{\pgfqpoint{1.150000in}{0.150000in}}{\pgfqpoint{5.700000in}{5.700000in}}%
\pgfusepath{clip}%
\pgfsetbuttcap%
\pgfsetroundjoin%
\definecolor{currentfill}{rgb}{0.272594,0.025563,0.353093}%
\pgfsetfillcolor{currentfill}%
\pgfsetfillopacity{0.700000}%
\pgfsetlinewidth{0.000000pt}%
\definecolor{currentstroke}{rgb}{0.000000,0.000000,0.000000}%
\pgfsetstrokecolor{currentstroke}%
\pgfsetdash{}{0pt}%
\pgfpathmoveto{\pgfqpoint{4.145458in}{2.662749in}}%
\pgfpathlineto{\pgfqpoint{4.158733in}{2.659045in}}%
\pgfpathlineto{\pgfqpoint{4.172015in}{2.655371in}}%
\pgfpathlineto{\pgfqpoint{4.185302in}{2.651728in}}%
\pgfpathlineto{\pgfqpoint{4.198595in}{2.648116in}}%
\pgfpathlineto{\pgfqpoint{4.190975in}{2.640512in}}%
\pgfpathlineto{\pgfqpoint{4.183349in}{2.632923in}}%
\pgfpathlineto{\pgfqpoint{4.175718in}{2.625348in}}%
\pgfpathlineto{\pgfqpoint{4.168082in}{2.617785in}}%
\pgfpathlineto{\pgfqpoint{4.154776in}{2.621348in}}%
\pgfpathlineto{\pgfqpoint{4.141477in}{2.624942in}}%
\pgfpathlineto{\pgfqpoint{4.128183in}{2.628566in}}%
\pgfpathlineto{\pgfqpoint{4.114896in}{2.632221in}}%
\pgfpathlineto{\pgfqpoint{4.122544in}{2.639829in}}%
\pgfpathlineto{\pgfqpoint{4.130188in}{2.647451in}}%
\pgfpathlineto{\pgfqpoint{4.137826in}{2.655091in}}%
\pgfpathlineto{\pgfqpoint{4.145458in}{2.662749in}}%
\pgfpathclose%
\pgfusepath{fill}%
\end{pgfscope}%
\begin{pgfscope}%
\pgfpathrectangle{\pgfqpoint{1.150000in}{0.150000in}}{\pgfqpoint{5.700000in}{5.700000in}}%
\pgfusepath{clip}%
\pgfsetbuttcap%
\pgfsetroundjoin%
\definecolor{currentfill}{rgb}{0.277018,0.050344,0.375715}%
\pgfsetfillcolor{currentfill}%
\pgfsetfillopacity{0.700000}%
\pgfsetlinewidth{0.000000pt}%
\definecolor{currentstroke}{rgb}{0.000000,0.000000,0.000000}%
\pgfsetstrokecolor{currentstroke}%
\pgfsetdash{}{0pt}%
\pgfpathmoveto{\pgfqpoint{4.722806in}{2.705536in}}%
\pgfpathlineto{\pgfqpoint{4.736214in}{2.702308in}}%
\pgfpathlineto{\pgfqpoint{4.749629in}{2.699107in}}%
\pgfpathlineto{\pgfqpoint{4.763050in}{2.695932in}}%
\pgfpathlineto{\pgfqpoint{4.776478in}{2.692784in}}%
\pgfpathlineto{\pgfqpoint{4.769065in}{2.685276in}}%
\pgfpathlineto{\pgfqpoint{4.761648in}{2.677815in}}%
\pgfpathlineto{\pgfqpoint{4.754226in}{2.670398in}}%
\pgfpathlineto{\pgfqpoint{4.746798in}{2.663022in}}%
\pgfpathlineto{\pgfqpoint{4.733357in}{2.666055in}}%
\pgfpathlineto{\pgfqpoint{4.719921in}{2.669115in}}%
\pgfpathlineto{\pgfqpoint{4.706493in}{2.672202in}}%
\pgfpathlineto{\pgfqpoint{4.693070in}{2.675315in}}%
\pgfpathlineto{\pgfqpoint{4.700511in}{2.682802in}}%
\pgfpathlineto{\pgfqpoint{4.707948in}{2.690332in}}%
\pgfpathlineto{\pgfqpoint{4.715379in}{2.697909in}}%
\pgfpathlineto{\pgfqpoint{4.722806in}{2.705536in}}%
\pgfpathclose%
\pgfusepath{fill}%
\end{pgfscope}%
\begin{pgfscope}%
\pgfpathrectangle{\pgfqpoint{1.150000in}{0.150000in}}{\pgfqpoint{5.700000in}{5.700000in}}%
\pgfusepath{clip}%
\pgfsetbuttcap%
\pgfsetroundjoin%
\definecolor{currentfill}{rgb}{0.274952,0.037752,0.364543}%
\pgfsetfillcolor{currentfill}%
\pgfsetfillopacity{0.700000}%
\pgfsetlinewidth{0.000000pt}%
\definecolor{currentstroke}{rgb}{0.000000,0.000000,0.000000}%
\pgfsetstrokecolor{currentstroke}%
\pgfsetdash{}{0pt}%
\pgfpathmoveto{\pgfqpoint{3.157933in}{2.681875in}}%
\pgfpathlineto{\pgfqpoint{3.171029in}{2.676117in}}%
\pgfpathlineto{\pgfqpoint{3.184130in}{2.670404in}}%
\pgfpathlineto{\pgfqpoint{3.197234in}{2.664736in}}%
\pgfpathlineto{\pgfqpoint{3.210343in}{2.659114in}}%
\pgfpathlineto{\pgfqpoint{3.202357in}{2.652195in}}%
\pgfpathlineto{\pgfqpoint{3.194365in}{2.645320in}}%
\pgfpathlineto{\pgfqpoint{3.186366in}{2.638489in}}%
\pgfpathlineto{\pgfqpoint{3.178361in}{2.631704in}}%
\pgfpathlineto{\pgfqpoint{3.165238in}{2.637382in}}%
\pgfpathlineto{\pgfqpoint{3.152119in}{2.643106in}}%
\pgfpathlineto{\pgfqpoint{3.139004in}{2.648875in}}%
\pgfpathlineto{\pgfqpoint{3.125893in}{2.654690in}}%
\pgfpathlineto{\pgfqpoint{3.133913in}{2.661414in}}%
\pgfpathlineto{\pgfqpoint{3.141927in}{2.668187in}}%
\pgfpathlineto{\pgfqpoint{3.149933in}{2.675008in}}%
\pgfpathlineto{\pgfqpoint{3.157933in}{2.681875in}}%
\pgfpathclose%
\pgfusepath{fill}%
\end{pgfscope}%
\begin{pgfscope}%
\pgfpathrectangle{\pgfqpoint{1.150000in}{0.150000in}}{\pgfqpoint{5.700000in}{5.700000in}}%
\pgfusepath{clip}%
\pgfsetbuttcap%
\pgfsetroundjoin%
\definecolor{currentfill}{rgb}{0.272594,0.025563,0.353093}%
\pgfsetfillcolor{currentfill}%
\pgfsetfillopacity{0.700000}%
\pgfsetlinewidth{0.000000pt}%
\definecolor{currentstroke}{rgb}{0.000000,0.000000,0.000000}%
\pgfsetstrokecolor{currentstroke}%
\pgfsetdash{}{0pt}%
\pgfpathmoveto{\pgfqpoint{3.294637in}{2.665333in}}%
\pgfpathlineto{\pgfqpoint{3.307752in}{2.659973in}}%
\pgfpathlineto{\pgfqpoint{3.320872in}{2.654655in}}%
\pgfpathlineto{\pgfqpoint{3.333996in}{2.649379in}}%
\pgfpathlineto{\pgfqpoint{3.347124in}{2.644145in}}%
\pgfpathlineto{\pgfqpoint{3.339193in}{2.636984in}}%
\pgfpathlineto{\pgfqpoint{3.331255in}{2.629856in}}%
\pgfpathlineto{\pgfqpoint{3.323310in}{2.622764in}}%
\pgfpathlineto{\pgfqpoint{3.315360in}{2.615707in}}%
\pgfpathlineto{\pgfqpoint{3.302218in}{2.620983in}}%
\pgfpathlineto{\pgfqpoint{3.289080in}{2.626302in}}%
\pgfpathlineto{\pgfqpoint{3.275947in}{2.631663in}}%
\pgfpathlineto{\pgfqpoint{3.262817in}{2.637066in}}%
\pgfpathlineto{\pgfqpoint{3.270782in}{2.644075in}}%
\pgfpathlineto{\pgfqpoint{3.278740in}{2.651123in}}%
\pgfpathlineto{\pgfqpoint{3.286692in}{2.658210in}}%
\pgfpathlineto{\pgfqpoint{3.294637in}{2.665333in}}%
\pgfpathclose%
\pgfusepath{fill}%
\end{pgfscope}%
\begin{pgfscope}%
\pgfpathrectangle{\pgfqpoint{1.150000in}{0.150000in}}{\pgfqpoint{5.700000in}{5.700000in}}%
\pgfusepath{clip}%
\pgfsetbuttcap%
\pgfsetroundjoin%
\definecolor{currentfill}{rgb}{0.274952,0.037752,0.364543}%
\pgfsetfillcolor{currentfill}%
\pgfsetfillopacity{0.700000}%
\pgfsetlinewidth{0.000000pt}%
\definecolor{currentstroke}{rgb}{0.000000,0.000000,0.000000}%
\pgfsetstrokecolor{currentstroke}%
\pgfsetdash{}{0pt}%
\pgfpathmoveto{\pgfqpoint{4.502530in}{2.684011in}}%
\pgfpathlineto{\pgfqpoint{4.515889in}{2.680669in}}%
\pgfpathlineto{\pgfqpoint{4.529254in}{2.677354in}}%
\pgfpathlineto{\pgfqpoint{4.542625in}{2.674068in}}%
\pgfpathlineto{\pgfqpoint{4.556003in}{2.670810in}}%
\pgfpathlineto{\pgfqpoint{4.548510in}{2.663301in}}%
\pgfpathlineto{\pgfqpoint{4.541012in}{2.655822in}}%
\pgfpathlineto{\pgfqpoint{4.533509in}{2.648370in}}%
\pgfpathlineto{\pgfqpoint{4.526001in}{2.640942in}}%
\pgfpathlineto{\pgfqpoint{4.512610in}{2.644112in}}%
\pgfpathlineto{\pgfqpoint{4.499225in}{2.647309in}}%
\pgfpathlineto{\pgfqpoint{4.485847in}{2.650535in}}%
\pgfpathlineto{\pgfqpoint{4.472475in}{2.653789in}}%
\pgfpathlineto{\pgfqpoint{4.479996in}{2.661301in}}%
\pgfpathlineto{\pgfqpoint{4.487513in}{2.668841in}}%
\pgfpathlineto{\pgfqpoint{4.495024in}{2.676409in}}%
\pgfpathlineto{\pgfqpoint{4.502530in}{2.684011in}}%
\pgfpathclose%
\pgfusepath{fill}%
\end{pgfscope}%
\begin{pgfscope}%
\pgfpathrectangle{\pgfqpoint{1.150000in}{0.150000in}}{\pgfqpoint{5.700000in}{5.700000in}}%
\pgfusepath{clip}%
\pgfsetbuttcap%
\pgfsetroundjoin%
\definecolor{currentfill}{rgb}{0.271305,0.019942,0.347269}%
\pgfsetfillcolor{currentfill}%
\pgfsetfillopacity{0.700000}%
\pgfsetlinewidth{0.000000pt}%
\definecolor{currentstroke}{rgb}{0.000000,0.000000,0.000000}%
\pgfsetstrokecolor{currentstroke}%
\pgfsetdash{}{0pt}%
\pgfpathmoveto{\pgfqpoint{3.431293in}{2.652720in}}%
\pgfpathlineto{\pgfqpoint{3.444430in}{2.647720in}}%
\pgfpathlineto{\pgfqpoint{3.457573in}{2.642760in}}%
\pgfpathlineto{\pgfqpoint{3.470720in}{2.637840in}}%
\pgfpathlineto{\pgfqpoint{3.483871in}{2.632959in}}%
\pgfpathlineto{\pgfqpoint{3.475991in}{2.625615in}}%
\pgfpathlineto{\pgfqpoint{3.468105in}{2.618297in}}%
\pgfpathlineto{\pgfqpoint{3.460212in}{2.611006in}}%
\pgfpathlineto{\pgfqpoint{3.452314in}{2.603742in}}%
\pgfpathlineto{\pgfqpoint{3.439149in}{2.608653in}}%
\pgfpathlineto{\pgfqpoint{3.425989in}{2.613603in}}%
\pgfpathlineto{\pgfqpoint{3.412833in}{2.618593in}}%
\pgfpathlineto{\pgfqpoint{3.399682in}{2.623622in}}%
\pgfpathlineto{\pgfqpoint{3.407594in}{2.630852in}}%
\pgfpathlineto{\pgfqpoint{3.415500in}{2.638112in}}%
\pgfpathlineto{\pgfqpoint{3.423399in}{2.645401in}}%
\pgfpathlineto{\pgfqpoint{3.431293in}{2.652720in}}%
\pgfpathclose%
\pgfusepath{fill}%
\end{pgfscope}%
\begin{pgfscope}%
\pgfpathrectangle{\pgfqpoint{1.150000in}{0.150000in}}{\pgfqpoint{5.700000in}{5.700000in}}%
\pgfusepath{clip}%
\pgfsetbuttcap%
\pgfsetroundjoin%
\definecolor{currentfill}{rgb}{0.277018,0.050344,0.375715}%
\pgfsetfillcolor{currentfill}%
\pgfsetfillopacity{0.700000}%
\pgfsetlinewidth{0.000000pt}%
\definecolor{currentstroke}{rgb}{0.000000,0.000000,0.000000}%
\pgfsetstrokecolor{currentstroke}%
\pgfsetdash{}{0pt}%
\pgfpathmoveto{\pgfqpoint{3.021138in}{2.702913in}}%
\pgfpathlineto{\pgfqpoint{3.034220in}{2.696715in}}%
\pgfpathlineto{\pgfqpoint{3.047305in}{2.690567in}}%
\pgfpathlineto{\pgfqpoint{3.060394in}{2.684467in}}%
\pgfpathlineto{\pgfqpoint{3.073486in}{2.678416in}}%
\pgfpathlineto{\pgfqpoint{3.065444in}{2.671807in}}%
\pgfpathlineto{\pgfqpoint{3.057394in}{2.665251in}}%
\pgfpathlineto{\pgfqpoint{3.049338in}{2.658752in}}%
\pgfpathlineto{\pgfqpoint{3.041274in}{2.652309in}}%
\pgfpathlineto{\pgfqpoint{3.028166in}{2.658429in}}%
\pgfpathlineto{\pgfqpoint{3.015062in}{2.664598in}}%
\pgfpathlineto{\pgfqpoint{3.001961in}{2.670816in}}%
\pgfpathlineto{\pgfqpoint{2.988864in}{2.677084in}}%
\pgfpathlineto{\pgfqpoint{2.996944in}{2.683452in}}%
\pgfpathlineto{\pgfqpoint{3.005016in}{2.689880in}}%
\pgfpathlineto{\pgfqpoint{3.013081in}{2.696368in}}%
\pgfpathlineto{\pgfqpoint{3.021138in}{2.702913in}}%
\pgfpathclose%
\pgfusepath{fill}%
\end{pgfscope}%
\begin{pgfscope}%
\pgfpathrectangle{\pgfqpoint{1.150000in}{0.150000in}}{\pgfqpoint{5.700000in}{5.700000in}}%
\pgfusepath{clip}%
\pgfsetbuttcap%
\pgfsetroundjoin%
\definecolor{currentfill}{rgb}{0.271305,0.019942,0.347269}%
\pgfsetfillcolor{currentfill}%
\pgfsetfillopacity{0.700000}%
\pgfsetlinewidth{0.000000pt}%
\definecolor{currentstroke}{rgb}{0.000000,0.000000,0.000000}%
\pgfsetstrokecolor{currentstroke}%
\pgfsetdash{}{0pt}%
\pgfpathmoveto{\pgfqpoint{3.925098in}{2.647788in}}%
\pgfpathlineto{\pgfqpoint{3.938329in}{2.643788in}}%
\pgfpathlineto{\pgfqpoint{3.951566in}{2.639820in}}%
\pgfpathlineto{\pgfqpoint{3.964808in}{2.635885in}}%
\pgfpathlineto{\pgfqpoint{3.978056in}{2.631982in}}%
\pgfpathlineto{\pgfqpoint{3.970355in}{2.624368in}}%
\pgfpathlineto{\pgfqpoint{3.962649in}{2.616767in}}%
\pgfpathlineto{\pgfqpoint{3.954937in}{2.609177in}}%
\pgfpathlineto{\pgfqpoint{3.947219in}{2.601596in}}%
\pgfpathlineto{\pgfqpoint{3.933959in}{2.605476in}}%
\pgfpathlineto{\pgfqpoint{3.920704in}{2.609388in}}%
\pgfpathlineto{\pgfqpoint{3.907455in}{2.613333in}}%
\pgfpathlineto{\pgfqpoint{3.894212in}{2.617310in}}%
\pgfpathlineto{\pgfqpoint{3.901942in}{2.624909in}}%
\pgfpathlineto{\pgfqpoint{3.909666in}{2.632521in}}%
\pgfpathlineto{\pgfqpoint{3.917385in}{2.640147in}}%
\pgfpathlineto{\pgfqpoint{3.925098in}{2.647788in}}%
\pgfpathclose%
\pgfusepath{fill}%
\end{pgfscope}%
\begin{pgfscope}%
\pgfpathrectangle{\pgfqpoint{1.150000in}{0.150000in}}{\pgfqpoint{5.700000in}{5.700000in}}%
\pgfusepath{clip}%
\pgfsetbuttcap%
\pgfsetroundjoin%
\definecolor{currentfill}{rgb}{0.273809,0.031497,0.358853}%
\pgfsetfillcolor{currentfill}%
\pgfsetfillopacity{0.700000}%
\pgfsetlinewidth{0.000000pt}%
\definecolor{currentstroke}{rgb}{0.000000,0.000000,0.000000}%
\pgfsetstrokecolor{currentstroke}%
\pgfsetdash{}{0pt}%
\pgfpathmoveto{\pgfqpoint{4.282204in}{2.664331in}}%
\pgfpathlineto{\pgfqpoint{4.295514in}{2.660806in}}%
\pgfpathlineto{\pgfqpoint{4.308831in}{2.657311in}}%
\pgfpathlineto{\pgfqpoint{4.322153in}{2.653845in}}%
\pgfpathlineto{\pgfqpoint{4.335482in}{2.650408in}}%
\pgfpathlineto{\pgfqpoint{4.327909in}{2.642855in}}%
\pgfpathlineto{\pgfqpoint{4.320330in}{2.635320in}}%
\pgfpathlineto{\pgfqpoint{4.312745in}{2.627800in}}%
\pgfpathlineto{\pgfqpoint{4.305156in}{2.620294in}}%
\pgfpathlineto{\pgfqpoint{4.291814in}{2.623668in}}%
\pgfpathlineto{\pgfqpoint{4.278479in}{2.627072in}}%
\pgfpathlineto{\pgfqpoint{4.265150in}{2.630505in}}%
\pgfpathlineto{\pgfqpoint{4.251827in}{2.633967in}}%
\pgfpathlineto{\pgfqpoint{4.259429in}{2.641531in}}%
\pgfpathlineto{\pgfqpoint{4.267026in}{2.649112in}}%
\pgfpathlineto{\pgfqpoint{4.274617in}{2.656711in}}%
\pgfpathlineto{\pgfqpoint{4.282204in}{2.664331in}}%
\pgfpathclose%
\pgfusepath{fill}%
\end{pgfscope}%
\begin{pgfscope}%
\pgfpathrectangle{\pgfqpoint{1.150000in}{0.150000in}}{\pgfqpoint{5.700000in}{5.700000in}}%
\pgfusepath{clip}%
\pgfsetbuttcap%
\pgfsetroundjoin%
\definecolor{currentfill}{rgb}{0.269944,0.014625,0.341379}%
\pgfsetfillcolor{currentfill}%
\pgfsetfillopacity{0.700000}%
\pgfsetlinewidth{0.000000pt}%
\definecolor{currentstroke}{rgb}{0.000000,0.000000,0.000000}%
\pgfsetstrokecolor{currentstroke}%
\pgfsetdash{}{0pt}%
\pgfpathmoveto{\pgfqpoint{3.567935in}{2.643522in}}%
\pgfpathlineto{\pgfqpoint{3.581098in}{2.638849in}}%
\pgfpathlineto{\pgfqpoint{3.594266in}{2.634213in}}%
\pgfpathlineto{\pgfqpoint{3.607439in}{2.629614in}}%
\pgfpathlineto{\pgfqpoint{3.620617in}{2.625053in}}%
\pgfpathlineto{\pgfqpoint{3.612786in}{2.617581in}}%
\pgfpathlineto{\pgfqpoint{3.604949in}{2.610130in}}%
\pgfpathlineto{\pgfqpoint{3.597107in}{2.602698in}}%
\pgfpathlineto{\pgfqpoint{3.589258in}{2.595285in}}%
\pgfpathlineto{\pgfqpoint{3.576068in}{2.599863in}}%
\pgfpathlineto{\pgfqpoint{3.562882in}{2.604478in}}%
\pgfpathlineto{\pgfqpoint{3.549701in}{2.609130in}}%
\pgfpathlineto{\pgfqpoint{3.536526in}{2.613820in}}%
\pgfpathlineto{\pgfqpoint{3.544387in}{2.621211in}}%
\pgfpathlineto{\pgfqpoint{3.552242in}{2.628625in}}%
\pgfpathlineto{\pgfqpoint{3.560092in}{2.636062in}}%
\pgfpathlineto{\pgfqpoint{3.567935in}{2.643522in}}%
\pgfpathclose%
\pgfusepath{fill}%
\end{pgfscope}%
\begin{pgfscope}%
\pgfpathrectangle{\pgfqpoint{1.150000in}{0.150000in}}{\pgfqpoint{5.700000in}{5.700000in}}%
\pgfusepath{clip}%
\pgfsetbuttcap%
\pgfsetroundjoin%
\definecolor{currentfill}{rgb}{0.280267,0.073417,0.397163}%
\pgfsetfillcolor{currentfill}%
\pgfsetfillopacity{0.700000}%
\pgfsetlinewidth{0.000000pt}%
\definecolor{currentstroke}{rgb}{0.000000,0.000000,0.000000}%
\pgfsetstrokecolor{currentstroke}%
\pgfsetdash{}{0pt}%
\pgfpathmoveto{\pgfqpoint{5.080141in}{2.734549in}}%
\pgfpathlineto{\pgfqpoint{5.093637in}{2.731421in}}%
\pgfpathlineto{\pgfqpoint{5.107140in}{2.728319in}}%
\pgfpathlineto{\pgfqpoint{5.120649in}{2.725243in}}%
\pgfpathlineto{\pgfqpoint{5.134165in}{2.722191in}}%
\pgfpathlineto{\pgfqpoint{5.126876in}{2.714576in}}%
\pgfpathlineto{\pgfqpoint{5.119584in}{2.707045in}}%
\pgfpathlineto{\pgfqpoint{5.112288in}{2.699593in}}%
\pgfpathlineto{\pgfqpoint{5.104988in}{2.692215in}}%
\pgfpathlineto{\pgfqpoint{5.091456in}{2.695113in}}%
\pgfpathlineto{\pgfqpoint{5.077931in}{2.698036in}}%
\pgfpathlineto{\pgfqpoint{5.064412in}{2.700984in}}%
\pgfpathlineto{\pgfqpoint{5.050901in}{2.703958in}}%
\pgfpathlineto{\pgfqpoint{5.058216in}{2.711485in}}%
\pgfpathlineto{\pgfqpoint{5.065528in}{2.719089in}}%
\pgfpathlineto{\pgfqpoint{5.072836in}{2.726775in}}%
\pgfpathlineto{\pgfqpoint{5.080141in}{2.734549in}}%
\pgfpathclose%
\pgfusepath{fill}%
\end{pgfscope}%
\begin{pgfscope}%
\pgfpathrectangle{\pgfqpoint{1.150000in}{0.150000in}}{\pgfqpoint{5.700000in}{5.700000in}}%
\pgfusepath{clip}%
\pgfsetbuttcap%
\pgfsetroundjoin%
\definecolor{currentfill}{rgb}{0.278791,0.062145,0.386592}%
\pgfsetfillcolor{currentfill}%
\pgfsetfillopacity{0.700000}%
\pgfsetlinewidth{0.000000pt}%
\definecolor{currentstroke}{rgb}{0.000000,0.000000,0.000000}%
\pgfsetstrokecolor{currentstroke}%
\pgfsetdash{}{0pt}%
\pgfpathmoveto{\pgfqpoint{4.859804in}{2.710535in}}%
\pgfpathlineto{\pgfqpoint{4.873250in}{2.707392in}}%
\pgfpathlineto{\pgfqpoint{4.886703in}{2.704275in}}%
\pgfpathlineto{\pgfqpoint{4.900163in}{2.701185in}}%
\pgfpathlineto{\pgfqpoint{4.913629in}{2.698120in}}%
\pgfpathlineto{\pgfqpoint{4.906263in}{2.690647in}}%
\pgfpathlineto{\pgfqpoint{4.898892in}{2.683231in}}%
\pgfpathlineto{\pgfqpoint{4.891517in}{2.675868in}}%
\pgfpathlineto{\pgfqpoint{4.884138in}{2.668554in}}%
\pgfpathlineto{\pgfqpoint{4.870657in}{2.671491in}}%
\pgfpathlineto{\pgfqpoint{4.857183in}{2.674454in}}%
\pgfpathlineto{\pgfqpoint{4.843716in}{2.677443in}}%
\pgfpathlineto{\pgfqpoint{4.830255in}{2.680458in}}%
\pgfpathlineto{\pgfqpoint{4.837649in}{2.687895in}}%
\pgfpathlineto{\pgfqpoint{4.845038in}{2.695384in}}%
\pgfpathlineto{\pgfqpoint{4.852423in}{2.702930in}}%
\pgfpathlineto{\pgfqpoint{4.859804in}{2.710535in}}%
\pgfpathclose%
\pgfusepath{fill}%
\end{pgfscope}%
\begin{pgfscope}%
\pgfpathrectangle{\pgfqpoint{1.150000in}{0.150000in}}{\pgfqpoint{5.700000in}{5.700000in}}%
\pgfusepath{clip}%
\pgfsetbuttcap%
\pgfsetroundjoin%
\definecolor{currentfill}{rgb}{0.271305,0.019942,0.347269}%
\pgfsetfillcolor{currentfill}%
\pgfsetfillopacity{0.700000}%
\pgfsetlinewidth{0.000000pt}%
\definecolor{currentstroke}{rgb}{0.000000,0.000000,0.000000}%
\pgfsetstrokecolor{currentstroke}%
\pgfsetdash{}{0pt}%
\pgfpathmoveto{\pgfqpoint{4.061804in}{2.647151in}}%
\pgfpathlineto{\pgfqpoint{4.075068in}{2.643372in}}%
\pgfpathlineto{\pgfqpoint{4.088338in}{2.639624in}}%
\pgfpathlineto{\pgfqpoint{4.101614in}{2.635907in}}%
\pgfpathlineto{\pgfqpoint{4.114896in}{2.632221in}}%
\pgfpathlineto{\pgfqpoint{4.107241in}{2.624627in}}%
\pgfpathlineto{\pgfqpoint{4.099582in}{2.617044in}}%
\pgfpathlineto{\pgfqpoint{4.091916in}{2.609471in}}%
\pgfpathlineto{\pgfqpoint{4.084246in}{2.601906in}}%
\pgfpathlineto{\pgfqpoint{4.070952in}{2.605556in}}%
\pgfpathlineto{\pgfqpoint{4.057663in}{2.609236in}}%
\pgfpathlineto{\pgfqpoint{4.044381in}{2.612948in}}%
\pgfpathlineto{\pgfqpoint{4.031104in}{2.616692in}}%
\pgfpathlineto{\pgfqpoint{4.038788in}{2.624287in}}%
\pgfpathlineto{\pgfqpoint{4.046465in}{2.631895in}}%
\pgfpathlineto{\pgfqpoint{4.054137in}{2.639516in}}%
\pgfpathlineto{\pgfqpoint{4.061804in}{2.647151in}}%
\pgfpathclose%
\pgfusepath{fill}%
\end{pgfscope}%
\begin{pgfscope}%
\pgfpathrectangle{\pgfqpoint{1.150000in}{0.150000in}}{\pgfqpoint{5.700000in}{5.700000in}}%
\pgfusepath{clip}%
\pgfsetbuttcap%
\pgfsetroundjoin%
\definecolor{currentfill}{rgb}{0.269944,0.014625,0.341379}%
\pgfsetfillcolor{currentfill}%
\pgfsetfillopacity{0.700000}%
\pgfsetlinewidth{0.000000pt}%
\definecolor{currentstroke}{rgb}{0.000000,0.000000,0.000000}%
\pgfsetstrokecolor{currentstroke}%
\pgfsetdash{}{0pt}%
\pgfpathmoveto{\pgfqpoint{3.704594in}{2.637273in}}%
\pgfpathlineto{\pgfqpoint{3.717785in}{2.632895in}}%
\pgfpathlineto{\pgfqpoint{3.730981in}{2.628553in}}%
\pgfpathlineto{\pgfqpoint{3.744183in}{2.624246in}}%
\pgfpathlineto{\pgfqpoint{3.757389in}{2.619974in}}%
\pgfpathlineto{\pgfqpoint{3.749607in}{2.612423in}}%
\pgfpathlineto{\pgfqpoint{3.741818in}{2.604888in}}%
\pgfpathlineto{\pgfqpoint{3.734024in}{2.597366in}}%
\pgfpathlineto{\pgfqpoint{3.726224in}{2.589858in}}%
\pgfpathlineto{\pgfqpoint{3.713005in}{2.594134in}}%
\pgfpathlineto{\pgfqpoint{3.699791in}{2.598444in}}%
\pgfpathlineto{\pgfqpoint{3.686583in}{2.602790in}}%
\pgfpathlineto{\pgfqpoint{3.673379in}{2.607170in}}%
\pgfpathlineto{\pgfqpoint{3.681191in}{2.614670in}}%
\pgfpathlineto{\pgfqpoint{3.688998in}{2.622187in}}%
\pgfpathlineto{\pgfqpoint{3.696799in}{2.629721in}}%
\pgfpathlineto{\pgfqpoint{3.704594in}{2.637273in}}%
\pgfpathclose%
\pgfusepath{fill}%
\end{pgfscope}%
\begin{pgfscope}%
\pgfpathrectangle{\pgfqpoint{1.150000in}{0.150000in}}{\pgfqpoint{5.700000in}{5.700000in}}%
\pgfusepath{clip}%
\pgfsetbuttcap%
\pgfsetroundjoin%
\definecolor{currentfill}{rgb}{0.277018,0.050344,0.375715}%
\pgfsetfillcolor{currentfill}%
\pgfsetfillopacity{0.700000}%
\pgfsetlinewidth{0.000000pt}%
\definecolor{currentstroke}{rgb}{0.000000,0.000000,0.000000}%
\pgfsetstrokecolor{currentstroke}%
\pgfsetdash{}{0pt}%
\pgfpathmoveto{\pgfqpoint{4.639447in}{2.688041in}}%
\pgfpathlineto{\pgfqpoint{4.652843in}{2.684819in}}%
\pgfpathlineto{\pgfqpoint{4.666246in}{2.681624in}}%
\pgfpathlineto{\pgfqpoint{4.679655in}{2.678456in}}%
\pgfpathlineto{\pgfqpoint{4.693070in}{2.675315in}}%
\pgfpathlineto{\pgfqpoint{4.685624in}{2.667869in}}%
\pgfpathlineto{\pgfqpoint{4.678174in}{2.660458in}}%
\pgfpathlineto{\pgfqpoint{4.670718in}{2.653081in}}%
\pgfpathlineto{\pgfqpoint{4.663257in}{2.645733in}}%
\pgfpathlineto{\pgfqpoint{4.649828in}{2.648772in}}%
\pgfpathlineto{\pgfqpoint{4.636405in}{2.651838in}}%
\pgfpathlineto{\pgfqpoint{4.622988in}{2.654931in}}%
\pgfpathlineto{\pgfqpoint{4.609578in}{2.658052in}}%
\pgfpathlineto{\pgfqpoint{4.617053in}{2.665497in}}%
\pgfpathlineto{\pgfqpoint{4.624522in}{2.672974in}}%
\pgfpathlineto{\pgfqpoint{4.631987in}{2.680488in}}%
\pgfpathlineto{\pgfqpoint{4.639447in}{2.688041in}}%
\pgfpathclose%
\pgfusepath{fill}%
\end{pgfscope}%
\begin{pgfscope}%
\pgfpathrectangle{\pgfqpoint{1.150000in}{0.150000in}}{\pgfqpoint{5.700000in}{5.700000in}}%
\pgfusepath{clip}%
\pgfsetbuttcap%
\pgfsetroundjoin%
\definecolor{currentfill}{rgb}{0.274952,0.037752,0.364543}%
\pgfsetfillcolor{currentfill}%
\pgfsetfillopacity{0.700000}%
\pgfsetlinewidth{0.000000pt}%
\definecolor{currentstroke}{rgb}{0.000000,0.000000,0.000000}%
\pgfsetstrokecolor{currentstroke}%
\pgfsetdash{}{0pt}%
\pgfpathmoveto{\pgfqpoint{4.419049in}{2.667089in}}%
\pgfpathlineto{\pgfqpoint{4.432396in}{2.663721in}}%
\pgfpathlineto{\pgfqpoint{4.445750in}{2.660382in}}%
\pgfpathlineto{\pgfqpoint{4.459109in}{2.657072in}}%
\pgfpathlineto{\pgfqpoint{4.472475in}{2.653789in}}%
\pgfpathlineto{\pgfqpoint{4.464948in}{2.646301in}}%
\pgfpathlineto{\pgfqpoint{4.457416in}{2.638835in}}%
\pgfpathlineto{\pgfqpoint{4.449879in}{2.631387in}}%
\pgfpathlineto{\pgfqpoint{4.442336in}{2.623955in}}%
\pgfpathlineto{\pgfqpoint{4.428957in}{2.627162in}}%
\pgfpathlineto{\pgfqpoint{4.415585in}{2.630397in}}%
\pgfpathlineto{\pgfqpoint{4.402219in}{2.633660in}}%
\pgfpathlineto{\pgfqpoint{4.388859in}{2.636952in}}%
\pgfpathlineto{\pgfqpoint{4.396415in}{2.644455in}}%
\pgfpathlineto{\pgfqpoint{4.403965in}{2.651976in}}%
\pgfpathlineto{\pgfqpoint{4.411510in}{2.659520in}}%
\pgfpathlineto{\pgfqpoint{4.419049in}{2.667089in}}%
\pgfpathclose%
\pgfusepath{fill}%
\end{pgfscope}%
\begin{pgfscope}%
\pgfpathrectangle{\pgfqpoint{1.150000in}{0.150000in}}{\pgfqpoint{5.700000in}{5.700000in}}%
\pgfusepath{clip}%
\pgfsetbuttcap%
\pgfsetroundjoin%
\definecolor{currentfill}{rgb}{0.280894,0.078907,0.402329}%
\pgfsetfillcolor{currentfill}%
\pgfsetfillopacity{0.700000}%
\pgfsetlinewidth{0.000000pt}%
\definecolor{currentstroke}{rgb}{0.000000,0.000000,0.000000}%
\pgfsetstrokecolor{currentstroke}%
\pgfsetdash{}{0pt}%
\pgfpathmoveto{\pgfqpoint{5.217358in}{2.740971in}}%
\pgfpathlineto{\pgfqpoint{5.230892in}{2.737878in}}%
\pgfpathlineto{\pgfqpoint{5.244433in}{2.734811in}}%
\pgfpathlineto{\pgfqpoint{5.257980in}{2.731768in}}%
\pgfpathlineto{\pgfqpoint{5.271535in}{2.728750in}}%
\pgfpathlineto{\pgfqpoint{5.264291in}{2.721091in}}%
\pgfpathlineto{\pgfqpoint{5.257044in}{2.713530in}}%
\pgfpathlineto{\pgfqpoint{5.249794in}{2.706063in}}%
\pgfpathlineto{\pgfqpoint{5.242541in}{2.698683in}}%
\pgfpathlineto{\pgfqpoint{5.228970in}{2.701534in}}%
\pgfpathlineto{\pgfqpoint{5.215406in}{2.704410in}}%
\pgfpathlineto{\pgfqpoint{5.201848in}{2.707311in}}%
\pgfpathlineto{\pgfqpoint{5.188298in}{2.710237in}}%
\pgfpathlineto{\pgfqpoint{5.195568in}{2.717779in}}%
\pgfpathlineto{\pgfqpoint{5.202834in}{2.725412in}}%
\pgfpathlineto{\pgfqpoint{5.210097in}{2.733141in}}%
\pgfpathlineto{\pgfqpoint{5.217358in}{2.740971in}}%
\pgfpathclose%
\pgfusepath{fill}%
\end{pgfscope}%
\begin{pgfscope}%
\pgfpathrectangle{\pgfqpoint{1.150000in}{0.150000in}}{\pgfqpoint{5.700000in}{5.700000in}}%
\pgfusepath{clip}%
\pgfsetbuttcap%
\pgfsetroundjoin%
\definecolor{currentfill}{rgb}{0.269944,0.014625,0.341379}%
\pgfsetfillcolor{currentfill}%
\pgfsetfillopacity{0.700000}%
\pgfsetlinewidth{0.000000pt}%
\definecolor{currentstroke}{rgb}{0.000000,0.000000,0.000000}%
\pgfsetstrokecolor{currentstroke}%
\pgfsetdash{}{0pt}%
\pgfpathmoveto{\pgfqpoint{3.841294in}{2.633553in}}%
\pgfpathlineto{\pgfqpoint{3.854515in}{2.629442in}}%
\pgfpathlineto{\pgfqpoint{3.867742in}{2.625365in}}%
\pgfpathlineto{\pgfqpoint{3.880974in}{2.621321in}}%
\pgfpathlineto{\pgfqpoint{3.894212in}{2.617310in}}%
\pgfpathlineto{\pgfqpoint{3.886477in}{2.609724in}}%
\pgfpathlineto{\pgfqpoint{3.878736in}{2.602148in}}%
\pgfpathlineto{\pgfqpoint{3.870989in}{2.594583in}}%
\pgfpathlineto{\pgfqpoint{3.863237in}{2.587027in}}%
\pgfpathlineto{\pgfqpoint{3.849987in}{2.591028in}}%
\pgfpathlineto{\pgfqpoint{3.836742in}{2.595062in}}%
\pgfpathlineto{\pgfqpoint{3.823503in}{2.599129in}}%
\pgfpathlineto{\pgfqpoint{3.810270in}{2.603230in}}%
\pgfpathlineto{\pgfqpoint{3.818034in}{2.610791in}}%
\pgfpathlineto{\pgfqpoint{3.825793in}{2.618364in}}%
\pgfpathlineto{\pgfqpoint{3.833546in}{2.625951in}}%
\pgfpathlineto{\pgfqpoint{3.841294in}{2.633553in}}%
\pgfpathclose%
\pgfusepath{fill}%
\end{pgfscope}%
\begin{pgfscope}%
\pgfpathrectangle{\pgfqpoint{1.150000in}{0.150000in}}{\pgfqpoint{5.700000in}{5.700000in}}%
\pgfusepath{clip}%
\pgfsetbuttcap%
\pgfsetroundjoin%
\definecolor{currentfill}{rgb}{0.273809,0.031497,0.358853}%
\pgfsetfillcolor{currentfill}%
\pgfsetfillopacity{0.700000}%
\pgfsetlinewidth{0.000000pt}%
\definecolor{currentstroke}{rgb}{0.000000,0.000000,0.000000}%
\pgfsetstrokecolor{currentstroke}%
\pgfsetdash{}{0pt}%
\pgfpathmoveto{\pgfqpoint{3.210343in}{2.659114in}}%
\pgfpathlineto{\pgfqpoint{3.223455in}{2.653536in}}%
\pgfpathlineto{\pgfqpoint{3.236572in}{2.648002in}}%
\pgfpathlineto{\pgfqpoint{3.249692in}{2.642513in}}%
\pgfpathlineto{\pgfqpoint{3.262817in}{2.637066in}}%
\pgfpathlineto{\pgfqpoint{3.254846in}{2.630096in}}%
\pgfpathlineto{\pgfqpoint{3.246868in}{2.623167in}}%
\pgfpathlineto{\pgfqpoint{3.238884in}{2.616278in}}%
\pgfpathlineto{\pgfqpoint{3.230893in}{2.609432in}}%
\pgfpathlineto{\pgfqpoint{3.217753in}{2.614935in}}%
\pgfpathlineto{\pgfqpoint{3.204618in}{2.620481in}}%
\pgfpathlineto{\pgfqpoint{3.191487in}{2.626070in}}%
\pgfpathlineto{\pgfqpoint{3.178361in}{2.631704in}}%
\pgfpathlineto{\pgfqpoint{3.186366in}{2.638489in}}%
\pgfpathlineto{\pgfqpoint{3.194365in}{2.645320in}}%
\pgfpathlineto{\pgfqpoint{3.202357in}{2.652195in}}%
\pgfpathlineto{\pgfqpoint{3.210343in}{2.659114in}}%
\pgfpathclose%
\pgfusepath{fill}%
\end{pgfscope}%
\begin{pgfscope}%
\pgfpathrectangle{\pgfqpoint{1.150000in}{0.150000in}}{\pgfqpoint{5.700000in}{5.700000in}}%
\pgfusepath{clip}%
\pgfsetbuttcap%
\pgfsetroundjoin%
\definecolor{currentfill}{rgb}{0.272594,0.025563,0.353093}%
\pgfsetfillcolor{currentfill}%
\pgfsetfillopacity{0.700000}%
\pgfsetlinewidth{0.000000pt}%
\definecolor{currentstroke}{rgb}{0.000000,0.000000,0.000000}%
\pgfsetstrokecolor{currentstroke}%
\pgfsetdash{}{0pt}%
\pgfpathmoveto{\pgfqpoint{4.198595in}{2.648116in}}%
\pgfpathlineto{\pgfqpoint{4.211894in}{2.644534in}}%
\pgfpathlineto{\pgfqpoint{4.225199in}{2.640982in}}%
\pgfpathlineto{\pgfqpoint{4.238510in}{2.637459in}}%
\pgfpathlineto{\pgfqpoint{4.251827in}{2.633967in}}%
\pgfpathlineto{\pgfqpoint{4.244219in}{2.626417in}}%
\pgfpathlineto{\pgfqpoint{4.236606in}{2.618880in}}%
\pgfpathlineto{\pgfqpoint{4.228987in}{2.611352in}}%
\pgfpathlineto{\pgfqpoint{4.221363in}{2.603834in}}%
\pgfpathlineto{\pgfqpoint{4.208034in}{2.607276in}}%
\pgfpathlineto{\pgfqpoint{4.194710in}{2.610749in}}%
\pgfpathlineto{\pgfqpoint{4.181393in}{2.614252in}}%
\pgfpathlineto{\pgfqpoint{4.168082in}{2.617785in}}%
\pgfpathlineto{\pgfqpoint{4.175718in}{2.625348in}}%
\pgfpathlineto{\pgfqpoint{4.183349in}{2.632923in}}%
\pgfpathlineto{\pgfqpoint{4.190975in}{2.640512in}}%
\pgfpathlineto{\pgfqpoint{4.198595in}{2.648116in}}%
\pgfpathclose%
\pgfusepath{fill}%
\end{pgfscope}%
\begin{pgfscope}%
\pgfpathrectangle{\pgfqpoint{1.150000in}{0.150000in}}{\pgfqpoint{5.700000in}{5.700000in}}%
\pgfusepath{clip}%
\pgfsetbuttcap%
\pgfsetroundjoin%
\definecolor{currentfill}{rgb}{0.279566,0.067836,0.391917}%
\pgfsetfillcolor{currentfill}%
\pgfsetfillopacity{0.700000}%
\pgfsetlinewidth{0.000000pt}%
\definecolor{currentstroke}{rgb}{0.000000,0.000000,0.000000}%
\pgfsetstrokecolor{currentstroke}%
\pgfsetdash{}{0pt}%
\pgfpathmoveto{\pgfqpoint{4.996923in}{2.716108in}}%
\pgfpathlineto{\pgfqpoint{5.010407in}{2.713032in}}%
\pgfpathlineto{\pgfqpoint{5.023898in}{2.709982in}}%
\pgfpathlineto{\pgfqpoint{5.037396in}{2.706957in}}%
\pgfpathlineto{\pgfqpoint{5.050901in}{2.703958in}}%
\pgfpathlineto{\pgfqpoint{5.043582in}{2.696504in}}%
\pgfpathlineto{\pgfqpoint{5.036258in}{2.689118in}}%
\pgfpathlineto{\pgfqpoint{5.028931in}{2.681796in}}%
\pgfpathlineto{\pgfqpoint{5.021600in}{2.674534in}}%
\pgfpathlineto{\pgfqpoint{5.008079in}{2.677392in}}%
\pgfpathlineto{\pgfqpoint{4.994566in}{2.680276in}}%
\pgfpathlineto{\pgfqpoint{4.981060in}{2.683185in}}%
\pgfpathlineto{\pgfqpoint{4.967560in}{2.686121in}}%
\pgfpathlineto{\pgfqpoint{4.974907in}{2.693519in}}%
\pgfpathlineto{\pgfqpoint{4.982249in}{2.700981in}}%
\pgfpathlineto{\pgfqpoint{4.989588in}{2.708509in}}%
\pgfpathlineto{\pgfqpoint{4.996923in}{2.716108in}}%
\pgfpathclose%
\pgfusepath{fill}%
\end{pgfscope}%
\begin{pgfscope}%
\pgfpathrectangle{\pgfqpoint{1.150000in}{0.150000in}}{\pgfqpoint{5.700000in}{5.700000in}}%
\pgfusepath{clip}%
\pgfsetbuttcap%
\pgfsetroundjoin%
\definecolor{currentfill}{rgb}{0.271305,0.019942,0.347269}%
\pgfsetfillcolor{currentfill}%
\pgfsetfillopacity{0.700000}%
\pgfsetlinewidth{0.000000pt}%
\definecolor{currentstroke}{rgb}{0.000000,0.000000,0.000000}%
\pgfsetstrokecolor{currentstroke}%
\pgfsetdash{}{0pt}%
\pgfpathmoveto{\pgfqpoint{3.347124in}{2.644145in}}%
\pgfpathlineto{\pgfqpoint{3.360257in}{2.638953in}}%
\pgfpathlineto{\pgfqpoint{3.373394in}{2.633802in}}%
\pgfpathlineto{\pgfqpoint{3.386536in}{2.628692in}}%
\pgfpathlineto{\pgfqpoint{3.399682in}{2.623622in}}%
\pgfpathlineto{\pgfqpoint{3.391764in}{2.616423in}}%
\pgfpathlineto{\pgfqpoint{3.383840in}{2.609254in}}%
\pgfpathlineto{\pgfqpoint{3.375909in}{2.602118in}}%
\pgfpathlineto{\pgfqpoint{3.367972in}{2.595013in}}%
\pgfpathlineto{\pgfqpoint{3.354812in}{2.600125in}}%
\pgfpathlineto{\pgfqpoint{3.341657in}{2.605278in}}%
\pgfpathlineto{\pgfqpoint{3.328506in}{2.610472in}}%
\pgfpathlineto{\pgfqpoint{3.315360in}{2.615707in}}%
\pgfpathlineto{\pgfqpoint{3.323310in}{2.622764in}}%
\pgfpathlineto{\pgfqpoint{3.331255in}{2.629856in}}%
\pgfpathlineto{\pgfqpoint{3.339193in}{2.636984in}}%
\pgfpathlineto{\pgfqpoint{3.347124in}{2.644145in}}%
\pgfpathclose%
\pgfusepath{fill}%
\end{pgfscope}%
\begin{pgfscope}%
\pgfpathrectangle{\pgfqpoint{1.150000in}{0.150000in}}{\pgfqpoint{5.700000in}{5.700000in}}%
\pgfusepath{clip}%
\pgfsetbuttcap%
\pgfsetroundjoin%
\definecolor{currentfill}{rgb}{0.276022,0.044167,0.370164}%
\pgfsetfillcolor{currentfill}%
\pgfsetfillopacity{0.700000}%
\pgfsetlinewidth{0.000000pt}%
\definecolor{currentstroke}{rgb}{0.000000,0.000000,0.000000}%
\pgfsetstrokecolor{currentstroke}%
\pgfsetdash{}{0pt}%
\pgfpathmoveto{\pgfqpoint{3.073486in}{2.678416in}}%
\pgfpathlineto{\pgfqpoint{3.086582in}{2.672414in}}%
\pgfpathlineto{\pgfqpoint{3.099682in}{2.666459in}}%
\pgfpathlineto{\pgfqpoint{3.112786in}{2.660551in}}%
\pgfpathlineto{\pgfqpoint{3.125893in}{2.654690in}}%
\pgfpathlineto{\pgfqpoint{3.117865in}{2.648015in}}%
\pgfpathlineto{\pgfqpoint{3.109831in}{2.641393in}}%
\pgfpathlineto{\pgfqpoint{3.101789in}{2.634822in}}%
\pgfpathlineto{\pgfqpoint{3.093741in}{2.628305in}}%
\pgfpathlineto{\pgfqpoint{3.080618in}{2.634236in}}%
\pgfpathlineto{\pgfqpoint{3.067500in}{2.640213in}}%
\pgfpathlineto{\pgfqpoint{3.054385in}{2.646237in}}%
\pgfpathlineto{\pgfqpoint{3.041274in}{2.652309in}}%
\pgfpathlineto{\pgfqpoint{3.049338in}{2.658752in}}%
\pgfpathlineto{\pgfqpoint{3.057394in}{2.665251in}}%
\pgfpathlineto{\pgfqpoint{3.065444in}{2.671807in}}%
\pgfpathlineto{\pgfqpoint{3.073486in}{2.678416in}}%
\pgfpathclose%
\pgfusepath{fill}%
\end{pgfscope}%
\begin{pgfscope}%
\pgfpathrectangle{\pgfqpoint{1.150000in}{0.150000in}}{\pgfqpoint{5.700000in}{5.700000in}}%
\pgfusepath{clip}%
\pgfsetbuttcap%
\pgfsetroundjoin%
\definecolor{currentfill}{rgb}{0.277941,0.056324,0.381191}%
\pgfsetfillcolor{currentfill}%
\pgfsetfillopacity{0.700000}%
\pgfsetlinewidth{0.000000pt}%
\definecolor{currentstroke}{rgb}{0.000000,0.000000,0.000000}%
\pgfsetstrokecolor{currentstroke}%
\pgfsetdash{}{0pt}%
\pgfpathmoveto{\pgfqpoint{4.776478in}{2.692784in}}%
\pgfpathlineto{\pgfqpoint{4.789912in}{2.689663in}}%
\pgfpathlineto{\pgfqpoint{4.803353in}{2.686568in}}%
\pgfpathlineto{\pgfqpoint{4.816801in}{2.683500in}}%
\pgfpathlineto{\pgfqpoint{4.830255in}{2.680458in}}%
\pgfpathlineto{\pgfqpoint{4.822856in}{2.673069in}}%
\pgfpathlineto{\pgfqpoint{4.815453in}{2.665725in}}%
\pgfpathlineto{\pgfqpoint{4.808045in}{2.658422in}}%
\pgfpathlineto{\pgfqpoint{4.800632in}{2.651155in}}%
\pgfpathlineto{\pgfqpoint{4.787164in}{2.654082in}}%
\pgfpathlineto{\pgfqpoint{4.773702in}{2.657035in}}%
\pgfpathlineto{\pgfqpoint{4.760247in}{2.660015in}}%
\pgfpathlineto{\pgfqpoint{4.746798in}{2.663022in}}%
\pgfpathlineto{\pgfqpoint{4.754226in}{2.670398in}}%
\pgfpathlineto{\pgfqpoint{4.761648in}{2.677815in}}%
\pgfpathlineto{\pgfqpoint{4.769065in}{2.685276in}}%
\pgfpathlineto{\pgfqpoint{4.776478in}{2.692784in}}%
\pgfpathclose%
\pgfusepath{fill}%
\end{pgfscope}%
\begin{pgfscope}%
\pgfpathrectangle{\pgfqpoint{1.150000in}{0.150000in}}{\pgfqpoint{5.700000in}{5.700000in}}%
\pgfusepath{clip}%
\pgfsetbuttcap%
\pgfsetroundjoin%
\definecolor{currentfill}{rgb}{0.271305,0.019942,0.347269}%
\pgfsetfillcolor{currentfill}%
\pgfsetfillopacity{0.700000}%
\pgfsetlinewidth{0.000000pt}%
\definecolor{currentstroke}{rgb}{0.000000,0.000000,0.000000}%
\pgfsetstrokecolor{currentstroke}%
\pgfsetdash{}{0pt}%
\pgfpathmoveto{\pgfqpoint{3.483871in}{2.632959in}}%
\pgfpathlineto{\pgfqpoint{3.497028in}{2.628116in}}%
\pgfpathlineto{\pgfqpoint{3.510189in}{2.623313in}}%
\pgfpathlineto{\pgfqpoint{3.523355in}{2.618547in}}%
\pgfpathlineto{\pgfqpoint{3.536526in}{2.613820in}}%
\pgfpathlineto{\pgfqpoint{3.528658in}{2.606451in}}%
\pgfpathlineto{\pgfqpoint{3.520785in}{2.599106in}}%
\pgfpathlineto{\pgfqpoint{3.512906in}{2.591784in}}%
\pgfpathlineto{\pgfqpoint{3.505020in}{2.584485in}}%
\pgfpathlineto{\pgfqpoint{3.491836in}{2.589242in}}%
\pgfpathlineto{\pgfqpoint{3.478657in}{2.594037in}}%
\pgfpathlineto{\pgfqpoint{3.465483in}{2.598870in}}%
\pgfpathlineto{\pgfqpoint{3.452314in}{2.603742in}}%
\pgfpathlineto{\pgfqpoint{3.460212in}{2.611006in}}%
\pgfpathlineto{\pgfqpoint{3.468105in}{2.618297in}}%
\pgfpathlineto{\pgfqpoint{3.475991in}{2.625615in}}%
\pgfpathlineto{\pgfqpoint{3.483871in}{2.632959in}}%
\pgfpathclose%
\pgfusepath{fill}%
\end{pgfscope}%
\begin{pgfscope}%
\pgfpathrectangle{\pgfqpoint{1.150000in}{0.150000in}}{\pgfqpoint{5.700000in}{5.700000in}}%
\pgfusepath{clip}%
\pgfsetbuttcap%
\pgfsetroundjoin%
\definecolor{currentfill}{rgb}{0.278791,0.062145,0.386592}%
\pgfsetfillcolor{currentfill}%
\pgfsetfillopacity{0.700000}%
\pgfsetlinewidth{0.000000pt}%
\definecolor{currentstroke}{rgb}{0.000000,0.000000,0.000000}%
\pgfsetstrokecolor{currentstroke}%
\pgfsetdash{}{0pt}%
\pgfpathmoveto{\pgfqpoint{2.936509in}{2.702660in}}%
\pgfpathlineto{\pgfqpoint{2.949593in}{2.696189in}}%
\pgfpathlineto{\pgfqpoint{2.962680in}{2.689770in}}%
\pgfpathlineto{\pgfqpoint{2.975770in}{2.683402in}}%
\pgfpathlineto{\pgfqpoint{2.988864in}{2.677084in}}%
\pgfpathlineto{\pgfqpoint{2.980777in}{2.670778in}}%
\pgfpathlineto{\pgfqpoint{2.972682in}{2.664535in}}%
\pgfpathlineto{\pgfqpoint{2.964579in}{2.658358in}}%
\pgfpathlineto{\pgfqpoint{2.956469in}{2.652247in}}%
\pgfpathlineto{\pgfqpoint{2.943359in}{2.658648in}}%
\pgfpathlineto{\pgfqpoint{2.930253in}{2.665099in}}%
\pgfpathlineto{\pgfqpoint{2.917149in}{2.671601in}}%
\pgfpathlineto{\pgfqpoint{2.904049in}{2.678155in}}%
\pgfpathlineto{\pgfqpoint{2.912176in}{2.684177in}}%
\pgfpathlineto{\pgfqpoint{2.920295in}{2.690270in}}%
\pgfpathlineto{\pgfqpoint{2.928406in}{2.696432in}}%
\pgfpathlineto{\pgfqpoint{2.936509in}{2.702660in}}%
\pgfpathclose%
\pgfusepath{fill}%
\end{pgfscope}%
\begin{pgfscope}%
\pgfpathrectangle{\pgfqpoint{1.150000in}{0.150000in}}{\pgfqpoint{5.700000in}{5.700000in}}%
\pgfusepath{clip}%
\pgfsetbuttcap%
\pgfsetroundjoin%
\definecolor{currentfill}{rgb}{0.276022,0.044167,0.370164}%
\pgfsetfillcolor{currentfill}%
\pgfsetfillopacity{0.700000}%
\pgfsetlinewidth{0.000000pt}%
\definecolor{currentstroke}{rgb}{0.000000,0.000000,0.000000}%
\pgfsetstrokecolor{currentstroke}%
\pgfsetdash{}{0pt}%
\pgfpathmoveto{\pgfqpoint{4.556003in}{2.670810in}}%
\pgfpathlineto{\pgfqpoint{4.569387in}{2.667579in}}%
\pgfpathlineto{\pgfqpoint{4.582778in}{2.664376in}}%
\pgfpathlineto{\pgfqpoint{4.596175in}{2.661200in}}%
\pgfpathlineto{\pgfqpoint{4.609578in}{2.658052in}}%
\pgfpathlineto{\pgfqpoint{4.602099in}{2.650637in}}%
\pgfpathlineto{\pgfqpoint{4.594614in}{2.643248in}}%
\pgfpathlineto{\pgfqpoint{4.587125in}{2.635883in}}%
\pgfpathlineto{\pgfqpoint{4.579630in}{2.628539in}}%
\pgfpathlineto{\pgfqpoint{4.566213in}{2.631598in}}%
\pgfpathlineto{\pgfqpoint{4.552802in}{2.634685in}}%
\pgfpathlineto{\pgfqpoint{4.539398in}{2.637799in}}%
\pgfpathlineto{\pgfqpoint{4.526001in}{2.640942in}}%
\pgfpathlineto{\pgfqpoint{4.533509in}{2.648370in}}%
\pgfpathlineto{\pgfqpoint{4.541012in}{2.655822in}}%
\pgfpathlineto{\pgfqpoint{4.548510in}{2.663301in}}%
\pgfpathlineto{\pgfqpoint{4.556003in}{2.670810in}}%
\pgfpathclose%
\pgfusepath{fill}%
\end{pgfscope}%
\begin{pgfscope}%
\pgfpathrectangle{\pgfqpoint{1.150000in}{0.150000in}}{\pgfqpoint{5.700000in}{5.700000in}}%
\pgfusepath{clip}%
\pgfsetbuttcap%
\pgfsetroundjoin%
\definecolor{currentfill}{rgb}{0.271305,0.019942,0.347269}%
\pgfsetfillcolor{currentfill}%
\pgfsetfillopacity{0.700000}%
\pgfsetlinewidth{0.000000pt}%
\definecolor{currentstroke}{rgb}{0.000000,0.000000,0.000000}%
\pgfsetstrokecolor{currentstroke}%
\pgfsetdash{}{0pt}%
\pgfpathmoveto{\pgfqpoint{3.978056in}{2.631982in}}%
\pgfpathlineto{\pgfqpoint{3.991309in}{2.628111in}}%
\pgfpathlineto{\pgfqpoint{4.004569in}{2.624273in}}%
\pgfpathlineto{\pgfqpoint{4.017834in}{2.620466in}}%
\pgfpathlineto{\pgfqpoint{4.031104in}{2.616692in}}%
\pgfpathlineto{\pgfqpoint{4.023416in}{2.609106in}}%
\pgfpathlineto{\pgfqpoint{4.015722in}{2.601529in}}%
\pgfpathlineto{\pgfqpoint{4.008022in}{2.593960in}}%
\pgfpathlineto{\pgfqpoint{4.000317in}{2.586398in}}%
\pgfpathlineto{\pgfqpoint{3.987034in}{2.590150in}}%
\pgfpathlineto{\pgfqpoint{3.973756in}{2.593933in}}%
\pgfpathlineto{\pgfqpoint{3.960485in}{2.597749in}}%
\pgfpathlineto{\pgfqpoint{3.947219in}{2.601596in}}%
\pgfpathlineto{\pgfqpoint{3.954937in}{2.609177in}}%
\pgfpathlineto{\pgfqpoint{3.962649in}{2.616767in}}%
\pgfpathlineto{\pgfqpoint{3.970355in}{2.624368in}}%
\pgfpathlineto{\pgfqpoint{3.978056in}{2.631982in}}%
\pgfpathclose%
\pgfusepath{fill}%
\end{pgfscope}%
\begin{pgfscope}%
\pgfpathrectangle{\pgfqpoint{1.150000in}{0.150000in}}{\pgfqpoint{5.700000in}{5.700000in}}%
\pgfusepath{clip}%
\pgfsetbuttcap%
\pgfsetroundjoin%
\definecolor{currentfill}{rgb}{0.269944,0.014625,0.341379}%
\pgfsetfillcolor{currentfill}%
\pgfsetfillopacity{0.700000}%
\pgfsetlinewidth{0.000000pt}%
\definecolor{currentstroke}{rgb}{0.000000,0.000000,0.000000}%
\pgfsetstrokecolor{currentstroke}%
\pgfsetdash{}{0pt}%
\pgfpathmoveto{\pgfqpoint{3.620617in}{2.625053in}}%
\pgfpathlineto{\pgfqpoint{3.633800in}{2.620528in}}%
\pgfpathlineto{\pgfqpoint{3.646988in}{2.616039in}}%
\pgfpathlineto{\pgfqpoint{3.660181in}{2.611587in}}%
\pgfpathlineto{\pgfqpoint{3.673379in}{2.607170in}}%
\pgfpathlineto{\pgfqpoint{3.665561in}{2.599688in}}%
\pgfpathlineto{\pgfqpoint{3.657737in}{2.592221in}}%
\pgfpathlineto{\pgfqpoint{3.649907in}{2.584771in}}%
\pgfpathlineto{\pgfqpoint{3.642071in}{2.577338in}}%
\pgfpathlineto{\pgfqpoint{3.628860in}{2.581770in}}%
\pgfpathlineto{\pgfqpoint{3.615655in}{2.586239in}}%
\pgfpathlineto{\pgfqpoint{3.602454in}{2.590744in}}%
\pgfpathlineto{\pgfqpoint{3.589258in}{2.595285in}}%
\pgfpathlineto{\pgfqpoint{3.597107in}{2.602698in}}%
\pgfpathlineto{\pgfqpoint{3.604949in}{2.610130in}}%
\pgfpathlineto{\pgfqpoint{3.612786in}{2.617581in}}%
\pgfpathlineto{\pgfqpoint{3.620617in}{2.625053in}}%
\pgfpathclose%
\pgfusepath{fill}%
\end{pgfscope}%
\begin{pgfscope}%
\pgfpathrectangle{\pgfqpoint{1.150000in}{0.150000in}}{\pgfqpoint{5.700000in}{5.700000in}}%
\pgfusepath{clip}%
\pgfsetbuttcap%
\pgfsetroundjoin%
\definecolor{currentfill}{rgb}{0.273809,0.031497,0.358853}%
\pgfsetfillcolor{currentfill}%
\pgfsetfillopacity{0.700000}%
\pgfsetlinewidth{0.000000pt}%
\definecolor{currentstroke}{rgb}{0.000000,0.000000,0.000000}%
\pgfsetstrokecolor{currentstroke}%
\pgfsetdash{}{0pt}%
\pgfpathmoveto{\pgfqpoint{4.335482in}{2.650408in}}%
\pgfpathlineto{\pgfqpoint{4.348817in}{2.647001in}}%
\pgfpathlineto{\pgfqpoint{4.362158in}{2.643622in}}%
\pgfpathlineto{\pgfqpoint{4.375506in}{2.640273in}}%
\pgfpathlineto{\pgfqpoint{4.388859in}{2.636952in}}%
\pgfpathlineto{\pgfqpoint{4.381298in}{2.629467in}}%
\pgfpathlineto{\pgfqpoint{4.373732in}{2.621995in}}%
\pgfpathlineto{\pgfqpoint{4.366161in}{2.614536in}}%
\pgfpathlineto{\pgfqpoint{4.358584in}{2.607087in}}%
\pgfpathlineto{\pgfqpoint{4.345217in}{2.610346in}}%
\pgfpathlineto{\pgfqpoint{4.331857in}{2.613633in}}%
\pgfpathlineto{\pgfqpoint{4.318503in}{2.616949in}}%
\pgfpathlineto{\pgfqpoint{4.305156in}{2.620294in}}%
\pgfpathlineto{\pgfqpoint{4.312745in}{2.627800in}}%
\pgfpathlineto{\pgfqpoint{4.320330in}{2.635320in}}%
\pgfpathlineto{\pgfqpoint{4.327909in}{2.642855in}}%
\pgfpathlineto{\pgfqpoint{4.335482in}{2.650408in}}%
\pgfpathclose%
\pgfusepath{fill}%
\end{pgfscope}%
\begin{pgfscope}%
\pgfpathrectangle{\pgfqpoint{1.150000in}{0.150000in}}{\pgfqpoint{5.700000in}{5.700000in}}%
\pgfusepath{clip}%
\pgfsetbuttcap%
\pgfsetroundjoin%
\definecolor{currentfill}{rgb}{0.280894,0.078907,0.402329}%
\pgfsetfillcolor{currentfill}%
\pgfsetfillopacity{0.700000}%
\pgfsetlinewidth{0.000000pt}%
\definecolor{currentstroke}{rgb}{0.000000,0.000000,0.000000}%
\pgfsetstrokecolor{currentstroke}%
\pgfsetdash{}{0pt}%
\pgfpathmoveto{\pgfqpoint{5.134165in}{2.722191in}}%
\pgfpathlineto{\pgfqpoint{5.147688in}{2.719165in}}%
\pgfpathlineto{\pgfqpoint{5.161218in}{2.716164in}}%
\pgfpathlineto{\pgfqpoint{5.174754in}{2.713188in}}%
\pgfpathlineto{\pgfqpoint{5.188298in}{2.710237in}}%
\pgfpathlineto{\pgfqpoint{5.181025in}{2.702780in}}%
\pgfpathlineto{\pgfqpoint{5.173749in}{2.695405in}}%
\pgfpathlineto{\pgfqpoint{5.166469in}{2.688105in}}%
\pgfpathlineto{\pgfqpoint{5.159185in}{2.680877in}}%
\pgfpathlineto{\pgfqpoint{5.145625in}{2.683674in}}%
\pgfpathlineto{\pgfqpoint{5.132073in}{2.686496in}}%
\pgfpathlineto{\pgfqpoint{5.118527in}{2.689343in}}%
\pgfpathlineto{\pgfqpoint{5.104988in}{2.692215in}}%
\pgfpathlineto{\pgfqpoint{5.112288in}{2.699593in}}%
\pgfpathlineto{\pgfqpoint{5.119584in}{2.707045in}}%
\pgfpathlineto{\pgfqpoint{5.126876in}{2.714576in}}%
\pgfpathlineto{\pgfqpoint{5.134165in}{2.722191in}}%
\pgfpathclose%
\pgfusepath{fill}%
\end{pgfscope}%
\begin{pgfscope}%
\pgfpathrectangle{\pgfqpoint{1.150000in}{0.150000in}}{\pgfqpoint{5.700000in}{5.700000in}}%
\pgfusepath{clip}%
\pgfsetbuttcap%
\pgfsetroundjoin%
\definecolor{currentfill}{rgb}{0.278791,0.062145,0.386592}%
\pgfsetfillcolor{currentfill}%
\pgfsetfillopacity{0.700000}%
\pgfsetlinewidth{0.000000pt}%
\definecolor{currentstroke}{rgb}{0.000000,0.000000,0.000000}%
\pgfsetstrokecolor{currentstroke}%
\pgfsetdash{}{0pt}%
\pgfpathmoveto{\pgfqpoint{4.913629in}{2.698120in}}%
\pgfpathlineto{\pgfqpoint{4.927101in}{2.695081in}}%
\pgfpathlineto{\pgfqpoint{4.940581in}{2.692069in}}%
\pgfpathlineto{\pgfqpoint{4.954067in}{2.689082in}}%
\pgfpathlineto{\pgfqpoint{4.967560in}{2.686121in}}%
\pgfpathlineto{\pgfqpoint{4.960209in}{2.678780in}}%
\pgfpathlineto{\pgfqpoint{4.952853in}{2.671494in}}%
\pgfpathlineto{\pgfqpoint{4.945494in}{2.664257in}}%
\pgfpathlineto{\pgfqpoint{4.938129in}{2.657066in}}%
\pgfpathlineto{\pgfqpoint{4.924621in}{2.659899in}}%
\pgfpathlineto{\pgfqpoint{4.911120in}{2.662758in}}%
\pgfpathlineto{\pgfqpoint{4.897626in}{2.665643in}}%
\pgfpathlineto{\pgfqpoint{4.884138in}{2.668554in}}%
\pgfpathlineto{\pgfqpoint{4.891517in}{2.675868in}}%
\pgfpathlineto{\pgfqpoint{4.898892in}{2.683231in}}%
\pgfpathlineto{\pgfqpoint{4.906263in}{2.690647in}}%
\pgfpathlineto{\pgfqpoint{4.913629in}{2.698120in}}%
\pgfpathclose%
\pgfusepath{fill}%
\end{pgfscope}%
\begin{pgfscope}%
\pgfpathrectangle{\pgfqpoint{1.150000in}{0.150000in}}{\pgfqpoint{5.700000in}{5.700000in}}%
\pgfusepath{clip}%
\pgfsetbuttcap%
\pgfsetroundjoin%
\definecolor{currentfill}{rgb}{0.269944,0.014625,0.341379}%
\pgfsetfillcolor{currentfill}%
\pgfsetfillopacity{0.700000}%
\pgfsetlinewidth{0.000000pt}%
\definecolor{currentstroke}{rgb}{0.000000,0.000000,0.000000}%
\pgfsetstrokecolor{currentstroke}%
\pgfsetdash{}{0pt}%
\pgfpathmoveto{\pgfqpoint{3.757389in}{2.619974in}}%
\pgfpathlineto{\pgfqpoint{3.770601in}{2.615736in}}%
\pgfpathlineto{\pgfqpoint{3.783819in}{2.611533in}}%
\pgfpathlineto{\pgfqpoint{3.797041in}{2.607365in}}%
\pgfpathlineto{\pgfqpoint{3.810270in}{2.603230in}}%
\pgfpathlineto{\pgfqpoint{3.802499in}{2.595681in}}%
\pgfpathlineto{\pgfqpoint{3.794723in}{2.588144in}}%
\pgfpathlineto{\pgfqpoint{3.786942in}{2.580618in}}%
\pgfpathlineto{\pgfqpoint{3.779154in}{2.573102in}}%
\pgfpathlineto{\pgfqpoint{3.765914in}{2.577240in}}%
\pgfpathlineto{\pgfqpoint{3.752678in}{2.581412in}}%
\pgfpathlineto{\pgfqpoint{3.739449in}{2.585618in}}%
\pgfpathlineto{\pgfqpoint{3.726224in}{2.589858in}}%
\pgfpathlineto{\pgfqpoint{3.734024in}{2.597366in}}%
\pgfpathlineto{\pgfqpoint{3.741818in}{2.604888in}}%
\pgfpathlineto{\pgfqpoint{3.749607in}{2.612423in}}%
\pgfpathlineto{\pgfqpoint{3.757389in}{2.619974in}}%
\pgfpathclose%
\pgfusepath{fill}%
\end{pgfscope}%
\begin{pgfscope}%
\pgfpathrectangle{\pgfqpoint{1.150000in}{0.150000in}}{\pgfqpoint{5.700000in}{5.700000in}}%
\pgfusepath{clip}%
\pgfsetbuttcap%
\pgfsetroundjoin%
\definecolor{currentfill}{rgb}{0.271305,0.019942,0.347269}%
\pgfsetfillcolor{currentfill}%
\pgfsetfillopacity{0.700000}%
\pgfsetlinewidth{0.000000pt}%
\definecolor{currentstroke}{rgb}{0.000000,0.000000,0.000000}%
\pgfsetstrokecolor{currentstroke}%
\pgfsetdash{}{0pt}%
\pgfpathmoveto{\pgfqpoint{4.114896in}{2.632221in}}%
\pgfpathlineto{\pgfqpoint{4.128183in}{2.628566in}}%
\pgfpathlineto{\pgfqpoint{4.141477in}{2.624942in}}%
\pgfpathlineto{\pgfqpoint{4.154776in}{2.621348in}}%
\pgfpathlineto{\pgfqpoint{4.168082in}{2.617785in}}%
\pgfpathlineto{\pgfqpoint{4.160440in}{2.610232in}}%
\pgfpathlineto{\pgfqpoint{4.152792in}{2.602687in}}%
\pgfpathlineto{\pgfqpoint{4.145140in}{2.595148in}}%
\pgfpathlineto{\pgfqpoint{4.137481in}{2.587615in}}%
\pgfpathlineto{\pgfqpoint{4.124163in}{2.591142in}}%
\pgfpathlineto{\pgfqpoint{4.110851in}{2.594699in}}%
\pgfpathlineto{\pgfqpoint{4.097546in}{2.598287in}}%
\pgfpathlineto{\pgfqpoint{4.084246in}{2.601906in}}%
\pgfpathlineto{\pgfqpoint{4.091916in}{2.609471in}}%
\pgfpathlineto{\pgfqpoint{4.099582in}{2.617044in}}%
\pgfpathlineto{\pgfqpoint{4.107241in}{2.624627in}}%
\pgfpathlineto{\pgfqpoint{4.114896in}{2.632221in}}%
\pgfpathclose%
\pgfusepath{fill}%
\end{pgfscope}%
\begin{pgfscope}%
\pgfpathrectangle{\pgfqpoint{1.150000in}{0.150000in}}{\pgfqpoint{5.700000in}{5.700000in}}%
\pgfusepath{clip}%
\pgfsetbuttcap%
\pgfsetroundjoin%
\definecolor{currentfill}{rgb}{0.277018,0.050344,0.375715}%
\pgfsetfillcolor{currentfill}%
\pgfsetfillopacity{0.700000}%
\pgfsetlinewidth{0.000000pt}%
\definecolor{currentstroke}{rgb}{0.000000,0.000000,0.000000}%
\pgfsetstrokecolor{currentstroke}%
\pgfsetdash{}{0pt}%
\pgfpathmoveto{\pgfqpoint{4.693070in}{2.675315in}}%
\pgfpathlineto{\pgfqpoint{4.706493in}{2.672202in}}%
\pgfpathlineto{\pgfqpoint{4.719921in}{2.669115in}}%
\pgfpathlineto{\pgfqpoint{4.733357in}{2.666055in}}%
\pgfpathlineto{\pgfqpoint{4.746798in}{2.663022in}}%
\pgfpathlineto{\pgfqpoint{4.739367in}{2.655682in}}%
\pgfpathlineto{\pgfqpoint{4.731930in}{2.648375in}}%
\pgfpathlineto{\pgfqpoint{4.724488in}{2.641097in}}%
\pgfpathlineto{\pgfqpoint{4.717041in}{2.633847in}}%
\pgfpathlineto{\pgfqpoint{4.703585in}{2.636778in}}%
\pgfpathlineto{\pgfqpoint{4.690136in}{2.639736in}}%
\pgfpathlineto{\pgfqpoint{4.676693in}{2.642721in}}%
\pgfpathlineto{\pgfqpoint{4.663257in}{2.645733in}}%
\pgfpathlineto{\pgfqpoint{4.670718in}{2.653081in}}%
\pgfpathlineto{\pgfqpoint{4.678174in}{2.660458in}}%
\pgfpathlineto{\pgfqpoint{4.685624in}{2.667869in}}%
\pgfpathlineto{\pgfqpoint{4.693070in}{2.675315in}}%
\pgfpathclose%
\pgfusepath{fill}%
\end{pgfscope}%
\begin{pgfscope}%
\pgfpathrectangle{\pgfqpoint{1.150000in}{0.150000in}}{\pgfqpoint{5.700000in}{5.700000in}}%
\pgfusepath{clip}%
\pgfsetbuttcap%
\pgfsetroundjoin%
\definecolor{currentfill}{rgb}{0.272594,0.025563,0.353093}%
\pgfsetfillcolor{currentfill}%
\pgfsetfillopacity{0.700000}%
\pgfsetlinewidth{0.000000pt}%
\definecolor{currentstroke}{rgb}{0.000000,0.000000,0.000000}%
\pgfsetstrokecolor{currentstroke}%
\pgfsetdash{}{0pt}%
\pgfpathmoveto{\pgfqpoint{3.262817in}{2.637066in}}%
\pgfpathlineto{\pgfqpoint{3.275947in}{2.631663in}}%
\pgfpathlineto{\pgfqpoint{3.289080in}{2.626302in}}%
\pgfpathlineto{\pgfqpoint{3.302218in}{2.620983in}}%
\pgfpathlineto{\pgfqpoint{3.315360in}{2.615707in}}%
\pgfpathlineto{\pgfqpoint{3.307402in}{2.608686in}}%
\pgfpathlineto{\pgfqpoint{3.299439in}{2.601702in}}%
\pgfpathlineto{\pgfqpoint{3.291468in}{2.594757in}}%
\pgfpathlineto{\pgfqpoint{3.283491in}{2.587850in}}%
\pgfpathlineto{\pgfqpoint{3.270335in}{2.593182in}}%
\pgfpathlineto{\pgfqpoint{3.257183in}{2.598557in}}%
\pgfpathlineto{\pgfqpoint{3.244036in}{2.603973in}}%
\pgfpathlineto{\pgfqpoint{3.230893in}{2.609432in}}%
\pgfpathlineto{\pgfqpoint{3.238884in}{2.616278in}}%
\pgfpathlineto{\pgfqpoint{3.246868in}{2.623167in}}%
\pgfpathlineto{\pgfqpoint{3.254846in}{2.630096in}}%
\pgfpathlineto{\pgfqpoint{3.262817in}{2.637066in}}%
\pgfpathclose%
\pgfusepath{fill}%
\end{pgfscope}%
\begin{pgfscope}%
\pgfpathrectangle{\pgfqpoint{1.150000in}{0.150000in}}{\pgfqpoint{5.700000in}{5.700000in}}%
\pgfusepath{clip}%
\pgfsetbuttcap%
\pgfsetroundjoin%
\definecolor{currentfill}{rgb}{0.274952,0.037752,0.364543}%
\pgfsetfillcolor{currentfill}%
\pgfsetfillopacity{0.700000}%
\pgfsetlinewidth{0.000000pt}%
\definecolor{currentstroke}{rgb}{0.000000,0.000000,0.000000}%
\pgfsetstrokecolor{currentstroke}%
\pgfsetdash{}{0pt}%
\pgfpathmoveto{\pgfqpoint{4.472475in}{2.653789in}}%
\pgfpathlineto{\pgfqpoint{4.485847in}{2.650535in}}%
\pgfpathlineto{\pgfqpoint{4.499225in}{2.647309in}}%
\pgfpathlineto{\pgfqpoint{4.512610in}{2.644112in}}%
\pgfpathlineto{\pgfqpoint{4.526001in}{2.640942in}}%
\pgfpathlineto{\pgfqpoint{4.518487in}{2.633534in}}%
\pgfpathlineto{\pgfqpoint{4.510968in}{2.626145in}}%
\pgfpathlineto{\pgfqpoint{4.503444in}{2.618771in}}%
\pgfpathlineto{\pgfqpoint{4.495915in}{2.611410in}}%
\pgfpathlineto{\pgfqpoint{4.482511in}{2.614504in}}%
\pgfpathlineto{\pgfqpoint{4.469113in}{2.617626in}}%
\pgfpathlineto{\pgfqpoint{4.455721in}{2.620777in}}%
\pgfpathlineto{\pgfqpoint{4.442336in}{2.623955in}}%
\pgfpathlineto{\pgfqpoint{4.449879in}{2.631387in}}%
\pgfpathlineto{\pgfqpoint{4.457416in}{2.638835in}}%
\pgfpathlineto{\pgfqpoint{4.464948in}{2.646301in}}%
\pgfpathlineto{\pgfqpoint{4.472475in}{2.653789in}}%
\pgfpathclose%
\pgfusepath{fill}%
\end{pgfscope}%
\begin{pgfscope}%
\pgfpathrectangle{\pgfqpoint{1.150000in}{0.150000in}}{\pgfqpoint{5.700000in}{5.700000in}}%
\pgfusepath{clip}%
\pgfsetbuttcap%
\pgfsetroundjoin%
\definecolor{currentfill}{rgb}{0.274952,0.037752,0.364543}%
\pgfsetfillcolor{currentfill}%
\pgfsetfillopacity{0.700000}%
\pgfsetlinewidth{0.000000pt}%
\definecolor{currentstroke}{rgb}{0.000000,0.000000,0.000000}%
\pgfsetstrokecolor{currentstroke}%
\pgfsetdash{}{0pt}%
\pgfpathmoveto{\pgfqpoint{3.125893in}{2.654690in}}%
\pgfpathlineto{\pgfqpoint{3.139004in}{2.648875in}}%
\pgfpathlineto{\pgfqpoint{3.152119in}{2.643106in}}%
\pgfpathlineto{\pgfqpoint{3.165238in}{2.637382in}}%
\pgfpathlineto{\pgfqpoint{3.178361in}{2.631704in}}%
\pgfpathlineto{\pgfqpoint{3.170348in}{2.624966in}}%
\pgfpathlineto{\pgfqpoint{3.162328in}{2.618275in}}%
\pgfpathlineto{\pgfqpoint{3.154302in}{2.611634in}}%
\pgfpathlineto{\pgfqpoint{3.146268in}{2.605043in}}%
\pgfpathlineto{\pgfqpoint{3.133131in}{2.610791in}}%
\pgfpathlineto{\pgfqpoint{3.119997in}{2.616583in}}%
\pgfpathlineto{\pgfqpoint{3.106867in}{2.622421in}}%
\pgfpathlineto{\pgfqpoint{3.093741in}{2.628305in}}%
\pgfpathlineto{\pgfqpoint{3.101789in}{2.634822in}}%
\pgfpathlineto{\pgfqpoint{3.109831in}{2.641393in}}%
\pgfpathlineto{\pgfqpoint{3.117865in}{2.648015in}}%
\pgfpathlineto{\pgfqpoint{3.125893in}{2.654690in}}%
\pgfpathclose%
\pgfusepath{fill}%
\end{pgfscope}%
\begin{pgfscope}%
\pgfpathrectangle{\pgfqpoint{1.150000in}{0.150000in}}{\pgfqpoint{5.700000in}{5.700000in}}%
\pgfusepath{clip}%
\pgfsetbuttcap%
\pgfsetroundjoin%
\definecolor{currentfill}{rgb}{0.269944,0.014625,0.341379}%
\pgfsetfillcolor{currentfill}%
\pgfsetfillopacity{0.700000}%
\pgfsetlinewidth{0.000000pt}%
\definecolor{currentstroke}{rgb}{0.000000,0.000000,0.000000}%
\pgfsetstrokecolor{currentstroke}%
\pgfsetdash{}{0pt}%
\pgfpathmoveto{\pgfqpoint{3.894212in}{2.617310in}}%
\pgfpathlineto{\pgfqpoint{3.907455in}{2.613333in}}%
\pgfpathlineto{\pgfqpoint{3.920704in}{2.609388in}}%
\pgfpathlineto{\pgfqpoint{3.933959in}{2.605476in}}%
\pgfpathlineto{\pgfqpoint{3.947219in}{2.601596in}}%
\pgfpathlineto{\pgfqpoint{3.939496in}{2.594024in}}%
\pgfpathlineto{\pgfqpoint{3.931767in}{2.586461in}}%
\pgfpathlineto{\pgfqpoint{3.924033in}{2.578904in}}%
\pgfpathlineto{\pgfqpoint{3.916293in}{2.571353in}}%
\pgfpathlineto{\pgfqpoint{3.903021in}{2.575223in}}%
\pgfpathlineto{\pgfqpoint{3.889754in}{2.579125in}}%
\pgfpathlineto{\pgfqpoint{3.876492in}{2.583060in}}%
\pgfpathlineto{\pgfqpoint{3.863237in}{2.587027in}}%
\pgfpathlineto{\pgfqpoint{3.870989in}{2.594583in}}%
\pgfpathlineto{\pgfqpoint{3.878736in}{2.602148in}}%
\pgfpathlineto{\pgfqpoint{3.886477in}{2.609724in}}%
\pgfpathlineto{\pgfqpoint{3.894212in}{2.617310in}}%
\pgfpathclose%
\pgfusepath{fill}%
\end{pgfscope}%
\begin{pgfscope}%
\pgfpathrectangle{\pgfqpoint{1.150000in}{0.150000in}}{\pgfqpoint{5.700000in}{5.700000in}}%
\pgfusepath{clip}%
\pgfsetbuttcap%
\pgfsetroundjoin%
\definecolor{currentfill}{rgb}{0.271305,0.019942,0.347269}%
\pgfsetfillcolor{currentfill}%
\pgfsetfillopacity{0.700000}%
\pgfsetlinewidth{0.000000pt}%
\definecolor{currentstroke}{rgb}{0.000000,0.000000,0.000000}%
\pgfsetstrokecolor{currentstroke}%
\pgfsetdash{}{0pt}%
\pgfpathmoveto{\pgfqpoint{3.399682in}{2.623622in}}%
\pgfpathlineto{\pgfqpoint{3.412833in}{2.618593in}}%
\pgfpathlineto{\pgfqpoint{3.425989in}{2.613603in}}%
\pgfpathlineto{\pgfqpoint{3.439149in}{2.608653in}}%
\pgfpathlineto{\pgfqpoint{3.452314in}{2.603742in}}%
\pgfpathlineto{\pgfqpoint{3.444409in}{2.596505in}}%
\pgfpathlineto{\pgfqpoint{3.436498in}{2.589296in}}%
\pgfpathlineto{\pgfqpoint{3.428581in}{2.582115in}}%
\pgfpathlineto{\pgfqpoint{3.420657in}{2.574963in}}%
\pgfpathlineto{\pgfqpoint{3.407479in}{2.579916in}}%
\pgfpathlineto{\pgfqpoint{3.394305in}{2.584909in}}%
\pgfpathlineto{\pgfqpoint{3.381137in}{2.589941in}}%
\pgfpathlineto{\pgfqpoint{3.367972in}{2.595013in}}%
\pgfpathlineto{\pgfqpoint{3.375909in}{2.602118in}}%
\pgfpathlineto{\pgfqpoint{3.383840in}{2.609254in}}%
\pgfpathlineto{\pgfqpoint{3.391764in}{2.616423in}}%
\pgfpathlineto{\pgfqpoint{3.399682in}{2.623622in}}%
\pgfpathclose%
\pgfusepath{fill}%
\end{pgfscope}%
\begin{pgfscope}%
\pgfpathrectangle{\pgfqpoint{1.150000in}{0.150000in}}{\pgfqpoint{5.700000in}{5.700000in}}%
\pgfusepath{clip}%
\pgfsetbuttcap%
\pgfsetroundjoin%
\definecolor{currentfill}{rgb}{0.281446,0.084320,0.407414}%
\pgfsetfillcolor{currentfill}%
\pgfsetfillopacity{0.700000}%
\pgfsetlinewidth{0.000000pt}%
\definecolor{currentstroke}{rgb}{0.000000,0.000000,0.000000}%
\pgfsetstrokecolor{currentstroke}%
\pgfsetdash{}{0pt}%
\pgfpathmoveto{\pgfqpoint{5.271535in}{2.728750in}}%
\pgfpathlineto{\pgfqpoint{5.285096in}{2.725756in}}%
\pgfpathlineto{\pgfqpoint{5.298665in}{2.722788in}}%
\pgfpathlineto{\pgfqpoint{5.312240in}{2.719844in}}%
\pgfpathlineto{\pgfqpoint{5.325822in}{2.716924in}}%
\pgfpathlineto{\pgfqpoint{5.318595in}{2.709438in}}%
\pgfpathlineto{\pgfqpoint{5.311365in}{2.702046in}}%
\pgfpathlineto{\pgfqpoint{5.304132in}{2.694743in}}%
\pgfpathlineto{\pgfqpoint{5.296896in}{2.687526in}}%
\pgfpathlineto{\pgfqpoint{5.283297in}{2.690278in}}%
\pgfpathlineto{\pgfqpoint{5.269704in}{2.693055in}}%
\pgfpathlineto{\pgfqpoint{5.256119in}{2.695856in}}%
\pgfpathlineto{\pgfqpoint{5.242541in}{2.698683in}}%
\pgfpathlineto{\pgfqpoint{5.249794in}{2.706063in}}%
\pgfpathlineto{\pgfqpoint{5.257044in}{2.713530in}}%
\pgfpathlineto{\pgfqpoint{5.264291in}{2.721091in}}%
\pgfpathlineto{\pgfqpoint{5.271535in}{2.728750in}}%
\pgfpathclose%
\pgfusepath{fill}%
\end{pgfscope}%
\begin{pgfscope}%
\pgfpathrectangle{\pgfqpoint{1.150000in}{0.150000in}}{\pgfqpoint{5.700000in}{5.700000in}}%
\pgfusepath{clip}%
\pgfsetbuttcap%
\pgfsetroundjoin%
\definecolor{currentfill}{rgb}{0.277018,0.050344,0.375715}%
\pgfsetfillcolor{currentfill}%
\pgfsetfillopacity{0.700000}%
\pgfsetlinewidth{0.000000pt}%
\definecolor{currentstroke}{rgb}{0.000000,0.000000,0.000000}%
\pgfsetstrokecolor{currentstroke}%
\pgfsetdash{}{0pt}%
\pgfpathmoveto{\pgfqpoint{2.988864in}{2.677084in}}%
\pgfpathlineto{\pgfqpoint{3.001961in}{2.670816in}}%
\pgfpathlineto{\pgfqpoint{3.015062in}{2.664598in}}%
\pgfpathlineto{\pgfqpoint{3.028166in}{2.658429in}}%
\pgfpathlineto{\pgfqpoint{3.041274in}{2.652309in}}%
\pgfpathlineto{\pgfqpoint{3.033202in}{2.645926in}}%
\pgfpathlineto{\pgfqpoint{3.025123in}{2.639602in}}%
\pgfpathlineto{\pgfqpoint{3.017037in}{2.633341in}}%
\pgfpathlineto{\pgfqpoint{3.008943in}{2.627143in}}%
\pgfpathlineto{\pgfqpoint{2.995819in}{2.633345in}}%
\pgfpathlineto{\pgfqpoint{2.982699in}{2.639597in}}%
\pgfpathlineto{\pgfqpoint{2.969582in}{2.645897in}}%
\pgfpathlineto{\pgfqpoint{2.956469in}{2.652247in}}%
\pgfpathlineto{\pgfqpoint{2.964579in}{2.658358in}}%
\pgfpathlineto{\pgfqpoint{2.972682in}{2.664535in}}%
\pgfpathlineto{\pgfqpoint{2.980777in}{2.670778in}}%
\pgfpathlineto{\pgfqpoint{2.988864in}{2.677084in}}%
\pgfpathclose%
\pgfusepath{fill}%
\end{pgfscope}%
\begin{pgfscope}%
\pgfpathrectangle{\pgfqpoint{1.150000in}{0.150000in}}{\pgfqpoint{5.700000in}{5.700000in}}%
\pgfusepath{clip}%
\pgfsetbuttcap%
\pgfsetroundjoin%
\definecolor{currentfill}{rgb}{0.272594,0.025563,0.353093}%
\pgfsetfillcolor{currentfill}%
\pgfsetfillopacity{0.700000}%
\pgfsetlinewidth{0.000000pt}%
\definecolor{currentstroke}{rgb}{0.000000,0.000000,0.000000}%
\pgfsetstrokecolor{currentstroke}%
\pgfsetdash{}{0pt}%
\pgfpathmoveto{\pgfqpoint{4.251827in}{2.633967in}}%
\pgfpathlineto{\pgfqpoint{4.265150in}{2.630505in}}%
\pgfpathlineto{\pgfqpoint{4.278479in}{2.627072in}}%
\pgfpathlineto{\pgfqpoint{4.291814in}{2.623668in}}%
\pgfpathlineto{\pgfqpoint{4.305156in}{2.620294in}}%
\pgfpathlineto{\pgfqpoint{4.297561in}{2.612798in}}%
\pgfpathlineto{\pgfqpoint{4.289960in}{2.605312in}}%
\pgfpathlineto{\pgfqpoint{4.282354in}{2.597832in}}%
\pgfpathlineto{\pgfqpoint{4.274743in}{2.590358in}}%
\pgfpathlineto{\pgfqpoint{4.261389in}{2.593683in}}%
\pgfpathlineto{\pgfqpoint{4.248041in}{2.597037in}}%
\pgfpathlineto{\pgfqpoint{4.234699in}{2.600420in}}%
\pgfpathlineto{\pgfqpoint{4.221363in}{2.603834in}}%
\pgfpathlineto{\pgfqpoint{4.228987in}{2.611352in}}%
\pgfpathlineto{\pgfqpoint{4.236606in}{2.618880in}}%
\pgfpathlineto{\pgfqpoint{4.244219in}{2.626417in}}%
\pgfpathlineto{\pgfqpoint{4.251827in}{2.633967in}}%
\pgfpathclose%
\pgfusepath{fill}%
\end{pgfscope}%
\begin{pgfscope}%
\pgfpathrectangle{\pgfqpoint{1.150000in}{0.150000in}}{\pgfqpoint{5.700000in}{5.700000in}}%
\pgfusepath{clip}%
\pgfsetbuttcap%
\pgfsetroundjoin%
\definecolor{currentfill}{rgb}{0.269944,0.014625,0.341379}%
\pgfsetfillcolor{currentfill}%
\pgfsetfillopacity{0.700000}%
\pgfsetlinewidth{0.000000pt}%
\definecolor{currentstroke}{rgb}{0.000000,0.000000,0.000000}%
\pgfsetstrokecolor{currentstroke}%
\pgfsetdash{}{0pt}%
\pgfpathmoveto{\pgfqpoint{3.536526in}{2.613820in}}%
\pgfpathlineto{\pgfqpoint{3.549701in}{2.609130in}}%
\pgfpathlineto{\pgfqpoint{3.562882in}{2.604478in}}%
\pgfpathlineto{\pgfqpoint{3.576068in}{2.599863in}}%
\pgfpathlineto{\pgfqpoint{3.589258in}{2.595285in}}%
\pgfpathlineto{\pgfqpoint{3.581404in}{2.587892in}}%
\pgfpathlineto{\pgfqpoint{3.573543in}{2.580519in}}%
\pgfpathlineto{\pgfqpoint{3.565677in}{2.573166in}}%
\pgfpathlineto{\pgfqpoint{3.557805in}{2.565833in}}%
\pgfpathlineto{\pgfqpoint{3.544601in}{2.570441in}}%
\pgfpathlineto{\pgfqpoint{3.531403in}{2.575085in}}%
\pgfpathlineto{\pgfqpoint{3.518209in}{2.579766in}}%
\pgfpathlineto{\pgfqpoint{3.505020in}{2.584485in}}%
\pgfpathlineto{\pgfqpoint{3.512906in}{2.591784in}}%
\pgfpathlineto{\pgfqpoint{3.520785in}{2.599106in}}%
\pgfpathlineto{\pgfqpoint{3.528658in}{2.606451in}}%
\pgfpathlineto{\pgfqpoint{3.536526in}{2.613820in}}%
\pgfpathclose%
\pgfusepath{fill}%
\end{pgfscope}%
\begin{pgfscope}%
\pgfpathrectangle{\pgfqpoint{1.150000in}{0.150000in}}{\pgfqpoint{5.700000in}{5.700000in}}%
\pgfusepath{clip}%
\pgfsetbuttcap%
\pgfsetroundjoin%
\definecolor{currentfill}{rgb}{0.280267,0.073417,0.397163}%
\pgfsetfillcolor{currentfill}%
\pgfsetfillopacity{0.700000}%
\pgfsetlinewidth{0.000000pt}%
\definecolor{currentstroke}{rgb}{0.000000,0.000000,0.000000}%
\pgfsetstrokecolor{currentstroke}%
\pgfsetdash{}{0pt}%
\pgfpathmoveto{\pgfqpoint{5.050901in}{2.703958in}}%
\pgfpathlineto{\pgfqpoint{5.064412in}{2.700984in}}%
\pgfpathlineto{\pgfqpoint{5.077931in}{2.698036in}}%
\pgfpathlineto{\pgfqpoint{5.091456in}{2.695113in}}%
\pgfpathlineto{\pgfqpoint{5.104988in}{2.692215in}}%
\pgfpathlineto{\pgfqpoint{5.097684in}{2.684907in}}%
\pgfpathlineto{\pgfqpoint{5.090377in}{2.677664in}}%
\pgfpathlineto{\pgfqpoint{5.083065in}{2.670482in}}%
\pgfpathlineto{\pgfqpoint{5.075749in}{2.663356in}}%
\pgfpathlineto{\pgfqpoint{5.062201in}{2.666112in}}%
\pgfpathlineto{\pgfqpoint{5.048661in}{2.668894in}}%
\pgfpathlineto{\pgfqpoint{5.035127in}{2.671701in}}%
\pgfpathlineto{\pgfqpoint{5.021600in}{2.674534in}}%
\pgfpathlineto{\pgfqpoint{5.028931in}{2.681796in}}%
\pgfpathlineto{\pgfqpoint{5.036258in}{2.689118in}}%
\pgfpathlineto{\pgfqpoint{5.043582in}{2.696504in}}%
\pgfpathlineto{\pgfqpoint{5.050901in}{2.703958in}}%
\pgfpathclose%
\pgfusepath{fill}%
\end{pgfscope}%
\begin{pgfscope}%
\pgfpathrectangle{\pgfqpoint{1.150000in}{0.150000in}}{\pgfqpoint{5.700000in}{5.700000in}}%
\pgfusepath{clip}%
\pgfsetbuttcap%
\pgfsetroundjoin%
\definecolor{currentfill}{rgb}{0.277941,0.056324,0.381191}%
\pgfsetfillcolor{currentfill}%
\pgfsetfillopacity{0.700000}%
\pgfsetlinewidth{0.000000pt}%
\definecolor{currentstroke}{rgb}{0.000000,0.000000,0.000000}%
\pgfsetstrokecolor{currentstroke}%
\pgfsetdash{}{0pt}%
\pgfpathmoveto{\pgfqpoint{4.830255in}{2.680458in}}%
\pgfpathlineto{\pgfqpoint{4.843716in}{2.677443in}}%
\pgfpathlineto{\pgfqpoint{4.857183in}{2.674454in}}%
\pgfpathlineto{\pgfqpoint{4.870657in}{2.671491in}}%
\pgfpathlineto{\pgfqpoint{4.884138in}{2.668554in}}%
\pgfpathlineto{\pgfqpoint{4.876754in}{2.661285in}}%
\pgfpathlineto{\pgfqpoint{4.869365in}{2.654057in}}%
\pgfpathlineto{\pgfqpoint{4.861972in}{2.646867in}}%
\pgfpathlineto{\pgfqpoint{4.854574in}{2.639711in}}%
\pgfpathlineto{\pgfqpoint{4.841078in}{2.642532in}}%
\pgfpathlineto{\pgfqpoint{4.827590in}{2.645380in}}%
\pgfpathlineto{\pgfqpoint{4.814108in}{2.648254in}}%
\pgfpathlineto{\pgfqpoint{4.800632in}{2.651155in}}%
\pgfpathlineto{\pgfqpoint{4.808045in}{2.658422in}}%
\pgfpathlineto{\pgfqpoint{4.815453in}{2.665725in}}%
\pgfpathlineto{\pgfqpoint{4.822856in}{2.673069in}}%
\pgfpathlineto{\pgfqpoint{4.830255in}{2.680458in}}%
\pgfpathclose%
\pgfusepath{fill}%
\end{pgfscope}%
\begin{pgfscope}%
\pgfpathrectangle{\pgfqpoint{1.150000in}{0.150000in}}{\pgfqpoint{5.700000in}{5.700000in}}%
\pgfusepath{clip}%
\pgfsetbuttcap%
\pgfsetroundjoin%
\definecolor{currentfill}{rgb}{0.271305,0.019942,0.347269}%
\pgfsetfillcolor{currentfill}%
\pgfsetfillopacity{0.700000}%
\pgfsetlinewidth{0.000000pt}%
\definecolor{currentstroke}{rgb}{0.000000,0.000000,0.000000}%
\pgfsetstrokecolor{currentstroke}%
\pgfsetdash{}{0pt}%
\pgfpathmoveto{\pgfqpoint{4.031104in}{2.616692in}}%
\pgfpathlineto{\pgfqpoint{4.044381in}{2.612948in}}%
\pgfpathlineto{\pgfqpoint{4.057663in}{2.609236in}}%
\pgfpathlineto{\pgfqpoint{4.070952in}{2.605556in}}%
\pgfpathlineto{\pgfqpoint{4.084246in}{2.601906in}}%
\pgfpathlineto{\pgfqpoint{4.076569in}{2.594348in}}%
\pgfpathlineto{\pgfqpoint{4.068888in}{2.586797in}}%
\pgfpathlineto{\pgfqpoint{4.061200in}{2.579249in}}%
\pgfpathlineto{\pgfqpoint{4.053507in}{2.571705in}}%
\pgfpathlineto{\pgfqpoint{4.040201in}{2.575332in}}%
\pgfpathlineto{\pgfqpoint{4.026900in}{2.578989in}}%
\pgfpathlineto{\pgfqpoint{4.013606in}{2.582678in}}%
\pgfpathlineto{\pgfqpoint{4.000317in}{2.586398in}}%
\pgfpathlineto{\pgfqpoint{4.008022in}{2.593960in}}%
\pgfpathlineto{\pgfqpoint{4.015722in}{2.601529in}}%
\pgfpathlineto{\pgfqpoint{4.023416in}{2.609106in}}%
\pgfpathlineto{\pgfqpoint{4.031104in}{2.616692in}}%
\pgfpathclose%
\pgfusepath{fill}%
\end{pgfscope}%
\begin{pgfscope}%
\pgfpathrectangle{\pgfqpoint{1.150000in}{0.150000in}}{\pgfqpoint{5.700000in}{5.700000in}}%
\pgfusepath{clip}%
\pgfsetbuttcap%
\pgfsetroundjoin%
\definecolor{currentfill}{rgb}{0.269944,0.014625,0.341379}%
\pgfsetfillcolor{currentfill}%
\pgfsetfillopacity{0.700000}%
\pgfsetlinewidth{0.000000pt}%
\definecolor{currentstroke}{rgb}{0.000000,0.000000,0.000000}%
\pgfsetstrokecolor{currentstroke}%
\pgfsetdash{}{0pt}%
\pgfpathmoveto{\pgfqpoint{3.673379in}{2.607170in}}%
\pgfpathlineto{\pgfqpoint{3.686583in}{2.602790in}}%
\pgfpathlineto{\pgfqpoint{3.699791in}{2.598444in}}%
\pgfpathlineto{\pgfqpoint{3.713005in}{2.594134in}}%
\pgfpathlineto{\pgfqpoint{3.726224in}{2.589858in}}%
\pgfpathlineto{\pgfqpoint{3.718419in}{2.582364in}}%
\pgfpathlineto{\pgfqpoint{3.710607in}{2.574883in}}%
\pgfpathlineto{\pgfqpoint{3.702790in}{2.567415in}}%
\pgfpathlineto{\pgfqpoint{3.694967in}{2.559961in}}%
\pgfpathlineto{\pgfqpoint{3.681735in}{2.564252in}}%
\pgfpathlineto{\pgfqpoint{3.668509in}{2.568579in}}%
\pgfpathlineto{\pgfqpoint{3.655287in}{2.572940in}}%
\pgfpathlineto{\pgfqpoint{3.642071in}{2.577338in}}%
\pgfpathlineto{\pgfqpoint{3.649907in}{2.584771in}}%
\pgfpathlineto{\pgfqpoint{3.657737in}{2.592221in}}%
\pgfpathlineto{\pgfqpoint{3.665561in}{2.599688in}}%
\pgfpathlineto{\pgfqpoint{3.673379in}{2.607170in}}%
\pgfpathclose%
\pgfusepath{fill}%
\end{pgfscope}%
\begin{pgfscope}%
\pgfpathrectangle{\pgfqpoint{1.150000in}{0.150000in}}{\pgfqpoint{5.700000in}{5.700000in}}%
\pgfusepath{clip}%
\pgfsetbuttcap%
\pgfsetroundjoin%
\definecolor{currentfill}{rgb}{0.276022,0.044167,0.370164}%
\pgfsetfillcolor{currentfill}%
\pgfsetfillopacity{0.700000}%
\pgfsetlinewidth{0.000000pt}%
\definecolor{currentstroke}{rgb}{0.000000,0.000000,0.000000}%
\pgfsetstrokecolor{currentstroke}%
\pgfsetdash{}{0pt}%
\pgfpathmoveto{\pgfqpoint{4.609578in}{2.658052in}}%
\pgfpathlineto{\pgfqpoint{4.622988in}{2.654931in}}%
\pgfpathlineto{\pgfqpoint{4.636405in}{2.651838in}}%
\pgfpathlineto{\pgfqpoint{4.649828in}{2.648772in}}%
\pgfpathlineto{\pgfqpoint{4.663257in}{2.645733in}}%
\pgfpathlineto{\pgfqpoint{4.655791in}{2.638412in}}%
\pgfpathlineto{\pgfqpoint{4.648320in}{2.631113in}}%
\pgfpathlineto{\pgfqpoint{4.640844in}{2.623836in}}%
\pgfpathlineto{\pgfqpoint{4.633363in}{2.616575in}}%
\pgfpathlineto{\pgfqpoint{4.619920in}{2.619525in}}%
\pgfpathlineto{\pgfqpoint{4.606483in}{2.622503in}}%
\pgfpathlineto{\pgfqpoint{4.593053in}{2.625507in}}%
\pgfpathlineto{\pgfqpoint{4.579630in}{2.628539in}}%
\pgfpathlineto{\pgfqpoint{4.587125in}{2.635883in}}%
\pgfpathlineto{\pgfqpoint{4.594614in}{2.643248in}}%
\pgfpathlineto{\pgfqpoint{4.602099in}{2.650637in}}%
\pgfpathlineto{\pgfqpoint{4.609578in}{2.658052in}}%
\pgfpathclose%
\pgfusepath{fill}%
\end{pgfscope}%
\begin{pgfscope}%
\pgfpathrectangle{\pgfqpoint{1.150000in}{0.150000in}}{\pgfqpoint{5.700000in}{5.700000in}}%
\pgfusepath{clip}%
\pgfsetbuttcap%
\pgfsetroundjoin%
\definecolor{currentfill}{rgb}{0.273809,0.031497,0.358853}%
\pgfsetfillcolor{currentfill}%
\pgfsetfillopacity{0.700000}%
\pgfsetlinewidth{0.000000pt}%
\definecolor{currentstroke}{rgb}{0.000000,0.000000,0.000000}%
\pgfsetstrokecolor{currentstroke}%
\pgfsetdash{}{0pt}%
\pgfpathmoveto{\pgfqpoint{4.388859in}{2.636952in}}%
\pgfpathlineto{\pgfqpoint{4.402219in}{2.633660in}}%
\pgfpathlineto{\pgfqpoint{4.415585in}{2.630397in}}%
\pgfpathlineto{\pgfqpoint{4.428957in}{2.627162in}}%
\pgfpathlineto{\pgfqpoint{4.442336in}{2.623955in}}%
\pgfpathlineto{\pgfqpoint{4.434788in}{2.616537in}}%
\pgfpathlineto{\pgfqpoint{4.427235in}{2.609130in}}%
\pgfpathlineto{\pgfqpoint{4.419676in}{2.601732in}}%
\pgfpathlineto{\pgfqpoint{4.412112in}{2.594341in}}%
\pgfpathlineto{\pgfqpoint{4.398721in}{2.597485in}}%
\pgfpathlineto{\pgfqpoint{4.385335in}{2.600657in}}%
\pgfpathlineto{\pgfqpoint{4.371956in}{2.603858in}}%
\pgfpathlineto{\pgfqpoint{4.358584in}{2.607087in}}%
\pgfpathlineto{\pgfqpoint{4.366161in}{2.614536in}}%
\pgfpathlineto{\pgfqpoint{4.373732in}{2.621995in}}%
\pgfpathlineto{\pgfqpoint{4.381298in}{2.629467in}}%
\pgfpathlineto{\pgfqpoint{4.388859in}{2.636952in}}%
\pgfpathclose%
\pgfusepath{fill}%
\end{pgfscope}%
\begin{pgfscope}%
\pgfpathrectangle{\pgfqpoint{1.150000in}{0.150000in}}{\pgfqpoint{5.700000in}{5.700000in}}%
\pgfusepath{clip}%
\pgfsetbuttcap%
\pgfsetroundjoin%
\definecolor{currentfill}{rgb}{0.269944,0.014625,0.341379}%
\pgfsetfillcolor{currentfill}%
\pgfsetfillopacity{0.700000}%
\pgfsetlinewidth{0.000000pt}%
\definecolor{currentstroke}{rgb}{0.000000,0.000000,0.000000}%
\pgfsetstrokecolor{currentstroke}%
\pgfsetdash{}{0pt}%
\pgfpathmoveto{\pgfqpoint{3.810270in}{2.603230in}}%
\pgfpathlineto{\pgfqpoint{3.823503in}{2.599129in}}%
\pgfpathlineto{\pgfqpoint{3.836742in}{2.595062in}}%
\pgfpathlineto{\pgfqpoint{3.849987in}{2.591028in}}%
\pgfpathlineto{\pgfqpoint{3.863237in}{2.587027in}}%
\pgfpathlineto{\pgfqpoint{3.855479in}{2.579480in}}%
\pgfpathlineto{\pgfqpoint{3.847715in}{2.571942in}}%
\pgfpathlineto{\pgfqpoint{3.839946in}{2.564411in}}%
\pgfpathlineto{\pgfqpoint{3.832171in}{2.556887in}}%
\pgfpathlineto{\pgfqpoint{3.818909in}{2.560891in}}%
\pgfpathlineto{\pgfqpoint{3.805652in}{2.564928in}}%
\pgfpathlineto{\pgfqpoint{3.792400in}{2.568998in}}%
\pgfpathlineto{\pgfqpoint{3.779154in}{2.573102in}}%
\pgfpathlineto{\pgfqpoint{3.786942in}{2.580618in}}%
\pgfpathlineto{\pgfqpoint{3.794723in}{2.588144in}}%
\pgfpathlineto{\pgfqpoint{3.802499in}{2.595681in}}%
\pgfpathlineto{\pgfqpoint{3.810270in}{2.603230in}}%
\pgfpathclose%
\pgfusepath{fill}%
\end{pgfscope}%
\begin{pgfscope}%
\pgfpathrectangle{\pgfqpoint{1.150000in}{0.150000in}}{\pgfqpoint{5.700000in}{5.700000in}}%
\pgfusepath{clip}%
\pgfsetbuttcap%
\pgfsetroundjoin%
\definecolor{currentfill}{rgb}{0.280894,0.078907,0.402329}%
\pgfsetfillcolor{currentfill}%
\pgfsetfillopacity{0.700000}%
\pgfsetlinewidth{0.000000pt}%
\definecolor{currentstroke}{rgb}{0.000000,0.000000,0.000000}%
\pgfsetstrokecolor{currentstroke}%
\pgfsetdash{}{0pt}%
\pgfpathmoveto{\pgfqpoint{5.188298in}{2.710237in}}%
\pgfpathlineto{\pgfqpoint{5.201848in}{2.707311in}}%
\pgfpathlineto{\pgfqpoint{5.215406in}{2.704410in}}%
\pgfpathlineto{\pgfqpoint{5.228970in}{2.701534in}}%
\pgfpathlineto{\pgfqpoint{5.242541in}{2.698683in}}%
\pgfpathlineto{\pgfqpoint{5.235285in}{2.691385in}}%
\pgfpathlineto{\pgfqpoint{5.228025in}{2.684165in}}%
\pgfpathlineto{\pgfqpoint{5.220761in}{2.677018in}}%
\pgfpathlineto{\pgfqpoint{5.213494in}{2.669940in}}%
\pgfpathlineto{\pgfqpoint{5.199906in}{2.672636in}}%
\pgfpathlineto{\pgfqpoint{5.186325in}{2.675358in}}%
\pgfpathlineto{\pgfqpoint{5.172752in}{2.678105in}}%
\pgfpathlineto{\pgfqpoint{5.159185in}{2.680877in}}%
\pgfpathlineto{\pgfqpoint{5.166469in}{2.688105in}}%
\pgfpathlineto{\pgfqpoint{5.173749in}{2.695405in}}%
\pgfpathlineto{\pgfqpoint{5.181025in}{2.702780in}}%
\pgfpathlineto{\pgfqpoint{5.188298in}{2.710237in}}%
\pgfpathclose%
\pgfusepath{fill}%
\end{pgfscope}%
\begin{pgfscope}%
\pgfpathrectangle{\pgfqpoint{1.150000in}{0.150000in}}{\pgfqpoint{5.700000in}{5.700000in}}%
\pgfusepath{clip}%
\pgfsetbuttcap%
\pgfsetroundjoin%
\definecolor{currentfill}{rgb}{0.272594,0.025563,0.353093}%
\pgfsetfillcolor{currentfill}%
\pgfsetfillopacity{0.700000}%
\pgfsetlinewidth{0.000000pt}%
\definecolor{currentstroke}{rgb}{0.000000,0.000000,0.000000}%
\pgfsetstrokecolor{currentstroke}%
\pgfsetdash{}{0pt}%
\pgfpathmoveto{\pgfqpoint{4.168082in}{2.617785in}}%
\pgfpathlineto{\pgfqpoint{4.181393in}{2.614252in}}%
\pgfpathlineto{\pgfqpoint{4.194710in}{2.610749in}}%
\pgfpathlineto{\pgfqpoint{4.208034in}{2.607276in}}%
\pgfpathlineto{\pgfqpoint{4.221363in}{2.603834in}}%
\pgfpathlineto{\pgfqpoint{4.213734in}{2.596321in}}%
\pgfpathlineto{\pgfqpoint{4.206099in}{2.588814in}}%
\pgfpathlineto{\pgfqpoint{4.198459in}{2.581311in}}%
\pgfpathlineto{\pgfqpoint{4.190813in}{2.573809in}}%
\pgfpathlineto{\pgfqpoint{4.177471in}{2.577216in}}%
\pgfpathlineto{\pgfqpoint{4.164135in}{2.580652in}}%
\pgfpathlineto{\pgfqpoint{4.150805in}{2.584118in}}%
\pgfpathlineto{\pgfqpoint{4.137481in}{2.587615in}}%
\pgfpathlineto{\pgfqpoint{4.145140in}{2.595148in}}%
\pgfpathlineto{\pgfqpoint{4.152792in}{2.602687in}}%
\pgfpathlineto{\pgfqpoint{4.160440in}{2.610232in}}%
\pgfpathlineto{\pgfqpoint{4.168082in}{2.617785in}}%
\pgfpathclose%
\pgfusepath{fill}%
\end{pgfscope}%
\begin{pgfscope}%
\pgfpathrectangle{\pgfqpoint{1.150000in}{0.150000in}}{\pgfqpoint{5.700000in}{5.700000in}}%
\pgfusepath{clip}%
\pgfsetbuttcap%
\pgfsetroundjoin%
\definecolor{currentfill}{rgb}{0.279566,0.067836,0.391917}%
\pgfsetfillcolor{currentfill}%
\pgfsetfillopacity{0.700000}%
\pgfsetlinewidth{0.000000pt}%
\definecolor{currentstroke}{rgb}{0.000000,0.000000,0.000000}%
\pgfsetstrokecolor{currentstroke}%
\pgfsetdash{}{0pt}%
\pgfpathmoveto{\pgfqpoint{4.967560in}{2.686121in}}%
\pgfpathlineto{\pgfqpoint{4.981060in}{2.683185in}}%
\pgfpathlineto{\pgfqpoint{4.994566in}{2.680276in}}%
\pgfpathlineto{\pgfqpoint{5.008079in}{2.677392in}}%
\pgfpathlineto{\pgfqpoint{5.021600in}{2.674534in}}%
\pgfpathlineto{\pgfqpoint{5.014264in}{2.667326in}}%
\pgfpathlineto{\pgfqpoint{5.006924in}{2.660170in}}%
\pgfpathlineto{\pgfqpoint{4.999579in}{2.653060in}}%
\pgfpathlineto{\pgfqpoint{4.992230in}{2.645993in}}%
\pgfpathlineto{\pgfqpoint{4.978694in}{2.648723in}}%
\pgfpathlineto{\pgfqpoint{4.965166in}{2.651478in}}%
\pgfpathlineto{\pgfqpoint{4.951644in}{2.654259in}}%
\pgfpathlineto{\pgfqpoint{4.938129in}{2.657066in}}%
\pgfpathlineto{\pgfqpoint{4.945494in}{2.664257in}}%
\pgfpathlineto{\pgfqpoint{4.952853in}{2.671494in}}%
\pgfpathlineto{\pgfqpoint{4.960209in}{2.678780in}}%
\pgfpathlineto{\pgfqpoint{4.967560in}{2.686121in}}%
\pgfpathclose%
\pgfusepath{fill}%
\end{pgfscope}%
\begin{pgfscope}%
\pgfpathrectangle{\pgfqpoint{1.150000in}{0.150000in}}{\pgfqpoint{5.700000in}{5.700000in}}%
\pgfusepath{clip}%
\pgfsetbuttcap%
\pgfsetroundjoin%
\definecolor{currentfill}{rgb}{0.273809,0.031497,0.358853}%
\pgfsetfillcolor{currentfill}%
\pgfsetfillopacity{0.700000}%
\pgfsetlinewidth{0.000000pt}%
\definecolor{currentstroke}{rgb}{0.000000,0.000000,0.000000}%
\pgfsetstrokecolor{currentstroke}%
\pgfsetdash{}{0pt}%
\pgfpathmoveto{\pgfqpoint{3.178361in}{2.631704in}}%
\pgfpathlineto{\pgfqpoint{3.191487in}{2.626070in}}%
\pgfpathlineto{\pgfqpoint{3.204618in}{2.620481in}}%
\pgfpathlineto{\pgfqpoint{3.217753in}{2.614935in}}%
\pgfpathlineto{\pgfqpoint{3.230893in}{2.609432in}}%
\pgfpathlineto{\pgfqpoint{3.222895in}{2.602630in}}%
\pgfpathlineto{\pgfqpoint{3.214890in}{2.595872in}}%
\pgfpathlineto{\pgfqpoint{3.206878in}{2.589160in}}%
\pgfpathlineto{\pgfqpoint{3.198860in}{2.582496in}}%
\pgfpathlineto{\pgfqpoint{3.185706in}{2.588067in}}%
\pgfpathlineto{\pgfqpoint{3.172556in}{2.593682in}}%
\pgfpathlineto{\pgfqpoint{3.159410in}{2.599340in}}%
\pgfpathlineto{\pgfqpoint{3.146268in}{2.605043in}}%
\pgfpathlineto{\pgfqpoint{3.154302in}{2.611634in}}%
\pgfpathlineto{\pgfqpoint{3.162328in}{2.618275in}}%
\pgfpathlineto{\pgfqpoint{3.170348in}{2.624966in}}%
\pgfpathlineto{\pgfqpoint{3.178361in}{2.631704in}}%
\pgfpathclose%
\pgfusepath{fill}%
\end{pgfscope}%
\begin{pgfscope}%
\pgfpathrectangle{\pgfqpoint{1.150000in}{0.150000in}}{\pgfqpoint{5.700000in}{5.700000in}}%
\pgfusepath{clip}%
\pgfsetbuttcap%
\pgfsetroundjoin%
\definecolor{currentfill}{rgb}{0.271305,0.019942,0.347269}%
\pgfsetfillcolor{currentfill}%
\pgfsetfillopacity{0.700000}%
\pgfsetlinewidth{0.000000pt}%
\definecolor{currentstroke}{rgb}{0.000000,0.000000,0.000000}%
\pgfsetstrokecolor{currentstroke}%
\pgfsetdash{}{0pt}%
\pgfpathmoveto{\pgfqpoint{3.315360in}{2.615707in}}%
\pgfpathlineto{\pgfqpoint{3.328506in}{2.610472in}}%
\pgfpathlineto{\pgfqpoint{3.341657in}{2.605278in}}%
\pgfpathlineto{\pgfqpoint{3.354812in}{2.600125in}}%
\pgfpathlineto{\pgfqpoint{3.367972in}{2.595013in}}%
\pgfpathlineto{\pgfqpoint{3.360029in}{2.587941in}}%
\pgfpathlineto{\pgfqpoint{3.352079in}{2.580903in}}%
\pgfpathlineto{\pgfqpoint{3.344123in}{2.573900in}}%
\pgfpathlineto{\pgfqpoint{3.336160in}{2.566933in}}%
\pgfpathlineto{\pgfqpoint{3.322986in}{2.572101in}}%
\pgfpathlineto{\pgfqpoint{3.309817in}{2.577310in}}%
\pgfpathlineto{\pgfqpoint{3.296652in}{2.582559in}}%
\pgfpathlineto{\pgfqpoint{3.283491in}{2.587850in}}%
\pgfpathlineto{\pgfqpoint{3.291468in}{2.594757in}}%
\pgfpathlineto{\pgfqpoint{3.299439in}{2.601702in}}%
\pgfpathlineto{\pgfqpoint{3.307402in}{2.608686in}}%
\pgfpathlineto{\pgfqpoint{3.315360in}{2.615707in}}%
\pgfpathclose%
\pgfusepath{fill}%
\end{pgfscope}%
\begin{pgfscope}%
\pgfpathrectangle{\pgfqpoint{1.150000in}{0.150000in}}{\pgfqpoint{5.700000in}{5.700000in}}%
\pgfusepath{clip}%
\pgfsetbuttcap%
\pgfsetroundjoin%
\definecolor{currentfill}{rgb}{0.277941,0.056324,0.381191}%
\pgfsetfillcolor{currentfill}%
\pgfsetfillopacity{0.700000}%
\pgfsetlinewidth{0.000000pt}%
\definecolor{currentstroke}{rgb}{0.000000,0.000000,0.000000}%
\pgfsetstrokecolor{currentstroke}%
\pgfsetdash{}{0pt}%
\pgfpathmoveto{\pgfqpoint{4.746798in}{2.663022in}}%
\pgfpathlineto{\pgfqpoint{4.760247in}{2.660015in}}%
\pgfpathlineto{\pgfqpoint{4.773702in}{2.657035in}}%
\pgfpathlineto{\pgfqpoint{4.787164in}{2.654082in}}%
\pgfpathlineto{\pgfqpoint{4.800632in}{2.651155in}}%
\pgfpathlineto{\pgfqpoint{4.793215in}{2.643922in}}%
\pgfpathlineto{\pgfqpoint{4.785792in}{2.636719in}}%
\pgfpathlineto{\pgfqpoint{4.778364in}{2.629542in}}%
\pgfpathlineto{\pgfqpoint{4.770932in}{2.622389in}}%
\pgfpathlineto{\pgfqpoint{4.757449in}{2.625213in}}%
\pgfpathlineto{\pgfqpoint{4.743973in}{2.628064in}}%
\pgfpathlineto{\pgfqpoint{4.730504in}{2.630942in}}%
\pgfpathlineto{\pgfqpoint{4.717041in}{2.633847in}}%
\pgfpathlineto{\pgfqpoint{4.724488in}{2.641097in}}%
\pgfpathlineto{\pgfqpoint{4.731930in}{2.648375in}}%
\pgfpathlineto{\pgfqpoint{4.739367in}{2.655682in}}%
\pgfpathlineto{\pgfqpoint{4.746798in}{2.663022in}}%
\pgfpathclose%
\pgfusepath{fill}%
\end{pgfscope}%
\begin{pgfscope}%
\pgfpathrectangle{\pgfqpoint{1.150000in}{0.150000in}}{\pgfqpoint{5.700000in}{5.700000in}}%
\pgfusepath{clip}%
\pgfsetbuttcap%
\pgfsetroundjoin%
\definecolor{currentfill}{rgb}{0.276022,0.044167,0.370164}%
\pgfsetfillcolor{currentfill}%
\pgfsetfillopacity{0.700000}%
\pgfsetlinewidth{0.000000pt}%
\definecolor{currentstroke}{rgb}{0.000000,0.000000,0.000000}%
\pgfsetstrokecolor{currentstroke}%
\pgfsetdash{}{0pt}%
\pgfpathmoveto{\pgfqpoint{3.041274in}{2.652309in}}%
\pgfpathlineto{\pgfqpoint{3.054385in}{2.646237in}}%
\pgfpathlineto{\pgfqpoint{3.067500in}{2.640213in}}%
\pgfpathlineto{\pgfqpoint{3.080618in}{2.634236in}}%
\pgfpathlineto{\pgfqpoint{3.093741in}{2.628305in}}%
\pgfpathlineto{\pgfqpoint{3.085685in}{2.621844in}}%
\pgfpathlineto{\pgfqpoint{3.077621in}{2.615440in}}%
\pgfpathlineto{\pgfqpoint{3.069551in}{2.609095in}}%
\pgfpathlineto{\pgfqpoint{3.061473in}{2.602809in}}%
\pgfpathlineto{\pgfqpoint{3.048335in}{2.608822in}}%
\pgfpathlineto{\pgfqpoint{3.035200in}{2.614882in}}%
\pgfpathlineto{\pgfqpoint{3.022070in}{2.620988in}}%
\pgfpathlineto{\pgfqpoint{3.008943in}{2.627143in}}%
\pgfpathlineto{\pgfqpoint{3.017037in}{2.633341in}}%
\pgfpathlineto{\pgfqpoint{3.025123in}{2.639602in}}%
\pgfpathlineto{\pgfqpoint{3.033202in}{2.645926in}}%
\pgfpathlineto{\pgfqpoint{3.041274in}{2.652309in}}%
\pgfpathclose%
\pgfusepath{fill}%
\end{pgfscope}%
\begin{pgfscope}%
\pgfpathrectangle{\pgfqpoint{1.150000in}{0.150000in}}{\pgfqpoint{5.700000in}{5.700000in}}%
\pgfusepath{clip}%
\pgfsetbuttcap%
\pgfsetroundjoin%
\definecolor{currentfill}{rgb}{0.269944,0.014625,0.341379}%
\pgfsetfillcolor{currentfill}%
\pgfsetfillopacity{0.700000}%
\pgfsetlinewidth{0.000000pt}%
\definecolor{currentstroke}{rgb}{0.000000,0.000000,0.000000}%
\pgfsetstrokecolor{currentstroke}%
\pgfsetdash{}{0pt}%
\pgfpathmoveto{\pgfqpoint{3.452314in}{2.603742in}}%
\pgfpathlineto{\pgfqpoint{3.465483in}{2.598870in}}%
\pgfpathlineto{\pgfqpoint{3.478657in}{2.594037in}}%
\pgfpathlineto{\pgfqpoint{3.491836in}{2.589242in}}%
\pgfpathlineto{\pgfqpoint{3.505020in}{2.584485in}}%
\pgfpathlineto{\pgfqpoint{3.497129in}{2.577211in}}%
\pgfpathlineto{\pgfqpoint{3.489231in}{2.569961in}}%
\pgfpathlineto{\pgfqpoint{3.481327in}{2.562735in}}%
\pgfpathlineto{\pgfqpoint{3.473418in}{2.555536in}}%
\pgfpathlineto{\pgfqpoint{3.460220in}{2.560335in}}%
\pgfpathlineto{\pgfqpoint{3.447028in}{2.565173in}}%
\pgfpathlineto{\pgfqpoint{3.433840in}{2.570048in}}%
\pgfpathlineto{\pgfqpoint{3.420657in}{2.574963in}}%
\pgfpathlineto{\pgfqpoint{3.428581in}{2.582115in}}%
\pgfpathlineto{\pgfqpoint{3.436498in}{2.589296in}}%
\pgfpathlineto{\pgfqpoint{3.444409in}{2.596505in}}%
\pgfpathlineto{\pgfqpoint{3.452314in}{2.603742in}}%
\pgfpathclose%
\pgfusepath{fill}%
\end{pgfscope}%
\begin{pgfscope}%
\pgfpathrectangle{\pgfqpoint{1.150000in}{0.150000in}}{\pgfqpoint{5.700000in}{5.700000in}}%
\pgfusepath{clip}%
\pgfsetbuttcap%
\pgfsetroundjoin%
\definecolor{currentfill}{rgb}{0.269944,0.014625,0.341379}%
\pgfsetfillcolor{currentfill}%
\pgfsetfillopacity{0.700000}%
\pgfsetlinewidth{0.000000pt}%
\definecolor{currentstroke}{rgb}{0.000000,0.000000,0.000000}%
\pgfsetstrokecolor{currentstroke}%
\pgfsetdash{}{0pt}%
\pgfpathmoveto{\pgfqpoint{3.947219in}{2.601596in}}%
\pgfpathlineto{\pgfqpoint{3.960485in}{2.597749in}}%
\pgfpathlineto{\pgfqpoint{3.973756in}{2.593933in}}%
\pgfpathlineto{\pgfqpoint{3.987034in}{2.590150in}}%
\pgfpathlineto{\pgfqpoint{4.000317in}{2.586398in}}%
\pgfpathlineto{\pgfqpoint{3.992606in}{2.578841in}}%
\pgfpathlineto{\pgfqpoint{3.984890in}{2.571289in}}%
\pgfpathlineto{\pgfqpoint{3.977168in}{2.563741in}}%
\pgfpathlineto{\pgfqpoint{3.969440in}{2.556196in}}%
\pgfpathlineto{\pgfqpoint{3.956145in}{2.559937in}}%
\pgfpathlineto{\pgfqpoint{3.942855in}{2.563711in}}%
\pgfpathlineto{\pgfqpoint{3.929571in}{2.567516in}}%
\pgfpathlineto{\pgfqpoint{3.916293in}{2.571353in}}%
\pgfpathlineto{\pgfqpoint{3.924033in}{2.578904in}}%
\pgfpathlineto{\pgfqpoint{3.931767in}{2.586461in}}%
\pgfpathlineto{\pgfqpoint{3.939496in}{2.594024in}}%
\pgfpathlineto{\pgfqpoint{3.947219in}{2.601596in}}%
\pgfpathclose%
\pgfusepath{fill}%
\end{pgfscope}%
\begin{pgfscope}%
\pgfpathrectangle{\pgfqpoint{1.150000in}{0.150000in}}{\pgfqpoint{5.700000in}{5.700000in}}%
\pgfusepath{clip}%
\pgfsetbuttcap%
\pgfsetroundjoin%
\definecolor{currentfill}{rgb}{0.276022,0.044167,0.370164}%
\pgfsetfillcolor{currentfill}%
\pgfsetfillopacity{0.700000}%
\pgfsetlinewidth{0.000000pt}%
\definecolor{currentstroke}{rgb}{0.000000,0.000000,0.000000}%
\pgfsetstrokecolor{currentstroke}%
\pgfsetdash{}{0pt}%
\pgfpathmoveto{\pgfqpoint{4.526001in}{2.640942in}}%
\pgfpathlineto{\pgfqpoint{4.539398in}{2.637799in}}%
\pgfpathlineto{\pgfqpoint{4.552802in}{2.634685in}}%
\pgfpathlineto{\pgfqpoint{4.566213in}{2.631598in}}%
\pgfpathlineto{\pgfqpoint{4.579630in}{2.628539in}}%
\pgfpathlineto{\pgfqpoint{4.572129in}{2.621212in}}%
\pgfpathlineto{\pgfqpoint{4.564624in}{2.613900in}}%
\pgfpathlineto{\pgfqpoint{4.557113in}{2.606601in}}%
\pgfpathlineto{\pgfqpoint{4.549597in}{2.599311in}}%
\pgfpathlineto{\pgfqpoint{4.536167in}{2.602294in}}%
\pgfpathlineto{\pgfqpoint{4.522743in}{2.605305in}}%
\pgfpathlineto{\pgfqpoint{4.509326in}{2.608344in}}%
\pgfpathlineto{\pgfqpoint{4.495915in}{2.611410in}}%
\pgfpathlineto{\pgfqpoint{4.503444in}{2.618771in}}%
\pgfpathlineto{\pgfqpoint{4.510968in}{2.626145in}}%
\pgfpathlineto{\pgfqpoint{4.518487in}{2.633534in}}%
\pgfpathlineto{\pgfqpoint{4.526001in}{2.640942in}}%
\pgfpathclose%
\pgfusepath{fill}%
\end{pgfscope}%
\begin{pgfscope}%
\pgfpathrectangle{\pgfqpoint{1.150000in}{0.150000in}}{\pgfqpoint{5.700000in}{5.700000in}}%
\pgfusepath{clip}%
\pgfsetbuttcap%
\pgfsetroundjoin%
\definecolor{currentfill}{rgb}{0.278791,0.062145,0.386592}%
\pgfsetfillcolor{currentfill}%
\pgfsetfillopacity{0.700000}%
\pgfsetlinewidth{0.000000pt}%
\definecolor{currentstroke}{rgb}{0.000000,0.000000,0.000000}%
\pgfsetstrokecolor{currentstroke}%
\pgfsetdash{}{0pt}%
\pgfpathmoveto{\pgfqpoint{2.904049in}{2.678155in}}%
\pgfpathlineto{\pgfqpoint{2.917149in}{2.671601in}}%
\pgfpathlineto{\pgfqpoint{2.930253in}{2.665099in}}%
\pgfpathlineto{\pgfqpoint{2.943359in}{2.658648in}}%
\pgfpathlineto{\pgfqpoint{2.956469in}{2.652247in}}%
\pgfpathlineto{\pgfqpoint{2.948351in}{2.646206in}}%
\pgfpathlineto{\pgfqpoint{2.940225in}{2.640235in}}%
\pgfpathlineto{\pgfqpoint{2.932091in}{2.634337in}}%
\pgfpathlineto{\pgfqpoint{2.923948in}{2.628514in}}%
\pgfpathlineto{\pgfqpoint{2.910822in}{2.635010in}}%
\pgfpathlineto{\pgfqpoint{2.897698in}{2.641557in}}%
\pgfpathlineto{\pgfqpoint{2.884578in}{2.648155in}}%
\pgfpathlineto{\pgfqpoint{2.871461in}{2.654805in}}%
\pgfpathlineto{\pgfqpoint{2.879620in}{2.660527in}}%
\pgfpathlineto{\pgfqpoint{2.887771in}{2.666327in}}%
\pgfpathlineto{\pgfqpoint{2.895914in}{2.672204in}}%
\pgfpathlineto{\pgfqpoint{2.904049in}{2.678155in}}%
\pgfpathclose%
\pgfusepath{fill}%
\end{pgfscope}%
\begin{pgfscope}%
\pgfpathrectangle{\pgfqpoint{1.150000in}{0.150000in}}{\pgfqpoint{5.700000in}{5.700000in}}%
\pgfusepath{clip}%
\pgfsetbuttcap%
\pgfsetroundjoin%
\definecolor{currentfill}{rgb}{0.269944,0.014625,0.341379}%
\pgfsetfillcolor{currentfill}%
\pgfsetfillopacity{0.700000}%
\pgfsetlinewidth{0.000000pt}%
\definecolor{currentstroke}{rgb}{0.000000,0.000000,0.000000}%
\pgfsetstrokecolor{currentstroke}%
\pgfsetdash{}{0pt}%
\pgfpathmoveto{\pgfqpoint{3.589258in}{2.595285in}}%
\pgfpathlineto{\pgfqpoint{3.602454in}{2.590744in}}%
\pgfpathlineto{\pgfqpoint{3.615655in}{2.586239in}}%
\pgfpathlineto{\pgfqpoint{3.628860in}{2.581770in}}%
\pgfpathlineto{\pgfqpoint{3.642071in}{2.577338in}}%
\pgfpathlineto{\pgfqpoint{3.634230in}{2.569920in}}%
\pgfpathlineto{\pgfqpoint{3.626382in}{2.562520in}}%
\pgfpathlineto{\pgfqpoint{3.618529in}{2.555135in}}%
\pgfpathlineto{\pgfqpoint{3.610670in}{2.547768in}}%
\pgfpathlineto{\pgfqpoint{3.597446in}{2.552231in}}%
\pgfpathlineto{\pgfqpoint{3.584227in}{2.556729in}}%
\pgfpathlineto{\pgfqpoint{3.571013in}{2.561263in}}%
\pgfpathlineto{\pgfqpoint{3.557805in}{2.565833in}}%
\pgfpathlineto{\pgfqpoint{3.565677in}{2.573166in}}%
\pgfpathlineto{\pgfqpoint{3.573543in}{2.580519in}}%
\pgfpathlineto{\pgfqpoint{3.581404in}{2.587892in}}%
\pgfpathlineto{\pgfqpoint{3.589258in}{2.595285in}}%
\pgfpathclose%
\pgfusepath{fill}%
\end{pgfscope}%
\begin{pgfscope}%
\pgfpathrectangle{\pgfqpoint{1.150000in}{0.150000in}}{\pgfqpoint{5.700000in}{5.700000in}}%
\pgfusepath{clip}%
\pgfsetbuttcap%
\pgfsetroundjoin%
\definecolor{currentfill}{rgb}{0.273809,0.031497,0.358853}%
\pgfsetfillcolor{currentfill}%
\pgfsetfillopacity{0.700000}%
\pgfsetlinewidth{0.000000pt}%
\definecolor{currentstroke}{rgb}{0.000000,0.000000,0.000000}%
\pgfsetstrokecolor{currentstroke}%
\pgfsetdash{}{0pt}%
\pgfpathmoveto{\pgfqpoint{4.305156in}{2.620294in}}%
\pgfpathlineto{\pgfqpoint{4.318503in}{2.616949in}}%
\pgfpathlineto{\pgfqpoint{4.331857in}{2.613633in}}%
\pgfpathlineto{\pgfqpoint{4.345217in}{2.610346in}}%
\pgfpathlineto{\pgfqpoint{4.358584in}{2.607087in}}%
\pgfpathlineto{\pgfqpoint{4.351001in}{2.599646in}}%
\pgfpathlineto{\pgfqpoint{4.343414in}{2.592211in}}%
\pgfpathlineto{\pgfqpoint{4.335820in}{2.584780in}}%
\pgfpathlineto{\pgfqpoint{4.328222in}{2.577350in}}%
\pgfpathlineto{\pgfqpoint{4.314843in}{2.580559in}}%
\pgfpathlineto{\pgfqpoint{4.301470in}{2.583796in}}%
\pgfpathlineto{\pgfqpoint{4.288103in}{2.587062in}}%
\pgfpathlineto{\pgfqpoint{4.274743in}{2.590358in}}%
\pgfpathlineto{\pgfqpoint{4.282354in}{2.597832in}}%
\pgfpathlineto{\pgfqpoint{4.289960in}{2.605312in}}%
\pgfpathlineto{\pgfqpoint{4.297561in}{2.612798in}}%
\pgfpathlineto{\pgfqpoint{4.305156in}{2.620294in}}%
\pgfpathclose%
\pgfusepath{fill}%
\end{pgfscope}%
\begin{pgfscope}%
\pgfpathrectangle{\pgfqpoint{1.150000in}{0.150000in}}{\pgfqpoint{5.700000in}{5.700000in}}%
\pgfusepath{clip}%
\pgfsetbuttcap%
\pgfsetroundjoin%
\definecolor{currentfill}{rgb}{0.281924,0.089666,0.412415}%
\pgfsetfillcolor{currentfill}%
\pgfsetfillopacity{0.700000}%
\pgfsetlinewidth{0.000000pt}%
\definecolor{currentstroke}{rgb}{0.000000,0.000000,0.000000}%
\pgfsetstrokecolor{currentstroke}%
\pgfsetdash{}{0pt}%
\pgfpathmoveto{\pgfqpoint{5.325822in}{2.716924in}}%
\pgfpathlineto{\pgfqpoint{5.339412in}{2.714030in}}%
\pgfpathlineto{\pgfqpoint{5.353008in}{2.711159in}}%
\pgfpathlineto{\pgfqpoint{5.366611in}{2.708314in}}%
\pgfpathlineto{\pgfqpoint{5.380222in}{2.705493in}}%
\pgfpathlineto{\pgfqpoint{5.373012in}{2.698178in}}%
\pgfpathlineto{\pgfqpoint{5.365799in}{2.690955in}}%
\pgfpathlineto{\pgfqpoint{5.358583in}{2.683819in}}%
\pgfpathlineto{\pgfqpoint{5.351364in}{2.676764in}}%
\pgfpathlineto{\pgfqpoint{5.337736in}{2.679417in}}%
\pgfpathlineto{\pgfqpoint{5.324116in}{2.682095in}}%
\pgfpathlineto{\pgfqpoint{5.310502in}{2.684798in}}%
\pgfpathlineto{\pgfqpoint{5.296896in}{2.687526in}}%
\pgfpathlineto{\pgfqpoint{5.304132in}{2.694743in}}%
\pgfpathlineto{\pgfqpoint{5.311365in}{2.702046in}}%
\pgfpathlineto{\pgfqpoint{5.318595in}{2.709438in}}%
\pgfpathlineto{\pgfqpoint{5.325822in}{2.716924in}}%
\pgfpathclose%
\pgfusepath{fill}%
\end{pgfscope}%
\begin{pgfscope}%
\pgfpathrectangle{\pgfqpoint{1.150000in}{0.150000in}}{\pgfqpoint{5.700000in}{5.700000in}}%
\pgfusepath{clip}%
\pgfsetbuttcap%
\pgfsetroundjoin%
\definecolor{currentfill}{rgb}{0.280267,0.073417,0.397163}%
\pgfsetfillcolor{currentfill}%
\pgfsetfillopacity{0.700000}%
\pgfsetlinewidth{0.000000pt}%
\definecolor{currentstroke}{rgb}{0.000000,0.000000,0.000000}%
\pgfsetstrokecolor{currentstroke}%
\pgfsetdash{}{0pt}%
\pgfpathmoveto{\pgfqpoint{5.104988in}{2.692215in}}%
\pgfpathlineto{\pgfqpoint{5.118527in}{2.689343in}}%
\pgfpathlineto{\pgfqpoint{5.132073in}{2.686496in}}%
\pgfpathlineto{\pgfqpoint{5.145625in}{2.683674in}}%
\pgfpathlineto{\pgfqpoint{5.159185in}{2.680877in}}%
\pgfpathlineto{\pgfqpoint{5.151897in}{2.673715in}}%
\pgfpathlineto{\pgfqpoint{5.144606in}{2.666615in}}%
\pgfpathlineto{\pgfqpoint{5.137310in}{2.659572in}}%
\pgfpathlineto{\pgfqpoint{5.130010in}{2.652583in}}%
\pgfpathlineto{\pgfqpoint{5.116434in}{2.655238in}}%
\pgfpathlineto{\pgfqpoint{5.102866in}{2.657919in}}%
\pgfpathlineto{\pgfqpoint{5.089304in}{2.660625in}}%
\pgfpathlineto{\pgfqpoint{5.075749in}{2.663356in}}%
\pgfpathlineto{\pgfqpoint{5.083065in}{2.670482in}}%
\pgfpathlineto{\pgfqpoint{5.090377in}{2.677664in}}%
\pgfpathlineto{\pgfqpoint{5.097684in}{2.684907in}}%
\pgfpathlineto{\pgfqpoint{5.104988in}{2.692215in}}%
\pgfpathclose%
\pgfusepath{fill}%
\end{pgfscope}%
\begin{pgfscope}%
\pgfpathrectangle{\pgfqpoint{1.150000in}{0.150000in}}{\pgfqpoint{5.700000in}{5.700000in}}%
\pgfusepath{clip}%
\pgfsetbuttcap%
\pgfsetroundjoin%
\definecolor{currentfill}{rgb}{0.269944,0.014625,0.341379}%
\pgfsetfillcolor{currentfill}%
\pgfsetfillopacity{0.700000}%
\pgfsetlinewidth{0.000000pt}%
\definecolor{currentstroke}{rgb}{0.000000,0.000000,0.000000}%
\pgfsetstrokecolor{currentstroke}%
\pgfsetdash{}{0pt}%
\pgfpathmoveto{\pgfqpoint{3.726224in}{2.589858in}}%
\pgfpathlineto{\pgfqpoint{3.739449in}{2.585618in}}%
\pgfpathlineto{\pgfqpoint{3.752678in}{2.581412in}}%
\pgfpathlineto{\pgfqpoint{3.765914in}{2.577240in}}%
\pgfpathlineto{\pgfqpoint{3.779154in}{2.573102in}}%
\pgfpathlineto{\pgfqpoint{3.771361in}{2.565597in}}%
\pgfpathlineto{\pgfqpoint{3.763562in}{2.558101in}}%
\pgfpathlineto{\pgfqpoint{3.755758in}{2.550616in}}%
\pgfpathlineto{\pgfqpoint{3.747947in}{2.543140in}}%
\pgfpathlineto{\pgfqpoint{3.734694in}{2.547294in}}%
\pgfpathlineto{\pgfqpoint{3.721446in}{2.551482in}}%
\pgfpathlineto{\pgfqpoint{3.708204in}{2.555704in}}%
\pgfpathlineto{\pgfqpoint{3.694967in}{2.559961in}}%
\pgfpathlineto{\pgfqpoint{3.702790in}{2.567415in}}%
\pgfpathlineto{\pgfqpoint{3.710607in}{2.574883in}}%
\pgfpathlineto{\pgfqpoint{3.718419in}{2.582364in}}%
\pgfpathlineto{\pgfqpoint{3.726224in}{2.589858in}}%
\pgfpathclose%
\pgfusepath{fill}%
\end{pgfscope}%
\begin{pgfscope}%
\pgfpathrectangle{\pgfqpoint{1.150000in}{0.150000in}}{\pgfqpoint{5.700000in}{5.700000in}}%
\pgfusepath{clip}%
\pgfsetbuttcap%
\pgfsetroundjoin%
\definecolor{currentfill}{rgb}{0.278791,0.062145,0.386592}%
\pgfsetfillcolor{currentfill}%
\pgfsetfillopacity{0.700000}%
\pgfsetlinewidth{0.000000pt}%
\definecolor{currentstroke}{rgb}{0.000000,0.000000,0.000000}%
\pgfsetstrokecolor{currentstroke}%
\pgfsetdash{}{0pt}%
\pgfpathmoveto{\pgfqpoint{4.884138in}{2.668554in}}%
\pgfpathlineto{\pgfqpoint{4.897626in}{2.665643in}}%
\pgfpathlineto{\pgfqpoint{4.911120in}{2.662758in}}%
\pgfpathlineto{\pgfqpoint{4.924621in}{2.659899in}}%
\pgfpathlineto{\pgfqpoint{4.938129in}{2.657066in}}%
\pgfpathlineto{\pgfqpoint{4.930760in}{2.649918in}}%
\pgfpathlineto{\pgfqpoint{4.923386in}{2.642807in}}%
\pgfpathlineto{\pgfqpoint{4.916008in}{2.635731in}}%
\pgfpathlineto{\pgfqpoint{4.908624in}{2.628685in}}%
\pgfpathlineto{\pgfqpoint{4.895101in}{2.631402in}}%
\pgfpathlineto{\pgfqpoint{4.881585in}{2.634146in}}%
\pgfpathlineto{\pgfqpoint{4.868076in}{2.636915in}}%
\pgfpathlineto{\pgfqpoint{4.854574in}{2.639711in}}%
\pgfpathlineto{\pgfqpoint{4.861972in}{2.646867in}}%
\pgfpathlineto{\pgfqpoint{4.869365in}{2.654057in}}%
\pgfpathlineto{\pgfqpoint{4.876754in}{2.661285in}}%
\pgfpathlineto{\pgfqpoint{4.884138in}{2.668554in}}%
\pgfpathclose%
\pgfusepath{fill}%
\end{pgfscope}%
\begin{pgfscope}%
\pgfpathrectangle{\pgfqpoint{1.150000in}{0.150000in}}{\pgfqpoint{5.700000in}{5.700000in}}%
\pgfusepath{clip}%
\pgfsetbuttcap%
\pgfsetroundjoin%
\definecolor{currentfill}{rgb}{0.271305,0.019942,0.347269}%
\pgfsetfillcolor{currentfill}%
\pgfsetfillopacity{0.700000}%
\pgfsetlinewidth{0.000000pt}%
\definecolor{currentstroke}{rgb}{0.000000,0.000000,0.000000}%
\pgfsetstrokecolor{currentstroke}%
\pgfsetdash{}{0pt}%
\pgfpathmoveto{\pgfqpoint{4.084246in}{2.601906in}}%
\pgfpathlineto{\pgfqpoint{4.097546in}{2.598287in}}%
\pgfpathlineto{\pgfqpoint{4.110851in}{2.594699in}}%
\pgfpathlineto{\pgfqpoint{4.124163in}{2.591142in}}%
\pgfpathlineto{\pgfqpoint{4.137481in}{2.587615in}}%
\pgfpathlineto{\pgfqpoint{4.129817in}{2.580086in}}%
\pgfpathlineto{\pgfqpoint{4.122148in}{2.572559in}}%
\pgfpathlineto{\pgfqpoint{4.114473in}{2.565033in}}%
\pgfpathlineto{\pgfqpoint{4.106792in}{2.557508in}}%
\pgfpathlineto{\pgfqpoint{4.093462in}{2.561011in}}%
\pgfpathlineto{\pgfqpoint{4.080138in}{2.564545in}}%
\pgfpathlineto{\pgfqpoint{4.066820in}{2.568110in}}%
\pgfpathlineto{\pgfqpoint{4.053507in}{2.571705in}}%
\pgfpathlineto{\pgfqpoint{4.061200in}{2.579249in}}%
\pgfpathlineto{\pgfqpoint{4.068888in}{2.586797in}}%
\pgfpathlineto{\pgfqpoint{4.076569in}{2.594348in}}%
\pgfpathlineto{\pgfqpoint{4.084246in}{2.601906in}}%
\pgfpathclose%
\pgfusepath{fill}%
\end{pgfscope}%
\begin{pgfscope}%
\pgfpathrectangle{\pgfqpoint{1.150000in}{0.150000in}}{\pgfqpoint{5.700000in}{5.700000in}}%
\pgfusepath{clip}%
\pgfsetbuttcap%
\pgfsetroundjoin%
\definecolor{currentfill}{rgb}{0.277018,0.050344,0.375715}%
\pgfsetfillcolor{currentfill}%
\pgfsetfillopacity{0.700000}%
\pgfsetlinewidth{0.000000pt}%
\definecolor{currentstroke}{rgb}{0.000000,0.000000,0.000000}%
\pgfsetstrokecolor{currentstroke}%
\pgfsetdash{}{0pt}%
\pgfpathmoveto{\pgfqpoint{4.663257in}{2.645733in}}%
\pgfpathlineto{\pgfqpoint{4.676693in}{2.642721in}}%
\pgfpathlineto{\pgfqpoint{4.690136in}{2.639736in}}%
\pgfpathlineto{\pgfqpoint{4.703585in}{2.636778in}}%
\pgfpathlineto{\pgfqpoint{4.717041in}{2.633847in}}%
\pgfpathlineto{\pgfqpoint{4.709589in}{2.626619in}}%
\pgfpathlineto{\pgfqpoint{4.702132in}{2.619412in}}%
\pgfpathlineto{\pgfqpoint{4.694670in}{2.612222in}}%
\pgfpathlineto{\pgfqpoint{4.687202in}{2.605046in}}%
\pgfpathlineto{\pgfqpoint{4.673732in}{2.607888in}}%
\pgfpathlineto{\pgfqpoint{4.660269in}{2.610757in}}%
\pgfpathlineto{\pgfqpoint{4.646813in}{2.613652in}}%
\pgfpathlineto{\pgfqpoint{4.633363in}{2.616575in}}%
\pgfpathlineto{\pgfqpoint{4.640844in}{2.623836in}}%
\pgfpathlineto{\pgfqpoint{4.648320in}{2.631113in}}%
\pgfpathlineto{\pgfqpoint{4.655791in}{2.638412in}}%
\pgfpathlineto{\pgfqpoint{4.663257in}{2.645733in}}%
\pgfpathclose%
\pgfusepath{fill}%
\end{pgfscope}%
\begin{pgfscope}%
\pgfpathrectangle{\pgfqpoint{1.150000in}{0.150000in}}{\pgfqpoint{5.700000in}{5.700000in}}%
\pgfusepath{clip}%
\pgfsetbuttcap%
\pgfsetroundjoin%
\definecolor{currentfill}{rgb}{0.274952,0.037752,0.364543}%
\pgfsetfillcolor{currentfill}%
\pgfsetfillopacity{0.700000}%
\pgfsetlinewidth{0.000000pt}%
\definecolor{currentstroke}{rgb}{0.000000,0.000000,0.000000}%
\pgfsetstrokecolor{currentstroke}%
\pgfsetdash{}{0pt}%
\pgfpathmoveto{\pgfqpoint{4.442336in}{2.623955in}}%
\pgfpathlineto{\pgfqpoint{4.455721in}{2.620777in}}%
\pgfpathlineto{\pgfqpoint{4.469113in}{2.617626in}}%
\pgfpathlineto{\pgfqpoint{4.482511in}{2.614504in}}%
\pgfpathlineto{\pgfqpoint{4.495915in}{2.611410in}}%
\pgfpathlineto{\pgfqpoint{4.488380in}{2.604060in}}%
\pgfpathlineto{\pgfqpoint{4.480840in}{2.596717in}}%
\pgfpathlineto{\pgfqpoint{4.473294in}{2.589381in}}%
\pgfpathlineto{\pgfqpoint{4.465743in}{2.582047in}}%
\pgfpathlineto{\pgfqpoint{4.452326in}{2.585079in}}%
\pgfpathlineto{\pgfqpoint{4.438915in}{2.588138in}}%
\pgfpathlineto{\pgfqpoint{4.425510in}{2.591225in}}%
\pgfpathlineto{\pgfqpoint{4.412112in}{2.594341in}}%
\pgfpathlineto{\pgfqpoint{4.419676in}{2.601732in}}%
\pgfpathlineto{\pgfqpoint{4.427235in}{2.609130in}}%
\pgfpathlineto{\pgfqpoint{4.434788in}{2.616537in}}%
\pgfpathlineto{\pgfqpoint{4.442336in}{2.623955in}}%
\pgfpathclose%
\pgfusepath{fill}%
\end{pgfscope}%
\begin{pgfscope}%
\pgfpathrectangle{\pgfqpoint{1.150000in}{0.150000in}}{\pgfqpoint{5.700000in}{5.700000in}}%
\pgfusepath{clip}%
\pgfsetbuttcap%
\pgfsetroundjoin%
\definecolor{currentfill}{rgb}{0.269944,0.014625,0.341379}%
\pgfsetfillcolor{currentfill}%
\pgfsetfillopacity{0.700000}%
\pgfsetlinewidth{0.000000pt}%
\definecolor{currentstroke}{rgb}{0.000000,0.000000,0.000000}%
\pgfsetstrokecolor{currentstroke}%
\pgfsetdash{}{0pt}%
\pgfpathmoveto{\pgfqpoint{3.863237in}{2.587027in}}%
\pgfpathlineto{\pgfqpoint{3.876492in}{2.583060in}}%
\pgfpathlineto{\pgfqpoint{3.889754in}{2.579125in}}%
\pgfpathlineto{\pgfqpoint{3.903021in}{2.575223in}}%
\pgfpathlineto{\pgfqpoint{3.916293in}{2.571353in}}%
\pgfpathlineto{\pgfqpoint{3.908547in}{2.563808in}}%
\pgfpathlineto{\pgfqpoint{3.900796in}{2.556268in}}%
\pgfpathlineto{\pgfqpoint{3.893039in}{2.548733in}}%
\pgfpathlineto{\pgfqpoint{3.885277in}{2.541202in}}%
\pgfpathlineto{\pgfqpoint{3.871992in}{2.545074in}}%
\pgfpathlineto{\pgfqpoint{3.858713in}{2.548979in}}%
\pgfpathlineto{\pgfqpoint{3.845439in}{2.552917in}}%
\pgfpathlineto{\pgfqpoint{3.832171in}{2.556887in}}%
\pgfpathlineto{\pgfqpoint{3.839946in}{2.564411in}}%
\pgfpathlineto{\pgfqpoint{3.847715in}{2.571942in}}%
\pgfpathlineto{\pgfqpoint{3.855479in}{2.579480in}}%
\pgfpathlineto{\pgfqpoint{3.863237in}{2.587027in}}%
\pgfpathclose%
\pgfusepath{fill}%
\end{pgfscope}%
\begin{pgfscope}%
\pgfpathrectangle{\pgfqpoint{1.150000in}{0.150000in}}{\pgfqpoint{5.700000in}{5.700000in}}%
\pgfusepath{clip}%
\pgfsetbuttcap%
\pgfsetroundjoin%
\definecolor{currentfill}{rgb}{0.272594,0.025563,0.353093}%
\pgfsetfillcolor{currentfill}%
\pgfsetfillopacity{0.700000}%
\pgfsetlinewidth{0.000000pt}%
\definecolor{currentstroke}{rgb}{0.000000,0.000000,0.000000}%
\pgfsetstrokecolor{currentstroke}%
\pgfsetdash{}{0pt}%
\pgfpathmoveto{\pgfqpoint{3.230893in}{2.609432in}}%
\pgfpathlineto{\pgfqpoint{3.244036in}{2.603973in}}%
\pgfpathlineto{\pgfqpoint{3.257183in}{2.598557in}}%
\pgfpathlineto{\pgfqpoint{3.270335in}{2.593182in}}%
\pgfpathlineto{\pgfqpoint{3.283491in}{2.587850in}}%
\pgfpathlineto{\pgfqpoint{3.275508in}{2.580983in}}%
\pgfpathlineto{\pgfqpoint{3.267518in}{2.574158in}}%
\pgfpathlineto{\pgfqpoint{3.259521in}{2.567376in}}%
\pgfpathlineto{\pgfqpoint{3.251517in}{2.560638in}}%
\pgfpathlineto{\pgfqpoint{3.238346in}{2.566039in}}%
\pgfpathlineto{\pgfqpoint{3.225180in}{2.571482in}}%
\pgfpathlineto{\pgfqpoint{3.212018in}{2.576967in}}%
\pgfpathlineto{\pgfqpoint{3.198860in}{2.582496in}}%
\pgfpathlineto{\pgfqpoint{3.206878in}{2.589160in}}%
\pgfpathlineto{\pgfqpoint{3.214890in}{2.595872in}}%
\pgfpathlineto{\pgfqpoint{3.222895in}{2.602630in}}%
\pgfpathlineto{\pgfqpoint{3.230893in}{2.609432in}}%
\pgfpathclose%
\pgfusepath{fill}%
\end{pgfscope}%
\begin{pgfscope}%
\pgfpathrectangle{\pgfqpoint{1.150000in}{0.150000in}}{\pgfqpoint{5.700000in}{5.700000in}}%
\pgfusepath{clip}%
\pgfsetbuttcap%
\pgfsetroundjoin%
\definecolor{currentfill}{rgb}{0.274952,0.037752,0.364543}%
\pgfsetfillcolor{currentfill}%
\pgfsetfillopacity{0.700000}%
\pgfsetlinewidth{0.000000pt}%
\definecolor{currentstroke}{rgb}{0.000000,0.000000,0.000000}%
\pgfsetstrokecolor{currentstroke}%
\pgfsetdash{}{0pt}%
\pgfpathmoveto{\pgfqpoint{3.093741in}{2.628305in}}%
\pgfpathlineto{\pgfqpoint{3.106867in}{2.622421in}}%
\pgfpathlineto{\pgfqpoint{3.119997in}{2.616583in}}%
\pgfpathlineto{\pgfqpoint{3.133131in}{2.610791in}}%
\pgfpathlineto{\pgfqpoint{3.146268in}{2.605043in}}%
\pgfpathlineto{\pgfqpoint{3.138228in}{2.598505in}}%
\pgfpathlineto{\pgfqpoint{3.130180in}{2.592020in}}%
\pgfpathlineto{\pgfqpoint{3.122125in}{2.585591in}}%
\pgfpathlineto{\pgfqpoint{3.114063in}{2.579218in}}%
\pgfpathlineto{\pgfqpoint{3.100909in}{2.585048in}}%
\pgfpathlineto{\pgfqpoint{3.087760in}{2.590923in}}%
\pgfpathlineto{\pgfqpoint{3.074614in}{2.596843in}}%
\pgfpathlineto{\pgfqpoint{3.061473in}{2.602809in}}%
\pgfpathlineto{\pgfqpoint{3.069551in}{2.609095in}}%
\pgfpathlineto{\pgfqpoint{3.077621in}{2.615440in}}%
\pgfpathlineto{\pgfqpoint{3.085685in}{2.621844in}}%
\pgfpathlineto{\pgfqpoint{3.093741in}{2.628305in}}%
\pgfpathclose%
\pgfusepath{fill}%
\end{pgfscope}%
\begin{pgfscope}%
\pgfpathrectangle{\pgfqpoint{1.150000in}{0.150000in}}{\pgfqpoint{5.700000in}{5.700000in}}%
\pgfusepath{clip}%
\pgfsetbuttcap%
\pgfsetroundjoin%
\definecolor{currentfill}{rgb}{0.271305,0.019942,0.347269}%
\pgfsetfillcolor{currentfill}%
\pgfsetfillopacity{0.700000}%
\pgfsetlinewidth{0.000000pt}%
\definecolor{currentstroke}{rgb}{0.000000,0.000000,0.000000}%
\pgfsetstrokecolor{currentstroke}%
\pgfsetdash{}{0pt}%
\pgfpathmoveto{\pgfqpoint{3.367972in}{2.595013in}}%
\pgfpathlineto{\pgfqpoint{3.381137in}{2.589941in}}%
\pgfpathlineto{\pgfqpoint{3.394305in}{2.584909in}}%
\pgfpathlineto{\pgfqpoint{3.407479in}{2.579916in}}%
\pgfpathlineto{\pgfqpoint{3.420657in}{2.574963in}}%
\pgfpathlineto{\pgfqpoint{3.412728in}{2.567840in}}%
\pgfpathlineto{\pgfqpoint{3.404792in}{2.560748in}}%
\pgfpathlineto{\pgfqpoint{3.396849in}{2.553688in}}%
\pgfpathlineto{\pgfqpoint{3.388901in}{2.546660in}}%
\pgfpathlineto{\pgfqpoint{3.375709in}{2.551669in}}%
\pgfpathlineto{\pgfqpoint{3.362521in}{2.556717in}}%
\pgfpathlineto{\pgfqpoint{3.349338in}{2.561805in}}%
\pgfpathlineto{\pgfqpoint{3.336160in}{2.566933in}}%
\pgfpathlineto{\pgfqpoint{3.344123in}{2.573900in}}%
\pgfpathlineto{\pgfqpoint{3.352079in}{2.580903in}}%
\pgfpathlineto{\pgfqpoint{3.360029in}{2.587941in}}%
\pgfpathlineto{\pgfqpoint{3.367972in}{2.595013in}}%
\pgfpathclose%
\pgfusepath{fill}%
\end{pgfscope}%
\begin{pgfscope}%
\pgfpathrectangle{\pgfqpoint{1.150000in}{0.150000in}}{\pgfqpoint{5.700000in}{5.700000in}}%
\pgfusepath{clip}%
\pgfsetbuttcap%
\pgfsetroundjoin%
\definecolor{currentfill}{rgb}{0.281446,0.084320,0.407414}%
\pgfsetfillcolor{currentfill}%
\pgfsetfillopacity{0.700000}%
\pgfsetlinewidth{0.000000pt}%
\definecolor{currentstroke}{rgb}{0.000000,0.000000,0.000000}%
\pgfsetstrokecolor{currentstroke}%
\pgfsetdash{}{0pt}%
\pgfpathmoveto{\pgfqpoint{5.242541in}{2.698683in}}%
\pgfpathlineto{\pgfqpoint{5.256119in}{2.695856in}}%
\pgfpathlineto{\pgfqpoint{5.269704in}{2.693055in}}%
\pgfpathlineto{\pgfqpoint{5.283297in}{2.690278in}}%
\pgfpathlineto{\pgfqpoint{5.296896in}{2.687526in}}%
\pgfpathlineto{\pgfqpoint{5.289656in}{2.680387in}}%
\pgfpathlineto{\pgfqpoint{5.282413in}{2.673324in}}%
\pgfpathlineto{\pgfqpoint{5.275167in}{2.666330in}}%
\pgfpathlineto{\pgfqpoint{5.267916in}{2.659401in}}%
\pgfpathlineto{\pgfqpoint{5.254300in}{2.661998in}}%
\pgfpathlineto{\pgfqpoint{5.240691in}{2.664621in}}%
\pgfpathlineto{\pgfqpoint{5.227089in}{2.667268in}}%
\pgfpathlineto{\pgfqpoint{5.213494in}{2.669940in}}%
\pgfpathlineto{\pgfqpoint{5.220761in}{2.677018in}}%
\pgfpathlineto{\pgfqpoint{5.228025in}{2.684165in}}%
\pgfpathlineto{\pgfqpoint{5.235285in}{2.691385in}}%
\pgfpathlineto{\pgfqpoint{5.242541in}{2.698683in}}%
\pgfpathclose%
\pgfusepath{fill}%
\end{pgfscope}%
\begin{pgfscope}%
\pgfpathrectangle{\pgfqpoint{1.150000in}{0.150000in}}{\pgfqpoint{5.700000in}{5.700000in}}%
\pgfusepath{clip}%
\pgfsetbuttcap%
\pgfsetroundjoin%
\definecolor{currentfill}{rgb}{0.272594,0.025563,0.353093}%
\pgfsetfillcolor{currentfill}%
\pgfsetfillopacity{0.700000}%
\pgfsetlinewidth{0.000000pt}%
\definecolor{currentstroke}{rgb}{0.000000,0.000000,0.000000}%
\pgfsetstrokecolor{currentstroke}%
\pgfsetdash{}{0pt}%
\pgfpathmoveto{\pgfqpoint{4.221363in}{2.603834in}}%
\pgfpathlineto{\pgfqpoint{4.234699in}{2.600420in}}%
\pgfpathlineto{\pgfqpoint{4.248041in}{2.597037in}}%
\pgfpathlineto{\pgfqpoint{4.261389in}{2.593683in}}%
\pgfpathlineto{\pgfqpoint{4.274743in}{2.590358in}}%
\pgfpathlineto{\pgfqpoint{4.267126in}{2.582887in}}%
\pgfpathlineto{\pgfqpoint{4.259504in}{2.575418in}}%
\pgfpathlineto{\pgfqpoint{4.251876in}{2.567950in}}%
\pgfpathlineto{\pgfqpoint{4.244242in}{2.560480in}}%
\pgfpathlineto{\pgfqpoint{4.230876in}{2.563768in}}%
\pgfpathlineto{\pgfqpoint{4.217515in}{2.567086in}}%
\pgfpathlineto{\pgfqpoint{4.204161in}{2.570433in}}%
\pgfpathlineto{\pgfqpoint{4.190813in}{2.573809in}}%
\pgfpathlineto{\pgfqpoint{4.198459in}{2.581311in}}%
\pgfpathlineto{\pgfqpoint{4.206099in}{2.588814in}}%
\pgfpathlineto{\pgfqpoint{4.213734in}{2.596321in}}%
\pgfpathlineto{\pgfqpoint{4.221363in}{2.603834in}}%
\pgfpathclose%
\pgfusepath{fill}%
\end{pgfscope}%
\begin{pgfscope}%
\pgfpathrectangle{\pgfqpoint{1.150000in}{0.150000in}}{\pgfqpoint{5.700000in}{5.700000in}}%
\pgfusepath{clip}%
\pgfsetbuttcap%
\pgfsetroundjoin%
\definecolor{currentfill}{rgb}{0.269944,0.014625,0.341379}%
\pgfsetfillcolor{currentfill}%
\pgfsetfillopacity{0.700000}%
\pgfsetlinewidth{0.000000pt}%
\definecolor{currentstroke}{rgb}{0.000000,0.000000,0.000000}%
\pgfsetstrokecolor{currentstroke}%
\pgfsetdash{}{0pt}%
\pgfpathmoveto{\pgfqpoint{3.505020in}{2.584485in}}%
\pgfpathlineto{\pgfqpoint{3.518209in}{2.579766in}}%
\pgfpathlineto{\pgfqpoint{3.531403in}{2.575085in}}%
\pgfpathlineto{\pgfqpoint{3.544601in}{2.570441in}}%
\pgfpathlineto{\pgfqpoint{3.557805in}{2.565833in}}%
\pgfpathlineto{\pgfqpoint{3.549927in}{2.558521in}}%
\pgfpathlineto{\pgfqpoint{3.542042in}{2.551230in}}%
\pgfpathlineto{\pgfqpoint{3.534152in}{2.543961in}}%
\pgfpathlineto{\pgfqpoint{3.526255in}{2.536714in}}%
\pgfpathlineto{\pgfqpoint{3.513039in}{2.541364in}}%
\pgfpathlineto{\pgfqpoint{3.499827in}{2.546051in}}%
\pgfpathlineto{\pgfqpoint{3.486620in}{2.550775in}}%
\pgfpathlineto{\pgfqpoint{3.473418in}{2.555536in}}%
\pgfpathlineto{\pgfqpoint{3.481327in}{2.562735in}}%
\pgfpathlineto{\pgfqpoint{3.489231in}{2.569961in}}%
\pgfpathlineto{\pgfqpoint{3.497129in}{2.577211in}}%
\pgfpathlineto{\pgfqpoint{3.505020in}{2.584485in}}%
\pgfpathclose%
\pgfusepath{fill}%
\end{pgfscope}%
\begin{pgfscope}%
\pgfpathrectangle{\pgfqpoint{1.150000in}{0.150000in}}{\pgfqpoint{5.700000in}{5.700000in}}%
\pgfusepath{clip}%
\pgfsetbuttcap%
\pgfsetroundjoin%
\definecolor{currentfill}{rgb}{0.279566,0.067836,0.391917}%
\pgfsetfillcolor{currentfill}%
\pgfsetfillopacity{0.700000}%
\pgfsetlinewidth{0.000000pt}%
\definecolor{currentstroke}{rgb}{0.000000,0.000000,0.000000}%
\pgfsetstrokecolor{currentstroke}%
\pgfsetdash{}{0pt}%
\pgfpathmoveto{\pgfqpoint{5.021600in}{2.674534in}}%
\pgfpathlineto{\pgfqpoint{5.035127in}{2.671701in}}%
\pgfpathlineto{\pgfqpoint{5.048661in}{2.668894in}}%
\pgfpathlineto{\pgfqpoint{5.062201in}{2.666112in}}%
\pgfpathlineto{\pgfqpoint{5.075749in}{2.663356in}}%
\pgfpathlineto{\pgfqpoint{5.068429in}{2.656281in}}%
\pgfpathlineto{\pgfqpoint{5.061104in}{2.649255in}}%
\pgfpathlineto{\pgfqpoint{5.053775in}{2.642272in}}%
\pgfpathlineto{\pgfqpoint{5.046441in}{2.635329in}}%
\pgfpathlineto{\pgfqpoint{5.032878in}{2.637956in}}%
\pgfpathlineto{\pgfqpoint{5.019322in}{2.640610in}}%
\pgfpathlineto{\pgfqpoint{5.005772in}{2.643288in}}%
\pgfpathlineto{\pgfqpoint{4.992230in}{2.645993in}}%
\pgfpathlineto{\pgfqpoint{4.999579in}{2.653060in}}%
\pgfpathlineto{\pgfqpoint{5.006924in}{2.660170in}}%
\pgfpathlineto{\pgfqpoint{5.014264in}{2.667326in}}%
\pgfpathlineto{\pgfqpoint{5.021600in}{2.674534in}}%
\pgfpathclose%
\pgfusepath{fill}%
\end{pgfscope}%
\begin{pgfscope}%
\pgfpathrectangle{\pgfqpoint{1.150000in}{0.150000in}}{\pgfqpoint{5.700000in}{5.700000in}}%
\pgfusepath{clip}%
\pgfsetbuttcap%
\pgfsetroundjoin%
\definecolor{currentfill}{rgb}{0.277941,0.056324,0.381191}%
\pgfsetfillcolor{currentfill}%
\pgfsetfillopacity{0.700000}%
\pgfsetlinewidth{0.000000pt}%
\definecolor{currentstroke}{rgb}{0.000000,0.000000,0.000000}%
\pgfsetstrokecolor{currentstroke}%
\pgfsetdash{}{0pt}%
\pgfpathmoveto{\pgfqpoint{2.956469in}{2.652247in}}%
\pgfpathlineto{\pgfqpoint{2.969582in}{2.645897in}}%
\pgfpathlineto{\pgfqpoint{2.982699in}{2.639597in}}%
\pgfpathlineto{\pgfqpoint{2.995819in}{2.633345in}}%
\pgfpathlineto{\pgfqpoint{3.008943in}{2.627143in}}%
\pgfpathlineto{\pgfqpoint{3.000841in}{2.621010in}}%
\pgfpathlineto{\pgfqpoint{2.992731in}{2.614946in}}%
\pgfpathlineto{\pgfqpoint{2.984614in}{2.608950in}}%
\pgfpathlineto{\pgfqpoint{2.976489in}{2.603027in}}%
\pgfpathlineto{\pgfqpoint{2.963348in}{2.609325in}}%
\pgfpathlineto{\pgfqpoint{2.950212in}{2.615672in}}%
\pgfpathlineto{\pgfqpoint{2.937078in}{2.622068in}}%
\pgfpathlineto{\pgfqpoint{2.923948in}{2.628514in}}%
\pgfpathlineto{\pgfqpoint{2.932091in}{2.634337in}}%
\pgfpathlineto{\pgfqpoint{2.940225in}{2.640235in}}%
\pgfpathlineto{\pgfqpoint{2.948351in}{2.646206in}}%
\pgfpathlineto{\pgfqpoint{2.956469in}{2.652247in}}%
\pgfpathclose%
\pgfusepath{fill}%
\end{pgfscope}%
\begin{pgfscope}%
\pgfpathrectangle{\pgfqpoint{1.150000in}{0.150000in}}{\pgfqpoint{5.700000in}{5.700000in}}%
\pgfusepath{clip}%
\pgfsetbuttcap%
\pgfsetroundjoin%
\definecolor{currentfill}{rgb}{0.277941,0.056324,0.381191}%
\pgfsetfillcolor{currentfill}%
\pgfsetfillopacity{0.700000}%
\pgfsetlinewidth{0.000000pt}%
\definecolor{currentstroke}{rgb}{0.000000,0.000000,0.000000}%
\pgfsetstrokecolor{currentstroke}%
\pgfsetdash{}{0pt}%
\pgfpathmoveto{\pgfqpoint{4.800632in}{2.651155in}}%
\pgfpathlineto{\pgfqpoint{4.814108in}{2.648254in}}%
\pgfpathlineto{\pgfqpoint{4.827590in}{2.645380in}}%
\pgfpathlineto{\pgfqpoint{4.841078in}{2.642532in}}%
\pgfpathlineto{\pgfqpoint{4.854574in}{2.639711in}}%
\pgfpathlineto{\pgfqpoint{4.847171in}{2.632585in}}%
\pgfpathlineto{\pgfqpoint{4.839762in}{2.625485in}}%
\pgfpathlineto{\pgfqpoint{4.832349in}{2.618410in}}%
\pgfpathlineto{\pgfqpoint{4.824931in}{2.611354in}}%
\pgfpathlineto{\pgfqpoint{4.811421in}{2.614073in}}%
\pgfpathlineto{\pgfqpoint{4.797918in}{2.616819in}}%
\pgfpathlineto{\pgfqpoint{4.784421in}{2.619590in}}%
\pgfpathlineto{\pgfqpoint{4.770932in}{2.622389in}}%
\pgfpathlineto{\pgfqpoint{4.778364in}{2.629542in}}%
\pgfpathlineto{\pgfqpoint{4.785792in}{2.636719in}}%
\pgfpathlineto{\pgfqpoint{4.793215in}{2.643922in}}%
\pgfpathlineto{\pgfqpoint{4.800632in}{2.651155in}}%
\pgfpathclose%
\pgfusepath{fill}%
\end{pgfscope}%
\begin{pgfscope}%
\pgfpathrectangle{\pgfqpoint{1.150000in}{0.150000in}}{\pgfqpoint{5.700000in}{5.700000in}}%
\pgfusepath{clip}%
\pgfsetbuttcap%
\pgfsetroundjoin%
\definecolor{currentfill}{rgb}{0.271305,0.019942,0.347269}%
\pgfsetfillcolor{currentfill}%
\pgfsetfillopacity{0.700000}%
\pgfsetlinewidth{0.000000pt}%
\definecolor{currentstroke}{rgb}{0.000000,0.000000,0.000000}%
\pgfsetstrokecolor{currentstroke}%
\pgfsetdash{}{0pt}%
\pgfpathmoveto{\pgfqpoint{4.000317in}{2.586398in}}%
\pgfpathlineto{\pgfqpoint{4.013606in}{2.582678in}}%
\pgfpathlineto{\pgfqpoint{4.026900in}{2.578989in}}%
\pgfpathlineto{\pgfqpoint{4.040201in}{2.575332in}}%
\pgfpathlineto{\pgfqpoint{4.053507in}{2.571705in}}%
\pgfpathlineto{\pgfqpoint{4.045809in}{2.564164in}}%
\pgfpathlineto{\pgfqpoint{4.038105in}{2.556624in}}%
\pgfpathlineto{\pgfqpoint{4.030395in}{2.549084in}}%
\pgfpathlineto{\pgfqpoint{4.022680in}{2.541544in}}%
\pgfpathlineto{\pgfqpoint{4.009361in}{2.545160in}}%
\pgfpathlineto{\pgfqpoint{3.996048in}{2.548808in}}%
\pgfpathlineto{\pgfqpoint{3.982741in}{2.552486in}}%
\pgfpathlineto{\pgfqpoint{3.969440in}{2.556196in}}%
\pgfpathlineto{\pgfqpoint{3.977168in}{2.563741in}}%
\pgfpathlineto{\pgfqpoint{3.984890in}{2.571289in}}%
\pgfpathlineto{\pgfqpoint{3.992606in}{2.578841in}}%
\pgfpathlineto{\pgfqpoint{4.000317in}{2.586398in}}%
\pgfpathclose%
\pgfusepath{fill}%
\end{pgfscope}%
\begin{pgfscope}%
\pgfpathrectangle{\pgfqpoint{1.150000in}{0.150000in}}{\pgfqpoint{5.700000in}{5.700000in}}%
\pgfusepath{clip}%
\pgfsetbuttcap%
\pgfsetroundjoin%
\definecolor{currentfill}{rgb}{0.269944,0.014625,0.341379}%
\pgfsetfillcolor{currentfill}%
\pgfsetfillopacity{0.700000}%
\pgfsetlinewidth{0.000000pt}%
\definecolor{currentstroke}{rgb}{0.000000,0.000000,0.000000}%
\pgfsetstrokecolor{currentstroke}%
\pgfsetdash{}{0pt}%
\pgfpathmoveto{\pgfqpoint{3.642071in}{2.577338in}}%
\pgfpathlineto{\pgfqpoint{3.655287in}{2.572940in}}%
\pgfpathlineto{\pgfqpoint{3.668509in}{2.568579in}}%
\pgfpathlineto{\pgfqpoint{3.681735in}{2.564252in}}%
\pgfpathlineto{\pgfqpoint{3.694967in}{2.559961in}}%
\pgfpathlineto{\pgfqpoint{3.687138in}{2.552519in}}%
\pgfpathlineto{\pgfqpoint{3.679304in}{2.545091in}}%
\pgfpathlineto{\pgfqpoint{3.671463in}{2.537676in}}%
\pgfpathlineto{\pgfqpoint{3.663617in}{2.530275in}}%
\pgfpathlineto{\pgfqpoint{3.650372in}{2.534595in}}%
\pgfpathlineto{\pgfqpoint{3.637133in}{2.538951in}}%
\pgfpathlineto{\pgfqpoint{3.623899in}{2.543342in}}%
\pgfpathlineto{\pgfqpoint{3.610670in}{2.547768in}}%
\pgfpathlineto{\pgfqpoint{3.618529in}{2.555135in}}%
\pgfpathlineto{\pgfqpoint{3.626382in}{2.562520in}}%
\pgfpathlineto{\pgfqpoint{3.634230in}{2.569920in}}%
\pgfpathlineto{\pgfqpoint{3.642071in}{2.577338in}}%
\pgfpathclose%
\pgfusepath{fill}%
\end{pgfscope}%
\begin{pgfscope}%
\pgfpathrectangle{\pgfqpoint{1.150000in}{0.150000in}}{\pgfqpoint{5.700000in}{5.700000in}}%
\pgfusepath{clip}%
\pgfsetbuttcap%
\pgfsetroundjoin%
\definecolor{currentfill}{rgb}{0.276022,0.044167,0.370164}%
\pgfsetfillcolor{currentfill}%
\pgfsetfillopacity{0.700000}%
\pgfsetlinewidth{0.000000pt}%
\definecolor{currentstroke}{rgb}{0.000000,0.000000,0.000000}%
\pgfsetstrokecolor{currentstroke}%
\pgfsetdash{}{0pt}%
\pgfpathmoveto{\pgfqpoint{4.579630in}{2.628539in}}%
\pgfpathlineto{\pgfqpoint{4.593053in}{2.625507in}}%
\pgfpathlineto{\pgfqpoint{4.606483in}{2.622503in}}%
\pgfpathlineto{\pgfqpoint{4.619920in}{2.619525in}}%
\pgfpathlineto{\pgfqpoint{4.633363in}{2.616575in}}%
\pgfpathlineto{\pgfqpoint{4.625876in}{2.609329in}}%
\pgfpathlineto{\pgfqpoint{4.618384in}{2.602095in}}%
\pgfpathlineto{\pgfqpoint{4.610887in}{2.594870in}}%
\pgfpathlineto{\pgfqpoint{4.603384in}{2.587652in}}%
\pgfpathlineto{\pgfqpoint{4.589927in}{2.590526in}}%
\pgfpathlineto{\pgfqpoint{4.576477in}{2.593427in}}%
\pgfpathlineto{\pgfqpoint{4.563034in}{2.596355in}}%
\pgfpathlineto{\pgfqpoint{4.549597in}{2.599311in}}%
\pgfpathlineto{\pgfqpoint{4.557113in}{2.606601in}}%
\pgfpathlineto{\pgfqpoint{4.564624in}{2.613900in}}%
\pgfpathlineto{\pgfqpoint{4.572129in}{2.621212in}}%
\pgfpathlineto{\pgfqpoint{4.579630in}{2.628539in}}%
\pgfpathclose%
\pgfusepath{fill}%
\end{pgfscope}%
\begin{pgfscope}%
\pgfpathrectangle{\pgfqpoint{1.150000in}{0.150000in}}{\pgfqpoint{5.700000in}{5.700000in}}%
\pgfusepath{clip}%
\pgfsetbuttcap%
\pgfsetroundjoin%
\definecolor{currentfill}{rgb}{0.273809,0.031497,0.358853}%
\pgfsetfillcolor{currentfill}%
\pgfsetfillopacity{0.700000}%
\pgfsetlinewidth{0.000000pt}%
\definecolor{currentstroke}{rgb}{0.000000,0.000000,0.000000}%
\pgfsetstrokecolor{currentstroke}%
\pgfsetdash{}{0pt}%
\pgfpathmoveto{\pgfqpoint{4.358584in}{2.607087in}}%
\pgfpathlineto{\pgfqpoint{4.371956in}{2.603858in}}%
\pgfpathlineto{\pgfqpoint{4.385335in}{2.600657in}}%
\pgfpathlineto{\pgfqpoint{4.398721in}{2.597485in}}%
\pgfpathlineto{\pgfqpoint{4.412112in}{2.594341in}}%
\pgfpathlineto{\pgfqpoint{4.404543in}{2.586955in}}%
\pgfpathlineto{\pgfqpoint{4.396968in}{2.579571in}}%
\pgfpathlineto{\pgfqpoint{4.389387in}{2.572187in}}%
\pgfpathlineto{\pgfqpoint{4.381801in}{2.564803in}}%
\pgfpathlineto{\pgfqpoint{4.368397in}{2.567897in}}%
\pgfpathlineto{\pgfqpoint{4.354999in}{2.571020in}}%
\pgfpathlineto{\pgfqpoint{4.341607in}{2.574171in}}%
\pgfpathlineto{\pgfqpoint{4.328222in}{2.577350in}}%
\pgfpathlineto{\pgfqpoint{4.335820in}{2.584780in}}%
\pgfpathlineto{\pgfqpoint{4.343414in}{2.592211in}}%
\pgfpathlineto{\pgfqpoint{4.351001in}{2.599646in}}%
\pgfpathlineto{\pgfqpoint{4.358584in}{2.607087in}}%
\pgfpathclose%
\pgfusepath{fill}%
\end{pgfscope}%
\begin{pgfscope}%
\pgfpathrectangle{\pgfqpoint{1.150000in}{0.150000in}}{\pgfqpoint{5.700000in}{5.700000in}}%
\pgfusepath{clip}%
\pgfsetbuttcap%
\pgfsetroundjoin%
\definecolor{currentfill}{rgb}{0.281924,0.089666,0.412415}%
\pgfsetfillcolor{currentfill}%
\pgfsetfillopacity{0.700000}%
\pgfsetlinewidth{0.000000pt}%
\definecolor{currentstroke}{rgb}{0.000000,0.000000,0.000000}%
\pgfsetstrokecolor{currentstroke}%
\pgfsetdash{}{0pt}%
\pgfpathmoveto{\pgfqpoint{5.380222in}{2.705493in}}%
\pgfpathlineto{\pgfqpoint{5.393839in}{2.702696in}}%
\pgfpathlineto{\pgfqpoint{5.407464in}{2.699924in}}%
\pgfpathlineto{\pgfqpoint{5.421095in}{2.697176in}}%
\pgfpathlineto{\pgfqpoint{5.434734in}{2.694452in}}%
\pgfpathlineto{\pgfqpoint{5.427542in}{2.687310in}}%
\pgfpathlineto{\pgfqpoint{5.420347in}{2.680256in}}%
\pgfpathlineto{\pgfqpoint{5.413149in}{2.673286in}}%
\pgfpathlineto{\pgfqpoint{5.405947in}{2.666394in}}%
\pgfpathlineto{\pgfqpoint{5.392291in}{2.668950in}}%
\pgfpathlineto{\pgfqpoint{5.378641in}{2.671530in}}%
\pgfpathlineto{\pgfqpoint{5.364999in}{2.674135in}}%
\pgfpathlineto{\pgfqpoint{5.351364in}{2.676764in}}%
\pgfpathlineto{\pgfqpoint{5.358583in}{2.683819in}}%
\pgfpathlineto{\pgfqpoint{5.365799in}{2.690955in}}%
\pgfpathlineto{\pgfqpoint{5.373012in}{2.698178in}}%
\pgfpathlineto{\pgfqpoint{5.380222in}{2.705493in}}%
\pgfpathclose%
\pgfusepath{fill}%
\end{pgfscope}%
\begin{pgfscope}%
\pgfpathrectangle{\pgfqpoint{1.150000in}{0.150000in}}{\pgfqpoint{5.700000in}{5.700000in}}%
\pgfusepath{clip}%
\pgfsetbuttcap%
\pgfsetroundjoin%
\definecolor{currentfill}{rgb}{0.269944,0.014625,0.341379}%
\pgfsetfillcolor{currentfill}%
\pgfsetfillopacity{0.700000}%
\pgfsetlinewidth{0.000000pt}%
\definecolor{currentstroke}{rgb}{0.000000,0.000000,0.000000}%
\pgfsetstrokecolor{currentstroke}%
\pgfsetdash{}{0pt}%
\pgfpathmoveto{\pgfqpoint{3.779154in}{2.573102in}}%
\pgfpathlineto{\pgfqpoint{3.792400in}{2.568998in}}%
\pgfpathlineto{\pgfqpoint{3.805652in}{2.564928in}}%
\pgfpathlineto{\pgfqpoint{3.818909in}{2.560891in}}%
\pgfpathlineto{\pgfqpoint{3.832171in}{2.556887in}}%
\pgfpathlineto{\pgfqpoint{3.824391in}{2.549371in}}%
\pgfpathlineto{\pgfqpoint{3.816604in}{2.541861in}}%
\pgfpathlineto{\pgfqpoint{3.808812in}{2.534358in}}%
\pgfpathlineto{\pgfqpoint{3.801015in}{2.526861in}}%
\pgfpathlineto{\pgfqpoint{3.787740in}{2.530881in}}%
\pgfpathlineto{\pgfqpoint{3.774470in}{2.534933in}}%
\pgfpathlineto{\pgfqpoint{3.761206in}{2.539020in}}%
\pgfpathlineto{\pgfqpoint{3.747947in}{2.543140in}}%
\pgfpathlineto{\pgfqpoint{3.755758in}{2.550616in}}%
\pgfpathlineto{\pgfqpoint{3.763562in}{2.558101in}}%
\pgfpathlineto{\pgfqpoint{3.771361in}{2.565597in}}%
\pgfpathlineto{\pgfqpoint{3.779154in}{2.573102in}}%
\pgfpathclose%
\pgfusepath{fill}%
\end{pgfscope}%
\begin{pgfscope}%
\pgfpathrectangle{\pgfqpoint{1.150000in}{0.150000in}}{\pgfqpoint{5.700000in}{5.700000in}}%
\pgfusepath{clip}%
\pgfsetbuttcap%
\pgfsetroundjoin%
\definecolor{currentfill}{rgb}{0.280894,0.078907,0.402329}%
\pgfsetfillcolor{currentfill}%
\pgfsetfillopacity{0.700000}%
\pgfsetlinewidth{0.000000pt}%
\definecolor{currentstroke}{rgb}{0.000000,0.000000,0.000000}%
\pgfsetstrokecolor{currentstroke}%
\pgfsetdash{}{0pt}%
\pgfpathmoveto{\pgfqpoint{5.159185in}{2.680877in}}%
\pgfpathlineto{\pgfqpoint{5.172752in}{2.678105in}}%
\pgfpathlineto{\pgfqpoint{5.186325in}{2.675358in}}%
\pgfpathlineto{\pgfqpoint{5.199906in}{2.672636in}}%
\pgfpathlineto{\pgfqpoint{5.213494in}{2.669940in}}%
\pgfpathlineto{\pgfqpoint{5.206223in}{2.662924in}}%
\pgfpathlineto{\pgfqpoint{5.198947in}{2.655968in}}%
\pgfpathlineto{\pgfqpoint{5.191668in}{2.649065in}}%
\pgfpathlineto{\pgfqpoint{5.184384in}{2.642213in}}%
\pgfpathlineto{\pgfqpoint{5.170780in}{2.644768in}}%
\pgfpathlineto{\pgfqpoint{5.157183in}{2.647348in}}%
\pgfpathlineto{\pgfqpoint{5.143593in}{2.649953in}}%
\pgfpathlineto{\pgfqpoint{5.130010in}{2.652583in}}%
\pgfpathlineto{\pgfqpoint{5.137310in}{2.659572in}}%
\pgfpathlineto{\pgfqpoint{5.144606in}{2.666615in}}%
\pgfpathlineto{\pgfqpoint{5.151897in}{2.673715in}}%
\pgfpathlineto{\pgfqpoint{5.159185in}{2.680877in}}%
\pgfpathclose%
\pgfusepath{fill}%
\end{pgfscope}%
\begin{pgfscope}%
\pgfpathrectangle{\pgfqpoint{1.150000in}{0.150000in}}{\pgfqpoint{5.700000in}{5.700000in}}%
\pgfusepath{clip}%
\pgfsetbuttcap%
\pgfsetroundjoin%
\definecolor{currentfill}{rgb}{0.271305,0.019942,0.347269}%
\pgfsetfillcolor{currentfill}%
\pgfsetfillopacity{0.700000}%
\pgfsetlinewidth{0.000000pt}%
\definecolor{currentstroke}{rgb}{0.000000,0.000000,0.000000}%
\pgfsetstrokecolor{currentstroke}%
\pgfsetdash{}{0pt}%
\pgfpathmoveto{\pgfqpoint{4.137481in}{2.587615in}}%
\pgfpathlineto{\pgfqpoint{4.150805in}{2.584118in}}%
\pgfpathlineto{\pgfqpoint{4.164135in}{2.580652in}}%
\pgfpathlineto{\pgfqpoint{4.177471in}{2.577216in}}%
\pgfpathlineto{\pgfqpoint{4.190813in}{2.573809in}}%
\pgfpathlineto{\pgfqpoint{4.183161in}{2.566308in}}%
\pgfpathlineto{\pgfqpoint{4.175504in}{2.558807in}}%
\pgfpathlineto{\pgfqpoint{4.167842in}{2.551303in}}%
\pgfpathlineto{\pgfqpoint{4.160174in}{2.543796in}}%
\pgfpathlineto{\pgfqpoint{4.146819in}{2.547179in}}%
\pgfpathlineto{\pgfqpoint{4.133471in}{2.550592in}}%
\pgfpathlineto{\pgfqpoint{4.120129in}{2.554035in}}%
\pgfpathlineto{\pgfqpoint{4.106792in}{2.557508in}}%
\pgfpathlineto{\pgfqpoint{4.114473in}{2.565033in}}%
\pgfpathlineto{\pgfqpoint{4.122148in}{2.572559in}}%
\pgfpathlineto{\pgfqpoint{4.129817in}{2.580086in}}%
\pgfpathlineto{\pgfqpoint{4.137481in}{2.587615in}}%
\pgfpathclose%
\pgfusepath{fill}%
\end{pgfscope}%
\begin{pgfscope}%
\pgfpathrectangle{\pgfqpoint{1.150000in}{0.150000in}}{\pgfqpoint{5.700000in}{5.700000in}}%
\pgfusepath{clip}%
\pgfsetbuttcap%
\pgfsetroundjoin%
\definecolor{currentfill}{rgb}{0.279566,0.067836,0.391917}%
\pgfsetfillcolor{currentfill}%
\pgfsetfillopacity{0.700000}%
\pgfsetlinewidth{0.000000pt}%
\definecolor{currentstroke}{rgb}{0.000000,0.000000,0.000000}%
\pgfsetstrokecolor{currentstroke}%
\pgfsetdash{}{0pt}%
\pgfpathmoveto{\pgfqpoint{4.938129in}{2.657066in}}%
\pgfpathlineto{\pgfqpoint{4.951644in}{2.654259in}}%
\pgfpathlineto{\pgfqpoint{4.965166in}{2.651478in}}%
\pgfpathlineto{\pgfqpoint{4.978694in}{2.648723in}}%
\pgfpathlineto{\pgfqpoint{4.992230in}{2.645993in}}%
\pgfpathlineto{\pgfqpoint{4.984876in}{2.638964in}}%
\pgfpathlineto{\pgfqpoint{4.977517in}{2.631971in}}%
\pgfpathlineto{\pgfqpoint{4.970154in}{2.625008in}}%
\pgfpathlineto{\pgfqpoint{4.962785in}{2.618073in}}%
\pgfpathlineto{\pgfqpoint{4.949234in}{2.620688in}}%
\pgfpathlineto{\pgfqpoint{4.935691in}{2.623328in}}%
\pgfpathlineto{\pgfqpoint{4.922154in}{2.625993in}}%
\pgfpathlineto{\pgfqpoint{4.908624in}{2.628685in}}%
\pgfpathlineto{\pgfqpoint{4.916008in}{2.635731in}}%
\pgfpathlineto{\pgfqpoint{4.923386in}{2.642807in}}%
\pgfpathlineto{\pgfqpoint{4.930760in}{2.649918in}}%
\pgfpathlineto{\pgfqpoint{4.938129in}{2.657066in}}%
\pgfpathclose%
\pgfusepath{fill}%
\end{pgfscope}%
\begin{pgfscope}%
\pgfpathrectangle{\pgfqpoint{1.150000in}{0.150000in}}{\pgfqpoint{5.700000in}{5.700000in}}%
\pgfusepath{clip}%
\pgfsetbuttcap%
\pgfsetroundjoin%
\definecolor{currentfill}{rgb}{0.271305,0.019942,0.347269}%
\pgfsetfillcolor{currentfill}%
\pgfsetfillopacity{0.700000}%
\pgfsetlinewidth{0.000000pt}%
\definecolor{currentstroke}{rgb}{0.000000,0.000000,0.000000}%
\pgfsetstrokecolor{currentstroke}%
\pgfsetdash{}{0pt}%
\pgfpathmoveto{\pgfqpoint{3.283491in}{2.587850in}}%
\pgfpathlineto{\pgfqpoint{3.296652in}{2.582559in}}%
\pgfpathlineto{\pgfqpoint{3.309817in}{2.577310in}}%
\pgfpathlineto{\pgfqpoint{3.322986in}{2.572101in}}%
\pgfpathlineto{\pgfqpoint{3.336160in}{2.566933in}}%
\pgfpathlineto{\pgfqpoint{3.328191in}{2.560003in}}%
\pgfpathlineto{\pgfqpoint{3.320215in}{2.553110in}}%
\pgfpathlineto{\pgfqpoint{3.312232in}{2.546257in}}%
\pgfpathlineto{\pgfqpoint{3.304243in}{2.539445in}}%
\pgfpathlineto{\pgfqpoint{3.291055in}{2.544682in}}%
\pgfpathlineto{\pgfqpoint{3.277871in}{2.549960in}}%
\pgfpathlineto{\pgfqpoint{3.264692in}{2.555278in}}%
\pgfpathlineto{\pgfqpoint{3.251517in}{2.560638in}}%
\pgfpathlineto{\pgfqpoint{3.259521in}{2.567376in}}%
\pgfpathlineto{\pgfqpoint{3.267518in}{2.574158in}}%
\pgfpathlineto{\pgfqpoint{3.275508in}{2.580983in}}%
\pgfpathlineto{\pgfqpoint{3.283491in}{2.587850in}}%
\pgfpathclose%
\pgfusepath{fill}%
\end{pgfscope}%
\begin{pgfscope}%
\pgfpathrectangle{\pgfqpoint{1.150000in}{0.150000in}}{\pgfqpoint{5.700000in}{5.700000in}}%
\pgfusepath{clip}%
\pgfsetbuttcap%
\pgfsetroundjoin%
\definecolor{currentfill}{rgb}{0.273809,0.031497,0.358853}%
\pgfsetfillcolor{currentfill}%
\pgfsetfillopacity{0.700000}%
\pgfsetlinewidth{0.000000pt}%
\definecolor{currentstroke}{rgb}{0.000000,0.000000,0.000000}%
\pgfsetstrokecolor{currentstroke}%
\pgfsetdash{}{0pt}%
\pgfpathmoveto{\pgfqpoint{3.146268in}{2.605043in}}%
\pgfpathlineto{\pgfqpoint{3.159410in}{2.599340in}}%
\pgfpathlineto{\pgfqpoint{3.172556in}{2.593682in}}%
\pgfpathlineto{\pgfqpoint{3.185706in}{2.588067in}}%
\pgfpathlineto{\pgfqpoint{3.198860in}{2.582496in}}%
\pgfpathlineto{\pgfqpoint{3.190834in}{2.575880in}}%
\pgfpathlineto{\pgfqpoint{3.182801in}{2.569315in}}%
\pgfpathlineto{\pgfqpoint{3.174762in}{2.562802in}}%
\pgfpathlineto{\pgfqpoint{3.166715in}{2.556342in}}%
\pgfpathlineto{\pgfqpoint{3.153546in}{2.561996in}}%
\pgfpathlineto{\pgfqpoint{3.140381in}{2.567692in}}%
\pgfpathlineto{\pgfqpoint{3.127220in}{2.573433in}}%
\pgfpathlineto{\pgfqpoint{3.114063in}{2.579218in}}%
\pgfpathlineto{\pgfqpoint{3.122125in}{2.585591in}}%
\pgfpathlineto{\pgfqpoint{3.130180in}{2.592020in}}%
\pgfpathlineto{\pgfqpoint{3.138228in}{2.598505in}}%
\pgfpathlineto{\pgfqpoint{3.146268in}{2.605043in}}%
\pgfpathclose%
\pgfusepath{fill}%
\end{pgfscope}%
\begin{pgfscope}%
\pgfpathrectangle{\pgfqpoint{1.150000in}{0.150000in}}{\pgfqpoint{5.700000in}{5.700000in}}%
\pgfusepath{clip}%
\pgfsetbuttcap%
\pgfsetroundjoin%
\definecolor{currentfill}{rgb}{0.277018,0.050344,0.375715}%
\pgfsetfillcolor{currentfill}%
\pgfsetfillopacity{0.700000}%
\pgfsetlinewidth{0.000000pt}%
\definecolor{currentstroke}{rgb}{0.000000,0.000000,0.000000}%
\pgfsetstrokecolor{currentstroke}%
\pgfsetdash{}{0pt}%
\pgfpathmoveto{\pgfqpoint{4.717041in}{2.633847in}}%
\pgfpathlineto{\pgfqpoint{4.730504in}{2.630942in}}%
\pgfpathlineto{\pgfqpoint{4.743973in}{2.628064in}}%
\pgfpathlineto{\pgfqpoint{4.757449in}{2.625213in}}%
\pgfpathlineto{\pgfqpoint{4.770932in}{2.622389in}}%
\pgfpathlineto{\pgfqpoint{4.763494in}{2.615255in}}%
\pgfpathlineto{\pgfqpoint{4.756051in}{2.608139in}}%
\pgfpathlineto{\pgfqpoint{4.748602in}{2.601036in}}%
\pgfpathlineto{\pgfqpoint{4.741149in}{2.593945in}}%
\pgfpathlineto{\pgfqpoint{4.727652in}{2.596680in}}%
\pgfpathlineto{\pgfqpoint{4.714162in}{2.599442in}}%
\pgfpathlineto{\pgfqpoint{4.700678in}{2.602230in}}%
\pgfpathlineto{\pgfqpoint{4.687202in}{2.605046in}}%
\pgfpathlineto{\pgfqpoint{4.694670in}{2.612222in}}%
\pgfpathlineto{\pgfqpoint{4.702132in}{2.619412in}}%
\pgfpathlineto{\pgfqpoint{4.709589in}{2.626619in}}%
\pgfpathlineto{\pgfqpoint{4.717041in}{2.633847in}}%
\pgfpathclose%
\pgfusepath{fill}%
\end{pgfscope}%
\begin{pgfscope}%
\pgfpathrectangle{\pgfqpoint{1.150000in}{0.150000in}}{\pgfqpoint{5.700000in}{5.700000in}}%
\pgfusepath{clip}%
\pgfsetbuttcap%
\pgfsetroundjoin%
\definecolor{currentfill}{rgb}{0.269944,0.014625,0.341379}%
\pgfsetfillcolor{currentfill}%
\pgfsetfillopacity{0.700000}%
\pgfsetlinewidth{0.000000pt}%
\definecolor{currentstroke}{rgb}{0.000000,0.000000,0.000000}%
\pgfsetstrokecolor{currentstroke}%
\pgfsetdash{}{0pt}%
\pgfpathmoveto{\pgfqpoint{3.420657in}{2.574963in}}%
\pgfpathlineto{\pgfqpoint{3.433840in}{2.570048in}}%
\pgfpathlineto{\pgfqpoint{3.447028in}{2.565173in}}%
\pgfpathlineto{\pgfqpoint{3.460220in}{2.560335in}}%
\pgfpathlineto{\pgfqpoint{3.473418in}{2.555536in}}%
\pgfpathlineto{\pgfqpoint{3.465502in}{2.548363in}}%
\pgfpathlineto{\pgfqpoint{3.457579in}{2.541217in}}%
\pgfpathlineto{\pgfqpoint{3.449651in}{2.534100in}}%
\pgfpathlineto{\pgfqpoint{3.441716in}{2.527011in}}%
\pgfpathlineto{\pgfqpoint{3.428505in}{2.531866in}}%
\pgfpathlineto{\pgfqpoint{3.415299in}{2.536759in}}%
\pgfpathlineto{\pgfqpoint{3.402097in}{2.541690in}}%
\pgfpathlineto{\pgfqpoint{3.388901in}{2.546660in}}%
\pgfpathlineto{\pgfqpoint{3.396849in}{2.553688in}}%
\pgfpathlineto{\pgfqpoint{3.404792in}{2.560748in}}%
\pgfpathlineto{\pgfqpoint{3.412728in}{2.567840in}}%
\pgfpathlineto{\pgfqpoint{3.420657in}{2.574963in}}%
\pgfpathclose%
\pgfusepath{fill}%
\end{pgfscope}%
\begin{pgfscope}%
\pgfpathrectangle{\pgfqpoint{1.150000in}{0.150000in}}{\pgfqpoint{5.700000in}{5.700000in}}%
\pgfusepath{clip}%
\pgfsetbuttcap%
\pgfsetroundjoin%
\definecolor{currentfill}{rgb}{0.269944,0.014625,0.341379}%
\pgfsetfillcolor{currentfill}%
\pgfsetfillopacity{0.700000}%
\pgfsetlinewidth{0.000000pt}%
\definecolor{currentstroke}{rgb}{0.000000,0.000000,0.000000}%
\pgfsetstrokecolor{currentstroke}%
\pgfsetdash{}{0pt}%
\pgfpathmoveto{\pgfqpoint{3.916293in}{2.571353in}}%
\pgfpathlineto{\pgfqpoint{3.929571in}{2.567516in}}%
\pgfpathlineto{\pgfqpoint{3.942855in}{2.563711in}}%
\pgfpathlineto{\pgfqpoint{3.956145in}{2.559937in}}%
\pgfpathlineto{\pgfqpoint{3.969440in}{2.556196in}}%
\pgfpathlineto{\pgfqpoint{3.961707in}{2.548653in}}%
\pgfpathlineto{\pgfqpoint{3.953968in}{2.541112in}}%
\pgfpathlineto{\pgfqpoint{3.946224in}{2.533572in}}%
\pgfpathlineto{\pgfqpoint{3.938474in}{2.526033in}}%
\pgfpathlineto{\pgfqpoint{3.925166in}{2.529777in}}%
\pgfpathlineto{\pgfqpoint{3.911864in}{2.533553in}}%
\pgfpathlineto{\pgfqpoint{3.898568in}{2.537361in}}%
\pgfpathlineto{\pgfqpoint{3.885277in}{2.541202in}}%
\pgfpathlineto{\pgfqpoint{3.893039in}{2.548733in}}%
\pgfpathlineto{\pgfqpoint{3.900796in}{2.556268in}}%
\pgfpathlineto{\pgfqpoint{3.908547in}{2.563808in}}%
\pgfpathlineto{\pgfqpoint{3.916293in}{2.571353in}}%
\pgfpathclose%
\pgfusepath{fill}%
\end{pgfscope}%
\begin{pgfscope}%
\pgfpathrectangle{\pgfqpoint{1.150000in}{0.150000in}}{\pgfqpoint{5.700000in}{5.700000in}}%
\pgfusepath{clip}%
\pgfsetbuttcap%
\pgfsetroundjoin%
\definecolor{currentfill}{rgb}{0.276022,0.044167,0.370164}%
\pgfsetfillcolor{currentfill}%
\pgfsetfillopacity{0.700000}%
\pgfsetlinewidth{0.000000pt}%
\definecolor{currentstroke}{rgb}{0.000000,0.000000,0.000000}%
\pgfsetstrokecolor{currentstroke}%
\pgfsetdash{}{0pt}%
\pgfpathmoveto{\pgfqpoint{3.008943in}{2.627143in}}%
\pgfpathlineto{\pgfqpoint{3.022070in}{2.620988in}}%
\pgfpathlineto{\pgfqpoint{3.035200in}{2.614882in}}%
\pgfpathlineto{\pgfqpoint{3.048335in}{2.608822in}}%
\pgfpathlineto{\pgfqpoint{3.061473in}{2.602809in}}%
\pgfpathlineto{\pgfqpoint{3.053387in}{2.596587in}}%
\pgfpathlineto{\pgfqpoint{3.045294in}{2.590428in}}%
\pgfpathlineto{\pgfqpoint{3.037193in}{2.584335in}}%
\pgfpathlineto{\pgfqpoint{3.029084in}{2.578311in}}%
\pgfpathlineto{\pgfqpoint{3.015930in}{2.584419in}}%
\pgfpathlineto{\pgfqpoint{3.002779in}{2.590574in}}%
\pgfpathlineto{\pgfqpoint{2.989632in}{2.596777in}}%
\pgfpathlineto{\pgfqpoint{2.976489in}{2.603027in}}%
\pgfpathlineto{\pgfqpoint{2.984614in}{2.608950in}}%
\pgfpathlineto{\pgfqpoint{2.992731in}{2.614946in}}%
\pgfpathlineto{\pgfqpoint{3.000841in}{2.621010in}}%
\pgfpathlineto{\pgfqpoint{3.008943in}{2.627143in}}%
\pgfpathclose%
\pgfusepath{fill}%
\end{pgfscope}%
\begin{pgfscope}%
\pgfpathrectangle{\pgfqpoint{1.150000in}{0.150000in}}{\pgfqpoint{5.700000in}{5.700000in}}%
\pgfusepath{clip}%
\pgfsetbuttcap%
\pgfsetroundjoin%
\definecolor{currentfill}{rgb}{0.274952,0.037752,0.364543}%
\pgfsetfillcolor{currentfill}%
\pgfsetfillopacity{0.700000}%
\pgfsetlinewidth{0.000000pt}%
\definecolor{currentstroke}{rgb}{0.000000,0.000000,0.000000}%
\pgfsetstrokecolor{currentstroke}%
\pgfsetdash{}{0pt}%
\pgfpathmoveto{\pgfqpoint{4.495915in}{2.611410in}}%
\pgfpathlineto{\pgfqpoint{4.509326in}{2.608344in}}%
\pgfpathlineto{\pgfqpoint{4.522743in}{2.605305in}}%
\pgfpathlineto{\pgfqpoint{4.536167in}{2.602294in}}%
\pgfpathlineto{\pgfqpoint{4.549597in}{2.599311in}}%
\pgfpathlineto{\pgfqpoint{4.542075in}{2.592028in}}%
\pgfpathlineto{\pgfqpoint{4.534548in}{2.584751in}}%
\pgfpathlineto{\pgfqpoint{4.527016in}{2.577475in}}%
\pgfpathlineto{\pgfqpoint{4.519478in}{2.570201in}}%
\pgfpathlineto{\pgfqpoint{4.506034in}{2.573121in}}%
\pgfpathlineto{\pgfqpoint{4.492598in}{2.576069in}}%
\pgfpathlineto{\pgfqpoint{4.479167in}{2.579044in}}%
\pgfpathlineto{\pgfqpoint{4.465743in}{2.582047in}}%
\pgfpathlineto{\pgfqpoint{4.473294in}{2.589381in}}%
\pgfpathlineto{\pgfqpoint{4.480840in}{2.596717in}}%
\pgfpathlineto{\pgfqpoint{4.488380in}{2.604060in}}%
\pgfpathlineto{\pgfqpoint{4.495915in}{2.611410in}}%
\pgfpathclose%
\pgfusepath{fill}%
\end{pgfscope}%
\begin{pgfscope}%
\pgfpathrectangle{\pgfqpoint{1.150000in}{0.150000in}}{\pgfqpoint{5.700000in}{5.700000in}}%
\pgfusepath{clip}%
\pgfsetbuttcap%
\pgfsetroundjoin%
\definecolor{currentfill}{rgb}{0.269944,0.014625,0.341379}%
\pgfsetfillcolor{currentfill}%
\pgfsetfillopacity{0.700000}%
\pgfsetlinewidth{0.000000pt}%
\definecolor{currentstroke}{rgb}{0.000000,0.000000,0.000000}%
\pgfsetstrokecolor{currentstroke}%
\pgfsetdash{}{0pt}%
\pgfpathmoveto{\pgfqpoint{3.557805in}{2.565833in}}%
\pgfpathlineto{\pgfqpoint{3.571013in}{2.561263in}}%
\pgfpathlineto{\pgfqpoint{3.584227in}{2.556729in}}%
\pgfpathlineto{\pgfqpoint{3.597446in}{2.552231in}}%
\pgfpathlineto{\pgfqpoint{3.610670in}{2.547768in}}%
\pgfpathlineto{\pgfqpoint{3.602804in}{2.540419in}}%
\pgfpathlineto{\pgfqpoint{3.594933in}{2.533087in}}%
\pgfpathlineto{\pgfqpoint{3.587056in}{2.525774in}}%
\pgfpathlineto{\pgfqpoint{3.579173in}{2.518480in}}%
\pgfpathlineto{\pgfqpoint{3.565936in}{2.522984in}}%
\pgfpathlineto{\pgfqpoint{3.552704in}{2.527525in}}%
\pgfpathlineto{\pgfqpoint{3.539477in}{2.532101in}}%
\pgfpathlineto{\pgfqpoint{3.526255in}{2.536714in}}%
\pgfpathlineto{\pgfqpoint{3.534152in}{2.543961in}}%
\pgfpathlineto{\pgfqpoint{3.542042in}{2.551230in}}%
\pgfpathlineto{\pgfqpoint{3.549927in}{2.558521in}}%
\pgfpathlineto{\pgfqpoint{3.557805in}{2.565833in}}%
\pgfpathclose%
\pgfusepath{fill}%
\end{pgfscope}%
\begin{pgfscope}%
\pgfpathrectangle{\pgfqpoint{1.150000in}{0.150000in}}{\pgfqpoint{5.700000in}{5.700000in}}%
\pgfusepath{clip}%
\pgfsetbuttcap%
\pgfsetroundjoin%
\definecolor{currentfill}{rgb}{0.272594,0.025563,0.353093}%
\pgfsetfillcolor{currentfill}%
\pgfsetfillopacity{0.700000}%
\pgfsetlinewidth{0.000000pt}%
\definecolor{currentstroke}{rgb}{0.000000,0.000000,0.000000}%
\pgfsetstrokecolor{currentstroke}%
\pgfsetdash{}{0pt}%
\pgfpathmoveto{\pgfqpoint{4.274743in}{2.590358in}}%
\pgfpathlineto{\pgfqpoint{4.288103in}{2.587062in}}%
\pgfpathlineto{\pgfqpoint{4.301470in}{2.583796in}}%
\pgfpathlineto{\pgfqpoint{4.314843in}{2.580559in}}%
\pgfpathlineto{\pgfqpoint{4.328222in}{2.577350in}}%
\pgfpathlineto{\pgfqpoint{4.320617in}{2.569921in}}%
\pgfpathlineto{\pgfqpoint{4.313008in}{2.562491in}}%
\pgfpathlineto{\pgfqpoint{4.305392in}{2.555057in}}%
\pgfpathlineto{\pgfqpoint{4.297772in}{2.547619in}}%
\pgfpathlineto{\pgfqpoint{4.284380in}{2.550791in}}%
\pgfpathlineto{\pgfqpoint{4.270994in}{2.553991in}}%
\pgfpathlineto{\pgfqpoint{4.257615in}{2.557221in}}%
\pgfpathlineto{\pgfqpoint{4.244242in}{2.560480in}}%
\pgfpathlineto{\pgfqpoint{4.251876in}{2.567950in}}%
\pgfpathlineto{\pgfqpoint{4.259504in}{2.575418in}}%
\pgfpathlineto{\pgfqpoint{4.267126in}{2.582887in}}%
\pgfpathlineto{\pgfqpoint{4.274743in}{2.590358in}}%
\pgfpathclose%
\pgfusepath{fill}%
\end{pgfscope}%
\begin{pgfscope}%
\pgfpathrectangle{\pgfqpoint{1.150000in}{0.150000in}}{\pgfqpoint{5.700000in}{5.700000in}}%
\pgfusepath{clip}%
\pgfsetbuttcap%
\pgfsetroundjoin%
\definecolor{currentfill}{rgb}{0.281446,0.084320,0.407414}%
\pgfsetfillcolor{currentfill}%
\pgfsetfillopacity{0.700000}%
\pgfsetlinewidth{0.000000pt}%
\definecolor{currentstroke}{rgb}{0.000000,0.000000,0.000000}%
\pgfsetstrokecolor{currentstroke}%
\pgfsetdash{}{0pt}%
\pgfpathmoveto{\pgfqpoint{5.296896in}{2.687526in}}%
\pgfpathlineto{\pgfqpoint{5.310502in}{2.684798in}}%
\pgfpathlineto{\pgfqpoint{5.324116in}{2.682095in}}%
\pgfpathlineto{\pgfqpoint{5.337736in}{2.679417in}}%
\pgfpathlineto{\pgfqpoint{5.351364in}{2.676764in}}%
\pgfpathlineto{\pgfqpoint{5.344142in}{2.669785in}}%
\pgfpathlineto{\pgfqpoint{5.336916in}{2.662878in}}%
\pgfpathlineto{\pgfqpoint{5.329686in}{2.656037in}}%
\pgfpathlineto{\pgfqpoint{5.322453in}{2.649259in}}%
\pgfpathlineto{\pgfqpoint{5.308808in}{2.651757in}}%
\pgfpathlineto{\pgfqpoint{5.295170in}{2.654281in}}%
\pgfpathlineto{\pgfqpoint{5.281539in}{2.656828in}}%
\pgfpathlineto{\pgfqpoint{5.267916in}{2.659401in}}%
\pgfpathlineto{\pgfqpoint{5.275167in}{2.666330in}}%
\pgfpathlineto{\pgfqpoint{5.282413in}{2.673324in}}%
\pgfpathlineto{\pgfqpoint{5.289656in}{2.680387in}}%
\pgfpathlineto{\pgfqpoint{5.296896in}{2.687526in}}%
\pgfpathclose%
\pgfusepath{fill}%
\end{pgfscope}%
\begin{pgfscope}%
\pgfpathrectangle{\pgfqpoint{1.150000in}{0.150000in}}{\pgfqpoint{5.700000in}{5.700000in}}%
\pgfusepath{clip}%
\pgfsetbuttcap%
\pgfsetroundjoin%
\definecolor{currentfill}{rgb}{0.279566,0.067836,0.391917}%
\pgfsetfillcolor{currentfill}%
\pgfsetfillopacity{0.700000}%
\pgfsetlinewidth{0.000000pt}%
\definecolor{currentstroke}{rgb}{0.000000,0.000000,0.000000}%
\pgfsetstrokecolor{currentstroke}%
\pgfsetdash{}{0pt}%
\pgfpathmoveto{\pgfqpoint{2.871461in}{2.654805in}}%
\pgfpathlineto{\pgfqpoint{2.884578in}{2.648155in}}%
\pgfpathlineto{\pgfqpoint{2.897698in}{2.641557in}}%
\pgfpathlineto{\pgfqpoint{2.910822in}{2.635010in}}%
\pgfpathlineto{\pgfqpoint{2.923948in}{2.628514in}}%
\pgfpathlineto{\pgfqpoint{2.915798in}{2.622768in}}%
\pgfpathlineto{\pgfqpoint{2.907639in}{2.617102in}}%
\pgfpathlineto{\pgfqpoint{2.899472in}{2.611517in}}%
\pgfpathlineto{\pgfqpoint{2.891297in}{2.606017in}}%
\pgfpathlineto{\pgfqpoint{2.878152in}{2.612622in}}%
\pgfpathlineto{\pgfqpoint{2.865011in}{2.619278in}}%
\pgfpathlineto{\pgfqpoint{2.851873in}{2.625985in}}%
\pgfpathlineto{\pgfqpoint{2.838738in}{2.632744in}}%
\pgfpathlineto{\pgfqpoint{2.846932in}{2.638130in}}%
\pgfpathlineto{\pgfqpoint{2.855117in}{2.643604in}}%
\pgfpathlineto{\pgfqpoint{2.863293in}{2.649163in}}%
\pgfpathlineto{\pgfqpoint{2.871461in}{2.654805in}}%
\pgfpathclose%
\pgfusepath{fill}%
\end{pgfscope}%
\begin{pgfscope}%
\pgfpathrectangle{\pgfqpoint{1.150000in}{0.150000in}}{\pgfqpoint{5.700000in}{5.700000in}}%
\pgfusepath{clip}%
\pgfsetbuttcap%
\pgfsetroundjoin%
\definecolor{currentfill}{rgb}{0.280267,0.073417,0.397163}%
\pgfsetfillcolor{currentfill}%
\pgfsetfillopacity{0.700000}%
\pgfsetlinewidth{0.000000pt}%
\definecolor{currentstroke}{rgb}{0.000000,0.000000,0.000000}%
\pgfsetstrokecolor{currentstroke}%
\pgfsetdash{}{0pt}%
\pgfpathmoveto{\pgfqpoint{5.075749in}{2.663356in}}%
\pgfpathlineto{\pgfqpoint{5.089304in}{2.660625in}}%
\pgfpathlineto{\pgfqpoint{5.102866in}{2.657919in}}%
\pgfpathlineto{\pgfqpoint{5.116434in}{2.655238in}}%
\pgfpathlineto{\pgfqpoint{5.130010in}{2.652583in}}%
\pgfpathlineto{\pgfqpoint{5.122706in}{2.645642in}}%
\pgfpathlineto{\pgfqpoint{5.115397in}{2.638746in}}%
\pgfpathlineto{\pgfqpoint{5.108084in}{2.631891in}}%
\pgfpathlineto{\pgfqpoint{5.100766in}{2.625071in}}%
\pgfpathlineto{\pgfqpoint{5.087174in}{2.627598in}}%
\pgfpathlineto{\pgfqpoint{5.073589in}{2.630149in}}%
\pgfpathlineto{\pgfqpoint{5.060012in}{2.632726in}}%
\pgfpathlineto{\pgfqpoint{5.046441in}{2.635329in}}%
\pgfpathlineto{\pgfqpoint{5.053775in}{2.642272in}}%
\pgfpathlineto{\pgfqpoint{5.061104in}{2.649255in}}%
\pgfpathlineto{\pgfqpoint{5.068429in}{2.656281in}}%
\pgfpathlineto{\pgfqpoint{5.075749in}{2.663356in}}%
\pgfpathclose%
\pgfusepath{fill}%
\end{pgfscope}%
\begin{pgfscope}%
\pgfpathrectangle{\pgfqpoint{1.150000in}{0.150000in}}{\pgfqpoint{5.700000in}{5.700000in}}%
\pgfusepath{clip}%
\pgfsetbuttcap%
\pgfsetroundjoin%
\definecolor{currentfill}{rgb}{0.268510,0.009605,0.335427}%
\pgfsetfillcolor{currentfill}%
\pgfsetfillopacity{0.700000}%
\pgfsetlinewidth{0.000000pt}%
\definecolor{currentstroke}{rgb}{0.000000,0.000000,0.000000}%
\pgfsetstrokecolor{currentstroke}%
\pgfsetdash{}{0pt}%
\pgfpathmoveto{\pgfqpoint{3.694967in}{2.559961in}}%
\pgfpathlineto{\pgfqpoint{3.708204in}{2.555704in}}%
\pgfpathlineto{\pgfqpoint{3.721446in}{2.551482in}}%
\pgfpathlineto{\pgfqpoint{3.734694in}{2.547294in}}%
\pgfpathlineto{\pgfqpoint{3.747947in}{2.543140in}}%
\pgfpathlineto{\pgfqpoint{3.740131in}{2.535674in}}%
\pgfpathlineto{\pgfqpoint{3.732309in}{2.528218in}}%
\pgfpathlineto{\pgfqpoint{3.724482in}{2.520772in}}%
\pgfpathlineto{\pgfqpoint{3.716648in}{2.513337in}}%
\pgfpathlineto{\pgfqpoint{3.703382in}{2.517520in}}%
\pgfpathlineto{\pgfqpoint{3.690122in}{2.521737in}}%
\pgfpathlineto{\pgfqpoint{3.676867in}{2.525989in}}%
\pgfpathlineto{\pgfqpoint{3.663617in}{2.530275in}}%
\pgfpathlineto{\pgfqpoint{3.671463in}{2.537676in}}%
\pgfpathlineto{\pgfqpoint{3.679304in}{2.545091in}}%
\pgfpathlineto{\pgfqpoint{3.687138in}{2.552519in}}%
\pgfpathlineto{\pgfqpoint{3.694967in}{2.559961in}}%
\pgfpathclose%
\pgfusepath{fill}%
\end{pgfscope}%
\begin{pgfscope}%
\pgfpathrectangle{\pgfqpoint{1.150000in}{0.150000in}}{\pgfqpoint{5.700000in}{5.700000in}}%
\pgfusepath{clip}%
\pgfsetbuttcap%
\pgfsetroundjoin%
\definecolor{currentfill}{rgb}{0.271305,0.019942,0.347269}%
\pgfsetfillcolor{currentfill}%
\pgfsetfillopacity{0.700000}%
\pgfsetlinewidth{0.000000pt}%
\definecolor{currentstroke}{rgb}{0.000000,0.000000,0.000000}%
\pgfsetstrokecolor{currentstroke}%
\pgfsetdash{}{0pt}%
\pgfpathmoveto{\pgfqpoint{4.053507in}{2.571705in}}%
\pgfpathlineto{\pgfqpoint{4.066820in}{2.568110in}}%
\pgfpathlineto{\pgfqpoint{4.080138in}{2.564545in}}%
\pgfpathlineto{\pgfqpoint{4.093462in}{2.561011in}}%
\pgfpathlineto{\pgfqpoint{4.106792in}{2.557508in}}%
\pgfpathlineto{\pgfqpoint{4.099106in}{2.549982in}}%
\pgfpathlineto{\pgfqpoint{4.091415in}{2.542454in}}%
\pgfpathlineto{\pgfqpoint{4.083717in}{2.534923in}}%
\pgfpathlineto{\pgfqpoint{4.076015in}{2.527388in}}%
\pgfpathlineto{\pgfqpoint{4.062672in}{2.530881in}}%
\pgfpathlineto{\pgfqpoint{4.049335in}{2.534405in}}%
\pgfpathlineto{\pgfqpoint{4.036005in}{2.537959in}}%
\pgfpathlineto{\pgfqpoint{4.022680in}{2.541544in}}%
\pgfpathlineto{\pgfqpoint{4.030395in}{2.549084in}}%
\pgfpathlineto{\pgfqpoint{4.038105in}{2.556624in}}%
\pgfpathlineto{\pgfqpoint{4.045809in}{2.564164in}}%
\pgfpathlineto{\pgfqpoint{4.053507in}{2.571705in}}%
\pgfpathclose%
\pgfusepath{fill}%
\end{pgfscope}%
\begin{pgfscope}%
\pgfpathrectangle{\pgfqpoint{1.150000in}{0.150000in}}{\pgfqpoint{5.700000in}{5.700000in}}%
\pgfusepath{clip}%
\pgfsetbuttcap%
\pgfsetroundjoin%
\definecolor{currentfill}{rgb}{0.278791,0.062145,0.386592}%
\pgfsetfillcolor{currentfill}%
\pgfsetfillopacity{0.700000}%
\pgfsetlinewidth{0.000000pt}%
\definecolor{currentstroke}{rgb}{0.000000,0.000000,0.000000}%
\pgfsetstrokecolor{currentstroke}%
\pgfsetdash{}{0pt}%
\pgfpathmoveto{\pgfqpoint{4.854574in}{2.639711in}}%
\pgfpathlineto{\pgfqpoint{4.868076in}{2.636915in}}%
\pgfpathlineto{\pgfqpoint{4.881585in}{2.634146in}}%
\pgfpathlineto{\pgfqpoint{4.895101in}{2.631402in}}%
\pgfpathlineto{\pgfqpoint{4.908624in}{2.628685in}}%
\pgfpathlineto{\pgfqpoint{4.901236in}{2.621666in}}%
\pgfpathlineto{\pgfqpoint{4.893842in}{2.614671in}}%
\pgfpathlineto{\pgfqpoint{4.886443in}{2.607696in}}%
\pgfpathlineto{\pgfqpoint{4.879040in}{2.600738in}}%
\pgfpathlineto{\pgfqpoint{4.865502in}{2.603353in}}%
\pgfpathlineto{\pgfqpoint{4.851971in}{2.605994in}}%
\pgfpathlineto{\pgfqpoint{4.838448in}{2.608661in}}%
\pgfpathlineto{\pgfqpoint{4.824931in}{2.611354in}}%
\pgfpathlineto{\pgfqpoint{4.832349in}{2.618410in}}%
\pgfpathlineto{\pgfqpoint{4.839762in}{2.625485in}}%
\pgfpathlineto{\pgfqpoint{4.847171in}{2.632585in}}%
\pgfpathlineto{\pgfqpoint{4.854574in}{2.639711in}}%
\pgfpathclose%
\pgfusepath{fill}%
\end{pgfscope}%
\begin{pgfscope}%
\pgfpathrectangle{\pgfqpoint{1.150000in}{0.150000in}}{\pgfqpoint{5.700000in}{5.700000in}}%
\pgfusepath{clip}%
\pgfsetbuttcap%
\pgfsetroundjoin%
\definecolor{currentfill}{rgb}{0.277018,0.050344,0.375715}%
\pgfsetfillcolor{currentfill}%
\pgfsetfillopacity{0.700000}%
\pgfsetlinewidth{0.000000pt}%
\definecolor{currentstroke}{rgb}{0.000000,0.000000,0.000000}%
\pgfsetstrokecolor{currentstroke}%
\pgfsetdash{}{0pt}%
\pgfpathmoveto{\pgfqpoint{4.633363in}{2.616575in}}%
\pgfpathlineto{\pgfqpoint{4.646813in}{2.613652in}}%
\pgfpathlineto{\pgfqpoint{4.660269in}{2.610757in}}%
\pgfpathlineto{\pgfqpoint{4.673732in}{2.607888in}}%
\pgfpathlineto{\pgfqpoint{4.687202in}{2.605046in}}%
\pgfpathlineto{\pgfqpoint{4.679729in}{2.597881in}}%
\pgfpathlineto{\pgfqpoint{4.672251in}{2.590725in}}%
\pgfpathlineto{\pgfqpoint{4.664767in}{2.583574in}}%
\pgfpathlineto{\pgfqpoint{4.657278in}{2.576427in}}%
\pgfpathlineto{\pgfqpoint{4.643794in}{2.579193in}}%
\pgfpathlineto{\pgfqpoint{4.630318in}{2.581986in}}%
\pgfpathlineto{\pgfqpoint{4.616847in}{2.584805in}}%
\pgfpathlineto{\pgfqpoint{4.603384in}{2.587652in}}%
\pgfpathlineto{\pgfqpoint{4.610887in}{2.594870in}}%
\pgfpathlineto{\pgfqpoint{4.618384in}{2.602095in}}%
\pgfpathlineto{\pgfqpoint{4.625876in}{2.609329in}}%
\pgfpathlineto{\pgfqpoint{4.633363in}{2.616575in}}%
\pgfpathclose%
\pgfusepath{fill}%
\end{pgfscope}%
\begin{pgfscope}%
\pgfpathrectangle{\pgfqpoint{1.150000in}{0.150000in}}{\pgfqpoint{5.700000in}{5.700000in}}%
\pgfusepath{clip}%
\pgfsetbuttcap%
\pgfsetroundjoin%
\definecolor{currentfill}{rgb}{0.274952,0.037752,0.364543}%
\pgfsetfillcolor{currentfill}%
\pgfsetfillopacity{0.700000}%
\pgfsetlinewidth{0.000000pt}%
\definecolor{currentstroke}{rgb}{0.000000,0.000000,0.000000}%
\pgfsetstrokecolor{currentstroke}%
\pgfsetdash{}{0pt}%
\pgfpathmoveto{\pgfqpoint{4.412112in}{2.594341in}}%
\pgfpathlineto{\pgfqpoint{4.425510in}{2.591225in}}%
\pgfpathlineto{\pgfqpoint{4.438915in}{2.588138in}}%
\pgfpathlineto{\pgfqpoint{4.452326in}{2.585079in}}%
\pgfpathlineto{\pgfqpoint{4.465743in}{2.582047in}}%
\pgfpathlineto{\pgfqpoint{4.458187in}{2.574716in}}%
\pgfpathlineto{\pgfqpoint{4.450625in}{2.567384in}}%
\pgfpathlineto{\pgfqpoint{4.443057in}{2.560049in}}%
\pgfpathlineto{\pgfqpoint{4.435484in}{2.552709in}}%
\pgfpathlineto{\pgfqpoint{4.422054in}{2.555691in}}%
\pgfpathlineto{\pgfqpoint{4.408630in}{2.558700in}}%
\pgfpathlineto{\pgfqpoint{4.395212in}{2.561737in}}%
\pgfpathlineto{\pgfqpoint{4.381801in}{2.564803in}}%
\pgfpathlineto{\pgfqpoint{4.389387in}{2.572187in}}%
\pgfpathlineto{\pgfqpoint{4.396968in}{2.579571in}}%
\pgfpathlineto{\pgfqpoint{4.404543in}{2.586955in}}%
\pgfpathlineto{\pgfqpoint{4.412112in}{2.594341in}}%
\pgfpathclose%
\pgfusepath{fill}%
\end{pgfscope}%
\begin{pgfscope}%
\pgfpathrectangle{\pgfqpoint{1.150000in}{0.150000in}}{\pgfqpoint{5.700000in}{5.700000in}}%
\pgfusepath{clip}%
\pgfsetbuttcap%
\pgfsetroundjoin%
\definecolor{currentfill}{rgb}{0.269944,0.014625,0.341379}%
\pgfsetfillcolor{currentfill}%
\pgfsetfillopacity{0.700000}%
\pgfsetlinewidth{0.000000pt}%
\definecolor{currentstroke}{rgb}{0.000000,0.000000,0.000000}%
\pgfsetstrokecolor{currentstroke}%
\pgfsetdash{}{0pt}%
\pgfpathmoveto{\pgfqpoint{3.832171in}{2.556887in}}%
\pgfpathlineto{\pgfqpoint{3.845439in}{2.552917in}}%
\pgfpathlineto{\pgfqpoint{3.858713in}{2.548979in}}%
\pgfpathlineto{\pgfqpoint{3.871992in}{2.545074in}}%
\pgfpathlineto{\pgfqpoint{3.885277in}{2.541202in}}%
\pgfpathlineto{\pgfqpoint{3.877509in}{2.533674in}}%
\pgfpathlineto{\pgfqpoint{3.869735in}{2.526150in}}%
\pgfpathlineto{\pgfqpoint{3.861956in}{2.518629in}}%
\pgfpathlineto{\pgfqpoint{3.854171in}{2.511112in}}%
\pgfpathlineto{\pgfqpoint{3.840873in}{2.515000in}}%
\pgfpathlineto{\pgfqpoint{3.827581in}{2.518921in}}%
\pgfpathlineto{\pgfqpoint{3.814295in}{2.522875in}}%
\pgfpathlineto{\pgfqpoint{3.801015in}{2.526861in}}%
\pgfpathlineto{\pgfqpoint{3.808812in}{2.534358in}}%
\pgfpathlineto{\pgfqpoint{3.816604in}{2.541861in}}%
\pgfpathlineto{\pgfqpoint{3.824391in}{2.549371in}}%
\pgfpathlineto{\pgfqpoint{3.832171in}{2.556887in}}%
\pgfpathclose%
\pgfusepath{fill}%
\end{pgfscope}%
\begin{pgfscope}%
\pgfpathrectangle{\pgfqpoint{1.150000in}{0.150000in}}{\pgfqpoint{5.700000in}{5.700000in}}%
\pgfusepath{clip}%
\pgfsetbuttcap%
\pgfsetroundjoin%
\definecolor{currentfill}{rgb}{0.282327,0.094955,0.417331}%
\pgfsetfillcolor{currentfill}%
\pgfsetfillopacity{0.700000}%
\pgfsetlinewidth{0.000000pt}%
\definecolor{currentstroke}{rgb}{0.000000,0.000000,0.000000}%
\pgfsetstrokecolor{currentstroke}%
\pgfsetdash{}{0pt}%
\pgfpathmoveto{\pgfqpoint{5.434734in}{2.694452in}}%
\pgfpathlineto{\pgfqpoint{5.448380in}{2.691753in}}%
\pgfpathlineto{\pgfqpoint{5.462033in}{2.689078in}}%
\pgfpathlineto{\pgfqpoint{5.475694in}{2.686427in}}%
\pgfpathlineto{\pgfqpoint{5.489362in}{2.683801in}}%
\pgfpathlineto{\pgfqpoint{5.482188in}{2.676831in}}%
\pgfpathlineto{\pgfqpoint{5.475011in}{2.669947in}}%
\pgfpathlineto{\pgfqpoint{5.467831in}{2.663144in}}%
\pgfpathlineto{\pgfqpoint{5.460647in}{2.656415in}}%
\pgfpathlineto{\pgfqpoint{5.446961in}{2.658874in}}%
\pgfpathlineto{\pgfqpoint{5.433283in}{2.661356in}}%
\pgfpathlineto{\pgfqpoint{5.419611in}{2.663863in}}%
\pgfpathlineto{\pgfqpoint{5.405947in}{2.666394in}}%
\pgfpathlineto{\pgfqpoint{5.413149in}{2.673286in}}%
\pgfpathlineto{\pgfqpoint{5.420347in}{2.680256in}}%
\pgfpathlineto{\pgfqpoint{5.427542in}{2.687310in}}%
\pgfpathlineto{\pgfqpoint{5.434734in}{2.694452in}}%
\pgfpathclose%
\pgfusepath{fill}%
\end{pgfscope}%
\begin{pgfscope}%
\pgfpathrectangle{\pgfqpoint{1.150000in}{0.150000in}}{\pgfqpoint{5.700000in}{5.700000in}}%
\pgfusepath{clip}%
\pgfsetbuttcap%
\pgfsetroundjoin%
\definecolor{currentfill}{rgb}{0.272594,0.025563,0.353093}%
\pgfsetfillcolor{currentfill}%
\pgfsetfillopacity{0.700000}%
\pgfsetlinewidth{0.000000pt}%
\definecolor{currentstroke}{rgb}{0.000000,0.000000,0.000000}%
\pgfsetstrokecolor{currentstroke}%
\pgfsetdash{}{0pt}%
\pgfpathmoveto{\pgfqpoint{3.198860in}{2.582496in}}%
\pgfpathlineto{\pgfqpoint{3.212018in}{2.576967in}}%
\pgfpathlineto{\pgfqpoint{3.225180in}{2.571482in}}%
\pgfpathlineto{\pgfqpoint{3.238346in}{2.566039in}}%
\pgfpathlineto{\pgfqpoint{3.251517in}{2.560638in}}%
\pgfpathlineto{\pgfqpoint{3.243506in}{2.553945in}}%
\pgfpathlineto{\pgfqpoint{3.235489in}{2.547299in}}%
\pgfpathlineto{\pgfqpoint{3.227465in}{2.540702in}}%
\pgfpathlineto{\pgfqpoint{3.219433in}{2.534156in}}%
\pgfpathlineto{\pgfqpoint{3.206247in}{2.539639in}}%
\pgfpathlineto{\pgfqpoint{3.193066in}{2.545164in}}%
\pgfpathlineto{\pgfqpoint{3.179888in}{2.550732in}}%
\pgfpathlineto{\pgfqpoint{3.166715in}{2.556342in}}%
\pgfpathlineto{\pgfqpoint{3.174762in}{2.562802in}}%
\pgfpathlineto{\pgfqpoint{3.182801in}{2.569315in}}%
\pgfpathlineto{\pgfqpoint{3.190834in}{2.575880in}}%
\pgfpathlineto{\pgfqpoint{3.198860in}{2.582496in}}%
\pgfpathclose%
\pgfusepath{fill}%
\end{pgfscope}%
\begin{pgfscope}%
\pgfpathrectangle{\pgfqpoint{1.150000in}{0.150000in}}{\pgfqpoint{5.700000in}{5.700000in}}%
\pgfusepath{clip}%
\pgfsetbuttcap%
\pgfsetroundjoin%
\definecolor{currentfill}{rgb}{0.271305,0.019942,0.347269}%
\pgfsetfillcolor{currentfill}%
\pgfsetfillopacity{0.700000}%
\pgfsetlinewidth{0.000000pt}%
\definecolor{currentstroke}{rgb}{0.000000,0.000000,0.000000}%
\pgfsetstrokecolor{currentstroke}%
\pgfsetdash{}{0pt}%
\pgfpathmoveto{\pgfqpoint{3.336160in}{2.566933in}}%
\pgfpathlineto{\pgfqpoint{3.349338in}{2.561805in}}%
\pgfpathlineto{\pgfqpoint{3.362521in}{2.556717in}}%
\pgfpathlineto{\pgfqpoint{3.375709in}{2.551669in}}%
\pgfpathlineto{\pgfqpoint{3.388901in}{2.546660in}}%
\pgfpathlineto{\pgfqpoint{3.380946in}{2.539666in}}%
\pgfpathlineto{\pgfqpoint{3.372984in}{2.532707in}}%
\pgfpathlineto{\pgfqpoint{3.365016in}{2.525783in}}%
\pgfpathlineto{\pgfqpoint{3.357041in}{2.518898in}}%
\pgfpathlineto{\pgfqpoint{3.343835in}{2.523975in}}%
\pgfpathlineto{\pgfqpoint{3.330633in}{2.529092in}}%
\pgfpathlineto{\pgfqpoint{3.317436in}{2.534249in}}%
\pgfpathlineto{\pgfqpoint{3.304243in}{2.539445in}}%
\pgfpathlineto{\pgfqpoint{3.312232in}{2.546257in}}%
\pgfpathlineto{\pgfqpoint{3.320215in}{2.553110in}}%
\pgfpathlineto{\pgfqpoint{3.328191in}{2.560003in}}%
\pgfpathlineto{\pgfqpoint{3.336160in}{2.566933in}}%
\pgfpathclose%
\pgfusepath{fill}%
\end{pgfscope}%
\begin{pgfscope}%
\pgfpathrectangle{\pgfqpoint{1.150000in}{0.150000in}}{\pgfqpoint{5.700000in}{5.700000in}}%
\pgfusepath{clip}%
\pgfsetbuttcap%
\pgfsetroundjoin%
\definecolor{currentfill}{rgb}{0.272594,0.025563,0.353093}%
\pgfsetfillcolor{currentfill}%
\pgfsetfillopacity{0.700000}%
\pgfsetlinewidth{0.000000pt}%
\definecolor{currentstroke}{rgb}{0.000000,0.000000,0.000000}%
\pgfsetstrokecolor{currentstroke}%
\pgfsetdash{}{0pt}%
\pgfpathmoveto{\pgfqpoint{4.190813in}{2.573809in}}%
\pgfpathlineto{\pgfqpoint{4.204161in}{2.570433in}}%
\pgfpathlineto{\pgfqpoint{4.217515in}{2.567086in}}%
\pgfpathlineto{\pgfqpoint{4.230876in}{2.563768in}}%
\pgfpathlineto{\pgfqpoint{4.244242in}{2.560480in}}%
\pgfpathlineto{\pgfqpoint{4.236603in}{2.553007in}}%
\pgfpathlineto{\pgfqpoint{4.228959in}{2.545531in}}%
\pgfpathlineto{\pgfqpoint{4.221309in}{2.538049in}}%
\pgfpathlineto{\pgfqpoint{4.213653in}{2.530561in}}%
\pgfpathlineto{\pgfqpoint{4.200274in}{2.533826in}}%
\pgfpathlineto{\pgfqpoint{4.186901in}{2.537120in}}%
\pgfpathlineto{\pgfqpoint{4.173534in}{2.540443in}}%
\pgfpathlineto{\pgfqpoint{4.160174in}{2.543796in}}%
\pgfpathlineto{\pgfqpoint{4.167842in}{2.551303in}}%
\pgfpathlineto{\pgfqpoint{4.175504in}{2.558807in}}%
\pgfpathlineto{\pgfqpoint{4.183161in}{2.566308in}}%
\pgfpathlineto{\pgfqpoint{4.190813in}{2.573809in}}%
\pgfpathclose%
\pgfusepath{fill}%
\end{pgfscope}%
\begin{pgfscope}%
\pgfpathrectangle{\pgfqpoint{1.150000in}{0.150000in}}{\pgfqpoint{5.700000in}{5.700000in}}%
\pgfusepath{clip}%
\pgfsetbuttcap%
\pgfsetroundjoin%
\definecolor{currentfill}{rgb}{0.281446,0.084320,0.407414}%
\pgfsetfillcolor{currentfill}%
\pgfsetfillopacity{0.700000}%
\pgfsetlinewidth{0.000000pt}%
\definecolor{currentstroke}{rgb}{0.000000,0.000000,0.000000}%
\pgfsetstrokecolor{currentstroke}%
\pgfsetdash{}{0pt}%
\pgfpathmoveto{\pgfqpoint{5.213494in}{2.669940in}}%
\pgfpathlineto{\pgfqpoint{5.227089in}{2.667268in}}%
\pgfpathlineto{\pgfqpoint{5.240691in}{2.664621in}}%
\pgfpathlineto{\pgfqpoint{5.254300in}{2.661998in}}%
\pgfpathlineto{\pgfqpoint{5.267916in}{2.659401in}}%
\pgfpathlineto{\pgfqpoint{5.260661in}{2.652532in}}%
\pgfpathlineto{\pgfqpoint{5.253403in}{2.645719in}}%
\pgfpathlineto{\pgfqpoint{5.246140in}{2.638958in}}%
\pgfpathlineto{\pgfqpoint{5.238873in}{2.632243in}}%
\pgfpathlineto{\pgfqpoint{5.225240in}{2.634698in}}%
\pgfpathlineto{\pgfqpoint{5.211614in}{2.637178in}}%
\pgfpathlineto{\pgfqpoint{5.197996in}{2.639683in}}%
\pgfpathlineto{\pgfqpoint{5.184384in}{2.642213in}}%
\pgfpathlineto{\pgfqpoint{5.191668in}{2.649065in}}%
\pgfpathlineto{\pgfqpoint{5.198947in}{2.655968in}}%
\pgfpathlineto{\pgfqpoint{5.206223in}{2.662924in}}%
\pgfpathlineto{\pgfqpoint{5.213494in}{2.669940in}}%
\pgfpathclose%
\pgfusepath{fill}%
\end{pgfscope}%
\begin{pgfscope}%
\pgfpathrectangle{\pgfqpoint{1.150000in}{0.150000in}}{\pgfqpoint{5.700000in}{5.700000in}}%
\pgfusepath{clip}%
\pgfsetbuttcap%
\pgfsetroundjoin%
\definecolor{currentfill}{rgb}{0.274952,0.037752,0.364543}%
\pgfsetfillcolor{currentfill}%
\pgfsetfillopacity{0.700000}%
\pgfsetlinewidth{0.000000pt}%
\definecolor{currentstroke}{rgb}{0.000000,0.000000,0.000000}%
\pgfsetstrokecolor{currentstroke}%
\pgfsetdash{}{0pt}%
\pgfpathmoveto{\pgfqpoint{3.061473in}{2.602809in}}%
\pgfpathlineto{\pgfqpoint{3.074614in}{2.596843in}}%
\pgfpathlineto{\pgfqpoint{3.087760in}{2.590923in}}%
\pgfpathlineto{\pgfqpoint{3.100909in}{2.585048in}}%
\pgfpathlineto{\pgfqpoint{3.114063in}{2.579218in}}%
\pgfpathlineto{\pgfqpoint{3.105993in}{2.572905in}}%
\pgfpathlineto{\pgfqpoint{3.097916in}{2.566653in}}%
\pgfpathlineto{\pgfqpoint{3.089831in}{2.560463in}}%
\pgfpathlineto{\pgfqpoint{3.081739in}{2.554339in}}%
\pgfpathlineto{\pgfqpoint{3.068570in}{2.560264in}}%
\pgfpathlineto{\pgfqpoint{3.055404in}{2.566234in}}%
\pgfpathlineto{\pgfqpoint{3.042242in}{2.572250in}}%
\pgfpathlineto{\pgfqpoint{3.029084in}{2.578311in}}%
\pgfpathlineto{\pgfqpoint{3.037193in}{2.584335in}}%
\pgfpathlineto{\pgfqpoint{3.045294in}{2.590428in}}%
\pgfpathlineto{\pgfqpoint{3.053387in}{2.596587in}}%
\pgfpathlineto{\pgfqpoint{3.061473in}{2.602809in}}%
\pgfpathclose%
\pgfusepath{fill}%
\end{pgfscope}%
\begin{pgfscope}%
\pgfpathrectangle{\pgfqpoint{1.150000in}{0.150000in}}{\pgfqpoint{5.700000in}{5.700000in}}%
\pgfusepath{clip}%
\pgfsetbuttcap%
\pgfsetroundjoin%
\definecolor{currentfill}{rgb}{0.269944,0.014625,0.341379}%
\pgfsetfillcolor{currentfill}%
\pgfsetfillopacity{0.700000}%
\pgfsetlinewidth{0.000000pt}%
\definecolor{currentstroke}{rgb}{0.000000,0.000000,0.000000}%
\pgfsetstrokecolor{currentstroke}%
\pgfsetdash{}{0pt}%
\pgfpathmoveto{\pgfqpoint{3.473418in}{2.555536in}}%
\pgfpathlineto{\pgfqpoint{3.486620in}{2.550775in}}%
\pgfpathlineto{\pgfqpoint{3.499827in}{2.546051in}}%
\pgfpathlineto{\pgfqpoint{3.513039in}{2.541364in}}%
\pgfpathlineto{\pgfqpoint{3.526255in}{2.536714in}}%
\pgfpathlineto{\pgfqpoint{3.518353in}{2.529491in}}%
\pgfpathlineto{\pgfqpoint{3.510444in}{2.522291in}}%
\pgfpathlineto{\pgfqpoint{3.502529in}{2.515116in}}%
\pgfpathlineto{\pgfqpoint{3.494608in}{2.507968in}}%
\pgfpathlineto{\pgfqpoint{3.481378in}{2.512673in}}%
\pgfpathlineto{\pgfqpoint{3.468152in}{2.517415in}}%
\pgfpathlineto{\pgfqpoint{3.454932in}{2.522194in}}%
\pgfpathlineto{\pgfqpoint{3.441716in}{2.527011in}}%
\pgfpathlineto{\pgfqpoint{3.449651in}{2.534100in}}%
\pgfpathlineto{\pgfqpoint{3.457579in}{2.541217in}}%
\pgfpathlineto{\pgfqpoint{3.465502in}{2.548363in}}%
\pgfpathlineto{\pgfqpoint{3.473418in}{2.555536in}}%
\pgfpathclose%
\pgfusepath{fill}%
\end{pgfscope}%
\begin{pgfscope}%
\pgfpathrectangle{\pgfqpoint{1.150000in}{0.150000in}}{\pgfqpoint{5.700000in}{5.700000in}}%
\pgfusepath{clip}%
\pgfsetbuttcap%
\pgfsetroundjoin%
\definecolor{currentfill}{rgb}{0.279566,0.067836,0.391917}%
\pgfsetfillcolor{currentfill}%
\pgfsetfillopacity{0.700000}%
\pgfsetlinewidth{0.000000pt}%
\definecolor{currentstroke}{rgb}{0.000000,0.000000,0.000000}%
\pgfsetstrokecolor{currentstroke}%
\pgfsetdash{}{0pt}%
\pgfpathmoveto{\pgfqpoint{4.992230in}{2.645993in}}%
\pgfpathlineto{\pgfqpoint{5.005772in}{2.643288in}}%
\pgfpathlineto{\pgfqpoint{5.019322in}{2.640610in}}%
\pgfpathlineto{\pgfqpoint{5.032878in}{2.637956in}}%
\pgfpathlineto{\pgfqpoint{5.046441in}{2.635329in}}%
\pgfpathlineto{\pgfqpoint{5.039103in}{2.628421in}}%
\pgfpathlineto{\pgfqpoint{5.031760in}{2.621545in}}%
\pgfpathlineto{\pgfqpoint{5.024411in}{2.614697in}}%
\pgfpathlineto{\pgfqpoint{5.017058in}{2.607873in}}%
\pgfpathlineto{\pgfqpoint{5.003479in}{2.610385in}}%
\pgfpathlineto{\pgfqpoint{4.989907in}{2.612922in}}%
\pgfpathlineto{\pgfqpoint{4.976343in}{2.615485in}}%
\pgfpathlineto{\pgfqpoint{4.962785in}{2.618073in}}%
\pgfpathlineto{\pgfqpoint{4.970154in}{2.625008in}}%
\pgfpathlineto{\pgfqpoint{4.977517in}{2.631971in}}%
\pgfpathlineto{\pgfqpoint{4.984876in}{2.638964in}}%
\pgfpathlineto{\pgfqpoint{4.992230in}{2.645993in}}%
\pgfpathclose%
\pgfusepath{fill}%
\end{pgfscope}%
\begin{pgfscope}%
\pgfpathrectangle{\pgfqpoint{1.150000in}{0.150000in}}{\pgfqpoint{5.700000in}{5.700000in}}%
\pgfusepath{clip}%
\pgfsetbuttcap%
\pgfsetroundjoin%
\definecolor{currentfill}{rgb}{0.277941,0.056324,0.381191}%
\pgfsetfillcolor{currentfill}%
\pgfsetfillopacity{0.700000}%
\pgfsetlinewidth{0.000000pt}%
\definecolor{currentstroke}{rgb}{0.000000,0.000000,0.000000}%
\pgfsetstrokecolor{currentstroke}%
\pgfsetdash{}{0pt}%
\pgfpathmoveto{\pgfqpoint{4.770932in}{2.622389in}}%
\pgfpathlineto{\pgfqpoint{4.784421in}{2.619590in}}%
\pgfpathlineto{\pgfqpoint{4.797918in}{2.616819in}}%
\pgfpathlineto{\pgfqpoint{4.811421in}{2.614073in}}%
\pgfpathlineto{\pgfqpoint{4.824931in}{2.611354in}}%
\pgfpathlineto{\pgfqpoint{4.817507in}{2.604315in}}%
\pgfpathlineto{\pgfqpoint{4.810078in}{2.597290in}}%
\pgfpathlineto{\pgfqpoint{4.802644in}{2.590275in}}%
\pgfpathlineto{\pgfqpoint{4.795205in}{2.583269in}}%
\pgfpathlineto{\pgfqpoint{4.781680in}{2.585898in}}%
\pgfpathlineto{\pgfqpoint{4.768163in}{2.588554in}}%
\pgfpathlineto{\pgfqpoint{4.754652in}{2.591236in}}%
\pgfpathlineto{\pgfqpoint{4.741149in}{2.593945in}}%
\pgfpathlineto{\pgfqpoint{4.748602in}{2.601036in}}%
\pgfpathlineto{\pgfqpoint{4.756051in}{2.608139in}}%
\pgfpathlineto{\pgfqpoint{4.763494in}{2.615255in}}%
\pgfpathlineto{\pgfqpoint{4.770932in}{2.622389in}}%
\pgfpathclose%
\pgfusepath{fill}%
\end{pgfscope}%
\begin{pgfscope}%
\pgfpathrectangle{\pgfqpoint{1.150000in}{0.150000in}}{\pgfqpoint{5.700000in}{5.700000in}}%
\pgfusepath{clip}%
\pgfsetbuttcap%
\pgfsetroundjoin%
\definecolor{currentfill}{rgb}{0.269944,0.014625,0.341379}%
\pgfsetfillcolor{currentfill}%
\pgfsetfillopacity{0.700000}%
\pgfsetlinewidth{0.000000pt}%
\definecolor{currentstroke}{rgb}{0.000000,0.000000,0.000000}%
\pgfsetstrokecolor{currentstroke}%
\pgfsetdash{}{0pt}%
\pgfpathmoveto{\pgfqpoint{3.969440in}{2.556196in}}%
\pgfpathlineto{\pgfqpoint{3.982741in}{2.552486in}}%
\pgfpathlineto{\pgfqpoint{3.996048in}{2.548808in}}%
\pgfpathlineto{\pgfqpoint{4.009361in}{2.545160in}}%
\pgfpathlineto{\pgfqpoint{4.022680in}{2.541544in}}%
\pgfpathlineto{\pgfqpoint{4.014959in}{2.534004in}}%
\pgfpathlineto{\pgfqpoint{4.007233in}{2.526461in}}%
\pgfpathlineto{\pgfqpoint{3.999501in}{2.518917in}}%
\pgfpathlineto{\pgfqpoint{3.991763in}{2.511370in}}%
\pgfpathlineto{\pgfqpoint{3.978432in}{2.514989in}}%
\pgfpathlineto{\pgfqpoint{3.965107in}{2.518639in}}%
\pgfpathlineto{\pgfqpoint{3.951787in}{2.522320in}}%
\pgfpathlineto{\pgfqpoint{3.938474in}{2.526033in}}%
\pgfpathlineto{\pgfqpoint{3.946224in}{2.533572in}}%
\pgfpathlineto{\pgfqpoint{3.953968in}{2.541112in}}%
\pgfpathlineto{\pgfqpoint{3.961707in}{2.548653in}}%
\pgfpathlineto{\pgfqpoint{3.969440in}{2.556196in}}%
\pgfpathclose%
\pgfusepath{fill}%
\end{pgfscope}%
\begin{pgfscope}%
\pgfpathrectangle{\pgfqpoint{1.150000in}{0.150000in}}{\pgfqpoint{5.700000in}{5.700000in}}%
\pgfusepath{clip}%
\pgfsetbuttcap%
\pgfsetroundjoin%
\definecolor{currentfill}{rgb}{0.277941,0.056324,0.381191}%
\pgfsetfillcolor{currentfill}%
\pgfsetfillopacity{0.700000}%
\pgfsetlinewidth{0.000000pt}%
\definecolor{currentstroke}{rgb}{0.000000,0.000000,0.000000}%
\pgfsetstrokecolor{currentstroke}%
\pgfsetdash{}{0pt}%
\pgfpathmoveto{\pgfqpoint{2.923948in}{2.628514in}}%
\pgfpathlineto{\pgfqpoint{2.937078in}{2.622068in}}%
\pgfpathlineto{\pgfqpoint{2.950212in}{2.615672in}}%
\pgfpathlineto{\pgfqpoint{2.963348in}{2.609325in}}%
\pgfpathlineto{\pgfqpoint{2.976489in}{2.603027in}}%
\pgfpathlineto{\pgfqpoint{2.968355in}{2.597177in}}%
\pgfpathlineto{\pgfqpoint{2.960214in}{2.591403in}}%
\pgfpathlineto{\pgfqpoint{2.952064in}{2.585708in}}%
\pgfpathlineto{\pgfqpoint{2.943906in}{2.580094in}}%
\pgfpathlineto{\pgfqpoint{2.930749in}{2.586501in}}%
\pgfpathlineto{\pgfqpoint{2.917595in}{2.592957in}}%
\pgfpathlineto{\pgfqpoint{2.904444in}{2.599462in}}%
\pgfpathlineto{\pgfqpoint{2.891297in}{2.606017in}}%
\pgfpathlineto{\pgfqpoint{2.899472in}{2.611517in}}%
\pgfpathlineto{\pgfqpoint{2.907639in}{2.617102in}}%
\pgfpathlineto{\pgfqpoint{2.915798in}{2.622768in}}%
\pgfpathlineto{\pgfqpoint{2.923948in}{2.628514in}}%
\pgfpathclose%
\pgfusepath{fill}%
\end{pgfscope}%
\begin{pgfscope}%
\pgfpathrectangle{\pgfqpoint{1.150000in}{0.150000in}}{\pgfqpoint{5.700000in}{5.700000in}}%
\pgfusepath{clip}%
\pgfsetbuttcap%
\pgfsetroundjoin%
\definecolor{currentfill}{rgb}{0.268510,0.009605,0.335427}%
\pgfsetfillcolor{currentfill}%
\pgfsetfillopacity{0.700000}%
\pgfsetlinewidth{0.000000pt}%
\definecolor{currentstroke}{rgb}{0.000000,0.000000,0.000000}%
\pgfsetstrokecolor{currentstroke}%
\pgfsetdash{}{0pt}%
\pgfpathmoveto{\pgfqpoint{3.610670in}{2.547768in}}%
\pgfpathlineto{\pgfqpoint{3.623899in}{2.543342in}}%
\pgfpathlineto{\pgfqpoint{3.637133in}{2.538951in}}%
\pgfpathlineto{\pgfqpoint{3.650372in}{2.534595in}}%
\pgfpathlineto{\pgfqpoint{3.663617in}{2.530275in}}%
\pgfpathlineto{\pgfqpoint{3.655764in}{2.522888in}}%
\pgfpathlineto{\pgfqpoint{3.647906in}{2.515515in}}%
\pgfpathlineto{\pgfqpoint{3.640042in}{2.508158in}}%
\pgfpathlineto{\pgfqpoint{3.632172in}{2.500817in}}%
\pgfpathlineto{\pgfqpoint{3.618915in}{2.505180in}}%
\pgfpathlineto{\pgfqpoint{3.605662in}{2.509578in}}%
\pgfpathlineto{\pgfqpoint{3.592415in}{2.514011in}}%
\pgfpathlineto{\pgfqpoint{3.579173in}{2.518480in}}%
\pgfpathlineto{\pgfqpoint{3.587056in}{2.525774in}}%
\pgfpathlineto{\pgfqpoint{3.594933in}{2.533087in}}%
\pgfpathlineto{\pgfqpoint{3.602804in}{2.540419in}}%
\pgfpathlineto{\pgfqpoint{3.610670in}{2.547768in}}%
\pgfpathclose%
\pgfusepath{fill}%
\end{pgfscope}%
\begin{pgfscope}%
\pgfpathrectangle{\pgfqpoint{1.150000in}{0.150000in}}{\pgfqpoint{5.700000in}{5.700000in}}%
\pgfusepath{clip}%
\pgfsetbuttcap%
\pgfsetroundjoin%
\definecolor{currentfill}{rgb}{0.276022,0.044167,0.370164}%
\pgfsetfillcolor{currentfill}%
\pgfsetfillopacity{0.700000}%
\pgfsetlinewidth{0.000000pt}%
\definecolor{currentstroke}{rgb}{0.000000,0.000000,0.000000}%
\pgfsetstrokecolor{currentstroke}%
\pgfsetdash{}{0pt}%
\pgfpathmoveto{\pgfqpoint{4.549597in}{2.599311in}}%
\pgfpathlineto{\pgfqpoint{4.563034in}{2.596355in}}%
\pgfpathlineto{\pgfqpoint{4.576477in}{2.593427in}}%
\pgfpathlineto{\pgfqpoint{4.589927in}{2.590526in}}%
\pgfpathlineto{\pgfqpoint{4.603384in}{2.587652in}}%
\pgfpathlineto{\pgfqpoint{4.595876in}{2.580437in}}%
\pgfpathlineto{\pgfqpoint{4.588362in}{2.573224in}}%
\pgfpathlineto{\pgfqpoint{4.580843in}{2.566011in}}%
\pgfpathlineto{\pgfqpoint{4.573318in}{2.558794in}}%
\pgfpathlineto{\pgfqpoint{4.559848in}{2.561605in}}%
\pgfpathlineto{\pgfqpoint{4.546385in}{2.564443in}}%
\pgfpathlineto{\pgfqpoint{4.532928in}{2.567308in}}%
\pgfpathlineto{\pgfqpoint{4.519478in}{2.570201in}}%
\pgfpathlineto{\pgfqpoint{4.527016in}{2.577475in}}%
\pgfpathlineto{\pgfqpoint{4.534548in}{2.584751in}}%
\pgfpathlineto{\pgfqpoint{4.542075in}{2.592028in}}%
\pgfpathlineto{\pgfqpoint{4.549597in}{2.599311in}}%
\pgfpathclose%
\pgfusepath{fill}%
\end{pgfscope}%
\begin{pgfscope}%
\pgfpathrectangle{\pgfqpoint{1.150000in}{0.150000in}}{\pgfqpoint{5.700000in}{5.700000in}}%
\pgfusepath{clip}%
\pgfsetbuttcap%
\pgfsetroundjoin%
\definecolor{currentfill}{rgb}{0.273809,0.031497,0.358853}%
\pgfsetfillcolor{currentfill}%
\pgfsetfillopacity{0.700000}%
\pgfsetlinewidth{0.000000pt}%
\definecolor{currentstroke}{rgb}{0.000000,0.000000,0.000000}%
\pgfsetstrokecolor{currentstroke}%
\pgfsetdash{}{0pt}%
\pgfpathmoveto{\pgfqpoint{4.328222in}{2.577350in}}%
\pgfpathlineto{\pgfqpoint{4.341607in}{2.574171in}}%
\pgfpathlineto{\pgfqpoint{4.354999in}{2.571020in}}%
\pgfpathlineto{\pgfqpoint{4.368397in}{2.567897in}}%
\pgfpathlineto{\pgfqpoint{4.381801in}{2.564803in}}%
\pgfpathlineto{\pgfqpoint{4.374210in}{2.557416in}}%
\pgfpathlineto{\pgfqpoint{4.366613in}{2.550024in}}%
\pgfpathlineto{\pgfqpoint{4.359010in}{2.542625in}}%
\pgfpathlineto{\pgfqpoint{4.351402in}{2.535219in}}%
\pgfpathlineto{\pgfqpoint{4.337985in}{2.538276in}}%
\pgfpathlineto{\pgfqpoint{4.324574in}{2.541362in}}%
\pgfpathlineto{\pgfqpoint{4.311170in}{2.544476in}}%
\pgfpathlineto{\pgfqpoint{4.297772in}{2.547619in}}%
\pgfpathlineto{\pgfqpoint{4.305392in}{2.555057in}}%
\pgfpathlineto{\pgfqpoint{4.313008in}{2.562491in}}%
\pgfpathlineto{\pgfqpoint{4.320617in}{2.569921in}}%
\pgfpathlineto{\pgfqpoint{4.328222in}{2.577350in}}%
\pgfpathclose%
\pgfusepath{fill}%
\end{pgfscope}%
\begin{pgfscope}%
\pgfpathrectangle{\pgfqpoint{1.150000in}{0.150000in}}{\pgfqpoint{5.700000in}{5.700000in}}%
\pgfusepath{clip}%
\pgfsetbuttcap%
\pgfsetroundjoin%
\definecolor{currentfill}{rgb}{0.281924,0.089666,0.412415}%
\pgfsetfillcolor{currentfill}%
\pgfsetfillopacity{0.700000}%
\pgfsetlinewidth{0.000000pt}%
\definecolor{currentstroke}{rgb}{0.000000,0.000000,0.000000}%
\pgfsetstrokecolor{currentstroke}%
\pgfsetdash{}{0pt}%
\pgfpathmoveto{\pgfqpoint{5.351364in}{2.676764in}}%
\pgfpathlineto{\pgfqpoint{5.364999in}{2.674135in}}%
\pgfpathlineto{\pgfqpoint{5.378641in}{2.671530in}}%
\pgfpathlineto{\pgfqpoint{5.392291in}{2.668950in}}%
\pgfpathlineto{\pgfqpoint{5.405947in}{2.666394in}}%
\pgfpathlineto{\pgfqpoint{5.398743in}{2.659576in}}%
\pgfpathlineto{\pgfqpoint{5.391534in}{2.652825in}}%
\pgfpathlineto{\pgfqpoint{5.384322in}{2.646139in}}%
\pgfpathlineto{\pgfqpoint{5.377106in}{2.639510in}}%
\pgfpathlineto{\pgfqpoint{5.363432in}{2.641911in}}%
\pgfpathlineto{\pgfqpoint{5.349765in}{2.644335in}}%
\pgfpathlineto{\pgfqpoint{5.336105in}{2.646785in}}%
\pgfpathlineto{\pgfqpoint{5.322453in}{2.649259in}}%
\pgfpathlineto{\pgfqpoint{5.329686in}{2.656037in}}%
\pgfpathlineto{\pgfqpoint{5.336916in}{2.662878in}}%
\pgfpathlineto{\pgfqpoint{5.344142in}{2.669785in}}%
\pgfpathlineto{\pgfqpoint{5.351364in}{2.676764in}}%
\pgfpathclose%
\pgfusepath{fill}%
\end{pgfscope}%
\begin{pgfscope}%
\pgfpathrectangle{\pgfqpoint{1.150000in}{0.150000in}}{\pgfqpoint{5.700000in}{5.700000in}}%
\pgfusepath{clip}%
\pgfsetbuttcap%
\pgfsetroundjoin%
\definecolor{currentfill}{rgb}{0.268510,0.009605,0.335427}%
\pgfsetfillcolor{currentfill}%
\pgfsetfillopacity{0.700000}%
\pgfsetlinewidth{0.000000pt}%
\definecolor{currentstroke}{rgb}{0.000000,0.000000,0.000000}%
\pgfsetstrokecolor{currentstroke}%
\pgfsetdash{}{0pt}%
\pgfpathmoveto{\pgfqpoint{3.747947in}{2.543140in}}%
\pgfpathlineto{\pgfqpoint{3.761206in}{2.539020in}}%
\pgfpathlineto{\pgfqpoint{3.774470in}{2.534933in}}%
\pgfpathlineto{\pgfqpoint{3.787740in}{2.530881in}}%
\pgfpathlineto{\pgfqpoint{3.801015in}{2.526861in}}%
\pgfpathlineto{\pgfqpoint{3.793211in}{2.519371in}}%
\pgfpathlineto{\pgfqpoint{3.785402in}{2.511888in}}%
\pgfpathlineto{\pgfqpoint{3.777587in}{2.504411in}}%
\pgfpathlineto{\pgfqpoint{3.769766in}{2.496942in}}%
\pgfpathlineto{\pgfqpoint{3.756479in}{2.500991in}}%
\pgfpathlineto{\pgfqpoint{3.743196in}{2.505073in}}%
\pgfpathlineto{\pgfqpoint{3.729920in}{2.509188in}}%
\pgfpathlineto{\pgfqpoint{3.716648in}{2.513337in}}%
\pgfpathlineto{\pgfqpoint{3.724482in}{2.520772in}}%
\pgfpathlineto{\pgfqpoint{3.732309in}{2.528218in}}%
\pgfpathlineto{\pgfqpoint{3.740131in}{2.535674in}}%
\pgfpathlineto{\pgfqpoint{3.747947in}{2.543140in}}%
\pgfpathclose%
\pgfusepath{fill}%
\end{pgfscope}%
\begin{pgfscope}%
\pgfpathrectangle{\pgfqpoint{1.150000in}{0.150000in}}{\pgfqpoint{5.700000in}{5.700000in}}%
\pgfusepath{clip}%
\pgfsetbuttcap%
\pgfsetroundjoin%
\definecolor{currentfill}{rgb}{0.271305,0.019942,0.347269}%
\pgfsetfillcolor{currentfill}%
\pgfsetfillopacity{0.700000}%
\pgfsetlinewidth{0.000000pt}%
\definecolor{currentstroke}{rgb}{0.000000,0.000000,0.000000}%
\pgfsetstrokecolor{currentstroke}%
\pgfsetdash{}{0pt}%
\pgfpathmoveto{\pgfqpoint{4.106792in}{2.557508in}}%
\pgfpathlineto{\pgfqpoint{4.120129in}{2.554035in}}%
\pgfpathlineto{\pgfqpoint{4.133471in}{2.550592in}}%
\pgfpathlineto{\pgfqpoint{4.146819in}{2.547179in}}%
\pgfpathlineto{\pgfqpoint{4.160174in}{2.543796in}}%
\pgfpathlineto{\pgfqpoint{4.152500in}{2.536285in}}%
\pgfpathlineto{\pgfqpoint{4.144821in}{2.528769in}}%
\pgfpathlineto{\pgfqpoint{4.137136in}{2.521247in}}%
\pgfpathlineto{\pgfqpoint{4.129446in}{2.513719in}}%
\pgfpathlineto{\pgfqpoint{4.116079in}{2.517091in}}%
\pgfpathlineto{\pgfqpoint{4.102718in}{2.520493in}}%
\pgfpathlineto{\pgfqpoint{4.089363in}{2.523926in}}%
\pgfpathlineto{\pgfqpoint{4.076015in}{2.527388in}}%
\pgfpathlineto{\pgfqpoint{4.083717in}{2.534923in}}%
\pgfpathlineto{\pgfqpoint{4.091415in}{2.542454in}}%
\pgfpathlineto{\pgfqpoint{4.099106in}{2.549982in}}%
\pgfpathlineto{\pgfqpoint{4.106792in}{2.557508in}}%
\pgfpathclose%
\pgfusepath{fill}%
\end{pgfscope}%
\begin{pgfscope}%
\pgfpathrectangle{\pgfqpoint{1.150000in}{0.150000in}}{\pgfqpoint{5.700000in}{5.700000in}}%
\pgfusepath{clip}%
\pgfsetbuttcap%
\pgfsetroundjoin%
\definecolor{currentfill}{rgb}{0.280894,0.078907,0.402329}%
\pgfsetfillcolor{currentfill}%
\pgfsetfillopacity{0.700000}%
\pgfsetlinewidth{0.000000pt}%
\definecolor{currentstroke}{rgb}{0.000000,0.000000,0.000000}%
\pgfsetstrokecolor{currentstroke}%
\pgfsetdash{}{0pt}%
\pgfpathmoveto{\pgfqpoint{5.130010in}{2.652583in}}%
\pgfpathlineto{\pgfqpoint{5.143593in}{2.649953in}}%
\pgfpathlineto{\pgfqpoint{5.157183in}{2.647348in}}%
\pgfpathlineto{\pgfqpoint{5.170780in}{2.644768in}}%
\pgfpathlineto{\pgfqpoint{5.184384in}{2.642213in}}%
\pgfpathlineto{\pgfqpoint{5.177096in}{2.635406in}}%
\pgfpathlineto{\pgfqpoint{5.169804in}{2.628641in}}%
\pgfpathlineto{\pgfqpoint{5.162506in}{2.621913in}}%
\pgfpathlineto{\pgfqpoint{5.155204in}{2.615218in}}%
\pgfpathlineto{\pgfqpoint{5.141584in}{2.617643in}}%
\pgfpathlineto{\pgfqpoint{5.127971in}{2.620094in}}%
\pgfpathlineto{\pgfqpoint{5.114365in}{2.622570in}}%
\pgfpathlineto{\pgfqpoint{5.100766in}{2.625071in}}%
\pgfpathlineto{\pgfqpoint{5.108084in}{2.631891in}}%
\pgfpathlineto{\pgfqpoint{5.115397in}{2.638746in}}%
\pgfpathlineto{\pgfqpoint{5.122706in}{2.645642in}}%
\pgfpathlineto{\pgfqpoint{5.130010in}{2.652583in}}%
\pgfpathclose%
\pgfusepath{fill}%
\end{pgfscope}%
\begin{pgfscope}%
\pgfpathrectangle{\pgfqpoint{1.150000in}{0.150000in}}{\pgfqpoint{5.700000in}{5.700000in}}%
\pgfusepath{clip}%
\pgfsetbuttcap%
\pgfsetroundjoin%
\definecolor{currentfill}{rgb}{0.279566,0.067836,0.391917}%
\pgfsetfillcolor{currentfill}%
\pgfsetfillopacity{0.700000}%
\pgfsetlinewidth{0.000000pt}%
\definecolor{currentstroke}{rgb}{0.000000,0.000000,0.000000}%
\pgfsetstrokecolor{currentstroke}%
\pgfsetdash{}{0pt}%
\pgfpathmoveto{\pgfqpoint{4.908624in}{2.628685in}}%
\pgfpathlineto{\pgfqpoint{4.922154in}{2.625993in}}%
\pgfpathlineto{\pgfqpoint{4.935691in}{2.623328in}}%
\pgfpathlineto{\pgfqpoint{4.949234in}{2.620688in}}%
\pgfpathlineto{\pgfqpoint{4.962785in}{2.618073in}}%
\pgfpathlineto{\pgfqpoint{4.955412in}{2.611162in}}%
\pgfpathlineto{\pgfqpoint{4.948033in}{2.604272in}}%
\pgfpathlineto{\pgfqpoint{4.940649in}{2.597398in}}%
\pgfpathlineto{\pgfqpoint{4.933260in}{2.590538in}}%
\pgfpathlineto{\pgfqpoint{4.919694in}{2.593050in}}%
\pgfpathlineto{\pgfqpoint{4.906136in}{2.595587in}}%
\pgfpathlineto{\pgfqpoint{4.892584in}{2.598150in}}%
\pgfpathlineto{\pgfqpoint{4.879040in}{2.600738in}}%
\pgfpathlineto{\pgfqpoint{4.886443in}{2.607696in}}%
\pgfpathlineto{\pgfqpoint{4.893842in}{2.614671in}}%
\pgfpathlineto{\pgfqpoint{4.901236in}{2.621666in}}%
\pgfpathlineto{\pgfqpoint{4.908624in}{2.628685in}}%
\pgfpathclose%
\pgfusepath{fill}%
\end{pgfscope}%
\begin{pgfscope}%
\pgfpathrectangle{\pgfqpoint{1.150000in}{0.150000in}}{\pgfqpoint{5.700000in}{5.700000in}}%
\pgfusepath{clip}%
\pgfsetbuttcap%
\pgfsetroundjoin%
\definecolor{currentfill}{rgb}{0.271305,0.019942,0.347269}%
\pgfsetfillcolor{currentfill}%
\pgfsetfillopacity{0.700000}%
\pgfsetlinewidth{0.000000pt}%
\definecolor{currentstroke}{rgb}{0.000000,0.000000,0.000000}%
\pgfsetstrokecolor{currentstroke}%
\pgfsetdash{}{0pt}%
\pgfpathmoveto{\pgfqpoint{3.251517in}{2.560638in}}%
\pgfpathlineto{\pgfqpoint{3.264692in}{2.555278in}}%
\pgfpathlineto{\pgfqpoint{3.277871in}{2.549960in}}%
\pgfpathlineto{\pgfqpoint{3.291055in}{2.544682in}}%
\pgfpathlineto{\pgfqpoint{3.304243in}{2.539445in}}%
\pgfpathlineto{\pgfqpoint{3.296248in}{2.532676in}}%
\pgfpathlineto{\pgfqpoint{3.288245in}{2.525950in}}%
\pgfpathlineto{\pgfqpoint{3.280236in}{2.519269in}}%
\pgfpathlineto{\pgfqpoint{3.272219in}{2.512636in}}%
\pgfpathlineto{\pgfqpoint{3.259016in}{2.517955in}}%
\pgfpathlineto{\pgfqpoint{3.245818in}{2.523314in}}%
\pgfpathlineto{\pgfqpoint{3.232623in}{2.528715in}}%
\pgfpathlineto{\pgfqpoint{3.219433in}{2.534156in}}%
\pgfpathlineto{\pgfqpoint{3.227465in}{2.540702in}}%
\pgfpathlineto{\pgfqpoint{3.235489in}{2.547299in}}%
\pgfpathlineto{\pgfqpoint{3.243506in}{2.553945in}}%
\pgfpathlineto{\pgfqpoint{3.251517in}{2.560638in}}%
\pgfpathclose%
\pgfusepath{fill}%
\end{pgfscope}%
\begin{pgfscope}%
\pgfpathrectangle{\pgfqpoint{1.150000in}{0.150000in}}{\pgfqpoint{5.700000in}{5.700000in}}%
\pgfusepath{clip}%
\pgfsetbuttcap%
\pgfsetroundjoin%
\definecolor{currentfill}{rgb}{0.277018,0.050344,0.375715}%
\pgfsetfillcolor{currentfill}%
\pgfsetfillopacity{0.700000}%
\pgfsetlinewidth{0.000000pt}%
\definecolor{currentstroke}{rgb}{0.000000,0.000000,0.000000}%
\pgfsetstrokecolor{currentstroke}%
\pgfsetdash{}{0pt}%
\pgfpathmoveto{\pgfqpoint{4.687202in}{2.605046in}}%
\pgfpathlineto{\pgfqpoint{4.700678in}{2.602230in}}%
\pgfpathlineto{\pgfqpoint{4.714162in}{2.599442in}}%
\pgfpathlineto{\pgfqpoint{4.727652in}{2.596680in}}%
\pgfpathlineto{\pgfqpoint{4.741149in}{2.593945in}}%
\pgfpathlineto{\pgfqpoint{4.733690in}{2.586862in}}%
\pgfpathlineto{\pgfqpoint{4.726225in}{2.579783in}}%
\pgfpathlineto{\pgfqpoint{4.718755in}{2.572708in}}%
\pgfpathlineto{\pgfqpoint{4.711280in}{2.565633in}}%
\pgfpathlineto{\pgfqpoint{4.697769in}{2.568291in}}%
\pgfpathlineto{\pgfqpoint{4.684265in}{2.570977in}}%
\pgfpathlineto{\pgfqpoint{4.670768in}{2.573689in}}%
\pgfpathlineto{\pgfqpoint{4.657278in}{2.576427in}}%
\pgfpathlineto{\pgfqpoint{4.664767in}{2.583574in}}%
\pgfpathlineto{\pgfqpoint{4.672251in}{2.590725in}}%
\pgfpathlineto{\pgfqpoint{4.679729in}{2.597881in}}%
\pgfpathlineto{\pgfqpoint{4.687202in}{2.605046in}}%
\pgfpathclose%
\pgfusepath{fill}%
\end{pgfscope}%
\begin{pgfscope}%
\pgfpathrectangle{\pgfqpoint{1.150000in}{0.150000in}}{\pgfqpoint{5.700000in}{5.700000in}}%
\pgfusepath{clip}%
\pgfsetbuttcap%
\pgfsetroundjoin%
\definecolor{currentfill}{rgb}{0.273809,0.031497,0.358853}%
\pgfsetfillcolor{currentfill}%
\pgfsetfillopacity{0.700000}%
\pgfsetlinewidth{0.000000pt}%
\definecolor{currentstroke}{rgb}{0.000000,0.000000,0.000000}%
\pgfsetstrokecolor{currentstroke}%
\pgfsetdash{}{0pt}%
\pgfpathmoveto{\pgfqpoint{3.114063in}{2.579218in}}%
\pgfpathlineto{\pgfqpoint{3.127220in}{2.573433in}}%
\pgfpathlineto{\pgfqpoint{3.140381in}{2.567692in}}%
\pgfpathlineto{\pgfqpoint{3.153546in}{2.561996in}}%
\pgfpathlineto{\pgfqpoint{3.166715in}{2.556342in}}%
\pgfpathlineto{\pgfqpoint{3.158661in}{2.549939in}}%
\pgfpathlineto{\pgfqpoint{3.150600in}{2.543593in}}%
\pgfpathlineto{\pgfqpoint{3.142531in}{2.537306in}}%
\pgfpathlineto{\pgfqpoint{3.134455in}{2.531082in}}%
\pgfpathlineto{\pgfqpoint{3.121270in}{2.536830in}}%
\pgfpathlineto{\pgfqpoint{3.108089in}{2.542622in}}%
\pgfpathlineto{\pgfqpoint{3.094912in}{2.548458in}}%
\pgfpathlineto{\pgfqpoint{3.081739in}{2.554339in}}%
\pgfpathlineto{\pgfqpoint{3.089831in}{2.560463in}}%
\pgfpathlineto{\pgfqpoint{3.097916in}{2.566653in}}%
\pgfpathlineto{\pgfqpoint{3.105993in}{2.572905in}}%
\pgfpathlineto{\pgfqpoint{3.114063in}{2.579218in}}%
\pgfpathclose%
\pgfusepath{fill}%
\end{pgfscope}%
\begin{pgfscope}%
\pgfpathrectangle{\pgfqpoint{1.150000in}{0.150000in}}{\pgfqpoint{5.700000in}{5.700000in}}%
\pgfusepath{clip}%
\pgfsetbuttcap%
\pgfsetroundjoin%
\definecolor{currentfill}{rgb}{0.269944,0.014625,0.341379}%
\pgfsetfillcolor{currentfill}%
\pgfsetfillopacity{0.700000}%
\pgfsetlinewidth{0.000000pt}%
\definecolor{currentstroke}{rgb}{0.000000,0.000000,0.000000}%
\pgfsetstrokecolor{currentstroke}%
\pgfsetdash{}{0pt}%
\pgfpathmoveto{\pgfqpoint{3.388901in}{2.546660in}}%
\pgfpathlineto{\pgfqpoint{3.402097in}{2.541690in}}%
\pgfpathlineto{\pgfqpoint{3.415299in}{2.536759in}}%
\pgfpathlineto{\pgfqpoint{3.428505in}{2.531866in}}%
\pgfpathlineto{\pgfqpoint{3.441716in}{2.527011in}}%
\pgfpathlineto{\pgfqpoint{3.433775in}{2.519953in}}%
\pgfpathlineto{\pgfqpoint{3.425827in}{2.512927in}}%
\pgfpathlineto{\pgfqpoint{3.417873in}{2.505934in}}%
\pgfpathlineto{\pgfqpoint{3.409913in}{2.498974in}}%
\pgfpathlineto{\pgfqpoint{3.396688in}{2.503898in}}%
\pgfpathlineto{\pgfqpoint{3.383468in}{2.508859in}}%
\pgfpathlineto{\pgfqpoint{3.370252in}{2.513859in}}%
\pgfpathlineto{\pgfqpoint{3.357041in}{2.518898in}}%
\pgfpathlineto{\pgfqpoint{3.365016in}{2.525783in}}%
\pgfpathlineto{\pgfqpoint{3.372984in}{2.532707in}}%
\pgfpathlineto{\pgfqpoint{3.380946in}{2.539666in}}%
\pgfpathlineto{\pgfqpoint{3.388901in}{2.546660in}}%
\pgfpathclose%
\pgfusepath{fill}%
\end{pgfscope}%
\begin{pgfscope}%
\pgfpathrectangle{\pgfqpoint{1.150000in}{0.150000in}}{\pgfqpoint{5.700000in}{5.700000in}}%
\pgfusepath{clip}%
\pgfsetbuttcap%
\pgfsetroundjoin%
\definecolor{currentfill}{rgb}{0.269944,0.014625,0.341379}%
\pgfsetfillcolor{currentfill}%
\pgfsetfillopacity{0.700000}%
\pgfsetlinewidth{0.000000pt}%
\definecolor{currentstroke}{rgb}{0.000000,0.000000,0.000000}%
\pgfsetstrokecolor{currentstroke}%
\pgfsetdash{}{0pt}%
\pgfpathmoveto{\pgfqpoint{3.885277in}{2.541202in}}%
\pgfpathlineto{\pgfqpoint{3.898568in}{2.537361in}}%
\pgfpathlineto{\pgfqpoint{3.911864in}{2.533553in}}%
\pgfpathlineto{\pgfqpoint{3.925166in}{2.529777in}}%
\pgfpathlineto{\pgfqpoint{3.938474in}{2.526033in}}%
\pgfpathlineto{\pgfqpoint{3.930718in}{2.518494in}}%
\pgfpathlineto{\pgfqpoint{3.922957in}{2.510956in}}%
\pgfpathlineto{\pgfqpoint{3.915190in}{2.503418in}}%
\pgfpathlineto{\pgfqpoint{3.907417in}{2.495880in}}%
\pgfpathlineto{\pgfqpoint{3.894097in}{2.499640in}}%
\pgfpathlineto{\pgfqpoint{3.880782in}{2.503432in}}%
\pgfpathlineto{\pgfqpoint{3.867474in}{2.507256in}}%
\pgfpathlineto{\pgfqpoint{3.854171in}{2.511112in}}%
\pgfpathlineto{\pgfqpoint{3.861956in}{2.518629in}}%
\pgfpathlineto{\pgfqpoint{3.869735in}{2.526150in}}%
\pgfpathlineto{\pgfqpoint{3.877509in}{2.533674in}}%
\pgfpathlineto{\pgfqpoint{3.885277in}{2.541202in}}%
\pgfpathclose%
\pgfusepath{fill}%
\end{pgfscope}%
\begin{pgfscope}%
\pgfpathrectangle{\pgfqpoint{1.150000in}{0.150000in}}{\pgfqpoint{5.700000in}{5.700000in}}%
\pgfusepath{clip}%
\pgfsetbuttcap%
\pgfsetroundjoin%
\definecolor{currentfill}{rgb}{0.274952,0.037752,0.364543}%
\pgfsetfillcolor{currentfill}%
\pgfsetfillopacity{0.700000}%
\pgfsetlinewidth{0.000000pt}%
\definecolor{currentstroke}{rgb}{0.000000,0.000000,0.000000}%
\pgfsetstrokecolor{currentstroke}%
\pgfsetdash{}{0pt}%
\pgfpathmoveto{\pgfqpoint{4.465743in}{2.582047in}}%
\pgfpathlineto{\pgfqpoint{4.479167in}{2.579044in}}%
\pgfpathlineto{\pgfqpoint{4.492598in}{2.576069in}}%
\pgfpathlineto{\pgfqpoint{4.506034in}{2.573121in}}%
\pgfpathlineto{\pgfqpoint{4.519478in}{2.570201in}}%
\pgfpathlineto{\pgfqpoint{4.511935in}{2.562924in}}%
\pgfpathlineto{\pgfqpoint{4.504386in}{2.555643in}}%
\pgfpathlineto{\pgfqpoint{4.496831in}{2.548357in}}%
\pgfpathlineto{\pgfqpoint{4.489271in}{2.541063in}}%
\pgfpathlineto{\pgfqpoint{4.475814in}{2.543933in}}%
\pgfpathlineto{\pgfqpoint{4.462364in}{2.546831in}}%
\pgfpathlineto{\pgfqpoint{4.448921in}{2.549756in}}%
\pgfpathlineto{\pgfqpoint{4.435484in}{2.552709in}}%
\pgfpathlineto{\pgfqpoint{4.443057in}{2.560049in}}%
\pgfpathlineto{\pgfqpoint{4.450625in}{2.567384in}}%
\pgfpathlineto{\pgfqpoint{4.458187in}{2.574716in}}%
\pgfpathlineto{\pgfqpoint{4.465743in}{2.582047in}}%
\pgfpathclose%
\pgfusepath{fill}%
\end{pgfscope}%
\begin{pgfscope}%
\pgfpathrectangle{\pgfqpoint{1.150000in}{0.150000in}}{\pgfqpoint{5.700000in}{5.700000in}}%
\pgfusepath{clip}%
\pgfsetbuttcap%
\pgfsetroundjoin%
\definecolor{currentfill}{rgb}{0.277018,0.050344,0.375715}%
\pgfsetfillcolor{currentfill}%
\pgfsetfillopacity{0.700000}%
\pgfsetlinewidth{0.000000pt}%
\definecolor{currentstroke}{rgb}{0.000000,0.000000,0.000000}%
\pgfsetstrokecolor{currentstroke}%
\pgfsetdash{}{0pt}%
\pgfpathmoveto{\pgfqpoint{2.976489in}{2.603027in}}%
\pgfpathlineto{\pgfqpoint{2.989632in}{2.596777in}}%
\pgfpathlineto{\pgfqpoint{3.002779in}{2.590574in}}%
\pgfpathlineto{\pgfqpoint{3.015930in}{2.584419in}}%
\pgfpathlineto{\pgfqpoint{3.029084in}{2.578311in}}%
\pgfpathlineto{\pgfqpoint{3.020968in}{2.572358in}}%
\pgfpathlineto{\pgfqpoint{3.012843in}{2.566477in}}%
\pgfpathlineto{\pgfqpoint{3.004711in}{2.560672in}}%
\pgfpathlineto{\pgfqpoint{2.996570in}{2.554944in}}%
\pgfpathlineto{\pgfqpoint{2.983399in}{2.561161in}}%
\pgfpathlineto{\pgfqpoint{2.970231in}{2.567425in}}%
\pgfpathlineto{\pgfqpoint{2.957067in}{2.573736in}}%
\pgfpathlineto{\pgfqpoint{2.943906in}{2.580094in}}%
\pgfpathlineto{\pgfqpoint{2.952064in}{2.585708in}}%
\pgfpathlineto{\pgfqpoint{2.960214in}{2.591403in}}%
\pgfpathlineto{\pgfqpoint{2.968355in}{2.597177in}}%
\pgfpathlineto{\pgfqpoint{2.976489in}{2.603027in}}%
\pgfpathclose%
\pgfusepath{fill}%
\end{pgfscope}%
\begin{pgfscope}%
\pgfpathrectangle{\pgfqpoint{1.150000in}{0.150000in}}{\pgfqpoint{5.700000in}{5.700000in}}%
\pgfusepath{clip}%
\pgfsetbuttcap%
\pgfsetroundjoin%
\definecolor{currentfill}{rgb}{0.268510,0.009605,0.335427}%
\pgfsetfillcolor{currentfill}%
\pgfsetfillopacity{0.700000}%
\pgfsetlinewidth{0.000000pt}%
\definecolor{currentstroke}{rgb}{0.000000,0.000000,0.000000}%
\pgfsetstrokecolor{currentstroke}%
\pgfsetdash{}{0pt}%
\pgfpathmoveto{\pgfqpoint{3.526255in}{2.536714in}}%
\pgfpathlineto{\pgfqpoint{3.539477in}{2.532101in}}%
\pgfpathlineto{\pgfqpoint{3.552704in}{2.527525in}}%
\pgfpathlineto{\pgfqpoint{3.565936in}{2.522984in}}%
\pgfpathlineto{\pgfqpoint{3.579173in}{2.518480in}}%
\pgfpathlineto{\pgfqpoint{3.571284in}{2.511206in}}%
\pgfpathlineto{\pgfqpoint{3.563388in}{2.503952in}}%
\pgfpathlineto{\pgfqpoint{3.555487in}{2.496721in}}%
\pgfpathlineto{\pgfqpoint{3.547580in}{2.489512in}}%
\pgfpathlineto{\pgfqpoint{3.534329in}{2.494072in}}%
\pgfpathlineto{\pgfqpoint{3.521084in}{2.498667in}}%
\pgfpathlineto{\pgfqpoint{3.507843in}{2.503299in}}%
\pgfpathlineto{\pgfqpoint{3.494608in}{2.507968in}}%
\pgfpathlineto{\pgfqpoint{3.502529in}{2.515116in}}%
\pgfpathlineto{\pgfqpoint{3.510444in}{2.522291in}}%
\pgfpathlineto{\pgfqpoint{3.518353in}{2.529491in}}%
\pgfpathlineto{\pgfqpoint{3.526255in}{2.536714in}}%
\pgfpathclose%
\pgfusepath{fill}%
\end{pgfscope}%
\begin{pgfscope}%
\pgfpathrectangle{\pgfqpoint{1.150000in}{0.150000in}}{\pgfqpoint{5.700000in}{5.700000in}}%
\pgfusepath{clip}%
\pgfsetbuttcap%
\pgfsetroundjoin%
\definecolor{currentfill}{rgb}{0.282656,0.100196,0.422160}%
\pgfsetfillcolor{currentfill}%
\pgfsetfillopacity{0.700000}%
\pgfsetlinewidth{0.000000pt}%
\definecolor{currentstroke}{rgb}{0.000000,0.000000,0.000000}%
\pgfsetstrokecolor{currentstroke}%
\pgfsetdash{}{0pt}%
\pgfpathmoveto{\pgfqpoint{5.489362in}{2.683801in}}%
\pgfpathlineto{\pgfqpoint{5.503037in}{2.681199in}}%
\pgfpathlineto{\pgfqpoint{5.516719in}{2.678621in}}%
\pgfpathlineto{\pgfqpoint{5.530409in}{2.676067in}}%
\pgfpathlineto{\pgfqpoint{5.544106in}{2.673537in}}%
\pgfpathlineto{\pgfqpoint{5.536950in}{2.666741in}}%
\pgfpathlineto{\pgfqpoint{5.529791in}{2.660027in}}%
\pgfpathlineto{\pgfqpoint{5.522630in}{2.653390in}}%
\pgfpathlineto{\pgfqpoint{5.515465in}{2.646825in}}%
\pgfpathlineto{\pgfqpoint{5.501749in}{2.649187in}}%
\pgfpathlineto{\pgfqpoint{5.488041in}{2.651572in}}%
\pgfpathlineto{\pgfqpoint{5.474340in}{2.653982in}}%
\pgfpathlineto{\pgfqpoint{5.460647in}{2.656415in}}%
\pgfpathlineto{\pgfqpoint{5.467831in}{2.663144in}}%
\pgfpathlineto{\pgfqpoint{5.475011in}{2.669947in}}%
\pgfpathlineto{\pgfqpoint{5.482188in}{2.676831in}}%
\pgfpathlineto{\pgfqpoint{5.489362in}{2.683801in}}%
\pgfpathclose%
\pgfusepath{fill}%
\end{pgfscope}%
\begin{pgfscope}%
\pgfpathrectangle{\pgfqpoint{1.150000in}{0.150000in}}{\pgfqpoint{5.700000in}{5.700000in}}%
\pgfusepath{clip}%
\pgfsetbuttcap%
\pgfsetroundjoin%
\definecolor{currentfill}{rgb}{0.272594,0.025563,0.353093}%
\pgfsetfillcolor{currentfill}%
\pgfsetfillopacity{0.700000}%
\pgfsetlinewidth{0.000000pt}%
\definecolor{currentstroke}{rgb}{0.000000,0.000000,0.000000}%
\pgfsetstrokecolor{currentstroke}%
\pgfsetdash{}{0pt}%
\pgfpathmoveto{\pgfqpoint{4.244242in}{2.560480in}}%
\pgfpathlineto{\pgfqpoint{4.257615in}{2.557221in}}%
\pgfpathlineto{\pgfqpoint{4.270994in}{2.553991in}}%
\pgfpathlineto{\pgfqpoint{4.284380in}{2.550791in}}%
\pgfpathlineto{\pgfqpoint{4.297772in}{2.547619in}}%
\pgfpathlineto{\pgfqpoint{4.290145in}{2.540175in}}%
\pgfpathlineto{\pgfqpoint{4.282513in}{2.532724in}}%
\pgfpathlineto{\pgfqpoint{4.274876in}{2.525265in}}%
\pgfpathlineto{\pgfqpoint{4.267233in}{2.517795in}}%
\pgfpathlineto{\pgfqpoint{4.253828in}{2.520943in}}%
\pgfpathlineto{\pgfqpoint{4.240430in}{2.524120in}}%
\pgfpathlineto{\pgfqpoint{4.227039in}{2.527326in}}%
\pgfpathlineto{\pgfqpoint{4.213653in}{2.530561in}}%
\pgfpathlineto{\pgfqpoint{4.221309in}{2.538049in}}%
\pgfpathlineto{\pgfqpoint{4.228959in}{2.545531in}}%
\pgfpathlineto{\pgfqpoint{4.236603in}{2.553007in}}%
\pgfpathlineto{\pgfqpoint{4.244242in}{2.560480in}}%
\pgfpathclose%
\pgfusepath{fill}%
\end{pgfscope}%
\begin{pgfscope}%
\pgfpathrectangle{\pgfqpoint{1.150000in}{0.150000in}}{\pgfqpoint{5.700000in}{5.700000in}}%
\pgfusepath{clip}%
\pgfsetbuttcap%
\pgfsetroundjoin%
\definecolor{currentfill}{rgb}{0.281446,0.084320,0.407414}%
\pgfsetfillcolor{currentfill}%
\pgfsetfillopacity{0.700000}%
\pgfsetlinewidth{0.000000pt}%
\definecolor{currentstroke}{rgb}{0.000000,0.000000,0.000000}%
\pgfsetstrokecolor{currentstroke}%
\pgfsetdash{}{0pt}%
\pgfpathmoveto{\pgfqpoint{5.267916in}{2.659401in}}%
\pgfpathlineto{\pgfqpoint{5.281539in}{2.656828in}}%
\pgfpathlineto{\pgfqpoint{5.295170in}{2.654281in}}%
\pgfpathlineto{\pgfqpoint{5.308808in}{2.651757in}}%
\pgfpathlineto{\pgfqpoint{5.322453in}{2.649259in}}%
\pgfpathlineto{\pgfqpoint{5.315215in}{2.642537in}}%
\pgfpathlineto{\pgfqpoint{5.307974in}{2.635868in}}%
\pgfpathlineto{\pgfqpoint{5.300728in}{2.629247in}}%
\pgfpathlineto{\pgfqpoint{5.293478in}{2.622670in}}%
\pgfpathlineto{\pgfqpoint{5.279816in}{2.625026in}}%
\pgfpathlineto{\pgfqpoint{5.266161in}{2.627407in}}%
\pgfpathlineto{\pgfqpoint{5.252513in}{2.629812in}}%
\pgfpathlineto{\pgfqpoint{5.238873in}{2.632243in}}%
\pgfpathlineto{\pgfqpoint{5.246140in}{2.638958in}}%
\pgfpathlineto{\pgfqpoint{5.253403in}{2.645719in}}%
\pgfpathlineto{\pgfqpoint{5.260661in}{2.652532in}}%
\pgfpathlineto{\pgfqpoint{5.267916in}{2.659401in}}%
\pgfpathclose%
\pgfusepath{fill}%
\end{pgfscope}%
\begin{pgfscope}%
\pgfpathrectangle{\pgfqpoint{1.150000in}{0.150000in}}{\pgfqpoint{5.700000in}{5.700000in}}%
\pgfusepath{clip}%
\pgfsetbuttcap%
\pgfsetroundjoin%
\definecolor{currentfill}{rgb}{0.280267,0.073417,0.397163}%
\pgfsetfillcolor{currentfill}%
\pgfsetfillopacity{0.700000}%
\pgfsetlinewidth{0.000000pt}%
\definecolor{currentstroke}{rgb}{0.000000,0.000000,0.000000}%
\pgfsetstrokecolor{currentstroke}%
\pgfsetdash{}{0pt}%
\pgfpathmoveto{\pgfqpoint{5.046441in}{2.635329in}}%
\pgfpathlineto{\pgfqpoint{5.060012in}{2.632726in}}%
\pgfpathlineto{\pgfqpoint{5.073589in}{2.630149in}}%
\pgfpathlineto{\pgfqpoint{5.087174in}{2.627598in}}%
\pgfpathlineto{\pgfqpoint{5.100766in}{2.625071in}}%
\pgfpathlineto{\pgfqpoint{5.093443in}{2.618284in}}%
\pgfpathlineto{\pgfqpoint{5.086115in}{2.611526in}}%
\pgfpathlineto{\pgfqpoint{5.078783in}{2.604793in}}%
\pgfpathlineto{\pgfqpoint{5.071445in}{2.598080in}}%
\pgfpathlineto{\pgfqpoint{5.057838in}{2.600490in}}%
\pgfpathlineto{\pgfqpoint{5.044237in}{2.602926in}}%
\pgfpathlineto{\pgfqpoint{5.030644in}{2.605387in}}%
\pgfpathlineto{\pgfqpoint{5.017058in}{2.607873in}}%
\pgfpathlineto{\pgfqpoint{5.024411in}{2.614697in}}%
\pgfpathlineto{\pgfqpoint{5.031760in}{2.621545in}}%
\pgfpathlineto{\pgfqpoint{5.039103in}{2.628421in}}%
\pgfpathlineto{\pgfqpoint{5.046441in}{2.635329in}}%
\pgfpathclose%
\pgfusepath{fill}%
\end{pgfscope}%
\begin{pgfscope}%
\pgfpathrectangle{\pgfqpoint{1.150000in}{0.150000in}}{\pgfqpoint{5.700000in}{5.700000in}}%
\pgfusepath{clip}%
\pgfsetbuttcap%
\pgfsetroundjoin%
\definecolor{currentfill}{rgb}{0.268510,0.009605,0.335427}%
\pgfsetfillcolor{currentfill}%
\pgfsetfillopacity{0.700000}%
\pgfsetlinewidth{0.000000pt}%
\definecolor{currentstroke}{rgb}{0.000000,0.000000,0.000000}%
\pgfsetstrokecolor{currentstroke}%
\pgfsetdash{}{0pt}%
\pgfpathmoveto{\pgfqpoint{3.663617in}{2.530275in}}%
\pgfpathlineto{\pgfqpoint{3.676867in}{2.525989in}}%
\pgfpathlineto{\pgfqpoint{3.690122in}{2.521737in}}%
\pgfpathlineto{\pgfqpoint{3.703382in}{2.517520in}}%
\pgfpathlineto{\pgfqpoint{3.716648in}{2.513337in}}%
\pgfpathlineto{\pgfqpoint{3.708809in}{2.505913in}}%
\pgfpathlineto{\pgfqpoint{3.700964in}{2.498500in}}%
\pgfpathlineto{\pgfqpoint{3.693113in}{2.491099in}}%
\pgfpathlineto{\pgfqpoint{3.685256in}{2.483711in}}%
\pgfpathlineto{\pgfqpoint{3.671977in}{2.487936in}}%
\pgfpathlineto{\pgfqpoint{3.658703in}{2.492195in}}%
\pgfpathlineto{\pgfqpoint{3.645435in}{2.496489in}}%
\pgfpathlineto{\pgfqpoint{3.632172in}{2.500817in}}%
\pgfpathlineto{\pgfqpoint{3.640042in}{2.508158in}}%
\pgfpathlineto{\pgfqpoint{3.647906in}{2.515515in}}%
\pgfpathlineto{\pgfqpoint{3.655764in}{2.522888in}}%
\pgfpathlineto{\pgfqpoint{3.663617in}{2.530275in}}%
\pgfpathclose%
\pgfusepath{fill}%
\end{pgfscope}%
\begin{pgfscope}%
\pgfpathrectangle{\pgfqpoint{1.150000in}{0.150000in}}{\pgfqpoint{5.700000in}{5.700000in}}%
\pgfusepath{clip}%
\pgfsetbuttcap%
\pgfsetroundjoin%
\definecolor{currentfill}{rgb}{0.269944,0.014625,0.341379}%
\pgfsetfillcolor{currentfill}%
\pgfsetfillopacity{0.700000}%
\pgfsetlinewidth{0.000000pt}%
\definecolor{currentstroke}{rgb}{0.000000,0.000000,0.000000}%
\pgfsetstrokecolor{currentstroke}%
\pgfsetdash{}{0pt}%
\pgfpathmoveto{\pgfqpoint{4.022680in}{2.541544in}}%
\pgfpathlineto{\pgfqpoint{4.036005in}{2.537959in}}%
\pgfpathlineto{\pgfqpoint{4.049335in}{2.534405in}}%
\pgfpathlineto{\pgfqpoint{4.062672in}{2.530881in}}%
\pgfpathlineto{\pgfqpoint{4.076015in}{2.527388in}}%
\pgfpathlineto{\pgfqpoint{4.068306in}{2.519850in}}%
\pgfpathlineto{\pgfqpoint{4.060592in}{2.512307in}}%
\pgfpathlineto{\pgfqpoint{4.052873in}{2.504758in}}%
\pgfpathlineto{\pgfqpoint{4.045147in}{2.497204in}}%
\pgfpathlineto{\pgfqpoint{4.031792in}{2.500700in}}%
\pgfpathlineto{\pgfqpoint{4.018443in}{2.504226in}}%
\pgfpathlineto{\pgfqpoint{4.005100in}{2.507783in}}%
\pgfpathlineto{\pgfqpoint{3.991763in}{2.511370in}}%
\pgfpathlineto{\pgfqpoint{3.999501in}{2.518917in}}%
\pgfpathlineto{\pgfqpoint{4.007233in}{2.526461in}}%
\pgfpathlineto{\pgfqpoint{4.014959in}{2.534004in}}%
\pgfpathlineto{\pgfqpoint{4.022680in}{2.541544in}}%
\pgfpathclose%
\pgfusepath{fill}%
\end{pgfscope}%
\begin{pgfscope}%
\pgfpathrectangle{\pgfqpoint{1.150000in}{0.150000in}}{\pgfqpoint{5.700000in}{5.700000in}}%
\pgfusepath{clip}%
\pgfsetbuttcap%
\pgfsetroundjoin%
\definecolor{currentfill}{rgb}{0.279566,0.067836,0.391917}%
\pgfsetfillcolor{currentfill}%
\pgfsetfillopacity{0.700000}%
\pgfsetlinewidth{0.000000pt}%
\definecolor{currentstroke}{rgb}{0.000000,0.000000,0.000000}%
\pgfsetstrokecolor{currentstroke}%
\pgfsetdash{}{0pt}%
\pgfpathmoveto{\pgfqpoint{2.838738in}{2.632744in}}%
\pgfpathlineto{\pgfqpoint{2.851873in}{2.625985in}}%
\pgfpathlineto{\pgfqpoint{2.865011in}{2.619278in}}%
\pgfpathlineto{\pgfqpoint{2.878152in}{2.612622in}}%
\pgfpathlineto{\pgfqpoint{2.891297in}{2.606017in}}%
\pgfpathlineto{\pgfqpoint{2.883112in}{2.600604in}}%
\pgfpathlineto{\pgfqpoint{2.874919in}{2.595280in}}%
\pgfpathlineto{\pgfqpoint{2.866717in}{2.590049in}}%
\pgfpathlineto{\pgfqpoint{2.858507in}{2.584913in}}%
\pgfpathlineto{\pgfqpoint{2.845344in}{2.591641in}}%
\pgfpathlineto{\pgfqpoint{2.832185in}{2.598419in}}%
\pgfpathlineto{\pgfqpoint{2.819028in}{2.605248in}}%
\pgfpathlineto{\pgfqpoint{2.805875in}{2.612129in}}%
\pgfpathlineto{\pgfqpoint{2.814104in}{2.617138in}}%
\pgfpathlineto{\pgfqpoint{2.822325in}{2.622245in}}%
\pgfpathlineto{\pgfqpoint{2.830536in}{2.627448in}}%
\pgfpathlineto{\pgfqpoint{2.838738in}{2.632744in}}%
\pgfpathclose%
\pgfusepath{fill}%
\end{pgfscope}%
\begin{pgfscope}%
\pgfpathrectangle{\pgfqpoint{1.150000in}{0.150000in}}{\pgfqpoint{5.700000in}{5.700000in}}%
\pgfusepath{clip}%
\pgfsetbuttcap%
\pgfsetroundjoin%
\definecolor{currentfill}{rgb}{0.278791,0.062145,0.386592}%
\pgfsetfillcolor{currentfill}%
\pgfsetfillopacity{0.700000}%
\pgfsetlinewidth{0.000000pt}%
\definecolor{currentstroke}{rgb}{0.000000,0.000000,0.000000}%
\pgfsetstrokecolor{currentstroke}%
\pgfsetdash{}{0pt}%
\pgfpathmoveto{\pgfqpoint{4.824931in}{2.611354in}}%
\pgfpathlineto{\pgfqpoint{4.838448in}{2.608661in}}%
\pgfpathlineto{\pgfqpoint{4.851971in}{2.605994in}}%
\pgfpathlineto{\pgfqpoint{4.865502in}{2.603353in}}%
\pgfpathlineto{\pgfqpoint{4.879040in}{2.600738in}}%
\pgfpathlineto{\pgfqpoint{4.871631in}{2.593794in}}%
\pgfpathlineto{\pgfqpoint{4.864216in}{2.586860in}}%
\pgfpathlineto{\pgfqpoint{4.856796in}{2.579934in}}%
\pgfpathlineto{\pgfqpoint{4.849371in}{2.573012in}}%
\pgfpathlineto{\pgfqpoint{4.835819in}{2.575537in}}%
\pgfpathlineto{\pgfqpoint{4.822274in}{2.578088in}}%
\pgfpathlineto{\pgfqpoint{4.808736in}{2.580665in}}%
\pgfpathlineto{\pgfqpoint{4.795205in}{2.583269in}}%
\pgfpathlineto{\pgfqpoint{4.802644in}{2.590275in}}%
\pgfpathlineto{\pgfqpoint{4.810078in}{2.597290in}}%
\pgfpathlineto{\pgfqpoint{4.817507in}{2.604315in}}%
\pgfpathlineto{\pgfqpoint{4.824931in}{2.611354in}}%
\pgfpathclose%
\pgfusepath{fill}%
\end{pgfscope}%
\begin{pgfscope}%
\pgfpathrectangle{\pgfqpoint{1.150000in}{0.150000in}}{\pgfqpoint{5.700000in}{5.700000in}}%
\pgfusepath{clip}%
\pgfsetbuttcap%
\pgfsetroundjoin%
\definecolor{currentfill}{rgb}{0.276022,0.044167,0.370164}%
\pgfsetfillcolor{currentfill}%
\pgfsetfillopacity{0.700000}%
\pgfsetlinewidth{0.000000pt}%
\definecolor{currentstroke}{rgb}{0.000000,0.000000,0.000000}%
\pgfsetstrokecolor{currentstroke}%
\pgfsetdash{}{0pt}%
\pgfpathmoveto{\pgfqpoint{4.603384in}{2.587652in}}%
\pgfpathlineto{\pgfqpoint{4.616847in}{2.584805in}}%
\pgfpathlineto{\pgfqpoint{4.630318in}{2.581986in}}%
\pgfpathlineto{\pgfqpoint{4.643794in}{2.579193in}}%
\pgfpathlineto{\pgfqpoint{4.657278in}{2.576427in}}%
\pgfpathlineto{\pgfqpoint{4.649783in}{2.569281in}}%
\pgfpathlineto{\pgfqpoint{4.642283in}{2.562133in}}%
\pgfpathlineto{\pgfqpoint{4.634777in}{2.554982in}}%
\pgfpathlineto{\pgfqpoint{4.627266in}{2.547824in}}%
\pgfpathlineto{\pgfqpoint{4.613769in}{2.550526in}}%
\pgfpathlineto{\pgfqpoint{4.600279in}{2.553255in}}%
\pgfpathlineto{\pgfqpoint{4.586795in}{2.556011in}}%
\pgfpathlineto{\pgfqpoint{4.573318in}{2.558794in}}%
\pgfpathlineto{\pgfqpoint{4.580843in}{2.566011in}}%
\pgfpathlineto{\pgfqpoint{4.588362in}{2.573224in}}%
\pgfpathlineto{\pgfqpoint{4.595876in}{2.580437in}}%
\pgfpathlineto{\pgfqpoint{4.603384in}{2.587652in}}%
\pgfpathclose%
\pgfusepath{fill}%
\end{pgfscope}%
\begin{pgfscope}%
\pgfpathrectangle{\pgfqpoint{1.150000in}{0.150000in}}{\pgfqpoint{5.700000in}{5.700000in}}%
\pgfusepath{clip}%
\pgfsetbuttcap%
\pgfsetroundjoin%
\definecolor{currentfill}{rgb}{0.273809,0.031497,0.358853}%
\pgfsetfillcolor{currentfill}%
\pgfsetfillopacity{0.700000}%
\pgfsetlinewidth{0.000000pt}%
\definecolor{currentstroke}{rgb}{0.000000,0.000000,0.000000}%
\pgfsetstrokecolor{currentstroke}%
\pgfsetdash{}{0pt}%
\pgfpathmoveto{\pgfqpoint{4.381801in}{2.564803in}}%
\pgfpathlineto{\pgfqpoint{4.395212in}{2.561737in}}%
\pgfpathlineto{\pgfqpoint{4.408630in}{2.558700in}}%
\pgfpathlineto{\pgfqpoint{4.422054in}{2.555691in}}%
\pgfpathlineto{\pgfqpoint{4.435484in}{2.552709in}}%
\pgfpathlineto{\pgfqpoint{4.427905in}{2.545364in}}%
\pgfpathlineto{\pgfqpoint{4.420321in}{2.538010in}}%
\pgfpathlineto{\pgfqpoint{4.412731in}{2.530648in}}%
\pgfpathlineto{\pgfqpoint{4.405136in}{2.523274in}}%
\pgfpathlineto{\pgfqpoint{4.391693in}{2.526218in}}%
\pgfpathlineto{\pgfqpoint{4.378256in}{2.529190in}}%
\pgfpathlineto{\pgfqpoint{4.364826in}{2.532191in}}%
\pgfpathlineto{\pgfqpoint{4.351402in}{2.535219in}}%
\pgfpathlineto{\pgfqpoint{4.359010in}{2.542625in}}%
\pgfpathlineto{\pgfqpoint{4.366613in}{2.550024in}}%
\pgfpathlineto{\pgfqpoint{4.374210in}{2.557416in}}%
\pgfpathlineto{\pgfqpoint{4.381801in}{2.564803in}}%
\pgfpathclose%
\pgfusepath{fill}%
\end{pgfscope}%
\begin{pgfscope}%
\pgfpathrectangle{\pgfqpoint{1.150000in}{0.150000in}}{\pgfqpoint{5.700000in}{5.700000in}}%
\pgfusepath{clip}%
\pgfsetbuttcap%
\pgfsetroundjoin%
\definecolor{currentfill}{rgb}{0.268510,0.009605,0.335427}%
\pgfsetfillcolor{currentfill}%
\pgfsetfillopacity{0.700000}%
\pgfsetlinewidth{0.000000pt}%
\definecolor{currentstroke}{rgb}{0.000000,0.000000,0.000000}%
\pgfsetstrokecolor{currentstroke}%
\pgfsetdash{}{0pt}%
\pgfpathmoveto{\pgfqpoint{3.801015in}{2.526861in}}%
\pgfpathlineto{\pgfqpoint{3.814295in}{2.522875in}}%
\pgfpathlineto{\pgfqpoint{3.827581in}{2.518921in}}%
\pgfpathlineto{\pgfqpoint{3.840873in}{2.515000in}}%
\pgfpathlineto{\pgfqpoint{3.854171in}{2.511112in}}%
\pgfpathlineto{\pgfqpoint{3.846380in}{2.503598in}}%
\pgfpathlineto{\pgfqpoint{3.838583in}{2.496087in}}%
\pgfpathlineto{\pgfqpoint{3.830781in}{2.488580in}}%
\pgfpathlineto{\pgfqpoint{3.822973in}{2.481077in}}%
\pgfpathlineto{\pgfqpoint{3.809663in}{2.484994in}}%
\pgfpathlineto{\pgfqpoint{3.796358in}{2.488944in}}%
\pgfpathlineto{\pgfqpoint{3.783060in}{2.492927in}}%
\pgfpathlineto{\pgfqpoint{3.769766in}{2.496942in}}%
\pgfpathlineto{\pgfqpoint{3.777587in}{2.504411in}}%
\pgfpathlineto{\pgfqpoint{3.785402in}{2.511888in}}%
\pgfpathlineto{\pgfqpoint{3.793211in}{2.519371in}}%
\pgfpathlineto{\pgfqpoint{3.801015in}{2.526861in}}%
\pgfpathclose%
\pgfusepath{fill}%
\end{pgfscope}%
\begin{pgfscope}%
\pgfpathrectangle{\pgfqpoint{1.150000in}{0.150000in}}{\pgfqpoint{5.700000in}{5.700000in}}%
\pgfusepath{clip}%
\pgfsetbuttcap%
\pgfsetroundjoin%
\definecolor{currentfill}{rgb}{0.282327,0.094955,0.417331}%
\pgfsetfillcolor{currentfill}%
\pgfsetfillopacity{0.700000}%
\pgfsetlinewidth{0.000000pt}%
\definecolor{currentstroke}{rgb}{0.000000,0.000000,0.000000}%
\pgfsetstrokecolor{currentstroke}%
\pgfsetdash{}{0pt}%
\pgfpathmoveto{\pgfqpoint{5.405947in}{2.666394in}}%
\pgfpathlineto{\pgfqpoint{5.419611in}{2.663863in}}%
\pgfpathlineto{\pgfqpoint{5.433283in}{2.661356in}}%
\pgfpathlineto{\pgfqpoint{5.446961in}{2.658874in}}%
\pgfpathlineto{\pgfqpoint{5.460647in}{2.656415in}}%
\pgfpathlineto{\pgfqpoint{5.453460in}{2.649757in}}%
\pgfpathlineto{\pgfqpoint{5.446270in}{2.643164in}}%
\pgfpathlineto{\pgfqpoint{5.439075in}{2.636631in}}%
\pgfpathlineto{\pgfqpoint{5.431877in}{2.630154in}}%
\pgfpathlineto{\pgfqpoint{5.418173in}{2.632456in}}%
\pgfpathlineto{\pgfqpoint{5.404477in}{2.634783in}}%
\pgfpathlineto{\pgfqpoint{5.390788in}{2.637135in}}%
\pgfpathlineto{\pgfqpoint{5.377106in}{2.639510in}}%
\pgfpathlineto{\pgfqpoint{5.384322in}{2.646139in}}%
\pgfpathlineto{\pgfqpoint{5.391534in}{2.652825in}}%
\pgfpathlineto{\pgfqpoint{5.398743in}{2.659576in}}%
\pgfpathlineto{\pgfqpoint{5.405947in}{2.666394in}}%
\pgfpathclose%
\pgfusepath{fill}%
\end{pgfscope}%
\begin{pgfscope}%
\pgfpathrectangle{\pgfqpoint{1.150000in}{0.150000in}}{\pgfqpoint{5.700000in}{5.700000in}}%
\pgfusepath{clip}%
\pgfsetbuttcap%
\pgfsetroundjoin%
\definecolor{currentfill}{rgb}{0.271305,0.019942,0.347269}%
\pgfsetfillcolor{currentfill}%
\pgfsetfillopacity{0.700000}%
\pgfsetlinewidth{0.000000pt}%
\definecolor{currentstroke}{rgb}{0.000000,0.000000,0.000000}%
\pgfsetstrokecolor{currentstroke}%
\pgfsetdash{}{0pt}%
\pgfpathmoveto{\pgfqpoint{4.160174in}{2.543796in}}%
\pgfpathlineto{\pgfqpoint{4.173534in}{2.540443in}}%
\pgfpathlineto{\pgfqpoint{4.186901in}{2.537120in}}%
\pgfpathlineto{\pgfqpoint{4.200274in}{2.533826in}}%
\pgfpathlineto{\pgfqpoint{4.213653in}{2.530561in}}%
\pgfpathlineto{\pgfqpoint{4.205992in}{2.523066in}}%
\pgfpathlineto{\pgfqpoint{4.198325in}{2.515562in}}%
\pgfpathlineto{\pgfqpoint{4.190653in}{2.508049in}}%
\pgfpathlineto{\pgfqpoint{4.182975in}{2.500526in}}%
\pgfpathlineto{\pgfqpoint{4.169583in}{2.503780in}}%
\pgfpathlineto{\pgfqpoint{4.156198in}{2.507063in}}%
\pgfpathlineto{\pgfqpoint{4.142819in}{2.510376in}}%
\pgfpathlineto{\pgfqpoint{4.129446in}{2.513719in}}%
\pgfpathlineto{\pgfqpoint{4.137136in}{2.521247in}}%
\pgfpathlineto{\pgfqpoint{4.144821in}{2.528769in}}%
\pgfpathlineto{\pgfqpoint{4.152500in}{2.536285in}}%
\pgfpathlineto{\pgfqpoint{4.160174in}{2.543796in}}%
\pgfpathclose%
\pgfusepath{fill}%
\end{pgfscope}%
\begin{pgfscope}%
\pgfpathrectangle{\pgfqpoint{1.150000in}{0.150000in}}{\pgfqpoint{5.700000in}{5.700000in}}%
\pgfusepath{clip}%
\pgfsetbuttcap%
\pgfsetroundjoin%
\definecolor{currentfill}{rgb}{0.271305,0.019942,0.347269}%
\pgfsetfillcolor{currentfill}%
\pgfsetfillopacity{0.700000}%
\pgfsetlinewidth{0.000000pt}%
\definecolor{currentstroke}{rgb}{0.000000,0.000000,0.000000}%
\pgfsetstrokecolor{currentstroke}%
\pgfsetdash{}{0pt}%
\pgfpathmoveto{\pgfqpoint{3.304243in}{2.539445in}}%
\pgfpathlineto{\pgfqpoint{3.317436in}{2.534249in}}%
\pgfpathlineto{\pgfqpoint{3.330633in}{2.529092in}}%
\pgfpathlineto{\pgfqpoint{3.343835in}{2.523975in}}%
\pgfpathlineto{\pgfqpoint{3.357041in}{2.518898in}}%
\pgfpathlineto{\pgfqpoint{3.349060in}{2.512051in}}%
\pgfpathlineto{\pgfqpoint{3.341072in}{2.505245in}}%
\pgfpathlineto{\pgfqpoint{3.333078in}{2.498481in}}%
\pgfpathlineto{\pgfqpoint{3.325076in}{2.491761in}}%
\pgfpathlineto{\pgfqpoint{3.311855in}{2.496921in}}%
\pgfpathlineto{\pgfqpoint{3.298639in}{2.502120in}}%
\pgfpathlineto{\pgfqpoint{3.285427in}{2.507358in}}%
\pgfpathlineto{\pgfqpoint{3.272219in}{2.512636in}}%
\pgfpathlineto{\pgfqpoint{3.280236in}{2.519269in}}%
\pgfpathlineto{\pgfqpoint{3.288245in}{2.525950in}}%
\pgfpathlineto{\pgfqpoint{3.296248in}{2.532676in}}%
\pgfpathlineto{\pgfqpoint{3.304243in}{2.539445in}}%
\pgfpathclose%
\pgfusepath{fill}%
\end{pgfscope}%
\begin{pgfscope}%
\pgfpathrectangle{\pgfqpoint{1.150000in}{0.150000in}}{\pgfqpoint{5.700000in}{5.700000in}}%
\pgfusepath{clip}%
\pgfsetbuttcap%
\pgfsetroundjoin%
\definecolor{currentfill}{rgb}{0.272594,0.025563,0.353093}%
\pgfsetfillcolor{currentfill}%
\pgfsetfillopacity{0.700000}%
\pgfsetlinewidth{0.000000pt}%
\definecolor{currentstroke}{rgb}{0.000000,0.000000,0.000000}%
\pgfsetstrokecolor{currentstroke}%
\pgfsetdash{}{0pt}%
\pgfpathmoveto{\pgfqpoint{3.166715in}{2.556342in}}%
\pgfpathlineto{\pgfqpoint{3.179888in}{2.550732in}}%
\pgfpathlineto{\pgfqpoint{3.193066in}{2.545164in}}%
\pgfpathlineto{\pgfqpoint{3.206247in}{2.539639in}}%
\pgfpathlineto{\pgfqpoint{3.219433in}{2.534156in}}%
\pgfpathlineto{\pgfqpoint{3.211395in}{2.527662in}}%
\pgfpathlineto{\pgfqpoint{3.203349in}{2.521223in}}%
\pgfpathlineto{\pgfqpoint{3.195296in}{2.514840in}}%
\pgfpathlineto{\pgfqpoint{3.187236in}{2.508515in}}%
\pgfpathlineto{\pgfqpoint{3.174035in}{2.514093in}}%
\pgfpathlineto{\pgfqpoint{3.160837in}{2.519713in}}%
\pgfpathlineto{\pgfqpoint{3.147644in}{2.525376in}}%
\pgfpathlineto{\pgfqpoint{3.134455in}{2.531082in}}%
\pgfpathlineto{\pgfqpoint{3.142531in}{2.537306in}}%
\pgfpathlineto{\pgfqpoint{3.150600in}{2.543593in}}%
\pgfpathlineto{\pgfqpoint{3.158661in}{2.549939in}}%
\pgfpathlineto{\pgfqpoint{3.166715in}{2.556342in}}%
\pgfpathclose%
\pgfusepath{fill}%
\end{pgfscope}%
\begin{pgfscope}%
\pgfpathrectangle{\pgfqpoint{1.150000in}{0.150000in}}{\pgfqpoint{5.700000in}{5.700000in}}%
\pgfusepath{clip}%
\pgfsetbuttcap%
\pgfsetroundjoin%
\definecolor{currentfill}{rgb}{0.281446,0.084320,0.407414}%
\pgfsetfillcolor{currentfill}%
\pgfsetfillopacity{0.700000}%
\pgfsetlinewidth{0.000000pt}%
\definecolor{currentstroke}{rgb}{0.000000,0.000000,0.000000}%
\pgfsetstrokecolor{currentstroke}%
\pgfsetdash{}{0pt}%
\pgfpathmoveto{\pgfqpoint{5.184384in}{2.642213in}}%
\pgfpathlineto{\pgfqpoint{5.197996in}{2.639683in}}%
\pgfpathlineto{\pgfqpoint{5.211614in}{2.637178in}}%
\pgfpathlineto{\pgfqpoint{5.225240in}{2.634698in}}%
\pgfpathlineto{\pgfqpoint{5.238873in}{2.632243in}}%
\pgfpathlineto{\pgfqpoint{5.231601in}{2.625570in}}%
\pgfpathlineto{\pgfqpoint{5.224325in}{2.618936in}}%
\pgfpathlineto{\pgfqpoint{5.217044in}{2.612335in}}%
\pgfpathlineto{\pgfqpoint{5.209759in}{2.605765in}}%
\pgfpathlineto{\pgfqpoint{5.196109in}{2.608091in}}%
\pgfpathlineto{\pgfqpoint{5.182467in}{2.610441in}}%
\pgfpathlineto{\pgfqpoint{5.168832in}{2.612817in}}%
\pgfpathlineto{\pgfqpoint{5.155204in}{2.615218in}}%
\pgfpathlineto{\pgfqpoint{5.162506in}{2.621913in}}%
\pgfpathlineto{\pgfqpoint{5.169804in}{2.628641in}}%
\pgfpathlineto{\pgfqpoint{5.177096in}{2.635406in}}%
\pgfpathlineto{\pgfqpoint{5.184384in}{2.642213in}}%
\pgfpathclose%
\pgfusepath{fill}%
\end{pgfscope}%
\begin{pgfscope}%
\pgfpathrectangle{\pgfqpoint{1.150000in}{0.150000in}}{\pgfqpoint{5.700000in}{5.700000in}}%
\pgfusepath{clip}%
\pgfsetbuttcap%
\pgfsetroundjoin%
\definecolor{currentfill}{rgb}{0.269944,0.014625,0.341379}%
\pgfsetfillcolor{currentfill}%
\pgfsetfillopacity{0.700000}%
\pgfsetlinewidth{0.000000pt}%
\definecolor{currentstroke}{rgb}{0.000000,0.000000,0.000000}%
\pgfsetstrokecolor{currentstroke}%
\pgfsetdash{}{0pt}%
\pgfpathmoveto{\pgfqpoint{3.441716in}{2.527011in}}%
\pgfpathlineto{\pgfqpoint{3.454932in}{2.522194in}}%
\pgfpathlineto{\pgfqpoint{3.468152in}{2.517415in}}%
\pgfpathlineto{\pgfqpoint{3.481378in}{2.512673in}}%
\pgfpathlineto{\pgfqpoint{3.494608in}{2.507968in}}%
\pgfpathlineto{\pgfqpoint{3.486681in}{2.500846in}}%
\pgfpathlineto{\pgfqpoint{3.478747in}{2.493753in}}%
\pgfpathlineto{\pgfqpoint{3.470807in}{2.486689in}}%
\pgfpathlineto{\pgfqpoint{3.462861in}{2.479657in}}%
\pgfpathlineto{\pgfqpoint{3.449617in}{2.484430in}}%
\pgfpathlineto{\pgfqpoint{3.436377in}{2.489241in}}%
\pgfpathlineto{\pgfqpoint{3.423143in}{2.494089in}}%
\pgfpathlineto{\pgfqpoint{3.409913in}{2.498974in}}%
\pgfpathlineto{\pgfqpoint{3.417873in}{2.505934in}}%
\pgfpathlineto{\pgfqpoint{3.425827in}{2.512927in}}%
\pgfpathlineto{\pgfqpoint{3.433775in}{2.519953in}}%
\pgfpathlineto{\pgfqpoint{3.441716in}{2.527011in}}%
\pgfpathclose%
\pgfusepath{fill}%
\end{pgfscope}%
\begin{pgfscope}%
\pgfpathrectangle{\pgfqpoint{1.150000in}{0.150000in}}{\pgfqpoint{5.700000in}{5.700000in}}%
\pgfusepath{clip}%
\pgfsetbuttcap%
\pgfsetroundjoin%
\definecolor{currentfill}{rgb}{0.276022,0.044167,0.370164}%
\pgfsetfillcolor{currentfill}%
\pgfsetfillopacity{0.700000}%
\pgfsetlinewidth{0.000000pt}%
\definecolor{currentstroke}{rgb}{0.000000,0.000000,0.000000}%
\pgfsetstrokecolor{currentstroke}%
\pgfsetdash{}{0pt}%
\pgfpathmoveto{\pgfqpoint{3.029084in}{2.578311in}}%
\pgfpathlineto{\pgfqpoint{3.042242in}{2.572250in}}%
\pgfpathlineto{\pgfqpoint{3.055404in}{2.566234in}}%
\pgfpathlineto{\pgfqpoint{3.068570in}{2.560264in}}%
\pgfpathlineto{\pgfqpoint{3.081739in}{2.554339in}}%
\pgfpathlineto{\pgfqpoint{3.073639in}{2.548282in}}%
\pgfpathlineto{\pgfqpoint{3.065531in}{2.542294in}}%
\pgfpathlineto{\pgfqpoint{3.057416in}{2.536378in}}%
\pgfpathlineto{\pgfqpoint{3.049292in}{2.530537in}}%
\pgfpathlineto{\pgfqpoint{3.036106in}{2.536571in}}%
\pgfpathlineto{\pgfqpoint{3.022924in}{2.542650in}}%
\pgfpathlineto{\pgfqpoint{3.009745in}{2.548774in}}%
\pgfpathlineto{\pgfqpoint{2.996570in}{2.554944in}}%
\pgfpathlineto{\pgfqpoint{3.004711in}{2.560672in}}%
\pgfpathlineto{\pgfqpoint{3.012843in}{2.566477in}}%
\pgfpathlineto{\pgfqpoint{3.020968in}{2.572358in}}%
\pgfpathlineto{\pgfqpoint{3.029084in}{2.578311in}}%
\pgfpathclose%
\pgfusepath{fill}%
\end{pgfscope}%
\begin{pgfscope}%
\pgfpathrectangle{\pgfqpoint{1.150000in}{0.150000in}}{\pgfqpoint{5.700000in}{5.700000in}}%
\pgfusepath{clip}%
\pgfsetbuttcap%
\pgfsetroundjoin%
\definecolor{currentfill}{rgb}{0.279566,0.067836,0.391917}%
\pgfsetfillcolor{currentfill}%
\pgfsetfillopacity{0.700000}%
\pgfsetlinewidth{0.000000pt}%
\definecolor{currentstroke}{rgb}{0.000000,0.000000,0.000000}%
\pgfsetstrokecolor{currentstroke}%
\pgfsetdash{}{0pt}%
\pgfpathmoveto{\pgfqpoint{4.962785in}{2.618073in}}%
\pgfpathlineto{\pgfqpoint{4.976343in}{2.615485in}}%
\pgfpathlineto{\pgfqpoint{4.989907in}{2.612922in}}%
\pgfpathlineto{\pgfqpoint{5.003479in}{2.610385in}}%
\pgfpathlineto{\pgfqpoint{5.017058in}{2.607873in}}%
\pgfpathlineto{\pgfqpoint{5.009700in}{2.601070in}}%
\pgfpathlineto{\pgfqpoint{5.002336in}{2.594284in}}%
\pgfpathlineto{\pgfqpoint{4.994968in}{2.587512in}}%
\pgfpathlineto{\pgfqpoint{4.987594in}{2.580750in}}%
\pgfpathlineto{\pgfqpoint{4.974000in}{2.583159in}}%
\pgfpathlineto{\pgfqpoint{4.960413in}{2.585593in}}%
\pgfpathlineto{\pgfqpoint{4.946833in}{2.588053in}}%
\pgfpathlineto{\pgfqpoint{4.933260in}{2.590538in}}%
\pgfpathlineto{\pgfqpoint{4.940649in}{2.597398in}}%
\pgfpathlineto{\pgfqpoint{4.948033in}{2.604272in}}%
\pgfpathlineto{\pgfqpoint{4.955412in}{2.611162in}}%
\pgfpathlineto{\pgfqpoint{4.962785in}{2.618073in}}%
\pgfpathclose%
\pgfusepath{fill}%
\end{pgfscope}%
\begin{pgfscope}%
\pgfpathrectangle{\pgfqpoint{1.150000in}{0.150000in}}{\pgfqpoint{5.700000in}{5.700000in}}%
\pgfusepath{clip}%
\pgfsetbuttcap%
\pgfsetroundjoin%
\definecolor{currentfill}{rgb}{0.269944,0.014625,0.341379}%
\pgfsetfillcolor{currentfill}%
\pgfsetfillopacity{0.700000}%
\pgfsetlinewidth{0.000000pt}%
\definecolor{currentstroke}{rgb}{0.000000,0.000000,0.000000}%
\pgfsetstrokecolor{currentstroke}%
\pgfsetdash{}{0pt}%
\pgfpathmoveto{\pgfqpoint{3.938474in}{2.526033in}}%
\pgfpathlineto{\pgfqpoint{3.951787in}{2.522320in}}%
\pgfpathlineto{\pgfqpoint{3.965107in}{2.518639in}}%
\pgfpathlineto{\pgfqpoint{3.978432in}{2.514989in}}%
\pgfpathlineto{\pgfqpoint{3.991763in}{2.511370in}}%
\pgfpathlineto{\pgfqpoint{3.984020in}{2.503821in}}%
\pgfpathlineto{\pgfqpoint{3.976271in}{2.496269in}}%
\pgfpathlineto{\pgfqpoint{3.968517in}{2.488713in}}%
\pgfpathlineto{\pgfqpoint{3.960757in}{2.481154in}}%
\pgfpathlineto{\pgfqpoint{3.947413in}{2.484789in}}%
\pgfpathlineto{\pgfqpoint{3.934075in}{2.488454in}}%
\pgfpathlineto{\pgfqpoint{3.920743in}{2.492151in}}%
\pgfpathlineto{\pgfqpoint{3.907417in}{2.495880in}}%
\pgfpathlineto{\pgfqpoint{3.915190in}{2.503418in}}%
\pgfpathlineto{\pgfqpoint{3.922957in}{2.510956in}}%
\pgfpathlineto{\pgfqpoint{3.930718in}{2.518494in}}%
\pgfpathlineto{\pgfqpoint{3.938474in}{2.526033in}}%
\pgfpathclose%
\pgfusepath{fill}%
\end{pgfscope}%
\begin{pgfscope}%
\pgfpathrectangle{\pgfqpoint{1.150000in}{0.150000in}}{\pgfqpoint{5.700000in}{5.700000in}}%
\pgfusepath{clip}%
\pgfsetbuttcap%
\pgfsetroundjoin%
\definecolor{currentfill}{rgb}{0.277941,0.056324,0.381191}%
\pgfsetfillcolor{currentfill}%
\pgfsetfillopacity{0.700000}%
\pgfsetlinewidth{0.000000pt}%
\definecolor{currentstroke}{rgb}{0.000000,0.000000,0.000000}%
\pgfsetstrokecolor{currentstroke}%
\pgfsetdash{}{0pt}%
\pgfpathmoveto{\pgfqpoint{4.741149in}{2.593945in}}%
\pgfpathlineto{\pgfqpoint{4.754652in}{2.591236in}}%
\pgfpathlineto{\pgfqpoint{4.768163in}{2.588554in}}%
\pgfpathlineto{\pgfqpoint{4.781680in}{2.585898in}}%
\pgfpathlineto{\pgfqpoint{4.795205in}{2.583269in}}%
\pgfpathlineto{\pgfqpoint{4.787760in}{2.576267in}}%
\pgfpathlineto{\pgfqpoint{4.780309in}{2.569267in}}%
\pgfpathlineto{\pgfqpoint{4.772853in}{2.562267in}}%
\pgfpathlineto{\pgfqpoint{4.765392in}{2.555263in}}%
\pgfpathlineto{\pgfqpoint{4.751854in}{2.557816in}}%
\pgfpathlineto{\pgfqpoint{4.738322in}{2.560395in}}%
\pgfpathlineto{\pgfqpoint{4.724798in}{2.563001in}}%
\pgfpathlineto{\pgfqpoint{4.711280in}{2.565633in}}%
\pgfpathlineto{\pgfqpoint{4.718755in}{2.572708in}}%
\pgfpathlineto{\pgfqpoint{4.726225in}{2.579783in}}%
\pgfpathlineto{\pgfqpoint{4.733690in}{2.586862in}}%
\pgfpathlineto{\pgfqpoint{4.741149in}{2.593945in}}%
\pgfpathclose%
\pgfusepath{fill}%
\end{pgfscope}%
\begin{pgfscope}%
\pgfpathrectangle{\pgfqpoint{1.150000in}{0.150000in}}{\pgfqpoint{5.700000in}{5.700000in}}%
\pgfusepath{clip}%
\pgfsetbuttcap%
\pgfsetroundjoin%
\definecolor{currentfill}{rgb}{0.268510,0.009605,0.335427}%
\pgfsetfillcolor{currentfill}%
\pgfsetfillopacity{0.700000}%
\pgfsetlinewidth{0.000000pt}%
\definecolor{currentstroke}{rgb}{0.000000,0.000000,0.000000}%
\pgfsetstrokecolor{currentstroke}%
\pgfsetdash{}{0pt}%
\pgfpathmoveto{\pgfqpoint{3.579173in}{2.518480in}}%
\pgfpathlineto{\pgfqpoint{3.592415in}{2.514011in}}%
\pgfpathlineto{\pgfqpoint{3.605662in}{2.509578in}}%
\pgfpathlineto{\pgfqpoint{3.618915in}{2.505180in}}%
\pgfpathlineto{\pgfqpoint{3.632172in}{2.500817in}}%
\pgfpathlineto{\pgfqpoint{3.624296in}{2.493493in}}%
\pgfpathlineto{\pgfqpoint{3.616414in}{2.486186in}}%
\pgfpathlineto{\pgfqpoint{3.608526in}{2.478897in}}%
\pgfpathlineto{\pgfqpoint{3.600632in}{2.471628in}}%
\pgfpathlineto{\pgfqpoint{3.587361in}{2.476046in}}%
\pgfpathlineto{\pgfqpoint{3.574096in}{2.480499in}}%
\pgfpathlineto{\pgfqpoint{3.560835in}{2.484988in}}%
\pgfpathlineto{\pgfqpoint{3.547580in}{2.489512in}}%
\pgfpathlineto{\pgfqpoint{3.555487in}{2.496721in}}%
\pgfpathlineto{\pgfqpoint{3.563388in}{2.503952in}}%
\pgfpathlineto{\pgfqpoint{3.571284in}{2.511206in}}%
\pgfpathlineto{\pgfqpoint{3.579173in}{2.518480in}}%
\pgfpathclose%
\pgfusepath{fill}%
\end{pgfscope}%
\begin{pgfscope}%
\pgfpathrectangle{\pgfqpoint{1.150000in}{0.150000in}}{\pgfqpoint{5.700000in}{5.700000in}}%
\pgfusepath{clip}%
\pgfsetbuttcap%
\pgfsetroundjoin%
\definecolor{currentfill}{rgb}{0.276022,0.044167,0.370164}%
\pgfsetfillcolor{currentfill}%
\pgfsetfillopacity{0.700000}%
\pgfsetlinewidth{0.000000pt}%
\definecolor{currentstroke}{rgb}{0.000000,0.000000,0.000000}%
\pgfsetstrokecolor{currentstroke}%
\pgfsetdash{}{0pt}%
\pgfpathmoveto{\pgfqpoint{4.519478in}{2.570201in}}%
\pgfpathlineto{\pgfqpoint{4.532928in}{2.567308in}}%
\pgfpathlineto{\pgfqpoint{4.546385in}{2.564443in}}%
\pgfpathlineto{\pgfqpoint{4.559848in}{2.561605in}}%
\pgfpathlineto{\pgfqpoint{4.573318in}{2.558794in}}%
\pgfpathlineto{\pgfqpoint{4.565788in}{2.551573in}}%
\pgfpathlineto{\pgfqpoint{4.558253in}{2.544344in}}%
\pgfpathlineto{\pgfqpoint{4.550711in}{2.537107in}}%
\pgfpathlineto{\pgfqpoint{4.543164in}{2.529858in}}%
\pgfpathlineto{\pgfqpoint{4.529681in}{2.532618in}}%
\pgfpathlineto{\pgfqpoint{4.516204in}{2.535406in}}%
\pgfpathlineto{\pgfqpoint{4.502734in}{2.538220in}}%
\pgfpathlineto{\pgfqpoint{4.489271in}{2.541063in}}%
\pgfpathlineto{\pgfqpoint{4.496831in}{2.548357in}}%
\pgfpathlineto{\pgfqpoint{4.504386in}{2.555643in}}%
\pgfpathlineto{\pgfqpoint{4.511935in}{2.562924in}}%
\pgfpathlineto{\pgfqpoint{4.519478in}{2.570201in}}%
\pgfpathclose%
\pgfusepath{fill}%
\end{pgfscope}%
\begin{pgfscope}%
\pgfpathrectangle{\pgfqpoint{1.150000in}{0.150000in}}{\pgfqpoint{5.700000in}{5.700000in}}%
\pgfusepath{clip}%
\pgfsetbuttcap%
\pgfsetroundjoin%
\definecolor{currentfill}{rgb}{0.278791,0.062145,0.386592}%
\pgfsetfillcolor{currentfill}%
\pgfsetfillopacity{0.700000}%
\pgfsetlinewidth{0.000000pt}%
\definecolor{currentstroke}{rgb}{0.000000,0.000000,0.000000}%
\pgfsetstrokecolor{currentstroke}%
\pgfsetdash{}{0pt}%
\pgfpathmoveto{\pgfqpoint{2.891297in}{2.606017in}}%
\pgfpathlineto{\pgfqpoint{2.904444in}{2.599462in}}%
\pgfpathlineto{\pgfqpoint{2.917595in}{2.592957in}}%
\pgfpathlineto{\pgfqpoint{2.930749in}{2.586501in}}%
\pgfpathlineto{\pgfqpoint{2.943906in}{2.580094in}}%
\pgfpathlineto{\pgfqpoint{2.935740in}{2.574564in}}%
\pgfpathlineto{\pgfqpoint{2.927565in}{2.569120in}}%
\pgfpathlineto{\pgfqpoint{2.919381in}{2.563765in}}%
\pgfpathlineto{\pgfqpoint{2.911189in}{2.558502in}}%
\pgfpathlineto{\pgfqpoint{2.898014in}{2.565031in}}%
\pgfpathlineto{\pgfqpoint{2.884841in}{2.571609in}}%
\pgfpathlineto{\pgfqpoint{2.871672in}{2.578236in}}%
\pgfpathlineto{\pgfqpoint{2.858507in}{2.584913in}}%
\pgfpathlineto{\pgfqpoint{2.866717in}{2.590049in}}%
\pgfpathlineto{\pgfqpoint{2.874919in}{2.595280in}}%
\pgfpathlineto{\pgfqpoint{2.883112in}{2.600604in}}%
\pgfpathlineto{\pgfqpoint{2.891297in}{2.606017in}}%
\pgfpathclose%
\pgfusepath{fill}%
\end{pgfscope}%
\begin{pgfscope}%
\pgfpathrectangle{\pgfqpoint{1.150000in}{0.150000in}}{\pgfqpoint{5.700000in}{5.700000in}}%
\pgfusepath{clip}%
\pgfsetbuttcap%
\pgfsetroundjoin%
\definecolor{currentfill}{rgb}{0.272594,0.025563,0.353093}%
\pgfsetfillcolor{currentfill}%
\pgfsetfillopacity{0.700000}%
\pgfsetlinewidth{0.000000pt}%
\definecolor{currentstroke}{rgb}{0.000000,0.000000,0.000000}%
\pgfsetstrokecolor{currentstroke}%
\pgfsetdash{}{0pt}%
\pgfpathmoveto{\pgfqpoint{4.297772in}{2.547619in}}%
\pgfpathlineto{\pgfqpoint{4.311170in}{2.544476in}}%
\pgfpathlineto{\pgfqpoint{4.324574in}{2.541362in}}%
\pgfpathlineto{\pgfqpoint{4.337985in}{2.538276in}}%
\pgfpathlineto{\pgfqpoint{4.351402in}{2.535219in}}%
\pgfpathlineto{\pgfqpoint{4.343789in}{2.527804in}}%
\pgfpathlineto{\pgfqpoint{4.336169in}{2.520378in}}%
\pgfpathlineto{\pgfqpoint{4.328544in}{2.512941in}}%
\pgfpathlineto{\pgfqpoint{4.320914in}{2.505491in}}%
\pgfpathlineto{\pgfqpoint{4.307484in}{2.508524in}}%
\pgfpathlineto{\pgfqpoint{4.294060in}{2.511586in}}%
\pgfpathlineto{\pgfqpoint{4.280643in}{2.514676in}}%
\pgfpathlineto{\pgfqpoint{4.267233in}{2.517795in}}%
\pgfpathlineto{\pgfqpoint{4.274876in}{2.525265in}}%
\pgfpathlineto{\pgfqpoint{4.282513in}{2.532724in}}%
\pgfpathlineto{\pgfqpoint{4.290145in}{2.540175in}}%
\pgfpathlineto{\pgfqpoint{4.297772in}{2.547619in}}%
\pgfpathclose%
\pgfusepath{fill}%
\end{pgfscope}%
\begin{pgfscope}%
\pgfpathrectangle{\pgfqpoint{1.150000in}{0.150000in}}{\pgfqpoint{5.700000in}{5.700000in}}%
\pgfusepath{clip}%
\pgfsetbuttcap%
\pgfsetroundjoin%
\definecolor{currentfill}{rgb}{0.282910,0.105393,0.426902}%
\pgfsetfillcolor{currentfill}%
\pgfsetfillopacity{0.700000}%
\pgfsetlinewidth{0.000000pt}%
\definecolor{currentstroke}{rgb}{0.000000,0.000000,0.000000}%
\pgfsetstrokecolor{currentstroke}%
\pgfsetdash{}{0pt}%
\pgfpathmoveto{\pgfqpoint{5.544106in}{2.673537in}}%
\pgfpathlineto{\pgfqpoint{5.557810in}{2.671032in}}%
\pgfpathlineto{\pgfqpoint{5.571522in}{2.668550in}}%
\pgfpathlineto{\pgfqpoint{5.585241in}{2.666093in}}%
\pgfpathlineto{\pgfqpoint{5.598967in}{2.663659in}}%
\pgfpathlineto{\pgfqpoint{5.591831in}{2.657036in}}%
\pgfpathlineto{\pgfqpoint{5.584691in}{2.650493in}}%
\pgfpathlineto{\pgfqpoint{5.577548in}{2.644023in}}%
\pgfpathlineto{\pgfqpoint{5.570402in}{2.637623in}}%
\pgfpathlineto{\pgfqpoint{5.556656in}{2.639887in}}%
\pgfpathlineto{\pgfqpoint{5.542918in}{2.642176in}}%
\pgfpathlineto{\pgfqpoint{5.529188in}{2.644488in}}%
\pgfpathlineto{\pgfqpoint{5.515465in}{2.646825in}}%
\pgfpathlineto{\pgfqpoint{5.522630in}{2.653390in}}%
\pgfpathlineto{\pgfqpoint{5.529791in}{2.660027in}}%
\pgfpathlineto{\pgfqpoint{5.536950in}{2.666741in}}%
\pgfpathlineto{\pgfqpoint{5.544106in}{2.673537in}}%
\pgfpathclose%
\pgfusepath{fill}%
\end{pgfscope}%
\begin{pgfscope}%
\pgfpathrectangle{\pgfqpoint{1.150000in}{0.150000in}}{\pgfqpoint{5.700000in}{5.700000in}}%
\pgfusepath{clip}%
\pgfsetbuttcap%
\pgfsetroundjoin%
\definecolor{currentfill}{rgb}{0.268510,0.009605,0.335427}%
\pgfsetfillcolor{currentfill}%
\pgfsetfillopacity{0.700000}%
\pgfsetlinewidth{0.000000pt}%
\definecolor{currentstroke}{rgb}{0.000000,0.000000,0.000000}%
\pgfsetstrokecolor{currentstroke}%
\pgfsetdash{}{0pt}%
\pgfpathmoveto{\pgfqpoint{3.716648in}{2.513337in}}%
\pgfpathlineto{\pgfqpoint{3.729920in}{2.509188in}}%
\pgfpathlineto{\pgfqpoint{3.743196in}{2.505073in}}%
\pgfpathlineto{\pgfqpoint{3.756479in}{2.500991in}}%
\pgfpathlineto{\pgfqpoint{3.769766in}{2.496942in}}%
\pgfpathlineto{\pgfqpoint{3.761940in}{2.489481in}}%
\pgfpathlineto{\pgfqpoint{3.754108in}{2.482028in}}%
\pgfpathlineto{\pgfqpoint{3.746269in}{2.474583in}}%
\pgfpathlineto{\pgfqpoint{3.738425in}{2.467148in}}%
\pgfpathlineto{\pgfqpoint{3.725125in}{2.471238in}}%
\pgfpathlineto{\pgfqpoint{3.711830in}{2.475362in}}%
\pgfpathlineto{\pgfqpoint{3.698540in}{2.479520in}}%
\pgfpathlineto{\pgfqpoint{3.685256in}{2.483711in}}%
\pgfpathlineto{\pgfqpoint{3.693113in}{2.491099in}}%
\pgfpathlineto{\pgfqpoint{3.700964in}{2.498500in}}%
\pgfpathlineto{\pgfqpoint{3.708809in}{2.505913in}}%
\pgfpathlineto{\pgfqpoint{3.716648in}{2.513337in}}%
\pgfpathclose%
\pgfusepath{fill}%
\end{pgfscope}%
\begin{pgfscope}%
\pgfpathrectangle{\pgfqpoint{1.150000in}{0.150000in}}{\pgfqpoint{5.700000in}{5.700000in}}%
\pgfusepath{clip}%
\pgfsetbuttcap%
\pgfsetroundjoin%
\definecolor{currentfill}{rgb}{0.281924,0.089666,0.412415}%
\pgfsetfillcolor{currentfill}%
\pgfsetfillopacity{0.700000}%
\pgfsetlinewidth{0.000000pt}%
\definecolor{currentstroke}{rgb}{0.000000,0.000000,0.000000}%
\pgfsetstrokecolor{currentstroke}%
\pgfsetdash{}{0pt}%
\pgfpathmoveto{\pgfqpoint{5.322453in}{2.649259in}}%
\pgfpathlineto{\pgfqpoint{5.336105in}{2.646785in}}%
\pgfpathlineto{\pgfqpoint{5.349765in}{2.644335in}}%
\pgfpathlineto{\pgfqpoint{5.363432in}{2.641911in}}%
\pgfpathlineto{\pgfqpoint{5.377106in}{2.639510in}}%
\pgfpathlineto{\pgfqpoint{5.369886in}{2.632936in}}%
\pgfpathlineto{\pgfqpoint{5.362662in}{2.626411in}}%
\pgfpathlineto{\pgfqpoint{5.355433in}{2.619932in}}%
\pgfpathlineto{\pgfqpoint{5.348200in}{2.613492in}}%
\pgfpathlineto{\pgfqpoint{5.334508in}{2.615750in}}%
\pgfpathlineto{\pgfqpoint{5.320824in}{2.618032in}}%
\pgfpathlineto{\pgfqpoint{5.307147in}{2.620338in}}%
\pgfpathlineto{\pgfqpoint{5.293478in}{2.622670in}}%
\pgfpathlineto{\pgfqpoint{5.300728in}{2.629247in}}%
\pgfpathlineto{\pgfqpoint{5.307974in}{2.635868in}}%
\pgfpathlineto{\pgfqpoint{5.315215in}{2.642537in}}%
\pgfpathlineto{\pgfqpoint{5.322453in}{2.649259in}}%
\pgfpathclose%
\pgfusepath{fill}%
\end{pgfscope}%
\begin{pgfscope}%
\pgfpathrectangle{\pgfqpoint{1.150000in}{0.150000in}}{\pgfqpoint{5.700000in}{5.700000in}}%
\pgfusepath{clip}%
\pgfsetbuttcap%
\pgfsetroundjoin%
\definecolor{currentfill}{rgb}{0.271305,0.019942,0.347269}%
\pgfsetfillcolor{currentfill}%
\pgfsetfillopacity{0.700000}%
\pgfsetlinewidth{0.000000pt}%
\definecolor{currentstroke}{rgb}{0.000000,0.000000,0.000000}%
\pgfsetstrokecolor{currentstroke}%
\pgfsetdash{}{0pt}%
\pgfpathmoveto{\pgfqpoint{4.076015in}{2.527388in}}%
\pgfpathlineto{\pgfqpoint{4.089363in}{2.523926in}}%
\pgfpathlineto{\pgfqpoint{4.102718in}{2.520493in}}%
\pgfpathlineto{\pgfqpoint{4.116079in}{2.517091in}}%
\pgfpathlineto{\pgfqpoint{4.129446in}{2.513719in}}%
\pgfpathlineto{\pgfqpoint{4.121750in}{2.506182in}}%
\pgfpathlineto{\pgfqpoint{4.114048in}{2.498638in}}%
\pgfpathlineto{\pgfqpoint{4.106341in}{2.491086in}}%
\pgfpathlineto{\pgfqpoint{4.098628in}{2.483524in}}%
\pgfpathlineto{\pgfqpoint{4.085249in}{2.486899in}}%
\pgfpathlineto{\pgfqpoint{4.071876in}{2.490304in}}%
\pgfpathlineto{\pgfqpoint{4.058508in}{2.493739in}}%
\pgfpathlineto{\pgfqpoint{4.045147in}{2.497204in}}%
\pgfpathlineto{\pgfqpoint{4.052873in}{2.504758in}}%
\pgfpathlineto{\pgfqpoint{4.060592in}{2.512307in}}%
\pgfpathlineto{\pgfqpoint{4.068306in}{2.519850in}}%
\pgfpathlineto{\pgfqpoint{4.076015in}{2.527388in}}%
\pgfpathclose%
\pgfusepath{fill}%
\end{pgfscope}%
\begin{pgfscope}%
\pgfpathrectangle{\pgfqpoint{1.150000in}{0.150000in}}{\pgfqpoint{5.700000in}{5.700000in}}%
\pgfusepath{clip}%
\pgfsetbuttcap%
\pgfsetroundjoin%
\definecolor{currentfill}{rgb}{0.280894,0.078907,0.402329}%
\pgfsetfillcolor{currentfill}%
\pgfsetfillopacity{0.700000}%
\pgfsetlinewidth{0.000000pt}%
\definecolor{currentstroke}{rgb}{0.000000,0.000000,0.000000}%
\pgfsetstrokecolor{currentstroke}%
\pgfsetdash{}{0pt}%
\pgfpathmoveto{\pgfqpoint{5.100766in}{2.625071in}}%
\pgfpathlineto{\pgfqpoint{5.114365in}{2.622570in}}%
\pgfpathlineto{\pgfqpoint{5.127971in}{2.620094in}}%
\pgfpathlineto{\pgfqpoint{5.141584in}{2.617643in}}%
\pgfpathlineto{\pgfqpoint{5.155204in}{2.615218in}}%
\pgfpathlineto{\pgfqpoint{5.147898in}{2.608552in}}%
\pgfpathlineto{\pgfqpoint{5.140586in}{2.601912in}}%
\pgfpathlineto{\pgfqpoint{5.133269in}{2.595293in}}%
\pgfpathlineto{\pgfqpoint{5.125947in}{2.588692in}}%
\pgfpathlineto{\pgfqpoint{5.112311in}{2.591001in}}%
\pgfpathlineto{\pgfqpoint{5.098682in}{2.593336in}}%
\pgfpathlineto{\pgfqpoint{5.085060in}{2.595695in}}%
\pgfpathlineto{\pgfqpoint{5.071445in}{2.598080in}}%
\pgfpathlineto{\pgfqpoint{5.078783in}{2.604793in}}%
\pgfpathlineto{\pgfqpoint{5.086115in}{2.611526in}}%
\pgfpathlineto{\pgfqpoint{5.093443in}{2.618284in}}%
\pgfpathlineto{\pgfqpoint{5.100766in}{2.625071in}}%
\pgfpathclose%
\pgfusepath{fill}%
\end{pgfscope}%
\begin{pgfscope}%
\pgfpathrectangle{\pgfqpoint{1.150000in}{0.150000in}}{\pgfqpoint{5.700000in}{5.700000in}}%
\pgfusepath{clip}%
\pgfsetbuttcap%
\pgfsetroundjoin%
\definecolor{currentfill}{rgb}{0.279566,0.067836,0.391917}%
\pgfsetfillcolor{currentfill}%
\pgfsetfillopacity{0.700000}%
\pgfsetlinewidth{0.000000pt}%
\definecolor{currentstroke}{rgb}{0.000000,0.000000,0.000000}%
\pgfsetstrokecolor{currentstroke}%
\pgfsetdash{}{0pt}%
\pgfpathmoveto{\pgfqpoint{4.879040in}{2.600738in}}%
\pgfpathlineto{\pgfqpoint{4.892584in}{2.598150in}}%
\pgfpathlineto{\pgfqpoint{4.906136in}{2.595587in}}%
\pgfpathlineto{\pgfqpoint{4.919694in}{2.593050in}}%
\pgfpathlineto{\pgfqpoint{4.933260in}{2.590538in}}%
\pgfpathlineto{\pgfqpoint{4.925866in}{2.583689in}}%
\pgfpathlineto{\pgfqpoint{4.918466in}{2.576847in}}%
\pgfpathlineto{\pgfqpoint{4.911061in}{2.570009in}}%
\pgfpathlineto{\pgfqpoint{4.903650in}{2.563172in}}%
\pgfpathlineto{\pgfqpoint{4.890070in}{2.565594in}}%
\pgfpathlineto{\pgfqpoint{4.876497in}{2.568041in}}%
\pgfpathlineto{\pgfqpoint{4.862930in}{2.570513in}}%
\pgfpathlineto{\pgfqpoint{4.849371in}{2.573012in}}%
\pgfpathlineto{\pgfqpoint{4.856796in}{2.579934in}}%
\pgfpathlineto{\pgfqpoint{4.864216in}{2.586860in}}%
\pgfpathlineto{\pgfqpoint{4.871631in}{2.593794in}}%
\pgfpathlineto{\pgfqpoint{4.879040in}{2.600738in}}%
\pgfpathclose%
\pgfusepath{fill}%
\end{pgfscope}%
\begin{pgfscope}%
\pgfpathrectangle{\pgfqpoint{1.150000in}{0.150000in}}{\pgfqpoint{5.700000in}{5.700000in}}%
\pgfusepath{clip}%
\pgfsetbuttcap%
\pgfsetroundjoin%
\definecolor{currentfill}{rgb}{0.277018,0.050344,0.375715}%
\pgfsetfillcolor{currentfill}%
\pgfsetfillopacity{0.700000}%
\pgfsetlinewidth{0.000000pt}%
\definecolor{currentstroke}{rgb}{0.000000,0.000000,0.000000}%
\pgfsetstrokecolor{currentstroke}%
\pgfsetdash{}{0pt}%
\pgfpathmoveto{\pgfqpoint{4.657278in}{2.576427in}}%
\pgfpathlineto{\pgfqpoint{4.670768in}{2.573689in}}%
\pgfpathlineto{\pgfqpoint{4.684265in}{2.570977in}}%
\pgfpathlineto{\pgfqpoint{4.697769in}{2.568291in}}%
\pgfpathlineto{\pgfqpoint{4.711280in}{2.565633in}}%
\pgfpathlineto{\pgfqpoint{4.703799in}{2.558555in}}%
\pgfpathlineto{\pgfqpoint{4.696313in}{2.551472in}}%
\pgfpathlineto{\pgfqpoint{4.688821in}{2.544382in}}%
\pgfpathlineto{\pgfqpoint{4.681323in}{2.537283in}}%
\pgfpathlineto{\pgfqpoint{4.667798in}{2.539878in}}%
\pgfpathlineto{\pgfqpoint{4.654281in}{2.542500in}}%
\pgfpathlineto{\pgfqpoint{4.640770in}{2.545148in}}%
\pgfpathlineto{\pgfqpoint{4.627266in}{2.547824in}}%
\pgfpathlineto{\pgfqpoint{4.634777in}{2.554982in}}%
\pgfpathlineto{\pgfqpoint{4.642283in}{2.562133in}}%
\pgfpathlineto{\pgfqpoint{4.649783in}{2.569281in}}%
\pgfpathlineto{\pgfqpoint{4.657278in}{2.576427in}}%
\pgfpathclose%
\pgfusepath{fill}%
\end{pgfscope}%
\begin{pgfscope}%
\pgfpathrectangle{\pgfqpoint{1.150000in}{0.150000in}}{\pgfqpoint{5.700000in}{5.700000in}}%
\pgfusepath{clip}%
\pgfsetbuttcap%
\pgfsetroundjoin%
\definecolor{currentfill}{rgb}{0.268510,0.009605,0.335427}%
\pgfsetfillcolor{currentfill}%
\pgfsetfillopacity{0.700000}%
\pgfsetlinewidth{0.000000pt}%
\definecolor{currentstroke}{rgb}{0.000000,0.000000,0.000000}%
\pgfsetstrokecolor{currentstroke}%
\pgfsetdash{}{0pt}%
\pgfpathmoveto{\pgfqpoint{3.854171in}{2.511112in}}%
\pgfpathlineto{\pgfqpoint{3.867474in}{2.507256in}}%
\pgfpathlineto{\pgfqpoint{3.880782in}{2.503432in}}%
\pgfpathlineto{\pgfqpoint{3.894097in}{2.499640in}}%
\pgfpathlineto{\pgfqpoint{3.907417in}{2.495880in}}%
\pgfpathlineto{\pgfqpoint{3.899639in}{2.488341in}}%
\pgfpathlineto{\pgfqpoint{3.891855in}{2.480804in}}%
\pgfpathlineto{\pgfqpoint{3.884065in}{2.473266in}}%
\pgfpathlineto{\pgfqpoint{3.876270in}{2.465729in}}%
\pgfpathlineto{\pgfqpoint{3.862937in}{2.469518in}}%
\pgfpathlineto{\pgfqpoint{3.849610in}{2.473339in}}%
\pgfpathlineto{\pgfqpoint{3.836289in}{2.477192in}}%
\pgfpathlineto{\pgfqpoint{3.822973in}{2.481077in}}%
\pgfpathlineto{\pgfqpoint{3.830781in}{2.488580in}}%
\pgfpathlineto{\pgfqpoint{3.838583in}{2.496087in}}%
\pgfpathlineto{\pgfqpoint{3.846380in}{2.503598in}}%
\pgfpathlineto{\pgfqpoint{3.854171in}{2.511112in}}%
\pgfpathclose%
\pgfusepath{fill}%
\end{pgfscope}%
\begin{pgfscope}%
\pgfpathrectangle{\pgfqpoint{1.150000in}{0.150000in}}{\pgfqpoint{5.700000in}{5.700000in}}%
\pgfusepath{clip}%
\pgfsetbuttcap%
\pgfsetroundjoin%
\definecolor{currentfill}{rgb}{0.272594,0.025563,0.353093}%
\pgfsetfillcolor{currentfill}%
\pgfsetfillopacity{0.700000}%
\pgfsetlinewidth{0.000000pt}%
\definecolor{currentstroke}{rgb}{0.000000,0.000000,0.000000}%
\pgfsetstrokecolor{currentstroke}%
\pgfsetdash{}{0pt}%
\pgfpathmoveto{\pgfqpoint{3.219433in}{2.534156in}}%
\pgfpathlineto{\pgfqpoint{3.232623in}{2.528715in}}%
\pgfpathlineto{\pgfqpoint{3.245818in}{2.523314in}}%
\pgfpathlineto{\pgfqpoint{3.259016in}{2.517955in}}%
\pgfpathlineto{\pgfqpoint{3.272219in}{2.512636in}}%
\pgfpathlineto{\pgfqpoint{3.264196in}{2.506052in}}%
\pgfpathlineto{\pgfqpoint{3.256166in}{2.499519in}}%
\pgfpathlineto{\pgfqpoint{3.248129in}{2.493040in}}%
\pgfpathlineto{\pgfqpoint{3.240084in}{2.486615in}}%
\pgfpathlineto{\pgfqpoint{3.226866in}{2.492029in}}%
\pgfpathlineto{\pgfqpoint{3.213652in}{2.497483in}}%
\pgfpathlineto{\pgfqpoint{3.200442in}{2.502978in}}%
\pgfpathlineto{\pgfqpoint{3.187236in}{2.508515in}}%
\pgfpathlineto{\pgfqpoint{3.195296in}{2.514840in}}%
\pgfpathlineto{\pgfqpoint{3.203349in}{2.521223in}}%
\pgfpathlineto{\pgfqpoint{3.211395in}{2.527662in}}%
\pgfpathlineto{\pgfqpoint{3.219433in}{2.534156in}}%
\pgfpathclose%
\pgfusepath{fill}%
\end{pgfscope}%
\begin{pgfscope}%
\pgfpathrectangle{\pgfqpoint{1.150000in}{0.150000in}}{\pgfqpoint{5.700000in}{5.700000in}}%
\pgfusepath{clip}%
\pgfsetbuttcap%
\pgfsetroundjoin%
\definecolor{currentfill}{rgb}{0.269944,0.014625,0.341379}%
\pgfsetfillcolor{currentfill}%
\pgfsetfillopacity{0.700000}%
\pgfsetlinewidth{0.000000pt}%
\definecolor{currentstroke}{rgb}{0.000000,0.000000,0.000000}%
\pgfsetstrokecolor{currentstroke}%
\pgfsetdash{}{0pt}%
\pgfpathmoveto{\pgfqpoint{3.357041in}{2.518898in}}%
\pgfpathlineto{\pgfqpoint{3.370252in}{2.513859in}}%
\pgfpathlineto{\pgfqpoint{3.383468in}{2.508859in}}%
\pgfpathlineto{\pgfqpoint{3.396688in}{2.503898in}}%
\pgfpathlineto{\pgfqpoint{3.409913in}{2.498974in}}%
\pgfpathlineto{\pgfqpoint{3.401946in}{2.492051in}}%
\pgfpathlineto{\pgfqpoint{3.393973in}{2.485165in}}%
\pgfpathlineto{\pgfqpoint{3.385993in}{2.478318in}}%
\pgfpathlineto{\pgfqpoint{3.378006in}{2.471511in}}%
\pgfpathlineto{\pgfqpoint{3.364767in}{2.476516in}}%
\pgfpathlineto{\pgfqpoint{3.351532in}{2.481559in}}%
\pgfpathlineto{\pgfqpoint{3.338302in}{2.486641in}}%
\pgfpathlineto{\pgfqpoint{3.325076in}{2.491761in}}%
\pgfpathlineto{\pgfqpoint{3.333078in}{2.498481in}}%
\pgfpathlineto{\pgfqpoint{3.341072in}{2.505245in}}%
\pgfpathlineto{\pgfqpoint{3.349060in}{2.512051in}}%
\pgfpathlineto{\pgfqpoint{3.357041in}{2.518898in}}%
\pgfpathclose%
\pgfusepath{fill}%
\end{pgfscope}%
\begin{pgfscope}%
\pgfpathrectangle{\pgfqpoint{1.150000in}{0.150000in}}{\pgfqpoint{5.700000in}{5.700000in}}%
\pgfusepath{clip}%
\pgfsetbuttcap%
\pgfsetroundjoin%
\definecolor{currentfill}{rgb}{0.274952,0.037752,0.364543}%
\pgfsetfillcolor{currentfill}%
\pgfsetfillopacity{0.700000}%
\pgfsetlinewidth{0.000000pt}%
\definecolor{currentstroke}{rgb}{0.000000,0.000000,0.000000}%
\pgfsetstrokecolor{currentstroke}%
\pgfsetdash{}{0pt}%
\pgfpathmoveto{\pgfqpoint{4.435484in}{2.552709in}}%
\pgfpathlineto{\pgfqpoint{4.448921in}{2.549756in}}%
\pgfpathlineto{\pgfqpoint{4.462364in}{2.546831in}}%
\pgfpathlineto{\pgfqpoint{4.475814in}{2.543933in}}%
\pgfpathlineto{\pgfqpoint{4.489271in}{2.541063in}}%
\pgfpathlineto{\pgfqpoint{4.481705in}{2.533759in}}%
\pgfpathlineto{\pgfqpoint{4.474134in}{2.526445in}}%
\pgfpathlineto{\pgfqpoint{4.466557in}{2.519117in}}%
\pgfpathlineto{\pgfqpoint{4.458975in}{2.511776in}}%
\pgfpathlineto{\pgfqpoint{4.445505in}{2.514609in}}%
\pgfpathlineto{\pgfqpoint{4.432042in}{2.517469in}}%
\pgfpathlineto{\pgfqpoint{4.418586in}{2.520357in}}%
\pgfpathlineto{\pgfqpoint{4.405136in}{2.523274in}}%
\pgfpathlineto{\pgfqpoint{4.412731in}{2.530648in}}%
\pgfpathlineto{\pgfqpoint{4.420321in}{2.538010in}}%
\pgfpathlineto{\pgfqpoint{4.427905in}{2.545364in}}%
\pgfpathlineto{\pgfqpoint{4.435484in}{2.552709in}}%
\pgfpathclose%
\pgfusepath{fill}%
\end{pgfscope}%
\begin{pgfscope}%
\pgfpathrectangle{\pgfqpoint{1.150000in}{0.150000in}}{\pgfqpoint{5.700000in}{5.700000in}}%
\pgfusepath{clip}%
\pgfsetbuttcap%
\pgfsetroundjoin%
\definecolor{currentfill}{rgb}{0.274952,0.037752,0.364543}%
\pgfsetfillcolor{currentfill}%
\pgfsetfillopacity{0.700000}%
\pgfsetlinewidth{0.000000pt}%
\definecolor{currentstroke}{rgb}{0.000000,0.000000,0.000000}%
\pgfsetstrokecolor{currentstroke}%
\pgfsetdash{}{0pt}%
\pgfpathmoveto{\pgfqpoint{3.081739in}{2.554339in}}%
\pgfpathlineto{\pgfqpoint{3.094912in}{2.548458in}}%
\pgfpathlineto{\pgfqpoint{3.108089in}{2.542622in}}%
\pgfpathlineto{\pgfqpoint{3.121270in}{2.536830in}}%
\pgfpathlineto{\pgfqpoint{3.134455in}{2.531082in}}%
\pgfpathlineto{\pgfqpoint{3.126372in}{2.524921in}}%
\pgfpathlineto{\pgfqpoint{3.118280in}{2.518827in}}%
\pgfpathlineto{\pgfqpoint{3.110182in}{2.512801in}}%
\pgfpathlineto{\pgfqpoint{3.102075in}{2.506846in}}%
\pgfpathlineto{\pgfqpoint{3.088874in}{2.512703in}}%
\pgfpathlineto{\pgfqpoint{3.075676in}{2.518604in}}%
\pgfpathlineto{\pgfqpoint{3.062482in}{2.524548in}}%
\pgfpathlineto{\pgfqpoint{3.049292in}{2.530537in}}%
\pgfpathlineto{\pgfqpoint{3.057416in}{2.536378in}}%
\pgfpathlineto{\pgfqpoint{3.065531in}{2.542294in}}%
\pgfpathlineto{\pgfqpoint{3.073639in}{2.548282in}}%
\pgfpathlineto{\pgfqpoint{3.081739in}{2.554339in}}%
\pgfpathclose%
\pgfusepath{fill}%
\end{pgfscope}%
\begin{pgfscope}%
\pgfpathrectangle{\pgfqpoint{1.150000in}{0.150000in}}{\pgfqpoint{5.700000in}{5.700000in}}%
\pgfusepath{clip}%
\pgfsetbuttcap%
\pgfsetroundjoin%
\definecolor{currentfill}{rgb}{0.268510,0.009605,0.335427}%
\pgfsetfillcolor{currentfill}%
\pgfsetfillopacity{0.700000}%
\pgfsetlinewidth{0.000000pt}%
\definecolor{currentstroke}{rgb}{0.000000,0.000000,0.000000}%
\pgfsetstrokecolor{currentstroke}%
\pgfsetdash{}{0pt}%
\pgfpathmoveto{\pgfqpoint{3.494608in}{2.507968in}}%
\pgfpathlineto{\pgfqpoint{3.507843in}{2.503299in}}%
\pgfpathlineto{\pgfqpoint{3.521084in}{2.498667in}}%
\pgfpathlineto{\pgfqpoint{3.534329in}{2.494072in}}%
\pgfpathlineto{\pgfqpoint{3.547580in}{2.489512in}}%
\pgfpathlineto{\pgfqpoint{3.539666in}{2.482327in}}%
\pgfpathlineto{\pgfqpoint{3.531746in}{2.475167in}}%
\pgfpathlineto{\pgfqpoint{3.523820in}{2.468033in}}%
\pgfpathlineto{\pgfqpoint{3.515888in}{2.460927in}}%
\pgfpathlineto{\pgfqpoint{3.502624in}{2.465555in}}%
\pgfpathlineto{\pgfqpoint{3.489365in}{2.470219in}}%
\pgfpathlineto{\pgfqpoint{3.476110in}{2.474920in}}%
\pgfpathlineto{\pgfqpoint{3.462861in}{2.479657in}}%
\pgfpathlineto{\pgfqpoint{3.470807in}{2.486689in}}%
\pgfpathlineto{\pgfqpoint{3.478747in}{2.493753in}}%
\pgfpathlineto{\pgfqpoint{3.486681in}{2.500846in}}%
\pgfpathlineto{\pgfqpoint{3.494608in}{2.507968in}}%
\pgfpathclose%
\pgfusepath{fill}%
\end{pgfscope}%
\begin{pgfscope}%
\pgfpathrectangle{\pgfqpoint{1.150000in}{0.150000in}}{\pgfqpoint{5.700000in}{5.700000in}}%
\pgfusepath{clip}%
\pgfsetbuttcap%
\pgfsetroundjoin%
\definecolor{currentfill}{rgb}{0.272594,0.025563,0.353093}%
\pgfsetfillcolor{currentfill}%
\pgfsetfillopacity{0.700000}%
\pgfsetlinewidth{0.000000pt}%
\definecolor{currentstroke}{rgb}{0.000000,0.000000,0.000000}%
\pgfsetstrokecolor{currentstroke}%
\pgfsetdash{}{0pt}%
\pgfpathmoveto{\pgfqpoint{4.213653in}{2.530561in}}%
\pgfpathlineto{\pgfqpoint{4.227039in}{2.527326in}}%
\pgfpathlineto{\pgfqpoint{4.240430in}{2.524120in}}%
\pgfpathlineto{\pgfqpoint{4.253828in}{2.520943in}}%
\pgfpathlineto{\pgfqpoint{4.267233in}{2.517795in}}%
\pgfpathlineto{\pgfqpoint{4.259584in}{2.510316in}}%
\pgfpathlineto{\pgfqpoint{4.251930in}{2.502824in}}%
\pgfpathlineto{\pgfqpoint{4.244270in}{2.495320in}}%
\pgfpathlineto{\pgfqpoint{4.236604in}{2.487803in}}%
\pgfpathlineto{\pgfqpoint{4.223187in}{2.490940in}}%
\pgfpathlineto{\pgfqpoint{4.209777in}{2.494106in}}%
\pgfpathlineto{\pgfqpoint{4.196373in}{2.497302in}}%
\pgfpathlineto{\pgfqpoint{4.182975in}{2.500526in}}%
\pgfpathlineto{\pgfqpoint{4.190653in}{2.508049in}}%
\pgfpathlineto{\pgfqpoint{4.198325in}{2.515562in}}%
\pgfpathlineto{\pgfqpoint{4.205992in}{2.523066in}}%
\pgfpathlineto{\pgfqpoint{4.213653in}{2.530561in}}%
\pgfpathclose%
\pgfusepath{fill}%
\end{pgfscope}%
\begin{pgfscope}%
\pgfpathrectangle{\pgfqpoint{1.150000in}{0.150000in}}{\pgfqpoint{5.700000in}{5.700000in}}%
\pgfusepath{clip}%
\pgfsetbuttcap%
\pgfsetroundjoin%
\definecolor{currentfill}{rgb}{0.282656,0.100196,0.422160}%
\pgfsetfillcolor{currentfill}%
\pgfsetfillopacity{0.700000}%
\pgfsetlinewidth{0.000000pt}%
\definecolor{currentstroke}{rgb}{0.000000,0.000000,0.000000}%
\pgfsetstrokecolor{currentstroke}%
\pgfsetdash{}{0pt}%
\pgfpathmoveto{\pgfqpoint{5.460647in}{2.656415in}}%
\pgfpathlineto{\pgfqpoint{5.474340in}{2.653982in}}%
\pgfpathlineto{\pgfqpoint{5.488041in}{2.651572in}}%
\pgfpathlineto{\pgfqpoint{5.501749in}{2.649187in}}%
\pgfpathlineto{\pgfqpoint{5.515465in}{2.646825in}}%
\pgfpathlineto{\pgfqpoint{5.508296in}{2.640328in}}%
\pgfpathlineto{\pgfqpoint{5.501124in}{2.633892in}}%
\pgfpathlineto{\pgfqpoint{5.493948in}{2.627514in}}%
\pgfpathlineto{\pgfqpoint{5.486768in}{2.621188in}}%
\pgfpathlineto{\pgfqpoint{5.473034in}{2.623393in}}%
\pgfpathlineto{\pgfqpoint{5.459307in}{2.625622in}}%
\pgfpathlineto{\pgfqpoint{5.445589in}{2.627876in}}%
\pgfpathlineto{\pgfqpoint{5.431877in}{2.630154in}}%
\pgfpathlineto{\pgfqpoint{5.439075in}{2.636631in}}%
\pgfpathlineto{\pgfqpoint{5.446270in}{2.643164in}}%
\pgfpathlineto{\pgfqpoint{5.453460in}{2.649757in}}%
\pgfpathlineto{\pgfqpoint{5.460647in}{2.656415in}}%
\pgfpathclose%
\pgfusepath{fill}%
\end{pgfscope}%
\begin{pgfscope}%
\pgfpathrectangle{\pgfqpoint{1.150000in}{0.150000in}}{\pgfqpoint{5.700000in}{5.700000in}}%
\pgfusepath{clip}%
\pgfsetbuttcap%
\pgfsetroundjoin%
\definecolor{currentfill}{rgb}{0.277941,0.056324,0.381191}%
\pgfsetfillcolor{currentfill}%
\pgfsetfillopacity{0.700000}%
\pgfsetlinewidth{0.000000pt}%
\definecolor{currentstroke}{rgb}{0.000000,0.000000,0.000000}%
\pgfsetstrokecolor{currentstroke}%
\pgfsetdash{}{0pt}%
\pgfpathmoveto{\pgfqpoint{2.943906in}{2.580094in}}%
\pgfpathlineto{\pgfqpoint{2.957067in}{2.573736in}}%
\pgfpathlineto{\pgfqpoint{2.970231in}{2.567425in}}%
\pgfpathlineto{\pgfqpoint{2.983399in}{2.561161in}}%
\pgfpathlineto{\pgfqpoint{2.996570in}{2.554944in}}%
\pgfpathlineto{\pgfqpoint{2.988422in}{2.549297in}}%
\pgfpathlineto{\pgfqpoint{2.980264in}{2.543733in}}%
\pgfpathlineto{\pgfqpoint{2.972099in}{2.538254in}}%
\pgfpathlineto{\pgfqpoint{2.963925in}{2.532864in}}%
\pgfpathlineto{\pgfqpoint{2.950736in}{2.539203in}}%
\pgfpathlineto{\pgfqpoint{2.937550in}{2.545588in}}%
\pgfpathlineto{\pgfqpoint{2.924368in}{2.552021in}}%
\pgfpathlineto{\pgfqpoint{2.911189in}{2.558502in}}%
\pgfpathlineto{\pgfqpoint{2.919381in}{2.563765in}}%
\pgfpathlineto{\pgfqpoint{2.927565in}{2.569120in}}%
\pgfpathlineto{\pgfqpoint{2.935740in}{2.574564in}}%
\pgfpathlineto{\pgfqpoint{2.943906in}{2.580094in}}%
\pgfpathclose%
\pgfusepath{fill}%
\end{pgfscope}%
\begin{pgfscope}%
\pgfpathrectangle{\pgfqpoint{1.150000in}{0.150000in}}{\pgfqpoint{5.700000in}{5.700000in}}%
\pgfusepath{clip}%
\pgfsetbuttcap%
\pgfsetroundjoin%
\definecolor{currentfill}{rgb}{0.281924,0.089666,0.412415}%
\pgfsetfillcolor{currentfill}%
\pgfsetfillopacity{0.700000}%
\pgfsetlinewidth{0.000000pt}%
\definecolor{currentstroke}{rgb}{0.000000,0.000000,0.000000}%
\pgfsetstrokecolor{currentstroke}%
\pgfsetdash{}{0pt}%
\pgfpathmoveto{\pgfqpoint{5.238873in}{2.632243in}}%
\pgfpathlineto{\pgfqpoint{5.252513in}{2.629812in}}%
\pgfpathlineto{\pgfqpoint{5.266161in}{2.627407in}}%
\pgfpathlineto{\pgfqpoint{5.279816in}{2.625026in}}%
\pgfpathlineto{\pgfqpoint{5.293478in}{2.622670in}}%
\pgfpathlineto{\pgfqpoint{5.286223in}{2.616132in}}%
\pgfpathlineto{\pgfqpoint{5.278964in}{2.609629in}}%
\pgfpathlineto{\pgfqpoint{5.271700in}{2.603156in}}%
\pgfpathlineto{\pgfqpoint{5.264431in}{2.596711in}}%
\pgfpathlineto{\pgfqpoint{5.250752in}{2.598937in}}%
\pgfpathlineto{\pgfqpoint{5.237080in}{2.601188in}}%
\pgfpathlineto{\pgfqpoint{5.223416in}{2.603464in}}%
\pgfpathlineto{\pgfqpoint{5.209759in}{2.605765in}}%
\pgfpathlineto{\pgfqpoint{5.217044in}{2.612335in}}%
\pgfpathlineto{\pgfqpoint{5.224325in}{2.618936in}}%
\pgfpathlineto{\pgfqpoint{5.231601in}{2.625570in}}%
\pgfpathlineto{\pgfqpoint{5.238873in}{2.632243in}}%
\pgfpathclose%
\pgfusepath{fill}%
\end{pgfscope}%
\begin{pgfscope}%
\pgfpathrectangle{\pgfqpoint{1.150000in}{0.150000in}}{\pgfqpoint{5.700000in}{5.700000in}}%
\pgfusepath{clip}%
\pgfsetbuttcap%
\pgfsetroundjoin%
\definecolor{currentfill}{rgb}{0.269944,0.014625,0.341379}%
\pgfsetfillcolor{currentfill}%
\pgfsetfillopacity{0.700000}%
\pgfsetlinewidth{0.000000pt}%
\definecolor{currentstroke}{rgb}{0.000000,0.000000,0.000000}%
\pgfsetstrokecolor{currentstroke}%
\pgfsetdash{}{0pt}%
\pgfpathmoveto{\pgfqpoint{3.991763in}{2.511370in}}%
\pgfpathlineto{\pgfqpoint{4.005100in}{2.507783in}}%
\pgfpathlineto{\pgfqpoint{4.018443in}{2.504226in}}%
\pgfpathlineto{\pgfqpoint{4.031792in}{2.500700in}}%
\pgfpathlineto{\pgfqpoint{4.045147in}{2.497204in}}%
\pgfpathlineto{\pgfqpoint{4.037417in}{2.489644in}}%
\pgfpathlineto{\pgfqpoint{4.029680in}{2.482077in}}%
\pgfpathlineto{\pgfqpoint{4.021938in}{2.474504in}}%
\pgfpathlineto{\pgfqpoint{4.014190in}{2.466925in}}%
\pgfpathlineto{\pgfqpoint{4.000823in}{2.470436in}}%
\pgfpathlineto{\pgfqpoint{3.987462in}{2.473978in}}%
\pgfpathlineto{\pgfqpoint{3.974106in}{2.477551in}}%
\pgfpathlineto{\pgfqpoint{3.960757in}{2.481154in}}%
\pgfpathlineto{\pgfqpoint{3.968517in}{2.488713in}}%
\pgfpathlineto{\pgfqpoint{3.976271in}{2.496269in}}%
\pgfpathlineto{\pgfqpoint{3.984020in}{2.503821in}}%
\pgfpathlineto{\pgfqpoint{3.991763in}{2.511370in}}%
\pgfpathclose%
\pgfusepath{fill}%
\end{pgfscope}%
\begin{pgfscope}%
\pgfpathrectangle{\pgfqpoint{1.150000in}{0.150000in}}{\pgfqpoint{5.700000in}{5.700000in}}%
\pgfusepath{clip}%
\pgfsetbuttcap%
\pgfsetroundjoin%
\definecolor{currentfill}{rgb}{0.268510,0.009605,0.335427}%
\pgfsetfillcolor{currentfill}%
\pgfsetfillopacity{0.700000}%
\pgfsetlinewidth{0.000000pt}%
\definecolor{currentstroke}{rgb}{0.000000,0.000000,0.000000}%
\pgfsetstrokecolor{currentstroke}%
\pgfsetdash{}{0pt}%
\pgfpathmoveto{\pgfqpoint{3.632172in}{2.500817in}}%
\pgfpathlineto{\pgfqpoint{3.645435in}{2.496489in}}%
\pgfpathlineto{\pgfqpoint{3.658703in}{2.492195in}}%
\pgfpathlineto{\pgfqpoint{3.671977in}{2.487936in}}%
\pgfpathlineto{\pgfqpoint{3.685256in}{2.483711in}}%
\pgfpathlineto{\pgfqpoint{3.677393in}{2.476336in}}%
\pgfpathlineto{\pgfqpoint{3.669524in}{2.468975in}}%
\pgfpathlineto{\pgfqpoint{3.661650in}{2.461630in}}%
\pgfpathlineto{\pgfqpoint{3.653769in}{2.454301in}}%
\pgfpathlineto{\pgfqpoint{3.640477in}{2.458581in}}%
\pgfpathlineto{\pgfqpoint{3.627190in}{2.462895in}}%
\pgfpathlineto{\pgfqpoint{3.613909in}{2.467244in}}%
\pgfpathlineto{\pgfqpoint{3.600632in}{2.471628in}}%
\pgfpathlineto{\pgfqpoint{3.608526in}{2.478897in}}%
\pgfpathlineto{\pgfqpoint{3.616414in}{2.486186in}}%
\pgfpathlineto{\pgfqpoint{3.624296in}{2.493493in}}%
\pgfpathlineto{\pgfqpoint{3.632172in}{2.500817in}}%
\pgfpathclose%
\pgfusepath{fill}%
\end{pgfscope}%
\begin{pgfscope}%
\pgfpathrectangle{\pgfqpoint{1.150000in}{0.150000in}}{\pgfqpoint{5.700000in}{5.700000in}}%
\pgfusepath{clip}%
\pgfsetbuttcap%
\pgfsetroundjoin%
\definecolor{currentfill}{rgb}{0.280267,0.073417,0.397163}%
\pgfsetfillcolor{currentfill}%
\pgfsetfillopacity{0.700000}%
\pgfsetlinewidth{0.000000pt}%
\definecolor{currentstroke}{rgb}{0.000000,0.000000,0.000000}%
\pgfsetstrokecolor{currentstroke}%
\pgfsetdash{}{0pt}%
\pgfpathmoveto{\pgfqpoint{5.017058in}{2.607873in}}%
\pgfpathlineto{\pgfqpoint{5.030644in}{2.605387in}}%
\pgfpathlineto{\pgfqpoint{5.044237in}{2.602926in}}%
\pgfpathlineto{\pgfqpoint{5.057838in}{2.600490in}}%
\pgfpathlineto{\pgfqpoint{5.071445in}{2.598080in}}%
\pgfpathlineto{\pgfqpoint{5.064102in}{2.591385in}}%
\pgfpathlineto{\pgfqpoint{5.056754in}{2.584704in}}%
\pgfpathlineto{\pgfqpoint{5.049401in}{2.578034in}}%
\pgfpathlineto{\pgfqpoint{5.042042in}{2.571371in}}%
\pgfpathlineto{\pgfqpoint{5.028419in}{2.573678in}}%
\pgfpathlineto{\pgfqpoint{5.014804in}{2.576010in}}%
\pgfpathlineto{\pgfqpoint{5.001195in}{2.578367in}}%
\pgfpathlineto{\pgfqpoint{4.987594in}{2.580750in}}%
\pgfpathlineto{\pgfqpoint{4.994968in}{2.587512in}}%
\pgfpathlineto{\pgfqpoint{5.002336in}{2.594284in}}%
\pgfpathlineto{\pgfqpoint{5.009700in}{2.601070in}}%
\pgfpathlineto{\pgfqpoint{5.017058in}{2.607873in}}%
\pgfpathclose%
\pgfusepath{fill}%
\end{pgfscope}%
\begin{pgfscope}%
\pgfpathrectangle{\pgfqpoint{1.150000in}{0.150000in}}{\pgfqpoint{5.700000in}{5.700000in}}%
\pgfusepath{clip}%
\pgfsetbuttcap%
\pgfsetroundjoin%
\definecolor{currentfill}{rgb}{0.278791,0.062145,0.386592}%
\pgfsetfillcolor{currentfill}%
\pgfsetfillopacity{0.700000}%
\pgfsetlinewidth{0.000000pt}%
\definecolor{currentstroke}{rgb}{0.000000,0.000000,0.000000}%
\pgfsetstrokecolor{currentstroke}%
\pgfsetdash{}{0pt}%
\pgfpathmoveto{\pgfqpoint{4.795205in}{2.583269in}}%
\pgfpathlineto{\pgfqpoint{4.808736in}{2.580665in}}%
\pgfpathlineto{\pgfqpoint{4.822274in}{2.578088in}}%
\pgfpathlineto{\pgfqpoint{4.835819in}{2.575537in}}%
\pgfpathlineto{\pgfqpoint{4.849371in}{2.573012in}}%
\pgfpathlineto{\pgfqpoint{4.841941in}{2.566092in}}%
\pgfpathlineto{\pgfqpoint{4.834505in}{2.559171in}}%
\pgfpathlineto{\pgfqpoint{4.827063in}{2.552246in}}%
\pgfpathlineto{\pgfqpoint{4.819616in}{2.545315in}}%
\pgfpathlineto{\pgfqpoint{4.806049in}{2.547763in}}%
\pgfpathlineto{\pgfqpoint{4.792490in}{2.550237in}}%
\pgfpathlineto{\pgfqpoint{4.778937in}{2.552737in}}%
\pgfpathlineto{\pgfqpoint{4.765392in}{2.555263in}}%
\pgfpathlineto{\pgfqpoint{4.772853in}{2.562267in}}%
\pgfpathlineto{\pgfqpoint{4.780309in}{2.569267in}}%
\pgfpathlineto{\pgfqpoint{4.787760in}{2.576267in}}%
\pgfpathlineto{\pgfqpoint{4.795205in}{2.583269in}}%
\pgfpathclose%
\pgfusepath{fill}%
\end{pgfscope}%
\begin{pgfscope}%
\pgfpathrectangle{\pgfqpoint{1.150000in}{0.150000in}}{\pgfqpoint{5.700000in}{5.700000in}}%
\pgfusepath{clip}%
\pgfsetbuttcap%
\pgfsetroundjoin%
\definecolor{currentfill}{rgb}{0.276022,0.044167,0.370164}%
\pgfsetfillcolor{currentfill}%
\pgfsetfillopacity{0.700000}%
\pgfsetlinewidth{0.000000pt}%
\definecolor{currentstroke}{rgb}{0.000000,0.000000,0.000000}%
\pgfsetstrokecolor{currentstroke}%
\pgfsetdash{}{0pt}%
\pgfpathmoveto{\pgfqpoint{4.573318in}{2.558794in}}%
\pgfpathlineto{\pgfqpoint{4.586795in}{2.556011in}}%
\pgfpathlineto{\pgfqpoint{4.600279in}{2.553255in}}%
\pgfpathlineto{\pgfqpoint{4.613769in}{2.550526in}}%
\pgfpathlineto{\pgfqpoint{4.627266in}{2.547824in}}%
\pgfpathlineto{\pgfqpoint{4.619749in}{2.540657in}}%
\pgfpathlineto{\pgfqpoint{4.612227in}{2.533481in}}%
\pgfpathlineto{\pgfqpoint{4.604699in}{2.526292in}}%
\pgfpathlineto{\pgfqpoint{4.597165in}{2.519089in}}%
\pgfpathlineto{\pgfqpoint{4.583655in}{2.521741in}}%
\pgfpathlineto{\pgfqpoint{4.570151in}{2.524419in}}%
\pgfpathlineto{\pgfqpoint{4.556654in}{2.527125in}}%
\pgfpathlineto{\pgfqpoint{4.543164in}{2.529858in}}%
\pgfpathlineto{\pgfqpoint{4.550711in}{2.537107in}}%
\pgfpathlineto{\pgfqpoint{4.558253in}{2.544344in}}%
\pgfpathlineto{\pgfqpoint{4.565788in}{2.551573in}}%
\pgfpathlineto{\pgfqpoint{4.573318in}{2.558794in}}%
\pgfpathclose%
\pgfusepath{fill}%
\end{pgfscope}%
\begin{pgfscope}%
\pgfpathrectangle{\pgfqpoint{1.150000in}{0.150000in}}{\pgfqpoint{5.700000in}{5.700000in}}%
\pgfusepath{clip}%
\pgfsetbuttcap%
\pgfsetroundjoin%
\definecolor{currentfill}{rgb}{0.280267,0.073417,0.397163}%
\pgfsetfillcolor{currentfill}%
\pgfsetfillopacity{0.700000}%
\pgfsetlinewidth{0.000000pt}%
\definecolor{currentstroke}{rgb}{0.000000,0.000000,0.000000}%
\pgfsetstrokecolor{currentstroke}%
\pgfsetdash{}{0pt}%
\pgfpathmoveto{\pgfqpoint{2.805875in}{2.612129in}}%
\pgfpathlineto{\pgfqpoint{2.819028in}{2.605248in}}%
\pgfpathlineto{\pgfqpoint{2.832185in}{2.598419in}}%
\pgfpathlineto{\pgfqpoint{2.845344in}{2.591641in}}%
\pgfpathlineto{\pgfqpoint{2.858507in}{2.584913in}}%
\pgfpathlineto{\pgfqpoint{2.850287in}{2.579876in}}%
\pgfpathlineto{\pgfqpoint{2.842058in}{2.574940in}}%
\pgfpathlineto{\pgfqpoint{2.833819in}{2.570108in}}%
\pgfpathlineto{\pgfqpoint{2.825571in}{2.565383in}}%
\pgfpathlineto{\pgfqpoint{2.812390in}{2.572246in}}%
\pgfpathlineto{\pgfqpoint{2.799211in}{2.579160in}}%
\pgfpathlineto{\pgfqpoint{2.786035in}{2.586125in}}%
\pgfpathlineto{\pgfqpoint{2.772862in}{2.593142in}}%
\pgfpathlineto{\pgfqpoint{2.781129in}{2.597726in}}%
\pgfpathlineto{\pgfqpoint{2.789387in}{2.602420in}}%
\pgfpathlineto{\pgfqpoint{2.797636in}{2.607222in}}%
\pgfpathlineto{\pgfqpoint{2.805875in}{2.612129in}}%
\pgfpathclose%
\pgfusepath{fill}%
\end{pgfscope}%
\begin{pgfscope}%
\pgfpathrectangle{\pgfqpoint{1.150000in}{0.150000in}}{\pgfqpoint{5.700000in}{5.700000in}}%
\pgfusepath{clip}%
\pgfsetbuttcap%
\pgfsetroundjoin%
\definecolor{currentfill}{rgb}{0.273809,0.031497,0.358853}%
\pgfsetfillcolor{currentfill}%
\pgfsetfillopacity{0.700000}%
\pgfsetlinewidth{0.000000pt}%
\definecolor{currentstroke}{rgb}{0.000000,0.000000,0.000000}%
\pgfsetstrokecolor{currentstroke}%
\pgfsetdash{}{0pt}%
\pgfpathmoveto{\pgfqpoint{4.351402in}{2.535219in}}%
\pgfpathlineto{\pgfqpoint{4.364826in}{2.532191in}}%
\pgfpathlineto{\pgfqpoint{4.378256in}{2.529190in}}%
\pgfpathlineto{\pgfqpoint{4.391693in}{2.526218in}}%
\pgfpathlineto{\pgfqpoint{4.405136in}{2.523274in}}%
\pgfpathlineto{\pgfqpoint{4.397535in}{2.515887in}}%
\pgfpathlineto{\pgfqpoint{4.389928in}{2.508487in}}%
\pgfpathlineto{\pgfqpoint{4.382316in}{2.501072in}}%
\pgfpathlineto{\pgfqpoint{4.374698in}{2.493641in}}%
\pgfpathlineto{\pgfqpoint{4.361242in}{2.496561in}}%
\pgfpathlineto{\pgfqpoint{4.347793in}{2.499510in}}%
\pgfpathlineto{\pgfqpoint{4.334350in}{2.502486in}}%
\pgfpathlineto{\pgfqpoint{4.320914in}{2.505491in}}%
\pgfpathlineto{\pgfqpoint{4.328544in}{2.512941in}}%
\pgfpathlineto{\pgfqpoint{4.336169in}{2.520378in}}%
\pgfpathlineto{\pgfqpoint{4.343789in}{2.527804in}}%
\pgfpathlineto{\pgfqpoint{4.351402in}{2.535219in}}%
\pgfpathclose%
\pgfusepath{fill}%
\end{pgfscope}%
\begin{pgfscope}%
\pgfpathrectangle{\pgfqpoint{1.150000in}{0.150000in}}{\pgfqpoint{5.700000in}{5.700000in}}%
\pgfusepath{clip}%
\pgfsetbuttcap%
\pgfsetroundjoin%
\definecolor{currentfill}{rgb}{0.268510,0.009605,0.335427}%
\pgfsetfillcolor{currentfill}%
\pgfsetfillopacity{0.700000}%
\pgfsetlinewidth{0.000000pt}%
\definecolor{currentstroke}{rgb}{0.000000,0.000000,0.000000}%
\pgfsetstrokecolor{currentstroke}%
\pgfsetdash{}{0pt}%
\pgfpathmoveto{\pgfqpoint{3.769766in}{2.496942in}}%
\pgfpathlineto{\pgfqpoint{3.783060in}{2.492927in}}%
\pgfpathlineto{\pgfqpoint{3.796358in}{2.488944in}}%
\pgfpathlineto{\pgfqpoint{3.809663in}{2.484994in}}%
\pgfpathlineto{\pgfqpoint{3.822973in}{2.481077in}}%
\pgfpathlineto{\pgfqpoint{3.815159in}{2.473578in}}%
\pgfpathlineto{\pgfqpoint{3.807340in}{2.466085in}}%
\pgfpathlineto{\pgfqpoint{3.799514in}{2.458596in}}%
\pgfpathlineto{\pgfqpoint{3.791683in}{2.451114in}}%
\pgfpathlineto{\pgfqpoint{3.778360in}{2.455073in}}%
\pgfpathlineto{\pgfqpoint{3.765043in}{2.459065in}}%
\pgfpathlineto{\pgfqpoint{3.751732in}{2.463090in}}%
\pgfpathlineto{\pgfqpoint{3.738425in}{2.467148in}}%
\pgfpathlineto{\pgfqpoint{3.746269in}{2.474583in}}%
\pgfpathlineto{\pgfqpoint{3.754108in}{2.482028in}}%
\pgfpathlineto{\pgfqpoint{3.761940in}{2.489481in}}%
\pgfpathlineto{\pgfqpoint{3.769766in}{2.496942in}}%
\pgfpathclose%
\pgfusepath{fill}%
\end{pgfscope}%
\begin{pgfscope}%
\pgfpathrectangle{\pgfqpoint{1.150000in}{0.150000in}}{\pgfqpoint{5.700000in}{5.700000in}}%
\pgfusepath{clip}%
\pgfsetbuttcap%
\pgfsetroundjoin%
\definecolor{currentfill}{rgb}{0.282910,0.105393,0.426902}%
\pgfsetfillcolor{currentfill}%
\pgfsetfillopacity{0.700000}%
\pgfsetlinewidth{0.000000pt}%
\definecolor{currentstroke}{rgb}{0.000000,0.000000,0.000000}%
\pgfsetstrokecolor{currentstroke}%
\pgfsetdash{}{0pt}%
\pgfpathmoveto{\pgfqpoint{5.598967in}{2.663659in}}%
\pgfpathlineto{\pgfqpoint{5.612701in}{2.661250in}}%
\pgfpathlineto{\pgfqpoint{5.626443in}{2.658864in}}%
\pgfpathlineto{\pgfqpoint{5.640192in}{2.656503in}}%
\pgfpathlineto{\pgfqpoint{5.653949in}{2.654165in}}%
\pgfpathlineto{\pgfqpoint{5.646831in}{2.647716in}}%
\pgfpathlineto{\pgfqpoint{5.639711in}{2.641344in}}%
\pgfpathlineto{\pgfqpoint{5.632587in}{2.635042in}}%
\pgfpathlineto{\pgfqpoint{5.625460in}{2.628806in}}%
\pgfpathlineto{\pgfqpoint{5.611684in}{2.630974in}}%
\pgfpathlineto{\pgfqpoint{5.597916in}{2.633166in}}%
\pgfpathlineto{\pgfqpoint{5.584155in}{2.635382in}}%
\pgfpathlineto{\pgfqpoint{5.570402in}{2.637623in}}%
\pgfpathlineto{\pgfqpoint{5.577548in}{2.644023in}}%
\pgfpathlineto{\pgfqpoint{5.584691in}{2.650493in}}%
\pgfpathlineto{\pgfqpoint{5.591831in}{2.657036in}}%
\pgfpathlineto{\pgfqpoint{5.598967in}{2.663659in}}%
\pgfpathclose%
\pgfusepath{fill}%
\end{pgfscope}%
\begin{pgfscope}%
\pgfpathrectangle{\pgfqpoint{1.150000in}{0.150000in}}{\pgfqpoint{5.700000in}{5.700000in}}%
\pgfusepath{clip}%
\pgfsetbuttcap%
\pgfsetroundjoin%
\definecolor{currentfill}{rgb}{0.271305,0.019942,0.347269}%
\pgfsetfillcolor{currentfill}%
\pgfsetfillopacity{0.700000}%
\pgfsetlinewidth{0.000000pt}%
\definecolor{currentstroke}{rgb}{0.000000,0.000000,0.000000}%
\pgfsetstrokecolor{currentstroke}%
\pgfsetdash{}{0pt}%
\pgfpathmoveto{\pgfqpoint{4.129446in}{2.513719in}}%
\pgfpathlineto{\pgfqpoint{4.142819in}{2.510376in}}%
\pgfpathlineto{\pgfqpoint{4.156198in}{2.507063in}}%
\pgfpathlineto{\pgfqpoint{4.169583in}{2.503780in}}%
\pgfpathlineto{\pgfqpoint{4.182975in}{2.500526in}}%
\pgfpathlineto{\pgfqpoint{4.175291in}{2.492992in}}%
\pgfpathlineto{\pgfqpoint{4.167602in}{2.485447in}}%
\pgfpathlineto{\pgfqpoint{4.159907in}{2.477891in}}%
\pgfpathlineto{\pgfqpoint{4.152207in}{2.470322in}}%
\pgfpathlineto{\pgfqpoint{4.138803in}{2.473578in}}%
\pgfpathlineto{\pgfqpoint{4.125405in}{2.476864in}}%
\pgfpathlineto{\pgfqpoint{4.112013in}{2.480179in}}%
\pgfpathlineto{\pgfqpoint{4.098628in}{2.483524in}}%
\pgfpathlineto{\pgfqpoint{4.106341in}{2.491086in}}%
\pgfpathlineto{\pgfqpoint{4.114048in}{2.498638in}}%
\pgfpathlineto{\pgfqpoint{4.121750in}{2.506182in}}%
\pgfpathlineto{\pgfqpoint{4.129446in}{2.513719in}}%
\pgfpathclose%
\pgfusepath{fill}%
\end{pgfscope}%
\begin{pgfscope}%
\pgfpathrectangle{\pgfqpoint{1.150000in}{0.150000in}}{\pgfqpoint{5.700000in}{5.700000in}}%
\pgfusepath{clip}%
\pgfsetbuttcap%
\pgfsetroundjoin%
\definecolor{currentfill}{rgb}{0.282327,0.094955,0.417331}%
\pgfsetfillcolor{currentfill}%
\pgfsetfillopacity{0.700000}%
\pgfsetlinewidth{0.000000pt}%
\definecolor{currentstroke}{rgb}{0.000000,0.000000,0.000000}%
\pgfsetstrokecolor{currentstroke}%
\pgfsetdash{}{0pt}%
\pgfpathmoveto{\pgfqpoint{5.377106in}{2.639510in}}%
\pgfpathlineto{\pgfqpoint{5.390788in}{2.637135in}}%
\pgfpathlineto{\pgfqpoint{5.404477in}{2.634783in}}%
\pgfpathlineto{\pgfqpoint{5.418173in}{2.632456in}}%
\pgfpathlineto{\pgfqpoint{5.431877in}{2.630154in}}%
\pgfpathlineto{\pgfqpoint{5.424675in}{2.623728in}}%
\pgfpathlineto{\pgfqpoint{5.417468in}{2.617347in}}%
\pgfpathlineto{\pgfqpoint{5.410257in}{2.611009in}}%
\pgfpathlineto{\pgfqpoint{5.403042in}{2.604708in}}%
\pgfpathlineto{\pgfqpoint{5.389320in}{2.606867in}}%
\pgfpathlineto{\pgfqpoint{5.375606in}{2.609051in}}%
\pgfpathlineto{\pgfqpoint{5.361899in}{2.611259in}}%
\pgfpathlineto{\pgfqpoint{5.348200in}{2.613492in}}%
\pgfpathlineto{\pgfqpoint{5.355433in}{2.619932in}}%
\pgfpathlineto{\pgfqpoint{5.362662in}{2.626411in}}%
\pgfpathlineto{\pgfqpoint{5.369886in}{2.632936in}}%
\pgfpathlineto{\pgfqpoint{5.377106in}{2.639510in}}%
\pgfpathclose%
\pgfusepath{fill}%
\end{pgfscope}%
\begin{pgfscope}%
\pgfpathrectangle{\pgfqpoint{1.150000in}{0.150000in}}{\pgfqpoint{5.700000in}{5.700000in}}%
\pgfusepath{clip}%
\pgfsetbuttcap%
\pgfsetroundjoin%
\definecolor{currentfill}{rgb}{0.271305,0.019942,0.347269}%
\pgfsetfillcolor{currentfill}%
\pgfsetfillopacity{0.700000}%
\pgfsetlinewidth{0.000000pt}%
\definecolor{currentstroke}{rgb}{0.000000,0.000000,0.000000}%
\pgfsetstrokecolor{currentstroke}%
\pgfsetdash{}{0pt}%
\pgfpathmoveto{\pgfqpoint{3.272219in}{2.512636in}}%
\pgfpathlineto{\pgfqpoint{3.285427in}{2.507358in}}%
\pgfpathlineto{\pgfqpoint{3.298639in}{2.502120in}}%
\pgfpathlineto{\pgfqpoint{3.311855in}{2.496921in}}%
\pgfpathlineto{\pgfqpoint{3.325076in}{2.491761in}}%
\pgfpathlineto{\pgfqpoint{3.317068in}{2.485087in}}%
\pgfpathlineto{\pgfqpoint{3.309053in}{2.478461in}}%
\pgfpathlineto{\pgfqpoint{3.301031in}{2.471885in}}%
\pgfpathlineto{\pgfqpoint{3.293002in}{2.465360in}}%
\pgfpathlineto{\pgfqpoint{3.279766in}{2.470615in}}%
\pgfpathlineto{\pgfqpoint{3.266534in}{2.475908in}}%
\pgfpathlineto{\pgfqpoint{3.253307in}{2.481242in}}%
\pgfpathlineto{\pgfqpoint{3.240084in}{2.486615in}}%
\pgfpathlineto{\pgfqpoint{3.248129in}{2.493040in}}%
\pgfpathlineto{\pgfqpoint{3.256166in}{2.499519in}}%
\pgfpathlineto{\pgfqpoint{3.264196in}{2.506052in}}%
\pgfpathlineto{\pgfqpoint{3.272219in}{2.512636in}}%
\pgfpathclose%
\pgfusepath{fill}%
\end{pgfscope}%
\begin{pgfscope}%
\pgfpathrectangle{\pgfqpoint{1.150000in}{0.150000in}}{\pgfqpoint{5.700000in}{5.700000in}}%
\pgfusepath{clip}%
\pgfsetbuttcap%
\pgfsetroundjoin%
\definecolor{currentfill}{rgb}{0.273809,0.031497,0.358853}%
\pgfsetfillcolor{currentfill}%
\pgfsetfillopacity{0.700000}%
\pgfsetlinewidth{0.000000pt}%
\definecolor{currentstroke}{rgb}{0.000000,0.000000,0.000000}%
\pgfsetstrokecolor{currentstroke}%
\pgfsetdash{}{0pt}%
\pgfpathmoveto{\pgfqpoint{3.134455in}{2.531082in}}%
\pgfpathlineto{\pgfqpoint{3.147644in}{2.525376in}}%
\pgfpathlineto{\pgfqpoint{3.160837in}{2.519713in}}%
\pgfpathlineto{\pgfqpoint{3.174035in}{2.514093in}}%
\pgfpathlineto{\pgfqpoint{3.187236in}{2.508515in}}%
\pgfpathlineto{\pgfqpoint{3.179169in}{2.502251in}}%
\pgfpathlineto{\pgfqpoint{3.171094in}{2.496050in}}%
\pgfpathlineto{\pgfqpoint{3.163011in}{2.489914in}}%
\pgfpathlineto{\pgfqpoint{3.154921in}{2.483846in}}%
\pgfpathlineto{\pgfqpoint{3.141704in}{2.489533in}}%
\pgfpathlineto{\pgfqpoint{3.128490in}{2.495262in}}%
\pgfpathlineto{\pgfqpoint{3.115281in}{2.501033in}}%
\pgfpathlineto{\pgfqpoint{3.102075in}{2.506846in}}%
\pgfpathlineto{\pgfqpoint{3.110182in}{2.512801in}}%
\pgfpathlineto{\pgfqpoint{3.118280in}{2.518827in}}%
\pgfpathlineto{\pgfqpoint{3.126372in}{2.524921in}}%
\pgfpathlineto{\pgfqpoint{3.134455in}{2.531082in}}%
\pgfpathclose%
\pgfusepath{fill}%
\end{pgfscope}%
\begin{pgfscope}%
\pgfpathrectangle{\pgfqpoint{1.150000in}{0.150000in}}{\pgfqpoint{5.700000in}{5.700000in}}%
\pgfusepath{clip}%
\pgfsetbuttcap%
\pgfsetroundjoin%
\definecolor{currentfill}{rgb}{0.281446,0.084320,0.407414}%
\pgfsetfillcolor{currentfill}%
\pgfsetfillopacity{0.700000}%
\pgfsetlinewidth{0.000000pt}%
\definecolor{currentstroke}{rgb}{0.000000,0.000000,0.000000}%
\pgfsetstrokecolor{currentstroke}%
\pgfsetdash{}{0pt}%
\pgfpathmoveto{\pgfqpoint{5.155204in}{2.615218in}}%
\pgfpathlineto{\pgfqpoint{5.168832in}{2.612817in}}%
\pgfpathlineto{\pgfqpoint{5.182467in}{2.610441in}}%
\pgfpathlineto{\pgfqpoint{5.196109in}{2.608091in}}%
\pgfpathlineto{\pgfqpoint{5.209759in}{2.605765in}}%
\pgfpathlineto{\pgfqpoint{5.202468in}{2.599221in}}%
\pgfpathlineto{\pgfqpoint{5.195173in}{2.592699in}}%
\pgfpathlineto{\pgfqpoint{5.187872in}{2.586195in}}%
\pgfpathlineto{\pgfqpoint{5.180566in}{2.579706in}}%
\pgfpathlineto{\pgfqpoint{5.166901in}{2.581915in}}%
\pgfpathlineto{\pgfqpoint{5.153242in}{2.584149in}}%
\pgfpathlineto{\pgfqpoint{5.139591in}{2.586408in}}%
\pgfpathlineto{\pgfqpoint{5.125947in}{2.588692in}}%
\pgfpathlineto{\pgfqpoint{5.133269in}{2.595293in}}%
\pgfpathlineto{\pgfqpoint{5.140586in}{2.601912in}}%
\pgfpathlineto{\pgfqpoint{5.147898in}{2.608552in}}%
\pgfpathlineto{\pgfqpoint{5.155204in}{2.615218in}}%
\pgfpathclose%
\pgfusepath{fill}%
\end{pgfscope}%
\begin{pgfscope}%
\pgfpathrectangle{\pgfqpoint{1.150000in}{0.150000in}}{\pgfqpoint{5.700000in}{5.700000in}}%
\pgfusepath{clip}%
\pgfsetbuttcap%
\pgfsetroundjoin%
\definecolor{currentfill}{rgb}{0.269944,0.014625,0.341379}%
\pgfsetfillcolor{currentfill}%
\pgfsetfillopacity{0.700000}%
\pgfsetlinewidth{0.000000pt}%
\definecolor{currentstroke}{rgb}{0.000000,0.000000,0.000000}%
\pgfsetstrokecolor{currentstroke}%
\pgfsetdash{}{0pt}%
\pgfpathmoveto{\pgfqpoint{3.409913in}{2.498974in}}%
\pgfpathlineto{\pgfqpoint{3.423143in}{2.494089in}}%
\pgfpathlineto{\pgfqpoint{3.436377in}{2.489241in}}%
\pgfpathlineto{\pgfqpoint{3.449617in}{2.484430in}}%
\pgfpathlineto{\pgfqpoint{3.462861in}{2.479657in}}%
\pgfpathlineto{\pgfqpoint{3.454909in}{2.472656in}}%
\pgfpathlineto{\pgfqpoint{3.446950in}{2.465690in}}%
\pgfpathlineto{\pgfqpoint{3.438984in}{2.458760in}}%
\pgfpathlineto{\pgfqpoint{3.431012in}{2.451867in}}%
\pgfpathlineto{\pgfqpoint{3.417753in}{2.456722in}}%
\pgfpathlineto{\pgfqpoint{3.404500in}{2.461614in}}%
\pgfpathlineto{\pgfqpoint{3.391251in}{2.466544in}}%
\pgfpathlineto{\pgfqpoint{3.378006in}{2.471511in}}%
\pgfpathlineto{\pgfqpoint{3.385993in}{2.478318in}}%
\pgfpathlineto{\pgfqpoint{3.393973in}{2.485165in}}%
\pgfpathlineto{\pgfqpoint{3.401946in}{2.492051in}}%
\pgfpathlineto{\pgfqpoint{3.409913in}{2.498974in}}%
\pgfpathclose%
\pgfusepath{fill}%
\end{pgfscope}%
\begin{pgfscope}%
\pgfpathrectangle{\pgfqpoint{1.150000in}{0.150000in}}{\pgfqpoint{5.700000in}{5.700000in}}%
\pgfusepath{clip}%
\pgfsetbuttcap%
\pgfsetroundjoin%
\definecolor{currentfill}{rgb}{0.268510,0.009605,0.335427}%
\pgfsetfillcolor{currentfill}%
\pgfsetfillopacity{0.700000}%
\pgfsetlinewidth{0.000000pt}%
\definecolor{currentstroke}{rgb}{0.000000,0.000000,0.000000}%
\pgfsetstrokecolor{currentstroke}%
\pgfsetdash{}{0pt}%
\pgfpathmoveto{\pgfqpoint{3.907417in}{2.495880in}}%
\pgfpathlineto{\pgfqpoint{3.920743in}{2.492151in}}%
\pgfpathlineto{\pgfqpoint{3.934075in}{2.488454in}}%
\pgfpathlineto{\pgfqpoint{3.947413in}{2.484789in}}%
\pgfpathlineto{\pgfqpoint{3.960757in}{2.481154in}}%
\pgfpathlineto{\pgfqpoint{3.952991in}{2.473592in}}%
\pgfpathlineto{\pgfqpoint{3.945219in}{2.466027in}}%
\pgfpathlineto{\pgfqpoint{3.937442in}{2.458459in}}%
\pgfpathlineto{\pgfqpoint{3.929659in}{2.450888in}}%
\pgfpathlineto{\pgfqpoint{3.916303in}{2.454552in}}%
\pgfpathlineto{\pgfqpoint{3.902953in}{2.458246in}}%
\pgfpathlineto{\pgfqpoint{3.889609in}{2.461972in}}%
\pgfpathlineto{\pgfqpoint{3.876270in}{2.465729in}}%
\pgfpathlineto{\pgfqpoint{3.884065in}{2.473266in}}%
\pgfpathlineto{\pgfqpoint{3.891855in}{2.480804in}}%
\pgfpathlineto{\pgfqpoint{3.899639in}{2.488341in}}%
\pgfpathlineto{\pgfqpoint{3.907417in}{2.495880in}}%
\pgfpathclose%
\pgfusepath{fill}%
\end{pgfscope}%
\begin{pgfscope}%
\pgfpathrectangle{\pgfqpoint{1.150000in}{0.150000in}}{\pgfqpoint{5.700000in}{5.700000in}}%
\pgfusepath{clip}%
\pgfsetbuttcap%
\pgfsetroundjoin%
\definecolor{currentfill}{rgb}{0.279566,0.067836,0.391917}%
\pgfsetfillcolor{currentfill}%
\pgfsetfillopacity{0.700000}%
\pgfsetlinewidth{0.000000pt}%
\definecolor{currentstroke}{rgb}{0.000000,0.000000,0.000000}%
\pgfsetstrokecolor{currentstroke}%
\pgfsetdash{}{0pt}%
\pgfpathmoveto{\pgfqpoint{4.933260in}{2.590538in}}%
\pgfpathlineto{\pgfqpoint{4.946833in}{2.588053in}}%
\pgfpathlineto{\pgfqpoint{4.960413in}{2.585593in}}%
\pgfpathlineto{\pgfqpoint{4.974000in}{2.583159in}}%
\pgfpathlineto{\pgfqpoint{4.987594in}{2.580750in}}%
\pgfpathlineto{\pgfqpoint{4.980214in}{2.573996in}}%
\pgfpathlineto{\pgfqpoint{4.972830in}{2.567246in}}%
\pgfpathlineto{\pgfqpoint{4.965439in}{2.560497in}}%
\pgfpathlineto{\pgfqpoint{4.958044in}{2.553745in}}%
\pgfpathlineto{\pgfqpoint{4.944435in}{2.556064in}}%
\pgfpathlineto{\pgfqpoint{4.930833in}{2.558408in}}%
\pgfpathlineto{\pgfqpoint{4.917238in}{2.560777in}}%
\pgfpathlineto{\pgfqpoint{4.903650in}{2.563172in}}%
\pgfpathlineto{\pgfqpoint{4.911061in}{2.570009in}}%
\pgfpathlineto{\pgfqpoint{4.918466in}{2.576847in}}%
\pgfpathlineto{\pgfqpoint{4.925866in}{2.583689in}}%
\pgfpathlineto{\pgfqpoint{4.933260in}{2.590538in}}%
\pgfpathclose%
\pgfusepath{fill}%
\end{pgfscope}%
\begin{pgfscope}%
\pgfpathrectangle{\pgfqpoint{1.150000in}{0.150000in}}{\pgfqpoint{5.700000in}{5.700000in}}%
\pgfusepath{clip}%
\pgfsetbuttcap%
\pgfsetroundjoin%
\definecolor{currentfill}{rgb}{0.277941,0.056324,0.381191}%
\pgfsetfillcolor{currentfill}%
\pgfsetfillopacity{0.700000}%
\pgfsetlinewidth{0.000000pt}%
\definecolor{currentstroke}{rgb}{0.000000,0.000000,0.000000}%
\pgfsetstrokecolor{currentstroke}%
\pgfsetdash{}{0pt}%
\pgfpathmoveto{\pgfqpoint{4.711280in}{2.565633in}}%
\pgfpathlineto{\pgfqpoint{4.724798in}{2.563001in}}%
\pgfpathlineto{\pgfqpoint{4.738322in}{2.560395in}}%
\pgfpathlineto{\pgfqpoint{4.751854in}{2.557816in}}%
\pgfpathlineto{\pgfqpoint{4.765392in}{2.555263in}}%
\pgfpathlineto{\pgfqpoint{4.757925in}{2.548254in}}%
\pgfpathlineto{\pgfqpoint{4.750452in}{2.541237in}}%
\pgfpathlineto{\pgfqpoint{4.742974in}{2.534209in}}%
\pgfpathlineto{\pgfqpoint{4.735490in}{2.527169in}}%
\pgfpathlineto{\pgfqpoint{4.721938in}{2.529658in}}%
\pgfpathlineto{\pgfqpoint{4.708393in}{2.532173in}}%
\pgfpathlineto{\pgfqpoint{4.694854in}{2.534715in}}%
\pgfpathlineto{\pgfqpoint{4.681323in}{2.537283in}}%
\pgfpathlineto{\pgfqpoint{4.688821in}{2.544382in}}%
\pgfpathlineto{\pgfqpoint{4.696313in}{2.551472in}}%
\pgfpathlineto{\pgfqpoint{4.703799in}{2.558555in}}%
\pgfpathlineto{\pgfqpoint{4.711280in}{2.565633in}}%
\pgfpathclose%
\pgfusepath{fill}%
\end{pgfscope}%
\begin{pgfscope}%
\pgfpathrectangle{\pgfqpoint{1.150000in}{0.150000in}}{\pgfqpoint{5.700000in}{5.700000in}}%
\pgfusepath{clip}%
\pgfsetbuttcap%
\pgfsetroundjoin%
\definecolor{currentfill}{rgb}{0.276022,0.044167,0.370164}%
\pgfsetfillcolor{currentfill}%
\pgfsetfillopacity{0.700000}%
\pgfsetlinewidth{0.000000pt}%
\definecolor{currentstroke}{rgb}{0.000000,0.000000,0.000000}%
\pgfsetstrokecolor{currentstroke}%
\pgfsetdash{}{0pt}%
\pgfpathmoveto{\pgfqpoint{2.996570in}{2.554944in}}%
\pgfpathlineto{\pgfqpoint{3.009745in}{2.548774in}}%
\pgfpathlineto{\pgfqpoint{3.022924in}{2.542650in}}%
\pgfpathlineto{\pgfqpoint{3.036106in}{2.536571in}}%
\pgfpathlineto{\pgfqpoint{3.049292in}{2.530537in}}%
\pgfpathlineto{\pgfqpoint{3.041161in}{2.524773in}}%
\pgfpathlineto{\pgfqpoint{3.033021in}{2.519089in}}%
\pgfpathlineto{\pgfqpoint{3.024873in}{2.513487in}}%
\pgfpathlineto{\pgfqpoint{3.016717in}{2.507970in}}%
\pgfpathlineto{\pgfqpoint{3.003514in}{2.514125in}}%
\pgfpathlineto{\pgfqpoint{2.990314in}{2.520326in}}%
\pgfpathlineto{\pgfqpoint{2.977118in}{2.526572in}}%
\pgfpathlineto{\pgfqpoint{2.963925in}{2.532864in}}%
\pgfpathlineto{\pgfqpoint{2.972099in}{2.538254in}}%
\pgfpathlineto{\pgfqpoint{2.980264in}{2.543733in}}%
\pgfpathlineto{\pgfqpoint{2.988422in}{2.549297in}}%
\pgfpathlineto{\pgfqpoint{2.996570in}{2.554944in}}%
\pgfpathclose%
\pgfusepath{fill}%
\end{pgfscope}%
\begin{pgfscope}%
\pgfpathrectangle{\pgfqpoint{1.150000in}{0.150000in}}{\pgfqpoint{5.700000in}{5.700000in}}%
\pgfusepath{clip}%
\pgfsetbuttcap%
\pgfsetroundjoin%
\definecolor{currentfill}{rgb}{0.268510,0.009605,0.335427}%
\pgfsetfillcolor{currentfill}%
\pgfsetfillopacity{0.700000}%
\pgfsetlinewidth{0.000000pt}%
\definecolor{currentstroke}{rgb}{0.000000,0.000000,0.000000}%
\pgfsetstrokecolor{currentstroke}%
\pgfsetdash{}{0pt}%
\pgfpathmoveto{\pgfqpoint{3.547580in}{2.489512in}}%
\pgfpathlineto{\pgfqpoint{3.560835in}{2.484988in}}%
\pgfpathlineto{\pgfqpoint{3.574096in}{2.480499in}}%
\pgfpathlineto{\pgfqpoint{3.587361in}{2.476046in}}%
\pgfpathlineto{\pgfqpoint{3.600632in}{2.471628in}}%
\pgfpathlineto{\pgfqpoint{3.592732in}{2.464379in}}%
\pgfpathlineto{\pgfqpoint{3.584826in}{2.457152in}}%
\pgfpathlineto{\pgfqpoint{3.576914in}{2.449948in}}%
\pgfpathlineto{\pgfqpoint{3.568995in}{2.442769in}}%
\pgfpathlineto{\pgfqpoint{3.555711in}{2.447256in}}%
\pgfpathlineto{\pgfqpoint{3.542431in}{2.451777in}}%
\pgfpathlineto{\pgfqpoint{3.529157in}{2.456334in}}%
\pgfpathlineto{\pgfqpoint{3.515888in}{2.460927in}}%
\pgfpathlineto{\pgfqpoint{3.523820in}{2.468033in}}%
\pgfpathlineto{\pgfqpoint{3.531746in}{2.475167in}}%
\pgfpathlineto{\pgfqpoint{3.539666in}{2.482327in}}%
\pgfpathlineto{\pgfqpoint{3.547580in}{2.489512in}}%
\pgfpathclose%
\pgfusepath{fill}%
\end{pgfscope}%
\begin{pgfscope}%
\pgfpathrectangle{\pgfqpoint{1.150000in}{0.150000in}}{\pgfqpoint{5.700000in}{5.700000in}}%
\pgfusepath{clip}%
\pgfsetbuttcap%
\pgfsetroundjoin%
\definecolor{currentfill}{rgb}{0.274952,0.037752,0.364543}%
\pgfsetfillcolor{currentfill}%
\pgfsetfillopacity{0.700000}%
\pgfsetlinewidth{0.000000pt}%
\definecolor{currentstroke}{rgb}{0.000000,0.000000,0.000000}%
\pgfsetstrokecolor{currentstroke}%
\pgfsetdash{}{0pt}%
\pgfpathmoveto{\pgfqpoint{4.489271in}{2.541063in}}%
\pgfpathlineto{\pgfqpoint{4.502734in}{2.538220in}}%
\pgfpathlineto{\pgfqpoint{4.516204in}{2.535406in}}%
\pgfpathlineto{\pgfqpoint{4.529681in}{2.532618in}}%
\pgfpathlineto{\pgfqpoint{4.543164in}{2.529858in}}%
\pgfpathlineto{\pgfqpoint{4.535612in}{2.522597in}}%
\pgfpathlineto{\pgfqpoint{4.528053in}{2.515321in}}%
\pgfpathlineto{\pgfqpoint{4.520489in}{2.508029in}}%
\pgfpathlineto{\pgfqpoint{4.512920in}{2.500720in}}%
\pgfpathlineto{\pgfqpoint{4.499423in}{2.503443in}}%
\pgfpathlineto{\pgfqpoint{4.485934in}{2.506193in}}%
\pgfpathlineto{\pgfqpoint{4.472451in}{2.508971in}}%
\pgfpathlineto{\pgfqpoint{4.458975in}{2.511776in}}%
\pgfpathlineto{\pgfqpoint{4.466557in}{2.519117in}}%
\pgfpathlineto{\pgfqpoint{4.474134in}{2.526445in}}%
\pgfpathlineto{\pgfqpoint{4.481705in}{2.533759in}}%
\pgfpathlineto{\pgfqpoint{4.489271in}{2.541063in}}%
\pgfpathclose%
\pgfusepath{fill}%
\end{pgfscope}%
\begin{pgfscope}%
\pgfpathrectangle{\pgfqpoint{1.150000in}{0.150000in}}{\pgfqpoint{5.700000in}{5.700000in}}%
\pgfusepath{clip}%
\pgfsetbuttcap%
\pgfsetroundjoin%
\definecolor{currentfill}{rgb}{0.272594,0.025563,0.353093}%
\pgfsetfillcolor{currentfill}%
\pgfsetfillopacity{0.700000}%
\pgfsetlinewidth{0.000000pt}%
\definecolor{currentstroke}{rgb}{0.000000,0.000000,0.000000}%
\pgfsetstrokecolor{currentstroke}%
\pgfsetdash{}{0pt}%
\pgfpathmoveto{\pgfqpoint{4.267233in}{2.517795in}}%
\pgfpathlineto{\pgfqpoint{4.280643in}{2.514676in}}%
\pgfpathlineto{\pgfqpoint{4.294060in}{2.511586in}}%
\pgfpathlineto{\pgfqpoint{4.307484in}{2.508524in}}%
\pgfpathlineto{\pgfqpoint{4.320914in}{2.505491in}}%
\pgfpathlineto{\pgfqpoint{4.313278in}{2.498027in}}%
\pgfpathlineto{\pgfqpoint{4.305636in}{2.490548in}}%
\pgfpathlineto{\pgfqpoint{4.297988in}{2.483053in}}%
\pgfpathlineto{\pgfqpoint{4.290335in}{2.475542in}}%
\pgfpathlineto{\pgfqpoint{4.276893in}{2.478564in}}%
\pgfpathlineto{\pgfqpoint{4.263457in}{2.481615in}}%
\pgfpathlineto{\pgfqpoint{4.250027in}{2.484695in}}%
\pgfpathlineto{\pgfqpoint{4.236604in}{2.487803in}}%
\pgfpathlineto{\pgfqpoint{4.244270in}{2.495320in}}%
\pgfpathlineto{\pgfqpoint{4.251930in}{2.502824in}}%
\pgfpathlineto{\pgfqpoint{4.259584in}{2.510316in}}%
\pgfpathlineto{\pgfqpoint{4.267233in}{2.517795in}}%
\pgfpathclose%
\pgfusepath{fill}%
\end{pgfscope}%
\begin{pgfscope}%
\pgfpathrectangle{\pgfqpoint{1.150000in}{0.150000in}}{\pgfqpoint{5.700000in}{5.700000in}}%
\pgfusepath{clip}%
\pgfsetbuttcap%
\pgfsetroundjoin%
\definecolor{currentfill}{rgb}{0.268510,0.009605,0.335427}%
\pgfsetfillcolor{currentfill}%
\pgfsetfillopacity{0.700000}%
\pgfsetlinewidth{0.000000pt}%
\definecolor{currentstroke}{rgb}{0.000000,0.000000,0.000000}%
\pgfsetstrokecolor{currentstroke}%
\pgfsetdash{}{0pt}%
\pgfpathmoveto{\pgfqpoint{3.685256in}{2.483711in}}%
\pgfpathlineto{\pgfqpoint{3.698540in}{2.479520in}}%
\pgfpathlineto{\pgfqpoint{3.711830in}{2.475362in}}%
\pgfpathlineto{\pgfqpoint{3.725125in}{2.471238in}}%
\pgfpathlineto{\pgfqpoint{3.738425in}{2.467148in}}%
\pgfpathlineto{\pgfqpoint{3.730576in}{2.459722in}}%
\pgfpathlineto{\pgfqpoint{3.722720in}{2.452308in}}%
\pgfpathlineto{\pgfqpoint{3.714859in}{2.444906in}}%
\pgfpathlineto{\pgfqpoint{3.706991in}{2.437516in}}%
\pgfpathlineto{\pgfqpoint{3.693677in}{2.441662in}}%
\pgfpathlineto{\pgfqpoint{3.680369in}{2.445841in}}%
\pgfpathlineto{\pgfqpoint{3.667066in}{2.450054in}}%
\pgfpathlineto{\pgfqpoint{3.653769in}{2.454301in}}%
\pgfpathlineto{\pgfqpoint{3.661650in}{2.461630in}}%
\pgfpathlineto{\pgfqpoint{3.669524in}{2.468975in}}%
\pgfpathlineto{\pgfqpoint{3.677393in}{2.476336in}}%
\pgfpathlineto{\pgfqpoint{3.685256in}{2.483711in}}%
\pgfpathclose%
\pgfusepath{fill}%
\end{pgfscope}%
\begin{pgfscope}%
\pgfpathrectangle{\pgfqpoint{1.150000in}{0.150000in}}{\pgfqpoint{5.700000in}{5.700000in}}%
\pgfusepath{clip}%
\pgfsetbuttcap%
\pgfsetroundjoin%
\definecolor{currentfill}{rgb}{0.282910,0.105393,0.426902}%
\pgfsetfillcolor{currentfill}%
\pgfsetfillopacity{0.700000}%
\pgfsetlinewidth{0.000000pt}%
\definecolor{currentstroke}{rgb}{0.000000,0.000000,0.000000}%
\pgfsetstrokecolor{currentstroke}%
\pgfsetdash{}{0pt}%
\pgfpathmoveto{\pgfqpoint{5.515465in}{2.646825in}}%
\pgfpathlineto{\pgfqpoint{5.529188in}{2.644488in}}%
\pgfpathlineto{\pgfqpoint{5.542918in}{2.642176in}}%
\pgfpathlineto{\pgfqpoint{5.556656in}{2.639887in}}%
\pgfpathlineto{\pgfqpoint{5.570402in}{2.637623in}}%
\pgfpathlineto{\pgfqpoint{5.563252in}{2.631286in}}%
\pgfpathlineto{\pgfqpoint{5.556098in}{2.625008in}}%
\pgfpathlineto{\pgfqpoint{5.548941in}{2.618785in}}%
\pgfpathlineto{\pgfqpoint{5.541779in}{2.612610in}}%
\pgfpathlineto{\pgfqpoint{5.528015in}{2.614718in}}%
\pgfpathlineto{\pgfqpoint{5.514258in}{2.616851in}}%
\pgfpathlineto{\pgfqpoint{5.500509in}{2.619007in}}%
\pgfpathlineto{\pgfqpoint{5.486768in}{2.621188in}}%
\pgfpathlineto{\pgfqpoint{5.493948in}{2.627514in}}%
\pgfpathlineto{\pgfqpoint{5.501124in}{2.633892in}}%
\pgfpathlineto{\pgfqpoint{5.508296in}{2.640328in}}%
\pgfpathlineto{\pgfqpoint{5.515465in}{2.646825in}}%
\pgfpathclose%
\pgfusepath{fill}%
\end{pgfscope}%
\begin{pgfscope}%
\pgfpathrectangle{\pgfqpoint{1.150000in}{0.150000in}}{\pgfqpoint{5.700000in}{5.700000in}}%
\pgfusepath{clip}%
\pgfsetbuttcap%
\pgfsetroundjoin%
\definecolor{currentfill}{rgb}{0.279566,0.067836,0.391917}%
\pgfsetfillcolor{currentfill}%
\pgfsetfillopacity{0.700000}%
\pgfsetlinewidth{0.000000pt}%
\definecolor{currentstroke}{rgb}{0.000000,0.000000,0.000000}%
\pgfsetstrokecolor{currentstroke}%
\pgfsetdash{}{0pt}%
\pgfpathmoveto{\pgfqpoint{2.858507in}{2.584913in}}%
\pgfpathlineto{\pgfqpoint{2.871672in}{2.578236in}}%
\pgfpathlineto{\pgfqpoint{2.884841in}{2.571609in}}%
\pgfpathlineto{\pgfqpoint{2.898014in}{2.565031in}}%
\pgfpathlineto{\pgfqpoint{2.911189in}{2.558502in}}%
\pgfpathlineto{\pgfqpoint{2.902988in}{2.553334in}}%
\pgfpathlineto{\pgfqpoint{2.894778in}{2.548264in}}%
\pgfpathlineto{\pgfqpoint{2.886559in}{2.543295in}}%
\pgfpathlineto{\pgfqpoint{2.878330in}{2.538431in}}%
\pgfpathlineto{\pgfqpoint{2.865136in}{2.545095in}}%
\pgfpathlineto{\pgfqpoint{2.851944in}{2.551808in}}%
\pgfpathlineto{\pgfqpoint{2.838756in}{2.558571in}}%
\pgfpathlineto{\pgfqpoint{2.825571in}{2.565383in}}%
\pgfpathlineto{\pgfqpoint{2.833819in}{2.570108in}}%
\pgfpathlineto{\pgfqpoint{2.842058in}{2.574940in}}%
\pgfpathlineto{\pgfqpoint{2.850287in}{2.579876in}}%
\pgfpathlineto{\pgfqpoint{2.858507in}{2.584913in}}%
\pgfpathclose%
\pgfusepath{fill}%
\end{pgfscope}%
\begin{pgfscope}%
\pgfpathrectangle{\pgfqpoint{1.150000in}{0.150000in}}{\pgfqpoint{5.700000in}{5.700000in}}%
\pgfusepath{clip}%
\pgfsetbuttcap%
\pgfsetroundjoin%
\definecolor{currentfill}{rgb}{0.269944,0.014625,0.341379}%
\pgfsetfillcolor{currentfill}%
\pgfsetfillopacity{0.700000}%
\pgfsetlinewidth{0.000000pt}%
\definecolor{currentstroke}{rgb}{0.000000,0.000000,0.000000}%
\pgfsetstrokecolor{currentstroke}%
\pgfsetdash{}{0pt}%
\pgfpathmoveto{\pgfqpoint{4.045147in}{2.497204in}}%
\pgfpathlineto{\pgfqpoint{4.058508in}{2.493739in}}%
\pgfpathlineto{\pgfqpoint{4.071876in}{2.490304in}}%
\pgfpathlineto{\pgfqpoint{4.085249in}{2.486899in}}%
\pgfpathlineto{\pgfqpoint{4.098628in}{2.483524in}}%
\pgfpathlineto{\pgfqpoint{4.090910in}{2.475953in}}%
\pgfpathlineto{\pgfqpoint{4.083186in}{2.468372in}}%
\pgfpathlineto{\pgfqpoint{4.075456in}{2.460782in}}%
\pgfpathlineto{\pgfqpoint{4.067721in}{2.453182in}}%
\pgfpathlineto{\pgfqpoint{4.054329in}{2.456573in}}%
\pgfpathlineto{\pgfqpoint{4.040943in}{2.459993in}}%
\pgfpathlineto{\pgfqpoint{4.027564in}{2.463444in}}%
\pgfpathlineto{\pgfqpoint{4.014190in}{2.466925in}}%
\pgfpathlineto{\pgfqpoint{4.021938in}{2.474504in}}%
\pgfpathlineto{\pgfqpoint{4.029680in}{2.482077in}}%
\pgfpathlineto{\pgfqpoint{4.037417in}{2.489644in}}%
\pgfpathlineto{\pgfqpoint{4.045147in}{2.497204in}}%
\pgfpathclose%
\pgfusepath{fill}%
\end{pgfscope}%
\begin{pgfscope}%
\pgfpathrectangle{\pgfqpoint{1.150000in}{0.150000in}}{\pgfqpoint{5.700000in}{5.700000in}}%
\pgfusepath{clip}%
\pgfsetbuttcap%
\pgfsetroundjoin%
\definecolor{currentfill}{rgb}{0.281924,0.089666,0.412415}%
\pgfsetfillcolor{currentfill}%
\pgfsetfillopacity{0.700000}%
\pgfsetlinewidth{0.000000pt}%
\definecolor{currentstroke}{rgb}{0.000000,0.000000,0.000000}%
\pgfsetstrokecolor{currentstroke}%
\pgfsetdash{}{0pt}%
\pgfpathmoveto{\pgfqpoint{5.293478in}{2.622670in}}%
\pgfpathlineto{\pgfqpoint{5.307147in}{2.620338in}}%
\pgfpathlineto{\pgfqpoint{5.320824in}{2.618032in}}%
\pgfpathlineto{\pgfqpoint{5.334508in}{2.615750in}}%
\pgfpathlineto{\pgfqpoint{5.348200in}{2.613492in}}%
\pgfpathlineto{\pgfqpoint{5.340963in}{2.607089in}}%
\pgfpathlineto{\pgfqpoint{5.333720in}{2.600717in}}%
\pgfpathlineto{\pgfqpoint{5.326473in}{2.594374in}}%
\pgfpathlineto{\pgfqpoint{5.319221in}{2.588054in}}%
\pgfpathlineto{\pgfqpoint{5.305512in}{2.590181in}}%
\pgfpathlineto{\pgfqpoint{5.291811in}{2.592333in}}%
\pgfpathlineto{\pgfqpoint{5.278117in}{2.594510in}}%
\pgfpathlineto{\pgfqpoint{5.264431in}{2.596711in}}%
\pgfpathlineto{\pgfqpoint{5.271700in}{2.603156in}}%
\pgfpathlineto{\pgfqpoint{5.278964in}{2.609629in}}%
\pgfpathlineto{\pgfqpoint{5.286223in}{2.616132in}}%
\pgfpathlineto{\pgfqpoint{5.293478in}{2.622670in}}%
\pgfpathclose%
\pgfusepath{fill}%
\end{pgfscope}%
\begin{pgfscope}%
\pgfpathrectangle{\pgfqpoint{1.150000in}{0.150000in}}{\pgfqpoint{5.700000in}{5.700000in}}%
\pgfusepath{clip}%
\pgfsetbuttcap%
\pgfsetroundjoin%
\definecolor{currentfill}{rgb}{0.280894,0.078907,0.402329}%
\pgfsetfillcolor{currentfill}%
\pgfsetfillopacity{0.700000}%
\pgfsetlinewidth{0.000000pt}%
\definecolor{currentstroke}{rgb}{0.000000,0.000000,0.000000}%
\pgfsetstrokecolor{currentstroke}%
\pgfsetdash{}{0pt}%
\pgfpathmoveto{\pgfqpoint{5.071445in}{2.598080in}}%
\pgfpathlineto{\pgfqpoint{5.085060in}{2.595695in}}%
\pgfpathlineto{\pgfqpoint{5.098682in}{2.593336in}}%
\pgfpathlineto{\pgfqpoint{5.112311in}{2.591001in}}%
\pgfpathlineto{\pgfqpoint{5.125947in}{2.588692in}}%
\pgfpathlineto{\pgfqpoint{5.118620in}{2.582106in}}%
\pgfpathlineto{\pgfqpoint{5.111288in}{2.575530in}}%
\pgfpathlineto{\pgfqpoint{5.103950in}{2.568962in}}%
\pgfpathlineto{\pgfqpoint{5.096607in}{2.562398in}}%
\pgfpathlineto{\pgfqpoint{5.082955in}{2.564603in}}%
\pgfpathlineto{\pgfqpoint{5.069310in}{2.566834in}}%
\pgfpathlineto{\pgfqpoint{5.055672in}{2.569090in}}%
\pgfpathlineto{\pgfqpoint{5.042042in}{2.571371in}}%
\pgfpathlineto{\pgfqpoint{5.049401in}{2.578034in}}%
\pgfpathlineto{\pgfqpoint{5.056754in}{2.584704in}}%
\pgfpathlineto{\pgfqpoint{5.064102in}{2.591385in}}%
\pgfpathlineto{\pgfqpoint{5.071445in}{2.598080in}}%
\pgfpathclose%
\pgfusepath{fill}%
\end{pgfscope}%
\begin{pgfscope}%
\pgfpathrectangle{\pgfqpoint{1.150000in}{0.150000in}}{\pgfqpoint{5.700000in}{5.700000in}}%
\pgfusepath{clip}%
\pgfsetbuttcap%
\pgfsetroundjoin%
\definecolor{currentfill}{rgb}{0.279566,0.067836,0.391917}%
\pgfsetfillcolor{currentfill}%
\pgfsetfillopacity{0.700000}%
\pgfsetlinewidth{0.000000pt}%
\definecolor{currentstroke}{rgb}{0.000000,0.000000,0.000000}%
\pgfsetstrokecolor{currentstroke}%
\pgfsetdash{}{0pt}%
\pgfpathmoveto{\pgfqpoint{4.849371in}{2.573012in}}%
\pgfpathlineto{\pgfqpoint{4.862930in}{2.570513in}}%
\pgfpathlineto{\pgfqpoint{4.876497in}{2.568041in}}%
\pgfpathlineto{\pgfqpoint{4.890070in}{2.565594in}}%
\pgfpathlineto{\pgfqpoint{4.903650in}{2.563172in}}%
\pgfpathlineto{\pgfqpoint{4.896234in}{2.556334in}}%
\pgfpathlineto{\pgfqpoint{4.888813in}{2.549492in}}%
\pgfpathlineto{\pgfqpoint{4.881385in}{2.542642in}}%
\pgfpathlineto{\pgfqpoint{4.873952in}{2.535783in}}%
\pgfpathlineto{\pgfqpoint{4.860357in}{2.538127in}}%
\pgfpathlineto{\pgfqpoint{4.846770in}{2.540497in}}%
\pgfpathlineto{\pgfqpoint{4.833189in}{2.542893in}}%
\pgfpathlineto{\pgfqpoint{4.819616in}{2.545315in}}%
\pgfpathlineto{\pgfqpoint{4.827063in}{2.552246in}}%
\pgfpathlineto{\pgfqpoint{4.834505in}{2.559171in}}%
\pgfpathlineto{\pgfqpoint{4.841941in}{2.566092in}}%
\pgfpathlineto{\pgfqpoint{4.849371in}{2.573012in}}%
\pgfpathclose%
\pgfusepath{fill}%
\end{pgfscope}%
\begin{pgfscope}%
\pgfpathrectangle{\pgfqpoint{1.150000in}{0.150000in}}{\pgfqpoint{5.700000in}{5.700000in}}%
\pgfusepath{clip}%
\pgfsetbuttcap%
\pgfsetroundjoin%
\definecolor{currentfill}{rgb}{0.268510,0.009605,0.335427}%
\pgfsetfillcolor{currentfill}%
\pgfsetfillopacity{0.700000}%
\pgfsetlinewidth{0.000000pt}%
\definecolor{currentstroke}{rgb}{0.000000,0.000000,0.000000}%
\pgfsetstrokecolor{currentstroke}%
\pgfsetdash{}{0pt}%
\pgfpathmoveto{\pgfqpoint{3.822973in}{2.481077in}}%
\pgfpathlineto{\pgfqpoint{3.836289in}{2.477192in}}%
\pgfpathlineto{\pgfqpoint{3.849610in}{2.473339in}}%
\pgfpathlineto{\pgfqpoint{3.862937in}{2.469518in}}%
\pgfpathlineto{\pgfqpoint{3.876270in}{2.465729in}}%
\pgfpathlineto{\pgfqpoint{3.868469in}{2.458194in}}%
\pgfpathlineto{\pgfqpoint{3.860662in}{2.450659in}}%
\pgfpathlineto{\pgfqpoint{3.852850in}{2.443127in}}%
\pgfpathlineto{\pgfqpoint{3.845031in}{2.435598in}}%
\pgfpathlineto{\pgfqpoint{3.831686in}{2.439429in}}%
\pgfpathlineto{\pgfqpoint{3.818346in}{2.443292in}}%
\pgfpathlineto{\pgfqpoint{3.805012in}{2.447187in}}%
\pgfpathlineto{\pgfqpoint{3.791683in}{2.451114in}}%
\pgfpathlineto{\pgfqpoint{3.799514in}{2.458596in}}%
\pgfpathlineto{\pgfqpoint{3.807340in}{2.466085in}}%
\pgfpathlineto{\pgfqpoint{3.815159in}{2.473578in}}%
\pgfpathlineto{\pgfqpoint{3.822973in}{2.481077in}}%
\pgfpathclose%
\pgfusepath{fill}%
\end{pgfscope}%
\begin{pgfscope}%
\pgfpathrectangle{\pgfqpoint{1.150000in}{0.150000in}}{\pgfqpoint{5.700000in}{5.700000in}}%
\pgfusepath{clip}%
\pgfsetbuttcap%
\pgfsetroundjoin%
\definecolor{currentfill}{rgb}{0.277018,0.050344,0.375715}%
\pgfsetfillcolor{currentfill}%
\pgfsetfillopacity{0.700000}%
\pgfsetlinewidth{0.000000pt}%
\definecolor{currentstroke}{rgb}{0.000000,0.000000,0.000000}%
\pgfsetstrokecolor{currentstroke}%
\pgfsetdash{}{0pt}%
\pgfpathmoveto{\pgfqpoint{4.627266in}{2.547824in}}%
\pgfpathlineto{\pgfqpoint{4.640770in}{2.545148in}}%
\pgfpathlineto{\pgfqpoint{4.654281in}{2.542500in}}%
\pgfpathlineto{\pgfqpoint{4.667798in}{2.539878in}}%
\pgfpathlineto{\pgfqpoint{4.681323in}{2.537283in}}%
\pgfpathlineto{\pgfqpoint{4.673820in}{2.530173in}}%
\pgfpathlineto{\pgfqpoint{4.666311in}{2.523048in}}%
\pgfpathlineto{\pgfqpoint{4.658796in}{2.515908in}}%
\pgfpathlineto{\pgfqpoint{4.651275in}{2.508751in}}%
\pgfpathlineto{\pgfqpoint{4.637737in}{2.511295in}}%
\pgfpathlineto{\pgfqpoint{4.624206in}{2.513866in}}%
\pgfpathlineto{\pgfqpoint{4.610682in}{2.516464in}}%
\pgfpathlineto{\pgfqpoint{4.597165in}{2.519089in}}%
\pgfpathlineto{\pgfqpoint{4.604699in}{2.526292in}}%
\pgfpathlineto{\pgfqpoint{4.612227in}{2.533481in}}%
\pgfpathlineto{\pgfqpoint{4.619749in}{2.540657in}}%
\pgfpathlineto{\pgfqpoint{4.627266in}{2.547824in}}%
\pgfpathclose%
\pgfusepath{fill}%
\end{pgfscope}%
\begin{pgfscope}%
\pgfpathrectangle{\pgfqpoint{1.150000in}{0.150000in}}{\pgfqpoint{5.700000in}{5.700000in}}%
\pgfusepath{clip}%
\pgfsetbuttcap%
\pgfsetroundjoin%
\definecolor{currentfill}{rgb}{0.274952,0.037752,0.364543}%
\pgfsetfillcolor{currentfill}%
\pgfsetfillopacity{0.700000}%
\pgfsetlinewidth{0.000000pt}%
\definecolor{currentstroke}{rgb}{0.000000,0.000000,0.000000}%
\pgfsetstrokecolor{currentstroke}%
\pgfsetdash{}{0pt}%
\pgfpathmoveto{\pgfqpoint{4.405136in}{2.523274in}}%
\pgfpathlineto{\pgfqpoint{4.418586in}{2.520357in}}%
\pgfpathlineto{\pgfqpoint{4.432042in}{2.517469in}}%
\pgfpathlineto{\pgfqpoint{4.445505in}{2.514609in}}%
\pgfpathlineto{\pgfqpoint{4.458975in}{2.511776in}}%
\pgfpathlineto{\pgfqpoint{4.451386in}{2.504419in}}%
\pgfpathlineto{\pgfqpoint{4.443793in}{2.497044in}}%
\pgfpathlineto{\pgfqpoint{4.436193in}{2.489652in}}%
\pgfpathlineto{\pgfqpoint{4.428588in}{2.482240in}}%
\pgfpathlineto{\pgfqpoint{4.415105in}{2.485049in}}%
\pgfpathlineto{\pgfqpoint{4.401630in}{2.487885in}}%
\pgfpathlineto{\pgfqpoint{4.388161in}{2.490749in}}%
\pgfpathlineto{\pgfqpoint{4.374698in}{2.493641in}}%
\pgfpathlineto{\pgfqpoint{4.382316in}{2.501072in}}%
\pgfpathlineto{\pgfqpoint{4.389928in}{2.508487in}}%
\pgfpathlineto{\pgfqpoint{4.397535in}{2.515887in}}%
\pgfpathlineto{\pgfqpoint{4.405136in}{2.523274in}}%
\pgfpathclose%
\pgfusepath{fill}%
\end{pgfscope}%
\begin{pgfscope}%
\pgfpathrectangle{\pgfqpoint{1.150000in}{0.150000in}}{\pgfqpoint{5.700000in}{5.700000in}}%
\pgfusepath{clip}%
\pgfsetbuttcap%
\pgfsetroundjoin%
\definecolor{currentfill}{rgb}{0.269944,0.014625,0.341379}%
\pgfsetfillcolor{currentfill}%
\pgfsetfillopacity{0.700000}%
\pgfsetlinewidth{0.000000pt}%
\definecolor{currentstroke}{rgb}{0.000000,0.000000,0.000000}%
\pgfsetstrokecolor{currentstroke}%
\pgfsetdash{}{0pt}%
\pgfpathmoveto{\pgfqpoint{3.325076in}{2.491761in}}%
\pgfpathlineto{\pgfqpoint{3.338302in}{2.486641in}}%
\pgfpathlineto{\pgfqpoint{3.351532in}{2.481559in}}%
\pgfpathlineto{\pgfqpoint{3.364767in}{2.476516in}}%
\pgfpathlineto{\pgfqpoint{3.378006in}{2.471511in}}%
\pgfpathlineto{\pgfqpoint{3.370013in}{2.464747in}}%
\pgfpathlineto{\pgfqpoint{3.362013in}{2.458028in}}%
\pgfpathlineto{\pgfqpoint{3.354006in}{2.451355in}}%
\pgfpathlineto{\pgfqpoint{3.345993in}{2.444730in}}%
\pgfpathlineto{\pgfqpoint{3.332738in}{2.449830in}}%
\pgfpathlineto{\pgfqpoint{3.319488in}{2.454968in}}%
\pgfpathlineto{\pgfqpoint{3.306243in}{2.460145in}}%
\pgfpathlineto{\pgfqpoint{3.293002in}{2.465360in}}%
\pgfpathlineto{\pgfqpoint{3.301031in}{2.471885in}}%
\pgfpathlineto{\pgfqpoint{3.309053in}{2.478461in}}%
\pgfpathlineto{\pgfqpoint{3.317068in}{2.485087in}}%
\pgfpathlineto{\pgfqpoint{3.325076in}{2.491761in}}%
\pgfpathclose%
\pgfusepath{fill}%
\end{pgfscope}%
\begin{pgfscope}%
\pgfpathrectangle{\pgfqpoint{1.150000in}{0.150000in}}{\pgfqpoint{5.700000in}{5.700000in}}%
\pgfusepath{clip}%
\pgfsetbuttcap%
\pgfsetroundjoin%
\definecolor{currentfill}{rgb}{0.272594,0.025563,0.353093}%
\pgfsetfillcolor{currentfill}%
\pgfsetfillopacity{0.700000}%
\pgfsetlinewidth{0.000000pt}%
\definecolor{currentstroke}{rgb}{0.000000,0.000000,0.000000}%
\pgfsetstrokecolor{currentstroke}%
\pgfsetdash{}{0pt}%
\pgfpathmoveto{\pgfqpoint{3.187236in}{2.508515in}}%
\pgfpathlineto{\pgfqpoint{3.200442in}{2.502978in}}%
\pgfpathlineto{\pgfqpoint{3.213652in}{2.497483in}}%
\pgfpathlineto{\pgfqpoint{3.226866in}{2.492029in}}%
\pgfpathlineto{\pgfqpoint{3.240084in}{2.486615in}}%
\pgfpathlineto{\pgfqpoint{3.232033in}{2.480248in}}%
\pgfpathlineto{\pgfqpoint{3.223974in}{2.473940in}}%
\pgfpathlineto{\pgfqpoint{3.215908in}{2.467695in}}%
\pgfpathlineto{\pgfqpoint{3.207834in}{2.461514in}}%
\pgfpathlineto{\pgfqpoint{3.194599in}{2.467035in}}%
\pgfpathlineto{\pgfqpoint{3.181369in}{2.472598in}}%
\pgfpathlineto{\pgfqpoint{3.168143in}{2.478202in}}%
\pgfpathlineto{\pgfqpoint{3.154921in}{2.483846in}}%
\pgfpathlineto{\pgfqpoint{3.163011in}{2.489914in}}%
\pgfpathlineto{\pgfqpoint{3.171094in}{2.496050in}}%
\pgfpathlineto{\pgfqpoint{3.179169in}{2.502251in}}%
\pgfpathlineto{\pgfqpoint{3.187236in}{2.508515in}}%
\pgfpathclose%
\pgfusepath{fill}%
\end{pgfscope}%
\begin{pgfscope}%
\pgfpathrectangle{\pgfqpoint{1.150000in}{0.150000in}}{\pgfqpoint{5.700000in}{5.700000in}}%
\pgfusepath{clip}%
\pgfsetbuttcap%
\pgfsetroundjoin%
\definecolor{currentfill}{rgb}{0.271305,0.019942,0.347269}%
\pgfsetfillcolor{currentfill}%
\pgfsetfillopacity{0.700000}%
\pgfsetlinewidth{0.000000pt}%
\definecolor{currentstroke}{rgb}{0.000000,0.000000,0.000000}%
\pgfsetstrokecolor{currentstroke}%
\pgfsetdash{}{0pt}%
\pgfpathmoveto{\pgfqpoint{4.182975in}{2.500526in}}%
\pgfpathlineto{\pgfqpoint{4.196373in}{2.497302in}}%
\pgfpathlineto{\pgfqpoint{4.209777in}{2.494106in}}%
\pgfpathlineto{\pgfqpoint{4.223187in}{2.490940in}}%
\pgfpathlineto{\pgfqpoint{4.236604in}{2.487803in}}%
\pgfpathlineto{\pgfqpoint{4.228933in}{2.480272in}}%
\pgfpathlineto{\pgfqpoint{4.221256in}{2.472726in}}%
\pgfpathlineto{\pgfqpoint{4.213574in}{2.465165in}}%
\pgfpathlineto{\pgfqpoint{4.205886in}{2.457589in}}%
\pgfpathlineto{\pgfqpoint{4.192457in}{2.460729in}}%
\pgfpathlineto{\pgfqpoint{4.179034in}{2.463897in}}%
\pgfpathlineto{\pgfqpoint{4.165617in}{2.467095in}}%
\pgfpathlineto{\pgfqpoint{4.152207in}{2.470322in}}%
\pgfpathlineto{\pgfqpoint{4.159907in}{2.477891in}}%
\pgfpathlineto{\pgfqpoint{4.167602in}{2.485447in}}%
\pgfpathlineto{\pgfqpoint{4.175291in}{2.492992in}}%
\pgfpathlineto{\pgfqpoint{4.182975in}{2.500526in}}%
\pgfpathclose%
\pgfusepath{fill}%
\end{pgfscope}%
\begin{pgfscope}%
\pgfpathrectangle{\pgfqpoint{1.150000in}{0.150000in}}{\pgfqpoint{5.700000in}{5.700000in}}%
\pgfusepath{clip}%
\pgfsetbuttcap%
\pgfsetroundjoin%
\definecolor{currentfill}{rgb}{0.268510,0.009605,0.335427}%
\pgfsetfillcolor{currentfill}%
\pgfsetfillopacity{0.700000}%
\pgfsetlinewidth{0.000000pt}%
\definecolor{currentstroke}{rgb}{0.000000,0.000000,0.000000}%
\pgfsetstrokecolor{currentstroke}%
\pgfsetdash{}{0pt}%
\pgfpathmoveto{\pgfqpoint{3.462861in}{2.479657in}}%
\pgfpathlineto{\pgfqpoint{3.476110in}{2.474920in}}%
\pgfpathlineto{\pgfqpoint{3.489365in}{2.470219in}}%
\pgfpathlineto{\pgfqpoint{3.502624in}{2.465555in}}%
\pgfpathlineto{\pgfqpoint{3.515888in}{2.460927in}}%
\pgfpathlineto{\pgfqpoint{3.507949in}{2.453850in}}%
\pgfpathlineto{\pgfqpoint{3.500004in}{2.446804in}}%
\pgfpathlineto{\pgfqpoint{3.492053in}{2.439790in}}%
\pgfpathlineto{\pgfqpoint{3.484096in}{2.432810in}}%
\pgfpathlineto{\pgfqpoint{3.470817in}{2.437520in}}%
\pgfpathlineto{\pgfqpoint{3.457544in}{2.442266in}}%
\pgfpathlineto{\pgfqpoint{3.444276in}{2.447048in}}%
\pgfpathlineto{\pgfqpoint{3.431012in}{2.451867in}}%
\pgfpathlineto{\pgfqpoint{3.438984in}{2.458760in}}%
\pgfpathlineto{\pgfqpoint{3.446950in}{2.465690in}}%
\pgfpathlineto{\pgfqpoint{3.454909in}{2.472656in}}%
\pgfpathlineto{\pgfqpoint{3.462861in}{2.479657in}}%
\pgfpathclose%
\pgfusepath{fill}%
\end{pgfscope}%
\begin{pgfscope}%
\pgfpathrectangle{\pgfqpoint{1.150000in}{0.150000in}}{\pgfqpoint{5.700000in}{5.700000in}}%
\pgfusepath{clip}%
\pgfsetbuttcap%
\pgfsetroundjoin%
\definecolor{currentfill}{rgb}{0.283091,0.110553,0.431554}%
\pgfsetfillcolor{currentfill}%
\pgfsetfillopacity{0.700000}%
\pgfsetlinewidth{0.000000pt}%
\definecolor{currentstroke}{rgb}{0.000000,0.000000,0.000000}%
\pgfsetstrokecolor{currentstroke}%
\pgfsetdash{}{0pt}%
\pgfpathmoveto{\pgfqpoint{5.653949in}{2.654165in}}%
\pgfpathlineto{\pgfqpoint{5.667713in}{2.651852in}}%
\pgfpathlineto{\pgfqpoint{5.681485in}{2.649562in}}%
\pgfpathlineto{\pgfqpoint{5.695264in}{2.647297in}}%
\pgfpathlineto{\pgfqpoint{5.709051in}{2.645055in}}%
\pgfpathlineto{\pgfqpoint{5.701953in}{2.638780in}}%
\pgfpathlineto{\pgfqpoint{5.694853in}{2.632578in}}%
\pgfpathlineto{\pgfqpoint{5.687748in}{2.626444in}}%
\pgfpathlineto{\pgfqpoint{5.680641in}{2.620373in}}%
\pgfpathlineto{\pgfqpoint{5.666834in}{2.622445in}}%
\pgfpathlineto{\pgfqpoint{5.653035in}{2.624541in}}%
\pgfpathlineto{\pgfqpoint{5.639244in}{2.626662in}}%
\pgfpathlineto{\pgfqpoint{5.625460in}{2.628806in}}%
\pgfpathlineto{\pgfqpoint{5.632587in}{2.635042in}}%
\pgfpathlineto{\pgfqpoint{5.639711in}{2.641344in}}%
\pgfpathlineto{\pgfqpoint{5.646831in}{2.647716in}}%
\pgfpathlineto{\pgfqpoint{5.653949in}{2.654165in}}%
\pgfpathclose%
\pgfusepath{fill}%
\end{pgfscope}%
\begin{pgfscope}%
\pgfpathrectangle{\pgfqpoint{1.150000in}{0.150000in}}{\pgfqpoint{5.700000in}{5.700000in}}%
\pgfusepath{clip}%
\pgfsetbuttcap%
\pgfsetroundjoin%
\definecolor{currentfill}{rgb}{0.274952,0.037752,0.364543}%
\pgfsetfillcolor{currentfill}%
\pgfsetfillopacity{0.700000}%
\pgfsetlinewidth{0.000000pt}%
\definecolor{currentstroke}{rgb}{0.000000,0.000000,0.000000}%
\pgfsetstrokecolor{currentstroke}%
\pgfsetdash{}{0pt}%
\pgfpathmoveto{\pgfqpoint{3.049292in}{2.530537in}}%
\pgfpathlineto{\pgfqpoint{3.062482in}{2.524548in}}%
\pgfpathlineto{\pgfqpoint{3.075676in}{2.518604in}}%
\pgfpathlineto{\pgfqpoint{3.088874in}{2.512703in}}%
\pgfpathlineto{\pgfqpoint{3.102075in}{2.506846in}}%
\pgfpathlineto{\pgfqpoint{3.093961in}{2.500966in}}%
\pgfpathlineto{\pgfqpoint{3.085838in}{2.495161in}}%
\pgfpathlineto{\pgfqpoint{3.077708in}{2.489436in}}%
\pgfpathlineto{\pgfqpoint{3.069569in}{2.483793in}}%
\pgfpathlineto{\pgfqpoint{3.056351in}{2.489771in}}%
\pgfpathlineto{\pgfqpoint{3.043136in}{2.495793in}}%
\pgfpathlineto{\pgfqpoint{3.029925in}{2.501860in}}%
\pgfpathlineto{\pgfqpoint{3.016717in}{2.507970in}}%
\pgfpathlineto{\pgfqpoint{3.024873in}{2.513487in}}%
\pgfpathlineto{\pgfqpoint{3.033021in}{2.519089in}}%
\pgfpathlineto{\pgfqpoint{3.041161in}{2.524773in}}%
\pgfpathlineto{\pgfqpoint{3.049292in}{2.530537in}}%
\pgfpathclose%
\pgfusepath{fill}%
\end{pgfscope}%
\begin{pgfscope}%
\pgfpathrectangle{\pgfqpoint{1.150000in}{0.150000in}}{\pgfqpoint{5.700000in}{5.700000in}}%
\pgfusepath{clip}%
\pgfsetbuttcap%
\pgfsetroundjoin%
\definecolor{currentfill}{rgb}{0.282656,0.100196,0.422160}%
\pgfsetfillcolor{currentfill}%
\pgfsetfillopacity{0.700000}%
\pgfsetlinewidth{0.000000pt}%
\definecolor{currentstroke}{rgb}{0.000000,0.000000,0.000000}%
\pgfsetstrokecolor{currentstroke}%
\pgfsetdash{}{0pt}%
\pgfpathmoveto{\pgfqpoint{5.431877in}{2.630154in}}%
\pgfpathlineto{\pgfqpoint{5.445589in}{2.627876in}}%
\pgfpathlineto{\pgfqpoint{5.459307in}{2.625622in}}%
\pgfpathlineto{\pgfqpoint{5.473034in}{2.623393in}}%
\pgfpathlineto{\pgfqpoint{5.486768in}{2.621188in}}%
\pgfpathlineto{\pgfqpoint{5.479584in}{2.614910in}}%
\pgfpathlineto{\pgfqpoint{5.472395in}{2.608674in}}%
\pgfpathlineto{\pgfqpoint{5.465202in}{2.602478in}}%
\pgfpathlineto{\pgfqpoint{5.458004in}{2.596315in}}%
\pgfpathlineto{\pgfqpoint{5.444252in}{2.598377in}}%
\pgfpathlineto{\pgfqpoint{5.430508in}{2.600463in}}%
\pgfpathlineto{\pgfqpoint{5.416771in}{2.602573in}}%
\pgfpathlineto{\pgfqpoint{5.403042in}{2.604708in}}%
\pgfpathlineto{\pgfqpoint{5.410257in}{2.611009in}}%
\pgfpathlineto{\pgfqpoint{5.417468in}{2.617347in}}%
\pgfpathlineto{\pgfqpoint{5.424675in}{2.623728in}}%
\pgfpathlineto{\pgfqpoint{5.431877in}{2.630154in}}%
\pgfpathclose%
\pgfusepath{fill}%
\end{pgfscope}%
\begin{pgfscope}%
\pgfpathrectangle{\pgfqpoint{1.150000in}{0.150000in}}{\pgfqpoint{5.700000in}{5.700000in}}%
\pgfusepath{clip}%
\pgfsetbuttcap%
\pgfsetroundjoin%
\definecolor{currentfill}{rgb}{0.269944,0.014625,0.341379}%
\pgfsetfillcolor{currentfill}%
\pgfsetfillopacity{0.700000}%
\pgfsetlinewidth{0.000000pt}%
\definecolor{currentstroke}{rgb}{0.000000,0.000000,0.000000}%
\pgfsetstrokecolor{currentstroke}%
\pgfsetdash{}{0pt}%
\pgfpathmoveto{\pgfqpoint{3.960757in}{2.481154in}}%
\pgfpathlineto{\pgfqpoint{3.974106in}{2.477551in}}%
\pgfpathlineto{\pgfqpoint{3.987462in}{2.473978in}}%
\pgfpathlineto{\pgfqpoint{4.000823in}{2.470436in}}%
\pgfpathlineto{\pgfqpoint{4.014190in}{2.466925in}}%
\pgfpathlineto{\pgfqpoint{4.006437in}{2.459339in}}%
\pgfpathlineto{\pgfqpoint{3.998678in}{2.451747in}}%
\pgfpathlineto{\pgfqpoint{3.990914in}{2.444148in}}%
\pgfpathlineto{\pgfqpoint{3.983143in}{2.436544in}}%
\pgfpathlineto{\pgfqpoint{3.969763in}{2.440084in}}%
\pgfpathlineto{\pgfqpoint{3.956389in}{2.443655in}}%
\pgfpathlineto{\pgfqpoint{3.943021in}{2.447256in}}%
\pgfpathlineto{\pgfqpoint{3.929659in}{2.450888in}}%
\pgfpathlineto{\pgfqpoint{3.937442in}{2.458459in}}%
\pgfpathlineto{\pgfqpoint{3.945219in}{2.466027in}}%
\pgfpathlineto{\pgfqpoint{3.952991in}{2.473592in}}%
\pgfpathlineto{\pgfqpoint{3.960757in}{2.481154in}}%
\pgfpathclose%
\pgfusepath{fill}%
\end{pgfscope}%
\begin{pgfscope}%
\pgfpathrectangle{\pgfqpoint{1.150000in}{0.150000in}}{\pgfqpoint{5.700000in}{5.700000in}}%
\pgfusepath{clip}%
\pgfsetbuttcap%
\pgfsetroundjoin%
\definecolor{currentfill}{rgb}{0.268510,0.009605,0.335427}%
\pgfsetfillcolor{currentfill}%
\pgfsetfillopacity{0.700000}%
\pgfsetlinewidth{0.000000pt}%
\definecolor{currentstroke}{rgb}{0.000000,0.000000,0.000000}%
\pgfsetstrokecolor{currentstroke}%
\pgfsetdash{}{0pt}%
\pgfpathmoveto{\pgfqpoint{3.600632in}{2.471628in}}%
\pgfpathlineto{\pgfqpoint{3.613909in}{2.467244in}}%
\pgfpathlineto{\pgfqpoint{3.627190in}{2.462895in}}%
\pgfpathlineto{\pgfqpoint{3.640477in}{2.458581in}}%
\pgfpathlineto{\pgfqpoint{3.653769in}{2.454301in}}%
\pgfpathlineto{\pgfqpoint{3.645882in}{2.446988in}}%
\pgfpathlineto{\pgfqpoint{3.637990in}{2.439695in}}%
\pgfpathlineto{\pgfqpoint{3.630091in}{2.432421in}}%
\pgfpathlineto{\pgfqpoint{3.622186in}{2.425168in}}%
\pgfpathlineto{\pgfqpoint{3.608881in}{2.429517in}}%
\pgfpathlineto{\pgfqpoint{3.595580in}{2.433900in}}%
\pgfpathlineto{\pgfqpoint{3.582285in}{2.438317in}}%
\pgfpathlineto{\pgfqpoint{3.568995in}{2.442769in}}%
\pgfpathlineto{\pgfqpoint{3.576914in}{2.449948in}}%
\pgfpathlineto{\pgfqpoint{3.584826in}{2.457152in}}%
\pgfpathlineto{\pgfqpoint{3.592732in}{2.464379in}}%
\pgfpathlineto{\pgfqpoint{3.600632in}{2.471628in}}%
\pgfpathclose%
\pgfusepath{fill}%
\end{pgfscope}%
\begin{pgfscope}%
\pgfpathrectangle{\pgfqpoint{1.150000in}{0.150000in}}{\pgfqpoint{5.700000in}{5.700000in}}%
\pgfusepath{clip}%
\pgfsetbuttcap%
\pgfsetroundjoin%
\definecolor{currentfill}{rgb}{0.281924,0.089666,0.412415}%
\pgfsetfillcolor{currentfill}%
\pgfsetfillopacity{0.700000}%
\pgfsetlinewidth{0.000000pt}%
\definecolor{currentstroke}{rgb}{0.000000,0.000000,0.000000}%
\pgfsetstrokecolor{currentstroke}%
\pgfsetdash{}{0pt}%
\pgfpathmoveto{\pgfqpoint{5.209759in}{2.605765in}}%
\pgfpathlineto{\pgfqpoint{5.223416in}{2.603464in}}%
\pgfpathlineto{\pgfqpoint{5.237080in}{2.601188in}}%
\pgfpathlineto{\pgfqpoint{5.250752in}{2.598937in}}%
\pgfpathlineto{\pgfqpoint{5.264431in}{2.596711in}}%
\pgfpathlineto{\pgfqpoint{5.257157in}{2.590289in}}%
\pgfpathlineto{\pgfqpoint{5.249878in}{2.583885in}}%
\pgfpathlineto{\pgfqpoint{5.242593in}{2.577497in}}%
\pgfpathlineto{\pgfqpoint{5.235304in}{2.571120in}}%
\pgfpathlineto{\pgfqpoint{5.221608in}{2.573229in}}%
\pgfpathlineto{\pgfqpoint{5.207920in}{2.575364in}}%
\pgfpathlineto{\pgfqpoint{5.194240in}{2.577522in}}%
\pgfpathlineto{\pgfqpoint{5.180566in}{2.579706in}}%
\pgfpathlineto{\pgfqpoint{5.187872in}{2.586195in}}%
\pgfpathlineto{\pgfqpoint{5.195173in}{2.592699in}}%
\pgfpathlineto{\pgfqpoint{5.202468in}{2.599221in}}%
\pgfpathlineto{\pgfqpoint{5.209759in}{2.605765in}}%
\pgfpathclose%
\pgfusepath{fill}%
\end{pgfscope}%
\begin{pgfscope}%
\pgfpathrectangle{\pgfqpoint{1.150000in}{0.150000in}}{\pgfqpoint{5.700000in}{5.700000in}}%
\pgfusepath{clip}%
\pgfsetbuttcap%
\pgfsetroundjoin%
\definecolor{currentfill}{rgb}{0.280267,0.073417,0.397163}%
\pgfsetfillcolor{currentfill}%
\pgfsetfillopacity{0.700000}%
\pgfsetlinewidth{0.000000pt}%
\definecolor{currentstroke}{rgb}{0.000000,0.000000,0.000000}%
\pgfsetstrokecolor{currentstroke}%
\pgfsetdash{}{0pt}%
\pgfpathmoveto{\pgfqpoint{4.987594in}{2.580750in}}%
\pgfpathlineto{\pgfqpoint{5.001195in}{2.578367in}}%
\pgfpathlineto{\pgfqpoint{5.014804in}{2.576010in}}%
\pgfpathlineto{\pgfqpoint{5.028419in}{2.573678in}}%
\pgfpathlineto{\pgfqpoint{5.042042in}{2.571371in}}%
\pgfpathlineto{\pgfqpoint{5.034678in}{2.564712in}}%
\pgfpathlineto{\pgfqpoint{5.027308in}{2.558054in}}%
\pgfpathlineto{\pgfqpoint{5.019933in}{2.551394in}}%
\pgfpathlineto{\pgfqpoint{5.012552in}{2.544728in}}%
\pgfpathlineto{\pgfqpoint{4.998914in}{2.546944in}}%
\pgfpathlineto{\pgfqpoint{4.985283in}{2.549186in}}%
\pgfpathlineto{\pgfqpoint{4.971660in}{2.551453in}}%
\pgfpathlineto{\pgfqpoint{4.958044in}{2.553745in}}%
\pgfpathlineto{\pgfqpoint{4.965439in}{2.560497in}}%
\pgfpathlineto{\pgfqpoint{4.972830in}{2.567246in}}%
\pgfpathlineto{\pgfqpoint{4.980214in}{2.573996in}}%
\pgfpathlineto{\pgfqpoint{4.987594in}{2.580750in}}%
\pgfpathclose%
\pgfusepath{fill}%
\end{pgfscope}%
\begin{pgfscope}%
\pgfpathrectangle{\pgfqpoint{1.150000in}{0.150000in}}{\pgfqpoint{5.700000in}{5.700000in}}%
\pgfusepath{clip}%
\pgfsetbuttcap%
\pgfsetroundjoin%
\definecolor{currentfill}{rgb}{0.278791,0.062145,0.386592}%
\pgfsetfillcolor{currentfill}%
\pgfsetfillopacity{0.700000}%
\pgfsetlinewidth{0.000000pt}%
\definecolor{currentstroke}{rgb}{0.000000,0.000000,0.000000}%
\pgfsetstrokecolor{currentstroke}%
\pgfsetdash{}{0pt}%
\pgfpathmoveto{\pgfqpoint{2.911189in}{2.558502in}}%
\pgfpathlineto{\pgfqpoint{2.924368in}{2.552021in}}%
\pgfpathlineto{\pgfqpoint{2.937550in}{2.545588in}}%
\pgfpathlineto{\pgfqpoint{2.950736in}{2.539203in}}%
\pgfpathlineto{\pgfqpoint{2.963925in}{2.532864in}}%
\pgfpathlineto{\pgfqpoint{2.955742in}{2.527566in}}%
\pgfpathlineto{\pgfqpoint{2.947551in}{2.522362in}}%
\pgfpathlineto{\pgfqpoint{2.939350in}{2.517256in}}%
\pgfpathlineto{\pgfqpoint{2.931141in}{2.512252in}}%
\pgfpathlineto{\pgfqpoint{2.917933in}{2.518725in}}%
\pgfpathlineto{\pgfqpoint{2.904729in}{2.525246in}}%
\pgfpathlineto{\pgfqpoint{2.891528in}{2.531814in}}%
\pgfpathlineto{\pgfqpoint{2.878330in}{2.538431in}}%
\pgfpathlineto{\pgfqpoint{2.886559in}{2.543295in}}%
\pgfpathlineto{\pgfqpoint{2.894778in}{2.548264in}}%
\pgfpathlineto{\pgfqpoint{2.902988in}{2.553334in}}%
\pgfpathlineto{\pgfqpoint{2.911189in}{2.558502in}}%
\pgfpathclose%
\pgfusepath{fill}%
\end{pgfscope}%
\begin{pgfscope}%
\pgfpathrectangle{\pgfqpoint{1.150000in}{0.150000in}}{\pgfqpoint{5.700000in}{5.700000in}}%
\pgfusepath{clip}%
\pgfsetbuttcap%
\pgfsetroundjoin%
\definecolor{currentfill}{rgb}{0.278791,0.062145,0.386592}%
\pgfsetfillcolor{currentfill}%
\pgfsetfillopacity{0.700000}%
\pgfsetlinewidth{0.000000pt}%
\definecolor{currentstroke}{rgb}{0.000000,0.000000,0.000000}%
\pgfsetstrokecolor{currentstroke}%
\pgfsetdash{}{0pt}%
\pgfpathmoveto{\pgfqpoint{4.765392in}{2.555263in}}%
\pgfpathlineto{\pgfqpoint{4.778937in}{2.552737in}}%
\pgfpathlineto{\pgfqpoint{4.792490in}{2.550237in}}%
\pgfpathlineto{\pgfqpoint{4.806049in}{2.547763in}}%
\pgfpathlineto{\pgfqpoint{4.819616in}{2.545315in}}%
\pgfpathlineto{\pgfqpoint{4.812163in}{2.538374in}}%
\pgfpathlineto{\pgfqpoint{4.804704in}{2.531423in}}%
\pgfpathlineto{\pgfqpoint{4.797240in}{2.524458in}}%
\pgfpathlineto{\pgfqpoint{4.789770in}{2.517476in}}%
\pgfpathlineto{\pgfqpoint{4.776189in}{2.519860in}}%
\pgfpathlineto{\pgfqpoint{4.762616in}{2.522270in}}%
\pgfpathlineto{\pgfqpoint{4.749049in}{2.524706in}}%
\pgfpathlineto{\pgfqpoint{4.735490in}{2.527169in}}%
\pgfpathlineto{\pgfqpoint{4.742974in}{2.534209in}}%
\pgfpathlineto{\pgfqpoint{4.750452in}{2.541237in}}%
\pgfpathlineto{\pgfqpoint{4.757925in}{2.548254in}}%
\pgfpathlineto{\pgfqpoint{4.765392in}{2.555263in}}%
\pgfpathclose%
\pgfusepath{fill}%
\end{pgfscope}%
\begin{pgfscope}%
\pgfpathrectangle{\pgfqpoint{1.150000in}{0.150000in}}{\pgfqpoint{5.700000in}{5.700000in}}%
\pgfusepath{clip}%
\pgfsetbuttcap%
\pgfsetroundjoin%
\definecolor{currentfill}{rgb}{0.276022,0.044167,0.370164}%
\pgfsetfillcolor{currentfill}%
\pgfsetfillopacity{0.700000}%
\pgfsetlinewidth{0.000000pt}%
\definecolor{currentstroke}{rgb}{0.000000,0.000000,0.000000}%
\pgfsetstrokecolor{currentstroke}%
\pgfsetdash{}{0pt}%
\pgfpathmoveto{\pgfqpoint{4.543164in}{2.529858in}}%
\pgfpathlineto{\pgfqpoint{4.556654in}{2.527125in}}%
\pgfpathlineto{\pgfqpoint{4.570151in}{2.524419in}}%
\pgfpathlineto{\pgfqpoint{4.583655in}{2.521741in}}%
\pgfpathlineto{\pgfqpoint{4.597165in}{2.519089in}}%
\pgfpathlineto{\pgfqpoint{4.589626in}{2.511870in}}%
\pgfpathlineto{\pgfqpoint{4.582080in}{2.504633in}}%
\pgfpathlineto{\pgfqpoint{4.574530in}{2.497377in}}%
\pgfpathlineto{\pgfqpoint{4.566973in}{2.490101in}}%
\pgfpathlineto{\pgfqpoint{4.553449in}{2.492715in}}%
\pgfpathlineto{\pgfqpoint{4.539933in}{2.495356in}}%
\pgfpathlineto{\pgfqpoint{4.526423in}{2.498024in}}%
\pgfpathlineto{\pgfqpoint{4.512920in}{2.500720in}}%
\pgfpathlineto{\pgfqpoint{4.520489in}{2.508029in}}%
\pgfpathlineto{\pgfqpoint{4.528053in}{2.515321in}}%
\pgfpathlineto{\pgfqpoint{4.535612in}{2.522597in}}%
\pgfpathlineto{\pgfqpoint{4.543164in}{2.529858in}}%
\pgfpathclose%
\pgfusepath{fill}%
\end{pgfscope}%
\begin{pgfscope}%
\pgfpathrectangle{\pgfqpoint{1.150000in}{0.150000in}}{\pgfqpoint{5.700000in}{5.700000in}}%
\pgfusepath{clip}%
\pgfsetbuttcap%
\pgfsetroundjoin%
\definecolor{currentfill}{rgb}{0.273809,0.031497,0.358853}%
\pgfsetfillcolor{currentfill}%
\pgfsetfillopacity{0.700000}%
\pgfsetlinewidth{0.000000pt}%
\definecolor{currentstroke}{rgb}{0.000000,0.000000,0.000000}%
\pgfsetstrokecolor{currentstroke}%
\pgfsetdash{}{0pt}%
\pgfpathmoveto{\pgfqpoint{4.320914in}{2.505491in}}%
\pgfpathlineto{\pgfqpoint{4.334350in}{2.502486in}}%
\pgfpathlineto{\pgfqpoint{4.347793in}{2.499510in}}%
\pgfpathlineto{\pgfqpoint{4.361242in}{2.496561in}}%
\pgfpathlineto{\pgfqpoint{4.374698in}{2.493641in}}%
\pgfpathlineto{\pgfqpoint{4.367075in}{2.486193in}}%
\pgfpathlineto{\pgfqpoint{4.359445in}{2.478727in}}%
\pgfpathlineto{\pgfqpoint{4.351811in}{2.471241in}}%
\pgfpathlineto{\pgfqpoint{4.344170in}{2.463736in}}%
\pgfpathlineto{\pgfqpoint{4.330702in}{2.466645in}}%
\pgfpathlineto{\pgfqpoint{4.317240in}{2.469582in}}%
\pgfpathlineto{\pgfqpoint{4.303784in}{2.472548in}}%
\pgfpathlineto{\pgfqpoint{4.290335in}{2.475542in}}%
\pgfpathlineto{\pgfqpoint{4.297988in}{2.483053in}}%
\pgfpathlineto{\pgfqpoint{4.305636in}{2.490548in}}%
\pgfpathlineto{\pgfqpoint{4.313278in}{2.498027in}}%
\pgfpathlineto{\pgfqpoint{4.320914in}{2.505491in}}%
\pgfpathclose%
\pgfusepath{fill}%
\end{pgfscope}%
\begin{pgfscope}%
\pgfpathrectangle{\pgfqpoint{1.150000in}{0.150000in}}{\pgfqpoint{5.700000in}{5.700000in}}%
\pgfusepath{clip}%
\pgfsetbuttcap%
\pgfsetroundjoin%
\definecolor{currentfill}{rgb}{0.268510,0.009605,0.335427}%
\pgfsetfillcolor{currentfill}%
\pgfsetfillopacity{0.700000}%
\pgfsetlinewidth{0.000000pt}%
\definecolor{currentstroke}{rgb}{0.000000,0.000000,0.000000}%
\pgfsetstrokecolor{currentstroke}%
\pgfsetdash{}{0pt}%
\pgfpathmoveto{\pgfqpoint{3.738425in}{2.467148in}}%
\pgfpathlineto{\pgfqpoint{3.751732in}{2.463090in}}%
\pgfpathlineto{\pgfqpoint{3.765043in}{2.459065in}}%
\pgfpathlineto{\pgfqpoint{3.778360in}{2.455073in}}%
\pgfpathlineto{\pgfqpoint{3.791683in}{2.451114in}}%
\pgfpathlineto{\pgfqpoint{3.783846in}{2.443639in}}%
\pgfpathlineto{\pgfqpoint{3.776004in}{2.436171in}}%
\pgfpathlineto{\pgfqpoint{3.768155in}{2.428712in}}%
\pgfpathlineto{\pgfqpoint{3.760301in}{2.421262in}}%
\pgfpathlineto{\pgfqpoint{3.746965in}{2.425276in}}%
\pgfpathlineto{\pgfqpoint{3.733635in}{2.429324in}}%
\pgfpathlineto{\pgfqpoint{3.720310in}{2.433403in}}%
\pgfpathlineto{\pgfqpoint{3.706991in}{2.437516in}}%
\pgfpathlineto{\pgfqpoint{3.714859in}{2.444906in}}%
\pgfpathlineto{\pgfqpoint{3.722720in}{2.452308in}}%
\pgfpathlineto{\pgfqpoint{3.730576in}{2.459722in}}%
\pgfpathlineto{\pgfqpoint{3.738425in}{2.467148in}}%
\pgfpathclose%
\pgfusepath{fill}%
\end{pgfscope}%
\begin{pgfscope}%
\pgfpathrectangle{\pgfqpoint{1.150000in}{0.150000in}}{\pgfqpoint{5.700000in}{5.700000in}}%
\pgfusepath{clip}%
\pgfsetbuttcap%
\pgfsetroundjoin%
\definecolor{currentfill}{rgb}{0.271305,0.019942,0.347269}%
\pgfsetfillcolor{currentfill}%
\pgfsetfillopacity{0.700000}%
\pgfsetlinewidth{0.000000pt}%
\definecolor{currentstroke}{rgb}{0.000000,0.000000,0.000000}%
\pgfsetstrokecolor{currentstroke}%
\pgfsetdash{}{0pt}%
\pgfpathmoveto{\pgfqpoint{4.098628in}{2.483524in}}%
\pgfpathlineto{\pgfqpoint{4.112013in}{2.480179in}}%
\pgfpathlineto{\pgfqpoint{4.125405in}{2.476864in}}%
\pgfpathlineto{\pgfqpoint{4.138803in}{2.473578in}}%
\pgfpathlineto{\pgfqpoint{4.152207in}{2.470322in}}%
\pgfpathlineto{\pgfqpoint{4.144501in}{2.462740in}}%
\pgfpathlineto{\pgfqpoint{4.136789in}{2.455146in}}%
\pgfpathlineto{\pgfqpoint{4.129072in}{2.447538in}}%
\pgfpathlineto{\pgfqpoint{4.121349in}{2.439918in}}%
\pgfpathlineto{\pgfqpoint{4.107933in}{2.443190in}}%
\pgfpathlineto{\pgfqpoint{4.094523in}{2.446491in}}%
\pgfpathlineto{\pgfqpoint{4.081119in}{2.449822in}}%
\pgfpathlineto{\pgfqpoint{4.067721in}{2.453182in}}%
\pgfpathlineto{\pgfqpoint{4.075456in}{2.460782in}}%
\pgfpathlineto{\pgfqpoint{4.083186in}{2.468372in}}%
\pgfpathlineto{\pgfqpoint{4.090910in}{2.475953in}}%
\pgfpathlineto{\pgfqpoint{4.098628in}{2.483524in}}%
\pgfpathclose%
\pgfusepath{fill}%
\end{pgfscope}%
\begin{pgfscope}%
\pgfpathrectangle{\pgfqpoint{1.150000in}{0.150000in}}{\pgfqpoint{5.700000in}{5.700000in}}%
\pgfusepath{clip}%
\pgfsetbuttcap%
\pgfsetroundjoin%
\definecolor{currentfill}{rgb}{0.283091,0.110553,0.431554}%
\pgfsetfillcolor{currentfill}%
\pgfsetfillopacity{0.700000}%
\pgfsetlinewidth{0.000000pt}%
\definecolor{currentstroke}{rgb}{0.000000,0.000000,0.000000}%
\pgfsetstrokecolor{currentstroke}%
\pgfsetdash{}{0pt}%
\pgfpathmoveto{\pgfqpoint{5.570402in}{2.637623in}}%
\pgfpathlineto{\pgfqpoint{5.584155in}{2.635382in}}%
\pgfpathlineto{\pgfqpoint{5.597916in}{2.633166in}}%
\pgfpathlineto{\pgfqpoint{5.611684in}{2.630974in}}%
\pgfpathlineto{\pgfqpoint{5.625460in}{2.628806in}}%
\pgfpathlineto{\pgfqpoint{5.618329in}{2.622630in}}%
\pgfpathlineto{\pgfqpoint{5.611195in}{2.616511in}}%
\pgfpathlineto{\pgfqpoint{5.604056in}{2.610442in}}%
\pgfpathlineto{\pgfqpoint{5.596914in}{2.604420in}}%
\pgfpathlineto{\pgfqpoint{5.583118in}{2.606431in}}%
\pgfpathlineto{\pgfqpoint{5.569331in}{2.608467in}}%
\pgfpathlineto{\pgfqpoint{5.555551in}{2.610526in}}%
\pgfpathlineto{\pgfqpoint{5.541779in}{2.612610in}}%
\pgfpathlineto{\pgfqpoint{5.548941in}{2.618785in}}%
\pgfpathlineto{\pgfqpoint{5.556098in}{2.625008in}}%
\pgfpathlineto{\pgfqpoint{5.563252in}{2.631286in}}%
\pgfpathlineto{\pgfqpoint{5.570402in}{2.637623in}}%
\pgfpathclose%
\pgfusepath{fill}%
\end{pgfscope}%
\begin{pgfscope}%
\pgfpathrectangle{\pgfqpoint{1.150000in}{0.150000in}}{\pgfqpoint{5.700000in}{5.700000in}}%
\pgfusepath{clip}%
\pgfsetbuttcap%
\pgfsetroundjoin%
\definecolor{currentfill}{rgb}{0.281446,0.084320,0.407414}%
\pgfsetfillcolor{currentfill}%
\pgfsetfillopacity{0.700000}%
\pgfsetlinewidth{0.000000pt}%
\definecolor{currentstroke}{rgb}{0.000000,0.000000,0.000000}%
\pgfsetstrokecolor{currentstroke}%
\pgfsetdash{}{0pt}%
\pgfpathmoveto{\pgfqpoint{2.772862in}{2.593142in}}%
\pgfpathlineto{\pgfqpoint{2.786035in}{2.586125in}}%
\pgfpathlineto{\pgfqpoint{2.799211in}{2.579160in}}%
\pgfpathlineto{\pgfqpoint{2.812390in}{2.572246in}}%
\pgfpathlineto{\pgfqpoint{2.825571in}{2.565383in}}%
\pgfpathlineto{\pgfqpoint{2.817314in}{2.560770in}}%
\pgfpathlineto{\pgfqpoint{2.809047in}{2.556271in}}%
\pgfpathlineto{\pgfqpoint{2.800769in}{2.551891in}}%
\pgfpathlineto{\pgfqpoint{2.792482in}{2.547631in}}%
\pgfpathlineto{\pgfqpoint{2.779280in}{2.554643in}}%
\pgfpathlineto{\pgfqpoint{2.766081in}{2.561706in}}%
\pgfpathlineto{\pgfqpoint{2.752884in}{2.568820in}}%
\pgfpathlineto{\pgfqpoint{2.739691in}{2.575986in}}%
\pgfpathlineto{\pgfqpoint{2.747999in}{2.580091in}}%
\pgfpathlineto{\pgfqpoint{2.756297in}{2.584321in}}%
\pgfpathlineto{\pgfqpoint{2.764584in}{2.588673in}}%
\pgfpathlineto{\pgfqpoint{2.772862in}{2.593142in}}%
\pgfpathclose%
\pgfusepath{fill}%
\end{pgfscope}%
\begin{pgfscope}%
\pgfpathrectangle{\pgfqpoint{1.150000in}{0.150000in}}{\pgfqpoint{5.700000in}{5.700000in}}%
\pgfusepath{clip}%
\pgfsetbuttcap%
\pgfsetroundjoin%
\definecolor{currentfill}{rgb}{0.282327,0.094955,0.417331}%
\pgfsetfillcolor{currentfill}%
\pgfsetfillopacity{0.700000}%
\pgfsetlinewidth{0.000000pt}%
\definecolor{currentstroke}{rgb}{0.000000,0.000000,0.000000}%
\pgfsetstrokecolor{currentstroke}%
\pgfsetdash{}{0pt}%
\pgfpathmoveto{\pgfqpoint{5.348200in}{2.613492in}}%
\pgfpathlineto{\pgfqpoint{5.361899in}{2.611259in}}%
\pgfpathlineto{\pgfqpoint{5.375606in}{2.609051in}}%
\pgfpathlineto{\pgfqpoint{5.389320in}{2.606867in}}%
\pgfpathlineto{\pgfqpoint{5.403042in}{2.604708in}}%
\pgfpathlineto{\pgfqpoint{5.395822in}{2.598440in}}%
\pgfpathlineto{\pgfqpoint{5.388597in}{2.592200in}}%
\pgfpathlineto{\pgfqpoint{5.381367in}{2.585985in}}%
\pgfpathlineto{\pgfqpoint{5.374132in}{2.579791in}}%
\pgfpathlineto{\pgfqpoint{5.360393in}{2.581820in}}%
\pgfpathlineto{\pgfqpoint{5.346662in}{2.583873in}}%
\pgfpathlineto{\pgfqpoint{5.332938in}{2.585951in}}%
\pgfpathlineto{\pgfqpoint{5.319221in}{2.588054in}}%
\pgfpathlineto{\pgfqpoint{5.326473in}{2.594374in}}%
\pgfpathlineto{\pgfqpoint{5.333720in}{2.600717in}}%
\pgfpathlineto{\pgfqpoint{5.340963in}{2.607089in}}%
\pgfpathlineto{\pgfqpoint{5.348200in}{2.613492in}}%
\pgfpathclose%
\pgfusepath{fill}%
\end{pgfscope}%
\begin{pgfscope}%
\pgfpathrectangle{\pgfqpoint{1.150000in}{0.150000in}}{\pgfqpoint{5.700000in}{5.700000in}}%
\pgfusepath{clip}%
\pgfsetbuttcap%
\pgfsetroundjoin%
\definecolor{currentfill}{rgb}{0.271305,0.019942,0.347269}%
\pgfsetfillcolor{currentfill}%
\pgfsetfillopacity{0.700000}%
\pgfsetlinewidth{0.000000pt}%
\definecolor{currentstroke}{rgb}{0.000000,0.000000,0.000000}%
\pgfsetstrokecolor{currentstroke}%
\pgfsetdash{}{0pt}%
\pgfpathmoveto{\pgfqpoint{3.240084in}{2.486615in}}%
\pgfpathlineto{\pgfqpoint{3.253307in}{2.481242in}}%
\pgfpathlineto{\pgfqpoint{3.266534in}{2.475908in}}%
\pgfpathlineto{\pgfqpoint{3.279766in}{2.470615in}}%
\pgfpathlineto{\pgfqpoint{3.293002in}{2.465360in}}%
\pgfpathlineto{\pgfqpoint{3.284966in}{2.458890in}}%
\pgfpathlineto{\pgfqpoint{3.276923in}{2.452476in}}%
\pgfpathlineto{\pgfqpoint{3.268873in}{2.446120in}}%
\pgfpathlineto{\pgfqpoint{3.260815in}{2.439826in}}%
\pgfpathlineto{\pgfqpoint{3.247563in}{2.445189in}}%
\pgfpathlineto{\pgfqpoint{3.234316in}{2.450591in}}%
\pgfpathlineto{\pgfqpoint{3.221073in}{2.456032in}}%
\pgfpathlineto{\pgfqpoint{3.207834in}{2.461514in}}%
\pgfpathlineto{\pgfqpoint{3.215908in}{2.467695in}}%
\pgfpathlineto{\pgfqpoint{3.223974in}{2.473940in}}%
\pgfpathlineto{\pgfqpoint{3.232033in}{2.480248in}}%
\pgfpathlineto{\pgfqpoint{3.240084in}{2.486615in}}%
\pgfpathclose%
\pgfusepath{fill}%
\end{pgfscope}%
\begin{pgfscope}%
\pgfpathrectangle{\pgfqpoint{1.150000in}{0.150000in}}{\pgfqpoint{5.700000in}{5.700000in}}%
\pgfusepath{clip}%
\pgfsetbuttcap%
\pgfsetroundjoin%
\definecolor{currentfill}{rgb}{0.268510,0.009605,0.335427}%
\pgfsetfillcolor{currentfill}%
\pgfsetfillopacity{0.700000}%
\pgfsetlinewidth{0.000000pt}%
\definecolor{currentstroke}{rgb}{0.000000,0.000000,0.000000}%
\pgfsetstrokecolor{currentstroke}%
\pgfsetdash{}{0pt}%
\pgfpathmoveto{\pgfqpoint{3.876270in}{2.465729in}}%
\pgfpathlineto{\pgfqpoint{3.889609in}{2.461972in}}%
\pgfpathlineto{\pgfqpoint{3.902953in}{2.458246in}}%
\pgfpathlineto{\pgfqpoint{3.916303in}{2.454552in}}%
\pgfpathlineto{\pgfqpoint{3.929659in}{2.450888in}}%
\pgfpathlineto{\pgfqpoint{3.921871in}{2.443316in}}%
\pgfpathlineto{\pgfqpoint{3.914077in}{2.435741in}}%
\pgfpathlineto{\pgfqpoint{3.906277in}{2.428166in}}%
\pgfpathlineto{\pgfqpoint{3.898471in}{2.420590in}}%
\pgfpathlineto{\pgfqpoint{3.885103in}{2.424295in}}%
\pgfpathlineto{\pgfqpoint{3.871740in}{2.428031in}}%
\pgfpathlineto{\pgfqpoint{3.858383in}{2.431799in}}%
\pgfpathlineto{\pgfqpoint{3.845031in}{2.435598in}}%
\pgfpathlineto{\pgfqpoint{3.852850in}{2.443127in}}%
\pgfpathlineto{\pgfqpoint{3.860662in}{2.450659in}}%
\pgfpathlineto{\pgfqpoint{3.868469in}{2.458194in}}%
\pgfpathlineto{\pgfqpoint{3.876270in}{2.465729in}}%
\pgfpathclose%
\pgfusepath{fill}%
\end{pgfscope}%
\begin{pgfscope}%
\pgfpathrectangle{\pgfqpoint{1.150000in}{0.150000in}}{\pgfqpoint{5.700000in}{5.700000in}}%
\pgfusepath{clip}%
\pgfsetbuttcap%
\pgfsetroundjoin%
\definecolor{currentfill}{rgb}{0.269944,0.014625,0.341379}%
\pgfsetfillcolor{currentfill}%
\pgfsetfillopacity{0.700000}%
\pgfsetlinewidth{0.000000pt}%
\definecolor{currentstroke}{rgb}{0.000000,0.000000,0.000000}%
\pgfsetstrokecolor{currentstroke}%
\pgfsetdash{}{0pt}%
\pgfpathmoveto{\pgfqpoint{3.378006in}{2.471511in}}%
\pgfpathlineto{\pgfqpoint{3.391251in}{2.466544in}}%
\pgfpathlineto{\pgfqpoint{3.404500in}{2.461614in}}%
\pgfpathlineto{\pgfqpoint{3.417753in}{2.456722in}}%
\pgfpathlineto{\pgfqpoint{3.431012in}{2.451867in}}%
\pgfpathlineto{\pgfqpoint{3.423033in}{2.445013in}}%
\pgfpathlineto{\pgfqpoint{3.415048in}{2.438200in}}%
\pgfpathlineto{\pgfqpoint{3.407056in}{2.431431in}}%
\pgfpathlineto{\pgfqpoint{3.399058in}{2.424706in}}%
\pgfpathlineto{\pgfqpoint{3.385784in}{2.429657in}}%
\pgfpathlineto{\pgfqpoint{3.372516in}{2.434644in}}%
\pgfpathlineto{\pgfqpoint{3.359252in}{2.439668in}}%
\pgfpathlineto{\pgfqpoint{3.345993in}{2.444730in}}%
\pgfpathlineto{\pgfqpoint{3.354006in}{2.451355in}}%
\pgfpathlineto{\pgfqpoint{3.362013in}{2.458028in}}%
\pgfpathlineto{\pgfqpoint{3.370013in}{2.464747in}}%
\pgfpathlineto{\pgfqpoint{3.378006in}{2.471511in}}%
\pgfpathclose%
\pgfusepath{fill}%
\end{pgfscope}%
\begin{pgfscope}%
\pgfpathrectangle{\pgfqpoint{1.150000in}{0.150000in}}{\pgfqpoint{5.700000in}{5.700000in}}%
\pgfusepath{clip}%
\pgfsetbuttcap%
\pgfsetroundjoin%
\definecolor{currentfill}{rgb}{0.281446,0.084320,0.407414}%
\pgfsetfillcolor{currentfill}%
\pgfsetfillopacity{0.700000}%
\pgfsetlinewidth{0.000000pt}%
\definecolor{currentstroke}{rgb}{0.000000,0.000000,0.000000}%
\pgfsetstrokecolor{currentstroke}%
\pgfsetdash{}{0pt}%
\pgfpathmoveto{\pgfqpoint{5.125947in}{2.588692in}}%
\pgfpathlineto{\pgfqpoint{5.139591in}{2.586408in}}%
\pgfpathlineto{\pgfqpoint{5.153242in}{2.584149in}}%
\pgfpathlineto{\pgfqpoint{5.166901in}{2.581915in}}%
\pgfpathlineto{\pgfqpoint{5.180566in}{2.579706in}}%
\pgfpathlineto{\pgfqpoint{5.173255in}{2.573229in}}%
\pgfpathlineto{\pgfqpoint{5.165939in}{2.566758in}}%
\pgfpathlineto{\pgfqpoint{5.158617in}{2.560292in}}%
\pgfpathlineto{\pgfqpoint{5.151290in}{2.553827in}}%
\pgfpathlineto{\pgfqpoint{5.137608in}{2.555932in}}%
\pgfpathlineto{\pgfqpoint{5.123933in}{2.558062in}}%
\pgfpathlineto{\pgfqpoint{5.110266in}{2.560217in}}%
\pgfpathlineto{\pgfqpoint{5.096607in}{2.562398in}}%
\pgfpathlineto{\pgfqpoint{5.103950in}{2.568962in}}%
\pgfpathlineto{\pgfqpoint{5.111288in}{2.575530in}}%
\pgfpathlineto{\pgfqpoint{5.118620in}{2.582106in}}%
\pgfpathlineto{\pgfqpoint{5.125947in}{2.588692in}}%
\pgfpathclose%
\pgfusepath{fill}%
\end{pgfscope}%
\begin{pgfscope}%
\pgfpathrectangle{\pgfqpoint{1.150000in}{0.150000in}}{\pgfqpoint{5.700000in}{5.700000in}}%
\pgfusepath{clip}%
\pgfsetbuttcap%
\pgfsetroundjoin%
\definecolor{currentfill}{rgb}{0.273809,0.031497,0.358853}%
\pgfsetfillcolor{currentfill}%
\pgfsetfillopacity{0.700000}%
\pgfsetlinewidth{0.000000pt}%
\definecolor{currentstroke}{rgb}{0.000000,0.000000,0.000000}%
\pgfsetstrokecolor{currentstroke}%
\pgfsetdash{}{0pt}%
\pgfpathmoveto{\pgfqpoint{3.102075in}{2.506846in}}%
\pgfpathlineto{\pgfqpoint{3.115281in}{2.501033in}}%
\pgfpathlineto{\pgfqpoint{3.128490in}{2.495262in}}%
\pgfpathlineto{\pgfqpoint{3.141704in}{2.489533in}}%
\pgfpathlineto{\pgfqpoint{3.154921in}{2.483846in}}%
\pgfpathlineto{\pgfqpoint{3.146824in}{2.477849in}}%
\pgfpathlineto{\pgfqpoint{3.138718in}{2.471925in}}%
\pgfpathlineto{\pgfqpoint{3.130605in}{2.466076in}}%
\pgfpathlineto{\pgfqpoint{3.122484in}{2.460306in}}%
\pgfpathlineto{\pgfqpoint{3.109249in}{2.466115in}}%
\pgfpathlineto{\pgfqpoint{3.096019in}{2.471965in}}%
\pgfpathlineto{\pgfqpoint{3.082792in}{2.477857in}}%
\pgfpathlineto{\pgfqpoint{3.069569in}{2.483793in}}%
\pgfpathlineto{\pgfqpoint{3.077708in}{2.489436in}}%
\pgfpathlineto{\pgfqpoint{3.085838in}{2.495161in}}%
\pgfpathlineto{\pgfqpoint{3.093961in}{2.500966in}}%
\pgfpathlineto{\pgfqpoint{3.102075in}{2.506846in}}%
\pgfpathclose%
\pgfusepath{fill}%
\end{pgfscope}%
\begin{pgfscope}%
\pgfpathrectangle{\pgfqpoint{1.150000in}{0.150000in}}{\pgfqpoint{5.700000in}{5.700000in}}%
\pgfusepath{clip}%
\pgfsetbuttcap%
\pgfsetroundjoin%
\definecolor{currentfill}{rgb}{0.279566,0.067836,0.391917}%
\pgfsetfillcolor{currentfill}%
\pgfsetfillopacity{0.700000}%
\pgfsetlinewidth{0.000000pt}%
\definecolor{currentstroke}{rgb}{0.000000,0.000000,0.000000}%
\pgfsetstrokecolor{currentstroke}%
\pgfsetdash{}{0pt}%
\pgfpathmoveto{\pgfqpoint{4.903650in}{2.563172in}}%
\pgfpathlineto{\pgfqpoint{4.917238in}{2.560777in}}%
\pgfpathlineto{\pgfqpoint{4.930833in}{2.558408in}}%
\pgfpathlineto{\pgfqpoint{4.944435in}{2.556064in}}%
\pgfpathlineto{\pgfqpoint{4.958044in}{2.553745in}}%
\pgfpathlineto{\pgfqpoint{4.950642in}{2.546990in}}%
\pgfpathlineto{\pgfqpoint{4.943235in}{2.540226in}}%
\pgfpathlineto{\pgfqpoint{4.935822in}{2.533452in}}%
\pgfpathlineto{\pgfqpoint{4.928404in}{2.526666in}}%
\pgfpathlineto{\pgfqpoint{4.914780in}{2.528907in}}%
\pgfpathlineto{\pgfqpoint{4.901164in}{2.531173in}}%
\pgfpathlineto{\pgfqpoint{4.887554in}{2.533465in}}%
\pgfpathlineto{\pgfqpoint{4.873952in}{2.535783in}}%
\pgfpathlineto{\pgfqpoint{4.881385in}{2.542642in}}%
\pgfpathlineto{\pgfqpoint{4.888813in}{2.549492in}}%
\pgfpathlineto{\pgfqpoint{4.896234in}{2.556334in}}%
\pgfpathlineto{\pgfqpoint{4.903650in}{2.563172in}}%
\pgfpathclose%
\pgfusepath{fill}%
\end{pgfscope}%
\begin{pgfscope}%
\pgfpathrectangle{\pgfqpoint{1.150000in}{0.150000in}}{\pgfqpoint{5.700000in}{5.700000in}}%
\pgfusepath{clip}%
\pgfsetbuttcap%
\pgfsetroundjoin%
\definecolor{currentfill}{rgb}{0.277941,0.056324,0.381191}%
\pgfsetfillcolor{currentfill}%
\pgfsetfillopacity{0.700000}%
\pgfsetlinewidth{0.000000pt}%
\definecolor{currentstroke}{rgb}{0.000000,0.000000,0.000000}%
\pgfsetstrokecolor{currentstroke}%
\pgfsetdash{}{0pt}%
\pgfpathmoveto{\pgfqpoint{4.681323in}{2.537283in}}%
\pgfpathlineto{\pgfqpoint{4.694854in}{2.534715in}}%
\pgfpathlineto{\pgfqpoint{4.708393in}{2.532173in}}%
\pgfpathlineto{\pgfqpoint{4.721938in}{2.529658in}}%
\pgfpathlineto{\pgfqpoint{4.735490in}{2.527169in}}%
\pgfpathlineto{\pgfqpoint{4.728000in}{2.520114in}}%
\pgfpathlineto{\pgfqpoint{4.720505in}{2.513042in}}%
\pgfpathlineto{\pgfqpoint{4.713004in}{2.505951in}}%
\pgfpathlineto{\pgfqpoint{4.705497in}{2.498840in}}%
\pgfpathlineto{\pgfqpoint{4.691931in}{2.501278in}}%
\pgfpathlineto{\pgfqpoint{4.678372in}{2.503743in}}%
\pgfpathlineto{\pgfqpoint{4.664820in}{2.506234in}}%
\pgfpathlineto{\pgfqpoint{4.651275in}{2.508751in}}%
\pgfpathlineto{\pgfqpoint{4.658796in}{2.515908in}}%
\pgfpathlineto{\pgfqpoint{4.666311in}{2.523048in}}%
\pgfpathlineto{\pgfqpoint{4.673820in}{2.530173in}}%
\pgfpathlineto{\pgfqpoint{4.681323in}{2.537283in}}%
\pgfpathclose%
\pgfusepath{fill}%
\end{pgfscope}%
\begin{pgfscope}%
\pgfpathrectangle{\pgfqpoint{1.150000in}{0.150000in}}{\pgfqpoint{5.700000in}{5.700000in}}%
\pgfusepath{clip}%
\pgfsetbuttcap%
\pgfsetroundjoin%
\definecolor{currentfill}{rgb}{0.268510,0.009605,0.335427}%
\pgfsetfillcolor{currentfill}%
\pgfsetfillopacity{0.700000}%
\pgfsetlinewidth{0.000000pt}%
\definecolor{currentstroke}{rgb}{0.000000,0.000000,0.000000}%
\pgfsetstrokecolor{currentstroke}%
\pgfsetdash{}{0pt}%
\pgfpathmoveto{\pgfqpoint{3.515888in}{2.460927in}}%
\pgfpathlineto{\pgfqpoint{3.529157in}{2.456334in}}%
\pgfpathlineto{\pgfqpoint{3.542431in}{2.451777in}}%
\pgfpathlineto{\pgfqpoint{3.555711in}{2.447256in}}%
\pgfpathlineto{\pgfqpoint{3.568995in}{2.442769in}}%
\pgfpathlineto{\pgfqpoint{3.561071in}{2.435615in}}%
\pgfpathlineto{\pgfqpoint{3.553140in}{2.428489in}}%
\pgfpathlineto{\pgfqpoint{3.545203in}{2.421392in}}%
\pgfpathlineto{\pgfqpoint{3.537259in}{2.414326in}}%
\pgfpathlineto{\pgfqpoint{3.523961in}{2.418894in}}%
\pgfpathlineto{\pgfqpoint{3.510667in}{2.423498in}}%
\pgfpathlineto{\pgfqpoint{3.497379in}{2.428136in}}%
\pgfpathlineto{\pgfqpoint{3.484096in}{2.432810in}}%
\pgfpathlineto{\pgfqpoint{3.492053in}{2.439790in}}%
\pgfpathlineto{\pgfqpoint{3.500004in}{2.446804in}}%
\pgfpathlineto{\pgfqpoint{3.507949in}{2.453850in}}%
\pgfpathlineto{\pgfqpoint{3.515888in}{2.460927in}}%
\pgfpathclose%
\pgfusepath{fill}%
\end{pgfscope}%
\begin{pgfscope}%
\pgfpathrectangle{\pgfqpoint{1.150000in}{0.150000in}}{\pgfqpoint{5.700000in}{5.700000in}}%
\pgfusepath{clip}%
\pgfsetbuttcap%
\pgfsetroundjoin%
\definecolor{currentfill}{rgb}{0.274952,0.037752,0.364543}%
\pgfsetfillcolor{currentfill}%
\pgfsetfillopacity{0.700000}%
\pgfsetlinewidth{0.000000pt}%
\definecolor{currentstroke}{rgb}{0.000000,0.000000,0.000000}%
\pgfsetstrokecolor{currentstroke}%
\pgfsetdash{}{0pt}%
\pgfpathmoveto{\pgfqpoint{4.458975in}{2.511776in}}%
\pgfpathlineto{\pgfqpoint{4.472451in}{2.508971in}}%
\pgfpathlineto{\pgfqpoint{4.485934in}{2.506193in}}%
\pgfpathlineto{\pgfqpoint{4.499423in}{2.503443in}}%
\pgfpathlineto{\pgfqpoint{4.512920in}{2.500720in}}%
\pgfpathlineto{\pgfqpoint{4.505344in}{2.493392in}}%
\pgfpathlineto{\pgfqpoint{4.497763in}{2.486043in}}%
\pgfpathlineto{\pgfqpoint{4.490177in}{2.478674in}}%
\pgfpathlineto{\pgfqpoint{4.482584in}{2.471281in}}%
\pgfpathlineto{\pgfqpoint{4.469075in}{2.473980in}}%
\pgfpathlineto{\pgfqpoint{4.455572in}{2.476705in}}%
\pgfpathlineto{\pgfqpoint{4.442077in}{2.479459in}}%
\pgfpathlineto{\pgfqpoint{4.428588in}{2.482240in}}%
\pgfpathlineto{\pgfqpoint{4.436193in}{2.489652in}}%
\pgfpathlineto{\pgfqpoint{4.443793in}{2.497044in}}%
\pgfpathlineto{\pgfqpoint{4.451386in}{2.504419in}}%
\pgfpathlineto{\pgfqpoint{4.458975in}{2.511776in}}%
\pgfpathclose%
\pgfusepath{fill}%
\end{pgfscope}%
\begin{pgfscope}%
\pgfpathrectangle{\pgfqpoint{1.150000in}{0.150000in}}{\pgfqpoint{5.700000in}{5.700000in}}%
\pgfusepath{clip}%
\pgfsetbuttcap%
\pgfsetroundjoin%
\definecolor{currentfill}{rgb}{0.272594,0.025563,0.353093}%
\pgfsetfillcolor{currentfill}%
\pgfsetfillopacity{0.700000}%
\pgfsetlinewidth{0.000000pt}%
\definecolor{currentstroke}{rgb}{0.000000,0.000000,0.000000}%
\pgfsetstrokecolor{currentstroke}%
\pgfsetdash{}{0pt}%
\pgfpathmoveto{\pgfqpoint{4.236604in}{2.487803in}}%
\pgfpathlineto{\pgfqpoint{4.250027in}{2.484695in}}%
\pgfpathlineto{\pgfqpoint{4.263457in}{2.481615in}}%
\pgfpathlineto{\pgfqpoint{4.276893in}{2.478564in}}%
\pgfpathlineto{\pgfqpoint{4.290335in}{2.475542in}}%
\pgfpathlineto{\pgfqpoint{4.282677in}{2.468013in}}%
\pgfpathlineto{\pgfqpoint{4.275012in}{2.460467in}}%
\pgfpathlineto{\pgfqpoint{4.267342in}{2.452902in}}%
\pgfpathlineto{\pgfqpoint{4.259667in}{2.445319in}}%
\pgfpathlineto{\pgfqpoint{4.246212in}{2.448344in}}%
\pgfpathlineto{\pgfqpoint{4.232763in}{2.451397in}}%
\pgfpathlineto{\pgfqpoint{4.219322in}{2.454479in}}%
\pgfpathlineto{\pgfqpoint{4.205886in}{2.457589in}}%
\pgfpathlineto{\pgfqpoint{4.213574in}{2.465165in}}%
\pgfpathlineto{\pgfqpoint{4.221256in}{2.472726in}}%
\pgfpathlineto{\pgfqpoint{4.228933in}{2.480272in}}%
\pgfpathlineto{\pgfqpoint{4.236604in}{2.487803in}}%
\pgfpathclose%
\pgfusepath{fill}%
\end{pgfscope}%
\begin{pgfscope}%
\pgfpathrectangle{\pgfqpoint{1.150000in}{0.150000in}}{\pgfqpoint{5.700000in}{5.700000in}}%
\pgfusepath{clip}%
\pgfsetbuttcap%
\pgfsetroundjoin%
\definecolor{currentfill}{rgb}{0.277018,0.050344,0.375715}%
\pgfsetfillcolor{currentfill}%
\pgfsetfillopacity{0.700000}%
\pgfsetlinewidth{0.000000pt}%
\definecolor{currentstroke}{rgb}{0.000000,0.000000,0.000000}%
\pgfsetstrokecolor{currentstroke}%
\pgfsetdash{}{0pt}%
\pgfpathmoveto{\pgfqpoint{2.963925in}{2.532864in}}%
\pgfpathlineto{\pgfqpoint{2.977118in}{2.526572in}}%
\pgfpathlineto{\pgfqpoint{2.990314in}{2.520326in}}%
\pgfpathlineto{\pgfqpoint{3.003514in}{2.514125in}}%
\pgfpathlineto{\pgfqpoint{3.016717in}{2.507970in}}%
\pgfpathlineto{\pgfqpoint{3.008553in}{2.502542in}}%
\pgfpathlineto{\pgfqpoint{3.000379in}{2.497205in}}%
\pgfpathlineto{\pgfqpoint{2.992198in}{2.491962in}}%
\pgfpathlineto{\pgfqpoint{2.984007in}{2.486817in}}%
\pgfpathlineto{\pgfqpoint{2.970785in}{2.493107in}}%
\pgfpathlineto{\pgfqpoint{2.957567in}{2.499443in}}%
\pgfpathlineto{\pgfqpoint{2.944352in}{2.505824in}}%
\pgfpathlineto{\pgfqpoint{2.931141in}{2.512252in}}%
\pgfpathlineto{\pgfqpoint{2.939350in}{2.517256in}}%
\pgfpathlineto{\pgfqpoint{2.947551in}{2.522362in}}%
\pgfpathlineto{\pgfqpoint{2.955742in}{2.527566in}}%
\pgfpathlineto{\pgfqpoint{2.963925in}{2.532864in}}%
\pgfpathclose%
\pgfusepath{fill}%
\end{pgfscope}%
\begin{pgfscope}%
\pgfpathrectangle{\pgfqpoint{1.150000in}{0.150000in}}{\pgfqpoint{5.700000in}{5.700000in}}%
\pgfusepath{clip}%
\pgfsetbuttcap%
\pgfsetroundjoin%
\definecolor{currentfill}{rgb}{0.267004,0.004874,0.329415}%
\pgfsetfillcolor{currentfill}%
\pgfsetfillopacity{0.700000}%
\pgfsetlinewidth{0.000000pt}%
\definecolor{currentstroke}{rgb}{0.000000,0.000000,0.000000}%
\pgfsetstrokecolor{currentstroke}%
\pgfsetdash{}{0pt}%
\pgfpathmoveto{\pgfqpoint{3.653769in}{2.454301in}}%
\pgfpathlineto{\pgfqpoint{3.667066in}{2.450054in}}%
\pgfpathlineto{\pgfqpoint{3.680369in}{2.445841in}}%
\pgfpathlineto{\pgfqpoint{3.693677in}{2.441662in}}%
\pgfpathlineto{\pgfqpoint{3.706991in}{2.437516in}}%
\pgfpathlineto{\pgfqpoint{3.699118in}{2.430141in}}%
\pgfpathlineto{\pgfqpoint{3.691238in}{2.422780in}}%
\pgfpathlineto{\pgfqpoint{3.683353in}{2.415436in}}%
\pgfpathlineto{\pgfqpoint{3.675462in}{2.408110in}}%
\pgfpathlineto{\pgfqpoint{3.662135in}{2.412325in}}%
\pgfpathlineto{\pgfqpoint{3.648813in}{2.416572in}}%
\pgfpathlineto{\pgfqpoint{3.635497in}{2.420853in}}%
\pgfpathlineto{\pgfqpoint{3.622186in}{2.425168in}}%
\pgfpathlineto{\pgfqpoint{3.630091in}{2.432421in}}%
\pgfpathlineto{\pgfqpoint{3.637990in}{2.439695in}}%
\pgfpathlineto{\pgfqpoint{3.645882in}{2.446988in}}%
\pgfpathlineto{\pgfqpoint{3.653769in}{2.454301in}}%
\pgfpathclose%
\pgfusepath{fill}%
\end{pgfscope}%
\begin{pgfscope}%
\pgfpathrectangle{\pgfqpoint{1.150000in}{0.150000in}}{\pgfqpoint{5.700000in}{5.700000in}}%
\pgfusepath{clip}%
\pgfsetbuttcap%
\pgfsetroundjoin%
\definecolor{currentfill}{rgb}{0.283197,0.115680,0.436115}%
\pgfsetfillcolor{currentfill}%
\pgfsetfillopacity{0.700000}%
\pgfsetlinewidth{0.000000pt}%
\definecolor{currentstroke}{rgb}{0.000000,0.000000,0.000000}%
\pgfsetstrokecolor{currentstroke}%
\pgfsetdash{}{0pt}%
\pgfpathmoveto{\pgfqpoint{5.709051in}{2.645055in}}%
\pgfpathlineto{\pgfqpoint{5.722846in}{2.642837in}}%
\pgfpathlineto{\pgfqpoint{5.736648in}{2.640643in}}%
\pgfpathlineto{\pgfqpoint{5.750458in}{2.638472in}}%
\pgfpathlineto{\pgfqpoint{5.764276in}{2.636326in}}%
\pgfpathlineto{\pgfqpoint{5.757198in}{2.630226in}}%
\pgfpathlineto{\pgfqpoint{5.750118in}{2.624195in}}%
\pgfpathlineto{\pgfqpoint{5.743034in}{2.618230in}}%
\pgfpathlineto{\pgfqpoint{5.735946in}{2.612324in}}%
\pgfpathlineto{\pgfqpoint{5.722108in}{2.614300in}}%
\pgfpathlineto{\pgfqpoint{5.708278in}{2.616301in}}%
\pgfpathlineto{\pgfqpoint{5.694455in}{2.618325in}}%
\pgfpathlineto{\pgfqpoint{5.680641in}{2.620373in}}%
\pgfpathlineto{\pgfqpoint{5.687748in}{2.626444in}}%
\pgfpathlineto{\pgfqpoint{5.694853in}{2.632578in}}%
\pgfpathlineto{\pgfqpoint{5.701953in}{2.638780in}}%
\pgfpathlineto{\pgfqpoint{5.709051in}{2.645055in}}%
\pgfpathclose%
\pgfusepath{fill}%
\end{pgfscope}%
\begin{pgfscope}%
\pgfpathrectangle{\pgfqpoint{1.150000in}{0.150000in}}{\pgfqpoint{5.700000in}{5.700000in}}%
\pgfusepath{clip}%
\pgfsetbuttcap%
\pgfsetroundjoin%
\definecolor{currentfill}{rgb}{0.269944,0.014625,0.341379}%
\pgfsetfillcolor{currentfill}%
\pgfsetfillopacity{0.700000}%
\pgfsetlinewidth{0.000000pt}%
\definecolor{currentstroke}{rgb}{0.000000,0.000000,0.000000}%
\pgfsetstrokecolor{currentstroke}%
\pgfsetdash{}{0pt}%
\pgfpathmoveto{\pgfqpoint{4.014190in}{2.466925in}}%
\pgfpathlineto{\pgfqpoint{4.027564in}{2.463444in}}%
\pgfpathlineto{\pgfqpoint{4.040943in}{2.459993in}}%
\pgfpathlineto{\pgfqpoint{4.054329in}{2.456573in}}%
\pgfpathlineto{\pgfqpoint{4.067721in}{2.453182in}}%
\pgfpathlineto{\pgfqpoint{4.059980in}{2.445573in}}%
\pgfpathlineto{\pgfqpoint{4.052233in}{2.437953in}}%
\pgfpathlineto{\pgfqpoint{4.044481in}{2.430324in}}%
\pgfpathlineto{\pgfqpoint{4.036723in}{2.422687in}}%
\pgfpathlineto{\pgfqpoint{4.023319in}{2.426106in}}%
\pgfpathlineto{\pgfqpoint{4.009921in}{2.429555in}}%
\pgfpathlineto{\pgfqpoint{3.996529in}{2.433034in}}%
\pgfpathlineto{\pgfqpoint{3.983143in}{2.436544in}}%
\pgfpathlineto{\pgfqpoint{3.990914in}{2.444148in}}%
\pgfpathlineto{\pgfqpoint{3.998678in}{2.451747in}}%
\pgfpathlineto{\pgfqpoint{4.006437in}{2.459339in}}%
\pgfpathlineto{\pgfqpoint{4.014190in}{2.466925in}}%
\pgfpathclose%
\pgfusepath{fill}%
\end{pgfscope}%
\begin{pgfscope}%
\pgfpathrectangle{\pgfqpoint{1.150000in}{0.150000in}}{\pgfqpoint{5.700000in}{5.700000in}}%
\pgfusepath{clip}%
\pgfsetbuttcap%
\pgfsetroundjoin%
\definecolor{currentfill}{rgb}{0.282910,0.105393,0.426902}%
\pgfsetfillcolor{currentfill}%
\pgfsetfillopacity{0.700000}%
\pgfsetlinewidth{0.000000pt}%
\definecolor{currentstroke}{rgb}{0.000000,0.000000,0.000000}%
\pgfsetstrokecolor{currentstroke}%
\pgfsetdash{}{0pt}%
\pgfpathmoveto{\pgfqpoint{5.486768in}{2.621188in}}%
\pgfpathlineto{\pgfqpoint{5.500509in}{2.619007in}}%
\pgfpathlineto{\pgfqpoint{5.514258in}{2.616851in}}%
\pgfpathlineto{\pgfqpoint{5.528015in}{2.614718in}}%
\pgfpathlineto{\pgfqpoint{5.541779in}{2.612610in}}%
\pgfpathlineto{\pgfqpoint{5.534614in}{2.606480in}}%
\pgfpathlineto{\pgfqpoint{5.527443in}{2.600390in}}%
\pgfpathlineto{\pgfqpoint{5.520269in}{2.594336in}}%
\pgfpathlineto{\pgfqpoint{5.513089in}{2.588312in}}%
\pgfpathlineto{\pgfqpoint{5.499307in}{2.590276in}}%
\pgfpathlineto{\pgfqpoint{5.485532in}{2.592265in}}%
\pgfpathlineto{\pgfqpoint{5.471764in}{2.594278in}}%
\pgfpathlineto{\pgfqpoint{5.458004in}{2.596315in}}%
\pgfpathlineto{\pgfqpoint{5.465202in}{2.602478in}}%
\pgfpathlineto{\pgfqpoint{5.472395in}{2.608674in}}%
\pgfpathlineto{\pgfqpoint{5.479584in}{2.614910in}}%
\pgfpathlineto{\pgfqpoint{5.486768in}{2.621188in}}%
\pgfpathclose%
\pgfusepath{fill}%
\end{pgfscope}%
\begin{pgfscope}%
\pgfpathrectangle{\pgfqpoint{1.150000in}{0.150000in}}{\pgfqpoint{5.700000in}{5.700000in}}%
\pgfusepath{clip}%
\pgfsetbuttcap%
\pgfsetroundjoin%
\definecolor{currentfill}{rgb}{0.282327,0.094955,0.417331}%
\pgfsetfillcolor{currentfill}%
\pgfsetfillopacity{0.700000}%
\pgfsetlinewidth{0.000000pt}%
\definecolor{currentstroke}{rgb}{0.000000,0.000000,0.000000}%
\pgfsetstrokecolor{currentstroke}%
\pgfsetdash{}{0pt}%
\pgfpathmoveto{\pgfqpoint{5.264431in}{2.596711in}}%
\pgfpathlineto{\pgfqpoint{5.278117in}{2.594510in}}%
\pgfpathlineto{\pgfqpoint{5.291811in}{2.592333in}}%
\pgfpathlineto{\pgfqpoint{5.305512in}{2.590181in}}%
\pgfpathlineto{\pgfqpoint{5.319221in}{2.588054in}}%
\pgfpathlineto{\pgfqpoint{5.311964in}{2.581753in}}%
\pgfpathlineto{\pgfqpoint{5.304702in}{2.575469in}}%
\pgfpathlineto{\pgfqpoint{5.297434in}{2.569196in}}%
\pgfpathlineto{\pgfqpoint{5.290162in}{2.562932in}}%
\pgfpathlineto{\pgfqpoint{5.276436in}{2.564942in}}%
\pgfpathlineto{\pgfqpoint{5.262718in}{2.566977in}}%
\pgfpathlineto{\pgfqpoint{5.249007in}{2.569036in}}%
\pgfpathlineto{\pgfqpoint{5.235304in}{2.571120in}}%
\pgfpathlineto{\pgfqpoint{5.242593in}{2.577497in}}%
\pgfpathlineto{\pgfqpoint{5.249878in}{2.583885in}}%
\pgfpathlineto{\pgfqpoint{5.257157in}{2.590289in}}%
\pgfpathlineto{\pgfqpoint{5.264431in}{2.596711in}}%
\pgfpathclose%
\pgfusepath{fill}%
\end{pgfscope}%
\begin{pgfscope}%
\pgfpathrectangle{\pgfqpoint{1.150000in}{0.150000in}}{\pgfqpoint{5.700000in}{5.700000in}}%
\pgfusepath{clip}%
\pgfsetbuttcap%
\pgfsetroundjoin%
\definecolor{currentfill}{rgb}{0.280894,0.078907,0.402329}%
\pgfsetfillcolor{currentfill}%
\pgfsetfillopacity{0.700000}%
\pgfsetlinewidth{0.000000pt}%
\definecolor{currentstroke}{rgb}{0.000000,0.000000,0.000000}%
\pgfsetstrokecolor{currentstroke}%
\pgfsetdash{}{0pt}%
\pgfpathmoveto{\pgfqpoint{5.042042in}{2.571371in}}%
\pgfpathlineto{\pgfqpoint{5.055672in}{2.569090in}}%
\pgfpathlineto{\pgfqpoint{5.069310in}{2.566834in}}%
\pgfpathlineto{\pgfqpoint{5.082955in}{2.564603in}}%
\pgfpathlineto{\pgfqpoint{5.096607in}{2.562398in}}%
\pgfpathlineto{\pgfqpoint{5.089258in}{2.555834in}}%
\pgfpathlineto{\pgfqpoint{5.081904in}{2.549268in}}%
\pgfpathlineto{\pgfqpoint{5.074544in}{2.542697in}}%
\pgfpathlineto{\pgfqpoint{5.067178in}{2.536118in}}%
\pgfpathlineto{\pgfqpoint{5.053511in}{2.538232in}}%
\pgfpathlineto{\pgfqpoint{5.039851in}{2.540372in}}%
\pgfpathlineto{\pgfqpoint{5.026198in}{2.542538in}}%
\pgfpathlineto{\pgfqpoint{5.012552in}{2.544728in}}%
\pgfpathlineto{\pgfqpoint{5.019933in}{2.551394in}}%
\pgfpathlineto{\pgfqpoint{5.027308in}{2.558054in}}%
\pgfpathlineto{\pgfqpoint{5.034678in}{2.564712in}}%
\pgfpathlineto{\pgfqpoint{5.042042in}{2.571371in}}%
\pgfpathclose%
\pgfusepath{fill}%
\end{pgfscope}%
\begin{pgfscope}%
\pgfpathrectangle{\pgfqpoint{1.150000in}{0.150000in}}{\pgfqpoint{5.700000in}{5.700000in}}%
\pgfusepath{clip}%
\pgfsetbuttcap%
\pgfsetroundjoin%
\definecolor{currentfill}{rgb}{0.280267,0.073417,0.397163}%
\pgfsetfillcolor{currentfill}%
\pgfsetfillopacity{0.700000}%
\pgfsetlinewidth{0.000000pt}%
\definecolor{currentstroke}{rgb}{0.000000,0.000000,0.000000}%
\pgfsetstrokecolor{currentstroke}%
\pgfsetdash{}{0pt}%
\pgfpathmoveto{\pgfqpoint{2.825571in}{2.565383in}}%
\pgfpathlineto{\pgfqpoint{2.838756in}{2.558571in}}%
\pgfpathlineto{\pgfqpoint{2.851944in}{2.551808in}}%
\pgfpathlineto{\pgfqpoint{2.865136in}{2.545095in}}%
\pgfpathlineto{\pgfqpoint{2.878330in}{2.538431in}}%
\pgfpathlineto{\pgfqpoint{2.870092in}{2.533674in}}%
\pgfpathlineto{\pgfqpoint{2.861845in}{2.529028in}}%
\pgfpathlineto{\pgfqpoint{2.853588in}{2.524496in}}%
\pgfpathlineto{\pgfqpoint{2.845321in}{2.520083in}}%
\pgfpathlineto{\pgfqpoint{2.832106in}{2.526896in}}%
\pgfpathlineto{\pgfqpoint{2.818895in}{2.533758in}}%
\pgfpathlineto{\pgfqpoint{2.805687in}{2.540670in}}%
\pgfpathlineto{\pgfqpoint{2.792482in}{2.547631in}}%
\pgfpathlineto{\pgfqpoint{2.800769in}{2.551891in}}%
\pgfpathlineto{\pgfqpoint{2.809047in}{2.556271in}}%
\pgfpathlineto{\pgfqpoint{2.817314in}{2.560770in}}%
\pgfpathlineto{\pgfqpoint{2.825571in}{2.565383in}}%
\pgfpathclose%
\pgfusepath{fill}%
\end{pgfscope}%
\begin{pgfscope}%
\pgfpathrectangle{\pgfqpoint{1.150000in}{0.150000in}}{\pgfqpoint{5.700000in}{5.700000in}}%
\pgfusepath{clip}%
\pgfsetbuttcap%
\pgfsetroundjoin%
\definecolor{currentfill}{rgb}{0.279566,0.067836,0.391917}%
\pgfsetfillcolor{currentfill}%
\pgfsetfillopacity{0.700000}%
\pgfsetlinewidth{0.000000pt}%
\definecolor{currentstroke}{rgb}{0.000000,0.000000,0.000000}%
\pgfsetstrokecolor{currentstroke}%
\pgfsetdash{}{0pt}%
\pgfpathmoveto{\pgfqpoint{4.819616in}{2.545315in}}%
\pgfpathlineto{\pgfqpoint{4.833189in}{2.542893in}}%
\pgfpathlineto{\pgfqpoint{4.846770in}{2.540497in}}%
\pgfpathlineto{\pgfqpoint{4.860357in}{2.538127in}}%
\pgfpathlineto{\pgfqpoint{4.873952in}{2.535783in}}%
\pgfpathlineto{\pgfqpoint{4.866514in}{2.528912in}}%
\pgfpathlineto{\pgfqpoint{4.859069in}{2.522026in}}%
\pgfpathlineto{\pgfqpoint{4.851619in}{2.515124in}}%
\pgfpathlineto{\pgfqpoint{4.844163in}{2.508202in}}%
\pgfpathlineto{\pgfqpoint{4.830554in}{2.510482in}}%
\pgfpathlineto{\pgfqpoint{4.816952in}{2.512787in}}%
\pgfpathlineto{\pgfqpoint{4.803357in}{2.515119in}}%
\pgfpathlineto{\pgfqpoint{4.789770in}{2.517476in}}%
\pgfpathlineto{\pgfqpoint{4.797240in}{2.524458in}}%
\pgfpathlineto{\pgfqpoint{4.804704in}{2.531423in}}%
\pgfpathlineto{\pgfqpoint{4.812163in}{2.538374in}}%
\pgfpathlineto{\pgfqpoint{4.819616in}{2.545315in}}%
\pgfpathclose%
\pgfusepath{fill}%
\end{pgfscope}%
\begin{pgfscope}%
\pgfpathrectangle{\pgfqpoint{1.150000in}{0.150000in}}{\pgfqpoint{5.700000in}{5.700000in}}%
\pgfusepath{clip}%
\pgfsetbuttcap%
\pgfsetroundjoin%
\definecolor{currentfill}{rgb}{0.268510,0.009605,0.335427}%
\pgfsetfillcolor{currentfill}%
\pgfsetfillopacity{0.700000}%
\pgfsetlinewidth{0.000000pt}%
\definecolor{currentstroke}{rgb}{0.000000,0.000000,0.000000}%
\pgfsetstrokecolor{currentstroke}%
\pgfsetdash{}{0pt}%
\pgfpathmoveto{\pgfqpoint{3.791683in}{2.451114in}}%
\pgfpathlineto{\pgfqpoint{3.805012in}{2.447187in}}%
\pgfpathlineto{\pgfqpoint{3.818346in}{2.443292in}}%
\pgfpathlineto{\pgfqpoint{3.831686in}{2.439429in}}%
\pgfpathlineto{\pgfqpoint{3.845031in}{2.435598in}}%
\pgfpathlineto{\pgfqpoint{3.837207in}{2.428073in}}%
\pgfpathlineto{\pgfqpoint{3.829378in}{2.420551in}}%
\pgfpathlineto{\pgfqpoint{3.821542in}{2.413035in}}%
\pgfpathlineto{\pgfqpoint{3.813701in}{2.405526in}}%
\pgfpathlineto{\pgfqpoint{3.800342in}{2.409412in}}%
\pgfpathlineto{\pgfqpoint{3.786989in}{2.413330in}}%
\pgfpathlineto{\pgfqpoint{3.773642in}{2.417280in}}%
\pgfpathlineto{\pgfqpoint{3.760301in}{2.421262in}}%
\pgfpathlineto{\pgfqpoint{3.768155in}{2.428712in}}%
\pgfpathlineto{\pgfqpoint{3.776004in}{2.436171in}}%
\pgfpathlineto{\pgfqpoint{3.783846in}{2.443639in}}%
\pgfpathlineto{\pgfqpoint{3.791683in}{2.451114in}}%
\pgfpathclose%
\pgfusepath{fill}%
\end{pgfscope}%
\begin{pgfscope}%
\pgfpathrectangle{\pgfqpoint{1.150000in}{0.150000in}}{\pgfqpoint{5.700000in}{5.700000in}}%
\pgfusepath{clip}%
\pgfsetbuttcap%
\pgfsetroundjoin%
\definecolor{currentfill}{rgb}{0.277018,0.050344,0.375715}%
\pgfsetfillcolor{currentfill}%
\pgfsetfillopacity{0.700000}%
\pgfsetlinewidth{0.000000pt}%
\definecolor{currentstroke}{rgb}{0.000000,0.000000,0.000000}%
\pgfsetstrokecolor{currentstroke}%
\pgfsetdash{}{0pt}%
\pgfpathmoveto{\pgfqpoint{4.597165in}{2.519089in}}%
\pgfpathlineto{\pgfqpoint{4.610682in}{2.516464in}}%
\pgfpathlineto{\pgfqpoint{4.624206in}{2.513866in}}%
\pgfpathlineto{\pgfqpoint{4.637737in}{2.511295in}}%
\pgfpathlineto{\pgfqpoint{4.651275in}{2.508751in}}%
\pgfpathlineto{\pgfqpoint{4.643749in}{2.501575in}}%
\pgfpathlineto{\pgfqpoint{4.636217in}{2.494377in}}%
\pgfpathlineto{\pgfqpoint{4.628680in}{2.487157in}}%
\pgfpathlineto{\pgfqpoint{4.621136in}{2.479913in}}%
\pgfpathlineto{\pgfqpoint{4.607585in}{2.482420in}}%
\pgfpathlineto{\pgfqpoint{4.594041in}{2.484953in}}%
\pgfpathlineto{\pgfqpoint{4.580503in}{2.487514in}}%
\pgfpathlineto{\pgfqpoint{4.566973in}{2.490101in}}%
\pgfpathlineto{\pgfqpoint{4.574530in}{2.497377in}}%
\pgfpathlineto{\pgfqpoint{4.582080in}{2.504633in}}%
\pgfpathlineto{\pgfqpoint{4.589626in}{2.511870in}}%
\pgfpathlineto{\pgfqpoint{4.597165in}{2.519089in}}%
\pgfpathclose%
\pgfusepath{fill}%
\end{pgfscope}%
\begin{pgfscope}%
\pgfpathrectangle{\pgfqpoint{1.150000in}{0.150000in}}{\pgfqpoint{5.700000in}{5.700000in}}%
\pgfusepath{clip}%
\pgfsetbuttcap%
\pgfsetroundjoin%
\definecolor{currentfill}{rgb}{0.273809,0.031497,0.358853}%
\pgfsetfillcolor{currentfill}%
\pgfsetfillopacity{0.700000}%
\pgfsetlinewidth{0.000000pt}%
\definecolor{currentstroke}{rgb}{0.000000,0.000000,0.000000}%
\pgfsetstrokecolor{currentstroke}%
\pgfsetdash{}{0pt}%
\pgfpathmoveto{\pgfqpoint{4.374698in}{2.493641in}}%
\pgfpathlineto{\pgfqpoint{4.388161in}{2.490749in}}%
\pgfpathlineto{\pgfqpoint{4.401630in}{2.487885in}}%
\pgfpathlineto{\pgfqpoint{4.415105in}{2.485049in}}%
\pgfpathlineto{\pgfqpoint{4.428588in}{2.482240in}}%
\pgfpathlineto{\pgfqpoint{4.420977in}{2.474808in}}%
\pgfpathlineto{\pgfqpoint{4.413360in}{2.467354in}}%
\pgfpathlineto{\pgfqpoint{4.405738in}{2.459878in}}%
\pgfpathlineto{\pgfqpoint{4.398110in}{2.452379in}}%
\pgfpathlineto{\pgfqpoint{4.384615in}{2.455176in}}%
\pgfpathlineto{\pgfqpoint{4.371127in}{2.458002in}}%
\pgfpathlineto{\pgfqpoint{4.357645in}{2.460855in}}%
\pgfpathlineto{\pgfqpoint{4.344170in}{2.463736in}}%
\pgfpathlineto{\pgfqpoint{4.351811in}{2.471241in}}%
\pgfpathlineto{\pgfqpoint{4.359445in}{2.478727in}}%
\pgfpathlineto{\pgfqpoint{4.367075in}{2.486193in}}%
\pgfpathlineto{\pgfqpoint{4.374698in}{2.493641in}}%
\pgfpathclose%
\pgfusepath{fill}%
\end{pgfscope}%
\begin{pgfscope}%
\pgfpathrectangle{\pgfqpoint{1.150000in}{0.150000in}}{\pgfqpoint{5.700000in}{5.700000in}}%
\pgfusepath{clip}%
\pgfsetbuttcap%
\pgfsetroundjoin%
\definecolor{currentfill}{rgb}{0.271305,0.019942,0.347269}%
\pgfsetfillcolor{currentfill}%
\pgfsetfillopacity{0.700000}%
\pgfsetlinewidth{0.000000pt}%
\definecolor{currentstroke}{rgb}{0.000000,0.000000,0.000000}%
\pgfsetstrokecolor{currentstroke}%
\pgfsetdash{}{0pt}%
\pgfpathmoveto{\pgfqpoint{4.152207in}{2.470322in}}%
\pgfpathlineto{\pgfqpoint{4.165617in}{2.467095in}}%
\pgfpathlineto{\pgfqpoint{4.179034in}{2.463897in}}%
\pgfpathlineto{\pgfqpoint{4.192457in}{2.460729in}}%
\pgfpathlineto{\pgfqpoint{4.205886in}{2.457589in}}%
\pgfpathlineto{\pgfqpoint{4.198192in}{2.449997in}}%
\pgfpathlineto{\pgfqpoint{4.190493in}{2.442389in}}%
\pgfpathlineto{\pgfqpoint{4.182788in}{2.434764in}}%
\pgfpathlineto{\pgfqpoint{4.175078in}{2.427123in}}%
\pgfpathlineto{\pgfqpoint{4.161636in}{2.430278in}}%
\pgfpathlineto{\pgfqpoint{4.148201in}{2.433462in}}%
\pgfpathlineto{\pgfqpoint{4.134772in}{2.436675in}}%
\pgfpathlineto{\pgfqpoint{4.121349in}{2.439918in}}%
\pgfpathlineto{\pgfqpoint{4.129072in}{2.447538in}}%
\pgfpathlineto{\pgfqpoint{4.136789in}{2.455146in}}%
\pgfpathlineto{\pgfqpoint{4.144501in}{2.462740in}}%
\pgfpathlineto{\pgfqpoint{4.152207in}{2.470322in}}%
\pgfpathclose%
\pgfusepath{fill}%
\end{pgfscope}%
\begin{pgfscope}%
\pgfpathrectangle{\pgfqpoint{1.150000in}{0.150000in}}{\pgfqpoint{5.700000in}{5.700000in}}%
\pgfusepath{clip}%
\pgfsetbuttcap%
\pgfsetroundjoin%
\definecolor{currentfill}{rgb}{0.269944,0.014625,0.341379}%
\pgfsetfillcolor{currentfill}%
\pgfsetfillopacity{0.700000}%
\pgfsetlinewidth{0.000000pt}%
\definecolor{currentstroke}{rgb}{0.000000,0.000000,0.000000}%
\pgfsetstrokecolor{currentstroke}%
\pgfsetdash{}{0pt}%
\pgfpathmoveto{\pgfqpoint{3.293002in}{2.465360in}}%
\pgfpathlineto{\pgfqpoint{3.306243in}{2.460145in}}%
\pgfpathlineto{\pgfqpoint{3.319488in}{2.454968in}}%
\pgfpathlineto{\pgfqpoint{3.332738in}{2.449830in}}%
\pgfpathlineto{\pgfqpoint{3.345993in}{2.444730in}}%
\pgfpathlineto{\pgfqpoint{3.337972in}{2.438157in}}%
\pgfpathlineto{\pgfqpoint{3.329945in}{2.431636in}}%
\pgfpathlineto{\pgfqpoint{3.321910in}{2.425171in}}%
\pgfpathlineto{\pgfqpoint{3.313868in}{2.418764in}}%
\pgfpathlineto{\pgfqpoint{3.300598in}{2.423972in}}%
\pgfpathlineto{\pgfqpoint{3.287333in}{2.429218in}}%
\pgfpathlineto{\pgfqpoint{3.274072in}{2.434503in}}%
\pgfpathlineto{\pgfqpoint{3.260815in}{2.439826in}}%
\pgfpathlineto{\pgfqpoint{3.268873in}{2.446120in}}%
\pgfpathlineto{\pgfqpoint{3.276923in}{2.452476in}}%
\pgfpathlineto{\pgfqpoint{3.284966in}{2.458890in}}%
\pgfpathlineto{\pgfqpoint{3.293002in}{2.465360in}}%
\pgfpathclose%
\pgfusepath{fill}%
\end{pgfscope}%
\begin{pgfscope}%
\pgfpathrectangle{\pgfqpoint{1.150000in}{0.150000in}}{\pgfqpoint{5.700000in}{5.700000in}}%
\pgfusepath{clip}%
\pgfsetbuttcap%
\pgfsetroundjoin%
\definecolor{currentfill}{rgb}{0.272594,0.025563,0.353093}%
\pgfsetfillcolor{currentfill}%
\pgfsetfillopacity{0.700000}%
\pgfsetlinewidth{0.000000pt}%
\definecolor{currentstroke}{rgb}{0.000000,0.000000,0.000000}%
\pgfsetstrokecolor{currentstroke}%
\pgfsetdash{}{0pt}%
\pgfpathmoveto{\pgfqpoint{3.154921in}{2.483846in}}%
\pgfpathlineto{\pgfqpoint{3.168143in}{2.478202in}}%
\pgfpathlineto{\pgfqpoint{3.181369in}{2.472598in}}%
\pgfpathlineto{\pgfqpoint{3.194599in}{2.467035in}}%
\pgfpathlineto{\pgfqpoint{3.207834in}{2.461514in}}%
\pgfpathlineto{\pgfqpoint{3.199753in}{2.455400in}}%
\pgfpathlineto{\pgfqpoint{3.191664in}{2.449356in}}%
\pgfpathlineto{\pgfqpoint{3.183568in}{2.443384in}}%
\pgfpathlineto{\pgfqpoint{3.175463in}{2.437488in}}%
\pgfpathlineto{\pgfqpoint{3.162212in}{2.443131in}}%
\pgfpathlineto{\pgfqpoint{3.148965in}{2.448815in}}%
\pgfpathlineto{\pgfqpoint{3.135722in}{2.454540in}}%
\pgfpathlineto{\pgfqpoint{3.122484in}{2.460306in}}%
\pgfpathlineto{\pgfqpoint{3.130605in}{2.466076in}}%
\pgfpathlineto{\pgfqpoint{3.138718in}{2.471925in}}%
\pgfpathlineto{\pgfqpoint{3.146824in}{2.477849in}}%
\pgfpathlineto{\pgfqpoint{3.154921in}{2.483846in}}%
\pgfpathclose%
\pgfusepath{fill}%
\end{pgfscope}%
\begin{pgfscope}%
\pgfpathrectangle{\pgfqpoint{1.150000in}{0.150000in}}{\pgfqpoint{5.700000in}{5.700000in}}%
\pgfusepath{clip}%
\pgfsetbuttcap%
\pgfsetroundjoin%
\definecolor{currentfill}{rgb}{0.268510,0.009605,0.335427}%
\pgfsetfillcolor{currentfill}%
\pgfsetfillopacity{0.700000}%
\pgfsetlinewidth{0.000000pt}%
\definecolor{currentstroke}{rgb}{0.000000,0.000000,0.000000}%
\pgfsetstrokecolor{currentstroke}%
\pgfsetdash{}{0pt}%
\pgfpathmoveto{\pgfqpoint{3.431012in}{2.451867in}}%
\pgfpathlineto{\pgfqpoint{3.444276in}{2.447048in}}%
\pgfpathlineto{\pgfqpoint{3.457544in}{2.442266in}}%
\pgfpathlineto{\pgfqpoint{3.470817in}{2.437520in}}%
\pgfpathlineto{\pgfqpoint{3.484096in}{2.432810in}}%
\pgfpathlineto{\pgfqpoint{3.476132in}{2.425867in}}%
\pgfpathlineto{\pgfqpoint{3.468161in}{2.418961in}}%
\pgfpathlineto{\pgfqpoint{3.460184in}{2.412095in}}%
\pgfpathlineto{\pgfqpoint{3.452200in}{2.405271in}}%
\pgfpathlineto{\pgfqpoint{3.438907in}{2.410076in}}%
\pgfpathlineto{\pgfqpoint{3.425619in}{2.414916in}}%
\pgfpathlineto{\pgfqpoint{3.412336in}{2.419793in}}%
\pgfpathlineto{\pgfqpoint{3.399058in}{2.424706in}}%
\pgfpathlineto{\pgfqpoint{3.407056in}{2.431431in}}%
\pgfpathlineto{\pgfqpoint{3.415048in}{2.438200in}}%
\pgfpathlineto{\pgfqpoint{3.423033in}{2.445013in}}%
\pgfpathlineto{\pgfqpoint{3.431012in}{2.451867in}}%
\pgfpathclose%
\pgfusepath{fill}%
\end{pgfscope}%
\begin{pgfscope}%
\pgfpathrectangle{\pgfqpoint{1.150000in}{0.150000in}}{\pgfqpoint{5.700000in}{5.700000in}}%
\pgfusepath{clip}%
\pgfsetbuttcap%
\pgfsetroundjoin%
\definecolor{currentfill}{rgb}{0.283197,0.115680,0.436115}%
\pgfsetfillcolor{currentfill}%
\pgfsetfillopacity{0.700000}%
\pgfsetlinewidth{0.000000pt}%
\definecolor{currentstroke}{rgb}{0.000000,0.000000,0.000000}%
\pgfsetstrokecolor{currentstroke}%
\pgfsetdash{}{0pt}%
\pgfpathmoveto{\pgfqpoint{5.625460in}{2.628806in}}%
\pgfpathlineto{\pgfqpoint{5.639244in}{2.626662in}}%
\pgfpathlineto{\pgfqpoint{5.653035in}{2.624541in}}%
\pgfpathlineto{\pgfqpoint{5.666834in}{2.622445in}}%
\pgfpathlineto{\pgfqpoint{5.680641in}{2.620373in}}%
\pgfpathlineto{\pgfqpoint{5.673530in}{2.614360in}}%
\pgfpathlineto{\pgfqpoint{5.666414in}{2.608399in}}%
\pgfpathlineto{\pgfqpoint{5.659295in}{2.602486in}}%
\pgfpathlineto{\pgfqpoint{5.652172in}{2.596616in}}%
\pgfpathlineto{\pgfqpoint{5.638346in}{2.598531in}}%
\pgfpathlineto{\pgfqpoint{5.624527in}{2.600470in}}%
\pgfpathlineto{\pgfqpoint{5.610716in}{2.602433in}}%
\pgfpathlineto{\pgfqpoint{5.596914in}{2.604420in}}%
\pgfpathlineto{\pgfqpoint{5.604056in}{2.610442in}}%
\pgfpathlineto{\pgfqpoint{5.611195in}{2.616511in}}%
\pgfpathlineto{\pgfqpoint{5.618329in}{2.622630in}}%
\pgfpathlineto{\pgfqpoint{5.625460in}{2.628806in}}%
\pgfpathclose%
\pgfusepath{fill}%
\end{pgfscope}%
\begin{pgfscope}%
\pgfpathrectangle{\pgfqpoint{1.150000in}{0.150000in}}{\pgfqpoint{5.700000in}{5.700000in}}%
\pgfusepath{clip}%
\pgfsetbuttcap%
\pgfsetroundjoin%
\definecolor{currentfill}{rgb}{0.268510,0.009605,0.335427}%
\pgfsetfillcolor{currentfill}%
\pgfsetfillopacity{0.700000}%
\pgfsetlinewidth{0.000000pt}%
\definecolor{currentstroke}{rgb}{0.000000,0.000000,0.000000}%
\pgfsetstrokecolor{currentstroke}%
\pgfsetdash{}{0pt}%
\pgfpathmoveto{\pgfqpoint{3.929659in}{2.450888in}}%
\pgfpathlineto{\pgfqpoint{3.943021in}{2.447256in}}%
\pgfpathlineto{\pgfqpoint{3.956389in}{2.443655in}}%
\pgfpathlineto{\pgfqpoint{3.969763in}{2.440084in}}%
\pgfpathlineto{\pgfqpoint{3.983143in}{2.436544in}}%
\pgfpathlineto{\pgfqpoint{3.975367in}{2.428934in}}%
\pgfpathlineto{\pgfqpoint{3.967586in}{2.421319in}}%
\pgfpathlineto{\pgfqpoint{3.959798in}{2.413700in}}%
\pgfpathlineto{\pgfqpoint{3.952005in}{2.406077in}}%
\pgfpathlineto{\pgfqpoint{3.938613in}{2.409659in}}%
\pgfpathlineto{\pgfqpoint{3.925227in}{2.413272in}}%
\pgfpathlineto{\pgfqpoint{3.911846in}{2.416915in}}%
\pgfpathlineto{\pgfqpoint{3.898471in}{2.420590in}}%
\pgfpathlineto{\pgfqpoint{3.906277in}{2.428166in}}%
\pgfpathlineto{\pgfqpoint{3.914077in}{2.435741in}}%
\pgfpathlineto{\pgfqpoint{3.921871in}{2.443316in}}%
\pgfpathlineto{\pgfqpoint{3.929659in}{2.450888in}}%
\pgfpathclose%
\pgfusepath{fill}%
\end{pgfscope}%
\begin{pgfscope}%
\pgfpathrectangle{\pgfqpoint{1.150000in}{0.150000in}}{\pgfqpoint{5.700000in}{5.700000in}}%
\pgfusepath{clip}%
\pgfsetbuttcap%
\pgfsetroundjoin%
\definecolor{currentfill}{rgb}{0.276022,0.044167,0.370164}%
\pgfsetfillcolor{currentfill}%
\pgfsetfillopacity{0.700000}%
\pgfsetlinewidth{0.000000pt}%
\definecolor{currentstroke}{rgb}{0.000000,0.000000,0.000000}%
\pgfsetstrokecolor{currentstroke}%
\pgfsetdash{}{0pt}%
\pgfpathmoveto{\pgfqpoint{3.016717in}{2.507970in}}%
\pgfpathlineto{\pgfqpoint{3.029925in}{2.501860in}}%
\pgfpathlineto{\pgfqpoint{3.043136in}{2.495793in}}%
\pgfpathlineto{\pgfqpoint{3.056351in}{2.489771in}}%
\pgfpathlineto{\pgfqpoint{3.069569in}{2.483793in}}%
\pgfpathlineto{\pgfqpoint{3.061422in}{2.478234in}}%
\pgfpathlineto{\pgfqpoint{3.053267in}{2.472764in}}%
\pgfpathlineto{\pgfqpoint{3.045104in}{2.467385in}}%
\pgfpathlineto{\pgfqpoint{3.036931in}{2.462100in}}%
\pgfpathlineto{\pgfqpoint{3.023695in}{2.468213in}}%
\pgfpathlineto{\pgfqpoint{3.010462in}{2.474370in}}%
\pgfpathlineto{\pgfqpoint{2.997232in}{2.480572in}}%
\pgfpathlineto{\pgfqpoint{2.984007in}{2.486817in}}%
\pgfpathlineto{\pgfqpoint{2.992198in}{2.491962in}}%
\pgfpathlineto{\pgfqpoint{3.000379in}{2.497205in}}%
\pgfpathlineto{\pgfqpoint{3.008553in}{2.502542in}}%
\pgfpathlineto{\pgfqpoint{3.016717in}{2.507970in}}%
\pgfpathclose%
\pgfusepath{fill}%
\end{pgfscope}%
\begin{pgfscope}%
\pgfpathrectangle{\pgfqpoint{1.150000in}{0.150000in}}{\pgfqpoint{5.700000in}{5.700000in}}%
\pgfusepath{clip}%
\pgfsetbuttcap%
\pgfsetroundjoin%
\definecolor{currentfill}{rgb}{0.267004,0.004874,0.329415}%
\pgfsetfillcolor{currentfill}%
\pgfsetfillopacity{0.700000}%
\pgfsetlinewidth{0.000000pt}%
\definecolor{currentstroke}{rgb}{0.000000,0.000000,0.000000}%
\pgfsetstrokecolor{currentstroke}%
\pgfsetdash{}{0pt}%
\pgfpathmoveto{\pgfqpoint{3.568995in}{2.442769in}}%
\pgfpathlineto{\pgfqpoint{3.582285in}{2.438317in}}%
\pgfpathlineto{\pgfqpoint{3.595580in}{2.433900in}}%
\pgfpathlineto{\pgfqpoint{3.608881in}{2.429517in}}%
\pgfpathlineto{\pgfqpoint{3.622186in}{2.425168in}}%
\pgfpathlineto{\pgfqpoint{3.614275in}{2.417938in}}%
\pgfpathlineto{\pgfqpoint{3.606358in}{2.410732in}}%
\pgfpathlineto{\pgfqpoint{3.598435in}{2.403552in}}%
\pgfpathlineto{\pgfqpoint{3.590505in}{2.396399in}}%
\pgfpathlineto{\pgfqpoint{3.577186in}{2.400830in}}%
\pgfpathlineto{\pgfqpoint{3.563872in}{2.405294in}}%
\pgfpathlineto{\pgfqpoint{3.550563in}{2.409793in}}%
\pgfpathlineto{\pgfqpoint{3.537259in}{2.414326in}}%
\pgfpathlineto{\pgfqpoint{3.545203in}{2.421392in}}%
\pgfpathlineto{\pgfqpoint{3.553140in}{2.428489in}}%
\pgfpathlineto{\pgfqpoint{3.561071in}{2.435615in}}%
\pgfpathlineto{\pgfqpoint{3.568995in}{2.442769in}}%
\pgfpathclose%
\pgfusepath{fill}%
\end{pgfscope}%
\begin{pgfscope}%
\pgfpathrectangle{\pgfqpoint{1.150000in}{0.150000in}}{\pgfqpoint{5.700000in}{5.700000in}}%
\pgfusepath{clip}%
\pgfsetbuttcap%
\pgfsetroundjoin%
\definecolor{currentfill}{rgb}{0.282656,0.100196,0.422160}%
\pgfsetfillcolor{currentfill}%
\pgfsetfillopacity{0.700000}%
\pgfsetlinewidth{0.000000pt}%
\definecolor{currentstroke}{rgb}{0.000000,0.000000,0.000000}%
\pgfsetstrokecolor{currentstroke}%
\pgfsetdash{}{0pt}%
\pgfpathmoveto{\pgfqpoint{5.403042in}{2.604708in}}%
\pgfpathlineto{\pgfqpoint{5.416771in}{2.602573in}}%
\pgfpathlineto{\pgfqpoint{5.430508in}{2.600463in}}%
\pgfpathlineto{\pgfqpoint{5.444252in}{2.598377in}}%
\pgfpathlineto{\pgfqpoint{5.458004in}{2.596315in}}%
\pgfpathlineto{\pgfqpoint{5.450802in}{2.590182in}}%
\pgfpathlineto{\pgfqpoint{5.443595in}{2.584075in}}%
\pgfpathlineto{\pgfqpoint{5.436383in}{2.577989in}}%
\pgfpathlineto{\pgfqpoint{5.429166in}{2.571921in}}%
\pgfpathlineto{\pgfqpoint{5.415396in}{2.573851in}}%
\pgfpathlineto{\pgfqpoint{5.401634in}{2.575807in}}%
\pgfpathlineto{\pgfqpoint{5.387879in}{2.577786in}}%
\pgfpathlineto{\pgfqpoint{5.374132in}{2.579791in}}%
\pgfpathlineto{\pgfqpoint{5.381367in}{2.585985in}}%
\pgfpathlineto{\pgfqpoint{5.388597in}{2.592200in}}%
\pgfpathlineto{\pgfqpoint{5.395822in}{2.598440in}}%
\pgfpathlineto{\pgfqpoint{5.403042in}{2.604708in}}%
\pgfpathclose%
\pgfusepath{fill}%
\end{pgfscope}%
\begin{pgfscope}%
\pgfpathrectangle{\pgfqpoint{1.150000in}{0.150000in}}{\pgfqpoint{5.700000in}{5.700000in}}%
\pgfusepath{clip}%
\pgfsetbuttcap%
\pgfsetroundjoin%
\definecolor{currentfill}{rgb}{0.281924,0.089666,0.412415}%
\pgfsetfillcolor{currentfill}%
\pgfsetfillopacity{0.700000}%
\pgfsetlinewidth{0.000000pt}%
\definecolor{currentstroke}{rgb}{0.000000,0.000000,0.000000}%
\pgfsetstrokecolor{currentstroke}%
\pgfsetdash{}{0pt}%
\pgfpathmoveto{\pgfqpoint{5.180566in}{2.579706in}}%
\pgfpathlineto{\pgfqpoint{5.194240in}{2.577522in}}%
\pgfpathlineto{\pgfqpoint{5.207920in}{2.575364in}}%
\pgfpathlineto{\pgfqpoint{5.221608in}{2.573229in}}%
\pgfpathlineto{\pgfqpoint{5.235304in}{2.571120in}}%
\pgfpathlineto{\pgfqpoint{5.228009in}{2.564752in}}%
\pgfpathlineto{\pgfqpoint{5.220709in}{2.558387in}}%
\pgfpathlineto{\pgfqpoint{5.213403in}{2.552024in}}%
\pgfpathlineto{\pgfqpoint{5.206092in}{2.545658in}}%
\pgfpathlineto{\pgfqpoint{5.192380in}{2.547663in}}%
\pgfpathlineto{\pgfqpoint{5.178676in}{2.549693in}}%
\pgfpathlineto{\pgfqpoint{5.164979in}{2.551748in}}%
\pgfpathlineto{\pgfqpoint{5.151290in}{2.553827in}}%
\pgfpathlineto{\pgfqpoint{5.158617in}{2.560292in}}%
\pgfpathlineto{\pgfqpoint{5.165939in}{2.566758in}}%
\pgfpathlineto{\pgfqpoint{5.173255in}{2.573229in}}%
\pgfpathlineto{\pgfqpoint{5.180566in}{2.579706in}}%
\pgfpathclose%
\pgfusepath{fill}%
\end{pgfscope}%
\begin{pgfscope}%
\pgfpathrectangle{\pgfqpoint{1.150000in}{0.150000in}}{\pgfqpoint{5.700000in}{5.700000in}}%
\pgfusepath{clip}%
\pgfsetbuttcap%
\pgfsetroundjoin%
\definecolor{currentfill}{rgb}{0.280267,0.073417,0.397163}%
\pgfsetfillcolor{currentfill}%
\pgfsetfillopacity{0.700000}%
\pgfsetlinewidth{0.000000pt}%
\definecolor{currentstroke}{rgb}{0.000000,0.000000,0.000000}%
\pgfsetstrokecolor{currentstroke}%
\pgfsetdash{}{0pt}%
\pgfpathmoveto{\pgfqpoint{4.958044in}{2.553745in}}%
\pgfpathlineto{\pgfqpoint{4.971660in}{2.551453in}}%
\pgfpathlineto{\pgfqpoint{4.985283in}{2.549186in}}%
\pgfpathlineto{\pgfqpoint{4.998914in}{2.546944in}}%
\pgfpathlineto{\pgfqpoint{5.012552in}{2.544728in}}%
\pgfpathlineto{\pgfqpoint{5.005166in}{2.538055in}}%
\pgfpathlineto{\pgfqpoint{4.997774in}{2.531371in}}%
\pgfpathlineto{\pgfqpoint{4.990376in}{2.524673in}}%
\pgfpathlineto{\pgfqpoint{4.982972in}{2.517959in}}%
\pgfpathlineto{\pgfqpoint{4.969319in}{2.520097in}}%
\pgfpathlineto{\pgfqpoint{4.955673in}{2.522261in}}%
\pgfpathlineto{\pgfqpoint{4.942035in}{2.524451in}}%
\pgfpathlineto{\pgfqpoint{4.928404in}{2.526666in}}%
\pgfpathlineto{\pgfqpoint{4.935822in}{2.533452in}}%
\pgfpathlineto{\pgfqpoint{4.943235in}{2.540226in}}%
\pgfpathlineto{\pgfqpoint{4.950642in}{2.546990in}}%
\pgfpathlineto{\pgfqpoint{4.958044in}{2.553745in}}%
\pgfpathclose%
\pgfusepath{fill}%
\end{pgfscope}%
\begin{pgfscope}%
\pgfpathrectangle{\pgfqpoint{1.150000in}{0.150000in}}{\pgfqpoint{5.700000in}{5.700000in}}%
\pgfusepath{clip}%
\pgfsetbuttcap%
\pgfsetroundjoin%
\definecolor{currentfill}{rgb}{0.278791,0.062145,0.386592}%
\pgfsetfillcolor{currentfill}%
\pgfsetfillopacity{0.700000}%
\pgfsetlinewidth{0.000000pt}%
\definecolor{currentstroke}{rgb}{0.000000,0.000000,0.000000}%
\pgfsetstrokecolor{currentstroke}%
\pgfsetdash{}{0pt}%
\pgfpathmoveto{\pgfqpoint{4.735490in}{2.527169in}}%
\pgfpathlineto{\pgfqpoint{4.749049in}{2.524706in}}%
\pgfpathlineto{\pgfqpoint{4.762616in}{2.522270in}}%
\pgfpathlineto{\pgfqpoint{4.776189in}{2.519860in}}%
\pgfpathlineto{\pgfqpoint{4.789770in}{2.517476in}}%
\pgfpathlineto{\pgfqpoint{4.782294in}{2.510477in}}%
\pgfpathlineto{\pgfqpoint{4.774812in}{2.503458in}}%
\pgfpathlineto{\pgfqpoint{4.767324in}{2.496417in}}%
\pgfpathlineto{\pgfqpoint{4.759831in}{2.489351in}}%
\pgfpathlineto{\pgfqpoint{4.746237in}{2.491684in}}%
\pgfpathlineto{\pgfqpoint{4.732650in}{2.494043in}}%
\pgfpathlineto{\pgfqpoint{4.719070in}{2.496428in}}%
\pgfpathlineto{\pgfqpoint{4.705497in}{2.498840in}}%
\pgfpathlineto{\pgfqpoint{4.713004in}{2.505951in}}%
\pgfpathlineto{\pgfqpoint{4.720505in}{2.513042in}}%
\pgfpathlineto{\pgfqpoint{4.728000in}{2.520114in}}%
\pgfpathlineto{\pgfqpoint{4.735490in}{2.527169in}}%
\pgfpathclose%
\pgfusepath{fill}%
\end{pgfscope}%
\begin{pgfscope}%
\pgfpathrectangle{\pgfqpoint{1.150000in}{0.150000in}}{\pgfqpoint{5.700000in}{5.700000in}}%
\pgfusepath{clip}%
\pgfsetbuttcap%
\pgfsetroundjoin%
\definecolor{currentfill}{rgb}{0.276022,0.044167,0.370164}%
\pgfsetfillcolor{currentfill}%
\pgfsetfillopacity{0.700000}%
\pgfsetlinewidth{0.000000pt}%
\definecolor{currentstroke}{rgb}{0.000000,0.000000,0.000000}%
\pgfsetstrokecolor{currentstroke}%
\pgfsetdash{}{0pt}%
\pgfpathmoveto{\pgfqpoint{4.512920in}{2.500720in}}%
\pgfpathlineto{\pgfqpoint{4.526423in}{2.498024in}}%
\pgfpathlineto{\pgfqpoint{4.539933in}{2.495356in}}%
\pgfpathlineto{\pgfqpoint{4.553449in}{2.492715in}}%
\pgfpathlineto{\pgfqpoint{4.566973in}{2.490101in}}%
\pgfpathlineto{\pgfqpoint{4.559411in}{2.482802in}}%
\pgfpathlineto{\pgfqpoint{4.551842in}{2.475480in}}%
\pgfpathlineto{\pgfqpoint{4.544269in}{2.468132in}}%
\pgfpathlineto{\pgfqpoint{4.536689in}{2.460759in}}%
\pgfpathlineto{\pgfqpoint{4.523152in}{2.463349in}}%
\pgfpathlineto{\pgfqpoint{4.509623in}{2.465966in}}%
\pgfpathlineto{\pgfqpoint{4.496100in}{2.468610in}}%
\pgfpathlineto{\pgfqpoint{4.482584in}{2.471281in}}%
\pgfpathlineto{\pgfqpoint{4.490177in}{2.478674in}}%
\pgfpathlineto{\pgfqpoint{4.497763in}{2.486043in}}%
\pgfpathlineto{\pgfqpoint{4.505344in}{2.493392in}}%
\pgfpathlineto{\pgfqpoint{4.512920in}{2.500720in}}%
\pgfpathclose%
\pgfusepath{fill}%
\end{pgfscope}%
\begin{pgfscope}%
\pgfpathrectangle{\pgfqpoint{1.150000in}{0.150000in}}{\pgfqpoint{5.700000in}{5.700000in}}%
\pgfusepath{clip}%
\pgfsetbuttcap%
\pgfsetroundjoin%
\definecolor{currentfill}{rgb}{0.272594,0.025563,0.353093}%
\pgfsetfillcolor{currentfill}%
\pgfsetfillopacity{0.700000}%
\pgfsetlinewidth{0.000000pt}%
\definecolor{currentstroke}{rgb}{0.000000,0.000000,0.000000}%
\pgfsetstrokecolor{currentstroke}%
\pgfsetdash{}{0pt}%
\pgfpathmoveto{\pgfqpoint{4.290335in}{2.475542in}}%
\pgfpathlineto{\pgfqpoint{4.303784in}{2.472548in}}%
\pgfpathlineto{\pgfqpoint{4.317240in}{2.469582in}}%
\pgfpathlineto{\pgfqpoint{4.330702in}{2.466645in}}%
\pgfpathlineto{\pgfqpoint{4.344170in}{2.463736in}}%
\pgfpathlineto{\pgfqpoint{4.336524in}{2.456210in}}%
\pgfpathlineto{\pgfqpoint{4.328872in}{2.448663in}}%
\pgfpathlineto{\pgfqpoint{4.321214in}{2.441095in}}%
\pgfpathlineto{\pgfqpoint{4.313551in}{2.433504in}}%
\pgfpathlineto{\pgfqpoint{4.300070in}{2.436416in}}%
\pgfpathlineto{\pgfqpoint{4.286596in}{2.439355in}}%
\pgfpathlineto{\pgfqpoint{4.273128in}{2.442323in}}%
\pgfpathlineto{\pgfqpoint{4.259667in}{2.445319in}}%
\pgfpathlineto{\pgfqpoint{4.267342in}{2.452902in}}%
\pgfpathlineto{\pgfqpoint{4.275012in}{2.460467in}}%
\pgfpathlineto{\pgfqpoint{4.282677in}{2.468013in}}%
\pgfpathlineto{\pgfqpoint{4.290335in}{2.475542in}}%
\pgfpathclose%
\pgfusepath{fill}%
\end{pgfscope}%
\begin{pgfscope}%
\pgfpathrectangle{\pgfqpoint{1.150000in}{0.150000in}}{\pgfqpoint{5.700000in}{5.700000in}}%
\pgfusepath{clip}%
\pgfsetbuttcap%
\pgfsetroundjoin%
\definecolor{currentfill}{rgb}{0.267004,0.004874,0.329415}%
\pgfsetfillcolor{currentfill}%
\pgfsetfillopacity{0.700000}%
\pgfsetlinewidth{0.000000pt}%
\definecolor{currentstroke}{rgb}{0.000000,0.000000,0.000000}%
\pgfsetstrokecolor{currentstroke}%
\pgfsetdash{}{0pt}%
\pgfpathmoveto{\pgfqpoint{3.706991in}{2.437516in}}%
\pgfpathlineto{\pgfqpoint{3.720310in}{2.433403in}}%
\pgfpathlineto{\pgfqpoint{3.733635in}{2.429324in}}%
\pgfpathlineto{\pgfqpoint{3.746965in}{2.425276in}}%
\pgfpathlineto{\pgfqpoint{3.760301in}{2.421262in}}%
\pgfpathlineto{\pgfqpoint{3.752441in}{2.413823in}}%
\pgfpathlineto{\pgfqpoint{3.744575in}{2.406396in}}%
\pgfpathlineto{\pgfqpoint{3.736703in}{2.398982in}}%
\pgfpathlineto{\pgfqpoint{3.728825in}{2.391583in}}%
\pgfpathlineto{\pgfqpoint{3.715476in}{2.395666in}}%
\pgfpathlineto{\pgfqpoint{3.702132in}{2.399781in}}%
\pgfpathlineto{\pgfqpoint{3.688794in}{2.403929in}}%
\pgfpathlineto{\pgfqpoint{3.675462in}{2.408110in}}%
\pgfpathlineto{\pgfqpoint{3.683353in}{2.415436in}}%
\pgfpathlineto{\pgfqpoint{3.691238in}{2.422780in}}%
\pgfpathlineto{\pgfqpoint{3.699118in}{2.430141in}}%
\pgfpathlineto{\pgfqpoint{3.706991in}{2.437516in}}%
\pgfpathclose%
\pgfusepath{fill}%
\end{pgfscope}%
\begin{pgfscope}%
\pgfpathrectangle{\pgfqpoint{1.150000in}{0.150000in}}{\pgfqpoint{5.700000in}{5.700000in}}%
\pgfusepath{clip}%
\pgfsetbuttcap%
\pgfsetroundjoin%
\definecolor{currentfill}{rgb}{0.279566,0.067836,0.391917}%
\pgfsetfillcolor{currentfill}%
\pgfsetfillopacity{0.700000}%
\pgfsetlinewidth{0.000000pt}%
\definecolor{currentstroke}{rgb}{0.000000,0.000000,0.000000}%
\pgfsetstrokecolor{currentstroke}%
\pgfsetdash{}{0pt}%
\pgfpathmoveto{\pgfqpoint{2.878330in}{2.538431in}}%
\pgfpathlineto{\pgfqpoint{2.891528in}{2.531814in}}%
\pgfpathlineto{\pgfqpoint{2.904729in}{2.525246in}}%
\pgfpathlineto{\pgfqpoint{2.917933in}{2.518725in}}%
\pgfpathlineto{\pgfqpoint{2.931141in}{2.512252in}}%
\pgfpathlineto{\pgfqpoint{2.922922in}{2.507351in}}%
\pgfpathlineto{\pgfqpoint{2.914694in}{2.502558in}}%
\pgfpathlineto{\pgfqpoint{2.906457in}{2.497876in}}%
\pgfpathlineto{\pgfqpoint{2.898210in}{2.493310in}}%
\pgfpathlineto{\pgfqpoint{2.884983in}{2.499932in}}%
\pgfpathlineto{\pgfqpoint{2.871759in}{2.506601in}}%
\pgfpathlineto{\pgfqpoint{2.858538in}{2.513318in}}%
\pgfpathlineto{\pgfqpoint{2.845321in}{2.520083in}}%
\pgfpathlineto{\pgfqpoint{2.853588in}{2.524496in}}%
\pgfpathlineto{\pgfqpoint{2.861845in}{2.529028in}}%
\pgfpathlineto{\pgfqpoint{2.870092in}{2.533674in}}%
\pgfpathlineto{\pgfqpoint{2.878330in}{2.538431in}}%
\pgfpathclose%
\pgfusepath{fill}%
\end{pgfscope}%
\begin{pgfscope}%
\pgfpathrectangle{\pgfqpoint{1.150000in}{0.150000in}}{\pgfqpoint{5.700000in}{5.700000in}}%
\pgfusepath{clip}%
\pgfsetbuttcap%
\pgfsetroundjoin%
\definecolor{currentfill}{rgb}{0.269944,0.014625,0.341379}%
\pgfsetfillcolor{currentfill}%
\pgfsetfillopacity{0.700000}%
\pgfsetlinewidth{0.000000pt}%
\definecolor{currentstroke}{rgb}{0.000000,0.000000,0.000000}%
\pgfsetstrokecolor{currentstroke}%
\pgfsetdash{}{0pt}%
\pgfpathmoveto{\pgfqpoint{4.067721in}{2.453182in}}%
\pgfpathlineto{\pgfqpoint{4.081119in}{2.449822in}}%
\pgfpathlineto{\pgfqpoint{4.094523in}{2.446491in}}%
\pgfpathlineto{\pgfqpoint{4.107933in}{2.443190in}}%
\pgfpathlineto{\pgfqpoint{4.121349in}{2.439918in}}%
\pgfpathlineto{\pgfqpoint{4.113621in}{2.432284in}}%
\pgfpathlineto{\pgfqpoint{4.105886in}{2.424638in}}%
\pgfpathlineto{\pgfqpoint{4.098147in}{2.416979in}}%
\pgfpathlineto{\pgfqpoint{4.090401in}{2.409307in}}%
\pgfpathlineto{\pgfqpoint{4.076972in}{2.412608in}}%
\pgfpathlineto{\pgfqpoint{4.063550in}{2.415938in}}%
\pgfpathlineto{\pgfqpoint{4.050133in}{2.419297in}}%
\pgfpathlineto{\pgfqpoint{4.036723in}{2.422687in}}%
\pgfpathlineto{\pgfqpoint{4.044481in}{2.430324in}}%
\pgfpathlineto{\pgfqpoint{4.052233in}{2.437953in}}%
\pgfpathlineto{\pgfqpoint{4.059980in}{2.445573in}}%
\pgfpathlineto{\pgfqpoint{4.067721in}{2.453182in}}%
\pgfpathclose%
\pgfusepath{fill}%
\end{pgfscope}%
\begin{pgfscope}%
\pgfpathrectangle{\pgfqpoint{1.150000in}{0.150000in}}{\pgfqpoint{5.700000in}{5.700000in}}%
\pgfusepath{clip}%
\pgfsetbuttcap%
\pgfsetroundjoin%
\definecolor{currentfill}{rgb}{0.283229,0.120777,0.440584}%
\pgfsetfillcolor{currentfill}%
\pgfsetfillopacity{0.700000}%
\pgfsetlinewidth{0.000000pt}%
\definecolor{currentstroke}{rgb}{0.000000,0.000000,0.000000}%
\pgfsetstrokecolor{currentstroke}%
\pgfsetdash{}{0pt}%
\pgfpathmoveto{\pgfqpoint{5.764276in}{2.636326in}}%
\pgfpathlineto{\pgfqpoint{5.778101in}{2.634203in}}%
\pgfpathlineto{\pgfqpoint{5.791935in}{2.632104in}}%
\pgfpathlineto{\pgfqpoint{5.805776in}{2.630029in}}%
\pgfpathlineto{\pgfqpoint{5.819625in}{2.627978in}}%
\pgfpathlineto{\pgfqpoint{5.812568in}{2.622053in}}%
\pgfpathlineto{\pgfqpoint{5.805508in}{2.616194in}}%
\pgfpathlineto{\pgfqpoint{5.798444in}{2.610398in}}%
\pgfpathlineto{\pgfqpoint{5.791377in}{2.604658in}}%
\pgfpathlineto{\pgfqpoint{5.777507in}{2.606538in}}%
\pgfpathlineto{\pgfqpoint{5.763645in}{2.608443in}}%
\pgfpathlineto{\pgfqpoint{5.749792in}{2.610372in}}%
\pgfpathlineto{\pgfqpoint{5.735946in}{2.612324in}}%
\pgfpathlineto{\pgfqpoint{5.743034in}{2.618230in}}%
\pgfpathlineto{\pgfqpoint{5.750118in}{2.624195in}}%
\pgfpathlineto{\pgfqpoint{5.757198in}{2.630226in}}%
\pgfpathlineto{\pgfqpoint{5.764276in}{2.636326in}}%
\pgfpathclose%
\pgfusepath{fill}%
\end{pgfscope}%
\begin{pgfscope}%
\pgfpathrectangle{\pgfqpoint{1.150000in}{0.150000in}}{\pgfqpoint{5.700000in}{5.700000in}}%
\pgfusepath{clip}%
\pgfsetbuttcap%
\pgfsetroundjoin%
\definecolor{currentfill}{rgb}{0.283091,0.110553,0.431554}%
\pgfsetfillcolor{currentfill}%
\pgfsetfillopacity{0.700000}%
\pgfsetlinewidth{0.000000pt}%
\definecolor{currentstroke}{rgb}{0.000000,0.000000,0.000000}%
\pgfsetstrokecolor{currentstroke}%
\pgfsetdash{}{0pt}%
\pgfpathmoveto{\pgfqpoint{5.541779in}{2.612610in}}%
\pgfpathlineto{\pgfqpoint{5.555551in}{2.610526in}}%
\pgfpathlineto{\pgfqpoint{5.569331in}{2.608467in}}%
\pgfpathlineto{\pgfqpoint{5.583118in}{2.606431in}}%
\pgfpathlineto{\pgfqpoint{5.596914in}{2.604420in}}%
\pgfpathlineto{\pgfqpoint{5.589767in}{2.598439in}}%
\pgfpathlineto{\pgfqpoint{5.582615in}{2.592495in}}%
\pgfpathlineto{\pgfqpoint{5.575459in}{2.586582in}}%
\pgfpathlineto{\pgfqpoint{5.568298in}{2.580698in}}%
\pgfpathlineto{\pgfqpoint{5.554484in}{2.582565in}}%
\pgfpathlineto{\pgfqpoint{5.540678in}{2.584457in}}%
\pgfpathlineto{\pgfqpoint{5.526880in}{2.586372in}}%
\pgfpathlineto{\pgfqpoint{5.513089in}{2.588312in}}%
\pgfpathlineto{\pgfqpoint{5.520269in}{2.594336in}}%
\pgfpathlineto{\pgfqpoint{5.527443in}{2.600390in}}%
\pgfpathlineto{\pgfqpoint{5.534614in}{2.606480in}}%
\pgfpathlineto{\pgfqpoint{5.541779in}{2.612610in}}%
\pgfpathclose%
\pgfusepath{fill}%
\end{pgfscope}%
\begin{pgfscope}%
\pgfpathrectangle{\pgfqpoint{1.150000in}{0.150000in}}{\pgfqpoint{5.700000in}{5.700000in}}%
\pgfusepath{clip}%
\pgfsetbuttcap%
\pgfsetroundjoin%
\definecolor{currentfill}{rgb}{0.268510,0.009605,0.335427}%
\pgfsetfillcolor{currentfill}%
\pgfsetfillopacity{0.700000}%
\pgfsetlinewidth{0.000000pt}%
\definecolor{currentstroke}{rgb}{0.000000,0.000000,0.000000}%
\pgfsetstrokecolor{currentstroke}%
\pgfsetdash{}{0pt}%
\pgfpathmoveto{\pgfqpoint{3.845031in}{2.435598in}}%
\pgfpathlineto{\pgfqpoint{3.858383in}{2.431799in}}%
\pgfpathlineto{\pgfqpoint{3.871740in}{2.428031in}}%
\pgfpathlineto{\pgfqpoint{3.885103in}{2.424295in}}%
\pgfpathlineto{\pgfqpoint{3.898471in}{2.420590in}}%
\pgfpathlineto{\pgfqpoint{3.890660in}{2.413014in}}%
\pgfpathlineto{\pgfqpoint{3.882843in}{2.405439in}}%
\pgfpathlineto{\pgfqpoint{3.875020in}{2.397866in}}%
\pgfpathlineto{\pgfqpoint{3.867192in}{2.390297in}}%
\pgfpathlineto{\pgfqpoint{3.853810in}{2.394057in}}%
\pgfpathlineto{\pgfqpoint{3.840435in}{2.397848in}}%
\pgfpathlineto{\pgfqpoint{3.827065in}{2.401671in}}%
\pgfpathlineto{\pgfqpoint{3.813701in}{2.405526in}}%
\pgfpathlineto{\pgfqpoint{3.821542in}{2.413035in}}%
\pgfpathlineto{\pgfqpoint{3.829378in}{2.420551in}}%
\pgfpathlineto{\pgfqpoint{3.837207in}{2.428073in}}%
\pgfpathlineto{\pgfqpoint{3.845031in}{2.435598in}}%
\pgfpathclose%
\pgfusepath{fill}%
\end{pgfscope}%
\begin{pgfscope}%
\pgfpathrectangle{\pgfqpoint{1.150000in}{0.150000in}}{\pgfqpoint{5.700000in}{5.700000in}}%
\pgfusepath{clip}%
\pgfsetbuttcap%
\pgfsetroundjoin%
\definecolor{currentfill}{rgb}{0.282656,0.100196,0.422160}%
\pgfsetfillcolor{currentfill}%
\pgfsetfillopacity{0.700000}%
\pgfsetlinewidth{0.000000pt}%
\definecolor{currentstroke}{rgb}{0.000000,0.000000,0.000000}%
\pgfsetstrokecolor{currentstroke}%
\pgfsetdash{}{0pt}%
\pgfpathmoveto{\pgfqpoint{5.319221in}{2.588054in}}%
\pgfpathlineto{\pgfqpoint{5.332938in}{2.585951in}}%
\pgfpathlineto{\pgfqpoint{5.346662in}{2.583873in}}%
\pgfpathlineto{\pgfqpoint{5.360393in}{2.581820in}}%
\pgfpathlineto{\pgfqpoint{5.374132in}{2.579791in}}%
\pgfpathlineto{\pgfqpoint{5.366892in}{2.573613in}}%
\pgfpathlineto{\pgfqpoint{5.359647in}{2.567447in}}%
\pgfpathlineto{\pgfqpoint{5.352397in}{2.561291in}}%
\pgfpathlineto{\pgfqpoint{5.345141in}{2.555140in}}%
\pgfpathlineto{\pgfqpoint{5.331385in}{2.557051in}}%
\pgfpathlineto{\pgfqpoint{5.317636in}{2.558986in}}%
\pgfpathlineto{\pgfqpoint{5.303895in}{2.560947in}}%
\pgfpathlineto{\pgfqpoint{5.290162in}{2.562932in}}%
\pgfpathlineto{\pgfqpoint{5.297434in}{2.569196in}}%
\pgfpathlineto{\pgfqpoint{5.304702in}{2.575469in}}%
\pgfpathlineto{\pgfqpoint{5.311964in}{2.581753in}}%
\pgfpathlineto{\pgfqpoint{5.319221in}{2.588054in}}%
\pgfpathclose%
\pgfusepath{fill}%
\end{pgfscope}%
\begin{pgfscope}%
\pgfpathrectangle{\pgfqpoint{1.150000in}{0.150000in}}{\pgfqpoint{5.700000in}{5.700000in}}%
\pgfusepath{clip}%
\pgfsetbuttcap%
\pgfsetroundjoin%
\definecolor{currentfill}{rgb}{0.269944,0.014625,0.341379}%
\pgfsetfillcolor{currentfill}%
\pgfsetfillopacity{0.700000}%
\pgfsetlinewidth{0.000000pt}%
\definecolor{currentstroke}{rgb}{0.000000,0.000000,0.000000}%
\pgfsetstrokecolor{currentstroke}%
\pgfsetdash{}{0pt}%
\pgfpathmoveto{\pgfqpoint{3.345993in}{2.444730in}}%
\pgfpathlineto{\pgfqpoint{3.359252in}{2.439668in}}%
\pgfpathlineto{\pgfqpoint{3.372516in}{2.434644in}}%
\pgfpathlineto{\pgfqpoint{3.385784in}{2.429657in}}%
\pgfpathlineto{\pgfqpoint{3.399058in}{2.424706in}}%
\pgfpathlineto{\pgfqpoint{3.391052in}{2.418030in}}%
\pgfpathlineto{\pgfqpoint{3.383040in}{2.411403in}}%
\pgfpathlineto{\pgfqpoint{3.375021in}{2.404828in}}%
\pgfpathlineto{\pgfqpoint{3.366995in}{2.398308in}}%
\pgfpathlineto{\pgfqpoint{3.353706in}{2.403366in}}%
\pgfpathlineto{\pgfqpoint{3.340422in}{2.408461in}}%
\pgfpathlineto{\pgfqpoint{3.327143in}{2.413594in}}%
\pgfpathlineto{\pgfqpoint{3.313868in}{2.418764in}}%
\pgfpathlineto{\pgfqpoint{3.321910in}{2.425171in}}%
\pgfpathlineto{\pgfqpoint{3.329945in}{2.431636in}}%
\pgfpathlineto{\pgfqpoint{3.337972in}{2.438157in}}%
\pgfpathlineto{\pgfqpoint{3.345993in}{2.444730in}}%
\pgfpathclose%
\pgfusepath{fill}%
\end{pgfscope}%
\begin{pgfscope}%
\pgfpathrectangle{\pgfqpoint{1.150000in}{0.150000in}}{\pgfqpoint{5.700000in}{5.700000in}}%
\pgfusepath{clip}%
\pgfsetbuttcap%
\pgfsetroundjoin%
\definecolor{currentfill}{rgb}{0.271305,0.019942,0.347269}%
\pgfsetfillcolor{currentfill}%
\pgfsetfillopacity{0.700000}%
\pgfsetlinewidth{0.000000pt}%
\definecolor{currentstroke}{rgb}{0.000000,0.000000,0.000000}%
\pgfsetstrokecolor{currentstroke}%
\pgfsetdash{}{0pt}%
\pgfpathmoveto{\pgfqpoint{3.207834in}{2.461514in}}%
\pgfpathlineto{\pgfqpoint{3.221073in}{2.456032in}}%
\pgfpathlineto{\pgfqpoint{3.234316in}{2.450591in}}%
\pgfpathlineto{\pgfqpoint{3.247563in}{2.445189in}}%
\pgfpathlineto{\pgfqpoint{3.260815in}{2.439826in}}%
\pgfpathlineto{\pgfqpoint{3.252751in}{2.433596in}}%
\pgfpathlineto{\pgfqpoint{3.244678in}{2.427432in}}%
\pgfpathlineto{\pgfqpoint{3.236598in}{2.421338in}}%
\pgfpathlineto{\pgfqpoint{3.228511in}{2.415315in}}%
\pgfpathlineto{\pgfqpoint{3.215243in}{2.420799in}}%
\pgfpathlineto{\pgfqpoint{3.201979in}{2.426322in}}%
\pgfpathlineto{\pgfqpoint{3.188719in}{2.431885in}}%
\pgfpathlineto{\pgfqpoint{3.175463in}{2.437488in}}%
\pgfpathlineto{\pgfqpoint{3.183568in}{2.443384in}}%
\pgfpathlineto{\pgfqpoint{3.191664in}{2.449356in}}%
\pgfpathlineto{\pgfqpoint{3.199753in}{2.455400in}}%
\pgfpathlineto{\pgfqpoint{3.207834in}{2.461514in}}%
\pgfpathclose%
\pgfusepath{fill}%
\end{pgfscope}%
\begin{pgfscope}%
\pgfpathrectangle{\pgfqpoint{1.150000in}{0.150000in}}{\pgfqpoint{5.700000in}{5.700000in}}%
\pgfusepath{clip}%
\pgfsetbuttcap%
\pgfsetroundjoin%
\definecolor{currentfill}{rgb}{0.281446,0.084320,0.407414}%
\pgfsetfillcolor{currentfill}%
\pgfsetfillopacity{0.700000}%
\pgfsetlinewidth{0.000000pt}%
\definecolor{currentstroke}{rgb}{0.000000,0.000000,0.000000}%
\pgfsetstrokecolor{currentstroke}%
\pgfsetdash{}{0pt}%
\pgfpathmoveto{\pgfqpoint{5.096607in}{2.562398in}}%
\pgfpathlineto{\pgfqpoint{5.110266in}{2.560217in}}%
\pgfpathlineto{\pgfqpoint{5.123933in}{2.558062in}}%
\pgfpathlineto{\pgfqpoint{5.137608in}{2.555932in}}%
\pgfpathlineto{\pgfqpoint{5.151290in}{2.553827in}}%
\pgfpathlineto{\pgfqpoint{5.143957in}{2.547360in}}%
\pgfpathlineto{\pgfqpoint{5.136618in}{2.540887in}}%
\pgfpathlineto{\pgfqpoint{5.129273in}{2.534405in}}%
\pgfpathlineto{\pgfqpoint{5.121923in}{2.527912in}}%
\pgfpathlineto{\pgfqpoint{5.108226in}{2.529926in}}%
\pgfpathlineto{\pgfqpoint{5.094536in}{2.531964in}}%
\pgfpathlineto{\pgfqpoint{5.080853in}{2.534029in}}%
\pgfpathlineto{\pgfqpoint{5.067178in}{2.536118in}}%
\pgfpathlineto{\pgfqpoint{5.074544in}{2.542697in}}%
\pgfpathlineto{\pgfqpoint{5.081904in}{2.549268in}}%
\pgfpathlineto{\pgfqpoint{5.089258in}{2.555834in}}%
\pgfpathlineto{\pgfqpoint{5.096607in}{2.562398in}}%
\pgfpathclose%
\pgfusepath{fill}%
\end{pgfscope}%
\begin{pgfscope}%
\pgfpathrectangle{\pgfqpoint{1.150000in}{0.150000in}}{\pgfqpoint{5.700000in}{5.700000in}}%
\pgfusepath{clip}%
\pgfsetbuttcap%
\pgfsetroundjoin%
\definecolor{currentfill}{rgb}{0.279566,0.067836,0.391917}%
\pgfsetfillcolor{currentfill}%
\pgfsetfillopacity{0.700000}%
\pgfsetlinewidth{0.000000pt}%
\definecolor{currentstroke}{rgb}{0.000000,0.000000,0.000000}%
\pgfsetstrokecolor{currentstroke}%
\pgfsetdash{}{0pt}%
\pgfpathmoveto{\pgfqpoint{4.873952in}{2.535783in}}%
\pgfpathlineto{\pgfqpoint{4.887554in}{2.533465in}}%
\pgfpathlineto{\pgfqpoint{4.901164in}{2.531173in}}%
\pgfpathlineto{\pgfqpoint{4.914780in}{2.528907in}}%
\pgfpathlineto{\pgfqpoint{4.928404in}{2.526666in}}%
\pgfpathlineto{\pgfqpoint{4.920980in}{2.519864in}}%
\pgfpathlineto{\pgfqpoint{4.913549in}{2.513044in}}%
\pgfpathlineto{\pgfqpoint{4.906113in}{2.506204in}}%
\pgfpathlineto{\pgfqpoint{4.898671in}{2.499342in}}%
\pgfpathlineto{\pgfqpoint{4.885033in}{2.501518in}}%
\pgfpathlineto{\pgfqpoint{4.871403in}{2.503720in}}%
\pgfpathlineto{\pgfqpoint{4.857779in}{2.505948in}}%
\pgfpathlineto{\pgfqpoint{4.844163in}{2.508202in}}%
\pgfpathlineto{\pgfqpoint{4.851619in}{2.515124in}}%
\pgfpathlineto{\pgfqpoint{4.859069in}{2.522026in}}%
\pgfpathlineto{\pgfqpoint{4.866514in}{2.528912in}}%
\pgfpathlineto{\pgfqpoint{4.873952in}{2.535783in}}%
\pgfpathclose%
\pgfusepath{fill}%
\end{pgfscope}%
\begin{pgfscope}%
\pgfpathrectangle{\pgfqpoint{1.150000in}{0.150000in}}{\pgfqpoint{5.700000in}{5.700000in}}%
\pgfusepath{clip}%
\pgfsetbuttcap%
\pgfsetroundjoin%
\definecolor{currentfill}{rgb}{0.277941,0.056324,0.381191}%
\pgfsetfillcolor{currentfill}%
\pgfsetfillopacity{0.700000}%
\pgfsetlinewidth{0.000000pt}%
\definecolor{currentstroke}{rgb}{0.000000,0.000000,0.000000}%
\pgfsetstrokecolor{currentstroke}%
\pgfsetdash{}{0pt}%
\pgfpathmoveto{\pgfqpoint{4.651275in}{2.508751in}}%
\pgfpathlineto{\pgfqpoint{4.664820in}{2.506234in}}%
\pgfpathlineto{\pgfqpoint{4.678372in}{2.503743in}}%
\pgfpathlineto{\pgfqpoint{4.691931in}{2.501278in}}%
\pgfpathlineto{\pgfqpoint{4.705497in}{2.498840in}}%
\pgfpathlineto{\pgfqpoint{4.697984in}{2.491706in}}%
\pgfpathlineto{\pgfqpoint{4.690465in}{2.484548in}}%
\pgfpathlineto{\pgfqpoint{4.682941in}{2.477364in}}%
\pgfpathlineto{\pgfqpoint{4.675411in}{2.470153in}}%
\pgfpathlineto{\pgfqpoint{4.661831in}{2.472554in}}%
\pgfpathlineto{\pgfqpoint{4.648259in}{2.474980in}}%
\pgfpathlineto{\pgfqpoint{4.634694in}{2.477434in}}%
\pgfpathlineto{\pgfqpoint{4.621136in}{2.479913in}}%
\pgfpathlineto{\pgfqpoint{4.628680in}{2.487157in}}%
\pgfpathlineto{\pgfqpoint{4.636217in}{2.494377in}}%
\pgfpathlineto{\pgfqpoint{4.643749in}{2.501575in}}%
\pgfpathlineto{\pgfqpoint{4.651275in}{2.508751in}}%
\pgfpathclose%
\pgfusepath{fill}%
\end{pgfscope}%
\begin{pgfscope}%
\pgfpathrectangle{\pgfqpoint{1.150000in}{0.150000in}}{\pgfqpoint{5.700000in}{5.700000in}}%
\pgfusepath{clip}%
\pgfsetbuttcap%
\pgfsetroundjoin%
\definecolor{currentfill}{rgb}{0.274952,0.037752,0.364543}%
\pgfsetfillcolor{currentfill}%
\pgfsetfillopacity{0.700000}%
\pgfsetlinewidth{0.000000pt}%
\definecolor{currentstroke}{rgb}{0.000000,0.000000,0.000000}%
\pgfsetstrokecolor{currentstroke}%
\pgfsetdash{}{0pt}%
\pgfpathmoveto{\pgfqpoint{4.428588in}{2.482240in}}%
\pgfpathlineto{\pgfqpoint{4.442077in}{2.479459in}}%
\pgfpathlineto{\pgfqpoint{4.455572in}{2.476705in}}%
\pgfpathlineto{\pgfqpoint{4.469075in}{2.473980in}}%
\pgfpathlineto{\pgfqpoint{4.482584in}{2.471281in}}%
\pgfpathlineto{\pgfqpoint{4.474986in}{2.463865in}}%
\pgfpathlineto{\pgfqpoint{4.467382in}{2.456424in}}%
\pgfpathlineto{\pgfqpoint{4.459772in}{2.448957in}}%
\pgfpathlineto{\pgfqpoint{4.452157in}{2.441464in}}%
\pgfpathlineto{\pgfqpoint{4.438635in}{2.444152in}}%
\pgfpathlineto{\pgfqpoint{4.425120in}{2.446867in}}%
\pgfpathlineto{\pgfqpoint{4.411612in}{2.449609in}}%
\pgfpathlineto{\pgfqpoint{4.398110in}{2.452379in}}%
\pgfpathlineto{\pgfqpoint{4.405738in}{2.459878in}}%
\pgfpathlineto{\pgfqpoint{4.413360in}{2.467354in}}%
\pgfpathlineto{\pgfqpoint{4.420977in}{2.474808in}}%
\pgfpathlineto{\pgfqpoint{4.428588in}{2.482240in}}%
\pgfpathclose%
\pgfusepath{fill}%
\end{pgfscope}%
\begin{pgfscope}%
\pgfpathrectangle{\pgfqpoint{1.150000in}{0.150000in}}{\pgfqpoint{5.700000in}{5.700000in}}%
\pgfusepath{clip}%
\pgfsetbuttcap%
\pgfsetroundjoin%
\definecolor{currentfill}{rgb}{0.281924,0.089666,0.412415}%
\pgfsetfillcolor{currentfill}%
\pgfsetfillopacity{0.700000}%
\pgfsetlinewidth{0.000000pt}%
\definecolor{currentstroke}{rgb}{0.000000,0.000000,0.000000}%
\pgfsetstrokecolor{currentstroke}%
\pgfsetdash{}{0pt}%
\pgfpathmoveto{\pgfqpoint{2.739691in}{2.575986in}}%
\pgfpathlineto{\pgfqpoint{2.752884in}{2.568820in}}%
\pgfpathlineto{\pgfqpoint{2.766081in}{2.561706in}}%
\pgfpathlineto{\pgfqpoint{2.779280in}{2.554643in}}%
\pgfpathlineto{\pgfqpoint{2.792482in}{2.547631in}}%
\pgfpathlineto{\pgfqpoint{2.784185in}{2.543497in}}%
\pgfpathlineto{\pgfqpoint{2.775877in}{2.539492in}}%
\pgfpathlineto{\pgfqpoint{2.767558in}{2.535620in}}%
\pgfpathlineto{\pgfqpoint{2.759229in}{2.531885in}}%
\pgfpathlineto{\pgfqpoint{2.746005in}{2.539059in}}%
\pgfpathlineto{\pgfqpoint{2.732785in}{2.546284in}}%
\pgfpathlineto{\pgfqpoint{2.719567in}{2.553561in}}%
\pgfpathlineto{\pgfqpoint{2.706352in}{2.560890in}}%
\pgfpathlineto{\pgfqpoint{2.714703in}{2.564457in}}%
\pgfpathlineto{\pgfqpoint{2.723043in}{2.568165in}}%
\pgfpathlineto{\pgfqpoint{2.731372in}{2.572009in}}%
\pgfpathlineto{\pgfqpoint{2.739691in}{2.575986in}}%
\pgfpathclose%
\pgfusepath{fill}%
\end{pgfscope}%
\begin{pgfscope}%
\pgfpathrectangle{\pgfqpoint{1.150000in}{0.150000in}}{\pgfqpoint{5.700000in}{5.700000in}}%
\pgfusepath{clip}%
\pgfsetbuttcap%
\pgfsetroundjoin%
\definecolor{currentfill}{rgb}{0.268510,0.009605,0.335427}%
\pgfsetfillcolor{currentfill}%
\pgfsetfillopacity{0.700000}%
\pgfsetlinewidth{0.000000pt}%
\definecolor{currentstroke}{rgb}{0.000000,0.000000,0.000000}%
\pgfsetstrokecolor{currentstroke}%
\pgfsetdash{}{0pt}%
\pgfpathmoveto{\pgfqpoint{3.484096in}{2.432810in}}%
\pgfpathlineto{\pgfqpoint{3.497379in}{2.428136in}}%
\pgfpathlineto{\pgfqpoint{3.510667in}{2.423498in}}%
\pgfpathlineto{\pgfqpoint{3.523961in}{2.418894in}}%
\pgfpathlineto{\pgfqpoint{3.537259in}{2.414326in}}%
\pgfpathlineto{\pgfqpoint{3.529310in}{2.407293in}}%
\pgfpathlineto{\pgfqpoint{3.521353in}{2.400294in}}%
\pgfpathlineto{\pgfqpoint{3.513391in}{2.393332in}}%
\pgfpathlineto{\pgfqpoint{3.505422in}{2.386408in}}%
\pgfpathlineto{\pgfqpoint{3.492109in}{2.391071in}}%
\pgfpathlineto{\pgfqpoint{3.478801in}{2.395769in}}%
\pgfpathlineto{\pgfqpoint{3.465498in}{2.400502in}}%
\pgfpathlineto{\pgfqpoint{3.452200in}{2.405271in}}%
\pgfpathlineto{\pgfqpoint{3.460184in}{2.412095in}}%
\pgfpathlineto{\pgfqpoint{3.468161in}{2.418961in}}%
\pgfpathlineto{\pgfqpoint{3.476132in}{2.425867in}}%
\pgfpathlineto{\pgfqpoint{3.484096in}{2.432810in}}%
\pgfpathclose%
\pgfusepath{fill}%
\end{pgfscope}%
\begin{pgfscope}%
\pgfpathrectangle{\pgfqpoint{1.150000in}{0.150000in}}{\pgfqpoint{5.700000in}{5.700000in}}%
\pgfusepath{clip}%
\pgfsetbuttcap%
\pgfsetroundjoin%
\definecolor{currentfill}{rgb}{0.274952,0.037752,0.364543}%
\pgfsetfillcolor{currentfill}%
\pgfsetfillopacity{0.700000}%
\pgfsetlinewidth{0.000000pt}%
\definecolor{currentstroke}{rgb}{0.000000,0.000000,0.000000}%
\pgfsetstrokecolor{currentstroke}%
\pgfsetdash{}{0pt}%
\pgfpathmoveto{\pgfqpoint{3.069569in}{2.483793in}}%
\pgfpathlineto{\pgfqpoint{3.082792in}{2.477857in}}%
\pgfpathlineto{\pgfqpoint{3.096019in}{2.471965in}}%
\pgfpathlineto{\pgfqpoint{3.109249in}{2.466115in}}%
\pgfpathlineto{\pgfqpoint{3.122484in}{2.460306in}}%
\pgfpathlineto{\pgfqpoint{3.114354in}{2.454618in}}%
\pgfpathlineto{\pgfqpoint{3.106217in}{2.449015in}}%
\pgfpathlineto{\pgfqpoint{3.098071in}{2.443499in}}%
\pgfpathlineto{\pgfqpoint{3.089917in}{2.438074in}}%
\pgfpathlineto{\pgfqpoint{3.076665in}{2.444017in}}%
\pgfpathlineto{\pgfqpoint{3.063416in}{2.450002in}}%
\pgfpathlineto{\pgfqpoint{3.050172in}{2.456030in}}%
\pgfpathlineto{\pgfqpoint{3.036931in}{2.462100in}}%
\pgfpathlineto{\pgfqpoint{3.045104in}{2.467385in}}%
\pgfpathlineto{\pgfqpoint{3.053267in}{2.472764in}}%
\pgfpathlineto{\pgfqpoint{3.061422in}{2.478234in}}%
\pgfpathlineto{\pgfqpoint{3.069569in}{2.483793in}}%
\pgfpathclose%
\pgfusepath{fill}%
\end{pgfscope}%
\begin{pgfscope}%
\pgfpathrectangle{\pgfqpoint{1.150000in}{0.150000in}}{\pgfqpoint{5.700000in}{5.700000in}}%
\pgfusepath{clip}%
\pgfsetbuttcap%
\pgfsetroundjoin%
\definecolor{currentfill}{rgb}{0.271305,0.019942,0.347269}%
\pgfsetfillcolor{currentfill}%
\pgfsetfillopacity{0.700000}%
\pgfsetlinewidth{0.000000pt}%
\definecolor{currentstroke}{rgb}{0.000000,0.000000,0.000000}%
\pgfsetstrokecolor{currentstroke}%
\pgfsetdash{}{0pt}%
\pgfpathmoveto{\pgfqpoint{4.205886in}{2.457589in}}%
\pgfpathlineto{\pgfqpoint{4.219322in}{2.454479in}}%
\pgfpathlineto{\pgfqpoint{4.232763in}{2.451397in}}%
\pgfpathlineto{\pgfqpoint{4.246212in}{2.448344in}}%
\pgfpathlineto{\pgfqpoint{4.259667in}{2.445319in}}%
\pgfpathlineto{\pgfqpoint{4.251985in}{2.437716in}}%
\pgfpathlineto{\pgfqpoint{4.244299in}{2.430094in}}%
\pgfpathlineto{\pgfqpoint{4.236606in}{2.422453in}}%
\pgfpathlineto{\pgfqpoint{4.228908in}{2.414792in}}%
\pgfpathlineto{\pgfqpoint{4.215441in}{2.417832in}}%
\pgfpathlineto{\pgfqpoint{4.201980in}{2.420900in}}%
\pgfpathlineto{\pgfqpoint{4.188526in}{2.423997in}}%
\pgfpathlineto{\pgfqpoint{4.175078in}{2.427123in}}%
\pgfpathlineto{\pgfqpoint{4.182788in}{2.434764in}}%
\pgfpathlineto{\pgfqpoint{4.190493in}{2.442389in}}%
\pgfpathlineto{\pgfqpoint{4.198192in}{2.449997in}}%
\pgfpathlineto{\pgfqpoint{4.205886in}{2.457589in}}%
\pgfpathclose%
\pgfusepath{fill}%
\end{pgfscope}%
\begin{pgfscope}%
\pgfpathrectangle{\pgfqpoint{1.150000in}{0.150000in}}{\pgfqpoint{5.700000in}{5.700000in}}%
\pgfusepath{clip}%
\pgfsetbuttcap%
\pgfsetroundjoin%
\definecolor{currentfill}{rgb}{0.267004,0.004874,0.329415}%
\pgfsetfillcolor{currentfill}%
\pgfsetfillopacity{0.700000}%
\pgfsetlinewidth{0.000000pt}%
\definecolor{currentstroke}{rgb}{0.000000,0.000000,0.000000}%
\pgfsetstrokecolor{currentstroke}%
\pgfsetdash{}{0pt}%
\pgfpathmoveto{\pgfqpoint{3.622186in}{2.425168in}}%
\pgfpathlineto{\pgfqpoint{3.635497in}{2.420853in}}%
\pgfpathlineto{\pgfqpoint{3.648813in}{2.416572in}}%
\pgfpathlineto{\pgfqpoint{3.662135in}{2.412325in}}%
\pgfpathlineto{\pgfqpoint{3.675462in}{2.408110in}}%
\pgfpathlineto{\pgfqpoint{3.667564in}{2.400804in}}%
\pgfpathlineto{\pgfqpoint{3.659661in}{2.393518in}}%
\pgfpathlineto{\pgfqpoint{3.651752in}{2.386255in}}%
\pgfpathlineto{\pgfqpoint{3.643836in}{2.379016in}}%
\pgfpathlineto{\pgfqpoint{3.630495in}{2.383311in}}%
\pgfpathlineto{\pgfqpoint{3.617160in}{2.387640in}}%
\pgfpathlineto{\pgfqpoint{3.603830in}{2.392003in}}%
\pgfpathlineto{\pgfqpoint{3.590505in}{2.396399in}}%
\pgfpathlineto{\pgfqpoint{3.598435in}{2.403552in}}%
\pgfpathlineto{\pgfqpoint{3.606358in}{2.410732in}}%
\pgfpathlineto{\pgfqpoint{3.614275in}{2.417938in}}%
\pgfpathlineto{\pgfqpoint{3.622186in}{2.425168in}}%
\pgfpathclose%
\pgfusepath{fill}%
\end{pgfscope}%
\begin{pgfscope}%
\pgfpathrectangle{\pgfqpoint{1.150000in}{0.150000in}}{\pgfqpoint{5.700000in}{5.700000in}}%
\pgfusepath{clip}%
\pgfsetbuttcap%
\pgfsetroundjoin%
\definecolor{currentfill}{rgb}{0.269944,0.014625,0.341379}%
\pgfsetfillcolor{currentfill}%
\pgfsetfillopacity{0.700000}%
\pgfsetlinewidth{0.000000pt}%
\definecolor{currentstroke}{rgb}{0.000000,0.000000,0.000000}%
\pgfsetstrokecolor{currentstroke}%
\pgfsetdash{}{0pt}%
\pgfpathmoveto{\pgfqpoint{3.983143in}{2.436544in}}%
\pgfpathlineto{\pgfqpoint{3.996529in}{2.433034in}}%
\pgfpathlineto{\pgfqpoint{4.009921in}{2.429555in}}%
\pgfpathlineto{\pgfqpoint{4.023319in}{2.426106in}}%
\pgfpathlineto{\pgfqpoint{4.036723in}{2.422687in}}%
\pgfpathlineto{\pgfqpoint{4.028960in}{2.415040in}}%
\pgfpathlineto{\pgfqpoint{4.021191in}{2.407385in}}%
\pgfpathlineto{\pgfqpoint{4.013416in}{2.399722in}}%
\pgfpathlineto{\pgfqpoint{4.005635in}{2.392053in}}%
\pgfpathlineto{\pgfqpoint{3.992219in}{2.395514in}}%
\pgfpathlineto{\pgfqpoint{3.978808in}{2.399005in}}%
\pgfpathlineto{\pgfqpoint{3.965404in}{2.402526in}}%
\pgfpathlineto{\pgfqpoint{3.952005in}{2.406077in}}%
\pgfpathlineto{\pgfqpoint{3.959798in}{2.413700in}}%
\pgfpathlineto{\pgfqpoint{3.967586in}{2.421319in}}%
\pgfpathlineto{\pgfqpoint{3.975367in}{2.428934in}}%
\pgfpathlineto{\pgfqpoint{3.983143in}{2.436544in}}%
\pgfpathclose%
\pgfusepath{fill}%
\end{pgfscope}%
\begin{pgfscope}%
\pgfpathrectangle{\pgfqpoint{1.150000in}{0.150000in}}{\pgfqpoint{5.700000in}{5.700000in}}%
\pgfusepath{clip}%
\pgfsetbuttcap%
\pgfsetroundjoin%
\definecolor{currentfill}{rgb}{0.277941,0.056324,0.381191}%
\pgfsetfillcolor{currentfill}%
\pgfsetfillopacity{0.700000}%
\pgfsetlinewidth{0.000000pt}%
\definecolor{currentstroke}{rgb}{0.000000,0.000000,0.000000}%
\pgfsetstrokecolor{currentstroke}%
\pgfsetdash{}{0pt}%
\pgfpathmoveto{\pgfqpoint{2.931141in}{2.512252in}}%
\pgfpathlineto{\pgfqpoint{2.944352in}{2.505824in}}%
\pgfpathlineto{\pgfqpoint{2.957567in}{2.499443in}}%
\pgfpathlineto{\pgfqpoint{2.970785in}{2.493107in}}%
\pgfpathlineto{\pgfqpoint{2.984007in}{2.486817in}}%
\pgfpathlineto{\pgfqpoint{2.975807in}{2.481773in}}%
\pgfpathlineto{\pgfqpoint{2.967599in}{2.476833in}}%
\pgfpathlineto{\pgfqpoint{2.959381in}{2.472002in}}%
\pgfpathlineto{\pgfqpoint{2.951154in}{2.467281in}}%
\pgfpathlineto{\pgfqpoint{2.937913in}{2.473720in}}%
\pgfpathlineto{\pgfqpoint{2.924675in}{2.480204in}}%
\pgfpathlineto{\pgfqpoint{2.911441in}{2.486734in}}%
\pgfpathlineto{\pgfqpoint{2.898210in}{2.493310in}}%
\pgfpathlineto{\pgfqpoint{2.906457in}{2.497876in}}%
\pgfpathlineto{\pgfqpoint{2.914694in}{2.502558in}}%
\pgfpathlineto{\pgfqpoint{2.922922in}{2.507351in}}%
\pgfpathlineto{\pgfqpoint{2.931141in}{2.512252in}}%
\pgfpathclose%
\pgfusepath{fill}%
\end{pgfscope}%
\begin{pgfscope}%
\pgfpathrectangle{\pgfqpoint{1.150000in}{0.150000in}}{\pgfqpoint{5.700000in}{5.700000in}}%
\pgfusepath{clip}%
\pgfsetbuttcap%
\pgfsetroundjoin%
\definecolor{currentfill}{rgb}{0.283229,0.120777,0.440584}%
\pgfsetfillcolor{currentfill}%
\pgfsetfillopacity{0.700000}%
\pgfsetlinewidth{0.000000pt}%
\definecolor{currentstroke}{rgb}{0.000000,0.000000,0.000000}%
\pgfsetstrokecolor{currentstroke}%
\pgfsetdash{}{0pt}%
\pgfpathmoveto{\pgfqpoint{5.680641in}{2.620373in}}%
\pgfpathlineto{\pgfqpoint{5.694455in}{2.618325in}}%
\pgfpathlineto{\pgfqpoint{5.708278in}{2.616301in}}%
\pgfpathlineto{\pgfqpoint{5.722108in}{2.614300in}}%
\pgfpathlineto{\pgfqpoint{5.735946in}{2.612324in}}%
\pgfpathlineto{\pgfqpoint{5.728854in}{2.606473in}}%
\pgfpathlineto{\pgfqpoint{5.721759in}{2.600671in}}%
\pgfpathlineto{\pgfqpoint{5.714660in}{2.594914in}}%
\pgfpathlineto{\pgfqpoint{5.707556in}{2.589196in}}%
\pgfpathlineto{\pgfqpoint{5.693698in}{2.591015in}}%
\pgfpathlineto{\pgfqpoint{5.679848in}{2.592858in}}%
\pgfpathlineto{\pgfqpoint{5.666006in}{2.594725in}}%
\pgfpathlineto{\pgfqpoint{5.652172in}{2.596616in}}%
\pgfpathlineto{\pgfqpoint{5.659295in}{2.602486in}}%
\pgfpathlineto{\pgfqpoint{5.666414in}{2.608399in}}%
\pgfpathlineto{\pgfqpoint{5.673530in}{2.614360in}}%
\pgfpathlineto{\pgfqpoint{5.680641in}{2.620373in}}%
\pgfpathclose%
\pgfusepath{fill}%
\end{pgfscope}%
\begin{pgfscope}%
\pgfpathrectangle{\pgfqpoint{1.150000in}{0.150000in}}{\pgfqpoint{5.700000in}{5.700000in}}%
\pgfusepath{clip}%
\pgfsetbuttcap%
\pgfsetroundjoin%
\definecolor{currentfill}{rgb}{0.283091,0.110553,0.431554}%
\pgfsetfillcolor{currentfill}%
\pgfsetfillopacity{0.700000}%
\pgfsetlinewidth{0.000000pt}%
\definecolor{currentstroke}{rgb}{0.000000,0.000000,0.000000}%
\pgfsetstrokecolor{currentstroke}%
\pgfsetdash{}{0pt}%
\pgfpathmoveto{\pgfqpoint{5.458004in}{2.596315in}}%
\pgfpathlineto{\pgfqpoint{5.471764in}{2.594278in}}%
\pgfpathlineto{\pgfqpoint{5.485532in}{2.592265in}}%
\pgfpathlineto{\pgfqpoint{5.499307in}{2.590276in}}%
\pgfpathlineto{\pgfqpoint{5.513089in}{2.588312in}}%
\pgfpathlineto{\pgfqpoint{5.505905in}{2.582315in}}%
\pgfpathlineto{\pgfqpoint{5.498716in}{2.576341in}}%
\pgfpathlineto{\pgfqpoint{5.491522in}{2.570384in}}%
\pgfpathlineto{\pgfqpoint{5.484323in}{2.564442in}}%
\pgfpathlineto{\pgfqpoint{5.470522in}{2.566275in}}%
\pgfpathlineto{\pgfqpoint{5.456729in}{2.568132in}}%
\pgfpathlineto{\pgfqpoint{5.442943in}{2.570014in}}%
\pgfpathlineto{\pgfqpoint{5.429166in}{2.571921in}}%
\pgfpathlineto{\pgfqpoint{5.436383in}{2.577989in}}%
\pgfpathlineto{\pgfqpoint{5.443595in}{2.584075in}}%
\pgfpathlineto{\pgfqpoint{5.450802in}{2.590182in}}%
\pgfpathlineto{\pgfqpoint{5.458004in}{2.596315in}}%
\pgfpathclose%
\pgfusepath{fill}%
\end{pgfscope}%
\begin{pgfscope}%
\pgfpathrectangle{\pgfqpoint{1.150000in}{0.150000in}}{\pgfqpoint{5.700000in}{5.700000in}}%
\pgfusepath{clip}%
\pgfsetbuttcap%
\pgfsetroundjoin%
\definecolor{currentfill}{rgb}{0.282327,0.094955,0.417331}%
\pgfsetfillcolor{currentfill}%
\pgfsetfillopacity{0.700000}%
\pgfsetlinewidth{0.000000pt}%
\definecolor{currentstroke}{rgb}{0.000000,0.000000,0.000000}%
\pgfsetstrokecolor{currentstroke}%
\pgfsetdash{}{0pt}%
\pgfpathmoveto{\pgfqpoint{5.235304in}{2.571120in}}%
\pgfpathlineto{\pgfqpoint{5.249007in}{2.569036in}}%
\pgfpathlineto{\pgfqpoint{5.262718in}{2.566977in}}%
\pgfpathlineto{\pgfqpoint{5.276436in}{2.564942in}}%
\pgfpathlineto{\pgfqpoint{5.290162in}{2.562932in}}%
\pgfpathlineto{\pgfqpoint{5.282883in}{2.556673in}}%
\pgfpathlineto{\pgfqpoint{5.275600in}{2.550414in}}%
\pgfpathlineto{\pgfqpoint{5.268310in}{2.544154in}}%
\pgfpathlineto{\pgfqpoint{5.261015in}{2.537888in}}%
\pgfpathlineto{\pgfqpoint{5.247273in}{2.539793in}}%
\pgfpathlineto{\pgfqpoint{5.233538in}{2.541723in}}%
\pgfpathlineto{\pgfqpoint{5.219811in}{2.543678in}}%
\pgfpathlineto{\pgfqpoint{5.206092in}{2.545658in}}%
\pgfpathlineto{\pgfqpoint{5.213403in}{2.552024in}}%
\pgfpathlineto{\pgfqpoint{5.220709in}{2.558387in}}%
\pgfpathlineto{\pgfqpoint{5.228009in}{2.564752in}}%
\pgfpathlineto{\pgfqpoint{5.235304in}{2.571120in}}%
\pgfpathclose%
\pgfusepath{fill}%
\end{pgfscope}%
\begin{pgfscope}%
\pgfpathrectangle{\pgfqpoint{1.150000in}{0.150000in}}{\pgfqpoint{5.700000in}{5.700000in}}%
\pgfusepath{clip}%
\pgfsetbuttcap%
\pgfsetroundjoin%
\definecolor{currentfill}{rgb}{0.267004,0.004874,0.329415}%
\pgfsetfillcolor{currentfill}%
\pgfsetfillopacity{0.700000}%
\pgfsetlinewidth{0.000000pt}%
\definecolor{currentstroke}{rgb}{0.000000,0.000000,0.000000}%
\pgfsetstrokecolor{currentstroke}%
\pgfsetdash{}{0pt}%
\pgfpathmoveto{\pgfqpoint{3.760301in}{2.421262in}}%
\pgfpathlineto{\pgfqpoint{3.773642in}{2.417280in}}%
\pgfpathlineto{\pgfqpoint{3.786989in}{2.413330in}}%
\pgfpathlineto{\pgfqpoint{3.800342in}{2.409412in}}%
\pgfpathlineto{\pgfqpoint{3.813701in}{2.405526in}}%
\pgfpathlineto{\pgfqpoint{3.805853in}{2.398023in}}%
\pgfpathlineto{\pgfqpoint{3.798000in}{2.390530in}}%
\pgfpathlineto{\pgfqpoint{3.790141in}{2.383046in}}%
\pgfpathlineto{\pgfqpoint{3.782277in}{2.375574in}}%
\pgfpathlineto{\pgfqpoint{3.768905in}{2.379528in}}%
\pgfpathlineto{\pgfqpoint{3.755539in}{2.383514in}}%
\pgfpathlineto{\pgfqpoint{3.742179in}{2.387532in}}%
\pgfpathlineto{\pgfqpoint{3.728825in}{2.391583in}}%
\pgfpathlineto{\pgfqpoint{3.736703in}{2.398982in}}%
\pgfpathlineto{\pgfqpoint{3.744575in}{2.406396in}}%
\pgfpathlineto{\pgfqpoint{3.752441in}{2.413823in}}%
\pgfpathlineto{\pgfqpoint{3.760301in}{2.421262in}}%
\pgfpathclose%
\pgfusepath{fill}%
\end{pgfscope}%
\begin{pgfscope}%
\pgfpathrectangle{\pgfqpoint{1.150000in}{0.150000in}}{\pgfqpoint{5.700000in}{5.700000in}}%
\pgfusepath{clip}%
\pgfsetbuttcap%
\pgfsetroundjoin%
\definecolor{currentfill}{rgb}{0.280894,0.078907,0.402329}%
\pgfsetfillcolor{currentfill}%
\pgfsetfillopacity{0.700000}%
\pgfsetlinewidth{0.000000pt}%
\definecolor{currentstroke}{rgb}{0.000000,0.000000,0.000000}%
\pgfsetstrokecolor{currentstroke}%
\pgfsetdash{}{0pt}%
\pgfpathmoveto{\pgfqpoint{5.012552in}{2.544728in}}%
\pgfpathlineto{\pgfqpoint{5.026198in}{2.542538in}}%
\pgfpathlineto{\pgfqpoint{5.039851in}{2.540372in}}%
\pgfpathlineto{\pgfqpoint{5.053511in}{2.538232in}}%
\pgfpathlineto{\pgfqpoint{5.067178in}{2.536118in}}%
\pgfpathlineto{\pgfqpoint{5.059807in}{2.529527in}}%
\pgfpathlineto{\pgfqpoint{5.052430in}{2.522922in}}%
\pgfpathlineto{\pgfqpoint{5.045047in}{2.516301in}}%
\pgfpathlineto{\pgfqpoint{5.037658in}{2.509660in}}%
\pgfpathlineto{\pgfqpoint{5.023975in}{2.511697in}}%
\pgfpathlineto{\pgfqpoint{5.010300in}{2.513759in}}%
\pgfpathlineto{\pgfqpoint{4.996632in}{2.515846in}}%
\pgfpathlineto{\pgfqpoint{4.982972in}{2.517959in}}%
\pgfpathlineto{\pgfqpoint{4.990376in}{2.524673in}}%
\pgfpathlineto{\pgfqpoint{4.997774in}{2.531371in}}%
\pgfpathlineto{\pgfqpoint{5.005166in}{2.538055in}}%
\pgfpathlineto{\pgfqpoint{5.012552in}{2.544728in}}%
\pgfpathclose%
\pgfusepath{fill}%
\end{pgfscope}%
\begin{pgfscope}%
\pgfpathrectangle{\pgfqpoint{1.150000in}{0.150000in}}{\pgfqpoint{5.700000in}{5.700000in}}%
\pgfusepath{clip}%
\pgfsetbuttcap%
\pgfsetroundjoin%
\definecolor{currentfill}{rgb}{0.279566,0.067836,0.391917}%
\pgfsetfillcolor{currentfill}%
\pgfsetfillopacity{0.700000}%
\pgfsetlinewidth{0.000000pt}%
\definecolor{currentstroke}{rgb}{0.000000,0.000000,0.000000}%
\pgfsetstrokecolor{currentstroke}%
\pgfsetdash{}{0pt}%
\pgfpathmoveto{\pgfqpoint{4.789770in}{2.517476in}}%
\pgfpathlineto{\pgfqpoint{4.803357in}{2.515119in}}%
\pgfpathlineto{\pgfqpoint{4.816952in}{2.512787in}}%
\pgfpathlineto{\pgfqpoint{4.830554in}{2.510482in}}%
\pgfpathlineto{\pgfqpoint{4.844163in}{2.508202in}}%
\pgfpathlineto{\pgfqpoint{4.836701in}{2.501259in}}%
\pgfpathlineto{\pgfqpoint{4.829233in}{2.494292in}}%
\pgfpathlineto{\pgfqpoint{4.821759in}{2.487301in}}%
\pgfpathlineto{\pgfqpoint{4.814279in}{2.480282in}}%
\pgfpathlineto{\pgfqpoint{4.800657in}{2.482510in}}%
\pgfpathlineto{\pgfqpoint{4.787041in}{2.484764in}}%
\pgfpathlineto{\pgfqpoint{4.773432in}{2.487045in}}%
\pgfpathlineto{\pgfqpoint{4.759831in}{2.489351in}}%
\pgfpathlineto{\pgfqpoint{4.767324in}{2.496417in}}%
\pgfpathlineto{\pgfqpoint{4.774812in}{2.503458in}}%
\pgfpathlineto{\pgfqpoint{4.782294in}{2.510477in}}%
\pgfpathlineto{\pgfqpoint{4.789770in}{2.517476in}}%
\pgfpathclose%
\pgfusepath{fill}%
\end{pgfscope}%
\begin{pgfscope}%
\pgfpathrectangle{\pgfqpoint{1.150000in}{0.150000in}}{\pgfqpoint{5.700000in}{5.700000in}}%
\pgfusepath{clip}%
\pgfsetbuttcap%
\pgfsetroundjoin%
\definecolor{currentfill}{rgb}{0.277018,0.050344,0.375715}%
\pgfsetfillcolor{currentfill}%
\pgfsetfillopacity{0.700000}%
\pgfsetlinewidth{0.000000pt}%
\definecolor{currentstroke}{rgb}{0.000000,0.000000,0.000000}%
\pgfsetstrokecolor{currentstroke}%
\pgfsetdash{}{0pt}%
\pgfpathmoveto{\pgfqpoint{4.566973in}{2.490101in}}%
\pgfpathlineto{\pgfqpoint{4.580503in}{2.487514in}}%
\pgfpathlineto{\pgfqpoint{4.594041in}{2.484953in}}%
\pgfpathlineto{\pgfqpoint{4.607585in}{2.482420in}}%
\pgfpathlineto{\pgfqpoint{4.621136in}{2.479913in}}%
\pgfpathlineto{\pgfqpoint{4.613587in}{2.472644in}}%
\pgfpathlineto{\pgfqpoint{4.606032in}{2.465348in}}%
\pgfpathlineto{\pgfqpoint{4.598471in}{2.458024in}}%
\pgfpathlineto{\pgfqpoint{4.590904in}{2.450670in}}%
\pgfpathlineto{\pgfqpoint{4.577340in}{2.453152in}}%
\pgfpathlineto{\pgfqpoint{4.563782in}{2.455661in}}%
\pgfpathlineto{\pgfqpoint{4.550232in}{2.458197in}}%
\pgfpathlineto{\pgfqpoint{4.536689in}{2.460759in}}%
\pgfpathlineto{\pgfqpoint{4.544269in}{2.468132in}}%
\pgfpathlineto{\pgfqpoint{4.551842in}{2.475480in}}%
\pgfpathlineto{\pgfqpoint{4.559411in}{2.482802in}}%
\pgfpathlineto{\pgfqpoint{4.566973in}{2.490101in}}%
\pgfpathclose%
\pgfusepath{fill}%
\end{pgfscope}%
\begin{pgfscope}%
\pgfpathrectangle{\pgfqpoint{1.150000in}{0.150000in}}{\pgfqpoint{5.700000in}{5.700000in}}%
\pgfusepath{clip}%
\pgfsetbuttcap%
\pgfsetroundjoin%
\definecolor{currentfill}{rgb}{0.273809,0.031497,0.358853}%
\pgfsetfillcolor{currentfill}%
\pgfsetfillopacity{0.700000}%
\pgfsetlinewidth{0.000000pt}%
\definecolor{currentstroke}{rgb}{0.000000,0.000000,0.000000}%
\pgfsetstrokecolor{currentstroke}%
\pgfsetdash{}{0pt}%
\pgfpathmoveto{\pgfqpoint{4.344170in}{2.463736in}}%
\pgfpathlineto{\pgfqpoint{4.357645in}{2.460855in}}%
\pgfpathlineto{\pgfqpoint{4.371127in}{2.458002in}}%
\pgfpathlineto{\pgfqpoint{4.384615in}{2.455176in}}%
\pgfpathlineto{\pgfqpoint{4.398110in}{2.452379in}}%
\pgfpathlineto{\pgfqpoint{4.390476in}{2.444856in}}%
\pgfpathlineto{\pgfqpoint{4.382837in}{2.437309in}}%
\pgfpathlineto{\pgfqpoint{4.375192in}{2.429736in}}%
\pgfpathlineto{\pgfqpoint{4.367541in}{2.422139in}}%
\pgfpathlineto{\pgfqpoint{4.354033in}{2.424938in}}%
\pgfpathlineto{\pgfqpoint{4.340533in}{2.427766in}}%
\pgfpathlineto{\pgfqpoint{4.327039in}{2.430621in}}%
\pgfpathlineto{\pgfqpoint{4.313551in}{2.433504in}}%
\pgfpathlineto{\pgfqpoint{4.321214in}{2.441095in}}%
\pgfpathlineto{\pgfqpoint{4.328872in}{2.448663in}}%
\pgfpathlineto{\pgfqpoint{4.336524in}{2.456210in}}%
\pgfpathlineto{\pgfqpoint{4.344170in}{2.463736in}}%
\pgfpathclose%
\pgfusepath{fill}%
\end{pgfscope}%
\begin{pgfscope}%
\pgfpathrectangle{\pgfqpoint{1.150000in}{0.150000in}}{\pgfqpoint{5.700000in}{5.700000in}}%
\pgfusepath{clip}%
\pgfsetbuttcap%
\pgfsetroundjoin%
\definecolor{currentfill}{rgb}{0.271305,0.019942,0.347269}%
\pgfsetfillcolor{currentfill}%
\pgfsetfillopacity{0.700000}%
\pgfsetlinewidth{0.000000pt}%
\definecolor{currentstroke}{rgb}{0.000000,0.000000,0.000000}%
\pgfsetstrokecolor{currentstroke}%
\pgfsetdash{}{0pt}%
\pgfpathmoveto{\pgfqpoint{4.121349in}{2.439918in}}%
\pgfpathlineto{\pgfqpoint{4.134772in}{2.436675in}}%
\pgfpathlineto{\pgfqpoint{4.148201in}{2.433462in}}%
\pgfpathlineto{\pgfqpoint{4.161636in}{2.430278in}}%
\pgfpathlineto{\pgfqpoint{4.175078in}{2.427123in}}%
\pgfpathlineto{\pgfqpoint{4.167361in}{2.419466in}}%
\pgfpathlineto{\pgfqpoint{4.159640in}{2.411793in}}%
\pgfpathlineto{\pgfqpoint{4.151912in}{2.404104in}}%
\pgfpathlineto{\pgfqpoint{4.144179in}{2.396399in}}%
\pgfpathlineto{\pgfqpoint{4.130725in}{2.399582in}}%
\pgfpathlineto{\pgfqpoint{4.117278in}{2.402795in}}%
\pgfpathlineto{\pgfqpoint{4.103836in}{2.406037in}}%
\pgfpathlineto{\pgfqpoint{4.090401in}{2.409307in}}%
\pgfpathlineto{\pgfqpoint{4.098147in}{2.416979in}}%
\pgfpathlineto{\pgfqpoint{4.105886in}{2.424638in}}%
\pgfpathlineto{\pgfqpoint{4.113621in}{2.432284in}}%
\pgfpathlineto{\pgfqpoint{4.121349in}{2.439918in}}%
\pgfpathclose%
\pgfusepath{fill}%
\end{pgfscope}%
\begin{pgfscope}%
\pgfpathrectangle{\pgfqpoint{1.150000in}{0.150000in}}{\pgfqpoint{5.700000in}{5.700000in}}%
\pgfusepath{clip}%
\pgfsetbuttcap%
\pgfsetroundjoin%
\definecolor{currentfill}{rgb}{0.281446,0.084320,0.407414}%
\pgfsetfillcolor{currentfill}%
\pgfsetfillopacity{0.700000}%
\pgfsetlinewidth{0.000000pt}%
\definecolor{currentstroke}{rgb}{0.000000,0.000000,0.000000}%
\pgfsetstrokecolor{currentstroke}%
\pgfsetdash{}{0pt}%
\pgfpathmoveto{\pgfqpoint{2.792482in}{2.547631in}}%
\pgfpathlineto{\pgfqpoint{2.805687in}{2.540670in}}%
\pgfpathlineto{\pgfqpoint{2.818895in}{2.533758in}}%
\pgfpathlineto{\pgfqpoint{2.832106in}{2.526896in}}%
\pgfpathlineto{\pgfqpoint{2.845321in}{2.520083in}}%
\pgfpathlineto{\pgfqpoint{2.837044in}{2.515792in}}%
\pgfpathlineto{\pgfqpoint{2.828757in}{2.511626in}}%
\pgfpathlineto{\pgfqpoint{2.820460in}{2.507590in}}%
\pgfpathlineto{\pgfqpoint{2.812152in}{2.503688in}}%
\pgfpathlineto{\pgfqpoint{2.798917in}{2.510663in}}%
\pgfpathlineto{\pgfqpoint{2.785684in}{2.517687in}}%
\pgfpathlineto{\pgfqpoint{2.772455in}{2.524761in}}%
\pgfpathlineto{\pgfqpoint{2.759229in}{2.531885in}}%
\pgfpathlineto{\pgfqpoint{2.767558in}{2.535620in}}%
\pgfpathlineto{\pgfqpoint{2.775877in}{2.539492in}}%
\pgfpathlineto{\pgfqpoint{2.784185in}{2.543497in}}%
\pgfpathlineto{\pgfqpoint{2.792482in}{2.547631in}}%
\pgfpathclose%
\pgfusepath{fill}%
\end{pgfscope}%
\begin{pgfscope}%
\pgfpathrectangle{\pgfqpoint{1.150000in}{0.150000in}}{\pgfqpoint{5.700000in}{5.700000in}}%
\pgfusepath{clip}%
\pgfsetbuttcap%
\pgfsetroundjoin%
\definecolor{currentfill}{rgb}{0.271305,0.019942,0.347269}%
\pgfsetfillcolor{currentfill}%
\pgfsetfillopacity{0.700000}%
\pgfsetlinewidth{0.000000pt}%
\definecolor{currentstroke}{rgb}{0.000000,0.000000,0.000000}%
\pgfsetstrokecolor{currentstroke}%
\pgfsetdash{}{0pt}%
\pgfpathmoveto{\pgfqpoint{3.260815in}{2.439826in}}%
\pgfpathlineto{\pgfqpoint{3.274072in}{2.434503in}}%
\pgfpathlineto{\pgfqpoint{3.287333in}{2.429218in}}%
\pgfpathlineto{\pgfqpoint{3.300598in}{2.423972in}}%
\pgfpathlineto{\pgfqpoint{3.313868in}{2.418764in}}%
\pgfpathlineto{\pgfqpoint{3.305819in}{2.412417in}}%
\pgfpathlineto{\pgfqpoint{3.297763in}{2.406134in}}%
\pgfpathlineto{\pgfqpoint{3.289700in}{2.399917in}}%
\pgfpathlineto{\pgfqpoint{3.281629in}{2.393768in}}%
\pgfpathlineto{\pgfqpoint{3.268343in}{2.399097in}}%
\pgfpathlineto{\pgfqpoint{3.255061in}{2.404464in}}%
\pgfpathlineto{\pgfqpoint{3.241784in}{2.409870in}}%
\pgfpathlineto{\pgfqpoint{3.228511in}{2.415315in}}%
\pgfpathlineto{\pgfqpoint{3.236598in}{2.421338in}}%
\pgfpathlineto{\pgfqpoint{3.244678in}{2.427432in}}%
\pgfpathlineto{\pgfqpoint{3.252751in}{2.433596in}}%
\pgfpathlineto{\pgfqpoint{3.260815in}{2.439826in}}%
\pgfpathclose%
\pgfusepath{fill}%
\end{pgfscope}%
\begin{pgfscope}%
\pgfpathrectangle{\pgfqpoint{1.150000in}{0.150000in}}{\pgfqpoint{5.700000in}{5.700000in}}%
\pgfusepath{clip}%
\pgfsetbuttcap%
\pgfsetroundjoin%
\definecolor{currentfill}{rgb}{0.268510,0.009605,0.335427}%
\pgfsetfillcolor{currentfill}%
\pgfsetfillopacity{0.700000}%
\pgfsetlinewidth{0.000000pt}%
\definecolor{currentstroke}{rgb}{0.000000,0.000000,0.000000}%
\pgfsetstrokecolor{currentstroke}%
\pgfsetdash{}{0pt}%
\pgfpathmoveto{\pgfqpoint{3.399058in}{2.424706in}}%
\pgfpathlineto{\pgfqpoint{3.412336in}{2.419793in}}%
\pgfpathlineto{\pgfqpoint{3.425619in}{2.414916in}}%
\pgfpathlineto{\pgfqpoint{3.438907in}{2.410076in}}%
\pgfpathlineto{\pgfqpoint{3.452200in}{2.405271in}}%
\pgfpathlineto{\pgfqpoint{3.444210in}{2.398491in}}%
\pgfpathlineto{\pgfqpoint{3.436212in}{2.391758in}}%
\pgfpathlineto{\pgfqpoint{3.428209in}{2.385074in}}%
\pgfpathlineto{\pgfqpoint{3.420198in}{2.378441in}}%
\pgfpathlineto{\pgfqpoint{3.406890in}{2.383353in}}%
\pgfpathlineto{\pgfqpoint{3.393587in}{2.388302in}}%
\pgfpathlineto{\pgfqpoint{3.380288in}{2.393287in}}%
\pgfpathlineto{\pgfqpoint{3.366995in}{2.398308in}}%
\pgfpathlineto{\pgfqpoint{3.375021in}{2.404828in}}%
\pgfpathlineto{\pgfqpoint{3.383040in}{2.411403in}}%
\pgfpathlineto{\pgfqpoint{3.391052in}{2.418030in}}%
\pgfpathlineto{\pgfqpoint{3.399058in}{2.424706in}}%
\pgfpathclose%
\pgfusepath{fill}%
\end{pgfscope}%
\begin{pgfscope}%
\pgfpathrectangle{\pgfqpoint{1.150000in}{0.150000in}}{\pgfqpoint{5.700000in}{5.700000in}}%
\pgfusepath{clip}%
\pgfsetbuttcap%
\pgfsetroundjoin%
\definecolor{currentfill}{rgb}{0.268510,0.009605,0.335427}%
\pgfsetfillcolor{currentfill}%
\pgfsetfillopacity{0.700000}%
\pgfsetlinewidth{0.000000pt}%
\definecolor{currentstroke}{rgb}{0.000000,0.000000,0.000000}%
\pgfsetstrokecolor{currentstroke}%
\pgfsetdash{}{0pt}%
\pgfpathmoveto{\pgfqpoint{3.898471in}{2.420590in}}%
\pgfpathlineto{\pgfqpoint{3.911846in}{2.416915in}}%
\pgfpathlineto{\pgfqpoint{3.925227in}{2.413272in}}%
\pgfpathlineto{\pgfqpoint{3.938613in}{2.409659in}}%
\pgfpathlineto{\pgfqpoint{3.952005in}{2.406077in}}%
\pgfpathlineto{\pgfqpoint{3.944207in}{2.398452in}}%
\pgfpathlineto{\pgfqpoint{3.936402in}{2.390823in}}%
\pgfpathlineto{\pgfqpoint{3.928592in}{2.383194in}}%
\pgfpathlineto{\pgfqpoint{3.920777in}{2.375564in}}%
\pgfpathlineto{\pgfqpoint{3.907372in}{2.379201in}}%
\pgfpathlineto{\pgfqpoint{3.893972in}{2.382869in}}%
\pgfpathlineto{\pgfqpoint{3.880579in}{2.386567in}}%
\pgfpathlineto{\pgfqpoint{3.867192in}{2.390297in}}%
\pgfpathlineto{\pgfqpoint{3.875020in}{2.397866in}}%
\pgfpathlineto{\pgfqpoint{3.882843in}{2.405439in}}%
\pgfpathlineto{\pgfqpoint{3.890660in}{2.413014in}}%
\pgfpathlineto{\pgfqpoint{3.898471in}{2.420590in}}%
\pgfpathclose%
\pgfusepath{fill}%
\end{pgfscope}%
\begin{pgfscope}%
\pgfpathrectangle{\pgfqpoint{1.150000in}{0.150000in}}{\pgfqpoint{5.700000in}{5.700000in}}%
\pgfusepath{clip}%
\pgfsetbuttcap%
\pgfsetroundjoin%
\definecolor{currentfill}{rgb}{0.283187,0.125848,0.444960}%
\pgfsetfillcolor{currentfill}%
\pgfsetfillopacity{0.700000}%
\pgfsetlinewidth{0.000000pt}%
\definecolor{currentstroke}{rgb}{0.000000,0.000000,0.000000}%
\pgfsetstrokecolor{currentstroke}%
\pgfsetdash{}{0pt}%
\pgfpathmoveto{\pgfqpoint{5.819625in}{2.627978in}}%
\pgfpathlineto{\pgfqpoint{5.833482in}{2.625950in}}%
\pgfpathlineto{\pgfqpoint{5.847346in}{2.623947in}}%
\pgfpathlineto{\pgfqpoint{5.861219in}{2.621967in}}%
\pgfpathlineto{\pgfqpoint{5.875099in}{2.620010in}}%
\pgfpathlineto{\pgfqpoint{5.868064in}{2.614260in}}%
\pgfpathlineto{\pgfqpoint{5.861024in}{2.608574in}}%
\pgfpathlineto{\pgfqpoint{5.853981in}{2.602947in}}%
\pgfpathlineto{\pgfqpoint{5.846935in}{2.597373in}}%
\pgfpathlineto{\pgfqpoint{5.833033in}{2.599158in}}%
\pgfpathlineto{\pgfqpoint{5.819140in}{2.600967in}}%
\pgfpathlineto{\pgfqpoint{5.805254in}{2.602801in}}%
\pgfpathlineto{\pgfqpoint{5.791377in}{2.604658in}}%
\pgfpathlineto{\pgfqpoint{5.798444in}{2.610398in}}%
\pgfpathlineto{\pgfqpoint{5.805508in}{2.616194in}}%
\pgfpathlineto{\pgfqpoint{5.812568in}{2.622053in}}%
\pgfpathlineto{\pgfqpoint{5.819625in}{2.627978in}}%
\pgfpathclose%
\pgfusepath{fill}%
\end{pgfscope}%
\begin{pgfscope}%
\pgfpathrectangle{\pgfqpoint{1.150000in}{0.150000in}}{\pgfqpoint{5.700000in}{5.700000in}}%
\pgfusepath{clip}%
\pgfsetbuttcap%
\pgfsetroundjoin%
\definecolor{currentfill}{rgb}{0.273809,0.031497,0.358853}%
\pgfsetfillcolor{currentfill}%
\pgfsetfillopacity{0.700000}%
\pgfsetlinewidth{0.000000pt}%
\definecolor{currentstroke}{rgb}{0.000000,0.000000,0.000000}%
\pgfsetstrokecolor{currentstroke}%
\pgfsetdash{}{0pt}%
\pgfpathmoveto{\pgfqpoint{3.122484in}{2.460306in}}%
\pgfpathlineto{\pgfqpoint{3.135722in}{2.454540in}}%
\pgfpathlineto{\pgfqpoint{3.148965in}{2.448815in}}%
\pgfpathlineto{\pgfqpoint{3.162212in}{2.443131in}}%
\pgfpathlineto{\pgfqpoint{3.175463in}{2.437488in}}%
\pgfpathlineto{\pgfqpoint{3.167351in}{2.431670in}}%
\pgfpathlineto{\pgfqpoint{3.159231in}{2.425933in}}%
\pgfpathlineto{\pgfqpoint{3.151103in}{2.420281in}}%
\pgfpathlineto{\pgfqpoint{3.142967in}{2.414717in}}%
\pgfpathlineto{\pgfqpoint{3.129698in}{2.420495in}}%
\pgfpathlineto{\pgfqpoint{3.116434in}{2.426314in}}%
\pgfpathlineto{\pgfqpoint{3.103173in}{2.432173in}}%
\pgfpathlineto{\pgfqpoint{3.089917in}{2.438074in}}%
\pgfpathlineto{\pgfqpoint{3.098071in}{2.443499in}}%
\pgfpathlineto{\pgfqpoint{3.106217in}{2.449015in}}%
\pgfpathlineto{\pgfqpoint{3.114354in}{2.454618in}}%
\pgfpathlineto{\pgfqpoint{3.122484in}{2.460306in}}%
\pgfpathclose%
\pgfusepath{fill}%
\end{pgfscope}%
\begin{pgfscope}%
\pgfpathrectangle{\pgfqpoint{1.150000in}{0.150000in}}{\pgfqpoint{5.700000in}{5.700000in}}%
\pgfusepath{clip}%
\pgfsetbuttcap%
\pgfsetroundjoin%
\definecolor{currentfill}{rgb}{0.267004,0.004874,0.329415}%
\pgfsetfillcolor{currentfill}%
\pgfsetfillopacity{0.700000}%
\pgfsetlinewidth{0.000000pt}%
\definecolor{currentstroke}{rgb}{0.000000,0.000000,0.000000}%
\pgfsetstrokecolor{currentstroke}%
\pgfsetdash{}{0pt}%
\pgfpathmoveto{\pgfqpoint{3.537259in}{2.414326in}}%
\pgfpathlineto{\pgfqpoint{3.550563in}{2.409793in}}%
\pgfpathlineto{\pgfqpoint{3.563872in}{2.405294in}}%
\pgfpathlineto{\pgfqpoint{3.577186in}{2.400830in}}%
\pgfpathlineto{\pgfqpoint{3.590505in}{2.396399in}}%
\pgfpathlineto{\pgfqpoint{3.582570in}{2.389276in}}%
\pgfpathlineto{\pgfqpoint{3.574628in}{2.382184in}}%
\pgfpathlineto{\pgfqpoint{3.566679in}{2.375126in}}%
\pgfpathlineto{\pgfqpoint{3.558725in}{2.368102in}}%
\pgfpathlineto{\pgfqpoint{3.545391in}{2.372627in}}%
\pgfpathlineto{\pgfqpoint{3.532063in}{2.377186in}}%
\pgfpathlineto{\pgfqpoint{3.518740in}{2.381780in}}%
\pgfpathlineto{\pgfqpoint{3.505422in}{2.386408in}}%
\pgfpathlineto{\pgfqpoint{3.513391in}{2.393332in}}%
\pgfpathlineto{\pgfqpoint{3.521353in}{2.400294in}}%
\pgfpathlineto{\pgfqpoint{3.529310in}{2.407293in}}%
\pgfpathlineto{\pgfqpoint{3.537259in}{2.414326in}}%
\pgfpathclose%
\pgfusepath{fill}%
\end{pgfscope}%
\begin{pgfscope}%
\pgfpathrectangle{\pgfqpoint{1.150000in}{0.150000in}}{\pgfqpoint{5.700000in}{5.700000in}}%
\pgfusepath{clip}%
\pgfsetbuttcap%
\pgfsetroundjoin%
\definecolor{currentfill}{rgb}{0.283197,0.115680,0.436115}%
\pgfsetfillcolor{currentfill}%
\pgfsetfillopacity{0.700000}%
\pgfsetlinewidth{0.000000pt}%
\definecolor{currentstroke}{rgb}{0.000000,0.000000,0.000000}%
\pgfsetstrokecolor{currentstroke}%
\pgfsetdash{}{0pt}%
\pgfpathmoveto{\pgfqpoint{5.596914in}{2.604420in}}%
\pgfpathlineto{\pgfqpoint{5.610716in}{2.602433in}}%
\pgfpathlineto{\pgfqpoint{5.624527in}{2.600470in}}%
\pgfpathlineto{\pgfqpoint{5.638346in}{2.598531in}}%
\pgfpathlineto{\pgfqpoint{5.652172in}{2.596616in}}%
\pgfpathlineto{\pgfqpoint{5.645044in}{2.590784in}}%
\pgfpathlineto{\pgfqpoint{5.637912in}{2.584985in}}%
\pgfpathlineto{\pgfqpoint{5.630775in}{2.579216in}}%
\pgfpathlineto{\pgfqpoint{5.623633in}{2.573471in}}%
\pgfpathlineto{\pgfqpoint{5.609787in}{2.575242in}}%
\pgfpathlineto{\pgfqpoint{5.595950in}{2.577036in}}%
\pgfpathlineto{\pgfqpoint{5.582120in}{2.578855in}}%
\pgfpathlineto{\pgfqpoint{5.568298in}{2.580698in}}%
\pgfpathlineto{\pgfqpoint{5.575459in}{2.586582in}}%
\pgfpathlineto{\pgfqpoint{5.582615in}{2.592495in}}%
\pgfpathlineto{\pgfqpoint{5.589767in}{2.598439in}}%
\pgfpathlineto{\pgfqpoint{5.596914in}{2.604420in}}%
\pgfpathclose%
\pgfusepath{fill}%
\end{pgfscope}%
\begin{pgfscope}%
\pgfpathrectangle{\pgfqpoint{1.150000in}{0.150000in}}{\pgfqpoint{5.700000in}{5.700000in}}%
\pgfusepath{clip}%
\pgfsetbuttcap%
\pgfsetroundjoin%
\definecolor{currentfill}{rgb}{0.282910,0.105393,0.426902}%
\pgfsetfillcolor{currentfill}%
\pgfsetfillopacity{0.700000}%
\pgfsetlinewidth{0.000000pt}%
\definecolor{currentstroke}{rgb}{0.000000,0.000000,0.000000}%
\pgfsetstrokecolor{currentstroke}%
\pgfsetdash{}{0pt}%
\pgfpathmoveto{\pgfqpoint{5.374132in}{2.579791in}}%
\pgfpathlineto{\pgfqpoint{5.387879in}{2.577786in}}%
\pgfpathlineto{\pgfqpoint{5.401634in}{2.575807in}}%
\pgfpathlineto{\pgfqpoint{5.415396in}{2.573851in}}%
\pgfpathlineto{\pgfqpoint{5.429166in}{2.571921in}}%
\pgfpathlineto{\pgfqpoint{5.421943in}{2.565865in}}%
\pgfpathlineto{\pgfqpoint{5.414715in}{2.559820in}}%
\pgfpathlineto{\pgfqpoint{5.407482in}{2.553779in}}%
\pgfpathlineto{\pgfqpoint{5.400244in}{2.547741in}}%
\pgfpathlineto{\pgfqpoint{5.386456in}{2.549554in}}%
\pgfpathlineto{\pgfqpoint{5.372677in}{2.551391in}}%
\pgfpathlineto{\pgfqpoint{5.358905in}{2.553253in}}%
\pgfpathlineto{\pgfqpoint{5.345141in}{2.555140in}}%
\pgfpathlineto{\pgfqpoint{5.352397in}{2.561291in}}%
\pgfpathlineto{\pgfqpoint{5.359647in}{2.567447in}}%
\pgfpathlineto{\pgfqpoint{5.366892in}{2.573613in}}%
\pgfpathlineto{\pgfqpoint{5.374132in}{2.579791in}}%
\pgfpathclose%
\pgfusepath{fill}%
\end{pgfscope}%
\begin{pgfscope}%
\pgfpathrectangle{\pgfqpoint{1.150000in}{0.150000in}}{\pgfqpoint{5.700000in}{5.700000in}}%
\pgfusepath{clip}%
\pgfsetbuttcap%
\pgfsetroundjoin%
\definecolor{currentfill}{rgb}{0.281924,0.089666,0.412415}%
\pgfsetfillcolor{currentfill}%
\pgfsetfillopacity{0.700000}%
\pgfsetlinewidth{0.000000pt}%
\definecolor{currentstroke}{rgb}{0.000000,0.000000,0.000000}%
\pgfsetstrokecolor{currentstroke}%
\pgfsetdash{}{0pt}%
\pgfpathmoveto{\pgfqpoint{5.151290in}{2.553827in}}%
\pgfpathlineto{\pgfqpoint{5.164979in}{2.551748in}}%
\pgfpathlineto{\pgfqpoint{5.178676in}{2.549693in}}%
\pgfpathlineto{\pgfqpoint{5.192380in}{2.547663in}}%
\pgfpathlineto{\pgfqpoint{5.206092in}{2.545658in}}%
\pgfpathlineto{\pgfqpoint{5.198775in}{2.539287in}}%
\pgfpathlineto{\pgfqpoint{5.191452in}{2.532907in}}%
\pgfpathlineto{\pgfqpoint{5.184123in}{2.526515in}}%
\pgfpathlineto{\pgfqpoint{5.176788in}{2.520108in}}%
\pgfpathlineto{\pgfqpoint{5.163061in}{2.522021in}}%
\pgfpathlineto{\pgfqpoint{5.149341in}{2.523960in}}%
\pgfpathlineto{\pgfqpoint{5.135628in}{2.525923in}}%
\pgfpathlineto{\pgfqpoint{5.121923in}{2.527912in}}%
\pgfpathlineto{\pgfqpoint{5.129273in}{2.534405in}}%
\pgfpathlineto{\pgfqpoint{5.136618in}{2.540887in}}%
\pgfpathlineto{\pgfqpoint{5.143957in}{2.547360in}}%
\pgfpathlineto{\pgfqpoint{5.151290in}{2.553827in}}%
\pgfpathclose%
\pgfusepath{fill}%
\end{pgfscope}%
\begin{pgfscope}%
\pgfpathrectangle{\pgfqpoint{1.150000in}{0.150000in}}{\pgfqpoint{5.700000in}{5.700000in}}%
\pgfusepath{clip}%
\pgfsetbuttcap%
\pgfsetroundjoin%
\definecolor{currentfill}{rgb}{0.278791,0.062145,0.386592}%
\pgfsetfillcolor{currentfill}%
\pgfsetfillopacity{0.700000}%
\pgfsetlinewidth{0.000000pt}%
\definecolor{currentstroke}{rgb}{0.000000,0.000000,0.000000}%
\pgfsetstrokecolor{currentstroke}%
\pgfsetdash{}{0pt}%
\pgfpathmoveto{\pgfqpoint{4.705497in}{2.498840in}}%
\pgfpathlineto{\pgfqpoint{4.719070in}{2.496428in}}%
\pgfpathlineto{\pgfqpoint{4.732650in}{2.494043in}}%
\pgfpathlineto{\pgfqpoint{4.746237in}{2.491684in}}%
\pgfpathlineto{\pgfqpoint{4.759831in}{2.489351in}}%
\pgfpathlineto{\pgfqpoint{4.752332in}{2.482260in}}%
\pgfpathlineto{\pgfqpoint{4.744827in}{2.475142in}}%
\pgfpathlineto{\pgfqpoint{4.737315in}{2.467994in}}%
\pgfpathlineto{\pgfqpoint{4.729798in}{2.460816in}}%
\pgfpathlineto{\pgfqpoint{4.716191in}{2.463111in}}%
\pgfpathlineto{\pgfqpoint{4.702590in}{2.465432in}}%
\pgfpathlineto{\pgfqpoint{4.688997in}{2.467780in}}%
\pgfpathlineto{\pgfqpoint{4.675411in}{2.470153in}}%
\pgfpathlineto{\pgfqpoint{4.682941in}{2.477364in}}%
\pgfpathlineto{\pgfqpoint{4.690465in}{2.484548in}}%
\pgfpathlineto{\pgfqpoint{4.697984in}{2.491706in}}%
\pgfpathlineto{\pgfqpoint{4.705497in}{2.498840in}}%
\pgfpathclose%
\pgfusepath{fill}%
\end{pgfscope}%
\begin{pgfscope}%
\pgfpathrectangle{\pgfqpoint{1.150000in}{0.150000in}}{\pgfqpoint{5.700000in}{5.700000in}}%
\pgfusepath{clip}%
\pgfsetbuttcap%
\pgfsetroundjoin%
\definecolor{currentfill}{rgb}{0.277018,0.050344,0.375715}%
\pgfsetfillcolor{currentfill}%
\pgfsetfillopacity{0.700000}%
\pgfsetlinewidth{0.000000pt}%
\definecolor{currentstroke}{rgb}{0.000000,0.000000,0.000000}%
\pgfsetstrokecolor{currentstroke}%
\pgfsetdash{}{0pt}%
\pgfpathmoveto{\pgfqpoint{2.984007in}{2.486817in}}%
\pgfpathlineto{\pgfqpoint{2.997232in}{2.480572in}}%
\pgfpathlineto{\pgfqpoint{3.010462in}{2.474370in}}%
\pgfpathlineto{\pgfqpoint{3.023695in}{2.468213in}}%
\pgfpathlineto{\pgfqpoint{3.036931in}{2.462100in}}%
\pgfpathlineto{\pgfqpoint{3.028750in}{2.456913in}}%
\pgfpathlineto{\pgfqpoint{3.020561in}{2.451826in}}%
\pgfpathlineto{\pgfqpoint{3.012362in}{2.446845in}}%
\pgfpathlineto{\pgfqpoint{3.004154in}{2.441971in}}%
\pgfpathlineto{\pgfqpoint{2.990898in}{2.448233in}}%
\pgfpathlineto{\pgfqpoint{2.977647in}{2.454538in}}%
\pgfpathlineto{\pgfqpoint{2.964398in}{2.460888in}}%
\pgfpathlineto{\pgfqpoint{2.951154in}{2.467281in}}%
\pgfpathlineto{\pgfqpoint{2.959381in}{2.472002in}}%
\pgfpathlineto{\pgfqpoint{2.967599in}{2.476833in}}%
\pgfpathlineto{\pgfqpoint{2.975807in}{2.481773in}}%
\pgfpathlineto{\pgfqpoint{2.984007in}{2.486817in}}%
\pgfpathclose%
\pgfusepath{fill}%
\end{pgfscope}%
\begin{pgfscope}%
\pgfpathrectangle{\pgfqpoint{1.150000in}{0.150000in}}{\pgfqpoint{5.700000in}{5.700000in}}%
\pgfusepath{clip}%
\pgfsetbuttcap%
\pgfsetroundjoin%
\definecolor{currentfill}{rgb}{0.276022,0.044167,0.370164}%
\pgfsetfillcolor{currentfill}%
\pgfsetfillopacity{0.700000}%
\pgfsetlinewidth{0.000000pt}%
\definecolor{currentstroke}{rgb}{0.000000,0.000000,0.000000}%
\pgfsetstrokecolor{currentstroke}%
\pgfsetdash{}{0pt}%
\pgfpathmoveto{\pgfqpoint{4.482584in}{2.471281in}}%
\pgfpathlineto{\pgfqpoint{4.496100in}{2.468610in}}%
\pgfpathlineto{\pgfqpoint{4.509623in}{2.465966in}}%
\pgfpathlineto{\pgfqpoint{4.523152in}{2.463349in}}%
\pgfpathlineto{\pgfqpoint{4.536689in}{2.460759in}}%
\pgfpathlineto{\pgfqpoint{4.529103in}{2.453359in}}%
\pgfpathlineto{\pgfqpoint{4.521512in}{2.445931in}}%
\pgfpathlineto{\pgfqpoint{4.513915in}{2.438474in}}%
\pgfpathlineto{\pgfqpoint{4.506312in}{2.430988in}}%
\pgfpathlineto{\pgfqpoint{4.492763in}{2.433566in}}%
\pgfpathlineto{\pgfqpoint{4.479221in}{2.436172in}}%
\pgfpathlineto{\pgfqpoint{4.465685in}{2.438805in}}%
\pgfpathlineto{\pgfqpoint{4.452157in}{2.441464in}}%
\pgfpathlineto{\pgfqpoint{4.459772in}{2.448957in}}%
\pgfpathlineto{\pgfqpoint{4.467382in}{2.456424in}}%
\pgfpathlineto{\pgfqpoint{4.474986in}{2.463865in}}%
\pgfpathlineto{\pgfqpoint{4.482584in}{2.471281in}}%
\pgfpathclose%
\pgfusepath{fill}%
\end{pgfscope}%
\begin{pgfscope}%
\pgfpathrectangle{\pgfqpoint{1.150000in}{0.150000in}}{\pgfqpoint{5.700000in}{5.700000in}}%
\pgfusepath{clip}%
\pgfsetbuttcap%
\pgfsetroundjoin%
\definecolor{currentfill}{rgb}{0.280267,0.073417,0.397163}%
\pgfsetfillcolor{currentfill}%
\pgfsetfillopacity{0.700000}%
\pgfsetlinewidth{0.000000pt}%
\definecolor{currentstroke}{rgb}{0.000000,0.000000,0.000000}%
\pgfsetstrokecolor{currentstroke}%
\pgfsetdash{}{0pt}%
\pgfpathmoveto{\pgfqpoint{4.928404in}{2.526666in}}%
\pgfpathlineto{\pgfqpoint{4.942035in}{2.524451in}}%
\pgfpathlineto{\pgfqpoint{4.955673in}{2.522261in}}%
\pgfpathlineto{\pgfqpoint{4.969319in}{2.520097in}}%
\pgfpathlineto{\pgfqpoint{4.982972in}{2.517959in}}%
\pgfpathlineto{\pgfqpoint{4.975562in}{2.511227in}}%
\pgfpathlineto{\pgfqpoint{4.968146in}{2.504473in}}%
\pgfpathlineto{\pgfqpoint{4.960725in}{2.497696in}}%
\pgfpathlineto{\pgfqpoint{4.953297in}{2.490894in}}%
\pgfpathlineto{\pgfqpoint{4.939630in}{2.492967in}}%
\pgfpathlineto{\pgfqpoint{4.925970in}{2.495067in}}%
\pgfpathlineto{\pgfqpoint{4.912317in}{2.497191in}}%
\pgfpathlineto{\pgfqpoint{4.898671in}{2.499342in}}%
\pgfpathlineto{\pgfqpoint{4.906113in}{2.506204in}}%
\pgfpathlineto{\pgfqpoint{4.913549in}{2.513044in}}%
\pgfpathlineto{\pgfqpoint{4.920980in}{2.519864in}}%
\pgfpathlineto{\pgfqpoint{4.928404in}{2.526666in}}%
\pgfpathclose%
\pgfusepath{fill}%
\end{pgfscope}%
\begin{pgfscope}%
\pgfpathrectangle{\pgfqpoint{1.150000in}{0.150000in}}{\pgfqpoint{5.700000in}{5.700000in}}%
\pgfusepath{clip}%
\pgfsetbuttcap%
\pgfsetroundjoin%
\definecolor{currentfill}{rgb}{0.272594,0.025563,0.353093}%
\pgfsetfillcolor{currentfill}%
\pgfsetfillopacity{0.700000}%
\pgfsetlinewidth{0.000000pt}%
\definecolor{currentstroke}{rgb}{0.000000,0.000000,0.000000}%
\pgfsetstrokecolor{currentstroke}%
\pgfsetdash{}{0pt}%
\pgfpathmoveto{\pgfqpoint{4.259667in}{2.445319in}}%
\pgfpathlineto{\pgfqpoint{4.273128in}{2.442323in}}%
\pgfpathlineto{\pgfqpoint{4.286596in}{2.439355in}}%
\pgfpathlineto{\pgfqpoint{4.300070in}{2.436416in}}%
\pgfpathlineto{\pgfqpoint{4.313551in}{2.433504in}}%
\pgfpathlineto{\pgfqpoint{4.305882in}{2.425891in}}%
\pgfpathlineto{\pgfqpoint{4.298208in}{2.418256in}}%
\pgfpathlineto{\pgfqpoint{4.290527in}{2.410597in}}%
\pgfpathlineto{\pgfqpoint{4.282841in}{2.402916in}}%
\pgfpathlineto{\pgfqpoint{4.269348in}{2.405842in}}%
\pgfpathlineto{\pgfqpoint{4.255862in}{2.408797in}}%
\pgfpathlineto{\pgfqpoint{4.242381in}{2.411780in}}%
\pgfpathlineto{\pgfqpoint{4.228908in}{2.414792in}}%
\pgfpathlineto{\pgfqpoint{4.236606in}{2.422453in}}%
\pgfpathlineto{\pgfqpoint{4.244299in}{2.430094in}}%
\pgfpathlineto{\pgfqpoint{4.251985in}{2.437716in}}%
\pgfpathlineto{\pgfqpoint{4.259667in}{2.445319in}}%
\pgfpathclose%
\pgfusepath{fill}%
\end{pgfscope}%
\begin{pgfscope}%
\pgfpathrectangle{\pgfqpoint{1.150000in}{0.150000in}}{\pgfqpoint{5.700000in}{5.700000in}}%
\pgfusepath{clip}%
\pgfsetbuttcap%
\pgfsetroundjoin%
\definecolor{currentfill}{rgb}{0.267004,0.004874,0.329415}%
\pgfsetfillcolor{currentfill}%
\pgfsetfillopacity{0.700000}%
\pgfsetlinewidth{0.000000pt}%
\definecolor{currentstroke}{rgb}{0.000000,0.000000,0.000000}%
\pgfsetstrokecolor{currentstroke}%
\pgfsetdash{}{0pt}%
\pgfpathmoveto{\pgfqpoint{3.675462in}{2.408110in}}%
\pgfpathlineto{\pgfqpoint{3.688794in}{2.403929in}}%
\pgfpathlineto{\pgfqpoint{3.702132in}{2.399781in}}%
\pgfpathlineto{\pgfqpoint{3.715476in}{2.395666in}}%
\pgfpathlineto{\pgfqpoint{3.728825in}{2.391583in}}%
\pgfpathlineto{\pgfqpoint{3.720941in}{2.384200in}}%
\pgfpathlineto{\pgfqpoint{3.713051in}{2.376834in}}%
\pgfpathlineto{\pgfqpoint{3.705155in}{2.369488in}}%
\pgfpathlineto{\pgfqpoint{3.697253in}{2.362162in}}%
\pgfpathlineto{\pgfqpoint{3.683890in}{2.366326in}}%
\pgfpathlineto{\pgfqpoint{3.670533in}{2.370523in}}%
\pgfpathlineto{\pgfqpoint{3.657182in}{2.374753in}}%
\pgfpathlineto{\pgfqpoint{3.643836in}{2.379016in}}%
\pgfpathlineto{\pgfqpoint{3.651752in}{2.386255in}}%
\pgfpathlineto{\pgfqpoint{3.659661in}{2.393518in}}%
\pgfpathlineto{\pgfqpoint{3.667564in}{2.400804in}}%
\pgfpathlineto{\pgfqpoint{3.675462in}{2.408110in}}%
\pgfpathclose%
\pgfusepath{fill}%
\end{pgfscope}%
\begin{pgfscope}%
\pgfpathrectangle{\pgfqpoint{1.150000in}{0.150000in}}{\pgfqpoint{5.700000in}{5.700000in}}%
\pgfusepath{clip}%
\pgfsetbuttcap%
\pgfsetroundjoin%
\definecolor{currentfill}{rgb}{0.269944,0.014625,0.341379}%
\pgfsetfillcolor{currentfill}%
\pgfsetfillopacity{0.700000}%
\pgfsetlinewidth{0.000000pt}%
\definecolor{currentstroke}{rgb}{0.000000,0.000000,0.000000}%
\pgfsetstrokecolor{currentstroke}%
\pgfsetdash{}{0pt}%
\pgfpathmoveto{\pgfqpoint{4.036723in}{2.422687in}}%
\pgfpathlineto{\pgfqpoint{4.050133in}{2.419297in}}%
\pgfpathlineto{\pgfqpoint{4.063550in}{2.415938in}}%
\pgfpathlineto{\pgfqpoint{4.076972in}{2.412608in}}%
\pgfpathlineto{\pgfqpoint{4.090401in}{2.409307in}}%
\pgfpathlineto{\pgfqpoint{4.082650in}{2.401624in}}%
\pgfpathlineto{\pgfqpoint{4.074893in}{2.393929in}}%
\pgfpathlineto{\pgfqpoint{4.067131in}{2.386223in}}%
\pgfpathlineto{\pgfqpoint{4.059363in}{2.378506in}}%
\pgfpathlineto{\pgfqpoint{4.045922in}{2.381848in}}%
\pgfpathlineto{\pgfqpoint{4.032487in}{2.385220in}}%
\pgfpathlineto{\pgfqpoint{4.019058in}{2.388622in}}%
\pgfpathlineto{\pgfqpoint{4.005635in}{2.392053in}}%
\pgfpathlineto{\pgfqpoint{4.013416in}{2.399722in}}%
\pgfpathlineto{\pgfqpoint{4.021191in}{2.407385in}}%
\pgfpathlineto{\pgfqpoint{4.028960in}{2.415040in}}%
\pgfpathlineto{\pgfqpoint{4.036723in}{2.422687in}}%
\pgfpathclose%
\pgfusepath{fill}%
\end{pgfscope}%
\begin{pgfscope}%
\pgfpathrectangle{\pgfqpoint{1.150000in}{0.150000in}}{\pgfqpoint{5.700000in}{5.700000in}}%
\pgfusepath{clip}%
\pgfsetbuttcap%
\pgfsetroundjoin%
\definecolor{currentfill}{rgb}{0.280267,0.073417,0.397163}%
\pgfsetfillcolor{currentfill}%
\pgfsetfillopacity{0.700000}%
\pgfsetlinewidth{0.000000pt}%
\definecolor{currentstroke}{rgb}{0.000000,0.000000,0.000000}%
\pgfsetstrokecolor{currentstroke}%
\pgfsetdash{}{0pt}%
\pgfpathmoveto{\pgfqpoint{2.845321in}{2.520083in}}%
\pgfpathlineto{\pgfqpoint{2.858538in}{2.513318in}}%
\pgfpathlineto{\pgfqpoint{2.871759in}{2.506601in}}%
\pgfpathlineto{\pgfqpoint{2.884983in}{2.499932in}}%
\pgfpathlineto{\pgfqpoint{2.898210in}{2.493310in}}%
\pgfpathlineto{\pgfqpoint{2.889954in}{2.488861in}}%
\pgfpathlineto{\pgfqpoint{2.881687in}{2.484535in}}%
\pgfpathlineto{\pgfqpoint{2.873411in}{2.480336in}}%
\pgfpathlineto{\pgfqpoint{2.865124in}{2.476266in}}%
\pgfpathlineto{\pgfqpoint{2.851876in}{2.483050in}}%
\pgfpathlineto{\pgfqpoint{2.838632in}{2.489882in}}%
\pgfpathlineto{\pgfqpoint{2.825390in}{2.496761in}}%
\pgfpathlineto{\pgfqpoint{2.812152in}{2.503688in}}%
\pgfpathlineto{\pgfqpoint{2.820460in}{2.507590in}}%
\pgfpathlineto{\pgfqpoint{2.828757in}{2.511626in}}%
\pgfpathlineto{\pgfqpoint{2.837044in}{2.515792in}}%
\pgfpathlineto{\pgfqpoint{2.845321in}{2.520083in}}%
\pgfpathclose%
\pgfusepath{fill}%
\end{pgfscope}%
\begin{pgfscope}%
\pgfpathrectangle{\pgfqpoint{1.150000in}{0.150000in}}{\pgfqpoint{5.700000in}{5.700000in}}%
\pgfusepath{clip}%
\pgfsetbuttcap%
\pgfsetroundjoin%
\definecolor{currentfill}{rgb}{0.267004,0.004874,0.329415}%
\pgfsetfillcolor{currentfill}%
\pgfsetfillopacity{0.700000}%
\pgfsetlinewidth{0.000000pt}%
\definecolor{currentstroke}{rgb}{0.000000,0.000000,0.000000}%
\pgfsetstrokecolor{currentstroke}%
\pgfsetdash{}{0pt}%
\pgfpathmoveto{\pgfqpoint{3.813701in}{2.405526in}}%
\pgfpathlineto{\pgfqpoint{3.827065in}{2.401671in}}%
\pgfpathlineto{\pgfqpoint{3.840435in}{2.397848in}}%
\pgfpathlineto{\pgfqpoint{3.853810in}{2.394057in}}%
\pgfpathlineto{\pgfqpoint{3.867192in}{2.390297in}}%
\pgfpathlineto{\pgfqpoint{3.859358in}{2.382731in}}%
\pgfpathlineto{\pgfqpoint{3.851517in}{2.375171in}}%
\pgfpathlineto{\pgfqpoint{3.843672in}{2.367617in}}%
\pgfpathlineto{\pgfqpoint{3.835820in}{2.360071in}}%
\pgfpathlineto{\pgfqpoint{3.822425in}{2.363900in}}%
\pgfpathlineto{\pgfqpoint{3.809037in}{2.367760in}}%
\pgfpathlineto{\pgfqpoint{3.795654in}{2.371651in}}%
\pgfpathlineto{\pgfqpoint{3.782277in}{2.375574in}}%
\pgfpathlineto{\pgfqpoint{3.790141in}{2.383046in}}%
\pgfpathlineto{\pgfqpoint{3.798000in}{2.390530in}}%
\pgfpathlineto{\pgfqpoint{3.805853in}{2.398023in}}%
\pgfpathlineto{\pgfqpoint{3.813701in}{2.405526in}}%
\pgfpathclose%
\pgfusepath{fill}%
\end{pgfscope}%
\begin{pgfscope}%
\pgfpathrectangle{\pgfqpoint{1.150000in}{0.150000in}}{\pgfqpoint{5.700000in}{5.700000in}}%
\pgfusepath{clip}%
\pgfsetbuttcap%
\pgfsetroundjoin%
\definecolor{currentfill}{rgb}{0.283187,0.125848,0.444960}%
\pgfsetfillcolor{currentfill}%
\pgfsetfillopacity{0.700000}%
\pgfsetlinewidth{0.000000pt}%
\definecolor{currentstroke}{rgb}{0.000000,0.000000,0.000000}%
\pgfsetstrokecolor{currentstroke}%
\pgfsetdash{}{0pt}%
\pgfpathmoveto{\pgfqpoint{5.735946in}{2.612324in}}%
\pgfpathlineto{\pgfqpoint{5.749792in}{2.610372in}}%
\pgfpathlineto{\pgfqpoint{5.763645in}{2.608443in}}%
\pgfpathlineto{\pgfqpoint{5.777507in}{2.606538in}}%
\pgfpathlineto{\pgfqpoint{5.791377in}{2.604658in}}%
\pgfpathlineto{\pgfqpoint{5.784306in}{2.598969in}}%
\pgfpathlineto{\pgfqpoint{5.777230in}{2.593326in}}%
\pgfpathlineto{\pgfqpoint{5.770151in}{2.587725in}}%
\pgfpathlineto{\pgfqpoint{5.763067in}{2.582161in}}%
\pgfpathlineto{\pgfqpoint{5.749177in}{2.583884in}}%
\pgfpathlineto{\pgfqpoint{5.735296in}{2.585630in}}%
\pgfpathlineto{\pgfqpoint{5.721422in}{2.587401in}}%
\pgfpathlineto{\pgfqpoint{5.707556in}{2.589196in}}%
\pgfpathlineto{\pgfqpoint{5.714660in}{2.594914in}}%
\pgfpathlineto{\pgfqpoint{5.721759in}{2.600671in}}%
\pgfpathlineto{\pgfqpoint{5.728854in}{2.606473in}}%
\pgfpathlineto{\pgfqpoint{5.735946in}{2.612324in}}%
\pgfpathclose%
\pgfusepath{fill}%
\end{pgfscope}%
\begin{pgfscope}%
\pgfpathrectangle{\pgfqpoint{1.150000in}{0.150000in}}{\pgfqpoint{5.700000in}{5.700000in}}%
\pgfusepath{clip}%
\pgfsetbuttcap%
\pgfsetroundjoin%
\definecolor{currentfill}{rgb}{0.283197,0.115680,0.436115}%
\pgfsetfillcolor{currentfill}%
\pgfsetfillopacity{0.700000}%
\pgfsetlinewidth{0.000000pt}%
\definecolor{currentstroke}{rgb}{0.000000,0.000000,0.000000}%
\pgfsetstrokecolor{currentstroke}%
\pgfsetdash{}{0pt}%
\pgfpathmoveto{\pgfqpoint{5.513089in}{2.588312in}}%
\pgfpathlineto{\pgfqpoint{5.526880in}{2.586372in}}%
\pgfpathlineto{\pgfqpoint{5.540678in}{2.584457in}}%
\pgfpathlineto{\pgfqpoint{5.554484in}{2.582565in}}%
\pgfpathlineto{\pgfqpoint{5.568298in}{2.580698in}}%
\pgfpathlineto{\pgfqpoint{5.561133in}{2.574837in}}%
\pgfpathlineto{\pgfqpoint{5.553962in}{2.568996in}}%
\pgfpathlineto{\pgfqpoint{5.546786in}{2.563169in}}%
\pgfpathlineto{\pgfqpoint{5.539605in}{2.557353in}}%
\pgfpathlineto{\pgfqpoint{5.525773in}{2.559089in}}%
\pgfpathlineto{\pgfqpoint{5.511948in}{2.560849in}}%
\pgfpathlineto{\pgfqpoint{5.498131in}{2.562633in}}%
\pgfpathlineto{\pgfqpoint{5.484323in}{2.564442in}}%
\pgfpathlineto{\pgfqpoint{5.491522in}{2.570384in}}%
\pgfpathlineto{\pgfqpoint{5.498716in}{2.576341in}}%
\pgfpathlineto{\pgfqpoint{5.505905in}{2.582315in}}%
\pgfpathlineto{\pgfqpoint{5.513089in}{2.588312in}}%
\pgfpathclose%
\pgfusepath{fill}%
\end{pgfscope}%
\begin{pgfscope}%
\pgfpathrectangle{\pgfqpoint{1.150000in}{0.150000in}}{\pgfqpoint{5.700000in}{5.700000in}}%
\pgfusepath{clip}%
\pgfsetbuttcap%
\pgfsetroundjoin%
\definecolor{currentfill}{rgb}{0.282656,0.100196,0.422160}%
\pgfsetfillcolor{currentfill}%
\pgfsetfillopacity{0.700000}%
\pgfsetlinewidth{0.000000pt}%
\definecolor{currentstroke}{rgb}{0.000000,0.000000,0.000000}%
\pgfsetstrokecolor{currentstroke}%
\pgfsetdash{}{0pt}%
\pgfpathmoveto{\pgfqpoint{5.290162in}{2.562932in}}%
\pgfpathlineto{\pgfqpoint{5.303895in}{2.560947in}}%
\pgfpathlineto{\pgfqpoint{5.317636in}{2.558986in}}%
\pgfpathlineto{\pgfqpoint{5.331385in}{2.557051in}}%
\pgfpathlineto{\pgfqpoint{5.345141in}{2.555140in}}%
\pgfpathlineto{\pgfqpoint{5.337879in}{2.548990in}}%
\pgfpathlineto{\pgfqpoint{5.330612in}{2.542838in}}%
\pgfpathlineto{\pgfqpoint{5.323340in}{2.536680in}}%
\pgfpathlineto{\pgfqpoint{5.316061in}{2.530514in}}%
\pgfpathlineto{\pgfqpoint{5.302288in}{2.532320in}}%
\pgfpathlineto{\pgfqpoint{5.288522in}{2.534151in}}%
\pgfpathlineto{\pgfqpoint{5.274765in}{2.536007in}}%
\pgfpathlineto{\pgfqpoint{5.261015in}{2.537888in}}%
\pgfpathlineto{\pgfqpoint{5.268310in}{2.544154in}}%
\pgfpathlineto{\pgfqpoint{5.275600in}{2.550414in}}%
\pgfpathlineto{\pgfqpoint{5.282883in}{2.556673in}}%
\pgfpathlineto{\pgfqpoint{5.290162in}{2.562932in}}%
\pgfpathclose%
\pgfusepath{fill}%
\end{pgfscope}%
\begin{pgfscope}%
\pgfpathrectangle{\pgfqpoint{1.150000in}{0.150000in}}{\pgfqpoint{5.700000in}{5.700000in}}%
\pgfusepath{clip}%
\pgfsetbuttcap%
\pgfsetroundjoin%
\definecolor{currentfill}{rgb}{0.277941,0.056324,0.381191}%
\pgfsetfillcolor{currentfill}%
\pgfsetfillopacity{0.700000}%
\pgfsetlinewidth{0.000000pt}%
\definecolor{currentstroke}{rgb}{0.000000,0.000000,0.000000}%
\pgfsetstrokecolor{currentstroke}%
\pgfsetdash{}{0pt}%
\pgfpathmoveto{\pgfqpoint{4.621136in}{2.479913in}}%
\pgfpathlineto{\pgfqpoint{4.634694in}{2.477434in}}%
\pgfpathlineto{\pgfqpoint{4.648259in}{2.474980in}}%
\pgfpathlineto{\pgfqpoint{4.661831in}{2.472554in}}%
\pgfpathlineto{\pgfqpoint{4.675411in}{2.470153in}}%
\pgfpathlineto{\pgfqpoint{4.667874in}{2.462913in}}%
\pgfpathlineto{\pgfqpoint{4.660332in}{2.455644in}}%
\pgfpathlineto{\pgfqpoint{4.652784in}{2.448342in}}%
\pgfpathlineto{\pgfqpoint{4.645230in}{2.441009in}}%
\pgfpathlineto{\pgfqpoint{4.631638in}{2.443384in}}%
\pgfpathlineto{\pgfqpoint{4.618053in}{2.445786in}}%
\pgfpathlineto{\pgfqpoint{4.604475in}{2.448215in}}%
\pgfpathlineto{\pgfqpoint{4.590904in}{2.450670in}}%
\pgfpathlineto{\pgfqpoint{4.598471in}{2.458024in}}%
\pgfpathlineto{\pgfqpoint{4.606032in}{2.465348in}}%
\pgfpathlineto{\pgfqpoint{4.613587in}{2.472644in}}%
\pgfpathlineto{\pgfqpoint{4.621136in}{2.479913in}}%
\pgfpathclose%
\pgfusepath{fill}%
\end{pgfscope}%
\begin{pgfscope}%
\pgfpathrectangle{\pgfqpoint{1.150000in}{0.150000in}}{\pgfqpoint{5.700000in}{5.700000in}}%
\pgfusepath{clip}%
\pgfsetbuttcap%
\pgfsetroundjoin%
\definecolor{currentfill}{rgb}{0.274952,0.037752,0.364543}%
\pgfsetfillcolor{currentfill}%
\pgfsetfillopacity{0.700000}%
\pgfsetlinewidth{0.000000pt}%
\definecolor{currentstroke}{rgb}{0.000000,0.000000,0.000000}%
\pgfsetstrokecolor{currentstroke}%
\pgfsetdash{}{0pt}%
\pgfpathmoveto{\pgfqpoint{4.398110in}{2.452379in}}%
\pgfpathlineto{\pgfqpoint{4.411612in}{2.449609in}}%
\pgfpathlineto{\pgfqpoint{4.425120in}{2.446867in}}%
\pgfpathlineto{\pgfqpoint{4.438635in}{2.444152in}}%
\pgfpathlineto{\pgfqpoint{4.452157in}{2.441464in}}%
\pgfpathlineto{\pgfqpoint{4.444536in}{2.433945in}}%
\pgfpathlineto{\pgfqpoint{4.436909in}{2.426397in}}%
\pgfpathlineto{\pgfqpoint{4.429276in}{2.418821in}}%
\pgfpathlineto{\pgfqpoint{4.421637in}{2.411216in}}%
\pgfpathlineto{\pgfqpoint{4.408103in}{2.413906in}}%
\pgfpathlineto{\pgfqpoint{4.394576in}{2.416623in}}%
\pgfpathlineto{\pgfqpoint{4.381055in}{2.419367in}}%
\pgfpathlineto{\pgfqpoint{4.367541in}{2.422139in}}%
\pgfpathlineto{\pgfqpoint{4.375192in}{2.429736in}}%
\pgfpathlineto{\pgfqpoint{4.382837in}{2.437309in}}%
\pgfpathlineto{\pgfqpoint{4.390476in}{2.444856in}}%
\pgfpathlineto{\pgfqpoint{4.398110in}{2.452379in}}%
\pgfpathclose%
\pgfusepath{fill}%
\end{pgfscope}%
\begin{pgfscope}%
\pgfpathrectangle{\pgfqpoint{1.150000in}{0.150000in}}{\pgfqpoint{5.700000in}{5.700000in}}%
\pgfusepath{clip}%
\pgfsetbuttcap%
\pgfsetroundjoin%
\definecolor{currentfill}{rgb}{0.279566,0.067836,0.391917}%
\pgfsetfillcolor{currentfill}%
\pgfsetfillopacity{0.700000}%
\pgfsetlinewidth{0.000000pt}%
\definecolor{currentstroke}{rgb}{0.000000,0.000000,0.000000}%
\pgfsetstrokecolor{currentstroke}%
\pgfsetdash{}{0pt}%
\pgfpathmoveto{\pgfqpoint{4.844163in}{2.508202in}}%
\pgfpathlineto{\pgfqpoint{4.857779in}{2.505948in}}%
\pgfpathlineto{\pgfqpoint{4.871403in}{2.503720in}}%
\pgfpathlineto{\pgfqpoint{4.885033in}{2.501518in}}%
\pgfpathlineto{\pgfqpoint{4.898671in}{2.499342in}}%
\pgfpathlineto{\pgfqpoint{4.891224in}{2.492455in}}%
\pgfpathlineto{\pgfqpoint{4.883770in}{2.485542in}}%
\pgfpathlineto{\pgfqpoint{4.876310in}{2.478600in}}%
\pgfpathlineto{\pgfqpoint{4.868844in}{2.471627in}}%
\pgfpathlineto{\pgfqpoint{4.855192in}{2.473752in}}%
\pgfpathlineto{\pgfqpoint{4.841547in}{2.475903in}}%
\pgfpathlineto{\pgfqpoint{4.827910in}{2.478079in}}%
\pgfpathlineto{\pgfqpoint{4.814279in}{2.480282in}}%
\pgfpathlineto{\pgfqpoint{4.821759in}{2.487301in}}%
\pgfpathlineto{\pgfqpoint{4.829233in}{2.494292in}}%
\pgfpathlineto{\pgfqpoint{4.836701in}{2.501259in}}%
\pgfpathlineto{\pgfqpoint{4.844163in}{2.508202in}}%
\pgfpathclose%
\pgfusepath{fill}%
\end{pgfscope}%
\begin{pgfscope}%
\pgfpathrectangle{\pgfqpoint{1.150000in}{0.150000in}}{\pgfqpoint{5.700000in}{5.700000in}}%
\pgfusepath{clip}%
\pgfsetbuttcap%
\pgfsetroundjoin%
\definecolor{currentfill}{rgb}{0.269944,0.014625,0.341379}%
\pgfsetfillcolor{currentfill}%
\pgfsetfillopacity{0.700000}%
\pgfsetlinewidth{0.000000pt}%
\definecolor{currentstroke}{rgb}{0.000000,0.000000,0.000000}%
\pgfsetstrokecolor{currentstroke}%
\pgfsetdash{}{0pt}%
\pgfpathmoveto{\pgfqpoint{3.313868in}{2.418764in}}%
\pgfpathlineto{\pgfqpoint{3.327143in}{2.413594in}}%
\pgfpathlineto{\pgfqpoint{3.340422in}{2.408461in}}%
\pgfpathlineto{\pgfqpoint{3.353706in}{2.403366in}}%
\pgfpathlineto{\pgfqpoint{3.366995in}{2.398308in}}%
\pgfpathlineto{\pgfqpoint{3.358962in}{2.391845in}}%
\pgfpathlineto{\pgfqpoint{3.350921in}{2.385442in}}%
\pgfpathlineto{\pgfqpoint{3.342874in}{2.379102in}}%
\pgfpathlineto{\pgfqpoint{3.334819in}{2.372827in}}%
\pgfpathlineto{\pgfqpoint{3.321515in}{2.378006in}}%
\pgfpathlineto{\pgfqpoint{3.308215in}{2.383223in}}%
\pgfpathlineto{\pgfqpoint{3.294919in}{2.388476in}}%
\pgfpathlineto{\pgfqpoint{3.281629in}{2.393768in}}%
\pgfpathlineto{\pgfqpoint{3.289700in}{2.399917in}}%
\pgfpathlineto{\pgfqpoint{3.297763in}{2.406134in}}%
\pgfpathlineto{\pgfqpoint{3.305819in}{2.412417in}}%
\pgfpathlineto{\pgfqpoint{3.313868in}{2.418764in}}%
\pgfpathclose%
\pgfusepath{fill}%
\end{pgfscope}%
\begin{pgfscope}%
\pgfpathrectangle{\pgfqpoint{1.150000in}{0.150000in}}{\pgfqpoint{5.700000in}{5.700000in}}%
\pgfusepath{clip}%
\pgfsetbuttcap%
\pgfsetroundjoin%
\definecolor{currentfill}{rgb}{0.281446,0.084320,0.407414}%
\pgfsetfillcolor{currentfill}%
\pgfsetfillopacity{0.700000}%
\pgfsetlinewidth{0.000000pt}%
\definecolor{currentstroke}{rgb}{0.000000,0.000000,0.000000}%
\pgfsetstrokecolor{currentstroke}%
\pgfsetdash{}{0pt}%
\pgfpathmoveto{\pgfqpoint{5.067178in}{2.536118in}}%
\pgfpathlineto{\pgfqpoint{5.080853in}{2.534029in}}%
\pgfpathlineto{\pgfqpoint{5.094536in}{2.531964in}}%
\pgfpathlineto{\pgfqpoint{5.108226in}{2.529926in}}%
\pgfpathlineto{\pgfqpoint{5.121923in}{2.527912in}}%
\pgfpathlineto{\pgfqpoint{5.114567in}{2.521404in}}%
\pgfpathlineto{\pgfqpoint{5.107205in}{2.514879in}}%
\pgfpathlineto{\pgfqpoint{5.099837in}{2.508334in}}%
\pgfpathlineto{\pgfqpoint{5.092463in}{2.501766in}}%
\pgfpathlineto{\pgfqpoint{5.078751in}{2.503702in}}%
\pgfpathlineto{\pgfqpoint{5.065045in}{2.505663in}}%
\pgfpathlineto{\pgfqpoint{5.051348in}{2.507649in}}%
\pgfpathlineto{\pgfqpoint{5.037658in}{2.509660in}}%
\pgfpathlineto{\pgfqpoint{5.045047in}{2.516301in}}%
\pgfpathlineto{\pgfqpoint{5.052430in}{2.522922in}}%
\pgfpathlineto{\pgfqpoint{5.059807in}{2.529527in}}%
\pgfpathlineto{\pgfqpoint{5.067178in}{2.536118in}}%
\pgfpathclose%
\pgfusepath{fill}%
\end{pgfscope}%
\begin{pgfscope}%
\pgfpathrectangle{\pgfqpoint{1.150000in}{0.150000in}}{\pgfqpoint{5.700000in}{5.700000in}}%
\pgfusepath{clip}%
\pgfsetbuttcap%
\pgfsetroundjoin%
\definecolor{currentfill}{rgb}{0.271305,0.019942,0.347269}%
\pgfsetfillcolor{currentfill}%
\pgfsetfillopacity{0.700000}%
\pgfsetlinewidth{0.000000pt}%
\definecolor{currentstroke}{rgb}{0.000000,0.000000,0.000000}%
\pgfsetstrokecolor{currentstroke}%
\pgfsetdash{}{0pt}%
\pgfpathmoveto{\pgfqpoint{4.175078in}{2.427123in}}%
\pgfpathlineto{\pgfqpoint{4.188526in}{2.423997in}}%
\pgfpathlineto{\pgfqpoint{4.201980in}{2.420900in}}%
\pgfpathlineto{\pgfqpoint{4.215441in}{2.417832in}}%
\pgfpathlineto{\pgfqpoint{4.228908in}{2.414792in}}%
\pgfpathlineto{\pgfqpoint{4.221204in}{2.407111in}}%
\pgfpathlineto{\pgfqpoint{4.213494in}{2.399411in}}%
\pgfpathlineto{\pgfqpoint{4.205779in}{2.391691in}}%
\pgfpathlineto{\pgfqpoint{4.198058in}{2.383953in}}%
\pgfpathlineto{\pgfqpoint{4.184579in}{2.387021in}}%
\pgfpathlineto{\pgfqpoint{4.171106in}{2.390118in}}%
\pgfpathlineto{\pgfqpoint{4.157639in}{2.393244in}}%
\pgfpathlineto{\pgfqpoint{4.144179in}{2.396399in}}%
\pgfpathlineto{\pgfqpoint{4.151912in}{2.404104in}}%
\pgfpathlineto{\pgfqpoint{4.159640in}{2.411793in}}%
\pgfpathlineto{\pgfqpoint{4.167361in}{2.419466in}}%
\pgfpathlineto{\pgfqpoint{4.175078in}{2.427123in}}%
\pgfpathclose%
\pgfusepath{fill}%
\end{pgfscope}%
\begin{pgfscope}%
\pgfpathrectangle{\pgfqpoint{1.150000in}{0.150000in}}{\pgfqpoint{5.700000in}{5.700000in}}%
\pgfusepath{clip}%
\pgfsetbuttcap%
\pgfsetroundjoin%
\definecolor{currentfill}{rgb}{0.272594,0.025563,0.353093}%
\pgfsetfillcolor{currentfill}%
\pgfsetfillopacity{0.700000}%
\pgfsetlinewidth{0.000000pt}%
\definecolor{currentstroke}{rgb}{0.000000,0.000000,0.000000}%
\pgfsetstrokecolor{currentstroke}%
\pgfsetdash{}{0pt}%
\pgfpathmoveto{\pgfqpoint{3.175463in}{2.437488in}}%
\pgfpathlineto{\pgfqpoint{3.188719in}{2.431885in}}%
\pgfpathlineto{\pgfqpoint{3.201979in}{2.426322in}}%
\pgfpathlineto{\pgfqpoint{3.215243in}{2.420799in}}%
\pgfpathlineto{\pgfqpoint{3.228511in}{2.415315in}}%
\pgfpathlineto{\pgfqpoint{3.220416in}{2.409368in}}%
\pgfpathlineto{\pgfqpoint{3.212313in}{2.403498in}}%
\pgfpathlineto{\pgfqpoint{3.204202in}{2.397710in}}%
\pgfpathlineto{\pgfqpoint{3.196083in}{2.392006in}}%
\pgfpathlineto{\pgfqpoint{3.182798in}{2.397624in}}%
\pgfpathlineto{\pgfqpoint{3.169517in}{2.403282in}}%
\pgfpathlineto{\pgfqpoint{3.156240in}{2.408979in}}%
\pgfpathlineto{\pgfqpoint{3.142967in}{2.414717in}}%
\pgfpathlineto{\pgfqpoint{3.151103in}{2.420281in}}%
\pgfpathlineto{\pgfqpoint{3.159231in}{2.425933in}}%
\pgfpathlineto{\pgfqpoint{3.167351in}{2.431670in}}%
\pgfpathlineto{\pgfqpoint{3.175463in}{2.437488in}}%
\pgfpathclose%
\pgfusepath{fill}%
\end{pgfscope}%
\begin{pgfscope}%
\pgfpathrectangle{\pgfqpoint{1.150000in}{0.150000in}}{\pgfqpoint{5.700000in}{5.700000in}}%
\pgfusepath{clip}%
\pgfsetbuttcap%
\pgfsetroundjoin%
\definecolor{currentfill}{rgb}{0.268510,0.009605,0.335427}%
\pgfsetfillcolor{currentfill}%
\pgfsetfillopacity{0.700000}%
\pgfsetlinewidth{0.000000pt}%
\definecolor{currentstroke}{rgb}{0.000000,0.000000,0.000000}%
\pgfsetstrokecolor{currentstroke}%
\pgfsetdash{}{0pt}%
\pgfpathmoveto{\pgfqpoint{3.452200in}{2.405271in}}%
\pgfpathlineto{\pgfqpoint{3.465498in}{2.400502in}}%
\pgfpathlineto{\pgfqpoint{3.478801in}{2.395769in}}%
\pgfpathlineto{\pgfqpoint{3.492109in}{2.391071in}}%
\pgfpathlineto{\pgfqpoint{3.505422in}{2.386408in}}%
\pgfpathlineto{\pgfqpoint{3.497446in}{2.379525in}}%
\pgfpathlineto{\pgfqpoint{3.489464in}{2.372686in}}%
\pgfpathlineto{\pgfqpoint{3.481475in}{2.365892in}}%
\pgfpathlineto{\pgfqpoint{3.473479in}{2.359146in}}%
\pgfpathlineto{\pgfqpoint{3.460151in}{2.363917in}}%
\pgfpathlineto{\pgfqpoint{3.446828in}{2.368723in}}%
\pgfpathlineto{\pgfqpoint{3.433511in}{2.373564in}}%
\pgfpathlineto{\pgfqpoint{3.420198in}{2.378441in}}%
\pgfpathlineto{\pgfqpoint{3.428209in}{2.385074in}}%
\pgfpathlineto{\pgfqpoint{3.436212in}{2.391758in}}%
\pgfpathlineto{\pgfqpoint{3.444210in}{2.398491in}}%
\pgfpathlineto{\pgfqpoint{3.452200in}{2.405271in}}%
\pgfpathclose%
\pgfusepath{fill}%
\end{pgfscope}%
\begin{pgfscope}%
\pgfpathrectangle{\pgfqpoint{1.150000in}{0.150000in}}{\pgfqpoint{5.700000in}{5.700000in}}%
\pgfusepath{clip}%
\pgfsetbuttcap%
\pgfsetroundjoin%
\definecolor{currentfill}{rgb}{0.268510,0.009605,0.335427}%
\pgfsetfillcolor{currentfill}%
\pgfsetfillopacity{0.700000}%
\pgfsetlinewidth{0.000000pt}%
\definecolor{currentstroke}{rgb}{0.000000,0.000000,0.000000}%
\pgfsetstrokecolor{currentstroke}%
\pgfsetdash{}{0pt}%
\pgfpathmoveto{\pgfqpoint{3.952005in}{2.406077in}}%
\pgfpathlineto{\pgfqpoint{3.965404in}{2.402526in}}%
\pgfpathlineto{\pgfqpoint{3.978808in}{2.399005in}}%
\pgfpathlineto{\pgfqpoint{3.992219in}{2.395514in}}%
\pgfpathlineto{\pgfqpoint{4.005635in}{2.392053in}}%
\pgfpathlineto{\pgfqpoint{3.997849in}{2.384377in}}%
\pgfpathlineto{\pgfqpoint{3.990058in}{2.376695in}}%
\pgfpathlineto{\pgfqpoint{3.982260in}{2.369009in}}%
\pgfpathlineto{\pgfqpoint{3.974457in}{2.361319in}}%
\pgfpathlineto{\pgfqpoint{3.961028in}{2.364835in}}%
\pgfpathlineto{\pgfqpoint{3.947605in}{2.368381in}}%
\pgfpathlineto{\pgfqpoint{3.934188in}{2.371958in}}%
\pgfpathlineto{\pgfqpoint{3.920777in}{2.375564in}}%
\pgfpathlineto{\pgfqpoint{3.928592in}{2.383194in}}%
\pgfpathlineto{\pgfqpoint{3.936402in}{2.390823in}}%
\pgfpathlineto{\pgfqpoint{3.944207in}{2.398452in}}%
\pgfpathlineto{\pgfqpoint{3.952005in}{2.406077in}}%
\pgfpathclose%
\pgfusepath{fill}%
\end{pgfscope}%
\begin{pgfscope}%
\pgfpathrectangle{\pgfqpoint{1.150000in}{0.150000in}}{\pgfqpoint{5.700000in}{5.700000in}}%
\pgfusepath{clip}%
\pgfsetbuttcap%
\pgfsetroundjoin%
\definecolor{currentfill}{rgb}{0.276022,0.044167,0.370164}%
\pgfsetfillcolor{currentfill}%
\pgfsetfillopacity{0.700000}%
\pgfsetlinewidth{0.000000pt}%
\definecolor{currentstroke}{rgb}{0.000000,0.000000,0.000000}%
\pgfsetstrokecolor{currentstroke}%
\pgfsetdash{}{0pt}%
\pgfpathmoveto{\pgfqpoint{3.036931in}{2.462100in}}%
\pgfpathlineto{\pgfqpoint{3.050172in}{2.456030in}}%
\pgfpathlineto{\pgfqpoint{3.063416in}{2.450002in}}%
\pgfpathlineto{\pgfqpoint{3.076665in}{2.444017in}}%
\pgfpathlineto{\pgfqpoint{3.089917in}{2.438074in}}%
\pgfpathlineto{\pgfqpoint{3.081755in}{2.432744in}}%
\pgfpathlineto{\pgfqpoint{3.073583in}{2.427511in}}%
\pgfpathlineto{\pgfqpoint{3.065403in}{2.422380in}}%
\pgfpathlineto{\pgfqpoint{3.057214in}{2.417353in}}%
\pgfpathlineto{\pgfqpoint{3.043944in}{2.423444in}}%
\pgfpathlineto{\pgfqpoint{3.030677in}{2.429577in}}%
\pgfpathlineto{\pgfqpoint{3.017413in}{2.435753in}}%
\pgfpathlineto{\pgfqpoint{3.004154in}{2.441971in}}%
\pgfpathlineto{\pgfqpoint{3.012362in}{2.446845in}}%
\pgfpathlineto{\pgfqpoint{3.020561in}{2.451826in}}%
\pgfpathlineto{\pgfqpoint{3.028750in}{2.456913in}}%
\pgfpathlineto{\pgfqpoint{3.036931in}{2.462100in}}%
\pgfpathclose%
\pgfusepath{fill}%
\end{pgfscope}%
\begin{pgfscope}%
\pgfpathrectangle{\pgfqpoint{1.150000in}{0.150000in}}{\pgfqpoint{5.700000in}{5.700000in}}%
\pgfusepath{clip}%
\pgfsetbuttcap%
\pgfsetroundjoin%
\definecolor{currentfill}{rgb}{0.267004,0.004874,0.329415}%
\pgfsetfillcolor{currentfill}%
\pgfsetfillopacity{0.700000}%
\pgfsetlinewidth{0.000000pt}%
\definecolor{currentstroke}{rgb}{0.000000,0.000000,0.000000}%
\pgfsetstrokecolor{currentstroke}%
\pgfsetdash{}{0pt}%
\pgfpathmoveto{\pgfqpoint{3.590505in}{2.396399in}}%
\pgfpathlineto{\pgfqpoint{3.603830in}{2.392003in}}%
\pgfpathlineto{\pgfqpoint{3.617160in}{2.387640in}}%
\pgfpathlineto{\pgfqpoint{3.630495in}{2.383311in}}%
\pgfpathlineto{\pgfqpoint{3.643836in}{2.379016in}}%
\pgfpathlineto{\pgfqpoint{3.635914in}{2.371803in}}%
\pgfpathlineto{\pgfqpoint{3.627986in}{2.364618in}}%
\pgfpathlineto{\pgfqpoint{3.620052in}{2.357463in}}%
\pgfpathlineto{\pgfqpoint{3.612111in}{2.350340in}}%
\pgfpathlineto{\pgfqpoint{3.598757in}{2.354730in}}%
\pgfpathlineto{\pgfqpoint{3.585407in}{2.359154in}}%
\pgfpathlineto{\pgfqpoint{3.572063in}{2.363611in}}%
\pgfpathlineto{\pgfqpoint{3.558725in}{2.368102in}}%
\pgfpathlineto{\pgfqpoint{3.566679in}{2.375126in}}%
\pgfpathlineto{\pgfqpoint{3.574628in}{2.382184in}}%
\pgfpathlineto{\pgfqpoint{3.582570in}{2.389276in}}%
\pgfpathlineto{\pgfqpoint{3.590505in}{2.396399in}}%
\pgfpathclose%
\pgfusepath{fill}%
\end{pgfscope}%
\begin{pgfscope}%
\pgfpathrectangle{\pgfqpoint{1.150000in}{0.150000in}}{\pgfqpoint{5.700000in}{5.700000in}}%
\pgfusepath{clip}%
\pgfsetbuttcap%
\pgfsetroundjoin%
\definecolor{currentfill}{rgb}{0.282656,0.100196,0.422160}%
\pgfsetfillcolor{currentfill}%
\pgfsetfillopacity{0.700000}%
\pgfsetlinewidth{0.000000pt}%
\definecolor{currentstroke}{rgb}{0.000000,0.000000,0.000000}%
\pgfsetstrokecolor{currentstroke}%
\pgfsetdash{}{0pt}%
\pgfpathmoveto{\pgfqpoint{2.706352in}{2.560890in}}%
\pgfpathlineto{\pgfqpoint{2.719567in}{2.553561in}}%
\pgfpathlineto{\pgfqpoint{2.732785in}{2.546284in}}%
\pgfpathlineto{\pgfqpoint{2.746005in}{2.539059in}}%
\pgfpathlineto{\pgfqpoint{2.759229in}{2.531885in}}%
\pgfpathlineto{\pgfqpoint{2.750889in}{2.528291in}}%
\pgfpathlineto{\pgfqpoint{2.742537in}{2.524842in}}%
\pgfpathlineto{\pgfqpoint{2.734175in}{2.521542in}}%
\pgfpathlineto{\pgfqpoint{2.725801in}{2.518397in}}%
\pgfpathlineto{\pgfqpoint{2.712555in}{2.525747in}}%
\pgfpathlineto{\pgfqpoint{2.699312in}{2.533148in}}%
\pgfpathlineto{\pgfqpoint{2.686071in}{2.540601in}}%
\pgfpathlineto{\pgfqpoint{2.672834in}{2.548106in}}%
\pgfpathlineto{\pgfqpoint{2.681231in}{2.551070in}}%
\pgfpathlineto{\pgfqpoint{2.689616in}{2.554192in}}%
\pgfpathlineto{\pgfqpoint{2.697990in}{2.557466in}}%
\pgfpathlineto{\pgfqpoint{2.706352in}{2.560890in}}%
\pgfpathclose%
\pgfusepath{fill}%
\end{pgfscope}%
\begin{pgfscope}%
\pgfpathrectangle{\pgfqpoint{1.150000in}{0.150000in}}{\pgfqpoint{5.700000in}{5.700000in}}%
\pgfusepath{clip}%
\pgfsetbuttcap%
\pgfsetroundjoin%
\definecolor{currentfill}{rgb}{0.282884,0.135920,0.453427}%
\pgfsetfillcolor{currentfill}%
\pgfsetfillopacity{0.700000}%
\pgfsetlinewidth{0.000000pt}%
\definecolor{currentstroke}{rgb}{0.000000,0.000000,0.000000}%
\pgfsetstrokecolor{currentstroke}%
\pgfsetdash{}{0pt}%
\pgfpathmoveto{\pgfqpoint{5.875099in}{2.620010in}}%
\pgfpathlineto{\pgfqpoint{5.888988in}{2.618077in}}%
\pgfpathlineto{\pgfqpoint{5.902884in}{2.616169in}}%
\pgfpathlineto{\pgfqpoint{5.916789in}{2.614283in}}%
\pgfpathlineto{\pgfqpoint{5.930701in}{2.612422in}}%
\pgfpathlineto{\pgfqpoint{5.923687in}{2.606848in}}%
\pgfpathlineto{\pgfqpoint{5.916669in}{2.601335in}}%
\pgfpathlineto{\pgfqpoint{5.909647in}{2.595877in}}%
\pgfpathlineto{\pgfqpoint{5.902622in}{2.590469in}}%
\pgfpathlineto{\pgfqpoint{5.888688in}{2.592159in}}%
\pgfpathlineto{\pgfqpoint{5.874762in}{2.593873in}}%
\pgfpathlineto{\pgfqpoint{5.860844in}{2.595611in}}%
\pgfpathlineto{\pgfqpoint{5.846935in}{2.597373in}}%
\pgfpathlineto{\pgfqpoint{5.853981in}{2.602947in}}%
\pgfpathlineto{\pgfqpoint{5.861024in}{2.608574in}}%
\pgfpathlineto{\pgfqpoint{5.868064in}{2.614260in}}%
\pgfpathlineto{\pgfqpoint{5.875099in}{2.620010in}}%
\pgfpathclose%
\pgfusepath{fill}%
\end{pgfscope}%
\begin{pgfscope}%
\pgfpathrectangle{\pgfqpoint{1.150000in}{0.150000in}}{\pgfqpoint{5.700000in}{5.700000in}}%
\pgfusepath{clip}%
\pgfsetbuttcap%
\pgfsetroundjoin%
\definecolor{currentfill}{rgb}{0.283229,0.120777,0.440584}%
\pgfsetfillcolor{currentfill}%
\pgfsetfillopacity{0.700000}%
\pgfsetlinewidth{0.000000pt}%
\definecolor{currentstroke}{rgb}{0.000000,0.000000,0.000000}%
\pgfsetstrokecolor{currentstroke}%
\pgfsetdash{}{0pt}%
\pgfpathmoveto{\pgfqpoint{5.652172in}{2.596616in}}%
\pgfpathlineto{\pgfqpoint{5.666006in}{2.594725in}}%
\pgfpathlineto{\pgfqpoint{5.679848in}{2.592858in}}%
\pgfpathlineto{\pgfqpoint{5.693698in}{2.591015in}}%
\pgfpathlineto{\pgfqpoint{5.707556in}{2.589196in}}%
\pgfpathlineto{\pgfqpoint{5.700448in}{2.583514in}}%
\pgfpathlineto{\pgfqpoint{5.693335in}{2.577862in}}%
\pgfpathlineto{\pgfqpoint{5.686217in}{2.572236in}}%
\pgfpathlineto{\pgfqpoint{5.679094in}{2.566631in}}%
\pgfpathlineto{\pgfqpoint{5.665217in}{2.568305in}}%
\pgfpathlineto{\pgfqpoint{5.651348in}{2.570003in}}%
\pgfpathlineto{\pgfqpoint{5.637486in}{2.571725in}}%
\pgfpathlineto{\pgfqpoint{5.623633in}{2.573471in}}%
\pgfpathlineto{\pgfqpoint{5.630775in}{2.579216in}}%
\pgfpathlineto{\pgfqpoint{5.637912in}{2.584985in}}%
\pgfpathlineto{\pgfqpoint{5.645044in}{2.590784in}}%
\pgfpathlineto{\pgfqpoint{5.652172in}{2.596616in}}%
\pgfpathclose%
\pgfusepath{fill}%
\end{pgfscope}%
\begin{pgfscope}%
\pgfpathrectangle{\pgfqpoint{1.150000in}{0.150000in}}{\pgfqpoint{5.700000in}{5.700000in}}%
\pgfusepath{clip}%
\pgfsetbuttcap%
\pgfsetroundjoin%
\definecolor{currentfill}{rgb}{0.267004,0.004874,0.329415}%
\pgfsetfillcolor{currentfill}%
\pgfsetfillopacity{0.700000}%
\pgfsetlinewidth{0.000000pt}%
\definecolor{currentstroke}{rgb}{0.000000,0.000000,0.000000}%
\pgfsetstrokecolor{currentstroke}%
\pgfsetdash{}{0pt}%
\pgfpathmoveto{\pgfqpoint{3.728825in}{2.391583in}}%
\pgfpathlineto{\pgfqpoint{3.742179in}{2.387532in}}%
\pgfpathlineto{\pgfqpoint{3.755539in}{2.383514in}}%
\pgfpathlineto{\pgfqpoint{3.768905in}{2.379528in}}%
\pgfpathlineto{\pgfqpoint{3.782277in}{2.375574in}}%
\pgfpathlineto{\pgfqpoint{3.774406in}{2.368114in}}%
\pgfpathlineto{\pgfqpoint{3.766529in}{2.360668in}}%
\pgfpathlineto{\pgfqpoint{3.758647in}{2.353239in}}%
\pgfpathlineto{\pgfqpoint{3.750758in}{2.345827in}}%
\pgfpathlineto{\pgfqpoint{3.737374in}{2.349863in}}%
\pgfpathlineto{\pgfqpoint{3.723994in}{2.353931in}}%
\pgfpathlineto{\pgfqpoint{3.710621in}{2.358030in}}%
\pgfpathlineto{\pgfqpoint{3.697253in}{2.362162in}}%
\pgfpathlineto{\pgfqpoint{3.705155in}{2.369488in}}%
\pgfpathlineto{\pgfqpoint{3.713051in}{2.376834in}}%
\pgfpathlineto{\pgfqpoint{3.720941in}{2.384200in}}%
\pgfpathlineto{\pgfqpoint{3.728825in}{2.391583in}}%
\pgfpathclose%
\pgfusepath{fill}%
\end{pgfscope}%
\begin{pgfscope}%
\pgfpathrectangle{\pgfqpoint{1.150000in}{0.150000in}}{\pgfqpoint{5.700000in}{5.700000in}}%
\pgfusepath{clip}%
\pgfsetbuttcap%
\pgfsetroundjoin%
\definecolor{currentfill}{rgb}{0.276022,0.044167,0.370164}%
\pgfsetfillcolor{currentfill}%
\pgfsetfillopacity{0.700000}%
\pgfsetlinewidth{0.000000pt}%
\definecolor{currentstroke}{rgb}{0.000000,0.000000,0.000000}%
\pgfsetstrokecolor{currentstroke}%
\pgfsetdash{}{0pt}%
\pgfpathmoveto{\pgfqpoint{4.536689in}{2.460759in}}%
\pgfpathlineto{\pgfqpoint{4.550232in}{2.458197in}}%
\pgfpathlineto{\pgfqpoint{4.563782in}{2.455661in}}%
\pgfpathlineto{\pgfqpoint{4.577340in}{2.453152in}}%
\pgfpathlineto{\pgfqpoint{4.590904in}{2.450670in}}%
\pgfpathlineto{\pgfqpoint{4.583331in}{2.443286in}}%
\pgfpathlineto{\pgfqpoint{4.575753in}{2.435871in}}%
\pgfpathlineto{\pgfqpoint{4.568168in}{2.428424in}}%
\pgfpathlineto{\pgfqpoint{4.560578in}{2.420944in}}%
\pgfpathlineto{\pgfqpoint{4.547001in}{2.423415in}}%
\pgfpathlineto{\pgfqpoint{4.533431in}{2.425912in}}%
\pgfpathlineto{\pgfqpoint{4.519868in}{2.428437in}}%
\pgfpathlineto{\pgfqpoint{4.506312in}{2.430988in}}%
\pgfpathlineto{\pgfqpoint{4.513915in}{2.438474in}}%
\pgfpathlineto{\pgfqpoint{4.521512in}{2.445931in}}%
\pgfpathlineto{\pgfqpoint{4.529103in}{2.453359in}}%
\pgfpathlineto{\pgfqpoint{4.536689in}{2.460759in}}%
\pgfpathclose%
\pgfusepath{fill}%
\end{pgfscope}%
\begin{pgfscope}%
\pgfpathrectangle{\pgfqpoint{1.150000in}{0.150000in}}{\pgfqpoint{5.700000in}{5.700000in}}%
\pgfusepath{clip}%
\pgfsetbuttcap%
\pgfsetroundjoin%
\definecolor{currentfill}{rgb}{0.283091,0.110553,0.431554}%
\pgfsetfillcolor{currentfill}%
\pgfsetfillopacity{0.700000}%
\pgfsetlinewidth{0.000000pt}%
\definecolor{currentstroke}{rgb}{0.000000,0.000000,0.000000}%
\pgfsetstrokecolor{currentstroke}%
\pgfsetdash{}{0pt}%
\pgfpathmoveto{\pgfqpoint{5.429166in}{2.571921in}}%
\pgfpathlineto{\pgfqpoint{5.442943in}{2.570014in}}%
\pgfpathlineto{\pgfqpoint{5.456729in}{2.568132in}}%
\pgfpathlineto{\pgfqpoint{5.470522in}{2.566275in}}%
\pgfpathlineto{\pgfqpoint{5.484323in}{2.564442in}}%
\pgfpathlineto{\pgfqpoint{5.477118in}{2.558510in}}%
\pgfpathlineto{\pgfqpoint{5.469908in}{2.552584in}}%
\pgfpathlineto{\pgfqpoint{5.462692in}{2.546660in}}%
\pgfpathlineto{\pgfqpoint{5.455471in}{2.540735in}}%
\pgfpathlineto{\pgfqpoint{5.441652in}{2.542450in}}%
\pgfpathlineto{\pgfqpoint{5.427842in}{2.544189in}}%
\pgfpathlineto{\pgfqpoint{5.414039in}{2.545953in}}%
\pgfpathlineto{\pgfqpoint{5.400244in}{2.547741in}}%
\pgfpathlineto{\pgfqpoint{5.407482in}{2.553779in}}%
\pgfpathlineto{\pgfqpoint{5.414715in}{2.559820in}}%
\pgfpathlineto{\pgfqpoint{5.421943in}{2.565865in}}%
\pgfpathlineto{\pgfqpoint{5.429166in}{2.571921in}}%
\pgfpathclose%
\pgfusepath{fill}%
\end{pgfscope}%
\begin{pgfscope}%
\pgfpathrectangle{\pgfqpoint{1.150000in}{0.150000in}}{\pgfqpoint{5.700000in}{5.700000in}}%
\pgfusepath{clip}%
\pgfsetbuttcap%
\pgfsetroundjoin%
\definecolor{currentfill}{rgb}{0.273809,0.031497,0.358853}%
\pgfsetfillcolor{currentfill}%
\pgfsetfillopacity{0.700000}%
\pgfsetlinewidth{0.000000pt}%
\definecolor{currentstroke}{rgb}{0.000000,0.000000,0.000000}%
\pgfsetstrokecolor{currentstroke}%
\pgfsetdash{}{0pt}%
\pgfpathmoveto{\pgfqpoint{4.313551in}{2.433504in}}%
\pgfpathlineto{\pgfqpoint{4.327039in}{2.430621in}}%
\pgfpathlineto{\pgfqpoint{4.340533in}{2.427766in}}%
\pgfpathlineto{\pgfqpoint{4.354033in}{2.424938in}}%
\pgfpathlineto{\pgfqpoint{4.367541in}{2.422139in}}%
\pgfpathlineto{\pgfqpoint{4.359884in}{2.414515in}}%
\pgfpathlineto{\pgfqpoint{4.352222in}{2.406866in}}%
\pgfpathlineto{\pgfqpoint{4.344554in}{2.399191in}}%
\pgfpathlineto{\pgfqpoint{4.336880in}{2.391489in}}%
\pgfpathlineto{\pgfqpoint{4.323361in}{2.394304in}}%
\pgfpathlineto{\pgfqpoint{4.309848in}{2.397147in}}%
\pgfpathlineto{\pgfqpoint{4.296341in}{2.400017in}}%
\pgfpathlineto{\pgfqpoint{4.282841in}{2.402916in}}%
\pgfpathlineto{\pgfqpoint{4.290527in}{2.410597in}}%
\pgfpathlineto{\pgfqpoint{4.298208in}{2.418256in}}%
\pgfpathlineto{\pgfqpoint{4.305882in}{2.425891in}}%
\pgfpathlineto{\pgfqpoint{4.313551in}{2.433504in}}%
\pgfpathclose%
\pgfusepath{fill}%
\end{pgfscope}%
\begin{pgfscope}%
\pgfpathrectangle{\pgfqpoint{1.150000in}{0.150000in}}{\pgfqpoint{5.700000in}{5.700000in}}%
\pgfusepath{clip}%
\pgfsetbuttcap%
\pgfsetroundjoin%
\definecolor{currentfill}{rgb}{0.279566,0.067836,0.391917}%
\pgfsetfillcolor{currentfill}%
\pgfsetfillopacity{0.700000}%
\pgfsetlinewidth{0.000000pt}%
\definecolor{currentstroke}{rgb}{0.000000,0.000000,0.000000}%
\pgfsetstrokecolor{currentstroke}%
\pgfsetdash{}{0pt}%
\pgfpathmoveto{\pgfqpoint{2.898210in}{2.493310in}}%
\pgfpathlineto{\pgfqpoint{2.911441in}{2.486734in}}%
\pgfpathlineto{\pgfqpoint{2.924675in}{2.480204in}}%
\pgfpathlineto{\pgfqpoint{2.937913in}{2.473720in}}%
\pgfpathlineto{\pgfqpoint{2.951154in}{2.467281in}}%
\pgfpathlineto{\pgfqpoint{2.942917in}{2.462676in}}%
\pgfpathlineto{\pgfqpoint{2.934671in}{2.458190in}}%
\pgfpathlineto{\pgfqpoint{2.926415in}{2.453827in}}%
\pgfpathlineto{\pgfqpoint{2.918149in}{2.449590in}}%
\pgfpathlineto{\pgfqpoint{2.904887in}{2.456191in}}%
\pgfpathlineto{\pgfqpoint{2.891630in}{2.462837in}}%
\pgfpathlineto{\pgfqpoint{2.878375in}{2.469528in}}%
\pgfpathlineto{\pgfqpoint{2.865124in}{2.476266in}}%
\pgfpathlineto{\pgfqpoint{2.873411in}{2.480336in}}%
\pgfpathlineto{\pgfqpoint{2.881687in}{2.484535in}}%
\pgfpathlineto{\pgfqpoint{2.889954in}{2.488861in}}%
\pgfpathlineto{\pgfqpoint{2.898210in}{2.493310in}}%
\pgfpathclose%
\pgfusepath{fill}%
\end{pgfscope}%
\begin{pgfscope}%
\pgfpathrectangle{\pgfqpoint{1.150000in}{0.150000in}}{\pgfqpoint{5.700000in}{5.700000in}}%
\pgfusepath{clip}%
\pgfsetbuttcap%
\pgfsetroundjoin%
\definecolor{currentfill}{rgb}{0.278791,0.062145,0.386592}%
\pgfsetfillcolor{currentfill}%
\pgfsetfillopacity{0.700000}%
\pgfsetlinewidth{0.000000pt}%
\definecolor{currentstroke}{rgb}{0.000000,0.000000,0.000000}%
\pgfsetstrokecolor{currentstroke}%
\pgfsetdash{}{0pt}%
\pgfpathmoveto{\pgfqpoint{4.759831in}{2.489351in}}%
\pgfpathlineto{\pgfqpoint{4.773432in}{2.487045in}}%
\pgfpathlineto{\pgfqpoint{4.787041in}{2.484764in}}%
\pgfpathlineto{\pgfqpoint{4.800657in}{2.482510in}}%
\pgfpathlineto{\pgfqpoint{4.814279in}{2.480282in}}%
\pgfpathlineto{\pgfqpoint{4.806794in}{2.473234in}}%
\pgfpathlineto{\pgfqpoint{4.799302in}{2.466155in}}%
\pgfpathlineto{\pgfqpoint{4.791805in}{2.459044in}}%
\pgfpathlineto{\pgfqpoint{4.784301in}{2.451899in}}%
\pgfpathlineto{\pgfqpoint{4.770664in}{2.454089in}}%
\pgfpathlineto{\pgfqpoint{4.757035in}{2.456305in}}%
\pgfpathlineto{\pgfqpoint{4.743413in}{2.458548in}}%
\pgfpathlineto{\pgfqpoint{4.729798in}{2.460816in}}%
\pgfpathlineto{\pgfqpoint{4.737315in}{2.467994in}}%
\pgfpathlineto{\pgfqpoint{4.744827in}{2.475142in}}%
\pgfpathlineto{\pgfqpoint{4.752332in}{2.482260in}}%
\pgfpathlineto{\pgfqpoint{4.759831in}{2.489351in}}%
\pgfpathclose%
\pgfusepath{fill}%
\end{pgfscope}%
\begin{pgfscope}%
\pgfpathrectangle{\pgfqpoint{1.150000in}{0.150000in}}{\pgfqpoint{5.700000in}{5.700000in}}%
\pgfusepath{clip}%
\pgfsetbuttcap%
\pgfsetroundjoin%
\definecolor{currentfill}{rgb}{0.282327,0.094955,0.417331}%
\pgfsetfillcolor{currentfill}%
\pgfsetfillopacity{0.700000}%
\pgfsetlinewidth{0.000000pt}%
\definecolor{currentstroke}{rgb}{0.000000,0.000000,0.000000}%
\pgfsetstrokecolor{currentstroke}%
\pgfsetdash{}{0pt}%
\pgfpathmoveto{\pgfqpoint{5.206092in}{2.545658in}}%
\pgfpathlineto{\pgfqpoint{5.219811in}{2.543678in}}%
\pgfpathlineto{\pgfqpoint{5.233538in}{2.541723in}}%
\pgfpathlineto{\pgfqpoint{5.247273in}{2.539793in}}%
\pgfpathlineto{\pgfqpoint{5.261015in}{2.537888in}}%
\pgfpathlineto{\pgfqpoint{5.253714in}{2.531613in}}%
\pgfpathlineto{\pgfqpoint{5.246407in}{2.525326in}}%
\pgfpathlineto{\pgfqpoint{5.239095in}{2.519024in}}%
\pgfpathlineto{\pgfqpoint{5.231776in}{2.512703in}}%
\pgfpathlineto{\pgfqpoint{5.218017in}{2.514517in}}%
\pgfpathlineto{\pgfqpoint{5.204267in}{2.516356in}}%
\pgfpathlineto{\pgfqpoint{5.190524in}{2.518219in}}%
\pgfpathlineto{\pgfqpoint{5.176788in}{2.520108in}}%
\pgfpathlineto{\pgfqpoint{5.184123in}{2.526515in}}%
\pgfpathlineto{\pgfqpoint{5.191452in}{2.532907in}}%
\pgfpathlineto{\pgfqpoint{5.198775in}{2.539287in}}%
\pgfpathlineto{\pgfqpoint{5.206092in}{2.545658in}}%
\pgfpathclose%
\pgfusepath{fill}%
\end{pgfscope}%
\begin{pgfscope}%
\pgfpathrectangle{\pgfqpoint{1.150000in}{0.150000in}}{\pgfqpoint{5.700000in}{5.700000in}}%
\pgfusepath{clip}%
\pgfsetbuttcap%
\pgfsetroundjoin%
\definecolor{currentfill}{rgb}{0.280894,0.078907,0.402329}%
\pgfsetfillcolor{currentfill}%
\pgfsetfillopacity{0.700000}%
\pgfsetlinewidth{0.000000pt}%
\definecolor{currentstroke}{rgb}{0.000000,0.000000,0.000000}%
\pgfsetstrokecolor{currentstroke}%
\pgfsetdash{}{0pt}%
\pgfpathmoveto{\pgfqpoint{4.982972in}{2.517959in}}%
\pgfpathlineto{\pgfqpoint{4.996632in}{2.515846in}}%
\pgfpathlineto{\pgfqpoint{5.010300in}{2.513759in}}%
\pgfpathlineto{\pgfqpoint{5.023975in}{2.511697in}}%
\pgfpathlineto{\pgfqpoint{5.037658in}{2.509660in}}%
\pgfpathlineto{\pgfqpoint{5.030263in}{2.502997in}}%
\pgfpathlineto{\pgfqpoint{5.022862in}{2.496310in}}%
\pgfpathlineto{\pgfqpoint{5.015455in}{2.489597in}}%
\pgfpathlineto{\pgfqpoint{5.008042in}{2.482854in}}%
\pgfpathlineto{\pgfqpoint{4.994344in}{2.484826in}}%
\pgfpathlineto{\pgfqpoint{4.980654in}{2.486823in}}%
\pgfpathlineto{\pgfqpoint{4.966972in}{2.488845in}}%
\pgfpathlineto{\pgfqpoint{4.953297in}{2.490894in}}%
\pgfpathlineto{\pgfqpoint{4.960725in}{2.497696in}}%
\pgfpathlineto{\pgfqpoint{4.968146in}{2.504473in}}%
\pgfpathlineto{\pgfqpoint{4.975562in}{2.511227in}}%
\pgfpathlineto{\pgfqpoint{4.982972in}{2.517959in}}%
\pgfpathclose%
\pgfusepath{fill}%
\end{pgfscope}%
\begin{pgfscope}%
\pgfpathrectangle{\pgfqpoint{1.150000in}{0.150000in}}{\pgfqpoint{5.700000in}{5.700000in}}%
\pgfusepath{clip}%
\pgfsetbuttcap%
\pgfsetroundjoin%
\definecolor{currentfill}{rgb}{0.269944,0.014625,0.341379}%
\pgfsetfillcolor{currentfill}%
\pgfsetfillopacity{0.700000}%
\pgfsetlinewidth{0.000000pt}%
\definecolor{currentstroke}{rgb}{0.000000,0.000000,0.000000}%
\pgfsetstrokecolor{currentstroke}%
\pgfsetdash{}{0pt}%
\pgfpathmoveto{\pgfqpoint{4.090401in}{2.409307in}}%
\pgfpathlineto{\pgfqpoint{4.103836in}{2.406037in}}%
\pgfpathlineto{\pgfqpoint{4.117278in}{2.402795in}}%
\pgfpathlineto{\pgfqpoint{4.130725in}{2.399582in}}%
\pgfpathlineto{\pgfqpoint{4.144179in}{2.396399in}}%
\pgfpathlineto{\pgfqpoint{4.136440in}{2.388678in}}%
\pgfpathlineto{\pgfqpoint{4.128696in}{2.380943in}}%
\pgfpathlineto{\pgfqpoint{4.120946in}{2.373194in}}%
\pgfpathlineto{\pgfqpoint{4.113190in}{2.365430in}}%
\pgfpathlineto{\pgfqpoint{4.099724in}{2.368656in}}%
\pgfpathlineto{\pgfqpoint{4.086264in}{2.371910in}}%
\pgfpathlineto{\pgfqpoint{4.072810in}{2.375194in}}%
\pgfpathlineto{\pgfqpoint{4.059363in}{2.378506in}}%
\pgfpathlineto{\pgfqpoint{4.067131in}{2.386223in}}%
\pgfpathlineto{\pgfqpoint{4.074893in}{2.393929in}}%
\pgfpathlineto{\pgfqpoint{4.082650in}{2.401624in}}%
\pgfpathlineto{\pgfqpoint{4.090401in}{2.409307in}}%
\pgfpathclose%
\pgfusepath{fill}%
\end{pgfscope}%
\begin{pgfscope}%
\pgfpathrectangle{\pgfqpoint{1.150000in}{0.150000in}}{\pgfqpoint{5.700000in}{5.700000in}}%
\pgfusepath{clip}%
\pgfsetbuttcap%
\pgfsetroundjoin%
\definecolor{currentfill}{rgb}{0.268510,0.009605,0.335427}%
\pgfsetfillcolor{currentfill}%
\pgfsetfillopacity{0.700000}%
\pgfsetlinewidth{0.000000pt}%
\definecolor{currentstroke}{rgb}{0.000000,0.000000,0.000000}%
\pgfsetstrokecolor{currentstroke}%
\pgfsetdash{}{0pt}%
\pgfpathmoveto{\pgfqpoint{3.867192in}{2.390297in}}%
\pgfpathlineto{\pgfqpoint{3.880579in}{2.386567in}}%
\pgfpathlineto{\pgfqpoint{3.893972in}{2.382869in}}%
\pgfpathlineto{\pgfqpoint{3.907372in}{2.379201in}}%
\pgfpathlineto{\pgfqpoint{3.920777in}{2.375564in}}%
\pgfpathlineto{\pgfqpoint{3.912955in}{2.367935in}}%
\pgfpathlineto{\pgfqpoint{3.905128in}{2.360309in}}%
\pgfpathlineto{\pgfqpoint{3.897295in}{2.352685in}}%
\pgfpathlineto{\pgfqpoint{3.889456in}{2.345066in}}%
\pgfpathlineto{\pgfqpoint{3.876038in}{2.348772in}}%
\pgfpathlineto{\pgfqpoint{3.862626in}{2.352507in}}%
\pgfpathlineto{\pgfqpoint{3.849220in}{2.356274in}}%
\pgfpathlineto{\pgfqpoint{3.835820in}{2.360071in}}%
\pgfpathlineto{\pgfqpoint{3.843672in}{2.367617in}}%
\pgfpathlineto{\pgfqpoint{3.851517in}{2.375171in}}%
\pgfpathlineto{\pgfqpoint{3.859358in}{2.382731in}}%
\pgfpathlineto{\pgfqpoint{3.867192in}{2.390297in}}%
\pgfpathclose%
\pgfusepath{fill}%
\end{pgfscope}%
\begin{pgfscope}%
\pgfpathrectangle{\pgfqpoint{1.150000in}{0.150000in}}{\pgfqpoint{5.700000in}{5.700000in}}%
\pgfusepath{clip}%
\pgfsetbuttcap%
\pgfsetroundjoin%
\definecolor{currentfill}{rgb}{0.268510,0.009605,0.335427}%
\pgfsetfillcolor{currentfill}%
\pgfsetfillopacity{0.700000}%
\pgfsetlinewidth{0.000000pt}%
\definecolor{currentstroke}{rgb}{0.000000,0.000000,0.000000}%
\pgfsetstrokecolor{currentstroke}%
\pgfsetdash{}{0pt}%
\pgfpathmoveto{\pgfqpoint{3.366995in}{2.398308in}}%
\pgfpathlineto{\pgfqpoint{3.380288in}{2.393287in}}%
\pgfpathlineto{\pgfqpoint{3.393587in}{2.388302in}}%
\pgfpathlineto{\pgfqpoint{3.406890in}{2.383353in}}%
\pgfpathlineto{\pgfqpoint{3.420198in}{2.378441in}}%
\pgfpathlineto{\pgfqpoint{3.412180in}{2.371862in}}%
\pgfpathlineto{\pgfqpoint{3.404155in}{2.365339in}}%
\pgfpathlineto{\pgfqpoint{3.396124in}{2.358876in}}%
\pgfpathlineto{\pgfqpoint{3.388085in}{2.352475in}}%
\pgfpathlineto{\pgfqpoint{3.374761in}{2.357509in}}%
\pgfpathlineto{\pgfqpoint{3.361443in}{2.362578in}}%
\pgfpathlineto{\pgfqpoint{3.348129in}{2.367684in}}%
\pgfpathlineto{\pgfqpoint{3.334819in}{2.372827in}}%
\pgfpathlineto{\pgfqpoint{3.342874in}{2.379102in}}%
\pgfpathlineto{\pgfqpoint{3.350921in}{2.385442in}}%
\pgfpathlineto{\pgfqpoint{3.358962in}{2.391845in}}%
\pgfpathlineto{\pgfqpoint{3.366995in}{2.398308in}}%
\pgfpathclose%
\pgfusepath{fill}%
\end{pgfscope}%
\begin{pgfscope}%
\pgfpathrectangle{\pgfqpoint{1.150000in}{0.150000in}}{\pgfqpoint{5.700000in}{5.700000in}}%
\pgfusepath{clip}%
\pgfsetbuttcap%
\pgfsetroundjoin%
\definecolor{currentfill}{rgb}{0.271305,0.019942,0.347269}%
\pgfsetfillcolor{currentfill}%
\pgfsetfillopacity{0.700000}%
\pgfsetlinewidth{0.000000pt}%
\definecolor{currentstroke}{rgb}{0.000000,0.000000,0.000000}%
\pgfsetstrokecolor{currentstroke}%
\pgfsetdash{}{0pt}%
\pgfpathmoveto{\pgfqpoint{3.228511in}{2.415315in}}%
\pgfpathlineto{\pgfqpoint{3.241784in}{2.409870in}}%
\pgfpathlineto{\pgfqpoint{3.255061in}{2.404464in}}%
\pgfpathlineto{\pgfqpoint{3.268343in}{2.399097in}}%
\pgfpathlineto{\pgfqpoint{3.281629in}{2.393768in}}%
\pgfpathlineto{\pgfqpoint{3.273550in}{2.387691in}}%
\pgfpathlineto{\pgfqpoint{3.265464in}{2.381688in}}%
\pgfpathlineto{\pgfqpoint{3.257370in}{2.375764in}}%
\pgfpathlineto{\pgfqpoint{3.249269in}{2.369920in}}%
\pgfpathlineto{\pgfqpoint{3.235966in}{2.375384in}}%
\pgfpathlineto{\pgfqpoint{3.222667in}{2.380886in}}%
\pgfpathlineto{\pgfqpoint{3.209373in}{2.386426in}}%
\pgfpathlineto{\pgfqpoint{3.196083in}{2.392006in}}%
\pgfpathlineto{\pgfqpoint{3.204202in}{2.397710in}}%
\pgfpathlineto{\pgfqpoint{3.212313in}{2.403498in}}%
\pgfpathlineto{\pgfqpoint{3.220416in}{2.409368in}}%
\pgfpathlineto{\pgfqpoint{3.228511in}{2.415315in}}%
\pgfpathclose%
\pgfusepath{fill}%
\end{pgfscope}%
\begin{pgfscope}%
\pgfpathrectangle{\pgfqpoint{1.150000in}{0.150000in}}{\pgfqpoint{5.700000in}{5.700000in}}%
\pgfusepath{clip}%
\pgfsetbuttcap%
\pgfsetroundjoin%
\definecolor{currentfill}{rgb}{0.267004,0.004874,0.329415}%
\pgfsetfillcolor{currentfill}%
\pgfsetfillopacity{0.700000}%
\pgfsetlinewidth{0.000000pt}%
\definecolor{currentstroke}{rgb}{0.000000,0.000000,0.000000}%
\pgfsetstrokecolor{currentstroke}%
\pgfsetdash{}{0pt}%
\pgfpathmoveto{\pgfqpoint{3.505422in}{2.386408in}}%
\pgfpathlineto{\pgfqpoint{3.518740in}{2.381780in}}%
\pgfpathlineto{\pgfqpoint{3.532063in}{2.377186in}}%
\pgfpathlineto{\pgfqpoint{3.545391in}{2.372627in}}%
\pgfpathlineto{\pgfqpoint{3.558725in}{2.368102in}}%
\pgfpathlineto{\pgfqpoint{3.550763in}{2.361117in}}%
\pgfpathlineto{\pgfqpoint{3.542796in}{2.354171in}}%
\pgfpathlineto{\pgfqpoint{3.534822in}{2.347268in}}%
\pgfpathlineto{\pgfqpoint{3.526841in}{2.340409in}}%
\pgfpathlineto{\pgfqpoint{3.513493in}{2.345042in}}%
\pgfpathlineto{\pgfqpoint{3.500150in}{2.349709in}}%
\pgfpathlineto{\pgfqpoint{3.486812in}{2.354410in}}%
\pgfpathlineto{\pgfqpoint{3.473479in}{2.359146in}}%
\pgfpathlineto{\pgfqpoint{3.481475in}{2.365892in}}%
\pgfpathlineto{\pgfqpoint{3.489464in}{2.372686in}}%
\pgfpathlineto{\pgfqpoint{3.497446in}{2.379525in}}%
\pgfpathlineto{\pgfqpoint{3.505422in}{2.386408in}}%
\pgfpathclose%
\pgfusepath{fill}%
\end{pgfscope}%
\begin{pgfscope}%
\pgfpathrectangle{\pgfqpoint{1.150000in}{0.150000in}}{\pgfqpoint{5.700000in}{5.700000in}}%
\pgfusepath{clip}%
\pgfsetbuttcap%
\pgfsetroundjoin%
\definecolor{currentfill}{rgb}{0.283072,0.130895,0.449241}%
\pgfsetfillcolor{currentfill}%
\pgfsetfillopacity{0.700000}%
\pgfsetlinewidth{0.000000pt}%
\definecolor{currentstroke}{rgb}{0.000000,0.000000,0.000000}%
\pgfsetstrokecolor{currentstroke}%
\pgfsetdash{}{0pt}%
\pgfpathmoveto{\pgfqpoint{5.791377in}{2.604658in}}%
\pgfpathlineto{\pgfqpoint{5.805254in}{2.602801in}}%
\pgfpathlineto{\pgfqpoint{5.819140in}{2.600967in}}%
\pgfpathlineto{\pgfqpoint{5.833033in}{2.599158in}}%
\pgfpathlineto{\pgfqpoint{5.846935in}{2.597373in}}%
\pgfpathlineto{\pgfqpoint{5.839884in}{2.591847in}}%
\pgfpathlineto{\pgfqpoint{5.832830in}{2.586364in}}%
\pgfpathlineto{\pgfqpoint{5.825771in}{2.580920in}}%
\pgfpathlineto{\pgfqpoint{5.818708in}{2.575509in}}%
\pgfpathlineto{\pgfqpoint{5.804785in}{2.577136in}}%
\pgfpathlineto{\pgfqpoint{5.790871in}{2.578787in}}%
\pgfpathlineto{\pgfqpoint{5.776965in}{2.580462in}}%
\pgfpathlineto{\pgfqpoint{5.763067in}{2.582161in}}%
\pgfpathlineto{\pgfqpoint{5.770151in}{2.587725in}}%
\pgfpathlineto{\pgfqpoint{5.777230in}{2.593326in}}%
\pgfpathlineto{\pgfqpoint{5.784306in}{2.598969in}}%
\pgfpathlineto{\pgfqpoint{5.791377in}{2.604658in}}%
\pgfpathclose%
\pgfusepath{fill}%
\end{pgfscope}%
\begin{pgfscope}%
\pgfpathrectangle{\pgfqpoint{1.150000in}{0.150000in}}{\pgfqpoint{5.700000in}{5.700000in}}%
\pgfusepath{clip}%
\pgfsetbuttcap%
\pgfsetroundjoin%
\definecolor{currentfill}{rgb}{0.281924,0.089666,0.412415}%
\pgfsetfillcolor{currentfill}%
\pgfsetfillopacity{0.700000}%
\pgfsetlinewidth{0.000000pt}%
\definecolor{currentstroke}{rgb}{0.000000,0.000000,0.000000}%
\pgfsetstrokecolor{currentstroke}%
\pgfsetdash{}{0pt}%
\pgfpathmoveto{\pgfqpoint{2.759229in}{2.531885in}}%
\pgfpathlineto{\pgfqpoint{2.772455in}{2.524761in}}%
\pgfpathlineto{\pgfqpoint{2.785684in}{2.517687in}}%
\pgfpathlineto{\pgfqpoint{2.798917in}{2.510663in}}%
\pgfpathlineto{\pgfqpoint{2.812152in}{2.503688in}}%
\pgfpathlineto{\pgfqpoint{2.803833in}{2.499923in}}%
\pgfpathlineto{\pgfqpoint{2.795504in}{2.496300in}}%
\pgfpathlineto{\pgfqpoint{2.787164in}{2.492823in}}%
\pgfpathlineto{\pgfqpoint{2.778813in}{2.489496in}}%
\pgfpathlineto{\pgfqpoint{2.765555in}{2.496648in}}%
\pgfpathlineto{\pgfqpoint{2.752301in}{2.503848in}}%
\pgfpathlineto{\pgfqpoint{2.739049in}{2.511097in}}%
\pgfpathlineto{\pgfqpoint{2.725801in}{2.518397in}}%
\pgfpathlineto{\pgfqpoint{2.734175in}{2.521542in}}%
\pgfpathlineto{\pgfqpoint{2.742537in}{2.524842in}}%
\pgfpathlineto{\pgfqpoint{2.750889in}{2.528291in}}%
\pgfpathlineto{\pgfqpoint{2.759229in}{2.531885in}}%
\pgfpathclose%
\pgfusepath{fill}%
\end{pgfscope}%
\begin{pgfscope}%
\pgfpathrectangle{\pgfqpoint{1.150000in}{0.150000in}}{\pgfqpoint{5.700000in}{5.700000in}}%
\pgfusepath{clip}%
\pgfsetbuttcap%
\pgfsetroundjoin%
\definecolor{currentfill}{rgb}{0.274952,0.037752,0.364543}%
\pgfsetfillcolor{currentfill}%
\pgfsetfillopacity{0.700000}%
\pgfsetlinewidth{0.000000pt}%
\definecolor{currentstroke}{rgb}{0.000000,0.000000,0.000000}%
\pgfsetstrokecolor{currentstroke}%
\pgfsetdash{}{0pt}%
\pgfpathmoveto{\pgfqpoint{3.089917in}{2.438074in}}%
\pgfpathlineto{\pgfqpoint{3.103173in}{2.432173in}}%
\pgfpathlineto{\pgfqpoint{3.116434in}{2.426314in}}%
\pgfpathlineto{\pgfqpoint{3.129698in}{2.420495in}}%
\pgfpathlineto{\pgfqpoint{3.142967in}{2.414717in}}%
\pgfpathlineto{\pgfqpoint{3.134822in}{2.409243in}}%
\pgfpathlineto{\pgfqpoint{3.126669in}{2.403864in}}%
\pgfpathlineto{\pgfqpoint{3.118508in}{2.398583in}}%
\pgfpathlineto{\pgfqpoint{3.110338in}{2.393403in}}%
\pgfpathlineto{\pgfqpoint{3.097051in}{2.399329in}}%
\pgfpathlineto{\pgfqpoint{3.083768in}{2.405296in}}%
\pgfpathlineto{\pgfqpoint{3.070489in}{2.411304in}}%
\pgfpathlineto{\pgfqpoint{3.057214in}{2.417353in}}%
\pgfpathlineto{\pgfqpoint{3.065403in}{2.422380in}}%
\pgfpathlineto{\pgfqpoint{3.073583in}{2.427511in}}%
\pgfpathlineto{\pgfqpoint{3.081755in}{2.432744in}}%
\pgfpathlineto{\pgfqpoint{3.089917in}{2.438074in}}%
\pgfpathclose%
\pgfusepath{fill}%
\end{pgfscope}%
\begin{pgfscope}%
\pgfpathrectangle{\pgfqpoint{1.150000in}{0.150000in}}{\pgfqpoint{5.700000in}{5.700000in}}%
\pgfusepath{clip}%
\pgfsetbuttcap%
\pgfsetroundjoin%
\definecolor{currentfill}{rgb}{0.274952,0.037752,0.364543}%
\pgfsetfillcolor{currentfill}%
\pgfsetfillopacity{0.700000}%
\pgfsetlinewidth{0.000000pt}%
\definecolor{currentstroke}{rgb}{0.000000,0.000000,0.000000}%
\pgfsetstrokecolor{currentstroke}%
\pgfsetdash{}{0pt}%
\pgfpathmoveto{\pgfqpoint{4.452157in}{2.441464in}}%
\pgfpathlineto{\pgfqpoint{4.465685in}{2.438805in}}%
\pgfpathlineto{\pgfqpoint{4.479221in}{2.436172in}}%
\pgfpathlineto{\pgfqpoint{4.492763in}{2.433566in}}%
\pgfpathlineto{\pgfqpoint{4.506312in}{2.430988in}}%
\pgfpathlineto{\pgfqpoint{4.498704in}{2.423471in}}%
\pgfpathlineto{\pgfqpoint{4.491089in}{2.415923in}}%
\pgfpathlineto{\pgfqpoint{4.483469in}{2.408344in}}%
\pgfpathlineto{\pgfqpoint{4.475843in}{2.400732in}}%
\pgfpathlineto{\pgfqpoint{4.462281in}{2.403312in}}%
\pgfpathlineto{\pgfqpoint{4.448726in}{2.405920in}}%
\pgfpathlineto{\pgfqpoint{4.435179in}{2.408555in}}%
\pgfpathlineto{\pgfqpoint{4.421637in}{2.411216in}}%
\pgfpathlineto{\pgfqpoint{4.429276in}{2.418821in}}%
\pgfpathlineto{\pgfqpoint{4.436909in}{2.426397in}}%
\pgfpathlineto{\pgfqpoint{4.444536in}{2.433945in}}%
\pgfpathlineto{\pgfqpoint{4.452157in}{2.441464in}}%
\pgfpathclose%
\pgfusepath{fill}%
\end{pgfscope}%
\begin{pgfscope}%
\pgfpathrectangle{\pgfqpoint{1.150000in}{0.150000in}}{\pgfqpoint{5.700000in}{5.700000in}}%
\pgfusepath{clip}%
\pgfsetbuttcap%
\pgfsetroundjoin%
\definecolor{currentfill}{rgb}{0.283229,0.120777,0.440584}%
\pgfsetfillcolor{currentfill}%
\pgfsetfillopacity{0.700000}%
\pgfsetlinewidth{0.000000pt}%
\definecolor{currentstroke}{rgb}{0.000000,0.000000,0.000000}%
\pgfsetstrokecolor{currentstroke}%
\pgfsetdash{}{0pt}%
\pgfpathmoveto{\pgfqpoint{5.568298in}{2.580698in}}%
\pgfpathlineto{\pgfqpoint{5.582120in}{2.578855in}}%
\pgfpathlineto{\pgfqpoint{5.595950in}{2.577036in}}%
\pgfpathlineto{\pgfqpoint{5.609787in}{2.575242in}}%
\pgfpathlineto{\pgfqpoint{5.623633in}{2.573471in}}%
\pgfpathlineto{\pgfqpoint{5.616486in}{2.567747in}}%
\pgfpathlineto{\pgfqpoint{5.609334in}{2.562039in}}%
\pgfpathlineto{\pgfqpoint{5.602177in}{2.556342in}}%
\pgfpathlineto{\pgfqpoint{5.595014in}{2.550653in}}%
\pgfpathlineto{\pgfqpoint{5.581150in}{2.552292in}}%
\pgfpathlineto{\pgfqpoint{5.567294in}{2.553955in}}%
\pgfpathlineto{\pgfqpoint{5.553445in}{2.555642in}}%
\pgfpathlineto{\pgfqpoint{5.539605in}{2.557353in}}%
\pgfpathlineto{\pgfqpoint{5.546786in}{2.563169in}}%
\pgfpathlineto{\pgfqpoint{5.553962in}{2.568996in}}%
\pgfpathlineto{\pgfqpoint{5.561133in}{2.574837in}}%
\pgfpathlineto{\pgfqpoint{5.568298in}{2.580698in}}%
\pgfpathclose%
\pgfusepath{fill}%
\end{pgfscope}%
\begin{pgfscope}%
\pgfpathrectangle{\pgfqpoint{1.150000in}{0.150000in}}{\pgfqpoint{5.700000in}{5.700000in}}%
\pgfusepath{clip}%
\pgfsetbuttcap%
\pgfsetroundjoin%
\definecolor{currentfill}{rgb}{0.277941,0.056324,0.381191}%
\pgfsetfillcolor{currentfill}%
\pgfsetfillopacity{0.700000}%
\pgfsetlinewidth{0.000000pt}%
\definecolor{currentstroke}{rgb}{0.000000,0.000000,0.000000}%
\pgfsetstrokecolor{currentstroke}%
\pgfsetdash{}{0pt}%
\pgfpathmoveto{\pgfqpoint{4.675411in}{2.470153in}}%
\pgfpathlineto{\pgfqpoint{4.688997in}{2.467780in}}%
\pgfpathlineto{\pgfqpoint{4.702590in}{2.465432in}}%
\pgfpathlineto{\pgfqpoint{4.716191in}{2.463111in}}%
\pgfpathlineto{\pgfqpoint{4.729798in}{2.460816in}}%
\pgfpathlineto{\pgfqpoint{4.722275in}{2.453606in}}%
\pgfpathlineto{\pgfqpoint{4.714746in}{2.446363in}}%
\pgfpathlineto{\pgfqpoint{4.707212in}{2.439085in}}%
\pgfpathlineto{\pgfqpoint{4.699671in}{2.431771in}}%
\pgfpathlineto{\pgfqpoint{4.686050in}{2.434041in}}%
\pgfpathlineto{\pgfqpoint{4.672436in}{2.436337in}}%
\pgfpathlineto{\pgfqpoint{4.658830in}{2.438659in}}%
\pgfpathlineto{\pgfqpoint{4.645230in}{2.441009in}}%
\pgfpathlineto{\pgfqpoint{4.652784in}{2.448342in}}%
\pgfpathlineto{\pgfqpoint{4.660332in}{2.455644in}}%
\pgfpathlineto{\pgfqpoint{4.667874in}{2.462913in}}%
\pgfpathlineto{\pgfqpoint{4.675411in}{2.470153in}}%
\pgfpathclose%
\pgfusepath{fill}%
\end{pgfscope}%
\begin{pgfscope}%
\pgfpathrectangle{\pgfqpoint{1.150000in}{0.150000in}}{\pgfqpoint{5.700000in}{5.700000in}}%
\pgfusepath{clip}%
\pgfsetbuttcap%
\pgfsetroundjoin%
\definecolor{currentfill}{rgb}{0.272594,0.025563,0.353093}%
\pgfsetfillcolor{currentfill}%
\pgfsetfillopacity{0.700000}%
\pgfsetlinewidth{0.000000pt}%
\definecolor{currentstroke}{rgb}{0.000000,0.000000,0.000000}%
\pgfsetstrokecolor{currentstroke}%
\pgfsetdash{}{0pt}%
\pgfpathmoveto{\pgfqpoint{4.228908in}{2.414792in}}%
\pgfpathlineto{\pgfqpoint{4.242381in}{2.411780in}}%
\pgfpathlineto{\pgfqpoint{4.255862in}{2.408797in}}%
\pgfpathlineto{\pgfqpoint{4.269348in}{2.405842in}}%
\pgfpathlineto{\pgfqpoint{4.282841in}{2.402916in}}%
\pgfpathlineto{\pgfqpoint{4.275150in}{2.395211in}}%
\pgfpathlineto{\pgfqpoint{4.267453in}{2.387484in}}%
\pgfpathlineto{\pgfqpoint{4.259750in}{2.379735in}}%
\pgfpathlineto{\pgfqpoint{4.252041in}{2.371963in}}%
\pgfpathlineto{\pgfqpoint{4.238536in}{2.374918in}}%
\pgfpathlineto{\pgfqpoint{4.225037in}{2.377901in}}%
\pgfpathlineto{\pgfqpoint{4.211544in}{2.380913in}}%
\pgfpathlineto{\pgfqpoint{4.198058in}{2.383953in}}%
\pgfpathlineto{\pgfqpoint{4.205779in}{2.391691in}}%
\pgfpathlineto{\pgfqpoint{4.213494in}{2.399411in}}%
\pgfpathlineto{\pgfqpoint{4.221204in}{2.407111in}}%
\pgfpathlineto{\pgfqpoint{4.228908in}{2.414792in}}%
\pgfpathclose%
\pgfusepath{fill}%
\end{pgfscope}%
\begin{pgfscope}%
\pgfpathrectangle{\pgfqpoint{1.150000in}{0.150000in}}{\pgfqpoint{5.700000in}{5.700000in}}%
\pgfusepath{clip}%
\pgfsetbuttcap%
\pgfsetroundjoin%
\definecolor{currentfill}{rgb}{0.280267,0.073417,0.397163}%
\pgfsetfillcolor{currentfill}%
\pgfsetfillopacity{0.700000}%
\pgfsetlinewidth{0.000000pt}%
\definecolor{currentstroke}{rgb}{0.000000,0.000000,0.000000}%
\pgfsetstrokecolor{currentstroke}%
\pgfsetdash{}{0pt}%
\pgfpathmoveto{\pgfqpoint{4.898671in}{2.499342in}}%
\pgfpathlineto{\pgfqpoint{4.912317in}{2.497191in}}%
\pgfpathlineto{\pgfqpoint{4.925970in}{2.495067in}}%
\pgfpathlineto{\pgfqpoint{4.939630in}{2.492967in}}%
\pgfpathlineto{\pgfqpoint{4.953297in}{2.490894in}}%
\pgfpathlineto{\pgfqpoint{4.945863in}{2.484063in}}%
\pgfpathlineto{\pgfqpoint{4.938424in}{2.477203in}}%
\pgfpathlineto{\pgfqpoint{4.930978in}{2.470311in}}%
\pgfpathlineto{\pgfqpoint{4.923526in}{2.463385in}}%
\pgfpathlineto{\pgfqpoint{4.909844in}{2.465407in}}%
\pgfpathlineto{\pgfqpoint{4.896170in}{2.467455in}}%
\pgfpathlineto{\pgfqpoint{4.882503in}{2.469528in}}%
\pgfpathlineto{\pgfqpoint{4.868844in}{2.471627in}}%
\pgfpathlineto{\pgfqpoint{4.876310in}{2.478600in}}%
\pgfpathlineto{\pgfqpoint{4.883770in}{2.485542in}}%
\pgfpathlineto{\pgfqpoint{4.891224in}{2.492455in}}%
\pgfpathlineto{\pgfqpoint{4.898671in}{2.499342in}}%
\pgfpathclose%
\pgfusepath{fill}%
\end{pgfscope}%
\begin{pgfscope}%
\pgfpathrectangle{\pgfqpoint{1.150000in}{0.150000in}}{\pgfqpoint{5.700000in}{5.700000in}}%
\pgfusepath{clip}%
\pgfsetbuttcap%
\pgfsetroundjoin%
\definecolor{currentfill}{rgb}{0.282910,0.105393,0.426902}%
\pgfsetfillcolor{currentfill}%
\pgfsetfillopacity{0.700000}%
\pgfsetlinewidth{0.000000pt}%
\definecolor{currentstroke}{rgb}{0.000000,0.000000,0.000000}%
\pgfsetstrokecolor{currentstroke}%
\pgfsetdash{}{0pt}%
\pgfpathmoveto{\pgfqpoint{5.345141in}{2.555140in}}%
\pgfpathlineto{\pgfqpoint{5.358905in}{2.553253in}}%
\pgfpathlineto{\pgfqpoint{5.372677in}{2.551391in}}%
\pgfpathlineto{\pgfqpoint{5.386456in}{2.549554in}}%
\pgfpathlineto{\pgfqpoint{5.400244in}{2.547741in}}%
\pgfpathlineto{\pgfqpoint{5.392999in}{2.541701in}}%
\pgfpathlineto{\pgfqpoint{5.385749in}{2.535656in}}%
\pgfpathlineto{\pgfqpoint{5.378493in}{2.529602in}}%
\pgfpathlineto{\pgfqpoint{5.371231in}{2.523536in}}%
\pgfpathlineto{\pgfqpoint{5.357427in}{2.525244in}}%
\pgfpathlineto{\pgfqpoint{5.343630in}{2.526976in}}%
\pgfpathlineto{\pgfqpoint{5.329842in}{2.528733in}}%
\pgfpathlineto{\pgfqpoint{5.316061in}{2.530514in}}%
\pgfpathlineto{\pgfqpoint{5.323340in}{2.536680in}}%
\pgfpathlineto{\pgfqpoint{5.330612in}{2.542838in}}%
\pgfpathlineto{\pgfqpoint{5.337879in}{2.548990in}}%
\pgfpathlineto{\pgfqpoint{5.345141in}{2.555140in}}%
\pgfpathclose%
\pgfusepath{fill}%
\end{pgfscope}%
\begin{pgfscope}%
\pgfpathrectangle{\pgfqpoint{1.150000in}{0.150000in}}{\pgfqpoint{5.700000in}{5.700000in}}%
\pgfusepath{clip}%
\pgfsetbuttcap%
\pgfsetroundjoin%
\definecolor{currentfill}{rgb}{0.282327,0.094955,0.417331}%
\pgfsetfillcolor{currentfill}%
\pgfsetfillopacity{0.700000}%
\pgfsetlinewidth{0.000000pt}%
\definecolor{currentstroke}{rgb}{0.000000,0.000000,0.000000}%
\pgfsetstrokecolor{currentstroke}%
\pgfsetdash{}{0pt}%
\pgfpathmoveto{\pgfqpoint{5.121923in}{2.527912in}}%
\pgfpathlineto{\pgfqpoint{5.135628in}{2.525923in}}%
\pgfpathlineto{\pgfqpoint{5.149341in}{2.523960in}}%
\pgfpathlineto{\pgfqpoint{5.163061in}{2.522021in}}%
\pgfpathlineto{\pgfqpoint{5.176788in}{2.520108in}}%
\pgfpathlineto{\pgfqpoint{5.169448in}{2.513683in}}%
\pgfpathlineto{\pgfqpoint{5.162101in}{2.507238in}}%
\pgfpathlineto{\pgfqpoint{5.154749in}{2.500770in}}%
\pgfpathlineto{\pgfqpoint{5.147390in}{2.494275in}}%
\pgfpathlineto{\pgfqpoint{5.133647in}{2.496110in}}%
\pgfpathlineto{\pgfqpoint{5.119911in}{2.497971in}}%
\pgfpathlineto{\pgfqpoint{5.106184in}{2.499856in}}%
\pgfpathlineto{\pgfqpoint{5.092463in}{2.501766in}}%
\pgfpathlineto{\pgfqpoint{5.099837in}{2.508334in}}%
\pgfpathlineto{\pgfqpoint{5.107205in}{2.514879in}}%
\pgfpathlineto{\pgfqpoint{5.114567in}{2.521404in}}%
\pgfpathlineto{\pgfqpoint{5.121923in}{2.527912in}}%
\pgfpathclose%
\pgfusepath{fill}%
\end{pgfscope}%
\begin{pgfscope}%
\pgfpathrectangle{\pgfqpoint{1.150000in}{0.150000in}}{\pgfqpoint{5.700000in}{5.700000in}}%
\pgfusepath{clip}%
\pgfsetbuttcap%
\pgfsetroundjoin%
\definecolor{currentfill}{rgb}{0.267004,0.004874,0.329415}%
\pgfsetfillcolor{currentfill}%
\pgfsetfillopacity{0.700000}%
\pgfsetlinewidth{0.000000pt}%
\definecolor{currentstroke}{rgb}{0.000000,0.000000,0.000000}%
\pgfsetstrokecolor{currentstroke}%
\pgfsetdash{}{0pt}%
\pgfpathmoveto{\pgfqpoint{3.643836in}{2.379016in}}%
\pgfpathlineto{\pgfqpoint{3.657182in}{2.374753in}}%
\pgfpathlineto{\pgfqpoint{3.670533in}{2.370523in}}%
\pgfpathlineto{\pgfqpoint{3.683890in}{2.366326in}}%
\pgfpathlineto{\pgfqpoint{3.697253in}{2.362162in}}%
\pgfpathlineto{\pgfqpoint{3.689345in}{2.354860in}}%
\pgfpathlineto{\pgfqpoint{3.681431in}{2.347582in}}%
\pgfpathlineto{\pgfqpoint{3.673510in}{2.340330in}}%
\pgfpathlineto{\pgfqpoint{3.665584in}{2.333108in}}%
\pgfpathlineto{\pgfqpoint{3.652207in}{2.337367in}}%
\pgfpathlineto{\pgfqpoint{3.638837in}{2.341658in}}%
\pgfpathlineto{\pgfqpoint{3.625471in}{2.345983in}}%
\pgfpathlineto{\pgfqpoint{3.612111in}{2.350340in}}%
\pgfpathlineto{\pgfqpoint{3.620052in}{2.357463in}}%
\pgfpathlineto{\pgfqpoint{3.627986in}{2.364618in}}%
\pgfpathlineto{\pgfqpoint{3.635914in}{2.371803in}}%
\pgfpathlineto{\pgfqpoint{3.643836in}{2.379016in}}%
\pgfpathclose%
\pgfusepath{fill}%
\end{pgfscope}%
\begin{pgfscope}%
\pgfpathrectangle{\pgfqpoint{1.150000in}{0.150000in}}{\pgfqpoint{5.700000in}{5.700000in}}%
\pgfusepath{clip}%
\pgfsetbuttcap%
\pgfsetroundjoin%
\definecolor{currentfill}{rgb}{0.268510,0.009605,0.335427}%
\pgfsetfillcolor{currentfill}%
\pgfsetfillopacity{0.700000}%
\pgfsetlinewidth{0.000000pt}%
\definecolor{currentstroke}{rgb}{0.000000,0.000000,0.000000}%
\pgfsetstrokecolor{currentstroke}%
\pgfsetdash{}{0pt}%
\pgfpathmoveto{\pgfqpoint{4.005635in}{2.392053in}}%
\pgfpathlineto{\pgfqpoint{4.019058in}{2.388622in}}%
\pgfpathlineto{\pgfqpoint{4.032487in}{2.385220in}}%
\pgfpathlineto{\pgfqpoint{4.045922in}{2.381848in}}%
\pgfpathlineto{\pgfqpoint{4.059363in}{2.378506in}}%
\pgfpathlineto{\pgfqpoint{4.051589in}{2.370780in}}%
\pgfpathlineto{\pgfqpoint{4.043810in}{2.363045in}}%
\pgfpathlineto{\pgfqpoint{4.036025in}{2.355303in}}%
\pgfpathlineto{\pgfqpoint{4.028234in}{2.347553in}}%
\pgfpathlineto{\pgfqpoint{4.014781in}{2.350950in}}%
\pgfpathlineto{\pgfqpoint{4.001333in}{2.354377in}}%
\pgfpathlineto{\pgfqpoint{3.987892in}{2.357833in}}%
\pgfpathlineto{\pgfqpoint{3.974457in}{2.361319in}}%
\pgfpathlineto{\pgfqpoint{3.982260in}{2.369009in}}%
\pgfpathlineto{\pgfqpoint{3.990058in}{2.376695in}}%
\pgfpathlineto{\pgfqpoint{3.997849in}{2.384377in}}%
\pgfpathlineto{\pgfqpoint{4.005635in}{2.392053in}}%
\pgfpathclose%
\pgfusepath{fill}%
\end{pgfscope}%
\begin{pgfscope}%
\pgfpathrectangle{\pgfqpoint{1.150000in}{0.150000in}}{\pgfqpoint{5.700000in}{5.700000in}}%
\pgfusepath{clip}%
\pgfsetbuttcap%
\pgfsetroundjoin%
\definecolor{currentfill}{rgb}{0.277941,0.056324,0.381191}%
\pgfsetfillcolor{currentfill}%
\pgfsetfillopacity{0.700000}%
\pgfsetlinewidth{0.000000pt}%
\definecolor{currentstroke}{rgb}{0.000000,0.000000,0.000000}%
\pgfsetstrokecolor{currentstroke}%
\pgfsetdash{}{0pt}%
\pgfpathmoveto{\pgfqpoint{2.951154in}{2.467281in}}%
\pgfpathlineto{\pgfqpoint{2.964398in}{2.460888in}}%
\pgfpathlineto{\pgfqpoint{2.977647in}{2.454538in}}%
\pgfpathlineto{\pgfqpoint{2.990898in}{2.448233in}}%
\pgfpathlineto{\pgfqpoint{3.004154in}{2.441971in}}%
\pgfpathlineto{\pgfqpoint{2.995937in}{2.437209in}}%
\pgfpathlineto{\pgfqpoint{2.987711in}{2.432563in}}%
\pgfpathlineto{\pgfqpoint{2.979475in}{2.428036in}}%
\pgfpathlineto{\pgfqpoint{2.971229in}{2.423633in}}%
\pgfpathlineto{\pgfqpoint{2.957954in}{2.430057in}}%
\pgfpathlineto{\pgfqpoint{2.944682in}{2.436524in}}%
\pgfpathlineto{\pgfqpoint{2.931413in}{2.443035in}}%
\pgfpathlineto{\pgfqpoint{2.918149in}{2.449590in}}%
\pgfpathlineto{\pgfqpoint{2.926415in}{2.453827in}}%
\pgfpathlineto{\pgfqpoint{2.934671in}{2.458190in}}%
\pgfpathlineto{\pgfqpoint{2.942917in}{2.462676in}}%
\pgfpathlineto{\pgfqpoint{2.951154in}{2.467281in}}%
\pgfpathclose%
\pgfusepath{fill}%
\end{pgfscope}%
\begin{pgfscope}%
\pgfpathrectangle{\pgfqpoint{1.150000in}{0.150000in}}{\pgfqpoint{5.700000in}{5.700000in}}%
\pgfusepath{clip}%
\pgfsetbuttcap%
\pgfsetroundjoin%
\definecolor{currentfill}{rgb}{0.267004,0.004874,0.329415}%
\pgfsetfillcolor{currentfill}%
\pgfsetfillopacity{0.700000}%
\pgfsetlinewidth{0.000000pt}%
\definecolor{currentstroke}{rgb}{0.000000,0.000000,0.000000}%
\pgfsetstrokecolor{currentstroke}%
\pgfsetdash{}{0pt}%
\pgfpathmoveto{\pgfqpoint{3.782277in}{2.375574in}}%
\pgfpathlineto{\pgfqpoint{3.795654in}{2.371651in}}%
\pgfpathlineto{\pgfqpoint{3.809037in}{2.367760in}}%
\pgfpathlineto{\pgfqpoint{3.822425in}{2.363900in}}%
\pgfpathlineto{\pgfqpoint{3.835820in}{2.360071in}}%
\pgfpathlineto{\pgfqpoint{3.827962in}{2.352535in}}%
\pgfpathlineto{\pgfqpoint{3.820099in}{2.345010in}}%
\pgfpathlineto{\pgfqpoint{3.812229in}{2.337497in}}%
\pgfpathlineto{\pgfqpoint{3.804354in}{2.329999in}}%
\pgfpathlineto{\pgfqpoint{3.790947in}{2.333909in}}%
\pgfpathlineto{\pgfqpoint{3.777545in}{2.337850in}}%
\pgfpathlineto{\pgfqpoint{3.764149in}{2.341823in}}%
\pgfpathlineto{\pgfqpoint{3.750758in}{2.345827in}}%
\pgfpathlineto{\pgfqpoint{3.758647in}{2.353239in}}%
\pgfpathlineto{\pgfqpoint{3.766529in}{2.360668in}}%
\pgfpathlineto{\pgfqpoint{3.774406in}{2.368114in}}%
\pgfpathlineto{\pgfqpoint{3.782277in}{2.375574in}}%
\pgfpathclose%
\pgfusepath{fill}%
\end{pgfscope}%
\begin{pgfscope}%
\pgfpathrectangle{\pgfqpoint{1.150000in}{0.150000in}}{\pgfqpoint{5.700000in}{5.700000in}}%
\pgfusepath{clip}%
\pgfsetbuttcap%
\pgfsetroundjoin%
\definecolor{currentfill}{rgb}{0.282623,0.140926,0.457517}%
\pgfsetfillcolor{currentfill}%
\pgfsetfillopacity{0.700000}%
\pgfsetlinewidth{0.000000pt}%
\definecolor{currentstroke}{rgb}{0.000000,0.000000,0.000000}%
\pgfsetstrokecolor{currentstroke}%
\pgfsetdash{}{0pt}%
\pgfpathmoveto{\pgfqpoint{5.930701in}{2.612422in}}%
\pgfpathlineto{\pgfqpoint{5.944622in}{2.610584in}}%
\pgfpathlineto{\pgfqpoint{5.958551in}{2.608770in}}%
\pgfpathlineto{\pgfqpoint{5.972487in}{2.606979in}}%
\pgfpathlineto{\pgfqpoint{5.986432in}{2.605212in}}%
\pgfpathlineto{\pgfqpoint{5.979440in}{2.599814in}}%
\pgfpathlineto{\pgfqpoint{5.972443in}{2.594474in}}%
\pgfpathlineto{\pgfqpoint{5.965443in}{2.589186in}}%
\pgfpathlineto{\pgfqpoint{5.958440in}{2.583946in}}%
\pgfpathlineto{\pgfqpoint{5.944473in}{2.585541in}}%
\pgfpathlineto{\pgfqpoint{5.930514in}{2.587160in}}%
\pgfpathlineto{\pgfqpoint{5.916564in}{2.588802in}}%
\pgfpathlineto{\pgfqpoint{5.902622in}{2.590469in}}%
\pgfpathlineto{\pgfqpoint{5.909647in}{2.595877in}}%
\pgfpathlineto{\pgfqpoint{5.916669in}{2.601335in}}%
\pgfpathlineto{\pgfqpoint{5.923687in}{2.606848in}}%
\pgfpathlineto{\pgfqpoint{5.930701in}{2.612422in}}%
\pgfpathclose%
\pgfusepath{fill}%
\end{pgfscope}%
\begin{pgfscope}%
\pgfpathrectangle{\pgfqpoint{1.150000in}{0.150000in}}{\pgfqpoint{5.700000in}{5.700000in}}%
\pgfusepath{clip}%
\pgfsetbuttcap%
\pgfsetroundjoin%
\definecolor{currentfill}{rgb}{0.273809,0.031497,0.358853}%
\pgfsetfillcolor{currentfill}%
\pgfsetfillopacity{0.700000}%
\pgfsetlinewidth{0.000000pt}%
\definecolor{currentstroke}{rgb}{0.000000,0.000000,0.000000}%
\pgfsetstrokecolor{currentstroke}%
\pgfsetdash{}{0pt}%
\pgfpathmoveto{\pgfqpoint{4.367541in}{2.422139in}}%
\pgfpathlineto{\pgfqpoint{4.381055in}{2.419367in}}%
\pgfpathlineto{\pgfqpoint{4.394576in}{2.416623in}}%
\pgfpathlineto{\pgfqpoint{4.408103in}{2.413906in}}%
\pgfpathlineto{\pgfqpoint{4.421637in}{2.411216in}}%
\pgfpathlineto{\pgfqpoint{4.413993in}{2.403583in}}%
\pgfpathlineto{\pgfqpoint{4.406343in}{2.395920in}}%
\pgfpathlineto{\pgfqpoint{4.398688in}{2.388228in}}%
\pgfpathlineto{\pgfqpoint{4.391026in}{2.380506in}}%
\pgfpathlineto{\pgfqpoint{4.377480in}{2.383211in}}%
\pgfpathlineto{\pgfqpoint{4.363940in}{2.385943in}}%
\pgfpathlineto{\pgfqpoint{4.350407in}{2.388702in}}%
\pgfpathlineto{\pgfqpoint{4.336880in}{2.391489in}}%
\pgfpathlineto{\pgfqpoint{4.344554in}{2.399191in}}%
\pgfpathlineto{\pgfqpoint{4.352222in}{2.406866in}}%
\pgfpathlineto{\pgfqpoint{4.359884in}{2.414515in}}%
\pgfpathlineto{\pgfqpoint{4.367541in}{2.422139in}}%
\pgfpathclose%
\pgfusepath{fill}%
\end{pgfscope}%
\begin{pgfscope}%
\pgfpathrectangle{\pgfqpoint{1.150000in}{0.150000in}}{\pgfqpoint{5.700000in}{5.700000in}}%
\pgfusepath{clip}%
\pgfsetbuttcap%
\pgfsetroundjoin%
\definecolor{currentfill}{rgb}{0.283072,0.130895,0.449241}%
\pgfsetfillcolor{currentfill}%
\pgfsetfillopacity{0.700000}%
\pgfsetlinewidth{0.000000pt}%
\definecolor{currentstroke}{rgb}{0.000000,0.000000,0.000000}%
\pgfsetstrokecolor{currentstroke}%
\pgfsetdash{}{0pt}%
\pgfpathmoveto{\pgfqpoint{5.707556in}{2.589196in}}%
\pgfpathlineto{\pgfqpoint{5.721422in}{2.587401in}}%
\pgfpathlineto{\pgfqpoint{5.735296in}{2.585630in}}%
\pgfpathlineto{\pgfqpoint{5.749177in}{2.583884in}}%
\pgfpathlineto{\pgfqpoint{5.763067in}{2.582161in}}%
\pgfpathlineto{\pgfqpoint{5.755979in}{2.576628in}}%
\pgfpathlineto{\pgfqpoint{5.748886in}{2.571123in}}%
\pgfpathlineto{\pgfqpoint{5.741788in}{2.565641in}}%
\pgfpathlineto{\pgfqpoint{5.734685in}{2.560176in}}%
\pgfpathlineto{\pgfqpoint{5.720775in}{2.561754in}}%
\pgfpathlineto{\pgfqpoint{5.706874in}{2.563356in}}%
\pgfpathlineto{\pgfqpoint{5.692980in}{2.564981in}}%
\pgfpathlineto{\pgfqpoint{5.679094in}{2.566631in}}%
\pgfpathlineto{\pgfqpoint{5.686217in}{2.572236in}}%
\pgfpathlineto{\pgfqpoint{5.693335in}{2.577862in}}%
\pgfpathlineto{\pgfqpoint{5.700448in}{2.583514in}}%
\pgfpathlineto{\pgfqpoint{5.707556in}{2.589196in}}%
\pgfpathclose%
\pgfusepath{fill}%
\end{pgfscope}%
\begin{pgfscope}%
\pgfpathrectangle{\pgfqpoint{1.150000in}{0.150000in}}{\pgfqpoint{5.700000in}{5.700000in}}%
\pgfusepath{clip}%
\pgfsetbuttcap%
\pgfsetroundjoin%
\definecolor{currentfill}{rgb}{0.277018,0.050344,0.375715}%
\pgfsetfillcolor{currentfill}%
\pgfsetfillopacity{0.700000}%
\pgfsetlinewidth{0.000000pt}%
\definecolor{currentstroke}{rgb}{0.000000,0.000000,0.000000}%
\pgfsetstrokecolor{currentstroke}%
\pgfsetdash{}{0pt}%
\pgfpathmoveto{\pgfqpoint{4.590904in}{2.450670in}}%
\pgfpathlineto{\pgfqpoint{4.604475in}{2.448215in}}%
\pgfpathlineto{\pgfqpoint{4.618053in}{2.445786in}}%
\pgfpathlineto{\pgfqpoint{4.631638in}{2.443384in}}%
\pgfpathlineto{\pgfqpoint{4.645230in}{2.441009in}}%
\pgfpathlineto{\pgfqpoint{4.637671in}{2.433641in}}%
\pgfpathlineto{\pgfqpoint{4.630105in}{2.426239in}}%
\pgfpathlineto{\pgfqpoint{4.622533in}{2.418802in}}%
\pgfpathlineto{\pgfqpoint{4.614956in}{2.411328in}}%
\pgfpathlineto{\pgfqpoint{4.601351in}{2.413692in}}%
\pgfpathlineto{\pgfqpoint{4.587753in}{2.416083in}}%
\pgfpathlineto{\pgfqpoint{4.574162in}{2.418500in}}%
\pgfpathlineto{\pgfqpoint{4.560578in}{2.420944in}}%
\pgfpathlineto{\pgfqpoint{4.568168in}{2.428424in}}%
\pgfpathlineto{\pgfqpoint{4.575753in}{2.435871in}}%
\pgfpathlineto{\pgfqpoint{4.583331in}{2.443286in}}%
\pgfpathlineto{\pgfqpoint{4.590904in}{2.450670in}}%
\pgfpathclose%
\pgfusepath{fill}%
\end{pgfscope}%
\begin{pgfscope}%
\pgfpathrectangle{\pgfqpoint{1.150000in}{0.150000in}}{\pgfqpoint{5.700000in}{5.700000in}}%
\pgfusepath{clip}%
\pgfsetbuttcap%
\pgfsetroundjoin%
\definecolor{currentfill}{rgb}{0.271305,0.019942,0.347269}%
\pgfsetfillcolor{currentfill}%
\pgfsetfillopacity{0.700000}%
\pgfsetlinewidth{0.000000pt}%
\definecolor{currentstroke}{rgb}{0.000000,0.000000,0.000000}%
\pgfsetstrokecolor{currentstroke}%
\pgfsetdash{}{0pt}%
\pgfpathmoveto{\pgfqpoint{4.144179in}{2.396399in}}%
\pgfpathlineto{\pgfqpoint{4.157639in}{2.393244in}}%
\pgfpathlineto{\pgfqpoint{4.171106in}{2.390118in}}%
\pgfpathlineto{\pgfqpoint{4.184579in}{2.387021in}}%
\pgfpathlineto{\pgfqpoint{4.198058in}{2.383953in}}%
\pgfpathlineto{\pgfqpoint{4.190332in}{2.376196in}}%
\pgfpathlineto{\pgfqpoint{4.182600in}{2.368420in}}%
\pgfpathlineto{\pgfqpoint{4.174862in}{2.360627in}}%
\pgfpathlineto{\pgfqpoint{4.167119in}{2.352817in}}%
\pgfpathlineto{\pgfqpoint{4.153627in}{2.355928in}}%
\pgfpathlineto{\pgfqpoint{4.140142in}{2.359066in}}%
\pgfpathlineto{\pgfqpoint{4.126663in}{2.362234in}}%
\pgfpathlineto{\pgfqpoint{4.113190in}{2.365430in}}%
\pgfpathlineto{\pgfqpoint{4.120946in}{2.373194in}}%
\pgfpathlineto{\pgfqpoint{4.128696in}{2.380943in}}%
\pgfpathlineto{\pgfqpoint{4.136440in}{2.388678in}}%
\pgfpathlineto{\pgfqpoint{4.144179in}{2.396399in}}%
\pgfpathclose%
\pgfusepath{fill}%
\end{pgfscope}%
\begin{pgfscope}%
\pgfpathrectangle{\pgfqpoint{1.150000in}{0.150000in}}{\pgfqpoint{5.700000in}{5.700000in}}%
\pgfusepath{clip}%
\pgfsetbuttcap%
\pgfsetroundjoin%
\definecolor{currentfill}{rgb}{0.279566,0.067836,0.391917}%
\pgfsetfillcolor{currentfill}%
\pgfsetfillopacity{0.700000}%
\pgfsetlinewidth{0.000000pt}%
\definecolor{currentstroke}{rgb}{0.000000,0.000000,0.000000}%
\pgfsetstrokecolor{currentstroke}%
\pgfsetdash{}{0pt}%
\pgfpathmoveto{\pgfqpoint{4.814279in}{2.480282in}}%
\pgfpathlineto{\pgfqpoint{4.827910in}{2.478079in}}%
\pgfpathlineto{\pgfqpoint{4.841547in}{2.475903in}}%
\pgfpathlineto{\pgfqpoint{4.855192in}{2.473752in}}%
\pgfpathlineto{\pgfqpoint{4.868844in}{2.471627in}}%
\pgfpathlineto{\pgfqpoint{4.861372in}{2.464622in}}%
\pgfpathlineto{\pgfqpoint{4.853894in}{2.457583in}}%
\pgfpathlineto{\pgfqpoint{4.846410in}{2.450509in}}%
\pgfpathlineto{\pgfqpoint{4.838920in}{2.443397in}}%
\pgfpathlineto{\pgfqpoint{4.825254in}{2.445484in}}%
\pgfpathlineto{\pgfqpoint{4.811596in}{2.447596in}}%
\pgfpathlineto{\pgfqpoint{4.797945in}{2.449734in}}%
\pgfpathlineto{\pgfqpoint{4.784301in}{2.451899in}}%
\pgfpathlineto{\pgfqpoint{4.791805in}{2.459044in}}%
\pgfpathlineto{\pgfqpoint{4.799302in}{2.466155in}}%
\pgfpathlineto{\pgfqpoint{4.806794in}{2.473234in}}%
\pgfpathlineto{\pgfqpoint{4.814279in}{2.480282in}}%
\pgfpathclose%
\pgfusepath{fill}%
\end{pgfscope}%
\begin{pgfscope}%
\pgfpathrectangle{\pgfqpoint{1.150000in}{0.150000in}}{\pgfqpoint{5.700000in}{5.700000in}}%
\pgfusepath{clip}%
\pgfsetbuttcap%
\pgfsetroundjoin%
\definecolor{currentfill}{rgb}{0.283197,0.115680,0.436115}%
\pgfsetfillcolor{currentfill}%
\pgfsetfillopacity{0.700000}%
\pgfsetlinewidth{0.000000pt}%
\definecolor{currentstroke}{rgb}{0.000000,0.000000,0.000000}%
\pgfsetstrokecolor{currentstroke}%
\pgfsetdash{}{0pt}%
\pgfpathmoveto{\pgfqpoint{5.484323in}{2.564442in}}%
\pgfpathlineto{\pgfqpoint{5.498131in}{2.562633in}}%
\pgfpathlineto{\pgfqpoint{5.511948in}{2.560849in}}%
\pgfpathlineto{\pgfqpoint{5.525773in}{2.559089in}}%
\pgfpathlineto{\pgfqpoint{5.539605in}{2.557353in}}%
\pgfpathlineto{\pgfqpoint{5.532418in}{2.551544in}}%
\pgfpathlineto{\pgfqpoint{5.525226in}{2.545738in}}%
\pgfpathlineto{\pgfqpoint{5.518028in}{2.539932in}}%
\pgfpathlineto{\pgfqpoint{5.510825in}{2.534121in}}%
\pgfpathlineto{\pgfqpoint{5.496974in}{2.535738in}}%
\pgfpathlineto{\pgfqpoint{5.483132in}{2.537379in}}%
\pgfpathlineto{\pgfqpoint{5.469298in}{2.539045in}}%
\pgfpathlineto{\pgfqpoint{5.455471in}{2.540735in}}%
\pgfpathlineto{\pgfqpoint{5.462692in}{2.546660in}}%
\pgfpathlineto{\pgfqpoint{5.469908in}{2.552584in}}%
\pgfpathlineto{\pgfqpoint{5.477118in}{2.558510in}}%
\pgfpathlineto{\pgfqpoint{5.484323in}{2.564442in}}%
\pgfpathclose%
\pgfusepath{fill}%
\end{pgfscope}%
\begin{pgfscope}%
\pgfpathrectangle{\pgfqpoint{1.150000in}{0.150000in}}{\pgfqpoint{5.700000in}{5.700000in}}%
\pgfusepath{clip}%
\pgfsetbuttcap%
\pgfsetroundjoin%
\definecolor{currentfill}{rgb}{0.281924,0.089666,0.412415}%
\pgfsetfillcolor{currentfill}%
\pgfsetfillopacity{0.700000}%
\pgfsetlinewidth{0.000000pt}%
\definecolor{currentstroke}{rgb}{0.000000,0.000000,0.000000}%
\pgfsetstrokecolor{currentstroke}%
\pgfsetdash{}{0pt}%
\pgfpathmoveto{\pgfqpoint{5.037658in}{2.509660in}}%
\pgfpathlineto{\pgfqpoint{5.051348in}{2.507649in}}%
\pgfpathlineto{\pgfqpoint{5.065045in}{2.505663in}}%
\pgfpathlineto{\pgfqpoint{5.078751in}{2.503702in}}%
\pgfpathlineto{\pgfqpoint{5.092463in}{2.501766in}}%
\pgfpathlineto{\pgfqpoint{5.085083in}{2.495173in}}%
\pgfpathlineto{\pgfqpoint{5.077697in}{2.488553in}}%
\pgfpathlineto{\pgfqpoint{5.070305in}{2.481903in}}%
\pgfpathlineto{\pgfqpoint{5.062906in}{2.475220in}}%
\pgfpathlineto{\pgfqpoint{5.049179in}{2.477091in}}%
\pgfpathlineto{\pgfqpoint{5.035459in}{2.478986in}}%
\pgfpathlineto{\pgfqpoint{5.021746in}{2.480907in}}%
\pgfpathlineto{\pgfqpoint{5.008042in}{2.482854in}}%
\pgfpathlineto{\pgfqpoint{5.015455in}{2.489597in}}%
\pgfpathlineto{\pgfqpoint{5.022862in}{2.496310in}}%
\pgfpathlineto{\pgfqpoint{5.030263in}{2.502997in}}%
\pgfpathlineto{\pgfqpoint{5.037658in}{2.509660in}}%
\pgfpathclose%
\pgfusepath{fill}%
\end{pgfscope}%
\begin{pgfscope}%
\pgfpathrectangle{\pgfqpoint{1.150000in}{0.150000in}}{\pgfqpoint{5.700000in}{5.700000in}}%
\pgfusepath{clip}%
\pgfsetbuttcap%
\pgfsetroundjoin%
\definecolor{currentfill}{rgb}{0.282910,0.105393,0.426902}%
\pgfsetfillcolor{currentfill}%
\pgfsetfillopacity{0.700000}%
\pgfsetlinewidth{0.000000pt}%
\definecolor{currentstroke}{rgb}{0.000000,0.000000,0.000000}%
\pgfsetstrokecolor{currentstroke}%
\pgfsetdash{}{0pt}%
\pgfpathmoveto{\pgfqpoint{5.261015in}{2.537888in}}%
\pgfpathlineto{\pgfqpoint{5.274765in}{2.536007in}}%
\pgfpathlineto{\pgfqpoint{5.288522in}{2.534151in}}%
\pgfpathlineto{\pgfqpoint{5.302288in}{2.532320in}}%
\pgfpathlineto{\pgfqpoint{5.316061in}{2.530514in}}%
\pgfpathlineto{\pgfqpoint{5.308776in}{2.524336in}}%
\pgfpathlineto{\pgfqpoint{5.301486in}{2.518142in}}%
\pgfpathlineto{\pgfqpoint{5.294189in}{2.511930in}}%
\pgfpathlineto{\pgfqpoint{5.286887in}{2.505697in}}%
\pgfpathlineto{\pgfqpoint{5.273097in}{2.507412in}}%
\pgfpathlineto{\pgfqpoint{5.259316in}{2.509151in}}%
\pgfpathlineto{\pgfqpoint{5.245542in}{2.510915in}}%
\pgfpathlineto{\pgfqpoint{5.231776in}{2.512703in}}%
\pgfpathlineto{\pgfqpoint{5.239095in}{2.519024in}}%
\pgfpathlineto{\pgfqpoint{5.246407in}{2.525326in}}%
\pgfpathlineto{\pgfqpoint{5.253714in}{2.531613in}}%
\pgfpathlineto{\pgfqpoint{5.261015in}{2.537888in}}%
\pgfpathclose%
\pgfusepath{fill}%
\end{pgfscope}%
\begin{pgfscope}%
\pgfpathrectangle{\pgfqpoint{1.150000in}{0.150000in}}{\pgfqpoint{5.700000in}{5.700000in}}%
\pgfusepath{clip}%
\pgfsetbuttcap%
\pgfsetroundjoin%
\definecolor{currentfill}{rgb}{0.281446,0.084320,0.407414}%
\pgfsetfillcolor{currentfill}%
\pgfsetfillopacity{0.700000}%
\pgfsetlinewidth{0.000000pt}%
\definecolor{currentstroke}{rgb}{0.000000,0.000000,0.000000}%
\pgfsetstrokecolor{currentstroke}%
\pgfsetdash{}{0pt}%
\pgfpathmoveto{\pgfqpoint{2.812152in}{2.503688in}}%
\pgfpathlineto{\pgfqpoint{2.825390in}{2.496761in}}%
\pgfpathlineto{\pgfqpoint{2.838632in}{2.489882in}}%
\pgfpathlineto{\pgfqpoint{2.851876in}{2.483050in}}%
\pgfpathlineto{\pgfqpoint{2.865124in}{2.476266in}}%
\pgfpathlineto{\pgfqpoint{2.856827in}{2.472331in}}%
\pgfpathlineto{\pgfqpoint{2.848519in}{2.468534in}}%
\pgfpathlineto{\pgfqpoint{2.840201in}{2.464880in}}%
\pgfpathlineto{\pgfqpoint{2.831872in}{2.461373in}}%
\pgfpathlineto{\pgfqpoint{2.818602in}{2.468332in}}%
\pgfpathlineto{\pgfqpoint{2.805336in}{2.475339in}}%
\pgfpathlineto{\pgfqpoint{2.792073in}{2.482394in}}%
\pgfpathlineto{\pgfqpoint{2.778813in}{2.489496in}}%
\pgfpathlineto{\pgfqpoint{2.787164in}{2.492823in}}%
\pgfpathlineto{\pgfqpoint{2.795504in}{2.496300in}}%
\pgfpathlineto{\pgfqpoint{2.803833in}{2.499923in}}%
\pgfpathlineto{\pgfqpoint{2.812152in}{2.503688in}}%
\pgfpathclose%
\pgfusepath{fill}%
\end{pgfscope}%
\begin{pgfscope}%
\pgfpathrectangle{\pgfqpoint{1.150000in}{0.150000in}}{\pgfqpoint{5.700000in}{5.700000in}}%
\pgfusepath{clip}%
\pgfsetbuttcap%
\pgfsetroundjoin%
\definecolor{currentfill}{rgb}{0.269944,0.014625,0.341379}%
\pgfsetfillcolor{currentfill}%
\pgfsetfillopacity{0.700000}%
\pgfsetlinewidth{0.000000pt}%
\definecolor{currentstroke}{rgb}{0.000000,0.000000,0.000000}%
\pgfsetstrokecolor{currentstroke}%
\pgfsetdash{}{0pt}%
\pgfpathmoveto{\pgfqpoint{3.281629in}{2.393768in}}%
\pgfpathlineto{\pgfqpoint{3.294919in}{2.388476in}}%
\pgfpathlineto{\pgfqpoint{3.308215in}{2.383223in}}%
\pgfpathlineto{\pgfqpoint{3.321515in}{2.378006in}}%
\pgfpathlineto{\pgfqpoint{3.334819in}{2.372827in}}%
\pgfpathlineto{\pgfqpoint{3.326757in}{2.366620in}}%
\pgfpathlineto{\pgfqpoint{3.318688in}{2.360485in}}%
\pgfpathlineto{\pgfqpoint{3.310610in}{2.354424in}}%
\pgfpathlineto{\pgfqpoint{3.302526in}{2.348441in}}%
\pgfpathlineto{\pgfqpoint{3.289205in}{2.353755in}}%
\pgfpathlineto{\pgfqpoint{3.275888in}{2.359106in}}%
\pgfpathlineto{\pgfqpoint{3.262576in}{2.364494in}}%
\pgfpathlineto{\pgfqpoint{3.249269in}{2.369920in}}%
\pgfpathlineto{\pgfqpoint{3.257370in}{2.375764in}}%
\pgfpathlineto{\pgfqpoint{3.265464in}{2.381688in}}%
\pgfpathlineto{\pgfqpoint{3.273550in}{2.387691in}}%
\pgfpathlineto{\pgfqpoint{3.281629in}{2.393768in}}%
\pgfpathclose%
\pgfusepath{fill}%
\end{pgfscope}%
\begin{pgfscope}%
\pgfpathrectangle{\pgfqpoint{1.150000in}{0.150000in}}{\pgfqpoint{5.700000in}{5.700000in}}%
\pgfusepath{clip}%
\pgfsetbuttcap%
\pgfsetroundjoin%
\definecolor{currentfill}{rgb}{0.268510,0.009605,0.335427}%
\pgfsetfillcolor{currentfill}%
\pgfsetfillopacity{0.700000}%
\pgfsetlinewidth{0.000000pt}%
\definecolor{currentstroke}{rgb}{0.000000,0.000000,0.000000}%
\pgfsetstrokecolor{currentstroke}%
\pgfsetdash{}{0pt}%
\pgfpathmoveto{\pgfqpoint{3.420198in}{2.378441in}}%
\pgfpathlineto{\pgfqpoint{3.433511in}{2.373564in}}%
\pgfpathlineto{\pgfqpoint{3.446828in}{2.368723in}}%
\pgfpathlineto{\pgfqpoint{3.460151in}{2.363917in}}%
\pgfpathlineto{\pgfqpoint{3.473479in}{2.359146in}}%
\pgfpathlineto{\pgfqpoint{3.465477in}{2.352451in}}%
\pgfpathlineto{\pgfqpoint{3.457467in}{2.345809in}}%
\pgfpathlineto{\pgfqpoint{3.449451in}{2.339223in}}%
\pgfpathlineto{\pgfqpoint{3.441428in}{2.332696in}}%
\pgfpathlineto{\pgfqpoint{3.428085in}{2.337588in}}%
\pgfpathlineto{\pgfqpoint{3.414747in}{2.342515in}}%
\pgfpathlineto{\pgfqpoint{3.401413in}{2.347477in}}%
\pgfpathlineto{\pgfqpoint{3.388085in}{2.352475in}}%
\pgfpathlineto{\pgfqpoint{3.396124in}{2.358876in}}%
\pgfpathlineto{\pgfqpoint{3.404155in}{2.365339in}}%
\pgfpathlineto{\pgfqpoint{3.412180in}{2.371862in}}%
\pgfpathlineto{\pgfqpoint{3.420198in}{2.378441in}}%
\pgfpathclose%
\pgfusepath{fill}%
\end{pgfscope}%
\begin{pgfscope}%
\pgfpathrectangle{\pgfqpoint{1.150000in}{0.150000in}}{\pgfqpoint{5.700000in}{5.700000in}}%
\pgfusepath{clip}%
\pgfsetbuttcap%
\pgfsetroundjoin%
\definecolor{currentfill}{rgb}{0.268510,0.009605,0.335427}%
\pgfsetfillcolor{currentfill}%
\pgfsetfillopacity{0.700000}%
\pgfsetlinewidth{0.000000pt}%
\definecolor{currentstroke}{rgb}{0.000000,0.000000,0.000000}%
\pgfsetstrokecolor{currentstroke}%
\pgfsetdash{}{0pt}%
\pgfpathmoveto{\pgfqpoint{3.920777in}{2.375564in}}%
\pgfpathlineto{\pgfqpoint{3.934188in}{2.371958in}}%
\pgfpathlineto{\pgfqpoint{3.947605in}{2.368381in}}%
\pgfpathlineto{\pgfqpoint{3.961028in}{2.364835in}}%
\pgfpathlineto{\pgfqpoint{3.974457in}{2.361319in}}%
\pgfpathlineto{\pgfqpoint{3.966648in}{2.353627in}}%
\pgfpathlineto{\pgfqpoint{3.958833in}{2.345934in}}%
\pgfpathlineto{\pgfqpoint{3.951013in}{2.338240in}}%
\pgfpathlineto{\pgfqpoint{3.943187in}{2.330548in}}%
\pgfpathlineto{\pgfqpoint{3.929745in}{2.334133in}}%
\pgfpathlineto{\pgfqpoint{3.916309in}{2.337747in}}%
\pgfpathlineto{\pgfqpoint{3.902880in}{2.341391in}}%
\pgfpathlineto{\pgfqpoint{3.889456in}{2.345066in}}%
\pgfpathlineto{\pgfqpoint{3.897295in}{2.352685in}}%
\pgfpathlineto{\pgfqpoint{3.905128in}{2.360309in}}%
\pgfpathlineto{\pgfqpoint{3.912955in}{2.367935in}}%
\pgfpathlineto{\pgfqpoint{3.920777in}{2.375564in}}%
\pgfpathclose%
\pgfusepath{fill}%
\end{pgfscope}%
\begin{pgfscope}%
\pgfpathrectangle{\pgfqpoint{1.150000in}{0.150000in}}{\pgfqpoint{5.700000in}{5.700000in}}%
\pgfusepath{clip}%
\pgfsetbuttcap%
\pgfsetroundjoin%
\definecolor{currentfill}{rgb}{0.273809,0.031497,0.358853}%
\pgfsetfillcolor{currentfill}%
\pgfsetfillopacity{0.700000}%
\pgfsetlinewidth{0.000000pt}%
\definecolor{currentstroke}{rgb}{0.000000,0.000000,0.000000}%
\pgfsetstrokecolor{currentstroke}%
\pgfsetdash{}{0pt}%
\pgfpathmoveto{\pgfqpoint{3.142967in}{2.414717in}}%
\pgfpathlineto{\pgfqpoint{3.156240in}{2.408979in}}%
\pgfpathlineto{\pgfqpoint{3.169517in}{2.403282in}}%
\pgfpathlineto{\pgfqpoint{3.182798in}{2.397624in}}%
\pgfpathlineto{\pgfqpoint{3.196083in}{2.392006in}}%
\pgfpathlineto{\pgfqpoint{3.187956in}{2.386389in}}%
\pgfpathlineto{\pgfqpoint{3.179821in}{2.380864in}}%
\pgfpathlineto{\pgfqpoint{3.171678in}{2.375433in}}%
\pgfpathlineto{\pgfqpoint{3.163526in}{2.370100in}}%
\pgfpathlineto{\pgfqpoint{3.150223in}{2.375867in}}%
\pgfpathlineto{\pgfqpoint{3.136923in}{2.381672in}}%
\pgfpathlineto{\pgfqpoint{3.123628in}{2.387518in}}%
\pgfpathlineto{\pgfqpoint{3.110338in}{2.393403in}}%
\pgfpathlineto{\pgfqpoint{3.118508in}{2.398583in}}%
\pgfpathlineto{\pgfqpoint{3.126669in}{2.403864in}}%
\pgfpathlineto{\pgfqpoint{3.134822in}{2.409243in}}%
\pgfpathlineto{\pgfqpoint{3.142967in}{2.414717in}}%
\pgfpathclose%
\pgfusepath{fill}%
\end{pgfscope}%
\begin{pgfscope}%
\pgfpathrectangle{\pgfqpoint{1.150000in}{0.150000in}}{\pgfqpoint{5.700000in}{5.700000in}}%
\pgfusepath{clip}%
\pgfsetbuttcap%
\pgfsetroundjoin%
\definecolor{currentfill}{rgb}{0.267004,0.004874,0.329415}%
\pgfsetfillcolor{currentfill}%
\pgfsetfillopacity{0.700000}%
\pgfsetlinewidth{0.000000pt}%
\definecolor{currentstroke}{rgb}{0.000000,0.000000,0.000000}%
\pgfsetstrokecolor{currentstroke}%
\pgfsetdash{}{0pt}%
\pgfpathmoveto{\pgfqpoint{3.558725in}{2.368102in}}%
\pgfpathlineto{\pgfqpoint{3.572063in}{2.363611in}}%
\pgfpathlineto{\pgfqpoint{3.585407in}{2.359154in}}%
\pgfpathlineto{\pgfqpoint{3.598757in}{2.354730in}}%
\pgfpathlineto{\pgfqpoint{3.612111in}{2.350340in}}%
\pgfpathlineto{\pgfqpoint{3.604164in}{2.343251in}}%
\pgfpathlineto{\pgfqpoint{3.596211in}{2.336200in}}%
\pgfpathlineto{\pgfqpoint{3.588252in}{2.329187in}}%
\pgfpathlineto{\pgfqpoint{3.580285in}{2.322215in}}%
\pgfpathlineto{\pgfqpoint{3.566916in}{2.326713in}}%
\pgfpathlineto{\pgfqpoint{3.553553in}{2.331245in}}%
\pgfpathlineto{\pgfqpoint{3.540194in}{2.335810in}}%
\pgfpathlineto{\pgfqpoint{3.526841in}{2.340409in}}%
\pgfpathlineto{\pgfqpoint{3.534822in}{2.347268in}}%
\pgfpathlineto{\pgfqpoint{3.542796in}{2.354171in}}%
\pgfpathlineto{\pgfqpoint{3.550763in}{2.361117in}}%
\pgfpathlineto{\pgfqpoint{3.558725in}{2.368102in}}%
\pgfpathclose%
\pgfusepath{fill}%
\end{pgfscope}%
\begin{pgfscope}%
\pgfpathrectangle{\pgfqpoint{1.150000in}{0.150000in}}{\pgfqpoint{5.700000in}{5.700000in}}%
\pgfusepath{clip}%
\pgfsetbuttcap%
\pgfsetroundjoin%
\definecolor{currentfill}{rgb}{0.277018,0.050344,0.375715}%
\pgfsetfillcolor{currentfill}%
\pgfsetfillopacity{0.700000}%
\pgfsetlinewidth{0.000000pt}%
\definecolor{currentstroke}{rgb}{0.000000,0.000000,0.000000}%
\pgfsetstrokecolor{currentstroke}%
\pgfsetdash{}{0pt}%
\pgfpathmoveto{\pgfqpoint{3.004154in}{2.441971in}}%
\pgfpathlineto{\pgfqpoint{3.017413in}{2.435753in}}%
\pgfpathlineto{\pgfqpoint{3.030677in}{2.429577in}}%
\pgfpathlineto{\pgfqpoint{3.043944in}{2.423444in}}%
\pgfpathlineto{\pgfqpoint{3.057214in}{2.417353in}}%
\pgfpathlineto{\pgfqpoint{3.049017in}{2.412435in}}%
\pgfpathlineto{\pgfqpoint{3.040810in}{2.407629in}}%
\pgfpathlineto{\pgfqpoint{3.032593in}{2.402939in}}%
\pgfpathlineto{\pgfqpoint{3.024368in}{2.398369in}}%
\pgfpathlineto{\pgfqpoint{3.011077in}{2.404621in}}%
\pgfpathlineto{\pgfqpoint{2.997791in}{2.410916in}}%
\pgfpathlineto{\pgfqpoint{2.984508in}{2.417253in}}%
\pgfpathlineto{\pgfqpoint{2.971229in}{2.423633in}}%
\pgfpathlineto{\pgfqpoint{2.979475in}{2.428036in}}%
\pgfpathlineto{\pgfqpoint{2.987711in}{2.432563in}}%
\pgfpathlineto{\pgfqpoint{2.995937in}{2.437209in}}%
\pgfpathlineto{\pgfqpoint{3.004154in}{2.441971in}}%
\pgfpathclose%
\pgfusepath{fill}%
\end{pgfscope}%
\begin{pgfscope}%
\pgfpathrectangle{\pgfqpoint{1.150000in}{0.150000in}}{\pgfqpoint{5.700000in}{5.700000in}}%
\pgfusepath{clip}%
\pgfsetbuttcap%
\pgfsetroundjoin%
\definecolor{currentfill}{rgb}{0.267004,0.004874,0.329415}%
\pgfsetfillcolor{currentfill}%
\pgfsetfillopacity{0.700000}%
\pgfsetlinewidth{0.000000pt}%
\definecolor{currentstroke}{rgb}{0.000000,0.000000,0.000000}%
\pgfsetstrokecolor{currentstroke}%
\pgfsetdash{}{0pt}%
\pgfpathmoveto{\pgfqpoint{3.697253in}{2.362162in}}%
\pgfpathlineto{\pgfqpoint{3.710621in}{2.358030in}}%
\pgfpathlineto{\pgfqpoint{3.723994in}{2.353931in}}%
\pgfpathlineto{\pgfqpoint{3.737374in}{2.349863in}}%
\pgfpathlineto{\pgfqpoint{3.750758in}{2.345827in}}%
\pgfpathlineto{\pgfqpoint{3.742864in}{2.338435in}}%
\pgfpathlineto{\pgfqpoint{3.734963in}{2.331064in}}%
\pgfpathlineto{\pgfqpoint{3.727057in}{2.323716in}}%
\pgfpathlineto{\pgfqpoint{3.719144in}{2.316394in}}%
\pgfpathlineto{\pgfqpoint{3.705745in}{2.320525in}}%
\pgfpathlineto{\pgfqpoint{3.692353in}{2.324687in}}%
\pgfpathlineto{\pgfqpoint{3.678965in}{2.328881in}}%
\pgfpathlineto{\pgfqpoint{3.665584in}{2.333108in}}%
\pgfpathlineto{\pgfqpoint{3.673510in}{2.340330in}}%
\pgfpathlineto{\pgfqpoint{3.681431in}{2.347582in}}%
\pgfpathlineto{\pgfqpoint{3.689345in}{2.354860in}}%
\pgfpathlineto{\pgfqpoint{3.697253in}{2.362162in}}%
\pgfpathclose%
\pgfusepath{fill}%
\end{pgfscope}%
\begin{pgfscope}%
\pgfpathrectangle{\pgfqpoint{1.150000in}{0.150000in}}{\pgfqpoint{5.700000in}{5.700000in}}%
\pgfusepath{clip}%
\pgfsetbuttcap%
\pgfsetroundjoin%
\definecolor{currentfill}{rgb}{0.282884,0.135920,0.453427}%
\pgfsetfillcolor{currentfill}%
\pgfsetfillopacity{0.700000}%
\pgfsetlinewidth{0.000000pt}%
\definecolor{currentstroke}{rgb}{0.000000,0.000000,0.000000}%
\pgfsetstrokecolor{currentstroke}%
\pgfsetdash{}{0pt}%
\pgfpathmoveto{\pgfqpoint{5.846935in}{2.597373in}}%
\pgfpathlineto{\pgfqpoint{5.860844in}{2.595611in}}%
\pgfpathlineto{\pgfqpoint{5.874762in}{2.593873in}}%
\pgfpathlineto{\pgfqpoint{5.888688in}{2.592159in}}%
\pgfpathlineto{\pgfqpoint{5.902622in}{2.590469in}}%
\pgfpathlineto{\pgfqpoint{5.895593in}{2.585106in}}%
\pgfpathlineto{\pgfqpoint{5.888559in}{2.579784in}}%
\pgfpathlineto{\pgfqpoint{5.881521in}{2.574496in}}%
\pgfpathlineto{\pgfqpoint{5.874478in}{2.569239in}}%
\pgfpathlineto{\pgfqpoint{5.860523in}{2.570771in}}%
\pgfpathlineto{\pgfqpoint{5.846577in}{2.572326in}}%
\pgfpathlineto{\pgfqpoint{5.832638in}{2.573905in}}%
\pgfpathlineto{\pgfqpoint{5.818708in}{2.575509in}}%
\pgfpathlineto{\pgfqpoint{5.825771in}{2.580920in}}%
\pgfpathlineto{\pgfqpoint{5.832830in}{2.586364in}}%
\pgfpathlineto{\pgfqpoint{5.839884in}{2.591847in}}%
\pgfpathlineto{\pgfqpoint{5.846935in}{2.597373in}}%
\pgfpathclose%
\pgfusepath{fill}%
\end{pgfscope}%
\begin{pgfscope}%
\pgfpathrectangle{\pgfqpoint{1.150000in}{0.150000in}}{\pgfqpoint{5.700000in}{5.700000in}}%
\pgfusepath{clip}%
\pgfsetbuttcap%
\pgfsetroundjoin%
\definecolor{currentfill}{rgb}{0.272594,0.025563,0.353093}%
\pgfsetfillcolor{currentfill}%
\pgfsetfillopacity{0.700000}%
\pgfsetlinewidth{0.000000pt}%
\definecolor{currentstroke}{rgb}{0.000000,0.000000,0.000000}%
\pgfsetstrokecolor{currentstroke}%
\pgfsetdash{}{0pt}%
\pgfpathmoveto{\pgfqpoint{4.282841in}{2.402916in}}%
\pgfpathlineto{\pgfqpoint{4.296341in}{2.400017in}}%
\pgfpathlineto{\pgfqpoint{4.309848in}{2.397147in}}%
\pgfpathlineto{\pgfqpoint{4.323361in}{2.394304in}}%
\pgfpathlineto{\pgfqpoint{4.336880in}{2.391489in}}%
\pgfpathlineto{\pgfqpoint{4.329201in}{2.383761in}}%
\pgfpathlineto{\pgfqpoint{4.321516in}{2.376007in}}%
\pgfpathlineto{\pgfqpoint{4.313825in}{2.368228in}}%
\pgfpathlineto{\pgfqpoint{4.306129in}{2.360422in}}%
\pgfpathlineto{\pgfqpoint{4.292597in}{2.363266in}}%
\pgfpathlineto{\pgfqpoint{4.279072in}{2.366137in}}%
\pgfpathlineto{\pgfqpoint{4.265553in}{2.369036in}}%
\pgfpathlineto{\pgfqpoint{4.252041in}{2.371963in}}%
\pgfpathlineto{\pgfqpoint{4.259750in}{2.379735in}}%
\pgfpathlineto{\pgfqpoint{4.267453in}{2.387484in}}%
\pgfpathlineto{\pgfqpoint{4.275150in}{2.395211in}}%
\pgfpathlineto{\pgfqpoint{4.282841in}{2.402916in}}%
\pgfpathclose%
\pgfusepath{fill}%
\end{pgfscope}%
\begin{pgfscope}%
\pgfpathrectangle{\pgfqpoint{1.150000in}{0.150000in}}{\pgfqpoint{5.700000in}{5.700000in}}%
\pgfusepath{clip}%
\pgfsetbuttcap%
\pgfsetroundjoin%
\definecolor{currentfill}{rgb}{0.276022,0.044167,0.370164}%
\pgfsetfillcolor{currentfill}%
\pgfsetfillopacity{0.700000}%
\pgfsetlinewidth{0.000000pt}%
\definecolor{currentstroke}{rgb}{0.000000,0.000000,0.000000}%
\pgfsetstrokecolor{currentstroke}%
\pgfsetdash{}{0pt}%
\pgfpathmoveto{\pgfqpoint{4.506312in}{2.430988in}}%
\pgfpathlineto{\pgfqpoint{4.519868in}{2.428437in}}%
\pgfpathlineto{\pgfqpoint{4.533431in}{2.425912in}}%
\pgfpathlineto{\pgfqpoint{4.547001in}{2.423415in}}%
\pgfpathlineto{\pgfqpoint{4.560578in}{2.420944in}}%
\pgfpathlineto{\pgfqpoint{4.552982in}{2.413430in}}%
\pgfpathlineto{\pgfqpoint{4.545380in}{2.405882in}}%
\pgfpathlineto{\pgfqpoint{4.537772in}{2.398299in}}%
\pgfpathlineto{\pgfqpoint{4.530159in}{2.390681in}}%
\pgfpathlineto{\pgfqpoint{4.516569in}{2.393153in}}%
\pgfpathlineto{\pgfqpoint{4.502987in}{2.395653in}}%
\pgfpathlineto{\pgfqpoint{4.489411in}{2.398179in}}%
\pgfpathlineto{\pgfqpoint{4.475843in}{2.400732in}}%
\pgfpathlineto{\pgfqpoint{4.483469in}{2.408344in}}%
\pgfpathlineto{\pgfqpoint{4.491089in}{2.415923in}}%
\pgfpathlineto{\pgfqpoint{4.498704in}{2.423471in}}%
\pgfpathlineto{\pgfqpoint{4.506312in}{2.430988in}}%
\pgfpathclose%
\pgfusepath{fill}%
\end{pgfscope}%
\begin{pgfscope}%
\pgfpathrectangle{\pgfqpoint{1.150000in}{0.150000in}}{\pgfqpoint{5.700000in}{5.700000in}}%
\pgfusepath{clip}%
\pgfsetbuttcap%
\pgfsetroundjoin%
\definecolor{currentfill}{rgb}{0.278791,0.062145,0.386592}%
\pgfsetfillcolor{currentfill}%
\pgfsetfillopacity{0.700000}%
\pgfsetlinewidth{0.000000pt}%
\definecolor{currentstroke}{rgb}{0.000000,0.000000,0.000000}%
\pgfsetstrokecolor{currentstroke}%
\pgfsetdash{}{0pt}%
\pgfpathmoveto{\pgfqpoint{4.729798in}{2.460816in}}%
\pgfpathlineto{\pgfqpoint{4.743413in}{2.458548in}}%
\pgfpathlineto{\pgfqpoint{4.757035in}{2.456305in}}%
\pgfpathlineto{\pgfqpoint{4.770664in}{2.454089in}}%
\pgfpathlineto{\pgfqpoint{4.784301in}{2.451899in}}%
\pgfpathlineto{\pgfqpoint{4.776791in}{2.444718in}}%
\pgfpathlineto{\pgfqpoint{4.769276in}{2.437502in}}%
\pgfpathlineto{\pgfqpoint{4.761754in}{2.430247in}}%
\pgfpathlineto{\pgfqpoint{4.754226in}{2.422953in}}%
\pgfpathlineto{\pgfqpoint{4.740576in}{2.425118in}}%
\pgfpathlineto{\pgfqpoint{4.726934in}{2.427309in}}%
\pgfpathlineto{\pgfqpoint{4.713299in}{2.429527in}}%
\pgfpathlineto{\pgfqpoint{4.699671in}{2.431771in}}%
\pgfpathlineto{\pgfqpoint{4.707212in}{2.439085in}}%
\pgfpathlineto{\pgfqpoint{4.714746in}{2.446363in}}%
\pgfpathlineto{\pgfqpoint{4.722275in}{2.453606in}}%
\pgfpathlineto{\pgfqpoint{4.729798in}{2.460816in}}%
\pgfpathclose%
\pgfusepath{fill}%
\end{pgfscope}%
\begin{pgfscope}%
\pgfpathrectangle{\pgfqpoint{1.150000in}{0.150000in}}{\pgfqpoint{5.700000in}{5.700000in}}%
\pgfusepath{clip}%
\pgfsetbuttcap%
\pgfsetroundjoin%
\definecolor{currentfill}{rgb}{0.269944,0.014625,0.341379}%
\pgfsetfillcolor{currentfill}%
\pgfsetfillopacity{0.700000}%
\pgfsetlinewidth{0.000000pt}%
\definecolor{currentstroke}{rgb}{0.000000,0.000000,0.000000}%
\pgfsetstrokecolor{currentstroke}%
\pgfsetdash{}{0pt}%
\pgfpathmoveto{\pgfqpoint{4.059363in}{2.378506in}}%
\pgfpathlineto{\pgfqpoint{4.072810in}{2.375194in}}%
\pgfpathlineto{\pgfqpoint{4.086264in}{2.371910in}}%
\pgfpathlineto{\pgfqpoint{4.099724in}{2.368656in}}%
\pgfpathlineto{\pgfqpoint{4.113190in}{2.365430in}}%
\pgfpathlineto{\pgfqpoint{4.105429in}{2.357654in}}%
\pgfpathlineto{\pgfqpoint{4.097662in}{2.349866in}}%
\pgfpathlineto{\pgfqpoint{4.089889in}{2.342067in}}%
\pgfpathlineto{\pgfqpoint{4.082111in}{2.334257in}}%
\pgfpathlineto{\pgfqpoint{4.068632in}{2.337537in}}%
\pgfpathlineto{\pgfqpoint{4.055160in}{2.340847in}}%
\pgfpathlineto{\pgfqpoint{4.041694in}{2.344185in}}%
\pgfpathlineto{\pgfqpoint{4.028234in}{2.347553in}}%
\pgfpathlineto{\pgfqpoint{4.036025in}{2.355303in}}%
\pgfpathlineto{\pgfqpoint{4.043810in}{2.363045in}}%
\pgfpathlineto{\pgfqpoint{4.051589in}{2.370780in}}%
\pgfpathlineto{\pgfqpoint{4.059363in}{2.378506in}}%
\pgfpathclose%
\pgfusepath{fill}%
\end{pgfscope}%
\begin{pgfscope}%
\pgfpathrectangle{\pgfqpoint{1.150000in}{0.150000in}}{\pgfqpoint{5.700000in}{5.700000in}}%
\pgfusepath{clip}%
\pgfsetbuttcap%
\pgfsetroundjoin%
\definecolor{currentfill}{rgb}{0.283187,0.125848,0.444960}%
\pgfsetfillcolor{currentfill}%
\pgfsetfillopacity{0.700000}%
\pgfsetlinewidth{0.000000pt}%
\definecolor{currentstroke}{rgb}{0.000000,0.000000,0.000000}%
\pgfsetstrokecolor{currentstroke}%
\pgfsetdash{}{0pt}%
\pgfpathmoveto{\pgfqpoint{5.623633in}{2.573471in}}%
\pgfpathlineto{\pgfqpoint{5.637486in}{2.571725in}}%
\pgfpathlineto{\pgfqpoint{5.651348in}{2.570003in}}%
\pgfpathlineto{\pgfqpoint{5.665217in}{2.568305in}}%
\pgfpathlineto{\pgfqpoint{5.679094in}{2.566631in}}%
\pgfpathlineto{\pgfqpoint{5.671967in}{2.561044in}}%
\pgfpathlineto{\pgfqpoint{5.664834in}{2.555469in}}%
\pgfpathlineto{\pgfqpoint{5.657696in}{2.549903in}}%
\pgfpathlineto{\pgfqpoint{5.650552in}{2.544341in}}%
\pgfpathlineto{\pgfqpoint{5.636655in}{2.545883in}}%
\pgfpathlineto{\pgfqpoint{5.622767in}{2.547449in}}%
\pgfpathlineto{\pgfqpoint{5.608886in}{2.549039in}}%
\pgfpathlineto{\pgfqpoint{5.595014in}{2.550653in}}%
\pgfpathlineto{\pgfqpoint{5.602177in}{2.556342in}}%
\pgfpathlineto{\pgfqpoint{5.609334in}{2.562039in}}%
\pgfpathlineto{\pgfqpoint{5.616486in}{2.567747in}}%
\pgfpathlineto{\pgfqpoint{5.623633in}{2.573471in}}%
\pgfpathclose%
\pgfusepath{fill}%
\end{pgfscope}%
\begin{pgfscope}%
\pgfpathrectangle{\pgfqpoint{1.150000in}{0.150000in}}{\pgfqpoint{5.700000in}{5.700000in}}%
\pgfusepath{clip}%
\pgfsetbuttcap%
\pgfsetroundjoin%
\definecolor{currentfill}{rgb}{0.281446,0.084320,0.407414}%
\pgfsetfillcolor{currentfill}%
\pgfsetfillopacity{0.700000}%
\pgfsetlinewidth{0.000000pt}%
\definecolor{currentstroke}{rgb}{0.000000,0.000000,0.000000}%
\pgfsetstrokecolor{currentstroke}%
\pgfsetdash{}{0pt}%
\pgfpathmoveto{\pgfqpoint{4.953297in}{2.490894in}}%
\pgfpathlineto{\pgfqpoint{4.966972in}{2.488845in}}%
\pgfpathlineto{\pgfqpoint{4.980654in}{2.486823in}}%
\pgfpathlineto{\pgfqpoint{4.994344in}{2.484826in}}%
\pgfpathlineto{\pgfqpoint{5.008042in}{2.482854in}}%
\pgfpathlineto{\pgfqpoint{5.000622in}{2.476080in}}%
\pgfpathlineto{\pgfqpoint{4.993197in}{2.469273in}}%
\pgfpathlineto{\pgfqpoint{4.985765in}{2.462431in}}%
\pgfpathlineto{\pgfqpoint{4.978327in}{2.455552in}}%
\pgfpathlineto{\pgfqpoint{4.964616in}{2.457472in}}%
\pgfpathlineto{\pgfqpoint{4.950912in}{2.459417in}}%
\pgfpathlineto{\pgfqpoint{4.937215in}{2.461388in}}%
\pgfpathlineto{\pgfqpoint{4.923526in}{2.463385in}}%
\pgfpathlineto{\pgfqpoint{4.930978in}{2.470311in}}%
\pgfpathlineto{\pgfqpoint{4.938424in}{2.477203in}}%
\pgfpathlineto{\pgfqpoint{4.945863in}{2.484063in}}%
\pgfpathlineto{\pgfqpoint{4.953297in}{2.490894in}}%
\pgfpathclose%
\pgfusepath{fill}%
\end{pgfscope}%
\begin{pgfscope}%
\pgfpathrectangle{\pgfqpoint{1.150000in}{0.150000in}}{\pgfqpoint{5.700000in}{5.700000in}}%
\pgfusepath{clip}%
\pgfsetbuttcap%
\pgfsetroundjoin%
\definecolor{currentfill}{rgb}{0.283197,0.115680,0.436115}%
\pgfsetfillcolor{currentfill}%
\pgfsetfillopacity{0.700000}%
\pgfsetlinewidth{0.000000pt}%
\definecolor{currentstroke}{rgb}{0.000000,0.000000,0.000000}%
\pgfsetstrokecolor{currentstroke}%
\pgfsetdash{}{0pt}%
\pgfpathmoveto{\pgfqpoint{5.400244in}{2.547741in}}%
\pgfpathlineto{\pgfqpoint{5.414039in}{2.545953in}}%
\pgfpathlineto{\pgfqpoint{5.427842in}{2.544189in}}%
\pgfpathlineto{\pgfqpoint{5.441652in}{2.542450in}}%
\pgfpathlineto{\pgfqpoint{5.455471in}{2.540735in}}%
\pgfpathlineto{\pgfqpoint{5.448244in}{2.534805in}}%
\pgfpathlineto{\pgfqpoint{5.441011in}{2.528867in}}%
\pgfpathlineto{\pgfqpoint{5.433772in}{2.522917in}}%
\pgfpathlineto{\pgfqpoint{5.426527in}{2.516951in}}%
\pgfpathlineto{\pgfqpoint{5.412691in}{2.518561in}}%
\pgfpathlineto{\pgfqpoint{5.398863in}{2.520195in}}%
\pgfpathlineto{\pgfqpoint{5.385043in}{2.521853in}}%
\pgfpathlineto{\pgfqpoint{5.371231in}{2.523536in}}%
\pgfpathlineto{\pgfqpoint{5.378493in}{2.529602in}}%
\pgfpathlineto{\pgfqpoint{5.385749in}{2.535656in}}%
\pgfpathlineto{\pgfqpoint{5.392999in}{2.541701in}}%
\pgfpathlineto{\pgfqpoint{5.400244in}{2.547741in}}%
\pgfpathclose%
\pgfusepath{fill}%
\end{pgfscope}%
\begin{pgfscope}%
\pgfpathrectangle{\pgfqpoint{1.150000in}{0.150000in}}{\pgfqpoint{5.700000in}{5.700000in}}%
\pgfusepath{clip}%
\pgfsetbuttcap%
\pgfsetroundjoin%
\definecolor{currentfill}{rgb}{0.282656,0.100196,0.422160}%
\pgfsetfillcolor{currentfill}%
\pgfsetfillopacity{0.700000}%
\pgfsetlinewidth{0.000000pt}%
\definecolor{currentstroke}{rgb}{0.000000,0.000000,0.000000}%
\pgfsetstrokecolor{currentstroke}%
\pgfsetdash{}{0pt}%
\pgfpathmoveto{\pgfqpoint{5.176788in}{2.520108in}}%
\pgfpathlineto{\pgfqpoint{5.190524in}{2.518219in}}%
\pgfpathlineto{\pgfqpoint{5.204267in}{2.516356in}}%
\pgfpathlineto{\pgfqpoint{5.218017in}{2.514517in}}%
\pgfpathlineto{\pgfqpoint{5.231776in}{2.512703in}}%
\pgfpathlineto{\pgfqpoint{5.224451in}{2.506362in}}%
\pgfpathlineto{\pgfqpoint{5.217120in}{2.499997in}}%
\pgfpathlineto{\pgfqpoint{5.209783in}{2.493606in}}%
\pgfpathlineto{\pgfqpoint{5.202440in}{2.487185in}}%
\pgfpathlineto{\pgfqpoint{5.188666in}{2.488920in}}%
\pgfpathlineto{\pgfqpoint{5.174899in}{2.490680in}}%
\pgfpathlineto{\pgfqpoint{5.161141in}{2.492465in}}%
\pgfpathlineto{\pgfqpoint{5.147390in}{2.494275in}}%
\pgfpathlineto{\pgfqpoint{5.154749in}{2.500770in}}%
\pgfpathlineto{\pgfqpoint{5.162101in}{2.507238in}}%
\pgfpathlineto{\pgfqpoint{5.169448in}{2.513683in}}%
\pgfpathlineto{\pgfqpoint{5.176788in}{2.520108in}}%
\pgfpathclose%
\pgfusepath{fill}%
\end{pgfscope}%
\begin{pgfscope}%
\pgfpathrectangle{\pgfqpoint{1.150000in}{0.150000in}}{\pgfqpoint{5.700000in}{5.700000in}}%
\pgfusepath{clip}%
\pgfsetbuttcap%
\pgfsetroundjoin%
\definecolor{currentfill}{rgb}{0.283091,0.110553,0.431554}%
\pgfsetfillcolor{currentfill}%
\pgfsetfillopacity{0.700000}%
\pgfsetlinewidth{0.000000pt}%
\definecolor{currentstroke}{rgb}{0.000000,0.000000,0.000000}%
\pgfsetstrokecolor{currentstroke}%
\pgfsetdash{}{0pt}%
\pgfpathmoveto{\pgfqpoint{2.672834in}{2.548106in}}%
\pgfpathlineto{\pgfqpoint{2.686071in}{2.540601in}}%
\pgfpathlineto{\pgfqpoint{2.699312in}{2.533148in}}%
\pgfpathlineto{\pgfqpoint{2.712555in}{2.525747in}}%
\pgfpathlineto{\pgfqpoint{2.725801in}{2.518397in}}%
\pgfpathlineto{\pgfqpoint{2.717415in}{2.515410in}}%
\pgfpathlineto{\pgfqpoint{2.709017in}{2.512585in}}%
\pgfpathlineto{\pgfqpoint{2.700608in}{2.509929in}}%
\pgfpathlineto{\pgfqpoint{2.692186in}{2.507444in}}%
\pgfpathlineto{\pgfqpoint{2.678916in}{2.514984in}}%
\pgfpathlineto{\pgfqpoint{2.665649in}{2.522575in}}%
\pgfpathlineto{\pgfqpoint{2.652385in}{2.530217in}}%
\pgfpathlineto{\pgfqpoint{2.639123in}{2.537912in}}%
\pgfpathlineto{\pgfqpoint{2.647569in}{2.540202in}}%
\pgfpathlineto{\pgfqpoint{2.656003in}{2.542667in}}%
\pgfpathlineto{\pgfqpoint{2.664424in}{2.545303in}}%
\pgfpathlineto{\pgfqpoint{2.672834in}{2.548106in}}%
\pgfpathclose%
\pgfusepath{fill}%
\end{pgfscope}%
\begin{pgfscope}%
\pgfpathrectangle{\pgfqpoint{1.150000in}{0.150000in}}{\pgfqpoint{5.700000in}{5.700000in}}%
\pgfusepath{clip}%
\pgfsetbuttcap%
\pgfsetroundjoin%
\definecolor{currentfill}{rgb}{0.267004,0.004874,0.329415}%
\pgfsetfillcolor{currentfill}%
\pgfsetfillopacity{0.700000}%
\pgfsetlinewidth{0.000000pt}%
\definecolor{currentstroke}{rgb}{0.000000,0.000000,0.000000}%
\pgfsetstrokecolor{currentstroke}%
\pgfsetdash{}{0pt}%
\pgfpathmoveto{\pgfqpoint{3.835820in}{2.360071in}}%
\pgfpathlineto{\pgfqpoint{3.849220in}{2.356274in}}%
\pgfpathlineto{\pgfqpoint{3.862626in}{2.352507in}}%
\pgfpathlineto{\pgfqpoint{3.876038in}{2.348772in}}%
\pgfpathlineto{\pgfqpoint{3.889456in}{2.345066in}}%
\pgfpathlineto{\pgfqpoint{3.881611in}{2.337454in}}%
\pgfpathlineto{\pgfqpoint{3.873761in}{2.329849in}}%
\pgfpathlineto{\pgfqpoint{3.865905in}{2.322253in}}%
\pgfpathlineto{\pgfqpoint{3.858042in}{2.314668in}}%
\pgfpathlineto{\pgfqpoint{3.844612in}{2.318455in}}%
\pgfpathlineto{\pgfqpoint{3.831187in}{2.322272in}}%
\pgfpathlineto{\pgfqpoint{3.817768in}{2.326120in}}%
\pgfpathlineto{\pgfqpoint{3.804354in}{2.329999in}}%
\pgfpathlineto{\pgfqpoint{3.812229in}{2.337497in}}%
\pgfpathlineto{\pgfqpoint{3.820099in}{2.345010in}}%
\pgfpathlineto{\pgfqpoint{3.827962in}{2.352535in}}%
\pgfpathlineto{\pgfqpoint{3.835820in}{2.360071in}}%
\pgfpathclose%
\pgfusepath{fill}%
\end{pgfscope}%
\begin{pgfscope}%
\pgfpathrectangle{\pgfqpoint{1.150000in}{0.150000in}}{\pgfqpoint{5.700000in}{5.700000in}}%
\pgfusepath{clip}%
\pgfsetbuttcap%
\pgfsetroundjoin%
\definecolor{currentfill}{rgb}{0.280267,0.073417,0.397163}%
\pgfsetfillcolor{currentfill}%
\pgfsetfillopacity{0.700000}%
\pgfsetlinewidth{0.000000pt}%
\definecolor{currentstroke}{rgb}{0.000000,0.000000,0.000000}%
\pgfsetstrokecolor{currentstroke}%
\pgfsetdash{}{0pt}%
\pgfpathmoveto{\pgfqpoint{2.865124in}{2.476266in}}%
\pgfpathlineto{\pgfqpoint{2.878375in}{2.469528in}}%
\pgfpathlineto{\pgfqpoint{2.891630in}{2.462837in}}%
\pgfpathlineto{\pgfqpoint{2.904887in}{2.456191in}}%
\pgfpathlineto{\pgfqpoint{2.918149in}{2.449590in}}%
\pgfpathlineto{\pgfqpoint{2.909873in}{2.445485in}}%
\pgfpathlineto{\pgfqpoint{2.901586in}{2.441514in}}%
\pgfpathlineto{\pgfqpoint{2.893290in}{2.437683in}}%
\pgfpathlineto{\pgfqpoint{2.884982in}{2.433995in}}%
\pgfpathlineto{\pgfqpoint{2.871700in}{2.440771in}}%
\pgfpathlineto{\pgfqpoint{2.858420in}{2.447592in}}%
\pgfpathlineto{\pgfqpoint{2.845145in}{2.454459in}}%
\pgfpathlineto{\pgfqpoint{2.831872in}{2.461373in}}%
\pgfpathlineto{\pgfqpoint{2.840201in}{2.464880in}}%
\pgfpathlineto{\pgfqpoint{2.848519in}{2.468534in}}%
\pgfpathlineto{\pgfqpoint{2.856827in}{2.472331in}}%
\pgfpathlineto{\pgfqpoint{2.865124in}{2.476266in}}%
\pgfpathclose%
\pgfusepath{fill}%
\end{pgfscope}%
\begin{pgfscope}%
\pgfpathrectangle{\pgfqpoint{1.150000in}{0.150000in}}{\pgfqpoint{5.700000in}{5.700000in}}%
\pgfusepath{clip}%
\pgfsetbuttcap%
\pgfsetroundjoin%
\definecolor{currentfill}{rgb}{0.269944,0.014625,0.341379}%
\pgfsetfillcolor{currentfill}%
\pgfsetfillopacity{0.700000}%
\pgfsetlinewidth{0.000000pt}%
\definecolor{currentstroke}{rgb}{0.000000,0.000000,0.000000}%
\pgfsetstrokecolor{currentstroke}%
\pgfsetdash{}{0pt}%
\pgfpathmoveto{\pgfqpoint{3.334819in}{2.372827in}}%
\pgfpathlineto{\pgfqpoint{3.348129in}{2.367684in}}%
\pgfpathlineto{\pgfqpoint{3.361443in}{2.362578in}}%
\pgfpathlineto{\pgfqpoint{3.374761in}{2.357509in}}%
\pgfpathlineto{\pgfqpoint{3.388085in}{2.352475in}}%
\pgfpathlineto{\pgfqpoint{3.380039in}{2.346139in}}%
\pgfpathlineto{\pgfqpoint{3.371986in}{2.339871in}}%
\pgfpathlineto{\pgfqpoint{3.363925in}{2.333674in}}%
\pgfpathlineto{\pgfqpoint{3.355857in}{2.327551in}}%
\pgfpathlineto{\pgfqpoint{3.342517in}{2.332720in}}%
\pgfpathlineto{\pgfqpoint{3.329182in}{2.337924in}}%
\pgfpathlineto{\pgfqpoint{3.315852in}{2.343164in}}%
\pgfpathlineto{\pgfqpoint{3.302526in}{2.348441in}}%
\pgfpathlineto{\pgfqpoint{3.310610in}{2.354424in}}%
\pgfpathlineto{\pgfqpoint{3.318688in}{2.360485in}}%
\pgfpathlineto{\pgfqpoint{3.326757in}{2.366620in}}%
\pgfpathlineto{\pgfqpoint{3.334819in}{2.372827in}}%
\pgfpathclose%
\pgfusepath{fill}%
\end{pgfscope}%
\begin{pgfscope}%
\pgfpathrectangle{\pgfqpoint{1.150000in}{0.150000in}}{\pgfqpoint{5.700000in}{5.700000in}}%
\pgfusepath{clip}%
\pgfsetbuttcap%
\pgfsetroundjoin%
\definecolor{currentfill}{rgb}{0.282290,0.145912,0.461510}%
\pgfsetfillcolor{currentfill}%
\pgfsetfillopacity{0.700000}%
\pgfsetlinewidth{0.000000pt}%
\definecolor{currentstroke}{rgb}{0.000000,0.000000,0.000000}%
\pgfsetstrokecolor{currentstroke}%
\pgfsetdash{}{0pt}%
\pgfpathmoveto{\pgfqpoint{5.986432in}{2.605212in}}%
\pgfpathlineto{\pgfqpoint{6.000385in}{2.603469in}}%
\pgfpathlineto{\pgfqpoint{6.014347in}{2.601749in}}%
\pgfpathlineto{\pgfqpoint{6.028316in}{2.600053in}}%
\pgfpathlineto{\pgfqpoint{6.042294in}{2.598381in}}%
\pgfpathlineto{\pgfqpoint{6.035323in}{2.593160in}}%
\pgfpathlineto{\pgfqpoint{6.028349in}{2.587993in}}%
\pgfpathlineto{\pgfqpoint{6.021371in}{2.582876in}}%
\pgfpathlineto{\pgfqpoint{6.014390in}{2.577803in}}%
\pgfpathlineto{\pgfqpoint{6.000390in}{2.579303in}}%
\pgfpathlineto{\pgfqpoint{5.986398in}{2.580827in}}%
\pgfpathlineto{\pgfqpoint{5.972415in}{2.582374in}}%
\pgfpathlineto{\pgfqpoint{5.958440in}{2.583946in}}%
\pgfpathlineto{\pgfqpoint{5.965443in}{2.589186in}}%
\pgfpathlineto{\pgfqpoint{5.972443in}{2.594474in}}%
\pgfpathlineto{\pgfqpoint{5.979440in}{2.599814in}}%
\pgfpathlineto{\pgfqpoint{5.986432in}{2.605212in}}%
\pgfpathclose%
\pgfusepath{fill}%
\end{pgfscope}%
\begin{pgfscope}%
\pgfpathrectangle{\pgfqpoint{1.150000in}{0.150000in}}{\pgfqpoint{5.700000in}{5.700000in}}%
\pgfusepath{clip}%
\pgfsetbuttcap%
\pgfsetroundjoin%
\definecolor{currentfill}{rgb}{0.268510,0.009605,0.335427}%
\pgfsetfillcolor{currentfill}%
\pgfsetfillopacity{0.700000}%
\pgfsetlinewidth{0.000000pt}%
\definecolor{currentstroke}{rgb}{0.000000,0.000000,0.000000}%
\pgfsetstrokecolor{currentstroke}%
\pgfsetdash{}{0pt}%
\pgfpathmoveto{\pgfqpoint{3.473479in}{2.359146in}}%
\pgfpathlineto{\pgfqpoint{3.486812in}{2.354410in}}%
\pgfpathlineto{\pgfqpoint{3.500150in}{2.349709in}}%
\pgfpathlineto{\pgfqpoint{3.513493in}{2.345042in}}%
\pgfpathlineto{\pgfqpoint{3.526841in}{2.340409in}}%
\pgfpathlineto{\pgfqpoint{3.518853in}{2.333598in}}%
\pgfpathlineto{\pgfqpoint{3.510859in}{2.326836in}}%
\pgfpathlineto{\pgfqpoint{3.502858in}{2.320127in}}%
\pgfpathlineto{\pgfqpoint{3.494851in}{2.313474in}}%
\pgfpathlineto{\pgfqpoint{3.481487in}{2.318228in}}%
\pgfpathlineto{\pgfqpoint{3.468129in}{2.323016in}}%
\pgfpathlineto{\pgfqpoint{3.454776in}{2.327838in}}%
\pgfpathlineto{\pgfqpoint{3.441428in}{2.332696in}}%
\pgfpathlineto{\pgfqpoint{3.449451in}{2.339223in}}%
\pgfpathlineto{\pgfqpoint{3.457467in}{2.345809in}}%
\pgfpathlineto{\pgfqpoint{3.465477in}{2.352451in}}%
\pgfpathlineto{\pgfqpoint{3.473479in}{2.359146in}}%
\pgfpathclose%
\pgfusepath{fill}%
\end{pgfscope}%
\begin{pgfscope}%
\pgfpathrectangle{\pgfqpoint{1.150000in}{0.150000in}}{\pgfqpoint{5.700000in}{5.700000in}}%
\pgfusepath{clip}%
\pgfsetbuttcap%
\pgfsetroundjoin%
\definecolor{currentfill}{rgb}{0.272594,0.025563,0.353093}%
\pgfsetfillcolor{currentfill}%
\pgfsetfillopacity{0.700000}%
\pgfsetlinewidth{0.000000pt}%
\definecolor{currentstroke}{rgb}{0.000000,0.000000,0.000000}%
\pgfsetstrokecolor{currentstroke}%
\pgfsetdash{}{0pt}%
\pgfpathmoveto{\pgfqpoint{3.196083in}{2.392006in}}%
\pgfpathlineto{\pgfqpoint{3.209373in}{2.386426in}}%
\pgfpathlineto{\pgfqpoint{3.222667in}{2.380886in}}%
\pgfpathlineto{\pgfqpoint{3.235966in}{2.375384in}}%
\pgfpathlineto{\pgfqpoint{3.249269in}{2.369920in}}%
\pgfpathlineto{\pgfqpoint{3.241159in}{2.364161in}}%
\pgfpathlineto{\pgfqpoint{3.233042in}{2.358489in}}%
\pgfpathlineto{\pgfqpoint{3.224916in}{2.352909in}}%
\pgfpathlineto{\pgfqpoint{3.216782in}{2.347423in}}%
\pgfpathlineto{\pgfqpoint{3.203462in}{2.353035in}}%
\pgfpathlineto{\pgfqpoint{3.190145in}{2.358685in}}%
\pgfpathlineto{\pgfqpoint{3.176833in}{2.364373in}}%
\pgfpathlineto{\pgfqpoint{3.163526in}{2.370100in}}%
\pgfpathlineto{\pgfqpoint{3.171678in}{2.375433in}}%
\pgfpathlineto{\pgfqpoint{3.179821in}{2.380864in}}%
\pgfpathlineto{\pgfqpoint{3.187956in}{2.386389in}}%
\pgfpathlineto{\pgfqpoint{3.196083in}{2.392006in}}%
\pgfpathclose%
\pgfusepath{fill}%
\end{pgfscope}%
\begin{pgfscope}%
\pgfpathrectangle{\pgfqpoint{1.150000in}{0.150000in}}{\pgfqpoint{5.700000in}{5.700000in}}%
\pgfusepath{clip}%
\pgfsetbuttcap%
\pgfsetroundjoin%
\definecolor{currentfill}{rgb}{0.274952,0.037752,0.364543}%
\pgfsetfillcolor{currentfill}%
\pgfsetfillopacity{0.700000}%
\pgfsetlinewidth{0.000000pt}%
\definecolor{currentstroke}{rgb}{0.000000,0.000000,0.000000}%
\pgfsetstrokecolor{currentstroke}%
\pgfsetdash{}{0pt}%
\pgfpathmoveto{\pgfqpoint{4.421637in}{2.411216in}}%
\pgfpathlineto{\pgfqpoint{4.435179in}{2.408555in}}%
\pgfpathlineto{\pgfqpoint{4.448726in}{2.405920in}}%
\pgfpathlineto{\pgfqpoint{4.462281in}{2.403312in}}%
\pgfpathlineto{\pgfqpoint{4.475843in}{2.400732in}}%
\pgfpathlineto{\pgfqpoint{4.468211in}{2.393089in}}%
\pgfpathlineto{\pgfqpoint{4.460573in}{2.385412in}}%
\pgfpathlineto{\pgfqpoint{4.452930in}{2.377703in}}%
\pgfpathlineto{\pgfqpoint{4.445281in}{2.369961in}}%
\pgfpathlineto{\pgfqpoint{4.431707in}{2.372557in}}%
\pgfpathlineto{\pgfqpoint{4.418140in}{2.375179in}}%
\pgfpathlineto{\pgfqpoint{4.404580in}{2.377829in}}%
\pgfpathlineto{\pgfqpoint{4.391026in}{2.380506in}}%
\pgfpathlineto{\pgfqpoint{4.398688in}{2.388228in}}%
\pgfpathlineto{\pgfqpoint{4.406343in}{2.395920in}}%
\pgfpathlineto{\pgfqpoint{4.413993in}{2.403583in}}%
\pgfpathlineto{\pgfqpoint{4.421637in}{2.411216in}}%
\pgfpathclose%
\pgfusepath{fill}%
\end{pgfscope}%
\begin{pgfscope}%
\pgfpathrectangle{\pgfqpoint{1.150000in}{0.150000in}}{\pgfqpoint{5.700000in}{5.700000in}}%
\pgfusepath{clip}%
\pgfsetbuttcap%
\pgfsetroundjoin%
\definecolor{currentfill}{rgb}{0.271305,0.019942,0.347269}%
\pgfsetfillcolor{currentfill}%
\pgfsetfillopacity{0.700000}%
\pgfsetlinewidth{0.000000pt}%
\definecolor{currentstroke}{rgb}{0.000000,0.000000,0.000000}%
\pgfsetstrokecolor{currentstroke}%
\pgfsetdash{}{0pt}%
\pgfpathmoveto{\pgfqpoint{4.198058in}{2.383953in}}%
\pgfpathlineto{\pgfqpoint{4.211544in}{2.380913in}}%
\pgfpathlineto{\pgfqpoint{4.225037in}{2.377901in}}%
\pgfpathlineto{\pgfqpoint{4.238536in}{2.374918in}}%
\pgfpathlineto{\pgfqpoint{4.252041in}{2.371963in}}%
\pgfpathlineto{\pgfqpoint{4.244327in}{2.364169in}}%
\pgfpathlineto{\pgfqpoint{4.236607in}{2.356353in}}%
\pgfpathlineto{\pgfqpoint{4.228881in}{2.348517in}}%
\pgfpathlineto{\pgfqpoint{4.221150in}{2.340660in}}%
\pgfpathlineto{\pgfqpoint{4.207632in}{2.343657in}}%
\pgfpathlineto{\pgfqpoint{4.194121in}{2.346682in}}%
\pgfpathlineto{\pgfqpoint{4.180617in}{2.349736in}}%
\pgfpathlineto{\pgfqpoint{4.167119in}{2.352817in}}%
\pgfpathlineto{\pgfqpoint{4.174862in}{2.360627in}}%
\pgfpathlineto{\pgfqpoint{4.182600in}{2.368420in}}%
\pgfpathlineto{\pgfqpoint{4.190332in}{2.376196in}}%
\pgfpathlineto{\pgfqpoint{4.198058in}{2.383953in}}%
\pgfpathclose%
\pgfusepath{fill}%
\end{pgfscope}%
\begin{pgfscope}%
\pgfpathrectangle{\pgfqpoint{1.150000in}{0.150000in}}{\pgfqpoint{5.700000in}{5.700000in}}%
\pgfusepath{clip}%
\pgfsetbuttcap%
\pgfsetroundjoin%
\definecolor{currentfill}{rgb}{0.277941,0.056324,0.381191}%
\pgfsetfillcolor{currentfill}%
\pgfsetfillopacity{0.700000}%
\pgfsetlinewidth{0.000000pt}%
\definecolor{currentstroke}{rgb}{0.000000,0.000000,0.000000}%
\pgfsetstrokecolor{currentstroke}%
\pgfsetdash{}{0pt}%
\pgfpathmoveto{\pgfqpoint{4.645230in}{2.441009in}}%
\pgfpathlineto{\pgfqpoint{4.658830in}{2.438659in}}%
\pgfpathlineto{\pgfqpoint{4.672436in}{2.436337in}}%
\pgfpathlineto{\pgfqpoint{4.686050in}{2.434041in}}%
\pgfpathlineto{\pgfqpoint{4.699671in}{2.431771in}}%
\pgfpathlineto{\pgfqpoint{4.692124in}{2.424420in}}%
\pgfpathlineto{\pgfqpoint{4.684571in}{2.417031in}}%
\pgfpathlineto{\pgfqpoint{4.677012in}{2.409604in}}%
\pgfpathlineto{\pgfqpoint{4.669447in}{2.402137in}}%
\pgfpathlineto{\pgfqpoint{4.655814in}{2.404395in}}%
\pgfpathlineto{\pgfqpoint{4.642187in}{2.406680in}}%
\pgfpathlineto{\pgfqpoint{4.628568in}{2.408991in}}%
\pgfpathlineto{\pgfqpoint{4.614956in}{2.411328in}}%
\pgfpathlineto{\pgfqpoint{4.622533in}{2.418802in}}%
\pgfpathlineto{\pgfqpoint{4.630105in}{2.426239in}}%
\pgfpathlineto{\pgfqpoint{4.637671in}{2.433641in}}%
\pgfpathlineto{\pgfqpoint{4.645230in}{2.441009in}}%
\pgfpathclose%
\pgfusepath{fill}%
\end{pgfscope}%
\begin{pgfscope}%
\pgfpathrectangle{\pgfqpoint{1.150000in}{0.150000in}}{\pgfqpoint{5.700000in}{5.700000in}}%
\pgfusepath{clip}%
\pgfsetbuttcap%
\pgfsetroundjoin%
\definecolor{currentfill}{rgb}{0.282884,0.135920,0.453427}%
\pgfsetfillcolor{currentfill}%
\pgfsetfillopacity{0.700000}%
\pgfsetlinewidth{0.000000pt}%
\definecolor{currentstroke}{rgb}{0.000000,0.000000,0.000000}%
\pgfsetstrokecolor{currentstroke}%
\pgfsetdash{}{0pt}%
\pgfpathmoveto{\pgfqpoint{5.763067in}{2.582161in}}%
\pgfpathlineto{\pgfqpoint{5.776965in}{2.580462in}}%
\pgfpathlineto{\pgfqpoint{5.790871in}{2.578787in}}%
\pgfpathlineto{\pgfqpoint{5.804785in}{2.577136in}}%
\pgfpathlineto{\pgfqpoint{5.818708in}{2.575509in}}%
\pgfpathlineto{\pgfqpoint{5.811640in}{2.570126in}}%
\pgfpathlineto{\pgfqpoint{5.804567in}{2.564768in}}%
\pgfpathlineto{\pgfqpoint{5.797489in}{2.559430in}}%
\pgfpathlineto{\pgfqpoint{5.790406in}{2.554106in}}%
\pgfpathlineto{\pgfqpoint{5.776463in}{2.555588in}}%
\pgfpathlineto{\pgfqpoint{5.762529in}{2.557093in}}%
\pgfpathlineto{\pgfqpoint{5.748603in}{2.558623in}}%
\pgfpathlineto{\pgfqpoint{5.734685in}{2.560176in}}%
\pgfpathlineto{\pgfqpoint{5.741788in}{2.565641in}}%
\pgfpathlineto{\pgfqpoint{5.748886in}{2.571123in}}%
\pgfpathlineto{\pgfqpoint{5.755979in}{2.576628in}}%
\pgfpathlineto{\pgfqpoint{5.763067in}{2.582161in}}%
\pgfpathclose%
\pgfusepath{fill}%
\end{pgfscope}%
\begin{pgfscope}%
\pgfpathrectangle{\pgfqpoint{1.150000in}{0.150000in}}{\pgfqpoint{5.700000in}{5.700000in}}%
\pgfusepath{clip}%
\pgfsetbuttcap%
\pgfsetroundjoin%
\definecolor{currentfill}{rgb}{0.280267,0.073417,0.397163}%
\pgfsetfillcolor{currentfill}%
\pgfsetfillopacity{0.700000}%
\pgfsetlinewidth{0.000000pt}%
\definecolor{currentstroke}{rgb}{0.000000,0.000000,0.000000}%
\pgfsetstrokecolor{currentstroke}%
\pgfsetdash{}{0pt}%
\pgfpathmoveto{\pgfqpoint{4.868844in}{2.471627in}}%
\pgfpathlineto{\pgfqpoint{4.882503in}{2.469528in}}%
\pgfpathlineto{\pgfqpoint{4.896170in}{2.467455in}}%
\pgfpathlineto{\pgfqpoint{4.909844in}{2.465407in}}%
\pgfpathlineto{\pgfqpoint{4.923526in}{2.463385in}}%
\pgfpathlineto{\pgfqpoint{4.916068in}{2.456423in}}%
\pgfpathlineto{\pgfqpoint{4.908604in}{2.449424in}}%
\pgfpathlineto{\pgfqpoint{4.901133in}{2.442387in}}%
\pgfpathlineto{\pgfqpoint{4.893657in}{2.435308in}}%
\pgfpathlineto{\pgfqpoint{4.879961in}{2.437292in}}%
\pgfpathlineto{\pgfqpoint{4.866274in}{2.439301in}}%
\pgfpathlineto{\pgfqpoint{4.852593in}{2.441336in}}%
\pgfpathlineto{\pgfqpoint{4.838920in}{2.443397in}}%
\pgfpathlineto{\pgfqpoint{4.846410in}{2.450509in}}%
\pgfpathlineto{\pgfqpoint{4.853894in}{2.457583in}}%
\pgfpathlineto{\pgfqpoint{4.861372in}{2.464622in}}%
\pgfpathlineto{\pgfqpoint{4.868844in}{2.471627in}}%
\pgfpathclose%
\pgfusepath{fill}%
\end{pgfscope}%
\begin{pgfscope}%
\pgfpathrectangle{\pgfqpoint{1.150000in}{0.150000in}}{\pgfqpoint{5.700000in}{5.700000in}}%
\pgfusepath{clip}%
\pgfsetbuttcap%
\pgfsetroundjoin%
\definecolor{currentfill}{rgb}{0.268510,0.009605,0.335427}%
\pgfsetfillcolor{currentfill}%
\pgfsetfillopacity{0.700000}%
\pgfsetlinewidth{0.000000pt}%
\definecolor{currentstroke}{rgb}{0.000000,0.000000,0.000000}%
\pgfsetstrokecolor{currentstroke}%
\pgfsetdash{}{0pt}%
\pgfpathmoveto{\pgfqpoint{3.974457in}{2.361319in}}%
\pgfpathlineto{\pgfqpoint{3.987892in}{2.357833in}}%
\pgfpathlineto{\pgfqpoint{4.001333in}{2.354377in}}%
\pgfpathlineto{\pgfqpoint{4.014781in}{2.350950in}}%
\pgfpathlineto{\pgfqpoint{4.028234in}{2.347553in}}%
\pgfpathlineto{\pgfqpoint{4.020438in}{2.339798in}}%
\pgfpathlineto{\pgfqpoint{4.012636in}{2.332038in}}%
\pgfpathlineto{\pgfqpoint{4.004828in}{2.324274in}}%
\pgfpathlineto{\pgfqpoint{3.997015in}{2.316509in}}%
\pgfpathlineto{\pgfqpoint{3.983549in}{2.319975in}}%
\pgfpathlineto{\pgfqpoint{3.970089in}{2.323469in}}%
\pgfpathlineto{\pgfqpoint{3.956635in}{2.326994in}}%
\pgfpathlineto{\pgfqpoint{3.943187in}{2.330548in}}%
\pgfpathlineto{\pgfqpoint{3.951013in}{2.338240in}}%
\pgfpathlineto{\pgfqpoint{3.958833in}{2.345934in}}%
\pgfpathlineto{\pgfqpoint{3.966648in}{2.353627in}}%
\pgfpathlineto{\pgfqpoint{3.974457in}{2.361319in}}%
\pgfpathclose%
\pgfusepath{fill}%
\end{pgfscope}%
\begin{pgfscope}%
\pgfpathrectangle{\pgfqpoint{1.150000in}{0.150000in}}{\pgfqpoint{5.700000in}{5.700000in}}%
\pgfusepath{clip}%
\pgfsetbuttcap%
\pgfsetroundjoin%
\definecolor{currentfill}{rgb}{0.267004,0.004874,0.329415}%
\pgfsetfillcolor{currentfill}%
\pgfsetfillopacity{0.700000}%
\pgfsetlinewidth{0.000000pt}%
\definecolor{currentstroke}{rgb}{0.000000,0.000000,0.000000}%
\pgfsetstrokecolor{currentstroke}%
\pgfsetdash{}{0pt}%
\pgfpathmoveto{\pgfqpoint{3.612111in}{2.350340in}}%
\pgfpathlineto{\pgfqpoint{3.625471in}{2.345983in}}%
\pgfpathlineto{\pgfqpoint{3.638837in}{2.341658in}}%
\pgfpathlineto{\pgfqpoint{3.652207in}{2.337367in}}%
\pgfpathlineto{\pgfqpoint{3.665584in}{2.333108in}}%
\pgfpathlineto{\pgfqpoint{3.657651in}{2.325917in}}%
\pgfpathlineto{\pgfqpoint{3.649712in}{2.318758in}}%
\pgfpathlineto{\pgfqpoint{3.641767in}{2.311636in}}%
\pgfpathlineto{\pgfqpoint{3.633815in}{2.304551in}}%
\pgfpathlineto{\pgfqpoint{3.620424in}{2.308918in}}%
\pgfpathlineto{\pgfqpoint{3.607039in}{2.313318in}}%
\pgfpathlineto{\pgfqpoint{3.593660in}{2.317750in}}%
\pgfpathlineto{\pgfqpoint{3.580285in}{2.322215in}}%
\pgfpathlineto{\pgfqpoint{3.588252in}{2.329187in}}%
\pgfpathlineto{\pgfqpoint{3.596211in}{2.336200in}}%
\pgfpathlineto{\pgfqpoint{3.604164in}{2.343251in}}%
\pgfpathlineto{\pgfqpoint{3.612111in}{2.350340in}}%
\pgfpathclose%
\pgfusepath{fill}%
\end{pgfscope}%
\begin{pgfscope}%
\pgfpathrectangle{\pgfqpoint{1.150000in}{0.150000in}}{\pgfqpoint{5.700000in}{5.700000in}}%
\pgfusepath{clip}%
\pgfsetbuttcap%
\pgfsetroundjoin%
\definecolor{currentfill}{rgb}{0.283229,0.120777,0.440584}%
\pgfsetfillcolor{currentfill}%
\pgfsetfillopacity{0.700000}%
\pgfsetlinewidth{0.000000pt}%
\definecolor{currentstroke}{rgb}{0.000000,0.000000,0.000000}%
\pgfsetstrokecolor{currentstroke}%
\pgfsetdash{}{0pt}%
\pgfpathmoveto{\pgfqpoint{5.539605in}{2.557353in}}%
\pgfpathlineto{\pgfqpoint{5.553445in}{2.555642in}}%
\pgfpathlineto{\pgfqpoint{5.567294in}{2.553955in}}%
\pgfpathlineto{\pgfqpoint{5.581150in}{2.552292in}}%
\pgfpathlineto{\pgfqpoint{5.595014in}{2.550653in}}%
\pgfpathlineto{\pgfqpoint{5.587846in}{2.544968in}}%
\pgfpathlineto{\pgfqpoint{5.580672in}{2.539283in}}%
\pgfpathlineto{\pgfqpoint{5.573492in}{2.533593in}}%
\pgfpathlineto{\pgfqpoint{5.566307in}{2.527896in}}%
\pgfpathlineto{\pgfqpoint{5.552424in}{2.529416in}}%
\pgfpathlineto{\pgfqpoint{5.538550in}{2.530960in}}%
\pgfpathlineto{\pgfqpoint{5.524683in}{2.532528in}}%
\pgfpathlineto{\pgfqpoint{5.510825in}{2.534121in}}%
\pgfpathlineto{\pgfqpoint{5.518028in}{2.539932in}}%
\pgfpathlineto{\pgfqpoint{5.525226in}{2.545738in}}%
\pgfpathlineto{\pgfqpoint{5.532418in}{2.551544in}}%
\pgfpathlineto{\pgfqpoint{5.539605in}{2.557353in}}%
\pgfpathclose%
\pgfusepath{fill}%
\end{pgfscope}%
\begin{pgfscope}%
\pgfpathrectangle{\pgfqpoint{1.150000in}{0.150000in}}{\pgfqpoint{5.700000in}{5.700000in}}%
\pgfusepath{clip}%
\pgfsetbuttcap%
\pgfsetroundjoin%
\definecolor{currentfill}{rgb}{0.282327,0.094955,0.417331}%
\pgfsetfillcolor{currentfill}%
\pgfsetfillopacity{0.700000}%
\pgfsetlinewidth{0.000000pt}%
\definecolor{currentstroke}{rgb}{0.000000,0.000000,0.000000}%
\pgfsetstrokecolor{currentstroke}%
\pgfsetdash{}{0pt}%
\pgfpathmoveto{\pgfqpoint{5.092463in}{2.501766in}}%
\pgfpathlineto{\pgfqpoint{5.106184in}{2.499856in}}%
\pgfpathlineto{\pgfqpoint{5.119911in}{2.497971in}}%
\pgfpathlineto{\pgfqpoint{5.133647in}{2.496110in}}%
\pgfpathlineto{\pgfqpoint{5.147390in}{2.494275in}}%
\pgfpathlineto{\pgfqpoint{5.140025in}{2.487753in}}%
\pgfpathlineto{\pgfqpoint{5.132654in}{2.481199in}}%
\pgfpathlineto{\pgfqpoint{5.125277in}{2.474612in}}%
\pgfpathlineto{\pgfqpoint{5.117893in}{2.467990in}}%
\pgfpathlineto{\pgfqpoint{5.104135in}{2.469760in}}%
\pgfpathlineto{\pgfqpoint{5.090384in}{2.471555in}}%
\pgfpathlineto{\pgfqpoint{5.076641in}{2.473375in}}%
\pgfpathlineto{\pgfqpoint{5.062906in}{2.475220in}}%
\pgfpathlineto{\pgfqpoint{5.070305in}{2.481903in}}%
\pgfpathlineto{\pgfqpoint{5.077697in}{2.488553in}}%
\pgfpathlineto{\pgfqpoint{5.085083in}{2.495173in}}%
\pgfpathlineto{\pgfqpoint{5.092463in}{2.501766in}}%
\pgfpathclose%
\pgfusepath{fill}%
\end{pgfscope}%
\begin{pgfscope}%
\pgfpathrectangle{\pgfqpoint{1.150000in}{0.150000in}}{\pgfqpoint{5.700000in}{5.700000in}}%
\pgfusepath{clip}%
\pgfsetbuttcap%
\pgfsetroundjoin%
\definecolor{currentfill}{rgb}{0.276022,0.044167,0.370164}%
\pgfsetfillcolor{currentfill}%
\pgfsetfillopacity{0.700000}%
\pgfsetlinewidth{0.000000pt}%
\definecolor{currentstroke}{rgb}{0.000000,0.000000,0.000000}%
\pgfsetstrokecolor{currentstroke}%
\pgfsetdash{}{0pt}%
\pgfpathmoveto{\pgfqpoint{3.057214in}{2.417353in}}%
\pgfpathlineto{\pgfqpoint{3.070489in}{2.411304in}}%
\pgfpathlineto{\pgfqpoint{3.083768in}{2.405296in}}%
\pgfpathlineto{\pgfqpoint{3.097051in}{2.399329in}}%
\pgfpathlineto{\pgfqpoint{3.110338in}{2.393403in}}%
\pgfpathlineto{\pgfqpoint{3.102159in}{2.388329in}}%
\pgfpathlineto{\pgfqpoint{3.093971in}{2.383363in}}%
\pgfpathlineto{\pgfqpoint{3.085774in}{2.378510in}}%
\pgfpathlineto{\pgfqpoint{3.077568in}{2.373774in}}%
\pgfpathlineto{\pgfqpoint{3.064262in}{2.379861in}}%
\pgfpathlineto{\pgfqpoint{3.050960in}{2.385989in}}%
\pgfpathlineto{\pgfqpoint{3.037662in}{2.392158in}}%
\pgfpathlineto{\pgfqpoint{3.024368in}{2.398369in}}%
\pgfpathlineto{\pgfqpoint{3.032593in}{2.402939in}}%
\pgfpathlineto{\pgfqpoint{3.040810in}{2.407629in}}%
\pgfpathlineto{\pgfqpoint{3.049017in}{2.412435in}}%
\pgfpathlineto{\pgfqpoint{3.057214in}{2.417353in}}%
\pgfpathclose%
\pgfusepath{fill}%
\end{pgfscope}%
\begin{pgfscope}%
\pgfpathrectangle{\pgfqpoint{1.150000in}{0.150000in}}{\pgfqpoint{5.700000in}{5.700000in}}%
\pgfusepath{clip}%
\pgfsetbuttcap%
\pgfsetroundjoin%
\definecolor{currentfill}{rgb}{0.283091,0.110553,0.431554}%
\pgfsetfillcolor{currentfill}%
\pgfsetfillopacity{0.700000}%
\pgfsetlinewidth{0.000000pt}%
\definecolor{currentstroke}{rgb}{0.000000,0.000000,0.000000}%
\pgfsetstrokecolor{currentstroke}%
\pgfsetdash{}{0pt}%
\pgfpathmoveto{\pgfqpoint{5.316061in}{2.530514in}}%
\pgfpathlineto{\pgfqpoint{5.329842in}{2.528733in}}%
\pgfpathlineto{\pgfqpoint{5.343630in}{2.526976in}}%
\pgfpathlineto{\pgfqpoint{5.357427in}{2.525244in}}%
\pgfpathlineto{\pgfqpoint{5.371231in}{2.523536in}}%
\pgfpathlineto{\pgfqpoint{5.363963in}{2.517455in}}%
\pgfpathlineto{\pgfqpoint{5.356689in}{2.511355in}}%
\pgfpathlineto{\pgfqpoint{5.349409in}{2.505233in}}%
\pgfpathlineto{\pgfqpoint{5.342123in}{2.499087in}}%
\pgfpathlineto{\pgfqpoint{5.328302in}{2.500703in}}%
\pgfpathlineto{\pgfqpoint{5.314489in}{2.502343in}}%
\pgfpathlineto{\pgfqpoint{5.300684in}{2.504008in}}%
\pgfpathlineto{\pgfqpoint{5.286887in}{2.505697in}}%
\pgfpathlineto{\pgfqpoint{5.294189in}{2.511930in}}%
\pgfpathlineto{\pgfqpoint{5.301486in}{2.518142in}}%
\pgfpathlineto{\pgfqpoint{5.308776in}{2.524336in}}%
\pgfpathlineto{\pgfqpoint{5.316061in}{2.530514in}}%
\pgfpathclose%
\pgfusepath{fill}%
\end{pgfscope}%
\begin{pgfscope}%
\pgfpathrectangle{\pgfqpoint{1.150000in}{0.150000in}}{\pgfqpoint{5.700000in}{5.700000in}}%
\pgfusepath{clip}%
\pgfsetbuttcap%
\pgfsetroundjoin%
\definecolor{currentfill}{rgb}{0.282656,0.100196,0.422160}%
\pgfsetfillcolor{currentfill}%
\pgfsetfillopacity{0.700000}%
\pgfsetlinewidth{0.000000pt}%
\definecolor{currentstroke}{rgb}{0.000000,0.000000,0.000000}%
\pgfsetstrokecolor{currentstroke}%
\pgfsetdash{}{0pt}%
\pgfpathmoveto{\pgfqpoint{2.725801in}{2.518397in}}%
\pgfpathlineto{\pgfqpoint{2.739049in}{2.511097in}}%
\pgfpathlineto{\pgfqpoint{2.752301in}{2.503848in}}%
\pgfpathlineto{\pgfqpoint{2.765555in}{2.496648in}}%
\pgfpathlineto{\pgfqpoint{2.778813in}{2.489496in}}%
\pgfpathlineto{\pgfqpoint{2.770450in}{2.486325in}}%
\pgfpathlineto{\pgfqpoint{2.762075in}{2.483313in}}%
\pgfpathlineto{\pgfqpoint{2.753689in}{2.480465in}}%
\pgfpathlineto{\pgfqpoint{2.745291in}{2.477786in}}%
\pgfpathlineto{\pgfqpoint{2.732011in}{2.485127in}}%
\pgfpathlineto{\pgfqpoint{2.718733in}{2.492516in}}%
\pgfpathlineto{\pgfqpoint{2.705458in}{2.499955in}}%
\pgfpathlineto{\pgfqpoint{2.692186in}{2.507444in}}%
\pgfpathlineto{\pgfqpoint{2.700608in}{2.509929in}}%
\pgfpathlineto{\pgfqpoint{2.709017in}{2.512585in}}%
\pgfpathlineto{\pgfqpoint{2.717415in}{2.515410in}}%
\pgfpathlineto{\pgfqpoint{2.725801in}{2.518397in}}%
\pgfpathclose%
\pgfusepath{fill}%
\end{pgfscope}%
\begin{pgfscope}%
\pgfpathrectangle{\pgfqpoint{1.150000in}{0.150000in}}{\pgfqpoint{5.700000in}{5.700000in}}%
\pgfusepath{clip}%
\pgfsetbuttcap%
\pgfsetroundjoin%
\definecolor{currentfill}{rgb}{0.267004,0.004874,0.329415}%
\pgfsetfillcolor{currentfill}%
\pgfsetfillopacity{0.700000}%
\pgfsetlinewidth{0.000000pt}%
\definecolor{currentstroke}{rgb}{0.000000,0.000000,0.000000}%
\pgfsetstrokecolor{currentstroke}%
\pgfsetdash{}{0pt}%
\pgfpathmoveto{\pgfqpoint{3.750758in}{2.345827in}}%
\pgfpathlineto{\pgfqpoint{3.764149in}{2.341823in}}%
\pgfpathlineto{\pgfqpoint{3.777545in}{2.337850in}}%
\pgfpathlineto{\pgfqpoint{3.790947in}{2.333909in}}%
\pgfpathlineto{\pgfqpoint{3.804354in}{2.329999in}}%
\pgfpathlineto{\pgfqpoint{3.796473in}{2.322517in}}%
\pgfpathlineto{\pgfqpoint{3.788586in}{2.315053in}}%
\pgfpathlineto{\pgfqpoint{3.780693in}{2.307609in}}%
\pgfpathlineto{\pgfqpoint{3.772794in}{2.300188in}}%
\pgfpathlineto{\pgfqpoint{3.759373in}{2.304192in}}%
\pgfpathlineto{\pgfqpoint{3.745957in}{2.308228in}}%
\pgfpathlineto{\pgfqpoint{3.732548in}{2.312295in}}%
\pgfpathlineto{\pgfqpoint{3.719144in}{2.316394in}}%
\pgfpathlineto{\pgfqpoint{3.727057in}{2.323716in}}%
\pgfpathlineto{\pgfqpoint{3.734963in}{2.331064in}}%
\pgfpathlineto{\pgfqpoint{3.742864in}{2.338435in}}%
\pgfpathlineto{\pgfqpoint{3.750758in}{2.345827in}}%
\pgfpathclose%
\pgfusepath{fill}%
\end{pgfscope}%
\begin{pgfscope}%
\pgfpathrectangle{\pgfqpoint{1.150000in}{0.150000in}}{\pgfqpoint{5.700000in}{5.700000in}}%
\pgfusepath{clip}%
\pgfsetbuttcap%
\pgfsetroundjoin%
\definecolor{currentfill}{rgb}{0.279566,0.067836,0.391917}%
\pgfsetfillcolor{currentfill}%
\pgfsetfillopacity{0.700000}%
\pgfsetlinewidth{0.000000pt}%
\definecolor{currentstroke}{rgb}{0.000000,0.000000,0.000000}%
\pgfsetstrokecolor{currentstroke}%
\pgfsetdash{}{0pt}%
\pgfpathmoveto{\pgfqpoint{2.918149in}{2.449590in}}%
\pgfpathlineto{\pgfqpoint{2.931413in}{2.443035in}}%
\pgfpathlineto{\pgfqpoint{2.944682in}{2.436524in}}%
\pgfpathlineto{\pgfqpoint{2.957954in}{2.430057in}}%
\pgfpathlineto{\pgfqpoint{2.971229in}{2.423633in}}%
\pgfpathlineto{\pgfqpoint{2.962974in}{2.419358in}}%
\pgfpathlineto{\pgfqpoint{2.954708in}{2.415213in}}%
\pgfpathlineto{\pgfqpoint{2.946432in}{2.411205in}}%
\pgfpathlineto{\pgfqpoint{2.938147in}{2.407338in}}%
\pgfpathlineto{\pgfqpoint{2.924850in}{2.413936in}}%
\pgfpathlineto{\pgfqpoint{2.911557in}{2.420578in}}%
\pgfpathlineto{\pgfqpoint{2.898268in}{2.427265in}}%
\pgfpathlineto{\pgfqpoint{2.884982in}{2.433995in}}%
\pgfpathlineto{\pgfqpoint{2.893290in}{2.437683in}}%
\pgfpathlineto{\pgfqpoint{2.901586in}{2.441514in}}%
\pgfpathlineto{\pgfqpoint{2.909873in}{2.445485in}}%
\pgfpathlineto{\pgfqpoint{2.918149in}{2.449590in}}%
\pgfpathclose%
\pgfusepath{fill}%
\end{pgfscope}%
\begin{pgfscope}%
\pgfpathrectangle{\pgfqpoint{1.150000in}{0.150000in}}{\pgfqpoint{5.700000in}{5.700000in}}%
\pgfusepath{clip}%
\pgfsetbuttcap%
\pgfsetroundjoin%
\definecolor{currentfill}{rgb}{0.273809,0.031497,0.358853}%
\pgfsetfillcolor{currentfill}%
\pgfsetfillopacity{0.700000}%
\pgfsetlinewidth{0.000000pt}%
\definecolor{currentstroke}{rgb}{0.000000,0.000000,0.000000}%
\pgfsetstrokecolor{currentstroke}%
\pgfsetdash{}{0pt}%
\pgfpathmoveto{\pgfqpoint{4.336880in}{2.391489in}}%
\pgfpathlineto{\pgfqpoint{4.350407in}{2.388702in}}%
\pgfpathlineto{\pgfqpoint{4.363940in}{2.385943in}}%
\pgfpathlineto{\pgfqpoint{4.377480in}{2.383211in}}%
\pgfpathlineto{\pgfqpoint{4.391026in}{2.380506in}}%
\pgfpathlineto{\pgfqpoint{4.383359in}{2.372755in}}%
\pgfpathlineto{\pgfqpoint{4.375686in}{2.364974in}}%
\pgfpathlineto{\pgfqpoint{4.368008in}{2.357164in}}%
\pgfpathlineto{\pgfqpoint{4.360323in}{2.349326in}}%
\pgfpathlineto{\pgfqpoint{4.346765in}{2.352058in}}%
\pgfpathlineto{\pgfqpoint{4.333213in}{2.354819in}}%
\pgfpathlineto{\pgfqpoint{4.319667in}{2.357607in}}%
\pgfpathlineto{\pgfqpoint{4.306129in}{2.360422in}}%
\pgfpathlineto{\pgfqpoint{4.313825in}{2.368228in}}%
\pgfpathlineto{\pgfqpoint{4.321516in}{2.376007in}}%
\pgfpathlineto{\pgfqpoint{4.329201in}{2.383761in}}%
\pgfpathlineto{\pgfqpoint{4.336880in}{2.391489in}}%
\pgfpathclose%
\pgfusepath{fill}%
\end{pgfscope}%
\begin{pgfscope}%
\pgfpathrectangle{\pgfqpoint{1.150000in}{0.150000in}}{\pgfqpoint{5.700000in}{5.700000in}}%
\pgfusepath{clip}%
\pgfsetbuttcap%
\pgfsetroundjoin%
\definecolor{currentfill}{rgb}{0.277018,0.050344,0.375715}%
\pgfsetfillcolor{currentfill}%
\pgfsetfillopacity{0.700000}%
\pgfsetlinewidth{0.000000pt}%
\definecolor{currentstroke}{rgb}{0.000000,0.000000,0.000000}%
\pgfsetstrokecolor{currentstroke}%
\pgfsetdash{}{0pt}%
\pgfpathmoveto{\pgfqpoint{4.560578in}{2.420944in}}%
\pgfpathlineto{\pgfqpoint{4.574162in}{2.418500in}}%
\pgfpathlineto{\pgfqpoint{4.587753in}{2.416083in}}%
\pgfpathlineto{\pgfqpoint{4.601351in}{2.413692in}}%
\pgfpathlineto{\pgfqpoint{4.614956in}{2.411328in}}%
\pgfpathlineto{\pgfqpoint{4.607372in}{2.403818in}}%
\pgfpathlineto{\pgfqpoint{4.599783in}{2.396270in}}%
\pgfpathlineto{\pgfqpoint{4.592188in}{2.388683in}}%
\pgfpathlineto{\pgfqpoint{4.584587in}{2.381058in}}%
\pgfpathlineto{\pgfqpoint{4.570969in}{2.383424in}}%
\pgfpathlineto{\pgfqpoint{4.557358in}{2.385816in}}%
\pgfpathlineto{\pgfqpoint{4.543755in}{2.388235in}}%
\pgfpathlineto{\pgfqpoint{4.530159in}{2.390681in}}%
\pgfpathlineto{\pgfqpoint{4.537772in}{2.398299in}}%
\pgfpathlineto{\pgfqpoint{4.545380in}{2.405882in}}%
\pgfpathlineto{\pgfqpoint{4.552982in}{2.413430in}}%
\pgfpathlineto{\pgfqpoint{4.560578in}{2.420944in}}%
\pgfpathclose%
\pgfusepath{fill}%
\end{pgfscope}%
\begin{pgfscope}%
\pgfpathrectangle{\pgfqpoint{1.150000in}{0.150000in}}{\pgfqpoint{5.700000in}{5.700000in}}%
\pgfusepath{clip}%
\pgfsetbuttcap%
\pgfsetroundjoin%
\definecolor{currentfill}{rgb}{0.269944,0.014625,0.341379}%
\pgfsetfillcolor{currentfill}%
\pgfsetfillopacity{0.700000}%
\pgfsetlinewidth{0.000000pt}%
\definecolor{currentstroke}{rgb}{0.000000,0.000000,0.000000}%
\pgfsetstrokecolor{currentstroke}%
\pgfsetdash{}{0pt}%
\pgfpathmoveto{\pgfqpoint{4.113190in}{2.365430in}}%
\pgfpathlineto{\pgfqpoint{4.126663in}{2.362234in}}%
\pgfpathlineto{\pgfqpoint{4.140142in}{2.359066in}}%
\pgfpathlineto{\pgfqpoint{4.153627in}{2.355928in}}%
\pgfpathlineto{\pgfqpoint{4.167119in}{2.352817in}}%
\pgfpathlineto{\pgfqpoint{4.159370in}{2.344991in}}%
\pgfpathlineto{\pgfqpoint{4.151615in}{2.337150in}}%
\pgfpathlineto{\pgfqpoint{4.143854in}{2.329294in}}%
\pgfpathlineto{\pgfqpoint{4.136088in}{2.321424in}}%
\pgfpathlineto{\pgfqpoint{4.122585in}{2.324589in}}%
\pgfpathlineto{\pgfqpoint{4.109087in}{2.327783in}}%
\pgfpathlineto{\pgfqpoint{4.095596in}{2.331006in}}%
\pgfpathlineto{\pgfqpoint{4.082111in}{2.334257in}}%
\pgfpathlineto{\pgfqpoint{4.089889in}{2.342067in}}%
\pgfpathlineto{\pgfqpoint{4.097662in}{2.349866in}}%
\pgfpathlineto{\pgfqpoint{4.105429in}{2.357654in}}%
\pgfpathlineto{\pgfqpoint{4.113190in}{2.365430in}}%
\pgfpathclose%
\pgfusepath{fill}%
\end{pgfscope}%
\begin{pgfscope}%
\pgfpathrectangle{\pgfqpoint{1.150000in}{0.150000in}}{\pgfqpoint{5.700000in}{5.700000in}}%
\pgfusepath{clip}%
\pgfsetbuttcap%
\pgfsetroundjoin%
\definecolor{currentfill}{rgb}{0.282290,0.145912,0.461510}%
\pgfsetfillcolor{currentfill}%
\pgfsetfillopacity{0.700000}%
\pgfsetlinewidth{0.000000pt}%
\definecolor{currentstroke}{rgb}{0.000000,0.000000,0.000000}%
\pgfsetstrokecolor{currentstroke}%
\pgfsetdash{}{0pt}%
\pgfpathmoveto{\pgfqpoint{5.902622in}{2.590469in}}%
\pgfpathlineto{\pgfqpoint{5.916564in}{2.588802in}}%
\pgfpathlineto{\pgfqpoint{5.930514in}{2.587160in}}%
\pgfpathlineto{\pgfqpoint{5.944473in}{2.585541in}}%
\pgfpathlineto{\pgfqpoint{5.958440in}{2.583946in}}%
\pgfpathlineto{\pgfqpoint{5.951432in}{2.578747in}}%
\pgfpathlineto{\pgfqpoint{5.944420in}{2.573585in}}%
\pgfpathlineto{\pgfqpoint{5.937403in}{2.568455in}}%
\pgfpathlineto{\pgfqpoint{5.930381in}{2.563352in}}%
\pgfpathlineto{\pgfqpoint{5.916393in}{2.564788in}}%
\pgfpathlineto{\pgfqpoint{5.902413in}{2.566248in}}%
\pgfpathlineto{\pgfqpoint{5.888442in}{2.567732in}}%
\pgfpathlineto{\pgfqpoint{5.874478in}{2.569239in}}%
\pgfpathlineto{\pgfqpoint{5.881521in}{2.574496in}}%
\pgfpathlineto{\pgfqpoint{5.888559in}{2.579784in}}%
\pgfpathlineto{\pgfqpoint{5.895593in}{2.585106in}}%
\pgfpathlineto{\pgfqpoint{5.902622in}{2.590469in}}%
\pgfpathclose%
\pgfusepath{fill}%
\end{pgfscope}%
\begin{pgfscope}%
\pgfpathrectangle{\pgfqpoint{1.150000in}{0.150000in}}{\pgfqpoint{5.700000in}{5.700000in}}%
\pgfusepath{clip}%
\pgfsetbuttcap%
\pgfsetroundjoin%
\definecolor{currentfill}{rgb}{0.279566,0.067836,0.391917}%
\pgfsetfillcolor{currentfill}%
\pgfsetfillopacity{0.700000}%
\pgfsetlinewidth{0.000000pt}%
\definecolor{currentstroke}{rgb}{0.000000,0.000000,0.000000}%
\pgfsetstrokecolor{currentstroke}%
\pgfsetdash{}{0pt}%
\pgfpathmoveto{\pgfqpoint{4.784301in}{2.451899in}}%
\pgfpathlineto{\pgfqpoint{4.797945in}{2.449734in}}%
\pgfpathlineto{\pgfqpoint{4.811596in}{2.447596in}}%
\pgfpathlineto{\pgfqpoint{4.825254in}{2.445484in}}%
\pgfpathlineto{\pgfqpoint{4.838920in}{2.443397in}}%
\pgfpathlineto{\pgfqpoint{4.831424in}{2.436247in}}%
\pgfpathlineto{\pgfqpoint{4.823921in}{2.429057in}}%
\pgfpathlineto{\pgfqpoint{4.816413in}{2.421825in}}%
\pgfpathlineto{\pgfqpoint{4.808898in}{2.414551in}}%
\pgfpathlineto{\pgfqpoint{4.795219in}{2.416613in}}%
\pgfpathlineto{\pgfqpoint{4.781547in}{2.418700in}}%
\pgfpathlineto{\pgfqpoint{4.767883in}{2.420813in}}%
\pgfpathlineto{\pgfqpoint{4.754226in}{2.422953in}}%
\pgfpathlineto{\pgfqpoint{4.761754in}{2.430247in}}%
\pgfpathlineto{\pgfqpoint{4.769276in}{2.437502in}}%
\pgfpathlineto{\pgfqpoint{4.776791in}{2.444718in}}%
\pgfpathlineto{\pgfqpoint{4.784301in}{2.451899in}}%
\pgfpathclose%
\pgfusepath{fill}%
\end{pgfscope}%
\begin{pgfscope}%
\pgfpathrectangle{\pgfqpoint{1.150000in}{0.150000in}}{\pgfqpoint{5.700000in}{5.700000in}}%
\pgfusepath{clip}%
\pgfsetbuttcap%
\pgfsetroundjoin%
\definecolor{currentfill}{rgb}{0.281924,0.089666,0.412415}%
\pgfsetfillcolor{currentfill}%
\pgfsetfillopacity{0.700000}%
\pgfsetlinewidth{0.000000pt}%
\definecolor{currentstroke}{rgb}{0.000000,0.000000,0.000000}%
\pgfsetstrokecolor{currentstroke}%
\pgfsetdash{}{0pt}%
\pgfpathmoveto{\pgfqpoint{5.008042in}{2.482854in}}%
\pgfpathlineto{\pgfqpoint{5.021746in}{2.480907in}}%
\pgfpathlineto{\pgfqpoint{5.035459in}{2.478986in}}%
\pgfpathlineto{\pgfqpoint{5.049179in}{2.477091in}}%
\pgfpathlineto{\pgfqpoint{5.062906in}{2.475220in}}%
\pgfpathlineto{\pgfqpoint{5.055502in}{2.468503in}}%
\pgfpathlineto{\pgfqpoint{5.048091in}{2.461750in}}%
\pgfpathlineto{\pgfqpoint{5.040673in}{2.454958in}}%
\pgfpathlineto{\pgfqpoint{5.033250in}{2.448126in}}%
\pgfpathlineto{\pgfqpoint{5.019508in}{2.449944in}}%
\pgfpathlineto{\pgfqpoint{5.005773in}{2.451788in}}%
\pgfpathlineto{\pgfqpoint{4.992047in}{2.453657in}}%
\pgfpathlineto{\pgfqpoint{4.978327in}{2.455552in}}%
\pgfpathlineto{\pgfqpoint{4.985765in}{2.462431in}}%
\pgfpathlineto{\pgfqpoint{4.993197in}{2.469273in}}%
\pgfpathlineto{\pgfqpoint{5.000622in}{2.476080in}}%
\pgfpathlineto{\pgfqpoint{5.008042in}{2.482854in}}%
\pgfpathclose%
\pgfusepath{fill}%
\end{pgfscope}%
\begin{pgfscope}%
\pgfpathrectangle{\pgfqpoint{1.150000in}{0.150000in}}{\pgfqpoint{5.700000in}{5.700000in}}%
\pgfusepath{clip}%
\pgfsetbuttcap%
\pgfsetroundjoin%
\definecolor{currentfill}{rgb}{0.283072,0.130895,0.449241}%
\pgfsetfillcolor{currentfill}%
\pgfsetfillopacity{0.700000}%
\pgfsetlinewidth{0.000000pt}%
\definecolor{currentstroke}{rgb}{0.000000,0.000000,0.000000}%
\pgfsetstrokecolor{currentstroke}%
\pgfsetdash{}{0pt}%
\pgfpathmoveto{\pgfqpoint{5.679094in}{2.566631in}}%
\pgfpathlineto{\pgfqpoint{5.692980in}{2.564981in}}%
\pgfpathlineto{\pgfqpoint{5.706874in}{2.563356in}}%
\pgfpathlineto{\pgfqpoint{5.720775in}{2.561754in}}%
\pgfpathlineto{\pgfqpoint{5.734685in}{2.560176in}}%
\pgfpathlineto{\pgfqpoint{5.727577in}{2.554726in}}%
\pgfpathlineto{\pgfqpoint{5.720463in}{2.549285in}}%
\pgfpathlineto{\pgfqpoint{5.713344in}{2.543850in}}%
\pgfpathlineto{\pgfqpoint{5.706219in}{2.538416in}}%
\pgfpathlineto{\pgfqpoint{5.692290in}{2.539861in}}%
\pgfpathlineto{\pgfqpoint{5.678369in}{2.541330in}}%
\pgfpathlineto{\pgfqpoint{5.664456in}{2.542823in}}%
\pgfpathlineto{\pgfqpoint{5.650552in}{2.544341in}}%
\pgfpathlineto{\pgfqpoint{5.657696in}{2.549903in}}%
\pgfpathlineto{\pgfqpoint{5.664834in}{2.555469in}}%
\pgfpathlineto{\pgfqpoint{5.671967in}{2.561044in}}%
\pgfpathlineto{\pgfqpoint{5.679094in}{2.566631in}}%
\pgfpathclose%
\pgfusepath{fill}%
\end{pgfscope}%
\begin{pgfscope}%
\pgfpathrectangle{\pgfqpoint{1.150000in}{0.150000in}}{\pgfqpoint{5.700000in}{5.700000in}}%
\pgfusepath{clip}%
\pgfsetbuttcap%
\pgfsetroundjoin%
\definecolor{currentfill}{rgb}{0.267004,0.004874,0.329415}%
\pgfsetfillcolor{currentfill}%
\pgfsetfillopacity{0.700000}%
\pgfsetlinewidth{0.000000pt}%
\definecolor{currentstroke}{rgb}{0.000000,0.000000,0.000000}%
\pgfsetstrokecolor{currentstroke}%
\pgfsetdash{}{0pt}%
\pgfpathmoveto{\pgfqpoint{3.889456in}{2.345066in}}%
\pgfpathlineto{\pgfqpoint{3.902880in}{2.341391in}}%
\pgfpathlineto{\pgfqpoint{3.916309in}{2.337747in}}%
\pgfpathlineto{\pgfqpoint{3.929745in}{2.334133in}}%
\pgfpathlineto{\pgfqpoint{3.943187in}{2.330548in}}%
\pgfpathlineto{\pgfqpoint{3.935355in}{2.322859in}}%
\pgfpathlineto{\pgfqpoint{3.927518in}{2.315174in}}%
\pgfpathlineto{\pgfqpoint{3.919674in}{2.307496in}}%
\pgfpathlineto{\pgfqpoint{3.911825in}{2.299825in}}%
\pgfpathlineto{\pgfqpoint{3.898370in}{2.303490in}}%
\pgfpathlineto{\pgfqpoint{3.884922in}{2.307186in}}%
\pgfpathlineto{\pgfqpoint{3.871479in}{2.310912in}}%
\pgfpathlineto{\pgfqpoint{3.858042in}{2.314668in}}%
\pgfpathlineto{\pgfqpoint{3.865905in}{2.322253in}}%
\pgfpathlineto{\pgfqpoint{3.873761in}{2.329849in}}%
\pgfpathlineto{\pgfqpoint{3.881611in}{2.337454in}}%
\pgfpathlineto{\pgfqpoint{3.889456in}{2.345066in}}%
\pgfpathclose%
\pgfusepath{fill}%
\end{pgfscope}%
\begin{pgfscope}%
\pgfpathrectangle{\pgfqpoint{1.150000in}{0.150000in}}{\pgfqpoint{5.700000in}{5.700000in}}%
\pgfusepath{clip}%
\pgfsetbuttcap%
\pgfsetroundjoin%
\definecolor{currentfill}{rgb}{0.282910,0.105393,0.426902}%
\pgfsetfillcolor{currentfill}%
\pgfsetfillopacity{0.700000}%
\pgfsetlinewidth{0.000000pt}%
\definecolor{currentstroke}{rgb}{0.000000,0.000000,0.000000}%
\pgfsetstrokecolor{currentstroke}%
\pgfsetdash{}{0pt}%
\pgfpathmoveto{\pgfqpoint{5.231776in}{2.512703in}}%
\pgfpathlineto{\pgfqpoint{5.245542in}{2.510915in}}%
\pgfpathlineto{\pgfqpoint{5.259316in}{2.509151in}}%
\pgfpathlineto{\pgfqpoint{5.273097in}{2.507412in}}%
\pgfpathlineto{\pgfqpoint{5.286887in}{2.505697in}}%
\pgfpathlineto{\pgfqpoint{5.279578in}{2.499440in}}%
\pgfpathlineto{\pgfqpoint{5.272263in}{2.493155in}}%
\pgfpathlineto{\pgfqpoint{5.264942in}{2.486841in}}%
\pgfpathlineto{\pgfqpoint{5.257614in}{2.480494in}}%
\pgfpathlineto{\pgfqpoint{5.243809in}{2.482130in}}%
\pgfpathlineto{\pgfqpoint{5.230011in}{2.483790in}}%
\pgfpathlineto{\pgfqpoint{5.216222in}{2.485475in}}%
\pgfpathlineto{\pgfqpoint{5.202440in}{2.487185in}}%
\pgfpathlineto{\pgfqpoint{5.209783in}{2.493606in}}%
\pgfpathlineto{\pgfqpoint{5.217120in}{2.499997in}}%
\pgfpathlineto{\pgfqpoint{5.224451in}{2.506362in}}%
\pgfpathlineto{\pgfqpoint{5.231776in}{2.512703in}}%
\pgfpathclose%
\pgfusepath{fill}%
\end{pgfscope}%
\begin{pgfscope}%
\pgfpathrectangle{\pgfqpoint{1.150000in}{0.150000in}}{\pgfqpoint{5.700000in}{5.700000in}}%
\pgfusepath{clip}%
\pgfsetbuttcap%
\pgfsetroundjoin%
\definecolor{currentfill}{rgb}{0.283229,0.120777,0.440584}%
\pgfsetfillcolor{currentfill}%
\pgfsetfillopacity{0.700000}%
\pgfsetlinewidth{0.000000pt}%
\definecolor{currentstroke}{rgb}{0.000000,0.000000,0.000000}%
\pgfsetstrokecolor{currentstroke}%
\pgfsetdash{}{0pt}%
\pgfpathmoveto{\pgfqpoint{5.455471in}{2.540735in}}%
\pgfpathlineto{\pgfqpoint{5.469298in}{2.539045in}}%
\pgfpathlineto{\pgfqpoint{5.483132in}{2.537379in}}%
\pgfpathlineto{\pgfqpoint{5.496974in}{2.535738in}}%
\pgfpathlineto{\pgfqpoint{5.510825in}{2.534121in}}%
\pgfpathlineto{\pgfqpoint{5.503616in}{2.528301in}}%
\pgfpathlineto{\pgfqpoint{5.496400in}{2.522470in}}%
\pgfpathlineto{\pgfqpoint{5.489179in}{2.516624in}}%
\pgfpathlineto{\pgfqpoint{5.481951in}{2.510759in}}%
\pgfpathlineto{\pgfqpoint{5.468083in}{2.512271in}}%
\pgfpathlineto{\pgfqpoint{5.454223in}{2.513806in}}%
\pgfpathlineto{\pgfqpoint{5.440371in}{2.515367in}}%
\pgfpathlineto{\pgfqpoint{5.426527in}{2.516951in}}%
\pgfpathlineto{\pgfqpoint{5.433772in}{2.522917in}}%
\pgfpathlineto{\pgfqpoint{5.441011in}{2.528867in}}%
\pgfpathlineto{\pgfqpoint{5.448244in}{2.534805in}}%
\pgfpathlineto{\pgfqpoint{5.455471in}{2.540735in}}%
\pgfpathclose%
\pgfusepath{fill}%
\end{pgfscope}%
\begin{pgfscope}%
\pgfpathrectangle{\pgfqpoint{1.150000in}{0.150000in}}{\pgfqpoint{5.700000in}{5.700000in}}%
\pgfusepath{clip}%
\pgfsetbuttcap%
\pgfsetroundjoin%
\definecolor{currentfill}{rgb}{0.268510,0.009605,0.335427}%
\pgfsetfillcolor{currentfill}%
\pgfsetfillopacity{0.700000}%
\pgfsetlinewidth{0.000000pt}%
\definecolor{currentstroke}{rgb}{0.000000,0.000000,0.000000}%
\pgfsetstrokecolor{currentstroke}%
\pgfsetdash{}{0pt}%
\pgfpathmoveto{\pgfqpoint{3.388085in}{2.352475in}}%
\pgfpathlineto{\pgfqpoint{3.401413in}{2.347477in}}%
\pgfpathlineto{\pgfqpoint{3.414747in}{2.342515in}}%
\pgfpathlineto{\pgfqpoint{3.428085in}{2.337588in}}%
\pgfpathlineto{\pgfqpoint{3.441428in}{2.332696in}}%
\pgfpathlineto{\pgfqpoint{3.433398in}{2.326230in}}%
\pgfpathlineto{\pgfqpoint{3.425360in}{2.319829in}}%
\pgfpathlineto{\pgfqpoint{3.417316in}{2.313496in}}%
\pgfpathlineto{\pgfqpoint{3.409264in}{2.307235in}}%
\pgfpathlineto{\pgfqpoint{3.395905in}{2.312261in}}%
\pgfpathlineto{\pgfqpoint{3.382551in}{2.317322in}}%
\pgfpathlineto{\pgfqpoint{3.369201in}{2.322419in}}%
\pgfpathlineto{\pgfqpoint{3.355857in}{2.327551in}}%
\pgfpathlineto{\pgfqpoint{3.363925in}{2.333674in}}%
\pgfpathlineto{\pgfqpoint{3.371986in}{2.339871in}}%
\pgfpathlineto{\pgfqpoint{3.380039in}{2.346139in}}%
\pgfpathlineto{\pgfqpoint{3.388085in}{2.352475in}}%
\pgfpathclose%
\pgfusepath{fill}%
\end{pgfscope}%
\begin{pgfscope}%
\pgfpathrectangle{\pgfqpoint{1.150000in}{0.150000in}}{\pgfqpoint{5.700000in}{5.700000in}}%
\pgfusepath{clip}%
\pgfsetbuttcap%
\pgfsetroundjoin%
\definecolor{currentfill}{rgb}{0.271305,0.019942,0.347269}%
\pgfsetfillcolor{currentfill}%
\pgfsetfillopacity{0.700000}%
\pgfsetlinewidth{0.000000pt}%
\definecolor{currentstroke}{rgb}{0.000000,0.000000,0.000000}%
\pgfsetstrokecolor{currentstroke}%
\pgfsetdash{}{0pt}%
\pgfpathmoveto{\pgfqpoint{3.249269in}{2.369920in}}%
\pgfpathlineto{\pgfqpoint{3.262576in}{2.364494in}}%
\pgfpathlineto{\pgfqpoint{3.275888in}{2.359106in}}%
\pgfpathlineto{\pgfqpoint{3.289205in}{2.353755in}}%
\pgfpathlineto{\pgfqpoint{3.302526in}{2.348441in}}%
\pgfpathlineto{\pgfqpoint{3.294433in}{2.342539in}}%
\pgfpathlineto{\pgfqpoint{3.286333in}{2.336721in}}%
\pgfpathlineto{\pgfqpoint{3.278225in}{2.330992in}}%
\pgfpathlineto{\pgfqpoint{3.270109in}{2.325353in}}%
\pgfpathlineto{\pgfqpoint{3.256770in}{2.330815in}}%
\pgfpathlineto{\pgfqpoint{3.243437in}{2.336313in}}%
\pgfpathlineto{\pgfqpoint{3.230107in}{2.341849in}}%
\pgfpathlineto{\pgfqpoint{3.216782in}{2.347423in}}%
\pgfpathlineto{\pgfqpoint{3.224916in}{2.352909in}}%
\pgfpathlineto{\pgfqpoint{3.233042in}{2.358489in}}%
\pgfpathlineto{\pgfqpoint{3.241159in}{2.364161in}}%
\pgfpathlineto{\pgfqpoint{3.249269in}{2.369920in}}%
\pgfpathclose%
\pgfusepath{fill}%
\end{pgfscope}%
\begin{pgfscope}%
\pgfpathrectangle{\pgfqpoint{1.150000in}{0.150000in}}{\pgfqpoint{5.700000in}{5.700000in}}%
\pgfusepath{clip}%
\pgfsetbuttcap%
\pgfsetroundjoin%
\definecolor{currentfill}{rgb}{0.267004,0.004874,0.329415}%
\pgfsetfillcolor{currentfill}%
\pgfsetfillopacity{0.700000}%
\pgfsetlinewidth{0.000000pt}%
\definecolor{currentstroke}{rgb}{0.000000,0.000000,0.000000}%
\pgfsetstrokecolor{currentstroke}%
\pgfsetdash{}{0pt}%
\pgfpathmoveto{\pgfqpoint{3.526841in}{2.340409in}}%
\pgfpathlineto{\pgfqpoint{3.540194in}{2.335810in}}%
\pgfpathlineto{\pgfqpoint{3.553553in}{2.331245in}}%
\pgfpathlineto{\pgfqpoint{3.566916in}{2.326713in}}%
\pgfpathlineto{\pgfqpoint{3.580285in}{2.322215in}}%
\pgfpathlineto{\pgfqpoint{3.572313in}{2.315287in}}%
\pgfpathlineto{\pgfqpoint{3.564333in}{2.308406in}}%
\pgfpathlineto{\pgfqpoint{3.556348in}{2.301575in}}%
\pgfpathlineto{\pgfqpoint{3.548355in}{2.294795in}}%
\pgfpathlineto{\pgfqpoint{3.534971in}{2.299415in}}%
\pgfpathlineto{\pgfqpoint{3.521592in}{2.304068in}}%
\pgfpathlineto{\pgfqpoint{3.508219in}{2.308754in}}%
\pgfpathlineto{\pgfqpoint{3.494851in}{2.313474in}}%
\pgfpathlineto{\pgfqpoint{3.502858in}{2.320127in}}%
\pgfpathlineto{\pgfqpoint{3.510859in}{2.326836in}}%
\pgfpathlineto{\pgfqpoint{3.518853in}{2.333598in}}%
\pgfpathlineto{\pgfqpoint{3.526841in}{2.340409in}}%
\pgfpathclose%
\pgfusepath{fill}%
\end{pgfscope}%
\begin{pgfscope}%
\pgfpathrectangle{\pgfqpoint{1.150000in}{0.150000in}}{\pgfqpoint{5.700000in}{5.700000in}}%
\pgfusepath{clip}%
\pgfsetbuttcap%
\pgfsetroundjoin%
\definecolor{currentfill}{rgb}{0.274952,0.037752,0.364543}%
\pgfsetfillcolor{currentfill}%
\pgfsetfillopacity{0.700000}%
\pgfsetlinewidth{0.000000pt}%
\definecolor{currentstroke}{rgb}{0.000000,0.000000,0.000000}%
\pgfsetstrokecolor{currentstroke}%
\pgfsetdash{}{0pt}%
\pgfpathmoveto{\pgfqpoint{3.110338in}{2.393403in}}%
\pgfpathlineto{\pgfqpoint{3.123628in}{2.387518in}}%
\pgfpathlineto{\pgfqpoint{3.136923in}{2.381672in}}%
\pgfpathlineto{\pgfqpoint{3.150223in}{2.375867in}}%
\pgfpathlineto{\pgfqpoint{3.163526in}{2.370100in}}%
\pgfpathlineto{\pgfqpoint{3.155365in}{2.364869in}}%
\pgfpathlineto{\pgfqpoint{3.147196in}{2.359744in}}%
\pgfpathlineto{\pgfqpoint{3.139018in}{2.354728in}}%
\pgfpathlineto{\pgfqpoint{3.130832in}{2.349825in}}%
\pgfpathlineto{\pgfqpoint{3.117510in}{2.355753in}}%
\pgfpathlineto{\pgfqpoint{3.104192in}{2.361720in}}%
\pgfpathlineto{\pgfqpoint{3.090878in}{2.367727in}}%
\pgfpathlineto{\pgfqpoint{3.077568in}{2.373774in}}%
\pgfpathlineto{\pgfqpoint{3.085774in}{2.378510in}}%
\pgfpathlineto{\pgfqpoint{3.093971in}{2.383363in}}%
\pgfpathlineto{\pgfqpoint{3.102159in}{2.388329in}}%
\pgfpathlineto{\pgfqpoint{3.110338in}{2.393403in}}%
\pgfpathclose%
\pgfusepath{fill}%
\end{pgfscope}%
\begin{pgfscope}%
\pgfpathrectangle{\pgfqpoint{1.150000in}{0.150000in}}{\pgfqpoint{5.700000in}{5.700000in}}%
\pgfusepath{clip}%
\pgfsetbuttcap%
\pgfsetroundjoin%
\definecolor{currentfill}{rgb}{0.282327,0.094955,0.417331}%
\pgfsetfillcolor{currentfill}%
\pgfsetfillopacity{0.700000}%
\pgfsetlinewidth{0.000000pt}%
\definecolor{currentstroke}{rgb}{0.000000,0.000000,0.000000}%
\pgfsetstrokecolor{currentstroke}%
\pgfsetdash{}{0pt}%
\pgfpathmoveto{\pgfqpoint{2.778813in}{2.489496in}}%
\pgfpathlineto{\pgfqpoint{2.792073in}{2.482394in}}%
\pgfpathlineto{\pgfqpoint{2.805336in}{2.475339in}}%
\pgfpathlineto{\pgfqpoint{2.818602in}{2.468332in}}%
\pgfpathlineto{\pgfqpoint{2.831872in}{2.461373in}}%
\pgfpathlineto{\pgfqpoint{2.823532in}{2.458017in}}%
\pgfpathlineto{\pgfqpoint{2.815180in}{2.454817in}}%
\pgfpathlineto{\pgfqpoint{2.806817in}{2.451779in}}%
\pgfpathlineto{\pgfqpoint{2.798442in}{2.448906in}}%
\pgfpathlineto{\pgfqpoint{2.785150in}{2.456054in}}%
\pgfpathlineto{\pgfqpoint{2.771861in}{2.463251in}}%
\pgfpathlineto{\pgfqpoint{2.758574in}{2.470494in}}%
\pgfpathlineto{\pgfqpoint{2.745291in}{2.477786in}}%
\pgfpathlineto{\pgfqpoint{2.753689in}{2.480465in}}%
\pgfpathlineto{\pgfqpoint{2.762075in}{2.483313in}}%
\pgfpathlineto{\pgfqpoint{2.770450in}{2.486325in}}%
\pgfpathlineto{\pgfqpoint{2.778813in}{2.489496in}}%
\pgfpathclose%
\pgfusepath{fill}%
\end{pgfscope}%
\begin{pgfscope}%
\pgfpathrectangle{\pgfqpoint{1.150000in}{0.150000in}}{\pgfqpoint{5.700000in}{5.700000in}}%
\pgfusepath{clip}%
\pgfsetbuttcap%
\pgfsetroundjoin%
\definecolor{currentfill}{rgb}{0.272594,0.025563,0.353093}%
\pgfsetfillcolor{currentfill}%
\pgfsetfillopacity{0.700000}%
\pgfsetlinewidth{0.000000pt}%
\definecolor{currentstroke}{rgb}{0.000000,0.000000,0.000000}%
\pgfsetstrokecolor{currentstroke}%
\pgfsetdash{}{0pt}%
\pgfpathmoveto{\pgfqpoint{4.252041in}{2.371963in}}%
\pgfpathlineto{\pgfqpoint{4.265553in}{2.369036in}}%
\pgfpathlineto{\pgfqpoint{4.279072in}{2.366137in}}%
\pgfpathlineto{\pgfqpoint{4.292597in}{2.363266in}}%
\pgfpathlineto{\pgfqpoint{4.306129in}{2.360422in}}%
\pgfpathlineto{\pgfqpoint{4.298427in}{2.352592in}}%
\pgfpathlineto{\pgfqpoint{4.290719in}{2.344736in}}%
\pgfpathlineto{\pgfqpoint{4.283005in}{2.336857in}}%
\pgfpathlineto{\pgfqpoint{4.275286in}{2.328953in}}%
\pgfpathlineto{\pgfqpoint{4.261742in}{2.331838in}}%
\pgfpathlineto{\pgfqpoint{4.248205in}{2.334751in}}%
\pgfpathlineto{\pgfqpoint{4.234674in}{2.337692in}}%
\pgfpathlineto{\pgfqpoint{4.221150in}{2.340660in}}%
\pgfpathlineto{\pgfqpoint{4.228881in}{2.348517in}}%
\pgfpathlineto{\pgfqpoint{4.236607in}{2.356353in}}%
\pgfpathlineto{\pgfqpoint{4.244327in}{2.364169in}}%
\pgfpathlineto{\pgfqpoint{4.252041in}{2.371963in}}%
\pgfpathclose%
\pgfusepath{fill}%
\end{pgfscope}%
\begin{pgfscope}%
\pgfpathrectangle{\pgfqpoint{1.150000in}{0.150000in}}{\pgfqpoint{5.700000in}{5.700000in}}%
\pgfusepath{clip}%
\pgfsetbuttcap%
\pgfsetroundjoin%
\definecolor{currentfill}{rgb}{0.276022,0.044167,0.370164}%
\pgfsetfillcolor{currentfill}%
\pgfsetfillopacity{0.700000}%
\pgfsetlinewidth{0.000000pt}%
\definecolor{currentstroke}{rgb}{0.000000,0.000000,0.000000}%
\pgfsetstrokecolor{currentstroke}%
\pgfsetdash{}{0pt}%
\pgfpathmoveto{\pgfqpoint{4.475843in}{2.400732in}}%
\pgfpathlineto{\pgfqpoint{4.489411in}{2.398179in}}%
\pgfpathlineto{\pgfqpoint{4.502987in}{2.395653in}}%
\pgfpathlineto{\pgfqpoint{4.516569in}{2.393153in}}%
\pgfpathlineto{\pgfqpoint{4.530159in}{2.390681in}}%
\pgfpathlineto{\pgfqpoint{4.522539in}{2.383027in}}%
\pgfpathlineto{\pgfqpoint{4.514914in}{2.375338in}}%
\pgfpathlineto{\pgfqpoint{4.507283in}{2.367612in}}%
\pgfpathlineto{\pgfqpoint{4.499646in}{2.359850in}}%
\pgfpathlineto{\pgfqpoint{4.486044in}{2.362337in}}%
\pgfpathlineto{\pgfqpoint{4.472449in}{2.364852in}}%
\pgfpathlineto{\pgfqpoint{4.458862in}{2.367393in}}%
\pgfpathlineto{\pgfqpoint{4.445281in}{2.369961in}}%
\pgfpathlineto{\pgfqpoint{4.452930in}{2.377703in}}%
\pgfpathlineto{\pgfqpoint{4.460573in}{2.385412in}}%
\pgfpathlineto{\pgfqpoint{4.468211in}{2.393089in}}%
\pgfpathlineto{\pgfqpoint{4.475843in}{2.400732in}}%
\pgfpathclose%
\pgfusepath{fill}%
\end{pgfscope}%
\begin{pgfscope}%
\pgfpathrectangle{\pgfqpoint{1.150000in}{0.150000in}}{\pgfqpoint{5.700000in}{5.700000in}}%
\pgfusepath{clip}%
\pgfsetbuttcap%
\pgfsetroundjoin%
\definecolor{currentfill}{rgb}{0.267004,0.004874,0.329415}%
\pgfsetfillcolor{currentfill}%
\pgfsetfillopacity{0.700000}%
\pgfsetlinewidth{0.000000pt}%
\definecolor{currentstroke}{rgb}{0.000000,0.000000,0.000000}%
\pgfsetstrokecolor{currentstroke}%
\pgfsetdash{}{0pt}%
\pgfpathmoveto{\pgfqpoint{3.665584in}{2.333108in}}%
\pgfpathlineto{\pgfqpoint{3.678965in}{2.328881in}}%
\pgfpathlineto{\pgfqpoint{3.692353in}{2.324687in}}%
\pgfpathlineto{\pgfqpoint{3.705745in}{2.320525in}}%
\pgfpathlineto{\pgfqpoint{3.719144in}{2.316394in}}%
\pgfpathlineto{\pgfqpoint{3.711225in}{2.309100in}}%
\pgfpathlineto{\pgfqpoint{3.703300in}{2.301836in}}%
\pgfpathlineto{\pgfqpoint{3.695369in}{2.294603in}}%
\pgfpathlineto{\pgfqpoint{3.687431in}{2.287406in}}%
\pgfpathlineto{\pgfqpoint{3.674019in}{2.291645in}}%
\pgfpathlineto{\pgfqpoint{3.660612in}{2.295915in}}%
\pgfpathlineto{\pgfqpoint{3.647211in}{2.300217in}}%
\pgfpathlineto{\pgfqpoint{3.633815in}{2.304551in}}%
\pgfpathlineto{\pgfqpoint{3.641767in}{2.311636in}}%
\pgfpathlineto{\pgfqpoint{3.649712in}{2.318758in}}%
\pgfpathlineto{\pgfqpoint{3.657651in}{2.325917in}}%
\pgfpathlineto{\pgfqpoint{3.665584in}{2.333108in}}%
\pgfpathclose%
\pgfusepath{fill}%
\end{pgfscope}%
\begin{pgfscope}%
\pgfpathrectangle{\pgfqpoint{1.150000in}{0.150000in}}{\pgfqpoint{5.700000in}{5.700000in}}%
\pgfusepath{clip}%
\pgfsetbuttcap%
\pgfsetroundjoin%
\definecolor{currentfill}{rgb}{0.281887,0.150881,0.465405}%
\pgfsetfillcolor{currentfill}%
\pgfsetfillopacity{0.700000}%
\pgfsetlinewidth{0.000000pt}%
\definecolor{currentstroke}{rgb}{0.000000,0.000000,0.000000}%
\pgfsetstrokecolor{currentstroke}%
\pgfsetdash{}{0pt}%
\pgfpathmoveto{\pgfqpoint{6.042294in}{2.598381in}}%
\pgfpathlineto{\pgfqpoint{6.056279in}{2.596732in}}%
\pgfpathlineto{\pgfqpoint{6.070274in}{2.595107in}}%
\pgfpathlineto{\pgfqpoint{6.084276in}{2.593506in}}%
\pgfpathlineto{\pgfqpoint{6.098287in}{2.591928in}}%
\pgfpathlineto{\pgfqpoint{6.091339in}{2.586884in}}%
\pgfpathlineto{\pgfqpoint{6.084388in}{2.581892in}}%
\pgfpathlineto{\pgfqpoint{6.077433in}{2.576945in}}%
\pgfpathlineto{\pgfqpoint{6.070473in}{2.572039in}}%
\pgfpathlineto{\pgfqpoint{6.056440in}{2.573444in}}%
\pgfpathlineto{\pgfqpoint{6.042415in}{2.574873in}}%
\pgfpathlineto{\pgfqpoint{6.028398in}{2.576326in}}%
\pgfpathlineto{\pgfqpoint{6.014390in}{2.577803in}}%
\pgfpathlineto{\pgfqpoint{6.021371in}{2.582876in}}%
\pgfpathlineto{\pgfqpoint{6.028349in}{2.587993in}}%
\pgfpathlineto{\pgfqpoint{6.035323in}{2.593160in}}%
\pgfpathlineto{\pgfqpoint{6.042294in}{2.598381in}}%
\pgfpathclose%
\pgfusepath{fill}%
\end{pgfscope}%
\begin{pgfscope}%
\pgfpathrectangle{\pgfqpoint{1.150000in}{0.150000in}}{\pgfqpoint{5.700000in}{5.700000in}}%
\pgfusepath{clip}%
\pgfsetbuttcap%
\pgfsetroundjoin%
\definecolor{currentfill}{rgb}{0.268510,0.009605,0.335427}%
\pgfsetfillcolor{currentfill}%
\pgfsetfillopacity{0.700000}%
\pgfsetlinewidth{0.000000pt}%
\definecolor{currentstroke}{rgb}{0.000000,0.000000,0.000000}%
\pgfsetstrokecolor{currentstroke}%
\pgfsetdash{}{0pt}%
\pgfpathmoveto{\pgfqpoint{4.028234in}{2.347553in}}%
\pgfpathlineto{\pgfqpoint{4.041694in}{2.344185in}}%
\pgfpathlineto{\pgfqpoint{4.055160in}{2.340847in}}%
\pgfpathlineto{\pgfqpoint{4.068632in}{2.337537in}}%
\pgfpathlineto{\pgfqpoint{4.082111in}{2.334257in}}%
\pgfpathlineto{\pgfqpoint{4.074327in}{2.326438in}}%
\pgfpathlineto{\pgfqpoint{4.066537in}{2.318612in}}%
\pgfpathlineto{\pgfqpoint{4.058742in}{2.310779in}}%
\pgfpathlineto{\pgfqpoint{4.050941in}{2.302940in}}%
\pgfpathlineto{\pgfqpoint{4.037450in}{2.306289in}}%
\pgfpathlineto{\pgfqpoint{4.023966in}{2.309666in}}%
\pgfpathlineto{\pgfqpoint{4.010487in}{2.313073in}}%
\pgfpathlineto{\pgfqpoint{3.997015in}{2.316509in}}%
\pgfpathlineto{\pgfqpoint{4.004828in}{2.324274in}}%
\pgfpathlineto{\pgfqpoint{4.012636in}{2.332038in}}%
\pgfpathlineto{\pgfqpoint{4.020438in}{2.339798in}}%
\pgfpathlineto{\pgfqpoint{4.028234in}{2.347553in}}%
\pgfpathclose%
\pgfusepath{fill}%
\end{pgfscope}%
\begin{pgfscope}%
\pgfpathrectangle{\pgfqpoint{1.150000in}{0.150000in}}{\pgfqpoint{5.700000in}{5.700000in}}%
\pgfusepath{clip}%
\pgfsetbuttcap%
\pgfsetroundjoin%
\definecolor{currentfill}{rgb}{0.278791,0.062145,0.386592}%
\pgfsetfillcolor{currentfill}%
\pgfsetfillopacity{0.700000}%
\pgfsetlinewidth{0.000000pt}%
\definecolor{currentstroke}{rgb}{0.000000,0.000000,0.000000}%
\pgfsetstrokecolor{currentstroke}%
\pgfsetdash{}{0pt}%
\pgfpathmoveto{\pgfqpoint{4.699671in}{2.431771in}}%
\pgfpathlineto{\pgfqpoint{4.713299in}{2.429527in}}%
\pgfpathlineto{\pgfqpoint{4.726934in}{2.427309in}}%
\pgfpathlineto{\pgfqpoint{4.740576in}{2.425118in}}%
\pgfpathlineto{\pgfqpoint{4.754226in}{2.422953in}}%
\pgfpathlineto{\pgfqpoint{4.746692in}{2.415619in}}%
\pgfpathlineto{\pgfqpoint{4.739152in}{2.408243in}}%
\pgfpathlineto{\pgfqpoint{4.731606in}{2.400826in}}%
\pgfpathlineto{\pgfqpoint{4.724054in}{2.393366in}}%
\pgfpathlineto{\pgfqpoint{4.710391in}{2.395519in}}%
\pgfpathlineto{\pgfqpoint{4.696736in}{2.397699in}}%
\pgfpathlineto{\pgfqpoint{4.683088in}{2.399905in}}%
\pgfpathlineto{\pgfqpoint{4.669447in}{2.402137in}}%
\pgfpathlineto{\pgfqpoint{4.677012in}{2.409604in}}%
\pgfpathlineto{\pgfqpoint{4.684571in}{2.417031in}}%
\pgfpathlineto{\pgfqpoint{4.692124in}{2.424420in}}%
\pgfpathlineto{\pgfqpoint{4.699671in}{2.431771in}}%
\pgfpathclose%
\pgfusepath{fill}%
\end{pgfscope}%
\begin{pgfscope}%
\pgfpathrectangle{\pgfqpoint{1.150000in}{0.150000in}}{\pgfqpoint{5.700000in}{5.700000in}}%
\pgfusepath{clip}%
\pgfsetbuttcap%
\pgfsetroundjoin%
\definecolor{currentfill}{rgb}{0.281446,0.084320,0.407414}%
\pgfsetfillcolor{currentfill}%
\pgfsetfillopacity{0.700000}%
\pgfsetlinewidth{0.000000pt}%
\definecolor{currentstroke}{rgb}{0.000000,0.000000,0.000000}%
\pgfsetstrokecolor{currentstroke}%
\pgfsetdash{}{0pt}%
\pgfpathmoveto{\pgfqpoint{4.923526in}{2.463385in}}%
\pgfpathlineto{\pgfqpoint{4.937215in}{2.461388in}}%
\pgfpathlineto{\pgfqpoint{4.950912in}{2.459417in}}%
\pgfpathlineto{\pgfqpoint{4.964616in}{2.457472in}}%
\pgfpathlineto{\pgfqpoint{4.978327in}{2.455552in}}%
\pgfpathlineto{\pgfqpoint{4.970883in}{2.448634in}}%
\pgfpathlineto{\pgfqpoint{4.963433in}{2.441675in}}%
\pgfpathlineto{\pgfqpoint{4.955977in}{2.434674in}}%
\pgfpathlineto{\pgfqpoint{4.948514in}{2.427630in}}%
\pgfpathlineto{\pgfqpoint{4.934788in}{2.429511in}}%
\pgfpathlineto{\pgfqpoint{4.921070in}{2.431418in}}%
\pgfpathlineto{\pgfqpoint{4.907360in}{2.433350in}}%
\pgfpathlineto{\pgfqpoint{4.893657in}{2.435308in}}%
\pgfpathlineto{\pgfqpoint{4.901133in}{2.442387in}}%
\pgfpathlineto{\pgfqpoint{4.908604in}{2.449424in}}%
\pgfpathlineto{\pgfqpoint{4.916068in}{2.456423in}}%
\pgfpathlineto{\pgfqpoint{4.923526in}{2.463385in}}%
\pgfpathclose%
\pgfusepath{fill}%
\end{pgfscope}%
\begin{pgfscope}%
\pgfpathrectangle{\pgfqpoint{1.150000in}{0.150000in}}{\pgfqpoint{5.700000in}{5.700000in}}%
\pgfusepath{clip}%
\pgfsetbuttcap%
\pgfsetroundjoin%
\definecolor{currentfill}{rgb}{0.282623,0.140926,0.457517}%
\pgfsetfillcolor{currentfill}%
\pgfsetfillopacity{0.700000}%
\pgfsetlinewidth{0.000000pt}%
\definecolor{currentstroke}{rgb}{0.000000,0.000000,0.000000}%
\pgfsetstrokecolor{currentstroke}%
\pgfsetdash{}{0pt}%
\pgfpathmoveto{\pgfqpoint{5.818708in}{2.575509in}}%
\pgfpathlineto{\pgfqpoint{5.832638in}{2.573905in}}%
\pgfpathlineto{\pgfqpoint{5.846577in}{2.572326in}}%
\pgfpathlineto{\pgfqpoint{5.860523in}{2.570771in}}%
\pgfpathlineto{\pgfqpoint{5.874478in}{2.569239in}}%
\pgfpathlineto{\pgfqpoint{5.867431in}{2.564008in}}%
\pgfpathlineto{\pgfqpoint{5.860379in}{2.558797in}}%
\pgfpathlineto{\pgfqpoint{5.853321in}{2.553603in}}%
\pgfpathlineto{\pgfqpoint{5.846259in}{2.548421in}}%
\pgfpathlineto{\pgfqpoint{5.832283in}{2.549806in}}%
\pgfpathlineto{\pgfqpoint{5.818316in}{2.551215in}}%
\pgfpathlineto{\pgfqpoint{5.804356in}{2.552649in}}%
\pgfpathlineto{\pgfqpoint{5.790406in}{2.554106in}}%
\pgfpathlineto{\pgfqpoint{5.797489in}{2.559430in}}%
\pgfpathlineto{\pgfqpoint{5.804567in}{2.564768in}}%
\pgfpathlineto{\pgfqpoint{5.811640in}{2.570126in}}%
\pgfpathlineto{\pgfqpoint{5.818708in}{2.575509in}}%
\pgfpathclose%
\pgfusepath{fill}%
\end{pgfscope}%
\begin{pgfscope}%
\pgfpathrectangle{\pgfqpoint{1.150000in}{0.150000in}}{\pgfqpoint{5.700000in}{5.700000in}}%
\pgfusepath{clip}%
\pgfsetbuttcap%
\pgfsetroundjoin%
\definecolor{currentfill}{rgb}{0.277941,0.056324,0.381191}%
\pgfsetfillcolor{currentfill}%
\pgfsetfillopacity{0.700000}%
\pgfsetlinewidth{0.000000pt}%
\definecolor{currentstroke}{rgb}{0.000000,0.000000,0.000000}%
\pgfsetstrokecolor{currentstroke}%
\pgfsetdash{}{0pt}%
\pgfpathmoveto{\pgfqpoint{2.971229in}{2.423633in}}%
\pgfpathlineto{\pgfqpoint{2.984508in}{2.417253in}}%
\pgfpathlineto{\pgfqpoint{2.997791in}{2.410916in}}%
\pgfpathlineto{\pgfqpoint{3.011077in}{2.404621in}}%
\pgfpathlineto{\pgfqpoint{3.024368in}{2.398369in}}%
\pgfpathlineto{\pgfqpoint{3.016132in}{2.393923in}}%
\pgfpathlineto{\pgfqpoint{3.007887in}{2.389606in}}%
\pgfpathlineto{\pgfqpoint{2.999633in}{2.385421in}}%
\pgfpathlineto{\pgfqpoint{2.991368in}{2.381373in}}%
\pgfpathlineto{\pgfqpoint{2.978057in}{2.387801in}}%
\pgfpathlineto{\pgfqpoint{2.964750in}{2.394270in}}%
\pgfpathlineto{\pgfqpoint{2.951446in}{2.400782in}}%
\pgfpathlineto{\pgfqpoint{2.938147in}{2.407338in}}%
\pgfpathlineto{\pgfqpoint{2.946432in}{2.411205in}}%
\pgfpathlineto{\pgfqpoint{2.954708in}{2.415213in}}%
\pgfpathlineto{\pgfqpoint{2.962974in}{2.419358in}}%
\pgfpathlineto{\pgfqpoint{2.971229in}{2.423633in}}%
\pgfpathclose%
\pgfusepath{fill}%
\end{pgfscope}%
\begin{pgfscope}%
\pgfpathrectangle{\pgfqpoint{1.150000in}{0.150000in}}{\pgfqpoint{5.700000in}{5.700000in}}%
\pgfusepath{clip}%
\pgfsetbuttcap%
\pgfsetroundjoin%
\definecolor{currentfill}{rgb}{0.282656,0.100196,0.422160}%
\pgfsetfillcolor{currentfill}%
\pgfsetfillopacity{0.700000}%
\pgfsetlinewidth{0.000000pt}%
\definecolor{currentstroke}{rgb}{0.000000,0.000000,0.000000}%
\pgfsetstrokecolor{currentstroke}%
\pgfsetdash{}{0pt}%
\pgfpathmoveto{\pgfqpoint{5.147390in}{2.494275in}}%
\pgfpathlineto{\pgfqpoint{5.161141in}{2.492465in}}%
\pgfpathlineto{\pgfqpoint{5.174899in}{2.490680in}}%
\pgfpathlineto{\pgfqpoint{5.188666in}{2.488920in}}%
\pgfpathlineto{\pgfqpoint{5.202440in}{2.487185in}}%
\pgfpathlineto{\pgfqpoint{5.195090in}{2.480733in}}%
\pgfpathlineto{\pgfqpoint{5.187734in}{2.474247in}}%
\pgfpathlineto{\pgfqpoint{5.180372in}{2.467724in}}%
\pgfpathlineto{\pgfqpoint{5.173003in}{2.461162in}}%
\pgfpathlineto{\pgfqpoint{5.159214in}{2.462831in}}%
\pgfpathlineto{\pgfqpoint{5.145433in}{2.464526in}}%
\pgfpathlineto{\pgfqpoint{5.131659in}{2.466246in}}%
\pgfpathlineto{\pgfqpoint{5.117893in}{2.467990in}}%
\pgfpathlineto{\pgfqpoint{5.125277in}{2.474612in}}%
\pgfpathlineto{\pgfqpoint{5.132654in}{2.481199in}}%
\pgfpathlineto{\pgfqpoint{5.140025in}{2.487753in}}%
\pgfpathlineto{\pgfqpoint{5.147390in}{2.494275in}}%
\pgfpathclose%
\pgfusepath{fill}%
\end{pgfscope}%
\begin{pgfscope}%
\pgfpathrectangle{\pgfqpoint{1.150000in}{0.150000in}}{\pgfqpoint{5.700000in}{5.700000in}}%
\pgfusepath{clip}%
\pgfsetbuttcap%
\pgfsetroundjoin%
\definecolor{currentfill}{rgb}{0.283072,0.130895,0.449241}%
\pgfsetfillcolor{currentfill}%
\pgfsetfillopacity{0.700000}%
\pgfsetlinewidth{0.000000pt}%
\definecolor{currentstroke}{rgb}{0.000000,0.000000,0.000000}%
\pgfsetstrokecolor{currentstroke}%
\pgfsetdash{}{0pt}%
\pgfpathmoveto{\pgfqpoint{5.595014in}{2.550653in}}%
\pgfpathlineto{\pgfqpoint{5.608886in}{2.549039in}}%
\pgfpathlineto{\pgfqpoint{5.622767in}{2.547449in}}%
\pgfpathlineto{\pgfqpoint{5.636655in}{2.545883in}}%
\pgfpathlineto{\pgfqpoint{5.650552in}{2.544341in}}%
\pgfpathlineto{\pgfqpoint{5.643402in}{2.538780in}}%
\pgfpathlineto{\pgfqpoint{5.636247in}{2.533215in}}%
\pgfpathlineto{\pgfqpoint{5.629086in}{2.527644in}}%
\pgfpathlineto{\pgfqpoint{5.621919in}{2.522061in}}%
\pgfpathlineto{\pgfqpoint{5.608003in}{2.523483in}}%
\pgfpathlineto{\pgfqpoint{5.594096in}{2.524930in}}%
\pgfpathlineto{\pgfqpoint{5.580198in}{2.526401in}}%
\pgfpathlineto{\pgfqpoint{5.566307in}{2.527896in}}%
\pgfpathlineto{\pgfqpoint{5.573492in}{2.533593in}}%
\pgfpathlineto{\pgfqpoint{5.580672in}{2.539283in}}%
\pgfpathlineto{\pgfqpoint{5.587846in}{2.544968in}}%
\pgfpathlineto{\pgfqpoint{5.595014in}{2.550653in}}%
\pgfpathclose%
\pgfusepath{fill}%
\end{pgfscope}%
\begin{pgfscope}%
\pgfpathrectangle{\pgfqpoint{1.150000in}{0.150000in}}{\pgfqpoint{5.700000in}{5.700000in}}%
\pgfusepath{clip}%
\pgfsetbuttcap%
\pgfsetroundjoin%
\definecolor{currentfill}{rgb}{0.283197,0.115680,0.436115}%
\pgfsetfillcolor{currentfill}%
\pgfsetfillopacity{0.700000}%
\pgfsetlinewidth{0.000000pt}%
\definecolor{currentstroke}{rgb}{0.000000,0.000000,0.000000}%
\pgfsetstrokecolor{currentstroke}%
\pgfsetdash{}{0pt}%
\pgfpathmoveto{\pgfqpoint{5.371231in}{2.523536in}}%
\pgfpathlineto{\pgfqpoint{5.385043in}{2.521853in}}%
\pgfpathlineto{\pgfqpoint{5.398863in}{2.520195in}}%
\pgfpathlineto{\pgfqpoint{5.412691in}{2.518561in}}%
\pgfpathlineto{\pgfqpoint{5.426527in}{2.516951in}}%
\pgfpathlineto{\pgfqpoint{5.419276in}{2.510967in}}%
\pgfpathlineto{\pgfqpoint{5.412019in}{2.504961in}}%
\pgfpathlineto{\pgfqpoint{5.404756in}{2.498931in}}%
\pgfpathlineto{\pgfqpoint{5.397486in}{2.492872in}}%
\pgfpathlineto{\pgfqpoint{5.383633in}{2.494389in}}%
\pgfpathlineto{\pgfqpoint{5.369789in}{2.495931in}}%
\pgfpathlineto{\pgfqpoint{5.355952in}{2.497497in}}%
\pgfpathlineto{\pgfqpoint{5.342123in}{2.499087in}}%
\pgfpathlineto{\pgfqpoint{5.349409in}{2.505233in}}%
\pgfpathlineto{\pgfqpoint{5.356689in}{2.511355in}}%
\pgfpathlineto{\pgfqpoint{5.363963in}{2.517455in}}%
\pgfpathlineto{\pgfqpoint{5.371231in}{2.523536in}}%
\pgfpathclose%
\pgfusepath{fill}%
\end{pgfscope}%
\begin{pgfscope}%
\pgfpathrectangle{\pgfqpoint{1.150000in}{0.150000in}}{\pgfqpoint{5.700000in}{5.700000in}}%
\pgfusepath{clip}%
\pgfsetbuttcap%
\pgfsetroundjoin%
\definecolor{currentfill}{rgb}{0.267004,0.004874,0.329415}%
\pgfsetfillcolor{currentfill}%
\pgfsetfillopacity{0.700000}%
\pgfsetlinewidth{0.000000pt}%
\definecolor{currentstroke}{rgb}{0.000000,0.000000,0.000000}%
\pgfsetstrokecolor{currentstroke}%
\pgfsetdash{}{0pt}%
\pgfpathmoveto{\pgfqpoint{3.804354in}{2.329999in}}%
\pgfpathlineto{\pgfqpoint{3.817768in}{2.326120in}}%
\pgfpathlineto{\pgfqpoint{3.831187in}{2.322272in}}%
\pgfpathlineto{\pgfqpoint{3.844612in}{2.318455in}}%
\pgfpathlineto{\pgfqpoint{3.858042in}{2.314668in}}%
\pgfpathlineto{\pgfqpoint{3.850174in}{2.307096in}}%
\pgfpathlineto{\pgfqpoint{3.842301in}{2.299539in}}%
\pgfpathlineto{\pgfqpoint{3.834421in}{2.291999in}}%
\pgfpathlineto{\pgfqpoint{3.826535in}{2.284478in}}%
\pgfpathlineto{\pgfqpoint{3.813091in}{2.288359in}}%
\pgfpathlineto{\pgfqpoint{3.799653in}{2.292271in}}%
\pgfpathlineto{\pgfqpoint{3.786220in}{2.296214in}}%
\pgfpathlineto{\pgfqpoint{3.772794in}{2.300188in}}%
\pgfpathlineto{\pgfqpoint{3.780693in}{2.307609in}}%
\pgfpathlineto{\pgfqpoint{3.788586in}{2.315053in}}%
\pgfpathlineto{\pgfqpoint{3.796473in}{2.322517in}}%
\pgfpathlineto{\pgfqpoint{3.804354in}{2.329999in}}%
\pgfpathclose%
\pgfusepath{fill}%
\end{pgfscope}%
\begin{pgfscope}%
\pgfpathrectangle{\pgfqpoint{1.150000in}{0.150000in}}{\pgfqpoint{5.700000in}{5.700000in}}%
\pgfusepath{clip}%
\pgfsetbuttcap%
\pgfsetroundjoin%
\definecolor{currentfill}{rgb}{0.274952,0.037752,0.364543}%
\pgfsetfillcolor{currentfill}%
\pgfsetfillopacity{0.700000}%
\pgfsetlinewidth{0.000000pt}%
\definecolor{currentstroke}{rgb}{0.000000,0.000000,0.000000}%
\pgfsetstrokecolor{currentstroke}%
\pgfsetdash{}{0pt}%
\pgfpathmoveto{\pgfqpoint{4.391026in}{2.380506in}}%
\pgfpathlineto{\pgfqpoint{4.404580in}{2.377829in}}%
\pgfpathlineto{\pgfqpoint{4.418140in}{2.375179in}}%
\pgfpathlineto{\pgfqpoint{4.431707in}{2.372557in}}%
\pgfpathlineto{\pgfqpoint{4.445281in}{2.369961in}}%
\pgfpathlineto{\pgfqpoint{4.437626in}{2.362187in}}%
\pgfpathlineto{\pgfqpoint{4.429965in}{2.354379in}}%
\pgfpathlineto{\pgfqpoint{4.422299in}{2.346539in}}%
\pgfpathlineto{\pgfqpoint{4.414626in}{2.338667in}}%
\pgfpathlineto{\pgfqpoint{4.401040in}{2.341291in}}%
\pgfpathlineto{\pgfqpoint{4.387461in}{2.343942in}}%
\pgfpathlineto{\pgfqpoint{4.373889in}{2.346620in}}%
\pgfpathlineto{\pgfqpoint{4.360323in}{2.349326in}}%
\pgfpathlineto{\pgfqpoint{4.368008in}{2.357164in}}%
\pgfpathlineto{\pgfqpoint{4.375686in}{2.364974in}}%
\pgfpathlineto{\pgfqpoint{4.383359in}{2.372755in}}%
\pgfpathlineto{\pgfqpoint{4.391026in}{2.380506in}}%
\pgfpathclose%
\pgfusepath{fill}%
\end{pgfscope}%
\begin{pgfscope}%
\pgfpathrectangle{\pgfqpoint{1.150000in}{0.150000in}}{\pgfqpoint{5.700000in}{5.700000in}}%
\pgfusepath{clip}%
\pgfsetbuttcap%
\pgfsetroundjoin%
\definecolor{currentfill}{rgb}{0.271305,0.019942,0.347269}%
\pgfsetfillcolor{currentfill}%
\pgfsetfillopacity{0.700000}%
\pgfsetlinewidth{0.000000pt}%
\definecolor{currentstroke}{rgb}{0.000000,0.000000,0.000000}%
\pgfsetstrokecolor{currentstroke}%
\pgfsetdash{}{0pt}%
\pgfpathmoveto{\pgfqpoint{4.167119in}{2.352817in}}%
\pgfpathlineto{\pgfqpoint{4.180617in}{2.349736in}}%
\pgfpathlineto{\pgfqpoint{4.194121in}{2.346682in}}%
\pgfpathlineto{\pgfqpoint{4.207632in}{2.343657in}}%
\pgfpathlineto{\pgfqpoint{4.221150in}{2.340660in}}%
\pgfpathlineto{\pgfqpoint{4.213413in}{2.332784in}}%
\pgfpathlineto{\pgfqpoint{4.205671in}{2.324890in}}%
\pgfpathlineto{\pgfqpoint{4.197922in}{2.316977in}}%
\pgfpathlineto{\pgfqpoint{4.190169in}{2.309047in}}%
\pgfpathlineto{\pgfqpoint{4.176639in}{2.312099in}}%
\pgfpathlineto{\pgfqpoint{4.163116in}{2.315179in}}%
\pgfpathlineto{\pgfqpoint{4.149599in}{2.318288in}}%
\pgfpathlineto{\pgfqpoint{4.136088in}{2.321424in}}%
\pgfpathlineto{\pgfqpoint{4.143854in}{2.329294in}}%
\pgfpathlineto{\pgfqpoint{4.151615in}{2.337150in}}%
\pgfpathlineto{\pgfqpoint{4.159370in}{2.344991in}}%
\pgfpathlineto{\pgfqpoint{4.167119in}{2.352817in}}%
\pgfpathclose%
\pgfusepath{fill}%
\end{pgfscope}%
\begin{pgfscope}%
\pgfpathrectangle{\pgfqpoint{1.150000in}{0.150000in}}{\pgfqpoint{5.700000in}{5.700000in}}%
\pgfusepath{clip}%
\pgfsetbuttcap%
\pgfsetroundjoin%
\definecolor{currentfill}{rgb}{0.277941,0.056324,0.381191}%
\pgfsetfillcolor{currentfill}%
\pgfsetfillopacity{0.700000}%
\pgfsetlinewidth{0.000000pt}%
\definecolor{currentstroke}{rgb}{0.000000,0.000000,0.000000}%
\pgfsetstrokecolor{currentstroke}%
\pgfsetdash{}{0pt}%
\pgfpathmoveto{\pgfqpoint{4.614956in}{2.411328in}}%
\pgfpathlineto{\pgfqpoint{4.628568in}{2.408991in}}%
\pgfpathlineto{\pgfqpoint{4.642187in}{2.406680in}}%
\pgfpathlineto{\pgfqpoint{4.655814in}{2.404395in}}%
\pgfpathlineto{\pgfqpoint{4.669447in}{2.402137in}}%
\pgfpathlineto{\pgfqpoint{4.661876in}{2.394630in}}%
\pgfpathlineto{\pgfqpoint{4.654300in}{2.387081in}}%
\pgfpathlineto{\pgfqpoint{4.646717in}{2.379492in}}%
\pgfpathlineto{\pgfqpoint{4.639128in}{2.371860in}}%
\pgfpathlineto{\pgfqpoint{4.625482in}{2.374120in}}%
\pgfpathlineto{\pgfqpoint{4.611843in}{2.376406in}}%
\pgfpathlineto{\pgfqpoint{4.598211in}{2.378719in}}%
\pgfpathlineto{\pgfqpoint{4.584587in}{2.381058in}}%
\pgfpathlineto{\pgfqpoint{4.592188in}{2.388683in}}%
\pgfpathlineto{\pgfqpoint{4.599783in}{2.396270in}}%
\pgfpathlineto{\pgfqpoint{4.607372in}{2.403818in}}%
\pgfpathlineto{\pgfqpoint{4.614956in}{2.411328in}}%
\pgfpathclose%
\pgfusepath{fill}%
\end{pgfscope}%
\begin{pgfscope}%
\pgfpathrectangle{\pgfqpoint{1.150000in}{0.150000in}}{\pgfqpoint{5.700000in}{5.700000in}}%
\pgfusepath{clip}%
\pgfsetbuttcap%
\pgfsetroundjoin%
\definecolor{currentfill}{rgb}{0.269944,0.014625,0.341379}%
\pgfsetfillcolor{currentfill}%
\pgfsetfillopacity{0.700000}%
\pgfsetlinewidth{0.000000pt}%
\definecolor{currentstroke}{rgb}{0.000000,0.000000,0.000000}%
\pgfsetstrokecolor{currentstroke}%
\pgfsetdash{}{0pt}%
\pgfpathmoveto{\pgfqpoint{3.302526in}{2.348441in}}%
\pgfpathlineto{\pgfqpoint{3.315852in}{2.343164in}}%
\pgfpathlineto{\pgfqpoint{3.329182in}{2.337924in}}%
\pgfpathlineto{\pgfqpoint{3.342517in}{2.332720in}}%
\pgfpathlineto{\pgfqpoint{3.355857in}{2.327551in}}%
\pgfpathlineto{\pgfqpoint{3.347781in}{2.321507in}}%
\pgfpathlineto{\pgfqpoint{3.339698in}{2.315543in}}%
\pgfpathlineto{\pgfqpoint{3.331607in}{2.309664in}}%
\pgfpathlineto{\pgfqpoint{3.323508in}{2.303872in}}%
\pgfpathlineto{\pgfqpoint{3.310151in}{2.309188in}}%
\pgfpathlineto{\pgfqpoint{3.296799in}{2.314540in}}%
\pgfpathlineto{\pgfqpoint{3.283452in}{2.319928in}}%
\pgfpathlineto{\pgfqpoint{3.270109in}{2.325353in}}%
\pgfpathlineto{\pgfqpoint{3.278225in}{2.330992in}}%
\pgfpathlineto{\pgfqpoint{3.286333in}{2.336721in}}%
\pgfpathlineto{\pgfqpoint{3.294433in}{2.342539in}}%
\pgfpathlineto{\pgfqpoint{3.302526in}{2.348441in}}%
\pgfpathclose%
\pgfusepath{fill}%
\end{pgfscope}%
\begin{pgfscope}%
\pgfpathrectangle{\pgfqpoint{1.150000in}{0.150000in}}{\pgfqpoint{5.700000in}{5.700000in}}%
\pgfusepath{clip}%
\pgfsetbuttcap%
\pgfsetroundjoin%
\definecolor{currentfill}{rgb}{0.268510,0.009605,0.335427}%
\pgfsetfillcolor{currentfill}%
\pgfsetfillopacity{0.700000}%
\pgfsetlinewidth{0.000000pt}%
\definecolor{currentstroke}{rgb}{0.000000,0.000000,0.000000}%
\pgfsetstrokecolor{currentstroke}%
\pgfsetdash{}{0pt}%
\pgfpathmoveto{\pgfqpoint{3.441428in}{2.332696in}}%
\pgfpathlineto{\pgfqpoint{3.454776in}{2.327838in}}%
\pgfpathlineto{\pgfqpoint{3.468129in}{2.323016in}}%
\pgfpathlineto{\pgfqpoint{3.481487in}{2.318228in}}%
\pgfpathlineto{\pgfqpoint{3.494851in}{2.313474in}}%
\pgfpathlineto{\pgfqpoint{3.486836in}{2.306879in}}%
\pgfpathlineto{\pgfqpoint{3.478814in}{2.300345in}}%
\pgfpathlineto{\pgfqpoint{3.470786in}{2.293876in}}%
\pgfpathlineto{\pgfqpoint{3.462750in}{2.287475in}}%
\pgfpathlineto{\pgfqpoint{3.449371in}{2.292363in}}%
\pgfpathlineto{\pgfqpoint{3.435997in}{2.297286in}}%
\pgfpathlineto{\pgfqpoint{3.422628in}{2.302243in}}%
\pgfpathlineto{\pgfqpoint{3.409264in}{2.307235in}}%
\pgfpathlineto{\pgfqpoint{3.417316in}{2.313496in}}%
\pgfpathlineto{\pgfqpoint{3.425360in}{2.319829in}}%
\pgfpathlineto{\pgfqpoint{3.433398in}{2.326230in}}%
\pgfpathlineto{\pgfqpoint{3.441428in}{2.332696in}}%
\pgfpathclose%
\pgfusepath{fill}%
\end{pgfscope}%
\begin{pgfscope}%
\pgfpathrectangle{\pgfqpoint{1.150000in}{0.150000in}}{\pgfqpoint{5.700000in}{5.700000in}}%
\pgfusepath{clip}%
\pgfsetbuttcap%
\pgfsetroundjoin%
\definecolor{currentfill}{rgb}{0.283187,0.125848,0.444960}%
\pgfsetfillcolor{currentfill}%
\pgfsetfillopacity{0.700000}%
\pgfsetlinewidth{0.000000pt}%
\definecolor{currentstroke}{rgb}{0.000000,0.000000,0.000000}%
\pgfsetstrokecolor{currentstroke}%
\pgfsetdash{}{0pt}%
\pgfpathmoveto{\pgfqpoint{2.639123in}{2.537912in}}%
\pgfpathlineto{\pgfqpoint{2.652385in}{2.530217in}}%
\pgfpathlineto{\pgfqpoint{2.665649in}{2.522575in}}%
\pgfpathlineto{\pgfqpoint{2.678916in}{2.514984in}}%
\pgfpathlineto{\pgfqpoint{2.692186in}{2.507444in}}%
\pgfpathlineto{\pgfqpoint{2.683751in}{2.505137in}}%
\pgfpathlineto{\pgfqpoint{2.675304in}{2.503012in}}%
\pgfpathlineto{\pgfqpoint{2.666843in}{2.501075in}}%
\pgfpathlineto{\pgfqpoint{2.658370in}{2.499330in}}%
\pgfpathlineto{\pgfqpoint{2.645075in}{2.507073in}}%
\pgfpathlineto{\pgfqpoint{2.631783in}{2.514867in}}%
\pgfpathlineto{\pgfqpoint{2.618493in}{2.522713in}}%
\pgfpathlineto{\pgfqpoint{2.605206in}{2.530611in}}%
\pgfpathlineto{\pgfqpoint{2.613705in}{2.532148in}}%
\pgfpathlineto{\pgfqpoint{2.622191in}{2.533880in}}%
\pgfpathlineto{\pgfqpoint{2.630664in}{2.535803in}}%
\pgfpathlineto{\pgfqpoint{2.639123in}{2.537912in}}%
\pgfpathclose%
\pgfusepath{fill}%
\end{pgfscope}%
\begin{pgfscope}%
\pgfpathrectangle{\pgfqpoint{1.150000in}{0.150000in}}{\pgfqpoint{5.700000in}{5.700000in}}%
\pgfusepath{clip}%
\pgfsetbuttcap%
\pgfsetroundjoin%
\definecolor{currentfill}{rgb}{0.281446,0.084320,0.407414}%
\pgfsetfillcolor{currentfill}%
\pgfsetfillopacity{0.700000}%
\pgfsetlinewidth{0.000000pt}%
\definecolor{currentstroke}{rgb}{0.000000,0.000000,0.000000}%
\pgfsetstrokecolor{currentstroke}%
\pgfsetdash{}{0pt}%
\pgfpathmoveto{\pgfqpoint{2.831872in}{2.461373in}}%
\pgfpathlineto{\pgfqpoint{2.845145in}{2.454459in}}%
\pgfpathlineto{\pgfqpoint{2.858420in}{2.447592in}}%
\pgfpathlineto{\pgfqpoint{2.871700in}{2.440771in}}%
\pgfpathlineto{\pgfqpoint{2.884982in}{2.433995in}}%
\pgfpathlineto{\pgfqpoint{2.876664in}{2.430456in}}%
\pgfpathlineto{\pgfqpoint{2.868335in}{2.427069in}}%
\pgfpathlineto{\pgfqpoint{2.859995in}{2.423840in}}%
\pgfpathlineto{\pgfqpoint{2.851643in}{2.420773in}}%
\pgfpathlineto{\pgfqpoint{2.838338in}{2.427737in}}%
\pgfpathlineto{\pgfqpoint{2.825036in}{2.434747in}}%
\pgfpathlineto{\pgfqpoint{2.811738in}{2.441803in}}%
\pgfpathlineto{\pgfqpoint{2.798442in}{2.448906in}}%
\pgfpathlineto{\pgfqpoint{2.806817in}{2.451779in}}%
\pgfpathlineto{\pgfqpoint{2.815180in}{2.454817in}}%
\pgfpathlineto{\pgfqpoint{2.823532in}{2.458017in}}%
\pgfpathlineto{\pgfqpoint{2.831872in}{2.461373in}}%
\pgfpathclose%
\pgfusepath{fill}%
\end{pgfscope}%
\begin{pgfscope}%
\pgfpathrectangle{\pgfqpoint{1.150000in}{0.150000in}}{\pgfqpoint{5.700000in}{5.700000in}}%
\pgfusepath{clip}%
\pgfsetbuttcap%
\pgfsetroundjoin%
\definecolor{currentfill}{rgb}{0.268510,0.009605,0.335427}%
\pgfsetfillcolor{currentfill}%
\pgfsetfillopacity{0.700000}%
\pgfsetlinewidth{0.000000pt}%
\definecolor{currentstroke}{rgb}{0.000000,0.000000,0.000000}%
\pgfsetstrokecolor{currentstroke}%
\pgfsetdash{}{0pt}%
\pgfpathmoveto{\pgfqpoint{3.943187in}{2.330548in}}%
\pgfpathlineto{\pgfqpoint{3.956635in}{2.326994in}}%
\pgfpathlineto{\pgfqpoint{3.970089in}{2.323469in}}%
\pgfpathlineto{\pgfqpoint{3.983549in}{2.319975in}}%
\pgfpathlineto{\pgfqpoint{3.997015in}{2.316509in}}%
\pgfpathlineto{\pgfqpoint{3.989196in}{2.308744in}}%
\pgfpathlineto{\pgfqpoint{3.981371in}{2.300979in}}%
\pgfpathlineto{\pgfqpoint{3.973540in}{2.293217in}}%
\pgfpathlineto{\pgfqpoint{3.965704in}{2.285460in}}%
\pgfpathlineto{\pgfqpoint{3.952225in}{2.289006in}}%
\pgfpathlineto{\pgfqpoint{3.938752in}{2.292583in}}%
\pgfpathlineto{\pgfqpoint{3.925286in}{2.296189in}}%
\pgfpathlineto{\pgfqpoint{3.911825in}{2.299825in}}%
\pgfpathlineto{\pgfqpoint{3.919674in}{2.307496in}}%
\pgfpathlineto{\pgfqpoint{3.927518in}{2.315174in}}%
\pgfpathlineto{\pgfqpoint{3.935355in}{2.322859in}}%
\pgfpathlineto{\pgfqpoint{3.943187in}{2.330548in}}%
\pgfpathclose%
\pgfusepath{fill}%
\end{pgfscope}%
\begin{pgfscope}%
\pgfpathrectangle{\pgfqpoint{1.150000in}{0.150000in}}{\pgfqpoint{5.700000in}{5.700000in}}%
\pgfusepath{clip}%
\pgfsetbuttcap%
\pgfsetroundjoin%
\definecolor{currentfill}{rgb}{0.280267,0.073417,0.397163}%
\pgfsetfillcolor{currentfill}%
\pgfsetfillopacity{0.700000}%
\pgfsetlinewidth{0.000000pt}%
\definecolor{currentstroke}{rgb}{0.000000,0.000000,0.000000}%
\pgfsetstrokecolor{currentstroke}%
\pgfsetdash{}{0pt}%
\pgfpathmoveto{\pgfqpoint{4.838920in}{2.443397in}}%
\pgfpathlineto{\pgfqpoint{4.852593in}{2.441336in}}%
\pgfpathlineto{\pgfqpoint{4.866274in}{2.439301in}}%
\pgfpathlineto{\pgfqpoint{4.879961in}{2.437292in}}%
\pgfpathlineto{\pgfqpoint{4.893657in}{2.435308in}}%
\pgfpathlineto{\pgfqpoint{4.886174in}{2.428188in}}%
\pgfpathlineto{\pgfqpoint{4.878685in}{2.421025in}}%
\pgfpathlineto{\pgfqpoint{4.871190in}{2.413817in}}%
\pgfpathlineto{\pgfqpoint{4.863688in}{2.406563in}}%
\pgfpathlineto{\pgfqpoint{4.849980in}{2.408522in}}%
\pgfpathlineto{\pgfqpoint{4.836278in}{2.410506in}}%
\pgfpathlineto{\pgfqpoint{4.822584in}{2.412516in}}%
\pgfpathlineto{\pgfqpoint{4.808898in}{2.414551in}}%
\pgfpathlineto{\pgfqpoint{4.816413in}{2.421825in}}%
\pgfpathlineto{\pgfqpoint{4.823921in}{2.429057in}}%
\pgfpathlineto{\pgfqpoint{4.831424in}{2.436247in}}%
\pgfpathlineto{\pgfqpoint{4.838920in}{2.443397in}}%
\pgfpathclose%
\pgfusepath{fill}%
\end{pgfscope}%
\begin{pgfscope}%
\pgfpathrectangle{\pgfqpoint{1.150000in}{0.150000in}}{\pgfqpoint{5.700000in}{5.700000in}}%
\pgfusepath{clip}%
\pgfsetbuttcap%
\pgfsetroundjoin%
\definecolor{currentfill}{rgb}{0.281887,0.150881,0.465405}%
\pgfsetfillcolor{currentfill}%
\pgfsetfillopacity{0.700000}%
\pgfsetlinewidth{0.000000pt}%
\definecolor{currentstroke}{rgb}{0.000000,0.000000,0.000000}%
\pgfsetstrokecolor{currentstroke}%
\pgfsetdash{}{0pt}%
\pgfpathmoveto{\pgfqpoint{5.958440in}{2.583946in}}%
\pgfpathlineto{\pgfqpoint{5.972415in}{2.582374in}}%
\pgfpathlineto{\pgfqpoint{5.986398in}{2.580827in}}%
\pgfpathlineto{\pgfqpoint{6.000390in}{2.579303in}}%
\pgfpathlineto{\pgfqpoint{6.014390in}{2.577803in}}%
\pgfpathlineto{\pgfqpoint{6.007404in}{2.572768in}}%
\pgfpathlineto{\pgfqpoint{6.000413in}{2.567767in}}%
\pgfpathlineto{\pgfqpoint{5.993418in}{2.562795in}}%
\pgfpathlineto{\pgfqpoint{5.986418in}{2.557847in}}%
\pgfpathlineto{\pgfqpoint{5.972396in}{2.559187in}}%
\pgfpathlineto{\pgfqpoint{5.958383in}{2.560552in}}%
\pgfpathlineto{\pgfqpoint{5.944378in}{2.561940in}}%
\pgfpathlineto{\pgfqpoint{5.930381in}{2.563352in}}%
\pgfpathlineto{\pgfqpoint{5.937403in}{2.568455in}}%
\pgfpathlineto{\pgfqpoint{5.944420in}{2.573585in}}%
\pgfpathlineto{\pgfqpoint{5.951432in}{2.578747in}}%
\pgfpathlineto{\pgfqpoint{5.958440in}{2.583946in}}%
\pgfpathclose%
\pgfusepath{fill}%
\end{pgfscope}%
\begin{pgfscope}%
\pgfpathrectangle{\pgfqpoint{1.150000in}{0.150000in}}{\pgfqpoint{5.700000in}{5.700000in}}%
\pgfusepath{clip}%
\pgfsetbuttcap%
\pgfsetroundjoin%
\definecolor{currentfill}{rgb}{0.273809,0.031497,0.358853}%
\pgfsetfillcolor{currentfill}%
\pgfsetfillopacity{0.700000}%
\pgfsetlinewidth{0.000000pt}%
\definecolor{currentstroke}{rgb}{0.000000,0.000000,0.000000}%
\pgfsetstrokecolor{currentstroke}%
\pgfsetdash{}{0pt}%
\pgfpathmoveto{\pgfqpoint{3.163526in}{2.370100in}}%
\pgfpathlineto{\pgfqpoint{3.176833in}{2.364373in}}%
\pgfpathlineto{\pgfqpoint{3.190145in}{2.358685in}}%
\pgfpathlineto{\pgfqpoint{3.203462in}{2.353035in}}%
\pgfpathlineto{\pgfqpoint{3.216782in}{2.347423in}}%
\pgfpathlineto{\pgfqpoint{3.208640in}{2.342036in}}%
\pgfpathlineto{\pgfqpoint{3.200489in}{2.336751in}}%
\pgfpathlineto{\pgfqpoint{3.192330in}{2.331572in}}%
\pgfpathlineto{\pgfqpoint{3.184162in}{2.326503in}}%
\pgfpathlineto{\pgfqpoint{3.170823in}{2.332276in}}%
\pgfpathlineto{\pgfqpoint{3.157488in}{2.338087in}}%
\pgfpathlineto{\pgfqpoint{3.144158in}{2.343937in}}%
\pgfpathlineto{\pgfqpoint{3.130832in}{2.349825in}}%
\pgfpathlineto{\pgfqpoint{3.139018in}{2.354728in}}%
\pgfpathlineto{\pgfqpoint{3.147196in}{2.359744in}}%
\pgfpathlineto{\pgfqpoint{3.155365in}{2.364869in}}%
\pgfpathlineto{\pgfqpoint{3.163526in}{2.370100in}}%
\pgfpathclose%
\pgfusepath{fill}%
\end{pgfscope}%
\begin{pgfscope}%
\pgfpathrectangle{\pgfqpoint{1.150000in}{0.150000in}}{\pgfqpoint{5.700000in}{5.700000in}}%
\pgfusepath{clip}%
\pgfsetbuttcap%
\pgfsetroundjoin%
\definecolor{currentfill}{rgb}{0.267004,0.004874,0.329415}%
\pgfsetfillcolor{currentfill}%
\pgfsetfillopacity{0.700000}%
\pgfsetlinewidth{0.000000pt}%
\definecolor{currentstroke}{rgb}{0.000000,0.000000,0.000000}%
\pgfsetstrokecolor{currentstroke}%
\pgfsetdash{}{0pt}%
\pgfpathmoveto{\pgfqpoint{3.580285in}{2.322215in}}%
\pgfpathlineto{\pgfqpoint{3.593660in}{2.317750in}}%
\pgfpathlineto{\pgfqpoint{3.607039in}{2.313318in}}%
\pgfpathlineto{\pgfqpoint{3.620424in}{2.308918in}}%
\pgfpathlineto{\pgfqpoint{3.633815in}{2.304551in}}%
\pgfpathlineto{\pgfqpoint{3.625857in}{2.297508in}}%
\pgfpathlineto{\pgfqpoint{3.617892in}{2.290507in}}%
\pgfpathlineto{\pgfqpoint{3.609921in}{2.283553in}}%
\pgfpathlineto{\pgfqpoint{3.601943in}{2.276647in}}%
\pgfpathlineto{\pgfqpoint{3.588538in}{2.281135in}}%
\pgfpathlineto{\pgfqpoint{3.575139in}{2.285656in}}%
\pgfpathlineto{\pgfqpoint{3.561744in}{2.290209in}}%
\pgfpathlineto{\pgfqpoint{3.548355in}{2.294795in}}%
\pgfpathlineto{\pgfqpoint{3.556348in}{2.301575in}}%
\pgfpathlineto{\pgfqpoint{3.564333in}{2.308406in}}%
\pgfpathlineto{\pgfqpoint{3.572313in}{2.315287in}}%
\pgfpathlineto{\pgfqpoint{3.580285in}{2.322215in}}%
\pgfpathclose%
\pgfusepath{fill}%
\end{pgfscope}%
\begin{pgfscope}%
\pgfpathrectangle{\pgfqpoint{1.150000in}{0.150000in}}{\pgfqpoint{5.700000in}{5.700000in}}%
\pgfusepath{clip}%
\pgfsetbuttcap%
\pgfsetroundjoin%
\definecolor{currentfill}{rgb}{0.282327,0.094955,0.417331}%
\pgfsetfillcolor{currentfill}%
\pgfsetfillopacity{0.700000}%
\pgfsetlinewidth{0.000000pt}%
\definecolor{currentstroke}{rgb}{0.000000,0.000000,0.000000}%
\pgfsetstrokecolor{currentstroke}%
\pgfsetdash{}{0pt}%
\pgfpathmoveto{\pgfqpoint{5.062906in}{2.475220in}}%
\pgfpathlineto{\pgfqpoint{5.076641in}{2.473375in}}%
\pgfpathlineto{\pgfqpoint{5.090384in}{2.471555in}}%
\pgfpathlineto{\pgfqpoint{5.104135in}{2.469760in}}%
\pgfpathlineto{\pgfqpoint{5.117893in}{2.467990in}}%
\pgfpathlineto{\pgfqpoint{5.110503in}{2.461330in}}%
\pgfpathlineto{\pgfqpoint{5.103107in}{2.454631in}}%
\pgfpathlineto{\pgfqpoint{5.095704in}{2.447889in}}%
\pgfpathlineto{\pgfqpoint{5.088295in}{2.441104in}}%
\pgfpathlineto{\pgfqpoint{5.074522in}{2.442822in}}%
\pgfpathlineto{\pgfqpoint{5.060757in}{2.444564in}}%
\pgfpathlineto{\pgfqpoint{5.046999in}{2.446332in}}%
\pgfpathlineto{\pgfqpoint{5.033250in}{2.448126in}}%
\pgfpathlineto{\pgfqpoint{5.040673in}{2.454958in}}%
\pgfpathlineto{\pgfqpoint{5.048091in}{2.461750in}}%
\pgfpathlineto{\pgfqpoint{5.055502in}{2.468503in}}%
\pgfpathlineto{\pgfqpoint{5.062906in}{2.475220in}}%
\pgfpathclose%
\pgfusepath{fill}%
\end{pgfscope}%
\begin{pgfscope}%
\pgfpathrectangle{\pgfqpoint{1.150000in}{0.150000in}}{\pgfqpoint{5.700000in}{5.700000in}}%
\pgfusepath{clip}%
\pgfsetbuttcap%
\pgfsetroundjoin%
\definecolor{currentfill}{rgb}{0.282623,0.140926,0.457517}%
\pgfsetfillcolor{currentfill}%
\pgfsetfillopacity{0.700000}%
\pgfsetlinewidth{0.000000pt}%
\definecolor{currentstroke}{rgb}{0.000000,0.000000,0.000000}%
\pgfsetstrokecolor{currentstroke}%
\pgfsetdash{}{0pt}%
\pgfpathmoveto{\pgfqpoint{5.734685in}{2.560176in}}%
\pgfpathlineto{\pgfqpoint{5.748603in}{2.558623in}}%
\pgfpathlineto{\pgfqpoint{5.762529in}{2.557093in}}%
\pgfpathlineto{\pgfqpoint{5.776463in}{2.555588in}}%
\pgfpathlineto{\pgfqpoint{5.790406in}{2.554106in}}%
\pgfpathlineto{\pgfqpoint{5.783317in}{2.548794in}}%
\pgfpathlineto{\pgfqpoint{5.776224in}{2.543487in}}%
\pgfpathlineto{\pgfqpoint{5.769124in}{2.538183in}}%
\pgfpathlineto{\pgfqpoint{5.762019in}{2.532877in}}%
\pgfpathlineto{\pgfqpoint{5.748057in}{2.534226in}}%
\pgfpathlineto{\pgfqpoint{5.734103in}{2.535598in}}%
\pgfpathlineto{\pgfqpoint{5.720157in}{2.536995in}}%
\pgfpathlineto{\pgfqpoint{5.706219in}{2.538416in}}%
\pgfpathlineto{\pgfqpoint{5.713344in}{2.543850in}}%
\pgfpathlineto{\pgfqpoint{5.720463in}{2.549285in}}%
\pgfpathlineto{\pgfqpoint{5.727577in}{2.554726in}}%
\pgfpathlineto{\pgfqpoint{5.734685in}{2.560176in}}%
\pgfpathclose%
\pgfusepath{fill}%
\end{pgfscope}%
\begin{pgfscope}%
\pgfpathrectangle{\pgfqpoint{1.150000in}{0.150000in}}{\pgfqpoint{5.700000in}{5.700000in}}%
\pgfusepath{clip}%
\pgfsetbuttcap%
\pgfsetroundjoin%
\definecolor{currentfill}{rgb}{0.283091,0.110553,0.431554}%
\pgfsetfillcolor{currentfill}%
\pgfsetfillopacity{0.700000}%
\pgfsetlinewidth{0.000000pt}%
\definecolor{currentstroke}{rgb}{0.000000,0.000000,0.000000}%
\pgfsetstrokecolor{currentstroke}%
\pgfsetdash{}{0pt}%
\pgfpathmoveto{\pgfqpoint{5.286887in}{2.505697in}}%
\pgfpathlineto{\pgfqpoint{5.300684in}{2.504008in}}%
\pgfpathlineto{\pgfqpoint{5.314489in}{2.502343in}}%
\pgfpathlineto{\pgfqpoint{5.328302in}{2.500703in}}%
\pgfpathlineto{\pgfqpoint{5.342123in}{2.499087in}}%
\pgfpathlineto{\pgfqpoint{5.334830in}{2.492914in}}%
\pgfpathlineto{\pgfqpoint{5.327531in}{2.486710in}}%
\pgfpathlineto{\pgfqpoint{5.320226in}{2.480473in}}%
\pgfpathlineto{\pgfqpoint{5.312914in}{2.474200in}}%
\pgfpathlineto{\pgfqpoint{5.299077in}{2.475737in}}%
\pgfpathlineto{\pgfqpoint{5.285248in}{2.477298in}}%
\pgfpathlineto{\pgfqpoint{5.271427in}{2.478884in}}%
\pgfpathlineto{\pgfqpoint{5.257614in}{2.480494in}}%
\pgfpathlineto{\pgfqpoint{5.264942in}{2.486841in}}%
\pgfpathlineto{\pgfqpoint{5.272263in}{2.493155in}}%
\pgfpathlineto{\pgfqpoint{5.279578in}{2.499440in}}%
\pgfpathlineto{\pgfqpoint{5.286887in}{2.505697in}}%
\pgfpathclose%
\pgfusepath{fill}%
\end{pgfscope}%
\begin{pgfscope}%
\pgfpathrectangle{\pgfqpoint{1.150000in}{0.150000in}}{\pgfqpoint{5.700000in}{5.700000in}}%
\pgfusepath{clip}%
\pgfsetbuttcap%
\pgfsetroundjoin%
\definecolor{currentfill}{rgb}{0.283187,0.125848,0.444960}%
\pgfsetfillcolor{currentfill}%
\pgfsetfillopacity{0.700000}%
\pgfsetlinewidth{0.000000pt}%
\definecolor{currentstroke}{rgb}{0.000000,0.000000,0.000000}%
\pgfsetstrokecolor{currentstroke}%
\pgfsetdash{}{0pt}%
\pgfpathmoveto{\pgfqpoint{5.510825in}{2.534121in}}%
\pgfpathlineto{\pgfqpoint{5.524683in}{2.532528in}}%
\pgfpathlineto{\pgfqpoint{5.538550in}{2.530960in}}%
\pgfpathlineto{\pgfqpoint{5.552424in}{2.529416in}}%
\pgfpathlineto{\pgfqpoint{5.566307in}{2.527896in}}%
\pgfpathlineto{\pgfqpoint{5.559115in}{2.522188in}}%
\pgfpathlineto{\pgfqpoint{5.551918in}{2.516464in}}%
\pgfpathlineto{\pgfqpoint{5.544714in}{2.510722in}}%
\pgfpathlineto{\pgfqpoint{5.537504in}{2.504959in}}%
\pgfpathlineto{\pgfqpoint{5.523604in}{2.506372in}}%
\pgfpathlineto{\pgfqpoint{5.509711in}{2.507810in}}%
\pgfpathlineto{\pgfqpoint{5.495827in}{2.509273in}}%
\pgfpathlineto{\pgfqpoint{5.481951in}{2.510759in}}%
\pgfpathlineto{\pgfqpoint{5.489179in}{2.516624in}}%
\pgfpathlineto{\pgfqpoint{5.496400in}{2.522470in}}%
\pgfpathlineto{\pgfqpoint{5.503616in}{2.528301in}}%
\pgfpathlineto{\pgfqpoint{5.510825in}{2.534121in}}%
\pgfpathclose%
\pgfusepath{fill}%
\end{pgfscope}%
\begin{pgfscope}%
\pgfpathrectangle{\pgfqpoint{1.150000in}{0.150000in}}{\pgfqpoint{5.700000in}{5.700000in}}%
\pgfusepath{clip}%
\pgfsetbuttcap%
\pgfsetroundjoin%
\definecolor{currentfill}{rgb}{0.267004,0.004874,0.329415}%
\pgfsetfillcolor{currentfill}%
\pgfsetfillopacity{0.700000}%
\pgfsetlinewidth{0.000000pt}%
\definecolor{currentstroke}{rgb}{0.000000,0.000000,0.000000}%
\pgfsetstrokecolor{currentstroke}%
\pgfsetdash{}{0pt}%
\pgfpathmoveto{\pgfqpoint{3.719144in}{2.316394in}}%
\pgfpathlineto{\pgfqpoint{3.732548in}{2.312295in}}%
\pgfpathlineto{\pgfqpoint{3.745957in}{2.308228in}}%
\pgfpathlineto{\pgfqpoint{3.759373in}{2.304192in}}%
\pgfpathlineto{\pgfqpoint{3.772794in}{2.300188in}}%
\pgfpathlineto{\pgfqpoint{3.764888in}{2.292790in}}%
\pgfpathlineto{\pgfqpoint{3.756977in}{2.285420in}}%
\pgfpathlineto{\pgfqpoint{3.749060in}{2.278078in}}%
\pgfpathlineto{\pgfqpoint{3.741136in}{2.270768in}}%
\pgfpathlineto{\pgfqpoint{3.727701in}{2.274881in}}%
\pgfpathlineto{\pgfqpoint{3.714272in}{2.279025in}}%
\pgfpathlineto{\pgfqpoint{3.700849in}{2.283200in}}%
\pgfpathlineto{\pgfqpoint{3.687431in}{2.287406in}}%
\pgfpathlineto{\pgfqpoint{3.695369in}{2.294603in}}%
\pgfpathlineto{\pgfqpoint{3.703300in}{2.301836in}}%
\pgfpathlineto{\pgfqpoint{3.711225in}{2.309100in}}%
\pgfpathlineto{\pgfqpoint{3.719144in}{2.316394in}}%
\pgfpathclose%
\pgfusepath{fill}%
\end{pgfscope}%
\begin{pgfscope}%
\pgfpathrectangle{\pgfqpoint{1.150000in}{0.150000in}}{\pgfqpoint{5.700000in}{5.700000in}}%
\pgfusepath{clip}%
\pgfsetbuttcap%
\pgfsetroundjoin%
\definecolor{currentfill}{rgb}{0.277018,0.050344,0.375715}%
\pgfsetfillcolor{currentfill}%
\pgfsetfillopacity{0.700000}%
\pgfsetlinewidth{0.000000pt}%
\definecolor{currentstroke}{rgb}{0.000000,0.000000,0.000000}%
\pgfsetstrokecolor{currentstroke}%
\pgfsetdash{}{0pt}%
\pgfpathmoveto{\pgfqpoint{3.024368in}{2.398369in}}%
\pgfpathlineto{\pgfqpoint{3.037662in}{2.392158in}}%
\pgfpathlineto{\pgfqpoint{3.050960in}{2.385989in}}%
\pgfpathlineto{\pgfqpoint{3.064262in}{2.379861in}}%
\pgfpathlineto{\pgfqpoint{3.077568in}{2.373774in}}%
\pgfpathlineto{\pgfqpoint{3.069352in}{2.369158in}}%
\pgfpathlineto{\pgfqpoint{3.061127in}{2.364668in}}%
\pgfpathlineto{\pgfqpoint{3.052893in}{2.360306in}}%
\pgfpathlineto{\pgfqpoint{3.044649in}{2.356078in}}%
\pgfpathlineto{\pgfqpoint{3.031323in}{2.362341in}}%
\pgfpathlineto{\pgfqpoint{3.018001in}{2.368644in}}%
\pgfpathlineto{\pgfqpoint{3.004682in}{2.374988in}}%
\pgfpathlineto{\pgfqpoint{2.991368in}{2.381373in}}%
\pgfpathlineto{\pgfqpoint{2.999633in}{2.385421in}}%
\pgfpathlineto{\pgfqpoint{3.007887in}{2.389606in}}%
\pgfpathlineto{\pgfqpoint{3.016132in}{2.393923in}}%
\pgfpathlineto{\pgfqpoint{3.024368in}{2.398369in}}%
\pgfpathclose%
\pgfusepath{fill}%
\end{pgfscope}%
\begin{pgfscope}%
\pgfpathrectangle{\pgfqpoint{1.150000in}{0.150000in}}{\pgfqpoint{5.700000in}{5.700000in}}%
\pgfusepath{clip}%
\pgfsetbuttcap%
\pgfsetroundjoin%
\definecolor{currentfill}{rgb}{0.272594,0.025563,0.353093}%
\pgfsetfillcolor{currentfill}%
\pgfsetfillopacity{0.700000}%
\pgfsetlinewidth{0.000000pt}%
\definecolor{currentstroke}{rgb}{0.000000,0.000000,0.000000}%
\pgfsetstrokecolor{currentstroke}%
\pgfsetdash{}{0pt}%
\pgfpathmoveto{\pgfqpoint{4.306129in}{2.360422in}}%
\pgfpathlineto{\pgfqpoint{4.319667in}{2.357607in}}%
\pgfpathlineto{\pgfqpoint{4.333213in}{2.354819in}}%
\pgfpathlineto{\pgfqpoint{4.346765in}{2.352058in}}%
\pgfpathlineto{\pgfqpoint{4.360323in}{2.349326in}}%
\pgfpathlineto{\pgfqpoint{4.352633in}{2.341458in}}%
\pgfpathlineto{\pgfqpoint{4.344938in}{2.333563in}}%
\pgfpathlineto{\pgfqpoint{4.337236in}{2.325640in}}%
\pgfpathlineto{\pgfqpoint{4.329529in}{2.317690in}}%
\pgfpathlineto{\pgfqpoint{4.315958in}{2.320464in}}%
\pgfpathlineto{\pgfqpoint{4.302394in}{2.323266in}}%
\pgfpathlineto{\pgfqpoint{4.288837in}{2.326096in}}%
\pgfpathlineto{\pgfqpoint{4.275286in}{2.328953in}}%
\pgfpathlineto{\pgfqpoint{4.283005in}{2.336857in}}%
\pgfpathlineto{\pgfqpoint{4.290719in}{2.344736in}}%
\pgfpathlineto{\pgfqpoint{4.298427in}{2.352592in}}%
\pgfpathlineto{\pgfqpoint{4.306129in}{2.360422in}}%
\pgfpathclose%
\pgfusepath{fill}%
\end{pgfscope}%
\begin{pgfscope}%
\pgfpathrectangle{\pgfqpoint{1.150000in}{0.150000in}}{\pgfqpoint{5.700000in}{5.700000in}}%
\pgfusepath{clip}%
\pgfsetbuttcap%
\pgfsetroundjoin%
\definecolor{currentfill}{rgb}{0.269944,0.014625,0.341379}%
\pgfsetfillcolor{currentfill}%
\pgfsetfillopacity{0.700000}%
\pgfsetlinewidth{0.000000pt}%
\definecolor{currentstroke}{rgb}{0.000000,0.000000,0.000000}%
\pgfsetstrokecolor{currentstroke}%
\pgfsetdash{}{0pt}%
\pgfpathmoveto{\pgfqpoint{4.082111in}{2.334257in}}%
\pgfpathlineto{\pgfqpoint{4.095596in}{2.331006in}}%
\pgfpathlineto{\pgfqpoint{4.109087in}{2.327783in}}%
\pgfpathlineto{\pgfqpoint{4.122585in}{2.324589in}}%
\pgfpathlineto{\pgfqpoint{4.136088in}{2.321424in}}%
\pgfpathlineto{\pgfqpoint{4.128317in}{2.313542in}}%
\pgfpathlineto{\pgfqpoint{4.120539in}{2.305649in}}%
\pgfpathlineto{\pgfqpoint{4.112756in}{2.297746in}}%
\pgfpathlineto{\pgfqpoint{4.104968in}{2.289835in}}%
\pgfpathlineto{\pgfqpoint{4.091452in}{2.293068in}}%
\pgfpathlineto{\pgfqpoint{4.077942in}{2.296330in}}%
\pgfpathlineto{\pgfqpoint{4.064438in}{2.299621in}}%
\pgfpathlineto{\pgfqpoint{4.050941in}{2.302940in}}%
\pgfpathlineto{\pgfqpoint{4.058742in}{2.310779in}}%
\pgfpathlineto{\pgfqpoint{4.066537in}{2.318612in}}%
\pgfpathlineto{\pgfqpoint{4.074327in}{2.326438in}}%
\pgfpathlineto{\pgfqpoint{4.082111in}{2.334257in}}%
\pgfpathclose%
\pgfusepath{fill}%
\end{pgfscope}%
\begin{pgfscope}%
\pgfpathrectangle{\pgfqpoint{1.150000in}{0.150000in}}{\pgfqpoint{5.700000in}{5.700000in}}%
\pgfusepath{clip}%
\pgfsetbuttcap%
\pgfsetroundjoin%
\definecolor{currentfill}{rgb}{0.277018,0.050344,0.375715}%
\pgfsetfillcolor{currentfill}%
\pgfsetfillopacity{0.700000}%
\pgfsetlinewidth{0.000000pt}%
\definecolor{currentstroke}{rgb}{0.000000,0.000000,0.000000}%
\pgfsetstrokecolor{currentstroke}%
\pgfsetdash{}{0pt}%
\pgfpathmoveto{\pgfqpoint{4.530159in}{2.390681in}}%
\pgfpathlineto{\pgfqpoint{4.543755in}{2.388235in}}%
\pgfpathlineto{\pgfqpoint{4.557358in}{2.385816in}}%
\pgfpathlineto{\pgfqpoint{4.570969in}{2.383424in}}%
\pgfpathlineto{\pgfqpoint{4.584587in}{2.381058in}}%
\pgfpathlineto{\pgfqpoint{4.576979in}{2.373395in}}%
\pgfpathlineto{\pgfqpoint{4.569366in}{2.365692in}}%
\pgfpathlineto{\pgfqpoint{4.561748in}{2.357949in}}%
\pgfpathlineto{\pgfqpoint{4.554123in}{2.350167in}}%
\pgfpathlineto{\pgfqpoint{4.540493in}{2.352548in}}%
\pgfpathlineto{\pgfqpoint{4.526870in}{2.354955in}}%
\pgfpathlineto{\pgfqpoint{4.513254in}{2.357389in}}%
\pgfpathlineto{\pgfqpoint{4.499646in}{2.359850in}}%
\pgfpathlineto{\pgfqpoint{4.507283in}{2.367612in}}%
\pgfpathlineto{\pgfqpoint{4.514914in}{2.375338in}}%
\pgfpathlineto{\pgfqpoint{4.522539in}{2.383027in}}%
\pgfpathlineto{\pgfqpoint{4.530159in}{2.390681in}}%
\pgfpathclose%
\pgfusepath{fill}%
\end{pgfscope}%
\begin{pgfscope}%
\pgfpathrectangle{\pgfqpoint{1.150000in}{0.150000in}}{\pgfqpoint{5.700000in}{5.700000in}}%
\pgfusepath{clip}%
\pgfsetbuttcap%
\pgfsetroundjoin%
\definecolor{currentfill}{rgb}{0.279566,0.067836,0.391917}%
\pgfsetfillcolor{currentfill}%
\pgfsetfillopacity{0.700000}%
\pgfsetlinewidth{0.000000pt}%
\definecolor{currentstroke}{rgb}{0.000000,0.000000,0.000000}%
\pgfsetstrokecolor{currentstroke}%
\pgfsetdash{}{0pt}%
\pgfpathmoveto{\pgfqpoint{4.754226in}{2.422953in}}%
\pgfpathlineto{\pgfqpoint{4.767883in}{2.420813in}}%
\pgfpathlineto{\pgfqpoint{4.781547in}{2.418700in}}%
\pgfpathlineto{\pgfqpoint{4.795219in}{2.416613in}}%
\pgfpathlineto{\pgfqpoint{4.808898in}{2.414551in}}%
\pgfpathlineto{\pgfqpoint{4.801377in}{2.407234in}}%
\pgfpathlineto{\pgfqpoint{4.793850in}{2.399872in}}%
\pgfpathlineto{\pgfqpoint{4.786317in}{2.392465in}}%
\pgfpathlineto{\pgfqpoint{4.778778in}{2.385011in}}%
\pgfpathlineto{\pgfqpoint{4.765086in}{2.387061in}}%
\pgfpathlineto{\pgfqpoint{4.751401in}{2.389137in}}%
\pgfpathlineto{\pgfqpoint{4.737724in}{2.391238in}}%
\pgfpathlineto{\pgfqpoint{4.724054in}{2.393366in}}%
\pgfpathlineto{\pgfqpoint{4.731606in}{2.400826in}}%
\pgfpathlineto{\pgfqpoint{4.739152in}{2.408243in}}%
\pgfpathlineto{\pgfqpoint{4.746692in}{2.415619in}}%
\pgfpathlineto{\pgfqpoint{4.754226in}{2.422953in}}%
\pgfpathclose%
\pgfusepath{fill}%
\end{pgfscope}%
\begin{pgfscope}%
\pgfpathrectangle{\pgfqpoint{1.150000in}{0.150000in}}{\pgfqpoint{5.700000in}{5.700000in}}%
\pgfusepath{clip}%
\pgfsetbuttcap%
\pgfsetroundjoin%
\definecolor{currentfill}{rgb}{0.280868,0.160771,0.472899}%
\pgfsetfillcolor{currentfill}%
\pgfsetfillopacity{0.700000}%
\pgfsetlinewidth{0.000000pt}%
\definecolor{currentstroke}{rgb}{0.000000,0.000000,0.000000}%
\pgfsetstrokecolor{currentstroke}%
\pgfsetdash{}{0pt}%
\pgfpathmoveto{\pgfqpoint{6.098287in}{2.591928in}}%
\pgfpathlineto{\pgfqpoint{6.112306in}{2.590374in}}%
\pgfpathlineto{\pgfqpoint{6.126334in}{2.588843in}}%
\pgfpathlineto{\pgfqpoint{6.140370in}{2.587336in}}%
\pgfpathlineto{\pgfqpoint{6.154414in}{2.585853in}}%
\pgfpathlineto{\pgfqpoint{6.147489in}{2.580987in}}%
\pgfpathlineto{\pgfqpoint{6.140561in}{2.576169in}}%
\pgfpathlineto{\pgfqpoint{6.133629in}{2.571393in}}%
\pgfpathlineto{\pgfqpoint{6.126692in}{2.566656in}}%
\pgfpathlineto{\pgfqpoint{6.112625in}{2.567966in}}%
\pgfpathlineto{\pgfqpoint{6.098566in}{2.569300in}}%
\pgfpathlineto{\pgfqpoint{6.084515in}{2.570658in}}%
\pgfpathlineto{\pgfqpoint{6.070473in}{2.572039in}}%
\pgfpathlineto{\pgfqpoint{6.077433in}{2.576945in}}%
\pgfpathlineto{\pgfqpoint{6.084388in}{2.581892in}}%
\pgfpathlineto{\pgfqpoint{6.091339in}{2.586884in}}%
\pgfpathlineto{\pgfqpoint{6.098287in}{2.591928in}}%
\pgfpathclose%
\pgfusepath{fill}%
\end{pgfscope}%
\begin{pgfscope}%
\pgfpathrectangle{\pgfqpoint{1.150000in}{0.150000in}}{\pgfqpoint{5.700000in}{5.700000in}}%
\pgfusepath{clip}%
\pgfsetbuttcap%
\pgfsetroundjoin%
\definecolor{currentfill}{rgb}{0.267004,0.004874,0.329415}%
\pgfsetfillcolor{currentfill}%
\pgfsetfillopacity{0.700000}%
\pgfsetlinewidth{0.000000pt}%
\definecolor{currentstroke}{rgb}{0.000000,0.000000,0.000000}%
\pgfsetstrokecolor{currentstroke}%
\pgfsetdash{}{0pt}%
\pgfpathmoveto{\pgfqpoint{3.858042in}{2.314668in}}%
\pgfpathlineto{\pgfqpoint{3.871479in}{2.310912in}}%
\pgfpathlineto{\pgfqpoint{3.884922in}{2.307186in}}%
\pgfpathlineto{\pgfqpoint{3.898370in}{2.303490in}}%
\pgfpathlineto{\pgfqpoint{3.911825in}{2.299825in}}%
\pgfpathlineto{\pgfqpoint{3.903970in}{2.292163in}}%
\pgfpathlineto{\pgfqpoint{3.896109in}{2.284513in}}%
\pgfpathlineto{\pgfqpoint{3.888243in}{2.276877in}}%
\pgfpathlineto{\pgfqpoint{3.880370in}{2.269256in}}%
\pgfpathlineto{\pgfqpoint{3.866902in}{2.273016in}}%
\pgfpathlineto{\pgfqpoint{3.853441in}{2.276807in}}%
\pgfpathlineto{\pgfqpoint{3.839985in}{2.280627in}}%
\pgfpathlineto{\pgfqpoint{3.826535in}{2.284478in}}%
\pgfpathlineto{\pgfqpoint{3.834421in}{2.291999in}}%
\pgfpathlineto{\pgfqpoint{3.842301in}{2.299539in}}%
\pgfpathlineto{\pgfqpoint{3.850174in}{2.307096in}}%
\pgfpathlineto{\pgfqpoint{3.858042in}{2.314668in}}%
\pgfpathclose%
\pgfusepath{fill}%
\end{pgfscope}%
\begin{pgfscope}%
\pgfpathrectangle{\pgfqpoint{1.150000in}{0.150000in}}{\pgfqpoint{5.700000in}{5.700000in}}%
\pgfusepath{clip}%
\pgfsetbuttcap%
\pgfsetroundjoin%
\definecolor{currentfill}{rgb}{0.283197,0.115680,0.436115}%
\pgfsetfillcolor{currentfill}%
\pgfsetfillopacity{0.700000}%
\pgfsetlinewidth{0.000000pt}%
\definecolor{currentstroke}{rgb}{0.000000,0.000000,0.000000}%
\pgfsetstrokecolor{currentstroke}%
\pgfsetdash{}{0pt}%
\pgfpathmoveto{\pgfqpoint{2.692186in}{2.507444in}}%
\pgfpathlineto{\pgfqpoint{2.705458in}{2.499955in}}%
\pgfpathlineto{\pgfqpoint{2.718733in}{2.492516in}}%
\pgfpathlineto{\pgfqpoint{2.732011in}{2.485127in}}%
\pgfpathlineto{\pgfqpoint{2.745291in}{2.477786in}}%
\pgfpathlineto{\pgfqpoint{2.736881in}{2.475281in}}%
\pgfpathlineto{\pgfqpoint{2.728458in}{2.472955in}}%
\pgfpathlineto{\pgfqpoint{2.720022in}{2.470812in}}%
\pgfpathlineto{\pgfqpoint{2.711574in}{2.468859in}}%
\pgfpathlineto{\pgfqpoint{2.698269in}{2.476402in}}%
\pgfpathlineto{\pgfqpoint{2.684966in}{2.483995in}}%
\pgfpathlineto{\pgfqpoint{2.671667in}{2.491637in}}%
\pgfpathlineto{\pgfqpoint{2.658370in}{2.499330in}}%
\pgfpathlineto{\pgfqpoint{2.666843in}{2.501075in}}%
\pgfpathlineto{\pgfqpoint{2.675304in}{2.503012in}}%
\pgfpathlineto{\pgfqpoint{2.683751in}{2.505137in}}%
\pgfpathlineto{\pgfqpoint{2.692186in}{2.507444in}}%
\pgfpathclose%
\pgfusepath{fill}%
\end{pgfscope}%
\begin{pgfscope}%
\pgfpathrectangle{\pgfqpoint{1.150000in}{0.150000in}}{\pgfqpoint{5.700000in}{5.700000in}}%
\pgfusepath{clip}%
\pgfsetbuttcap%
\pgfsetroundjoin%
\definecolor{currentfill}{rgb}{0.281924,0.089666,0.412415}%
\pgfsetfillcolor{currentfill}%
\pgfsetfillopacity{0.700000}%
\pgfsetlinewidth{0.000000pt}%
\definecolor{currentstroke}{rgb}{0.000000,0.000000,0.000000}%
\pgfsetstrokecolor{currentstroke}%
\pgfsetdash{}{0pt}%
\pgfpathmoveto{\pgfqpoint{4.978327in}{2.455552in}}%
\pgfpathlineto{\pgfqpoint{4.992047in}{2.453657in}}%
\pgfpathlineto{\pgfqpoint{5.005773in}{2.451788in}}%
\pgfpathlineto{\pgfqpoint{5.019508in}{2.449944in}}%
\pgfpathlineto{\pgfqpoint{5.033250in}{2.448126in}}%
\pgfpathlineto{\pgfqpoint{5.025820in}{2.441251in}}%
\pgfpathlineto{\pgfqpoint{5.018383in}{2.434333in}}%
\pgfpathlineto{\pgfqpoint{5.010941in}{2.427369in}}%
\pgfpathlineto{\pgfqpoint{5.003492in}{2.420358in}}%
\pgfpathlineto{\pgfqpoint{4.989736in}{2.422138in}}%
\pgfpathlineto{\pgfqpoint{4.975987in}{2.423943in}}%
\pgfpathlineto{\pgfqpoint{4.962247in}{2.425774in}}%
\pgfpathlineto{\pgfqpoint{4.948514in}{2.427630in}}%
\pgfpathlineto{\pgfqpoint{4.955977in}{2.434674in}}%
\pgfpathlineto{\pgfqpoint{4.963433in}{2.441675in}}%
\pgfpathlineto{\pgfqpoint{4.970883in}{2.448634in}}%
\pgfpathlineto{\pgfqpoint{4.978327in}{2.455552in}}%
\pgfpathclose%
\pgfusepath{fill}%
\end{pgfscope}%
\begin{pgfscope}%
\pgfpathrectangle{\pgfqpoint{1.150000in}{0.150000in}}{\pgfqpoint{5.700000in}{5.700000in}}%
\pgfusepath{clip}%
\pgfsetbuttcap%
\pgfsetroundjoin%
\definecolor{currentfill}{rgb}{0.281887,0.150881,0.465405}%
\pgfsetfillcolor{currentfill}%
\pgfsetfillopacity{0.700000}%
\pgfsetlinewidth{0.000000pt}%
\definecolor{currentstroke}{rgb}{0.000000,0.000000,0.000000}%
\pgfsetstrokecolor{currentstroke}%
\pgfsetdash{}{0pt}%
\pgfpathmoveto{\pgfqpoint{5.874478in}{2.569239in}}%
\pgfpathlineto{\pgfqpoint{5.888442in}{2.567732in}}%
\pgfpathlineto{\pgfqpoint{5.902413in}{2.566248in}}%
\pgfpathlineto{\pgfqpoint{5.916393in}{2.564788in}}%
\pgfpathlineto{\pgfqpoint{5.930381in}{2.563352in}}%
\pgfpathlineto{\pgfqpoint{5.923355in}{2.558272in}}%
\pgfpathlineto{\pgfqpoint{5.916324in}{2.553209in}}%
\pgfpathlineto{\pgfqpoint{5.909287in}{2.548159in}}%
\pgfpathlineto{\pgfqpoint{5.902245in}{2.543119in}}%
\pgfpathlineto{\pgfqpoint{5.888236in}{2.544408in}}%
\pgfpathlineto{\pgfqpoint{5.874235in}{2.545722in}}%
\pgfpathlineto{\pgfqpoint{5.860243in}{2.547059in}}%
\pgfpathlineto{\pgfqpoint{5.846259in}{2.548421in}}%
\pgfpathlineto{\pgfqpoint{5.853321in}{2.553603in}}%
\pgfpathlineto{\pgfqpoint{5.860379in}{2.558797in}}%
\pgfpathlineto{\pgfqpoint{5.867431in}{2.564008in}}%
\pgfpathlineto{\pgfqpoint{5.874478in}{2.569239in}}%
\pgfpathclose%
\pgfusepath{fill}%
\end{pgfscope}%
\begin{pgfscope}%
\pgfpathrectangle{\pgfqpoint{1.150000in}{0.150000in}}{\pgfqpoint{5.700000in}{5.700000in}}%
\pgfusepath{clip}%
\pgfsetbuttcap%
\pgfsetroundjoin%
\definecolor{currentfill}{rgb}{0.280267,0.073417,0.397163}%
\pgfsetfillcolor{currentfill}%
\pgfsetfillopacity{0.700000}%
\pgfsetlinewidth{0.000000pt}%
\definecolor{currentstroke}{rgb}{0.000000,0.000000,0.000000}%
\pgfsetstrokecolor{currentstroke}%
\pgfsetdash{}{0pt}%
\pgfpathmoveto{\pgfqpoint{2.884982in}{2.433995in}}%
\pgfpathlineto{\pgfqpoint{2.898268in}{2.427265in}}%
\pgfpathlineto{\pgfqpoint{2.911557in}{2.420578in}}%
\pgfpathlineto{\pgfqpoint{2.924850in}{2.413936in}}%
\pgfpathlineto{\pgfqpoint{2.938147in}{2.407338in}}%
\pgfpathlineto{\pgfqpoint{2.929850in}{2.403615in}}%
\pgfpathlineto{\pgfqpoint{2.921543in}{2.400041in}}%
\pgfpathlineto{\pgfqpoint{2.913225in}{2.396621in}}%
\pgfpathlineto{\pgfqpoint{2.904896in}{2.393360in}}%
\pgfpathlineto{\pgfqpoint{2.891578in}{2.400147in}}%
\pgfpathlineto{\pgfqpoint{2.878263in}{2.406978in}}%
\pgfpathlineto{\pgfqpoint{2.864951in}{2.413853in}}%
\pgfpathlineto{\pgfqpoint{2.851643in}{2.420773in}}%
\pgfpathlineto{\pgfqpoint{2.859995in}{2.423840in}}%
\pgfpathlineto{\pgfqpoint{2.868335in}{2.427069in}}%
\pgfpathlineto{\pgfqpoint{2.876664in}{2.430456in}}%
\pgfpathlineto{\pgfqpoint{2.884982in}{2.433995in}}%
\pgfpathclose%
\pgfusepath{fill}%
\end{pgfscope}%
\begin{pgfscope}%
\pgfpathrectangle{\pgfqpoint{1.150000in}{0.150000in}}{\pgfqpoint{5.700000in}{5.700000in}}%
\pgfusepath{clip}%
\pgfsetbuttcap%
\pgfsetroundjoin%
\definecolor{currentfill}{rgb}{0.282910,0.105393,0.426902}%
\pgfsetfillcolor{currentfill}%
\pgfsetfillopacity{0.700000}%
\pgfsetlinewidth{0.000000pt}%
\definecolor{currentstroke}{rgb}{0.000000,0.000000,0.000000}%
\pgfsetstrokecolor{currentstroke}%
\pgfsetdash{}{0pt}%
\pgfpathmoveto{\pgfqpoint{5.202440in}{2.487185in}}%
\pgfpathlineto{\pgfqpoint{5.216222in}{2.485475in}}%
\pgfpathlineto{\pgfqpoint{5.230011in}{2.483790in}}%
\pgfpathlineto{\pgfqpoint{5.243809in}{2.482130in}}%
\pgfpathlineto{\pgfqpoint{5.257614in}{2.480494in}}%
\pgfpathlineto{\pgfqpoint{5.250280in}{2.474112in}}%
\pgfpathlineto{\pgfqpoint{5.242939in}{2.467693in}}%
\pgfpathlineto{\pgfqpoint{5.235592in}{2.461234in}}%
\pgfpathlineto{\pgfqpoint{5.228239in}{2.454734in}}%
\pgfpathlineto{\pgfqpoint{5.214418in}{2.456303in}}%
\pgfpathlineto{\pgfqpoint{5.200605in}{2.457898in}}%
\pgfpathlineto{\pgfqpoint{5.186800in}{2.459517in}}%
\pgfpathlineto{\pgfqpoint{5.173003in}{2.461162in}}%
\pgfpathlineto{\pgfqpoint{5.180372in}{2.467724in}}%
\pgfpathlineto{\pgfqpoint{5.187734in}{2.474247in}}%
\pgfpathlineto{\pgfqpoint{5.195090in}{2.480733in}}%
\pgfpathlineto{\pgfqpoint{5.202440in}{2.487185in}}%
\pgfpathclose%
\pgfusepath{fill}%
\end{pgfscope}%
\begin{pgfscope}%
\pgfpathrectangle{\pgfqpoint{1.150000in}{0.150000in}}{\pgfqpoint{5.700000in}{5.700000in}}%
\pgfusepath{clip}%
\pgfsetbuttcap%
\pgfsetroundjoin%
\definecolor{currentfill}{rgb}{0.282884,0.135920,0.453427}%
\pgfsetfillcolor{currentfill}%
\pgfsetfillopacity{0.700000}%
\pgfsetlinewidth{0.000000pt}%
\definecolor{currentstroke}{rgb}{0.000000,0.000000,0.000000}%
\pgfsetstrokecolor{currentstroke}%
\pgfsetdash{}{0pt}%
\pgfpathmoveto{\pgfqpoint{5.650552in}{2.544341in}}%
\pgfpathlineto{\pgfqpoint{5.664456in}{2.542823in}}%
\pgfpathlineto{\pgfqpoint{5.678369in}{2.541330in}}%
\pgfpathlineto{\pgfqpoint{5.692290in}{2.539861in}}%
\pgfpathlineto{\pgfqpoint{5.706219in}{2.538416in}}%
\pgfpathlineto{\pgfqpoint{5.699089in}{2.532979in}}%
\pgfpathlineto{\pgfqpoint{5.691953in}{2.527536in}}%
\pgfpathlineto{\pgfqpoint{5.684810in}{2.522082in}}%
\pgfpathlineto{\pgfqpoint{5.677662in}{2.516614in}}%
\pgfpathlineto{\pgfqpoint{5.663714in}{2.517939in}}%
\pgfpathlineto{\pgfqpoint{5.649774in}{2.519289in}}%
\pgfpathlineto{\pgfqpoint{5.635842in}{2.520663in}}%
\pgfpathlineto{\pgfqpoint{5.621919in}{2.522061in}}%
\pgfpathlineto{\pgfqpoint{5.629086in}{2.527644in}}%
\pgfpathlineto{\pgfqpoint{5.636247in}{2.533215in}}%
\pgfpathlineto{\pgfqpoint{5.643402in}{2.538780in}}%
\pgfpathlineto{\pgfqpoint{5.650552in}{2.544341in}}%
\pgfpathclose%
\pgfusepath{fill}%
\end{pgfscope}%
\begin{pgfscope}%
\pgfpathrectangle{\pgfqpoint{1.150000in}{0.150000in}}{\pgfqpoint{5.700000in}{5.700000in}}%
\pgfusepath{clip}%
\pgfsetbuttcap%
\pgfsetroundjoin%
\definecolor{currentfill}{rgb}{0.283229,0.120777,0.440584}%
\pgfsetfillcolor{currentfill}%
\pgfsetfillopacity{0.700000}%
\pgfsetlinewidth{0.000000pt}%
\definecolor{currentstroke}{rgb}{0.000000,0.000000,0.000000}%
\pgfsetstrokecolor{currentstroke}%
\pgfsetdash{}{0pt}%
\pgfpathmoveto{\pgfqpoint{5.426527in}{2.516951in}}%
\pgfpathlineto{\pgfqpoint{5.440371in}{2.515367in}}%
\pgfpathlineto{\pgfqpoint{5.454223in}{2.513806in}}%
\pgfpathlineto{\pgfqpoint{5.468083in}{2.512271in}}%
\pgfpathlineto{\pgfqpoint{5.481951in}{2.510759in}}%
\pgfpathlineto{\pgfqpoint{5.474717in}{2.504873in}}%
\pgfpathlineto{\pgfqpoint{5.467477in}{2.498961in}}%
\pgfpathlineto{\pgfqpoint{5.460231in}{2.493022in}}%
\pgfpathlineto{\pgfqpoint{5.452978in}{2.487051in}}%
\pgfpathlineto{\pgfqpoint{5.439093in}{2.488469in}}%
\pgfpathlineto{\pgfqpoint{5.425216in}{2.489912in}}%
\pgfpathlineto{\pgfqpoint{5.411347in}{2.491380in}}%
\pgfpathlineto{\pgfqpoint{5.397486in}{2.492872in}}%
\pgfpathlineto{\pgfqpoint{5.404756in}{2.498931in}}%
\pgfpathlineto{\pgfqpoint{5.412019in}{2.504961in}}%
\pgfpathlineto{\pgfqpoint{5.419276in}{2.510967in}}%
\pgfpathlineto{\pgfqpoint{5.426527in}{2.516951in}}%
\pgfpathclose%
\pgfusepath{fill}%
\end{pgfscope}%
\begin{pgfscope}%
\pgfpathrectangle{\pgfqpoint{1.150000in}{0.150000in}}{\pgfqpoint{5.700000in}{5.700000in}}%
\pgfusepath{clip}%
\pgfsetbuttcap%
\pgfsetroundjoin%
\definecolor{currentfill}{rgb}{0.269944,0.014625,0.341379}%
\pgfsetfillcolor{currentfill}%
\pgfsetfillopacity{0.700000}%
\pgfsetlinewidth{0.000000pt}%
\definecolor{currentstroke}{rgb}{0.000000,0.000000,0.000000}%
\pgfsetstrokecolor{currentstroke}%
\pgfsetdash{}{0pt}%
\pgfpathmoveto{\pgfqpoint{3.355857in}{2.327551in}}%
\pgfpathlineto{\pgfqpoint{3.369201in}{2.322419in}}%
\pgfpathlineto{\pgfqpoint{3.382551in}{2.317322in}}%
\pgfpathlineto{\pgfqpoint{3.395905in}{2.312261in}}%
\pgfpathlineto{\pgfqpoint{3.409264in}{2.307235in}}%
\pgfpathlineto{\pgfqpoint{3.401205in}{2.301047in}}%
\pgfpathlineto{\pgfqpoint{3.393138in}{2.294937in}}%
\pgfpathlineto{\pgfqpoint{3.385064in}{2.288909in}}%
\pgfpathlineto{\pgfqpoint{3.376982in}{2.282965in}}%
\pgfpathlineto{\pgfqpoint{3.363607in}{2.288139in}}%
\pgfpathlineto{\pgfqpoint{3.350236in}{2.293348in}}%
\pgfpathlineto{\pgfqpoint{3.336870in}{2.298592in}}%
\pgfpathlineto{\pgfqpoint{3.323508in}{2.303872in}}%
\pgfpathlineto{\pgfqpoint{3.331607in}{2.309664in}}%
\pgfpathlineto{\pgfqpoint{3.339698in}{2.315543in}}%
\pgfpathlineto{\pgfqpoint{3.347781in}{2.321507in}}%
\pgfpathlineto{\pgfqpoint{3.355857in}{2.327551in}}%
\pgfpathclose%
\pgfusepath{fill}%
\end{pgfscope}%
\begin{pgfscope}%
\pgfpathrectangle{\pgfqpoint{1.150000in}{0.150000in}}{\pgfqpoint{5.700000in}{5.700000in}}%
\pgfusepath{clip}%
\pgfsetbuttcap%
\pgfsetroundjoin%
\definecolor{currentfill}{rgb}{0.267004,0.004874,0.329415}%
\pgfsetfillcolor{currentfill}%
\pgfsetfillopacity{0.700000}%
\pgfsetlinewidth{0.000000pt}%
\definecolor{currentstroke}{rgb}{0.000000,0.000000,0.000000}%
\pgfsetstrokecolor{currentstroke}%
\pgfsetdash{}{0pt}%
\pgfpathmoveto{\pgfqpoint{3.494851in}{2.313474in}}%
\pgfpathlineto{\pgfqpoint{3.508219in}{2.308754in}}%
\pgfpathlineto{\pgfqpoint{3.521592in}{2.304068in}}%
\pgfpathlineto{\pgfqpoint{3.534971in}{2.299415in}}%
\pgfpathlineto{\pgfqpoint{3.548355in}{2.294795in}}%
\pgfpathlineto{\pgfqpoint{3.540356in}{2.288071in}}%
\pgfpathlineto{\pgfqpoint{3.532350in}{2.281405in}}%
\pgfpathlineto{\pgfqpoint{3.524336in}{2.274800in}}%
\pgfpathlineto{\pgfqpoint{3.516316in}{2.268259in}}%
\pgfpathlineto{\pgfqpoint{3.502917in}{2.273013in}}%
\pgfpathlineto{\pgfqpoint{3.489523in}{2.277800in}}%
\pgfpathlineto{\pgfqpoint{3.476134in}{2.282621in}}%
\pgfpathlineto{\pgfqpoint{3.462750in}{2.287475in}}%
\pgfpathlineto{\pgfqpoint{3.470786in}{2.293876in}}%
\pgfpathlineto{\pgfqpoint{3.478814in}{2.300345in}}%
\pgfpathlineto{\pgfqpoint{3.486836in}{2.306879in}}%
\pgfpathlineto{\pgfqpoint{3.494851in}{2.313474in}}%
\pgfpathclose%
\pgfusepath{fill}%
\end{pgfscope}%
\begin{pgfscope}%
\pgfpathrectangle{\pgfqpoint{1.150000in}{0.150000in}}{\pgfqpoint{5.700000in}{5.700000in}}%
\pgfusepath{clip}%
\pgfsetbuttcap%
\pgfsetroundjoin%
\definecolor{currentfill}{rgb}{0.272594,0.025563,0.353093}%
\pgfsetfillcolor{currentfill}%
\pgfsetfillopacity{0.700000}%
\pgfsetlinewidth{0.000000pt}%
\definecolor{currentstroke}{rgb}{0.000000,0.000000,0.000000}%
\pgfsetstrokecolor{currentstroke}%
\pgfsetdash{}{0pt}%
\pgfpathmoveto{\pgfqpoint{3.216782in}{2.347423in}}%
\pgfpathlineto{\pgfqpoint{3.230107in}{2.341849in}}%
\pgfpathlineto{\pgfqpoint{3.243437in}{2.336313in}}%
\pgfpathlineto{\pgfqpoint{3.256770in}{2.330815in}}%
\pgfpathlineto{\pgfqpoint{3.270109in}{2.325353in}}%
\pgfpathlineto{\pgfqpoint{3.261984in}{2.319810in}}%
\pgfpathlineto{\pgfqpoint{3.253852in}{2.314365in}}%
\pgfpathlineto{\pgfqpoint{3.245711in}{2.309024in}}%
\pgfpathlineto{\pgfqpoint{3.237562in}{2.303789in}}%
\pgfpathlineto{\pgfqpoint{3.224205in}{2.309411in}}%
\pgfpathlineto{\pgfqpoint{3.210853in}{2.315071in}}%
\pgfpathlineto{\pgfqpoint{3.197505in}{2.320768in}}%
\pgfpathlineto{\pgfqpoint{3.184162in}{2.326503in}}%
\pgfpathlineto{\pgfqpoint{3.192330in}{2.331572in}}%
\pgfpathlineto{\pgfqpoint{3.200489in}{2.336751in}}%
\pgfpathlineto{\pgfqpoint{3.208640in}{2.342036in}}%
\pgfpathlineto{\pgfqpoint{3.216782in}{2.347423in}}%
\pgfpathclose%
\pgfusepath{fill}%
\end{pgfscope}%
\begin{pgfscope}%
\pgfpathrectangle{\pgfqpoint{1.150000in}{0.150000in}}{\pgfqpoint{5.700000in}{5.700000in}}%
\pgfusepath{clip}%
\pgfsetbuttcap%
\pgfsetroundjoin%
\definecolor{currentfill}{rgb}{0.271305,0.019942,0.347269}%
\pgfsetfillcolor{currentfill}%
\pgfsetfillopacity{0.700000}%
\pgfsetlinewidth{0.000000pt}%
\definecolor{currentstroke}{rgb}{0.000000,0.000000,0.000000}%
\pgfsetstrokecolor{currentstroke}%
\pgfsetdash{}{0pt}%
\pgfpathmoveto{\pgfqpoint{4.221150in}{2.340660in}}%
\pgfpathlineto{\pgfqpoint{4.234674in}{2.337692in}}%
\pgfpathlineto{\pgfqpoint{4.248205in}{2.334751in}}%
\pgfpathlineto{\pgfqpoint{4.261742in}{2.331838in}}%
\pgfpathlineto{\pgfqpoint{4.275286in}{2.328953in}}%
\pgfpathlineto{\pgfqpoint{4.267562in}{2.321027in}}%
\pgfpathlineto{\pgfqpoint{4.259831in}{2.313079in}}%
\pgfpathlineto{\pgfqpoint{4.252095in}{2.305110in}}%
\pgfpathlineto{\pgfqpoint{4.244353in}{2.297120in}}%
\pgfpathlineto{\pgfqpoint{4.230797in}{2.300060in}}%
\pgfpathlineto{\pgfqpoint{4.217248in}{2.303028in}}%
\pgfpathlineto{\pgfqpoint{4.203705in}{2.306024in}}%
\pgfpathlineto{\pgfqpoint{4.190169in}{2.309047in}}%
\pgfpathlineto{\pgfqpoint{4.197922in}{2.316977in}}%
\pgfpathlineto{\pgfqpoint{4.205671in}{2.324890in}}%
\pgfpathlineto{\pgfqpoint{4.213413in}{2.332784in}}%
\pgfpathlineto{\pgfqpoint{4.221150in}{2.340660in}}%
\pgfpathclose%
\pgfusepath{fill}%
\end{pgfscope}%
\begin{pgfscope}%
\pgfpathrectangle{\pgfqpoint{1.150000in}{0.150000in}}{\pgfqpoint{5.700000in}{5.700000in}}%
\pgfusepath{clip}%
\pgfsetbuttcap%
\pgfsetroundjoin%
\definecolor{currentfill}{rgb}{0.274952,0.037752,0.364543}%
\pgfsetfillcolor{currentfill}%
\pgfsetfillopacity{0.700000}%
\pgfsetlinewidth{0.000000pt}%
\definecolor{currentstroke}{rgb}{0.000000,0.000000,0.000000}%
\pgfsetstrokecolor{currentstroke}%
\pgfsetdash{}{0pt}%
\pgfpathmoveto{\pgfqpoint{4.445281in}{2.369961in}}%
\pgfpathlineto{\pgfqpoint{4.458862in}{2.367393in}}%
\pgfpathlineto{\pgfqpoint{4.472449in}{2.364852in}}%
\pgfpathlineto{\pgfqpoint{4.486044in}{2.362337in}}%
\pgfpathlineto{\pgfqpoint{4.499646in}{2.359850in}}%
\pgfpathlineto{\pgfqpoint{4.492003in}{2.352052in}}%
\pgfpathlineto{\pgfqpoint{4.484354in}{2.344218in}}%
\pgfpathlineto{\pgfqpoint{4.476700in}{2.336348in}}%
\pgfpathlineto{\pgfqpoint{4.469040in}{2.328443in}}%
\pgfpathlineto{\pgfqpoint{4.455426in}{2.330958in}}%
\pgfpathlineto{\pgfqpoint{4.441819in}{2.333501in}}%
\pgfpathlineto{\pgfqpoint{4.428219in}{2.336071in}}%
\pgfpathlineto{\pgfqpoint{4.414626in}{2.338667in}}%
\pgfpathlineto{\pgfqpoint{4.422299in}{2.346539in}}%
\pgfpathlineto{\pgfqpoint{4.429965in}{2.354379in}}%
\pgfpathlineto{\pgfqpoint{4.437626in}{2.362187in}}%
\pgfpathlineto{\pgfqpoint{4.445281in}{2.369961in}}%
\pgfpathclose%
\pgfusepath{fill}%
\end{pgfscope}%
\begin{pgfscope}%
\pgfpathrectangle{\pgfqpoint{1.150000in}{0.150000in}}{\pgfqpoint{5.700000in}{5.700000in}}%
\pgfusepath{clip}%
\pgfsetbuttcap%
\pgfsetroundjoin%
\definecolor{currentfill}{rgb}{0.267004,0.004874,0.329415}%
\pgfsetfillcolor{currentfill}%
\pgfsetfillopacity{0.700000}%
\pgfsetlinewidth{0.000000pt}%
\definecolor{currentstroke}{rgb}{0.000000,0.000000,0.000000}%
\pgfsetstrokecolor{currentstroke}%
\pgfsetdash{}{0pt}%
\pgfpathmoveto{\pgfqpoint{3.633815in}{2.304551in}}%
\pgfpathlineto{\pgfqpoint{3.647211in}{2.300217in}}%
\pgfpathlineto{\pgfqpoint{3.660612in}{2.295915in}}%
\pgfpathlineto{\pgfqpoint{3.674019in}{2.291645in}}%
\pgfpathlineto{\pgfqpoint{3.687431in}{2.287406in}}%
\pgfpathlineto{\pgfqpoint{3.679487in}{2.280246in}}%
\pgfpathlineto{\pgfqpoint{3.671537in}{2.273126in}}%
\pgfpathlineto{\pgfqpoint{3.663580in}{2.266049in}}%
\pgfpathlineto{\pgfqpoint{3.655617in}{2.259018in}}%
\pgfpathlineto{\pgfqpoint{3.642191in}{2.263377in}}%
\pgfpathlineto{\pgfqpoint{3.628769in}{2.267769in}}%
\pgfpathlineto{\pgfqpoint{3.615354in}{2.272192in}}%
\pgfpathlineto{\pgfqpoint{3.601943in}{2.276647in}}%
\pgfpathlineto{\pgfqpoint{3.609921in}{2.283553in}}%
\pgfpathlineto{\pgfqpoint{3.617892in}{2.290507in}}%
\pgfpathlineto{\pgfqpoint{3.625857in}{2.297508in}}%
\pgfpathlineto{\pgfqpoint{3.633815in}{2.304551in}}%
\pgfpathclose%
\pgfusepath{fill}%
\end{pgfscope}%
\begin{pgfscope}%
\pgfpathrectangle{\pgfqpoint{1.150000in}{0.150000in}}{\pgfqpoint{5.700000in}{5.700000in}}%
\pgfusepath{clip}%
\pgfsetbuttcap%
\pgfsetroundjoin%
\definecolor{currentfill}{rgb}{0.268510,0.009605,0.335427}%
\pgfsetfillcolor{currentfill}%
\pgfsetfillopacity{0.700000}%
\pgfsetlinewidth{0.000000pt}%
\definecolor{currentstroke}{rgb}{0.000000,0.000000,0.000000}%
\pgfsetstrokecolor{currentstroke}%
\pgfsetdash{}{0pt}%
\pgfpathmoveto{\pgfqpoint{3.997015in}{2.316509in}}%
\pgfpathlineto{\pgfqpoint{4.010487in}{2.313073in}}%
\pgfpathlineto{\pgfqpoint{4.023966in}{2.309666in}}%
\pgfpathlineto{\pgfqpoint{4.037450in}{2.306289in}}%
\pgfpathlineto{\pgfqpoint{4.050941in}{2.302940in}}%
\pgfpathlineto{\pgfqpoint{4.043134in}{2.295098in}}%
\pgfpathlineto{\pgfqpoint{4.035322in}{2.287254in}}%
\pgfpathlineto{\pgfqpoint{4.027504in}{2.279409in}}%
\pgfpathlineto{\pgfqpoint{4.019680in}{2.271565in}}%
\pgfpathlineto{\pgfqpoint{4.006177in}{2.274995in}}%
\pgfpathlineto{\pgfqpoint{3.992680in}{2.278454in}}%
\pgfpathlineto{\pgfqpoint{3.979189in}{2.281942in}}%
\pgfpathlineto{\pgfqpoint{3.965704in}{2.285460in}}%
\pgfpathlineto{\pgfqpoint{3.973540in}{2.293217in}}%
\pgfpathlineto{\pgfqpoint{3.981371in}{2.300979in}}%
\pgfpathlineto{\pgfqpoint{3.989196in}{2.308744in}}%
\pgfpathlineto{\pgfqpoint{3.997015in}{2.316509in}}%
\pgfpathclose%
\pgfusepath{fill}%
\end{pgfscope}%
\begin{pgfscope}%
\pgfpathrectangle{\pgfqpoint{1.150000in}{0.150000in}}{\pgfqpoint{5.700000in}{5.700000in}}%
\pgfusepath{clip}%
\pgfsetbuttcap%
\pgfsetroundjoin%
\definecolor{currentfill}{rgb}{0.278791,0.062145,0.386592}%
\pgfsetfillcolor{currentfill}%
\pgfsetfillopacity{0.700000}%
\pgfsetlinewidth{0.000000pt}%
\definecolor{currentstroke}{rgb}{0.000000,0.000000,0.000000}%
\pgfsetstrokecolor{currentstroke}%
\pgfsetdash{}{0pt}%
\pgfpathmoveto{\pgfqpoint{4.669447in}{2.402137in}}%
\pgfpathlineto{\pgfqpoint{4.683088in}{2.399905in}}%
\pgfpathlineto{\pgfqpoint{4.696736in}{2.397699in}}%
\pgfpathlineto{\pgfqpoint{4.710391in}{2.395519in}}%
\pgfpathlineto{\pgfqpoint{4.724054in}{2.393366in}}%
\pgfpathlineto{\pgfqpoint{4.716496in}{2.385862in}}%
\pgfpathlineto{\pgfqpoint{4.708932in}{2.378314in}}%
\pgfpathlineto{\pgfqpoint{4.701361in}{2.370721in}}%
\pgfpathlineto{\pgfqpoint{4.693785in}{2.363083in}}%
\pgfpathlineto{\pgfqpoint{4.680110in}{2.365238in}}%
\pgfpathlineto{\pgfqpoint{4.666442in}{2.367419in}}%
\pgfpathlineto{\pgfqpoint{4.652781in}{2.369627in}}%
\pgfpathlineto{\pgfqpoint{4.639128in}{2.371860in}}%
\pgfpathlineto{\pgfqpoint{4.646717in}{2.379492in}}%
\pgfpathlineto{\pgfqpoint{4.654300in}{2.387081in}}%
\pgfpathlineto{\pgfqpoint{4.661876in}{2.394630in}}%
\pgfpathlineto{\pgfqpoint{4.669447in}{2.402137in}}%
\pgfpathclose%
\pgfusepath{fill}%
\end{pgfscope}%
\begin{pgfscope}%
\pgfpathrectangle{\pgfqpoint{1.150000in}{0.150000in}}{\pgfqpoint{5.700000in}{5.700000in}}%
\pgfusepath{clip}%
\pgfsetbuttcap%
\pgfsetroundjoin%
\definecolor{currentfill}{rgb}{0.276022,0.044167,0.370164}%
\pgfsetfillcolor{currentfill}%
\pgfsetfillopacity{0.700000}%
\pgfsetlinewidth{0.000000pt}%
\definecolor{currentstroke}{rgb}{0.000000,0.000000,0.000000}%
\pgfsetstrokecolor{currentstroke}%
\pgfsetdash{}{0pt}%
\pgfpathmoveto{\pgfqpoint{3.077568in}{2.373774in}}%
\pgfpathlineto{\pgfqpoint{3.090878in}{2.367727in}}%
\pgfpathlineto{\pgfqpoint{3.104192in}{2.361720in}}%
\pgfpathlineto{\pgfqpoint{3.117510in}{2.355753in}}%
\pgfpathlineto{\pgfqpoint{3.130832in}{2.349825in}}%
\pgfpathlineto{\pgfqpoint{3.122636in}{2.345040in}}%
\pgfpathlineto{\pgfqpoint{3.114431in}{2.340376in}}%
\pgfpathlineto{\pgfqpoint{3.106216in}{2.335839in}}%
\pgfpathlineto{\pgfqpoint{3.097992in}{2.331431in}}%
\pgfpathlineto{\pgfqpoint{3.084650in}{2.337533in}}%
\pgfpathlineto{\pgfqpoint{3.071312in}{2.343675in}}%
\pgfpathlineto{\pgfqpoint{3.057979in}{2.349857in}}%
\pgfpathlineto{\pgfqpoint{3.044649in}{2.356078in}}%
\pgfpathlineto{\pgfqpoint{3.052893in}{2.360306in}}%
\pgfpathlineto{\pgfqpoint{3.061127in}{2.364668in}}%
\pgfpathlineto{\pgfqpoint{3.069352in}{2.369158in}}%
\pgfpathlineto{\pgfqpoint{3.077568in}{2.373774in}}%
\pgfpathclose%
\pgfusepath{fill}%
\end{pgfscope}%
\begin{pgfscope}%
\pgfpathrectangle{\pgfqpoint{1.150000in}{0.150000in}}{\pgfqpoint{5.700000in}{5.700000in}}%
\pgfusepath{clip}%
\pgfsetbuttcap%
\pgfsetroundjoin%
\definecolor{currentfill}{rgb}{0.281446,0.084320,0.407414}%
\pgfsetfillcolor{currentfill}%
\pgfsetfillopacity{0.700000}%
\pgfsetlinewidth{0.000000pt}%
\definecolor{currentstroke}{rgb}{0.000000,0.000000,0.000000}%
\pgfsetstrokecolor{currentstroke}%
\pgfsetdash{}{0pt}%
\pgfpathmoveto{\pgfqpoint{4.893657in}{2.435308in}}%
\pgfpathlineto{\pgfqpoint{4.907360in}{2.433350in}}%
\pgfpathlineto{\pgfqpoint{4.921070in}{2.431418in}}%
\pgfpathlineto{\pgfqpoint{4.934788in}{2.429511in}}%
\pgfpathlineto{\pgfqpoint{4.948514in}{2.427630in}}%
\pgfpathlineto{\pgfqpoint{4.941044in}{2.420540in}}%
\pgfpathlineto{\pgfqpoint{4.933569in}{2.413403in}}%
\pgfpathlineto{\pgfqpoint{4.926087in}{2.406219in}}%
\pgfpathlineto{\pgfqpoint{4.918599in}{2.398986in}}%
\pgfpathlineto{\pgfqpoint{4.904860in}{2.400842in}}%
\pgfpathlineto{\pgfqpoint{4.891129in}{2.402723in}}%
\pgfpathlineto{\pgfqpoint{4.877405in}{2.404630in}}%
\pgfpathlineto{\pgfqpoint{4.863688in}{2.406563in}}%
\pgfpathlineto{\pgfqpoint{4.871190in}{2.413817in}}%
\pgfpathlineto{\pgfqpoint{4.878685in}{2.421025in}}%
\pgfpathlineto{\pgfqpoint{4.886174in}{2.428188in}}%
\pgfpathlineto{\pgfqpoint{4.893657in}{2.435308in}}%
\pgfpathclose%
\pgfusepath{fill}%
\end{pgfscope}%
\begin{pgfscope}%
\pgfpathrectangle{\pgfqpoint{1.150000in}{0.150000in}}{\pgfqpoint{5.700000in}{5.700000in}}%
\pgfusepath{clip}%
\pgfsetbuttcap%
\pgfsetroundjoin%
\definecolor{currentfill}{rgb}{0.281412,0.155834,0.469201}%
\pgfsetfillcolor{currentfill}%
\pgfsetfillopacity{0.700000}%
\pgfsetlinewidth{0.000000pt}%
\definecolor{currentstroke}{rgb}{0.000000,0.000000,0.000000}%
\pgfsetstrokecolor{currentstroke}%
\pgfsetdash{}{0pt}%
\pgfpathmoveto{\pgfqpoint{6.014390in}{2.577803in}}%
\pgfpathlineto{\pgfqpoint{6.028398in}{2.576326in}}%
\pgfpathlineto{\pgfqpoint{6.042415in}{2.574873in}}%
\pgfpathlineto{\pgfqpoint{6.056440in}{2.573444in}}%
\pgfpathlineto{\pgfqpoint{6.070473in}{2.572039in}}%
\pgfpathlineto{\pgfqpoint{6.063509in}{2.567169in}}%
\pgfpathlineto{\pgfqpoint{6.056541in}{2.562330in}}%
\pgfpathlineto{\pgfqpoint{6.049568in}{2.557516in}}%
\pgfpathlineto{\pgfqpoint{6.042590in}{2.552723in}}%
\pgfpathlineto{\pgfqpoint{6.028534in}{2.553968in}}%
\pgfpathlineto{\pgfqpoint{6.014487in}{2.555237in}}%
\pgfpathlineto{\pgfqpoint{6.000448in}{2.556530in}}%
\pgfpathlineto{\pgfqpoint{5.986418in}{2.557847in}}%
\pgfpathlineto{\pgfqpoint{5.993418in}{2.562795in}}%
\pgfpathlineto{\pgfqpoint{6.000413in}{2.567767in}}%
\pgfpathlineto{\pgfqpoint{6.007404in}{2.572768in}}%
\pgfpathlineto{\pgfqpoint{6.014390in}{2.577803in}}%
\pgfpathclose%
\pgfusepath{fill}%
\end{pgfscope}%
\begin{pgfscope}%
\pgfpathrectangle{\pgfqpoint{1.150000in}{0.150000in}}{\pgfqpoint{5.700000in}{5.700000in}}%
\pgfusepath{clip}%
\pgfsetbuttcap%
\pgfsetroundjoin%
\definecolor{currentfill}{rgb}{0.267004,0.004874,0.329415}%
\pgfsetfillcolor{currentfill}%
\pgfsetfillopacity{0.700000}%
\pgfsetlinewidth{0.000000pt}%
\definecolor{currentstroke}{rgb}{0.000000,0.000000,0.000000}%
\pgfsetstrokecolor{currentstroke}%
\pgfsetdash{}{0pt}%
\pgfpathmoveto{\pgfqpoint{3.772794in}{2.300188in}}%
\pgfpathlineto{\pgfqpoint{3.786220in}{2.296214in}}%
\pgfpathlineto{\pgfqpoint{3.799653in}{2.292271in}}%
\pgfpathlineto{\pgfqpoint{3.813091in}{2.288359in}}%
\pgfpathlineto{\pgfqpoint{3.826535in}{2.284478in}}%
\pgfpathlineto{\pgfqpoint{3.818643in}{2.276978in}}%
\pgfpathlineto{\pgfqpoint{3.810746in}{2.269501in}}%
\pgfpathlineto{\pgfqpoint{3.802842in}{2.262050in}}%
\pgfpathlineto{\pgfqpoint{3.794932in}{2.254627in}}%
\pgfpathlineto{\pgfqpoint{3.781474in}{2.258616in}}%
\pgfpathlineto{\pgfqpoint{3.768022in}{2.262636in}}%
\pgfpathlineto{\pgfqpoint{3.754576in}{2.266687in}}%
\pgfpathlineto{\pgfqpoint{3.741136in}{2.270768in}}%
\pgfpathlineto{\pgfqpoint{3.749060in}{2.278078in}}%
\pgfpathlineto{\pgfqpoint{3.756977in}{2.285420in}}%
\pgfpathlineto{\pgfqpoint{3.764888in}{2.292790in}}%
\pgfpathlineto{\pgfqpoint{3.772794in}{2.300188in}}%
\pgfpathclose%
\pgfusepath{fill}%
\end{pgfscope}%
\begin{pgfscope}%
\pgfpathrectangle{\pgfqpoint{1.150000in}{0.150000in}}{\pgfqpoint{5.700000in}{5.700000in}}%
\pgfusepath{clip}%
\pgfsetbuttcap%
\pgfsetroundjoin%
\definecolor{currentfill}{rgb}{0.282656,0.100196,0.422160}%
\pgfsetfillcolor{currentfill}%
\pgfsetfillopacity{0.700000}%
\pgfsetlinewidth{0.000000pt}%
\definecolor{currentstroke}{rgb}{0.000000,0.000000,0.000000}%
\pgfsetstrokecolor{currentstroke}%
\pgfsetdash{}{0pt}%
\pgfpathmoveto{\pgfqpoint{5.117893in}{2.467990in}}%
\pgfpathlineto{\pgfqpoint{5.131659in}{2.466246in}}%
\pgfpathlineto{\pgfqpoint{5.145433in}{2.464526in}}%
\pgfpathlineto{\pgfqpoint{5.159214in}{2.462831in}}%
\pgfpathlineto{\pgfqpoint{5.173003in}{2.461162in}}%
\pgfpathlineto{\pgfqpoint{5.165628in}{2.454559in}}%
\pgfpathlineto{\pgfqpoint{5.158246in}{2.447914in}}%
\pgfpathlineto{\pgfqpoint{5.150858in}{2.441223in}}%
\pgfpathlineto{\pgfqpoint{5.143463in}{2.434485in}}%
\pgfpathlineto{\pgfqpoint{5.129660in}{2.436102in}}%
\pgfpathlineto{\pgfqpoint{5.115863in}{2.437744in}}%
\pgfpathlineto{\pgfqpoint{5.102075in}{2.439412in}}%
\pgfpathlineto{\pgfqpoint{5.088295in}{2.441104in}}%
\pgfpathlineto{\pgfqpoint{5.095704in}{2.447889in}}%
\pgfpathlineto{\pgfqpoint{5.103107in}{2.454631in}}%
\pgfpathlineto{\pgfqpoint{5.110503in}{2.461330in}}%
\pgfpathlineto{\pgfqpoint{5.117893in}{2.467990in}}%
\pgfpathclose%
\pgfusepath{fill}%
\end{pgfscope}%
\begin{pgfscope}%
\pgfpathrectangle{\pgfqpoint{1.150000in}{0.150000in}}{\pgfqpoint{5.700000in}{5.700000in}}%
\pgfusepath{clip}%
\pgfsetbuttcap%
\pgfsetroundjoin%
\definecolor{currentfill}{rgb}{0.282910,0.105393,0.426902}%
\pgfsetfillcolor{currentfill}%
\pgfsetfillopacity{0.700000}%
\pgfsetlinewidth{0.000000pt}%
\definecolor{currentstroke}{rgb}{0.000000,0.000000,0.000000}%
\pgfsetstrokecolor{currentstroke}%
\pgfsetdash{}{0pt}%
\pgfpathmoveto{\pgfqpoint{2.745291in}{2.477786in}}%
\pgfpathlineto{\pgfqpoint{2.758574in}{2.470494in}}%
\pgfpathlineto{\pgfqpoint{2.771861in}{2.463251in}}%
\pgfpathlineto{\pgfqpoint{2.785150in}{2.456054in}}%
\pgfpathlineto{\pgfqpoint{2.798442in}{2.448906in}}%
\pgfpathlineto{\pgfqpoint{2.790056in}{2.446203in}}%
\pgfpathlineto{\pgfqpoint{2.781657in}{2.443675in}}%
\pgfpathlineto{\pgfqpoint{2.773246in}{2.441328in}}%
\pgfpathlineto{\pgfqpoint{2.764823in}{2.439166in}}%
\pgfpathlineto{\pgfqpoint{2.751506in}{2.446518in}}%
\pgfpathlineto{\pgfqpoint{2.738193in}{2.453917in}}%
\pgfpathlineto{\pgfqpoint{2.724882in}{2.461364in}}%
\pgfpathlineto{\pgfqpoint{2.711574in}{2.468859in}}%
\pgfpathlineto{\pgfqpoint{2.720022in}{2.470812in}}%
\pgfpathlineto{\pgfqpoint{2.728458in}{2.472955in}}%
\pgfpathlineto{\pgfqpoint{2.736881in}{2.475281in}}%
\pgfpathlineto{\pgfqpoint{2.745291in}{2.477786in}}%
\pgfpathclose%
\pgfusepath{fill}%
\end{pgfscope}%
\begin{pgfscope}%
\pgfpathrectangle{\pgfqpoint{1.150000in}{0.150000in}}{\pgfqpoint{5.700000in}{5.700000in}}%
\pgfusepath{clip}%
\pgfsetbuttcap%
\pgfsetroundjoin%
\definecolor{currentfill}{rgb}{0.282290,0.145912,0.461510}%
\pgfsetfillcolor{currentfill}%
\pgfsetfillopacity{0.700000}%
\pgfsetlinewidth{0.000000pt}%
\definecolor{currentstroke}{rgb}{0.000000,0.000000,0.000000}%
\pgfsetstrokecolor{currentstroke}%
\pgfsetdash{}{0pt}%
\pgfpathmoveto{\pgfqpoint{5.790406in}{2.554106in}}%
\pgfpathlineto{\pgfqpoint{5.804356in}{2.552649in}}%
\pgfpathlineto{\pgfqpoint{5.818316in}{2.551215in}}%
\pgfpathlineto{\pgfqpoint{5.832283in}{2.549806in}}%
\pgfpathlineto{\pgfqpoint{5.846259in}{2.548421in}}%
\pgfpathlineto{\pgfqpoint{5.839190in}{2.543246in}}%
\pgfpathlineto{\pgfqpoint{5.832117in}{2.538074in}}%
\pgfpathlineto{\pgfqpoint{5.825037in}{2.532902in}}%
\pgfpathlineto{\pgfqpoint{5.817952in}{2.527724in}}%
\pgfpathlineto{\pgfqpoint{5.803956in}{2.528976in}}%
\pgfpathlineto{\pgfqpoint{5.789968in}{2.530252in}}%
\pgfpathlineto{\pgfqpoint{5.775989in}{2.531553in}}%
\pgfpathlineto{\pgfqpoint{5.762019in}{2.532877in}}%
\pgfpathlineto{\pgfqpoint{5.769124in}{2.538183in}}%
\pgfpathlineto{\pgfqpoint{5.776224in}{2.543487in}}%
\pgfpathlineto{\pgfqpoint{5.783317in}{2.548794in}}%
\pgfpathlineto{\pgfqpoint{5.790406in}{2.554106in}}%
\pgfpathclose%
\pgfusepath{fill}%
\end{pgfscope}%
\begin{pgfscope}%
\pgfpathrectangle{\pgfqpoint{1.150000in}{0.150000in}}{\pgfqpoint{5.700000in}{5.700000in}}%
\pgfusepath{clip}%
\pgfsetbuttcap%
\pgfsetroundjoin%
\definecolor{currentfill}{rgb}{0.283197,0.115680,0.436115}%
\pgfsetfillcolor{currentfill}%
\pgfsetfillopacity{0.700000}%
\pgfsetlinewidth{0.000000pt}%
\definecolor{currentstroke}{rgb}{0.000000,0.000000,0.000000}%
\pgfsetstrokecolor{currentstroke}%
\pgfsetdash{}{0pt}%
\pgfpathmoveto{\pgfqpoint{5.342123in}{2.499087in}}%
\pgfpathlineto{\pgfqpoint{5.355952in}{2.497497in}}%
\pgfpathlineto{\pgfqpoint{5.369789in}{2.495931in}}%
\pgfpathlineto{\pgfqpoint{5.383633in}{2.494389in}}%
\pgfpathlineto{\pgfqpoint{5.397486in}{2.492872in}}%
\pgfpathlineto{\pgfqpoint{5.390210in}{2.486783in}}%
\pgfpathlineto{\pgfqpoint{5.382927in}{2.480660in}}%
\pgfpathlineto{\pgfqpoint{5.375638in}{2.474501in}}%
\pgfpathlineto{\pgfqpoint{5.368342in}{2.468303in}}%
\pgfpathlineto{\pgfqpoint{5.354473in}{2.469740in}}%
\pgfpathlineto{\pgfqpoint{5.340612in}{2.471202in}}%
\pgfpathlineto{\pgfqpoint{5.326759in}{2.472689in}}%
\pgfpathlineto{\pgfqpoint{5.312914in}{2.474200in}}%
\pgfpathlineto{\pgfqpoint{5.320226in}{2.480473in}}%
\pgfpathlineto{\pgfqpoint{5.327531in}{2.486710in}}%
\pgfpathlineto{\pgfqpoint{5.334830in}{2.492914in}}%
\pgfpathlineto{\pgfqpoint{5.342123in}{2.499087in}}%
\pgfpathclose%
\pgfusepath{fill}%
\end{pgfscope}%
\begin{pgfscope}%
\pgfpathrectangle{\pgfqpoint{1.150000in}{0.150000in}}{\pgfqpoint{5.700000in}{5.700000in}}%
\pgfusepath{clip}%
\pgfsetbuttcap%
\pgfsetroundjoin%
\definecolor{currentfill}{rgb}{0.282884,0.135920,0.453427}%
\pgfsetfillcolor{currentfill}%
\pgfsetfillopacity{0.700000}%
\pgfsetlinewidth{0.000000pt}%
\definecolor{currentstroke}{rgb}{0.000000,0.000000,0.000000}%
\pgfsetstrokecolor{currentstroke}%
\pgfsetdash{}{0pt}%
\pgfpathmoveto{\pgfqpoint{5.566307in}{2.527896in}}%
\pgfpathlineto{\pgfqpoint{5.580198in}{2.526401in}}%
\pgfpathlineto{\pgfqpoint{5.594096in}{2.524930in}}%
\pgfpathlineto{\pgfqpoint{5.608003in}{2.523483in}}%
\pgfpathlineto{\pgfqpoint{5.621919in}{2.522061in}}%
\pgfpathlineto{\pgfqpoint{5.614745in}{2.516464in}}%
\pgfpathlineto{\pgfqpoint{5.607566in}{2.510848in}}%
\pgfpathlineto{\pgfqpoint{5.600380in}{2.505211in}}%
\pgfpathlineto{\pgfqpoint{5.593188in}{2.499549in}}%
\pgfpathlineto{\pgfqpoint{5.579255in}{2.500865in}}%
\pgfpathlineto{\pgfqpoint{5.565330in}{2.502205in}}%
\pgfpathlineto{\pgfqpoint{5.551413in}{2.503570in}}%
\pgfpathlineto{\pgfqpoint{5.537504in}{2.504959in}}%
\pgfpathlineto{\pgfqpoint{5.544714in}{2.510722in}}%
\pgfpathlineto{\pgfqpoint{5.551918in}{2.516464in}}%
\pgfpathlineto{\pgfqpoint{5.559115in}{2.522188in}}%
\pgfpathlineto{\pgfqpoint{5.566307in}{2.527896in}}%
\pgfpathclose%
\pgfusepath{fill}%
\end{pgfscope}%
\begin{pgfscope}%
\pgfpathrectangle{\pgfqpoint{1.150000in}{0.150000in}}{\pgfqpoint{5.700000in}{5.700000in}}%
\pgfusepath{clip}%
\pgfsetbuttcap%
\pgfsetroundjoin%
\definecolor{currentfill}{rgb}{0.279566,0.067836,0.391917}%
\pgfsetfillcolor{currentfill}%
\pgfsetfillopacity{0.700000}%
\pgfsetlinewidth{0.000000pt}%
\definecolor{currentstroke}{rgb}{0.000000,0.000000,0.000000}%
\pgfsetstrokecolor{currentstroke}%
\pgfsetdash{}{0pt}%
\pgfpathmoveto{\pgfqpoint{2.938147in}{2.407338in}}%
\pgfpathlineto{\pgfqpoint{2.951446in}{2.400782in}}%
\pgfpathlineto{\pgfqpoint{2.964750in}{2.394270in}}%
\pgfpathlineto{\pgfqpoint{2.978057in}{2.387801in}}%
\pgfpathlineto{\pgfqpoint{2.991368in}{2.381373in}}%
\pgfpathlineto{\pgfqpoint{2.983093in}{2.377467in}}%
\pgfpathlineto{\pgfqpoint{2.974807in}{2.373706in}}%
\pgfpathlineto{\pgfqpoint{2.966511in}{2.370096in}}%
\pgfpathlineto{\pgfqpoint{2.958204in}{2.366641in}}%
\pgfpathlineto{\pgfqpoint{2.944872in}{2.373257in}}%
\pgfpathlineto{\pgfqpoint{2.931543in}{2.379915in}}%
\pgfpathlineto{\pgfqpoint{2.918218in}{2.386616in}}%
\pgfpathlineto{\pgfqpoint{2.904896in}{2.393360in}}%
\pgfpathlineto{\pgfqpoint{2.913225in}{2.396621in}}%
\pgfpathlineto{\pgfqpoint{2.921543in}{2.400041in}}%
\pgfpathlineto{\pgfqpoint{2.929850in}{2.403615in}}%
\pgfpathlineto{\pgfqpoint{2.938147in}{2.407338in}}%
\pgfpathclose%
\pgfusepath{fill}%
\end{pgfscope}%
\begin{pgfscope}%
\pgfpathrectangle{\pgfqpoint{1.150000in}{0.150000in}}{\pgfqpoint{5.700000in}{5.700000in}}%
\pgfusepath{clip}%
\pgfsetbuttcap%
\pgfsetroundjoin%
\definecolor{currentfill}{rgb}{0.273809,0.031497,0.358853}%
\pgfsetfillcolor{currentfill}%
\pgfsetfillopacity{0.700000}%
\pgfsetlinewidth{0.000000pt}%
\definecolor{currentstroke}{rgb}{0.000000,0.000000,0.000000}%
\pgfsetstrokecolor{currentstroke}%
\pgfsetdash{}{0pt}%
\pgfpathmoveto{\pgfqpoint{4.360323in}{2.349326in}}%
\pgfpathlineto{\pgfqpoint{4.373889in}{2.346620in}}%
\pgfpathlineto{\pgfqpoint{4.387461in}{2.343942in}}%
\pgfpathlineto{\pgfqpoint{4.401040in}{2.341291in}}%
\pgfpathlineto{\pgfqpoint{4.414626in}{2.338667in}}%
\pgfpathlineto{\pgfqpoint{4.406948in}{2.330763in}}%
\pgfpathlineto{\pgfqpoint{4.399265in}{2.322828in}}%
\pgfpathlineto{\pgfqpoint{4.391575in}{2.314862in}}%
\pgfpathlineto{\pgfqpoint{4.383880in}{2.306865in}}%
\pgfpathlineto{\pgfqpoint{4.370282in}{2.309530in}}%
\pgfpathlineto{\pgfqpoint{4.356691in}{2.312223in}}%
\pgfpathlineto{\pgfqpoint{4.343107in}{2.314943in}}%
\pgfpathlineto{\pgfqpoint{4.329529in}{2.317690in}}%
\pgfpathlineto{\pgfqpoint{4.337236in}{2.325640in}}%
\pgfpathlineto{\pgfqpoint{4.344938in}{2.333563in}}%
\pgfpathlineto{\pgfqpoint{4.352633in}{2.341458in}}%
\pgfpathlineto{\pgfqpoint{4.360323in}{2.349326in}}%
\pgfpathclose%
\pgfusepath{fill}%
\end{pgfscope}%
\begin{pgfscope}%
\pgfpathrectangle{\pgfqpoint{1.150000in}{0.150000in}}{\pgfqpoint{5.700000in}{5.700000in}}%
\pgfusepath{clip}%
\pgfsetbuttcap%
\pgfsetroundjoin%
\definecolor{currentfill}{rgb}{0.269944,0.014625,0.341379}%
\pgfsetfillcolor{currentfill}%
\pgfsetfillopacity{0.700000}%
\pgfsetlinewidth{0.000000pt}%
\definecolor{currentstroke}{rgb}{0.000000,0.000000,0.000000}%
\pgfsetstrokecolor{currentstroke}%
\pgfsetdash{}{0pt}%
\pgfpathmoveto{\pgfqpoint{4.136088in}{2.321424in}}%
\pgfpathlineto{\pgfqpoint{4.149599in}{2.318288in}}%
\pgfpathlineto{\pgfqpoint{4.163116in}{2.315179in}}%
\pgfpathlineto{\pgfqpoint{4.176639in}{2.312099in}}%
\pgfpathlineto{\pgfqpoint{4.190169in}{2.309047in}}%
\pgfpathlineto{\pgfqpoint{4.182409in}{2.301102in}}%
\pgfpathlineto{\pgfqpoint{4.174644in}{2.293143in}}%
\pgfpathlineto{\pgfqpoint{4.166873in}{2.285170in}}%
\pgfpathlineto{\pgfqpoint{4.159097in}{2.277185in}}%
\pgfpathlineto{\pgfqpoint{4.145555in}{2.280305in}}%
\pgfpathlineto{\pgfqpoint{4.132019in}{2.283453in}}%
\pgfpathlineto{\pgfqpoint{4.118490in}{2.286630in}}%
\pgfpathlineto{\pgfqpoint{4.104968in}{2.289835in}}%
\pgfpathlineto{\pgfqpoint{4.112756in}{2.297746in}}%
\pgfpathlineto{\pgfqpoint{4.120539in}{2.305649in}}%
\pgfpathlineto{\pgfqpoint{4.128317in}{2.313542in}}%
\pgfpathlineto{\pgfqpoint{4.136088in}{2.321424in}}%
\pgfpathclose%
\pgfusepath{fill}%
\end{pgfscope}%
\begin{pgfscope}%
\pgfpathrectangle{\pgfqpoint{1.150000in}{0.150000in}}{\pgfqpoint{5.700000in}{5.700000in}}%
\pgfusepath{clip}%
\pgfsetbuttcap%
\pgfsetroundjoin%
\definecolor{currentfill}{rgb}{0.277941,0.056324,0.381191}%
\pgfsetfillcolor{currentfill}%
\pgfsetfillopacity{0.700000}%
\pgfsetlinewidth{0.000000pt}%
\definecolor{currentstroke}{rgb}{0.000000,0.000000,0.000000}%
\pgfsetstrokecolor{currentstroke}%
\pgfsetdash{}{0pt}%
\pgfpathmoveto{\pgfqpoint{4.584587in}{2.381058in}}%
\pgfpathlineto{\pgfqpoint{4.598211in}{2.378719in}}%
\pgfpathlineto{\pgfqpoint{4.611843in}{2.376406in}}%
\pgfpathlineto{\pgfqpoint{4.625482in}{2.374120in}}%
\pgfpathlineto{\pgfqpoint{4.639128in}{2.371860in}}%
\pgfpathlineto{\pgfqpoint{4.631533in}{2.364186in}}%
\pgfpathlineto{\pgfqpoint{4.623933in}{2.356470in}}%
\pgfpathlineto{\pgfqpoint{4.616326in}{2.348711in}}%
\pgfpathlineto{\pgfqpoint{4.608713in}{2.340909in}}%
\pgfpathlineto{\pgfqpoint{4.595055in}{2.343184in}}%
\pgfpathlineto{\pgfqpoint{4.581404in}{2.345486in}}%
\pgfpathlineto{\pgfqpoint{4.567760in}{2.347813in}}%
\pgfpathlineto{\pgfqpoint{4.554123in}{2.350167in}}%
\pgfpathlineto{\pgfqpoint{4.561748in}{2.357949in}}%
\pgfpathlineto{\pgfqpoint{4.569366in}{2.365692in}}%
\pgfpathlineto{\pgfqpoint{4.576979in}{2.373395in}}%
\pgfpathlineto{\pgfqpoint{4.584587in}{2.381058in}}%
\pgfpathclose%
\pgfusepath{fill}%
\end{pgfscope}%
\begin{pgfscope}%
\pgfpathrectangle{\pgfqpoint{1.150000in}{0.150000in}}{\pgfqpoint{5.700000in}{5.700000in}}%
\pgfusepath{clip}%
\pgfsetbuttcap%
\pgfsetroundjoin%
\definecolor{currentfill}{rgb}{0.280255,0.165693,0.476498}%
\pgfsetfillcolor{currentfill}%
\pgfsetfillopacity{0.700000}%
\pgfsetlinewidth{0.000000pt}%
\definecolor{currentstroke}{rgb}{0.000000,0.000000,0.000000}%
\pgfsetstrokecolor{currentstroke}%
\pgfsetdash{}{0pt}%
\pgfpathmoveto{\pgfqpoint{6.154414in}{2.585853in}}%
\pgfpathlineto{\pgfqpoint{6.168467in}{2.584393in}}%
\pgfpathlineto{\pgfqpoint{6.182528in}{2.582957in}}%
\pgfpathlineto{\pgfqpoint{6.196598in}{2.581544in}}%
\pgfpathlineto{\pgfqpoint{6.189691in}{2.576812in}}%
\pgfpathlineto{\pgfqpoint{6.182780in}{2.572125in}}%
\pgfpathlineto{\pgfqpoint{6.175865in}{2.567478in}}%
\pgfpathlineto{\pgfqpoint{6.168946in}{2.562867in}}%
\pgfpathlineto{\pgfqpoint{6.154853in}{2.564106in}}%
\pgfpathlineto{\pgfqpoint{6.140768in}{2.565369in}}%
\pgfpathlineto{\pgfqpoint{6.126692in}{2.566656in}}%
\pgfpathlineto{\pgfqpoint{6.133629in}{2.571393in}}%
\pgfpathlineto{\pgfqpoint{6.140561in}{2.576169in}}%
\pgfpathlineto{\pgfqpoint{6.147489in}{2.580987in}}%
\pgfpathlineto{\pgfqpoint{6.154414in}{2.585853in}}%
\pgfpathclose%
\pgfusepath{fill}%
\end{pgfscope}%
\begin{pgfscope}%
\pgfpathrectangle{\pgfqpoint{1.150000in}{0.150000in}}{\pgfqpoint{5.700000in}{5.700000in}}%
\pgfusepath{clip}%
\pgfsetbuttcap%
\pgfsetroundjoin%
\definecolor{currentfill}{rgb}{0.267004,0.004874,0.329415}%
\pgfsetfillcolor{currentfill}%
\pgfsetfillopacity{0.700000}%
\pgfsetlinewidth{0.000000pt}%
\definecolor{currentstroke}{rgb}{0.000000,0.000000,0.000000}%
\pgfsetstrokecolor{currentstroke}%
\pgfsetdash{}{0pt}%
\pgfpathmoveto{\pgfqpoint{3.911825in}{2.299825in}}%
\pgfpathlineto{\pgfqpoint{3.925286in}{2.296189in}}%
\pgfpathlineto{\pgfqpoint{3.938752in}{2.292583in}}%
\pgfpathlineto{\pgfqpoint{3.952225in}{2.289006in}}%
\pgfpathlineto{\pgfqpoint{3.965704in}{2.285460in}}%
\pgfpathlineto{\pgfqpoint{3.957862in}{2.277708in}}%
\pgfpathlineto{\pgfqpoint{3.950014in}{2.269965in}}%
\pgfpathlineto{\pgfqpoint{3.942160in}{2.262233in}}%
\pgfpathlineto{\pgfqpoint{3.934300in}{2.254512in}}%
\pgfpathlineto{\pgfqpoint{3.920809in}{2.258153in}}%
\pgfpathlineto{\pgfqpoint{3.907323in}{2.261825in}}%
\pgfpathlineto{\pgfqpoint{3.893843in}{2.265525in}}%
\pgfpathlineto{\pgfqpoint{3.880370in}{2.269256in}}%
\pgfpathlineto{\pgfqpoint{3.888243in}{2.276877in}}%
\pgfpathlineto{\pgfqpoint{3.896109in}{2.284513in}}%
\pgfpathlineto{\pgfqpoint{3.903970in}{2.292163in}}%
\pgfpathlineto{\pgfqpoint{3.911825in}{2.299825in}}%
\pgfpathclose%
\pgfusepath{fill}%
\end{pgfscope}%
\begin{pgfscope}%
\pgfpathrectangle{\pgfqpoint{1.150000in}{0.150000in}}{\pgfqpoint{5.700000in}{5.700000in}}%
\pgfusepath{clip}%
\pgfsetbuttcap%
\pgfsetroundjoin%
\definecolor{currentfill}{rgb}{0.280267,0.073417,0.397163}%
\pgfsetfillcolor{currentfill}%
\pgfsetfillopacity{0.700000}%
\pgfsetlinewidth{0.000000pt}%
\definecolor{currentstroke}{rgb}{0.000000,0.000000,0.000000}%
\pgfsetstrokecolor{currentstroke}%
\pgfsetdash{}{0pt}%
\pgfpathmoveto{\pgfqpoint{4.808898in}{2.414551in}}%
\pgfpathlineto{\pgfqpoint{4.822584in}{2.412516in}}%
\pgfpathlineto{\pgfqpoint{4.836278in}{2.410506in}}%
\pgfpathlineto{\pgfqpoint{4.849980in}{2.408522in}}%
\pgfpathlineto{\pgfqpoint{4.863688in}{2.406563in}}%
\pgfpathlineto{\pgfqpoint{4.856181in}{2.399263in}}%
\pgfpathlineto{\pgfqpoint{4.848667in}{2.391914in}}%
\pgfpathlineto{\pgfqpoint{4.841147in}{2.384517in}}%
\pgfpathlineto{\pgfqpoint{4.833620in}{2.377071in}}%
\pgfpathlineto{\pgfqpoint{4.819898in}{2.379017in}}%
\pgfpathlineto{\pgfqpoint{4.806184in}{2.380990in}}%
\pgfpathlineto{\pgfqpoint{4.792477in}{2.382988in}}%
\pgfpathlineto{\pgfqpoint{4.778778in}{2.385011in}}%
\pgfpathlineto{\pgfqpoint{4.786317in}{2.392465in}}%
\pgfpathlineto{\pgfqpoint{4.793850in}{2.399872in}}%
\pgfpathlineto{\pgfqpoint{4.801377in}{2.407234in}}%
\pgfpathlineto{\pgfqpoint{4.808898in}{2.414551in}}%
\pgfpathclose%
\pgfusepath{fill}%
\end{pgfscope}%
\begin{pgfscope}%
\pgfpathrectangle{\pgfqpoint{1.150000in}{0.150000in}}{\pgfqpoint{5.700000in}{5.700000in}}%
\pgfusepath{clip}%
\pgfsetbuttcap%
\pgfsetroundjoin%
\definecolor{currentfill}{rgb}{0.268510,0.009605,0.335427}%
\pgfsetfillcolor{currentfill}%
\pgfsetfillopacity{0.700000}%
\pgfsetlinewidth{0.000000pt}%
\definecolor{currentstroke}{rgb}{0.000000,0.000000,0.000000}%
\pgfsetstrokecolor{currentstroke}%
\pgfsetdash{}{0pt}%
\pgfpathmoveto{\pgfqpoint{3.409264in}{2.307235in}}%
\pgfpathlineto{\pgfqpoint{3.422628in}{2.302243in}}%
\pgfpathlineto{\pgfqpoint{3.435997in}{2.297286in}}%
\pgfpathlineto{\pgfqpoint{3.449371in}{2.292363in}}%
\pgfpathlineto{\pgfqpoint{3.462750in}{2.287475in}}%
\pgfpathlineto{\pgfqpoint{3.454707in}{2.281145in}}%
\pgfpathlineto{\pgfqpoint{3.446657in}{2.274889in}}%
\pgfpathlineto{\pgfqpoint{3.438599in}{2.268711in}}%
\pgfpathlineto{\pgfqpoint{3.430534in}{2.262614in}}%
\pgfpathlineto{\pgfqpoint{3.417139in}{2.267651in}}%
\pgfpathlineto{\pgfqpoint{3.403748in}{2.272721in}}%
\pgfpathlineto{\pgfqpoint{3.390363in}{2.277825in}}%
\pgfpathlineto{\pgfqpoint{3.376982in}{2.282965in}}%
\pgfpathlineto{\pgfqpoint{3.385064in}{2.288909in}}%
\pgfpathlineto{\pgfqpoint{3.393138in}{2.294937in}}%
\pgfpathlineto{\pgfqpoint{3.401205in}{2.301047in}}%
\pgfpathlineto{\pgfqpoint{3.409264in}{2.307235in}}%
\pgfpathclose%
\pgfusepath{fill}%
\end{pgfscope}%
\begin{pgfscope}%
\pgfpathrectangle{\pgfqpoint{1.150000in}{0.150000in}}{\pgfqpoint{5.700000in}{5.700000in}}%
\pgfusepath{clip}%
\pgfsetbuttcap%
\pgfsetroundjoin%
\definecolor{currentfill}{rgb}{0.271305,0.019942,0.347269}%
\pgfsetfillcolor{currentfill}%
\pgfsetfillopacity{0.700000}%
\pgfsetlinewidth{0.000000pt}%
\definecolor{currentstroke}{rgb}{0.000000,0.000000,0.000000}%
\pgfsetstrokecolor{currentstroke}%
\pgfsetdash{}{0pt}%
\pgfpathmoveto{\pgfqpoint{3.270109in}{2.325353in}}%
\pgfpathlineto{\pgfqpoint{3.283452in}{2.319928in}}%
\pgfpathlineto{\pgfqpoint{3.296799in}{2.314540in}}%
\pgfpathlineto{\pgfqpoint{3.310151in}{2.309188in}}%
\pgfpathlineto{\pgfqpoint{3.323508in}{2.303872in}}%
\pgfpathlineto{\pgfqpoint{3.315401in}{2.298173in}}%
\pgfpathlineto{\pgfqpoint{3.307287in}{2.292569in}}%
\pgfpathlineto{\pgfqpoint{3.299164in}{2.287065in}}%
\pgfpathlineto{\pgfqpoint{3.291032in}{2.281664in}}%
\pgfpathlineto{\pgfqpoint{3.277658in}{2.287141in}}%
\pgfpathlineto{\pgfqpoint{3.264288in}{2.292653in}}%
\pgfpathlineto{\pgfqpoint{3.250922in}{2.298203in}}%
\pgfpathlineto{\pgfqpoint{3.237562in}{2.303789in}}%
\pgfpathlineto{\pgfqpoint{3.245711in}{2.309024in}}%
\pgfpathlineto{\pgfqpoint{3.253852in}{2.314365in}}%
\pgfpathlineto{\pgfqpoint{3.261984in}{2.319810in}}%
\pgfpathlineto{\pgfqpoint{3.270109in}{2.325353in}}%
\pgfpathclose%
\pgfusepath{fill}%
\end{pgfscope}%
\begin{pgfscope}%
\pgfpathrectangle{\pgfqpoint{1.150000in}{0.150000in}}{\pgfqpoint{5.700000in}{5.700000in}}%
\pgfusepath{clip}%
\pgfsetbuttcap%
\pgfsetroundjoin%
\definecolor{currentfill}{rgb}{0.267004,0.004874,0.329415}%
\pgfsetfillcolor{currentfill}%
\pgfsetfillopacity{0.700000}%
\pgfsetlinewidth{0.000000pt}%
\definecolor{currentstroke}{rgb}{0.000000,0.000000,0.000000}%
\pgfsetstrokecolor{currentstroke}%
\pgfsetdash{}{0pt}%
\pgfpathmoveto{\pgfqpoint{3.548355in}{2.294795in}}%
\pgfpathlineto{\pgfqpoint{3.561744in}{2.290209in}}%
\pgfpathlineto{\pgfqpoint{3.575139in}{2.285656in}}%
\pgfpathlineto{\pgfqpoint{3.588538in}{2.281135in}}%
\pgfpathlineto{\pgfqpoint{3.601943in}{2.276647in}}%
\pgfpathlineto{\pgfqpoint{3.593959in}{2.269794in}}%
\pgfpathlineto{\pgfqpoint{3.585968in}{2.262995in}}%
\pgfpathlineto{\pgfqpoint{3.577970in}{2.256254in}}%
\pgfpathlineto{\pgfqpoint{3.569966in}{2.249574in}}%
\pgfpathlineto{\pgfqpoint{3.556545in}{2.254196in}}%
\pgfpathlineto{\pgfqpoint{3.543131in}{2.258851in}}%
\pgfpathlineto{\pgfqpoint{3.529721in}{2.263539in}}%
\pgfpathlineto{\pgfqpoint{3.516316in}{2.268259in}}%
\pgfpathlineto{\pgfqpoint{3.524336in}{2.274800in}}%
\pgfpathlineto{\pgfqpoint{3.532350in}{2.281405in}}%
\pgfpathlineto{\pgfqpoint{3.540356in}{2.288071in}}%
\pgfpathlineto{\pgfqpoint{3.548355in}{2.294795in}}%
\pgfpathclose%
\pgfusepath{fill}%
\end{pgfscope}%
\begin{pgfscope}%
\pgfpathrectangle{\pgfqpoint{1.150000in}{0.150000in}}{\pgfqpoint{5.700000in}{5.700000in}}%
\pgfusepath{clip}%
\pgfsetbuttcap%
\pgfsetroundjoin%
\definecolor{currentfill}{rgb}{0.282327,0.094955,0.417331}%
\pgfsetfillcolor{currentfill}%
\pgfsetfillopacity{0.700000}%
\pgfsetlinewidth{0.000000pt}%
\definecolor{currentstroke}{rgb}{0.000000,0.000000,0.000000}%
\pgfsetstrokecolor{currentstroke}%
\pgfsetdash{}{0pt}%
\pgfpathmoveto{\pgfqpoint{5.033250in}{2.448126in}}%
\pgfpathlineto{\pgfqpoint{5.046999in}{2.446332in}}%
\pgfpathlineto{\pgfqpoint{5.060757in}{2.444564in}}%
\pgfpathlineto{\pgfqpoint{5.074522in}{2.442822in}}%
\pgfpathlineto{\pgfqpoint{5.088295in}{2.441104in}}%
\pgfpathlineto{\pgfqpoint{5.080879in}{2.434274in}}%
\pgfpathlineto{\pgfqpoint{5.073457in}{2.427396in}}%
\pgfpathlineto{\pgfqpoint{5.066028in}{2.420469in}}%
\pgfpathlineto{\pgfqpoint{5.058593in}{2.413492in}}%
\pgfpathlineto{\pgfqpoint{5.044806in}{2.415171in}}%
\pgfpathlineto{\pgfqpoint{5.031027in}{2.416875in}}%
\pgfpathlineto{\pgfqpoint{5.017255in}{2.418604in}}%
\pgfpathlineto{\pgfqpoint{5.003492in}{2.420358in}}%
\pgfpathlineto{\pgfqpoint{5.010941in}{2.427369in}}%
\pgfpathlineto{\pgfqpoint{5.018383in}{2.434333in}}%
\pgfpathlineto{\pgfqpoint{5.025820in}{2.441251in}}%
\pgfpathlineto{\pgfqpoint{5.033250in}{2.448126in}}%
\pgfpathclose%
\pgfusepath{fill}%
\end{pgfscope}%
\begin{pgfscope}%
\pgfpathrectangle{\pgfqpoint{1.150000in}{0.150000in}}{\pgfqpoint{5.700000in}{5.700000in}}%
\pgfusepath{clip}%
\pgfsetbuttcap%
\pgfsetroundjoin%
\definecolor{currentfill}{rgb}{0.281412,0.155834,0.469201}%
\pgfsetfillcolor{currentfill}%
\pgfsetfillopacity{0.700000}%
\pgfsetlinewidth{0.000000pt}%
\definecolor{currentstroke}{rgb}{0.000000,0.000000,0.000000}%
\pgfsetstrokecolor{currentstroke}%
\pgfsetdash{}{0pt}%
\pgfpathmoveto{\pgfqpoint{5.930381in}{2.563352in}}%
\pgfpathlineto{\pgfqpoint{5.944378in}{2.561940in}}%
\pgfpathlineto{\pgfqpoint{5.958383in}{2.560552in}}%
\pgfpathlineto{\pgfqpoint{5.972396in}{2.559187in}}%
\pgfpathlineto{\pgfqpoint{5.986418in}{2.557847in}}%
\pgfpathlineto{\pgfqpoint{5.979413in}{2.552918in}}%
\pgfpathlineto{\pgfqpoint{5.972403in}{2.548003in}}%
\pgfpathlineto{\pgfqpoint{5.965388in}{2.543099in}}%
\pgfpathlineto{\pgfqpoint{5.958367in}{2.538200in}}%
\pgfpathlineto{\pgfqpoint{5.944324in}{2.539394in}}%
\pgfpathlineto{\pgfqpoint{5.930289in}{2.540612in}}%
\pgfpathlineto{\pgfqpoint{5.916263in}{2.541853in}}%
\pgfpathlineto{\pgfqpoint{5.902245in}{2.543119in}}%
\pgfpathlineto{\pgfqpoint{5.909287in}{2.548159in}}%
\pgfpathlineto{\pgfqpoint{5.916324in}{2.553209in}}%
\pgfpathlineto{\pgfqpoint{5.923355in}{2.558272in}}%
\pgfpathlineto{\pgfqpoint{5.930381in}{2.563352in}}%
\pgfpathclose%
\pgfusepath{fill}%
\end{pgfscope}%
\begin{pgfscope}%
\pgfpathrectangle{\pgfqpoint{1.150000in}{0.150000in}}{\pgfqpoint{5.700000in}{5.700000in}}%
\pgfusepath{clip}%
\pgfsetbuttcap%
\pgfsetroundjoin%
\definecolor{currentfill}{rgb}{0.283091,0.110553,0.431554}%
\pgfsetfillcolor{currentfill}%
\pgfsetfillopacity{0.700000}%
\pgfsetlinewidth{0.000000pt}%
\definecolor{currentstroke}{rgb}{0.000000,0.000000,0.000000}%
\pgfsetstrokecolor{currentstroke}%
\pgfsetdash{}{0pt}%
\pgfpathmoveto{\pgfqpoint{5.257614in}{2.480494in}}%
\pgfpathlineto{\pgfqpoint{5.271427in}{2.478884in}}%
\pgfpathlineto{\pgfqpoint{5.285248in}{2.477298in}}%
\pgfpathlineto{\pgfqpoint{5.299077in}{2.475737in}}%
\pgfpathlineto{\pgfqpoint{5.312914in}{2.474200in}}%
\pgfpathlineto{\pgfqpoint{5.305596in}{2.467890in}}%
\pgfpathlineto{\pgfqpoint{5.298271in}{2.461538in}}%
\pgfpathlineto{\pgfqpoint{5.290939in}{2.455143in}}%
\pgfpathlineto{\pgfqpoint{5.283601in}{2.448703in}}%
\pgfpathlineto{\pgfqpoint{5.269748in}{2.450174in}}%
\pgfpathlineto{\pgfqpoint{5.255904in}{2.451669in}}%
\pgfpathlineto{\pgfqpoint{5.242067in}{2.453189in}}%
\pgfpathlineto{\pgfqpoint{5.228239in}{2.454734in}}%
\pgfpathlineto{\pgfqpoint{5.235592in}{2.461234in}}%
\pgfpathlineto{\pgfqpoint{5.242939in}{2.467693in}}%
\pgfpathlineto{\pgfqpoint{5.250280in}{2.474112in}}%
\pgfpathlineto{\pgfqpoint{5.257614in}{2.480494in}}%
\pgfpathclose%
\pgfusepath{fill}%
\end{pgfscope}%
\begin{pgfscope}%
\pgfpathrectangle{\pgfqpoint{1.150000in}{0.150000in}}{\pgfqpoint{5.700000in}{5.700000in}}%
\pgfusepath{clip}%
\pgfsetbuttcap%
\pgfsetroundjoin%
\definecolor{currentfill}{rgb}{0.274952,0.037752,0.364543}%
\pgfsetfillcolor{currentfill}%
\pgfsetfillopacity{0.700000}%
\pgfsetlinewidth{0.000000pt}%
\definecolor{currentstroke}{rgb}{0.000000,0.000000,0.000000}%
\pgfsetstrokecolor{currentstroke}%
\pgfsetdash{}{0pt}%
\pgfpathmoveto{\pgfqpoint{3.130832in}{2.349825in}}%
\pgfpathlineto{\pgfqpoint{3.144158in}{2.343937in}}%
\pgfpathlineto{\pgfqpoint{3.157488in}{2.338087in}}%
\pgfpathlineto{\pgfqpoint{3.170823in}{2.332276in}}%
\pgfpathlineto{\pgfqpoint{3.184162in}{2.326503in}}%
\pgfpathlineto{\pgfqpoint{3.175985in}{2.321549in}}%
\pgfpathlineto{\pgfqpoint{3.167800in}{2.316712in}}%
\pgfpathlineto{\pgfqpoint{3.159605in}{2.311998in}}%
\pgfpathlineto{\pgfqpoint{3.151400in}{2.307411in}}%
\pgfpathlineto{\pgfqpoint{3.138042in}{2.313358in}}%
\pgfpathlineto{\pgfqpoint{3.124688in}{2.319344in}}%
\pgfpathlineto{\pgfqpoint{3.111338in}{2.325368in}}%
\pgfpathlineto{\pgfqpoint{3.097992in}{2.331431in}}%
\pgfpathlineto{\pgfqpoint{3.106216in}{2.335839in}}%
\pgfpathlineto{\pgfqpoint{3.114431in}{2.340376in}}%
\pgfpathlineto{\pgfqpoint{3.122636in}{2.345040in}}%
\pgfpathlineto{\pgfqpoint{3.130832in}{2.349825in}}%
\pgfpathclose%
\pgfusepath{fill}%
\end{pgfscope}%
\begin{pgfscope}%
\pgfpathrectangle{\pgfqpoint{1.150000in}{0.150000in}}{\pgfqpoint{5.700000in}{5.700000in}}%
\pgfusepath{clip}%
\pgfsetbuttcap%
\pgfsetroundjoin%
\definecolor{currentfill}{rgb}{0.282290,0.145912,0.461510}%
\pgfsetfillcolor{currentfill}%
\pgfsetfillopacity{0.700000}%
\pgfsetlinewidth{0.000000pt}%
\definecolor{currentstroke}{rgb}{0.000000,0.000000,0.000000}%
\pgfsetstrokecolor{currentstroke}%
\pgfsetdash{}{0pt}%
\pgfpathmoveto{\pgfqpoint{5.706219in}{2.538416in}}%
\pgfpathlineto{\pgfqpoint{5.720157in}{2.536995in}}%
\pgfpathlineto{\pgfqpoint{5.734103in}{2.535598in}}%
\pgfpathlineto{\pgfqpoint{5.748057in}{2.534226in}}%
\pgfpathlineto{\pgfqpoint{5.762019in}{2.532877in}}%
\pgfpathlineto{\pgfqpoint{5.754908in}{2.527565in}}%
\pgfpathlineto{\pgfqpoint{5.747791in}{2.522243in}}%
\pgfpathlineto{\pgfqpoint{5.740667in}{2.516908in}}%
\pgfpathlineto{\pgfqpoint{5.733538in}{2.511555in}}%
\pgfpathlineto{\pgfqpoint{5.719556in}{2.512783in}}%
\pgfpathlineto{\pgfqpoint{5.705583in}{2.514036in}}%
\pgfpathlineto{\pgfqpoint{5.691618in}{2.515313in}}%
\pgfpathlineto{\pgfqpoint{5.677662in}{2.516614in}}%
\pgfpathlineto{\pgfqpoint{5.684810in}{2.522082in}}%
\pgfpathlineto{\pgfqpoint{5.691953in}{2.527536in}}%
\pgfpathlineto{\pgfqpoint{5.699089in}{2.532979in}}%
\pgfpathlineto{\pgfqpoint{5.706219in}{2.538416in}}%
\pgfpathclose%
\pgfusepath{fill}%
\end{pgfscope}%
\begin{pgfscope}%
\pgfpathrectangle{\pgfqpoint{1.150000in}{0.150000in}}{\pgfqpoint{5.700000in}{5.700000in}}%
\pgfusepath{clip}%
\pgfsetbuttcap%
\pgfsetroundjoin%
\definecolor{currentfill}{rgb}{0.283072,0.130895,0.449241}%
\pgfsetfillcolor{currentfill}%
\pgfsetfillopacity{0.700000}%
\pgfsetlinewidth{0.000000pt}%
\definecolor{currentstroke}{rgb}{0.000000,0.000000,0.000000}%
\pgfsetstrokecolor{currentstroke}%
\pgfsetdash{}{0pt}%
\pgfpathmoveto{\pgfqpoint{5.481951in}{2.510759in}}%
\pgfpathlineto{\pgfqpoint{5.495827in}{2.509273in}}%
\pgfpathlineto{\pgfqpoint{5.509711in}{2.507810in}}%
\pgfpathlineto{\pgfqpoint{5.523604in}{2.506372in}}%
\pgfpathlineto{\pgfqpoint{5.537504in}{2.504959in}}%
\pgfpathlineto{\pgfqpoint{5.530288in}{2.499170in}}%
\pgfpathlineto{\pgfqpoint{5.523065in}{2.493353in}}%
\pgfpathlineto{\pgfqpoint{5.515836in}{2.487505in}}%
\pgfpathlineto{\pgfqpoint{5.508600in}{2.481622in}}%
\pgfpathlineto{\pgfqpoint{5.494682in}{2.482942in}}%
\pgfpathlineto{\pgfqpoint{5.480772in}{2.484287in}}%
\pgfpathlineto{\pgfqpoint{5.466871in}{2.485657in}}%
\pgfpathlineto{\pgfqpoint{5.452978in}{2.487051in}}%
\pgfpathlineto{\pgfqpoint{5.460231in}{2.493022in}}%
\pgfpathlineto{\pgfqpoint{5.467477in}{2.498961in}}%
\pgfpathlineto{\pgfqpoint{5.474717in}{2.504873in}}%
\pgfpathlineto{\pgfqpoint{5.481951in}{2.510759in}}%
\pgfpathclose%
\pgfusepath{fill}%
\end{pgfscope}%
\begin{pgfscope}%
\pgfpathrectangle{\pgfqpoint{1.150000in}{0.150000in}}{\pgfqpoint{5.700000in}{5.700000in}}%
\pgfusepath{clip}%
\pgfsetbuttcap%
\pgfsetroundjoin%
\definecolor{currentfill}{rgb}{0.267004,0.004874,0.329415}%
\pgfsetfillcolor{currentfill}%
\pgfsetfillopacity{0.700000}%
\pgfsetlinewidth{0.000000pt}%
\definecolor{currentstroke}{rgb}{0.000000,0.000000,0.000000}%
\pgfsetstrokecolor{currentstroke}%
\pgfsetdash{}{0pt}%
\pgfpathmoveto{\pgfqpoint{3.687431in}{2.287406in}}%
\pgfpathlineto{\pgfqpoint{3.700849in}{2.283200in}}%
\pgfpathlineto{\pgfqpoint{3.714272in}{2.279025in}}%
\pgfpathlineto{\pgfqpoint{3.727701in}{2.274881in}}%
\pgfpathlineto{\pgfqpoint{3.741136in}{2.270768in}}%
\pgfpathlineto{\pgfqpoint{3.733206in}{2.263492in}}%
\pgfpathlineto{\pgfqpoint{3.725270in}{2.256253in}}%
\pgfpathlineto{\pgfqpoint{3.717328in}{2.249053in}}%
\pgfpathlineto{\pgfqpoint{3.709379in}{2.241895in}}%
\pgfpathlineto{\pgfqpoint{3.695930in}{2.246129in}}%
\pgfpathlineto{\pgfqpoint{3.682487in}{2.250394in}}%
\pgfpathlineto{\pgfqpoint{3.669049in}{2.254690in}}%
\pgfpathlineto{\pgfqpoint{3.655617in}{2.259018in}}%
\pgfpathlineto{\pgfqpoint{3.663580in}{2.266049in}}%
\pgfpathlineto{\pgfqpoint{3.671537in}{2.273126in}}%
\pgfpathlineto{\pgfqpoint{3.679487in}{2.280246in}}%
\pgfpathlineto{\pgfqpoint{3.687431in}{2.287406in}}%
\pgfpathclose%
\pgfusepath{fill}%
\end{pgfscope}%
\begin{pgfscope}%
\pgfpathrectangle{\pgfqpoint{1.150000in}{0.150000in}}{\pgfqpoint{5.700000in}{5.700000in}}%
\pgfusepath{clip}%
\pgfsetbuttcap%
\pgfsetroundjoin%
\definecolor{currentfill}{rgb}{0.272594,0.025563,0.353093}%
\pgfsetfillcolor{currentfill}%
\pgfsetfillopacity{0.700000}%
\pgfsetlinewidth{0.000000pt}%
\definecolor{currentstroke}{rgb}{0.000000,0.000000,0.000000}%
\pgfsetstrokecolor{currentstroke}%
\pgfsetdash{}{0pt}%
\pgfpathmoveto{\pgfqpoint{4.275286in}{2.328953in}}%
\pgfpathlineto{\pgfqpoint{4.288837in}{2.326096in}}%
\pgfpathlineto{\pgfqpoint{4.302394in}{2.323266in}}%
\pgfpathlineto{\pgfqpoint{4.315958in}{2.320464in}}%
\pgfpathlineto{\pgfqpoint{4.329529in}{2.317690in}}%
\pgfpathlineto{\pgfqpoint{4.321816in}{2.309714in}}%
\pgfpathlineto{\pgfqpoint{4.314098in}{2.301712in}}%
\pgfpathlineto{\pgfqpoint{4.306374in}{2.293687in}}%
\pgfpathlineto{\pgfqpoint{4.298644in}{2.285637in}}%
\pgfpathlineto{\pgfqpoint{4.285061in}{2.288467in}}%
\pgfpathlineto{\pgfqpoint{4.271485in}{2.291324in}}%
\pgfpathlineto{\pgfqpoint{4.257916in}{2.294208in}}%
\pgfpathlineto{\pgfqpoint{4.244353in}{2.297120in}}%
\pgfpathlineto{\pgfqpoint{4.252095in}{2.305110in}}%
\pgfpathlineto{\pgfqpoint{4.259831in}{2.313079in}}%
\pgfpathlineto{\pgfqpoint{4.267562in}{2.321027in}}%
\pgfpathlineto{\pgfqpoint{4.275286in}{2.328953in}}%
\pgfpathclose%
\pgfusepath{fill}%
\end{pgfscope}%
\begin{pgfscope}%
\pgfpathrectangle{\pgfqpoint{1.150000in}{0.150000in}}{\pgfqpoint{5.700000in}{5.700000in}}%
\pgfusepath{clip}%
\pgfsetbuttcap%
\pgfsetroundjoin%
\definecolor{currentfill}{rgb}{0.282327,0.094955,0.417331}%
\pgfsetfillcolor{currentfill}%
\pgfsetfillopacity{0.700000}%
\pgfsetlinewidth{0.000000pt}%
\definecolor{currentstroke}{rgb}{0.000000,0.000000,0.000000}%
\pgfsetstrokecolor{currentstroke}%
\pgfsetdash{}{0pt}%
\pgfpathmoveto{\pgfqpoint{2.798442in}{2.448906in}}%
\pgfpathlineto{\pgfqpoint{2.811738in}{2.441803in}}%
\pgfpathlineto{\pgfqpoint{2.825036in}{2.434747in}}%
\pgfpathlineto{\pgfqpoint{2.838338in}{2.427737in}}%
\pgfpathlineto{\pgfqpoint{2.851643in}{2.420773in}}%
\pgfpathlineto{\pgfqpoint{2.843280in}{2.417872in}}%
\pgfpathlineto{\pgfqpoint{2.834905in}{2.415144in}}%
\pgfpathlineto{\pgfqpoint{2.826518in}{2.412592in}}%
\pgfpathlineto{\pgfqpoint{2.818119in}{2.410223in}}%
\pgfpathlineto{\pgfqpoint{2.804790in}{2.417390in}}%
\pgfpathlineto{\pgfqpoint{2.791465in}{2.424603in}}%
\pgfpathlineto{\pgfqpoint{2.778142in}{2.431861in}}%
\pgfpathlineto{\pgfqpoint{2.764823in}{2.439166in}}%
\pgfpathlineto{\pgfqpoint{2.773246in}{2.441328in}}%
\pgfpathlineto{\pgfqpoint{2.781657in}{2.443675in}}%
\pgfpathlineto{\pgfqpoint{2.790056in}{2.446203in}}%
\pgfpathlineto{\pgfqpoint{2.798442in}{2.448906in}}%
\pgfpathclose%
\pgfusepath{fill}%
\end{pgfscope}%
\begin{pgfscope}%
\pgfpathrectangle{\pgfqpoint{1.150000in}{0.150000in}}{\pgfqpoint{5.700000in}{5.700000in}}%
\pgfusepath{clip}%
\pgfsetbuttcap%
\pgfsetroundjoin%
\definecolor{currentfill}{rgb}{0.268510,0.009605,0.335427}%
\pgfsetfillcolor{currentfill}%
\pgfsetfillopacity{0.700000}%
\pgfsetlinewidth{0.000000pt}%
\definecolor{currentstroke}{rgb}{0.000000,0.000000,0.000000}%
\pgfsetstrokecolor{currentstroke}%
\pgfsetdash{}{0pt}%
\pgfpathmoveto{\pgfqpoint{4.050941in}{2.302940in}}%
\pgfpathlineto{\pgfqpoint{4.064438in}{2.299621in}}%
\pgfpathlineto{\pgfqpoint{4.077942in}{2.296330in}}%
\pgfpathlineto{\pgfqpoint{4.091452in}{2.293068in}}%
\pgfpathlineto{\pgfqpoint{4.104968in}{2.289835in}}%
\pgfpathlineto{\pgfqpoint{4.097174in}{2.281916in}}%
\pgfpathlineto{\pgfqpoint{4.089374in}{2.273992in}}%
\pgfpathlineto{\pgfqpoint{4.081568in}{2.266064in}}%
\pgfpathlineto{\pgfqpoint{4.073757in}{2.258133in}}%
\pgfpathlineto{\pgfqpoint{4.060228in}{2.261448in}}%
\pgfpathlineto{\pgfqpoint{4.046706in}{2.264792in}}%
\pgfpathlineto{\pgfqpoint{4.033190in}{2.268164in}}%
\pgfpathlineto{\pgfqpoint{4.019680in}{2.271565in}}%
\pgfpathlineto{\pgfqpoint{4.027504in}{2.279409in}}%
\pgfpathlineto{\pgfqpoint{4.035322in}{2.287254in}}%
\pgfpathlineto{\pgfqpoint{4.043134in}{2.295098in}}%
\pgfpathlineto{\pgfqpoint{4.050941in}{2.302940in}}%
\pgfpathclose%
\pgfusepath{fill}%
\end{pgfscope}%
\begin{pgfscope}%
\pgfpathrectangle{\pgfqpoint{1.150000in}{0.150000in}}{\pgfqpoint{5.700000in}{5.700000in}}%
\pgfusepath{clip}%
\pgfsetbuttcap%
\pgfsetroundjoin%
\definecolor{currentfill}{rgb}{0.276022,0.044167,0.370164}%
\pgfsetfillcolor{currentfill}%
\pgfsetfillopacity{0.700000}%
\pgfsetlinewidth{0.000000pt}%
\definecolor{currentstroke}{rgb}{0.000000,0.000000,0.000000}%
\pgfsetstrokecolor{currentstroke}%
\pgfsetdash{}{0pt}%
\pgfpathmoveto{\pgfqpoint{4.499646in}{2.359850in}}%
\pgfpathlineto{\pgfqpoint{4.513254in}{2.357389in}}%
\pgfpathlineto{\pgfqpoint{4.526870in}{2.354955in}}%
\pgfpathlineto{\pgfqpoint{4.540493in}{2.352548in}}%
\pgfpathlineto{\pgfqpoint{4.554123in}{2.350167in}}%
\pgfpathlineto{\pgfqpoint{4.546492in}{2.342346in}}%
\pgfpathlineto{\pgfqpoint{4.538856in}{2.334485in}}%
\pgfpathlineto{\pgfqpoint{4.531213in}{2.326586in}}%
\pgfpathlineto{\pgfqpoint{4.523565in}{2.318647in}}%
\pgfpathlineto{\pgfqpoint{4.509923in}{2.321056in}}%
\pgfpathlineto{\pgfqpoint{4.496288in}{2.323491in}}%
\pgfpathlineto{\pgfqpoint{4.482660in}{2.325954in}}%
\pgfpathlineto{\pgfqpoint{4.469040in}{2.328443in}}%
\pgfpathlineto{\pgfqpoint{4.476700in}{2.336348in}}%
\pgfpathlineto{\pgfqpoint{4.484354in}{2.344218in}}%
\pgfpathlineto{\pgfqpoint{4.492003in}{2.352052in}}%
\pgfpathlineto{\pgfqpoint{4.499646in}{2.359850in}}%
\pgfpathclose%
\pgfusepath{fill}%
\end{pgfscope}%
\begin{pgfscope}%
\pgfpathrectangle{\pgfqpoint{1.150000in}{0.150000in}}{\pgfqpoint{5.700000in}{5.700000in}}%
\pgfusepath{clip}%
\pgfsetbuttcap%
\pgfsetroundjoin%
\definecolor{currentfill}{rgb}{0.282623,0.140926,0.457517}%
\pgfsetfillcolor{currentfill}%
\pgfsetfillopacity{0.700000}%
\pgfsetlinewidth{0.000000pt}%
\definecolor{currentstroke}{rgb}{0.000000,0.000000,0.000000}%
\pgfsetstrokecolor{currentstroke}%
\pgfsetdash{}{0pt}%
\pgfpathmoveto{\pgfqpoint{2.605206in}{2.530611in}}%
\pgfpathlineto{\pgfqpoint{2.618493in}{2.522713in}}%
\pgfpathlineto{\pgfqpoint{2.631783in}{2.514867in}}%
\pgfpathlineto{\pgfqpoint{2.645075in}{2.507073in}}%
\pgfpathlineto{\pgfqpoint{2.658370in}{2.499330in}}%
\pgfpathlineto{\pgfqpoint{2.649882in}{2.497782in}}%
\pgfpathlineto{\pgfqpoint{2.641381in}{2.496438in}}%
\pgfpathlineto{\pgfqpoint{2.632866in}{2.495302in}}%
\pgfpathlineto{\pgfqpoint{2.624337in}{2.494380in}}%
\pgfpathlineto{\pgfqpoint{2.611016in}{2.502341in}}%
\pgfpathlineto{\pgfqpoint{2.597697in}{2.510353in}}%
\pgfpathlineto{\pgfqpoint{2.584381in}{2.518416in}}%
\pgfpathlineto{\pgfqpoint{2.571066in}{2.526532in}}%
\pgfpathlineto{\pgfqpoint{2.579623in}{2.527231in}}%
\pgfpathlineto{\pgfqpoint{2.588165in}{2.528148in}}%
\pgfpathlineto{\pgfqpoint{2.596692in}{2.529276in}}%
\pgfpathlineto{\pgfqpoint{2.605206in}{2.530611in}}%
\pgfpathclose%
\pgfusepath{fill}%
\end{pgfscope}%
\begin{pgfscope}%
\pgfpathrectangle{\pgfqpoint{1.150000in}{0.150000in}}{\pgfqpoint{5.700000in}{5.700000in}}%
\pgfusepath{clip}%
\pgfsetbuttcap%
\pgfsetroundjoin%
\definecolor{currentfill}{rgb}{0.279566,0.067836,0.391917}%
\pgfsetfillcolor{currentfill}%
\pgfsetfillopacity{0.700000}%
\pgfsetlinewidth{0.000000pt}%
\definecolor{currentstroke}{rgb}{0.000000,0.000000,0.000000}%
\pgfsetstrokecolor{currentstroke}%
\pgfsetdash{}{0pt}%
\pgfpathmoveto{\pgfqpoint{4.724054in}{2.393366in}}%
\pgfpathlineto{\pgfqpoint{4.737724in}{2.391238in}}%
\pgfpathlineto{\pgfqpoint{4.751401in}{2.389137in}}%
\pgfpathlineto{\pgfqpoint{4.765086in}{2.387061in}}%
\pgfpathlineto{\pgfqpoint{4.778778in}{2.385011in}}%
\pgfpathlineto{\pgfqpoint{4.771232in}{2.377511in}}%
\pgfpathlineto{\pgfqpoint{4.763681in}{2.369963in}}%
\pgfpathlineto{\pgfqpoint{4.756123in}{2.362367in}}%
\pgfpathlineto{\pgfqpoint{4.748559in}{2.354723in}}%
\pgfpathlineto{\pgfqpoint{4.734855in}{2.356774in}}%
\pgfpathlineto{\pgfqpoint{4.721157in}{2.358851in}}%
\pgfpathlineto{\pgfqpoint{4.707468in}{2.360954in}}%
\pgfpathlineto{\pgfqpoint{4.693785in}{2.363083in}}%
\pgfpathlineto{\pgfqpoint{4.701361in}{2.370721in}}%
\pgfpathlineto{\pgfqpoint{4.708932in}{2.378314in}}%
\pgfpathlineto{\pgfqpoint{4.716496in}{2.385862in}}%
\pgfpathlineto{\pgfqpoint{4.724054in}{2.393366in}}%
\pgfpathclose%
\pgfusepath{fill}%
\end{pgfscope}%
\begin{pgfscope}%
\pgfpathrectangle{\pgfqpoint{1.150000in}{0.150000in}}{\pgfqpoint{5.700000in}{5.700000in}}%
\pgfusepath{clip}%
\pgfsetbuttcap%
\pgfsetroundjoin%
\definecolor{currentfill}{rgb}{0.278791,0.062145,0.386592}%
\pgfsetfillcolor{currentfill}%
\pgfsetfillopacity{0.700000}%
\pgfsetlinewidth{0.000000pt}%
\definecolor{currentstroke}{rgb}{0.000000,0.000000,0.000000}%
\pgfsetstrokecolor{currentstroke}%
\pgfsetdash{}{0pt}%
\pgfpathmoveto{\pgfqpoint{2.991368in}{2.381373in}}%
\pgfpathlineto{\pgfqpoint{3.004682in}{2.374988in}}%
\pgfpathlineto{\pgfqpoint{3.018001in}{2.368644in}}%
\pgfpathlineto{\pgfqpoint{3.031323in}{2.362341in}}%
\pgfpathlineto{\pgfqpoint{3.044649in}{2.356078in}}%
\pgfpathlineto{\pgfqpoint{3.036394in}{2.351989in}}%
\pgfpathlineto{\pgfqpoint{3.028130in}{2.348041in}}%
\pgfpathlineto{\pgfqpoint{3.019856in}{2.344241in}}%
\pgfpathlineto{\pgfqpoint{3.011570in}{2.340593in}}%
\pgfpathlineto{\pgfqpoint{2.998223in}{2.347043in}}%
\pgfpathlineto{\pgfqpoint{2.984880in}{2.353535in}}%
\pgfpathlineto{\pgfqpoint{2.971540in}{2.360067in}}%
\pgfpathlineto{\pgfqpoint{2.958204in}{2.366641in}}%
\pgfpathlineto{\pgfqpoint{2.966511in}{2.370096in}}%
\pgfpathlineto{\pgfqpoint{2.974807in}{2.373706in}}%
\pgfpathlineto{\pgfqpoint{2.983093in}{2.377467in}}%
\pgfpathlineto{\pgfqpoint{2.991368in}{2.381373in}}%
\pgfpathclose%
\pgfusepath{fill}%
\end{pgfscope}%
\begin{pgfscope}%
\pgfpathrectangle{\pgfqpoint{1.150000in}{0.150000in}}{\pgfqpoint{5.700000in}{5.700000in}}%
\pgfusepath{clip}%
\pgfsetbuttcap%
\pgfsetroundjoin%
\definecolor{currentfill}{rgb}{0.267004,0.004874,0.329415}%
\pgfsetfillcolor{currentfill}%
\pgfsetfillopacity{0.700000}%
\pgfsetlinewidth{0.000000pt}%
\definecolor{currentstroke}{rgb}{0.000000,0.000000,0.000000}%
\pgfsetstrokecolor{currentstroke}%
\pgfsetdash{}{0pt}%
\pgfpathmoveto{\pgfqpoint{3.826535in}{2.284478in}}%
\pgfpathlineto{\pgfqpoint{3.839985in}{2.280627in}}%
\pgfpathlineto{\pgfqpoint{3.853441in}{2.276807in}}%
\pgfpathlineto{\pgfqpoint{3.866902in}{2.273016in}}%
\pgfpathlineto{\pgfqpoint{3.880370in}{2.269256in}}%
\pgfpathlineto{\pgfqpoint{3.872491in}{2.261653in}}%
\pgfpathlineto{\pgfqpoint{3.864607in}{2.254069in}}%
\pgfpathlineto{\pgfqpoint{3.856717in}{2.246509in}}%
\pgfpathlineto{\pgfqpoint{3.848820in}{2.238973in}}%
\pgfpathlineto{\pgfqpoint{3.835339in}{2.242841in}}%
\pgfpathlineto{\pgfqpoint{3.821864in}{2.246740in}}%
\pgfpathlineto{\pgfqpoint{3.808395in}{2.250668in}}%
\pgfpathlineto{\pgfqpoint{3.794932in}{2.254627in}}%
\pgfpathlineto{\pgfqpoint{3.802842in}{2.262050in}}%
\pgfpathlineto{\pgfqpoint{3.810746in}{2.269501in}}%
\pgfpathlineto{\pgfqpoint{3.818643in}{2.276978in}}%
\pgfpathlineto{\pgfqpoint{3.826535in}{2.284478in}}%
\pgfpathclose%
\pgfusepath{fill}%
\end{pgfscope}%
\begin{pgfscope}%
\pgfpathrectangle{\pgfqpoint{1.150000in}{0.150000in}}{\pgfqpoint{5.700000in}{5.700000in}}%
\pgfusepath{clip}%
\pgfsetbuttcap%
\pgfsetroundjoin%
\definecolor{currentfill}{rgb}{0.281924,0.089666,0.412415}%
\pgfsetfillcolor{currentfill}%
\pgfsetfillopacity{0.700000}%
\pgfsetlinewidth{0.000000pt}%
\definecolor{currentstroke}{rgb}{0.000000,0.000000,0.000000}%
\pgfsetstrokecolor{currentstroke}%
\pgfsetdash{}{0pt}%
\pgfpathmoveto{\pgfqpoint{4.948514in}{2.427630in}}%
\pgfpathlineto{\pgfqpoint{4.962247in}{2.425774in}}%
\pgfpathlineto{\pgfqpoint{4.975987in}{2.423943in}}%
\pgfpathlineto{\pgfqpoint{4.989736in}{2.422138in}}%
\pgfpathlineto{\pgfqpoint{5.003492in}{2.420358in}}%
\pgfpathlineto{\pgfqpoint{4.996036in}{2.413299in}}%
\pgfpathlineto{\pgfqpoint{4.988574in}{2.406190in}}%
\pgfpathlineto{\pgfqpoint{4.981106in}{2.399029in}}%
\pgfpathlineto{\pgfqpoint{4.973631in}{2.391816in}}%
\pgfpathlineto{\pgfqpoint{4.959862in}{2.393570in}}%
\pgfpathlineto{\pgfqpoint{4.946100in}{2.395350in}}%
\pgfpathlineto{\pgfqpoint{4.932346in}{2.397155in}}%
\pgfpathlineto{\pgfqpoint{4.918599in}{2.398986in}}%
\pgfpathlineto{\pgfqpoint{4.926087in}{2.406219in}}%
\pgfpathlineto{\pgfqpoint{4.933569in}{2.413403in}}%
\pgfpathlineto{\pgfqpoint{4.941044in}{2.420540in}}%
\pgfpathlineto{\pgfqpoint{4.948514in}{2.427630in}}%
\pgfpathclose%
\pgfusepath{fill}%
\end{pgfscope}%
\begin{pgfscope}%
\pgfpathrectangle{\pgfqpoint{1.150000in}{0.150000in}}{\pgfqpoint{5.700000in}{5.700000in}}%
\pgfusepath{clip}%
\pgfsetbuttcap%
\pgfsetroundjoin%
\definecolor{currentfill}{rgb}{0.280255,0.165693,0.476498}%
\pgfsetfillcolor{currentfill}%
\pgfsetfillopacity{0.700000}%
\pgfsetlinewidth{0.000000pt}%
\definecolor{currentstroke}{rgb}{0.000000,0.000000,0.000000}%
\pgfsetstrokecolor{currentstroke}%
\pgfsetdash{}{0pt}%
\pgfpathmoveto{\pgfqpoint{6.070473in}{2.572039in}}%
\pgfpathlineto{\pgfqpoint{6.084515in}{2.570658in}}%
\pgfpathlineto{\pgfqpoint{6.098566in}{2.569300in}}%
\pgfpathlineto{\pgfqpoint{6.112625in}{2.567966in}}%
\pgfpathlineto{\pgfqpoint{6.126692in}{2.566656in}}%
\pgfpathlineto{\pgfqpoint{6.119751in}{2.561951in}}%
\pgfpathlineto{\pgfqpoint{6.112806in}{2.557273in}}%
\pgfpathlineto{\pgfqpoint{6.105855in}{2.552618in}}%
\pgfpathlineto{\pgfqpoint{6.098900in}{2.547980in}}%
\pgfpathlineto{\pgfqpoint{6.084809in}{2.549130in}}%
\pgfpathlineto{\pgfqpoint{6.070728in}{2.550304in}}%
\pgfpathlineto{\pgfqpoint{6.056655in}{2.551501in}}%
\pgfpathlineto{\pgfqpoint{6.042590in}{2.552723in}}%
\pgfpathlineto{\pgfqpoint{6.049568in}{2.557516in}}%
\pgfpathlineto{\pgfqpoint{6.056541in}{2.562330in}}%
\pgfpathlineto{\pgfqpoint{6.063509in}{2.567169in}}%
\pgfpathlineto{\pgfqpoint{6.070473in}{2.572039in}}%
\pgfpathclose%
\pgfusepath{fill}%
\end{pgfscope}%
\begin{pgfscope}%
\pgfpathrectangle{\pgfqpoint{1.150000in}{0.150000in}}{\pgfqpoint{5.700000in}{5.700000in}}%
\pgfusepath{clip}%
\pgfsetbuttcap%
\pgfsetroundjoin%
\definecolor{currentfill}{rgb}{0.282910,0.105393,0.426902}%
\pgfsetfillcolor{currentfill}%
\pgfsetfillopacity{0.700000}%
\pgfsetlinewidth{0.000000pt}%
\definecolor{currentstroke}{rgb}{0.000000,0.000000,0.000000}%
\pgfsetstrokecolor{currentstroke}%
\pgfsetdash{}{0pt}%
\pgfpathmoveto{\pgfqpoint{5.173003in}{2.461162in}}%
\pgfpathlineto{\pgfqpoint{5.186800in}{2.459517in}}%
\pgfpathlineto{\pgfqpoint{5.200605in}{2.457898in}}%
\pgfpathlineto{\pgfqpoint{5.214418in}{2.456303in}}%
\pgfpathlineto{\pgfqpoint{5.228239in}{2.454734in}}%
\pgfpathlineto{\pgfqpoint{5.220879in}{2.448188in}}%
\pgfpathlineto{\pgfqpoint{5.213512in}{2.441597in}}%
\pgfpathlineto{\pgfqpoint{5.206138in}{2.434957in}}%
\pgfpathlineto{\pgfqpoint{5.198758in}{2.428267in}}%
\pgfpathlineto{\pgfqpoint{5.184923in}{2.429784in}}%
\pgfpathlineto{\pgfqpoint{5.171095in}{2.431326in}}%
\pgfpathlineto{\pgfqpoint{5.157275in}{2.432893in}}%
\pgfpathlineto{\pgfqpoint{5.143463in}{2.434485in}}%
\pgfpathlineto{\pgfqpoint{5.150858in}{2.441223in}}%
\pgfpathlineto{\pgfqpoint{5.158246in}{2.447914in}}%
\pgfpathlineto{\pgfqpoint{5.165628in}{2.454559in}}%
\pgfpathlineto{\pgfqpoint{5.173003in}{2.461162in}}%
\pgfpathclose%
\pgfusepath{fill}%
\end{pgfscope}%
\begin{pgfscope}%
\pgfpathrectangle{\pgfqpoint{1.150000in}{0.150000in}}{\pgfqpoint{5.700000in}{5.700000in}}%
\pgfusepath{clip}%
\pgfsetbuttcap%
\pgfsetroundjoin%
\definecolor{currentfill}{rgb}{0.281412,0.155834,0.469201}%
\pgfsetfillcolor{currentfill}%
\pgfsetfillopacity{0.700000}%
\pgfsetlinewidth{0.000000pt}%
\definecolor{currentstroke}{rgb}{0.000000,0.000000,0.000000}%
\pgfsetstrokecolor{currentstroke}%
\pgfsetdash{}{0pt}%
\pgfpathmoveto{\pgfqpoint{5.846259in}{2.548421in}}%
\pgfpathlineto{\pgfqpoint{5.860243in}{2.547059in}}%
\pgfpathlineto{\pgfqpoint{5.874235in}{2.545722in}}%
\pgfpathlineto{\pgfqpoint{5.888236in}{2.544408in}}%
\pgfpathlineto{\pgfqpoint{5.902245in}{2.543119in}}%
\pgfpathlineto{\pgfqpoint{5.895197in}{2.538082in}}%
\pgfpathlineto{\pgfqpoint{5.888144in}{2.533046in}}%
\pgfpathlineto{\pgfqpoint{5.881085in}{2.528005in}}%
\pgfpathlineto{\pgfqpoint{5.874020in}{2.522956in}}%
\pgfpathlineto{\pgfqpoint{5.859990in}{2.524112in}}%
\pgfpathlineto{\pgfqpoint{5.845969in}{2.525292in}}%
\pgfpathlineto{\pgfqpoint{5.831956in}{2.526496in}}%
\pgfpathlineto{\pgfqpoint{5.817952in}{2.527724in}}%
\pgfpathlineto{\pgfqpoint{5.825037in}{2.532902in}}%
\pgfpathlineto{\pgfqpoint{5.832117in}{2.538074in}}%
\pgfpathlineto{\pgfqpoint{5.839190in}{2.543246in}}%
\pgfpathlineto{\pgfqpoint{5.846259in}{2.548421in}}%
\pgfpathclose%
\pgfusepath{fill}%
\end{pgfscope}%
\begin{pgfscope}%
\pgfpathrectangle{\pgfqpoint{1.150000in}{0.150000in}}{\pgfqpoint{5.700000in}{5.700000in}}%
\pgfusepath{clip}%
\pgfsetbuttcap%
\pgfsetroundjoin%
\definecolor{currentfill}{rgb}{0.271305,0.019942,0.347269}%
\pgfsetfillcolor{currentfill}%
\pgfsetfillopacity{0.700000}%
\pgfsetlinewidth{0.000000pt}%
\definecolor{currentstroke}{rgb}{0.000000,0.000000,0.000000}%
\pgfsetstrokecolor{currentstroke}%
\pgfsetdash{}{0pt}%
\pgfpathmoveto{\pgfqpoint{4.190169in}{2.309047in}}%
\pgfpathlineto{\pgfqpoint{4.203705in}{2.306024in}}%
\pgfpathlineto{\pgfqpoint{4.217248in}{2.303028in}}%
\pgfpathlineto{\pgfqpoint{4.230797in}{2.300060in}}%
\pgfpathlineto{\pgfqpoint{4.244353in}{2.297120in}}%
\pgfpathlineto{\pgfqpoint{4.236606in}{2.289112in}}%
\pgfpathlineto{\pgfqpoint{4.228853in}{2.281086in}}%
\pgfpathlineto{\pgfqpoint{4.221094in}{2.273043in}}%
\pgfpathlineto{\pgfqpoint{4.213330in}{2.264985in}}%
\pgfpathlineto{\pgfqpoint{4.199762in}{2.267993in}}%
\pgfpathlineto{\pgfqpoint{4.186200in}{2.271029in}}%
\pgfpathlineto{\pgfqpoint{4.172645in}{2.274093in}}%
\pgfpathlineto{\pgfqpoint{4.159097in}{2.277185in}}%
\pgfpathlineto{\pgfqpoint{4.166873in}{2.285170in}}%
\pgfpathlineto{\pgfqpoint{4.174644in}{2.293143in}}%
\pgfpathlineto{\pgfqpoint{4.182409in}{2.301102in}}%
\pgfpathlineto{\pgfqpoint{4.190169in}{2.309047in}}%
\pgfpathclose%
\pgfusepath{fill}%
\end{pgfscope}%
\begin{pgfscope}%
\pgfpathrectangle{\pgfqpoint{1.150000in}{0.150000in}}{\pgfqpoint{5.700000in}{5.700000in}}%
\pgfusepath{clip}%
\pgfsetbuttcap%
\pgfsetroundjoin%
\definecolor{currentfill}{rgb}{0.283187,0.125848,0.444960}%
\pgfsetfillcolor{currentfill}%
\pgfsetfillopacity{0.700000}%
\pgfsetlinewidth{0.000000pt}%
\definecolor{currentstroke}{rgb}{0.000000,0.000000,0.000000}%
\pgfsetstrokecolor{currentstroke}%
\pgfsetdash{}{0pt}%
\pgfpathmoveto{\pgfqpoint{5.397486in}{2.492872in}}%
\pgfpathlineto{\pgfqpoint{5.411347in}{2.491380in}}%
\pgfpathlineto{\pgfqpoint{5.425216in}{2.489912in}}%
\pgfpathlineto{\pgfqpoint{5.439093in}{2.488469in}}%
\pgfpathlineto{\pgfqpoint{5.452978in}{2.487051in}}%
\pgfpathlineto{\pgfqpoint{5.445718in}{2.481046in}}%
\pgfpathlineto{\pgfqpoint{5.438452in}{2.475004in}}%
\pgfpathlineto{\pgfqpoint{5.431179in}{2.468923in}}%
\pgfpathlineto{\pgfqpoint{5.423900in}{2.462799in}}%
\pgfpathlineto{\pgfqpoint{5.409998in}{2.464138in}}%
\pgfpathlineto{\pgfqpoint{5.396105in}{2.465502in}}%
\pgfpathlineto{\pgfqpoint{5.382219in}{2.466890in}}%
\pgfpathlineto{\pgfqpoint{5.368342in}{2.468303in}}%
\pgfpathlineto{\pgfqpoint{5.375638in}{2.474501in}}%
\pgfpathlineto{\pgfqpoint{5.382927in}{2.480660in}}%
\pgfpathlineto{\pgfqpoint{5.390210in}{2.486783in}}%
\pgfpathlineto{\pgfqpoint{5.397486in}{2.492872in}}%
\pgfpathclose%
\pgfusepath{fill}%
\end{pgfscope}%
\begin{pgfscope}%
\pgfpathrectangle{\pgfqpoint{1.150000in}{0.150000in}}{\pgfqpoint{5.700000in}{5.700000in}}%
\pgfusepath{clip}%
\pgfsetbuttcap%
\pgfsetroundjoin%
\definecolor{currentfill}{rgb}{0.282623,0.140926,0.457517}%
\pgfsetfillcolor{currentfill}%
\pgfsetfillopacity{0.700000}%
\pgfsetlinewidth{0.000000pt}%
\definecolor{currentstroke}{rgb}{0.000000,0.000000,0.000000}%
\pgfsetstrokecolor{currentstroke}%
\pgfsetdash{}{0pt}%
\pgfpathmoveto{\pgfqpoint{5.621919in}{2.522061in}}%
\pgfpathlineto{\pgfqpoint{5.635842in}{2.520663in}}%
\pgfpathlineto{\pgfqpoint{5.649774in}{2.519289in}}%
\pgfpathlineto{\pgfqpoint{5.663714in}{2.517939in}}%
\pgfpathlineto{\pgfqpoint{5.677662in}{2.516614in}}%
\pgfpathlineto{\pgfqpoint{5.670507in}{2.511128in}}%
\pgfpathlineto{\pgfqpoint{5.663346in}{2.505621in}}%
\pgfpathlineto{\pgfqpoint{5.656178in}{2.500088in}}%
\pgfpathlineto{\pgfqpoint{5.649004in}{2.494528in}}%
\pgfpathlineto{\pgfqpoint{5.635038in}{2.495747in}}%
\pgfpathlineto{\pgfqpoint{5.621079in}{2.496990in}}%
\pgfpathlineto{\pgfqpoint{5.607130in}{2.498257in}}%
\pgfpathlineto{\pgfqpoint{5.593188in}{2.499549in}}%
\pgfpathlineto{\pgfqpoint{5.600380in}{2.505211in}}%
\pgfpathlineto{\pgfqpoint{5.607566in}{2.510848in}}%
\pgfpathlineto{\pgfqpoint{5.614745in}{2.516464in}}%
\pgfpathlineto{\pgfqpoint{5.621919in}{2.522061in}}%
\pgfpathclose%
\pgfusepath{fill}%
\end{pgfscope}%
\begin{pgfscope}%
\pgfpathrectangle{\pgfqpoint{1.150000in}{0.150000in}}{\pgfqpoint{5.700000in}{5.700000in}}%
\pgfusepath{clip}%
\pgfsetbuttcap%
\pgfsetroundjoin%
\definecolor{currentfill}{rgb}{0.274952,0.037752,0.364543}%
\pgfsetfillcolor{currentfill}%
\pgfsetfillopacity{0.700000}%
\pgfsetlinewidth{0.000000pt}%
\definecolor{currentstroke}{rgb}{0.000000,0.000000,0.000000}%
\pgfsetstrokecolor{currentstroke}%
\pgfsetdash{}{0pt}%
\pgfpathmoveto{\pgfqpoint{4.414626in}{2.338667in}}%
\pgfpathlineto{\pgfqpoint{4.428219in}{2.336071in}}%
\pgfpathlineto{\pgfqpoint{4.441819in}{2.333501in}}%
\pgfpathlineto{\pgfqpoint{4.455426in}{2.330958in}}%
\pgfpathlineto{\pgfqpoint{4.469040in}{2.328443in}}%
\pgfpathlineto{\pgfqpoint{4.461374in}{2.320502in}}%
\pgfpathlineto{\pgfqpoint{4.453702in}{2.312527in}}%
\pgfpathlineto{\pgfqpoint{4.446024in}{2.304517in}}%
\pgfpathlineto{\pgfqpoint{4.438341in}{2.296474in}}%
\pgfpathlineto{\pgfqpoint{4.424715in}{2.299031in}}%
\pgfpathlineto{\pgfqpoint{4.411097in}{2.301616in}}%
\pgfpathlineto{\pgfqpoint{4.397485in}{2.304227in}}%
\pgfpathlineto{\pgfqpoint{4.383880in}{2.306865in}}%
\pgfpathlineto{\pgfqpoint{4.391575in}{2.314862in}}%
\pgfpathlineto{\pgfqpoint{4.399265in}{2.322828in}}%
\pgfpathlineto{\pgfqpoint{4.406948in}{2.330763in}}%
\pgfpathlineto{\pgfqpoint{4.414626in}{2.338667in}}%
\pgfpathclose%
\pgfusepath{fill}%
\end{pgfscope}%
\begin{pgfscope}%
\pgfpathrectangle{\pgfqpoint{1.150000in}{0.150000in}}{\pgfqpoint{5.700000in}{5.700000in}}%
\pgfusepath{clip}%
\pgfsetbuttcap%
\pgfsetroundjoin%
\definecolor{currentfill}{rgb}{0.268510,0.009605,0.335427}%
\pgfsetfillcolor{currentfill}%
\pgfsetfillopacity{0.700000}%
\pgfsetlinewidth{0.000000pt}%
\definecolor{currentstroke}{rgb}{0.000000,0.000000,0.000000}%
\pgfsetstrokecolor{currentstroke}%
\pgfsetdash{}{0pt}%
\pgfpathmoveto{\pgfqpoint{3.462750in}{2.287475in}}%
\pgfpathlineto{\pgfqpoint{3.476134in}{2.282621in}}%
\pgfpathlineto{\pgfqpoint{3.489523in}{2.277800in}}%
\pgfpathlineto{\pgfqpoint{3.502917in}{2.273013in}}%
\pgfpathlineto{\pgfqpoint{3.516316in}{2.268259in}}%
\pgfpathlineto{\pgfqpoint{3.508289in}{2.261786in}}%
\pgfpathlineto{\pgfqpoint{3.500255in}{2.255384in}}%
\pgfpathlineto{\pgfqpoint{3.492214in}{2.249057in}}%
\pgfpathlineto{\pgfqpoint{3.484165in}{2.242808in}}%
\pgfpathlineto{\pgfqpoint{3.470750in}{2.247709in}}%
\pgfpathlineto{\pgfqpoint{3.457340in}{2.252644in}}%
\pgfpathlineto{\pgfqpoint{3.443934in}{2.257612in}}%
\pgfpathlineto{\pgfqpoint{3.430534in}{2.262614in}}%
\pgfpathlineto{\pgfqpoint{3.438599in}{2.268711in}}%
\pgfpathlineto{\pgfqpoint{3.446657in}{2.274889in}}%
\pgfpathlineto{\pgfqpoint{3.454707in}{2.281145in}}%
\pgfpathlineto{\pgfqpoint{3.462750in}{2.287475in}}%
\pgfpathclose%
\pgfusepath{fill}%
\end{pgfscope}%
\begin{pgfscope}%
\pgfpathrectangle{\pgfqpoint{1.150000in}{0.150000in}}{\pgfqpoint{5.700000in}{5.700000in}}%
\pgfusepath{clip}%
\pgfsetbuttcap%
\pgfsetroundjoin%
\definecolor{currentfill}{rgb}{0.269944,0.014625,0.341379}%
\pgfsetfillcolor{currentfill}%
\pgfsetfillopacity{0.700000}%
\pgfsetlinewidth{0.000000pt}%
\definecolor{currentstroke}{rgb}{0.000000,0.000000,0.000000}%
\pgfsetstrokecolor{currentstroke}%
\pgfsetdash{}{0pt}%
\pgfpathmoveto{\pgfqpoint{3.323508in}{2.303872in}}%
\pgfpathlineto{\pgfqpoint{3.336870in}{2.298592in}}%
\pgfpathlineto{\pgfqpoint{3.350236in}{2.293348in}}%
\pgfpathlineto{\pgfqpoint{3.363607in}{2.288139in}}%
\pgfpathlineto{\pgfqpoint{3.376982in}{2.282965in}}%
\pgfpathlineto{\pgfqpoint{3.368893in}{2.277109in}}%
\pgfpathlineto{\pgfqpoint{3.360796in}{2.271346in}}%
\pgfpathlineto{\pgfqpoint{3.352690in}{2.265679in}}%
\pgfpathlineto{\pgfqpoint{3.344577in}{2.260112in}}%
\pgfpathlineto{\pgfqpoint{3.331184in}{2.265447in}}%
\pgfpathlineto{\pgfqpoint{3.317795in}{2.270817in}}%
\pgfpathlineto{\pgfqpoint{3.304411in}{2.276222in}}%
\pgfpathlineto{\pgfqpoint{3.291032in}{2.281664in}}%
\pgfpathlineto{\pgfqpoint{3.299164in}{2.287065in}}%
\pgfpathlineto{\pgfqpoint{3.307287in}{2.292569in}}%
\pgfpathlineto{\pgfqpoint{3.315401in}{2.298173in}}%
\pgfpathlineto{\pgfqpoint{3.323508in}{2.303872in}}%
\pgfpathclose%
\pgfusepath{fill}%
\end{pgfscope}%
\begin{pgfscope}%
\pgfpathrectangle{\pgfqpoint{1.150000in}{0.150000in}}{\pgfqpoint{5.700000in}{5.700000in}}%
\pgfusepath{clip}%
\pgfsetbuttcap%
\pgfsetroundjoin%
\definecolor{currentfill}{rgb}{0.268510,0.009605,0.335427}%
\pgfsetfillcolor{currentfill}%
\pgfsetfillopacity{0.700000}%
\pgfsetlinewidth{0.000000pt}%
\definecolor{currentstroke}{rgb}{0.000000,0.000000,0.000000}%
\pgfsetstrokecolor{currentstroke}%
\pgfsetdash{}{0pt}%
\pgfpathmoveto{\pgfqpoint{3.965704in}{2.285460in}}%
\pgfpathlineto{\pgfqpoint{3.979189in}{2.281942in}}%
\pgfpathlineto{\pgfqpoint{3.992680in}{2.278454in}}%
\pgfpathlineto{\pgfqpoint{4.006177in}{2.274995in}}%
\pgfpathlineto{\pgfqpoint{4.019680in}{2.271565in}}%
\pgfpathlineto{\pgfqpoint{4.011851in}{2.263724in}}%
\pgfpathlineto{\pgfqpoint{4.004016in}{2.255888in}}%
\pgfpathlineto{\pgfqpoint{3.996175in}{2.248059in}}%
\pgfpathlineto{\pgfqpoint{3.988328in}{2.240238in}}%
\pgfpathlineto{\pgfqpoint{3.974812in}{2.243763in}}%
\pgfpathlineto{\pgfqpoint{3.961302in}{2.247317in}}%
\pgfpathlineto{\pgfqpoint{3.947798in}{2.250900in}}%
\pgfpathlineto{\pgfqpoint{3.934300in}{2.254512in}}%
\pgfpathlineto{\pgfqpoint{3.942160in}{2.262233in}}%
\pgfpathlineto{\pgfqpoint{3.950014in}{2.269965in}}%
\pgfpathlineto{\pgfqpoint{3.957862in}{2.277708in}}%
\pgfpathlineto{\pgfqpoint{3.965704in}{2.285460in}}%
\pgfpathclose%
\pgfusepath{fill}%
\end{pgfscope}%
\begin{pgfscope}%
\pgfpathrectangle{\pgfqpoint{1.150000in}{0.150000in}}{\pgfqpoint{5.700000in}{5.700000in}}%
\pgfusepath{clip}%
\pgfsetbuttcap%
\pgfsetroundjoin%
\definecolor{currentfill}{rgb}{0.278791,0.062145,0.386592}%
\pgfsetfillcolor{currentfill}%
\pgfsetfillopacity{0.700000}%
\pgfsetlinewidth{0.000000pt}%
\definecolor{currentstroke}{rgb}{0.000000,0.000000,0.000000}%
\pgfsetstrokecolor{currentstroke}%
\pgfsetdash{}{0pt}%
\pgfpathmoveto{\pgfqpoint{4.639128in}{2.371860in}}%
\pgfpathlineto{\pgfqpoint{4.652781in}{2.369627in}}%
\pgfpathlineto{\pgfqpoint{4.666442in}{2.367419in}}%
\pgfpathlineto{\pgfqpoint{4.680110in}{2.365238in}}%
\pgfpathlineto{\pgfqpoint{4.693785in}{2.363083in}}%
\pgfpathlineto{\pgfqpoint{4.686203in}{2.355399in}}%
\pgfpathlineto{\pgfqpoint{4.678614in}{2.347670in}}%
\pgfpathlineto{\pgfqpoint{4.671020in}{2.339894in}}%
\pgfpathlineto{\pgfqpoint{4.663419in}{2.332072in}}%
\pgfpathlineto{\pgfqpoint{4.649732in}{2.334242in}}%
\pgfpathlineto{\pgfqpoint{4.636052in}{2.336439in}}%
\pgfpathlineto{\pgfqpoint{4.622379in}{2.338661in}}%
\pgfpathlineto{\pgfqpoint{4.608713in}{2.340909in}}%
\pgfpathlineto{\pgfqpoint{4.616326in}{2.348711in}}%
\pgfpathlineto{\pgfqpoint{4.623933in}{2.356470in}}%
\pgfpathlineto{\pgfqpoint{4.631533in}{2.364186in}}%
\pgfpathlineto{\pgfqpoint{4.639128in}{2.371860in}}%
\pgfpathclose%
\pgfusepath{fill}%
\end{pgfscope}%
\begin{pgfscope}%
\pgfpathrectangle{\pgfqpoint{1.150000in}{0.150000in}}{\pgfqpoint{5.700000in}{5.700000in}}%
\pgfusepath{clip}%
\pgfsetbuttcap%
\pgfsetroundjoin%
\definecolor{currentfill}{rgb}{0.267004,0.004874,0.329415}%
\pgfsetfillcolor{currentfill}%
\pgfsetfillopacity{0.700000}%
\pgfsetlinewidth{0.000000pt}%
\definecolor{currentstroke}{rgb}{0.000000,0.000000,0.000000}%
\pgfsetstrokecolor{currentstroke}%
\pgfsetdash{}{0pt}%
\pgfpathmoveto{\pgfqpoint{3.601943in}{2.276647in}}%
\pgfpathlineto{\pgfqpoint{3.615354in}{2.272192in}}%
\pgfpathlineto{\pgfqpoint{3.628769in}{2.267769in}}%
\pgfpathlineto{\pgfqpoint{3.642191in}{2.263377in}}%
\pgfpathlineto{\pgfqpoint{3.655617in}{2.259018in}}%
\pgfpathlineto{\pgfqpoint{3.647648in}{2.252035in}}%
\pgfpathlineto{\pgfqpoint{3.639672in}{2.245103in}}%
\pgfpathlineto{\pgfqpoint{3.631689in}{2.238226in}}%
\pgfpathlineto{\pgfqpoint{3.623699in}{2.231406in}}%
\pgfpathlineto{\pgfqpoint{3.610258in}{2.235900in}}%
\pgfpathlineto{\pgfqpoint{3.596822in}{2.240426in}}%
\pgfpathlineto{\pgfqpoint{3.583391in}{2.244984in}}%
\pgfpathlineto{\pgfqpoint{3.569966in}{2.249574in}}%
\pgfpathlineto{\pgfqpoint{3.577970in}{2.256254in}}%
\pgfpathlineto{\pgfqpoint{3.585968in}{2.262995in}}%
\pgfpathlineto{\pgfqpoint{3.593959in}{2.269794in}}%
\pgfpathlineto{\pgfqpoint{3.601943in}{2.276647in}}%
\pgfpathclose%
\pgfusepath{fill}%
\end{pgfscope}%
\begin{pgfscope}%
\pgfpathrectangle{\pgfqpoint{1.150000in}{0.150000in}}{\pgfqpoint{5.700000in}{5.700000in}}%
\pgfusepath{clip}%
\pgfsetbuttcap%
\pgfsetroundjoin%
\definecolor{currentfill}{rgb}{0.273809,0.031497,0.358853}%
\pgfsetfillcolor{currentfill}%
\pgfsetfillopacity{0.700000}%
\pgfsetlinewidth{0.000000pt}%
\definecolor{currentstroke}{rgb}{0.000000,0.000000,0.000000}%
\pgfsetstrokecolor{currentstroke}%
\pgfsetdash{}{0pt}%
\pgfpathmoveto{\pgfqpoint{3.184162in}{2.326503in}}%
\pgfpathlineto{\pgfqpoint{3.197505in}{2.320768in}}%
\pgfpathlineto{\pgfqpoint{3.210853in}{2.315071in}}%
\pgfpathlineto{\pgfqpoint{3.224205in}{2.309411in}}%
\pgfpathlineto{\pgfqpoint{3.237562in}{2.303789in}}%
\pgfpathlineto{\pgfqpoint{3.229403in}{2.298664in}}%
\pgfpathlineto{\pgfqpoint{3.221237in}{2.293655in}}%
\pgfpathlineto{\pgfqpoint{3.213061in}{2.288765in}}%
\pgfpathlineto{\pgfqpoint{3.204876in}{2.283998in}}%
\pgfpathlineto{\pgfqpoint{3.191501in}{2.289795in}}%
\pgfpathlineto{\pgfqpoint{3.178130in}{2.295629in}}%
\pgfpathlineto{\pgfqpoint{3.164763in}{2.301501in}}%
\pgfpathlineto{\pgfqpoint{3.151400in}{2.307411in}}%
\pgfpathlineto{\pgfqpoint{3.159605in}{2.311998in}}%
\pgfpathlineto{\pgfqpoint{3.167800in}{2.316712in}}%
\pgfpathlineto{\pgfqpoint{3.175985in}{2.321549in}}%
\pgfpathlineto{\pgfqpoint{3.184162in}{2.326503in}}%
\pgfpathclose%
\pgfusepath{fill}%
\end{pgfscope}%
\begin{pgfscope}%
\pgfpathrectangle{\pgfqpoint{1.150000in}{0.150000in}}{\pgfqpoint{5.700000in}{5.700000in}}%
\pgfusepath{clip}%
\pgfsetbuttcap%
\pgfsetroundjoin%
\definecolor{currentfill}{rgb}{0.281924,0.089666,0.412415}%
\pgfsetfillcolor{currentfill}%
\pgfsetfillopacity{0.700000}%
\pgfsetlinewidth{0.000000pt}%
\definecolor{currentstroke}{rgb}{0.000000,0.000000,0.000000}%
\pgfsetstrokecolor{currentstroke}%
\pgfsetdash{}{0pt}%
\pgfpathmoveto{\pgfqpoint{2.851643in}{2.420773in}}%
\pgfpathlineto{\pgfqpoint{2.864951in}{2.413853in}}%
\pgfpathlineto{\pgfqpoint{2.878263in}{2.406978in}}%
\pgfpathlineto{\pgfqpoint{2.891578in}{2.400147in}}%
\pgfpathlineto{\pgfqpoint{2.904896in}{2.393360in}}%
\pgfpathlineto{\pgfqpoint{2.896556in}{2.390262in}}%
\pgfpathlineto{\pgfqpoint{2.888204in}{2.387333in}}%
\pgfpathlineto{\pgfqpoint{2.879841in}{2.384577in}}%
\pgfpathlineto{\pgfqpoint{2.871465in}{2.382000in}}%
\pgfpathlineto{\pgfqpoint{2.858124in}{2.388990in}}%
\pgfpathlineto{\pgfqpoint{2.844786in}{2.396023in}}%
\pgfpathlineto{\pgfqpoint{2.831451in}{2.403101in}}%
\pgfpathlineto{\pgfqpoint{2.818119in}{2.410223in}}%
\pgfpathlineto{\pgfqpoint{2.826518in}{2.412592in}}%
\pgfpathlineto{\pgfqpoint{2.834905in}{2.415144in}}%
\pgfpathlineto{\pgfqpoint{2.843280in}{2.417872in}}%
\pgfpathlineto{\pgfqpoint{2.851643in}{2.420773in}}%
\pgfpathclose%
\pgfusepath{fill}%
\end{pgfscope}%
\begin{pgfscope}%
\pgfpathrectangle{\pgfqpoint{1.150000in}{0.150000in}}{\pgfqpoint{5.700000in}{5.700000in}}%
\pgfusepath{clip}%
\pgfsetbuttcap%
\pgfsetroundjoin%
\definecolor{currentfill}{rgb}{0.283072,0.130895,0.449241}%
\pgfsetfillcolor{currentfill}%
\pgfsetfillopacity{0.700000}%
\pgfsetlinewidth{0.000000pt}%
\definecolor{currentstroke}{rgb}{0.000000,0.000000,0.000000}%
\pgfsetstrokecolor{currentstroke}%
\pgfsetdash{}{0pt}%
\pgfpathmoveto{\pgfqpoint{2.658370in}{2.499330in}}%
\pgfpathlineto{\pgfqpoint{2.671667in}{2.491637in}}%
\pgfpathlineto{\pgfqpoint{2.684966in}{2.483995in}}%
\pgfpathlineto{\pgfqpoint{2.698269in}{2.476402in}}%
\pgfpathlineto{\pgfqpoint{2.711574in}{2.468859in}}%
\pgfpathlineto{\pgfqpoint{2.703112in}{2.467099in}}%
\pgfpathlineto{\pgfqpoint{2.694638in}{2.465540in}}%
\pgfpathlineto{\pgfqpoint{2.686149in}{2.464185in}}%
\pgfpathlineto{\pgfqpoint{2.677647in}{2.463041in}}%
\pgfpathlineto{\pgfqpoint{2.664315in}{2.470801in}}%
\pgfpathlineto{\pgfqpoint{2.650987in}{2.478611in}}%
\pgfpathlineto{\pgfqpoint{2.637661in}{2.486471in}}%
\pgfpathlineto{\pgfqpoint{2.624337in}{2.494380in}}%
\pgfpathlineto{\pgfqpoint{2.632866in}{2.495302in}}%
\pgfpathlineto{\pgfqpoint{2.641381in}{2.496438in}}%
\pgfpathlineto{\pgfqpoint{2.649882in}{2.497782in}}%
\pgfpathlineto{\pgfqpoint{2.658370in}{2.499330in}}%
\pgfpathclose%
\pgfusepath{fill}%
\end{pgfscope}%
\begin{pgfscope}%
\pgfpathrectangle{\pgfqpoint{1.150000in}{0.150000in}}{\pgfqpoint{5.700000in}{5.700000in}}%
\pgfusepath{clip}%
\pgfsetbuttcap%
\pgfsetroundjoin%
\definecolor{currentfill}{rgb}{0.280894,0.078907,0.402329}%
\pgfsetfillcolor{currentfill}%
\pgfsetfillopacity{0.700000}%
\pgfsetlinewidth{0.000000pt}%
\definecolor{currentstroke}{rgb}{0.000000,0.000000,0.000000}%
\pgfsetstrokecolor{currentstroke}%
\pgfsetdash{}{0pt}%
\pgfpathmoveto{\pgfqpoint{4.863688in}{2.406563in}}%
\pgfpathlineto{\pgfqpoint{4.877405in}{2.404630in}}%
\pgfpathlineto{\pgfqpoint{4.891129in}{2.402723in}}%
\pgfpathlineto{\pgfqpoint{4.904860in}{2.400842in}}%
\pgfpathlineto{\pgfqpoint{4.918599in}{2.398986in}}%
\pgfpathlineto{\pgfqpoint{4.911104in}{2.391702in}}%
\pgfpathlineto{\pgfqpoint{4.903604in}{2.384368in}}%
\pgfpathlineto{\pgfqpoint{4.896096in}{2.376981in}}%
\pgfpathlineto{\pgfqpoint{4.888583in}{2.369542in}}%
\pgfpathlineto{\pgfqpoint{4.874831in}{2.371385in}}%
\pgfpathlineto{\pgfqpoint{4.861086in}{2.373255in}}%
\pgfpathlineto{\pgfqpoint{4.847350in}{2.375150in}}%
\pgfpathlineto{\pgfqpoint{4.833620in}{2.377071in}}%
\pgfpathlineto{\pgfqpoint{4.841147in}{2.384517in}}%
\pgfpathlineto{\pgfqpoint{4.848667in}{2.391914in}}%
\pgfpathlineto{\pgfqpoint{4.856181in}{2.399263in}}%
\pgfpathlineto{\pgfqpoint{4.863688in}{2.406563in}}%
\pgfpathclose%
\pgfusepath{fill}%
\end{pgfscope}%
\begin{pgfscope}%
\pgfpathrectangle{\pgfqpoint{1.150000in}{0.150000in}}{\pgfqpoint{5.700000in}{5.700000in}}%
\pgfusepath{clip}%
\pgfsetbuttcap%
\pgfsetroundjoin%
\definecolor{currentfill}{rgb}{0.267004,0.004874,0.329415}%
\pgfsetfillcolor{currentfill}%
\pgfsetfillopacity{0.700000}%
\pgfsetlinewidth{0.000000pt}%
\definecolor{currentstroke}{rgb}{0.000000,0.000000,0.000000}%
\pgfsetstrokecolor{currentstroke}%
\pgfsetdash{}{0pt}%
\pgfpathmoveto{\pgfqpoint{3.741136in}{2.270768in}}%
\pgfpathlineto{\pgfqpoint{3.754576in}{2.266687in}}%
\pgfpathlineto{\pgfqpoint{3.768022in}{2.262636in}}%
\pgfpathlineto{\pgfqpoint{3.781474in}{2.258616in}}%
\pgfpathlineto{\pgfqpoint{3.794932in}{2.254627in}}%
\pgfpathlineto{\pgfqpoint{3.787016in}{2.247235in}}%
\pgfpathlineto{\pgfqpoint{3.779094in}{2.239876in}}%
\pgfpathlineto{\pgfqpoint{3.771165in}{2.232553in}}%
\pgfpathlineto{\pgfqpoint{3.763231in}{2.225269in}}%
\pgfpathlineto{\pgfqpoint{3.749759in}{2.229380in}}%
\pgfpathlineto{\pgfqpoint{3.736294in}{2.233521in}}%
\pgfpathlineto{\pgfqpoint{3.722834in}{2.237692in}}%
\pgfpathlineto{\pgfqpoint{3.709379in}{2.241895in}}%
\pgfpathlineto{\pgfqpoint{3.717328in}{2.249053in}}%
\pgfpathlineto{\pgfqpoint{3.725270in}{2.256253in}}%
\pgfpathlineto{\pgfqpoint{3.733206in}{2.263492in}}%
\pgfpathlineto{\pgfqpoint{3.741136in}{2.270768in}}%
\pgfpathclose%
\pgfusepath{fill}%
\end{pgfscope}%
\begin{pgfscope}%
\pgfpathrectangle{\pgfqpoint{1.150000in}{0.150000in}}{\pgfqpoint{5.700000in}{5.700000in}}%
\pgfusepath{clip}%
\pgfsetbuttcap%
\pgfsetroundjoin%
\definecolor{currentfill}{rgb}{0.282656,0.100196,0.422160}%
\pgfsetfillcolor{currentfill}%
\pgfsetfillopacity{0.700000}%
\pgfsetlinewidth{0.000000pt}%
\definecolor{currentstroke}{rgb}{0.000000,0.000000,0.000000}%
\pgfsetstrokecolor{currentstroke}%
\pgfsetdash{}{0pt}%
\pgfpathmoveto{\pgfqpoint{5.088295in}{2.441104in}}%
\pgfpathlineto{\pgfqpoint{5.102075in}{2.439412in}}%
\pgfpathlineto{\pgfqpoint{5.115863in}{2.437744in}}%
\pgfpathlineto{\pgfqpoint{5.129660in}{2.436102in}}%
\pgfpathlineto{\pgfqpoint{5.143463in}{2.434485in}}%
\pgfpathlineto{\pgfqpoint{5.136062in}{2.427698in}}%
\pgfpathlineto{\pgfqpoint{5.128654in}{2.420861in}}%
\pgfpathlineto{\pgfqpoint{5.121240in}{2.413972in}}%
\pgfpathlineto{\pgfqpoint{5.113819in}{2.407029in}}%
\pgfpathlineto{\pgfqpoint{5.100000in}{2.408607in}}%
\pgfpathlineto{\pgfqpoint{5.086190in}{2.410210in}}%
\pgfpathlineto{\pgfqpoint{5.072388in}{2.411839in}}%
\pgfpathlineto{\pgfqpoint{5.058593in}{2.413492in}}%
\pgfpathlineto{\pgfqpoint{5.066028in}{2.420469in}}%
\pgfpathlineto{\pgfqpoint{5.073457in}{2.427396in}}%
\pgfpathlineto{\pgfqpoint{5.080879in}{2.434274in}}%
\pgfpathlineto{\pgfqpoint{5.088295in}{2.441104in}}%
\pgfpathclose%
\pgfusepath{fill}%
\end{pgfscope}%
\begin{pgfscope}%
\pgfpathrectangle{\pgfqpoint{1.150000in}{0.150000in}}{\pgfqpoint{5.700000in}{5.700000in}}%
\pgfusepath{clip}%
\pgfsetbuttcap%
\pgfsetroundjoin%
\definecolor{currentfill}{rgb}{0.277018,0.050344,0.375715}%
\pgfsetfillcolor{currentfill}%
\pgfsetfillopacity{0.700000}%
\pgfsetlinewidth{0.000000pt}%
\definecolor{currentstroke}{rgb}{0.000000,0.000000,0.000000}%
\pgfsetstrokecolor{currentstroke}%
\pgfsetdash{}{0pt}%
\pgfpathmoveto{\pgfqpoint{3.044649in}{2.356078in}}%
\pgfpathlineto{\pgfqpoint{3.057979in}{2.349857in}}%
\pgfpathlineto{\pgfqpoint{3.071312in}{2.343675in}}%
\pgfpathlineto{\pgfqpoint{3.084650in}{2.337533in}}%
\pgfpathlineto{\pgfqpoint{3.097992in}{2.331431in}}%
\pgfpathlineto{\pgfqpoint{3.089758in}{2.327158in}}%
\pgfpathlineto{\pgfqpoint{3.081515in}{2.323024in}}%
\pgfpathlineto{\pgfqpoint{3.073261in}{2.319034in}}%
\pgfpathlineto{\pgfqpoint{3.064998in}{2.315192in}}%
\pgfpathlineto{\pgfqpoint{3.051635in}{2.321483in}}%
\pgfpathlineto{\pgfqpoint{3.038276in}{2.327813in}}%
\pgfpathlineto{\pgfqpoint{3.024921in}{2.334183in}}%
\pgfpathlineto{\pgfqpoint{3.011570in}{2.340593in}}%
\pgfpathlineto{\pgfqpoint{3.019856in}{2.344241in}}%
\pgfpathlineto{\pgfqpoint{3.028130in}{2.348041in}}%
\pgfpathlineto{\pgfqpoint{3.036394in}{2.351989in}}%
\pgfpathlineto{\pgfqpoint{3.044649in}{2.356078in}}%
\pgfpathclose%
\pgfusepath{fill}%
\end{pgfscope}%
\begin{pgfscope}%
\pgfpathrectangle{\pgfqpoint{1.150000in}{0.150000in}}{\pgfqpoint{5.700000in}{5.700000in}}%
\pgfusepath{clip}%
\pgfsetbuttcap%
\pgfsetroundjoin%
\definecolor{currentfill}{rgb}{0.273809,0.031497,0.358853}%
\pgfsetfillcolor{currentfill}%
\pgfsetfillopacity{0.700000}%
\pgfsetlinewidth{0.000000pt}%
\definecolor{currentstroke}{rgb}{0.000000,0.000000,0.000000}%
\pgfsetstrokecolor{currentstroke}%
\pgfsetdash{}{0pt}%
\pgfpathmoveto{\pgfqpoint{4.329529in}{2.317690in}}%
\pgfpathlineto{\pgfqpoint{4.343107in}{2.314943in}}%
\pgfpathlineto{\pgfqpoint{4.356691in}{2.312223in}}%
\pgfpathlineto{\pgfqpoint{4.370282in}{2.309530in}}%
\pgfpathlineto{\pgfqpoint{4.383880in}{2.306865in}}%
\pgfpathlineto{\pgfqpoint{4.376179in}{2.298839in}}%
\pgfpathlineto{\pgfqpoint{4.368473in}{2.290784in}}%
\pgfpathlineto{\pgfqpoint{4.360760in}{2.282702in}}%
\pgfpathlineto{\pgfqpoint{4.353042in}{2.274593in}}%
\pgfpathlineto{\pgfqpoint{4.339433in}{2.277313in}}%
\pgfpathlineto{\pgfqpoint{4.325830in}{2.280060in}}%
\pgfpathlineto{\pgfqpoint{4.312233in}{2.282835in}}%
\pgfpathlineto{\pgfqpoint{4.298644in}{2.285637in}}%
\pgfpathlineto{\pgfqpoint{4.306374in}{2.293687in}}%
\pgfpathlineto{\pgfqpoint{4.314098in}{2.301712in}}%
\pgfpathlineto{\pgfqpoint{4.321816in}{2.309714in}}%
\pgfpathlineto{\pgfqpoint{4.329529in}{2.317690in}}%
\pgfpathclose%
\pgfusepath{fill}%
\end{pgfscope}%
\begin{pgfscope}%
\pgfpathrectangle{\pgfqpoint{1.150000in}{0.150000in}}{\pgfqpoint{5.700000in}{5.700000in}}%
\pgfusepath{clip}%
\pgfsetbuttcap%
\pgfsetroundjoin%
\definecolor{currentfill}{rgb}{0.269944,0.014625,0.341379}%
\pgfsetfillcolor{currentfill}%
\pgfsetfillopacity{0.700000}%
\pgfsetlinewidth{0.000000pt}%
\definecolor{currentstroke}{rgb}{0.000000,0.000000,0.000000}%
\pgfsetstrokecolor{currentstroke}%
\pgfsetdash{}{0pt}%
\pgfpathmoveto{\pgfqpoint{4.104968in}{2.289835in}}%
\pgfpathlineto{\pgfqpoint{4.118490in}{2.286630in}}%
\pgfpathlineto{\pgfqpoint{4.132019in}{2.283453in}}%
\pgfpathlineto{\pgfqpoint{4.145555in}{2.280305in}}%
\pgfpathlineto{\pgfqpoint{4.159097in}{2.277185in}}%
\pgfpathlineto{\pgfqpoint{4.151315in}{2.269190in}}%
\pgfpathlineto{\pgfqpoint{4.143527in}{2.261186in}}%
\pgfpathlineto{\pgfqpoint{4.135734in}{2.253174in}}%
\pgfpathlineto{\pgfqpoint{4.127935in}{2.245158in}}%
\pgfpathlineto{\pgfqpoint{4.114380in}{2.248359in}}%
\pgfpathlineto{\pgfqpoint{4.100833in}{2.251589in}}%
\pgfpathlineto{\pgfqpoint{4.087292in}{2.254847in}}%
\pgfpathlineto{\pgfqpoint{4.073757in}{2.258133in}}%
\pgfpathlineto{\pgfqpoint{4.081568in}{2.266064in}}%
\pgfpathlineto{\pgfqpoint{4.089374in}{2.273992in}}%
\pgfpathlineto{\pgfqpoint{4.097174in}{2.281916in}}%
\pgfpathlineto{\pgfqpoint{4.104968in}{2.289835in}}%
\pgfpathclose%
\pgfusepath{fill}%
\end{pgfscope}%
\begin{pgfscope}%
\pgfpathrectangle{\pgfqpoint{1.150000in}{0.150000in}}{\pgfqpoint{5.700000in}{5.700000in}}%
\pgfusepath{clip}%
\pgfsetbuttcap%
\pgfsetroundjoin%
\definecolor{currentfill}{rgb}{0.283229,0.120777,0.440584}%
\pgfsetfillcolor{currentfill}%
\pgfsetfillopacity{0.700000}%
\pgfsetlinewidth{0.000000pt}%
\definecolor{currentstroke}{rgb}{0.000000,0.000000,0.000000}%
\pgfsetstrokecolor{currentstroke}%
\pgfsetdash{}{0pt}%
\pgfpathmoveto{\pgfqpoint{5.312914in}{2.474200in}}%
\pgfpathlineto{\pgfqpoint{5.326759in}{2.472689in}}%
\pgfpathlineto{\pgfqpoint{5.340612in}{2.471202in}}%
\pgfpathlineto{\pgfqpoint{5.354473in}{2.469740in}}%
\pgfpathlineto{\pgfqpoint{5.368342in}{2.468303in}}%
\pgfpathlineto{\pgfqpoint{5.361040in}{2.462063in}}%
\pgfpathlineto{\pgfqpoint{5.353730in}{2.455779in}}%
\pgfpathlineto{\pgfqpoint{5.346415in}{2.449449in}}%
\pgfpathlineto{\pgfqpoint{5.339092in}{2.443070in}}%
\pgfpathlineto{\pgfqpoint{5.325207in}{2.444441in}}%
\pgfpathlineto{\pgfqpoint{5.311330in}{2.445837in}}%
\pgfpathlineto{\pgfqpoint{5.297462in}{2.447258in}}%
\pgfpathlineto{\pgfqpoint{5.283601in}{2.448703in}}%
\pgfpathlineto{\pgfqpoint{5.290939in}{2.455143in}}%
\pgfpathlineto{\pgfqpoint{5.298271in}{2.461538in}}%
\pgfpathlineto{\pgfqpoint{5.305596in}{2.467890in}}%
\pgfpathlineto{\pgfqpoint{5.312914in}{2.474200in}}%
\pgfpathclose%
\pgfusepath{fill}%
\end{pgfscope}%
\begin{pgfscope}%
\pgfpathrectangle{\pgfqpoint{1.150000in}{0.150000in}}{\pgfqpoint{5.700000in}{5.700000in}}%
\pgfusepath{clip}%
\pgfsetbuttcap%
\pgfsetroundjoin%
\definecolor{currentfill}{rgb}{0.280255,0.165693,0.476498}%
\pgfsetfillcolor{currentfill}%
\pgfsetfillopacity{0.700000}%
\pgfsetlinewidth{0.000000pt}%
\definecolor{currentstroke}{rgb}{0.000000,0.000000,0.000000}%
\pgfsetstrokecolor{currentstroke}%
\pgfsetdash{}{0pt}%
\pgfpathmoveto{\pgfqpoint{5.986418in}{2.557847in}}%
\pgfpathlineto{\pgfqpoint{6.000448in}{2.556530in}}%
\pgfpathlineto{\pgfqpoint{6.014487in}{2.555237in}}%
\pgfpathlineto{\pgfqpoint{6.028534in}{2.553968in}}%
\pgfpathlineto{\pgfqpoint{6.042590in}{2.552723in}}%
\pgfpathlineto{\pgfqpoint{6.035607in}{2.547946in}}%
\pgfpathlineto{\pgfqpoint{6.028619in}{2.543180in}}%
\pgfpathlineto{\pgfqpoint{6.021625in}{2.538421in}}%
\pgfpathlineto{\pgfqpoint{6.014625in}{2.533665in}}%
\pgfpathlineto{\pgfqpoint{6.000548in}{2.534763in}}%
\pgfpathlineto{\pgfqpoint{5.986479in}{2.535885in}}%
\pgfpathlineto{\pgfqpoint{5.972418in}{2.537030in}}%
\pgfpathlineto{\pgfqpoint{5.958367in}{2.538200in}}%
\pgfpathlineto{\pgfqpoint{5.965388in}{2.543099in}}%
\pgfpathlineto{\pgfqpoint{5.972403in}{2.548003in}}%
\pgfpathlineto{\pgfqpoint{5.979413in}{2.552918in}}%
\pgfpathlineto{\pgfqpoint{5.986418in}{2.557847in}}%
\pgfpathclose%
\pgfusepath{fill}%
\end{pgfscope}%
\begin{pgfscope}%
\pgfpathrectangle{\pgfqpoint{1.150000in}{0.150000in}}{\pgfqpoint{5.700000in}{5.700000in}}%
\pgfusepath{clip}%
\pgfsetbuttcap%
\pgfsetroundjoin%
\definecolor{currentfill}{rgb}{0.282884,0.135920,0.453427}%
\pgfsetfillcolor{currentfill}%
\pgfsetfillopacity{0.700000}%
\pgfsetlinewidth{0.000000pt}%
\definecolor{currentstroke}{rgb}{0.000000,0.000000,0.000000}%
\pgfsetstrokecolor{currentstroke}%
\pgfsetdash{}{0pt}%
\pgfpathmoveto{\pgfqpoint{5.537504in}{2.504959in}}%
\pgfpathlineto{\pgfqpoint{5.551413in}{2.503570in}}%
\pgfpathlineto{\pgfqpoint{5.565330in}{2.502205in}}%
\pgfpathlineto{\pgfqpoint{5.579255in}{2.500865in}}%
\pgfpathlineto{\pgfqpoint{5.593188in}{2.499549in}}%
\pgfpathlineto{\pgfqpoint{5.585989in}{2.493858in}}%
\pgfpathlineto{\pgfqpoint{5.578784in}{2.488136in}}%
\pgfpathlineto{\pgfqpoint{5.571572in}{2.482379in}}%
\pgfpathlineto{\pgfqpoint{5.564353in}{2.476585in}}%
\pgfpathlineto{\pgfqpoint{5.550402in}{2.477807in}}%
\pgfpathlineto{\pgfqpoint{5.536460in}{2.479054in}}%
\pgfpathlineto{\pgfqpoint{5.522526in}{2.480326in}}%
\pgfpathlineto{\pgfqpoint{5.508600in}{2.481622in}}%
\pgfpathlineto{\pgfqpoint{5.515836in}{2.487505in}}%
\pgfpathlineto{\pgfqpoint{5.523065in}{2.493353in}}%
\pgfpathlineto{\pgfqpoint{5.530288in}{2.499170in}}%
\pgfpathlineto{\pgfqpoint{5.537504in}{2.504959in}}%
\pgfpathclose%
\pgfusepath{fill}%
\end{pgfscope}%
\begin{pgfscope}%
\pgfpathrectangle{\pgfqpoint{1.150000in}{0.150000in}}{\pgfqpoint{5.700000in}{5.700000in}}%
\pgfusepath{clip}%
\pgfsetbuttcap%
\pgfsetroundjoin%
\definecolor{currentfill}{rgb}{0.281887,0.150881,0.465405}%
\pgfsetfillcolor{currentfill}%
\pgfsetfillopacity{0.700000}%
\pgfsetlinewidth{0.000000pt}%
\definecolor{currentstroke}{rgb}{0.000000,0.000000,0.000000}%
\pgfsetstrokecolor{currentstroke}%
\pgfsetdash{}{0pt}%
\pgfpathmoveto{\pgfqpoint{5.762019in}{2.532877in}}%
\pgfpathlineto{\pgfqpoint{5.775989in}{2.531553in}}%
\pgfpathlineto{\pgfqpoint{5.789968in}{2.530252in}}%
\pgfpathlineto{\pgfqpoint{5.803956in}{2.528976in}}%
\pgfpathlineto{\pgfqpoint{5.817952in}{2.527724in}}%
\pgfpathlineto{\pgfqpoint{5.810860in}{2.522537in}}%
\pgfpathlineto{\pgfqpoint{5.803763in}{2.517337in}}%
\pgfpathlineto{\pgfqpoint{5.796659in}{2.512120in}}%
\pgfpathlineto{\pgfqpoint{5.789548in}{2.506883in}}%
\pgfpathlineto{\pgfqpoint{5.775533in}{2.508014in}}%
\pgfpathlineto{\pgfqpoint{5.761526in}{2.509170in}}%
\pgfpathlineto{\pgfqpoint{5.747528in}{2.510350in}}%
\pgfpathlineto{\pgfqpoint{5.733538in}{2.511555in}}%
\pgfpathlineto{\pgfqpoint{5.740667in}{2.516908in}}%
\pgfpathlineto{\pgfqpoint{5.747791in}{2.522243in}}%
\pgfpathlineto{\pgfqpoint{5.754908in}{2.527565in}}%
\pgfpathlineto{\pgfqpoint{5.762019in}{2.532877in}}%
\pgfpathclose%
\pgfusepath{fill}%
\end{pgfscope}%
\begin{pgfscope}%
\pgfpathrectangle{\pgfqpoint{1.150000in}{0.150000in}}{\pgfqpoint{5.700000in}{5.700000in}}%
\pgfusepath{clip}%
\pgfsetbuttcap%
\pgfsetroundjoin%
\definecolor{currentfill}{rgb}{0.277018,0.050344,0.375715}%
\pgfsetfillcolor{currentfill}%
\pgfsetfillopacity{0.700000}%
\pgfsetlinewidth{0.000000pt}%
\definecolor{currentstroke}{rgb}{0.000000,0.000000,0.000000}%
\pgfsetstrokecolor{currentstroke}%
\pgfsetdash{}{0pt}%
\pgfpathmoveto{\pgfqpoint{4.554123in}{2.350167in}}%
\pgfpathlineto{\pgfqpoint{4.567760in}{2.347813in}}%
\pgfpathlineto{\pgfqpoint{4.581404in}{2.345486in}}%
\pgfpathlineto{\pgfqpoint{4.595055in}{2.343184in}}%
\pgfpathlineto{\pgfqpoint{4.608713in}{2.340909in}}%
\pgfpathlineto{\pgfqpoint{4.601095in}{2.333065in}}%
\pgfpathlineto{\pgfqpoint{4.593470in}{2.325178in}}%
\pgfpathlineto{\pgfqpoint{4.585840in}{2.317248in}}%
\pgfpathlineto{\pgfqpoint{4.578204in}{2.309276in}}%
\pgfpathlineto{\pgfqpoint{4.564533in}{2.311579in}}%
\pgfpathlineto{\pgfqpoint{4.550870in}{2.313908in}}%
\pgfpathlineto{\pgfqpoint{4.537214in}{2.316264in}}%
\pgfpathlineto{\pgfqpoint{4.523565in}{2.318647in}}%
\pgfpathlineto{\pgfqpoint{4.531213in}{2.326586in}}%
\pgfpathlineto{\pgfqpoint{4.538856in}{2.334485in}}%
\pgfpathlineto{\pgfqpoint{4.546492in}{2.342346in}}%
\pgfpathlineto{\pgfqpoint{4.554123in}{2.350167in}}%
\pgfpathclose%
\pgfusepath{fill}%
\end{pgfscope}%
\begin{pgfscope}%
\pgfpathrectangle{\pgfqpoint{1.150000in}{0.150000in}}{\pgfqpoint{5.700000in}{5.700000in}}%
\pgfusepath{clip}%
\pgfsetbuttcap%
\pgfsetroundjoin%
\definecolor{currentfill}{rgb}{0.267004,0.004874,0.329415}%
\pgfsetfillcolor{currentfill}%
\pgfsetfillopacity{0.700000}%
\pgfsetlinewidth{0.000000pt}%
\definecolor{currentstroke}{rgb}{0.000000,0.000000,0.000000}%
\pgfsetstrokecolor{currentstroke}%
\pgfsetdash{}{0pt}%
\pgfpathmoveto{\pgfqpoint{3.880370in}{2.269256in}}%
\pgfpathlineto{\pgfqpoint{3.893843in}{2.265525in}}%
\pgfpathlineto{\pgfqpoint{3.907323in}{2.261825in}}%
\pgfpathlineto{\pgfqpoint{3.920809in}{2.258153in}}%
\pgfpathlineto{\pgfqpoint{3.934300in}{2.254512in}}%
\pgfpathlineto{\pgfqpoint{3.926435in}{2.246806in}}%
\pgfpathlineto{\pgfqpoint{3.918564in}{2.239116in}}%
\pgfpathlineto{\pgfqpoint{3.910686in}{2.231446in}}%
\pgfpathlineto{\pgfqpoint{3.902803in}{2.223797in}}%
\pgfpathlineto{\pgfqpoint{3.889299in}{2.227547in}}%
\pgfpathlineto{\pgfqpoint{3.875800in}{2.231326in}}%
\pgfpathlineto{\pgfqpoint{3.862307in}{2.235134in}}%
\pgfpathlineto{\pgfqpoint{3.848820in}{2.238973in}}%
\pgfpathlineto{\pgfqpoint{3.856717in}{2.246509in}}%
\pgfpathlineto{\pgfqpoint{3.864607in}{2.254069in}}%
\pgfpathlineto{\pgfqpoint{3.872491in}{2.261653in}}%
\pgfpathlineto{\pgfqpoint{3.880370in}{2.269256in}}%
\pgfpathclose%
\pgfusepath{fill}%
\end{pgfscope}%
\begin{pgfscope}%
\pgfpathrectangle{\pgfqpoint{1.150000in}{0.150000in}}{\pgfqpoint{5.700000in}{5.700000in}}%
\pgfusepath{clip}%
\pgfsetbuttcap%
\pgfsetroundjoin%
\definecolor{currentfill}{rgb}{0.280267,0.073417,0.397163}%
\pgfsetfillcolor{currentfill}%
\pgfsetfillopacity{0.700000}%
\pgfsetlinewidth{0.000000pt}%
\definecolor{currentstroke}{rgb}{0.000000,0.000000,0.000000}%
\pgfsetstrokecolor{currentstroke}%
\pgfsetdash{}{0pt}%
\pgfpathmoveto{\pgfqpoint{4.778778in}{2.385011in}}%
\pgfpathlineto{\pgfqpoint{4.792477in}{2.382988in}}%
\pgfpathlineto{\pgfqpoint{4.806184in}{2.380990in}}%
\pgfpathlineto{\pgfqpoint{4.819898in}{2.379017in}}%
\pgfpathlineto{\pgfqpoint{4.833620in}{2.377071in}}%
\pgfpathlineto{\pgfqpoint{4.826088in}{2.369574in}}%
\pgfpathlineto{\pgfqpoint{4.818549in}{2.362027in}}%
\pgfpathlineto{\pgfqpoint{4.811003in}{2.354427in}}%
\pgfpathlineto{\pgfqpoint{4.803452in}{2.346777in}}%
\pgfpathlineto{\pgfqpoint{4.789718in}{2.348724in}}%
\pgfpathlineto{\pgfqpoint{4.775991in}{2.350698in}}%
\pgfpathlineto{\pgfqpoint{4.762271in}{2.352697in}}%
\pgfpathlineto{\pgfqpoint{4.748559in}{2.354723in}}%
\pgfpathlineto{\pgfqpoint{4.756123in}{2.362367in}}%
\pgfpathlineto{\pgfqpoint{4.763681in}{2.369963in}}%
\pgfpathlineto{\pgfqpoint{4.771232in}{2.377511in}}%
\pgfpathlineto{\pgfqpoint{4.778778in}{2.385011in}}%
\pgfpathclose%
\pgfusepath{fill}%
\end{pgfscope}%
\begin{pgfscope}%
\pgfpathrectangle{\pgfqpoint{1.150000in}{0.150000in}}{\pgfqpoint{5.700000in}{5.700000in}}%
\pgfusepath{clip}%
\pgfsetbuttcap%
\pgfsetroundjoin%
\definecolor{currentfill}{rgb}{0.283229,0.120777,0.440584}%
\pgfsetfillcolor{currentfill}%
\pgfsetfillopacity{0.700000}%
\pgfsetlinewidth{0.000000pt}%
\definecolor{currentstroke}{rgb}{0.000000,0.000000,0.000000}%
\pgfsetstrokecolor{currentstroke}%
\pgfsetdash{}{0pt}%
\pgfpathmoveto{\pgfqpoint{2.711574in}{2.468859in}}%
\pgfpathlineto{\pgfqpoint{2.724882in}{2.461364in}}%
\pgfpathlineto{\pgfqpoint{2.738193in}{2.453917in}}%
\pgfpathlineto{\pgfqpoint{2.751506in}{2.446518in}}%
\pgfpathlineto{\pgfqpoint{2.764823in}{2.439166in}}%
\pgfpathlineto{\pgfqpoint{2.756386in}{2.437195in}}%
\pgfpathlineto{\pgfqpoint{2.747937in}{2.435421in}}%
\pgfpathlineto{\pgfqpoint{2.739474in}{2.433848in}}%
\pgfpathlineto{\pgfqpoint{2.730998in}{2.432482in}}%
\pgfpathlineto{\pgfqpoint{2.717656in}{2.440050in}}%
\pgfpathlineto{\pgfqpoint{2.704317in}{2.447666in}}%
\pgfpathlineto{\pgfqpoint{2.690980in}{2.455329in}}%
\pgfpathlineto{\pgfqpoint{2.677647in}{2.463041in}}%
\pgfpathlineto{\pgfqpoint{2.686149in}{2.464185in}}%
\pgfpathlineto{\pgfqpoint{2.694638in}{2.465540in}}%
\pgfpathlineto{\pgfqpoint{2.703112in}{2.467099in}}%
\pgfpathlineto{\pgfqpoint{2.711574in}{2.468859in}}%
\pgfpathclose%
\pgfusepath{fill}%
\end{pgfscope}%
\begin{pgfscope}%
\pgfpathrectangle{\pgfqpoint{1.150000in}{0.150000in}}{\pgfqpoint{5.700000in}{5.700000in}}%
\pgfusepath{clip}%
\pgfsetbuttcap%
\pgfsetroundjoin%
\definecolor{currentfill}{rgb}{0.269944,0.014625,0.341379}%
\pgfsetfillcolor{currentfill}%
\pgfsetfillopacity{0.700000}%
\pgfsetlinewidth{0.000000pt}%
\definecolor{currentstroke}{rgb}{0.000000,0.000000,0.000000}%
\pgfsetstrokecolor{currentstroke}%
\pgfsetdash{}{0pt}%
\pgfpathmoveto{\pgfqpoint{3.376982in}{2.282965in}}%
\pgfpathlineto{\pgfqpoint{3.390363in}{2.277825in}}%
\pgfpathlineto{\pgfqpoint{3.403748in}{2.272721in}}%
\pgfpathlineto{\pgfqpoint{3.417139in}{2.267651in}}%
\pgfpathlineto{\pgfqpoint{3.430534in}{2.262614in}}%
\pgfpathlineto{\pgfqpoint{3.422461in}{2.256603in}}%
\pgfpathlineto{\pgfqpoint{3.414381in}{2.250680in}}%
\pgfpathlineto{\pgfqpoint{3.406293in}{2.244851in}}%
\pgfpathlineto{\pgfqpoint{3.398197in}{2.239117in}}%
\pgfpathlineto{\pgfqpoint{3.384785in}{2.244314in}}%
\pgfpathlineto{\pgfqpoint{3.371377in}{2.249546in}}%
\pgfpathlineto{\pgfqpoint{3.357975in}{2.254811in}}%
\pgfpathlineto{\pgfqpoint{3.344577in}{2.260112in}}%
\pgfpathlineto{\pgfqpoint{3.352690in}{2.265679in}}%
\pgfpathlineto{\pgfqpoint{3.360796in}{2.271346in}}%
\pgfpathlineto{\pgfqpoint{3.368893in}{2.277109in}}%
\pgfpathlineto{\pgfqpoint{3.376982in}{2.282965in}}%
\pgfpathclose%
\pgfusepath{fill}%
\end{pgfscope}%
\begin{pgfscope}%
\pgfpathrectangle{\pgfqpoint{1.150000in}{0.150000in}}{\pgfqpoint{5.700000in}{5.700000in}}%
\pgfusepath{clip}%
\pgfsetbuttcap%
\pgfsetroundjoin%
\definecolor{currentfill}{rgb}{0.280894,0.078907,0.402329}%
\pgfsetfillcolor{currentfill}%
\pgfsetfillopacity{0.700000}%
\pgfsetlinewidth{0.000000pt}%
\definecolor{currentstroke}{rgb}{0.000000,0.000000,0.000000}%
\pgfsetstrokecolor{currentstroke}%
\pgfsetdash{}{0pt}%
\pgfpathmoveto{\pgfqpoint{2.904896in}{2.393360in}}%
\pgfpathlineto{\pgfqpoint{2.918218in}{2.386616in}}%
\pgfpathlineto{\pgfqpoint{2.931543in}{2.379915in}}%
\pgfpathlineto{\pgfqpoint{2.944872in}{2.373257in}}%
\pgfpathlineto{\pgfqpoint{2.958204in}{2.366641in}}%
\pgfpathlineto{\pgfqpoint{2.949886in}{2.363346in}}%
\pgfpathlineto{\pgfqpoint{2.941557in}{2.360217in}}%
\pgfpathlineto{\pgfqpoint{2.933217in}{2.357257in}}%
\pgfpathlineto{\pgfqpoint{2.924865in}{2.354473in}}%
\pgfpathlineto{\pgfqpoint{2.911510in}{2.361291in}}%
\pgfpathlineto{\pgfqpoint{2.898159in}{2.368151in}}%
\pgfpathlineto{\pgfqpoint{2.884810in}{2.375054in}}%
\pgfpathlineto{\pgfqpoint{2.871465in}{2.382000in}}%
\pgfpathlineto{\pgfqpoint{2.879841in}{2.384577in}}%
\pgfpathlineto{\pgfqpoint{2.888204in}{2.387333in}}%
\pgfpathlineto{\pgfqpoint{2.896556in}{2.390262in}}%
\pgfpathlineto{\pgfqpoint{2.904896in}{2.393360in}}%
\pgfpathclose%
\pgfusepath{fill}%
\end{pgfscope}%
\begin{pgfscope}%
\pgfpathrectangle{\pgfqpoint{1.150000in}{0.150000in}}{\pgfqpoint{5.700000in}{5.700000in}}%
\pgfusepath{clip}%
\pgfsetbuttcap%
\pgfsetroundjoin%
\definecolor{currentfill}{rgb}{0.279574,0.170599,0.479997}%
\pgfsetfillcolor{currentfill}%
\pgfsetfillopacity{0.700000}%
\pgfsetlinewidth{0.000000pt}%
\definecolor{currentstroke}{rgb}{0.000000,0.000000,0.000000}%
\pgfsetstrokecolor{currentstroke}%
\pgfsetdash{}{0pt}%
\pgfpathmoveto{\pgfqpoint{6.126692in}{2.566656in}}%
\pgfpathlineto{\pgfqpoint{6.140768in}{2.565369in}}%
\pgfpathlineto{\pgfqpoint{6.154853in}{2.564106in}}%
\pgfpathlineto{\pgfqpoint{6.168946in}{2.562867in}}%
\pgfpathlineto{\pgfqpoint{6.162023in}{2.558286in}}%
\pgfpathlineto{\pgfqpoint{6.155094in}{2.553730in}}%
\pgfpathlineto{\pgfqpoint{6.148161in}{2.549194in}}%
\pgfpathlineto{\pgfqpoint{6.141222in}{2.544674in}}%
\pgfpathlineto{\pgfqpoint{6.127106in}{2.545752in}}%
\pgfpathlineto{\pgfqpoint{6.112998in}{2.546854in}}%
\pgfpathlineto{\pgfqpoint{6.098900in}{2.547980in}}%
\pgfpathlineto{\pgfqpoint{6.105855in}{2.552618in}}%
\pgfpathlineto{\pgfqpoint{6.112806in}{2.557273in}}%
\pgfpathlineto{\pgfqpoint{6.119751in}{2.561951in}}%
\pgfpathlineto{\pgfqpoint{6.126692in}{2.566656in}}%
\pgfpathclose%
\pgfusepath{fill}%
\end{pgfscope}%
\begin{pgfscope}%
\pgfpathrectangle{\pgfqpoint{1.150000in}{0.150000in}}{\pgfqpoint{5.700000in}{5.700000in}}%
\pgfusepath{clip}%
\pgfsetbuttcap%
\pgfsetroundjoin%
\definecolor{currentfill}{rgb}{0.282327,0.094955,0.417331}%
\pgfsetfillcolor{currentfill}%
\pgfsetfillopacity{0.700000}%
\pgfsetlinewidth{0.000000pt}%
\definecolor{currentstroke}{rgb}{0.000000,0.000000,0.000000}%
\pgfsetstrokecolor{currentstroke}%
\pgfsetdash{}{0pt}%
\pgfpathmoveto{\pgfqpoint{5.003492in}{2.420358in}}%
\pgfpathlineto{\pgfqpoint{5.017255in}{2.418604in}}%
\pgfpathlineto{\pgfqpoint{5.031027in}{2.416875in}}%
\pgfpathlineto{\pgfqpoint{5.044806in}{2.415171in}}%
\pgfpathlineto{\pgfqpoint{5.058593in}{2.413492in}}%
\pgfpathlineto{\pgfqpoint{5.051151in}{2.406463in}}%
\pgfpathlineto{\pgfqpoint{5.043703in}{2.399381in}}%
\pgfpathlineto{\pgfqpoint{5.036248in}{2.392245in}}%
\pgfpathlineto{\pgfqpoint{5.028787in}{2.385052in}}%
\pgfpathlineto{\pgfqpoint{5.014986in}{2.386705in}}%
\pgfpathlineto{\pgfqpoint{5.001193in}{2.388384in}}%
\pgfpathlineto{\pgfqpoint{4.987408in}{2.390087in}}%
\pgfpathlineto{\pgfqpoint{4.973631in}{2.391816in}}%
\pgfpathlineto{\pgfqpoint{4.981106in}{2.399029in}}%
\pgfpathlineto{\pgfqpoint{4.988574in}{2.406190in}}%
\pgfpathlineto{\pgfqpoint{4.996036in}{2.413299in}}%
\pgfpathlineto{\pgfqpoint{5.003492in}{2.420358in}}%
\pgfpathclose%
\pgfusepath{fill}%
\end{pgfscope}%
\begin{pgfscope}%
\pgfpathrectangle{\pgfqpoint{1.150000in}{0.150000in}}{\pgfqpoint{5.700000in}{5.700000in}}%
\pgfusepath{clip}%
\pgfsetbuttcap%
\pgfsetroundjoin%
\definecolor{currentfill}{rgb}{0.267004,0.004874,0.329415}%
\pgfsetfillcolor{currentfill}%
\pgfsetfillopacity{0.700000}%
\pgfsetlinewidth{0.000000pt}%
\definecolor{currentstroke}{rgb}{0.000000,0.000000,0.000000}%
\pgfsetstrokecolor{currentstroke}%
\pgfsetdash{}{0pt}%
\pgfpathmoveto{\pgfqpoint{3.516316in}{2.268259in}}%
\pgfpathlineto{\pgfqpoint{3.529721in}{2.263539in}}%
\pgfpathlineto{\pgfqpoint{3.543131in}{2.258851in}}%
\pgfpathlineto{\pgfqpoint{3.556545in}{2.254196in}}%
\pgfpathlineto{\pgfqpoint{3.569966in}{2.249574in}}%
\pgfpathlineto{\pgfqpoint{3.561954in}{2.242958in}}%
\pgfpathlineto{\pgfqpoint{3.553936in}{2.236410in}}%
\pgfpathlineto{\pgfqpoint{3.545910in}{2.229933in}}%
\pgfpathlineto{\pgfqpoint{3.537878in}{2.223532in}}%
\pgfpathlineto{\pgfqpoint{3.524442in}{2.228302in}}%
\pgfpathlineto{\pgfqpoint{3.511011in}{2.233104in}}%
\pgfpathlineto{\pgfqpoint{3.497585in}{2.237940in}}%
\pgfpathlineto{\pgfqpoint{3.484165in}{2.242808in}}%
\pgfpathlineto{\pgfqpoint{3.492214in}{2.249057in}}%
\pgfpathlineto{\pgfqpoint{3.500255in}{2.255384in}}%
\pgfpathlineto{\pgfqpoint{3.508289in}{2.261786in}}%
\pgfpathlineto{\pgfqpoint{3.516316in}{2.268259in}}%
\pgfpathclose%
\pgfusepath{fill}%
\end{pgfscope}%
\begin{pgfscope}%
\pgfpathrectangle{\pgfqpoint{1.150000in}{0.150000in}}{\pgfqpoint{5.700000in}{5.700000in}}%
\pgfusepath{clip}%
\pgfsetbuttcap%
\pgfsetroundjoin%
\definecolor{currentfill}{rgb}{0.272594,0.025563,0.353093}%
\pgfsetfillcolor{currentfill}%
\pgfsetfillopacity{0.700000}%
\pgfsetlinewidth{0.000000pt}%
\definecolor{currentstroke}{rgb}{0.000000,0.000000,0.000000}%
\pgfsetstrokecolor{currentstroke}%
\pgfsetdash{}{0pt}%
\pgfpathmoveto{\pgfqpoint{3.237562in}{2.303789in}}%
\pgfpathlineto{\pgfqpoint{3.250922in}{2.298203in}}%
\pgfpathlineto{\pgfqpoint{3.264288in}{2.292653in}}%
\pgfpathlineto{\pgfqpoint{3.277658in}{2.287141in}}%
\pgfpathlineto{\pgfqpoint{3.291032in}{2.281664in}}%
\pgfpathlineto{\pgfqpoint{3.282893in}{2.276370in}}%
\pgfpathlineto{\pgfqpoint{3.274745in}{2.271188in}}%
\pgfpathlineto{\pgfqpoint{3.266588in}{2.266121in}}%
\pgfpathlineto{\pgfqpoint{3.258422in}{2.261174in}}%
\pgfpathlineto{\pgfqpoint{3.245029in}{2.266826in}}%
\pgfpathlineto{\pgfqpoint{3.231640in}{2.272513in}}%
\pgfpathlineto{\pgfqpoint{3.218256in}{2.278237in}}%
\pgfpathlineto{\pgfqpoint{3.204876in}{2.283998in}}%
\pgfpathlineto{\pgfqpoint{3.213061in}{2.288765in}}%
\pgfpathlineto{\pgfqpoint{3.221237in}{2.293655in}}%
\pgfpathlineto{\pgfqpoint{3.229403in}{2.298664in}}%
\pgfpathlineto{\pgfqpoint{3.237562in}{2.303789in}}%
\pgfpathclose%
\pgfusepath{fill}%
\end{pgfscope}%
\begin{pgfscope}%
\pgfpathrectangle{\pgfqpoint{1.150000in}{0.150000in}}{\pgfqpoint{5.700000in}{5.700000in}}%
\pgfusepath{clip}%
\pgfsetbuttcap%
\pgfsetroundjoin%
\definecolor{currentfill}{rgb}{0.271305,0.019942,0.347269}%
\pgfsetfillcolor{currentfill}%
\pgfsetfillopacity{0.700000}%
\pgfsetlinewidth{0.000000pt}%
\definecolor{currentstroke}{rgb}{0.000000,0.000000,0.000000}%
\pgfsetstrokecolor{currentstroke}%
\pgfsetdash{}{0pt}%
\pgfpathmoveto{\pgfqpoint{4.244353in}{2.297120in}}%
\pgfpathlineto{\pgfqpoint{4.257916in}{2.294208in}}%
\pgfpathlineto{\pgfqpoint{4.271485in}{2.291324in}}%
\pgfpathlineto{\pgfqpoint{4.285061in}{2.288467in}}%
\pgfpathlineto{\pgfqpoint{4.298644in}{2.285637in}}%
\pgfpathlineto{\pgfqpoint{4.290908in}{2.277566in}}%
\pgfpathlineto{\pgfqpoint{4.283167in}{2.269473in}}%
\pgfpathlineto{\pgfqpoint{4.275420in}{2.261360in}}%
\pgfpathlineto{\pgfqpoint{4.267668in}{2.253229in}}%
\pgfpathlineto{\pgfqpoint{4.254073in}{2.256127in}}%
\pgfpathlineto{\pgfqpoint{4.240485in}{2.259052in}}%
\pgfpathlineto{\pgfqpoint{4.226904in}{2.262005in}}%
\pgfpathlineto{\pgfqpoint{4.213330in}{2.264985in}}%
\pgfpathlineto{\pgfqpoint{4.221094in}{2.273043in}}%
\pgfpathlineto{\pgfqpoint{4.228853in}{2.281086in}}%
\pgfpathlineto{\pgfqpoint{4.236606in}{2.289112in}}%
\pgfpathlineto{\pgfqpoint{4.244353in}{2.297120in}}%
\pgfpathclose%
\pgfusepath{fill}%
\end{pgfscope}%
\begin{pgfscope}%
\pgfpathrectangle{\pgfqpoint{1.150000in}{0.150000in}}{\pgfqpoint{5.700000in}{5.700000in}}%
\pgfusepath{clip}%
\pgfsetbuttcap%
\pgfsetroundjoin%
\definecolor{currentfill}{rgb}{0.283197,0.115680,0.436115}%
\pgfsetfillcolor{currentfill}%
\pgfsetfillopacity{0.700000}%
\pgfsetlinewidth{0.000000pt}%
\definecolor{currentstroke}{rgb}{0.000000,0.000000,0.000000}%
\pgfsetstrokecolor{currentstroke}%
\pgfsetdash{}{0pt}%
\pgfpathmoveto{\pgfqpoint{5.228239in}{2.454734in}}%
\pgfpathlineto{\pgfqpoint{5.242067in}{2.453189in}}%
\pgfpathlineto{\pgfqpoint{5.255904in}{2.451669in}}%
\pgfpathlineto{\pgfqpoint{5.269748in}{2.450174in}}%
\pgfpathlineto{\pgfqpoint{5.283601in}{2.448703in}}%
\pgfpathlineto{\pgfqpoint{5.276256in}{2.442216in}}%
\pgfpathlineto{\pgfqpoint{5.268904in}{2.435679in}}%
\pgfpathlineto{\pgfqpoint{5.261546in}{2.429090in}}%
\pgfpathlineto{\pgfqpoint{5.254180in}{2.422447in}}%
\pgfpathlineto{\pgfqpoint{5.240313in}{2.423865in}}%
\pgfpathlineto{\pgfqpoint{5.226453in}{2.425307in}}%
\pgfpathlineto{\pgfqpoint{5.212602in}{2.426774in}}%
\pgfpathlineto{\pgfqpoint{5.198758in}{2.428267in}}%
\pgfpathlineto{\pgfqpoint{5.206138in}{2.434957in}}%
\pgfpathlineto{\pgfqpoint{5.213512in}{2.441597in}}%
\pgfpathlineto{\pgfqpoint{5.220879in}{2.448188in}}%
\pgfpathlineto{\pgfqpoint{5.228239in}{2.454734in}}%
\pgfpathclose%
\pgfusepath{fill}%
\end{pgfscope}%
\begin{pgfscope}%
\pgfpathrectangle{\pgfqpoint{1.150000in}{0.150000in}}{\pgfqpoint{5.700000in}{5.700000in}}%
\pgfusepath{clip}%
\pgfsetbuttcap%
\pgfsetroundjoin%
\definecolor{currentfill}{rgb}{0.267004,0.004874,0.329415}%
\pgfsetfillcolor{currentfill}%
\pgfsetfillopacity{0.700000}%
\pgfsetlinewidth{0.000000pt}%
\definecolor{currentstroke}{rgb}{0.000000,0.000000,0.000000}%
\pgfsetstrokecolor{currentstroke}%
\pgfsetdash{}{0pt}%
\pgfpathmoveto{\pgfqpoint{3.655617in}{2.259018in}}%
\pgfpathlineto{\pgfqpoint{3.669049in}{2.254690in}}%
\pgfpathlineto{\pgfqpoint{3.682487in}{2.250394in}}%
\pgfpathlineto{\pgfqpoint{3.695930in}{2.246129in}}%
\pgfpathlineto{\pgfqpoint{3.709379in}{2.241895in}}%
\pgfpathlineto{\pgfqpoint{3.701424in}{2.234782in}}%
\pgfpathlineto{\pgfqpoint{3.693463in}{2.227718in}}%
\pgfpathlineto{\pgfqpoint{3.685495in}{2.220705in}}%
\pgfpathlineto{\pgfqpoint{3.677520in}{2.213746in}}%
\pgfpathlineto{\pgfqpoint{3.664057in}{2.218114in}}%
\pgfpathlineto{\pgfqpoint{3.650599in}{2.222513in}}%
\pgfpathlineto{\pgfqpoint{3.637146in}{2.226944in}}%
\pgfpathlineto{\pgfqpoint{3.623699in}{2.231406in}}%
\pgfpathlineto{\pgfqpoint{3.631689in}{2.238226in}}%
\pgfpathlineto{\pgfqpoint{3.639672in}{2.245103in}}%
\pgfpathlineto{\pgfqpoint{3.647648in}{2.252035in}}%
\pgfpathlineto{\pgfqpoint{3.655617in}{2.259018in}}%
\pgfpathclose%
\pgfusepath{fill}%
\end{pgfscope}%
\begin{pgfscope}%
\pgfpathrectangle{\pgfqpoint{1.150000in}{0.150000in}}{\pgfqpoint{5.700000in}{5.700000in}}%
\pgfusepath{clip}%
\pgfsetbuttcap%
\pgfsetroundjoin%
\definecolor{currentfill}{rgb}{0.268510,0.009605,0.335427}%
\pgfsetfillcolor{currentfill}%
\pgfsetfillopacity{0.700000}%
\pgfsetlinewidth{0.000000pt}%
\definecolor{currentstroke}{rgb}{0.000000,0.000000,0.000000}%
\pgfsetstrokecolor{currentstroke}%
\pgfsetdash{}{0pt}%
\pgfpathmoveto{\pgfqpoint{4.019680in}{2.271565in}}%
\pgfpathlineto{\pgfqpoint{4.033190in}{2.268164in}}%
\pgfpathlineto{\pgfqpoint{4.046706in}{2.264792in}}%
\pgfpathlineto{\pgfqpoint{4.060228in}{2.261448in}}%
\pgfpathlineto{\pgfqpoint{4.073757in}{2.258133in}}%
\pgfpathlineto{\pgfqpoint{4.065940in}{2.250203in}}%
\pgfpathlineto{\pgfqpoint{4.058117in}{2.242273in}}%
\pgfpathlineto{\pgfqpoint{4.050288in}{2.234348in}}%
\pgfpathlineto{\pgfqpoint{4.042454in}{2.226428in}}%
\pgfpathlineto{\pgfqpoint{4.028913in}{2.229837in}}%
\pgfpathlineto{\pgfqpoint{4.015379in}{2.233276in}}%
\pgfpathlineto{\pgfqpoint{4.001850in}{2.236743in}}%
\pgfpathlineto{\pgfqpoint{3.988328in}{2.240238in}}%
\pgfpathlineto{\pgfqpoint{3.996175in}{2.248059in}}%
\pgfpathlineto{\pgfqpoint{4.004016in}{2.255888in}}%
\pgfpathlineto{\pgfqpoint{4.011851in}{2.263724in}}%
\pgfpathlineto{\pgfqpoint{4.019680in}{2.271565in}}%
\pgfpathclose%
\pgfusepath{fill}%
\end{pgfscope}%
\begin{pgfscope}%
\pgfpathrectangle{\pgfqpoint{1.150000in}{0.150000in}}{\pgfqpoint{5.700000in}{5.700000in}}%
\pgfusepath{clip}%
\pgfsetbuttcap%
\pgfsetroundjoin%
\definecolor{currentfill}{rgb}{0.276022,0.044167,0.370164}%
\pgfsetfillcolor{currentfill}%
\pgfsetfillopacity{0.700000}%
\pgfsetlinewidth{0.000000pt}%
\definecolor{currentstroke}{rgb}{0.000000,0.000000,0.000000}%
\pgfsetstrokecolor{currentstroke}%
\pgfsetdash{}{0pt}%
\pgfpathmoveto{\pgfqpoint{4.469040in}{2.328443in}}%
\pgfpathlineto{\pgfqpoint{4.482660in}{2.325954in}}%
\pgfpathlineto{\pgfqpoint{4.496288in}{2.323491in}}%
\pgfpathlineto{\pgfqpoint{4.509923in}{2.321056in}}%
\pgfpathlineto{\pgfqpoint{4.523565in}{2.318647in}}%
\pgfpathlineto{\pgfqpoint{4.515911in}{2.310670in}}%
\pgfpathlineto{\pgfqpoint{4.508251in}{2.302654in}}%
\pgfpathlineto{\pgfqpoint{4.500585in}{2.294601in}}%
\pgfpathlineto{\pgfqpoint{4.492914in}{2.286511in}}%
\pgfpathlineto{\pgfqpoint{4.479260in}{2.288962in}}%
\pgfpathlineto{\pgfqpoint{4.465613in}{2.291439in}}%
\pgfpathlineto{\pgfqpoint{4.451974in}{2.293943in}}%
\pgfpathlineto{\pgfqpoint{4.438341in}{2.296474in}}%
\pgfpathlineto{\pgfqpoint{4.446024in}{2.304517in}}%
\pgfpathlineto{\pgfqpoint{4.453702in}{2.312527in}}%
\pgfpathlineto{\pgfqpoint{4.461374in}{2.320502in}}%
\pgfpathlineto{\pgfqpoint{4.469040in}{2.328443in}}%
\pgfpathclose%
\pgfusepath{fill}%
\end{pgfscope}%
\begin{pgfscope}%
\pgfpathrectangle{\pgfqpoint{1.150000in}{0.150000in}}{\pgfqpoint{5.700000in}{5.700000in}}%
\pgfusepath{clip}%
\pgfsetbuttcap%
\pgfsetroundjoin%
\definecolor{currentfill}{rgb}{0.283072,0.130895,0.449241}%
\pgfsetfillcolor{currentfill}%
\pgfsetfillopacity{0.700000}%
\pgfsetlinewidth{0.000000pt}%
\definecolor{currentstroke}{rgb}{0.000000,0.000000,0.000000}%
\pgfsetstrokecolor{currentstroke}%
\pgfsetdash{}{0pt}%
\pgfpathmoveto{\pgfqpoint{5.452978in}{2.487051in}}%
\pgfpathlineto{\pgfqpoint{5.466871in}{2.485657in}}%
\pgfpathlineto{\pgfqpoint{5.480772in}{2.484287in}}%
\pgfpathlineto{\pgfqpoint{5.494682in}{2.482942in}}%
\pgfpathlineto{\pgfqpoint{5.508600in}{2.481622in}}%
\pgfpathlineto{\pgfqpoint{5.501357in}{2.475702in}}%
\pgfpathlineto{\pgfqpoint{5.494107in}{2.469742in}}%
\pgfpathlineto{\pgfqpoint{5.486851in}{2.463738in}}%
\pgfpathlineto{\pgfqpoint{5.479588in}{2.457690in}}%
\pgfpathlineto{\pgfqpoint{5.465654in}{2.458930in}}%
\pgfpathlineto{\pgfqpoint{5.451727in}{2.460195in}}%
\pgfpathlineto{\pgfqpoint{5.437809in}{2.461485in}}%
\pgfpathlineto{\pgfqpoint{5.423900in}{2.462799in}}%
\pgfpathlineto{\pgfqpoint{5.431179in}{2.468923in}}%
\pgfpathlineto{\pgfqpoint{5.438452in}{2.475004in}}%
\pgfpathlineto{\pgfqpoint{5.445718in}{2.481046in}}%
\pgfpathlineto{\pgfqpoint{5.452978in}{2.487051in}}%
\pgfpathclose%
\pgfusepath{fill}%
\end{pgfscope}%
\begin{pgfscope}%
\pgfpathrectangle{\pgfqpoint{1.150000in}{0.150000in}}{\pgfqpoint{5.700000in}{5.700000in}}%
\pgfusepath{clip}%
\pgfsetbuttcap%
\pgfsetroundjoin%
\definecolor{currentfill}{rgb}{0.280868,0.160771,0.472899}%
\pgfsetfillcolor{currentfill}%
\pgfsetfillopacity{0.700000}%
\pgfsetlinewidth{0.000000pt}%
\definecolor{currentstroke}{rgb}{0.000000,0.000000,0.000000}%
\pgfsetstrokecolor{currentstroke}%
\pgfsetdash{}{0pt}%
\pgfpathmoveto{\pgfqpoint{5.902245in}{2.543119in}}%
\pgfpathlineto{\pgfqpoint{5.916263in}{2.541853in}}%
\pgfpathlineto{\pgfqpoint{5.930289in}{2.540612in}}%
\pgfpathlineto{\pgfqpoint{5.944324in}{2.539394in}}%
\pgfpathlineto{\pgfqpoint{5.958367in}{2.538200in}}%
\pgfpathlineto{\pgfqpoint{5.951340in}{2.533302in}}%
\pgfpathlineto{\pgfqpoint{5.944307in}{2.528402in}}%
\pgfpathlineto{\pgfqpoint{5.937269in}{2.523493in}}%
\pgfpathlineto{\pgfqpoint{5.930224in}{2.518573in}}%
\pgfpathlineto{\pgfqpoint{5.916160in}{2.519633in}}%
\pgfpathlineto{\pgfqpoint{5.902105in}{2.520716in}}%
\pgfpathlineto{\pgfqpoint{5.888058in}{2.521824in}}%
\pgfpathlineto{\pgfqpoint{5.874020in}{2.522956in}}%
\pgfpathlineto{\pgfqpoint{5.881085in}{2.528005in}}%
\pgfpathlineto{\pgfqpoint{5.888144in}{2.533046in}}%
\pgfpathlineto{\pgfqpoint{5.895197in}{2.538082in}}%
\pgfpathlineto{\pgfqpoint{5.902245in}{2.543119in}}%
\pgfpathclose%
\pgfusepath{fill}%
\end{pgfscope}%
\begin{pgfscope}%
\pgfpathrectangle{\pgfqpoint{1.150000in}{0.150000in}}{\pgfqpoint{5.700000in}{5.700000in}}%
\pgfusepath{clip}%
\pgfsetbuttcap%
\pgfsetroundjoin%
\definecolor{currentfill}{rgb}{0.282290,0.145912,0.461510}%
\pgfsetfillcolor{currentfill}%
\pgfsetfillopacity{0.700000}%
\pgfsetlinewidth{0.000000pt}%
\definecolor{currentstroke}{rgb}{0.000000,0.000000,0.000000}%
\pgfsetstrokecolor{currentstroke}%
\pgfsetdash{}{0pt}%
\pgfpathmoveto{\pgfqpoint{5.677662in}{2.516614in}}%
\pgfpathlineto{\pgfqpoint{5.691618in}{2.515313in}}%
\pgfpathlineto{\pgfqpoint{5.705583in}{2.514036in}}%
\pgfpathlineto{\pgfqpoint{5.719556in}{2.512783in}}%
\pgfpathlineto{\pgfqpoint{5.733538in}{2.511555in}}%
\pgfpathlineto{\pgfqpoint{5.726402in}{2.506181in}}%
\pgfpathlineto{\pgfqpoint{5.719259in}{2.500782in}}%
\pgfpathlineto{\pgfqpoint{5.712110in}{2.495355in}}%
\pgfpathlineto{\pgfqpoint{5.704955in}{2.489896in}}%
\pgfpathlineto{\pgfqpoint{5.690954in}{2.491018in}}%
\pgfpathlineto{\pgfqpoint{5.676963in}{2.492164in}}%
\pgfpathlineto{\pgfqpoint{5.662979in}{2.493334in}}%
\pgfpathlineto{\pgfqpoint{5.649004in}{2.494528in}}%
\pgfpathlineto{\pgfqpoint{5.656178in}{2.500088in}}%
\pgfpathlineto{\pgfqpoint{5.663346in}{2.505621in}}%
\pgfpathlineto{\pgfqpoint{5.670507in}{2.511128in}}%
\pgfpathlineto{\pgfqpoint{5.677662in}{2.516614in}}%
\pgfpathclose%
\pgfusepath{fill}%
\end{pgfscope}%
\begin{pgfscope}%
\pgfpathrectangle{\pgfqpoint{1.150000in}{0.150000in}}{\pgfqpoint{5.700000in}{5.700000in}}%
\pgfusepath{clip}%
\pgfsetbuttcap%
\pgfsetroundjoin%
\definecolor{currentfill}{rgb}{0.276022,0.044167,0.370164}%
\pgfsetfillcolor{currentfill}%
\pgfsetfillopacity{0.700000}%
\pgfsetlinewidth{0.000000pt}%
\definecolor{currentstroke}{rgb}{0.000000,0.000000,0.000000}%
\pgfsetstrokecolor{currentstroke}%
\pgfsetdash{}{0pt}%
\pgfpathmoveto{\pgfqpoint{3.097992in}{2.331431in}}%
\pgfpathlineto{\pgfqpoint{3.111338in}{2.325368in}}%
\pgfpathlineto{\pgfqpoint{3.124688in}{2.319344in}}%
\pgfpathlineto{\pgfqpoint{3.138042in}{2.313358in}}%
\pgfpathlineto{\pgfqpoint{3.151400in}{2.307411in}}%
\pgfpathlineto{\pgfqpoint{3.143187in}{2.302954in}}%
\pgfpathlineto{\pgfqpoint{3.134964in}{2.298634in}}%
\pgfpathlineto{\pgfqpoint{3.126731in}{2.294454in}}%
\pgfpathlineto{\pgfqpoint{3.118488in}{2.290419in}}%
\pgfpathlineto{\pgfqpoint{3.105109in}{2.296555in}}%
\pgfpathlineto{\pgfqpoint{3.091735in}{2.302729in}}%
\pgfpathlineto{\pgfqpoint{3.078364in}{2.308941in}}%
\pgfpathlineto{\pgfqpoint{3.064998in}{2.315192in}}%
\pgfpathlineto{\pgfqpoint{3.073261in}{2.319034in}}%
\pgfpathlineto{\pgfqpoint{3.081515in}{2.323024in}}%
\pgfpathlineto{\pgfqpoint{3.089758in}{2.327158in}}%
\pgfpathlineto{\pgfqpoint{3.097992in}{2.331431in}}%
\pgfpathclose%
\pgfusepath{fill}%
\end{pgfscope}%
\begin{pgfscope}%
\pgfpathrectangle{\pgfqpoint{1.150000in}{0.150000in}}{\pgfqpoint{5.700000in}{5.700000in}}%
\pgfusepath{clip}%
\pgfsetbuttcap%
\pgfsetroundjoin%
\definecolor{currentfill}{rgb}{0.279566,0.067836,0.391917}%
\pgfsetfillcolor{currentfill}%
\pgfsetfillopacity{0.700000}%
\pgfsetlinewidth{0.000000pt}%
\definecolor{currentstroke}{rgb}{0.000000,0.000000,0.000000}%
\pgfsetstrokecolor{currentstroke}%
\pgfsetdash{}{0pt}%
\pgfpathmoveto{\pgfqpoint{4.693785in}{2.363083in}}%
\pgfpathlineto{\pgfqpoint{4.707468in}{2.360954in}}%
\pgfpathlineto{\pgfqpoint{4.721157in}{2.358851in}}%
\pgfpathlineto{\pgfqpoint{4.734855in}{2.356774in}}%
\pgfpathlineto{\pgfqpoint{4.748559in}{2.354723in}}%
\pgfpathlineto{\pgfqpoint{4.740989in}{2.347029in}}%
\pgfpathlineto{\pgfqpoint{4.733413in}{2.339286in}}%
\pgfpathlineto{\pgfqpoint{4.725831in}{2.331494in}}%
\pgfpathlineto{\pgfqpoint{4.718243in}{2.323653in}}%
\pgfpathlineto{\pgfqpoint{4.704526in}{2.325719in}}%
\pgfpathlineto{\pgfqpoint{4.690816in}{2.327811in}}%
\pgfpathlineto{\pgfqpoint{4.677114in}{2.329929in}}%
\pgfpathlineto{\pgfqpoint{4.663419in}{2.332072in}}%
\pgfpathlineto{\pgfqpoint{4.671020in}{2.339894in}}%
\pgfpathlineto{\pgfqpoint{4.678614in}{2.347670in}}%
\pgfpathlineto{\pgfqpoint{4.686203in}{2.355399in}}%
\pgfpathlineto{\pgfqpoint{4.693785in}{2.363083in}}%
\pgfpathclose%
\pgfusepath{fill}%
\end{pgfscope}%
\begin{pgfscope}%
\pgfpathrectangle{\pgfqpoint{1.150000in}{0.150000in}}{\pgfqpoint{5.700000in}{5.700000in}}%
\pgfusepath{clip}%
\pgfsetbuttcap%
\pgfsetroundjoin%
\definecolor{currentfill}{rgb}{0.267004,0.004874,0.329415}%
\pgfsetfillcolor{currentfill}%
\pgfsetfillopacity{0.700000}%
\pgfsetlinewidth{0.000000pt}%
\definecolor{currentstroke}{rgb}{0.000000,0.000000,0.000000}%
\pgfsetstrokecolor{currentstroke}%
\pgfsetdash{}{0pt}%
\pgfpathmoveto{\pgfqpoint{3.794932in}{2.254627in}}%
\pgfpathlineto{\pgfqpoint{3.808395in}{2.250668in}}%
\pgfpathlineto{\pgfqpoint{3.821864in}{2.246740in}}%
\pgfpathlineto{\pgfqpoint{3.835339in}{2.242841in}}%
\pgfpathlineto{\pgfqpoint{3.848820in}{2.238973in}}%
\pgfpathlineto{\pgfqpoint{3.840918in}{2.231464in}}%
\pgfpathlineto{\pgfqpoint{3.833009in}{2.223986in}}%
\pgfpathlineto{\pgfqpoint{3.825095in}{2.216540in}}%
\pgfpathlineto{\pgfqpoint{3.817174in}{2.209130in}}%
\pgfpathlineto{\pgfqpoint{3.803680in}{2.213120in}}%
\pgfpathlineto{\pgfqpoint{3.790191in}{2.217139in}}%
\pgfpathlineto{\pgfqpoint{3.776708in}{2.221189in}}%
\pgfpathlineto{\pgfqpoint{3.763231in}{2.225269in}}%
\pgfpathlineto{\pgfqpoint{3.771165in}{2.232553in}}%
\pgfpathlineto{\pgfqpoint{3.779094in}{2.239876in}}%
\pgfpathlineto{\pgfqpoint{3.787016in}{2.247235in}}%
\pgfpathlineto{\pgfqpoint{3.794932in}{2.254627in}}%
\pgfpathclose%
\pgfusepath{fill}%
\end{pgfscope}%
\begin{pgfscope}%
\pgfpathrectangle{\pgfqpoint{1.150000in}{0.150000in}}{\pgfqpoint{5.700000in}{5.700000in}}%
\pgfusepath{clip}%
\pgfsetbuttcap%
\pgfsetroundjoin%
\definecolor{currentfill}{rgb}{0.281924,0.089666,0.412415}%
\pgfsetfillcolor{currentfill}%
\pgfsetfillopacity{0.700000}%
\pgfsetlinewidth{0.000000pt}%
\definecolor{currentstroke}{rgb}{0.000000,0.000000,0.000000}%
\pgfsetstrokecolor{currentstroke}%
\pgfsetdash{}{0pt}%
\pgfpathmoveto{\pgfqpoint{4.918599in}{2.398986in}}%
\pgfpathlineto{\pgfqpoint{4.932346in}{2.397155in}}%
\pgfpathlineto{\pgfqpoint{4.946100in}{2.395350in}}%
\pgfpathlineto{\pgfqpoint{4.959862in}{2.393570in}}%
\pgfpathlineto{\pgfqpoint{4.973631in}{2.391816in}}%
\pgfpathlineto{\pgfqpoint{4.966150in}{2.384550in}}%
\pgfpathlineto{\pgfqpoint{4.958662in}{2.377229in}}%
\pgfpathlineto{\pgfqpoint{4.951168in}{2.369853in}}%
\pgfpathlineto{\pgfqpoint{4.943667in}{2.362421in}}%
\pgfpathlineto{\pgfqpoint{4.929885in}{2.364163in}}%
\pgfpathlineto{\pgfqpoint{4.916110in}{2.365930in}}%
\pgfpathlineto{\pgfqpoint{4.902343in}{2.367723in}}%
\pgfpathlineto{\pgfqpoint{4.888583in}{2.369542in}}%
\pgfpathlineto{\pgfqpoint{4.896096in}{2.376981in}}%
\pgfpathlineto{\pgfqpoint{4.903604in}{2.384368in}}%
\pgfpathlineto{\pgfqpoint{4.911104in}{2.391702in}}%
\pgfpathlineto{\pgfqpoint{4.918599in}{2.398986in}}%
\pgfpathclose%
\pgfusepath{fill}%
\end{pgfscope}%
\begin{pgfscope}%
\pgfpathrectangle{\pgfqpoint{1.150000in}{0.150000in}}{\pgfqpoint{5.700000in}{5.700000in}}%
\pgfusepath{clip}%
\pgfsetbuttcap%
\pgfsetroundjoin%
\definecolor{currentfill}{rgb}{0.283091,0.110553,0.431554}%
\pgfsetfillcolor{currentfill}%
\pgfsetfillopacity{0.700000}%
\pgfsetlinewidth{0.000000pt}%
\definecolor{currentstroke}{rgb}{0.000000,0.000000,0.000000}%
\pgfsetstrokecolor{currentstroke}%
\pgfsetdash{}{0pt}%
\pgfpathmoveto{\pgfqpoint{2.764823in}{2.439166in}}%
\pgfpathlineto{\pgfqpoint{2.778142in}{2.431861in}}%
\pgfpathlineto{\pgfqpoint{2.791465in}{2.424603in}}%
\pgfpathlineto{\pgfqpoint{2.804790in}{2.417390in}}%
\pgfpathlineto{\pgfqpoint{2.818119in}{2.410223in}}%
\pgfpathlineto{\pgfqpoint{2.809707in}{2.408041in}}%
\pgfpathlineto{\pgfqpoint{2.801283in}{2.406051in}}%
\pgfpathlineto{\pgfqpoint{2.792846in}{2.404260in}}%
\pgfpathlineto{\pgfqpoint{2.784395in}{2.402672in}}%
\pgfpathlineto{\pgfqpoint{2.771042in}{2.410056in}}%
\pgfpathlineto{\pgfqpoint{2.757691in}{2.417485in}}%
\pgfpathlineto{\pgfqpoint{2.744343in}{2.424960in}}%
\pgfpathlineto{\pgfqpoint{2.730998in}{2.432482in}}%
\pgfpathlineto{\pgfqpoint{2.739474in}{2.433848in}}%
\pgfpathlineto{\pgfqpoint{2.747937in}{2.435421in}}%
\pgfpathlineto{\pgfqpoint{2.756386in}{2.437195in}}%
\pgfpathlineto{\pgfqpoint{2.764823in}{2.439166in}}%
\pgfpathclose%
\pgfusepath{fill}%
\end{pgfscope}%
\begin{pgfscope}%
\pgfpathrectangle{\pgfqpoint{1.150000in}{0.150000in}}{\pgfqpoint{5.700000in}{5.700000in}}%
\pgfusepath{clip}%
\pgfsetbuttcap%
\pgfsetroundjoin%
\definecolor{currentfill}{rgb}{0.269944,0.014625,0.341379}%
\pgfsetfillcolor{currentfill}%
\pgfsetfillopacity{0.700000}%
\pgfsetlinewidth{0.000000pt}%
\definecolor{currentstroke}{rgb}{0.000000,0.000000,0.000000}%
\pgfsetstrokecolor{currentstroke}%
\pgfsetdash{}{0pt}%
\pgfpathmoveto{\pgfqpoint{4.159097in}{2.277185in}}%
\pgfpathlineto{\pgfqpoint{4.172645in}{2.274093in}}%
\pgfpathlineto{\pgfqpoint{4.186200in}{2.271029in}}%
\pgfpathlineto{\pgfqpoint{4.199762in}{2.267993in}}%
\pgfpathlineto{\pgfqpoint{4.213330in}{2.264985in}}%
\pgfpathlineto{\pgfqpoint{4.205560in}{2.256913in}}%
\pgfpathlineto{\pgfqpoint{4.197784in}{2.248829in}}%
\pgfpathlineto{\pgfqpoint{4.190003in}{2.240735in}}%
\pgfpathlineto{\pgfqpoint{4.182216in}{2.232631in}}%
\pgfpathlineto{\pgfqpoint{4.168636in}{2.235721in}}%
\pgfpathlineto{\pgfqpoint{4.155062in}{2.238839in}}%
\pgfpathlineto{\pgfqpoint{4.141495in}{2.241984in}}%
\pgfpathlineto{\pgfqpoint{4.127935in}{2.245158in}}%
\pgfpathlineto{\pgfqpoint{4.135734in}{2.253174in}}%
\pgfpathlineto{\pgfqpoint{4.143527in}{2.261186in}}%
\pgfpathlineto{\pgfqpoint{4.151315in}{2.269190in}}%
\pgfpathlineto{\pgfqpoint{4.159097in}{2.277185in}}%
\pgfpathclose%
\pgfusepath{fill}%
\end{pgfscope}%
\begin{pgfscope}%
\pgfpathrectangle{\pgfqpoint{1.150000in}{0.150000in}}{\pgfqpoint{5.700000in}{5.700000in}}%
\pgfusepath{clip}%
\pgfsetbuttcap%
\pgfsetroundjoin%
\definecolor{currentfill}{rgb}{0.283091,0.110553,0.431554}%
\pgfsetfillcolor{currentfill}%
\pgfsetfillopacity{0.700000}%
\pgfsetlinewidth{0.000000pt}%
\definecolor{currentstroke}{rgb}{0.000000,0.000000,0.000000}%
\pgfsetstrokecolor{currentstroke}%
\pgfsetdash{}{0pt}%
\pgfpathmoveto{\pgfqpoint{5.143463in}{2.434485in}}%
\pgfpathlineto{\pgfqpoint{5.157275in}{2.432893in}}%
\pgfpathlineto{\pgfqpoint{5.171095in}{2.431326in}}%
\pgfpathlineto{\pgfqpoint{5.184923in}{2.429784in}}%
\pgfpathlineto{\pgfqpoint{5.198758in}{2.428267in}}%
\pgfpathlineto{\pgfqpoint{5.191371in}{2.421524in}}%
\pgfpathlineto{\pgfqpoint{5.183978in}{2.414728in}}%
\pgfpathlineto{\pgfqpoint{5.176578in}{2.407876in}}%
\pgfpathlineto{\pgfqpoint{5.169171in}{2.400968in}}%
\pgfpathlineto{\pgfqpoint{5.155321in}{2.402446in}}%
\pgfpathlineto{\pgfqpoint{5.141479in}{2.403948in}}%
\pgfpathlineto{\pgfqpoint{5.127645in}{2.405476in}}%
\pgfpathlineto{\pgfqpoint{5.113819in}{2.407029in}}%
\pgfpathlineto{\pgfqpoint{5.121240in}{2.413972in}}%
\pgfpathlineto{\pgfqpoint{5.128654in}{2.420861in}}%
\pgfpathlineto{\pgfqpoint{5.136062in}{2.427698in}}%
\pgfpathlineto{\pgfqpoint{5.143463in}{2.434485in}}%
\pgfpathclose%
\pgfusepath{fill}%
\end{pgfscope}%
\begin{pgfscope}%
\pgfpathrectangle{\pgfqpoint{1.150000in}{0.150000in}}{\pgfqpoint{5.700000in}{5.700000in}}%
\pgfusepath{clip}%
\pgfsetbuttcap%
\pgfsetroundjoin%
\definecolor{currentfill}{rgb}{0.273809,0.031497,0.358853}%
\pgfsetfillcolor{currentfill}%
\pgfsetfillopacity{0.700000}%
\pgfsetlinewidth{0.000000pt}%
\definecolor{currentstroke}{rgb}{0.000000,0.000000,0.000000}%
\pgfsetstrokecolor{currentstroke}%
\pgfsetdash{}{0pt}%
\pgfpathmoveto{\pgfqpoint{4.383880in}{2.306865in}}%
\pgfpathlineto{\pgfqpoint{4.397485in}{2.304227in}}%
\pgfpathlineto{\pgfqpoint{4.411097in}{2.301616in}}%
\pgfpathlineto{\pgfqpoint{4.424715in}{2.299031in}}%
\pgfpathlineto{\pgfqpoint{4.438341in}{2.296474in}}%
\pgfpathlineto{\pgfqpoint{4.430652in}{2.288398in}}%
\pgfpathlineto{\pgfqpoint{4.422957in}{2.280290in}}%
\pgfpathlineto{\pgfqpoint{4.415257in}{2.272151in}}%
\pgfpathlineto{\pgfqpoint{4.407551in}{2.263981in}}%
\pgfpathlineto{\pgfqpoint{4.393913in}{2.266594in}}%
\pgfpathlineto{\pgfqpoint{4.380283in}{2.269233in}}%
\pgfpathlineto{\pgfqpoint{4.366659in}{2.271899in}}%
\pgfpathlineto{\pgfqpoint{4.353042in}{2.274593in}}%
\pgfpathlineto{\pgfqpoint{4.360760in}{2.282702in}}%
\pgfpathlineto{\pgfqpoint{4.368473in}{2.290784in}}%
\pgfpathlineto{\pgfqpoint{4.376179in}{2.298839in}}%
\pgfpathlineto{\pgfqpoint{4.383880in}{2.306865in}}%
\pgfpathclose%
\pgfusepath{fill}%
\end{pgfscope}%
\begin{pgfscope}%
\pgfpathrectangle{\pgfqpoint{1.150000in}{0.150000in}}{\pgfqpoint{5.700000in}{5.700000in}}%
\pgfusepath{clip}%
\pgfsetbuttcap%
\pgfsetroundjoin%
\definecolor{currentfill}{rgb}{0.280267,0.073417,0.397163}%
\pgfsetfillcolor{currentfill}%
\pgfsetfillopacity{0.700000}%
\pgfsetlinewidth{0.000000pt}%
\definecolor{currentstroke}{rgb}{0.000000,0.000000,0.000000}%
\pgfsetstrokecolor{currentstroke}%
\pgfsetdash{}{0pt}%
\pgfpathmoveto{\pgfqpoint{2.958204in}{2.366641in}}%
\pgfpathlineto{\pgfqpoint{2.971540in}{2.360067in}}%
\pgfpathlineto{\pgfqpoint{2.984880in}{2.353535in}}%
\pgfpathlineto{\pgfqpoint{2.998223in}{2.347043in}}%
\pgfpathlineto{\pgfqpoint{3.011570in}{2.340593in}}%
\pgfpathlineto{\pgfqpoint{3.003275in}{2.337101in}}%
\pgfpathlineto{\pgfqpoint{2.994968in}{2.333771in}}%
\pgfpathlineto{\pgfqpoint{2.986651in}{2.330608in}}%
\pgfpathlineto{\pgfqpoint{2.978322in}{2.327616in}}%
\pgfpathlineto{\pgfqpoint{2.964952in}{2.334269in}}%
\pgfpathlineto{\pgfqpoint{2.951586in}{2.340962in}}%
\pgfpathlineto{\pgfqpoint{2.938224in}{2.347697in}}%
\pgfpathlineto{\pgfqpoint{2.924865in}{2.354473in}}%
\pgfpathlineto{\pgfqpoint{2.933217in}{2.357257in}}%
\pgfpathlineto{\pgfqpoint{2.941557in}{2.360217in}}%
\pgfpathlineto{\pgfqpoint{2.949886in}{2.363346in}}%
\pgfpathlineto{\pgfqpoint{2.958204in}{2.366641in}}%
\pgfpathclose%
\pgfusepath{fill}%
\end{pgfscope}%
\begin{pgfscope}%
\pgfpathrectangle{\pgfqpoint{1.150000in}{0.150000in}}{\pgfqpoint{5.700000in}{5.700000in}}%
\pgfusepath{clip}%
\pgfsetbuttcap%
\pgfsetroundjoin%
\definecolor{currentfill}{rgb}{0.267004,0.004874,0.329415}%
\pgfsetfillcolor{currentfill}%
\pgfsetfillopacity{0.700000}%
\pgfsetlinewidth{0.000000pt}%
\definecolor{currentstroke}{rgb}{0.000000,0.000000,0.000000}%
\pgfsetstrokecolor{currentstroke}%
\pgfsetdash{}{0pt}%
\pgfpathmoveto{\pgfqpoint{3.934300in}{2.254512in}}%
\pgfpathlineto{\pgfqpoint{3.947798in}{2.250900in}}%
\pgfpathlineto{\pgfqpoint{3.961302in}{2.247317in}}%
\pgfpathlineto{\pgfqpoint{3.974812in}{2.243763in}}%
\pgfpathlineto{\pgfqpoint{3.988328in}{2.240238in}}%
\pgfpathlineto{\pgfqpoint{3.980475in}{2.232429in}}%
\pgfpathlineto{\pgfqpoint{3.972617in}{2.224633in}}%
\pgfpathlineto{\pgfqpoint{3.964753in}{2.216853in}}%
\pgfpathlineto{\pgfqpoint{3.956883in}{2.209092in}}%
\pgfpathlineto{\pgfqpoint{3.943354in}{2.212724in}}%
\pgfpathlineto{\pgfqpoint{3.929831in}{2.216386in}}%
\pgfpathlineto{\pgfqpoint{3.916314in}{2.220077in}}%
\pgfpathlineto{\pgfqpoint{3.902803in}{2.223797in}}%
\pgfpathlineto{\pgfqpoint{3.910686in}{2.231446in}}%
\pgfpathlineto{\pgfqpoint{3.918564in}{2.239116in}}%
\pgfpathlineto{\pgfqpoint{3.926435in}{2.246806in}}%
\pgfpathlineto{\pgfqpoint{3.934300in}{2.254512in}}%
\pgfpathclose%
\pgfusepath{fill}%
\end{pgfscope}%
\begin{pgfscope}%
\pgfpathrectangle{\pgfqpoint{1.150000in}{0.150000in}}{\pgfqpoint{5.700000in}{5.700000in}}%
\pgfusepath{clip}%
\pgfsetbuttcap%
\pgfsetroundjoin%
\definecolor{currentfill}{rgb}{0.283187,0.125848,0.444960}%
\pgfsetfillcolor{currentfill}%
\pgfsetfillopacity{0.700000}%
\pgfsetlinewidth{0.000000pt}%
\definecolor{currentstroke}{rgb}{0.000000,0.000000,0.000000}%
\pgfsetstrokecolor{currentstroke}%
\pgfsetdash{}{0pt}%
\pgfpathmoveto{\pgfqpoint{5.368342in}{2.468303in}}%
\pgfpathlineto{\pgfqpoint{5.382219in}{2.466890in}}%
\pgfpathlineto{\pgfqpoint{5.396105in}{2.465502in}}%
\pgfpathlineto{\pgfqpoint{5.409998in}{2.464138in}}%
\pgfpathlineto{\pgfqpoint{5.423900in}{2.462799in}}%
\pgfpathlineto{\pgfqpoint{5.416613in}{2.456631in}}%
\pgfpathlineto{\pgfqpoint{5.409320in}{2.450415in}}%
\pgfpathlineto{\pgfqpoint{5.402020in}{2.444150in}}%
\pgfpathlineto{\pgfqpoint{5.394713in}{2.437832in}}%
\pgfpathlineto{\pgfqpoint{5.380795in}{2.439105in}}%
\pgfpathlineto{\pgfqpoint{5.366886in}{2.440402in}}%
\pgfpathlineto{\pgfqpoint{5.352985in}{2.441724in}}%
\pgfpathlineto{\pgfqpoint{5.339092in}{2.443070in}}%
\pgfpathlineto{\pgfqpoint{5.346415in}{2.449449in}}%
\pgfpathlineto{\pgfqpoint{5.353730in}{2.455779in}}%
\pgfpathlineto{\pgfqpoint{5.361040in}{2.462063in}}%
\pgfpathlineto{\pgfqpoint{5.368342in}{2.468303in}}%
\pgfpathclose%
\pgfusepath{fill}%
\end{pgfscope}%
\begin{pgfscope}%
\pgfpathrectangle{\pgfqpoint{1.150000in}{0.150000in}}{\pgfqpoint{5.700000in}{5.700000in}}%
\pgfusepath{clip}%
\pgfsetbuttcap%
\pgfsetroundjoin%
\definecolor{currentfill}{rgb}{0.277941,0.056324,0.381191}%
\pgfsetfillcolor{currentfill}%
\pgfsetfillopacity{0.700000}%
\pgfsetlinewidth{0.000000pt}%
\definecolor{currentstroke}{rgb}{0.000000,0.000000,0.000000}%
\pgfsetstrokecolor{currentstroke}%
\pgfsetdash{}{0pt}%
\pgfpathmoveto{\pgfqpoint{4.608713in}{2.340909in}}%
\pgfpathlineto{\pgfqpoint{4.622379in}{2.338661in}}%
\pgfpathlineto{\pgfqpoint{4.636052in}{2.336439in}}%
\pgfpathlineto{\pgfqpoint{4.649732in}{2.334242in}}%
\pgfpathlineto{\pgfqpoint{4.663419in}{2.332072in}}%
\pgfpathlineto{\pgfqpoint{4.655813in}{2.324205in}}%
\pgfpathlineto{\pgfqpoint{4.648201in}{2.316291in}}%
\pgfpathlineto{\pgfqpoint{4.640582in}{2.308331in}}%
\pgfpathlineto{\pgfqpoint{4.632958in}{2.300326in}}%
\pgfpathlineto{\pgfqpoint{4.619258in}{2.302524in}}%
\pgfpathlineto{\pgfqpoint{4.605566in}{2.304748in}}%
\pgfpathlineto{\pgfqpoint{4.591881in}{2.306999in}}%
\pgfpathlineto{\pgfqpoint{4.578204in}{2.309276in}}%
\pgfpathlineto{\pgfqpoint{4.585840in}{2.317248in}}%
\pgfpathlineto{\pgfqpoint{4.593470in}{2.325178in}}%
\pgfpathlineto{\pgfqpoint{4.601095in}{2.333065in}}%
\pgfpathlineto{\pgfqpoint{4.608713in}{2.340909in}}%
\pgfpathclose%
\pgfusepath{fill}%
\end{pgfscope}%
\begin{pgfscope}%
\pgfpathrectangle{\pgfqpoint{1.150000in}{0.150000in}}{\pgfqpoint{5.700000in}{5.700000in}}%
\pgfusepath{clip}%
\pgfsetbuttcap%
\pgfsetroundjoin%
\definecolor{currentfill}{rgb}{0.281412,0.155834,0.469201}%
\pgfsetfillcolor{currentfill}%
\pgfsetfillopacity{0.700000}%
\pgfsetlinewidth{0.000000pt}%
\definecolor{currentstroke}{rgb}{0.000000,0.000000,0.000000}%
\pgfsetstrokecolor{currentstroke}%
\pgfsetdash{}{0pt}%
\pgfpathmoveto{\pgfqpoint{2.571066in}{2.526532in}}%
\pgfpathlineto{\pgfqpoint{2.584381in}{2.518416in}}%
\pgfpathlineto{\pgfqpoint{2.597697in}{2.510353in}}%
\pgfpathlineto{\pgfqpoint{2.611016in}{2.502341in}}%
\pgfpathlineto{\pgfqpoint{2.624337in}{2.494380in}}%
\pgfpathlineto{\pgfqpoint{2.615793in}{2.493679in}}%
\pgfpathlineto{\pgfqpoint{2.607235in}{2.493202in}}%
\pgfpathlineto{\pgfqpoint{2.598661in}{2.492958in}}%
\pgfpathlineto{\pgfqpoint{2.590072in}{2.492951in}}%
\pgfpathlineto{\pgfqpoint{2.576722in}{2.501143in}}%
\pgfpathlineto{\pgfqpoint{2.563375in}{2.509386in}}%
\pgfpathlineto{\pgfqpoint{2.550030in}{2.517681in}}%
\pgfpathlineto{\pgfqpoint{2.536688in}{2.526029in}}%
\pgfpathlineto{\pgfqpoint{2.545306in}{2.525799in}}%
\pgfpathlineto{\pgfqpoint{2.553908in}{2.525810in}}%
\pgfpathlineto{\pgfqpoint{2.562495in}{2.526057in}}%
\pgfpathlineto{\pgfqpoint{2.571066in}{2.526532in}}%
\pgfpathclose%
\pgfusepath{fill}%
\end{pgfscope}%
\begin{pgfscope}%
\pgfpathrectangle{\pgfqpoint{1.150000in}{0.150000in}}{\pgfqpoint{5.700000in}{5.700000in}}%
\pgfusepath{clip}%
\pgfsetbuttcap%
\pgfsetroundjoin%
\definecolor{currentfill}{rgb}{0.279574,0.170599,0.479997}%
\pgfsetfillcolor{currentfill}%
\pgfsetfillopacity{0.700000}%
\pgfsetlinewidth{0.000000pt}%
\definecolor{currentstroke}{rgb}{0.000000,0.000000,0.000000}%
\pgfsetstrokecolor{currentstroke}%
\pgfsetdash{}{0pt}%
\pgfpathmoveto{\pgfqpoint{6.042590in}{2.552723in}}%
\pgfpathlineto{\pgfqpoint{6.056655in}{2.551501in}}%
\pgfpathlineto{\pgfqpoint{6.070728in}{2.550304in}}%
\pgfpathlineto{\pgfqpoint{6.084809in}{2.549130in}}%
\pgfpathlineto{\pgfqpoint{6.098900in}{2.547980in}}%
\pgfpathlineto{\pgfqpoint{6.091939in}{2.543356in}}%
\pgfpathlineto{\pgfqpoint{6.084972in}{2.538739in}}%
\pgfpathlineto{\pgfqpoint{6.078000in}{2.534126in}}%
\pgfpathlineto{\pgfqpoint{6.071023in}{2.529513in}}%
\pgfpathlineto{\pgfqpoint{6.056910in}{2.530515in}}%
\pgfpathlineto{\pgfqpoint{6.042807in}{2.531541in}}%
\pgfpathlineto{\pgfqpoint{6.028712in}{2.532591in}}%
\pgfpathlineto{\pgfqpoint{6.014625in}{2.533665in}}%
\pgfpathlineto{\pgfqpoint{6.021625in}{2.538421in}}%
\pgfpathlineto{\pgfqpoint{6.028619in}{2.543180in}}%
\pgfpathlineto{\pgfqpoint{6.035607in}{2.547946in}}%
\pgfpathlineto{\pgfqpoint{6.042590in}{2.552723in}}%
\pgfpathclose%
\pgfusepath{fill}%
\end{pgfscope}%
\begin{pgfscope}%
\pgfpathrectangle{\pgfqpoint{1.150000in}{0.150000in}}{\pgfqpoint{5.700000in}{5.700000in}}%
\pgfusepath{clip}%
\pgfsetbuttcap%
\pgfsetroundjoin%
\definecolor{currentfill}{rgb}{0.268510,0.009605,0.335427}%
\pgfsetfillcolor{currentfill}%
\pgfsetfillopacity{0.700000}%
\pgfsetlinewidth{0.000000pt}%
\definecolor{currentstroke}{rgb}{0.000000,0.000000,0.000000}%
\pgfsetstrokecolor{currentstroke}%
\pgfsetdash{}{0pt}%
\pgfpathmoveto{\pgfqpoint{3.430534in}{2.262614in}}%
\pgfpathlineto{\pgfqpoint{3.443934in}{2.257612in}}%
\pgfpathlineto{\pgfqpoint{3.457340in}{2.252644in}}%
\pgfpathlineto{\pgfqpoint{3.470750in}{2.247709in}}%
\pgfpathlineto{\pgfqpoint{3.484165in}{2.242808in}}%
\pgfpathlineto{\pgfqpoint{3.476109in}{2.236640in}}%
\pgfpathlineto{\pgfqpoint{3.468045in}{2.230558in}}%
\pgfpathlineto{\pgfqpoint{3.459974in}{2.224566in}}%
\pgfpathlineto{\pgfqpoint{3.451896in}{2.218667in}}%
\pgfpathlineto{\pgfqpoint{3.438464in}{2.223729in}}%
\pgfpathlineto{\pgfqpoint{3.425036in}{2.228825in}}%
\pgfpathlineto{\pgfqpoint{3.411614in}{2.233954in}}%
\pgfpathlineto{\pgfqpoint{3.398197in}{2.239117in}}%
\pgfpathlineto{\pgfqpoint{3.406293in}{2.244851in}}%
\pgfpathlineto{\pgfqpoint{3.414381in}{2.250680in}}%
\pgfpathlineto{\pgfqpoint{3.422461in}{2.256603in}}%
\pgfpathlineto{\pgfqpoint{3.430534in}{2.262614in}}%
\pgfpathclose%
\pgfusepath{fill}%
\end{pgfscope}%
\begin{pgfscope}%
\pgfpathrectangle{\pgfqpoint{1.150000in}{0.150000in}}{\pgfqpoint{5.700000in}{5.700000in}}%
\pgfusepath{clip}%
\pgfsetbuttcap%
\pgfsetroundjoin%
\definecolor{currentfill}{rgb}{0.282290,0.145912,0.461510}%
\pgfsetfillcolor{currentfill}%
\pgfsetfillopacity{0.700000}%
\pgfsetlinewidth{0.000000pt}%
\definecolor{currentstroke}{rgb}{0.000000,0.000000,0.000000}%
\pgfsetstrokecolor{currentstroke}%
\pgfsetdash{}{0pt}%
\pgfpathmoveto{\pgfqpoint{5.593188in}{2.499549in}}%
\pgfpathlineto{\pgfqpoint{5.607130in}{2.498257in}}%
\pgfpathlineto{\pgfqpoint{5.621079in}{2.496990in}}%
\pgfpathlineto{\pgfqpoint{5.635038in}{2.495747in}}%
\pgfpathlineto{\pgfqpoint{5.649004in}{2.494528in}}%
\pgfpathlineto{\pgfqpoint{5.641823in}{2.488936in}}%
\pgfpathlineto{\pgfqpoint{5.634636in}{2.483309in}}%
\pgfpathlineto{\pgfqpoint{5.627442in}{2.477644in}}%
\pgfpathlineto{\pgfqpoint{5.620240in}{2.471938in}}%
\pgfpathlineto{\pgfqpoint{5.606256in}{2.473063in}}%
\pgfpathlineto{\pgfqpoint{5.592280in}{2.474213in}}%
\pgfpathlineto{\pgfqpoint{5.578312in}{2.475386in}}%
\pgfpathlineto{\pgfqpoint{5.564353in}{2.476585in}}%
\pgfpathlineto{\pgfqpoint{5.571572in}{2.482379in}}%
\pgfpathlineto{\pgfqpoint{5.578784in}{2.488136in}}%
\pgfpathlineto{\pgfqpoint{5.585989in}{2.493858in}}%
\pgfpathlineto{\pgfqpoint{5.593188in}{2.499549in}}%
\pgfpathclose%
\pgfusepath{fill}%
\end{pgfscope}%
\begin{pgfscope}%
\pgfpathrectangle{\pgfqpoint{1.150000in}{0.150000in}}{\pgfqpoint{5.700000in}{5.700000in}}%
\pgfusepath{clip}%
\pgfsetbuttcap%
\pgfsetroundjoin%
\definecolor{currentfill}{rgb}{0.280868,0.160771,0.472899}%
\pgfsetfillcolor{currentfill}%
\pgfsetfillopacity{0.700000}%
\pgfsetlinewidth{0.000000pt}%
\definecolor{currentstroke}{rgb}{0.000000,0.000000,0.000000}%
\pgfsetstrokecolor{currentstroke}%
\pgfsetdash{}{0pt}%
\pgfpathmoveto{\pgfqpoint{5.817952in}{2.527724in}}%
\pgfpathlineto{\pgfqpoint{5.831956in}{2.526496in}}%
\pgfpathlineto{\pgfqpoint{5.845969in}{2.525292in}}%
\pgfpathlineto{\pgfqpoint{5.859990in}{2.524112in}}%
\pgfpathlineto{\pgfqpoint{5.874020in}{2.522956in}}%
\pgfpathlineto{\pgfqpoint{5.866948in}{2.517895in}}%
\pgfpathlineto{\pgfqpoint{5.859870in}{2.512817in}}%
\pgfpathlineto{\pgfqpoint{5.852786in}{2.507719in}}%
\pgfpathlineto{\pgfqpoint{5.845695in}{2.502597in}}%
\pgfpathlineto{\pgfqpoint{5.831646in}{2.503632in}}%
\pgfpathlineto{\pgfqpoint{5.817605in}{2.504692in}}%
\pgfpathlineto{\pgfqpoint{5.803572in}{2.505775in}}%
\pgfpathlineto{\pgfqpoint{5.789548in}{2.506883in}}%
\pgfpathlineto{\pgfqpoint{5.796659in}{2.512120in}}%
\pgfpathlineto{\pgfqpoint{5.803763in}{2.517337in}}%
\pgfpathlineto{\pgfqpoint{5.810860in}{2.522537in}}%
\pgfpathlineto{\pgfqpoint{5.817952in}{2.527724in}}%
\pgfpathclose%
\pgfusepath{fill}%
\end{pgfscope}%
\begin{pgfscope}%
\pgfpathrectangle{\pgfqpoint{1.150000in}{0.150000in}}{\pgfqpoint{5.700000in}{5.700000in}}%
\pgfusepath{clip}%
\pgfsetbuttcap%
\pgfsetroundjoin%
\definecolor{currentfill}{rgb}{0.271305,0.019942,0.347269}%
\pgfsetfillcolor{currentfill}%
\pgfsetfillopacity{0.700000}%
\pgfsetlinewidth{0.000000pt}%
\definecolor{currentstroke}{rgb}{0.000000,0.000000,0.000000}%
\pgfsetstrokecolor{currentstroke}%
\pgfsetdash{}{0pt}%
\pgfpathmoveto{\pgfqpoint{3.291032in}{2.281664in}}%
\pgfpathlineto{\pgfqpoint{3.304411in}{2.276222in}}%
\pgfpathlineto{\pgfqpoint{3.317795in}{2.270817in}}%
\pgfpathlineto{\pgfqpoint{3.331184in}{2.265447in}}%
\pgfpathlineto{\pgfqpoint{3.344577in}{2.260112in}}%
\pgfpathlineto{\pgfqpoint{3.336455in}{2.254648in}}%
\pgfpathlineto{\pgfqpoint{3.328325in}{2.249293in}}%
\pgfpathlineto{\pgfqpoint{3.320187in}{2.244051in}}%
\pgfpathlineto{\pgfqpoint{3.312040in}{2.238925in}}%
\pgfpathlineto{\pgfqpoint{3.298629in}{2.244434in}}%
\pgfpathlineto{\pgfqpoint{3.285222in}{2.249979in}}%
\pgfpathlineto{\pgfqpoint{3.271820in}{2.255559in}}%
\pgfpathlineto{\pgfqpoint{3.258422in}{2.261174in}}%
\pgfpathlineto{\pgfqpoint{3.266588in}{2.266121in}}%
\pgfpathlineto{\pgfqpoint{3.274745in}{2.271188in}}%
\pgfpathlineto{\pgfqpoint{3.282893in}{2.276370in}}%
\pgfpathlineto{\pgfqpoint{3.291032in}{2.281664in}}%
\pgfpathclose%
\pgfusepath{fill}%
\end{pgfscope}%
\begin{pgfscope}%
\pgfpathrectangle{\pgfqpoint{1.150000in}{0.150000in}}{\pgfqpoint{5.700000in}{5.700000in}}%
\pgfusepath{clip}%
\pgfsetbuttcap%
\pgfsetroundjoin%
\definecolor{currentfill}{rgb}{0.267004,0.004874,0.329415}%
\pgfsetfillcolor{currentfill}%
\pgfsetfillopacity{0.700000}%
\pgfsetlinewidth{0.000000pt}%
\definecolor{currentstroke}{rgb}{0.000000,0.000000,0.000000}%
\pgfsetstrokecolor{currentstroke}%
\pgfsetdash{}{0pt}%
\pgfpathmoveto{\pgfqpoint{3.569966in}{2.249574in}}%
\pgfpathlineto{\pgfqpoint{3.583391in}{2.244984in}}%
\pgfpathlineto{\pgfqpoint{3.596822in}{2.240426in}}%
\pgfpathlineto{\pgfqpoint{3.610258in}{2.235900in}}%
\pgfpathlineto{\pgfqpoint{3.623699in}{2.231406in}}%
\pgfpathlineto{\pgfqpoint{3.615703in}{2.224648in}}%
\pgfpathlineto{\pgfqpoint{3.607700in}{2.217954in}}%
\pgfpathlineto{\pgfqpoint{3.599690in}{2.211328in}}%
\pgfpathlineto{\pgfqpoint{3.591674in}{2.204773in}}%
\pgfpathlineto{\pgfqpoint{3.578217in}{2.209415in}}%
\pgfpathlineto{\pgfqpoint{3.564765in}{2.214088in}}%
\pgfpathlineto{\pgfqpoint{3.551319in}{2.218794in}}%
\pgfpathlineto{\pgfqpoint{3.537878in}{2.223532in}}%
\pgfpathlineto{\pgfqpoint{3.545910in}{2.229933in}}%
\pgfpathlineto{\pgfqpoint{3.553936in}{2.236410in}}%
\pgfpathlineto{\pgfqpoint{3.561954in}{2.242958in}}%
\pgfpathlineto{\pgfqpoint{3.569966in}{2.249574in}}%
\pgfpathclose%
\pgfusepath{fill}%
\end{pgfscope}%
\begin{pgfscope}%
\pgfpathrectangle{\pgfqpoint{1.150000in}{0.150000in}}{\pgfqpoint{5.700000in}{5.700000in}}%
\pgfusepath{clip}%
\pgfsetbuttcap%
\pgfsetroundjoin%
\definecolor{currentfill}{rgb}{0.280894,0.078907,0.402329}%
\pgfsetfillcolor{currentfill}%
\pgfsetfillopacity{0.700000}%
\pgfsetlinewidth{0.000000pt}%
\definecolor{currentstroke}{rgb}{0.000000,0.000000,0.000000}%
\pgfsetstrokecolor{currentstroke}%
\pgfsetdash{}{0pt}%
\pgfpathmoveto{\pgfqpoint{4.833620in}{2.377071in}}%
\pgfpathlineto{\pgfqpoint{4.847350in}{2.375150in}}%
\pgfpathlineto{\pgfqpoint{4.861086in}{2.373255in}}%
\pgfpathlineto{\pgfqpoint{4.874831in}{2.371385in}}%
\pgfpathlineto{\pgfqpoint{4.888583in}{2.369542in}}%
\pgfpathlineto{\pgfqpoint{4.881063in}{2.362048in}}%
\pgfpathlineto{\pgfqpoint{4.873537in}{2.354501in}}%
\pgfpathlineto{\pgfqpoint{4.866004in}{2.346899in}}%
\pgfpathlineto{\pgfqpoint{4.858465in}{2.339242in}}%
\pgfpathlineto{\pgfqpoint{4.844701in}{2.341087in}}%
\pgfpathlineto{\pgfqpoint{4.830944in}{2.342958in}}%
\pgfpathlineto{\pgfqpoint{4.817194in}{2.344854in}}%
\pgfpathlineto{\pgfqpoint{4.803452in}{2.346777in}}%
\pgfpathlineto{\pgfqpoint{4.811003in}{2.354427in}}%
\pgfpathlineto{\pgfqpoint{4.818549in}{2.362027in}}%
\pgfpathlineto{\pgfqpoint{4.826088in}{2.369574in}}%
\pgfpathlineto{\pgfqpoint{4.833620in}{2.377071in}}%
\pgfpathclose%
\pgfusepath{fill}%
\end{pgfscope}%
\begin{pgfscope}%
\pgfpathrectangle{\pgfqpoint{1.150000in}{0.150000in}}{\pgfqpoint{5.700000in}{5.700000in}}%
\pgfusepath{clip}%
\pgfsetbuttcap%
\pgfsetroundjoin%
\definecolor{currentfill}{rgb}{0.267004,0.004874,0.329415}%
\pgfsetfillcolor{currentfill}%
\pgfsetfillopacity{0.700000}%
\pgfsetlinewidth{0.000000pt}%
\definecolor{currentstroke}{rgb}{0.000000,0.000000,0.000000}%
\pgfsetstrokecolor{currentstroke}%
\pgfsetdash{}{0pt}%
\pgfpathmoveto{\pgfqpoint{3.709379in}{2.241895in}}%
\pgfpathlineto{\pgfqpoint{3.722834in}{2.237692in}}%
\pgfpathlineto{\pgfqpoint{3.736294in}{2.233521in}}%
\pgfpathlineto{\pgfqpoint{3.749759in}{2.229380in}}%
\pgfpathlineto{\pgfqpoint{3.763231in}{2.225269in}}%
\pgfpathlineto{\pgfqpoint{3.755290in}{2.218027in}}%
\pgfpathlineto{\pgfqpoint{3.747343in}{2.210830in}}%
\pgfpathlineto{\pgfqpoint{3.739389in}{2.203681in}}%
\pgfpathlineto{\pgfqpoint{3.731430in}{2.196582in}}%
\pgfpathlineto{\pgfqpoint{3.717944in}{2.200827in}}%
\pgfpathlineto{\pgfqpoint{3.704464in}{2.205103in}}%
\pgfpathlineto{\pgfqpoint{3.690989in}{2.209409in}}%
\pgfpathlineto{\pgfqpoint{3.677520in}{2.213746in}}%
\pgfpathlineto{\pgfqpoint{3.685495in}{2.220705in}}%
\pgfpathlineto{\pgfqpoint{3.693463in}{2.227718in}}%
\pgfpathlineto{\pgfqpoint{3.701424in}{2.234782in}}%
\pgfpathlineto{\pgfqpoint{3.709379in}{2.241895in}}%
\pgfpathclose%
\pgfusepath{fill}%
\end{pgfscope}%
\begin{pgfscope}%
\pgfpathrectangle{\pgfqpoint{1.150000in}{0.150000in}}{\pgfqpoint{5.700000in}{5.700000in}}%
\pgfusepath{clip}%
\pgfsetbuttcap%
\pgfsetroundjoin%
\definecolor{currentfill}{rgb}{0.274952,0.037752,0.364543}%
\pgfsetfillcolor{currentfill}%
\pgfsetfillopacity{0.700000}%
\pgfsetlinewidth{0.000000pt}%
\definecolor{currentstroke}{rgb}{0.000000,0.000000,0.000000}%
\pgfsetstrokecolor{currentstroke}%
\pgfsetdash{}{0pt}%
\pgfpathmoveto{\pgfqpoint{3.151400in}{2.307411in}}%
\pgfpathlineto{\pgfqpoint{3.164763in}{2.301501in}}%
\pgfpathlineto{\pgfqpoint{3.178130in}{2.295629in}}%
\pgfpathlineto{\pgfqpoint{3.191501in}{2.289795in}}%
\pgfpathlineto{\pgfqpoint{3.204876in}{2.283998in}}%
\pgfpathlineto{\pgfqpoint{3.196683in}{2.279358in}}%
\pgfpathlineto{\pgfqpoint{3.188479in}{2.274852in}}%
\pgfpathlineto{\pgfqpoint{3.180267in}{2.270482in}}%
\pgfpathlineto{\pgfqpoint{3.172045in}{2.266254in}}%
\pgfpathlineto{\pgfqpoint{3.158649in}{2.272239in}}%
\pgfpathlineto{\pgfqpoint{3.145258in}{2.278262in}}%
\pgfpathlineto{\pgfqpoint{3.131871in}{2.284322in}}%
\pgfpathlineto{\pgfqpoint{3.118488in}{2.290419in}}%
\pgfpathlineto{\pgfqpoint{3.126731in}{2.294454in}}%
\pgfpathlineto{\pgfqpoint{3.134964in}{2.298634in}}%
\pgfpathlineto{\pgfqpoint{3.143187in}{2.302954in}}%
\pgfpathlineto{\pgfqpoint{3.151400in}{2.307411in}}%
\pgfpathclose%
\pgfusepath{fill}%
\end{pgfscope}%
\begin{pgfscope}%
\pgfpathrectangle{\pgfqpoint{1.150000in}{0.150000in}}{\pgfqpoint{5.700000in}{5.700000in}}%
\pgfusepath{clip}%
\pgfsetbuttcap%
\pgfsetroundjoin%
\definecolor{currentfill}{rgb}{0.282656,0.100196,0.422160}%
\pgfsetfillcolor{currentfill}%
\pgfsetfillopacity{0.700000}%
\pgfsetlinewidth{0.000000pt}%
\definecolor{currentstroke}{rgb}{0.000000,0.000000,0.000000}%
\pgfsetstrokecolor{currentstroke}%
\pgfsetdash{}{0pt}%
\pgfpathmoveto{\pgfqpoint{5.058593in}{2.413492in}}%
\pgfpathlineto{\pgfqpoint{5.072388in}{2.411839in}}%
\pgfpathlineto{\pgfqpoint{5.086190in}{2.410210in}}%
\pgfpathlineto{\pgfqpoint{5.100000in}{2.408607in}}%
\pgfpathlineto{\pgfqpoint{5.113819in}{2.407029in}}%
\pgfpathlineto{\pgfqpoint{5.106391in}{2.400031in}}%
\pgfpathlineto{\pgfqpoint{5.098956in}{2.392977in}}%
\pgfpathlineto{\pgfqpoint{5.091515in}{2.385864in}}%
\pgfpathlineto{\pgfqpoint{5.084067in}{2.378692in}}%
\pgfpathlineto{\pgfqpoint{5.070235in}{2.380245in}}%
\pgfpathlineto{\pgfqpoint{5.056411in}{2.381822in}}%
\pgfpathlineto{\pgfqpoint{5.042595in}{2.383425in}}%
\pgfpathlineto{\pgfqpoint{5.028787in}{2.385052in}}%
\pgfpathlineto{\pgfqpoint{5.036248in}{2.392245in}}%
\pgfpathlineto{\pgfqpoint{5.043703in}{2.399381in}}%
\pgfpathlineto{\pgfqpoint{5.051151in}{2.406463in}}%
\pgfpathlineto{\pgfqpoint{5.058593in}{2.413492in}}%
\pgfpathclose%
\pgfusepath{fill}%
\end{pgfscope}%
\begin{pgfscope}%
\pgfpathrectangle{\pgfqpoint{1.150000in}{0.150000in}}{\pgfqpoint{5.700000in}{5.700000in}}%
\pgfusepath{clip}%
\pgfsetbuttcap%
\pgfsetroundjoin%
\definecolor{currentfill}{rgb}{0.272594,0.025563,0.353093}%
\pgfsetfillcolor{currentfill}%
\pgfsetfillopacity{0.700000}%
\pgfsetlinewidth{0.000000pt}%
\definecolor{currentstroke}{rgb}{0.000000,0.000000,0.000000}%
\pgfsetstrokecolor{currentstroke}%
\pgfsetdash{}{0pt}%
\pgfpathmoveto{\pgfqpoint{4.298644in}{2.285637in}}%
\pgfpathlineto{\pgfqpoint{4.312233in}{2.282835in}}%
\pgfpathlineto{\pgfqpoint{4.325830in}{2.280060in}}%
\pgfpathlineto{\pgfqpoint{4.339433in}{2.277313in}}%
\pgfpathlineto{\pgfqpoint{4.353042in}{2.274593in}}%
\pgfpathlineto{\pgfqpoint{4.345319in}{2.266458in}}%
\pgfpathlineto{\pgfqpoint{4.337589in}{2.258298in}}%
\pgfpathlineto{\pgfqpoint{4.329854in}{2.250116in}}%
\pgfpathlineto{\pgfqpoint{4.322114in}{2.241911in}}%
\pgfpathlineto{\pgfqpoint{4.308492in}{2.244700in}}%
\pgfpathlineto{\pgfqpoint{4.294877in}{2.247515in}}%
\pgfpathlineto{\pgfqpoint{4.281269in}{2.250358in}}%
\pgfpathlineto{\pgfqpoint{4.267668in}{2.253229in}}%
\pgfpathlineto{\pgfqpoint{4.275420in}{2.261360in}}%
\pgfpathlineto{\pgfqpoint{4.283167in}{2.269473in}}%
\pgfpathlineto{\pgfqpoint{4.290908in}{2.277566in}}%
\pgfpathlineto{\pgfqpoint{4.298644in}{2.285637in}}%
\pgfpathclose%
\pgfusepath{fill}%
\end{pgfscope}%
\begin{pgfscope}%
\pgfpathrectangle{\pgfqpoint{1.150000in}{0.150000in}}{\pgfqpoint{5.700000in}{5.700000in}}%
\pgfusepath{clip}%
\pgfsetbuttcap%
\pgfsetroundjoin%
\definecolor{currentfill}{rgb}{0.268510,0.009605,0.335427}%
\pgfsetfillcolor{currentfill}%
\pgfsetfillopacity{0.700000}%
\pgfsetlinewidth{0.000000pt}%
\definecolor{currentstroke}{rgb}{0.000000,0.000000,0.000000}%
\pgfsetstrokecolor{currentstroke}%
\pgfsetdash{}{0pt}%
\pgfpathmoveto{\pgfqpoint{4.073757in}{2.258133in}}%
\pgfpathlineto{\pgfqpoint{4.087292in}{2.254847in}}%
\pgfpathlineto{\pgfqpoint{4.100833in}{2.251589in}}%
\pgfpathlineto{\pgfqpoint{4.114380in}{2.248359in}}%
\pgfpathlineto{\pgfqpoint{4.127935in}{2.245158in}}%
\pgfpathlineto{\pgfqpoint{4.120130in}{2.237137in}}%
\pgfpathlineto{\pgfqpoint{4.112319in}{2.229115in}}%
\pgfpathlineto{\pgfqpoint{4.104503in}{2.221093in}}%
\pgfpathlineto{\pgfqpoint{4.096682in}{2.213073in}}%
\pgfpathlineto{\pgfqpoint{4.083115in}{2.216369in}}%
\pgfpathlineto{\pgfqpoint{4.069555in}{2.219694in}}%
\pgfpathlineto{\pgfqpoint{4.056002in}{2.223046in}}%
\pgfpathlineto{\pgfqpoint{4.042454in}{2.226428in}}%
\pgfpathlineto{\pgfqpoint{4.050288in}{2.234348in}}%
\pgfpathlineto{\pgfqpoint{4.058117in}{2.242273in}}%
\pgfpathlineto{\pgfqpoint{4.065940in}{2.250203in}}%
\pgfpathlineto{\pgfqpoint{4.073757in}{2.258133in}}%
\pgfpathclose%
\pgfusepath{fill}%
\end{pgfscope}%
\begin{pgfscope}%
\pgfpathrectangle{\pgfqpoint{1.150000in}{0.150000in}}{\pgfqpoint{5.700000in}{5.700000in}}%
\pgfusepath{clip}%
\pgfsetbuttcap%
\pgfsetroundjoin%
\definecolor{currentfill}{rgb}{0.277018,0.050344,0.375715}%
\pgfsetfillcolor{currentfill}%
\pgfsetfillopacity{0.700000}%
\pgfsetlinewidth{0.000000pt}%
\definecolor{currentstroke}{rgb}{0.000000,0.000000,0.000000}%
\pgfsetstrokecolor{currentstroke}%
\pgfsetdash{}{0pt}%
\pgfpathmoveto{\pgfqpoint{4.523565in}{2.318647in}}%
\pgfpathlineto{\pgfqpoint{4.537214in}{2.316264in}}%
\pgfpathlineto{\pgfqpoint{4.550870in}{2.313908in}}%
\pgfpathlineto{\pgfqpoint{4.564533in}{2.311579in}}%
\pgfpathlineto{\pgfqpoint{4.578204in}{2.309276in}}%
\pgfpathlineto{\pgfqpoint{4.570561in}{2.301262in}}%
\pgfpathlineto{\pgfqpoint{4.562913in}{2.293207in}}%
\pgfpathlineto{\pgfqpoint{4.555259in}{2.285111in}}%
\pgfpathlineto{\pgfqpoint{4.547600in}{2.276974in}}%
\pgfpathlineto{\pgfqpoint{4.533917in}{2.279319in}}%
\pgfpathlineto{\pgfqpoint{4.520242in}{2.281690in}}%
\pgfpathlineto{\pgfqpoint{4.506575in}{2.284087in}}%
\pgfpathlineto{\pgfqpoint{4.492914in}{2.286511in}}%
\pgfpathlineto{\pgfqpoint{4.500585in}{2.294601in}}%
\pgfpathlineto{\pgfqpoint{4.508251in}{2.302654in}}%
\pgfpathlineto{\pgfqpoint{4.515911in}{2.310670in}}%
\pgfpathlineto{\pgfqpoint{4.523565in}{2.318647in}}%
\pgfpathclose%
\pgfusepath{fill}%
\end{pgfscope}%
\begin{pgfscope}%
\pgfpathrectangle{\pgfqpoint{1.150000in}{0.150000in}}{\pgfqpoint{5.700000in}{5.700000in}}%
\pgfusepath{clip}%
\pgfsetbuttcap%
\pgfsetroundjoin%
\definecolor{currentfill}{rgb}{0.282656,0.100196,0.422160}%
\pgfsetfillcolor{currentfill}%
\pgfsetfillopacity{0.700000}%
\pgfsetlinewidth{0.000000pt}%
\definecolor{currentstroke}{rgb}{0.000000,0.000000,0.000000}%
\pgfsetstrokecolor{currentstroke}%
\pgfsetdash{}{0pt}%
\pgfpathmoveto{\pgfqpoint{2.818119in}{2.410223in}}%
\pgfpathlineto{\pgfqpoint{2.831451in}{2.403101in}}%
\pgfpathlineto{\pgfqpoint{2.844786in}{2.396023in}}%
\pgfpathlineto{\pgfqpoint{2.858124in}{2.388990in}}%
\pgfpathlineto{\pgfqpoint{2.871465in}{2.382000in}}%
\pgfpathlineto{\pgfqpoint{2.863078in}{2.379607in}}%
\pgfpathlineto{\pgfqpoint{2.854678in}{2.377403in}}%
\pgfpathlineto{\pgfqpoint{2.846266in}{2.375394in}}%
\pgfpathlineto{\pgfqpoint{2.837841in}{2.373585in}}%
\pgfpathlineto{\pgfqpoint{2.824475in}{2.380790in}}%
\pgfpathlineto{\pgfqpoint{2.811112in}{2.388040in}}%
\pgfpathlineto{\pgfqpoint{2.797752in}{2.395334in}}%
\pgfpathlineto{\pgfqpoint{2.784395in}{2.402672in}}%
\pgfpathlineto{\pgfqpoint{2.792846in}{2.404260in}}%
\pgfpathlineto{\pgfqpoint{2.801283in}{2.406051in}}%
\pgfpathlineto{\pgfqpoint{2.809707in}{2.408041in}}%
\pgfpathlineto{\pgfqpoint{2.818119in}{2.410223in}}%
\pgfpathclose%
\pgfusepath{fill}%
\end{pgfscope}%
\begin{pgfscope}%
\pgfpathrectangle{\pgfqpoint{1.150000in}{0.150000in}}{\pgfqpoint{5.700000in}{5.700000in}}%
\pgfusepath{clip}%
\pgfsetbuttcap%
\pgfsetroundjoin%
\definecolor{currentfill}{rgb}{0.283229,0.120777,0.440584}%
\pgfsetfillcolor{currentfill}%
\pgfsetfillopacity{0.700000}%
\pgfsetlinewidth{0.000000pt}%
\definecolor{currentstroke}{rgb}{0.000000,0.000000,0.000000}%
\pgfsetstrokecolor{currentstroke}%
\pgfsetdash{}{0pt}%
\pgfpathmoveto{\pgfqpoint{5.283601in}{2.448703in}}%
\pgfpathlineto{\pgfqpoint{5.297462in}{2.447258in}}%
\pgfpathlineto{\pgfqpoint{5.311330in}{2.445837in}}%
\pgfpathlineto{\pgfqpoint{5.325207in}{2.444441in}}%
\pgfpathlineto{\pgfqpoint{5.339092in}{2.443070in}}%
\pgfpathlineto{\pgfqpoint{5.331762in}{2.436640in}}%
\pgfpathlineto{\pgfqpoint{5.324426in}{2.430158in}}%
\pgfpathlineto{\pgfqpoint{5.317082in}{2.423620in}}%
\pgfpathlineto{\pgfqpoint{5.309732in}{2.417025in}}%
\pgfpathlineto{\pgfqpoint{5.295832in}{2.418344in}}%
\pgfpathlineto{\pgfqpoint{5.281940in}{2.419687in}}%
\pgfpathlineto{\pgfqpoint{5.268056in}{2.421054in}}%
\pgfpathlineto{\pgfqpoint{5.254180in}{2.422447in}}%
\pgfpathlineto{\pgfqpoint{5.261546in}{2.429090in}}%
\pgfpathlineto{\pgfqpoint{5.268904in}{2.435679in}}%
\pgfpathlineto{\pgfqpoint{5.276256in}{2.442216in}}%
\pgfpathlineto{\pgfqpoint{5.283601in}{2.448703in}}%
\pgfpathclose%
\pgfusepath{fill}%
\end{pgfscope}%
\begin{pgfscope}%
\pgfpathrectangle{\pgfqpoint{1.150000in}{0.150000in}}{\pgfqpoint{5.700000in}{5.700000in}}%
\pgfusepath{clip}%
\pgfsetbuttcap%
\pgfsetroundjoin%
\definecolor{currentfill}{rgb}{0.267004,0.004874,0.329415}%
\pgfsetfillcolor{currentfill}%
\pgfsetfillopacity{0.700000}%
\pgfsetlinewidth{0.000000pt}%
\definecolor{currentstroke}{rgb}{0.000000,0.000000,0.000000}%
\pgfsetstrokecolor{currentstroke}%
\pgfsetdash{}{0pt}%
\pgfpathmoveto{\pgfqpoint{3.848820in}{2.238973in}}%
\pgfpathlineto{\pgfqpoint{3.862307in}{2.235134in}}%
\pgfpathlineto{\pgfqpoint{3.875800in}{2.231326in}}%
\pgfpathlineto{\pgfqpoint{3.889299in}{2.227547in}}%
\pgfpathlineto{\pgfqpoint{3.902803in}{2.223797in}}%
\pgfpathlineto{\pgfqpoint{3.894914in}{2.216172in}}%
\pgfpathlineto{\pgfqpoint{3.887019in}{2.208574in}}%
\pgfpathlineto{\pgfqpoint{3.879118in}{2.201006in}}%
\pgfpathlineto{\pgfqpoint{3.871211in}{2.193470in}}%
\pgfpathlineto{\pgfqpoint{3.857693in}{2.197340in}}%
\pgfpathlineto{\pgfqpoint{3.844181in}{2.201241in}}%
\pgfpathlineto{\pgfqpoint{3.830675in}{2.205171in}}%
\pgfpathlineto{\pgfqpoint{3.817174in}{2.209130in}}%
\pgfpathlineto{\pgfqpoint{3.825095in}{2.216540in}}%
\pgfpathlineto{\pgfqpoint{3.833009in}{2.223986in}}%
\pgfpathlineto{\pgfqpoint{3.840918in}{2.231464in}}%
\pgfpathlineto{\pgfqpoint{3.848820in}{2.238973in}}%
\pgfpathclose%
\pgfusepath{fill}%
\end{pgfscope}%
\begin{pgfscope}%
\pgfpathrectangle{\pgfqpoint{1.150000in}{0.150000in}}{\pgfqpoint{5.700000in}{5.700000in}}%
\pgfusepath{clip}%
\pgfsetbuttcap%
\pgfsetroundjoin%
\definecolor{currentfill}{rgb}{0.282623,0.140926,0.457517}%
\pgfsetfillcolor{currentfill}%
\pgfsetfillopacity{0.700000}%
\pgfsetlinewidth{0.000000pt}%
\definecolor{currentstroke}{rgb}{0.000000,0.000000,0.000000}%
\pgfsetstrokecolor{currentstroke}%
\pgfsetdash{}{0pt}%
\pgfpathmoveto{\pgfqpoint{5.508600in}{2.481622in}}%
\pgfpathlineto{\pgfqpoint{5.522526in}{2.480326in}}%
\pgfpathlineto{\pgfqpoint{5.536460in}{2.479054in}}%
\pgfpathlineto{\pgfqpoint{5.550402in}{2.477807in}}%
\pgfpathlineto{\pgfqpoint{5.564353in}{2.476585in}}%
\pgfpathlineto{\pgfqpoint{5.557128in}{2.470750in}}%
\pgfpathlineto{\pgfqpoint{5.549895in}{2.464871in}}%
\pgfpathlineto{\pgfqpoint{5.542656in}{2.458946in}}%
\pgfpathlineto{\pgfqpoint{5.535409in}{2.452973in}}%
\pgfpathlineto{\pgfqpoint{5.521441in}{2.454115in}}%
\pgfpathlineto{\pgfqpoint{5.507482in}{2.455282in}}%
\pgfpathlineto{\pgfqpoint{5.493531in}{2.456474in}}%
\pgfpathlineto{\pgfqpoint{5.479588in}{2.457690in}}%
\pgfpathlineto{\pgfqpoint{5.486851in}{2.463738in}}%
\pgfpathlineto{\pgfqpoint{5.494107in}{2.469742in}}%
\pgfpathlineto{\pgfqpoint{5.501357in}{2.475702in}}%
\pgfpathlineto{\pgfqpoint{5.508600in}{2.481622in}}%
\pgfpathclose%
\pgfusepath{fill}%
\end{pgfscope}%
\begin{pgfscope}%
\pgfpathrectangle{\pgfqpoint{1.150000in}{0.150000in}}{\pgfqpoint{5.700000in}{5.700000in}}%
\pgfusepath{clip}%
\pgfsetbuttcap%
\pgfsetroundjoin%
\definecolor{currentfill}{rgb}{0.282290,0.145912,0.461510}%
\pgfsetfillcolor{currentfill}%
\pgfsetfillopacity{0.700000}%
\pgfsetlinewidth{0.000000pt}%
\definecolor{currentstroke}{rgb}{0.000000,0.000000,0.000000}%
\pgfsetstrokecolor{currentstroke}%
\pgfsetdash{}{0pt}%
\pgfpathmoveto{\pgfqpoint{2.624337in}{2.494380in}}%
\pgfpathlineto{\pgfqpoint{2.637661in}{2.486471in}}%
\pgfpathlineto{\pgfqpoint{2.650987in}{2.478611in}}%
\pgfpathlineto{\pgfqpoint{2.664315in}{2.470801in}}%
\pgfpathlineto{\pgfqpoint{2.677647in}{2.463041in}}%
\pgfpathlineto{\pgfqpoint{2.669130in}{2.462113in}}%
\pgfpathlineto{\pgfqpoint{2.660599in}{2.461408in}}%
\pgfpathlineto{\pgfqpoint{2.652053in}{2.460930in}}%
\pgfpathlineto{\pgfqpoint{2.643492in}{2.460687in}}%
\pgfpathlineto{\pgfqpoint{2.630133in}{2.468678in}}%
\pgfpathlineto{\pgfqpoint{2.616777in}{2.476719in}}%
\pgfpathlineto{\pgfqpoint{2.603423in}{2.484810in}}%
\pgfpathlineto{\pgfqpoint{2.590072in}{2.492951in}}%
\pgfpathlineto{\pgfqpoint{2.598661in}{2.492958in}}%
\pgfpathlineto{\pgfqpoint{2.607235in}{2.493202in}}%
\pgfpathlineto{\pgfqpoint{2.615793in}{2.493679in}}%
\pgfpathlineto{\pgfqpoint{2.624337in}{2.494380in}}%
\pgfpathclose%
\pgfusepath{fill}%
\end{pgfscope}%
\begin{pgfscope}%
\pgfpathrectangle{\pgfqpoint{1.150000in}{0.150000in}}{\pgfqpoint{5.700000in}{5.700000in}}%
\pgfusepath{clip}%
\pgfsetbuttcap%
\pgfsetroundjoin%
\definecolor{currentfill}{rgb}{0.279574,0.170599,0.479997}%
\pgfsetfillcolor{currentfill}%
\pgfsetfillopacity{0.700000}%
\pgfsetlinewidth{0.000000pt}%
\definecolor{currentstroke}{rgb}{0.000000,0.000000,0.000000}%
\pgfsetstrokecolor{currentstroke}%
\pgfsetdash{}{0pt}%
\pgfpathmoveto{\pgfqpoint{5.958367in}{2.538200in}}%
\pgfpathlineto{\pgfqpoint{5.972418in}{2.537030in}}%
\pgfpathlineto{\pgfqpoint{5.986479in}{2.535885in}}%
\pgfpathlineto{\pgfqpoint{6.000548in}{2.534763in}}%
\pgfpathlineto{\pgfqpoint{6.014625in}{2.533665in}}%
\pgfpathlineto{\pgfqpoint{6.007620in}{2.528906in}}%
\pgfpathlineto{\pgfqpoint{6.000609in}{2.524141in}}%
\pgfpathlineto{\pgfqpoint{5.993591in}{2.519366in}}%
\pgfpathlineto{\pgfqpoint{5.986567in}{2.514575in}}%
\pgfpathlineto{\pgfqpoint{5.972468in}{2.515539in}}%
\pgfpathlineto{\pgfqpoint{5.958378in}{2.516526in}}%
\pgfpathlineto{\pgfqpoint{5.944297in}{2.517538in}}%
\pgfpathlineto{\pgfqpoint{5.930224in}{2.518573in}}%
\pgfpathlineto{\pgfqpoint{5.937269in}{2.523493in}}%
\pgfpathlineto{\pgfqpoint{5.944307in}{2.528402in}}%
\pgfpathlineto{\pgfqpoint{5.951340in}{2.533302in}}%
\pgfpathlineto{\pgfqpoint{5.958367in}{2.538200in}}%
\pgfpathclose%
\pgfusepath{fill}%
\end{pgfscope}%
\begin{pgfscope}%
\pgfpathrectangle{\pgfqpoint{1.150000in}{0.150000in}}{\pgfqpoint{5.700000in}{5.700000in}}%
\pgfusepath{clip}%
\pgfsetbuttcap%
\pgfsetroundjoin%
\definecolor{currentfill}{rgb}{0.280267,0.073417,0.397163}%
\pgfsetfillcolor{currentfill}%
\pgfsetfillopacity{0.700000}%
\pgfsetlinewidth{0.000000pt}%
\definecolor{currentstroke}{rgb}{0.000000,0.000000,0.000000}%
\pgfsetstrokecolor{currentstroke}%
\pgfsetdash{}{0pt}%
\pgfpathmoveto{\pgfqpoint{4.748559in}{2.354723in}}%
\pgfpathlineto{\pgfqpoint{4.762271in}{2.352697in}}%
\pgfpathlineto{\pgfqpoint{4.775991in}{2.350698in}}%
\pgfpathlineto{\pgfqpoint{4.789718in}{2.348724in}}%
\pgfpathlineto{\pgfqpoint{4.803452in}{2.346777in}}%
\pgfpathlineto{\pgfqpoint{4.795895in}{2.339073in}}%
\pgfpathlineto{\pgfqpoint{4.788331in}{2.331318in}}%
\pgfpathlineto{\pgfqpoint{4.780761in}{2.323509in}}%
\pgfpathlineto{\pgfqpoint{4.773185in}{2.315648in}}%
\pgfpathlineto{\pgfqpoint{4.759438in}{2.317610in}}%
\pgfpathlineto{\pgfqpoint{4.745699in}{2.319599in}}%
\pgfpathlineto{\pgfqpoint{4.731967in}{2.321613in}}%
\pgfpathlineto{\pgfqpoint{4.718243in}{2.323653in}}%
\pgfpathlineto{\pgfqpoint{4.725831in}{2.331494in}}%
\pgfpathlineto{\pgfqpoint{4.733413in}{2.339286in}}%
\pgfpathlineto{\pgfqpoint{4.740989in}{2.347029in}}%
\pgfpathlineto{\pgfqpoint{4.748559in}{2.354723in}}%
\pgfpathclose%
\pgfusepath{fill}%
\end{pgfscope}%
\begin{pgfscope}%
\pgfpathrectangle{\pgfqpoint{1.150000in}{0.150000in}}{\pgfqpoint{5.700000in}{5.700000in}}%
\pgfusepath{clip}%
\pgfsetbuttcap%
\pgfsetroundjoin%
\definecolor{currentfill}{rgb}{0.281412,0.155834,0.469201}%
\pgfsetfillcolor{currentfill}%
\pgfsetfillopacity{0.700000}%
\pgfsetlinewidth{0.000000pt}%
\definecolor{currentstroke}{rgb}{0.000000,0.000000,0.000000}%
\pgfsetstrokecolor{currentstroke}%
\pgfsetdash{}{0pt}%
\pgfpathmoveto{\pgfqpoint{5.733538in}{2.511555in}}%
\pgfpathlineto{\pgfqpoint{5.747528in}{2.510350in}}%
\pgfpathlineto{\pgfqpoint{5.761526in}{2.509170in}}%
\pgfpathlineto{\pgfqpoint{5.775533in}{2.508014in}}%
\pgfpathlineto{\pgfqpoint{5.789548in}{2.506883in}}%
\pgfpathlineto{\pgfqpoint{5.782432in}{2.501621in}}%
\pgfpathlineto{\pgfqpoint{5.775308in}{2.496331in}}%
\pgfpathlineto{\pgfqpoint{5.768178in}{2.491009in}}%
\pgfpathlineto{\pgfqpoint{5.761041in}{2.485653in}}%
\pgfpathlineto{\pgfqpoint{5.747006in}{2.486678in}}%
\pgfpathlineto{\pgfqpoint{5.732981in}{2.487726in}}%
\pgfpathlineto{\pgfqpoint{5.718963in}{2.488799in}}%
\pgfpathlineto{\pgfqpoint{5.704955in}{2.489896in}}%
\pgfpathlineto{\pgfqpoint{5.712110in}{2.495355in}}%
\pgfpathlineto{\pgfqpoint{5.719259in}{2.500782in}}%
\pgfpathlineto{\pgfqpoint{5.726402in}{2.506181in}}%
\pgfpathlineto{\pgfqpoint{5.733538in}{2.511555in}}%
\pgfpathclose%
\pgfusepath{fill}%
\end{pgfscope}%
\begin{pgfscope}%
\pgfpathrectangle{\pgfqpoint{1.150000in}{0.150000in}}{\pgfqpoint{5.700000in}{5.700000in}}%
\pgfusepath{clip}%
\pgfsetbuttcap%
\pgfsetroundjoin%
\definecolor{currentfill}{rgb}{0.278791,0.062145,0.386592}%
\pgfsetfillcolor{currentfill}%
\pgfsetfillopacity{0.700000}%
\pgfsetlinewidth{0.000000pt}%
\definecolor{currentstroke}{rgb}{0.000000,0.000000,0.000000}%
\pgfsetstrokecolor{currentstroke}%
\pgfsetdash{}{0pt}%
\pgfpathmoveto{\pgfqpoint{3.011570in}{2.340593in}}%
\pgfpathlineto{\pgfqpoint{3.024921in}{2.334183in}}%
\pgfpathlineto{\pgfqpoint{3.038276in}{2.327813in}}%
\pgfpathlineto{\pgfqpoint{3.051635in}{2.321483in}}%
\pgfpathlineto{\pgfqpoint{3.064998in}{2.315192in}}%
\pgfpathlineto{\pgfqpoint{3.056724in}{2.311504in}}%
\pgfpathlineto{\pgfqpoint{3.048439in}{2.307974in}}%
\pgfpathlineto{\pgfqpoint{3.040144in}{2.304607in}}%
\pgfpathlineto{\pgfqpoint{3.031838in}{2.301408in}}%
\pgfpathlineto{\pgfqpoint{3.018453in}{2.307901in}}%
\pgfpathlineto{\pgfqpoint{3.005072in}{2.314433in}}%
\pgfpathlineto{\pgfqpoint{2.991695in}{2.321004in}}%
\pgfpathlineto{\pgfqpoint{2.978322in}{2.327616in}}%
\pgfpathlineto{\pgfqpoint{2.986651in}{2.330608in}}%
\pgfpathlineto{\pgfqpoint{2.994968in}{2.333771in}}%
\pgfpathlineto{\pgfqpoint{3.003275in}{2.337101in}}%
\pgfpathlineto{\pgfqpoint{3.011570in}{2.340593in}}%
\pgfpathclose%
\pgfusepath{fill}%
\end{pgfscope}%
\begin{pgfscope}%
\pgfpathrectangle{\pgfqpoint{1.150000in}{0.150000in}}{\pgfqpoint{5.700000in}{5.700000in}}%
\pgfusepath{clip}%
\pgfsetbuttcap%
\pgfsetroundjoin%
\definecolor{currentfill}{rgb}{0.282327,0.094955,0.417331}%
\pgfsetfillcolor{currentfill}%
\pgfsetfillopacity{0.700000}%
\pgfsetlinewidth{0.000000pt}%
\definecolor{currentstroke}{rgb}{0.000000,0.000000,0.000000}%
\pgfsetstrokecolor{currentstroke}%
\pgfsetdash{}{0pt}%
\pgfpathmoveto{\pgfqpoint{4.973631in}{2.391816in}}%
\pgfpathlineto{\pgfqpoint{4.987408in}{2.390087in}}%
\pgfpathlineto{\pgfqpoint{5.001193in}{2.388384in}}%
\pgfpathlineto{\pgfqpoint{5.014986in}{2.386705in}}%
\pgfpathlineto{\pgfqpoint{5.028787in}{2.385052in}}%
\pgfpathlineto{\pgfqpoint{5.021319in}{2.377803in}}%
\pgfpathlineto{\pgfqpoint{5.013844in}{2.370497in}}%
\pgfpathlineto{\pgfqpoint{5.006363in}{2.363131in}}%
\pgfpathlineto{\pgfqpoint{4.998876in}{2.355706in}}%
\pgfpathlineto{\pgfqpoint{4.985062in}{2.357347in}}%
\pgfpathlineto{\pgfqpoint{4.971256in}{2.359013in}}%
\pgfpathlineto{\pgfqpoint{4.957458in}{2.360704in}}%
\pgfpathlineto{\pgfqpoint{4.943667in}{2.362421in}}%
\pgfpathlineto{\pgfqpoint{4.951168in}{2.369853in}}%
\pgfpathlineto{\pgfqpoint{4.958662in}{2.377229in}}%
\pgfpathlineto{\pgfqpoint{4.966150in}{2.384550in}}%
\pgfpathlineto{\pgfqpoint{4.973631in}{2.391816in}}%
\pgfpathclose%
\pgfusepath{fill}%
\end{pgfscope}%
\begin{pgfscope}%
\pgfpathrectangle{\pgfqpoint{1.150000in}{0.150000in}}{\pgfqpoint{5.700000in}{5.700000in}}%
\pgfusepath{clip}%
\pgfsetbuttcap%
\pgfsetroundjoin%
\definecolor{currentfill}{rgb}{0.271305,0.019942,0.347269}%
\pgfsetfillcolor{currentfill}%
\pgfsetfillopacity{0.700000}%
\pgfsetlinewidth{0.000000pt}%
\definecolor{currentstroke}{rgb}{0.000000,0.000000,0.000000}%
\pgfsetstrokecolor{currentstroke}%
\pgfsetdash{}{0pt}%
\pgfpathmoveto{\pgfqpoint{4.213330in}{2.264985in}}%
\pgfpathlineto{\pgfqpoint{4.226904in}{2.262005in}}%
\pgfpathlineto{\pgfqpoint{4.240485in}{2.259052in}}%
\pgfpathlineto{\pgfqpoint{4.254073in}{2.256127in}}%
\pgfpathlineto{\pgfqpoint{4.267668in}{2.253229in}}%
\pgfpathlineto{\pgfqpoint{4.259910in}{2.245080in}}%
\pgfpathlineto{\pgfqpoint{4.252146in}{2.236917in}}%
\pgfpathlineto{\pgfqpoint{4.244377in}{2.228739in}}%
\pgfpathlineto{\pgfqpoint{4.236602in}{2.220549in}}%
\pgfpathlineto{\pgfqpoint{4.222995in}{2.223528in}}%
\pgfpathlineto{\pgfqpoint{4.209396in}{2.226535in}}%
\pgfpathlineto{\pgfqpoint{4.195802in}{2.229569in}}%
\pgfpathlineto{\pgfqpoint{4.182216in}{2.232631in}}%
\pgfpathlineto{\pgfqpoint{4.190003in}{2.240735in}}%
\pgfpathlineto{\pgfqpoint{4.197784in}{2.248829in}}%
\pgfpathlineto{\pgfqpoint{4.205560in}{2.256913in}}%
\pgfpathlineto{\pgfqpoint{4.213330in}{2.264985in}}%
\pgfpathclose%
\pgfusepath{fill}%
\end{pgfscope}%
\begin{pgfscope}%
\pgfpathrectangle{\pgfqpoint{1.150000in}{0.150000in}}{\pgfqpoint{5.700000in}{5.700000in}}%
\pgfusepath{clip}%
\pgfsetbuttcap%
\pgfsetroundjoin%
\definecolor{currentfill}{rgb}{0.268510,0.009605,0.335427}%
\pgfsetfillcolor{currentfill}%
\pgfsetfillopacity{0.700000}%
\pgfsetlinewidth{0.000000pt}%
\definecolor{currentstroke}{rgb}{0.000000,0.000000,0.000000}%
\pgfsetstrokecolor{currentstroke}%
\pgfsetdash{}{0pt}%
\pgfpathmoveto{\pgfqpoint{3.484165in}{2.242808in}}%
\pgfpathlineto{\pgfqpoint{3.497585in}{2.237940in}}%
\pgfpathlineto{\pgfqpoint{3.511011in}{2.233104in}}%
\pgfpathlineto{\pgfqpoint{3.524442in}{2.228302in}}%
\pgfpathlineto{\pgfqpoint{3.537878in}{2.223532in}}%
\pgfpathlineto{\pgfqpoint{3.529838in}{2.217208in}}%
\pgfpathlineto{\pgfqpoint{3.521791in}{2.210967in}}%
\pgfpathlineto{\pgfqpoint{3.513736in}{2.204811in}}%
\pgfpathlineto{\pgfqpoint{3.505674in}{2.198746in}}%
\pgfpathlineto{\pgfqpoint{3.492222in}{2.203677in}}%
\pgfpathlineto{\pgfqpoint{3.478775in}{2.208641in}}%
\pgfpathlineto{\pgfqpoint{3.465333in}{2.213637in}}%
\pgfpathlineto{\pgfqpoint{3.451896in}{2.218667in}}%
\pgfpathlineto{\pgfqpoint{3.459974in}{2.224566in}}%
\pgfpathlineto{\pgfqpoint{3.468045in}{2.230558in}}%
\pgfpathlineto{\pgfqpoint{3.476109in}{2.236640in}}%
\pgfpathlineto{\pgfqpoint{3.484165in}{2.242808in}}%
\pgfpathclose%
\pgfusepath{fill}%
\end{pgfscope}%
\begin{pgfscope}%
\pgfpathrectangle{\pgfqpoint{1.150000in}{0.150000in}}{\pgfqpoint{5.700000in}{5.700000in}}%
\pgfusepath{clip}%
\pgfsetbuttcap%
\pgfsetroundjoin%
\definecolor{currentfill}{rgb}{0.274952,0.037752,0.364543}%
\pgfsetfillcolor{currentfill}%
\pgfsetfillopacity{0.700000}%
\pgfsetlinewidth{0.000000pt}%
\definecolor{currentstroke}{rgb}{0.000000,0.000000,0.000000}%
\pgfsetstrokecolor{currentstroke}%
\pgfsetdash{}{0pt}%
\pgfpathmoveto{\pgfqpoint{4.438341in}{2.296474in}}%
\pgfpathlineto{\pgfqpoint{4.451974in}{2.293943in}}%
\pgfpathlineto{\pgfqpoint{4.465613in}{2.291439in}}%
\pgfpathlineto{\pgfqpoint{4.479260in}{2.288962in}}%
\pgfpathlineto{\pgfqpoint{4.492914in}{2.286511in}}%
\pgfpathlineto{\pgfqpoint{4.485236in}{2.278386in}}%
\pgfpathlineto{\pgfqpoint{4.477553in}{2.270224in}}%
\pgfpathlineto{\pgfqpoint{4.469865in}{2.262029in}}%
\pgfpathlineto{\pgfqpoint{4.462170in}{2.253799in}}%
\pgfpathlineto{\pgfqpoint{4.448505in}{2.256305in}}%
\pgfpathlineto{\pgfqpoint{4.434846in}{2.258837in}}%
\pgfpathlineto{\pgfqpoint{4.421195in}{2.261396in}}%
\pgfpathlineto{\pgfqpoint{4.407551in}{2.263981in}}%
\pgfpathlineto{\pgfqpoint{4.415257in}{2.272151in}}%
\pgfpathlineto{\pgfqpoint{4.422957in}{2.280290in}}%
\pgfpathlineto{\pgfqpoint{4.430652in}{2.288398in}}%
\pgfpathlineto{\pgfqpoint{4.438341in}{2.296474in}}%
\pgfpathclose%
\pgfusepath{fill}%
\end{pgfscope}%
\begin{pgfscope}%
\pgfpathrectangle{\pgfqpoint{1.150000in}{0.150000in}}{\pgfqpoint{5.700000in}{5.700000in}}%
\pgfusepath{clip}%
\pgfsetbuttcap%
\pgfsetroundjoin%
\definecolor{currentfill}{rgb}{0.271305,0.019942,0.347269}%
\pgfsetfillcolor{currentfill}%
\pgfsetfillopacity{0.700000}%
\pgfsetlinewidth{0.000000pt}%
\definecolor{currentstroke}{rgb}{0.000000,0.000000,0.000000}%
\pgfsetstrokecolor{currentstroke}%
\pgfsetdash{}{0pt}%
\pgfpathmoveto{\pgfqpoint{3.344577in}{2.260112in}}%
\pgfpathlineto{\pgfqpoint{3.357975in}{2.254811in}}%
\pgfpathlineto{\pgfqpoint{3.371377in}{2.249546in}}%
\pgfpathlineto{\pgfqpoint{3.384785in}{2.244314in}}%
\pgfpathlineto{\pgfqpoint{3.398197in}{2.239117in}}%
\pgfpathlineto{\pgfqpoint{3.390093in}{2.233485in}}%
\pgfpathlineto{\pgfqpoint{3.381981in}{2.227957in}}%
\pgfpathlineto{\pgfqpoint{3.373861in}{2.222538in}}%
\pgfpathlineto{\pgfqpoint{3.365733in}{2.217233in}}%
\pgfpathlineto{\pgfqpoint{3.352302in}{2.222604in}}%
\pgfpathlineto{\pgfqpoint{3.338877in}{2.228010in}}%
\pgfpathlineto{\pgfqpoint{3.325456in}{2.233450in}}%
\pgfpathlineto{\pgfqpoint{3.312040in}{2.238925in}}%
\pgfpathlineto{\pgfqpoint{3.320187in}{2.244051in}}%
\pgfpathlineto{\pgfqpoint{3.328325in}{2.249293in}}%
\pgfpathlineto{\pgfqpoint{3.336455in}{2.254648in}}%
\pgfpathlineto{\pgfqpoint{3.344577in}{2.260112in}}%
\pgfpathclose%
\pgfusepath{fill}%
\end{pgfscope}%
\begin{pgfscope}%
\pgfpathrectangle{\pgfqpoint{1.150000in}{0.150000in}}{\pgfqpoint{5.700000in}{5.700000in}}%
\pgfusepath{clip}%
\pgfsetbuttcap%
\pgfsetroundjoin%
\definecolor{currentfill}{rgb}{0.268510,0.009605,0.335427}%
\pgfsetfillcolor{currentfill}%
\pgfsetfillopacity{0.700000}%
\pgfsetlinewidth{0.000000pt}%
\definecolor{currentstroke}{rgb}{0.000000,0.000000,0.000000}%
\pgfsetstrokecolor{currentstroke}%
\pgfsetdash{}{0pt}%
\pgfpathmoveto{\pgfqpoint{3.988328in}{2.240238in}}%
\pgfpathlineto{\pgfqpoint{4.001850in}{2.236743in}}%
\pgfpathlineto{\pgfqpoint{4.015379in}{2.233276in}}%
\pgfpathlineto{\pgfqpoint{4.028913in}{2.229837in}}%
\pgfpathlineto{\pgfqpoint{4.042454in}{2.226428in}}%
\pgfpathlineto{\pgfqpoint{4.034614in}{2.218515in}}%
\pgfpathlineto{\pgfqpoint{4.026769in}{2.210613in}}%
\pgfpathlineto{\pgfqpoint{4.018917in}{2.202724in}}%
\pgfpathlineto{\pgfqpoint{4.011060in}{2.194849in}}%
\pgfpathlineto{\pgfqpoint{3.997507in}{2.198366in}}%
\pgfpathlineto{\pgfqpoint{3.983959in}{2.201913in}}%
\pgfpathlineto{\pgfqpoint{3.970418in}{2.205488in}}%
\pgfpathlineto{\pgfqpoint{3.956883in}{2.209092in}}%
\pgfpathlineto{\pgfqpoint{3.964753in}{2.216853in}}%
\pgfpathlineto{\pgfqpoint{3.972617in}{2.224633in}}%
\pgfpathlineto{\pgfqpoint{3.980475in}{2.232429in}}%
\pgfpathlineto{\pgfqpoint{3.988328in}{2.240238in}}%
\pgfpathclose%
\pgfusepath{fill}%
\end{pgfscope}%
\begin{pgfscope}%
\pgfpathrectangle{\pgfqpoint{1.150000in}{0.150000in}}{\pgfqpoint{5.700000in}{5.700000in}}%
\pgfusepath{clip}%
\pgfsetbuttcap%
\pgfsetroundjoin%
\definecolor{currentfill}{rgb}{0.283197,0.115680,0.436115}%
\pgfsetfillcolor{currentfill}%
\pgfsetfillopacity{0.700000}%
\pgfsetlinewidth{0.000000pt}%
\definecolor{currentstroke}{rgb}{0.000000,0.000000,0.000000}%
\pgfsetstrokecolor{currentstroke}%
\pgfsetdash{}{0pt}%
\pgfpathmoveto{\pgfqpoint{5.198758in}{2.428267in}}%
\pgfpathlineto{\pgfqpoint{5.212602in}{2.426774in}}%
\pgfpathlineto{\pgfqpoint{5.226453in}{2.425307in}}%
\pgfpathlineto{\pgfqpoint{5.240313in}{2.423865in}}%
\pgfpathlineto{\pgfqpoint{5.254180in}{2.422447in}}%
\pgfpathlineto{\pgfqpoint{5.246808in}{2.415749in}}%
\pgfpathlineto{\pgfqpoint{5.239429in}{2.408994in}}%
\pgfpathlineto{\pgfqpoint{5.232043in}{2.402180in}}%
\pgfpathlineto{\pgfqpoint{5.224651in}{2.395306in}}%
\pgfpathlineto{\pgfqpoint{5.210769in}{2.396684in}}%
\pgfpathlineto{\pgfqpoint{5.196895in}{2.398087in}}%
\pgfpathlineto{\pgfqpoint{5.183029in}{2.399515in}}%
\pgfpathlineto{\pgfqpoint{5.169171in}{2.400968in}}%
\pgfpathlineto{\pgfqpoint{5.176578in}{2.407876in}}%
\pgfpathlineto{\pgfqpoint{5.183978in}{2.414728in}}%
\pgfpathlineto{\pgfqpoint{5.191371in}{2.421524in}}%
\pgfpathlineto{\pgfqpoint{5.198758in}{2.428267in}}%
\pgfpathclose%
\pgfusepath{fill}%
\end{pgfscope}%
\begin{pgfscope}%
\pgfpathrectangle{\pgfqpoint{1.150000in}{0.150000in}}{\pgfqpoint{5.700000in}{5.700000in}}%
\pgfusepath{clip}%
\pgfsetbuttcap%
\pgfsetroundjoin%
\definecolor{currentfill}{rgb}{0.267004,0.004874,0.329415}%
\pgfsetfillcolor{currentfill}%
\pgfsetfillopacity{0.700000}%
\pgfsetlinewidth{0.000000pt}%
\definecolor{currentstroke}{rgb}{0.000000,0.000000,0.000000}%
\pgfsetstrokecolor{currentstroke}%
\pgfsetdash{}{0pt}%
\pgfpathmoveto{\pgfqpoint{3.623699in}{2.231406in}}%
\pgfpathlineto{\pgfqpoint{3.637146in}{2.226944in}}%
\pgfpathlineto{\pgfqpoint{3.650599in}{2.222513in}}%
\pgfpathlineto{\pgfqpoint{3.664057in}{2.218114in}}%
\pgfpathlineto{\pgfqpoint{3.677520in}{2.213746in}}%
\pgfpathlineto{\pgfqpoint{3.669539in}{2.206845in}}%
\pgfpathlineto{\pgfqpoint{3.661551in}{2.200005in}}%
\pgfpathlineto{\pgfqpoint{3.653557in}{2.193230in}}%
\pgfpathlineto{\pgfqpoint{3.645555in}{2.186522in}}%
\pgfpathlineto{\pgfqpoint{3.632077in}{2.191038in}}%
\pgfpathlineto{\pgfqpoint{3.618604in}{2.195585in}}%
\pgfpathlineto{\pgfqpoint{3.605136in}{2.200164in}}%
\pgfpathlineto{\pgfqpoint{3.591674in}{2.204773in}}%
\pgfpathlineto{\pgfqpoint{3.599690in}{2.211328in}}%
\pgfpathlineto{\pgfqpoint{3.607700in}{2.217954in}}%
\pgfpathlineto{\pgfqpoint{3.615703in}{2.224648in}}%
\pgfpathlineto{\pgfqpoint{3.623699in}{2.231406in}}%
\pgfpathclose%
\pgfusepath{fill}%
\end{pgfscope}%
\begin{pgfscope}%
\pgfpathrectangle{\pgfqpoint{1.150000in}{0.150000in}}{\pgfqpoint{5.700000in}{5.700000in}}%
\pgfusepath{clip}%
\pgfsetbuttcap%
\pgfsetroundjoin%
\definecolor{currentfill}{rgb}{0.278826,0.175490,0.483397}%
\pgfsetfillcolor{currentfill}%
\pgfsetfillopacity{0.700000}%
\pgfsetlinewidth{0.000000pt}%
\definecolor{currentstroke}{rgb}{0.000000,0.000000,0.000000}%
\pgfsetstrokecolor{currentstroke}%
\pgfsetdash{}{0pt}%
\pgfpathmoveto{\pgfqpoint{6.098900in}{2.547980in}}%
\pgfpathlineto{\pgfqpoint{6.112998in}{2.546854in}}%
\pgfpathlineto{\pgfqpoint{6.127106in}{2.545752in}}%
\pgfpathlineto{\pgfqpoint{6.141222in}{2.544674in}}%
\pgfpathlineto{\pgfqpoint{6.134278in}{2.540164in}}%
\pgfpathlineto{\pgfqpoint{6.127329in}{2.535660in}}%
\pgfpathlineto{\pgfqpoint{6.120374in}{2.531156in}}%
\pgfpathlineto{\pgfqpoint{6.113413in}{2.526650in}}%
\pgfpathlineto{\pgfqpoint{6.099274in}{2.527580in}}%
\pgfpathlineto{\pgfqpoint{6.085144in}{2.528535in}}%
\pgfpathlineto{\pgfqpoint{6.071023in}{2.529513in}}%
\pgfpathlineto{\pgfqpoint{6.078000in}{2.534126in}}%
\pgfpathlineto{\pgfqpoint{6.084972in}{2.538739in}}%
\pgfpathlineto{\pgfqpoint{6.091939in}{2.543356in}}%
\pgfpathlineto{\pgfqpoint{6.098900in}{2.547980in}}%
\pgfpathclose%
\pgfusepath{fill}%
\end{pgfscope}%
\begin{pgfscope}%
\pgfpathrectangle{\pgfqpoint{1.150000in}{0.150000in}}{\pgfqpoint{5.700000in}{5.700000in}}%
\pgfusepath{clip}%
\pgfsetbuttcap%
\pgfsetroundjoin%
\definecolor{currentfill}{rgb}{0.278791,0.062145,0.386592}%
\pgfsetfillcolor{currentfill}%
\pgfsetfillopacity{0.700000}%
\pgfsetlinewidth{0.000000pt}%
\definecolor{currentstroke}{rgb}{0.000000,0.000000,0.000000}%
\pgfsetstrokecolor{currentstroke}%
\pgfsetdash{}{0pt}%
\pgfpathmoveto{\pgfqpoint{4.663419in}{2.332072in}}%
\pgfpathlineto{\pgfqpoint{4.677114in}{2.329929in}}%
\pgfpathlineto{\pgfqpoint{4.690816in}{2.327811in}}%
\pgfpathlineto{\pgfqpoint{4.704526in}{2.325719in}}%
\pgfpathlineto{\pgfqpoint{4.718243in}{2.323653in}}%
\pgfpathlineto{\pgfqpoint{4.710648in}{2.315762in}}%
\pgfpathlineto{\pgfqpoint{4.703048in}{2.307822in}}%
\pgfpathlineto{\pgfqpoint{4.695441in}{2.299832in}}%
\pgfpathlineto{\pgfqpoint{4.687829in}{2.291794in}}%
\pgfpathlineto{\pgfqpoint{4.674100in}{2.293888in}}%
\pgfpathlineto{\pgfqpoint{4.660378in}{2.296008in}}%
\pgfpathlineto{\pgfqpoint{4.646664in}{2.298154in}}%
\pgfpathlineto{\pgfqpoint{4.632958in}{2.300326in}}%
\pgfpathlineto{\pgfqpoint{4.640582in}{2.308331in}}%
\pgfpathlineto{\pgfqpoint{4.648201in}{2.316291in}}%
\pgfpathlineto{\pgfqpoint{4.655813in}{2.324205in}}%
\pgfpathlineto{\pgfqpoint{4.663419in}{2.332072in}}%
\pgfpathclose%
\pgfusepath{fill}%
\end{pgfscope}%
\begin{pgfscope}%
\pgfpathrectangle{\pgfqpoint{1.150000in}{0.150000in}}{\pgfqpoint{5.700000in}{5.700000in}}%
\pgfusepath{clip}%
\pgfsetbuttcap%
\pgfsetroundjoin%
\definecolor{currentfill}{rgb}{0.273809,0.031497,0.358853}%
\pgfsetfillcolor{currentfill}%
\pgfsetfillopacity{0.700000}%
\pgfsetlinewidth{0.000000pt}%
\definecolor{currentstroke}{rgb}{0.000000,0.000000,0.000000}%
\pgfsetstrokecolor{currentstroke}%
\pgfsetdash{}{0pt}%
\pgfpathmoveto{\pgfqpoint{3.204876in}{2.283998in}}%
\pgfpathlineto{\pgfqpoint{3.218256in}{2.278237in}}%
\pgfpathlineto{\pgfqpoint{3.231640in}{2.272513in}}%
\pgfpathlineto{\pgfqpoint{3.245029in}{2.266826in}}%
\pgfpathlineto{\pgfqpoint{3.258422in}{2.261174in}}%
\pgfpathlineto{\pgfqpoint{3.250248in}{2.256352in}}%
\pgfpathlineto{\pgfqpoint{3.242064in}{2.251659in}}%
\pgfpathlineto{\pgfqpoint{3.233871in}{2.247100in}}%
\pgfpathlineto{\pgfqpoint{3.225669in}{2.242679in}}%
\pgfpathlineto{\pgfqpoint{3.212257in}{2.248518in}}%
\pgfpathlineto{\pgfqpoint{3.198848in}{2.254394in}}%
\pgfpathlineto{\pgfqpoint{3.185444in}{2.260306in}}%
\pgfpathlineto{\pgfqpoint{3.172045in}{2.266254in}}%
\pgfpathlineto{\pgfqpoint{3.180267in}{2.270482in}}%
\pgfpathlineto{\pgfqpoint{3.188479in}{2.274852in}}%
\pgfpathlineto{\pgfqpoint{3.196683in}{2.279358in}}%
\pgfpathlineto{\pgfqpoint{3.204876in}{2.283998in}}%
\pgfpathclose%
\pgfusepath{fill}%
\end{pgfscope}%
\begin{pgfscope}%
\pgfpathrectangle{\pgfqpoint{1.150000in}{0.150000in}}{\pgfqpoint{5.700000in}{5.700000in}}%
\pgfusepath{clip}%
\pgfsetbuttcap%
\pgfsetroundjoin%
\definecolor{currentfill}{rgb}{0.282884,0.135920,0.453427}%
\pgfsetfillcolor{currentfill}%
\pgfsetfillopacity{0.700000}%
\pgfsetlinewidth{0.000000pt}%
\definecolor{currentstroke}{rgb}{0.000000,0.000000,0.000000}%
\pgfsetstrokecolor{currentstroke}%
\pgfsetdash{}{0pt}%
\pgfpathmoveto{\pgfqpoint{5.423900in}{2.462799in}}%
\pgfpathlineto{\pgfqpoint{5.437809in}{2.461485in}}%
\pgfpathlineto{\pgfqpoint{5.451727in}{2.460195in}}%
\pgfpathlineto{\pgfqpoint{5.465654in}{2.458930in}}%
\pgfpathlineto{\pgfqpoint{5.479588in}{2.457690in}}%
\pgfpathlineto{\pgfqpoint{5.472318in}{2.451593in}}%
\pgfpathlineto{\pgfqpoint{5.465041in}{2.445445in}}%
\pgfpathlineto{\pgfqpoint{5.457757in}{2.439245in}}%
\pgfpathlineto{\pgfqpoint{5.450465in}{2.432989in}}%
\pgfpathlineto{\pgfqpoint{5.436515in}{2.434163in}}%
\pgfpathlineto{\pgfqpoint{5.422572in}{2.435361in}}%
\pgfpathlineto{\pgfqpoint{5.408638in}{2.436585in}}%
\pgfpathlineto{\pgfqpoint{5.394713in}{2.437832in}}%
\pgfpathlineto{\pgfqpoint{5.402020in}{2.444150in}}%
\pgfpathlineto{\pgfqpoint{5.409320in}{2.450415in}}%
\pgfpathlineto{\pgfqpoint{5.416613in}{2.456631in}}%
\pgfpathlineto{\pgfqpoint{5.423900in}{2.462799in}}%
\pgfpathclose%
\pgfusepath{fill}%
\end{pgfscope}%
\begin{pgfscope}%
\pgfpathrectangle{\pgfqpoint{1.150000in}{0.150000in}}{\pgfqpoint{5.700000in}{5.700000in}}%
\pgfusepath{clip}%
\pgfsetbuttcap%
\pgfsetroundjoin%
\definecolor{currentfill}{rgb}{0.281924,0.089666,0.412415}%
\pgfsetfillcolor{currentfill}%
\pgfsetfillopacity{0.700000}%
\pgfsetlinewidth{0.000000pt}%
\definecolor{currentstroke}{rgb}{0.000000,0.000000,0.000000}%
\pgfsetstrokecolor{currentstroke}%
\pgfsetdash{}{0pt}%
\pgfpathmoveto{\pgfqpoint{2.871465in}{2.382000in}}%
\pgfpathlineto{\pgfqpoint{2.884810in}{2.375054in}}%
\pgfpathlineto{\pgfqpoint{2.898159in}{2.368151in}}%
\pgfpathlineto{\pgfqpoint{2.911510in}{2.361291in}}%
\pgfpathlineto{\pgfqpoint{2.924865in}{2.354473in}}%
\pgfpathlineto{\pgfqpoint{2.916502in}{2.351869in}}%
\pgfpathlineto{\pgfqpoint{2.908126in}{2.349451in}}%
\pgfpathlineto{\pgfqpoint{2.899739in}{2.347224in}}%
\pgfpathlineto{\pgfqpoint{2.891339in}{2.345193in}}%
\pgfpathlineto{\pgfqpoint{2.877959in}{2.352227in}}%
\pgfpathlineto{\pgfqpoint{2.864583in}{2.359304in}}%
\pgfpathlineto{\pgfqpoint{2.851211in}{2.366423in}}%
\pgfpathlineto{\pgfqpoint{2.837841in}{2.373585in}}%
\pgfpathlineto{\pgfqpoint{2.846266in}{2.375394in}}%
\pgfpathlineto{\pgfqpoint{2.854678in}{2.377403in}}%
\pgfpathlineto{\pgfqpoint{2.863078in}{2.379607in}}%
\pgfpathlineto{\pgfqpoint{2.871465in}{2.382000in}}%
\pgfpathclose%
\pgfusepath{fill}%
\end{pgfscope}%
\begin{pgfscope}%
\pgfpathrectangle{\pgfqpoint{1.150000in}{0.150000in}}{\pgfqpoint{5.700000in}{5.700000in}}%
\pgfusepath{clip}%
\pgfsetbuttcap%
\pgfsetroundjoin%
\definecolor{currentfill}{rgb}{0.281887,0.150881,0.465405}%
\pgfsetfillcolor{currentfill}%
\pgfsetfillopacity{0.700000}%
\pgfsetlinewidth{0.000000pt}%
\definecolor{currentstroke}{rgb}{0.000000,0.000000,0.000000}%
\pgfsetstrokecolor{currentstroke}%
\pgfsetdash{}{0pt}%
\pgfpathmoveto{\pgfqpoint{5.649004in}{2.494528in}}%
\pgfpathlineto{\pgfqpoint{5.662979in}{2.493334in}}%
\pgfpathlineto{\pgfqpoint{5.676963in}{2.492164in}}%
\pgfpathlineto{\pgfqpoint{5.690954in}{2.491018in}}%
\pgfpathlineto{\pgfqpoint{5.704955in}{2.489896in}}%
\pgfpathlineto{\pgfqpoint{5.697792in}{2.484403in}}%
\pgfpathlineto{\pgfqpoint{5.690623in}{2.478871in}}%
\pgfpathlineto{\pgfqpoint{5.683446in}{2.473299in}}%
\pgfpathlineto{\pgfqpoint{5.676263in}{2.467682in}}%
\pgfpathlineto{\pgfqpoint{5.662244in}{2.468709in}}%
\pgfpathlineto{\pgfqpoint{5.648234in}{2.469761in}}%
\pgfpathlineto{\pgfqpoint{5.634233in}{2.470837in}}%
\pgfpathlineto{\pgfqpoint{5.620240in}{2.471938in}}%
\pgfpathlineto{\pgfqpoint{5.627442in}{2.477644in}}%
\pgfpathlineto{\pgfqpoint{5.634636in}{2.483309in}}%
\pgfpathlineto{\pgfqpoint{5.641823in}{2.488936in}}%
\pgfpathlineto{\pgfqpoint{5.649004in}{2.494528in}}%
\pgfpathclose%
\pgfusepath{fill}%
\end{pgfscope}%
\begin{pgfscope}%
\pgfpathrectangle{\pgfqpoint{1.150000in}{0.150000in}}{\pgfqpoint{5.700000in}{5.700000in}}%
\pgfusepath{clip}%
\pgfsetbuttcap%
\pgfsetroundjoin%
\definecolor{currentfill}{rgb}{0.267004,0.004874,0.329415}%
\pgfsetfillcolor{currentfill}%
\pgfsetfillopacity{0.700000}%
\pgfsetlinewidth{0.000000pt}%
\definecolor{currentstroke}{rgb}{0.000000,0.000000,0.000000}%
\pgfsetstrokecolor{currentstroke}%
\pgfsetdash{}{0pt}%
\pgfpathmoveto{\pgfqpoint{3.763231in}{2.225269in}}%
\pgfpathlineto{\pgfqpoint{3.776708in}{2.221189in}}%
\pgfpathlineto{\pgfqpoint{3.790191in}{2.217139in}}%
\pgfpathlineto{\pgfqpoint{3.803680in}{2.213120in}}%
\pgfpathlineto{\pgfqpoint{3.817174in}{2.209130in}}%
\pgfpathlineto{\pgfqpoint{3.809247in}{2.201759in}}%
\pgfpathlineto{\pgfqpoint{3.801314in}{2.194429in}}%
\pgfpathlineto{\pgfqpoint{3.793375in}{2.187143in}}%
\pgfpathlineto{\pgfqpoint{3.785429in}{2.179905in}}%
\pgfpathlineto{\pgfqpoint{3.771921in}{2.184029in}}%
\pgfpathlineto{\pgfqpoint{3.758418in}{2.188184in}}%
\pgfpathlineto{\pgfqpoint{3.744921in}{2.192368in}}%
\pgfpathlineto{\pgfqpoint{3.731430in}{2.196582in}}%
\pgfpathlineto{\pgfqpoint{3.739389in}{2.203681in}}%
\pgfpathlineto{\pgfqpoint{3.747343in}{2.210830in}}%
\pgfpathlineto{\pgfqpoint{3.755290in}{2.218027in}}%
\pgfpathlineto{\pgfqpoint{3.763231in}{2.225269in}}%
\pgfpathclose%
\pgfusepath{fill}%
\end{pgfscope}%
\begin{pgfscope}%
\pgfpathrectangle{\pgfqpoint{1.150000in}{0.150000in}}{\pgfqpoint{5.700000in}{5.700000in}}%
\pgfusepath{clip}%
\pgfsetbuttcap%
\pgfsetroundjoin%
\definecolor{currentfill}{rgb}{0.280255,0.165693,0.476498}%
\pgfsetfillcolor{currentfill}%
\pgfsetfillopacity{0.700000}%
\pgfsetlinewidth{0.000000pt}%
\definecolor{currentstroke}{rgb}{0.000000,0.000000,0.000000}%
\pgfsetstrokecolor{currentstroke}%
\pgfsetdash{}{0pt}%
\pgfpathmoveto{\pgfqpoint{5.874020in}{2.522956in}}%
\pgfpathlineto{\pgfqpoint{5.888058in}{2.521824in}}%
\pgfpathlineto{\pgfqpoint{5.902105in}{2.520716in}}%
\pgfpathlineto{\pgfqpoint{5.916160in}{2.519633in}}%
\pgfpathlineto{\pgfqpoint{5.930224in}{2.518573in}}%
\pgfpathlineto{\pgfqpoint{5.923173in}{2.513638in}}%
\pgfpathlineto{\pgfqpoint{5.916115in}{2.508683in}}%
\pgfpathlineto{\pgfqpoint{5.909051in}{2.503704in}}%
\pgfpathlineto{\pgfqpoint{5.901980in}{2.498698in}}%
\pgfpathlineto{\pgfqpoint{5.887896in}{2.499637in}}%
\pgfpathlineto{\pgfqpoint{5.873821in}{2.500599in}}%
\pgfpathlineto{\pgfqpoint{5.859754in}{2.501586in}}%
\pgfpathlineto{\pgfqpoint{5.845695in}{2.502597in}}%
\pgfpathlineto{\pgfqpoint{5.852786in}{2.507719in}}%
\pgfpathlineto{\pgfqpoint{5.859870in}{2.512817in}}%
\pgfpathlineto{\pgfqpoint{5.866948in}{2.517895in}}%
\pgfpathlineto{\pgfqpoint{5.874020in}{2.522956in}}%
\pgfpathclose%
\pgfusepath{fill}%
\end{pgfscope}%
\begin{pgfscope}%
\pgfpathrectangle{\pgfqpoint{1.150000in}{0.150000in}}{\pgfqpoint{5.700000in}{5.700000in}}%
\pgfusepath{clip}%
\pgfsetbuttcap%
\pgfsetroundjoin%
\definecolor{currentfill}{rgb}{0.282884,0.135920,0.453427}%
\pgfsetfillcolor{currentfill}%
\pgfsetfillopacity{0.700000}%
\pgfsetlinewidth{0.000000pt}%
\definecolor{currentstroke}{rgb}{0.000000,0.000000,0.000000}%
\pgfsetstrokecolor{currentstroke}%
\pgfsetdash{}{0pt}%
\pgfpathmoveto{\pgfqpoint{2.677647in}{2.463041in}}%
\pgfpathlineto{\pgfqpoint{2.690980in}{2.455329in}}%
\pgfpathlineto{\pgfqpoint{2.704317in}{2.447666in}}%
\pgfpathlineto{\pgfqpoint{2.717656in}{2.440050in}}%
\pgfpathlineto{\pgfqpoint{2.730998in}{2.432482in}}%
\pgfpathlineto{\pgfqpoint{2.722508in}{2.431328in}}%
\pgfpathlineto{\pgfqpoint{2.714004in}{2.430394in}}%
\pgfpathlineto{\pgfqpoint{2.705486in}{2.429684in}}%
\pgfpathlineto{\pgfqpoint{2.696953in}{2.429204in}}%
\pgfpathlineto{\pgfqpoint{2.683584in}{2.437003in}}%
\pgfpathlineto{\pgfqpoint{2.670217in}{2.444850in}}%
\pgfpathlineto{\pgfqpoint{2.656854in}{2.452744in}}%
\pgfpathlineto{\pgfqpoint{2.643492in}{2.460687in}}%
\pgfpathlineto{\pgfqpoint{2.652053in}{2.460930in}}%
\pgfpathlineto{\pgfqpoint{2.660599in}{2.461408in}}%
\pgfpathlineto{\pgfqpoint{2.669130in}{2.462113in}}%
\pgfpathlineto{\pgfqpoint{2.677647in}{2.463041in}}%
\pgfpathclose%
\pgfusepath{fill}%
\end{pgfscope}%
\begin{pgfscope}%
\pgfpathrectangle{\pgfqpoint{1.150000in}{0.150000in}}{\pgfqpoint{5.700000in}{5.700000in}}%
\pgfusepath{clip}%
\pgfsetbuttcap%
\pgfsetroundjoin%
\definecolor{currentfill}{rgb}{0.281446,0.084320,0.407414}%
\pgfsetfillcolor{currentfill}%
\pgfsetfillopacity{0.700000}%
\pgfsetlinewidth{0.000000pt}%
\definecolor{currentstroke}{rgb}{0.000000,0.000000,0.000000}%
\pgfsetstrokecolor{currentstroke}%
\pgfsetdash{}{0pt}%
\pgfpathmoveto{\pgfqpoint{4.888583in}{2.369542in}}%
\pgfpathlineto{\pgfqpoint{4.902343in}{2.367723in}}%
\pgfpathlineto{\pgfqpoint{4.916110in}{2.365930in}}%
\pgfpathlineto{\pgfqpoint{4.929885in}{2.364163in}}%
\pgfpathlineto{\pgfqpoint{4.943667in}{2.362421in}}%
\pgfpathlineto{\pgfqpoint{4.936160in}{2.354931in}}%
\pgfpathlineto{\pgfqpoint{4.928647in}{2.347385in}}%
\pgfpathlineto{\pgfqpoint{4.921127in}{2.339780in}}%
\pgfpathlineto{\pgfqpoint{4.913601in}{2.332116in}}%
\pgfpathlineto{\pgfqpoint{4.899805in}{2.333859in}}%
\pgfpathlineto{\pgfqpoint{4.886018in}{2.335628in}}%
\pgfpathlineto{\pgfqpoint{4.872238in}{2.337422in}}%
\pgfpathlineto{\pgfqpoint{4.858465in}{2.339242in}}%
\pgfpathlineto{\pgfqpoint{4.866004in}{2.346899in}}%
\pgfpathlineto{\pgfqpoint{4.873537in}{2.354501in}}%
\pgfpathlineto{\pgfqpoint{4.881063in}{2.362048in}}%
\pgfpathlineto{\pgfqpoint{4.888583in}{2.369542in}}%
\pgfpathclose%
\pgfusepath{fill}%
\end{pgfscope}%
\begin{pgfscope}%
\pgfpathrectangle{\pgfqpoint{1.150000in}{0.150000in}}{\pgfqpoint{5.700000in}{5.700000in}}%
\pgfusepath{clip}%
\pgfsetbuttcap%
\pgfsetroundjoin%
\definecolor{currentfill}{rgb}{0.269944,0.014625,0.341379}%
\pgfsetfillcolor{currentfill}%
\pgfsetfillopacity{0.700000}%
\pgfsetlinewidth{0.000000pt}%
\definecolor{currentstroke}{rgb}{0.000000,0.000000,0.000000}%
\pgfsetstrokecolor{currentstroke}%
\pgfsetdash{}{0pt}%
\pgfpathmoveto{\pgfqpoint{4.127935in}{2.245158in}}%
\pgfpathlineto{\pgfqpoint{4.141495in}{2.241984in}}%
\pgfpathlineto{\pgfqpoint{4.155062in}{2.238839in}}%
\pgfpathlineto{\pgfqpoint{4.168636in}{2.235721in}}%
\pgfpathlineto{\pgfqpoint{4.182216in}{2.232631in}}%
\pgfpathlineto{\pgfqpoint{4.174423in}{2.224521in}}%
\pgfpathlineto{\pgfqpoint{4.166625in}{2.216405in}}%
\pgfpathlineto{\pgfqpoint{4.158821in}{2.208287in}}%
\pgfpathlineto{\pgfqpoint{4.151012in}{2.200167in}}%
\pgfpathlineto{\pgfqpoint{4.137419in}{2.203352in}}%
\pgfpathlineto{\pgfqpoint{4.123834in}{2.206564in}}%
\pgfpathlineto{\pgfqpoint{4.110254in}{2.209804in}}%
\pgfpathlineto{\pgfqpoint{4.096682in}{2.213073in}}%
\pgfpathlineto{\pgfqpoint{4.104503in}{2.221093in}}%
\pgfpathlineto{\pgfqpoint{4.112319in}{2.229115in}}%
\pgfpathlineto{\pgfqpoint{4.120130in}{2.237137in}}%
\pgfpathlineto{\pgfqpoint{4.127935in}{2.245158in}}%
\pgfpathclose%
\pgfusepath{fill}%
\end{pgfscope}%
\begin{pgfscope}%
\pgfpathrectangle{\pgfqpoint{1.150000in}{0.150000in}}{\pgfqpoint{5.700000in}{5.700000in}}%
\pgfusepath{clip}%
\pgfsetbuttcap%
\pgfsetroundjoin%
\definecolor{currentfill}{rgb}{0.273809,0.031497,0.358853}%
\pgfsetfillcolor{currentfill}%
\pgfsetfillopacity{0.700000}%
\pgfsetlinewidth{0.000000pt}%
\definecolor{currentstroke}{rgb}{0.000000,0.000000,0.000000}%
\pgfsetstrokecolor{currentstroke}%
\pgfsetdash{}{0pt}%
\pgfpathmoveto{\pgfqpoint{4.353042in}{2.274593in}}%
\pgfpathlineto{\pgfqpoint{4.366659in}{2.271899in}}%
\pgfpathlineto{\pgfqpoint{4.380283in}{2.269233in}}%
\pgfpathlineto{\pgfqpoint{4.393913in}{2.266594in}}%
\pgfpathlineto{\pgfqpoint{4.407551in}{2.263981in}}%
\pgfpathlineto{\pgfqpoint{4.399839in}{2.255783in}}%
\pgfpathlineto{\pgfqpoint{4.392121in}{2.247557in}}%
\pgfpathlineto{\pgfqpoint{4.384398in}{2.239305in}}%
\pgfpathlineto{\pgfqpoint{4.376669in}{2.231027in}}%
\pgfpathlineto{\pgfqpoint{4.363020in}{2.233707in}}%
\pgfpathlineto{\pgfqpoint{4.349378in}{2.236415in}}%
\pgfpathlineto{\pgfqpoint{4.335742in}{2.239149in}}%
\pgfpathlineto{\pgfqpoint{4.322114in}{2.241911in}}%
\pgfpathlineto{\pgfqpoint{4.329854in}{2.250116in}}%
\pgfpathlineto{\pgfqpoint{4.337589in}{2.258298in}}%
\pgfpathlineto{\pgfqpoint{4.345319in}{2.266458in}}%
\pgfpathlineto{\pgfqpoint{4.353042in}{2.274593in}}%
\pgfpathclose%
\pgfusepath{fill}%
\end{pgfscope}%
\begin{pgfscope}%
\pgfpathrectangle{\pgfqpoint{1.150000in}{0.150000in}}{\pgfqpoint{5.700000in}{5.700000in}}%
\pgfusepath{clip}%
\pgfsetbuttcap%
\pgfsetroundjoin%
\definecolor{currentfill}{rgb}{0.277941,0.056324,0.381191}%
\pgfsetfillcolor{currentfill}%
\pgfsetfillopacity{0.700000}%
\pgfsetlinewidth{0.000000pt}%
\definecolor{currentstroke}{rgb}{0.000000,0.000000,0.000000}%
\pgfsetstrokecolor{currentstroke}%
\pgfsetdash{}{0pt}%
\pgfpathmoveto{\pgfqpoint{3.064998in}{2.315192in}}%
\pgfpathlineto{\pgfqpoint{3.078364in}{2.308941in}}%
\pgfpathlineto{\pgfqpoint{3.091735in}{2.302729in}}%
\pgfpathlineto{\pgfqpoint{3.105109in}{2.296555in}}%
\pgfpathlineto{\pgfqpoint{3.118488in}{2.290419in}}%
\pgfpathlineto{\pgfqpoint{3.110235in}{2.286534in}}%
\pgfpathlineto{\pgfqpoint{3.101972in}{2.282804in}}%
\pgfpathlineto{\pgfqpoint{3.093699in}{2.279234in}}%
\pgfpathlineto{\pgfqpoint{3.085415in}{2.275828in}}%
\pgfpathlineto{\pgfqpoint{3.072015in}{2.282165in}}%
\pgfpathlineto{\pgfqpoint{3.058618in}{2.288541in}}%
\pgfpathlineto{\pgfqpoint{3.045226in}{2.294955in}}%
\pgfpathlineto{\pgfqpoint{3.031838in}{2.301408in}}%
\pgfpathlineto{\pgfqpoint{3.040144in}{2.304607in}}%
\pgfpathlineto{\pgfqpoint{3.048439in}{2.307974in}}%
\pgfpathlineto{\pgfqpoint{3.056724in}{2.311504in}}%
\pgfpathlineto{\pgfqpoint{3.064998in}{2.315192in}}%
\pgfpathclose%
\pgfusepath{fill}%
\end{pgfscope}%
\begin{pgfscope}%
\pgfpathrectangle{\pgfqpoint{1.150000in}{0.150000in}}{\pgfqpoint{5.700000in}{5.700000in}}%
\pgfusepath{clip}%
\pgfsetbuttcap%
\pgfsetroundjoin%
\definecolor{currentfill}{rgb}{0.283091,0.110553,0.431554}%
\pgfsetfillcolor{currentfill}%
\pgfsetfillopacity{0.700000}%
\pgfsetlinewidth{0.000000pt}%
\definecolor{currentstroke}{rgb}{0.000000,0.000000,0.000000}%
\pgfsetstrokecolor{currentstroke}%
\pgfsetdash{}{0pt}%
\pgfpathmoveto{\pgfqpoint{5.113819in}{2.407029in}}%
\pgfpathlineto{\pgfqpoint{5.127645in}{2.405476in}}%
\pgfpathlineto{\pgfqpoint{5.141479in}{2.403948in}}%
\pgfpathlineto{\pgfqpoint{5.155321in}{2.402446in}}%
\pgfpathlineto{\pgfqpoint{5.169171in}{2.400968in}}%
\pgfpathlineto{\pgfqpoint{5.161757in}{2.394001in}}%
\pgfpathlineto{\pgfqpoint{5.154336in}{2.386973in}}%
\pgfpathlineto{\pgfqpoint{5.146909in}{2.379885in}}%
\pgfpathlineto{\pgfqpoint{5.139475in}{2.372734in}}%
\pgfpathlineto{\pgfqpoint{5.125611in}{2.374186in}}%
\pgfpathlineto{\pgfqpoint{5.111755in}{2.375663in}}%
\pgfpathlineto{\pgfqpoint{5.097907in}{2.377165in}}%
\pgfpathlineto{\pgfqpoint{5.084067in}{2.378692in}}%
\pgfpathlineto{\pgfqpoint{5.091515in}{2.385864in}}%
\pgfpathlineto{\pgfqpoint{5.098956in}{2.392977in}}%
\pgfpathlineto{\pgfqpoint{5.106391in}{2.400031in}}%
\pgfpathlineto{\pgfqpoint{5.113819in}{2.407029in}}%
\pgfpathclose%
\pgfusepath{fill}%
\end{pgfscope}%
\begin{pgfscope}%
\pgfpathrectangle{\pgfqpoint{1.150000in}{0.150000in}}{\pgfqpoint{5.700000in}{5.700000in}}%
\pgfusepath{clip}%
\pgfsetbuttcap%
\pgfsetroundjoin%
\definecolor{currentfill}{rgb}{0.267004,0.004874,0.329415}%
\pgfsetfillcolor{currentfill}%
\pgfsetfillopacity{0.700000}%
\pgfsetlinewidth{0.000000pt}%
\definecolor{currentstroke}{rgb}{0.000000,0.000000,0.000000}%
\pgfsetstrokecolor{currentstroke}%
\pgfsetdash{}{0pt}%
\pgfpathmoveto{\pgfqpoint{3.902803in}{2.223797in}}%
\pgfpathlineto{\pgfqpoint{3.916314in}{2.220077in}}%
\pgfpathlineto{\pgfqpoint{3.929831in}{2.216386in}}%
\pgfpathlineto{\pgfqpoint{3.943354in}{2.212724in}}%
\pgfpathlineto{\pgfqpoint{3.956883in}{2.209092in}}%
\pgfpathlineto{\pgfqpoint{3.949007in}{2.201351in}}%
\pgfpathlineto{\pgfqpoint{3.941125in}{2.193633in}}%
\pgfpathlineto{\pgfqpoint{3.933237in}{2.185941in}}%
\pgfpathlineto{\pgfqpoint{3.925344in}{2.178279in}}%
\pgfpathlineto{\pgfqpoint{3.911801in}{2.182033in}}%
\pgfpathlineto{\pgfqpoint{3.898265in}{2.185816in}}%
\pgfpathlineto{\pgfqpoint{3.884735in}{2.189628in}}%
\pgfpathlineto{\pgfqpoint{3.871211in}{2.193470in}}%
\pgfpathlineto{\pgfqpoint{3.879118in}{2.201006in}}%
\pgfpathlineto{\pgfqpoint{3.887019in}{2.208574in}}%
\pgfpathlineto{\pgfqpoint{3.894914in}{2.216172in}}%
\pgfpathlineto{\pgfqpoint{3.902803in}{2.223797in}}%
\pgfpathclose%
\pgfusepath{fill}%
\end{pgfscope}%
\begin{pgfscope}%
\pgfpathrectangle{\pgfqpoint{1.150000in}{0.150000in}}{\pgfqpoint{5.700000in}{5.700000in}}%
\pgfusepath{clip}%
\pgfsetbuttcap%
\pgfsetroundjoin%
\definecolor{currentfill}{rgb}{0.277941,0.056324,0.381191}%
\pgfsetfillcolor{currentfill}%
\pgfsetfillopacity{0.700000}%
\pgfsetlinewidth{0.000000pt}%
\definecolor{currentstroke}{rgb}{0.000000,0.000000,0.000000}%
\pgfsetstrokecolor{currentstroke}%
\pgfsetdash{}{0pt}%
\pgfpathmoveto{\pgfqpoint{4.578204in}{2.309276in}}%
\pgfpathlineto{\pgfqpoint{4.591881in}{2.306999in}}%
\pgfpathlineto{\pgfqpoint{4.605566in}{2.304748in}}%
\pgfpathlineto{\pgfqpoint{4.619258in}{2.302524in}}%
\pgfpathlineto{\pgfqpoint{4.632958in}{2.300326in}}%
\pgfpathlineto{\pgfqpoint{4.625327in}{2.292276in}}%
\pgfpathlineto{\pgfqpoint{4.617691in}{2.284180in}}%
\pgfpathlineto{\pgfqpoint{4.610049in}{2.276041in}}%
\pgfpathlineto{\pgfqpoint{4.602401in}{2.267858in}}%
\pgfpathlineto{\pgfqpoint{4.588690in}{2.270098in}}%
\pgfpathlineto{\pgfqpoint{4.574986in}{2.272364in}}%
\pgfpathlineto{\pgfqpoint{4.561289in}{2.274656in}}%
\pgfpathlineto{\pgfqpoint{4.547600in}{2.276974in}}%
\pgfpathlineto{\pgfqpoint{4.555259in}{2.285111in}}%
\pgfpathlineto{\pgfqpoint{4.562913in}{2.293207in}}%
\pgfpathlineto{\pgfqpoint{4.570561in}{2.301262in}}%
\pgfpathlineto{\pgfqpoint{4.578204in}{2.309276in}}%
\pgfpathclose%
\pgfusepath{fill}%
\end{pgfscope}%
\begin{pgfscope}%
\pgfpathrectangle{\pgfqpoint{1.150000in}{0.150000in}}{\pgfqpoint{5.700000in}{5.700000in}}%
\pgfusepath{clip}%
\pgfsetbuttcap%
\pgfsetroundjoin%
\definecolor{currentfill}{rgb}{0.283072,0.130895,0.449241}%
\pgfsetfillcolor{currentfill}%
\pgfsetfillopacity{0.700000}%
\pgfsetlinewidth{0.000000pt}%
\definecolor{currentstroke}{rgb}{0.000000,0.000000,0.000000}%
\pgfsetstrokecolor{currentstroke}%
\pgfsetdash{}{0pt}%
\pgfpathmoveto{\pgfqpoint{5.339092in}{2.443070in}}%
\pgfpathlineto{\pgfqpoint{5.352985in}{2.441724in}}%
\pgfpathlineto{\pgfqpoint{5.366886in}{2.440402in}}%
\pgfpathlineto{\pgfqpoint{5.380795in}{2.439105in}}%
\pgfpathlineto{\pgfqpoint{5.394713in}{2.437832in}}%
\pgfpathlineto{\pgfqpoint{5.387399in}{2.431461in}}%
\pgfpathlineto{\pgfqpoint{5.380077in}{2.425033in}}%
\pgfpathlineto{\pgfqpoint{5.372749in}{2.418547in}}%
\pgfpathlineto{\pgfqpoint{5.365414in}{2.412000in}}%
\pgfpathlineto{\pgfqpoint{5.351481in}{2.413219in}}%
\pgfpathlineto{\pgfqpoint{5.337556in}{2.414463in}}%
\pgfpathlineto{\pgfqpoint{5.323640in}{2.415732in}}%
\pgfpathlineto{\pgfqpoint{5.309732in}{2.417025in}}%
\pgfpathlineto{\pgfqpoint{5.317082in}{2.423620in}}%
\pgfpathlineto{\pgfqpoint{5.324426in}{2.430158in}}%
\pgfpathlineto{\pgfqpoint{5.331762in}{2.436640in}}%
\pgfpathlineto{\pgfqpoint{5.339092in}{2.443070in}}%
\pgfpathclose%
\pgfusepath{fill}%
\end{pgfscope}%
\begin{pgfscope}%
\pgfpathrectangle{\pgfqpoint{1.150000in}{0.150000in}}{\pgfqpoint{5.700000in}{5.700000in}}%
\pgfusepath{clip}%
\pgfsetbuttcap%
\pgfsetroundjoin%
\definecolor{currentfill}{rgb}{0.280894,0.078907,0.402329}%
\pgfsetfillcolor{currentfill}%
\pgfsetfillopacity{0.700000}%
\pgfsetlinewidth{0.000000pt}%
\definecolor{currentstroke}{rgb}{0.000000,0.000000,0.000000}%
\pgfsetstrokecolor{currentstroke}%
\pgfsetdash{}{0pt}%
\pgfpathmoveto{\pgfqpoint{4.803452in}{2.346777in}}%
\pgfpathlineto{\pgfqpoint{4.817194in}{2.344854in}}%
\pgfpathlineto{\pgfqpoint{4.830944in}{2.342958in}}%
\pgfpathlineto{\pgfqpoint{4.844701in}{2.341087in}}%
\pgfpathlineto{\pgfqpoint{4.858465in}{2.339242in}}%
\pgfpathlineto{\pgfqpoint{4.850920in}{2.331529in}}%
\pgfpathlineto{\pgfqpoint{4.843369in}{2.323760in}}%
\pgfpathlineto{\pgfqpoint{4.835811in}{2.315935in}}%
\pgfpathlineto{\pgfqpoint{4.828247in}{2.308054in}}%
\pgfpathlineto{\pgfqpoint{4.814470in}{2.309914in}}%
\pgfpathlineto{\pgfqpoint{4.800701in}{2.311800in}}%
\pgfpathlineto{\pgfqpoint{4.786939in}{2.313711in}}%
\pgfpathlineto{\pgfqpoint{4.773185in}{2.315648in}}%
\pgfpathlineto{\pgfqpoint{4.780761in}{2.323509in}}%
\pgfpathlineto{\pgfqpoint{4.788331in}{2.331318in}}%
\pgfpathlineto{\pgfqpoint{4.795895in}{2.339073in}}%
\pgfpathlineto{\pgfqpoint{4.803452in}{2.346777in}}%
\pgfpathclose%
\pgfusepath{fill}%
\end{pgfscope}%
\begin{pgfscope}%
\pgfpathrectangle{\pgfqpoint{1.150000in}{0.150000in}}{\pgfqpoint{5.700000in}{5.700000in}}%
\pgfusepath{clip}%
\pgfsetbuttcap%
\pgfsetroundjoin%
\definecolor{currentfill}{rgb}{0.282290,0.145912,0.461510}%
\pgfsetfillcolor{currentfill}%
\pgfsetfillopacity{0.700000}%
\pgfsetlinewidth{0.000000pt}%
\definecolor{currentstroke}{rgb}{0.000000,0.000000,0.000000}%
\pgfsetstrokecolor{currentstroke}%
\pgfsetdash{}{0pt}%
\pgfpathmoveto{\pgfqpoint{5.564353in}{2.476585in}}%
\pgfpathlineto{\pgfqpoint{5.578312in}{2.475386in}}%
\pgfpathlineto{\pgfqpoint{5.592280in}{2.474213in}}%
\pgfpathlineto{\pgfqpoint{5.606256in}{2.473063in}}%
\pgfpathlineto{\pgfqpoint{5.620240in}{2.471938in}}%
\pgfpathlineto{\pgfqpoint{5.613032in}{2.466188in}}%
\pgfpathlineto{\pgfqpoint{5.605817in}{2.460392in}}%
\pgfpathlineto{\pgfqpoint{5.598594in}{2.454546in}}%
\pgfpathlineto{\pgfqpoint{5.591365in}{2.448648in}}%
\pgfpathlineto{\pgfqpoint{5.577363in}{2.449692in}}%
\pgfpathlineto{\pgfqpoint{5.563370in}{2.450761in}}%
\pgfpathlineto{\pgfqpoint{5.549385in}{2.451855in}}%
\pgfpathlineto{\pgfqpoint{5.535409in}{2.452973in}}%
\pgfpathlineto{\pgfqpoint{5.542656in}{2.458946in}}%
\pgfpathlineto{\pgfqpoint{5.549895in}{2.464871in}}%
\pgfpathlineto{\pgfqpoint{5.557128in}{2.470750in}}%
\pgfpathlineto{\pgfqpoint{5.564353in}{2.476585in}}%
\pgfpathclose%
\pgfusepath{fill}%
\end{pgfscope}%
\begin{pgfscope}%
\pgfpathrectangle{\pgfqpoint{1.150000in}{0.150000in}}{\pgfqpoint{5.700000in}{5.700000in}}%
\pgfusepath{clip}%
\pgfsetbuttcap%
\pgfsetroundjoin%
\definecolor{currentfill}{rgb}{0.269944,0.014625,0.341379}%
\pgfsetfillcolor{currentfill}%
\pgfsetfillopacity{0.700000}%
\pgfsetlinewidth{0.000000pt}%
\definecolor{currentstroke}{rgb}{0.000000,0.000000,0.000000}%
\pgfsetstrokecolor{currentstroke}%
\pgfsetdash{}{0pt}%
\pgfpathmoveto{\pgfqpoint{3.398197in}{2.239117in}}%
\pgfpathlineto{\pgfqpoint{3.411614in}{2.233954in}}%
\pgfpathlineto{\pgfqpoint{3.425036in}{2.228825in}}%
\pgfpathlineto{\pgfqpoint{3.438464in}{2.223729in}}%
\pgfpathlineto{\pgfqpoint{3.451896in}{2.218667in}}%
\pgfpathlineto{\pgfqpoint{3.443809in}{2.212865in}}%
\pgfpathlineto{\pgfqpoint{3.435715in}{2.207164in}}%
\pgfpathlineto{\pgfqpoint{3.427612in}{2.201569in}}%
\pgfpathlineto{\pgfqpoint{3.419502in}{2.196084in}}%
\pgfpathlineto{\pgfqpoint{3.406052in}{2.201321in}}%
\pgfpathlineto{\pgfqpoint{3.392608in}{2.206591in}}%
\pgfpathlineto{\pgfqpoint{3.379168in}{2.211895in}}%
\pgfpathlineto{\pgfqpoint{3.365733in}{2.217233in}}%
\pgfpathlineto{\pgfqpoint{3.373861in}{2.222538in}}%
\pgfpathlineto{\pgfqpoint{3.381981in}{2.227957in}}%
\pgfpathlineto{\pgfqpoint{3.390093in}{2.233485in}}%
\pgfpathlineto{\pgfqpoint{3.398197in}{2.239117in}}%
\pgfpathclose%
\pgfusepath{fill}%
\end{pgfscope}%
\begin{pgfscope}%
\pgfpathrectangle{\pgfqpoint{1.150000in}{0.150000in}}{\pgfqpoint{5.700000in}{5.700000in}}%
\pgfusepath{clip}%
\pgfsetbuttcap%
\pgfsetroundjoin%
\definecolor{currentfill}{rgb}{0.267004,0.004874,0.329415}%
\pgfsetfillcolor{currentfill}%
\pgfsetfillopacity{0.700000}%
\pgfsetlinewidth{0.000000pt}%
\definecolor{currentstroke}{rgb}{0.000000,0.000000,0.000000}%
\pgfsetstrokecolor{currentstroke}%
\pgfsetdash{}{0pt}%
\pgfpathmoveto{\pgfqpoint{3.537878in}{2.223532in}}%
\pgfpathlineto{\pgfqpoint{3.551319in}{2.218794in}}%
\pgfpathlineto{\pgfqpoint{3.564765in}{2.214088in}}%
\pgfpathlineto{\pgfqpoint{3.578217in}{2.209415in}}%
\pgfpathlineto{\pgfqpoint{3.591674in}{2.204773in}}%
\pgfpathlineto{\pgfqpoint{3.583650in}{2.198294in}}%
\pgfpathlineto{\pgfqpoint{3.575619in}{2.191893in}}%
\pgfpathlineto{\pgfqpoint{3.567581in}{2.185575in}}%
\pgfpathlineto{\pgfqpoint{3.559535in}{2.179344in}}%
\pgfpathlineto{\pgfqpoint{3.546062in}{2.184147in}}%
\pgfpathlineto{\pgfqpoint{3.532594in}{2.188981in}}%
\pgfpathlineto{\pgfqpoint{3.519132in}{2.193847in}}%
\pgfpathlineto{\pgfqpoint{3.505674in}{2.198746in}}%
\pgfpathlineto{\pgfqpoint{3.513736in}{2.204811in}}%
\pgfpathlineto{\pgfqpoint{3.521791in}{2.210967in}}%
\pgfpathlineto{\pgfqpoint{3.529838in}{2.217208in}}%
\pgfpathlineto{\pgfqpoint{3.537878in}{2.223532in}}%
\pgfpathclose%
\pgfusepath{fill}%
\end{pgfscope}%
\begin{pgfscope}%
\pgfpathrectangle{\pgfqpoint{1.150000in}{0.150000in}}{\pgfqpoint{5.700000in}{5.700000in}}%
\pgfusepath{clip}%
\pgfsetbuttcap%
\pgfsetroundjoin%
\definecolor{currentfill}{rgb}{0.278826,0.175490,0.483397}%
\pgfsetfillcolor{currentfill}%
\pgfsetfillopacity{0.700000}%
\pgfsetlinewidth{0.000000pt}%
\definecolor{currentstroke}{rgb}{0.000000,0.000000,0.000000}%
\pgfsetstrokecolor{currentstroke}%
\pgfsetdash{}{0pt}%
\pgfpathmoveto{\pgfqpoint{6.014625in}{2.533665in}}%
\pgfpathlineto{\pgfqpoint{6.028712in}{2.532591in}}%
\pgfpathlineto{\pgfqpoint{6.042807in}{2.531541in}}%
\pgfpathlineto{\pgfqpoint{6.056910in}{2.530515in}}%
\pgfpathlineto{\pgfqpoint{6.071023in}{2.529513in}}%
\pgfpathlineto{\pgfqpoint{6.064039in}{2.524894in}}%
\pgfpathlineto{\pgfqpoint{6.057049in}{2.520265in}}%
\pgfpathlineto{\pgfqpoint{6.050053in}{2.515622in}}%
\pgfpathlineto{\pgfqpoint{6.043050in}{2.510962in}}%
\pgfpathlineto{\pgfqpoint{6.028916in}{2.511829in}}%
\pgfpathlineto{\pgfqpoint{6.014791in}{2.512720in}}%
\pgfpathlineto{\pgfqpoint{6.000674in}{2.513636in}}%
\pgfpathlineto{\pgfqpoint{5.986567in}{2.514575in}}%
\pgfpathlineto{\pgfqpoint{5.993591in}{2.519366in}}%
\pgfpathlineto{\pgfqpoint{6.000609in}{2.524141in}}%
\pgfpathlineto{\pgfqpoint{6.007620in}{2.528906in}}%
\pgfpathlineto{\pgfqpoint{6.014625in}{2.533665in}}%
\pgfpathclose%
\pgfusepath{fill}%
\end{pgfscope}%
\begin{pgfscope}%
\pgfpathrectangle{\pgfqpoint{1.150000in}{0.150000in}}{\pgfqpoint{5.700000in}{5.700000in}}%
\pgfusepath{clip}%
\pgfsetbuttcap%
\pgfsetroundjoin%
\definecolor{currentfill}{rgb}{0.280255,0.165693,0.476498}%
\pgfsetfillcolor{currentfill}%
\pgfsetfillopacity{0.700000}%
\pgfsetlinewidth{0.000000pt}%
\definecolor{currentstroke}{rgb}{0.000000,0.000000,0.000000}%
\pgfsetstrokecolor{currentstroke}%
\pgfsetdash{}{0pt}%
\pgfpathmoveto{\pgfqpoint{5.789548in}{2.506883in}}%
\pgfpathlineto{\pgfqpoint{5.803572in}{2.505775in}}%
\pgfpathlineto{\pgfqpoint{5.817605in}{2.504692in}}%
\pgfpathlineto{\pgfqpoint{5.831646in}{2.503632in}}%
\pgfpathlineto{\pgfqpoint{5.845695in}{2.502597in}}%
\pgfpathlineto{\pgfqpoint{5.838598in}{2.497448in}}%
\pgfpathlineto{\pgfqpoint{5.831494in}{2.492267in}}%
\pgfpathlineto{\pgfqpoint{5.824383in}{2.487052in}}%
\pgfpathlineto{\pgfqpoint{5.817264in}{2.481798in}}%
\pgfpathlineto{\pgfqpoint{5.803195in}{2.482726in}}%
\pgfpathlineto{\pgfqpoint{5.789135in}{2.483677in}}%
\pgfpathlineto{\pgfqpoint{5.775084in}{2.484653in}}%
\pgfpathlineto{\pgfqpoint{5.761041in}{2.485653in}}%
\pgfpathlineto{\pgfqpoint{5.768178in}{2.491009in}}%
\pgfpathlineto{\pgfqpoint{5.775308in}{2.496331in}}%
\pgfpathlineto{\pgfqpoint{5.782432in}{2.501621in}}%
\pgfpathlineto{\pgfqpoint{5.789548in}{2.506883in}}%
\pgfpathclose%
\pgfusepath{fill}%
\end{pgfscope}%
\begin{pgfscope}%
\pgfpathrectangle{\pgfqpoint{1.150000in}{0.150000in}}{\pgfqpoint{5.700000in}{5.700000in}}%
\pgfusepath{clip}%
\pgfsetbuttcap%
\pgfsetroundjoin%
\definecolor{currentfill}{rgb}{0.281446,0.084320,0.407414}%
\pgfsetfillcolor{currentfill}%
\pgfsetfillopacity{0.700000}%
\pgfsetlinewidth{0.000000pt}%
\definecolor{currentstroke}{rgb}{0.000000,0.000000,0.000000}%
\pgfsetstrokecolor{currentstroke}%
\pgfsetdash{}{0pt}%
\pgfpathmoveto{\pgfqpoint{2.924865in}{2.354473in}}%
\pgfpathlineto{\pgfqpoint{2.938224in}{2.347697in}}%
\pgfpathlineto{\pgfqpoint{2.951586in}{2.340962in}}%
\pgfpathlineto{\pgfqpoint{2.964952in}{2.334269in}}%
\pgfpathlineto{\pgfqpoint{2.978322in}{2.327616in}}%
\pgfpathlineto{\pgfqpoint{2.969982in}{2.324802in}}%
\pgfpathlineto{\pgfqpoint{2.961630in}{2.322169in}}%
\pgfpathlineto{\pgfqpoint{2.953266in}{2.319725in}}%
\pgfpathlineto{\pgfqpoint{2.944890in}{2.317474in}}%
\pgfpathlineto{\pgfqpoint{2.931497in}{2.324342in}}%
\pgfpathlineto{\pgfqpoint{2.918108in}{2.331251in}}%
\pgfpathlineto{\pgfqpoint{2.904721in}{2.338201in}}%
\pgfpathlineto{\pgfqpoint{2.891339in}{2.345193in}}%
\pgfpathlineto{\pgfqpoint{2.899739in}{2.347224in}}%
\pgfpathlineto{\pgfqpoint{2.908126in}{2.349451in}}%
\pgfpathlineto{\pgfqpoint{2.916502in}{2.351869in}}%
\pgfpathlineto{\pgfqpoint{2.924865in}{2.354473in}}%
\pgfpathclose%
\pgfusepath{fill}%
\end{pgfscope}%
\begin{pgfscope}%
\pgfpathrectangle{\pgfqpoint{1.150000in}{0.150000in}}{\pgfqpoint{5.700000in}{5.700000in}}%
\pgfusepath{clip}%
\pgfsetbuttcap%
\pgfsetroundjoin%
\definecolor{currentfill}{rgb}{0.283187,0.125848,0.444960}%
\pgfsetfillcolor{currentfill}%
\pgfsetfillopacity{0.700000}%
\pgfsetlinewidth{0.000000pt}%
\definecolor{currentstroke}{rgb}{0.000000,0.000000,0.000000}%
\pgfsetstrokecolor{currentstroke}%
\pgfsetdash{}{0pt}%
\pgfpathmoveto{\pgfqpoint{2.730998in}{2.432482in}}%
\pgfpathlineto{\pgfqpoint{2.744343in}{2.424960in}}%
\pgfpathlineto{\pgfqpoint{2.757691in}{2.417485in}}%
\pgfpathlineto{\pgfqpoint{2.771042in}{2.410056in}}%
\pgfpathlineto{\pgfqpoint{2.784395in}{2.402672in}}%
\pgfpathlineto{\pgfqpoint{2.775932in}{2.401294in}}%
\pgfpathlineto{\pgfqpoint{2.767454in}{2.400131in}}%
\pgfpathlineto{\pgfqpoint{2.758963in}{2.399188in}}%
\pgfpathlineto{\pgfqpoint{2.750457in}{2.398473in}}%
\pgfpathlineto{\pgfqpoint{2.737077in}{2.406087in}}%
\pgfpathlineto{\pgfqpoint{2.723700in}{2.413747in}}%
\pgfpathlineto{\pgfqpoint{2.710325in}{2.421452in}}%
\pgfpathlineto{\pgfqpoint{2.696953in}{2.429204in}}%
\pgfpathlineto{\pgfqpoint{2.705486in}{2.429684in}}%
\pgfpathlineto{\pgfqpoint{2.714004in}{2.430394in}}%
\pgfpathlineto{\pgfqpoint{2.722508in}{2.431328in}}%
\pgfpathlineto{\pgfqpoint{2.730998in}{2.432482in}}%
\pgfpathclose%
\pgfusepath{fill}%
\end{pgfscope}%
\begin{pgfscope}%
\pgfpathrectangle{\pgfqpoint{1.150000in}{0.150000in}}{\pgfqpoint{5.700000in}{5.700000in}}%
\pgfusepath{clip}%
\pgfsetbuttcap%
\pgfsetroundjoin%
\definecolor{currentfill}{rgb}{0.272594,0.025563,0.353093}%
\pgfsetfillcolor{currentfill}%
\pgfsetfillopacity{0.700000}%
\pgfsetlinewidth{0.000000pt}%
\definecolor{currentstroke}{rgb}{0.000000,0.000000,0.000000}%
\pgfsetstrokecolor{currentstroke}%
\pgfsetdash{}{0pt}%
\pgfpathmoveto{\pgfqpoint{4.267668in}{2.253229in}}%
\pgfpathlineto{\pgfqpoint{4.281269in}{2.250358in}}%
\pgfpathlineto{\pgfqpoint{4.294877in}{2.247515in}}%
\pgfpathlineto{\pgfqpoint{4.308492in}{2.244700in}}%
\pgfpathlineto{\pgfqpoint{4.322114in}{2.241911in}}%
\pgfpathlineto{\pgfqpoint{4.314368in}{2.233686in}}%
\pgfpathlineto{\pgfqpoint{4.306616in}{2.225442in}}%
\pgfpathlineto{\pgfqpoint{4.298858in}{2.217181in}}%
\pgfpathlineto{\pgfqpoint{4.291095in}{2.208905in}}%
\pgfpathlineto{\pgfqpoint{4.277462in}{2.211775in}}%
\pgfpathlineto{\pgfqpoint{4.263835in}{2.214672in}}%
\pgfpathlineto{\pgfqpoint{4.250215in}{2.217597in}}%
\pgfpathlineto{\pgfqpoint{4.236602in}{2.220549in}}%
\pgfpathlineto{\pgfqpoint{4.244377in}{2.228739in}}%
\pgfpathlineto{\pgfqpoint{4.252146in}{2.236917in}}%
\pgfpathlineto{\pgfqpoint{4.259910in}{2.245080in}}%
\pgfpathlineto{\pgfqpoint{4.267668in}{2.253229in}}%
\pgfpathclose%
\pgfusepath{fill}%
\end{pgfscope}%
\begin{pgfscope}%
\pgfpathrectangle{\pgfqpoint{1.150000in}{0.150000in}}{\pgfqpoint{5.700000in}{5.700000in}}%
\pgfusepath{clip}%
\pgfsetbuttcap%
\pgfsetroundjoin%
\definecolor{currentfill}{rgb}{0.272594,0.025563,0.353093}%
\pgfsetfillcolor{currentfill}%
\pgfsetfillopacity{0.700000}%
\pgfsetlinewidth{0.000000pt}%
\definecolor{currentstroke}{rgb}{0.000000,0.000000,0.000000}%
\pgfsetstrokecolor{currentstroke}%
\pgfsetdash{}{0pt}%
\pgfpathmoveto{\pgfqpoint{3.258422in}{2.261174in}}%
\pgfpathlineto{\pgfqpoint{3.271820in}{2.255559in}}%
\pgfpathlineto{\pgfqpoint{3.285222in}{2.249979in}}%
\pgfpathlineto{\pgfqpoint{3.298629in}{2.244434in}}%
\pgfpathlineto{\pgfqpoint{3.312040in}{2.238925in}}%
\pgfpathlineto{\pgfqpoint{3.303885in}{2.233920in}}%
\pgfpathlineto{\pgfqpoint{3.295720in}{2.229041in}}%
\pgfpathlineto{\pgfqpoint{3.287547in}{2.224292in}}%
\pgfpathlineto{\pgfqpoint{3.279365in}{2.219677in}}%
\pgfpathlineto{\pgfqpoint{3.265934in}{2.225375in}}%
\pgfpathlineto{\pgfqpoint{3.252508in}{2.231108in}}%
\pgfpathlineto{\pgfqpoint{3.239087in}{2.236875in}}%
\pgfpathlineto{\pgfqpoint{3.225669in}{2.242679in}}%
\pgfpathlineto{\pgfqpoint{3.233871in}{2.247100in}}%
\pgfpathlineto{\pgfqpoint{3.242064in}{2.251659in}}%
\pgfpathlineto{\pgfqpoint{3.250248in}{2.256352in}}%
\pgfpathlineto{\pgfqpoint{3.258422in}{2.261174in}}%
\pgfpathclose%
\pgfusepath{fill}%
\end{pgfscope}%
\begin{pgfscope}%
\pgfpathrectangle{\pgfqpoint{1.150000in}{0.150000in}}{\pgfqpoint{5.700000in}{5.700000in}}%
\pgfusepath{clip}%
\pgfsetbuttcap%
\pgfsetroundjoin%
\definecolor{currentfill}{rgb}{0.267004,0.004874,0.329415}%
\pgfsetfillcolor{currentfill}%
\pgfsetfillopacity{0.700000}%
\pgfsetlinewidth{0.000000pt}%
\definecolor{currentstroke}{rgb}{0.000000,0.000000,0.000000}%
\pgfsetstrokecolor{currentstroke}%
\pgfsetdash{}{0pt}%
\pgfpathmoveto{\pgfqpoint{3.677520in}{2.213746in}}%
\pgfpathlineto{\pgfqpoint{3.690989in}{2.209409in}}%
\pgfpathlineto{\pgfqpoint{3.704464in}{2.205103in}}%
\pgfpathlineto{\pgfqpoint{3.717944in}{2.200827in}}%
\pgfpathlineto{\pgfqpoint{3.731430in}{2.196582in}}%
\pgfpathlineto{\pgfqpoint{3.723463in}{2.189538in}}%
\pgfpathlineto{\pgfqpoint{3.715490in}{2.182552in}}%
\pgfpathlineto{\pgfqpoint{3.707511in}{2.175627in}}%
\pgfpathlineto{\pgfqpoint{3.699524in}{2.168767in}}%
\pgfpathlineto{\pgfqpoint{3.686024in}{2.173160in}}%
\pgfpathlineto{\pgfqpoint{3.672529in}{2.177583in}}%
\pgfpathlineto{\pgfqpoint{3.659039in}{2.182037in}}%
\pgfpathlineto{\pgfqpoint{3.645555in}{2.186522in}}%
\pgfpathlineto{\pgfqpoint{3.653557in}{2.193230in}}%
\pgfpathlineto{\pgfqpoint{3.661551in}{2.200005in}}%
\pgfpathlineto{\pgfqpoint{3.669539in}{2.206845in}}%
\pgfpathlineto{\pgfqpoint{3.677520in}{2.213746in}}%
\pgfpathclose%
\pgfusepath{fill}%
\end{pgfscope}%
\begin{pgfscope}%
\pgfpathrectangle{\pgfqpoint{1.150000in}{0.150000in}}{\pgfqpoint{5.700000in}{5.700000in}}%
\pgfusepath{clip}%
\pgfsetbuttcap%
\pgfsetroundjoin%
\definecolor{currentfill}{rgb}{0.268510,0.009605,0.335427}%
\pgfsetfillcolor{currentfill}%
\pgfsetfillopacity{0.700000}%
\pgfsetlinewidth{0.000000pt}%
\definecolor{currentstroke}{rgb}{0.000000,0.000000,0.000000}%
\pgfsetstrokecolor{currentstroke}%
\pgfsetdash{}{0pt}%
\pgfpathmoveto{\pgfqpoint{4.042454in}{2.226428in}}%
\pgfpathlineto{\pgfqpoint{4.056002in}{2.223046in}}%
\pgfpathlineto{\pgfqpoint{4.069555in}{2.219694in}}%
\pgfpathlineto{\pgfqpoint{4.083115in}{2.216369in}}%
\pgfpathlineto{\pgfqpoint{4.096682in}{2.213073in}}%
\pgfpathlineto{\pgfqpoint{4.088854in}{2.205057in}}%
\pgfpathlineto{\pgfqpoint{4.081021in}{2.197049in}}%
\pgfpathlineto{\pgfqpoint{4.073182in}{2.189049in}}%
\pgfpathlineto{\pgfqpoint{4.065338in}{2.181061in}}%
\pgfpathlineto{\pgfqpoint{4.051759in}{2.184466in}}%
\pgfpathlineto{\pgfqpoint{4.038186in}{2.187898in}}%
\pgfpathlineto{\pgfqpoint{4.024620in}{2.191359in}}%
\pgfpathlineto{\pgfqpoint{4.011060in}{2.194849in}}%
\pgfpathlineto{\pgfqpoint{4.018917in}{2.202724in}}%
\pgfpathlineto{\pgfqpoint{4.026769in}{2.210613in}}%
\pgfpathlineto{\pgfqpoint{4.034614in}{2.218515in}}%
\pgfpathlineto{\pgfqpoint{4.042454in}{2.226428in}}%
\pgfpathclose%
\pgfusepath{fill}%
\end{pgfscope}%
\begin{pgfscope}%
\pgfpathrectangle{\pgfqpoint{1.150000in}{0.150000in}}{\pgfqpoint{5.700000in}{5.700000in}}%
\pgfusepath{clip}%
\pgfsetbuttcap%
\pgfsetroundjoin%
\definecolor{currentfill}{rgb}{0.282656,0.100196,0.422160}%
\pgfsetfillcolor{currentfill}%
\pgfsetfillopacity{0.700000}%
\pgfsetlinewidth{0.000000pt}%
\definecolor{currentstroke}{rgb}{0.000000,0.000000,0.000000}%
\pgfsetstrokecolor{currentstroke}%
\pgfsetdash{}{0pt}%
\pgfpathmoveto{\pgfqpoint{5.028787in}{2.385052in}}%
\pgfpathlineto{\pgfqpoint{5.042595in}{2.383425in}}%
\pgfpathlineto{\pgfqpoint{5.056411in}{2.381822in}}%
\pgfpathlineto{\pgfqpoint{5.070235in}{2.380245in}}%
\pgfpathlineto{\pgfqpoint{5.084067in}{2.378692in}}%
\pgfpathlineto{\pgfqpoint{5.076613in}{2.371461in}}%
\pgfpathlineto{\pgfqpoint{5.069152in}{2.364168in}}%
\pgfpathlineto{\pgfqpoint{5.061684in}{2.356813in}}%
\pgfpathlineto{\pgfqpoint{5.054209in}{2.349395in}}%
\pgfpathlineto{\pgfqpoint{5.040364in}{2.350935in}}%
\pgfpathlineto{\pgfqpoint{5.026527in}{2.352500in}}%
\pgfpathlineto{\pgfqpoint{5.012697in}{2.354090in}}%
\pgfpathlineto{\pgfqpoint{4.998876in}{2.355706in}}%
\pgfpathlineto{\pgfqpoint{5.006363in}{2.363131in}}%
\pgfpathlineto{\pgfqpoint{5.013844in}{2.370497in}}%
\pgfpathlineto{\pgfqpoint{5.021319in}{2.377803in}}%
\pgfpathlineto{\pgfqpoint{5.028787in}{2.385052in}}%
\pgfpathclose%
\pgfusepath{fill}%
\end{pgfscope}%
\begin{pgfscope}%
\pgfpathrectangle{\pgfqpoint{1.150000in}{0.150000in}}{\pgfqpoint{5.700000in}{5.700000in}}%
\pgfusepath{clip}%
\pgfsetbuttcap%
\pgfsetroundjoin%
\definecolor{currentfill}{rgb}{0.276022,0.044167,0.370164}%
\pgfsetfillcolor{currentfill}%
\pgfsetfillopacity{0.700000}%
\pgfsetlinewidth{0.000000pt}%
\definecolor{currentstroke}{rgb}{0.000000,0.000000,0.000000}%
\pgfsetstrokecolor{currentstroke}%
\pgfsetdash{}{0pt}%
\pgfpathmoveto{\pgfqpoint{4.492914in}{2.286511in}}%
\pgfpathlineto{\pgfqpoint{4.506575in}{2.284087in}}%
\pgfpathlineto{\pgfqpoint{4.520242in}{2.281690in}}%
\pgfpathlineto{\pgfqpoint{4.533917in}{2.279319in}}%
\pgfpathlineto{\pgfqpoint{4.547600in}{2.276974in}}%
\pgfpathlineto{\pgfqpoint{4.539934in}{2.268798in}}%
\pgfpathlineto{\pgfqpoint{4.532263in}{2.260584in}}%
\pgfpathlineto{\pgfqpoint{4.524585in}{2.252331in}}%
\pgfpathlineto{\pgfqpoint{4.516902in}{2.244042in}}%
\pgfpathlineto{\pgfqpoint{4.503209in}{2.246442in}}%
\pgfpathlineto{\pgfqpoint{4.489522in}{2.248868in}}%
\pgfpathlineto{\pgfqpoint{4.475842in}{2.251320in}}%
\pgfpathlineto{\pgfqpoint{4.462170in}{2.253799in}}%
\pgfpathlineto{\pgfqpoint{4.469865in}{2.262029in}}%
\pgfpathlineto{\pgfqpoint{4.477553in}{2.270224in}}%
\pgfpathlineto{\pgfqpoint{4.485236in}{2.278386in}}%
\pgfpathlineto{\pgfqpoint{4.492914in}{2.286511in}}%
\pgfpathclose%
\pgfusepath{fill}%
\end{pgfscope}%
\begin{pgfscope}%
\pgfpathrectangle{\pgfqpoint{1.150000in}{0.150000in}}{\pgfqpoint{5.700000in}{5.700000in}}%
\pgfusepath{clip}%
\pgfsetbuttcap%
\pgfsetroundjoin%
\definecolor{currentfill}{rgb}{0.283229,0.120777,0.440584}%
\pgfsetfillcolor{currentfill}%
\pgfsetfillopacity{0.700000}%
\pgfsetlinewidth{0.000000pt}%
\definecolor{currentstroke}{rgb}{0.000000,0.000000,0.000000}%
\pgfsetstrokecolor{currentstroke}%
\pgfsetdash{}{0pt}%
\pgfpathmoveto{\pgfqpoint{5.254180in}{2.422447in}}%
\pgfpathlineto{\pgfqpoint{5.268056in}{2.421054in}}%
\pgfpathlineto{\pgfqpoint{5.281940in}{2.419687in}}%
\pgfpathlineto{\pgfqpoint{5.295832in}{2.418344in}}%
\pgfpathlineto{\pgfqpoint{5.309732in}{2.417025in}}%
\pgfpathlineto{\pgfqpoint{5.302374in}{2.410372in}}%
\pgfpathlineto{\pgfqpoint{5.295010in}{2.403658in}}%
\pgfpathlineto{\pgfqpoint{5.287639in}{2.396882in}}%
\pgfpathlineto{\pgfqpoint{5.280260in}{2.390042in}}%
\pgfpathlineto{\pgfqpoint{5.266346in}{2.391321in}}%
\pgfpathlineto{\pgfqpoint{5.252439in}{2.392624in}}%
\pgfpathlineto{\pgfqpoint{5.238541in}{2.393953in}}%
\pgfpathlineto{\pgfqpoint{5.224651in}{2.395306in}}%
\pgfpathlineto{\pgfqpoint{5.232043in}{2.402180in}}%
\pgfpathlineto{\pgfqpoint{5.239429in}{2.408994in}}%
\pgfpathlineto{\pgfqpoint{5.246808in}{2.415749in}}%
\pgfpathlineto{\pgfqpoint{5.254180in}{2.422447in}}%
\pgfpathclose%
\pgfusepath{fill}%
\end{pgfscope}%
\begin{pgfscope}%
\pgfpathrectangle{\pgfqpoint{1.150000in}{0.150000in}}{\pgfqpoint{5.700000in}{5.700000in}}%
\pgfusepath{clip}%
\pgfsetbuttcap%
\pgfsetroundjoin%
\definecolor{currentfill}{rgb}{0.267004,0.004874,0.329415}%
\pgfsetfillcolor{currentfill}%
\pgfsetfillopacity{0.700000}%
\pgfsetlinewidth{0.000000pt}%
\definecolor{currentstroke}{rgb}{0.000000,0.000000,0.000000}%
\pgfsetstrokecolor{currentstroke}%
\pgfsetdash{}{0pt}%
\pgfpathmoveto{\pgfqpoint{3.817174in}{2.209130in}}%
\pgfpathlineto{\pgfqpoint{3.830675in}{2.205171in}}%
\pgfpathlineto{\pgfqpoint{3.844181in}{2.201241in}}%
\pgfpathlineto{\pgfqpoint{3.857693in}{2.197340in}}%
\pgfpathlineto{\pgfqpoint{3.871211in}{2.193470in}}%
\pgfpathlineto{\pgfqpoint{3.863298in}{2.185968in}}%
\pgfpathlineto{\pgfqpoint{3.855379in}{2.178505in}}%
\pgfpathlineto{\pgfqpoint{3.847454in}{2.171084in}}%
\pgfpathlineto{\pgfqpoint{3.839522in}{2.163707in}}%
\pgfpathlineto{\pgfqpoint{3.825990in}{2.167712in}}%
\pgfpathlineto{\pgfqpoint{3.812464in}{2.171747in}}%
\pgfpathlineto{\pgfqpoint{3.798944in}{2.175811in}}%
\pgfpathlineto{\pgfqpoint{3.785429in}{2.179905in}}%
\pgfpathlineto{\pgfqpoint{3.793375in}{2.187143in}}%
\pgfpathlineto{\pgfqpoint{3.801314in}{2.194429in}}%
\pgfpathlineto{\pgfqpoint{3.809247in}{2.201759in}}%
\pgfpathlineto{\pgfqpoint{3.817174in}{2.209130in}}%
\pgfpathclose%
\pgfusepath{fill}%
\end{pgfscope}%
\begin{pgfscope}%
\pgfpathrectangle{\pgfqpoint{1.150000in}{0.150000in}}{\pgfqpoint{5.700000in}{5.700000in}}%
\pgfusepath{clip}%
\pgfsetbuttcap%
\pgfsetroundjoin%
\definecolor{currentfill}{rgb}{0.277018,0.050344,0.375715}%
\pgfsetfillcolor{currentfill}%
\pgfsetfillopacity{0.700000}%
\pgfsetlinewidth{0.000000pt}%
\definecolor{currentstroke}{rgb}{0.000000,0.000000,0.000000}%
\pgfsetstrokecolor{currentstroke}%
\pgfsetdash{}{0pt}%
\pgfpathmoveto{\pgfqpoint{3.118488in}{2.290419in}}%
\pgfpathlineto{\pgfqpoint{3.131871in}{2.284322in}}%
\pgfpathlineto{\pgfqpoint{3.145258in}{2.278262in}}%
\pgfpathlineto{\pgfqpoint{3.158649in}{2.272239in}}%
\pgfpathlineto{\pgfqpoint{3.172045in}{2.266254in}}%
\pgfpathlineto{\pgfqpoint{3.163813in}{2.262172in}}%
\pgfpathlineto{\pgfqpoint{3.155571in}{2.258242in}}%
\pgfpathlineto{\pgfqpoint{3.147319in}{2.254469in}}%
\pgfpathlineto{\pgfqpoint{3.139056in}{2.250857in}}%
\pgfpathlineto{\pgfqpoint{3.125640in}{2.257043in}}%
\pgfpathlineto{\pgfqpoint{3.112227in}{2.263268in}}%
\pgfpathlineto{\pgfqpoint{3.098819in}{2.269529in}}%
\pgfpathlineto{\pgfqpoint{3.085415in}{2.275828in}}%
\pgfpathlineto{\pgfqpoint{3.093699in}{2.279234in}}%
\pgfpathlineto{\pgfqpoint{3.101972in}{2.282804in}}%
\pgfpathlineto{\pgfqpoint{3.110235in}{2.286534in}}%
\pgfpathlineto{\pgfqpoint{3.118488in}{2.290419in}}%
\pgfpathclose%
\pgfusepath{fill}%
\end{pgfscope}%
\begin{pgfscope}%
\pgfpathrectangle{\pgfqpoint{1.150000in}{0.150000in}}{\pgfqpoint{5.700000in}{5.700000in}}%
\pgfusepath{clip}%
\pgfsetbuttcap%
\pgfsetroundjoin%
\definecolor{currentfill}{rgb}{0.279566,0.067836,0.391917}%
\pgfsetfillcolor{currentfill}%
\pgfsetfillopacity{0.700000}%
\pgfsetlinewidth{0.000000pt}%
\definecolor{currentstroke}{rgb}{0.000000,0.000000,0.000000}%
\pgfsetstrokecolor{currentstroke}%
\pgfsetdash{}{0pt}%
\pgfpathmoveto{\pgfqpoint{4.718243in}{2.323653in}}%
\pgfpathlineto{\pgfqpoint{4.731967in}{2.321613in}}%
\pgfpathlineto{\pgfqpoint{4.745699in}{2.319599in}}%
\pgfpathlineto{\pgfqpoint{4.759438in}{2.317610in}}%
\pgfpathlineto{\pgfqpoint{4.773185in}{2.315648in}}%
\pgfpathlineto{\pgfqpoint{4.765602in}{2.307734in}}%
\pgfpathlineto{\pgfqpoint{4.758014in}{2.299767in}}%
\pgfpathlineto{\pgfqpoint{4.750419in}{2.291748in}}%
\pgfpathlineto{\pgfqpoint{4.742818in}{2.283676in}}%
\pgfpathlineto{\pgfqpoint{4.729060in}{2.285667in}}%
\pgfpathlineto{\pgfqpoint{4.715309in}{2.287683in}}%
\pgfpathlineto{\pgfqpoint{4.701565in}{2.289726in}}%
\pgfpathlineto{\pgfqpoint{4.687829in}{2.291794in}}%
\pgfpathlineto{\pgfqpoint{4.695441in}{2.299832in}}%
\pgfpathlineto{\pgfqpoint{4.703048in}{2.307822in}}%
\pgfpathlineto{\pgfqpoint{4.710648in}{2.315762in}}%
\pgfpathlineto{\pgfqpoint{4.718243in}{2.323653in}}%
\pgfpathclose%
\pgfusepath{fill}%
\end{pgfscope}%
\begin{pgfscope}%
\pgfpathrectangle{\pgfqpoint{1.150000in}{0.150000in}}{\pgfqpoint{5.700000in}{5.700000in}}%
\pgfusepath{clip}%
\pgfsetbuttcap%
\pgfsetroundjoin%
\definecolor{currentfill}{rgb}{0.278826,0.175490,0.483397}%
\pgfsetfillcolor{currentfill}%
\pgfsetfillopacity{0.700000}%
\pgfsetlinewidth{0.000000pt}%
\definecolor{currentstroke}{rgb}{0.000000,0.000000,0.000000}%
\pgfsetstrokecolor{currentstroke}%
\pgfsetdash{}{0pt}%
\pgfpathmoveto{\pgfqpoint{2.536688in}{2.526029in}}%
\pgfpathlineto{\pgfqpoint{2.550030in}{2.517681in}}%
\pgfpathlineto{\pgfqpoint{2.563375in}{2.509386in}}%
\pgfpathlineto{\pgfqpoint{2.576722in}{2.501143in}}%
\pgfpathlineto{\pgfqpoint{2.590072in}{2.492951in}}%
\pgfpathlineto{\pgfqpoint{2.581467in}{2.493187in}}%
\pgfpathlineto{\pgfqpoint{2.572846in}{2.493674in}}%
\pgfpathlineto{\pgfqpoint{2.564208in}{2.494416in}}%
\pgfpathlineto{\pgfqpoint{2.555554in}{2.495421in}}%
\pgfpathlineto{\pgfqpoint{2.542175in}{2.503859in}}%
\pgfpathlineto{\pgfqpoint{2.528798in}{2.512348in}}%
\pgfpathlineto{\pgfqpoint{2.515423in}{2.520889in}}%
\pgfpathlineto{\pgfqpoint{2.502050in}{2.529482in}}%
\pgfpathlineto{\pgfqpoint{2.510735in}{2.528226in}}%
\pgfpathlineto{\pgfqpoint{2.519402in}{2.527236in}}%
\pgfpathlineto{\pgfqpoint{2.528053in}{2.526506in}}%
\pgfpathlineto{\pgfqpoint{2.536688in}{2.526029in}}%
\pgfpathclose%
\pgfusepath{fill}%
\end{pgfscope}%
\begin{pgfscope}%
\pgfpathrectangle{\pgfqpoint{1.150000in}{0.150000in}}{\pgfqpoint{5.700000in}{5.700000in}}%
\pgfusepath{clip}%
\pgfsetbuttcap%
\pgfsetroundjoin%
\definecolor{currentfill}{rgb}{0.282623,0.140926,0.457517}%
\pgfsetfillcolor{currentfill}%
\pgfsetfillopacity{0.700000}%
\pgfsetlinewidth{0.000000pt}%
\definecolor{currentstroke}{rgb}{0.000000,0.000000,0.000000}%
\pgfsetstrokecolor{currentstroke}%
\pgfsetdash{}{0pt}%
\pgfpathmoveto{\pgfqpoint{5.479588in}{2.457690in}}%
\pgfpathlineto{\pgfqpoint{5.493531in}{2.456474in}}%
\pgfpathlineto{\pgfqpoint{5.507482in}{2.455282in}}%
\pgfpathlineto{\pgfqpoint{5.521441in}{2.454115in}}%
\pgfpathlineto{\pgfqpoint{5.535409in}{2.452973in}}%
\pgfpathlineto{\pgfqpoint{5.528156in}{2.446948in}}%
\pgfpathlineto{\pgfqpoint{5.520895in}{2.440869in}}%
\pgfpathlineto{\pgfqpoint{5.513627in}{2.434733in}}%
\pgfpathlineto{\pgfqpoint{5.506352in}{2.428540in}}%
\pgfpathlineto{\pgfqpoint{5.492367in}{2.429615in}}%
\pgfpathlineto{\pgfqpoint{5.478392in}{2.430715in}}%
\pgfpathlineto{\pgfqpoint{5.464424in}{2.431840in}}%
\pgfpathlineto{\pgfqpoint{5.450465in}{2.432989in}}%
\pgfpathlineto{\pgfqpoint{5.457757in}{2.439245in}}%
\pgfpathlineto{\pgfqpoint{5.465041in}{2.445445in}}%
\pgfpathlineto{\pgfqpoint{5.472318in}{2.451593in}}%
\pgfpathlineto{\pgfqpoint{5.479588in}{2.457690in}}%
\pgfpathclose%
\pgfusepath{fill}%
\end{pgfscope}%
\begin{pgfscope}%
\pgfpathrectangle{\pgfqpoint{1.150000in}{0.150000in}}{\pgfqpoint{5.700000in}{5.700000in}}%
\pgfusepath{clip}%
\pgfsetbuttcap%
\pgfsetroundjoin%
\definecolor{currentfill}{rgb}{0.269944,0.014625,0.341379}%
\pgfsetfillcolor{currentfill}%
\pgfsetfillopacity{0.700000}%
\pgfsetlinewidth{0.000000pt}%
\definecolor{currentstroke}{rgb}{0.000000,0.000000,0.000000}%
\pgfsetstrokecolor{currentstroke}%
\pgfsetdash{}{0pt}%
\pgfpathmoveto{\pgfqpoint{4.182216in}{2.232631in}}%
\pgfpathlineto{\pgfqpoint{4.195802in}{2.229569in}}%
\pgfpathlineto{\pgfqpoint{4.209396in}{2.226535in}}%
\pgfpathlineto{\pgfqpoint{4.222995in}{2.223528in}}%
\pgfpathlineto{\pgfqpoint{4.236602in}{2.220549in}}%
\pgfpathlineto{\pgfqpoint{4.228821in}{2.212349in}}%
\pgfpathlineto{\pgfqpoint{4.221035in}{2.204140in}}%
\pgfpathlineto{\pgfqpoint{4.213243in}{2.195925in}}%
\pgfpathlineto{\pgfqpoint{4.205446in}{2.187705in}}%
\pgfpathlineto{\pgfqpoint{4.191827in}{2.190779in}}%
\pgfpathlineto{\pgfqpoint{4.178216in}{2.193881in}}%
\pgfpathlineto{\pgfqpoint{4.164610in}{2.197010in}}%
\pgfpathlineto{\pgfqpoint{4.151012in}{2.200167in}}%
\pgfpathlineto{\pgfqpoint{4.158821in}{2.208287in}}%
\pgfpathlineto{\pgfqpoint{4.166625in}{2.216405in}}%
\pgfpathlineto{\pgfqpoint{4.174423in}{2.224521in}}%
\pgfpathlineto{\pgfqpoint{4.182216in}{2.232631in}}%
\pgfpathclose%
\pgfusepath{fill}%
\end{pgfscope}%
\begin{pgfscope}%
\pgfpathrectangle{\pgfqpoint{1.150000in}{0.150000in}}{\pgfqpoint{5.700000in}{5.700000in}}%
\pgfusepath{clip}%
\pgfsetbuttcap%
\pgfsetroundjoin%
\definecolor{currentfill}{rgb}{0.280868,0.160771,0.472899}%
\pgfsetfillcolor{currentfill}%
\pgfsetfillopacity{0.700000}%
\pgfsetlinewidth{0.000000pt}%
\definecolor{currentstroke}{rgb}{0.000000,0.000000,0.000000}%
\pgfsetstrokecolor{currentstroke}%
\pgfsetdash{}{0pt}%
\pgfpathmoveto{\pgfqpoint{5.704955in}{2.489896in}}%
\pgfpathlineto{\pgfqpoint{5.718963in}{2.488799in}}%
\pgfpathlineto{\pgfqpoint{5.732981in}{2.487726in}}%
\pgfpathlineto{\pgfqpoint{5.747006in}{2.486678in}}%
\pgfpathlineto{\pgfqpoint{5.761041in}{2.485653in}}%
\pgfpathlineto{\pgfqpoint{5.753897in}{2.480259in}}%
\pgfpathlineto{\pgfqpoint{5.746746in}{2.474823in}}%
\pgfpathlineto{\pgfqpoint{5.739587in}{2.469343in}}%
\pgfpathlineto{\pgfqpoint{5.732422in}{2.463815in}}%
\pgfpathlineto{\pgfqpoint{5.718369in}{2.464745in}}%
\pgfpathlineto{\pgfqpoint{5.704325in}{2.465700in}}%
\pgfpathlineto{\pgfqpoint{5.690289in}{2.466679in}}%
\pgfpathlineto{\pgfqpoint{5.676263in}{2.467682in}}%
\pgfpathlineto{\pgfqpoint{5.683446in}{2.473299in}}%
\pgfpathlineto{\pgfqpoint{5.690623in}{2.478871in}}%
\pgfpathlineto{\pgfqpoint{5.697792in}{2.484403in}}%
\pgfpathlineto{\pgfqpoint{5.704955in}{2.489896in}}%
\pgfpathclose%
\pgfusepath{fill}%
\end{pgfscope}%
\begin{pgfscope}%
\pgfpathrectangle{\pgfqpoint{1.150000in}{0.150000in}}{\pgfqpoint{5.700000in}{5.700000in}}%
\pgfusepath{clip}%
\pgfsetbuttcap%
\pgfsetroundjoin%
\definecolor{currentfill}{rgb}{0.282327,0.094955,0.417331}%
\pgfsetfillcolor{currentfill}%
\pgfsetfillopacity{0.700000}%
\pgfsetlinewidth{0.000000pt}%
\definecolor{currentstroke}{rgb}{0.000000,0.000000,0.000000}%
\pgfsetstrokecolor{currentstroke}%
\pgfsetdash{}{0pt}%
\pgfpathmoveto{\pgfqpoint{4.943667in}{2.362421in}}%
\pgfpathlineto{\pgfqpoint{4.957458in}{2.360704in}}%
\pgfpathlineto{\pgfqpoint{4.971256in}{2.359013in}}%
\pgfpathlineto{\pgfqpoint{4.985062in}{2.357347in}}%
\pgfpathlineto{\pgfqpoint{4.998876in}{2.355706in}}%
\pgfpathlineto{\pgfqpoint{4.991382in}{2.348221in}}%
\pgfpathlineto{\pgfqpoint{4.983881in}{2.340674in}}%
\pgfpathlineto{\pgfqpoint{4.976374in}{2.333067in}}%
\pgfpathlineto{\pgfqpoint{4.968860in}{2.325397in}}%
\pgfpathlineto{\pgfqpoint{4.955033in}{2.327039in}}%
\pgfpathlineto{\pgfqpoint{4.941215in}{2.328706in}}%
\pgfpathlineto{\pgfqpoint{4.927404in}{2.330398in}}%
\pgfpathlineto{\pgfqpoint{4.913601in}{2.332116in}}%
\pgfpathlineto{\pgfqpoint{4.921127in}{2.339780in}}%
\pgfpathlineto{\pgfqpoint{4.928647in}{2.347385in}}%
\pgfpathlineto{\pgfqpoint{4.936160in}{2.354931in}}%
\pgfpathlineto{\pgfqpoint{4.943667in}{2.362421in}}%
\pgfpathclose%
\pgfusepath{fill}%
\end{pgfscope}%
\begin{pgfscope}%
\pgfpathrectangle{\pgfqpoint{1.150000in}{0.150000in}}{\pgfqpoint{5.700000in}{5.700000in}}%
\pgfusepath{clip}%
\pgfsetbuttcap%
\pgfsetroundjoin%
\definecolor{currentfill}{rgb}{0.278826,0.175490,0.483397}%
\pgfsetfillcolor{currentfill}%
\pgfsetfillopacity{0.700000}%
\pgfsetlinewidth{0.000000pt}%
\definecolor{currentstroke}{rgb}{0.000000,0.000000,0.000000}%
\pgfsetstrokecolor{currentstroke}%
\pgfsetdash{}{0pt}%
\pgfpathmoveto{\pgfqpoint{5.930224in}{2.518573in}}%
\pgfpathlineto{\pgfqpoint{5.944297in}{2.517538in}}%
\pgfpathlineto{\pgfqpoint{5.958378in}{2.516526in}}%
\pgfpathlineto{\pgfqpoint{5.972468in}{2.515539in}}%
\pgfpathlineto{\pgfqpoint{5.986567in}{2.514575in}}%
\pgfpathlineto{\pgfqpoint{5.979537in}{2.509766in}}%
\pgfpathlineto{\pgfqpoint{5.972499in}{2.504934in}}%
\pgfpathlineto{\pgfqpoint{5.965456in}{2.500075in}}%
\pgfpathlineto{\pgfqpoint{5.958405in}{2.495186in}}%
\pgfpathlineto{\pgfqpoint{5.944286in}{2.496028in}}%
\pgfpathlineto{\pgfqpoint{5.930175in}{2.496894in}}%
\pgfpathlineto{\pgfqpoint{5.916073in}{2.497784in}}%
\pgfpathlineto{\pgfqpoint{5.901980in}{2.498698in}}%
\pgfpathlineto{\pgfqpoint{5.909051in}{2.503704in}}%
\pgfpathlineto{\pgfqpoint{5.916115in}{2.508683in}}%
\pgfpathlineto{\pgfqpoint{5.923173in}{2.513638in}}%
\pgfpathlineto{\pgfqpoint{5.930224in}{2.518573in}}%
\pgfpathclose%
\pgfusepath{fill}%
\end{pgfscope}%
\begin{pgfscope}%
\pgfpathrectangle{\pgfqpoint{1.150000in}{0.150000in}}{\pgfqpoint{5.700000in}{5.700000in}}%
\pgfusepath{clip}%
\pgfsetbuttcap%
\pgfsetroundjoin%
\definecolor{currentfill}{rgb}{0.274952,0.037752,0.364543}%
\pgfsetfillcolor{currentfill}%
\pgfsetfillopacity{0.700000}%
\pgfsetlinewidth{0.000000pt}%
\definecolor{currentstroke}{rgb}{0.000000,0.000000,0.000000}%
\pgfsetstrokecolor{currentstroke}%
\pgfsetdash{}{0pt}%
\pgfpathmoveto{\pgfqpoint{4.407551in}{2.263981in}}%
\pgfpathlineto{\pgfqpoint{4.421195in}{2.261396in}}%
\pgfpathlineto{\pgfqpoint{4.434846in}{2.258837in}}%
\pgfpathlineto{\pgfqpoint{4.448505in}{2.256305in}}%
\pgfpathlineto{\pgfqpoint{4.462170in}{2.253799in}}%
\pgfpathlineto{\pgfqpoint{4.454470in}{2.245538in}}%
\pgfpathlineto{\pgfqpoint{4.446764in}{2.237245in}}%
\pgfpathlineto{\pgfqpoint{4.439052in}{2.228922in}}%
\pgfpathlineto{\pgfqpoint{4.431335in}{2.220571in}}%
\pgfpathlineto{\pgfqpoint{4.417658in}{2.223145in}}%
\pgfpathlineto{\pgfqpoint{4.403988in}{2.225746in}}%
\pgfpathlineto{\pgfqpoint{4.390325in}{2.228373in}}%
\pgfpathlineto{\pgfqpoint{4.376669in}{2.231027in}}%
\pgfpathlineto{\pgfqpoint{4.384398in}{2.239305in}}%
\pgfpathlineto{\pgfqpoint{4.392121in}{2.247557in}}%
\pgfpathlineto{\pgfqpoint{4.399839in}{2.255783in}}%
\pgfpathlineto{\pgfqpoint{4.407551in}{2.263981in}}%
\pgfpathclose%
\pgfusepath{fill}%
\end{pgfscope}%
\begin{pgfscope}%
\pgfpathrectangle{\pgfqpoint{1.150000in}{0.150000in}}{\pgfqpoint{5.700000in}{5.700000in}}%
\pgfusepath{clip}%
\pgfsetbuttcap%
\pgfsetroundjoin%
\definecolor{currentfill}{rgb}{0.267004,0.004874,0.329415}%
\pgfsetfillcolor{currentfill}%
\pgfsetfillopacity{0.700000}%
\pgfsetlinewidth{0.000000pt}%
\definecolor{currentstroke}{rgb}{0.000000,0.000000,0.000000}%
\pgfsetstrokecolor{currentstroke}%
\pgfsetdash{}{0pt}%
\pgfpathmoveto{\pgfqpoint{3.956883in}{2.209092in}}%
\pgfpathlineto{\pgfqpoint{3.970418in}{2.205488in}}%
\pgfpathlineto{\pgfqpoint{3.983959in}{2.201913in}}%
\pgfpathlineto{\pgfqpoint{3.997507in}{2.198366in}}%
\pgfpathlineto{\pgfqpoint{4.011060in}{2.194849in}}%
\pgfpathlineto{\pgfqpoint{4.003197in}{2.186991in}}%
\pgfpathlineto{\pgfqpoint{3.995328in}{2.179154in}}%
\pgfpathlineto{\pgfqpoint{3.987454in}{2.171339in}}%
\pgfpathlineto{\pgfqpoint{3.979573in}{2.163550in}}%
\pgfpathlineto{\pgfqpoint{3.966007in}{2.167190in}}%
\pgfpathlineto{\pgfqpoint{3.952446in}{2.170857in}}%
\pgfpathlineto{\pgfqpoint{3.938892in}{2.174554in}}%
\pgfpathlineto{\pgfqpoint{3.925344in}{2.178279in}}%
\pgfpathlineto{\pgfqpoint{3.933237in}{2.185941in}}%
\pgfpathlineto{\pgfqpoint{3.941125in}{2.193633in}}%
\pgfpathlineto{\pgfqpoint{3.949007in}{2.201351in}}%
\pgfpathlineto{\pgfqpoint{3.956883in}{2.209092in}}%
\pgfpathclose%
\pgfusepath{fill}%
\end{pgfscope}%
\begin{pgfscope}%
\pgfpathrectangle{\pgfqpoint{1.150000in}{0.150000in}}{\pgfqpoint{5.700000in}{5.700000in}}%
\pgfusepath{clip}%
\pgfsetbuttcap%
\pgfsetroundjoin%
\definecolor{currentfill}{rgb}{0.283197,0.115680,0.436115}%
\pgfsetfillcolor{currentfill}%
\pgfsetfillopacity{0.700000}%
\pgfsetlinewidth{0.000000pt}%
\definecolor{currentstroke}{rgb}{0.000000,0.000000,0.000000}%
\pgfsetstrokecolor{currentstroke}%
\pgfsetdash{}{0pt}%
\pgfpathmoveto{\pgfqpoint{2.784395in}{2.402672in}}%
\pgfpathlineto{\pgfqpoint{2.797752in}{2.395334in}}%
\pgfpathlineto{\pgfqpoint{2.811112in}{2.388040in}}%
\pgfpathlineto{\pgfqpoint{2.824475in}{2.380790in}}%
\pgfpathlineto{\pgfqpoint{2.837841in}{2.373585in}}%
\pgfpathlineto{\pgfqpoint{2.829403in}{2.371982in}}%
\pgfpathlineto{\pgfqpoint{2.820952in}{2.370590in}}%
\pgfpathlineto{\pgfqpoint{2.812487in}{2.369416in}}%
\pgfpathlineto{\pgfqpoint{2.804008in}{2.368465in}}%
\pgfpathlineto{\pgfqpoint{2.790616in}{2.375901in}}%
\pgfpathlineto{\pgfqpoint{2.777227in}{2.383380in}}%
\pgfpathlineto{\pgfqpoint{2.763841in}{2.390904in}}%
\pgfpathlineto{\pgfqpoint{2.750457in}{2.398473in}}%
\pgfpathlineto{\pgfqpoint{2.758963in}{2.399188in}}%
\pgfpathlineto{\pgfqpoint{2.767454in}{2.400131in}}%
\pgfpathlineto{\pgfqpoint{2.775932in}{2.401294in}}%
\pgfpathlineto{\pgfqpoint{2.784395in}{2.402672in}}%
\pgfpathclose%
\pgfusepath{fill}%
\end{pgfscope}%
\begin{pgfscope}%
\pgfpathrectangle{\pgfqpoint{1.150000in}{0.150000in}}{\pgfqpoint{5.700000in}{5.700000in}}%
\pgfusepath{clip}%
\pgfsetbuttcap%
\pgfsetroundjoin%
\definecolor{currentfill}{rgb}{0.280894,0.078907,0.402329}%
\pgfsetfillcolor{currentfill}%
\pgfsetfillopacity{0.700000}%
\pgfsetlinewidth{0.000000pt}%
\definecolor{currentstroke}{rgb}{0.000000,0.000000,0.000000}%
\pgfsetstrokecolor{currentstroke}%
\pgfsetdash{}{0pt}%
\pgfpathmoveto{\pgfqpoint{2.978322in}{2.327616in}}%
\pgfpathlineto{\pgfqpoint{2.991695in}{2.321004in}}%
\pgfpathlineto{\pgfqpoint{3.005072in}{2.314433in}}%
\pgfpathlineto{\pgfqpoint{3.018453in}{2.307901in}}%
\pgfpathlineto{\pgfqpoint{3.031838in}{2.301408in}}%
\pgfpathlineto{\pgfqpoint{3.023520in}{2.298383in}}%
\pgfpathlineto{\pgfqpoint{3.015192in}{2.295537in}}%
\pgfpathlineto{\pgfqpoint{3.006852in}{2.292876in}}%
\pgfpathlineto{\pgfqpoint{2.998500in}{2.290404in}}%
\pgfpathlineto{\pgfqpoint{2.985092in}{2.297112in}}%
\pgfpathlineto{\pgfqpoint{2.971688in}{2.303859in}}%
\pgfpathlineto{\pgfqpoint{2.958287in}{2.310646in}}%
\pgfpathlineto{\pgfqpoint{2.944890in}{2.317474in}}%
\pgfpathlineto{\pgfqpoint{2.953266in}{2.319725in}}%
\pgfpathlineto{\pgfqpoint{2.961630in}{2.322169in}}%
\pgfpathlineto{\pgfqpoint{2.969982in}{2.324802in}}%
\pgfpathlineto{\pgfqpoint{2.978322in}{2.327616in}}%
\pgfpathclose%
\pgfusepath{fill}%
\end{pgfscope}%
\begin{pgfscope}%
\pgfpathrectangle{\pgfqpoint{1.150000in}{0.150000in}}{\pgfqpoint{5.700000in}{5.700000in}}%
\pgfusepath{clip}%
\pgfsetbuttcap%
\pgfsetroundjoin%
\definecolor{currentfill}{rgb}{0.278791,0.062145,0.386592}%
\pgfsetfillcolor{currentfill}%
\pgfsetfillopacity{0.700000}%
\pgfsetlinewidth{0.000000pt}%
\definecolor{currentstroke}{rgb}{0.000000,0.000000,0.000000}%
\pgfsetstrokecolor{currentstroke}%
\pgfsetdash{}{0pt}%
\pgfpathmoveto{\pgfqpoint{4.632958in}{2.300326in}}%
\pgfpathlineto{\pgfqpoint{4.646664in}{2.298154in}}%
\pgfpathlineto{\pgfqpoint{4.660378in}{2.296008in}}%
\pgfpathlineto{\pgfqpoint{4.674100in}{2.293888in}}%
\pgfpathlineto{\pgfqpoint{4.687829in}{2.291794in}}%
\pgfpathlineto{\pgfqpoint{4.680210in}{2.283707in}}%
\pgfpathlineto{\pgfqpoint{4.672586in}{2.275572in}}%
\pgfpathlineto{\pgfqpoint{4.664955in}{2.267389in}}%
\pgfpathlineto{\pgfqpoint{4.657318in}{2.259159in}}%
\pgfpathlineto{\pgfqpoint{4.643578in}{2.261295in}}%
\pgfpathlineto{\pgfqpoint{4.629845in}{2.263457in}}%
\pgfpathlineto{\pgfqpoint{4.616119in}{2.265644in}}%
\pgfpathlineto{\pgfqpoint{4.602401in}{2.267858in}}%
\pgfpathlineto{\pgfqpoint{4.610049in}{2.276041in}}%
\pgfpathlineto{\pgfqpoint{4.617691in}{2.284180in}}%
\pgfpathlineto{\pgfqpoint{4.625327in}{2.292276in}}%
\pgfpathlineto{\pgfqpoint{4.632958in}{2.300326in}}%
\pgfpathclose%
\pgfusepath{fill}%
\end{pgfscope}%
\begin{pgfscope}%
\pgfpathrectangle{\pgfqpoint{1.150000in}{0.150000in}}{\pgfqpoint{5.700000in}{5.700000in}}%
\pgfusepath{clip}%
\pgfsetbuttcap%
\pgfsetroundjoin%
\definecolor{currentfill}{rgb}{0.283197,0.115680,0.436115}%
\pgfsetfillcolor{currentfill}%
\pgfsetfillopacity{0.700000}%
\pgfsetlinewidth{0.000000pt}%
\definecolor{currentstroke}{rgb}{0.000000,0.000000,0.000000}%
\pgfsetstrokecolor{currentstroke}%
\pgfsetdash{}{0pt}%
\pgfpathmoveto{\pgfqpoint{5.169171in}{2.400968in}}%
\pgfpathlineto{\pgfqpoint{5.183029in}{2.399515in}}%
\pgfpathlineto{\pgfqpoint{5.196895in}{2.398087in}}%
\pgfpathlineto{\pgfqpoint{5.210769in}{2.396684in}}%
\pgfpathlineto{\pgfqpoint{5.224651in}{2.395306in}}%
\pgfpathlineto{\pgfqpoint{5.217251in}{2.388370in}}%
\pgfpathlineto{\pgfqpoint{5.209844in}{2.381370in}}%
\pgfpathlineto{\pgfqpoint{5.202431in}{2.374306in}}%
\pgfpathlineto{\pgfqpoint{5.195010in}{2.367177in}}%
\pgfpathlineto{\pgfqpoint{5.181114in}{2.368529in}}%
\pgfpathlineto{\pgfqpoint{5.167226in}{2.369906in}}%
\pgfpathlineto{\pgfqpoint{5.153346in}{2.371308in}}%
\pgfpathlineto{\pgfqpoint{5.139475in}{2.372734in}}%
\pgfpathlineto{\pgfqpoint{5.146909in}{2.379885in}}%
\pgfpathlineto{\pgfqpoint{5.154336in}{2.386973in}}%
\pgfpathlineto{\pgfqpoint{5.161757in}{2.394001in}}%
\pgfpathlineto{\pgfqpoint{5.169171in}{2.400968in}}%
\pgfpathclose%
\pgfusepath{fill}%
\end{pgfscope}%
\begin{pgfscope}%
\pgfpathrectangle{\pgfqpoint{1.150000in}{0.150000in}}{\pgfqpoint{5.700000in}{5.700000in}}%
\pgfusepath{clip}%
\pgfsetbuttcap%
\pgfsetroundjoin%
\definecolor{currentfill}{rgb}{0.268510,0.009605,0.335427}%
\pgfsetfillcolor{currentfill}%
\pgfsetfillopacity{0.700000}%
\pgfsetlinewidth{0.000000pt}%
\definecolor{currentstroke}{rgb}{0.000000,0.000000,0.000000}%
\pgfsetstrokecolor{currentstroke}%
\pgfsetdash{}{0pt}%
\pgfpathmoveto{\pgfqpoint{3.451896in}{2.218667in}}%
\pgfpathlineto{\pgfqpoint{3.465333in}{2.213637in}}%
\pgfpathlineto{\pgfqpoint{3.478775in}{2.208641in}}%
\pgfpathlineto{\pgfqpoint{3.492222in}{2.203677in}}%
\pgfpathlineto{\pgfqpoint{3.505674in}{2.198746in}}%
\pgfpathlineto{\pgfqpoint{3.497605in}{2.192775in}}%
\pgfpathlineto{\pgfqpoint{3.489528in}{2.186902in}}%
\pgfpathlineto{\pgfqpoint{3.481443in}{2.181131in}}%
\pgfpathlineto{\pgfqpoint{3.473350in}{2.175466in}}%
\pgfpathlineto{\pgfqpoint{3.459880in}{2.180572in}}%
\pgfpathlineto{\pgfqpoint{3.446416in}{2.185710in}}%
\pgfpathlineto{\pgfqpoint{3.432956in}{2.190881in}}%
\pgfpathlineto{\pgfqpoint{3.419502in}{2.196084in}}%
\pgfpathlineto{\pgfqpoint{3.427612in}{2.201569in}}%
\pgfpathlineto{\pgfqpoint{3.435715in}{2.207164in}}%
\pgfpathlineto{\pgfqpoint{3.443809in}{2.212865in}}%
\pgfpathlineto{\pgfqpoint{3.451896in}{2.218667in}}%
\pgfpathclose%
\pgfusepath{fill}%
\end{pgfscope}%
\begin{pgfscope}%
\pgfpathrectangle{\pgfqpoint{1.150000in}{0.150000in}}{\pgfqpoint{5.700000in}{5.700000in}}%
\pgfusepath{clip}%
\pgfsetbuttcap%
\pgfsetroundjoin%
\definecolor{currentfill}{rgb}{0.267004,0.004874,0.329415}%
\pgfsetfillcolor{currentfill}%
\pgfsetfillopacity{0.700000}%
\pgfsetlinewidth{0.000000pt}%
\definecolor{currentstroke}{rgb}{0.000000,0.000000,0.000000}%
\pgfsetstrokecolor{currentstroke}%
\pgfsetdash{}{0pt}%
\pgfpathmoveto{\pgfqpoint{3.591674in}{2.204773in}}%
\pgfpathlineto{\pgfqpoint{3.605136in}{2.200164in}}%
\pgfpathlineto{\pgfqpoint{3.618604in}{2.195585in}}%
\pgfpathlineto{\pgfqpoint{3.632077in}{2.191038in}}%
\pgfpathlineto{\pgfqpoint{3.645555in}{2.186522in}}%
\pgfpathlineto{\pgfqpoint{3.637547in}{2.179887in}}%
\pgfpathlineto{\pgfqpoint{3.629532in}{2.173327in}}%
\pgfpathlineto{\pgfqpoint{3.621510in}{2.166846in}}%
\pgfpathlineto{\pgfqpoint{3.613480in}{2.160448in}}%
\pgfpathlineto{\pgfqpoint{3.599986in}{2.165125in}}%
\pgfpathlineto{\pgfqpoint{3.586497in}{2.169833in}}%
\pgfpathlineto{\pgfqpoint{3.573013in}{2.174573in}}%
\pgfpathlineto{\pgfqpoint{3.559535in}{2.179344in}}%
\pgfpathlineto{\pgfqpoint{3.567581in}{2.185575in}}%
\pgfpathlineto{\pgfqpoint{3.575619in}{2.191893in}}%
\pgfpathlineto{\pgfqpoint{3.583650in}{2.198294in}}%
\pgfpathlineto{\pgfqpoint{3.591674in}{2.204773in}}%
\pgfpathclose%
\pgfusepath{fill}%
\end{pgfscope}%
\begin{pgfscope}%
\pgfpathrectangle{\pgfqpoint{1.150000in}{0.150000in}}{\pgfqpoint{5.700000in}{5.700000in}}%
\pgfusepath{clip}%
\pgfsetbuttcap%
\pgfsetroundjoin%
\definecolor{currentfill}{rgb}{0.272594,0.025563,0.353093}%
\pgfsetfillcolor{currentfill}%
\pgfsetfillopacity{0.700000}%
\pgfsetlinewidth{0.000000pt}%
\definecolor{currentstroke}{rgb}{0.000000,0.000000,0.000000}%
\pgfsetstrokecolor{currentstroke}%
\pgfsetdash{}{0pt}%
\pgfpathmoveto{\pgfqpoint{3.312040in}{2.238925in}}%
\pgfpathlineto{\pgfqpoint{3.325456in}{2.233450in}}%
\pgfpathlineto{\pgfqpoint{3.338877in}{2.228010in}}%
\pgfpathlineto{\pgfqpoint{3.352302in}{2.222604in}}%
\pgfpathlineto{\pgfqpoint{3.365733in}{2.217233in}}%
\pgfpathlineto{\pgfqpoint{3.357596in}{2.212045in}}%
\pgfpathlineto{\pgfqpoint{3.349450in}{2.206980in}}%
\pgfpathlineto{\pgfqpoint{3.341296in}{2.202041in}}%
\pgfpathlineto{\pgfqpoint{3.333133in}{2.197234in}}%
\pgfpathlineto{\pgfqpoint{3.319684in}{2.202793in}}%
\pgfpathlineto{\pgfqpoint{3.306240in}{2.208387in}}%
\pgfpathlineto{\pgfqpoint{3.292800in}{2.214015in}}%
\pgfpathlineto{\pgfqpoint{3.279365in}{2.219677in}}%
\pgfpathlineto{\pgfqpoint{3.287547in}{2.224292in}}%
\pgfpathlineto{\pgfqpoint{3.295720in}{2.229041in}}%
\pgfpathlineto{\pgfqpoint{3.303885in}{2.233920in}}%
\pgfpathlineto{\pgfqpoint{3.312040in}{2.238925in}}%
\pgfpathclose%
\pgfusepath{fill}%
\end{pgfscope}%
\begin{pgfscope}%
\pgfpathrectangle{\pgfqpoint{1.150000in}{0.150000in}}{\pgfqpoint{5.700000in}{5.700000in}}%
\pgfusepath{clip}%
\pgfsetbuttcap%
\pgfsetroundjoin%
\definecolor{currentfill}{rgb}{0.280255,0.165693,0.476498}%
\pgfsetfillcolor{currentfill}%
\pgfsetfillopacity{0.700000}%
\pgfsetlinewidth{0.000000pt}%
\definecolor{currentstroke}{rgb}{0.000000,0.000000,0.000000}%
\pgfsetstrokecolor{currentstroke}%
\pgfsetdash{}{0pt}%
\pgfpathmoveto{\pgfqpoint{2.590072in}{2.492951in}}%
\pgfpathlineto{\pgfqpoint{2.603423in}{2.484810in}}%
\pgfpathlineto{\pgfqpoint{2.616777in}{2.476719in}}%
\pgfpathlineto{\pgfqpoint{2.630133in}{2.468678in}}%
\pgfpathlineto{\pgfqpoint{2.643492in}{2.460687in}}%
\pgfpathlineto{\pgfqpoint{2.634916in}{2.460683in}}%
\pgfpathlineto{\pgfqpoint{2.626324in}{2.460926in}}%
\pgfpathlineto{\pgfqpoint{2.617717in}{2.461422in}}%
\pgfpathlineto{\pgfqpoint{2.609093in}{2.462176in}}%
\pgfpathlineto{\pgfqpoint{2.595705in}{2.470413in}}%
\pgfpathlineto{\pgfqpoint{2.582319in}{2.478699in}}%
\pgfpathlineto{\pgfqpoint{2.568935in}{2.487035in}}%
\pgfpathlineto{\pgfqpoint{2.555554in}{2.495421in}}%
\pgfpathlineto{\pgfqpoint{2.564208in}{2.494416in}}%
\pgfpathlineto{\pgfqpoint{2.572846in}{2.493674in}}%
\pgfpathlineto{\pgfqpoint{2.581467in}{2.493187in}}%
\pgfpathlineto{\pgfqpoint{2.590072in}{2.492951in}}%
\pgfpathclose%
\pgfusepath{fill}%
\end{pgfscope}%
\begin{pgfscope}%
\pgfpathrectangle{\pgfqpoint{1.150000in}{0.150000in}}{\pgfqpoint{5.700000in}{5.700000in}}%
\pgfusepath{clip}%
\pgfsetbuttcap%
\pgfsetroundjoin%
\definecolor{currentfill}{rgb}{0.282884,0.135920,0.453427}%
\pgfsetfillcolor{currentfill}%
\pgfsetfillopacity{0.700000}%
\pgfsetlinewidth{0.000000pt}%
\definecolor{currentstroke}{rgb}{0.000000,0.000000,0.000000}%
\pgfsetstrokecolor{currentstroke}%
\pgfsetdash{}{0pt}%
\pgfpathmoveto{\pgfqpoint{5.394713in}{2.437832in}}%
\pgfpathlineto{\pgfqpoint{5.408638in}{2.436585in}}%
\pgfpathlineto{\pgfqpoint{5.422572in}{2.435361in}}%
\pgfpathlineto{\pgfqpoint{5.436515in}{2.434163in}}%
\pgfpathlineto{\pgfqpoint{5.450465in}{2.432989in}}%
\pgfpathlineto{\pgfqpoint{5.443167in}{2.426676in}}%
\pgfpathlineto{\pgfqpoint{5.435861in}{2.420303in}}%
\pgfpathlineto{\pgfqpoint{5.428549in}{2.413868in}}%
\pgfpathlineto{\pgfqpoint{5.421228in}{2.407370in}}%
\pgfpathlineto{\pgfqpoint{5.407262in}{2.408491in}}%
\pgfpathlineto{\pgfqpoint{5.393305in}{2.409636in}}%
\pgfpathlineto{\pgfqpoint{5.379355in}{2.410806in}}%
\pgfpathlineto{\pgfqpoint{5.365414in}{2.412000in}}%
\pgfpathlineto{\pgfqpoint{5.372749in}{2.418547in}}%
\pgfpathlineto{\pgfqpoint{5.380077in}{2.425033in}}%
\pgfpathlineto{\pgfqpoint{5.387399in}{2.431461in}}%
\pgfpathlineto{\pgfqpoint{5.394713in}{2.437832in}}%
\pgfpathclose%
\pgfusepath{fill}%
\end{pgfscope}%
\begin{pgfscope}%
\pgfpathrectangle{\pgfqpoint{1.150000in}{0.150000in}}{\pgfqpoint{5.700000in}{5.700000in}}%
\pgfusepath{clip}%
\pgfsetbuttcap%
\pgfsetroundjoin%
\definecolor{currentfill}{rgb}{0.277134,0.185228,0.489898}%
\pgfsetfillcolor{currentfill}%
\pgfsetfillopacity{0.700000}%
\pgfsetlinewidth{0.000000pt}%
\definecolor{currentstroke}{rgb}{0.000000,0.000000,0.000000}%
\pgfsetstrokecolor{currentstroke}%
\pgfsetdash{}{0pt}%
\pgfpathmoveto{\pgfqpoint{6.071023in}{2.529513in}}%
\pgfpathlineto{\pgfqpoint{6.085144in}{2.528535in}}%
\pgfpathlineto{\pgfqpoint{6.099274in}{2.527580in}}%
\pgfpathlineto{\pgfqpoint{6.113413in}{2.526650in}}%
\pgfpathlineto{\pgfqpoint{6.106445in}{2.522136in}}%
\pgfpathlineto{\pgfqpoint{6.099472in}{2.517610in}}%
\pgfpathlineto{\pgfqpoint{6.092492in}{2.513067in}}%
\pgfpathlineto{\pgfqpoint{6.085505in}{2.508504in}}%
\pgfpathlineto{\pgfqpoint{6.071344in}{2.509299in}}%
\pgfpathlineto{\pgfqpoint{6.057193in}{2.510119in}}%
\pgfpathlineto{\pgfqpoint{6.043050in}{2.510962in}}%
\pgfpathlineto{\pgfqpoint{6.050053in}{2.515622in}}%
\pgfpathlineto{\pgfqpoint{6.057049in}{2.520265in}}%
\pgfpathlineto{\pgfqpoint{6.064039in}{2.524894in}}%
\pgfpathlineto{\pgfqpoint{6.071023in}{2.529513in}}%
\pgfpathclose%
\pgfusepath{fill}%
\end{pgfscope}%
\begin{pgfscope}%
\pgfpathrectangle{\pgfqpoint{1.150000in}{0.150000in}}{\pgfqpoint{5.700000in}{5.700000in}}%
\pgfusepath{clip}%
\pgfsetbuttcap%
\pgfsetroundjoin%
\definecolor{currentfill}{rgb}{0.267004,0.004874,0.329415}%
\pgfsetfillcolor{currentfill}%
\pgfsetfillopacity{0.700000}%
\pgfsetlinewidth{0.000000pt}%
\definecolor{currentstroke}{rgb}{0.000000,0.000000,0.000000}%
\pgfsetstrokecolor{currentstroke}%
\pgfsetdash{}{0pt}%
\pgfpathmoveto{\pgfqpoint{3.731430in}{2.196582in}}%
\pgfpathlineto{\pgfqpoint{3.744921in}{2.192368in}}%
\pgfpathlineto{\pgfqpoint{3.758418in}{2.188184in}}%
\pgfpathlineto{\pgfqpoint{3.771921in}{2.184029in}}%
\pgfpathlineto{\pgfqpoint{3.785429in}{2.179905in}}%
\pgfpathlineto{\pgfqpoint{3.777478in}{2.172719in}}%
\pgfpathlineto{\pgfqpoint{3.769519in}{2.165587in}}%
\pgfpathlineto{\pgfqpoint{3.761554in}{2.158512in}}%
\pgfpathlineto{\pgfqpoint{3.753583in}{2.151499in}}%
\pgfpathlineto{\pgfqpoint{3.740060in}{2.155771in}}%
\pgfpathlineto{\pgfqpoint{3.726542in}{2.160073in}}%
\pgfpathlineto{\pgfqpoint{3.713031in}{2.164405in}}%
\pgfpathlineto{\pgfqpoint{3.699524in}{2.168767in}}%
\pgfpathlineto{\pgfqpoint{3.707511in}{2.175627in}}%
\pgfpathlineto{\pgfqpoint{3.715490in}{2.182552in}}%
\pgfpathlineto{\pgfqpoint{3.723463in}{2.189538in}}%
\pgfpathlineto{\pgfqpoint{3.731430in}{2.196582in}}%
\pgfpathclose%
\pgfusepath{fill}%
\end{pgfscope}%
\begin{pgfscope}%
\pgfpathrectangle{\pgfqpoint{1.150000in}{0.150000in}}{\pgfqpoint{5.700000in}{5.700000in}}%
\pgfusepath{clip}%
\pgfsetbuttcap%
\pgfsetroundjoin%
\definecolor{currentfill}{rgb}{0.281446,0.084320,0.407414}%
\pgfsetfillcolor{currentfill}%
\pgfsetfillopacity{0.700000}%
\pgfsetlinewidth{0.000000pt}%
\definecolor{currentstroke}{rgb}{0.000000,0.000000,0.000000}%
\pgfsetstrokecolor{currentstroke}%
\pgfsetdash{}{0pt}%
\pgfpathmoveto{\pgfqpoint{4.858465in}{2.339242in}}%
\pgfpathlineto{\pgfqpoint{4.872238in}{2.337422in}}%
\pgfpathlineto{\pgfqpoint{4.886018in}{2.335628in}}%
\pgfpathlineto{\pgfqpoint{4.899805in}{2.333859in}}%
\pgfpathlineto{\pgfqpoint{4.913601in}{2.332116in}}%
\pgfpathlineto{\pgfqpoint{4.906068in}{2.324394in}}%
\pgfpathlineto{\pgfqpoint{4.898529in}{2.316612in}}%
\pgfpathlineto{\pgfqpoint{4.890983in}{2.308771in}}%
\pgfpathlineto{\pgfqpoint{4.883431in}{2.300870in}}%
\pgfpathlineto{\pgfqpoint{4.869624in}{2.302628in}}%
\pgfpathlineto{\pgfqpoint{4.855824in}{2.304411in}}%
\pgfpathlineto{\pgfqpoint{4.842031in}{2.306220in}}%
\pgfpathlineto{\pgfqpoint{4.828247in}{2.308054in}}%
\pgfpathlineto{\pgfqpoint{4.835811in}{2.315935in}}%
\pgfpathlineto{\pgfqpoint{4.843369in}{2.323760in}}%
\pgfpathlineto{\pgfqpoint{4.850920in}{2.331529in}}%
\pgfpathlineto{\pgfqpoint{4.858465in}{2.339242in}}%
\pgfpathclose%
\pgfusepath{fill}%
\end{pgfscope}%
\begin{pgfscope}%
\pgfpathrectangle{\pgfqpoint{1.150000in}{0.150000in}}{\pgfqpoint{5.700000in}{5.700000in}}%
\pgfusepath{clip}%
\pgfsetbuttcap%
\pgfsetroundjoin%
\definecolor{currentfill}{rgb}{0.268510,0.009605,0.335427}%
\pgfsetfillcolor{currentfill}%
\pgfsetfillopacity{0.700000}%
\pgfsetlinewidth{0.000000pt}%
\definecolor{currentstroke}{rgb}{0.000000,0.000000,0.000000}%
\pgfsetstrokecolor{currentstroke}%
\pgfsetdash{}{0pt}%
\pgfpathmoveto{\pgfqpoint{4.096682in}{2.213073in}}%
\pgfpathlineto{\pgfqpoint{4.110254in}{2.209804in}}%
\pgfpathlineto{\pgfqpoint{4.123834in}{2.206564in}}%
\pgfpathlineto{\pgfqpoint{4.137419in}{2.203352in}}%
\pgfpathlineto{\pgfqpoint{4.151012in}{2.200167in}}%
\pgfpathlineto{\pgfqpoint{4.143196in}{2.192049in}}%
\pgfpathlineto{\pgfqpoint{4.135376in}{2.183934in}}%
\pgfpathlineto{\pgfqpoint{4.127549in}{2.175824in}}%
\pgfpathlineto{\pgfqpoint{4.119717in}{2.167723in}}%
\pgfpathlineto{\pgfqpoint{4.106113in}{2.171016in}}%
\pgfpathlineto{\pgfqpoint{4.092515in}{2.174336in}}%
\pgfpathlineto{\pgfqpoint{4.078923in}{2.177685in}}%
\pgfpathlineto{\pgfqpoint{4.065338in}{2.181061in}}%
\pgfpathlineto{\pgfqpoint{4.073182in}{2.189049in}}%
\pgfpathlineto{\pgfqpoint{4.081021in}{2.197049in}}%
\pgfpathlineto{\pgfqpoint{4.088854in}{2.205057in}}%
\pgfpathlineto{\pgfqpoint{4.096682in}{2.213073in}}%
\pgfpathclose%
\pgfusepath{fill}%
\end{pgfscope}%
\begin{pgfscope}%
\pgfpathrectangle{\pgfqpoint{1.150000in}{0.150000in}}{\pgfqpoint{5.700000in}{5.700000in}}%
\pgfusepath{clip}%
\pgfsetbuttcap%
\pgfsetroundjoin%
\definecolor{currentfill}{rgb}{0.272594,0.025563,0.353093}%
\pgfsetfillcolor{currentfill}%
\pgfsetfillopacity{0.700000}%
\pgfsetlinewidth{0.000000pt}%
\definecolor{currentstroke}{rgb}{0.000000,0.000000,0.000000}%
\pgfsetstrokecolor{currentstroke}%
\pgfsetdash{}{0pt}%
\pgfpathmoveto{\pgfqpoint{4.322114in}{2.241911in}}%
\pgfpathlineto{\pgfqpoint{4.335742in}{2.239149in}}%
\pgfpathlineto{\pgfqpoint{4.349378in}{2.236415in}}%
\pgfpathlineto{\pgfqpoint{4.363020in}{2.233707in}}%
\pgfpathlineto{\pgfqpoint{4.376669in}{2.231027in}}%
\pgfpathlineto{\pgfqpoint{4.368934in}{2.222725in}}%
\pgfpathlineto{\pgfqpoint{4.361194in}{2.214401in}}%
\pgfpathlineto{\pgfqpoint{4.353448in}{2.206057in}}%
\pgfpathlineto{\pgfqpoint{4.345697in}{2.197694in}}%
\pgfpathlineto{\pgfqpoint{4.332036in}{2.200456in}}%
\pgfpathlineto{\pgfqpoint{4.318382in}{2.203245in}}%
\pgfpathlineto{\pgfqpoint{4.304735in}{2.206061in}}%
\pgfpathlineto{\pgfqpoint{4.291095in}{2.208905in}}%
\pgfpathlineto{\pgfqpoint{4.298858in}{2.217181in}}%
\pgfpathlineto{\pgfqpoint{4.306616in}{2.225442in}}%
\pgfpathlineto{\pgfqpoint{4.314368in}{2.233686in}}%
\pgfpathlineto{\pgfqpoint{4.322114in}{2.241911in}}%
\pgfpathclose%
\pgfusepath{fill}%
\end{pgfscope}%
\begin{pgfscope}%
\pgfpathrectangle{\pgfqpoint{1.150000in}{0.150000in}}{\pgfqpoint{5.700000in}{5.700000in}}%
\pgfusepath{clip}%
\pgfsetbuttcap%
\pgfsetroundjoin%
\definecolor{currentfill}{rgb}{0.276022,0.044167,0.370164}%
\pgfsetfillcolor{currentfill}%
\pgfsetfillopacity{0.700000}%
\pgfsetlinewidth{0.000000pt}%
\definecolor{currentstroke}{rgb}{0.000000,0.000000,0.000000}%
\pgfsetstrokecolor{currentstroke}%
\pgfsetdash{}{0pt}%
\pgfpathmoveto{\pgfqpoint{3.172045in}{2.266254in}}%
\pgfpathlineto{\pgfqpoint{3.185444in}{2.260306in}}%
\pgfpathlineto{\pgfqpoint{3.198848in}{2.254394in}}%
\pgfpathlineto{\pgfqpoint{3.212257in}{2.248518in}}%
\pgfpathlineto{\pgfqpoint{3.225669in}{2.242679in}}%
\pgfpathlineto{\pgfqpoint{3.217458in}{2.238401in}}%
\pgfpathlineto{\pgfqpoint{3.209237in}{2.234271in}}%
\pgfpathlineto{\pgfqpoint{3.201005in}{2.230294in}}%
\pgfpathlineto{\pgfqpoint{3.192764in}{2.226475in}}%
\pgfpathlineto{\pgfqpoint{3.179331in}{2.232516in}}%
\pgfpathlineto{\pgfqpoint{3.165902in}{2.238593in}}%
\pgfpathlineto{\pgfqpoint{3.152477in}{2.244707in}}%
\pgfpathlineto{\pgfqpoint{3.139056in}{2.250857in}}%
\pgfpathlineto{\pgfqpoint{3.147319in}{2.254469in}}%
\pgfpathlineto{\pgfqpoint{3.155571in}{2.258242in}}%
\pgfpathlineto{\pgfqpoint{3.163813in}{2.262172in}}%
\pgfpathlineto{\pgfqpoint{3.172045in}{2.266254in}}%
\pgfpathclose%
\pgfusepath{fill}%
\end{pgfscope}%
\begin{pgfscope}%
\pgfpathrectangle{\pgfqpoint{1.150000in}{0.150000in}}{\pgfqpoint{5.700000in}{5.700000in}}%
\pgfusepath{clip}%
\pgfsetbuttcap%
\pgfsetroundjoin%
\definecolor{currentfill}{rgb}{0.281412,0.155834,0.469201}%
\pgfsetfillcolor{currentfill}%
\pgfsetfillopacity{0.700000}%
\pgfsetlinewidth{0.000000pt}%
\definecolor{currentstroke}{rgb}{0.000000,0.000000,0.000000}%
\pgfsetstrokecolor{currentstroke}%
\pgfsetdash{}{0pt}%
\pgfpathmoveto{\pgfqpoint{5.620240in}{2.471938in}}%
\pgfpathlineto{\pgfqpoint{5.634233in}{2.470837in}}%
\pgfpathlineto{\pgfqpoint{5.648234in}{2.469761in}}%
\pgfpathlineto{\pgfqpoint{5.662244in}{2.468709in}}%
\pgfpathlineto{\pgfqpoint{5.676263in}{2.467682in}}%
\pgfpathlineto{\pgfqpoint{5.669072in}{2.462018in}}%
\pgfpathlineto{\pgfqpoint{5.661874in}{2.456303in}}%
\pgfpathlineto{\pgfqpoint{5.654669in}{2.450536in}}%
\pgfpathlineto{\pgfqpoint{5.647456in}{2.444714in}}%
\pgfpathlineto{\pgfqpoint{5.633420in}{2.445661in}}%
\pgfpathlineto{\pgfqpoint{5.619393in}{2.446632in}}%
\pgfpathlineto{\pgfqpoint{5.605375in}{2.447628in}}%
\pgfpathlineto{\pgfqpoint{5.591365in}{2.448648in}}%
\pgfpathlineto{\pgfqpoint{5.598594in}{2.454546in}}%
\pgfpathlineto{\pgfqpoint{5.605817in}{2.460392in}}%
\pgfpathlineto{\pgfqpoint{5.613032in}{2.466188in}}%
\pgfpathlineto{\pgfqpoint{5.620240in}{2.471938in}}%
\pgfpathclose%
\pgfusepath{fill}%
\end{pgfscope}%
\begin{pgfscope}%
\pgfpathrectangle{\pgfqpoint{1.150000in}{0.150000in}}{\pgfqpoint{5.700000in}{5.700000in}}%
\pgfusepath{clip}%
\pgfsetbuttcap%
\pgfsetroundjoin%
\definecolor{currentfill}{rgb}{0.279574,0.170599,0.479997}%
\pgfsetfillcolor{currentfill}%
\pgfsetfillopacity{0.700000}%
\pgfsetlinewidth{0.000000pt}%
\definecolor{currentstroke}{rgb}{0.000000,0.000000,0.000000}%
\pgfsetstrokecolor{currentstroke}%
\pgfsetdash{}{0pt}%
\pgfpathmoveto{\pgfqpoint{5.845695in}{2.502597in}}%
\pgfpathlineto{\pgfqpoint{5.859754in}{2.501586in}}%
\pgfpathlineto{\pgfqpoint{5.873821in}{2.500599in}}%
\pgfpathlineto{\pgfqpoint{5.887896in}{2.499637in}}%
\pgfpathlineto{\pgfqpoint{5.901980in}{2.498698in}}%
\pgfpathlineto{\pgfqpoint{5.894903in}{2.493662in}}%
\pgfpathlineto{\pgfqpoint{5.887818in}{2.488591in}}%
\pgfpathlineto{\pgfqpoint{5.880726in}{2.483482in}}%
\pgfpathlineto{\pgfqpoint{5.873627in}{2.478331in}}%
\pgfpathlineto{\pgfqpoint{5.859523in}{2.479161in}}%
\pgfpathlineto{\pgfqpoint{5.845428in}{2.480016in}}%
\pgfpathlineto{\pgfqpoint{5.831342in}{2.480895in}}%
\pgfpathlineto{\pgfqpoint{5.817264in}{2.481798in}}%
\pgfpathlineto{\pgfqpoint{5.824383in}{2.487052in}}%
\pgfpathlineto{\pgfqpoint{5.831494in}{2.492267in}}%
\pgfpathlineto{\pgfqpoint{5.838598in}{2.497448in}}%
\pgfpathlineto{\pgfqpoint{5.845695in}{2.502597in}}%
\pgfpathclose%
\pgfusepath{fill}%
\end{pgfscope}%
\begin{pgfscope}%
\pgfpathrectangle{\pgfqpoint{1.150000in}{0.150000in}}{\pgfqpoint{5.700000in}{5.700000in}}%
\pgfusepath{clip}%
\pgfsetbuttcap%
\pgfsetroundjoin%
\definecolor{currentfill}{rgb}{0.277018,0.050344,0.375715}%
\pgfsetfillcolor{currentfill}%
\pgfsetfillopacity{0.700000}%
\pgfsetlinewidth{0.000000pt}%
\definecolor{currentstroke}{rgb}{0.000000,0.000000,0.000000}%
\pgfsetstrokecolor{currentstroke}%
\pgfsetdash{}{0pt}%
\pgfpathmoveto{\pgfqpoint{4.547600in}{2.276974in}}%
\pgfpathlineto{\pgfqpoint{4.561289in}{2.274656in}}%
\pgfpathlineto{\pgfqpoint{4.574986in}{2.272364in}}%
\pgfpathlineto{\pgfqpoint{4.588690in}{2.270098in}}%
\pgfpathlineto{\pgfqpoint{4.602401in}{2.267858in}}%
\pgfpathlineto{\pgfqpoint{4.594747in}{2.259632in}}%
\pgfpathlineto{\pgfqpoint{4.587087in}{2.251364in}}%
\pgfpathlineto{\pgfqpoint{4.579421in}{2.243055in}}%
\pgfpathlineto{\pgfqpoint{4.571749in}{2.234706in}}%
\pgfpathlineto{\pgfqpoint{4.558027in}{2.237001in}}%
\pgfpathlineto{\pgfqpoint{4.544311in}{2.239322in}}%
\pgfpathlineto{\pgfqpoint{4.530603in}{2.241669in}}%
\pgfpathlineto{\pgfqpoint{4.516902in}{2.244042in}}%
\pgfpathlineto{\pgfqpoint{4.524585in}{2.252331in}}%
\pgfpathlineto{\pgfqpoint{4.532263in}{2.260584in}}%
\pgfpathlineto{\pgfqpoint{4.539934in}{2.268798in}}%
\pgfpathlineto{\pgfqpoint{4.547600in}{2.276974in}}%
\pgfpathclose%
\pgfusepath{fill}%
\end{pgfscope}%
\begin{pgfscope}%
\pgfpathrectangle{\pgfqpoint{1.150000in}{0.150000in}}{\pgfqpoint{5.700000in}{5.700000in}}%
\pgfusepath{clip}%
\pgfsetbuttcap%
\pgfsetroundjoin%
\definecolor{currentfill}{rgb}{0.283091,0.110553,0.431554}%
\pgfsetfillcolor{currentfill}%
\pgfsetfillopacity{0.700000}%
\pgfsetlinewidth{0.000000pt}%
\definecolor{currentstroke}{rgb}{0.000000,0.000000,0.000000}%
\pgfsetstrokecolor{currentstroke}%
\pgfsetdash{}{0pt}%
\pgfpathmoveto{\pgfqpoint{5.084067in}{2.378692in}}%
\pgfpathlineto{\pgfqpoint{5.097907in}{2.377165in}}%
\pgfpathlineto{\pgfqpoint{5.111755in}{2.375663in}}%
\pgfpathlineto{\pgfqpoint{5.125611in}{2.374186in}}%
\pgfpathlineto{\pgfqpoint{5.139475in}{2.372734in}}%
\pgfpathlineto{\pgfqpoint{5.132034in}{2.365520in}}%
\pgfpathlineto{\pgfqpoint{5.124586in}{2.358242in}}%
\pgfpathlineto{\pgfqpoint{5.117131in}{2.350898in}}%
\pgfpathlineto{\pgfqpoint{5.109670in}{2.343487in}}%
\pgfpathlineto{\pgfqpoint{5.095793in}{2.344927in}}%
\pgfpathlineto{\pgfqpoint{5.081923in}{2.346391in}}%
\pgfpathlineto{\pgfqpoint{5.068062in}{2.347881in}}%
\pgfpathlineto{\pgfqpoint{5.054209in}{2.349395in}}%
\pgfpathlineto{\pgfqpoint{5.061684in}{2.356813in}}%
\pgfpathlineto{\pgfqpoint{5.069152in}{2.364168in}}%
\pgfpathlineto{\pgfqpoint{5.076613in}{2.371461in}}%
\pgfpathlineto{\pgfqpoint{5.084067in}{2.378692in}}%
\pgfpathclose%
\pgfusepath{fill}%
\end{pgfscope}%
\begin{pgfscope}%
\pgfpathrectangle{\pgfqpoint{1.150000in}{0.150000in}}{\pgfqpoint{5.700000in}{5.700000in}}%
\pgfusepath{clip}%
\pgfsetbuttcap%
\pgfsetroundjoin%
\definecolor{currentfill}{rgb}{0.267004,0.004874,0.329415}%
\pgfsetfillcolor{currentfill}%
\pgfsetfillopacity{0.700000}%
\pgfsetlinewidth{0.000000pt}%
\definecolor{currentstroke}{rgb}{0.000000,0.000000,0.000000}%
\pgfsetstrokecolor{currentstroke}%
\pgfsetdash{}{0pt}%
\pgfpathmoveto{\pgfqpoint{3.871211in}{2.193470in}}%
\pgfpathlineto{\pgfqpoint{3.884735in}{2.189628in}}%
\pgfpathlineto{\pgfqpoint{3.898265in}{2.185816in}}%
\pgfpathlineto{\pgfqpoint{3.911801in}{2.182033in}}%
\pgfpathlineto{\pgfqpoint{3.925344in}{2.178279in}}%
\pgfpathlineto{\pgfqpoint{3.917444in}{2.170648in}}%
\pgfpathlineto{\pgfqpoint{3.909538in}{2.163052in}}%
\pgfpathlineto{\pgfqpoint{3.901627in}{2.155494in}}%
\pgfpathlineto{\pgfqpoint{3.893709in}{2.147977in}}%
\pgfpathlineto{\pgfqpoint{3.880153in}{2.151866in}}%
\pgfpathlineto{\pgfqpoint{3.866603in}{2.155784in}}%
\pgfpathlineto{\pgfqpoint{3.853060in}{2.159730in}}%
\pgfpathlineto{\pgfqpoint{3.839522in}{2.163707in}}%
\pgfpathlineto{\pgfqpoint{3.847454in}{2.171084in}}%
\pgfpathlineto{\pgfqpoint{3.855379in}{2.178505in}}%
\pgfpathlineto{\pgfqpoint{3.863298in}{2.185968in}}%
\pgfpathlineto{\pgfqpoint{3.871211in}{2.193470in}}%
\pgfpathclose%
\pgfusepath{fill}%
\end{pgfscope}%
\begin{pgfscope}%
\pgfpathrectangle{\pgfqpoint{1.150000in}{0.150000in}}{\pgfqpoint{5.700000in}{5.700000in}}%
\pgfusepath{clip}%
\pgfsetbuttcap%
\pgfsetroundjoin%
\definecolor{currentfill}{rgb}{0.282910,0.105393,0.426902}%
\pgfsetfillcolor{currentfill}%
\pgfsetfillopacity{0.700000}%
\pgfsetlinewidth{0.000000pt}%
\definecolor{currentstroke}{rgb}{0.000000,0.000000,0.000000}%
\pgfsetstrokecolor{currentstroke}%
\pgfsetdash{}{0pt}%
\pgfpathmoveto{\pgfqpoint{2.837841in}{2.373585in}}%
\pgfpathlineto{\pgfqpoint{2.851211in}{2.366423in}}%
\pgfpathlineto{\pgfqpoint{2.864583in}{2.359304in}}%
\pgfpathlineto{\pgfqpoint{2.877959in}{2.352227in}}%
\pgfpathlineto{\pgfqpoint{2.891339in}{2.345193in}}%
\pgfpathlineto{\pgfqpoint{2.882926in}{2.343366in}}%
\pgfpathlineto{\pgfqpoint{2.874500in}{2.341746in}}%
\pgfpathlineto{\pgfqpoint{2.866061in}{2.340340in}}%
\pgfpathlineto{\pgfqpoint{2.857609in}{2.339154in}}%
\pgfpathlineto{\pgfqpoint{2.844204in}{2.346418in}}%
\pgfpathlineto{\pgfqpoint{2.830802in}{2.353724in}}%
\pgfpathlineto{\pgfqpoint{2.817404in}{2.361073in}}%
\pgfpathlineto{\pgfqpoint{2.804008in}{2.368465in}}%
\pgfpathlineto{\pgfqpoint{2.812487in}{2.369416in}}%
\pgfpathlineto{\pgfqpoint{2.820952in}{2.370590in}}%
\pgfpathlineto{\pgfqpoint{2.829403in}{2.371982in}}%
\pgfpathlineto{\pgfqpoint{2.837841in}{2.373585in}}%
\pgfpathclose%
\pgfusepath{fill}%
\end{pgfscope}%
\begin{pgfscope}%
\pgfpathrectangle{\pgfqpoint{1.150000in}{0.150000in}}{\pgfqpoint{5.700000in}{5.700000in}}%
\pgfusepath{clip}%
\pgfsetbuttcap%
\pgfsetroundjoin%
\definecolor{currentfill}{rgb}{0.280267,0.073417,0.397163}%
\pgfsetfillcolor{currentfill}%
\pgfsetfillopacity{0.700000}%
\pgfsetlinewidth{0.000000pt}%
\definecolor{currentstroke}{rgb}{0.000000,0.000000,0.000000}%
\pgfsetstrokecolor{currentstroke}%
\pgfsetdash{}{0pt}%
\pgfpathmoveto{\pgfqpoint{4.773185in}{2.315648in}}%
\pgfpathlineto{\pgfqpoint{4.786939in}{2.313711in}}%
\pgfpathlineto{\pgfqpoint{4.800701in}{2.311800in}}%
\pgfpathlineto{\pgfqpoint{4.814470in}{2.309914in}}%
\pgfpathlineto{\pgfqpoint{4.828247in}{2.308054in}}%
\pgfpathlineto{\pgfqpoint{4.820677in}{2.300117in}}%
\pgfpathlineto{\pgfqpoint{4.813100in}{2.292124in}}%
\pgfpathlineto{\pgfqpoint{4.805517in}{2.284075in}}%
\pgfpathlineto{\pgfqpoint{4.797928in}{2.275970in}}%
\pgfpathlineto{\pgfqpoint{4.784139in}{2.277858in}}%
\pgfpathlineto{\pgfqpoint{4.770358in}{2.279772in}}%
\pgfpathlineto{\pgfqpoint{4.756584in}{2.281711in}}%
\pgfpathlineto{\pgfqpoint{4.742818in}{2.283676in}}%
\pgfpathlineto{\pgfqpoint{4.750419in}{2.291748in}}%
\pgfpathlineto{\pgfqpoint{4.758014in}{2.299767in}}%
\pgfpathlineto{\pgfqpoint{4.765602in}{2.307734in}}%
\pgfpathlineto{\pgfqpoint{4.773185in}{2.315648in}}%
\pgfpathclose%
\pgfusepath{fill}%
\end{pgfscope}%
\begin{pgfscope}%
\pgfpathrectangle{\pgfqpoint{1.150000in}{0.150000in}}{\pgfqpoint{5.700000in}{5.700000in}}%
\pgfusepath{clip}%
\pgfsetbuttcap%
\pgfsetroundjoin%
\definecolor{currentfill}{rgb}{0.283072,0.130895,0.449241}%
\pgfsetfillcolor{currentfill}%
\pgfsetfillopacity{0.700000}%
\pgfsetlinewidth{0.000000pt}%
\definecolor{currentstroke}{rgb}{0.000000,0.000000,0.000000}%
\pgfsetstrokecolor{currentstroke}%
\pgfsetdash{}{0pt}%
\pgfpathmoveto{\pgfqpoint{5.309732in}{2.417025in}}%
\pgfpathlineto{\pgfqpoint{5.323640in}{2.415732in}}%
\pgfpathlineto{\pgfqpoint{5.337556in}{2.414463in}}%
\pgfpathlineto{\pgfqpoint{5.351481in}{2.413219in}}%
\pgfpathlineto{\pgfqpoint{5.365414in}{2.412000in}}%
\pgfpathlineto{\pgfqpoint{5.358071in}{2.405391in}}%
\pgfpathlineto{\pgfqpoint{5.350722in}{2.398719in}}%
\pgfpathlineto{\pgfqpoint{5.343365in}{2.391981in}}%
\pgfpathlineto{\pgfqpoint{5.336001in}{2.385176in}}%
\pgfpathlineto{\pgfqpoint{5.322054in}{2.386355in}}%
\pgfpathlineto{\pgfqpoint{5.308114in}{2.387560in}}%
\pgfpathlineto{\pgfqpoint{5.294183in}{2.388789in}}%
\pgfpathlineto{\pgfqpoint{5.280260in}{2.390042in}}%
\pgfpathlineto{\pgfqpoint{5.287639in}{2.396882in}}%
\pgfpathlineto{\pgfqpoint{5.295010in}{2.403658in}}%
\pgfpathlineto{\pgfqpoint{5.302374in}{2.410372in}}%
\pgfpathlineto{\pgfqpoint{5.309732in}{2.417025in}}%
\pgfpathclose%
\pgfusepath{fill}%
\end{pgfscope}%
\begin{pgfscope}%
\pgfpathrectangle{\pgfqpoint{1.150000in}{0.150000in}}{\pgfqpoint{5.700000in}{5.700000in}}%
\pgfusepath{clip}%
\pgfsetbuttcap%
\pgfsetroundjoin%
\definecolor{currentfill}{rgb}{0.279566,0.067836,0.391917}%
\pgfsetfillcolor{currentfill}%
\pgfsetfillopacity{0.700000}%
\pgfsetlinewidth{0.000000pt}%
\definecolor{currentstroke}{rgb}{0.000000,0.000000,0.000000}%
\pgfsetstrokecolor{currentstroke}%
\pgfsetdash{}{0pt}%
\pgfpathmoveto{\pgfqpoint{3.031838in}{2.301408in}}%
\pgfpathlineto{\pgfqpoint{3.045226in}{2.294955in}}%
\pgfpathlineto{\pgfqpoint{3.058618in}{2.288541in}}%
\pgfpathlineto{\pgfqpoint{3.072015in}{2.282165in}}%
\pgfpathlineto{\pgfqpoint{3.085415in}{2.275828in}}%
\pgfpathlineto{\pgfqpoint{3.077120in}{2.272593in}}%
\pgfpathlineto{\pgfqpoint{3.068814in}{2.269533in}}%
\pgfpathlineto{\pgfqpoint{3.060497in}{2.266654in}}%
\pgfpathlineto{\pgfqpoint{3.052169in}{2.263962in}}%
\pgfpathlineto{\pgfqpoint{3.038746in}{2.270515in}}%
\pgfpathlineto{\pgfqpoint{3.025327in}{2.277105in}}%
\pgfpathlineto{\pgfqpoint{3.011911in}{2.283735in}}%
\pgfpathlineto{\pgfqpoint{2.998500in}{2.290404in}}%
\pgfpathlineto{\pgfqpoint{3.006852in}{2.292876in}}%
\pgfpathlineto{\pgfqpoint{3.015192in}{2.295537in}}%
\pgfpathlineto{\pgfqpoint{3.023520in}{2.298383in}}%
\pgfpathlineto{\pgfqpoint{3.031838in}{2.301408in}}%
\pgfpathclose%
\pgfusepath{fill}%
\end{pgfscope}%
\begin{pgfscope}%
\pgfpathrectangle{\pgfqpoint{1.150000in}{0.150000in}}{\pgfqpoint{5.700000in}{5.700000in}}%
\pgfusepath{clip}%
\pgfsetbuttcap%
\pgfsetroundjoin%
\definecolor{currentfill}{rgb}{0.281887,0.150881,0.465405}%
\pgfsetfillcolor{currentfill}%
\pgfsetfillopacity{0.700000}%
\pgfsetlinewidth{0.000000pt}%
\definecolor{currentstroke}{rgb}{0.000000,0.000000,0.000000}%
\pgfsetstrokecolor{currentstroke}%
\pgfsetdash{}{0pt}%
\pgfpathmoveto{\pgfqpoint{2.643492in}{2.460687in}}%
\pgfpathlineto{\pgfqpoint{2.656854in}{2.452744in}}%
\pgfpathlineto{\pgfqpoint{2.670217in}{2.444850in}}%
\pgfpathlineto{\pgfqpoint{2.683584in}{2.437003in}}%
\pgfpathlineto{\pgfqpoint{2.696953in}{2.429204in}}%
\pgfpathlineto{\pgfqpoint{2.688405in}{2.428961in}}%
\pgfpathlineto{\pgfqpoint{2.679842in}{2.428961in}}%
\pgfpathlineto{\pgfqpoint{2.671264in}{2.429210in}}%
\pgfpathlineto{\pgfqpoint{2.662670in}{2.429714in}}%
\pgfpathlineto{\pgfqpoint{2.649272in}{2.437758in}}%
\pgfpathlineto{\pgfqpoint{2.635876in}{2.445849in}}%
\pgfpathlineto{\pgfqpoint{2.622483in}{2.453988in}}%
\pgfpathlineto{\pgfqpoint{2.609093in}{2.462176in}}%
\pgfpathlineto{\pgfqpoint{2.617717in}{2.461422in}}%
\pgfpathlineto{\pgfqpoint{2.626324in}{2.460926in}}%
\pgfpathlineto{\pgfqpoint{2.634916in}{2.460683in}}%
\pgfpathlineto{\pgfqpoint{2.643492in}{2.460687in}}%
\pgfpathclose%
\pgfusepath{fill}%
\end{pgfscope}%
\begin{pgfscope}%
\pgfpathrectangle{\pgfqpoint{1.150000in}{0.150000in}}{\pgfqpoint{5.700000in}{5.700000in}}%
\pgfusepath{clip}%
\pgfsetbuttcap%
\pgfsetroundjoin%
\definecolor{currentfill}{rgb}{0.271305,0.019942,0.347269}%
\pgfsetfillcolor{currentfill}%
\pgfsetfillopacity{0.700000}%
\pgfsetlinewidth{0.000000pt}%
\definecolor{currentstroke}{rgb}{0.000000,0.000000,0.000000}%
\pgfsetstrokecolor{currentstroke}%
\pgfsetdash{}{0pt}%
\pgfpathmoveto{\pgfqpoint{4.236602in}{2.220549in}}%
\pgfpathlineto{\pgfqpoint{4.250215in}{2.217597in}}%
\pgfpathlineto{\pgfqpoint{4.263835in}{2.214672in}}%
\pgfpathlineto{\pgfqpoint{4.277462in}{2.211775in}}%
\pgfpathlineto{\pgfqpoint{4.291095in}{2.208905in}}%
\pgfpathlineto{\pgfqpoint{4.283326in}{2.200614in}}%
\pgfpathlineto{\pgfqpoint{4.275552in}{2.192312in}}%
\pgfpathlineto{\pgfqpoint{4.267772in}{2.184001in}}%
\pgfpathlineto{\pgfqpoint{4.259986in}{2.175681in}}%
\pgfpathlineto{\pgfqpoint{4.246341in}{2.178646in}}%
\pgfpathlineto{\pgfqpoint{4.232703in}{2.181639in}}%
\pgfpathlineto{\pgfqpoint{4.219071in}{2.184658in}}%
\pgfpathlineto{\pgfqpoint{4.205446in}{2.187705in}}%
\pgfpathlineto{\pgfqpoint{4.213243in}{2.195925in}}%
\pgfpathlineto{\pgfqpoint{4.221035in}{2.204140in}}%
\pgfpathlineto{\pgfqpoint{4.228821in}{2.212349in}}%
\pgfpathlineto{\pgfqpoint{4.236602in}{2.220549in}}%
\pgfpathclose%
\pgfusepath{fill}%
\end{pgfscope}%
\begin{pgfscope}%
\pgfpathrectangle{\pgfqpoint{1.150000in}{0.150000in}}{\pgfqpoint{5.700000in}{5.700000in}}%
\pgfusepath{clip}%
\pgfsetbuttcap%
\pgfsetroundjoin%
\definecolor{currentfill}{rgb}{0.281887,0.150881,0.465405}%
\pgfsetfillcolor{currentfill}%
\pgfsetfillopacity{0.700000}%
\pgfsetlinewidth{0.000000pt}%
\definecolor{currentstroke}{rgb}{0.000000,0.000000,0.000000}%
\pgfsetstrokecolor{currentstroke}%
\pgfsetdash{}{0pt}%
\pgfpathmoveto{\pgfqpoint{5.535409in}{2.452973in}}%
\pgfpathlineto{\pgfqpoint{5.549385in}{2.451855in}}%
\pgfpathlineto{\pgfqpoint{5.563370in}{2.450761in}}%
\pgfpathlineto{\pgfqpoint{5.577363in}{2.449692in}}%
\pgfpathlineto{\pgfqpoint{5.591365in}{2.448648in}}%
\pgfpathlineto{\pgfqpoint{5.584128in}{2.442695in}}%
\pgfpathlineto{\pgfqpoint{5.576884in}{2.436685in}}%
\pgfpathlineto{\pgfqpoint{5.569632in}{2.430615in}}%
\pgfpathlineto{\pgfqpoint{5.562373in}{2.424483in}}%
\pgfpathlineto{\pgfqpoint{5.548355in}{2.425460in}}%
\pgfpathlineto{\pgfqpoint{5.534345in}{2.426462in}}%
\pgfpathlineto{\pgfqpoint{5.520344in}{2.427489in}}%
\pgfpathlineto{\pgfqpoint{5.506352in}{2.428540in}}%
\pgfpathlineto{\pgfqpoint{5.513627in}{2.434733in}}%
\pgfpathlineto{\pgfqpoint{5.520895in}{2.440869in}}%
\pgfpathlineto{\pgfqpoint{5.528156in}{2.446948in}}%
\pgfpathlineto{\pgfqpoint{5.535409in}{2.452973in}}%
\pgfpathclose%
\pgfusepath{fill}%
\end{pgfscope}%
\begin{pgfscope}%
\pgfpathrectangle{\pgfqpoint{1.150000in}{0.150000in}}{\pgfqpoint{5.700000in}{5.700000in}}%
\pgfusepath{clip}%
\pgfsetbuttcap%
\pgfsetroundjoin%
\definecolor{currentfill}{rgb}{0.267004,0.004874,0.329415}%
\pgfsetfillcolor{currentfill}%
\pgfsetfillopacity{0.700000}%
\pgfsetlinewidth{0.000000pt}%
\definecolor{currentstroke}{rgb}{0.000000,0.000000,0.000000}%
\pgfsetstrokecolor{currentstroke}%
\pgfsetdash{}{0pt}%
\pgfpathmoveto{\pgfqpoint{4.011060in}{2.194849in}}%
\pgfpathlineto{\pgfqpoint{4.024620in}{2.191359in}}%
\pgfpathlineto{\pgfqpoint{4.038186in}{2.187898in}}%
\pgfpathlineto{\pgfqpoint{4.051759in}{2.184466in}}%
\pgfpathlineto{\pgfqpoint{4.065338in}{2.181061in}}%
\pgfpathlineto{\pgfqpoint{4.057487in}{2.173087in}}%
\pgfpathlineto{\pgfqpoint{4.049631in}{2.165130in}}%
\pgfpathlineto{\pgfqpoint{4.041769in}{2.157193in}}%
\pgfpathlineto{\pgfqpoint{4.033902in}{2.149277in}}%
\pgfpathlineto{\pgfqpoint{4.020310in}{2.152803in}}%
\pgfpathlineto{\pgfqpoint{4.006725in}{2.156357in}}%
\pgfpathlineto{\pgfqpoint{3.993146in}{2.159940in}}%
\pgfpathlineto{\pgfqpoint{3.979573in}{2.163550in}}%
\pgfpathlineto{\pgfqpoint{3.987454in}{2.171339in}}%
\pgfpathlineto{\pgfqpoint{3.995328in}{2.179154in}}%
\pgfpathlineto{\pgfqpoint{4.003197in}{2.186991in}}%
\pgfpathlineto{\pgfqpoint{4.011060in}{2.194849in}}%
\pgfpathclose%
\pgfusepath{fill}%
\end{pgfscope}%
\begin{pgfscope}%
\pgfpathrectangle{\pgfqpoint{1.150000in}{0.150000in}}{\pgfqpoint{5.700000in}{5.700000in}}%
\pgfusepath{clip}%
\pgfsetbuttcap%
\pgfsetroundjoin%
\definecolor{currentfill}{rgb}{0.268510,0.009605,0.335427}%
\pgfsetfillcolor{currentfill}%
\pgfsetfillopacity{0.700000}%
\pgfsetlinewidth{0.000000pt}%
\definecolor{currentstroke}{rgb}{0.000000,0.000000,0.000000}%
\pgfsetstrokecolor{currentstroke}%
\pgfsetdash{}{0pt}%
\pgfpathmoveto{\pgfqpoint{3.505674in}{2.198746in}}%
\pgfpathlineto{\pgfqpoint{3.519132in}{2.193847in}}%
\pgfpathlineto{\pgfqpoint{3.532594in}{2.188981in}}%
\pgfpathlineto{\pgfqpoint{3.546062in}{2.184147in}}%
\pgfpathlineto{\pgfqpoint{3.559535in}{2.179344in}}%
\pgfpathlineto{\pgfqpoint{3.551482in}{2.173203in}}%
\pgfpathlineto{\pgfqpoint{3.543422in}{2.167157in}}%
\pgfpathlineto{\pgfqpoint{3.535354in}{2.161210in}}%
\pgfpathlineto{\pgfqpoint{3.527278in}{2.155366in}}%
\pgfpathlineto{\pgfqpoint{3.513789in}{2.160343in}}%
\pgfpathlineto{\pgfqpoint{3.500304in}{2.165352in}}%
\pgfpathlineto{\pgfqpoint{3.486824in}{2.170393in}}%
\pgfpathlineto{\pgfqpoint{3.473350in}{2.175466in}}%
\pgfpathlineto{\pgfqpoint{3.481443in}{2.181131in}}%
\pgfpathlineto{\pgfqpoint{3.489528in}{2.186902in}}%
\pgfpathlineto{\pgfqpoint{3.497605in}{2.192775in}}%
\pgfpathlineto{\pgfqpoint{3.505674in}{2.198746in}}%
\pgfpathclose%
\pgfusepath{fill}%
\end{pgfscope}%
\begin{pgfscope}%
\pgfpathrectangle{\pgfqpoint{1.150000in}{0.150000in}}{\pgfqpoint{5.700000in}{5.700000in}}%
\pgfusepath{clip}%
\pgfsetbuttcap%
\pgfsetroundjoin%
\definecolor{currentfill}{rgb}{0.274952,0.037752,0.364543}%
\pgfsetfillcolor{currentfill}%
\pgfsetfillopacity{0.700000}%
\pgfsetlinewidth{0.000000pt}%
\definecolor{currentstroke}{rgb}{0.000000,0.000000,0.000000}%
\pgfsetstrokecolor{currentstroke}%
\pgfsetdash{}{0pt}%
\pgfpathmoveto{\pgfqpoint{4.462170in}{2.253799in}}%
\pgfpathlineto{\pgfqpoint{4.475842in}{2.251320in}}%
\pgfpathlineto{\pgfqpoint{4.489522in}{2.248868in}}%
\pgfpathlineto{\pgfqpoint{4.503209in}{2.246442in}}%
\pgfpathlineto{\pgfqpoint{4.516902in}{2.244042in}}%
\pgfpathlineto{\pgfqpoint{4.509214in}{2.235717in}}%
\pgfpathlineto{\pgfqpoint{4.501519in}{2.227358in}}%
\pgfpathlineto{\pgfqpoint{4.493819in}{2.218965in}}%
\pgfpathlineto{\pgfqpoint{4.486113in}{2.210541in}}%
\pgfpathlineto{\pgfqpoint{4.472408in}{2.213009in}}%
\pgfpathlineto{\pgfqpoint{4.458709in}{2.215503in}}%
\pgfpathlineto{\pgfqpoint{4.445018in}{2.218024in}}%
\pgfpathlineto{\pgfqpoint{4.431335in}{2.220571in}}%
\pgfpathlineto{\pgfqpoint{4.439052in}{2.228922in}}%
\pgfpathlineto{\pgfqpoint{4.446764in}{2.237245in}}%
\pgfpathlineto{\pgfqpoint{4.454470in}{2.245538in}}%
\pgfpathlineto{\pgfqpoint{4.462170in}{2.253799in}}%
\pgfpathclose%
\pgfusepath{fill}%
\end{pgfscope}%
\begin{pgfscope}%
\pgfpathrectangle{\pgfqpoint{1.150000in}{0.150000in}}{\pgfqpoint{5.700000in}{5.700000in}}%
\pgfusepath{clip}%
\pgfsetbuttcap%
\pgfsetroundjoin%
\definecolor{currentfill}{rgb}{0.282656,0.100196,0.422160}%
\pgfsetfillcolor{currentfill}%
\pgfsetfillopacity{0.700000}%
\pgfsetlinewidth{0.000000pt}%
\definecolor{currentstroke}{rgb}{0.000000,0.000000,0.000000}%
\pgfsetstrokecolor{currentstroke}%
\pgfsetdash{}{0pt}%
\pgfpathmoveto{\pgfqpoint{4.998876in}{2.355706in}}%
\pgfpathlineto{\pgfqpoint{5.012697in}{2.354090in}}%
\pgfpathlineto{\pgfqpoint{5.026527in}{2.352500in}}%
\pgfpathlineto{\pgfqpoint{5.040364in}{2.350935in}}%
\pgfpathlineto{\pgfqpoint{5.054209in}{2.349395in}}%
\pgfpathlineto{\pgfqpoint{5.046728in}{2.341914in}}%
\pgfpathlineto{\pgfqpoint{5.039240in}{2.334368in}}%
\pgfpathlineto{\pgfqpoint{5.031745in}{2.326758in}}%
\pgfpathlineto{\pgfqpoint{5.024244in}{2.319082in}}%
\pgfpathlineto{\pgfqpoint{5.010386in}{2.320623in}}%
\pgfpathlineto{\pgfqpoint{4.996536in}{2.322189in}}%
\pgfpathlineto{\pgfqpoint{4.982694in}{2.323780in}}%
\pgfpathlineto{\pgfqpoint{4.968860in}{2.325397in}}%
\pgfpathlineto{\pgfqpoint{4.976374in}{2.333067in}}%
\pgfpathlineto{\pgfqpoint{4.983881in}{2.340674in}}%
\pgfpathlineto{\pgfqpoint{4.991382in}{2.348221in}}%
\pgfpathlineto{\pgfqpoint{4.998876in}{2.355706in}}%
\pgfpathclose%
\pgfusepath{fill}%
\end{pgfscope}%
\begin{pgfscope}%
\pgfpathrectangle{\pgfqpoint{1.150000in}{0.150000in}}{\pgfqpoint{5.700000in}{5.700000in}}%
\pgfusepath{clip}%
\pgfsetbuttcap%
\pgfsetroundjoin%
\definecolor{currentfill}{rgb}{0.271305,0.019942,0.347269}%
\pgfsetfillcolor{currentfill}%
\pgfsetfillopacity{0.700000}%
\pgfsetlinewidth{0.000000pt}%
\definecolor{currentstroke}{rgb}{0.000000,0.000000,0.000000}%
\pgfsetstrokecolor{currentstroke}%
\pgfsetdash{}{0pt}%
\pgfpathmoveto{\pgfqpoint{3.365733in}{2.217233in}}%
\pgfpathlineto{\pgfqpoint{3.379168in}{2.211895in}}%
\pgfpathlineto{\pgfqpoint{3.392608in}{2.206591in}}%
\pgfpathlineto{\pgfqpoint{3.406052in}{2.201321in}}%
\pgfpathlineto{\pgfqpoint{3.419502in}{2.196084in}}%
\pgfpathlineto{\pgfqpoint{3.411383in}{2.190714in}}%
\pgfpathlineto{\pgfqpoint{3.403256in}{2.185462in}}%
\pgfpathlineto{\pgfqpoint{3.395121in}{2.180334in}}%
\pgfpathlineto{\pgfqpoint{3.386977in}{2.175334in}}%
\pgfpathlineto{\pgfqpoint{3.373509in}{2.180759in}}%
\pgfpathlineto{\pgfqpoint{3.360045in}{2.186217in}}%
\pgfpathlineto{\pgfqpoint{3.346587in}{2.191709in}}%
\pgfpathlineto{\pgfqpoint{3.333133in}{2.197234in}}%
\pgfpathlineto{\pgfqpoint{3.341296in}{2.202041in}}%
\pgfpathlineto{\pgfqpoint{3.349450in}{2.206980in}}%
\pgfpathlineto{\pgfqpoint{3.357596in}{2.212045in}}%
\pgfpathlineto{\pgfqpoint{3.365733in}{2.217233in}}%
\pgfpathclose%
\pgfusepath{fill}%
\end{pgfscope}%
\begin{pgfscope}%
\pgfpathrectangle{\pgfqpoint{1.150000in}{0.150000in}}{\pgfqpoint{5.700000in}{5.700000in}}%
\pgfusepath{clip}%
\pgfsetbuttcap%
\pgfsetroundjoin%
\definecolor{currentfill}{rgb}{0.267004,0.004874,0.329415}%
\pgfsetfillcolor{currentfill}%
\pgfsetfillopacity{0.700000}%
\pgfsetlinewidth{0.000000pt}%
\definecolor{currentstroke}{rgb}{0.000000,0.000000,0.000000}%
\pgfsetstrokecolor{currentstroke}%
\pgfsetdash{}{0pt}%
\pgfpathmoveto{\pgfqpoint{3.645555in}{2.186522in}}%
\pgfpathlineto{\pgfqpoint{3.659039in}{2.182037in}}%
\pgfpathlineto{\pgfqpoint{3.672529in}{2.177583in}}%
\pgfpathlineto{\pgfqpoint{3.686024in}{2.173160in}}%
\pgfpathlineto{\pgfqpoint{3.699524in}{2.168767in}}%
\pgfpathlineto{\pgfqpoint{3.691531in}{2.161976in}}%
\pgfpathlineto{\pgfqpoint{3.683532in}{2.155256in}}%
\pgfpathlineto{\pgfqpoint{3.675525in}{2.148613in}}%
\pgfpathlineto{\pgfqpoint{3.667512in}{2.142049in}}%
\pgfpathlineto{\pgfqpoint{3.653996in}{2.146603in}}%
\pgfpathlineto{\pgfqpoint{3.640485in}{2.151187in}}%
\pgfpathlineto{\pgfqpoint{3.626980in}{2.155802in}}%
\pgfpathlineto{\pgfqpoint{3.613480in}{2.160448in}}%
\pgfpathlineto{\pgfqpoint{3.621510in}{2.166846in}}%
\pgfpathlineto{\pgfqpoint{3.629532in}{2.173327in}}%
\pgfpathlineto{\pgfqpoint{3.637547in}{2.179887in}}%
\pgfpathlineto{\pgfqpoint{3.645555in}{2.186522in}}%
\pgfpathclose%
\pgfusepath{fill}%
\end{pgfscope}%
\begin{pgfscope}%
\pgfpathrectangle{\pgfqpoint{1.150000in}{0.150000in}}{\pgfqpoint{5.700000in}{5.700000in}}%
\pgfusepath{clip}%
\pgfsetbuttcap%
\pgfsetroundjoin%
\definecolor{currentfill}{rgb}{0.280255,0.165693,0.476498}%
\pgfsetfillcolor{currentfill}%
\pgfsetfillopacity{0.700000}%
\pgfsetlinewidth{0.000000pt}%
\definecolor{currentstroke}{rgb}{0.000000,0.000000,0.000000}%
\pgfsetstrokecolor{currentstroke}%
\pgfsetdash{}{0pt}%
\pgfpathmoveto{\pgfqpoint{5.761041in}{2.485653in}}%
\pgfpathlineto{\pgfqpoint{5.775084in}{2.484653in}}%
\pgfpathlineto{\pgfqpoint{5.789135in}{2.483677in}}%
\pgfpathlineto{\pgfqpoint{5.803195in}{2.482726in}}%
\pgfpathlineto{\pgfqpoint{5.817264in}{2.481798in}}%
\pgfpathlineto{\pgfqpoint{5.810139in}{2.476503in}}%
\pgfpathlineto{\pgfqpoint{5.803006in}{2.471164in}}%
\pgfpathlineto{\pgfqpoint{5.795867in}{2.465776in}}%
\pgfpathlineto{\pgfqpoint{5.788719in}{2.460338in}}%
\pgfpathlineto{\pgfqpoint{5.774632in}{2.461171in}}%
\pgfpathlineto{\pgfqpoint{5.760553in}{2.462028in}}%
\pgfpathlineto{\pgfqpoint{5.746483in}{2.462909in}}%
\pgfpathlineto{\pgfqpoint{5.732422in}{2.463815in}}%
\pgfpathlineto{\pgfqpoint{5.739587in}{2.469343in}}%
\pgfpathlineto{\pgfqpoint{5.746746in}{2.474823in}}%
\pgfpathlineto{\pgfqpoint{5.753897in}{2.480259in}}%
\pgfpathlineto{\pgfqpoint{5.761041in}{2.485653in}}%
\pgfpathclose%
\pgfusepath{fill}%
\end{pgfscope}%
\begin{pgfscope}%
\pgfpathrectangle{\pgfqpoint{1.150000in}{0.150000in}}{\pgfqpoint{5.700000in}{5.700000in}}%
\pgfusepath{clip}%
\pgfsetbuttcap%
\pgfsetroundjoin%
\definecolor{currentfill}{rgb}{0.277134,0.185228,0.489898}%
\pgfsetfillcolor{currentfill}%
\pgfsetfillopacity{0.700000}%
\pgfsetlinewidth{0.000000pt}%
\definecolor{currentstroke}{rgb}{0.000000,0.000000,0.000000}%
\pgfsetstrokecolor{currentstroke}%
\pgfsetdash{}{0pt}%
\pgfpathmoveto{\pgfqpoint{5.986567in}{2.514575in}}%
\pgfpathlineto{\pgfqpoint{6.000674in}{2.513636in}}%
\pgfpathlineto{\pgfqpoint{6.014791in}{2.512720in}}%
\pgfpathlineto{\pgfqpoint{6.028916in}{2.511829in}}%
\pgfpathlineto{\pgfqpoint{6.043050in}{2.510962in}}%
\pgfpathlineto{\pgfqpoint{6.036040in}{2.506279in}}%
\pgfpathlineto{\pgfqpoint{6.029024in}{2.501570in}}%
\pgfpathlineto{\pgfqpoint{6.022001in}{2.496832in}}%
\pgfpathlineto{\pgfqpoint{6.014971in}{2.492059in}}%
\pgfpathlineto{\pgfqpoint{6.000816in}{2.492805in}}%
\pgfpathlineto{\pgfqpoint{5.986670in}{2.493574in}}%
\pgfpathlineto{\pgfqpoint{5.972533in}{2.494368in}}%
\pgfpathlineto{\pgfqpoint{5.958405in}{2.495186in}}%
\pgfpathlineto{\pgfqpoint{5.965456in}{2.500075in}}%
\pgfpathlineto{\pgfqpoint{5.972499in}{2.504934in}}%
\pgfpathlineto{\pgfqpoint{5.979537in}{2.509766in}}%
\pgfpathlineto{\pgfqpoint{5.986567in}{2.514575in}}%
\pgfpathclose%
\pgfusepath{fill}%
\end{pgfscope}%
\begin{pgfscope}%
\pgfpathrectangle{\pgfqpoint{1.150000in}{0.150000in}}{\pgfqpoint{5.700000in}{5.700000in}}%
\pgfusepath{clip}%
\pgfsetbuttcap%
\pgfsetroundjoin%
\definecolor{currentfill}{rgb}{0.279566,0.067836,0.391917}%
\pgfsetfillcolor{currentfill}%
\pgfsetfillopacity{0.700000}%
\pgfsetlinewidth{0.000000pt}%
\definecolor{currentstroke}{rgb}{0.000000,0.000000,0.000000}%
\pgfsetstrokecolor{currentstroke}%
\pgfsetdash{}{0pt}%
\pgfpathmoveto{\pgfqpoint{4.687829in}{2.291794in}}%
\pgfpathlineto{\pgfqpoint{4.701565in}{2.289726in}}%
\pgfpathlineto{\pgfqpoint{4.715309in}{2.287683in}}%
\pgfpathlineto{\pgfqpoint{4.729060in}{2.285667in}}%
\pgfpathlineto{\pgfqpoint{4.742818in}{2.283676in}}%
\pgfpathlineto{\pgfqpoint{4.735211in}{2.275553in}}%
\pgfpathlineto{\pgfqpoint{4.727599in}{2.267378in}}%
\pgfpathlineto{\pgfqpoint{4.719980in}{2.259152in}}%
\pgfpathlineto{\pgfqpoint{4.712354in}{2.250875in}}%
\pgfpathlineto{\pgfqpoint{4.698584in}{2.252908in}}%
\pgfpathlineto{\pgfqpoint{4.684822in}{2.254966in}}%
\pgfpathlineto{\pgfqpoint{4.671066in}{2.257050in}}%
\pgfpathlineto{\pgfqpoint{4.657318in}{2.259159in}}%
\pgfpathlineto{\pgfqpoint{4.664955in}{2.267389in}}%
\pgfpathlineto{\pgfqpoint{4.672586in}{2.275572in}}%
\pgfpathlineto{\pgfqpoint{4.680210in}{2.283707in}}%
\pgfpathlineto{\pgfqpoint{4.687829in}{2.291794in}}%
\pgfpathclose%
\pgfusepath{fill}%
\end{pgfscope}%
\begin{pgfscope}%
\pgfpathrectangle{\pgfqpoint{1.150000in}{0.150000in}}{\pgfqpoint{5.700000in}{5.700000in}}%
\pgfusepath{clip}%
\pgfsetbuttcap%
\pgfsetroundjoin%
\definecolor{currentfill}{rgb}{0.274952,0.037752,0.364543}%
\pgfsetfillcolor{currentfill}%
\pgfsetfillopacity{0.700000}%
\pgfsetlinewidth{0.000000pt}%
\definecolor{currentstroke}{rgb}{0.000000,0.000000,0.000000}%
\pgfsetstrokecolor{currentstroke}%
\pgfsetdash{}{0pt}%
\pgfpathmoveto{\pgfqpoint{3.225669in}{2.242679in}}%
\pgfpathlineto{\pgfqpoint{3.239087in}{2.236875in}}%
\pgfpathlineto{\pgfqpoint{3.252508in}{2.231108in}}%
\pgfpathlineto{\pgfqpoint{3.265934in}{2.225375in}}%
\pgfpathlineto{\pgfqpoint{3.279365in}{2.219677in}}%
\pgfpathlineto{\pgfqpoint{3.271173in}{2.215203in}}%
\pgfpathlineto{\pgfqpoint{3.262972in}{2.210873in}}%
\pgfpathlineto{\pgfqpoint{3.254762in}{2.206693in}}%
\pgfpathlineto{\pgfqpoint{3.246542in}{2.202668in}}%
\pgfpathlineto{\pgfqpoint{3.233091in}{2.208567in}}%
\pgfpathlineto{\pgfqpoint{3.219644in}{2.214501in}}%
\pgfpathlineto{\pgfqpoint{3.206202in}{2.220470in}}%
\pgfpathlineto{\pgfqpoint{3.192764in}{2.226475in}}%
\pgfpathlineto{\pgfqpoint{3.201005in}{2.230294in}}%
\pgfpathlineto{\pgfqpoint{3.209237in}{2.234271in}}%
\pgfpathlineto{\pgfqpoint{3.217458in}{2.238401in}}%
\pgfpathlineto{\pgfqpoint{3.225669in}{2.242679in}}%
\pgfpathclose%
\pgfusepath{fill}%
\end{pgfscope}%
\begin{pgfscope}%
\pgfpathrectangle{\pgfqpoint{1.150000in}{0.150000in}}{\pgfqpoint{5.700000in}{5.700000in}}%
\pgfusepath{clip}%
\pgfsetbuttcap%
\pgfsetroundjoin%
\definecolor{currentfill}{rgb}{0.267004,0.004874,0.329415}%
\pgfsetfillcolor{currentfill}%
\pgfsetfillopacity{0.700000}%
\pgfsetlinewidth{0.000000pt}%
\definecolor{currentstroke}{rgb}{0.000000,0.000000,0.000000}%
\pgfsetstrokecolor{currentstroke}%
\pgfsetdash{}{0pt}%
\pgfpathmoveto{\pgfqpoint{3.785429in}{2.179905in}}%
\pgfpathlineto{\pgfqpoint{3.798944in}{2.175811in}}%
\pgfpathlineto{\pgfqpoint{3.812464in}{2.171747in}}%
\pgfpathlineto{\pgfqpoint{3.825990in}{2.167712in}}%
\pgfpathlineto{\pgfqpoint{3.839522in}{2.163707in}}%
\pgfpathlineto{\pgfqpoint{3.831584in}{2.156377in}}%
\pgfpathlineto{\pgfqpoint{3.823640in}{2.149099in}}%
\pgfpathlineto{\pgfqpoint{3.815690in}{2.141875in}}%
\pgfpathlineto{\pgfqpoint{3.807733in}{2.134709in}}%
\pgfpathlineto{\pgfqpoint{3.794187in}{2.138862in}}%
\pgfpathlineto{\pgfqpoint{3.780646in}{2.143045in}}%
\pgfpathlineto{\pgfqpoint{3.767112in}{2.147257in}}%
\pgfpathlineto{\pgfqpoint{3.753583in}{2.151499in}}%
\pgfpathlineto{\pgfqpoint{3.761554in}{2.158512in}}%
\pgfpathlineto{\pgfqpoint{3.769519in}{2.165587in}}%
\pgfpathlineto{\pgfqpoint{3.777478in}{2.172719in}}%
\pgfpathlineto{\pgfqpoint{3.785429in}{2.179905in}}%
\pgfpathclose%
\pgfusepath{fill}%
\end{pgfscope}%
\begin{pgfscope}%
\pgfpathrectangle{\pgfqpoint{1.150000in}{0.150000in}}{\pgfqpoint{5.700000in}{5.700000in}}%
\pgfusepath{clip}%
\pgfsetbuttcap%
\pgfsetroundjoin%
\definecolor{currentfill}{rgb}{0.283229,0.120777,0.440584}%
\pgfsetfillcolor{currentfill}%
\pgfsetfillopacity{0.700000}%
\pgfsetlinewidth{0.000000pt}%
\definecolor{currentstroke}{rgb}{0.000000,0.000000,0.000000}%
\pgfsetstrokecolor{currentstroke}%
\pgfsetdash{}{0pt}%
\pgfpathmoveto{\pgfqpoint{5.224651in}{2.395306in}}%
\pgfpathlineto{\pgfqpoint{5.238541in}{2.393953in}}%
\pgfpathlineto{\pgfqpoint{5.252439in}{2.392624in}}%
\pgfpathlineto{\pgfqpoint{5.266346in}{2.391321in}}%
\pgfpathlineto{\pgfqpoint{5.280260in}{2.390042in}}%
\pgfpathlineto{\pgfqpoint{5.272875in}{2.383137in}}%
\pgfpathlineto{\pgfqpoint{5.265482in}{2.376166in}}%
\pgfpathlineto{\pgfqpoint{5.258083in}{2.369126in}}%
\pgfpathlineto{\pgfqpoint{5.250676in}{2.362018in}}%
\pgfpathlineto{\pgfqpoint{5.236747in}{2.363270in}}%
\pgfpathlineto{\pgfqpoint{5.222827in}{2.364548in}}%
\pgfpathlineto{\pgfqpoint{5.208914in}{2.365850in}}%
\pgfpathlineto{\pgfqpoint{5.195010in}{2.367177in}}%
\pgfpathlineto{\pgfqpoint{5.202431in}{2.374306in}}%
\pgfpathlineto{\pgfqpoint{5.209844in}{2.381370in}}%
\pgfpathlineto{\pgfqpoint{5.217251in}{2.388370in}}%
\pgfpathlineto{\pgfqpoint{5.224651in}{2.395306in}}%
\pgfpathclose%
\pgfusepath{fill}%
\end{pgfscope}%
\begin{pgfscope}%
\pgfpathrectangle{\pgfqpoint{1.150000in}{0.150000in}}{\pgfqpoint{5.700000in}{5.700000in}}%
\pgfusepath{clip}%
\pgfsetbuttcap%
\pgfsetroundjoin%
\definecolor{currentfill}{rgb}{0.282656,0.100196,0.422160}%
\pgfsetfillcolor{currentfill}%
\pgfsetfillopacity{0.700000}%
\pgfsetlinewidth{0.000000pt}%
\definecolor{currentstroke}{rgb}{0.000000,0.000000,0.000000}%
\pgfsetstrokecolor{currentstroke}%
\pgfsetdash{}{0pt}%
\pgfpathmoveto{\pgfqpoint{2.891339in}{2.345193in}}%
\pgfpathlineto{\pgfqpoint{2.904721in}{2.338201in}}%
\pgfpathlineto{\pgfqpoint{2.918108in}{2.331251in}}%
\pgfpathlineto{\pgfqpoint{2.931497in}{2.324342in}}%
\pgfpathlineto{\pgfqpoint{2.944890in}{2.317474in}}%
\pgfpathlineto{\pgfqpoint{2.936502in}{2.315422in}}%
\pgfpathlineto{\pgfqpoint{2.928102in}{2.313574in}}%
\pgfpathlineto{\pgfqpoint{2.919689in}{2.311937in}}%
\pgfpathlineto{\pgfqpoint{2.911262in}{2.310516in}}%
\pgfpathlineto{\pgfqpoint{2.897844in}{2.317614in}}%
\pgfpathlineto{\pgfqpoint{2.884429in}{2.324753in}}%
\pgfpathlineto{\pgfqpoint{2.871017in}{2.331933in}}%
\pgfpathlineto{\pgfqpoint{2.857609in}{2.339154in}}%
\pgfpathlineto{\pgfqpoint{2.866061in}{2.340340in}}%
\pgfpathlineto{\pgfqpoint{2.874500in}{2.341746in}}%
\pgfpathlineto{\pgfqpoint{2.882926in}{2.343366in}}%
\pgfpathlineto{\pgfqpoint{2.891339in}{2.345193in}}%
\pgfpathclose%
\pgfusepath{fill}%
\end{pgfscope}%
\begin{pgfscope}%
\pgfpathrectangle{\pgfqpoint{1.150000in}{0.150000in}}{\pgfqpoint{5.700000in}{5.700000in}}%
\pgfusepath{clip}%
\pgfsetbuttcap%
\pgfsetroundjoin%
\definecolor{currentfill}{rgb}{0.269944,0.014625,0.341379}%
\pgfsetfillcolor{currentfill}%
\pgfsetfillopacity{0.700000}%
\pgfsetlinewidth{0.000000pt}%
\definecolor{currentstroke}{rgb}{0.000000,0.000000,0.000000}%
\pgfsetstrokecolor{currentstroke}%
\pgfsetdash{}{0pt}%
\pgfpathmoveto{\pgfqpoint{4.151012in}{2.200167in}}%
\pgfpathlineto{\pgfqpoint{4.164610in}{2.197010in}}%
\pgfpathlineto{\pgfqpoint{4.178216in}{2.193881in}}%
\pgfpathlineto{\pgfqpoint{4.191827in}{2.190779in}}%
\pgfpathlineto{\pgfqpoint{4.205446in}{2.187705in}}%
\pgfpathlineto{\pgfqpoint{4.197643in}{2.179483in}}%
\pgfpathlineto{\pgfqpoint{4.189834in}{2.171262in}}%
\pgfpathlineto{\pgfqpoint{4.182020in}{2.163042in}}%
\pgfpathlineto{\pgfqpoint{4.174200in}{2.154828in}}%
\pgfpathlineto{\pgfqpoint{4.160569in}{2.158010in}}%
\pgfpathlineto{\pgfqpoint{4.146945in}{2.161220in}}%
\pgfpathlineto{\pgfqpoint{4.133328in}{2.164458in}}%
\pgfpathlineto{\pgfqpoint{4.119717in}{2.167723in}}%
\pgfpathlineto{\pgfqpoint{4.127549in}{2.175824in}}%
\pgfpathlineto{\pgfqpoint{4.135376in}{2.183934in}}%
\pgfpathlineto{\pgfqpoint{4.143196in}{2.192049in}}%
\pgfpathlineto{\pgfqpoint{4.151012in}{2.200167in}}%
\pgfpathclose%
\pgfusepath{fill}%
\end{pgfscope}%
\begin{pgfscope}%
\pgfpathrectangle{\pgfqpoint{1.150000in}{0.150000in}}{\pgfqpoint{5.700000in}{5.700000in}}%
\pgfusepath{clip}%
\pgfsetbuttcap%
\pgfsetroundjoin%
\definecolor{currentfill}{rgb}{0.281924,0.089666,0.412415}%
\pgfsetfillcolor{currentfill}%
\pgfsetfillopacity{0.700000}%
\pgfsetlinewidth{0.000000pt}%
\definecolor{currentstroke}{rgb}{0.000000,0.000000,0.000000}%
\pgfsetstrokecolor{currentstroke}%
\pgfsetdash{}{0pt}%
\pgfpathmoveto{\pgfqpoint{4.913601in}{2.332116in}}%
\pgfpathlineto{\pgfqpoint{4.927404in}{2.330398in}}%
\pgfpathlineto{\pgfqpoint{4.941215in}{2.328706in}}%
\pgfpathlineto{\pgfqpoint{4.955033in}{2.327039in}}%
\pgfpathlineto{\pgfqpoint{4.968860in}{2.325397in}}%
\pgfpathlineto{\pgfqpoint{4.961339in}{2.317665in}}%
\pgfpathlineto{\pgfqpoint{4.953813in}{2.309870in}}%
\pgfpathlineto{\pgfqpoint{4.946279in}{2.302013in}}%
\pgfpathlineto{\pgfqpoint{4.938739in}{2.294092in}}%
\pgfpathlineto{\pgfqpoint{4.924901in}{2.295749in}}%
\pgfpathlineto{\pgfqpoint{4.911070in}{2.297431in}}%
\pgfpathlineto{\pgfqpoint{4.897247in}{2.299138in}}%
\pgfpathlineto{\pgfqpoint{4.883431in}{2.300870in}}%
\pgfpathlineto{\pgfqpoint{4.890983in}{2.308771in}}%
\pgfpathlineto{\pgfqpoint{4.898529in}{2.316612in}}%
\pgfpathlineto{\pgfqpoint{4.906068in}{2.324394in}}%
\pgfpathlineto{\pgfqpoint{4.913601in}{2.332116in}}%
\pgfpathclose%
\pgfusepath{fill}%
\end{pgfscope}%
\begin{pgfscope}%
\pgfpathrectangle{\pgfqpoint{1.150000in}{0.150000in}}{\pgfqpoint{5.700000in}{5.700000in}}%
\pgfusepath{clip}%
\pgfsetbuttcap%
\pgfsetroundjoin%
\definecolor{currentfill}{rgb}{0.273809,0.031497,0.358853}%
\pgfsetfillcolor{currentfill}%
\pgfsetfillopacity{0.700000}%
\pgfsetlinewidth{0.000000pt}%
\definecolor{currentstroke}{rgb}{0.000000,0.000000,0.000000}%
\pgfsetstrokecolor{currentstroke}%
\pgfsetdash{}{0pt}%
\pgfpathmoveto{\pgfqpoint{4.376669in}{2.231027in}}%
\pgfpathlineto{\pgfqpoint{4.390325in}{2.228373in}}%
\pgfpathlineto{\pgfqpoint{4.403988in}{2.225746in}}%
\pgfpathlineto{\pgfqpoint{4.417658in}{2.223145in}}%
\pgfpathlineto{\pgfqpoint{4.431335in}{2.220571in}}%
\pgfpathlineto{\pgfqpoint{4.423611in}{2.212193in}}%
\pgfpathlineto{\pgfqpoint{4.415883in}{2.203789in}}%
\pgfpathlineto{\pgfqpoint{4.408148in}{2.195361in}}%
\pgfpathlineto{\pgfqpoint{4.400408in}{2.186911in}}%
\pgfpathlineto{\pgfqpoint{4.386720in}{2.189567in}}%
\pgfpathlineto{\pgfqpoint{4.373038in}{2.192249in}}%
\pgfpathlineto{\pgfqpoint{4.359364in}{2.194958in}}%
\pgfpathlineto{\pgfqpoint{4.345697in}{2.197694in}}%
\pgfpathlineto{\pgfqpoint{4.353448in}{2.206057in}}%
\pgfpathlineto{\pgfqpoint{4.361194in}{2.214401in}}%
\pgfpathlineto{\pgfqpoint{4.368934in}{2.222725in}}%
\pgfpathlineto{\pgfqpoint{4.376669in}{2.231027in}}%
\pgfpathclose%
\pgfusepath{fill}%
\end{pgfscope}%
\begin{pgfscope}%
\pgfpathrectangle{\pgfqpoint{1.150000in}{0.150000in}}{\pgfqpoint{5.700000in}{5.700000in}}%
\pgfusepath{clip}%
\pgfsetbuttcap%
\pgfsetroundjoin%
\definecolor{currentfill}{rgb}{0.282290,0.145912,0.461510}%
\pgfsetfillcolor{currentfill}%
\pgfsetfillopacity{0.700000}%
\pgfsetlinewidth{0.000000pt}%
\definecolor{currentstroke}{rgb}{0.000000,0.000000,0.000000}%
\pgfsetstrokecolor{currentstroke}%
\pgfsetdash{}{0pt}%
\pgfpathmoveto{\pgfqpoint{5.450465in}{2.432989in}}%
\pgfpathlineto{\pgfqpoint{5.464424in}{2.431840in}}%
\pgfpathlineto{\pgfqpoint{5.478392in}{2.430715in}}%
\pgfpathlineto{\pgfqpoint{5.492367in}{2.429615in}}%
\pgfpathlineto{\pgfqpoint{5.506352in}{2.428540in}}%
\pgfpathlineto{\pgfqpoint{5.499069in}{2.422285in}}%
\pgfpathlineto{\pgfqpoint{5.491779in}{2.415967in}}%
\pgfpathlineto{\pgfqpoint{5.484482in}{2.409584in}}%
\pgfpathlineto{\pgfqpoint{5.477177in}{2.403135in}}%
\pgfpathlineto{\pgfqpoint{5.463177in}{2.404157in}}%
\pgfpathlineto{\pgfqpoint{5.449186in}{2.405203in}}%
\pgfpathlineto{\pgfqpoint{5.435203in}{2.406274in}}%
\pgfpathlineto{\pgfqpoint{5.421228in}{2.407370in}}%
\pgfpathlineto{\pgfqpoint{5.428549in}{2.413868in}}%
\pgfpathlineto{\pgfqpoint{5.435861in}{2.420303in}}%
\pgfpathlineto{\pgfqpoint{5.443167in}{2.426676in}}%
\pgfpathlineto{\pgfqpoint{5.450465in}{2.432989in}}%
\pgfpathclose%
\pgfusepath{fill}%
\end{pgfscope}%
\begin{pgfscope}%
\pgfpathrectangle{\pgfqpoint{1.150000in}{0.150000in}}{\pgfqpoint{5.700000in}{5.700000in}}%
\pgfusepath{clip}%
\pgfsetbuttcap%
\pgfsetroundjoin%
\definecolor{currentfill}{rgb}{0.282623,0.140926,0.457517}%
\pgfsetfillcolor{currentfill}%
\pgfsetfillopacity{0.700000}%
\pgfsetlinewidth{0.000000pt}%
\definecolor{currentstroke}{rgb}{0.000000,0.000000,0.000000}%
\pgfsetstrokecolor{currentstroke}%
\pgfsetdash{}{0pt}%
\pgfpathmoveto{\pgfqpoint{2.696953in}{2.429204in}}%
\pgfpathlineto{\pgfqpoint{2.710325in}{2.421452in}}%
\pgfpathlineto{\pgfqpoint{2.723700in}{2.413747in}}%
\pgfpathlineto{\pgfqpoint{2.737077in}{2.406087in}}%
\pgfpathlineto{\pgfqpoint{2.750457in}{2.398473in}}%
\pgfpathlineto{\pgfqpoint{2.741937in}{2.397991in}}%
\pgfpathlineto{\pgfqpoint{2.733403in}{2.397748in}}%
\pgfpathlineto{\pgfqpoint{2.724853in}{2.397750in}}%
\pgfpathlineto{\pgfqpoint{2.716288in}{2.398004in}}%
\pgfpathlineto{\pgfqpoint{2.702879in}{2.405863in}}%
\pgfpathlineto{\pgfqpoint{2.689473in}{2.413767in}}%
\pgfpathlineto{\pgfqpoint{2.676070in}{2.421717in}}%
\pgfpathlineto{\pgfqpoint{2.662670in}{2.429714in}}%
\pgfpathlineto{\pgfqpoint{2.671264in}{2.429210in}}%
\pgfpathlineto{\pgfqpoint{2.679842in}{2.428961in}}%
\pgfpathlineto{\pgfqpoint{2.688405in}{2.428961in}}%
\pgfpathlineto{\pgfqpoint{2.696953in}{2.429204in}}%
\pgfpathclose%
\pgfusepath{fill}%
\end{pgfscope}%
\begin{pgfscope}%
\pgfpathrectangle{\pgfqpoint{1.150000in}{0.150000in}}{\pgfqpoint{5.700000in}{5.700000in}}%
\pgfusepath{clip}%
\pgfsetbuttcap%
\pgfsetroundjoin%
\definecolor{currentfill}{rgb}{0.267004,0.004874,0.329415}%
\pgfsetfillcolor{currentfill}%
\pgfsetfillopacity{0.700000}%
\pgfsetlinewidth{0.000000pt}%
\definecolor{currentstroke}{rgb}{0.000000,0.000000,0.000000}%
\pgfsetstrokecolor{currentstroke}%
\pgfsetdash{}{0pt}%
\pgfpathmoveto{\pgfqpoint{3.925344in}{2.178279in}}%
\pgfpathlineto{\pgfqpoint{3.938892in}{2.174554in}}%
\pgfpathlineto{\pgfqpoint{3.952446in}{2.170857in}}%
\pgfpathlineto{\pgfqpoint{3.966007in}{2.167190in}}%
\pgfpathlineto{\pgfqpoint{3.979573in}{2.163550in}}%
\pgfpathlineto{\pgfqpoint{3.971687in}{2.155790in}}%
\pgfpathlineto{\pgfqpoint{3.963794in}{2.148061in}}%
\pgfpathlineto{\pgfqpoint{3.955896in}{2.140367in}}%
\pgfpathlineto{\pgfqpoint{3.947992in}{2.132710in}}%
\pgfpathlineto{\pgfqpoint{3.934412in}{2.136484in}}%
\pgfpathlineto{\pgfqpoint{3.920838in}{2.140286in}}%
\pgfpathlineto{\pgfqpoint{3.907270in}{2.144117in}}%
\pgfpathlineto{\pgfqpoint{3.893709in}{2.147977in}}%
\pgfpathlineto{\pgfqpoint{3.901627in}{2.155494in}}%
\pgfpathlineto{\pgfqpoint{3.909538in}{2.163052in}}%
\pgfpathlineto{\pgfqpoint{3.917444in}{2.170648in}}%
\pgfpathlineto{\pgfqpoint{3.925344in}{2.178279in}}%
\pgfpathclose%
\pgfusepath{fill}%
\end{pgfscope}%
\begin{pgfscope}%
\pgfpathrectangle{\pgfqpoint{1.150000in}{0.150000in}}{\pgfqpoint{5.700000in}{5.700000in}}%
\pgfusepath{clip}%
\pgfsetbuttcap%
\pgfsetroundjoin%
\definecolor{currentfill}{rgb}{0.278791,0.062145,0.386592}%
\pgfsetfillcolor{currentfill}%
\pgfsetfillopacity{0.700000}%
\pgfsetlinewidth{0.000000pt}%
\definecolor{currentstroke}{rgb}{0.000000,0.000000,0.000000}%
\pgfsetstrokecolor{currentstroke}%
\pgfsetdash{}{0pt}%
\pgfpathmoveto{\pgfqpoint{3.085415in}{2.275828in}}%
\pgfpathlineto{\pgfqpoint{3.098819in}{2.269529in}}%
\pgfpathlineto{\pgfqpoint{3.112227in}{2.263268in}}%
\pgfpathlineto{\pgfqpoint{3.125640in}{2.257043in}}%
\pgfpathlineto{\pgfqpoint{3.139056in}{2.250857in}}%
\pgfpathlineto{\pgfqpoint{3.130783in}{2.247411in}}%
\pgfpathlineto{\pgfqpoint{3.122500in}{2.244138in}}%
\pgfpathlineto{\pgfqpoint{3.114206in}{2.241042in}}%
\pgfpathlineto{\pgfqpoint{3.105900in}{2.238129in}}%
\pgfpathlineto{\pgfqpoint{3.092461in}{2.244531in}}%
\pgfpathlineto{\pgfqpoint{3.079027in}{2.250971in}}%
\pgfpathlineto{\pgfqpoint{3.065596in}{2.257447in}}%
\pgfpathlineto{\pgfqpoint{3.052169in}{2.263962in}}%
\pgfpathlineto{\pgfqpoint{3.060497in}{2.266654in}}%
\pgfpathlineto{\pgfqpoint{3.068814in}{2.269533in}}%
\pgfpathlineto{\pgfqpoint{3.077120in}{2.272593in}}%
\pgfpathlineto{\pgfqpoint{3.085415in}{2.275828in}}%
\pgfpathclose%
\pgfusepath{fill}%
\end{pgfscope}%
\begin{pgfscope}%
\pgfpathrectangle{\pgfqpoint{1.150000in}{0.150000in}}{\pgfqpoint{5.700000in}{5.700000in}}%
\pgfusepath{clip}%
\pgfsetbuttcap%
\pgfsetroundjoin%
\definecolor{currentfill}{rgb}{0.280255,0.165693,0.476498}%
\pgfsetfillcolor{currentfill}%
\pgfsetfillopacity{0.700000}%
\pgfsetlinewidth{0.000000pt}%
\definecolor{currentstroke}{rgb}{0.000000,0.000000,0.000000}%
\pgfsetstrokecolor{currentstroke}%
\pgfsetdash{}{0pt}%
\pgfpathmoveto{\pgfqpoint{5.676263in}{2.467682in}}%
\pgfpathlineto{\pgfqpoint{5.690289in}{2.466679in}}%
\pgfpathlineto{\pgfqpoint{5.704325in}{2.465700in}}%
\pgfpathlineto{\pgfqpoint{5.718369in}{2.464745in}}%
\pgfpathlineto{\pgfqpoint{5.732422in}{2.463815in}}%
\pgfpathlineto{\pgfqpoint{5.725249in}{2.458237in}}%
\pgfpathlineto{\pgfqpoint{5.718069in}{2.452605in}}%
\pgfpathlineto{\pgfqpoint{5.710881in}{2.446918in}}%
\pgfpathlineto{\pgfqpoint{5.703686in}{2.441171in}}%
\pgfpathlineto{\pgfqpoint{5.689615in}{2.442020in}}%
\pgfpathlineto{\pgfqpoint{5.675554in}{2.442894in}}%
\pgfpathlineto{\pgfqpoint{5.661501in}{2.443792in}}%
\pgfpathlineto{\pgfqpoint{5.647456in}{2.444714in}}%
\pgfpathlineto{\pgfqpoint{5.654669in}{2.450536in}}%
\pgfpathlineto{\pgfqpoint{5.661874in}{2.456303in}}%
\pgfpathlineto{\pgfqpoint{5.669072in}{2.462018in}}%
\pgfpathlineto{\pgfqpoint{5.676263in}{2.467682in}}%
\pgfpathclose%
\pgfusepath{fill}%
\end{pgfscope}%
\begin{pgfscope}%
\pgfpathrectangle{\pgfqpoint{1.150000in}{0.150000in}}{\pgfqpoint{5.700000in}{5.700000in}}%
\pgfusepath{clip}%
\pgfsetbuttcap%
\pgfsetroundjoin%
\definecolor{currentfill}{rgb}{0.277941,0.056324,0.381191}%
\pgfsetfillcolor{currentfill}%
\pgfsetfillopacity{0.700000}%
\pgfsetlinewidth{0.000000pt}%
\definecolor{currentstroke}{rgb}{0.000000,0.000000,0.000000}%
\pgfsetstrokecolor{currentstroke}%
\pgfsetdash{}{0pt}%
\pgfpathmoveto{\pgfqpoint{4.602401in}{2.267858in}}%
\pgfpathlineto{\pgfqpoint{4.616119in}{2.265644in}}%
\pgfpathlineto{\pgfqpoint{4.629845in}{2.263457in}}%
\pgfpathlineto{\pgfqpoint{4.643578in}{2.261295in}}%
\pgfpathlineto{\pgfqpoint{4.657318in}{2.259159in}}%
\pgfpathlineto{\pgfqpoint{4.649676in}{2.250884in}}%
\pgfpathlineto{\pgfqpoint{4.642027in}{2.242562in}}%
\pgfpathlineto{\pgfqpoint{4.634373in}{2.234197in}}%
\pgfpathlineto{\pgfqpoint{4.626713in}{2.225787in}}%
\pgfpathlineto{\pgfqpoint{4.612961in}{2.227978in}}%
\pgfpathlineto{\pgfqpoint{4.599216in}{2.230195in}}%
\pgfpathlineto{\pgfqpoint{4.585479in}{2.232437in}}%
\pgfpathlineto{\pgfqpoint{4.571749in}{2.234706in}}%
\pgfpathlineto{\pgfqpoint{4.579421in}{2.243055in}}%
\pgfpathlineto{\pgfqpoint{4.587087in}{2.251364in}}%
\pgfpathlineto{\pgfqpoint{4.594747in}{2.259632in}}%
\pgfpathlineto{\pgfqpoint{4.602401in}{2.267858in}}%
\pgfpathclose%
\pgfusepath{fill}%
\end{pgfscope}%
\begin{pgfscope}%
\pgfpathrectangle{\pgfqpoint{1.150000in}{0.150000in}}{\pgfqpoint{5.700000in}{5.700000in}}%
\pgfusepath{clip}%
\pgfsetbuttcap%
\pgfsetroundjoin%
\definecolor{currentfill}{rgb}{0.278012,0.180367,0.486697}%
\pgfsetfillcolor{currentfill}%
\pgfsetfillopacity{0.700000}%
\pgfsetlinewidth{0.000000pt}%
\definecolor{currentstroke}{rgb}{0.000000,0.000000,0.000000}%
\pgfsetstrokecolor{currentstroke}%
\pgfsetdash{}{0pt}%
\pgfpathmoveto{\pgfqpoint{5.901980in}{2.498698in}}%
\pgfpathlineto{\pgfqpoint{5.916073in}{2.497784in}}%
\pgfpathlineto{\pgfqpoint{5.930175in}{2.496894in}}%
\pgfpathlineto{\pgfqpoint{5.944286in}{2.496028in}}%
\pgfpathlineto{\pgfqpoint{5.958405in}{2.495186in}}%
\pgfpathlineto{\pgfqpoint{5.951347in}{2.490262in}}%
\pgfpathlineto{\pgfqpoint{5.944282in}{2.485301in}}%
\pgfpathlineto{\pgfqpoint{5.937210in}{2.480299in}}%
\pgfpathlineto{\pgfqpoint{5.930131in}{2.475252in}}%
\pgfpathlineto{\pgfqpoint{5.915991in}{2.475985in}}%
\pgfpathlineto{\pgfqpoint{5.901861in}{2.476743in}}%
\pgfpathlineto{\pgfqpoint{5.887740in}{2.477525in}}%
\pgfpathlineto{\pgfqpoint{5.873627in}{2.478331in}}%
\pgfpathlineto{\pgfqpoint{5.880726in}{2.483482in}}%
\pgfpathlineto{\pgfqpoint{5.887818in}{2.488591in}}%
\pgfpathlineto{\pgfqpoint{5.894903in}{2.493662in}}%
\pgfpathlineto{\pgfqpoint{5.901980in}{2.498698in}}%
\pgfpathclose%
\pgfusepath{fill}%
\end{pgfscope}%
\begin{pgfscope}%
\pgfpathrectangle{\pgfqpoint{1.150000in}{0.150000in}}{\pgfqpoint{5.700000in}{5.700000in}}%
\pgfusepath{clip}%
\pgfsetbuttcap%
\pgfsetroundjoin%
\definecolor{currentfill}{rgb}{0.283197,0.115680,0.436115}%
\pgfsetfillcolor{currentfill}%
\pgfsetfillopacity{0.700000}%
\pgfsetlinewidth{0.000000pt}%
\definecolor{currentstroke}{rgb}{0.000000,0.000000,0.000000}%
\pgfsetstrokecolor{currentstroke}%
\pgfsetdash{}{0pt}%
\pgfpathmoveto{\pgfqpoint{5.139475in}{2.372734in}}%
\pgfpathlineto{\pgfqpoint{5.153346in}{2.371308in}}%
\pgfpathlineto{\pgfqpoint{5.167226in}{2.369906in}}%
\pgfpathlineto{\pgfqpoint{5.181114in}{2.368529in}}%
\pgfpathlineto{\pgfqpoint{5.195010in}{2.367177in}}%
\pgfpathlineto{\pgfqpoint{5.187583in}{2.359980in}}%
\pgfpathlineto{\pgfqpoint{5.180148in}{2.352716in}}%
\pgfpathlineto{\pgfqpoint{5.172707in}{2.345383in}}%
\pgfpathlineto{\pgfqpoint{5.165259in}{2.337980in}}%
\pgfpathlineto{\pgfqpoint{5.151349in}{2.339319in}}%
\pgfpathlineto{\pgfqpoint{5.137448in}{2.340684in}}%
\pgfpathlineto{\pgfqpoint{5.123555in}{2.342073in}}%
\pgfpathlineto{\pgfqpoint{5.109670in}{2.343487in}}%
\pgfpathlineto{\pgfqpoint{5.117131in}{2.350898in}}%
\pgfpathlineto{\pgfqpoint{5.124586in}{2.358242in}}%
\pgfpathlineto{\pgfqpoint{5.132034in}{2.365520in}}%
\pgfpathlineto{\pgfqpoint{5.139475in}{2.372734in}}%
\pgfpathclose%
\pgfusepath{fill}%
\end{pgfscope}%
\begin{pgfscope}%
\pgfpathrectangle{\pgfqpoint{1.150000in}{0.150000in}}{\pgfqpoint{5.700000in}{5.700000in}}%
\pgfusepath{clip}%
\pgfsetbuttcap%
\pgfsetroundjoin%
\definecolor{currentfill}{rgb}{0.268510,0.009605,0.335427}%
\pgfsetfillcolor{currentfill}%
\pgfsetfillopacity{0.700000}%
\pgfsetlinewidth{0.000000pt}%
\definecolor{currentstroke}{rgb}{0.000000,0.000000,0.000000}%
\pgfsetstrokecolor{currentstroke}%
\pgfsetdash{}{0pt}%
\pgfpathmoveto{\pgfqpoint{3.559535in}{2.179344in}}%
\pgfpathlineto{\pgfqpoint{3.573013in}{2.174573in}}%
\pgfpathlineto{\pgfqpoint{3.586497in}{2.169833in}}%
\pgfpathlineto{\pgfqpoint{3.599986in}{2.165125in}}%
\pgfpathlineto{\pgfqpoint{3.613480in}{2.160448in}}%
\pgfpathlineto{\pgfqpoint{3.605444in}{2.154138in}}%
\pgfpathlineto{\pgfqpoint{3.597400in}{2.147920in}}%
\pgfpathlineto{\pgfqpoint{3.589349in}{2.141796in}}%
\pgfpathlineto{\pgfqpoint{3.581290in}{2.135773in}}%
\pgfpathlineto{\pgfqpoint{3.567779in}{2.140625in}}%
\pgfpathlineto{\pgfqpoint{3.554274in}{2.145507in}}%
\pgfpathlineto{\pgfqpoint{3.540773in}{2.150421in}}%
\pgfpathlineto{\pgfqpoint{3.527278in}{2.155366in}}%
\pgfpathlineto{\pgfqpoint{3.535354in}{2.161210in}}%
\pgfpathlineto{\pgfqpoint{3.543422in}{2.167157in}}%
\pgfpathlineto{\pgfqpoint{3.551482in}{2.173203in}}%
\pgfpathlineto{\pgfqpoint{3.559535in}{2.179344in}}%
\pgfpathclose%
\pgfusepath{fill}%
\end{pgfscope}%
\begin{pgfscope}%
\pgfpathrectangle{\pgfqpoint{1.150000in}{0.150000in}}{\pgfqpoint{5.700000in}{5.700000in}}%
\pgfusepath{clip}%
\pgfsetbuttcap%
\pgfsetroundjoin%
\definecolor{currentfill}{rgb}{0.269944,0.014625,0.341379}%
\pgfsetfillcolor{currentfill}%
\pgfsetfillopacity{0.700000}%
\pgfsetlinewidth{0.000000pt}%
\definecolor{currentstroke}{rgb}{0.000000,0.000000,0.000000}%
\pgfsetstrokecolor{currentstroke}%
\pgfsetdash{}{0pt}%
\pgfpathmoveto{\pgfqpoint{3.419502in}{2.196084in}}%
\pgfpathlineto{\pgfqpoint{3.432956in}{2.190881in}}%
\pgfpathlineto{\pgfqpoint{3.446416in}{2.185710in}}%
\pgfpathlineto{\pgfqpoint{3.459880in}{2.180572in}}%
\pgfpathlineto{\pgfqpoint{3.473350in}{2.175466in}}%
\pgfpathlineto{\pgfqpoint{3.465249in}{2.169913in}}%
\pgfpathlineto{\pgfqpoint{3.457140in}{2.164475in}}%
\pgfpathlineto{\pgfqpoint{3.449023in}{2.159158in}}%
\pgfpathlineto{\pgfqpoint{3.440897in}{2.153965in}}%
\pgfpathlineto{\pgfqpoint{3.427410in}{2.159258in}}%
\pgfpathlineto{\pgfqpoint{3.413927in}{2.164584in}}%
\pgfpathlineto{\pgfqpoint{3.400449in}{2.169943in}}%
\pgfpathlineto{\pgfqpoint{3.386977in}{2.175334in}}%
\pgfpathlineto{\pgfqpoint{3.395121in}{2.180334in}}%
\pgfpathlineto{\pgfqpoint{3.403256in}{2.185462in}}%
\pgfpathlineto{\pgfqpoint{3.411383in}{2.190714in}}%
\pgfpathlineto{\pgfqpoint{3.419502in}{2.196084in}}%
\pgfpathclose%
\pgfusepath{fill}%
\end{pgfscope}%
\begin{pgfscope}%
\pgfpathrectangle{\pgfqpoint{1.150000in}{0.150000in}}{\pgfqpoint{5.700000in}{5.700000in}}%
\pgfusepath{clip}%
\pgfsetbuttcap%
\pgfsetroundjoin%
\definecolor{currentfill}{rgb}{0.281446,0.084320,0.407414}%
\pgfsetfillcolor{currentfill}%
\pgfsetfillopacity{0.700000}%
\pgfsetlinewidth{0.000000pt}%
\definecolor{currentstroke}{rgb}{0.000000,0.000000,0.000000}%
\pgfsetstrokecolor{currentstroke}%
\pgfsetdash{}{0pt}%
\pgfpathmoveto{\pgfqpoint{4.828247in}{2.308054in}}%
\pgfpathlineto{\pgfqpoint{4.842031in}{2.306220in}}%
\pgfpathlineto{\pgfqpoint{4.855824in}{2.304411in}}%
\pgfpathlineto{\pgfqpoint{4.869624in}{2.302628in}}%
\pgfpathlineto{\pgfqpoint{4.883431in}{2.300870in}}%
\pgfpathlineto{\pgfqpoint{4.875873in}{2.292910in}}%
\pgfpathlineto{\pgfqpoint{4.868308in}{2.284890in}}%
\pgfpathlineto{\pgfqpoint{4.860737in}{2.276811in}}%
\pgfpathlineto{\pgfqpoint{4.853160in}{2.268673in}}%
\pgfpathlineto{\pgfqpoint{4.839341in}{2.270459in}}%
\pgfpathlineto{\pgfqpoint{4.825529in}{2.272271in}}%
\pgfpathlineto{\pgfqpoint{4.811725in}{2.274108in}}%
\pgfpathlineto{\pgfqpoint{4.797928in}{2.275970in}}%
\pgfpathlineto{\pgfqpoint{4.805517in}{2.284075in}}%
\pgfpathlineto{\pgfqpoint{4.813100in}{2.292124in}}%
\pgfpathlineto{\pgfqpoint{4.820677in}{2.300117in}}%
\pgfpathlineto{\pgfqpoint{4.828247in}{2.308054in}}%
\pgfpathclose%
\pgfusepath{fill}%
\end{pgfscope}%
\begin{pgfscope}%
\pgfpathrectangle{\pgfqpoint{1.150000in}{0.150000in}}{\pgfqpoint{5.700000in}{5.700000in}}%
\pgfusepath{clip}%
\pgfsetbuttcap%
\pgfsetroundjoin%
\definecolor{currentfill}{rgb}{0.272594,0.025563,0.353093}%
\pgfsetfillcolor{currentfill}%
\pgfsetfillopacity{0.700000}%
\pgfsetlinewidth{0.000000pt}%
\definecolor{currentstroke}{rgb}{0.000000,0.000000,0.000000}%
\pgfsetstrokecolor{currentstroke}%
\pgfsetdash{}{0pt}%
\pgfpathmoveto{\pgfqpoint{4.291095in}{2.208905in}}%
\pgfpathlineto{\pgfqpoint{4.304735in}{2.206061in}}%
\pgfpathlineto{\pgfqpoint{4.318382in}{2.203245in}}%
\pgfpathlineto{\pgfqpoint{4.332036in}{2.200456in}}%
\pgfpathlineto{\pgfqpoint{4.345697in}{2.197694in}}%
\pgfpathlineto{\pgfqpoint{4.337939in}{2.189313in}}%
\pgfpathlineto{\pgfqpoint{4.330177in}{2.180918in}}%
\pgfpathlineto{\pgfqpoint{4.322408in}{2.172509in}}%
\pgfpathlineto{\pgfqpoint{4.314634in}{2.164090in}}%
\pgfpathlineto{\pgfqpoint{4.300962in}{2.166947in}}%
\pgfpathlineto{\pgfqpoint{4.287297in}{2.169832in}}%
\pgfpathlineto{\pgfqpoint{4.273638in}{2.172743in}}%
\pgfpathlineto{\pgfqpoint{4.259986in}{2.175681in}}%
\pgfpathlineto{\pgfqpoint{4.267772in}{2.184001in}}%
\pgfpathlineto{\pgfqpoint{4.275552in}{2.192312in}}%
\pgfpathlineto{\pgfqpoint{4.283326in}{2.200614in}}%
\pgfpathlineto{\pgfqpoint{4.291095in}{2.208905in}}%
\pgfpathclose%
\pgfusepath{fill}%
\end{pgfscope}%
\begin{pgfscope}%
\pgfpathrectangle{\pgfqpoint{1.150000in}{0.150000in}}{\pgfqpoint{5.700000in}{5.700000in}}%
\pgfusepath{clip}%
\pgfsetbuttcap%
\pgfsetroundjoin%
\definecolor{currentfill}{rgb}{0.268510,0.009605,0.335427}%
\pgfsetfillcolor{currentfill}%
\pgfsetfillopacity{0.700000}%
\pgfsetlinewidth{0.000000pt}%
\definecolor{currentstroke}{rgb}{0.000000,0.000000,0.000000}%
\pgfsetstrokecolor{currentstroke}%
\pgfsetdash{}{0pt}%
\pgfpathmoveto{\pgfqpoint{4.065338in}{2.181061in}}%
\pgfpathlineto{\pgfqpoint{4.078923in}{2.177685in}}%
\pgfpathlineto{\pgfqpoint{4.092515in}{2.174336in}}%
\pgfpathlineto{\pgfqpoint{4.106113in}{2.171016in}}%
\pgfpathlineto{\pgfqpoint{4.119717in}{2.167723in}}%
\pgfpathlineto{\pgfqpoint{4.111879in}{2.159633in}}%
\pgfpathlineto{\pgfqpoint{4.104036in}{2.151556in}}%
\pgfpathlineto{\pgfqpoint{4.096186in}{2.143495in}}%
\pgfpathlineto{\pgfqpoint{4.088331in}{2.135453in}}%
\pgfpathlineto{\pgfqpoint{4.074714in}{2.138867in}}%
\pgfpathlineto{\pgfqpoint{4.061104in}{2.142309in}}%
\pgfpathlineto{\pgfqpoint{4.047500in}{2.145779in}}%
\pgfpathlineto{\pgfqpoint{4.033902in}{2.149277in}}%
\pgfpathlineto{\pgfqpoint{4.041769in}{2.157193in}}%
\pgfpathlineto{\pgfqpoint{4.049631in}{2.165130in}}%
\pgfpathlineto{\pgfqpoint{4.057487in}{2.173087in}}%
\pgfpathlineto{\pgfqpoint{4.065338in}{2.181061in}}%
\pgfpathclose%
\pgfusepath{fill}%
\end{pgfscope}%
\begin{pgfscope}%
\pgfpathrectangle{\pgfqpoint{1.150000in}{0.150000in}}{\pgfqpoint{5.700000in}{5.700000in}}%
\pgfusepath{clip}%
\pgfsetbuttcap%
\pgfsetroundjoin%
\definecolor{currentfill}{rgb}{0.282884,0.135920,0.453427}%
\pgfsetfillcolor{currentfill}%
\pgfsetfillopacity{0.700000}%
\pgfsetlinewidth{0.000000pt}%
\definecolor{currentstroke}{rgb}{0.000000,0.000000,0.000000}%
\pgfsetstrokecolor{currentstroke}%
\pgfsetdash{}{0pt}%
\pgfpathmoveto{\pgfqpoint{5.365414in}{2.412000in}}%
\pgfpathlineto{\pgfqpoint{5.379355in}{2.410806in}}%
\pgfpathlineto{\pgfqpoint{5.393305in}{2.409636in}}%
\pgfpathlineto{\pgfqpoint{5.407262in}{2.408491in}}%
\pgfpathlineto{\pgfqpoint{5.421228in}{2.407370in}}%
\pgfpathlineto{\pgfqpoint{5.413901in}{2.400806in}}%
\pgfpathlineto{\pgfqpoint{5.406566in}{2.394176in}}%
\pgfpathlineto{\pgfqpoint{5.399224in}{2.387476in}}%
\pgfpathlineto{\pgfqpoint{5.391875in}{2.380706in}}%
\pgfpathlineto{\pgfqpoint{5.377894in}{2.381786in}}%
\pgfpathlineto{\pgfqpoint{5.363921in}{2.382891in}}%
\pgfpathlineto{\pgfqpoint{5.349957in}{2.384021in}}%
\pgfpathlineto{\pgfqpoint{5.336001in}{2.385176in}}%
\pgfpathlineto{\pgfqpoint{5.343365in}{2.391981in}}%
\pgfpathlineto{\pgfqpoint{5.350722in}{2.398719in}}%
\pgfpathlineto{\pgfqpoint{5.358071in}{2.405391in}}%
\pgfpathlineto{\pgfqpoint{5.365414in}{2.412000in}}%
\pgfpathclose%
\pgfusepath{fill}%
\end{pgfscope}%
\begin{pgfscope}%
\pgfpathrectangle{\pgfqpoint{1.150000in}{0.150000in}}{\pgfqpoint{5.700000in}{5.700000in}}%
\pgfusepath{clip}%
\pgfsetbuttcap%
\pgfsetroundjoin%
\definecolor{currentfill}{rgb}{0.267004,0.004874,0.329415}%
\pgfsetfillcolor{currentfill}%
\pgfsetfillopacity{0.700000}%
\pgfsetlinewidth{0.000000pt}%
\definecolor{currentstroke}{rgb}{0.000000,0.000000,0.000000}%
\pgfsetstrokecolor{currentstroke}%
\pgfsetdash{}{0pt}%
\pgfpathmoveto{\pgfqpoint{3.699524in}{2.168767in}}%
\pgfpathlineto{\pgfqpoint{3.713031in}{2.164405in}}%
\pgfpathlineto{\pgfqpoint{3.726542in}{2.160073in}}%
\pgfpathlineto{\pgfqpoint{3.740060in}{2.155771in}}%
\pgfpathlineto{\pgfqpoint{3.753583in}{2.151499in}}%
\pgfpathlineto{\pgfqpoint{3.745605in}{2.144551in}}%
\pgfpathlineto{\pgfqpoint{3.737620in}{2.137673in}}%
\pgfpathlineto{\pgfqpoint{3.729629in}{2.130866in}}%
\pgfpathlineto{\pgfqpoint{3.721631in}{2.124136in}}%
\pgfpathlineto{\pgfqpoint{3.708093in}{2.128570in}}%
\pgfpathlineto{\pgfqpoint{3.694560in}{2.133033in}}%
\pgfpathlineto{\pgfqpoint{3.681033in}{2.137526in}}%
\pgfpathlineto{\pgfqpoint{3.667512in}{2.142049in}}%
\pgfpathlineto{\pgfqpoint{3.675525in}{2.148613in}}%
\pgfpathlineto{\pgfqpoint{3.683532in}{2.155256in}}%
\pgfpathlineto{\pgfqpoint{3.691531in}{2.161976in}}%
\pgfpathlineto{\pgfqpoint{3.699524in}{2.168767in}}%
\pgfpathclose%
\pgfusepath{fill}%
\end{pgfscope}%
\begin{pgfscope}%
\pgfpathrectangle{\pgfqpoint{1.150000in}{0.150000in}}{\pgfqpoint{5.700000in}{5.700000in}}%
\pgfusepath{clip}%
\pgfsetbuttcap%
\pgfsetroundjoin%
\definecolor{currentfill}{rgb}{0.275191,0.194905,0.496005}%
\pgfsetfillcolor{currentfill}%
\pgfsetfillopacity{0.700000}%
\pgfsetlinewidth{0.000000pt}%
\definecolor{currentstroke}{rgb}{0.000000,0.000000,0.000000}%
\pgfsetstrokecolor{currentstroke}%
\pgfsetdash{}{0pt}%
\pgfpathmoveto{\pgfqpoint{2.502050in}{2.529482in}}%
\pgfpathlineto{\pgfqpoint{2.515423in}{2.520889in}}%
\pgfpathlineto{\pgfqpoint{2.528798in}{2.512348in}}%
\pgfpathlineto{\pgfqpoint{2.542175in}{2.503859in}}%
\pgfpathlineto{\pgfqpoint{2.555554in}{2.495421in}}%
\pgfpathlineto{\pgfqpoint{2.546883in}{2.496696in}}%
\pgfpathlineto{\pgfqpoint{2.538195in}{2.498246in}}%
\pgfpathlineto{\pgfqpoint{2.529488in}{2.500079in}}%
\pgfpathlineto{\pgfqpoint{2.520764in}{2.502201in}}%
\pgfpathlineto{\pgfqpoint{2.507353in}{2.510898in}}%
\pgfpathlineto{\pgfqpoint{2.493944in}{2.519647in}}%
\pgfpathlineto{\pgfqpoint{2.480537in}{2.528449in}}%
\pgfpathlineto{\pgfqpoint{2.467132in}{2.537303in}}%
\pgfpathlineto{\pgfqpoint{2.475889in}{2.534915in}}%
\pgfpathlineto{\pgfqpoint{2.484627in}{2.532820in}}%
\pgfpathlineto{\pgfqpoint{2.493347in}{2.531012in}}%
\pgfpathlineto{\pgfqpoint{2.502050in}{2.529482in}}%
\pgfpathclose%
\pgfusepath{fill}%
\end{pgfscope}%
\begin{pgfscope}%
\pgfpathrectangle{\pgfqpoint{1.150000in}{0.150000in}}{\pgfqpoint{5.700000in}{5.700000in}}%
\pgfusepath{clip}%
\pgfsetbuttcap%
\pgfsetroundjoin%
\definecolor{currentfill}{rgb}{0.273809,0.031497,0.358853}%
\pgfsetfillcolor{currentfill}%
\pgfsetfillopacity{0.700000}%
\pgfsetlinewidth{0.000000pt}%
\definecolor{currentstroke}{rgb}{0.000000,0.000000,0.000000}%
\pgfsetstrokecolor{currentstroke}%
\pgfsetdash{}{0pt}%
\pgfpathmoveto{\pgfqpoint{3.279365in}{2.219677in}}%
\pgfpathlineto{\pgfqpoint{3.292800in}{2.214015in}}%
\pgfpathlineto{\pgfqpoint{3.306240in}{2.208387in}}%
\pgfpathlineto{\pgfqpoint{3.319684in}{2.202793in}}%
\pgfpathlineto{\pgfqpoint{3.333133in}{2.197234in}}%
\pgfpathlineto{\pgfqpoint{3.324961in}{2.192563in}}%
\pgfpathlineto{\pgfqpoint{3.316780in}{2.188034in}}%
\pgfpathlineto{\pgfqpoint{3.308590in}{2.183651in}}%
\pgfpathlineto{\pgfqpoint{3.300390in}{2.179419in}}%
\pgfpathlineto{\pgfqpoint{3.286921in}{2.185180in}}%
\pgfpathlineto{\pgfqpoint{3.273457in}{2.190975in}}%
\pgfpathlineto{\pgfqpoint{3.259997in}{2.196804in}}%
\pgfpathlineto{\pgfqpoint{3.246542in}{2.202668in}}%
\pgfpathlineto{\pgfqpoint{3.254762in}{2.206693in}}%
\pgfpathlineto{\pgfqpoint{3.262972in}{2.210873in}}%
\pgfpathlineto{\pgfqpoint{3.271173in}{2.215203in}}%
\pgfpathlineto{\pgfqpoint{3.279365in}{2.219677in}}%
\pgfpathclose%
\pgfusepath{fill}%
\end{pgfscope}%
\begin{pgfscope}%
\pgfpathrectangle{\pgfqpoint{1.150000in}{0.150000in}}{\pgfqpoint{5.700000in}{5.700000in}}%
\pgfusepath{clip}%
\pgfsetbuttcap%
\pgfsetroundjoin%
\definecolor{currentfill}{rgb}{0.276022,0.044167,0.370164}%
\pgfsetfillcolor{currentfill}%
\pgfsetfillopacity{0.700000}%
\pgfsetlinewidth{0.000000pt}%
\definecolor{currentstroke}{rgb}{0.000000,0.000000,0.000000}%
\pgfsetstrokecolor{currentstroke}%
\pgfsetdash{}{0pt}%
\pgfpathmoveto{\pgfqpoint{4.516902in}{2.244042in}}%
\pgfpathlineto{\pgfqpoint{4.530603in}{2.241669in}}%
\pgfpathlineto{\pgfqpoint{4.544311in}{2.239322in}}%
\pgfpathlineto{\pgfqpoint{4.558027in}{2.237001in}}%
\pgfpathlineto{\pgfqpoint{4.571749in}{2.234706in}}%
\pgfpathlineto{\pgfqpoint{4.564072in}{2.226318in}}%
\pgfpathlineto{\pgfqpoint{4.556389in}{2.217891in}}%
\pgfpathlineto{\pgfqpoint{4.548700in}{2.209429in}}%
\pgfpathlineto{\pgfqpoint{4.541005in}{2.200931in}}%
\pgfpathlineto{\pgfqpoint{4.527271in}{2.203294in}}%
\pgfpathlineto{\pgfqpoint{4.513545in}{2.205683in}}%
\pgfpathlineto{\pgfqpoint{4.499825in}{2.208099in}}%
\pgfpathlineto{\pgfqpoint{4.486113in}{2.210541in}}%
\pgfpathlineto{\pgfqpoint{4.493819in}{2.218965in}}%
\pgfpathlineto{\pgfqpoint{4.501519in}{2.227358in}}%
\pgfpathlineto{\pgfqpoint{4.509214in}{2.235717in}}%
\pgfpathlineto{\pgfqpoint{4.516902in}{2.244042in}}%
\pgfpathclose%
\pgfusepath{fill}%
\end{pgfscope}%
\begin{pgfscope}%
\pgfpathrectangle{\pgfqpoint{1.150000in}{0.150000in}}{\pgfqpoint{5.700000in}{5.700000in}}%
\pgfusepath{clip}%
\pgfsetbuttcap%
\pgfsetroundjoin%
\definecolor{currentfill}{rgb}{0.281924,0.089666,0.412415}%
\pgfsetfillcolor{currentfill}%
\pgfsetfillopacity{0.700000}%
\pgfsetlinewidth{0.000000pt}%
\definecolor{currentstroke}{rgb}{0.000000,0.000000,0.000000}%
\pgfsetstrokecolor{currentstroke}%
\pgfsetdash{}{0pt}%
\pgfpathmoveto{\pgfqpoint{2.944890in}{2.317474in}}%
\pgfpathlineto{\pgfqpoint{2.958287in}{2.310646in}}%
\pgfpathlineto{\pgfqpoint{2.971688in}{2.303859in}}%
\pgfpathlineto{\pgfqpoint{2.985092in}{2.297112in}}%
\pgfpathlineto{\pgfqpoint{2.998500in}{2.290404in}}%
\pgfpathlineto{\pgfqpoint{2.990136in}{2.288127in}}%
\pgfpathlineto{\pgfqpoint{2.981760in}{2.286052in}}%
\pgfpathlineto{\pgfqpoint{2.973372in}{2.284184in}}%
\pgfpathlineto{\pgfqpoint{2.964971in}{2.282528in}}%
\pgfpathlineto{\pgfqpoint{2.951538in}{2.289466in}}%
\pgfpathlineto{\pgfqpoint{2.938109in}{2.296442in}}%
\pgfpathlineto{\pgfqpoint{2.924684in}{2.303459in}}%
\pgfpathlineto{\pgfqpoint{2.911262in}{2.310516in}}%
\pgfpathlineto{\pgfqpoint{2.919689in}{2.311937in}}%
\pgfpathlineto{\pgfqpoint{2.928102in}{2.313574in}}%
\pgfpathlineto{\pgfqpoint{2.936502in}{2.315422in}}%
\pgfpathlineto{\pgfqpoint{2.944890in}{2.317474in}}%
\pgfpathclose%
\pgfusepath{fill}%
\end{pgfscope}%
\begin{pgfscope}%
\pgfpathrectangle{\pgfqpoint{1.150000in}{0.150000in}}{\pgfqpoint{5.700000in}{5.700000in}}%
\pgfusepath{clip}%
\pgfsetbuttcap%
\pgfsetroundjoin%
\definecolor{currentfill}{rgb}{0.276194,0.190074,0.493001}%
\pgfsetfillcolor{currentfill}%
\pgfsetfillopacity{0.700000}%
\pgfsetlinewidth{0.000000pt}%
\definecolor{currentstroke}{rgb}{0.000000,0.000000,0.000000}%
\pgfsetstrokecolor{currentstroke}%
\pgfsetdash{}{0pt}%
\pgfpathmoveto{\pgfqpoint{6.043050in}{2.510962in}}%
\pgfpathlineto{\pgfqpoint{6.057193in}{2.510119in}}%
\pgfpathlineto{\pgfqpoint{6.071344in}{2.509299in}}%
\pgfpathlineto{\pgfqpoint{6.085505in}{2.508504in}}%
\pgfpathlineto{\pgfqpoint{6.078511in}{2.503917in}}%
\pgfpathlineto{\pgfqpoint{6.071511in}{2.499301in}}%
\pgfpathlineto{\pgfqpoint{6.064504in}{2.494653in}}%
\pgfpathlineto{\pgfqpoint{6.057489in}{2.489968in}}%
\pgfpathlineto{\pgfqpoint{6.043307in}{2.490641in}}%
\pgfpathlineto{\pgfqpoint{6.029135in}{2.491338in}}%
\pgfpathlineto{\pgfqpoint{6.014971in}{2.492059in}}%
\pgfpathlineto{\pgfqpoint{6.022001in}{2.496832in}}%
\pgfpathlineto{\pgfqpoint{6.029024in}{2.501570in}}%
\pgfpathlineto{\pgfqpoint{6.036040in}{2.506279in}}%
\pgfpathlineto{\pgfqpoint{6.043050in}{2.510962in}}%
\pgfpathclose%
\pgfusepath{fill}%
\end{pgfscope}%
\begin{pgfscope}%
\pgfpathrectangle{\pgfqpoint{1.150000in}{0.150000in}}{\pgfqpoint{5.700000in}{5.700000in}}%
\pgfusepath{clip}%
\pgfsetbuttcap%
\pgfsetroundjoin%
\definecolor{currentfill}{rgb}{0.280868,0.160771,0.472899}%
\pgfsetfillcolor{currentfill}%
\pgfsetfillopacity{0.700000}%
\pgfsetlinewidth{0.000000pt}%
\definecolor{currentstroke}{rgb}{0.000000,0.000000,0.000000}%
\pgfsetstrokecolor{currentstroke}%
\pgfsetdash{}{0pt}%
\pgfpathmoveto{\pgfqpoint{5.591365in}{2.448648in}}%
\pgfpathlineto{\pgfqpoint{5.605375in}{2.447628in}}%
\pgfpathlineto{\pgfqpoint{5.619393in}{2.446632in}}%
\pgfpathlineto{\pgfqpoint{5.633420in}{2.445661in}}%
\pgfpathlineto{\pgfqpoint{5.647456in}{2.444714in}}%
\pgfpathlineto{\pgfqpoint{5.640236in}{2.438833in}}%
\pgfpathlineto{\pgfqpoint{5.633009in}{2.432892in}}%
\pgfpathlineto{\pgfqpoint{5.625774in}{2.426888in}}%
\pgfpathlineto{\pgfqpoint{5.618531in}{2.420818in}}%
\pgfpathlineto{\pgfqpoint{5.604479in}{2.421698in}}%
\pgfpathlineto{\pgfqpoint{5.590435in}{2.422602in}}%
\pgfpathlineto{\pgfqpoint{5.576400in}{2.423530in}}%
\pgfpathlineto{\pgfqpoint{5.562373in}{2.424483in}}%
\pgfpathlineto{\pgfqpoint{5.569632in}{2.430615in}}%
\pgfpathlineto{\pgfqpoint{5.576884in}{2.436685in}}%
\pgfpathlineto{\pgfqpoint{5.584128in}{2.442695in}}%
\pgfpathlineto{\pgfqpoint{5.591365in}{2.448648in}}%
\pgfpathclose%
\pgfusepath{fill}%
\end{pgfscope}%
\begin{pgfscope}%
\pgfpathrectangle{\pgfqpoint{1.150000in}{0.150000in}}{\pgfqpoint{5.700000in}{5.700000in}}%
\pgfusepath{clip}%
\pgfsetbuttcap%
\pgfsetroundjoin%
\definecolor{currentfill}{rgb}{0.283072,0.130895,0.449241}%
\pgfsetfillcolor{currentfill}%
\pgfsetfillopacity{0.700000}%
\pgfsetlinewidth{0.000000pt}%
\definecolor{currentstroke}{rgb}{0.000000,0.000000,0.000000}%
\pgfsetstrokecolor{currentstroke}%
\pgfsetdash{}{0pt}%
\pgfpathmoveto{\pgfqpoint{2.750457in}{2.398473in}}%
\pgfpathlineto{\pgfqpoint{2.763841in}{2.390904in}}%
\pgfpathlineto{\pgfqpoint{2.777227in}{2.383380in}}%
\pgfpathlineto{\pgfqpoint{2.790616in}{2.375901in}}%
\pgfpathlineto{\pgfqpoint{2.804008in}{2.368465in}}%
\pgfpathlineto{\pgfqpoint{2.795516in}{2.367744in}}%
\pgfpathlineto{\pgfqpoint{2.787009in}{2.367258in}}%
\pgfpathlineto{\pgfqpoint{2.778487in}{2.367015in}}%
\pgfpathlineto{\pgfqpoint{2.769951in}{2.367020in}}%
\pgfpathlineto{\pgfqpoint{2.756530in}{2.374699in}}%
\pgfpathlineto{\pgfqpoint{2.743113in}{2.382423in}}%
\pgfpathlineto{\pgfqpoint{2.729699in}{2.390191in}}%
\pgfpathlineto{\pgfqpoint{2.716288in}{2.398004in}}%
\pgfpathlineto{\pgfqpoint{2.724853in}{2.397750in}}%
\pgfpathlineto{\pgfqpoint{2.733403in}{2.397748in}}%
\pgfpathlineto{\pgfqpoint{2.741937in}{2.397991in}}%
\pgfpathlineto{\pgfqpoint{2.750457in}{2.398473in}}%
\pgfpathclose%
\pgfusepath{fill}%
\end{pgfscope}%
\begin{pgfscope}%
\pgfpathrectangle{\pgfqpoint{1.150000in}{0.150000in}}{\pgfqpoint{5.700000in}{5.700000in}}%
\pgfusepath{clip}%
\pgfsetbuttcap%
\pgfsetroundjoin%
\definecolor{currentfill}{rgb}{0.282910,0.105393,0.426902}%
\pgfsetfillcolor{currentfill}%
\pgfsetfillopacity{0.700000}%
\pgfsetlinewidth{0.000000pt}%
\definecolor{currentstroke}{rgb}{0.000000,0.000000,0.000000}%
\pgfsetstrokecolor{currentstroke}%
\pgfsetdash{}{0pt}%
\pgfpathmoveto{\pgfqpoint{5.054209in}{2.349395in}}%
\pgfpathlineto{\pgfqpoint{5.068062in}{2.347881in}}%
\pgfpathlineto{\pgfqpoint{5.081923in}{2.346391in}}%
\pgfpathlineto{\pgfqpoint{5.095793in}{2.344927in}}%
\pgfpathlineto{\pgfqpoint{5.109670in}{2.343487in}}%
\pgfpathlineto{\pgfqpoint{5.102201in}{2.336010in}}%
\pgfpathlineto{\pgfqpoint{5.094726in}{2.328465in}}%
\pgfpathlineto{\pgfqpoint{5.087244in}{2.320852in}}%
\pgfpathlineto{\pgfqpoint{5.079756in}{2.313170in}}%
\pgfpathlineto{\pgfqpoint{5.065866in}{2.314610in}}%
\pgfpathlineto{\pgfqpoint{5.051984in}{2.316076in}}%
\pgfpathlineto{\pgfqpoint{5.038110in}{2.317566in}}%
\pgfpathlineto{\pgfqpoint{5.024244in}{2.319082in}}%
\pgfpathlineto{\pgfqpoint{5.031745in}{2.326758in}}%
\pgfpathlineto{\pgfqpoint{5.039240in}{2.334368in}}%
\pgfpathlineto{\pgfqpoint{5.046728in}{2.341914in}}%
\pgfpathlineto{\pgfqpoint{5.054209in}{2.349395in}}%
\pgfpathclose%
\pgfusepath{fill}%
\end{pgfscope}%
\begin{pgfscope}%
\pgfpathrectangle{\pgfqpoint{1.150000in}{0.150000in}}{\pgfqpoint{5.700000in}{5.700000in}}%
\pgfusepath{clip}%
\pgfsetbuttcap%
\pgfsetroundjoin%
\definecolor{currentfill}{rgb}{0.267004,0.004874,0.329415}%
\pgfsetfillcolor{currentfill}%
\pgfsetfillopacity{0.700000}%
\pgfsetlinewidth{0.000000pt}%
\definecolor{currentstroke}{rgb}{0.000000,0.000000,0.000000}%
\pgfsetstrokecolor{currentstroke}%
\pgfsetdash{}{0pt}%
\pgfpathmoveto{\pgfqpoint{3.839522in}{2.163707in}}%
\pgfpathlineto{\pgfqpoint{3.853060in}{2.159730in}}%
\pgfpathlineto{\pgfqpoint{3.866603in}{2.155784in}}%
\pgfpathlineto{\pgfqpoint{3.880153in}{2.151866in}}%
\pgfpathlineto{\pgfqpoint{3.893709in}{2.147977in}}%
\pgfpathlineto{\pgfqpoint{3.885785in}{2.140505in}}%
\pgfpathlineto{\pgfqpoint{3.877855in}{2.133080in}}%
\pgfpathlineto{\pgfqpoint{3.869918in}{2.125707in}}%
\pgfpathlineto{\pgfqpoint{3.861976in}{2.118388in}}%
\pgfpathlineto{\pgfqpoint{3.848406in}{2.122425in}}%
\pgfpathlineto{\pgfqpoint{3.834843in}{2.126490in}}%
\pgfpathlineto{\pgfqpoint{3.821285in}{2.130585in}}%
\pgfpathlineto{\pgfqpoint{3.807733in}{2.134709in}}%
\pgfpathlineto{\pgfqpoint{3.815690in}{2.141875in}}%
\pgfpathlineto{\pgfqpoint{3.823640in}{2.149099in}}%
\pgfpathlineto{\pgfqpoint{3.831584in}{2.156377in}}%
\pgfpathlineto{\pgfqpoint{3.839522in}{2.163707in}}%
\pgfpathclose%
\pgfusepath{fill}%
\end{pgfscope}%
\begin{pgfscope}%
\pgfpathrectangle{\pgfqpoint{1.150000in}{0.150000in}}{\pgfqpoint{5.700000in}{5.700000in}}%
\pgfusepath{clip}%
\pgfsetbuttcap%
\pgfsetroundjoin%
\definecolor{currentfill}{rgb}{0.278826,0.175490,0.483397}%
\pgfsetfillcolor{currentfill}%
\pgfsetfillopacity{0.700000}%
\pgfsetlinewidth{0.000000pt}%
\definecolor{currentstroke}{rgb}{0.000000,0.000000,0.000000}%
\pgfsetstrokecolor{currentstroke}%
\pgfsetdash{}{0pt}%
\pgfpathmoveto{\pgfqpoint{5.817264in}{2.481798in}}%
\pgfpathlineto{\pgfqpoint{5.831342in}{2.480895in}}%
\pgfpathlineto{\pgfqpoint{5.845428in}{2.480016in}}%
\pgfpathlineto{\pgfqpoint{5.859523in}{2.479161in}}%
\pgfpathlineto{\pgfqpoint{5.873627in}{2.478331in}}%
\pgfpathlineto{\pgfqpoint{5.866521in}{2.473136in}}%
\pgfpathlineto{\pgfqpoint{5.859407in}{2.467893in}}%
\pgfpathlineto{\pgfqpoint{5.852286in}{2.462598in}}%
\pgfpathlineto{\pgfqpoint{5.845157in}{2.457250in}}%
\pgfpathlineto{\pgfqpoint{5.831034in}{2.457986in}}%
\pgfpathlineto{\pgfqpoint{5.816920in}{2.458745in}}%
\pgfpathlineto{\pgfqpoint{5.802815in}{2.459530in}}%
\pgfpathlineto{\pgfqpoint{5.788719in}{2.460338in}}%
\pgfpathlineto{\pgfqpoint{5.795867in}{2.465776in}}%
\pgfpathlineto{\pgfqpoint{5.803006in}{2.471164in}}%
\pgfpathlineto{\pgfqpoint{5.810139in}{2.476503in}}%
\pgfpathlineto{\pgfqpoint{5.817264in}{2.481798in}}%
\pgfpathclose%
\pgfusepath{fill}%
\end{pgfscope}%
\begin{pgfscope}%
\pgfpathrectangle{\pgfqpoint{1.150000in}{0.150000in}}{\pgfqpoint{5.700000in}{5.700000in}}%
\pgfusepath{clip}%
\pgfsetbuttcap%
\pgfsetroundjoin%
\definecolor{currentfill}{rgb}{0.280267,0.073417,0.397163}%
\pgfsetfillcolor{currentfill}%
\pgfsetfillopacity{0.700000}%
\pgfsetlinewidth{0.000000pt}%
\definecolor{currentstroke}{rgb}{0.000000,0.000000,0.000000}%
\pgfsetstrokecolor{currentstroke}%
\pgfsetdash{}{0pt}%
\pgfpathmoveto{\pgfqpoint{4.742818in}{2.283676in}}%
\pgfpathlineto{\pgfqpoint{4.756584in}{2.281711in}}%
\pgfpathlineto{\pgfqpoint{4.770358in}{2.279772in}}%
\pgfpathlineto{\pgfqpoint{4.784139in}{2.277858in}}%
\pgfpathlineto{\pgfqpoint{4.797928in}{2.275970in}}%
\pgfpathlineto{\pgfqpoint{4.790333in}{2.267810in}}%
\pgfpathlineto{\pgfqpoint{4.782732in}{2.259595in}}%
\pgfpathlineto{\pgfqpoint{4.775124in}{2.251326in}}%
\pgfpathlineto{\pgfqpoint{4.767511in}{2.243003in}}%
\pgfpathlineto{\pgfqpoint{4.753710in}{2.244933in}}%
\pgfpathlineto{\pgfqpoint{4.739917in}{2.246888in}}%
\pgfpathlineto{\pgfqpoint{4.726132in}{2.248869in}}%
\pgfpathlineto{\pgfqpoint{4.712354in}{2.250875in}}%
\pgfpathlineto{\pgfqpoint{4.719980in}{2.259152in}}%
\pgfpathlineto{\pgfqpoint{4.727599in}{2.267378in}}%
\pgfpathlineto{\pgfqpoint{4.735211in}{2.275553in}}%
\pgfpathlineto{\pgfqpoint{4.742818in}{2.283676in}}%
\pgfpathclose%
\pgfusepath{fill}%
\end{pgfscope}%
\begin{pgfscope}%
\pgfpathrectangle{\pgfqpoint{1.150000in}{0.150000in}}{\pgfqpoint{5.700000in}{5.700000in}}%
\pgfusepath{clip}%
\pgfsetbuttcap%
\pgfsetroundjoin%
\definecolor{currentfill}{rgb}{0.277941,0.056324,0.381191}%
\pgfsetfillcolor{currentfill}%
\pgfsetfillopacity{0.700000}%
\pgfsetlinewidth{0.000000pt}%
\definecolor{currentstroke}{rgb}{0.000000,0.000000,0.000000}%
\pgfsetstrokecolor{currentstroke}%
\pgfsetdash{}{0pt}%
\pgfpathmoveto{\pgfqpoint{3.139056in}{2.250857in}}%
\pgfpathlineto{\pgfqpoint{3.152477in}{2.244707in}}%
\pgfpathlineto{\pgfqpoint{3.165902in}{2.238593in}}%
\pgfpathlineto{\pgfqpoint{3.179331in}{2.232516in}}%
\pgfpathlineto{\pgfqpoint{3.192764in}{2.226475in}}%
\pgfpathlineto{\pgfqpoint{3.184513in}{2.222820in}}%
\pgfpathlineto{\pgfqpoint{3.176252in}{2.219333in}}%
\pgfpathlineto{\pgfqpoint{3.167979in}{2.216021in}}%
\pgfpathlineto{\pgfqpoint{3.159697in}{2.212887in}}%
\pgfpathlineto{\pgfqpoint{3.146241in}{2.219143in}}%
\pgfpathlineto{\pgfqpoint{3.132790in}{2.225435in}}%
\pgfpathlineto{\pgfqpoint{3.119343in}{2.231764in}}%
\pgfpathlineto{\pgfqpoint{3.105900in}{2.238129in}}%
\pgfpathlineto{\pgfqpoint{3.114206in}{2.241042in}}%
\pgfpathlineto{\pgfqpoint{3.122500in}{2.244138in}}%
\pgfpathlineto{\pgfqpoint{3.130783in}{2.247411in}}%
\pgfpathlineto{\pgfqpoint{3.139056in}{2.250857in}}%
\pgfpathclose%
\pgfusepath{fill}%
\end{pgfscope}%
\begin{pgfscope}%
\pgfpathrectangle{\pgfqpoint{1.150000in}{0.150000in}}{\pgfqpoint{5.700000in}{5.700000in}}%
\pgfusepath{clip}%
\pgfsetbuttcap%
\pgfsetroundjoin%
\definecolor{currentfill}{rgb}{0.283072,0.130895,0.449241}%
\pgfsetfillcolor{currentfill}%
\pgfsetfillopacity{0.700000}%
\pgfsetlinewidth{0.000000pt}%
\definecolor{currentstroke}{rgb}{0.000000,0.000000,0.000000}%
\pgfsetstrokecolor{currentstroke}%
\pgfsetdash{}{0pt}%
\pgfpathmoveto{\pgfqpoint{5.280260in}{2.390042in}}%
\pgfpathlineto{\pgfqpoint{5.294183in}{2.388789in}}%
\pgfpathlineto{\pgfqpoint{5.308114in}{2.387560in}}%
\pgfpathlineto{\pgfqpoint{5.322054in}{2.386355in}}%
\pgfpathlineto{\pgfqpoint{5.336001in}{2.385176in}}%
\pgfpathlineto{\pgfqpoint{5.328630in}{2.378302in}}%
\pgfpathlineto{\pgfqpoint{5.321252in}{2.371359in}}%
\pgfpathlineto{\pgfqpoint{5.313866in}{2.364344in}}%
\pgfpathlineto{\pgfqpoint{5.306473in}{2.357257in}}%
\pgfpathlineto{\pgfqpoint{5.292511in}{2.358410in}}%
\pgfpathlineto{\pgfqpoint{5.278558in}{2.359588in}}%
\pgfpathlineto{\pgfqpoint{5.264613in}{2.360790in}}%
\pgfpathlineto{\pgfqpoint{5.250676in}{2.362018in}}%
\pgfpathlineto{\pgfqpoint{5.258083in}{2.369126in}}%
\pgfpathlineto{\pgfqpoint{5.265482in}{2.376166in}}%
\pgfpathlineto{\pgfqpoint{5.272875in}{2.383137in}}%
\pgfpathlineto{\pgfqpoint{5.280260in}{2.390042in}}%
\pgfpathclose%
\pgfusepath{fill}%
\end{pgfscope}%
\begin{pgfscope}%
\pgfpathrectangle{\pgfqpoint{1.150000in}{0.150000in}}{\pgfqpoint{5.700000in}{5.700000in}}%
\pgfusepath{clip}%
\pgfsetbuttcap%
\pgfsetroundjoin%
\definecolor{currentfill}{rgb}{0.269944,0.014625,0.341379}%
\pgfsetfillcolor{currentfill}%
\pgfsetfillopacity{0.700000}%
\pgfsetlinewidth{0.000000pt}%
\definecolor{currentstroke}{rgb}{0.000000,0.000000,0.000000}%
\pgfsetstrokecolor{currentstroke}%
\pgfsetdash{}{0pt}%
\pgfpathmoveto{\pgfqpoint{4.205446in}{2.187705in}}%
\pgfpathlineto{\pgfqpoint{4.219071in}{2.184658in}}%
\pgfpathlineto{\pgfqpoint{4.232703in}{2.181639in}}%
\pgfpathlineto{\pgfqpoint{4.246341in}{2.178646in}}%
\pgfpathlineto{\pgfqpoint{4.259986in}{2.175681in}}%
\pgfpathlineto{\pgfqpoint{4.252195in}{2.167356in}}%
\pgfpathlineto{\pgfqpoint{4.244398in}{2.159027in}}%
\pgfpathlineto{\pgfqpoint{4.236596in}{2.150698in}}%
\pgfpathlineto{\pgfqpoint{4.228788in}{2.142371in}}%
\pgfpathlineto{\pgfqpoint{4.215131in}{2.145444in}}%
\pgfpathlineto{\pgfqpoint{4.201480in}{2.148545in}}%
\pgfpathlineto{\pgfqpoint{4.187837in}{2.151673in}}%
\pgfpathlineto{\pgfqpoint{4.174200in}{2.154828in}}%
\pgfpathlineto{\pgfqpoint{4.182020in}{2.163042in}}%
\pgfpathlineto{\pgfqpoint{4.189834in}{2.171262in}}%
\pgfpathlineto{\pgfqpoint{4.197643in}{2.179483in}}%
\pgfpathlineto{\pgfqpoint{4.205446in}{2.187705in}}%
\pgfpathclose%
\pgfusepath{fill}%
\end{pgfscope}%
\begin{pgfscope}%
\pgfpathrectangle{\pgfqpoint{1.150000in}{0.150000in}}{\pgfqpoint{5.700000in}{5.700000in}}%
\pgfusepath{clip}%
\pgfsetbuttcap%
\pgfsetroundjoin%
\definecolor{currentfill}{rgb}{0.274952,0.037752,0.364543}%
\pgfsetfillcolor{currentfill}%
\pgfsetfillopacity{0.700000}%
\pgfsetlinewidth{0.000000pt}%
\definecolor{currentstroke}{rgb}{0.000000,0.000000,0.000000}%
\pgfsetstrokecolor{currentstroke}%
\pgfsetdash{}{0pt}%
\pgfpathmoveto{\pgfqpoint{4.431335in}{2.220571in}}%
\pgfpathlineto{\pgfqpoint{4.445018in}{2.218024in}}%
\pgfpathlineto{\pgfqpoint{4.458709in}{2.215503in}}%
\pgfpathlineto{\pgfqpoint{4.472408in}{2.213009in}}%
\pgfpathlineto{\pgfqpoint{4.486113in}{2.210541in}}%
\pgfpathlineto{\pgfqpoint{4.478401in}{2.202085in}}%
\pgfpathlineto{\pgfqpoint{4.470684in}{2.193602in}}%
\pgfpathlineto{\pgfqpoint{4.462961in}{2.185090in}}%
\pgfpathlineto{\pgfqpoint{4.455232in}{2.176554in}}%
\pgfpathlineto{\pgfqpoint{4.441515in}{2.179103in}}%
\pgfpathlineto{\pgfqpoint{4.427806in}{2.181680in}}%
\pgfpathlineto{\pgfqpoint{4.414104in}{2.184282in}}%
\pgfpathlineto{\pgfqpoint{4.400408in}{2.186911in}}%
\pgfpathlineto{\pgfqpoint{4.408148in}{2.195361in}}%
\pgfpathlineto{\pgfqpoint{4.415883in}{2.203789in}}%
\pgfpathlineto{\pgfqpoint{4.423611in}{2.212193in}}%
\pgfpathlineto{\pgfqpoint{4.431335in}{2.220571in}}%
\pgfpathclose%
\pgfusepath{fill}%
\end{pgfscope}%
\begin{pgfscope}%
\pgfpathrectangle{\pgfqpoint{1.150000in}{0.150000in}}{\pgfqpoint{5.700000in}{5.700000in}}%
\pgfusepath{clip}%
\pgfsetbuttcap%
\pgfsetroundjoin%
\definecolor{currentfill}{rgb}{0.267004,0.004874,0.329415}%
\pgfsetfillcolor{currentfill}%
\pgfsetfillopacity{0.700000}%
\pgfsetlinewidth{0.000000pt}%
\definecolor{currentstroke}{rgb}{0.000000,0.000000,0.000000}%
\pgfsetstrokecolor{currentstroke}%
\pgfsetdash{}{0pt}%
\pgfpathmoveto{\pgfqpoint{3.979573in}{2.163550in}}%
\pgfpathlineto{\pgfqpoint{3.993146in}{2.159940in}}%
\pgfpathlineto{\pgfqpoint{4.006725in}{2.156357in}}%
\pgfpathlineto{\pgfqpoint{4.020310in}{2.152803in}}%
\pgfpathlineto{\pgfqpoint{4.033902in}{2.149277in}}%
\pgfpathlineto{\pgfqpoint{4.026028in}{2.141387in}}%
\pgfpathlineto{\pgfqpoint{4.018149in}{2.133525in}}%
\pgfpathlineto{\pgfqpoint{4.010264in}{2.125694in}}%
\pgfpathlineto{\pgfqpoint{4.002373in}{2.117898in}}%
\pgfpathlineto{\pgfqpoint{3.988768in}{2.121559in}}%
\pgfpathlineto{\pgfqpoint{3.975170in}{2.125248in}}%
\pgfpathlineto{\pgfqpoint{3.961578in}{2.128965in}}%
\pgfpathlineto{\pgfqpoint{3.947992in}{2.132710in}}%
\pgfpathlineto{\pgfqpoint{3.955896in}{2.140367in}}%
\pgfpathlineto{\pgfqpoint{3.963794in}{2.148061in}}%
\pgfpathlineto{\pgfqpoint{3.971687in}{2.155790in}}%
\pgfpathlineto{\pgfqpoint{3.979573in}{2.163550in}}%
\pgfpathclose%
\pgfusepath{fill}%
\end{pgfscope}%
\begin{pgfscope}%
\pgfpathrectangle{\pgfqpoint{1.150000in}{0.150000in}}{\pgfqpoint{5.700000in}{5.700000in}}%
\pgfusepath{clip}%
\pgfsetbuttcap%
\pgfsetroundjoin%
\definecolor{currentfill}{rgb}{0.277134,0.185228,0.489898}%
\pgfsetfillcolor{currentfill}%
\pgfsetfillopacity{0.700000}%
\pgfsetlinewidth{0.000000pt}%
\definecolor{currentstroke}{rgb}{0.000000,0.000000,0.000000}%
\pgfsetstrokecolor{currentstroke}%
\pgfsetdash{}{0pt}%
\pgfpathmoveto{\pgfqpoint{2.555554in}{2.495421in}}%
\pgfpathlineto{\pgfqpoint{2.568935in}{2.487035in}}%
\pgfpathlineto{\pgfqpoint{2.582319in}{2.478699in}}%
\pgfpathlineto{\pgfqpoint{2.595705in}{2.470413in}}%
\pgfpathlineto{\pgfqpoint{2.609093in}{2.462176in}}%
\pgfpathlineto{\pgfqpoint{2.600452in}{2.463196in}}%
\pgfpathlineto{\pgfqpoint{2.591795in}{2.464488in}}%
\pgfpathlineto{\pgfqpoint{2.583121in}{2.466059in}}%
\pgfpathlineto{\pgfqpoint{2.574429in}{2.467916in}}%
\pgfpathlineto{\pgfqpoint{2.561009in}{2.476413in}}%
\pgfpathlineto{\pgfqpoint{2.547592in}{2.484958in}}%
\pgfpathlineto{\pgfqpoint{2.534177in}{2.493554in}}%
\pgfpathlineto{\pgfqpoint{2.520764in}{2.502201in}}%
\pgfpathlineto{\pgfqpoint{2.529488in}{2.500079in}}%
\pgfpathlineto{\pgfqpoint{2.538195in}{2.498246in}}%
\pgfpathlineto{\pgfqpoint{2.546883in}{2.496696in}}%
\pgfpathlineto{\pgfqpoint{2.555554in}{2.495421in}}%
\pgfpathclose%
\pgfusepath{fill}%
\end{pgfscope}%
\begin{pgfscope}%
\pgfpathrectangle{\pgfqpoint{1.150000in}{0.150000in}}{\pgfqpoint{5.700000in}{5.700000in}}%
\pgfusepath{clip}%
\pgfsetbuttcap%
\pgfsetroundjoin%
\definecolor{currentfill}{rgb}{0.282656,0.100196,0.422160}%
\pgfsetfillcolor{currentfill}%
\pgfsetfillopacity{0.700000}%
\pgfsetlinewidth{0.000000pt}%
\definecolor{currentstroke}{rgb}{0.000000,0.000000,0.000000}%
\pgfsetstrokecolor{currentstroke}%
\pgfsetdash{}{0pt}%
\pgfpathmoveto{\pgfqpoint{4.968860in}{2.325397in}}%
\pgfpathlineto{\pgfqpoint{4.982694in}{2.323780in}}%
\pgfpathlineto{\pgfqpoint{4.996536in}{2.322189in}}%
\pgfpathlineto{\pgfqpoint{5.010386in}{2.320623in}}%
\pgfpathlineto{\pgfqpoint{5.024244in}{2.319082in}}%
\pgfpathlineto{\pgfqpoint{5.016736in}{2.311341in}}%
\pgfpathlineto{\pgfqpoint{5.009222in}{2.303533in}}%
\pgfpathlineto{\pgfqpoint{5.001701in}{2.295660in}}%
\pgfpathlineto{\pgfqpoint{4.994173in}{2.287720in}}%
\pgfpathlineto{\pgfqpoint{4.980303in}{2.289275in}}%
\pgfpathlineto{\pgfqpoint{4.966440in}{2.290856in}}%
\pgfpathlineto{\pgfqpoint{4.952586in}{2.292461in}}%
\pgfpathlineto{\pgfqpoint{4.938739in}{2.294092in}}%
\pgfpathlineto{\pgfqpoint{4.946279in}{2.302013in}}%
\pgfpathlineto{\pgfqpoint{4.953813in}{2.309870in}}%
\pgfpathlineto{\pgfqpoint{4.961339in}{2.317665in}}%
\pgfpathlineto{\pgfqpoint{4.968860in}{2.325397in}}%
\pgfpathclose%
\pgfusepath{fill}%
\end{pgfscope}%
\begin{pgfscope}%
\pgfpathrectangle{\pgfqpoint{1.150000in}{0.150000in}}{\pgfqpoint{5.700000in}{5.700000in}}%
\pgfusepath{clip}%
\pgfsetbuttcap%
\pgfsetroundjoin%
\definecolor{currentfill}{rgb}{0.281887,0.150881,0.465405}%
\pgfsetfillcolor{currentfill}%
\pgfsetfillopacity{0.700000}%
\pgfsetlinewidth{0.000000pt}%
\definecolor{currentstroke}{rgb}{0.000000,0.000000,0.000000}%
\pgfsetstrokecolor{currentstroke}%
\pgfsetdash{}{0pt}%
\pgfpathmoveto{\pgfqpoint{5.506352in}{2.428540in}}%
\pgfpathlineto{\pgfqpoint{5.520344in}{2.427489in}}%
\pgfpathlineto{\pgfqpoint{5.534345in}{2.426462in}}%
\pgfpathlineto{\pgfqpoint{5.548355in}{2.425460in}}%
\pgfpathlineto{\pgfqpoint{5.562373in}{2.424483in}}%
\pgfpathlineto{\pgfqpoint{5.555106in}{2.418287in}}%
\pgfpathlineto{\pgfqpoint{5.547832in}{2.412025in}}%
\pgfpathlineto{\pgfqpoint{5.540551in}{2.405694in}}%
\pgfpathlineto{\pgfqpoint{5.533262in}{2.399293in}}%
\pgfpathlineto{\pgfqpoint{5.519228in}{2.400216in}}%
\pgfpathlineto{\pgfqpoint{5.505202in}{2.401165in}}%
\pgfpathlineto{\pgfqpoint{5.491185in}{2.402137in}}%
\pgfpathlineto{\pgfqpoint{5.477177in}{2.403135in}}%
\pgfpathlineto{\pgfqpoint{5.484482in}{2.409584in}}%
\pgfpathlineto{\pgfqpoint{5.491779in}{2.415967in}}%
\pgfpathlineto{\pgfqpoint{5.499069in}{2.422285in}}%
\pgfpathlineto{\pgfqpoint{5.506352in}{2.428540in}}%
\pgfpathclose%
\pgfusepath{fill}%
\end{pgfscope}%
\begin{pgfscope}%
\pgfpathrectangle{\pgfqpoint{1.150000in}{0.150000in}}{\pgfqpoint{5.700000in}{5.700000in}}%
\pgfusepath{clip}%
\pgfsetbuttcap%
\pgfsetroundjoin%
\definecolor{currentfill}{rgb}{0.269944,0.014625,0.341379}%
\pgfsetfillcolor{currentfill}%
\pgfsetfillopacity{0.700000}%
\pgfsetlinewidth{0.000000pt}%
\definecolor{currentstroke}{rgb}{0.000000,0.000000,0.000000}%
\pgfsetstrokecolor{currentstroke}%
\pgfsetdash{}{0pt}%
\pgfpathmoveto{\pgfqpoint{3.473350in}{2.175466in}}%
\pgfpathlineto{\pgfqpoint{3.486824in}{2.170393in}}%
\pgfpathlineto{\pgfqpoint{3.500304in}{2.165352in}}%
\pgfpathlineto{\pgfqpoint{3.513789in}{2.160343in}}%
\pgfpathlineto{\pgfqpoint{3.527278in}{2.155366in}}%
\pgfpathlineto{\pgfqpoint{3.519195in}{2.149630in}}%
\pgfpathlineto{\pgfqpoint{3.511104in}{2.144006in}}%
\pgfpathlineto{\pgfqpoint{3.503005in}{2.138499in}}%
\pgfpathlineto{\pgfqpoint{3.494897in}{2.133114in}}%
\pgfpathlineto{\pgfqpoint{3.481390in}{2.138279in}}%
\pgfpathlineto{\pgfqpoint{3.467887in}{2.143475in}}%
\pgfpathlineto{\pgfqpoint{3.454390in}{2.148704in}}%
\pgfpathlineto{\pgfqpoint{3.440897in}{2.153965in}}%
\pgfpathlineto{\pgfqpoint{3.449023in}{2.159158in}}%
\pgfpathlineto{\pgfqpoint{3.457140in}{2.164475in}}%
\pgfpathlineto{\pgfqpoint{3.465249in}{2.169913in}}%
\pgfpathlineto{\pgfqpoint{3.473350in}{2.175466in}}%
\pgfpathclose%
\pgfusepath{fill}%
\end{pgfscope}%
\begin{pgfscope}%
\pgfpathrectangle{\pgfqpoint{1.150000in}{0.150000in}}{\pgfqpoint{5.700000in}{5.700000in}}%
\pgfusepath{clip}%
\pgfsetbuttcap%
\pgfsetroundjoin%
\definecolor{currentfill}{rgb}{0.278791,0.062145,0.386592}%
\pgfsetfillcolor{currentfill}%
\pgfsetfillopacity{0.700000}%
\pgfsetlinewidth{0.000000pt}%
\definecolor{currentstroke}{rgb}{0.000000,0.000000,0.000000}%
\pgfsetstrokecolor{currentstroke}%
\pgfsetdash{}{0pt}%
\pgfpathmoveto{\pgfqpoint{4.657318in}{2.259159in}}%
\pgfpathlineto{\pgfqpoint{4.671066in}{2.257050in}}%
\pgfpathlineto{\pgfqpoint{4.684822in}{2.254966in}}%
\pgfpathlineto{\pgfqpoint{4.698584in}{2.252908in}}%
\pgfpathlineto{\pgfqpoint{4.712354in}{2.250875in}}%
\pgfpathlineto{\pgfqpoint{4.704723in}{2.242550in}}%
\pgfpathlineto{\pgfqpoint{4.697086in}{2.234175in}}%
\pgfpathlineto{\pgfqpoint{4.689443in}{2.225753in}}%
\pgfpathlineto{\pgfqpoint{4.681794in}{2.217283in}}%
\pgfpathlineto{\pgfqpoint{4.668013in}{2.219371in}}%
\pgfpathlineto{\pgfqpoint{4.654239in}{2.221484in}}%
\pgfpathlineto{\pgfqpoint{4.640472in}{2.223623in}}%
\pgfpathlineto{\pgfqpoint{4.626713in}{2.225787in}}%
\pgfpathlineto{\pgfqpoint{4.634373in}{2.234197in}}%
\pgfpathlineto{\pgfqpoint{4.642027in}{2.242562in}}%
\pgfpathlineto{\pgfqpoint{4.649676in}{2.250884in}}%
\pgfpathlineto{\pgfqpoint{4.657318in}{2.259159in}}%
\pgfpathclose%
\pgfusepath{fill}%
\end{pgfscope}%
\begin{pgfscope}%
\pgfpathrectangle{\pgfqpoint{1.150000in}{0.150000in}}{\pgfqpoint{5.700000in}{5.700000in}}%
\pgfusepath{clip}%
\pgfsetbuttcap%
\pgfsetroundjoin%
\definecolor{currentfill}{rgb}{0.267004,0.004874,0.329415}%
\pgfsetfillcolor{currentfill}%
\pgfsetfillopacity{0.700000}%
\pgfsetlinewidth{0.000000pt}%
\definecolor{currentstroke}{rgb}{0.000000,0.000000,0.000000}%
\pgfsetstrokecolor{currentstroke}%
\pgfsetdash{}{0pt}%
\pgfpathmoveto{\pgfqpoint{3.613480in}{2.160448in}}%
\pgfpathlineto{\pgfqpoint{3.626980in}{2.155802in}}%
\pgfpathlineto{\pgfqpoint{3.640485in}{2.151187in}}%
\pgfpathlineto{\pgfqpoint{3.653996in}{2.146603in}}%
\pgfpathlineto{\pgfqpoint{3.667512in}{2.142049in}}%
\pgfpathlineto{\pgfqpoint{3.659491in}{2.135570in}}%
\pgfpathlineto{\pgfqpoint{3.651463in}{2.129178in}}%
\pgfpathlineto{\pgfqpoint{3.643428in}{2.122879in}}%
\pgfpathlineto{\pgfqpoint{3.635386in}{2.116676in}}%
\pgfpathlineto{\pgfqpoint{3.621854in}{2.121404in}}%
\pgfpathlineto{\pgfqpoint{3.608327in}{2.126163in}}%
\pgfpathlineto{\pgfqpoint{3.594806in}{2.130953in}}%
\pgfpathlineto{\pgfqpoint{3.581290in}{2.135773in}}%
\pgfpathlineto{\pgfqpoint{3.589349in}{2.141796in}}%
\pgfpathlineto{\pgfqpoint{3.597400in}{2.147920in}}%
\pgfpathlineto{\pgfqpoint{3.605444in}{2.154138in}}%
\pgfpathlineto{\pgfqpoint{3.613480in}{2.160448in}}%
\pgfpathclose%
\pgfusepath{fill}%
\end{pgfscope}%
\begin{pgfscope}%
\pgfpathrectangle{\pgfqpoint{1.150000in}{0.150000in}}{\pgfqpoint{5.700000in}{5.700000in}}%
\pgfusepath{clip}%
\pgfsetbuttcap%
\pgfsetroundjoin%
\definecolor{currentfill}{rgb}{0.279574,0.170599,0.479997}%
\pgfsetfillcolor{currentfill}%
\pgfsetfillopacity{0.700000}%
\pgfsetlinewidth{0.000000pt}%
\definecolor{currentstroke}{rgb}{0.000000,0.000000,0.000000}%
\pgfsetstrokecolor{currentstroke}%
\pgfsetdash{}{0pt}%
\pgfpathmoveto{\pgfqpoint{5.732422in}{2.463815in}}%
\pgfpathlineto{\pgfqpoint{5.746483in}{2.462909in}}%
\pgfpathlineto{\pgfqpoint{5.760553in}{2.462028in}}%
\pgfpathlineto{\pgfqpoint{5.774632in}{2.461171in}}%
\pgfpathlineto{\pgfqpoint{5.788719in}{2.460338in}}%
\pgfpathlineto{\pgfqpoint{5.781564in}{2.454846in}}%
\pgfpathlineto{\pgfqpoint{5.774402in}{2.449297in}}%
\pgfpathlineto{\pgfqpoint{5.767232in}{2.443689in}}%
\pgfpathlineto{\pgfqpoint{5.760054in}{2.438019in}}%
\pgfpathlineto{\pgfqpoint{5.745949in}{2.438770in}}%
\pgfpathlineto{\pgfqpoint{5.731852in}{2.439546in}}%
\pgfpathlineto{\pgfqpoint{5.717765in}{2.440346in}}%
\pgfpathlineto{\pgfqpoint{5.703686in}{2.441171in}}%
\pgfpathlineto{\pgfqpoint{5.710881in}{2.446918in}}%
\pgfpathlineto{\pgfqpoint{5.718069in}{2.452605in}}%
\pgfpathlineto{\pgfqpoint{5.725249in}{2.458237in}}%
\pgfpathlineto{\pgfqpoint{5.732422in}{2.463815in}}%
\pgfpathclose%
\pgfusepath{fill}%
\end{pgfscope}%
\begin{pgfscope}%
\pgfpathrectangle{\pgfqpoint{1.150000in}{0.150000in}}{\pgfqpoint{5.700000in}{5.700000in}}%
\pgfusepath{clip}%
\pgfsetbuttcap%
\pgfsetroundjoin%
\definecolor{currentfill}{rgb}{0.281446,0.084320,0.407414}%
\pgfsetfillcolor{currentfill}%
\pgfsetfillopacity{0.700000}%
\pgfsetlinewidth{0.000000pt}%
\definecolor{currentstroke}{rgb}{0.000000,0.000000,0.000000}%
\pgfsetstrokecolor{currentstroke}%
\pgfsetdash{}{0pt}%
\pgfpathmoveto{\pgfqpoint{2.998500in}{2.290404in}}%
\pgfpathlineto{\pgfqpoint{3.011911in}{2.283735in}}%
\pgfpathlineto{\pgfqpoint{3.025327in}{2.277105in}}%
\pgfpathlineto{\pgfqpoint{3.038746in}{2.270515in}}%
\pgfpathlineto{\pgfqpoint{3.052169in}{2.263962in}}%
\pgfpathlineto{\pgfqpoint{3.043829in}{2.261461in}}%
\pgfpathlineto{\pgfqpoint{3.035477in}{2.259159in}}%
\pgfpathlineto{\pgfqpoint{3.027113in}{2.257060in}}%
\pgfpathlineto{\pgfqpoint{3.018737in}{2.255170in}}%
\pgfpathlineto{\pgfqpoint{3.005290in}{2.261952in}}%
\pgfpathlineto{\pgfqpoint{2.991846in}{2.268772in}}%
\pgfpathlineto{\pgfqpoint{2.978407in}{2.275631in}}%
\pgfpathlineto{\pgfqpoint{2.964971in}{2.282528in}}%
\pgfpathlineto{\pgfqpoint{2.973372in}{2.284184in}}%
\pgfpathlineto{\pgfqpoint{2.981760in}{2.286052in}}%
\pgfpathlineto{\pgfqpoint{2.990136in}{2.288127in}}%
\pgfpathlineto{\pgfqpoint{2.998500in}{2.290404in}}%
\pgfpathclose%
\pgfusepath{fill}%
\end{pgfscope}%
\begin{pgfscope}%
\pgfpathrectangle{\pgfqpoint{1.150000in}{0.150000in}}{\pgfqpoint{5.700000in}{5.700000in}}%
\pgfusepath{clip}%
\pgfsetbuttcap%
\pgfsetroundjoin%
\definecolor{currentfill}{rgb}{0.283187,0.125848,0.444960}%
\pgfsetfillcolor{currentfill}%
\pgfsetfillopacity{0.700000}%
\pgfsetlinewidth{0.000000pt}%
\definecolor{currentstroke}{rgb}{0.000000,0.000000,0.000000}%
\pgfsetstrokecolor{currentstroke}%
\pgfsetdash{}{0pt}%
\pgfpathmoveto{\pgfqpoint{2.804008in}{2.368465in}}%
\pgfpathlineto{\pgfqpoint{2.817404in}{2.361073in}}%
\pgfpathlineto{\pgfqpoint{2.830802in}{2.353724in}}%
\pgfpathlineto{\pgfqpoint{2.844204in}{2.346418in}}%
\pgfpathlineto{\pgfqpoint{2.857609in}{2.339154in}}%
\pgfpathlineto{\pgfqpoint{2.849143in}{2.338195in}}%
\pgfpathlineto{\pgfqpoint{2.840663in}{2.337467in}}%
\pgfpathlineto{\pgfqpoint{2.832169in}{2.336978in}}%
\pgfpathlineto{\pgfqpoint{2.823661in}{2.336733in}}%
\pgfpathlineto{\pgfqpoint{2.810229in}{2.344241in}}%
\pgfpathlineto{\pgfqpoint{2.796799in}{2.351791in}}%
\pgfpathlineto{\pgfqpoint{2.783373in}{2.359383in}}%
\pgfpathlineto{\pgfqpoint{2.769951in}{2.367020in}}%
\pgfpathlineto{\pgfqpoint{2.778487in}{2.367015in}}%
\pgfpathlineto{\pgfqpoint{2.787009in}{2.367258in}}%
\pgfpathlineto{\pgfqpoint{2.795516in}{2.367744in}}%
\pgfpathlineto{\pgfqpoint{2.804008in}{2.368465in}}%
\pgfpathclose%
\pgfusepath{fill}%
\end{pgfscope}%
\begin{pgfscope}%
\pgfpathrectangle{\pgfqpoint{1.150000in}{0.150000in}}{\pgfqpoint{5.700000in}{5.700000in}}%
\pgfusepath{clip}%
\pgfsetbuttcap%
\pgfsetroundjoin%
\definecolor{currentfill}{rgb}{0.272594,0.025563,0.353093}%
\pgfsetfillcolor{currentfill}%
\pgfsetfillopacity{0.700000}%
\pgfsetlinewidth{0.000000pt}%
\definecolor{currentstroke}{rgb}{0.000000,0.000000,0.000000}%
\pgfsetstrokecolor{currentstroke}%
\pgfsetdash{}{0pt}%
\pgfpathmoveto{\pgfqpoint{3.333133in}{2.197234in}}%
\pgfpathlineto{\pgfqpoint{3.346587in}{2.191709in}}%
\pgfpathlineto{\pgfqpoint{3.360045in}{2.186217in}}%
\pgfpathlineto{\pgfqpoint{3.373509in}{2.180759in}}%
\pgfpathlineto{\pgfqpoint{3.386977in}{2.175334in}}%
\pgfpathlineto{\pgfqpoint{3.378824in}{2.170467in}}%
\pgfpathlineto{\pgfqpoint{3.370662in}{2.165738in}}%
\pgfpathlineto{\pgfqpoint{3.362491in}{2.161152in}}%
\pgfpathlineto{\pgfqpoint{3.354312in}{2.156714in}}%
\pgfpathlineto{\pgfqpoint{3.340824in}{2.162340in}}%
\pgfpathlineto{\pgfqpoint{3.327341in}{2.167999in}}%
\pgfpathlineto{\pgfqpoint{3.313863in}{2.173692in}}%
\pgfpathlineto{\pgfqpoint{3.300390in}{2.179419in}}%
\pgfpathlineto{\pgfqpoint{3.308590in}{2.183651in}}%
\pgfpathlineto{\pgfqpoint{3.316780in}{2.188034in}}%
\pgfpathlineto{\pgfqpoint{3.324961in}{2.192563in}}%
\pgfpathlineto{\pgfqpoint{3.333133in}{2.197234in}}%
\pgfpathclose%
\pgfusepath{fill}%
\end{pgfscope}%
\begin{pgfscope}%
\pgfpathrectangle{\pgfqpoint{1.150000in}{0.150000in}}{\pgfqpoint{5.700000in}{5.700000in}}%
\pgfusepath{clip}%
\pgfsetbuttcap%
\pgfsetroundjoin%
\definecolor{currentfill}{rgb}{0.283229,0.120777,0.440584}%
\pgfsetfillcolor{currentfill}%
\pgfsetfillopacity{0.700000}%
\pgfsetlinewidth{0.000000pt}%
\definecolor{currentstroke}{rgb}{0.000000,0.000000,0.000000}%
\pgfsetstrokecolor{currentstroke}%
\pgfsetdash{}{0pt}%
\pgfpathmoveto{\pgfqpoint{5.195010in}{2.367177in}}%
\pgfpathlineto{\pgfqpoint{5.208914in}{2.365850in}}%
\pgfpathlineto{\pgfqpoint{5.222827in}{2.364548in}}%
\pgfpathlineto{\pgfqpoint{5.236747in}{2.363270in}}%
\pgfpathlineto{\pgfqpoint{5.250676in}{2.362018in}}%
\pgfpathlineto{\pgfqpoint{5.243262in}{2.354839in}}%
\pgfpathlineto{\pgfqpoint{5.235841in}{2.347589in}}%
\pgfpathlineto{\pgfqpoint{5.228413in}{2.340267in}}%
\pgfpathlineto{\pgfqpoint{5.220978in}{2.332872in}}%
\pgfpathlineto{\pgfqpoint{5.207036in}{2.334112in}}%
\pgfpathlineto{\pgfqpoint{5.193102in}{2.335376in}}%
\pgfpathlineto{\pgfqpoint{5.179176in}{2.336666in}}%
\pgfpathlineto{\pgfqpoint{5.165259in}{2.337980in}}%
\pgfpathlineto{\pgfqpoint{5.172707in}{2.345383in}}%
\pgfpathlineto{\pgfqpoint{5.180148in}{2.352716in}}%
\pgfpathlineto{\pgfqpoint{5.187583in}{2.359980in}}%
\pgfpathlineto{\pgfqpoint{5.195010in}{2.367177in}}%
\pgfpathclose%
\pgfusepath{fill}%
\end{pgfscope}%
\begin{pgfscope}%
\pgfpathrectangle{\pgfqpoint{1.150000in}{0.150000in}}{\pgfqpoint{5.700000in}{5.700000in}}%
\pgfusepath{clip}%
\pgfsetbuttcap%
\pgfsetroundjoin%
\definecolor{currentfill}{rgb}{0.276194,0.190074,0.493001}%
\pgfsetfillcolor{currentfill}%
\pgfsetfillopacity{0.700000}%
\pgfsetlinewidth{0.000000pt}%
\definecolor{currentstroke}{rgb}{0.000000,0.000000,0.000000}%
\pgfsetstrokecolor{currentstroke}%
\pgfsetdash{}{0pt}%
\pgfpathmoveto{\pgfqpoint{5.958405in}{2.495186in}}%
\pgfpathlineto{\pgfqpoint{5.972533in}{2.494368in}}%
\pgfpathlineto{\pgfqpoint{5.986670in}{2.493574in}}%
\pgfpathlineto{\pgfqpoint{6.000816in}{2.492805in}}%
\pgfpathlineto{\pgfqpoint{6.014971in}{2.492059in}}%
\pgfpathlineto{\pgfqpoint{6.007933in}{2.487250in}}%
\pgfpathlineto{\pgfqpoint{6.000889in}{2.482399in}}%
\pgfpathlineto{\pgfqpoint{5.993836in}{2.477503in}}%
\pgfpathlineto{\pgfqpoint{5.986777in}{2.472560in}}%
\pgfpathlineto{\pgfqpoint{5.972602in}{2.473196in}}%
\pgfpathlineto{\pgfqpoint{5.958436in}{2.473857in}}%
\pgfpathlineto{\pgfqpoint{5.944279in}{2.474542in}}%
\pgfpathlineto{\pgfqpoint{5.930131in}{2.475252in}}%
\pgfpathlineto{\pgfqpoint{5.937210in}{2.480299in}}%
\pgfpathlineto{\pgfqpoint{5.944282in}{2.485301in}}%
\pgfpathlineto{\pgfqpoint{5.951347in}{2.490262in}}%
\pgfpathlineto{\pgfqpoint{5.958405in}{2.495186in}}%
\pgfpathclose%
\pgfusepath{fill}%
\end{pgfscope}%
\begin{pgfscope}%
\pgfpathrectangle{\pgfqpoint{1.150000in}{0.150000in}}{\pgfqpoint{5.700000in}{5.700000in}}%
\pgfusepath{clip}%
\pgfsetbuttcap%
\pgfsetroundjoin%
\definecolor{currentfill}{rgb}{0.267004,0.004874,0.329415}%
\pgfsetfillcolor{currentfill}%
\pgfsetfillopacity{0.700000}%
\pgfsetlinewidth{0.000000pt}%
\definecolor{currentstroke}{rgb}{0.000000,0.000000,0.000000}%
\pgfsetstrokecolor{currentstroke}%
\pgfsetdash{}{0pt}%
\pgfpathmoveto{\pgfqpoint{3.753583in}{2.151499in}}%
\pgfpathlineto{\pgfqpoint{3.767112in}{2.147257in}}%
\pgfpathlineto{\pgfqpoint{3.780646in}{2.143045in}}%
\pgfpathlineto{\pgfqpoint{3.794187in}{2.138862in}}%
\pgfpathlineto{\pgfqpoint{3.807733in}{2.134709in}}%
\pgfpathlineto{\pgfqpoint{3.799770in}{2.127605in}}%
\pgfpathlineto{\pgfqpoint{3.791800in}{2.120566in}}%
\pgfpathlineto{\pgfqpoint{3.783823in}{2.113597in}}%
\pgfpathlineto{\pgfqpoint{3.775841in}{2.106701in}}%
\pgfpathlineto{\pgfqpoint{3.762280in}{2.111016in}}%
\pgfpathlineto{\pgfqpoint{3.748724in}{2.115360in}}%
\pgfpathlineto{\pgfqpoint{3.735175in}{2.119733in}}%
\pgfpathlineto{\pgfqpoint{3.721631in}{2.124136in}}%
\pgfpathlineto{\pgfqpoint{3.729629in}{2.130866in}}%
\pgfpathlineto{\pgfqpoint{3.737620in}{2.137673in}}%
\pgfpathlineto{\pgfqpoint{3.745605in}{2.144551in}}%
\pgfpathlineto{\pgfqpoint{3.753583in}{2.151499in}}%
\pgfpathclose%
\pgfusepath{fill}%
\end{pgfscope}%
\begin{pgfscope}%
\pgfpathrectangle{\pgfqpoint{1.150000in}{0.150000in}}{\pgfqpoint{5.700000in}{5.700000in}}%
\pgfusepath{clip}%
\pgfsetbuttcap%
\pgfsetroundjoin%
\definecolor{currentfill}{rgb}{0.268510,0.009605,0.335427}%
\pgfsetfillcolor{currentfill}%
\pgfsetfillopacity{0.700000}%
\pgfsetlinewidth{0.000000pt}%
\definecolor{currentstroke}{rgb}{0.000000,0.000000,0.000000}%
\pgfsetstrokecolor{currentstroke}%
\pgfsetdash{}{0pt}%
\pgfpathmoveto{\pgfqpoint{4.119717in}{2.167723in}}%
\pgfpathlineto{\pgfqpoint{4.133328in}{2.164458in}}%
\pgfpathlineto{\pgfqpoint{4.146945in}{2.161220in}}%
\pgfpathlineto{\pgfqpoint{4.160569in}{2.158010in}}%
\pgfpathlineto{\pgfqpoint{4.174200in}{2.154828in}}%
\pgfpathlineto{\pgfqpoint{4.166374in}{2.146621in}}%
\pgfpathlineto{\pgfqpoint{4.158543in}{2.138424in}}%
\pgfpathlineto{\pgfqpoint{4.150706in}{2.130240in}}%
\pgfpathlineto{\pgfqpoint{4.142863in}{2.122071in}}%
\pgfpathlineto{\pgfqpoint{4.129221in}{2.125376in}}%
\pgfpathlineto{\pgfqpoint{4.115584in}{2.128707in}}%
\pgfpathlineto{\pgfqpoint{4.101955in}{2.132066in}}%
\pgfpathlineto{\pgfqpoint{4.088331in}{2.135453in}}%
\pgfpathlineto{\pgfqpoint{4.096186in}{2.143495in}}%
\pgfpathlineto{\pgfqpoint{4.104036in}{2.151556in}}%
\pgfpathlineto{\pgfqpoint{4.111879in}{2.159633in}}%
\pgfpathlineto{\pgfqpoint{4.119717in}{2.167723in}}%
\pgfpathclose%
\pgfusepath{fill}%
\end{pgfscope}%
\begin{pgfscope}%
\pgfpathrectangle{\pgfqpoint{1.150000in}{0.150000in}}{\pgfqpoint{5.700000in}{5.700000in}}%
\pgfusepath{clip}%
\pgfsetbuttcap%
\pgfsetroundjoin%
\definecolor{currentfill}{rgb}{0.272594,0.025563,0.353093}%
\pgfsetfillcolor{currentfill}%
\pgfsetfillopacity{0.700000}%
\pgfsetlinewidth{0.000000pt}%
\definecolor{currentstroke}{rgb}{0.000000,0.000000,0.000000}%
\pgfsetstrokecolor{currentstroke}%
\pgfsetdash{}{0pt}%
\pgfpathmoveto{\pgfqpoint{4.345697in}{2.197694in}}%
\pgfpathlineto{\pgfqpoint{4.359364in}{2.194958in}}%
\pgfpathlineto{\pgfqpoint{4.373038in}{2.192249in}}%
\pgfpathlineto{\pgfqpoint{4.386720in}{2.189567in}}%
\pgfpathlineto{\pgfqpoint{4.400408in}{2.186911in}}%
\pgfpathlineto{\pgfqpoint{4.392663in}{2.178441in}}%
\pgfpathlineto{\pgfqpoint{4.384911in}{2.169952in}}%
\pgfpathlineto{\pgfqpoint{4.377154in}{2.161447in}}%
\pgfpathlineto{\pgfqpoint{4.369392in}{2.152927in}}%
\pgfpathlineto{\pgfqpoint{4.355692in}{2.155678in}}%
\pgfpathlineto{\pgfqpoint{4.341999in}{2.158455in}}%
\pgfpathlineto{\pgfqpoint{4.328313in}{2.161259in}}%
\pgfpathlineto{\pgfqpoint{4.314634in}{2.164090in}}%
\pgfpathlineto{\pgfqpoint{4.322408in}{2.172509in}}%
\pgfpathlineto{\pgfqpoint{4.330177in}{2.180918in}}%
\pgfpathlineto{\pgfqpoint{4.337939in}{2.189313in}}%
\pgfpathlineto{\pgfqpoint{4.345697in}{2.197694in}}%
\pgfpathclose%
\pgfusepath{fill}%
\end{pgfscope}%
\begin{pgfscope}%
\pgfpathrectangle{\pgfqpoint{1.150000in}{0.150000in}}{\pgfqpoint{5.700000in}{5.700000in}}%
\pgfusepath{clip}%
\pgfsetbuttcap%
\pgfsetroundjoin%
\definecolor{currentfill}{rgb}{0.281924,0.089666,0.412415}%
\pgfsetfillcolor{currentfill}%
\pgfsetfillopacity{0.700000}%
\pgfsetlinewidth{0.000000pt}%
\definecolor{currentstroke}{rgb}{0.000000,0.000000,0.000000}%
\pgfsetstrokecolor{currentstroke}%
\pgfsetdash{}{0pt}%
\pgfpathmoveto{\pgfqpoint{4.883431in}{2.300870in}}%
\pgfpathlineto{\pgfqpoint{4.897247in}{2.299138in}}%
\pgfpathlineto{\pgfqpoint{4.911070in}{2.297431in}}%
\pgfpathlineto{\pgfqpoint{4.924901in}{2.295749in}}%
\pgfpathlineto{\pgfqpoint{4.938739in}{2.294092in}}%
\pgfpathlineto{\pgfqpoint{4.931193in}{2.286109in}}%
\pgfpathlineto{\pgfqpoint{4.923640in}{2.278063in}}%
\pgfpathlineto{\pgfqpoint{4.916081in}{2.269955in}}%
\pgfpathlineto{\pgfqpoint{4.908516in}{2.261784in}}%
\pgfpathlineto{\pgfqpoint{4.894665in}{2.263468in}}%
\pgfpathlineto{\pgfqpoint{4.880822in}{2.265178in}}%
\pgfpathlineto{\pgfqpoint{4.866987in}{2.266913in}}%
\pgfpathlineto{\pgfqpoint{4.853160in}{2.268673in}}%
\pgfpathlineto{\pgfqpoint{4.860737in}{2.276811in}}%
\pgfpathlineto{\pgfqpoint{4.868308in}{2.284890in}}%
\pgfpathlineto{\pgfqpoint{4.875873in}{2.292910in}}%
\pgfpathlineto{\pgfqpoint{4.883431in}{2.300870in}}%
\pgfpathclose%
\pgfusepath{fill}%
\end{pgfscope}%
\begin{pgfscope}%
\pgfpathrectangle{\pgfqpoint{1.150000in}{0.150000in}}{\pgfqpoint{5.700000in}{5.700000in}}%
\pgfusepath{clip}%
\pgfsetbuttcap%
\pgfsetroundjoin%
\definecolor{currentfill}{rgb}{0.277018,0.050344,0.375715}%
\pgfsetfillcolor{currentfill}%
\pgfsetfillopacity{0.700000}%
\pgfsetlinewidth{0.000000pt}%
\definecolor{currentstroke}{rgb}{0.000000,0.000000,0.000000}%
\pgfsetstrokecolor{currentstroke}%
\pgfsetdash{}{0pt}%
\pgfpathmoveto{\pgfqpoint{3.192764in}{2.226475in}}%
\pgfpathlineto{\pgfqpoint{3.206202in}{2.220470in}}%
\pgfpathlineto{\pgfqpoint{3.219644in}{2.214501in}}%
\pgfpathlineto{\pgfqpoint{3.233091in}{2.208567in}}%
\pgfpathlineto{\pgfqpoint{3.246542in}{2.202668in}}%
\pgfpathlineto{\pgfqpoint{3.238311in}{2.198803in}}%
\pgfpathlineto{\pgfqpoint{3.230071in}{2.195103in}}%
\pgfpathlineto{\pgfqpoint{3.221821in}{2.191573in}}%
\pgfpathlineto{\pgfqpoint{3.213560in}{2.188220in}}%
\pgfpathlineto{\pgfqpoint{3.200088in}{2.194334in}}%
\pgfpathlineto{\pgfqpoint{3.186620in}{2.200483in}}%
\pgfpathlineto{\pgfqpoint{3.173156in}{2.206667in}}%
\pgfpathlineto{\pgfqpoint{3.159697in}{2.212887in}}%
\pgfpathlineto{\pgfqpoint{3.167979in}{2.216021in}}%
\pgfpathlineto{\pgfqpoint{3.176252in}{2.219333in}}%
\pgfpathlineto{\pgfqpoint{3.184513in}{2.222820in}}%
\pgfpathlineto{\pgfqpoint{3.192764in}{2.226475in}}%
\pgfpathclose%
\pgfusepath{fill}%
\end{pgfscope}%
\begin{pgfscope}%
\pgfpathrectangle{\pgfqpoint{1.150000in}{0.150000in}}{\pgfqpoint{5.700000in}{5.700000in}}%
\pgfusepath{clip}%
\pgfsetbuttcap%
\pgfsetroundjoin%
\definecolor{currentfill}{rgb}{0.282290,0.145912,0.461510}%
\pgfsetfillcolor{currentfill}%
\pgfsetfillopacity{0.700000}%
\pgfsetlinewidth{0.000000pt}%
\definecolor{currentstroke}{rgb}{0.000000,0.000000,0.000000}%
\pgfsetstrokecolor{currentstroke}%
\pgfsetdash{}{0pt}%
\pgfpathmoveto{\pgfqpoint{5.421228in}{2.407370in}}%
\pgfpathlineto{\pgfqpoint{5.435203in}{2.406274in}}%
\pgfpathlineto{\pgfqpoint{5.449186in}{2.405203in}}%
\pgfpathlineto{\pgfqpoint{5.463177in}{2.404157in}}%
\pgfpathlineto{\pgfqpoint{5.477177in}{2.403135in}}%
\pgfpathlineto{\pgfqpoint{5.469865in}{2.396616in}}%
\pgfpathlineto{\pgfqpoint{5.462545in}{2.390027in}}%
\pgfpathlineto{\pgfqpoint{5.455218in}{2.383366in}}%
\pgfpathlineto{\pgfqpoint{5.447884in}{2.376631in}}%
\pgfpathlineto{\pgfqpoint{5.433869in}{2.377612in}}%
\pgfpathlineto{\pgfqpoint{5.419862in}{2.378619in}}%
\pgfpathlineto{\pgfqpoint{5.405865in}{2.379650in}}%
\pgfpathlineto{\pgfqpoint{5.391875in}{2.380706in}}%
\pgfpathlineto{\pgfqpoint{5.399224in}{2.387476in}}%
\pgfpathlineto{\pgfqpoint{5.406566in}{2.394176in}}%
\pgfpathlineto{\pgfqpoint{5.413901in}{2.400806in}}%
\pgfpathlineto{\pgfqpoint{5.421228in}{2.407370in}}%
\pgfpathclose%
\pgfusepath{fill}%
\end{pgfscope}%
\begin{pgfscope}%
\pgfpathrectangle{\pgfqpoint{1.150000in}{0.150000in}}{\pgfqpoint{5.700000in}{5.700000in}}%
\pgfusepath{clip}%
\pgfsetbuttcap%
\pgfsetroundjoin%
\definecolor{currentfill}{rgb}{0.278826,0.175490,0.483397}%
\pgfsetfillcolor{currentfill}%
\pgfsetfillopacity{0.700000}%
\pgfsetlinewidth{0.000000pt}%
\definecolor{currentstroke}{rgb}{0.000000,0.000000,0.000000}%
\pgfsetstrokecolor{currentstroke}%
\pgfsetdash{}{0pt}%
\pgfpathmoveto{\pgfqpoint{2.609093in}{2.462176in}}%
\pgfpathlineto{\pgfqpoint{2.622483in}{2.453988in}}%
\pgfpathlineto{\pgfqpoint{2.635876in}{2.445849in}}%
\pgfpathlineto{\pgfqpoint{2.649272in}{2.437758in}}%
\pgfpathlineto{\pgfqpoint{2.662670in}{2.429714in}}%
\pgfpathlineto{\pgfqpoint{2.654059in}{2.430480in}}%
\pgfpathlineto{\pgfqpoint{2.645433in}{2.431515in}}%
\pgfpathlineto{\pgfqpoint{2.636789in}{2.432825in}}%
\pgfpathlineto{\pgfqpoint{2.628129in}{2.434417in}}%
\pgfpathlineto{\pgfqpoint{2.614700in}{2.442720in}}%
\pgfpathlineto{\pgfqpoint{2.601274in}{2.451070in}}%
\pgfpathlineto{\pgfqpoint{2.587850in}{2.459469in}}%
\pgfpathlineto{\pgfqpoint{2.574429in}{2.467916in}}%
\pgfpathlineto{\pgfqpoint{2.583121in}{2.466059in}}%
\pgfpathlineto{\pgfqpoint{2.591795in}{2.464488in}}%
\pgfpathlineto{\pgfqpoint{2.600452in}{2.463196in}}%
\pgfpathlineto{\pgfqpoint{2.609093in}{2.462176in}}%
\pgfpathclose%
\pgfusepath{fill}%
\end{pgfscope}%
\begin{pgfscope}%
\pgfpathrectangle{\pgfqpoint{1.150000in}{0.150000in}}{\pgfqpoint{5.700000in}{5.700000in}}%
\pgfusepath{clip}%
\pgfsetbuttcap%
\pgfsetroundjoin%
\definecolor{currentfill}{rgb}{0.267004,0.004874,0.329415}%
\pgfsetfillcolor{currentfill}%
\pgfsetfillopacity{0.700000}%
\pgfsetlinewidth{0.000000pt}%
\definecolor{currentstroke}{rgb}{0.000000,0.000000,0.000000}%
\pgfsetstrokecolor{currentstroke}%
\pgfsetdash{}{0pt}%
\pgfpathmoveto{\pgfqpoint{3.893709in}{2.147977in}}%
\pgfpathlineto{\pgfqpoint{3.907270in}{2.144117in}}%
\pgfpathlineto{\pgfqpoint{3.920838in}{2.140286in}}%
\pgfpathlineto{\pgfqpoint{3.934412in}{2.136484in}}%
\pgfpathlineto{\pgfqpoint{3.947992in}{2.132710in}}%
\pgfpathlineto{\pgfqpoint{3.940081in}{2.125095in}}%
\pgfpathlineto{\pgfqpoint{3.932165in}{2.117524in}}%
\pgfpathlineto{\pgfqpoint{3.924243in}{2.110000in}}%
\pgfpathlineto{\pgfqpoint{3.916314in}{2.102529in}}%
\pgfpathlineto{\pgfqpoint{3.902720in}{2.106451in}}%
\pgfpathlineto{\pgfqpoint{3.889133in}{2.110401in}}%
\pgfpathlineto{\pgfqpoint{3.875551in}{2.114380in}}%
\pgfpathlineto{\pgfqpoint{3.861976in}{2.118388in}}%
\pgfpathlineto{\pgfqpoint{3.869918in}{2.125707in}}%
\pgfpathlineto{\pgfqpoint{3.877855in}{2.133080in}}%
\pgfpathlineto{\pgfqpoint{3.885785in}{2.140505in}}%
\pgfpathlineto{\pgfqpoint{3.893709in}{2.147977in}}%
\pgfpathclose%
\pgfusepath{fill}%
\end{pgfscope}%
\begin{pgfscope}%
\pgfpathrectangle{\pgfqpoint{1.150000in}{0.150000in}}{\pgfqpoint{5.700000in}{5.700000in}}%
\pgfusepath{clip}%
\pgfsetbuttcap%
\pgfsetroundjoin%
\definecolor{currentfill}{rgb}{0.277018,0.050344,0.375715}%
\pgfsetfillcolor{currentfill}%
\pgfsetfillopacity{0.700000}%
\pgfsetlinewidth{0.000000pt}%
\definecolor{currentstroke}{rgb}{0.000000,0.000000,0.000000}%
\pgfsetstrokecolor{currentstroke}%
\pgfsetdash{}{0pt}%
\pgfpathmoveto{\pgfqpoint{4.571749in}{2.234706in}}%
\pgfpathlineto{\pgfqpoint{4.585479in}{2.232437in}}%
\pgfpathlineto{\pgfqpoint{4.599216in}{2.230195in}}%
\pgfpathlineto{\pgfqpoint{4.612961in}{2.227978in}}%
\pgfpathlineto{\pgfqpoint{4.626713in}{2.225787in}}%
\pgfpathlineto{\pgfqpoint{4.619047in}{2.217336in}}%
\pgfpathlineto{\pgfqpoint{4.611375in}{2.208843in}}%
\pgfpathlineto{\pgfqpoint{4.603697in}{2.200310in}}%
\pgfpathlineto{\pgfqpoint{4.596013in}{2.191738in}}%
\pgfpathlineto{\pgfqpoint{4.582250in}{2.193998in}}%
\pgfpathlineto{\pgfqpoint{4.568495in}{2.196283in}}%
\pgfpathlineto{\pgfqpoint{4.554746in}{2.198594in}}%
\pgfpathlineto{\pgfqpoint{4.541005in}{2.200931in}}%
\pgfpathlineto{\pgfqpoint{4.548700in}{2.209429in}}%
\pgfpathlineto{\pgfqpoint{4.556389in}{2.217891in}}%
\pgfpathlineto{\pgfqpoint{4.564072in}{2.226318in}}%
\pgfpathlineto{\pgfqpoint{4.571749in}{2.234706in}}%
\pgfpathclose%
\pgfusepath{fill}%
\end{pgfscope}%
\begin{pgfscope}%
\pgfpathrectangle{\pgfqpoint{1.150000in}{0.150000in}}{\pgfqpoint{5.700000in}{5.700000in}}%
\pgfusepath{clip}%
\pgfsetbuttcap%
\pgfsetroundjoin%
\definecolor{currentfill}{rgb}{0.283197,0.115680,0.436115}%
\pgfsetfillcolor{currentfill}%
\pgfsetfillopacity{0.700000}%
\pgfsetlinewidth{0.000000pt}%
\definecolor{currentstroke}{rgb}{0.000000,0.000000,0.000000}%
\pgfsetstrokecolor{currentstroke}%
\pgfsetdash{}{0pt}%
\pgfpathmoveto{\pgfqpoint{5.109670in}{2.343487in}}%
\pgfpathlineto{\pgfqpoint{5.123555in}{2.342073in}}%
\pgfpathlineto{\pgfqpoint{5.137448in}{2.340684in}}%
\pgfpathlineto{\pgfqpoint{5.151349in}{2.339319in}}%
\pgfpathlineto{\pgfqpoint{5.165259in}{2.337980in}}%
\pgfpathlineto{\pgfqpoint{5.157803in}{2.330507in}}%
\pgfpathlineto{\pgfqpoint{5.150341in}{2.322963in}}%
\pgfpathlineto{\pgfqpoint{5.142872in}{2.315347in}}%
\pgfpathlineto{\pgfqpoint{5.135396in}{2.307659in}}%
\pgfpathlineto{\pgfqpoint{5.121474in}{2.308999in}}%
\pgfpathlineto{\pgfqpoint{5.107560in}{2.310365in}}%
\pgfpathlineto{\pgfqpoint{5.093654in}{2.311755in}}%
\pgfpathlineto{\pgfqpoint{5.079756in}{2.313170in}}%
\pgfpathlineto{\pgfqpoint{5.087244in}{2.320852in}}%
\pgfpathlineto{\pgfqpoint{5.094726in}{2.328465in}}%
\pgfpathlineto{\pgfqpoint{5.102201in}{2.336010in}}%
\pgfpathlineto{\pgfqpoint{5.109670in}{2.343487in}}%
\pgfpathclose%
\pgfusepath{fill}%
\end{pgfscope}%
\begin{pgfscope}%
\pgfpathrectangle{\pgfqpoint{1.150000in}{0.150000in}}{\pgfqpoint{5.700000in}{5.700000in}}%
\pgfusepath{clip}%
\pgfsetbuttcap%
\pgfsetroundjoin%
\definecolor{currentfill}{rgb}{0.280255,0.165693,0.476498}%
\pgfsetfillcolor{currentfill}%
\pgfsetfillopacity{0.700000}%
\pgfsetlinewidth{0.000000pt}%
\definecolor{currentstroke}{rgb}{0.000000,0.000000,0.000000}%
\pgfsetstrokecolor{currentstroke}%
\pgfsetdash{}{0pt}%
\pgfpathmoveto{\pgfqpoint{5.647456in}{2.444714in}}%
\pgfpathlineto{\pgfqpoint{5.661501in}{2.443792in}}%
\pgfpathlineto{\pgfqpoint{5.675554in}{2.442894in}}%
\pgfpathlineto{\pgfqpoint{5.689615in}{2.442020in}}%
\pgfpathlineto{\pgfqpoint{5.703686in}{2.441171in}}%
\pgfpathlineto{\pgfqpoint{5.696483in}{2.435363in}}%
\pgfpathlineto{\pgfqpoint{5.689272in}{2.429491in}}%
\pgfpathlineto{\pgfqpoint{5.682054in}{2.423553in}}%
\pgfpathlineto{\pgfqpoint{5.674828in}{2.417546in}}%
\pgfpathlineto{\pgfqpoint{5.660740in}{2.418327in}}%
\pgfpathlineto{\pgfqpoint{5.646662in}{2.419133in}}%
\pgfpathlineto{\pgfqpoint{5.632592in}{2.419964in}}%
\pgfpathlineto{\pgfqpoint{5.618531in}{2.420818in}}%
\pgfpathlineto{\pgfqpoint{5.625774in}{2.426888in}}%
\pgfpathlineto{\pgfqpoint{5.633009in}{2.432892in}}%
\pgfpathlineto{\pgfqpoint{5.640236in}{2.438833in}}%
\pgfpathlineto{\pgfqpoint{5.647456in}{2.444714in}}%
\pgfpathclose%
\pgfusepath{fill}%
\end{pgfscope}%
\begin{pgfscope}%
\pgfpathrectangle{\pgfqpoint{1.150000in}{0.150000in}}{\pgfqpoint{5.700000in}{5.700000in}}%
\pgfusepath{clip}%
\pgfsetbuttcap%
\pgfsetroundjoin%
\definecolor{currentfill}{rgb}{0.280894,0.078907,0.402329}%
\pgfsetfillcolor{currentfill}%
\pgfsetfillopacity{0.700000}%
\pgfsetlinewidth{0.000000pt}%
\definecolor{currentstroke}{rgb}{0.000000,0.000000,0.000000}%
\pgfsetstrokecolor{currentstroke}%
\pgfsetdash{}{0pt}%
\pgfpathmoveto{\pgfqpoint{4.797928in}{2.275970in}}%
\pgfpathlineto{\pgfqpoint{4.811725in}{2.274108in}}%
\pgfpathlineto{\pgfqpoint{4.825529in}{2.272271in}}%
\pgfpathlineto{\pgfqpoint{4.839341in}{2.270459in}}%
\pgfpathlineto{\pgfqpoint{4.853160in}{2.268673in}}%
\pgfpathlineto{\pgfqpoint{4.845577in}{2.260477in}}%
\pgfpathlineto{\pgfqpoint{4.837987in}{2.252222in}}%
\pgfpathlineto{\pgfqpoint{4.830391in}{2.243910in}}%
\pgfpathlineto{\pgfqpoint{4.822789in}{2.235540in}}%
\pgfpathlineto{\pgfqpoint{4.808958in}{2.237368in}}%
\pgfpathlineto{\pgfqpoint{4.795134in}{2.239221in}}%
\pgfpathlineto{\pgfqpoint{4.781319in}{2.241099in}}%
\pgfpathlineto{\pgfqpoint{4.767511in}{2.243003in}}%
\pgfpathlineto{\pgfqpoint{4.775124in}{2.251326in}}%
\pgfpathlineto{\pgfqpoint{4.782732in}{2.259595in}}%
\pgfpathlineto{\pgfqpoint{4.790333in}{2.267810in}}%
\pgfpathlineto{\pgfqpoint{4.797928in}{2.275970in}}%
\pgfpathclose%
\pgfusepath{fill}%
\end{pgfscope}%
\begin{pgfscope}%
\pgfpathrectangle{\pgfqpoint{1.150000in}{0.150000in}}{\pgfqpoint{5.700000in}{5.700000in}}%
\pgfusepath{clip}%
\pgfsetbuttcap%
\pgfsetroundjoin%
\definecolor{currentfill}{rgb}{0.277134,0.185228,0.489898}%
\pgfsetfillcolor{currentfill}%
\pgfsetfillopacity{0.700000}%
\pgfsetlinewidth{0.000000pt}%
\definecolor{currentstroke}{rgb}{0.000000,0.000000,0.000000}%
\pgfsetstrokecolor{currentstroke}%
\pgfsetdash{}{0pt}%
\pgfpathmoveto{\pgfqpoint{5.873627in}{2.478331in}}%
\pgfpathlineto{\pgfqpoint{5.887740in}{2.477525in}}%
\pgfpathlineto{\pgfqpoint{5.901861in}{2.476743in}}%
\pgfpathlineto{\pgfqpoint{5.915991in}{2.475985in}}%
\pgfpathlineto{\pgfqpoint{5.930131in}{2.475252in}}%
\pgfpathlineto{\pgfqpoint{5.923044in}{2.470156in}}%
\pgfpathlineto{\pgfqpoint{5.915949in}{2.465010in}}%
\pgfpathlineto{\pgfqpoint{5.908847in}{2.459809in}}%
\pgfpathlineto{\pgfqpoint{5.901737in}{2.454551in}}%
\pgfpathlineto{\pgfqpoint{5.887578in}{2.455189in}}%
\pgfpathlineto{\pgfqpoint{5.873429in}{2.455852in}}%
\pgfpathlineto{\pgfqpoint{5.859288in}{2.456539in}}%
\pgfpathlineto{\pgfqpoint{5.845157in}{2.457250in}}%
\pgfpathlineto{\pgfqpoint{5.852286in}{2.462598in}}%
\pgfpathlineto{\pgfqpoint{5.859407in}{2.467893in}}%
\pgfpathlineto{\pgfqpoint{5.866521in}{2.473136in}}%
\pgfpathlineto{\pgfqpoint{5.873627in}{2.478331in}}%
\pgfpathclose%
\pgfusepath{fill}%
\end{pgfscope}%
\begin{pgfscope}%
\pgfpathrectangle{\pgfqpoint{1.150000in}{0.150000in}}{\pgfqpoint{5.700000in}{5.700000in}}%
\pgfusepath{clip}%
\pgfsetbuttcap%
\pgfsetroundjoin%
\definecolor{currentfill}{rgb}{0.271305,0.019942,0.347269}%
\pgfsetfillcolor{currentfill}%
\pgfsetfillopacity{0.700000}%
\pgfsetlinewidth{0.000000pt}%
\definecolor{currentstroke}{rgb}{0.000000,0.000000,0.000000}%
\pgfsetstrokecolor{currentstroke}%
\pgfsetdash{}{0pt}%
\pgfpathmoveto{\pgfqpoint{4.259986in}{2.175681in}}%
\pgfpathlineto{\pgfqpoint{4.273638in}{2.172743in}}%
\pgfpathlineto{\pgfqpoint{4.287297in}{2.169832in}}%
\pgfpathlineto{\pgfqpoint{4.300962in}{2.166947in}}%
\pgfpathlineto{\pgfqpoint{4.314634in}{2.164090in}}%
\pgfpathlineto{\pgfqpoint{4.306855in}{2.155661in}}%
\pgfpathlineto{\pgfqpoint{4.299070in}{2.147226in}}%
\pgfpathlineto{\pgfqpoint{4.291279in}{2.138787in}}%
\pgfpathlineto{\pgfqpoint{4.283483in}{2.130346in}}%
\pgfpathlineto{\pgfqpoint{4.269799in}{2.133312in}}%
\pgfpathlineto{\pgfqpoint{4.256122in}{2.136304in}}%
\pgfpathlineto{\pgfqpoint{4.242451in}{2.139324in}}%
\pgfpathlineto{\pgfqpoint{4.228788in}{2.142371in}}%
\pgfpathlineto{\pgfqpoint{4.236596in}{2.150698in}}%
\pgfpathlineto{\pgfqpoint{4.244398in}{2.159027in}}%
\pgfpathlineto{\pgfqpoint{4.252195in}{2.167356in}}%
\pgfpathlineto{\pgfqpoint{4.259986in}{2.175681in}}%
\pgfpathclose%
\pgfusepath{fill}%
\end{pgfscope}%
\begin{pgfscope}%
\pgfpathrectangle{\pgfqpoint{1.150000in}{0.150000in}}{\pgfqpoint{5.700000in}{5.700000in}}%
\pgfusepath{clip}%
\pgfsetbuttcap%
\pgfsetroundjoin%
\definecolor{currentfill}{rgb}{0.283197,0.115680,0.436115}%
\pgfsetfillcolor{currentfill}%
\pgfsetfillopacity{0.700000}%
\pgfsetlinewidth{0.000000pt}%
\definecolor{currentstroke}{rgb}{0.000000,0.000000,0.000000}%
\pgfsetstrokecolor{currentstroke}%
\pgfsetdash{}{0pt}%
\pgfpathmoveto{\pgfqpoint{2.857609in}{2.339154in}}%
\pgfpathlineto{\pgfqpoint{2.871017in}{2.331933in}}%
\pgfpathlineto{\pgfqpoint{2.884429in}{2.324753in}}%
\pgfpathlineto{\pgfqpoint{2.897844in}{2.317614in}}%
\pgfpathlineto{\pgfqpoint{2.911262in}{2.310516in}}%
\pgfpathlineto{\pgfqpoint{2.902823in}{2.309318in}}%
\pgfpathlineto{\pgfqpoint{2.894369in}{2.308349in}}%
\pgfpathlineto{\pgfqpoint{2.885902in}{2.307614in}}%
\pgfpathlineto{\pgfqpoint{2.877421in}{2.307120in}}%
\pgfpathlineto{\pgfqpoint{2.863976in}{2.314462in}}%
\pgfpathlineto{\pgfqpoint{2.850535in}{2.321844in}}%
\pgfpathlineto{\pgfqpoint{2.837096in}{2.329268in}}%
\pgfpathlineto{\pgfqpoint{2.823661in}{2.336733in}}%
\pgfpathlineto{\pgfqpoint{2.832169in}{2.336978in}}%
\pgfpathlineto{\pgfqpoint{2.840663in}{2.337467in}}%
\pgfpathlineto{\pgfqpoint{2.849143in}{2.338195in}}%
\pgfpathlineto{\pgfqpoint{2.857609in}{2.339154in}}%
\pgfpathclose%
\pgfusepath{fill}%
\end{pgfscope}%
\begin{pgfscope}%
\pgfpathrectangle{\pgfqpoint{1.150000in}{0.150000in}}{\pgfqpoint{5.700000in}{5.700000in}}%
\pgfusepath{clip}%
\pgfsetbuttcap%
\pgfsetroundjoin%
\definecolor{currentfill}{rgb}{0.267004,0.004874,0.329415}%
\pgfsetfillcolor{currentfill}%
\pgfsetfillopacity{0.700000}%
\pgfsetlinewidth{0.000000pt}%
\definecolor{currentstroke}{rgb}{0.000000,0.000000,0.000000}%
\pgfsetstrokecolor{currentstroke}%
\pgfsetdash{}{0pt}%
\pgfpathmoveto{\pgfqpoint{4.033902in}{2.149277in}}%
\pgfpathlineto{\pgfqpoint{4.047500in}{2.145779in}}%
\pgfpathlineto{\pgfqpoint{4.061104in}{2.142309in}}%
\pgfpathlineto{\pgfqpoint{4.074714in}{2.138867in}}%
\pgfpathlineto{\pgfqpoint{4.088331in}{2.135453in}}%
\pgfpathlineto{\pgfqpoint{4.080470in}{2.127433in}}%
\pgfpathlineto{\pgfqpoint{4.072604in}{2.119438in}}%
\pgfpathlineto{\pgfqpoint{4.064732in}{2.111470in}}%
\pgfpathlineto{\pgfqpoint{4.056853in}{2.103534in}}%
\pgfpathlineto{\pgfqpoint{4.043224in}{2.107083in}}%
\pgfpathlineto{\pgfqpoint{4.029601in}{2.110660in}}%
\pgfpathlineto{\pgfqpoint{4.015983in}{2.114265in}}%
\pgfpathlineto{\pgfqpoint{4.002373in}{2.117898in}}%
\pgfpathlineto{\pgfqpoint{4.010264in}{2.125694in}}%
\pgfpathlineto{\pgfqpoint{4.018149in}{2.133525in}}%
\pgfpathlineto{\pgfqpoint{4.026028in}{2.141387in}}%
\pgfpathlineto{\pgfqpoint{4.033902in}{2.149277in}}%
\pgfpathclose%
\pgfusepath{fill}%
\end{pgfscope}%
\begin{pgfscope}%
\pgfpathrectangle{\pgfqpoint{1.150000in}{0.150000in}}{\pgfqpoint{5.700000in}{5.700000in}}%
\pgfusepath{clip}%
\pgfsetbuttcap%
\pgfsetroundjoin%
\definecolor{currentfill}{rgb}{0.280267,0.073417,0.397163}%
\pgfsetfillcolor{currentfill}%
\pgfsetfillopacity{0.700000}%
\pgfsetlinewidth{0.000000pt}%
\definecolor{currentstroke}{rgb}{0.000000,0.000000,0.000000}%
\pgfsetstrokecolor{currentstroke}%
\pgfsetdash{}{0pt}%
\pgfpathmoveto{\pgfqpoint{3.052169in}{2.263962in}}%
\pgfpathlineto{\pgfqpoint{3.065596in}{2.257447in}}%
\pgfpathlineto{\pgfqpoint{3.079027in}{2.250971in}}%
\pgfpathlineto{\pgfqpoint{3.092461in}{2.244531in}}%
\pgfpathlineto{\pgfqpoint{3.105900in}{2.238129in}}%
\pgfpathlineto{\pgfqpoint{3.097584in}{2.235405in}}%
\pgfpathlineto{\pgfqpoint{3.089255in}{2.232875in}}%
\pgfpathlineto{\pgfqpoint{3.080916in}{2.230545in}}%
\pgfpathlineto{\pgfqpoint{3.072564in}{2.228421in}}%
\pgfpathlineto{\pgfqpoint{3.059101in}{2.235052in}}%
\pgfpathlineto{\pgfqpoint{3.045643in}{2.241720in}}%
\pgfpathlineto{\pgfqpoint{3.032188in}{2.248426in}}%
\pgfpathlineto{\pgfqpoint{3.018737in}{2.255170in}}%
\pgfpathlineto{\pgfqpoint{3.027113in}{2.257060in}}%
\pgfpathlineto{\pgfqpoint{3.035477in}{2.259159in}}%
\pgfpathlineto{\pgfqpoint{3.043829in}{2.261461in}}%
\pgfpathlineto{\pgfqpoint{3.052169in}{2.263962in}}%
\pgfpathclose%
\pgfusepath{fill}%
\end{pgfscope}%
\begin{pgfscope}%
\pgfpathrectangle{\pgfqpoint{1.150000in}{0.150000in}}{\pgfqpoint{5.700000in}{5.700000in}}%
\pgfusepath{clip}%
\pgfsetbuttcap%
\pgfsetroundjoin%
\definecolor{currentfill}{rgb}{0.268510,0.009605,0.335427}%
\pgfsetfillcolor{currentfill}%
\pgfsetfillopacity{0.700000}%
\pgfsetlinewidth{0.000000pt}%
\definecolor{currentstroke}{rgb}{0.000000,0.000000,0.000000}%
\pgfsetstrokecolor{currentstroke}%
\pgfsetdash{}{0pt}%
\pgfpathmoveto{\pgfqpoint{3.527278in}{2.155366in}}%
\pgfpathlineto{\pgfqpoint{3.540773in}{2.150421in}}%
\pgfpathlineto{\pgfqpoint{3.554274in}{2.145507in}}%
\pgfpathlineto{\pgfqpoint{3.567779in}{2.140625in}}%
\pgfpathlineto{\pgfqpoint{3.581290in}{2.135773in}}%
\pgfpathlineto{\pgfqpoint{3.573224in}{2.129854in}}%
\pgfpathlineto{\pgfqpoint{3.565150in}{2.124044in}}%
\pgfpathlineto{\pgfqpoint{3.557068in}{2.118348in}}%
\pgfpathlineto{\pgfqpoint{3.548978in}{2.112769in}}%
\pgfpathlineto{\pgfqpoint{3.535450in}{2.117808in}}%
\pgfpathlineto{\pgfqpoint{3.521928in}{2.122879in}}%
\pgfpathlineto{\pgfqpoint{3.508410in}{2.127980in}}%
\pgfpathlineto{\pgfqpoint{3.494897in}{2.133114in}}%
\pgfpathlineto{\pgfqpoint{3.503005in}{2.138499in}}%
\pgfpathlineto{\pgfqpoint{3.511104in}{2.144006in}}%
\pgfpathlineto{\pgfqpoint{3.519195in}{2.149630in}}%
\pgfpathlineto{\pgfqpoint{3.527278in}{2.155366in}}%
\pgfpathclose%
\pgfusepath{fill}%
\end{pgfscope}%
\begin{pgfscope}%
\pgfpathrectangle{\pgfqpoint{1.150000in}{0.150000in}}{\pgfqpoint{5.700000in}{5.700000in}}%
\pgfusepath{clip}%
\pgfsetbuttcap%
\pgfsetroundjoin%
\definecolor{currentfill}{rgb}{0.276022,0.044167,0.370164}%
\pgfsetfillcolor{currentfill}%
\pgfsetfillopacity{0.700000}%
\pgfsetlinewidth{0.000000pt}%
\definecolor{currentstroke}{rgb}{0.000000,0.000000,0.000000}%
\pgfsetstrokecolor{currentstroke}%
\pgfsetdash{}{0pt}%
\pgfpathmoveto{\pgfqpoint{4.486113in}{2.210541in}}%
\pgfpathlineto{\pgfqpoint{4.499825in}{2.208099in}}%
\pgfpathlineto{\pgfqpoint{4.513545in}{2.205683in}}%
\pgfpathlineto{\pgfqpoint{4.527271in}{2.203294in}}%
\pgfpathlineto{\pgfqpoint{4.541005in}{2.200931in}}%
\pgfpathlineto{\pgfqpoint{4.533305in}{2.192399in}}%
\pgfpathlineto{\pgfqpoint{4.525599in}{2.183835in}}%
\pgfpathlineto{\pgfqpoint{4.517887in}{2.175240in}}%
\pgfpathlineto{\pgfqpoint{4.510169in}{2.166616in}}%
\pgfpathlineto{\pgfqpoint{4.496424in}{2.169062in}}%
\pgfpathlineto{\pgfqpoint{4.482686in}{2.171533in}}%
\pgfpathlineto{\pgfqpoint{4.468955in}{2.174030in}}%
\pgfpathlineto{\pgfqpoint{4.455232in}{2.176554in}}%
\pgfpathlineto{\pgfqpoint{4.462961in}{2.185090in}}%
\pgfpathlineto{\pgfqpoint{4.470684in}{2.193602in}}%
\pgfpathlineto{\pgfqpoint{4.478401in}{2.202085in}}%
\pgfpathlineto{\pgfqpoint{4.486113in}{2.210541in}}%
\pgfpathclose%
\pgfusepath{fill}%
\end{pgfscope}%
\begin{pgfscope}%
\pgfpathrectangle{\pgfqpoint{1.150000in}{0.150000in}}{\pgfqpoint{5.700000in}{5.700000in}}%
\pgfusepath{clip}%
\pgfsetbuttcap%
\pgfsetroundjoin%
\definecolor{currentfill}{rgb}{0.282623,0.140926,0.457517}%
\pgfsetfillcolor{currentfill}%
\pgfsetfillopacity{0.700000}%
\pgfsetlinewidth{0.000000pt}%
\definecolor{currentstroke}{rgb}{0.000000,0.000000,0.000000}%
\pgfsetstrokecolor{currentstroke}%
\pgfsetdash{}{0pt}%
\pgfpathmoveto{\pgfqpoint{5.336001in}{2.385176in}}%
\pgfpathlineto{\pgfqpoint{5.349957in}{2.384021in}}%
\pgfpathlineto{\pgfqpoint{5.363921in}{2.382891in}}%
\pgfpathlineto{\pgfqpoint{5.377894in}{2.381786in}}%
\pgfpathlineto{\pgfqpoint{5.391875in}{2.380706in}}%
\pgfpathlineto{\pgfqpoint{5.384518in}{2.373864in}}%
\pgfpathlineto{\pgfqpoint{5.377154in}{2.366948in}}%
\pgfpathlineto{\pgfqpoint{5.369783in}{2.359958in}}%
\pgfpathlineto{\pgfqpoint{5.362404in}{2.352892in}}%
\pgfpathlineto{\pgfqpoint{5.348409in}{2.353946in}}%
\pgfpathlineto{\pgfqpoint{5.334422in}{2.355025in}}%
\pgfpathlineto{\pgfqpoint{5.320443in}{2.356129in}}%
\pgfpathlineto{\pgfqpoint{5.306473in}{2.357257in}}%
\pgfpathlineto{\pgfqpoint{5.313866in}{2.364344in}}%
\pgfpathlineto{\pgfqpoint{5.321252in}{2.371359in}}%
\pgfpathlineto{\pgfqpoint{5.328630in}{2.378302in}}%
\pgfpathlineto{\pgfqpoint{5.336001in}{2.385176in}}%
\pgfpathclose%
\pgfusepath{fill}%
\end{pgfscope}%
\begin{pgfscope}%
\pgfpathrectangle{\pgfqpoint{1.150000in}{0.150000in}}{\pgfqpoint{5.700000in}{5.700000in}}%
\pgfusepath{clip}%
\pgfsetbuttcap%
\pgfsetroundjoin%
\definecolor{currentfill}{rgb}{0.267004,0.004874,0.329415}%
\pgfsetfillcolor{currentfill}%
\pgfsetfillopacity{0.700000}%
\pgfsetlinewidth{0.000000pt}%
\definecolor{currentstroke}{rgb}{0.000000,0.000000,0.000000}%
\pgfsetstrokecolor{currentstroke}%
\pgfsetdash{}{0pt}%
\pgfpathmoveto{\pgfqpoint{3.667512in}{2.142049in}}%
\pgfpathlineto{\pgfqpoint{3.681033in}{2.137526in}}%
\pgfpathlineto{\pgfqpoint{3.694560in}{2.133033in}}%
\pgfpathlineto{\pgfqpoint{3.708093in}{2.128570in}}%
\pgfpathlineto{\pgfqpoint{3.721631in}{2.124136in}}%
\pgfpathlineto{\pgfqpoint{3.713626in}{2.117487in}}%
\pgfpathlineto{\pgfqpoint{3.705614in}{2.110923in}}%
\pgfpathlineto{\pgfqpoint{3.697595in}{2.104447in}}%
\pgfpathlineto{\pgfqpoint{3.689569in}{2.098065in}}%
\pgfpathlineto{\pgfqpoint{3.676015in}{2.102673in}}%
\pgfpathlineto{\pgfqpoint{3.662467in}{2.107310in}}%
\pgfpathlineto{\pgfqpoint{3.648924in}{2.111978in}}%
\pgfpathlineto{\pgfqpoint{3.635386in}{2.116676in}}%
\pgfpathlineto{\pgfqpoint{3.643428in}{2.122879in}}%
\pgfpathlineto{\pgfqpoint{3.651463in}{2.129178in}}%
\pgfpathlineto{\pgfqpoint{3.659491in}{2.135570in}}%
\pgfpathlineto{\pgfqpoint{3.667512in}{2.142049in}}%
\pgfpathclose%
\pgfusepath{fill}%
\end{pgfscope}%
\begin{pgfscope}%
\pgfpathrectangle{\pgfqpoint{1.150000in}{0.150000in}}{\pgfqpoint{5.700000in}{5.700000in}}%
\pgfusepath{clip}%
\pgfsetbuttcap%
\pgfsetroundjoin%
\definecolor{currentfill}{rgb}{0.271305,0.019942,0.347269}%
\pgfsetfillcolor{currentfill}%
\pgfsetfillopacity{0.700000}%
\pgfsetlinewidth{0.000000pt}%
\definecolor{currentstroke}{rgb}{0.000000,0.000000,0.000000}%
\pgfsetstrokecolor{currentstroke}%
\pgfsetdash{}{0pt}%
\pgfpathmoveto{\pgfqpoint{3.386977in}{2.175334in}}%
\pgfpathlineto{\pgfqpoint{3.400449in}{2.169943in}}%
\pgfpathlineto{\pgfqpoint{3.413927in}{2.164584in}}%
\pgfpathlineto{\pgfqpoint{3.427410in}{2.159258in}}%
\pgfpathlineto{\pgfqpoint{3.440897in}{2.153965in}}%
\pgfpathlineto{\pgfqpoint{3.432763in}{2.148902in}}%
\pgfpathlineto{\pgfqpoint{3.424621in}{2.143973in}}%
\pgfpathlineto{\pgfqpoint{3.416469in}{2.139184in}}%
\pgfpathlineto{\pgfqpoint{3.408309in}{2.134539in}}%
\pgfpathlineto{\pgfqpoint{3.394803in}{2.140034in}}%
\pgfpathlineto{\pgfqpoint{3.381301in}{2.145561in}}%
\pgfpathlineto{\pgfqpoint{3.367804in}{2.151121in}}%
\pgfpathlineto{\pgfqpoint{3.354312in}{2.156714in}}%
\pgfpathlineto{\pgfqpoint{3.362491in}{2.161152in}}%
\pgfpathlineto{\pgfqpoint{3.370662in}{2.165738in}}%
\pgfpathlineto{\pgfqpoint{3.378824in}{2.170467in}}%
\pgfpathlineto{\pgfqpoint{3.386977in}{2.175334in}}%
\pgfpathclose%
\pgfusepath{fill}%
\end{pgfscope}%
\begin{pgfscope}%
\pgfpathrectangle{\pgfqpoint{1.150000in}{0.150000in}}{\pgfqpoint{5.700000in}{5.700000in}}%
\pgfusepath{clip}%
\pgfsetbuttcap%
\pgfsetroundjoin%
\definecolor{currentfill}{rgb}{0.282910,0.105393,0.426902}%
\pgfsetfillcolor{currentfill}%
\pgfsetfillopacity{0.700000}%
\pgfsetlinewidth{0.000000pt}%
\definecolor{currentstroke}{rgb}{0.000000,0.000000,0.000000}%
\pgfsetstrokecolor{currentstroke}%
\pgfsetdash{}{0pt}%
\pgfpathmoveto{\pgfqpoint{5.024244in}{2.319082in}}%
\pgfpathlineto{\pgfqpoint{5.038110in}{2.317566in}}%
\pgfpathlineto{\pgfqpoint{5.051984in}{2.316076in}}%
\pgfpathlineto{\pgfqpoint{5.065866in}{2.314610in}}%
\pgfpathlineto{\pgfqpoint{5.079756in}{2.313170in}}%
\pgfpathlineto{\pgfqpoint{5.072260in}{2.305419in}}%
\pgfpathlineto{\pgfqpoint{5.064758in}{2.297599in}}%
\pgfpathlineto{\pgfqpoint{5.057249in}{2.289709in}}%
\pgfpathlineto{\pgfqpoint{5.049734in}{2.281750in}}%
\pgfpathlineto{\pgfqpoint{5.035831in}{2.283205in}}%
\pgfpathlineto{\pgfqpoint{5.021937in}{2.284684in}}%
\pgfpathlineto{\pgfqpoint{5.008051in}{2.286189in}}%
\pgfpathlineto{\pgfqpoint{4.994173in}{2.287720in}}%
\pgfpathlineto{\pgfqpoint{5.001701in}{2.295660in}}%
\pgfpathlineto{\pgfqpoint{5.009222in}{2.303533in}}%
\pgfpathlineto{\pgfqpoint{5.016736in}{2.311341in}}%
\pgfpathlineto{\pgfqpoint{5.024244in}{2.319082in}}%
\pgfpathclose%
\pgfusepath{fill}%
\end{pgfscope}%
\begin{pgfscope}%
\pgfpathrectangle{\pgfqpoint{1.150000in}{0.150000in}}{\pgfqpoint{5.700000in}{5.700000in}}%
\pgfusepath{clip}%
\pgfsetbuttcap%
\pgfsetroundjoin%
\definecolor{currentfill}{rgb}{0.280868,0.160771,0.472899}%
\pgfsetfillcolor{currentfill}%
\pgfsetfillopacity{0.700000}%
\pgfsetlinewidth{0.000000pt}%
\definecolor{currentstroke}{rgb}{0.000000,0.000000,0.000000}%
\pgfsetstrokecolor{currentstroke}%
\pgfsetdash{}{0pt}%
\pgfpathmoveto{\pgfqpoint{2.662670in}{2.429714in}}%
\pgfpathlineto{\pgfqpoint{2.676070in}{2.421717in}}%
\pgfpathlineto{\pgfqpoint{2.689473in}{2.413767in}}%
\pgfpathlineto{\pgfqpoint{2.702879in}{2.405863in}}%
\pgfpathlineto{\pgfqpoint{2.716288in}{2.398004in}}%
\pgfpathlineto{\pgfqpoint{2.707707in}{2.398517in}}%
\pgfpathlineto{\pgfqpoint{2.699110in}{2.399295in}}%
\pgfpathlineto{\pgfqpoint{2.690497in}{2.400344in}}%
\pgfpathlineto{\pgfqpoint{2.681868in}{2.401672in}}%
\pgfpathlineto{\pgfqpoint{2.668429in}{2.409789in}}%
\pgfpathlineto{\pgfqpoint{2.654993in}{2.417952in}}%
\pgfpathlineto{\pgfqpoint{2.641560in}{2.426161in}}%
\pgfpathlineto{\pgfqpoint{2.628129in}{2.434417in}}%
\pgfpathlineto{\pgfqpoint{2.636789in}{2.432825in}}%
\pgfpathlineto{\pgfqpoint{2.645433in}{2.431515in}}%
\pgfpathlineto{\pgfqpoint{2.654059in}{2.430480in}}%
\pgfpathlineto{\pgfqpoint{2.662670in}{2.429714in}}%
\pgfpathclose%
\pgfusepath{fill}%
\end{pgfscope}%
\begin{pgfscope}%
\pgfpathrectangle{\pgfqpoint{1.150000in}{0.150000in}}{\pgfqpoint{5.700000in}{5.700000in}}%
\pgfusepath{clip}%
\pgfsetbuttcap%
\pgfsetroundjoin%
\definecolor{currentfill}{rgb}{0.267004,0.004874,0.329415}%
\pgfsetfillcolor{currentfill}%
\pgfsetfillopacity{0.700000}%
\pgfsetlinewidth{0.000000pt}%
\definecolor{currentstroke}{rgb}{0.000000,0.000000,0.000000}%
\pgfsetstrokecolor{currentstroke}%
\pgfsetdash{}{0pt}%
\pgfpathmoveto{\pgfqpoint{3.807733in}{2.134709in}}%
\pgfpathlineto{\pgfqpoint{3.821285in}{2.130585in}}%
\pgfpathlineto{\pgfqpoint{3.834843in}{2.126490in}}%
\pgfpathlineto{\pgfqpoint{3.848406in}{2.122425in}}%
\pgfpathlineto{\pgfqpoint{3.861976in}{2.118388in}}%
\pgfpathlineto{\pgfqpoint{3.854027in}{2.111128in}}%
\pgfpathlineto{\pgfqpoint{3.846072in}{2.103929in}}%
\pgfpathlineto{\pgfqpoint{3.838110in}{2.096797in}}%
\pgfpathlineto{\pgfqpoint{3.830142in}{2.089735in}}%
\pgfpathlineto{\pgfqpoint{3.816558in}{2.093933in}}%
\pgfpathlineto{\pgfqpoint{3.802980in}{2.098160in}}%
\pgfpathlineto{\pgfqpoint{3.789407in}{2.102416in}}%
\pgfpathlineto{\pgfqpoint{3.775841in}{2.106701in}}%
\pgfpathlineto{\pgfqpoint{3.783823in}{2.113597in}}%
\pgfpathlineto{\pgfqpoint{3.791800in}{2.120566in}}%
\pgfpathlineto{\pgfqpoint{3.799770in}{2.127605in}}%
\pgfpathlineto{\pgfqpoint{3.807733in}{2.134709in}}%
\pgfpathclose%
\pgfusepath{fill}%
\end{pgfscope}%
\begin{pgfscope}%
\pgfpathrectangle{\pgfqpoint{1.150000in}{0.150000in}}{\pgfqpoint{5.700000in}{5.700000in}}%
\pgfusepath{clip}%
\pgfsetbuttcap%
\pgfsetroundjoin%
\definecolor{currentfill}{rgb}{0.279566,0.067836,0.391917}%
\pgfsetfillcolor{currentfill}%
\pgfsetfillopacity{0.700000}%
\pgfsetlinewidth{0.000000pt}%
\definecolor{currentstroke}{rgb}{0.000000,0.000000,0.000000}%
\pgfsetstrokecolor{currentstroke}%
\pgfsetdash{}{0pt}%
\pgfpathmoveto{\pgfqpoint{4.712354in}{2.250875in}}%
\pgfpathlineto{\pgfqpoint{4.726132in}{2.248869in}}%
\pgfpathlineto{\pgfqpoint{4.739917in}{2.246888in}}%
\pgfpathlineto{\pgfqpoint{4.753710in}{2.244933in}}%
\pgfpathlineto{\pgfqpoint{4.767511in}{2.243003in}}%
\pgfpathlineto{\pgfqpoint{4.759891in}{2.234627in}}%
\pgfpathlineto{\pgfqpoint{4.752265in}{2.226199in}}%
\pgfpathlineto{\pgfqpoint{4.744633in}{2.217720in}}%
\pgfpathlineto{\pgfqpoint{4.736995in}{2.209191in}}%
\pgfpathlineto{\pgfqpoint{4.723184in}{2.211176in}}%
\pgfpathlineto{\pgfqpoint{4.709380in}{2.213186in}}%
\pgfpathlineto{\pgfqpoint{4.695583in}{2.215222in}}%
\pgfpathlineto{\pgfqpoint{4.681794in}{2.217283in}}%
\pgfpathlineto{\pgfqpoint{4.689443in}{2.225753in}}%
\pgfpathlineto{\pgfqpoint{4.697086in}{2.234175in}}%
\pgfpathlineto{\pgfqpoint{4.704723in}{2.242550in}}%
\pgfpathlineto{\pgfqpoint{4.712354in}{2.250875in}}%
\pgfpathclose%
\pgfusepath{fill}%
\end{pgfscope}%
\begin{pgfscope}%
\pgfpathrectangle{\pgfqpoint{1.150000in}{0.150000in}}{\pgfqpoint{5.700000in}{5.700000in}}%
\pgfusepath{clip}%
\pgfsetbuttcap%
\pgfsetroundjoin%
\definecolor{currentfill}{rgb}{0.280868,0.160771,0.472899}%
\pgfsetfillcolor{currentfill}%
\pgfsetfillopacity{0.700000}%
\pgfsetlinewidth{0.000000pt}%
\definecolor{currentstroke}{rgb}{0.000000,0.000000,0.000000}%
\pgfsetstrokecolor{currentstroke}%
\pgfsetdash{}{0pt}%
\pgfpathmoveto{\pgfqpoint{5.562373in}{2.424483in}}%
\pgfpathlineto{\pgfqpoint{5.576400in}{2.423530in}}%
\pgfpathlineto{\pgfqpoint{5.590435in}{2.422602in}}%
\pgfpathlineto{\pgfqpoint{5.604479in}{2.421698in}}%
\pgfpathlineto{\pgfqpoint{5.618531in}{2.420818in}}%
\pgfpathlineto{\pgfqpoint{5.611281in}{2.414681in}}%
\pgfpathlineto{\pgfqpoint{5.604023in}{2.408475in}}%
\pgfpathlineto{\pgfqpoint{5.596757in}{2.402196in}}%
\pgfpathlineto{\pgfqpoint{5.589484in}{2.395844in}}%
\pgfpathlineto{\pgfqpoint{5.575415in}{2.396670in}}%
\pgfpathlineto{\pgfqpoint{5.561355in}{2.397519in}}%
\pgfpathlineto{\pgfqpoint{5.547304in}{2.398394in}}%
\pgfpathlineto{\pgfqpoint{5.533262in}{2.399293in}}%
\pgfpathlineto{\pgfqpoint{5.540551in}{2.405694in}}%
\pgfpathlineto{\pgfqpoint{5.547832in}{2.412025in}}%
\pgfpathlineto{\pgfqpoint{5.555106in}{2.418287in}}%
\pgfpathlineto{\pgfqpoint{5.562373in}{2.424483in}}%
\pgfpathclose%
\pgfusepath{fill}%
\end{pgfscope}%
\begin{pgfscope}%
\pgfpathrectangle{\pgfqpoint{1.150000in}{0.150000in}}{\pgfqpoint{5.700000in}{5.700000in}}%
\pgfusepath{clip}%
\pgfsetbuttcap%
\pgfsetroundjoin%
\definecolor{currentfill}{rgb}{0.276022,0.044167,0.370164}%
\pgfsetfillcolor{currentfill}%
\pgfsetfillopacity{0.700000}%
\pgfsetlinewidth{0.000000pt}%
\definecolor{currentstroke}{rgb}{0.000000,0.000000,0.000000}%
\pgfsetstrokecolor{currentstroke}%
\pgfsetdash{}{0pt}%
\pgfpathmoveto{\pgfqpoint{3.246542in}{2.202668in}}%
\pgfpathlineto{\pgfqpoint{3.259997in}{2.196804in}}%
\pgfpathlineto{\pgfqpoint{3.273457in}{2.190975in}}%
\pgfpathlineto{\pgfqpoint{3.286921in}{2.185180in}}%
\pgfpathlineto{\pgfqpoint{3.300390in}{2.179419in}}%
\pgfpathlineto{\pgfqpoint{3.292180in}{2.175344in}}%
\pgfpathlineto{\pgfqpoint{3.283961in}{2.171430in}}%
\pgfpathlineto{\pgfqpoint{3.275732in}{2.167684in}}%
\pgfpathlineto{\pgfqpoint{3.267493in}{2.164111in}}%
\pgfpathlineto{\pgfqpoint{3.254003in}{2.170086in}}%
\pgfpathlineto{\pgfqpoint{3.240518in}{2.176096in}}%
\pgfpathlineto{\pgfqpoint{3.227037in}{2.182141in}}%
\pgfpathlineto{\pgfqpoint{3.213560in}{2.188220in}}%
\pgfpathlineto{\pgfqpoint{3.221821in}{2.191573in}}%
\pgfpathlineto{\pgfqpoint{3.230071in}{2.195103in}}%
\pgfpathlineto{\pgfqpoint{3.238311in}{2.198803in}}%
\pgfpathlineto{\pgfqpoint{3.246542in}{2.202668in}}%
\pgfpathclose%
\pgfusepath{fill}%
\end{pgfscope}%
\begin{pgfscope}%
\pgfpathrectangle{\pgfqpoint{1.150000in}{0.150000in}}{\pgfqpoint{5.700000in}{5.700000in}}%
\pgfusepath{clip}%
\pgfsetbuttcap%
\pgfsetroundjoin%
\definecolor{currentfill}{rgb}{0.275191,0.194905,0.496005}%
\pgfsetfillcolor{currentfill}%
\pgfsetfillopacity{0.700000}%
\pgfsetlinewidth{0.000000pt}%
\definecolor{currentstroke}{rgb}{0.000000,0.000000,0.000000}%
\pgfsetstrokecolor{currentstroke}%
\pgfsetdash{}{0pt}%
\pgfpathmoveto{\pgfqpoint{6.014971in}{2.492059in}}%
\pgfpathlineto{\pgfqpoint{6.029135in}{2.491338in}}%
\pgfpathlineto{\pgfqpoint{6.043307in}{2.490641in}}%
\pgfpathlineto{\pgfqpoint{6.057489in}{2.489968in}}%
\pgfpathlineto{\pgfqpoint{6.050467in}{2.485244in}}%
\pgfpathlineto{\pgfqpoint{6.043438in}{2.480476in}}%
\pgfpathlineto{\pgfqpoint{6.036401in}{2.475661in}}%
\pgfpathlineto{\pgfqpoint{6.029356in}{2.470795in}}%
\pgfpathlineto{\pgfqpoint{6.015154in}{2.471359in}}%
\pgfpathlineto{\pgfqpoint{6.000961in}{2.471947in}}%
\pgfpathlineto{\pgfqpoint{5.986777in}{2.472560in}}%
\pgfpathlineto{\pgfqpoint{5.993836in}{2.477503in}}%
\pgfpathlineto{\pgfqpoint{6.000889in}{2.482399in}}%
\pgfpathlineto{\pgfqpoint{6.007933in}{2.487250in}}%
\pgfpathlineto{\pgfqpoint{6.014971in}{2.492059in}}%
\pgfpathclose%
\pgfusepath{fill}%
\end{pgfscope}%
\begin{pgfscope}%
\pgfpathrectangle{\pgfqpoint{1.150000in}{0.150000in}}{\pgfqpoint{5.700000in}{5.700000in}}%
\pgfusepath{clip}%
\pgfsetbuttcap%
\pgfsetroundjoin%
\definecolor{currentfill}{rgb}{0.269944,0.014625,0.341379}%
\pgfsetfillcolor{currentfill}%
\pgfsetfillopacity{0.700000}%
\pgfsetlinewidth{0.000000pt}%
\definecolor{currentstroke}{rgb}{0.000000,0.000000,0.000000}%
\pgfsetstrokecolor{currentstroke}%
\pgfsetdash{}{0pt}%
\pgfpathmoveto{\pgfqpoint{4.174200in}{2.154828in}}%
\pgfpathlineto{\pgfqpoint{4.187837in}{2.151673in}}%
\pgfpathlineto{\pgfqpoint{4.201480in}{2.148545in}}%
\pgfpathlineto{\pgfqpoint{4.215131in}{2.145444in}}%
\pgfpathlineto{\pgfqpoint{4.228788in}{2.142371in}}%
\pgfpathlineto{\pgfqpoint{4.220974in}{2.134047in}}%
\pgfpathlineto{\pgfqpoint{4.213155in}{2.125730in}}%
\pgfpathlineto{\pgfqpoint{4.205330in}{2.117423in}}%
\pgfpathlineto{\pgfqpoint{4.197499in}{2.109127in}}%
\pgfpathlineto{\pgfqpoint{4.183831in}{2.112323in}}%
\pgfpathlineto{\pgfqpoint{4.170168in}{2.115545in}}%
\pgfpathlineto{\pgfqpoint{4.156512in}{2.118795in}}%
\pgfpathlineto{\pgfqpoint{4.142863in}{2.122071in}}%
\pgfpathlineto{\pgfqpoint{4.150706in}{2.130240in}}%
\pgfpathlineto{\pgfqpoint{4.158543in}{2.138424in}}%
\pgfpathlineto{\pgfqpoint{4.166374in}{2.146621in}}%
\pgfpathlineto{\pgfqpoint{4.174200in}{2.154828in}}%
\pgfpathclose%
\pgfusepath{fill}%
\end{pgfscope}%
\begin{pgfscope}%
\pgfpathrectangle{\pgfqpoint{1.150000in}{0.150000in}}{\pgfqpoint{5.700000in}{5.700000in}}%
\pgfusepath{clip}%
\pgfsetbuttcap%
\pgfsetroundjoin%
\definecolor{currentfill}{rgb}{0.278012,0.180367,0.486697}%
\pgfsetfillcolor{currentfill}%
\pgfsetfillopacity{0.700000}%
\pgfsetlinewidth{0.000000pt}%
\definecolor{currentstroke}{rgb}{0.000000,0.000000,0.000000}%
\pgfsetstrokecolor{currentstroke}%
\pgfsetdash{}{0pt}%
\pgfpathmoveto{\pgfqpoint{5.788719in}{2.460338in}}%
\pgfpathlineto{\pgfqpoint{5.802815in}{2.459530in}}%
\pgfpathlineto{\pgfqpoint{5.816920in}{2.458745in}}%
\pgfpathlineto{\pgfqpoint{5.831034in}{2.457986in}}%
\pgfpathlineto{\pgfqpoint{5.845157in}{2.457250in}}%
\pgfpathlineto{\pgfqpoint{5.838020in}{2.451844in}}%
\pgfpathlineto{\pgfqpoint{5.830876in}{2.446379in}}%
\pgfpathlineto{\pgfqpoint{5.823724in}{2.440850in}}%
\pgfpathlineto{\pgfqpoint{5.816564in}{2.435257in}}%
\pgfpathlineto{\pgfqpoint{5.802423in}{2.435911in}}%
\pgfpathlineto{\pgfqpoint{5.788291in}{2.436589in}}%
\pgfpathlineto{\pgfqpoint{5.774168in}{2.437292in}}%
\pgfpathlineto{\pgfqpoint{5.760054in}{2.438019in}}%
\pgfpathlineto{\pgfqpoint{5.767232in}{2.443689in}}%
\pgfpathlineto{\pgfqpoint{5.774402in}{2.449297in}}%
\pgfpathlineto{\pgfqpoint{5.781564in}{2.454846in}}%
\pgfpathlineto{\pgfqpoint{5.788719in}{2.460338in}}%
\pgfpathclose%
\pgfusepath{fill}%
\end{pgfscope}%
\begin{pgfscope}%
\pgfpathrectangle{\pgfqpoint{1.150000in}{0.150000in}}{\pgfqpoint{5.700000in}{5.700000in}}%
\pgfusepath{clip}%
\pgfsetbuttcap%
\pgfsetroundjoin%
\definecolor{currentfill}{rgb}{0.273809,0.031497,0.358853}%
\pgfsetfillcolor{currentfill}%
\pgfsetfillopacity{0.700000}%
\pgfsetlinewidth{0.000000pt}%
\definecolor{currentstroke}{rgb}{0.000000,0.000000,0.000000}%
\pgfsetstrokecolor{currentstroke}%
\pgfsetdash{}{0pt}%
\pgfpathmoveto{\pgfqpoint{4.400408in}{2.186911in}}%
\pgfpathlineto{\pgfqpoint{4.414104in}{2.184282in}}%
\pgfpathlineto{\pgfqpoint{4.427806in}{2.181680in}}%
\pgfpathlineto{\pgfqpoint{4.441515in}{2.179103in}}%
\pgfpathlineto{\pgfqpoint{4.455232in}{2.176554in}}%
\pgfpathlineto{\pgfqpoint{4.447497in}{2.167993in}}%
\pgfpathlineto{\pgfqpoint{4.439758in}{2.159411in}}%
\pgfpathlineto{\pgfqpoint{4.432012in}{2.150809in}}%
\pgfpathlineto{\pgfqpoint{4.424261in}{2.142189in}}%
\pgfpathlineto{\pgfqpoint{4.410533in}{2.144834in}}%
\pgfpathlineto{\pgfqpoint{4.396812in}{2.147505in}}%
\pgfpathlineto{\pgfqpoint{4.383099in}{2.150203in}}%
\pgfpathlineto{\pgfqpoint{4.369392in}{2.152927in}}%
\pgfpathlineto{\pgfqpoint{4.377154in}{2.161447in}}%
\pgfpathlineto{\pgfqpoint{4.384911in}{2.169952in}}%
\pgfpathlineto{\pgfqpoint{4.392663in}{2.178441in}}%
\pgfpathlineto{\pgfqpoint{4.400408in}{2.186911in}}%
\pgfpathclose%
\pgfusepath{fill}%
\end{pgfscope}%
\begin{pgfscope}%
\pgfpathrectangle{\pgfqpoint{1.150000in}{0.150000in}}{\pgfqpoint{5.700000in}{5.700000in}}%
\pgfusepath{clip}%
\pgfsetbuttcap%
\pgfsetroundjoin%
\definecolor{currentfill}{rgb}{0.283072,0.130895,0.449241}%
\pgfsetfillcolor{currentfill}%
\pgfsetfillopacity{0.700000}%
\pgfsetlinewidth{0.000000pt}%
\definecolor{currentstroke}{rgb}{0.000000,0.000000,0.000000}%
\pgfsetstrokecolor{currentstroke}%
\pgfsetdash{}{0pt}%
\pgfpathmoveto{\pgfqpoint{5.250676in}{2.362018in}}%
\pgfpathlineto{\pgfqpoint{5.264613in}{2.360790in}}%
\pgfpathlineto{\pgfqpoint{5.278558in}{2.359588in}}%
\pgfpathlineto{\pgfqpoint{5.292511in}{2.358410in}}%
\pgfpathlineto{\pgfqpoint{5.306473in}{2.357257in}}%
\pgfpathlineto{\pgfqpoint{5.299073in}{2.350096in}}%
\pgfpathlineto{\pgfqpoint{5.291666in}{2.342861in}}%
\pgfpathlineto{\pgfqpoint{5.284251in}{2.335550in}}%
\pgfpathlineto{\pgfqpoint{5.276829in}{2.328162in}}%
\pgfpathlineto{\pgfqpoint{5.262854in}{2.329302in}}%
\pgfpathlineto{\pgfqpoint{5.248887in}{2.330467in}}%
\pgfpathlineto{\pgfqpoint{5.234928in}{2.331657in}}%
\pgfpathlineto{\pgfqpoint{5.220978in}{2.332872in}}%
\pgfpathlineto{\pgfqpoint{5.228413in}{2.340267in}}%
\pgfpathlineto{\pgfqpoint{5.235841in}{2.347589in}}%
\pgfpathlineto{\pgfqpoint{5.243262in}{2.354839in}}%
\pgfpathlineto{\pgfqpoint{5.250676in}{2.362018in}}%
\pgfpathclose%
\pgfusepath{fill}%
\end{pgfscope}%
\begin{pgfscope}%
\pgfpathrectangle{\pgfqpoint{1.150000in}{0.150000in}}{\pgfqpoint{5.700000in}{5.700000in}}%
\pgfusepath{clip}%
\pgfsetbuttcap%
\pgfsetroundjoin%
\definecolor{currentfill}{rgb}{0.267004,0.004874,0.329415}%
\pgfsetfillcolor{currentfill}%
\pgfsetfillopacity{0.700000}%
\pgfsetlinewidth{0.000000pt}%
\definecolor{currentstroke}{rgb}{0.000000,0.000000,0.000000}%
\pgfsetstrokecolor{currentstroke}%
\pgfsetdash{}{0pt}%
\pgfpathmoveto{\pgfqpoint{3.947992in}{2.132710in}}%
\pgfpathlineto{\pgfqpoint{3.961578in}{2.128965in}}%
\pgfpathlineto{\pgfqpoint{3.975170in}{2.125248in}}%
\pgfpathlineto{\pgfqpoint{3.988768in}{2.121559in}}%
\pgfpathlineto{\pgfqpoint{4.002373in}{2.117898in}}%
\pgfpathlineto{\pgfqpoint{3.994476in}{2.110139in}}%
\pgfpathlineto{\pgfqpoint{3.986573in}{2.102422in}}%
\pgfpathlineto{\pgfqpoint{3.978664in}{2.094749in}}%
\pgfpathlineto{\pgfqpoint{3.970749in}{2.087124in}}%
\pgfpathlineto{\pgfqpoint{3.957131in}{2.090933in}}%
\pgfpathlineto{\pgfqpoint{3.943519in}{2.094770in}}%
\pgfpathlineto{\pgfqpoint{3.929913in}{2.098635in}}%
\pgfpathlineto{\pgfqpoint{3.916314in}{2.102529in}}%
\pgfpathlineto{\pgfqpoint{3.924243in}{2.110000in}}%
\pgfpathlineto{\pgfqpoint{3.932165in}{2.117524in}}%
\pgfpathlineto{\pgfqpoint{3.940081in}{2.125095in}}%
\pgfpathlineto{\pgfqpoint{3.947992in}{2.132710in}}%
\pgfpathclose%
\pgfusepath{fill}%
\end{pgfscope}%
\begin{pgfscope}%
\pgfpathrectangle{\pgfqpoint{1.150000in}{0.150000in}}{\pgfqpoint{5.700000in}{5.700000in}}%
\pgfusepath{clip}%
\pgfsetbuttcap%
\pgfsetroundjoin%
\definecolor{currentfill}{rgb}{0.282327,0.094955,0.417331}%
\pgfsetfillcolor{currentfill}%
\pgfsetfillopacity{0.700000}%
\pgfsetlinewidth{0.000000pt}%
\definecolor{currentstroke}{rgb}{0.000000,0.000000,0.000000}%
\pgfsetstrokecolor{currentstroke}%
\pgfsetdash{}{0pt}%
\pgfpathmoveto{\pgfqpoint{4.938739in}{2.294092in}}%
\pgfpathlineto{\pgfqpoint{4.952586in}{2.292461in}}%
\pgfpathlineto{\pgfqpoint{4.966440in}{2.290856in}}%
\pgfpathlineto{\pgfqpoint{4.980303in}{2.289275in}}%
\pgfpathlineto{\pgfqpoint{4.994173in}{2.287720in}}%
\pgfpathlineto{\pgfqpoint{4.986639in}{2.279713in}}%
\pgfpathlineto{\pgfqpoint{4.979098in}{2.271641in}}%
\pgfpathlineto{\pgfqpoint{4.971550in}{2.263503in}}%
\pgfpathlineto{\pgfqpoint{4.963996in}{2.255299in}}%
\pgfpathlineto{\pgfqpoint{4.950114in}{2.256882in}}%
\pgfpathlineto{\pgfqpoint{4.936240in}{2.258491in}}%
\pgfpathlineto{\pgfqpoint{4.922374in}{2.260124in}}%
\pgfpathlineto{\pgfqpoint{4.908516in}{2.261784in}}%
\pgfpathlineto{\pgfqpoint{4.916081in}{2.269955in}}%
\pgfpathlineto{\pgfqpoint{4.923640in}{2.278063in}}%
\pgfpathlineto{\pgfqpoint{4.931193in}{2.286109in}}%
\pgfpathlineto{\pgfqpoint{4.938739in}{2.294092in}}%
\pgfpathclose%
\pgfusepath{fill}%
\end{pgfscope}%
\begin{pgfscope}%
\pgfpathrectangle{\pgfqpoint{1.150000in}{0.150000in}}{\pgfqpoint{5.700000in}{5.700000in}}%
\pgfusepath{clip}%
\pgfsetbuttcap%
\pgfsetroundjoin%
\definecolor{currentfill}{rgb}{0.282910,0.105393,0.426902}%
\pgfsetfillcolor{currentfill}%
\pgfsetfillopacity{0.700000}%
\pgfsetlinewidth{0.000000pt}%
\definecolor{currentstroke}{rgb}{0.000000,0.000000,0.000000}%
\pgfsetstrokecolor{currentstroke}%
\pgfsetdash{}{0pt}%
\pgfpathmoveto{\pgfqpoint{2.911262in}{2.310516in}}%
\pgfpathlineto{\pgfqpoint{2.924684in}{2.303459in}}%
\pgfpathlineto{\pgfqpoint{2.938109in}{2.296442in}}%
\pgfpathlineto{\pgfqpoint{2.951538in}{2.289466in}}%
\pgfpathlineto{\pgfqpoint{2.964971in}{2.282528in}}%
\pgfpathlineto{\pgfqpoint{2.956557in}{2.281092in}}%
\pgfpathlineto{\pgfqpoint{2.948130in}{2.279881in}}%
\pgfpathlineto{\pgfqpoint{2.939689in}{2.278901in}}%
\pgfpathlineto{\pgfqpoint{2.931235in}{2.278159in}}%
\pgfpathlineto{\pgfqpoint{2.917777in}{2.285339in}}%
\pgfpathlineto{\pgfqpoint{2.904322in}{2.292560in}}%
\pgfpathlineto{\pgfqpoint{2.890870in}{2.299820in}}%
\pgfpathlineto{\pgfqpoint{2.877421in}{2.307120in}}%
\pgfpathlineto{\pgfqpoint{2.885902in}{2.307614in}}%
\pgfpathlineto{\pgfqpoint{2.894369in}{2.308349in}}%
\pgfpathlineto{\pgfqpoint{2.902823in}{2.309318in}}%
\pgfpathlineto{\pgfqpoint{2.911262in}{2.310516in}}%
\pgfpathclose%
\pgfusepath{fill}%
\end{pgfscope}%
\begin{pgfscope}%
\pgfpathrectangle{\pgfqpoint{1.150000in}{0.150000in}}{\pgfqpoint{5.700000in}{5.700000in}}%
\pgfusepath{clip}%
\pgfsetbuttcap%
\pgfsetroundjoin%
\definecolor{currentfill}{rgb}{0.277941,0.056324,0.381191}%
\pgfsetfillcolor{currentfill}%
\pgfsetfillopacity{0.700000}%
\pgfsetlinewidth{0.000000pt}%
\definecolor{currentstroke}{rgb}{0.000000,0.000000,0.000000}%
\pgfsetstrokecolor{currentstroke}%
\pgfsetdash{}{0pt}%
\pgfpathmoveto{\pgfqpoint{4.626713in}{2.225787in}}%
\pgfpathlineto{\pgfqpoint{4.640472in}{2.223623in}}%
\pgfpathlineto{\pgfqpoint{4.654239in}{2.221484in}}%
\pgfpathlineto{\pgfqpoint{4.668013in}{2.219371in}}%
\pgfpathlineto{\pgfqpoint{4.681794in}{2.217283in}}%
\pgfpathlineto{\pgfqpoint{4.674139in}{2.208768in}}%
\pgfpathlineto{\pgfqpoint{4.666479in}{2.200208in}}%
\pgfpathlineto{\pgfqpoint{4.658812in}{2.191605in}}%
\pgfpathlineto{\pgfqpoint{4.651139in}{2.182960in}}%
\pgfpathlineto{\pgfqpoint{4.637347in}{2.185116in}}%
\pgfpathlineto{\pgfqpoint{4.623562in}{2.187298in}}%
\pgfpathlineto{\pgfqpoint{4.609784in}{2.189505in}}%
\pgfpathlineto{\pgfqpoint{4.596013in}{2.191738in}}%
\pgfpathlineto{\pgfqpoint{4.603697in}{2.200310in}}%
\pgfpathlineto{\pgfqpoint{4.611375in}{2.208843in}}%
\pgfpathlineto{\pgfqpoint{4.619047in}{2.217336in}}%
\pgfpathlineto{\pgfqpoint{4.626713in}{2.225787in}}%
\pgfpathclose%
\pgfusepath{fill}%
\end{pgfscope}%
\begin{pgfscope}%
\pgfpathrectangle{\pgfqpoint{1.150000in}{0.150000in}}{\pgfqpoint{5.700000in}{5.700000in}}%
\pgfusepath{clip}%
\pgfsetbuttcap%
\pgfsetroundjoin%
\definecolor{currentfill}{rgb}{0.279566,0.067836,0.391917}%
\pgfsetfillcolor{currentfill}%
\pgfsetfillopacity{0.700000}%
\pgfsetlinewidth{0.000000pt}%
\definecolor{currentstroke}{rgb}{0.000000,0.000000,0.000000}%
\pgfsetstrokecolor{currentstroke}%
\pgfsetdash{}{0pt}%
\pgfpathmoveto{\pgfqpoint{3.105900in}{2.238129in}}%
\pgfpathlineto{\pgfqpoint{3.119343in}{2.231764in}}%
\pgfpathlineto{\pgfqpoint{3.132790in}{2.225435in}}%
\pgfpathlineto{\pgfqpoint{3.146241in}{2.219143in}}%
\pgfpathlineto{\pgfqpoint{3.159697in}{2.212887in}}%
\pgfpathlineto{\pgfqpoint{3.151403in}{2.209939in}}%
\pgfpathlineto{\pgfqpoint{3.143098in}{2.207182in}}%
\pgfpathlineto{\pgfqpoint{3.134782in}{2.204621in}}%
\pgfpathlineto{\pgfqpoint{3.126454in}{2.202263in}}%
\pgfpathlineto{\pgfqpoint{3.112975in}{2.208748in}}%
\pgfpathlineto{\pgfqpoint{3.099501in}{2.215269in}}%
\pgfpathlineto{\pgfqpoint{3.086030in}{2.221827in}}%
\pgfpathlineto{\pgfqpoint{3.072564in}{2.228421in}}%
\pgfpathlineto{\pgfqpoint{3.080916in}{2.230545in}}%
\pgfpathlineto{\pgfqpoint{3.089255in}{2.232875in}}%
\pgfpathlineto{\pgfqpoint{3.097584in}{2.235405in}}%
\pgfpathlineto{\pgfqpoint{3.105900in}{2.238129in}}%
\pgfpathclose%
\pgfusepath{fill}%
\end{pgfscope}%
\begin{pgfscope}%
\pgfpathrectangle{\pgfqpoint{1.150000in}{0.150000in}}{\pgfqpoint{5.700000in}{5.700000in}}%
\pgfusepath{clip}%
\pgfsetbuttcap%
\pgfsetroundjoin%
\definecolor{currentfill}{rgb}{0.281412,0.155834,0.469201}%
\pgfsetfillcolor{currentfill}%
\pgfsetfillopacity{0.700000}%
\pgfsetlinewidth{0.000000pt}%
\definecolor{currentstroke}{rgb}{0.000000,0.000000,0.000000}%
\pgfsetstrokecolor{currentstroke}%
\pgfsetdash{}{0pt}%
\pgfpathmoveto{\pgfqpoint{5.477177in}{2.403135in}}%
\pgfpathlineto{\pgfqpoint{5.491185in}{2.402137in}}%
\pgfpathlineto{\pgfqpoint{5.505202in}{2.401165in}}%
\pgfpathlineto{\pgfqpoint{5.519228in}{2.400216in}}%
\pgfpathlineto{\pgfqpoint{5.533262in}{2.399293in}}%
\pgfpathlineto{\pgfqpoint{5.525965in}{2.392820in}}%
\pgfpathlineto{\pgfqpoint{5.518660in}{2.386273in}}%
\pgfpathlineto{\pgfqpoint{5.511348in}{2.379650in}}%
\pgfpathlineto{\pgfqpoint{5.504029in}{2.372950in}}%
\pgfpathlineto{\pgfqpoint{5.489979in}{2.373833in}}%
\pgfpathlineto{\pgfqpoint{5.475939in}{2.374741in}}%
\pgfpathlineto{\pgfqpoint{5.461907in}{2.375674in}}%
\pgfpathlineto{\pgfqpoint{5.447884in}{2.376631in}}%
\pgfpathlineto{\pgfqpoint{5.455218in}{2.383366in}}%
\pgfpathlineto{\pgfqpoint{5.462545in}{2.390027in}}%
\pgfpathlineto{\pgfqpoint{5.469865in}{2.396616in}}%
\pgfpathlineto{\pgfqpoint{5.477177in}{2.403135in}}%
\pgfpathclose%
\pgfusepath{fill}%
\end{pgfscope}%
\begin{pgfscope}%
\pgfpathrectangle{\pgfqpoint{1.150000in}{0.150000in}}{\pgfqpoint{5.700000in}{5.700000in}}%
\pgfusepath{clip}%
\pgfsetbuttcap%
\pgfsetroundjoin%
\definecolor{currentfill}{rgb}{0.269308,0.218818,0.509577}%
\pgfsetfillcolor{currentfill}%
\pgfsetfillopacity{0.700000}%
\pgfsetlinewidth{0.000000pt}%
\definecolor{currentstroke}{rgb}{0.000000,0.000000,0.000000}%
\pgfsetstrokecolor{currentstroke}%
\pgfsetdash{}{0pt}%
\pgfpathmoveto{\pgfqpoint{2.467132in}{2.537303in}}%
\pgfpathlineto{\pgfqpoint{2.480537in}{2.528449in}}%
\pgfpathlineto{\pgfqpoint{2.493944in}{2.519647in}}%
\pgfpathlineto{\pgfqpoint{2.507353in}{2.510898in}}%
\pgfpathlineto{\pgfqpoint{2.520764in}{2.502201in}}%
\pgfpathlineto{\pgfqpoint{2.512022in}{2.504619in}}%
\pgfpathlineto{\pgfqpoint{2.503260in}{2.507342in}}%
\pgfpathlineto{\pgfqpoint{2.494479in}{2.510375in}}%
\pgfpathlineto{\pgfqpoint{2.485679in}{2.513726in}}%
\pgfpathlineto{\pgfqpoint{2.472235in}{2.522698in}}%
\pgfpathlineto{\pgfqpoint{2.458792in}{2.531722in}}%
\pgfpathlineto{\pgfqpoint{2.445350in}{2.540799in}}%
\pgfpathlineto{\pgfqpoint{2.431911in}{2.549928in}}%
\pgfpathlineto{\pgfqpoint{2.440746in}{2.546296in}}%
\pgfpathlineto{\pgfqpoint{2.449561in}{2.542986in}}%
\pgfpathlineto{\pgfqpoint{2.458356in}{2.539991in}}%
\pgfpathlineto{\pgfqpoint{2.467132in}{2.537303in}}%
\pgfpathclose%
\pgfusepath{fill}%
\end{pgfscope}%
\begin{pgfscope}%
\pgfpathrectangle{\pgfqpoint{1.150000in}{0.150000in}}{\pgfqpoint{5.700000in}{5.700000in}}%
\pgfusepath{clip}%
\pgfsetbuttcap%
\pgfsetroundjoin%
\definecolor{currentfill}{rgb}{0.281887,0.150881,0.465405}%
\pgfsetfillcolor{currentfill}%
\pgfsetfillopacity{0.700000}%
\pgfsetlinewidth{0.000000pt}%
\definecolor{currentstroke}{rgb}{0.000000,0.000000,0.000000}%
\pgfsetstrokecolor{currentstroke}%
\pgfsetdash{}{0pt}%
\pgfpathmoveto{\pgfqpoint{2.716288in}{2.398004in}}%
\pgfpathlineto{\pgfqpoint{2.729699in}{2.390191in}}%
\pgfpathlineto{\pgfqpoint{2.743113in}{2.382423in}}%
\pgfpathlineto{\pgfqpoint{2.756530in}{2.374699in}}%
\pgfpathlineto{\pgfqpoint{2.769951in}{2.367020in}}%
\pgfpathlineto{\pgfqpoint{2.761399in}{2.367279in}}%
\pgfpathlineto{\pgfqpoint{2.752831in}{2.367800in}}%
\pgfpathlineto{\pgfqpoint{2.744248in}{2.368589in}}%
\pgfpathlineto{\pgfqpoint{2.735649in}{2.369653in}}%
\pgfpathlineto{\pgfqpoint{2.722200in}{2.377591in}}%
\pgfpathlineto{\pgfqpoint{2.708753in}{2.385573in}}%
\pgfpathlineto{\pgfqpoint{2.695309in}{2.393600in}}%
\pgfpathlineto{\pgfqpoint{2.681868in}{2.401672in}}%
\pgfpathlineto{\pgfqpoint{2.690497in}{2.400344in}}%
\pgfpathlineto{\pgfqpoint{2.699110in}{2.399295in}}%
\pgfpathlineto{\pgfqpoint{2.707707in}{2.398517in}}%
\pgfpathlineto{\pgfqpoint{2.716288in}{2.398004in}}%
\pgfpathclose%
\pgfusepath{fill}%
\end{pgfscope}%
\begin{pgfscope}%
\pgfpathrectangle{\pgfqpoint{1.150000in}{0.150000in}}{\pgfqpoint{5.700000in}{5.700000in}}%
\pgfusepath{clip}%
\pgfsetbuttcap%
\pgfsetroundjoin%
\definecolor{currentfill}{rgb}{0.268510,0.009605,0.335427}%
\pgfsetfillcolor{currentfill}%
\pgfsetfillopacity{0.700000}%
\pgfsetlinewidth{0.000000pt}%
\definecolor{currentstroke}{rgb}{0.000000,0.000000,0.000000}%
\pgfsetstrokecolor{currentstroke}%
\pgfsetdash{}{0pt}%
\pgfpathmoveto{\pgfqpoint{3.581290in}{2.135773in}}%
\pgfpathlineto{\pgfqpoint{3.594806in}{2.130953in}}%
\pgfpathlineto{\pgfqpoint{3.608327in}{2.126163in}}%
\pgfpathlineto{\pgfqpoint{3.621854in}{2.121404in}}%
\pgfpathlineto{\pgfqpoint{3.635386in}{2.116676in}}%
\pgfpathlineto{\pgfqpoint{3.627337in}{2.110574in}}%
\pgfpathlineto{\pgfqpoint{3.619280in}{2.104578in}}%
\pgfpathlineto{\pgfqpoint{3.611215in}{2.098692in}}%
\pgfpathlineto{\pgfqpoint{3.603143in}{2.092920in}}%
\pgfpathlineto{\pgfqpoint{3.589594in}{2.097837in}}%
\pgfpathlineto{\pgfqpoint{3.576050in}{2.102783in}}%
\pgfpathlineto{\pgfqpoint{3.562512in}{2.107761in}}%
\pgfpathlineto{\pgfqpoint{3.548978in}{2.112769in}}%
\pgfpathlineto{\pgfqpoint{3.557068in}{2.118348in}}%
\pgfpathlineto{\pgfqpoint{3.565150in}{2.124044in}}%
\pgfpathlineto{\pgfqpoint{3.573224in}{2.129854in}}%
\pgfpathlineto{\pgfqpoint{3.581290in}{2.135773in}}%
\pgfpathclose%
\pgfusepath{fill}%
\end{pgfscope}%
\begin{pgfscope}%
\pgfpathrectangle{\pgfqpoint{1.150000in}{0.150000in}}{\pgfqpoint{5.700000in}{5.700000in}}%
\pgfusepath{clip}%
\pgfsetbuttcap%
\pgfsetroundjoin%
\definecolor{currentfill}{rgb}{0.283229,0.120777,0.440584}%
\pgfsetfillcolor{currentfill}%
\pgfsetfillopacity{0.700000}%
\pgfsetlinewidth{0.000000pt}%
\definecolor{currentstroke}{rgb}{0.000000,0.000000,0.000000}%
\pgfsetstrokecolor{currentstroke}%
\pgfsetdash{}{0pt}%
\pgfpathmoveto{\pgfqpoint{5.165259in}{2.337980in}}%
\pgfpathlineto{\pgfqpoint{5.179176in}{2.336666in}}%
\pgfpathlineto{\pgfqpoint{5.193102in}{2.335376in}}%
\pgfpathlineto{\pgfqpoint{5.207036in}{2.334112in}}%
\pgfpathlineto{\pgfqpoint{5.220978in}{2.332872in}}%
\pgfpathlineto{\pgfqpoint{5.213536in}{2.325403in}}%
\pgfpathlineto{\pgfqpoint{5.206087in}{2.317860in}}%
\pgfpathlineto{\pgfqpoint{5.198630in}{2.310242in}}%
\pgfpathlineto{\pgfqpoint{5.191167in}{2.302548in}}%
\pgfpathlineto{\pgfqpoint{5.177212in}{2.303788in}}%
\pgfpathlineto{\pgfqpoint{5.163265in}{2.305054in}}%
\pgfpathlineto{\pgfqpoint{5.149326in}{2.306344in}}%
\pgfpathlineto{\pgfqpoint{5.135396in}{2.307659in}}%
\pgfpathlineto{\pgfqpoint{5.142872in}{2.315347in}}%
\pgfpathlineto{\pgfqpoint{5.150341in}{2.322963in}}%
\pgfpathlineto{\pgfqpoint{5.157803in}{2.330507in}}%
\pgfpathlineto{\pgfqpoint{5.165259in}{2.337980in}}%
\pgfpathclose%
\pgfusepath{fill}%
\end{pgfscope}%
\begin{pgfscope}%
\pgfpathrectangle{\pgfqpoint{1.150000in}{0.150000in}}{\pgfqpoint{5.700000in}{5.700000in}}%
\pgfusepath{clip}%
\pgfsetbuttcap%
\pgfsetroundjoin%
\definecolor{currentfill}{rgb}{0.271305,0.019942,0.347269}%
\pgfsetfillcolor{currentfill}%
\pgfsetfillopacity{0.700000}%
\pgfsetlinewidth{0.000000pt}%
\definecolor{currentstroke}{rgb}{0.000000,0.000000,0.000000}%
\pgfsetstrokecolor{currentstroke}%
\pgfsetdash{}{0pt}%
\pgfpathmoveto{\pgfqpoint{3.440897in}{2.153965in}}%
\pgfpathlineto{\pgfqpoint{3.454390in}{2.148704in}}%
\pgfpathlineto{\pgfqpoint{3.467887in}{2.143475in}}%
\pgfpathlineto{\pgfqpoint{3.481390in}{2.138279in}}%
\pgfpathlineto{\pgfqpoint{3.494897in}{2.133114in}}%
\pgfpathlineto{\pgfqpoint{3.486782in}{2.127854in}}%
\pgfpathlineto{\pgfqpoint{3.478658in}{2.122726in}}%
\pgfpathlineto{\pgfqpoint{3.470525in}{2.117734in}}%
\pgfpathlineto{\pgfqpoint{3.462384in}{2.112883in}}%
\pgfpathlineto{\pgfqpoint{3.448858in}{2.118249in}}%
\pgfpathlineto{\pgfqpoint{3.435337in}{2.123647in}}%
\pgfpathlineto{\pgfqpoint{3.421820in}{2.129077in}}%
\pgfpathlineto{\pgfqpoint{3.408309in}{2.134539in}}%
\pgfpathlineto{\pgfqpoint{3.416469in}{2.139184in}}%
\pgfpathlineto{\pgfqpoint{3.424621in}{2.143973in}}%
\pgfpathlineto{\pgfqpoint{3.432763in}{2.148902in}}%
\pgfpathlineto{\pgfqpoint{3.440897in}{2.153965in}}%
\pgfpathclose%
\pgfusepath{fill}%
\end{pgfscope}%
\begin{pgfscope}%
\pgfpathrectangle{\pgfqpoint{1.150000in}{0.150000in}}{\pgfqpoint{5.700000in}{5.700000in}}%
\pgfusepath{clip}%
\pgfsetbuttcap%
\pgfsetroundjoin%
\definecolor{currentfill}{rgb}{0.272594,0.025563,0.353093}%
\pgfsetfillcolor{currentfill}%
\pgfsetfillopacity{0.700000}%
\pgfsetlinewidth{0.000000pt}%
\definecolor{currentstroke}{rgb}{0.000000,0.000000,0.000000}%
\pgfsetstrokecolor{currentstroke}%
\pgfsetdash{}{0pt}%
\pgfpathmoveto{\pgfqpoint{4.314634in}{2.164090in}}%
\pgfpathlineto{\pgfqpoint{4.328313in}{2.161259in}}%
\pgfpathlineto{\pgfqpoint{4.341999in}{2.158455in}}%
\pgfpathlineto{\pgfqpoint{4.355692in}{2.155678in}}%
\pgfpathlineto{\pgfqpoint{4.369392in}{2.152927in}}%
\pgfpathlineto{\pgfqpoint{4.361624in}{2.144395in}}%
\pgfpathlineto{\pgfqpoint{4.353850in}{2.135853in}}%
\pgfpathlineto{\pgfqpoint{4.346071in}{2.127304in}}%
\pgfpathlineto{\pgfqpoint{4.338286in}{2.118749in}}%
\pgfpathlineto{\pgfqpoint{4.324575in}{2.121608in}}%
\pgfpathlineto{\pgfqpoint{4.310871in}{2.124494in}}%
\pgfpathlineto{\pgfqpoint{4.297173in}{2.127407in}}%
\pgfpathlineto{\pgfqpoint{4.283483in}{2.130346in}}%
\pgfpathlineto{\pgfqpoint{4.291279in}{2.138787in}}%
\pgfpathlineto{\pgfqpoint{4.299070in}{2.147226in}}%
\pgfpathlineto{\pgfqpoint{4.306855in}{2.155661in}}%
\pgfpathlineto{\pgfqpoint{4.314634in}{2.164090in}}%
\pgfpathclose%
\pgfusepath{fill}%
\end{pgfscope}%
\begin{pgfscope}%
\pgfpathrectangle{\pgfqpoint{1.150000in}{0.150000in}}{\pgfqpoint{5.700000in}{5.700000in}}%
\pgfusepath{clip}%
\pgfsetbuttcap%
\pgfsetroundjoin%
\definecolor{currentfill}{rgb}{0.268510,0.009605,0.335427}%
\pgfsetfillcolor{currentfill}%
\pgfsetfillopacity{0.700000}%
\pgfsetlinewidth{0.000000pt}%
\definecolor{currentstroke}{rgb}{0.000000,0.000000,0.000000}%
\pgfsetstrokecolor{currentstroke}%
\pgfsetdash{}{0pt}%
\pgfpathmoveto{\pgfqpoint{4.088331in}{2.135453in}}%
\pgfpathlineto{\pgfqpoint{4.101955in}{2.132066in}}%
\pgfpathlineto{\pgfqpoint{4.115584in}{2.128707in}}%
\pgfpathlineto{\pgfqpoint{4.129221in}{2.125376in}}%
\pgfpathlineto{\pgfqpoint{4.142863in}{2.122071in}}%
\pgfpathlineto{\pgfqpoint{4.135015in}{2.113921in}}%
\pgfpathlineto{\pgfqpoint{4.127161in}{2.105793in}}%
\pgfpathlineto{\pgfqpoint{4.119301in}{2.097689in}}%
\pgfpathlineto{\pgfqpoint{4.111436in}{2.089613in}}%
\pgfpathlineto{\pgfqpoint{4.097780in}{2.093052in}}%
\pgfpathlineto{\pgfqpoint{4.084132in}{2.096519in}}%
\pgfpathlineto{\pgfqpoint{4.070489in}{2.100013in}}%
\pgfpathlineto{\pgfqpoint{4.056853in}{2.103534in}}%
\pgfpathlineto{\pgfqpoint{4.064732in}{2.111470in}}%
\pgfpathlineto{\pgfqpoint{4.072604in}{2.119438in}}%
\pgfpathlineto{\pgfqpoint{4.080470in}{2.127433in}}%
\pgfpathlineto{\pgfqpoint{4.088331in}{2.135453in}}%
\pgfpathclose%
\pgfusepath{fill}%
\end{pgfscope}%
\begin{pgfscope}%
\pgfpathrectangle{\pgfqpoint{1.150000in}{0.150000in}}{\pgfqpoint{5.700000in}{5.700000in}}%
\pgfusepath{clip}%
\pgfsetbuttcap%
\pgfsetroundjoin%
\definecolor{currentfill}{rgb}{0.278826,0.175490,0.483397}%
\pgfsetfillcolor{currentfill}%
\pgfsetfillopacity{0.700000}%
\pgfsetlinewidth{0.000000pt}%
\definecolor{currentstroke}{rgb}{0.000000,0.000000,0.000000}%
\pgfsetstrokecolor{currentstroke}%
\pgfsetdash{}{0pt}%
\pgfpathmoveto{\pgfqpoint{5.703686in}{2.441171in}}%
\pgfpathlineto{\pgfqpoint{5.717765in}{2.440346in}}%
\pgfpathlineto{\pgfqpoint{5.731852in}{2.439546in}}%
\pgfpathlineto{\pgfqpoint{5.745949in}{2.438770in}}%
\pgfpathlineto{\pgfqpoint{5.760054in}{2.438019in}}%
\pgfpathlineto{\pgfqpoint{5.752869in}{2.432284in}}%
\pgfpathlineto{\pgfqpoint{5.745675in}{2.426481in}}%
\pgfpathlineto{\pgfqpoint{5.738474in}{2.420609in}}%
\pgfpathlineto{\pgfqpoint{5.731265in}{2.414665in}}%
\pgfpathlineto{\pgfqpoint{5.717142in}{2.415349in}}%
\pgfpathlineto{\pgfqpoint{5.703029in}{2.416057in}}%
\pgfpathlineto{\pgfqpoint{5.688924in}{2.416789in}}%
\pgfpathlineto{\pgfqpoint{5.674828in}{2.417546in}}%
\pgfpathlineto{\pgfqpoint{5.682054in}{2.423553in}}%
\pgfpathlineto{\pgfqpoint{5.689272in}{2.429491in}}%
\pgfpathlineto{\pgfqpoint{5.696483in}{2.435363in}}%
\pgfpathlineto{\pgfqpoint{5.703686in}{2.441171in}}%
\pgfpathclose%
\pgfusepath{fill}%
\end{pgfscope}%
\begin{pgfscope}%
\pgfpathrectangle{\pgfqpoint{1.150000in}{0.150000in}}{\pgfqpoint{5.700000in}{5.700000in}}%
\pgfusepath{clip}%
\pgfsetbuttcap%
\pgfsetroundjoin%
\definecolor{currentfill}{rgb}{0.267004,0.004874,0.329415}%
\pgfsetfillcolor{currentfill}%
\pgfsetfillopacity{0.700000}%
\pgfsetlinewidth{0.000000pt}%
\definecolor{currentstroke}{rgb}{0.000000,0.000000,0.000000}%
\pgfsetstrokecolor{currentstroke}%
\pgfsetdash{}{0pt}%
\pgfpathmoveto{\pgfqpoint{3.721631in}{2.124136in}}%
\pgfpathlineto{\pgfqpoint{3.735175in}{2.119733in}}%
\pgfpathlineto{\pgfqpoint{3.748724in}{2.115360in}}%
\pgfpathlineto{\pgfqpoint{3.762280in}{2.111016in}}%
\pgfpathlineto{\pgfqpoint{3.775841in}{2.106701in}}%
\pgfpathlineto{\pgfqpoint{3.767851in}{2.099883in}}%
\pgfpathlineto{\pgfqpoint{3.759854in}{2.093145in}}%
\pgfpathlineto{\pgfqpoint{3.751851in}{2.086493in}}%
\pgfpathlineto{\pgfqpoint{3.743841in}{2.079932in}}%
\pgfpathlineto{\pgfqpoint{3.730265in}{2.084421in}}%
\pgfpathlineto{\pgfqpoint{3.716694in}{2.088939in}}%
\pgfpathlineto{\pgfqpoint{3.703129in}{2.093487in}}%
\pgfpathlineto{\pgfqpoint{3.689569in}{2.098065in}}%
\pgfpathlineto{\pgfqpoint{3.697595in}{2.104447in}}%
\pgfpathlineto{\pgfqpoint{3.705614in}{2.110923in}}%
\pgfpathlineto{\pgfqpoint{3.713626in}{2.117487in}}%
\pgfpathlineto{\pgfqpoint{3.721631in}{2.124136in}}%
\pgfpathclose%
\pgfusepath{fill}%
\end{pgfscope}%
\begin{pgfscope}%
\pgfpathrectangle{\pgfqpoint{1.150000in}{0.150000in}}{\pgfqpoint{5.700000in}{5.700000in}}%
\pgfusepath{clip}%
\pgfsetbuttcap%
\pgfsetroundjoin%
\definecolor{currentfill}{rgb}{0.281446,0.084320,0.407414}%
\pgfsetfillcolor{currentfill}%
\pgfsetfillopacity{0.700000}%
\pgfsetlinewidth{0.000000pt}%
\definecolor{currentstroke}{rgb}{0.000000,0.000000,0.000000}%
\pgfsetstrokecolor{currentstroke}%
\pgfsetdash{}{0pt}%
\pgfpathmoveto{\pgfqpoint{4.853160in}{2.268673in}}%
\pgfpathlineto{\pgfqpoint{4.866987in}{2.266913in}}%
\pgfpathlineto{\pgfqpoint{4.880822in}{2.265178in}}%
\pgfpathlineto{\pgfqpoint{4.894665in}{2.263468in}}%
\pgfpathlineto{\pgfqpoint{4.908516in}{2.261784in}}%
\pgfpathlineto{\pgfqpoint{4.900944in}{2.253551in}}%
\pgfpathlineto{\pgfqpoint{4.893365in}{2.245256in}}%
\pgfpathlineto{\pgfqpoint{4.885781in}{2.236900in}}%
\pgfpathlineto{\pgfqpoint{4.878190in}{2.228484in}}%
\pgfpathlineto{\pgfqpoint{4.864328in}{2.230210in}}%
\pgfpathlineto{\pgfqpoint{4.850474in}{2.231961in}}%
\pgfpathlineto{\pgfqpoint{4.836627in}{2.233738in}}%
\pgfpathlineto{\pgfqpoint{4.822789in}{2.235540in}}%
\pgfpathlineto{\pgfqpoint{4.830391in}{2.243910in}}%
\pgfpathlineto{\pgfqpoint{4.837987in}{2.252222in}}%
\pgfpathlineto{\pgfqpoint{4.845577in}{2.260477in}}%
\pgfpathlineto{\pgfqpoint{4.853160in}{2.268673in}}%
\pgfpathclose%
\pgfusepath{fill}%
\end{pgfscope}%
\begin{pgfscope}%
\pgfpathrectangle{\pgfqpoint{1.150000in}{0.150000in}}{\pgfqpoint{5.700000in}{5.700000in}}%
\pgfusepath{clip}%
\pgfsetbuttcap%
\pgfsetroundjoin%
\definecolor{currentfill}{rgb}{0.277018,0.050344,0.375715}%
\pgfsetfillcolor{currentfill}%
\pgfsetfillopacity{0.700000}%
\pgfsetlinewidth{0.000000pt}%
\definecolor{currentstroke}{rgb}{0.000000,0.000000,0.000000}%
\pgfsetstrokecolor{currentstroke}%
\pgfsetdash{}{0pt}%
\pgfpathmoveto{\pgfqpoint{4.541005in}{2.200931in}}%
\pgfpathlineto{\pgfqpoint{4.554746in}{2.198594in}}%
\pgfpathlineto{\pgfqpoint{4.568495in}{2.196283in}}%
\pgfpathlineto{\pgfqpoint{4.582250in}{2.193998in}}%
\pgfpathlineto{\pgfqpoint{4.596013in}{2.191738in}}%
\pgfpathlineto{\pgfqpoint{4.588324in}{2.183130in}}%
\pgfpathlineto{\pgfqpoint{4.580629in}{2.174485in}}%
\pgfpathlineto{\pgfqpoint{4.572928in}{2.165807in}}%
\pgfpathlineto{\pgfqpoint{4.565222in}{2.157097in}}%
\pgfpathlineto{\pgfqpoint{4.551448in}{2.159438in}}%
\pgfpathlineto{\pgfqpoint{4.537681in}{2.161805in}}%
\pgfpathlineto{\pgfqpoint{4.523921in}{2.164198in}}%
\pgfpathlineto{\pgfqpoint{4.510169in}{2.166616in}}%
\pgfpathlineto{\pgfqpoint{4.517887in}{2.175240in}}%
\pgfpathlineto{\pgfqpoint{4.525599in}{2.183835in}}%
\pgfpathlineto{\pgfqpoint{4.533305in}{2.192399in}}%
\pgfpathlineto{\pgfqpoint{4.541005in}{2.200931in}}%
\pgfpathclose%
\pgfusepath{fill}%
\end{pgfscope}%
\begin{pgfscope}%
\pgfpathrectangle{\pgfqpoint{1.150000in}{0.150000in}}{\pgfqpoint{5.700000in}{5.700000in}}%
\pgfusepath{clip}%
\pgfsetbuttcap%
\pgfsetroundjoin%
\definecolor{currentfill}{rgb}{0.275191,0.194905,0.496005}%
\pgfsetfillcolor{currentfill}%
\pgfsetfillopacity{0.700000}%
\pgfsetlinewidth{0.000000pt}%
\definecolor{currentstroke}{rgb}{0.000000,0.000000,0.000000}%
\pgfsetstrokecolor{currentstroke}%
\pgfsetdash{}{0pt}%
\pgfpathmoveto{\pgfqpoint{5.930131in}{2.475252in}}%
\pgfpathlineto{\pgfqpoint{5.944279in}{2.474542in}}%
\pgfpathlineto{\pgfqpoint{5.958436in}{2.473857in}}%
\pgfpathlineto{\pgfqpoint{5.972602in}{2.473196in}}%
\pgfpathlineto{\pgfqpoint{5.986777in}{2.472560in}}%
\pgfpathlineto{\pgfqpoint{5.979709in}{2.467565in}}%
\pgfpathlineto{\pgfqpoint{5.972634in}{2.462516in}}%
\pgfpathlineto{\pgfqpoint{5.965551in}{2.457409in}}%
\pgfpathlineto{\pgfqpoint{5.958460in}{2.452241in}}%
\pgfpathlineto{\pgfqpoint{5.944265in}{2.452782in}}%
\pgfpathlineto{\pgfqpoint{5.930080in}{2.453347in}}%
\pgfpathlineto{\pgfqpoint{5.915904in}{2.453937in}}%
\pgfpathlineto{\pgfqpoint{5.901737in}{2.454551in}}%
\pgfpathlineto{\pgfqpoint{5.908847in}{2.459809in}}%
\pgfpathlineto{\pgfqpoint{5.915949in}{2.465010in}}%
\pgfpathlineto{\pgfqpoint{5.923044in}{2.470156in}}%
\pgfpathlineto{\pgfqpoint{5.930131in}{2.475252in}}%
\pgfpathclose%
\pgfusepath{fill}%
\end{pgfscope}%
\begin{pgfscope}%
\pgfpathrectangle{\pgfqpoint{1.150000in}{0.150000in}}{\pgfqpoint{5.700000in}{5.700000in}}%
\pgfusepath{clip}%
\pgfsetbuttcap%
\pgfsetroundjoin%
\definecolor{currentfill}{rgb}{0.274952,0.037752,0.364543}%
\pgfsetfillcolor{currentfill}%
\pgfsetfillopacity{0.700000}%
\pgfsetlinewidth{0.000000pt}%
\definecolor{currentstroke}{rgb}{0.000000,0.000000,0.000000}%
\pgfsetstrokecolor{currentstroke}%
\pgfsetdash{}{0pt}%
\pgfpathmoveto{\pgfqpoint{3.300390in}{2.179419in}}%
\pgfpathlineto{\pgfqpoint{3.313863in}{2.173692in}}%
\pgfpathlineto{\pgfqpoint{3.327341in}{2.167999in}}%
\pgfpathlineto{\pgfqpoint{3.340824in}{2.162340in}}%
\pgfpathlineto{\pgfqpoint{3.354312in}{2.156714in}}%
\pgfpathlineto{\pgfqpoint{3.346123in}{2.152429in}}%
\pgfpathlineto{\pgfqpoint{3.337924in}{2.148302in}}%
\pgfpathlineto{\pgfqpoint{3.329716in}{2.144339in}}%
\pgfpathlineto{\pgfqpoint{3.321498in}{2.140546in}}%
\pgfpathlineto{\pgfqpoint{3.307990in}{2.146387in}}%
\pgfpathlineto{\pgfqpoint{3.294486in}{2.152261in}}%
\pgfpathlineto{\pgfqpoint{3.280987in}{2.158169in}}%
\pgfpathlineto{\pgfqpoint{3.267493in}{2.164111in}}%
\pgfpathlineto{\pgfqpoint{3.275732in}{2.167684in}}%
\pgfpathlineto{\pgfqpoint{3.283961in}{2.171430in}}%
\pgfpathlineto{\pgfqpoint{3.292180in}{2.175344in}}%
\pgfpathlineto{\pgfqpoint{3.300390in}{2.179419in}}%
\pgfpathclose%
\pgfusepath{fill}%
\end{pgfscope}%
\begin{pgfscope}%
\pgfpathrectangle{\pgfqpoint{1.150000in}{0.150000in}}{\pgfqpoint{5.700000in}{5.700000in}}%
\pgfusepath{clip}%
\pgfsetbuttcap%
\pgfsetroundjoin%
\definecolor{currentfill}{rgb}{0.267004,0.004874,0.329415}%
\pgfsetfillcolor{currentfill}%
\pgfsetfillopacity{0.700000}%
\pgfsetlinewidth{0.000000pt}%
\definecolor{currentstroke}{rgb}{0.000000,0.000000,0.000000}%
\pgfsetstrokecolor{currentstroke}%
\pgfsetdash{}{0pt}%
\pgfpathmoveto{\pgfqpoint{3.861976in}{2.118388in}}%
\pgfpathlineto{\pgfqpoint{3.875551in}{2.114380in}}%
\pgfpathlineto{\pgfqpoint{3.889133in}{2.110401in}}%
\pgfpathlineto{\pgfqpoint{3.902720in}{2.106451in}}%
\pgfpathlineto{\pgfqpoint{3.916314in}{2.102529in}}%
\pgfpathlineto{\pgfqpoint{3.908379in}{2.095112in}}%
\pgfpathlineto{\pgfqpoint{3.900438in}{2.087754in}}%
\pgfpathlineto{\pgfqpoint{3.892491in}{2.080459in}}%
\pgfpathlineto{\pgfqpoint{3.884537in}{2.073230in}}%
\pgfpathlineto{\pgfqpoint{3.870929in}{2.077314in}}%
\pgfpathlineto{\pgfqpoint{3.857328in}{2.081425in}}%
\pgfpathlineto{\pgfqpoint{3.843732in}{2.085566in}}%
\pgfpathlineto{\pgfqpoint{3.830142in}{2.089735in}}%
\pgfpathlineto{\pgfqpoint{3.838110in}{2.096797in}}%
\pgfpathlineto{\pgfqpoint{3.846072in}{2.103929in}}%
\pgfpathlineto{\pgfqpoint{3.854027in}{2.111128in}}%
\pgfpathlineto{\pgfqpoint{3.861976in}{2.118388in}}%
\pgfpathclose%
\pgfusepath{fill}%
\end{pgfscope}%
\begin{pgfscope}%
\pgfpathrectangle{\pgfqpoint{1.150000in}{0.150000in}}{\pgfqpoint{5.700000in}{5.700000in}}%
\pgfusepath{clip}%
\pgfsetbuttcap%
\pgfsetroundjoin%
\definecolor{currentfill}{rgb}{0.282290,0.145912,0.461510}%
\pgfsetfillcolor{currentfill}%
\pgfsetfillopacity{0.700000}%
\pgfsetlinewidth{0.000000pt}%
\definecolor{currentstroke}{rgb}{0.000000,0.000000,0.000000}%
\pgfsetstrokecolor{currentstroke}%
\pgfsetdash{}{0pt}%
\pgfpathmoveto{\pgfqpoint{5.391875in}{2.380706in}}%
\pgfpathlineto{\pgfqpoint{5.405865in}{2.379650in}}%
\pgfpathlineto{\pgfqpoint{5.419862in}{2.378619in}}%
\pgfpathlineto{\pgfqpoint{5.433869in}{2.377612in}}%
\pgfpathlineto{\pgfqpoint{5.447884in}{2.376631in}}%
\pgfpathlineto{\pgfqpoint{5.440542in}{2.369820in}}%
\pgfpathlineto{\pgfqpoint{5.433192in}{2.362933in}}%
\pgfpathlineto{\pgfqpoint{5.425835in}{2.355968in}}%
\pgfpathlineto{\pgfqpoint{5.418470in}{2.348924in}}%
\pgfpathlineto{\pgfqpoint{5.404441in}{2.349879in}}%
\pgfpathlineto{\pgfqpoint{5.390420in}{2.350859in}}%
\pgfpathlineto{\pgfqpoint{5.376408in}{2.351863in}}%
\pgfpathlineto{\pgfqpoint{5.362404in}{2.352892in}}%
\pgfpathlineto{\pgfqpoint{5.369783in}{2.359958in}}%
\pgfpathlineto{\pgfqpoint{5.377154in}{2.366948in}}%
\pgfpathlineto{\pgfqpoint{5.384518in}{2.373864in}}%
\pgfpathlineto{\pgfqpoint{5.391875in}{2.380706in}}%
\pgfpathclose%
\pgfusepath{fill}%
\end{pgfscope}%
\begin{pgfscope}%
\pgfpathrectangle{\pgfqpoint{1.150000in}{0.150000in}}{\pgfqpoint{5.700000in}{5.700000in}}%
\pgfusepath{clip}%
\pgfsetbuttcap%
\pgfsetroundjoin%
\definecolor{currentfill}{rgb}{0.282656,0.100196,0.422160}%
\pgfsetfillcolor{currentfill}%
\pgfsetfillopacity{0.700000}%
\pgfsetlinewidth{0.000000pt}%
\definecolor{currentstroke}{rgb}{0.000000,0.000000,0.000000}%
\pgfsetstrokecolor{currentstroke}%
\pgfsetdash{}{0pt}%
\pgfpathmoveto{\pgfqpoint{2.964971in}{2.282528in}}%
\pgfpathlineto{\pgfqpoint{2.978407in}{2.275631in}}%
\pgfpathlineto{\pgfqpoint{2.991846in}{2.268772in}}%
\pgfpathlineto{\pgfqpoint{3.005290in}{2.261952in}}%
\pgfpathlineto{\pgfqpoint{3.018737in}{2.255170in}}%
\pgfpathlineto{\pgfqpoint{3.010348in}{2.253496in}}%
\pgfpathlineto{\pgfqpoint{3.001947in}{2.252043in}}%
\pgfpathlineto{\pgfqpoint{2.993533in}{2.250818in}}%
\pgfpathlineto{\pgfqpoint{2.985105in}{2.249828in}}%
\pgfpathlineto{\pgfqpoint{2.971632in}{2.256853in}}%
\pgfpathlineto{\pgfqpoint{2.958163in}{2.263916in}}%
\pgfpathlineto{\pgfqpoint{2.944697in}{2.271018in}}%
\pgfpathlineto{\pgfqpoint{2.931235in}{2.278159in}}%
\pgfpathlineto{\pgfqpoint{2.939689in}{2.278901in}}%
\pgfpathlineto{\pgfqpoint{2.948130in}{2.279881in}}%
\pgfpathlineto{\pgfqpoint{2.956557in}{2.281092in}}%
\pgfpathlineto{\pgfqpoint{2.964971in}{2.282528in}}%
\pgfpathclose%
\pgfusepath{fill}%
\end{pgfscope}%
\begin{pgfscope}%
\pgfpathrectangle{\pgfqpoint{1.150000in}{0.150000in}}{\pgfqpoint{5.700000in}{5.700000in}}%
\pgfusepath{clip}%
\pgfsetbuttcap%
\pgfsetroundjoin%
\definecolor{currentfill}{rgb}{0.283091,0.110553,0.431554}%
\pgfsetfillcolor{currentfill}%
\pgfsetfillopacity{0.700000}%
\pgfsetlinewidth{0.000000pt}%
\definecolor{currentstroke}{rgb}{0.000000,0.000000,0.000000}%
\pgfsetstrokecolor{currentstroke}%
\pgfsetdash{}{0pt}%
\pgfpathmoveto{\pgfqpoint{5.079756in}{2.313170in}}%
\pgfpathlineto{\pgfqpoint{5.093654in}{2.311755in}}%
\pgfpathlineto{\pgfqpoint{5.107560in}{2.310365in}}%
\pgfpathlineto{\pgfqpoint{5.121474in}{2.308999in}}%
\pgfpathlineto{\pgfqpoint{5.135396in}{2.307659in}}%
\pgfpathlineto{\pgfqpoint{5.127913in}{2.299899in}}%
\pgfpathlineto{\pgfqpoint{5.120423in}{2.292066in}}%
\pgfpathlineto{\pgfqpoint{5.112927in}{2.284160in}}%
\pgfpathlineto{\pgfqpoint{5.105423in}{2.276181in}}%
\pgfpathlineto{\pgfqpoint{5.091489in}{2.277536in}}%
\pgfpathlineto{\pgfqpoint{5.077562in}{2.278915in}}%
\pgfpathlineto{\pgfqpoint{5.063644in}{2.280320in}}%
\pgfpathlineto{\pgfqpoint{5.049734in}{2.281750in}}%
\pgfpathlineto{\pgfqpoint{5.057249in}{2.289709in}}%
\pgfpathlineto{\pgfqpoint{5.064758in}{2.297599in}}%
\pgfpathlineto{\pgfqpoint{5.072260in}{2.305419in}}%
\pgfpathlineto{\pgfqpoint{5.079756in}{2.313170in}}%
\pgfpathclose%
\pgfusepath{fill}%
\end{pgfscope}%
\begin{pgfscope}%
\pgfpathrectangle{\pgfqpoint{1.150000in}{0.150000in}}{\pgfqpoint{5.700000in}{5.700000in}}%
\pgfusepath{clip}%
\pgfsetbuttcap%
\pgfsetroundjoin%
\definecolor{currentfill}{rgb}{0.271828,0.209303,0.504434}%
\pgfsetfillcolor{currentfill}%
\pgfsetfillopacity{0.700000}%
\pgfsetlinewidth{0.000000pt}%
\definecolor{currentstroke}{rgb}{0.000000,0.000000,0.000000}%
\pgfsetstrokecolor{currentstroke}%
\pgfsetdash{}{0pt}%
\pgfpathmoveto{\pgfqpoint{2.520764in}{2.502201in}}%
\pgfpathlineto{\pgfqpoint{2.534177in}{2.493554in}}%
\pgfpathlineto{\pgfqpoint{2.547592in}{2.484958in}}%
\pgfpathlineto{\pgfqpoint{2.561009in}{2.476413in}}%
\pgfpathlineto{\pgfqpoint{2.574429in}{2.467916in}}%
\pgfpathlineto{\pgfqpoint{2.565719in}{2.470066in}}%
\pgfpathlineto{\pgfqpoint{2.556990in}{2.472516in}}%
\pgfpathlineto{\pgfqpoint{2.548243in}{2.475272in}}%
\pgfpathlineto{\pgfqpoint{2.539477in}{2.478344in}}%
\pgfpathlineto{\pgfqpoint{2.526025in}{2.487114in}}%
\pgfpathlineto{\pgfqpoint{2.512574in}{2.495934in}}%
\pgfpathlineto{\pgfqpoint{2.499126in}{2.504805in}}%
\pgfpathlineto{\pgfqpoint{2.485679in}{2.513726in}}%
\pgfpathlineto{\pgfqpoint{2.494479in}{2.510375in}}%
\pgfpathlineto{\pgfqpoint{2.503260in}{2.507342in}}%
\pgfpathlineto{\pgfqpoint{2.512022in}{2.504619in}}%
\pgfpathlineto{\pgfqpoint{2.520764in}{2.502201in}}%
\pgfpathclose%
\pgfusepath{fill}%
\end{pgfscope}%
\begin{pgfscope}%
\pgfpathrectangle{\pgfqpoint{1.150000in}{0.150000in}}{\pgfqpoint{5.700000in}{5.700000in}}%
\pgfusepath{clip}%
\pgfsetbuttcap%
\pgfsetroundjoin%
\definecolor{currentfill}{rgb}{0.280267,0.073417,0.397163}%
\pgfsetfillcolor{currentfill}%
\pgfsetfillopacity{0.700000}%
\pgfsetlinewidth{0.000000pt}%
\definecolor{currentstroke}{rgb}{0.000000,0.000000,0.000000}%
\pgfsetstrokecolor{currentstroke}%
\pgfsetdash{}{0pt}%
\pgfpathmoveto{\pgfqpoint{4.767511in}{2.243003in}}%
\pgfpathlineto{\pgfqpoint{4.781319in}{2.241099in}}%
\pgfpathlineto{\pgfqpoint{4.795134in}{2.239221in}}%
\pgfpathlineto{\pgfqpoint{4.808958in}{2.237368in}}%
\pgfpathlineto{\pgfqpoint{4.822789in}{2.235540in}}%
\pgfpathlineto{\pgfqpoint{4.815180in}{2.227114in}}%
\pgfpathlineto{\pgfqpoint{4.807566in}{2.218633in}}%
\pgfpathlineto{\pgfqpoint{4.799945in}{2.210097in}}%
\pgfpathlineto{\pgfqpoint{4.792318in}{2.201508in}}%
\pgfpathlineto{\pgfqpoint{4.778476in}{2.203390in}}%
\pgfpathlineto{\pgfqpoint{4.764641in}{2.205298in}}%
\pgfpathlineto{\pgfqpoint{4.750815in}{2.207232in}}%
\pgfpathlineto{\pgfqpoint{4.736995in}{2.209191in}}%
\pgfpathlineto{\pgfqpoint{4.744633in}{2.217720in}}%
\pgfpathlineto{\pgfqpoint{4.752265in}{2.226199in}}%
\pgfpathlineto{\pgfqpoint{4.759891in}{2.234627in}}%
\pgfpathlineto{\pgfqpoint{4.767511in}{2.243003in}}%
\pgfpathclose%
\pgfusepath{fill}%
\end{pgfscope}%
\begin{pgfscope}%
\pgfpathrectangle{\pgfqpoint{1.150000in}{0.150000in}}{\pgfqpoint{5.700000in}{5.700000in}}%
\pgfusepath{clip}%
\pgfsetbuttcap%
\pgfsetroundjoin%
\definecolor{currentfill}{rgb}{0.269944,0.014625,0.341379}%
\pgfsetfillcolor{currentfill}%
\pgfsetfillopacity{0.700000}%
\pgfsetlinewidth{0.000000pt}%
\definecolor{currentstroke}{rgb}{0.000000,0.000000,0.000000}%
\pgfsetstrokecolor{currentstroke}%
\pgfsetdash{}{0pt}%
\pgfpathmoveto{\pgfqpoint{4.228788in}{2.142371in}}%
\pgfpathlineto{\pgfqpoint{4.242451in}{2.139324in}}%
\pgfpathlineto{\pgfqpoint{4.256122in}{2.136304in}}%
\pgfpathlineto{\pgfqpoint{4.269799in}{2.133312in}}%
\pgfpathlineto{\pgfqpoint{4.283483in}{2.130346in}}%
\pgfpathlineto{\pgfqpoint{4.275681in}{2.121905in}}%
\pgfpathlineto{\pgfqpoint{4.267873in}{2.113468in}}%
\pgfpathlineto{\pgfqpoint{4.260060in}{2.105037in}}%
\pgfpathlineto{\pgfqpoint{4.252242in}{2.096615in}}%
\pgfpathlineto{\pgfqpoint{4.238546in}{2.099703in}}%
\pgfpathlineto{\pgfqpoint{4.224857in}{2.102818in}}%
\pgfpathlineto{\pgfqpoint{4.211175in}{2.105959in}}%
\pgfpathlineto{\pgfqpoint{4.197499in}{2.109127in}}%
\pgfpathlineto{\pgfqpoint{4.205330in}{2.117423in}}%
\pgfpathlineto{\pgfqpoint{4.213155in}{2.125730in}}%
\pgfpathlineto{\pgfqpoint{4.220974in}{2.134047in}}%
\pgfpathlineto{\pgfqpoint{4.228788in}{2.142371in}}%
\pgfpathclose%
\pgfusepath{fill}%
\end{pgfscope}%
\begin{pgfscope}%
\pgfpathrectangle{\pgfqpoint{1.150000in}{0.150000in}}{\pgfqpoint{5.700000in}{5.700000in}}%
\pgfusepath{clip}%
\pgfsetbuttcap%
\pgfsetroundjoin%
\definecolor{currentfill}{rgb}{0.282623,0.140926,0.457517}%
\pgfsetfillcolor{currentfill}%
\pgfsetfillopacity{0.700000}%
\pgfsetlinewidth{0.000000pt}%
\definecolor{currentstroke}{rgb}{0.000000,0.000000,0.000000}%
\pgfsetstrokecolor{currentstroke}%
\pgfsetdash{}{0pt}%
\pgfpathmoveto{\pgfqpoint{2.769951in}{2.367020in}}%
\pgfpathlineto{\pgfqpoint{2.783373in}{2.359383in}}%
\pgfpathlineto{\pgfqpoint{2.796799in}{2.351791in}}%
\pgfpathlineto{\pgfqpoint{2.810229in}{2.344241in}}%
\pgfpathlineto{\pgfqpoint{2.823661in}{2.336733in}}%
\pgfpathlineto{\pgfqpoint{2.815137in}{2.336740in}}%
\pgfpathlineto{\pgfqpoint{2.806599in}{2.337004in}}%
\pgfpathlineto{\pgfqpoint{2.798045in}{2.337533in}}%
\pgfpathlineto{\pgfqpoint{2.789476in}{2.338334in}}%
\pgfpathlineto{\pgfqpoint{2.776015in}{2.346100in}}%
\pgfpathlineto{\pgfqpoint{2.762557in}{2.353908in}}%
\pgfpathlineto{\pgfqpoint{2.749102in}{2.361759in}}%
\pgfpathlineto{\pgfqpoint{2.735649in}{2.369653in}}%
\pgfpathlineto{\pgfqpoint{2.744248in}{2.368589in}}%
\pgfpathlineto{\pgfqpoint{2.752831in}{2.367800in}}%
\pgfpathlineto{\pgfqpoint{2.761399in}{2.367279in}}%
\pgfpathlineto{\pgfqpoint{2.769951in}{2.367020in}}%
\pgfpathclose%
\pgfusepath{fill}%
\end{pgfscope}%
\begin{pgfscope}%
\pgfpathrectangle{\pgfqpoint{1.150000in}{0.150000in}}{\pgfqpoint{5.700000in}{5.700000in}}%
\pgfusepath{clip}%
\pgfsetbuttcap%
\pgfsetroundjoin%
\definecolor{currentfill}{rgb}{0.279574,0.170599,0.479997}%
\pgfsetfillcolor{currentfill}%
\pgfsetfillopacity{0.700000}%
\pgfsetlinewidth{0.000000pt}%
\definecolor{currentstroke}{rgb}{0.000000,0.000000,0.000000}%
\pgfsetstrokecolor{currentstroke}%
\pgfsetdash{}{0pt}%
\pgfpathmoveto{\pgfqpoint{5.618531in}{2.420818in}}%
\pgfpathlineto{\pgfqpoint{5.632592in}{2.419964in}}%
\pgfpathlineto{\pgfqpoint{5.646662in}{2.419133in}}%
\pgfpathlineto{\pgfqpoint{5.660740in}{2.418327in}}%
\pgfpathlineto{\pgfqpoint{5.674828in}{2.417546in}}%
\pgfpathlineto{\pgfqpoint{5.667594in}{2.411468in}}%
\pgfpathlineto{\pgfqpoint{5.660352in}{2.405317in}}%
\pgfpathlineto{\pgfqpoint{5.653102in}{2.399091in}}%
\pgfpathlineto{\pgfqpoint{5.645845in}{2.392789in}}%
\pgfpathlineto{\pgfqpoint{5.631741in}{2.393516in}}%
\pgfpathlineto{\pgfqpoint{5.617647in}{2.394267in}}%
\pgfpathlineto{\pgfqpoint{5.603561in}{2.395044in}}%
\pgfpathlineto{\pgfqpoint{5.589484in}{2.395844in}}%
\pgfpathlineto{\pgfqpoint{5.596757in}{2.402196in}}%
\pgfpathlineto{\pgfqpoint{5.604023in}{2.408475in}}%
\pgfpathlineto{\pgfqpoint{5.611281in}{2.414681in}}%
\pgfpathlineto{\pgfqpoint{5.618531in}{2.420818in}}%
\pgfpathclose%
\pgfusepath{fill}%
\end{pgfscope}%
\begin{pgfscope}%
\pgfpathrectangle{\pgfqpoint{1.150000in}{0.150000in}}{\pgfqpoint{5.700000in}{5.700000in}}%
\pgfusepath{clip}%
\pgfsetbuttcap%
\pgfsetroundjoin%
\definecolor{currentfill}{rgb}{0.278791,0.062145,0.386592}%
\pgfsetfillcolor{currentfill}%
\pgfsetfillopacity{0.700000}%
\pgfsetlinewidth{0.000000pt}%
\definecolor{currentstroke}{rgb}{0.000000,0.000000,0.000000}%
\pgfsetstrokecolor{currentstroke}%
\pgfsetdash{}{0pt}%
\pgfpathmoveto{\pgfqpoint{3.159697in}{2.212887in}}%
\pgfpathlineto{\pgfqpoint{3.173156in}{2.206667in}}%
\pgfpathlineto{\pgfqpoint{3.186620in}{2.200483in}}%
\pgfpathlineto{\pgfqpoint{3.200088in}{2.194334in}}%
\pgfpathlineto{\pgfqpoint{3.213560in}{2.188220in}}%
\pgfpathlineto{\pgfqpoint{3.205289in}{2.185048in}}%
\pgfpathlineto{\pgfqpoint{3.197006in}{2.182064in}}%
\pgfpathlineto{\pgfqpoint{3.188713in}{2.179273in}}%
\pgfpathlineto{\pgfqpoint{3.180409in}{2.176681in}}%
\pgfpathlineto{\pgfqpoint{3.166914in}{2.183023in}}%
\pgfpathlineto{\pgfqpoint{3.153423in}{2.189401in}}%
\pgfpathlineto{\pgfqpoint{3.139936in}{2.195814in}}%
\pgfpathlineto{\pgfqpoint{3.126454in}{2.202263in}}%
\pgfpathlineto{\pgfqpoint{3.134782in}{2.204621in}}%
\pgfpathlineto{\pgfqpoint{3.143098in}{2.207182in}}%
\pgfpathlineto{\pgfqpoint{3.151403in}{2.209939in}}%
\pgfpathlineto{\pgfqpoint{3.159697in}{2.212887in}}%
\pgfpathclose%
\pgfusepath{fill}%
\end{pgfscope}%
\begin{pgfscope}%
\pgfpathrectangle{\pgfqpoint{1.150000in}{0.150000in}}{\pgfqpoint{5.700000in}{5.700000in}}%
\pgfusepath{clip}%
\pgfsetbuttcap%
\pgfsetroundjoin%
\definecolor{currentfill}{rgb}{0.267004,0.004874,0.329415}%
\pgfsetfillcolor{currentfill}%
\pgfsetfillopacity{0.700000}%
\pgfsetlinewidth{0.000000pt}%
\definecolor{currentstroke}{rgb}{0.000000,0.000000,0.000000}%
\pgfsetstrokecolor{currentstroke}%
\pgfsetdash{}{0pt}%
\pgfpathmoveto{\pgfqpoint{4.002373in}{2.117898in}}%
\pgfpathlineto{\pgfqpoint{4.015983in}{2.114265in}}%
\pgfpathlineto{\pgfqpoint{4.029601in}{2.110660in}}%
\pgfpathlineto{\pgfqpoint{4.043224in}{2.107083in}}%
\pgfpathlineto{\pgfqpoint{4.056853in}{2.103534in}}%
\pgfpathlineto{\pgfqpoint{4.048969in}{2.095632in}}%
\pgfpathlineto{\pgfqpoint{4.041080in}{2.087768in}}%
\pgfpathlineto{\pgfqpoint{4.033184in}{2.079946in}}%
\pgfpathlineto{\pgfqpoint{4.025282in}{2.072167in}}%
\pgfpathlineto{\pgfqpoint{4.011639in}{2.075865in}}%
\pgfpathlineto{\pgfqpoint{3.998003in}{2.079590in}}%
\pgfpathlineto{\pgfqpoint{3.984373in}{2.083343in}}%
\pgfpathlineto{\pgfqpoint{3.970749in}{2.087124in}}%
\pgfpathlineto{\pgfqpoint{3.978664in}{2.094749in}}%
\pgfpathlineto{\pgfqpoint{3.986573in}{2.102422in}}%
\pgfpathlineto{\pgfqpoint{3.994476in}{2.110139in}}%
\pgfpathlineto{\pgfqpoint{4.002373in}{2.117898in}}%
\pgfpathclose%
\pgfusepath{fill}%
\end{pgfscope}%
\begin{pgfscope}%
\pgfpathrectangle{\pgfqpoint{1.150000in}{0.150000in}}{\pgfqpoint{5.700000in}{5.700000in}}%
\pgfusepath{clip}%
\pgfsetbuttcap%
\pgfsetroundjoin%
\definecolor{currentfill}{rgb}{0.274952,0.037752,0.364543}%
\pgfsetfillcolor{currentfill}%
\pgfsetfillopacity{0.700000}%
\pgfsetlinewidth{0.000000pt}%
\definecolor{currentstroke}{rgb}{0.000000,0.000000,0.000000}%
\pgfsetstrokecolor{currentstroke}%
\pgfsetdash{}{0pt}%
\pgfpathmoveto{\pgfqpoint{4.455232in}{2.176554in}}%
\pgfpathlineto{\pgfqpoint{4.468955in}{2.174030in}}%
\pgfpathlineto{\pgfqpoint{4.482686in}{2.171533in}}%
\pgfpathlineto{\pgfqpoint{4.496424in}{2.169062in}}%
\pgfpathlineto{\pgfqpoint{4.510169in}{2.166616in}}%
\pgfpathlineto{\pgfqpoint{4.502446in}{2.157966in}}%
\pgfpathlineto{\pgfqpoint{4.494717in}{2.149290in}}%
\pgfpathlineto{\pgfqpoint{4.486983in}{2.140591in}}%
\pgfpathlineto{\pgfqpoint{4.479242in}{2.131871in}}%
\pgfpathlineto{\pgfqpoint{4.465486in}{2.134411in}}%
\pgfpathlineto{\pgfqpoint{4.451737in}{2.136977in}}%
\pgfpathlineto{\pgfqpoint{4.437996in}{2.139570in}}%
\pgfpathlineto{\pgfqpoint{4.424261in}{2.142189in}}%
\pgfpathlineto{\pgfqpoint{4.432012in}{2.150809in}}%
\pgfpathlineto{\pgfqpoint{4.439758in}{2.159411in}}%
\pgfpathlineto{\pgfqpoint{4.447497in}{2.167993in}}%
\pgfpathlineto{\pgfqpoint{4.455232in}{2.176554in}}%
\pgfpathclose%
\pgfusepath{fill}%
\end{pgfscope}%
\begin{pgfscope}%
\pgfpathrectangle{\pgfqpoint{1.150000in}{0.150000in}}{\pgfqpoint{5.700000in}{5.700000in}}%
\pgfusepath{clip}%
\pgfsetbuttcap%
\pgfsetroundjoin%
\definecolor{currentfill}{rgb}{0.276194,0.190074,0.493001}%
\pgfsetfillcolor{currentfill}%
\pgfsetfillopacity{0.700000}%
\pgfsetlinewidth{0.000000pt}%
\definecolor{currentstroke}{rgb}{0.000000,0.000000,0.000000}%
\pgfsetstrokecolor{currentstroke}%
\pgfsetdash{}{0pt}%
\pgfpathmoveto{\pgfqpoint{5.845157in}{2.457250in}}%
\pgfpathlineto{\pgfqpoint{5.859288in}{2.456539in}}%
\pgfpathlineto{\pgfqpoint{5.873429in}{2.455852in}}%
\pgfpathlineto{\pgfqpoint{5.887578in}{2.455189in}}%
\pgfpathlineto{\pgfqpoint{5.901737in}{2.454551in}}%
\pgfpathlineto{\pgfqpoint{5.894619in}{2.449232in}}%
\pgfpathlineto{\pgfqpoint{5.887493in}{2.443850in}}%
\pgfpathlineto{\pgfqpoint{5.880359in}{2.438402in}}%
\pgfpathlineto{\pgfqpoint{5.873217in}{2.432885in}}%
\pgfpathlineto{\pgfqpoint{5.859040in}{2.433441in}}%
\pgfpathlineto{\pgfqpoint{5.844872in}{2.434022in}}%
\pgfpathlineto{\pgfqpoint{5.830714in}{2.434627in}}%
\pgfpathlineto{\pgfqpoint{5.816564in}{2.435257in}}%
\pgfpathlineto{\pgfqpoint{5.823724in}{2.440850in}}%
\pgfpathlineto{\pgfqpoint{5.830876in}{2.446379in}}%
\pgfpathlineto{\pgfqpoint{5.838020in}{2.451844in}}%
\pgfpathlineto{\pgfqpoint{5.845157in}{2.457250in}}%
\pgfpathclose%
\pgfusepath{fill}%
\end{pgfscope}%
\begin{pgfscope}%
\pgfpathrectangle{\pgfqpoint{1.150000in}{0.150000in}}{\pgfqpoint{5.700000in}{5.700000in}}%
\pgfusepath{clip}%
\pgfsetbuttcap%
\pgfsetroundjoin%
\definecolor{currentfill}{rgb}{0.282623,0.140926,0.457517}%
\pgfsetfillcolor{currentfill}%
\pgfsetfillopacity{0.700000}%
\pgfsetlinewidth{0.000000pt}%
\definecolor{currentstroke}{rgb}{0.000000,0.000000,0.000000}%
\pgfsetstrokecolor{currentstroke}%
\pgfsetdash{}{0pt}%
\pgfpathmoveto{\pgfqpoint{5.306473in}{2.357257in}}%
\pgfpathlineto{\pgfqpoint{5.320443in}{2.356129in}}%
\pgfpathlineto{\pgfqpoint{5.334422in}{2.355025in}}%
\pgfpathlineto{\pgfqpoint{5.348409in}{2.353946in}}%
\pgfpathlineto{\pgfqpoint{5.362404in}{2.352892in}}%
\pgfpathlineto{\pgfqpoint{5.355018in}{2.345750in}}%
\pgfpathlineto{\pgfqpoint{5.347624in}{2.338529in}}%
\pgfpathlineto{\pgfqpoint{5.340223in}{2.331229in}}%
\pgfpathlineto{\pgfqpoint{5.332814in}{2.323850in}}%
\pgfpathlineto{\pgfqpoint{5.318805in}{2.324891in}}%
\pgfpathlineto{\pgfqpoint{5.304805in}{2.325956in}}%
\pgfpathlineto{\pgfqpoint{5.290813in}{2.327047in}}%
\pgfpathlineto{\pgfqpoint{5.276829in}{2.328162in}}%
\pgfpathlineto{\pgfqpoint{5.284251in}{2.335550in}}%
\pgfpathlineto{\pgfqpoint{5.291666in}{2.342861in}}%
\pgfpathlineto{\pgfqpoint{5.299073in}{2.350096in}}%
\pgfpathlineto{\pgfqpoint{5.306473in}{2.357257in}}%
\pgfpathclose%
\pgfusepath{fill}%
\end{pgfscope}%
\begin{pgfscope}%
\pgfpathrectangle{\pgfqpoint{1.150000in}{0.150000in}}{\pgfqpoint{5.700000in}{5.700000in}}%
\pgfusepath{clip}%
\pgfsetbuttcap%
\pgfsetroundjoin%
\definecolor{currentfill}{rgb}{0.282910,0.105393,0.426902}%
\pgfsetfillcolor{currentfill}%
\pgfsetfillopacity{0.700000}%
\pgfsetlinewidth{0.000000pt}%
\definecolor{currentstroke}{rgb}{0.000000,0.000000,0.000000}%
\pgfsetstrokecolor{currentstroke}%
\pgfsetdash{}{0pt}%
\pgfpathmoveto{\pgfqpoint{4.994173in}{2.287720in}}%
\pgfpathlineto{\pgfqpoint{5.008051in}{2.286189in}}%
\pgfpathlineto{\pgfqpoint{5.021937in}{2.284684in}}%
\pgfpathlineto{\pgfqpoint{5.035831in}{2.283205in}}%
\pgfpathlineto{\pgfqpoint{5.049734in}{2.281750in}}%
\pgfpathlineto{\pgfqpoint{5.042211in}{2.273721in}}%
\pgfpathlineto{\pgfqpoint{5.034682in}{2.265622in}}%
\pgfpathlineto{\pgfqpoint{5.027147in}{2.257454in}}%
\pgfpathlineto{\pgfqpoint{5.019604in}{2.249217in}}%
\pgfpathlineto{\pgfqpoint{5.005690in}{2.250699in}}%
\pgfpathlineto{\pgfqpoint{4.991784in}{2.252207in}}%
\pgfpathlineto{\pgfqpoint{4.977886in}{2.253740in}}%
\pgfpathlineto{\pgfqpoint{4.963996in}{2.255299in}}%
\pgfpathlineto{\pgfqpoint{4.971550in}{2.263503in}}%
\pgfpathlineto{\pgfqpoint{4.979098in}{2.271641in}}%
\pgfpathlineto{\pgfqpoint{4.986639in}{2.279713in}}%
\pgfpathlineto{\pgfqpoint{4.994173in}{2.287720in}}%
\pgfpathclose%
\pgfusepath{fill}%
\end{pgfscope}%
\begin{pgfscope}%
\pgfpathrectangle{\pgfqpoint{1.150000in}{0.150000in}}{\pgfqpoint{5.700000in}{5.700000in}}%
\pgfusepath{clip}%
\pgfsetbuttcap%
\pgfsetroundjoin%
\definecolor{currentfill}{rgb}{0.267004,0.004874,0.329415}%
\pgfsetfillcolor{currentfill}%
\pgfsetfillopacity{0.700000}%
\pgfsetlinewidth{0.000000pt}%
\definecolor{currentstroke}{rgb}{0.000000,0.000000,0.000000}%
\pgfsetstrokecolor{currentstroke}%
\pgfsetdash{}{0pt}%
\pgfpathmoveto{\pgfqpoint{3.635386in}{2.116676in}}%
\pgfpathlineto{\pgfqpoint{3.648924in}{2.111978in}}%
\pgfpathlineto{\pgfqpoint{3.662467in}{2.107310in}}%
\pgfpathlineto{\pgfqpoint{3.676015in}{2.102673in}}%
\pgfpathlineto{\pgfqpoint{3.689569in}{2.098065in}}%
\pgfpathlineto{\pgfqpoint{3.681536in}{2.091781in}}%
\pgfpathlineto{\pgfqpoint{3.673496in}{2.085598in}}%
\pgfpathlineto{\pgfqpoint{3.665448in}{2.079522in}}%
\pgfpathlineto{\pgfqpoint{3.657393in}{2.073558in}}%
\pgfpathlineto{\pgfqpoint{3.643822in}{2.078353in}}%
\pgfpathlineto{\pgfqpoint{3.630257in}{2.083179in}}%
\pgfpathlineto{\pgfqpoint{3.616697in}{2.088034in}}%
\pgfpathlineto{\pgfqpoint{3.603143in}{2.092920in}}%
\pgfpathlineto{\pgfqpoint{3.611215in}{2.098692in}}%
\pgfpathlineto{\pgfqpoint{3.619280in}{2.104578in}}%
\pgfpathlineto{\pgfqpoint{3.627337in}{2.110574in}}%
\pgfpathlineto{\pgfqpoint{3.635386in}{2.116676in}}%
\pgfpathclose%
\pgfusepath{fill}%
\end{pgfscope}%
\begin{pgfscope}%
\pgfpathrectangle{\pgfqpoint{1.150000in}{0.150000in}}{\pgfqpoint{5.700000in}{5.700000in}}%
\pgfusepath{clip}%
\pgfsetbuttcap%
\pgfsetroundjoin%
\definecolor{currentfill}{rgb}{0.269944,0.014625,0.341379}%
\pgfsetfillcolor{currentfill}%
\pgfsetfillopacity{0.700000}%
\pgfsetlinewidth{0.000000pt}%
\definecolor{currentstroke}{rgb}{0.000000,0.000000,0.000000}%
\pgfsetstrokecolor{currentstroke}%
\pgfsetdash{}{0pt}%
\pgfpathmoveto{\pgfqpoint{3.494897in}{2.133114in}}%
\pgfpathlineto{\pgfqpoint{3.508410in}{2.127980in}}%
\pgfpathlineto{\pgfqpoint{3.521928in}{2.122879in}}%
\pgfpathlineto{\pgfqpoint{3.535450in}{2.117808in}}%
\pgfpathlineto{\pgfqpoint{3.548978in}{2.112769in}}%
\pgfpathlineto{\pgfqpoint{3.540881in}{2.107313in}}%
\pgfpathlineto{\pgfqpoint{3.532775in}{2.101985in}}%
\pgfpathlineto{\pgfqpoint{3.524661in}{2.096790in}}%
\pgfpathlineto{\pgfqpoint{3.516539in}{2.091733in}}%
\pgfpathlineto{\pgfqpoint{3.502993in}{2.096973in}}%
\pgfpathlineto{\pgfqpoint{3.489451in}{2.102245in}}%
\pgfpathlineto{\pgfqpoint{3.475915in}{2.107548in}}%
\pgfpathlineto{\pgfqpoint{3.462384in}{2.112883in}}%
\pgfpathlineto{\pgfqpoint{3.470525in}{2.117734in}}%
\pgfpathlineto{\pgfqpoint{3.478658in}{2.122726in}}%
\pgfpathlineto{\pgfqpoint{3.486782in}{2.127854in}}%
\pgfpathlineto{\pgfqpoint{3.494897in}{2.133114in}}%
\pgfpathclose%
\pgfusepath{fill}%
\end{pgfscope}%
\begin{pgfscope}%
\pgfpathrectangle{\pgfqpoint{1.150000in}{0.150000in}}{\pgfqpoint{5.700000in}{5.700000in}}%
\pgfusepath{clip}%
\pgfsetbuttcap%
\pgfsetroundjoin%
\definecolor{currentfill}{rgb}{0.278791,0.062145,0.386592}%
\pgfsetfillcolor{currentfill}%
\pgfsetfillopacity{0.700000}%
\pgfsetlinewidth{0.000000pt}%
\definecolor{currentstroke}{rgb}{0.000000,0.000000,0.000000}%
\pgfsetstrokecolor{currentstroke}%
\pgfsetdash{}{0pt}%
\pgfpathmoveto{\pgfqpoint{4.681794in}{2.217283in}}%
\pgfpathlineto{\pgfqpoint{4.695583in}{2.215222in}}%
\pgfpathlineto{\pgfqpoint{4.709380in}{2.213186in}}%
\pgfpathlineto{\pgfqpoint{4.723184in}{2.211176in}}%
\pgfpathlineto{\pgfqpoint{4.736995in}{2.209191in}}%
\pgfpathlineto{\pgfqpoint{4.729352in}{2.200612in}}%
\pgfpathlineto{\pgfqpoint{4.721702in}{2.191986in}}%
\pgfpathlineto{\pgfqpoint{4.714046in}{2.183312in}}%
\pgfpathlineto{\pgfqpoint{4.706384in}{2.174594in}}%
\pgfpathlineto{\pgfqpoint{4.692562in}{2.176647in}}%
\pgfpathlineto{\pgfqpoint{4.678747in}{2.178726in}}%
\pgfpathlineto{\pgfqpoint{4.664939in}{2.180830in}}%
\pgfpathlineto{\pgfqpoint{4.651139in}{2.182960in}}%
\pgfpathlineto{\pgfqpoint{4.658812in}{2.191605in}}%
\pgfpathlineto{\pgfqpoint{4.666479in}{2.200208in}}%
\pgfpathlineto{\pgfqpoint{4.674139in}{2.208768in}}%
\pgfpathlineto{\pgfqpoint{4.681794in}{2.217283in}}%
\pgfpathclose%
\pgfusepath{fill}%
\end{pgfscope}%
\begin{pgfscope}%
\pgfpathrectangle{\pgfqpoint{1.150000in}{0.150000in}}{\pgfqpoint{5.700000in}{5.700000in}}%
\pgfusepath{clip}%
\pgfsetbuttcap%
\pgfsetroundjoin%
\definecolor{currentfill}{rgb}{0.267004,0.004874,0.329415}%
\pgfsetfillcolor{currentfill}%
\pgfsetfillopacity{0.700000}%
\pgfsetlinewidth{0.000000pt}%
\definecolor{currentstroke}{rgb}{0.000000,0.000000,0.000000}%
\pgfsetstrokecolor{currentstroke}%
\pgfsetdash{}{0pt}%
\pgfpathmoveto{\pgfqpoint{3.775841in}{2.106701in}}%
\pgfpathlineto{\pgfqpoint{3.789407in}{2.102416in}}%
\pgfpathlineto{\pgfqpoint{3.802980in}{2.098160in}}%
\pgfpathlineto{\pgfqpoint{3.816558in}{2.093933in}}%
\pgfpathlineto{\pgfqpoint{3.830142in}{2.089735in}}%
\pgfpathlineto{\pgfqpoint{3.822167in}{2.082747in}}%
\pgfpathlineto{\pgfqpoint{3.814186in}{2.075836in}}%
\pgfpathlineto{\pgfqpoint{3.806198in}{2.069008in}}%
\pgfpathlineto{\pgfqpoint{3.798203in}{2.062267in}}%
\pgfpathlineto{\pgfqpoint{3.784604in}{2.066639in}}%
\pgfpathlineto{\pgfqpoint{3.771011in}{2.071041in}}%
\pgfpathlineto{\pgfqpoint{3.757423in}{2.075472in}}%
\pgfpathlineto{\pgfqpoint{3.743841in}{2.079932in}}%
\pgfpathlineto{\pgfqpoint{3.751851in}{2.086493in}}%
\pgfpathlineto{\pgfqpoint{3.759854in}{2.093145in}}%
\pgfpathlineto{\pgfqpoint{3.767851in}{2.099883in}}%
\pgfpathlineto{\pgfqpoint{3.775841in}{2.106701in}}%
\pgfpathclose%
\pgfusepath{fill}%
\end{pgfscope}%
\begin{pgfscope}%
\pgfpathrectangle{\pgfqpoint{1.150000in}{0.150000in}}{\pgfqpoint{5.700000in}{5.700000in}}%
\pgfusepath{clip}%
\pgfsetbuttcap%
\pgfsetroundjoin%
\definecolor{currentfill}{rgb}{0.275191,0.194905,0.496005}%
\pgfsetfillcolor{currentfill}%
\pgfsetfillopacity{0.700000}%
\pgfsetlinewidth{0.000000pt}%
\definecolor{currentstroke}{rgb}{0.000000,0.000000,0.000000}%
\pgfsetstrokecolor{currentstroke}%
\pgfsetdash{}{0pt}%
\pgfpathmoveto{\pgfqpoint{2.574429in}{2.467916in}}%
\pgfpathlineto{\pgfqpoint{2.587850in}{2.459469in}}%
\pgfpathlineto{\pgfqpoint{2.601274in}{2.451070in}}%
\pgfpathlineto{\pgfqpoint{2.614700in}{2.442720in}}%
\pgfpathlineto{\pgfqpoint{2.628129in}{2.434417in}}%
\pgfpathlineto{\pgfqpoint{2.619451in}{2.436298in}}%
\pgfpathlineto{\pgfqpoint{2.610755in}{2.438476in}}%
\pgfpathlineto{\pgfqpoint{2.602041in}{2.440957in}}%
\pgfpathlineto{\pgfqpoint{2.593308in}{2.443749in}}%
\pgfpathlineto{\pgfqpoint{2.579847in}{2.452325in}}%
\pgfpathlineto{\pgfqpoint{2.566388in}{2.460950in}}%
\pgfpathlineto{\pgfqpoint{2.552932in}{2.469622in}}%
\pgfpathlineto{\pgfqpoint{2.539477in}{2.478344in}}%
\pgfpathlineto{\pgfqpoint{2.548243in}{2.475272in}}%
\pgfpathlineto{\pgfqpoint{2.556990in}{2.472516in}}%
\pgfpathlineto{\pgfqpoint{2.565719in}{2.470066in}}%
\pgfpathlineto{\pgfqpoint{2.574429in}{2.467916in}}%
\pgfpathclose%
\pgfusepath{fill}%
\end{pgfscope}%
\begin{pgfscope}%
\pgfpathrectangle{\pgfqpoint{1.150000in}{0.150000in}}{\pgfqpoint{5.700000in}{5.700000in}}%
\pgfusepath{clip}%
\pgfsetbuttcap%
\pgfsetroundjoin%
\definecolor{currentfill}{rgb}{0.273809,0.031497,0.358853}%
\pgfsetfillcolor{currentfill}%
\pgfsetfillopacity{0.700000}%
\pgfsetlinewidth{0.000000pt}%
\definecolor{currentstroke}{rgb}{0.000000,0.000000,0.000000}%
\pgfsetstrokecolor{currentstroke}%
\pgfsetdash{}{0pt}%
\pgfpathmoveto{\pgfqpoint{3.354312in}{2.156714in}}%
\pgfpathlineto{\pgfqpoint{3.367804in}{2.151121in}}%
\pgfpathlineto{\pgfqpoint{3.381301in}{2.145561in}}%
\pgfpathlineto{\pgfqpoint{3.394803in}{2.140034in}}%
\pgfpathlineto{\pgfqpoint{3.408309in}{2.134539in}}%
\pgfpathlineto{\pgfqpoint{3.400140in}{2.130044in}}%
\pgfpathlineto{\pgfqpoint{3.391961in}{2.125705in}}%
\pgfpathlineto{\pgfqpoint{3.383774in}{2.121525in}}%
\pgfpathlineto{\pgfqpoint{3.375576in}{2.117512in}}%
\pgfpathlineto{\pgfqpoint{3.362050in}{2.123221in}}%
\pgfpathlineto{\pgfqpoint{3.348528in}{2.128963in}}%
\pgfpathlineto{\pgfqpoint{3.335010in}{2.134738in}}%
\pgfpathlineto{\pgfqpoint{3.321498in}{2.140546in}}%
\pgfpathlineto{\pgfqpoint{3.329716in}{2.144339in}}%
\pgfpathlineto{\pgfqpoint{3.337924in}{2.148302in}}%
\pgfpathlineto{\pgfqpoint{3.346123in}{2.152429in}}%
\pgfpathlineto{\pgfqpoint{3.354312in}{2.156714in}}%
\pgfpathclose%
\pgfusepath{fill}%
\end{pgfscope}%
\begin{pgfscope}%
\pgfpathrectangle{\pgfqpoint{1.150000in}{0.150000in}}{\pgfqpoint{5.700000in}{5.700000in}}%
\pgfusepath{clip}%
\pgfsetbuttcap%
\pgfsetroundjoin%
\definecolor{currentfill}{rgb}{0.268510,0.009605,0.335427}%
\pgfsetfillcolor{currentfill}%
\pgfsetfillopacity{0.700000}%
\pgfsetlinewidth{0.000000pt}%
\definecolor{currentstroke}{rgb}{0.000000,0.000000,0.000000}%
\pgfsetstrokecolor{currentstroke}%
\pgfsetdash{}{0pt}%
\pgfpathmoveto{\pgfqpoint{4.142863in}{2.122071in}}%
\pgfpathlineto{\pgfqpoint{4.156512in}{2.118795in}}%
\pgfpathlineto{\pgfqpoint{4.170168in}{2.115545in}}%
\pgfpathlineto{\pgfqpoint{4.183831in}{2.112323in}}%
\pgfpathlineto{\pgfqpoint{4.197499in}{2.109127in}}%
\pgfpathlineto{\pgfqpoint{4.189663in}{2.100847in}}%
\pgfpathlineto{\pgfqpoint{4.181822in}{2.092585in}}%
\pgfpathlineto{\pgfqpoint{4.173974in}{2.084345in}}%
\pgfpathlineto{\pgfqpoint{4.166121in}{2.076129in}}%
\pgfpathlineto{\pgfqpoint{4.152440in}{2.079459in}}%
\pgfpathlineto{\pgfqpoint{4.138765in}{2.082816in}}%
\pgfpathlineto{\pgfqpoint{4.125097in}{2.086201in}}%
\pgfpathlineto{\pgfqpoint{4.111436in}{2.089613in}}%
\pgfpathlineto{\pgfqpoint{4.119301in}{2.097689in}}%
\pgfpathlineto{\pgfqpoint{4.127161in}{2.105793in}}%
\pgfpathlineto{\pgfqpoint{4.135015in}{2.113921in}}%
\pgfpathlineto{\pgfqpoint{4.142863in}{2.122071in}}%
\pgfpathclose%
\pgfusepath{fill}%
\end{pgfscope}%
\begin{pgfscope}%
\pgfpathrectangle{\pgfqpoint{1.150000in}{0.150000in}}{\pgfqpoint{5.700000in}{5.700000in}}%
\pgfusepath{clip}%
\pgfsetbuttcap%
\pgfsetroundjoin%
\definecolor{currentfill}{rgb}{0.280868,0.160771,0.472899}%
\pgfsetfillcolor{currentfill}%
\pgfsetfillopacity{0.700000}%
\pgfsetlinewidth{0.000000pt}%
\definecolor{currentstroke}{rgb}{0.000000,0.000000,0.000000}%
\pgfsetstrokecolor{currentstroke}%
\pgfsetdash{}{0pt}%
\pgfpathmoveto{\pgfqpoint{5.533262in}{2.399293in}}%
\pgfpathlineto{\pgfqpoint{5.547304in}{2.398394in}}%
\pgfpathlineto{\pgfqpoint{5.561355in}{2.397519in}}%
\pgfpathlineto{\pgfqpoint{5.575415in}{2.396670in}}%
\pgfpathlineto{\pgfqpoint{5.589484in}{2.395844in}}%
\pgfpathlineto{\pgfqpoint{5.582202in}{2.389417in}}%
\pgfpathlineto{\pgfqpoint{5.574913in}{2.382912in}}%
\pgfpathlineto{\pgfqpoint{5.567616in}{2.376328in}}%
\pgfpathlineto{\pgfqpoint{5.560312in}{2.369664in}}%
\pgfpathlineto{\pgfqpoint{5.546228in}{2.370448in}}%
\pgfpathlineto{\pgfqpoint{5.532153in}{2.371258in}}%
\pgfpathlineto{\pgfqpoint{5.518086in}{2.372092in}}%
\pgfpathlineto{\pgfqpoint{5.504029in}{2.372950in}}%
\pgfpathlineto{\pgfqpoint{5.511348in}{2.379650in}}%
\pgfpathlineto{\pgfqpoint{5.518660in}{2.386273in}}%
\pgfpathlineto{\pgfqpoint{5.525965in}{2.392820in}}%
\pgfpathlineto{\pgfqpoint{5.533262in}{2.399293in}}%
\pgfpathclose%
\pgfusepath{fill}%
\end{pgfscope}%
\begin{pgfscope}%
\pgfpathrectangle{\pgfqpoint{1.150000in}{0.150000in}}{\pgfqpoint{5.700000in}{5.700000in}}%
\pgfusepath{clip}%
\pgfsetbuttcap%
\pgfsetroundjoin%
\definecolor{currentfill}{rgb}{0.272594,0.025563,0.353093}%
\pgfsetfillcolor{currentfill}%
\pgfsetfillopacity{0.700000}%
\pgfsetlinewidth{0.000000pt}%
\definecolor{currentstroke}{rgb}{0.000000,0.000000,0.000000}%
\pgfsetstrokecolor{currentstroke}%
\pgfsetdash{}{0pt}%
\pgfpathmoveto{\pgfqpoint{4.369392in}{2.152927in}}%
\pgfpathlineto{\pgfqpoint{4.383099in}{2.150203in}}%
\pgfpathlineto{\pgfqpoint{4.396812in}{2.147505in}}%
\pgfpathlineto{\pgfqpoint{4.410533in}{2.144834in}}%
\pgfpathlineto{\pgfqpoint{4.424261in}{2.142189in}}%
\pgfpathlineto{\pgfqpoint{4.416504in}{2.133553in}}%
\pgfpathlineto{\pgfqpoint{4.408742in}{2.124904in}}%
\pgfpathlineto{\pgfqpoint{4.400974in}{2.116245in}}%
\pgfpathlineto{\pgfqpoint{4.393200in}{2.107576in}}%
\pgfpathlineto{\pgfqpoint{4.379461in}{2.110330in}}%
\pgfpathlineto{\pgfqpoint{4.365729in}{2.113110in}}%
\pgfpathlineto{\pgfqpoint{4.352004in}{2.115916in}}%
\pgfpathlineto{\pgfqpoint{4.338286in}{2.118749in}}%
\pgfpathlineto{\pgfqpoint{4.346071in}{2.127304in}}%
\pgfpathlineto{\pgfqpoint{4.353850in}{2.135853in}}%
\pgfpathlineto{\pgfqpoint{4.361624in}{2.144395in}}%
\pgfpathlineto{\pgfqpoint{4.369392in}{2.152927in}}%
\pgfpathclose%
\pgfusepath{fill}%
\end{pgfscope}%
\begin{pgfscope}%
\pgfpathrectangle{\pgfqpoint{1.150000in}{0.150000in}}{\pgfqpoint{5.700000in}{5.700000in}}%
\pgfusepath{clip}%
\pgfsetbuttcap%
\pgfsetroundjoin%
\definecolor{currentfill}{rgb}{0.281924,0.089666,0.412415}%
\pgfsetfillcolor{currentfill}%
\pgfsetfillopacity{0.700000}%
\pgfsetlinewidth{0.000000pt}%
\definecolor{currentstroke}{rgb}{0.000000,0.000000,0.000000}%
\pgfsetstrokecolor{currentstroke}%
\pgfsetdash{}{0pt}%
\pgfpathmoveto{\pgfqpoint{3.018737in}{2.255170in}}%
\pgfpathlineto{\pgfqpoint{3.032188in}{2.248426in}}%
\pgfpathlineto{\pgfqpoint{3.045643in}{2.241720in}}%
\pgfpathlineto{\pgfqpoint{3.059101in}{2.235052in}}%
\pgfpathlineto{\pgfqpoint{3.072564in}{2.228421in}}%
\pgfpathlineto{\pgfqpoint{3.064200in}{2.226509in}}%
\pgfpathlineto{\pgfqpoint{3.055824in}{2.224815in}}%
\pgfpathlineto{\pgfqpoint{3.047435in}{2.223346in}}%
\pgfpathlineto{\pgfqpoint{3.039033in}{2.222107in}}%
\pgfpathlineto{\pgfqpoint{3.025546in}{2.228981in}}%
\pgfpathlineto{\pgfqpoint{3.012062in}{2.235892in}}%
\pgfpathlineto{\pgfqpoint{2.998582in}{2.242841in}}%
\pgfpathlineto{\pgfqpoint{2.985105in}{2.249828in}}%
\pgfpathlineto{\pgfqpoint{2.993533in}{2.250818in}}%
\pgfpathlineto{\pgfqpoint{3.001947in}{2.252043in}}%
\pgfpathlineto{\pgfqpoint{3.010348in}{2.253496in}}%
\pgfpathlineto{\pgfqpoint{3.018737in}{2.255170in}}%
\pgfpathclose%
\pgfusepath{fill}%
\end{pgfscope}%
\begin{pgfscope}%
\pgfpathrectangle{\pgfqpoint{1.150000in}{0.150000in}}{\pgfqpoint{5.700000in}{5.700000in}}%
\pgfusepath{clip}%
\pgfsetbuttcap%
\pgfsetroundjoin%
\definecolor{currentfill}{rgb}{0.283072,0.130895,0.449241}%
\pgfsetfillcolor{currentfill}%
\pgfsetfillopacity{0.700000}%
\pgfsetlinewidth{0.000000pt}%
\definecolor{currentstroke}{rgb}{0.000000,0.000000,0.000000}%
\pgfsetstrokecolor{currentstroke}%
\pgfsetdash{}{0pt}%
\pgfpathmoveto{\pgfqpoint{5.220978in}{2.332872in}}%
\pgfpathlineto{\pgfqpoint{5.234928in}{2.331657in}}%
\pgfpathlineto{\pgfqpoint{5.248887in}{2.330467in}}%
\pgfpathlineto{\pgfqpoint{5.262854in}{2.329302in}}%
\pgfpathlineto{\pgfqpoint{5.276829in}{2.328162in}}%
\pgfpathlineto{\pgfqpoint{5.269400in}{2.320698in}}%
\pgfpathlineto{\pgfqpoint{5.261964in}{2.313156in}}%
\pgfpathlineto{\pgfqpoint{5.254520in}{2.305535in}}%
\pgfpathlineto{\pgfqpoint{5.247070in}{2.297835in}}%
\pgfpathlineto{\pgfqpoint{5.233082in}{2.298976in}}%
\pgfpathlineto{\pgfqpoint{5.219102in}{2.300142in}}%
\pgfpathlineto{\pgfqpoint{5.205130in}{2.301333in}}%
\pgfpathlineto{\pgfqpoint{5.191167in}{2.302548in}}%
\pgfpathlineto{\pgfqpoint{5.198630in}{2.310242in}}%
\pgfpathlineto{\pgfqpoint{5.206087in}{2.317860in}}%
\pgfpathlineto{\pgfqpoint{5.213536in}{2.325403in}}%
\pgfpathlineto{\pgfqpoint{5.220978in}{2.332872in}}%
\pgfpathclose%
\pgfusepath{fill}%
\end{pgfscope}%
\begin{pgfscope}%
\pgfpathrectangle{\pgfqpoint{1.150000in}{0.150000in}}{\pgfqpoint{5.700000in}{5.700000in}}%
\pgfusepath{clip}%
\pgfsetbuttcap%
\pgfsetroundjoin%
\definecolor{currentfill}{rgb}{0.283072,0.130895,0.449241}%
\pgfsetfillcolor{currentfill}%
\pgfsetfillopacity{0.700000}%
\pgfsetlinewidth{0.000000pt}%
\definecolor{currentstroke}{rgb}{0.000000,0.000000,0.000000}%
\pgfsetstrokecolor{currentstroke}%
\pgfsetdash{}{0pt}%
\pgfpathmoveto{\pgfqpoint{2.823661in}{2.336733in}}%
\pgfpathlineto{\pgfqpoint{2.837096in}{2.329268in}}%
\pgfpathlineto{\pgfqpoint{2.850535in}{2.321844in}}%
\pgfpathlineto{\pgfqpoint{2.863976in}{2.314462in}}%
\pgfpathlineto{\pgfqpoint{2.877421in}{2.307120in}}%
\pgfpathlineto{\pgfqpoint{2.868926in}{2.306875in}}%
\pgfpathlineto{\pgfqpoint{2.860416in}{2.306883in}}%
\pgfpathlineto{\pgfqpoint{2.851891in}{2.307153in}}%
\pgfpathlineto{\pgfqpoint{2.843351in}{2.307690in}}%
\pgfpathlineto{\pgfqpoint{2.829878in}{2.315289in}}%
\pgfpathlineto{\pgfqpoint{2.816407in}{2.322929in}}%
\pgfpathlineto{\pgfqpoint{2.802940in}{2.330611in}}%
\pgfpathlineto{\pgfqpoint{2.789476in}{2.338334in}}%
\pgfpathlineto{\pgfqpoint{2.798045in}{2.337533in}}%
\pgfpathlineto{\pgfqpoint{2.806599in}{2.337004in}}%
\pgfpathlineto{\pgfqpoint{2.815137in}{2.336740in}}%
\pgfpathlineto{\pgfqpoint{2.823661in}{2.336733in}}%
\pgfpathclose%
\pgfusepath{fill}%
\end{pgfscope}%
\begin{pgfscope}%
\pgfpathrectangle{\pgfqpoint{1.150000in}{0.150000in}}{\pgfqpoint{5.700000in}{5.700000in}}%
\pgfusepath{clip}%
\pgfsetbuttcap%
\pgfsetroundjoin%
\definecolor{currentfill}{rgb}{0.282327,0.094955,0.417331}%
\pgfsetfillcolor{currentfill}%
\pgfsetfillopacity{0.700000}%
\pgfsetlinewidth{0.000000pt}%
\definecolor{currentstroke}{rgb}{0.000000,0.000000,0.000000}%
\pgfsetstrokecolor{currentstroke}%
\pgfsetdash{}{0pt}%
\pgfpathmoveto{\pgfqpoint{4.908516in}{2.261784in}}%
\pgfpathlineto{\pgfqpoint{4.922374in}{2.260124in}}%
\pgfpathlineto{\pgfqpoint{4.936240in}{2.258491in}}%
\pgfpathlineto{\pgfqpoint{4.950114in}{2.256882in}}%
\pgfpathlineto{\pgfqpoint{4.963996in}{2.255299in}}%
\pgfpathlineto{\pgfqpoint{4.956436in}{2.247029in}}%
\pgfpathlineto{\pgfqpoint{4.948869in}{2.238695in}}%
\pgfpathlineto{\pgfqpoint{4.941296in}{2.230296in}}%
\pgfpathlineto{\pgfqpoint{4.933716in}{2.221833in}}%
\pgfpathlineto{\pgfqpoint{4.919823in}{2.223458in}}%
\pgfpathlineto{\pgfqpoint{4.905937in}{2.225108in}}%
\pgfpathlineto{\pgfqpoint{4.892060in}{2.226783in}}%
\pgfpathlineto{\pgfqpoint{4.878190in}{2.228484in}}%
\pgfpathlineto{\pgfqpoint{4.885781in}{2.236900in}}%
\pgfpathlineto{\pgfqpoint{4.893365in}{2.245256in}}%
\pgfpathlineto{\pgfqpoint{4.900944in}{2.253551in}}%
\pgfpathlineto{\pgfqpoint{4.908516in}{2.261784in}}%
\pgfpathclose%
\pgfusepath{fill}%
\end{pgfscope}%
\begin{pgfscope}%
\pgfpathrectangle{\pgfqpoint{1.150000in}{0.150000in}}{\pgfqpoint{5.700000in}{5.700000in}}%
\pgfusepath{clip}%
\pgfsetbuttcap%
\pgfsetroundjoin%
\definecolor{currentfill}{rgb}{0.267004,0.004874,0.329415}%
\pgfsetfillcolor{currentfill}%
\pgfsetfillopacity{0.700000}%
\pgfsetlinewidth{0.000000pt}%
\definecolor{currentstroke}{rgb}{0.000000,0.000000,0.000000}%
\pgfsetstrokecolor{currentstroke}%
\pgfsetdash{}{0pt}%
\pgfpathmoveto{\pgfqpoint{3.916314in}{2.102529in}}%
\pgfpathlineto{\pgfqpoint{3.929913in}{2.098635in}}%
\pgfpathlineto{\pgfqpoint{3.943519in}{2.094770in}}%
\pgfpathlineto{\pgfqpoint{3.957131in}{2.090933in}}%
\pgfpathlineto{\pgfqpoint{3.970749in}{2.087124in}}%
\pgfpathlineto{\pgfqpoint{3.962828in}{2.079551in}}%
\pgfpathlineto{\pgfqpoint{3.954901in}{2.072033in}}%
\pgfpathlineto{\pgfqpoint{3.946967in}{2.064575in}}%
\pgfpathlineto{\pgfqpoint{3.939028in}{2.057180in}}%
\pgfpathlineto{\pgfqpoint{3.925396in}{2.061150in}}%
\pgfpathlineto{\pgfqpoint{3.911770in}{2.065149in}}%
\pgfpathlineto{\pgfqpoint{3.898151in}{2.069175in}}%
\pgfpathlineto{\pgfqpoint{3.884537in}{2.073230in}}%
\pgfpathlineto{\pgfqpoint{3.892491in}{2.080459in}}%
\pgfpathlineto{\pgfqpoint{3.900438in}{2.087754in}}%
\pgfpathlineto{\pgfqpoint{3.908379in}{2.095112in}}%
\pgfpathlineto{\pgfqpoint{3.916314in}{2.102529in}}%
\pgfpathclose%
\pgfusepath{fill}%
\end{pgfscope}%
\begin{pgfscope}%
\pgfpathrectangle{\pgfqpoint{1.150000in}{0.150000in}}{\pgfqpoint{5.700000in}{5.700000in}}%
\pgfusepath{clip}%
\pgfsetbuttcap%
\pgfsetroundjoin%
\definecolor{currentfill}{rgb}{0.274128,0.199721,0.498911}%
\pgfsetfillcolor{currentfill}%
\pgfsetfillopacity{0.700000}%
\pgfsetlinewidth{0.000000pt}%
\definecolor{currentstroke}{rgb}{0.000000,0.000000,0.000000}%
\pgfsetstrokecolor{currentstroke}%
\pgfsetdash{}{0pt}%
\pgfpathmoveto{\pgfqpoint{5.986777in}{2.472560in}}%
\pgfpathlineto{\pgfqpoint{6.000961in}{2.471947in}}%
\pgfpathlineto{\pgfqpoint{6.015154in}{2.471359in}}%
\pgfpathlineto{\pgfqpoint{6.029356in}{2.470795in}}%
\pgfpathlineto{\pgfqpoint{6.022303in}{2.465876in}}%
\pgfpathlineto{\pgfqpoint{6.015243in}{2.460900in}}%
\pgfpathlineto{\pgfqpoint{6.008174in}{2.455863in}}%
\pgfpathlineto{\pgfqpoint{6.001097in}{2.450764in}}%
\pgfpathlineto{\pgfqpoint{5.986876in}{2.451232in}}%
\pgfpathlineto{\pgfqpoint{5.972663in}{2.451724in}}%
\pgfpathlineto{\pgfqpoint{5.958460in}{2.452241in}}%
\pgfpathlineto{\pgfqpoint{5.965551in}{2.457409in}}%
\pgfpathlineto{\pgfqpoint{5.972634in}{2.462516in}}%
\pgfpathlineto{\pgfqpoint{5.979709in}{2.467565in}}%
\pgfpathlineto{\pgfqpoint{5.986777in}{2.472560in}}%
\pgfpathclose%
\pgfusepath{fill}%
\end{pgfscope}%
\begin{pgfscope}%
\pgfpathrectangle{\pgfqpoint{1.150000in}{0.150000in}}{\pgfqpoint{5.700000in}{5.700000in}}%
\pgfusepath{clip}%
\pgfsetbuttcap%
\pgfsetroundjoin%
\definecolor{currentfill}{rgb}{0.277134,0.185228,0.489898}%
\pgfsetfillcolor{currentfill}%
\pgfsetfillopacity{0.700000}%
\pgfsetlinewidth{0.000000pt}%
\definecolor{currentstroke}{rgb}{0.000000,0.000000,0.000000}%
\pgfsetstrokecolor{currentstroke}%
\pgfsetdash{}{0pt}%
\pgfpathmoveto{\pgfqpoint{5.760054in}{2.438019in}}%
\pgfpathlineto{\pgfqpoint{5.774168in}{2.437292in}}%
\pgfpathlineto{\pgfqpoint{5.788291in}{2.436589in}}%
\pgfpathlineto{\pgfqpoint{5.802423in}{2.435911in}}%
\pgfpathlineto{\pgfqpoint{5.816564in}{2.435257in}}%
\pgfpathlineto{\pgfqpoint{5.809396in}{2.429595in}}%
\pgfpathlineto{\pgfqpoint{5.802220in}{2.423862in}}%
\pgfpathlineto{\pgfqpoint{5.795036in}{2.418056in}}%
\pgfpathlineto{\pgfqpoint{5.787843in}{2.412175in}}%
\pgfpathlineto{\pgfqpoint{5.773685in}{2.412761in}}%
\pgfpathlineto{\pgfqpoint{5.759536in}{2.413371in}}%
\pgfpathlineto{\pgfqpoint{5.745396in}{2.414006in}}%
\pgfpathlineto{\pgfqpoint{5.731265in}{2.414665in}}%
\pgfpathlineto{\pgfqpoint{5.738474in}{2.420609in}}%
\pgfpathlineto{\pgfqpoint{5.745675in}{2.426481in}}%
\pgfpathlineto{\pgfqpoint{5.752869in}{2.432284in}}%
\pgfpathlineto{\pgfqpoint{5.760054in}{2.438019in}}%
\pgfpathclose%
\pgfusepath{fill}%
\end{pgfscope}%
\begin{pgfscope}%
\pgfpathrectangle{\pgfqpoint{1.150000in}{0.150000in}}{\pgfqpoint{5.700000in}{5.700000in}}%
\pgfusepath{clip}%
\pgfsetbuttcap%
\pgfsetroundjoin%
\definecolor{currentfill}{rgb}{0.277941,0.056324,0.381191}%
\pgfsetfillcolor{currentfill}%
\pgfsetfillopacity{0.700000}%
\pgfsetlinewidth{0.000000pt}%
\definecolor{currentstroke}{rgb}{0.000000,0.000000,0.000000}%
\pgfsetstrokecolor{currentstroke}%
\pgfsetdash{}{0pt}%
\pgfpathmoveto{\pgfqpoint{4.596013in}{2.191738in}}%
\pgfpathlineto{\pgfqpoint{4.609784in}{2.189505in}}%
\pgfpathlineto{\pgfqpoint{4.623562in}{2.187298in}}%
\pgfpathlineto{\pgfqpoint{4.637347in}{2.185116in}}%
\pgfpathlineto{\pgfqpoint{4.651139in}{2.182960in}}%
\pgfpathlineto{\pgfqpoint{4.643461in}{2.174275in}}%
\pgfpathlineto{\pgfqpoint{4.635777in}{2.165550in}}%
\pgfpathlineto{\pgfqpoint{4.628087in}{2.156788in}}%
\pgfpathlineto{\pgfqpoint{4.620391in}{2.147991in}}%
\pgfpathlineto{\pgfqpoint{4.606588in}{2.150229in}}%
\pgfpathlineto{\pgfqpoint{4.592792in}{2.152492in}}%
\pgfpathlineto{\pgfqpoint{4.579003in}{2.154781in}}%
\pgfpathlineto{\pgfqpoint{4.565222in}{2.157097in}}%
\pgfpathlineto{\pgfqpoint{4.572928in}{2.165807in}}%
\pgfpathlineto{\pgfqpoint{4.580629in}{2.174485in}}%
\pgfpathlineto{\pgfqpoint{4.588324in}{2.183130in}}%
\pgfpathlineto{\pgfqpoint{4.596013in}{2.191738in}}%
\pgfpathclose%
\pgfusepath{fill}%
\end{pgfscope}%
\begin{pgfscope}%
\pgfpathrectangle{\pgfqpoint{1.150000in}{0.150000in}}{\pgfqpoint{5.700000in}{5.700000in}}%
\pgfusepath{clip}%
\pgfsetbuttcap%
\pgfsetroundjoin%
\definecolor{currentfill}{rgb}{0.277941,0.056324,0.381191}%
\pgfsetfillcolor{currentfill}%
\pgfsetfillopacity{0.700000}%
\pgfsetlinewidth{0.000000pt}%
\definecolor{currentstroke}{rgb}{0.000000,0.000000,0.000000}%
\pgfsetstrokecolor{currentstroke}%
\pgfsetdash{}{0pt}%
\pgfpathmoveto{\pgfqpoint{3.213560in}{2.188220in}}%
\pgfpathlineto{\pgfqpoint{3.227037in}{2.182141in}}%
\pgfpathlineto{\pgfqpoint{3.240518in}{2.176096in}}%
\pgfpathlineto{\pgfqpoint{3.254003in}{2.170086in}}%
\pgfpathlineto{\pgfqpoint{3.267493in}{2.164111in}}%
\pgfpathlineto{\pgfqpoint{3.259244in}{2.160715in}}%
\pgfpathlineto{\pgfqpoint{3.250984in}{2.157504in}}%
\pgfpathlineto{\pgfqpoint{3.242713in}{2.154483in}}%
\pgfpathlineto{\pgfqpoint{3.234432in}{2.151657in}}%
\pgfpathlineto{\pgfqpoint{3.220919in}{2.157861in}}%
\pgfpathlineto{\pgfqpoint{3.207412in}{2.164100in}}%
\pgfpathlineto{\pgfqpoint{3.193908in}{2.170373in}}%
\pgfpathlineto{\pgfqpoint{3.180409in}{2.176681in}}%
\pgfpathlineto{\pgfqpoint{3.188713in}{2.179273in}}%
\pgfpathlineto{\pgfqpoint{3.197006in}{2.182064in}}%
\pgfpathlineto{\pgfqpoint{3.205289in}{2.185048in}}%
\pgfpathlineto{\pgfqpoint{3.213560in}{2.188220in}}%
\pgfpathclose%
\pgfusepath{fill}%
\end{pgfscope}%
\begin{pgfscope}%
\pgfpathrectangle{\pgfqpoint{1.150000in}{0.150000in}}{\pgfqpoint{5.700000in}{5.700000in}}%
\pgfusepath{clip}%
\pgfsetbuttcap%
\pgfsetroundjoin%
\definecolor{currentfill}{rgb}{0.281412,0.155834,0.469201}%
\pgfsetfillcolor{currentfill}%
\pgfsetfillopacity{0.700000}%
\pgfsetlinewidth{0.000000pt}%
\definecolor{currentstroke}{rgb}{0.000000,0.000000,0.000000}%
\pgfsetstrokecolor{currentstroke}%
\pgfsetdash{}{0pt}%
\pgfpathmoveto{\pgfqpoint{5.447884in}{2.376631in}}%
\pgfpathlineto{\pgfqpoint{5.461907in}{2.375674in}}%
\pgfpathlineto{\pgfqpoint{5.475939in}{2.374741in}}%
\pgfpathlineto{\pgfqpoint{5.489979in}{2.373833in}}%
\pgfpathlineto{\pgfqpoint{5.504029in}{2.372950in}}%
\pgfpathlineto{\pgfqpoint{5.496701in}{2.366172in}}%
\pgfpathlineto{\pgfqpoint{5.489366in}{2.359313in}}%
\pgfpathlineto{\pgfqpoint{5.482023in}{2.352373in}}%
\pgfpathlineto{\pgfqpoint{5.474672in}{2.345351in}}%
\pgfpathlineto{\pgfqpoint{5.460609in}{2.346207in}}%
\pgfpathlineto{\pgfqpoint{5.446554in}{2.347088in}}%
\pgfpathlineto{\pgfqpoint{5.432507in}{2.347994in}}%
\pgfpathlineto{\pgfqpoint{5.418470in}{2.348924in}}%
\pgfpathlineto{\pgfqpoint{5.425835in}{2.355968in}}%
\pgfpathlineto{\pgfqpoint{5.433192in}{2.362933in}}%
\pgfpathlineto{\pgfqpoint{5.440542in}{2.369820in}}%
\pgfpathlineto{\pgfqpoint{5.447884in}{2.376631in}}%
\pgfpathclose%
\pgfusepath{fill}%
\end{pgfscope}%
\begin{pgfscope}%
\pgfpathrectangle{\pgfqpoint{1.150000in}{0.150000in}}{\pgfqpoint{5.700000in}{5.700000in}}%
\pgfusepath{clip}%
\pgfsetbuttcap%
\pgfsetroundjoin%
\definecolor{currentfill}{rgb}{0.271305,0.019942,0.347269}%
\pgfsetfillcolor{currentfill}%
\pgfsetfillopacity{0.700000}%
\pgfsetlinewidth{0.000000pt}%
\definecolor{currentstroke}{rgb}{0.000000,0.000000,0.000000}%
\pgfsetstrokecolor{currentstroke}%
\pgfsetdash{}{0pt}%
\pgfpathmoveto{\pgfqpoint{4.283483in}{2.130346in}}%
\pgfpathlineto{\pgfqpoint{4.297173in}{2.127407in}}%
\pgfpathlineto{\pgfqpoint{4.310871in}{2.124494in}}%
\pgfpathlineto{\pgfqpoint{4.324575in}{2.121608in}}%
\pgfpathlineto{\pgfqpoint{4.338286in}{2.118749in}}%
\pgfpathlineto{\pgfqpoint{4.330496in}{2.110192in}}%
\pgfpathlineto{\pgfqpoint{4.322700in}{2.101635in}}%
\pgfpathlineto{\pgfqpoint{4.314899in}{2.093080in}}%
\pgfpathlineto{\pgfqpoint{4.307092in}{2.084531in}}%
\pgfpathlineto{\pgfqpoint{4.293369in}{2.087512in}}%
\pgfpathlineto{\pgfqpoint{4.279653in}{2.090520in}}%
\pgfpathlineto{\pgfqpoint{4.265944in}{2.093554in}}%
\pgfpathlineto{\pgfqpoint{4.252242in}{2.096615in}}%
\pgfpathlineto{\pgfqpoint{4.260060in}{2.105037in}}%
\pgfpathlineto{\pgfqpoint{4.267873in}{2.113468in}}%
\pgfpathlineto{\pgfqpoint{4.275681in}{2.121905in}}%
\pgfpathlineto{\pgfqpoint{4.283483in}{2.130346in}}%
\pgfpathclose%
\pgfusepath{fill}%
\end{pgfscope}%
\begin{pgfscope}%
\pgfpathrectangle{\pgfqpoint{1.150000in}{0.150000in}}{\pgfqpoint{5.700000in}{5.700000in}}%
\pgfusepath{clip}%
\pgfsetbuttcap%
\pgfsetroundjoin%
\definecolor{currentfill}{rgb}{0.283229,0.120777,0.440584}%
\pgfsetfillcolor{currentfill}%
\pgfsetfillopacity{0.700000}%
\pgfsetlinewidth{0.000000pt}%
\definecolor{currentstroke}{rgb}{0.000000,0.000000,0.000000}%
\pgfsetstrokecolor{currentstroke}%
\pgfsetdash{}{0pt}%
\pgfpathmoveto{\pgfqpoint{5.135396in}{2.307659in}}%
\pgfpathlineto{\pgfqpoint{5.149326in}{2.306344in}}%
\pgfpathlineto{\pgfqpoint{5.163265in}{2.305054in}}%
\pgfpathlineto{\pgfqpoint{5.177212in}{2.303788in}}%
\pgfpathlineto{\pgfqpoint{5.191167in}{2.302548in}}%
\pgfpathlineto{\pgfqpoint{5.183696in}{2.294779in}}%
\pgfpathlineto{\pgfqpoint{5.176219in}{2.286933in}}%
\pgfpathlineto{\pgfqpoint{5.168734in}{2.279011in}}%
\pgfpathlineto{\pgfqpoint{5.161243in}{2.271013in}}%
\pgfpathlineto{\pgfqpoint{5.147276in}{2.272267in}}%
\pgfpathlineto{\pgfqpoint{5.133317in}{2.273547in}}%
\pgfpathlineto{\pgfqpoint{5.119366in}{2.274852in}}%
\pgfpathlineto{\pgfqpoint{5.105423in}{2.276181in}}%
\pgfpathlineto{\pgfqpoint{5.112927in}{2.284160in}}%
\pgfpathlineto{\pgfqpoint{5.120423in}{2.292066in}}%
\pgfpathlineto{\pgfqpoint{5.127913in}{2.299899in}}%
\pgfpathlineto{\pgfqpoint{5.135396in}{2.307659in}}%
\pgfpathclose%
\pgfusepath{fill}%
\end{pgfscope}%
\begin{pgfscope}%
\pgfpathrectangle{\pgfqpoint{1.150000in}{0.150000in}}{\pgfqpoint{5.700000in}{5.700000in}}%
\pgfusepath{clip}%
\pgfsetbuttcap%
\pgfsetroundjoin%
\definecolor{currentfill}{rgb}{0.267004,0.004874,0.329415}%
\pgfsetfillcolor{currentfill}%
\pgfsetfillopacity{0.700000}%
\pgfsetlinewidth{0.000000pt}%
\definecolor{currentstroke}{rgb}{0.000000,0.000000,0.000000}%
\pgfsetstrokecolor{currentstroke}%
\pgfsetdash{}{0pt}%
\pgfpathmoveto{\pgfqpoint{4.056853in}{2.103534in}}%
\pgfpathlineto{\pgfqpoint{4.070489in}{2.100013in}}%
\pgfpathlineto{\pgfqpoint{4.084132in}{2.096519in}}%
\pgfpathlineto{\pgfqpoint{4.097780in}{2.093052in}}%
\pgfpathlineto{\pgfqpoint{4.111436in}{2.089613in}}%
\pgfpathlineto{\pgfqpoint{4.103564in}{2.081568in}}%
\pgfpathlineto{\pgfqpoint{4.095687in}{2.073557in}}%
\pgfpathlineto{\pgfqpoint{4.087805in}{2.065584in}}%
\pgfpathlineto{\pgfqpoint{4.079916in}{2.057653in}}%
\pgfpathlineto{\pgfqpoint{4.066248in}{2.061240in}}%
\pgfpathlineto{\pgfqpoint{4.052586in}{2.064855in}}%
\pgfpathlineto{\pgfqpoint{4.038931in}{2.068498in}}%
\pgfpathlineto{\pgfqpoint{4.025282in}{2.072167in}}%
\pgfpathlineto{\pgfqpoint{4.033184in}{2.079946in}}%
\pgfpathlineto{\pgfqpoint{4.041080in}{2.087768in}}%
\pgfpathlineto{\pgfqpoint{4.048969in}{2.095632in}}%
\pgfpathlineto{\pgfqpoint{4.056853in}{2.103534in}}%
\pgfpathclose%
\pgfusepath{fill}%
\end{pgfscope}%
\begin{pgfscope}%
\pgfpathrectangle{\pgfqpoint{1.150000in}{0.150000in}}{\pgfqpoint{5.700000in}{5.700000in}}%
\pgfusepath{clip}%
\pgfsetbuttcap%
\pgfsetroundjoin%
\definecolor{currentfill}{rgb}{0.277134,0.185228,0.489898}%
\pgfsetfillcolor{currentfill}%
\pgfsetfillopacity{0.700000}%
\pgfsetlinewidth{0.000000pt}%
\definecolor{currentstroke}{rgb}{0.000000,0.000000,0.000000}%
\pgfsetstrokecolor{currentstroke}%
\pgfsetdash{}{0pt}%
\pgfpathmoveto{\pgfqpoint{2.628129in}{2.434417in}}%
\pgfpathlineto{\pgfqpoint{2.641560in}{2.426161in}}%
\pgfpathlineto{\pgfqpoint{2.654993in}{2.417952in}}%
\pgfpathlineto{\pgfqpoint{2.668429in}{2.409789in}}%
\pgfpathlineto{\pgfqpoint{2.681868in}{2.401672in}}%
\pgfpathlineto{\pgfqpoint{2.673221in}{2.403285in}}%
\pgfpathlineto{\pgfqpoint{2.664557in}{2.405191in}}%
\pgfpathlineto{\pgfqpoint{2.655876in}{2.407397in}}%
\pgfpathlineto{\pgfqpoint{2.647176in}{2.409910in}}%
\pgfpathlineto{\pgfqpoint{2.633706in}{2.418300in}}%
\pgfpathlineto{\pgfqpoint{2.620238in}{2.426736in}}%
\pgfpathlineto{\pgfqpoint{2.606772in}{2.435219in}}%
\pgfpathlineto{\pgfqpoint{2.593308in}{2.443749in}}%
\pgfpathlineto{\pgfqpoint{2.602041in}{2.440957in}}%
\pgfpathlineto{\pgfqpoint{2.610755in}{2.438476in}}%
\pgfpathlineto{\pgfqpoint{2.619451in}{2.436298in}}%
\pgfpathlineto{\pgfqpoint{2.628129in}{2.434417in}}%
\pgfpathclose%
\pgfusepath{fill}%
\end{pgfscope}%
\begin{pgfscope}%
\pgfpathrectangle{\pgfqpoint{1.150000in}{0.150000in}}{\pgfqpoint{5.700000in}{5.700000in}}%
\pgfusepath{clip}%
\pgfsetbuttcap%
\pgfsetroundjoin%
\definecolor{currentfill}{rgb}{0.281446,0.084320,0.407414}%
\pgfsetfillcolor{currentfill}%
\pgfsetfillopacity{0.700000}%
\pgfsetlinewidth{0.000000pt}%
\definecolor{currentstroke}{rgb}{0.000000,0.000000,0.000000}%
\pgfsetstrokecolor{currentstroke}%
\pgfsetdash{}{0pt}%
\pgfpathmoveto{\pgfqpoint{4.822789in}{2.235540in}}%
\pgfpathlineto{\pgfqpoint{4.836627in}{2.233738in}}%
\pgfpathlineto{\pgfqpoint{4.850474in}{2.231961in}}%
\pgfpathlineto{\pgfqpoint{4.864328in}{2.230210in}}%
\pgfpathlineto{\pgfqpoint{4.878190in}{2.228484in}}%
\pgfpathlineto{\pgfqpoint{4.870593in}{2.220008in}}%
\pgfpathlineto{\pgfqpoint{4.862989in}{2.211473in}}%
\pgfpathlineto{\pgfqpoint{4.855380in}{2.202881in}}%
\pgfpathlineto{\pgfqpoint{4.847764in}{2.194231in}}%
\pgfpathlineto{\pgfqpoint{4.833891in}{2.196012in}}%
\pgfpathlineto{\pgfqpoint{4.820025in}{2.197819in}}%
\pgfpathlineto{\pgfqpoint{4.806168in}{2.199650in}}%
\pgfpathlineto{\pgfqpoint{4.792318in}{2.201508in}}%
\pgfpathlineto{\pgfqpoint{4.799945in}{2.210097in}}%
\pgfpathlineto{\pgfqpoint{4.807566in}{2.218633in}}%
\pgfpathlineto{\pgfqpoint{4.815180in}{2.227114in}}%
\pgfpathlineto{\pgfqpoint{4.822789in}{2.235540in}}%
\pgfpathclose%
\pgfusepath{fill}%
\end{pgfscope}%
\begin{pgfscope}%
\pgfpathrectangle{\pgfqpoint{1.150000in}{0.150000in}}{\pgfqpoint{5.700000in}{5.700000in}}%
\pgfusepath{clip}%
\pgfsetbuttcap%
\pgfsetroundjoin%
\definecolor{currentfill}{rgb}{0.276022,0.044167,0.370164}%
\pgfsetfillcolor{currentfill}%
\pgfsetfillopacity{0.700000}%
\pgfsetlinewidth{0.000000pt}%
\definecolor{currentstroke}{rgb}{0.000000,0.000000,0.000000}%
\pgfsetstrokecolor{currentstroke}%
\pgfsetdash{}{0pt}%
\pgfpathmoveto{\pgfqpoint{4.510169in}{2.166616in}}%
\pgfpathlineto{\pgfqpoint{4.523921in}{2.164198in}}%
\pgfpathlineto{\pgfqpoint{4.537681in}{2.161805in}}%
\pgfpathlineto{\pgfqpoint{4.551448in}{2.159438in}}%
\pgfpathlineto{\pgfqpoint{4.565222in}{2.157097in}}%
\pgfpathlineto{\pgfqpoint{4.557509in}{2.148356in}}%
\pgfpathlineto{\pgfqpoint{4.549791in}{2.139586in}}%
\pgfpathlineto{\pgfqpoint{4.542068in}{2.130790in}}%
\pgfpathlineto{\pgfqpoint{4.534339in}{2.121969in}}%
\pgfpathlineto{\pgfqpoint{4.520554in}{2.124406in}}%
\pgfpathlineto{\pgfqpoint{4.506776in}{2.126868in}}%
\pgfpathlineto{\pgfqpoint{4.493006in}{2.129356in}}%
\pgfpathlineto{\pgfqpoint{4.479242in}{2.131871in}}%
\pgfpathlineto{\pgfqpoint{4.486983in}{2.140591in}}%
\pgfpathlineto{\pgfqpoint{4.494717in}{2.149290in}}%
\pgfpathlineto{\pgfqpoint{4.502446in}{2.157966in}}%
\pgfpathlineto{\pgfqpoint{4.510169in}{2.166616in}}%
\pgfpathclose%
\pgfusepath{fill}%
\end{pgfscope}%
\begin{pgfscope}%
\pgfpathrectangle{\pgfqpoint{1.150000in}{0.150000in}}{\pgfqpoint{5.700000in}{5.700000in}}%
\pgfusepath{clip}%
\pgfsetbuttcap%
\pgfsetroundjoin%
\definecolor{currentfill}{rgb}{0.269944,0.014625,0.341379}%
\pgfsetfillcolor{currentfill}%
\pgfsetfillopacity{0.700000}%
\pgfsetlinewidth{0.000000pt}%
\definecolor{currentstroke}{rgb}{0.000000,0.000000,0.000000}%
\pgfsetstrokecolor{currentstroke}%
\pgfsetdash{}{0pt}%
\pgfpathmoveto{\pgfqpoint{3.548978in}{2.112769in}}%
\pgfpathlineto{\pgfqpoint{3.562512in}{2.107761in}}%
\pgfpathlineto{\pgfqpoint{3.576050in}{2.102783in}}%
\pgfpathlineto{\pgfqpoint{3.589594in}{2.097837in}}%
\pgfpathlineto{\pgfqpoint{3.603143in}{2.092920in}}%
\pgfpathlineto{\pgfqpoint{3.595063in}{2.087268in}}%
\pgfpathlineto{\pgfqpoint{3.586975in}{2.081741in}}%
\pgfpathlineto{\pgfqpoint{3.578879in}{2.076342in}}%
\pgfpathlineto{\pgfqpoint{3.570775in}{2.071079in}}%
\pgfpathlineto{\pgfqpoint{3.557208in}{2.076196in}}%
\pgfpathlineto{\pgfqpoint{3.543647in}{2.081344in}}%
\pgfpathlineto{\pgfqpoint{3.530090in}{2.086523in}}%
\pgfpathlineto{\pgfqpoint{3.516539in}{2.091733in}}%
\pgfpathlineto{\pgfqpoint{3.524661in}{2.096790in}}%
\pgfpathlineto{\pgfqpoint{3.532775in}{2.101985in}}%
\pgfpathlineto{\pgfqpoint{3.540881in}{2.107313in}}%
\pgfpathlineto{\pgfqpoint{3.548978in}{2.112769in}}%
\pgfpathclose%
\pgfusepath{fill}%
\end{pgfscope}%
\begin{pgfscope}%
\pgfpathrectangle{\pgfqpoint{1.150000in}{0.150000in}}{\pgfqpoint{5.700000in}{5.700000in}}%
\pgfusepath{clip}%
\pgfsetbuttcap%
\pgfsetroundjoin%
\definecolor{currentfill}{rgb}{0.267004,0.004874,0.329415}%
\pgfsetfillcolor{currentfill}%
\pgfsetfillopacity{0.700000}%
\pgfsetlinewidth{0.000000pt}%
\definecolor{currentstroke}{rgb}{0.000000,0.000000,0.000000}%
\pgfsetstrokecolor{currentstroke}%
\pgfsetdash{}{0pt}%
\pgfpathmoveto{\pgfqpoint{3.689569in}{2.098065in}}%
\pgfpathlineto{\pgfqpoint{3.703129in}{2.093487in}}%
\pgfpathlineto{\pgfqpoint{3.716694in}{2.088939in}}%
\pgfpathlineto{\pgfqpoint{3.730265in}{2.084421in}}%
\pgfpathlineto{\pgfqpoint{3.743841in}{2.079932in}}%
\pgfpathlineto{\pgfqpoint{3.735824in}{2.073464in}}%
\pgfpathlineto{\pgfqpoint{3.727800in}{2.067095in}}%
\pgfpathlineto{\pgfqpoint{3.719768in}{2.060830in}}%
\pgfpathlineto{\pgfqpoint{3.711730in}{2.054672in}}%
\pgfpathlineto{\pgfqpoint{3.698137in}{2.059349in}}%
\pgfpathlineto{\pgfqpoint{3.684550in}{2.064056in}}%
\pgfpathlineto{\pgfqpoint{3.670969in}{2.068792in}}%
\pgfpathlineto{\pgfqpoint{3.657393in}{2.073558in}}%
\pgfpathlineto{\pgfqpoint{3.665448in}{2.079522in}}%
\pgfpathlineto{\pgfqpoint{3.673496in}{2.085598in}}%
\pgfpathlineto{\pgfqpoint{3.681536in}{2.091781in}}%
\pgfpathlineto{\pgfqpoint{3.689569in}{2.098065in}}%
\pgfpathclose%
\pgfusepath{fill}%
\end{pgfscope}%
\begin{pgfscope}%
\pgfpathrectangle{\pgfqpoint{1.150000in}{0.150000in}}{\pgfqpoint{5.700000in}{5.700000in}}%
\pgfusepath{clip}%
\pgfsetbuttcap%
\pgfsetroundjoin%
\definecolor{currentfill}{rgb}{0.278012,0.180367,0.486697}%
\pgfsetfillcolor{currentfill}%
\pgfsetfillopacity{0.700000}%
\pgfsetlinewidth{0.000000pt}%
\definecolor{currentstroke}{rgb}{0.000000,0.000000,0.000000}%
\pgfsetstrokecolor{currentstroke}%
\pgfsetdash{}{0pt}%
\pgfpathmoveto{\pgfqpoint{5.674828in}{2.417546in}}%
\pgfpathlineto{\pgfqpoint{5.688924in}{2.416789in}}%
\pgfpathlineto{\pgfqpoint{5.703029in}{2.416057in}}%
\pgfpathlineto{\pgfqpoint{5.717142in}{2.415349in}}%
\pgfpathlineto{\pgfqpoint{5.731265in}{2.414665in}}%
\pgfpathlineto{\pgfqpoint{5.724047in}{2.408647in}}%
\pgfpathlineto{\pgfqpoint{5.716822in}{2.402552in}}%
\pgfpathlineto{\pgfqpoint{5.709588in}{2.396379in}}%
\pgfpathlineto{\pgfqpoint{5.702347in}{2.390125in}}%
\pgfpathlineto{\pgfqpoint{5.688208in}{2.390754in}}%
\pgfpathlineto{\pgfqpoint{5.674078in}{2.391408in}}%
\pgfpathlineto{\pgfqpoint{5.659957in}{2.392086in}}%
\pgfpathlineto{\pgfqpoint{5.645845in}{2.392789in}}%
\pgfpathlineto{\pgfqpoint{5.653102in}{2.399091in}}%
\pgfpathlineto{\pgfqpoint{5.660352in}{2.405317in}}%
\pgfpathlineto{\pgfqpoint{5.667594in}{2.411468in}}%
\pgfpathlineto{\pgfqpoint{5.674828in}{2.417546in}}%
\pgfpathclose%
\pgfusepath{fill}%
\end{pgfscope}%
\begin{pgfscope}%
\pgfpathrectangle{\pgfqpoint{1.150000in}{0.150000in}}{\pgfqpoint{5.700000in}{5.700000in}}%
\pgfusepath{clip}%
\pgfsetbuttcap%
\pgfsetroundjoin%
\definecolor{currentfill}{rgb}{0.272594,0.025563,0.353093}%
\pgfsetfillcolor{currentfill}%
\pgfsetfillopacity{0.700000}%
\pgfsetlinewidth{0.000000pt}%
\definecolor{currentstroke}{rgb}{0.000000,0.000000,0.000000}%
\pgfsetstrokecolor{currentstroke}%
\pgfsetdash{}{0pt}%
\pgfpathmoveto{\pgfqpoint{3.408309in}{2.134539in}}%
\pgfpathlineto{\pgfqpoint{3.421820in}{2.129077in}}%
\pgfpathlineto{\pgfqpoint{3.435337in}{2.123647in}}%
\pgfpathlineto{\pgfqpoint{3.448858in}{2.118249in}}%
\pgfpathlineto{\pgfqpoint{3.462384in}{2.112883in}}%
\pgfpathlineto{\pgfqpoint{3.454234in}{2.108178in}}%
\pgfpathlineto{\pgfqpoint{3.446075in}{2.103625in}}%
\pgfpathlineto{\pgfqpoint{3.437908in}{2.099229in}}%
\pgfpathlineto{\pgfqpoint{3.429731in}{2.094996in}}%
\pgfpathlineto{\pgfqpoint{3.416185in}{2.100577in}}%
\pgfpathlineto{\pgfqpoint{3.402644in}{2.106190in}}%
\pgfpathlineto{\pgfqpoint{3.389108in}{2.111834in}}%
\pgfpathlineto{\pgfqpoint{3.375576in}{2.117512in}}%
\pgfpathlineto{\pgfqpoint{3.383774in}{2.121525in}}%
\pgfpathlineto{\pgfqpoint{3.391961in}{2.125705in}}%
\pgfpathlineto{\pgfqpoint{3.400140in}{2.130044in}}%
\pgfpathlineto{\pgfqpoint{3.408309in}{2.134539in}}%
\pgfpathclose%
\pgfusepath{fill}%
\end{pgfscope}%
\begin{pgfscope}%
\pgfpathrectangle{\pgfqpoint{1.150000in}{0.150000in}}{\pgfqpoint{5.700000in}{5.700000in}}%
\pgfusepath{clip}%
\pgfsetbuttcap%
\pgfsetroundjoin%
\definecolor{currentfill}{rgb}{0.281446,0.084320,0.407414}%
\pgfsetfillcolor{currentfill}%
\pgfsetfillopacity{0.700000}%
\pgfsetlinewidth{0.000000pt}%
\definecolor{currentstroke}{rgb}{0.000000,0.000000,0.000000}%
\pgfsetstrokecolor{currentstroke}%
\pgfsetdash{}{0pt}%
\pgfpathmoveto{\pgfqpoint{3.072564in}{2.228421in}}%
\pgfpathlineto{\pgfqpoint{3.086030in}{2.221827in}}%
\pgfpathlineto{\pgfqpoint{3.099501in}{2.215269in}}%
\pgfpathlineto{\pgfqpoint{3.112975in}{2.208748in}}%
\pgfpathlineto{\pgfqpoint{3.126454in}{2.202263in}}%
\pgfpathlineto{\pgfqpoint{3.118114in}{2.200114in}}%
\pgfpathlineto{\pgfqpoint{3.109763in}{2.198179in}}%
\pgfpathlineto{\pgfqpoint{3.101399in}{2.196465in}}%
\pgfpathlineto{\pgfqpoint{3.093023in}{2.194978in}}%
\pgfpathlineto{\pgfqpoint{3.079519in}{2.201706in}}%
\pgfpathlineto{\pgfqpoint{3.066020in}{2.208470in}}%
\pgfpathlineto{\pgfqpoint{3.052525in}{2.215270in}}%
\pgfpathlineto{\pgfqpoint{3.039033in}{2.222107in}}%
\pgfpathlineto{\pgfqpoint{3.047435in}{2.223346in}}%
\pgfpathlineto{\pgfqpoint{3.055824in}{2.224815in}}%
\pgfpathlineto{\pgfqpoint{3.064200in}{2.226509in}}%
\pgfpathlineto{\pgfqpoint{3.072564in}{2.228421in}}%
\pgfpathclose%
\pgfusepath{fill}%
\end{pgfscope}%
\begin{pgfscope}%
\pgfpathrectangle{\pgfqpoint{1.150000in}{0.150000in}}{\pgfqpoint{5.700000in}{5.700000in}}%
\pgfusepath{clip}%
\pgfsetbuttcap%
\pgfsetroundjoin%
\definecolor{currentfill}{rgb}{0.283187,0.125848,0.444960}%
\pgfsetfillcolor{currentfill}%
\pgfsetfillopacity{0.700000}%
\pgfsetlinewidth{0.000000pt}%
\definecolor{currentstroke}{rgb}{0.000000,0.000000,0.000000}%
\pgfsetstrokecolor{currentstroke}%
\pgfsetdash{}{0pt}%
\pgfpathmoveto{\pgfqpoint{2.877421in}{2.307120in}}%
\pgfpathlineto{\pgfqpoint{2.890870in}{2.299820in}}%
\pgfpathlineto{\pgfqpoint{2.904322in}{2.292560in}}%
\pgfpathlineto{\pgfqpoint{2.917777in}{2.285339in}}%
\pgfpathlineto{\pgfqpoint{2.931235in}{2.278159in}}%
\pgfpathlineto{\pgfqpoint{2.922767in}{2.277661in}}%
\pgfpathlineto{\pgfqpoint{2.914285in}{2.277414in}}%
\pgfpathlineto{\pgfqpoint{2.905789in}{2.277424in}}%
\pgfpathlineto{\pgfqpoint{2.897277in}{2.277699in}}%
\pgfpathlineto{\pgfqpoint{2.883791in}{2.285136in}}%
\pgfpathlineto{\pgfqpoint{2.870307in}{2.292614in}}%
\pgfpathlineto{\pgfqpoint{2.856828in}{2.300132in}}%
\pgfpathlineto{\pgfqpoint{2.843351in}{2.307690in}}%
\pgfpathlineto{\pgfqpoint{2.851891in}{2.307153in}}%
\pgfpathlineto{\pgfqpoint{2.860416in}{2.306883in}}%
\pgfpathlineto{\pgfqpoint{2.868926in}{2.306875in}}%
\pgfpathlineto{\pgfqpoint{2.877421in}{2.307120in}}%
\pgfpathclose%
\pgfusepath{fill}%
\end{pgfscope}%
\begin{pgfscope}%
\pgfpathrectangle{\pgfqpoint{1.150000in}{0.150000in}}{\pgfqpoint{5.700000in}{5.700000in}}%
\pgfusepath{clip}%
\pgfsetbuttcap%
\pgfsetroundjoin%
\definecolor{currentfill}{rgb}{0.267004,0.004874,0.329415}%
\pgfsetfillcolor{currentfill}%
\pgfsetfillopacity{0.700000}%
\pgfsetlinewidth{0.000000pt}%
\definecolor{currentstroke}{rgb}{0.000000,0.000000,0.000000}%
\pgfsetstrokecolor{currentstroke}%
\pgfsetdash{}{0pt}%
\pgfpathmoveto{\pgfqpoint{3.830142in}{2.089735in}}%
\pgfpathlineto{\pgfqpoint{3.843732in}{2.085566in}}%
\pgfpathlineto{\pgfqpoint{3.857328in}{2.081425in}}%
\pgfpathlineto{\pgfqpoint{3.870929in}{2.077314in}}%
\pgfpathlineto{\pgfqpoint{3.884537in}{2.073230in}}%
\pgfpathlineto{\pgfqpoint{3.876577in}{2.066072in}}%
\pgfpathlineto{\pgfqpoint{3.868610in}{2.058989in}}%
\pgfpathlineto{\pgfqpoint{3.860637in}{2.051984in}}%
\pgfpathlineto{\pgfqpoint{3.852658in}{2.045063in}}%
\pgfpathlineto{\pgfqpoint{3.839035in}{2.049321in}}%
\pgfpathlineto{\pgfqpoint{3.825419in}{2.053608in}}%
\pgfpathlineto{\pgfqpoint{3.811808in}{2.057923in}}%
\pgfpathlineto{\pgfqpoint{3.798203in}{2.062267in}}%
\pgfpathlineto{\pgfqpoint{3.806198in}{2.069008in}}%
\pgfpathlineto{\pgfqpoint{3.814186in}{2.075836in}}%
\pgfpathlineto{\pgfqpoint{3.822167in}{2.082747in}}%
\pgfpathlineto{\pgfqpoint{3.830142in}{2.089735in}}%
\pgfpathclose%
\pgfusepath{fill}%
\end{pgfscope}%
\begin{pgfscope}%
\pgfpathrectangle{\pgfqpoint{1.150000in}{0.150000in}}{\pgfqpoint{5.700000in}{5.700000in}}%
\pgfusepath{clip}%
\pgfsetbuttcap%
\pgfsetroundjoin%
\definecolor{currentfill}{rgb}{0.282290,0.145912,0.461510}%
\pgfsetfillcolor{currentfill}%
\pgfsetfillopacity{0.700000}%
\pgfsetlinewidth{0.000000pt}%
\definecolor{currentstroke}{rgb}{0.000000,0.000000,0.000000}%
\pgfsetstrokecolor{currentstroke}%
\pgfsetdash{}{0pt}%
\pgfpathmoveto{\pgfqpoint{5.362404in}{2.352892in}}%
\pgfpathlineto{\pgfqpoint{5.376408in}{2.351863in}}%
\pgfpathlineto{\pgfqpoint{5.390420in}{2.350859in}}%
\pgfpathlineto{\pgfqpoint{5.404441in}{2.349879in}}%
\pgfpathlineto{\pgfqpoint{5.418470in}{2.348924in}}%
\pgfpathlineto{\pgfqpoint{5.411097in}{2.341799in}}%
\pgfpathlineto{\pgfqpoint{5.403717in}{2.334593in}}%
\pgfpathlineto{\pgfqpoint{5.396330in}{2.327305in}}%
\pgfpathlineto{\pgfqpoint{5.388935in}{2.319933in}}%
\pgfpathlineto{\pgfqpoint{5.374892in}{2.320875in}}%
\pgfpathlineto{\pgfqpoint{5.360857in}{2.321842in}}%
\pgfpathlineto{\pgfqpoint{5.346832in}{2.322834in}}%
\pgfpathlineto{\pgfqpoint{5.332814in}{2.323850in}}%
\pgfpathlineto{\pgfqpoint{5.340223in}{2.331229in}}%
\pgfpathlineto{\pgfqpoint{5.347624in}{2.338529in}}%
\pgfpathlineto{\pgfqpoint{5.355018in}{2.345750in}}%
\pgfpathlineto{\pgfqpoint{5.362404in}{2.352892in}}%
\pgfpathclose%
\pgfusepath{fill}%
\end{pgfscope}%
\begin{pgfscope}%
\pgfpathrectangle{\pgfqpoint{1.150000in}{0.150000in}}{\pgfqpoint{5.700000in}{5.700000in}}%
\pgfusepath{clip}%
\pgfsetbuttcap%
\pgfsetroundjoin%
\definecolor{currentfill}{rgb}{0.274128,0.199721,0.498911}%
\pgfsetfillcolor{currentfill}%
\pgfsetfillopacity{0.700000}%
\pgfsetlinewidth{0.000000pt}%
\definecolor{currentstroke}{rgb}{0.000000,0.000000,0.000000}%
\pgfsetstrokecolor{currentstroke}%
\pgfsetdash{}{0pt}%
\pgfpathmoveto{\pgfqpoint{5.901737in}{2.454551in}}%
\pgfpathlineto{\pgfqpoint{5.915904in}{2.453937in}}%
\pgfpathlineto{\pgfqpoint{5.930080in}{2.453347in}}%
\pgfpathlineto{\pgfqpoint{5.944265in}{2.452782in}}%
\pgfpathlineto{\pgfqpoint{5.958460in}{2.452241in}}%
\pgfpathlineto{\pgfqpoint{5.951361in}{2.447009in}}%
\pgfpathlineto{\pgfqpoint{5.944253in}{2.441711in}}%
\pgfpathlineto{\pgfqpoint{5.937138in}{2.436343in}}%
\pgfpathlineto{\pgfqpoint{5.930014in}{2.430903in}}%
\pgfpathlineto{\pgfqpoint{5.915801in}{2.431362in}}%
\pgfpathlineto{\pgfqpoint{5.901597in}{2.431845in}}%
\pgfpathlineto{\pgfqpoint{5.887402in}{2.432353in}}%
\pgfpathlineto{\pgfqpoint{5.873217in}{2.432885in}}%
\pgfpathlineto{\pgfqpoint{5.880359in}{2.438402in}}%
\pgfpathlineto{\pgfqpoint{5.887493in}{2.443850in}}%
\pgfpathlineto{\pgfqpoint{5.894619in}{2.449232in}}%
\pgfpathlineto{\pgfqpoint{5.901737in}{2.454551in}}%
\pgfpathclose%
\pgfusepath{fill}%
\end{pgfscope}%
\begin{pgfscope}%
\pgfpathrectangle{\pgfqpoint{1.150000in}{0.150000in}}{\pgfqpoint{5.700000in}{5.700000in}}%
\pgfusepath{clip}%
\pgfsetbuttcap%
\pgfsetroundjoin%
\definecolor{currentfill}{rgb}{0.283091,0.110553,0.431554}%
\pgfsetfillcolor{currentfill}%
\pgfsetfillopacity{0.700000}%
\pgfsetlinewidth{0.000000pt}%
\definecolor{currentstroke}{rgb}{0.000000,0.000000,0.000000}%
\pgfsetstrokecolor{currentstroke}%
\pgfsetdash{}{0pt}%
\pgfpathmoveto{\pgfqpoint{5.049734in}{2.281750in}}%
\pgfpathlineto{\pgfqpoint{5.063644in}{2.280320in}}%
\pgfpathlineto{\pgfqpoint{5.077562in}{2.278915in}}%
\pgfpathlineto{\pgfqpoint{5.091489in}{2.277536in}}%
\pgfpathlineto{\pgfqpoint{5.105423in}{2.276181in}}%
\pgfpathlineto{\pgfqpoint{5.097913in}{2.268129in}}%
\pgfpathlineto{\pgfqpoint{5.090396in}{2.260004in}}%
\pgfpathlineto{\pgfqpoint{5.082872in}{2.251807in}}%
\pgfpathlineto{\pgfqpoint{5.075341in}{2.243536in}}%
\pgfpathlineto{\pgfqpoint{5.061395in}{2.244919in}}%
\pgfpathlineto{\pgfqpoint{5.047456in}{2.246326in}}%
\pgfpathlineto{\pgfqpoint{5.033526in}{2.247759in}}%
\pgfpathlineto{\pgfqpoint{5.019604in}{2.249217in}}%
\pgfpathlineto{\pgfqpoint{5.027147in}{2.257454in}}%
\pgfpathlineto{\pgfqpoint{5.034682in}{2.265622in}}%
\pgfpathlineto{\pgfqpoint{5.042211in}{2.273721in}}%
\pgfpathlineto{\pgfqpoint{5.049734in}{2.281750in}}%
\pgfpathclose%
\pgfusepath{fill}%
\end{pgfscope}%
\begin{pgfscope}%
\pgfpathrectangle{\pgfqpoint{1.150000in}{0.150000in}}{\pgfqpoint{5.700000in}{5.700000in}}%
\pgfusepath{clip}%
\pgfsetbuttcap%
\pgfsetroundjoin%
\definecolor{currentfill}{rgb}{0.280267,0.073417,0.397163}%
\pgfsetfillcolor{currentfill}%
\pgfsetfillopacity{0.700000}%
\pgfsetlinewidth{0.000000pt}%
\definecolor{currentstroke}{rgb}{0.000000,0.000000,0.000000}%
\pgfsetstrokecolor{currentstroke}%
\pgfsetdash{}{0pt}%
\pgfpathmoveto{\pgfqpoint{4.736995in}{2.209191in}}%
\pgfpathlineto{\pgfqpoint{4.750815in}{2.207232in}}%
\pgfpathlineto{\pgfqpoint{4.764641in}{2.205298in}}%
\pgfpathlineto{\pgfqpoint{4.778476in}{2.203390in}}%
\pgfpathlineto{\pgfqpoint{4.792318in}{2.201508in}}%
\pgfpathlineto{\pgfqpoint{4.784685in}{2.192865in}}%
\pgfpathlineto{\pgfqpoint{4.777046in}{2.184172in}}%
\pgfpathlineto{\pgfqpoint{4.769401in}{2.175428in}}%
\pgfpathlineto{\pgfqpoint{4.761751in}{2.166636in}}%
\pgfpathlineto{\pgfqpoint{4.747898in}{2.168587in}}%
\pgfpathlineto{\pgfqpoint{4.734052in}{2.170564in}}%
\pgfpathlineto{\pgfqpoint{4.720215in}{2.172566in}}%
\pgfpathlineto{\pgfqpoint{4.706384in}{2.174594in}}%
\pgfpathlineto{\pgfqpoint{4.714046in}{2.183312in}}%
\pgfpathlineto{\pgfqpoint{4.721702in}{2.191986in}}%
\pgfpathlineto{\pgfqpoint{4.729352in}{2.200612in}}%
\pgfpathlineto{\pgfqpoint{4.736995in}{2.209191in}}%
\pgfpathclose%
\pgfusepath{fill}%
\end{pgfscope}%
\begin{pgfscope}%
\pgfpathrectangle{\pgfqpoint{1.150000in}{0.150000in}}{\pgfqpoint{5.700000in}{5.700000in}}%
\pgfusepath{clip}%
\pgfsetbuttcap%
\pgfsetroundjoin%
\definecolor{currentfill}{rgb}{0.269944,0.014625,0.341379}%
\pgfsetfillcolor{currentfill}%
\pgfsetfillopacity{0.700000}%
\pgfsetlinewidth{0.000000pt}%
\definecolor{currentstroke}{rgb}{0.000000,0.000000,0.000000}%
\pgfsetstrokecolor{currentstroke}%
\pgfsetdash{}{0pt}%
\pgfpathmoveto{\pgfqpoint{4.197499in}{2.109127in}}%
\pgfpathlineto{\pgfqpoint{4.211175in}{2.105959in}}%
\pgfpathlineto{\pgfqpoint{4.224857in}{2.102818in}}%
\pgfpathlineto{\pgfqpoint{4.238546in}{2.099703in}}%
\pgfpathlineto{\pgfqpoint{4.252242in}{2.096615in}}%
\pgfpathlineto{\pgfqpoint{4.244418in}{2.088205in}}%
\pgfpathlineto{\pgfqpoint{4.236588in}{2.079810in}}%
\pgfpathlineto{\pgfqpoint{4.228752in}{2.071432in}}%
\pgfpathlineto{\pgfqpoint{4.220911in}{2.063076in}}%
\pgfpathlineto{\pgfqpoint{4.207204in}{2.066299in}}%
\pgfpathlineto{\pgfqpoint{4.193503in}{2.069549in}}%
\pgfpathlineto{\pgfqpoint{4.179809in}{2.072825in}}%
\pgfpathlineto{\pgfqpoint{4.166121in}{2.076129in}}%
\pgfpathlineto{\pgfqpoint{4.173974in}{2.084345in}}%
\pgfpathlineto{\pgfqpoint{4.181822in}{2.092585in}}%
\pgfpathlineto{\pgfqpoint{4.189663in}{2.100847in}}%
\pgfpathlineto{\pgfqpoint{4.197499in}{2.109127in}}%
\pgfpathclose%
\pgfusepath{fill}%
\end{pgfscope}%
\begin{pgfscope}%
\pgfpathrectangle{\pgfqpoint{1.150000in}{0.150000in}}{\pgfqpoint{5.700000in}{5.700000in}}%
\pgfusepath{clip}%
\pgfsetbuttcap%
\pgfsetroundjoin%
\definecolor{currentfill}{rgb}{0.273809,0.031497,0.358853}%
\pgfsetfillcolor{currentfill}%
\pgfsetfillopacity{0.700000}%
\pgfsetlinewidth{0.000000pt}%
\definecolor{currentstroke}{rgb}{0.000000,0.000000,0.000000}%
\pgfsetstrokecolor{currentstroke}%
\pgfsetdash{}{0pt}%
\pgfpathmoveto{\pgfqpoint{4.424261in}{2.142189in}}%
\pgfpathlineto{\pgfqpoint{4.437996in}{2.139570in}}%
\pgfpathlineto{\pgfqpoint{4.451737in}{2.136977in}}%
\pgfpathlineto{\pgfqpoint{4.465486in}{2.134411in}}%
\pgfpathlineto{\pgfqpoint{4.479242in}{2.131871in}}%
\pgfpathlineto{\pgfqpoint{4.471497in}{2.123131in}}%
\pgfpathlineto{\pgfqpoint{4.463746in}{2.114376in}}%
\pgfpathlineto{\pgfqpoint{4.455989in}{2.105605in}}%
\pgfpathlineto{\pgfqpoint{4.448226in}{2.096823in}}%
\pgfpathlineto{\pgfqpoint{4.434459in}{2.099472in}}%
\pgfpathlineto{\pgfqpoint{4.420699in}{2.102148in}}%
\pgfpathlineto{\pgfqpoint{4.406946in}{2.104849in}}%
\pgfpathlineto{\pgfqpoint{4.393200in}{2.107576in}}%
\pgfpathlineto{\pgfqpoint{4.400974in}{2.116245in}}%
\pgfpathlineto{\pgfqpoint{4.408742in}{2.124904in}}%
\pgfpathlineto{\pgfqpoint{4.416504in}{2.133553in}}%
\pgfpathlineto{\pgfqpoint{4.424261in}{2.142189in}}%
\pgfpathclose%
\pgfusepath{fill}%
\end{pgfscope}%
\begin{pgfscope}%
\pgfpathrectangle{\pgfqpoint{1.150000in}{0.150000in}}{\pgfqpoint{5.700000in}{5.700000in}}%
\pgfusepath{clip}%
\pgfsetbuttcap%
\pgfsetroundjoin%
\definecolor{currentfill}{rgb}{0.277018,0.050344,0.375715}%
\pgfsetfillcolor{currentfill}%
\pgfsetfillopacity{0.700000}%
\pgfsetlinewidth{0.000000pt}%
\definecolor{currentstroke}{rgb}{0.000000,0.000000,0.000000}%
\pgfsetstrokecolor{currentstroke}%
\pgfsetdash{}{0pt}%
\pgfpathmoveto{\pgfqpoint{3.267493in}{2.164111in}}%
\pgfpathlineto{\pgfqpoint{3.280987in}{2.158169in}}%
\pgfpathlineto{\pgfqpoint{3.294486in}{2.152261in}}%
\pgfpathlineto{\pgfqpoint{3.307990in}{2.146387in}}%
\pgfpathlineto{\pgfqpoint{3.321498in}{2.140546in}}%
\pgfpathlineto{\pgfqpoint{3.313270in}{2.136927in}}%
\pgfpathlineto{\pgfqpoint{3.305032in}{2.133489in}}%
\pgfpathlineto{\pgfqpoint{3.296783in}{2.130237in}}%
\pgfpathlineto{\pgfqpoint{3.288524in}{2.127177in}}%
\pgfpathlineto{\pgfqpoint{3.274994in}{2.133247in}}%
\pgfpathlineto{\pgfqpoint{3.261469in}{2.139350in}}%
\pgfpathlineto{\pgfqpoint{3.247948in}{2.145486in}}%
\pgfpathlineto{\pgfqpoint{3.234432in}{2.151657in}}%
\pgfpathlineto{\pgfqpoint{3.242713in}{2.154483in}}%
\pgfpathlineto{\pgfqpoint{3.250984in}{2.157504in}}%
\pgfpathlineto{\pgfqpoint{3.259244in}{2.160715in}}%
\pgfpathlineto{\pgfqpoint{3.267493in}{2.164111in}}%
\pgfpathclose%
\pgfusepath{fill}%
\end{pgfscope}%
\begin{pgfscope}%
\pgfpathrectangle{\pgfqpoint{1.150000in}{0.150000in}}{\pgfqpoint{5.700000in}{5.700000in}}%
\pgfusepath{clip}%
\pgfsetbuttcap%
\pgfsetroundjoin%
\definecolor{currentfill}{rgb}{0.267004,0.004874,0.329415}%
\pgfsetfillcolor{currentfill}%
\pgfsetfillopacity{0.700000}%
\pgfsetlinewidth{0.000000pt}%
\definecolor{currentstroke}{rgb}{0.000000,0.000000,0.000000}%
\pgfsetstrokecolor{currentstroke}%
\pgfsetdash{}{0pt}%
\pgfpathmoveto{\pgfqpoint{3.970749in}{2.087124in}}%
\pgfpathlineto{\pgfqpoint{3.984373in}{2.083343in}}%
\pgfpathlineto{\pgfqpoint{3.998003in}{2.079590in}}%
\pgfpathlineto{\pgfqpoint{4.011639in}{2.075865in}}%
\pgfpathlineto{\pgfqpoint{4.025282in}{2.072167in}}%
\pgfpathlineto{\pgfqpoint{4.017375in}{2.064438in}}%
\pgfpathlineto{\pgfqpoint{4.009461in}{2.056760in}}%
\pgfpathlineto{\pgfqpoint{4.001541in}{2.049139in}}%
\pgfpathlineto{\pgfqpoint{3.993616in}{2.041577in}}%
\pgfpathlineto{\pgfqpoint{3.979959in}{2.045436in}}%
\pgfpathlineto{\pgfqpoint{3.966309in}{2.049323in}}%
\pgfpathlineto{\pgfqpoint{3.952665in}{2.053237in}}%
\pgfpathlineto{\pgfqpoint{3.939028in}{2.057180in}}%
\pgfpathlineto{\pgfqpoint{3.946967in}{2.064575in}}%
\pgfpathlineto{\pgfqpoint{3.954901in}{2.072033in}}%
\pgfpathlineto{\pgfqpoint{3.962828in}{2.079551in}}%
\pgfpathlineto{\pgfqpoint{3.970749in}{2.087124in}}%
\pgfpathclose%
\pgfusepath{fill}%
\end{pgfscope}%
\begin{pgfscope}%
\pgfpathrectangle{\pgfqpoint{1.150000in}{0.150000in}}{\pgfqpoint{5.700000in}{5.700000in}}%
\pgfusepath{clip}%
\pgfsetbuttcap%
\pgfsetroundjoin%
\definecolor{currentfill}{rgb}{0.278826,0.175490,0.483397}%
\pgfsetfillcolor{currentfill}%
\pgfsetfillopacity{0.700000}%
\pgfsetlinewidth{0.000000pt}%
\definecolor{currentstroke}{rgb}{0.000000,0.000000,0.000000}%
\pgfsetstrokecolor{currentstroke}%
\pgfsetdash{}{0pt}%
\pgfpathmoveto{\pgfqpoint{2.681868in}{2.401672in}}%
\pgfpathlineto{\pgfqpoint{2.695309in}{2.393600in}}%
\pgfpathlineto{\pgfqpoint{2.708753in}{2.385573in}}%
\pgfpathlineto{\pgfqpoint{2.722200in}{2.377591in}}%
\pgfpathlineto{\pgfqpoint{2.735649in}{2.369653in}}%
\pgfpathlineto{\pgfqpoint{2.727034in}{2.370999in}}%
\pgfpathlineto{\pgfqpoint{2.718401in}{2.372634in}}%
\pgfpathlineto{\pgfqpoint{2.709751in}{2.374565in}}%
\pgfpathlineto{\pgfqpoint{2.701084in}{2.376799in}}%
\pgfpathlineto{\pgfqpoint{2.687603in}{2.385010in}}%
\pgfpathlineto{\pgfqpoint{2.674125in}{2.393265in}}%
\pgfpathlineto{\pgfqpoint{2.660650in}{2.401565in}}%
\pgfpathlineto{\pgfqpoint{2.647176in}{2.409910in}}%
\pgfpathlineto{\pgfqpoint{2.655876in}{2.407397in}}%
\pgfpathlineto{\pgfqpoint{2.664557in}{2.405191in}}%
\pgfpathlineto{\pgfqpoint{2.673221in}{2.403285in}}%
\pgfpathlineto{\pgfqpoint{2.681868in}{2.401672in}}%
\pgfpathclose%
\pgfusepath{fill}%
\end{pgfscope}%
\begin{pgfscope}%
\pgfpathrectangle{\pgfqpoint{1.150000in}{0.150000in}}{\pgfqpoint{5.700000in}{5.700000in}}%
\pgfusepath{clip}%
\pgfsetbuttcap%
\pgfsetroundjoin%
\definecolor{currentfill}{rgb}{0.279574,0.170599,0.479997}%
\pgfsetfillcolor{currentfill}%
\pgfsetfillopacity{0.700000}%
\pgfsetlinewidth{0.000000pt}%
\definecolor{currentstroke}{rgb}{0.000000,0.000000,0.000000}%
\pgfsetstrokecolor{currentstroke}%
\pgfsetdash{}{0pt}%
\pgfpathmoveto{\pgfqpoint{5.589484in}{2.395844in}}%
\pgfpathlineto{\pgfqpoint{5.603561in}{2.395044in}}%
\pgfpathlineto{\pgfqpoint{5.617647in}{2.394267in}}%
\pgfpathlineto{\pgfqpoint{5.631741in}{2.393516in}}%
\pgfpathlineto{\pgfqpoint{5.645845in}{2.392789in}}%
\pgfpathlineto{\pgfqpoint{5.638579in}{2.386407in}}%
\pgfpathlineto{\pgfqpoint{5.631306in}{2.379944in}}%
\pgfpathlineto{\pgfqpoint{5.624024in}{2.373399in}}%
\pgfpathlineto{\pgfqpoint{5.616734in}{2.366771in}}%
\pgfpathlineto{\pgfqpoint{5.602615in}{2.367457in}}%
\pgfpathlineto{\pgfqpoint{5.588505in}{2.368168in}}%
\pgfpathlineto{\pgfqpoint{5.574404in}{2.368903in}}%
\pgfpathlineto{\pgfqpoint{5.560312in}{2.369664in}}%
\pgfpathlineto{\pgfqpoint{5.567616in}{2.376328in}}%
\pgfpathlineto{\pgfqpoint{5.574913in}{2.382912in}}%
\pgfpathlineto{\pgfqpoint{5.582202in}{2.389417in}}%
\pgfpathlineto{\pgfqpoint{5.589484in}{2.395844in}}%
\pgfpathclose%
\pgfusepath{fill}%
\end{pgfscope}%
\begin{pgfscope}%
\pgfpathrectangle{\pgfqpoint{1.150000in}{0.150000in}}{\pgfqpoint{5.700000in}{5.700000in}}%
\pgfusepath{clip}%
\pgfsetbuttcap%
\pgfsetroundjoin%
\definecolor{currentfill}{rgb}{0.282884,0.135920,0.453427}%
\pgfsetfillcolor{currentfill}%
\pgfsetfillopacity{0.700000}%
\pgfsetlinewidth{0.000000pt}%
\definecolor{currentstroke}{rgb}{0.000000,0.000000,0.000000}%
\pgfsetstrokecolor{currentstroke}%
\pgfsetdash{}{0pt}%
\pgfpathmoveto{\pgfqpoint{5.276829in}{2.328162in}}%
\pgfpathlineto{\pgfqpoint{5.290813in}{2.327047in}}%
\pgfpathlineto{\pgfqpoint{5.304805in}{2.325956in}}%
\pgfpathlineto{\pgfqpoint{5.318805in}{2.324891in}}%
\pgfpathlineto{\pgfqpoint{5.332814in}{2.323850in}}%
\pgfpathlineto{\pgfqpoint{5.325398in}{2.316390in}}%
\pgfpathlineto{\pgfqpoint{5.317975in}{2.308849in}}%
\pgfpathlineto{\pgfqpoint{5.310544in}{2.301226in}}%
\pgfpathlineto{\pgfqpoint{5.303106in}{2.293520in}}%
\pgfpathlineto{\pgfqpoint{5.289085in}{2.294562in}}%
\pgfpathlineto{\pgfqpoint{5.275071in}{2.295628in}}%
\pgfpathlineto{\pgfqpoint{5.261066in}{2.296719in}}%
\pgfpathlineto{\pgfqpoint{5.247070in}{2.297835in}}%
\pgfpathlineto{\pgfqpoint{5.254520in}{2.305535in}}%
\pgfpathlineto{\pgfqpoint{5.261964in}{2.313156in}}%
\pgfpathlineto{\pgfqpoint{5.269400in}{2.320698in}}%
\pgfpathlineto{\pgfqpoint{5.276829in}{2.328162in}}%
\pgfpathclose%
\pgfusepath{fill}%
\end{pgfscope}%
\begin{pgfscope}%
\pgfpathrectangle{\pgfqpoint{1.150000in}{0.150000in}}{\pgfqpoint{5.700000in}{5.700000in}}%
\pgfusepath{clip}%
\pgfsetbuttcap%
\pgfsetroundjoin%
\definecolor{currentfill}{rgb}{0.282656,0.100196,0.422160}%
\pgfsetfillcolor{currentfill}%
\pgfsetfillopacity{0.700000}%
\pgfsetlinewidth{0.000000pt}%
\definecolor{currentstroke}{rgb}{0.000000,0.000000,0.000000}%
\pgfsetstrokecolor{currentstroke}%
\pgfsetdash{}{0pt}%
\pgfpathmoveto{\pgfqpoint{4.963996in}{2.255299in}}%
\pgfpathlineto{\pgfqpoint{4.977886in}{2.253740in}}%
\pgfpathlineto{\pgfqpoint{4.991784in}{2.252207in}}%
\pgfpathlineto{\pgfqpoint{5.005690in}{2.250699in}}%
\pgfpathlineto{\pgfqpoint{5.019604in}{2.249217in}}%
\pgfpathlineto{\pgfqpoint{5.012055in}{2.240911in}}%
\pgfpathlineto{\pgfqpoint{5.004500in}{2.232536in}}%
\pgfpathlineto{\pgfqpoint{4.996938in}{2.224094in}}%
\pgfpathlineto{\pgfqpoint{4.989369in}{2.215585in}}%
\pgfpathlineto{\pgfqpoint{4.975444in}{2.217109in}}%
\pgfpathlineto{\pgfqpoint{4.961527in}{2.218658in}}%
\pgfpathlineto{\pgfqpoint{4.947618in}{2.220233in}}%
\pgfpathlineto{\pgfqpoint{4.933716in}{2.221833in}}%
\pgfpathlineto{\pgfqpoint{4.941296in}{2.230296in}}%
\pgfpathlineto{\pgfqpoint{4.948869in}{2.238695in}}%
\pgfpathlineto{\pgfqpoint{4.956436in}{2.247029in}}%
\pgfpathlineto{\pgfqpoint{4.963996in}{2.255299in}}%
\pgfpathclose%
\pgfusepath{fill}%
\end{pgfscope}%
\begin{pgfscope}%
\pgfpathrectangle{\pgfqpoint{1.150000in}{0.150000in}}{\pgfqpoint{5.700000in}{5.700000in}}%
\pgfusepath{clip}%
\pgfsetbuttcap%
\pgfsetroundjoin%
\definecolor{currentfill}{rgb}{0.260571,0.246922,0.522828}%
\pgfsetfillcolor{currentfill}%
\pgfsetfillopacity{0.700000}%
\pgfsetlinewidth{0.000000pt}%
\definecolor{currentstroke}{rgb}{0.000000,0.000000,0.000000}%
\pgfsetstrokecolor{currentstroke}%
\pgfsetdash{}{0pt}%
\pgfpathmoveto{\pgfqpoint{2.431911in}{2.549928in}}%
\pgfpathlineto{\pgfqpoint{2.445350in}{2.540799in}}%
\pgfpathlineto{\pgfqpoint{2.458792in}{2.531722in}}%
\pgfpathlineto{\pgfqpoint{2.472235in}{2.522698in}}%
\pgfpathlineto{\pgfqpoint{2.485679in}{2.513726in}}%
\pgfpathlineto{\pgfqpoint{2.476859in}{2.517403in}}%
\pgfpathlineto{\pgfqpoint{2.468019in}{2.521412in}}%
\pgfpathlineto{\pgfqpoint{2.459157in}{2.525763in}}%
\pgfpathlineto{\pgfqpoint{2.450275in}{2.530463in}}%
\pgfpathlineto{\pgfqpoint{2.436794in}{2.539725in}}%
\pgfpathlineto{\pgfqpoint{2.423315in}{2.549039in}}%
\pgfpathlineto{\pgfqpoint{2.409837in}{2.558405in}}%
\pgfpathlineto{\pgfqpoint{2.396361in}{2.567825in}}%
\pgfpathlineto{\pgfqpoint{2.405281in}{2.562830in}}%
\pgfpathlineto{\pgfqpoint{2.414179in}{2.558187in}}%
\pgfpathlineto{\pgfqpoint{2.423055in}{2.553889in}}%
\pgfpathlineto{\pgfqpoint{2.431911in}{2.549928in}}%
\pgfpathclose%
\pgfusepath{fill}%
\end{pgfscope}%
\begin{pgfscope}%
\pgfpathrectangle{\pgfqpoint{1.150000in}{0.150000in}}{\pgfqpoint{5.700000in}{5.700000in}}%
\pgfusepath{clip}%
\pgfsetbuttcap%
\pgfsetroundjoin%
\definecolor{currentfill}{rgb}{0.278791,0.062145,0.386592}%
\pgfsetfillcolor{currentfill}%
\pgfsetfillopacity{0.700000}%
\pgfsetlinewidth{0.000000pt}%
\definecolor{currentstroke}{rgb}{0.000000,0.000000,0.000000}%
\pgfsetstrokecolor{currentstroke}%
\pgfsetdash{}{0pt}%
\pgfpathmoveto{\pgfqpoint{4.651139in}{2.182960in}}%
\pgfpathlineto{\pgfqpoint{4.664939in}{2.180830in}}%
\pgfpathlineto{\pgfqpoint{4.678747in}{2.178726in}}%
\pgfpathlineto{\pgfqpoint{4.692562in}{2.176647in}}%
\pgfpathlineto{\pgfqpoint{4.706384in}{2.174594in}}%
\pgfpathlineto{\pgfqpoint{4.698717in}{2.165831in}}%
\pgfpathlineto{\pgfqpoint{4.691044in}{2.157026in}}%
\pgfpathlineto{\pgfqpoint{4.683364in}{2.148181in}}%
\pgfpathlineto{\pgfqpoint{4.675679in}{2.139296in}}%
\pgfpathlineto{\pgfqpoint{4.661846in}{2.141431in}}%
\pgfpathlineto{\pgfqpoint{4.648020in}{2.143592in}}%
\pgfpathlineto{\pgfqpoint{4.634202in}{2.145779in}}%
\pgfpathlineto{\pgfqpoint{4.620391in}{2.147991in}}%
\pgfpathlineto{\pgfqpoint{4.628087in}{2.156788in}}%
\pgfpathlineto{\pgfqpoint{4.635777in}{2.165550in}}%
\pgfpathlineto{\pgfqpoint{4.643461in}{2.174275in}}%
\pgfpathlineto{\pgfqpoint{4.651139in}{2.182960in}}%
\pgfpathclose%
\pgfusepath{fill}%
\end{pgfscope}%
\begin{pgfscope}%
\pgfpathrectangle{\pgfqpoint{1.150000in}{0.150000in}}{\pgfqpoint{5.700000in}{5.700000in}}%
\pgfusepath{clip}%
\pgfsetbuttcap%
\pgfsetroundjoin%
\definecolor{currentfill}{rgb}{0.275191,0.194905,0.496005}%
\pgfsetfillcolor{currentfill}%
\pgfsetfillopacity{0.700000}%
\pgfsetlinewidth{0.000000pt}%
\definecolor{currentstroke}{rgb}{0.000000,0.000000,0.000000}%
\pgfsetstrokecolor{currentstroke}%
\pgfsetdash{}{0pt}%
\pgfpathmoveto{\pgfqpoint{5.816564in}{2.435257in}}%
\pgfpathlineto{\pgfqpoint{5.830714in}{2.434627in}}%
\pgfpathlineto{\pgfqpoint{5.844872in}{2.434022in}}%
\pgfpathlineto{\pgfqpoint{5.859040in}{2.433441in}}%
\pgfpathlineto{\pgfqpoint{5.873217in}{2.432885in}}%
\pgfpathlineto{\pgfqpoint{5.866066in}{2.427296in}}%
\pgfpathlineto{\pgfqpoint{5.858908in}{2.421634in}}%
\pgfpathlineto{\pgfqpoint{5.851741in}{2.415895in}}%
\pgfpathlineto{\pgfqpoint{5.844566in}{2.410077in}}%
\pgfpathlineto{\pgfqpoint{5.830372in}{2.410565in}}%
\pgfpathlineto{\pgfqpoint{5.816187in}{2.411077in}}%
\pgfpathlineto{\pgfqpoint{5.802011in}{2.411614in}}%
\pgfpathlineto{\pgfqpoint{5.787843in}{2.412175in}}%
\pgfpathlineto{\pgfqpoint{5.795036in}{2.418056in}}%
\pgfpathlineto{\pgfqpoint{5.802220in}{2.423862in}}%
\pgfpathlineto{\pgfqpoint{5.809396in}{2.429595in}}%
\pgfpathlineto{\pgfqpoint{5.816564in}{2.435257in}}%
\pgfpathclose%
\pgfusepath{fill}%
\end{pgfscope}%
\begin{pgfscope}%
\pgfpathrectangle{\pgfqpoint{1.150000in}{0.150000in}}{\pgfqpoint{5.700000in}{5.700000in}}%
\pgfusepath{clip}%
\pgfsetbuttcap%
\pgfsetroundjoin%
\definecolor{currentfill}{rgb}{0.283197,0.115680,0.436115}%
\pgfsetfillcolor{currentfill}%
\pgfsetfillopacity{0.700000}%
\pgfsetlinewidth{0.000000pt}%
\definecolor{currentstroke}{rgb}{0.000000,0.000000,0.000000}%
\pgfsetstrokecolor{currentstroke}%
\pgfsetdash{}{0pt}%
\pgfpathmoveto{\pgfqpoint{2.931235in}{2.278159in}}%
\pgfpathlineto{\pgfqpoint{2.944697in}{2.271018in}}%
\pgfpathlineto{\pgfqpoint{2.958163in}{2.263916in}}%
\pgfpathlineto{\pgfqpoint{2.971632in}{2.256853in}}%
\pgfpathlineto{\pgfqpoint{2.985105in}{2.249828in}}%
\pgfpathlineto{\pgfqpoint{2.976664in}{2.249078in}}%
\pgfpathlineto{\pgfqpoint{2.968209in}{2.248575in}}%
\pgfpathlineto{\pgfqpoint{2.959740in}{2.248326in}}%
\pgfpathlineto{\pgfqpoint{2.951257in}{2.248338in}}%
\pgfpathlineto{\pgfqpoint{2.937757in}{2.255620in}}%
\pgfpathlineto{\pgfqpoint{2.924260in}{2.262941in}}%
\pgfpathlineto{\pgfqpoint{2.910767in}{2.270300in}}%
\pgfpathlineto{\pgfqpoint{2.897277in}{2.277699in}}%
\pgfpathlineto{\pgfqpoint{2.905789in}{2.277424in}}%
\pgfpathlineto{\pgfqpoint{2.914285in}{2.277414in}}%
\pgfpathlineto{\pgfqpoint{2.922767in}{2.277661in}}%
\pgfpathlineto{\pgfqpoint{2.931235in}{2.278159in}}%
\pgfpathclose%
\pgfusepath{fill}%
\end{pgfscope}%
\begin{pgfscope}%
\pgfpathrectangle{\pgfqpoint{1.150000in}{0.150000in}}{\pgfqpoint{5.700000in}{5.700000in}}%
\pgfusepath{clip}%
\pgfsetbuttcap%
\pgfsetroundjoin%
\definecolor{currentfill}{rgb}{0.272594,0.025563,0.353093}%
\pgfsetfillcolor{currentfill}%
\pgfsetfillopacity{0.700000}%
\pgfsetlinewidth{0.000000pt}%
\definecolor{currentstroke}{rgb}{0.000000,0.000000,0.000000}%
\pgfsetstrokecolor{currentstroke}%
\pgfsetdash{}{0pt}%
\pgfpathmoveto{\pgfqpoint{4.338286in}{2.118749in}}%
\pgfpathlineto{\pgfqpoint{4.352004in}{2.115916in}}%
\pgfpathlineto{\pgfqpoint{4.365729in}{2.113110in}}%
\pgfpathlineto{\pgfqpoint{4.379461in}{2.110330in}}%
\pgfpathlineto{\pgfqpoint{4.393200in}{2.107576in}}%
\pgfpathlineto{\pgfqpoint{4.385421in}{2.098902in}}%
\pgfpathlineto{\pgfqpoint{4.377637in}{2.090224in}}%
\pgfpathlineto{\pgfqpoint{4.369847in}{2.081546in}}%
\pgfpathlineto{\pgfqpoint{4.362051in}{2.072870in}}%
\pgfpathlineto{\pgfqpoint{4.348301in}{2.075746in}}%
\pgfpathlineto{\pgfqpoint{4.334558in}{2.078648in}}%
\pgfpathlineto{\pgfqpoint{4.320821in}{2.081576in}}%
\pgfpathlineto{\pgfqpoint{4.307092in}{2.084531in}}%
\pgfpathlineto{\pgfqpoint{4.314899in}{2.093080in}}%
\pgfpathlineto{\pgfqpoint{4.322700in}{2.101635in}}%
\pgfpathlineto{\pgfqpoint{4.330496in}{2.110192in}}%
\pgfpathlineto{\pgfqpoint{4.338286in}{2.118749in}}%
\pgfpathclose%
\pgfusepath{fill}%
\end{pgfscope}%
\begin{pgfscope}%
\pgfpathrectangle{\pgfqpoint{1.150000in}{0.150000in}}{\pgfqpoint{5.700000in}{5.700000in}}%
\pgfusepath{clip}%
\pgfsetbuttcap%
\pgfsetroundjoin%
\definecolor{currentfill}{rgb}{0.268510,0.009605,0.335427}%
\pgfsetfillcolor{currentfill}%
\pgfsetfillopacity{0.700000}%
\pgfsetlinewidth{0.000000pt}%
\definecolor{currentstroke}{rgb}{0.000000,0.000000,0.000000}%
\pgfsetstrokecolor{currentstroke}%
\pgfsetdash{}{0pt}%
\pgfpathmoveto{\pgfqpoint{4.111436in}{2.089613in}}%
\pgfpathlineto{\pgfqpoint{4.125097in}{2.086201in}}%
\pgfpathlineto{\pgfqpoint{4.138765in}{2.082816in}}%
\pgfpathlineto{\pgfqpoint{4.152440in}{2.079459in}}%
\pgfpathlineto{\pgfqpoint{4.166121in}{2.076129in}}%
\pgfpathlineto{\pgfqpoint{4.158262in}{2.067940in}}%
\pgfpathlineto{\pgfqpoint{4.150398in}{2.059783in}}%
\pgfpathlineto{\pgfqpoint{4.142528in}{2.051660in}}%
\pgfpathlineto{\pgfqpoint{4.134652in}{2.043575in}}%
\pgfpathlineto{\pgfqpoint{4.120958in}{2.047054in}}%
\pgfpathlineto{\pgfqpoint{4.107271in}{2.050560in}}%
\pgfpathlineto{\pgfqpoint{4.093590in}{2.054093in}}%
\pgfpathlineto{\pgfqpoint{4.079916in}{2.057653in}}%
\pgfpathlineto{\pgfqpoint{4.087805in}{2.065584in}}%
\pgfpathlineto{\pgfqpoint{4.095687in}{2.073557in}}%
\pgfpathlineto{\pgfqpoint{4.103564in}{2.081568in}}%
\pgfpathlineto{\pgfqpoint{4.111436in}{2.089613in}}%
\pgfpathclose%
\pgfusepath{fill}%
\end{pgfscope}%
\begin{pgfscope}%
\pgfpathrectangle{\pgfqpoint{1.150000in}{0.150000in}}{\pgfqpoint{5.700000in}{5.700000in}}%
\pgfusepath{clip}%
\pgfsetbuttcap%
\pgfsetroundjoin%
\definecolor{currentfill}{rgb}{0.280267,0.073417,0.397163}%
\pgfsetfillcolor{currentfill}%
\pgfsetfillopacity{0.700000}%
\pgfsetlinewidth{0.000000pt}%
\definecolor{currentstroke}{rgb}{0.000000,0.000000,0.000000}%
\pgfsetstrokecolor{currentstroke}%
\pgfsetdash{}{0pt}%
\pgfpathmoveto{\pgfqpoint{3.126454in}{2.202263in}}%
\pgfpathlineto{\pgfqpoint{3.139936in}{2.195814in}}%
\pgfpathlineto{\pgfqpoint{3.153423in}{2.189401in}}%
\pgfpathlineto{\pgfqpoint{3.166914in}{2.183023in}}%
\pgfpathlineto{\pgfqpoint{3.180409in}{2.176681in}}%
\pgfpathlineto{\pgfqpoint{3.172093in}{2.174294in}}%
\pgfpathlineto{\pgfqpoint{3.163766in}{2.172118in}}%
\pgfpathlineto{\pgfqpoint{3.155426in}{2.170159in}}%
\pgfpathlineto{\pgfqpoint{3.147075in}{2.168425in}}%
\pgfpathlineto{\pgfqpoint{3.133556in}{2.175010in}}%
\pgfpathlineto{\pgfqpoint{3.120041in}{2.181631in}}%
\pgfpathlineto{\pgfqpoint{3.106530in}{2.188287in}}%
\pgfpathlineto{\pgfqpoint{3.093023in}{2.194978in}}%
\pgfpathlineto{\pgfqpoint{3.101399in}{2.196465in}}%
\pgfpathlineto{\pgfqpoint{3.109763in}{2.198179in}}%
\pgfpathlineto{\pgfqpoint{3.118114in}{2.200114in}}%
\pgfpathlineto{\pgfqpoint{3.126454in}{2.202263in}}%
\pgfpathclose%
\pgfusepath{fill}%
\end{pgfscope}%
\begin{pgfscope}%
\pgfpathrectangle{\pgfqpoint{1.150000in}{0.150000in}}{\pgfqpoint{5.700000in}{5.700000in}}%
\pgfusepath{clip}%
\pgfsetbuttcap%
\pgfsetroundjoin%
\definecolor{currentfill}{rgb}{0.268510,0.009605,0.335427}%
\pgfsetfillcolor{currentfill}%
\pgfsetfillopacity{0.700000}%
\pgfsetlinewidth{0.000000pt}%
\definecolor{currentstroke}{rgb}{0.000000,0.000000,0.000000}%
\pgfsetstrokecolor{currentstroke}%
\pgfsetdash{}{0pt}%
\pgfpathmoveto{\pgfqpoint{3.603143in}{2.092920in}}%
\pgfpathlineto{\pgfqpoint{3.616697in}{2.088034in}}%
\pgfpathlineto{\pgfqpoint{3.630257in}{2.083179in}}%
\pgfpathlineto{\pgfqpoint{3.643822in}{2.078353in}}%
\pgfpathlineto{\pgfqpoint{3.657393in}{2.073558in}}%
\pgfpathlineto{\pgfqpoint{3.649330in}{2.067709in}}%
\pgfpathlineto{\pgfqpoint{3.641259in}{2.061982in}}%
\pgfpathlineto{\pgfqpoint{3.633181in}{2.056381in}}%
\pgfpathlineto{\pgfqpoint{3.625095in}{2.050911in}}%
\pgfpathlineto{\pgfqpoint{3.611507in}{2.055907in}}%
\pgfpathlineto{\pgfqpoint{3.597925in}{2.060934in}}%
\pgfpathlineto{\pgfqpoint{3.584347in}{2.065991in}}%
\pgfpathlineto{\pgfqpoint{3.570775in}{2.071079in}}%
\pgfpathlineto{\pgfqpoint{3.578879in}{2.076342in}}%
\pgfpathlineto{\pgfqpoint{3.586975in}{2.081741in}}%
\pgfpathlineto{\pgfqpoint{3.595063in}{2.087268in}}%
\pgfpathlineto{\pgfqpoint{3.603143in}{2.092920in}}%
\pgfpathclose%
\pgfusepath{fill}%
\end{pgfscope}%
\begin{pgfscope}%
\pgfpathrectangle{\pgfqpoint{1.150000in}{0.150000in}}{\pgfqpoint{5.700000in}{5.700000in}}%
\pgfusepath{clip}%
\pgfsetbuttcap%
\pgfsetroundjoin%
\definecolor{currentfill}{rgb}{0.267004,0.004874,0.329415}%
\pgfsetfillcolor{currentfill}%
\pgfsetfillopacity{0.700000}%
\pgfsetlinewidth{0.000000pt}%
\definecolor{currentstroke}{rgb}{0.000000,0.000000,0.000000}%
\pgfsetstrokecolor{currentstroke}%
\pgfsetdash{}{0pt}%
\pgfpathmoveto{\pgfqpoint{3.743841in}{2.079932in}}%
\pgfpathlineto{\pgfqpoint{3.757423in}{2.075472in}}%
\pgfpathlineto{\pgfqpoint{3.771011in}{2.071041in}}%
\pgfpathlineto{\pgfqpoint{3.784604in}{2.066639in}}%
\pgfpathlineto{\pgfqpoint{3.798203in}{2.062267in}}%
\pgfpathlineto{\pgfqpoint{3.790202in}{2.055616in}}%
\pgfpathlineto{\pgfqpoint{3.782193in}{2.049061in}}%
\pgfpathlineto{\pgfqpoint{3.774178in}{2.042606in}}%
\pgfpathlineto{\pgfqpoint{3.766155in}{2.036255in}}%
\pgfpathlineto{\pgfqpoint{3.752541in}{2.040816in}}%
\pgfpathlineto{\pgfqpoint{3.738931in}{2.045406in}}%
\pgfpathlineto{\pgfqpoint{3.725328in}{2.050024in}}%
\pgfpathlineto{\pgfqpoint{3.711730in}{2.054672in}}%
\pgfpathlineto{\pgfqpoint{3.719768in}{2.060830in}}%
\pgfpathlineto{\pgfqpoint{3.727800in}{2.067095in}}%
\pgfpathlineto{\pgfqpoint{3.735824in}{2.073464in}}%
\pgfpathlineto{\pgfqpoint{3.743841in}{2.079932in}}%
\pgfpathclose%
\pgfusepath{fill}%
\end{pgfscope}%
\begin{pgfscope}%
\pgfpathrectangle{\pgfqpoint{1.150000in}{0.150000in}}{\pgfqpoint{5.700000in}{5.700000in}}%
\pgfusepath{clip}%
\pgfsetbuttcap%
\pgfsetroundjoin%
\definecolor{currentfill}{rgb}{0.271305,0.019942,0.347269}%
\pgfsetfillcolor{currentfill}%
\pgfsetfillopacity{0.700000}%
\pgfsetlinewidth{0.000000pt}%
\definecolor{currentstroke}{rgb}{0.000000,0.000000,0.000000}%
\pgfsetstrokecolor{currentstroke}%
\pgfsetdash{}{0pt}%
\pgfpathmoveto{\pgfqpoint{3.462384in}{2.112883in}}%
\pgfpathlineto{\pgfqpoint{3.475915in}{2.107548in}}%
\pgfpathlineto{\pgfqpoint{3.489451in}{2.102245in}}%
\pgfpathlineto{\pgfqpoint{3.502993in}{2.096973in}}%
\pgfpathlineto{\pgfqpoint{3.516539in}{2.091733in}}%
\pgfpathlineto{\pgfqpoint{3.508408in}{2.086818in}}%
\pgfpathlineto{\pgfqpoint{3.500268in}{2.082052in}}%
\pgfpathlineto{\pgfqpoint{3.492120in}{2.077440in}}%
\pgfpathlineto{\pgfqpoint{3.483963in}{2.072986in}}%
\pgfpathlineto{\pgfqpoint{3.470397in}{2.078441in}}%
\pgfpathlineto{\pgfqpoint{3.456837in}{2.083928in}}%
\pgfpathlineto{\pgfqpoint{3.443281in}{2.089446in}}%
\pgfpathlineto{\pgfqpoint{3.429731in}{2.094996in}}%
\pgfpathlineto{\pgfqpoint{3.437908in}{2.099229in}}%
\pgfpathlineto{\pgfqpoint{3.446075in}{2.103625in}}%
\pgfpathlineto{\pgfqpoint{3.454234in}{2.108178in}}%
\pgfpathlineto{\pgfqpoint{3.462384in}{2.112883in}}%
\pgfpathclose%
\pgfusepath{fill}%
\end{pgfscope}%
\begin{pgfscope}%
\pgfpathrectangle{\pgfqpoint{1.150000in}{0.150000in}}{\pgfqpoint{5.700000in}{5.700000in}}%
\pgfusepath{clip}%
\pgfsetbuttcap%
\pgfsetroundjoin%
\definecolor{currentfill}{rgb}{0.280255,0.165693,0.476498}%
\pgfsetfillcolor{currentfill}%
\pgfsetfillopacity{0.700000}%
\pgfsetlinewidth{0.000000pt}%
\definecolor{currentstroke}{rgb}{0.000000,0.000000,0.000000}%
\pgfsetstrokecolor{currentstroke}%
\pgfsetdash{}{0pt}%
\pgfpathmoveto{\pgfqpoint{5.504029in}{2.372950in}}%
\pgfpathlineto{\pgfqpoint{5.518086in}{2.372092in}}%
\pgfpathlineto{\pgfqpoint{5.532153in}{2.371258in}}%
\pgfpathlineto{\pgfqpoint{5.546228in}{2.370448in}}%
\pgfpathlineto{\pgfqpoint{5.560312in}{2.369664in}}%
\pgfpathlineto{\pgfqpoint{5.552999in}{2.362917in}}%
\pgfpathlineto{\pgfqpoint{5.545678in}{2.356087in}}%
\pgfpathlineto{\pgfqpoint{5.538350in}{2.349172in}}%
\pgfpathlineto{\pgfqpoint{5.531013in}{2.342172in}}%
\pgfpathlineto{\pgfqpoint{5.516915in}{2.342929in}}%
\pgfpathlineto{\pgfqpoint{5.502825in}{2.343712in}}%
\pgfpathlineto{\pgfqpoint{5.488744in}{2.344519in}}%
\pgfpathlineto{\pgfqpoint{5.474672in}{2.345351in}}%
\pgfpathlineto{\pgfqpoint{5.482023in}{2.352373in}}%
\pgfpathlineto{\pgfqpoint{5.489366in}{2.359313in}}%
\pgfpathlineto{\pgfqpoint{5.496701in}{2.366172in}}%
\pgfpathlineto{\pgfqpoint{5.504029in}{2.372950in}}%
\pgfpathclose%
\pgfusepath{fill}%
\end{pgfscope}%
\begin{pgfscope}%
\pgfpathrectangle{\pgfqpoint{1.150000in}{0.150000in}}{\pgfqpoint{5.700000in}{5.700000in}}%
\pgfusepath{clip}%
\pgfsetbuttcap%
\pgfsetroundjoin%
\definecolor{currentfill}{rgb}{0.281924,0.089666,0.412415}%
\pgfsetfillcolor{currentfill}%
\pgfsetfillopacity{0.700000}%
\pgfsetlinewidth{0.000000pt}%
\definecolor{currentstroke}{rgb}{0.000000,0.000000,0.000000}%
\pgfsetstrokecolor{currentstroke}%
\pgfsetdash{}{0pt}%
\pgfpathmoveto{\pgfqpoint{4.878190in}{2.228484in}}%
\pgfpathlineto{\pgfqpoint{4.892060in}{2.226783in}}%
\pgfpathlineto{\pgfqpoint{4.905937in}{2.225108in}}%
\pgfpathlineto{\pgfqpoint{4.919823in}{2.223458in}}%
\pgfpathlineto{\pgfqpoint{4.933716in}{2.221833in}}%
\pgfpathlineto{\pgfqpoint{4.926130in}{2.213307in}}%
\pgfpathlineto{\pgfqpoint{4.918538in}{2.204719in}}%
\pgfpathlineto{\pgfqpoint{4.910939in}{2.196069in}}%
\pgfpathlineto{\pgfqpoint{4.903334in}{2.187359in}}%
\pgfpathlineto{\pgfqpoint{4.889430in}{2.189039in}}%
\pgfpathlineto{\pgfqpoint{4.875533in}{2.190745in}}%
\pgfpathlineto{\pgfqpoint{4.861645in}{2.192475in}}%
\pgfpathlineto{\pgfqpoint{4.847764in}{2.194231in}}%
\pgfpathlineto{\pgfqpoint{4.855380in}{2.202881in}}%
\pgfpathlineto{\pgfqpoint{4.862989in}{2.211473in}}%
\pgfpathlineto{\pgfqpoint{4.870593in}{2.220008in}}%
\pgfpathlineto{\pgfqpoint{4.878190in}{2.228484in}}%
\pgfpathclose%
\pgfusepath{fill}%
\end{pgfscope}%
\begin{pgfscope}%
\pgfpathrectangle{\pgfqpoint{1.150000in}{0.150000in}}{\pgfqpoint{5.700000in}{5.700000in}}%
\pgfusepath{clip}%
\pgfsetbuttcap%
\pgfsetroundjoin%
\definecolor{currentfill}{rgb}{0.283187,0.125848,0.444960}%
\pgfsetfillcolor{currentfill}%
\pgfsetfillopacity{0.700000}%
\pgfsetlinewidth{0.000000pt}%
\definecolor{currentstroke}{rgb}{0.000000,0.000000,0.000000}%
\pgfsetstrokecolor{currentstroke}%
\pgfsetdash{}{0pt}%
\pgfpathmoveto{\pgfqpoint{5.191167in}{2.302548in}}%
\pgfpathlineto{\pgfqpoint{5.205130in}{2.301333in}}%
\pgfpathlineto{\pgfqpoint{5.219102in}{2.300142in}}%
\pgfpathlineto{\pgfqpoint{5.233082in}{2.298976in}}%
\pgfpathlineto{\pgfqpoint{5.247070in}{2.297835in}}%
\pgfpathlineto{\pgfqpoint{5.239612in}{2.290057in}}%
\pgfpathlineto{\pgfqpoint{5.232147in}{2.282199in}}%
\pgfpathlineto{\pgfqpoint{5.224674in}{2.274261in}}%
\pgfpathlineto{\pgfqpoint{5.217195in}{2.266243in}}%
\pgfpathlineto{\pgfqpoint{5.203195in}{2.267398in}}%
\pgfpathlineto{\pgfqpoint{5.189202in}{2.268578in}}%
\pgfpathlineto{\pgfqpoint{5.175219in}{2.269783in}}%
\pgfpathlineto{\pgfqpoint{5.161243in}{2.271013in}}%
\pgfpathlineto{\pgfqpoint{5.168734in}{2.279011in}}%
\pgfpathlineto{\pgfqpoint{5.176219in}{2.286933in}}%
\pgfpathlineto{\pgfqpoint{5.183696in}{2.294779in}}%
\pgfpathlineto{\pgfqpoint{5.191167in}{2.302548in}}%
\pgfpathclose%
\pgfusepath{fill}%
\end{pgfscope}%
\begin{pgfscope}%
\pgfpathrectangle{\pgfqpoint{1.150000in}{0.150000in}}{\pgfqpoint{5.700000in}{5.700000in}}%
\pgfusepath{clip}%
\pgfsetbuttcap%
\pgfsetroundjoin%
\definecolor{currentfill}{rgb}{0.277018,0.050344,0.375715}%
\pgfsetfillcolor{currentfill}%
\pgfsetfillopacity{0.700000}%
\pgfsetlinewidth{0.000000pt}%
\definecolor{currentstroke}{rgb}{0.000000,0.000000,0.000000}%
\pgfsetstrokecolor{currentstroke}%
\pgfsetdash{}{0pt}%
\pgfpathmoveto{\pgfqpoint{4.565222in}{2.157097in}}%
\pgfpathlineto{\pgfqpoint{4.579003in}{2.154781in}}%
\pgfpathlineto{\pgfqpoint{4.592792in}{2.152492in}}%
\pgfpathlineto{\pgfqpoint{4.606588in}{2.150229in}}%
\pgfpathlineto{\pgfqpoint{4.620391in}{2.147991in}}%
\pgfpathlineto{\pgfqpoint{4.612690in}{2.139159in}}%
\pgfpathlineto{\pgfqpoint{4.604983in}{2.130296in}}%
\pgfpathlineto{\pgfqpoint{4.597270in}{2.121403in}}%
\pgfpathlineto{\pgfqpoint{4.589551in}{2.112482in}}%
\pgfpathlineto{\pgfqpoint{4.575737in}{2.114815in}}%
\pgfpathlineto{\pgfqpoint{4.561930in}{2.117174in}}%
\pgfpathlineto{\pgfqpoint{4.548131in}{2.119559in}}%
\pgfpathlineto{\pgfqpoint{4.534339in}{2.121969in}}%
\pgfpathlineto{\pgfqpoint{4.542068in}{2.130790in}}%
\pgfpathlineto{\pgfqpoint{4.549791in}{2.139586in}}%
\pgfpathlineto{\pgfqpoint{4.557509in}{2.148356in}}%
\pgfpathlineto{\pgfqpoint{4.565222in}{2.157097in}}%
\pgfpathclose%
\pgfusepath{fill}%
\end{pgfscope}%
\begin{pgfscope}%
\pgfpathrectangle{\pgfqpoint{1.150000in}{0.150000in}}{\pgfqpoint{5.700000in}{5.700000in}}%
\pgfusepath{clip}%
\pgfsetbuttcap%
\pgfsetroundjoin%
\definecolor{currentfill}{rgb}{0.280255,0.165693,0.476498}%
\pgfsetfillcolor{currentfill}%
\pgfsetfillopacity{0.700000}%
\pgfsetlinewidth{0.000000pt}%
\definecolor{currentstroke}{rgb}{0.000000,0.000000,0.000000}%
\pgfsetstrokecolor{currentstroke}%
\pgfsetdash{}{0pt}%
\pgfpathmoveto{\pgfqpoint{2.735649in}{2.369653in}}%
\pgfpathlineto{\pgfqpoint{2.749102in}{2.361759in}}%
\pgfpathlineto{\pgfqpoint{2.762557in}{2.353908in}}%
\pgfpathlineto{\pgfqpoint{2.776015in}{2.346100in}}%
\pgfpathlineto{\pgfqpoint{2.789476in}{2.338334in}}%
\pgfpathlineto{\pgfqpoint{2.780891in}{2.339413in}}%
\pgfpathlineto{\pgfqpoint{2.772289in}{2.340777in}}%
\pgfpathlineto{\pgfqpoint{2.763671in}{2.342433in}}%
\pgfpathlineto{\pgfqpoint{2.755035in}{2.344390in}}%
\pgfpathlineto{\pgfqpoint{2.741543in}{2.352428in}}%
\pgfpathlineto{\pgfqpoint{2.728054in}{2.360508in}}%
\pgfpathlineto{\pgfqpoint{2.714568in}{2.368632in}}%
\pgfpathlineto{\pgfqpoint{2.701084in}{2.376799in}}%
\pgfpathlineto{\pgfqpoint{2.709751in}{2.374565in}}%
\pgfpathlineto{\pgfqpoint{2.718401in}{2.372634in}}%
\pgfpathlineto{\pgfqpoint{2.727034in}{2.370999in}}%
\pgfpathlineto{\pgfqpoint{2.735649in}{2.369653in}}%
\pgfpathclose%
\pgfusepath{fill}%
\end{pgfscope}%
\begin{pgfscope}%
\pgfpathrectangle{\pgfqpoint{1.150000in}{0.150000in}}{\pgfqpoint{5.700000in}{5.700000in}}%
\pgfusepath{clip}%
\pgfsetbuttcap%
\pgfsetroundjoin%
\definecolor{currentfill}{rgb}{0.267004,0.004874,0.329415}%
\pgfsetfillcolor{currentfill}%
\pgfsetfillopacity{0.700000}%
\pgfsetlinewidth{0.000000pt}%
\definecolor{currentstroke}{rgb}{0.000000,0.000000,0.000000}%
\pgfsetstrokecolor{currentstroke}%
\pgfsetdash{}{0pt}%
\pgfpathmoveto{\pgfqpoint{3.884537in}{2.073230in}}%
\pgfpathlineto{\pgfqpoint{3.898151in}{2.069175in}}%
\pgfpathlineto{\pgfqpoint{3.911770in}{2.065149in}}%
\pgfpathlineto{\pgfqpoint{3.925396in}{2.061150in}}%
\pgfpathlineto{\pgfqpoint{3.939028in}{2.057180in}}%
\pgfpathlineto{\pgfqpoint{3.931082in}{2.049852in}}%
\pgfpathlineto{\pgfqpoint{3.923130in}{2.042595in}}%
\pgfpathlineto{\pgfqpoint{3.915171in}{2.035414in}}%
\pgfpathlineto{\pgfqpoint{3.907206in}{2.028313in}}%
\pgfpathlineto{\pgfqpoint{3.893560in}{2.032458in}}%
\pgfpathlineto{\pgfqpoint{3.879920in}{2.036632in}}%
\pgfpathlineto{\pgfqpoint{3.866286in}{2.040833in}}%
\pgfpathlineto{\pgfqpoint{3.852658in}{2.045063in}}%
\pgfpathlineto{\pgfqpoint{3.860637in}{2.051984in}}%
\pgfpathlineto{\pgfqpoint{3.868610in}{2.058989in}}%
\pgfpathlineto{\pgfqpoint{3.876577in}{2.066072in}}%
\pgfpathlineto{\pgfqpoint{3.884537in}{2.073230in}}%
\pgfpathclose%
\pgfusepath{fill}%
\end{pgfscope}%
\begin{pgfscope}%
\pgfpathrectangle{\pgfqpoint{1.150000in}{0.150000in}}{\pgfqpoint{5.700000in}{5.700000in}}%
\pgfusepath{clip}%
\pgfsetbuttcap%
\pgfsetroundjoin%
\definecolor{currentfill}{rgb}{0.265145,0.232956,0.516599}%
\pgfsetfillcolor{currentfill}%
\pgfsetfillopacity{0.700000}%
\pgfsetlinewidth{0.000000pt}%
\definecolor{currentstroke}{rgb}{0.000000,0.000000,0.000000}%
\pgfsetstrokecolor{currentstroke}%
\pgfsetdash{}{0pt}%
\pgfpathmoveto{\pgfqpoint{2.485679in}{2.513726in}}%
\pgfpathlineto{\pgfqpoint{2.499126in}{2.504805in}}%
\pgfpathlineto{\pgfqpoint{2.512574in}{2.495934in}}%
\pgfpathlineto{\pgfqpoint{2.526025in}{2.487114in}}%
\pgfpathlineto{\pgfqpoint{2.539477in}{2.478344in}}%
\pgfpathlineto{\pgfqpoint{2.530692in}{2.481737in}}%
\pgfpathlineto{\pgfqpoint{2.521886in}{2.485459in}}%
\pgfpathlineto{\pgfqpoint{2.513061in}{2.489519in}}%
\pgfpathlineto{\pgfqpoint{2.504215in}{2.493923in}}%
\pgfpathlineto{\pgfqpoint{2.490727in}{2.502983in}}%
\pgfpathlineto{\pgfqpoint{2.477241in}{2.512092in}}%
\pgfpathlineto{\pgfqpoint{2.463757in}{2.521252in}}%
\pgfpathlineto{\pgfqpoint{2.450275in}{2.530463in}}%
\pgfpathlineto{\pgfqpoint{2.459157in}{2.525763in}}%
\pgfpathlineto{\pgfqpoint{2.468019in}{2.521412in}}%
\pgfpathlineto{\pgfqpoint{2.476859in}{2.517403in}}%
\pgfpathlineto{\pgfqpoint{2.485679in}{2.513726in}}%
\pgfpathclose%
\pgfusepath{fill}%
\end{pgfscope}%
\begin{pgfscope}%
\pgfpathrectangle{\pgfqpoint{1.150000in}{0.150000in}}{\pgfqpoint{5.700000in}{5.700000in}}%
\pgfusepath{clip}%
\pgfsetbuttcap%
\pgfsetroundjoin%
\definecolor{currentfill}{rgb}{0.276022,0.044167,0.370164}%
\pgfsetfillcolor{currentfill}%
\pgfsetfillopacity{0.700000}%
\pgfsetlinewidth{0.000000pt}%
\definecolor{currentstroke}{rgb}{0.000000,0.000000,0.000000}%
\pgfsetstrokecolor{currentstroke}%
\pgfsetdash{}{0pt}%
\pgfpathmoveto{\pgfqpoint{3.321498in}{2.140546in}}%
\pgfpathlineto{\pgfqpoint{3.335010in}{2.134738in}}%
\pgfpathlineto{\pgfqpoint{3.348528in}{2.128963in}}%
\pgfpathlineto{\pgfqpoint{3.362050in}{2.123221in}}%
\pgfpathlineto{\pgfqpoint{3.375576in}{2.117512in}}%
\pgfpathlineto{\pgfqpoint{3.367369in}{2.113669in}}%
\pgfpathlineto{\pgfqpoint{3.359152in}{2.110004in}}%
\pgfpathlineto{\pgfqpoint{3.350926in}{2.106522in}}%
\pgfpathlineto{\pgfqpoint{3.342689in}{2.103229in}}%
\pgfpathlineto{\pgfqpoint{3.329141in}{2.109167in}}%
\pgfpathlineto{\pgfqpoint{3.315597in}{2.115138in}}%
\pgfpathlineto{\pgfqpoint{3.302058in}{2.121141in}}%
\pgfpathlineto{\pgfqpoint{3.288524in}{2.127177in}}%
\pgfpathlineto{\pgfqpoint{3.296783in}{2.130237in}}%
\pgfpathlineto{\pgfqpoint{3.305032in}{2.133489in}}%
\pgfpathlineto{\pgfqpoint{3.313270in}{2.136927in}}%
\pgfpathlineto{\pgfqpoint{3.321498in}{2.140546in}}%
\pgfpathclose%
\pgfusepath{fill}%
\end{pgfscope}%
\begin{pgfscope}%
\pgfpathrectangle{\pgfqpoint{1.150000in}{0.150000in}}{\pgfqpoint{5.700000in}{5.700000in}}%
\pgfusepath{clip}%
\pgfsetbuttcap%
\pgfsetroundjoin%
\definecolor{currentfill}{rgb}{0.277134,0.185228,0.489898}%
\pgfsetfillcolor{currentfill}%
\pgfsetfillopacity{0.700000}%
\pgfsetlinewidth{0.000000pt}%
\definecolor{currentstroke}{rgb}{0.000000,0.000000,0.000000}%
\pgfsetstrokecolor{currentstroke}%
\pgfsetdash{}{0pt}%
\pgfpathmoveto{\pgfqpoint{5.731265in}{2.414665in}}%
\pgfpathlineto{\pgfqpoint{5.745396in}{2.414006in}}%
\pgfpathlineto{\pgfqpoint{5.759536in}{2.413371in}}%
\pgfpathlineto{\pgfqpoint{5.773685in}{2.412761in}}%
\pgfpathlineto{\pgfqpoint{5.787843in}{2.412175in}}%
\pgfpathlineto{\pgfqpoint{5.780643in}{2.406217in}}%
\pgfpathlineto{\pgfqpoint{5.773434in}{2.400178in}}%
\pgfpathlineto{\pgfqpoint{5.766217in}{2.394058in}}%
\pgfpathlineto{\pgfqpoint{5.758992in}{2.387854in}}%
\pgfpathlineto{\pgfqpoint{5.744817in}{2.388385in}}%
\pgfpathlineto{\pgfqpoint{5.730651in}{2.388941in}}%
\pgfpathlineto{\pgfqpoint{5.716495in}{2.389521in}}%
\pgfpathlineto{\pgfqpoint{5.702347in}{2.390125in}}%
\pgfpathlineto{\pgfqpoint{5.709588in}{2.396379in}}%
\pgfpathlineto{\pgfqpoint{5.716822in}{2.402552in}}%
\pgfpathlineto{\pgfqpoint{5.724047in}{2.408647in}}%
\pgfpathlineto{\pgfqpoint{5.731265in}{2.414665in}}%
\pgfpathclose%
\pgfusepath{fill}%
\end{pgfscope}%
\begin{pgfscope}%
\pgfpathrectangle{\pgfqpoint{1.150000in}{0.150000in}}{\pgfqpoint{5.700000in}{5.700000in}}%
\pgfusepath{clip}%
\pgfsetbuttcap%
\pgfsetroundjoin%
\definecolor{currentfill}{rgb}{0.273006,0.204520,0.501721}%
\pgfsetfillcolor{currentfill}%
\pgfsetfillopacity{0.700000}%
\pgfsetlinewidth{0.000000pt}%
\definecolor{currentstroke}{rgb}{0.000000,0.000000,0.000000}%
\pgfsetstrokecolor{currentstroke}%
\pgfsetdash{}{0pt}%
\pgfpathmoveto{\pgfqpoint{5.958460in}{2.452241in}}%
\pgfpathlineto{\pgfqpoint{5.972663in}{2.451724in}}%
\pgfpathlineto{\pgfqpoint{5.986876in}{2.451232in}}%
\pgfpathlineto{\pgfqpoint{6.001097in}{2.450764in}}%
\pgfpathlineto{\pgfqpoint{5.994013in}{2.445598in}}%
\pgfpathlineto{\pgfqpoint{5.986919in}{2.440362in}}%
\pgfpathlineto{\pgfqpoint{5.979818in}{2.435055in}}%
\pgfpathlineto{\pgfqpoint{5.972708in}{2.429673in}}%
\pgfpathlineto{\pgfqpoint{5.958467in}{2.430059in}}%
\pgfpathlineto{\pgfqpoint{5.944236in}{2.430469in}}%
\pgfpathlineto{\pgfqpoint{5.930014in}{2.430903in}}%
\pgfpathlineto{\pgfqpoint{5.937138in}{2.436343in}}%
\pgfpathlineto{\pgfqpoint{5.944253in}{2.441711in}}%
\pgfpathlineto{\pgfqpoint{5.951361in}{2.447009in}}%
\pgfpathlineto{\pgfqpoint{5.958460in}{2.452241in}}%
\pgfpathclose%
\pgfusepath{fill}%
\end{pgfscope}%
\begin{pgfscope}%
\pgfpathrectangle{\pgfqpoint{1.150000in}{0.150000in}}{\pgfqpoint{5.700000in}{5.700000in}}%
\pgfusepath{clip}%
\pgfsetbuttcap%
\pgfsetroundjoin%
\definecolor{currentfill}{rgb}{0.269944,0.014625,0.341379}%
\pgfsetfillcolor{currentfill}%
\pgfsetfillopacity{0.700000}%
\pgfsetlinewidth{0.000000pt}%
\definecolor{currentstroke}{rgb}{0.000000,0.000000,0.000000}%
\pgfsetstrokecolor{currentstroke}%
\pgfsetdash{}{0pt}%
\pgfpathmoveto{\pgfqpoint{4.252242in}{2.096615in}}%
\pgfpathlineto{\pgfqpoint{4.265944in}{2.093554in}}%
\pgfpathlineto{\pgfqpoint{4.279653in}{2.090520in}}%
\pgfpathlineto{\pgfqpoint{4.293369in}{2.087512in}}%
\pgfpathlineto{\pgfqpoint{4.307092in}{2.084531in}}%
\pgfpathlineto{\pgfqpoint{4.299279in}{2.075991in}}%
\pgfpathlineto{\pgfqpoint{4.291461in}{2.067462in}}%
\pgfpathlineto{\pgfqpoint{4.283638in}{2.058947in}}%
\pgfpathlineto{\pgfqpoint{4.275809in}{2.050451in}}%
\pgfpathlineto{\pgfqpoint{4.262074in}{2.053567in}}%
\pgfpathlineto{\pgfqpoint{4.248347in}{2.056710in}}%
\pgfpathlineto{\pgfqpoint{4.234626in}{2.059880in}}%
\pgfpathlineto{\pgfqpoint{4.220911in}{2.063076in}}%
\pgfpathlineto{\pgfqpoint{4.228752in}{2.071432in}}%
\pgfpathlineto{\pgfqpoint{4.236588in}{2.079810in}}%
\pgfpathlineto{\pgfqpoint{4.244418in}{2.088205in}}%
\pgfpathlineto{\pgfqpoint{4.252242in}{2.096615in}}%
\pgfpathclose%
\pgfusepath{fill}%
\end{pgfscope}%
\begin{pgfscope}%
\pgfpathrectangle{\pgfqpoint{1.150000in}{0.150000in}}{\pgfqpoint{5.700000in}{5.700000in}}%
\pgfusepath{clip}%
\pgfsetbuttcap%
\pgfsetroundjoin%
\definecolor{currentfill}{rgb}{0.280894,0.078907,0.402329}%
\pgfsetfillcolor{currentfill}%
\pgfsetfillopacity{0.700000}%
\pgfsetlinewidth{0.000000pt}%
\definecolor{currentstroke}{rgb}{0.000000,0.000000,0.000000}%
\pgfsetstrokecolor{currentstroke}%
\pgfsetdash{}{0pt}%
\pgfpathmoveto{\pgfqpoint{4.792318in}{2.201508in}}%
\pgfpathlineto{\pgfqpoint{4.806168in}{2.199650in}}%
\pgfpathlineto{\pgfqpoint{4.820025in}{2.197819in}}%
\pgfpathlineto{\pgfqpoint{4.833891in}{2.196012in}}%
\pgfpathlineto{\pgfqpoint{4.847764in}{2.194231in}}%
\pgfpathlineto{\pgfqpoint{4.840142in}{2.185525in}}%
\pgfpathlineto{\pgfqpoint{4.832514in}{2.176765in}}%
\pgfpathlineto{\pgfqpoint{4.824879in}{2.167951in}}%
\pgfpathlineto{\pgfqpoint{4.817239in}{2.159085in}}%
\pgfpathlineto{\pgfqpoint{4.803355in}{2.160935in}}%
\pgfpathlineto{\pgfqpoint{4.789479in}{2.162810in}}%
\pgfpathlineto{\pgfqpoint{4.775611in}{2.164710in}}%
\pgfpathlineto{\pgfqpoint{4.761751in}{2.166636in}}%
\pgfpathlineto{\pgfqpoint{4.769401in}{2.175428in}}%
\pgfpathlineto{\pgfqpoint{4.777046in}{2.184172in}}%
\pgfpathlineto{\pgfqpoint{4.784685in}{2.192865in}}%
\pgfpathlineto{\pgfqpoint{4.792318in}{2.201508in}}%
\pgfpathclose%
\pgfusepath{fill}%
\end{pgfscope}%
\begin{pgfscope}%
\pgfpathrectangle{\pgfqpoint{1.150000in}{0.150000in}}{\pgfqpoint{5.700000in}{5.700000in}}%
\pgfusepath{clip}%
\pgfsetbuttcap%
\pgfsetroundjoin%
\definecolor{currentfill}{rgb}{0.281412,0.155834,0.469201}%
\pgfsetfillcolor{currentfill}%
\pgfsetfillopacity{0.700000}%
\pgfsetlinewidth{0.000000pt}%
\definecolor{currentstroke}{rgb}{0.000000,0.000000,0.000000}%
\pgfsetstrokecolor{currentstroke}%
\pgfsetdash{}{0pt}%
\pgfpathmoveto{\pgfqpoint{5.418470in}{2.348924in}}%
\pgfpathlineto{\pgfqpoint{5.432507in}{2.347994in}}%
\pgfpathlineto{\pgfqpoint{5.446554in}{2.347088in}}%
\pgfpathlineto{\pgfqpoint{5.460609in}{2.346207in}}%
\pgfpathlineto{\pgfqpoint{5.474672in}{2.345351in}}%
\pgfpathlineto{\pgfqpoint{5.467314in}{2.338244in}}%
\pgfpathlineto{\pgfqpoint{5.459948in}{2.331053in}}%
\pgfpathlineto{\pgfqpoint{5.452574in}{2.323776in}}%
\pgfpathlineto{\pgfqpoint{5.445192in}{2.316413in}}%
\pgfpathlineto{\pgfqpoint{5.431115in}{2.317256in}}%
\pgfpathlineto{\pgfqpoint{5.417046in}{2.318124in}}%
\pgfpathlineto{\pgfqpoint{5.402986in}{2.319016in}}%
\pgfpathlineto{\pgfqpoint{5.388935in}{2.319933in}}%
\pgfpathlineto{\pgfqpoint{5.396330in}{2.327305in}}%
\pgfpathlineto{\pgfqpoint{5.403717in}{2.334593in}}%
\pgfpathlineto{\pgfqpoint{5.411097in}{2.341799in}}%
\pgfpathlineto{\pgfqpoint{5.418470in}{2.348924in}}%
\pgfpathclose%
\pgfusepath{fill}%
\end{pgfscope}%
\begin{pgfscope}%
\pgfpathrectangle{\pgfqpoint{1.150000in}{0.150000in}}{\pgfqpoint{5.700000in}{5.700000in}}%
\pgfusepath{clip}%
\pgfsetbuttcap%
\pgfsetroundjoin%
\definecolor{currentfill}{rgb}{0.283197,0.115680,0.436115}%
\pgfsetfillcolor{currentfill}%
\pgfsetfillopacity{0.700000}%
\pgfsetlinewidth{0.000000pt}%
\definecolor{currentstroke}{rgb}{0.000000,0.000000,0.000000}%
\pgfsetstrokecolor{currentstroke}%
\pgfsetdash{}{0pt}%
\pgfpathmoveto{\pgfqpoint{5.105423in}{2.276181in}}%
\pgfpathlineto{\pgfqpoint{5.119366in}{2.274852in}}%
\pgfpathlineto{\pgfqpoint{5.133317in}{2.273547in}}%
\pgfpathlineto{\pgfqpoint{5.147276in}{2.272267in}}%
\pgfpathlineto{\pgfqpoint{5.161243in}{2.271013in}}%
\pgfpathlineto{\pgfqpoint{5.153745in}{2.262938in}}%
\pgfpathlineto{\pgfqpoint{5.146239in}{2.254787in}}%
\pgfpathlineto{\pgfqpoint{5.138727in}{2.246559in}}%
\pgfpathlineto{\pgfqpoint{5.131208in}{2.238256in}}%
\pgfpathlineto{\pgfqpoint{5.117229in}{2.239538in}}%
\pgfpathlineto{\pgfqpoint{5.103258in}{2.240846in}}%
\pgfpathlineto{\pgfqpoint{5.089295in}{2.242179in}}%
\pgfpathlineto{\pgfqpoint{5.075341in}{2.243536in}}%
\pgfpathlineto{\pgfqpoint{5.082872in}{2.251807in}}%
\pgfpathlineto{\pgfqpoint{5.090396in}{2.260004in}}%
\pgfpathlineto{\pgfqpoint{5.097913in}{2.268129in}}%
\pgfpathlineto{\pgfqpoint{5.105423in}{2.276181in}}%
\pgfpathclose%
\pgfusepath{fill}%
\end{pgfscope}%
\begin{pgfscope}%
\pgfpathrectangle{\pgfqpoint{1.150000in}{0.150000in}}{\pgfqpoint{5.700000in}{5.700000in}}%
\pgfusepath{clip}%
\pgfsetbuttcap%
\pgfsetroundjoin%
\definecolor{currentfill}{rgb}{0.267004,0.004874,0.329415}%
\pgfsetfillcolor{currentfill}%
\pgfsetfillopacity{0.700000}%
\pgfsetlinewidth{0.000000pt}%
\definecolor{currentstroke}{rgb}{0.000000,0.000000,0.000000}%
\pgfsetstrokecolor{currentstroke}%
\pgfsetdash{}{0pt}%
\pgfpathmoveto{\pgfqpoint{4.025282in}{2.072167in}}%
\pgfpathlineto{\pgfqpoint{4.038931in}{2.068498in}}%
\pgfpathlineto{\pgfqpoint{4.052586in}{2.064855in}}%
\pgfpathlineto{\pgfqpoint{4.066248in}{2.061240in}}%
\pgfpathlineto{\pgfqpoint{4.079916in}{2.057653in}}%
\pgfpathlineto{\pgfqpoint{4.072021in}{2.049767in}}%
\pgfpathlineto{\pgfqpoint{4.064121in}{2.041929in}}%
\pgfpathlineto{\pgfqpoint{4.056215in}{2.034144in}}%
\pgfpathlineto{\pgfqpoint{4.048302in}{2.026416in}}%
\pgfpathlineto{\pgfqpoint{4.034621in}{2.030165in}}%
\pgfpathlineto{\pgfqpoint{4.020947in}{2.033941in}}%
\pgfpathlineto{\pgfqpoint{4.007278in}{2.037745in}}%
\pgfpathlineto{\pgfqpoint{3.993616in}{2.041577in}}%
\pgfpathlineto{\pgfqpoint{4.001541in}{2.049139in}}%
\pgfpathlineto{\pgfqpoint{4.009461in}{2.056760in}}%
\pgfpathlineto{\pgfqpoint{4.017375in}{2.064438in}}%
\pgfpathlineto{\pgfqpoint{4.025282in}{2.072167in}}%
\pgfpathclose%
\pgfusepath{fill}%
\end{pgfscope}%
\begin{pgfscope}%
\pgfpathrectangle{\pgfqpoint{1.150000in}{0.150000in}}{\pgfqpoint{5.700000in}{5.700000in}}%
\pgfusepath{clip}%
\pgfsetbuttcap%
\pgfsetroundjoin%
\definecolor{currentfill}{rgb}{0.274952,0.037752,0.364543}%
\pgfsetfillcolor{currentfill}%
\pgfsetfillopacity{0.700000}%
\pgfsetlinewidth{0.000000pt}%
\definecolor{currentstroke}{rgb}{0.000000,0.000000,0.000000}%
\pgfsetstrokecolor{currentstroke}%
\pgfsetdash{}{0pt}%
\pgfpathmoveto{\pgfqpoint{4.479242in}{2.131871in}}%
\pgfpathlineto{\pgfqpoint{4.493006in}{2.129356in}}%
\pgfpathlineto{\pgfqpoint{4.506776in}{2.126868in}}%
\pgfpathlineto{\pgfqpoint{4.520554in}{2.124406in}}%
\pgfpathlineto{\pgfqpoint{4.534339in}{2.121969in}}%
\pgfpathlineto{\pgfqpoint{4.526604in}{2.113126in}}%
\pgfpathlineto{\pgfqpoint{4.518864in}{2.104263in}}%
\pgfpathlineto{\pgfqpoint{4.511118in}{2.095383in}}%
\pgfpathlineto{\pgfqpoint{4.503366in}{2.086487in}}%
\pgfpathlineto{\pgfqpoint{4.489571in}{2.089032in}}%
\pgfpathlineto{\pgfqpoint{4.475782in}{2.091603in}}%
\pgfpathlineto{\pgfqpoint{4.462001in}{2.094200in}}%
\pgfpathlineto{\pgfqpoint{4.448226in}{2.096823in}}%
\pgfpathlineto{\pgfqpoint{4.455989in}{2.105605in}}%
\pgfpathlineto{\pgfqpoint{4.463746in}{2.114376in}}%
\pgfpathlineto{\pgfqpoint{4.471497in}{2.123131in}}%
\pgfpathlineto{\pgfqpoint{4.479242in}{2.131871in}}%
\pgfpathclose%
\pgfusepath{fill}%
\end{pgfscope}%
\begin{pgfscope}%
\pgfpathrectangle{\pgfqpoint{1.150000in}{0.150000in}}{\pgfqpoint{5.700000in}{5.700000in}}%
\pgfusepath{clip}%
\pgfsetbuttcap%
\pgfsetroundjoin%
\definecolor{currentfill}{rgb}{0.282910,0.105393,0.426902}%
\pgfsetfillcolor{currentfill}%
\pgfsetfillopacity{0.700000}%
\pgfsetlinewidth{0.000000pt}%
\definecolor{currentstroke}{rgb}{0.000000,0.000000,0.000000}%
\pgfsetstrokecolor{currentstroke}%
\pgfsetdash{}{0pt}%
\pgfpathmoveto{\pgfqpoint{2.985105in}{2.249828in}}%
\pgfpathlineto{\pgfqpoint{2.998582in}{2.242841in}}%
\pgfpathlineto{\pgfqpoint{3.012062in}{2.235892in}}%
\pgfpathlineto{\pgfqpoint{3.025546in}{2.228981in}}%
\pgfpathlineto{\pgfqpoint{3.039033in}{2.222107in}}%
\pgfpathlineto{\pgfqpoint{3.030619in}{2.221105in}}%
\pgfpathlineto{\pgfqpoint{3.022190in}{2.220347in}}%
\pgfpathlineto{\pgfqpoint{3.013749in}{2.219840in}}%
\pgfpathlineto{\pgfqpoint{3.005293in}{2.219589in}}%
\pgfpathlineto{\pgfqpoint{2.991779in}{2.226720in}}%
\pgfpathlineto{\pgfqpoint{2.978268in}{2.233888in}}%
\pgfpathlineto{\pgfqpoint{2.964761in}{2.241094in}}%
\pgfpathlineto{\pgfqpoint{2.951257in}{2.248338in}}%
\pgfpathlineto{\pgfqpoint{2.959740in}{2.248326in}}%
\pgfpathlineto{\pgfqpoint{2.968209in}{2.248575in}}%
\pgfpathlineto{\pgfqpoint{2.976664in}{2.249078in}}%
\pgfpathlineto{\pgfqpoint{2.985105in}{2.249828in}}%
\pgfpathclose%
\pgfusepath{fill}%
\end{pgfscope}%
\begin{pgfscope}%
\pgfpathrectangle{\pgfqpoint{1.150000in}{0.150000in}}{\pgfqpoint{5.700000in}{5.700000in}}%
\pgfusepath{clip}%
\pgfsetbuttcap%
\pgfsetroundjoin%
\definecolor{currentfill}{rgb}{0.279566,0.067836,0.391917}%
\pgfsetfillcolor{currentfill}%
\pgfsetfillopacity{0.700000}%
\pgfsetlinewidth{0.000000pt}%
\definecolor{currentstroke}{rgb}{0.000000,0.000000,0.000000}%
\pgfsetstrokecolor{currentstroke}%
\pgfsetdash{}{0pt}%
\pgfpathmoveto{\pgfqpoint{3.180409in}{2.176681in}}%
\pgfpathlineto{\pgfqpoint{3.193908in}{2.170373in}}%
\pgfpathlineto{\pgfqpoint{3.207412in}{2.164100in}}%
\pgfpathlineto{\pgfqpoint{3.220919in}{2.157861in}}%
\pgfpathlineto{\pgfqpoint{3.234432in}{2.151657in}}%
\pgfpathlineto{\pgfqpoint{3.226139in}{2.149032in}}%
\pgfpathlineto{\pgfqpoint{3.217835in}{2.146616in}}%
\pgfpathlineto{\pgfqpoint{3.209520in}{2.144413in}}%
\pgfpathlineto{\pgfqpoint{3.201193in}{2.142431in}}%
\pgfpathlineto{\pgfqpoint{3.187657in}{2.148878in}}%
\pgfpathlineto{\pgfqpoint{3.174126in}{2.155359in}}%
\pgfpathlineto{\pgfqpoint{3.160598in}{2.161874in}}%
\pgfpathlineto{\pgfqpoint{3.147075in}{2.168425in}}%
\pgfpathlineto{\pgfqpoint{3.155426in}{2.170159in}}%
\pgfpathlineto{\pgfqpoint{3.163766in}{2.172118in}}%
\pgfpathlineto{\pgfqpoint{3.172093in}{2.174294in}}%
\pgfpathlineto{\pgfqpoint{3.180409in}{2.176681in}}%
\pgfpathclose%
\pgfusepath{fill}%
\end{pgfscope}%
\begin{pgfscope}%
\pgfpathrectangle{\pgfqpoint{1.150000in}{0.150000in}}{\pgfqpoint{5.700000in}{5.700000in}}%
\pgfusepath{clip}%
\pgfsetbuttcap%
\pgfsetroundjoin%
\definecolor{currentfill}{rgb}{0.267968,0.223549,0.512008}%
\pgfsetfillcolor{currentfill}%
\pgfsetfillopacity{0.700000}%
\pgfsetlinewidth{0.000000pt}%
\definecolor{currentstroke}{rgb}{0.000000,0.000000,0.000000}%
\pgfsetstrokecolor{currentstroke}%
\pgfsetdash{}{0pt}%
\pgfpathmoveto{\pgfqpoint{2.539477in}{2.478344in}}%
\pgfpathlineto{\pgfqpoint{2.552932in}{2.469622in}}%
\pgfpathlineto{\pgfqpoint{2.566388in}{2.460950in}}%
\pgfpathlineto{\pgfqpoint{2.579847in}{2.452325in}}%
\pgfpathlineto{\pgfqpoint{2.593308in}{2.443749in}}%
\pgfpathlineto{\pgfqpoint{2.584557in}{2.446858in}}%
\pgfpathlineto{\pgfqpoint{2.575786in}{2.450294in}}%
\pgfpathlineto{\pgfqpoint{2.566996in}{2.454063in}}%
\pgfpathlineto{\pgfqpoint{2.558186in}{2.458173in}}%
\pgfpathlineto{\pgfqpoint{2.544690in}{2.467038in}}%
\pgfpathlineto{\pgfqpoint{2.531196in}{2.475952in}}%
\pgfpathlineto{\pgfqpoint{2.517705in}{2.484913in}}%
\pgfpathlineto{\pgfqpoint{2.504215in}{2.493923in}}%
\pgfpathlineto{\pgfqpoint{2.513061in}{2.489519in}}%
\pgfpathlineto{\pgfqpoint{2.521886in}{2.485459in}}%
\pgfpathlineto{\pgfqpoint{2.530692in}{2.481737in}}%
\pgfpathlineto{\pgfqpoint{2.539477in}{2.478344in}}%
\pgfpathclose%
\pgfusepath{fill}%
\end{pgfscope}%
\begin{pgfscope}%
\pgfpathrectangle{\pgfqpoint{1.150000in}{0.150000in}}{\pgfqpoint{5.700000in}{5.700000in}}%
\pgfusepath{clip}%
\pgfsetbuttcap%
\pgfsetroundjoin%
\definecolor{currentfill}{rgb}{0.281412,0.155834,0.469201}%
\pgfsetfillcolor{currentfill}%
\pgfsetfillopacity{0.700000}%
\pgfsetlinewidth{0.000000pt}%
\definecolor{currentstroke}{rgb}{0.000000,0.000000,0.000000}%
\pgfsetstrokecolor{currentstroke}%
\pgfsetdash{}{0pt}%
\pgfpathmoveto{\pgfqpoint{2.789476in}{2.338334in}}%
\pgfpathlineto{\pgfqpoint{2.802940in}{2.330611in}}%
\pgfpathlineto{\pgfqpoint{2.816407in}{2.322929in}}%
\pgfpathlineto{\pgfqpoint{2.829878in}{2.315289in}}%
\pgfpathlineto{\pgfqpoint{2.843351in}{2.307690in}}%
\pgfpathlineto{\pgfqpoint{2.834795in}{2.308502in}}%
\pgfpathlineto{\pgfqpoint{2.826224in}{2.309596in}}%
\pgfpathlineto{\pgfqpoint{2.817636in}{2.310978in}}%
\pgfpathlineto{\pgfqpoint{2.809032in}{2.312657in}}%
\pgfpathlineto{\pgfqpoint{2.795528in}{2.320528in}}%
\pgfpathlineto{\pgfqpoint{2.782028in}{2.328440in}}%
\pgfpathlineto{\pgfqpoint{2.768530in}{2.336394in}}%
\pgfpathlineto{\pgfqpoint{2.755035in}{2.344390in}}%
\pgfpathlineto{\pgfqpoint{2.763671in}{2.342433in}}%
\pgfpathlineto{\pgfqpoint{2.772289in}{2.340777in}}%
\pgfpathlineto{\pgfqpoint{2.780891in}{2.339413in}}%
\pgfpathlineto{\pgfqpoint{2.789476in}{2.338334in}}%
\pgfpathclose%
\pgfusepath{fill}%
\end{pgfscope}%
\begin{pgfscope}%
\pgfpathrectangle{\pgfqpoint{1.150000in}{0.150000in}}{\pgfqpoint{5.700000in}{5.700000in}}%
\pgfusepath{clip}%
\pgfsetbuttcap%
\pgfsetroundjoin%
\definecolor{currentfill}{rgb}{0.278012,0.180367,0.486697}%
\pgfsetfillcolor{currentfill}%
\pgfsetfillopacity{0.700000}%
\pgfsetlinewidth{0.000000pt}%
\definecolor{currentstroke}{rgb}{0.000000,0.000000,0.000000}%
\pgfsetstrokecolor{currentstroke}%
\pgfsetdash{}{0pt}%
\pgfpathmoveto{\pgfqpoint{5.645845in}{2.392789in}}%
\pgfpathlineto{\pgfqpoint{5.659957in}{2.392086in}}%
\pgfpathlineto{\pgfqpoint{5.674078in}{2.391408in}}%
\pgfpathlineto{\pgfqpoint{5.688208in}{2.390754in}}%
\pgfpathlineto{\pgfqpoint{5.702347in}{2.390125in}}%
\pgfpathlineto{\pgfqpoint{5.695097in}{2.383789in}}%
\pgfpathlineto{\pgfqpoint{5.687839in}{2.377369in}}%
\pgfpathlineto{\pgfqpoint{5.680573in}{2.370864in}}%
\pgfpathlineto{\pgfqpoint{5.673298in}{2.364271in}}%
\pgfpathlineto{\pgfqpoint{5.659144in}{2.364859in}}%
\pgfpathlineto{\pgfqpoint{5.644999in}{2.365471in}}%
\pgfpathlineto{\pgfqpoint{5.630862in}{2.366109in}}%
\pgfpathlineto{\pgfqpoint{5.616734in}{2.366771in}}%
\pgfpathlineto{\pgfqpoint{5.624024in}{2.373399in}}%
\pgfpathlineto{\pgfqpoint{5.631306in}{2.379944in}}%
\pgfpathlineto{\pgfqpoint{5.638579in}{2.386407in}}%
\pgfpathlineto{\pgfqpoint{5.645845in}{2.392789in}}%
\pgfpathclose%
\pgfusepath{fill}%
\end{pgfscope}%
\begin{pgfscope}%
\pgfpathrectangle{\pgfqpoint{1.150000in}{0.150000in}}{\pgfqpoint{5.700000in}{5.700000in}}%
\pgfusepath{clip}%
\pgfsetbuttcap%
\pgfsetroundjoin%
\definecolor{currentfill}{rgb}{0.268510,0.009605,0.335427}%
\pgfsetfillcolor{currentfill}%
\pgfsetfillopacity{0.700000}%
\pgfsetlinewidth{0.000000pt}%
\definecolor{currentstroke}{rgb}{0.000000,0.000000,0.000000}%
\pgfsetstrokecolor{currentstroke}%
\pgfsetdash{}{0pt}%
\pgfpathmoveto{\pgfqpoint{3.657393in}{2.073558in}}%
\pgfpathlineto{\pgfqpoint{3.670969in}{2.068792in}}%
\pgfpathlineto{\pgfqpoint{3.684550in}{2.064056in}}%
\pgfpathlineto{\pgfqpoint{3.698137in}{2.059349in}}%
\pgfpathlineto{\pgfqpoint{3.711730in}{2.054672in}}%
\pgfpathlineto{\pgfqpoint{3.703684in}{2.048627in}}%
\pgfpathlineto{\pgfqpoint{3.695630in}{2.042700in}}%
\pgfpathlineto{\pgfqpoint{3.687569in}{2.036896in}}%
\pgfpathlineto{\pgfqpoint{3.679501in}{2.031219in}}%
\pgfpathlineto{\pgfqpoint{3.665891in}{2.036098in}}%
\pgfpathlineto{\pgfqpoint{3.652287in}{2.041006in}}%
\pgfpathlineto{\pgfqpoint{3.638688in}{2.045943in}}%
\pgfpathlineto{\pgfqpoint{3.625095in}{2.050911in}}%
\pgfpathlineto{\pgfqpoint{3.633181in}{2.056381in}}%
\pgfpathlineto{\pgfqpoint{3.641259in}{2.061982in}}%
\pgfpathlineto{\pgfqpoint{3.649330in}{2.067709in}}%
\pgfpathlineto{\pgfqpoint{3.657393in}{2.073558in}}%
\pgfpathclose%
\pgfusepath{fill}%
\end{pgfscope}%
\begin{pgfscope}%
\pgfpathrectangle{\pgfqpoint{1.150000in}{0.150000in}}{\pgfqpoint{5.700000in}{5.700000in}}%
\pgfusepath{clip}%
\pgfsetbuttcap%
\pgfsetroundjoin%
\definecolor{currentfill}{rgb}{0.279566,0.067836,0.391917}%
\pgfsetfillcolor{currentfill}%
\pgfsetfillopacity{0.700000}%
\pgfsetlinewidth{0.000000pt}%
\definecolor{currentstroke}{rgb}{0.000000,0.000000,0.000000}%
\pgfsetstrokecolor{currentstroke}%
\pgfsetdash{}{0pt}%
\pgfpathmoveto{\pgfqpoint{4.706384in}{2.174594in}}%
\pgfpathlineto{\pgfqpoint{4.720215in}{2.172566in}}%
\pgfpathlineto{\pgfqpoint{4.734052in}{2.170564in}}%
\pgfpathlineto{\pgfqpoint{4.747898in}{2.168587in}}%
\pgfpathlineto{\pgfqpoint{4.761751in}{2.166636in}}%
\pgfpathlineto{\pgfqpoint{4.754094in}{2.157797in}}%
\pgfpathlineto{\pgfqpoint{4.746431in}{2.148911in}}%
\pgfpathlineto{\pgfqpoint{4.738762in}{2.139982in}}%
\pgfpathlineto{\pgfqpoint{4.731088in}{2.131010in}}%
\pgfpathlineto{\pgfqpoint{4.717224in}{2.133043in}}%
\pgfpathlineto{\pgfqpoint{4.703368in}{2.135102in}}%
\pgfpathlineto{\pgfqpoint{4.689520in}{2.137186in}}%
\pgfpathlineto{\pgfqpoint{4.675679in}{2.139296in}}%
\pgfpathlineto{\pgfqpoint{4.683364in}{2.148181in}}%
\pgfpathlineto{\pgfqpoint{4.691044in}{2.157026in}}%
\pgfpathlineto{\pgfqpoint{4.698717in}{2.165831in}}%
\pgfpathlineto{\pgfqpoint{4.706384in}{2.174594in}}%
\pgfpathclose%
\pgfusepath{fill}%
\end{pgfscope}%
\begin{pgfscope}%
\pgfpathrectangle{\pgfqpoint{1.150000in}{0.150000in}}{\pgfqpoint{5.700000in}{5.700000in}}%
\pgfusepath{clip}%
\pgfsetbuttcap%
\pgfsetroundjoin%
\definecolor{currentfill}{rgb}{0.282910,0.105393,0.426902}%
\pgfsetfillcolor{currentfill}%
\pgfsetfillopacity{0.700000}%
\pgfsetlinewidth{0.000000pt}%
\definecolor{currentstroke}{rgb}{0.000000,0.000000,0.000000}%
\pgfsetstrokecolor{currentstroke}%
\pgfsetdash{}{0pt}%
\pgfpathmoveto{\pgfqpoint{5.019604in}{2.249217in}}%
\pgfpathlineto{\pgfqpoint{5.033526in}{2.247759in}}%
\pgfpathlineto{\pgfqpoint{5.047456in}{2.246326in}}%
\pgfpathlineto{\pgfqpoint{5.061395in}{2.244919in}}%
\pgfpathlineto{\pgfqpoint{5.075341in}{2.243536in}}%
\pgfpathlineto{\pgfqpoint{5.067803in}{2.235194in}}%
\pgfpathlineto{\pgfqpoint{5.060259in}{2.226779in}}%
\pgfpathlineto{\pgfqpoint{5.052709in}{2.218294in}}%
\pgfpathlineto{\pgfqpoint{5.045151in}{2.209738in}}%
\pgfpathlineto{\pgfqpoint{5.031194in}{2.211162in}}%
\pgfpathlineto{\pgfqpoint{5.017244in}{2.212611in}}%
\pgfpathlineto{\pgfqpoint{5.003303in}{2.214085in}}%
\pgfpathlineto{\pgfqpoint{4.989369in}{2.215585in}}%
\pgfpathlineto{\pgfqpoint{4.996938in}{2.224094in}}%
\pgfpathlineto{\pgfqpoint{5.004500in}{2.232536in}}%
\pgfpathlineto{\pgfqpoint{5.012055in}{2.240911in}}%
\pgfpathlineto{\pgfqpoint{5.019604in}{2.249217in}}%
\pgfpathclose%
\pgfusepath{fill}%
\end{pgfscope}%
\begin{pgfscope}%
\pgfpathrectangle{\pgfqpoint{1.150000in}{0.150000in}}{\pgfqpoint{5.700000in}{5.700000in}}%
\pgfusepath{clip}%
\pgfsetbuttcap%
\pgfsetroundjoin%
\definecolor{currentfill}{rgb}{0.271305,0.019942,0.347269}%
\pgfsetfillcolor{currentfill}%
\pgfsetfillopacity{0.700000}%
\pgfsetlinewidth{0.000000pt}%
\definecolor{currentstroke}{rgb}{0.000000,0.000000,0.000000}%
\pgfsetstrokecolor{currentstroke}%
\pgfsetdash{}{0pt}%
\pgfpathmoveto{\pgfqpoint{3.516539in}{2.091733in}}%
\pgfpathlineto{\pgfqpoint{3.530090in}{2.086523in}}%
\pgfpathlineto{\pgfqpoint{3.543647in}{2.081344in}}%
\pgfpathlineto{\pgfqpoint{3.557208in}{2.076196in}}%
\pgfpathlineto{\pgfqpoint{3.570775in}{2.071079in}}%
\pgfpathlineto{\pgfqpoint{3.562663in}{2.065954in}}%
\pgfpathlineto{\pgfqpoint{3.554542in}{2.060975in}}%
\pgfpathlineto{\pgfqpoint{3.546413in}{2.056146in}}%
\pgfpathlineto{\pgfqpoint{3.538275in}{2.051472in}}%
\pgfpathlineto{\pgfqpoint{3.524689in}{2.056805in}}%
\pgfpathlineto{\pgfqpoint{3.511109in}{2.062168in}}%
\pgfpathlineto{\pgfqpoint{3.497533in}{2.067562in}}%
\pgfpathlineto{\pgfqpoint{3.483963in}{2.072986in}}%
\pgfpathlineto{\pgfqpoint{3.492120in}{2.077440in}}%
\pgfpathlineto{\pgfqpoint{3.500268in}{2.082052in}}%
\pgfpathlineto{\pgfqpoint{3.508408in}{2.086818in}}%
\pgfpathlineto{\pgfqpoint{3.516539in}{2.091733in}}%
\pgfpathclose%
\pgfusepath{fill}%
\end{pgfscope}%
\begin{pgfscope}%
\pgfpathrectangle{\pgfqpoint{1.150000in}{0.150000in}}{\pgfqpoint{5.700000in}{5.700000in}}%
\pgfusepath{clip}%
\pgfsetbuttcap%
\pgfsetroundjoin%
\definecolor{currentfill}{rgb}{0.282290,0.145912,0.461510}%
\pgfsetfillcolor{currentfill}%
\pgfsetfillopacity{0.700000}%
\pgfsetlinewidth{0.000000pt}%
\definecolor{currentstroke}{rgb}{0.000000,0.000000,0.000000}%
\pgfsetstrokecolor{currentstroke}%
\pgfsetdash{}{0pt}%
\pgfpathmoveto{\pgfqpoint{5.332814in}{2.323850in}}%
\pgfpathlineto{\pgfqpoint{5.346832in}{2.322834in}}%
\pgfpathlineto{\pgfqpoint{5.360857in}{2.321842in}}%
\pgfpathlineto{\pgfqpoint{5.374892in}{2.320875in}}%
\pgfpathlineto{\pgfqpoint{5.388935in}{2.319933in}}%
\pgfpathlineto{\pgfqpoint{5.381532in}{2.312478in}}%
\pgfpathlineto{\pgfqpoint{5.374122in}{2.304938in}}%
\pgfpathlineto{\pgfqpoint{5.366704in}{2.297313in}}%
\pgfpathlineto{\pgfqpoint{5.359279in}{2.289602in}}%
\pgfpathlineto{\pgfqpoint{5.345223in}{2.290544in}}%
\pgfpathlineto{\pgfqpoint{5.331176in}{2.291512in}}%
\pgfpathlineto{\pgfqpoint{5.317137in}{2.292504in}}%
\pgfpathlineto{\pgfqpoint{5.303106in}{2.293520in}}%
\pgfpathlineto{\pgfqpoint{5.310544in}{2.301226in}}%
\pgfpathlineto{\pgfqpoint{5.317975in}{2.308849in}}%
\pgfpathlineto{\pgfqpoint{5.325398in}{2.316390in}}%
\pgfpathlineto{\pgfqpoint{5.332814in}{2.323850in}}%
\pgfpathclose%
\pgfusepath{fill}%
\end{pgfscope}%
\begin{pgfscope}%
\pgfpathrectangle{\pgfqpoint{1.150000in}{0.150000in}}{\pgfqpoint{5.700000in}{5.700000in}}%
\pgfusepath{clip}%
\pgfsetbuttcap%
\pgfsetroundjoin%
\definecolor{currentfill}{rgb}{0.267004,0.004874,0.329415}%
\pgfsetfillcolor{currentfill}%
\pgfsetfillopacity{0.700000}%
\pgfsetlinewidth{0.000000pt}%
\definecolor{currentstroke}{rgb}{0.000000,0.000000,0.000000}%
\pgfsetstrokecolor{currentstroke}%
\pgfsetdash{}{0pt}%
\pgfpathmoveto{\pgfqpoint{3.798203in}{2.062267in}}%
\pgfpathlineto{\pgfqpoint{3.811808in}{2.057923in}}%
\pgfpathlineto{\pgfqpoint{3.825419in}{2.053608in}}%
\pgfpathlineto{\pgfqpoint{3.839035in}{2.049321in}}%
\pgfpathlineto{\pgfqpoint{3.852658in}{2.045063in}}%
\pgfpathlineto{\pgfqpoint{3.844672in}{2.038229in}}%
\pgfpathlineto{\pgfqpoint{3.836679in}{2.031488in}}%
\pgfpathlineto{\pgfqpoint{3.828679in}{2.024843in}}%
\pgfpathlineto{\pgfqpoint{3.820672in}{2.018299in}}%
\pgfpathlineto{\pgfqpoint{3.807034in}{2.022745in}}%
\pgfpathlineto{\pgfqpoint{3.793402in}{2.027220in}}%
\pgfpathlineto{\pgfqpoint{3.779776in}{2.031723in}}%
\pgfpathlineto{\pgfqpoint{3.766155in}{2.036255in}}%
\pgfpathlineto{\pgfqpoint{3.774178in}{2.042606in}}%
\pgfpathlineto{\pgfqpoint{3.782193in}{2.049061in}}%
\pgfpathlineto{\pgfqpoint{3.790202in}{2.055616in}}%
\pgfpathlineto{\pgfqpoint{3.798203in}{2.062267in}}%
\pgfpathclose%
\pgfusepath{fill}%
\end{pgfscope}%
\begin{pgfscope}%
\pgfpathrectangle{\pgfqpoint{1.150000in}{0.150000in}}{\pgfqpoint{5.700000in}{5.700000in}}%
\pgfusepath{clip}%
\pgfsetbuttcap%
\pgfsetroundjoin%
\definecolor{currentfill}{rgb}{0.268510,0.009605,0.335427}%
\pgfsetfillcolor{currentfill}%
\pgfsetfillopacity{0.700000}%
\pgfsetlinewidth{0.000000pt}%
\definecolor{currentstroke}{rgb}{0.000000,0.000000,0.000000}%
\pgfsetstrokecolor{currentstroke}%
\pgfsetdash{}{0pt}%
\pgfpathmoveto{\pgfqpoint{4.166121in}{2.076129in}}%
\pgfpathlineto{\pgfqpoint{4.179809in}{2.072825in}}%
\pgfpathlineto{\pgfqpoint{4.193503in}{2.069549in}}%
\pgfpathlineto{\pgfqpoint{4.207204in}{2.066299in}}%
\pgfpathlineto{\pgfqpoint{4.220911in}{2.063076in}}%
\pgfpathlineto{\pgfqpoint{4.213065in}{2.054744in}}%
\pgfpathlineto{\pgfqpoint{4.205213in}{2.046440in}}%
\pgfpathlineto{\pgfqpoint{4.197355in}{2.038167in}}%
\pgfpathlineto{\pgfqpoint{4.189491in}{2.029928in}}%
\pgfpathlineto{\pgfqpoint{4.175772in}{2.033300in}}%
\pgfpathlineto{\pgfqpoint{4.162058in}{2.036698in}}%
\pgfpathlineto{\pgfqpoint{4.148352in}{2.040123in}}%
\pgfpathlineto{\pgfqpoint{4.134652in}{2.043575in}}%
\pgfpathlineto{\pgfqpoint{4.142528in}{2.051660in}}%
\pgfpathlineto{\pgfqpoint{4.150398in}{2.059783in}}%
\pgfpathlineto{\pgfqpoint{4.158262in}{2.067940in}}%
\pgfpathlineto{\pgfqpoint{4.166121in}{2.076129in}}%
\pgfpathclose%
\pgfusepath{fill}%
\end{pgfscope}%
\begin{pgfscope}%
\pgfpathrectangle{\pgfqpoint{1.150000in}{0.150000in}}{\pgfqpoint{5.700000in}{5.700000in}}%
\pgfusepath{clip}%
\pgfsetbuttcap%
\pgfsetroundjoin%
\definecolor{currentfill}{rgb}{0.272594,0.025563,0.353093}%
\pgfsetfillcolor{currentfill}%
\pgfsetfillopacity{0.700000}%
\pgfsetlinewidth{0.000000pt}%
\definecolor{currentstroke}{rgb}{0.000000,0.000000,0.000000}%
\pgfsetstrokecolor{currentstroke}%
\pgfsetdash{}{0pt}%
\pgfpathmoveto{\pgfqpoint{4.393200in}{2.107576in}}%
\pgfpathlineto{\pgfqpoint{4.406946in}{2.104849in}}%
\pgfpathlineto{\pgfqpoint{4.420699in}{2.102148in}}%
\pgfpathlineto{\pgfqpoint{4.434459in}{2.099472in}}%
\pgfpathlineto{\pgfqpoint{4.448226in}{2.096823in}}%
\pgfpathlineto{\pgfqpoint{4.440459in}{2.088032in}}%
\pgfpathlineto{\pgfqpoint{4.432685in}{2.079234in}}%
\pgfpathlineto{\pgfqpoint{4.424906in}{2.070432in}}%
\pgfpathlineto{\pgfqpoint{4.417122in}{2.061629in}}%
\pgfpathlineto{\pgfqpoint{4.403344in}{2.064400in}}%
\pgfpathlineto{\pgfqpoint{4.389573in}{2.067197in}}%
\pgfpathlineto{\pgfqpoint{4.375808in}{2.070021in}}%
\pgfpathlineto{\pgfqpoint{4.362051in}{2.072870in}}%
\pgfpathlineto{\pgfqpoint{4.369847in}{2.081546in}}%
\pgfpathlineto{\pgfqpoint{4.377637in}{2.090224in}}%
\pgfpathlineto{\pgfqpoint{4.385421in}{2.098902in}}%
\pgfpathlineto{\pgfqpoint{4.393200in}{2.107576in}}%
\pgfpathclose%
\pgfusepath{fill}%
\end{pgfscope}%
\begin{pgfscope}%
\pgfpathrectangle{\pgfqpoint{1.150000in}{0.150000in}}{\pgfqpoint{5.700000in}{5.700000in}}%
\pgfusepath{clip}%
\pgfsetbuttcap%
\pgfsetroundjoin%
\definecolor{currentfill}{rgb}{0.273006,0.204520,0.501721}%
\pgfsetfillcolor{currentfill}%
\pgfsetfillopacity{0.700000}%
\pgfsetlinewidth{0.000000pt}%
\definecolor{currentstroke}{rgb}{0.000000,0.000000,0.000000}%
\pgfsetstrokecolor{currentstroke}%
\pgfsetdash{}{0pt}%
\pgfpathmoveto{\pgfqpoint{5.873217in}{2.432885in}}%
\pgfpathlineto{\pgfqpoint{5.887402in}{2.432353in}}%
\pgfpathlineto{\pgfqpoint{5.901597in}{2.431845in}}%
\pgfpathlineto{\pgfqpoint{5.915801in}{2.431362in}}%
\pgfpathlineto{\pgfqpoint{5.930014in}{2.430903in}}%
\pgfpathlineto{\pgfqpoint{5.922882in}{2.425388in}}%
\pgfpathlineto{\pgfqpoint{5.915741in}{2.419796in}}%
\pgfpathlineto{\pgfqpoint{5.908592in}{2.414124in}}%
\pgfpathlineto{\pgfqpoint{5.901434in}{2.408370in}}%
\pgfpathlineto{\pgfqpoint{5.887203in}{2.408760in}}%
\pgfpathlineto{\pgfqpoint{5.872982in}{2.409175in}}%
\pgfpathlineto{\pgfqpoint{5.858769in}{2.409614in}}%
\pgfpathlineto{\pgfqpoint{5.844566in}{2.410077in}}%
\pgfpathlineto{\pgfqpoint{5.851741in}{2.415895in}}%
\pgfpathlineto{\pgfqpoint{5.858908in}{2.421634in}}%
\pgfpathlineto{\pgfqpoint{5.866066in}{2.427296in}}%
\pgfpathlineto{\pgfqpoint{5.873217in}{2.432885in}}%
\pgfpathclose%
\pgfusepath{fill}%
\end{pgfscope}%
\begin{pgfscope}%
\pgfpathrectangle{\pgfqpoint{1.150000in}{0.150000in}}{\pgfqpoint{5.700000in}{5.700000in}}%
\pgfusepath{clip}%
\pgfsetbuttcap%
\pgfsetroundjoin%
\definecolor{currentfill}{rgb}{0.274952,0.037752,0.364543}%
\pgfsetfillcolor{currentfill}%
\pgfsetfillopacity{0.700000}%
\pgfsetlinewidth{0.000000pt}%
\definecolor{currentstroke}{rgb}{0.000000,0.000000,0.000000}%
\pgfsetstrokecolor{currentstroke}%
\pgfsetdash{}{0pt}%
\pgfpathmoveto{\pgfqpoint{3.375576in}{2.117512in}}%
\pgfpathlineto{\pgfqpoint{3.389108in}{2.111834in}}%
\pgfpathlineto{\pgfqpoint{3.402644in}{2.106190in}}%
\pgfpathlineto{\pgfqpoint{3.416185in}{2.100577in}}%
\pgfpathlineto{\pgfqpoint{3.429731in}{2.094996in}}%
\pgfpathlineto{\pgfqpoint{3.421544in}{2.090930in}}%
\pgfpathlineto{\pgfqpoint{3.413348in}{2.087038in}}%
\pgfpathlineto{\pgfqpoint{3.405143in}{2.083326in}}%
\pgfpathlineto{\pgfqpoint{3.396927in}{2.079799in}}%
\pgfpathlineto{\pgfqpoint{3.383360in}{2.085608in}}%
\pgfpathlineto{\pgfqpoint{3.369798in}{2.091450in}}%
\pgfpathlineto{\pgfqpoint{3.356241in}{2.097323in}}%
\pgfpathlineto{\pgfqpoint{3.342689in}{2.103229in}}%
\pgfpathlineto{\pgfqpoint{3.350926in}{2.106522in}}%
\pgfpathlineto{\pgfqpoint{3.359152in}{2.110004in}}%
\pgfpathlineto{\pgfqpoint{3.367369in}{2.113669in}}%
\pgfpathlineto{\pgfqpoint{3.375576in}{2.117512in}}%
\pgfpathclose%
\pgfusepath{fill}%
\end{pgfscope}%
\begin{pgfscope}%
\pgfpathrectangle{\pgfqpoint{1.150000in}{0.150000in}}{\pgfqpoint{5.700000in}{5.700000in}}%
\pgfusepath{clip}%
\pgfsetbuttcap%
\pgfsetroundjoin%
\definecolor{currentfill}{rgb}{0.267004,0.004874,0.329415}%
\pgfsetfillcolor{currentfill}%
\pgfsetfillopacity{0.700000}%
\pgfsetlinewidth{0.000000pt}%
\definecolor{currentstroke}{rgb}{0.000000,0.000000,0.000000}%
\pgfsetstrokecolor{currentstroke}%
\pgfsetdash{}{0pt}%
\pgfpathmoveto{\pgfqpoint{3.939028in}{2.057180in}}%
\pgfpathlineto{\pgfqpoint{3.952665in}{2.053237in}}%
\pgfpathlineto{\pgfqpoint{3.966309in}{2.049323in}}%
\pgfpathlineto{\pgfqpoint{3.979959in}{2.045436in}}%
\pgfpathlineto{\pgfqpoint{3.993616in}{2.041577in}}%
\pgfpathlineto{\pgfqpoint{3.985684in}{2.034079in}}%
\pgfpathlineto{\pgfqpoint{3.977746in}{2.026649in}}%
\pgfpathlineto{\pgfqpoint{3.969801in}{2.019292in}}%
\pgfpathlineto{\pgfqpoint{3.961851in}{2.012011in}}%
\pgfpathlineto{\pgfqpoint{3.948181in}{2.016045in}}%
\pgfpathlineto{\pgfqpoint{3.934517in}{2.020106in}}%
\pgfpathlineto{\pgfqpoint{3.920858in}{2.024196in}}%
\pgfpathlineto{\pgfqpoint{3.907206in}{2.028313in}}%
\pgfpathlineto{\pgfqpoint{3.915171in}{2.035414in}}%
\pgfpathlineto{\pgfqpoint{3.923130in}{2.042595in}}%
\pgfpathlineto{\pgfqpoint{3.931082in}{2.049852in}}%
\pgfpathlineto{\pgfqpoint{3.939028in}{2.057180in}}%
\pgfpathclose%
\pgfusepath{fill}%
\end{pgfscope}%
\begin{pgfscope}%
\pgfpathrectangle{\pgfqpoint{1.150000in}{0.150000in}}{\pgfqpoint{5.700000in}{5.700000in}}%
\pgfusepath{clip}%
\pgfsetbuttcap%
\pgfsetroundjoin%
\definecolor{currentfill}{rgb}{0.277941,0.056324,0.381191}%
\pgfsetfillcolor{currentfill}%
\pgfsetfillopacity{0.700000}%
\pgfsetlinewidth{0.000000pt}%
\definecolor{currentstroke}{rgb}{0.000000,0.000000,0.000000}%
\pgfsetstrokecolor{currentstroke}%
\pgfsetdash{}{0pt}%
\pgfpathmoveto{\pgfqpoint{4.620391in}{2.147991in}}%
\pgfpathlineto{\pgfqpoint{4.634202in}{2.145779in}}%
\pgfpathlineto{\pgfqpoint{4.648020in}{2.143592in}}%
\pgfpathlineto{\pgfqpoint{4.661846in}{2.141431in}}%
\pgfpathlineto{\pgfqpoint{4.675679in}{2.139296in}}%
\pgfpathlineto{\pgfqpoint{4.667989in}{2.130374in}}%
\pgfpathlineto{\pgfqpoint{4.660292in}{2.121417in}}%
\pgfpathlineto{\pgfqpoint{4.652590in}{2.112427in}}%
\pgfpathlineto{\pgfqpoint{4.644882in}{2.103405in}}%
\pgfpathlineto{\pgfqpoint{4.631038in}{2.105636in}}%
\pgfpathlineto{\pgfqpoint{4.617202in}{2.107892in}}%
\pgfpathlineto{\pgfqpoint{4.603373in}{2.110174in}}%
\pgfpathlineto{\pgfqpoint{4.589551in}{2.112482in}}%
\pgfpathlineto{\pgfqpoint{4.597270in}{2.121403in}}%
\pgfpathlineto{\pgfqpoint{4.604983in}{2.130296in}}%
\pgfpathlineto{\pgfqpoint{4.612690in}{2.139159in}}%
\pgfpathlineto{\pgfqpoint{4.620391in}{2.147991in}}%
\pgfpathclose%
\pgfusepath{fill}%
\end{pgfscope}%
\begin{pgfscope}%
\pgfpathrectangle{\pgfqpoint{1.150000in}{0.150000in}}{\pgfqpoint{5.700000in}{5.700000in}}%
\pgfusepath{clip}%
\pgfsetbuttcap%
\pgfsetroundjoin%
\definecolor{currentfill}{rgb}{0.282327,0.094955,0.417331}%
\pgfsetfillcolor{currentfill}%
\pgfsetfillopacity{0.700000}%
\pgfsetlinewidth{0.000000pt}%
\definecolor{currentstroke}{rgb}{0.000000,0.000000,0.000000}%
\pgfsetstrokecolor{currentstroke}%
\pgfsetdash{}{0pt}%
\pgfpathmoveto{\pgfqpoint{4.933716in}{2.221833in}}%
\pgfpathlineto{\pgfqpoint{4.947618in}{2.220233in}}%
\pgfpathlineto{\pgfqpoint{4.961527in}{2.218658in}}%
\pgfpathlineto{\pgfqpoint{4.975444in}{2.217109in}}%
\pgfpathlineto{\pgfqpoint{4.989369in}{2.215585in}}%
\pgfpathlineto{\pgfqpoint{4.981795in}{2.207009in}}%
\pgfpathlineto{\pgfqpoint{4.974213in}{2.198367in}}%
\pgfpathlineto{\pgfqpoint{4.966625in}{2.189661in}}%
\pgfpathlineto{\pgfqpoint{4.959031in}{2.180890in}}%
\pgfpathlineto{\pgfqpoint{4.945095in}{2.182470in}}%
\pgfpathlineto{\pgfqpoint{4.931167in}{2.184075in}}%
\pgfpathlineto{\pgfqpoint{4.917246in}{2.185704in}}%
\pgfpathlineto{\pgfqpoint{4.903334in}{2.187359in}}%
\pgfpathlineto{\pgfqpoint{4.910939in}{2.196069in}}%
\pgfpathlineto{\pgfqpoint{4.918538in}{2.204719in}}%
\pgfpathlineto{\pgfqpoint{4.926130in}{2.213307in}}%
\pgfpathlineto{\pgfqpoint{4.933716in}{2.221833in}}%
\pgfpathclose%
\pgfusepath{fill}%
\end{pgfscope}%
\begin{pgfscope}%
\pgfpathrectangle{\pgfqpoint{1.150000in}{0.150000in}}{\pgfqpoint{5.700000in}{5.700000in}}%
\pgfusepath{clip}%
\pgfsetbuttcap%
\pgfsetroundjoin%
\definecolor{currentfill}{rgb}{0.279574,0.170599,0.479997}%
\pgfsetfillcolor{currentfill}%
\pgfsetfillopacity{0.700000}%
\pgfsetlinewidth{0.000000pt}%
\definecolor{currentstroke}{rgb}{0.000000,0.000000,0.000000}%
\pgfsetstrokecolor{currentstroke}%
\pgfsetdash{}{0pt}%
\pgfpathmoveto{\pgfqpoint{5.560312in}{2.369664in}}%
\pgfpathlineto{\pgfqpoint{5.574404in}{2.368903in}}%
\pgfpathlineto{\pgfqpoint{5.588505in}{2.368168in}}%
\pgfpathlineto{\pgfqpoint{5.602615in}{2.367457in}}%
\pgfpathlineto{\pgfqpoint{5.616734in}{2.366771in}}%
\pgfpathlineto{\pgfqpoint{5.609436in}{2.360056in}}%
\pgfpathlineto{\pgfqpoint{5.602131in}{2.353255in}}%
\pgfpathlineto{\pgfqpoint{5.594817in}{2.346366in}}%
\pgfpathlineto{\pgfqpoint{5.587494in}{2.339387in}}%
\pgfpathlineto{\pgfqpoint{5.573361in}{2.340046in}}%
\pgfpathlineto{\pgfqpoint{5.559236in}{2.340730in}}%
\pgfpathlineto{\pgfqpoint{5.545120in}{2.341439in}}%
\pgfpathlineto{\pgfqpoint{5.531013in}{2.342172in}}%
\pgfpathlineto{\pgfqpoint{5.538350in}{2.349172in}}%
\pgfpathlineto{\pgfqpoint{5.545678in}{2.356087in}}%
\pgfpathlineto{\pgfqpoint{5.552999in}{2.362917in}}%
\pgfpathlineto{\pgfqpoint{5.560312in}{2.369664in}}%
\pgfpathclose%
\pgfusepath{fill}%
\end{pgfscope}%
\begin{pgfscope}%
\pgfpathrectangle{\pgfqpoint{1.150000in}{0.150000in}}{\pgfqpoint{5.700000in}{5.700000in}}%
\pgfusepath{clip}%
\pgfsetbuttcap%
\pgfsetroundjoin%
\definecolor{currentfill}{rgb}{0.282884,0.135920,0.453427}%
\pgfsetfillcolor{currentfill}%
\pgfsetfillopacity{0.700000}%
\pgfsetlinewidth{0.000000pt}%
\definecolor{currentstroke}{rgb}{0.000000,0.000000,0.000000}%
\pgfsetstrokecolor{currentstroke}%
\pgfsetdash{}{0pt}%
\pgfpathmoveto{\pgfqpoint{5.247070in}{2.297835in}}%
\pgfpathlineto{\pgfqpoint{5.261066in}{2.296719in}}%
\pgfpathlineto{\pgfqpoint{5.275071in}{2.295628in}}%
\pgfpathlineto{\pgfqpoint{5.289085in}{2.294562in}}%
\pgfpathlineto{\pgfqpoint{5.303106in}{2.293520in}}%
\pgfpathlineto{\pgfqpoint{5.295661in}{2.285733in}}%
\pgfpathlineto{\pgfqpoint{5.288208in}{2.277862in}}%
\pgfpathlineto{\pgfqpoint{5.280748in}{2.269908in}}%
\pgfpathlineto{\pgfqpoint{5.273281in}{2.261871in}}%
\pgfpathlineto{\pgfqpoint{5.259247in}{2.262927in}}%
\pgfpathlineto{\pgfqpoint{5.245221in}{2.264007in}}%
\pgfpathlineto{\pgfqpoint{5.231204in}{2.265113in}}%
\pgfpathlineto{\pgfqpoint{5.217195in}{2.266243in}}%
\pgfpathlineto{\pgfqpoint{5.224674in}{2.274261in}}%
\pgfpathlineto{\pgfqpoint{5.232147in}{2.282199in}}%
\pgfpathlineto{\pgfqpoint{5.239612in}{2.290057in}}%
\pgfpathlineto{\pgfqpoint{5.247070in}{2.297835in}}%
\pgfpathclose%
\pgfusepath{fill}%
\end{pgfscope}%
\begin{pgfscope}%
\pgfpathrectangle{\pgfqpoint{1.150000in}{0.150000in}}{\pgfqpoint{5.700000in}{5.700000in}}%
\pgfusepath{clip}%
\pgfsetbuttcap%
\pgfsetroundjoin%
\definecolor{currentfill}{rgb}{0.282656,0.100196,0.422160}%
\pgfsetfillcolor{currentfill}%
\pgfsetfillopacity{0.700000}%
\pgfsetlinewidth{0.000000pt}%
\definecolor{currentstroke}{rgb}{0.000000,0.000000,0.000000}%
\pgfsetstrokecolor{currentstroke}%
\pgfsetdash{}{0pt}%
\pgfpathmoveto{\pgfqpoint{3.039033in}{2.222107in}}%
\pgfpathlineto{\pgfqpoint{3.052525in}{2.215270in}}%
\pgfpathlineto{\pgfqpoint{3.066020in}{2.208470in}}%
\pgfpathlineto{\pgfqpoint{3.079519in}{2.201706in}}%
\pgfpathlineto{\pgfqpoint{3.093023in}{2.194978in}}%
\pgfpathlineto{\pgfqpoint{3.084634in}{2.193725in}}%
\pgfpathlineto{\pgfqpoint{3.076232in}{2.192712in}}%
\pgfpathlineto{\pgfqpoint{3.067817in}{2.191946in}}%
\pgfpathlineto{\pgfqpoint{3.059388in}{2.191433in}}%
\pgfpathlineto{\pgfqpoint{3.045859in}{2.198418in}}%
\pgfpathlineto{\pgfqpoint{3.032333in}{2.205438in}}%
\pgfpathlineto{\pgfqpoint{3.018811in}{2.212495in}}%
\pgfpathlineto{\pgfqpoint{3.005293in}{2.219589in}}%
\pgfpathlineto{\pgfqpoint{3.013749in}{2.219840in}}%
\pgfpathlineto{\pgfqpoint{3.022190in}{2.220347in}}%
\pgfpathlineto{\pgfqpoint{3.030619in}{2.221105in}}%
\pgfpathlineto{\pgfqpoint{3.039033in}{2.222107in}}%
\pgfpathclose%
\pgfusepath{fill}%
\end{pgfscope}%
\begin{pgfscope}%
\pgfpathrectangle{\pgfqpoint{1.150000in}{0.150000in}}{\pgfqpoint{5.700000in}{5.700000in}}%
\pgfusepath{clip}%
\pgfsetbuttcap%
\pgfsetroundjoin%
\definecolor{currentfill}{rgb}{0.271828,0.209303,0.504434}%
\pgfsetfillcolor{currentfill}%
\pgfsetfillopacity{0.700000}%
\pgfsetlinewidth{0.000000pt}%
\definecolor{currentstroke}{rgb}{0.000000,0.000000,0.000000}%
\pgfsetstrokecolor{currentstroke}%
\pgfsetdash{}{0pt}%
\pgfpathmoveto{\pgfqpoint{2.593308in}{2.443749in}}%
\pgfpathlineto{\pgfqpoint{2.606772in}{2.435219in}}%
\pgfpathlineto{\pgfqpoint{2.620238in}{2.426736in}}%
\pgfpathlineto{\pgfqpoint{2.633706in}{2.418300in}}%
\pgfpathlineto{\pgfqpoint{2.647176in}{2.409910in}}%
\pgfpathlineto{\pgfqpoint{2.638459in}{2.412737in}}%
\pgfpathlineto{\pgfqpoint{2.629722in}{2.415886in}}%
\pgfpathlineto{\pgfqpoint{2.620966in}{2.419365in}}%
\pgfpathlineto{\pgfqpoint{2.612191in}{2.423182in}}%
\pgfpathlineto{\pgfqpoint{2.598686in}{2.431860in}}%
\pgfpathlineto{\pgfqpoint{2.585184in}{2.440584in}}%
\pgfpathlineto{\pgfqpoint{2.571684in}{2.449355in}}%
\pgfpathlineto{\pgfqpoint{2.558186in}{2.458173in}}%
\pgfpathlineto{\pgfqpoint{2.566996in}{2.454063in}}%
\pgfpathlineto{\pgfqpoint{2.575786in}{2.450294in}}%
\pgfpathlineto{\pgfqpoint{2.584557in}{2.446858in}}%
\pgfpathlineto{\pgfqpoint{2.593308in}{2.443749in}}%
\pgfpathclose%
\pgfusepath{fill}%
\end{pgfscope}%
\begin{pgfscope}%
\pgfpathrectangle{\pgfqpoint{1.150000in}{0.150000in}}{\pgfqpoint{5.700000in}{5.700000in}}%
\pgfusepath{clip}%
\pgfsetbuttcap%
\pgfsetroundjoin%
\definecolor{currentfill}{rgb}{0.271305,0.019942,0.347269}%
\pgfsetfillcolor{currentfill}%
\pgfsetfillopacity{0.700000}%
\pgfsetlinewidth{0.000000pt}%
\definecolor{currentstroke}{rgb}{0.000000,0.000000,0.000000}%
\pgfsetstrokecolor{currentstroke}%
\pgfsetdash{}{0pt}%
\pgfpathmoveto{\pgfqpoint{4.307092in}{2.084531in}}%
\pgfpathlineto{\pgfqpoint{4.320821in}{2.081576in}}%
\pgfpathlineto{\pgfqpoint{4.334558in}{2.078648in}}%
\pgfpathlineto{\pgfqpoint{4.348301in}{2.075746in}}%
\pgfpathlineto{\pgfqpoint{4.362051in}{2.072870in}}%
\pgfpathlineto{\pgfqpoint{4.354250in}{2.064199in}}%
\pgfpathlineto{\pgfqpoint{4.346444in}{2.055537in}}%
\pgfpathlineto{\pgfqpoint{4.338632in}{2.046885in}}%
\pgfpathlineto{\pgfqpoint{4.330814in}{2.038248in}}%
\pgfpathlineto{\pgfqpoint{4.317052in}{2.041259in}}%
\pgfpathlineto{\pgfqpoint{4.303298in}{2.044297in}}%
\pgfpathlineto{\pgfqpoint{4.289550in}{2.047360in}}%
\pgfpathlineto{\pgfqpoint{4.275809in}{2.050451in}}%
\pgfpathlineto{\pgfqpoint{4.283638in}{2.058947in}}%
\pgfpathlineto{\pgfqpoint{4.291461in}{2.067462in}}%
\pgfpathlineto{\pgfqpoint{4.299279in}{2.075991in}}%
\pgfpathlineto{\pgfqpoint{4.307092in}{2.084531in}}%
\pgfpathclose%
\pgfusepath{fill}%
\end{pgfscope}%
\begin{pgfscope}%
\pgfpathrectangle{\pgfqpoint{1.150000in}{0.150000in}}{\pgfqpoint{5.700000in}{5.700000in}}%
\pgfusepath{clip}%
\pgfsetbuttcap%
\pgfsetroundjoin%
\definecolor{currentfill}{rgb}{0.282290,0.145912,0.461510}%
\pgfsetfillcolor{currentfill}%
\pgfsetfillopacity{0.700000}%
\pgfsetlinewidth{0.000000pt}%
\definecolor{currentstroke}{rgb}{0.000000,0.000000,0.000000}%
\pgfsetstrokecolor{currentstroke}%
\pgfsetdash{}{0pt}%
\pgfpathmoveto{\pgfqpoint{2.843351in}{2.307690in}}%
\pgfpathlineto{\pgfqpoint{2.856828in}{2.300132in}}%
\pgfpathlineto{\pgfqpoint{2.870307in}{2.292614in}}%
\pgfpathlineto{\pgfqpoint{2.883791in}{2.285136in}}%
\pgfpathlineto{\pgfqpoint{2.897277in}{2.277699in}}%
\pgfpathlineto{\pgfqpoint{2.888751in}{2.278244in}}%
\pgfpathlineto{\pgfqpoint{2.880209in}{2.279068in}}%
\pgfpathlineto{\pgfqpoint{2.871651in}{2.280177in}}%
\pgfpathlineto{\pgfqpoint{2.863078in}{2.281578in}}%
\pgfpathlineto{\pgfqpoint{2.849561in}{2.289288in}}%
\pgfpathlineto{\pgfqpoint{2.836048in}{2.297037in}}%
\pgfpathlineto{\pgfqpoint{2.822539in}{2.304827in}}%
\pgfpathlineto{\pgfqpoint{2.809032in}{2.312657in}}%
\pgfpathlineto{\pgfqpoint{2.817636in}{2.310978in}}%
\pgfpathlineto{\pgfqpoint{2.826224in}{2.309596in}}%
\pgfpathlineto{\pgfqpoint{2.834795in}{2.308502in}}%
\pgfpathlineto{\pgfqpoint{2.843351in}{2.307690in}}%
\pgfpathclose%
\pgfusepath{fill}%
\end{pgfscope}%
\begin{pgfscope}%
\pgfpathrectangle{\pgfqpoint{1.150000in}{0.150000in}}{\pgfqpoint{5.700000in}{5.700000in}}%
\pgfusepath{clip}%
\pgfsetbuttcap%
\pgfsetroundjoin%
\definecolor{currentfill}{rgb}{0.278791,0.062145,0.386592}%
\pgfsetfillcolor{currentfill}%
\pgfsetfillopacity{0.700000}%
\pgfsetlinewidth{0.000000pt}%
\definecolor{currentstroke}{rgb}{0.000000,0.000000,0.000000}%
\pgfsetstrokecolor{currentstroke}%
\pgfsetdash{}{0pt}%
\pgfpathmoveto{\pgfqpoint{3.234432in}{2.151657in}}%
\pgfpathlineto{\pgfqpoint{3.247948in}{2.145486in}}%
\pgfpathlineto{\pgfqpoint{3.261469in}{2.139350in}}%
\pgfpathlineto{\pgfqpoint{3.274994in}{2.133247in}}%
\pgfpathlineto{\pgfqpoint{3.288524in}{2.127177in}}%
\pgfpathlineto{\pgfqpoint{3.280254in}{2.124316in}}%
\pgfpathlineto{\pgfqpoint{3.271974in}{2.121659in}}%
\pgfpathlineto{\pgfqpoint{3.263682in}{2.119212in}}%
\pgfpathlineto{\pgfqpoint{3.255379in}{2.116982in}}%
\pgfpathlineto{\pgfqpoint{3.241826in}{2.123294in}}%
\pgfpathlineto{\pgfqpoint{3.228277in}{2.129639in}}%
\pgfpathlineto{\pgfqpoint{3.214733in}{2.136018in}}%
\pgfpathlineto{\pgfqpoint{3.201193in}{2.142431in}}%
\pgfpathlineto{\pgfqpoint{3.209520in}{2.144413in}}%
\pgfpathlineto{\pgfqpoint{3.217835in}{2.146616in}}%
\pgfpathlineto{\pgfqpoint{3.226139in}{2.149032in}}%
\pgfpathlineto{\pgfqpoint{3.234432in}{2.151657in}}%
\pgfpathclose%
\pgfusepath{fill}%
\end{pgfscope}%
\begin{pgfscope}%
\pgfpathrectangle{\pgfqpoint{1.150000in}{0.150000in}}{\pgfqpoint{5.700000in}{5.700000in}}%
\pgfusepath{clip}%
\pgfsetbuttcap%
\pgfsetroundjoin%
\definecolor{currentfill}{rgb}{0.267004,0.004874,0.329415}%
\pgfsetfillcolor{currentfill}%
\pgfsetfillopacity{0.700000}%
\pgfsetlinewidth{0.000000pt}%
\definecolor{currentstroke}{rgb}{0.000000,0.000000,0.000000}%
\pgfsetstrokecolor{currentstroke}%
\pgfsetdash{}{0pt}%
\pgfpathmoveto{\pgfqpoint{4.079916in}{2.057653in}}%
\pgfpathlineto{\pgfqpoint{4.093590in}{2.054093in}}%
\pgfpathlineto{\pgfqpoint{4.107271in}{2.050560in}}%
\pgfpathlineto{\pgfqpoint{4.120958in}{2.047054in}}%
\pgfpathlineto{\pgfqpoint{4.134652in}{2.043575in}}%
\pgfpathlineto{\pgfqpoint{4.126770in}{2.035532in}}%
\pgfpathlineto{\pgfqpoint{4.118883in}{2.027534in}}%
\pgfpathlineto{\pgfqpoint{4.110989in}{2.019586in}}%
\pgfpathlineto{\pgfqpoint{4.103090in}{2.011690in}}%
\pgfpathlineto{\pgfqpoint{4.089384in}{2.015331in}}%
\pgfpathlineto{\pgfqpoint{4.075684in}{2.018999in}}%
\pgfpathlineto{\pgfqpoint{4.061990in}{2.022694in}}%
\pgfpathlineto{\pgfqpoint{4.048302in}{2.026416in}}%
\pgfpathlineto{\pgfqpoint{4.056215in}{2.034144in}}%
\pgfpathlineto{\pgfqpoint{4.064121in}{2.041929in}}%
\pgfpathlineto{\pgfqpoint{4.072021in}{2.049767in}}%
\pgfpathlineto{\pgfqpoint{4.079916in}{2.057653in}}%
\pgfpathclose%
\pgfusepath{fill}%
\end{pgfscope}%
\begin{pgfscope}%
\pgfpathrectangle{\pgfqpoint{1.150000in}{0.150000in}}{\pgfqpoint{5.700000in}{5.700000in}}%
\pgfusepath{clip}%
\pgfsetbuttcap%
\pgfsetroundjoin%
\definecolor{currentfill}{rgb}{0.275191,0.194905,0.496005}%
\pgfsetfillcolor{currentfill}%
\pgfsetfillopacity{0.700000}%
\pgfsetlinewidth{0.000000pt}%
\definecolor{currentstroke}{rgb}{0.000000,0.000000,0.000000}%
\pgfsetstrokecolor{currentstroke}%
\pgfsetdash{}{0pt}%
\pgfpathmoveto{\pgfqpoint{5.787843in}{2.412175in}}%
\pgfpathlineto{\pgfqpoint{5.802011in}{2.411614in}}%
\pgfpathlineto{\pgfqpoint{5.816187in}{2.411077in}}%
\pgfpathlineto{\pgfqpoint{5.830372in}{2.410565in}}%
\pgfpathlineto{\pgfqpoint{5.844566in}{2.410077in}}%
\pgfpathlineto{\pgfqpoint{5.837382in}{2.404178in}}%
\pgfpathlineto{\pgfqpoint{5.830190in}{2.398197in}}%
\pgfpathlineto{\pgfqpoint{5.822990in}{2.392130in}}%
\pgfpathlineto{\pgfqpoint{5.815781in}{2.385975in}}%
\pgfpathlineto{\pgfqpoint{5.801570in}{2.386408in}}%
\pgfpathlineto{\pgfqpoint{5.787368in}{2.386866in}}%
\pgfpathlineto{\pgfqpoint{5.773175in}{2.387348in}}%
\pgfpathlineto{\pgfqpoint{5.758992in}{2.387854in}}%
\pgfpathlineto{\pgfqpoint{5.766217in}{2.394058in}}%
\pgfpathlineto{\pgfqpoint{5.773434in}{2.400178in}}%
\pgfpathlineto{\pgfqpoint{5.780643in}{2.406217in}}%
\pgfpathlineto{\pgfqpoint{5.787843in}{2.412175in}}%
\pgfpathclose%
\pgfusepath{fill}%
\end{pgfscope}%
\begin{pgfscope}%
\pgfpathrectangle{\pgfqpoint{1.150000in}{0.150000in}}{\pgfqpoint{5.700000in}{5.700000in}}%
\pgfusepath{clip}%
\pgfsetbuttcap%
\pgfsetroundjoin%
\definecolor{currentfill}{rgb}{0.281446,0.084320,0.407414}%
\pgfsetfillcolor{currentfill}%
\pgfsetfillopacity{0.700000}%
\pgfsetlinewidth{0.000000pt}%
\definecolor{currentstroke}{rgb}{0.000000,0.000000,0.000000}%
\pgfsetstrokecolor{currentstroke}%
\pgfsetdash{}{0pt}%
\pgfpathmoveto{\pgfqpoint{4.847764in}{2.194231in}}%
\pgfpathlineto{\pgfqpoint{4.861645in}{2.192475in}}%
\pgfpathlineto{\pgfqpoint{4.875533in}{2.190745in}}%
\pgfpathlineto{\pgfqpoint{4.889430in}{2.189039in}}%
\pgfpathlineto{\pgfqpoint{4.903334in}{2.187359in}}%
\pgfpathlineto{\pgfqpoint{4.895723in}{2.178590in}}%
\pgfpathlineto{\pgfqpoint{4.888106in}{2.169763in}}%
\pgfpathlineto{\pgfqpoint{4.880482in}{2.160878in}}%
\pgfpathlineto{\pgfqpoint{4.872852in}{2.151939in}}%
\pgfpathlineto{\pgfqpoint{4.858937in}{2.153687in}}%
\pgfpathlineto{\pgfqpoint{4.845030in}{2.155461in}}%
\pgfpathlineto{\pgfqpoint{4.831131in}{2.157261in}}%
\pgfpathlineto{\pgfqpoint{4.817239in}{2.159085in}}%
\pgfpathlineto{\pgfqpoint{4.824879in}{2.167951in}}%
\pgfpathlineto{\pgfqpoint{4.832514in}{2.176765in}}%
\pgfpathlineto{\pgfqpoint{4.840142in}{2.185525in}}%
\pgfpathlineto{\pgfqpoint{4.847764in}{2.194231in}}%
\pgfpathclose%
\pgfusepath{fill}%
\end{pgfscope}%
\begin{pgfscope}%
\pgfpathrectangle{\pgfqpoint{1.150000in}{0.150000in}}{\pgfqpoint{5.700000in}{5.700000in}}%
\pgfusepath{clip}%
\pgfsetbuttcap%
\pgfsetroundjoin%
\definecolor{currentfill}{rgb}{0.276022,0.044167,0.370164}%
\pgfsetfillcolor{currentfill}%
\pgfsetfillopacity{0.700000}%
\pgfsetlinewidth{0.000000pt}%
\definecolor{currentstroke}{rgb}{0.000000,0.000000,0.000000}%
\pgfsetstrokecolor{currentstroke}%
\pgfsetdash{}{0pt}%
\pgfpathmoveto{\pgfqpoint{4.534339in}{2.121969in}}%
\pgfpathlineto{\pgfqpoint{4.548131in}{2.119559in}}%
\pgfpathlineto{\pgfqpoint{4.561930in}{2.117174in}}%
\pgfpathlineto{\pgfqpoint{4.575737in}{2.114815in}}%
\pgfpathlineto{\pgfqpoint{4.589551in}{2.112482in}}%
\pgfpathlineto{\pgfqpoint{4.581827in}{2.103535in}}%
\pgfpathlineto{\pgfqpoint{4.574098in}{2.094565in}}%
\pgfpathlineto{\pgfqpoint{4.566362in}{2.085574in}}%
\pgfpathlineto{\pgfqpoint{4.558622in}{2.076564in}}%
\pgfpathlineto{\pgfqpoint{4.544797in}{2.079006in}}%
\pgfpathlineto{\pgfqpoint{4.530979in}{2.081474in}}%
\pgfpathlineto{\pgfqpoint{4.517169in}{2.083967in}}%
\pgfpathlineto{\pgfqpoint{4.503366in}{2.086487in}}%
\pgfpathlineto{\pgfqpoint{4.511118in}{2.095383in}}%
\pgfpathlineto{\pgfqpoint{4.518864in}{2.104263in}}%
\pgfpathlineto{\pgfqpoint{4.526604in}{2.113126in}}%
\pgfpathlineto{\pgfqpoint{4.534339in}{2.121969in}}%
\pgfpathclose%
\pgfusepath{fill}%
\end{pgfscope}%
\begin{pgfscope}%
\pgfpathrectangle{\pgfqpoint{1.150000in}{0.150000in}}{\pgfqpoint{5.700000in}{5.700000in}}%
\pgfusepath{clip}%
\pgfsetbuttcap%
\pgfsetroundjoin%
\definecolor{currentfill}{rgb}{0.283187,0.125848,0.444960}%
\pgfsetfillcolor{currentfill}%
\pgfsetfillopacity{0.700000}%
\pgfsetlinewidth{0.000000pt}%
\definecolor{currentstroke}{rgb}{0.000000,0.000000,0.000000}%
\pgfsetstrokecolor{currentstroke}%
\pgfsetdash{}{0pt}%
\pgfpathmoveto{\pgfqpoint{5.161243in}{2.271013in}}%
\pgfpathlineto{\pgfqpoint{5.175219in}{2.269783in}}%
\pgfpathlineto{\pgfqpoint{5.189202in}{2.268578in}}%
\pgfpathlineto{\pgfqpoint{5.203195in}{2.267398in}}%
\pgfpathlineto{\pgfqpoint{5.217195in}{2.266243in}}%
\pgfpathlineto{\pgfqpoint{5.209709in}{2.258145in}}%
\pgfpathlineto{\pgfqpoint{5.202215in}{2.249968in}}%
\pgfpathlineto{\pgfqpoint{5.194714in}{2.241711in}}%
\pgfpathlineto{\pgfqpoint{5.187207in}{2.233374in}}%
\pgfpathlineto{\pgfqpoint{5.173195in}{2.234557in}}%
\pgfpathlineto{\pgfqpoint{5.159191in}{2.235765in}}%
\pgfpathlineto{\pgfqpoint{5.145195in}{2.236998in}}%
\pgfpathlineto{\pgfqpoint{5.131208in}{2.238256in}}%
\pgfpathlineto{\pgfqpoint{5.138727in}{2.246559in}}%
\pgfpathlineto{\pgfqpoint{5.146239in}{2.254787in}}%
\pgfpathlineto{\pgfqpoint{5.153745in}{2.262938in}}%
\pgfpathlineto{\pgfqpoint{5.161243in}{2.271013in}}%
\pgfpathclose%
\pgfusepath{fill}%
\end{pgfscope}%
\begin{pgfscope}%
\pgfpathrectangle{\pgfqpoint{1.150000in}{0.150000in}}{\pgfqpoint{5.700000in}{5.700000in}}%
\pgfusepath{clip}%
\pgfsetbuttcap%
\pgfsetroundjoin%
\definecolor{currentfill}{rgb}{0.280255,0.165693,0.476498}%
\pgfsetfillcolor{currentfill}%
\pgfsetfillopacity{0.700000}%
\pgfsetlinewidth{0.000000pt}%
\definecolor{currentstroke}{rgb}{0.000000,0.000000,0.000000}%
\pgfsetstrokecolor{currentstroke}%
\pgfsetdash{}{0pt}%
\pgfpathmoveto{\pgfqpoint{5.474672in}{2.345351in}}%
\pgfpathlineto{\pgfqpoint{5.488744in}{2.344519in}}%
\pgfpathlineto{\pgfqpoint{5.502825in}{2.343712in}}%
\pgfpathlineto{\pgfqpoint{5.516915in}{2.342929in}}%
\pgfpathlineto{\pgfqpoint{5.531013in}{2.342172in}}%
\pgfpathlineto{\pgfqpoint{5.523669in}{2.335084in}}%
\pgfpathlineto{\pgfqpoint{5.516317in}{2.327908in}}%
\pgfpathlineto{\pgfqpoint{5.508956in}{2.320642in}}%
\pgfpathlineto{\pgfqpoint{5.501588in}{2.313287in}}%
\pgfpathlineto{\pgfqpoint{5.487476in}{2.314032in}}%
\pgfpathlineto{\pgfqpoint{5.473373in}{2.314801in}}%
\pgfpathlineto{\pgfqpoint{5.459278in}{2.315594in}}%
\pgfpathlineto{\pgfqpoint{5.445192in}{2.316413in}}%
\pgfpathlineto{\pgfqpoint{5.452574in}{2.323776in}}%
\pgfpathlineto{\pgfqpoint{5.459948in}{2.331053in}}%
\pgfpathlineto{\pgfqpoint{5.467314in}{2.338244in}}%
\pgfpathlineto{\pgfqpoint{5.474672in}{2.345351in}}%
\pgfpathclose%
\pgfusepath{fill}%
\end{pgfscope}%
\begin{pgfscope}%
\pgfpathrectangle{\pgfqpoint{1.150000in}{0.150000in}}{\pgfqpoint{5.700000in}{5.700000in}}%
\pgfusepath{clip}%
\pgfsetbuttcap%
\pgfsetroundjoin%
\definecolor{currentfill}{rgb}{0.268510,0.009605,0.335427}%
\pgfsetfillcolor{currentfill}%
\pgfsetfillopacity{0.700000}%
\pgfsetlinewidth{0.000000pt}%
\definecolor{currentstroke}{rgb}{0.000000,0.000000,0.000000}%
\pgfsetstrokecolor{currentstroke}%
\pgfsetdash{}{0pt}%
\pgfpathmoveto{\pgfqpoint{3.711730in}{2.054672in}}%
\pgfpathlineto{\pgfqpoint{3.725328in}{2.050024in}}%
\pgfpathlineto{\pgfqpoint{3.738931in}{2.045406in}}%
\pgfpathlineto{\pgfqpoint{3.752541in}{2.040816in}}%
\pgfpathlineto{\pgfqpoint{3.766155in}{2.036255in}}%
\pgfpathlineto{\pgfqpoint{3.758126in}{2.030014in}}%
\pgfpathlineto{\pgfqpoint{3.750089in}{2.023887in}}%
\pgfpathlineto{\pgfqpoint{3.742045in}{2.017879in}}%
\pgfpathlineto{\pgfqpoint{3.733993in}{2.011996in}}%
\pgfpathlineto{\pgfqpoint{3.720362in}{2.016758in}}%
\pgfpathlineto{\pgfqpoint{3.706736in}{2.021550in}}%
\pgfpathlineto{\pgfqpoint{3.693115in}{2.026370in}}%
\pgfpathlineto{\pgfqpoint{3.679501in}{2.031219in}}%
\pgfpathlineto{\pgfqpoint{3.687569in}{2.036896in}}%
\pgfpathlineto{\pgfqpoint{3.695630in}{2.042700in}}%
\pgfpathlineto{\pgfqpoint{3.703684in}{2.048627in}}%
\pgfpathlineto{\pgfqpoint{3.711730in}{2.054672in}}%
\pgfpathclose%
\pgfusepath{fill}%
\end{pgfscope}%
\begin{pgfscope}%
\pgfpathrectangle{\pgfqpoint{1.150000in}{0.150000in}}{\pgfqpoint{5.700000in}{5.700000in}}%
\pgfusepath{clip}%
\pgfsetbuttcap%
\pgfsetroundjoin%
\definecolor{currentfill}{rgb}{0.269944,0.014625,0.341379}%
\pgfsetfillcolor{currentfill}%
\pgfsetfillopacity{0.700000}%
\pgfsetlinewidth{0.000000pt}%
\definecolor{currentstroke}{rgb}{0.000000,0.000000,0.000000}%
\pgfsetstrokecolor{currentstroke}%
\pgfsetdash{}{0pt}%
\pgfpathmoveto{\pgfqpoint{3.570775in}{2.071079in}}%
\pgfpathlineto{\pgfqpoint{3.584347in}{2.065991in}}%
\pgfpathlineto{\pgfqpoint{3.597925in}{2.060934in}}%
\pgfpathlineto{\pgfqpoint{3.611507in}{2.055907in}}%
\pgfpathlineto{\pgfqpoint{3.625095in}{2.050911in}}%
\pgfpathlineto{\pgfqpoint{3.617001in}{2.045576in}}%
\pgfpathlineto{\pgfqpoint{3.608899in}{2.040384in}}%
\pgfpathlineto{\pgfqpoint{3.600788in}{2.035338in}}%
\pgfpathlineto{\pgfqpoint{3.592669in}{2.030445in}}%
\pgfpathlineto{\pgfqpoint{3.579063in}{2.035656in}}%
\pgfpathlineto{\pgfqpoint{3.565462in}{2.040898in}}%
\pgfpathlineto{\pgfqpoint{3.551866in}{2.046170in}}%
\pgfpathlineto{\pgfqpoint{3.538275in}{2.051472in}}%
\pgfpathlineto{\pgfqpoint{3.546413in}{2.056146in}}%
\pgfpathlineto{\pgfqpoint{3.554542in}{2.060975in}}%
\pgfpathlineto{\pgfqpoint{3.562663in}{2.065954in}}%
\pgfpathlineto{\pgfqpoint{3.570775in}{2.071079in}}%
\pgfpathclose%
\pgfusepath{fill}%
\end{pgfscope}%
\begin{pgfscope}%
\pgfpathrectangle{\pgfqpoint{1.150000in}{0.150000in}}{\pgfqpoint{5.700000in}{5.700000in}}%
\pgfusepath{clip}%
\pgfsetbuttcap%
\pgfsetroundjoin%
\definecolor{currentfill}{rgb}{0.267004,0.004874,0.329415}%
\pgfsetfillcolor{currentfill}%
\pgfsetfillopacity{0.700000}%
\pgfsetlinewidth{0.000000pt}%
\definecolor{currentstroke}{rgb}{0.000000,0.000000,0.000000}%
\pgfsetstrokecolor{currentstroke}%
\pgfsetdash{}{0pt}%
\pgfpathmoveto{\pgfqpoint{3.852658in}{2.045063in}}%
\pgfpathlineto{\pgfqpoint{3.866286in}{2.040833in}}%
\pgfpathlineto{\pgfqpoint{3.879920in}{2.036632in}}%
\pgfpathlineto{\pgfqpoint{3.893560in}{2.032458in}}%
\pgfpathlineto{\pgfqpoint{3.907206in}{2.028313in}}%
\pgfpathlineto{\pgfqpoint{3.899235in}{2.021296in}}%
\pgfpathlineto{\pgfqpoint{3.891257in}{2.014368in}}%
\pgfpathlineto{\pgfqpoint{3.883273in}{2.007533in}}%
\pgfpathlineto{\pgfqpoint{3.875281in}{2.000796in}}%
\pgfpathlineto{\pgfqpoint{3.861620in}{2.005129in}}%
\pgfpathlineto{\pgfqpoint{3.847965in}{2.009491in}}%
\pgfpathlineto{\pgfqpoint{3.834316in}{2.013881in}}%
\pgfpathlineto{\pgfqpoint{3.820672in}{2.018299in}}%
\pgfpathlineto{\pgfqpoint{3.828679in}{2.024843in}}%
\pgfpathlineto{\pgfqpoint{3.836679in}{2.031488in}}%
\pgfpathlineto{\pgfqpoint{3.844672in}{2.038229in}}%
\pgfpathlineto{\pgfqpoint{3.852658in}{2.045063in}}%
\pgfpathclose%
\pgfusepath{fill}%
\end{pgfscope}%
\begin{pgfscope}%
\pgfpathrectangle{\pgfqpoint{1.150000in}{0.150000in}}{\pgfqpoint{5.700000in}{5.700000in}}%
\pgfusepath{clip}%
\pgfsetbuttcap%
\pgfsetroundjoin%
\definecolor{currentfill}{rgb}{0.273809,0.031497,0.358853}%
\pgfsetfillcolor{currentfill}%
\pgfsetfillopacity{0.700000}%
\pgfsetlinewidth{0.000000pt}%
\definecolor{currentstroke}{rgb}{0.000000,0.000000,0.000000}%
\pgfsetstrokecolor{currentstroke}%
\pgfsetdash{}{0pt}%
\pgfpathmoveto{\pgfqpoint{3.429731in}{2.094996in}}%
\pgfpathlineto{\pgfqpoint{3.443281in}{2.089446in}}%
\pgfpathlineto{\pgfqpoint{3.456837in}{2.083928in}}%
\pgfpathlineto{\pgfqpoint{3.470397in}{2.078441in}}%
\pgfpathlineto{\pgfqpoint{3.483963in}{2.072986in}}%
\pgfpathlineto{\pgfqpoint{3.475797in}{2.068697in}}%
\pgfpathlineto{\pgfqpoint{3.467621in}{2.064578in}}%
\pgfpathlineto{\pgfqpoint{3.459436in}{2.060636in}}%
\pgfpathlineto{\pgfqpoint{3.451242in}{2.056875in}}%
\pgfpathlineto{\pgfqpoint{3.437656in}{2.062559in}}%
\pgfpathlineto{\pgfqpoint{3.424075in}{2.068274in}}%
\pgfpathlineto{\pgfqpoint{3.410499in}{2.074021in}}%
\pgfpathlineto{\pgfqpoint{3.396927in}{2.079799in}}%
\pgfpathlineto{\pgfqpoint{3.405143in}{2.083326in}}%
\pgfpathlineto{\pgfqpoint{3.413348in}{2.087038in}}%
\pgfpathlineto{\pgfqpoint{3.421544in}{2.090930in}}%
\pgfpathlineto{\pgfqpoint{3.429731in}{2.094996in}}%
\pgfpathclose%
\pgfusepath{fill}%
\end{pgfscope}%
\begin{pgfscope}%
\pgfpathrectangle{\pgfqpoint{1.150000in}{0.150000in}}{\pgfqpoint{5.700000in}{5.700000in}}%
\pgfusepath{clip}%
\pgfsetbuttcap%
\pgfsetroundjoin%
\definecolor{currentfill}{rgb}{0.280267,0.073417,0.397163}%
\pgfsetfillcolor{currentfill}%
\pgfsetfillopacity{0.700000}%
\pgfsetlinewidth{0.000000pt}%
\definecolor{currentstroke}{rgb}{0.000000,0.000000,0.000000}%
\pgfsetstrokecolor{currentstroke}%
\pgfsetdash{}{0pt}%
\pgfpathmoveto{\pgfqpoint{4.761751in}{2.166636in}}%
\pgfpathlineto{\pgfqpoint{4.775611in}{2.164710in}}%
\pgfpathlineto{\pgfqpoint{4.789479in}{2.162810in}}%
\pgfpathlineto{\pgfqpoint{4.803355in}{2.160935in}}%
\pgfpathlineto{\pgfqpoint{4.817239in}{2.159085in}}%
\pgfpathlineto{\pgfqpoint{4.809593in}{2.150168in}}%
\pgfpathlineto{\pgfqpoint{4.801941in}{2.141203in}}%
\pgfpathlineto{\pgfqpoint{4.794283in}{2.132189in}}%
\pgfpathlineto{\pgfqpoint{4.786618in}{2.123130in}}%
\pgfpathlineto{\pgfqpoint{4.772724in}{2.125062in}}%
\pgfpathlineto{\pgfqpoint{4.758838in}{2.127019in}}%
\pgfpathlineto{\pgfqpoint{4.744959in}{2.129002in}}%
\pgfpathlineto{\pgfqpoint{4.731088in}{2.131010in}}%
\pgfpathlineto{\pgfqpoint{4.738762in}{2.139982in}}%
\pgfpathlineto{\pgfqpoint{4.746431in}{2.148911in}}%
\pgfpathlineto{\pgfqpoint{4.754094in}{2.157797in}}%
\pgfpathlineto{\pgfqpoint{4.761751in}{2.166636in}}%
\pgfpathclose%
\pgfusepath{fill}%
\end{pgfscope}%
\begin{pgfscope}%
\pgfpathrectangle{\pgfqpoint{1.150000in}{0.150000in}}{\pgfqpoint{5.700000in}{5.700000in}}%
\pgfusepath{clip}%
\pgfsetbuttcap%
\pgfsetroundjoin%
\definecolor{currentfill}{rgb}{0.269944,0.014625,0.341379}%
\pgfsetfillcolor{currentfill}%
\pgfsetfillopacity{0.700000}%
\pgfsetlinewidth{0.000000pt}%
\definecolor{currentstroke}{rgb}{0.000000,0.000000,0.000000}%
\pgfsetstrokecolor{currentstroke}%
\pgfsetdash{}{0pt}%
\pgfpathmoveto{\pgfqpoint{4.220911in}{2.063076in}}%
\pgfpathlineto{\pgfqpoint{4.234626in}{2.059880in}}%
\pgfpathlineto{\pgfqpoint{4.248347in}{2.056710in}}%
\pgfpathlineto{\pgfqpoint{4.262074in}{2.053567in}}%
\pgfpathlineto{\pgfqpoint{4.275809in}{2.050451in}}%
\pgfpathlineto{\pgfqpoint{4.267974in}{2.041975in}}%
\pgfpathlineto{\pgfqpoint{4.260134in}{2.033524in}}%
\pgfpathlineto{\pgfqpoint{4.252288in}{2.025100in}}%
\pgfpathlineto{\pgfqpoint{4.244437in}{2.016708in}}%
\pgfpathlineto{\pgfqpoint{4.230690in}{2.019973in}}%
\pgfpathlineto{\pgfqpoint{4.216951in}{2.023265in}}%
\pgfpathlineto{\pgfqpoint{4.203218in}{2.026583in}}%
\pgfpathlineto{\pgfqpoint{4.189491in}{2.029928in}}%
\pgfpathlineto{\pgfqpoint{4.197355in}{2.038167in}}%
\pgfpathlineto{\pgfqpoint{4.205213in}{2.046440in}}%
\pgfpathlineto{\pgfqpoint{4.213065in}{2.054744in}}%
\pgfpathlineto{\pgfqpoint{4.220911in}{2.063076in}}%
\pgfpathclose%
\pgfusepath{fill}%
\end{pgfscope}%
\begin{pgfscope}%
\pgfpathrectangle{\pgfqpoint{1.150000in}{0.150000in}}{\pgfqpoint{5.700000in}{5.700000in}}%
\pgfusepath{clip}%
\pgfsetbuttcap%
\pgfsetroundjoin%
\definecolor{currentfill}{rgb}{0.274128,0.199721,0.498911}%
\pgfsetfillcolor{currentfill}%
\pgfsetfillopacity{0.700000}%
\pgfsetlinewidth{0.000000pt}%
\definecolor{currentstroke}{rgb}{0.000000,0.000000,0.000000}%
\pgfsetstrokecolor{currentstroke}%
\pgfsetdash{}{0pt}%
\pgfpathmoveto{\pgfqpoint{2.647176in}{2.409910in}}%
\pgfpathlineto{\pgfqpoint{2.660650in}{2.401565in}}%
\pgfpathlineto{\pgfqpoint{2.674125in}{2.393265in}}%
\pgfpathlineto{\pgfqpoint{2.687603in}{2.385010in}}%
\pgfpathlineto{\pgfqpoint{2.701084in}{2.376799in}}%
\pgfpathlineto{\pgfqpoint{2.692399in}{2.379344in}}%
\pgfpathlineto{\pgfqpoint{2.683696in}{2.382208in}}%
\pgfpathlineto{\pgfqpoint{2.674974in}{2.385397in}}%
\pgfpathlineto{\pgfqpoint{2.666234in}{2.388920in}}%
\pgfpathlineto{\pgfqpoint{2.652719in}{2.397418in}}%
\pgfpathlineto{\pgfqpoint{2.639207in}{2.405961in}}%
\pgfpathlineto{\pgfqpoint{2.625698in}{2.414549in}}%
\pgfpathlineto{\pgfqpoint{2.612191in}{2.423182in}}%
\pgfpathlineto{\pgfqpoint{2.620966in}{2.419365in}}%
\pgfpathlineto{\pgfqpoint{2.629722in}{2.415886in}}%
\pgfpathlineto{\pgfqpoint{2.638459in}{2.412737in}}%
\pgfpathlineto{\pgfqpoint{2.647176in}{2.409910in}}%
\pgfpathclose%
\pgfusepath{fill}%
\end{pgfscope}%
\begin{pgfscope}%
\pgfpathrectangle{\pgfqpoint{1.150000in}{0.150000in}}{\pgfqpoint{5.700000in}{5.700000in}}%
\pgfusepath{clip}%
\pgfsetbuttcap%
\pgfsetroundjoin%
\definecolor{currentfill}{rgb}{0.283197,0.115680,0.436115}%
\pgfsetfillcolor{currentfill}%
\pgfsetfillopacity{0.700000}%
\pgfsetlinewidth{0.000000pt}%
\definecolor{currentstroke}{rgb}{0.000000,0.000000,0.000000}%
\pgfsetstrokecolor{currentstroke}%
\pgfsetdash{}{0pt}%
\pgfpathmoveto{\pgfqpoint{5.075341in}{2.243536in}}%
\pgfpathlineto{\pgfqpoint{5.089295in}{2.242179in}}%
\pgfpathlineto{\pgfqpoint{5.103258in}{2.240846in}}%
\pgfpathlineto{\pgfqpoint{5.117229in}{2.239538in}}%
\pgfpathlineto{\pgfqpoint{5.131208in}{2.238256in}}%
\pgfpathlineto{\pgfqpoint{5.123682in}{2.229877in}}%
\pgfpathlineto{\pgfqpoint{5.116149in}{2.221423in}}%
\pgfpathlineto{\pgfqpoint{5.108609in}{2.212894in}}%
\pgfpathlineto{\pgfqpoint{5.101063in}{2.204291in}}%
\pgfpathlineto{\pgfqpoint{5.087073in}{2.205615in}}%
\pgfpathlineto{\pgfqpoint{5.073091in}{2.206964in}}%
\pgfpathlineto{\pgfqpoint{5.059117in}{2.208339in}}%
\pgfpathlineto{\pgfqpoint{5.045151in}{2.209738in}}%
\pgfpathlineto{\pgfqpoint{5.052709in}{2.218294in}}%
\pgfpathlineto{\pgfqpoint{5.060259in}{2.226779in}}%
\pgfpathlineto{\pgfqpoint{5.067803in}{2.235194in}}%
\pgfpathlineto{\pgfqpoint{5.075341in}{2.243536in}}%
\pgfpathclose%
\pgfusepath{fill}%
\end{pgfscope}%
\begin{pgfscope}%
\pgfpathrectangle{\pgfqpoint{1.150000in}{0.150000in}}{\pgfqpoint{5.700000in}{5.700000in}}%
\pgfusepath{clip}%
\pgfsetbuttcap%
\pgfsetroundjoin%
\definecolor{currentfill}{rgb}{0.281924,0.089666,0.412415}%
\pgfsetfillcolor{currentfill}%
\pgfsetfillopacity{0.700000}%
\pgfsetlinewidth{0.000000pt}%
\definecolor{currentstroke}{rgb}{0.000000,0.000000,0.000000}%
\pgfsetstrokecolor{currentstroke}%
\pgfsetdash{}{0pt}%
\pgfpathmoveto{\pgfqpoint{3.093023in}{2.194978in}}%
\pgfpathlineto{\pgfqpoint{3.106530in}{2.188287in}}%
\pgfpathlineto{\pgfqpoint{3.120041in}{2.181631in}}%
\pgfpathlineto{\pgfqpoint{3.133556in}{2.175010in}}%
\pgfpathlineto{\pgfqpoint{3.147075in}{2.168425in}}%
\pgfpathlineto{\pgfqpoint{3.138711in}{2.166920in}}%
\pgfpathlineto{\pgfqpoint{3.130335in}{2.165652in}}%
\pgfpathlineto{\pgfqpoint{3.121946in}{2.164628in}}%
\pgfpathlineto{\pgfqpoint{3.113544in}{2.163853in}}%
\pgfpathlineto{\pgfqpoint{3.099999in}{2.170695in}}%
\pgfpathlineto{\pgfqpoint{3.086458in}{2.177572in}}%
\pgfpathlineto{\pgfqpoint{3.072921in}{2.184485in}}%
\pgfpathlineto{\pgfqpoint{3.059388in}{2.191433in}}%
\pgfpathlineto{\pgfqpoint{3.067817in}{2.191946in}}%
\pgfpathlineto{\pgfqpoint{3.076232in}{2.192712in}}%
\pgfpathlineto{\pgfqpoint{3.084634in}{2.193725in}}%
\pgfpathlineto{\pgfqpoint{3.093023in}{2.194978in}}%
\pgfpathclose%
\pgfusepath{fill}%
\end{pgfscope}%
\begin{pgfscope}%
\pgfpathrectangle{\pgfqpoint{1.150000in}{0.150000in}}{\pgfqpoint{5.700000in}{5.700000in}}%
\pgfusepath{clip}%
\pgfsetbuttcap%
\pgfsetroundjoin%
\definecolor{currentfill}{rgb}{0.276194,0.190074,0.493001}%
\pgfsetfillcolor{currentfill}%
\pgfsetfillopacity{0.700000}%
\pgfsetlinewidth{0.000000pt}%
\definecolor{currentstroke}{rgb}{0.000000,0.000000,0.000000}%
\pgfsetstrokecolor{currentstroke}%
\pgfsetdash{}{0pt}%
\pgfpathmoveto{\pgfqpoint{5.702347in}{2.390125in}}%
\pgfpathlineto{\pgfqpoint{5.716495in}{2.389521in}}%
\pgfpathlineto{\pgfqpoint{5.730651in}{2.388941in}}%
\pgfpathlineto{\pgfqpoint{5.744817in}{2.388385in}}%
\pgfpathlineto{\pgfqpoint{5.758992in}{2.387854in}}%
\pgfpathlineto{\pgfqpoint{5.751758in}{2.381564in}}%
\pgfpathlineto{\pgfqpoint{5.744516in}{2.375187in}}%
\pgfpathlineto{\pgfqpoint{5.737265in}{2.368721in}}%
\pgfpathlineto{\pgfqpoint{5.730006in}{2.362164in}}%
\pgfpathlineto{\pgfqpoint{5.715815in}{2.362654in}}%
\pgfpathlineto{\pgfqpoint{5.701634in}{2.363168in}}%
\pgfpathlineto{\pgfqpoint{5.687462in}{2.363707in}}%
\pgfpathlineto{\pgfqpoint{5.673298in}{2.364271in}}%
\pgfpathlineto{\pgfqpoint{5.680573in}{2.370864in}}%
\pgfpathlineto{\pgfqpoint{5.687839in}{2.377369in}}%
\pgfpathlineto{\pgfqpoint{5.695097in}{2.383789in}}%
\pgfpathlineto{\pgfqpoint{5.702347in}{2.390125in}}%
\pgfpathclose%
\pgfusepath{fill}%
\end{pgfscope}%
\begin{pgfscope}%
\pgfpathrectangle{\pgfqpoint{1.150000in}{0.150000in}}{\pgfqpoint{5.700000in}{5.700000in}}%
\pgfusepath{clip}%
\pgfsetbuttcap%
\pgfsetroundjoin%
\definecolor{currentfill}{rgb}{0.273809,0.031497,0.358853}%
\pgfsetfillcolor{currentfill}%
\pgfsetfillopacity{0.700000}%
\pgfsetlinewidth{0.000000pt}%
\definecolor{currentstroke}{rgb}{0.000000,0.000000,0.000000}%
\pgfsetstrokecolor{currentstroke}%
\pgfsetdash{}{0pt}%
\pgfpathmoveto{\pgfqpoint{4.448226in}{2.096823in}}%
\pgfpathlineto{\pgfqpoint{4.462001in}{2.094200in}}%
\pgfpathlineto{\pgfqpoint{4.475782in}{2.091603in}}%
\pgfpathlineto{\pgfqpoint{4.489571in}{2.089032in}}%
\pgfpathlineto{\pgfqpoint{4.503366in}{2.086487in}}%
\pgfpathlineto{\pgfqpoint{4.495609in}{2.077578in}}%
\pgfpathlineto{\pgfqpoint{4.487847in}{2.068660in}}%
\pgfpathlineto{\pgfqpoint{4.480079in}{2.059734in}}%
\pgfpathlineto{\pgfqpoint{4.472305in}{2.050804in}}%
\pgfpathlineto{\pgfqpoint{4.458499in}{2.053471in}}%
\pgfpathlineto{\pgfqpoint{4.444699in}{2.056164in}}%
\pgfpathlineto{\pgfqpoint{4.430907in}{2.058884in}}%
\pgfpathlineto{\pgfqpoint{4.417122in}{2.061629in}}%
\pgfpathlineto{\pgfqpoint{4.424906in}{2.070432in}}%
\pgfpathlineto{\pgfqpoint{4.432685in}{2.079234in}}%
\pgfpathlineto{\pgfqpoint{4.440459in}{2.088032in}}%
\pgfpathlineto{\pgfqpoint{4.448226in}{2.096823in}}%
\pgfpathclose%
\pgfusepath{fill}%
\end{pgfscope}%
\begin{pgfscope}%
\pgfpathrectangle{\pgfqpoint{1.150000in}{0.150000in}}{\pgfqpoint{5.700000in}{5.700000in}}%
\pgfusepath{clip}%
\pgfsetbuttcap%
\pgfsetroundjoin%
\definecolor{currentfill}{rgb}{0.281412,0.155834,0.469201}%
\pgfsetfillcolor{currentfill}%
\pgfsetfillopacity{0.700000}%
\pgfsetlinewidth{0.000000pt}%
\definecolor{currentstroke}{rgb}{0.000000,0.000000,0.000000}%
\pgfsetstrokecolor{currentstroke}%
\pgfsetdash{}{0pt}%
\pgfpathmoveto{\pgfqpoint{5.388935in}{2.319933in}}%
\pgfpathlineto{\pgfqpoint{5.402986in}{2.319016in}}%
\pgfpathlineto{\pgfqpoint{5.417046in}{2.318124in}}%
\pgfpathlineto{\pgfqpoint{5.431115in}{2.317256in}}%
\pgfpathlineto{\pgfqpoint{5.445192in}{2.316413in}}%
\pgfpathlineto{\pgfqpoint{5.437803in}{2.308962in}}%
\pgfpathlineto{\pgfqpoint{5.430406in}{2.301423in}}%
\pgfpathlineto{\pgfqpoint{5.423001in}{2.293796in}}%
\pgfpathlineto{\pgfqpoint{5.415588in}{2.286080in}}%
\pgfpathlineto{\pgfqpoint{5.401498in}{2.286923in}}%
\pgfpathlineto{\pgfqpoint{5.387416in}{2.287791in}}%
\pgfpathlineto{\pgfqpoint{5.373343in}{2.288684in}}%
\pgfpathlineto{\pgfqpoint{5.359279in}{2.289602in}}%
\pgfpathlineto{\pgfqpoint{5.366704in}{2.297313in}}%
\pgfpathlineto{\pgfqpoint{5.374122in}{2.304938in}}%
\pgfpathlineto{\pgfqpoint{5.381532in}{2.312478in}}%
\pgfpathlineto{\pgfqpoint{5.388935in}{2.319933in}}%
\pgfpathclose%
\pgfusepath{fill}%
\end{pgfscope}%
\begin{pgfscope}%
\pgfpathrectangle{\pgfqpoint{1.150000in}{0.150000in}}{\pgfqpoint{5.700000in}{5.700000in}}%
\pgfusepath{clip}%
\pgfsetbuttcap%
\pgfsetroundjoin%
\definecolor{currentfill}{rgb}{0.267004,0.004874,0.329415}%
\pgfsetfillcolor{currentfill}%
\pgfsetfillopacity{0.700000}%
\pgfsetlinewidth{0.000000pt}%
\definecolor{currentstroke}{rgb}{0.000000,0.000000,0.000000}%
\pgfsetstrokecolor{currentstroke}%
\pgfsetdash{}{0pt}%
\pgfpathmoveto{\pgfqpoint{3.993616in}{2.041577in}}%
\pgfpathlineto{\pgfqpoint{4.007278in}{2.037745in}}%
\pgfpathlineto{\pgfqpoint{4.020947in}{2.033941in}}%
\pgfpathlineto{\pgfqpoint{4.034621in}{2.030165in}}%
\pgfpathlineto{\pgfqpoint{4.048302in}{2.026416in}}%
\pgfpathlineto{\pgfqpoint{4.040384in}{2.018748in}}%
\pgfpathlineto{\pgfqpoint{4.032460in}{2.011145in}}%
\pgfpathlineto{\pgfqpoint{4.024530in}{2.003610in}}%
\pgfpathlineto{\pgfqpoint{4.016593in}{1.996149in}}%
\pgfpathlineto{\pgfqpoint{4.002898in}{2.000073in}}%
\pgfpathlineto{\pgfqpoint{3.989210in}{2.004025in}}%
\pgfpathlineto{\pgfqpoint{3.975527in}{2.008004in}}%
\pgfpathlineto{\pgfqpoint{3.961851in}{2.012011in}}%
\pgfpathlineto{\pgfqpoint{3.969801in}{2.019292in}}%
\pgfpathlineto{\pgfqpoint{3.977746in}{2.026649in}}%
\pgfpathlineto{\pgfqpoint{3.985684in}{2.034079in}}%
\pgfpathlineto{\pgfqpoint{3.993616in}{2.041577in}}%
\pgfpathclose%
\pgfusepath{fill}%
\end{pgfscope}%
\begin{pgfscope}%
\pgfpathrectangle{\pgfqpoint{1.150000in}{0.150000in}}{\pgfqpoint{5.700000in}{5.700000in}}%
\pgfusepath{clip}%
\pgfsetbuttcap%
\pgfsetroundjoin%
\definecolor{currentfill}{rgb}{0.282884,0.135920,0.453427}%
\pgfsetfillcolor{currentfill}%
\pgfsetfillopacity{0.700000}%
\pgfsetlinewidth{0.000000pt}%
\definecolor{currentstroke}{rgb}{0.000000,0.000000,0.000000}%
\pgfsetstrokecolor{currentstroke}%
\pgfsetdash{}{0pt}%
\pgfpathmoveto{\pgfqpoint{2.897277in}{2.277699in}}%
\pgfpathlineto{\pgfqpoint{2.910767in}{2.270300in}}%
\pgfpathlineto{\pgfqpoint{2.924260in}{2.262941in}}%
\pgfpathlineto{\pgfqpoint{2.937757in}{2.255620in}}%
\pgfpathlineto{\pgfqpoint{2.951257in}{2.248338in}}%
\pgfpathlineto{\pgfqpoint{2.942759in}{2.248618in}}%
\pgfpathlineto{\pgfqpoint{2.934246in}{2.249171in}}%
\pgfpathlineto{\pgfqpoint{2.925718in}{2.250007in}}%
\pgfpathlineto{\pgfqpoint{2.917175in}{2.251131in}}%
\pgfpathlineto{\pgfqpoint{2.903645in}{2.258685in}}%
\pgfpathlineto{\pgfqpoint{2.890119in}{2.266277in}}%
\pgfpathlineto{\pgfqpoint{2.876597in}{2.273908in}}%
\pgfpathlineto{\pgfqpoint{2.863078in}{2.281578in}}%
\pgfpathlineto{\pgfqpoint{2.871651in}{2.280177in}}%
\pgfpathlineto{\pgfqpoint{2.880209in}{2.279068in}}%
\pgfpathlineto{\pgfqpoint{2.888751in}{2.278244in}}%
\pgfpathlineto{\pgfqpoint{2.897277in}{2.277699in}}%
\pgfpathclose%
\pgfusepath{fill}%
\end{pgfscope}%
\begin{pgfscope}%
\pgfpathrectangle{\pgfqpoint{1.150000in}{0.150000in}}{\pgfqpoint{5.700000in}{5.700000in}}%
\pgfusepath{clip}%
\pgfsetbuttcap%
\pgfsetroundjoin%
\definecolor{currentfill}{rgb}{0.271828,0.209303,0.504434}%
\pgfsetfillcolor{currentfill}%
\pgfsetfillopacity{0.700000}%
\pgfsetlinewidth{0.000000pt}%
\definecolor{currentstroke}{rgb}{0.000000,0.000000,0.000000}%
\pgfsetstrokecolor{currentstroke}%
\pgfsetdash{}{0pt}%
\pgfpathmoveto{\pgfqpoint{5.930014in}{2.430903in}}%
\pgfpathlineto{\pgfqpoint{5.944236in}{2.430469in}}%
\pgfpathlineto{\pgfqpoint{5.958467in}{2.430059in}}%
\pgfpathlineto{\pgfqpoint{5.972708in}{2.429673in}}%
\pgfpathlineto{\pgfqpoint{5.965589in}{2.424214in}}%
\pgfpathlineto{\pgfqpoint{5.958462in}{2.418674in}}%
\pgfpathlineto{\pgfqpoint{5.951326in}{2.413053in}}%
\pgfpathlineto{\pgfqpoint{5.944181in}{2.407347in}}%
\pgfpathlineto{\pgfqpoint{5.929923in}{2.407663in}}%
\pgfpathlineto{\pgfqpoint{5.915674in}{2.408004in}}%
\pgfpathlineto{\pgfqpoint{5.901434in}{2.408370in}}%
\pgfpathlineto{\pgfqpoint{5.908592in}{2.414124in}}%
\pgfpathlineto{\pgfqpoint{5.915741in}{2.419796in}}%
\pgfpathlineto{\pgfqpoint{5.922882in}{2.425388in}}%
\pgfpathlineto{\pgfqpoint{5.930014in}{2.430903in}}%
\pgfpathclose%
\pgfusepath{fill}%
\end{pgfscope}%
\begin{pgfscope}%
\pgfpathrectangle{\pgfqpoint{1.150000in}{0.150000in}}{\pgfqpoint{5.700000in}{5.700000in}}%
\pgfusepath{clip}%
\pgfsetbuttcap%
\pgfsetroundjoin%
\definecolor{currentfill}{rgb}{0.277941,0.056324,0.381191}%
\pgfsetfillcolor{currentfill}%
\pgfsetfillopacity{0.700000}%
\pgfsetlinewidth{0.000000pt}%
\definecolor{currentstroke}{rgb}{0.000000,0.000000,0.000000}%
\pgfsetstrokecolor{currentstroke}%
\pgfsetdash{}{0pt}%
\pgfpathmoveto{\pgfqpoint{3.288524in}{2.127177in}}%
\pgfpathlineto{\pgfqpoint{3.302058in}{2.121141in}}%
\pgfpathlineto{\pgfqpoint{3.315597in}{2.115138in}}%
\pgfpathlineto{\pgfqpoint{3.329141in}{2.109167in}}%
\pgfpathlineto{\pgfqpoint{3.342689in}{2.103229in}}%
\pgfpathlineto{\pgfqpoint{3.334441in}{2.100130in}}%
\pgfpathlineto{\pgfqpoint{3.326183in}{2.097232in}}%
\pgfpathlineto{\pgfqpoint{3.317915in}{2.094542in}}%
\pgfpathlineto{\pgfqpoint{3.309635in}{2.092064in}}%
\pgfpathlineto{\pgfqpoint{3.296064in}{2.098244in}}%
\pgfpathlineto{\pgfqpoint{3.282498in}{2.104457in}}%
\pgfpathlineto{\pgfqpoint{3.268936in}{2.110703in}}%
\pgfpathlineto{\pgfqpoint{3.255379in}{2.116982in}}%
\pgfpathlineto{\pgfqpoint{3.263682in}{2.119212in}}%
\pgfpathlineto{\pgfqpoint{3.271974in}{2.121659in}}%
\pgfpathlineto{\pgfqpoint{3.280254in}{2.124316in}}%
\pgfpathlineto{\pgfqpoint{3.288524in}{2.127177in}}%
\pgfpathclose%
\pgfusepath{fill}%
\end{pgfscope}%
\begin{pgfscope}%
\pgfpathrectangle{\pgfqpoint{1.150000in}{0.150000in}}{\pgfqpoint{5.700000in}{5.700000in}}%
\pgfusepath{clip}%
\pgfsetbuttcap%
\pgfsetroundjoin%
\definecolor{currentfill}{rgb}{0.250425,0.274290,0.533103}%
\pgfsetfillcolor{currentfill}%
\pgfsetfillopacity{0.700000}%
\pgfsetlinewidth{0.000000pt}%
\definecolor{currentstroke}{rgb}{0.000000,0.000000,0.000000}%
\pgfsetstrokecolor{currentstroke}%
\pgfsetdash{}{0pt}%
\pgfpathmoveto{\pgfqpoint{2.396361in}{2.567825in}}%
\pgfpathlineto{\pgfqpoint{2.409837in}{2.558405in}}%
\pgfpathlineto{\pgfqpoint{2.423315in}{2.549039in}}%
\pgfpathlineto{\pgfqpoint{2.436794in}{2.539725in}}%
\pgfpathlineto{\pgfqpoint{2.450275in}{2.530463in}}%
\pgfpathlineto{\pgfqpoint{2.441370in}{2.535519in}}%
\pgfpathlineto{\pgfqpoint{2.432444in}{2.540940in}}%
\pgfpathlineto{\pgfqpoint{2.423495in}{2.546734in}}%
\pgfpathlineto{\pgfqpoint{2.414523in}{2.552909in}}%
\pgfpathlineto{\pgfqpoint{2.401004in}{2.562476in}}%
\pgfpathlineto{\pgfqpoint{2.387487in}{2.572095in}}%
\pgfpathlineto{\pgfqpoint{2.373971in}{2.581767in}}%
\pgfpathlineto{\pgfqpoint{2.360456in}{2.591492in}}%
\pgfpathlineto{\pgfqpoint{2.369467in}{2.585006in}}%
\pgfpathlineto{\pgfqpoint{2.378455in}{2.578905in}}%
\pgfpathlineto{\pgfqpoint{2.387420in}{2.573181in}}%
\pgfpathlineto{\pgfqpoint{2.396361in}{2.567825in}}%
\pgfpathclose%
\pgfusepath{fill}%
\end{pgfscope}%
\begin{pgfscope}%
\pgfpathrectangle{\pgfqpoint{1.150000in}{0.150000in}}{\pgfqpoint{5.700000in}{5.700000in}}%
\pgfusepath{clip}%
\pgfsetbuttcap%
\pgfsetroundjoin%
\definecolor{currentfill}{rgb}{0.278791,0.062145,0.386592}%
\pgfsetfillcolor{currentfill}%
\pgfsetfillopacity{0.700000}%
\pgfsetlinewidth{0.000000pt}%
\definecolor{currentstroke}{rgb}{0.000000,0.000000,0.000000}%
\pgfsetstrokecolor{currentstroke}%
\pgfsetdash{}{0pt}%
\pgfpathmoveto{\pgfqpoint{4.675679in}{2.139296in}}%
\pgfpathlineto{\pgfqpoint{4.689520in}{2.137186in}}%
\pgfpathlineto{\pgfqpoint{4.703368in}{2.135102in}}%
\pgfpathlineto{\pgfqpoint{4.717224in}{2.133043in}}%
\pgfpathlineto{\pgfqpoint{4.731088in}{2.131010in}}%
\pgfpathlineto{\pgfqpoint{4.723408in}{2.121998in}}%
\pgfpathlineto{\pgfqpoint{4.715722in}{2.112946in}}%
\pgfpathlineto{\pgfqpoint{4.708030in}{2.103859in}}%
\pgfpathlineto{\pgfqpoint{4.700332in}{2.094736in}}%
\pgfpathlineto{\pgfqpoint{4.686458in}{2.096865in}}%
\pgfpathlineto{\pgfqpoint{4.672592in}{2.099020in}}%
\pgfpathlineto{\pgfqpoint{4.658733in}{2.101200in}}%
\pgfpathlineto{\pgfqpoint{4.644882in}{2.103405in}}%
\pgfpathlineto{\pgfqpoint{4.652590in}{2.112427in}}%
\pgfpathlineto{\pgfqpoint{4.660292in}{2.121417in}}%
\pgfpathlineto{\pgfqpoint{4.667989in}{2.130374in}}%
\pgfpathlineto{\pgfqpoint{4.675679in}{2.139296in}}%
\pgfpathclose%
\pgfusepath{fill}%
\end{pgfscope}%
\begin{pgfscope}%
\pgfpathrectangle{\pgfqpoint{1.150000in}{0.150000in}}{\pgfqpoint{5.700000in}{5.700000in}}%
\pgfusepath{clip}%
\pgfsetbuttcap%
\pgfsetroundjoin%
\definecolor{currentfill}{rgb}{0.282910,0.105393,0.426902}%
\pgfsetfillcolor{currentfill}%
\pgfsetfillopacity{0.700000}%
\pgfsetlinewidth{0.000000pt}%
\definecolor{currentstroke}{rgb}{0.000000,0.000000,0.000000}%
\pgfsetstrokecolor{currentstroke}%
\pgfsetdash{}{0pt}%
\pgfpathmoveto{\pgfqpoint{4.989369in}{2.215585in}}%
\pgfpathlineto{\pgfqpoint{5.003303in}{2.214085in}}%
\pgfpathlineto{\pgfqpoint{5.017244in}{2.212611in}}%
\pgfpathlineto{\pgfqpoint{5.031194in}{2.211162in}}%
\pgfpathlineto{\pgfqpoint{5.045151in}{2.209738in}}%
\pgfpathlineto{\pgfqpoint{5.037587in}{2.201112in}}%
\pgfpathlineto{\pgfqpoint{5.030017in}{2.192417in}}%
\pgfpathlineto{\pgfqpoint{5.022440in}{2.183653in}}%
\pgfpathlineto{\pgfqpoint{5.014856in}{2.174823in}}%
\pgfpathlineto{\pgfqpoint{5.000888in}{2.176302in}}%
\pgfpathlineto{\pgfqpoint{4.986927in}{2.177807in}}%
\pgfpathlineto{\pgfqpoint{4.972975in}{2.179336in}}%
\pgfpathlineto{\pgfqpoint{4.959031in}{2.180890in}}%
\pgfpathlineto{\pgfqpoint{4.966625in}{2.189661in}}%
\pgfpathlineto{\pgfqpoint{4.974213in}{2.198367in}}%
\pgfpathlineto{\pgfqpoint{4.981795in}{2.207009in}}%
\pgfpathlineto{\pgfqpoint{4.989369in}{2.215585in}}%
\pgfpathclose%
\pgfusepath{fill}%
\end{pgfscope}%
\begin{pgfscope}%
\pgfpathrectangle{\pgfqpoint{1.150000in}{0.150000in}}{\pgfqpoint{5.700000in}{5.700000in}}%
\pgfusepath{clip}%
\pgfsetbuttcap%
\pgfsetroundjoin%
\definecolor{currentfill}{rgb}{0.282290,0.145912,0.461510}%
\pgfsetfillcolor{currentfill}%
\pgfsetfillopacity{0.700000}%
\pgfsetlinewidth{0.000000pt}%
\definecolor{currentstroke}{rgb}{0.000000,0.000000,0.000000}%
\pgfsetstrokecolor{currentstroke}%
\pgfsetdash{}{0pt}%
\pgfpathmoveto{\pgfqpoint{5.303106in}{2.293520in}}%
\pgfpathlineto{\pgfqpoint{5.317137in}{2.292504in}}%
\pgfpathlineto{\pgfqpoint{5.331176in}{2.291512in}}%
\pgfpathlineto{\pgfqpoint{5.345223in}{2.290544in}}%
\pgfpathlineto{\pgfqpoint{5.359279in}{2.289602in}}%
\pgfpathlineto{\pgfqpoint{5.351846in}{2.281805in}}%
\pgfpathlineto{\pgfqpoint{5.344406in}{2.273922in}}%
\pgfpathlineto{\pgfqpoint{5.336958in}{2.265952in}}%
\pgfpathlineto{\pgfqpoint{5.329503in}{2.257896in}}%
\pgfpathlineto{\pgfqpoint{5.315434in}{2.258852in}}%
\pgfpathlineto{\pgfqpoint{5.301375in}{2.259834in}}%
\pgfpathlineto{\pgfqpoint{5.287324in}{2.260840in}}%
\pgfpathlineto{\pgfqpoint{5.273281in}{2.261871in}}%
\pgfpathlineto{\pgfqpoint{5.280748in}{2.269908in}}%
\pgfpathlineto{\pgfqpoint{5.288208in}{2.277862in}}%
\pgfpathlineto{\pgfqpoint{5.295661in}{2.285733in}}%
\pgfpathlineto{\pgfqpoint{5.303106in}{2.293520in}}%
\pgfpathclose%
\pgfusepath{fill}%
\end{pgfscope}%
\begin{pgfscope}%
\pgfpathrectangle{\pgfqpoint{1.150000in}{0.150000in}}{\pgfqpoint{5.700000in}{5.700000in}}%
\pgfusepath{clip}%
\pgfsetbuttcap%
\pgfsetroundjoin%
\definecolor{currentfill}{rgb}{0.278012,0.180367,0.486697}%
\pgfsetfillcolor{currentfill}%
\pgfsetfillopacity{0.700000}%
\pgfsetlinewidth{0.000000pt}%
\definecolor{currentstroke}{rgb}{0.000000,0.000000,0.000000}%
\pgfsetstrokecolor{currentstroke}%
\pgfsetdash{}{0pt}%
\pgfpathmoveto{\pgfqpoint{5.616734in}{2.366771in}}%
\pgfpathlineto{\pgfqpoint{5.630862in}{2.366109in}}%
\pgfpathlineto{\pgfqpoint{5.644999in}{2.365471in}}%
\pgfpathlineto{\pgfqpoint{5.659144in}{2.364859in}}%
\pgfpathlineto{\pgfqpoint{5.673298in}{2.364271in}}%
\pgfpathlineto{\pgfqpoint{5.666016in}{2.357589in}}%
\pgfpathlineto{\pgfqpoint{5.658725in}{2.350817in}}%
\pgfpathlineto{\pgfqpoint{5.651425in}{2.343953in}}%
\pgfpathlineto{\pgfqpoint{5.644118in}{2.336996in}}%
\pgfpathlineto{\pgfqpoint{5.629948in}{2.337557in}}%
\pgfpathlineto{\pgfqpoint{5.615788in}{2.338142in}}%
\pgfpathlineto{\pgfqpoint{5.601637in}{2.338752in}}%
\pgfpathlineto{\pgfqpoint{5.587494in}{2.339387in}}%
\pgfpathlineto{\pgfqpoint{5.594817in}{2.346366in}}%
\pgfpathlineto{\pgfqpoint{5.602131in}{2.353255in}}%
\pgfpathlineto{\pgfqpoint{5.609436in}{2.360056in}}%
\pgfpathlineto{\pgfqpoint{5.616734in}{2.366771in}}%
\pgfpathclose%
\pgfusepath{fill}%
\end{pgfscope}%
\begin{pgfscope}%
\pgfpathrectangle{\pgfqpoint{1.150000in}{0.150000in}}{\pgfqpoint{5.700000in}{5.700000in}}%
\pgfusepath{clip}%
\pgfsetbuttcap%
\pgfsetroundjoin%
\definecolor{currentfill}{rgb}{0.271305,0.019942,0.347269}%
\pgfsetfillcolor{currentfill}%
\pgfsetfillopacity{0.700000}%
\pgfsetlinewidth{0.000000pt}%
\definecolor{currentstroke}{rgb}{0.000000,0.000000,0.000000}%
\pgfsetstrokecolor{currentstroke}%
\pgfsetdash{}{0pt}%
\pgfpathmoveto{\pgfqpoint{4.362051in}{2.072870in}}%
\pgfpathlineto{\pgfqpoint{4.375808in}{2.070021in}}%
\pgfpathlineto{\pgfqpoint{4.389573in}{2.067197in}}%
\pgfpathlineto{\pgfqpoint{4.403344in}{2.064400in}}%
\pgfpathlineto{\pgfqpoint{4.417122in}{2.061629in}}%
\pgfpathlineto{\pgfqpoint{4.409332in}{2.052828in}}%
\pgfpathlineto{\pgfqpoint{4.401537in}{2.044031in}}%
\pgfpathlineto{\pgfqpoint{4.393736in}{2.035242in}}%
\pgfpathlineto{\pgfqpoint{4.385930in}{2.026465in}}%
\pgfpathlineto{\pgfqpoint{4.372140in}{2.029371in}}%
\pgfpathlineto{\pgfqpoint{4.358358in}{2.032304in}}%
\pgfpathlineto{\pgfqpoint{4.344583in}{2.035263in}}%
\pgfpathlineto{\pgfqpoint{4.330814in}{2.038248in}}%
\pgfpathlineto{\pgfqpoint{4.338632in}{2.046885in}}%
\pgfpathlineto{\pgfqpoint{4.346444in}{2.055537in}}%
\pgfpathlineto{\pgfqpoint{4.354250in}{2.064199in}}%
\pgfpathlineto{\pgfqpoint{4.362051in}{2.072870in}}%
\pgfpathclose%
\pgfusepath{fill}%
\end{pgfscope}%
\begin{pgfscope}%
\pgfpathrectangle{\pgfqpoint{1.150000in}{0.150000in}}{\pgfqpoint{5.700000in}{5.700000in}}%
\pgfusepath{clip}%
\pgfsetbuttcap%
\pgfsetroundjoin%
\definecolor{currentfill}{rgb}{0.268510,0.009605,0.335427}%
\pgfsetfillcolor{currentfill}%
\pgfsetfillopacity{0.700000}%
\pgfsetlinewidth{0.000000pt}%
\definecolor{currentstroke}{rgb}{0.000000,0.000000,0.000000}%
\pgfsetstrokecolor{currentstroke}%
\pgfsetdash{}{0pt}%
\pgfpathmoveto{\pgfqpoint{4.134652in}{2.043575in}}%
\pgfpathlineto{\pgfqpoint{4.148352in}{2.040123in}}%
\pgfpathlineto{\pgfqpoint{4.162058in}{2.036698in}}%
\pgfpathlineto{\pgfqpoint{4.175772in}{2.033300in}}%
\pgfpathlineto{\pgfqpoint{4.189491in}{2.029928in}}%
\pgfpathlineto{\pgfqpoint{4.181622in}{2.021728in}}%
\pgfpathlineto{\pgfqpoint{4.173747in}{2.013570in}}%
\pgfpathlineto{\pgfqpoint{4.165867in}{2.005458in}}%
\pgfpathlineto{\pgfqpoint{4.157981in}{1.997396in}}%
\pgfpathlineto{\pgfqpoint{4.144248in}{2.000929in}}%
\pgfpathlineto{\pgfqpoint{4.130522in}{2.004490in}}%
\pgfpathlineto{\pgfqpoint{4.116803in}{2.008077in}}%
\pgfpathlineto{\pgfqpoint{4.103090in}{2.011690in}}%
\pgfpathlineto{\pgfqpoint{4.110989in}{2.019586in}}%
\pgfpathlineto{\pgfqpoint{4.118883in}{2.027534in}}%
\pgfpathlineto{\pgfqpoint{4.126770in}{2.035532in}}%
\pgfpathlineto{\pgfqpoint{4.134652in}{2.043575in}}%
\pgfpathclose%
\pgfusepath{fill}%
\end{pgfscope}%
\begin{pgfscope}%
\pgfpathrectangle{\pgfqpoint{1.150000in}{0.150000in}}{\pgfqpoint{5.700000in}{5.700000in}}%
\pgfusepath{clip}%
\pgfsetbuttcap%
\pgfsetroundjoin%
\definecolor{currentfill}{rgb}{0.276194,0.190074,0.493001}%
\pgfsetfillcolor{currentfill}%
\pgfsetfillopacity{0.700000}%
\pgfsetlinewidth{0.000000pt}%
\definecolor{currentstroke}{rgb}{0.000000,0.000000,0.000000}%
\pgfsetstrokecolor{currentstroke}%
\pgfsetdash{}{0pt}%
\pgfpathmoveto{\pgfqpoint{2.701084in}{2.376799in}}%
\pgfpathlineto{\pgfqpoint{2.714568in}{2.368632in}}%
\pgfpathlineto{\pgfqpoint{2.728054in}{2.360508in}}%
\pgfpathlineto{\pgfqpoint{2.741543in}{2.352428in}}%
\pgfpathlineto{\pgfqpoint{2.755035in}{2.344390in}}%
\pgfpathlineto{\pgfqpoint{2.746382in}{2.346653in}}%
\pgfpathlineto{\pgfqpoint{2.737712in}{2.349231in}}%
\pgfpathlineto{\pgfqpoint{2.729024in}{2.352131in}}%
\pgfpathlineto{\pgfqpoint{2.720317in}{2.355362in}}%
\pgfpathlineto{\pgfqpoint{2.706792in}{2.363687in}}%
\pgfpathlineto{\pgfqpoint{2.693270in}{2.372054in}}%
\pgfpathlineto{\pgfqpoint{2.679750in}{2.380465in}}%
\pgfpathlineto{\pgfqpoint{2.666234in}{2.388920in}}%
\pgfpathlineto{\pgfqpoint{2.674974in}{2.385397in}}%
\pgfpathlineto{\pgfqpoint{2.683696in}{2.382208in}}%
\pgfpathlineto{\pgfqpoint{2.692399in}{2.379344in}}%
\pgfpathlineto{\pgfqpoint{2.701084in}{2.376799in}}%
\pgfpathclose%
\pgfusepath{fill}%
\end{pgfscope}%
\begin{pgfscope}%
\pgfpathrectangle{\pgfqpoint{1.150000in}{0.150000in}}{\pgfqpoint{5.700000in}{5.700000in}}%
\pgfusepath{clip}%
\pgfsetbuttcap%
\pgfsetroundjoin%
\definecolor{currentfill}{rgb}{0.269944,0.014625,0.341379}%
\pgfsetfillcolor{currentfill}%
\pgfsetfillopacity{0.700000}%
\pgfsetlinewidth{0.000000pt}%
\definecolor{currentstroke}{rgb}{0.000000,0.000000,0.000000}%
\pgfsetstrokecolor{currentstroke}%
\pgfsetdash{}{0pt}%
\pgfpathmoveto{\pgfqpoint{3.625095in}{2.050911in}}%
\pgfpathlineto{\pgfqpoint{3.638688in}{2.045943in}}%
\pgfpathlineto{\pgfqpoint{3.652287in}{2.041006in}}%
\pgfpathlineto{\pgfqpoint{3.665891in}{2.036098in}}%
\pgfpathlineto{\pgfqpoint{3.679501in}{2.031219in}}%
\pgfpathlineto{\pgfqpoint{3.671424in}{2.025675in}}%
\pgfpathlineto{\pgfqpoint{3.663340in}{2.020269in}}%
\pgfpathlineto{\pgfqpoint{3.655247in}{2.015007in}}%
\pgfpathlineto{\pgfqpoint{3.647147in}{2.009893in}}%
\pgfpathlineto{\pgfqpoint{3.633520in}{2.014987in}}%
\pgfpathlineto{\pgfqpoint{3.619898in}{2.020110in}}%
\pgfpathlineto{\pgfqpoint{3.606281in}{2.025262in}}%
\pgfpathlineto{\pgfqpoint{3.592669in}{2.030445in}}%
\pgfpathlineto{\pgfqpoint{3.600788in}{2.035338in}}%
\pgfpathlineto{\pgfqpoint{3.608899in}{2.040384in}}%
\pgfpathlineto{\pgfqpoint{3.617001in}{2.045576in}}%
\pgfpathlineto{\pgfqpoint{3.625095in}{2.050911in}}%
\pgfpathclose%
\pgfusepath{fill}%
\end{pgfscope}%
\begin{pgfscope}%
\pgfpathrectangle{\pgfqpoint{1.150000in}{0.150000in}}{\pgfqpoint{5.700000in}{5.700000in}}%
\pgfusepath{clip}%
\pgfsetbuttcap%
\pgfsetroundjoin%
\definecolor{currentfill}{rgb}{0.267004,0.004874,0.329415}%
\pgfsetfillcolor{currentfill}%
\pgfsetfillopacity{0.700000}%
\pgfsetlinewidth{0.000000pt}%
\definecolor{currentstroke}{rgb}{0.000000,0.000000,0.000000}%
\pgfsetstrokecolor{currentstroke}%
\pgfsetdash{}{0pt}%
\pgfpathmoveto{\pgfqpoint{3.766155in}{2.036255in}}%
\pgfpathlineto{\pgfqpoint{3.779776in}{2.031723in}}%
\pgfpathlineto{\pgfqpoint{3.793402in}{2.027220in}}%
\pgfpathlineto{\pgfqpoint{3.807034in}{2.022745in}}%
\pgfpathlineto{\pgfqpoint{3.820672in}{2.018299in}}%
\pgfpathlineto{\pgfqpoint{3.812659in}{2.011861in}}%
\pgfpathlineto{\pgfqpoint{3.804638in}{2.005534in}}%
\pgfpathlineto{\pgfqpoint{3.796610in}{1.999323in}}%
\pgfpathlineto{\pgfqpoint{3.788575in}{1.993233in}}%
\pgfpathlineto{\pgfqpoint{3.774921in}{1.997881in}}%
\pgfpathlineto{\pgfqpoint{3.761273in}{2.002558in}}%
\pgfpathlineto{\pgfqpoint{3.747630in}{2.007262in}}%
\pgfpathlineto{\pgfqpoint{3.733993in}{2.011996in}}%
\pgfpathlineto{\pgfqpoint{3.742045in}{2.017879in}}%
\pgfpathlineto{\pgfqpoint{3.750089in}{2.023887in}}%
\pgfpathlineto{\pgfqpoint{3.758126in}{2.030014in}}%
\pgfpathlineto{\pgfqpoint{3.766155in}{2.036255in}}%
\pgfpathclose%
\pgfusepath{fill}%
\end{pgfscope}%
\begin{pgfscope}%
\pgfpathrectangle{\pgfqpoint{1.150000in}{0.150000in}}{\pgfqpoint{5.700000in}{5.700000in}}%
\pgfusepath{clip}%
\pgfsetbuttcap%
\pgfsetroundjoin%
\definecolor{currentfill}{rgb}{0.277018,0.050344,0.375715}%
\pgfsetfillcolor{currentfill}%
\pgfsetfillopacity{0.700000}%
\pgfsetlinewidth{0.000000pt}%
\definecolor{currentstroke}{rgb}{0.000000,0.000000,0.000000}%
\pgfsetstrokecolor{currentstroke}%
\pgfsetdash{}{0pt}%
\pgfpathmoveto{\pgfqpoint{4.589551in}{2.112482in}}%
\pgfpathlineto{\pgfqpoint{4.603373in}{2.110174in}}%
\pgfpathlineto{\pgfqpoint{4.617202in}{2.107892in}}%
\pgfpathlineto{\pgfqpoint{4.631038in}{2.105636in}}%
\pgfpathlineto{\pgfqpoint{4.644882in}{2.103405in}}%
\pgfpathlineto{\pgfqpoint{4.637168in}{2.094354in}}%
\pgfpathlineto{\pgfqpoint{4.629449in}{2.085277in}}%
\pgfpathlineto{\pgfqpoint{4.621724in}{2.076175in}}%
\pgfpathlineto{\pgfqpoint{4.613994in}{2.067051in}}%
\pgfpathlineto{\pgfqpoint{4.600140in}{2.069391in}}%
\pgfpathlineto{\pgfqpoint{4.586293in}{2.071756in}}%
\pgfpathlineto{\pgfqpoint{4.572454in}{2.074147in}}%
\pgfpathlineto{\pgfqpoint{4.558622in}{2.076564in}}%
\pgfpathlineto{\pgfqpoint{4.566362in}{2.085574in}}%
\pgfpathlineto{\pgfqpoint{4.574098in}{2.094565in}}%
\pgfpathlineto{\pgfqpoint{4.581827in}{2.103535in}}%
\pgfpathlineto{\pgfqpoint{4.589551in}{2.112482in}}%
\pgfpathclose%
\pgfusepath{fill}%
\end{pgfscope}%
\begin{pgfscope}%
\pgfpathrectangle{\pgfqpoint{1.150000in}{0.150000in}}{\pgfqpoint{5.700000in}{5.700000in}}%
\pgfusepath{clip}%
\pgfsetbuttcap%
\pgfsetroundjoin%
\definecolor{currentfill}{rgb}{0.281924,0.089666,0.412415}%
\pgfsetfillcolor{currentfill}%
\pgfsetfillopacity{0.700000}%
\pgfsetlinewidth{0.000000pt}%
\definecolor{currentstroke}{rgb}{0.000000,0.000000,0.000000}%
\pgfsetstrokecolor{currentstroke}%
\pgfsetdash{}{0pt}%
\pgfpathmoveto{\pgfqpoint{4.903334in}{2.187359in}}%
\pgfpathlineto{\pgfqpoint{4.917246in}{2.185704in}}%
\pgfpathlineto{\pgfqpoint{4.931167in}{2.184075in}}%
\pgfpathlineto{\pgfqpoint{4.945095in}{2.182470in}}%
\pgfpathlineto{\pgfqpoint{4.959031in}{2.180890in}}%
\pgfpathlineto{\pgfqpoint{4.951430in}{2.172057in}}%
\pgfpathlineto{\pgfqpoint{4.943824in}{2.163163in}}%
\pgfpathlineto{\pgfqpoint{4.936211in}{2.154208in}}%
\pgfpathlineto{\pgfqpoint{4.928591in}{2.145195in}}%
\pgfpathlineto{\pgfqpoint{4.914645in}{2.146843in}}%
\pgfpathlineto{\pgfqpoint{4.900706in}{2.148516in}}%
\pgfpathlineto{\pgfqpoint{4.886775in}{2.150215in}}%
\pgfpathlineto{\pgfqpoint{4.872852in}{2.151939in}}%
\pgfpathlineto{\pgfqpoint{4.880482in}{2.160878in}}%
\pgfpathlineto{\pgfqpoint{4.888106in}{2.169763in}}%
\pgfpathlineto{\pgfqpoint{4.895723in}{2.178590in}}%
\pgfpathlineto{\pgfqpoint{4.903334in}{2.187359in}}%
\pgfpathclose%
\pgfusepath{fill}%
\end{pgfscope}%
\begin{pgfscope}%
\pgfpathrectangle{\pgfqpoint{1.150000in}{0.150000in}}{\pgfqpoint{5.700000in}{5.700000in}}%
\pgfusepath{clip}%
\pgfsetbuttcap%
\pgfsetroundjoin%
\definecolor{currentfill}{rgb}{0.272594,0.025563,0.353093}%
\pgfsetfillcolor{currentfill}%
\pgfsetfillopacity{0.700000}%
\pgfsetlinewidth{0.000000pt}%
\definecolor{currentstroke}{rgb}{0.000000,0.000000,0.000000}%
\pgfsetstrokecolor{currentstroke}%
\pgfsetdash{}{0pt}%
\pgfpathmoveto{\pgfqpoint{3.483963in}{2.072986in}}%
\pgfpathlineto{\pgfqpoint{3.497533in}{2.067562in}}%
\pgfpathlineto{\pgfqpoint{3.511109in}{2.062168in}}%
\pgfpathlineto{\pgfqpoint{3.524689in}{2.056805in}}%
\pgfpathlineto{\pgfqpoint{3.538275in}{2.051472in}}%
\pgfpathlineto{\pgfqpoint{3.530128in}{2.046960in}}%
\pgfpathlineto{\pgfqpoint{3.521973in}{2.042615in}}%
\pgfpathlineto{\pgfqpoint{3.513808in}{2.038442in}}%
\pgfpathlineto{\pgfqpoint{3.505634in}{2.034447in}}%
\pgfpathlineto{\pgfqpoint{3.492029in}{2.040008in}}%
\pgfpathlineto{\pgfqpoint{3.478428in}{2.045600in}}%
\pgfpathlineto{\pgfqpoint{3.464832in}{2.051222in}}%
\pgfpathlineto{\pgfqpoint{3.451242in}{2.056875in}}%
\pgfpathlineto{\pgfqpoint{3.459436in}{2.060636in}}%
\pgfpathlineto{\pgfqpoint{3.467621in}{2.064578in}}%
\pgfpathlineto{\pgfqpoint{3.475797in}{2.068697in}}%
\pgfpathlineto{\pgfqpoint{3.483963in}{2.072986in}}%
\pgfpathclose%
\pgfusepath{fill}%
\end{pgfscope}%
\begin{pgfscope}%
\pgfpathrectangle{\pgfqpoint{1.150000in}{0.150000in}}{\pgfqpoint{5.700000in}{5.700000in}}%
\pgfusepath{clip}%
\pgfsetbuttcap%
\pgfsetroundjoin%
\definecolor{currentfill}{rgb}{0.281446,0.084320,0.407414}%
\pgfsetfillcolor{currentfill}%
\pgfsetfillopacity{0.700000}%
\pgfsetlinewidth{0.000000pt}%
\definecolor{currentstroke}{rgb}{0.000000,0.000000,0.000000}%
\pgfsetstrokecolor{currentstroke}%
\pgfsetdash{}{0pt}%
\pgfpathmoveto{\pgfqpoint{3.147075in}{2.168425in}}%
\pgfpathlineto{\pgfqpoint{3.160598in}{2.161874in}}%
\pgfpathlineto{\pgfqpoint{3.174126in}{2.155359in}}%
\pgfpathlineto{\pgfqpoint{3.187657in}{2.148878in}}%
\pgfpathlineto{\pgfqpoint{3.201193in}{2.142431in}}%
\pgfpathlineto{\pgfqpoint{3.192854in}{2.140675in}}%
\pgfpathlineto{\pgfqpoint{3.184503in}{2.139153in}}%
\pgfpathlineto{\pgfqpoint{3.176140in}{2.137870in}}%
\pgfpathlineto{\pgfqpoint{3.167764in}{2.136833in}}%
\pgfpathlineto{\pgfqpoint{3.154203in}{2.143537in}}%
\pgfpathlineto{\pgfqpoint{3.140646in}{2.150274in}}%
\pgfpathlineto{\pgfqpoint{3.127093in}{2.157046in}}%
\pgfpathlineto{\pgfqpoint{3.113544in}{2.163853in}}%
\pgfpathlineto{\pgfqpoint{3.121946in}{2.164628in}}%
\pgfpathlineto{\pgfqpoint{3.130335in}{2.165652in}}%
\pgfpathlineto{\pgfqpoint{3.138711in}{2.166920in}}%
\pgfpathlineto{\pgfqpoint{3.147075in}{2.168425in}}%
\pgfpathclose%
\pgfusepath{fill}%
\end{pgfscope}%
\begin{pgfscope}%
\pgfpathrectangle{\pgfqpoint{1.150000in}{0.150000in}}{\pgfqpoint{5.700000in}{5.700000in}}%
\pgfusepath{clip}%
\pgfsetbuttcap%
\pgfsetroundjoin%
\definecolor{currentfill}{rgb}{0.267004,0.004874,0.329415}%
\pgfsetfillcolor{currentfill}%
\pgfsetfillopacity{0.700000}%
\pgfsetlinewidth{0.000000pt}%
\definecolor{currentstroke}{rgb}{0.000000,0.000000,0.000000}%
\pgfsetstrokecolor{currentstroke}%
\pgfsetdash{}{0pt}%
\pgfpathmoveto{\pgfqpoint{3.907206in}{2.028313in}}%
\pgfpathlineto{\pgfqpoint{3.920858in}{2.024196in}}%
\pgfpathlineto{\pgfqpoint{3.934517in}{2.020106in}}%
\pgfpathlineto{\pgfqpoint{3.948181in}{2.016045in}}%
\pgfpathlineto{\pgfqpoint{3.961851in}{2.012011in}}%
\pgfpathlineto{\pgfqpoint{3.953894in}{2.004810in}}%
\pgfpathlineto{\pgfqpoint{3.945931in}{1.997695in}}%
\pgfpathlineto{\pgfqpoint{3.937961in}{1.990670in}}%
\pgfpathlineto{\pgfqpoint{3.929985in}{1.983740in}}%
\pgfpathlineto{\pgfqpoint{3.916300in}{1.987962in}}%
\pgfpathlineto{\pgfqpoint{3.902621in}{1.992212in}}%
\pgfpathlineto{\pgfqpoint{3.888948in}{1.996490in}}%
\pgfpathlineto{\pgfqpoint{3.875281in}{2.000796in}}%
\pgfpathlineto{\pgfqpoint{3.883273in}{2.007533in}}%
\pgfpathlineto{\pgfqpoint{3.891257in}{2.014368in}}%
\pgfpathlineto{\pgfqpoint{3.899235in}{2.021296in}}%
\pgfpathlineto{\pgfqpoint{3.907206in}{2.028313in}}%
\pgfpathclose%
\pgfusepath{fill}%
\end{pgfscope}%
\begin{pgfscope}%
\pgfpathrectangle{\pgfqpoint{1.150000in}{0.150000in}}{\pgfqpoint{5.700000in}{5.700000in}}%
\pgfusepath{clip}%
\pgfsetbuttcap%
\pgfsetroundjoin%
\definecolor{currentfill}{rgb}{0.273006,0.204520,0.501721}%
\pgfsetfillcolor{currentfill}%
\pgfsetfillopacity{0.700000}%
\pgfsetlinewidth{0.000000pt}%
\definecolor{currentstroke}{rgb}{0.000000,0.000000,0.000000}%
\pgfsetstrokecolor{currentstroke}%
\pgfsetdash{}{0pt}%
\pgfpathmoveto{\pgfqpoint{5.844566in}{2.410077in}}%
\pgfpathlineto{\pgfqpoint{5.858769in}{2.409614in}}%
\pgfpathlineto{\pgfqpoint{5.872982in}{2.409175in}}%
\pgfpathlineto{\pgfqpoint{5.887203in}{2.408760in}}%
\pgfpathlineto{\pgfqpoint{5.901434in}{2.408370in}}%
\pgfpathlineto{\pgfqpoint{5.894268in}{2.402532in}}%
\pgfpathlineto{\pgfqpoint{5.887092in}{2.396607in}}%
\pgfpathlineto{\pgfqpoint{5.879909in}{2.390593in}}%
\pgfpathlineto{\pgfqpoint{5.872716in}{2.384489in}}%
\pgfpathlineto{\pgfqpoint{5.858468in}{2.384824in}}%
\pgfpathlineto{\pgfqpoint{5.844230in}{2.385183in}}%
\pgfpathlineto{\pgfqpoint{5.830001in}{2.385567in}}%
\pgfpathlineto{\pgfqpoint{5.815781in}{2.385975in}}%
\pgfpathlineto{\pgfqpoint{5.822990in}{2.392130in}}%
\pgfpathlineto{\pgfqpoint{5.830190in}{2.398197in}}%
\pgfpathlineto{\pgfqpoint{5.837382in}{2.404178in}}%
\pgfpathlineto{\pgfqpoint{5.844566in}{2.410077in}}%
\pgfpathclose%
\pgfusepath{fill}%
\end{pgfscope}%
\begin{pgfscope}%
\pgfpathrectangle{\pgfqpoint{1.150000in}{0.150000in}}{\pgfqpoint{5.700000in}{5.700000in}}%
\pgfusepath{clip}%
\pgfsetbuttcap%
\pgfsetroundjoin%
\definecolor{currentfill}{rgb}{0.283187,0.125848,0.444960}%
\pgfsetfillcolor{currentfill}%
\pgfsetfillopacity{0.700000}%
\pgfsetlinewidth{0.000000pt}%
\definecolor{currentstroke}{rgb}{0.000000,0.000000,0.000000}%
\pgfsetstrokecolor{currentstroke}%
\pgfsetdash{}{0pt}%
\pgfpathmoveto{\pgfqpoint{2.951257in}{2.248338in}}%
\pgfpathlineto{\pgfqpoint{2.964761in}{2.241094in}}%
\pgfpathlineto{\pgfqpoint{2.978268in}{2.233888in}}%
\pgfpathlineto{\pgfqpoint{2.991779in}{2.226720in}}%
\pgfpathlineto{\pgfqpoint{3.005293in}{2.219589in}}%
\pgfpathlineto{\pgfqpoint{2.996823in}{2.219603in}}%
\pgfpathlineto{\pgfqpoint{2.988339in}{2.219887in}}%
\pgfpathlineto{\pgfqpoint{2.979840in}{2.220449in}}%
\pgfpathlineto{\pgfqpoint{2.971326in}{2.221297in}}%
\pgfpathlineto{\pgfqpoint{2.957783in}{2.228699in}}%
\pgfpathlineto{\pgfqpoint{2.944243in}{2.236139in}}%
\pgfpathlineto{\pgfqpoint{2.930707in}{2.243616in}}%
\pgfpathlineto{\pgfqpoint{2.917175in}{2.251131in}}%
\pgfpathlineto{\pgfqpoint{2.925718in}{2.250007in}}%
\pgfpathlineto{\pgfqpoint{2.934246in}{2.249171in}}%
\pgfpathlineto{\pgfqpoint{2.942759in}{2.248618in}}%
\pgfpathlineto{\pgfqpoint{2.951257in}{2.248338in}}%
\pgfpathclose%
\pgfusepath{fill}%
\end{pgfscope}%
\begin{pgfscope}%
\pgfpathrectangle{\pgfqpoint{1.150000in}{0.150000in}}{\pgfqpoint{5.700000in}{5.700000in}}%
\pgfusepath{clip}%
\pgfsetbuttcap%
\pgfsetroundjoin%
\definecolor{currentfill}{rgb}{0.253935,0.265254,0.529983}%
\pgfsetfillcolor{currentfill}%
\pgfsetfillopacity{0.700000}%
\pgfsetlinewidth{0.000000pt}%
\definecolor{currentstroke}{rgb}{0.000000,0.000000,0.000000}%
\pgfsetstrokecolor{currentstroke}%
\pgfsetdash{}{0pt}%
\pgfpathmoveto{\pgfqpoint{2.450275in}{2.530463in}}%
\pgfpathlineto{\pgfqpoint{2.463757in}{2.521252in}}%
\pgfpathlineto{\pgfqpoint{2.477241in}{2.512092in}}%
\pgfpathlineto{\pgfqpoint{2.490727in}{2.502983in}}%
\pgfpathlineto{\pgfqpoint{2.504215in}{2.493923in}}%
\pgfpathlineto{\pgfqpoint{2.495348in}{2.498681in}}%
\pgfpathlineto{\pgfqpoint{2.486459in}{2.503799in}}%
\pgfpathlineto{\pgfqpoint{2.477548in}{2.509286in}}%
\pgfpathlineto{\pgfqpoint{2.468615in}{2.515151in}}%
\pgfpathlineto{\pgfqpoint{2.455090in}{2.524515in}}%
\pgfpathlineto{\pgfqpoint{2.441566in}{2.533929in}}%
\pgfpathlineto{\pgfqpoint{2.428044in}{2.543394in}}%
\pgfpathlineto{\pgfqpoint{2.414523in}{2.552909in}}%
\pgfpathlineto{\pgfqpoint{2.423495in}{2.546734in}}%
\pgfpathlineto{\pgfqpoint{2.432444in}{2.540940in}}%
\pgfpathlineto{\pgfqpoint{2.441370in}{2.535519in}}%
\pgfpathlineto{\pgfqpoint{2.450275in}{2.530463in}}%
\pgfpathclose%
\pgfusepath{fill}%
\end{pgfscope}%
\begin{pgfscope}%
\pgfpathrectangle{\pgfqpoint{1.150000in}{0.150000in}}{\pgfqpoint{5.700000in}{5.700000in}}%
\pgfusepath{clip}%
\pgfsetbuttcap%
\pgfsetroundjoin%
\definecolor{currentfill}{rgb}{0.282884,0.135920,0.453427}%
\pgfsetfillcolor{currentfill}%
\pgfsetfillopacity{0.700000}%
\pgfsetlinewidth{0.000000pt}%
\definecolor{currentstroke}{rgb}{0.000000,0.000000,0.000000}%
\pgfsetstrokecolor{currentstroke}%
\pgfsetdash{}{0pt}%
\pgfpathmoveto{\pgfqpoint{5.217195in}{2.266243in}}%
\pgfpathlineto{\pgfqpoint{5.231204in}{2.265113in}}%
\pgfpathlineto{\pgfqpoint{5.245221in}{2.264007in}}%
\pgfpathlineto{\pgfqpoint{5.259247in}{2.262927in}}%
\pgfpathlineto{\pgfqpoint{5.273281in}{2.261871in}}%
\pgfpathlineto{\pgfqpoint{5.265807in}{2.253750in}}%
\pgfpathlineto{\pgfqpoint{5.258325in}{2.245547in}}%
\pgfpathlineto{\pgfqpoint{5.250836in}{2.237260in}}%
\pgfpathlineto{\pgfqpoint{5.243340in}{2.228891in}}%
\pgfpathlineto{\pgfqpoint{5.229294in}{2.229974in}}%
\pgfpathlineto{\pgfqpoint{5.215256in}{2.231083in}}%
\pgfpathlineto{\pgfqpoint{5.201227in}{2.232216in}}%
\pgfpathlineto{\pgfqpoint{5.187207in}{2.233374in}}%
\pgfpathlineto{\pgfqpoint{5.194714in}{2.241711in}}%
\pgfpathlineto{\pgfqpoint{5.202215in}{2.249968in}}%
\pgfpathlineto{\pgfqpoint{5.209709in}{2.258145in}}%
\pgfpathlineto{\pgfqpoint{5.217195in}{2.266243in}}%
\pgfpathclose%
\pgfusepath{fill}%
\end{pgfscope}%
\begin{pgfscope}%
\pgfpathrectangle{\pgfqpoint{1.150000in}{0.150000in}}{\pgfqpoint{5.700000in}{5.700000in}}%
\pgfusepath{clip}%
\pgfsetbuttcap%
\pgfsetroundjoin%
\definecolor{currentfill}{rgb}{0.279574,0.170599,0.479997}%
\pgfsetfillcolor{currentfill}%
\pgfsetfillopacity{0.700000}%
\pgfsetlinewidth{0.000000pt}%
\definecolor{currentstroke}{rgb}{0.000000,0.000000,0.000000}%
\pgfsetstrokecolor{currentstroke}%
\pgfsetdash{}{0pt}%
\pgfpathmoveto{\pgfqpoint{5.531013in}{2.342172in}}%
\pgfpathlineto{\pgfqpoint{5.545120in}{2.341439in}}%
\pgfpathlineto{\pgfqpoint{5.559236in}{2.340730in}}%
\pgfpathlineto{\pgfqpoint{5.573361in}{2.340046in}}%
\pgfpathlineto{\pgfqpoint{5.587494in}{2.339387in}}%
\pgfpathlineto{\pgfqpoint{5.580164in}{2.332318in}}%
\pgfpathlineto{\pgfqpoint{5.572826in}{2.325157in}}%
\pgfpathlineto{\pgfqpoint{5.565479in}{2.317903in}}%
\pgfpathlineto{\pgfqpoint{5.558125in}{2.310556in}}%
\pgfpathlineto{\pgfqpoint{5.543977in}{2.311202in}}%
\pgfpathlineto{\pgfqpoint{5.529839in}{2.311872in}}%
\pgfpathlineto{\pgfqpoint{5.515709in}{2.312568in}}%
\pgfpathlineto{\pgfqpoint{5.501588in}{2.313287in}}%
\pgfpathlineto{\pgfqpoint{5.508956in}{2.320642in}}%
\pgfpathlineto{\pgfqpoint{5.516317in}{2.327908in}}%
\pgfpathlineto{\pgfqpoint{5.523669in}{2.335084in}}%
\pgfpathlineto{\pgfqpoint{5.531013in}{2.342172in}}%
\pgfpathclose%
\pgfusepath{fill}%
\end{pgfscope}%
\begin{pgfscope}%
\pgfpathrectangle{\pgfqpoint{1.150000in}{0.150000in}}{\pgfqpoint{5.700000in}{5.700000in}}%
\pgfusepath{clip}%
\pgfsetbuttcap%
\pgfsetroundjoin%
\definecolor{currentfill}{rgb}{0.269944,0.014625,0.341379}%
\pgfsetfillcolor{currentfill}%
\pgfsetfillopacity{0.700000}%
\pgfsetlinewidth{0.000000pt}%
\definecolor{currentstroke}{rgb}{0.000000,0.000000,0.000000}%
\pgfsetstrokecolor{currentstroke}%
\pgfsetdash{}{0pt}%
\pgfpathmoveto{\pgfqpoint{4.275809in}{2.050451in}}%
\pgfpathlineto{\pgfqpoint{4.289550in}{2.047360in}}%
\pgfpathlineto{\pgfqpoint{4.303298in}{2.044297in}}%
\pgfpathlineto{\pgfqpoint{4.317052in}{2.041259in}}%
\pgfpathlineto{\pgfqpoint{4.330814in}{2.038248in}}%
\pgfpathlineto{\pgfqpoint{4.322991in}{2.029629in}}%
\pgfpathlineto{\pgfqpoint{4.315163in}{2.021030in}}%
\pgfpathlineto{\pgfqpoint{4.307329in}{2.012456in}}%
\pgfpathlineto{\pgfqpoint{4.299489in}{2.003910in}}%
\pgfpathlineto{\pgfqpoint{4.285716in}{2.007070in}}%
\pgfpathlineto{\pgfqpoint{4.271949in}{2.010257in}}%
\pgfpathlineto{\pgfqpoint{4.258190in}{2.013469in}}%
\pgfpathlineto{\pgfqpoint{4.244437in}{2.016708in}}%
\pgfpathlineto{\pgfqpoint{4.252288in}{2.025100in}}%
\pgfpathlineto{\pgfqpoint{4.260134in}{2.033524in}}%
\pgfpathlineto{\pgfqpoint{4.267974in}{2.041975in}}%
\pgfpathlineto{\pgfqpoint{4.275809in}{2.050451in}}%
\pgfpathclose%
\pgfusepath{fill}%
\end{pgfscope}%
\begin{pgfscope}%
\pgfpathrectangle{\pgfqpoint{1.150000in}{0.150000in}}{\pgfqpoint{5.700000in}{5.700000in}}%
\pgfusepath{clip}%
\pgfsetbuttcap%
\pgfsetroundjoin%
\definecolor{currentfill}{rgb}{0.277018,0.050344,0.375715}%
\pgfsetfillcolor{currentfill}%
\pgfsetfillopacity{0.700000}%
\pgfsetlinewidth{0.000000pt}%
\definecolor{currentstroke}{rgb}{0.000000,0.000000,0.000000}%
\pgfsetstrokecolor{currentstroke}%
\pgfsetdash{}{0pt}%
\pgfpathmoveto{\pgfqpoint{3.342689in}{2.103229in}}%
\pgfpathlineto{\pgfqpoint{3.356241in}{2.097323in}}%
\pgfpathlineto{\pgfqpoint{3.369798in}{2.091450in}}%
\pgfpathlineto{\pgfqpoint{3.383360in}{2.085608in}}%
\pgfpathlineto{\pgfqpoint{3.396927in}{2.079799in}}%
\pgfpathlineto{\pgfqpoint{3.388702in}{2.076463in}}%
\pgfpathlineto{\pgfqpoint{3.380466in}{2.073325in}}%
\pgfpathlineto{\pgfqpoint{3.372220in}{2.070390in}}%
\pgfpathlineto{\pgfqpoint{3.363963in}{2.067665in}}%
\pgfpathlineto{\pgfqpoint{3.350374in}{2.073717in}}%
\pgfpathlineto{\pgfqpoint{3.336790in}{2.079800in}}%
\pgfpathlineto{\pgfqpoint{3.323210in}{2.085916in}}%
\pgfpathlineto{\pgfqpoint{3.309635in}{2.092064in}}%
\pgfpathlineto{\pgfqpoint{3.317915in}{2.094542in}}%
\pgfpathlineto{\pgfqpoint{3.326183in}{2.097232in}}%
\pgfpathlineto{\pgfqpoint{3.334441in}{2.100130in}}%
\pgfpathlineto{\pgfqpoint{3.342689in}{2.103229in}}%
\pgfpathclose%
\pgfusepath{fill}%
\end{pgfscope}%
\begin{pgfscope}%
\pgfpathrectangle{\pgfqpoint{1.150000in}{0.150000in}}{\pgfqpoint{5.700000in}{5.700000in}}%
\pgfusepath{clip}%
\pgfsetbuttcap%
\pgfsetroundjoin%
\definecolor{currentfill}{rgb}{0.280894,0.078907,0.402329}%
\pgfsetfillcolor{currentfill}%
\pgfsetfillopacity{0.700000}%
\pgfsetlinewidth{0.000000pt}%
\definecolor{currentstroke}{rgb}{0.000000,0.000000,0.000000}%
\pgfsetstrokecolor{currentstroke}%
\pgfsetdash{}{0pt}%
\pgfpathmoveto{\pgfqpoint{4.817239in}{2.159085in}}%
\pgfpathlineto{\pgfqpoint{4.831131in}{2.157261in}}%
\pgfpathlineto{\pgfqpoint{4.845030in}{2.155461in}}%
\pgfpathlineto{\pgfqpoint{4.858937in}{2.153687in}}%
\pgfpathlineto{\pgfqpoint{4.872852in}{2.151939in}}%
\pgfpathlineto{\pgfqpoint{4.865216in}{2.142945in}}%
\pgfpathlineto{\pgfqpoint{4.857575in}{2.133898in}}%
\pgfpathlineto{\pgfqpoint{4.849927in}{2.124801in}}%
\pgfpathlineto{\pgfqpoint{4.842273in}{2.115654in}}%
\pgfpathlineto{\pgfqpoint{4.828348in}{2.117486in}}%
\pgfpathlineto{\pgfqpoint{4.814430in}{2.119342in}}%
\pgfpathlineto{\pgfqpoint{4.800520in}{2.121223in}}%
\pgfpathlineto{\pgfqpoint{4.786618in}{2.123130in}}%
\pgfpathlineto{\pgfqpoint{4.794283in}{2.132189in}}%
\pgfpathlineto{\pgfqpoint{4.801941in}{2.141203in}}%
\pgfpathlineto{\pgfqpoint{4.809593in}{2.150168in}}%
\pgfpathlineto{\pgfqpoint{4.817239in}{2.159085in}}%
\pgfpathclose%
\pgfusepath{fill}%
\end{pgfscope}%
\begin{pgfscope}%
\pgfpathrectangle{\pgfqpoint{1.150000in}{0.150000in}}{\pgfqpoint{5.700000in}{5.700000in}}%
\pgfusepath{clip}%
\pgfsetbuttcap%
\pgfsetroundjoin%
\definecolor{currentfill}{rgb}{0.267004,0.004874,0.329415}%
\pgfsetfillcolor{currentfill}%
\pgfsetfillopacity{0.700000}%
\pgfsetlinewidth{0.000000pt}%
\definecolor{currentstroke}{rgb}{0.000000,0.000000,0.000000}%
\pgfsetstrokecolor{currentstroke}%
\pgfsetdash{}{0pt}%
\pgfpathmoveto{\pgfqpoint{4.048302in}{2.026416in}}%
\pgfpathlineto{\pgfqpoint{4.061990in}{2.022694in}}%
\pgfpathlineto{\pgfqpoint{4.075684in}{2.018999in}}%
\pgfpathlineto{\pgfqpoint{4.089384in}{2.015331in}}%
\pgfpathlineto{\pgfqpoint{4.103090in}{2.011690in}}%
\pgfpathlineto{\pgfqpoint{4.095185in}{2.003852in}}%
\pgfpathlineto{\pgfqpoint{4.087274in}{1.996076in}}%
\pgfpathlineto{\pgfqpoint{4.079358in}{1.988364in}}%
\pgfpathlineto{\pgfqpoint{4.071435in}{1.980723in}}%
\pgfpathlineto{\pgfqpoint{4.057715in}{1.984539in}}%
\pgfpathlineto{\pgfqpoint{4.044001in}{1.988382in}}%
\pgfpathlineto{\pgfqpoint{4.030294in}{1.992252in}}%
\pgfpathlineto{\pgfqpoint{4.016593in}{1.996149in}}%
\pgfpathlineto{\pgfqpoint{4.024530in}{2.003610in}}%
\pgfpathlineto{\pgfqpoint{4.032460in}{2.011145in}}%
\pgfpathlineto{\pgfqpoint{4.040384in}{2.018748in}}%
\pgfpathlineto{\pgfqpoint{4.048302in}{2.026416in}}%
\pgfpathclose%
\pgfusepath{fill}%
\end{pgfscope}%
\begin{pgfscope}%
\pgfpathrectangle{\pgfqpoint{1.150000in}{0.150000in}}{\pgfqpoint{5.700000in}{5.700000in}}%
\pgfusepath{clip}%
\pgfsetbuttcap%
\pgfsetroundjoin%
\definecolor{currentfill}{rgb}{0.274952,0.037752,0.364543}%
\pgfsetfillcolor{currentfill}%
\pgfsetfillopacity{0.700000}%
\pgfsetlinewidth{0.000000pt}%
\definecolor{currentstroke}{rgb}{0.000000,0.000000,0.000000}%
\pgfsetstrokecolor{currentstroke}%
\pgfsetdash{}{0pt}%
\pgfpathmoveto{\pgfqpoint{4.503366in}{2.086487in}}%
\pgfpathlineto{\pgfqpoint{4.517169in}{2.083967in}}%
\pgfpathlineto{\pgfqpoint{4.530979in}{2.081474in}}%
\pgfpathlineto{\pgfqpoint{4.544797in}{2.079006in}}%
\pgfpathlineto{\pgfqpoint{4.558622in}{2.076564in}}%
\pgfpathlineto{\pgfqpoint{4.550875in}{2.067538in}}%
\pgfpathlineto{\pgfqpoint{4.543123in}{2.058499in}}%
\pgfpathlineto{\pgfqpoint{4.535366in}{2.049449in}}%
\pgfpathlineto{\pgfqpoint{4.527603in}{2.040391in}}%
\pgfpathlineto{\pgfqpoint{4.513768in}{2.042955in}}%
\pgfpathlineto{\pgfqpoint{4.499940in}{2.045546in}}%
\pgfpathlineto{\pgfqpoint{4.486119in}{2.048162in}}%
\pgfpathlineto{\pgfqpoint{4.472305in}{2.050804in}}%
\pgfpathlineto{\pgfqpoint{4.480079in}{2.059734in}}%
\pgfpathlineto{\pgfqpoint{4.487847in}{2.068660in}}%
\pgfpathlineto{\pgfqpoint{4.495609in}{2.077578in}}%
\pgfpathlineto{\pgfqpoint{4.503366in}{2.086487in}}%
\pgfpathclose%
\pgfusepath{fill}%
\end{pgfscope}%
\begin{pgfscope}%
\pgfpathrectangle{\pgfqpoint{1.150000in}{0.150000in}}{\pgfqpoint{5.700000in}{5.700000in}}%
\pgfusepath{clip}%
\pgfsetbuttcap%
\pgfsetroundjoin%
\definecolor{currentfill}{rgb}{0.278012,0.180367,0.486697}%
\pgfsetfillcolor{currentfill}%
\pgfsetfillopacity{0.700000}%
\pgfsetlinewidth{0.000000pt}%
\definecolor{currentstroke}{rgb}{0.000000,0.000000,0.000000}%
\pgfsetstrokecolor{currentstroke}%
\pgfsetdash{}{0pt}%
\pgfpathmoveto{\pgfqpoint{2.755035in}{2.344390in}}%
\pgfpathlineto{\pgfqpoint{2.768530in}{2.336394in}}%
\pgfpathlineto{\pgfqpoint{2.782028in}{2.328440in}}%
\pgfpathlineto{\pgfqpoint{2.795528in}{2.320528in}}%
\pgfpathlineto{\pgfqpoint{2.809032in}{2.312657in}}%
\pgfpathlineto{\pgfqpoint{2.800411in}{2.314639in}}%
\pgfpathlineto{\pgfqpoint{2.791773in}{2.316932in}}%
\pgfpathlineto{\pgfqpoint{2.783117in}{2.319544in}}%
\pgfpathlineto{\pgfqpoint{2.774443in}{2.322482in}}%
\pgfpathlineto{\pgfqpoint{2.760907in}{2.330639in}}%
\pgfpathlineto{\pgfqpoint{2.747374in}{2.338838in}}%
\pgfpathlineto{\pgfqpoint{2.733844in}{2.347079in}}%
\pgfpathlineto{\pgfqpoint{2.720317in}{2.355362in}}%
\pgfpathlineto{\pgfqpoint{2.729024in}{2.352131in}}%
\pgfpathlineto{\pgfqpoint{2.737712in}{2.349231in}}%
\pgfpathlineto{\pgfqpoint{2.746382in}{2.346653in}}%
\pgfpathlineto{\pgfqpoint{2.755035in}{2.344390in}}%
\pgfpathclose%
\pgfusepath{fill}%
\end{pgfscope}%
\begin{pgfscope}%
\pgfpathrectangle{\pgfqpoint{1.150000in}{0.150000in}}{\pgfqpoint{5.700000in}{5.700000in}}%
\pgfusepath{clip}%
\pgfsetbuttcap%
\pgfsetroundjoin%
\definecolor{currentfill}{rgb}{0.283229,0.120777,0.440584}%
\pgfsetfillcolor{currentfill}%
\pgfsetfillopacity{0.700000}%
\pgfsetlinewidth{0.000000pt}%
\definecolor{currentstroke}{rgb}{0.000000,0.000000,0.000000}%
\pgfsetstrokecolor{currentstroke}%
\pgfsetdash{}{0pt}%
\pgfpathmoveto{\pgfqpoint{5.131208in}{2.238256in}}%
\pgfpathlineto{\pgfqpoint{5.145195in}{2.236998in}}%
\pgfpathlineto{\pgfqpoint{5.159191in}{2.235765in}}%
\pgfpathlineto{\pgfqpoint{5.173195in}{2.234557in}}%
\pgfpathlineto{\pgfqpoint{5.187207in}{2.233374in}}%
\pgfpathlineto{\pgfqpoint{5.179692in}{2.224959in}}%
\pgfpathlineto{\pgfqpoint{5.172171in}{2.216465in}}%
\pgfpathlineto{\pgfqpoint{5.164642in}{2.207893in}}%
\pgfpathlineto{\pgfqpoint{5.157107in}{2.199243in}}%
\pgfpathlineto{\pgfqpoint{5.143083in}{2.200468in}}%
\pgfpathlineto{\pgfqpoint{5.129068in}{2.201717in}}%
\pgfpathlineto{\pgfqpoint{5.115062in}{2.202992in}}%
\pgfpathlineto{\pgfqpoint{5.101063in}{2.204291in}}%
\pgfpathlineto{\pgfqpoint{5.108609in}{2.212894in}}%
\pgfpathlineto{\pgfqpoint{5.116149in}{2.221423in}}%
\pgfpathlineto{\pgfqpoint{5.123682in}{2.229877in}}%
\pgfpathlineto{\pgfqpoint{5.131208in}{2.238256in}}%
\pgfpathclose%
\pgfusepath{fill}%
\end{pgfscope}%
\begin{pgfscope}%
\pgfpathrectangle{\pgfqpoint{1.150000in}{0.150000in}}{\pgfqpoint{5.700000in}{5.700000in}}%
\pgfusepath{clip}%
\pgfsetbuttcap%
\pgfsetroundjoin%
\definecolor{currentfill}{rgb}{0.274128,0.199721,0.498911}%
\pgfsetfillcolor{currentfill}%
\pgfsetfillopacity{0.700000}%
\pgfsetlinewidth{0.000000pt}%
\definecolor{currentstroke}{rgb}{0.000000,0.000000,0.000000}%
\pgfsetstrokecolor{currentstroke}%
\pgfsetdash{}{0pt}%
\pgfpathmoveto{\pgfqpoint{5.758992in}{2.387854in}}%
\pgfpathlineto{\pgfqpoint{5.773175in}{2.387348in}}%
\pgfpathlineto{\pgfqpoint{5.787368in}{2.386866in}}%
\pgfpathlineto{\pgfqpoint{5.801570in}{2.386408in}}%
\pgfpathlineto{\pgfqpoint{5.815781in}{2.385975in}}%
\pgfpathlineto{\pgfqpoint{5.808563in}{2.379732in}}%
\pgfpathlineto{\pgfqpoint{5.801337in}{2.373398in}}%
\pgfpathlineto{\pgfqpoint{5.794102in}{2.366971in}}%
\pgfpathlineto{\pgfqpoint{5.786858in}{2.360450in}}%
\pgfpathlineto{\pgfqpoint{5.772631in}{2.360842in}}%
\pgfpathlineto{\pgfqpoint{5.758414in}{2.361258in}}%
\pgfpathlineto{\pgfqpoint{5.744205in}{2.361699in}}%
\pgfpathlineto{\pgfqpoint{5.730006in}{2.362164in}}%
\pgfpathlineto{\pgfqpoint{5.737265in}{2.368721in}}%
\pgfpathlineto{\pgfqpoint{5.744516in}{2.375187in}}%
\pgfpathlineto{\pgfqpoint{5.751758in}{2.381564in}}%
\pgfpathlineto{\pgfqpoint{5.758992in}{2.387854in}}%
\pgfpathclose%
\pgfusepath{fill}%
\end{pgfscope}%
\begin{pgfscope}%
\pgfpathrectangle{\pgfqpoint{1.150000in}{0.150000in}}{\pgfqpoint{5.700000in}{5.700000in}}%
\pgfusepath{clip}%
\pgfsetbuttcap%
\pgfsetroundjoin%
\definecolor{currentfill}{rgb}{0.280868,0.160771,0.472899}%
\pgfsetfillcolor{currentfill}%
\pgfsetfillopacity{0.700000}%
\pgfsetlinewidth{0.000000pt}%
\definecolor{currentstroke}{rgb}{0.000000,0.000000,0.000000}%
\pgfsetstrokecolor{currentstroke}%
\pgfsetdash{}{0pt}%
\pgfpathmoveto{\pgfqpoint{5.445192in}{2.316413in}}%
\pgfpathlineto{\pgfqpoint{5.459278in}{2.315594in}}%
\pgfpathlineto{\pgfqpoint{5.473373in}{2.314801in}}%
\pgfpathlineto{\pgfqpoint{5.487476in}{2.314032in}}%
\pgfpathlineto{\pgfqpoint{5.501588in}{2.313287in}}%
\pgfpathlineto{\pgfqpoint{5.494212in}{2.305841in}}%
\pgfpathlineto{\pgfqpoint{5.486828in}{2.298304in}}%
\pgfpathlineto{\pgfqpoint{5.479436in}{2.290674in}}%
\pgfpathlineto{\pgfqpoint{5.472037in}{2.282953in}}%
\pgfpathlineto{\pgfqpoint{5.457911in}{2.283697in}}%
\pgfpathlineto{\pgfqpoint{5.443795in}{2.284467in}}%
\pgfpathlineto{\pgfqpoint{5.429687in}{2.285261in}}%
\pgfpathlineto{\pgfqpoint{5.415588in}{2.286080in}}%
\pgfpathlineto{\pgfqpoint{5.423001in}{2.293796in}}%
\pgfpathlineto{\pgfqpoint{5.430406in}{2.301423in}}%
\pgfpathlineto{\pgfqpoint{5.437803in}{2.308962in}}%
\pgfpathlineto{\pgfqpoint{5.445192in}{2.316413in}}%
\pgfpathclose%
\pgfusepath{fill}%
\end{pgfscope}%
\begin{pgfscope}%
\pgfpathrectangle{\pgfqpoint{1.150000in}{0.150000in}}{\pgfqpoint{5.700000in}{5.700000in}}%
\pgfusepath{clip}%
\pgfsetbuttcap%
\pgfsetroundjoin%
\definecolor{currentfill}{rgb}{0.258965,0.251537,0.524736}%
\pgfsetfillcolor{currentfill}%
\pgfsetfillopacity{0.700000}%
\pgfsetlinewidth{0.000000pt}%
\definecolor{currentstroke}{rgb}{0.000000,0.000000,0.000000}%
\pgfsetstrokecolor{currentstroke}%
\pgfsetdash{}{0pt}%
\pgfpathmoveto{\pgfqpoint{2.504215in}{2.493923in}}%
\pgfpathlineto{\pgfqpoint{2.517705in}{2.484913in}}%
\pgfpathlineto{\pgfqpoint{2.531196in}{2.475952in}}%
\pgfpathlineto{\pgfqpoint{2.544690in}{2.467038in}}%
\pgfpathlineto{\pgfqpoint{2.558186in}{2.458173in}}%
\pgfpathlineto{\pgfqpoint{2.549355in}{2.462632in}}%
\pgfpathlineto{\pgfqpoint{2.540503in}{2.467449in}}%
\pgfpathlineto{\pgfqpoint{2.531630in}{2.472630in}}%
\pgfpathlineto{\pgfqpoint{2.522735in}{2.478185in}}%
\pgfpathlineto{\pgfqpoint{2.509202in}{2.487354in}}%
\pgfpathlineto{\pgfqpoint{2.495671in}{2.496571in}}%
\pgfpathlineto{\pgfqpoint{2.482142in}{2.505837in}}%
\pgfpathlineto{\pgfqpoint{2.468615in}{2.515151in}}%
\pgfpathlineto{\pgfqpoint{2.477548in}{2.509286in}}%
\pgfpathlineto{\pgfqpoint{2.486459in}{2.503799in}}%
\pgfpathlineto{\pgfqpoint{2.495348in}{2.498681in}}%
\pgfpathlineto{\pgfqpoint{2.504215in}{2.493923in}}%
\pgfpathclose%
\pgfusepath{fill}%
\end{pgfscope}%
\begin{pgfscope}%
\pgfpathrectangle{\pgfqpoint{1.150000in}{0.150000in}}{\pgfqpoint{5.700000in}{5.700000in}}%
\pgfusepath{clip}%
\pgfsetbuttcap%
\pgfsetroundjoin%
\definecolor{currentfill}{rgb}{0.279566,0.067836,0.391917}%
\pgfsetfillcolor{currentfill}%
\pgfsetfillopacity{0.700000}%
\pgfsetlinewidth{0.000000pt}%
\definecolor{currentstroke}{rgb}{0.000000,0.000000,0.000000}%
\pgfsetstrokecolor{currentstroke}%
\pgfsetdash{}{0pt}%
\pgfpathmoveto{\pgfqpoint{4.731088in}{2.131010in}}%
\pgfpathlineto{\pgfqpoint{4.744959in}{2.129002in}}%
\pgfpathlineto{\pgfqpoint{4.758838in}{2.127019in}}%
\pgfpathlineto{\pgfqpoint{4.772724in}{2.125062in}}%
\pgfpathlineto{\pgfqpoint{4.786618in}{2.123130in}}%
\pgfpathlineto{\pgfqpoint{4.778949in}{2.114027in}}%
\pgfpathlineto{\pgfqpoint{4.771273in}{2.104882in}}%
\pgfpathlineto{\pgfqpoint{4.763591in}{2.095697in}}%
\pgfpathlineto{\pgfqpoint{4.755904in}{2.086473in}}%
\pgfpathlineto{\pgfqpoint{4.741999in}{2.088501in}}%
\pgfpathlineto{\pgfqpoint{4.728103in}{2.090554in}}%
\pgfpathlineto{\pgfqpoint{4.714214in}{2.092633in}}%
\pgfpathlineto{\pgfqpoint{4.700332in}{2.094736in}}%
\pgfpathlineto{\pgfqpoint{4.708030in}{2.103859in}}%
\pgfpathlineto{\pgfqpoint{4.715722in}{2.112946in}}%
\pgfpathlineto{\pgfqpoint{4.723408in}{2.121998in}}%
\pgfpathlineto{\pgfqpoint{4.731088in}{2.131010in}}%
\pgfpathclose%
\pgfusepath{fill}%
\end{pgfscope}%
\begin{pgfscope}%
\pgfpathrectangle{\pgfqpoint{1.150000in}{0.150000in}}{\pgfqpoint{5.700000in}{5.700000in}}%
\pgfusepath{clip}%
\pgfsetbuttcap%
\pgfsetroundjoin%
\definecolor{currentfill}{rgb}{0.268510,0.009605,0.335427}%
\pgfsetfillcolor{currentfill}%
\pgfsetfillopacity{0.700000}%
\pgfsetlinewidth{0.000000pt}%
\definecolor{currentstroke}{rgb}{0.000000,0.000000,0.000000}%
\pgfsetstrokecolor{currentstroke}%
\pgfsetdash{}{0pt}%
\pgfpathmoveto{\pgfqpoint{3.679501in}{2.031219in}}%
\pgfpathlineto{\pgfqpoint{3.693115in}{2.026370in}}%
\pgfpathlineto{\pgfqpoint{3.706736in}{2.021550in}}%
\pgfpathlineto{\pgfqpoint{3.720362in}{2.016758in}}%
\pgfpathlineto{\pgfqpoint{3.733993in}{2.011996in}}%
\pgfpathlineto{\pgfqpoint{3.725934in}{2.006242in}}%
\pgfpathlineto{\pgfqpoint{3.717867in}{2.000623in}}%
\pgfpathlineto{\pgfqpoint{3.709793in}{1.995144in}}%
\pgfpathlineto{\pgfqpoint{3.701711in}{1.989810in}}%
\pgfpathlineto{\pgfqpoint{3.688062in}{1.994787in}}%
\pgfpathlineto{\pgfqpoint{3.674418in}{1.999794in}}%
\pgfpathlineto{\pgfqpoint{3.660780in}{2.004829in}}%
\pgfpathlineto{\pgfqpoint{3.647147in}{2.009893in}}%
\pgfpathlineto{\pgfqpoint{3.655247in}{2.015007in}}%
\pgfpathlineto{\pgfqpoint{3.663340in}{2.020269in}}%
\pgfpathlineto{\pgfqpoint{3.671424in}{2.025675in}}%
\pgfpathlineto{\pgfqpoint{3.679501in}{2.031219in}}%
\pgfpathclose%
\pgfusepath{fill}%
\end{pgfscope}%
\begin{pgfscope}%
\pgfpathrectangle{\pgfqpoint{1.150000in}{0.150000in}}{\pgfqpoint{5.700000in}{5.700000in}}%
\pgfusepath{clip}%
\pgfsetbuttcap%
\pgfsetroundjoin%
\definecolor{currentfill}{rgb}{0.283229,0.120777,0.440584}%
\pgfsetfillcolor{currentfill}%
\pgfsetfillopacity{0.700000}%
\pgfsetlinewidth{0.000000pt}%
\definecolor{currentstroke}{rgb}{0.000000,0.000000,0.000000}%
\pgfsetstrokecolor{currentstroke}%
\pgfsetdash{}{0pt}%
\pgfpathmoveto{\pgfqpoint{3.005293in}{2.219589in}}%
\pgfpathlineto{\pgfqpoint{3.018811in}{2.212495in}}%
\pgfpathlineto{\pgfqpoint{3.032333in}{2.205438in}}%
\pgfpathlineto{\pgfqpoint{3.045859in}{2.198418in}}%
\pgfpathlineto{\pgfqpoint{3.059388in}{2.191433in}}%
\pgfpathlineto{\pgfqpoint{3.050946in}{2.191181in}}%
\pgfpathlineto{\pgfqpoint{3.042489in}{2.191196in}}%
\pgfpathlineto{\pgfqpoint{3.034019in}{2.191486in}}%
\pgfpathlineto{\pgfqpoint{3.025533in}{2.192057in}}%
\pgfpathlineto{\pgfqpoint{3.011976in}{2.199312in}}%
\pgfpathlineto{\pgfqpoint{2.998422in}{2.206604in}}%
\pgfpathlineto{\pgfqpoint{2.984872in}{2.213932in}}%
\pgfpathlineto{\pgfqpoint{2.971326in}{2.221297in}}%
\pgfpathlineto{\pgfqpoint{2.979840in}{2.220449in}}%
\pgfpathlineto{\pgfqpoint{2.988339in}{2.219887in}}%
\pgfpathlineto{\pgfqpoint{2.996823in}{2.219603in}}%
\pgfpathlineto{\pgfqpoint{3.005293in}{2.219589in}}%
\pgfpathclose%
\pgfusepath{fill}%
\end{pgfscope}%
\begin{pgfscope}%
\pgfpathrectangle{\pgfqpoint{1.150000in}{0.150000in}}{\pgfqpoint{5.700000in}{5.700000in}}%
\pgfusepath{clip}%
\pgfsetbuttcap%
\pgfsetroundjoin%
\definecolor{currentfill}{rgb}{0.268510,0.009605,0.335427}%
\pgfsetfillcolor{currentfill}%
\pgfsetfillopacity{0.700000}%
\pgfsetlinewidth{0.000000pt}%
\definecolor{currentstroke}{rgb}{0.000000,0.000000,0.000000}%
\pgfsetstrokecolor{currentstroke}%
\pgfsetdash{}{0pt}%
\pgfpathmoveto{\pgfqpoint{4.189491in}{2.029928in}}%
\pgfpathlineto{\pgfqpoint{4.203218in}{2.026583in}}%
\pgfpathlineto{\pgfqpoint{4.216951in}{2.023265in}}%
\pgfpathlineto{\pgfqpoint{4.230690in}{2.019973in}}%
\pgfpathlineto{\pgfqpoint{4.244437in}{2.016708in}}%
\pgfpathlineto{\pgfqpoint{4.236580in}{2.008351in}}%
\pgfpathlineto{\pgfqpoint{4.228717in}{2.000033in}}%
\pgfpathlineto{\pgfqpoint{4.220849in}{1.991757in}}%
\pgfpathlineto{\pgfqpoint{4.212975in}{1.983527in}}%
\pgfpathlineto{\pgfqpoint{4.199217in}{1.986955in}}%
\pgfpathlineto{\pgfqpoint{4.185465in}{1.990409in}}%
\pgfpathlineto{\pgfqpoint{4.171719in}{1.993889in}}%
\pgfpathlineto{\pgfqpoint{4.157981in}{1.997396in}}%
\pgfpathlineto{\pgfqpoint{4.165867in}{2.005458in}}%
\pgfpathlineto{\pgfqpoint{4.173747in}{2.013570in}}%
\pgfpathlineto{\pgfqpoint{4.181622in}{2.021728in}}%
\pgfpathlineto{\pgfqpoint{4.189491in}{2.029928in}}%
\pgfpathclose%
\pgfusepath{fill}%
\end{pgfscope}%
\begin{pgfscope}%
\pgfpathrectangle{\pgfqpoint{1.150000in}{0.150000in}}{\pgfqpoint{5.700000in}{5.700000in}}%
\pgfusepath{clip}%
\pgfsetbuttcap%
\pgfsetroundjoin%
\definecolor{currentfill}{rgb}{0.280894,0.078907,0.402329}%
\pgfsetfillcolor{currentfill}%
\pgfsetfillopacity{0.700000}%
\pgfsetlinewidth{0.000000pt}%
\definecolor{currentstroke}{rgb}{0.000000,0.000000,0.000000}%
\pgfsetstrokecolor{currentstroke}%
\pgfsetdash{}{0pt}%
\pgfpathmoveto{\pgfqpoint{3.201193in}{2.142431in}}%
\pgfpathlineto{\pgfqpoint{3.214733in}{2.136018in}}%
\pgfpathlineto{\pgfqpoint{3.228277in}{2.129639in}}%
\pgfpathlineto{\pgfqpoint{3.241826in}{2.123294in}}%
\pgfpathlineto{\pgfqpoint{3.255379in}{2.116982in}}%
\pgfpathlineto{\pgfqpoint{3.247064in}{2.114975in}}%
\pgfpathlineto{\pgfqpoint{3.238738in}{2.113198in}}%
\pgfpathlineto{\pgfqpoint{3.230400in}{2.111657in}}%
\pgfpathlineto{\pgfqpoint{3.222049in}{2.110359in}}%
\pgfpathlineto{\pgfqpoint{3.208471in}{2.116927in}}%
\pgfpathlineto{\pgfqpoint{3.194898in}{2.123529in}}%
\pgfpathlineto{\pgfqpoint{3.181329in}{2.130164in}}%
\pgfpathlineto{\pgfqpoint{3.167764in}{2.136833in}}%
\pgfpathlineto{\pgfqpoint{3.176140in}{2.137870in}}%
\pgfpathlineto{\pgfqpoint{3.184503in}{2.139153in}}%
\pgfpathlineto{\pgfqpoint{3.192854in}{2.140675in}}%
\pgfpathlineto{\pgfqpoint{3.201193in}{2.142431in}}%
\pgfpathclose%
\pgfusepath{fill}%
\end{pgfscope}%
\begin{pgfscope}%
\pgfpathrectangle{\pgfqpoint{1.150000in}{0.150000in}}{\pgfqpoint{5.700000in}{5.700000in}}%
\pgfusepath{clip}%
\pgfsetbuttcap%
\pgfsetroundjoin%
\definecolor{currentfill}{rgb}{0.283091,0.110553,0.431554}%
\pgfsetfillcolor{currentfill}%
\pgfsetfillopacity{0.700000}%
\pgfsetlinewidth{0.000000pt}%
\definecolor{currentstroke}{rgb}{0.000000,0.000000,0.000000}%
\pgfsetstrokecolor{currentstroke}%
\pgfsetdash{}{0pt}%
\pgfpathmoveto{\pgfqpoint{5.045151in}{2.209738in}}%
\pgfpathlineto{\pgfqpoint{5.059117in}{2.208339in}}%
\pgfpathlineto{\pgfqpoint{5.073091in}{2.206964in}}%
\pgfpathlineto{\pgfqpoint{5.087073in}{2.205615in}}%
\pgfpathlineto{\pgfqpoint{5.101063in}{2.204291in}}%
\pgfpathlineto{\pgfqpoint{5.093510in}{2.195615in}}%
\pgfpathlineto{\pgfqpoint{5.085950in}{2.186866in}}%
\pgfpathlineto{\pgfqpoint{5.078384in}{2.178046in}}%
\pgfpathlineto{\pgfqpoint{5.070811in}{2.169155in}}%
\pgfpathlineto{\pgfqpoint{5.056810in}{2.170535in}}%
\pgfpathlineto{\pgfqpoint{5.042817in}{2.171939in}}%
\pgfpathlineto{\pgfqpoint{5.028833in}{2.173368in}}%
\pgfpathlineto{\pgfqpoint{5.014856in}{2.174823in}}%
\pgfpathlineto{\pgfqpoint{5.022440in}{2.183653in}}%
\pgfpathlineto{\pgfqpoint{5.030017in}{2.192417in}}%
\pgfpathlineto{\pgfqpoint{5.037587in}{2.201112in}}%
\pgfpathlineto{\pgfqpoint{5.045151in}{2.209738in}}%
\pgfpathclose%
\pgfusepath{fill}%
\end{pgfscope}%
\begin{pgfscope}%
\pgfpathrectangle{\pgfqpoint{1.150000in}{0.150000in}}{\pgfqpoint{5.700000in}{5.700000in}}%
\pgfusepath{clip}%
\pgfsetbuttcap%
\pgfsetroundjoin%
\definecolor{currentfill}{rgb}{0.272594,0.025563,0.353093}%
\pgfsetfillcolor{currentfill}%
\pgfsetfillopacity{0.700000}%
\pgfsetlinewidth{0.000000pt}%
\definecolor{currentstroke}{rgb}{0.000000,0.000000,0.000000}%
\pgfsetstrokecolor{currentstroke}%
\pgfsetdash{}{0pt}%
\pgfpathmoveto{\pgfqpoint{4.417122in}{2.061629in}}%
\pgfpathlineto{\pgfqpoint{4.430907in}{2.058884in}}%
\pgfpathlineto{\pgfqpoint{4.444699in}{2.056164in}}%
\pgfpathlineto{\pgfqpoint{4.458499in}{2.053471in}}%
\pgfpathlineto{\pgfqpoint{4.472305in}{2.050804in}}%
\pgfpathlineto{\pgfqpoint{4.464526in}{2.041872in}}%
\pgfpathlineto{\pgfqpoint{4.456742in}{2.032941in}}%
\pgfpathlineto{\pgfqpoint{4.448952in}{2.024015in}}%
\pgfpathlineto{\pgfqpoint{4.441157in}{2.015096in}}%
\pgfpathlineto{\pgfqpoint{4.427340in}{2.017900in}}%
\pgfpathlineto{\pgfqpoint{4.413529in}{2.020729in}}%
\pgfpathlineto{\pgfqpoint{4.399726in}{2.023584in}}%
\pgfpathlineto{\pgfqpoint{4.385930in}{2.026465in}}%
\pgfpathlineto{\pgfqpoint{4.393736in}{2.035242in}}%
\pgfpathlineto{\pgfqpoint{4.401537in}{2.044031in}}%
\pgfpathlineto{\pgfqpoint{4.409332in}{2.052828in}}%
\pgfpathlineto{\pgfqpoint{4.417122in}{2.061629in}}%
\pgfpathclose%
\pgfusepath{fill}%
\end{pgfscope}%
\begin{pgfscope}%
\pgfpathrectangle{\pgfqpoint{1.150000in}{0.150000in}}{\pgfqpoint{5.700000in}{5.700000in}}%
\pgfusepath{clip}%
\pgfsetbuttcap%
\pgfsetroundjoin%
\definecolor{currentfill}{rgb}{0.267004,0.004874,0.329415}%
\pgfsetfillcolor{currentfill}%
\pgfsetfillopacity{0.700000}%
\pgfsetlinewidth{0.000000pt}%
\definecolor{currentstroke}{rgb}{0.000000,0.000000,0.000000}%
\pgfsetstrokecolor{currentstroke}%
\pgfsetdash{}{0pt}%
\pgfpathmoveto{\pgfqpoint{3.820672in}{2.018299in}}%
\pgfpathlineto{\pgfqpoint{3.834316in}{2.013881in}}%
\pgfpathlineto{\pgfqpoint{3.847965in}{2.009491in}}%
\pgfpathlineto{\pgfqpoint{3.861620in}{2.005129in}}%
\pgfpathlineto{\pgfqpoint{3.875281in}{2.000796in}}%
\pgfpathlineto{\pgfqpoint{3.867283in}{1.994161in}}%
\pgfpathlineto{\pgfqpoint{3.859279in}{1.987635in}}%
\pgfpathlineto{\pgfqpoint{3.851267in}{1.981220in}}%
\pgfpathlineto{\pgfqpoint{3.843248in}{1.974924in}}%
\pgfpathlineto{\pgfqpoint{3.829571in}{1.979459in}}%
\pgfpathlineto{\pgfqpoint{3.815900in}{1.984022in}}%
\pgfpathlineto{\pgfqpoint{3.802235in}{1.988614in}}%
\pgfpathlineto{\pgfqpoint{3.788575in}{1.993233in}}%
\pgfpathlineto{\pgfqpoint{3.796610in}{1.999323in}}%
\pgfpathlineto{\pgfqpoint{3.804638in}{2.005534in}}%
\pgfpathlineto{\pgfqpoint{3.812659in}{2.011861in}}%
\pgfpathlineto{\pgfqpoint{3.820672in}{2.018299in}}%
\pgfpathclose%
\pgfusepath{fill}%
\end{pgfscope}%
\begin{pgfscope}%
\pgfpathrectangle{\pgfqpoint{1.150000in}{0.150000in}}{\pgfqpoint{5.700000in}{5.700000in}}%
\pgfusepath{clip}%
\pgfsetbuttcap%
\pgfsetroundjoin%
\definecolor{currentfill}{rgb}{0.272594,0.025563,0.353093}%
\pgfsetfillcolor{currentfill}%
\pgfsetfillopacity{0.700000}%
\pgfsetlinewidth{0.000000pt}%
\definecolor{currentstroke}{rgb}{0.000000,0.000000,0.000000}%
\pgfsetstrokecolor{currentstroke}%
\pgfsetdash{}{0pt}%
\pgfpathmoveto{\pgfqpoint{3.538275in}{2.051472in}}%
\pgfpathlineto{\pgfqpoint{3.551866in}{2.046170in}}%
\pgfpathlineto{\pgfqpoint{3.565462in}{2.040898in}}%
\pgfpathlineto{\pgfqpoint{3.579063in}{2.035656in}}%
\pgfpathlineto{\pgfqpoint{3.592669in}{2.030445in}}%
\pgfpathlineto{\pgfqpoint{3.584542in}{2.025709in}}%
\pgfpathlineto{\pgfqpoint{3.576406in}{2.021137in}}%
\pgfpathlineto{\pgfqpoint{3.568261in}{2.016734in}}%
\pgfpathlineto{\pgfqpoint{3.560107in}{2.012506in}}%
\pgfpathlineto{\pgfqpoint{3.546481in}{2.017946in}}%
\pgfpathlineto{\pgfqpoint{3.532861in}{2.023416in}}%
\pgfpathlineto{\pgfqpoint{3.519245in}{2.028917in}}%
\pgfpathlineto{\pgfqpoint{3.505634in}{2.034447in}}%
\pgfpathlineto{\pgfqpoint{3.513808in}{2.038442in}}%
\pgfpathlineto{\pgfqpoint{3.521973in}{2.042615in}}%
\pgfpathlineto{\pgfqpoint{3.530128in}{2.046960in}}%
\pgfpathlineto{\pgfqpoint{3.538275in}{2.051472in}}%
\pgfpathclose%
\pgfusepath{fill}%
\end{pgfscope}%
\begin{pgfscope}%
\pgfpathrectangle{\pgfqpoint{1.150000in}{0.150000in}}{\pgfqpoint{5.700000in}{5.700000in}}%
\pgfusepath{clip}%
\pgfsetbuttcap%
\pgfsetroundjoin%
\definecolor{currentfill}{rgb}{0.281887,0.150881,0.465405}%
\pgfsetfillcolor{currentfill}%
\pgfsetfillopacity{0.700000}%
\pgfsetlinewidth{0.000000pt}%
\definecolor{currentstroke}{rgb}{0.000000,0.000000,0.000000}%
\pgfsetstrokecolor{currentstroke}%
\pgfsetdash{}{0pt}%
\pgfpathmoveto{\pgfqpoint{5.359279in}{2.289602in}}%
\pgfpathlineto{\pgfqpoint{5.373343in}{2.288684in}}%
\pgfpathlineto{\pgfqpoint{5.387416in}{2.287791in}}%
\pgfpathlineto{\pgfqpoint{5.401498in}{2.286923in}}%
\pgfpathlineto{\pgfqpoint{5.415588in}{2.286080in}}%
\pgfpathlineto{\pgfqpoint{5.408168in}{2.278274in}}%
\pgfpathlineto{\pgfqpoint{5.400740in}{2.270378in}}%
\pgfpathlineto{\pgfqpoint{5.393305in}{2.262392in}}%
\pgfpathlineto{\pgfqpoint{5.385861in}{2.254317in}}%
\pgfpathlineto{\pgfqpoint{5.371759in}{2.255174in}}%
\pgfpathlineto{\pgfqpoint{5.357665in}{2.256057in}}%
\pgfpathlineto{\pgfqpoint{5.343579in}{2.256964in}}%
\pgfpathlineto{\pgfqpoint{5.329503in}{2.257896in}}%
\pgfpathlineto{\pgfqpoint{5.336958in}{2.265952in}}%
\pgfpathlineto{\pgfqpoint{5.344406in}{2.273922in}}%
\pgfpathlineto{\pgfqpoint{5.351846in}{2.281805in}}%
\pgfpathlineto{\pgfqpoint{5.359279in}{2.289602in}}%
\pgfpathclose%
\pgfusepath{fill}%
\end{pgfscope}%
\begin{pgfscope}%
\pgfpathrectangle{\pgfqpoint{1.150000in}{0.150000in}}{\pgfqpoint{5.700000in}{5.700000in}}%
\pgfusepath{clip}%
\pgfsetbuttcap%
\pgfsetroundjoin%
\definecolor{currentfill}{rgb}{0.276194,0.190074,0.493001}%
\pgfsetfillcolor{currentfill}%
\pgfsetfillopacity{0.700000}%
\pgfsetlinewidth{0.000000pt}%
\definecolor{currentstroke}{rgb}{0.000000,0.000000,0.000000}%
\pgfsetstrokecolor{currentstroke}%
\pgfsetdash{}{0pt}%
\pgfpathmoveto{\pgfqpoint{5.673298in}{2.364271in}}%
\pgfpathlineto{\pgfqpoint{5.687462in}{2.363707in}}%
\pgfpathlineto{\pgfqpoint{5.701634in}{2.363168in}}%
\pgfpathlineto{\pgfqpoint{5.715815in}{2.362654in}}%
\pgfpathlineto{\pgfqpoint{5.730006in}{2.362164in}}%
\pgfpathlineto{\pgfqpoint{5.722738in}{2.355515in}}%
\pgfpathlineto{\pgfqpoint{5.715462in}{2.348771in}}%
\pgfpathlineto{\pgfqpoint{5.708178in}{2.341933in}}%
\pgfpathlineto{\pgfqpoint{5.700884in}{2.334999in}}%
\pgfpathlineto{\pgfqpoint{5.686679in}{2.335461in}}%
\pgfpathlineto{\pgfqpoint{5.672483in}{2.335948in}}%
\pgfpathlineto{\pgfqpoint{5.658296in}{2.336460in}}%
\pgfpathlineto{\pgfqpoint{5.644118in}{2.336996in}}%
\pgfpathlineto{\pgfqpoint{5.651425in}{2.343953in}}%
\pgfpathlineto{\pgfqpoint{5.658725in}{2.350817in}}%
\pgfpathlineto{\pgfqpoint{5.666016in}{2.357589in}}%
\pgfpathlineto{\pgfqpoint{5.673298in}{2.364271in}}%
\pgfpathclose%
\pgfusepath{fill}%
\end{pgfscope}%
\begin{pgfscope}%
\pgfpathrectangle{\pgfqpoint{1.150000in}{0.150000in}}{\pgfqpoint{5.700000in}{5.700000in}}%
\pgfusepath{clip}%
\pgfsetbuttcap%
\pgfsetroundjoin%
\definecolor{currentfill}{rgb}{0.267004,0.004874,0.329415}%
\pgfsetfillcolor{currentfill}%
\pgfsetfillopacity{0.700000}%
\pgfsetlinewidth{0.000000pt}%
\definecolor{currentstroke}{rgb}{0.000000,0.000000,0.000000}%
\pgfsetstrokecolor{currentstroke}%
\pgfsetdash{}{0pt}%
\pgfpathmoveto{\pgfqpoint{3.961851in}{2.012011in}}%
\pgfpathlineto{\pgfqpoint{3.975527in}{2.008004in}}%
\pgfpathlineto{\pgfqpoint{3.989210in}{2.004025in}}%
\pgfpathlineto{\pgfqpoint{4.002898in}{2.000073in}}%
\pgfpathlineto{\pgfqpoint{4.016593in}{1.996149in}}%
\pgfpathlineto{\pgfqpoint{4.008650in}{1.988765in}}%
\pgfpathlineto{\pgfqpoint{4.000702in}{1.981463in}}%
\pgfpathlineto{\pgfqpoint{3.992746in}{1.974248in}}%
\pgfpathlineto{\pgfqpoint{3.984785in}{1.967124in}}%
\pgfpathlineto{\pgfqpoint{3.971076in}{1.971237in}}%
\pgfpathlineto{\pgfqpoint{3.957373in}{1.975377in}}%
\pgfpathlineto{\pgfqpoint{3.943676in}{1.979545in}}%
\pgfpathlineto{\pgfqpoint{3.929985in}{1.983740in}}%
\pgfpathlineto{\pgfqpoint{3.937961in}{1.990670in}}%
\pgfpathlineto{\pgfqpoint{3.945931in}{1.997695in}}%
\pgfpathlineto{\pgfqpoint{3.953894in}{2.004810in}}%
\pgfpathlineto{\pgfqpoint{3.961851in}{2.012011in}}%
\pgfpathclose%
\pgfusepath{fill}%
\end{pgfscope}%
\begin{pgfscope}%
\pgfpathrectangle{\pgfqpoint{1.150000in}{0.150000in}}{\pgfqpoint{5.700000in}{5.700000in}}%
\pgfusepath{clip}%
\pgfsetbuttcap%
\pgfsetroundjoin%
\definecolor{currentfill}{rgb}{0.277941,0.056324,0.381191}%
\pgfsetfillcolor{currentfill}%
\pgfsetfillopacity{0.700000}%
\pgfsetlinewidth{0.000000pt}%
\definecolor{currentstroke}{rgb}{0.000000,0.000000,0.000000}%
\pgfsetstrokecolor{currentstroke}%
\pgfsetdash{}{0pt}%
\pgfpathmoveto{\pgfqpoint{4.644882in}{2.103405in}}%
\pgfpathlineto{\pgfqpoint{4.658733in}{2.101200in}}%
\pgfpathlineto{\pgfqpoint{4.672592in}{2.099020in}}%
\pgfpathlineto{\pgfqpoint{4.686458in}{2.096865in}}%
\pgfpathlineto{\pgfqpoint{4.700332in}{2.094736in}}%
\pgfpathlineto{\pgfqpoint{4.692629in}{2.085581in}}%
\pgfpathlineto{\pgfqpoint{4.684920in}{2.076396in}}%
\pgfpathlineto{\pgfqpoint{4.677206in}{2.067184in}}%
\pgfpathlineto{\pgfqpoint{4.669485in}{2.057946in}}%
\pgfpathlineto{\pgfqpoint{4.655601in}{2.060184in}}%
\pgfpathlineto{\pgfqpoint{4.641725in}{2.062447in}}%
\pgfpathlineto{\pgfqpoint{4.627856in}{2.064736in}}%
\pgfpathlineto{\pgfqpoint{4.613994in}{2.067051in}}%
\pgfpathlineto{\pgfqpoint{4.621724in}{2.076175in}}%
\pgfpathlineto{\pgfqpoint{4.629449in}{2.085277in}}%
\pgfpathlineto{\pgfqpoint{4.637168in}{2.094354in}}%
\pgfpathlineto{\pgfqpoint{4.644882in}{2.103405in}}%
\pgfpathclose%
\pgfusepath{fill}%
\end{pgfscope}%
\begin{pgfscope}%
\pgfpathrectangle{\pgfqpoint{1.150000in}{0.150000in}}{\pgfqpoint{5.700000in}{5.700000in}}%
\pgfusepath{clip}%
\pgfsetbuttcap%
\pgfsetroundjoin%
\definecolor{currentfill}{rgb}{0.279574,0.170599,0.479997}%
\pgfsetfillcolor{currentfill}%
\pgfsetfillopacity{0.700000}%
\pgfsetlinewidth{0.000000pt}%
\definecolor{currentstroke}{rgb}{0.000000,0.000000,0.000000}%
\pgfsetstrokecolor{currentstroke}%
\pgfsetdash{}{0pt}%
\pgfpathmoveto{\pgfqpoint{2.809032in}{2.312657in}}%
\pgfpathlineto{\pgfqpoint{2.822539in}{2.304827in}}%
\pgfpathlineto{\pgfqpoint{2.836048in}{2.297037in}}%
\pgfpathlineto{\pgfqpoint{2.849561in}{2.289288in}}%
\pgfpathlineto{\pgfqpoint{2.863078in}{2.281578in}}%
\pgfpathlineto{\pgfqpoint{2.854488in}{2.283279in}}%
\pgfpathlineto{\pgfqpoint{2.845881in}{2.285288in}}%
\pgfpathlineto{\pgfqpoint{2.837257in}{2.287611in}}%
\pgfpathlineto{\pgfqpoint{2.828616in}{2.290257in}}%
\pgfpathlineto{\pgfqpoint{2.815069in}{2.298253in}}%
\pgfpathlineto{\pgfqpoint{2.801524in}{2.306289in}}%
\pgfpathlineto{\pgfqpoint{2.787982in}{2.314365in}}%
\pgfpathlineto{\pgfqpoint{2.774443in}{2.322482in}}%
\pgfpathlineto{\pgfqpoint{2.783117in}{2.319544in}}%
\pgfpathlineto{\pgfqpoint{2.791773in}{2.316932in}}%
\pgfpathlineto{\pgfqpoint{2.800411in}{2.314639in}}%
\pgfpathlineto{\pgfqpoint{2.809032in}{2.312657in}}%
\pgfpathclose%
\pgfusepath{fill}%
\end{pgfscope}%
\begin{pgfscope}%
\pgfpathrectangle{\pgfqpoint{1.150000in}{0.150000in}}{\pgfqpoint{5.700000in}{5.700000in}}%
\pgfusepath{clip}%
\pgfsetbuttcap%
\pgfsetroundjoin%
\definecolor{currentfill}{rgb}{0.276022,0.044167,0.370164}%
\pgfsetfillcolor{currentfill}%
\pgfsetfillopacity{0.700000}%
\pgfsetlinewidth{0.000000pt}%
\definecolor{currentstroke}{rgb}{0.000000,0.000000,0.000000}%
\pgfsetstrokecolor{currentstroke}%
\pgfsetdash{}{0pt}%
\pgfpathmoveto{\pgfqpoint{3.396927in}{2.079799in}}%
\pgfpathlineto{\pgfqpoint{3.410499in}{2.074021in}}%
\pgfpathlineto{\pgfqpoint{3.424075in}{2.068274in}}%
\pgfpathlineto{\pgfqpoint{3.437656in}{2.062559in}}%
\pgfpathlineto{\pgfqpoint{3.451242in}{2.056875in}}%
\pgfpathlineto{\pgfqpoint{3.443037in}{2.053302in}}%
\pgfpathlineto{\pgfqpoint{3.434823in}{2.049923in}}%
\pgfpathlineto{\pgfqpoint{3.426599in}{2.046745in}}%
\pgfpathlineto{\pgfqpoint{3.418365in}{2.043772in}}%
\pgfpathlineto{\pgfqpoint{3.404757in}{2.049698in}}%
\pgfpathlineto{\pgfqpoint{3.391154in}{2.055656in}}%
\pgfpathlineto{\pgfqpoint{3.377556in}{2.061644in}}%
\pgfpathlineto{\pgfqpoint{3.363963in}{2.067665in}}%
\pgfpathlineto{\pgfqpoint{3.372220in}{2.070390in}}%
\pgfpathlineto{\pgfqpoint{3.380466in}{2.073325in}}%
\pgfpathlineto{\pgfqpoint{3.388702in}{2.076463in}}%
\pgfpathlineto{\pgfqpoint{3.396927in}{2.079799in}}%
\pgfpathclose%
\pgfusepath{fill}%
\end{pgfscope}%
\begin{pgfscope}%
\pgfpathrectangle{\pgfqpoint{1.150000in}{0.150000in}}{\pgfqpoint{5.700000in}{5.700000in}}%
\pgfusepath{clip}%
\pgfsetbuttcap%
\pgfsetroundjoin%
\definecolor{currentfill}{rgb}{0.282656,0.100196,0.422160}%
\pgfsetfillcolor{currentfill}%
\pgfsetfillopacity{0.700000}%
\pgfsetlinewidth{0.000000pt}%
\definecolor{currentstroke}{rgb}{0.000000,0.000000,0.000000}%
\pgfsetstrokecolor{currentstroke}%
\pgfsetdash{}{0pt}%
\pgfpathmoveto{\pgfqpoint{4.959031in}{2.180890in}}%
\pgfpathlineto{\pgfqpoint{4.972975in}{2.179336in}}%
\pgfpathlineto{\pgfqpoint{4.986927in}{2.177807in}}%
\pgfpathlineto{\pgfqpoint{5.000888in}{2.176302in}}%
\pgfpathlineto{\pgfqpoint{5.014856in}{2.174823in}}%
\pgfpathlineto{\pgfqpoint{5.007266in}{2.165926in}}%
\pgfpathlineto{\pgfqpoint{4.999670in}{2.156965in}}%
\pgfpathlineto{\pgfqpoint{4.992067in}{2.147939in}}%
\pgfpathlineto{\pgfqpoint{4.984458in}{2.138852in}}%
\pgfpathlineto{\pgfqpoint{4.970479in}{2.140400in}}%
\pgfpathlineto{\pgfqpoint{4.956509in}{2.141973in}}%
\pgfpathlineto{\pgfqpoint{4.942546in}{2.143571in}}%
\pgfpathlineto{\pgfqpoint{4.928591in}{2.145195in}}%
\pgfpathlineto{\pgfqpoint{4.936211in}{2.154208in}}%
\pgfpathlineto{\pgfqpoint{4.943824in}{2.163163in}}%
\pgfpathlineto{\pgfqpoint{4.951430in}{2.172057in}}%
\pgfpathlineto{\pgfqpoint{4.959031in}{2.180890in}}%
\pgfpathclose%
\pgfusepath{fill}%
\end{pgfscope}%
\begin{pgfscope}%
\pgfpathrectangle{\pgfqpoint{1.150000in}{0.150000in}}{\pgfqpoint{5.700000in}{5.700000in}}%
\pgfusepath{clip}%
\pgfsetbuttcap%
\pgfsetroundjoin%
\definecolor{currentfill}{rgb}{0.270595,0.214069,0.507052}%
\pgfsetfillcolor{currentfill}%
\pgfsetfillopacity{0.700000}%
\pgfsetlinewidth{0.000000pt}%
\definecolor{currentstroke}{rgb}{0.000000,0.000000,0.000000}%
\pgfsetstrokecolor{currentstroke}%
\pgfsetdash{}{0pt}%
\pgfpathmoveto{\pgfqpoint{5.901434in}{2.408370in}}%
\pgfpathlineto{\pgfqpoint{5.915674in}{2.408004in}}%
\pgfpathlineto{\pgfqpoint{5.929923in}{2.407663in}}%
\pgfpathlineto{\pgfqpoint{5.944181in}{2.407347in}}%
\pgfpathlineto{\pgfqpoint{5.937028in}{2.401553in}}%
\pgfpathlineto{\pgfqpoint{5.929866in}{2.395671in}}%
\pgfpathlineto{\pgfqpoint{5.922694in}{2.389698in}}%
\pgfpathlineto{\pgfqpoint{5.915514in}{2.383631in}}%
\pgfpathlineto{\pgfqpoint{5.901239in}{2.383893in}}%
\pgfpathlineto{\pgfqpoint{5.886973in}{2.384178in}}%
\pgfpathlineto{\pgfqpoint{5.872716in}{2.384489in}}%
\pgfpathlineto{\pgfqpoint{5.879909in}{2.390593in}}%
\pgfpathlineto{\pgfqpoint{5.887092in}{2.396607in}}%
\pgfpathlineto{\pgfqpoint{5.894268in}{2.402532in}}%
\pgfpathlineto{\pgfqpoint{5.901434in}{2.408370in}}%
\pgfpathclose%
\pgfusepath{fill}%
\end{pgfscope}%
\begin{pgfscope}%
\pgfpathrectangle{\pgfqpoint{1.150000in}{0.150000in}}{\pgfqpoint{5.700000in}{5.700000in}}%
\pgfusepath{clip}%
\pgfsetbuttcap%
\pgfsetroundjoin%
\definecolor{currentfill}{rgb}{0.263663,0.237631,0.518762}%
\pgfsetfillcolor{currentfill}%
\pgfsetfillopacity{0.700000}%
\pgfsetlinewidth{0.000000pt}%
\definecolor{currentstroke}{rgb}{0.000000,0.000000,0.000000}%
\pgfsetstrokecolor{currentstroke}%
\pgfsetdash{}{0pt}%
\pgfpathmoveto{\pgfqpoint{2.558186in}{2.458173in}}%
\pgfpathlineto{\pgfqpoint{2.571684in}{2.449355in}}%
\pgfpathlineto{\pgfqpoint{2.585184in}{2.440584in}}%
\pgfpathlineto{\pgfqpoint{2.598686in}{2.431860in}}%
\pgfpathlineto{\pgfqpoint{2.612191in}{2.423182in}}%
\pgfpathlineto{\pgfqpoint{2.603396in}{2.427343in}}%
\pgfpathlineto{\pgfqpoint{2.594580in}{2.431858in}}%
\pgfpathlineto{\pgfqpoint{2.585744in}{2.436735in}}%
\pgfpathlineto{\pgfqpoint{2.576886in}{2.441981in}}%
\pgfpathlineto{\pgfqpoint{2.563345in}{2.450962in}}%
\pgfpathlineto{\pgfqpoint{2.549807in}{2.459990in}}%
\pgfpathlineto{\pgfqpoint{2.536270in}{2.469064in}}%
\pgfpathlineto{\pgfqpoint{2.522735in}{2.478185in}}%
\pgfpathlineto{\pgfqpoint{2.531630in}{2.472630in}}%
\pgfpathlineto{\pgfqpoint{2.540503in}{2.467449in}}%
\pgfpathlineto{\pgfqpoint{2.549355in}{2.462632in}}%
\pgfpathlineto{\pgfqpoint{2.558186in}{2.458173in}}%
\pgfpathclose%
\pgfusepath{fill}%
\end{pgfscope}%
\begin{pgfscope}%
\pgfpathrectangle{\pgfqpoint{1.150000in}{0.150000in}}{\pgfqpoint{5.700000in}{5.700000in}}%
\pgfusepath{clip}%
\pgfsetbuttcap%
\pgfsetroundjoin%
\definecolor{currentfill}{rgb}{0.271305,0.019942,0.347269}%
\pgfsetfillcolor{currentfill}%
\pgfsetfillopacity{0.700000}%
\pgfsetlinewidth{0.000000pt}%
\definecolor{currentstroke}{rgb}{0.000000,0.000000,0.000000}%
\pgfsetstrokecolor{currentstroke}%
\pgfsetdash{}{0pt}%
\pgfpathmoveto{\pgfqpoint{4.330814in}{2.038248in}}%
\pgfpathlineto{\pgfqpoint{4.344583in}{2.035263in}}%
\pgfpathlineto{\pgfqpoint{4.358358in}{2.032304in}}%
\pgfpathlineto{\pgfqpoint{4.372140in}{2.029371in}}%
\pgfpathlineto{\pgfqpoint{4.385930in}{2.026465in}}%
\pgfpathlineto{\pgfqpoint{4.378118in}{2.017701in}}%
\pgfpathlineto{\pgfqpoint{4.370301in}{2.008955in}}%
\pgfpathlineto{\pgfqpoint{4.362478in}{2.000231in}}%
\pgfpathlineto{\pgfqpoint{4.354650in}{1.991531in}}%
\pgfpathlineto{\pgfqpoint{4.340850in}{1.994587in}}%
\pgfpathlineto{\pgfqpoint{4.327056in}{1.997668in}}%
\pgfpathlineto{\pgfqpoint{4.313269in}{2.000776in}}%
\pgfpathlineto{\pgfqpoint{4.299489in}{2.003910in}}%
\pgfpathlineto{\pgfqpoint{4.307329in}{2.012456in}}%
\pgfpathlineto{\pgfqpoint{4.315163in}{2.021030in}}%
\pgfpathlineto{\pgfqpoint{4.322991in}{2.029629in}}%
\pgfpathlineto{\pgfqpoint{4.330814in}{2.038248in}}%
\pgfpathclose%
\pgfusepath{fill}%
\end{pgfscope}%
\begin{pgfscope}%
\pgfpathrectangle{\pgfqpoint{1.150000in}{0.150000in}}{\pgfqpoint{5.700000in}{5.700000in}}%
\pgfusepath{clip}%
\pgfsetbuttcap%
\pgfsetroundjoin%
\definecolor{currentfill}{rgb}{0.282623,0.140926,0.457517}%
\pgfsetfillcolor{currentfill}%
\pgfsetfillopacity{0.700000}%
\pgfsetlinewidth{0.000000pt}%
\definecolor{currentstroke}{rgb}{0.000000,0.000000,0.000000}%
\pgfsetstrokecolor{currentstroke}%
\pgfsetdash{}{0pt}%
\pgfpathmoveto{\pgfqpoint{5.273281in}{2.261871in}}%
\pgfpathlineto{\pgfqpoint{5.287324in}{2.260840in}}%
\pgfpathlineto{\pgfqpoint{5.301375in}{2.259834in}}%
\pgfpathlineto{\pgfqpoint{5.315434in}{2.258852in}}%
\pgfpathlineto{\pgfqpoint{5.329503in}{2.257896in}}%
\pgfpathlineto{\pgfqpoint{5.322040in}{2.249752in}}%
\pgfpathlineto{\pgfqpoint{5.314570in}{2.241523in}}%
\pgfpathlineto{\pgfqpoint{5.307093in}{2.233206in}}%
\pgfpathlineto{\pgfqpoint{5.299608in}{2.224804in}}%
\pgfpathlineto{\pgfqpoint{5.285528in}{2.225788in}}%
\pgfpathlineto{\pgfqpoint{5.271457in}{2.226798in}}%
\pgfpathlineto{\pgfqpoint{5.257394in}{2.227832in}}%
\pgfpathlineto{\pgfqpoint{5.243340in}{2.228891in}}%
\pgfpathlineto{\pgfqpoint{5.250836in}{2.237260in}}%
\pgfpathlineto{\pgfqpoint{5.258325in}{2.245547in}}%
\pgfpathlineto{\pgfqpoint{5.265807in}{2.253750in}}%
\pgfpathlineto{\pgfqpoint{5.273281in}{2.261871in}}%
\pgfpathclose%
\pgfusepath{fill}%
\end{pgfscope}%
\begin{pgfscope}%
\pgfpathrectangle{\pgfqpoint{1.150000in}{0.150000in}}{\pgfqpoint{5.700000in}{5.700000in}}%
\pgfusepath{clip}%
\pgfsetbuttcap%
\pgfsetroundjoin%
\definecolor{currentfill}{rgb}{0.267004,0.004874,0.329415}%
\pgfsetfillcolor{currentfill}%
\pgfsetfillopacity{0.700000}%
\pgfsetlinewidth{0.000000pt}%
\definecolor{currentstroke}{rgb}{0.000000,0.000000,0.000000}%
\pgfsetstrokecolor{currentstroke}%
\pgfsetdash{}{0pt}%
\pgfpathmoveto{\pgfqpoint{4.103090in}{2.011690in}}%
\pgfpathlineto{\pgfqpoint{4.116803in}{2.008077in}}%
\pgfpathlineto{\pgfqpoint{4.130522in}{2.004490in}}%
\pgfpathlineto{\pgfqpoint{4.144248in}{2.000929in}}%
\pgfpathlineto{\pgfqpoint{4.157981in}{1.997396in}}%
\pgfpathlineto{\pgfqpoint{4.150088in}{1.989388in}}%
\pgfpathlineto{\pgfqpoint{4.142191in}{1.981437in}}%
\pgfpathlineto{\pgfqpoint{4.134287in}{1.973549in}}%
\pgfpathlineto{\pgfqpoint{4.126377in}{1.965727in}}%
\pgfpathlineto{\pgfqpoint{4.112632in}{1.969436in}}%
\pgfpathlineto{\pgfqpoint{4.098893in}{1.973171in}}%
\pgfpathlineto{\pgfqpoint{4.085161in}{1.976934in}}%
\pgfpathlineto{\pgfqpoint{4.071435in}{1.980723in}}%
\pgfpathlineto{\pgfqpoint{4.079358in}{1.988364in}}%
\pgfpathlineto{\pgfqpoint{4.087274in}{1.996076in}}%
\pgfpathlineto{\pgfqpoint{4.095185in}{2.003852in}}%
\pgfpathlineto{\pgfqpoint{4.103090in}{2.011690in}}%
\pgfpathclose%
\pgfusepath{fill}%
\end{pgfscope}%
\begin{pgfscope}%
\pgfpathrectangle{\pgfqpoint{1.150000in}{0.150000in}}{\pgfqpoint{5.700000in}{5.700000in}}%
\pgfusepath{clip}%
\pgfsetbuttcap%
\pgfsetroundjoin%
\definecolor{currentfill}{rgb}{0.278012,0.180367,0.486697}%
\pgfsetfillcolor{currentfill}%
\pgfsetfillopacity{0.700000}%
\pgfsetlinewidth{0.000000pt}%
\definecolor{currentstroke}{rgb}{0.000000,0.000000,0.000000}%
\pgfsetstrokecolor{currentstroke}%
\pgfsetdash{}{0pt}%
\pgfpathmoveto{\pgfqpoint{5.587494in}{2.339387in}}%
\pgfpathlineto{\pgfqpoint{5.601637in}{2.338752in}}%
\pgfpathlineto{\pgfqpoint{5.615788in}{2.338142in}}%
\pgfpathlineto{\pgfqpoint{5.629948in}{2.337557in}}%
\pgfpathlineto{\pgfqpoint{5.644118in}{2.336996in}}%
\pgfpathlineto{\pgfqpoint{5.636802in}{2.329946in}}%
\pgfpathlineto{\pgfqpoint{5.629477in}{2.322800in}}%
\pgfpathlineto{\pgfqpoint{5.622145in}{2.315558in}}%
\pgfpathlineto{\pgfqpoint{5.614804in}{2.308220in}}%
\pgfpathlineto{\pgfqpoint{5.600621in}{2.308767in}}%
\pgfpathlineto{\pgfqpoint{5.586446in}{2.309339in}}%
\pgfpathlineto{\pgfqpoint{5.572281in}{2.309935in}}%
\pgfpathlineto{\pgfqpoint{5.558125in}{2.310556in}}%
\pgfpathlineto{\pgfqpoint{5.565479in}{2.317903in}}%
\pgfpathlineto{\pgfqpoint{5.572826in}{2.325157in}}%
\pgfpathlineto{\pgfqpoint{5.580164in}{2.332318in}}%
\pgfpathlineto{\pgfqpoint{5.587494in}{2.339387in}}%
\pgfpathclose%
\pgfusepath{fill}%
\end{pgfscope}%
\begin{pgfscope}%
\pgfpathrectangle{\pgfqpoint{1.150000in}{0.150000in}}{\pgfqpoint{5.700000in}{5.700000in}}%
\pgfusepath{clip}%
\pgfsetbuttcap%
\pgfsetroundjoin%
\definecolor{currentfill}{rgb}{0.276022,0.044167,0.370164}%
\pgfsetfillcolor{currentfill}%
\pgfsetfillopacity{0.700000}%
\pgfsetlinewidth{0.000000pt}%
\definecolor{currentstroke}{rgb}{0.000000,0.000000,0.000000}%
\pgfsetstrokecolor{currentstroke}%
\pgfsetdash{}{0pt}%
\pgfpathmoveto{\pgfqpoint{4.558622in}{2.076564in}}%
\pgfpathlineto{\pgfqpoint{4.572454in}{2.074147in}}%
\pgfpathlineto{\pgfqpoint{4.586293in}{2.071756in}}%
\pgfpathlineto{\pgfqpoint{4.600140in}{2.069391in}}%
\pgfpathlineto{\pgfqpoint{4.613994in}{2.067051in}}%
\pgfpathlineto{\pgfqpoint{4.606258in}{2.057907in}}%
\pgfpathlineto{\pgfqpoint{4.598517in}{2.048747in}}%
\pgfpathlineto{\pgfqpoint{4.590770in}{2.039573in}}%
\pgfpathlineto{\pgfqpoint{4.583018in}{2.030388in}}%
\pgfpathlineto{\pgfqpoint{4.569153in}{2.032850in}}%
\pgfpathlineto{\pgfqpoint{4.555296in}{2.035338in}}%
\pgfpathlineto{\pgfqpoint{4.541446in}{2.037852in}}%
\pgfpathlineto{\pgfqpoint{4.527603in}{2.040391in}}%
\pgfpathlineto{\pgfqpoint{4.535366in}{2.049449in}}%
\pgfpathlineto{\pgfqpoint{4.543123in}{2.058499in}}%
\pgfpathlineto{\pgfqpoint{4.550875in}{2.067538in}}%
\pgfpathlineto{\pgfqpoint{4.558622in}{2.076564in}}%
\pgfpathclose%
\pgfusepath{fill}%
\end{pgfscope}%
\begin{pgfscope}%
\pgfpathrectangle{\pgfqpoint{1.150000in}{0.150000in}}{\pgfqpoint{5.700000in}{5.700000in}}%
\pgfusepath{clip}%
\pgfsetbuttcap%
\pgfsetroundjoin%
\definecolor{currentfill}{rgb}{0.283091,0.110553,0.431554}%
\pgfsetfillcolor{currentfill}%
\pgfsetfillopacity{0.700000}%
\pgfsetlinewidth{0.000000pt}%
\definecolor{currentstroke}{rgb}{0.000000,0.000000,0.000000}%
\pgfsetstrokecolor{currentstroke}%
\pgfsetdash{}{0pt}%
\pgfpathmoveto{\pgfqpoint{3.059388in}{2.191433in}}%
\pgfpathlineto{\pgfqpoint{3.072921in}{2.184485in}}%
\pgfpathlineto{\pgfqpoint{3.086458in}{2.177572in}}%
\pgfpathlineto{\pgfqpoint{3.099999in}{2.170695in}}%
\pgfpathlineto{\pgfqpoint{3.113544in}{2.163853in}}%
\pgfpathlineto{\pgfqpoint{3.105129in}{2.163336in}}%
\pgfpathlineto{\pgfqpoint{3.096700in}{2.163082in}}%
\pgfpathlineto{\pgfqpoint{3.088257in}{2.163099in}}%
\pgfpathlineto{\pgfqpoint{3.079800in}{2.163394in}}%
\pgfpathlineto{\pgfqpoint{3.066228in}{2.170506in}}%
\pgfpathlineto{\pgfqpoint{3.052659in}{2.177654in}}%
\pgfpathlineto{\pgfqpoint{3.039094in}{2.184838in}}%
\pgfpathlineto{\pgfqpoint{3.025533in}{2.192057in}}%
\pgfpathlineto{\pgfqpoint{3.034019in}{2.191486in}}%
\pgfpathlineto{\pgfqpoint{3.042489in}{2.191196in}}%
\pgfpathlineto{\pgfqpoint{3.050946in}{2.191181in}}%
\pgfpathlineto{\pgfqpoint{3.059388in}{2.191433in}}%
\pgfpathclose%
\pgfusepath{fill}%
\end{pgfscope}%
\begin{pgfscope}%
\pgfpathrectangle{\pgfqpoint{1.150000in}{0.150000in}}{\pgfqpoint{5.700000in}{5.700000in}}%
\pgfusepath{clip}%
\pgfsetbuttcap%
\pgfsetroundjoin%
\definecolor{currentfill}{rgb}{0.281446,0.084320,0.407414}%
\pgfsetfillcolor{currentfill}%
\pgfsetfillopacity{0.700000}%
\pgfsetlinewidth{0.000000pt}%
\definecolor{currentstroke}{rgb}{0.000000,0.000000,0.000000}%
\pgfsetstrokecolor{currentstroke}%
\pgfsetdash{}{0pt}%
\pgfpathmoveto{\pgfqpoint{4.872852in}{2.151939in}}%
\pgfpathlineto{\pgfqpoint{4.886775in}{2.150215in}}%
\pgfpathlineto{\pgfqpoint{4.900706in}{2.148516in}}%
\pgfpathlineto{\pgfqpoint{4.914645in}{2.146843in}}%
\pgfpathlineto{\pgfqpoint{4.928591in}{2.145195in}}%
\pgfpathlineto{\pgfqpoint{4.920966in}{2.136123in}}%
\pgfpathlineto{\pgfqpoint{4.913334in}{2.126996in}}%
\pgfpathlineto{\pgfqpoint{4.905696in}{2.117815in}}%
\pgfpathlineto{\pgfqpoint{4.898053in}{2.108581in}}%
\pgfpathlineto{\pgfqpoint{4.884096in}{2.110312in}}%
\pgfpathlineto{\pgfqpoint{4.870147in}{2.112068in}}%
\pgfpathlineto{\pgfqpoint{4.856206in}{2.113849in}}%
\pgfpathlineto{\pgfqpoint{4.842273in}{2.115654in}}%
\pgfpathlineto{\pgfqpoint{4.849927in}{2.124801in}}%
\pgfpathlineto{\pgfqpoint{4.857575in}{2.133898in}}%
\pgfpathlineto{\pgfqpoint{4.865216in}{2.142945in}}%
\pgfpathlineto{\pgfqpoint{4.872852in}{2.151939in}}%
\pgfpathclose%
\pgfusepath{fill}%
\end{pgfscope}%
\begin{pgfscope}%
\pgfpathrectangle{\pgfqpoint{1.150000in}{0.150000in}}{\pgfqpoint{5.700000in}{5.700000in}}%
\pgfusepath{clip}%
\pgfsetbuttcap%
\pgfsetroundjoin%
\definecolor{currentfill}{rgb}{0.280267,0.073417,0.397163}%
\pgfsetfillcolor{currentfill}%
\pgfsetfillopacity{0.700000}%
\pgfsetlinewidth{0.000000pt}%
\definecolor{currentstroke}{rgb}{0.000000,0.000000,0.000000}%
\pgfsetstrokecolor{currentstroke}%
\pgfsetdash{}{0pt}%
\pgfpathmoveto{\pgfqpoint{3.255379in}{2.116982in}}%
\pgfpathlineto{\pgfqpoint{3.268936in}{2.110703in}}%
\pgfpathlineto{\pgfqpoint{3.282498in}{2.104457in}}%
\pgfpathlineto{\pgfqpoint{3.296064in}{2.098244in}}%
\pgfpathlineto{\pgfqpoint{3.309635in}{2.092064in}}%
\pgfpathlineto{\pgfqpoint{3.301344in}{2.089806in}}%
\pgfpathlineto{\pgfqpoint{3.293042in}{2.087774in}}%
\pgfpathlineto{\pgfqpoint{3.284728in}{2.085975in}}%
\pgfpathlineto{\pgfqpoint{3.276402in}{2.084416in}}%
\pgfpathlineto{\pgfqpoint{3.262807in}{2.090853in}}%
\pgfpathlineto{\pgfqpoint{3.249217in}{2.097322in}}%
\pgfpathlineto{\pgfqpoint{3.235631in}{2.103824in}}%
\pgfpathlineto{\pgfqpoint{3.222049in}{2.110359in}}%
\pgfpathlineto{\pgfqpoint{3.230400in}{2.111657in}}%
\pgfpathlineto{\pgfqpoint{3.238738in}{2.113198in}}%
\pgfpathlineto{\pgfqpoint{3.247064in}{2.114975in}}%
\pgfpathlineto{\pgfqpoint{3.255379in}{2.116982in}}%
\pgfpathclose%
\pgfusepath{fill}%
\end{pgfscope}%
\begin{pgfscope}%
\pgfpathrectangle{\pgfqpoint{1.150000in}{0.150000in}}{\pgfqpoint{5.700000in}{5.700000in}}%
\pgfusepath{clip}%
\pgfsetbuttcap%
\pgfsetroundjoin%
\definecolor{currentfill}{rgb}{0.283072,0.130895,0.449241}%
\pgfsetfillcolor{currentfill}%
\pgfsetfillopacity{0.700000}%
\pgfsetlinewidth{0.000000pt}%
\definecolor{currentstroke}{rgb}{0.000000,0.000000,0.000000}%
\pgfsetstrokecolor{currentstroke}%
\pgfsetdash{}{0pt}%
\pgfpathmoveto{\pgfqpoint{5.187207in}{2.233374in}}%
\pgfpathlineto{\pgfqpoint{5.201227in}{2.232216in}}%
\pgfpathlineto{\pgfqpoint{5.215256in}{2.231083in}}%
\pgfpathlineto{\pgfqpoint{5.229294in}{2.229974in}}%
\pgfpathlineto{\pgfqpoint{5.243340in}{2.228891in}}%
\pgfpathlineto{\pgfqpoint{5.235836in}{2.220439in}}%
\pgfpathlineto{\pgfqpoint{5.228326in}{2.211905in}}%
\pgfpathlineto{\pgfqpoint{5.220809in}{2.203289in}}%
\pgfpathlineto{\pgfqpoint{5.213284in}{2.194593in}}%
\pgfpathlineto{\pgfqpoint{5.199227in}{2.195718in}}%
\pgfpathlineto{\pgfqpoint{5.185179in}{2.196868in}}%
\pgfpathlineto{\pgfqpoint{5.171139in}{2.198043in}}%
\pgfpathlineto{\pgfqpoint{5.157107in}{2.199243in}}%
\pgfpathlineto{\pgfqpoint{5.164642in}{2.207893in}}%
\pgfpathlineto{\pgfqpoint{5.172171in}{2.216465in}}%
\pgfpathlineto{\pgfqpoint{5.179692in}{2.224959in}}%
\pgfpathlineto{\pgfqpoint{5.187207in}{2.233374in}}%
\pgfpathclose%
\pgfusepath{fill}%
\end{pgfscope}%
\begin{pgfscope}%
\pgfpathrectangle{\pgfqpoint{1.150000in}{0.150000in}}{\pgfqpoint{5.700000in}{5.700000in}}%
\pgfusepath{clip}%
\pgfsetbuttcap%
\pgfsetroundjoin%
\definecolor{currentfill}{rgb}{0.268510,0.009605,0.335427}%
\pgfsetfillcolor{currentfill}%
\pgfsetfillopacity{0.700000}%
\pgfsetlinewidth{0.000000pt}%
\definecolor{currentstroke}{rgb}{0.000000,0.000000,0.000000}%
\pgfsetstrokecolor{currentstroke}%
\pgfsetdash{}{0pt}%
\pgfpathmoveto{\pgfqpoint{3.733993in}{2.011996in}}%
\pgfpathlineto{\pgfqpoint{3.747630in}{2.007262in}}%
\pgfpathlineto{\pgfqpoint{3.761273in}{2.002558in}}%
\pgfpathlineto{\pgfqpoint{3.774921in}{1.997881in}}%
\pgfpathlineto{\pgfqpoint{3.788575in}{1.993233in}}%
\pgfpathlineto{\pgfqpoint{3.780533in}{1.987269in}}%
\pgfpathlineto{\pgfqpoint{3.772483in}{1.981437in}}%
\pgfpathlineto{\pgfqpoint{3.764426in}{1.975741in}}%
\pgfpathlineto{\pgfqpoint{3.756361in}{1.970187in}}%
\pgfpathlineto{\pgfqpoint{3.742690in}{1.975050in}}%
\pgfpathlineto{\pgfqpoint{3.729025in}{1.979942in}}%
\pgfpathlineto{\pgfqpoint{3.715365in}{1.984862in}}%
\pgfpathlineto{\pgfqpoint{3.701711in}{1.989810in}}%
\pgfpathlineto{\pgfqpoint{3.709793in}{1.995144in}}%
\pgfpathlineto{\pgfqpoint{3.717867in}{2.000623in}}%
\pgfpathlineto{\pgfqpoint{3.725934in}{2.006242in}}%
\pgfpathlineto{\pgfqpoint{3.733993in}{2.011996in}}%
\pgfpathclose%
\pgfusepath{fill}%
\end{pgfscope}%
\begin{pgfscope}%
\pgfpathrectangle{\pgfqpoint{1.150000in}{0.150000in}}{\pgfqpoint{5.700000in}{5.700000in}}%
\pgfusepath{clip}%
\pgfsetbuttcap%
\pgfsetroundjoin%
\definecolor{currentfill}{rgb}{0.271305,0.019942,0.347269}%
\pgfsetfillcolor{currentfill}%
\pgfsetfillopacity{0.700000}%
\pgfsetlinewidth{0.000000pt}%
\definecolor{currentstroke}{rgb}{0.000000,0.000000,0.000000}%
\pgfsetstrokecolor{currentstroke}%
\pgfsetdash{}{0pt}%
\pgfpathmoveto{\pgfqpoint{3.592669in}{2.030445in}}%
\pgfpathlineto{\pgfqpoint{3.606281in}{2.025262in}}%
\pgfpathlineto{\pgfqpoint{3.619898in}{2.020110in}}%
\pgfpathlineto{\pgfqpoint{3.633520in}{2.014987in}}%
\pgfpathlineto{\pgfqpoint{3.647147in}{2.009893in}}%
\pgfpathlineto{\pgfqpoint{3.639038in}{2.004934in}}%
\pgfpathlineto{\pgfqpoint{3.630921in}{2.000135in}}%
\pgfpathlineto{\pgfqpoint{3.622796in}{1.995502in}}%
\pgfpathlineto{\pgfqpoint{3.614662in}{1.991040in}}%
\pgfpathlineto{\pgfqpoint{3.601015in}{1.996362in}}%
\pgfpathlineto{\pgfqpoint{3.587374in}{2.001714in}}%
\pgfpathlineto{\pgfqpoint{3.573738in}{2.007095in}}%
\pgfpathlineto{\pgfqpoint{3.560107in}{2.012506in}}%
\pgfpathlineto{\pgfqpoint{3.568261in}{2.016734in}}%
\pgfpathlineto{\pgfqpoint{3.576406in}{2.021137in}}%
\pgfpathlineto{\pgfqpoint{3.584542in}{2.025709in}}%
\pgfpathlineto{\pgfqpoint{3.592669in}{2.030445in}}%
\pgfpathclose%
\pgfusepath{fill}%
\end{pgfscope}%
\begin{pgfscope}%
\pgfpathrectangle{\pgfqpoint{1.150000in}{0.150000in}}{\pgfqpoint{5.700000in}{5.700000in}}%
\pgfusepath{clip}%
\pgfsetbuttcap%
\pgfsetroundjoin%
\definecolor{currentfill}{rgb}{0.280868,0.160771,0.472899}%
\pgfsetfillcolor{currentfill}%
\pgfsetfillopacity{0.700000}%
\pgfsetlinewidth{0.000000pt}%
\definecolor{currentstroke}{rgb}{0.000000,0.000000,0.000000}%
\pgfsetstrokecolor{currentstroke}%
\pgfsetdash{}{0pt}%
\pgfpathmoveto{\pgfqpoint{2.863078in}{2.281578in}}%
\pgfpathlineto{\pgfqpoint{2.876597in}{2.273908in}}%
\pgfpathlineto{\pgfqpoint{2.890119in}{2.266277in}}%
\pgfpathlineto{\pgfqpoint{2.903645in}{2.258685in}}%
\pgfpathlineto{\pgfqpoint{2.917175in}{2.251131in}}%
\pgfpathlineto{\pgfqpoint{2.908615in}{2.252552in}}%
\pgfpathlineto{\pgfqpoint{2.900040in}{2.254276in}}%
\pgfpathlineto{\pgfqpoint{2.891447in}{2.256312in}}%
\pgfpathlineto{\pgfqpoint{2.882839in}{2.258666in}}%
\pgfpathlineto{\pgfqpoint{2.869278in}{2.266505in}}%
\pgfpathlineto{\pgfqpoint{2.855721in}{2.274383in}}%
\pgfpathlineto{\pgfqpoint{2.842167in}{2.282300in}}%
\pgfpathlineto{\pgfqpoint{2.828616in}{2.290257in}}%
\pgfpathlineto{\pgfqpoint{2.837257in}{2.287611in}}%
\pgfpathlineto{\pgfqpoint{2.845881in}{2.285288in}}%
\pgfpathlineto{\pgfqpoint{2.854488in}{2.283279in}}%
\pgfpathlineto{\pgfqpoint{2.863078in}{2.281578in}}%
\pgfpathclose%
\pgfusepath{fill}%
\end{pgfscope}%
\begin{pgfscope}%
\pgfpathrectangle{\pgfqpoint{1.150000in}{0.150000in}}{\pgfqpoint{5.700000in}{5.700000in}}%
\pgfusepath{clip}%
\pgfsetbuttcap%
\pgfsetroundjoin%
\definecolor{currentfill}{rgb}{0.266580,0.228262,0.514349}%
\pgfsetfillcolor{currentfill}%
\pgfsetfillopacity{0.700000}%
\pgfsetlinewidth{0.000000pt}%
\definecolor{currentstroke}{rgb}{0.000000,0.000000,0.000000}%
\pgfsetstrokecolor{currentstroke}%
\pgfsetdash{}{0pt}%
\pgfpathmoveto{\pgfqpoint{2.612191in}{2.423182in}}%
\pgfpathlineto{\pgfqpoint{2.625698in}{2.414549in}}%
\pgfpathlineto{\pgfqpoint{2.639207in}{2.405961in}}%
\pgfpathlineto{\pgfqpoint{2.652719in}{2.397418in}}%
\pgfpathlineto{\pgfqpoint{2.666234in}{2.388920in}}%
\pgfpathlineto{\pgfqpoint{2.657473in}{2.392785in}}%
\pgfpathlineto{\pgfqpoint{2.648694in}{2.396999in}}%
\pgfpathlineto{\pgfqpoint{2.639893in}{2.401570in}}%
\pgfpathlineto{\pgfqpoint{2.631073in}{2.406508in}}%
\pgfpathlineto{\pgfqpoint{2.617523in}{2.415309in}}%
\pgfpathlineto{\pgfqpoint{2.603975in}{2.424155in}}%
\pgfpathlineto{\pgfqpoint{2.590430in}{2.433045in}}%
\pgfpathlineto{\pgfqpoint{2.576886in}{2.441981in}}%
\pgfpathlineto{\pgfqpoint{2.585744in}{2.436735in}}%
\pgfpathlineto{\pgfqpoint{2.594580in}{2.431858in}}%
\pgfpathlineto{\pgfqpoint{2.603396in}{2.427343in}}%
\pgfpathlineto{\pgfqpoint{2.612191in}{2.423182in}}%
\pgfpathclose%
\pgfusepath{fill}%
\end{pgfscope}%
\begin{pgfscope}%
\pgfpathrectangle{\pgfqpoint{1.150000in}{0.150000in}}{\pgfqpoint{5.700000in}{5.700000in}}%
\pgfusepath{clip}%
\pgfsetbuttcap%
\pgfsetroundjoin%
\definecolor{currentfill}{rgb}{0.267004,0.004874,0.329415}%
\pgfsetfillcolor{currentfill}%
\pgfsetfillopacity{0.700000}%
\pgfsetlinewidth{0.000000pt}%
\definecolor{currentstroke}{rgb}{0.000000,0.000000,0.000000}%
\pgfsetstrokecolor{currentstroke}%
\pgfsetdash{}{0pt}%
\pgfpathmoveto{\pgfqpoint{3.875281in}{2.000796in}}%
\pgfpathlineto{\pgfqpoint{3.888948in}{1.996490in}}%
\pgfpathlineto{\pgfqpoint{3.902621in}{1.992212in}}%
\pgfpathlineto{\pgfqpoint{3.916300in}{1.987962in}}%
\pgfpathlineto{\pgfqpoint{3.929985in}{1.983740in}}%
\pgfpathlineto{\pgfqpoint{3.922002in}{1.976909in}}%
\pgfpathlineto{\pgfqpoint{3.914013in}{1.970182in}}%
\pgfpathlineto{\pgfqpoint{3.906017in}{1.963564in}}%
\pgfpathlineto{\pgfqpoint{3.898014in}{1.957060in}}%
\pgfpathlineto{\pgfqpoint{3.884314in}{1.961485in}}%
\pgfpathlineto{\pgfqpoint{3.870619in}{1.965937in}}%
\pgfpathlineto{\pgfqpoint{3.856931in}{1.970416in}}%
\pgfpathlineto{\pgfqpoint{3.843248in}{1.974924in}}%
\pgfpathlineto{\pgfqpoint{3.851267in}{1.981220in}}%
\pgfpathlineto{\pgfqpoint{3.859279in}{1.987635in}}%
\pgfpathlineto{\pgfqpoint{3.867283in}{1.994161in}}%
\pgfpathlineto{\pgfqpoint{3.875281in}{2.000796in}}%
\pgfpathclose%
\pgfusepath{fill}%
\end{pgfscope}%
\begin{pgfscope}%
\pgfpathrectangle{\pgfqpoint{1.150000in}{0.150000in}}{\pgfqpoint{5.700000in}{5.700000in}}%
\pgfusepath{clip}%
\pgfsetbuttcap%
\pgfsetroundjoin%
\definecolor{currentfill}{rgb}{0.271828,0.209303,0.504434}%
\pgfsetfillcolor{currentfill}%
\pgfsetfillopacity{0.700000}%
\pgfsetlinewidth{0.000000pt}%
\definecolor{currentstroke}{rgb}{0.000000,0.000000,0.000000}%
\pgfsetstrokecolor{currentstroke}%
\pgfsetdash{}{0pt}%
\pgfpathmoveto{\pgfqpoint{5.815781in}{2.385975in}}%
\pgfpathlineto{\pgfqpoint{5.830001in}{2.385567in}}%
\pgfpathlineto{\pgfqpoint{5.844230in}{2.385183in}}%
\pgfpathlineto{\pgfqpoint{5.858468in}{2.384824in}}%
\pgfpathlineto{\pgfqpoint{5.872716in}{2.384489in}}%
\pgfpathlineto{\pgfqpoint{5.865515in}{2.378292in}}%
\pgfpathlineto{\pgfqpoint{5.858304in}{2.372001in}}%
\pgfpathlineto{\pgfqpoint{5.851085in}{2.365614in}}%
\pgfpathlineto{\pgfqpoint{5.843857in}{2.359129in}}%
\pgfpathlineto{\pgfqpoint{5.829594in}{2.359423in}}%
\pgfpathlineto{\pgfqpoint{5.815339in}{2.359741in}}%
\pgfpathlineto{\pgfqpoint{5.801094in}{2.360083in}}%
\pgfpathlineto{\pgfqpoint{5.786858in}{2.360450in}}%
\pgfpathlineto{\pgfqpoint{5.794102in}{2.366971in}}%
\pgfpathlineto{\pgfqpoint{5.801337in}{2.373398in}}%
\pgfpathlineto{\pgfqpoint{5.808563in}{2.379732in}}%
\pgfpathlineto{\pgfqpoint{5.815781in}{2.385975in}}%
\pgfpathclose%
\pgfusepath{fill}%
\end{pgfscope}%
\begin{pgfscope}%
\pgfpathrectangle{\pgfqpoint{1.150000in}{0.150000in}}{\pgfqpoint{5.700000in}{5.700000in}}%
\pgfusepath{clip}%
\pgfsetbuttcap%
\pgfsetroundjoin%
\definecolor{currentfill}{rgb}{0.279574,0.170599,0.479997}%
\pgfsetfillcolor{currentfill}%
\pgfsetfillopacity{0.700000}%
\pgfsetlinewidth{0.000000pt}%
\definecolor{currentstroke}{rgb}{0.000000,0.000000,0.000000}%
\pgfsetstrokecolor{currentstroke}%
\pgfsetdash{}{0pt}%
\pgfpathmoveto{\pgfqpoint{5.501588in}{2.313287in}}%
\pgfpathlineto{\pgfqpoint{5.515709in}{2.312568in}}%
\pgfpathlineto{\pgfqpoint{5.529839in}{2.311872in}}%
\pgfpathlineto{\pgfqpoint{5.543977in}{2.311202in}}%
\pgfpathlineto{\pgfqpoint{5.558125in}{2.310556in}}%
\pgfpathlineto{\pgfqpoint{5.550762in}{2.303115in}}%
\pgfpathlineto{\pgfqpoint{5.543391in}{2.295579in}}%
\pgfpathlineto{\pgfqpoint{5.536013in}{2.287948in}}%
\pgfpathlineto{\pgfqpoint{5.528626in}{2.280221in}}%
\pgfpathlineto{\pgfqpoint{5.514465in}{2.280867in}}%
\pgfpathlineto{\pgfqpoint{5.500313in}{2.281537in}}%
\pgfpathlineto{\pgfqpoint{5.486171in}{2.282233in}}%
\pgfpathlineto{\pgfqpoint{5.472037in}{2.282953in}}%
\pgfpathlineto{\pgfqpoint{5.479436in}{2.290674in}}%
\pgfpathlineto{\pgfqpoint{5.486828in}{2.298304in}}%
\pgfpathlineto{\pgfqpoint{5.494212in}{2.305841in}}%
\pgfpathlineto{\pgfqpoint{5.501588in}{2.313287in}}%
\pgfpathclose%
\pgfusepath{fill}%
\end{pgfscope}%
\begin{pgfscope}%
\pgfpathrectangle{\pgfqpoint{1.150000in}{0.150000in}}{\pgfqpoint{5.700000in}{5.700000in}}%
\pgfusepath{clip}%
\pgfsetbuttcap%
\pgfsetroundjoin%
\definecolor{currentfill}{rgb}{0.268510,0.009605,0.335427}%
\pgfsetfillcolor{currentfill}%
\pgfsetfillopacity{0.700000}%
\pgfsetlinewidth{0.000000pt}%
\definecolor{currentstroke}{rgb}{0.000000,0.000000,0.000000}%
\pgfsetstrokecolor{currentstroke}%
\pgfsetdash{}{0pt}%
\pgfpathmoveto{\pgfqpoint{4.244437in}{2.016708in}}%
\pgfpathlineto{\pgfqpoint{4.258190in}{2.013469in}}%
\pgfpathlineto{\pgfqpoint{4.271949in}{2.010257in}}%
\pgfpathlineto{\pgfqpoint{4.285716in}{2.007070in}}%
\pgfpathlineto{\pgfqpoint{4.299489in}{2.003910in}}%
\pgfpathlineto{\pgfqpoint{4.291644in}{1.995396in}}%
\pgfpathlineto{\pgfqpoint{4.283793in}{1.986917in}}%
\pgfpathlineto{\pgfqpoint{4.275937in}{1.978477in}}%
\pgfpathlineto{\pgfqpoint{4.268076in}{1.970081in}}%
\pgfpathlineto{\pgfqpoint{4.254290in}{1.973403in}}%
\pgfpathlineto{\pgfqpoint{4.240512in}{1.976752in}}%
\pgfpathlineto{\pgfqpoint{4.226740in}{1.980126in}}%
\pgfpathlineto{\pgfqpoint{4.212975in}{1.983527in}}%
\pgfpathlineto{\pgfqpoint{4.220849in}{1.991757in}}%
\pgfpathlineto{\pgfqpoint{4.228717in}{2.000033in}}%
\pgfpathlineto{\pgfqpoint{4.236580in}{2.008351in}}%
\pgfpathlineto{\pgfqpoint{4.244437in}{2.016708in}}%
\pgfpathclose%
\pgfusepath{fill}%
\end{pgfscope}%
\begin{pgfscope}%
\pgfpathrectangle{\pgfqpoint{1.150000in}{0.150000in}}{\pgfqpoint{5.700000in}{5.700000in}}%
\pgfusepath{clip}%
\pgfsetbuttcap%
\pgfsetroundjoin%
\definecolor{currentfill}{rgb}{0.280267,0.073417,0.397163}%
\pgfsetfillcolor{currentfill}%
\pgfsetfillopacity{0.700000}%
\pgfsetlinewidth{0.000000pt}%
\definecolor{currentstroke}{rgb}{0.000000,0.000000,0.000000}%
\pgfsetstrokecolor{currentstroke}%
\pgfsetdash{}{0pt}%
\pgfpathmoveto{\pgfqpoint{4.786618in}{2.123130in}}%
\pgfpathlineto{\pgfqpoint{4.800520in}{2.121223in}}%
\pgfpathlineto{\pgfqpoint{4.814430in}{2.119342in}}%
\pgfpathlineto{\pgfqpoint{4.828348in}{2.117486in}}%
\pgfpathlineto{\pgfqpoint{4.842273in}{2.115654in}}%
\pgfpathlineto{\pgfqpoint{4.834613in}{2.106461in}}%
\pgfpathlineto{\pgfqpoint{4.826947in}{2.097221in}}%
\pgfpathlineto{\pgfqpoint{4.819276in}{2.087939in}}%
\pgfpathlineto{\pgfqpoint{4.811598in}{2.078614in}}%
\pgfpathlineto{\pgfqpoint{4.797663in}{2.080541in}}%
\pgfpathlineto{\pgfqpoint{4.783736in}{2.082494in}}%
\pgfpathlineto{\pgfqpoint{4.769816in}{2.084471in}}%
\pgfpathlineto{\pgfqpoint{4.755904in}{2.086473in}}%
\pgfpathlineto{\pgfqpoint{4.763591in}{2.095697in}}%
\pgfpathlineto{\pgfqpoint{4.771273in}{2.104882in}}%
\pgfpathlineto{\pgfqpoint{4.778949in}{2.114027in}}%
\pgfpathlineto{\pgfqpoint{4.786618in}{2.123130in}}%
\pgfpathclose%
\pgfusepath{fill}%
\end{pgfscope}%
\begin{pgfscope}%
\pgfpathrectangle{\pgfqpoint{1.150000in}{0.150000in}}{\pgfqpoint{5.700000in}{5.700000in}}%
\pgfusepath{clip}%
\pgfsetbuttcap%
\pgfsetroundjoin%
\definecolor{currentfill}{rgb}{0.273809,0.031497,0.358853}%
\pgfsetfillcolor{currentfill}%
\pgfsetfillopacity{0.700000}%
\pgfsetlinewidth{0.000000pt}%
\definecolor{currentstroke}{rgb}{0.000000,0.000000,0.000000}%
\pgfsetstrokecolor{currentstroke}%
\pgfsetdash{}{0pt}%
\pgfpathmoveto{\pgfqpoint{4.472305in}{2.050804in}}%
\pgfpathlineto{\pgfqpoint{4.486119in}{2.048162in}}%
\pgfpathlineto{\pgfqpoint{4.499940in}{2.045546in}}%
\pgfpathlineto{\pgfqpoint{4.513768in}{2.042955in}}%
\pgfpathlineto{\pgfqpoint{4.527603in}{2.040391in}}%
\pgfpathlineto{\pgfqpoint{4.519835in}{2.031328in}}%
\pgfpathlineto{\pgfqpoint{4.512062in}{2.022263in}}%
\pgfpathlineto{\pgfqpoint{4.504282in}{2.013200in}}%
\pgfpathlineto{\pgfqpoint{4.496498in}{2.004140in}}%
\pgfpathlineto{\pgfqpoint{4.482652in}{2.006841in}}%
\pgfpathlineto{\pgfqpoint{4.468813in}{2.009567in}}%
\pgfpathlineto{\pgfqpoint{4.454982in}{2.012319in}}%
\pgfpathlineto{\pgfqpoint{4.441157in}{2.015096in}}%
\pgfpathlineto{\pgfqpoint{4.448952in}{2.024015in}}%
\pgfpathlineto{\pgfqpoint{4.456742in}{2.032941in}}%
\pgfpathlineto{\pgfqpoint{4.464526in}{2.041872in}}%
\pgfpathlineto{\pgfqpoint{4.472305in}{2.050804in}}%
\pgfpathclose%
\pgfusepath{fill}%
\end{pgfscope}%
\begin{pgfscope}%
\pgfpathrectangle{\pgfqpoint{1.150000in}{0.150000in}}{\pgfqpoint{5.700000in}{5.700000in}}%
\pgfusepath{clip}%
\pgfsetbuttcap%
\pgfsetroundjoin%
\definecolor{currentfill}{rgb}{0.274952,0.037752,0.364543}%
\pgfsetfillcolor{currentfill}%
\pgfsetfillopacity{0.700000}%
\pgfsetlinewidth{0.000000pt}%
\definecolor{currentstroke}{rgb}{0.000000,0.000000,0.000000}%
\pgfsetstrokecolor{currentstroke}%
\pgfsetdash{}{0pt}%
\pgfpathmoveto{\pgfqpoint{3.451242in}{2.056875in}}%
\pgfpathlineto{\pgfqpoint{3.464832in}{2.051222in}}%
\pgfpathlineto{\pgfqpoint{3.478428in}{2.045600in}}%
\pgfpathlineto{\pgfqpoint{3.492029in}{2.040008in}}%
\pgfpathlineto{\pgfqpoint{3.505634in}{2.034447in}}%
\pgfpathlineto{\pgfqpoint{3.497451in}{2.030638in}}%
\pgfpathlineto{\pgfqpoint{3.489258in}{2.027018in}}%
\pgfpathlineto{\pgfqpoint{3.481055in}{2.023596in}}%
\pgfpathlineto{\pgfqpoint{3.472843in}{2.020376in}}%
\pgfpathlineto{\pgfqpoint{3.459216in}{2.026179in}}%
\pgfpathlineto{\pgfqpoint{3.445594in}{2.032013in}}%
\pgfpathlineto{\pgfqpoint{3.431977in}{2.037877in}}%
\pgfpathlineto{\pgfqpoint{3.418365in}{2.043772in}}%
\pgfpathlineto{\pgfqpoint{3.426599in}{2.046745in}}%
\pgfpathlineto{\pgfqpoint{3.434823in}{2.049923in}}%
\pgfpathlineto{\pgfqpoint{3.443037in}{2.053302in}}%
\pgfpathlineto{\pgfqpoint{3.451242in}{2.056875in}}%
\pgfpathclose%
\pgfusepath{fill}%
\end{pgfscope}%
\begin{pgfscope}%
\pgfpathrectangle{\pgfqpoint{1.150000in}{0.150000in}}{\pgfqpoint{5.700000in}{5.700000in}}%
\pgfusepath{clip}%
\pgfsetbuttcap%
\pgfsetroundjoin%
\definecolor{currentfill}{rgb}{0.283229,0.120777,0.440584}%
\pgfsetfillcolor{currentfill}%
\pgfsetfillopacity{0.700000}%
\pgfsetlinewidth{0.000000pt}%
\definecolor{currentstroke}{rgb}{0.000000,0.000000,0.000000}%
\pgfsetstrokecolor{currentstroke}%
\pgfsetdash{}{0pt}%
\pgfpathmoveto{\pgfqpoint{5.101063in}{2.204291in}}%
\pgfpathlineto{\pgfqpoint{5.115062in}{2.202992in}}%
\pgfpathlineto{\pgfqpoint{5.129068in}{2.201717in}}%
\pgfpathlineto{\pgfqpoint{5.143083in}{2.200468in}}%
\pgfpathlineto{\pgfqpoint{5.157107in}{2.199243in}}%
\pgfpathlineto{\pgfqpoint{5.149565in}{2.190517in}}%
\pgfpathlineto{\pgfqpoint{5.142016in}{2.181715in}}%
\pgfpathlineto{\pgfqpoint{5.134460in}{2.172837in}}%
\pgfpathlineto{\pgfqpoint{5.126897in}{2.163886in}}%
\pgfpathlineto{\pgfqpoint{5.112863in}{2.165166in}}%
\pgfpathlineto{\pgfqpoint{5.098838in}{2.166471in}}%
\pgfpathlineto{\pgfqpoint{5.084820in}{2.167801in}}%
\pgfpathlineto{\pgfqpoint{5.070811in}{2.169155in}}%
\pgfpathlineto{\pgfqpoint{5.078384in}{2.178046in}}%
\pgfpathlineto{\pgfqpoint{5.085950in}{2.186866in}}%
\pgfpathlineto{\pgfqpoint{5.093510in}{2.195615in}}%
\pgfpathlineto{\pgfqpoint{5.101063in}{2.204291in}}%
\pgfpathclose%
\pgfusepath{fill}%
\end{pgfscope}%
\begin{pgfscope}%
\pgfpathrectangle{\pgfqpoint{1.150000in}{0.150000in}}{\pgfqpoint{5.700000in}{5.700000in}}%
\pgfusepath{clip}%
\pgfsetbuttcap%
\pgfsetroundjoin%
\definecolor{currentfill}{rgb}{0.267004,0.004874,0.329415}%
\pgfsetfillcolor{currentfill}%
\pgfsetfillopacity{0.700000}%
\pgfsetlinewidth{0.000000pt}%
\definecolor{currentstroke}{rgb}{0.000000,0.000000,0.000000}%
\pgfsetstrokecolor{currentstroke}%
\pgfsetdash{}{0pt}%
\pgfpathmoveto{\pgfqpoint{4.016593in}{1.996149in}}%
\pgfpathlineto{\pgfqpoint{4.030294in}{1.992252in}}%
\pgfpathlineto{\pgfqpoint{4.044001in}{1.988382in}}%
\pgfpathlineto{\pgfqpoint{4.057715in}{1.984539in}}%
\pgfpathlineto{\pgfqpoint{4.071435in}{1.980723in}}%
\pgfpathlineto{\pgfqpoint{4.063506in}{1.973156in}}%
\pgfpathlineto{\pgfqpoint{4.055571in}{1.965667in}}%
\pgfpathlineto{\pgfqpoint{4.047630in}{1.958261in}}%
\pgfpathlineto{\pgfqpoint{4.039683in}{1.950944in}}%
\pgfpathlineto{\pgfqpoint{4.025949in}{1.954948in}}%
\pgfpathlineto{\pgfqpoint{4.012221in}{1.958980in}}%
\pgfpathlineto{\pgfqpoint{3.998500in}{1.963039in}}%
\pgfpathlineto{\pgfqpoint{3.984785in}{1.967124in}}%
\pgfpathlineto{\pgfqpoint{3.992746in}{1.974248in}}%
\pgfpathlineto{\pgfqpoint{4.000702in}{1.981463in}}%
\pgfpathlineto{\pgfqpoint{4.008650in}{1.988765in}}%
\pgfpathlineto{\pgfqpoint{4.016593in}{1.996149in}}%
\pgfpathclose%
\pgfusepath{fill}%
\end{pgfscope}%
\begin{pgfscope}%
\pgfpathrectangle{\pgfqpoint{1.150000in}{0.150000in}}{\pgfqpoint{5.700000in}{5.700000in}}%
\pgfusepath{clip}%
\pgfsetbuttcap%
\pgfsetroundjoin%
\definecolor{currentfill}{rgb}{0.280868,0.160771,0.472899}%
\pgfsetfillcolor{currentfill}%
\pgfsetfillopacity{0.700000}%
\pgfsetlinewidth{0.000000pt}%
\definecolor{currentstroke}{rgb}{0.000000,0.000000,0.000000}%
\pgfsetstrokecolor{currentstroke}%
\pgfsetdash{}{0pt}%
\pgfpathmoveto{\pgfqpoint{5.415588in}{2.286080in}}%
\pgfpathlineto{\pgfqpoint{5.429687in}{2.285261in}}%
\pgfpathlineto{\pgfqpoint{5.443795in}{2.284467in}}%
\pgfpathlineto{\pgfqpoint{5.457911in}{2.283697in}}%
\pgfpathlineto{\pgfqpoint{5.472037in}{2.282953in}}%
\pgfpathlineto{\pgfqpoint{5.464629in}{2.275138in}}%
\pgfpathlineto{\pgfqpoint{5.457214in}{2.267230in}}%
\pgfpathlineto{\pgfqpoint{5.449790in}{2.259228in}}%
\pgfpathlineto{\pgfqpoint{5.442359in}{2.251133in}}%
\pgfpathlineto{\pgfqpoint{5.428222in}{2.251892in}}%
\pgfpathlineto{\pgfqpoint{5.414093in}{2.252675in}}%
\pgfpathlineto{\pgfqpoint{5.399973in}{2.253484in}}%
\pgfpathlineto{\pgfqpoint{5.385861in}{2.254317in}}%
\pgfpathlineto{\pgfqpoint{5.393305in}{2.262392in}}%
\pgfpathlineto{\pgfqpoint{5.400740in}{2.270378in}}%
\pgfpathlineto{\pgfqpoint{5.408168in}{2.278274in}}%
\pgfpathlineto{\pgfqpoint{5.415588in}{2.286080in}}%
\pgfpathclose%
\pgfusepath{fill}%
\end{pgfscope}%
\begin{pgfscope}%
\pgfpathrectangle{\pgfqpoint{1.150000in}{0.150000in}}{\pgfqpoint{5.700000in}{5.700000in}}%
\pgfusepath{clip}%
\pgfsetbuttcap%
\pgfsetroundjoin%
\definecolor{currentfill}{rgb}{0.278791,0.062145,0.386592}%
\pgfsetfillcolor{currentfill}%
\pgfsetfillopacity{0.700000}%
\pgfsetlinewidth{0.000000pt}%
\definecolor{currentstroke}{rgb}{0.000000,0.000000,0.000000}%
\pgfsetstrokecolor{currentstroke}%
\pgfsetdash{}{0pt}%
\pgfpathmoveto{\pgfqpoint{4.700332in}{2.094736in}}%
\pgfpathlineto{\pgfqpoint{4.714214in}{2.092633in}}%
\pgfpathlineto{\pgfqpoint{4.728103in}{2.090554in}}%
\pgfpathlineto{\pgfqpoint{4.741999in}{2.088501in}}%
\pgfpathlineto{\pgfqpoint{4.755904in}{2.086473in}}%
\pgfpathlineto{\pgfqpoint{4.748211in}{2.077214in}}%
\pgfpathlineto{\pgfqpoint{4.740512in}{2.067922in}}%
\pgfpathlineto{\pgfqpoint{4.732807in}{2.058598in}}%
\pgfpathlineto{\pgfqpoint{4.725097in}{2.049246in}}%
\pgfpathlineto{\pgfqpoint{4.711183in}{2.051383in}}%
\pgfpathlineto{\pgfqpoint{4.697276in}{2.053545in}}%
\pgfpathlineto{\pgfqpoint{4.683377in}{2.055733in}}%
\pgfpathlineto{\pgfqpoint{4.669485in}{2.057946in}}%
\pgfpathlineto{\pgfqpoint{4.677206in}{2.067184in}}%
\pgfpathlineto{\pgfqpoint{4.684920in}{2.076396in}}%
\pgfpathlineto{\pgfqpoint{4.692629in}{2.085581in}}%
\pgfpathlineto{\pgfqpoint{4.700332in}{2.094736in}}%
\pgfpathclose%
\pgfusepath{fill}%
\end{pgfscope}%
\begin{pgfscope}%
\pgfpathrectangle{\pgfqpoint{1.150000in}{0.150000in}}{\pgfqpoint{5.700000in}{5.700000in}}%
\pgfusepath{clip}%
\pgfsetbuttcap%
\pgfsetroundjoin%
\definecolor{currentfill}{rgb}{0.282910,0.105393,0.426902}%
\pgfsetfillcolor{currentfill}%
\pgfsetfillopacity{0.700000}%
\pgfsetlinewidth{0.000000pt}%
\definecolor{currentstroke}{rgb}{0.000000,0.000000,0.000000}%
\pgfsetstrokecolor{currentstroke}%
\pgfsetdash{}{0pt}%
\pgfpathmoveto{\pgfqpoint{3.113544in}{2.163853in}}%
\pgfpathlineto{\pgfqpoint{3.127093in}{2.157046in}}%
\pgfpathlineto{\pgfqpoint{3.140646in}{2.150274in}}%
\pgfpathlineto{\pgfqpoint{3.154203in}{2.143537in}}%
\pgfpathlineto{\pgfqpoint{3.167764in}{2.136833in}}%
\pgfpathlineto{\pgfqpoint{3.159375in}{2.136050in}}%
\pgfpathlineto{\pgfqpoint{3.150972in}{2.135528in}}%
\pgfpathlineto{\pgfqpoint{3.142557in}{2.135272in}}%
\pgfpathlineto{\pgfqpoint{3.134128in}{2.135291in}}%
\pgfpathlineto{\pgfqpoint{3.120540in}{2.142265in}}%
\pgfpathlineto{\pgfqpoint{3.106956in}{2.149273in}}%
\pgfpathlineto{\pgfqpoint{3.093376in}{2.156316in}}%
\pgfpathlineto{\pgfqpoint{3.079800in}{2.163394in}}%
\pgfpathlineto{\pgfqpoint{3.088257in}{2.163099in}}%
\pgfpathlineto{\pgfqpoint{3.096700in}{2.163082in}}%
\pgfpathlineto{\pgfqpoint{3.105129in}{2.163336in}}%
\pgfpathlineto{\pgfqpoint{3.113544in}{2.163853in}}%
\pgfpathclose%
\pgfusepath{fill}%
\end{pgfscope}%
\begin{pgfscope}%
\pgfpathrectangle{\pgfqpoint{1.150000in}{0.150000in}}{\pgfqpoint{5.700000in}{5.700000in}}%
\pgfusepath{clip}%
\pgfsetbuttcap%
\pgfsetroundjoin%
\definecolor{currentfill}{rgb}{0.274128,0.199721,0.498911}%
\pgfsetfillcolor{currentfill}%
\pgfsetfillopacity{0.700000}%
\pgfsetlinewidth{0.000000pt}%
\definecolor{currentstroke}{rgb}{0.000000,0.000000,0.000000}%
\pgfsetstrokecolor{currentstroke}%
\pgfsetdash{}{0pt}%
\pgfpathmoveto{\pgfqpoint{5.730006in}{2.362164in}}%
\pgfpathlineto{\pgfqpoint{5.744205in}{2.361699in}}%
\pgfpathlineto{\pgfqpoint{5.758414in}{2.361258in}}%
\pgfpathlineto{\pgfqpoint{5.772631in}{2.360842in}}%
\pgfpathlineto{\pgfqpoint{5.786858in}{2.360450in}}%
\pgfpathlineto{\pgfqpoint{5.779606in}{2.353834in}}%
\pgfpathlineto{\pgfqpoint{5.772345in}{2.347120in}}%
\pgfpathlineto{\pgfqpoint{5.765075in}{2.340308in}}%
\pgfpathlineto{\pgfqpoint{5.757797in}{2.333396in}}%
\pgfpathlineto{\pgfqpoint{5.743555in}{2.333760in}}%
\pgfpathlineto{\pgfqpoint{5.729322in}{2.334148in}}%
\pgfpathlineto{\pgfqpoint{5.715099in}{2.334561in}}%
\pgfpathlineto{\pgfqpoint{5.700884in}{2.334999in}}%
\pgfpathlineto{\pgfqpoint{5.708178in}{2.341933in}}%
\pgfpathlineto{\pgfqpoint{5.715462in}{2.348771in}}%
\pgfpathlineto{\pgfqpoint{5.722738in}{2.355515in}}%
\pgfpathlineto{\pgfqpoint{5.730006in}{2.362164in}}%
\pgfpathclose%
\pgfusepath{fill}%
\end{pgfscope}%
\begin{pgfscope}%
\pgfpathrectangle{\pgfqpoint{1.150000in}{0.150000in}}{\pgfqpoint{5.700000in}{5.700000in}}%
\pgfusepath{clip}%
\pgfsetbuttcap%
\pgfsetroundjoin%
\definecolor{currentfill}{rgb}{0.235526,0.309527,0.542944}%
\pgfsetfillcolor{currentfill}%
\pgfsetfillopacity{0.700000}%
\pgfsetlinewidth{0.000000pt}%
\definecolor{currentstroke}{rgb}{0.000000,0.000000,0.000000}%
\pgfsetstrokecolor{currentstroke}%
\pgfsetdash{}{0pt}%
\pgfpathmoveto{\pgfqpoint{2.360456in}{2.591492in}}%
\pgfpathlineto{\pgfqpoint{2.373971in}{2.581767in}}%
\pgfpathlineto{\pgfqpoint{2.387487in}{2.572095in}}%
\pgfpathlineto{\pgfqpoint{2.401004in}{2.562476in}}%
\pgfpathlineto{\pgfqpoint{2.414523in}{2.552909in}}%
\pgfpathlineto{\pgfqpoint{2.405528in}{2.559474in}}%
\pgfpathlineto{\pgfqpoint{2.396508in}{2.566437in}}%
\pgfpathlineto{\pgfqpoint{2.387464in}{2.573808in}}%
\pgfpathlineto{\pgfqpoint{2.378396in}{2.581594in}}%
\pgfpathlineto{\pgfqpoint{2.364836in}{2.591482in}}%
\pgfpathlineto{\pgfqpoint{2.351277in}{2.601422in}}%
\pgfpathlineto{\pgfqpoint{2.337720in}{2.611415in}}%
\pgfpathlineto{\pgfqpoint{2.324164in}{2.621461in}}%
\pgfpathlineto{\pgfqpoint{2.333275in}{2.613348in}}%
\pgfpathlineto{\pgfqpoint{2.342360in}{2.605655in}}%
\pgfpathlineto{\pgfqpoint{2.351420in}{2.598372in}}%
\pgfpathlineto{\pgfqpoint{2.360456in}{2.591492in}}%
\pgfpathclose%
\pgfusepath{fill}%
\end{pgfscope}%
\begin{pgfscope}%
\pgfpathrectangle{\pgfqpoint{1.150000in}{0.150000in}}{\pgfqpoint{5.700000in}{5.700000in}}%
\pgfusepath{clip}%
\pgfsetbuttcap%
\pgfsetroundjoin%
\definecolor{currentfill}{rgb}{0.282910,0.105393,0.426902}%
\pgfsetfillcolor{currentfill}%
\pgfsetfillopacity{0.700000}%
\pgfsetlinewidth{0.000000pt}%
\definecolor{currentstroke}{rgb}{0.000000,0.000000,0.000000}%
\pgfsetstrokecolor{currentstroke}%
\pgfsetdash{}{0pt}%
\pgfpathmoveto{\pgfqpoint{5.014856in}{2.174823in}}%
\pgfpathlineto{\pgfqpoint{5.028833in}{2.173368in}}%
\pgfpathlineto{\pgfqpoint{5.042817in}{2.171939in}}%
\pgfpathlineto{\pgfqpoint{5.056810in}{2.170535in}}%
\pgfpathlineto{\pgfqpoint{5.070811in}{2.169155in}}%
\pgfpathlineto{\pgfqpoint{5.063231in}{2.160195in}}%
\pgfpathlineto{\pgfqpoint{5.055645in}{2.151166in}}%
\pgfpathlineto{\pgfqpoint{5.048053in}{2.142070in}}%
\pgfpathlineto{\pgfqpoint{5.040454in}{2.132908in}}%
\pgfpathlineto{\pgfqpoint{5.026443in}{2.134357in}}%
\pgfpathlineto{\pgfqpoint{5.012440in}{2.135830in}}%
\pgfpathlineto{\pgfqpoint{4.998445in}{2.137329in}}%
\pgfpathlineto{\pgfqpoint{4.984458in}{2.138852in}}%
\pgfpathlineto{\pgfqpoint{4.992067in}{2.147939in}}%
\pgfpathlineto{\pgfqpoint{4.999670in}{2.156965in}}%
\pgfpathlineto{\pgfqpoint{5.007266in}{2.165926in}}%
\pgfpathlineto{\pgfqpoint{5.014856in}{2.174823in}}%
\pgfpathclose%
\pgfusepath{fill}%
\end{pgfscope}%
\begin{pgfscope}%
\pgfpathrectangle{\pgfqpoint{1.150000in}{0.150000in}}{\pgfqpoint{5.700000in}{5.700000in}}%
\pgfusepath{clip}%
\pgfsetbuttcap%
\pgfsetroundjoin%
\definecolor{currentfill}{rgb}{0.279566,0.067836,0.391917}%
\pgfsetfillcolor{currentfill}%
\pgfsetfillopacity{0.700000}%
\pgfsetlinewidth{0.000000pt}%
\definecolor{currentstroke}{rgb}{0.000000,0.000000,0.000000}%
\pgfsetstrokecolor{currentstroke}%
\pgfsetdash{}{0pt}%
\pgfpathmoveto{\pgfqpoint{3.309635in}{2.092064in}}%
\pgfpathlineto{\pgfqpoint{3.323210in}{2.085916in}}%
\pgfpathlineto{\pgfqpoint{3.336790in}{2.079800in}}%
\pgfpathlineto{\pgfqpoint{3.350374in}{2.073717in}}%
\pgfpathlineto{\pgfqpoint{3.363963in}{2.067665in}}%
\pgfpathlineto{\pgfqpoint{3.355695in}{2.065156in}}%
\pgfpathlineto{\pgfqpoint{3.347416in}{2.062870in}}%
\pgfpathlineto{\pgfqpoint{3.339127in}{2.060813in}}%
\pgfpathlineto{\pgfqpoint{3.330825in}{2.058992in}}%
\pgfpathlineto{\pgfqpoint{3.317213in}{2.065300in}}%
\pgfpathlineto{\pgfqpoint{3.303605in}{2.071640in}}%
\pgfpathlineto{\pgfqpoint{3.290001in}{2.078012in}}%
\pgfpathlineto{\pgfqpoint{3.276402in}{2.084416in}}%
\pgfpathlineto{\pgfqpoint{3.284728in}{2.085975in}}%
\pgfpathlineto{\pgfqpoint{3.293042in}{2.087774in}}%
\pgfpathlineto{\pgfqpoint{3.301344in}{2.089806in}}%
\pgfpathlineto{\pgfqpoint{3.309635in}{2.092064in}}%
\pgfpathclose%
\pgfusepath{fill}%
\end{pgfscope}%
\begin{pgfscope}%
\pgfpathrectangle{\pgfqpoint{1.150000in}{0.150000in}}{\pgfqpoint{5.700000in}{5.700000in}}%
\pgfusepath{clip}%
\pgfsetbuttcap%
\pgfsetroundjoin%
\definecolor{currentfill}{rgb}{0.271305,0.019942,0.347269}%
\pgfsetfillcolor{currentfill}%
\pgfsetfillopacity{0.700000}%
\pgfsetlinewidth{0.000000pt}%
\definecolor{currentstroke}{rgb}{0.000000,0.000000,0.000000}%
\pgfsetstrokecolor{currentstroke}%
\pgfsetdash{}{0pt}%
\pgfpathmoveto{\pgfqpoint{4.385930in}{2.026465in}}%
\pgfpathlineto{\pgfqpoint{4.399726in}{2.023584in}}%
\pgfpathlineto{\pgfqpoint{4.413529in}{2.020729in}}%
\pgfpathlineto{\pgfqpoint{4.427340in}{2.017900in}}%
\pgfpathlineto{\pgfqpoint{4.441157in}{2.015096in}}%
\pgfpathlineto{\pgfqpoint{4.433357in}{2.006189in}}%
\pgfpathlineto{\pgfqpoint{4.425550in}{1.997296in}}%
\pgfpathlineto{\pgfqpoint{4.417739in}{1.988420in}}%
\pgfpathlineto{\pgfqpoint{4.409922in}{1.979566in}}%
\pgfpathlineto{\pgfqpoint{4.396094in}{1.982518in}}%
\pgfpathlineto{\pgfqpoint{4.382272in}{1.985497in}}%
\pgfpathlineto{\pgfqpoint{4.368458in}{1.988501in}}%
\pgfpathlineto{\pgfqpoint{4.354650in}{1.991531in}}%
\pgfpathlineto{\pgfqpoint{4.362478in}{2.000231in}}%
\pgfpathlineto{\pgfqpoint{4.370301in}{2.008955in}}%
\pgfpathlineto{\pgfqpoint{4.378118in}{2.017701in}}%
\pgfpathlineto{\pgfqpoint{4.385930in}{2.026465in}}%
\pgfpathclose%
\pgfusepath{fill}%
\end{pgfscope}%
\begin{pgfscope}%
\pgfpathrectangle{\pgfqpoint{1.150000in}{0.150000in}}{\pgfqpoint{5.700000in}{5.700000in}}%
\pgfusepath{clip}%
\pgfsetbuttcap%
\pgfsetroundjoin%
\definecolor{currentfill}{rgb}{0.270595,0.214069,0.507052}%
\pgfsetfillcolor{currentfill}%
\pgfsetfillopacity{0.700000}%
\pgfsetlinewidth{0.000000pt}%
\definecolor{currentstroke}{rgb}{0.000000,0.000000,0.000000}%
\pgfsetstrokecolor{currentstroke}%
\pgfsetdash{}{0pt}%
\pgfpathmoveto{\pgfqpoint{2.666234in}{2.388920in}}%
\pgfpathlineto{\pgfqpoint{2.679750in}{2.380465in}}%
\pgfpathlineto{\pgfqpoint{2.693270in}{2.372054in}}%
\pgfpathlineto{\pgfqpoint{2.706792in}{2.363687in}}%
\pgfpathlineto{\pgfqpoint{2.720317in}{2.355362in}}%
\pgfpathlineto{\pgfqpoint{2.711591in}{2.358930in}}%
\pgfpathlineto{\pgfqpoint{2.702846in}{2.362843in}}%
\pgfpathlineto{\pgfqpoint{2.694082in}{2.367111in}}%
\pgfpathlineto{\pgfqpoint{2.685297in}{2.371741in}}%
\pgfpathlineto{\pgfqpoint{2.671737in}{2.380368in}}%
\pgfpathlineto{\pgfqpoint{2.658180in}{2.389038in}}%
\pgfpathlineto{\pgfqpoint{2.644625in}{2.397751in}}%
\pgfpathlineto{\pgfqpoint{2.631073in}{2.406508in}}%
\pgfpathlineto{\pgfqpoint{2.639893in}{2.401570in}}%
\pgfpathlineto{\pgfqpoint{2.648694in}{2.396999in}}%
\pgfpathlineto{\pgfqpoint{2.657473in}{2.392785in}}%
\pgfpathlineto{\pgfqpoint{2.666234in}{2.388920in}}%
\pgfpathclose%
\pgfusepath{fill}%
\end{pgfscope}%
\begin{pgfscope}%
\pgfpathrectangle{\pgfqpoint{1.150000in}{0.150000in}}{\pgfqpoint{5.700000in}{5.700000in}}%
\pgfusepath{clip}%
\pgfsetbuttcap%
\pgfsetroundjoin%
\definecolor{currentfill}{rgb}{0.268510,0.009605,0.335427}%
\pgfsetfillcolor{currentfill}%
\pgfsetfillopacity{0.700000}%
\pgfsetlinewidth{0.000000pt}%
\definecolor{currentstroke}{rgb}{0.000000,0.000000,0.000000}%
\pgfsetstrokecolor{currentstroke}%
\pgfsetdash{}{0pt}%
\pgfpathmoveto{\pgfqpoint{4.157981in}{1.997396in}}%
\pgfpathlineto{\pgfqpoint{4.171719in}{1.993889in}}%
\pgfpathlineto{\pgfqpoint{4.185465in}{1.990409in}}%
\pgfpathlineto{\pgfqpoint{4.199217in}{1.986955in}}%
\pgfpathlineto{\pgfqpoint{4.212975in}{1.983527in}}%
\pgfpathlineto{\pgfqpoint{4.205096in}{1.975349in}}%
\pgfpathlineto{\pgfqpoint{4.197210in}{1.967224in}}%
\pgfpathlineto{\pgfqpoint{4.189320in}{1.959159in}}%
\pgfpathlineto{\pgfqpoint{4.181423in}{1.951157in}}%
\pgfpathlineto{\pgfqpoint{4.167652in}{1.954760in}}%
\pgfpathlineto{\pgfqpoint{4.153887in}{1.958389in}}%
\pgfpathlineto{\pgfqpoint{4.140129in}{1.962045in}}%
\pgfpathlineto{\pgfqpoint{4.126377in}{1.965727in}}%
\pgfpathlineto{\pgfqpoint{4.134287in}{1.973549in}}%
\pgfpathlineto{\pgfqpoint{4.142191in}{1.981437in}}%
\pgfpathlineto{\pgfqpoint{4.150088in}{1.989388in}}%
\pgfpathlineto{\pgfqpoint{4.157981in}{1.997396in}}%
\pgfpathclose%
\pgfusepath{fill}%
\end{pgfscope}%
\begin{pgfscope}%
\pgfpathrectangle{\pgfqpoint{1.150000in}{0.150000in}}{\pgfqpoint{5.700000in}{5.700000in}}%
\pgfusepath{clip}%
\pgfsetbuttcap%
\pgfsetroundjoin%
\definecolor{currentfill}{rgb}{0.281887,0.150881,0.465405}%
\pgfsetfillcolor{currentfill}%
\pgfsetfillopacity{0.700000}%
\pgfsetlinewidth{0.000000pt}%
\definecolor{currentstroke}{rgb}{0.000000,0.000000,0.000000}%
\pgfsetstrokecolor{currentstroke}%
\pgfsetdash{}{0pt}%
\pgfpathmoveto{\pgfqpoint{2.917175in}{2.251131in}}%
\pgfpathlineto{\pgfqpoint{2.930707in}{2.243616in}}%
\pgfpathlineto{\pgfqpoint{2.944243in}{2.236139in}}%
\pgfpathlineto{\pgfqpoint{2.957783in}{2.228699in}}%
\pgfpathlineto{\pgfqpoint{2.971326in}{2.221297in}}%
\pgfpathlineto{\pgfqpoint{2.962796in}{2.222437in}}%
\pgfpathlineto{\pgfqpoint{2.954251in}{2.223878in}}%
\pgfpathlineto{\pgfqpoint{2.945690in}{2.225626in}}%
\pgfpathlineto{\pgfqpoint{2.937112in}{2.227689in}}%
\pgfpathlineto{\pgfqpoint{2.923539in}{2.235376in}}%
\pgfpathlineto{\pgfqpoint{2.909969in}{2.243102in}}%
\pgfpathlineto{\pgfqpoint{2.896402in}{2.250865in}}%
\pgfpathlineto{\pgfqpoint{2.882839in}{2.258666in}}%
\pgfpathlineto{\pgfqpoint{2.891447in}{2.256312in}}%
\pgfpathlineto{\pgfqpoint{2.900040in}{2.254276in}}%
\pgfpathlineto{\pgfqpoint{2.908615in}{2.252552in}}%
\pgfpathlineto{\pgfqpoint{2.917175in}{2.251131in}}%
\pgfpathclose%
\pgfusepath{fill}%
\end{pgfscope}%
\begin{pgfscope}%
\pgfpathrectangle{\pgfqpoint{1.150000in}{0.150000in}}{\pgfqpoint{5.700000in}{5.700000in}}%
\pgfusepath{clip}%
\pgfsetbuttcap%
\pgfsetroundjoin%
\definecolor{currentfill}{rgb}{0.281887,0.150881,0.465405}%
\pgfsetfillcolor{currentfill}%
\pgfsetfillopacity{0.700000}%
\pgfsetlinewidth{0.000000pt}%
\definecolor{currentstroke}{rgb}{0.000000,0.000000,0.000000}%
\pgfsetstrokecolor{currentstroke}%
\pgfsetdash{}{0pt}%
\pgfpathmoveto{\pgfqpoint{5.329503in}{2.257896in}}%
\pgfpathlineto{\pgfqpoint{5.343579in}{2.256964in}}%
\pgfpathlineto{\pgfqpoint{5.357665in}{2.256057in}}%
\pgfpathlineto{\pgfqpoint{5.371759in}{2.255174in}}%
\pgfpathlineto{\pgfqpoint{5.385861in}{2.254317in}}%
\pgfpathlineto{\pgfqpoint{5.378411in}{2.246151in}}%
\pgfpathlineto{\pgfqpoint{5.370953in}{2.237895in}}%
\pgfpathlineto{\pgfqpoint{5.363487in}{2.229549in}}%
\pgfpathlineto{\pgfqpoint{5.356014in}{2.221113in}}%
\pgfpathlineto{\pgfqpoint{5.341899in}{2.221999in}}%
\pgfpathlineto{\pgfqpoint{5.327794in}{2.222909in}}%
\pgfpathlineto{\pgfqpoint{5.313696in}{2.223844in}}%
\pgfpathlineto{\pgfqpoint{5.299608in}{2.224804in}}%
\pgfpathlineto{\pgfqpoint{5.307093in}{2.233206in}}%
\pgfpathlineto{\pgfqpoint{5.314570in}{2.241523in}}%
\pgfpathlineto{\pgfqpoint{5.322040in}{2.249752in}}%
\pgfpathlineto{\pgfqpoint{5.329503in}{2.257896in}}%
\pgfpathclose%
\pgfusepath{fill}%
\end{pgfscope}%
\begin{pgfscope}%
\pgfpathrectangle{\pgfqpoint{1.150000in}{0.150000in}}{\pgfqpoint{5.700000in}{5.700000in}}%
\pgfusepath{clip}%
\pgfsetbuttcap%
\pgfsetroundjoin%
\definecolor{currentfill}{rgb}{0.268510,0.009605,0.335427}%
\pgfsetfillcolor{currentfill}%
\pgfsetfillopacity{0.700000}%
\pgfsetlinewidth{0.000000pt}%
\definecolor{currentstroke}{rgb}{0.000000,0.000000,0.000000}%
\pgfsetstrokecolor{currentstroke}%
\pgfsetdash{}{0pt}%
\pgfpathmoveto{\pgfqpoint{3.788575in}{1.993233in}}%
\pgfpathlineto{\pgfqpoint{3.802235in}{1.988614in}}%
\pgfpathlineto{\pgfqpoint{3.815900in}{1.984022in}}%
\pgfpathlineto{\pgfqpoint{3.829571in}{1.979459in}}%
\pgfpathlineto{\pgfqpoint{3.843248in}{1.974924in}}%
\pgfpathlineto{\pgfqpoint{3.835222in}{1.968750in}}%
\pgfpathlineto{\pgfqpoint{3.827190in}{1.962704in}}%
\pgfpathlineto{\pgfqpoint{3.819149in}{1.956791in}}%
\pgfpathlineto{\pgfqpoint{3.811102in}{1.951017in}}%
\pgfpathlineto{\pgfqpoint{3.797408in}{1.955767in}}%
\pgfpathlineto{\pgfqpoint{3.783720in}{1.960546in}}%
\pgfpathlineto{\pgfqpoint{3.770038in}{1.965352in}}%
\pgfpathlineto{\pgfqpoint{3.756361in}{1.970187in}}%
\pgfpathlineto{\pgfqpoint{3.764426in}{1.975741in}}%
\pgfpathlineto{\pgfqpoint{3.772483in}{1.981437in}}%
\pgfpathlineto{\pgfqpoint{3.780533in}{1.987269in}}%
\pgfpathlineto{\pgfqpoint{3.788575in}{1.993233in}}%
\pgfpathclose%
\pgfusepath{fill}%
\end{pgfscope}%
\begin{pgfscope}%
\pgfpathrectangle{\pgfqpoint{1.150000in}{0.150000in}}{\pgfqpoint{5.700000in}{5.700000in}}%
\pgfusepath{clip}%
\pgfsetbuttcap%
\pgfsetroundjoin%
\definecolor{currentfill}{rgb}{0.277018,0.050344,0.375715}%
\pgfsetfillcolor{currentfill}%
\pgfsetfillopacity{0.700000}%
\pgfsetlinewidth{0.000000pt}%
\definecolor{currentstroke}{rgb}{0.000000,0.000000,0.000000}%
\pgfsetstrokecolor{currentstroke}%
\pgfsetdash{}{0pt}%
\pgfpathmoveto{\pgfqpoint{4.613994in}{2.067051in}}%
\pgfpathlineto{\pgfqpoint{4.627856in}{2.064736in}}%
\pgfpathlineto{\pgfqpoint{4.641725in}{2.062447in}}%
\pgfpathlineto{\pgfqpoint{4.655601in}{2.060184in}}%
\pgfpathlineto{\pgfqpoint{4.669485in}{2.057946in}}%
\pgfpathlineto{\pgfqpoint{4.661760in}{2.048685in}}%
\pgfpathlineto{\pgfqpoint{4.654029in}{2.039404in}}%
\pgfpathlineto{\pgfqpoint{4.646292in}{2.030105in}}%
\pgfpathlineto{\pgfqpoint{4.638550in}{2.020792in}}%
\pgfpathlineto{\pgfqpoint{4.624656in}{2.023153in}}%
\pgfpathlineto{\pgfqpoint{4.610769in}{2.025539in}}%
\pgfpathlineto{\pgfqpoint{4.596890in}{2.027951in}}%
\pgfpathlineto{\pgfqpoint{4.583018in}{2.030388in}}%
\pgfpathlineto{\pgfqpoint{4.590770in}{2.039573in}}%
\pgfpathlineto{\pgfqpoint{4.598517in}{2.048747in}}%
\pgfpathlineto{\pgfqpoint{4.606258in}{2.057907in}}%
\pgfpathlineto{\pgfqpoint{4.613994in}{2.067051in}}%
\pgfpathclose%
\pgfusepath{fill}%
\end{pgfscope}%
\begin{pgfscope}%
\pgfpathrectangle{\pgfqpoint{1.150000in}{0.150000in}}{\pgfqpoint{5.700000in}{5.700000in}}%
\pgfusepath{clip}%
\pgfsetbuttcap%
\pgfsetroundjoin%
\definecolor{currentfill}{rgb}{0.271305,0.019942,0.347269}%
\pgfsetfillcolor{currentfill}%
\pgfsetfillopacity{0.700000}%
\pgfsetlinewidth{0.000000pt}%
\definecolor{currentstroke}{rgb}{0.000000,0.000000,0.000000}%
\pgfsetstrokecolor{currentstroke}%
\pgfsetdash{}{0pt}%
\pgfpathmoveto{\pgfqpoint{3.647147in}{2.009893in}}%
\pgfpathlineto{\pgfqpoint{3.660780in}{2.004829in}}%
\pgfpathlineto{\pgfqpoint{3.674418in}{1.999794in}}%
\pgfpathlineto{\pgfqpoint{3.688062in}{1.994787in}}%
\pgfpathlineto{\pgfqpoint{3.701711in}{1.989810in}}%
\pgfpathlineto{\pgfqpoint{3.693620in}{1.984628in}}%
\pgfpathlineto{\pgfqpoint{3.685522in}{1.979602in}}%
\pgfpathlineto{\pgfqpoint{3.677415in}{1.974738in}}%
\pgfpathlineto{\pgfqpoint{3.669300in}{1.970043in}}%
\pgfpathlineto{\pgfqpoint{3.655632in}{1.975249in}}%
\pgfpathlineto{\pgfqpoint{3.641970in}{1.980484in}}%
\pgfpathlineto{\pgfqpoint{3.628313in}{1.985747in}}%
\pgfpathlineto{\pgfqpoint{3.614662in}{1.991040in}}%
\pgfpathlineto{\pgfqpoint{3.622796in}{1.995502in}}%
\pgfpathlineto{\pgfqpoint{3.630921in}{2.000135in}}%
\pgfpathlineto{\pgfqpoint{3.639038in}{2.004934in}}%
\pgfpathlineto{\pgfqpoint{3.647147in}{2.009893in}}%
\pgfpathclose%
\pgfusepath{fill}%
\end{pgfscope}%
\begin{pgfscope}%
\pgfpathrectangle{\pgfqpoint{1.150000in}{0.150000in}}{\pgfqpoint{5.700000in}{5.700000in}}%
\pgfusepath{clip}%
\pgfsetbuttcap%
\pgfsetroundjoin%
\definecolor{currentfill}{rgb}{0.276194,0.190074,0.493001}%
\pgfsetfillcolor{currentfill}%
\pgfsetfillopacity{0.700000}%
\pgfsetlinewidth{0.000000pt}%
\definecolor{currentstroke}{rgb}{0.000000,0.000000,0.000000}%
\pgfsetstrokecolor{currentstroke}%
\pgfsetdash{}{0pt}%
\pgfpathmoveto{\pgfqpoint{5.644118in}{2.336996in}}%
\pgfpathlineto{\pgfqpoint{5.658296in}{2.336460in}}%
\pgfpathlineto{\pgfqpoint{5.672483in}{2.335948in}}%
\pgfpathlineto{\pgfqpoint{5.686679in}{2.335461in}}%
\pgfpathlineto{\pgfqpoint{5.700884in}{2.334999in}}%
\pgfpathlineto{\pgfqpoint{5.693583in}{2.327967in}}%
\pgfpathlineto{\pgfqpoint{5.686273in}{2.320837in}}%
\pgfpathlineto{\pgfqpoint{5.678954in}{2.313607in}}%
\pgfpathlineto{\pgfqpoint{5.671627in}{2.306277in}}%
\pgfpathlineto{\pgfqpoint{5.657407in}{2.306726in}}%
\pgfpathlineto{\pgfqpoint{5.643197in}{2.307199in}}%
\pgfpathlineto{\pgfqpoint{5.628996in}{2.307697in}}%
\pgfpathlineto{\pgfqpoint{5.614804in}{2.308220in}}%
\pgfpathlineto{\pgfqpoint{5.622145in}{2.315558in}}%
\pgfpathlineto{\pgfqpoint{5.629477in}{2.322800in}}%
\pgfpathlineto{\pgfqpoint{5.636802in}{2.329946in}}%
\pgfpathlineto{\pgfqpoint{5.644118in}{2.336996in}}%
\pgfpathclose%
\pgfusepath{fill}%
\end{pgfscope}%
\begin{pgfscope}%
\pgfpathrectangle{\pgfqpoint{1.150000in}{0.150000in}}{\pgfqpoint{5.700000in}{5.700000in}}%
\pgfusepath{clip}%
\pgfsetbuttcap%
\pgfsetroundjoin%
\definecolor{currentfill}{rgb}{0.282327,0.094955,0.417331}%
\pgfsetfillcolor{currentfill}%
\pgfsetfillopacity{0.700000}%
\pgfsetlinewidth{0.000000pt}%
\definecolor{currentstroke}{rgb}{0.000000,0.000000,0.000000}%
\pgfsetstrokecolor{currentstroke}%
\pgfsetdash{}{0pt}%
\pgfpathmoveto{\pgfqpoint{4.928591in}{2.145195in}}%
\pgfpathlineto{\pgfqpoint{4.942546in}{2.143571in}}%
\pgfpathlineto{\pgfqpoint{4.956509in}{2.141973in}}%
\pgfpathlineto{\pgfqpoint{4.970479in}{2.140400in}}%
\pgfpathlineto{\pgfqpoint{4.984458in}{2.138852in}}%
\pgfpathlineto{\pgfqpoint{4.976843in}{2.129703in}}%
\pgfpathlineto{\pgfqpoint{4.969221in}{2.120495in}}%
\pgfpathlineto{\pgfqpoint{4.961593in}{2.111230in}}%
\pgfpathlineto{\pgfqpoint{4.953959in}{2.101908in}}%
\pgfpathlineto{\pgfqpoint{4.939971in}{2.103539in}}%
\pgfpathlineto{\pgfqpoint{4.925990in}{2.105195in}}%
\pgfpathlineto{\pgfqpoint{4.912017in}{2.106875in}}%
\pgfpathlineto{\pgfqpoint{4.898053in}{2.108581in}}%
\pgfpathlineto{\pgfqpoint{4.905696in}{2.117815in}}%
\pgfpathlineto{\pgfqpoint{4.913334in}{2.126996in}}%
\pgfpathlineto{\pgfqpoint{4.920966in}{2.136123in}}%
\pgfpathlineto{\pgfqpoint{4.928591in}{2.145195in}}%
\pgfpathclose%
\pgfusepath{fill}%
\end{pgfscope}%
\begin{pgfscope}%
\pgfpathrectangle{\pgfqpoint{1.150000in}{0.150000in}}{\pgfqpoint{5.700000in}{5.700000in}}%
\pgfusepath{clip}%
\pgfsetbuttcap%
\pgfsetroundjoin%
\definecolor{currentfill}{rgb}{0.267004,0.004874,0.329415}%
\pgfsetfillcolor{currentfill}%
\pgfsetfillopacity{0.700000}%
\pgfsetlinewidth{0.000000pt}%
\definecolor{currentstroke}{rgb}{0.000000,0.000000,0.000000}%
\pgfsetstrokecolor{currentstroke}%
\pgfsetdash{}{0pt}%
\pgfpathmoveto{\pgfqpoint{3.929985in}{1.983740in}}%
\pgfpathlineto{\pgfqpoint{3.943676in}{1.979545in}}%
\pgfpathlineto{\pgfqpoint{3.957373in}{1.975377in}}%
\pgfpathlineto{\pgfqpoint{3.971076in}{1.971237in}}%
\pgfpathlineto{\pgfqpoint{3.984785in}{1.967124in}}%
\pgfpathlineto{\pgfqpoint{3.976817in}{1.960096in}}%
\pgfpathlineto{\pgfqpoint{3.968843in}{1.953169in}}%
\pgfpathlineto{\pgfqpoint{3.960862in}{1.946348in}}%
\pgfpathlineto{\pgfqpoint{3.952874in}{1.939637in}}%
\pgfpathlineto{\pgfqpoint{3.939150in}{1.943952in}}%
\pgfpathlineto{\pgfqpoint{3.925432in}{1.948294in}}%
\pgfpathlineto{\pgfqpoint{3.911720in}{1.952664in}}%
\pgfpathlineto{\pgfqpoint{3.898014in}{1.957060in}}%
\pgfpathlineto{\pgfqpoint{3.906017in}{1.963564in}}%
\pgfpathlineto{\pgfqpoint{3.914013in}{1.970182in}}%
\pgfpathlineto{\pgfqpoint{3.922002in}{1.976909in}}%
\pgfpathlineto{\pgfqpoint{3.929985in}{1.983740in}}%
\pgfpathclose%
\pgfusepath{fill}%
\end{pgfscope}%
\begin{pgfscope}%
\pgfpathrectangle{\pgfqpoint{1.150000in}{0.150000in}}{\pgfqpoint{5.700000in}{5.700000in}}%
\pgfusepath{clip}%
\pgfsetbuttcap%
\pgfsetroundjoin%
\definecolor{currentfill}{rgb}{0.273809,0.031497,0.358853}%
\pgfsetfillcolor{currentfill}%
\pgfsetfillopacity{0.700000}%
\pgfsetlinewidth{0.000000pt}%
\definecolor{currentstroke}{rgb}{0.000000,0.000000,0.000000}%
\pgfsetstrokecolor{currentstroke}%
\pgfsetdash{}{0pt}%
\pgfpathmoveto{\pgfqpoint{3.505634in}{2.034447in}}%
\pgfpathlineto{\pgfqpoint{3.519245in}{2.028917in}}%
\pgfpathlineto{\pgfqpoint{3.532861in}{2.023416in}}%
\pgfpathlineto{\pgfqpoint{3.546481in}{2.017946in}}%
\pgfpathlineto{\pgfqpoint{3.560107in}{2.012506in}}%
\pgfpathlineto{\pgfqpoint{3.551944in}{2.008459in}}%
\pgfpathlineto{\pgfqpoint{3.543772in}{2.004599in}}%
\pgfpathlineto{\pgfqpoint{3.535590in}{2.000932in}}%
\pgfpathlineto{\pgfqpoint{3.527399in}{1.997465in}}%
\pgfpathlineto{\pgfqpoint{3.513753in}{2.003148in}}%
\pgfpathlineto{\pgfqpoint{3.500111in}{2.008860in}}%
\pgfpathlineto{\pgfqpoint{3.486474in}{2.014603in}}%
\pgfpathlineto{\pgfqpoint{3.472843in}{2.020376in}}%
\pgfpathlineto{\pgfqpoint{3.481055in}{2.023596in}}%
\pgfpathlineto{\pgfqpoint{3.489258in}{2.027018in}}%
\pgfpathlineto{\pgfqpoint{3.497451in}{2.030638in}}%
\pgfpathlineto{\pgfqpoint{3.505634in}{2.034447in}}%
\pgfpathclose%
\pgfusepath{fill}%
\end{pgfscope}%
\begin{pgfscope}%
\pgfpathrectangle{\pgfqpoint{1.150000in}{0.150000in}}{\pgfqpoint{5.700000in}{5.700000in}}%
\pgfusepath{clip}%
\pgfsetbuttcap%
\pgfsetroundjoin%
\definecolor{currentfill}{rgb}{0.241237,0.296485,0.539709}%
\pgfsetfillcolor{currentfill}%
\pgfsetfillopacity{0.700000}%
\pgfsetlinewidth{0.000000pt}%
\definecolor{currentstroke}{rgb}{0.000000,0.000000,0.000000}%
\pgfsetstrokecolor{currentstroke}%
\pgfsetdash{}{0pt}%
\pgfpathmoveto{\pgfqpoint{2.414523in}{2.552909in}}%
\pgfpathlineto{\pgfqpoint{2.428044in}{2.543394in}}%
\pgfpathlineto{\pgfqpoint{2.441566in}{2.533929in}}%
\pgfpathlineto{\pgfqpoint{2.455090in}{2.524515in}}%
\pgfpathlineto{\pgfqpoint{2.468615in}{2.515151in}}%
\pgfpathlineto{\pgfqpoint{2.459659in}{2.521402in}}%
\pgfpathlineto{\pgfqpoint{2.450680in}{2.528047in}}%
\pgfpathlineto{\pgfqpoint{2.441677in}{2.535095in}}%
\pgfpathlineto{\pgfqpoint{2.432649in}{2.542555in}}%
\pgfpathlineto{\pgfqpoint{2.419083in}{2.552239in}}%
\pgfpathlineto{\pgfqpoint{2.405519in}{2.561973in}}%
\pgfpathlineto{\pgfqpoint{2.391957in}{2.571758in}}%
\pgfpathlineto{\pgfqpoint{2.378396in}{2.581594in}}%
\pgfpathlineto{\pgfqpoint{2.387464in}{2.573808in}}%
\pgfpathlineto{\pgfqpoint{2.396508in}{2.566437in}}%
\pgfpathlineto{\pgfqpoint{2.405528in}{2.559474in}}%
\pgfpathlineto{\pgfqpoint{2.414523in}{2.552909in}}%
\pgfpathclose%
\pgfusepath{fill}%
\end{pgfscope}%
\begin{pgfscope}%
\pgfpathrectangle{\pgfqpoint{1.150000in}{0.150000in}}{\pgfqpoint{5.700000in}{5.700000in}}%
\pgfusepath{clip}%
\pgfsetbuttcap%
\pgfsetroundjoin%
\definecolor{currentfill}{rgb}{0.269944,0.014625,0.341379}%
\pgfsetfillcolor{currentfill}%
\pgfsetfillopacity{0.700000}%
\pgfsetlinewidth{0.000000pt}%
\definecolor{currentstroke}{rgb}{0.000000,0.000000,0.000000}%
\pgfsetstrokecolor{currentstroke}%
\pgfsetdash{}{0pt}%
\pgfpathmoveto{\pgfqpoint{4.299489in}{2.003910in}}%
\pgfpathlineto{\pgfqpoint{4.313269in}{2.000776in}}%
\pgfpathlineto{\pgfqpoint{4.327056in}{1.997668in}}%
\pgfpathlineto{\pgfqpoint{4.340850in}{1.994587in}}%
\pgfpathlineto{\pgfqpoint{4.354650in}{1.991531in}}%
\pgfpathlineto{\pgfqpoint{4.346817in}{1.982859in}}%
\pgfpathlineto{\pgfqpoint{4.338978in}{1.974219in}}%
\pgfpathlineto{\pgfqpoint{4.331134in}{1.965616in}}%
\pgfpathlineto{\pgfqpoint{4.323284in}{1.957052in}}%
\pgfpathlineto{\pgfqpoint{4.309472in}{1.960270in}}%
\pgfpathlineto{\pgfqpoint{4.295666in}{1.963514in}}%
\pgfpathlineto{\pgfqpoint{4.281867in}{1.966784in}}%
\pgfpathlineto{\pgfqpoint{4.268076in}{1.970081in}}%
\pgfpathlineto{\pgfqpoint{4.275937in}{1.978477in}}%
\pgfpathlineto{\pgfqpoint{4.283793in}{1.986917in}}%
\pgfpathlineto{\pgfqpoint{4.291644in}{1.995396in}}%
\pgfpathlineto{\pgfqpoint{4.299489in}{2.003910in}}%
\pgfpathclose%
\pgfusepath{fill}%
\end{pgfscope}%
\begin{pgfscope}%
\pgfpathrectangle{\pgfqpoint{1.150000in}{0.150000in}}{\pgfqpoint{5.700000in}{5.700000in}}%
\pgfusepath{clip}%
\pgfsetbuttcap%
\pgfsetroundjoin%
\definecolor{currentfill}{rgb}{0.282623,0.140926,0.457517}%
\pgfsetfillcolor{currentfill}%
\pgfsetfillopacity{0.700000}%
\pgfsetlinewidth{0.000000pt}%
\definecolor{currentstroke}{rgb}{0.000000,0.000000,0.000000}%
\pgfsetstrokecolor{currentstroke}%
\pgfsetdash{}{0pt}%
\pgfpathmoveto{\pgfqpoint{5.243340in}{2.228891in}}%
\pgfpathlineto{\pgfqpoint{5.257394in}{2.227832in}}%
\pgfpathlineto{\pgfqpoint{5.271457in}{2.226798in}}%
\pgfpathlineto{\pgfqpoint{5.285528in}{2.225788in}}%
\pgfpathlineto{\pgfqpoint{5.299608in}{2.224804in}}%
\pgfpathlineto{\pgfqpoint{5.292116in}{2.216315in}}%
\pgfpathlineto{\pgfqpoint{5.284617in}{2.207742in}}%
\pgfpathlineto{\pgfqpoint{5.277110in}{2.199083in}}%
\pgfpathlineto{\pgfqpoint{5.269597in}{2.190340in}}%
\pgfpathlineto{\pgfqpoint{5.255506in}{2.191366in}}%
\pgfpathlineto{\pgfqpoint{5.241424in}{2.192417in}}%
\pgfpathlineto{\pgfqpoint{5.227350in}{2.193492in}}%
\pgfpathlineto{\pgfqpoint{5.213284in}{2.194593in}}%
\pgfpathlineto{\pgfqpoint{5.220809in}{2.203289in}}%
\pgfpathlineto{\pgfqpoint{5.228326in}{2.211905in}}%
\pgfpathlineto{\pgfqpoint{5.235836in}{2.220439in}}%
\pgfpathlineto{\pgfqpoint{5.243340in}{2.228891in}}%
\pgfpathclose%
\pgfusepath{fill}%
\end{pgfscope}%
\begin{pgfscope}%
\pgfpathrectangle{\pgfqpoint{1.150000in}{0.150000in}}{\pgfqpoint{5.700000in}{5.700000in}}%
\pgfusepath{clip}%
\pgfsetbuttcap%
\pgfsetroundjoin%
\definecolor{currentfill}{rgb}{0.270595,0.214069,0.507052}%
\pgfsetfillcolor{currentfill}%
\pgfsetfillopacity{0.700000}%
\pgfsetlinewidth{0.000000pt}%
\definecolor{currentstroke}{rgb}{0.000000,0.000000,0.000000}%
\pgfsetstrokecolor{currentstroke}%
\pgfsetdash{}{0pt}%
\pgfpathmoveto{\pgfqpoint{5.872716in}{2.384489in}}%
\pgfpathlineto{\pgfqpoint{5.886973in}{2.384178in}}%
\pgfpathlineto{\pgfqpoint{5.901239in}{2.383893in}}%
\pgfpathlineto{\pgfqpoint{5.915514in}{2.383631in}}%
\pgfpathlineto{\pgfqpoint{5.908325in}{2.377470in}}%
\pgfpathlineto{\pgfqpoint{5.901127in}{2.371211in}}%
\pgfpathlineto{\pgfqpoint{5.893920in}{2.364854in}}%
\pgfpathlineto{\pgfqpoint{5.886704in}{2.358397in}}%
\pgfpathlineto{\pgfqpoint{5.872412in}{2.358617in}}%
\pgfpathlineto{\pgfqpoint{5.858130in}{2.358861in}}%
\pgfpathlineto{\pgfqpoint{5.843857in}{2.359129in}}%
\pgfpathlineto{\pgfqpoint{5.851085in}{2.365614in}}%
\pgfpathlineto{\pgfqpoint{5.858304in}{2.372001in}}%
\pgfpathlineto{\pgfqpoint{5.865515in}{2.378292in}}%
\pgfpathlineto{\pgfqpoint{5.872716in}{2.384489in}}%
\pgfpathclose%
\pgfusepath{fill}%
\end{pgfscope}%
\begin{pgfscope}%
\pgfpathrectangle{\pgfqpoint{1.150000in}{0.150000in}}{\pgfqpoint{5.700000in}{5.700000in}}%
\pgfusepath{clip}%
\pgfsetbuttcap%
\pgfsetroundjoin%
\definecolor{currentfill}{rgb}{0.273006,0.204520,0.501721}%
\pgfsetfillcolor{currentfill}%
\pgfsetfillopacity{0.700000}%
\pgfsetlinewidth{0.000000pt}%
\definecolor{currentstroke}{rgb}{0.000000,0.000000,0.000000}%
\pgfsetstrokecolor{currentstroke}%
\pgfsetdash{}{0pt}%
\pgfpathmoveto{\pgfqpoint{2.720317in}{2.355362in}}%
\pgfpathlineto{\pgfqpoint{2.733844in}{2.347079in}}%
\pgfpathlineto{\pgfqpoint{2.747374in}{2.338838in}}%
\pgfpathlineto{\pgfqpoint{2.760907in}{2.330639in}}%
\pgfpathlineto{\pgfqpoint{2.774443in}{2.322482in}}%
\pgfpathlineto{\pgfqpoint{2.765752in}{2.325753in}}%
\pgfpathlineto{\pgfqpoint{2.757041in}{2.329367in}}%
\pgfpathlineto{\pgfqpoint{2.748311in}{2.333332in}}%
\pgfpathlineto{\pgfqpoint{2.739562in}{2.337654in}}%
\pgfpathlineto{\pgfqpoint{2.725992in}{2.346113in}}%
\pgfpathlineto{\pgfqpoint{2.712424in}{2.354614in}}%
\pgfpathlineto{\pgfqpoint{2.698859in}{2.363156in}}%
\pgfpathlineto{\pgfqpoint{2.685297in}{2.371741in}}%
\pgfpathlineto{\pgfqpoint{2.694082in}{2.367111in}}%
\pgfpathlineto{\pgfqpoint{2.702846in}{2.362843in}}%
\pgfpathlineto{\pgfqpoint{2.711591in}{2.358930in}}%
\pgfpathlineto{\pgfqpoint{2.720317in}{2.355362in}}%
\pgfpathclose%
\pgfusepath{fill}%
\end{pgfscope}%
\begin{pgfscope}%
\pgfpathrectangle{\pgfqpoint{1.150000in}{0.150000in}}{\pgfqpoint{5.700000in}{5.700000in}}%
\pgfusepath{clip}%
\pgfsetbuttcap%
\pgfsetroundjoin%
\definecolor{currentfill}{rgb}{0.282327,0.094955,0.417331}%
\pgfsetfillcolor{currentfill}%
\pgfsetfillopacity{0.700000}%
\pgfsetlinewidth{0.000000pt}%
\definecolor{currentstroke}{rgb}{0.000000,0.000000,0.000000}%
\pgfsetstrokecolor{currentstroke}%
\pgfsetdash{}{0pt}%
\pgfpathmoveto{\pgfqpoint{3.167764in}{2.136833in}}%
\pgfpathlineto{\pgfqpoint{3.181329in}{2.130164in}}%
\pgfpathlineto{\pgfqpoint{3.194898in}{2.123529in}}%
\pgfpathlineto{\pgfqpoint{3.208471in}{2.116927in}}%
\pgfpathlineto{\pgfqpoint{3.222049in}{2.110359in}}%
\pgfpathlineto{\pgfqpoint{3.213686in}{2.109311in}}%
\pgfpathlineto{\pgfqpoint{3.205310in}{2.108519in}}%
\pgfpathlineto{\pgfqpoint{3.196921in}{2.107992in}}%
\pgfpathlineto{\pgfqpoint{3.188519in}{2.107735in}}%
\pgfpathlineto{\pgfqpoint{3.174915in}{2.114574in}}%
\pgfpathlineto{\pgfqpoint{3.161315in}{2.121446in}}%
\pgfpathlineto{\pgfqpoint{3.147719in}{2.128352in}}%
\pgfpathlineto{\pgfqpoint{3.134128in}{2.135291in}}%
\pgfpathlineto{\pgfqpoint{3.142557in}{2.135272in}}%
\pgfpathlineto{\pgfqpoint{3.150972in}{2.135528in}}%
\pgfpathlineto{\pgfqpoint{3.159375in}{2.136050in}}%
\pgfpathlineto{\pgfqpoint{3.167764in}{2.136833in}}%
\pgfpathclose%
\pgfusepath{fill}%
\end{pgfscope}%
\begin{pgfscope}%
\pgfpathrectangle{\pgfqpoint{1.150000in}{0.150000in}}{\pgfqpoint{5.700000in}{5.700000in}}%
\pgfusepath{clip}%
\pgfsetbuttcap%
\pgfsetroundjoin%
\definecolor{currentfill}{rgb}{0.267004,0.004874,0.329415}%
\pgfsetfillcolor{currentfill}%
\pgfsetfillopacity{0.700000}%
\pgfsetlinewidth{0.000000pt}%
\definecolor{currentstroke}{rgb}{0.000000,0.000000,0.000000}%
\pgfsetstrokecolor{currentstroke}%
\pgfsetdash{}{0pt}%
\pgfpathmoveto{\pgfqpoint{4.071435in}{1.980723in}}%
\pgfpathlineto{\pgfqpoint{4.085161in}{1.976934in}}%
\pgfpathlineto{\pgfqpoint{4.098893in}{1.973171in}}%
\pgfpathlineto{\pgfqpoint{4.112632in}{1.969436in}}%
\pgfpathlineto{\pgfqpoint{4.126377in}{1.965727in}}%
\pgfpathlineto{\pgfqpoint{4.118462in}{1.957976in}}%
\pgfpathlineto{\pgfqpoint{4.110541in}{1.950300in}}%
\pgfpathlineto{\pgfqpoint{4.102613in}{1.942704in}}%
\pgfpathlineto{\pgfqpoint{4.094680in}{1.935193in}}%
\pgfpathlineto{\pgfqpoint{4.080921in}{1.939091in}}%
\pgfpathlineto{\pgfqpoint{4.067169in}{1.943015in}}%
\pgfpathlineto{\pgfqpoint{4.053423in}{1.946966in}}%
\pgfpathlineto{\pgfqpoint{4.039683in}{1.950944in}}%
\pgfpathlineto{\pgfqpoint{4.047630in}{1.958261in}}%
\pgfpathlineto{\pgfqpoint{4.055571in}{1.965667in}}%
\pgfpathlineto{\pgfqpoint{4.063506in}{1.973156in}}%
\pgfpathlineto{\pgfqpoint{4.071435in}{1.980723in}}%
\pgfpathclose%
\pgfusepath{fill}%
\end{pgfscope}%
\begin{pgfscope}%
\pgfpathrectangle{\pgfqpoint{1.150000in}{0.150000in}}{\pgfqpoint{5.700000in}{5.700000in}}%
\pgfusepath{clip}%
\pgfsetbuttcap%
\pgfsetroundjoin%
\definecolor{currentfill}{rgb}{0.274952,0.037752,0.364543}%
\pgfsetfillcolor{currentfill}%
\pgfsetfillopacity{0.700000}%
\pgfsetlinewidth{0.000000pt}%
\definecolor{currentstroke}{rgb}{0.000000,0.000000,0.000000}%
\pgfsetstrokecolor{currentstroke}%
\pgfsetdash{}{0pt}%
\pgfpathmoveto{\pgfqpoint{4.527603in}{2.040391in}}%
\pgfpathlineto{\pgfqpoint{4.541446in}{2.037852in}}%
\pgfpathlineto{\pgfqpoint{4.555296in}{2.035338in}}%
\pgfpathlineto{\pgfqpoint{4.569153in}{2.032850in}}%
\pgfpathlineto{\pgfqpoint{4.583018in}{2.030388in}}%
\pgfpathlineto{\pgfqpoint{4.575260in}{2.021194in}}%
\pgfpathlineto{\pgfqpoint{4.567497in}{2.011995in}}%
\pgfpathlineto{\pgfqpoint{4.559728in}{2.002794in}}%
\pgfpathlineto{\pgfqpoint{4.551954in}{1.993593in}}%
\pgfpathlineto{\pgfqpoint{4.538079in}{1.996192in}}%
\pgfpathlineto{\pgfqpoint{4.524212in}{1.998816in}}%
\pgfpathlineto{\pgfqpoint{4.510351in}{2.001465in}}%
\pgfpathlineto{\pgfqpoint{4.496498in}{2.004140in}}%
\pgfpathlineto{\pgfqpoint{4.504282in}{2.013200in}}%
\pgfpathlineto{\pgfqpoint{4.512062in}{2.022263in}}%
\pgfpathlineto{\pgfqpoint{4.519835in}{2.031328in}}%
\pgfpathlineto{\pgfqpoint{4.527603in}{2.040391in}}%
\pgfpathclose%
\pgfusepath{fill}%
\end{pgfscope}%
\begin{pgfscope}%
\pgfpathrectangle{\pgfqpoint{1.150000in}{0.150000in}}{\pgfqpoint{5.700000in}{5.700000in}}%
\pgfusepath{clip}%
\pgfsetbuttcap%
\pgfsetroundjoin%
\definecolor{currentfill}{rgb}{0.280894,0.078907,0.402329}%
\pgfsetfillcolor{currentfill}%
\pgfsetfillopacity{0.700000}%
\pgfsetlinewidth{0.000000pt}%
\definecolor{currentstroke}{rgb}{0.000000,0.000000,0.000000}%
\pgfsetstrokecolor{currentstroke}%
\pgfsetdash{}{0pt}%
\pgfpathmoveto{\pgfqpoint{4.842273in}{2.115654in}}%
\pgfpathlineto{\pgfqpoint{4.856206in}{2.113849in}}%
\pgfpathlineto{\pgfqpoint{4.870147in}{2.112068in}}%
\pgfpathlineto{\pgfqpoint{4.884096in}{2.110312in}}%
\pgfpathlineto{\pgfqpoint{4.898053in}{2.108581in}}%
\pgfpathlineto{\pgfqpoint{4.890403in}{2.099296in}}%
\pgfpathlineto{\pgfqpoint{4.882747in}{2.089963in}}%
\pgfpathlineto{\pgfqpoint{4.875085in}{2.080582in}}%
\pgfpathlineto{\pgfqpoint{4.867418in}{2.071157in}}%
\pgfpathlineto{\pgfqpoint{4.853451in}{2.072984in}}%
\pgfpathlineto{\pgfqpoint{4.839492in}{2.074836in}}%
\pgfpathlineto{\pgfqpoint{4.825541in}{2.076712in}}%
\pgfpathlineto{\pgfqpoint{4.811598in}{2.078614in}}%
\pgfpathlineto{\pgfqpoint{4.819276in}{2.087939in}}%
\pgfpathlineto{\pgfqpoint{4.826947in}{2.097221in}}%
\pgfpathlineto{\pgfqpoint{4.834613in}{2.106461in}}%
\pgfpathlineto{\pgfqpoint{4.842273in}{2.115654in}}%
\pgfpathclose%
\pgfusepath{fill}%
\end{pgfscope}%
\begin{pgfscope}%
\pgfpathrectangle{\pgfqpoint{1.150000in}{0.150000in}}{\pgfqpoint{5.700000in}{5.700000in}}%
\pgfusepath{clip}%
\pgfsetbuttcap%
\pgfsetroundjoin%
\definecolor{currentfill}{rgb}{0.278012,0.180367,0.486697}%
\pgfsetfillcolor{currentfill}%
\pgfsetfillopacity{0.700000}%
\pgfsetlinewidth{0.000000pt}%
\definecolor{currentstroke}{rgb}{0.000000,0.000000,0.000000}%
\pgfsetstrokecolor{currentstroke}%
\pgfsetdash{}{0pt}%
\pgfpathmoveto{\pgfqpoint{5.558125in}{2.310556in}}%
\pgfpathlineto{\pgfqpoint{5.572281in}{2.309935in}}%
\pgfpathlineto{\pgfqpoint{5.586446in}{2.309339in}}%
\pgfpathlineto{\pgfqpoint{5.600621in}{2.308767in}}%
\pgfpathlineto{\pgfqpoint{5.614804in}{2.308220in}}%
\pgfpathlineto{\pgfqpoint{5.607455in}{2.300784in}}%
\pgfpathlineto{\pgfqpoint{5.600097in}{2.293249in}}%
\pgfpathlineto{\pgfqpoint{5.592731in}{2.285616in}}%
\pgfpathlineto{\pgfqpoint{5.585357in}{2.277883in}}%
\pgfpathlineto{\pgfqpoint{5.571161in}{2.278431in}}%
\pgfpathlineto{\pgfqpoint{5.556974in}{2.279003in}}%
\pgfpathlineto{\pgfqpoint{5.542795in}{2.279599in}}%
\pgfpathlineto{\pgfqpoint{5.528626in}{2.280221in}}%
\pgfpathlineto{\pgfqpoint{5.536013in}{2.287948in}}%
\pgfpathlineto{\pgfqpoint{5.543391in}{2.295579in}}%
\pgfpathlineto{\pgfqpoint{5.550762in}{2.303115in}}%
\pgfpathlineto{\pgfqpoint{5.558125in}{2.310556in}}%
\pgfpathclose%
\pgfusepath{fill}%
\end{pgfscope}%
\begin{pgfscope}%
\pgfpathrectangle{\pgfqpoint{1.150000in}{0.150000in}}{\pgfqpoint{5.700000in}{5.700000in}}%
\pgfusepath{clip}%
\pgfsetbuttcap%
\pgfsetroundjoin%
\definecolor{currentfill}{rgb}{0.282623,0.140926,0.457517}%
\pgfsetfillcolor{currentfill}%
\pgfsetfillopacity{0.700000}%
\pgfsetlinewidth{0.000000pt}%
\definecolor{currentstroke}{rgb}{0.000000,0.000000,0.000000}%
\pgfsetstrokecolor{currentstroke}%
\pgfsetdash{}{0pt}%
\pgfpathmoveto{\pgfqpoint{2.971326in}{2.221297in}}%
\pgfpathlineto{\pgfqpoint{2.984872in}{2.213932in}}%
\pgfpathlineto{\pgfqpoint{2.998422in}{2.206604in}}%
\pgfpathlineto{\pgfqpoint{3.011976in}{2.199312in}}%
\pgfpathlineto{\pgfqpoint{3.025533in}{2.192057in}}%
\pgfpathlineto{\pgfqpoint{3.017033in}{2.192917in}}%
\pgfpathlineto{\pgfqpoint{3.008518in}{2.194074in}}%
\pgfpathlineto{\pgfqpoint{2.999987in}{2.195535in}}%
\pgfpathlineto{\pgfqpoint{2.991440in}{2.197307in}}%
\pgfpathlineto{\pgfqpoint{2.977853in}{2.204847in}}%
\pgfpathlineto{\pgfqpoint{2.964269in}{2.212424in}}%
\pgfpathlineto{\pgfqpoint{2.950689in}{2.220038in}}%
\pgfpathlineto{\pgfqpoint{2.937112in}{2.227689in}}%
\pgfpathlineto{\pgfqpoint{2.945690in}{2.225626in}}%
\pgfpathlineto{\pgfqpoint{2.954251in}{2.223878in}}%
\pgfpathlineto{\pgfqpoint{2.962796in}{2.222437in}}%
\pgfpathlineto{\pgfqpoint{2.971326in}{2.221297in}}%
\pgfpathclose%
\pgfusepath{fill}%
\end{pgfscope}%
\begin{pgfscope}%
\pgfpathrectangle{\pgfqpoint{1.150000in}{0.150000in}}{\pgfqpoint{5.700000in}{5.700000in}}%
\pgfusepath{clip}%
\pgfsetbuttcap%
\pgfsetroundjoin%
\definecolor{currentfill}{rgb}{0.277941,0.056324,0.381191}%
\pgfsetfillcolor{currentfill}%
\pgfsetfillopacity{0.700000}%
\pgfsetlinewidth{0.000000pt}%
\definecolor{currentstroke}{rgb}{0.000000,0.000000,0.000000}%
\pgfsetstrokecolor{currentstroke}%
\pgfsetdash{}{0pt}%
\pgfpathmoveto{\pgfqpoint{3.363963in}{2.067665in}}%
\pgfpathlineto{\pgfqpoint{3.377556in}{2.061644in}}%
\pgfpathlineto{\pgfqpoint{3.391154in}{2.055656in}}%
\pgfpathlineto{\pgfqpoint{3.404757in}{2.049698in}}%
\pgfpathlineto{\pgfqpoint{3.418365in}{2.043772in}}%
\pgfpathlineto{\pgfqpoint{3.410120in}{2.041013in}}%
\pgfpathlineto{\pgfqpoint{3.401864in}{2.038472in}}%
\pgfpathlineto{\pgfqpoint{3.393598in}{2.036158in}}%
\pgfpathlineto{\pgfqpoint{3.385320in}{2.034075in}}%
\pgfpathlineto{\pgfqpoint{3.371690in}{2.040258in}}%
\pgfpathlineto{\pgfqpoint{3.358064in}{2.046471in}}%
\pgfpathlineto{\pgfqpoint{3.344442in}{2.052716in}}%
\pgfpathlineto{\pgfqpoint{3.330825in}{2.058992in}}%
\pgfpathlineto{\pgfqpoint{3.339127in}{2.060813in}}%
\pgfpathlineto{\pgfqpoint{3.347416in}{2.062870in}}%
\pgfpathlineto{\pgfqpoint{3.355695in}{2.065156in}}%
\pgfpathlineto{\pgfqpoint{3.363963in}{2.067665in}}%
\pgfpathclose%
\pgfusepath{fill}%
\end{pgfscope}%
\begin{pgfscope}%
\pgfpathrectangle{\pgfqpoint{1.150000in}{0.150000in}}{\pgfqpoint{5.700000in}{5.700000in}}%
\pgfusepath{clip}%
\pgfsetbuttcap%
\pgfsetroundjoin%
\definecolor{currentfill}{rgb}{0.283187,0.125848,0.444960}%
\pgfsetfillcolor{currentfill}%
\pgfsetfillopacity{0.700000}%
\pgfsetlinewidth{0.000000pt}%
\definecolor{currentstroke}{rgb}{0.000000,0.000000,0.000000}%
\pgfsetstrokecolor{currentstroke}%
\pgfsetdash{}{0pt}%
\pgfpathmoveto{\pgfqpoint{5.157107in}{2.199243in}}%
\pgfpathlineto{\pgfqpoint{5.171139in}{2.198043in}}%
\pgfpathlineto{\pgfqpoint{5.185179in}{2.196868in}}%
\pgfpathlineto{\pgfqpoint{5.199227in}{2.195718in}}%
\pgfpathlineto{\pgfqpoint{5.213284in}{2.194593in}}%
\pgfpathlineto{\pgfqpoint{5.205753in}{2.185816in}}%
\pgfpathlineto{\pgfqpoint{5.198214in}{2.176961in}}%
\pgfpathlineto{\pgfqpoint{5.190669in}{2.168026in}}%
\pgfpathlineto{\pgfqpoint{5.183117in}{2.159015in}}%
\pgfpathlineto{\pgfqpoint{5.169050in}{2.160195in}}%
\pgfpathlineto{\pgfqpoint{5.154990in}{2.161401in}}%
\pgfpathlineto{\pgfqpoint{5.140940in}{2.162631in}}%
\pgfpathlineto{\pgfqpoint{5.126897in}{2.163886in}}%
\pgfpathlineto{\pgfqpoint{5.134460in}{2.172837in}}%
\pgfpathlineto{\pgfqpoint{5.142016in}{2.181715in}}%
\pgfpathlineto{\pgfqpoint{5.149565in}{2.190517in}}%
\pgfpathlineto{\pgfqpoint{5.157107in}{2.199243in}}%
\pgfpathclose%
\pgfusepath{fill}%
\end{pgfscope}%
\begin{pgfscope}%
\pgfpathrectangle{\pgfqpoint{1.150000in}{0.150000in}}{\pgfqpoint{5.700000in}{5.700000in}}%
\pgfusepath{clip}%
\pgfsetbuttcap%
\pgfsetroundjoin%
\definecolor{currentfill}{rgb}{0.279566,0.067836,0.391917}%
\pgfsetfillcolor{currentfill}%
\pgfsetfillopacity{0.700000}%
\pgfsetlinewidth{0.000000pt}%
\definecolor{currentstroke}{rgb}{0.000000,0.000000,0.000000}%
\pgfsetstrokecolor{currentstroke}%
\pgfsetdash{}{0pt}%
\pgfpathmoveto{\pgfqpoint{4.755904in}{2.086473in}}%
\pgfpathlineto{\pgfqpoint{4.769816in}{2.084471in}}%
\pgfpathlineto{\pgfqpoint{4.783736in}{2.082494in}}%
\pgfpathlineto{\pgfqpoint{4.797663in}{2.080541in}}%
\pgfpathlineto{\pgfqpoint{4.811598in}{2.078614in}}%
\pgfpathlineto{\pgfqpoint{4.803915in}{2.069251in}}%
\pgfpathlineto{\pgfqpoint{4.796226in}{2.059851in}}%
\pgfpathlineto{\pgfqpoint{4.788532in}{2.050416in}}%
\pgfpathlineto{\pgfqpoint{4.780831in}{2.040949in}}%
\pgfpathlineto{\pgfqpoint{4.766886in}{2.042985in}}%
\pgfpathlineto{\pgfqpoint{4.752949in}{2.045047in}}%
\pgfpathlineto{\pgfqpoint{4.739019in}{2.047134in}}%
\pgfpathlineto{\pgfqpoint{4.725097in}{2.049246in}}%
\pgfpathlineto{\pgfqpoint{4.732807in}{2.058598in}}%
\pgfpathlineto{\pgfqpoint{4.740512in}{2.067922in}}%
\pgfpathlineto{\pgfqpoint{4.748211in}{2.077214in}}%
\pgfpathlineto{\pgfqpoint{4.755904in}{2.086473in}}%
\pgfpathclose%
\pgfusepath{fill}%
\end{pgfscope}%
\begin{pgfscope}%
\pgfpathrectangle{\pgfqpoint{1.150000in}{0.150000in}}{\pgfqpoint{5.700000in}{5.700000in}}%
\pgfusepath{clip}%
\pgfsetbuttcap%
\pgfsetroundjoin%
\definecolor{currentfill}{rgb}{0.246811,0.283237,0.535941}%
\pgfsetfillcolor{currentfill}%
\pgfsetfillopacity{0.700000}%
\pgfsetlinewidth{0.000000pt}%
\definecolor{currentstroke}{rgb}{0.000000,0.000000,0.000000}%
\pgfsetstrokecolor{currentstroke}%
\pgfsetdash{}{0pt}%
\pgfpathmoveto{\pgfqpoint{2.468615in}{2.515151in}}%
\pgfpathlineto{\pgfqpoint{2.482142in}{2.505837in}}%
\pgfpathlineto{\pgfqpoint{2.495671in}{2.496571in}}%
\pgfpathlineto{\pgfqpoint{2.509202in}{2.487354in}}%
\pgfpathlineto{\pgfqpoint{2.522735in}{2.478185in}}%
\pgfpathlineto{\pgfqpoint{2.513818in}{2.484123in}}%
\pgfpathlineto{\pgfqpoint{2.504878in}{2.490450in}}%
\pgfpathlineto{\pgfqpoint{2.495915in}{2.497177in}}%
\pgfpathlineto{\pgfqpoint{2.486928in}{2.504312in}}%
\pgfpathlineto{\pgfqpoint{2.473356in}{2.513800in}}%
\pgfpathlineto{\pgfqpoint{2.459785in}{2.523336in}}%
\pgfpathlineto{\pgfqpoint{2.446216in}{2.532921in}}%
\pgfpathlineto{\pgfqpoint{2.432649in}{2.542555in}}%
\pgfpathlineto{\pgfqpoint{2.441677in}{2.535095in}}%
\pgfpathlineto{\pgfqpoint{2.450680in}{2.528047in}}%
\pgfpathlineto{\pgfqpoint{2.459659in}{2.521402in}}%
\pgfpathlineto{\pgfqpoint{2.468615in}{2.515151in}}%
\pgfpathclose%
\pgfusepath{fill}%
\end{pgfscope}%
\begin{pgfscope}%
\pgfpathrectangle{\pgfqpoint{1.150000in}{0.150000in}}{\pgfqpoint{5.700000in}{5.700000in}}%
\pgfusepath{clip}%
\pgfsetbuttcap%
\pgfsetroundjoin%
\definecolor{currentfill}{rgb}{0.268510,0.009605,0.335427}%
\pgfsetfillcolor{currentfill}%
\pgfsetfillopacity{0.700000}%
\pgfsetlinewidth{0.000000pt}%
\definecolor{currentstroke}{rgb}{0.000000,0.000000,0.000000}%
\pgfsetstrokecolor{currentstroke}%
\pgfsetdash{}{0pt}%
\pgfpathmoveto{\pgfqpoint{4.212975in}{1.983527in}}%
\pgfpathlineto{\pgfqpoint{4.226740in}{1.980126in}}%
\pgfpathlineto{\pgfqpoint{4.240512in}{1.976752in}}%
\pgfpathlineto{\pgfqpoint{4.254290in}{1.973403in}}%
\pgfpathlineto{\pgfqpoint{4.268076in}{1.970081in}}%
\pgfpathlineto{\pgfqpoint{4.260208in}{1.961731in}}%
\pgfpathlineto{\pgfqpoint{4.252336in}{1.953433in}}%
\pgfpathlineto{\pgfqpoint{4.244457in}{1.945190in}}%
\pgfpathlineto{\pgfqpoint{4.236573in}{1.937007in}}%
\pgfpathlineto{\pgfqpoint{4.222776in}{1.940505in}}%
\pgfpathlineto{\pgfqpoint{4.208985in}{1.944030in}}%
\pgfpathlineto{\pgfqpoint{4.195201in}{1.947580in}}%
\pgfpathlineto{\pgfqpoint{4.181423in}{1.951157in}}%
\pgfpathlineto{\pgfqpoint{4.189320in}{1.959159in}}%
\pgfpathlineto{\pgfqpoint{4.197210in}{1.967224in}}%
\pgfpathlineto{\pgfqpoint{4.205096in}{1.975349in}}%
\pgfpathlineto{\pgfqpoint{4.212975in}{1.983527in}}%
\pgfpathclose%
\pgfusepath{fill}%
\end{pgfscope}%
\begin{pgfscope}%
\pgfpathrectangle{\pgfqpoint{1.150000in}{0.150000in}}{\pgfqpoint{5.700000in}{5.700000in}}%
\pgfusepath{clip}%
\pgfsetbuttcap%
\pgfsetroundjoin%
\definecolor{currentfill}{rgb}{0.279574,0.170599,0.479997}%
\pgfsetfillcolor{currentfill}%
\pgfsetfillopacity{0.700000}%
\pgfsetlinewidth{0.000000pt}%
\definecolor{currentstroke}{rgb}{0.000000,0.000000,0.000000}%
\pgfsetstrokecolor{currentstroke}%
\pgfsetdash{}{0pt}%
\pgfpathmoveto{\pgfqpoint{5.472037in}{2.282953in}}%
\pgfpathlineto{\pgfqpoint{5.486171in}{2.282233in}}%
\pgfpathlineto{\pgfqpoint{5.500313in}{2.281537in}}%
\pgfpathlineto{\pgfqpoint{5.514465in}{2.280867in}}%
\pgfpathlineto{\pgfqpoint{5.528626in}{2.280221in}}%
\pgfpathlineto{\pgfqpoint{5.521231in}{2.272397in}}%
\pgfpathlineto{\pgfqpoint{5.513828in}{2.264476in}}%
\pgfpathlineto{\pgfqpoint{5.506417in}{2.256459in}}%
\pgfpathlineto{\pgfqpoint{5.498998in}{2.248345in}}%
\pgfpathlineto{\pgfqpoint{5.484825in}{2.249005in}}%
\pgfpathlineto{\pgfqpoint{5.470661in}{2.249690in}}%
\pgfpathlineto{\pgfqpoint{5.456506in}{2.250399in}}%
\pgfpathlineto{\pgfqpoint{5.442359in}{2.251133in}}%
\pgfpathlineto{\pgfqpoint{5.449790in}{2.259228in}}%
\pgfpathlineto{\pgfqpoint{5.457214in}{2.267230in}}%
\pgfpathlineto{\pgfqpoint{5.464629in}{2.275138in}}%
\pgfpathlineto{\pgfqpoint{5.472037in}{2.282953in}}%
\pgfpathclose%
\pgfusepath{fill}%
\end{pgfscope}%
\begin{pgfscope}%
\pgfpathrectangle{\pgfqpoint{1.150000in}{0.150000in}}{\pgfqpoint{5.700000in}{5.700000in}}%
\pgfusepath{clip}%
\pgfsetbuttcap%
\pgfsetroundjoin%
\definecolor{currentfill}{rgb}{0.272594,0.025563,0.353093}%
\pgfsetfillcolor{currentfill}%
\pgfsetfillopacity{0.700000}%
\pgfsetlinewidth{0.000000pt}%
\definecolor{currentstroke}{rgb}{0.000000,0.000000,0.000000}%
\pgfsetstrokecolor{currentstroke}%
\pgfsetdash{}{0pt}%
\pgfpathmoveto{\pgfqpoint{4.441157in}{2.015096in}}%
\pgfpathlineto{\pgfqpoint{4.454982in}{2.012319in}}%
\pgfpathlineto{\pgfqpoint{4.468813in}{2.009567in}}%
\pgfpathlineto{\pgfqpoint{4.482652in}{2.006841in}}%
\pgfpathlineto{\pgfqpoint{4.496498in}{2.004140in}}%
\pgfpathlineto{\pgfqpoint{4.488708in}{1.995088in}}%
\pgfpathlineto{\pgfqpoint{4.480913in}{1.986048in}}%
\pgfpathlineto{\pgfqpoint{4.473112in}{1.977021in}}%
\pgfpathlineto{\pgfqpoint{4.465307in}{1.968013in}}%
\pgfpathlineto{\pgfqpoint{4.451450in}{1.970863in}}%
\pgfpathlineto{\pgfqpoint{4.437600in}{1.973738in}}%
\pgfpathlineto{\pgfqpoint{4.423758in}{1.976639in}}%
\pgfpathlineto{\pgfqpoint{4.409922in}{1.979566in}}%
\pgfpathlineto{\pgfqpoint{4.417739in}{1.988420in}}%
\pgfpathlineto{\pgfqpoint{4.425550in}{1.997296in}}%
\pgfpathlineto{\pgfqpoint{4.433357in}{2.006189in}}%
\pgfpathlineto{\pgfqpoint{4.441157in}{2.015096in}}%
\pgfpathclose%
\pgfusepath{fill}%
\end{pgfscope}%
\begin{pgfscope}%
\pgfpathrectangle{\pgfqpoint{1.150000in}{0.150000in}}{\pgfqpoint{5.700000in}{5.700000in}}%
\pgfusepath{clip}%
\pgfsetbuttcap%
\pgfsetroundjoin%
\definecolor{currentfill}{rgb}{0.269944,0.014625,0.341379}%
\pgfsetfillcolor{currentfill}%
\pgfsetfillopacity{0.700000}%
\pgfsetlinewidth{0.000000pt}%
\definecolor{currentstroke}{rgb}{0.000000,0.000000,0.000000}%
\pgfsetstrokecolor{currentstroke}%
\pgfsetdash{}{0pt}%
\pgfpathmoveto{\pgfqpoint{3.701711in}{1.989810in}}%
\pgfpathlineto{\pgfqpoint{3.715365in}{1.984862in}}%
\pgfpathlineto{\pgfqpoint{3.729025in}{1.979942in}}%
\pgfpathlineto{\pgfqpoint{3.742690in}{1.975050in}}%
\pgfpathlineto{\pgfqpoint{3.756361in}{1.970187in}}%
\pgfpathlineto{\pgfqpoint{3.748289in}{1.964781in}}%
\pgfpathlineto{\pgfqpoint{3.740209in}{1.959528in}}%
\pgfpathlineto{\pgfqpoint{3.732120in}{1.954434in}}%
\pgfpathlineto{\pgfqpoint{3.724024in}{1.949505in}}%
\pgfpathlineto{\pgfqpoint{3.710335in}{1.954597in}}%
\pgfpathlineto{\pgfqpoint{3.696651in}{1.959717in}}%
\pgfpathlineto{\pgfqpoint{3.682973in}{1.964866in}}%
\pgfpathlineto{\pgfqpoint{3.669300in}{1.970043in}}%
\pgfpathlineto{\pgfqpoint{3.677415in}{1.974738in}}%
\pgfpathlineto{\pgfqpoint{3.685522in}{1.979602in}}%
\pgfpathlineto{\pgfqpoint{3.693620in}{1.984628in}}%
\pgfpathlineto{\pgfqpoint{3.701711in}{1.989810in}}%
\pgfpathclose%
\pgfusepath{fill}%
\end{pgfscope}%
\begin{pgfscope}%
\pgfpathrectangle{\pgfqpoint{1.150000in}{0.150000in}}{\pgfqpoint{5.700000in}{5.700000in}}%
\pgfusepath{clip}%
\pgfsetbuttcap%
\pgfsetroundjoin%
\definecolor{currentfill}{rgb}{0.268510,0.009605,0.335427}%
\pgfsetfillcolor{currentfill}%
\pgfsetfillopacity{0.700000}%
\pgfsetlinewidth{0.000000pt}%
\definecolor{currentstroke}{rgb}{0.000000,0.000000,0.000000}%
\pgfsetstrokecolor{currentstroke}%
\pgfsetdash{}{0pt}%
\pgfpathmoveto{\pgfqpoint{3.843248in}{1.974924in}}%
\pgfpathlineto{\pgfqpoint{3.856931in}{1.970416in}}%
\pgfpathlineto{\pgfqpoint{3.870619in}{1.965937in}}%
\pgfpathlineto{\pgfqpoint{3.884314in}{1.961485in}}%
\pgfpathlineto{\pgfqpoint{3.898014in}{1.957060in}}%
\pgfpathlineto{\pgfqpoint{3.890004in}{1.950676in}}%
\pgfpathlineto{\pgfqpoint{3.881988in}{1.944417in}}%
\pgfpathlineto{\pgfqpoint{3.873964in}{1.938287in}}%
\pgfpathlineto{\pgfqpoint{3.865933in}{1.932292in}}%
\pgfpathlineto{\pgfqpoint{3.852216in}{1.936932in}}%
\pgfpathlineto{\pgfqpoint{3.838506in}{1.941599in}}%
\pgfpathlineto{\pgfqpoint{3.824801in}{1.946294in}}%
\pgfpathlineto{\pgfqpoint{3.811102in}{1.951017in}}%
\pgfpathlineto{\pgfqpoint{3.819149in}{1.956791in}}%
\pgfpathlineto{\pgfqpoint{3.827190in}{1.962704in}}%
\pgfpathlineto{\pgfqpoint{3.835222in}{1.968750in}}%
\pgfpathlineto{\pgfqpoint{3.843248in}{1.974924in}}%
\pgfpathclose%
\pgfusepath{fill}%
\end{pgfscope}%
\begin{pgfscope}%
\pgfpathrectangle{\pgfqpoint{1.150000in}{0.150000in}}{\pgfqpoint{5.700000in}{5.700000in}}%
\pgfusepath{clip}%
\pgfsetbuttcap%
\pgfsetroundjoin%
\definecolor{currentfill}{rgb}{0.271828,0.209303,0.504434}%
\pgfsetfillcolor{currentfill}%
\pgfsetfillopacity{0.700000}%
\pgfsetlinewidth{0.000000pt}%
\definecolor{currentstroke}{rgb}{0.000000,0.000000,0.000000}%
\pgfsetstrokecolor{currentstroke}%
\pgfsetdash{}{0pt}%
\pgfpathmoveto{\pgfqpoint{5.786858in}{2.360450in}}%
\pgfpathlineto{\pgfqpoint{5.801094in}{2.360083in}}%
\pgfpathlineto{\pgfqpoint{5.815339in}{2.359741in}}%
\pgfpathlineto{\pgfqpoint{5.829594in}{2.359423in}}%
\pgfpathlineto{\pgfqpoint{5.843857in}{2.359129in}}%
\pgfpathlineto{\pgfqpoint{5.836620in}{2.352546in}}%
\pgfpathlineto{\pgfqpoint{5.829375in}{2.345861in}}%
\pgfpathlineto{\pgfqpoint{5.822120in}{2.339075in}}%
\pgfpathlineto{\pgfqpoint{5.814856in}{2.332186in}}%
\pgfpathlineto{\pgfqpoint{5.800577in}{2.332451in}}%
\pgfpathlineto{\pgfqpoint{5.786308in}{2.332742in}}%
\pgfpathlineto{\pgfqpoint{5.772048in}{2.333056in}}%
\pgfpathlineto{\pgfqpoint{5.757797in}{2.333396in}}%
\pgfpathlineto{\pgfqpoint{5.765075in}{2.340308in}}%
\pgfpathlineto{\pgfqpoint{5.772345in}{2.347120in}}%
\pgfpathlineto{\pgfqpoint{5.779606in}{2.353834in}}%
\pgfpathlineto{\pgfqpoint{5.786858in}{2.360450in}}%
\pgfpathclose%
\pgfusepath{fill}%
\end{pgfscope}%
\begin{pgfscope}%
\pgfpathrectangle{\pgfqpoint{1.150000in}{0.150000in}}{\pgfqpoint{5.700000in}{5.700000in}}%
\pgfusepath{clip}%
\pgfsetbuttcap%
\pgfsetroundjoin%
\definecolor{currentfill}{rgb}{0.283197,0.115680,0.436115}%
\pgfsetfillcolor{currentfill}%
\pgfsetfillopacity{0.700000}%
\pgfsetlinewidth{0.000000pt}%
\definecolor{currentstroke}{rgb}{0.000000,0.000000,0.000000}%
\pgfsetstrokecolor{currentstroke}%
\pgfsetdash{}{0pt}%
\pgfpathmoveto{\pgfqpoint{5.070811in}{2.169155in}}%
\pgfpathlineto{\pgfqpoint{5.084820in}{2.167801in}}%
\pgfpathlineto{\pgfqpoint{5.098838in}{2.166471in}}%
\pgfpathlineto{\pgfqpoint{5.112863in}{2.165166in}}%
\pgfpathlineto{\pgfqpoint{5.126897in}{2.163886in}}%
\pgfpathlineto{\pgfqpoint{5.119328in}{2.154862in}}%
\pgfpathlineto{\pgfqpoint{5.111752in}{2.145766in}}%
\pgfpathlineto{\pgfqpoint{5.104170in}{2.136599in}}%
\pgfpathlineto{\pgfqpoint{5.096581in}{2.127364in}}%
\pgfpathlineto{\pgfqpoint{5.082537in}{2.128713in}}%
\pgfpathlineto{\pgfqpoint{5.068501in}{2.130086in}}%
\pgfpathlineto{\pgfqpoint{5.054473in}{2.131485in}}%
\pgfpathlineto{\pgfqpoint{5.040454in}{2.132908in}}%
\pgfpathlineto{\pgfqpoint{5.048053in}{2.142070in}}%
\pgfpathlineto{\pgfqpoint{5.055645in}{2.151166in}}%
\pgfpathlineto{\pgfqpoint{5.063231in}{2.160195in}}%
\pgfpathlineto{\pgfqpoint{5.070811in}{2.169155in}}%
\pgfpathclose%
\pgfusepath{fill}%
\end{pgfscope}%
\begin{pgfscope}%
\pgfpathrectangle{\pgfqpoint{1.150000in}{0.150000in}}{\pgfqpoint{5.700000in}{5.700000in}}%
\pgfusepath{clip}%
\pgfsetbuttcap%
\pgfsetroundjoin%
\definecolor{currentfill}{rgb}{0.275191,0.194905,0.496005}%
\pgfsetfillcolor{currentfill}%
\pgfsetfillopacity{0.700000}%
\pgfsetlinewidth{0.000000pt}%
\definecolor{currentstroke}{rgb}{0.000000,0.000000,0.000000}%
\pgfsetstrokecolor{currentstroke}%
\pgfsetdash{}{0pt}%
\pgfpathmoveto{\pgfqpoint{2.774443in}{2.322482in}}%
\pgfpathlineto{\pgfqpoint{2.787982in}{2.314365in}}%
\pgfpathlineto{\pgfqpoint{2.801524in}{2.306289in}}%
\pgfpathlineto{\pgfqpoint{2.815069in}{2.298253in}}%
\pgfpathlineto{\pgfqpoint{2.828616in}{2.290257in}}%
\pgfpathlineto{\pgfqpoint{2.819958in}{2.293233in}}%
\pgfpathlineto{\pgfqpoint{2.811281in}{2.296548in}}%
\pgfpathlineto{\pgfqpoint{2.802586in}{2.300209in}}%
\pgfpathlineto{\pgfqpoint{2.793871in}{2.304224in}}%
\pgfpathlineto{\pgfqpoint{2.780290in}{2.312521in}}%
\pgfpathlineto{\pgfqpoint{2.766711in}{2.320858in}}%
\pgfpathlineto{\pgfqpoint{2.753135in}{2.329236in}}%
\pgfpathlineto{\pgfqpoint{2.739562in}{2.337654in}}%
\pgfpathlineto{\pgfqpoint{2.748311in}{2.333332in}}%
\pgfpathlineto{\pgfqpoint{2.757041in}{2.329367in}}%
\pgfpathlineto{\pgfqpoint{2.765752in}{2.325753in}}%
\pgfpathlineto{\pgfqpoint{2.774443in}{2.322482in}}%
\pgfpathclose%
\pgfusepath{fill}%
\end{pgfscope}%
\begin{pgfscope}%
\pgfpathrectangle{\pgfqpoint{1.150000in}{0.150000in}}{\pgfqpoint{5.700000in}{5.700000in}}%
\pgfusepath{clip}%
\pgfsetbuttcap%
\pgfsetroundjoin%
\definecolor{currentfill}{rgb}{0.273809,0.031497,0.358853}%
\pgfsetfillcolor{currentfill}%
\pgfsetfillopacity{0.700000}%
\pgfsetlinewidth{0.000000pt}%
\definecolor{currentstroke}{rgb}{0.000000,0.000000,0.000000}%
\pgfsetstrokecolor{currentstroke}%
\pgfsetdash{}{0pt}%
\pgfpathmoveto{\pgfqpoint{3.560107in}{2.012506in}}%
\pgfpathlineto{\pgfqpoint{3.573738in}{2.007095in}}%
\pgfpathlineto{\pgfqpoint{3.587374in}{2.001714in}}%
\pgfpathlineto{\pgfqpoint{3.601015in}{1.996362in}}%
\pgfpathlineto{\pgfqpoint{3.614662in}{1.991040in}}%
\pgfpathlineto{\pgfqpoint{3.606519in}{1.986756in}}%
\pgfpathlineto{\pgfqpoint{3.598367in}{1.982656in}}%
\pgfpathlineto{\pgfqpoint{3.590205in}{1.978745in}}%
\pgfpathlineto{\pgfqpoint{3.582035in}{1.975031in}}%
\pgfpathlineto{\pgfqpoint{3.568368in}{1.980595in}}%
\pgfpathlineto{\pgfqpoint{3.554707in}{1.986189in}}%
\pgfpathlineto{\pgfqpoint{3.541050in}{1.991812in}}%
\pgfpathlineto{\pgfqpoint{3.527399in}{1.997465in}}%
\pgfpathlineto{\pgfqpoint{3.535590in}{2.000932in}}%
\pgfpathlineto{\pgfqpoint{3.543772in}{2.004599in}}%
\pgfpathlineto{\pgfqpoint{3.551944in}{2.008459in}}%
\pgfpathlineto{\pgfqpoint{3.560107in}{2.012506in}}%
\pgfpathclose%
\pgfusepath{fill}%
\end{pgfscope}%
\begin{pgfscope}%
\pgfpathrectangle{\pgfqpoint{1.150000in}{0.150000in}}{\pgfqpoint{5.700000in}{5.700000in}}%
\pgfusepath{clip}%
\pgfsetbuttcap%
\pgfsetroundjoin%
\definecolor{currentfill}{rgb}{0.267004,0.004874,0.329415}%
\pgfsetfillcolor{currentfill}%
\pgfsetfillopacity{0.700000}%
\pgfsetlinewidth{0.000000pt}%
\definecolor{currentstroke}{rgb}{0.000000,0.000000,0.000000}%
\pgfsetstrokecolor{currentstroke}%
\pgfsetdash{}{0pt}%
\pgfpathmoveto{\pgfqpoint{3.984785in}{1.967124in}}%
\pgfpathlineto{\pgfqpoint{3.998500in}{1.963039in}}%
\pgfpathlineto{\pgfqpoint{4.012221in}{1.958980in}}%
\pgfpathlineto{\pgfqpoint{4.025949in}{1.954948in}}%
\pgfpathlineto{\pgfqpoint{4.039683in}{1.950944in}}%
\pgfpathlineto{\pgfqpoint{4.031729in}{1.943719in}}%
\pgfpathlineto{\pgfqpoint{4.023770in}{1.936591in}}%
\pgfpathlineto{\pgfqpoint{4.015803in}{1.929566in}}%
\pgfpathlineto{\pgfqpoint{4.007831in}{1.922649in}}%
\pgfpathlineto{\pgfqpoint{3.994083in}{1.926855in}}%
\pgfpathlineto{\pgfqpoint{3.980341in}{1.931089in}}%
\pgfpathlineto{\pgfqpoint{3.966604in}{1.935350in}}%
\pgfpathlineto{\pgfqpoint{3.952874in}{1.939637in}}%
\pgfpathlineto{\pgfqpoint{3.960862in}{1.946348in}}%
\pgfpathlineto{\pgfqpoint{3.968843in}{1.953169in}}%
\pgfpathlineto{\pgfqpoint{3.976817in}{1.960096in}}%
\pgfpathlineto{\pgfqpoint{3.984785in}{1.967124in}}%
\pgfpathclose%
\pgfusepath{fill}%
\end{pgfscope}%
\begin{pgfscope}%
\pgfpathrectangle{\pgfqpoint{1.150000in}{0.150000in}}{\pgfqpoint{5.700000in}{5.700000in}}%
\pgfusepath{clip}%
\pgfsetbuttcap%
\pgfsetroundjoin%
\definecolor{currentfill}{rgb}{0.277941,0.056324,0.381191}%
\pgfsetfillcolor{currentfill}%
\pgfsetfillopacity{0.700000}%
\pgfsetlinewidth{0.000000pt}%
\definecolor{currentstroke}{rgb}{0.000000,0.000000,0.000000}%
\pgfsetstrokecolor{currentstroke}%
\pgfsetdash{}{0pt}%
\pgfpathmoveto{\pgfqpoint{4.669485in}{2.057946in}}%
\pgfpathlineto{\pgfqpoint{4.683377in}{2.055733in}}%
\pgfpathlineto{\pgfqpoint{4.697276in}{2.053545in}}%
\pgfpathlineto{\pgfqpoint{4.711183in}{2.051383in}}%
\pgfpathlineto{\pgfqpoint{4.725097in}{2.049246in}}%
\pgfpathlineto{\pgfqpoint{4.717382in}{2.039867in}}%
\pgfpathlineto{\pgfqpoint{4.709660in}{2.030465in}}%
\pgfpathlineto{\pgfqpoint{4.701934in}{2.021042in}}%
\pgfpathlineto{\pgfqpoint{4.694202in}{2.011601in}}%
\pgfpathlineto{\pgfqpoint{4.680277in}{2.013861in}}%
\pgfpathlineto{\pgfqpoint{4.666361in}{2.016146in}}%
\pgfpathlineto{\pgfqpoint{4.652451in}{2.018456in}}%
\pgfpathlineto{\pgfqpoint{4.638550in}{2.020792in}}%
\pgfpathlineto{\pgfqpoint{4.646292in}{2.030105in}}%
\pgfpathlineto{\pgfqpoint{4.654029in}{2.039404in}}%
\pgfpathlineto{\pgfqpoint{4.661760in}{2.048685in}}%
\pgfpathlineto{\pgfqpoint{4.669485in}{2.057946in}}%
\pgfpathclose%
\pgfusepath{fill}%
\end{pgfscope}%
\begin{pgfscope}%
\pgfpathrectangle{\pgfqpoint{1.150000in}{0.150000in}}{\pgfqpoint{5.700000in}{5.700000in}}%
\pgfusepath{clip}%
\pgfsetbuttcap%
\pgfsetroundjoin%
\definecolor{currentfill}{rgb}{0.281924,0.089666,0.412415}%
\pgfsetfillcolor{currentfill}%
\pgfsetfillopacity{0.700000}%
\pgfsetlinewidth{0.000000pt}%
\definecolor{currentstroke}{rgb}{0.000000,0.000000,0.000000}%
\pgfsetstrokecolor{currentstroke}%
\pgfsetdash{}{0pt}%
\pgfpathmoveto{\pgfqpoint{3.222049in}{2.110359in}}%
\pgfpathlineto{\pgfqpoint{3.235631in}{2.103824in}}%
\pgfpathlineto{\pgfqpoint{3.249217in}{2.097322in}}%
\pgfpathlineto{\pgfqpoint{3.262807in}{2.090853in}}%
\pgfpathlineto{\pgfqpoint{3.276402in}{2.084416in}}%
\pgfpathlineto{\pgfqpoint{3.268064in}{2.083103in}}%
\pgfpathlineto{\pgfqpoint{3.259714in}{2.082043in}}%
\pgfpathlineto{\pgfqpoint{3.251351in}{2.081243in}}%
\pgfpathlineto{\pgfqpoint{3.242976in}{2.080711in}}%
\pgfpathlineto{\pgfqpoint{3.229355in}{2.087418in}}%
\pgfpathlineto{\pgfqpoint{3.215739in}{2.094157in}}%
\pgfpathlineto{\pgfqpoint{3.202127in}{2.100930in}}%
\pgfpathlineto{\pgfqpoint{3.188519in}{2.107735in}}%
\pgfpathlineto{\pgfqpoint{3.196921in}{2.107992in}}%
\pgfpathlineto{\pgfqpoint{3.205310in}{2.108519in}}%
\pgfpathlineto{\pgfqpoint{3.213686in}{2.109311in}}%
\pgfpathlineto{\pgfqpoint{3.222049in}{2.110359in}}%
\pgfpathclose%
\pgfusepath{fill}%
\end{pgfscope}%
\begin{pgfscope}%
\pgfpathrectangle{\pgfqpoint{1.150000in}{0.150000in}}{\pgfqpoint{5.700000in}{5.700000in}}%
\pgfusepath{clip}%
\pgfsetbuttcap%
\pgfsetroundjoin%
\definecolor{currentfill}{rgb}{0.280868,0.160771,0.472899}%
\pgfsetfillcolor{currentfill}%
\pgfsetfillopacity{0.700000}%
\pgfsetlinewidth{0.000000pt}%
\definecolor{currentstroke}{rgb}{0.000000,0.000000,0.000000}%
\pgfsetstrokecolor{currentstroke}%
\pgfsetdash{}{0pt}%
\pgfpathmoveto{\pgfqpoint{5.385861in}{2.254317in}}%
\pgfpathlineto{\pgfqpoint{5.399973in}{2.253484in}}%
\pgfpathlineto{\pgfqpoint{5.414093in}{2.252675in}}%
\pgfpathlineto{\pgfqpoint{5.428222in}{2.251892in}}%
\pgfpathlineto{\pgfqpoint{5.442359in}{2.251133in}}%
\pgfpathlineto{\pgfqpoint{5.434921in}{2.242945in}}%
\pgfpathlineto{\pgfqpoint{5.427474in}{2.234663in}}%
\pgfpathlineto{\pgfqpoint{5.420020in}{2.226287in}}%
\pgfpathlineto{\pgfqpoint{5.412558in}{2.217818in}}%
\pgfpathlineto{\pgfqpoint{5.398409in}{2.218605in}}%
\pgfpathlineto{\pgfqpoint{5.384268in}{2.219416in}}%
\pgfpathlineto{\pgfqpoint{5.370137in}{2.220252in}}%
\pgfpathlineto{\pgfqpoint{5.356014in}{2.221113in}}%
\pgfpathlineto{\pgfqpoint{5.363487in}{2.229549in}}%
\pgfpathlineto{\pgfqpoint{5.370953in}{2.237895in}}%
\pgfpathlineto{\pgfqpoint{5.378411in}{2.246151in}}%
\pgfpathlineto{\pgfqpoint{5.385861in}{2.254317in}}%
\pgfpathclose%
\pgfusepath{fill}%
\end{pgfscope}%
\begin{pgfscope}%
\pgfpathrectangle{\pgfqpoint{1.150000in}{0.150000in}}{\pgfqpoint{5.700000in}{5.700000in}}%
\pgfusepath{clip}%
\pgfsetbuttcap%
\pgfsetroundjoin%
\definecolor{currentfill}{rgb}{0.283072,0.130895,0.449241}%
\pgfsetfillcolor{currentfill}%
\pgfsetfillopacity{0.700000}%
\pgfsetlinewidth{0.000000pt}%
\definecolor{currentstroke}{rgb}{0.000000,0.000000,0.000000}%
\pgfsetstrokecolor{currentstroke}%
\pgfsetdash{}{0pt}%
\pgfpathmoveto{\pgfqpoint{3.025533in}{2.192057in}}%
\pgfpathlineto{\pgfqpoint{3.039094in}{2.184838in}}%
\pgfpathlineto{\pgfqpoint{3.052659in}{2.177654in}}%
\pgfpathlineto{\pgfqpoint{3.066228in}{2.170506in}}%
\pgfpathlineto{\pgfqpoint{3.079800in}{2.163394in}}%
\pgfpathlineto{\pgfqpoint{3.071328in}{2.163974in}}%
\pgfpathlineto{\pgfqpoint{3.062842in}{2.164848in}}%
\pgfpathlineto{\pgfqpoint{3.054341in}{2.166021in}}%
\pgfpathlineto{\pgfqpoint{3.045825in}{2.167503in}}%
\pgfpathlineto{\pgfqpoint{3.032223in}{2.174900in}}%
\pgfpathlineto{\pgfqpoint{3.018625in}{2.182333in}}%
\pgfpathlineto{\pgfqpoint{3.005031in}{2.189802in}}%
\pgfpathlineto{\pgfqpoint{2.991440in}{2.197307in}}%
\pgfpathlineto{\pgfqpoint{2.999987in}{2.195535in}}%
\pgfpathlineto{\pgfqpoint{3.008518in}{2.194074in}}%
\pgfpathlineto{\pgfqpoint{3.017033in}{2.192917in}}%
\pgfpathlineto{\pgfqpoint{3.025533in}{2.192057in}}%
\pgfpathclose%
\pgfusepath{fill}%
\end{pgfscope}%
\begin{pgfscope}%
\pgfpathrectangle{\pgfqpoint{1.150000in}{0.150000in}}{\pgfqpoint{5.700000in}{5.700000in}}%
\pgfusepath{clip}%
\pgfsetbuttcap%
\pgfsetroundjoin%
\definecolor{currentfill}{rgb}{0.282656,0.100196,0.422160}%
\pgfsetfillcolor{currentfill}%
\pgfsetfillopacity{0.700000}%
\pgfsetlinewidth{0.000000pt}%
\definecolor{currentstroke}{rgb}{0.000000,0.000000,0.000000}%
\pgfsetstrokecolor{currentstroke}%
\pgfsetdash{}{0pt}%
\pgfpathmoveto{\pgfqpoint{4.984458in}{2.138852in}}%
\pgfpathlineto{\pgfqpoint{4.998445in}{2.137329in}}%
\pgfpathlineto{\pgfqpoint{5.012440in}{2.135830in}}%
\pgfpathlineto{\pgfqpoint{5.026443in}{2.134357in}}%
\pgfpathlineto{\pgfqpoint{5.040454in}{2.132908in}}%
\pgfpathlineto{\pgfqpoint{5.032849in}{2.123683in}}%
\pgfpathlineto{\pgfqpoint{5.025237in}{2.114394in}}%
\pgfpathlineto{\pgfqpoint{5.017619in}{2.105044in}}%
\pgfpathlineto{\pgfqpoint{5.009995in}{2.095635in}}%
\pgfpathlineto{\pgfqpoint{4.995974in}{2.097166in}}%
\pgfpathlineto{\pgfqpoint{4.981961in}{2.098722in}}%
\pgfpathlineto{\pgfqpoint{4.967956in}{2.100303in}}%
\pgfpathlineto{\pgfqpoint{4.953959in}{2.101908in}}%
\pgfpathlineto{\pgfqpoint{4.961593in}{2.111230in}}%
\pgfpathlineto{\pgfqpoint{4.969221in}{2.120495in}}%
\pgfpathlineto{\pgfqpoint{4.976843in}{2.129703in}}%
\pgfpathlineto{\pgfqpoint{4.984458in}{2.138852in}}%
\pgfpathclose%
\pgfusepath{fill}%
\end{pgfscope}%
\begin{pgfscope}%
\pgfpathrectangle{\pgfqpoint{1.150000in}{0.150000in}}{\pgfqpoint{5.700000in}{5.700000in}}%
\pgfusepath{clip}%
\pgfsetbuttcap%
\pgfsetroundjoin%
\definecolor{currentfill}{rgb}{0.252194,0.269783,0.531579}%
\pgfsetfillcolor{currentfill}%
\pgfsetfillopacity{0.700000}%
\pgfsetlinewidth{0.000000pt}%
\definecolor{currentstroke}{rgb}{0.000000,0.000000,0.000000}%
\pgfsetstrokecolor{currentstroke}%
\pgfsetdash{}{0pt}%
\pgfpathmoveto{\pgfqpoint{2.522735in}{2.478185in}}%
\pgfpathlineto{\pgfqpoint{2.536270in}{2.469064in}}%
\pgfpathlineto{\pgfqpoint{2.549807in}{2.459990in}}%
\pgfpathlineto{\pgfqpoint{2.563345in}{2.450962in}}%
\pgfpathlineto{\pgfqpoint{2.576886in}{2.441981in}}%
\pgfpathlineto{\pgfqpoint{2.568007in}{2.447605in}}%
\pgfpathlineto{\pgfqpoint{2.559106in}{2.453616in}}%
\pgfpathlineto{\pgfqpoint{2.550182in}{2.460022in}}%
\pgfpathlineto{\pgfqpoint{2.541236in}{2.466832in}}%
\pgfpathlineto{\pgfqpoint{2.527656in}{2.476132in}}%
\pgfpathlineto{\pgfqpoint{2.514078in}{2.485478in}}%
\pgfpathlineto{\pgfqpoint{2.500502in}{2.494872in}}%
\pgfpathlineto{\pgfqpoint{2.486928in}{2.504312in}}%
\pgfpathlineto{\pgfqpoint{2.495915in}{2.497177in}}%
\pgfpathlineto{\pgfqpoint{2.504878in}{2.490450in}}%
\pgfpathlineto{\pgfqpoint{2.513818in}{2.484123in}}%
\pgfpathlineto{\pgfqpoint{2.522735in}{2.478185in}}%
\pgfpathclose%
\pgfusepath{fill}%
\end{pgfscope}%
\begin{pgfscope}%
\pgfpathrectangle{\pgfqpoint{1.150000in}{0.150000in}}{\pgfqpoint{5.700000in}{5.700000in}}%
\pgfusepath{clip}%
\pgfsetbuttcap%
\pgfsetroundjoin%
\definecolor{currentfill}{rgb}{0.271305,0.019942,0.347269}%
\pgfsetfillcolor{currentfill}%
\pgfsetfillopacity{0.700000}%
\pgfsetlinewidth{0.000000pt}%
\definecolor{currentstroke}{rgb}{0.000000,0.000000,0.000000}%
\pgfsetstrokecolor{currentstroke}%
\pgfsetdash{}{0pt}%
\pgfpathmoveto{\pgfqpoint{4.354650in}{1.991531in}}%
\pgfpathlineto{\pgfqpoint{4.368458in}{1.988501in}}%
\pgfpathlineto{\pgfqpoint{4.382272in}{1.985497in}}%
\pgfpathlineto{\pgfqpoint{4.396094in}{1.982518in}}%
\pgfpathlineto{\pgfqpoint{4.409922in}{1.979566in}}%
\pgfpathlineto{\pgfqpoint{4.402100in}{1.970737in}}%
\pgfpathlineto{\pgfqpoint{4.394273in}{1.961936in}}%
\pgfpathlineto{\pgfqpoint{4.386440in}{1.953168in}}%
\pgfpathlineto{\pgfqpoint{4.378601in}{1.944437in}}%
\pgfpathlineto{\pgfqpoint{4.364762in}{1.947552in}}%
\pgfpathlineto{\pgfqpoint{4.350929in}{1.950693in}}%
\pgfpathlineto{\pgfqpoint{4.337103in}{1.953859in}}%
\pgfpathlineto{\pgfqpoint{4.323284in}{1.957052in}}%
\pgfpathlineto{\pgfqpoint{4.331134in}{1.965616in}}%
\pgfpathlineto{\pgfqpoint{4.338978in}{1.974219in}}%
\pgfpathlineto{\pgfqpoint{4.346817in}{1.982859in}}%
\pgfpathlineto{\pgfqpoint{4.354650in}{1.991531in}}%
\pgfpathclose%
\pgfusepath{fill}%
\end{pgfscope}%
\begin{pgfscope}%
\pgfpathrectangle{\pgfqpoint{1.150000in}{0.150000in}}{\pgfqpoint{5.700000in}{5.700000in}}%
\pgfusepath{clip}%
\pgfsetbuttcap%
\pgfsetroundjoin%
\definecolor{currentfill}{rgb}{0.274128,0.199721,0.498911}%
\pgfsetfillcolor{currentfill}%
\pgfsetfillopacity{0.700000}%
\pgfsetlinewidth{0.000000pt}%
\definecolor{currentstroke}{rgb}{0.000000,0.000000,0.000000}%
\pgfsetstrokecolor{currentstroke}%
\pgfsetdash{}{0pt}%
\pgfpathmoveto{\pgfqpoint{5.700884in}{2.334999in}}%
\pgfpathlineto{\pgfqpoint{5.715099in}{2.334561in}}%
\pgfpathlineto{\pgfqpoint{5.729322in}{2.334148in}}%
\pgfpathlineto{\pgfqpoint{5.743555in}{2.333760in}}%
\pgfpathlineto{\pgfqpoint{5.757797in}{2.333396in}}%
\pgfpathlineto{\pgfqpoint{5.750509in}{2.326383in}}%
\pgfpathlineto{\pgfqpoint{5.743214in}{2.319268in}}%
\pgfpathlineto{\pgfqpoint{5.735909in}{2.312051in}}%
\pgfpathlineto{\pgfqpoint{5.728595in}{2.304729in}}%
\pgfpathlineto{\pgfqpoint{5.714340in}{2.305079in}}%
\pgfpathlineto{\pgfqpoint{5.700093in}{2.305454in}}%
\pgfpathlineto{\pgfqpoint{5.685855in}{2.305853in}}%
\pgfpathlineto{\pgfqpoint{5.671627in}{2.306277in}}%
\pgfpathlineto{\pgfqpoint{5.678954in}{2.313607in}}%
\pgfpathlineto{\pgfqpoint{5.686273in}{2.320837in}}%
\pgfpathlineto{\pgfqpoint{5.693583in}{2.327967in}}%
\pgfpathlineto{\pgfqpoint{5.700884in}{2.334999in}}%
\pgfpathclose%
\pgfusepath{fill}%
\end{pgfscope}%
\begin{pgfscope}%
\pgfpathrectangle{\pgfqpoint{1.150000in}{0.150000in}}{\pgfqpoint{5.700000in}{5.700000in}}%
\pgfusepath{clip}%
\pgfsetbuttcap%
\pgfsetroundjoin%
\definecolor{currentfill}{rgb}{0.277018,0.050344,0.375715}%
\pgfsetfillcolor{currentfill}%
\pgfsetfillopacity{0.700000}%
\pgfsetlinewidth{0.000000pt}%
\definecolor{currentstroke}{rgb}{0.000000,0.000000,0.000000}%
\pgfsetstrokecolor{currentstroke}%
\pgfsetdash{}{0pt}%
\pgfpathmoveto{\pgfqpoint{3.418365in}{2.043772in}}%
\pgfpathlineto{\pgfqpoint{3.431977in}{2.037877in}}%
\pgfpathlineto{\pgfqpoint{3.445594in}{2.032013in}}%
\pgfpathlineto{\pgfqpoint{3.459216in}{2.026179in}}%
\pgfpathlineto{\pgfqpoint{3.472843in}{2.020376in}}%
\pgfpathlineto{\pgfqpoint{3.464620in}{2.017365in}}%
\pgfpathlineto{\pgfqpoint{3.456387in}{2.014571in}}%
\pgfpathlineto{\pgfqpoint{3.448144in}{2.011998in}}%
\pgfpathlineto{\pgfqpoint{3.439889in}{2.009655in}}%
\pgfpathlineto{\pgfqpoint{3.426240in}{2.015714in}}%
\pgfpathlineto{\pgfqpoint{3.412596in}{2.021804in}}%
\pgfpathlineto{\pgfqpoint{3.398956in}{2.027924in}}%
\pgfpathlineto{\pgfqpoint{3.385320in}{2.034075in}}%
\pgfpathlineto{\pgfqpoint{3.393598in}{2.036158in}}%
\pgfpathlineto{\pgfqpoint{3.401864in}{2.038472in}}%
\pgfpathlineto{\pgfqpoint{3.410120in}{2.041013in}}%
\pgfpathlineto{\pgfqpoint{3.418365in}{2.043772in}}%
\pgfpathclose%
\pgfusepath{fill}%
\end{pgfscope}%
\begin{pgfscope}%
\pgfpathrectangle{\pgfqpoint{1.150000in}{0.150000in}}{\pgfqpoint{5.700000in}{5.700000in}}%
\pgfusepath{clip}%
\pgfsetbuttcap%
\pgfsetroundjoin%
\definecolor{currentfill}{rgb}{0.267004,0.004874,0.329415}%
\pgfsetfillcolor{currentfill}%
\pgfsetfillopacity{0.700000}%
\pgfsetlinewidth{0.000000pt}%
\definecolor{currentstroke}{rgb}{0.000000,0.000000,0.000000}%
\pgfsetstrokecolor{currentstroke}%
\pgfsetdash{}{0pt}%
\pgfpathmoveto{\pgfqpoint{4.126377in}{1.965727in}}%
\pgfpathlineto{\pgfqpoint{4.140129in}{1.962045in}}%
\pgfpathlineto{\pgfqpoint{4.153887in}{1.958389in}}%
\pgfpathlineto{\pgfqpoint{4.167652in}{1.954760in}}%
\pgfpathlineto{\pgfqpoint{4.181423in}{1.951157in}}%
\pgfpathlineto{\pgfqpoint{4.173521in}{1.943222in}}%
\pgfpathlineto{\pgfqpoint{4.165613in}{1.935359in}}%
\pgfpathlineto{\pgfqpoint{4.157699in}{1.927572in}}%
\pgfpathlineto{\pgfqpoint{4.149779in}{1.919867in}}%
\pgfpathlineto{\pgfqpoint{4.135995in}{1.923659in}}%
\pgfpathlineto{\pgfqpoint{4.122217in}{1.927477in}}%
\pgfpathlineto{\pgfqpoint{4.108445in}{1.931322in}}%
\pgfpathlineto{\pgfqpoint{4.094680in}{1.935193in}}%
\pgfpathlineto{\pgfqpoint{4.102613in}{1.942704in}}%
\pgfpathlineto{\pgfqpoint{4.110541in}{1.950300in}}%
\pgfpathlineto{\pgfqpoint{4.118462in}{1.957976in}}%
\pgfpathlineto{\pgfqpoint{4.126377in}{1.965727in}}%
\pgfpathclose%
\pgfusepath{fill}%
\end{pgfscope}%
\begin{pgfscope}%
\pgfpathrectangle{\pgfqpoint{1.150000in}{0.150000in}}{\pgfqpoint{5.700000in}{5.700000in}}%
\pgfusepath{clip}%
\pgfsetbuttcap%
\pgfsetroundjoin%
\definecolor{currentfill}{rgb}{0.276022,0.044167,0.370164}%
\pgfsetfillcolor{currentfill}%
\pgfsetfillopacity{0.700000}%
\pgfsetlinewidth{0.000000pt}%
\definecolor{currentstroke}{rgb}{0.000000,0.000000,0.000000}%
\pgfsetstrokecolor{currentstroke}%
\pgfsetdash{}{0pt}%
\pgfpathmoveto{\pgfqpoint{4.583018in}{2.030388in}}%
\pgfpathlineto{\pgfqpoint{4.596890in}{2.027951in}}%
\pgfpathlineto{\pgfqpoint{4.610769in}{2.025539in}}%
\pgfpathlineto{\pgfqpoint{4.624656in}{2.023153in}}%
\pgfpathlineto{\pgfqpoint{4.638550in}{2.020792in}}%
\pgfpathlineto{\pgfqpoint{4.630802in}{2.011467in}}%
\pgfpathlineto{\pgfqpoint{4.623049in}{2.002134in}}%
\pgfpathlineto{\pgfqpoint{4.615291in}{1.992794in}}%
\pgfpathlineto{\pgfqpoint{4.607527in}{1.983453in}}%
\pgfpathlineto{\pgfqpoint{4.593623in}{1.985950in}}%
\pgfpathlineto{\pgfqpoint{4.579726in}{1.988472in}}%
\pgfpathlineto{\pgfqpoint{4.565836in}{1.991020in}}%
\pgfpathlineto{\pgfqpoint{4.551954in}{1.993593in}}%
\pgfpathlineto{\pgfqpoint{4.559728in}{2.002794in}}%
\pgfpathlineto{\pgfqpoint{4.567497in}{2.011995in}}%
\pgfpathlineto{\pgfqpoint{4.575260in}{2.021194in}}%
\pgfpathlineto{\pgfqpoint{4.583018in}{2.030388in}}%
\pgfpathclose%
\pgfusepath{fill}%
\end{pgfscope}%
\begin{pgfscope}%
\pgfpathrectangle{\pgfqpoint{1.150000in}{0.150000in}}{\pgfqpoint{5.700000in}{5.700000in}}%
\pgfusepath{clip}%
\pgfsetbuttcap%
\pgfsetroundjoin%
\definecolor{currentfill}{rgb}{0.282290,0.145912,0.461510}%
\pgfsetfillcolor{currentfill}%
\pgfsetfillopacity{0.700000}%
\pgfsetlinewidth{0.000000pt}%
\definecolor{currentstroke}{rgb}{0.000000,0.000000,0.000000}%
\pgfsetstrokecolor{currentstroke}%
\pgfsetdash{}{0pt}%
\pgfpathmoveto{\pgfqpoint{5.299608in}{2.224804in}}%
\pgfpathlineto{\pgfqpoint{5.313696in}{2.223844in}}%
\pgfpathlineto{\pgfqpoint{5.327794in}{2.222909in}}%
\pgfpathlineto{\pgfqpoint{5.341899in}{2.221999in}}%
\pgfpathlineto{\pgfqpoint{5.356014in}{2.221113in}}%
\pgfpathlineto{\pgfqpoint{5.348533in}{2.212588in}}%
\pgfpathlineto{\pgfqpoint{5.341045in}{2.203975in}}%
\pgfpathlineto{\pgfqpoint{5.333549in}{2.195273in}}%
\pgfpathlineto{\pgfqpoint{5.326047in}{2.186483in}}%
\pgfpathlineto{\pgfqpoint{5.311921in}{2.187410in}}%
\pgfpathlineto{\pgfqpoint{5.297804in}{2.188362in}}%
\pgfpathlineto{\pgfqpoint{5.283696in}{2.189338in}}%
\pgfpathlineto{\pgfqpoint{5.269597in}{2.190340in}}%
\pgfpathlineto{\pgfqpoint{5.277110in}{2.199083in}}%
\pgfpathlineto{\pgfqpoint{5.284617in}{2.207742in}}%
\pgfpathlineto{\pgfqpoint{5.292116in}{2.216315in}}%
\pgfpathlineto{\pgfqpoint{5.299608in}{2.224804in}}%
\pgfpathclose%
\pgfusepath{fill}%
\end{pgfscope}%
\begin{pgfscope}%
\pgfpathrectangle{\pgfqpoint{1.150000in}{0.150000in}}{\pgfqpoint{5.700000in}{5.700000in}}%
\pgfusepath{clip}%
\pgfsetbuttcap%
\pgfsetroundjoin%
\definecolor{currentfill}{rgb}{0.281924,0.089666,0.412415}%
\pgfsetfillcolor{currentfill}%
\pgfsetfillopacity{0.700000}%
\pgfsetlinewidth{0.000000pt}%
\definecolor{currentstroke}{rgb}{0.000000,0.000000,0.000000}%
\pgfsetstrokecolor{currentstroke}%
\pgfsetdash{}{0pt}%
\pgfpathmoveto{\pgfqpoint{4.898053in}{2.108581in}}%
\pgfpathlineto{\pgfqpoint{4.912017in}{2.106875in}}%
\pgfpathlineto{\pgfqpoint{4.925990in}{2.105195in}}%
\pgfpathlineto{\pgfqpoint{4.939971in}{2.103539in}}%
\pgfpathlineto{\pgfqpoint{4.953959in}{2.101908in}}%
\pgfpathlineto{\pgfqpoint{4.946319in}{2.092533in}}%
\pgfpathlineto{\pgfqpoint{4.938673in}{2.083105in}}%
\pgfpathlineto{\pgfqpoint{4.931021in}{2.073627in}}%
\pgfpathlineto{\pgfqpoint{4.923364in}{2.064100in}}%
\pgfpathlineto{\pgfqpoint{4.909365in}{2.065827in}}%
\pgfpathlineto{\pgfqpoint{4.895375in}{2.067579in}}%
\pgfpathlineto{\pgfqpoint{4.881392in}{2.069355in}}%
\pgfpathlineto{\pgfqpoint{4.867418in}{2.071157in}}%
\pgfpathlineto{\pgfqpoint{4.875085in}{2.080582in}}%
\pgfpathlineto{\pgfqpoint{4.882747in}{2.089963in}}%
\pgfpathlineto{\pgfqpoint{4.890403in}{2.099296in}}%
\pgfpathlineto{\pgfqpoint{4.898053in}{2.108581in}}%
\pgfpathclose%
\pgfusepath{fill}%
\end{pgfscope}%
\begin{pgfscope}%
\pgfpathrectangle{\pgfqpoint{1.150000in}{0.150000in}}{\pgfqpoint{5.700000in}{5.700000in}}%
\pgfusepath{clip}%
\pgfsetbuttcap%
\pgfsetroundjoin%
\definecolor{currentfill}{rgb}{0.277134,0.185228,0.489898}%
\pgfsetfillcolor{currentfill}%
\pgfsetfillopacity{0.700000}%
\pgfsetlinewidth{0.000000pt}%
\definecolor{currentstroke}{rgb}{0.000000,0.000000,0.000000}%
\pgfsetstrokecolor{currentstroke}%
\pgfsetdash{}{0pt}%
\pgfpathmoveto{\pgfqpoint{2.828616in}{2.290257in}}%
\pgfpathlineto{\pgfqpoint{2.842167in}{2.282300in}}%
\pgfpathlineto{\pgfqpoint{2.855721in}{2.274383in}}%
\pgfpathlineto{\pgfqpoint{2.869278in}{2.266505in}}%
\pgfpathlineto{\pgfqpoint{2.882839in}{2.258666in}}%
\pgfpathlineto{\pgfqpoint{2.874212in}{2.261347in}}%
\pgfpathlineto{\pgfqpoint{2.865569in}{2.264362in}}%
\pgfpathlineto{\pgfqpoint{2.856907in}{2.267721in}}%
\pgfpathlineto{\pgfqpoint{2.848227in}{2.271430in}}%
\pgfpathlineto{\pgfqpoint{2.834634in}{2.279570in}}%
\pgfpathlineto{\pgfqpoint{2.821043in}{2.287749in}}%
\pgfpathlineto{\pgfqpoint{2.807456in}{2.295967in}}%
\pgfpathlineto{\pgfqpoint{2.793871in}{2.304224in}}%
\pgfpathlineto{\pgfqpoint{2.802586in}{2.300209in}}%
\pgfpathlineto{\pgfqpoint{2.811281in}{2.296548in}}%
\pgfpathlineto{\pgfqpoint{2.819958in}{2.293233in}}%
\pgfpathlineto{\pgfqpoint{2.828616in}{2.290257in}}%
\pgfpathclose%
\pgfusepath{fill}%
\end{pgfscope}%
\begin{pgfscope}%
\pgfpathrectangle{\pgfqpoint{1.150000in}{0.150000in}}{\pgfqpoint{5.700000in}{5.700000in}}%
\pgfusepath{clip}%
\pgfsetbuttcap%
\pgfsetroundjoin%
\definecolor{currentfill}{rgb}{0.276194,0.190074,0.493001}%
\pgfsetfillcolor{currentfill}%
\pgfsetfillopacity{0.700000}%
\pgfsetlinewidth{0.000000pt}%
\definecolor{currentstroke}{rgb}{0.000000,0.000000,0.000000}%
\pgfsetstrokecolor{currentstroke}%
\pgfsetdash{}{0pt}%
\pgfpathmoveto{\pgfqpoint{5.614804in}{2.308220in}}%
\pgfpathlineto{\pgfqpoint{5.628996in}{2.307697in}}%
\pgfpathlineto{\pgfqpoint{5.643197in}{2.307199in}}%
\pgfpathlineto{\pgfqpoint{5.657407in}{2.306726in}}%
\pgfpathlineto{\pgfqpoint{5.671627in}{2.306277in}}%
\pgfpathlineto{\pgfqpoint{5.664291in}{2.298846in}}%
\pgfpathlineto{\pgfqpoint{5.656947in}{2.291314in}}%
\pgfpathlineto{\pgfqpoint{5.649594in}{2.283679in}}%
\pgfpathlineto{\pgfqpoint{5.642233in}{2.275941in}}%
\pgfpathlineto{\pgfqpoint{5.628001in}{2.276389in}}%
\pgfpathlineto{\pgfqpoint{5.613777in}{2.276863in}}%
\pgfpathlineto{\pgfqpoint{5.599563in}{2.277361in}}%
\pgfpathlineto{\pgfqpoint{5.585357in}{2.277883in}}%
\pgfpathlineto{\pgfqpoint{5.592731in}{2.285616in}}%
\pgfpathlineto{\pgfqpoint{5.600097in}{2.293249in}}%
\pgfpathlineto{\pgfqpoint{5.607455in}{2.300784in}}%
\pgfpathlineto{\pgfqpoint{5.614804in}{2.308220in}}%
\pgfpathclose%
\pgfusepath{fill}%
\end{pgfscope}%
\begin{pgfscope}%
\pgfpathrectangle{\pgfqpoint{1.150000in}{0.150000in}}{\pgfqpoint{5.700000in}{5.700000in}}%
\pgfusepath{clip}%
\pgfsetbuttcap%
\pgfsetroundjoin%
\definecolor{currentfill}{rgb}{0.269944,0.014625,0.341379}%
\pgfsetfillcolor{currentfill}%
\pgfsetfillopacity{0.700000}%
\pgfsetlinewidth{0.000000pt}%
\definecolor{currentstroke}{rgb}{0.000000,0.000000,0.000000}%
\pgfsetstrokecolor{currentstroke}%
\pgfsetdash{}{0pt}%
\pgfpathmoveto{\pgfqpoint{3.756361in}{1.970187in}}%
\pgfpathlineto{\pgfqpoint{3.770038in}{1.965352in}}%
\pgfpathlineto{\pgfqpoint{3.783720in}{1.960546in}}%
\pgfpathlineto{\pgfqpoint{3.797408in}{1.955767in}}%
\pgfpathlineto{\pgfqpoint{3.811102in}{1.951017in}}%
\pgfpathlineto{\pgfqpoint{3.803047in}{1.945387in}}%
\pgfpathlineto{\pgfqpoint{3.794984in}{1.939907in}}%
\pgfpathlineto{\pgfqpoint{3.786913in}{1.934583in}}%
\pgfpathlineto{\pgfqpoint{3.778835in}{1.929420in}}%
\pgfpathlineto{\pgfqpoint{3.765124in}{1.934399in}}%
\pgfpathlineto{\pgfqpoint{3.751419in}{1.939407in}}%
\pgfpathlineto{\pgfqpoint{3.737718in}{1.944442in}}%
\pgfpathlineto{\pgfqpoint{3.724024in}{1.949505in}}%
\pgfpathlineto{\pgfqpoint{3.732120in}{1.954434in}}%
\pgfpathlineto{\pgfqpoint{3.740209in}{1.959528in}}%
\pgfpathlineto{\pgfqpoint{3.748289in}{1.964781in}}%
\pgfpathlineto{\pgfqpoint{3.756361in}{1.970187in}}%
\pgfpathclose%
\pgfusepath{fill}%
\end{pgfscope}%
\begin{pgfscope}%
\pgfpathrectangle{\pgfqpoint{1.150000in}{0.150000in}}{\pgfqpoint{5.700000in}{5.700000in}}%
\pgfusepath{clip}%
\pgfsetbuttcap%
\pgfsetroundjoin%
\definecolor{currentfill}{rgb}{0.282884,0.135920,0.453427}%
\pgfsetfillcolor{currentfill}%
\pgfsetfillopacity{0.700000}%
\pgfsetlinewidth{0.000000pt}%
\definecolor{currentstroke}{rgb}{0.000000,0.000000,0.000000}%
\pgfsetstrokecolor{currentstroke}%
\pgfsetdash{}{0pt}%
\pgfpathmoveto{\pgfqpoint{5.213284in}{2.194593in}}%
\pgfpathlineto{\pgfqpoint{5.227350in}{2.193492in}}%
\pgfpathlineto{\pgfqpoint{5.241424in}{2.192417in}}%
\pgfpathlineto{\pgfqpoint{5.255506in}{2.191366in}}%
\pgfpathlineto{\pgfqpoint{5.269597in}{2.190340in}}%
\pgfpathlineto{\pgfqpoint{5.262076in}{2.181513in}}%
\pgfpathlineto{\pgfqpoint{5.254548in}{2.172604in}}%
\pgfpathlineto{\pgfqpoint{5.247014in}{2.163612in}}%
\pgfpathlineto{\pgfqpoint{5.239472in}{2.154540in}}%
\pgfpathlineto{\pgfqpoint{5.225370in}{2.155621in}}%
\pgfpathlineto{\pgfqpoint{5.211277in}{2.156728in}}%
\pgfpathlineto{\pgfqpoint{5.197193in}{2.157859in}}%
\pgfpathlineto{\pgfqpoint{5.183117in}{2.159015in}}%
\pgfpathlineto{\pgfqpoint{5.190669in}{2.168026in}}%
\pgfpathlineto{\pgfqpoint{5.198214in}{2.176961in}}%
\pgfpathlineto{\pgfqpoint{5.205753in}{2.185816in}}%
\pgfpathlineto{\pgfqpoint{5.213284in}{2.194593in}}%
\pgfpathclose%
\pgfusepath{fill}%
\end{pgfscope}%
\begin{pgfscope}%
\pgfpathrectangle{\pgfqpoint{1.150000in}{0.150000in}}{\pgfqpoint{5.700000in}{5.700000in}}%
\pgfusepath{clip}%
\pgfsetbuttcap%
\pgfsetroundjoin%
\definecolor{currentfill}{rgb}{0.268510,0.009605,0.335427}%
\pgfsetfillcolor{currentfill}%
\pgfsetfillopacity{0.700000}%
\pgfsetlinewidth{0.000000pt}%
\definecolor{currentstroke}{rgb}{0.000000,0.000000,0.000000}%
\pgfsetstrokecolor{currentstroke}%
\pgfsetdash{}{0pt}%
\pgfpathmoveto{\pgfqpoint{4.268076in}{1.970081in}}%
\pgfpathlineto{\pgfqpoint{4.281867in}{1.966784in}}%
\pgfpathlineto{\pgfqpoint{4.295666in}{1.963514in}}%
\pgfpathlineto{\pgfqpoint{4.309472in}{1.960270in}}%
\pgfpathlineto{\pgfqpoint{4.323284in}{1.957052in}}%
\pgfpathlineto{\pgfqpoint{4.315428in}{1.948531in}}%
\pgfpathlineto{\pgfqpoint{4.307568in}{1.940059in}}%
\pgfpathlineto{\pgfqpoint{4.299701in}{1.931639in}}%
\pgfpathlineto{\pgfqpoint{4.291830in}{1.923275in}}%
\pgfpathlineto{\pgfqpoint{4.278006in}{1.926669in}}%
\pgfpathlineto{\pgfqpoint{4.264188in}{1.930089in}}%
\pgfpathlineto{\pgfqpoint{4.250377in}{1.933535in}}%
\pgfpathlineto{\pgfqpoint{4.236573in}{1.937007in}}%
\pgfpathlineto{\pgfqpoint{4.244457in}{1.945190in}}%
\pgfpathlineto{\pgfqpoint{4.252336in}{1.953433in}}%
\pgfpathlineto{\pgfqpoint{4.260208in}{1.961731in}}%
\pgfpathlineto{\pgfqpoint{4.268076in}{1.970081in}}%
\pgfpathclose%
\pgfusepath{fill}%
\end{pgfscope}%
\begin{pgfscope}%
\pgfpathrectangle{\pgfqpoint{1.150000in}{0.150000in}}{\pgfqpoint{5.700000in}{5.700000in}}%
\pgfusepath{clip}%
\pgfsetbuttcap%
\pgfsetroundjoin%
\definecolor{currentfill}{rgb}{0.257322,0.256130,0.526563}%
\pgfsetfillcolor{currentfill}%
\pgfsetfillopacity{0.700000}%
\pgfsetlinewidth{0.000000pt}%
\definecolor{currentstroke}{rgb}{0.000000,0.000000,0.000000}%
\pgfsetstrokecolor{currentstroke}%
\pgfsetdash{}{0pt}%
\pgfpathmoveto{\pgfqpoint{2.576886in}{2.441981in}}%
\pgfpathlineto{\pgfqpoint{2.590430in}{2.433045in}}%
\pgfpathlineto{\pgfqpoint{2.603975in}{2.424155in}}%
\pgfpathlineto{\pgfqpoint{2.617523in}{2.415309in}}%
\pgfpathlineto{\pgfqpoint{2.631073in}{2.406508in}}%
\pgfpathlineto{\pgfqpoint{2.622231in}{2.411820in}}%
\pgfpathlineto{\pgfqpoint{2.613368in}{2.417515in}}%
\pgfpathlineto{\pgfqpoint{2.604483in}{2.423601in}}%
\pgfpathlineto{\pgfqpoint{2.595575in}{2.430087in}}%
\pgfpathlineto{\pgfqpoint{2.581987in}{2.439206in}}%
\pgfpathlineto{\pgfqpoint{2.568401in}{2.448370in}}%
\pgfpathlineto{\pgfqpoint{2.554817in}{2.457578in}}%
\pgfpathlineto{\pgfqpoint{2.541236in}{2.466832in}}%
\pgfpathlineto{\pgfqpoint{2.550182in}{2.460022in}}%
\pgfpathlineto{\pgfqpoint{2.559106in}{2.453616in}}%
\pgfpathlineto{\pgfqpoint{2.568007in}{2.447605in}}%
\pgfpathlineto{\pgfqpoint{2.576886in}{2.441981in}}%
\pgfpathclose%
\pgfusepath{fill}%
\end{pgfscope}%
\begin{pgfscope}%
\pgfpathrectangle{\pgfqpoint{1.150000in}{0.150000in}}{\pgfqpoint{5.700000in}{5.700000in}}%
\pgfusepath{clip}%
\pgfsetbuttcap%
\pgfsetroundjoin%
\definecolor{currentfill}{rgb}{0.267004,0.004874,0.329415}%
\pgfsetfillcolor{currentfill}%
\pgfsetfillopacity{0.700000}%
\pgfsetlinewidth{0.000000pt}%
\definecolor{currentstroke}{rgb}{0.000000,0.000000,0.000000}%
\pgfsetstrokecolor{currentstroke}%
\pgfsetdash{}{0pt}%
\pgfpathmoveto{\pgfqpoint{3.898014in}{1.957060in}}%
\pgfpathlineto{\pgfqpoint{3.911720in}{1.952664in}}%
\pgfpathlineto{\pgfqpoint{3.925432in}{1.948294in}}%
\pgfpathlineto{\pgfqpoint{3.939150in}{1.943952in}}%
\pgfpathlineto{\pgfqpoint{3.952874in}{1.939637in}}%
\pgfpathlineto{\pgfqpoint{3.944880in}{1.933043in}}%
\pgfpathlineto{\pgfqpoint{3.936879in}{1.926569in}}%
\pgfpathlineto{\pgfqpoint{3.928872in}{1.920223in}}%
\pgfpathlineto{\pgfqpoint{3.920857in}{1.914008in}}%
\pgfpathlineto{\pgfqpoint{3.907117in}{1.918538in}}%
\pgfpathlineto{\pgfqpoint{3.893383in}{1.923096in}}%
\pgfpathlineto{\pgfqpoint{3.879655in}{1.927680in}}%
\pgfpathlineto{\pgfqpoint{3.865933in}{1.932292in}}%
\pgfpathlineto{\pgfqpoint{3.873964in}{1.938287in}}%
\pgfpathlineto{\pgfqpoint{3.881988in}{1.944417in}}%
\pgfpathlineto{\pgfqpoint{3.890004in}{1.950676in}}%
\pgfpathlineto{\pgfqpoint{3.898014in}{1.957060in}}%
\pgfpathclose%
\pgfusepath{fill}%
\end{pgfscope}%
\begin{pgfscope}%
\pgfpathrectangle{\pgfqpoint{1.150000in}{0.150000in}}{\pgfqpoint{5.700000in}{5.700000in}}%
\pgfusepath{clip}%
\pgfsetbuttcap%
\pgfsetroundjoin%
\definecolor{currentfill}{rgb}{0.281446,0.084320,0.407414}%
\pgfsetfillcolor{currentfill}%
\pgfsetfillopacity{0.700000}%
\pgfsetlinewidth{0.000000pt}%
\definecolor{currentstroke}{rgb}{0.000000,0.000000,0.000000}%
\pgfsetstrokecolor{currentstroke}%
\pgfsetdash{}{0pt}%
\pgfpathmoveto{\pgfqpoint{3.276402in}{2.084416in}}%
\pgfpathlineto{\pgfqpoint{3.290001in}{2.078012in}}%
\pgfpathlineto{\pgfqpoint{3.303605in}{2.071640in}}%
\pgfpathlineto{\pgfqpoint{3.317213in}{2.065300in}}%
\pgfpathlineto{\pgfqpoint{3.330825in}{2.058992in}}%
\pgfpathlineto{\pgfqpoint{3.322512in}{2.057414in}}%
\pgfpathlineto{\pgfqpoint{3.314187in}{2.056086in}}%
\pgfpathlineto{\pgfqpoint{3.305850in}{2.055014in}}%
\pgfpathlineto{\pgfqpoint{3.297501in}{2.054207in}}%
\pgfpathlineto{\pgfqpoint{3.283863in}{2.060785in}}%
\pgfpathlineto{\pgfqpoint{3.270230in}{2.067395in}}%
\pgfpathlineto{\pgfqpoint{3.256601in}{2.074037in}}%
\pgfpathlineto{\pgfqpoint{3.242976in}{2.080711in}}%
\pgfpathlineto{\pgfqpoint{3.251351in}{2.081243in}}%
\pgfpathlineto{\pgfqpoint{3.259714in}{2.082043in}}%
\pgfpathlineto{\pgfqpoint{3.268064in}{2.083103in}}%
\pgfpathlineto{\pgfqpoint{3.276402in}{2.084416in}}%
\pgfpathclose%
\pgfusepath{fill}%
\end{pgfscope}%
\begin{pgfscope}%
\pgfpathrectangle{\pgfqpoint{1.150000in}{0.150000in}}{\pgfqpoint{5.700000in}{5.700000in}}%
\pgfusepath{clip}%
\pgfsetbuttcap%
\pgfsetroundjoin%
\definecolor{currentfill}{rgb}{0.280267,0.073417,0.397163}%
\pgfsetfillcolor{currentfill}%
\pgfsetfillopacity{0.700000}%
\pgfsetlinewidth{0.000000pt}%
\definecolor{currentstroke}{rgb}{0.000000,0.000000,0.000000}%
\pgfsetstrokecolor{currentstroke}%
\pgfsetdash{}{0pt}%
\pgfpathmoveto{\pgfqpoint{4.811598in}{2.078614in}}%
\pgfpathlineto{\pgfqpoint{4.825541in}{2.076712in}}%
\pgfpathlineto{\pgfqpoint{4.839492in}{2.074836in}}%
\pgfpathlineto{\pgfqpoint{4.853451in}{2.072984in}}%
\pgfpathlineto{\pgfqpoint{4.867418in}{2.071157in}}%
\pgfpathlineto{\pgfqpoint{4.859744in}{2.061689in}}%
\pgfpathlineto{\pgfqpoint{4.852065in}{2.052181in}}%
\pgfpathlineto{\pgfqpoint{4.844380in}{2.042635in}}%
\pgfpathlineto{\pgfqpoint{4.836689in}{2.033053in}}%
\pgfpathlineto{\pgfqpoint{4.822713in}{2.034990in}}%
\pgfpathlineto{\pgfqpoint{4.808745in}{2.036951in}}%
\pgfpathlineto{\pgfqpoint{4.794784in}{2.038938in}}%
\pgfpathlineto{\pgfqpoint{4.780831in}{2.040949in}}%
\pgfpathlineto{\pgfqpoint{4.788532in}{2.050416in}}%
\pgfpathlineto{\pgfqpoint{4.796226in}{2.059851in}}%
\pgfpathlineto{\pgfqpoint{4.803915in}{2.069251in}}%
\pgfpathlineto{\pgfqpoint{4.811598in}{2.078614in}}%
\pgfpathclose%
\pgfusepath{fill}%
\end{pgfscope}%
\begin{pgfscope}%
\pgfpathrectangle{\pgfqpoint{1.150000in}{0.150000in}}{\pgfqpoint{5.700000in}{5.700000in}}%
\pgfusepath{clip}%
\pgfsetbuttcap%
\pgfsetroundjoin%
\definecolor{currentfill}{rgb}{0.272594,0.025563,0.353093}%
\pgfsetfillcolor{currentfill}%
\pgfsetfillopacity{0.700000}%
\pgfsetlinewidth{0.000000pt}%
\definecolor{currentstroke}{rgb}{0.000000,0.000000,0.000000}%
\pgfsetstrokecolor{currentstroke}%
\pgfsetdash{}{0pt}%
\pgfpathmoveto{\pgfqpoint{3.614662in}{1.991040in}}%
\pgfpathlineto{\pgfqpoint{3.628313in}{1.985747in}}%
\pgfpathlineto{\pgfqpoint{3.641970in}{1.980484in}}%
\pgfpathlineto{\pgfqpoint{3.655632in}{1.975249in}}%
\pgfpathlineto{\pgfqpoint{3.669300in}{1.970043in}}%
\pgfpathlineto{\pgfqpoint{3.661176in}{1.965522in}}%
\pgfpathlineto{\pgfqpoint{3.653044in}{1.961181in}}%
\pgfpathlineto{\pgfqpoint{3.644903in}{1.957026in}}%
\pgfpathlineto{\pgfqpoint{3.636753in}{1.953065in}}%
\pgfpathlineto{\pgfqpoint{3.623066in}{1.958513in}}%
\pgfpathlineto{\pgfqpoint{3.609384in}{1.963990in}}%
\pgfpathlineto{\pgfqpoint{3.595707in}{1.969496in}}%
\pgfpathlineto{\pgfqpoint{3.582035in}{1.975031in}}%
\pgfpathlineto{\pgfqpoint{3.590205in}{1.978745in}}%
\pgfpathlineto{\pgfqpoint{3.598367in}{1.982656in}}%
\pgfpathlineto{\pgfqpoint{3.606519in}{1.986756in}}%
\pgfpathlineto{\pgfqpoint{3.614662in}{1.991040in}}%
\pgfpathclose%
\pgfusepath{fill}%
\end{pgfscope}%
\begin{pgfscope}%
\pgfpathrectangle{\pgfqpoint{1.150000in}{0.150000in}}{\pgfqpoint{5.700000in}{5.700000in}}%
\pgfusepath{clip}%
\pgfsetbuttcap%
\pgfsetroundjoin%
\definecolor{currentfill}{rgb}{0.283187,0.125848,0.444960}%
\pgfsetfillcolor{currentfill}%
\pgfsetfillopacity{0.700000}%
\pgfsetlinewidth{0.000000pt}%
\definecolor{currentstroke}{rgb}{0.000000,0.000000,0.000000}%
\pgfsetstrokecolor{currentstroke}%
\pgfsetdash{}{0pt}%
\pgfpathmoveto{\pgfqpoint{3.079800in}{2.163394in}}%
\pgfpathlineto{\pgfqpoint{3.093376in}{2.156316in}}%
\pgfpathlineto{\pgfqpoint{3.106956in}{2.149273in}}%
\pgfpathlineto{\pgfqpoint{3.120540in}{2.142265in}}%
\pgfpathlineto{\pgfqpoint{3.134128in}{2.135291in}}%
\pgfpathlineto{\pgfqpoint{3.125684in}{2.135592in}}%
\pgfpathlineto{\pgfqpoint{3.117227in}{2.136183in}}%
\pgfpathlineto{\pgfqpoint{3.108755in}{2.137069in}}%
\pgfpathlineto{\pgfqpoint{3.100268in}{2.138261in}}%
\pgfpathlineto{\pgfqpoint{3.086652in}{2.145519in}}%
\pgfpathlineto{\pgfqpoint{3.073039in}{2.152812in}}%
\pgfpathlineto{\pgfqpoint{3.059430in}{2.160140in}}%
\pgfpathlineto{\pgfqpoint{3.045825in}{2.167503in}}%
\pgfpathlineto{\pgfqpoint{3.054341in}{2.166021in}}%
\pgfpathlineto{\pgfqpoint{3.062842in}{2.164848in}}%
\pgfpathlineto{\pgfqpoint{3.071328in}{2.163974in}}%
\pgfpathlineto{\pgfqpoint{3.079800in}{2.163394in}}%
\pgfpathclose%
\pgfusepath{fill}%
\end{pgfscope}%
\begin{pgfscope}%
\pgfpathrectangle{\pgfqpoint{1.150000in}{0.150000in}}{\pgfqpoint{5.700000in}{5.700000in}}%
\pgfusepath{clip}%
\pgfsetbuttcap%
\pgfsetroundjoin%
\definecolor{currentfill}{rgb}{0.273809,0.031497,0.358853}%
\pgfsetfillcolor{currentfill}%
\pgfsetfillopacity{0.700000}%
\pgfsetlinewidth{0.000000pt}%
\definecolor{currentstroke}{rgb}{0.000000,0.000000,0.000000}%
\pgfsetstrokecolor{currentstroke}%
\pgfsetdash{}{0pt}%
\pgfpathmoveto{\pgfqpoint{4.496498in}{2.004140in}}%
\pgfpathlineto{\pgfqpoint{4.510351in}{2.001465in}}%
\pgfpathlineto{\pgfqpoint{4.524212in}{1.998816in}}%
\pgfpathlineto{\pgfqpoint{4.538079in}{1.996192in}}%
\pgfpathlineto{\pgfqpoint{4.551954in}{1.993593in}}%
\pgfpathlineto{\pgfqpoint{4.544175in}{1.984397in}}%
\pgfpathlineto{\pgfqpoint{4.536390in}{1.975208in}}%
\pgfpathlineto{\pgfqpoint{4.528600in}{1.966031in}}%
\pgfpathlineto{\pgfqpoint{4.520805in}{1.956868in}}%
\pgfpathlineto{\pgfqpoint{4.506920in}{1.959616in}}%
\pgfpathlineto{\pgfqpoint{4.493041in}{1.962389in}}%
\pgfpathlineto{\pgfqpoint{4.479170in}{1.965188in}}%
\pgfpathlineto{\pgfqpoint{4.465307in}{1.968013in}}%
\pgfpathlineto{\pgfqpoint{4.473112in}{1.977021in}}%
\pgfpathlineto{\pgfqpoint{4.480913in}{1.986048in}}%
\pgfpathlineto{\pgfqpoint{4.488708in}{1.995088in}}%
\pgfpathlineto{\pgfqpoint{4.496498in}{2.004140in}}%
\pgfpathclose%
\pgfusepath{fill}%
\end{pgfscope}%
\begin{pgfscope}%
\pgfpathrectangle{\pgfqpoint{1.150000in}{0.150000in}}{\pgfqpoint{5.700000in}{5.700000in}}%
\pgfusepath{clip}%
\pgfsetbuttcap%
\pgfsetroundjoin%
\definecolor{currentfill}{rgb}{0.269308,0.218818,0.509577}%
\pgfsetfillcolor{currentfill}%
\pgfsetfillopacity{0.700000}%
\pgfsetlinewidth{0.000000pt}%
\definecolor{currentstroke}{rgb}{0.000000,0.000000,0.000000}%
\pgfsetstrokecolor{currentstroke}%
\pgfsetdash{}{0pt}%
\pgfpathmoveto{\pgfqpoint{5.843857in}{2.359129in}}%
\pgfpathlineto{\pgfqpoint{5.858130in}{2.358861in}}%
\pgfpathlineto{\pgfqpoint{5.872412in}{2.358617in}}%
\pgfpathlineto{\pgfqpoint{5.886704in}{2.358397in}}%
\pgfpathlineto{\pgfqpoint{5.879479in}{2.351838in}}%
\pgfpathlineto{\pgfqpoint{5.872244in}{2.345176in}}%
\pgfpathlineto{\pgfqpoint{5.865001in}{2.338409in}}%
\pgfpathlineto{\pgfqpoint{5.857748in}{2.331537in}}%
\pgfpathlineto{\pgfqpoint{5.843442in}{2.331729in}}%
\pgfpathlineto{\pgfqpoint{5.829144in}{2.331945in}}%
\pgfpathlineto{\pgfqpoint{5.814856in}{2.332186in}}%
\pgfpathlineto{\pgfqpoint{5.822120in}{2.339075in}}%
\pgfpathlineto{\pgfqpoint{5.829375in}{2.345861in}}%
\pgfpathlineto{\pgfqpoint{5.836620in}{2.352546in}}%
\pgfpathlineto{\pgfqpoint{5.843857in}{2.359129in}}%
\pgfpathclose%
\pgfusepath{fill}%
\end{pgfscope}%
\begin{pgfscope}%
\pgfpathrectangle{\pgfqpoint{1.150000in}{0.150000in}}{\pgfqpoint{5.700000in}{5.700000in}}%
\pgfusepath{clip}%
\pgfsetbuttcap%
\pgfsetroundjoin%
\definecolor{currentfill}{rgb}{0.267004,0.004874,0.329415}%
\pgfsetfillcolor{currentfill}%
\pgfsetfillopacity{0.700000}%
\pgfsetlinewidth{0.000000pt}%
\definecolor{currentstroke}{rgb}{0.000000,0.000000,0.000000}%
\pgfsetstrokecolor{currentstroke}%
\pgfsetdash{}{0pt}%
\pgfpathmoveto{\pgfqpoint{4.039683in}{1.950944in}}%
\pgfpathlineto{\pgfqpoint{4.053423in}{1.946966in}}%
\pgfpathlineto{\pgfqpoint{4.067169in}{1.943015in}}%
\pgfpathlineto{\pgfqpoint{4.080921in}{1.939091in}}%
\pgfpathlineto{\pgfqpoint{4.094680in}{1.935193in}}%
\pgfpathlineto{\pgfqpoint{4.086741in}{1.927771in}}%
\pgfpathlineto{\pgfqpoint{4.078795in}{1.920443in}}%
\pgfpathlineto{\pgfqpoint{4.070844in}{1.913214in}}%
\pgfpathlineto{\pgfqpoint{4.062886in}{1.906089in}}%
\pgfpathlineto{\pgfqpoint{4.049113in}{1.910189in}}%
\pgfpathlineto{\pgfqpoint{4.035346in}{1.914316in}}%
\pgfpathlineto{\pgfqpoint{4.021585in}{1.918469in}}%
\pgfpathlineto{\pgfqpoint{4.007831in}{1.922649in}}%
\pgfpathlineto{\pgfqpoint{4.015803in}{1.929566in}}%
\pgfpathlineto{\pgfqpoint{4.023770in}{1.936591in}}%
\pgfpathlineto{\pgfqpoint{4.031729in}{1.943719in}}%
\pgfpathlineto{\pgfqpoint{4.039683in}{1.950944in}}%
\pgfpathclose%
\pgfusepath{fill}%
\end{pgfscope}%
\begin{pgfscope}%
\pgfpathrectangle{\pgfqpoint{1.150000in}{0.150000in}}{\pgfqpoint{5.700000in}{5.700000in}}%
\pgfusepath{clip}%
\pgfsetbuttcap%
\pgfsetroundjoin%
\definecolor{currentfill}{rgb}{0.278012,0.180367,0.486697}%
\pgfsetfillcolor{currentfill}%
\pgfsetfillopacity{0.700000}%
\pgfsetlinewidth{0.000000pt}%
\definecolor{currentstroke}{rgb}{0.000000,0.000000,0.000000}%
\pgfsetstrokecolor{currentstroke}%
\pgfsetdash{}{0pt}%
\pgfpathmoveto{\pgfqpoint{5.528626in}{2.280221in}}%
\pgfpathlineto{\pgfqpoint{5.542795in}{2.279599in}}%
\pgfpathlineto{\pgfqpoint{5.556974in}{2.279003in}}%
\pgfpathlineto{\pgfqpoint{5.571161in}{2.278431in}}%
\pgfpathlineto{\pgfqpoint{5.585357in}{2.277883in}}%
\pgfpathlineto{\pgfqpoint{5.577975in}{2.270051in}}%
\pgfpathlineto{\pgfqpoint{5.570585in}{2.262118in}}%
\pgfpathlineto{\pgfqpoint{5.563186in}{2.254085in}}%
\pgfpathlineto{\pgfqpoint{5.555779in}{2.245952in}}%
\pgfpathlineto{\pgfqpoint{5.541570in}{2.246513in}}%
\pgfpathlineto{\pgfqpoint{5.527371in}{2.247099in}}%
\pgfpathlineto{\pgfqpoint{5.513180in}{2.247710in}}%
\pgfpathlineto{\pgfqpoint{5.498998in}{2.248345in}}%
\pgfpathlineto{\pgfqpoint{5.506417in}{2.256459in}}%
\pgfpathlineto{\pgfqpoint{5.513828in}{2.264476in}}%
\pgfpathlineto{\pgfqpoint{5.521231in}{2.272397in}}%
\pgfpathlineto{\pgfqpoint{5.528626in}{2.280221in}}%
\pgfpathclose%
\pgfusepath{fill}%
\end{pgfscope}%
\begin{pgfscope}%
\pgfpathrectangle{\pgfqpoint{1.150000in}{0.150000in}}{\pgfqpoint{5.700000in}{5.700000in}}%
\pgfusepath{clip}%
\pgfsetbuttcap%
\pgfsetroundjoin%
\definecolor{currentfill}{rgb}{0.283229,0.120777,0.440584}%
\pgfsetfillcolor{currentfill}%
\pgfsetfillopacity{0.700000}%
\pgfsetlinewidth{0.000000pt}%
\definecolor{currentstroke}{rgb}{0.000000,0.000000,0.000000}%
\pgfsetstrokecolor{currentstroke}%
\pgfsetdash{}{0pt}%
\pgfpathmoveto{\pgfqpoint{5.126897in}{2.163886in}}%
\pgfpathlineto{\pgfqpoint{5.140940in}{2.162631in}}%
\pgfpathlineto{\pgfqpoint{5.154990in}{2.161401in}}%
\pgfpathlineto{\pgfqpoint{5.169050in}{2.160195in}}%
\pgfpathlineto{\pgfqpoint{5.183117in}{2.159015in}}%
\pgfpathlineto{\pgfqpoint{5.175558in}{2.149927in}}%
\pgfpathlineto{\pgfqpoint{5.167993in}{2.140763in}}%
\pgfpathlineto{\pgfqpoint{5.160420in}{2.131526in}}%
\pgfpathlineto{\pgfqpoint{5.152841in}{2.122216in}}%
\pgfpathlineto{\pgfqpoint{5.138764in}{2.123466in}}%
\pgfpathlineto{\pgfqpoint{5.124695in}{2.124740in}}%
\pgfpathlineto{\pgfqpoint{5.110634in}{2.126039in}}%
\pgfpathlineto{\pgfqpoint{5.096581in}{2.127364in}}%
\pgfpathlineto{\pgfqpoint{5.104170in}{2.136599in}}%
\pgfpathlineto{\pgfqpoint{5.111752in}{2.145766in}}%
\pgfpathlineto{\pgfqpoint{5.119328in}{2.154862in}}%
\pgfpathlineto{\pgfqpoint{5.126897in}{2.163886in}}%
\pgfpathclose%
\pgfusepath{fill}%
\end{pgfscope}%
\begin{pgfscope}%
\pgfpathrectangle{\pgfqpoint{1.150000in}{0.150000in}}{\pgfqpoint{5.700000in}{5.700000in}}%
\pgfusepath{clip}%
\pgfsetbuttcap%
\pgfsetroundjoin%
\definecolor{currentfill}{rgb}{0.277018,0.050344,0.375715}%
\pgfsetfillcolor{currentfill}%
\pgfsetfillopacity{0.700000}%
\pgfsetlinewidth{0.000000pt}%
\definecolor{currentstroke}{rgb}{0.000000,0.000000,0.000000}%
\pgfsetstrokecolor{currentstroke}%
\pgfsetdash{}{0pt}%
\pgfpathmoveto{\pgfqpoint{3.472843in}{2.020376in}}%
\pgfpathlineto{\pgfqpoint{3.486474in}{2.014603in}}%
\pgfpathlineto{\pgfqpoint{3.500111in}{2.008860in}}%
\pgfpathlineto{\pgfqpoint{3.513753in}{2.003148in}}%
\pgfpathlineto{\pgfqpoint{3.527399in}{1.997465in}}%
\pgfpathlineto{\pgfqpoint{3.519198in}{1.994204in}}%
\pgfpathlineto{\pgfqpoint{3.510987in}{1.991155in}}%
\pgfpathlineto{\pgfqpoint{3.502766in}{1.988325in}}%
\pgfpathlineto{\pgfqpoint{3.494535in}{1.985720in}}%
\pgfpathlineto{\pgfqpoint{3.480866in}{1.991659in}}%
\pgfpathlineto{\pgfqpoint{3.467202in}{1.997627in}}%
\pgfpathlineto{\pgfqpoint{3.453544in}{2.003626in}}%
\pgfpathlineto{\pgfqpoint{3.439889in}{2.009655in}}%
\pgfpathlineto{\pgfqpoint{3.448144in}{2.011998in}}%
\pgfpathlineto{\pgfqpoint{3.456387in}{2.014571in}}%
\pgfpathlineto{\pgfqpoint{3.464620in}{2.017365in}}%
\pgfpathlineto{\pgfqpoint{3.472843in}{2.020376in}}%
\pgfpathclose%
\pgfusepath{fill}%
\end{pgfscope}%
\begin{pgfscope}%
\pgfpathrectangle{\pgfqpoint{1.150000in}{0.150000in}}{\pgfqpoint{5.700000in}{5.700000in}}%
\pgfusepath{clip}%
\pgfsetbuttcap%
\pgfsetroundjoin%
\definecolor{currentfill}{rgb}{0.278791,0.062145,0.386592}%
\pgfsetfillcolor{currentfill}%
\pgfsetfillopacity{0.700000}%
\pgfsetlinewidth{0.000000pt}%
\definecolor{currentstroke}{rgb}{0.000000,0.000000,0.000000}%
\pgfsetstrokecolor{currentstroke}%
\pgfsetdash{}{0pt}%
\pgfpathmoveto{\pgfqpoint{4.725097in}{2.049246in}}%
\pgfpathlineto{\pgfqpoint{4.739019in}{2.047134in}}%
\pgfpathlineto{\pgfqpoint{4.752949in}{2.045047in}}%
\pgfpathlineto{\pgfqpoint{4.766886in}{2.042985in}}%
\pgfpathlineto{\pgfqpoint{4.780831in}{2.040949in}}%
\pgfpathlineto{\pgfqpoint{4.773126in}{2.031452in}}%
\pgfpathlineto{\pgfqpoint{4.765414in}{2.021929in}}%
\pgfpathlineto{\pgfqpoint{4.757697in}{2.012381in}}%
\pgfpathlineto{\pgfqpoint{4.749975in}{2.002812in}}%
\pgfpathlineto{\pgfqpoint{4.736020in}{2.004972in}}%
\pgfpathlineto{\pgfqpoint{4.722073in}{2.007156in}}%
\pgfpathlineto{\pgfqpoint{4.708133in}{2.009366in}}%
\pgfpathlineto{\pgfqpoint{4.694202in}{2.011601in}}%
\pgfpathlineto{\pgfqpoint{4.701934in}{2.021042in}}%
\pgfpathlineto{\pgfqpoint{4.709660in}{2.030465in}}%
\pgfpathlineto{\pgfqpoint{4.717382in}{2.039867in}}%
\pgfpathlineto{\pgfqpoint{4.725097in}{2.049246in}}%
\pgfpathclose%
\pgfusepath{fill}%
\end{pgfscope}%
\begin{pgfscope}%
\pgfpathrectangle{\pgfqpoint{1.150000in}{0.150000in}}{\pgfqpoint{5.700000in}{5.700000in}}%
\pgfusepath{clip}%
\pgfsetbuttcap%
\pgfsetroundjoin%
\definecolor{currentfill}{rgb}{0.278826,0.175490,0.483397}%
\pgfsetfillcolor{currentfill}%
\pgfsetfillopacity{0.700000}%
\pgfsetlinewidth{0.000000pt}%
\definecolor{currentstroke}{rgb}{0.000000,0.000000,0.000000}%
\pgfsetstrokecolor{currentstroke}%
\pgfsetdash{}{0pt}%
\pgfpathmoveto{\pgfqpoint{2.882839in}{2.258666in}}%
\pgfpathlineto{\pgfqpoint{2.896402in}{2.250865in}}%
\pgfpathlineto{\pgfqpoint{2.909969in}{2.243102in}}%
\pgfpathlineto{\pgfqpoint{2.923539in}{2.235376in}}%
\pgfpathlineto{\pgfqpoint{2.937112in}{2.227689in}}%
\pgfpathlineto{\pgfqpoint{2.928518in}{2.230075in}}%
\pgfpathlineto{\pgfqpoint{2.919907in}{2.232792in}}%
\pgfpathlineto{\pgfqpoint{2.911278in}{2.235848in}}%
\pgfpathlineto{\pgfqpoint{2.902632in}{2.239251in}}%
\pgfpathlineto{\pgfqpoint{2.889026in}{2.247239in}}%
\pgfpathlineto{\pgfqpoint{2.875423in}{2.255265in}}%
\pgfpathlineto{\pgfqpoint{2.861824in}{2.263328in}}%
\pgfpathlineto{\pgfqpoint{2.848227in}{2.271430in}}%
\pgfpathlineto{\pgfqpoint{2.856907in}{2.267721in}}%
\pgfpathlineto{\pgfqpoint{2.865569in}{2.264362in}}%
\pgfpathlineto{\pgfqpoint{2.874212in}{2.261347in}}%
\pgfpathlineto{\pgfqpoint{2.882839in}{2.258666in}}%
\pgfpathclose%
\pgfusepath{fill}%
\end{pgfscope}%
\begin{pgfscope}%
\pgfpathrectangle{\pgfqpoint{1.150000in}{0.150000in}}{\pgfqpoint{5.700000in}{5.700000in}}%
\pgfusepath{clip}%
\pgfsetbuttcap%
\pgfsetroundjoin%
\definecolor{currentfill}{rgb}{0.271305,0.019942,0.347269}%
\pgfsetfillcolor{currentfill}%
\pgfsetfillopacity{0.700000}%
\pgfsetlinewidth{0.000000pt}%
\definecolor{currentstroke}{rgb}{0.000000,0.000000,0.000000}%
\pgfsetstrokecolor{currentstroke}%
\pgfsetdash{}{0pt}%
\pgfpathmoveto{\pgfqpoint{4.409922in}{1.979566in}}%
\pgfpathlineto{\pgfqpoint{4.423758in}{1.976639in}}%
\pgfpathlineto{\pgfqpoint{4.437600in}{1.973738in}}%
\pgfpathlineto{\pgfqpoint{4.451450in}{1.970863in}}%
\pgfpathlineto{\pgfqpoint{4.465307in}{1.968013in}}%
\pgfpathlineto{\pgfqpoint{4.457495in}{1.959026in}}%
\pgfpathlineto{\pgfqpoint{4.449679in}{1.950064in}}%
\pgfpathlineto{\pgfqpoint{4.441857in}{1.941132in}}%
\pgfpathlineto{\pgfqpoint{4.434030in}{1.932232in}}%
\pgfpathlineto{\pgfqpoint{4.420162in}{1.935245in}}%
\pgfpathlineto{\pgfqpoint{4.406302in}{1.938283in}}%
\pgfpathlineto{\pgfqpoint{4.392448in}{1.941347in}}%
\pgfpathlineto{\pgfqpoint{4.378601in}{1.944437in}}%
\pgfpathlineto{\pgfqpoint{4.386440in}{1.953168in}}%
\pgfpathlineto{\pgfqpoint{4.394273in}{1.961936in}}%
\pgfpathlineto{\pgfqpoint{4.402100in}{1.970737in}}%
\pgfpathlineto{\pgfqpoint{4.409922in}{1.979566in}}%
\pgfpathclose%
\pgfusepath{fill}%
\end{pgfscope}%
\begin{pgfscope}%
\pgfpathrectangle{\pgfqpoint{1.150000in}{0.150000in}}{\pgfqpoint{5.700000in}{5.700000in}}%
\pgfusepath{clip}%
\pgfsetbuttcap%
\pgfsetroundjoin%
\definecolor{currentfill}{rgb}{0.268510,0.009605,0.335427}%
\pgfsetfillcolor{currentfill}%
\pgfsetfillopacity{0.700000}%
\pgfsetlinewidth{0.000000pt}%
\definecolor{currentstroke}{rgb}{0.000000,0.000000,0.000000}%
\pgfsetstrokecolor{currentstroke}%
\pgfsetdash{}{0pt}%
\pgfpathmoveto{\pgfqpoint{4.181423in}{1.951157in}}%
\pgfpathlineto{\pgfqpoint{4.195201in}{1.947580in}}%
\pgfpathlineto{\pgfqpoint{4.208985in}{1.944030in}}%
\pgfpathlineto{\pgfqpoint{4.222776in}{1.940505in}}%
\pgfpathlineto{\pgfqpoint{4.236573in}{1.937007in}}%
\pgfpathlineto{\pgfqpoint{4.228684in}{1.928888in}}%
\pgfpathlineto{\pgfqpoint{4.220788in}{1.920838in}}%
\pgfpathlineto{\pgfqpoint{4.212887in}{1.912861in}}%
\pgfpathlineto{\pgfqpoint{4.204981in}{1.904961in}}%
\pgfpathlineto{\pgfqpoint{4.191170in}{1.908649in}}%
\pgfpathlineto{\pgfqpoint{4.177367in}{1.912362in}}%
\pgfpathlineto{\pgfqpoint{4.163570in}{1.916101in}}%
\pgfpathlineto{\pgfqpoint{4.149779in}{1.919867in}}%
\pgfpathlineto{\pgfqpoint{4.157699in}{1.927572in}}%
\pgfpathlineto{\pgfqpoint{4.165613in}{1.935359in}}%
\pgfpathlineto{\pgfqpoint{4.173521in}{1.943222in}}%
\pgfpathlineto{\pgfqpoint{4.181423in}{1.951157in}}%
\pgfpathclose%
\pgfusepath{fill}%
\end{pgfscope}%
\begin{pgfscope}%
\pgfpathrectangle{\pgfqpoint{1.150000in}{0.150000in}}{\pgfqpoint{5.700000in}{5.700000in}}%
\pgfusepath{clip}%
\pgfsetbuttcap%
\pgfsetroundjoin%
\definecolor{currentfill}{rgb}{0.280255,0.165693,0.476498}%
\pgfsetfillcolor{currentfill}%
\pgfsetfillopacity{0.700000}%
\pgfsetlinewidth{0.000000pt}%
\definecolor{currentstroke}{rgb}{0.000000,0.000000,0.000000}%
\pgfsetstrokecolor{currentstroke}%
\pgfsetdash{}{0pt}%
\pgfpathmoveto{\pgfqpoint{5.442359in}{2.251133in}}%
\pgfpathlineto{\pgfqpoint{5.456506in}{2.250399in}}%
\pgfpathlineto{\pgfqpoint{5.470661in}{2.249690in}}%
\pgfpathlineto{\pgfqpoint{5.484825in}{2.249005in}}%
\pgfpathlineto{\pgfqpoint{5.498998in}{2.248345in}}%
\pgfpathlineto{\pgfqpoint{5.491571in}{2.240134in}}%
\pgfpathlineto{\pgfqpoint{5.484136in}{2.231826in}}%
\pgfpathlineto{\pgfqpoint{5.476694in}{2.223421in}}%
\pgfpathlineto{\pgfqpoint{5.469243in}{2.214919in}}%
\pgfpathlineto{\pgfqpoint{5.455059in}{2.215607in}}%
\pgfpathlineto{\pgfqpoint{5.440883in}{2.216319in}}%
\pgfpathlineto{\pgfqpoint{5.426716in}{2.217057in}}%
\pgfpathlineto{\pgfqpoint{5.412558in}{2.217818in}}%
\pgfpathlineto{\pgfqpoint{5.420020in}{2.226287in}}%
\pgfpathlineto{\pgfqpoint{5.427474in}{2.234663in}}%
\pgfpathlineto{\pgfqpoint{5.434921in}{2.242945in}}%
\pgfpathlineto{\pgfqpoint{5.442359in}{2.251133in}}%
\pgfpathclose%
\pgfusepath{fill}%
\end{pgfscope}%
\begin{pgfscope}%
\pgfpathrectangle{\pgfqpoint{1.150000in}{0.150000in}}{\pgfqpoint{5.700000in}{5.700000in}}%
\pgfusepath{clip}%
\pgfsetbuttcap%
\pgfsetroundjoin%
\definecolor{currentfill}{rgb}{0.260571,0.246922,0.522828}%
\pgfsetfillcolor{currentfill}%
\pgfsetfillopacity{0.700000}%
\pgfsetlinewidth{0.000000pt}%
\definecolor{currentstroke}{rgb}{0.000000,0.000000,0.000000}%
\pgfsetstrokecolor{currentstroke}%
\pgfsetdash{}{0pt}%
\pgfpathmoveto{\pgfqpoint{2.631073in}{2.406508in}}%
\pgfpathlineto{\pgfqpoint{2.644625in}{2.397751in}}%
\pgfpathlineto{\pgfqpoint{2.658180in}{2.389038in}}%
\pgfpathlineto{\pgfqpoint{2.671737in}{2.380368in}}%
\pgfpathlineto{\pgfqpoint{2.685297in}{2.371741in}}%
\pgfpathlineto{\pgfqpoint{2.676492in}{2.376741in}}%
\pgfpathlineto{\pgfqpoint{2.667666in}{2.382120in}}%
\pgfpathlineto{\pgfqpoint{2.658819in}{2.387887in}}%
\pgfpathlineto{\pgfqpoint{2.649950in}{2.394050in}}%
\pgfpathlineto{\pgfqpoint{2.636353in}{2.402995in}}%
\pgfpathlineto{\pgfqpoint{2.622758in}{2.411982in}}%
\pgfpathlineto{\pgfqpoint{2.609166in}{2.421013in}}%
\pgfpathlineto{\pgfqpoint{2.595575in}{2.430087in}}%
\pgfpathlineto{\pgfqpoint{2.604483in}{2.423601in}}%
\pgfpathlineto{\pgfqpoint{2.613368in}{2.417515in}}%
\pgfpathlineto{\pgfqpoint{2.622231in}{2.411820in}}%
\pgfpathlineto{\pgfqpoint{2.631073in}{2.406508in}}%
\pgfpathclose%
\pgfusepath{fill}%
\end{pgfscope}%
\begin{pgfscope}%
\pgfpathrectangle{\pgfqpoint{1.150000in}{0.150000in}}{\pgfqpoint{5.700000in}{5.700000in}}%
\pgfusepath{clip}%
\pgfsetbuttcap%
\pgfsetroundjoin%
\definecolor{currentfill}{rgb}{0.283091,0.110553,0.431554}%
\pgfsetfillcolor{currentfill}%
\pgfsetfillopacity{0.700000}%
\pgfsetlinewidth{0.000000pt}%
\definecolor{currentstroke}{rgb}{0.000000,0.000000,0.000000}%
\pgfsetstrokecolor{currentstroke}%
\pgfsetdash{}{0pt}%
\pgfpathmoveto{\pgfqpoint{5.040454in}{2.132908in}}%
\pgfpathlineto{\pgfqpoint{5.054473in}{2.131485in}}%
\pgfpathlineto{\pgfqpoint{5.068501in}{2.130086in}}%
\pgfpathlineto{\pgfqpoint{5.082537in}{2.128713in}}%
\pgfpathlineto{\pgfqpoint{5.096581in}{2.127364in}}%
\pgfpathlineto{\pgfqpoint{5.088986in}{2.118060in}}%
\pgfpathlineto{\pgfqpoint{5.081384in}{2.108691in}}%
\pgfpathlineto{\pgfqpoint{5.073776in}{2.099256in}}%
\pgfpathlineto{\pgfqpoint{5.066161in}{2.089759in}}%
\pgfpathlineto{\pgfqpoint{5.052108in}{2.091191in}}%
\pgfpathlineto{\pgfqpoint{5.038062in}{2.092648in}}%
\pgfpathlineto{\pgfqpoint{5.024024in}{2.094129in}}%
\pgfpathlineto{\pgfqpoint{5.009995in}{2.095635in}}%
\pgfpathlineto{\pgfqpoint{5.017619in}{2.105044in}}%
\pgfpathlineto{\pgfqpoint{5.025237in}{2.114394in}}%
\pgfpathlineto{\pgfqpoint{5.032849in}{2.123683in}}%
\pgfpathlineto{\pgfqpoint{5.040454in}{2.132908in}}%
\pgfpathclose%
\pgfusepath{fill}%
\end{pgfscope}%
\begin{pgfscope}%
\pgfpathrectangle{\pgfqpoint{1.150000in}{0.150000in}}{\pgfqpoint{5.700000in}{5.700000in}}%
\pgfusepath{clip}%
\pgfsetbuttcap%
\pgfsetroundjoin%
\definecolor{currentfill}{rgb}{0.271828,0.209303,0.504434}%
\pgfsetfillcolor{currentfill}%
\pgfsetfillopacity{0.700000}%
\pgfsetlinewidth{0.000000pt}%
\definecolor{currentstroke}{rgb}{0.000000,0.000000,0.000000}%
\pgfsetstrokecolor{currentstroke}%
\pgfsetdash{}{0pt}%
\pgfpathmoveto{\pgfqpoint{5.757797in}{2.333396in}}%
\pgfpathlineto{\pgfqpoint{5.772048in}{2.333056in}}%
\pgfpathlineto{\pgfqpoint{5.786308in}{2.332742in}}%
\pgfpathlineto{\pgfqpoint{5.800577in}{2.332451in}}%
\pgfpathlineto{\pgfqpoint{5.814856in}{2.332186in}}%
\pgfpathlineto{\pgfqpoint{5.807583in}{2.325192in}}%
\pgfpathlineto{\pgfqpoint{5.800302in}{2.318093in}}%
\pgfpathlineto{\pgfqpoint{5.793011in}{2.310888in}}%
\pgfpathlineto{\pgfqpoint{5.785712in}{2.303575in}}%
\pgfpathlineto{\pgfqpoint{5.771419in}{2.303827in}}%
\pgfpathlineto{\pgfqpoint{5.757135in}{2.304103in}}%
\pgfpathlineto{\pgfqpoint{5.742861in}{2.304404in}}%
\pgfpathlineto{\pgfqpoint{5.728595in}{2.304729in}}%
\pgfpathlineto{\pgfqpoint{5.735909in}{2.312051in}}%
\pgfpathlineto{\pgfqpoint{5.743214in}{2.319268in}}%
\pgfpathlineto{\pgfqpoint{5.750509in}{2.326383in}}%
\pgfpathlineto{\pgfqpoint{5.757797in}{2.333396in}}%
\pgfpathclose%
\pgfusepath{fill}%
\end{pgfscope}%
\begin{pgfscope}%
\pgfpathrectangle{\pgfqpoint{1.150000in}{0.150000in}}{\pgfqpoint{5.700000in}{5.700000in}}%
\pgfusepath{clip}%
\pgfsetbuttcap%
\pgfsetroundjoin%
\definecolor{currentfill}{rgb}{0.277018,0.050344,0.375715}%
\pgfsetfillcolor{currentfill}%
\pgfsetfillopacity{0.700000}%
\pgfsetlinewidth{0.000000pt}%
\definecolor{currentstroke}{rgb}{0.000000,0.000000,0.000000}%
\pgfsetstrokecolor{currentstroke}%
\pgfsetdash{}{0pt}%
\pgfpathmoveto{\pgfqpoint{4.638550in}{2.020792in}}%
\pgfpathlineto{\pgfqpoint{4.652451in}{2.018456in}}%
\pgfpathlineto{\pgfqpoint{4.666361in}{2.016146in}}%
\pgfpathlineto{\pgfqpoint{4.680277in}{2.013861in}}%
\pgfpathlineto{\pgfqpoint{4.694202in}{2.011601in}}%
\pgfpathlineto{\pgfqpoint{4.686464in}{2.002145in}}%
\pgfpathlineto{\pgfqpoint{4.678721in}{1.992676in}}%
\pgfpathlineto{\pgfqpoint{4.670972in}{1.983199in}}%
\pgfpathlineto{\pgfqpoint{4.663218in}{1.973716in}}%
\pgfpathlineto{\pgfqpoint{4.649284in}{1.976113in}}%
\pgfpathlineto{\pgfqpoint{4.635358in}{1.978534in}}%
\pgfpathlineto{\pgfqpoint{4.621439in}{1.980981in}}%
\pgfpathlineto{\pgfqpoint{4.607527in}{1.983453in}}%
\pgfpathlineto{\pgfqpoint{4.615291in}{1.992794in}}%
\pgfpathlineto{\pgfqpoint{4.623049in}{2.002134in}}%
\pgfpathlineto{\pgfqpoint{4.630802in}{2.011467in}}%
\pgfpathlineto{\pgfqpoint{4.638550in}{2.020792in}}%
\pgfpathclose%
\pgfusepath{fill}%
\end{pgfscope}%
\begin{pgfscope}%
\pgfpathrectangle{\pgfqpoint{1.150000in}{0.150000in}}{\pgfqpoint{5.700000in}{5.700000in}}%
\pgfusepath{clip}%
\pgfsetbuttcap%
\pgfsetroundjoin%
\definecolor{currentfill}{rgb}{0.283197,0.115680,0.436115}%
\pgfsetfillcolor{currentfill}%
\pgfsetfillopacity{0.700000}%
\pgfsetlinewidth{0.000000pt}%
\definecolor{currentstroke}{rgb}{0.000000,0.000000,0.000000}%
\pgfsetstrokecolor{currentstroke}%
\pgfsetdash{}{0pt}%
\pgfpathmoveto{\pgfqpoint{3.134128in}{2.135291in}}%
\pgfpathlineto{\pgfqpoint{3.147719in}{2.128352in}}%
\pgfpathlineto{\pgfqpoint{3.161315in}{2.121446in}}%
\pgfpathlineto{\pgfqpoint{3.174915in}{2.114574in}}%
\pgfpathlineto{\pgfqpoint{3.188519in}{2.107735in}}%
\pgfpathlineto{\pgfqpoint{3.180103in}{2.107757in}}%
\pgfpathlineto{\pgfqpoint{3.171674in}{2.108064in}}%
\pgfpathlineto{\pgfqpoint{3.163231in}{2.108664in}}%
\pgfpathlineto{\pgfqpoint{3.154773in}{2.109565in}}%
\pgfpathlineto{\pgfqpoint{3.141141in}{2.116689in}}%
\pgfpathlineto{\pgfqpoint{3.127513in}{2.123845in}}%
\pgfpathlineto{\pgfqpoint{3.113888in}{2.131036in}}%
\pgfpathlineto{\pgfqpoint{3.100268in}{2.138261in}}%
\pgfpathlineto{\pgfqpoint{3.108755in}{2.137069in}}%
\pgfpathlineto{\pgfqpoint{3.117227in}{2.136183in}}%
\pgfpathlineto{\pgfqpoint{3.125684in}{2.135592in}}%
\pgfpathlineto{\pgfqpoint{3.134128in}{2.135291in}}%
\pgfpathclose%
\pgfusepath{fill}%
\end{pgfscope}%
\begin{pgfscope}%
\pgfpathrectangle{\pgfqpoint{1.150000in}{0.150000in}}{\pgfqpoint{5.700000in}{5.700000in}}%
\pgfusepath{clip}%
\pgfsetbuttcap%
\pgfsetroundjoin%
\definecolor{currentfill}{rgb}{0.220057,0.343307,0.549413}%
\pgfsetfillcolor{currentfill}%
\pgfsetfillopacity{0.700000}%
\pgfsetlinewidth{0.000000pt}%
\definecolor{currentstroke}{rgb}{0.000000,0.000000,0.000000}%
\pgfsetstrokecolor{currentstroke}%
\pgfsetdash{}{0pt}%
\pgfpathmoveto{\pgfqpoint{2.324164in}{2.621461in}}%
\pgfpathlineto{\pgfqpoint{2.337720in}{2.611415in}}%
\pgfpathlineto{\pgfqpoint{2.351277in}{2.601422in}}%
\pgfpathlineto{\pgfqpoint{2.364836in}{2.591482in}}%
\pgfpathlineto{\pgfqpoint{2.378396in}{2.581594in}}%
\pgfpathlineto{\pgfqpoint{2.369301in}{2.589805in}}%
\pgfpathlineto{\pgfqpoint{2.360181in}{2.598450in}}%
\pgfpathlineto{\pgfqpoint{2.351034in}{2.607539in}}%
\pgfpathlineto{\pgfqpoint{2.341860in}{2.617080in}}%
\pgfpathlineto{\pgfqpoint{2.328256in}{2.627304in}}%
\pgfpathlineto{\pgfqpoint{2.314654in}{2.637581in}}%
\pgfpathlineto{\pgfqpoint{2.301052in}{2.647911in}}%
\pgfpathlineto{\pgfqpoint{2.287452in}{2.658295in}}%
\pgfpathlineto{\pgfqpoint{2.296671in}{2.648411in}}%
\pgfpathlineto{\pgfqpoint{2.305862in}{2.638983in}}%
\pgfpathlineto{\pgfqpoint{2.315026in}{2.630003in}}%
\pgfpathlineto{\pgfqpoint{2.324164in}{2.621461in}}%
\pgfpathclose%
\pgfusepath{fill}%
\end{pgfscope}%
\begin{pgfscope}%
\pgfpathrectangle{\pgfqpoint{1.150000in}{0.150000in}}{\pgfqpoint{5.700000in}{5.700000in}}%
\pgfusepath{clip}%
\pgfsetbuttcap%
\pgfsetroundjoin%
\definecolor{currentfill}{rgb}{0.280267,0.073417,0.397163}%
\pgfsetfillcolor{currentfill}%
\pgfsetfillopacity{0.700000}%
\pgfsetlinewidth{0.000000pt}%
\definecolor{currentstroke}{rgb}{0.000000,0.000000,0.000000}%
\pgfsetstrokecolor{currentstroke}%
\pgfsetdash{}{0pt}%
\pgfpathmoveto{\pgfqpoint{3.330825in}{2.058992in}}%
\pgfpathlineto{\pgfqpoint{3.344442in}{2.052716in}}%
\pgfpathlineto{\pgfqpoint{3.358064in}{2.046471in}}%
\pgfpathlineto{\pgfqpoint{3.371690in}{2.040258in}}%
\pgfpathlineto{\pgfqpoint{3.385320in}{2.034075in}}%
\pgfpathlineto{\pgfqpoint{3.377032in}{2.032233in}}%
\pgfpathlineto{\pgfqpoint{3.368731in}{2.030636in}}%
\pgfpathlineto{\pgfqpoint{3.360419in}{2.029293in}}%
\pgfpathlineto{\pgfqpoint{3.352095in}{2.028209in}}%
\pgfpathlineto{\pgfqpoint{3.338440in}{2.034662in}}%
\pgfpathlineto{\pgfqpoint{3.324789in}{2.041145in}}%
\pgfpathlineto{\pgfqpoint{3.311143in}{2.047660in}}%
\pgfpathlineto{\pgfqpoint{3.297501in}{2.054207in}}%
\pgfpathlineto{\pgfqpoint{3.305850in}{2.055014in}}%
\pgfpathlineto{\pgfqpoint{3.314187in}{2.056086in}}%
\pgfpathlineto{\pgfqpoint{3.322512in}{2.057414in}}%
\pgfpathlineto{\pgfqpoint{3.330825in}{2.058992in}}%
\pgfpathclose%
\pgfusepath{fill}%
\end{pgfscope}%
\begin{pgfscope}%
\pgfpathrectangle{\pgfqpoint{1.150000in}{0.150000in}}{\pgfqpoint{5.700000in}{5.700000in}}%
\pgfusepath{clip}%
\pgfsetbuttcap%
\pgfsetroundjoin%
\definecolor{currentfill}{rgb}{0.268510,0.009605,0.335427}%
\pgfsetfillcolor{currentfill}%
\pgfsetfillopacity{0.700000}%
\pgfsetlinewidth{0.000000pt}%
\definecolor{currentstroke}{rgb}{0.000000,0.000000,0.000000}%
\pgfsetstrokecolor{currentstroke}%
\pgfsetdash{}{0pt}%
\pgfpathmoveto{\pgfqpoint{3.811102in}{1.951017in}}%
\pgfpathlineto{\pgfqpoint{3.824801in}{1.946294in}}%
\pgfpathlineto{\pgfqpoint{3.838506in}{1.941599in}}%
\pgfpathlineto{\pgfqpoint{3.852216in}{1.936932in}}%
\pgfpathlineto{\pgfqpoint{3.865933in}{1.932292in}}%
\pgfpathlineto{\pgfqpoint{3.857895in}{1.926439in}}%
\pgfpathlineto{\pgfqpoint{3.849849in}{1.920732in}}%
\pgfpathlineto{\pgfqpoint{3.841796in}{1.915177in}}%
\pgfpathlineto{\pgfqpoint{3.833736in}{1.909781in}}%
\pgfpathlineto{\pgfqpoint{3.820002in}{1.914649in}}%
\pgfpathlineto{\pgfqpoint{3.806274in}{1.919545in}}%
\pgfpathlineto{\pgfqpoint{3.792552in}{1.924469in}}%
\pgfpathlineto{\pgfqpoint{3.778835in}{1.929420in}}%
\pgfpathlineto{\pgfqpoint{3.786913in}{1.934583in}}%
\pgfpathlineto{\pgfqpoint{3.794984in}{1.939907in}}%
\pgfpathlineto{\pgfqpoint{3.803047in}{1.945387in}}%
\pgfpathlineto{\pgfqpoint{3.811102in}{1.951017in}}%
\pgfpathclose%
\pgfusepath{fill}%
\end{pgfscope}%
\begin{pgfscope}%
\pgfpathrectangle{\pgfqpoint{1.150000in}{0.150000in}}{\pgfqpoint{5.700000in}{5.700000in}}%
\pgfusepath{clip}%
\pgfsetbuttcap%
\pgfsetroundjoin%
\definecolor{currentfill}{rgb}{0.281412,0.155834,0.469201}%
\pgfsetfillcolor{currentfill}%
\pgfsetfillopacity{0.700000}%
\pgfsetlinewidth{0.000000pt}%
\definecolor{currentstroke}{rgb}{0.000000,0.000000,0.000000}%
\pgfsetstrokecolor{currentstroke}%
\pgfsetdash{}{0pt}%
\pgfpathmoveto{\pgfqpoint{5.356014in}{2.221113in}}%
\pgfpathlineto{\pgfqpoint{5.370137in}{2.220252in}}%
\pgfpathlineto{\pgfqpoint{5.384268in}{2.219416in}}%
\pgfpathlineto{\pgfqpoint{5.398409in}{2.218605in}}%
\pgfpathlineto{\pgfqpoint{5.412558in}{2.217818in}}%
\pgfpathlineto{\pgfqpoint{5.405089in}{2.209257in}}%
\pgfpathlineto{\pgfqpoint{5.397612in}{2.200603in}}%
\pgfpathlineto{\pgfqpoint{5.390127in}{2.191858in}}%
\pgfpathlineto{\pgfqpoint{5.382635in}{2.183021in}}%
\pgfpathlineto{\pgfqpoint{5.368475in}{2.183850in}}%
\pgfpathlineto{\pgfqpoint{5.354323in}{2.184703in}}%
\pgfpathlineto{\pgfqpoint{5.340181in}{2.185580in}}%
\pgfpathlineto{\pgfqpoint{5.326047in}{2.186483in}}%
\pgfpathlineto{\pgfqpoint{5.333549in}{2.195273in}}%
\pgfpathlineto{\pgfqpoint{5.341045in}{2.203975in}}%
\pgfpathlineto{\pgfqpoint{5.348533in}{2.212588in}}%
\pgfpathlineto{\pgfqpoint{5.356014in}{2.221113in}}%
\pgfpathclose%
\pgfusepath{fill}%
\end{pgfscope}%
\begin{pgfscope}%
\pgfpathrectangle{\pgfqpoint{1.150000in}{0.150000in}}{\pgfqpoint{5.700000in}{5.700000in}}%
\pgfusepath{clip}%
\pgfsetbuttcap%
\pgfsetroundjoin%
\definecolor{currentfill}{rgb}{0.282327,0.094955,0.417331}%
\pgfsetfillcolor{currentfill}%
\pgfsetfillopacity{0.700000}%
\pgfsetlinewidth{0.000000pt}%
\definecolor{currentstroke}{rgb}{0.000000,0.000000,0.000000}%
\pgfsetstrokecolor{currentstroke}%
\pgfsetdash{}{0pt}%
\pgfpathmoveto{\pgfqpoint{4.953959in}{2.101908in}}%
\pgfpathlineto{\pgfqpoint{4.967956in}{2.100303in}}%
\pgfpathlineto{\pgfqpoint{4.981961in}{2.098722in}}%
\pgfpathlineto{\pgfqpoint{4.995974in}{2.097166in}}%
\pgfpathlineto{\pgfqpoint{5.009995in}{2.095635in}}%
\pgfpathlineto{\pgfqpoint{5.002365in}{2.086168in}}%
\pgfpathlineto{\pgfqpoint{4.994728in}{2.076646in}}%
\pgfpathlineto{\pgfqpoint{4.987086in}{2.067070in}}%
\pgfpathlineto{\pgfqpoint{4.979437in}{2.057442in}}%
\pgfpathlineto{\pgfqpoint{4.965407in}{2.059069in}}%
\pgfpathlineto{\pgfqpoint{4.951384in}{2.060721in}}%
\pgfpathlineto{\pgfqpoint{4.937370in}{2.062398in}}%
\pgfpathlineto{\pgfqpoint{4.923364in}{2.064100in}}%
\pgfpathlineto{\pgfqpoint{4.931021in}{2.073627in}}%
\pgfpathlineto{\pgfqpoint{4.938673in}{2.083105in}}%
\pgfpathlineto{\pgfqpoint{4.946319in}{2.092533in}}%
\pgfpathlineto{\pgfqpoint{4.953959in}{2.101908in}}%
\pgfpathclose%
\pgfusepath{fill}%
\end{pgfscope}%
\begin{pgfscope}%
\pgfpathrectangle{\pgfqpoint{1.150000in}{0.150000in}}{\pgfqpoint{5.700000in}{5.700000in}}%
\pgfusepath{clip}%
\pgfsetbuttcap%
\pgfsetroundjoin%
\definecolor{currentfill}{rgb}{0.271305,0.019942,0.347269}%
\pgfsetfillcolor{currentfill}%
\pgfsetfillopacity{0.700000}%
\pgfsetlinewidth{0.000000pt}%
\definecolor{currentstroke}{rgb}{0.000000,0.000000,0.000000}%
\pgfsetstrokecolor{currentstroke}%
\pgfsetdash{}{0pt}%
\pgfpathmoveto{\pgfqpoint{3.669300in}{1.970043in}}%
\pgfpathlineto{\pgfqpoint{3.682973in}{1.964866in}}%
\pgfpathlineto{\pgfqpoint{3.696651in}{1.959717in}}%
\pgfpathlineto{\pgfqpoint{3.710335in}{1.954597in}}%
\pgfpathlineto{\pgfqpoint{3.724024in}{1.949505in}}%
\pgfpathlineto{\pgfqpoint{3.715919in}{1.944747in}}%
\pgfpathlineto{\pgfqpoint{3.707806in}{1.940166in}}%
\pgfpathlineto{\pgfqpoint{3.699685in}{1.935767in}}%
\pgfpathlineto{\pgfqpoint{3.691555in}{1.931558in}}%
\pgfpathlineto{\pgfqpoint{3.677846in}{1.936892in}}%
\pgfpathlineto{\pgfqpoint{3.664143in}{1.942254in}}%
\pgfpathlineto{\pgfqpoint{3.650446in}{1.947645in}}%
\pgfpathlineto{\pgfqpoint{3.636753in}{1.953065in}}%
\pgfpathlineto{\pgfqpoint{3.644903in}{1.957026in}}%
\pgfpathlineto{\pgfqpoint{3.653044in}{1.961181in}}%
\pgfpathlineto{\pgfqpoint{3.661176in}{1.965522in}}%
\pgfpathlineto{\pgfqpoint{3.669300in}{1.970043in}}%
\pgfpathclose%
\pgfusepath{fill}%
\end{pgfscope}%
\begin{pgfscope}%
\pgfpathrectangle{\pgfqpoint{1.150000in}{0.150000in}}{\pgfqpoint{5.700000in}{5.700000in}}%
\pgfusepath{clip}%
\pgfsetbuttcap%
\pgfsetroundjoin%
\definecolor{currentfill}{rgb}{0.267004,0.004874,0.329415}%
\pgfsetfillcolor{currentfill}%
\pgfsetfillopacity{0.700000}%
\pgfsetlinewidth{0.000000pt}%
\definecolor{currentstroke}{rgb}{0.000000,0.000000,0.000000}%
\pgfsetstrokecolor{currentstroke}%
\pgfsetdash{}{0pt}%
\pgfpathmoveto{\pgfqpoint{3.952874in}{1.939637in}}%
\pgfpathlineto{\pgfqpoint{3.966604in}{1.935350in}}%
\pgfpathlineto{\pgfqpoint{3.980341in}{1.931089in}}%
\pgfpathlineto{\pgfqpoint{3.994083in}{1.926855in}}%
\pgfpathlineto{\pgfqpoint{4.007831in}{1.922649in}}%
\pgfpathlineto{\pgfqpoint{3.999852in}{1.915844in}}%
\pgfpathlineto{\pgfqpoint{3.991867in}{1.909157in}}%
\pgfpathlineto{\pgfqpoint{3.983874in}{1.902593in}}%
\pgfpathlineto{\pgfqpoint{3.975876in}{1.896157in}}%
\pgfpathlineto{\pgfqpoint{3.962112in}{1.900580in}}%
\pgfpathlineto{\pgfqpoint{3.948354in}{1.905029in}}%
\pgfpathlineto{\pgfqpoint{3.934603in}{1.909505in}}%
\pgfpathlineto{\pgfqpoint{3.920857in}{1.914008in}}%
\pgfpathlineto{\pgfqpoint{3.928872in}{1.920223in}}%
\pgfpathlineto{\pgfqpoint{3.936879in}{1.926569in}}%
\pgfpathlineto{\pgfqpoint{3.944880in}{1.933043in}}%
\pgfpathlineto{\pgfqpoint{3.952874in}{1.939637in}}%
\pgfpathclose%
\pgfusepath{fill}%
\end{pgfscope}%
\begin{pgfscope}%
\pgfpathrectangle{\pgfqpoint{1.150000in}{0.150000in}}{\pgfqpoint{5.700000in}{5.700000in}}%
\pgfusepath{clip}%
\pgfsetbuttcap%
\pgfsetroundjoin%
\definecolor{currentfill}{rgb}{0.269944,0.014625,0.341379}%
\pgfsetfillcolor{currentfill}%
\pgfsetfillopacity{0.700000}%
\pgfsetlinewidth{0.000000pt}%
\definecolor{currentstroke}{rgb}{0.000000,0.000000,0.000000}%
\pgfsetstrokecolor{currentstroke}%
\pgfsetdash{}{0pt}%
\pgfpathmoveto{\pgfqpoint{4.323284in}{1.957052in}}%
\pgfpathlineto{\pgfqpoint{4.337103in}{1.953859in}}%
\pgfpathlineto{\pgfqpoint{4.350929in}{1.950693in}}%
\pgfpathlineto{\pgfqpoint{4.364762in}{1.947552in}}%
\pgfpathlineto{\pgfqpoint{4.378601in}{1.944437in}}%
\pgfpathlineto{\pgfqpoint{4.370758in}{1.935746in}}%
\pgfpathlineto{\pgfqpoint{4.362908in}{1.927099in}}%
\pgfpathlineto{\pgfqpoint{4.355054in}{1.918501in}}%
\pgfpathlineto{\pgfqpoint{4.347194in}{1.909956in}}%
\pgfpathlineto{\pgfqpoint{4.333343in}{1.913247in}}%
\pgfpathlineto{\pgfqpoint{4.319498in}{1.916564in}}%
\pgfpathlineto{\pgfqpoint{4.305661in}{1.919907in}}%
\pgfpathlineto{\pgfqpoint{4.291830in}{1.923275in}}%
\pgfpathlineto{\pgfqpoint{4.299701in}{1.931639in}}%
\pgfpathlineto{\pgfqpoint{4.307568in}{1.940059in}}%
\pgfpathlineto{\pgfqpoint{4.315428in}{1.948531in}}%
\pgfpathlineto{\pgfqpoint{4.323284in}{1.957052in}}%
\pgfpathclose%
\pgfusepath{fill}%
\end{pgfscope}%
\begin{pgfscope}%
\pgfpathrectangle{\pgfqpoint{1.150000in}{0.150000in}}{\pgfqpoint{5.700000in}{5.700000in}}%
\pgfusepath{clip}%
\pgfsetbuttcap%
\pgfsetroundjoin%
\definecolor{currentfill}{rgb}{0.280255,0.165693,0.476498}%
\pgfsetfillcolor{currentfill}%
\pgfsetfillopacity{0.700000}%
\pgfsetlinewidth{0.000000pt}%
\definecolor{currentstroke}{rgb}{0.000000,0.000000,0.000000}%
\pgfsetstrokecolor{currentstroke}%
\pgfsetdash{}{0pt}%
\pgfpathmoveto{\pgfqpoint{2.937112in}{2.227689in}}%
\pgfpathlineto{\pgfqpoint{2.950689in}{2.220038in}}%
\pgfpathlineto{\pgfqpoint{2.964269in}{2.212424in}}%
\pgfpathlineto{\pgfqpoint{2.977853in}{2.204847in}}%
\pgfpathlineto{\pgfqpoint{2.991440in}{2.197307in}}%
\pgfpathlineto{\pgfqpoint{2.982877in}{2.199398in}}%
\pgfpathlineto{\pgfqpoint{2.974298in}{2.201817in}}%
\pgfpathlineto{\pgfqpoint{2.965702in}{2.204571in}}%
\pgfpathlineto{\pgfqpoint{2.957088in}{2.207668in}}%
\pgfpathlineto{\pgfqpoint{2.943469in}{2.215509in}}%
\pgfpathlineto{\pgfqpoint{2.929853in}{2.223386in}}%
\pgfpathlineto{\pgfqpoint{2.916241in}{2.231300in}}%
\pgfpathlineto{\pgfqpoint{2.902632in}{2.239251in}}%
\pgfpathlineto{\pgfqpoint{2.911278in}{2.235848in}}%
\pgfpathlineto{\pgfqpoint{2.919907in}{2.232792in}}%
\pgfpathlineto{\pgfqpoint{2.928518in}{2.230075in}}%
\pgfpathlineto{\pgfqpoint{2.937112in}{2.227689in}}%
\pgfpathclose%
\pgfusepath{fill}%
\end{pgfscope}%
\begin{pgfscope}%
\pgfpathrectangle{\pgfqpoint{1.150000in}{0.150000in}}{\pgfqpoint{5.700000in}{5.700000in}}%
\pgfusepath{clip}%
\pgfsetbuttcap%
\pgfsetroundjoin%
\definecolor{currentfill}{rgb}{0.274128,0.199721,0.498911}%
\pgfsetfillcolor{currentfill}%
\pgfsetfillopacity{0.700000}%
\pgfsetlinewidth{0.000000pt}%
\definecolor{currentstroke}{rgb}{0.000000,0.000000,0.000000}%
\pgfsetstrokecolor{currentstroke}%
\pgfsetdash{}{0pt}%
\pgfpathmoveto{\pgfqpoint{5.671627in}{2.306277in}}%
\pgfpathlineto{\pgfqpoint{5.685855in}{2.305853in}}%
\pgfpathlineto{\pgfqpoint{5.700093in}{2.305454in}}%
\pgfpathlineto{\pgfqpoint{5.714340in}{2.305079in}}%
\pgfpathlineto{\pgfqpoint{5.728595in}{2.304729in}}%
\pgfpathlineto{\pgfqpoint{5.721273in}{2.297303in}}%
\pgfpathlineto{\pgfqpoint{5.713943in}{2.289772in}}%
\pgfpathlineto{\pgfqpoint{5.706603in}{2.282136in}}%
\pgfpathlineto{\pgfqpoint{5.699255in}{2.274392in}}%
\pgfpathlineto{\pgfqpoint{5.684986in}{2.274742in}}%
\pgfpathlineto{\pgfqpoint{5.670726in}{2.275117in}}%
\pgfpathlineto{\pgfqpoint{5.656475in}{2.275517in}}%
\pgfpathlineto{\pgfqpoint{5.642233in}{2.275941in}}%
\pgfpathlineto{\pgfqpoint{5.649594in}{2.283679in}}%
\pgfpathlineto{\pgfqpoint{5.656947in}{2.291314in}}%
\pgfpathlineto{\pgfqpoint{5.664291in}{2.298846in}}%
\pgfpathlineto{\pgfqpoint{5.671627in}{2.306277in}}%
\pgfpathclose%
\pgfusepath{fill}%
\end{pgfscope}%
\begin{pgfscope}%
\pgfpathrectangle{\pgfqpoint{1.150000in}{0.150000in}}{\pgfqpoint{5.700000in}{5.700000in}}%
\pgfusepath{clip}%
\pgfsetbuttcap%
\pgfsetroundjoin%
\definecolor{currentfill}{rgb}{0.265145,0.232956,0.516599}%
\pgfsetfillcolor{currentfill}%
\pgfsetfillopacity{0.700000}%
\pgfsetlinewidth{0.000000pt}%
\definecolor{currentstroke}{rgb}{0.000000,0.000000,0.000000}%
\pgfsetstrokecolor{currentstroke}%
\pgfsetdash{}{0pt}%
\pgfpathmoveto{\pgfqpoint{2.685297in}{2.371741in}}%
\pgfpathlineto{\pgfqpoint{2.698859in}{2.363156in}}%
\pgfpathlineto{\pgfqpoint{2.712424in}{2.354614in}}%
\pgfpathlineto{\pgfqpoint{2.725992in}{2.346113in}}%
\pgfpathlineto{\pgfqpoint{2.739562in}{2.337654in}}%
\pgfpathlineto{\pgfqpoint{2.730793in}{2.342343in}}%
\pgfpathlineto{\pgfqpoint{2.722004in}{2.347407in}}%
\pgfpathlineto{\pgfqpoint{2.713195in}{2.352855in}}%
\pgfpathlineto{\pgfqpoint{2.704364in}{2.358696in}}%
\pgfpathlineto{\pgfqpoint{2.690757in}{2.367472in}}%
\pgfpathlineto{\pgfqpoint{2.677152in}{2.376289in}}%
\pgfpathlineto{\pgfqpoint{2.663550in}{2.385149in}}%
\pgfpathlineto{\pgfqpoint{2.649950in}{2.394050in}}%
\pgfpathlineto{\pgfqpoint{2.658819in}{2.387887in}}%
\pgfpathlineto{\pgfqpoint{2.667666in}{2.382120in}}%
\pgfpathlineto{\pgfqpoint{2.676492in}{2.376741in}}%
\pgfpathlineto{\pgfqpoint{2.685297in}{2.371741in}}%
\pgfpathclose%
\pgfusepath{fill}%
\end{pgfscope}%
\begin{pgfscope}%
\pgfpathrectangle{\pgfqpoint{1.150000in}{0.150000in}}{\pgfqpoint{5.700000in}{5.700000in}}%
\pgfusepath{clip}%
\pgfsetbuttcap%
\pgfsetroundjoin%
\definecolor{currentfill}{rgb}{0.274952,0.037752,0.364543}%
\pgfsetfillcolor{currentfill}%
\pgfsetfillopacity{0.700000}%
\pgfsetlinewidth{0.000000pt}%
\definecolor{currentstroke}{rgb}{0.000000,0.000000,0.000000}%
\pgfsetstrokecolor{currentstroke}%
\pgfsetdash{}{0pt}%
\pgfpathmoveto{\pgfqpoint{4.551954in}{1.993593in}}%
\pgfpathlineto{\pgfqpoint{4.565836in}{1.991020in}}%
\pgfpathlineto{\pgfqpoint{4.579726in}{1.988472in}}%
\pgfpathlineto{\pgfqpoint{4.593623in}{1.985950in}}%
\pgfpathlineto{\pgfqpoint{4.607527in}{1.983453in}}%
\pgfpathlineto{\pgfqpoint{4.599758in}{1.974112in}}%
\pgfpathlineto{\pgfqpoint{4.591984in}{1.964775in}}%
\pgfpathlineto{\pgfqpoint{4.584204in}{1.955447in}}%
\pgfpathlineto{\pgfqpoint{4.576419in}{1.946129in}}%
\pgfpathlineto{\pgfqpoint{4.562504in}{1.948776in}}%
\pgfpathlineto{\pgfqpoint{4.548597in}{1.951448in}}%
\pgfpathlineto{\pgfqpoint{4.534697in}{1.954145in}}%
\pgfpathlineto{\pgfqpoint{4.520805in}{1.956868in}}%
\pgfpathlineto{\pgfqpoint{4.528600in}{1.966031in}}%
\pgfpathlineto{\pgfqpoint{4.536390in}{1.975208in}}%
\pgfpathlineto{\pgfqpoint{4.544175in}{1.984397in}}%
\pgfpathlineto{\pgfqpoint{4.551954in}{1.993593in}}%
\pgfpathclose%
\pgfusepath{fill}%
\end{pgfscope}%
\begin{pgfscope}%
\pgfpathrectangle{\pgfqpoint{1.150000in}{0.150000in}}{\pgfqpoint{5.700000in}{5.700000in}}%
\pgfusepath{clip}%
\pgfsetbuttcap%
\pgfsetroundjoin%
\definecolor{currentfill}{rgb}{0.276022,0.044167,0.370164}%
\pgfsetfillcolor{currentfill}%
\pgfsetfillopacity{0.700000}%
\pgfsetlinewidth{0.000000pt}%
\definecolor{currentstroke}{rgb}{0.000000,0.000000,0.000000}%
\pgfsetstrokecolor{currentstroke}%
\pgfsetdash{}{0pt}%
\pgfpathmoveto{\pgfqpoint{3.527399in}{1.997465in}}%
\pgfpathlineto{\pgfqpoint{3.541050in}{1.991812in}}%
\pgfpathlineto{\pgfqpoint{3.554707in}{1.986189in}}%
\pgfpathlineto{\pgfqpoint{3.568368in}{1.980595in}}%
\pgfpathlineto{\pgfqpoint{3.582035in}{1.975031in}}%
\pgfpathlineto{\pgfqpoint{3.573855in}{1.971519in}}%
\pgfpathlineto{\pgfqpoint{3.565666in}{1.968216in}}%
\pgfpathlineto{\pgfqpoint{3.557467in}{1.965128in}}%
\pgfpathlineto{\pgfqpoint{3.549258in}{1.962262in}}%
\pgfpathlineto{\pgfqpoint{3.535570in}{1.968082in}}%
\pgfpathlineto{\pgfqpoint{3.521886in}{1.973932in}}%
\pgfpathlineto{\pgfqpoint{3.508208in}{1.979811in}}%
\pgfpathlineto{\pgfqpoint{3.494535in}{1.985720in}}%
\pgfpathlineto{\pgfqpoint{3.502766in}{1.988325in}}%
\pgfpathlineto{\pgfqpoint{3.510987in}{1.991155in}}%
\pgfpathlineto{\pgfqpoint{3.519198in}{1.994204in}}%
\pgfpathlineto{\pgfqpoint{3.527399in}{1.997465in}}%
\pgfpathclose%
\pgfusepath{fill}%
\end{pgfscope}%
\begin{pgfscope}%
\pgfpathrectangle{\pgfqpoint{1.150000in}{0.150000in}}{\pgfqpoint{5.700000in}{5.700000in}}%
\pgfusepath{clip}%
\pgfsetbuttcap%
\pgfsetroundjoin%
\definecolor{currentfill}{rgb}{0.282623,0.140926,0.457517}%
\pgfsetfillcolor{currentfill}%
\pgfsetfillopacity{0.700000}%
\pgfsetlinewidth{0.000000pt}%
\definecolor{currentstroke}{rgb}{0.000000,0.000000,0.000000}%
\pgfsetstrokecolor{currentstroke}%
\pgfsetdash{}{0pt}%
\pgfpathmoveto{\pgfqpoint{5.269597in}{2.190340in}}%
\pgfpathlineto{\pgfqpoint{5.283696in}{2.189338in}}%
\pgfpathlineto{\pgfqpoint{5.297804in}{2.188362in}}%
\pgfpathlineto{\pgfqpoint{5.311921in}{2.187410in}}%
\pgfpathlineto{\pgfqpoint{5.326047in}{2.186483in}}%
\pgfpathlineto{\pgfqpoint{5.318537in}{2.177606in}}%
\pgfpathlineto{\pgfqpoint{5.311019in}{2.168643in}}%
\pgfpathlineto{\pgfqpoint{5.303495in}{2.159594in}}%
\pgfpathlineto{\pgfqpoint{5.295963in}{2.150461in}}%
\pgfpathlineto{\pgfqpoint{5.281827in}{2.151444in}}%
\pgfpathlineto{\pgfqpoint{5.267700in}{2.152451in}}%
\pgfpathlineto{\pgfqpoint{5.253582in}{2.153483in}}%
\pgfpathlineto{\pgfqpoint{5.239472in}{2.154540in}}%
\pgfpathlineto{\pgfqpoint{5.247014in}{2.163612in}}%
\pgfpathlineto{\pgfqpoint{5.254548in}{2.172604in}}%
\pgfpathlineto{\pgfqpoint{5.262076in}{2.181513in}}%
\pgfpathlineto{\pgfqpoint{5.269597in}{2.190340in}}%
\pgfpathclose%
\pgfusepath{fill}%
\end{pgfscope}%
\begin{pgfscope}%
\pgfpathrectangle{\pgfqpoint{1.150000in}{0.150000in}}{\pgfqpoint{5.700000in}{5.700000in}}%
\pgfusepath{clip}%
\pgfsetbuttcap%
\pgfsetroundjoin%
\definecolor{currentfill}{rgb}{0.267004,0.004874,0.329415}%
\pgfsetfillcolor{currentfill}%
\pgfsetfillopacity{0.700000}%
\pgfsetlinewidth{0.000000pt}%
\definecolor{currentstroke}{rgb}{0.000000,0.000000,0.000000}%
\pgfsetstrokecolor{currentstroke}%
\pgfsetdash{}{0pt}%
\pgfpathmoveto{\pgfqpoint{4.094680in}{1.935193in}}%
\pgfpathlineto{\pgfqpoint{4.108445in}{1.931322in}}%
\pgfpathlineto{\pgfqpoint{4.122217in}{1.927477in}}%
\pgfpathlineto{\pgfqpoint{4.135995in}{1.923659in}}%
\pgfpathlineto{\pgfqpoint{4.149779in}{1.919867in}}%
\pgfpathlineto{\pgfqpoint{4.141853in}{1.912248in}}%
\pgfpathlineto{\pgfqpoint{4.133921in}{1.904719in}}%
\pgfpathlineto{\pgfqpoint{4.125984in}{1.897286in}}%
\pgfpathlineto{\pgfqpoint{4.118040in}{1.889954in}}%
\pgfpathlineto{\pgfqpoint{4.104242in}{1.893948in}}%
\pgfpathlineto{\pgfqpoint{4.090450in}{1.897969in}}%
\pgfpathlineto{\pgfqpoint{4.076665in}{1.902016in}}%
\pgfpathlineto{\pgfqpoint{4.062886in}{1.906089in}}%
\pgfpathlineto{\pgfqpoint{4.070844in}{1.913214in}}%
\pgfpathlineto{\pgfqpoint{4.078795in}{1.920443in}}%
\pgfpathlineto{\pgfqpoint{4.086741in}{1.927771in}}%
\pgfpathlineto{\pgfqpoint{4.094680in}{1.935193in}}%
\pgfpathclose%
\pgfusepath{fill}%
\end{pgfscope}%
\begin{pgfscope}%
\pgfpathrectangle{\pgfqpoint{1.150000in}{0.150000in}}{\pgfqpoint{5.700000in}{5.700000in}}%
\pgfusepath{clip}%
\pgfsetbuttcap%
\pgfsetroundjoin%
\definecolor{currentfill}{rgb}{0.280894,0.078907,0.402329}%
\pgfsetfillcolor{currentfill}%
\pgfsetfillopacity{0.700000}%
\pgfsetlinewidth{0.000000pt}%
\definecolor{currentstroke}{rgb}{0.000000,0.000000,0.000000}%
\pgfsetstrokecolor{currentstroke}%
\pgfsetdash{}{0pt}%
\pgfpathmoveto{\pgfqpoint{4.867418in}{2.071157in}}%
\pgfpathlineto{\pgfqpoint{4.881392in}{2.069355in}}%
\pgfpathlineto{\pgfqpoint{4.895375in}{2.067579in}}%
\pgfpathlineto{\pgfqpoint{4.909365in}{2.065827in}}%
\pgfpathlineto{\pgfqpoint{4.923364in}{2.064100in}}%
\pgfpathlineto{\pgfqpoint{4.915700in}{2.054528in}}%
\pgfpathlineto{\pgfqpoint{4.908030in}{2.044912in}}%
\pgfpathlineto{\pgfqpoint{4.900354in}{2.035254in}}%
\pgfpathlineto{\pgfqpoint{4.892673in}{2.025558in}}%
\pgfpathlineto{\pgfqpoint{4.878665in}{2.027394in}}%
\pgfpathlineto{\pgfqpoint{4.864665in}{2.029256in}}%
\pgfpathlineto{\pgfqpoint{4.850674in}{2.031142in}}%
\pgfpathlineto{\pgfqpoint{4.836689in}{2.033053in}}%
\pgfpathlineto{\pgfqpoint{4.844380in}{2.042635in}}%
\pgfpathlineto{\pgfqpoint{4.852065in}{2.052181in}}%
\pgfpathlineto{\pgfqpoint{4.859744in}{2.061689in}}%
\pgfpathlineto{\pgfqpoint{4.867418in}{2.071157in}}%
\pgfpathclose%
\pgfusepath{fill}%
\end{pgfscope}%
\begin{pgfscope}%
\pgfpathrectangle{\pgfqpoint{1.150000in}{0.150000in}}{\pgfqpoint{5.700000in}{5.700000in}}%
\pgfusepath{clip}%
\pgfsetbuttcap%
\pgfsetroundjoin%
\definecolor{currentfill}{rgb}{0.225863,0.330805,0.547314}%
\pgfsetfillcolor{currentfill}%
\pgfsetfillopacity{0.700000}%
\pgfsetlinewidth{0.000000pt}%
\definecolor{currentstroke}{rgb}{0.000000,0.000000,0.000000}%
\pgfsetstrokecolor{currentstroke}%
\pgfsetdash{}{0pt}%
\pgfpathmoveto{\pgfqpoint{2.378396in}{2.581594in}}%
\pgfpathlineto{\pgfqpoint{2.391957in}{2.571758in}}%
\pgfpathlineto{\pgfqpoint{2.405519in}{2.561973in}}%
\pgfpathlineto{\pgfqpoint{2.419083in}{2.552239in}}%
\pgfpathlineto{\pgfqpoint{2.432649in}{2.542555in}}%
\pgfpathlineto{\pgfqpoint{2.423597in}{2.550436in}}%
\pgfpathlineto{\pgfqpoint{2.414519in}{2.558747in}}%
\pgfpathlineto{\pgfqpoint{2.405416in}{2.567498in}}%
\pgfpathlineto{\pgfqpoint{2.396286in}{2.576697in}}%
\pgfpathlineto{\pgfqpoint{2.382677in}{2.586716in}}%
\pgfpathlineto{\pgfqpoint{2.369070in}{2.596786in}}%
\pgfpathlineto{\pgfqpoint{2.355464in}{2.606907in}}%
\pgfpathlineto{\pgfqpoint{2.341860in}{2.617080in}}%
\pgfpathlineto{\pgfqpoint{2.351034in}{2.607539in}}%
\pgfpathlineto{\pgfqpoint{2.360181in}{2.598450in}}%
\pgfpathlineto{\pgfqpoint{2.369301in}{2.589805in}}%
\pgfpathlineto{\pgfqpoint{2.378396in}{2.581594in}}%
\pgfpathclose%
\pgfusepath{fill}%
\end{pgfscope}%
\begin{pgfscope}%
\pgfpathrectangle{\pgfqpoint{1.150000in}{0.150000in}}{\pgfqpoint{5.700000in}{5.700000in}}%
\pgfusepath{clip}%
\pgfsetbuttcap%
\pgfsetroundjoin%
\definecolor{currentfill}{rgb}{0.276194,0.190074,0.493001}%
\pgfsetfillcolor{currentfill}%
\pgfsetfillopacity{0.700000}%
\pgfsetlinewidth{0.000000pt}%
\definecolor{currentstroke}{rgb}{0.000000,0.000000,0.000000}%
\pgfsetstrokecolor{currentstroke}%
\pgfsetdash{}{0pt}%
\pgfpathmoveto{\pgfqpoint{5.585357in}{2.277883in}}%
\pgfpathlineto{\pgfqpoint{5.599563in}{2.277361in}}%
\pgfpathlineto{\pgfqpoint{5.613777in}{2.276863in}}%
\pgfpathlineto{\pgfqpoint{5.628001in}{2.276389in}}%
\pgfpathlineto{\pgfqpoint{5.642233in}{2.275941in}}%
\pgfpathlineto{\pgfqpoint{5.634864in}{2.268099in}}%
\pgfpathlineto{\pgfqpoint{5.627486in}{2.260155in}}%
\pgfpathlineto{\pgfqpoint{5.620099in}{2.252106in}}%
\pgfpathlineto{\pgfqpoint{5.612705in}{2.243953in}}%
\pgfpathlineto{\pgfqpoint{5.598460in}{2.244416in}}%
\pgfpathlineto{\pgfqpoint{5.584224in}{2.244903in}}%
\pgfpathlineto{\pgfqpoint{5.569997in}{2.245415in}}%
\pgfpathlineto{\pgfqpoint{5.555779in}{2.245952in}}%
\pgfpathlineto{\pgfqpoint{5.563186in}{2.254085in}}%
\pgfpathlineto{\pgfqpoint{5.570585in}{2.262118in}}%
\pgfpathlineto{\pgfqpoint{5.577975in}{2.270051in}}%
\pgfpathlineto{\pgfqpoint{5.585357in}{2.277883in}}%
\pgfpathclose%
\pgfusepath{fill}%
\end{pgfscope}%
\begin{pgfscope}%
\pgfpathrectangle{\pgfqpoint{1.150000in}{0.150000in}}{\pgfqpoint{5.700000in}{5.700000in}}%
\pgfusepath{clip}%
\pgfsetbuttcap%
\pgfsetroundjoin%
\definecolor{currentfill}{rgb}{0.283072,0.130895,0.449241}%
\pgfsetfillcolor{currentfill}%
\pgfsetfillopacity{0.700000}%
\pgfsetlinewidth{0.000000pt}%
\definecolor{currentstroke}{rgb}{0.000000,0.000000,0.000000}%
\pgfsetstrokecolor{currentstroke}%
\pgfsetdash{}{0pt}%
\pgfpathmoveto{\pgfqpoint{5.183117in}{2.159015in}}%
\pgfpathlineto{\pgfqpoint{5.197193in}{2.157859in}}%
\pgfpathlineto{\pgfqpoint{5.211277in}{2.156728in}}%
\pgfpathlineto{\pgfqpoint{5.225370in}{2.155621in}}%
\pgfpathlineto{\pgfqpoint{5.239472in}{2.154540in}}%
\pgfpathlineto{\pgfqpoint{5.231923in}{2.145388in}}%
\pgfpathlineto{\pgfqpoint{5.224367in}{2.136157in}}%
\pgfpathlineto{\pgfqpoint{5.216805in}{2.126849in}}%
\pgfpathlineto{\pgfqpoint{5.209236in}{2.117465in}}%
\pgfpathlineto{\pgfqpoint{5.195124in}{2.118616in}}%
\pgfpathlineto{\pgfqpoint{5.181022in}{2.119791in}}%
\pgfpathlineto{\pgfqpoint{5.166927in}{2.120991in}}%
\pgfpathlineto{\pgfqpoint{5.152841in}{2.122216in}}%
\pgfpathlineto{\pgfqpoint{5.160420in}{2.131526in}}%
\pgfpathlineto{\pgfqpoint{5.167993in}{2.140763in}}%
\pgfpathlineto{\pgfqpoint{5.175558in}{2.149927in}}%
\pgfpathlineto{\pgfqpoint{5.183117in}{2.159015in}}%
\pgfpathclose%
\pgfusepath{fill}%
\end{pgfscope}%
\begin{pgfscope}%
\pgfpathrectangle{\pgfqpoint{1.150000in}{0.150000in}}{\pgfqpoint{5.700000in}{5.700000in}}%
\pgfusepath{clip}%
\pgfsetbuttcap%
\pgfsetroundjoin%
\definecolor{currentfill}{rgb}{0.283091,0.110553,0.431554}%
\pgfsetfillcolor{currentfill}%
\pgfsetfillopacity{0.700000}%
\pgfsetlinewidth{0.000000pt}%
\definecolor{currentstroke}{rgb}{0.000000,0.000000,0.000000}%
\pgfsetstrokecolor{currentstroke}%
\pgfsetdash{}{0pt}%
\pgfpathmoveto{\pgfqpoint{3.188519in}{2.107735in}}%
\pgfpathlineto{\pgfqpoint{3.202127in}{2.100930in}}%
\pgfpathlineto{\pgfqpoint{3.215739in}{2.094157in}}%
\pgfpathlineto{\pgfqpoint{3.229355in}{2.087418in}}%
\pgfpathlineto{\pgfqpoint{3.242976in}{2.080711in}}%
\pgfpathlineto{\pgfqpoint{3.234587in}{2.080454in}}%
\pgfpathlineto{\pgfqpoint{3.226185in}{2.080478in}}%
\pgfpathlineto{\pgfqpoint{3.217770in}{2.080792in}}%
\pgfpathlineto{\pgfqpoint{3.209341in}{2.081403in}}%
\pgfpathlineto{\pgfqpoint{3.195693in}{2.088394in}}%
\pgfpathlineto{\pgfqpoint{3.182049in}{2.095418in}}%
\pgfpathlineto{\pgfqpoint{3.168409in}{2.102475in}}%
\pgfpathlineto{\pgfqpoint{3.154773in}{2.109565in}}%
\pgfpathlineto{\pgfqpoint{3.163231in}{2.108664in}}%
\pgfpathlineto{\pgfqpoint{3.171674in}{2.108064in}}%
\pgfpathlineto{\pgfqpoint{3.180103in}{2.107757in}}%
\pgfpathlineto{\pgfqpoint{3.188519in}{2.107735in}}%
\pgfpathclose%
\pgfusepath{fill}%
\end{pgfscope}%
\begin{pgfscope}%
\pgfpathrectangle{\pgfqpoint{1.150000in}{0.150000in}}{\pgfqpoint{5.700000in}{5.700000in}}%
\pgfusepath{clip}%
\pgfsetbuttcap%
\pgfsetroundjoin%
\definecolor{currentfill}{rgb}{0.279566,0.067836,0.391917}%
\pgfsetfillcolor{currentfill}%
\pgfsetfillopacity{0.700000}%
\pgfsetlinewidth{0.000000pt}%
\definecolor{currentstroke}{rgb}{0.000000,0.000000,0.000000}%
\pgfsetstrokecolor{currentstroke}%
\pgfsetdash{}{0pt}%
\pgfpathmoveto{\pgfqpoint{4.780831in}{2.040949in}}%
\pgfpathlineto{\pgfqpoint{4.794784in}{2.038938in}}%
\pgfpathlineto{\pgfqpoint{4.808745in}{2.036951in}}%
\pgfpathlineto{\pgfqpoint{4.822713in}{2.034990in}}%
\pgfpathlineto{\pgfqpoint{4.836689in}{2.033053in}}%
\pgfpathlineto{\pgfqpoint{4.828993in}{2.023439in}}%
\pgfpathlineto{\pgfqpoint{4.821291in}{2.013794in}}%
\pgfpathlineto{\pgfqpoint{4.813584in}{2.004121in}}%
\pgfpathlineto{\pgfqpoint{4.805871in}{1.994424in}}%
\pgfpathlineto{\pgfqpoint{4.791885in}{1.996484in}}%
\pgfpathlineto{\pgfqpoint{4.777907in}{1.998568in}}%
\pgfpathlineto{\pgfqpoint{4.763937in}{2.000678in}}%
\pgfpathlineto{\pgfqpoint{4.749975in}{2.002812in}}%
\pgfpathlineto{\pgfqpoint{4.757697in}{2.012381in}}%
\pgfpathlineto{\pgfqpoint{4.765414in}{2.021929in}}%
\pgfpathlineto{\pgfqpoint{4.773126in}{2.031452in}}%
\pgfpathlineto{\pgfqpoint{4.780831in}{2.040949in}}%
\pgfpathclose%
\pgfusepath{fill}%
\end{pgfscope}%
\begin{pgfscope}%
\pgfpathrectangle{\pgfqpoint{1.150000in}{0.150000in}}{\pgfqpoint{5.700000in}{5.700000in}}%
\pgfusepath{clip}%
\pgfsetbuttcap%
\pgfsetroundjoin%
\definecolor{currentfill}{rgb}{0.268510,0.009605,0.335427}%
\pgfsetfillcolor{currentfill}%
\pgfsetfillopacity{0.700000}%
\pgfsetlinewidth{0.000000pt}%
\definecolor{currentstroke}{rgb}{0.000000,0.000000,0.000000}%
\pgfsetstrokecolor{currentstroke}%
\pgfsetdash{}{0pt}%
\pgfpathmoveto{\pgfqpoint{4.236573in}{1.937007in}}%
\pgfpathlineto{\pgfqpoint{4.250377in}{1.933535in}}%
\pgfpathlineto{\pgfqpoint{4.264188in}{1.930089in}}%
\pgfpathlineto{\pgfqpoint{4.278006in}{1.926669in}}%
\pgfpathlineto{\pgfqpoint{4.291830in}{1.923275in}}%
\pgfpathlineto{\pgfqpoint{4.283952in}{1.914972in}}%
\pgfpathlineto{\pgfqpoint{4.276070in}{1.906734in}}%
\pgfpathlineto{\pgfqpoint{4.268181in}{1.898566in}}%
\pgfpathlineto{\pgfqpoint{4.260287in}{1.890472in}}%
\pgfpathlineto{\pgfqpoint{4.246451in}{1.894056in}}%
\pgfpathlineto{\pgfqpoint{4.232621in}{1.897665in}}%
\pgfpathlineto{\pgfqpoint{4.218797in}{1.901300in}}%
\pgfpathlineto{\pgfqpoint{4.204981in}{1.904961in}}%
\pgfpathlineto{\pgfqpoint{4.212887in}{1.912861in}}%
\pgfpathlineto{\pgfqpoint{4.220788in}{1.920838in}}%
\pgfpathlineto{\pgfqpoint{4.228684in}{1.928888in}}%
\pgfpathlineto{\pgfqpoint{4.236573in}{1.937007in}}%
\pgfpathclose%
\pgfusepath{fill}%
\end{pgfscope}%
\begin{pgfscope}%
\pgfpathrectangle{\pgfqpoint{1.150000in}{0.150000in}}{\pgfqpoint{5.700000in}{5.700000in}}%
\pgfusepath{clip}%
\pgfsetbuttcap%
\pgfsetroundjoin%
\definecolor{currentfill}{rgb}{0.272594,0.025563,0.353093}%
\pgfsetfillcolor{currentfill}%
\pgfsetfillopacity{0.700000}%
\pgfsetlinewidth{0.000000pt}%
\definecolor{currentstroke}{rgb}{0.000000,0.000000,0.000000}%
\pgfsetstrokecolor{currentstroke}%
\pgfsetdash{}{0pt}%
\pgfpathmoveto{\pgfqpoint{4.465307in}{1.968013in}}%
\pgfpathlineto{\pgfqpoint{4.479170in}{1.965188in}}%
\pgfpathlineto{\pgfqpoint{4.493041in}{1.962389in}}%
\pgfpathlineto{\pgfqpoint{4.506920in}{1.959616in}}%
\pgfpathlineto{\pgfqpoint{4.520805in}{1.956868in}}%
\pgfpathlineto{\pgfqpoint{4.513004in}{1.947723in}}%
\pgfpathlineto{\pgfqpoint{4.505199in}{1.938600in}}%
\pgfpathlineto{\pgfqpoint{4.497387in}{1.929503in}}%
\pgfpathlineto{\pgfqpoint{4.489571in}{1.920436in}}%
\pgfpathlineto{\pgfqpoint{4.475675in}{1.923347in}}%
\pgfpathlineto{\pgfqpoint{4.461786in}{1.926283in}}%
\pgfpathlineto{\pgfqpoint{4.447905in}{1.929245in}}%
\pgfpathlineto{\pgfqpoint{4.434030in}{1.932232in}}%
\pgfpathlineto{\pgfqpoint{4.441857in}{1.941132in}}%
\pgfpathlineto{\pgfqpoint{4.449679in}{1.950064in}}%
\pgfpathlineto{\pgfqpoint{4.457495in}{1.959026in}}%
\pgfpathlineto{\pgfqpoint{4.465307in}{1.968013in}}%
\pgfpathclose%
\pgfusepath{fill}%
\end{pgfscope}%
\begin{pgfscope}%
\pgfpathrectangle{\pgfqpoint{1.150000in}{0.150000in}}{\pgfqpoint{5.700000in}{5.700000in}}%
\pgfusepath{clip}%
\pgfsetbuttcap%
\pgfsetroundjoin%
\definecolor{currentfill}{rgb}{0.279566,0.067836,0.391917}%
\pgfsetfillcolor{currentfill}%
\pgfsetfillopacity{0.700000}%
\pgfsetlinewidth{0.000000pt}%
\definecolor{currentstroke}{rgb}{0.000000,0.000000,0.000000}%
\pgfsetstrokecolor{currentstroke}%
\pgfsetdash{}{0pt}%
\pgfpathmoveto{\pgfqpoint{3.385320in}{2.034075in}}%
\pgfpathlineto{\pgfqpoint{3.398956in}{2.027924in}}%
\pgfpathlineto{\pgfqpoint{3.412596in}{2.021804in}}%
\pgfpathlineto{\pgfqpoint{3.426240in}{2.015714in}}%
\pgfpathlineto{\pgfqpoint{3.439889in}{2.009655in}}%
\pgfpathlineto{\pgfqpoint{3.431624in}{2.007547in}}%
\pgfpathlineto{\pgfqpoint{3.423348in}{2.005682in}}%
\pgfpathlineto{\pgfqpoint{3.415061in}{2.004067in}}%
\pgfpathlineto{\pgfqpoint{3.406762in}{2.002709in}}%
\pgfpathlineto{\pgfqpoint{3.393088in}{2.009038in}}%
\pgfpathlineto{\pgfqpoint{3.379419in}{2.015398in}}%
\pgfpathlineto{\pgfqpoint{3.365755in}{2.021788in}}%
\pgfpathlineto{\pgfqpoint{3.352095in}{2.028209in}}%
\pgfpathlineto{\pgfqpoint{3.360419in}{2.029293in}}%
\pgfpathlineto{\pgfqpoint{3.368731in}{2.030636in}}%
\pgfpathlineto{\pgfqpoint{3.377032in}{2.032233in}}%
\pgfpathlineto{\pgfqpoint{3.385320in}{2.034075in}}%
\pgfpathclose%
\pgfusepath{fill}%
\end{pgfscope}%
\begin{pgfscope}%
\pgfpathrectangle{\pgfqpoint{1.150000in}{0.150000in}}{\pgfqpoint{5.700000in}{5.700000in}}%
\pgfusepath{clip}%
\pgfsetbuttcap%
\pgfsetroundjoin%
\definecolor{currentfill}{rgb}{0.267968,0.223549,0.512008}%
\pgfsetfillcolor{currentfill}%
\pgfsetfillopacity{0.700000}%
\pgfsetlinewidth{0.000000pt}%
\definecolor{currentstroke}{rgb}{0.000000,0.000000,0.000000}%
\pgfsetstrokecolor{currentstroke}%
\pgfsetdash{}{0pt}%
\pgfpathmoveto{\pgfqpoint{2.739562in}{2.337654in}}%
\pgfpathlineto{\pgfqpoint{2.753135in}{2.329236in}}%
\pgfpathlineto{\pgfqpoint{2.766711in}{2.320858in}}%
\pgfpathlineto{\pgfqpoint{2.780290in}{2.312521in}}%
\pgfpathlineto{\pgfqpoint{2.793871in}{2.304224in}}%
\pgfpathlineto{\pgfqpoint{2.785138in}{2.308603in}}%
\pgfpathlineto{\pgfqpoint{2.776385in}{2.313353in}}%
\pgfpathlineto{\pgfqpoint{2.767612in}{2.318482in}}%
\pgfpathlineto{\pgfqpoint{2.758818in}{2.324000in}}%
\pgfpathlineto{\pgfqpoint{2.745200in}{2.332614in}}%
\pgfpathlineto{\pgfqpoint{2.731586in}{2.341267in}}%
\pgfpathlineto{\pgfqpoint{2.717973in}{2.349961in}}%
\pgfpathlineto{\pgfqpoint{2.704364in}{2.358696in}}%
\pgfpathlineto{\pgfqpoint{2.713195in}{2.352855in}}%
\pgfpathlineto{\pgfqpoint{2.722004in}{2.347407in}}%
\pgfpathlineto{\pgfqpoint{2.730793in}{2.342343in}}%
\pgfpathlineto{\pgfqpoint{2.739562in}{2.337654in}}%
\pgfpathclose%
\pgfusepath{fill}%
\end{pgfscope}%
\begin{pgfscope}%
\pgfpathrectangle{\pgfqpoint{1.150000in}{0.150000in}}{\pgfqpoint{5.700000in}{5.700000in}}%
\pgfusepath{clip}%
\pgfsetbuttcap%
\pgfsetroundjoin%
\definecolor{currentfill}{rgb}{0.281412,0.155834,0.469201}%
\pgfsetfillcolor{currentfill}%
\pgfsetfillopacity{0.700000}%
\pgfsetlinewidth{0.000000pt}%
\definecolor{currentstroke}{rgb}{0.000000,0.000000,0.000000}%
\pgfsetstrokecolor{currentstroke}%
\pgfsetdash{}{0pt}%
\pgfpathmoveto{\pgfqpoint{2.991440in}{2.197307in}}%
\pgfpathlineto{\pgfqpoint{3.005031in}{2.189802in}}%
\pgfpathlineto{\pgfqpoint{3.018625in}{2.182333in}}%
\pgfpathlineto{\pgfqpoint{3.032223in}{2.174900in}}%
\pgfpathlineto{\pgfqpoint{3.045825in}{2.167503in}}%
\pgfpathlineto{\pgfqpoint{3.037293in}{2.169300in}}%
\pgfpathlineto{\pgfqpoint{3.028744in}{2.171421in}}%
\pgfpathlineto{\pgfqpoint{3.020180in}{2.173873in}}%
\pgfpathlineto{\pgfqpoint{3.011599in}{2.176665in}}%
\pgfpathlineto{\pgfqpoint{2.997966in}{2.184363in}}%
\pgfpathlineto{\pgfqpoint{2.984337in}{2.192095in}}%
\pgfpathlineto{\pgfqpoint{2.970711in}{2.199864in}}%
\pgfpathlineto{\pgfqpoint{2.957088in}{2.207668in}}%
\pgfpathlineto{\pgfqpoint{2.965702in}{2.204571in}}%
\pgfpathlineto{\pgfqpoint{2.974298in}{2.201817in}}%
\pgfpathlineto{\pgfqpoint{2.982877in}{2.199398in}}%
\pgfpathlineto{\pgfqpoint{2.991440in}{2.197307in}}%
\pgfpathclose%
\pgfusepath{fill}%
\end{pgfscope}%
\begin{pgfscope}%
\pgfpathrectangle{\pgfqpoint{1.150000in}{0.150000in}}{\pgfqpoint{5.700000in}{5.700000in}}%
\pgfusepath{clip}%
\pgfsetbuttcap%
\pgfsetroundjoin%
\definecolor{currentfill}{rgb}{0.231674,0.318106,0.544834}%
\pgfsetfillcolor{currentfill}%
\pgfsetfillopacity{0.700000}%
\pgfsetlinewidth{0.000000pt}%
\definecolor{currentstroke}{rgb}{0.000000,0.000000,0.000000}%
\pgfsetstrokecolor{currentstroke}%
\pgfsetdash{}{0pt}%
\pgfpathmoveto{\pgfqpoint{2.432649in}{2.542555in}}%
\pgfpathlineto{\pgfqpoint{2.446216in}{2.532921in}}%
\pgfpathlineto{\pgfqpoint{2.459785in}{2.523336in}}%
\pgfpathlineto{\pgfqpoint{2.473356in}{2.513800in}}%
\pgfpathlineto{\pgfqpoint{2.486928in}{2.504312in}}%
\pgfpathlineto{\pgfqpoint{2.477917in}{2.511864in}}%
\pgfpathlineto{\pgfqpoint{2.468881in}{2.519841in}}%
\pgfpathlineto{\pgfqpoint{2.459821in}{2.528255in}}%
\pgfpathlineto{\pgfqpoint{2.450734in}{2.537112in}}%
\pgfpathlineto{\pgfqpoint{2.437120in}{2.546935in}}%
\pgfpathlineto{\pgfqpoint{2.423507in}{2.556807in}}%
\pgfpathlineto{\pgfqpoint{2.409896in}{2.566727in}}%
\pgfpathlineto{\pgfqpoint{2.396286in}{2.576697in}}%
\pgfpathlineto{\pgfqpoint{2.405416in}{2.567498in}}%
\pgfpathlineto{\pgfqpoint{2.414519in}{2.558747in}}%
\pgfpathlineto{\pgfqpoint{2.423597in}{2.550436in}}%
\pgfpathlineto{\pgfqpoint{2.432649in}{2.542555in}}%
\pgfpathclose%
\pgfusepath{fill}%
\end{pgfscope}%
\begin{pgfscope}%
\pgfpathrectangle{\pgfqpoint{1.150000in}{0.150000in}}{\pgfqpoint{5.700000in}{5.700000in}}%
\pgfusepath{clip}%
\pgfsetbuttcap%
\pgfsetroundjoin%
\definecolor{currentfill}{rgb}{0.268510,0.009605,0.335427}%
\pgfsetfillcolor{currentfill}%
\pgfsetfillopacity{0.700000}%
\pgfsetlinewidth{0.000000pt}%
\definecolor{currentstroke}{rgb}{0.000000,0.000000,0.000000}%
\pgfsetstrokecolor{currentstroke}%
\pgfsetdash{}{0pt}%
\pgfpathmoveto{\pgfqpoint{3.865933in}{1.932292in}}%
\pgfpathlineto{\pgfqpoint{3.879655in}{1.927680in}}%
\pgfpathlineto{\pgfqpoint{3.893383in}{1.923096in}}%
\pgfpathlineto{\pgfqpoint{3.907117in}{1.918538in}}%
\pgfpathlineto{\pgfqpoint{3.920857in}{1.914008in}}%
\pgfpathlineto{\pgfqpoint{3.912835in}{1.907931in}}%
\pgfpathlineto{\pgfqpoint{3.904807in}{1.901996in}}%
\pgfpathlineto{\pgfqpoint{3.896771in}{1.896211in}}%
\pgfpathlineto{\pgfqpoint{3.888728in}{1.890580in}}%
\pgfpathlineto{\pgfqpoint{3.874971in}{1.895340in}}%
\pgfpathlineto{\pgfqpoint{3.861220in}{1.900126in}}%
\pgfpathlineto{\pgfqpoint{3.847475in}{1.904940in}}%
\pgfpathlineto{\pgfqpoint{3.833736in}{1.909781in}}%
\pgfpathlineto{\pgfqpoint{3.841796in}{1.915177in}}%
\pgfpathlineto{\pgfqpoint{3.849849in}{1.920732in}}%
\pgfpathlineto{\pgfqpoint{3.857895in}{1.926439in}}%
\pgfpathlineto{\pgfqpoint{3.865933in}{1.932292in}}%
\pgfpathclose%
\pgfusepath{fill}%
\end{pgfscope}%
\begin{pgfscope}%
\pgfpathrectangle{\pgfqpoint{1.150000in}{0.150000in}}{\pgfqpoint{5.700000in}{5.700000in}}%
\pgfusepath{clip}%
\pgfsetbuttcap%
\pgfsetroundjoin%
\definecolor{currentfill}{rgb}{0.283197,0.115680,0.436115}%
\pgfsetfillcolor{currentfill}%
\pgfsetfillopacity{0.700000}%
\pgfsetlinewidth{0.000000pt}%
\definecolor{currentstroke}{rgb}{0.000000,0.000000,0.000000}%
\pgfsetstrokecolor{currentstroke}%
\pgfsetdash{}{0pt}%
\pgfpathmoveto{\pgfqpoint{5.096581in}{2.127364in}}%
\pgfpathlineto{\pgfqpoint{5.110634in}{2.126039in}}%
\pgfpathlineto{\pgfqpoint{5.124695in}{2.124740in}}%
\pgfpathlineto{\pgfqpoint{5.138764in}{2.123466in}}%
\pgfpathlineto{\pgfqpoint{5.152841in}{2.122216in}}%
\pgfpathlineto{\pgfqpoint{5.145256in}{2.112835in}}%
\pgfpathlineto{\pgfqpoint{5.137664in}{2.103384in}}%
\pgfpathlineto{\pgfqpoint{5.130065in}{2.093866in}}%
\pgfpathlineto{\pgfqpoint{5.122460in}{2.084281in}}%
\pgfpathlineto{\pgfqpoint{5.108373in}{2.085613in}}%
\pgfpathlineto{\pgfqpoint{5.094294in}{2.086971in}}%
\pgfpathlineto{\pgfqpoint{5.080224in}{2.088353in}}%
\pgfpathlineto{\pgfqpoint{5.066161in}{2.089759in}}%
\pgfpathlineto{\pgfqpoint{5.073776in}{2.099256in}}%
\pgfpathlineto{\pgfqpoint{5.081384in}{2.108691in}}%
\pgfpathlineto{\pgfqpoint{5.088986in}{2.118060in}}%
\pgfpathlineto{\pgfqpoint{5.096581in}{2.127364in}}%
\pgfpathclose%
\pgfusepath{fill}%
\end{pgfscope}%
\begin{pgfscope}%
\pgfpathrectangle{\pgfqpoint{1.150000in}{0.150000in}}{\pgfqpoint{5.700000in}{5.700000in}}%
\pgfusepath{clip}%
\pgfsetbuttcap%
\pgfsetroundjoin%
\definecolor{currentfill}{rgb}{0.271305,0.019942,0.347269}%
\pgfsetfillcolor{currentfill}%
\pgfsetfillopacity{0.700000}%
\pgfsetlinewidth{0.000000pt}%
\definecolor{currentstroke}{rgb}{0.000000,0.000000,0.000000}%
\pgfsetstrokecolor{currentstroke}%
\pgfsetdash{}{0pt}%
\pgfpathmoveto{\pgfqpoint{3.724024in}{1.949505in}}%
\pgfpathlineto{\pgfqpoint{3.737718in}{1.944442in}}%
\pgfpathlineto{\pgfqpoint{3.751419in}{1.939407in}}%
\pgfpathlineto{\pgfqpoint{3.765124in}{1.934399in}}%
\pgfpathlineto{\pgfqpoint{3.778835in}{1.929420in}}%
\pgfpathlineto{\pgfqpoint{3.770749in}{1.924425in}}%
\pgfpathlineto{\pgfqpoint{3.762655in}{1.919603in}}%
\pgfpathlineto{\pgfqpoint{3.754553in}{1.914960in}}%
\pgfpathlineto{\pgfqpoint{3.746442in}{1.910503in}}%
\pgfpathlineto{\pgfqpoint{3.732712in}{1.915725in}}%
\pgfpathlineto{\pgfqpoint{3.718987in}{1.920975in}}%
\pgfpathlineto{\pgfqpoint{3.705268in}{1.926252in}}%
\pgfpathlineto{\pgfqpoint{3.691555in}{1.931558in}}%
\pgfpathlineto{\pgfqpoint{3.699685in}{1.935767in}}%
\pgfpathlineto{\pgfqpoint{3.707806in}{1.940166in}}%
\pgfpathlineto{\pgfqpoint{3.715919in}{1.944747in}}%
\pgfpathlineto{\pgfqpoint{3.724024in}{1.949505in}}%
\pgfpathclose%
\pgfusepath{fill}%
\end{pgfscope}%
\begin{pgfscope}%
\pgfpathrectangle{\pgfqpoint{1.150000in}{0.150000in}}{\pgfqpoint{5.700000in}{5.700000in}}%
\pgfusepath{clip}%
\pgfsetbuttcap%
\pgfsetroundjoin%
\definecolor{currentfill}{rgb}{0.278826,0.175490,0.483397}%
\pgfsetfillcolor{currentfill}%
\pgfsetfillopacity{0.700000}%
\pgfsetlinewidth{0.000000pt}%
\definecolor{currentstroke}{rgb}{0.000000,0.000000,0.000000}%
\pgfsetstrokecolor{currentstroke}%
\pgfsetdash{}{0pt}%
\pgfpathmoveto{\pgfqpoint{5.498998in}{2.248345in}}%
\pgfpathlineto{\pgfqpoint{5.513180in}{2.247710in}}%
\pgfpathlineto{\pgfqpoint{5.527371in}{2.247099in}}%
\pgfpathlineto{\pgfqpoint{5.541570in}{2.246513in}}%
\pgfpathlineto{\pgfqpoint{5.555779in}{2.245952in}}%
\pgfpathlineto{\pgfqpoint{5.548364in}{2.237718in}}%
\pgfpathlineto{\pgfqpoint{5.540941in}{2.229384in}}%
\pgfpathlineto{\pgfqpoint{5.533510in}{2.220949in}}%
\pgfpathlineto{\pgfqpoint{5.526071in}{2.212415in}}%
\pgfpathlineto{\pgfqpoint{5.511851in}{2.213004in}}%
\pgfpathlineto{\pgfqpoint{5.497639in}{2.213617in}}%
\pgfpathlineto{\pgfqpoint{5.483437in}{2.214256in}}%
\pgfpathlineto{\pgfqpoint{5.469243in}{2.214919in}}%
\pgfpathlineto{\pgfqpoint{5.476694in}{2.223421in}}%
\pgfpathlineto{\pgfqpoint{5.484136in}{2.231826in}}%
\pgfpathlineto{\pgfqpoint{5.491571in}{2.240134in}}%
\pgfpathlineto{\pgfqpoint{5.498998in}{2.248345in}}%
\pgfpathclose%
\pgfusepath{fill}%
\end{pgfscope}%
\begin{pgfscope}%
\pgfpathrectangle{\pgfqpoint{1.150000in}{0.150000in}}{\pgfqpoint{5.700000in}{5.700000in}}%
\pgfusepath{clip}%
\pgfsetbuttcap%
\pgfsetroundjoin%
\definecolor{currentfill}{rgb}{0.269308,0.218818,0.509577}%
\pgfsetfillcolor{currentfill}%
\pgfsetfillopacity{0.700000}%
\pgfsetlinewidth{0.000000pt}%
\definecolor{currentstroke}{rgb}{0.000000,0.000000,0.000000}%
\pgfsetstrokecolor{currentstroke}%
\pgfsetdash{}{0pt}%
\pgfpathmoveto{\pgfqpoint{5.814856in}{2.332186in}}%
\pgfpathlineto{\pgfqpoint{5.829144in}{2.331945in}}%
\pgfpathlineto{\pgfqpoint{5.843442in}{2.331729in}}%
\pgfpathlineto{\pgfqpoint{5.857748in}{2.331537in}}%
\pgfpathlineto{\pgfqpoint{5.850487in}{2.324558in}}%
\pgfpathlineto{\pgfqpoint{5.843216in}{2.317471in}}%
\pgfpathlineto{\pgfqpoint{5.835936in}{2.310275in}}%
\pgfpathlineto{\pgfqpoint{5.828647in}{2.302969in}}%
\pgfpathlineto{\pgfqpoint{5.814326in}{2.303146in}}%
\pgfpathlineto{\pgfqpoint{5.800014in}{2.303349in}}%
\pgfpathlineto{\pgfqpoint{5.785712in}{2.303575in}}%
\pgfpathlineto{\pgfqpoint{5.793011in}{2.310888in}}%
\pgfpathlineto{\pgfqpoint{5.800302in}{2.318093in}}%
\pgfpathlineto{\pgfqpoint{5.807583in}{2.325192in}}%
\pgfpathlineto{\pgfqpoint{5.814856in}{2.332186in}}%
\pgfpathclose%
\pgfusepath{fill}%
\end{pgfscope}%
\begin{pgfscope}%
\pgfpathrectangle{\pgfqpoint{1.150000in}{0.150000in}}{\pgfqpoint{5.700000in}{5.700000in}}%
\pgfusepath{clip}%
\pgfsetbuttcap%
\pgfsetroundjoin%
\definecolor{currentfill}{rgb}{0.277941,0.056324,0.381191}%
\pgfsetfillcolor{currentfill}%
\pgfsetfillopacity{0.700000}%
\pgfsetlinewidth{0.000000pt}%
\definecolor{currentstroke}{rgb}{0.000000,0.000000,0.000000}%
\pgfsetstrokecolor{currentstroke}%
\pgfsetdash{}{0pt}%
\pgfpathmoveto{\pgfqpoint{4.694202in}{2.011601in}}%
\pgfpathlineto{\pgfqpoint{4.708133in}{2.009366in}}%
\pgfpathlineto{\pgfqpoint{4.722073in}{2.007156in}}%
\pgfpathlineto{\pgfqpoint{4.736020in}{2.004972in}}%
\pgfpathlineto{\pgfqpoint{4.749975in}{2.002812in}}%
\pgfpathlineto{\pgfqpoint{4.742247in}{1.993225in}}%
\pgfpathlineto{\pgfqpoint{4.734513in}{1.983622in}}%
\pgfpathlineto{\pgfqpoint{4.726774in}{1.974006in}}%
\pgfpathlineto{\pgfqpoint{4.719030in}{1.964382in}}%
\pgfpathlineto{\pgfqpoint{4.705066in}{1.966678in}}%
\pgfpathlineto{\pgfqpoint{4.691109in}{1.968999in}}%
\pgfpathlineto{\pgfqpoint{4.677160in}{1.971345in}}%
\pgfpathlineto{\pgfqpoint{4.663218in}{1.973716in}}%
\pgfpathlineto{\pgfqpoint{4.670972in}{1.983199in}}%
\pgfpathlineto{\pgfqpoint{4.678721in}{1.992676in}}%
\pgfpathlineto{\pgfqpoint{4.686464in}{2.002145in}}%
\pgfpathlineto{\pgfqpoint{4.694202in}{2.011601in}}%
\pgfpathclose%
\pgfusepath{fill}%
\end{pgfscope}%
\begin{pgfscope}%
\pgfpathrectangle{\pgfqpoint{1.150000in}{0.150000in}}{\pgfqpoint{5.700000in}{5.700000in}}%
\pgfusepath{clip}%
\pgfsetbuttcap%
\pgfsetroundjoin%
\definecolor{currentfill}{rgb}{0.267004,0.004874,0.329415}%
\pgfsetfillcolor{currentfill}%
\pgfsetfillopacity{0.700000}%
\pgfsetlinewidth{0.000000pt}%
\definecolor{currentstroke}{rgb}{0.000000,0.000000,0.000000}%
\pgfsetstrokecolor{currentstroke}%
\pgfsetdash{}{0pt}%
\pgfpathmoveto{\pgfqpoint{4.007831in}{1.922649in}}%
\pgfpathlineto{\pgfqpoint{4.021585in}{1.918469in}}%
\pgfpathlineto{\pgfqpoint{4.035346in}{1.914316in}}%
\pgfpathlineto{\pgfqpoint{4.049113in}{1.910189in}}%
\pgfpathlineto{\pgfqpoint{4.062886in}{1.906089in}}%
\pgfpathlineto{\pgfqpoint{4.054921in}{1.899074in}}%
\pgfpathlineto{\pgfqpoint{4.046951in}{1.892172in}}%
\pgfpathlineto{\pgfqpoint{4.038974in}{1.885391in}}%
\pgfpathlineto{\pgfqpoint{4.030991in}{1.878735in}}%
\pgfpathlineto{\pgfqpoint{4.017203in}{1.883051in}}%
\pgfpathlineto{\pgfqpoint{4.003421in}{1.887393in}}%
\pgfpathlineto{\pgfqpoint{3.989645in}{1.891762in}}%
\pgfpathlineto{\pgfqpoint{3.975876in}{1.896157in}}%
\pgfpathlineto{\pgfqpoint{3.983874in}{1.902593in}}%
\pgfpathlineto{\pgfqpoint{3.991867in}{1.909157in}}%
\pgfpathlineto{\pgfqpoint{3.999852in}{1.915844in}}%
\pgfpathlineto{\pgfqpoint{4.007831in}{1.922649in}}%
\pgfpathclose%
\pgfusepath{fill}%
\end{pgfscope}%
\begin{pgfscope}%
\pgfpathrectangle{\pgfqpoint{1.150000in}{0.150000in}}{\pgfqpoint{5.700000in}{5.700000in}}%
\pgfusepath{clip}%
\pgfsetbuttcap%
\pgfsetroundjoin%
\definecolor{currentfill}{rgb}{0.274952,0.037752,0.364543}%
\pgfsetfillcolor{currentfill}%
\pgfsetfillopacity{0.700000}%
\pgfsetlinewidth{0.000000pt}%
\definecolor{currentstroke}{rgb}{0.000000,0.000000,0.000000}%
\pgfsetstrokecolor{currentstroke}%
\pgfsetdash{}{0pt}%
\pgfpathmoveto{\pgfqpoint{3.582035in}{1.975031in}}%
\pgfpathlineto{\pgfqpoint{3.595707in}{1.969496in}}%
\pgfpathlineto{\pgfqpoint{3.609384in}{1.963990in}}%
\pgfpathlineto{\pgfqpoint{3.623066in}{1.958513in}}%
\pgfpathlineto{\pgfqpoint{3.636753in}{1.953065in}}%
\pgfpathlineto{\pgfqpoint{3.628594in}{1.949302in}}%
\pgfpathlineto{\pgfqpoint{3.620426in}{1.945744in}}%
\pgfpathlineto{\pgfqpoint{3.612248in}{1.942399in}}%
\pgfpathlineto{\pgfqpoint{3.604061in}{1.939272in}}%
\pgfpathlineto{\pgfqpoint{3.590352in}{1.944976in}}%
\pgfpathlineto{\pgfqpoint{3.576649in}{1.950709in}}%
\pgfpathlineto{\pgfqpoint{3.562951in}{1.956471in}}%
\pgfpathlineto{\pgfqpoint{3.549258in}{1.962262in}}%
\pgfpathlineto{\pgfqpoint{3.557467in}{1.965128in}}%
\pgfpathlineto{\pgfqpoint{3.565666in}{1.968216in}}%
\pgfpathlineto{\pgfqpoint{3.573855in}{1.971519in}}%
\pgfpathlineto{\pgfqpoint{3.582035in}{1.975031in}}%
\pgfpathclose%
\pgfusepath{fill}%
\end{pgfscope}%
\begin{pgfscope}%
\pgfpathrectangle{\pgfqpoint{1.150000in}{0.150000in}}{\pgfqpoint{5.700000in}{5.700000in}}%
\pgfusepath{clip}%
\pgfsetbuttcap%
\pgfsetroundjoin%
\definecolor{currentfill}{rgb}{0.269944,0.014625,0.341379}%
\pgfsetfillcolor{currentfill}%
\pgfsetfillopacity{0.700000}%
\pgfsetlinewidth{0.000000pt}%
\definecolor{currentstroke}{rgb}{0.000000,0.000000,0.000000}%
\pgfsetstrokecolor{currentstroke}%
\pgfsetdash{}{0pt}%
\pgfpathmoveto{\pgfqpoint{4.378601in}{1.944437in}}%
\pgfpathlineto{\pgfqpoint{4.392448in}{1.941347in}}%
\pgfpathlineto{\pgfqpoint{4.406302in}{1.938283in}}%
\pgfpathlineto{\pgfqpoint{4.420162in}{1.935245in}}%
\pgfpathlineto{\pgfqpoint{4.434030in}{1.932232in}}%
\pgfpathlineto{\pgfqpoint{4.426197in}{1.923370in}}%
\pgfpathlineto{\pgfqpoint{4.418360in}{1.914549in}}%
\pgfpathlineto{\pgfqpoint{4.410517in}{1.905773in}}%
\pgfpathlineto{\pgfqpoint{4.402668in}{1.897047in}}%
\pgfpathlineto{\pgfqpoint{4.388789in}{1.900236in}}%
\pgfpathlineto{\pgfqpoint{4.374917in}{1.903451in}}%
\pgfpathlineto{\pgfqpoint{4.361052in}{1.906691in}}%
\pgfpathlineto{\pgfqpoint{4.347194in}{1.909956in}}%
\pgfpathlineto{\pgfqpoint{4.355054in}{1.918501in}}%
\pgfpathlineto{\pgfqpoint{4.362908in}{1.927099in}}%
\pgfpathlineto{\pgfqpoint{4.370758in}{1.935746in}}%
\pgfpathlineto{\pgfqpoint{4.378601in}{1.944437in}}%
\pgfpathclose%
\pgfusepath{fill}%
\end{pgfscope}%
\begin{pgfscope}%
\pgfpathrectangle{\pgfqpoint{1.150000in}{0.150000in}}{\pgfqpoint{5.700000in}{5.700000in}}%
\pgfusepath{clip}%
\pgfsetbuttcap%
\pgfsetroundjoin%
\definecolor{currentfill}{rgb}{0.282656,0.100196,0.422160}%
\pgfsetfillcolor{currentfill}%
\pgfsetfillopacity{0.700000}%
\pgfsetlinewidth{0.000000pt}%
\definecolor{currentstroke}{rgb}{0.000000,0.000000,0.000000}%
\pgfsetstrokecolor{currentstroke}%
\pgfsetdash{}{0pt}%
\pgfpathmoveto{\pgfqpoint{5.009995in}{2.095635in}}%
\pgfpathlineto{\pgfqpoint{5.024024in}{2.094129in}}%
\pgfpathlineto{\pgfqpoint{5.038062in}{2.092648in}}%
\pgfpathlineto{\pgfqpoint{5.052108in}{2.091191in}}%
\pgfpathlineto{\pgfqpoint{5.066161in}{2.089759in}}%
\pgfpathlineto{\pgfqpoint{5.058541in}{2.080201in}}%
\pgfpathlineto{\pgfqpoint{5.050914in}{2.070584in}}%
\pgfpathlineto{\pgfqpoint{5.043280in}{2.060910in}}%
\pgfpathlineto{\pgfqpoint{5.035641in}{2.051181in}}%
\pgfpathlineto{\pgfqpoint{5.021578in}{2.052709in}}%
\pgfpathlineto{\pgfqpoint{5.007523in}{2.054262in}}%
\pgfpathlineto{\pgfqpoint{4.993476in}{2.055839in}}%
\pgfpathlineto{\pgfqpoint{4.979437in}{2.057442in}}%
\pgfpathlineto{\pgfqpoint{4.987086in}{2.067070in}}%
\pgfpathlineto{\pgfqpoint{4.994728in}{2.076646in}}%
\pgfpathlineto{\pgfqpoint{5.002365in}{2.086168in}}%
\pgfpathlineto{\pgfqpoint{5.009995in}{2.095635in}}%
\pgfpathclose%
\pgfusepath{fill}%
\end{pgfscope}%
\begin{pgfscope}%
\pgfpathrectangle{\pgfqpoint{1.150000in}{0.150000in}}{\pgfqpoint{5.700000in}{5.700000in}}%
\pgfusepath{clip}%
\pgfsetbuttcap%
\pgfsetroundjoin%
\definecolor{currentfill}{rgb}{0.280255,0.165693,0.476498}%
\pgfsetfillcolor{currentfill}%
\pgfsetfillopacity{0.700000}%
\pgfsetlinewidth{0.000000pt}%
\definecolor{currentstroke}{rgb}{0.000000,0.000000,0.000000}%
\pgfsetstrokecolor{currentstroke}%
\pgfsetdash{}{0pt}%
\pgfpathmoveto{\pgfqpoint{5.412558in}{2.217818in}}%
\pgfpathlineto{\pgfqpoint{5.426716in}{2.217057in}}%
\pgfpathlineto{\pgfqpoint{5.440883in}{2.216319in}}%
\pgfpathlineto{\pgfqpoint{5.455059in}{2.215607in}}%
\pgfpathlineto{\pgfqpoint{5.469243in}{2.214919in}}%
\pgfpathlineto{\pgfqpoint{5.461785in}{2.206321in}}%
\pgfpathlineto{\pgfqpoint{5.454319in}{2.197628in}}%
\pgfpathlineto{\pgfqpoint{5.446845in}{2.188839in}}%
\pgfpathlineto{\pgfqpoint{5.439364in}{2.179955in}}%
\pgfpathlineto{\pgfqpoint{5.425168in}{2.180685in}}%
\pgfpathlineto{\pgfqpoint{5.410982in}{2.181439in}}%
\pgfpathlineto{\pgfqpoint{5.396804in}{2.182218in}}%
\pgfpathlineto{\pgfqpoint{5.382635in}{2.183021in}}%
\pgfpathlineto{\pgfqpoint{5.390127in}{2.191858in}}%
\pgfpathlineto{\pgfqpoint{5.397612in}{2.200603in}}%
\pgfpathlineto{\pgfqpoint{5.405089in}{2.209257in}}%
\pgfpathlineto{\pgfqpoint{5.412558in}{2.217818in}}%
\pgfpathclose%
\pgfusepath{fill}%
\end{pgfscope}%
\begin{pgfscope}%
\pgfpathrectangle{\pgfqpoint{1.150000in}{0.150000in}}{\pgfqpoint{5.700000in}{5.700000in}}%
\pgfusepath{clip}%
\pgfsetbuttcap%
\pgfsetroundjoin%
\definecolor{currentfill}{rgb}{0.267004,0.004874,0.329415}%
\pgfsetfillcolor{currentfill}%
\pgfsetfillopacity{0.700000}%
\pgfsetlinewidth{0.000000pt}%
\definecolor{currentstroke}{rgb}{0.000000,0.000000,0.000000}%
\pgfsetstrokecolor{currentstroke}%
\pgfsetdash{}{0pt}%
\pgfpathmoveto{\pgfqpoint{4.149779in}{1.919867in}}%
\pgfpathlineto{\pgfqpoint{4.163570in}{1.916101in}}%
\pgfpathlineto{\pgfqpoint{4.177367in}{1.912362in}}%
\pgfpathlineto{\pgfqpoint{4.191170in}{1.908649in}}%
\pgfpathlineto{\pgfqpoint{4.204981in}{1.904961in}}%
\pgfpathlineto{\pgfqpoint{4.197068in}{1.897145in}}%
\pgfpathlineto{\pgfqpoint{4.189150in}{1.889415in}}%
\pgfpathlineto{\pgfqpoint{4.181226in}{1.881778in}}%
\pgfpathlineto{\pgfqpoint{4.173296in}{1.874238in}}%
\pgfpathlineto{\pgfqpoint{4.159472in}{1.878128in}}%
\pgfpathlineto{\pgfqpoint{4.145655in}{1.882044in}}%
\pgfpathlineto{\pgfqpoint{4.131844in}{1.885986in}}%
\pgfpathlineto{\pgfqpoint{4.118040in}{1.889954in}}%
\pgfpathlineto{\pgfqpoint{4.125984in}{1.897286in}}%
\pgfpathlineto{\pgfqpoint{4.133921in}{1.904719in}}%
\pgfpathlineto{\pgfqpoint{4.141853in}{1.912248in}}%
\pgfpathlineto{\pgfqpoint{4.149779in}{1.919867in}}%
\pgfpathclose%
\pgfusepath{fill}%
\end{pgfscope}%
\begin{pgfscope}%
\pgfpathrectangle{\pgfqpoint{1.150000in}{0.150000in}}{\pgfqpoint{5.700000in}{5.700000in}}%
\pgfusepath{clip}%
\pgfsetbuttcap%
\pgfsetroundjoin%
\definecolor{currentfill}{rgb}{0.276022,0.044167,0.370164}%
\pgfsetfillcolor{currentfill}%
\pgfsetfillopacity{0.700000}%
\pgfsetlinewidth{0.000000pt}%
\definecolor{currentstroke}{rgb}{0.000000,0.000000,0.000000}%
\pgfsetstrokecolor{currentstroke}%
\pgfsetdash{}{0pt}%
\pgfpathmoveto{\pgfqpoint{4.607527in}{1.983453in}}%
\pgfpathlineto{\pgfqpoint{4.621439in}{1.980981in}}%
\pgfpathlineto{\pgfqpoint{4.635358in}{1.978534in}}%
\pgfpathlineto{\pgfqpoint{4.649284in}{1.976113in}}%
\pgfpathlineto{\pgfqpoint{4.663218in}{1.973716in}}%
\pgfpathlineto{\pgfqpoint{4.655459in}{1.964231in}}%
\pgfpathlineto{\pgfqpoint{4.647695in}{1.954746in}}%
\pgfpathlineto{\pgfqpoint{4.639925in}{1.945266in}}%
\pgfpathlineto{\pgfqpoint{4.632150in}{1.935793in}}%
\pgfpathlineto{\pgfqpoint{4.618206in}{1.938339in}}%
\pgfpathlineto{\pgfqpoint{4.604270in}{1.940911in}}%
\pgfpathlineto{\pgfqpoint{4.590341in}{1.943507in}}%
\pgfpathlineto{\pgfqpoint{4.576419in}{1.946129in}}%
\pgfpathlineto{\pgfqpoint{4.584204in}{1.955447in}}%
\pgfpathlineto{\pgfqpoint{4.591984in}{1.964775in}}%
\pgfpathlineto{\pgfqpoint{4.599758in}{1.974112in}}%
\pgfpathlineto{\pgfqpoint{4.607527in}{1.983453in}}%
\pgfpathclose%
\pgfusepath{fill}%
\end{pgfscope}%
\begin{pgfscope}%
\pgfpathrectangle{\pgfqpoint{1.150000in}{0.150000in}}{\pgfqpoint{5.700000in}{5.700000in}}%
\pgfusepath{clip}%
\pgfsetbuttcap%
\pgfsetroundjoin%
\definecolor{currentfill}{rgb}{0.282656,0.100196,0.422160}%
\pgfsetfillcolor{currentfill}%
\pgfsetfillopacity{0.700000}%
\pgfsetlinewidth{0.000000pt}%
\definecolor{currentstroke}{rgb}{0.000000,0.000000,0.000000}%
\pgfsetstrokecolor{currentstroke}%
\pgfsetdash{}{0pt}%
\pgfpathmoveto{\pgfqpoint{3.242976in}{2.080711in}}%
\pgfpathlineto{\pgfqpoint{3.256601in}{2.074037in}}%
\pgfpathlineto{\pgfqpoint{3.270230in}{2.067395in}}%
\pgfpathlineto{\pgfqpoint{3.283863in}{2.060785in}}%
\pgfpathlineto{\pgfqpoint{3.297501in}{2.054207in}}%
\pgfpathlineto{\pgfqpoint{3.289139in}{2.053670in}}%
\pgfpathlineto{\pgfqpoint{3.280764in}{2.053412in}}%
\pgfpathlineto{\pgfqpoint{3.272376in}{2.053440in}}%
\pgfpathlineto{\pgfqpoint{3.263974in}{2.053761in}}%
\pgfpathlineto{\pgfqpoint{3.250309in}{2.060623in}}%
\pgfpathlineto{\pgfqpoint{3.236649in}{2.067518in}}%
\pgfpathlineto{\pgfqpoint{3.222993in}{2.074444in}}%
\pgfpathlineto{\pgfqpoint{3.209341in}{2.081403in}}%
\pgfpathlineto{\pgfqpoint{3.217770in}{2.080792in}}%
\pgfpathlineto{\pgfqpoint{3.226185in}{2.080478in}}%
\pgfpathlineto{\pgfqpoint{3.234587in}{2.080454in}}%
\pgfpathlineto{\pgfqpoint{3.242976in}{2.080711in}}%
\pgfpathclose%
\pgfusepath{fill}%
\end{pgfscope}%
\begin{pgfscope}%
\pgfpathrectangle{\pgfqpoint{1.150000in}{0.150000in}}{\pgfqpoint{5.700000in}{5.700000in}}%
\pgfusepath{clip}%
\pgfsetbuttcap%
\pgfsetroundjoin%
\definecolor{currentfill}{rgb}{0.237441,0.305202,0.541921}%
\pgfsetfillcolor{currentfill}%
\pgfsetfillopacity{0.700000}%
\pgfsetlinewidth{0.000000pt}%
\definecolor{currentstroke}{rgb}{0.000000,0.000000,0.000000}%
\pgfsetstrokecolor{currentstroke}%
\pgfsetdash{}{0pt}%
\pgfpathmoveto{\pgfqpoint{2.486928in}{2.504312in}}%
\pgfpathlineto{\pgfqpoint{2.500502in}{2.494872in}}%
\pgfpathlineto{\pgfqpoint{2.514078in}{2.485478in}}%
\pgfpathlineto{\pgfqpoint{2.527656in}{2.476132in}}%
\pgfpathlineto{\pgfqpoint{2.541236in}{2.466832in}}%
\pgfpathlineto{\pgfqpoint{2.532265in}{2.474056in}}%
\pgfpathlineto{\pgfqpoint{2.523271in}{2.481701in}}%
\pgfpathlineto{\pgfqpoint{2.514252in}{2.489777in}}%
\pgfpathlineto{\pgfqpoint{2.505209in}{2.498295in}}%
\pgfpathlineto{\pgfqpoint{2.491587in}{2.507929in}}%
\pgfpathlineto{\pgfqpoint{2.477968in}{2.517609in}}%
\pgfpathlineto{\pgfqpoint{2.464350in}{2.527337in}}%
\pgfpathlineto{\pgfqpoint{2.450734in}{2.537112in}}%
\pgfpathlineto{\pgfqpoint{2.459821in}{2.528255in}}%
\pgfpathlineto{\pgfqpoint{2.468881in}{2.519841in}}%
\pgfpathlineto{\pgfqpoint{2.477917in}{2.511864in}}%
\pgfpathlineto{\pgfqpoint{2.486928in}{2.504312in}}%
\pgfpathclose%
\pgfusepath{fill}%
\end{pgfscope}%
\begin{pgfscope}%
\pgfpathrectangle{\pgfqpoint{1.150000in}{0.150000in}}{\pgfqpoint{5.700000in}{5.700000in}}%
\pgfusepath{clip}%
\pgfsetbuttcap%
\pgfsetroundjoin%
\definecolor{currentfill}{rgb}{0.270595,0.214069,0.507052}%
\pgfsetfillcolor{currentfill}%
\pgfsetfillopacity{0.700000}%
\pgfsetlinewidth{0.000000pt}%
\definecolor{currentstroke}{rgb}{0.000000,0.000000,0.000000}%
\pgfsetstrokecolor{currentstroke}%
\pgfsetdash{}{0pt}%
\pgfpathmoveto{\pgfqpoint{2.793871in}{2.304224in}}%
\pgfpathlineto{\pgfqpoint{2.807456in}{2.295967in}}%
\pgfpathlineto{\pgfqpoint{2.821043in}{2.287749in}}%
\pgfpathlineto{\pgfqpoint{2.834634in}{2.279570in}}%
\pgfpathlineto{\pgfqpoint{2.848227in}{2.271430in}}%
\pgfpathlineto{\pgfqpoint{2.839528in}{2.275498in}}%
\pgfpathlineto{\pgfqpoint{2.830811in}{2.279934in}}%
\pgfpathlineto{\pgfqpoint{2.822074in}{2.284746in}}%
\pgfpathlineto{\pgfqpoint{2.813316in}{2.289942in}}%
\pgfpathlineto{\pgfqpoint{2.799688in}{2.298398in}}%
\pgfpathlineto{\pgfqpoint{2.786062in}{2.306893in}}%
\pgfpathlineto{\pgfqpoint{2.772438in}{2.315427in}}%
\pgfpathlineto{\pgfqpoint{2.758818in}{2.324000in}}%
\pgfpathlineto{\pgfqpoint{2.767612in}{2.318482in}}%
\pgfpathlineto{\pgfqpoint{2.776385in}{2.313353in}}%
\pgfpathlineto{\pgfqpoint{2.785138in}{2.308603in}}%
\pgfpathlineto{\pgfqpoint{2.793871in}{2.304224in}}%
\pgfpathclose%
\pgfusepath{fill}%
\end{pgfscope}%
\begin{pgfscope}%
\pgfpathrectangle{\pgfqpoint{1.150000in}{0.150000in}}{\pgfqpoint{5.700000in}{5.700000in}}%
\pgfusepath{clip}%
\pgfsetbuttcap%
\pgfsetroundjoin%
\definecolor{currentfill}{rgb}{0.281924,0.089666,0.412415}%
\pgfsetfillcolor{currentfill}%
\pgfsetfillopacity{0.700000}%
\pgfsetlinewidth{0.000000pt}%
\definecolor{currentstroke}{rgb}{0.000000,0.000000,0.000000}%
\pgfsetstrokecolor{currentstroke}%
\pgfsetdash{}{0pt}%
\pgfpathmoveto{\pgfqpoint{4.923364in}{2.064100in}}%
\pgfpathlineto{\pgfqpoint{4.937370in}{2.062398in}}%
\pgfpathlineto{\pgfqpoint{4.951384in}{2.060721in}}%
\pgfpathlineto{\pgfqpoint{4.965407in}{2.059069in}}%
\pgfpathlineto{\pgfqpoint{4.979437in}{2.057442in}}%
\pgfpathlineto{\pgfqpoint{4.971783in}{2.047765in}}%
\pgfpathlineto{\pgfqpoint{4.964122in}{2.038040in}}%
\pgfpathlineto{\pgfqpoint{4.956456in}{2.028271in}}%
\pgfpathlineto{\pgfqpoint{4.948784in}{2.018460in}}%
\pgfpathlineto{\pgfqpoint{4.934744in}{2.020197in}}%
\pgfpathlineto{\pgfqpoint{4.920713in}{2.021959in}}%
\pgfpathlineto{\pgfqpoint{4.906689in}{2.023746in}}%
\pgfpathlineto{\pgfqpoint{4.892673in}{2.025558in}}%
\pgfpathlineto{\pgfqpoint{4.900354in}{2.035254in}}%
\pgfpathlineto{\pgfqpoint{4.908030in}{2.044912in}}%
\pgfpathlineto{\pgfqpoint{4.915700in}{2.054528in}}%
\pgfpathlineto{\pgfqpoint{4.923364in}{2.064100in}}%
\pgfpathclose%
\pgfusepath{fill}%
\end{pgfscope}%
\begin{pgfscope}%
\pgfpathrectangle{\pgfqpoint{1.150000in}{0.150000in}}{\pgfqpoint{5.700000in}{5.700000in}}%
\pgfusepath{clip}%
\pgfsetbuttcap%
\pgfsetroundjoin%
\definecolor{currentfill}{rgb}{0.271828,0.209303,0.504434}%
\pgfsetfillcolor{currentfill}%
\pgfsetfillopacity{0.700000}%
\pgfsetlinewidth{0.000000pt}%
\definecolor{currentstroke}{rgb}{0.000000,0.000000,0.000000}%
\pgfsetstrokecolor{currentstroke}%
\pgfsetdash{}{0pt}%
\pgfpathmoveto{\pgfqpoint{5.728595in}{2.304729in}}%
\pgfpathlineto{\pgfqpoint{5.742861in}{2.304404in}}%
\pgfpathlineto{\pgfqpoint{5.757135in}{2.304103in}}%
\pgfpathlineto{\pgfqpoint{5.771419in}{2.303827in}}%
\pgfpathlineto{\pgfqpoint{5.785712in}{2.303575in}}%
\pgfpathlineto{\pgfqpoint{5.778403in}{2.296155in}}%
\pgfpathlineto{\pgfqpoint{5.771086in}{2.288626in}}%
\pgfpathlineto{\pgfqpoint{5.763760in}{2.280987in}}%
\pgfpathlineto{\pgfqpoint{5.756424in}{2.273239in}}%
\pgfpathlineto{\pgfqpoint{5.742118in}{2.273490in}}%
\pgfpathlineto{\pgfqpoint{5.727821in}{2.273766in}}%
\pgfpathlineto{\pgfqpoint{5.713533in}{2.274067in}}%
\pgfpathlineto{\pgfqpoint{5.699255in}{2.274392in}}%
\pgfpathlineto{\pgfqpoint{5.706603in}{2.282136in}}%
\pgfpathlineto{\pgfqpoint{5.713943in}{2.289772in}}%
\pgfpathlineto{\pgfqpoint{5.721273in}{2.297303in}}%
\pgfpathlineto{\pgfqpoint{5.728595in}{2.304729in}}%
\pgfpathclose%
\pgfusepath{fill}%
\end{pgfscope}%
\begin{pgfscope}%
\pgfpathrectangle{\pgfqpoint{1.150000in}{0.150000in}}{\pgfqpoint{5.700000in}{5.700000in}}%
\pgfusepath{clip}%
\pgfsetbuttcap%
\pgfsetroundjoin%
\definecolor{currentfill}{rgb}{0.278791,0.062145,0.386592}%
\pgfsetfillcolor{currentfill}%
\pgfsetfillopacity{0.700000}%
\pgfsetlinewidth{0.000000pt}%
\definecolor{currentstroke}{rgb}{0.000000,0.000000,0.000000}%
\pgfsetstrokecolor{currentstroke}%
\pgfsetdash{}{0pt}%
\pgfpathmoveto{\pgfqpoint{3.439889in}{2.009655in}}%
\pgfpathlineto{\pgfqpoint{3.453544in}{2.003626in}}%
\pgfpathlineto{\pgfqpoint{3.467202in}{1.997627in}}%
\pgfpathlineto{\pgfqpoint{3.480866in}{1.991659in}}%
\pgfpathlineto{\pgfqpoint{3.494535in}{1.985720in}}%
\pgfpathlineto{\pgfqpoint{3.486293in}{1.983348in}}%
\pgfpathlineto{\pgfqpoint{3.478040in}{1.981215in}}%
\pgfpathlineto{\pgfqpoint{3.469777in}{1.979328in}}%
\pgfpathlineto{\pgfqpoint{3.461502in}{1.977695in}}%
\pgfpathlineto{\pgfqpoint{3.447810in}{1.983903in}}%
\pgfpathlineto{\pgfqpoint{3.434122in}{1.990142in}}%
\pgfpathlineto{\pgfqpoint{3.420440in}{1.996410in}}%
\pgfpathlineto{\pgfqpoint{3.406762in}{2.002709in}}%
\pgfpathlineto{\pgfqpoint{3.415061in}{2.004067in}}%
\pgfpathlineto{\pgfqpoint{3.423348in}{2.005682in}}%
\pgfpathlineto{\pgfqpoint{3.431624in}{2.007547in}}%
\pgfpathlineto{\pgfqpoint{3.439889in}{2.009655in}}%
\pgfpathclose%
\pgfusepath{fill}%
\end{pgfscope}%
\begin{pgfscope}%
\pgfpathrectangle{\pgfqpoint{1.150000in}{0.150000in}}{\pgfqpoint{5.700000in}{5.700000in}}%
\pgfusepath{clip}%
\pgfsetbuttcap%
\pgfsetroundjoin%
\definecolor{currentfill}{rgb}{0.281887,0.150881,0.465405}%
\pgfsetfillcolor{currentfill}%
\pgfsetfillopacity{0.700000}%
\pgfsetlinewidth{0.000000pt}%
\definecolor{currentstroke}{rgb}{0.000000,0.000000,0.000000}%
\pgfsetstrokecolor{currentstroke}%
\pgfsetdash{}{0pt}%
\pgfpathmoveto{\pgfqpoint{5.326047in}{2.186483in}}%
\pgfpathlineto{\pgfqpoint{5.340181in}{2.185580in}}%
\pgfpathlineto{\pgfqpoint{5.354323in}{2.184703in}}%
\pgfpathlineto{\pgfqpoint{5.368475in}{2.183850in}}%
\pgfpathlineto{\pgfqpoint{5.382635in}{2.183021in}}%
\pgfpathlineto{\pgfqpoint{5.375136in}{2.174094in}}%
\pgfpathlineto{\pgfqpoint{5.367629in}{2.165077in}}%
\pgfpathlineto{\pgfqpoint{5.360115in}{2.155972in}}%
\pgfpathlineto{\pgfqpoint{5.352593in}{2.146778in}}%
\pgfpathlineto{\pgfqpoint{5.338423in}{2.147662in}}%
\pgfpathlineto{\pgfqpoint{5.324261in}{2.148570in}}%
\pgfpathlineto{\pgfqpoint{5.310108in}{2.149503in}}%
\pgfpathlineto{\pgfqpoint{5.295963in}{2.150461in}}%
\pgfpathlineto{\pgfqpoint{5.303495in}{2.159594in}}%
\pgfpathlineto{\pgfqpoint{5.311019in}{2.168643in}}%
\pgfpathlineto{\pgfqpoint{5.318537in}{2.177606in}}%
\pgfpathlineto{\pgfqpoint{5.326047in}{2.186483in}}%
\pgfpathclose%
\pgfusepath{fill}%
\end{pgfscope}%
\begin{pgfscope}%
\pgfpathrectangle{\pgfqpoint{1.150000in}{0.150000in}}{\pgfqpoint{5.700000in}{5.700000in}}%
\pgfusepath{clip}%
\pgfsetbuttcap%
\pgfsetroundjoin%
\definecolor{currentfill}{rgb}{0.282290,0.145912,0.461510}%
\pgfsetfillcolor{currentfill}%
\pgfsetfillopacity{0.700000}%
\pgfsetlinewidth{0.000000pt}%
\definecolor{currentstroke}{rgb}{0.000000,0.000000,0.000000}%
\pgfsetstrokecolor{currentstroke}%
\pgfsetdash{}{0pt}%
\pgfpathmoveto{\pgfqpoint{3.045825in}{2.167503in}}%
\pgfpathlineto{\pgfqpoint{3.059430in}{2.160140in}}%
\pgfpathlineto{\pgfqpoint{3.073039in}{2.152812in}}%
\pgfpathlineto{\pgfqpoint{3.086652in}{2.145519in}}%
\pgfpathlineto{\pgfqpoint{3.100268in}{2.138261in}}%
\pgfpathlineto{\pgfqpoint{3.091766in}{2.139764in}}%
\pgfpathlineto{\pgfqpoint{3.083249in}{2.141587in}}%
\pgfpathlineto{\pgfqpoint{3.074715in}{2.143738in}}%
\pgfpathlineto{\pgfqpoint{3.066166in}{2.146225in}}%
\pgfpathlineto{\pgfqpoint{3.052519in}{2.153783in}}%
\pgfpathlineto{\pgfqpoint{3.038875in}{2.161376in}}%
\pgfpathlineto{\pgfqpoint{3.025235in}{2.169003in}}%
\pgfpathlineto{\pgfqpoint{3.011599in}{2.176665in}}%
\pgfpathlineto{\pgfqpoint{3.020180in}{2.173873in}}%
\pgfpathlineto{\pgfqpoint{3.028744in}{2.171421in}}%
\pgfpathlineto{\pgfqpoint{3.037293in}{2.169300in}}%
\pgfpathlineto{\pgfqpoint{3.045825in}{2.167503in}}%
\pgfpathclose%
\pgfusepath{fill}%
\end{pgfscope}%
\begin{pgfscope}%
\pgfpathrectangle{\pgfqpoint{1.150000in}{0.150000in}}{\pgfqpoint{5.700000in}{5.700000in}}%
\pgfusepath{clip}%
\pgfsetbuttcap%
\pgfsetroundjoin%
\definecolor{currentfill}{rgb}{0.268510,0.009605,0.335427}%
\pgfsetfillcolor{currentfill}%
\pgfsetfillopacity{0.700000}%
\pgfsetlinewidth{0.000000pt}%
\definecolor{currentstroke}{rgb}{0.000000,0.000000,0.000000}%
\pgfsetstrokecolor{currentstroke}%
\pgfsetdash{}{0pt}%
\pgfpathmoveto{\pgfqpoint{4.291830in}{1.923275in}}%
\pgfpathlineto{\pgfqpoint{4.305661in}{1.919907in}}%
\pgfpathlineto{\pgfqpoint{4.319498in}{1.916564in}}%
\pgfpathlineto{\pgfqpoint{4.333343in}{1.913247in}}%
\pgfpathlineto{\pgfqpoint{4.347194in}{1.909956in}}%
\pgfpathlineto{\pgfqpoint{4.339329in}{1.901469in}}%
\pgfpathlineto{\pgfqpoint{4.331458in}{1.893043in}}%
\pgfpathlineto{\pgfqpoint{4.323582in}{1.884684in}}%
\pgfpathlineto{\pgfqpoint{4.315700in}{1.876396in}}%
\pgfpathlineto{\pgfqpoint{4.301837in}{1.879876in}}%
\pgfpathlineto{\pgfqpoint{4.287980in}{1.883383in}}%
\pgfpathlineto{\pgfqpoint{4.274130in}{1.886915in}}%
\pgfpathlineto{\pgfqpoint{4.260287in}{1.890472in}}%
\pgfpathlineto{\pgfqpoint{4.268181in}{1.898566in}}%
\pgfpathlineto{\pgfqpoint{4.276070in}{1.906734in}}%
\pgfpathlineto{\pgfqpoint{4.283952in}{1.914972in}}%
\pgfpathlineto{\pgfqpoint{4.291830in}{1.923275in}}%
\pgfpathclose%
\pgfusepath{fill}%
\end{pgfscope}%
\begin{pgfscope}%
\pgfpathrectangle{\pgfqpoint{1.150000in}{0.150000in}}{\pgfqpoint{5.700000in}{5.700000in}}%
\pgfusepath{clip}%
\pgfsetbuttcap%
\pgfsetroundjoin%
\definecolor{currentfill}{rgb}{0.273809,0.031497,0.358853}%
\pgfsetfillcolor{currentfill}%
\pgfsetfillopacity{0.700000}%
\pgfsetlinewidth{0.000000pt}%
\definecolor{currentstroke}{rgb}{0.000000,0.000000,0.000000}%
\pgfsetstrokecolor{currentstroke}%
\pgfsetdash{}{0pt}%
\pgfpathmoveto{\pgfqpoint{4.520805in}{1.956868in}}%
\pgfpathlineto{\pgfqpoint{4.534697in}{1.954145in}}%
\pgfpathlineto{\pgfqpoint{4.548597in}{1.951448in}}%
\pgfpathlineto{\pgfqpoint{4.562504in}{1.948776in}}%
\pgfpathlineto{\pgfqpoint{4.576419in}{1.946129in}}%
\pgfpathlineto{\pgfqpoint{4.568629in}{1.936826in}}%
\pgfpathlineto{\pgfqpoint{4.560833in}{1.927542in}}%
\pgfpathlineto{\pgfqpoint{4.553033in}{1.918280in}}%
\pgfpathlineto{\pgfqpoint{4.545227in}{1.909044in}}%
\pgfpathlineto{\pgfqpoint{4.531302in}{1.911854in}}%
\pgfpathlineto{\pgfqpoint{4.517385in}{1.914690in}}%
\pgfpathlineto{\pgfqpoint{4.503474in}{1.917550in}}%
\pgfpathlineto{\pgfqpoint{4.489571in}{1.920436in}}%
\pgfpathlineto{\pgfqpoint{4.497387in}{1.929503in}}%
\pgfpathlineto{\pgfqpoint{4.505199in}{1.938600in}}%
\pgfpathlineto{\pgfqpoint{4.513004in}{1.947723in}}%
\pgfpathlineto{\pgfqpoint{4.520805in}{1.956868in}}%
\pgfpathclose%
\pgfusepath{fill}%
\end{pgfscope}%
\begin{pgfscope}%
\pgfpathrectangle{\pgfqpoint{1.150000in}{0.150000in}}{\pgfqpoint{5.700000in}{5.700000in}}%
\pgfusepath{clip}%
\pgfsetbuttcap%
\pgfsetroundjoin%
\definecolor{currentfill}{rgb}{0.280267,0.073417,0.397163}%
\pgfsetfillcolor{currentfill}%
\pgfsetfillopacity{0.700000}%
\pgfsetlinewidth{0.000000pt}%
\definecolor{currentstroke}{rgb}{0.000000,0.000000,0.000000}%
\pgfsetstrokecolor{currentstroke}%
\pgfsetdash{}{0pt}%
\pgfpathmoveto{\pgfqpoint{4.836689in}{2.033053in}}%
\pgfpathlineto{\pgfqpoint{4.850674in}{2.031142in}}%
\pgfpathlineto{\pgfqpoint{4.864665in}{2.029256in}}%
\pgfpathlineto{\pgfqpoint{4.878665in}{2.027394in}}%
\pgfpathlineto{\pgfqpoint{4.892673in}{2.025558in}}%
\pgfpathlineto{\pgfqpoint{4.884986in}{2.015825in}}%
\pgfpathlineto{\pgfqpoint{4.877293in}{2.006058in}}%
\pgfpathlineto{\pgfqpoint{4.869595in}{1.996261in}}%
\pgfpathlineto{\pgfqpoint{4.861891in}{1.986435in}}%
\pgfpathlineto{\pgfqpoint{4.847874in}{1.988395in}}%
\pgfpathlineto{\pgfqpoint{4.833865in}{1.990380in}}%
\pgfpathlineto{\pgfqpoint{4.819864in}{1.992390in}}%
\pgfpathlineto{\pgfqpoint{4.805871in}{1.994424in}}%
\pgfpathlineto{\pgfqpoint{4.813584in}{2.004121in}}%
\pgfpathlineto{\pgfqpoint{4.821291in}{2.013794in}}%
\pgfpathlineto{\pgfqpoint{4.828993in}{2.023439in}}%
\pgfpathlineto{\pgfqpoint{4.836689in}{2.033053in}}%
\pgfpathclose%
\pgfusepath{fill}%
\end{pgfscope}%
\begin{pgfscope}%
\pgfpathrectangle{\pgfqpoint{1.150000in}{0.150000in}}{\pgfqpoint{5.700000in}{5.700000in}}%
\pgfusepath{clip}%
\pgfsetbuttcap%
\pgfsetroundjoin%
\definecolor{currentfill}{rgb}{0.269944,0.014625,0.341379}%
\pgfsetfillcolor{currentfill}%
\pgfsetfillopacity{0.700000}%
\pgfsetlinewidth{0.000000pt}%
\definecolor{currentstroke}{rgb}{0.000000,0.000000,0.000000}%
\pgfsetstrokecolor{currentstroke}%
\pgfsetdash{}{0pt}%
\pgfpathmoveto{\pgfqpoint{3.778835in}{1.929420in}}%
\pgfpathlineto{\pgfqpoint{3.792552in}{1.924469in}}%
\pgfpathlineto{\pgfqpoint{3.806274in}{1.919545in}}%
\pgfpathlineto{\pgfqpoint{3.820002in}{1.914649in}}%
\pgfpathlineto{\pgfqpoint{3.833736in}{1.909781in}}%
\pgfpathlineto{\pgfqpoint{3.825668in}{1.904548in}}%
\pgfpathlineto{\pgfqpoint{3.817592in}{1.899485in}}%
\pgfpathlineto{\pgfqpoint{3.809508in}{1.894599in}}%
\pgfpathlineto{\pgfqpoint{3.801416in}{1.889894in}}%
\pgfpathlineto{\pgfqpoint{3.787664in}{1.895005in}}%
\pgfpathlineto{\pgfqpoint{3.773918in}{1.900144in}}%
\pgfpathlineto{\pgfqpoint{3.760177in}{1.905310in}}%
\pgfpathlineto{\pgfqpoint{3.746442in}{1.910503in}}%
\pgfpathlineto{\pgfqpoint{3.754553in}{1.914960in}}%
\pgfpathlineto{\pgfqpoint{3.762655in}{1.919603in}}%
\pgfpathlineto{\pgfqpoint{3.770749in}{1.924425in}}%
\pgfpathlineto{\pgfqpoint{3.778835in}{1.929420in}}%
\pgfpathclose%
\pgfusepath{fill}%
\end{pgfscope}%
\begin{pgfscope}%
\pgfpathrectangle{\pgfqpoint{1.150000in}{0.150000in}}{\pgfqpoint{5.700000in}{5.700000in}}%
\pgfusepath{clip}%
\pgfsetbuttcap%
\pgfsetroundjoin%
\definecolor{currentfill}{rgb}{0.282884,0.135920,0.453427}%
\pgfsetfillcolor{currentfill}%
\pgfsetfillopacity{0.700000}%
\pgfsetlinewidth{0.000000pt}%
\definecolor{currentstroke}{rgb}{0.000000,0.000000,0.000000}%
\pgfsetstrokecolor{currentstroke}%
\pgfsetdash{}{0pt}%
\pgfpathmoveto{\pgfqpoint{5.239472in}{2.154540in}}%
\pgfpathlineto{\pgfqpoint{5.253582in}{2.153483in}}%
\pgfpathlineto{\pgfqpoint{5.267700in}{2.152451in}}%
\pgfpathlineto{\pgfqpoint{5.281827in}{2.151444in}}%
\pgfpathlineto{\pgfqpoint{5.295963in}{2.150461in}}%
\pgfpathlineto{\pgfqpoint{5.288425in}{2.141245in}}%
\pgfpathlineto{\pgfqpoint{5.280879in}{2.131947in}}%
\pgfpathlineto{\pgfqpoint{5.273326in}{2.122568in}}%
\pgfpathlineto{\pgfqpoint{5.265767in}{2.113109in}}%
\pgfpathlineto{\pgfqpoint{5.251621in}{2.114161in}}%
\pgfpathlineto{\pgfqpoint{5.237484in}{2.115238in}}%
\pgfpathlineto{\pgfqpoint{5.223356in}{2.116339in}}%
\pgfpathlineto{\pgfqpoint{5.209236in}{2.117465in}}%
\pgfpathlineto{\pgfqpoint{5.216805in}{2.126849in}}%
\pgfpathlineto{\pgfqpoint{5.224367in}{2.136157in}}%
\pgfpathlineto{\pgfqpoint{5.231923in}{2.145388in}}%
\pgfpathlineto{\pgfqpoint{5.239472in}{2.154540in}}%
\pgfpathclose%
\pgfusepath{fill}%
\end{pgfscope}%
\begin{pgfscope}%
\pgfpathrectangle{\pgfqpoint{1.150000in}{0.150000in}}{\pgfqpoint{5.700000in}{5.700000in}}%
\pgfusepath{clip}%
\pgfsetbuttcap%
\pgfsetroundjoin%
\definecolor{currentfill}{rgb}{0.268510,0.009605,0.335427}%
\pgfsetfillcolor{currentfill}%
\pgfsetfillopacity{0.700000}%
\pgfsetlinewidth{0.000000pt}%
\definecolor{currentstroke}{rgb}{0.000000,0.000000,0.000000}%
\pgfsetstrokecolor{currentstroke}%
\pgfsetdash{}{0pt}%
\pgfpathmoveto{\pgfqpoint{3.920857in}{1.914008in}}%
\pgfpathlineto{\pgfqpoint{3.934603in}{1.909505in}}%
\pgfpathlineto{\pgfqpoint{3.948354in}{1.905029in}}%
\pgfpathlineto{\pgfqpoint{3.962112in}{1.900580in}}%
\pgfpathlineto{\pgfqpoint{3.975876in}{1.896157in}}%
\pgfpathlineto{\pgfqpoint{3.967870in}{1.889856in}}%
\pgfpathlineto{\pgfqpoint{3.959858in}{1.883694in}}%
\pgfpathlineto{\pgfqpoint{3.951838in}{1.877678in}}%
\pgfpathlineto{\pgfqpoint{3.943812in}{1.871814in}}%
\pgfpathlineto{\pgfqpoint{3.930032in}{1.876465in}}%
\pgfpathlineto{\pgfqpoint{3.916258in}{1.881143in}}%
\pgfpathlineto{\pgfqpoint{3.902490in}{1.885848in}}%
\pgfpathlineto{\pgfqpoint{3.888728in}{1.890580in}}%
\pgfpathlineto{\pgfqpoint{3.896771in}{1.896211in}}%
\pgfpathlineto{\pgfqpoint{3.904807in}{1.901996in}}%
\pgfpathlineto{\pgfqpoint{3.912835in}{1.907931in}}%
\pgfpathlineto{\pgfqpoint{3.920857in}{1.914008in}}%
\pgfpathclose%
\pgfusepath{fill}%
\end{pgfscope}%
\begin{pgfscope}%
\pgfpathrectangle{\pgfqpoint{1.150000in}{0.150000in}}{\pgfqpoint{5.700000in}{5.700000in}}%
\pgfusepath{clip}%
\pgfsetbuttcap%
\pgfsetroundjoin%
\definecolor{currentfill}{rgb}{0.274128,0.199721,0.498911}%
\pgfsetfillcolor{currentfill}%
\pgfsetfillopacity{0.700000}%
\pgfsetlinewidth{0.000000pt}%
\definecolor{currentstroke}{rgb}{0.000000,0.000000,0.000000}%
\pgfsetstrokecolor{currentstroke}%
\pgfsetdash{}{0pt}%
\pgfpathmoveto{\pgfqpoint{5.642233in}{2.275941in}}%
\pgfpathlineto{\pgfqpoint{5.656475in}{2.275517in}}%
\pgfpathlineto{\pgfqpoint{5.670726in}{2.275117in}}%
\pgfpathlineto{\pgfqpoint{5.684986in}{2.274742in}}%
\pgfpathlineto{\pgfqpoint{5.699255in}{2.274392in}}%
\pgfpathlineto{\pgfqpoint{5.691898in}{2.266543in}}%
\pgfpathlineto{\pgfqpoint{5.684533in}{2.258586in}}%
\pgfpathlineto{\pgfqpoint{5.677159in}{2.250521in}}%
\pgfpathlineto{\pgfqpoint{5.669776in}{2.242350in}}%
\pgfpathlineto{\pgfqpoint{5.655494in}{2.242714in}}%
\pgfpathlineto{\pgfqpoint{5.641222in}{2.243102in}}%
\pgfpathlineto{\pgfqpoint{5.626959in}{2.243516in}}%
\pgfpathlineto{\pgfqpoint{5.612705in}{2.243953in}}%
\pgfpathlineto{\pgfqpoint{5.620099in}{2.252106in}}%
\pgfpathlineto{\pgfqpoint{5.627486in}{2.260155in}}%
\pgfpathlineto{\pgfqpoint{5.634864in}{2.268099in}}%
\pgfpathlineto{\pgfqpoint{5.642233in}{2.275941in}}%
\pgfpathclose%
\pgfusepath{fill}%
\end{pgfscope}%
\begin{pgfscope}%
\pgfpathrectangle{\pgfqpoint{1.150000in}{0.150000in}}{\pgfqpoint{5.700000in}{5.700000in}}%
\pgfusepath{clip}%
\pgfsetbuttcap%
\pgfsetroundjoin%
\definecolor{currentfill}{rgb}{0.243113,0.292092,0.538516}%
\pgfsetfillcolor{currentfill}%
\pgfsetfillopacity{0.700000}%
\pgfsetlinewidth{0.000000pt}%
\definecolor{currentstroke}{rgb}{0.000000,0.000000,0.000000}%
\pgfsetstrokecolor{currentstroke}%
\pgfsetdash{}{0pt}%
\pgfpathmoveto{\pgfqpoint{2.541236in}{2.466832in}}%
\pgfpathlineto{\pgfqpoint{2.554817in}{2.457578in}}%
\pgfpathlineto{\pgfqpoint{2.568401in}{2.448370in}}%
\pgfpathlineto{\pgfqpoint{2.581987in}{2.439206in}}%
\pgfpathlineto{\pgfqpoint{2.595575in}{2.430087in}}%
\pgfpathlineto{\pgfqpoint{2.586645in}{2.436983in}}%
\pgfpathlineto{\pgfqpoint{2.577691in}{2.444296in}}%
\pgfpathlineto{\pgfqpoint{2.568714in}{2.452037in}}%
\pgfpathlineto{\pgfqpoint{2.559712in}{2.460215in}}%
\pgfpathlineto{\pgfqpoint{2.546083in}{2.469667in}}%
\pgfpathlineto{\pgfqpoint{2.532457in}{2.479164in}}%
\pgfpathlineto{\pgfqpoint{2.518832in}{2.488707in}}%
\pgfpathlineto{\pgfqpoint{2.505209in}{2.498295in}}%
\pgfpathlineto{\pgfqpoint{2.514252in}{2.489777in}}%
\pgfpathlineto{\pgfqpoint{2.523271in}{2.481701in}}%
\pgfpathlineto{\pgfqpoint{2.532265in}{2.474056in}}%
\pgfpathlineto{\pgfqpoint{2.541236in}{2.466832in}}%
\pgfpathclose%
\pgfusepath{fill}%
\end{pgfscope}%
\begin{pgfscope}%
\pgfpathrectangle{\pgfqpoint{1.150000in}{0.150000in}}{\pgfqpoint{5.700000in}{5.700000in}}%
\pgfusepath{clip}%
\pgfsetbuttcap%
\pgfsetroundjoin%
\definecolor{currentfill}{rgb}{0.273809,0.031497,0.358853}%
\pgfsetfillcolor{currentfill}%
\pgfsetfillopacity{0.700000}%
\pgfsetlinewidth{0.000000pt}%
\definecolor{currentstroke}{rgb}{0.000000,0.000000,0.000000}%
\pgfsetstrokecolor{currentstroke}%
\pgfsetdash{}{0pt}%
\pgfpathmoveto{\pgfqpoint{3.636753in}{1.953065in}}%
\pgfpathlineto{\pgfqpoint{3.650446in}{1.947645in}}%
\pgfpathlineto{\pgfqpoint{3.664143in}{1.942254in}}%
\pgfpathlineto{\pgfqpoint{3.677846in}{1.936892in}}%
\pgfpathlineto{\pgfqpoint{3.691555in}{1.931558in}}%
\pgfpathlineto{\pgfqpoint{3.683416in}{1.927544in}}%
\pgfpathlineto{\pgfqpoint{3.675268in}{1.923732in}}%
\pgfpathlineto{\pgfqpoint{3.667111in}{1.920129in}}%
\pgfpathlineto{\pgfqpoint{3.658945in}{1.916741in}}%
\pgfpathlineto{\pgfqpoint{3.645216in}{1.922331in}}%
\pgfpathlineto{\pgfqpoint{3.631492in}{1.927949in}}%
\pgfpathlineto{\pgfqpoint{3.617774in}{1.933596in}}%
\pgfpathlineto{\pgfqpoint{3.604061in}{1.939272in}}%
\pgfpathlineto{\pgfqpoint{3.612248in}{1.942399in}}%
\pgfpathlineto{\pgfqpoint{3.620426in}{1.945744in}}%
\pgfpathlineto{\pgfqpoint{3.628594in}{1.949302in}}%
\pgfpathlineto{\pgfqpoint{3.636753in}{1.953065in}}%
\pgfpathclose%
\pgfusepath{fill}%
\end{pgfscope}%
\begin{pgfscope}%
\pgfpathrectangle{\pgfqpoint{1.150000in}{0.150000in}}{\pgfqpoint{5.700000in}{5.700000in}}%
\pgfusepath{clip}%
\pgfsetbuttcap%
\pgfsetroundjoin%
\definecolor{currentfill}{rgb}{0.267004,0.004874,0.329415}%
\pgfsetfillcolor{currentfill}%
\pgfsetfillopacity{0.700000}%
\pgfsetlinewidth{0.000000pt}%
\definecolor{currentstroke}{rgb}{0.000000,0.000000,0.000000}%
\pgfsetstrokecolor{currentstroke}%
\pgfsetdash{}{0pt}%
\pgfpathmoveto{\pgfqpoint{4.062886in}{1.906089in}}%
\pgfpathlineto{\pgfqpoint{4.076665in}{1.902016in}}%
\pgfpathlineto{\pgfqpoint{4.090450in}{1.897969in}}%
\pgfpathlineto{\pgfqpoint{4.104242in}{1.893948in}}%
\pgfpathlineto{\pgfqpoint{4.118040in}{1.889954in}}%
\pgfpathlineto{\pgfqpoint{4.110090in}{1.882728in}}%
\pgfpathlineto{\pgfqpoint{4.102134in}{1.875612in}}%
\pgfpathlineto{\pgfqpoint{4.094172in}{1.868614in}}%
\pgfpathlineto{\pgfqpoint{4.086204in}{1.861737in}}%
\pgfpathlineto{\pgfqpoint{4.072391in}{1.865947in}}%
\pgfpathlineto{\pgfqpoint{4.058585in}{1.870183in}}%
\pgfpathlineto{\pgfqpoint{4.044785in}{1.874446in}}%
\pgfpathlineto{\pgfqpoint{4.030991in}{1.878735in}}%
\pgfpathlineto{\pgfqpoint{4.038974in}{1.885391in}}%
\pgfpathlineto{\pgfqpoint{4.046951in}{1.892172in}}%
\pgfpathlineto{\pgfqpoint{4.054921in}{1.899074in}}%
\pgfpathlineto{\pgfqpoint{4.062886in}{1.906089in}}%
\pgfpathclose%
\pgfusepath{fill}%
\end{pgfscope}%
\begin{pgfscope}%
\pgfpathrectangle{\pgfqpoint{1.150000in}{0.150000in}}{\pgfqpoint{5.700000in}{5.700000in}}%
\pgfusepath{clip}%
\pgfsetbuttcap%
\pgfsetroundjoin%
\definecolor{currentfill}{rgb}{0.283187,0.125848,0.444960}%
\pgfsetfillcolor{currentfill}%
\pgfsetfillopacity{0.700000}%
\pgfsetlinewidth{0.000000pt}%
\definecolor{currentstroke}{rgb}{0.000000,0.000000,0.000000}%
\pgfsetstrokecolor{currentstroke}%
\pgfsetdash{}{0pt}%
\pgfpathmoveto{\pgfqpoint{5.152841in}{2.122216in}}%
\pgfpathlineto{\pgfqpoint{5.166927in}{2.120991in}}%
\pgfpathlineto{\pgfqpoint{5.181022in}{2.119791in}}%
\pgfpathlineto{\pgfqpoint{5.195124in}{2.118616in}}%
\pgfpathlineto{\pgfqpoint{5.209236in}{2.117465in}}%
\pgfpathlineto{\pgfqpoint{5.201660in}{2.108006in}}%
\pgfpathlineto{\pgfqpoint{5.194077in}{2.098474in}}%
\pgfpathlineto{\pgfqpoint{5.186488in}{2.088871in}}%
\pgfpathlineto{\pgfqpoint{5.178892in}{2.079198in}}%
\pgfpathlineto{\pgfqpoint{5.164772in}{2.080432in}}%
\pgfpathlineto{\pgfqpoint{5.150659in}{2.081690in}}%
\pgfpathlineto{\pgfqpoint{5.136555in}{2.082973in}}%
\pgfpathlineto{\pgfqpoint{5.122460in}{2.084281in}}%
\pgfpathlineto{\pgfqpoint{5.130065in}{2.093866in}}%
\pgfpathlineto{\pgfqpoint{5.137664in}{2.103384in}}%
\pgfpathlineto{\pgfqpoint{5.145256in}{2.112835in}}%
\pgfpathlineto{\pgfqpoint{5.152841in}{2.122216in}}%
\pgfpathclose%
\pgfusepath{fill}%
\end{pgfscope}%
\begin{pgfscope}%
\pgfpathrectangle{\pgfqpoint{1.150000in}{0.150000in}}{\pgfqpoint{5.700000in}{5.700000in}}%
\pgfusepath{clip}%
\pgfsetbuttcap%
\pgfsetroundjoin%
\definecolor{currentfill}{rgb}{0.273006,0.204520,0.501721}%
\pgfsetfillcolor{currentfill}%
\pgfsetfillopacity{0.700000}%
\pgfsetlinewidth{0.000000pt}%
\definecolor{currentstroke}{rgb}{0.000000,0.000000,0.000000}%
\pgfsetstrokecolor{currentstroke}%
\pgfsetdash{}{0pt}%
\pgfpathmoveto{\pgfqpoint{2.848227in}{2.271430in}}%
\pgfpathlineto{\pgfqpoint{2.861824in}{2.263328in}}%
\pgfpathlineto{\pgfqpoint{2.875423in}{2.255265in}}%
\pgfpathlineto{\pgfqpoint{2.889026in}{2.247239in}}%
\pgfpathlineto{\pgfqpoint{2.902632in}{2.239251in}}%
\pgfpathlineto{\pgfqpoint{2.893967in}{2.243009in}}%
\pgfpathlineto{\pgfqpoint{2.885284in}{2.247132in}}%
\pgfpathlineto{\pgfqpoint{2.876583in}{2.251626in}}%
\pgfpathlineto{\pgfqpoint{2.867861in}{2.256502in}}%
\pgfpathlineto{\pgfqpoint{2.854221in}{2.264805in}}%
\pgfpathlineto{\pgfqpoint{2.840583in}{2.273146in}}%
\pgfpathlineto{\pgfqpoint{2.826948in}{2.281525in}}%
\pgfpathlineto{\pgfqpoint{2.813316in}{2.289942in}}%
\pgfpathlineto{\pgfqpoint{2.822074in}{2.284746in}}%
\pgfpathlineto{\pgfqpoint{2.830811in}{2.279934in}}%
\pgfpathlineto{\pgfqpoint{2.839528in}{2.275498in}}%
\pgfpathlineto{\pgfqpoint{2.848227in}{2.271430in}}%
\pgfpathclose%
\pgfusepath{fill}%
\end{pgfscope}%
\begin{pgfscope}%
\pgfpathrectangle{\pgfqpoint{1.150000in}{0.150000in}}{\pgfqpoint{5.700000in}{5.700000in}}%
\pgfusepath{clip}%
\pgfsetbuttcap%
\pgfsetroundjoin%
\definecolor{currentfill}{rgb}{0.278791,0.062145,0.386592}%
\pgfsetfillcolor{currentfill}%
\pgfsetfillopacity{0.700000}%
\pgfsetlinewidth{0.000000pt}%
\definecolor{currentstroke}{rgb}{0.000000,0.000000,0.000000}%
\pgfsetstrokecolor{currentstroke}%
\pgfsetdash{}{0pt}%
\pgfpathmoveto{\pgfqpoint{4.749975in}{2.002812in}}%
\pgfpathlineto{\pgfqpoint{4.763937in}{2.000678in}}%
\pgfpathlineto{\pgfqpoint{4.777907in}{1.998568in}}%
\pgfpathlineto{\pgfqpoint{4.791885in}{1.996484in}}%
\pgfpathlineto{\pgfqpoint{4.805871in}{1.994424in}}%
\pgfpathlineto{\pgfqpoint{4.798152in}{1.984705in}}%
\pgfpathlineto{\pgfqpoint{4.790428in}{1.974967in}}%
\pgfpathlineto{\pgfqpoint{4.782699in}{1.965213in}}%
\pgfpathlineto{\pgfqpoint{4.774964in}{1.955447in}}%
\pgfpathlineto{\pgfqpoint{4.760969in}{1.957643in}}%
\pgfpathlineto{\pgfqpoint{4.746982in}{1.959864in}}%
\pgfpathlineto{\pgfqpoint{4.733002in}{1.962111in}}%
\pgfpathlineto{\pgfqpoint{4.719030in}{1.964382in}}%
\pgfpathlineto{\pgfqpoint{4.726774in}{1.974006in}}%
\pgfpathlineto{\pgfqpoint{4.734513in}{1.983622in}}%
\pgfpathlineto{\pgfqpoint{4.742247in}{1.993225in}}%
\pgfpathlineto{\pgfqpoint{4.749975in}{2.002812in}}%
\pgfpathclose%
\pgfusepath{fill}%
\end{pgfscope}%
\begin{pgfscope}%
\pgfpathrectangle{\pgfqpoint{1.150000in}{0.150000in}}{\pgfqpoint{5.700000in}{5.700000in}}%
\pgfusepath{clip}%
\pgfsetbuttcap%
\pgfsetroundjoin%
\definecolor{currentfill}{rgb}{0.282327,0.094955,0.417331}%
\pgfsetfillcolor{currentfill}%
\pgfsetfillopacity{0.700000}%
\pgfsetlinewidth{0.000000pt}%
\definecolor{currentstroke}{rgb}{0.000000,0.000000,0.000000}%
\pgfsetstrokecolor{currentstroke}%
\pgfsetdash{}{0pt}%
\pgfpathmoveto{\pgfqpoint{3.297501in}{2.054207in}}%
\pgfpathlineto{\pgfqpoint{3.311143in}{2.047660in}}%
\pgfpathlineto{\pgfqpoint{3.324789in}{2.041145in}}%
\pgfpathlineto{\pgfqpoint{3.338440in}{2.034662in}}%
\pgfpathlineto{\pgfqpoint{3.352095in}{2.028209in}}%
\pgfpathlineto{\pgfqpoint{3.343759in}{2.027394in}}%
\pgfpathlineto{\pgfqpoint{3.335410in}{2.026853in}}%
\pgfpathlineto{\pgfqpoint{3.327049in}{2.026595in}}%
\pgfpathlineto{\pgfqpoint{3.318675in}{2.026627in}}%
\pgfpathlineto{\pgfqpoint{3.304993in}{2.033363in}}%
\pgfpathlineto{\pgfqpoint{3.291316in}{2.040131in}}%
\pgfpathlineto{\pgfqpoint{3.277643in}{2.046930in}}%
\pgfpathlineto{\pgfqpoint{3.263974in}{2.053761in}}%
\pgfpathlineto{\pgfqpoint{3.272376in}{2.053440in}}%
\pgfpathlineto{\pgfqpoint{3.280764in}{2.053412in}}%
\pgfpathlineto{\pgfqpoint{3.289139in}{2.053670in}}%
\pgfpathlineto{\pgfqpoint{3.297501in}{2.054207in}}%
\pgfpathclose%
\pgfusepath{fill}%
\end{pgfscope}%
\begin{pgfscope}%
\pgfpathrectangle{\pgfqpoint{1.150000in}{0.150000in}}{\pgfqpoint{5.700000in}{5.700000in}}%
\pgfusepath{clip}%
\pgfsetbuttcap%
\pgfsetroundjoin%
\definecolor{currentfill}{rgb}{0.277134,0.185228,0.489898}%
\pgfsetfillcolor{currentfill}%
\pgfsetfillopacity{0.700000}%
\pgfsetlinewidth{0.000000pt}%
\definecolor{currentstroke}{rgb}{0.000000,0.000000,0.000000}%
\pgfsetstrokecolor{currentstroke}%
\pgfsetdash{}{0pt}%
\pgfpathmoveto{\pgfqpoint{5.555779in}{2.245952in}}%
\pgfpathlineto{\pgfqpoint{5.569997in}{2.245415in}}%
\pgfpathlineto{\pgfqpoint{5.584224in}{2.244903in}}%
\pgfpathlineto{\pgfqpoint{5.598460in}{2.244416in}}%
\pgfpathlineto{\pgfqpoint{5.612705in}{2.243953in}}%
\pgfpathlineto{\pgfqpoint{5.605302in}{2.235697in}}%
\pgfpathlineto{\pgfqpoint{5.597890in}{2.227337in}}%
\pgfpathlineto{\pgfqpoint{5.590471in}{2.218873in}}%
\pgfpathlineto{\pgfqpoint{5.583043in}{2.210305in}}%
\pgfpathlineto{\pgfqpoint{5.568786in}{2.210795in}}%
\pgfpathlineto{\pgfqpoint{5.554539in}{2.211310in}}%
\pgfpathlineto{\pgfqpoint{5.540300in}{2.211850in}}%
\pgfpathlineto{\pgfqpoint{5.526071in}{2.212415in}}%
\pgfpathlineto{\pgfqpoint{5.533510in}{2.220949in}}%
\pgfpathlineto{\pgfqpoint{5.540941in}{2.229384in}}%
\pgfpathlineto{\pgfqpoint{5.548364in}{2.237718in}}%
\pgfpathlineto{\pgfqpoint{5.555779in}{2.245952in}}%
\pgfpathclose%
\pgfusepath{fill}%
\end{pgfscope}%
\begin{pgfscope}%
\pgfpathrectangle{\pgfqpoint{1.150000in}{0.150000in}}{\pgfqpoint{5.700000in}{5.700000in}}%
\pgfusepath{clip}%
\pgfsetbuttcap%
\pgfsetroundjoin%
\definecolor{currentfill}{rgb}{0.271305,0.019942,0.347269}%
\pgfsetfillcolor{currentfill}%
\pgfsetfillopacity{0.700000}%
\pgfsetlinewidth{0.000000pt}%
\definecolor{currentstroke}{rgb}{0.000000,0.000000,0.000000}%
\pgfsetstrokecolor{currentstroke}%
\pgfsetdash{}{0pt}%
\pgfpathmoveto{\pgfqpoint{4.434030in}{1.932232in}}%
\pgfpathlineto{\pgfqpoint{4.447905in}{1.929245in}}%
\pgfpathlineto{\pgfqpoint{4.461786in}{1.926283in}}%
\pgfpathlineto{\pgfqpoint{4.475675in}{1.923347in}}%
\pgfpathlineto{\pgfqpoint{4.489571in}{1.920436in}}%
\pgfpathlineto{\pgfqpoint{4.481750in}{1.911402in}}%
\pgfpathlineto{\pgfqpoint{4.473923in}{1.902407in}}%
\pgfpathlineto{\pgfqpoint{4.466091in}{1.893453in}}%
\pgfpathlineto{\pgfqpoint{4.458254in}{1.884545in}}%
\pgfpathlineto{\pgfqpoint{4.444347in}{1.887633in}}%
\pgfpathlineto{\pgfqpoint{4.430447in}{1.890746in}}%
\pgfpathlineto{\pgfqpoint{4.416554in}{1.893884in}}%
\pgfpathlineto{\pgfqpoint{4.402668in}{1.897047in}}%
\pgfpathlineto{\pgfqpoint{4.410517in}{1.905773in}}%
\pgfpathlineto{\pgfqpoint{4.418360in}{1.914549in}}%
\pgfpathlineto{\pgfqpoint{4.426197in}{1.923370in}}%
\pgfpathlineto{\pgfqpoint{4.434030in}{1.932232in}}%
\pgfpathclose%
\pgfusepath{fill}%
\end{pgfscope}%
\begin{pgfscope}%
\pgfpathrectangle{\pgfqpoint{1.150000in}{0.150000in}}{\pgfqpoint{5.700000in}{5.700000in}}%
\pgfusepath{clip}%
\pgfsetbuttcap%
\pgfsetroundjoin%
\definecolor{currentfill}{rgb}{0.282623,0.140926,0.457517}%
\pgfsetfillcolor{currentfill}%
\pgfsetfillopacity{0.700000}%
\pgfsetlinewidth{0.000000pt}%
\definecolor{currentstroke}{rgb}{0.000000,0.000000,0.000000}%
\pgfsetstrokecolor{currentstroke}%
\pgfsetdash{}{0pt}%
\pgfpathmoveto{\pgfqpoint{3.100268in}{2.138261in}}%
\pgfpathlineto{\pgfqpoint{3.113888in}{2.131036in}}%
\pgfpathlineto{\pgfqpoint{3.127513in}{2.123845in}}%
\pgfpathlineto{\pgfqpoint{3.141141in}{2.116689in}}%
\pgfpathlineto{\pgfqpoint{3.154773in}{2.109565in}}%
\pgfpathlineto{\pgfqpoint{3.146300in}{2.110775in}}%
\pgfpathlineto{\pgfqpoint{3.137813in}{2.112300in}}%
\pgfpathlineto{\pgfqpoint{3.129310in}{2.114150in}}%
\pgfpathlineto{\pgfqpoint{3.120792in}{2.116333in}}%
\pgfpathlineto{\pgfqpoint{3.107130in}{2.123755in}}%
\pgfpathlineto{\pgfqpoint{3.093471in}{2.131211in}}%
\pgfpathlineto{\pgfqpoint{3.079817in}{2.138701in}}%
\pgfpathlineto{\pgfqpoint{3.066166in}{2.146225in}}%
\pgfpathlineto{\pgfqpoint{3.074715in}{2.143738in}}%
\pgfpathlineto{\pgfqpoint{3.083249in}{2.141587in}}%
\pgfpathlineto{\pgfqpoint{3.091766in}{2.139764in}}%
\pgfpathlineto{\pgfqpoint{3.100268in}{2.138261in}}%
\pgfpathclose%
\pgfusepath{fill}%
\end{pgfscope}%
\begin{pgfscope}%
\pgfpathrectangle{\pgfqpoint{1.150000in}{0.150000in}}{\pgfqpoint{5.700000in}{5.700000in}}%
\pgfusepath{clip}%
\pgfsetbuttcap%
\pgfsetroundjoin%
\definecolor{currentfill}{rgb}{0.268510,0.009605,0.335427}%
\pgfsetfillcolor{currentfill}%
\pgfsetfillopacity{0.700000}%
\pgfsetlinewidth{0.000000pt}%
\definecolor{currentstroke}{rgb}{0.000000,0.000000,0.000000}%
\pgfsetstrokecolor{currentstroke}%
\pgfsetdash{}{0pt}%
\pgfpathmoveto{\pgfqpoint{4.204981in}{1.904961in}}%
\pgfpathlineto{\pgfqpoint{4.218797in}{1.901300in}}%
\pgfpathlineto{\pgfqpoint{4.232621in}{1.897665in}}%
\pgfpathlineto{\pgfqpoint{4.246451in}{1.894056in}}%
\pgfpathlineto{\pgfqpoint{4.260287in}{1.890472in}}%
\pgfpathlineto{\pgfqpoint{4.252388in}{1.882458in}}%
\pgfpathlineto{\pgfqpoint{4.244482in}{1.874527in}}%
\pgfpathlineto{\pgfqpoint{4.236571in}{1.866686in}}%
\pgfpathlineto{\pgfqpoint{4.228655in}{1.858938in}}%
\pgfpathlineto{\pgfqpoint{4.214805in}{1.862725in}}%
\pgfpathlineto{\pgfqpoint{4.200962in}{1.866537in}}%
\pgfpathlineto{\pgfqpoint{4.187126in}{1.870375in}}%
\pgfpathlineto{\pgfqpoint{4.173296in}{1.874238in}}%
\pgfpathlineto{\pgfqpoint{4.181226in}{1.881778in}}%
\pgfpathlineto{\pgfqpoint{4.189150in}{1.889415in}}%
\pgfpathlineto{\pgfqpoint{4.197068in}{1.897145in}}%
\pgfpathlineto{\pgfqpoint{4.204981in}{1.904961in}}%
\pgfpathclose%
\pgfusepath{fill}%
\end{pgfscope}%
\begin{pgfscope}%
\pgfpathrectangle{\pgfqpoint{1.150000in}{0.150000in}}{\pgfqpoint{5.700000in}{5.700000in}}%
\pgfusepath{clip}%
\pgfsetbuttcap%
\pgfsetroundjoin%
\definecolor{currentfill}{rgb}{0.277941,0.056324,0.381191}%
\pgfsetfillcolor{currentfill}%
\pgfsetfillopacity{0.700000}%
\pgfsetlinewidth{0.000000pt}%
\definecolor{currentstroke}{rgb}{0.000000,0.000000,0.000000}%
\pgfsetstrokecolor{currentstroke}%
\pgfsetdash{}{0pt}%
\pgfpathmoveto{\pgfqpoint{3.494535in}{1.985720in}}%
\pgfpathlineto{\pgfqpoint{3.508208in}{1.979811in}}%
\pgfpathlineto{\pgfqpoint{3.521886in}{1.973932in}}%
\pgfpathlineto{\pgfqpoint{3.535570in}{1.968082in}}%
\pgfpathlineto{\pgfqpoint{3.549258in}{1.962262in}}%
\pgfpathlineto{\pgfqpoint{3.541038in}{1.959625in}}%
\pgfpathlineto{\pgfqpoint{3.532809in}{1.957224in}}%
\pgfpathlineto{\pgfqpoint{3.524569in}{1.955066in}}%
\pgfpathlineto{\pgfqpoint{3.516318in}{1.953157in}}%
\pgfpathlineto{\pgfqpoint{3.502607in}{1.959247in}}%
\pgfpathlineto{\pgfqpoint{3.488900in}{1.965367in}}%
\pgfpathlineto{\pgfqpoint{3.475199in}{1.971516in}}%
\pgfpathlineto{\pgfqpoint{3.461502in}{1.977695in}}%
\pgfpathlineto{\pgfqpoint{3.469777in}{1.979328in}}%
\pgfpathlineto{\pgfqpoint{3.478040in}{1.981215in}}%
\pgfpathlineto{\pgfqpoint{3.486293in}{1.983348in}}%
\pgfpathlineto{\pgfqpoint{3.494535in}{1.985720in}}%
\pgfpathclose%
\pgfusepath{fill}%
\end{pgfscope}%
\begin{pgfscope}%
\pgfpathrectangle{\pgfqpoint{1.150000in}{0.150000in}}{\pgfqpoint{5.700000in}{5.700000in}}%
\pgfusepath{clip}%
\pgfsetbuttcap%
\pgfsetroundjoin%
\definecolor{currentfill}{rgb}{0.283091,0.110553,0.431554}%
\pgfsetfillcolor{currentfill}%
\pgfsetfillopacity{0.700000}%
\pgfsetlinewidth{0.000000pt}%
\definecolor{currentstroke}{rgb}{0.000000,0.000000,0.000000}%
\pgfsetstrokecolor{currentstroke}%
\pgfsetdash{}{0pt}%
\pgfpathmoveto{\pgfqpoint{5.066161in}{2.089759in}}%
\pgfpathlineto{\pgfqpoint{5.080224in}{2.088353in}}%
\pgfpathlineto{\pgfqpoint{5.094294in}{2.086971in}}%
\pgfpathlineto{\pgfqpoint{5.108373in}{2.085613in}}%
\pgfpathlineto{\pgfqpoint{5.122460in}{2.084281in}}%
\pgfpathlineto{\pgfqpoint{5.114848in}{2.074632in}}%
\pgfpathlineto{\pgfqpoint{5.107231in}{2.064920in}}%
\pgfpathlineto{\pgfqpoint{5.099607in}{2.055147in}}%
\pgfpathlineto{\pgfqpoint{5.091976in}{2.045316in}}%
\pgfpathlineto{\pgfqpoint{5.077880in}{2.046746in}}%
\pgfpathlineto{\pgfqpoint{5.063792in}{2.048199in}}%
\pgfpathlineto{\pgfqpoint{5.049713in}{2.049678in}}%
\pgfpathlineto{\pgfqpoint{5.035641in}{2.051181in}}%
\pgfpathlineto{\pgfqpoint{5.043280in}{2.060910in}}%
\pgfpathlineto{\pgfqpoint{5.050914in}{2.070584in}}%
\pgfpathlineto{\pgfqpoint{5.058541in}{2.080201in}}%
\pgfpathlineto{\pgfqpoint{5.066161in}{2.089759in}}%
\pgfpathclose%
\pgfusepath{fill}%
\end{pgfscope}%
\begin{pgfscope}%
\pgfpathrectangle{\pgfqpoint{1.150000in}{0.150000in}}{\pgfqpoint{5.700000in}{5.700000in}}%
\pgfusepath{clip}%
\pgfsetbuttcap%
\pgfsetroundjoin%
\definecolor{currentfill}{rgb}{0.248629,0.278775,0.534556}%
\pgfsetfillcolor{currentfill}%
\pgfsetfillopacity{0.700000}%
\pgfsetlinewidth{0.000000pt}%
\definecolor{currentstroke}{rgb}{0.000000,0.000000,0.000000}%
\pgfsetstrokecolor{currentstroke}%
\pgfsetdash{}{0pt}%
\pgfpathmoveto{\pgfqpoint{2.595575in}{2.430087in}}%
\pgfpathlineto{\pgfqpoint{2.609166in}{2.421013in}}%
\pgfpathlineto{\pgfqpoint{2.622758in}{2.411982in}}%
\pgfpathlineto{\pgfqpoint{2.636353in}{2.402995in}}%
\pgfpathlineto{\pgfqpoint{2.649950in}{2.394050in}}%
\pgfpathlineto{\pgfqpoint{2.641059in}{2.400619in}}%
\pgfpathlineto{\pgfqpoint{2.632146in}{2.407601in}}%
\pgfpathlineto{\pgfqpoint{2.623209in}{2.415007in}}%
\pgfpathlineto{\pgfqpoint{2.614248in}{2.422845in}}%
\pgfpathlineto{\pgfqpoint{2.600611in}{2.432122in}}%
\pgfpathlineto{\pgfqpoint{2.586976in}{2.441443in}}%
\pgfpathlineto{\pgfqpoint{2.573343in}{2.450807in}}%
\pgfpathlineto{\pgfqpoint{2.559712in}{2.460215in}}%
\pgfpathlineto{\pgfqpoint{2.568714in}{2.452037in}}%
\pgfpathlineto{\pgfqpoint{2.577691in}{2.444296in}}%
\pgfpathlineto{\pgfqpoint{2.586645in}{2.436983in}}%
\pgfpathlineto{\pgfqpoint{2.595575in}{2.430087in}}%
\pgfpathclose%
\pgfusepath{fill}%
\end{pgfscope}%
\begin{pgfscope}%
\pgfpathrectangle{\pgfqpoint{1.150000in}{0.150000in}}{\pgfqpoint{5.700000in}{5.700000in}}%
\pgfusepath{clip}%
\pgfsetbuttcap%
\pgfsetroundjoin%
\definecolor{currentfill}{rgb}{0.277018,0.050344,0.375715}%
\pgfsetfillcolor{currentfill}%
\pgfsetfillopacity{0.700000}%
\pgfsetlinewidth{0.000000pt}%
\definecolor{currentstroke}{rgb}{0.000000,0.000000,0.000000}%
\pgfsetstrokecolor{currentstroke}%
\pgfsetdash{}{0pt}%
\pgfpathmoveto{\pgfqpoint{4.663218in}{1.973716in}}%
\pgfpathlineto{\pgfqpoint{4.677160in}{1.971345in}}%
\pgfpathlineto{\pgfqpoint{4.691109in}{1.968999in}}%
\pgfpathlineto{\pgfqpoint{4.705066in}{1.966678in}}%
\pgfpathlineto{\pgfqpoint{4.719030in}{1.964382in}}%
\pgfpathlineto{\pgfqpoint{4.711281in}{1.954751in}}%
\pgfpathlineto{\pgfqpoint{4.703526in}{1.945118in}}%
\pgfpathlineto{\pgfqpoint{4.695766in}{1.935486in}}%
\pgfpathlineto{\pgfqpoint{4.688001in}{1.925858in}}%
\pgfpathlineto{\pgfqpoint{4.674027in}{1.928304in}}%
\pgfpathlineto{\pgfqpoint{4.660060in}{1.930776in}}%
\pgfpathlineto{\pgfqpoint{4.646102in}{1.933272in}}%
\pgfpathlineto{\pgfqpoint{4.632150in}{1.935793in}}%
\pgfpathlineto{\pgfqpoint{4.639925in}{1.945266in}}%
\pgfpathlineto{\pgfqpoint{4.647695in}{1.954746in}}%
\pgfpathlineto{\pgfqpoint{4.655459in}{1.964231in}}%
\pgfpathlineto{\pgfqpoint{4.663218in}{1.973716in}}%
\pgfpathclose%
\pgfusepath{fill}%
\end{pgfscope}%
\begin{pgfscope}%
\pgfpathrectangle{\pgfqpoint{1.150000in}{0.150000in}}{\pgfqpoint{5.700000in}{5.700000in}}%
\pgfusepath{clip}%
\pgfsetbuttcap%
\pgfsetroundjoin%
\definecolor{currentfill}{rgb}{0.278826,0.175490,0.483397}%
\pgfsetfillcolor{currentfill}%
\pgfsetfillopacity{0.700000}%
\pgfsetlinewidth{0.000000pt}%
\definecolor{currentstroke}{rgb}{0.000000,0.000000,0.000000}%
\pgfsetstrokecolor{currentstroke}%
\pgfsetdash{}{0pt}%
\pgfpathmoveto{\pgfqpoint{5.469243in}{2.214919in}}%
\pgfpathlineto{\pgfqpoint{5.483437in}{2.214256in}}%
\pgfpathlineto{\pgfqpoint{5.497639in}{2.213617in}}%
\pgfpathlineto{\pgfqpoint{5.511851in}{2.213004in}}%
\pgfpathlineto{\pgfqpoint{5.526071in}{2.212415in}}%
\pgfpathlineto{\pgfqpoint{5.518624in}{2.203780in}}%
\pgfpathlineto{\pgfqpoint{5.511169in}{2.195047in}}%
\pgfpathlineto{\pgfqpoint{5.503706in}{2.186214in}}%
\pgfpathlineto{\pgfqpoint{5.496235in}{2.177284in}}%
\pgfpathlineto{\pgfqpoint{5.482004in}{2.177915in}}%
\pgfpathlineto{\pgfqpoint{5.467782in}{2.178570in}}%
\pgfpathlineto{\pgfqpoint{5.453568in}{2.179250in}}%
\pgfpathlineto{\pgfqpoint{5.439364in}{2.179955in}}%
\pgfpathlineto{\pgfqpoint{5.446845in}{2.188839in}}%
\pgfpathlineto{\pgfqpoint{5.454319in}{2.197628in}}%
\pgfpathlineto{\pgfqpoint{5.461785in}{2.206321in}}%
\pgfpathlineto{\pgfqpoint{5.469243in}{2.214919in}}%
\pgfpathclose%
\pgfusepath{fill}%
\end{pgfscope}%
\begin{pgfscope}%
\pgfpathrectangle{\pgfqpoint{1.150000in}{0.150000in}}{\pgfqpoint{5.700000in}{5.700000in}}%
\pgfusepath{clip}%
\pgfsetbuttcap%
\pgfsetroundjoin%
\definecolor{currentfill}{rgb}{0.269308,0.218818,0.509577}%
\pgfsetfillcolor{currentfill}%
\pgfsetfillopacity{0.700000}%
\pgfsetlinewidth{0.000000pt}%
\definecolor{currentstroke}{rgb}{0.000000,0.000000,0.000000}%
\pgfsetstrokecolor{currentstroke}%
\pgfsetdash{}{0pt}%
\pgfpathmoveto{\pgfqpoint{5.785712in}{2.303575in}}%
\pgfpathlineto{\pgfqpoint{5.800014in}{2.303349in}}%
\pgfpathlineto{\pgfqpoint{5.814326in}{2.303146in}}%
\pgfpathlineto{\pgfqpoint{5.828647in}{2.302969in}}%
\pgfpathlineto{\pgfqpoint{5.821349in}{2.295552in}}%
\pgfpathlineto{\pgfqpoint{5.814041in}{2.288025in}}%
\pgfpathlineto{\pgfqpoint{5.806725in}{2.280385in}}%
\pgfpathlineto{\pgfqpoint{5.799400in}{2.272633in}}%
\pgfpathlineto{\pgfqpoint{5.785065in}{2.272810in}}%
\pgfpathlineto{\pgfqpoint{5.770740in}{2.273012in}}%
\pgfpathlineto{\pgfqpoint{5.756424in}{2.273239in}}%
\pgfpathlineto{\pgfqpoint{5.763760in}{2.280987in}}%
\pgfpathlineto{\pgfqpoint{5.771086in}{2.288626in}}%
\pgfpathlineto{\pgfqpoint{5.778403in}{2.296155in}}%
\pgfpathlineto{\pgfqpoint{5.785712in}{2.303575in}}%
\pgfpathclose%
\pgfusepath{fill}%
\end{pgfscope}%
\begin{pgfscope}%
\pgfpathrectangle{\pgfqpoint{1.150000in}{0.150000in}}{\pgfqpoint{5.700000in}{5.700000in}}%
\pgfusepath{clip}%
\pgfsetbuttcap%
\pgfsetroundjoin%
\definecolor{currentfill}{rgb}{0.282327,0.094955,0.417331}%
\pgfsetfillcolor{currentfill}%
\pgfsetfillopacity{0.700000}%
\pgfsetlinewidth{0.000000pt}%
\definecolor{currentstroke}{rgb}{0.000000,0.000000,0.000000}%
\pgfsetstrokecolor{currentstroke}%
\pgfsetdash{}{0pt}%
\pgfpathmoveto{\pgfqpoint{4.979437in}{2.057442in}}%
\pgfpathlineto{\pgfqpoint{4.993476in}{2.055839in}}%
\pgfpathlineto{\pgfqpoint{5.007523in}{2.054262in}}%
\pgfpathlineto{\pgfqpoint{5.021578in}{2.052709in}}%
\pgfpathlineto{\pgfqpoint{5.035641in}{2.051181in}}%
\pgfpathlineto{\pgfqpoint{5.027996in}{2.041399in}}%
\pgfpathlineto{\pgfqpoint{5.020344in}{2.031566in}}%
\pgfpathlineto{\pgfqpoint{5.012687in}{2.021686in}}%
\pgfpathlineto{\pgfqpoint{5.005024in}{2.011759in}}%
\pgfpathlineto{\pgfqpoint{4.990952in}{2.013397in}}%
\pgfpathlineto{\pgfqpoint{4.976888in}{2.015060in}}%
\pgfpathlineto{\pgfqpoint{4.962832in}{2.016748in}}%
\pgfpathlineto{\pgfqpoint{4.948784in}{2.018460in}}%
\pgfpathlineto{\pgfqpoint{4.956456in}{2.028271in}}%
\pgfpathlineto{\pgfqpoint{4.964122in}{2.038040in}}%
\pgfpathlineto{\pgfqpoint{4.971783in}{2.047765in}}%
\pgfpathlineto{\pgfqpoint{4.979437in}{2.057442in}}%
\pgfpathclose%
\pgfusepath{fill}%
\end{pgfscope}%
\begin{pgfscope}%
\pgfpathrectangle{\pgfqpoint{1.150000in}{0.150000in}}{\pgfqpoint{5.700000in}{5.700000in}}%
\pgfusepath{clip}%
\pgfsetbuttcap%
\pgfsetroundjoin%
\definecolor{currentfill}{rgb}{0.269944,0.014625,0.341379}%
\pgfsetfillcolor{currentfill}%
\pgfsetfillopacity{0.700000}%
\pgfsetlinewidth{0.000000pt}%
\definecolor{currentstroke}{rgb}{0.000000,0.000000,0.000000}%
\pgfsetstrokecolor{currentstroke}%
\pgfsetdash{}{0pt}%
\pgfpathmoveto{\pgfqpoint{3.833736in}{1.909781in}}%
\pgfpathlineto{\pgfqpoint{3.847475in}{1.904940in}}%
\pgfpathlineto{\pgfqpoint{3.861220in}{1.900126in}}%
\pgfpathlineto{\pgfqpoint{3.874971in}{1.895340in}}%
\pgfpathlineto{\pgfqpoint{3.888728in}{1.890580in}}%
\pgfpathlineto{\pgfqpoint{3.880677in}{1.885110in}}%
\pgfpathlineto{\pgfqpoint{3.872619in}{1.879807in}}%
\pgfpathlineto{\pgfqpoint{3.864553in}{1.874676in}}%
\pgfpathlineto{\pgfqpoint{3.856480in}{1.869724in}}%
\pgfpathlineto{\pgfqpoint{3.842705in}{1.874726in}}%
\pgfpathlineto{\pgfqpoint{3.828937in}{1.879755in}}%
\pgfpathlineto{\pgfqpoint{3.815174in}{1.884811in}}%
\pgfpathlineto{\pgfqpoint{3.801416in}{1.889894in}}%
\pgfpathlineto{\pgfqpoint{3.809508in}{1.894599in}}%
\pgfpathlineto{\pgfqpoint{3.817592in}{1.899485in}}%
\pgfpathlineto{\pgfqpoint{3.825668in}{1.904548in}}%
\pgfpathlineto{\pgfqpoint{3.833736in}{1.909781in}}%
\pgfpathclose%
\pgfusepath{fill}%
\end{pgfscope}%
\begin{pgfscope}%
\pgfpathrectangle{\pgfqpoint{1.150000in}{0.150000in}}{\pgfqpoint{5.700000in}{5.700000in}}%
\pgfusepath{clip}%
\pgfsetbuttcap%
\pgfsetroundjoin%
\definecolor{currentfill}{rgb}{0.276194,0.190074,0.493001}%
\pgfsetfillcolor{currentfill}%
\pgfsetfillopacity{0.700000}%
\pgfsetlinewidth{0.000000pt}%
\definecolor{currentstroke}{rgb}{0.000000,0.000000,0.000000}%
\pgfsetstrokecolor{currentstroke}%
\pgfsetdash{}{0pt}%
\pgfpathmoveto{\pgfqpoint{2.902632in}{2.239251in}}%
\pgfpathlineto{\pgfqpoint{2.916241in}{2.231300in}}%
\pgfpathlineto{\pgfqpoint{2.929853in}{2.223386in}}%
\pgfpathlineto{\pgfqpoint{2.943469in}{2.215509in}}%
\pgfpathlineto{\pgfqpoint{2.957088in}{2.207668in}}%
\pgfpathlineto{\pgfqpoint{2.948457in}{2.211117in}}%
\pgfpathlineto{\pgfqpoint{2.939808in}{2.214927in}}%
\pgfpathlineto{\pgfqpoint{2.931141in}{2.219104in}}%
\pgfpathlineto{\pgfqpoint{2.922455in}{2.223659in}}%
\pgfpathlineto{\pgfqpoint{2.908802in}{2.231814in}}%
\pgfpathlineto{\pgfqpoint{2.895152in}{2.240007in}}%
\pgfpathlineto{\pgfqpoint{2.881505in}{2.248236in}}%
\pgfpathlineto{\pgfqpoint{2.867861in}{2.256502in}}%
\pgfpathlineto{\pgfqpoint{2.876583in}{2.251626in}}%
\pgfpathlineto{\pgfqpoint{2.885284in}{2.247132in}}%
\pgfpathlineto{\pgfqpoint{2.893967in}{2.243009in}}%
\pgfpathlineto{\pgfqpoint{2.902632in}{2.239251in}}%
\pgfpathclose%
\pgfusepath{fill}%
\end{pgfscope}%
\begin{pgfscope}%
\pgfpathrectangle{\pgfqpoint{1.150000in}{0.150000in}}{\pgfqpoint{5.700000in}{5.700000in}}%
\pgfusepath{clip}%
\pgfsetbuttcap%
\pgfsetroundjoin%
\definecolor{currentfill}{rgb}{0.269944,0.014625,0.341379}%
\pgfsetfillcolor{currentfill}%
\pgfsetfillopacity{0.700000}%
\pgfsetlinewidth{0.000000pt}%
\definecolor{currentstroke}{rgb}{0.000000,0.000000,0.000000}%
\pgfsetstrokecolor{currentstroke}%
\pgfsetdash{}{0pt}%
\pgfpathmoveto{\pgfqpoint{4.347194in}{1.909956in}}%
\pgfpathlineto{\pgfqpoint{4.361052in}{1.906691in}}%
\pgfpathlineto{\pgfqpoint{4.374917in}{1.903451in}}%
\pgfpathlineto{\pgfqpoint{4.388789in}{1.900236in}}%
\pgfpathlineto{\pgfqpoint{4.402668in}{1.897047in}}%
\pgfpathlineto{\pgfqpoint{4.394814in}{1.888375in}}%
\pgfpathlineto{\pgfqpoint{4.386955in}{1.879762in}}%
\pgfpathlineto{\pgfqpoint{4.379091in}{1.871211in}}%
\pgfpathlineto{\pgfqpoint{4.371221in}{1.862728in}}%
\pgfpathlineto{\pgfqpoint{4.357331in}{1.866107in}}%
\pgfpathlineto{\pgfqpoint{4.343447in}{1.869511in}}%
\pgfpathlineto{\pgfqpoint{4.329570in}{1.872941in}}%
\pgfpathlineto{\pgfqpoint{4.315700in}{1.876396in}}%
\pgfpathlineto{\pgfqpoint{4.323582in}{1.884684in}}%
\pgfpathlineto{\pgfqpoint{4.331458in}{1.893043in}}%
\pgfpathlineto{\pgfqpoint{4.339329in}{1.901469in}}%
\pgfpathlineto{\pgfqpoint{4.347194in}{1.909956in}}%
\pgfpathclose%
\pgfusepath{fill}%
\end{pgfscope}%
\begin{pgfscope}%
\pgfpathrectangle{\pgfqpoint{1.150000in}{0.150000in}}{\pgfqpoint{5.700000in}{5.700000in}}%
\pgfusepath{clip}%
\pgfsetbuttcap%
\pgfsetroundjoin%
\definecolor{currentfill}{rgb}{0.268510,0.009605,0.335427}%
\pgfsetfillcolor{currentfill}%
\pgfsetfillopacity{0.700000}%
\pgfsetlinewidth{0.000000pt}%
\definecolor{currentstroke}{rgb}{0.000000,0.000000,0.000000}%
\pgfsetstrokecolor{currentstroke}%
\pgfsetdash{}{0pt}%
\pgfpathmoveto{\pgfqpoint{3.975876in}{1.896157in}}%
\pgfpathlineto{\pgfqpoint{3.989645in}{1.891762in}}%
\pgfpathlineto{\pgfqpoint{4.003421in}{1.887393in}}%
\pgfpathlineto{\pgfqpoint{4.017203in}{1.883051in}}%
\pgfpathlineto{\pgfqpoint{4.030991in}{1.878735in}}%
\pgfpathlineto{\pgfqpoint{4.023001in}{1.872210in}}%
\pgfpathlineto{\pgfqpoint{4.015004in}{1.865821in}}%
\pgfpathlineto{\pgfqpoint{4.007001in}{1.859574in}}%
\pgfpathlineto{\pgfqpoint{3.998991in}{1.853475in}}%
\pgfpathlineto{\pgfqpoint{3.985187in}{1.858020in}}%
\pgfpathlineto{\pgfqpoint{3.971389in}{1.862591in}}%
\pgfpathlineto{\pgfqpoint{3.957598in}{1.867189in}}%
\pgfpathlineto{\pgfqpoint{3.943812in}{1.871814in}}%
\pgfpathlineto{\pgfqpoint{3.951838in}{1.877678in}}%
\pgfpathlineto{\pgfqpoint{3.959858in}{1.883694in}}%
\pgfpathlineto{\pgfqpoint{3.967870in}{1.889856in}}%
\pgfpathlineto{\pgfqpoint{3.975876in}{1.896157in}}%
\pgfpathclose%
\pgfusepath{fill}%
\end{pgfscope}%
\begin{pgfscope}%
\pgfpathrectangle{\pgfqpoint{1.150000in}{0.150000in}}{\pgfqpoint{5.700000in}{5.700000in}}%
\pgfusepath{clip}%
\pgfsetbuttcap%
\pgfsetroundjoin%
\definecolor{currentfill}{rgb}{0.280868,0.160771,0.472899}%
\pgfsetfillcolor{currentfill}%
\pgfsetfillopacity{0.700000}%
\pgfsetlinewidth{0.000000pt}%
\definecolor{currentstroke}{rgb}{0.000000,0.000000,0.000000}%
\pgfsetstrokecolor{currentstroke}%
\pgfsetdash{}{0pt}%
\pgfpathmoveto{\pgfqpoint{5.382635in}{2.183021in}}%
\pgfpathlineto{\pgfqpoint{5.396804in}{2.182218in}}%
\pgfpathlineto{\pgfqpoint{5.410982in}{2.181439in}}%
\pgfpathlineto{\pgfqpoint{5.425168in}{2.180685in}}%
\pgfpathlineto{\pgfqpoint{5.439364in}{2.179955in}}%
\pgfpathlineto{\pgfqpoint{5.431875in}{2.170978in}}%
\pgfpathlineto{\pgfqpoint{5.424379in}{2.161907in}}%
\pgfpathlineto{\pgfqpoint{5.416875in}{2.152744in}}%
\pgfpathlineto{\pgfqpoint{5.409363in}{2.143490in}}%
\pgfpathlineto{\pgfqpoint{5.395157in}{2.144275in}}%
\pgfpathlineto{\pgfqpoint{5.380960in}{2.145085in}}%
\pgfpathlineto{\pgfqpoint{5.366772in}{2.145919in}}%
\pgfpathlineto{\pgfqpoint{5.352593in}{2.146778in}}%
\pgfpathlineto{\pgfqpoint{5.360115in}{2.155972in}}%
\pgfpathlineto{\pgfqpoint{5.367629in}{2.165077in}}%
\pgfpathlineto{\pgfqpoint{5.375136in}{2.174094in}}%
\pgfpathlineto{\pgfqpoint{5.382635in}{2.183021in}}%
\pgfpathclose%
\pgfusepath{fill}%
\end{pgfscope}%
\begin{pgfscope}%
\pgfpathrectangle{\pgfqpoint{1.150000in}{0.150000in}}{\pgfqpoint{5.700000in}{5.700000in}}%
\pgfusepath{clip}%
\pgfsetbuttcap%
\pgfsetroundjoin%
\definecolor{currentfill}{rgb}{0.273809,0.031497,0.358853}%
\pgfsetfillcolor{currentfill}%
\pgfsetfillopacity{0.700000}%
\pgfsetlinewidth{0.000000pt}%
\definecolor{currentstroke}{rgb}{0.000000,0.000000,0.000000}%
\pgfsetstrokecolor{currentstroke}%
\pgfsetdash{}{0pt}%
\pgfpathmoveto{\pgfqpoint{3.691555in}{1.931558in}}%
\pgfpathlineto{\pgfqpoint{3.705268in}{1.926252in}}%
\pgfpathlineto{\pgfqpoint{3.718987in}{1.920975in}}%
\pgfpathlineto{\pgfqpoint{3.732712in}{1.915725in}}%
\pgfpathlineto{\pgfqpoint{3.746442in}{1.910503in}}%
\pgfpathlineto{\pgfqpoint{3.738323in}{1.906239in}}%
\pgfpathlineto{\pgfqpoint{3.730195in}{1.902173in}}%
\pgfpathlineto{\pgfqpoint{3.722058in}{1.898312in}}%
\pgfpathlineto{\pgfqpoint{3.713913in}{1.894663in}}%
\pgfpathlineto{\pgfqpoint{3.700163in}{1.900140in}}%
\pgfpathlineto{\pgfqpoint{3.686418in}{1.905646in}}%
\pgfpathlineto{\pgfqpoint{3.672679in}{1.911179in}}%
\pgfpathlineto{\pgfqpoint{3.658945in}{1.916741in}}%
\pgfpathlineto{\pgfqpoint{3.667111in}{1.920129in}}%
\pgfpathlineto{\pgfqpoint{3.675268in}{1.923732in}}%
\pgfpathlineto{\pgfqpoint{3.683416in}{1.927544in}}%
\pgfpathlineto{\pgfqpoint{3.691555in}{1.931558in}}%
\pgfpathclose%
\pgfusepath{fill}%
\end{pgfscope}%
\begin{pgfscope}%
\pgfpathrectangle{\pgfqpoint{1.150000in}{0.150000in}}{\pgfqpoint{5.700000in}{5.700000in}}%
\pgfusepath{clip}%
\pgfsetbuttcap%
\pgfsetroundjoin%
\definecolor{currentfill}{rgb}{0.274952,0.037752,0.364543}%
\pgfsetfillcolor{currentfill}%
\pgfsetfillopacity{0.700000}%
\pgfsetlinewidth{0.000000pt}%
\definecolor{currentstroke}{rgb}{0.000000,0.000000,0.000000}%
\pgfsetstrokecolor{currentstroke}%
\pgfsetdash{}{0pt}%
\pgfpathmoveto{\pgfqpoint{4.576419in}{1.946129in}}%
\pgfpathlineto{\pgfqpoint{4.590341in}{1.943507in}}%
\pgfpathlineto{\pgfqpoint{4.604270in}{1.940911in}}%
\pgfpathlineto{\pgfqpoint{4.618206in}{1.938339in}}%
\pgfpathlineto{\pgfqpoint{4.632150in}{1.935793in}}%
\pgfpathlineto{\pgfqpoint{4.624370in}{1.926332in}}%
\pgfpathlineto{\pgfqpoint{4.616585in}{1.916886in}}%
\pgfpathlineto{\pgfqpoint{4.608794in}{1.907459in}}%
\pgfpathlineto{\pgfqpoint{4.600999in}{1.898055in}}%
\pgfpathlineto{\pgfqpoint{4.587045in}{1.900765in}}%
\pgfpathlineto{\pgfqpoint{4.573098in}{1.903500in}}%
\pgfpathlineto{\pgfqpoint{4.559159in}{1.906260in}}%
\pgfpathlineto{\pgfqpoint{4.545227in}{1.909044in}}%
\pgfpathlineto{\pgfqpoint{4.553033in}{1.918280in}}%
\pgfpathlineto{\pgfqpoint{4.560833in}{1.927542in}}%
\pgfpathlineto{\pgfqpoint{4.568629in}{1.936826in}}%
\pgfpathlineto{\pgfqpoint{4.576419in}{1.946129in}}%
\pgfpathclose%
\pgfusepath{fill}%
\end{pgfscope}%
\begin{pgfscope}%
\pgfpathrectangle{\pgfqpoint{1.150000in}{0.150000in}}{\pgfqpoint{5.700000in}{5.700000in}}%
\pgfusepath{clip}%
\pgfsetbuttcap%
\pgfsetroundjoin%
\definecolor{currentfill}{rgb}{0.281924,0.089666,0.412415}%
\pgfsetfillcolor{currentfill}%
\pgfsetfillopacity{0.700000}%
\pgfsetlinewidth{0.000000pt}%
\definecolor{currentstroke}{rgb}{0.000000,0.000000,0.000000}%
\pgfsetstrokecolor{currentstroke}%
\pgfsetdash{}{0pt}%
\pgfpathmoveto{\pgfqpoint{3.352095in}{2.028209in}}%
\pgfpathlineto{\pgfqpoint{3.365755in}{2.021788in}}%
\pgfpathlineto{\pgfqpoint{3.379419in}{2.015398in}}%
\pgfpathlineto{\pgfqpoint{3.393088in}{2.009038in}}%
\pgfpathlineto{\pgfqpoint{3.406762in}{2.002709in}}%
\pgfpathlineto{\pgfqpoint{3.398451in}{2.001615in}}%
\pgfpathlineto{\pgfqpoint{3.390128in}{2.000792in}}%
\pgfpathlineto{\pgfqpoint{3.381793in}{2.000248in}}%
\pgfpathlineto{\pgfqpoint{3.373445in}{1.999990in}}%
\pgfpathlineto{\pgfqpoint{3.359746in}{2.006603in}}%
\pgfpathlineto{\pgfqpoint{3.346051in}{2.013247in}}%
\pgfpathlineto{\pgfqpoint{3.332361in}{2.019921in}}%
\pgfpathlineto{\pgfqpoint{3.318675in}{2.026627in}}%
\pgfpathlineto{\pgfqpoint{3.327049in}{2.026595in}}%
\pgfpathlineto{\pgfqpoint{3.335410in}{2.026853in}}%
\pgfpathlineto{\pgfqpoint{3.343759in}{2.027394in}}%
\pgfpathlineto{\pgfqpoint{3.352095in}{2.028209in}}%
\pgfpathclose%
\pgfusepath{fill}%
\end{pgfscope}%
\begin{pgfscope}%
\pgfpathrectangle{\pgfqpoint{1.150000in}{0.150000in}}{\pgfqpoint{5.700000in}{5.700000in}}%
\pgfusepath{clip}%
\pgfsetbuttcap%
\pgfsetroundjoin%
\definecolor{currentfill}{rgb}{0.280894,0.078907,0.402329}%
\pgfsetfillcolor{currentfill}%
\pgfsetfillopacity{0.700000}%
\pgfsetlinewidth{0.000000pt}%
\definecolor{currentstroke}{rgb}{0.000000,0.000000,0.000000}%
\pgfsetstrokecolor{currentstroke}%
\pgfsetdash{}{0pt}%
\pgfpathmoveto{\pgfqpoint{4.892673in}{2.025558in}}%
\pgfpathlineto{\pgfqpoint{4.906689in}{2.023746in}}%
\pgfpathlineto{\pgfqpoint{4.920713in}{2.021959in}}%
\pgfpathlineto{\pgfqpoint{4.934744in}{2.020197in}}%
\pgfpathlineto{\pgfqpoint{4.948784in}{2.018460in}}%
\pgfpathlineto{\pgfqpoint{4.941106in}{2.008609in}}%
\pgfpathlineto{\pgfqpoint{4.933423in}{1.998720in}}%
\pgfpathlineto{\pgfqpoint{4.925733in}{1.988798in}}%
\pgfpathlineto{\pgfqpoint{4.918038in}{1.978844in}}%
\pgfpathlineto{\pgfqpoint{4.903990in}{1.980704in}}%
\pgfpathlineto{\pgfqpoint{4.889949in}{1.982590in}}%
\pgfpathlineto{\pgfqpoint{4.875916in}{1.984500in}}%
\pgfpathlineto{\pgfqpoint{4.861891in}{1.986435in}}%
\pgfpathlineto{\pgfqpoint{4.869595in}{1.996261in}}%
\pgfpathlineto{\pgfqpoint{4.877293in}{2.006058in}}%
\pgfpathlineto{\pgfqpoint{4.884986in}{2.015825in}}%
\pgfpathlineto{\pgfqpoint{4.892673in}{2.025558in}}%
\pgfpathclose%
\pgfusepath{fill}%
\end{pgfscope}%
\begin{pgfscope}%
\pgfpathrectangle{\pgfqpoint{1.150000in}{0.150000in}}{\pgfqpoint{5.700000in}{5.700000in}}%
\pgfusepath{clip}%
\pgfsetbuttcap%
\pgfsetroundjoin%
\definecolor{currentfill}{rgb}{0.203063,0.379716,0.553925}%
\pgfsetfillcolor{currentfill}%
\pgfsetfillopacity{0.700000}%
\pgfsetlinewidth{0.000000pt}%
\definecolor{currentstroke}{rgb}{0.000000,0.000000,0.000000}%
\pgfsetstrokecolor{currentstroke}%
\pgfsetdash{}{0pt}%
\pgfpathmoveto{\pgfqpoint{2.287452in}{2.658295in}}%
\pgfpathlineto{\pgfqpoint{2.301052in}{2.647911in}}%
\pgfpathlineto{\pgfqpoint{2.314654in}{2.637581in}}%
\pgfpathlineto{\pgfqpoint{2.328256in}{2.627304in}}%
\pgfpathlineto{\pgfqpoint{2.341860in}{2.617080in}}%
\pgfpathlineto{\pgfqpoint{2.332658in}{2.627083in}}%
\pgfpathlineto{\pgfqpoint{2.323428in}{2.637559in}}%
\pgfpathlineto{\pgfqpoint{2.314169in}{2.648516in}}%
\pgfpathlineto{\pgfqpoint{2.304881in}{2.659965in}}%
\pgfpathlineto{\pgfqpoint{2.291231in}{2.670543in}}%
\pgfpathlineto{\pgfqpoint{2.277581in}{2.681173in}}%
\pgfpathlineto{\pgfqpoint{2.263933in}{2.691857in}}%
\pgfpathlineto{\pgfqpoint{2.250286in}{2.702595in}}%
\pgfpathlineto{\pgfqpoint{2.259622in}{2.690786in}}%
\pgfpathlineto{\pgfqpoint{2.268928in}{2.679473in}}%
\pgfpathlineto{\pgfqpoint{2.278205in}{2.668646in}}%
\pgfpathlineto{\pgfqpoint{2.287452in}{2.658295in}}%
\pgfpathclose%
\pgfusepath{fill}%
\end{pgfscope}%
\begin{pgfscope}%
\pgfpathrectangle{\pgfqpoint{1.150000in}{0.150000in}}{\pgfqpoint{5.700000in}{5.700000in}}%
\pgfusepath{clip}%
\pgfsetbuttcap%
\pgfsetroundjoin%
\definecolor{currentfill}{rgb}{0.267004,0.004874,0.329415}%
\pgfsetfillcolor{currentfill}%
\pgfsetfillopacity{0.700000}%
\pgfsetlinewidth{0.000000pt}%
\definecolor{currentstroke}{rgb}{0.000000,0.000000,0.000000}%
\pgfsetstrokecolor{currentstroke}%
\pgfsetdash{}{0pt}%
\pgfpathmoveto{\pgfqpoint{4.118040in}{1.889954in}}%
\pgfpathlineto{\pgfqpoint{4.131844in}{1.885986in}}%
\pgfpathlineto{\pgfqpoint{4.145655in}{1.882044in}}%
\pgfpathlineto{\pgfqpoint{4.159472in}{1.878128in}}%
\pgfpathlineto{\pgfqpoint{4.173296in}{1.874238in}}%
\pgfpathlineto{\pgfqpoint{4.165360in}{1.866801in}}%
\pgfpathlineto{\pgfqpoint{4.157418in}{1.859472in}}%
\pgfpathlineto{\pgfqpoint{4.149470in}{1.852255in}}%
\pgfpathlineto{\pgfqpoint{4.141516in}{1.845157in}}%
\pgfpathlineto{\pgfqpoint{4.127679in}{1.849263in}}%
\pgfpathlineto{\pgfqpoint{4.113847in}{1.853395in}}%
\pgfpathlineto{\pgfqpoint{4.100022in}{1.857553in}}%
\pgfpathlineto{\pgfqpoint{4.086204in}{1.861737in}}%
\pgfpathlineto{\pgfqpoint{4.094172in}{1.868614in}}%
\pgfpathlineto{\pgfqpoint{4.102134in}{1.875612in}}%
\pgfpathlineto{\pgfqpoint{4.110090in}{1.882728in}}%
\pgfpathlineto{\pgfqpoint{4.118040in}{1.889954in}}%
\pgfpathclose%
\pgfusepath{fill}%
\end{pgfscope}%
\begin{pgfscope}%
\pgfpathrectangle{\pgfqpoint{1.150000in}{0.150000in}}{\pgfqpoint{5.700000in}{5.700000in}}%
\pgfusepath{clip}%
\pgfsetbuttcap%
\pgfsetroundjoin%
\definecolor{currentfill}{rgb}{0.283072,0.130895,0.449241}%
\pgfsetfillcolor{currentfill}%
\pgfsetfillopacity{0.700000}%
\pgfsetlinewidth{0.000000pt}%
\definecolor{currentstroke}{rgb}{0.000000,0.000000,0.000000}%
\pgfsetstrokecolor{currentstroke}%
\pgfsetdash{}{0pt}%
\pgfpathmoveto{\pgfqpoint{3.154773in}{2.109565in}}%
\pgfpathlineto{\pgfqpoint{3.168409in}{2.102475in}}%
\pgfpathlineto{\pgfqpoint{3.182049in}{2.095418in}}%
\pgfpathlineto{\pgfqpoint{3.195693in}{2.088394in}}%
\pgfpathlineto{\pgfqpoint{3.209341in}{2.081403in}}%
\pgfpathlineto{\pgfqpoint{3.200897in}{2.082319in}}%
\pgfpathlineto{\pgfqpoint{3.192439in}{2.083547in}}%
\pgfpathlineto{\pgfqpoint{3.183966in}{2.085097in}}%
\pgfpathlineto{\pgfqpoint{3.175478in}{2.086974in}}%
\pgfpathlineto{\pgfqpoint{3.161801in}{2.094265in}}%
\pgfpathlineto{\pgfqpoint{3.148127in}{2.101588in}}%
\pgfpathlineto{\pgfqpoint{3.134458in}{2.108944in}}%
\pgfpathlineto{\pgfqpoint{3.120792in}{2.116333in}}%
\pgfpathlineto{\pgfqpoint{3.129310in}{2.114150in}}%
\pgfpathlineto{\pgfqpoint{3.137813in}{2.112300in}}%
\pgfpathlineto{\pgfqpoint{3.146300in}{2.110775in}}%
\pgfpathlineto{\pgfqpoint{3.154773in}{2.109565in}}%
\pgfpathclose%
\pgfusepath{fill}%
\end{pgfscope}%
\begin{pgfscope}%
\pgfpathrectangle{\pgfqpoint{1.150000in}{0.150000in}}{\pgfqpoint{5.700000in}{5.700000in}}%
\pgfusepath{clip}%
\pgfsetbuttcap%
\pgfsetroundjoin%
\definecolor{currentfill}{rgb}{0.253935,0.265254,0.529983}%
\pgfsetfillcolor{currentfill}%
\pgfsetfillopacity{0.700000}%
\pgfsetlinewidth{0.000000pt}%
\definecolor{currentstroke}{rgb}{0.000000,0.000000,0.000000}%
\pgfsetstrokecolor{currentstroke}%
\pgfsetdash{}{0pt}%
\pgfpathmoveto{\pgfqpoint{2.649950in}{2.394050in}}%
\pgfpathlineto{\pgfqpoint{2.663550in}{2.385149in}}%
\pgfpathlineto{\pgfqpoint{2.677152in}{2.376289in}}%
\pgfpathlineto{\pgfqpoint{2.690757in}{2.367472in}}%
\pgfpathlineto{\pgfqpoint{2.704364in}{2.358696in}}%
\pgfpathlineto{\pgfqpoint{2.695511in}{2.364938in}}%
\pgfpathlineto{\pgfqpoint{2.686637in}{2.371590in}}%
\pgfpathlineto{\pgfqpoint{2.677740in}{2.378661in}}%
\pgfpathlineto{\pgfqpoint{2.668820in}{2.386161in}}%
\pgfpathlineto{\pgfqpoint{2.655173in}{2.395269in}}%
\pgfpathlineto{\pgfqpoint{2.641529in}{2.404419in}}%
\pgfpathlineto{\pgfqpoint{2.627888in}{2.413611in}}%
\pgfpathlineto{\pgfqpoint{2.614248in}{2.422845in}}%
\pgfpathlineto{\pgfqpoint{2.623209in}{2.415007in}}%
\pgfpathlineto{\pgfqpoint{2.632146in}{2.407601in}}%
\pgfpathlineto{\pgfqpoint{2.641059in}{2.400619in}}%
\pgfpathlineto{\pgfqpoint{2.649950in}{2.394050in}}%
\pgfpathclose%
\pgfusepath{fill}%
\end{pgfscope}%
\begin{pgfscope}%
\pgfpathrectangle{\pgfqpoint{1.150000in}{0.150000in}}{\pgfqpoint{5.700000in}{5.700000in}}%
\pgfusepath{clip}%
\pgfsetbuttcap%
\pgfsetroundjoin%
\definecolor{currentfill}{rgb}{0.282290,0.145912,0.461510}%
\pgfsetfillcolor{currentfill}%
\pgfsetfillopacity{0.700000}%
\pgfsetlinewidth{0.000000pt}%
\definecolor{currentstroke}{rgb}{0.000000,0.000000,0.000000}%
\pgfsetstrokecolor{currentstroke}%
\pgfsetdash{}{0pt}%
\pgfpathmoveto{\pgfqpoint{5.295963in}{2.150461in}}%
\pgfpathlineto{\pgfqpoint{5.310108in}{2.149503in}}%
\pgfpathlineto{\pgfqpoint{5.324261in}{2.148570in}}%
\pgfpathlineto{\pgfqpoint{5.338423in}{2.147662in}}%
\pgfpathlineto{\pgfqpoint{5.352593in}{2.146778in}}%
\pgfpathlineto{\pgfqpoint{5.345064in}{2.137498in}}%
\pgfpathlineto{\pgfqpoint{5.337529in}{2.128132in}}%
\pgfpathlineto{\pgfqpoint{5.329986in}{2.118682in}}%
\pgfpathlineto{\pgfqpoint{5.322436in}{2.109149in}}%
\pgfpathlineto{\pgfqpoint{5.308255in}{2.110102in}}%
\pgfpathlineto{\pgfqpoint{5.294084in}{2.111080in}}%
\pgfpathlineto{\pgfqpoint{5.279921in}{2.112082in}}%
\pgfpathlineto{\pgfqpoint{5.265767in}{2.113109in}}%
\pgfpathlineto{\pgfqpoint{5.273326in}{2.122568in}}%
\pgfpathlineto{\pgfqpoint{5.280879in}{2.131947in}}%
\pgfpathlineto{\pgfqpoint{5.288425in}{2.141245in}}%
\pgfpathlineto{\pgfqpoint{5.295963in}{2.150461in}}%
\pgfpathclose%
\pgfusepath{fill}%
\end{pgfscope}%
\begin{pgfscope}%
\pgfpathrectangle{\pgfqpoint{1.150000in}{0.150000in}}{\pgfqpoint{5.700000in}{5.700000in}}%
\pgfusepath{clip}%
\pgfsetbuttcap%
\pgfsetroundjoin%
\definecolor{currentfill}{rgb}{0.271828,0.209303,0.504434}%
\pgfsetfillcolor{currentfill}%
\pgfsetfillopacity{0.700000}%
\pgfsetlinewidth{0.000000pt}%
\definecolor{currentstroke}{rgb}{0.000000,0.000000,0.000000}%
\pgfsetstrokecolor{currentstroke}%
\pgfsetdash{}{0pt}%
\pgfpathmoveto{\pgfqpoint{5.699255in}{2.274392in}}%
\pgfpathlineto{\pgfqpoint{5.713533in}{2.274067in}}%
\pgfpathlineto{\pgfqpoint{5.727821in}{2.273766in}}%
\pgfpathlineto{\pgfqpoint{5.742118in}{2.273490in}}%
\pgfpathlineto{\pgfqpoint{5.756424in}{2.273239in}}%
\pgfpathlineto{\pgfqpoint{5.749080in}{2.265380in}}%
\pgfpathlineto{\pgfqpoint{5.741727in}{2.257411in}}%
\pgfpathlineto{\pgfqpoint{5.734366in}{2.249332in}}%
\pgfpathlineto{\pgfqpoint{5.726995in}{2.241141in}}%
\pgfpathlineto{\pgfqpoint{5.712677in}{2.241406in}}%
\pgfpathlineto{\pgfqpoint{5.698367in}{2.241696in}}%
\pgfpathlineto{\pgfqpoint{5.684067in}{2.242011in}}%
\pgfpathlineto{\pgfqpoint{5.669776in}{2.242350in}}%
\pgfpathlineto{\pgfqpoint{5.677159in}{2.250521in}}%
\pgfpathlineto{\pgfqpoint{5.684533in}{2.258586in}}%
\pgfpathlineto{\pgfqpoint{5.691898in}{2.266543in}}%
\pgfpathlineto{\pgfqpoint{5.699255in}{2.274392in}}%
\pgfpathclose%
\pgfusepath{fill}%
\end{pgfscope}%
\begin{pgfscope}%
\pgfpathrectangle{\pgfqpoint{1.150000in}{0.150000in}}{\pgfqpoint{5.700000in}{5.700000in}}%
\pgfusepath{clip}%
\pgfsetbuttcap%
\pgfsetroundjoin%
\definecolor{currentfill}{rgb}{0.277018,0.050344,0.375715}%
\pgfsetfillcolor{currentfill}%
\pgfsetfillopacity{0.700000}%
\pgfsetlinewidth{0.000000pt}%
\definecolor{currentstroke}{rgb}{0.000000,0.000000,0.000000}%
\pgfsetstrokecolor{currentstroke}%
\pgfsetdash{}{0pt}%
\pgfpathmoveto{\pgfqpoint{3.549258in}{1.962262in}}%
\pgfpathlineto{\pgfqpoint{3.562951in}{1.956471in}}%
\pgfpathlineto{\pgfqpoint{3.576649in}{1.950709in}}%
\pgfpathlineto{\pgfqpoint{3.590352in}{1.944976in}}%
\pgfpathlineto{\pgfqpoint{3.604061in}{1.939272in}}%
\pgfpathlineto{\pgfqpoint{3.595863in}{1.936370in}}%
\pgfpathlineto{\pgfqpoint{3.587656in}{1.933701in}}%
\pgfpathlineto{\pgfqpoint{3.579439in}{1.931271in}}%
\pgfpathlineto{\pgfqpoint{3.571211in}{1.929087in}}%
\pgfpathlineto{\pgfqpoint{3.557481in}{1.935061in}}%
\pgfpathlineto{\pgfqpoint{3.543755in}{1.941064in}}%
\pgfpathlineto{\pgfqpoint{3.530034in}{1.947096in}}%
\pgfpathlineto{\pgfqpoint{3.516318in}{1.953157in}}%
\pgfpathlineto{\pgfqpoint{3.524569in}{1.955066in}}%
\pgfpathlineto{\pgfqpoint{3.532809in}{1.957224in}}%
\pgfpathlineto{\pgfqpoint{3.541038in}{1.959625in}}%
\pgfpathlineto{\pgfqpoint{3.549258in}{1.962262in}}%
\pgfpathclose%
\pgfusepath{fill}%
\end{pgfscope}%
\begin{pgfscope}%
\pgfpathrectangle{\pgfqpoint{1.150000in}{0.150000in}}{\pgfqpoint{5.700000in}{5.700000in}}%
\pgfusepath{clip}%
\pgfsetbuttcap%
\pgfsetroundjoin%
\definecolor{currentfill}{rgb}{0.279566,0.067836,0.391917}%
\pgfsetfillcolor{currentfill}%
\pgfsetfillopacity{0.700000}%
\pgfsetlinewidth{0.000000pt}%
\definecolor{currentstroke}{rgb}{0.000000,0.000000,0.000000}%
\pgfsetstrokecolor{currentstroke}%
\pgfsetdash{}{0pt}%
\pgfpathmoveto{\pgfqpoint{4.805871in}{1.994424in}}%
\pgfpathlineto{\pgfqpoint{4.819864in}{1.992390in}}%
\pgfpathlineto{\pgfqpoint{4.833865in}{1.990380in}}%
\pgfpathlineto{\pgfqpoint{4.847874in}{1.988395in}}%
\pgfpathlineto{\pgfqpoint{4.861891in}{1.986435in}}%
\pgfpathlineto{\pgfqpoint{4.854182in}{1.976584in}}%
\pgfpathlineto{\pgfqpoint{4.846467in}{1.966711in}}%
\pgfpathlineto{\pgfqpoint{4.838747in}{1.956819in}}%
\pgfpathlineto{\pgfqpoint{4.831021in}{1.946910in}}%
\pgfpathlineto{\pgfqpoint{4.816995in}{1.949007in}}%
\pgfpathlineto{\pgfqpoint{4.802977in}{1.951129in}}%
\pgfpathlineto{\pgfqpoint{4.788966in}{1.953275in}}%
\pgfpathlineto{\pgfqpoint{4.774964in}{1.955447in}}%
\pgfpathlineto{\pgfqpoint{4.782699in}{1.965213in}}%
\pgfpathlineto{\pgfqpoint{4.790428in}{1.974967in}}%
\pgfpathlineto{\pgfqpoint{4.798152in}{1.984705in}}%
\pgfpathlineto{\pgfqpoint{4.805871in}{1.994424in}}%
\pgfpathclose%
\pgfusepath{fill}%
\end{pgfscope}%
\begin{pgfscope}%
\pgfpathrectangle{\pgfqpoint{1.150000in}{0.150000in}}{\pgfqpoint{5.700000in}{5.700000in}}%
\pgfusepath{clip}%
\pgfsetbuttcap%
\pgfsetroundjoin%
\definecolor{currentfill}{rgb}{0.272594,0.025563,0.353093}%
\pgfsetfillcolor{currentfill}%
\pgfsetfillopacity{0.700000}%
\pgfsetlinewidth{0.000000pt}%
\definecolor{currentstroke}{rgb}{0.000000,0.000000,0.000000}%
\pgfsetstrokecolor{currentstroke}%
\pgfsetdash{}{0pt}%
\pgfpathmoveto{\pgfqpoint{4.489571in}{1.920436in}}%
\pgfpathlineto{\pgfqpoint{4.503474in}{1.917550in}}%
\pgfpathlineto{\pgfqpoint{4.517385in}{1.914690in}}%
\pgfpathlineto{\pgfqpoint{4.531302in}{1.911854in}}%
\pgfpathlineto{\pgfqpoint{4.545227in}{1.909044in}}%
\pgfpathlineto{\pgfqpoint{4.537416in}{1.899839in}}%
\pgfpathlineto{\pgfqpoint{4.529600in}{1.890669in}}%
\pgfpathlineto{\pgfqpoint{4.521778in}{1.881537in}}%
\pgfpathlineto{\pgfqpoint{4.513952in}{1.872447in}}%
\pgfpathlineto{\pgfqpoint{4.500017in}{1.875434in}}%
\pgfpathlineto{\pgfqpoint{4.486088in}{1.878446in}}%
\pgfpathlineto{\pgfqpoint{4.472167in}{1.881483in}}%
\pgfpathlineto{\pgfqpoint{4.458254in}{1.884545in}}%
\pgfpathlineto{\pgfqpoint{4.466091in}{1.893453in}}%
\pgfpathlineto{\pgfqpoint{4.473923in}{1.902407in}}%
\pgfpathlineto{\pgfqpoint{4.481750in}{1.911402in}}%
\pgfpathlineto{\pgfqpoint{4.489571in}{1.920436in}}%
\pgfpathclose%
\pgfusepath{fill}%
\end{pgfscope}%
\begin{pgfscope}%
\pgfpathrectangle{\pgfqpoint{1.150000in}{0.150000in}}{\pgfqpoint{5.700000in}{5.700000in}}%
\pgfusepath{clip}%
\pgfsetbuttcap%
\pgfsetroundjoin%
\definecolor{currentfill}{rgb}{0.283072,0.130895,0.449241}%
\pgfsetfillcolor{currentfill}%
\pgfsetfillopacity{0.700000}%
\pgfsetlinewidth{0.000000pt}%
\definecolor{currentstroke}{rgb}{0.000000,0.000000,0.000000}%
\pgfsetstrokecolor{currentstroke}%
\pgfsetdash{}{0pt}%
\pgfpathmoveto{\pgfqpoint{5.209236in}{2.117465in}}%
\pgfpathlineto{\pgfqpoint{5.223356in}{2.116339in}}%
\pgfpathlineto{\pgfqpoint{5.237484in}{2.115238in}}%
\pgfpathlineto{\pgfqpoint{5.251621in}{2.114161in}}%
\pgfpathlineto{\pgfqpoint{5.265767in}{2.113109in}}%
\pgfpathlineto{\pgfqpoint{5.258200in}{2.103573in}}%
\pgfpathlineto{\pgfqpoint{5.250627in}{2.093960in}}%
\pgfpathlineto{\pgfqpoint{5.243047in}{2.084272in}}%
\pgfpathlineto{\pgfqpoint{5.235460in}{2.074511in}}%
\pgfpathlineto{\pgfqpoint{5.221306in}{2.075646in}}%
\pgfpathlineto{\pgfqpoint{5.207159in}{2.076805in}}%
\pgfpathlineto{\pgfqpoint{5.193022in}{2.077989in}}%
\pgfpathlineto{\pgfqpoint{5.178892in}{2.079198in}}%
\pgfpathlineto{\pgfqpoint{5.186488in}{2.088871in}}%
\pgfpathlineto{\pgfqpoint{5.194077in}{2.098474in}}%
\pgfpathlineto{\pgfqpoint{5.201660in}{2.108006in}}%
\pgfpathlineto{\pgfqpoint{5.209236in}{2.117465in}}%
\pgfpathclose%
\pgfusepath{fill}%
\end{pgfscope}%
\begin{pgfscope}%
\pgfpathrectangle{\pgfqpoint{1.150000in}{0.150000in}}{\pgfqpoint{5.700000in}{5.700000in}}%
\pgfusepath{clip}%
\pgfsetbuttcap%
\pgfsetroundjoin%
\definecolor{currentfill}{rgb}{0.268510,0.009605,0.335427}%
\pgfsetfillcolor{currentfill}%
\pgfsetfillopacity{0.700000}%
\pgfsetlinewidth{0.000000pt}%
\definecolor{currentstroke}{rgb}{0.000000,0.000000,0.000000}%
\pgfsetstrokecolor{currentstroke}%
\pgfsetdash{}{0pt}%
\pgfpathmoveto{\pgfqpoint{4.260287in}{1.890472in}}%
\pgfpathlineto{\pgfqpoint{4.274130in}{1.886915in}}%
\pgfpathlineto{\pgfqpoint{4.287980in}{1.883383in}}%
\pgfpathlineto{\pgfqpoint{4.301837in}{1.879876in}}%
\pgfpathlineto{\pgfqpoint{4.315700in}{1.876396in}}%
\pgfpathlineto{\pgfqpoint{4.307813in}{1.868183in}}%
\pgfpathlineto{\pgfqpoint{4.299921in}{1.860052in}}%
\pgfpathlineto{\pgfqpoint{4.292022in}{1.852006in}}%
\pgfpathlineto{\pgfqpoint{4.284119in}{1.844051in}}%
\pgfpathlineto{\pgfqpoint{4.270243in}{1.847734in}}%
\pgfpathlineto{\pgfqpoint{4.256374in}{1.851443in}}%
\pgfpathlineto{\pgfqpoint{4.242511in}{1.855178in}}%
\pgfpathlineto{\pgfqpoint{4.228655in}{1.858938in}}%
\pgfpathlineto{\pgfqpoint{4.236571in}{1.866686in}}%
\pgfpathlineto{\pgfqpoint{4.244482in}{1.874527in}}%
\pgfpathlineto{\pgfqpoint{4.252388in}{1.882458in}}%
\pgfpathlineto{\pgfqpoint{4.260287in}{1.890472in}}%
\pgfpathclose%
\pgfusepath{fill}%
\end{pgfscope}%
\begin{pgfscope}%
\pgfpathrectangle{\pgfqpoint{1.150000in}{0.150000in}}{\pgfqpoint{5.700000in}{5.700000in}}%
\pgfusepath{clip}%
\pgfsetbuttcap%
\pgfsetroundjoin%
\definecolor{currentfill}{rgb}{0.278012,0.180367,0.486697}%
\pgfsetfillcolor{currentfill}%
\pgfsetfillopacity{0.700000}%
\pgfsetlinewidth{0.000000pt}%
\definecolor{currentstroke}{rgb}{0.000000,0.000000,0.000000}%
\pgfsetstrokecolor{currentstroke}%
\pgfsetdash{}{0pt}%
\pgfpathmoveto{\pgfqpoint{2.957088in}{2.207668in}}%
\pgfpathlineto{\pgfqpoint{2.970711in}{2.199864in}}%
\pgfpathlineto{\pgfqpoint{2.984337in}{2.192095in}}%
\pgfpathlineto{\pgfqpoint{2.997966in}{2.184363in}}%
\pgfpathlineto{\pgfqpoint{3.011599in}{2.176665in}}%
\pgfpathlineto{\pgfqpoint{3.003001in}{2.179805in}}%
\pgfpathlineto{\pgfqpoint{2.994385in}{2.183301in}}%
\pgfpathlineto{\pgfqpoint{2.985752in}{2.187163in}}%
\pgfpathlineto{\pgfqpoint{2.977101in}{2.191397in}}%
\pgfpathlineto{\pgfqpoint{2.963435in}{2.199409in}}%
\pgfpathlineto{\pgfqpoint{2.949772in}{2.207456in}}%
\pgfpathlineto{\pgfqpoint{2.936112in}{2.215540in}}%
\pgfpathlineto{\pgfqpoint{2.922455in}{2.223659in}}%
\pgfpathlineto{\pgfqpoint{2.931141in}{2.219104in}}%
\pgfpathlineto{\pgfqpoint{2.939808in}{2.214927in}}%
\pgfpathlineto{\pgfqpoint{2.948457in}{2.211117in}}%
\pgfpathlineto{\pgfqpoint{2.957088in}{2.207668in}}%
\pgfpathclose%
\pgfusepath{fill}%
\end{pgfscope}%
\begin{pgfscope}%
\pgfpathrectangle{\pgfqpoint{1.150000in}{0.150000in}}{\pgfqpoint{5.700000in}{5.700000in}}%
\pgfusepath{clip}%
\pgfsetbuttcap%
\pgfsetroundjoin%
\definecolor{currentfill}{rgb}{0.210503,0.363727,0.552206}%
\pgfsetfillcolor{currentfill}%
\pgfsetfillopacity{0.700000}%
\pgfsetlinewidth{0.000000pt}%
\definecolor{currentstroke}{rgb}{0.000000,0.000000,0.000000}%
\pgfsetstrokecolor{currentstroke}%
\pgfsetdash{}{0pt}%
\pgfpathmoveto{\pgfqpoint{2.341860in}{2.617080in}}%
\pgfpathlineto{\pgfqpoint{2.355464in}{2.606907in}}%
\pgfpathlineto{\pgfqpoint{2.369070in}{2.596786in}}%
\pgfpathlineto{\pgfqpoint{2.382677in}{2.586716in}}%
\pgfpathlineto{\pgfqpoint{2.396286in}{2.576697in}}%
\pgfpathlineto{\pgfqpoint{2.387129in}{2.586354in}}%
\pgfpathlineto{\pgfqpoint{2.377944in}{2.596479in}}%
\pgfpathlineto{\pgfqpoint{2.368732in}{2.607082in}}%
\pgfpathlineto{\pgfqpoint{2.359491in}{2.618172in}}%
\pgfpathlineto{\pgfqpoint{2.345837in}{2.628544in}}%
\pgfpathlineto{\pgfqpoint{2.332183in}{2.638967in}}%
\pgfpathlineto{\pgfqpoint{2.318531in}{2.649440in}}%
\pgfpathlineto{\pgfqpoint{2.304881in}{2.659965in}}%
\pgfpathlineto{\pgfqpoint{2.314169in}{2.648516in}}%
\pgfpathlineto{\pgfqpoint{2.323428in}{2.637559in}}%
\pgfpathlineto{\pgfqpoint{2.332658in}{2.627083in}}%
\pgfpathlineto{\pgfqpoint{2.341860in}{2.617080in}}%
\pgfpathclose%
\pgfusepath{fill}%
\end{pgfscope}%
\begin{pgfscope}%
\pgfpathrectangle{\pgfqpoint{1.150000in}{0.150000in}}{\pgfqpoint{5.700000in}{5.700000in}}%
\pgfusepath{clip}%
\pgfsetbuttcap%
\pgfsetroundjoin%
\definecolor{currentfill}{rgb}{0.275191,0.194905,0.496005}%
\pgfsetfillcolor{currentfill}%
\pgfsetfillopacity{0.700000}%
\pgfsetlinewidth{0.000000pt}%
\definecolor{currentstroke}{rgb}{0.000000,0.000000,0.000000}%
\pgfsetstrokecolor{currentstroke}%
\pgfsetdash{}{0pt}%
\pgfpathmoveto{\pgfqpoint{5.612705in}{2.243953in}}%
\pgfpathlineto{\pgfqpoint{5.626959in}{2.243516in}}%
\pgfpathlineto{\pgfqpoint{5.641222in}{2.243102in}}%
\pgfpathlineto{\pgfqpoint{5.655494in}{2.242714in}}%
\pgfpathlineto{\pgfqpoint{5.669776in}{2.242350in}}%
\pgfpathlineto{\pgfqpoint{5.662385in}{2.234071in}}%
\pgfpathlineto{\pgfqpoint{5.654985in}{2.225684in}}%
\pgfpathlineto{\pgfqpoint{5.647577in}{2.217191in}}%
\pgfpathlineto{\pgfqpoint{5.640161in}{2.208590in}}%
\pgfpathlineto{\pgfqpoint{5.625867in}{2.208982in}}%
\pgfpathlineto{\pgfqpoint{5.611583in}{2.209398in}}%
\pgfpathlineto{\pgfqpoint{5.597309in}{2.209839in}}%
\pgfpathlineto{\pgfqpoint{5.583043in}{2.210305in}}%
\pgfpathlineto{\pgfqpoint{5.590471in}{2.218873in}}%
\pgfpathlineto{\pgfqpoint{5.597890in}{2.227337in}}%
\pgfpathlineto{\pgfqpoint{5.605302in}{2.235697in}}%
\pgfpathlineto{\pgfqpoint{5.612705in}{2.243953in}}%
\pgfpathclose%
\pgfusepath{fill}%
\end{pgfscope}%
\begin{pgfscope}%
\pgfpathrectangle{\pgfqpoint{1.150000in}{0.150000in}}{\pgfqpoint{5.700000in}{5.700000in}}%
\pgfusepath{clip}%
\pgfsetbuttcap%
\pgfsetroundjoin%
\definecolor{currentfill}{rgb}{0.283197,0.115680,0.436115}%
\pgfsetfillcolor{currentfill}%
\pgfsetfillopacity{0.700000}%
\pgfsetlinewidth{0.000000pt}%
\definecolor{currentstroke}{rgb}{0.000000,0.000000,0.000000}%
\pgfsetstrokecolor{currentstroke}%
\pgfsetdash{}{0pt}%
\pgfpathmoveto{\pgfqpoint{5.122460in}{2.084281in}}%
\pgfpathlineto{\pgfqpoint{5.136555in}{2.082973in}}%
\pgfpathlineto{\pgfqpoint{5.150659in}{2.081690in}}%
\pgfpathlineto{\pgfqpoint{5.164772in}{2.080432in}}%
\pgfpathlineto{\pgfqpoint{5.178892in}{2.079198in}}%
\pgfpathlineto{\pgfqpoint{5.171290in}{2.069457in}}%
\pgfpathlineto{\pgfqpoint{5.163681in}{2.059651in}}%
\pgfpathlineto{\pgfqpoint{5.156066in}{2.049780in}}%
\pgfpathlineto{\pgfqpoint{5.148445in}{2.039847in}}%
\pgfpathlineto{\pgfqpoint{5.134315in}{2.041178in}}%
\pgfpathlineto{\pgfqpoint{5.120194in}{2.042533in}}%
\pgfpathlineto{\pgfqpoint{5.106081in}{2.043912in}}%
\pgfpathlineto{\pgfqpoint{5.091976in}{2.045316in}}%
\pgfpathlineto{\pgfqpoint{5.099607in}{2.055147in}}%
\pgfpathlineto{\pgfqpoint{5.107231in}{2.064920in}}%
\pgfpathlineto{\pgfqpoint{5.114848in}{2.074632in}}%
\pgfpathlineto{\pgfqpoint{5.122460in}{2.084281in}}%
\pgfpathclose%
\pgfusepath{fill}%
\end{pgfscope}%
\begin{pgfscope}%
\pgfpathrectangle{\pgfqpoint{1.150000in}{0.150000in}}{\pgfqpoint{5.700000in}{5.700000in}}%
\pgfusepath{clip}%
\pgfsetbuttcap%
\pgfsetroundjoin%
\definecolor{currentfill}{rgb}{0.277941,0.056324,0.381191}%
\pgfsetfillcolor{currentfill}%
\pgfsetfillopacity{0.700000}%
\pgfsetlinewidth{0.000000pt}%
\definecolor{currentstroke}{rgb}{0.000000,0.000000,0.000000}%
\pgfsetstrokecolor{currentstroke}%
\pgfsetdash{}{0pt}%
\pgfpathmoveto{\pgfqpoint{4.719030in}{1.964382in}}%
\pgfpathlineto{\pgfqpoint{4.733002in}{1.962111in}}%
\pgfpathlineto{\pgfqpoint{4.746982in}{1.959864in}}%
\pgfpathlineto{\pgfqpoint{4.760969in}{1.957643in}}%
\pgfpathlineto{\pgfqpoint{4.774964in}{1.955447in}}%
\pgfpathlineto{\pgfqpoint{4.767224in}{1.945671in}}%
\pgfpathlineto{\pgfqpoint{4.759478in}{1.935890in}}%
\pgfpathlineto{\pgfqpoint{4.751728in}{1.926106in}}%
\pgfpathlineto{\pgfqpoint{4.743972in}{1.916323in}}%
\pgfpathlineto{\pgfqpoint{4.729968in}{1.918669in}}%
\pgfpathlineto{\pgfqpoint{4.715971in}{1.921041in}}%
\pgfpathlineto{\pgfqpoint{4.701982in}{1.923437in}}%
\pgfpathlineto{\pgfqpoint{4.688001in}{1.925858in}}%
\pgfpathlineto{\pgfqpoint{4.695766in}{1.935486in}}%
\pgfpathlineto{\pgfqpoint{4.703526in}{1.945118in}}%
\pgfpathlineto{\pgfqpoint{4.711281in}{1.954751in}}%
\pgfpathlineto{\pgfqpoint{4.719030in}{1.964382in}}%
\pgfpathclose%
\pgfusepath{fill}%
\end{pgfscope}%
\begin{pgfscope}%
\pgfpathrectangle{\pgfqpoint{1.150000in}{0.150000in}}{\pgfqpoint{5.700000in}{5.700000in}}%
\pgfusepath{clip}%
\pgfsetbuttcap%
\pgfsetroundjoin%
\definecolor{currentfill}{rgb}{0.257322,0.256130,0.526563}%
\pgfsetfillcolor{currentfill}%
\pgfsetfillopacity{0.700000}%
\pgfsetlinewidth{0.000000pt}%
\definecolor{currentstroke}{rgb}{0.000000,0.000000,0.000000}%
\pgfsetstrokecolor{currentstroke}%
\pgfsetdash{}{0pt}%
\pgfpathmoveto{\pgfqpoint{2.704364in}{2.358696in}}%
\pgfpathlineto{\pgfqpoint{2.717973in}{2.349961in}}%
\pgfpathlineto{\pgfqpoint{2.731586in}{2.341267in}}%
\pgfpathlineto{\pgfqpoint{2.745200in}{2.332614in}}%
\pgfpathlineto{\pgfqpoint{2.758818in}{2.324000in}}%
\pgfpathlineto{\pgfqpoint{2.750004in}{2.329916in}}%
\pgfpathlineto{\pgfqpoint{2.741168in}{2.336238in}}%
\pgfpathlineto{\pgfqpoint{2.732310in}{2.342976in}}%
\pgfpathlineto{\pgfqpoint{2.723430in}{2.350138in}}%
\pgfpathlineto{\pgfqpoint{2.709774in}{2.359083in}}%
\pgfpathlineto{\pgfqpoint{2.696120in}{2.368068in}}%
\pgfpathlineto{\pgfqpoint{2.682469in}{2.377094in}}%
\pgfpathlineto{\pgfqpoint{2.668820in}{2.386161in}}%
\pgfpathlineto{\pgfqpoint{2.677740in}{2.378661in}}%
\pgfpathlineto{\pgfqpoint{2.686637in}{2.371590in}}%
\pgfpathlineto{\pgfqpoint{2.695511in}{2.364938in}}%
\pgfpathlineto{\pgfqpoint{2.704364in}{2.358696in}}%
\pgfpathclose%
\pgfusepath{fill}%
\end{pgfscope}%
\begin{pgfscope}%
\pgfpathrectangle{\pgfqpoint{1.150000in}{0.150000in}}{\pgfqpoint{5.700000in}{5.700000in}}%
\pgfusepath{clip}%
\pgfsetbuttcap%
\pgfsetroundjoin%
\definecolor{currentfill}{rgb}{0.269944,0.014625,0.341379}%
\pgfsetfillcolor{currentfill}%
\pgfsetfillopacity{0.700000}%
\pgfsetlinewidth{0.000000pt}%
\definecolor{currentstroke}{rgb}{0.000000,0.000000,0.000000}%
\pgfsetstrokecolor{currentstroke}%
\pgfsetdash{}{0pt}%
\pgfpathmoveto{\pgfqpoint{3.888728in}{1.890580in}}%
\pgfpathlineto{\pgfqpoint{3.902490in}{1.885848in}}%
\pgfpathlineto{\pgfqpoint{3.916258in}{1.881143in}}%
\pgfpathlineto{\pgfqpoint{3.930032in}{1.876465in}}%
\pgfpathlineto{\pgfqpoint{3.943812in}{1.871814in}}%
\pgfpathlineto{\pgfqpoint{3.935778in}{1.866106in}}%
\pgfpathlineto{\pgfqpoint{3.927738in}{1.860562in}}%
\pgfpathlineto{\pgfqpoint{3.919689in}{1.855187in}}%
\pgfpathlineto{\pgfqpoint{3.911634in}{1.849987in}}%
\pgfpathlineto{\pgfqpoint{3.897837in}{1.854881in}}%
\pgfpathlineto{\pgfqpoint{3.884045in}{1.859802in}}%
\pgfpathlineto{\pgfqpoint{3.870260in}{1.864750in}}%
\pgfpathlineto{\pgfqpoint{3.856480in}{1.869724in}}%
\pgfpathlineto{\pgfqpoint{3.864553in}{1.874676in}}%
\pgfpathlineto{\pgfqpoint{3.872619in}{1.879807in}}%
\pgfpathlineto{\pgfqpoint{3.880677in}{1.885110in}}%
\pgfpathlineto{\pgfqpoint{3.888728in}{1.890580in}}%
\pgfpathclose%
\pgfusepath{fill}%
\end{pgfscope}%
\begin{pgfscope}%
\pgfpathrectangle{\pgfqpoint{1.150000in}{0.150000in}}{\pgfqpoint{5.700000in}{5.700000in}}%
\pgfusepath{clip}%
\pgfsetbuttcap%
\pgfsetroundjoin%
\definecolor{currentfill}{rgb}{0.280894,0.078907,0.402329}%
\pgfsetfillcolor{currentfill}%
\pgfsetfillopacity{0.700000}%
\pgfsetlinewidth{0.000000pt}%
\definecolor{currentstroke}{rgb}{0.000000,0.000000,0.000000}%
\pgfsetstrokecolor{currentstroke}%
\pgfsetdash{}{0pt}%
\pgfpathmoveto{\pgfqpoint{3.406762in}{2.002709in}}%
\pgfpathlineto{\pgfqpoint{3.420440in}{1.996410in}}%
\pgfpathlineto{\pgfqpoint{3.434122in}{1.990142in}}%
\pgfpathlineto{\pgfqpoint{3.447810in}{1.983903in}}%
\pgfpathlineto{\pgfqpoint{3.461502in}{1.977695in}}%
\pgfpathlineto{\pgfqpoint{3.453216in}{1.976322in}}%
\pgfpathlineto{\pgfqpoint{3.444918in}{1.975216in}}%
\pgfpathlineto{\pgfqpoint{3.436609in}{1.974387in}}%
\pgfpathlineto{\pgfqpoint{3.428287in}{1.973839in}}%
\pgfpathlineto{\pgfqpoint{3.414570in}{1.980332in}}%
\pgfpathlineto{\pgfqpoint{3.400857in}{1.986854in}}%
\pgfpathlineto{\pgfqpoint{3.387149in}{1.993407in}}%
\pgfpathlineto{\pgfqpoint{3.373445in}{1.999990in}}%
\pgfpathlineto{\pgfqpoint{3.381793in}{2.000248in}}%
\pgfpathlineto{\pgfqpoint{3.390128in}{2.000792in}}%
\pgfpathlineto{\pgfqpoint{3.398451in}{2.001615in}}%
\pgfpathlineto{\pgfqpoint{3.406762in}{2.002709in}}%
\pgfpathclose%
\pgfusepath{fill}%
\end{pgfscope}%
\begin{pgfscope}%
\pgfpathrectangle{\pgfqpoint{1.150000in}{0.150000in}}{\pgfqpoint{5.700000in}{5.700000in}}%
\pgfusepath{clip}%
\pgfsetbuttcap%
\pgfsetroundjoin%
\definecolor{currentfill}{rgb}{0.283229,0.120777,0.440584}%
\pgfsetfillcolor{currentfill}%
\pgfsetfillopacity{0.700000}%
\pgfsetlinewidth{0.000000pt}%
\definecolor{currentstroke}{rgb}{0.000000,0.000000,0.000000}%
\pgfsetstrokecolor{currentstroke}%
\pgfsetdash{}{0pt}%
\pgfpathmoveto{\pgfqpoint{3.209341in}{2.081403in}}%
\pgfpathlineto{\pgfqpoint{3.222993in}{2.074444in}}%
\pgfpathlineto{\pgfqpoint{3.236649in}{2.067518in}}%
\pgfpathlineto{\pgfqpoint{3.250309in}{2.060623in}}%
\pgfpathlineto{\pgfqpoint{3.263974in}{2.053761in}}%
\pgfpathlineto{\pgfqpoint{3.255559in}{2.054383in}}%
\pgfpathlineto{\pgfqpoint{3.247130in}{2.055315in}}%
\pgfpathlineto{\pgfqpoint{3.238686in}{2.056564in}}%
\pgfpathlineto{\pgfqpoint{3.230228in}{2.058137in}}%
\pgfpathlineto{\pgfqpoint{3.216535in}{2.065298in}}%
\pgfpathlineto{\pgfqpoint{3.202845in}{2.072491in}}%
\pgfpathlineto{\pgfqpoint{3.189160in}{2.079717in}}%
\pgfpathlineto{\pgfqpoint{3.175478in}{2.086974in}}%
\pgfpathlineto{\pgfqpoint{3.183966in}{2.085097in}}%
\pgfpathlineto{\pgfqpoint{3.192439in}{2.083547in}}%
\pgfpathlineto{\pgfqpoint{3.200897in}{2.082319in}}%
\pgfpathlineto{\pgfqpoint{3.209341in}{2.081403in}}%
\pgfpathclose%
\pgfusepath{fill}%
\end{pgfscope}%
\begin{pgfscope}%
\pgfpathrectangle{\pgfqpoint{1.150000in}{0.150000in}}{\pgfqpoint{5.700000in}{5.700000in}}%
\pgfusepath{clip}%
\pgfsetbuttcap%
\pgfsetroundjoin%
\definecolor{currentfill}{rgb}{0.268510,0.009605,0.335427}%
\pgfsetfillcolor{currentfill}%
\pgfsetfillopacity{0.700000}%
\pgfsetlinewidth{0.000000pt}%
\definecolor{currentstroke}{rgb}{0.000000,0.000000,0.000000}%
\pgfsetstrokecolor{currentstroke}%
\pgfsetdash{}{0pt}%
\pgfpathmoveto{\pgfqpoint{4.030991in}{1.878735in}}%
\pgfpathlineto{\pgfqpoint{4.044785in}{1.874446in}}%
\pgfpathlineto{\pgfqpoint{4.058585in}{1.870183in}}%
\pgfpathlineto{\pgfqpoint{4.072391in}{1.865947in}}%
\pgfpathlineto{\pgfqpoint{4.086204in}{1.861737in}}%
\pgfpathlineto{\pgfqpoint{4.078229in}{1.854987in}}%
\pgfpathlineto{\pgfqpoint{4.070248in}{1.848371in}}%
\pgfpathlineto{\pgfqpoint{4.062260in}{1.841893in}}%
\pgfpathlineto{\pgfqpoint{4.054266in}{1.835559in}}%
\pgfpathlineto{\pgfqpoint{4.040438in}{1.839999in}}%
\pgfpathlineto{\pgfqpoint{4.026616in}{1.844465in}}%
\pgfpathlineto{\pgfqpoint{4.012800in}{1.848957in}}%
\pgfpathlineto{\pgfqpoint{3.998991in}{1.853475in}}%
\pgfpathlineto{\pgfqpoint{4.007001in}{1.859574in}}%
\pgfpathlineto{\pgfqpoint{4.015004in}{1.865821in}}%
\pgfpathlineto{\pgfqpoint{4.023001in}{1.872210in}}%
\pgfpathlineto{\pgfqpoint{4.030991in}{1.878735in}}%
\pgfpathclose%
\pgfusepath{fill}%
\end{pgfscope}%
\begin{pgfscope}%
\pgfpathrectangle{\pgfqpoint{1.150000in}{0.150000in}}{\pgfqpoint{5.700000in}{5.700000in}}%
\pgfusepath{clip}%
\pgfsetbuttcap%
\pgfsetroundjoin%
\definecolor{currentfill}{rgb}{0.272594,0.025563,0.353093}%
\pgfsetfillcolor{currentfill}%
\pgfsetfillopacity{0.700000}%
\pgfsetlinewidth{0.000000pt}%
\definecolor{currentstroke}{rgb}{0.000000,0.000000,0.000000}%
\pgfsetstrokecolor{currentstroke}%
\pgfsetdash{}{0pt}%
\pgfpathmoveto{\pgfqpoint{3.746442in}{1.910503in}}%
\pgfpathlineto{\pgfqpoint{3.760177in}{1.905310in}}%
\pgfpathlineto{\pgfqpoint{3.773918in}{1.900144in}}%
\pgfpathlineto{\pgfqpoint{3.787664in}{1.895005in}}%
\pgfpathlineto{\pgfqpoint{3.801416in}{1.889894in}}%
\pgfpathlineto{\pgfqpoint{3.793316in}{1.885379in}}%
\pgfpathlineto{\pgfqpoint{3.785208in}{1.881058in}}%
\pgfpathlineto{\pgfqpoint{3.777091in}{1.876940in}}%
\pgfpathlineto{\pgfqpoint{3.768966in}{1.873029in}}%
\pgfpathlineto{\pgfqpoint{3.755195in}{1.878396in}}%
\pgfpathlineto{\pgfqpoint{3.741429in}{1.883791in}}%
\pgfpathlineto{\pgfqpoint{3.727668in}{1.889213in}}%
\pgfpathlineto{\pgfqpoint{3.713913in}{1.894663in}}%
\pgfpathlineto{\pgfqpoint{3.722058in}{1.898312in}}%
\pgfpathlineto{\pgfqpoint{3.730195in}{1.902173in}}%
\pgfpathlineto{\pgfqpoint{3.738323in}{1.906239in}}%
\pgfpathlineto{\pgfqpoint{3.746442in}{1.910503in}}%
\pgfpathclose%
\pgfusepath{fill}%
\end{pgfscope}%
\begin{pgfscope}%
\pgfpathrectangle{\pgfqpoint{1.150000in}{0.150000in}}{\pgfqpoint{5.700000in}{5.700000in}}%
\pgfusepath{clip}%
\pgfsetbuttcap%
\pgfsetroundjoin%
\definecolor{currentfill}{rgb}{0.277134,0.185228,0.489898}%
\pgfsetfillcolor{currentfill}%
\pgfsetfillopacity{0.700000}%
\pgfsetlinewidth{0.000000pt}%
\definecolor{currentstroke}{rgb}{0.000000,0.000000,0.000000}%
\pgfsetstrokecolor{currentstroke}%
\pgfsetdash{}{0pt}%
\pgfpathmoveto{\pgfqpoint{5.526071in}{2.212415in}}%
\pgfpathlineto{\pgfqpoint{5.540300in}{2.211850in}}%
\pgfpathlineto{\pgfqpoint{5.554539in}{2.211310in}}%
\pgfpathlineto{\pgfqpoint{5.568786in}{2.210795in}}%
\pgfpathlineto{\pgfqpoint{5.583043in}{2.210305in}}%
\pgfpathlineto{\pgfqpoint{5.575607in}{2.201634in}}%
\pgfpathlineto{\pgfqpoint{5.568163in}{2.192861in}}%
\pgfpathlineto{\pgfqpoint{5.560711in}{2.183985in}}%
\pgfpathlineto{\pgfqpoint{5.553251in}{2.175008in}}%
\pgfpathlineto{\pgfqpoint{5.538983in}{2.175540in}}%
\pgfpathlineto{\pgfqpoint{5.524725in}{2.176096in}}%
\pgfpathlineto{\pgfqpoint{5.510475in}{2.176678in}}%
\pgfpathlineto{\pgfqpoint{5.496235in}{2.177284in}}%
\pgfpathlineto{\pgfqpoint{5.503706in}{2.186214in}}%
\pgfpathlineto{\pgfqpoint{5.511169in}{2.195047in}}%
\pgfpathlineto{\pgfqpoint{5.518624in}{2.203780in}}%
\pgfpathlineto{\pgfqpoint{5.526071in}{2.212415in}}%
\pgfpathclose%
\pgfusepath{fill}%
\end{pgfscope}%
\begin{pgfscope}%
\pgfpathrectangle{\pgfqpoint{1.150000in}{0.150000in}}{\pgfqpoint{5.700000in}{5.700000in}}%
\pgfusepath{clip}%
\pgfsetbuttcap%
\pgfsetroundjoin%
\definecolor{currentfill}{rgb}{0.269944,0.014625,0.341379}%
\pgfsetfillcolor{currentfill}%
\pgfsetfillopacity{0.700000}%
\pgfsetlinewidth{0.000000pt}%
\definecolor{currentstroke}{rgb}{0.000000,0.000000,0.000000}%
\pgfsetstrokecolor{currentstroke}%
\pgfsetdash{}{0pt}%
\pgfpathmoveto{\pgfqpoint{4.402668in}{1.897047in}}%
\pgfpathlineto{\pgfqpoint{4.416554in}{1.893884in}}%
\pgfpathlineto{\pgfqpoint{4.430447in}{1.890746in}}%
\pgfpathlineto{\pgfqpoint{4.444347in}{1.887633in}}%
\pgfpathlineto{\pgfqpoint{4.458254in}{1.884545in}}%
\pgfpathlineto{\pgfqpoint{4.450411in}{1.875688in}}%
\pgfpathlineto{\pgfqpoint{4.442563in}{1.866887in}}%
\pgfpathlineto{\pgfqpoint{4.434711in}{1.858145in}}%
\pgfpathlineto{\pgfqpoint{4.426852in}{1.849467in}}%
\pgfpathlineto{\pgfqpoint{4.412934in}{1.852744in}}%
\pgfpathlineto{\pgfqpoint{4.399023in}{1.856047in}}%
\pgfpathlineto{\pgfqpoint{4.385119in}{1.859375in}}%
\pgfpathlineto{\pgfqpoint{4.371221in}{1.862728in}}%
\pgfpathlineto{\pgfqpoint{4.379091in}{1.871211in}}%
\pgfpathlineto{\pgfqpoint{4.386955in}{1.879762in}}%
\pgfpathlineto{\pgfqpoint{4.394814in}{1.888375in}}%
\pgfpathlineto{\pgfqpoint{4.402668in}{1.897047in}}%
\pgfpathclose%
\pgfusepath{fill}%
\end{pgfscope}%
\begin{pgfscope}%
\pgfpathrectangle{\pgfqpoint{1.150000in}{0.150000in}}{\pgfqpoint{5.700000in}{5.700000in}}%
\pgfusepath{clip}%
\pgfsetbuttcap%
\pgfsetroundjoin%
\definecolor{currentfill}{rgb}{0.282910,0.105393,0.426902}%
\pgfsetfillcolor{currentfill}%
\pgfsetfillopacity{0.700000}%
\pgfsetlinewidth{0.000000pt}%
\definecolor{currentstroke}{rgb}{0.000000,0.000000,0.000000}%
\pgfsetstrokecolor{currentstroke}%
\pgfsetdash{}{0pt}%
\pgfpathmoveto{\pgfqpoint{5.035641in}{2.051181in}}%
\pgfpathlineto{\pgfqpoint{5.049713in}{2.049678in}}%
\pgfpathlineto{\pgfqpoint{5.063792in}{2.048199in}}%
\pgfpathlineto{\pgfqpoint{5.077880in}{2.046746in}}%
\pgfpathlineto{\pgfqpoint{5.091976in}{2.045316in}}%
\pgfpathlineto{\pgfqpoint{5.084340in}{2.035429in}}%
\pgfpathlineto{\pgfqpoint{5.076698in}{2.025488in}}%
\pgfpathlineto{\pgfqpoint{5.069049in}{2.015496in}}%
\pgfpathlineto{\pgfqpoint{5.061395in}{2.005454in}}%
\pgfpathlineto{\pgfqpoint{5.047290in}{2.006993in}}%
\pgfpathlineto{\pgfqpoint{5.033193in}{2.008557in}}%
\pgfpathlineto{\pgfqpoint{5.019104in}{2.010146in}}%
\pgfpathlineto{\pgfqpoint{5.005024in}{2.011759in}}%
\pgfpathlineto{\pgfqpoint{5.012687in}{2.021686in}}%
\pgfpathlineto{\pgfqpoint{5.020344in}{2.031566in}}%
\pgfpathlineto{\pgfqpoint{5.027996in}{2.041399in}}%
\pgfpathlineto{\pgfqpoint{5.035641in}{2.051181in}}%
\pgfpathclose%
\pgfusepath{fill}%
\end{pgfscope}%
\begin{pgfscope}%
\pgfpathrectangle{\pgfqpoint{1.150000in}{0.150000in}}{\pgfqpoint{5.700000in}{5.700000in}}%
\pgfusepath{clip}%
\pgfsetbuttcap%
\pgfsetroundjoin%
\definecolor{currentfill}{rgb}{0.216210,0.351535,0.550627}%
\pgfsetfillcolor{currentfill}%
\pgfsetfillopacity{0.700000}%
\pgfsetlinewidth{0.000000pt}%
\definecolor{currentstroke}{rgb}{0.000000,0.000000,0.000000}%
\pgfsetstrokecolor{currentstroke}%
\pgfsetdash{}{0pt}%
\pgfpathmoveto{\pgfqpoint{2.396286in}{2.576697in}}%
\pgfpathlineto{\pgfqpoint{2.409896in}{2.566727in}}%
\pgfpathlineto{\pgfqpoint{2.423507in}{2.556807in}}%
\pgfpathlineto{\pgfqpoint{2.437120in}{2.546935in}}%
\pgfpathlineto{\pgfqpoint{2.450734in}{2.537112in}}%
\pgfpathlineto{\pgfqpoint{2.441622in}{2.546424in}}%
\pgfpathlineto{\pgfqpoint{2.432482in}{2.556200in}}%
\pgfpathlineto{\pgfqpoint{2.423315in}{2.566449in}}%
\pgfpathlineto{\pgfqpoint{2.414121in}{2.577181in}}%
\pgfpathlineto{\pgfqpoint{2.400461in}{2.587356in}}%
\pgfpathlineto{\pgfqpoint{2.386803in}{2.597579in}}%
\pgfpathlineto{\pgfqpoint{2.373146in}{2.607851in}}%
\pgfpathlineto{\pgfqpoint{2.359491in}{2.618172in}}%
\pgfpathlineto{\pgfqpoint{2.368732in}{2.607082in}}%
\pgfpathlineto{\pgfqpoint{2.377944in}{2.596479in}}%
\pgfpathlineto{\pgfqpoint{2.387129in}{2.586354in}}%
\pgfpathlineto{\pgfqpoint{2.396286in}{2.576697in}}%
\pgfpathclose%
\pgfusepath{fill}%
\end{pgfscope}%
\begin{pgfscope}%
\pgfpathrectangle{\pgfqpoint{1.150000in}{0.150000in}}{\pgfqpoint{5.700000in}{5.700000in}}%
\pgfusepath{clip}%
\pgfsetbuttcap%
\pgfsetroundjoin%
\definecolor{currentfill}{rgb}{0.268510,0.009605,0.335427}%
\pgfsetfillcolor{currentfill}%
\pgfsetfillopacity{0.700000}%
\pgfsetlinewidth{0.000000pt}%
\definecolor{currentstroke}{rgb}{0.000000,0.000000,0.000000}%
\pgfsetstrokecolor{currentstroke}%
\pgfsetdash{}{0pt}%
\pgfpathmoveto{\pgfqpoint{4.173296in}{1.874238in}}%
\pgfpathlineto{\pgfqpoint{4.187126in}{1.870375in}}%
\pgfpathlineto{\pgfqpoint{4.200962in}{1.866537in}}%
\pgfpathlineto{\pgfqpoint{4.214805in}{1.862725in}}%
\pgfpathlineto{\pgfqpoint{4.228655in}{1.858938in}}%
\pgfpathlineto{\pgfqpoint{4.220732in}{1.851290in}}%
\pgfpathlineto{\pgfqpoint{4.212804in}{1.843746in}}%
\pgfpathlineto{\pgfqpoint{4.204870in}{1.836312in}}%
\pgfpathlineto{\pgfqpoint{4.196930in}{1.828993in}}%
\pgfpathlineto{\pgfqpoint{4.183067in}{1.832995in}}%
\pgfpathlineto{\pgfqpoint{4.169211in}{1.837024in}}%
\pgfpathlineto{\pgfqpoint{4.155360in}{1.841078in}}%
\pgfpathlineto{\pgfqpoint{4.141516in}{1.845157in}}%
\pgfpathlineto{\pgfqpoint{4.149470in}{1.852255in}}%
\pgfpathlineto{\pgfqpoint{4.157418in}{1.859472in}}%
\pgfpathlineto{\pgfqpoint{4.165360in}{1.866801in}}%
\pgfpathlineto{\pgfqpoint{4.173296in}{1.874238in}}%
\pgfpathclose%
\pgfusepath{fill}%
\end{pgfscope}%
\begin{pgfscope}%
\pgfpathrectangle{\pgfqpoint{1.150000in}{0.150000in}}{\pgfqpoint{5.700000in}{5.700000in}}%
\pgfusepath{clip}%
\pgfsetbuttcap%
\pgfsetroundjoin%
\definecolor{currentfill}{rgb}{0.274952,0.037752,0.364543}%
\pgfsetfillcolor{currentfill}%
\pgfsetfillopacity{0.700000}%
\pgfsetlinewidth{0.000000pt}%
\definecolor{currentstroke}{rgb}{0.000000,0.000000,0.000000}%
\pgfsetstrokecolor{currentstroke}%
\pgfsetdash{}{0pt}%
\pgfpathmoveto{\pgfqpoint{4.632150in}{1.935793in}}%
\pgfpathlineto{\pgfqpoint{4.646102in}{1.933272in}}%
\pgfpathlineto{\pgfqpoint{4.660060in}{1.930776in}}%
\pgfpathlineto{\pgfqpoint{4.674027in}{1.928304in}}%
\pgfpathlineto{\pgfqpoint{4.688001in}{1.925858in}}%
\pgfpathlineto{\pgfqpoint{4.680230in}{1.916239in}}%
\pgfpathlineto{\pgfqpoint{4.672455in}{1.906631in}}%
\pgfpathlineto{\pgfqpoint{4.664674in}{1.897039in}}%
\pgfpathlineto{\pgfqpoint{4.656888in}{1.887466in}}%
\pgfpathlineto{\pgfqpoint{4.642905in}{1.890076in}}%
\pgfpathlineto{\pgfqpoint{4.628929in}{1.892711in}}%
\pgfpathlineto{\pgfqpoint{4.614960in}{1.895371in}}%
\pgfpathlineto{\pgfqpoint{4.600999in}{1.898055in}}%
\pgfpathlineto{\pgfqpoint{4.608794in}{1.907459in}}%
\pgfpathlineto{\pgfqpoint{4.616585in}{1.916886in}}%
\pgfpathlineto{\pgfqpoint{4.624370in}{1.926332in}}%
\pgfpathlineto{\pgfqpoint{4.632150in}{1.935793in}}%
\pgfpathclose%
\pgfusepath{fill}%
\end{pgfscope}%
\begin{pgfscope}%
\pgfpathrectangle{\pgfqpoint{1.150000in}{0.150000in}}{\pgfqpoint{5.700000in}{5.700000in}}%
\pgfusepath{clip}%
\pgfsetbuttcap%
\pgfsetroundjoin%
\definecolor{currentfill}{rgb}{0.276022,0.044167,0.370164}%
\pgfsetfillcolor{currentfill}%
\pgfsetfillopacity{0.700000}%
\pgfsetlinewidth{0.000000pt}%
\definecolor{currentstroke}{rgb}{0.000000,0.000000,0.000000}%
\pgfsetstrokecolor{currentstroke}%
\pgfsetdash{}{0pt}%
\pgfpathmoveto{\pgfqpoint{3.604061in}{1.939272in}}%
\pgfpathlineto{\pgfqpoint{3.617774in}{1.933596in}}%
\pgfpathlineto{\pgfqpoint{3.631492in}{1.927949in}}%
\pgfpathlineto{\pgfqpoint{3.645216in}{1.922331in}}%
\pgfpathlineto{\pgfqpoint{3.658945in}{1.916741in}}%
\pgfpathlineto{\pgfqpoint{3.650769in}{1.913575in}}%
\pgfpathlineto{\pgfqpoint{3.642584in}{1.910638in}}%
\pgfpathlineto{\pgfqpoint{3.634389in}{1.907936in}}%
\pgfpathlineto{\pgfqpoint{3.626184in}{1.905477in}}%
\pgfpathlineto{\pgfqpoint{3.612433in}{1.911337in}}%
\pgfpathlineto{\pgfqpoint{3.598688in}{1.917225in}}%
\pgfpathlineto{\pgfqpoint{3.584947in}{1.923142in}}%
\pgfpathlineto{\pgfqpoint{3.571211in}{1.929087in}}%
\pgfpathlineto{\pgfqpoint{3.579439in}{1.931271in}}%
\pgfpathlineto{\pgfqpoint{3.587656in}{1.933701in}}%
\pgfpathlineto{\pgfqpoint{3.595863in}{1.936370in}}%
\pgfpathlineto{\pgfqpoint{3.604061in}{1.939272in}}%
\pgfpathclose%
\pgfusepath{fill}%
\end{pgfscope}%
\begin{pgfscope}%
\pgfpathrectangle{\pgfqpoint{1.150000in}{0.150000in}}{\pgfqpoint{5.700000in}{5.700000in}}%
\pgfusepath{clip}%
\pgfsetbuttcap%
\pgfsetroundjoin%
\definecolor{currentfill}{rgb}{0.279574,0.170599,0.479997}%
\pgfsetfillcolor{currentfill}%
\pgfsetfillopacity{0.700000}%
\pgfsetlinewidth{0.000000pt}%
\definecolor{currentstroke}{rgb}{0.000000,0.000000,0.000000}%
\pgfsetstrokecolor{currentstroke}%
\pgfsetdash{}{0pt}%
\pgfpathmoveto{\pgfqpoint{5.439364in}{2.179955in}}%
\pgfpathlineto{\pgfqpoint{5.453568in}{2.179250in}}%
\pgfpathlineto{\pgfqpoint{5.467782in}{2.178570in}}%
\pgfpathlineto{\pgfqpoint{5.482004in}{2.177915in}}%
\pgfpathlineto{\pgfqpoint{5.496235in}{2.177284in}}%
\pgfpathlineto{\pgfqpoint{5.488757in}{2.168256in}}%
\pgfpathlineto{\pgfqpoint{5.481271in}{2.159132in}}%
\pgfpathlineto{\pgfqpoint{5.473777in}{2.149911in}}%
\pgfpathlineto{\pgfqpoint{5.466275in}{2.140596in}}%
\pgfpathlineto{\pgfqpoint{5.452034in}{2.141283in}}%
\pgfpathlineto{\pgfqpoint{5.437801in}{2.141994in}}%
\pgfpathlineto{\pgfqpoint{5.423578in}{2.142730in}}%
\pgfpathlineto{\pgfqpoint{5.409363in}{2.143490in}}%
\pgfpathlineto{\pgfqpoint{5.416875in}{2.152744in}}%
\pgfpathlineto{\pgfqpoint{5.424379in}{2.161907in}}%
\pgfpathlineto{\pgfqpoint{5.431875in}{2.170978in}}%
\pgfpathlineto{\pgfqpoint{5.439364in}{2.179955in}}%
\pgfpathclose%
\pgfusepath{fill}%
\end{pgfscope}%
\begin{pgfscope}%
\pgfpathrectangle{\pgfqpoint{1.150000in}{0.150000in}}{\pgfqpoint{5.700000in}{5.700000in}}%
\pgfusepath{clip}%
\pgfsetbuttcap%
\pgfsetroundjoin%
\definecolor{currentfill}{rgb}{0.278826,0.175490,0.483397}%
\pgfsetfillcolor{currentfill}%
\pgfsetfillopacity{0.700000}%
\pgfsetlinewidth{0.000000pt}%
\definecolor{currentstroke}{rgb}{0.000000,0.000000,0.000000}%
\pgfsetstrokecolor{currentstroke}%
\pgfsetdash{}{0pt}%
\pgfpathmoveto{\pgfqpoint{3.011599in}{2.176665in}}%
\pgfpathlineto{\pgfqpoint{3.025235in}{2.169003in}}%
\pgfpathlineto{\pgfqpoint{3.038875in}{2.161376in}}%
\pgfpathlineto{\pgfqpoint{3.052519in}{2.153783in}}%
\pgfpathlineto{\pgfqpoint{3.066166in}{2.146225in}}%
\pgfpathlineto{\pgfqpoint{3.057600in}{2.149056in}}%
\pgfpathlineto{\pgfqpoint{3.049017in}{2.152240in}}%
\pgfpathlineto{\pgfqpoint{3.040418in}{2.155785in}}%
\pgfpathlineto{\pgfqpoint{3.031800in}{2.159699in}}%
\pgfpathlineto{\pgfqpoint{3.018120in}{2.167572in}}%
\pgfpathlineto{\pgfqpoint{3.004444in}{2.175479in}}%
\pgfpathlineto{\pgfqpoint{2.990771in}{2.183420in}}%
\pgfpathlineto{\pgfqpoint{2.977101in}{2.191397in}}%
\pgfpathlineto{\pgfqpoint{2.985752in}{2.187163in}}%
\pgfpathlineto{\pgfqpoint{2.994385in}{2.183301in}}%
\pgfpathlineto{\pgfqpoint{3.003001in}{2.179805in}}%
\pgfpathlineto{\pgfqpoint{3.011599in}{2.176665in}}%
\pgfpathclose%
\pgfusepath{fill}%
\end{pgfscope}%
\begin{pgfscope}%
\pgfpathrectangle{\pgfqpoint{1.150000in}{0.150000in}}{\pgfqpoint{5.700000in}{5.700000in}}%
\pgfusepath{clip}%
\pgfsetbuttcap%
\pgfsetroundjoin%
\definecolor{currentfill}{rgb}{0.281924,0.089666,0.412415}%
\pgfsetfillcolor{currentfill}%
\pgfsetfillopacity{0.700000}%
\pgfsetlinewidth{0.000000pt}%
\definecolor{currentstroke}{rgb}{0.000000,0.000000,0.000000}%
\pgfsetstrokecolor{currentstroke}%
\pgfsetdash{}{0pt}%
\pgfpathmoveto{\pgfqpoint{4.948784in}{2.018460in}}%
\pgfpathlineto{\pgfqpoint{4.962832in}{2.016748in}}%
\pgfpathlineto{\pgfqpoint{4.976888in}{2.015060in}}%
\pgfpathlineto{\pgfqpoint{4.990952in}{2.013397in}}%
\pgfpathlineto{\pgfqpoint{5.005024in}{2.011759in}}%
\pgfpathlineto{\pgfqpoint{4.997355in}{2.001789in}}%
\pgfpathlineto{\pgfqpoint{4.989680in}{1.991779in}}%
\pgfpathlineto{\pgfqpoint{4.982000in}{1.981731in}}%
\pgfpathlineto{\pgfqpoint{4.974313in}{1.971648in}}%
\pgfpathlineto{\pgfqpoint{4.960233in}{1.973410in}}%
\pgfpathlineto{\pgfqpoint{4.946160in}{1.975196in}}%
\pgfpathlineto{\pgfqpoint{4.932095in}{1.977008in}}%
\pgfpathlineto{\pgfqpoint{4.918038in}{1.978844in}}%
\pgfpathlineto{\pgfqpoint{4.925733in}{1.988798in}}%
\pgfpathlineto{\pgfqpoint{4.933423in}{1.998720in}}%
\pgfpathlineto{\pgfqpoint{4.941106in}{2.008609in}}%
\pgfpathlineto{\pgfqpoint{4.948784in}{2.018460in}}%
\pgfpathclose%
\pgfusepath{fill}%
\end{pgfscope}%
\begin{pgfscope}%
\pgfpathrectangle{\pgfqpoint{1.150000in}{0.150000in}}{\pgfqpoint{5.700000in}{5.700000in}}%
\pgfusepath{clip}%
\pgfsetbuttcap%
\pgfsetroundjoin%
\definecolor{currentfill}{rgb}{0.260571,0.246922,0.522828}%
\pgfsetfillcolor{currentfill}%
\pgfsetfillopacity{0.700000}%
\pgfsetlinewidth{0.000000pt}%
\definecolor{currentstroke}{rgb}{0.000000,0.000000,0.000000}%
\pgfsetstrokecolor{currentstroke}%
\pgfsetdash{}{0pt}%
\pgfpathmoveto{\pgfqpoint{2.758818in}{2.324000in}}%
\pgfpathlineto{\pgfqpoint{2.772438in}{2.315427in}}%
\pgfpathlineto{\pgfqpoint{2.786062in}{2.306893in}}%
\pgfpathlineto{\pgfqpoint{2.799688in}{2.298398in}}%
\pgfpathlineto{\pgfqpoint{2.813316in}{2.289942in}}%
\pgfpathlineto{\pgfqpoint{2.804539in}{2.295533in}}%
\pgfpathlineto{\pgfqpoint{2.795741in}{2.301525in}}%
\pgfpathlineto{\pgfqpoint{2.786922in}{2.307930in}}%
\pgfpathlineto{\pgfqpoint{2.778081in}{2.314755in}}%
\pgfpathlineto{\pgfqpoint{2.764414in}{2.323542in}}%
\pgfpathlineto{\pgfqpoint{2.750750in}{2.332368in}}%
\pgfpathlineto{\pgfqpoint{2.737089in}{2.341233in}}%
\pgfpathlineto{\pgfqpoint{2.723430in}{2.350138in}}%
\pgfpathlineto{\pgfqpoint{2.732310in}{2.342976in}}%
\pgfpathlineto{\pgfqpoint{2.741168in}{2.336238in}}%
\pgfpathlineto{\pgfqpoint{2.750004in}{2.329916in}}%
\pgfpathlineto{\pgfqpoint{2.758818in}{2.324000in}}%
\pgfpathclose%
\pgfusepath{fill}%
\end{pgfscope}%
\begin{pgfscope}%
\pgfpathrectangle{\pgfqpoint{1.150000in}{0.150000in}}{\pgfqpoint{5.700000in}{5.700000in}}%
\pgfusepath{clip}%
\pgfsetbuttcap%
\pgfsetroundjoin%
\definecolor{currentfill}{rgb}{0.269308,0.218818,0.509577}%
\pgfsetfillcolor{currentfill}%
\pgfsetfillopacity{0.700000}%
\pgfsetlinewidth{0.000000pt}%
\definecolor{currentstroke}{rgb}{0.000000,0.000000,0.000000}%
\pgfsetstrokecolor{currentstroke}%
\pgfsetdash{}{0pt}%
\pgfpathmoveto{\pgfqpoint{5.756424in}{2.273239in}}%
\pgfpathlineto{\pgfqpoint{5.770740in}{2.273012in}}%
\pgfpathlineto{\pgfqpoint{5.785065in}{2.272810in}}%
\pgfpathlineto{\pgfqpoint{5.799400in}{2.272633in}}%
\pgfpathlineto{\pgfqpoint{5.792065in}{2.264768in}}%
\pgfpathlineto{\pgfqpoint{5.784722in}{2.256790in}}%
\pgfpathlineto{\pgfqpoint{5.777369in}{2.248699in}}%
\pgfpathlineto{\pgfqpoint{5.770008in}{2.240494in}}%
\pgfpathlineto{\pgfqpoint{5.755661in}{2.240685in}}%
\pgfpathlineto{\pgfqpoint{5.741323in}{2.240901in}}%
\pgfpathlineto{\pgfqpoint{5.726995in}{2.241141in}}%
\pgfpathlineto{\pgfqpoint{5.734366in}{2.249332in}}%
\pgfpathlineto{\pgfqpoint{5.741727in}{2.257411in}}%
\pgfpathlineto{\pgfqpoint{5.749080in}{2.265380in}}%
\pgfpathlineto{\pgfqpoint{5.756424in}{2.273239in}}%
\pgfpathclose%
\pgfusepath{fill}%
\end{pgfscope}%
\begin{pgfscope}%
\pgfpathrectangle{\pgfqpoint{1.150000in}{0.150000in}}{\pgfqpoint{5.700000in}{5.700000in}}%
\pgfusepath{clip}%
\pgfsetbuttcap%
\pgfsetroundjoin%
\definecolor{currentfill}{rgb}{0.281412,0.155834,0.469201}%
\pgfsetfillcolor{currentfill}%
\pgfsetfillopacity{0.700000}%
\pgfsetlinewidth{0.000000pt}%
\definecolor{currentstroke}{rgb}{0.000000,0.000000,0.000000}%
\pgfsetstrokecolor{currentstroke}%
\pgfsetdash{}{0pt}%
\pgfpathmoveto{\pgfqpoint{5.352593in}{2.146778in}}%
\pgfpathlineto{\pgfqpoint{5.366772in}{2.145919in}}%
\pgfpathlineto{\pgfqpoint{5.380960in}{2.145085in}}%
\pgfpathlineto{\pgfqpoint{5.395157in}{2.144275in}}%
\pgfpathlineto{\pgfqpoint{5.409363in}{2.143490in}}%
\pgfpathlineto{\pgfqpoint{5.401844in}{2.134146in}}%
\pgfpathlineto{\pgfqpoint{5.394318in}{2.124712in}}%
\pgfpathlineto{\pgfqpoint{5.386785in}{2.115191in}}%
\pgfpathlineto{\pgfqpoint{5.379244in}{2.105583in}}%
\pgfpathlineto{\pgfqpoint{5.365029in}{2.106438in}}%
\pgfpathlineto{\pgfqpoint{5.350822in}{2.107317in}}%
\pgfpathlineto{\pgfqpoint{5.336624in}{2.108221in}}%
\pgfpathlineto{\pgfqpoint{5.322436in}{2.109149in}}%
\pgfpathlineto{\pgfqpoint{5.329986in}{2.118682in}}%
\pgfpathlineto{\pgfqpoint{5.337529in}{2.128132in}}%
\pgfpathlineto{\pgfqpoint{5.345064in}{2.137498in}}%
\pgfpathlineto{\pgfqpoint{5.352593in}{2.146778in}}%
\pgfpathclose%
\pgfusepath{fill}%
\end{pgfscope}%
\begin{pgfscope}%
\pgfpathrectangle{\pgfqpoint{1.150000in}{0.150000in}}{\pgfqpoint{5.700000in}{5.700000in}}%
\pgfusepath{clip}%
\pgfsetbuttcap%
\pgfsetroundjoin%
\definecolor{currentfill}{rgb}{0.268510,0.009605,0.335427}%
\pgfsetfillcolor{currentfill}%
\pgfsetfillopacity{0.700000}%
\pgfsetlinewidth{0.000000pt}%
\definecolor{currentstroke}{rgb}{0.000000,0.000000,0.000000}%
\pgfsetstrokecolor{currentstroke}%
\pgfsetdash{}{0pt}%
\pgfpathmoveto{\pgfqpoint{4.315700in}{1.876396in}}%
\pgfpathlineto{\pgfqpoint{4.329570in}{1.872941in}}%
\pgfpathlineto{\pgfqpoint{4.343447in}{1.869511in}}%
\pgfpathlineto{\pgfqpoint{4.357331in}{1.866107in}}%
\pgfpathlineto{\pgfqpoint{4.371221in}{1.862728in}}%
\pgfpathlineto{\pgfqpoint{4.363346in}{1.854318in}}%
\pgfpathlineto{\pgfqpoint{4.355466in}{1.845985in}}%
\pgfpathlineto{\pgfqpoint{4.347580in}{1.837734in}}%
\pgfpathlineto{\pgfqpoint{4.339689in}{1.829571in}}%
\pgfpathlineto{\pgfqpoint{4.325786in}{1.833153in}}%
\pgfpathlineto{\pgfqpoint{4.311890in}{1.836760in}}%
\pgfpathlineto{\pgfqpoint{4.298001in}{1.840392in}}%
\pgfpathlineto{\pgfqpoint{4.284119in}{1.844051in}}%
\pgfpathlineto{\pgfqpoint{4.292022in}{1.852006in}}%
\pgfpathlineto{\pgfqpoint{4.299921in}{1.860052in}}%
\pgfpathlineto{\pgfqpoint{4.307813in}{1.868183in}}%
\pgfpathlineto{\pgfqpoint{4.315700in}{1.876396in}}%
\pgfpathclose%
\pgfusepath{fill}%
\end{pgfscope}%
\begin{pgfscope}%
\pgfpathrectangle{\pgfqpoint{1.150000in}{0.150000in}}{\pgfqpoint{5.700000in}{5.700000in}}%
\pgfusepath{clip}%
\pgfsetbuttcap%
\pgfsetroundjoin%
\definecolor{currentfill}{rgb}{0.221989,0.339161,0.548752}%
\pgfsetfillcolor{currentfill}%
\pgfsetfillopacity{0.700000}%
\pgfsetlinewidth{0.000000pt}%
\definecolor{currentstroke}{rgb}{0.000000,0.000000,0.000000}%
\pgfsetstrokecolor{currentstroke}%
\pgfsetdash{}{0pt}%
\pgfpathmoveto{\pgfqpoint{2.450734in}{2.537112in}}%
\pgfpathlineto{\pgfqpoint{2.464350in}{2.527337in}}%
\pgfpathlineto{\pgfqpoint{2.477968in}{2.517609in}}%
\pgfpathlineto{\pgfqpoint{2.491587in}{2.507929in}}%
\pgfpathlineto{\pgfqpoint{2.505209in}{2.498295in}}%
\pgfpathlineto{\pgfqpoint{2.496139in}{2.507262in}}%
\pgfpathlineto{\pgfqpoint{2.487044in}{2.516689in}}%
\pgfpathlineto{\pgfqpoint{2.477922in}{2.526585in}}%
\pgfpathlineto{\pgfqpoint{2.468773in}{2.536961in}}%
\pgfpathlineto{\pgfqpoint{2.455108in}{2.546945in}}%
\pgfpathlineto{\pgfqpoint{2.441444in}{2.556977in}}%
\pgfpathlineto{\pgfqpoint{2.427781in}{2.567055in}}%
\pgfpathlineto{\pgfqpoint{2.414121in}{2.577181in}}%
\pgfpathlineto{\pgfqpoint{2.423315in}{2.566449in}}%
\pgfpathlineto{\pgfqpoint{2.432482in}{2.556200in}}%
\pgfpathlineto{\pgfqpoint{2.441622in}{2.546424in}}%
\pgfpathlineto{\pgfqpoint{2.450734in}{2.537112in}}%
\pgfpathclose%
\pgfusepath{fill}%
\end{pgfscope}%
\begin{pgfscope}%
\pgfpathrectangle{\pgfqpoint{1.150000in}{0.150000in}}{\pgfqpoint{5.700000in}{5.700000in}}%
\pgfusepath{clip}%
\pgfsetbuttcap%
\pgfsetroundjoin%
\definecolor{currentfill}{rgb}{0.272594,0.025563,0.353093}%
\pgfsetfillcolor{currentfill}%
\pgfsetfillopacity{0.700000}%
\pgfsetlinewidth{0.000000pt}%
\definecolor{currentstroke}{rgb}{0.000000,0.000000,0.000000}%
\pgfsetstrokecolor{currentstroke}%
\pgfsetdash{}{0pt}%
\pgfpathmoveto{\pgfqpoint{4.545227in}{1.909044in}}%
\pgfpathlineto{\pgfqpoint{4.559159in}{1.906260in}}%
\pgfpathlineto{\pgfqpoint{4.573098in}{1.903500in}}%
\pgfpathlineto{\pgfqpoint{4.587045in}{1.900765in}}%
\pgfpathlineto{\pgfqpoint{4.600999in}{1.898055in}}%
\pgfpathlineto{\pgfqpoint{4.593198in}{1.888678in}}%
\pgfpathlineto{\pgfqpoint{4.585392in}{1.879333in}}%
\pgfpathlineto{\pgfqpoint{4.577581in}{1.870022in}}%
\pgfpathlineto{\pgfqpoint{4.569765in}{1.860751in}}%
\pgfpathlineto{\pgfqpoint{4.555801in}{1.863638in}}%
\pgfpathlineto{\pgfqpoint{4.541844in}{1.866549in}}%
\pgfpathlineto{\pgfqpoint{4.527894in}{1.869486in}}%
\pgfpathlineto{\pgfqpoint{4.513952in}{1.872447in}}%
\pgfpathlineto{\pgfqpoint{4.521778in}{1.881537in}}%
\pgfpathlineto{\pgfqpoint{4.529600in}{1.890669in}}%
\pgfpathlineto{\pgfqpoint{4.537416in}{1.899839in}}%
\pgfpathlineto{\pgfqpoint{4.545227in}{1.909044in}}%
\pgfpathclose%
\pgfusepath{fill}%
\end{pgfscope}%
\begin{pgfscope}%
\pgfpathrectangle{\pgfqpoint{1.150000in}{0.150000in}}{\pgfqpoint{5.700000in}{5.700000in}}%
\pgfusepath{clip}%
\pgfsetbuttcap%
\pgfsetroundjoin%
\definecolor{currentfill}{rgb}{0.280267,0.073417,0.397163}%
\pgfsetfillcolor{currentfill}%
\pgfsetfillopacity{0.700000}%
\pgfsetlinewidth{0.000000pt}%
\definecolor{currentstroke}{rgb}{0.000000,0.000000,0.000000}%
\pgfsetstrokecolor{currentstroke}%
\pgfsetdash{}{0pt}%
\pgfpathmoveto{\pgfqpoint{4.861891in}{1.986435in}}%
\pgfpathlineto{\pgfqpoint{4.875916in}{1.984500in}}%
\pgfpathlineto{\pgfqpoint{4.889949in}{1.982590in}}%
\pgfpathlineto{\pgfqpoint{4.903990in}{1.980704in}}%
\pgfpathlineto{\pgfqpoint{4.918038in}{1.978844in}}%
\pgfpathlineto{\pgfqpoint{4.910338in}{1.968861in}}%
\pgfpathlineto{\pgfqpoint{4.902632in}{1.958852in}}%
\pgfpathlineto{\pgfqpoint{4.894920in}{1.948821in}}%
\pgfpathlineto{\pgfqpoint{4.887204in}{1.938770in}}%
\pgfpathlineto{\pgfqpoint{4.873146in}{1.940768in}}%
\pgfpathlineto{\pgfqpoint{4.859097in}{1.942791in}}%
\pgfpathlineto{\pgfqpoint{4.845055in}{1.944838in}}%
\pgfpathlineto{\pgfqpoint{4.831021in}{1.946910in}}%
\pgfpathlineto{\pgfqpoint{4.838747in}{1.956819in}}%
\pgfpathlineto{\pgfqpoint{4.846467in}{1.966711in}}%
\pgfpathlineto{\pgfqpoint{4.854182in}{1.976584in}}%
\pgfpathlineto{\pgfqpoint{4.861891in}{1.986435in}}%
\pgfpathclose%
\pgfusepath{fill}%
\end{pgfscope}%
\begin{pgfscope}%
\pgfpathrectangle{\pgfqpoint{1.150000in}{0.150000in}}{\pgfqpoint{5.700000in}{5.700000in}}%
\pgfusepath{clip}%
\pgfsetbuttcap%
\pgfsetroundjoin%
\definecolor{currentfill}{rgb}{0.283197,0.115680,0.436115}%
\pgfsetfillcolor{currentfill}%
\pgfsetfillopacity{0.700000}%
\pgfsetlinewidth{0.000000pt}%
\definecolor{currentstroke}{rgb}{0.000000,0.000000,0.000000}%
\pgfsetstrokecolor{currentstroke}%
\pgfsetdash{}{0pt}%
\pgfpathmoveto{\pgfqpoint{3.263974in}{2.053761in}}%
\pgfpathlineto{\pgfqpoint{3.277643in}{2.046930in}}%
\pgfpathlineto{\pgfqpoint{3.291316in}{2.040131in}}%
\pgfpathlineto{\pgfqpoint{3.304993in}{2.033363in}}%
\pgfpathlineto{\pgfqpoint{3.318675in}{2.026627in}}%
\pgfpathlineto{\pgfqpoint{3.310287in}{2.026956in}}%
\pgfpathlineto{\pgfqpoint{3.301886in}{2.027591in}}%
\pgfpathlineto{\pgfqpoint{3.293472in}{2.028539in}}%
\pgfpathlineto{\pgfqpoint{3.285043in}{2.029809in}}%
\pgfpathlineto{\pgfqpoint{3.271333in}{2.036844in}}%
\pgfpathlineto{\pgfqpoint{3.257627in}{2.043910in}}%
\pgfpathlineto{\pgfqpoint{3.243926in}{2.051008in}}%
\pgfpathlineto{\pgfqpoint{3.230228in}{2.058137in}}%
\pgfpathlineto{\pgfqpoint{3.238686in}{2.056564in}}%
\pgfpathlineto{\pgfqpoint{3.247130in}{2.055315in}}%
\pgfpathlineto{\pgfqpoint{3.255559in}{2.054383in}}%
\pgfpathlineto{\pgfqpoint{3.263974in}{2.053761in}}%
\pgfpathclose%
\pgfusepath{fill}%
\end{pgfscope}%
\begin{pgfscope}%
\pgfpathrectangle{\pgfqpoint{1.150000in}{0.150000in}}{\pgfqpoint{5.700000in}{5.700000in}}%
\pgfusepath{clip}%
\pgfsetbuttcap%
\pgfsetroundjoin%
\definecolor{currentfill}{rgb}{0.280267,0.073417,0.397163}%
\pgfsetfillcolor{currentfill}%
\pgfsetfillopacity{0.700000}%
\pgfsetlinewidth{0.000000pt}%
\definecolor{currentstroke}{rgb}{0.000000,0.000000,0.000000}%
\pgfsetstrokecolor{currentstroke}%
\pgfsetdash{}{0pt}%
\pgfpathmoveto{\pgfqpoint{3.461502in}{1.977695in}}%
\pgfpathlineto{\pgfqpoint{3.475199in}{1.971516in}}%
\pgfpathlineto{\pgfqpoint{3.488900in}{1.965367in}}%
\pgfpathlineto{\pgfqpoint{3.502607in}{1.959247in}}%
\pgfpathlineto{\pgfqpoint{3.516318in}{1.953157in}}%
\pgfpathlineto{\pgfqpoint{3.508056in}{1.951505in}}%
\pgfpathlineto{\pgfqpoint{3.499783in}{1.950118in}}%
\pgfpathlineto{\pgfqpoint{3.491499in}{1.949002in}}%
\pgfpathlineto{\pgfqpoint{3.483203in}{1.948166in}}%
\pgfpathlineto{\pgfqpoint{3.469467in}{1.954540in}}%
\pgfpathlineto{\pgfqpoint{3.455736in}{1.960944in}}%
\pgfpathlineto{\pgfqpoint{3.442009in}{1.967377in}}%
\pgfpathlineto{\pgfqpoint{3.428287in}{1.973839in}}%
\pgfpathlineto{\pgfqpoint{3.436609in}{1.974387in}}%
\pgfpathlineto{\pgfqpoint{3.444918in}{1.975216in}}%
\pgfpathlineto{\pgfqpoint{3.453216in}{1.976322in}}%
\pgfpathlineto{\pgfqpoint{3.461502in}{1.977695in}}%
\pgfpathclose%
\pgfusepath{fill}%
\end{pgfscope}%
\begin{pgfscope}%
\pgfpathrectangle{\pgfqpoint{1.150000in}{0.150000in}}{\pgfqpoint{5.700000in}{5.700000in}}%
\pgfusepath{clip}%
\pgfsetbuttcap%
\pgfsetroundjoin%
\definecolor{currentfill}{rgb}{0.282623,0.140926,0.457517}%
\pgfsetfillcolor{currentfill}%
\pgfsetfillopacity{0.700000}%
\pgfsetlinewidth{0.000000pt}%
\definecolor{currentstroke}{rgb}{0.000000,0.000000,0.000000}%
\pgfsetstrokecolor{currentstroke}%
\pgfsetdash{}{0pt}%
\pgfpathmoveto{\pgfqpoint{5.265767in}{2.113109in}}%
\pgfpathlineto{\pgfqpoint{5.279921in}{2.112082in}}%
\pgfpathlineto{\pgfqpoint{5.294084in}{2.111080in}}%
\pgfpathlineto{\pgfqpoint{5.308255in}{2.110102in}}%
\pgfpathlineto{\pgfqpoint{5.322436in}{2.109149in}}%
\pgfpathlineto{\pgfqpoint{5.314878in}{2.099535in}}%
\pgfpathlineto{\pgfqpoint{5.307315in}{2.089840in}}%
\pgfpathlineto{\pgfqpoint{5.299744in}{2.080067in}}%
\pgfpathlineto{\pgfqpoint{5.292166in}{2.070218in}}%
\pgfpathlineto{\pgfqpoint{5.277977in}{2.071254in}}%
\pgfpathlineto{\pgfqpoint{5.263796in}{2.072315in}}%
\pgfpathlineto{\pgfqpoint{5.249624in}{2.073401in}}%
\pgfpathlineto{\pgfqpoint{5.235460in}{2.074511in}}%
\pgfpathlineto{\pgfqpoint{5.243047in}{2.084272in}}%
\pgfpathlineto{\pgfqpoint{5.250627in}{2.093960in}}%
\pgfpathlineto{\pgfqpoint{5.258200in}{2.103573in}}%
\pgfpathlineto{\pgfqpoint{5.265767in}{2.113109in}}%
\pgfpathclose%
\pgfusepath{fill}%
\end{pgfscope}%
\begin{pgfscope}%
\pgfpathrectangle{\pgfqpoint{1.150000in}{0.150000in}}{\pgfqpoint{5.700000in}{5.700000in}}%
\pgfusepath{clip}%
\pgfsetbuttcap%
\pgfsetroundjoin%
\definecolor{currentfill}{rgb}{0.269944,0.014625,0.341379}%
\pgfsetfillcolor{currentfill}%
\pgfsetfillopacity{0.700000}%
\pgfsetlinewidth{0.000000pt}%
\definecolor{currentstroke}{rgb}{0.000000,0.000000,0.000000}%
\pgfsetstrokecolor{currentstroke}%
\pgfsetdash{}{0pt}%
\pgfpathmoveto{\pgfqpoint{3.943812in}{1.871814in}}%
\pgfpathlineto{\pgfqpoint{3.957598in}{1.867189in}}%
\pgfpathlineto{\pgfqpoint{3.971389in}{1.862591in}}%
\pgfpathlineto{\pgfqpoint{3.985187in}{1.858020in}}%
\pgfpathlineto{\pgfqpoint{3.998991in}{1.853475in}}%
\pgfpathlineto{\pgfqpoint{3.990974in}{1.847530in}}%
\pgfpathlineto{\pgfqpoint{3.982950in}{1.841745in}}%
\pgfpathlineto{\pgfqpoint{3.974919in}{1.836125in}}%
\pgfpathlineto{\pgfqpoint{3.966881in}{1.830678in}}%
\pgfpathlineto{\pgfqpoint{3.953060in}{1.835465in}}%
\pgfpathlineto{\pgfqpoint{3.939246in}{1.840279in}}%
\pgfpathlineto{\pgfqpoint{3.925437in}{1.845120in}}%
\pgfpathlineto{\pgfqpoint{3.911634in}{1.849987in}}%
\pgfpathlineto{\pgfqpoint{3.919689in}{1.855187in}}%
\pgfpathlineto{\pgfqpoint{3.927738in}{1.860562in}}%
\pgfpathlineto{\pgfqpoint{3.935778in}{1.866106in}}%
\pgfpathlineto{\pgfqpoint{3.943812in}{1.871814in}}%
\pgfpathclose%
\pgfusepath{fill}%
\end{pgfscope}%
\begin{pgfscope}%
\pgfpathrectangle{\pgfqpoint{1.150000in}{0.150000in}}{\pgfqpoint{5.700000in}{5.700000in}}%
\pgfusepath{clip}%
\pgfsetbuttcap%
\pgfsetroundjoin%
\definecolor{currentfill}{rgb}{0.272594,0.025563,0.353093}%
\pgfsetfillcolor{currentfill}%
\pgfsetfillopacity{0.700000}%
\pgfsetlinewidth{0.000000pt}%
\definecolor{currentstroke}{rgb}{0.000000,0.000000,0.000000}%
\pgfsetstrokecolor{currentstroke}%
\pgfsetdash{}{0pt}%
\pgfpathmoveto{\pgfqpoint{3.801416in}{1.889894in}}%
\pgfpathlineto{\pgfqpoint{3.815174in}{1.884811in}}%
\pgfpathlineto{\pgfqpoint{3.828937in}{1.879755in}}%
\pgfpathlineto{\pgfqpoint{3.842705in}{1.874726in}}%
\pgfpathlineto{\pgfqpoint{3.856480in}{1.869724in}}%
\pgfpathlineto{\pgfqpoint{3.848398in}{1.864958in}}%
\pgfpathlineto{\pgfqpoint{3.840309in}{1.860383in}}%
\pgfpathlineto{\pgfqpoint{3.832212in}{1.856006in}}%
\pgfpathlineto{\pgfqpoint{3.824106in}{1.851834in}}%
\pgfpathlineto{\pgfqpoint{3.810313in}{1.857092in}}%
\pgfpathlineto{\pgfqpoint{3.796525in}{1.862377in}}%
\pgfpathlineto{\pgfqpoint{3.782743in}{1.867690in}}%
\pgfpathlineto{\pgfqpoint{3.768966in}{1.873029in}}%
\pgfpathlineto{\pgfqpoint{3.777091in}{1.876940in}}%
\pgfpathlineto{\pgfqpoint{3.785208in}{1.881058in}}%
\pgfpathlineto{\pgfqpoint{3.793316in}{1.885379in}}%
\pgfpathlineto{\pgfqpoint{3.801416in}{1.889894in}}%
\pgfpathclose%
\pgfusepath{fill}%
\end{pgfscope}%
\begin{pgfscope}%
\pgfpathrectangle{\pgfqpoint{1.150000in}{0.150000in}}{\pgfqpoint{5.700000in}{5.700000in}}%
\pgfusepath{clip}%
\pgfsetbuttcap%
\pgfsetroundjoin%
\definecolor{currentfill}{rgb}{0.268510,0.009605,0.335427}%
\pgfsetfillcolor{currentfill}%
\pgfsetfillopacity{0.700000}%
\pgfsetlinewidth{0.000000pt}%
\definecolor{currentstroke}{rgb}{0.000000,0.000000,0.000000}%
\pgfsetstrokecolor{currentstroke}%
\pgfsetdash{}{0pt}%
\pgfpathmoveto{\pgfqpoint{4.086204in}{1.861737in}}%
\pgfpathlineto{\pgfqpoint{4.100022in}{1.857553in}}%
\pgfpathlineto{\pgfqpoint{4.113847in}{1.853395in}}%
\pgfpathlineto{\pgfqpoint{4.127679in}{1.849263in}}%
\pgfpathlineto{\pgfqpoint{4.141516in}{1.845157in}}%
\pgfpathlineto{\pgfqpoint{4.133556in}{1.838184in}}%
\pgfpathlineto{\pgfqpoint{4.125590in}{1.831339in}}%
\pgfpathlineto{\pgfqpoint{4.117617in}{1.824630in}}%
\pgfpathlineto{\pgfqpoint{4.109639in}{1.818062in}}%
\pgfpathlineto{\pgfqpoint{4.095786in}{1.822398in}}%
\pgfpathlineto{\pgfqpoint{4.081940in}{1.826759in}}%
\pgfpathlineto{\pgfqpoint{4.068100in}{1.831146in}}%
\pgfpathlineto{\pgfqpoint{4.054266in}{1.835559in}}%
\pgfpathlineto{\pgfqpoint{4.062260in}{1.841893in}}%
\pgfpathlineto{\pgfqpoint{4.070248in}{1.848371in}}%
\pgfpathlineto{\pgfqpoint{4.078229in}{1.854987in}}%
\pgfpathlineto{\pgfqpoint{4.086204in}{1.861737in}}%
\pgfpathclose%
\pgfusepath{fill}%
\end{pgfscope}%
\begin{pgfscope}%
\pgfpathrectangle{\pgfqpoint{1.150000in}{0.150000in}}{\pgfqpoint{5.700000in}{5.700000in}}%
\pgfusepath{clip}%
\pgfsetbuttcap%
\pgfsetroundjoin%
\definecolor{currentfill}{rgb}{0.278791,0.062145,0.386592}%
\pgfsetfillcolor{currentfill}%
\pgfsetfillopacity{0.700000}%
\pgfsetlinewidth{0.000000pt}%
\definecolor{currentstroke}{rgb}{0.000000,0.000000,0.000000}%
\pgfsetstrokecolor{currentstroke}%
\pgfsetdash{}{0pt}%
\pgfpathmoveto{\pgfqpoint{4.774964in}{1.955447in}}%
\pgfpathlineto{\pgfqpoint{4.788966in}{1.953275in}}%
\pgfpathlineto{\pgfqpoint{4.802977in}{1.951129in}}%
\pgfpathlineto{\pgfqpoint{4.816995in}{1.949007in}}%
\pgfpathlineto{\pgfqpoint{4.831021in}{1.946910in}}%
\pgfpathlineto{\pgfqpoint{4.823290in}{1.936989in}}%
\pgfpathlineto{\pgfqpoint{4.815554in}{1.927059in}}%
\pgfpathlineto{\pgfqpoint{4.807812in}{1.917123in}}%
\pgfpathlineto{\pgfqpoint{4.800066in}{1.907184in}}%
\pgfpathlineto{\pgfqpoint{4.786031in}{1.909432in}}%
\pgfpathlineto{\pgfqpoint{4.772003in}{1.911704in}}%
\pgfpathlineto{\pgfqpoint{4.757984in}{1.914001in}}%
\pgfpathlineto{\pgfqpoint{4.743972in}{1.916323in}}%
\pgfpathlineto{\pgfqpoint{4.751728in}{1.926106in}}%
\pgfpathlineto{\pgfqpoint{4.759478in}{1.935890in}}%
\pgfpathlineto{\pgfqpoint{4.767224in}{1.945671in}}%
\pgfpathlineto{\pgfqpoint{4.774964in}{1.955447in}}%
\pgfpathclose%
\pgfusepath{fill}%
\end{pgfscope}%
\begin{pgfscope}%
\pgfpathrectangle{\pgfqpoint{1.150000in}{0.150000in}}{\pgfqpoint{5.700000in}{5.700000in}}%
\pgfusepath{clip}%
\pgfsetbuttcap%
\pgfsetroundjoin%
\definecolor{currentfill}{rgb}{0.273006,0.204520,0.501721}%
\pgfsetfillcolor{currentfill}%
\pgfsetfillopacity{0.700000}%
\pgfsetlinewidth{0.000000pt}%
\definecolor{currentstroke}{rgb}{0.000000,0.000000,0.000000}%
\pgfsetstrokecolor{currentstroke}%
\pgfsetdash{}{0pt}%
\pgfpathmoveto{\pgfqpoint{5.669776in}{2.242350in}}%
\pgfpathlineto{\pgfqpoint{5.684067in}{2.242011in}}%
\pgfpathlineto{\pgfqpoint{5.698367in}{2.241696in}}%
\pgfpathlineto{\pgfqpoint{5.712677in}{2.241406in}}%
\pgfpathlineto{\pgfqpoint{5.726995in}{2.241141in}}%
\pgfpathlineto{\pgfqpoint{5.719616in}{2.232839in}}%
\pgfpathlineto{\pgfqpoint{5.712228in}{2.224427in}}%
\pgfpathlineto{\pgfqpoint{5.704832in}{2.215904in}}%
\pgfpathlineto{\pgfqpoint{5.697426in}{2.207270in}}%
\pgfpathlineto{\pgfqpoint{5.683096in}{2.207563in}}%
\pgfpathlineto{\pgfqpoint{5.668775in}{2.207881in}}%
\pgfpathlineto{\pgfqpoint{5.654463in}{2.208223in}}%
\pgfpathlineto{\pgfqpoint{5.640161in}{2.208590in}}%
\pgfpathlineto{\pgfqpoint{5.647577in}{2.217191in}}%
\pgfpathlineto{\pgfqpoint{5.654985in}{2.225684in}}%
\pgfpathlineto{\pgfqpoint{5.662385in}{2.234071in}}%
\pgfpathlineto{\pgfqpoint{5.669776in}{2.242350in}}%
\pgfpathclose%
\pgfusepath{fill}%
\end{pgfscope}%
\begin{pgfscope}%
\pgfpathrectangle{\pgfqpoint{1.150000in}{0.150000in}}{\pgfqpoint{5.700000in}{5.700000in}}%
\pgfusepath{clip}%
\pgfsetbuttcap%
\pgfsetroundjoin%
\definecolor{currentfill}{rgb}{0.280255,0.165693,0.476498}%
\pgfsetfillcolor{currentfill}%
\pgfsetfillopacity{0.700000}%
\pgfsetlinewidth{0.000000pt}%
\definecolor{currentstroke}{rgb}{0.000000,0.000000,0.000000}%
\pgfsetstrokecolor{currentstroke}%
\pgfsetdash{}{0pt}%
\pgfpathmoveto{\pgfqpoint{3.066166in}{2.146225in}}%
\pgfpathlineto{\pgfqpoint{3.079817in}{2.138701in}}%
\pgfpathlineto{\pgfqpoint{3.093471in}{2.131211in}}%
\pgfpathlineto{\pgfqpoint{3.107130in}{2.123755in}}%
\pgfpathlineto{\pgfqpoint{3.120792in}{2.116333in}}%
\pgfpathlineto{\pgfqpoint{3.112257in}{2.118856in}}%
\pgfpathlineto{\pgfqpoint{3.103707in}{2.121727in}}%
\pgfpathlineto{\pgfqpoint{3.095140in}{2.124956in}}%
\pgfpathlineto{\pgfqpoint{3.086556in}{2.128551in}}%
\pgfpathlineto{\pgfqpoint{3.072861in}{2.136287in}}%
\pgfpathlineto{\pgfqpoint{3.059171in}{2.144057in}}%
\pgfpathlineto{\pgfqpoint{3.045484in}{2.151861in}}%
\pgfpathlineto{\pgfqpoint{3.031800in}{2.159699in}}%
\pgfpathlineto{\pgfqpoint{3.040418in}{2.155785in}}%
\pgfpathlineto{\pgfqpoint{3.049017in}{2.152240in}}%
\pgfpathlineto{\pgfqpoint{3.057600in}{2.149056in}}%
\pgfpathlineto{\pgfqpoint{3.066166in}{2.146225in}}%
\pgfpathclose%
\pgfusepath{fill}%
\end{pgfscope}%
\begin{pgfscope}%
\pgfpathrectangle{\pgfqpoint{1.150000in}{0.150000in}}{\pgfqpoint{5.700000in}{5.700000in}}%
\pgfusepath{clip}%
\pgfsetbuttcap%
\pgfsetroundjoin%
\definecolor{currentfill}{rgb}{0.283187,0.125848,0.444960}%
\pgfsetfillcolor{currentfill}%
\pgfsetfillopacity{0.700000}%
\pgfsetlinewidth{0.000000pt}%
\definecolor{currentstroke}{rgb}{0.000000,0.000000,0.000000}%
\pgfsetstrokecolor{currentstroke}%
\pgfsetdash{}{0pt}%
\pgfpathmoveto{\pgfqpoint{5.178892in}{2.079198in}}%
\pgfpathlineto{\pgfqpoint{5.193022in}{2.077989in}}%
\pgfpathlineto{\pgfqpoint{5.207159in}{2.076805in}}%
\pgfpathlineto{\pgfqpoint{5.221306in}{2.075646in}}%
\pgfpathlineto{\pgfqpoint{5.235460in}{2.074511in}}%
\pgfpathlineto{\pgfqpoint{5.227867in}{2.064679in}}%
\pgfpathlineto{\pgfqpoint{5.220268in}{2.054777in}}%
\pgfpathlineto{\pgfqpoint{5.212661in}{2.044808in}}%
\pgfpathlineto{\pgfqpoint{5.205049in}{2.034773in}}%
\pgfpathlineto{\pgfqpoint{5.190885in}{2.036005in}}%
\pgfpathlineto{\pgfqpoint{5.176730in}{2.037261in}}%
\pgfpathlineto{\pgfqpoint{5.162583in}{2.038542in}}%
\pgfpathlineto{\pgfqpoint{5.148445in}{2.039847in}}%
\pgfpathlineto{\pgfqpoint{5.156066in}{2.049780in}}%
\pgfpathlineto{\pgfqpoint{5.163681in}{2.059651in}}%
\pgfpathlineto{\pgfqpoint{5.171290in}{2.069457in}}%
\pgfpathlineto{\pgfqpoint{5.178892in}{2.079198in}}%
\pgfpathclose%
\pgfusepath{fill}%
\end{pgfscope}%
\begin{pgfscope}%
\pgfpathrectangle{\pgfqpoint{1.150000in}{0.150000in}}{\pgfqpoint{5.700000in}{5.700000in}}%
\pgfusepath{clip}%
\pgfsetbuttcap%
\pgfsetroundjoin%
\definecolor{currentfill}{rgb}{0.265145,0.232956,0.516599}%
\pgfsetfillcolor{currentfill}%
\pgfsetfillopacity{0.700000}%
\pgfsetlinewidth{0.000000pt}%
\definecolor{currentstroke}{rgb}{0.000000,0.000000,0.000000}%
\pgfsetstrokecolor{currentstroke}%
\pgfsetdash{}{0pt}%
\pgfpathmoveto{\pgfqpoint{2.813316in}{2.289942in}}%
\pgfpathlineto{\pgfqpoint{2.826948in}{2.281525in}}%
\pgfpathlineto{\pgfqpoint{2.840583in}{2.273146in}}%
\pgfpathlineto{\pgfqpoint{2.854221in}{2.264805in}}%
\pgfpathlineto{\pgfqpoint{2.867861in}{2.256502in}}%
\pgfpathlineto{\pgfqpoint{2.859121in}{2.261767in}}%
\pgfpathlineto{\pgfqpoint{2.850360in}{2.267431in}}%
\pgfpathlineto{\pgfqpoint{2.841578in}{2.273502in}}%
\pgfpathlineto{\pgfqpoint{2.832776in}{2.279991in}}%
\pgfpathlineto{\pgfqpoint{2.819098in}{2.288625in}}%
\pgfpathlineto{\pgfqpoint{2.805423in}{2.297297in}}%
\pgfpathlineto{\pgfqpoint{2.791750in}{2.306007in}}%
\pgfpathlineto{\pgfqpoint{2.778081in}{2.314755in}}%
\pgfpathlineto{\pgfqpoint{2.786922in}{2.307930in}}%
\pgfpathlineto{\pgfqpoint{2.795741in}{2.301525in}}%
\pgfpathlineto{\pgfqpoint{2.804539in}{2.295533in}}%
\pgfpathlineto{\pgfqpoint{2.813316in}{2.289942in}}%
\pgfpathclose%
\pgfusepath{fill}%
\end{pgfscope}%
\begin{pgfscope}%
\pgfpathrectangle{\pgfqpoint{1.150000in}{0.150000in}}{\pgfqpoint{5.700000in}{5.700000in}}%
\pgfusepath{clip}%
\pgfsetbuttcap%
\pgfsetroundjoin%
\definecolor{currentfill}{rgb}{0.271305,0.019942,0.347269}%
\pgfsetfillcolor{currentfill}%
\pgfsetfillopacity{0.700000}%
\pgfsetlinewidth{0.000000pt}%
\definecolor{currentstroke}{rgb}{0.000000,0.000000,0.000000}%
\pgfsetstrokecolor{currentstroke}%
\pgfsetdash{}{0pt}%
\pgfpathmoveto{\pgfqpoint{4.458254in}{1.884545in}}%
\pgfpathlineto{\pgfqpoint{4.472167in}{1.881483in}}%
\pgfpathlineto{\pgfqpoint{4.486088in}{1.878446in}}%
\pgfpathlineto{\pgfqpoint{4.500017in}{1.875434in}}%
\pgfpathlineto{\pgfqpoint{4.513952in}{1.872447in}}%
\pgfpathlineto{\pgfqpoint{4.506120in}{1.863406in}}%
\pgfpathlineto{\pgfqpoint{4.498284in}{1.854416in}}%
\pgfpathlineto{\pgfqpoint{4.490442in}{1.845482in}}%
\pgfpathlineto{\pgfqpoint{4.482595in}{1.836609in}}%
\pgfpathlineto{\pgfqpoint{4.468649in}{1.839786in}}%
\pgfpathlineto{\pgfqpoint{4.454710in}{1.842988in}}%
\pgfpathlineto{\pgfqpoint{4.440777in}{1.846215in}}%
\pgfpathlineto{\pgfqpoint{4.426852in}{1.849467in}}%
\pgfpathlineto{\pgfqpoint{4.434711in}{1.858145in}}%
\pgfpathlineto{\pgfqpoint{4.442563in}{1.866887in}}%
\pgfpathlineto{\pgfqpoint{4.450411in}{1.875688in}}%
\pgfpathlineto{\pgfqpoint{4.458254in}{1.884545in}}%
\pgfpathclose%
\pgfusepath{fill}%
\end{pgfscope}%
\begin{pgfscope}%
\pgfpathrectangle{\pgfqpoint{1.150000in}{0.150000in}}{\pgfqpoint{5.700000in}{5.700000in}}%
\pgfusepath{clip}%
\pgfsetbuttcap%
\pgfsetroundjoin%
\definecolor{currentfill}{rgb}{0.276022,0.044167,0.370164}%
\pgfsetfillcolor{currentfill}%
\pgfsetfillopacity{0.700000}%
\pgfsetlinewidth{0.000000pt}%
\definecolor{currentstroke}{rgb}{0.000000,0.000000,0.000000}%
\pgfsetstrokecolor{currentstroke}%
\pgfsetdash{}{0pt}%
\pgfpathmoveto{\pgfqpoint{3.658945in}{1.916741in}}%
\pgfpathlineto{\pgfqpoint{3.672679in}{1.911179in}}%
\pgfpathlineto{\pgfqpoint{3.686418in}{1.905646in}}%
\pgfpathlineto{\pgfqpoint{3.700163in}{1.900140in}}%
\pgfpathlineto{\pgfqpoint{3.713913in}{1.894663in}}%
\pgfpathlineto{\pgfqpoint{3.705758in}{1.891232in}}%
\pgfpathlineto{\pgfqpoint{3.697594in}{1.888026in}}%
\pgfpathlineto{\pgfqpoint{3.689421in}{1.885053in}}%
\pgfpathlineto{\pgfqpoint{3.681238in}{1.882319in}}%
\pgfpathlineto{\pgfqpoint{3.667467in}{1.888067in}}%
\pgfpathlineto{\pgfqpoint{3.653701in}{1.893842in}}%
\pgfpathlineto{\pgfqpoint{3.639940in}{1.899646in}}%
\pgfpathlineto{\pgfqpoint{3.626184in}{1.905477in}}%
\pgfpathlineto{\pgfqpoint{3.634389in}{1.907936in}}%
\pgfpathlineto{\pgfqpoint{3.642584in}{1.910638in}}%
\pgfpathlineto{\pgfqpoint{3.650769in}{1.913575in}}%
\pgfpathlineto{\pgfqpoint{3.658945in}{1.916741in}}%
\pgfpathclose%
\pgfusepath{fill}%
\end{pgfscope}%
\begin{pgfscope}%
\pgfpathrectangle{\pgfqpoint{1.150000in}{0.150000in}}{\pgfqpoint{5.700000in}{5.700000in}}%
\pgfusepath{clip}%
\pgfsetbuttcap%
\pgfsetroundjoin%
\definecolor{currentfill}{rgb}{0.227802,0.326594,0.546532}%
\pgfsetfillcolor{currentfill}%
\pgfsetfillopacity{0.700000}%
\pgfsetlinewidth{0.000000pt}%
\definecolor{currentstroke}{rgb}{0.000000,0.000000,0.000000}%
\pgfsetstrokecolor{currentstroke}%
\pgfsetdash{}{0pt}%
\pgfpathmoveto{\pgfqpoint{2.505209in}{2.498295in}}%
\pgfpathlineto{\pgfqpoint{2.518832in}{2.488707in}}%
\pgfpathlineto{\pgfqpoint{2.532457in}{2.479164in}}%
\pgfpathlineto{\pgfqpoint{2.546083in}{2.469667in}}%
\pgfpathlineto{\pgfqpoint{2.559712in}{2.460215in}}%
\pgfpathlineto{\pgfqpoint{2.550686in}{2.468838in}}%
\pgfpathlineto{\pgfqpoint{2.541634in}{2.477917in}}%
\pgfpathlineto{\pgfqpoint{2.532556in}{2.487462in}}%
\pgfpathlineto{\pgfqpoint{2.523452in}{2.497481in}}%
\pgfpathlineto{\pgfqpoint{2.509780in}{2.507283in}}%
\pgfpathlineto{\pgfqpoint{2.496109in}{2.517130in}}%
\pgfpathlineto{\pgfqpoint{2.482440in}{2.527023in}}%
\pgfpathlineto{\pgfqpoint{2.468773in}{2.536961in}}%
\pgfpathlineto{\pgfqpoint{2.477922in}{2.526585in}}%
\pgfpathlineto{\pgfqpoint{2.487044in}{2.516689in}}%
\pgfpathlineto{\pgfqpoint{2.496139in}{2.507262in}}%
\pgfpathlineto{\pgfqpoint{2.505209in}{2.498295in}}%
\pgfpathclose%
\pgfusepath{fill}%
\end{pgfscope}%
\begin{pgfscope}%
\pgfpathrectangle{\pgfqpoint{1.150000in}{0.150000in}}{\pgfqpoint{5.700000in}{5.700000in}}%
\pgfusepath{clip}%
\pgfsetbuttcap%
\pgfsetroundjoin%
\definecolor{currentfill}{rgb}{0.268510,0.009605,0.335427}%
\pgfsetfillcolor{currentfill}%
\pgfsetfillopacity{0.700000}%
\pgfsetlinewidth{0.000000pt}%
\definecolor{currentstroke}{rgb}{0.000000,0.000000,0.000000}%
\pgfsetstrokecolor{currentstroke}%
\pgfsetdash{}{0pt}%
\pgfpathmoveto{\pgfqpoint{4.228655in}{1.858938in}}%
\pgfpathlineto{\pgfqpoint{4.242511in}{1.855178in}}%
\pgfpathlineto{\pgfqpoint{4.256374in}{1.851443in}}%
\pgfpathlineto{\pgfqpoint{4.270243in}{1.847734in}}%
\pgfpathlineto{\pgfqpoint{4.284119in}{1.844051in}}%
\pgfpathlineto{\pgfqpoint{4.276209in}{1.836191in}}%
\pgfpathlineto{\pgfqpoint{4.268294in}{1.828432in}}%
\pgfpathlineto{\pgfqpoint{4.260374in}{1.820780in}}%
\pgfpathlineto{\pgfqpoint{4.252448in}{1.813240in}}%
\pgfpathlineto{\pgfqpoint{4.238559in}{1.817140in}}%
\pgfpathlineto{\pgfqpoint{4.224676in}{1.821065in}}%
\pgfpathlineto{\pgfqpoint{4.210800in}{1.825016in}}%
\pgfpathlineto{\pgfqpoint{4.196930in}{1.828993in}}%
\pgfpathlineto{\pgfqpoint{4.204870in}{1.836312in}}%
\pgfpathlineto{\pgfqpoint{4.212804in}{1.843746in}}%
\pgfpathlineto{\pgfqpoint{4.220732in}{1.851290in}}%
\pgfpathlineto{\pgfqpoint{4.228655in}{1.858938in}}%
\pgfpathclose%
\pgfusepath{fill}%
\end{pgfscope}%
\begin{pgfscope}%
\pgfpathrectangle{\pgfqpoint{1.150000in}{0.150000in}}{\pgfqpoint{5.700000in}{5.700000in}}%
\pgfusepath{clip}%
\pgfsetbuttcap%
\pgfsetroundjoin%
\definecolor{currentfill}{rgb}{0.276194,0.190074,0.493001}%
\pgfsetfillcolor{currentfill}%
\pgfsetfillopacity{0.700000}%
\pgfsetlinewidth{0.000000pt}%
\definecolor{currentstroke}{rgb}{0.000000,0.000000,0.000000}%
\pgfsetstrokecolor{currentstroke}%
\pgfsetdash{}{0pt}%
\pgfpathmoveto{\pgfqpoint{5.583043in}{2.210305in}}%
\pgfpathlineto{\pgfqpoint{5.597309in}{2.209839in}}%
\pgfpathlineto{\pgfqpoint{5.611583in}{2.209398in}}%
\pgfpathlineto{\pgfqpoint{5.625867in}{2.208982in}}%
\pgfpathlineto{\pgfqpoint{5.640161in}{2.208590in}}%
\pgfpathlineto{\pgfqpoint{5.632736in}{2.199883in}}%
\pgfpathlineto{\pgfqpoint{5.625303in}{2.191069in}}%
\pgfpathlineto{\pgfqpoint{5.617861in}{2.182150in}}%
\pgfpathlineto{\pgfqpoint{5.610412in}{2.173126in}}%
\pgfpathlineto{\pgfqpoint{5.596108in}{2.173559in}}%
\pgfpathlineto{\pgfqpoint{5.581813in}{2.174017in}}%
\pgfpathlineto{\pgfqpoint{5.567527in}{2.174500in}}%
\pgfpathlineto{\pgfqpoint{5.553251in}{2.175008in}}%
\pgfpathlineto{\pgfqpoint{5.560711in}{2.183985in}}%
\pgfpathlineto{\pgfqpoint{5.568163in}{2.192861in}}%
\pgfpathlineto{\pgfqpoint{5.575607in}{2.201634in}}%
\pgfpathlineto{\pgfqpoint{5.583043in}{2.210305in}}%
\pgfpathclose%
\pgfusepath{fill}%
\end{pgfscope}%
\begin{pgfscope}%
\pgfpathrectangle{\pgfqpoint{1.150000in}{0.150000in}}{\pgfqpoint{5.700000in}{5.700000in}}%
\pgfusepath{clip}%
\pgfsetbuttcap%
\pgfsetroundjoin%
\definecolor{currentfill}{rgb}{0.283091,0.110553,0.431554}%
\pgfsetfillcolor{currentfill}%
\pgfsetfillopacity{0.700000}%
\pgfsetlinewidth{0.000000pt}%
\definecolor{currentstroke}{rgb}{0.000000,0.000000,0.000000}%
\pgfsetstrokecolor{currentstroke}%
\pgfsetdash{}{0pt}%
\pgfpathmoveto{\pgfqpoint{5.091976in}{2.045316in}}%
\pgfpathlineto{\pgfqpoint{5.106081in}{2.043912in}}%
\pgfpathlineto{\pgfqpoint{5.120194in}{2.042533in}}%
\pgfpathlineto{\pgfqpoint{5.134315in}{2.041178in}}%
\pgfpathlineto{\pgfqpoint{5.148445in}{2.039847in}}%
\pgfpathlineto{\pgfqpoint{5.140817in}{2.029855in}}%
\pgfpathlineto{\pgfqpoint{5.133184in}{2.019806in}}%
\pgfpathlineto{\pgfqpoint{5.125544in}{2.009701in}}%
\pgfpathlineto{\pgfqpoint{5.117898in}{1.999543in}}%
\pgfpathlineto{\pgfqpoint{5.103760in}{2.000984in}}%
\pgfpathlineto{\pgfqpoint{5.089630in}{2.002449in}}%
\pgfpathlineto{\pgfqpoint{5.075508in}{2.003939in}}%
\pgfpathlineto{\pgfqpoint{5.061395in}{2.005454in}}%
\pgfpathlineto{\pgfqpoint{5.069049in}{2.015496in}}%
\pgfpathlineto{\pgfqpoint{5.076698in}{2.025488in}}%
\pgfpathlineto{\pgfqpoint{5.084340in}{2.035429in}}%
\pgfpathlineto{\pgfqpoint{5.091976in}{2.045316in}}%
\pgfpathclose%
\pgfusepath{fill}%
\end{pgfscope}%
\begin{pgfscope}%
\pgfpathrectangle{\pgfqpoint{1.150000in}{0.150000in}}{\pgfqpoint{5.700000in}{5.700000in}}%
\pgfusepath{clip}%
\pgfsetbuttcap%
\pgfsetroundjoin%
\definecolor{currentfill}{rgb}{0.276022,0.044167,0.370164}%
\pgfsetfillcolor{currentfill}%
\pgfsetfillopacity{0.700000}%
\pgfsetlinewidth{0.000000pt}%
\definecolor{currentstroke}{rgb}{0.000000,0.000000,0.000000}%
\pgfsetstrokecolor{currentstroke}%
\pgfsetdash{}{0pt}%
\pgfpathmoveto{\pgfqpoint{4.688001in}{1.925858in}}%
\pgfpathlineto{\pgfqpoint{4.701982in}{1.923437in}}%
\pgfpathlineto{\pgfqpoint{4.715971in}{1.921041in}}%
\pgfpathlineto{\pgfqpoint{4.729968in}{1.918669in}}%
\pgfpathlineto{\pgfqpoint{4.743972in}{1.916323in}}%
\pgfpathlineto{\pgfqpoint{4.736211in}{1.906544in}}%
\pgfpathlineto{\pgfqpoint{4.728445in}{1.896775in}}%
\pgfpathlineto{\pgfqpoint{4.720673in}{1.887017in}}%
\pgfpathlineto{\pgfqpoint{4.712897in}{1.877276in}}%
\pgfpathlineto{\pgfqpoint{4.698884in}{1.879786in}}%
\pgfpathlineto{\pgfqpoint{4.684878in}{1.882321in}}%
\pgfpathlineto{\pgfqpoint{4.670879in}{1.884882in}}%
\pgfpathlineto{\pgfqpoint{4.656888in}{1.887466in}}%
\pgfpathlineto{\pgfqpoint{4.664674in}{1.897039in}}%
\pgfpathlineto{\pgfqpoint{4.672455in}{1.906631in}}%
\pgfpathlineto{\pgfqpoint{4.680230in}{1.916239in}}%
\pgfpathlineto{\pgfqpoint{4.688001in}{1.925858in}}%
\pgfpathclose%
\pgfusepath{fill}%
\end{pgfscope}%
\begin{pgfscope}%
\pgfpathrectangle{\pgfqpoint{1.150000in}{0.150000in}}{\pgfqpoint{5.700000in}{5.700000in}}%
\pgfusepath{clip}%
\pgfsetbuttcap%
\pgfsetroundjoin%
\definecolor{currentfill}{rgb}{0.282910,0.105393,0.426902}%
\pgfsetfillcolor{currentfill}%
\pgfsetfillopacity{0.700000}%
\pgfsetlinewidth{0.000000pt}%
\definecolor{currentstroke}{rgb}{0.000000,0.000000,0.000000}%
\pgfsetstrokecolor{currentstroke}%
\pgfsetdash{}{0pt}%
\pgfpathmoveto{\pgfqpoint{3.318675in}{2.026627in}}%
\pgfpathlineto{\pgfqpoint{3.332361in}{2.019921in}}%
\pgfpathlineto{\pgfqpoint{3.346051in}{2.013247in}}%
\pgfpathlineto{\pgfqpoint{3.359746in}{2.006603in}}%
\pgfpathlineto{\pgfqpoint{3.373445in}{1.999990in}}%
\pgfpathlineto{\pgfqpoint{3.365085in}{2.000026in}}%
\pgfpathlineto{\pgfqpoint{3.356712in}{2.000364in}}%
\pgfpathlineto{\pgfqpoint{3.348325in}{2.001012in}}%
\pgfpathlineto{\pgfqpoint{3.339925in}{2.001978in}}%
\pgfpathlineto{\pgfqpoint{3.326198in}{2.008889in}}%
\pgfpathlineto{\pgfqpoint{3.312475in}{2.015832in}}%
\pgfpathlineto{\pgfqpoint{3.298757in}{2.022805in}}%
\pgfpathlineto{\pgfqpoint{3.285043in}{2.029809in}}%
\pgfpathlineto{\pgfqpoint{3.293472in}{2.028539in}}%
\pgfpathlineto{\pgfqpoint{3.301886in}{2.027591in}}%
\pgfpathlineto{\pgfqpoint{3.310287in}{2.026956in}}%
\pgfpathlineto{\pgfqpoint{3.318675in}{2.026627in}}%
\pgfpathclose%
\pgfusepath{fill}%
\end{pgfscope}%
\begin{pgfscope}%
\pgfpathrectangle{\pgfqpoint{1.150000in}{0.150000in}}{\pgfqpoint{5.700000in}{5.700000in}}%
\pgfusepath{clip}%
\pgfsetbuttcap%
\pgfsetroundjoin%
\definecolor{currentfill}{rgb}{0.278012,0.180367,0.486697}%
\pgfsetfillcolor{currentfill}%
\pgfsetfillopacity{0.700000}%
\pgfsetlinewidth{0.000000pt}%
\definecolor{currentstroke}{rgb}{0.000000,0.000000,0.000000}%
\pgfsetstrokecolor{currentstroke}%
\pgfsetdash{}{0pt}%
\pgfpathmoveto{\pgfqpoint{5.496235in}{2.177284in}}%
\pgfpathlineto{\pgfqpoint{5.510475in}{2.176678in}}%
\pgfpathlineto{\pgfqpoint{5.524725in}{2.176096in}}%
\pgfpathlineto{\pgfqpoint{5.538983in}{2.175540in}}%
\pgfpathlineto{\pgfqpoint{5.553251in}{2.175008in}}%
\pgfpathlineto{\pgfqpoint{5.545782in}{2.165929in}}%
\pgfpathlineto{\pgfqpoint{5.538306in}{2.156751in}}%
\pgfpathlineto{\pgfqpoint{5.530822in}{2.147473in}}%
\pgfpathlineto{\pgfqpoint{5.523331in}{2.138098in}}%
\pgfpathlineto{\pgfqpoint{5.509053in}{2.138685in}}%
\pgfpathlineto{\pgfqpoint{5.494785in}{2.139298in}}%
\pgfpathlineto{\pgfqpoint{5.480526in}{2.139935in}}%
\pgfpathlineto{\pgfqpoint{5.466275in}{2.140596in}}%
\pgfpathlineto{\pgfqpoint{5.473777in}{2.149911in}}%
\pgfpathlineto{\pgfqpoint{5.481271in}{2.159132in}}%
\pgfpathlineto{\pgfqpoint{5.488757in}{2.168256in}}%
\pgfpathlineto{\pgfqpoint{5.496235in}{2.177284in}}%
\pgfpathclose%
\pgfusepath{fill}%
\end{pgfscope}%
\begin{pgfscope}%
\pgfpathrectangle{\pgfqpoint{1.150000in}{0.150000in}}{\pgfqpoint{5.700000in}{5.700000in}}%
\pgfusepath{clip}%
\pgfsetbuttcap%
\pgfsetroundjoin%
\definecolor{currentfill}{rgb}{0.279566,0.067836,0.391917}%
\pgfsetfillcolor{currentfill}%
\pgfsetfillopacity{0.700000}%
\pgfsetlinewidth{0.000000pt}%
\definecolor{currentstroke}{rgb}{0.000000,0.000000,0.000000}%
\pgfsetstrokecolor{currentstroke}%
\pgfsetdash{}{0pt}%
\pgfpathmoveto{\pgfqpoint{3.516318in}{1.953157in}}%
\pgfpathlineto{\pgfqpoint{3.530034in}{1.947096in}}%
\pgfpathlineto{\pgfqpoint{3.543755in}{1.941064in}}%
\pgfpathlineto{\pgfqpoint{3.557481in}{1.935061in}}%
\pgfpathlineto{\pgfqpoint{3.571211in}{1.929087in}}%
\pgfpathlineto{\pgfqpoint{3.562973in}{1.927157in}}%
\pgfpathlineto{\pgfqpoint{3.554724in}{1.925487in}}%
\pgfpathlineto{\pgfqpoint{3.546464in}{1.924086in}}%
\pgfpathlineto{\pgfqpoint{3.538193in}{1.922961in}}%
\pgfpathlineto{\pgfqpoint{3.524438in}{1.929219in}}%
\pgfpathlineto{\pgfqpoint{3.510688in}{1.935505in}}%
\pgfpathlineto{\pgfqpoint{3.496943in}{1.941821in}}%
\pgfpathlineto{\pgfqpoint{3.483203in}{1.948166in}}%
\pgfpathlineto{\pgfqpoint{3.491499in}{1.949002in}}%
\pgfpathlineto{\pgfqpoint{3.499783in}{1.950118in}}%
\pgfpathlineto{\pgfqpoint{3.508056in}{1.951505in}}%
\pgfpathlineto{\pgfqpoint{3.516318in}{1.953157in}}%
\pgfpathclose%
\pgfusepath{fill}%
\end{pgfscope}%
\begin{pgfscope}%
\pgfpathrectangle{\pgfqpoint{1.150000in}{0.150000in}}{\pgfqpoint{5.700000in}{5.700000in}}%
\pgfusepath{clip}%
\pgfsetbuttcap%
\pgfsetroundjoin%
\definecolor{currentfill}{rgb}{0.282327,0.094955,0.417331}%
\pgfsetfillcolor{currentfill}%
\pgfsetfillopacity{0.700000}%
\pgfsetlinewidth{0.000000pt}%
\definecolor{currentstroke}{rgb}{0.000000,0.000000,0.000000}%
\pgfsetstrokecolor{currentstroke}%
\pgfsetdash{}{0pt}%
\pgfpathmoveto{\pgfqpoint{5.005024in}{2.011759in}}%
\pgfpathlineto{\pgfqpoint{5.019104in}{2.010146in}}%
\pgfpathlineto{\pgfqpoint{5.033193in}{2.008557in}}%
\pgfpathlineto{\pgfqpoint{5.047290in}{2.006993in}}%
\pgfpathlineto{\pgfqpoint{5.061395in}{2.005454in}}%
\pgfpathlineto{\pgfqpoint{5.053734in}{1.995365in}}%
\pgfpathlineto{\pgfqpoint{5.046068in}{1.985233in}}%
\pgfpathlineto{\pgfqpoint{5.038396in}{1.975059in}}%
\pgfpathlineto{\pgfqpoint{5.030718in}{1.964847in}}%
\pgfpathlineto{\pgfqpoint{5.016605in}{1.966511in}}%
\pgfpathlineto{\pgfqpoint{5.002500in}{1.968198in}}%
\pgfpathlineto{\pgfqpoint{4.988402in}{1.969911in}}%
\pgfpathlineto{\pgfqpoint{4.974313in}{1.971648in}}%
\pgfpathlineto{\pgfqpoint{4.982000in}{1.981731in}}%
\pgfpathlineto{\pgfqpoint{4.989680in}{1.991779in}}%
\pgfpathlineto{\pgfqpoint{4.997355in}{2.001789in}}%
\pgfpathlineto{\pgfqpoint{5.005024in}{2.011759in}}%
\pgfpathclose%
\pgfusepath{fill}%
\end{pgfscope}%
\begin{pgfscope}%
\pgfpathrectangle{\pgfqpoint{1.150000in}{0.150000in}}{\pgfqpoint{5.700000in}{5.700000in}}%
\pgfusepath{clip}%
\pgfsetbuttcap%
\pgfsetroundjoin%
\definecolor{currentfill}{rgb}{0.269944,0.014625,0.341379}%
\pgfsetfillcolor{currentfill}%
\pgfsetfillopacity{0.700000}%
\pgfsetlinewidth{0.000000pt}%
\definecolor{currentstroke}{rgb}{0.000000,0.000000,0.000000}%
\pgfsetstrokecolor{currentstroke}%
\pgfsetdash{}{0pt}%
\pgfpathmoveto{\pgfqpoint{4.371221in}{1.862728in}}%
\pgfpathlineto{\pgfqpoint{4.385119in}{1.859375in}}%
\pgfpathlineto{\pgfqpoint{4.399023in}{1.856047in}}%
\pgfpathlineto{\pgfqpoint{4.412934in}{1.852744in}}%
\pgfpathlineto{\pgfqpoint{4.426852in}{1.849467in}}%
\pgfpathlineto{\pgfqpoint{4.418989in}{1.840859in}}%
\pgfpathlineto{\pgfqpoint{4.411120in}{1.832324in}}%
\pgfpathlineto{\pgfqpoint{4.403247in}{1.823868in}}%
\pgfpathlineto{\pgfqpoint{4.395368in}{1.815497in}}%
\pgfpathlineto{\pgfqpoint{4.381438in}{1.818977in}}%
\pgfpathlineto{\pgfqpoint{4.367515in}{1.822483in}}%
\pgfpathlineto{\pgfqpoint{4.353598in}{1.826015in}}%
\pgfpathlineto{\pgfqpoint{4.339689in}{1.829571in}}%
\pgfpathlineto{\pgfqpoint{4.347580in}{1.837734in}}%
\pgfpathlineto{\pgfqpoint{4.355466in}{1.845985in}}%
\pgfpathlineto{\pgfqpoint{4.363346in}{1.854318in}}%
\pgfpathlineto{\pgfqpoint{4.371221in}{1.862728in}}%
\pgfpathclose%
\pgfusepath{fill}%
\end{pgfscope}%
\begin{pgfscope}%
\pgfpathrectangle{\pgfqpoint{1.150000in}{0.150000in}}{\pgfqpoint{5.700000in}{5.700000in}}%
\pgfusepath{clip}%
\pgfsetbuttcap%
\pgfsetroundjoin%
\definecolor{currentfill}{rgb}{0.233603,0.313828,0.543914}%
\pgfsetfillcolor{currentfill}%
\pgfsetfillopacity{0.700000}%
\pgfsetlinewidth{0.000000pt}%
\definecolor{currentstroke}{rgb}{0.000000,0.000000,0.000000}%
\pgfsetstrokecolor{currentstroke}%
\pgfsetdash{}{0pt}%
\pgfpathmoveto{\pgfqpoint{2.559712in}{2.460215in}}%
\pgfpathlineto{\pgfqpoint{2.573343in}{2.450807in}}%
\pgfpathlineto{\pgfqpoint{2.586976in}{2.441443in}}%
\pgfpathlineto{\pgfqpoint{2.600611in}{2.432122in}}%
\pgfpathlineto{\pgfqpoint{2.614248in}{2.422845in}}%
\pgfpathlineto{\pgfqpoint{2.605264in}{2.431126in}}%
\pgfpathlineto{\pgfqpoint{2.596255in}{2.439858in}}%
\pgfpathlineto{\pgfqpoint{2.587220in}{2.449051in}}%
\pgfpathlineto{\pgfqpoint{2.578160in}{2.458715in}}%
\pgfpathlineto{\pgfqpoint{2.564480in}{2.468341in}}%
\pgfpathlineto{\pgfqpoint{2.550802in}{2.478011in}}%
\pgfpathlineto{\pgfqpoint{2.537126in}{2.487724in}}%
\pgfpathlineto{\pgfqpoint{2.523452in}{2.497481in}}%
\pgfpathlineto{\pgfqpoint{2.532556in}{2.487462in}}%
\pgfpathlineto{\pgfqpoint{2.541634in}{2.477917in}}%
\pgfpathlineto{\pgfqpoint{2.550686in}{2.468838in}}%
\pgfpathlineto{\pgfqpoint{2.559712in}{2.460215in}}%
\pgfpathclose%
\pgfusepath{fill}%
\end{pgfscope}%
\begin{pgfscope}%
\pgfpathrectangle{\pgfqpoint{1.150000in}{0.150000in}}{\pgfqpoint{5.700000in}{5.700000in}}%
\pgfusepath{clip}%
\pgfsetbuttcap%
\pgfsetroundjoin%
\definecolor{currentfill}{rgb}{0.267968,0.223549,0.512008}%
\pgfsetfillcolor{currentfill}%
\pgfsetfillopacity{0.700000}%
\pgfsetlinewidth{0.000000pt}%
\definecolor{currentstroke}{rgb}{0.000000,0.000000,0.000000}%
\pgfsetstrokecolor{currentstroke}%
\pgfsetdash{}{0pt}%
\pgfpathmoveto{\pgfqpoint{2.867861in}{2.256502in}}%
\pgfpathlineto{\pgfqpoint{2.881505in}{2.248236in}}%
\pgfpathlineto{\pgfqpoint{2.895152in}{2.240007in}}%
\pgfpathlineto{\pgfqpoint{2.908802in}{2.231814in}}%
\pgfpathlineto{\pgfqpoint{2.922455in}{2.223659in}}%
\pgfpathlineto{\pgfqpoint{2.913750in}{2.228600in}}%
\pgfpathlineto{\pgfqpoint{2.905026in}{2.233935in}}%
\pgfpathlineto{\pgfqpoint{2.896282in}{2.239674in}}%
\pgfpathlineto{\pgfqpoint{2.887517in}{2.245827in}}%
\pgfpathlineto{\pgfqpoint{2.873827in}{2.254312in}}%
\pgfpathlineto{\pgfqpoint{2.860140in}{2.262835in}}%
\pgfpathlineto{\pgfqpoint{2.846456in}{2.271394in}}%
\pgfpathlineto{\pgfqpoint{2.832776in}{2.279991in}}%
\pgfpathlineto{\pgfqpoint{2.841578in}{2.273502in}}%
\pgfpathlineto{\pgfqpoint{2.850360in}{2.267431in}}%
\pgfpathlineto{\pgfqpoint{2.859121in}{2.261767in}}%
\pgfpathlineto{\pgfqpoint{2.867861in}{2.256502in}}%
\pgfpathclose%
\pgfusepath{fill}%
\end{pgfscope}%
\begin{pgfscope}%
\pgfpathrectangle{\pgfqpoint{1.150000in}{0.150000in}}{\pgfqpoint{5.700000in}{5.700000in}}%
\pgfusepath{clip}%
\pgfsetbuttcap%
\pgfsetroundjoin%
\definecolor{currentfill}{rgb}{0.281412,0.155834,0.469201}%
\pgfsetfillcolor{currentfill}%
\pgfsetfillopacity{0.700000}%
\pgfsetlinewidth{0.000000pt}%
\definecolor{currentstroke}{rgb}{0.000000,0.000000,0.000000}%
\pgfsetstrokecolor{currentstroke}%
\pgfsetdash{}{0pt}%
\pgfpathmoveto{\pgfqpoint{3.120792in}{2.116333in}}%
\pgfpathlineto{\pgfqpoint{3.134458in}{2.108944in}}%
\pgfpathlineto{\pgfqpoint{3.148127in}{2.101588in}}%
\pgfpathlineto{\pgfqpoint{3.161801in}{2.094265in}}%
\pgfpathlineto{\pgfqpoint{3.175478in}{2.086974in}}%
\pgfpathlineto{\pgfqpoint{3.166975in}{2.089189in}}%
\pgfpathlineto{\pgfqpoint{3.158456in}{2.091749in}}%
\pgfpathlineto{\pgfqpoint{3.149921in}{2.094662in}}%
\pgfpathlineto{\pgfqpoint{3.141370in}{2.097938in}}%
\pgfpathlineto{\pgfqpoint{3.127661in}{2.105542in}}%
\pgfpathlineto{\pgfqpoint{3.113955in}{2.113178in}}%
\pgfpathlineto{\pgfqpoint{3.100254in}{2.120848in}}%
\pgfpathlineto{\pgfqpoint{3.086556in}{2.128551in}}%
\pgfpathlineto{\pgfqpoint{3.095140in}{2.124956in}}%
\pgfpathlineto{\pgfqpoint{3.103707in}{2.121727in}}%
\pgfpathlineto{\pgfqpoint{3.112257in}{2.118856in}}%
\pgfpathlineto{\pgfqpoint{3.120792in}{2.116333in}}%
\pgfpathclose%
\pgfusepath{fill}%
\end{pgfscope}%
\begin{pgfscope}%
\pgfpathrectangle{\pgfqpoint{1.150000in}{0.150000in}}{\pgfqpoint{5.700000in}{5.700000in}}%
\pgfusepath{clip}%
\pgfsetbuttcap%
\pgfsetroundjoin%
\definecolor{currentfill}{rgb}{0.273809,0.031497,0.358853}%
\pgfsetfillcolor{currentfill}%
\pgfsetfillopacity{0.700000}%
\pgfsetlinewidth{0.000000pt}%
\definecolor{currentstroke}{rgb}{0.000000,0.000000,0.000000}%
\pgfsetstrokecolor{currentstroke}%
\pgfsetdash{}{0pt}%
\pgfpathmoveto{\pgfqpoint{4.600999in}{1.898055in}}%
\pgfpathlineto{\pgfqpoint{4.614960in}{1.895371in}}%
\pgfpathlineto{\pgfqpoint{4.628929in}{1.892711in}}%
\pgfpathlineto{\pgfqpoint{4.642905in}{1.890076in}}%
\pgfpathlineto{\pgfqpoint{4.656888in}{1.887466in}}%
\pgfpathlineto{\pgfqpoint{4.649097in}{1.877918in}}%
\pgfpathlineto{\pgfqpoint{4.641301in}{1.868396in}}%
\pgfpathlineto{\pgfqpoint{4.633500in}{1.858907in}}%
\pgfpathlineto{\pgfqpoint{4.625694in}{1.849454in}}%
\pgfpathlineto{\pgfqpoint{4.611701in}{1.852241in}}%
\pgfpathlineto{\pgfqpoint{4.597715in}{1.855053in}}%
\pgfpathlineto{\pgfqpoint{4.583736in}{1.857890in}}%
\pgfpathlineto{\pgfqpoint{4.569765in}{1.860751in}}%
\pgfpathlineto{\pgfqpoint{4.577581in}{1.870022in}}%
\pgfpathlineto{\pgfqpoint{4.585392in}{1.879333in}}%
\pgfpathlineto{\pgfqpoint{4.593198in}{1.888678in}}%
\pgfpathlineto{\pgfqpoint{4.600999in}{1.898055in}}%
\pgfpathclose%
\pgfusepath{fill}%
\end{pgfscope}%
\begin{pgfscope}%
\pgfpathrectangle{\pgfqpoint{1.150000in}{0.150000in}}{\pgfqpoint{5.700000in}{5.700000in}}%
\pgfusepath{clip}%
\pgfsetbuttcap%
\pgfsetroundjoin%
\definecolor{currentfill}{rgb}{0.268510,0.009605,0.335427}%
\pgfsetfillcolor{currentfill}%
\pgfsetfillopacity{0.700000}%
\pgfsetlinewidth{0.000000pt}%
\definecolor{currentstroke}{rgb}{0.000000,0.000000,0.000000}%
\pgfsetstrokecolor{currentstroke}%
\pgfsetdash{}{0pt}%
\pgfpathmoveto{\pgfqpoint{3.998991in}{1.853475in}}%
\pgfpathlineto{\pgfqpoint{4.012800in}{1.848957in}}%
\pgfpathlineto{\pgfqpoint{4.026616in}{1.844465in}}%
\pgfpathlineto{\pgfqpoint{4.040438in}{1.839999in}}%
\pgfpathlineto{\pgfqpoint{4.054266in}{1.835559in}}%
\pgfpathlineto{\pgfqpoint{4.046265in}{1.829377in}}%
\pgfpathlineto{\pgfqpoint{4.038257in}{1.823350in}}%
\pgfpathlineto{\pgfqpoint{4.030243in}{1.817486in}}%
\pgfpathlineto{\pgfqpoint{4.022222in}{1.811791in}}%
\pgfpathlineto{\pgfqpoint{4.008377in}{1.816473in}}%
\pgfpathlineto{\pgfqpoint{3.994539in}{1.821182in}}%
\pgfpathlineto{\pgfqpoint{3.980707in}{1.825917in}}%
\pgfpathlineto{\pgfqpoint{3.966881in}{1.830678in}}%
\pgfpathlineto{\pgfqpoint{3.974919in}{1.836125in}}%
\pgfpathlineto{\pgfqpoint{3.982950in}{1.841745in}}%
\pgfpathlineto{\pgfqpoint{3.990974in}{1.847530in}}%
\pgfpathlineto{\pgfqpoint{3.998991in}{1.853475in}}%
\pgfpathclose%
\pgfusepath{fill}%
\end{pgfscope}%
\begin{pgfscope}%
\pgfpathrectangle{\pgfqpoint{1.150000in}{0.150000in}}{\pgfqpoint{5.700000in}{5.700000in}}%
\pgfusepath{clip}%
\pgfsetbuttcap%
\pgfsetroundjoin%
\definecolor{currentfill}{rgb}{0.271305,0.019942,0.347269}%
\pgfsetfillcolor{currentfill}%
\pgfsetfillopacity{0.700000}%
\pgfsetlinewidth{0.000000pt}%
\definecolor{currentstroke}{rgb}{0.000000,0.000000,0.000000}%
\pgfsetstrokecolor{currentstroke}%
\pgfsetdash{}{0pt}%
\pgfpathmoveto{\pgfqpoint{3.856480in}{1.869724in}}%
\pgfpathlineto{\pgfqpoint{3.870260in}{1.864750in}}%
\pgfpathlineto{\pgfqpoint{3.884045in}{1.859802in}}%
\pgfpathlineto{\pgfqpoint{3.897837in}{1.854881in}}%
\pgfpathlineto{\pgfqpoint{3.911634in}{1.849987in}}%
\pgfpathlineto{\pgfqpoint{3.903571in}{1.844970in}}%
\pgfpathlineto{\pgfqpoint{3.895500in}{1.840140in}}%
\pgfpathlineto{\pgfqpoint{3.887421in}{1.835506in}}%
\pgfpathlineto{\pgfqpoint{3.879335in}{1.831072in}}%
\pgfpathlineto{\pgfqpoint{3.865519in}{1.836223in}}%
\pgfpathlineto{\pgfqpoint{3.851709in}{1.841400in}}%
\pgfpathlineto{\pgfqpoint{3.837905in}{1.846604in}}%
\pgfpathlineto{\pgfqpoint{3.824106in}{1.851834in}}%
\pgfpathlineto{\pgfqpoint{3.832212in}{1.856006in}}%
\pgfpathlineto{\pgfqpoint{3.840309in}{1.860383in}}%
\pgfpathlineto{\pgfqpoint{3.848398in}{1.864958in}}%
\pgfpathlineto{\pgfqpoint{3.856480in}{1.869724in}}%
\pgfpathclose%
\pgfusepath{fill}%
\end{pgfscope}%
\begin{pgfscope}%
\pgfpathrectangle{\pgfqpoint{1.150000in}{0.150000in}}{\pgfqpoint{5.700000in}{5.700000in}}%
\pgfusepath{clip}%
\pgfsetbuttcap%
\pgfsetroundjoin%
\definecolor{currentfill}{rgb}{0.280255,0.165693,0.476498}%
\pgfsetfillcolor{currentfill}%
\pgfsetfillopacity{0.700000}%
\pgfsetlinewidth{0.000000pt}%
\definecolor{currentstroke}{rgb}{0.000000,0.000000,0.000000}%
\pgfsetstrokecolor{currentstroke}%
\pgfsetdash{}{0pt}%
\pgfpathmoveto{\pgfqpoint{5.409363in}{2.143490in}}%
\pgfpathlineto{\pgfqpoint{5.423578in}{2.142730in}}%
\pgfpathlineto{\pgfqpoint{5.437801in}{2.141994in}}%
\pgfpathlineto{\pgfqpoint{5.452034in}{2.141283in}}%
\pgfpathlineto{\pgfqpoint{5.466275in}{2.140596in}}%
\pgfpathlineto{\pgfqpoint{5.458766in}{2.131188in}}%
\pgfpathlineto{\pgfqpoint{5.451250in}{2.121687in}}%
\pgfpathlineto{\pgfqpoint{5.443726in}{2.112094in}}%
\pgfpathlineto{\pgfqpoint{5.436194in}{2.102412in}}%
\pgfpathlineto{\pgfqpoint{5.421943in}{2.103168in}}%
\pgfpathlineto{\pgfqpoint{5.407701in}{2.103949in}}%
\pgfpathlineto{\pgfqpoint{5.393468in}{2.104754in}}%
\pgfpathlineto{\pgfqpoint{5.379244in}{2.105583in}}%
\pgfpathlineto{\pgfqpoint{5.386785in}{2.115191in}}%
\pgfpathlineto{\pgfqpoint{5.394318in}{2.124712in}}%
\pgfpathlineto{\pgfqpoint{5.401844in}{2.134146in}}%
\pgfpathlineto{\pgfqpoint{5.409363in}{2.143490in}}%
\pgfpathclose%
\pgfusepath{fill}%
\end{pgfscope}%
\begin{pgfscope}%
\pgfpathrectangle{\pgfqpoint{1.150000in}{0.150000in}}{\pgfqpoint{5.700000in}{5.700000in}}%
\pgfusepath{clip}%
\pgfsetbuttcap%
\pgfsetroundjoin%
\definecolor{currentfill}{rgb}{0.280894,0.078907,0.402329}%
\pgfsetfillcolor{currentfill}%
\pgfsetfillopacity{0.700000}%
\pgfsetlinewidth{0.000000pt}%
\definecolor{currentstroke}{rgb}{0.000000,0.000000,0.000000}%
\pgfsetstrokecolor{currentstroke}%
\pgfsetdash{}{0pt}%
\pgfpathmoveto{\pgfqpoint{4.918038in}{1.978844in}}%
\pgfpathlineto{\pgfqpoint{4.932095in}{1.977008in}}%
\pgfpathlineto{\pgfqpoint{4.946160in}{1.975196in}}%
\pgfpathlineto{\pgfqpoint{4.960233in}{1.973410in}}%
\pgfpathlineto{\pgfqpoint{4.974313in}{1.971648in}}%
\pgfpathlineto{\pgfqpoint{4.966622in}{1.961533in}}%
\pgfpathlineto{\pgfqpoint{4.958924in}{1.951389in}}%
\pgfpathlineto{\pgfqpoint{4.951222in}{1.941219in}}%
\pgfpathlineto{\pgfqpoint{4.943513in}{1.931026in}}%
\pgfpathlineto{\pgfqpoint{4.929424in}{1.932925in}}%
\pgfpathlineto{\pgfqpoint{4.915342in}{1.934849in}}%
\pgfpathlineto{\pgfqpoint{4.901269in}{1.936797in}}%
\pgfpathlineto{\pgfqpoint{4.887204in}{1.938770in}}%
\pgfpathlineto{\pgfqpoint{4.894920in}{1.948821in}}%
\pgfpathlineto{\pgfqpoint{4.902632in}{1.958852in}}%
\pgfpathlineto{\pgfqpoint{4.910338in}{1.968861in}}%
\pgfpathlineto{\pgfqpoint{4.918038in}{1.978844in}}%
\pgfpathclose%
\pgfusepath{fill}%
\end{pgfscope}%
\begin{pgfscope}%
\pgfpathrectangle{\pgfqpoint{1.150000in}{0.150000in}}{\pgfqpoint{5.700000in}{5.700000in}}%
\pgfusepath{clip}%
\pgfsetbuttcap%
\pgfsetroundjoin%
\definecolor{currentfill}{rgb}{0.268510,0.009605,0.335427}%
\pgfsetfillcolor{currentfill}%
\pgfsetfillopacity{0.700000}%
\pgfsetlinewidth{0.000000pt}%
\definecolor{currentstroke}{rgb}{0.000000,0.000000,0.000000}%
\pgfsetstrokecolor{currentstroke}%
\pgfsetdash{}{0pt}%
\pgfpathmoveto{\pgfqpoint{4.141516in}{1.845157in}}%
\pgfpathlineto{\pgfqpoint{4.155360in}{1.841078in}}%
\pgfpathlineto{\pgfqpoint{4.169211in}{1.837024in}}%
\pgfpathlineto{\pgfqpoint{4.183067in}{1.832995in}}%
\pgfpathlineto{\pgfqpoint{4.196930in}{1.828993in}}%
\pgfpathlineto{\pgfqpoint{4.188985in}{1.821795in}}%
\pgfpathlineto{\pgfqpoint{4.181033in}{1.814722in}}%
\pgfpathlineto{\pgfqpoint{4.173075in}{1.807782in}}%
\pgfpathlineto{\pgfqpoint{4.165111in}{1.800980in}}%
\pgfpathlineto{\pgfqpoint{4.151234in}{1.805212in}}%
\pgfpathlineto{\pgfqpoint{4.137362in}{1.809470in}}%
\pgfpathlineto{\pgfqpoint{4.123497in}{1.813753in}}%
\pgfpathlineto{\pgfqpoint{4.109639in}{1.818062in}}%
\pgfpathlineto{\pgfqpoint{4.117617in}{1.824630in}}%
\pgfpathlineto{\pgfqpoint{4.125590in}{1.831339in}}%
\pgfpathlineto{\pgfqpoint{4.133556in}{1.838184in}}%
\pgfpathlineto{\pgfqpoint{4.141516in}{1.845157in}}%
\pgfpathclose%
\pgfusepath{fill}%
\end{pgfscope}%
\begin{pgfscope}%
\pgfpathrectangle{\pgfqpoint{1.150000in}{0.150000in}}{\pgfqpoint{5.700000in}{5.700000in}}%
\pgfusepath{clip}%
\pgfsetbuttcap%
\pgfsetroundjoin%
\definecolor{currentfill}{rgb}{0.274952,0.037752,0.364543}%
\pgfsetfillcolor{currentfill}%
\pgfsetfillopacity{0.700000}%
\pgfsetlinewidth{0.000000pt}%
\definecolor{currentstroke}{rgb}{0.000000,0.000000,0.000000}%
\pgfsetstrokecolor{currentstroke}%
\pgfsetdash{}{0pt}%
\pgfpathmoveto{\pgfqpoint{3.713913in}{1.894663in}}%
\pgfpathlineto{\pgfqpoint{3.727668in}{1.889213in}}%
\pgfpathlineto{\pgfqpoint{3.741429in}{1.883791in}}%
\pgfpathlineto{\pgfqpoint{3.755195in}{1.878396in}}%
\pgfpathlineto{\pgfqpoint{3.768966in}{1.873029in}}%
\pgfpathlineto{\pgfqpoint{3.760832in}{1.869334in}}%
\pgfpathlineto{\pgfqpoint{3.752689in}{1.865860in}}%
\pgfpathlineto{\pgfqpoint{3.744537in}{1.862615in}}%
\pgfpathlineto{\pgfqpoint{3.736376in}{1.859606in}}%
\pgfpathlineto{\pgfqpoint{3.722584in}{1.865243in}}%
\pgfpathlineto{\pgfqpoint{3.708797in}{1.870907in}}%
\pgfpathlineto{\pgfqpoint{3.695015in}{1.876599in}}%
\pgfpathlineto{\pgfqpoint{3.681238in}{1.882319in}}%
\pgfpathlineto{\pgfqpoint{3.689421in}{1.885053in}}%
\pgfpathlineto{\pgfqpoint{3.697594in}{1.888026in}}%
\pgfpathlineto{\pgfqpoint{3.705758in}{1.891232in}}%
\pgfpathlineto{\pgfqpoint{3.713913in}{1.894663in}}%
\pgfpathclose%
\pgfusepath{fill}%
\end{pgfscope}%
\begin{pgfscope}%
\pgfpathrectangle{\pgfqpoint{1.150000in}{0.150000in}}{\pgfqpoint{5.700000in}{5.700000in}}%
\pgfusepath{clip}%
\pgfsetbuttcap%
\pgfsetroundjoin%
\definecolor{currentfill}{rgb}{0.281887,0.150881,0.465405}%
\pgfsetfillcolor{currentfill}%
\pgfsetfillopacity{0.700000}%
\pgfsetlinewidth{0.000000pt}%
\definecolor{currentstroke}{rgb}{0.000000,0.000000,0.000000}%
\pgfsetstrokecolor{currentstroke}%
\pgfsetdash{}{0pt}%
\pgfpathmoveto{\pgfqpoint{5.322436in}{2.109149in}}%
\pgfpathlineto{\pgfqpoint{5.336624in}{2.108221in}}%
\pgfpathlineto{\pgfqpoint{5.350822in}{2.107317in}}%
\pgfpathlineto{\pgfqpoint{5.365029in}{2.106438in}}%
\pgfpathlineto{\pgfqpoint{5.379244in}{2.105583in}}%
\pgfpathlineto{\pgfqpoint{5.371696in}{2.095891in}}%
\pgfpathlineto{\pgfqpoint{5.364141in}{2.086115in}}%
\pgfpathlineto{\pgfqpoint{5.356580in}{2.076257in}}%
\pgfpathlineto{\pgfqpoint{5.349011in}{2.066320in}}%
\pgfpathlineto{\pgfqpoint{5.334787in}{2.067257in}}%
\pgfpathlineto{\pgfqpoint{5.320571in}{2.068220in}}%
\pgfpathlineto{\pgfqpoint{5.306364in}{2.069207in}}%
\pgfpathlineto{\pgfqpoint{5.292166in}{2.070218in}}%
\pgfpathlineto{\pgfqpoint{5.299744in}{2.080067in}}%
\pgfpathlineto{\pgfqpoint{5.307315in}{2.089840in}}%
\pgfpathlineto{\pgfqpoint{5.314878in}{2.099535in}}%
\pgfpathlineto{\pgfqpoint{5.322436in}{2.109149in}}%
\pgfpathclose%
\pgfusepath{fill}%
\end{pgfscope}%
\begin{pgfscope}%
\pgfpathrectangle{\pgfqpoint{1.150000in}{0.150000in}}{\pgfqpoint{5.700000in}{5.700000in}}%
\pgfusepath{clip}%
\pgfsetbuttcap%
\pgfsetroundjoin%
\definecolor{currentfill}{rgb}{0.270595,0.214069,0.507052}%
\pgfsetfillcolor{currentfill}%
\pgfsetfillopacity{0.700000}%
\pgfsetlinewidth{0.000000pt}%
\definecolor{currentstroke}{rgb}{0.000000,0.000000,0.000000}%
\pgfsetstrokecolor{currentstroke}%
\pgfsetdash{}{0pt}%
\pgfpathmoveto{\pgfqpoint{5.726995in}{2.241141in}}%
\pgfpathlineto{\pgfqpoint{5.741323in}{2.240901in}}%
\pgfpathlineto{\pgfqpoint{5.755661in}{2.240685in}}%
\pgfpathlineto{\pgfqpoint{5.770008in}{2.240494in}}%
\pgfpathlineto{\pgfqpoint{5.762638in}{2.232175in}}%
\pgfpathlineto{\pgfqpoint{5.755258in}{2.223743in}}%
\pgfpathlineto{\pgfqpoint{5.747870in}{2.215198in}}%
\pgfpathlineto{\pgfqpoint{5.740474in}{2.206540in}}%
\pgfpathlineto{\pgfqpoint{5.726115in}{2.206758in}}%
\pgfpathlineto{\pgfqpoint{5.711766in}{2.207002in}}%
\pgfpathlineto{\pgfqpoint{5.697426in}{2.207270in}}%
\pgfpathlineto{\pgfqpoint{5.704832in}{2.215904in}}%
\pgfpathlineto{\pgfqpoint{5.712228in}{2.224427in}}%
\pgfpathlineto{\pgfqpoint{5.719616in}{2.232839in}}%
\pgfpathlineto{\pgfqpoint{5.726995in}{2.241141in}}%
\pgfpathclose%
\pgfusepath{fill}%
\end{pgfscope}%
\begin{pgfscope}%
\pgfpathrectangle{\pgfqpoint{1.150000in}{0.150000in}}{\pgfqpoint{5.700000in}{5.700000in}}%
\pgfusepath{clip}%
\pgfsetbuttcap%
\pgfsetroundjoin%
\definecolor{currentfill}{rgb}{0.279566,0.067836,0.391917}%
\pgfsetfillcolor{currentfill}%
\pgfsetfillopacity{0.700000}%
\pgfsetlinewidth{0.000000pt}%
\definecolor{currentstroke}{rgb}{0.000000,0.000000,0.000000}%
\pgfsetstrokecolor{currentstroke}%
\pgfsetdash{}{0pt}%
\pgfpathmoveto{\pgfqpoint{4.831021in}{1.946910in}}%
\pgfpathlineto{\pgfqpoint{4.845055in}{1.944838in}}%
\pgfpathlineto{\pgfqpoint{4.859097in}{1.942791in}}%
\pgfpathlineto{\pgfqpoint{4.873146in}{1.940768in}}%
\pgfpathlineto{\pgfqpoint{4.887204in}{1.938770in}}%
\pgfpathlineto{\pgfqpoint{4.879481in}{1.928704in}}%
\pgfpathlineto{\pgfqpoint{4.871754in}{1.918624in}}%
\pgfpathlineto{\pgfqpoint{4.864021in}{1.908536in}}%
\pgfpathlineto{\pgfqpoint{4.856283in}{1.898442in}}%
\pgfpathlineto{\pgfqpoint{4.842217in}{1.900590in}}%
\pgfpathlineto{\pgfqpoint{4.828159in}{1.902764in}}%
\pgfpathlineto{\pgfqpoint{4.814108in}{1.904962in}}%
\pgfpathlineto{\pgfqpoint{4.800066in}{1.907184in}}%
\pgfpathlineto{\pgfqpoint{4.807812in}{1.917123in}}%
\pgfpathlineto{\pgfqpoint{4.815554in}{1.927059in}}%
\pgfpathlineto{\pgfqpoint{4.823290in}{1.936989in}}%
\pgfpathlineto{\pgfqpoint{4.831021in}{1.946910in}}%
\pgfpathclose%
\pgfusepath{fill}%
\end{pgfscope}%
\begin{pgfscope}%
\pgfpathrectangle{\pgfqpoint{1.150000in}{0.150000in}}{\pgfqpoint{5.700000in}{5.700000in}}%
\pgfusepath{clip}%
\pgfsetbuttcap%
\pgfsetroundjoin%
\definecolor{currentfill}{rgb}{0.272594,0.025563,0.353093}%
\pgfsetfillcolor{currentfill}%
\pgfsetfillopacity{0.700000}%
\pgfsetlinewidth{0.000000pt}%
\definecolor{currentstroke}{rgb}{0.000000,0.000000,0.000000}%
\pgfsetstrokecolor{currentstroke}%
\pgfsetdash{}{0pt}%
\pgfpathmoveto{\pgfqpoint{4.513952in}{1.872447in}}%
\pgfpathlineto{\pgfqpoint{4.527894in}{1.869486in}}%
\pgfpathlineto{\pgfqpoint{4.541844in}{1.866549in}}%
\pgfpathlineto{\pgfqpoint{4.555801in}{1.863638in}}%
\pgfpathlineto{\pgfqpoint{4.569765in}{1.860751in}}%
\pgfpathlineto{\pgfqpoint{4.561944in}{1.851524in}}%
\pgfpathlineto{\pgfqpoint{4.554118in}{1.842346in}}%
\pgfpathlineto{\pgfqpoint{4.546287in}{1.833220in}}%
\pgfpathlineto{\pgfqpoint{4.538451in}{1.824152in}}%
\pgfpathlineto{\pgfqpoint{4.524476in}{1.827229in}}%
\pgfpathlineto{\pgfqpoint{4.510508in}{1.830331in}}%
\pgfpathlineto{\pgfqpoint{4.496548in}{1.833457in}}%
\pgfpathlineto{\pgfqpoint{4.482595in}{1.836609in}}%
\pgfpathlineto{\pgfqpoint{4.490442in}{1.845482in}}%
\pgfpathlineto{\pgfqpoint{4.498284in}{1.854416in}}%
\pgfpathlineto{\pgfqpoint{4.506120in}{1.863406in}}%
\pgfpathlineto{\pgfqpoint{4.513952in}{1.872447in}}%
\pgfpathclose%
\pgfusepath{fill}%
\end{pgfscope}%
\begin{pgfscope}%
\pgfpathrectangle{\pgfqpoint{1.150000in}{0.150000in}}{\pgfqpoint{5.700000in}{5.700000in}}%
\pgfusepath{clip}%
\pgfsetbuttcap%
\pgfsetroundjoin%
\definecolor{currentfill}{rgb}{0.239346,0.300855,0.540844}%
\pgfsetfillcolor{currentfill}%
\pgfsetfillopacity{0.700000}%
\pgfsetlinewidth{0.000000pt}%
\definecolor{currentstroke}{rgb}{0.000000,0.000000,0.000000}%
\pgfsetstrokecolor{currentstroke}%
\pgfsetdash{}{0pt}%
\pgfpathmoveto{\pgfqpoint{2.614248in}{2.422845in}}%
\pgfpathlineto{\pgfqpoint{2.627888in}{2.413611in}}%
\pgfpathlineto{\pgfqpoint{2.641529in}{2.404419in}}%
\pgfpathlineto{\pgfqpoint{2.655173in}{2.395269in}}%
\pgfpathlineto{\pgfqpoint{2.668820in}{2.386161in}}%
\pgfpathlineto{\pgfqpoint{2.659877in}{2.394099in}}%
\pgfpathlineto{\pgfqpoint{2.650909in}{2.402485in}}%
\pgfpathlineto{\pgfqpoint{2.641918in}{2.411327in}}%
\pgfpathlineto{\pgfqpoint{2.632901in}{2.420637in}}%
\pgfpathlineto{\pgfqpoint{2.619213in}{2.430094in}}%
\pgfpathlineto{\pgfqpoint{2.605526in}{2.439592in}}%
\pgfpathlineto{\pgfqpoint{2.591842in}{2.449132in}}%
\pgfpathlineto{\pgfqpoint{2.578160in}{2.458715in}}%
\pgfpathlineto{\pgfqpoint{2.587220in}{2.449051in}}%
\pgfpathlineto{\pgfqpoint{2.596255in}{2.439858in}}%
\pgfpathlineto{\pgfqpoint{2.605264in}{2.431126in}}%
\pgfpathlineto{\pgfqpoint{2.614248in}{2.422845in}}%
\pgfpathclose%
\pgfusepath{fill}%
\end{pgfscope}%
\begin{pgfscope}%
\pgfpathrectangle{\pgfqpoint{1.150000in}{0.150000in}}{\pgfqpoint{5.700000in}{5.700000in}}%
\pgfusepath{clip}%
\pgfsetbuttcap%
\pgfsetroundjoin%
\definecolor{currentfill}{rgb}{0.282656,0.100196,0.422160}%
\pgfsetfillcolor{currentfill}%
\pgfsetfillopacity{0.700000}%
\pgfsetlinewidth{0.000000pt}%
\definecolor{currentstroke}{rgb}{0.000000,0.000000,0.000000}%
\pgfsetstrokecolor{currentstroke}%
\pgfsetdash{}{0pt}%
\pgfpathmoveto{\pgfqpoint{3.373445in}{1.999990in}}%
\pgfpathlineto{\pgfqpoint{3.387149in}{1.993407in}}%
\pgfpathlineto{\pgfqpoint{3.400857in}{1.986854in}}%
\pgfpathlineto{\pgfqpoint{3.414570in}{1.980332in}}%
\pgfpathlineto{\pgfqpoint{3.428287in}{1.973839in}}%
\pgfpathlineto{\pgfqpoint{3.419953in}{1.973583in}}%
\pgfpathlineto{\pgfqpoint{3.411607in}{1.973624in}}%
\pgfpathlineto{\pgfqpoint{3.403248in}{1.973972in}}%
\pgfpathlineto{\pgfqpoint{3.394876in}{1.974634in}}%
\pgfpathlineto{\pgfqpoint{3.381131in}{1.981425in}}%
\pgfpathlineto{\pgfqpoint{3.367391in}{1.988246in}}%
\pgfpathlineto{\pgfqpoint{3.353656in}{1.995096in}}%
\pgfpathlineto{\pgfqpoint{3.339925in}{2.001978in}}%
\pgfpathlineto{\pgfqpoint{3.348325in}{2.001012in}}%
\pgfpathlineto{\pgfqpoint{3.356712in}{2.000364in}}%
\pgfpathlineto{\pgfqpoint{3.365085in}{2.000026in}}%
\pgfpathlineto{\pgfqpoint{3.373445in}{1.999990in}}%
\pgfpathclose%
\pgfusepath{fill}%
\end{pgfscope}%
\begin{pgfscope}%
\pgfpathrectangle{\pgfqpoint{1.150000in}{0.150000in}}{\pgfqpoint{5.700000in}{5.700000in}}%
\pgfusepath{clip}%
\pgfsetbuttcap%
\pgfsetroundjoin%
\definecolor{currentfill}{rgb}{0.268510,0.009605,0.335427}%
\pgfsetfillcolor{currentfill}%
\pgfsetfillopacity{0.700000}%
\pgfsetlinewidth{0.000000pt}%
\definecolor{currentstroke}{rgb}{0.000000,0.000000,0.000000}%
\pgfsetstrokecolor{currentstroke}%
\pgfsetdash{}{0pt}%
\pgfpathmoveto{\pgfqpoint{4.284119in}{1.844051in}}%
\pgfpathlineto{\pgfqpoint{4.298001in}{1.840392in}}%
\pgfpathlineto{\pgfqpoint{4.311890in}{1.836760in}}%
\pgfpathlineto{\pgfqpoint{4.325786in}{1.833153in}}%
\pgfpathlineto{\pgfqpoint{4.339689in}{1.829571in}}%
\pgfpathlineto{\pgfqpoint{4.331792in}{1.821500in}}%
\pgfpathlineto{\pgfqpoint{4.323890in}{1.813527in}}%
\pgfpathlineto{\pgfqpoint{4.315983in}{1.805656in}}%
\pgfpathlineto{\pgfqpoint{4.308070in}{1.797894in}}%
\pgfpathlineto{\pgfqpoint{4.294154in}{1.801693in}}%
\pgfpathlineto{\pgfqpoint{4.280245in}{1.805516in}}%
\pgfpathlineto{\pgfqpoint{4.266343in}{1.809365in}}%
\pgfpathlineto{\pgfqpoint{4.252448in}{1.813240in}}%
\pgfpathlineto{\pgfqpoint{4.260374in}{1.820780in}}%
\pgfpathlineto{\pgfqpoint{4.268294in}{1.828432in}}%
\pgfpathlineto{\pgfqpoint{4.276209in}{1.836191in}}%
\pgfpathlineto{\pgfqpoint{4.284119in}{1.844051in}}%
\pgfpathclose%
\pgfusepath{fill}%
\end{pgfscope}%
\begin{pgfscope}%
\pgfpathrectangle{\pgfqpoint{1.150000in}{0.150000in}}{\pgfqpoint{5.700000in}{5.700000in}}%
\pgfusepath{clip}%
\pgfsetbuttcap%
\pgfsetroundjoin%
\definecolor{currentfill}{rgb}{0.282884,0.135920,0.453427}%
\pgfsetfillcolor{currentfill}%
\pgfsetfillopacity{0.700000}%
\pgfsetlinewidth{0.000000pt}%
\definecolor{currentstroke}{rgb}{0.000000,0.000000,0.000000}%
\pgfsetstrokecolor{currentstroke}%
\pgfsetdash{}{0pt}%
\pgfpathmoveto{\pgfqpoint{5.235460in}{2.074511in}}%
\pgfpathlineto{\pgfqpoint{5.249624in}{2.073401in}}%
\pgfpathlineto{\pgfqpoint{5.263796in}{2.072315in}}%
\pgfpathlineto{\pgfqpoint{5.277977in}{2.071254in}}%
\pgfpathlineto{\pgfqpoint{5.292166in}{2.070218in}}%
\pgfpathlineto{\pgfqpoint{5.284582in}{2.060294in}}%
\pgfpathlineto{\pgfqpoint{5.276991in}{2.050297in}}%
\pgfpathlineto{\pgfqpoint{5.269393in}{2.040229in}}%
\pgfpathlineto{\pgfqpoint{5.261789in}{2.030093in}}%
\pgfpathlineto{\pgfqpoint{5.247591in}{2.031226in}}%
\pgfpathlineto{\pgfqpoint{5.233402in}{2.032384in}}%
\pgfpathlineto{\pgfqpoint{5.219221in}{2.033566in}}%
\pgfpathlineto{\pgfqpoint{5.205049in}{2.034773in}}%
\pgfpathlineto{\pgfqpoint{5.212661in}{2.044808in}}%
\pgfpathlineto{\pgfqpoint{5.220268in}{2.054777in}}%
\pgfpathlineto{\pgfqpoint{5.227867in}{2.064679in}}%
\pgfpathlineto{\pgfqpoint{5.235460in}{2.074511in}}%
\pgfpathclose%
\pgfusepath{fill}%
\end{pgfscope}%
\begin{pgfscope}%
\pgfpathrectangle{\pgfqpoint{1.150000in}{0.150000in}}{\pgfqpoint{5.700000in}{5.700000in}}%
\pgfusepath{clip}%
\pgfsetbuttcap%
\pgfsetroundjoin%
\definecolor{currentfill}{rgb}{0.270595,0.214069,0.507052}%
\pgfsetfillcolor{currentfill}%
\pgfsetfillopacity{0.700000}%
\pgfsetlinewidth{0.000000pt}%
\definecolor{currentstroke}{rgb}{0.000000,0.000000,0.000000}%
\pgfsetstrokecolor{currentstroke}%
\pgfsetdash{}{0pt}%
\pgfpathmoveto{\pgfqpoint{2.922455in}{2.223659in}}%
\pgfpathlineto{\pgfqpoint{2.936112in}{2.215540in}}%
\pgfpathlineto{\pgfqpoint{2.949772in}{2.207456in}}%
\pgfpathlineto{\pgfqpoint{2.963435in}{2.199409in}}%
\pgfpathlineto{\pgfqpoint{2.977101in}{2.191397in}}%
\pgfpathlineto{\pgfqpoint{2.968431in}{2.196014in}}%
\pgfpathlineto{\pgfqpoint{2.959742in}{2.201021in}}%
\pgfpathlineto{\pgfqpoint{2.951034in}{2.206429in}}%
\pgfpathlineto{\pgfqpoint{2.942307in}{2.212245in}}%
\pgfpathlineto{\pgfqpoint{2.928605in}{2.220587in}}%
\pgfpathlineto{\pgfqpoint{2.914906in}{2.228964in}}%
\pgfpathlineto{\pgfqpoint{2.901210in}{2.237377in}}%
\pgfpathlineto{\pgfqpoint{2.887517in}{2.245827in}}%
\pgfpathlineto{\pgfqpoint{2.896282in}{2.239674in}}%
\pgfpathlineto{\pgfqpoint{2.905026in}{2.233935in}}%
\pgfpathlineto{\pgfqpoint{2.913750in}{2.228600in}}%
\pgfpathlineto{\pgfqpoint{2.922455in}{2.223659in}}%
\pgfpathclose%
\pgfusepath{fill}%
\end{pgfscope}%
\begin{pgfscope}%
\pgfpathrectangle{\pgfqpoint{1.150000in}{0.150000in}}{\pgfqpoint{5.700000in}{5.700000in}}%
\pgfusepath{clip}%
\pgfsetbuttcap%
\pgfsetroundjoin%
\definecolor{currentfill}{rgb}{0.278791,0.062145,0.386592}%
\pgfsetfillcolor{currentfill}%
\pgfsetfillopacity{0.700000}%
\pgfsetlinewidth{0.000000pt}%
\definecolor{currentstroke}{rgb}{0.000000,0.000000,0.000000}%
\pgfsetstrokecolor{currentstroke}%
\pgfsetdash{}{0pt}%
\pgfpathmoveto{\pgfqpoint{3.571211in}{1.929087in}}%
\pgfpathlineto{\pgfqpoint{3.584947in}{1.923142in}}%
\pgfpathlineto{\pgfqpoint{3.598688in}{1.917225in}}%
\pgfpathlineto{\pgfqpoint{3.612433in}{1.911337in}}%
\pgfpathlineto{\pgfqpoint{3.626184in}{1.905477in}}%
\pgfpathlineto{\pgfqpoint{3.617969in}{1.903268in}}%
\pgfpathlineto{\pgfqpoint{3.609744in}{1.901317in}}%
\pgfpathlineto{\pgfqpoint{3.601508in}{1.899630in}}%
\pgfpathlineto{\pgfqpoint{3.593261in}{1.898215in}}%
\pgfpathlineto{\pgfqpoint{3.579487in}{1.904359in}}%
\pgfpathlineto{\pgfqpoint{3.565717in}{1.910531in}}%
\pgfpathlineto{\pgfqpoint{3.551953in}{1.916731in}}%
\pgfpathlineto{\pgfqpoint{3.538193in}{1.922961in}}%
\pgfpathlineto{\pgfqpoint{3.546464in}{1.924086in}}%
\pgfpathlineto{\pgfqpoint{3.554724in}{1.925487in}}%
\pgfpathlineto{\pgfqpoint{3.562973in}{1.927157in}}%
\pgfpathlineto{\pgfqpoint{3.571211in}{1.929087in}}%
\pgfpathclose%
\pgfusepath{fill}%
\end{pgfscope}%
\begin{pgfscope}%
\pgfpathrectangle{\pgfqpoint{1.150000in}{0.150000in}}{\pgfqpoint{5.700000in}{5.700000in}}%
\pgfusepath{clip}%
\pgfsetbuttcap%
\pgfsetroundjoin%
\definecolor{currentfill}{rgb}{0.282290,0.145912,0.461510}%
\pgfsetfillcolor{currentfill}%
\pgfsetfillopacity{0.700000}%
\pgfsetlinewidth{0.000000pt}%
\definecolor{currentstroke}{rgb}{0.000000,0.000000,0.000000}%
\pgfsetstrokecolor{currentstroke}%
\pgfsetdash{}{0pt}%
\pgfpathmoveto{\pgfqpoint{3.175478in}{2.086974in}}%
\pgfpathlineto{\pgfqpoint{3.189160in}{2.079717in}}%
\pgfpathlineto{\pgfqpoint{3.202845in}{2.072491in}}%
\pgfpathlineto{\pgfqpoint{3.216535in}{2.065298in}}%
\pgfpathlineto{\pgfqpoint{3.230228in}{2.058137in}}%
\pgfpathlineto{\pgfqpoint{3.221755in}{2.060044in}}%
\pgfpathlineto{\pgfqpoint{3.213267in}{2.062292in}}%
\pgfpathlineto{\pgfqpoint{3.204763in}{2.064890in}}%
\pgfpathlineto{\pgfqpoint{3.196244in}{2.067847in}}%
\pgfpathlineto{\pgfqpoint{3.182520in}{2.075321in}}%
\pgfpathlineto{\pgfqpoint{3.168799in}{2.082828in}}%
\pgfpathlineto{\pgfqpoint{3.155082in}{2.090367in}}%
\pgfpathlineto{\pgfqpoint{3.141370in}{2.097938in}}%
\pgfpathlineto{\pgfqpoint{3.149921in}{2.094662in}}%
\pgfpathlineto{\pgfqpoint{3.158456in}{2.091749in}}%
\pgfpathlineto{\pgfqpoint{3.166975in}{2.089189in}}%
\pgfpathlineto{\pgfqpoint{3.175478in}{2.086974in}}%
\pgfpathclose%
\pgfusepath{fill}%
\end{pgfscope}%
\begin{pgfscope}%
\pgfpathrectangle{\pgfqpoint{1.150000in}{0.150000in}}{\pgfqpoint{5.700000in}{5.700000in}}%
\pgfusepath{clip}%
\pgfsetbuttcap%
\pgfsetroundjoin%
\definecolor{currentfill}{rgb}{0.185556,0.418570,0.556753}%
\pgfsetfillcolor{currentfill}%
\pgfsetfillopacity{0.700000}%
\pgfsetlinewidth{0.000000pt}%
\definecolor{currentstroke}{rgb}{0.000000,0.000000,0.000000}%
\pgfsetstrokecolor{currentstroke}%
\pgfsetdash{}{0pt}%
\pgfpathmoveto{\pgfqpoint{2.250286in}{2.702595in}}%
\pgfpathlineto{\pgfqpoint{2.263933in}{2.691857in}}%
\pgfpathlineto{\pgfqpoint{2.277581in}{2.681173in}}%
\pgfpathlineto{\pgfqpoint{2.291231in}{2.670543in}}%
\pgfpathlineto{\pgfqpoint{2.304881in}{2.659965in}}%
\pgfpathlineto{\pgfqpoint{2.295562in}{2.671917in}}%
\pgfpathlineto{\pgfqpoint{2.286213in}{2.684380in}}%
\pgfpathlineto{\pgfqpoint{2.276833in}{2.697367in}}%
\pgfpathlineto{\pgfqpoint{2.267420in}{2.710887in}}%
\pgfpathlineto{\pgfqpoint{2.253721in}{2.721834in}}%
\pgfpathlineto{\pgfqpoint{2.240022in}{2.732835in}}%
\pgfpathlineto{\pgfqpoint{2.226323in}{2.743890in}}%
\pgfpathlineto{\pgfqpoint{2.212625in}{2.754998in}}%
\pgfpathlineto{\pgfqpoint{2.222089in}{2.741101in}}%
\pgfpathlineto{\pgfqpoint{2.231520in}{2.727742in}}%
\pgfpathlineto{\pgfqpoint{2.240919in}{2.714910in}}%
\pgfpathlineto{\pgfqpoint{2.250286in}{2.702595in}}%
\pgfpathclose%
\pgfusepath{fill}%
\end{pgfscope}%
\begin{pgfscope}%
\pgfpathrectangle{\pgfqpoint{1.150000in}{0.150000in}}{\pgfqpoint{5.700000in}{5.700000in}}%
\pgfusepath{clip}%
\pgfsetbuttcap%
\pgfsetroundjoin%
\definecolor{currentfill}{rgb}{0.277018,0.050344,0.375715}%
\pgfsetfillcolor{currentfill}%
\pgfsetfillopacity{0.700000}%
\pgfsetlinewidth{0.000000pt}%
\definecolor{currentstroke}{rgb}{0.000000,0.000000,0.000000}%
\pgfsetstrokecolor{currentstroke}%
\pgfsetdash{}{0pt}%
\pgfpathmoveto{\pgfqpoint{4.743972in}{1.916323in}}%
\pgfpathlineto{\pgfqpoint{4.757984in}{1.914001in}}%
\pgfpathlineto{\pgfqpoint{4.772003in}{1.911704in}}%
\pgfpathlineto{\pgfqpoint{4.786031in}{1.909432in}}%
\pgfpathlineto{\pgfqpoint{4.800066in}{1.907184in}}%
\pgfpathlineto{\pgfqpoint{4.792314in}{1.897247in}}%
\pgfpathlineto{\pgfqpoint{4.784557in}{1.887315in}}%
\pgfpathlineto{\pgfqpoint{4.776794in}{1.877392in}}%
\pgfpathlineto{\pgfqpoint{4.769027in}{1.867482in}}%
\pgfpathlineto{\pgfqpoint{4.754983in}{1.869893in}}%
\pgfpathlineto{\pgfqpoint{4.740947in}{1.872329in}}%
\pgfpathlineto{\pgfqpoint{4.726918in}{1.874790in}}%
\pgfpathlineto{\pgfqpoint{4.712897in}{1.877276in}}%
\pgfpathlineto{\pgfqpoint{4.720673in}{1.887017in}}%
\pgfpathlineto{\pgfqpoint{4.728445in}{1.896775in}}%
\pgfpathlineto{\pgfqpoint{4.736211in}{1.906544in}}%
\pgfpathlineto{\pgfqpoint{4.743972in}{1.916323in}}%
\pgfpathclose%
\pgfusepath{fill}%
\end{pgfscope}%
\begin{pgfscope}%
\pgfpathrectangle{\pgfqpoint{1.150000in}{0.150000in}}{\pgfqpoint{5.700000in}{5.700000in}}%
\pgfusepath{clip}%
\pgfsetbuttcap%
\pgfsetroundjoin%
\definecolor{currentfill}{rgb}{0.283229,0.120777,0.440584}%
\pgfsetfillcolor{currentfill}%
\pgfsetfillopacity{0.700000}%
\pgfsetlinewidth{0.000000pt}%
\definecolor{currentstroke}{rgb}{0.000000,0.000000,0.000000}%
\pgfsetstrokecolor{currentstroke}%
\pgfsetdash{}{0pt}%
\pgfpathmoveto{\pgfqpoint{5.148445in}{2.039847in}}%
\pgfpathlineto{\pgfqpoint{5.162583in}{2.038542in}}%
\pgfpathlineto{\pgfqpoint{5.176730in}{2.037261in}}%
\pgfpathlineto{\pgfqpoint{5.190885in}{2.036005in}}%
\pgfpathlineto{\pgfqpoint{5.205049in}{2.034773in}}%
\pgfpathlineto{\pgfqpoint{5.197430in}{2.024675in}}%
\pgfpathlineto{\pgfqpoint{5.189804in}{2.014517in}}%
\pgfpathlineto{\pgfqpoint{5.182173in}{2.004300in}}%
\pgfpathlineto{\pgfqpoint{5.174535in}{1.994027in}}%
\pgfpathlineto{\pgfqpoint{5.160363in}{1.995369in}}%
\pgfpathlineto{\pgfqpoint{5.146200in}{1.996736in}}%
\pgfpathlineto{\pgfqpoint{5.132045in}{1.998127in}}%
\pgfpathlineto{\pgfqpoint{5.117898in}{1.999543in}}%
\pgfpathlineto{\pgfqpoint{5.125544in}{2.009701in}}%
\pgfpathlineto{\pgfqpoint{5.133184in}{2.019806in}}%
\pgfpathlineto{\pgfqpoint{5.140817in}{2.029855in}}%
\pgfpathlineto{\pgfqpoint{5.148445in}{2.039847in}}%
\pgfpathclose%
\pgfusepath{fill}%
\end{pgfscope}%
\begin{pgfscope}%
\pgfpathrectangle{\pgfqpoint{1.150000in}{0.150000in}}{\pgfqpoint{5.700000in}{5.700000in}}%
\pgfusepath{clip}%
\pgfsetbuttcap%
\pgfsetroundjoin%
\definecolor{currentfill}{rgb}{0.274128,0.199721,0.498911}%
\pgfsetfillcolor{currentfill}%
\pgfsetfillopacity{0.700000}%
\pgfsetlinewidth{0.000000pt}%
\definecolor{currentstroke}{rgb}{0.000000,0.000000,0.000000}%
\pgfsetstrokecolor{currentstroke}%
\pgfsetdash{}{0pt}%
\pgfpathmoveto{\pgfqpoint{5.640161in}{2.208590in}}%
\pgfpathlineto{\pgfqpoint{5.654463in}{2.208223in}}%
\pgfpathlineto{\pgfqpoint{5.668775in}{2.207881in}}%
\pgfpathlineto{\pgfqpoint{5.683096in}{2.207563in}}%
\pgfpathlineto{\pgfqpoint{5.697426in}{2.207270in}}%
\pgfpathlineto{\pgfqpoint{5.690013in}{2.198526in}}%
\pgfpathlineto{\pgfqpoint{5.682590in}{2.189673in}}%
\pgfpathlineto{\pgfqpoint{5.675159in}{2.180710in}}%
\pgfpathlineto{\pgfqpoint{5.667720in}{2.171639in}}%
\pgfpathlineto{\pgfqpoint{5.653379in}{2.171974in}}%
\pgfpathlineto{\pgfqpoint{5.639047in}{2.172333in}}%
\pgfpathlineto{\pgfqpoint{5.624725in}{2.172717in}}%
\pgfpathlineto{\pgfqpoint{5.610412in}{2.173126in}}%
\pgfpathlineto{\pgfqpoint{5.617861in}{2.182150in}}%
\pgfpathlineto{\pgfqpoint{5.625303in}{2.191069in}}%
\pgfpathlineto{\pgfqpoint{5.632736in}{2.199883in}}%
\pgfpathlineto{\pgfqpoint{5.640161in}{2.208590in}}%
\pgfpathclose%
\pgfusepath{fill}%
\end{pgfscope}%
\begin{pgfscope}%
\pgfpathrectangle{\pgfqpoint{1.150000in}{0.150000in}}{\pgfqpoint{5.700000in}{5.700000in}}%
\pgfusepath{clip}%
\pgfsetbuttcap%
\pgfsetroundjoin%
\definecolor{currentfill}{rgb}{0.269944,0.014625,0.341379}%
\pgfsetfillcolor{currentfill}%
\pgfsetfillopacity{0.700000}%
\pgfsetlinewidth{0.000000pt}%
\definecolor{currentstroke}{rgb}{0.000000,0.000000,0.000000}%
\pgfsetstrokecolor{currentstroke}%
\pgfsetdash{}{0pt}%
\pgfpathmoveto{\pgfqpoint{4.426852in}{1.849467in}}%
\pgfpathlineto{\pgfqpoint{4.440777in}{1.846215in}}%
\pgfpathlineto{\pgfqpoint{4.454710in}{1.842988in}}%
\pgfpathlineto{\pgfqpoint{4.468649in}{1.839786in}}%
\pgfpathlineto{\pgfqpoint{4.482595in}{1.836609in}}%
\pgfpathlineto{\pgfqpoint{4.474743in}{1.827802in}}%
\pgfpathlineto{\pgfqpoint{4.466886in}{1.819066in}}%
\pgfpathlineto{\pgfqpoint{4.459023in}{1.810405in}}%
\pgfpathlineto{\pgfqpoint{4.451156in}{1.801825in}}%
\pgfpathlineto{\pgfqpoint{4.437198in}{1.805205in}}%
\pgfpathlineto{\pgfqpoint{4.423248in}{1.808611in}}%
\pgfpathlineto{\pgfqpoint{4.409304in}{1.812041in}}%
\pgfpathlineto{\pgfqpoint{4.395368in}{1.815497in}}%
\pgfpathlineto{\pgfqpoint{4.403247in}{1.823868in}}%
\pgfpathlineto{\pgfqpoint{4.411120in}{1.832324in}}%
\pgfpathlineto{\pgfqpoint{4.418989in}{1.840859in}}%
\pgfpathlineto{\pgfqpoint{4.426852in}{1.849467in}}%
\pgfpathclose%
\pgfusepath{fill}%
\end{pgfscope}%
\begin{pgfscope}%
\pgfpathrectangle{\pgfqpoint{1.150000in}{0.150000in}}{\pgfqpoint{5.700000in}{5.700000in}}%
\pgfusepath{clip}%
\pgfsetbuttcap%
\pgfsetroundjoin%
\definecolor{currentfill}{rgb}{0.271305,0.019942,0.347269}%
\pgfsetfillcolor{currentfill}%
\pgfsetfillopacity{0.700000}%
\pgfsetlinewidth{0.000000pt}%
\definecolor{currentstroke}{rgb}{0.000000,0.000000,0.000000}%
\pgfsetstrokecolor{currentstroke}%
\pgfsetdash{}{0pt}%
\pgfpathmoveto{\pgfqpoint{3.911634in}{1.849987in}}%
\pgfpathlineto{\pgfqpoint{3.925437in}{1.845120in}}%
\pgfpathlineto{\pgfqpoint{3.939246in}{1.840279in}}%
\pgfpathlineto{\pgfqpoint{3.953060in}{1.835465in}}%
\pgfpathlineto{\pgfqpoint{3.966881in}{1.830678in}}%
\pgfpathlineto{\pgfqpoint{3.958835in}{1.825409in}}%
\pgfpathlineto{\pgfqpoint{3.950782in}{1.820325in}}%
\pgfpathlineto{\pgfqpoint{3.942722in}{1.815432in}}%
\pgfpathlineto{\pgfqpoint{3.934654in}{1.810738in}}%
\pgfpathlineto{\pgfqpoint{3.920816in}{1.815781in}}%
\pgfpathlineto{\pgfqpoint{3.906983in}{1.820852in}}%
\pgfpathlineto{\pgfqpoint{3.893156in}{1.825949in}}%
\pgfpathlineto{\pgfqpoint{3.879335in}{1.831072in}}%
\pgfpathlineto{\pgfqpoint{3.887421in}{1.835506in}}%
\pgfpathlineto{\pgfqpoint{3.895500in}{1.840140in}}%
\pgfpathlineto{\pgfqpoint{3.903571in}{1.844970in}}%
\pgfpathlineto{\pgfqpoint{3.911634in}{1.849987in}}%
\pgfpathclose%
\pgfusepath{fill}%
\end{pgfscope}%
\begin{pgfscope}%
\pgfpathrectangle{\pgfqpoint{1.150000in}{0.150000in}}{\pgfqpoint{5.700000in}{5.700000in}}%
\pgfusepath{clip}%
\pgfsetbuttcap%
\pgfsetroundjoin%
\definecolor{currentfill}{rgb}{0.268510,0.009605,0.335427}%
\pgfsetfillcolor{currentfill}%
\pgfsetfillopacity{0.700000}%
\pgfsetlinewidth{0.000000pt}%
\definecolor{currentstroke}{rgb}{0.000000,0.000000,0.000000}%
\pgfsetstrokecolor{currentstroke}%
\pgfsetdash{}{0pt}%
\pgfpathmoveto{\pgfqpoint{4.054266in}{1.835559in}}%
\pgfpathlineto{\pgfqpoint{4.068100in}{1.831146in}}%
\pgfpathlineto{\pgfqpoint{4.081940in}{1.826759in}}%
\pgfpathlineto{\pgfqpoint{4.095786in}{1.822398in}}%
\pgfpathlineto{\pgfqpoint{4.109639in}{1.818062in}}%
\pgfpathlineto{\pgfqpoint{4.101653in}{1.811642in}}%
\pgfpathlineto{\pgfqpoint{4.093662in}{1.805374in}}%
\pgfpathlineto{\pgfqpoint{4.085663in}{1.799265in}}%
\pgfpathlineto{\pgfqpoint{4.077658in}{1.793322in}}%
\pgfpathlineto{\pgfqpoint{4.063790in}{1.797900in}}%
\pgfpathlineto{\pgfqpoint{4.049928in}{1.802504in}}%
\pgfpathlineto{\pgfqpoint{4.036072in}{1.807135in}}%
\pgfpathlineto{\pgfqpoint{4.022222in}{1.811791in}}%
\pgfpathlineto{\pgfqpoint{4.030243in}{1.817486in}}%
\pgfpathlineto{\pgfqpoint{4.038257in}{1.823350in}}%
\pgfpathlineto{\pgfqpoint{4.046265in}{1.829377in}}%
\pgfpathlineto{\pgfqpoint{4.054266in}{1.835559in}}%
\pgfpathclose%
\pgfusepath{fill}%
\end{pgfscope}%
\begin{pgfscope}%
\pgfpathrectangle{\pgfqpoint{1.150000in}{0.150000in}}{\pgfqpoint{5.700000in}{5.700000in}}%
\pgfusepath{clip}%
\pgfsetbuttcap%
\pgfsetroundjoin%
\definecolor{currentfill}{rgb}{0.282910,0.105393,0.426902}%
\pgfsetfillcolor{currentfill}%
\pgfsetfillopacity{0.700000}%
\pgfsetlinewidth{0.000000pt}%
\definecolor{currentstroke}{rgb}{0.000000,0.000000,0.000000}%
\pgfsetstrokecolor{currentstroke}%
\pgfsetdash{}{0pt}%
\pgfpathmoveto{\pgfqpoint{5.061395in}{2.005454in}}%
\pgfpathlineto{\pgfqpoint{5.075508in}{2.003939in}}%
\pgfpathlineto{\pgfqpoint{5.089630in}{2.002449in}}%
\pgfpathlineto{\pgfqpoint{5.103760in}{2.000984in}}%
\pgfpathlineto{\pgfqpoint{5.117898in}{1.999543in}}%
\pgfpathlineto{\pgfqpoint{5.110246in}{1.989336in}}%
\pgfpathlineto{\pgfqpoint{5.102588in}{1.979081in}}%
\pgfpathlineto{\pgfqpoint{5.094924in}{1.968782in}}%
\pgfpathlineto{\pgfqpoint{5.087255in}{1.958441in}}%
\pgfpathlineto{\pgfqpoint{5.073108in}{1.960006in}}%
\pgfpathlineto{\pgfqpoint{5.058970in}{1.961595in}}%
\pgfpathlineto{\pgfqpoint{5.044840in}{1.963209in}}%
\pgfpathlineto{\pgfqpoint{5.030718in}{1.964847in}}%
\pgfpathlineto{\pgfqpoint{5.038396in}{1.975059in}}%
\pgfpathlineto{\pgfqpoint{5.046068in}{1.985233in}}%
\pgfpathlineto{\pgfqpoint{5.053734in}{1.995365in}}%
\pgfpathlineto{\pgfqpoint{5.061395in}{2.005454in}}%
\pgfpathclose%
\pgfusepath{fill}%
\end{pgfscope}%
\begin{pgfscope}%
\pgfpathrectangle{\pgfqpoint{1.150000in}{0.150000in}}{\pgfqpoint{5.700000in}{5.700000in}}%
\pgfusepath{clip}%
\pgfsetbuttcap%
\pgfsetroundjoin%
\definecolor{currentfill}{rgb}{0.243113,0.292092,0.538516}%
\pgfsetfillcolor{currentfill}%
\pgfsetfillopacity{0.700000}%
\pgfsetlinewidth{0.000000pt}%
\definecolor{currentstroke}{rgb}{0.000000,0.000000,0.000000}%
\pgfsetstrokecolor{currentstroke}%
\pgfsetdash{}{0pt}%
\pgfpathmoveto{\pgfqpoint{2.668820in}{2.386161in}}%
\pgfpathlineto{\pgfqpoint{2.682469in}{2.377094in}}%
\pgfpathlineto{\pgfqpoint{2.696120in}{2.368068in}}%
\pgfpathlineto{\pgfqpoint{2.709774in}{2.359083in}}%
\pgfpathlineto{\pgfqpoint{2.723430in}{2.350138in}}%
\pgfpathlineto{\pgfqpoint{2.714527in}{2.357734in}}%
\pgfpathlineto{\pgfqpoint{2.705601in}{2.365774in}}%
\pgfpathlineto{\pgfqpoint{2.696651in}{2.374267in}}%
\pgfpathlineto{\pgfqpoint{2.687677in}{2.383224in}}%
\pgfpathlineto{\pgfqpoint{2.673980in}{2.392516in}}%
\pgfpathlineto{\pgfqpoint{2.660284in}{2.401849in}}%
\pgfpathlineto{\pgfqpoint{2.646592in}{2.411223in}}%
\pgfpathlineto{\pgfqpoint{2.632901in}{2.420637in}}%
\pgfpathlineto{\pgfqpoint{2.641918in}{2.411327in}}%
\pgfpathlineto{\pgfqpoint{2.650909in}{2.402485in}}%
\pgfpathlineto{\pgfqpoint{2.659877in}{2.394099in}}%
\pgfpathlineto{\pgfqpoint{2.668820in}{2.386161in}}%
\pgfpathclose%
\pgfusepath{fill}%
\end{pgfscope}%
\begin{pgfscope}%
\pgfpathrectangle{\pgfqpoint{1.150000in}{0.150000in}}{\pgfqpoint{5.700000in}{5.700000in}}%
\pgfusepath{clip}%
\pgfsetbuttcap%
\pgfsetroundjoin%
\definecolor{currentfill}{rgb}{0.276194,0.190074,0.493001}%
\pgfsetfillcolor{currentfill}%
\pgfsetfillopacity{0.700000}%
\pgfsetlinewidth{0.000000pt}%
\definecolor{currentstroke}{rgb}{0.000000,0.000000,0.000000}%
\pgfsetstrokecolor{currentstroke}%
\pgfsetdash{}{0pt}%
\pgfpathmoveto{\pgfqpoint{5.553251in}{2.175008in}}%
\pgfpathlineto{\pgfqpoint{5.567527in}{2.174500in}}%
\pgfpathlineto{\pgfqpoint{5.581813in}{2.174017in}}%
\pgfpathlineto{\pgfqpoint{5.596108in}{2.173559in}}%
\pgfpathlineto{\pgfqpoint{5.610412in}{2.173126in}}%
\pgfpathlineto{\pgfqpoint{5.602954in}{2.163997in}}%
\pgfpathlineto{\pgfqpoint{5.595488in}{2.154765in}}%
\pgfpathlineto{\pgfqpoint{5.588014in}{2.145430in}}%
\pgfpathlineto{\pgfqpoint{5.580532in}{2.135993in}}%
\pgfpathlineto{\pgfqpoint{5.566218in}{2.136482in}}%
\pgfpathlineto{\pgfqpoint{5.551913in}{2.136996in}}%
\pgfpathlineto{\pgfqpoint{5.537617in}{2.137534in}}%
\pgfpathlineto{\pgfqpoint{5.523331in}{2.138098in}}%
\pgfpathlineto{\pgfqpoint{5.530822in}{2.147473in}}%
\pgfpathlineto{\pgfqpoint{5.538306in}{2.156751in}}%
\pgfpathlineto{\pgfqpoint{5.545782in}{2.165929in}}%
\pgfpathlineto{\pgfqpoint{5.553251in}{2.175008in}}%
\pgfpathclose%
\pgfusepath{fill}%
\end{pgfscope}%
\begin{pgfscope}%
\pgfpathrectangle{\pgfqpoint{1.150000in}{0.150000in}}{\pgfqpoint{5.700000in}{5.700000in}}%
\pgfusepath{clip}%
\pgfsetbuttcap%
\pgfsetroundjoin%
\definecolor{currentfill}{rgb}{0.274952,0.037752,0.364543}%
\pgfsetfillcolor{currentfill}%
\pgfsetfillopacity{0.700000}%
\pgfsetlinewidth{0.000000pt}%
\definecolor{currentstroke}{rgb}{0.000000,0.000000,0.000000}%
\pgfsetstrokecolor{currentstroke}%
\pgfsetdash{}{0pt}%
\pgfpathmoveto{\pgfqpoint{4.656888in}{1.887466in}}%
\pgfpathlineto{\pgfqpoint{4.670879in}{1.884882in}}%
\pgfpathlineto{\pgfqpoint{4.684878in}{1.882321in}}%
\pgfpathlineto{\pgfqpoint{4.698884in}{1.879786in}}%
\pgfpathlineto{\pgfqpoint{4.712897in}{1.877276in}}%
\pgfpathlineto{\pgfqpoint{4.705116in}{1.867555in}}%
\pgfpathlineto{\pgfqpoint{4.697329in}{1.857858in}}%
\pgfpathlineto{\pgfqpoint{4.689538in}{1.848190in}}%
\pgfpathlineto{\pgfqpoint{4.681742in}{1.838555in}}%
\pgfpathlineto{\pgfqpoint{4.667719in}{1.841242in}}%
\pgfpathlineto{\pgfqpoint{4.653703in}{1.843955in}}%
\pgfpathlineto{\pgfqpoint{4.639695in}{1.846692in}}%
\pgfpathlineto{\pgfqpoint{4.625694in}{1.849454in}}%
\pgfpathlineto{\pgfqpoint{4.633500in}{1.858907in}}%
\pgfpathlineto{\pgfqpoint{4.641301in}{1.868396in}}%
\pgfpathlineto{\pgfqpoint{4.649097in}{1.877918in}}%
\pgfpathlineto{\pgfqpoint{4.656888in}{1.887466in}}%
\pgfpathclose%
\pgfusepath{fill}%
\end{pgfscope}%
\begin{pgfscope}%
\pgfpathrectangle{\pgfqpoint{1.150000in}{0.150000in}}{\pgfqpoint{5.700000in}{5.700000in}}%
\pgfusepath{clip}%
\pgfsetbuttcap%
\pgfsetroundjoin%
\definecolor{currentfill}{rgb}{0.273809,0.031497,0.358853}%
\pgfsetfillcolor{currentfill}%
\pgfsetfillopacity{0.700000}%
\pgfsetlinewidth{0.000000pt}%
\definecolor{currentstroke}{rgb}{0.000000,0.000000,0.000000}%
\pgfsetstrokecolor{currentstroke}%
\pgfsetdash{}{0pt}%
\pgfpathmoveto{\pgfqpoint{3.768966in}{1.873029in}}%
\pgfpathlineto{\pgfqpoint{3.782743in}{1.867690in}}%
\pgfpathlineto{\pgfqpoint{3.796525in}{1.862377in}}%
\pgfpathlineto{\pgfqpoint{3.810313in}{1.857092in}}%
\pgfpathlineto{\pgfqpoint{3.824106in}{1.851834in}}%
\pgfpathlineto{\pgfqpoint{3.815992in}{1.847874in}}%
\pgfpathlineto{\pgfqpoint{3.807869in}{1.844132in}}%
\pgfpathlineto{\pgfqpoint{3.799738in}{1.840616in}}%
\pgfpathlineto{\pgfqpoint{3.791598in}{1.837332in}}%
\pgfpathlineto{\pgfqpoint{3.777784in}{1.842859in}}%
\pgfpathlineto{\pgfqpoint{3.763976in}{1.848414in}}%
\pgfpathlineto{\pgfqpoint{3.750173in}{1.853997in}}%
\pgfpathlineto{\pgfqpoint{3.736376in}{1.859606in}}%
\pgfpathlineto{\pgfqpoint{3.744537in}{1.862615in}}%
\pgfpathlineto{\pgfqpoint{3.752689in}{1.865860in}}%
\pgfpathlineto{\pgfqpoint{3.760832in}{1.869334in}}%
\pgfpathlineto{\pgfqpoint{3.768966in}{1.873029in}}%
\pgfpathclose%
\pgfusepath{fill}%
\end{pgfscope}%
\begin{pgfscope}%
\pgfpathrectangle{\pgfqpoint{1.150000in}{0.150000in}}{\pgfqpoint{5.700000in}{5.700000in}}%
\pgfusepath{clip}%
\pgfsetbuttcap%
\pgfsetroundjoin%
\definecolor{currentfill}{rgb}{0.192357,0.403199,0.555836}%
\pgfsetfillcolor{currentfill}%
\pgfsetfillopacity{0.700000}%
\pgfsetlinewidth{0.000000pt}%
\definecolor{currentstroke}{rgb}{0.000000,0.000000,0.000000}%
\pgfsetstrokecolor{currentstroke}%
\pgfsetdash{}{0pt}%
\pgfpathmoveto{\pgfqpoint{2.304881in}{2.659965in}}%
\pgfpathlineto{\pgfqpoint{2.318531in}{2.649440in}}%
\pgfpathlineto{\pgfqpoint{2.332183in}{2.638967in}}%
\pgfpathlineto{\pgfqpoint{2.345837in}{2.628544in}}%
\pgfpathlineto{\pgfqpoint{2.359491in}{2.618172in}}%
\pgfpathlineto{\pgfqpoint{2.350220in}{2.629761in}}%
\pgfpathlineto{\pgfqpoint{2.340920in}{2.641857in}}%
\pgfpathlineto{\pgfqpoint{2.331589in}{2.654472in}}%
\pgfpathlineto{\pgfqpoint{2.322227in}{2.667616in}}%
\pgfpathlineto{\pgfqpoint{2.308524in}{2.678357in}}%
\pgfpathlineto{\pgfqpoint{2.294822in}{2.689148in}}%
\pgfpathlineto{\pgfqpoint{2.281121in}{2.699991in}}%
\pgfpathlineto{\pgfqpoint{2.267420in}{2.710887in}}%
\pgfpathlineto{\pgfqpoint{2.276833in}{2.697367in}}%
\pgfpathlineto{\pgfqpoint{2.286213in}{2.684380in}}%
\pgfpathlineto{\pgfqpoint{2.295562in}{2.671917in}}%
\pgfpathlineto{\pgfqpoint{2.304881in}{2.659965in}}%
\pgfpathclose%
\pgfusepath{fill}%
\end{pgfscope}%
\begin{pgfscope}%
\pgfpathrectangle{\pgfqpoint{1.150000in}{0.150000in}}{\pgfqpoint{5.700000in}{5.700000in}}%
\pgfusepath{clip}%
\pgfsetbuttcap%
\pgfsetroundjoin%
\definecolor{currentfill}{rgb}{0.268510,0.009605,0.335427}%
\pgfsetfillcolor{currentfill}%
\pgfsetfillopacity{0.700000}%
\pgfsetlinewidth{0.000000pt}%
\definecolor{currentstroke}{rgb}{0.000000,0.000000,0.000000}%
\pgfsetstrokecolor{currentstroke}%
\pgfsetdash{}{0pt}%
\pgfpathmoveto{\pgfqpoint{4.196930in}{1.828993in}}%
\pgfpathlineto{\pgfqpoint{4.210800in}{1.825016in}}%
\pgfpathlineto{\pgfqpoint{4.224676in}{1.821065in}}%
\pgfpathlineto{\pgfqpoint{4.238559in}{1.817140in}}%
\pgfpathlineto{\pgfqpoint{4.252448in}{1.813240in}}%
\pgfpathlineto{\pgfqpoint{4.244516in}{1.805817in}}%
\pgfpathlineto{\pgfqpoint{4.236578in}{1.798516in}}%
\pgfpathlineto{\pgfqpoint{4.228634in}{1.791345in}}%
\pgfpathlineto{\pgfqpoint{4.220685in}{1.784307in}}%
\pgfpathlineto{\pgfqpoint{4.206782in}{1.788437in}}%
\pgfpathlineto{\pgfqpoint{4.192885in}{1.792592in}}%
\pgfpathlineto{\pgfqpoint{4.178995in}{1.796773in}}%
\pgfpathlineto{\pgfqpoint{4.165111in}{1.800980in}}%
\pgfpathlineto{\pgfqpoint{4.173075in}{1.807782in}}%
\pgfpathlineto{\pgfqpoint{4.181033in}{1.814722in}}%
\pgfpathlineto{\pgfqpoint{4.188985in}{1.821795in}}%
\pgfpathlineto{\pgfqpoint{4.196930in}{1.828993in}}%
\pgfpathclose%
\pgfusepath{fill}%
\end{pgfscope}%
\begin{pgfscope}%
\pgfpathrectangle{\pgfqpoint{1.150000in}{0.150000in}}{\pgfqpoint{5.700000in}{5.700000in}}%
\pgfusepath{clip}%
\pgfsetbuttcap%
\pgfsetroundjoin%
\definecolor{currentfill}{rgb}{0.273006,0.204520,0.501721}%
\pgfsetfillcolor{currentfill}%
\pgfsetfillopacity{0.700000}%
\pgfsetlinewidth{0.000000pt}%
\definecolor{currentstroke}{rgb}{0.000000,0.000000,0.000000}%
\pgfsetstrokecolor{currentstroke}%
\pgfsetdash{}{0pt}%
\pgfpathmoveto{\pgfqpoint{2.977101in}{2.191397in}}%
\pgfpathlineto{\pgfqpoint{2.990771in}{2.183420in}}%
\pgfpathlineto{\pgfqpoint{3.004444in}{2.175479in}}%
\pgfpathlineto{\pgfqpoint{3.018120in}{2.167572in}}%
\pgfpathlineto{\pgfqpoint{3.031800in}{2.159699in}}%
\pgfpathlineto{\pgfqpoint{3.023165in}{2.163992in}}%
\pgfpathlineto{\pgfqpoint{3.014511in}{2.168672in}}%
\pgfpathlineto{\pgfqpoint{3.005839in}{2.173748in}}%
\pgfpathlineto{\pgfqpoint{2.997148in}{2.179230in}}%
\pgfpathlineto{\pgfqpoint{2.983433in}{2.187431in}}%
\pgfpathlineto{\pgfqpoint{2.969721in}{2.195668in}}%
\pgfpathlineto{\pgfqpoint{2.956012in}{2.203939in}}%
\pgfpathlineto{\pgfqpoint{2.942307in}{2.212245in}}%
\pgfpathlineto{\pgfqpoint{2.951034in}{2.206429in}}%
\pgfpathlineto{\pgfqpoint{2.959742in}{2.201021in}}%
\pgfpathlineto{\pgfqpoint{2.968431in}{2.196014in}}%
\pgfpathlineto{\pgfqpoint{2.977101in}{2.191397in}}%
\pgfpathclose%
\pgfusepath{fill}%
\end{pgfscope}%
\begin{pgfscope}%
\pgfpathrectangle{\pgfqpoint{1.150000in}{0.150000in}}{\pgfqpoint{5.700000in}{5.700000in}}%
\pgfusepath{clip}%
\pgfsetbuttcap%
\pgfsetroundjoin%
\definecolor{currentfill}{rgb}{0.282327,0.094955,0.417331}%
\pgfsetfillcolor{currentfill}%
\pgfsetfillopacity{0.700000}%
\pgfsetlinewidth{0.000000pt}%
\definecolor{currentstroke}{rgb}{0.000000,0.000000,0.000000}%
\pgfsetstrokecolor{currentstroke}%
\pgfsetdash{}{0pt}%
\pgfpathmoveto{\pgfqpoint{3.428287in}{1.973839in}}%
\pgfpathlineto{\pgfqpoint{3.442009in}{1.967377in}}%
\pgfpathlineto{\pgfqpoint{3.455736in}{1.960944in}}%
\pgfpathlineto{\pgfqpoint{3.469467in}{1.954540in}}%
\pgfpathlineto{\pgfqpoint{3.483203in}{1.948166in}}%
\pgfpathlineto{\pgfqpoint{3.474895in}{1.947617in}}%
\pgfpathlineto{\pgfqpoint{3.466575in}{1.947362in}}%
\pgfpathlineto{\pgfqpoint{3.458242in}{1.947409in}}%
\pgfpathlineto{\pgfqpoint{3.449898in}{1.947768in}}%
\pgfpathlineto{\pgfqpoint{3.436135in}{1.954440in}}%
\pgfpathlineto{\pgfqpoint{3.422378in}{1.961142in}}%
\pgfpathlineto{\pgfqpoint{3.408624in}{1.967873in}}%
\pgfpathlineto{\pgfqpoint{3.394876in}{1.974634in}}%
\pgfpathlineto{\pgfqpoint{3.403248in}{1.973972in}}%
\pgfpathlineto{\pgfqpoint{3.411607in}{1.973624in}}%
\pgfpathlineto{\pgfqpoint{3.419953in}{1.973583in}}%
\pgfpathlineto{\pgfqpoint{3.428287in}{1.973839in}}%
\pgfpathclose%
\pgfusepath{fill}%
\end{pgfscope}%
\begin{pgfscope}%
\pgfpathrectangle{\pgfqpoint{1.150000in}{0.150000in}}{\pgfqpoint{5.700000in}{5.700000in}}%
\pgfusepath{clip}%
\pgfsetbuttcap%
\pgfsetroundjoin%
\definecolor{currentfill}{rgb}{0.281924,0.089666,0.412415}%
\pgfsetfillcolor{currentfill}%
\pgfsetfillopacity{0.700000}%
\pgfsetlinewidth{0.000000pt}%
\definecolor{currentstroke}{rgb}{0.000000,0.000000,0.000000}%
\pgfsetstrokecolor{currentstroke}%
\pgfsetdash{}{0pt}%
\pgfpathmoveto{\pgfqpoint{4.974313in}{1.971648in}}%
\pgfpathlineto{\pgfqpoint{4.988402in}{1.969911in}}%
\pgfpathlineto{\pgfqpoint{5.002500in}{1.968198in}}%
\pgfpathlineto{\pgfqpoint{5.016605in}{1.966511in}}%
\pgfpathlineto{\pgfqpoint{5.030718in}{1.964847in}}%
\pgfpathlineto{\pgfqpoint{5.023035in}{1.954600in}}%
\pgfpathlineto{\pgfqpoint{5.015346in}{1.944320in}}%
\pgfpathlineto{\pgfqpoint{5.007652in}{1.934011in}}%
\pgfpathlineto{\pgfqpoint{4.999952in}{1.923675in}}%
\pgfpathlineto{\pgfqpoint{4.985830in}{1.925476in}}%
\pgfpathlineto{\pgfqpoint{4.971716in}{1.927301in}}%
\pgfpathlineto{\pgfqpoint{4.957611in}{1.929151in}}%
\pgfpathlineto{\pgfqpoint{4.943513in}{1.931026in}}%
\pgfpathlineto{\pgfqpoint{4.951222in}{1.941219in}}%
\pgfpathlineto{\pgfqpoint{4.958924in}{1.951389in}}%
\pgfpathlineto{\pgfqpoint{4.966622in}{1.961533in}}%
\pgfpathlineto{\pgfqpoint{4.974313in}{1.971648in}}%
\pgfpathclose%
\pgfusepath{fill}%
\end{pgfscope}%
\begin{pgfscope}%
\pgfpathrectangle{\pgfqpoint{1.150000in}{0.150000in}}{\pgfqpoint{5.700000in}{5.700000in}}%
\pgfusepath{clip}%
\pgfsetbuttcap%
\pgfsetroundjoin%
\definecolor{currentfill}{rgb}{0.279574,0.170599,0.479997}%
\pgfsetfillcolor{currentfill}%
\pgfsetfillopacity{0.700000}%
\pgfsetlinewidth{0.000000pt}%
\definecolor{currentstroke}{rgb}{0.000000,0.000000,0.000000}%
\pgfsetstrokecolor{currentstroke}%
\pgfsetdash{}{0pt}%
\pgfpathmoveto{\pgfqpoint{5.466275in}{2.140596in}}%
\pgfpathlineto{\pgfqpoint{5.480526in}{2.139935in}}%
\pgfpathlineto{\pgfqpoint{5.494785in}{2.139298in}}%
\pgfpathlineto{\pgfqpoint{5.509053in}{2.138685in}}%
\pgfpathlineto{\pgfqpoint{5.523331in}{2.138098in}}%
\pgfpathlineto{\pgfqpoint{5.515831in}{2.128625in}}%
\pgfpathlineto{\pgfqpoint{5.508324in}{2.119056in}}%
\pgfpathlineto{\pgfqpoint{5.500810in}{2.109392in}}%
\pgfpathlineto{\pgfqpoint{5.493288in}{2.099635in}}%
\pgfpathlineto{\pgfqpoint{5.479001in}{2.100292in}}%
\pgfpathlineto{\pgfqpoint{5.464723in}{2.100974in}}%
\pgfpathlineto{\pgfqpoint{5.450454in}{2.101681in}}%
\pgfpathlineto{\pgfqpoint{5.436194in}{2.102412in}}%
\pgfpathlineto{\pgfqpoint{5.443726in}{2.112094in}}%
\pgfpathlineto{\pgfqpoint{5.451250in}{2.121687in}}%
\pgfpathlineto{\pgfqpoint{5.458766in}{2.131188in}}%
\pgfpathlineto{\pgfqpoint{5.466275in}{2.140596in}}%
\pgfpathclose%
\pgfusepath{fill}%
\end{pgfscope}%
\begin{pgfscope}%
\pgfpathrectangle{\pgfqpoint{1.150000in}{0.150000in}}{\pgfqpoint{5.700000in}{5.700000in}}%
\pgfusepath{clip}%
\pgfsetbuttcap%
\pgfsetroundjoin%
\definecolor{currentfill}{rgb}{0.282623,0.140926,0.457517}%
\pgfsetfillcolor{currentfill}%
\pgfsetfillopacity{0.700000}%
\pgfsetlinewidth{0.000000pt}%
\definecolor{currentstroke}{rgb}{0.000000,0.000000,0.000000}%
\pgfsetstrokecolor{currentstroke}%
\pgfsetdash{}{0pt}%
\pgfpathmoveto{\pgfqpoint{3.230228in}{2.058137in}}%
\pgfpathlineto{\pgfqpoint{3.243926in}{2.051008in}}%
\pgfpathlineto{\pgfqpoint{3.257627in}{2.043910in}}%
\pgfpathlineto{\pgfqpoint{3.271333in}{2.036844in}}%
\pgfpathlineto{\pgfqpoint{3.285043in}{2.029809in}}%
\pgfpathlineto{\pgfqpoint{3.276600in}{2.031408in}}%
\pgfpathlineto{\pgfqpoint{3.268142in}{2.033344in}}%
\pgfpathlineto{\pgfqpoint{3.259669in}{2.035627in}}%
\pgfpathlineto{\pgfqpoint{3.251181in}{2.038265in}}%
\pgfpathlineto{\pgfqpoint{3.237441in}{2.045613in}}%
\pgfpathlineto{\pgfqpoint{3.223704in}{2.052993in}}%
\pgfpathlineto{\pgfqpoint{3.209972in}{2.060404in}}%
\pgfpathlineto{\pgfqpoint{3.196244in}{2.067847in}}%
\pgfpathlineto{\pgfqpoint{3.204763in}{2.064890in}}%
\pgfpathlineto{\pgfqpoint{3.213267in}{2.062292in}}%
\pgfpathlineto{\pgfqpoint{3.221755in}{2.060044in}}%
\pgfpathlineto{\pgfqpoint{3.230228in}{2.058137in}}%
\pgfpathclose%
\pgfusepath{fill}%
\end{pgfscope}%
\begin{pgfscope}%
\pgfpathrectangle{\pgfqpoint{1.150000in}{0.150000in}}{\pgfqpoint{5.700000in}{5.700000in}}%
\pgfusepath{clip}%
\pgfsetbuttcap%
\pgfsetroundjoin%
\definecolor{currentfill}{rgb}{0.277941,0.056324,0.381191}%
\pgfsetfillcolor{currentfill}%
\pgfsetfillopacity{0.700000}%
\pgfsetlinewidth{0.000000pt}%
\definecolor{currentstroke}{rgb}{0.000000,0.000000,0.000000}%
\pgfsetstrokecolor{currentstroke}%
\pgfsetdash{}{0pt}%
\pgfpathmoveto{\pgfqpoint{3.626184in}{1.905477in}}%
\pgfpathlineto{\pgfqpoint{3.639940in}{1.899646in}}%
\pgfpathlineto{\pgfqpoint{3.653701in}{1.893842in}}%
\pgfpathlineto{\pgfqpoint{3.667467in}{1.888067in}}%
\pgfpathlineto{\pgfqpoint{3.681238in}{1.882319in}}%
\pgfpathlineto{\pgfqpoint{3.673046in}{1.879832in}}%
\pgfpathlineto{\pgfqpoint{3.664843in}{1.877598in}}%
\pgfpathlineto{\pgfqpoint{3.656631in}{1.875626in}}%
\pgfpathlineto{\pgfqpoint{3.648408in}{1.873922in}}%
\pgfpathlineto{\pgfqpoint{3.634614in}{1.879953in}}%
\pgfpathlineto{\pgfqpoint{3.620824in}{1.886012in}}%
\pgfpathlineto{\pgfqpoint{3.607040in}{1.892100in}}%
\pgfpathlineto{\pgfqpoint{3.593261in}{1.898215in}}%
\pgfpathlineto{\pgfqpoint{3.601508in}{1.899630in}}%
\pgfpathlineto{\pgfqpoint{3.609744in}{1.901317in}}%
\pgfpathlineto{\pgfqpoint{3.617969in}{1.903268in}}%
\pgfpathlineto{\pgfqpoint{3.626184in}{1.905477in}}%
\pgfpathclose%
\pgfusepath{fill}%
\end{pgfscope}%
\begin{pgfscope}%
\pgfpathrectangle{\pgfqpoint{1.150000in}{0.150000in}}{\pgfqpoint{5.700000in}{5.700000in}}%
\pgfusepath{clip}%
\pgfsetbuttcap%
\pgfsetroundjoin%
\definecolor{currentfill}{rgb}{0.268510,0.009605,0.335427}%
\pgfsetfillcolor{currentfill}%
\pgfsetfillopacity{0.700000}%
\pgfsetlinewidth{0.000000pt}%
\definecolor{currentstroke}{rgb}{0.000000,0.000000,0.000000}%
\pgfsetstrokecolor{currentstroke}%
\pgfsetdash{}{0pt}%
\pgfpathmoveto{\pgfqpoint{4.339689in}{1.829571in}}%
\pgfpathlineto{\pgfqpoint{4.353598in}{1.826015in}}%
\pgfpathlineto{\pgfqpoint{4.367515in}{1.822483in}}%
\pgfpathlineto{\pgfqpoint{4.381438in}{1.818977in}}%
\pgfpathlineto{\pgfqpoint{4.395368in}{1.815497in}}%
\pgfpathlineto{\pgfqpoint{4.387483in}{1.807214in}}%
\pgfpathlineto{\pgfqpoint{4.379593in}{1.799026in}}%
\pgfpathlineto{\pgfqpoint{4.371698in}{1.790937in}}%
\pgfpathlineto{\pgfqpoint{4.363798in}{1.782953in}}%
\pgfpathlineto{\pgfqpoint{4.349856in}{1.786651in}}%
\pgfpathlineto{\pgfqpoint{4.335920in}{1.790373in}}%
\pgfpathlineto{\pgfqpoint{4.321992in}{1.794121in}}%
\pgfpathlineto{\pgfqpoint{4.308070in}{1.797894in}}%
\pgfpathlineto{\pgfqpoint{4.315983in}{1.805656in}}%
\pgfpathlineto{\pgfqpoint{4.323890in}{1.813527in}}%
\pgfpathlineto{\pgfqpoint{4.331792in}{1.821500in}}%
\pgfpathlineto{\pgfqpoint{4.339689in}{1.829571in}}%
\pgfpathclose%
\pgfusepath{fill}%
\end{pgfscope}%
\begin{pgfscope}%
\pgfpathrectangle{\pgfqpoint{1.150000in}{0.150000in}}{\pgfqpoint{5.700000in}{5.700000in}}%
\pgfusepath{clip}%
\pgfsetbuttcap%
\pgfsetroundjoin%
\definecolor{currentfill}{rgb}{0.272594,0.025563,0.353093}%
\pgfsetfillcolor{currentfill}%
\pgfsetfillopacity{0.700000}%
\pgfsetlinewidth{0.000000pt}%
\definecolor{currentstroke}{rgb}{0.000000,0.000000,0.000000}%
\pgfsetstrokecolor{currentstroke}%
\pgfsetdash{}{0pt}%
\pgfpathmoveto{\pgfqpoint{4.569765in}{1.860751in}}%
\pgfpathlineto{\pgfqpoint{4.583736in}{1.857890in}}%
\pgfpathlineto{\pgfqpoint{4.597715in}{1.855053in}}%
\pgfpathlineto{\pgfqpoint{4.611701in}{1.852241in}}%
\pgfpathlineto{\pgfqpoint{4.625694in}{1.849454in}}%
\pgfpathlineto{\pgfqpoint{4.617884in}{1.840042in}}%
\pgfpathlineto{\pgfqpoint{4.610068in}{1.830674in}}%
\pgfpathlineto{\pgfqpoint{4.602247in}{1.821357in}}%
\pgfpathlineto{\pgfqpoint{4.594421in}{1.812093in}}%
\pgfpathlineto{\pgfqpoint{4.580418in}{1.815071in}}%
\pgfpathlineto{\pgfqpoint{4.566421in}{1.818073in}}%
\pgfpathlineto{\pgfqpoint{4.552432in}{1.821100in}}%
\pgfpathlineto{\pgfqpoint{4.538451in}{1.824152in}}%
\pgfpathlineto{\pgfqpoint{4.546287in}{1.833220in}}%
\pgfpathlineto{\pgfqpoint{4.554118in}{1.842346in}}%
\pgfpathlineto{\pgfqpoint{4.561944in}{1.851524in}}%
\pgfpathlineto{\pgfqpoint{4.569765in}{1.860751in}}%
\pgfpathclose%
\pgfusepath{fill}%
\end{pgfscope}%
\begin{pgfscope}%
\pgfpathrectangle{\pgfqpoint{1.150000in}{0.150000in}}{\pgfqpoint{5.700000in}{5.700000in}}%
\pgfusepath{clip}%
\pgfsetbuttcap%
\pgfsetroundjoin%
\definecolor{currentfill}{rgb}{0.281412,0.155834,0.469201}%
\pgfsetfillcolor{currentfill}%
\pgfsetfillopacity{0.700000}%
\pgfsetlinewidth{0.000000pt}%
\definecolor{currentstroke}{rgb}{0.000000,0.000000,0.000000}%
\pgfsetstrokecolor{currentstroke}%
\pgfsetdash{}{0pt}%
\pgfpathmoveto{\pgfqpoint{5.379244in}{2.105583in}}%
\pgfpathlineto{\pgfqpoint{5.393468in}{2.104754in}}%
\pgfpathlineto{\pgfqpoint{5.407701in}{2.103949in}}%
\pgfpathlineto{\pgfqpoint{5.421943in}{2.103168in}}%
\pgfpathlineto{\pgfqpoint{5.436194in}{2.102412in}}%
\pgfpathlineto{\pgfqpoint{5.428656in}{2.092642in}}%
\pgfpathlineto{\pgfqpoint{5.421110in}{2.082784in}}%
\pgfpathlineto{\pgfqpoint{5.413557in}{2.072841in}}%
\pgfpathlineto{\pgfqpoint{5.405997in}{2.062815in}}%
\pgfpathlineto{\pgfqpoint{5.391737in}{2.063654in}}%
\pgfpathlineto{\pgfqpoint{5.377486in}{2.064518in}}%
\pgfpathlineto{\pgfqpoint{5.363244in}{2.065407in}}%
\pgfpathlineto{\pgfqpoint{5.349011in}{2.066320in}}%
\pgfpathlineto{\pgfqpoint{5.356580in}{2.076257in}}%
\pgfpathlineto{\pgfqpoint{5.364141in}{2.086115in}}%
\pgfpathlineto{\pgfqpoint{5.371696in}{2.095891in}}%
\pgfpathlineto{\pgfqpoint{5.379244in}{2.105583in}}%
\pgfpathclose%
\pgfusepath{fill}%
\end{pgfscope}%
\begin{pgfscope}%
\pgfpathrectangle{\pgfqpoint{1.150000in}{0.150000in}}{\pgfqpoint{5.700000in}{5.700000in}}%
\pgfusepath{clip}%
\pgfsetbuttcap%
\pgfsetroundjoin%
\definecolor{currentfill}{rgb}{0.280267,0.073417,0.397163}%
\pgfsetfillcolor{currentfill}%
\pgfsetfillopacity{0.700000}%
\pgfsetlinewidth{0.000000pt}%
\definecolor{currentstroke}{rgb}{0.000000,0.000000,0.000000}%
\pgfsetstrokecolor{currentstroke}%
\pgfsetdash{}{0pt}%
\pgfpathmoveto{\pgfqpoint{4.887204in}{1.938770in}}%
\pgfpathlineto{\pgfqpoint{4.901269in}{1.936797in}}%
\pgfpathlineto{\pgfqpoint{4.915342in}{1.934849in}}%
\pgfpathlineto{\pgfqpoint{4.929424in}{1.932925in}}%
\pgfpathlineto{\pgfqpoint{4.943513in}{1.931026in}}%
\pgfpathlineto{\pgfqpoint{4.935800in}{1.920813in}}%
\pgfpathlineto{\pgfqpoint{4.928081in}{1.910585in}}%
\pgfpathlineto{\pgfqpoint{4.920356in}{1.900344in}}%
\pgfpathlineto{\pgfqpoint{4.912627in}{1.890093in}}%
\pgfpathlineto{\pgfqpoint{4.898529in}{1.892144in}}%
\pgfpathlineto{\pgfqpoint{4.884439in}{1.894218in}}%
\pgfpathlineto{\pgfqpoint{4.870357in}{1.896318in}}%
\pgfpathlineto{\pgfqpoint{4.856283in}{1.898442in}}%
\pgfpathlineto{\pgfqpoint{4.864021in}{1.908536in}}%
\pgfpathlineto{\pgfqpoint{4.871754in}{1.918624in}}%
\pgfpathlineto{\pgfqpoint{4.879481in}{1.928704in}}%
\pgfpathlineto{\pgfqpoint{4.887204in}{1.938770in}}%
\pgfpathclose%
\pgfusepath{fill}%
\end{pgfscope}%
\begin{pgfscope}%
\pgfpathrectangle{\pgfqpoint{1.150000in}{0.150000in}}{\pgfqpoint{5.700000in}{5.700000in}}%
\pgfusepath{clip}%
\pgfsetbuttcap%
\pgfsetroundjoin%
\definecolor{currentfill}{rgb}{0.248629,0.278775,0.534556}%
\pgfsetfillcolor{currentfill}%
\pgfsetfillopacity{0.700000}%
\pgfsetlinewidth{0.000000pt}%
\definecolor{currentstroke}{rgb}{0.000000,0.000000,0.000000}%
\pgfsetstrokecolor{currentstroke}%
\pgfsetdash{}{0pt}%
\pgfpathmoveto{\pgfqpoint{2.723430in}{2.350138in}}%
\pgfpathlineto{\pgfqpoint{2.737089in}{2.341233in}}%
\pgfpathlineto{\pgfqpoint{2.750750in}{2.332368in}}%
\pgfpathlineto{\pgfqpoint{2.764414in}{2.323542in}}%
\pgfpathlineto{\pgfqpoint{2.778081in}{2.314755in}}%
\pgfpathlineto{\pgfqpoint{2.769218in}{2.322010in}}%
\pgfpathlineto{\pgfqpoint{2.760332in}{2.329705in}}%
\pgfpathlineto{\pgfqpoint{2.751424in}{2.337849in}}%
\pgfpathlineto{\pgfqpoint{2.742492in}{2.346452in}}%
\pgfpathlineto{\pgfqpoint{2.728784in}{2.355586in}}%
\pgfpathlineto{\pgfqpoint{2.715079in}{2.364759in}}%
\pgfpathlineto{\pgfqpoint{2.701377in}{2.373971in}}%
\pgfpathlineto{\pgfqpoint{2.687677in}{2.383224in}}%
\pgfpathlineto{\pgfqpoint{2.696651in}{2.374267in}}%
\pgfpathlineto{\pgfqpoint{2.705601in}{2.365774in}}%
\pgfpathlineto{\pgfqpoint{2.714527in}{2.357734in}}%
\pgfpathlineto{\pgfqpoint{2.723430in}{2.350138in}}%
\pgfpathclose%
\pgfusepath{fill}%
\end{pgfscope}%
\begin{pgfscope}%
\pgfpathrectangle{\pgfqpoint{1.150000in}{0.150000in}}{\pgfqpoint{5.700000in}{5.700000in}}%
\pgfusepath{clip}%
\pgfsetbuttcap%
\pgfsetroundjoin%
\definecolor{currentfill}{rgb}{0.197636,0.391528,0.554969}%
\pgfsetfillcolor{currentfill}%
\pgfsetfillopacity{0.700000}%
\pgfsetlinewidth{0.000000pt}%
\definecolor{currentstroke}{rgb}{0.000000,0.000000,0.000000}%
\pgfsetstrokecolor{currentstroke}%
\pgfsetdash{}{0pt}%
\pgfpathmoveto{\pgfqpoint{2.359491in}{2.618172in}}%
\pgfpathlineto{\pgfqpoint{2.373146in}{2.607851in}}%
\pgfpathlineto{\pgfqpoint{2.386803in}{2.597579in}}%
\pgfpathlineto{\pgfqpoint{2.400461in}{2.587356in}}%
\pgfpathlineto{\pgfqpoint{2.414121in}{2.577181in}}%
\pgfpathlineto{\pgfqpoint{2.404897in}{2.588408in}}%
\pgfpathlineto{\pgfqpoint{2.395645in}{2.600138in}}%
\pgfpathlineto{\pgfqpoint{2.386363in}{2.612382in}}%
\pgfpathlineto{\pgfqpoint{2.377051in}{2.625151in}}%
\pgfpathlineto{\pgfqpoint{2.363343in}{2.635693in}}%
\pgfpathlineto{\pgfqpoint{2.349637in}{2.646285in}}%
\pgfpathlineto{\pgfqpoint{2.335932in}{2.656925in}}%
\pgfpathlineto{\pgfqpoint{2.322227in}{2.667616in}}%
\pgfpathlineto{\pgfqpoint{2.331589in}{2.654472in}}%
\pgfpathlineto{\pgfqpoint{2.340920in}{2.641857in}}%
\pgfpathlineto{\pgfqpoint{2.350220in}{2.629761in}}%
\pgfpathlineto{\pgfqpoint{2.359491in}{2.618172in}}%
\pgfpathclose%
\pgfusepath{fill}%
\end{pgfscope}%
\begin{pgfscope}%
\pgfpathrectangle{\pgfqpoint{1.150000in}{0.150000in}}{\pgfqpoint{5.700000in}{5.700000in}}%
\pgfusepath{clip}%
\pgfsetbuttcap%
\pgfsetroundjoin%
\definecolor{currentfill}{rgb}{0.282623,0.140926,0.457517}%
\pgfsetfillcolor{currentfill}%
\pgfsetfillopacity{0.700000}%
\pgfsetlinewidth{0.000000pt}%
\definecolor{currentstroke}{rgb}{0.000000,0.000000,0.000000}%
\pgfsetstrokecolor{currentstroke}%
\pgfsetdash{}{0pt}%
\pgfpathmoveto{\pgfqpoint{5.292166in}{2.070218in}}%
\pgfpathlineto{\pgfqpoint{5.306364in}{2.069207in}}%
\pgfpathlineto{\pgfqpoint{5.320571in}{2.068220in}}%
\pgfpathlineto{\pgfqpoint{5.334787in}{2.067257in}}%
\pgfpathlineto{\pgfqpoint{5.349011in}{2.066320in}}%
\pgfpathlineto{\pgfqpoint{5.341435in}{2.056304in}}%
\pgfpathlineto{\pgfqpoint{5.333853in}{2.046211in}}%
\pgfpathlineto{\pgfqpoint{5.326264in}{2.036045in}}%
\pgfpathlineto{\pgfqpoint{5.318668in}{2.025806in}}%
\pgfpathlineto{\pgfqpoint{5.304435in}{2.026841in}}%
\pgfpathlineto{\pgfqpoint{5.290211in}{2.027900in}}%
\pgfpathlineto{\pgfqpoint{5.275996in}{2.028984in}}%
\pgfpathlineto{\pgfqpoint{5.261789in}{2.030093in}}%
\pgfpathlineto{\pgfqpoint{5.269393in}{2.040229in}}%
\pgfpathlineto{\pgfqpoint{5.276991in}{2.050297in}}%
\pgfpathlineto{\pgfqpoint{5.284582in}{2.060294in}}%
\pgfpathlineto{\pgfqpoint{5.292166in}{2.070218in}}%
\pgfpathclose%
\pgfusepath{fill}%
\end{pgfscope}%
\begin{pgfscope}%
\pgfpathrectangle{\pgfqpoint{1.150000in}{0.150000in}}{\pgfqpoint{5.700000in}{5.700000in}}%
\pgfusepath{clip}%
\pgfsetbuttcap%
\pgfsetroundjoin%
\definecolor{currentfill}{rgb}{0.277941,0.056324,0.381191}%
\pgfsetfillcolor{currentfill}%
\pgfsetfillopacity{0.700000}%
\pgfsetlinewidth{0.000000pt}%
\definecolor{currentstroke}{rgb}{0.000000,0.000000,0.000000}%
\pgfsetstrokecolor{currentstroke}%
\pgfsetdash{}{0pt}%
\pgfpathmoveto{\pgfqpoint{4.800066in}{1.907184in}}%
\pgfpathlineto{\pgfqpoint{4.814108in}{1.904962in}}%
\pgfpathlineto{\pgfqpoint{4.828159in}{1.902764in}}%
\pgfpathlineto{\pgfqpoint{4.842217in}{1.900590in}}%
\pgfpathlineto{\pgfqpoint{4.856283in}{1.898442in}}%
\pgfpathlineto{\pgfqpoint{4.848540in}{1.888345in}}%
\pgfpathlineto{\pgfqpoint{4.840792in}{1.878251in}}%
\pgfpathlineto{\pgfqpoint{4.833039in}{1.868162in}}%
\pgfpathlineto{\pgfqpoint{4.825280in}{1.858082in}}%
\pgfpathlineto{\pgfqpoint{4.811205in}{1.860395in}}%
\pgfpathlineto{\pgfqpoint{4.797138in}{1.862733in}}%
\pgfpathlineto{\pgfqpoint{4.783079in}{1.865095in}}%
\pgfpathlineto{\pgfqpoint{4.769027in}{1.867482in}}%
\pgfpathlineto{\pgfqpoint{4.776794in}{1.877392in}}%
\pgfpathlineto{\pgfqpoint{4.784557in}{1.887315in}}%
\pgfpathlineto{\pgfqpoint{4.792314in}{1.897247in}}%
\pgfpathlineto{\pgfqpoint{4.800066in}{1.907184in}}%
\pgfpathclose%
\pgfusepath{fill}%
\end{pgfscope}%
\begin{pgfscope}%
\pgfpathrectangle{\pgfqpoint{1.150000in}{0.150000in}}{\pgfqpoint{5.700000in}{5.700000in}}%
\pgfusepath{clip}%
\pgfsetbuttcap%
\pgfsetroundjoin%
\definecolor{currentfill}{rgb}{0.269944,0.014625,0.341379}%
\pgfsetfillcolor{currentfill}%
\pgfsetfillopacity{0.700000}%
\pgfsetlinewidth{0.000000pt}%
\definecolor{currentstroke}{rgb}{0.000000,0.000000,0.000000}%
\pgfsetstrokecolor{currentstroke}%
\pgfsetdash{}{0pt}%
\pgfpathmoveto{\pgfqpoint{3.966881in}{1.830678in}}%
\pgfpathlineto{\pgfqpoint{3.980707in}{1.825917in}}%
\pgfpathlineto{\pgfqpoint{3.994539in}{1.821182in}}%
\pgfpathlineto{\pgfqpoint{4.008377in}{1.816473in}}%
\pgfpathlineto{\pgfqpoint{4.022222in}{1.811791in}}%
\pgfpathlineto{\pgfqpoint{4.014193in}{1.806271in}}%
\pgfpathlineto{\pgfqpoint{4.006158in}{1.800932in}}%
\pgfpathlineto{\pgfqpoint{3.998115in}{1.795781in}}%
\pgfpathlineto{\pgfqpoint{3.990065in}{1.790825in}}%
\pgfpathlineto{\pgfqpoint{3.976204in}{1.795764in}}%
\pgfpathlineto{\pgfqpoint{3.962348in}{1.800729in}}%
\pgfpathlineto{\pgfqpoint{3.948498in}{1.805720in}}%
\pgfpathlineto{\pgfqpoint{3.934654in}{1.810738in}}%
\pgfpathlineto{\pgfqpoint{3.942722in}{1.815432in}}%
\pgfpathlineto{\pgfqpoint{3.950782in}{1.820325in}}%
\pgfpathlineto{\pgfqpoint{3.958835in}{1.825409in}}%
\pgfpathlineto{\pgfqpoint{3.966881in}{1.830678in}}%
\pgfpathclose%
\pgfusepath{fill}%
\end{pgfscope}%
\begin{pgfscope}%
\pgfpathrectangle{\pgfqpoint{1.150000in}{0.150000in}}{\pgfqpoint{5.700000in}{5.700000in}}%
\pgfusepath{clip}%
\pgfsetbuttcap%
\pgfsetroundjoin%
\definecolor{currentfill}{rgb}{0.275191,0.194905,0.496005}%
\pgfsetfillcolor{currentfill}%
\pgfsetfillopacity{0.700000}%
\pgfsetlinewidth{0.000000pt}%
\definecolor{currentstroke}{rgb}{0.000000,0.000000,0.000000}%
\pgfsetstrokecolor{currentstroke}%
\pgfsetdash{}{0pt}%
\pgfpathmoveto{\pgfqpoint{3.031800in}{2.159699in}}%
\pgfpathlineto{\pgfqpoint{3.045484in}{2.151861in}}%
\pgfpathlineto{\pgfqpoint{3.059171in}{2.144057in}}%
\pgfpathlineto{\pgfqpoint{3.072861in}{2.136287in}}%
\pgfpathlineto{\pgfqpoint{3.086556in}{2.128551in}}%
\pgfpathlineto{\pgfqpoint{3.077954in}{2.132520in}}%
\pgfpathlineto{\pgfqpoint{3.069335in}{2.136873in}}%
\pgfpathlineto{\pgfqpoint{3.060698in}{2.141618in}}%
\pgfpathlineto{\pgfqpoint{3.052043in}{2.146765in}}%
\pgfpathlineto{\pgfqpoint{3.038314in}{2.154830in}}%
\pgfpathlineto{\pgfqpoint{3.024588in}{2.162930in}}%
\pgfpathlineto{\pgfqpoint{3.010867in}{2.171063in}}%
\pgfpathlineto{\pgfqpoint{2.997148in}{2.179230in}}%
\pgfpathlineto{\pgfqpoint{3.005839in}{2.173748in}}%
\pgfpathlineto{\pgfqpoint{3.014511in}{2.168672in}}%
\pgfpathlineto{\pgfqpoint{3.023165in}{2.163992in}}%
\pgfpathlineto{\pgfqpoint{3.031800in}{2.159699in}}%
\pgfpathclose%
\pgfusepath{fill}%
\end{pgfscope}%
\begin{pgfscope}%
\pgfpathrectangle{\pgfqpoint{1.150000in}{0.150000in}}{\pgfqpoint{5.700000in}{5.700000in}}%
\pgfusepath{clip}%
\pgfsetbuttcap%
\pgfsetroundjoin%
\definecolor{currentfill}{rgb}{0.268510,0.009605,0.335427}%
\pgfsetfillcolor{currentfill}%
\pgfsetfillopacity{0.700000}%
\pgfsetlinewidth{0.000000pt}%
\definecolor{currentstroke}{rgb}{0.000000,0.000000,0.000000}%
\pgfsetstrokecolor{currentstroke}%
\pgfsetdash{}{0pt}%
\pgfpathmoveto{\pgfqpoint{4.109639in}{1.818062in}}%
\pgfpathlineto{\pgfqpoint{4.123497in}{1.813753in}}%
\pgfpathlineto{\pgfqpoint{4.137362in}{1.809470in}}%
\pgfpathlineto{\pgfqpoint{4.151234in}{1.805212in}}%
\pgfpathlineto{\pgfqpoint{4.165111in}{1.800980in}}%
\pgfpathlineto{\pgfqpoint{4.157141in}{1.794321in}}%
\pgfpathlineto{\pgfqpoint{4.149165in}{1.787812in}}%
\pgfpathlineto{\pgfqpoint{4.141182in}{1.781458in}}%
\pgfpathlineto{\pgfqpoint{4.133193in}{1.775267in}}%
\pgfpathlineto{\pgfqpoint{4.119300in}{1.779742in}}%
\pgfpathlineto{\pgfqpoint{4.105413in}{1.784243in}}%
\pgfpathlineto{\pgfqpoint{4.091533in}{1.788770in}}%
\pgfpathlineto{\pgfqpoint{4.077658in}{1.793322in}}%
\pgfpathlineto{\pgfqpoint{4.085663in}{1.799265in}}%
\pgfpathlineto{\pgfqpoint{4.093662in}{1.805374in}}%
\pgfpathlineto{\pgfqpoint{4.101653in}{1.811642in}}%
\pgfpathlineto{\pgfqpoint{4.109639in}{1.818062in}}%
\pgfpathclose%
\pgfusepath{fill}%
\end{pgfscope}%
\begin{pgfscope}%
\pgfpathrectangle{\pgfqpoint{1.150000in}{0.150000in}}{\pgfqpoint{5.700000in}{5.700000in}}%
\pgfusepath{clip}%
\pgfsetbuttcap%
\pgfsetroundjoin%
\definecolor{currentfill}{rgb}{0.271828,0.209303,0.504434}%
\pgfsetfillcolor{currentfill}%
\pgfsetfillopacity{0.700000}%
\pgfsetlinewidth{0.000000pt}%
\definecolor{currentstroke}{rgb}{0.000000,0.000000,0.000000}%
\pgfsetstrokecolor{currentstroke}%
\pgfsetdash{}{0pt}%
\pgfpathmoveto{\pgfqpoint{5.697426in}{2.207270in}}%
\pgfpathlineto{\pgfqpoint{5.711766in}{2.207002in}}%
\pgfpathlineto{\pgfqpoint{5.726115in}{2.206758in}}%
\pgfpathlineto{\pgfqpoint{5.740474in}{2.206540in}}%
\pgfpathlineto{\pgfqpoint{5.733068in}{2.197768in}}%
\pgfpathlineto{\pgfqpoint{5.725654in}{2.188885in}}%
\pgfpathlineto{\pgfqpoint{5.718231in}{2.179890in}}%
\pgfpathlineto{\pgfqpoint{5.710800in}{2.170783in}}%
\pgfpathlineto{\pgfqpoint{5.696431in}{2.171044in}}%
\pgfpathlineto{\pgfqpoint{5.682071in}{2.171329in}}%
\pgfpathlineto{\pgfqpoint{5.667720in}{2.171639in}}%
\pgfpathlineto{\pgfqpoint{5.675159in}{2.180710in}}%
\pgfpathlineto{\pgfqpoint{5.682590in}{2.189673in}}%
\pgfpathlineto{\pgfqpoint{5.690013in}{2.198526in}}%
\pgfpathlineto{\pgfqpoint{5.697426in}{2.207270in}}%
\pgfpathclose%
\pgfusepath{fill}%
\end{pgfscope}%
\begin{pgfscope}%
\pgfpathrectangle{\pgfqpoint{1.150000in}{0.150000in}}{\pgfqpoint{5.700000in}{5.700000in}}%
\pgfusepath{clip}%
\pgfsetbuttcap%
\pgfsetroundjoin%
\definecolor{currentfill}{rgb}{0.283187,0.125848,0.444960}%
\pgfsetfillcolor{currentfill}%
\pgfsetfillopacity{0.700000}%
\pgfsetlinewidth{0.000000pt}%
\definecolor{currentstroke}{rgb}{0.000000,0.000000,0.000000}%
\pgfsetstrokecolor{currentstroke}%
\pgfsetdash{}{0pt}%
\pgfpathmoveto{\pgfqpoint{5.205049in}{2.034773in}}%
\pgfpathlineto{\pgfqpoint{5.219221in}{2.033566in}}%
\pgfpathlineto{\pgfqpoint{5.233402in}{2.032384in}}%
\pgfpathlineto{\pgfqpoint{5.247591in}{2.031226in}}%
\pgfpathlineto{\pgfqpoint{5.261789in}{2.030093in}}%
\pgfpathlineto{\pgfqpoint{5.254178in}{2.019890in}}%
\pgfpathlineto{\pgfqpoint{5.246561in}{2.009622in}}%
\pgfpathlineto{\pgfqpoint{5.238938in}{1.999293in}}%
\pgfpathlineto{\pgfqpoint{5.231308in}{1.988904in}}%
\pgfpathlineto{\pgfqpoint{5.217102in}{1.990148in}}%
\pgfpathlineto{\pgfqpoint{5.202905in}{1.991416in}}%
\pgfpathlineto{\pgfqpoint{5.188716in}{1.992709in}}%
\pgfpathlineto{\pgfqpoint{5.174535in}{1.994027in}}%
\pgfpathlineto{\pgfqpoint{5.182173in}{2.004300in}}%
\pgfpathlineto{\pgfqpoint{5.189804in}{2.014517in}}%
\pgfpathlineto{\pgfqpoint{5.197430in}{2.024675in}}%
\pgfpathlineto{\pgfqpoint{5.205049in}{2.034773in}}%
\pgfpathclose%
\pgfusepath{fill}%
\end{pgfscope}%
\begin{pgfscope}%
\pgfpathrectangle{\pgfqpoint{1.150000in}{0.150000in}}{\pgfqpoint{5.700000in}{5.700000in}}%
\pgfusepath{clip}%
\pgfsetbuttcap%
\pgfsetroundjoin%
\definecolor{currentfill}{rgb}{0.273809,0.031497,0.358853}%
\pgfsetfillcolor{currentfill}%
\pgfsetfillopacity{0.700000}%
\pgfsetlinewidth{0.000000pt}%
\definecolor{currentstroke}{rgb}{0.000000,0.000000,0.000000}%
\pgfsetstrokecolor{currentstroke}%
\pgfsetdash{}{0pt}%
\pgfpathmoveto{\pgfqpoint{3.824106in}{1.851834in}}%
\pgfpathlineto{\pgfqpoint{3.837905in}{1.846604in}}%
\pgfpathlineto{\pgfqpoint{3.851709in}{1.841400in}}%
\pgfpathlineto{\pgfqpoint{3.865519in}{1.836223in}}%
\pgfpathlineto{\pgfqpoint{3.879335in}{1.831072in}}%
\pgfpathlineto{\pgfqpoint{3.871240in}{1.826847in}}%
\pgfpathlineto{\pgfqpoint{3.863137in}{1.822837in}}%
\pgfpathlineto{\pgfqpoint{3.855026in}{1.819049in}}%
\pgfpathlineto{\pgfqpoint{3.846906in}{1.815490in}}%
\pgfpathlineto{\pgfqpoint{3.833071in}{1.820910in}}%
\pgfpathlineto{\pgfqpoint{3.819241in}{1.826357in}}%
\pgfpathlineto{\pgfqpoint{3.805417in}{1.831831in}}%
\pgfpathlineto{\pgfqpoint{3.791598in}{1.837332in}}%
\pgfpathlineto{\pgfqpoint{3.799738in}{1.840616in}}%
\pgfpathlineto{\pgfqpoint{3.807869in}{1.844132in}}%
\pgfpathlineto{\pgfqpoint{3.815992in}{1.847874in}}%
\pgfpathlineto{\pgfqpoint{3.824106in}{1.851834in}}%
\pgfpathclose%
\pgfusepath{fill}%
\end{pgfscope}%
\begin{pgfscope}%
\pgfpathrectangle{\pgfqpoint{1.150000in}{0.150000in}}{\pgfqpoint{5.700000in}{5.700000in}}%
\pgfusepath{clip}%
\pgfsetbuttcap%
\pgfsetroundjoin%
\definecolor{currentfill}{rgb}{0.271305,0.019942,0.347269}%
\pgfsetfillcolor{currentfill}%
\pgfsetfillopacity{0.700000}%
\pgfsetlinewidth{0.000000pt}%
\definecolor{currentstroke}{rgb}{0.000000,0.000000,0.000000}%
\pgfsetstrokecolor{currentstroke}%
\pgfsetdash{}{0pt}%
\pgfpathmoveto{\pgfqpoint{4.482595in}{1.836609in}}%
\pgfpathlineto{\pgfqpoint{4.496548in}{1.833457in}}%
\pgfpathlineto{\pgfqpoint{4.510508in}{1.830331in}}%
\pgfpathlineto{\pgfqpoint{4.524476in}{1.827229in}}%
\pgfpathlineto{\pgfqpoint{4.538451in}{1.824152in}}%
\pgfpathlineto{\pgfqpoint{4.530609in}{1.815146in}}%
\pgfpathlineto{\pgfqpoint{4.522763in}{1.806208in}}%
\pgfpathlineto{\pgfqpoint{4.514912in}{1.797342in}}%
\pgfpathlineto{\pgfqpoint{4.507056in}{1.788553in}}%
\pgfpathlineto{\pgfqpoint{4.493070in}{1.791834in}}%
\pgfpathlineto{\pgfqpoint{4.479092in}{1.795139in}}%
\pgfpathlineto{\pgfqpoint{4.465120in}{1.798469in}}%
\pgfpathlineto{\pgfqpoint{4.451156in}{1.801825in}}%
\pgfpathlineto{\pgfqpoint{4.459023in}{1.810405in}}%
\pgfpathlineto{\pgfqpoint{4.466886in}{1.819066in}}%
\pgfpathlineto{\pgfqpoint{4.474743in}{1.827802in}}%
\pgfpathlineto{\pgfqpoint{4.482595in}{1.836609in}}%
\pgfpathclose%
\pgfusepath{fill}%
\end{pgfscope}%
\begin{pgfscope}%
\pgfpathrectangle{\pgfqpoint{1.150000in}{0.150000in}}{\pgfqpoint{5.700000in}{5.700000in}}%
\pgfusepath{clip}%
\pgfsetbuttcap%
\pgfsetroundjoin%
\definecolor{currentfill}{rgb}{0.281924,0.089666,0.412415}%
\pgfsetfillcolor{currentfill}%
\pgfsetfillopacity{0.700000}%
\pgfsetlinewidth{0.000000pt}%
\definecolor{currentstroke}{rgb}{0.000000,0.000000,0.000000}%
\pgfsetstrokecolor{currentstroke}%
\pgfsetdash{}{0pt}%
\pgfpathmoveto{\pgfqpoint{3.483203in}{1.948166in}}%
\pgfpathlineto{\pgfqpoint{3.496943in}{1.941821in}}%
\pgfpathlineto{\pgfqpoint{3.510688in}{1.935505in}}%
\pgfpathlineto{\pgfqpoint{3.524438in}{1.929219in}}%
\pgfpathlineto{\pgfqpoint{3.538193in}{1.922961in}}%
\pgfpathlineto{\pgfqpoint{3.529911in}{1.922118in}}%
\pgfpathlineto{\pgfqpoint{3.521617in}{1.921567in}}%
\pgfpathlineto{\pgfqpoint{3.513311in}{1.921315in}}%
\pgfpathlineto{\pgfqpoint{3.504992in}{1.921370in}}%
\pgfpathlineto{\pgfqpoint{3.491212in}{1.927926in}}%
\pgfpathlineto{\pgfqpoint{3.477436in}{1.934511in}}%
\pgfpathlineto{\pgfqpoint{3.463664in}{1.941125in}}%
\pgfpathlineto{\pgfqpoint{3.449898in}{1.947768in}}%
\pgfpathlineto{\pgfqpoint{3.458242in}{1.947409in}}%
\pgfpathlineto{\pgfqpoint{3.466575in}{1.947362in}}%
\pgfpathlineto{\pgfqpoint{3.474895in}{1.947617in}}%
\pgfpathlineto{\pgfqpoint{3.483203in}{1.948166in}}%
\pgfpathclose%
\pgfusepath{fill}%
\end{pgfscope}%
\begin{pgfscope}%
\pgfpathrectangle{\pgfqpoint{1.150000in}{0.150000in}}{\pgfqpoint{5.700000in}{5.700000in}}%
\pgfusepath{clip}%
\pgfsetbuttcap%
\pgfsetroundjoin%
\definecolor{currentfill}{rgb}{0.204903,0.375746,0.553533}%
\pgfsetfillcolor{currentfill}%
\pgfsetfillopacity{0.700000}%
\pgfsetlinewidth{0.000000pt}%
\definecolor{currentstroke}{rgb}{0.000000,0.000000,0.000000}%
\pgfsetstrokecolor{currentstroke}%
\pgfsetdash{}{0pt}%
\pgfpathmoveto{\pgfqpoint{2.414121in}{2.577181in}}%
\pgfpathlineto{\pgfqpoint{2.427781in}{2.567055in}}%
\pgfpathlineto{\pgfqpoint{2.441444in}{2.556977in}}%
\pgfpathlineto{\pgfqpoint{2.455108in}{2.546945in}}%
\pgfpathlineto{\pgfqpoint{2.468773in}{2.536961in}}%
\pgfpathlineto{\pgfqpoint{2.459596in}{2.547826in}}%
\pgfpathlineto{\pgfqpoint{2.450391in}{2.559191in}}%
\pgfpathlineto{\pgfqpoint{2.441157in}{2.571066in}}%
\pgfpathlineto{\pgfqpoint{2.431894in}{2.583461in}}%
\pgfpathlineto{\pgfqpoint{2.418181in}{2.593812in}}%
\pgfpathlineto{\pgfqpoint{2.404470in}{2.604211in}}%
\pgfpathlineto{\pgfqpoint{2.390759in}{2.614657in}}%
\pgfpathlineto{\pgfqpoint{2.377051in}{2.625151in}}%
\pgfpathlineto{\pgfqpoint{2.386363in}{2.612382in}}%
\pgfpathlineto{\pgfqpoint{2.395645in}{2.600138in}}%
\pgfpathlineto{\pgfqpoint{2.404897in}{2.588408in}}%
\pgfpathlineto{\pgfqpoint{2.414121in}{2.577181in}}%
\pgfpathclose%
\pgfusepath{fill}%
\end{pgfscope}%
\begin{pgfscope}%
\pgfpathrectangle{\pgfqpoint{1.150000in}{0.150000in}}{\pgfqpoint{5.700000in}{5.700000in}}%
\pgfusepath{clip}%
\pgfsetbuttcap%
\pgfsetroundjoin%
\definecolor{currentfill}{rgb}{0.283072,0.130895,0.449241}%
\pgfsetfillcolor{currentfill}%
\pgfsetfillopacity{0.700000}%
\pgfsetlinewidth{0.000000pt}%
\definecolor{currentstroke}{rgb}{0.000000,0.000000,0.000000}%
\pgfsetstrokecolor{currentstroke}%
\pgfsetdash{}{0pt}%
\pgfpathmoveto{\pgfqpoint{3.285043in}{2.029809in}}%
\pgfpathlineto{\pgfqpoint{3.298757in}{2.022805in}}%
\pgfpathlineto{\pgfqpoint{3.312475in}{2.015832in}}%
\pgfpathlineto{\pgfqpoint{3.326198in}{2.008889in}}%
\pgfpathlineto{\pgfqpoint{3.339925in}{2.001978in}}%
\pgfpathlineto{\pgfqpoint{3.331511in}{2.003269in}}%
\pgfpathlineto{\pgfqpoint{3.323082in}{2.004895in}}%
\pgfpathlineto{\pgfqpoint{3.314640in}{2.006863in}}%
\pgfpathlineto{\pgfqpoint{3.306182in}{2.009182in}}%
\pgfpathlineto{\pgfqpoint{3.292426in}{2.016407in}}%
\pgfpathlineto{\pgfqpoint{3.278673in}{2.023662in}}%
\pgfpathlineto{\pgfqpoint{3.264925in}{2.030948in}}%
\pgfpathlineto{\pgfqpoint{3.251181in}{2.038265in}}%
\pgfpathlineto{\pgfqpoint{3.259669in}{2.035627in}}%
\pgfpathlineto{\pgfqpoint{3.268142in}{2.033344in}}%
\pgfpathlineto{\pgfqpoint{3.276600in}{2.031408in}}%
\pgfpathlineto{\pgfqpoint{3.285043in}{2.029809in}}%
\pgfpathclose%
\pgfusepath{fill}%
\end{pgfscope}%
\begin{pgfscope}%
\pgfpathrectangle{\pgfqpoint{1.150000in}{0.150000in}}{\pgfqpoint{5.700000in}{5.700000in}}%
\pgfusepath{clip}%
\pgfsetbuttcap%
\pgfsetroundjoin%
\definecolor{currentfill}{rgb}{0.268510,0.009605,0.335427}%
\pgfsetfillcolor{currentfill}%
\pgfsetfillopacity{0.700000}%
\pgfsetlinewidth{0.000000pt}%
\definecolor{currentstroke}{rgb}{0.000000,0.000000,0.000000}%
\pgfsetstrokecolor{currentstroke}%
\pgfsetdash{}{0pt}%
\pgfpathmoveto{\pgfqpoint{4.252448in}{1.813240in}}%
\pgfpathlineto{\pgfqpoint{4.266343in}{1.809365in}}%
\pgfpathlineto{\pgfqpoint{4.280245in}{1.805516in}}%
\pgfpathlineto{\pgfqpoint{4.294154in}{1.801693in}}%
\pgfpathlineto{\pgfqpoint{4.308070in}{1.797894in}}%
\pgfpathlineto{\pgfqpoint{4.300151in}{1.790246in}}%
\pgfpathlineto{\pgfqpoint{4.292227in}{1.782718in}}%
\pgfpathlineto{\pgfqpoint{4.284297in}{1.775314in}}%
\pgfpathlineto{\pgfqpoint{4.276361in}{1.768042in}}%
\pgfpathlineto{\pgfqpoint{4.262432in}{1.772070in}}%
\pgfpathlineto{\pgfqpoint{4.248510in}{1.776124in}}%
\pgfpathlineto{\pgfqpoint{4.234594in}{1.780203in}}%
\pgfpathlineto{\pgfqpoint{4.220685in}{1.784307in}}%
\pgfpathlineto{\pgfqpoint{4.228634in}{1.791345in}}%
\pgfpathlineto{\pgfqpoint{4.236578in}{1.798516in}}%
\pgfpathlineto{\pgfqpoint{4.244516in}{1.805817in}}%
\pgfpathlineto{\pgfqpoint{4.252448in}{1.813240in}}%
\pgfpathclose%
\pgfusepath{fill}%
\end{pgfscope}%
\begin{pgfscope}%
\pgfpathrectangle{\pgfqpoint{1.150000in}{0.150000in}}{\pgfqpoint{5.700000in}{5.700000in}}%
\pgfusepath{clip}%
\pgfsetbuttcap%
\pgfsetroundjoin%
\definecolor{currentfill}{rgb}{0.252194,0.269783,0.531579}%
\pgfsetfillcolor{currentfill}%
\pgfsetfillopacity{0.700000}%
\pgfsetlinewidth{0.000000pt}%
\definecolor{currentstroke}{rgb}{0.000000,0.000000,0.000000}%
\pgfsetstrokecolor{currentstroke}%
\pgfsetdash{}{0pt}%
\pgfpathmoveto{\pgfqpoint{2.778081in}{2.314755in}}%
\pgfpathlineto{\pgfqpoint{2.791750in}{2.306007in}}%
\pgfpathlineto{\pgfqpoint{2.805423in}{2.297297in}}%
\pgfpathlineto{\pgfqpoint{2.819098in}{2.288625in}}%
\pgfpathlineto{\pgfqpoint{2.832776in}{2.279991in}}%
\pgfpathlineto{\pgfqpoint{2.823952in}{2.286905in}}%
\pgfpathlineto{\pgfqpoint{2.815106in}{2.294256in}}%
\pgfpathlineto{\pgfqpoint{2.806238in}{2.302051in}}%
\pgfpathlineto{\pgfqpoint{2.797347in}{2.310302in}}%
\pgfpathlineto{\pgfqpoint{2.783629in}{2.319282in}}%
\pgfpathlineto{\pgfqpoint{2.769914in}{2.328300in}}%
\pgfpathlineto{\pgfqpoint{2.756202in}{2.337357in}}%
\pgfpathlineto{\pgfqpoint{2.742492in}{2.346452in}}%
\pgfpathlineto{\pgfqpoint{2.751424in}{2.337849in}}%
\pgfpathlineto{\pgfqpoint{2.760332in}{2.329705in}}%
\pgfpathlineto{\pgfqpoint{2.769218in}{2.322010in}}%
\pgfpathlineto{\pgfqpoint{2.778081in}{2.314755in}}%
\pgfpathclose%
\pgfusepath{fill}%
\end{pgfscope}%
\begin{pgfscope}%
\pgfpathrectangle{\pgfqpoint{1.150000in}{0.150000in}}{\pgfqpoint{5.700000in}{5.700000in}}%
\pgfusepath{clip}%
\pgfsetbuttcap%
\pgfsetroundjoin%
\definecolor{currentfill}{rgb}{0.283091,0.110553,0.431554}%
\pgfsetfillcolor{currentfill}%
\pgfsetfillopacity{0.700000}%
\pgfsetlinewidth{0.000000pt}%
\definecolor{currentstroke}{rgb}{0.000000,0.000000,0.000000}%
\pgfsetstrokecolor{currentstroke}%
\pgfsetdash{}{0pt}%
\pgfpathmoveto{\pgfqpoint{5.117898in}{1.999543in}}%
\pgfpathlineto{\pgfqpoint{5.132045in}{1.998127in}}%
\pgfpathlineto{\pgfqpoint{5.146200in}{1.996736in}}%
\pgfpathlineto{\pgfqpoint{5.160363in}{1.995369in}}%
\pgfpathlineto{\pgfqpoint{5.174535in}{1.994027in}}%
\pgfpathlineto{\pgfqpoint{5.166891in}{1.983701in}}%
\pgfpathlineto{\pgfqpoint{5.159242in}{1.973323in}}%
\pgfpathlineto{\pgfqpoint{5.151586in}{1.962898in}}%
\pgfpathlineto{\pgfqpoint{5.143925in}{1.952428in}}%
\pgfpathlineto{\pgfqpoint{5.129745in}{1.953894in}}%
\pgfpathlineto{\pgfqpoint{5.115573in}{1.955385in}}%
\pgfpathlineto{\pgfqpoint{5.101410in}{1.956901in}}%
\pgfpathlineto{\pgfqpoint{5.087255in}{1.958441in}}%
\pgfpathlineto{\pgfqpoint{5.094924in}{1.968782in}}%
\pgfpathlineto{\pgfqpoint{5.102588in}{1.979081in}}%
\pgfpathlineto{\pgfqpoint{5.110246in}{1.989336in}}%
\pgfpathlineto{\pgfqpoint{5.117898in}{1.999543in}}%
\pgfpathclose%
\pgfusepath{fill}%
\end{pgfscope}%
\begin{pgfscope}%
\pgfpathrectangle{\pgfqpoint{1.150000in}{0.150000in}}{\pgfqpoint{5.700000in}{5.700000in}}%
\pgfusepath{clip}%
\pgfsetbuttcap%
\pgfsetroundjoin%
\definecolor{currentfill}{rgb}{0.276022,0.044167,0.370164}%
\pgfsetfillcolor{currentfill}%
\pgfsetfillopacity{0.700000}%
\pgfsetlinewidth{0.000000pt}%
\definecolor{currentstroke}{rgb}{0.000000,0.000000,0.000000}%
\pgfsetstrokecolor{currentstroke}%
\pgfsetdash{}{0pt}%
\pgfpathmoveto{\pgfqpoint{4.712897in}{1.877276in}}%
\pgfpathlineto{\pgfqpoint{4.726918in}{1.874790in}}%
\pgfpathlineto{\pgfqpoint{4.740947in}{1.872329in}}%
\pgfpathlineto{\pgfqpoint{4.754983in}{1.869893in}}%
\pgfpathlineto{\pgfqpoint{4.769027in}{1.867482in}}%
\pgfpathlineto{\pgfqpoint{4.761255in}{1.857588in}}%
\pgfpathlineto{\pgfqpoint{4.753478in}{1.847716in}}%
\pgfpathlineto{\pgfqpoint{4.745696in}{1.837868in}}%
\pgfpathlineto{\pgfqpoint{4.737909in}{1.828050in}}%
\pgfpathlineto{\pgfqpoint{4.723856in}{1.830639in}}%
\pgfpathlineto{\pgfqpoint{4.709810in}{1.833253in}}%
\pgfpathlineto{\pgfqpoint{4.695772in}{1.835892in}}%
\pgfpathlineto{\pgfqpoint{4.681742in}{1.838555in}}%
\pgfpathlineto{\pgfqpoint{4.689538in}{1.848190in}}%
\pgfpathlineto{\pgfqpoint{4.697329in}{1.857858in}}%
\pgfpathlineto{\pgfqpoint{4.705116in}{1.867555in}}%
\pgfpathlineto{\pgfqpoint{4.712897in}{1.877276in}}%
\pgfpathclose%
\pgfusepath{fill}%
\end{pgfscope}%
\begin{pgfscope}%
\pgfpathrectangle{\pgfqpoint{1.150000in}{0.150000in}}{\pgfqpoint{5.700000in}{5.700000in}}%
\pgfusepath{clip}%
\pgfsetbuttcap%
\pgfsetroundjoin%
\definecolor{currentfill}{rgb}{0.277018,0.050344,0.375715}%
\pgfsetfillcolor{currentfill}%
\pgfsetfillopacity{0.700000}%
\pgfsetlinewidth{0.000000pt}%
\definecolor{currentstroke}{rgb}{0.000000,0.000000,0.000000}%
\pgfsetstrokecolor{currentstroke}%
\pgfsetdash{}{0pt}%
\pgfpathmoveto{\pgfqpoint{3.681238in}{1.882319in}}%
\pgfpathlineto{\pgfqpoint{3.695015in}{1.876599in}}%
\pgfpathlineto{\pgfqpoint{3.708797in}{1.870907in}}%
\pgfpathlineto{\pgfqpoint{3.722584in}{1.865243in}}%
\pgfpathlineto{\pgfqpoint{3.736376in}{1.859606in}}%
\pgfpathlineto{\pgfqpoint{3.728205in}{1.856840in}}%
\pgfpathlineto{\pgfqpoint{3.720025in}{1.854324in}}%
\pgfpathlineto{\pgfqpoint{3.711835in}{1.852066in}}%
\pgfpathlineto{\pgfqpoint{3.703635in}{1.850073in}}%
\pgfpathlineto{\pgfqpoint{3.689821in}{1.855994in}}%
\pgfpathlineto{\pgfqpoint{3.676011in}{1.861942in}}%
\pgfpathlineto{\pgfqpoint{3.662207in}{1.867918in}}%
\pgfpathlineto{\pgfqpoint{3.648408in}{1.873922in}}%
\pgfpathlineto{\pgfqpoint{3.656631in}{1.875626in}}%
\pgfpathlineto{\pgfqpoint{3.664843in}{1.877598in}}%
\pgfpathlineto{\pgfqpoint{3.673046in}{1.879832in}}%
\pgfpathlineto{\pgfqpoint{3.681238in}{1.882319in}}%
\pgfpathclose%
\pgfusepath{fill}%
\end{pgfscope}%
\begin{pgfscope}%
\pgfpathrectangle{\pgfqpoint{1.150000in}{0.150000in}}{\pgfqpoint{5.700000in}{5.700000in}}%
\pgfusepath{clip}%
\pgfsetbuttcap%
\pgfsetroundjoin%
\definecolor{currentfill}{rgb}{0.274128,0.199721,0.498911}%
\pgfsetfillcolor{currentfill}%
\pgfsetfillopacity{0.700000}%
\pgfsetlinewidth{0.000000pt}%
\definecolor{currentstroke}{rgb}{0.000000,0.000000,0.000000}%
\pgfsetstrokecolor{currentstroke}%
\pgfsetdash{}{0pt}%
\pgfpathmoveto{\pgfqpoint{5.610412in}{2.173126in}}%
\pgfpathlineto{\pgfqpoint{5.624725in}{2.172717in}}%
\pgfpathlineto{\pgfqpoint{5.639047in}{2.172333in}}%
\pgfpathlineto{\pgfqpoint{5.653379in}{2.171974in}}%
\pgfpathlineto{\pgfqpoint{5.667720in}{2.171639in}}%
\pgfpathlineto{\pgfqpoint{5.660273in}{2.162460in}}%
\pgfpathlineto{\pgfqpoint{5.652817in}{2.153173in}}%
\pgfpathlineto{\pgfqpoint{5.645353in}{2.143781in}}%
\pgfpathlineto{\pgfqpoint{5.637881in}{2.134283in}}%
\pgfpathlineto{\pgfqpoint{5.623530in}{2.134674in}}%
\pgfpathlineto{\pgfqpoint{5.609188in}{2.135089in}}%
\pgfpathlineto{\pgfqpoint{5.594855in}{2.135529in}}%
\pgfpathlineto{\pgfqpoint{5.580532in}{2.135993in}}%
\pgfpathlineto{\pgfqpoint{5.588014in}{2.145430in}}%
\pgfpathlineto{\pgfqpoint{5.595488in}{2.154765in}}%
\pgfpathlineto{\pgfqpoint{5.602954in}{2.163997in}}%
\pgfpathlineto{\pgfqpoint{5.610412in}{2.173126in}}%
\pgfpathclose%
\pgfusepath{fill}%
\end{pgfscope}%
\begin{pgfscope}%
\pgfpathrectangle{\pgfqpoint{1.150000in}{0.150000in}}{\pgfqpoint{5.700000in}{5.700000in}}%
\pgfusepath{clip}%
\pgfsetbuttcap%
\pgfsetroundjoin%
\definecolor{currentfill}{rgb}{0.282327,0.094955,0.417331}%
\pgfsetfillcolor{currentfill}%
\pgfsetfillopacity{0.700000}%
\pgfsetlinewidth{0.000000pt}%
\definecolor{currentstroke}{rgb}{0.000000,0.000000,0.000000}%
\pgfsetstrokecolor{currentstroke}%
\pgfsetdash{}{0pt}%
\pgfpathmoveto{\pgfqpoint{5.030718in}{1.964847in}}%
\pgfpathlineto{\pgfqpoint{5.044840in}{1.963209in}}%
\pgfpathlineto{\pgfqpoint{5.058970in}{1.961595in}}%
\pgfpathlineto{\pgfqpoint{5.073108in}{1.960006in}}%
\pgfpathlineto{\pgfqpoint{5.087255in}{1.958441in}}%
\pgfpathlineto{\pgfqpoint{5.079580in}{1.948061in}}%
\pgfpathlineto{\pgfqpoint{5.071899in}{1.937645in}}%
\pgfpathlineto{\pgfqpoint{5.064212in}{1.927196in}}%
\pgfpathlineto{\pgfqpoint{5.056520in}{1.916718in}}%
\pgfpathlineto{\pgfqpoint{5.042366in}{1.918420in}}%
\pgfpathlineto{\pgfqpoint{5.028220in}{1.920147in}}%
\pgfpathlineto{\pgfqpoint{5.014081in}{1.921899in}}%
\pgfpathlineto{\pgfqpoint{4.999952in}{1.923675in}}%
\pgfpathlineto{\pgfqpoint{5.007652in}{1.934011in}}%
\pgfpathlineto{\pgfqpoint{5.015346in}{1.944320in}}%
\pgfpathlineto{\pgfqpoint{5.023035in}{1.954600in}}%
\pgfpathlineto{\pgfqpoint{5.030718in}{1.964847in}}%
\pgfpathclose%
\pgfusepath{fill}%
\end{pgfscope}%
\begin{pgfscope}%
\pgfpathrectangle{\pgfqpoint{1.150000in}{0.150000in}}{\pgfqpoint{5.700000in}{5.700000in}}%
\pgfusepath{clip}%
\pgfsetbuttcap%
\pgfsetroundjoin%
\definecolor{currentfill}{rgb}{0.210503,0.363727,0.552206}%
\pgfsetfillcolor{currentfill}%
\pgfsetfillopacity{0.700000}%
\pgfsetlinewidth{0.000000pt}%
\definecolor{currentstroke}{rgb}{0.000000,0.000000,0.000000}%
\pgfsetstrokecolor{currentstroke}%
\pgfsetdash{}{0pt}%
\pgfpathmoveto{\pgfqpoint{2.468773in}{2.536961in}}%
\pgfpathlineto{\pgfqpoint{2.482440in}{2.527023in}}%
\pgfpathlineto{\pgfqpoint{2.496109in}{2.517130in}}%
\pgfpathlineto{\pgfqpoint{2.509780in}{2.507283in}}%
\pgfpathlineto{\pgfqpoint{2.523452in}{2.497481in}}%
\pgfpathlineto{\pgfqpoint{2.514321in}{2.507986in}}%
\pgfpathlineto{\pgfqpoint{2.505162in}{2.518987in}}%
\pgfpathlineto{\pgfqpoint{2.495975in}{2.530493in}}%
\pgfpathlineto{\pgfqpoint{2.486760in}{2.542515in}}%
\pgfpathlineto{\pgfqpoint{2.473041in}{2.552683in}}%
\pgfpathlineto{\pgfqpoint{2.459324in}{2.562897in}}%
\pgfpathlineto{\pgfqpoint{2.445608in}{2.573156in}}%
\pgfpathlineto{\pgfqpoint{2.431894in}{2.583461in}}%
\pgfpathlineto{\pgfqpoint{2.441157in}{2.571066in}}%
\pgfpathlineto{\pgfqpoint{2.450391in}{2.559191in}}%
\pgfpathlineto{\pgfqpoint{2.459596in}{2.547826in}}%
\pgfpathlineto{\pgfqpoint{2.468773in}{2.536961in}}%
\pgfpathclose%
\pgfusepath{fill}%
\end{pgfscope}%
\begin{pgfscope}%
\pgfpathrectangle{\pgfqpoint{1.150000in}{0.150000in}}{\pgfqpoint{5.700000in}{5.700000in}}%
\pgfusepath{clip}%
\pgfsetbuttcap%
\pgfsetroundjoin%
\definecolor{currentfill}{rgb}{0.269944,0.014625,0.341379}%
\pgfsetfillcolor{currentfill}%
\pgfsetfillopacity{0.700000}%
\pgfsetlinewidth{0.000000pt}%
\definecolor{currentstroke}{rgb}{0.000000,0.000000,0.000000}%
\pgfsetstrokecolor{currentstroke}%
\pgfsetdash{}{0pt}%
\pgfpathmoveto{\pgfqpoint{4.395368in}{1.815497in}}%
\pgfpathlineto{\pgfqpoint{4.409304in}{1.812041in}}%
\pgfpathlineto{\pgfqpoint{4.423248in}{1.808611in}}%
\pgfpathlineto{\pgfqpoint{4.437198in}{1.805205in}}%
\pgfpathlineto{\pgfqpoint{4.451156in}{1.801825in}}%
\pgfpathlineto{\pgfqpoint{4.443283in}{1.793330in}}%
\pgfpathlineto{\pgfqpoint{4.435406in}{1.784927in}}%
\pgfpathlineto{\pgfqpoint{4.427523in}{1.776619in}}%
\pgfpathlineto{\pgfqpoint{4.419635in}{1.768414in}}%
\pgfpathlineto{\pgfqpoint{4.405665in}{1.772011in}}%
\pgfpathlineto{\pgfqpoint{4.391703in}{1.775634in}}%
\pgfpathlineto{\pgfqpoint{4.377747in}{1.779281in}}%
\pgfpathlineto{\pgfqpoint{4.363798in}{1.782953in}}%
\pgfpathlineto{\pgfqpoint{4.371698in}{1.790937in}}%
\pgfpathlineto{\pgfqpoint{4.379593in}{1.799026in}}%
\pgfpathlineto{\pgfqpoint{4.387483in}{1.807214in}}%
\pgfpathlineto{\pgfqpoint{4.395368in}{1.815497in}}%
\pgfpathclose%
\pgfusepath{fill}%
\end{pgfscope}%
\begin{pgfscope}%
\pgfpathrectangle{\pgfqpoint{1.150000in}{0.150000in}}{\pgfqpoint{5.700000in}{5.700000in}}%
\pgfusepath{clip}%
\pgfsetbuttcap%
\pgfsetroundjoin%
\definecolor{currentfill}{rgb}{0.277134,0.185228,0.489898}%
\pgfsetfillcolor{currentfill}%
\pgfsetfillopacity{0.700000}%
\pgfsetlinewidth{0.000000pt}%
\definecolor{currentstroke}{rgb}{0.000000,0.000000,0.000000}%
\pgfsetstrokecolor{currentstroke}%
\pgfsetdash{}{0pt}%
\pgfpathmoveto{\pgfqpoint{3.086556in}{2.128551in}}%
\pgfpathlineto{\pgfqpoint{3.100254in}{2.120848in}}%
\pgfpathlineto{\pgfqpoint{3.113955in}{2.113178in}}%
\pgfpathlineto{\pgfqpoint{3.127661in}{2.105542in}}%
\pgfpathlineto{\pgfqpoint{3.141370in}{2.097938in}}%
\pgfpathlineto{\pgfqpoint{3.132801in}{2.101584in}}%
\pgfpathlineto{\pgfqpoint{3.124216in}{2.105610in}}%
\pgfpathlineto{\pgfqpoint{3.115613in}{2.110025in}}%
\pgfpathlineto{\pgfqpoint{3.106993in}{2.114837in}}%
\pgfpathlineto{\pgfqpoint{3.093250in}{2.122769in}}%
\pgfpathlineto{\pgfqpoint{3.079511in}{2.130735in}}%
\pgfpathlineto{\pgfqpoint{3.065775in}{2.138733in}}%
\pgfpathlineto{\pgfqpoint{3.052043in}{2.146765in}}%
\pgfpathlineto{\pgfqpoint{3.060698in}{2.141618in}}%
\pgfpathlineto{\pgfqpoint{3.069335in}{2.136873in}}%
\pgfpathlineto{\pgfqpoint{3.077954in}{2.132520in}}%
\pgfpathlineto{\pgfqpoint{3.086556in}{2.128551in}}%
\pgfpathclose%
\pgfusepath{fill}%
\end{pgfscope}%
\begin{pgfscope}%
\pgfpathrectangle{\pgfqpoint{1.150000in}{0.150000in}}{\pgfqpoint{5.700000in}{5.700000in}}%
\pgfusepath{clip}%
\pgfsetbuttcap%
\pgfsetroundjoin%
\definecolor{currentfill}{rgb}{0.278012,0.180367,0.486697}%
\pgfsetfillcolor{currentfill}%
\pgfsetfillopacity{0.700000}%
\pgfsetlinewidth{0.000000pt}%
\definecolor{currentstroke}{rgb}{0.000000,0.000000,0.000000}%
\pgfsetstrokecolor{currentstroke}%
\pgfsetdash{}{0pt}%
\pgfpathmoveto{\pgfqpoint{5.523331in}{2.138098in}}%
\pgfpathlineto{\pgfqpoint{5.537617in}{2.137534in}}%
\pgfpathlineto{\pgfqpoint{5.551913in}{2.136996in}}%
\pgfpathlineto{\pgfqpoint{5.566218in}{2.136482in}}%
\pgfpathlineto{\pgfqpoint{5.580532in}{2.135993in}}%
\pgfpathlineto{\pgfqpoint{5.573042in}{2.126456in}}%
\pgfpathlineto{\pgfqpoint{5.565545in}{2.116819in}}%
\pgfpathlineto{\pgfqpoint{5.558039in}{2.107084in}}%
\pgfpathlineto{\pgfqpoint{5.550526in}{2.097252in}}%
\pgfpathlineto{\pgfqpoint{5.536203in}{2.097811in}}%
\pgfpathlineto{\pgfqpoint{5.521889in}{2.098394in}}%
\pgfpathlineto{\pgfqpoint{5.507583in}{2.099002in}}%
\pgfpathlineto{\pgfqpoint{5.493288in}{2.099635in}}%
\pgfpathlineto{\pgfqpoint{5.500810in}{2.109392in}}%
\pgfpathlineto{\pgfqpoint{5.508324in}{2.119056in}}%
\pgfpathlineto{\pgfqpoint{5.515831in}{2.128625in}}%
\pgfpathlineto{\pgfqpoint{5.523331in}{2.138098in}}%
\pgfpathclose%
\pgfusepath{fill}%
\end{pgfscope}%
\begin{pgfscope}%
\pgfpathrectangle{\pgfqpoint{1.150000in}{0.150000in}}{\pgfqpoint{5.700000in}{5.700000in}}%
\pgfusepath{clip}%
\pgfsetbuttcap%
\pgfsetroundjoin%
\definecolor{currentfill}{rgb}{0.273809,0.031497,0.358853}%
\pgfsetfillcolor{currentfill}%
\pgfsetfillopacity{0.700000}%
\pgfsetlinewidth{0.000000pt}%
\definecolor{currentstroke}{rgb}{0.000000,0.000000,0.000000}%
\pgfsetstrokecolor{currentstroke}%
\pgfsetdash{}{0pt}%
\pgfpathmoveto{\pgfqpoint{4.625694in}{1.849454in}}%
\pgfpathlineto{\pgfqpoint{4.639695in}{1.846692in}}%
\pgfpathlineto{\pgfqpoint{4.653703in}{1.843955in}}%
\pgfpathlineto{\pgfqpoint{4.667719in}{1.841242in}}%
\pgfpathlineto{\pgfqpoint{4.681742in}{1.838555in}}%
\pgfpathlineto{\pgfqpoint{4.673941in}{1.828956in}}%
\pgfpathlineto{\pgfqpoint{4.666135in}{1.819400in}}%
\pgfpathlineto{\pgfqpoint{4.658324in}{1.809890in}}%
\pgfpathlineto{\pgfqpoint{4.650508in}{1.800431in}}%
\pgfpathlineto{\pgfqpoint{4.636475in}{1.803309in}}%
\pgfpathlineto{\pgfqpoint{4.622450in}{1.806212in}}%
\pgfpathlineto{\pgfqpoint{4.608432in}{1.809140in}}%
\pgfpathlineto{\pgfqpoint{4.594421in}{1.812093in}}%
\pgfpathlineto{\pgfqpoint{4.602247in}{1.821357in}}%
\pgfpathlineto{\pgfqpoint{4.610068in}{1.830674in}}%
\pgfpathlineto{\pgfqpoint{4.617884in}{1.840042in}}%
\pgfpathlineto{\pgfqpoint{4.625694in}{1.849454in}}%
\pgfpathclose%
\pgfusepath{fill}%
\end{pgfscope}%
\begin{pgfscope}%
\pgfpathrectangle{\pgfqpoint{1.150000in}{0.150000in}}{\pgfqpoint{5.700000in}{5.700000in}}%
\pgfusepath{clip}%
\pgfsetbuttcap%
\pgfsetroundjoin%
\definecolor{currentfill}{rgb}{0.280894,0.078907,0.402329}%
\pgfsetfillcolor{currentfill}%
\pgfsetfillopacity{0.700000}%
\pgfsetlinewidth{0.000000pt}%
\definecolor{currentstroke}{rgb}{0.000000,0.000000,0.000000}%
\pgfsetstrokecolor{currentstroke}%
\pgfsetdash{}{0pt}%
\pgfpathmoveto{\pgfqpoint{4.943513in}{1.931026in}}%
\pgfpathlineto{\pgfqpoint{4.957611in}{1.929151in}}%
\pgfpathlineto{\pgfqpoint{4.971716in}{1.927301in}}%
\pgfpathlineto{\pgfqpoint{4.985830in}{1.925476in}}%
\pgfpathlineto{\pgfqpoint{4.999952in}{1.923675in}}%
\pgfpathlineto{\pgfqpoint{4.992246in}{1.913317in}}%
\pgfpathlineto{\pgfqpoint{4.984535in}{1.902939in}}%
\pgfpathlineto{\pgfqpoint{4.976819in}{1.892545in}}%
\pgfpathlineto{\pgfqpoint{4.969098in}{1.882138in}}%
\pgfpathlineto{\pgfqpoint{4.954968in}{1.884090in}}%
\pgfpathlineto{\pgfqpoint{4.940846in}{1.886067in}}%
\pgfpathlineto{\pgfqpoint{4.926733in}{1.888068in}}%
\pgfpathlineto{\pgfqpoint{4.912627in}{1.890093in}}%
\pgfpathlineto{\pgfqpoint{4.920356in}{1.900344in}}%
\pgfpathlineto{\pgfqpoint{4.928081in}{1.910585in}}%
\pgfpathlineto{\pgfqpoint{4.935800in}{1.920813in}}%
\pgfpathlineto{\pgfqpoint{4.943513in}{1.931026in}}%
\pgfpathclose%
\pgfusepath{fill}%
\end{pgfscope}%
\begin{pgfscope}%
\pgfpathrectangle{\pgfqpoint{1.150000in}{0.150000in}}{\pgfqpoint{5.700000in}{5.700000in}}%
\pgfusepath{clip}%
\pgfsetbuttcap%
\pgfsetroundjoin%
\definecolor{currentfill}{rgb}{0.257322,0.256130,0.526563}%
\pgfsetfillcolor{currentfill}%
\pgfsetfillopacity{0.700000}%
\pgfsetlinewidth{0.000000pt}%
\definecolor{currentstroke}{rgb}{0.000000,0.000000,0.000000}%
\pgfsetstrokecolor{currentstroke}%
\pgfsetdash{}{0pt}%
\pgfpathmoveto{\pgfqpoint{2.832776in}{2.279991in}}%
\pgfpathlineto{\pgfqpoint{2.846456in}{2.271394in}}%
\pgfpathlineto{\pgfqpoint{2.860140in}{2.262835in}}%
\pgfpathlineto{\pgfqpoint{2.873827in}{2.254312in}}%
\pgfpathlineto{\pgfqpoint{2.887517in}{2.245827in}}%
\pgfpathlineto{\pgfqpoint{2.878731in}{2.252401in}}%
\pgfpathlineto{\pgfqpoint{2.869925in}{2.259408in}}%
\pgfpathlineto{\pgfqpoint{2.861096in}{2.266855in}}%
\pgfpathlineto{\pgfqpoint{2.852246in}{2.274754in}}%
\pgfpathlineto{\pgfqpoint{2.838517in}{2.283585in}}%
\pgfpathlineto{\pgfqpoint{2.824791in}{2.292454in}}%
\pgfpathlineto{\pgfqpoint{2.811068in}{2.301359in}}%
\pgfpathlineto{\pgfqpoint{2.797347in}{2.310302in}}%
\pgfpathlineto{\pgfqpoint{2.806238in}{2.302051in}}%
\pgfpathlineto{\pgfqpoint{2.815106in}{2.294256in}}%
\pgfpathlineto{\pgfqpoint{2.823952in}{2.286905in}}%
\pgfpathlineto{\pgfqpoint{2.832776in}{2.279991in}}%
\pgfpathclose%
\pgfusepath{fill}%
\end{pgfscope}%
\begin{pgfscope}%
\pgfpathrectangle{\pgfqpoint{1.150000in}{0.150000in}}{\pgfqpoint{5.700000in}{5.700000in}}%
\pgfusepath{clip}%
\pgfsetbuttcap%
\pgfsetroundjoin%
\definecolor{currentfill}{rgb}{0.280255,0.165693,0.476498}%
\pgfsetfillcolor{currentfill}%
\pgfsetfillopacity{0.700000}%
\pgfsetlinewidth{0.000000pt}%
\definecolor{currentstroke}{rgb}{0.000000,0.000000,0.000000}%
\pgfsetstrokecolor{currentstroke}%
\pgfsetdash{}{0pt}%
\pgfpathmoveto{\pgfqpoint{5.436194in}{2.102412in}}%
\pgfpathlineto{\pgfqpoint{5.450454in}{2.101681in}}%
\pgfpathlineto{\pgfqpoint{5.464723in}{2.100974in}}%
\pgfpathlineto{\pgfqpoint{5.479001in}{2.100292in}}%
\pgfpathlineto{\pgfqpoint{5.493288in}{2.099635in}}%
\pgfpathlineto{\pgfqpoint{5.485758in}{2.089786in}}%
\pgfpathlineto{\pgfqpoint{5.478221in}{2.079847in}}%
\pgfpathlineto{\pgfqpoint{5.470677in}{2.069819in}}%
\pgfpathlineto{\pgfqpoint{5.463125in}{2.059704in}}%
\pgfpathlineto{\pgfqpoint{5.448829in}{2.060445in}}%
\pgfpathlineto{\pgfqpoint{5.434543in}{2.061210in}}%
\pgfpathlineto{\pgfqpoint{5.420265in}{2.062000in}}%
\pgfpathlineto{\pgfqpoint{5.405997in}{2.062815in}}%
\pgfpathlineto{\pgfqpoint{5.413557in}{2.072841in}}%
\pgfpathlineto{\pgfqpoint{5.421110in}{2.082784in}}%
\pgfpathlineto{\pgfqpoint{5.428656in}{2.092642in}}%
\pgfpathlineto{\pgfqpoint{5.436194in}{2.102412in}}%
\pgfpathclose%
\pgfusepath{fill}%
\end{pgfscope}%
\begin{pgfscope}%
\pgfpathrectangle{\pgfqpoint{1.150000in}{0.150000in}}{\pgfqpoint{5.700000in}{5.700000in}}%
\pgfusepath{clip}%
\pgfsetbuttcap%
\pgfsetroundjoin%
\definecolor{currentfill}{rgb}{0.269944,0.014625,0.341379}%
\pgfsetfillcolor{currentfill}%
\pgfsetfillopacity{0.700000}%
\pgfsetlinewidth{0.000000pt}%
\definecolor{currentstroke}{rgb}{0.000000,0.000000,0.000000}%
\pgfsetstrokecolor{currentstroke}%
\pgfsetdash{}{0pt}%
\pgfpathmoveto{\pgfqpoint{4.022222in}{1.811791in}}%
\pgfpathlineto{\pgfqpoint{4.036072in}{1.807135in}}%
\pgfpathlineto{\pgfqpoint{4.049928in}{1.802504in}}%
\pgfpathlineto{\pgfqpoint{4.063790in}{1.797900in}}%
\pgfpathlineto{\pgfqpoint{4.077658in}{1.793322in}}%
\pgfpathlineto{\pgfqpoint{4.069647in}{1.787550in}}%
\pgfpathlineto{\pgfqpoint{4.061628in}{1.781957in}}%
\pgfpathlineto{\pgfqpoint{4.053603in}{1.776548in}}%
\pgfpathlineto{\pgfqpoint{4.045570in}{1.771330in}}%
\pgfpathlineto{\pgfqpoint{4.031685in}{1.776165in}}%
\pgfpathlineto{\pgfqpoint{4.017806in}{1.781025in}}%
\pgfpathlineto{\pgfqpoint{4.003933in}{1.785912in}}%
\pgfpathlineto{\pgfqpoint{3.990065in}{1.790825in}}%
\pgfpathlineto{\pgfqpoint{3.998115in}{1.795781in}}%
\pgfpathlineto{\pgfqpoint{4.006158in}{1.800932in}}%
\pgfpathlineto{\pgfqpoint{4.014193in}{1.806271in}}%
\pgfpathlineto{\pgfqpoint{4.022222in}{1.811791in}}%
\pgfpathclose%
\pgfusepath{fill}%
\end{pgfscope}%
\begin{pgfscope}%
\pgfpathrectangle{\pgfqpoint{1.150000in}{0.150000in}}{\pgfqpoint{5.700000in}{5.700000in}}%
\pgfusepath{clip}%
\pgfsetbuttcap%
\pgfsetroundjoin%
\definecolor{currentfill}{rgb}{0.281446,0.084320,0.407414}%
\pgfsetfillcolor{currentfill}%
\pgfsetfillopacity{0.700000}%
\pgfsetlinewidth{0.000000pt}%
\definecolor{currentstroke}{rgb}{0.000000,0.000000,0.000000}%
\pgfsetstrokecolor{currentstroke}%
\pgfsetdash{}{0pt}%
\pgfpathmoveto{\pgfqpoint{3.538193in}{1.922961in}}%
\pgfpathlineto{\pgfqpoint{3.551953in}{1.916731in}}%
\pgfpathlineto{\pgfqpoint{3.565717in}{1.910531in}}%
\pgfpathlineto{\pgfqpoint{3.579487in}{1.904359in}}%
\pgfpathlineto{\pgfqpoint{3.593261in}{1.898215in}}%
\pgfpathlineto{\pgfqpoint{3.585003in}{1.897080in}}%
\pgfpathlineto{\pgfqpoint{3.576734in}{1.896233in}}%
\pgfpathlineto{\pgfqpoint{3.568454in}{1.895681in}}%
\pgfpathlineto{\pgfqpoint{3.560162in}{1.895432in}}%
\pgfpathlineto{\pgfqpoint{3.546363in}{1.901874in}}%
\pgfpathlineto{\pgfqpoint{3.532568in}{1.908344in}}%
\pgfpathlineto{\pgfqpoint{3.518778in}{1.914843in}}%
\pgfpathlineto{\pgfqpoint{3.504992in}{1.921370in}}%
\pgfpathlineto{\pgfqpoint{3.513311in}{1.921315in}}%
\pgfpathlineto{\pgfqpoint{3.521617in}{1.921567in}}%
\pgfpathlineto{\pgfqpoint{3.529911in}{1.922118in}}%
\pgfpathlineto{\pgfqpoint{3.538193in}{1.922961in}}%
\pgfpathclose%
\pgfusepath{fill}%
\end{pgfscope}%
\begin{pgfscope}%
\pgfpathrectangle{\pgfqpoint{1.150000in}{0.150000in}}{\pgfqpoint{5.700000in}{5.700000in}}%
\pgfusepath{clip}%
\pgfsetbuttcap%
\pgfsetroundjoin%
\definecolor{currentfill}{rgb}{0.283187,0.125848,0.444960}%
\pgfsetfillcolor{currentfill}%
\pgfsetfillopacity{0.700000}%
\pgfsetlinewidth{0.000000pt}%
\definecolor{currentstroke}{rgb}{0.000000,0.000000,0.000000}%
\pgfsetstrokecolor{currentstroke}%
\pgfsetdash{}{0pt}%
\pgfpathmoveto{\pgfqpoint{3.339925in}{2.001978in}}%
\pgfpathlineto{\pgfqpoint{3.353656in}{1.995096in}}%
\pgfpathlineto{\pgfqpoint{3.367391in}{1.988246in}}%
\pgfpathlineto{\pgfqpoint{3.381131in}{1.981425in}}%
\pgfpathlineto{\pgfqpoint{3.394876in}{1.974634in}}%
\pgfpathlineto{\pgfqpoint{3.386490in}{1.975618in}}%
\pgfpathlineto{\pgfqpoint{3.378091in}{1.976933in}}%
\pgfpathlineto{\pgfqpoint{3.369678in}{1.978586in}}%
\pgfpathlineto{\pgfqpoint{3.361250in}{1.980587in}}%
\pgfpathlineto{\pgfqpoint{3.347477in}{1.987691in}}%
\pgfpathlineto{\pgfqpoint{3.333708in}{1.994824in}}%
\pgfpathlineto{\pgfqpoint{3.319943in}{2.001988in}}%
\pgfpathlineto{\pgfqpoint{3.306182in}{2.009182in}}%
\pgfpathlineto{\pgfqpoint{3.314640in}{2.006863in}}%
\pgfpathlineto{\pgfqpoint{3.323082in}{2.004895in}}%
\pgfpathlineto{\pgfqpoint{3.331511in}{2.003269in}}%
\pgfpathlineto{\pgfqpoint{3.339925in}{2.001978in}}%
\pgfpathclose%
\pgfusepath{fill}%
\end{pgfscope}%
\begin{pgfscope}%
\pgfpathrectangle{\pgfqpoint{1.150000in}{0.150000in}}{\pgfqpoint{5.700000in}{5.700000in}}%
\pgfusepath{clip}%
\pgfsetbuttcap%
\pgfsetroundjoin%
\definecolor{currentfill}{rgb}{0.272594,0.025563,0.353093}%
\pgfsetfillcolor{currentfill}%
\pgfsetfillopacity{0.700000}%
\pgfsetlinewidth{0.000000pt}%
\definecolor{currentstroke}{rgb}{0.000000,0.000000,0.000000}%
\pgfsetstrokecolor{currentstroke}%
\pgfsetdash{}{0pt}%
\pgfpathmoveto{\pgfqpoint{3.879335in}{1.831072in}}%
\pgfpathlineto{\pgfqpoint{3.893156in}{1.825949in}}%
\pgfpathlineto{\pgfqpoint{3.906983in}{1.820852in}}%
\pgfpathlineto{\pgfqpoint{3.920816in}{1.815781in}}%
\pgfpathlineto{\pgfqpoint{3.934654in}{1.810738in}}%
\pgfpathlineto{\pgfqpoint{3.926578in}{1.806248in}}%
\pgfpathlineto{\pgfqpoint{3.918495in}{1.801969in}}%
\pgfpathlineto{\pgfqpoint{3.910403in}{1.797909in}}%
\pgfpathlineto{\pgfqpoint{3.902303in}{1.794074in}}%
\pgfpathlineto{\pgfqpoint{3.888446in}{1.799389in}}%
\pgfpathlineto{\pgfqpoint{3.874594in}{1.804729in}}%
\pgfpathlineto{\pgfqpoint{3.860747in}{1.810096in}}%
\pgfpathlineto{\pgfqpoint{3.846906in}{1.815490in}}%
\pgfpathlineto{\pgfqpoint{3.855026in}{1.819049in}}%
\pgfpathlineto{\pgfqpoint{3.863137in}{1.822837in}}%
\pgfpathlineto{\pgfqpoint{3.871240in}{1.826847in}}%
\pgfpathlineto{\pgfqpoint{3.879335in}{1.831072in}}%
\pgfpathclose%
\pgfusepath{fill}%
\end{pgfscope}%
\begin{pgfscope}%
\pgfpathrectangle{\pgfqpoint{1.150000in}{0.150000in}}{\pgfqpoint{5.700000in}{5.700000in}}%
\pgfusepath{clip}%
\pgfsetbuttcap%
\pgfsetroundjoin%
\definecolor{currentfill}{rgb}{0.268510,0.009605,0.335427}%
\pgfsetfillcolor{currentfill}%
\pgfsetfillopacity{0.700000}%
\pgfsetlinewidth{0.000000pt}%
\definecolor{currentstroke}{rgb}{0.000000,0.000000,0.000000}%
\pgfsetstrokecolor{currentstroke}%
\pgfsetdash{}{0pt}%
\pgfpathmoveto{\pgfqpoint{4.165111in}{1.800980in}}%
\pgfpathlineto{\pgfqpoint{4.178995in}{1.796773in}}%
\pgfpathlineto{\pgfqpoint{4.192885in}{1.792592in}}%
\pgfpathlineto{\pgfqpoint{4.206782in}{1.788437in}}%
\pgfpathlineto{\pgfqpoint{4.220685in}{1.784307in}}%
\pgfpathlineto{\pgfqpoint{4.212729in}{1.777410in}}%
\pgfpathlineto{\pgfqpoint{4.204768in}{1.770660in}}%
\pgfpathlineto{\pgfqpoint{4.196800in}{1.764061in}}%
\pgfpathlineto{\pgfqpoint{4.188827in}{1.757621in}}%
\pgfpathlineto{\pgfqpoint{4.174909in}{1.761994in}}%
\pgfpathlineto{\pgfqpoint{4.160997in}{1.766393in}}%
\pgfpathlineto{\pgfqpoint{4.147092in}{1.770817in}}%
\pgfpathlineto{\pgfqpoint{4.133193in}{1.775267in}}%
\pgfpathlineto{\pgfqpoint{4.141182in}{1.781458in}}%
\pgfpathlineto{\pgfqpoint{4.149165in}{1.787812in}}%
\pgfpathlineto{\pgfqpoint{4.157141in}{1.794321in}}%
\pgfpathlineto{\pgfqpoint{4.165111in}{1.800980in}}%
\pgfpathclose%
\pgfusepath{fill}%
\end{pgfscope}%
\begin{pgfscope}%
\pgfpathrectangle{\pgfqpoint{1.150000in}{0.150000in}}{\pgfqpoint{5.700000in}{5.700000in}}%
\pgfusepath{clip}%
\pgfsetbuttcap%
\pgfsetroundjoin%
\definecolor{currentfill}{rgb}{0.281887,0.150881,0.465405}%
\pgfsetfillcolor{currentfill}%
\pgfsetfillopacity{0.700000}%
\pgfsetlinewidth{0.000000pt}%
\definecolor{currentstroke}{rgb}{0.000000,0.000000,0.000000}%
\pgfsetstrokecolor{currentstroke}%
\pgfsetdash{}{0pt}%
\pgfpathmoveto{\pgfqpoint{5.349011in}{2.066320in}}%
\pgfpathlineto{\pgfqpoint{5.363244in}{2.065407in}}%
\pgfpathlineto{\pgfqpoint{5.377486in}{2.064518in}}%
\pgfpathlineto{\pgfqpoint{5.391737in}{2.063654in}}%
\pgfpathlineto{\pgfqpoint{5.405997in}{2.062815in}}%
\pgfpathlineto{\pgfqpoint{5.398429in}{2.052707in}}%
\pgfpathlineto{\pgfqpoint{5.390855in}{2.042519in}}%
\pgfpathlineto{\pgfqpoint{5.383274in}{2.032254in}}%
\pgfpathlineto{\pgfqpoint{5.375687in}{2.021913in}}%
\pgfpathlineto{\pgfqpoint{5.361419in}{2.022849in}}%
\pgfpathlineto{\pgfqpoint{5.347160in}{2.023810in}}%
\pgfpathlineto{\pgfqpoint{5.332909in}{2.024796in}}%
\pgfpathlineto{\pgfqpoint{5.318668in}{2.025806in}}%
\pgfpathlineto{\pgfqpoint{5.326264in}{2.036045in}}%
\pgfpathlineto{\pgfqpoint{5.333853in}{2.046211in}}%
\pgfpathlineto{\pgfqpoint{5.341435in}{2.056304in}}%
\pgfpathlineto{\pgfqpoint{5.349011in}{2.066320in}}%
\pgfpathclose%
\pgfusepath{fill}%
\end{pgfscope}%
\begin{pgfscope}%
\pgfpathrectangle{\pgfqpoint{1.150000in}{0.150000in}}{\pgfqpoint{5.700000in}{5.700000in}}%
\pgfusepath{clip}%
\pgfsetbuttcap%
\pgfsetroundjoin%
\definecolor{currentfill}{rgb}{0.278791,0.062145,0.386592}%
\pgfsetfillcolor{currentfill}%
\pgfsetfillopacity{0.700000}%
\pgfsetlinewidth{0.000000pt}%
\definecolor{currentstroke}{rgb}{0.000000,0.000000,0.000000}%
\pgfsetstrokecolor{currentstroke}%
\pgfsetdash{}{0pt}%
\pgfpathmoveto{\pgfqpoint{4.856283in}{1.898442in}}%
\pgfpathlineto{\pgfqpoint{4.870357in}{1.896318in}}%
\pgfpathlineto{\pgfqpoint{4.884439in}{1.894218in}}%
\pgfpathlineto{\pgfqpoint{4.898529in}{1.892144in}}%
\pgfpathlineto{\pgfqpoint{4.912627in}{1.890093in}}%
\pgfpathlineto{\pgfqpoint{4.904892in}{1.879838in}}%
\pgfpathlineto{\pgfqpoint{4.897153in}{1.869580in}}%
\pgfpathlineto{\pgfqpoint{4.889408in}{1.859325in}}%
\pgfpathlineto{\pgfqpoint{4.881658in}{1.849076in}}%
\pgfpathlineto{\pgfqpoint{4.867552in}{1.851291in}}%
\pgfpathlineto{\pgfqpoint{4.853453in}{1.853530in}}%
\pgfpathlineto{\pgfqpoint{4.839363in}{1.855794in}}%
\pgfpathlineto{\pgfqpoint{4.825280in}{1.858082in}}%
\pgfpathlineto{\pgfqpoint{4.833039in}{1.868162in}}%
\pgfpathlineto{\pgfqpoint{4.840792in}{1.878251in}}%
\pgfpathlineto{\pgfqpoint{4.848540in}{1.888345in}}%
\pgfpathlineto{\pgfqpoint{4.856283in}{1.898442in}}%
\pgfpathclose%
\pgfusepath{fill}%
\end{pgfscope}%
\begin{pgfscope}%
\pgfpathrectangle{\pgfqpoint{1.150000in}{0.150000in}}{\pgfqpoint{5.700000in}{5.700000in}}%
\pgfusepath{clip}%
\pgfsetbuttcap%
\pgfsetroundjoin%
\definecolor{currentfill}{rgb}{0.216210,0.351535,0.550627}%
\pgfsetfillcolor{currentfill}%
\pgfsetfillopacity{0.700000}%
\pgfsetlinewidth{0.000000pt}%
\definecolor{currentstroke}{rgb}{0.000000,0.000000,0.000000}%
\pgfsetstrokecolor{currentstroke}%
\pgfsetdash{}{0pt}%
\pgfpathmoveto{\pgfqpoint{2.523452in}{2.497481in}}%
\pgfpathlineto{\pgfqpoint{2.537126in}{2.487724in}}%
\pgfpathlineto{\pgfqpoint{2.550802in}{2.478011in}}%
\pgfpathlineto{\pgfqpoint{2.564480in}{2.468341in}}%
\pgfpathlineto{\pgfqpoint{2.578160in}{2.458715in}}%
\pgfpathlineto{\pgfqpoint{2.569074in}{2.468861in}}%
\pgfpathlineto{\pgfqpoint{2.559961in}{2.479498in}}%
\pgfpathlineto{\pgfqpoint{2.550821in}{2.490636in}}%
\pgfpathlineto{\pgfqpoint{2.541652in}{2.502287in}}%
\pgfpathlineto{\pgfqpoint{2.527927in}{2.512278in}}%
\pgfpathlineto{\pgfqpoint{2.514203in}{2.522313in}}%
\pgfpathlineto{\pgfqpoint{2.500480in}{2.532392in}}%
\pgfpathlineto{\pgfqpoint{2.486760in}{2.542515in}}%
\pgfpathlineto{\pgfqpoint{2.495975in}{2.530493in}}%
\pgfpathlineto{\pgfqpoint{2.505162in}{2.518987in}}%
\pgfpathlineto{\pgfqpoint{2.514321in}{2.507986in}}%
\pgfpathlineto{\pgfqpoint{2.523452in}{2.497481in}}%
\pgfpathclose%
\pgfusepath{fill}%
\end{pgfscope}%
\begin{pgfscope}%
\pgfpathrectangle{\pgfqpoint{1.150000in}{0.150000in}}{\pgfqpoint{5.700000in}{5.700000in}}%
\pgfusepath{clip}%
\pgfsetbuttcap%
\pgfsetroundjoin%
\definecolor{currentfill}{rgb}{0.271305,0.019942,0.347269}%
\pgfsetfillcolor{currentfill}%
\pgfsetfillopacity{0.700000}%
\pgfsetlinewidth{0.000000pt}%
\definecolor{currentstroke}{rgb}{0.000000,0.000000,0.000000}%
\pgfsetstrokecolor{currentstroke}%
\pgfsetdash{}{0pt}%
\pgfpathmoveto{\pgfqpoint{4.538451in}{1.824152in}}%
\pgfpathlineto{\pgfqpoint{4.552432in}{1.821100in}}%
\pgfpathlineto{\pgfqpoint{4.566421in}{1.818073in}}%
\pgfpathlineto{\pgfqpoint{4.580418in}{1.815071in}}%
\pgfpathlineto{\pgfqpoint{4.594421in}{1.812093in}}%
\pgfpathlineto{\pgfqpoint{4.586590in}{1.802889in}}%
\pgfpathlineto{\pgfqpoint{4.578755in}{1.793748in}}%
\pgfpathlineto{\pgfqpoint{4.570914in}{1.784676in}}%
\pgfpathlineto{\pgfqpoint{4.563069in}{1.775678in}}%
\pgfpathlineto{\pgfqpoint{4.549055in}{1.778860in}}%
\pgfpathlineto{\pgfqpoint{4.535048in}{1.782066in}}%
\pgfpathlineto{\pgfqpoint{4.521048in}{1.785297in}}%
\pgfpathlineto{\pgfqpoint{4.507056in}{1.788553in}}%
\pgfpathlineto{\pgfqpoint{4.514912in}{1.797342in}}%
\pgfpathlineto{\pgfqpoint{4.522763in}{1.806208in}}%
\pgfpathlineto{\pgfqpoint{4.530609in}{1.815146in}}%
\pgfpathlineto{\pgfqpoint{4.538451in}{1.824152in}}%
\pgfpathclose%
\pgfusepath{fill}%
\end{pgfscope}%
\begin{pgfscope}%
\pgfpathrectangle{\pgfqpoint{1.150000in}{0.150000in}}{\pgfqpoint{5.700000in}{5.700000in}}%
\pgfusepath{clip}%
\pgfsetbuttcap%
\pgfsetroundjoin%
\definecolor{currentfill}{rgb}{0.277018,0.050344,0.375715}%
\pgfsetfillcolor{currentfill}%
\pgfsetfillopacity{0.700000}%
\pgfsetlinewidth{0.000000pt}%
\definecolor{currentstroke}{rgb}{0.000000,0.000000,0.000000}%
\pgfsetstrokecolor{currentstroke}%
\pgfsetdash{}{0pt}%
\pgfpathmoveto{\pgfqpoint{3.736376in}{1.859606in}}%
\pgfpathlineto{\pgfqpoint{3.750173in}{1.853997in}}%
\pgfpathlineto{\pgfqpoint{3.763976in}{1.848414in}}%
\pgfpathlineto{\pgfqpoint{3.777784in}{1.842859in}}%
\pgfpathlineto{\pgfqpoint{3.791598in}{1.837332in}}%
\pgfpathlineto{\pgfqpoint{3.783449in}{1.834287in}}%
\pgfpathlineto{\pgfqpoint{3.775290in}{1.831489in}}%
\pgfpathlineto{\pgfqpoint{3.767123in}{1.828946in}}%
\pgfpathlineto{\pgfqpoint{3.758945in}{1.826664in}}%
\pgfpathlineto{\pgfqpoint{3.745110in}{1.832475in}}%
\pgfpathlineto{\pgfqpoint{3.731280in}{1.838314in}}%
\pgfpathlineto{\pgfqpoint{3.717455in}{1.844180in}}%
\pgfpathlineto{\pgfqpoint{3.703635in}{1.850073in}}%
\pgfpathlineto{\pgfqpoint{3.711835in}{1.852066in}}%
\pgfpathlineto{\pgfqpoint{3.720025in}{1.854324in}}%
\pgfpathlineto{\pgfqpoint{3.728205in}{1.856840in}}%
\pgfpathlineto{\pgfqpoint{3.736376in}{1.859606in}}%
\pgfpathclose%
\pgfusepath{fill}%
\end{pgfscope}%
\begin{pgfscope}%
\pgfpathrectangle{\pgfqpoint{1.150000in}{0.150000in}}{\pgfqpoint{5.700000in}{5.700000in}}%
\pgfusepath{clip}%
\pgfsetbuttcap%
\pgfsetroundjoin%
\definecolor{currentfill}{rgb}{0.268510,0.009605,0.335427}%
\pgfsetfillcolor{currentfill}%
\pgfsetfillopacity{0.700000}%
\pgfsetlinewidth{0.000000pt}%
\definecolor{currentstroke}{rgb}{0.000000,0.000000,0.000000}%
\pgfsetstrokecolor{currentstroke}%
\pgfsetdash{}{0pt}%
\pgfpathmoveto{\pgfqpoint{4.308070in}{1.797894in}}%
\pgfpathlineto{\pgfqpoint{4.321992in}{1.794121in}}%
\pgfpathlineto{\pgfqpoint{4.335920in}{1.790373in}}%
\pgfpathlineto{\pgfqpoint{4.349856in}{1.786651in}}%
\pgfpathlineto{\pgfqpoint{4.363798in}{1.782953in}}%
\pgfpathlineto{\pgfqpoint{4.355892in}{1.775080in}}%
\pgfpathlineto{\pgfqpoint{4.347981in}{1.767323in}}%
\pgfpathlineto{\pgfqpoint{4.340065in}{1.759688in}}%
\pgfpathlineto{\pgfqpoint{4.332143in}{1.752180in}}%
\pgfpathlineto{\pgfqpoint{4.318187in}{1.756108in}}%
\pgfpathlineto{\pgfqpoint{4.304239in}{1.760061in}}%
\pgfpathlineto{\pgfqpoint{4.290297in}{1.764039in}}%
\pgfpathlineto{\pgfqpoint{4.276361in}{1.768042in}}%
\pgfpathlineto{\pgfqpoint{4.284297in}{1.775314in}}%
\pgfpathlineto{\pgfqpoint{4.292227in}{1.782718in}}%
\pgfpathlineto{\pgfqpoint{4.300151in}{1.790246in}}%
\pgfpathlineto{\pgfqpoint{4.308070in}{1.797894in}}%
\pgfpathclose%
\pgfusepath{fill}%
\end{pgfscope}%
\begin{pgfscope}%
\pgfpathrectangle{\pgfqpoint{1.150000in}{0.150000in}}{\pgfqpoint{5.700000in}{5.700000in}}%
\pgfusepath{clip}%
\pgfsetbuttcap%
\pgfsetroundjoin%
\definecolor{currentfill}{rgb}{0.282884,0.135920,0.453427}%
\pgfsetfillcolor{currentfill}%
\pgfsetfillopacity{0.700000}%
\pgfsetlinewidth{0.000000pt}%
\definecolor{currentstroke}{rgb}{0.000000,0.000000,0.000000}%
\pgfsetstrokecolor{currentstroke}%
\pgfsetdash{}{0pt}%
\pgfpathmoveto{\pgfqpoint{5.261789in}{2.030093in}}%
\pgfpathlineto{\pgfqpoint{5.275996in}{2.028984in}}%
\pgfpathlineto{\pgfqpoint{5.290211in}{2.027900in}}%
\pgfpathlineto{\pgfqpoint{5.304435in}{2.026841in}}%
\pgfpathlineto{\pgfqpoint{5.318668in}{2.025806in}}%
\pgfpathlineto{\pgfqpoint{5.311065in}{2.015497in}}%
\pgfpathlineto{\pgfqpoint{5.303456in}{2.005121in}}%
\pgfpathlineto{\pgfqpoint{5.295841in}{1.994679in}}%
\pgfpathlineto{\pgfqpoint{5.288219in}{1.984174in}}%
\pgfpathlineto{\pgfqpoint{5.273979in}{1.985320in}}%
\pgfpathlineto{\pgfqpoint{5.259747in}{1.986490in}}%
\pgfpathlineto{\pgfqpoint{5.245523in}{1.987685in}}%
\pgfpathlineto{\pgfqpoint{5.231308in}{1.988904in}}%
\pgfpathlineto{\pgfqpoint{5.238938in}{1.999293in}}%
\pgfpathlineto{\pgfqpoint{5.246561in}{2.009622in}}%
\pgfpathlineto{\pgfqpoint{5.254178in}{2.019890in}}%
\pgfpathlineto{\pgfqpoint{5.261789in}{2.030093in}}%
\pgfpathclose%
\pgfusepath{fill}%
\end{pgfscope}%
\begin{pgfscope}%
\pgfpathrectangle{\pgfqpoint{1.150000in}{0.150000in}}{\pgfqpoint{5.700000in}{5.700000in}}%
\pgfusepath{clip}%
\pgfsetbuttcap%
\pgfsetroundjoin%
\definecolor{currentfill}{rgb}{0.278826,0.175490,0.483397}%
\pgfsetfillcolor{currentfill}%
\pgfsetfillopacity{0.700000}%
\pgfsetlinewidth{0.000000pt}%
\definecolor{currentstroke}{rgb}{0.000000,0.000000,0.000000}%
\pgfsetstrokecolor{currentstroke}%
\pgfsetdash{}{0pt}%
\pgfpathmoveto{\pgfqpoint{3.141370in}{2.097938in}}%
\pgfpathlineto{\pgfqpoint{3.155082in}{2.090367in}}%
\pgfpathlineto{\pgfqpoint{3.168799in}{2.082828in}}%
\pgfpathlineto{\pgfqpoint{3.182520in}{2.075321in}}%
\pgfpathlineto{\pgfqpoint{3.196244in}{2.067847in}}%
\pgfpathlineto{\pgfqpoint{3.187708in}{2.071170in}}%
\pgfpathlineto{\pgfqpoint{3.179156in}{2.074870in}}%
\pgfpathlineto{\pgfqpoint{3.170587in}{2.078954in}}%
\pgfpathlineto{\pgfqpoint{3.162001in}{2.083432in}}%
\pgfpathlineto{\pgfqpoint{3.148243in}{2.091235in}}%
\pgfpathlineto{\pgfqpoint{3.134489in}{2.099070in}}%
\pgfpathlineto{\pgfqpoint{3.120739in}{2.106937in}}%
\pgfpathlineto{\pgfqpoint{3.106993in}{2.114837in}}%
\pgfpathlineto{\pgfqpoint{3.115613in}{2.110025in}}%
\pgfpathlineto{\pgfqpoint{3.124216in}{2.105610in}}%
\pgfpathlineto{\pgfqpoint{3.132801in}{2.101584in}}%
\pgfpathlineto{\pgfqpoint{3.141370in}{2.097938in}}%
\pgfpathclose%
\pgfusepath{fill}%
\end{pgfscope}%
\begin{pgfscope}%
\pgfpathrectangle{\pgfqpoint{1.150000in}{0.150000in}}{\pgfqpoint{5.700000in}{5.700000in}}%
\pgfusepath{clip}%
\pgfsetbuttcap%
\pgfsetroundjoin%
\definecolor{currentfill}{rgb}{0.277018,0.050344,0.375715}%
\pgfsetfillcolor{currentfill}%
\pgfsetfillopacity{0.700000}%
\pgfsetlinewidth{0.000000pt}%
\definecolor{currentstroke}{rgb}{0.000000,0.000000,0.000000}%
\pgfsetstrokecolor{currentstroke}%
\pgfsetdash{}{0pt}%
\pgfpathmoveto{\pgfqpoint{4.769027in}{1.867482in}}%
\pgfpathlineto{\pgfqpoint{4.783079in}{1.865095in}}%
\pgfpathlineto{\pgfqpoint{4.797138in}{1.862733in}}%
\pgfpathlineto{\pgfqpoint{4.811205in}{1.860395in}}%
\pgfpathlineto{\pgfqpoint{4.825280in}{1.858082in}}%
\pgfpathlineto{\pgfqpoint{4.817517in}{1.848016in}}%
\pgfpathlineto{\pgfqpoint{4.809749in}{1.837968in}}%
\pgfpathlineto{\pgfqpoint{4.801976in}{1.827941in}}%
\pgfpathlineto{\pgfqpoint{4.794198in}{1.817940in}}%
\pgfpathlineto{\pgfqpoint{4.780114in}{1.820431in}}%
\pgfpathlineto{\pgfqpoint{4.766038in}{1.822946in}}%
\pgfpathlineto{\pgfqpoint{4.751970in}{1.825486in}}%
\pgfpathlineto{\pgfqpoint{4.737909in}{1.828050in}}%
\pgfpathlineto{\pgfqpoint{4.745696in}{1.837868in}}%
\pgfpathlineto{\pgfqpoint{4.753478in}{1.847716in}}%
\pgfpathlineto{\pgfqpoint{4.761255in}{1.857588in}}%
\pgfpathlineto{\pgfqpoint{4.769027in}{1.867482in}}%
\pgfpathclose%
\pgfusepath{fill}%
\end{pgfscope}%
\begin{pgfscope}%
\pgfpathrectangle{\pgfqpoint{1.150000in}{0.150000in}}{\pgfqpoint{5.700000in}{5.700000in}}%
\pgfusepath{clip}%
\pgfsetbuttcap%
\pgfsetroundjoin%
\definecolor{currentfill}{rgb}{0.260571,0.246922,0.522828}%
\pgfsetfillcolor{currentfill}%
\pgfsetfillopacity{0.700000}%
\pgfsetlinewidth{0.000000pt}%
\definecolor{currentstroke}{rgb}{0.000000,0.000000,0.000000}%
\pgfsetstrokecolor{currentstroke}%
\pgfsetdash{}{0pt}%
\pgfpathmoveto{\pgfqpoint{2.887517in}{2.245827in}}%
\pgfpathlineto{\pgfqpoint{2.901210in}{2.237377in}}%
\pgfpathlineto{\pgfqpoint{2.914906in}{2.228964in}}%
\pgfpathlineto{\pgfqpoint{2.928605in}{2.220587in}}%
\pgfpathlineto{\pgfqpoint{2.942307in}{2.212245in}}%
\pgfpathlineto{\pgfqpoint{2.933559in}{2.218480in}}%
\pgfpathlineto{\pgfqpoint{2.924791in}{2.225143in}}%
\pgfpathlineto{\pgfqpoint{2.916001in}{2.232243in}}%
\pgfpathlineto{\pgfqpoint{2.907191in}{2.239791in}}%
\pgfpathlineto{\pgfqpoint{2.893450in}{2.248478in}}%
\pgfpathlineto{\pgfqpoint{2.879712in}{2.257200in}}%
\pgfpathlineto{\pgfqpoint{2.865978in}{2.265959in}}%
\pgfpathlineto{\pgfqpoint{2.852246in}{2.274754in}}%
\pgfpathlineto{\pgfqpoint{2.861096in}{2.266855in}}%
\pgfpathlineto{\pgfqpoint{2.869925in}{2.259408in}}%
\pgfpathlineto{\pgfqpoint{2.878731in}{2.252401in}}%
\pgfpathlineto{\pgfqpoint{2.887517in}{2.245827in}}%
\pgfpathclose%
\pgfusepath{fill}%
\end{pgfscope}%
\begin{pgfscope}%
\pgfpathrectangle{\pgfqpoint{1.150000in}{0.150000in}}{\pgfqpoint{5.700000in}{5.700000in}}%
\pgfusepath{clip}%
\pgfsetbuttcap%
\pgfsetroundjoin%
\definecolor{currentfill}{rgb}{0.283229,0.120777,0.440584}%
\pgfsetfillcolor{currentfill}%
\pgfsetfillopacity{0.700000}%
\pgfsetlinewidth{0.000000pt}%
\definecolor{currentstroke}{rgb}{0.000000,0.000000,0.000000}%
\pgfsetstrokecolor{currentstroke}%
\pgfsetdash{}{0pt}%
\pgfpathmoveto{\pgfqpoint{5.174535in}{1.994027in}}%
\pgfpathlineto{\pgfqpoint{5.188716in}{1.992709in}}%
\pgfpathlineto{\pgfqpoint{5.202905in}{1.991416in}}%
\pgfpathlineto{\pgfqpoint{5.217102in}{1.990148in}}%
\pgfpathlineto{\pgfqpoint{5.231308in}{1.988904in}}%
\pgfpathlineto{\pgfqpoint{5.223673in}{1.978458in}}%
\pgfpathlineto{\pgfqpoint{5.216031in}{1.967958in}}%
\pgfpathlineto{\pgfqpoint{5.208383in}{1.957407in}}%
\pgfpathlineto{\pgfqpoint{5.200729in}{1.946807in}}%
\pgfpathlineto{\pgfqpoint{5.186515in}{1.948175in}}%
\pgfpathlineto{\pgfqpoint{5.172310in}{1.949568in}}%
\pgfpathlineto{\pgfqpoint{5.158113in}{1.950986in}}%
\pgfpathlineto{\pgfqpoint{5.143925in}{1.952428in}}%
\pgfpathlineto{\pgfqpoint{5.151586in}{1.962898in}}%
\pgfpathlineto{\pgfqpoint{5.159242in}{1.973323in}}%
\pgfpathlineto{\pgfqpoint{5.166891in}{1.983701in}}%
\pgfpathlineto{\pgfqpoint{5.174535in}{1.994027in}}%
\pgfpathclose%
\pgfusepath{fill}%
\end{pgfscope}%
\begin{pgfscope}%
\pgfpathrectangle{\pgfqpoint{1.150000in}{0.150000in}}{\pgfqpoint{5.700000in}{5.700000in}}%
\pgfusepath{clip}%
\pgfsetbuttcap%
\pgfsetroundjoin%
\definecolor{currentfill}{rgb}{0.273006,0.204520,0.501721}%
\pgfsetfillcolor{currentfill}%
\pgfsetfillopacity{0.700000}%
\pgfsetlinewidth{0.000000pt}%
\definecolor{currentstroke}{rgb}{0.000000,0.000000,0.000000}%
\pgfsetstrokecolor{currentstroke}%
\pgfsetdash{}{0pt}%
\pgfpathmoveto{\pgfqpoint{5.667720in}{2.171639in}}%
\pgfpathlineto{\pgfqpoint{5.682071in}{2.171329in}}%
\pgfpathlineto{\pgfqpoint{5.696431in}{2.171044in}}%
\pgfpathlineto{\pgfqpoint{5.710800in}{2.170783in}}%
\pgfpathlineto{\pgfqpoint{5.703360in}{2.161566in}}%
\pgfpathlineto{\pgfqpoint{5.695911in}{2.152239in}}%
\pgfpathlineto{\pgfqpoint{5.688455in}{2.142803in}}%
\pgfpathlineto{\pgfqpoint{5.680990in}{2.133260in}}%
\pgfpathlineto{\pgfqpoint{5.666611in}{2.133576in}}%
\pgfpathlineto{\pgfqpoint{5.652241in}{2.133917in}}%
\pgfpathlineto{\pgfqpoint{5.637881in}{2.134283in}}%
\pgfpathlineto{\pgfqpoint{5.645353in}{2.143781in}}%
\pgfpathlineto{\pgfqpoint{5.652817in}{2.153173in}}%
\pgfpathlineto{\pgfqpoint{5.660273in}{2.162460in}}%
\pgfpathlineto{\pgfqpoint{5.667720in}{2.171639in}}%
\pgfpathclose%
\pgfusepath{fill}%
\end{pgfscope}%
\begin{pgfscope}%
\pgfpathrectangle{\pgfqpoint{1.150000in}{0.150000in}}{\pgfqpoint{5.700000in}{5.700000in}}%
\pgfusepath{clip}%
\pgfsetbuttcap%
\pgfsetroundjoin%
\definecolor{currentfill}{rgb}{0.221989,0.339161,0.548752}%
\pgfsetfillcolor{currentfill}%
\pgfsetfillopacity{0.700000}%
\pgfsetlinewidth{0.000000pt}%
\definecolor{currentstroke}{rgb}{0.000000,0.000000,0.000000}%
\pgfsetstrokecolor{currentstroke}%
\pgfsetdash{}{0pt}%
\pgfpathmoveto{\pgfqpoint{2.578160in}{2.458715in}}%
\pgfpathlineto{\pgfqpoint{2.591842in}{2.449132in}}%
\pgfpathlineto{\pgfqpoint{2.605526in}{2.439592in}}%
\pgfpathlineto{\pgfqpoint{2.619213in}{2.430094in}}%
\pgfpathlineto{\pgfqpoint{2.632901in}{2.420637in}}%
\pgfpathlineto{\pgfqpoint{2.623859in}{2.430424in}}%
\pgfpathlineto{\pgfqpoint{2.614791in}{2.440698in}}%
\pgfpathlineto{\pgfqpoint{2.605696in}{2.451470in}}%
\pgfpathlineto{\pgfqpoint{2.596574in}{2.462750in}}%
\pgfpathlineto{\pgfqpoint{2.582841in}{2.472570in}}%
\pgfpathlineto{\pgfqpoint{2.569109in}{2.482433in}}%
\pgfpathlineto{\pgfqpoint{2.555380in}{2.492339in}}%
\pgfpathlineto{\pgfqpoint{2.541652in}{2.502287in}}%
\pgfpathlineto{\pgfqpoint{2.550821in}{2.490636in}}%
\pgfpathlineto{\pgfqpoint{2.559961in}{2.479498in}}%
\pgfpathlineto{\pgfqpoint{2.569074in}{2.468861in}}%
\pgfpathlineto{\pgfqpoint{2.578160in}{2.458715in}}%
\pgfpathclose%
\pgfusepath{fill}%
\end{pgfscope}%
\begin{pgfscope}%
\pgfpathrectangle{\pgfqpoint{1.150000in}{0.150000in}}{\pgfqpoint{5.700000in}{5.700000in}}%
\pgfusepath{clip}%
\pgfsetbuttcap%
\pgfsetroundjoin%
\definecolor{currentfill}{rgb}{0.269944,0.014625,0.341379}%
\pgfsetfillcolor{currentfill}%
\pgfsetfillopacity{0.700000}%
\pgfsetlinewidth{0.000000pt}%
\definecolor{currentstroke}{rgb}{0.000000,0.000000,0.000000}%
\pgfsetstrokecolor{currentstroke}%
\pgfsetdash{}{0pt}%
\pgfpathmoveto{\pgfqpoint{4.451156in}{1.801825in}}%
\pgfpathlineto{\pgfqpoint{4.465120in}{1.798469in}}%
\pgfpathlineto{\pgfqpoint{4.479092in}{1.795139in}}%
\pgfpathlineto{\pgfqpoint{4.493070in}{1.791834in}}%
\pgfpathlineto{\pgfqpoint{4.507056in}{1.788553in}}%
\pgfpathlineto{\pgfqpoint{4.499195in}{1.779846in}}%
\pgfpathlineto{\pgfqpoint{4.491328in}{1.771227in}}%
\pgfpathlineto{\pgfqpoint{4.483457in}{1.762701in}}%
\pgfpathlineto{\pgfqpoint{4.475581in}{1.754273in}}%
\pgfpathlineto{\pgfqpoint{4.461584in}{1.757771in}}%
\pgfpathlineto{\pgfqpoint{4.447594in}{1.761294in}}%
\pgfpathlineto{\pgfqpoint{4.433611in}{1.764841in}}%
\pgfpathlineto{\pgfqpoint{4.419635in}{1.768414in}}%
\pgfpathlineto{\pgfqpoint{4.427523in}{1.776619in}}%
\pgfpathlineto{\pgfqpoint{4.435406in}{1.784927in}}%
\pgfpathlineto{\pgfqpoint{4.443283in}{1.793330in}}%
\pgfpathlineto{\pgfqpoint{4.451156in}{1.801825in}}%
\pgfpathclose%
\pgfusepath{fill}%
\end{pgfscope}%
\begin{pgfscope}%
\pgfpathrectangle{\pgfqpoint{1.150000in}{0.150000in}}{\pgfqpoint{5.700000in}{5.700000in}}%
\pgfusepath{clip}%
\pgfsetbuttcap%
\pgfsetroundjoin%
\definecolor{currentfill}{rgb}{0.283197,0.115680,0.436115}%
\pgfsetfillcolor{currentfill}%
\pgfsetfillopacity{0.700000}%
\pgfsetlinewidth{0.000000pt}%
\definecolor{currentstroke}{rgb}{0.000000,0.000000,0.000000}%
\pgfsetstrokecolor{currentstroke}%
\pgfsetdash{}{0pt}%
\pgfpathmoveto{\pgfqpoint{3.394876in}{1.974634in}}%
\pgfpathlineto{\pgfqpoint{3.408624in}{1.967873in}}%
\pgfpathlineto{\pgfqpoint{3.422378in}{1.961142in}}%
\pgfpathlineto{\pgfqpoint{3.436135in}{1.954440in}}%
\pgfpathlineto{\pgfqpoint{3.449898in}{1.947768in}}%
\pgfpathlineto{\pgfqpoint{3.441540in}{1.948445in}}%
\pgfpathlineto{\pgfqpoint{3.433169in}{1.949448in}}%
\pgfpathlineto{\pgfqpoint{3.424785in}{1.950787in}}%
\pgfpathlineto{\pgfqpoint{3.416387in}{1.952470in}}%
\pgfpathlineto{\pgfqpoint{3.402596in}{1.959455in}}%
\pgfpathlineto{\pgfqpoint{3.388810in}{1.966469in}}%
\pgfpathlineto{\pgfqpoint{3.375028in}{1.973513in}}%
\pgfpathlineto{\pgfqpoint{3.361250in}{1.980587in}}%
\pgfpathlineto{\pgfqpoint{3.369678in}{1.978586in}}%
\pgfpathlineto{\pgfqpoint{3.378091in}{1.976933in}}%
\pgfpathlineto{\pgfqpoint{3.386490in}{1.975618in}}%
\pgfpathlineto{\pgfqpoint{3.394876in}{1.974634in}}%
\pgfpathclose%
\pgfusepath{fill}%
\end{pgfscope}%
\begin{pgfscope}%
\pgfpathrectangle{\pgfqpoint{1.150000in}{0.150000in}}{\pgfqpoint{5.700000in}{5.700000in}}%
\pgfusepath{clip}%
\pgfsetbuttcap%
\pgfsetroundjoin%
\definecolor{currentfill}{rgb}{0.282656,0.100196,0.422160}%
\pgfsetfillcolor{currentfill}%
\pgfsetfillopacity{0.700000}%
\pgfsetlinewidth{0.000000pt}%
\definecolor{currentstroke}{rgb}{0.000000,0.000000,0.000000}%
\pgfsetstrokecolor{currentstroke}%
\pgfsetdash{}{0pt}%
\pgfpathmoveto{\pgfqpoint{5.087255in}{1.958441in}}%
\pgfpathlineto{\pgfqpoint{5.101410in}{1.956901in}}%
\pgfpathlineto{\pgfqpoint{5.115573in}{1.955385in}}%
\pgfpathlineto{\pgfqpoint{5.129745in}{1.953894in}}%
\pgfpathlineto{\pgfqpoint{5.143925in}{1.952428in}}%
\pgfpathlineto{\pgfqpoint{5.136257in}{1.941915in}}%
\pgfpathlineto{\pgfqpoint{5.128584in}{1.931362in}}%
\pgfpathlineto{\pgfqpoint{5.120906in}{1.920774in}}%
\pgfpathlineto{\pgfqpoint{5.113221in}{1.910153in}}%
\pgfpathlineto{\pgfqpoint{5.099034in}{1.911757in}}%
\pgfpathlineto{\pgfqpoint{5.084854in}{1.913386in}}%
\pgfpathlineto{\pgfqpoint{5.070683in}{1.915040in}}%
\pgfpathlineto{\pgfqpoint{5.056520in}{1.916718in}}%
\pgfpathlineto{\pgfqpoint{5.064212in}{1.927196in}}%
\pgfpathlineto{\pgfqpoint{5.071899in}{1.937645in}}%
\pgfpathlineto{\pgfqpoint{5.079580in}{1.948061in}}%
\pgfpathlineto{\pgfqpoint{5.087255in}{1.958441in}}%
\pgfpathclose%
\pgfusepath{fill}%
\end{pgfscope}%
\begin{pgfscope}%
\pgfpathrectangle{\pgfqpoint{1.150000in}{0.150000in}}{\pgfqpoint{5.700000in}{5.700000in}}%
\pgfusepath{clip}%
\pgfsetbuttcap%
\pgfsetroundjoin%
\definecolor{currentfill}{rgb}{0.280894,0.078907,0.402329}%
\pgfsetfillcolor{currentfill}%
\pgfsetfillopacity{0.700000}%
\pgfsetlinewidth{0.000000pt}%
\definecolor{currentstroke}{rgb}{0.000000,0.000000,0.000000}%
\pgfsetstrokecolor{currentstroke}%
\pgfsetdash{}{0pt}%
\pgfpathmoveto{\pgfqpoint{3.593261in}{1.898215in}}%
\pgfpathlineto{\pgfqpoint{3.607040in}{1.892100in}}%
\pgfpathlineto{\pgfqpoint{3.620824in}{1.886012in}}%
\pgfpathlineto{\pgfqpoint{3.634614in}{1.879953in}}%
\pgfpathlineto{\pgfqpoint{3.648408in}{1.873922in}}%
\pgfpathlineto{\pgfqpoint{3.640174in}{1.872494in}}%
\pgfpathlineto{\pgfqpoint{3.631930in}{1.871351in}}%
\pgfpathlineto{\pgfqpoint{3.623675in}{1.870499in}}%
\pgfpathlineto{\pgfqpoint{3.615408in}{1.869947in}}%
\pgfpathlineto{\pgfqpoint{3.601589in}{1.876276in}}%
\pgfpathlineto{\pgfqpoint{3.587775in}{1.882634in}}%
\pgfpathlineto{\pgfqpoint{3.573966in}{1.889019in}}%
\pgfpathlineto{\pgfqpoint{3.560162in}{1.895432in}}%
\pgfpathlineto{\pgfqpoint{3.568454in}{1.895681in}}%
\pgfpathlineto{\pgfqpoint{3.576734in}{1.896233in}}%
\pgfpathlineto{\pgfqpoint{3.585003in}{1.897080in}}%
\pgfpathlineto{\pgfqpoint{3.593261in}{1.898215in}}%
\pgfpathclose%
\pgfusepath{fill}%
\end{pgfscope}%
\begin{pgfscope}%
\pgfpathrectangle{\pgfqpoint{1.150000in}{0.150000in}}{\pgfqpoint{5.700000in}{5.700000in}}%
\pgfusepath{clip}%
\pgfsetbuttcap%
\pgfsetroundjoin%
\definecolor{currentfill}{rgb}{0.274952,0.037752,0.364543}%
\pgfsetfillcolor{currentfill}%
\pgfsetfillopacity{0.700000}%
\pgfsetlinewidth{0.000000pt}%
\definecolor{currentstroke}{rgb}{0.000000,0.000000,0.000000}%
\pgfsetstrokecolor{currentstroke}%
\pgfsetdash{}{0pt}%
\pgfpathmoveto{\pgfqpoint{4.681742in}{1.838555in}}%
\pgfpathlineto{\pgfqpoint{4.695772in}{1.835892in}}%
\pgfpathlineto{\pgfqpoint{4.709810in}{1.833253in}}%
\pgfpathlineto{\pgfqpoint{4.723856in}{1.830639in}}%
\pgfpathlineto{\pgfqpoint{4.737909in}{1.828050in}}%
\pgfpathlineto{\pgfqpoint{4.730117in}{1.818266in}}%
\pgfpathlineto{\pgfqpoint{4.722321in}{1.808521in}}%
\pgfpathlineto{\pgfqpoint{4.714519in}{1.798818in}}%
\pgfpathlineto{\pgfqpoint{4.706713in}{1.789162in}}%
\pgfpathlineto{\pgfqpoint{4.692651in}{1.791943in}}%
\pgfpathlineto{\pgfqpoint{4.678596in}{1.794747in}}%
\pgfpathlineto{\pgfqpoint{4.664548in}{1.797577in}}%
\pgfpathlineto{\pgfqpoint{4.650508in}{1.800431in}}%
\pgfpathlineto{\pgfqpoint{4.658324in}{1.809890in}}%
\pgfpathlineto{\pgfqpoint{4.666135in}{1.819400in}}%
\pgfpathlineto{\pgfqpoint{4.673941in}{1.828956in}}%
\pgfpathlineto{\pgfqpoint{4.681742in}{1.838555in}}%
\pgfpathclose%
\pgfusepath{fill}%
\end{pgfscope}%
\begin{pgfscope}%
\pgfpathrectangle{\pgfqpoint{1.150000in}{0.150000in}}{\pgfqpoint{5.700000in}{5.700000in}}%
\pgfusepath{clip}%
\pgfsetbuttcap%
\pgfsetroundjoin%
\definecolor{currentfill}{rgb}{0.269944,0.014625,0.341379}%
\pgfsetfillcolor{currentfill}%
\pgfsetfillopacity{0.700000}%
\pgfsetlinewidth{0.000000pt}%
\definecolor{currentstroke}{rgb}{0.000000,0.000000,0.000000}%
\pgfsetstrokecolor{currentstroke}%
\pgfsetdash{}{0pt}%
\pgfpathmoveto{\pgfqpoint{4.077658in}{1.793322in}}%
\pgfpathlineto{\pgfqpoint{4.091533in}{1.788770in}}%
\pgfpathlineto{\pgfqpoint{4.105413in}{1.784243in}}%
\pgfpathlineto{\pgfqpoint{4.119300in}{1.779742in}}%
\pgfpathlineto{\pgfqpoint{4.133193in}{1.775267in}}%
\pgfpathlineto{\pgfqpoint{4.125197in}{1.769243in}}%
\pgfpathlineto{\pgfqpoint{4.117195in}{1.763395in}}%
\pgfpathlineto{\pgfqpoint{4.109187in}{1.757727in}}%
\pgfpathlineto{\pgfqpoint{4.101171in}{1.752247in}}%
\pgfpathlineto{\pgfqpoint{4.087262in}{1.756979in}}%
\pgfpathlineto{\pgfqpoint{4.073359in}{1.761737in}}%
\pgfpathlineto{\pgfqpoint{4.059462in}{1.766520in}}%
\pgfpathlineto{\pgfqpoint{4.045570in}{1.771330in}}%
\pgfpathlineto{\pgfqpoint{4.053603in}{1.776548in}}%
\pgfpathlineto{\pgfqpoint{4.061628in}{1.781957in}}%
\pgfpathlineto{\pgfqpoint{4.069647in}{1.787550in}}%
\pgfpathlineto{\pgfqpoint{4.077658in}{1.793322in}}%
\pgfpathclose%
\pgfusepath{fill}%
\end{pgfscope}%
\begin{pgfscope}%
\pgfpathrectangle{\pgfqpoint{1.150000in}{0.150000in}}{\pgfqpoint{5.700000in}{5.700000in}}%
\pgfusepath{clip}%
\pgfsetbuttcap%
\pgfsetroundjoin%
\definecolor{currentfill}{rgb}{0.272594,0.025563,0.353093}%
\pgfsetfillcolor{currentfill}%
\pgfsetfillopacity{0.700000}%
\pgfsetlinewidth{0.000000pt}%
\definecolor{currentstroke}{rgb}{0.000000,0.000000,0.000000}%
\pgfsetstrokecolor{currentstroke}%
\pgfsetdash{}{0pt}%
\pgfpathmoveto{\pgfqpoint{3.934654in}{1.810738in}}%
\pgfpathlineto{\pgfqpoint{3.948498in}{1.805720in}}%
\pgfpathlineto{\pgfqpoint{3.962348in}{1.800729in}}%
\pgfpathlineto{\pgfqpoint{3.976204in}{1.795764in}}%
\pgfpathlineto{\pgfqpoint{3.990065in}{1.790825in}}%
\pgfpathlineto{\pgfqpoint{3.982008in}{1.786070in}}%
\pgfpathlineto{\pgfqpoint{3.973943in}{1.781523in}}%
\pgfpathlineto{\pgfqpoint{3.965870in}{1.777191in}}%
\pgfpathlineto{\pgfqpoint{3.957790in}{1.773081in}}%
\pgfpathlineto{\pgfqpoint{3.943910in}{1.778290in}}%
\pgfpathlineto{\pgfqpoint{3.930035in}{1.783525in}}%
\pgfpathlineto{\pgfqpoint{3.916166in}{1.788787in}}%
\pgfpathlineto{\pgfqpoint{3.902303in}{1.794074in}}%
\pgfpathlineto{\pgfqpoint{3.910403in}{1.797909in}}%
\pgfpathlineto{\pgfqpoint{3.918495in}{1.801969in}}%
\pgfpathlineto{\pgfqpoint{3.926578in}{1.806248in}}%
\pgfpathlineto{\pgfqpoint{3.934654in}{1.810738in}}%
\pgfpathclose%
\pgfusepath{fill}%
\end{pgfscope}%
\begin{pgfscope}%
\pgfpathrectangle{\pgfqpoint{1.150000in}{0.150000in}}{\pgfqpoint{5.700000in}{5.700000in}}%
\pgfusepath{clip}%
\pgfsetbuttcap%
\pgfsetroundjoin%
\definecolor{currentfill}{rgb}{0.276194,0.190074,0.493001}%
\pgfsetfillcolor{currentfill}%
\pgfsetfillopacity{0.700000}%
\pgfsetlinewidth{0.000000pt}%
\definecolor{currentstroke}{rgb}{0.000000,0.000000,0.000000}%
\pgfsetstrokecolor{currentstroke}%
\pgfsetdash{}{0pt}%
\pgfpathmoveto{\pgfqpoint{5.580532in}{2.135993in}}%
\pgfpathlineto{\pgfqpoint{5.594855in}{2.135529in}}%
\pgfpathlineto{\pgfqpoint{5.609188in}{2.135089in}}%
\pgfpathlineto{\pgfqpoint{5.623530in}{2.134674in}}%
\pgfpathlineto{\pgfqpoint{5.637881in}{2.134283in}}%
\pgfpathlineto{\pgfqpoint{5.630400in}{2.124681in}}%
\pgfpathlineto{\pgfqpoint{5.622912in}{2.114977in}}%
\pgfpathlineto{\pgfqpoint{5.615415in}{2.105170in}}%
\pgfpathlineto{\pgfqpoint{5.607911in}{2.095263in}}%
\pgfpathlineto{\pgfqpoint{5.593551in}{2.095724in}}%
\pgfpathlineto{\pgfqpoint{5.579200in}{2.096208in}}%
\pgfpathlineto{\pgfqpoint{5.564858in}{2.096718in}}%
\pgfpathlineto{\pgfqpoint{5.550526in}{2.097252in}}%
\pgfpathlineto{\pgfqpoint{5.558039in}{2.107084in}}%
\pgfpathlineto{\pgfqpoint{5.565545in}{2.116819in}}%
\pgfpathlineto{\pgfqpoint{5.573042in}{2.126456in}}%
\pgfpathlineto{\pgfqpoint{5.580532in}{2.135993in}}%
\pgfpathclose%
\pgfusepath{fill}%
\end{pgfscope}%
\begin{pgfscope}%
\pgfpathrectangle{\pgfqpoint{1.150000in}{0.150000in}}{\pgfqpoint{5.700000in}{5.700000in}}%
\pgfusepath{clip}%
\pgfsetbuttcap%
\pgfsetroundjoin%
\definecolor{currentfill}{rgb}{0.268510,0.009605,0.335427}%
\pgfsetfillcolor{currentfill}%
\pgfsetfillopacity{0.700000}%
\pgfsetlinewidth{0.000000pt}%
\definecolor{currentstroke}{rgb}{0.000000,0.000000,0.000000}%
\pgfsetstrokecolor{currentstroke}%
\pgfsetdash{}{0pt}%
\pgfpathmoveto{\pgfqpoint{4.220685in}{1.784307in}}%
\pgfpathlineto{\pgfqpoint{4.234594in}{1.780203in}}%
\pgfpathlineto{\pgfqpoint{4.248510in}{1.776124in}}%
\pgfpathlineto{\pgfqpoint{4.262432in}{1.772070in}}%
\pgfpathlineto{\pgfqpoint{4.276361in}{1.768042in}}%
\pgfpathlineto{\pgfqpoint{4.268420in}{1.760907in}}%
\pgfpathlineto{\pgfqpoint{4.260473in}{1.753914in}}%
\pgfpathlineto{\pgfqpoint{4.252520in}{1.747070in}}%
\pgfpathlineto{\pgfqpoint{4.244561in}{1.740382in}}%
\pgfpathlineto{\pgfqpoint{4.230618in}{1.744654in}}%
\pgfpathlineto{\pgfqpoint{4.216681in}{1.748951in}}%
\pgfpathlineto{\pgfqpoint{4.202751in}{1.753273in}}%
\pgfpathlineto{\pgfqpoint{4.188827in}{1.757621in}}%
\pgfpathlineto{\pgfqpoint{4.196800in}{1.764061in}}%
\pgfpathlineto{\pgfqpoint{4.204768in}{1.770660in}}%
\pgfpathlineto{\pgfqpoint{4.212729in}{1.777410in}}%
\pgfpathlineto{\pgfqpoint{4.220685in}{1.784307in}}%
\pgfpathclose%
\pgfusepath{fill}%
\end{pgfscope}%
\begin{pgfscope}%
\pgfpathrectangle{\pgfqpoint{1.150000in}{0.150000in}}{\pgfqpoint{5.700000in}{5.700000in}}%
\pgfusepath{clip}%
\pgfsetbuttcap%
\pgfsetroundjoin%
\definecolor{currentfill}{rgb}{0.281446,0.084320,0.407414}%
\pgfsetfillcolor{currentfill}%
\pgfsetfillopacity{0.700000}%
\pgfsetlinewidth{0.000000pt}%
\definecolor{currentstroke}{rgb}{0.000000,0.000000,0.000000}%
\pgfsetstrokecolor{currentstroke}%
\pgfsetdash{}{0pt}%
\pgfpathmoveto{\pgfqpoint{4.999952in}{1.923675in}}%
\pgfpathlineto{\pgfqpoint{5.014081in}{1.921899in}}%
\pgfpathlineto{\pgfqpoint{5.028220in}{1.920147in}}%
\pgfpathlineto{\pgfqpoint{5.042366in}{1.918420in}}%
\pgfpathlineto{\pgfqpoint{5.056520in}{1.916718in}}%
\pgfpathlineto{\pgfqpoint{5.048823in}{1.906213in}}%
\pgfpathlineto{\pgfqpoint{5.041120in}{1.895686in}}%
\pgfpathlineto{\pgfqpoint{5.033412in}{1.885139in}}%
\pgfpathlineto{\pgfqpoint{5.025698in}{1.874576in}}%
\pgfpathlineto{\pgfqpoint{5.011536in}{1.876430in}}%
\pgfpathlineto{\pgfqpoint{4.997382in}{1.878308in}}%
\pgfpathlineto{\pgfqpoint{4.983236in}{1.880211in}}%
\pgfpathlineto{\pgfqpoint{4.969098in}{1.882138in}}%
\pgfpathlineto{\pgfqpoint{4.976819in}{1.892545in}}%
\pgfpathlineto{\pgfqpoint{4.984535in}{1.902939in}}%
\pgfpathlineto{\pgfqpoint{4.992246in}{1.913317in}}%
\pgfpathlineto{\pgfqpoint{4.999952in}{1.923675in}}%
\pgfpathclose%
\pgfusepath{fill}%
\end{pgfscope}%
\begin{pgfscope}%
\pgfpathrectangle{\pgfqpoint{1.150000in}{0.150000in}}{\pgfqpoint{5.700000in}{5.700000in}}%
\pgfusepath{clip}%
\pgfsetbuttcap%
\pgfsetroundjoin%
\definecolor{currentfill}{rgb}{0.280255,0.165693,0.476498}%
\pgfsetfillcolor{currentfill}%
\pgfsetfillopacity{0.700000}%
\pgfsetlinewidth{0.000000pt}%
\definecolor{currentstroke}{rgb}{0.000000,0.000000,0.000000}%
\pgfsetstrokecolor{currentstroke}%
\pgfsetdash{}{0pt}%
\pgfpathmoveto{\pgfqpoint{3.196244in}{2.067847in}}%
\pgfpathlineto{\pgfqpoint{3.209972in}{2.060404in}}%
\pgfpathlineto{\pgfqpoint{3.223704in}{2.052993in}}%
\pgfpathlineto{\pgfqpoint{3.237441in}{2.045613in}}%
\pgfpathlineto{\pgfqpoint{3.251181in}{2.038265in}}%
\pgfpathlineto{\pgfqpoint{3.242677in}{2.041266in}}%
\pgfpathlineto{\pgfqpoint{3.234157in}{2.044639in}}%
\pgfpathlineto{\pgfqpoint{3.225621in}{2.048394in}}%
\pgfpathlineto{\pgfqpoint{3.217069in}{2.052538in}}%
\pgfpathlineto{\pgfqpoint{3.203296in}{2.060214in}}%
\pgfpathlineto{\pgfqpoint{3.189527in}{2.067922in}}%
\pgfpathlineto{\pgfqpoint{3.175762in}{2.075661in}}%
\pgfpathlineto{\pgfqpoint{3.162001in}{2.083432in}}%
\pgfpathlineto{\pgfqpoint{3.170587in}{2.078954in}}%
\pgfpathlineto{\pgfqpoint{3.179156in}{2.074870in}}%
\pgfpathlineto{\pgfqpoint{3.187708in}{2.071170in}}%
\pgfpathlineto{\pgfqpoint{3.196244in}{2.067847in}}%
\pgfpathclose%
\pgfusepath{fill}%
\end{pgfscope}%
\begin{pgfscope}%
\pgfpathrectangle{\pgfqpoint{1.150000in}{0.150000in}}{\pgfqpoint{5.700000in}{5.700000in}}%
\pgfusepath{clip}%
\pgfsetbuttcap%
\pgfsetroundjoin%
\definecolor{currentfill}{rgb}{0.263663,0.237631,0.518762}%
\pgfsetfillcolor{currentfill}%
\pgfsetfillopacity{0.700000}%
\pgfsetlinewidth{0.000000pt}%
\definecolor{currentstroke}{rgb}{0.000000,0.000000,0.000000}%
\pgfsetstrokecolor{currentstroke}%
\pgfsetdash{}{0pt}%
\pgfpathmoveto{\pgfqpoint{2.942307in}{2.212245in}}%
\pgfpathlineto{\pgfqpoint{2.956012in}{2.203939in}}%
\pgfpathlineto{\pgfqpoint{2.969721in}{2.195668in}}%
\pgfpathlineto{\pgfqpoint{2.983433in}{2.187431in}}%
\pgfpathlineto{\pgfqpoint{2.997148in}{2.179230in}}%
\pgfpathlineto{\pgfqpoint{2.988437in}{2.185126in}}%
\pgfpathlineto{\pgfqpoint{2.979707in}{2.191446in}}%
\pgfpathlineto{\pgfqpoint{2.970956in}{2.198199in}}%
\pgfpathlineto{\pgfqpoint{2.962184in}{2.205395in}}%
\pgfpathlineto{\pgfqpoint{2.948431in}{2.213942in}}%
\pgfpathlineto{\pgfqpoint{2.934681in}{2.222523in}}%
\pgfpathlineto{\pgfqpoint{2.920934in}{2.231139in}}%
\pgfpathlineto{\pgfqpoint{2.907191in}{2.239791in}}%
\pgfpathlineto{\pgfqpoint{2.916001in}{2.232243in}}%
\pgfpathlineto{\pgfqpoint{2.924791in}{2.225143in}}%
\pgfpathlineto{\pgfqpoint{2.933559in}{2.218480in}}%
\pgfpathlineto{\pgfqpoint{2.942307in}{2.212245in}}%
\pgfpathclose%
\pgfusepath{fill}%
\end{pgfscope}%
\begin{pgfscope}%
\pgfpathrectangle{\pgfqpoint{1.150000in}{0.150000in}}{\pgfqpoint{5.700000in}{5.700000in}}%
\pgfusepath{clip}%
\pgfsetbuttcap%
\pgfsetroundjoin%
\definecolor{currentfill}{rgb}{0.276022,0.044167,0.370164}%
\pgfsetfillcolor{currentfill}%
\pgfsetfillopacity{0.700000}%
\pgfsetlinewidth{0.000000pt}%
\definecolor{currentstroke}{rgb}{0.000000,0.000000,0.000000}%
\pgfsetstrokecolor{currentstroke}%
\pgfsetdash{}{0pt}%
\pgfpathmoveto{\pgfqpoint{3.791598in}{1.837332in}}%
\pgfpathlineto{\pgfqpoint{3.805417in}{1.831831in}}%
\pgfpathlineto{\pgfqpoint{3.819241in}{1.826357in}}%
\pgfpathlineto{\pgfqpoint{3.833071in}{1.820910in}}%
\pgfpathlineto{\pgfqpoint{3.846906in}{1.815490in}}%
\pgfpathlineto{\pgfqpoint{3.838778in}{1.812166in}}%
\pgfpathlineto{\pgfqpoint{3.830641in}{1.809086in}}%
\pgfpathlineto{\pgfqpoint{3.822495in}{1.806257in}}%
\pgfpathlineto{\pgfqpoint{3.814340in}{1.803686in}}%
\pgfpathlineto{\pgfqpoint{3.800483in}{1.809390in}}%
\pgfpathlineto{\pgfqpoint{3.786632in}{1.815121in}}%
\pgfpathlineto{\pgfqpoint{3.772786in}{1.820879in}}%
\pgfpathlineto{\pgfqpoint{3.758945in}{1.826664in}}%
\pgfpathlineto{\pgfqpoint{3.767123in}{1.828946in}}%
\pgfpathlineto{\pgfqpoint{3.775290in}{1.831489in}}%
\pgfpathlineto{\pgfqpoint{3.783449in}{1.834287in}}%
\pgfpathlineto{\pgfqpoint{3.791598in}{1.837332in}}%
\pgfpathclose%
\pgfusepath{fill}%
\end{pgfscope}%
\begin{pgfscope}%
\pgfpathrectangle{\pgfqpoint{1.150000in}{0.150000in}}{\pgfqpoint{5.700000in}{5.700000in}}%
\pgfusepath{clip}%
\pgfsetbuttcap%
\pgfsetroundjoin%
\definecolor{currentfill}{rgb}{0.227802,0.326594,0.546532}%
\pgfsetfillcolor{currentfill}%
\pgfsetfillopacity{0.700000}%
\pgfsetlinewidth{0.000000pt}%
\definecolor{currentstroke}{rgb}{0.000000,0.000000,0.000000}%
\pgfsetstrokecolor{currentstroke}%
\pgfsetdash{}{0pt}%
\pgfpathmoveto{\pgfqpoint{2.632901in}{2.420637in}}%
\pgfpathlineto{\pgfqpoint{2.646592in}{2.411223in}}%
\pgfpathlineto{\pgfqpoint{2.660284in}{2.401849in}}%
\pgfpathlineto{\pgfqpoint{2.673980in}{2.392516in}}%
\pgfpathlineto{\pgfqpoint{2.687677in}{2.383224in}}%
\pgfpathlineto{\pgfqpoint{2.678678in}{2.392653in}}%
\pgfpathlineto{\pgfqpoint{2.669654in}{2.402565in}}%
\pgfpathlineto{\pgfqpoint{2.660605in}{2.412970in}}%
\pgfpathlineto{\pgfqpoint{2.651528in}{2.423880in}}%
\pgfpathlineto{\pgfqpoint{2.637787in}{2.433536in}}%
\pgfpathlineto{\pgfqpoint{2.624047in}{2.443233in}}%
\pgfpathlineto{\pgfqpoint{2.610310in}{2.452970in}}%
\pgfpathlineto{\pgfqpoint{2.596574in}{2.462750in}}%
\pgfpathlineto{\pgfqpoint{2.605696in}{2.451470in}}%
\pgfpathlineto{\pgfqpoint{2.614791in}{2.440698in}}%
\pgfpathlineto{\pgfqpoint{2.623859in}{2.430424in}}%
\pgfpathlineto{\pgfqpoint{2.632901in}{2.420637in}}%
\pgfpathclose%
\pgfusepath{fill}%
\end{pgfscope}%
\begin{pgfscope}%
\pgfpathrectangle{\pgfqpoint{1.150000in}{0.150000in}}{\pgfqpoint{5.700000in}{5.700000in}}%
\pgfusepath{clip}%
\pgfsetbuttcap%
\pgfsetroundjoin%
\definecolor{currentfill}{rgb}{0.278826,0.175490,0.483397}%
\pgfsetfillcolor{currentfill}%
\pgfsetfillopacity{0.700000}%
\pgfsetlinewidth{0.000000pt}%
\definecolor{currentstroke}{rgb}{0.000000,0.000000,0.000000}%
\pgfsetstrokecolor{currentstroke}%
\pgfsetdash{}{0pt}%
\pgfpathmoveto{\pgfqpoint{5.493288in}{2.099635in}}%
\pgfpathlineto{\pgfqpoint{5.507583in}{2.099002in}}%
\pgfpathlineto{\pgfqpoint{5.521889in}{2.098394in}}%
\pgfpathlineto{\pgfqpoint{5.536203in}{2.097811in}}%
\pgfpathlineto{\pgfqpoint{5.550526in}{2.097252in}}%
\pgfpathlineto{\pgfqpoint{5.543005in}{2.087325in}}%
\pgfpathlineto{\pgfqpoint{5.535477in}{2.077304in}}%
\pgfpathlineto{\pgfqpoint{5.527941in}{2.067191in}}%
\pgfpathlineto{\pgfqpoint{5.520398in}{2.056987in}}%
\pgfpathlineto{\pgfqpoint{5.506066in}{2.057629in}}%
\pgfpathlineto{\pgfqpoint{5.491743in}{2.058296in}}%
\pgfpathlineto{\pgfqpoint{5.477430in}{2.058988in}}%
\pgfpathlineto{\pgfqpoint{5.463125in}{2.059704in}}%
\pgfpathlineto{\pgfqpoint{5.470677in}{2.069819in}}%
\pgfpathlineto{\pgfqpoint{5.478221in}{2.079847in}}%
\pgfpathlineto{\pgfqpoint{5.485758in}{2.089786in}}%
\pgfpathlineto{\pgfqpoint{5.493288in}{2.099635in}}%
\pgfpathclose%
\pgfusepath{fill}%
\end{pgfscope}%
\begin{pgfscope}%
\pgfpathrectangle{\pgfqpoint{1.150000in}{0.150000in}}{\pgfqpoint{5.700000in}{5.700000in}}%
\pgfusepath{clip}%
\pgfsetbuttcap%
\pgfsetroundjoin%
\definecolor{currentfill}{rgb}{0.168126,0.459988,0.558082}%
\pgfsetfillcolor{currentfill}%
\pgfsetfillopacity{0.700000}%
\pgfsetlinewidth{0.000000pt}%
\definecolor{currentstroke}{rgb}{0.000000,0.000000,0.000000}%
\pgfsetstrokecolor{currentstroke}%
\pgfsetdash{}{0pt}%
\pgfpathmoveto{\pgfqpoint{2.212625in}{2.754998in}}%
\pgfpathlineto{\pgfqpoint{2.226323in}{2.743890in}}%
\pgfpathlineto{\pgfqpoint{2.240022in}{2.732835in}}%
\pgfpathlineto{\pgfqpoint{2.253721in}{2.721834in}}%
\pgfpathlineto{\pgfqpoint{2.267420in}{2.710887in}}%
\pgfpathlineto{\pgfqpoint{2.257975in}{2.724951in}}%
\pgfpathlineto{\pgfqpoint{2.248497in}{2.739570in}}%
\pgfpathlineto{\pgfqpoint{2.238985in}{2.754756in}}%
\pgfpathlineto{\pgfqpoint{2.229438in}{2.770519in}}%
\pgfpathlineto{\pgfqpoint{2.215685in}{2.781855in}}%
\pgfpathlineto{\pgfqpoint{2.201933in}{2.793243in}}%
\pgfpathlineto{\pgfqpoint{2.188181in}{2.804686in}}%
\pgfpathlineto{\pgfqpoint{2.174429in}{2.816183in}}%
\pgfpathlineto{\pgfqpoint{2.184031in}{2.800025in}}%
\pgfpathlineto{\pgfqpoint{2.193597in}{2.784449in}}%
\pgfpathlineto{\pgfqpoint{2.203128in}{2.769444in}}%
\pgfpathlineto{\pgfqpoint{2.212625in}{2.754998in}}%
\pgfpathclose%
\pgfusepath{fill}%
\end{pgfscope}%
\begin{pgfscope}%
\pgfpathrectangle{\pgfqpoint{1.150000in}{0.150000in}}{\pgfqpoint{5.700000in}{5.700000in}}%
\pgfusepath{clip}%
\pgfsetbuttcap%
\pgfsetroundjoin%
\definecolor{currentfill}{rgb}{0.272594,0.025563,0.353093}%
\pgfsetfillcolor{currentfill}%
\pgfsetfillopacity{0.700000}%
\pgfsetlinewidth{0.000000pt}%
\definecolor{currentstroke}{rgb}{0.000000,0.000000,0.000000}%
\pgfsetstrokecolor{currentstroke}%
\pgfsetdash{}{0pt}%
\pgfpathmoveto{\pgfqpoint{4.594421in}{1.812093in}}%
\pgfpathlineto{\pgfqpoint{4.608432in}{1.809140in}}%
\pgfpathlineto{\pgfqpoint{4.622450in}{1.806212in}}%
\pgfpathlineto{\pgfqpoint{4.636475in}{1.803309in}}%
\pgfpathlineto{\pgfqpoint{4.650508in}{1.800431in}}%
\pgfpathlineto{\pgfqpoint{4.642687in}{1.791027in}}%
\pgfpathlineto{\pgfqpoint{4.634862in}{1.781684in}}%
\pgfpathlineto{\pgfqpoint{4.627032in}{1.772406in}}%
\pgfpathlineto{\pgfqpoint{4.619197in}{1.763199in}}%
\pgfpathlineto{\pgfqpoint{4.605154in}{1.766282in}}%
\pgfpathlineto{\pgfqpoint{4.591118in}{1.769389in}}%
\pgfpathlineto{\pgfqpoint{4.577090in}{1.772521in}}%
\pgfpathlineto{\pgfqpoint{4.563069in}{1.775678in}}%
\pgfpathlineto{\pgfqpoint{4.570914in}{1.784676in}}%
\pgfpathlineto{\pgfqpoint{4.578755in}{1.793748in}}%
\pgfpathlineto{\pgfqpoint{4.586590in}{1.802889in}}%
\pgfpathlineto{\pgfqpoint{4.594421in}{1.812093in}}%
\pgfpathclose%
\pgfusepath{fill}%
\end{pgfscope}%
\begin{pgfscope}%
\pgfpathrectangle{\pgfqpoint{1.150000in}{0.150000in}}{\pgfqpoint{5.700000in}{5.700000in}}%
\pgfusepath{clip}%
\pgfsetbuttcap%
\pgfsetroundjoin%
\definecolor{currentfill}{rgb}{0.280267,0.073417,0.397163}%
\pgfsetfillcolor{currentfill}%
\pgfsetfillopacity{0.700000}%
\pgfsetlinewidth{0.000000pt}%
\definecolor{currentstroke}{rgb}{0.000000,0.000000,0.000000}%
\pgfsetstrokecolor{currentstroke}%
\pgfsetdash{}{0pt}%
\pgfpathmoveto{\pgfqpoint{4.912627in}{1.890093in}}%
\pgfpathlineto{\pgfqpoint{4.926733in}{1.888068in}}%
\pgfpathlineto{\pgfqpoint{4.940846in}{1.886067in}}%
\pgfpathlineto{\pgfqpoint{4.954968in}{1.884090in}}%
\pgfpathlineto{\pgfqpoint{4.969098in}{1.882138in}}%
\pgfpathlineto{\pgfqpoint{4.961372in}{1.871723in}}%
\pgfpathlineto{\pgfqpoint{4.953640in}{1.861303in}}%
\pgfpathlineto{\pgfqpoint{4.945903in}{1.850881in}}%
\pgfpathlineto{\pgfqpoint{4.938162in}{1.840462in}}%
\pgfpathlineto{\pgfqpoint{4.924024in}{1.842579in}}%
\pgfpathlineto{\pgfqpoint{4.909894in}{1.844720in}}%
\pgfpathlineto{\pgfqpoint{4.895772in}{1.846886in}}%
\pgfpathlineto{\pgfqpoint{4.881658in}{1.849076in}}%
\pgfpathlineto{\pgfqpoint{4.889408in}{1.859325in}}%
\pgfpathlineto{\pgfqpoint{4.897153in}{1.869580in}}%
\pgfpathlineto{\pgfqpoint{4.904892in}{1.879838in}}%
\pgfpathlineto{\pgfqpoint{4.912627in}{1.890093in}}%
\pgfpathclose%
\pgfusepath{fill}%
\end{pgfscope}%
\begin{pgfscope}%
\pgfpathrectangle{\pgfqpoint{1.150000in}{0.150000in}}{\pgfqpoint{5.700000in}{5.700000in}}%
\pgfusepath{clip}%
\pgfsetbuttcap%
\pgfsetroundjoin%
\definecolor{currentfill}{rgb}{0.268510,0.009605,0.335427}%
\pgfsetfillcolor{currentfill}%
\pgfsetfillopacity{0.700000}%
\pgfsetlinewidth{0.000000pt}%
\definecolor{currentstroke}{rgb}{0.000000,0.000000,0.000000}%
\pgfsetstrokecolor{currentstroke}%
\pgfsetdash{}{0pt}%
\pgfpathmoveto{\pgfqpoint{4.363798in}{1.782953in}}%
\pgfpathlineto{\pgfqpoint{4.377747in}{1.779281in}}%
\pgfpathlineto{\pgfqpoint{4.391703in}{1.775634in}}%
\pgfpathlineto{\pgfqpoint{4.405665in}{1.772011in}}%
\pgfpathlineto{\pgfqpoint{4.419635in}{1.768414in}}%
\pgfpathlineto{\pgfqpoint{4.411741in}{1.760315in}}%
\pgfpathlineto{\pgfqpoint{4.403843in}{1.752329in}}%
\pgfpathlineto{\pgfqpoint{4.395939in}{1.744462in}}%
\pgfpathlineto{\pgfqpoint{4.388030in}{1.736719in}}%
\pgfpathlineto{\pgfqpoint{4.374048in}{1.740547in}}%
\pgfpathlineto{\pgfqpoint{4.360073in}{1.744400in}}%
\pgfpathlineto{\pgfqpoint{4.346104in}{1.748278in}}%
\pgfpathlineto{\pgfqpoint{4.332143in}{1.752180in}}%
\pgfpathlineto{\pgfqpoint{4.340065in}{1.759688in}}%
\pgfpathlineto{\pgfqpoint{4.347981in}{1.767323in}}%
\pgfpathlineto{\pgfqpoint{4.355892in}{1.775080in}}%
\pgfpathlineto{\pgfqpoint{4.363798in}{1.782953in}}%
\pgfpathclose%
\pgfusepath{fill}%
\end{pgfscope}%
\begin{pgfscope}%
\pgfpathrectangle{\pgfqpoint{1.150000in}{0.150000in}}{\pgfqpoint{5.700000in}{5.700000in}}%
\pgfusepath{clip}%
\pgfsetbuttcap%
\pgfsetroundjoin%
\definecolor{currentfill}{rgb}{0.280868,0.160771,0.472899}%
\pgfsetfillcolor{currentfill}%
\pgfsetfillopacity{0.700000}%
\pgfsetlinewidth{0.000000pt}%
\definecolor{currentstroke}{rgb}{0.000000,0.000000,0.000000}%
\pgfsetstrokecolor{currentstroke}%
\pgfsetdash{}{0pt}%
\pgfpathmoveto{\pgfqpoint{5.405997in}{2.062815in}}%
\pgfpathlineto{\pgfqpoint{5.420265in}{2.062000in}}%
\pgfpathlineto{\pgfqpoint{5.434543in}{2.061210in}}%
\pgfpathlineto{\pgfqpoint{5.448829in}{2.060445in}}%
\pgfpathlineto{\pgfqpoint{5.463125in}{2.059704in}}%
\pgfpathlineto{\pgfqpoint{5.455566in}{2.049504in}}%
\pgfpathlineto{\pgfqpoint{5.448000in}{2.039221in}}%
\pgfpathlineto{\pgfqpoint{5.440428in}{2.028856in}}%
\pgfpathlineto{\pgfqpoint{5.432848in}{2.018412in}}%
\pgfpathlineto{\pgfqpoint{5.418544in}{2.019250in}}%
\pgfpathlineto{\pgfqpoint{5.404249in}{2.020113in}}%
\pgfpathlineto{\pgfqpoint{5.389964in}{2.021001in}}%
\pgfpathlineto{\pgfqpoint{5.375687in}{2.021913in}}%
\pgfpathlineto{\pgfqpoint{5.383274in}{2.032254in}}%
\pgfpathlineto{\pgfqpoint{5.390855in}{2.042519in}}%
\pgfpathlineto{\pgfqpoint{5.398429in}{2.052707in}}%
\pgfpathlineto{\pgfqpoint{5.405997in}{2.062815in}}%
\pgfpathclose%
\pgfusepath{fill}%
\end{pgfscope}%
\begin{pgfscope}%
\pgfpathrectangle{\pgfqpoint{1.150000in}{0.150000in}}{\pgfqpoint{5.700000in}{5.700000in}}%
\pgfusepath{clip}%
\pgfsetbuttcap%
\pgfsetroundjoin%
\definecolor{currentfill}{rgb}{0.283091,0.110553,0.431554}%
\pgfsetfillcolor{currentfill}%
\pgfsetfillopacity{0.700000}%
\pgfsetlinewidth{0.000000pt}%
\definecolor{currentstroke}{rgb}{0.000000,0.000000,0.000000}%
\pgfsetstrokecolor{currentstroke}%
\pgfsetdash{}{0pt}%
\pgfpathmoveto{\pgfqpoint{3.449898in}{1.947768in}}%
\pgfpathlineto{\pgfqpoint{3.463664in}{1.941125in}}%
\pgfpathlineto{\pgfqpoint{3.477436in}{1.934511in}}%
\pgfpathlineto{\pgfqpoint{3.491212in}{1.927926in}}%
\pgfpathlineto{\pgfqpoint{3.504992in}{1.921370in}}%
\pgfpathlineto{\pgfqpoint{3.496662in}{1.921740in}}%
\pgfpathlineto{\pgfqpoint{3.488319in}{1.922433in}}%
\pgfpathlineto{\pgfqpoint{3.479963in}{1.923458in}}%
\pgfpathlineto{\pgfqpoint{3.471594in}{1.924822in}}%
\pgfpathlineto{\pgfqpoint{3.457785in}{1.931691in}}%
\pgfpathlineto{\pgfqpoint{3.443981in}{1.938588in}}%
\pgfpathlineto{\pgfqpoint{3.430182in}{1.945515in}}%
\pgfpathlineto{\pgfqpoint{3.416387in}{1.952470in}}%
\pgfpathlineto{\pgfqpoint{3.424785in}{1.950787in}}%
\pgfpathlineto{\pgfqpoint{3.433169in}{1.949448in}}%
\pgfpathlineto{\pgfqpoint{3.441540in}{1.948445in}}%
\pgfpathlineto{\pgfqpoint{3.449898in}{1.947768in}}%
\pgfpathclose%
\pgfusepath{fill}%
\end{pgfscope}%
\begin{pgfscope}%
\pgfpathrectangle{\pgfqpoint{1.150000in}{0.150000in}}{\pgfqpoint{5.700000in}{5.700000in}}%
\pgfusepath{clip}%
\pgfsetbuttcap%
\pgfsetroundjoin%
\definecolor{currentfill}{rgb}{0.282623,0.140926,0.457517}%
\pgfsetfillcolor{currentfill}%
\pgfsetfillopacity{0.700000}%
\pgfsetlinewidth{0.000000pt}%
\definecolor{currentstroke}{rgb}{0.000000,0.000000,0.000000}%
\pgfsetstrokecolor{currentstroke}%
\pgfsetdash{}{0pt}%
\pgfpathmoveto{\pgfqpoint{5.318668in}{2.025806in}}%
\pgfpathlineto{\pgfqpoint{5.332909in}{2.024796in}}%
\pgfpathlineto{\pgfqpoint{5.347160in}{2.023810in}}%
\pgfpathlineto{\pgfqpoint{5.361419in}{2.022849in}}%
\pgfpathlineto{\pgfqpoint{5.375687in}{2.021913in}}%
\pgfpathlineto{\pgfqpoint{5.368092in}{2.011498in}}%
\pgfpathlineto{\pgfqpoint{5.360491in}{2.001012in}}%
\pgfpathlineto{\pgfqpoint{5.352884in}{1.990457in}}%
\pgfpathlineto{\pgfqpoint{5.345270in}{1.979836in}}%
\pgfpathlineto{\pgfqpoint{5.330994in}{1.980884in}}%
\pgfpathlineto{\pgfqpoint{5.316727in}{1.981956in}}%
\pgfpathlineto{\pgfqpoint{5.302469in}{1.983053in}}%
\pgfpathlineto{\pgfqpoint{5.288219in}{1.984174in}}%
\pgfpathlineto{\pgfqpoint{5.295841in}{1.994679in}}%
\pgfpathlineto{\pgfqpoint{5.303456in}{2.005121in}}%
\pgfpathlineto{\pgfqpoint{5.311065in}{2.015497in}}%
\pgfpathlineto{\pgfqpoint{5.318668in}{2.025806in}}%
\pgfpathclose%
\pgfusepath{fill}%
\end{pgfscope}%
\begin{pgfscope}%
\pgfpathrectangle{\pgfqpoint{1.150000in}{0.150000in}}{\pgfqpoint{5.700000in}{5.700000in}}%
\pgfusepath{clip}%
\pgfsetbuttcap%
\pgfsetroundjoin%
\definecolor{currentfill}{rgb}{0.280267,0.073417,0.397163}%
\pgfsetfillcolor{currentfill}%
\pgfsetfillopacity{0.700000}%
\pgfsetlinewidth{0.000000pt}%
\definecolor{currentstroke}{rgb}{0.000000,0.000000,0.000000}%
\pgfsetstrokecolor{currentstroke}%
\pgfsetdash{}{0pt}%
\pgfpathmoveto{\pgfqpoint{3.648408in}{1.873922in}}%
\pgfpathlineto{\pgfqpoint{3.662207in}{1.867918in}}%
\pgfpathlineto{\pgfqpoint{3.676011in}{1.861942in}}%
\pgfpathlineto{\pgfqpoint{3.689821in}{1.855994in}}%
\pgfpathlineto{\pgfqpoint{3.703635in}{1.850073in}}%
\pgfpathlineto{\pgfqpoint{3.695425in}{1.848353in}}%
\pgfpathlineto{\pgfqpoint{3.687205in}{1.846914in}}%
\pgfpathlineto{\pgfqpoint{3.678974in}{1.845762in}}%
\pgfpathlineto{\pgfqpoint{3.670733in}{1.844907in}}%
\pgfpathlineto{\pgfqpoint{3.656894in}{1.851126in}}%
\pgfpathlineto{\pgfqpoint{3.643061in}{1.857372in}}%
\pgfpathlineto{\pgfqpoint{3.629232in}{1.863646in}}%
\pgfpathlineto{\pgfqpoint{3.615408in}{1.869947in}}%
\pgfpathlineto{\pgfqpoint{3.623675in}{1.870499in}}%
\pgfpathlineto{\pgfqpoint{3.631930in}{1.871351in}}%
\pgfpathlineto{\pgfqpoint{3.640174in}{1.872494in}}%
\pgfpathlineto{\pgfqpoint{3.648408in}{1.873922in}}%
\pgfpathclose%
\pgfusepath{fill}%
\end{pgfscope}%
\begin{pgfscope}%
\pgfpathrectangle{\pgfqpoint{1.150000in}{0.150000in}}{\pgfqpoint{5.700000in}{5.700000in}}%
\pgfusepath{clip}%
\pgfsetbuttcap%
\pgfsetroundjoin%
\definecolor{currentfill}{rgb}{0.277941,0.056324,0.381191}%
\pgfsetfillcolor{currentfill}%
\pgfsetfillopacity{0.700000}%
\pgfsetlinewidth{0.000000pt}%
\definecolor{currentstroke}{rgb}{0.000000,0.000000,0.000000}%
\pgfsetstrokecolor{currentstroke}%
\pgfsetdash{}{0pt}%
\pgfpathmoveto{\pgfqpoint{4.825280in}{1.858082in}}%
\pgfpathlineto{\pgfqpoint{4.839363in}{1.855794in}}%
\pgfpathlineto{\pgfqpoint{4.853453in}{1.853530in}}%
\pgfpathlineto{\pgfqpoint{4.867552in}{1.851291in}}%
\pgfpathlineto{\pgfqpoint{4.881658in}{1.849076in}}%
\pgfpathlineto{\pgfqpoint{4.873903in}{1.838837in}}%
\pgfpathlineto{\pgfqpoint{4.866143in}{1.828612in}}%
\pgfpathlineto{\pgfqpoint{4.858379in}{1.818406in}}%
\pgfpathlineto{\pgfqpoint{4.850610in}{1.808222in}}%
\pgfpathlineto{\pgfqpoint{4.836495in}{1.810615in}}%
\pgfpathlineto{\pgfqpoint{4.822388in}{1.813032in}}%
\pgfpathlineto{\pgfqpoint{4.808289in}{1.815474in}}%
\pgfpathlineto{\pgfqpoint{4.794198in}{1.817940in}}%
\pgfpathlineto{\pgfqpoint{4.801976in}{1.827941in}}%
\pgfpathlineto{\pgfqpoint{4.809749in}{1.837968in}}%
\pgfpathlineto{\pgfqpoint{4.817517in}{1.848016in}}%
\pgfpathlineto{\pgfqpoint{4.825280in}{1.858082in}}%
\pgfpathclose%
\pgfusepath{fill}%
\end{pgfscope}%
\begin{pgfscope}%
\pgfpathrectangle{\pgfqpoint{1.150000in}{0.150000in}}{\pgfqpoint{5.700000in}{5.700000in}}%
\pgfusepath{clip}%
\pgfsetbuttcap%
\pgfsetroundjoin%
\definecolor{currentfill}{rgb}{0.174274,0.445044,0.557792}%
\pgfsetfillcolor{currentfill}%
\pgfsetfillopacity{0.700000}%
\pgfsetlinewidth{0.000000pt}%
\definecolor{currentstroke}{rgb}{0.000000,0.000000,0.000000}%
\pgfsetstrokecolor{currentstroke}%
\pgfsetdash{}{0pt}%
\pgfpathmoveto{\pgfqpoint{2.267420in}{2.710887in}}%
\pgfpathlineto{\pgfqpoint{2.281121in}{2.699991in}}%
\pgfpathlineto{\pgfqpoint{2.294822in}{2.689148in}}%
\pgfpathlineto{\pgfqpoint{2.308524in}{2.678357in}}%
\pgfpathlineto{\pgfqpoint{2.322227in}{2.667616in}}%
\pgfpathlineto{\pgfqpoint{2.312834in}{2.681300in}}%
\pgfpathlineto{\pgfqpoint{2.303408in}{2.695534in}}%
\pgfpathlineto{\pgfqpoint{2.293948in}{2.710331in}}%
\pgfpathlineto{\pgfqpoint{2.284455in}{2.725700in}}%
\pgfpathlineto{\pgfqpoint{2.270700in}{2.736828in}}%
\pgfpathlineto{\pgfqpoint{2.256945in}{2.748006in}}%
\pgfpathlineto{\pgfqpoint{2.243191in}{2.759237in}}%
\pgfpathlineto{\pgfqpoint{2.229438in}{2.770519in}}%
\pgfpathlineto{\pgfqpoint{2.238985in}{2.754756in}}%
\pgfpathlineto{\pgfqpoint{2.248497in}{2.739570in}}%
\pgfpathlineto{\pgfqpoint{2.257975in}{2.724951in}}%
\pgfpathlineto{\pgfqpoint{2.267420in}{2.710887in}}%
\pgfpathclose%
\pgfusepath{fill}%
\end{pgfscope}%
\begin{pgfscope}%
\pgfpathrectangle{\pgfqpoint{1.150000in}{0.150000in}}{\pgfqpoint{5.700000in}{5.700000in}}%
\pgfusepath{clip}%
\pgfsetbuttcap%
\pgfsetroundjoin%
\definecolor{currentfill}{rgb}{0.231674,0.318106,0.544834}%
\pgfsetfillcolor{currentfill}%
\pgfsetfillopacity{0.700000}%
\pgfsetlinewidth{0.000000pt}%
\definecolor{currentstroke}{rgb}{0.000000,0.000000,0.000000}%
\pgfsetstrokecolor{currentstroke}%
\pgfsetdash{}{0pt}%
\pgfpathmoveto{\pgfqpoint{2.687677in}{2.383224in}}%
\pgfpathlineto{\pgfqpoint{2.701377in}{2.373971in}}%
\pgfpathlineto{\pgfqpoint{2.715079in}{2.364759in}}%
\pgfpathlineto{\pgfqpoint{2.728784in}{2.355586in}}%
\pgfpathlineto{\pgfqpoint{2.742492in}{2.346452in}}%
\pgfpathlineto{\pgfqpoint{2.733536in}{2.355524in}}%
\pgfpathlineto{\pgfqpoint{2.724555in}{2.365075in}}%
\pgfpathlineto{\pgfqpoint{2.715549in}{2.375115in}}%
\pgfpathlineto{\pgfqpoint{2.706518in}{2.385655in}}%
\pgfpathlineto{\pgfqpoint{2.692767in}{2.395151in}}%
\pgfpathlineto{\pgfqpoint{2.679019in}{2.404688in}}%
\pgfpathlineto{\pgfqpoint{2.665272in}{2.414264in}}%
\pgfpathlineto{\pgfqpoint{2.651528in}{2.423880in}}%
\pgfpathlineto{\pgfqpoint{2.660605in}{2.412970in}}%
\pgfpathlineto{\pgfqpoint{2.669654in}{2.402565in}}%
\pgfpathlineto{\pgfqpoint{2.678678in}{2.392653in}}%
\pgfpathlineto{\pgfqpoint{2.687677in}{2.383224in}}%
\pgfpathclose%
\pgfusepath{fill}%
\end{pgfscope}%
\begin{pgfscope}%
\pgfpathrectangle{\pgfqpoint{1.150000in}{0.150000in}}{\pgfqpoint{5.700000in}{5.700000in}}%
\pgfusepath{clip}%
\pgfsetbuttcap%
\pgfsetroundjoin%
\definecolor{currentfill}{rgb}{0.267968,0.223549,0.512008}%
\pgfsetfillcolor{currentfill}%
\pgfsetfillopacity{0.700000}%
\pgfsetlinewidth{0.000000pt}%
\definecolor{currentstroke}{rgb}{0.000000,0.000000,0.000000}%
\pgfsetstrokecolor{currentstroke}%
\pgfsetdash{}{0pt}%
\pgfpathmoveto{\pgfqpoint{2.997148in}{2.179230in}}%
\pgfpathlineto{\pgfqpoint{3.010867in}{2.171063in}}%
\pgfpathlineto{\pgfqpoint{3.024588in}{2.162930in}}%
\pgfpathlineto{\pgfqpoint{3.038314in}{2.154830in}}%
\pgfpathlineto{\pgfqpoint{3.052043in}{2.146765in}}%
\pgfpathlineto{\pgfqpoint{3.043368in}{2.152322in}}%
\pgfpathlineto{\pgfqpoint{3.034674in}{2.158300in}}%
\pgfpathlineto{\pgfqpoint{3.025961in}{2.164707in}}%
\pgfpathlineto{\pgfqpoint{3.017228in}{2.171553in}}%
\pgfpathlineto{\pgfqpoint{3.003462in}{2.179962in}}%
\pgfpathlineto{\pgfqpoint{2.989699in}{2.188406in}}%
\pgfpathlineto{\pgfqpoint{2.975940in}{2.196883in}}%
\pgfpathlineto{\pgfqpoint{2.962184in}{2.205395in}}%
\pgfpathlineto{\pgfqpoint{2.970956in}{2.198199in}}%
\pgfpathlineto{\pgfqpoint{2.979707in}{2.191446in}}%
\pgfpathlineto{\pgfqpoint{2.988437in}{2.185126in}}%
\pgfpathlineto{\pgfqpoint{2.997148in}{2.179230in}}%
\pgfpathclose%
\pgfusepath{fill}%
\end{pgfscope}%
\begin{pgfscope}%
\pgfpathrectangle{\pgfqpoint{1.150000in}{0.150000in}}{\pgfqpoint{5.700000in}{5.700000in}}%
\pgfusepath{clip}%
\pgfsetbuttcap%
\pgfsetroundjoin%
\definecolor{currentfill}{rgb}{0.281412,0.155834,0.469201}%
\pgfsetfillcolor{currentfill}%
\pgfsetfillopacity{0.700000}%
\pgfsetlinewidth{0.000000pt}%
\definecolor{currentstroke}{rgb}{0.000000,0.000000,0.000000}%
\pgfsetstrokecolor{currentstroke}%
\pgfsetdash{}{0pt}%
\pgfpathmoveto{\pgfqpoint{3.251181in}{2.038265in}}%
\pgfpathlineto{\pgfqpoint{3.264925in}{2.030948in}}%
\pgfpathlineto{\pgfqpoint{3.278673in}{2.023662in}}%
\pgfpathlineto{\pgfqpoint{3.292426in}{2.016407in}}%
\pgfpathlineto{\pgfqpoint{3.306182in}{2.009182in}}%
\pgfpathlineto{\pgfqpoint{3.297710in}{2.011861in}}%
\pgfpathlineto{\pgfqpoint{3.289222in}{2.014908in}}%
\pgfpathlineto{\pgfqpoint{3.280718in}{2.018333in}}%
\pgfpathlineto{\pgfqpoint{3.272199in}{2.022143in}}%
\pgfpathlineto{\pgfqpoint{3.258410in}{2.029696in}}%
\pgfpathlineto{\pgfqpoint{3.244626in}{2.037279in}}%
\pgfpathlineto{\pgfqpoint{3.230845in}{2.044893in}}%
\pgfpathlineto{\pgfqpoint{3.217069in}{2.052538in}}%
\pgfpathlineto{\pgfqpoint{3.225621in}{2.048394in}}%
\pgfpathlineto{\pgfqpoint{3.234157in}{2.044639in}}%
\pgfpathlineto{\pgfqpoint{3.242677in}{2.041266in}}%
\pgfpathlineto{\pgfqpoint{3.251181in}{2.038265in}}%
\pgfpathclose%
\pgfusepath{fill}%
\end{pgfscope}%
\begin{pgfscope}%
\pgfpathrectangle{\pgfqpoint{1.150000in}{0.150000in}}{\pgfqpoint{5.700000in}{5.700000in}}%
\pgfusepath{clip}%
\pgfsetbuttcap%
\pgfsetroundjoin%
\definecolor{currentfill}{rgb}{0.271305,0.019942,0.347269}%
\pgfsetfillcolor{currentfill}%
\pgfsetfillopacity{0.700000}%
\pgfsetlinewidth{0.000000pt}%
\definecolor{currentstroke}{rgb}{0.000000,0.000000,0.000000}%
\pgfsetstrokecolor{currentstroke}%
\pgfsetdash{}{0pt}%
\pgfpathmoveto{\pgfqpoint{4.507056in}{1.788553in}}%
\pgfpathlineto{\pgfqpoint{4.521048in}{1.785297in}}%
\pgfpathlineto{\pgfqpoint{4.535048in}{1.782066in}}%
\pgfpathlineto{\pgfqpoint{4.549055in}{1.778860in}}%
\pgfpathlineto{\pgfqpoint{4.563069in}{1.775678in}}%
\pgfpathlineto{\pgfqpoint{4.555219in}{1.766759in}}%
\pgfpathlineto{\pgfqpoint{4.547364in}{1.757925in}}%
\pgfpathlineto{\pgfqpoint{4.539504in}{1.749179in}}%
\pgfpathlineto{\pgfqpoint{4.531639in}{1.740529in}}%
\pgfpathlineto{\pgfqpoint{4.517614in}{1.743928in}}%
\pgfpathlineto{\pgfqpoint{4.503596in}{1.747352in}}%
\pgfpathlineto{\pgfqpoint{4.489585in}{1.750800in}}%
\pgfpathlineto{\pgfqpoint{4.475581in}{1.754273in}}%
\pgfpathlineto{\pgfqpoint{4.483457in}{1.762701in}}%
\pgfpathlineto{\pgfqpoint{4.491328in}{1.771227in}}%
\pgfpathlineto{\pgfqpoint{4.499195in}{1.779846in}}%
\pgfpathlineto{\pgfqpoint{4.507056in}{1.788553in}}%
\pgfpathclose%
\pgfusepath{fill}%
\end{pgfscope}%
\begin{pgfscope}%
\pgfpathrectangle{\pgfqpoint{1.150000in}{0.150000in}}{\pgfqpoint{5.700000in}{5.700000in}}%
\pgfusepath{clip}%
\pgfsetbuttcap%
\pgfsetroundjoin%
\definecolor{currentfill}{rgb}{0.283187,0.125848,0.444960}%
\pgfsetfillcolor{currentfill}%
\pgfsetfillopacity{0.700000}%
\pgfsetlinewidth{0.000000pt}%
\definecolor{currentstroke}{rgb}{0.000000,0.000000,0.000000}%
\pgfsetstrokecolor{currentstroke}%
\pgfsetdash{}{0pt}%
\pgfpathmoveto{\pgfqpoint{5.231308in}{1.988904in}}%
\pgfpathlineto{\pgfqpoint{5.245523in}{1.987685in}}%
\pgfpathlineto{\pgfqpoint{5.259747in}{1.986490in}}%
\pgfpathlineto{\pgfqpoint{5.273979in}{1.985320in}}%
\pgfpathlineto{\pgfqpoint{5.288219in}{1.984174in}}%
\pgfpathlineto{\pgfqpoint{5.280591in}{1.973609in}}%
\pgfpathlineto{\pgfqpoint{5.272957in}{1.962986in}}%
\pgfpathlineto{\pgfqpoint{5.265317in}{1.952308in}}%
\pgfpathlineto{\pgfqpoint{5.257671in}{1.941578in}}%
\pgfpathlineto{\pgfqpoint{5.243422in}{1.942848in}}%
\pgfpathlineto{\pgfqpoint{5.229183in}{1.944143in}}%
\pgfpathlineto{\pgfqpoint{5.214952in}{1.945463in}}%
\pgfpathlineto{\pgfqpoint{5.200729in}{1.946807in}}%
\pgfpathlineto{\pgfqpoint{5.208383in}{1.957407in}}%
\pgfpathlineto{\pgfqpoint{5.216031in}{1.967958in}}%
\pgfpathlineto{\pgfqpoint{5.223673in}{1.978458in}}%
\pgfpathlineto{\pgfqpoint{5.231308in}{1.988904in}}%
\pgfpathclose%
\pgfusepath{fill}%
\end{pgfscope}%
\begin{pgfscope}%
\pgfpathrectangle{\pgfqpoint{1.150000in}{0.150000in}}{\pgfqpoint{5.700000in}{5.700000in}}%
\pgfusepath{clip}%
\pgfsetbuttcap%
\pgfsetroundjoin%
\definecolor{currentfill}{rgb}{0.269944,0.014625,0.341379}%
\pgfsetfillcolor{currentfill}%
\pgfsetfillopacity{0.700000}%
\pgfsetlinewidth{0.000000pt}%
\definecolor{currentstroke}{rgb}{0.000000,0.000000,0.000000}%
\pgfsetstrokecolor{currentstroke}%
\pgfsetdash{}{0pt}%
\pgfpathmoveto{\pgfqpoint{4.133193in}{1.775267in}}%
\pgfpathlineto{\pgfqpoint{4.147092in}{1.770817in}}%
\pgfpathlineto{\pgfqpoint{4.160997in}{1.766393in}}%
\pgfpathlineto{\pgfqpoint{4.174909in}{1.761994in}}%
\pgfpathlineto{\pgfqpoint{4.188827in}{1.757621in}}%
\pgfpathlineto{\pgfqpoint{4.180847in}{1.751346in}}%
\pgfpathlineto{\pgfqpoint{4.172861in}{1.745242in}}%
\pgfpathlineto{\pgfqpoint{4.164868in}{1.739316in}}%
\pgfpathlineto{\pgfqpoint{4.156869in}{1.733574in}}%
\pgfpathlineto{\pgfqpoint{4.142935in}{1.738205in}}%
\pgfpathlineto{\pgfqpoint{4.129008in}{1.742860in}}%
\pgfpathlineto{\pgfqpoint{4.115087in}{1.747541in}}%
\pgfpathlineto{\pgfqpoint{4.101171in}{1.752247in}}%
\pgfpathlineto{\pgfqpoint{4.109187in}{1.757727in}}%
\pgfpathlineto{\pgfqpoint{4.117195in}{1.763395in}}%
\pgfpathlineto{\pgfqpoint{4.125197in}{1.769243in}}%
\pgfpathlineto{\pgfqpoint{4.133193in}{1.775267in}}%
\pgfpathclose%
\pgfusepath{fill}%
\end{pgfscope}%
\begin{pgfscope}%
\pgfpathrectangle{\pgfqpoint{1.150000in}{0.150000in}}{\pgfqpoint{5.700000in}{5.700000in}}%
\pgfusepath{clip}%
\pgfsetbuttcap%
\pgfsetroundjoin%
\definecolor{currentfill}{rgb}{0.271305,0.019942,0.347269}%
\pgfsetfillcolor{currentfill}%
\pgfsetfillopacity{0.700000}%
\pgfsetlinewidth{0.000000pt}%
\definecolor{currentstroke}{rgb}{0.000000,0.000000,0.000000}%
\pgfsetstrokecolor{currentstroke}%
\pgfsetdash{}{0pt}%
\pgfpathmoveto{\pgfqpoint{3.990065in}{1.790825in}}%
\pgfpathlineto{\pgfqpoint{4.003933in}{1.785912in}}%
\pgfpathlineto{\pgfqpoint{4.017806in}{1.781025in}}%
\pgfpathlineto{\pgfqpoint{4.031685in}{1.776165in}}%
\pgfpathlineto{\pgfqpoint{4.045570in}{1.771330in}}%
\pgfpathlineto{\pgfqpoint{4.037531in}{1.766309in}}%
\pgfpathlineto{\pgfqpoint{4.029484in}{1.761494in}}%
\pgfpathlineto{\pgfqpoint{4.021430in}{1.756890in}}%
\pgfpathlineto{\pgfqpoint{4.013368in}{1.752505in}}%
\pgfpathlineto{\pgfqpoint{3.999465in}{1.757610in}}%
\pgfpathlineto{\pgfqpoint{3.985568in}{1.762741in}}%
\pgfpathlineto{\pgfqpoint{3.971676in}{1.767898in}}%
\pgfpathlineto{\pgfqpoint{3.957790in}{1.773081in}}%
\pgfpathlineto{\pgfqpoint{3.965870in}{1.777191in}}%
\pgfpathlineto{\pgfqpoint{3.973943in}{1.781523in}}%
\pgfpathlineto{\pgfqpoint{3.982008in}{1.786070in}}%
\pgfpathlineto{\pgfqpoint{3.990065in}{1.790825in}}%
\pgfpathclose%
\pgfusepath{fill}%
\end{pgfscope}%
\begin{pgfscope}%
\pgfpathrectangle{\pgfqpoint{1.150000in}{0.150000in}}{\pgfqpoint{5.700000in}{5.700000in}}%
\pgfusepath{clip}%
\pgfsetbuttcap%
\pgfsetroundjoin%
\definecolor{currentfill}{rgb}{0.276022,0.044167,0.370164}%
\pgfsetfillcolor{currentfill}%
\pgfsetfillopacity{0.700000}%
\pgfsetlinewidth{0.000000pt}%
\definecolor{currentstroke}{rgb}{0.000000,0.000000,0.000000}%
\pgfsetstrokecolor{currentstroke}%
\pgfsetdash{}{0pt}%
\pgfpathmoveto{\pgfqpoint{4.737909in}{1.828050in}}%
\pgfpathlineto{\pgfqpoint{4.751970in}{1.825486in}}%
\pgfpathlineto{\pgfqpoint{4.766038in}{1.822946in}}%
\pgfpathlineto{\pgfqpoint{4.780114in}{1.820431in}}%
\pgfpathlineto{\pgfqpoint{4.794198in}{1.817940in}}%
\pgfpathlineto{\pgfqpoint{4.786415in}{1.807970in}}%
\pgfpathlineto{\pgfqpoint{4.778628in}{1.798035in}}%
\pgfpathlineto{\pgfqpoint{4.770835in}{1.788139in}}%
\pgfpathlineto{\pgfqpoint{4.763038in}{1.778287in}}%
\pgfpathlineto{\pgfqpoint{4.748946in}{1.780969in}}%
\pgfpathlineto{\pgfqpoint{4.734861in}{1.783676in}}%
\pgfpathlineto{\pgfqpoint{4.720783in}{1.786407in}}%
\pgfpathlineto{\pgfqpoint{4.706713in}{1.789162in}}%
\pgfpathlineto{\pgfqpoint{4.714519in}{1.798818in}}%
\pgfpathlineto{\pgfqpoint{4.722321in}{1.808521in}}%
\pgfpathlineto{\pgfqpoint{4.730117in}{1.818266in}}%
\pgfpathlineto{\pgfqpoint{4.737909in}{1.828050in}}%
\pgfpathclose%
\pgfusepath{fill}%
\end{pgfscope}%
\begin{pgfscope}%
\pgfpathrectangle{\pgfqpoint{1.150000in}{0.150000in}}{\pgfqpoint{5.700000in}{5.700000in}}%
\pgfusepath{clip}%
\pgfsetbuttcap%
\pgfsetroundjoin%
\definecolor{currentfill}{rgb}{0.283091,0.110553,0.431554}%
\pgfsetfillcolor{currentfill}%
\pgfsetfillopacity{0.700000}%
\pgfsetlinewidth{0.000000pt}%
\definecolor{currentstroke}{rgb}{0.000000,0.000000,0.000000}%
\pgfsetstrokecolor{currentstroke}%
\pgfsetdash{}{0pt}%
\pgfpathmoveto{\pgfqpoint{5.143925in}{1.952428in}}%
\pgfpathlineto{\pgfqpoint{5.158113in}{1.950986in}}%
\pgfpathlineto{\pgfqpoint{5.172310in}{1.949568in}}%
\pgfpathlineto{\pgfqpoint{5.186515in}{1.948175in}}%
\pgfpathlineto{\pgfqpoint{5.200729in}{1.946807in}}%
\pgfpathlineto{\pgfqpoint{5.193070in}{1.936161in}}%
\pgfpathlineto{\pgfqpoint{5.185404in}{1.925472in}}%
\pgfpathlineto{\pgfqpoint{5.177733in}{1.914744in}}%
\pgfpathlineto{\pgfqpoint{5.170056in}{1.903979in}}%
\pgfpathlineto{\pgfqpoint{5.155835in}{1.905486in}}%
\pgfpathlineto{\pgfqpoint{5.141622in}{1.907017in}}%
\pgfpathlineto{\pgfqpoint{5.127417in}{1.908573in}}%
\pgfpathlineto{\pgfqpoint{5.113221in}{1.910153in}}%
\pgfpathlineto{\pgfqpoint{5.120906in}{1.920774in}}%
\pgfpathlineto{\pgfqpoint{5.128584in}{1.931362in}}%
\pgfpathlineto{\pgfqpoint{5.136257in}{1.941915in}}%
\pgfpathlineto{\pgfqpoint{5.143925in}{1.952428in}}%
\pgfpathclose%
\pgfusepath{fill}%
\end{pgfscope}%
\begin{pgfscope}%
\pgfpathrectangle{\pgfqpoint{1.150000in}{0.150000in}}{\pgfqpoint{5.700000in}{5.700000in}}%
\pgfusepath{clip}%
\pgfsetbuttcap%
\pgfsetroundjoin%
\definecolor{currentfill}{rgb}{0.268510,0.009605,0.335427}%
\pgfsetfillcolor{currentfill}%
\pgfsetfillopacity{0.700000}%
\pgfsetlinewidth{0.000000pt}%
\definecolor{currentstroke}{rgb}{0.000000,0.000000,0.000000}%
\pgfsetstrokecolor{currentstroke}%
\pgfsetdash{}{0pt}%
\pgfpathmoveto{\pgfqpoint{4.276361in}{1.768042in}}%
\pgfpathlineto{\pgfqpoint{4.290297in}{1.764039in}}%
\pgfpathlineto{\pgfqpoint{4.304239in}{1.760061in}}%
\pgfpathlineto{\pgfqpoint{4.318187in}{1.756108in}}%
\pgfpathlineto{\pgfqpoint{4.332143in}{1.752180in}}%
\pgfpathlineto{\pgfqpoint{4.324215in}{1.744806in}}%
\pgfpathlineto{\pgfqpoint{4.316282in}{1.737572in}}%
\pgfpathlineto{\pgfqpoint{4.308343in}{1.730483in}}%
\pgfpathlineto{\pgfqpoint{4.300399in}{1.723545in}}%
\pgfpathlineto{\pgfqpoint{4.286430in}{1.727717in}}%
\pgfpathlineto{\pgfqpoint{4.272467in}{1.731913in}}%
\pgfpathlineto{\pgfqpoint{4.258511in}{1.736135in}}%
\pgfpathlineto{\pgfqpoint{4.244561in}{1.740382in}}%
\pgfpathlineto{\pgfqpoint{4.252520in}{1.747070in}}%
\pgfpathlineto{\pgfqpoint{4.260473in}{1.753914in}}%
\pgfpathlineto{\pgfqpoint{4.268420in}{1.760907in}}%
\pgfpathlineto{\pgfqpoint{4.276361in}{1.768042in}}%
\pgfpathclose%
\pgfusepath{fill}%
\end{pgfscope}%
\begin{pgfscope}%
\pgfpathrectangle{\pgfqpoint{1.150000in}{0.150000in}}{\pgfqpoint{5.700000in}{5.700000in}}%
\pgfusepath{clip}%
\pgfsetbuttcap%
\pgfsetroundjoin%
\definecolor{currentfill}{rgb}{0.274952,0.037752,0.364543}%
\pgfsetfillcolor{currentfill}%
\pgfsetfillopacity{0.700000}%
\pgfsetlinewidth{0.000000pt}%
\definecolor{currentstroke}{rgb}{0.000000,0.000000,0.000000}%
\pgfsetstrokecolor{currentstroke}%
\pgfsetdash{}{0pt}%
\pgfpathmoveto{\pgfqpoint{3.846906in}{1.815490in}}%
\pgfpathlineto{\pgfqpoint{3.860747in}{1.810096in}}%
\pgfpathlineto{\pgfqpoint{3.874594in}{1.804729in}}%
\pgfpathlineto{\pgfqpoint{3.888446in}{1.799389in}}%
\pgfpathlineto{\pgfqpoint{3.902303in}{1.794074in}}%
\pgfpathlineto{\pgfqpoint{3.894195in}{1.790473in}}%
\pgfpathlineto{\pgfqpoint{3.886079in}{1.787110in}}%
\pgfpathlineto{\pgfqpoint{3.877954in}{1.783996in}}%
\pgfpathlineto{\pgfqpoint{3.869820in}{1.781135in}}%
\pgfpathlineto{\pgfqpoint{3.855942in}{1.786733in}}%
\pgfpathlineto{\pgfqpoint{3.842069in}{1.792358in}}%
\pgfpathlineto{\pgfqpoint{3.828202in}{1.798009in}}%
\pgfpathlineto{\pgfqpoint{3.814340in}{1.803686in}}%
\pgfpathlineto{\pgfqpoint{3.822495in}{1.806257in}}%
\pgfpathlineto{\pgfqpoint{3.830641in}{1.809086in}}%
\pgfpathlineto{\pgfqpoint{3.838778in}{1.812166in}}%
\pgfpathlineto{\pgfqpoint{3.846906in}{1.815490in}}%
\pgfpathclose%
\pgfusepath{fill}%
\end{pgfscope}%
\begin{pgfscope}%
\pgfpathrectangle{\pgfqpoint{1.150000in}{0.150000in}}{\pgfqpoint{5.700000in}{5.700000in}}%
\pgfusepath{clip}%
\pgfsetbuttcap%
\pgfsetroundjoin%
\definecolor{currentfill}{rgb}{0.180629,0.429975,0.557282}%
\pgfsetfillcolor{currentfill}%
\pgfsetfillopacity{0.700000}%
\pgfsetlinewidth{0.000000pt}%
\definecolor{currentstroke}{rgb}{0.000000,0.000000,0.000000}%
\pgfsetstrokecolor{currentstroke}%
\pgfsetdash{}{0pt}%
\pgfpathmoveto{\pgfqpoint{2.322227in}{2.667616in}}%
\pgfpathlineto{\pgfqpoint{2.335932in}{2.656925in}}%
\pgfpathlineto{\pgfqpoint{2.349637in}{2.646285in}}%
\pgfpathlineto{\pgfqpoint{2.363343in}{2.635693in}}%
\pgfpathlineto{\pgfqpoint{2.377051in}{2.625151in}}%
\pgfpathlineto{\pgfqpoint{2.367707in}{2.638456in}}%
\pgfpathlineto{\pgfqpoint{2.358333in}{2.652307in}}%
\pgfpathlineto{\pgfqpoint{2.348926in}{2.666715in}}%
\pgfpathlineto{\pgfqpoint{2.339486in}{2.681693in}}%
\pgfpathlineto{\pgfqpoint{2.325727in}{2.692620in}}%
\pgfpathlineto{\pgfqpoint{2.311969in}{2.703597in}}%
\pgfpathlineto{\pgfqpoint{2.298212in}{2.714624in}}%
\pgfpathlineto{\pgfqpoint{2.284455in}{2.725700in}}%
\pgfpathlineto{\pgfqpoint{2.293948in}{2.710331in}}%
\pgfpathlineto{\pgfqpoint{2.303408in}{2.695534in}}%
\pgfpathlineto{\pgfqpoint{2.312834in}{2.681300in}}%
\pgfpathlineto{\pgfqpoint{2.322227in}{2.667616in}}%
\pgfpathclose%
\pgfusepath{fill}%
\end{pgfscope}%
\begin{pgfscope}%
\pgfpathrectangle{\pgfqpoint{1.150000in}{0.150000in}}{\pgfqpoint{5.700000in}{5.700000in}}%
\pgfusepath{clip}%
\pgfsetbuttcap%
\pgfsetroundjoin%
\definecolor{currentfill}{rgb}{0.274128,0.199721,0.498911}%
\pgfsetfillcolor{currentfill}%
\pgfsetfillopacity{0.700000}%
\pgfsetlinewidth{0.000000pt}%
\definecolor{currentstroke}{rgb}{0.000000,0.000000,0.000000}%
\pgfsetstrokecolor{currentstroke}%
\pgfsetdash{}{0pt}%
\pgfpathmoveto{\pgfqpoint{5.637881in}{2.134283in}}%
\pgfpathlineto{\pgfqpoint{5.652241in}{2.133917in}}%
\pgfpathlineto{\pgfqpoint{5.666611in}{2.133576in}}%
\pgfpathlineto{\pgfqpoint{5.680990in}{2.133260in}}%
\pgfpathlineto{\pgfqpoint{5.673516in}{2.123610in}}%
\pgfpathlineto{\pgfqpoint{5.666035in}{2.113854in}}%
\pgfpathlineto{\pgfqpoint{5.658545in}{2.103994in}}%
\pgfpathlineto{\pgfqpoint{5.651048in}{2.094030in}}%
\pgfpathlineto{\pgfqpoint{5.636659in}{2.094417in}}%
\pgfpathlineto{\pgfqpoint{5.622281in}{2.094828in}}%
\pgfpathlineto{\pgfqpoint{5.607911in}{2.095263in}}%
\pgfpathlineto{\pgfqpoint{5.615415in}{2.105170in}}%
\pgfpathlineto{\pgfqpoint{5.622912in}{2.114977in}}%
\pgfpathlineto{\pgfqpoint{5.630400in}{2.124681in}}%
\pgfpathlineto{\pgfqpoint{5.637881in}{2.134283in}}%
\pgfpathclose%
\pgfusepath{fill}%
\end{pgfscope}%
\begin{pgfscope}%
\pgfpathrectangle{\pgfqpoint{1.150000in}{0.150000in}}{\pgfqpoint{5.700000in}{5.700000in}}%
\pgfusepath{clip}%
\pgfsetbuttcap%
\pgfsetroundjoin%
\definecolor{currentfill}{rgb}{0.282327,0.094955,0.417331}%
\pgfsetfillcolor{currentfill}%
\pgfsetfillopacity{0.700000}%
\pgfsetlinewidth{0.000000pt}%
\definecolor{currentstroke}{rgb}{0.000000,0.000000,0.000000}%
\pgfsetstrokecolor{currentstroke}%
\pgfsetdash{}{0pt}%
\pgfpathmoveto{\pgfqpoint{5.056520in}{1.916718in}}%
\pgfpathlineto{\pgfqpoint{5.070683in}{1.915040in}}%
\pgfpathlineto{\pgfqpoint{5.084854in}{1.913386in}}%
\pgfpathlineto{\pgfqpoint{5.099034in}{1.911757in}}%
\pgfpathlineto{\pgfqpoint{5.113221in}{1.910153in}}%
\pgfpathlineto{\pgfqpoint{5.105531in}{1.899502in}}%
\pgfpathlineto{\pgfqpoint{5.097836in}{1.888824in}}%
\pgfpathlineto{\pgfqpoint{5.090136in}{1.878124in}}%
\pgfpathlineto{\pgfqpoint{5.082430in}{1.867404in}}%
\pgfpathlineto{\pgfqpoint{5.068235in}{1.869160in}}%
\pgfpathlineto{\pgfqpoint{5.054048in}{1.870941in}}%
\pgfpathlineto{\pgfqpoint{5.039869in}{1.872746in}}%
\pgfpathlineto{\pgfqpoint{5.025698in}{1.874576in}}%
\pgfpathlineto{\pgfqpoint{5.033412in}{1.885139in}}%
\pgfpathlineto{\pgfqpoint{5.041120in}{1.895686in}}%
\pgfpathlineto{\pgfqpoint{5.048823in}{1.906213in}}%
\pgfpathlineto{\pgfqpoint{5.056520in}{1.916718in}}%
\pgfpathclose%
\pgfusepath{fill}%
\end{pgfscope}%
\begin{pgfscope}%
\pgfpathrectangle{\pgfqpoint{1.150000in}{0.150000in}}{\pgfqpoint{5.700000in}{5.700000in}}%
\pgfusepath{clip}%
\pgfsetbuttcap%
\pgfsetroundjoin%
\definecolor{currentfill}{rgb}{0.282910,0.105393,0.426902}%
\pgfsetfillcolor{currentfill}%
\pgfsetfillopacity{0.700000}%
\pgfsetlinewidth{0.000000pt}%
\definecolor{currentstroke}{rgb}{0.000000,0.000000,0.000000}%
\pgfsetstrokecolor{currentstroke}%
\pgfsetdash{}{0pt}%
\pgfpathmoveto{\pgfqpoint{3.504992in}{1.921370in}}%
\pgfpathlineto{\pgfqpoint{3.518778in}{1.914843in}}%
\pgfpathlineto{\pgfqpoint{3.532568in}{1.908344in}}%
\pgfpathlineto{\pgfqpoint{3.546363in}{1.901874in}}%
\pgfpathlineto{\pgfqpoint{3.560162in}{1.895432in}}%
\pgfpathlineto{\pgfqpoint{3.551858in}{1.895495in}}%
\pgfpathlineto{\pgfqpoint{3.543542in}{1.895878in}}%
\pgfpathlineto{\pgfqpoint{3.535214in}{1.896588in}}%
\pgfpathlineto{\pgfqpoint{3.526873in}{1.897635in}}%
\pgfpathlineto{\pgfqpoint{3.513046in}{1.904389in}}%
\pgfpathlineto{\pgfqpoint{3.499224in}{1.911172in}}%
\pgfpathlineto{\pgfqpoint{3.485407in}{1.917983in}}%
\pgfpathlineto{\pgfqpoint{3.471594in}{1.924822in}}%
\pgfpathlineto{\pgfqpoint{3.479963in}{1.923458in}}%
\pgfpathlineto{\pgfqpoint{3.488319in}{1.922433in}}%
\pgfpathlineto{\pgfqpoint{3.496662in}{1.921740in}}%
\pgfpathlineto{\pgfqpoint{3.504992in}{1.921370in}}%
\pgfpathclose%
\pgfusepath{fill}%
\end{pgfscope}%
\begin{pgfscope}%
\pgfpathrectangle{\pgfqpoint{1.150000in}{0.150000in}}{\pgfqpoint{5.700000in}{5.700000in}}%
\pgfusepath{clip}%
\pgfsetbuttcap%
\pgfsetroundjoin%
\definecolor{currentfill}{rgb}{0.237441,0.305202,0.541921}%
\pgfsetfillcolor{currentfill}%
\pgfsetfillopacity{0.700000}%
\pgfsetlinewidth{0.000000pt}%
\definecolor{currentstroke}{rgb}{0.000000,0.000000,0.000000}%
\pgfsetstrokecolor{currentstroke}%
\pgfsetdash{}{0pt}%
\pgfpathmoveto{\pgfqpoint{2.742492in}{2.346452in}}%
\pgfpathlineto{\pgfqpoint{2.756202in}{2.337357in}}%
\pgfpathlineto{\pgfqpoint{2.769914in}{2.328300in}}%
\pgfpathlineto{\pgfqpoint{2.783629in}{2.319282in}}%
\pgfpathlineto{\pgfqpoint{2.797347in}{2.310302in}}%
\pgfpathlineto{\pgfqpoint{2.788433in}{2.319017in}}%
\pgfpathlineto{\pgfqpoint{2.779495in}{2.328207in}}%
\pgfpathlineto{\pgfqpoint{2.770533in}{2.337883in}}%
\pgfpathlineto{\pgfqpoint{2.761545in}{2.348054in}}%
\pgfpathlineto{\pgfqpoint{2.747785in}{2.357396in}}%
\pgfpathlineto{\pgfqpoint{2.734027in}{2.366777in}}%
\pgfpathlineto{\pgfqpoint{2.720271in}{2.376197in}}%
\pgfpathlineto{\pgfqpoint{2.706518in}{2.385655in}}%
\pgfpathlineto{\pgfqpoint{2.715549in}{2.375115in}}%
\pgfpathlineto{\pgfqpoint{2.724555in}{2.365075in}}%
\pgfpathlineto{\pgfqpoint{2.733536in}{2.355524in}}%
\pgfpathlineto{\pgfqpoint{2.742492in}{2.346452in}}%
\pgfpathclose%
\pgfusepath{fill}%
\end{pgfscope}%
\begin{pgfscope}%
\pgfpathrectangle{\pgfqpoint{1.150000in}{0.150000in}}{\pgfqpoint{5.700000in}{5.700000in}}%
\pgfusepath{clip}%
\pgfsetbuttcap%
\pgfsetroundjoin%
\definecolor{currentfill}{rgb}{0.269944,0.014625,0.341379}%
\pgfsetfillcolor{currentfill}%
\pgfsetfillopacity{0.700000}%
\pgfsetlinewidth{0.000000pt}%
\definecolor{currentstroke}{rgb}{0.000000,0.000000,0.000000}%
\pgfsetstrokecolor{currentstroke}%
\pgfsetdash{}{0pt}%
\pgfpathmoveto{\pgfqpoint{4.419635in}{1.768414in}}%
\pgfpathlineto{\pgfqpoint{4.433611in}{1.764841in}}%
\pgfpathlineto{\pgfqpoint{4.447594in}{1.761294in}}%
\pgfpathlineto{\pgfqpoint{4.461584in}{1.757771in}}%
\pgfpathlineto{\pgfqpoint{4.475581in}{1.754273in}}%
\pgfpathlineto{\pgfqpoint{4.467700in}{1.745949in}}%
\pgfpathlineto{\pgfqpoint{4.459813in}{1.737734in}}%
\pgfpathlineto{\pgfqpoint{4.451922in}{1.729635in}}%
\pgfpathlineto{\pgfqpoint{4.444025in}{1.721656in}}%
\pgfpathlineto{\pgfqpoint{4.430016in}{1.725385in}}%
\pgfpathlineto{\pgfqpoint{4.416014in}{1.729138in}}%
\pgfpathlineto{\pgfqpoint{4.402019in}{1.732916in}}%
\pgfpathlineto{\pgfqpoint{4.388030in}{1.736719in}}%
\pgfpathlineto{\pgfqpoint{4.395939in}{1.744462in}}%
\pgfpathlineto{\pgfqpoint{4.403843in}{1.752329in}}%
\pgfpathlineto{\pgfqpoint{4.411741in}{1.760315in}}%
\pgfpathlineto{\pgfqpoint{4.419635in}{1.768414in}}%
\pgfpathclose%
\pgfusepath{fill}%
\end{pgfscope}%
\begin{pgfscope}%
\pgfpathrectangle{\pgfqpoint{1.150000in}{0.150000in}}{\pgfqpoint{5.700000in}{5.700000in}}%
\pgfusepath{clip}%
\pgfsetbuttcap%
\pgfsetroundjoin%
\definecolor{currentfill}{rgb}{0.273809,0.031497,0.358853}%
\pgfsetfillcolor{currentfill}%
\pgfsetfillopacity{0.700000}%
\pgfsetlinewidth{0.000000pt}%
\definecolor{currentstroke}{rgb}{0.000000,0.000000,0.000000}%
\pgfsetstrokecolor{currentstroke}%
\pgfsetdash{}{0pt}%
\pgfpathmoveto{\pgfqpoint{4.650508in}{1.800431in}}%
\pgfpathlineto{\pgfqpoint{4.664548in}{1.797577in}}%
\pgfpathlineto{\pgfqpoint{4.678596in}{1.794747in}}%
\pgfpathlineto{\pgfqpoint{4.692651in}{1.791943in}}%
\pgfpathlineto{\pgfqpoint{4.706713in}{1.789162in}}%
\pgfpathlineto{\pgfqpoint{4.698902in}{1.779559in}}%
\pgfpathlineto{\pgfqpoint{4.691087in}{1.770014in}}%
\pgfpathlineto{\pgfqpoint{4.683266in}{1.760530in}}%
\pgfpathlineto{\pgfqpoint{4.675441in}{1.751113in}}%
\pgfpathlineto{\pgfqpoint{4.661369in}{1.754098in}}%
\pgfpathlineto{\pgfqpoint{4.647304in}{1.757107in}}%
\pgfpathlineto{\pgfqpoint{4.633247in}{1.760141in}}%
\pgfpathlineto{\pgfqpoint{4.619197in}{1.763199in}}%
\pgfpathlineto{\pgfqpoint{4.627032in}{1.772406in}}%
\pgfpathlineto{\pgfqpoint{4.634862in}{1.781684in}}%
\pgfpathlineto{\pgfqpoint{4.642687in}{1.791027in}}%
\pgfpathlineto{\pgfqpoint{4.650508in}{1.800431in}}%
\pgfpathclose%
\pgfusepath{fill}%
\end{pgfscope}%
\begin{pgfscope}%
\pgfpathrectangle{\pgfqpoint{1.150000in}{0.150000in}}{\pgfqpoint{5.700000in}{5.700000in}}%
\pgfusepath{clip}%
\pgfsetbuttcap%
\pgfsetroundjoin%
\definecolor{currentfill}{rgb}{0.270595,0.214069,0.507052}%
\pgfsetfillcolor{currentfill}%
\pgfsetfillopacity{0.700000}%
\pgfsetlinewidth{0.000000pt}%
\definecolor{currentstroke}{rgb}{0.000000,0.000000,0.000000}%
\pgfsetstrokecolor{currentstroke}%
\pgfsetdash{}{0pt}%
\pgfpathmoveto{\pgfqpoint{3.052043in}{2.146765in}}%
\pgfpathlineto{\pgfqpoint{3.065775in}{2.138733in}}%
\pgfpathlineto{\pgfqpoint{3.079511in}{2.130735in}}%
\pgfpathlineto{\pgfqpoint{3.093250in}{2.122769in}}%
\pgfpathlineto{\pgfqpoint{3.106993in}{2.114837in}}%
\pgfpathlineto{\pgfqpoint{3.098354in}{2.120056in}}%
\pgfpathlineto{\pgfqpoint{3.089697in}{2.125691in}}%
\pgfpathlineto{\pgfqpoint{3.081020in}{2.131752in}}%
\pgfpathlineto{\pgfqpoint{3.072324in}{2.138248in}}%
\pgfpathlineto{\pgfqpoint{3.058545in}{2.146524in}}%
\pgfpathlineto{\pgfqpoint{3.044769in}{2.154834in}}%
\pgfpathlineto{\pgfqpoint{3.030997in}{2.163177in}}%
\pgfpathlineto{\pgfqpoint{3.017228in}{2.171553in}}%
\pgfpathlineto{\pgfqpoint{3.025961in}{2.164707in}}%
\pgfpathlineto{\pgfqpoint{3.034674in}{2.158300in}}%
\pgfpathlineto{\pgfqpoint{3.043368in}{2.152322in}}%
\pgfpathlineto{\pgfqpoint{3.052043in}{2.146765in}}%
\pgfpathclose%
\pgfusepath{fill}%
\end{pgfscope}%
\begin{pgfscope}%
\pgfpathrectangle{\pgfqpoint{1.150000in}{0.150000in}}{\pgfqpoint{5.700000in}{5.700000in}}%
\pgfusepath{clip}%
\pgfsetbuttcap%
\pgfsetroundjoin%
\definecolor{currentfill}{rgb}{0.279566,0.067836,0.391917}%
\pgfsetfillcolor{currentfill}%
\pgfsetfillopacity{0.700000}%
\pgfsetlinewidth{0.000000pt}%
\definecolor{currentstroke}{rgb}{0.000000,0.000000,0.000000}%
\pgfsetstrokecolor{currentstroke}%
\pgfsetdash{}{0pt}%
\pgfpathmoveto{\pgfqpoint{3.703635in}{1.850073in}}%
\pgfpathlineto{\pgfqpoint{3.717455in}{1.844180in}}%
\pgfpathlineto{\pgfqpoint{3.731280in}{1.838314in}}%
\pgfpathlineto{\pgfqpoint{3.745110in}{1.832475in}}%
\pgfpathlineto{\pgfqpoint{3.758945in}{1.826664in}}%
\pgfpathlineto{\pgfqpoint{3.750758in}{1.824651in}}%
\pgfpathlineto{\pgfqpoint{3.742562in}{1.822915in}}%
\pgfpathlineto{\pgfqpoint{3.734355in}{1.821464in}}%
\pgfpathlineto{\pgfqpoint{3.726137in}{1.820306in}}%
\pgfpathlineto{\pgfqpoint{3.712279in}{1.826415in}}%
\pgfpathlineto{\pgfqpoint{3.698425in}{1.832552in}}%
\pgfpathlineto{\pgfqpoint{3.684576in}{1.838716in}}%
\pgfpathlineto{\pgfqpoint{3.670733in}{1.844907in}}%
\pgfpathlineto{\pgfqpoint{3.678974in}{1.845762in}}%
\pgfpathlineto{\pgfqpoint{3.687205in}{1.846914in}}%
\pgfpathlineto{\pgfqpoint{3.695425in}{1.848353in}}%
\pgfpathlineto{\pgfqpoint{3.703635in}{1.850073in}}%
\pgfpathclose%
\pgfusepath{fill}%
\end{pgfscope}%
\begin{pgfscope}%
\pgfpathrectangle{\pgfqpoint{1.150000in}{0.150000in}}{\pgfqpoint{5.700000in}{5.700000in}}%
\pgfusepath{clip}%
\pgfsetbuttcap%
\pgfsetroundjoin%
\definecolor{currentfill}{rgb}{0.280894,0.078907,0.402329}%
\pgfsetfillcolor{currentfill}%
\pgfsetfillopacity{0.700000}%
\pgfsetlinewidth{0.000000pt}%
\definecolor{currentstroke}{rgb}{0.000000,0.000000,0.000000}%
\pgfsetstrokecolor{currentstroke}%
\pgfsetdash{}{0pt}%
\pgfpathmoveto{\pgfqpoint{4.969098in}{1.882138in}}%
\pgfpathlineto{\pgfqpoint{4.983236in}{1.880211in}}%
\pgfpathlineto{\pgfqpoint{4.997382in}{1.878308in}}%
\pgfpathlineto{\pgfqpoint{5.011536in}{1.876430in}}%
\pgfpathlineto{\pgfqpoint{5.025698in}{1.874576in}}%
\pgfpathlineto{\pgfqpoint{5.017980in}{1.864001in}}%
\pgfpathlineto{\pgfqpoint{5.010256in}{1.853417in}}%
\pgfpathlineto{\pgfqpoint{5.002527in}{1.842829in}}%
\pgfpathlineto{\pgfqpoint{4.994793in}{1.832240in}}%
\pgfpathlineto{\pgfqpoint{4.980623in}{1.834259in}}%
\pgfpathlineto{\pgfqpoint{4.966461in}{1.836302in}}%
\pgfpathlineto{\pgfqpoint{4.952308in}{1.838370in}}%
\pgfpathlineto{\pgfqpoint{4.938162in}{1.840462in}}%
\pgfpathlineto{\pgfqpoint{4.945903in}{1.850881in}}%
\pgfpathlineto{\pgfqpoint{4.953640in}{1.861303in}}%
\pgfpathlineto{\pgfqpoint{4.961372in}{1.871723in}}%
\pgfpathlineto{\pgfqpoint{4.969098in}{1.882138in}}%
\pgfpathclose%
\pgfusepath{fill}%
\end{pgfscope}%
\begin{pgfscope}%
\pgfpathrectangle{\pgfqpoint{1.150000in}{0.150000in}}{\pgfqpoint{5.700000in}{5.700000in}}%
\pgfusepath{clip}%
\pgfsetbuttcap%
\pgfsetroundjoin%
\definecolor{currentfill}{rgb}{0.277134,0.185228,0.489898}%
\pgfsetfillcolor{currentfill}%
\pgfsetfillopacity{0.700000}%
\pgfsetlinewidth{0.000000pt}%
\definecolor{currentstroke}{rgb}{0.000000,0.000000,0.000000}%
\pgfsetstrokecolor{currentstroke}%
\pgfsetdash{}{0pt}%
\pgfpathmoveto{\pgfqpoint{5.550526in}{2.097252in}}%
\pgfpathlineto{\pgfqpoint{5.564858in}{2.096718in}}%
\pgfpathlineto{\pgfqpoint{5.579200in}{2.096208in}}%
\pgfpathlineto{\pgfqpoint{5.593551in}{2.095724in}}%
\pgfpathlineto{\pgfqpoint{5.607911in}{2.095263in}}%
\pgfpathlineto{\pgfqpoint{5.600399in}{2.085258in}}%
\pgfpathlineto{\pgfqpoint{5.592879in}{2.075155in}}%
\pgfpathlineto{\pgfqpoint{5.585352in}{2.064956in}}%
\pgfpathlineto{\pgfqpoint{5.577817in}{2.054663in}}%
\pgfpathlineto{\pgfqpoint{5.563448in}{2.055207in}}%
\pgfpathlineto{\pgfqpoint{5.549089in}{2.055776in}}%
\pgfpathlineto{\pgfqpoint{5.534739in}{2.056369in}}%
\pgfpathlineto{\pgfqpoint{5.520398in}{2.056987in}}%
\pgfpathlineto{\pgfqpoint{5.527941in}{2.067191in}}%
\pgfpathlineto{\pgfqpoint{5.535477in}{2.077304in}}%
\pgfpathlineto{\pgfqpoint{5.543005in}{2.087325in}}%
\pgfpathlineto{\pgfqpoint{5.550526in}{2.097252in}}%
\pgfpathclose%
\pgfusepath{fill}%
\end{pgfscope}%
\begin{pgfscope}%
\pgfpathrectangle{\pgfqpoint{1.150000in}{0.150000in}}{\pgfqpoint{5.700000in}{5.700000in}}%
\pgfusepath{clip}%
\pgfsetbuttcap%
\pgfsetroundjoin%
\definecolor{currentfill}{rgb}{0.281887,0.150881,0.465405}%
\pgfsetfillcolor{currentfill}%
\pgfsetfillopacity{0.700000}%
\pgfsetlinewidth{0.000000pt}%
\definecolor{currentstroke}{rgb}{0.000000,0.000000,0.000000}%
\pgfsetstrokecolor{currentstroke}%
\pgfsetdash{}{0pt}%
\pgfpathmoveto{\pgfqpoint{3.306182in}{2.009182in}}%
\pgfpathlineto{\pgfqpoint{3.319943in}{2.001988in}}%
\pgfpathlineto{\pgfqpoint{3.333708in}{1.994824in}}%
\pgfpathlineto{\pgfqpoint{3.347477in}{1.987691in}}%
\pgfpathlineto{\pgfqpoint{3.361250in}{1.980587in}}%
\pgfpathlineto{\pgfqpoint{3.352808in}{1.982944in}}%
\pgfpathlineto{\pgfqpoint{3.344352in}{1.985665in}}%
\pgfpathlineto{\pgfqpoint{3.335880in}{1.988760in}}%
\pgfpathlineto{\pgfqpoint{3.327393in}{1.992238in}}%
\pgfpathlineto{\pgfqpoint{3.313588in}{1.999669in}}%
\pgfpathlineto{\pgfqpoint{3.299788in}{2.007130in}}%
\pgfpathlineto{\pgfqpoint{3.285991in}{2.014622in}}%
\pgfpathlineto{\pgfqpoint{3.272199in}{2.022143in}}%
\pgfpathlineto{\pgfqpoint{3.280718in}{2.018333in}}%
\pgfpathlineto{\pgfqpoint{3.289222in}{2.014908in}}%
\pgfpathlineto{\pgfqpoint{3.297710in}{2.011861in}}%
\pgfpathlineto{\pgfqpoint{3.306182in}{2.009182in}}%
\pgfpathclose%
\pgfusepath{fill}%
\end{pgfscope}%
\begin{pgfscope}%
\pgfpathrectangle{\pgfqpoint{1.150000in}{0.150000in}}{\pgfqpoint{5.700000in}{5.700000in}}%
\pgfusepath{clip}%
\pgfsetbuttcap%
\pgfsetroundjoin%
\definecolor{currentfill}{rgb}{0.185556,0.418570,0.556753}%
\pgfsetfillcolor{currentfill}%
\pgfsetfillopacity{0.700000}%
\pgfsetlinewidth{0.000000pt}%
\definecolor{currentstroke}{rgb}{0.000000,0.000000,0.000000}%
\pgfsetstrokecolor{currentstroke}%
\pgfsetdash{}{0pt}%
\pgfpathmoveto{\pgfqpoint{2.377051in}{2.625151in}}%
\pgfpathlineto{\pgfqpoint{2.390759in}{2.614657in}}%
\pgfpathlineto{\pgfqpoint{2.404470in}{2.604211in}}%
\pgfpathlineto{\pgfqpoint{2.418181in}{2.593812in}}%
\pgfpathlineto{\pgfqpoint{2.431894in}{2.583461in}}%
\pgfpathlineto{\pgfqpoint{2.422600in}{2.596387in}}%
\pgfpathlineto{\pgfqpoint{2.413276in}{2.609856in}}%
\pgfpathlineto{\pgfqpoint{2.403920in}{2.623878in}}%
\pgfpathlineto{\pgfqpoint{2.394532in}{2.638464in}}%
\pgfpathlineto{\pgfqpoint{2.380769in}{2.649200in}}%
\pgfpathlineto{\pgfqpoint{2.367007in}{2.659983in}}%
\pgfpathlineto{\pgfqpoint{2.353246in}{2.670814in}}%
\pgfpathlineto{\pgfqpoint{2.339486in}{2.681693in}}%
\pgfpathlineto{\pgfqpoint{2.348926in}{2.666715in}}%
\pgfpathlineto{\pgfqpoint{2.358333in}{2.652307in}}%
\pgfpathlineto{\pgfqpoint{2.367707in}{2.638456in}}%
\pgfpathlineto{\pgfqpoint{2.377051in}{2.625151in}}%
\pgfpathclose%
\pgfusepath{fill}%
\end{pgfscope}%
\begin{pgfscope}%
\pgfpathrectangle{\pgfqpoint{1.150000in}{0.150000in}}{\pgfqpoint{5.700000in}{5.700000in}}%
\pgfusepath{clip}%
\pgfsetbuttcap%
\pgfsetroundjoin%
\definecolor{currentfill}{rgb}{0.279574,0.170599,0.479997}%
\pgfsetfillcolor{currentfill}%
\pgfsetfillopacity{0.700000}%
\pgfsetlinewidth{0.000000pt}%
\definecolor{currentstroke}{rgb}{0.000000,0.000000,0.000000}%
\pgfsetstrokecolor{currentstroke}%
\pgfsetdash{}{0pt}%
\pgfpathmoveto{\pgfqpoint{5.463125in}{2.059704in}}%
\pgfpathlineto{\pgfqpoint{5.477430in}{2.058988in}}%
\pgfpathlineto{\pgfqpoint{5.491743in}{2.058296in}}%
\pgfpathlineto{\pgfqpoint{5.506066in}{2.057629in}}%
\pgfpathlineto{\pgfqpoint{5.520398in}{2.056987in}}%
\pgfpathlineto{\pgfqpoint{5.512847in}{2.046694in}}%
\pgfpathlineto{\pgfqpoint{5.505289in}{2.036315in}}%
\pgfpathlineto{\pgfqpoint{5.497725in}{2.025851in}}%
\pgfpathlineto{\pgfqpoint{5.490153in}{2.015305in}}%
\pgfpathlineto{\pgfqpoint{5.475813in}{2.016045in}}%
\pgfpathlineto{\pgfqpoint{5.461482in}{2.016809in}}%
\pgfpathlineto{\pgfqpoint{5.447160in}{2.017598in}}%
\pgfpathlineto{\pgfqpoint{5.432848in}{2.018412in}}%
\pgfpathlineto{\pgfqpoint{5.440428in}{2.028856in}}%
\pgfpathlineto{\pgfqpoint{5.448000in}{2.039221in}}%
\pgfpathlineto{\pgfqpoint{5.455566in}{2.049504in}}%
\pgfpathlineto{\pgfqpoint{5.463125in}{2.059704in}}%
\pgfpathclose%
\pgfusepath{fill}%
\end{pgfscope}%
\begin{pgfscope}%
\pgfpathrectangle{\pgfqpoint{1.150000in}{0.150000in}}{\pgfqpoint{5.700000in}{5.700000in}}%
\pgfusepath{clip}%
\pgfsetbuttcap%
\pgfsetroundjoin%
\definecolor{currentfill}{rgb}{0.278791,0.062145,0.386592}%
\pgfsetfillcolor{currentfill}%
\pgfsetfillopacity{0.700000}%
\pgfsetlinewidth{0.000000pt}%
\definecolor{currentstroke}{rgb}{0.000000,0.000000,0.000000}%
\pgfsetstrokecolor{currentstroke}%
\pgfsetdash{}{0pt}%
\pgfpathmoveto{\pgfqpoint{4.881658in}{1.849076in}}%
\pgfpathlineto{\pgfqpoint{4.895772in}{1.846886in}}%
\pgfpathlineto{\pgfqpoint{4.909894in}{1.844720in}}%
\pgfpathlineto{\pgfqpoint{4.924024in}{1.842579in}}%
\pgfpathlineto{\pgfqpoint{4.938162in}{1.840462in}}%
\pgfpathlineto{\pgfqpoint{4.930415in}{1.830050in}}%
\pgfpathlineto{\pgfqpoint{4.922664in}{1.819649in}}%
\pgfpathlineto{\pgfqpoint{4.914907in}{1.809262in}}%
\pgfpathlineto{\pgfqpoint{4.907146in}{1.798895in}}%
\pgfpathlineto{\pgfqpoint{4.893000in}{1.801190in}}%
\pgfpathlineto{\pgfqpoint{4.878862in}{1.803510in}}%
\pgfpathlineto{\pgfqpoint{4.864732in}{1.805854in}}%
\pgfpathlineto{\pgfqpoint{4.850610in}{1.808222in}}%
\pgfpathlineto{\pgfqpoint{4.858379in}{1.818406in}}%
\pgfpathlineto{\pgfqpoint{4.866143in}{1.828612in}}%
\pgfpathlineto{\pgfqpoint{4.873903in}{1.838837in}}%
\pgfpathlineto{\pgfqpoint{4.881658in}{1.849076in}}%
\pgfpathclose%
\pgfusepath{fill}%
\end{pgfscope}%
\begin{pgfscope}%
\pgfpathrectangle{\pgfqpoint{1.150000in}{0.150000in}}{\pgfqpoint{5.700000in}{5.700000in}}%
\pgfusepath{clip}%
\pgfsetbuttcap%
\pgfsetroundjoin%
\definecolor{currentfill}{rgb}{0.271305,0.019942,0.347269}%
\pgfsetfillcolor{currentfill}%
\pgfsetfillopacity{0.700000}%
\pgfsetlinewidth{0.000000pt}%
\definecolor{currentstroke}{rgb}{0.000000,0.000000,0.000000}%
\pgfsetstrokecolor{currentstroke}%
\pgfsetdash{}{0pt}%
\pgfpathmoveto{\pgfqpoint{4.045570in}{1.771330in}}%
\pgfpathlineto{\pgfqpoint{4.059462in}{1.766520in}}%
\pgfpathlineto{\pgfqpoint{4.073359in}{1.761737in}}%
\pgfpathlineto{\pgfqpoint{4.087262in}{1.756979in}}%
\pgfpathlineto{\pgfqpoint{4.101171in}{1.752247in}}%
\pgfpathlineto{\pgfqpoint{4.093149in}{1.746962in}}%
\pgfpathlineto{\pgfqpoint{4.085120in}{1.741878in}}%
\pgfpathlineto{\pgfqpoint{4.077084in}{1.737002in}}%
\pgfpathlineto{\pgfqpoint{4.069040in}{1.732341in}}%
\pgfpathlineto{\pgfqpoint{4.055114in}{1.737344in}}%
\pgfpathlineto{\pgfqpoint{4.041193in}{1.742372in}}%
\pgfpathlineto{\pgfqpoint{4.027278in}{1.747425in}}%
\pgfpathlineto{\pgfqpoint{4.013368in}{1.752505in}}%
\pgfpathlineto{\pgfqpoint{4.021430in}{1.756890in}}%
\pgfpathlineto{\pgfqpoint{4.029484in}{1.761494in}}%
\pgfpathlineto{\pgfqpoint{4.037531in}{1.766309in}}%
\pgfpathlineto{\pgfqpoint{4.045570in}{1.771330in}}%
\pgfpathclose%
\pgfusepath{fill}%
\end{pgfscope}%
\begin{pgfscope}%
\pgfpathrectangle{\pgfqpoint{1.150000in}{0.150000in}}{\pgfqpoint{5.700000in}{5.700000in}}%
\pgfusepath{clip}%
\pgfsetbuttcap%
\pgfsetroundjoin%
\definecolor{currentfill}{rgb}{0.269944,0.014625,0.341379}%
\pgfsetfillcolor{currentfill}%
\pgfsetfillopacity{0.700000}%
\pgfsetlinewidth{0.000000pt}%
\definecolor{currentstroke}{rgb}{0.000000,0.000000,0.000000}%
\pgfsetstrokecolor{currentstroke}%
\pgfsetdash{}{0pt}%
\pgfpathmoveto{\pgfqpoint{4.188827in}{1.757621in}}%
\pgfpathlineto{\pgfqpoint{4.202751in}{1.753273in}}%
\pgfpathlineto{\pgfqpoint{4.216681in}{1.748951in}}%
\pgfpathlineto{\pgfqpoint{4.230618in}{1.744654in}}%
\pgfpathlineto{\pgfqpoint{4.244561in}{1.740382in}}%
\pgfpathlineto{\pgfqpoint{4.236597in}{1.733855in}}%
\pgfpathlineto{\pgfqpoint{4.228626in}{1.727496in}}%
\pgfpathlineto{\pgfqpoint{4.220649in}{1.721311in}}%
\pgfpathlineto{\pgfqpoint{4.212666in}{1.715307in}}%
\pgfpathlineto{\pgfqpoint{4.198707in}{1.719836in}}%
\pgfpathlineto{\pgfqpoint{4.184755in}{1.724390in}}%
\pgfpathlineto{\pgfqpoint{4.170809in}{1.728970in}}%
\pgfpathlineto{\pgfqpoint{4.156869in}{1.733574in}}%
\pgfpathlineto{\pgfqpoint{4.164868in}{1.739316in}}%
\pgfpathlineto{\pgfqpoint{4.172861in}{1.745242in}}%
\pgfpathlineto{\pgfqpoint{4.180847in}{1.751346in}}%
\pgfpathlineto{\pgfqpoint{4.188827in}{1.757621in}}%
\pgfpathclose%
\pgfusepath{fill}%
\end{pgfscope}%
\begin{pgfscope}%
\pgfpathrectangle{\pgfqpoint{1.150000in}{0.150000in}}{\pgfqpoint{5.700000in}{5.700000in}}%
\pgfusepath{clip}%
\pgfsetbuttcap%
\pgfsetroundjoin%
\definecolor{currentfill}{rgb}{0.281887,0.150881,0.465405}%
\pgfsetfillcolor{currentfill}%
\pgfsetfillopacity{0.700000}%
\pgfsetlinewidth{0.000000pt}%
\definecolor{currentstroke}{rgb}{0.000000,0.000000,0.000000}%
\pgfsetstrokecolor{currentstroke}%
\pgfsetdash{}{0pt}%
\pgfpathmoveto{\pgfqpoint{5.375687in}{2.021913in}}%
\pgfpathlineto{\pgfqpoint{5.389964in}{2.021001in}}%
\pgfpathlineto{\pgfqpoint{5.404249in}{2.020113in}}%
\pgfpathlineto{\pgfqpoint{5.418544in}{2.019250in}}%
\pgfpathlineto{\pgfqpoint{5.432848in}{2.018412in}}%
\pgfpathlineto{\pgfqpoint{5.425261in}{2.007891in}}%
\pgfpathlineto{\pgfqpoint{5.417668in}{1.997296in}}%
\pgfpathlineto{\pgfqpoint{5.410068in}{1.986628in}}%
\pgfpathlineto{\pgfqpoint{5.402461in}{1.975891in}}%
\pgfpathlineto{\pgfqpoint{5.388150in}{1.976840in}}%
\pgfpathlineto{\pgfqpoint{5.373848in}{1.977814in}}%
\pgfpathlineto{\pgfqpoint{5.359554in}{1.978813in}}%
\pgfpathlineto{\pgfqpoint{5.345270in}{1.979836in}}%
\pgfpathlineto{\pgfqpoint{5.352884in}{1.990457in}}%
\pgfpathlineto{\pgfqpoint{5.360491in}{2.001012in}}%
\pgfpathlineto{\pgfqpoint{5.368092in}{2.011498in}}%
\pgfpathlineto{\pgfqpoint{5.375687in}{2.021913in}}%
\pgfpathclose%
\pgfusepath{fill}%
\end{pgfscope}%
\begin{pgfscope}%
\pgfpathrectangle{\pgfqpoint{1.150000in}{0.150000in}}{\pgfqpoint{5.700000in}{5.700000in}}%
\pgfusepath{clip}%
\pgfsetbuttcap%
\pgfsetroundjoin%
\definecolor{currentfill}{rgb}{0.271305,0.019942,0.347269}%
\pgfsetfillcolor{currentfill}%
\pgfsetfillopacity{0.700000}%
\pgfsetlinewidth{0.000000pt}%
\definecolor{currentstroke}{rgb}{0.000000,0.000000,0.000000}%
\pgfsetstrokecolor{currentstroke}%
\pgfsetdash{}{0pt}%
\pgfpathmoveto{\pgfqpoint{4.563069in}{1.775678in}}%
\pgfpathlineto{\pgfqpoint{4.577090in}{1.772521in}}%
\pgfpathlineto{\pgfqpoint{4.591118in}{1.769389in}}%
\pgfpathlineto{\pgfqpoint{4.605154in}{1.766282in}}%
\pgfpathlineto{\pgfqpoint{4.619197in}{1.763199in}}%
\pgfpathlineto{\pgfqpoint{4.611357in}{1.754067in}}%
\pgfpathlineto{\pgfqpoint{4.603513in}{1.745017in}}%
\pgfpathlineto{\pgfqpoint{4.595664in}{1.736052in}}%
\pgfpathlineto{\pgfqpoint{4.587810in}{1.727179in}}%
\pgfpathlineto{\pgfqpoint{4.573756in}{1.730480in}}%
\pgfpathlineto{\pgfqpoint{4.559710in}{1.733805in}}%
\pgfpathlineto{\pgfqpoint{4.545671in}{1.737155in}}%
\pgfpathlineto{\pgfqpoint{4.531639in}{1.740529in}}%
\pgfpathlineto{\pgfqpoint{4.539504in}{1.749179in}}%
\pgfpathlineto{\pgfqpoint{4.547364in}{1.757925in}}%
\pgfpathlineto{\pgfqpoint{4.555219in}{1.766759in}}%
\pgfpathlineto{\pgfqpoint{4.563069in}{1.775678in}}%
\pgfpathclose%
\pgfusepath{fill}%
\end{pgfscope}%
\begin{pgfscope}%
\pgfpathrectangle{\pgfqpoint{1.150000in}{0.150000in}}{\pgfqpoint{5.700000in}{5.700000in}}%
\pgfusepath{clip}%
\pgfsetbuttcap%
\pgfsetroundjoin%
\definecolor{currentfill}{rgb}{0.243113,0.292092,0.538516}%
\pgfsetfillcolor{currentfill}%
\pgfsetfillopacity{0.700000}%
\pgfsetlinewidth{0.000000pt}%
\definecolor{currentstroke}{rgb}{0.000000,0.000000,0.000000}%
\pgfsetstrokecolor{currentstroke}%
\pgfsetdash{}{0pt}%
\pgfpathmoveto{\pgfqpoint{2.797347in}{2.310302in}}%
\pgfpathlineto{\pgfqpoint{2.811068in}{2.301359in}}%
\pgfpathlineto{\pgfqpoint{2.824791in}{2.292454in}}%
\pgfpathlineto{\pgfqpoint{2.838517in}{2.283585in}}%
\pgfpathlineto{\pgfqpoint{2.852246in}{2.274754in}}%
\pgfpathlineto{\pgfqpoint{2.843373in}{2.283113in}}%
\pgfpathlineto{\pgfqpoint{2.834477in}{2.291944in}}%
\pgfpathlineto{\pgfqpoint{2.825557in}{2.301255in}}%
\pgfpathlineto{\pgfqpoint{2.816613in}{2.311058in}}%
\pgfpathlineto{\pgfqpoint{2.802842in}{2.320251in}}%
\pgfpathlineto{\pgfqpoint{2.789074in}{2.329481in}}%
\pgfpathlineto{\pgfqpoint{2.775308in}{2.338749in}}%
\pgfpathlineto{\pgfqpoint{2.761545in}{2.348054in}}%
\pgfpathlineto{\pgfqpoint{2.770533in}{2.337883in}}%
\pgfpathlineto{\pgfqpoint{2.779495in}{2.328207in}}%
\pgfpathlineto{\pgfqpoint{2.788433in}{2.319017in}}%
\pgfpathlineto{\pgfqpoint{2.797347in}{2.310302in}}%
\pgfpathclose%
\pgfusepath{fill}%
\end{pgfscope}%
\begin{pgfscope}%
\pgfpathrectangle{\pgfqpoint{1.150000in}{0.150000in}}{\pgfqpoint{5.700000in}{5.700000in}}%
\pgfusepath{clip}%
\pgfsetbuttcap%
\pgfsetroundjoin%
\definecolor{currentfill}{rgb}{0.274952,0.037752,0.364543}%
\pgfsetfillcolor{currentfill}%
\pgfsetfillopacity{0.700000}%
\pgfsetlinewidth{0.000000pt}%
\definecolor{currentstroke}{rgb}{0.000000,0.000000,0.000000}%
\pgfsetstrokecolor{currentstroke}%
\pgfsetdash{}{0pt}%
\pgfpathmoveto{\pgfqpoint{3.902303in}{1.794074in}}%
\pgfpathlineto{\pgfqpoint{3.916166in}{1.788787in}}%
\pgfpathlineto{\pgfqpoint{3.930035in}{1.783525in}}%
\pgfpathlineto{\pgfqpoint{3.943910in}{1.778290in}}%
\pgfpathlineto{\pgfqpoint{3.957790in}{1.773081in}}%
\pgfpathlineto{\pgfqpoint{3.949702in}{1.769200in}}%
\pgfpathlineto{\pgfqpoint{3.941605in}{1.765556in}}%
\pgfpathlineto{\pgfqpoint{3.933501in}{1.762155in}}%
\pgfpathlineto{\pgfqpoint{3.925388in}{1.759006in}}%
\pgfpathlineto{\pgfqpoint{3.911487in}{1.764499in}}%
\pgfpathlineto{\pgfqpoint{3.897593in}{1.770018in}}%
\pgfpathlineto{\pgfqpoint{3.883704in}{1.775564in}}%
\pgfpathlineto{\pgfqpoint{3.869820in}{1.781135in}}%
\pgfpathlineto{\pgfqpoint{3.877954in}{1.783996in}}%
\pgfpathlineto{\pgfqpoint{3.886079in}{1.787110in}}%
\pgfpathlineto{\pgfqpoint{3.894195in}{1.790473in}}%
\pgfpathlineto{\pgfqpoint{3.902303in}{1.794074in}}%
\pgfpathclose%
\pgfusepath{fill}%
\end{pgfscope}%
\begin{pgfscope}%
\pgfpathrectangle{\pgfqpoint{1.150000in}{0.150000in}}{\pgfqpoint{5.700000in}{5.700000in}}%
\pgfusepath{clip}%
\pgfsetbuttcap%
\pgfsetroundjoin%
\definecolor{currentfill}{rgb}{0.190631,0.407061,0.556089}%
\pgfsetfillcolor{currentfill}%
\pgfsetfillopacity{0.700000}%
\pgfsetlinewidth{0.000000pt}%
\definecolor{currentstroke}{rgb}{0.000000,0.000000,0.000000}%
\pgfsetstrokecolor{currentstroke}%
\pgfsetdash{}{0pt}%
\pgfpathmoveto{\pgfqpoint{2.431894in}{2.583461in}}%
\pgfpathlineto{\pgfqpoint{2.445608in}{2.573156in}}%
\pgfpathlineto{\pgfqpoint{2.459324in}{2.562897in}}%
\pgfpathlineto{\pgfqpoint{2.473041in}{2.552683in}}%
\pgfpathlineto{\pgfqpoint{2.486760in}{2.542515in}}%
\pgfpathlineto{\pgfqpoint{2.477515in}{2.555065in}}%
\pgfpathlineto{\pgfqpoint{2.468240in}{2.568152in}}%
\pgfpathlineto{\pgfqpoint{2.458935in}{2.581788in}}%
\pgfpathlineto{\pgfqpoint{2.449599in}{2.595984in}}%
\pgfpathlineto{\pgfqpoint{2.435830in}{2.606535in}}%
\pgfpathlineto{\pgfqpoint{2.422063in}{2.617132in}}%
\pgfpathlineto{\pgfqpoint{2.408297in}{2.627775in}}%
\pgfpathlineto{\pgfqpoint{2.394532in}{2.638464in}}%
\pgfpathlineto{\pgfqpoint{2.403920in}{2.623878in}}%
\pgfpathlineto{\pgfqpoint{2.413276in}{2.609856in}}%
\pgfpathlineto{\pgfqpoint{2.422600in}{2.596387in}}%
\pgfpathlineto{\pgfqpoint{2.431894in}{2.583461in}}%
\pgfpathclose%
\pgfusepath{fill}%
\end{pgfscope}%
\begin{pgfscope}%
\pgfpathrectangle{\pgfqpoint{1.150000in}{0.150000in}}{\pgfqpoint{5.700000in}{5.700000in}}%
\pgfusepath{clip}%
\pgfsetbuttcap%
\pgfsetroundjoin%
\definecolor{currentfill}{rgb}{0.268510,0.009605,0.335427}%
\pgfsetfillcolor{currentfill}%
\pgfsetfillopacity{0.700000}%
\pgfsetlinewidth{0.000000pt}%
\definecolor{currentstroke}{rgb}{0.000000,0.000000,0.000000}%
\pgfsetstrokecolor{currentstroke}%
\pgfsetdash{}{0pt}%
\pgfpathmoveto{\pgfqpoint{4.332143in}{1.752180in}}%
\pgfpathlineto{\pgfqpoint{4.346104in}{1.748278in}}%
\pgfpathlineto{\pgfqpoint{4.360073in}{1.744400in}}%
\pgfpathlineto{\pgfqpoint{4.374048in}{1.740547in}}%
\pgfpathlineto{\pgfqpoint{4.388030in}{1.736719in}}%
\pgfpathlineto{\pgfqpoint{4.380116in}{1.729106in}}%
\pgfpathlineto{\pgfqpoint{4.372196in}{1.721629in}}%
\pgfpathlineto{\pgfqpoint{4.364271in}{1.714295in}}%
\pgfpathlineto{\pgfqpoint{4.356340in}{1.707109in}}%
\pgfpathlineto{\pgfqpoint{4.342345in}{1.711181in}}%
\pgfpathlineto{\pgfqpoint{4.328356in}{1.715277in}}%
\pgfpathlineto{\pgfqpoint{4.314374in}{1.719399in}}%
\pgfpathlineto{\pgfqpoint{4.300399in}{1.723545in}}%
\pgfpathlineto{\pgfqpoint{4.308343in}{1.730483in}}%
\pgfpathlineto{\pgfqpoint{4.316282in}{1.737572in}}%
\pgfpathlineto{\pgfqpoint{4.324215in}{1.744806in}}%
\pgfpathlineto{\pgfqpoint{4.332143in}{1.752180in}}%
\pgfpathclose%
\pgfusepath{fill}%
\end{pgfscope}%
\begin{pgfscope}%
\pgfpathrectangle{\pgfqpoint{1.150000in}{0.150000in}}{\pgfqpoint{5.700000in}{5.700000in}}%
\pgfusepath{clip}%
\pgfsetbuttcap%
\pgfsetroundjoin%
\definecolor{currentfill}{rgb}{0.282884,0.135920,0.453427}%
\pgfsetfillcolor{currentfill}%
\pgfsetfillopacity{0.700000}%
\pgfsetlinewidth{0.000000pt}%
\definecolor{currentstroke}{rgb}{0.000000,0.000000,0.000000}%
\pgfsetstrokecolor{currentstroke}%
\pgfsetdash{}{0pt}%
\pgfpathmoveto{\pgfqpoint{5.288219in}{1.984174in}}%
\pgfpathlineto{\pgfqpoint{5.302469in}{1.983053in}}%
\pgfpathlineto{\pgfqpoint{5.316727in}{1.981956in}}%
\pgfpathlineto{\pgfqpoint{5.330994in}{1.980884in}}%
\pgfpathlineto{\pgfqpoint{5.345270in}{1.979836in}}%
\pgfpathlineto{\pgfqpoint{5.337649in}{1.969151in}}%
\pgfpathlineto{\pgfqpoint{5.330022in}{1.958405in}}%
\pgfpathlineto{\pgfqpoint{5.322389in}{1.947601in}}%
\pgfpathlineto{\pgfqpoint{5.314750in}{1.936741in}}%
\pgfpathlineto{\pgfqpoint{5.300467in}{1.937913in}}%
\pgfpathlineto{\pgfqpoint{5.286193in}{1.939110in}}%
\pgfpathlineto{\pgfqpoint{5.271927in}{1.940332in}}%
\pgfpathlineto{\pgfqpoint{5.257671in}{1.941578in}}%
\pgfpathlineto{\pgfqpoint{5.265317in}{1.952308in}}%
\pgfpathlineto{\pgfqpoint{5.272957in}{1.962986in}}%
\pgfpathlineto{\pgfqpoint{5.280591in}{1.973609in}}%
\pgfpathlineto{\pgfqpoint{5.288219in}{1.984174in}}%
\pgfpathclose%
\pgfusepath{fill}%
\end{pgfscope}%
\begin{pgfscope}%
\pgfpathrectangle{\pgfqpoint{1.150000in}{0.150000in}}{\pgfqpoint{5.700000in}{5.700000in}}%
\pgfusepath{clip}%
\pgfsetbuttcap%
\pgfsetroundjoin%
\definecolor{currentfill}{rgb}{0.273006,0.204520,0.501721}%
\pgfsetfillcolor{currentfill}%
\pgfsetfillopacity{0.700000}%
\pgfsetlinewidth{0.000000pt}%
\definecolor{currentstroke}{rgb}{0.000000,0.000000,0.000000}%
\pgfsetstrokecolor{currentstroke}%
\pgfsetdash{}{0pt}%
\pgfpathmoveto{\pgfqpoint{3.106993in}{2.114837in}}%
\pgfpathlineto{\pgfqpoint{3.120739in}{2.106937in}}%
\pgfpathlineto{\pgfqpoint{3.134489in}{2.099070in}}%
\pgfpathlineto{\pgfqpoint{3.148243in}{2.091235in}}%
\pgfpathlineto{\pgfqpoint{3.162001in}{2.083432in}}%
\pgfpathlineto{\pgfqpoint{3.153397in}{2.088313in}}%
\pgfpathlineto{\pgfqpoint{3.144775in}{2.093606in}}%
\pgfpathlineto{\pgfqpoint{3.136135in}{2.099322in}}%
\pgfpathlineto{\pgfqpoint{3.127476in}{2.105468in}}%
\pgfpathlineto{\pgfqpoint{3.113683in}{2.113615in}}%
\pgfpathlineto{\pgfqpoint{3.099893in}{2.121793in}}%
\pgfpathlineto{\pgfqpoint{3.086107in}{2.130004in}}%
\pgfpathlineto{\pgfqpoint{3.072324in}{2.138248in}}%
\pgfpathlineto{\pgfqpoint{3.081020in}{2.131752in}}%
\pgfpathlineto{\pgfqpoint{3.089697in}{2.125691in}}%
\pgfpathlineto{\pgfqpoint{3.098354in}{2.120056in}}%
\pgfpathlineto{\pgfqpoint{3.106993in}{2.114837in}}%
\pgfpathclose%
\pgfusepath{fill}%
\end{pgfscope}%
\begin{pgfscope}%
\pgfpathrectangle{\pgfqpoint{1.150000in}{0.150000in}}{\pgfqpoint{5.700000in}{5.700000in}}%
\pgfusepath{clip}%
\pgfsetbuttcap%
\pgfsetroundjoin%
\definecolor{currentfill}{rgb}{0.282327,0.094955,0.417331}%
\pgfsetfillcolor{currentfill}%
\pgfsetfillopacity{0.700000}%
\pgfsetlinewidth{0.000000pt}%
\definecolor{currentstroke}{rgb}{0.000000,0.000000,0.000000}%
\pgfsetstrokecolor{currentstroke}%
\pgfsetdash{}{0pt}%
\pgfpathmoveto{\pgfqpoint{3.560162in}{1.895432in}}%
\pgfpathlineto{\pgfqpoint{3.573966in}{1.889019in}}%
\pgfpathlineto{\pgfqpoint{3.587775in}{1.882634in}}%
\pgfpathlineto{\pgfqpoint{3.601589in}{1.876276in}}%
\pgfpathlineto{\pgfqpoint{3.615408in}{1.869947in}}%
\pgfpathlineto{\pgfqpoint{3.607130in}{1.869703in}}%
\pgfpathlineto{\pgfqpoint{3.598841in}{1.869775in}}%
\pgfpathlineto{\pgfqpoint{3.590539in}{1.870171in}}%
\pgfpathlineto{\pgfqpoint{3.582226in}{1.870900in}}%
\pgfpathlineto{\pgfqpoint{3.568380in}{1.877542in}}%
\pgfpathlineto{\pgfqpoint{3.554540in}{1.884212in}}%
\pgfpathlineto{\pgfqpoint{3.540704in}{1.890909in}}%
\pgfpathlineto{\pgfqpoint{3.526873in}{1.897635in}}%
\pgfpathlineto{\pgfqpoint{3.535214in}{1.896588in}}%
\pgfpathlineto{\pgfqpoint{3.543542in}{1.895878in}}%
\pgfpathlineto{\pgfqpoint{3.551858in}{1.895495in}}%
\pgfpathlineto{\pgfqpoint{3.560162in}{1.895432in}}%
\pgfpathclose%
\pgfusepath{fill}%
\end{pgfscope}%
\begin{pgfscope}%
\pgfpathrectangle{\pgfqpoint{1.150000in}{0.150000in}}{\pgfqpoint{5.700000in}{5.700000in}}%
\pgfusepath{clip}%
\pgfsetbuttcap%
\pgfsetroundjoin%
\definecolor{currentfill}{rgb}{0.277018,0.050344,0.375715}%
\pgfsetfillcolor{currentfill}%
\pgfsetfillopacity{0.700000}%
\pgfsetlinewidth{0.000000pt}%
\definecolor{currentstroke}{rgb}{0.000000,0.000000,0.000000}%
\pgfsetstrokecolor{currentstroke}%
\pgfsetdash{}{0pt}%
\pgfpathmoveto{\pgfqpoint{4.794198in}{1.817940in}}%
\pgfpathlineto{\pgfqpoint{4.808289in}{1.815474in}}%
\pgfpathlineto{\pgfqpoint{4.822388in}{1.813032in}}%
\pgfpathlineto{\pgfqpoint{4.836495in}{1.810615in}}%
\pgfpathlineto{\pgfqpoint{4.850610in}{1.808222in}}%
\pgfpathlineto{\pgfqpoint{4.842835in}{1.798065in}}%
\pgfpathlineto{\pgfqpoint{4.835057in}{1.787940in}}%
\pgfpathlineto{\pgfqpoint{4.827273in}{1.777851in}}%
\pgfpathlineto{\pgfqpoint{4.819485in}{1.767803in}}%
\pgfpathlineto{\pgfqpoint{4.805362in}{1.770388in}}%
\pgfpathlineto{\pgfqpoint{4.791246in}{1.772996in}}%
\pgfpathlineto{\pgfqpoint{4.777139in}{1.775630in}}%
\pgfpathlineto{\pgfqpoint{4.763038in}{1.778287in}}%
\pgfpathlineto{\pgfqpoint{4.770835in}{1.788139in}}%
\pgfpathlineto{\pgfqpoint{4.778628in}{1.798035in}}%
\pgfpathlineto{\pgfqpoint{4.786415in}{1.807970in}}%
\pgfpathlineto{\pgfqpoint{4.794198in}{1.817940in}}%
\pgfpathclose%
\pgfusepath{fill}%
\end{pgfscope}%
\begin{pgfscope}%
\pgfpathrectangle{\pgfqpoint{1.150000in}{0.150000in}}{\pgfqpoint{5.700000in}{5.700000in}}%
\pgfusepath{clip}%
\pgfsetbuttcap%
\pgfsetroundjoin%
\definecolor{currentfill}{rgb}{0.283197,0.115680,0.436115}%
\pgfsetfillcolor{currentfill}%
\pgfsetfillopacity{0.700000}%
\pgfsetlinewidth{0.000000pt}%
\definecolor{currentstroke}{rgb}{0.000000,0.000000,0.000000}%
\pgfsetstrokecolor{currentstroke}%
\pgfsetdash{}{0pt}%
\pgfpathmoveto{\pgfqpoint{5.200729in}{1.946807in}}%
\pgfpathlineto{\pgfqpoint{5.214952in}{1.945463in}}%
\pgfpathlineto{\pgfqpoint{5.229183in}{1.944143in}}%
\pgfpathlineto{\pgfqpoint{5.243422in}{1.942848in}}%
\pgfpathlineto{\pgfqpoint{5.257671in}{1.941578in}}%
\pgfpathlineto{\pgfqpoint{5.250018in}{1.930799in}}%
\pgfpathlineto{\pgfqpoint{5.242360in}{1.919973in}}%
\pgfpathlineto{\pgfqpoint{5.234696in}{1.909105in}}%
\pgfpathlineto{\pgfqpoint{5.227026in}{1.898197in}}%
\pgfpathlineto{\pgfqpoint{5.212771in}{1.899606in}}%
\pgfpathlineto{\pgfqpoint{5.198524in}{1.901039in}}%
\pgfpathlineto{\pgfqpoint{5.184286in}{1.902497in}}%
\pgfpathlineto{\pgfqpoint{5.170056in}{1.903979in}}%
\pgfpathlineto{\pgfqpoint{5.177733in}{1.914744in}}%
\pgfpathlineto{\pgfqpoint{5.185404in}{1.925472in}}%
\pgfpathlineto{\pgfqpoint{5.193070in}{1.936161in}}%
\pgfpathlineto{\pgfqpoint{5.200729in}{1.946807in}}%
\pgfpathclose%
\pgfusepath{fill}%
\end{pgfscope}%
\begin{pgfscope}%
\pgfpathrectangle{\pgfqpoint{1.150000in}{0.150000in}}{\pgfqpoint{5.700000in}{5.700000in}}%
\pgfusepath{clip}%
\pgfsetbuttcap%
\pgfsetroundjoin%
\definecolor{currentfill}{rgb}{0.282623,0.140926,0.457517}%
\pgfsetfillcolor{currentfill}%
\pgfsetfillopacity{0.700000}%
\pgfsetlinewidth{0.000000pt}%
\definecolor{currentstroke}{rgb}{0.000000,0.000000,0.000000}%
\pgfsetstrokecolor{currentstroke}%
\pgfsetdash{}{0pt}%
\pgfpathmoveto{\pgfqpoint{3.361250in}{1.980587in}}%
\pgfpathlineto{\pgfqpoint{3.375028in}{1.973513in}}%
\pgfpathlineto{\pgfqpoint{3.388810in}{1.966469in}}%
\pgfpathlineto{\pgfqpoint{3.402596in}{1.959455in}}%
\pgfpathlineto{\pgfqpoint{3.416387in}{1.952470in}}%
\pgfpathlineto{\pgfqpoint{3.407975in}{1.954505in}}%
\pgfpathlineto{\pgfqpoint{3.399549in}{1.956901in}}%
\pgfpathlineto{\pgfqpoint{3.391108in}{1.959667in}}%
\pgfpathlineto{\pgfqpoint{3.382652in}{1.962811in}}%
\pgfpathlineto{\pgfqpoint{3.368831in}{1.970124in}}%
\pgfpathlineto{\pgfqpoint{3.355014in}{1.977465in}}%
\pgfpathlineto{\pgfqpoint{3.341201in}{1.984837in}}%
\pgfpathlineto{\pgfqpoint{3.327393in}{1.992238in}}%
\pgfpathlineto{\pgfqpoint{3.335880in}{1.988760in}}%
\pgfpathlineto{\pgfqpoint{3.344352in}{1.985665in}}%
\pgfpathlineto{\pgfqpoint{3.352808in}{1.982944in}}%
\pgfpathlineto{\pgfqpoint{3.361250in}{1.980587in}}%
\pgfpathclose%
\pgfusepath{fill}%
\end{pgfscope}%
\begin{pgfscope}%
\pgfpathrectangle{\pgfqpoint{1.150000in}{0.150000in}}{\pgfqpoint{5.700000in}{5.700000in}}%
\pgfusepath{clip}%
\pgfsetbuttcap%
\pgfsetroundjoin%
\definecolor{currentfill}{rgb}{0.278791,0.062145,0.386592}%
\pgfsetfillcolor{currentfill}%
\pgfsetfillopacity{0.700000}%
\pgfsetlinewidth{0.000000pt}%
\definecolor{currentstroke}{rgb}{0.000000,0.000000,0.000000}%
\pgfsetstrokecolor{currentstroke}%
\pgfsetdash{}{0pt}%
\pgfpathmoveto{\pgfqpoint{3.758945in}{1.826664in}}%
\pgfpathlineto{\pgfqpoint{3.772786in}{1.820879in}}%
\pgfpathlineto{\pgfqpoint{3.786632in}{1.815121in}}%
\pgfpathlineto{\pgfqpoint{3.800483in}{1.809390in}}%
\pgfpathlineto{\pgfqpoint{3.814340in}{1.803686in}}%
\pgfpathlineto{\pgfqpoint{3.806175in}{1.801381in}}%
\pgfpathlineto{\pgfqpoint{3.798001in}{1.799349in}}%
\pgfpathlineto{\pgfqpoint{3.789817in}{1.797598in}}%
\pgfpathlineto{\pgfqpoint{3.781624in}{1.796136in}}%
\pgfpathlineto{\pgfqpoint{3.767744in}{1.802138in}}%
\pgfpathlineto{\pgfqpoint{3.753870in}{1.808167in}}%
\pgfpathlineto{\pgfqpoint{3.740001in}{1.814223in}}%
\pgfpathlineto{\pgfqpoint{3.726137in}{1.820306in}}%
\pgfpathlineto{\pgfqpoint{3.734355in}{1.821464in}}%
\pgfpathlineto{\pgfqpoint{3.742562in}{1.822915in}}%
\pgfpathlineto{\pgfqpoint{3.750758in}{1.824651in}}%
\pgfpathlineto{\pgfqpoint{3.758945in}{1.826664in}}%
\pgfpathclose%
\pgfusepath{fill}%
\end{pgfscope}%
\begin{pgfscope}%
\pgfpathrectangle{\pgfqpoint{1.150000in}{0.150000in}}{\pgfqpoint{5.700000in}{5.700000in}}%
\pgfusepath{clip}%
\pgfsetbuttcap%
\pgfsetroundjoin%
\definecolor{currentfill}{rgb}{0.269944,0.014625,0.341379}%
\pgfsetfillcolor{currentfill}%
\pgfsetfillopacity{0.700000}%
\pgfsetlinewidth{0.000000pt}%
\definecolor{currentstroke}{rgb}{0.000000,0.000000,0.000000}%
\pgfsetstrokecolor{currentstroke}%
\pgfsetdash{}{0pt}%
\pgfpathmoveto{\pgfqpoint{4.475581in}{1.754273in}}%
\pgfpathlineto{\pgfqpoint{4.489585in}{1.750800in}}%
\pgfpathlineto{\pgfqpoint{4.503596in}{1.747352in}}%
\pgfpathlineto{\pgfqpoint{4.517614in}{1.743928in}}%
\pgfpathlineto{\pgfqpoint{4.531639in}{1.740529in}}%
\pgfpathlineto{\pgfqpoint{4.523769in}{1.731979in}}%
\pgfpathlineto{\pgfqpoint{4.515894in}{1.723535in}}%
\pgfpathlineto{\pgfqpoint{4.508015in}{1.715203in}}%
\pgfpathlineto{\pgfqpoint{4.500130in}{1.706989in}}%
\pgfpathlineto{\pgfqpoint{4.486094in}{1.710618in}}%
\pgfpathlineto{\pgfqpoint{4.472064in}{1.714273in}}%
\pgfpathlineto{\pgfqpoint{4.458041in}{1.717952in}}%
\pgfpathlineto{\pgfqpoint{4.444025in}{1.721656in}}%
\pgfpathlineto{\pgfqpoint{4.451922in}{1.729635in}}%
\pgfpathlineto{\pgfqpoint{4.459813in}{1.737734in}}%
\pgfpathlineto{\pgfqpoint{4.467700in}{1.745949in}}%
\pgfpathlineto{\pgfqpoint{4.475581in}{1.754273in}}%
\pgfpathclose%
\pgfusepath{fill}%
\end{pgfscope}%
\begin{pgfscope}%
\pgfpathrectangle{\pgfqpoint{1.150000in}{0.150000in}}{\pgfqpoint{5.700000in}{5.700000in}}%
\pgfusepath{clip}%
\pgfsetbuttcap%
\pgfsetroundjoin%
\definecolor{currentfill}{rgb}{0.282656,0.100196,0.422160}%
\pgfsetfillcolor{currentfill}%
\pgfsetfillopacity{0.700000}%
\pgfsetlinewidth{0.000000pt}%
\definecolor{currentstroke}{rgb}{0.000000,0.000000,0.000000}%
\pgfsetstrokecolor{currentstroke}%
\pgfsetdash{}{0pt}%
\pgfpathmoveto{\pgfqpoint{5.113221in}{1.910153in}}%
\pgfpathlineto{\pgfqpoint{5.127417in}{1.908573in}}%
\pgfpathlineto{\pgfqpoint{5.141622in}{1.907017in}}%
\pgfpathlineto{\pgfqpoint{5.155835in}{1.905486in}}%
\pgfpathlineto{\pgfqpoint{5.170056in}{1.903979in}}%
\pgfpathlineto{\pgfqpoint{5.162374in}{1.893181in}}%
\pgfpathlineto{\pgfqpoint{5.154686in}{1.882354in}}%
\pgfpathlineto{\pgfqpoint{5.146992in}{1.871500in}}%
\pgfpathlineto{\pgfqpoint{5.139294in}{1.860623in}}%
\pgfpathlineto{\pgfqpoint{5.125065in}{1.862282in}}%
\pgfpathlineto{\pgfqpoint{5.110845in}{1.863965in}}%
\pgfpathlineto{\pgfqpoint{5.096633in}{1.865672in}}%
\pgfpathlineto{\pgfqpoint{5.082430in}{1.867404in}}%
\pgfpathlineto{\pgfqpoint{5.090136in}{1.878124in}}%
\pgfpathlineto{\pgfqpoint{5.097836in}{1.888824in}}%
\pgfpathlineto{\pgfqpoint{5.105531in}{1.899502in}}%
\pgfpathlineto{\pgfqpoint{5.113221in}{1.910153in}}%
\pgfpathclose%
\pgfusepath{fill}%
\end{pgfscope}%
\begin{pgfscope}%
\pgfpathrectangle{\pgfqpoint{1.150000in}{0.150000in}}{\pgfqpoint{5.700000in}{5.700000in}}%
\pgfusepath{clip}%
\pgfsetbuttcap%
\pgfsetroundjoin%
\definecolor{currentfill}{rgb}{0.197636,0.391528,0.554969}%
\pgfsetfillcolor{currentfill}%
\pgfsetfillopacity{0.700000}%
\pgfsetlinewidth{0.000000pt}%
\definecolor{currentstroke}{rgb}{0.000000,0.000000,0.000000}%
\pgfsetstrokecolor{currentstroke}%
\pgfsetdash{}{0pt}%
\pgfpathmoveto{\pgfqpoint{2.486760in}{2.542515in}}%
\pgfpathlineto{\pgfqpoint{2.500480in}{2.532392in}}%
\pgfpathlineto{\pgfqpoint{2.514203in}{2.522313in}}%
\pgfpathlineto{\pgfqpoint{2.527927in}{2.512278in}}%
\pgfpathlineto{\pgfqpoint{2.541652in}{2.502287in}}%
\pgfpathlineto{\pgfqpoint{2.532456in}{2.514460in}}%
\pgfpathlineto{\pgfqpoint{2.523230in}{2.527167in}}%
\pgfpathlineto{\pgfqpoint{2.513974in}{2.540418in}}%
\pgfpathlineto{\pgfqpoint{2.504688in}{2.554224in}}%
\pgfpathlineto{\pgfqpoint{2.490913in}{2.564598in}}%
\pgfpathlineto{\pgfqpoint{2.477140in}{2.575015in}}%
\pgfpathlineto{\pgfqpoint{2.463369in}{2.585477in}}%
\pgfpathlineto{\pgfqpoint{2.449599in}{2.595984in}}%
\pgfpathlineto{\pgfqpoint{2.458935in}{2.581788in}}%
\pgfpathlineto{\pgfqpoint{2.468240in}{2.568152in}}%
\pgfpathlineto{\pgfqpoint{2.477515in}{2.555065in}}%
\pgfpathlineto{\pgfqpoint{2.486760in}{2.542515in}}%
\pgfpathclose%
\pgfusepath{fill}%
\end{pgfscope}%
\begin{pgfscope}%
\pgfpathrectangle{\pgfqpoint{1.150000in}{0.150000in}}{\pgfqpoint{5.700000in}{5.700000in}}%
\pgfusepath{clip}%
\pgfsetbuttcap%
\pgfsetroundjoin%
\definecolor{currentfill}{rgb}{0.246811,0.283237,0.535941}%
\pgfsetfillcolor{currentfill}%
\pgfsetfillopacity{0.700000}%
\pgfsetlinewidth{0.000000pt}%
\definecolor{currentstroke}{rgb}{0.000000,0.000000,0.000000}%
\pgfsetstrokecolor{currentstroke}%
\pgfsetdash{}{0pt}%
\pgfpathmoveto{\pgfqpoint{2.852246in}{2.274754in}}%
\pgfpathlineto{\pgfqpoint{2.865978in}{2.265959in}}%
\pgfpathlineto{\pgfqpoint{2.879712in}{2.257200in}}%
\pgfpathlineto{\pgfqpoint{2.893450in}{2.248478in}}%
\pgfpathlineto{\pgfqpoint{2.907191in}{2.239791in}}%
\pgfpathlineto{\pgfqpoint{2.898358in}{2.247795in}}%
\pgfpathlineto{\pgfqpoint{2.889503in}{2.256266in}}%
\pgfpathlineto{\pgfqpoint{2.880625in}{2.265214in}}%
\pgfpathlineto{\pgfqpoint{2.871724in}{2.274649in}}%
\pgfpathlineto{\pgfqpoint{2.857942in}{2.283697in}}%
\pgfpathlineto{\pgfqpoint{2.844163in}{2.292781in}}%
\pgfpathlineto{\pgfqpoint{2.830387in}{2.301901in}}%
\pgfpathlineto{\pgfqpoint{2.816613in}{2.311058in}}%
\pgfpathlineto{\pgfqpoint{2.825557in}{2.301255in}}%
\pgfpathlineto{\pgfqpoint{2.834477in}{2.291944in}}%
\pgfpathlineto{\pgfqpoint{2.843373in}{2.283113in}}%
\pgfpathlineto{\pgfqpoint{2.852246in}{2.274754in}}%
\pgfpathclose%
\pgfusepath{fill}%
\end{pgfscope}%
\begin{pgfscope}%
\pgfpathrectangle{\pgfqpoint{1.150000in}{0.150000in}}{\pgfqpoint{5.700000in}{5.700000in}}%
\pgfusepath{clip}%
\pgfsetbuttcap%
\pgfsetroundjoin%
\definecolor{currentfill}{rgb}{0.273809,0.031497,0.358853}%
\pgfsetfillcolor{currentfill}%
\pgfsetfillopacity{0.700000}%
\pgfsetlinewidth{0.000000pt}%
\definecolor{currentstroke}{rgb}{0.000000,0.000000,0.000000}%
\pgfsetstrokecolor{currentstroke}%
\pgfsetdash{}{0pt}%
\pgfpathmoveto{\pgfqpoint{4.706713in}{1.789162in}}%
\pgfpathlineto{\pgfqpoint{4.720783in}{1.786407in}}%
\pgfpathlineto{\pgfqpoint{4.734861in}{1.783676in}}%
\pgfpathlineto{\pgfqpoint{4.748946in}{1.780969in}}%
\pgfpathlineto{\pgfqpoint{4.763038in}{1.778287in}}%
\pgfpathlineto{\pgfqpoint{4.755237in}{1.768485in}}%
\pgfpathlineto{\pgfqpoint{4.747430in}{1.758736in}}%
\pgfpathlineto{\pgfqpoint{4.739620in}{1.749045in}}%
\pgfpathlineto{\pgfqpoint{4.731804in}{1.739419in}}%
\pgfpathlineto{\pgfqpoint{4.717702in}{1.742306in}}%
\pgfpathlineto{\pgfqpoint{4.703608in}{1.745217in}}%
\pgfpathlineto{\pgfqpoint{4.689521in}{1.748153in}}%
\pgfpathlineto{\pgfqpoint{4.675441in}{1.751113in}}%
\pgfpathlineto{\pgfqpoint{4.683266in}{1.760530in}}%
\pgfpathlineto{\pgfqpoint{4.691087in}{1.770014in}}%
\pgfpathlineto{\pgfqpoint{4.698902in}{1.779559in}}%
\pgfpathlineto{\pgfqpoint{4.706713in}{1.789162in}}%
\pgfpathclose%
\pgfusepath{fill}%
\end{pgfscope}%
\begin{pgfscope}%
\pgfpathrectangle{\pgfqpoint{1.150000in}{0.150000in}}{\pgfqpoint{5.700000in}{5.700000in}}%
\pgfusepath{clip}%
\pgfsetbuttcap%
\pgfsetroundjoin%
\definecolor{currentfill}{rgb}{0.275191,0.194905,0.496005}%
\pgfsetfillcolor{currentfill}%
\pgfsetfillopacity{0.700000}%
\pgfsetlinewidth{0.000000pt}%
\definecolor{currentstroke}{rgb}{0.000000,0.000000,0.000000}%
\pgfsetstrokecolor{currentstroke}%
\pgfsetdash{}{0pt}%
\pgfpathmoveto{\pgfqpoint{5.607911in}{2.095263in}}%
\pgfpathlineto{\pgfqpoint{5.622281in}{2.094828in}}%
\pgfpathlineto{\pgfqpoint{5.636659in}{2.094417in}}%
\pgfpathlineto{\pgfqpoint{5.651048in}{2.094030in}}%
\pgfpathlineto{\pgfqpoint{5.643542in}{2.083966in}}%
\pgfpathlineto{\pgfqpoint{5.636029in}{2.073801in}}%
\pgfpathlineto{\pgfqpoint{5.628507in}{2.063538in}}%
\pgfpathlineto{\pgfqpoint{5.620978in}{2.053179in}}%
\pgfpathlineto{\pgfqpoint{5.606582in}{2.053649in}}%
\pgfpathlineto{\pgfqpoint{5.592195in}{2.054144in}}%
\pgfpathlineto{\pgfqpoint{5.577817in}{2.054663in}}%
\pgfpathlineto{\pgfqpoint{5.585352in}{2.064956in}}%
\pgfpathlineto{\pgfqpoint{5.592879in}{2.075155in}}%
\pgfpathlineto{\pgfqpoint{5.600399in}{2.085258in}}%
\pgfpathlineto{\pgfqpoint{5.607911in}{2.095263in}}%
\pgfpathclose%
\pgfusepath{fill}%
\end{pgfscope}%
\begin{pgfscope}%
\pgfpathrectangle{\pgfqpoint{1.150000in}{0.150000in}}{\pgfqpoint{5.700000in}{5.700000in}}%
\pgfusepath{clip}%
\pgfsetbuttcap%
\pgfsetroundjoin%
\definecolor{currentfill}{rgb}{0.281446,0.084320,0.407414}%
\pgfsetfillcolor{currentfill}%
\pgfsetfillopacity{0.700000}%
\pgfsetlinewidth{0.000000pt}%
\definecolor{currentstroke}{rgb}{0.000000,0.000000,0.000000}%
\pgfsetstrokecolor{currentstroke}%
\pgfsetdash{}{0pt}%
\pgfpathmoveto{\pgfqpoint{5.025698in}{1.874576in}}%
\pgfpathlineto{\pgfqpoint{5.039869in}{1.872746in}}%
\pgfpathlineto{\pgfqpoint{5.054048in}{1.870941in}}%
\pgfpathlineto{\pgfqpoint{5.068235in}{1.869160in}}%
\pgfpathlineto{\pgfqpoint{5.082430in}{1.867404in}}%
\pgfpathlineto{\pgfqpoint{5.074719in}{1.856669in}}%
\pgfpathlineto{\pgfqpoint{5.067002in}{1.845922in}}%
\pgfpathlineto{\pgfqpoint{5.059281in}{1.835167in}}%
\pgfpathlineto{\pgfqpoint{5.051555in}{1.824407in}}%
\pgfpathlineto{\pgfqpoint{5.037352in}{1.826329in}}%
\pgfpathlineto{\pgfqpoint{5.023158in}{1.828275in}}%
\pgfpathlineto{\pgfqpoint{5.008971in}{1.830245in}}%
\pgfpathlineto{\pgfqpoint{4.994793in}{1.832240in}}%
\pgfpathlineto{\pgfqpoint{5.002527in}{1.842829in}}%
\pgfpathlineto{\pgfqpoint{5.010256in}{1.853417in}}%
\pgfpathlineto{\pgfqpoint{5.017980in}{1.864001in}}%
\pgfpathlineto{\pgfqpoint{5.025698in}{1.874576in}}%
\pgfpathclose%
\pgfusepath{fill}%
\end{pgfscope}%
\begin{pgfscope}%
\pgfpathrectangle{\pgfqpoint{1.150000in}{0.150000in}}{\pgfqpoint{5.700000in}{5.700000in}}%
\pgfusepath{clip}%
\pgfsetbuttcap%
\pgfsetroundjoin%
\definecolor{currentfill}{rgb}{0.271305,0.019942,0.347269}%
\pgfsetfillcolor{currentfill}%
\pgfsetfillopacity{0.700000}%
\pgfsetlinewidth{0.000000pt}%
\definecolor{currentstroke}{rgb}{0.000000,0.000000,0.000000}%
\pgfsetstrokecolor{currentstroke}%
\pgfsetdash{}{0pt}%
\pgfpathmoveto{\pgfqpoint{4.101171in}{1.752247in}}%
\pgfpathlineto{\pgfqpoint{4.115087in}{1.747541in}}%
\pgfpathlineto{\pgfqpoint{4.129008in}{1.742860in}}%
\pgfpathlineto{\pgfqpoint{4.142935in}{1.738205in}}%
\pgfpathlineto{\pgfqpoint{4.156869in}{1.733574in}}%
\pgfpathlineto{\pgfqpoint{4.148864in}{1.728024in}}%
\pgfpathlineto{\pgfqpoint{4.140852in}{1.722671in}}%
\pgfpathlineto{\pgfqpoint{4.132833in}{1.717523in}}%
\pgfpathlineto{\pgfqpoint{4.124807in}{1.712586in}}%
\pgfpathlineto{\pgfqpoint{4.110856in}{1.717487in}}%
\pgfpathlineto{\pgfqpoint{4.096912in}{1.722413in}}%
\pgfpathlineto{\pgfqpoint{4.082973in}{1.727364in}}%
\pgfpathlineto{\pgfqpoint{4.069040in}{1.732341in}}%
\pgfpathlineto{\pgfqpoint{4.077084in}{1.737002in}}%
\pgfpathlineto{\pgfqpoint{4.085120in}{1.741878in}}%
\pgfpathlineto{\pgfqpoint{4.093149in}{1.746962in}}%
\pgfpathlineto{\pgfqpoint{4.101171in}{1.752247in}}%
\pgfpathclose%
\pgfusepath{fill}%
\end{pgfscope}%
\begin{pgfscope}%
\pgfpathrectangle{\pgfqpoint{1.150000in}{0.150000in}}{\pgfqpoint{5.700000in}{5.700000in}}%
\pgfusepath{clip}%
\pgfsetbuttcap%
\pgfsetroundjoin%
\definecolor{currentfill}{rgb}{0.275191,0.194905,0.496005}%
\pgfsetfillcolor{currentfill}%
\pgfsetfillopacity{0.700000}%
\pgfsetlinewidth{0.000000pt}%
\definecolor{currentstroke}{rgb}{0.000000,0.000000,0.000000}%
\pgfsetstrokecolor{currentstroke}%
\pgfsetdash{}{0pt}%
\pgfpathmoveto{\pgfqpoint{3.162001in}{2.083432in}}%
\pgfpathlineto{\pgfqpoint{3.175762in}{2.075661in}}%
\pgfpathlineto{\pgfqpoint{3.189527in}{2.067922in}}%
\pgfpathlineto{\pgfqpoint{3.203296in}{2.060214in}}%
\pgfpathlineto{\pgfqpoint{3.217069in}{2.052538in}}%
\pgfpathlineto{\pgfqpoint{3.208499in}{2.057081in}}%
\pgfpathlineto{\pgfqpoint{3.199912in}{2.062033in}}%
\pgfpathlineto{\pgfqpoint{3.191308in}{2.067403in}}%
\pgfpathlineto{\pgfqpoint{3.182685in}{2.073201in}}%
\pgfpathlineto{\pgfqpoint{3.168877in}{2.081220in}}%
\pgfpathlineto{\pgfqpoint{3.155073in}{2.089271in}}%
\pgfpathlineto{\pgfqpoint{3.141273in}{2.097354in}}%
\pgfpathlineto{\pgfqpoint{3.127476in}{2.105468in}}%
\pgfpathlineto{\pgfqpoint{3.136135in}{2.099322in}}%
\pgfpathlineto{\pgfqpoint{3.144775in}{2.093606in}}%
\pgfpathlineto{\pgfqpoint{3.153397in}{2.088313in}}%
\pgfpathlineto{\pgfqpoint{3.162001in}{2.083432in}}%
\pgfpathclose%
\pgfusepath{fill}%
\end{pgfscope}%
\begin{pgfscope}%
\pgfpathrectangle{\pgfqpoint{1.150000in}{0.150000in}}{\pgfqpoint{5.700000in}{5.700000in}}%
\pgfusepath{clip}%
\pgfsetbuttcap%
\pgfsetroundjoin%
\definecolor{currentfill}{rgb}{0.269944,0.014625,0.341379}%
\pgfsetfillcolor{currentfill}%
\pgfsetfillopacity{0.700000}%
\pgfsetlinewidth{0.000000pt}%
\definecolor{currentstroke}{rgb}{0.000000,0.000000,0.000000}%
\pgfsetstrokecolor{currentstroke}%
\pgfsetdash{}{0pt}%
\pgfpathmoveto{\pgfqpoint{4.244561in}{1.740382in}}%
\pgfpathlineto{\pgfqpoint{4.258511in}{1.736135in}}%
\pgfpathlineto{\pgfqpoint{4.272467in}{1.731913in}}%
\pgfpathlineto{\pgfqpoint{4.286430in}{1.727717in}}%
\pgfpathlineto{\pgfqpoint{4.300399in}{1.723545in}}%
\pgfpathlineto{\pgfqpoint{4.292449in}{1.716766in}}%
\pgfpathlineto{\pgfqpoint{4.284493in}{1.710152in}}%
\pgfpathlineto{\pgfqpoint{4.276531in}{1.703708in}}%
\pgfpathlineto{\pgfqpoint{4.268563in}{1.697442in}}%
\pgfpathlineto{\pgfqpoint{4.254579in}{1.701870in}}%
\pgfpathlineto{\pgfqpoint{4.240602in}{1.706324in}}%
\pgfpathlineto{\pgfqpoint{4.226631in}{1.710803in}}%
\pgfpathlineto{\pgfqpoint{4.212666in}{1.715307in}}%
\pgfpathlineto{\pgfqpoint{4.220649in}{1.721311in}}%
\pgfpathlineto{\pgfqpoint{4.228626in}{1.727496in}}%
\pgfpathlineto{\pgfqpoint{4.236597in}{1.733855in}}%
\pgfpathlineto{\pgfqpoint{4.244561in}{1.740382in}}%
\pgfpathclose%
\pgfusepath{fill}%
\end{pgfscope}%
\begin{pgfscope}%
\pgfpathrectangle{\pgfqpoint{1.150000in}{0.150000in}}{\pgfqpoint{5.700000in}{5.700000in}}%
\pgfusepath{clip}%
\pgfsetbuttcap%
\pgfsetroundjoin%
\definecolor{currentfill}{rgb}{0.273809,0.031497,0.358853}%
\pgfsetfillcolor{currentfill}%
\pgfsetfillopacity{0.700000}%
\pgfsetlinewidth{0.000000pt}%
\definecolor{currentstroke}{rgb}{0.000000,0.000000,0.000000}%
\pgfsetstrokecolor{currentstroke}%
\pgfsetdash{}{0pt}%
\pgfpathmoveto{\pgfqpoint{3.957790in}{1.773081in}}%
\pgfpathlineto{\pgfqpoint{3.971676in}{1.767898in}}%
\pgfpathlineto{\pgfqpoint{3.985568in}{1.762741in}}%
\pgfpathlineto{\pgfqpoint{3.999465in}{1.757610in}}%
\pgfpathlineto{\pgfqpoint{4.013368in}{1.752505in}}%
\pgfpathlineto{\pgfqpoint{4.005299in}{1.748345in}}%
\pgfpathlineto{\pgfqpoint{3.997222in}{1.744418in}}%
\pgfpathlineto{\pgfqpoint{3.989138in}{1.740732in}}%
\pgfpathlineto{\pgfqpoint{3.981045in}{1.737293in}}%
\pgfpathlineto{\pgfqpoint{3.967122in}{1.742683in}}%
\pgfpathlineto{\pgfqpoint{3.953205in}{1.748098in}}%
\pgfpathlineto{\pgfqpoint{3.939294in}{1.753539in}}%
\pgfpathlineto{\pgfqpoint{3.925388in}{1.759006in}}%
\pgfpathlineto{\pgfqpoint{3.933501in}{1.762155in}}%
\pgfpathlineto{\pgfqpoint{3.941605in}{1.765556in}}%
\pgfpathlineto{\pgfqpoint{3.949702in}{1.769200in}}%
\pgfpathlineto{\pgfqpoint{3.957790in}{1.773081in}}%
\pgfpathclose%
\pgfusepath{fill}%
\end{pgfscope}%
\begin{pgfscope}%
\pgfpathrectangle{\pgfqpoint{1.150000in}{0.150000in}}{\pgfqpoint{5.700000in}{5.700000in}}%
\pgfusepath{clip}%
\pgfsetbuttcap%
\pgfsetroundjoin%
\definecolor{currentfill}{rgb}{0.281924,0.089666,0.412415}%
\pgfsetfillcolor{currentfill}%
\pgfsetfillopacity{0.700000}%
\pgfsetlinewidth{0.000000pt}%
\definecolor{currentstroke}{rgb}{0.000000,0.000000,0.000000}%
\pgfsetstrokecolor{currentstroke}%
\pgfsetdash{}{0pt}%
\pgfpathmoveto{\pgfqpoint{3.615408in}{1.869947in}}%
\pgfpathlineto{\pgfqpoint{3.629232in}{1.863646in}}%
\pgfpathlineto{\pgfqpoint{3.643061in}{1.857372in}}%
\pgfpathlineto{\pgfqpoint{3.656894in}{1.851126in}}%
\pgfpathlineto{\pgfqpoint{3.670733in}{1.844907in}}%
\pgfpathlineto{\pgfqpoint{3.662480in}{1.844356in}}%
\pgfpathlineto{\pgfqpoint{3.654217in}{1.844118in}}%
\pgfpathlineto{\pgfqpoint{3.645941in}{1.844200in}}%
\pgfpathlineto{\pgfqpoint{3.637655in}{1.844611in}}%
\pgfpathlineto{\pgfqpoint{3.623790in}{1.851142in}}%
\pgfpathlineto{\pgfqpoint{3.609931in}{1.857701in}}%
\pgfpathlineto{\pgfqpoint{3.596076in}{1.864287in}}%
\pgfpathlineto{\pgfqpoint{3.582226in}{1.870900in}}%
\pgfpathlineto{\pgfqpoint{3.590539in}{1.870171in}}%
\pgfpathlineto{\pgfqpoint{3.598841in}{1.869775in}}%
\pgfpathlineto{\pgfqpoint{3.607130in}{1.869703in}}%
\pgfpathlineto{\pgfqpoint{3.615408in}{1.869947in}}%
\pgfpathclose%
\pgfusepath{fill}%
\end{pgfscope}%
\begin{pgfscope}%
\pgfpathrectangle{\pgfqpoint{1.150000in}{0.150000in}}{\pgfqpoint{5.700000in}{5.700000in}}%
\pgfusepath{clip}%
\pgfsetbuttcap%
\pgfsetroundjoin%
\definecolor{currentfill}{rgb}{0.279566,0.067836,0.391917}%
\pgfsetfillcolor{currentfill}%
\pgfsetfillopacity{0.700000}%
\pgfsetlinewidth{0.000000pt}%
\definecolor{currentstroke}{rgb}{0.000000,0.000000,0.000000}%
\pgfsetstrokecolor{currentstroke}%
\pgfsetdash{}{0pt}%
\pgfpathmoveto{\pgfqpoint{4.938162in}{1.840462in}}%
\pgfpathlineto{\pgfqpoint{4.952308in}{1.838370in}}%
\pgfpathlineto{\pgfqpoint{4.966461in}{1.836302in}}%
\pgfpathlineto{\pgfqpoint{4.980623in}{1.834259in}}%
\pgfpathlineto{\pgfqpoint{4.994793in}{1.832240in}}%
\pgfpathlineto{\pgfqpoint{4.987055in}{1.821654in}}%
\pgfpathlineto{\pgfqpoint{4.979311in}{1.811076in}}%
\pgfpathlineto{\pgfqpoint{4.971562in}{1.800509in}}%
\pgfpathlineto{\pgfqpoint{4.963809in}{1.789958in}}%
\pgfpathlineto{\pgfqpoint{4.949631in}{1.792156in}}%
\pgfpathlineto{\pgfqpoint{4.935462in}{1.794378in}}%
\pgfpathlineto{\pgfqpoint{4.921300in}{1.796625in}}%
\pgfpathlineto{\pgfqpoint{4.907146in}{1.798895in}}%
\pgfpathlineto{\pgfqpoint{4.914907in}{1.809262in}}%
\pgfpathlineto{\pgfqpoint{4.922664in}{1.819649in}}%
\pgfpathlineto{\pgfqpoint{4.930415in}{1.830050in}}%
\pgfpathlineto{\pgfqpoint{4.938162in}{1.840462in}}%
\pgfpathclose%
\pgfusepath{fill}%
\end{pgfscope}%
\begin{pgfscope}%
\pgfpathrectangle{\pgfqpoint{1.150000in}{0.150000in}}{\pgfqpoint{5.700000in}{5.700000in}}%
\pgfusepath{clip}%
\pgfsetbuttcap%
\pgfsetroundjoin%
\definecolor{currentfill}{rgb}{0.272594,0.025563,0.353093}%
\pgfsetfillcolor{currentfill}%
\pgfsetfillopacity{0.700000}%
\pgfsetlinewidth{0.000000pt}%
\definecolor{currentstroke}{rgb}{0.000000,0.000000,0.000000}%
\pgfsetstrokecolor{currentstroke}%
\pgfsetdash{}{0pt}%
\pgfpathmoveto{\pgfqpoint{4.619197in}{1.763199in}}%
\pgfpathlineto{\pgfqpoint{4.633247in}{1.760141in}}%
\pgfpathlineto{\pgfqpoint{4.647304in}{1.757107in}}%
\pgfpathlineto{\pgfqpoint{4.661369in}{1.754098in}}%
\pgfpathlineto{\pgfqpoint{4.675441in}{1.751113in}}%
\pgfpathlineto{\pgfqpoint{4.667612in}{1.741769in}}%
\pgfpathlineto{\pgfqpoint{4.659778in}{1.732502in}}%
\pgfpathlineto{\pgfqpoint{4.651939in}{1.723318in}}%
\pgfpathlineto{\pgfqpoint{4.644095in}{1.714222in}}%
\pgfpathlineto{\pgfqpoint{4.630013in}{1.717424in}}%
\pgfpathlineto{\pgfqpoint{4.615938in}{1.720652in}}%
\pgfpathlineto{\pgfqpoint{4.601870in}{1.723903in}}%
\pgfpathlineto{\pgfqpoint{4.587810in}{1.727179in}}%
\pgfpathlineto{\pgfqpoint{4.595664in}{1.736052in}}%
\pgfpathlineto{\pgfqpoint{4.603513in}{1.745017in}}%
\pgfpathlineto{\pgfqpoint{4.611357in}{1.754067in}}%
\pgfpathlineto{\pgfqpoint{4.619197in}{1.763199in}}%
\pgfpathclose%
\pgfusepath{fill}%
\end{pgfscope}%
\begin{pgfscope}%
\pgfpathrectangle{\pgfqpoint{1.150000in}{0.150000in}}{\pgfqpoint{5.700000in}{5.700000in}}%
\pgfusepath{clip}%
\pgfsetbuttcap%
\pgfsetroundjoin%
\definecolor{currentfill}{rgb}{0.203063,0.379716,0.553925}%
\pgfsetfillcolor{currentfill}%
\pgfsetfillopacity{0.700000}%
\pgfsetlinewidth{0.000000pt}%
\definecolor{currentstroke}{rgb}{0.000000,0.000000,0.000000}%
\pgfsetstrokecolor{currentstroke}%
\pgfsetdash{}{0pt}%
\pgfpathmoveto{\pgfqpoint{2.541652in}{2.502287in}}%
\pgfpathlineto{\pgfqpoint{2.555380in}{2.492339in}}%
\pgfpathlineto{\pgfqpoint{2.569109in}{2.482433in}}%
\pgfpathlineto{\pgfqpoint{2.582841in}{2.472570in}}%
\pgfpathlineto{\pgfqpoint{2.596574in}{2.462750in}}%
\pgfpathlineto{\pgfqpoint{2.587425in}{2.474547in}}%
\pgfpathlineto{\pgfqpoint{2.578247in}{2.486874in}}%
\pgfpathlineto{\pgfqpoint{2.569040in}{2.499742in}}%
\pgfpathlineto{\pgfqpoint{2.559804in}{2.513160in}}%
\pgfpathlineto{\pgfqpoint{2.546022in}{2.523362in}}%
\pgfpathlineto{\pgfqpoint{2.532243in}{2.533607in}}%
\pgfpathlineto{\pgfqpoint{2.518465in}{2.543894in}}%
\pgfpathlineto{\pgfqpoint{2.504688in}{2.554224in}}%
\pgfpathlineto{\pgfqpoint{2.513974in}{2.540418in}}%
\pgfpathlineto{\pgfqpoint{2.523230in}{2.527167in}}%
\pgfpathlineto{\pgfqpoint{2.532456in}{2.514460in}}%
\pgfpathlineto{\pgfqpoint{2.541652in}{2.502287in}}%
\pgfpathclose%
\pgfusepath{fill}%
\end{pgfscope}%
\begin{pgfscope}%
\pgfpathrectangle{\pgfqpoint{1.150000in}{0.150000in}}{\pgfqpoint{5.700000in}{5.700000in}}%
\pgfusepath{clip}%
\pgfsetbuttcap%
\pgfsetroundjoin%
\definecolor{currentfill}{rgb}{0.278826,0.175490,0.483397}%
\pgfsetfillcolor{currentfill}%
\pgfsetfillopacity{0.700000}%
\pgfsetlinewidth{0.000000pt}%
\definecolor{currentstroke}{rgb}{0.000000,0.000000,0.000000}%
\pgfsetstrokecolor{currentstroke}%
\pgfsetdash{}{0pt}%
\pgfpathmoveto{\pgfqpoint{5.520398in}{2.056987in}}%
\pgfpathlineto{\pgfqpoint{5.534739in}{2.056369in}}%
\pgfpathlineto{\pgfqpoint{5.549089in}{2.055776in}}%
\pgfpathlineto{\pgfqpoint{5.563448in}{2.055207in}}%
\pgfpathlineto{\pgfqpoint{5.577817in}{2.054663in}}%
\pgfpathlineto{\pgfqpoint{5.570274in}{2.044278in}}%
\pgfpathlineto{\pgfqpoint{5.562725in}{2.033803in}}%
\pgfpathlineto{\pgfqpoint{5.555167in}{2.023240in}}%
\pgfpathlineto{\pgfqpoint{5.547603in}{2.012590in}}%
\pgfpathlineto{\pgfqpoint{5.533227in}{2.013232in}}%
\pgfpathlineto{\pgfqpoint{5.518859in}{2.013898in}}%
\pgfpathlineto{\pgfqpoint{5.504501in}{2.014589in}}%
\pgfpathlineto{\pgfqpoint{5.490153in}{2.015305in}}%
\pgfpathlineto{\pgfqpoint{5.497725in}{2.025851in}}%
\pgfpathlineto{\pgfqpoint{5.505289in}{2.036315in}}%
\pgfpathlineto{\pgfqpoint{5.512847in}{2.046694in}}%
\pgfpathlineto{\pgfqpoint{5.520398in}{2.056987in}}%
\pgfpathclose%
\pgfusepath{fill}%
\end{pgfscope}%
\begin{pgfscope}%
\pgfpathrectangle{\pgfqpoint{1.150000in}{0.150000in}}{\pgfqpoint{5.700000in}{5.700000in}}%
\pgfusepath{clip}%
\pgfsetbuttcap%
\pgfsetroundjoin%
\definecolor{currentfill}{rgb}{0.282884,0.135920,0.453427}%
\pgfsetfillcolor{currentfill}%
\pgfsetfillopacity{0.700000}%
\pgfsetlinewidth{0.000000pt}%
\definecolor{currentstroke}{rgb}{0.000000,0.000000,0.000000}%
\pgfsetstrokecolor{currentstroke}%
\pgfsetdash{}{0pt}%
\pgfpathmoveto{\pgfqpoint{3.416387in}{1.952470in}}%
\pgfpathlineto{\pgfqpoint{3.430182in}{1.945515in}}%
\pgfpathlineto{\pgfqpoint{3.443981in}{1.938588in}}%
\pgfpathlineto{\pgfqpoint{3.457785in}{1.931691in}}%
\pgfpathlineto{\pgfqpoint{3.471594in}{1.924822in}}%
\pgfpathlineto{\pgfqpoint{3.463211in}{1.926535in}}%
\pgfpathlineto{\pgfqpoint{3.454815in}{1.928606in}}%
\pgfpathlineto{\pgfqpoint{3.446404in}{1.931043in}}%
\pgfpathlineto{\pgfqpoint{3.437979in}{1.933854in}}%
\pgfpathlineto{\pgfqpoint{3.424141in}{1.941050in}}%
\pgfpathlineto{\pgfqpoint{3.410307in}{1.948275in}}%
\pgfpathlineto{\pgfqpoint{3.396478in}{1.955528in}}%
\pgfpathlineto{\pgfqpoint{3.382652in}{1.962811in}}%
\pgfpathlineto{\pgfqpoint{3.391108in}{1.959667in}}%
\pgfpathlineto{\pgfqpoint{3.399549in}{1.956901in}}%
\pgfpathlineto{\pgfqpoint{3.407975in}{1.954505in}}%
\pgfpathlineto{\pgfqpoint{3.416387in}{1.952470in}}%
\pgfpathclose%
\pgfusepath{fill}%
\end{pgfscope}%
\begin{pgfscope}%
\pgfpathrectangle{\pgfqpoint{1.150000in}{0.150000in}}{\pgfqpoint{5.700000in}{5.700000in}}%
\pgfusepath{clip}%
\pgfsetbuttcap%
\pgfsetroundjoin%
\definecolor{currentfill}{rgb}{0.269944,0.014625,0.341379}%
\pgfsetfillcolor{currentfill}%
\pgfsetfillopacity{0.700000}%
\pgfsetlinewidth{0.000000pt}%
\definecolor{currentstroke}{rgb}{0.000000,0.000000,0.000000}%
\pgfsetstrokecolor{currentstroke}%
\pgfsetdash{}{0pt}%
\pgfpathmoveto{\pgfqpoint{4.388030in}{1.736719in}}%
\pgfpathlineto{\pgfqpoint{4.402019in}{1.732916in}}%
\pgfpathlineto{\pgfqpoint{4.416014in}{1.729138in}}%
\pgfpathlineto{\pgfqpoint{4.430016in}{1.725385in}}%
\pgfpathlineto{\pgfqpoint{4.444025in}{1.721656in}}%
\pgfpathlineto{\pgfqpoint{4.436124in}{1.713804in}}%
\pgfpathlineto{\pgfqpoint{4.428217in}{1.706085in}}%
\pgfpathlineto{\pgfqpoint{4.420305in}{1.698505in}}%
\pgfpathlineto{\pgfqpoint{4.412388in}{1.691069in}}%
\pgfpathlineto{\pgfqpoint{4.398366in}{1.695042in}}%
\pgfpathlineto{\pgfqpoint{4.384351in}{1.699040in}}%
\pgfpathlineto{\pgfqpoint{4.370342in}{1.703062in}}%
\pgfpathlineto{\pgfqpoint{4.356340in}{1.707109in}}%
\pgfpathlineto{\pgfqpoint{4.364271in}{1.714295in}}%
\pgfpathlineto{\pgfqpoint{4.372196in}{1.721629in}}%
\pgfpathlineto{\pgfqpoint{4.380116in}{1.729106in}}%
\pgfpathlineto{\pgfqpoint{4.388030in}{1.736719in}}%
\pgfpathclose%
\pgfusepath{fill}%
\end{pgfscope}%
\begin{pgfscope}%
\pgfpathrectangle{\pgfqpoint{1.150000in}{0.150000in}}{\pgfqpoint{5.700000in}{5.700000in}}%
\pgfusepath{clip}%
\pgfsetbuttcap%
\pgfsetroundjoin%
\definecolor{currentfill}{rgb}{0.252194,0.269783,0.531579}%
\pgfsetfillcolor{currentfill}%
\pgfsetfillopacity{0.700000}%
\pgfsetlinewidth{0.000000pt}%
\definecolor{currentstroke}{rgb}{0.000000,0.000000,0.000000}%
\pgfsetstrokecolor{currentstroke}%
\pgfsetdash{}{0pt}%
\pgfpathmoveto{\pgfqpoint{2.907191in}{2.239791in}}%
\pgfpathlineto{\pgfqpoint{2.920934in}{2.231139in}}%
\pgfpathlineto{\pgfqpoint{2.934681in}{2.222523in}}%
\pgfpathlineto{\pgfqpoint{2.948431in}{2.213942in}}%
\pgfpathlineto{\pgfqpoint{2.962184in}{2.205395in}}%
\pgfpathlineto{\pgfqpoint{2.953391in}{2.213045in}}%
\pgfpathlineto{\pgfqpoint{2.944576in}{2.221157in}}%
\pgfpathlineto{\pgfqpoint{2.935740in}{2.229742in}}%
\pgfpathlineto{\pgfqpoint{2.926880in}{2.238810in}}%
\pgfpathlineto{\pgfqpoint{2.913087in}{2.247717in}}%
\pgfpathlineto{\pgfqpoint{2.899296in}{2.256659in}}%
\pgfpathlineto{\pgfqpoint{2.885509in}{2.265636in}}%
\pgfpathlineto{\pgfqpoint{2.871724in}{2.274649in}}%
\pgfpathlineto{\pgfqpoint{2.880625in}{2.265214in}}%
\pgfpathlineto{\pgfqpoint{2.889503in}{2.256266in}}%
\pgfpathlineto{\pgfqpoint{2.898358in}{2.247795in}}%
\pgfpathlineto{\pgfqpoint{2.907191in}{2.239791in}}%
\pgfpathclose%
\pgfusepath{fill}%
\end{pgfscope}%
\begin{pgfscope}%
\pgfpathrectangle{\pgfqpoint{1.150000in}{0.150000in}}{\pgfqpoint{5.700000in}{5.700000in}}%
\pgfusepath{clip}%
\pgfsetbuttcap%
\pgfsetroundjoin%
\definecolor{currentfill}{rgb}{0.280868,0.160771,0.472899}%
\pgfsetfillcolor{currentfill}%
\pgfsetfillopacity{0.700000}%
\pgfsetlinewidth{0.000000pt}%
\definecolor{currentstroke}{rgb}{0.000000,0.000000,0.000000}%
\pgfsetstrokecolor{currentstroke}%
\pgfsetdash{}{0pt}%
\pgfpathmoveto{\pgfqpoint{5.432848in}{2.018412in}}%
\pgfpathlineto{\pgfqpoint{5.447160in}{2.017598in}}%
\pgfpathlineto{\pgfqpoint{5.461482in}{2.016809in}}%
\pgfpathlineto{\pgfqpoint{5.475813in}{2.016045in}}%
\pgfpathlineto{\pgfqpoint{5.490153in}{2.015305in}}%
\pgfpathlineto{\pgfqpoint{5.482574in}{2.004678in}}%
\pgfpathlineto{\pgfqpoint{5.474988in}{1.993973in}}%
\pgfpathlineto{\pgfqpoint{5.467395in}{1.983192in}}%
\pgfpathlineto{\pgfqpoint{5.459795in}{1.972338in}}%
\pgfpathlineto{\pgfqpoint{5.445448in}{1.973189in}}%
\pgfpathlineto{\pgfqpoint{5.431110in}{1.974065in}}%
\pgfpathlineto{\pgfqpoint{5.416781in}{1.974966in}}%
\pgfpathlineto{\pgfqpoint{5.402461in}{1.975891in}}%
\pgfpathlineto{\pgfqpoint{5.410068in}{1.986628in}}%
\pgfpathlineto{\pgfqpoint{5.417668in}{1.997296in}}%
\pgfpathlineto{\pgfqpoint{5.425261in}{2.007891in}}%
\pgfpathlineto{\pgfqpoint{5.432848in}{2.018412in}}%
\pgfpathclose%
\pgfusepath{fill}%
\end{pgfscope}%
\begin{pgfscope}%
\pgfpathrectangle{\pgfqpoint{1.150000in}{0.150000in}}{\pgfqpoint{5.700000in}{5.700000in}}%
\pgfusepath{clip}%
\pgfsetbuttcap%
\pgfsetroundjoin%
\definecolor{currentfill}{rgb}{0.277941,0.056324,0.381191}%
\pgfsetfillcolor{currentfill}%
\pgfsetfillopacity{0.700000}%
\pgfsetlinewidth{0.000000pt}%
\definecolor{currentstroke}{rgb}{0.000000,0.000000,0.000000}%
\pgfsetstrokecolor{currentstroke}%
\pgfsetdash{}{0pt}%
\pgfpathmoveto{\pgfqpoint{3.814340in}{1.803686in}}%
\pgfpathlineto{\pgfqpoint{3.828202in}{1.798009in}}%
\pgfpathlineto{\pgfqpoint{3.842069in}{1.792358in}}%
\pgfpathlineto{\pgfqpoint{3.855942in}{1.786733in}}%
\pgfpathlineto{\pgfqpoint{3.869820in}{1.781135in}}%
\pgfpathlineto{\pgfqpoint{3.861677in}{1.778537in}}%
\pgfpathlineto{\pgfqpoint{3.853525in}{1.776209in}}%
\pgfpathlineto{\pgfqpoint{3.845364in}{1.774159in}}%
\pgfpathlineto{\pgfqpoint{3.837194in}{1.772394in}}%
\pgfpathlineto{\pgfqpoint{3.823293in}{1.778290in}}%
\pgfpathlineto{\pgfqpoint{3.809398in}{1.784212in}}%
\pgfpathlineto{\pgfqpoint{3.795508in}{1.790161in}}%
\pgfpathlineto{\pgfqpoint{3.781624in}{1.796136in}}%
\pgfpathlineto{\pgfqpoint{3.789817in}{1.797598in}}%
\pgfpathlineto{\pgfqpoint{3.798001in}{1.799349in}}%
\pgfpathlineto{\pgfqpoint{3.806175in}{1.801381in}}%
\pgfpathlineto{\pgfqpoint{3.814340in}{1.803686in}}%
\pgfpathclose%
\pgfusepath{fill}%
\end{pgfscope}%
\begin{pgfscope}%
\pgfpathrectangle{\pgfqpoint{1.150000in}{0.150000in}}{\pgfqpoint{5.700000in}{5.700000in}}%
\pgfusepath{clip}%
\pgfsetbuttcap%
\pgfsetroundjoin%
\definecolor{currentfill}{rgb}{0.277941,0.056324,0.381191}%
\pgfsetfillcolor{currentfill}%
\pgfsetfillopacity{0.700000}%
\pgfsetlinewidth{0.000000pt}%
\definecolor{currentstroke}{rgb}{0.000000,0.000000,0.000000}%
\pgfsetstrokecolor{currentstroke}%
\pgfsetdash{}{0pt}%
\pgfpathmoveto{\pgfqpoint{4.850610in}{1.808222in}}%
\pgfpathlineto{\pgfqpoint{4.864732in}{1.805854in}}%
\pgfpathlineto{\pgfqpoint{4.878862in}{1.803510in}}%
\pgfpathlineto{\pgfqpoint{4.893000in}{1.801190in}}%
\pgfpathlineto{\pgfqpoint{4.907146in}{1.798895in}}%
\pgfpathlineto{\pgfqpoint{4.899380in}{1.788552in}}%
\pgfpathlineto{\pgfqpoint{4.891610in}{1.778237in}}%
\pgfpathlineto{\pgfqpoint{4.883835in}{1.767954in}}%
\pgfpathlineto{\pgfqpoint{4.876055in}{1.757709in}}%
\pgfpathlineto{\pgfqpoint{4.861901in}{1.760196in}}%
\pgfpathlineto{\pgfqpoint{4.847754in}{1.762708in}}%
\pgfpathlineto{\pgfqpoint{4.833616in}{1.765243in}}%
\pgfpathlineto{\pgfqpoint{4.819485in}{1.767803in}}%
\pgfpathlineto{\pgfqpoint{4.827273in}{1.777851in}}%
\pgfpathlineto{\pgfqpoint{4.835057in}{1.787940in}}%
\pgfpathlineto{\pgfqpoint{4.842835in}{1.798065in}}%
\pgfpathlineto{\pgfqpoint{4.850610in}{1.808222in}}%
\pgfpathclose%
\pgfusepath{fill}%
\end{pgfscope}%
\begin{pgfscope}%
\pgfpathrectangle{\pgfqpoint{1.150000in}{0.150000in}}{\pgfqpoint{5.700000in}{5.700000in}}%
\pgfusepath{clip}%
\pgfsetbuttcap%
\pgfsetroundjoin%
\definecolor{currentfill}{rgb}{0.282623,0.140926,0.457517}%
\pgfsetfillcolor{currentfill}%
\pgfsetfillopacity{0.700000}%
\pgfsetlinewidth{0.000000pt}%
\definecolor{currentstroke}{rgb}{0.000000,0.000000,0.000000}%
\pgfsetstrokecolor{currentstroke}%
\pgfsetdash{}{0pt}%
\pgfpathmoveto{\pgfqpoint{5.345270in}{1.979836in}}%
\pgfpathlineto{\pgfqpoint{5.359554in}{1.978813in}}%
\pgfpathlineto{\pgfqpoint{5.373848in}{1.977814in}}%
\pgfpathlineto{\pgfqpoint{5.388150in}{1.976840in}}%
\pgfpathlineto{\pgfqpoint{5.402461in}{1.975891in}}%
\pgfpathlineto{\pgfqpoint{5.394848in}{1.965086in}}%
\pgfpathlineto{\pgfqpoint{5.387228in}{1.954217in}}%
\pgfpathlineto{\pgfqpoint{5.379602in}{1.943285in}}%
\pgfpathlineto{\pgfqpoint{5.371970in}{1.932295in}}%
\pgfpathlineto{\pgfqpoint{5.357652in}{1.933370in}}%
\pgfpathlineto{\pgfqpoint{5.343343in}{1.934469in}}%
\pgfpathlineto{\pgfqpoint{5.329042in}{1.935593in}}%
\pgfpathlineto{\pgfqpoint{5.314750in}{1.936741in}}%
\pgfpathlineto{\pgfqpoint{5.322389in}{1.947601in}}%
\pgfpathlineto{\pgfqpoint{5.330022in}{1.958405in}}%
\pgfpathlineto{\pgfqpoint{5.337649in}{1.969151in}}%
\pgfpathlineto{\pgfqpoint{5.345270in}{1.979836in}}%
\pgfpathclose%
\pgfusepath{fill}%
\end{pgfscope}%
\begin{pgfscope}%
\pgfpathrectangle{\pgfqpoint{1.150000in}{0.150000in}}{\pgfqpoint{5.700000in}{5.700000in}}%
\pgfusepath{clip}%
\pgfsetbuttcap%
\pgfsetroundjoin%
\definecolor{currentfill}{rgb}{0.277134,0.185228,0.489898}%
\pgfsetfillcolor{currentfill}%
\pgfsetfillopacity{0.700000}%
\pgfsetlinewidth{0.000000pt}%
\definecolor{currentstroke}{rgb}{0.000000,0.000000,0.000000}%
\pgfsetstrokecolor{currentstroke}%
\pgfsetdash{}{0pt}%
\pgfpathmoveto{\pgfqpoint{3.217069in}{2.052538in}}%
\pgfpathlineto{\pgfqpoint{3.230845in}{2.044893in}}%
\pgfpathlineto{\pgfqpoint{3.244626in}{2.037279in}}%
\pgfpathlineto{\pgfqpoint{3.258410in}{2.029696in}}%
\pgfpathlineto{\pgfqpoint{3.272199in}{2.022143in}}%
\pgfpathlineto{\pgfqpoint{3.263663in}{2.026350in}}%
\pgfpathlineto{\pgfqpoint{3.255110in}{2.030961in}}%
\pgfpathlineto{\pgfqpoint{3.246540in}{2.035985in}}%
\pgfpathlineto{\pgfqpoint{3.237953in}{2.041434in}}%
\pgfpathlineto{\pgfqpoint{3.224130in}{2.049329in}}%
\pgfpathlineto{\pgfqpoint{3.210311in}{2.057255in}}%
\pgfpathlineto{\pgfqpoint{3.196496in}{2.065213in}}%
\pgfpathlineto{\pgfqpoint{3.182685in}{2.073201in}}%
\pgfpathlineto{\pgfqpoint{3.191308in}{2.067403in}}%
\pgfpathlineto{\pgfqpoint{3.199912in}{2.062033in}}%
\pgfpathlineto{\pgfqpoint{3.208499in}{2.057081in}}%
\pgfpathlineto{\pgfqpoint{3.217069in}{2.052538in}}%
\pgfpathclose%
\pgfusepath{fill}%
\end{pgfscope}%
\begin{pgfscope}%
\pgfpathrectangle{\pgfqpoint{1.150000in}{0.150000in}}{\pgfqpoint{5.700000in}{5.700000in}}%
\pgfusepath{clip}%
\pgfsetbuttcap%
\pgfsetroundjoin%
\definecolor{currentfill}{rgb}{0.208623,0.367752,0.552675}%
\pgfsetfillcolor{currentfill}%
\pgfsetfillopacity{0.700000}%
\pgfsetlinewidth{0.000000pt}%
\definecolor{currentstroke}{rgb}{0.000000,0.000000,0.000000}%
\pgfsetstrokecolor{currentstroke}%
\pgfsetdash{}{0pt}%
\pgfpathmoveto{\pgfqpoint{2.596574in}{2.462750in}}%
\pgfpathlineto{\pgfqpoint{2.610310in}{2.452970in}}%
\pgfpathlineto{\pgfqpoint{2.624047in}{2.443233in}}%
\pgfpathlineto{\pgfqpoint{2.637787in}{2.433536in}}%
\pgfpathlineto{\pgfqpoint{2.651528in}{2.423880in}}%
\pgfpathlineto{\pgfqpoint{2.642425in}{2.435303in}}%
\pgfpathlineto{\pgfqpoint{2.633295in}{2.447251in}}%
\pgfpathlineto{\pgfqpoint{2.624136in}{2.459735in}}%
\pgfpathlineto{\pgfqpoint{2.614949in}{2.472766in}}%
\pgfpathlineto{\pgfqpoint{2.601160in}{2.482803in}}%
\pgfpathlineto{\pgfqpoint{2.587373in}{2.492881in}}%
\pgfpathlineto{\pgfqpoint{2.573587in}{2.503000in}}%
\pgfpathlineto{\pgfqpoint{2.559804in}{2.513160in}}%
\pgfpathlineto{\pgfqpoint{2.569040in}{2.499742in}}%
\pgfpathlineto{\pgfqpoint{2.578247in}{2.486874in}}%
\pgfpathlineto{\pgfqpoint{2.587425in}{2.474547in}}%
\pgfpathlineto{\pgfqpoint{2.596574in}{2.462750in}}%
\pgfpathclose%
\pgfusepath{fill}%
\end{pgfscope}%
\begin{pgfscope}%
\pgfpathrectangle{\pgfqpoint{1.150000in}{0.150000in}}{\pgfqpoint{5.700000in}{5.700000in}}%
\pgfusepath{clip}%
\pgfsetbuttcap%
\pgfsetroundjoin%
\definecolor{currentfill}{rgb}{0.283187,0.125848,0.444960}%
\pgfsetfillcolor{currentfill}%
\pgfsetfillopacity{0.700000}%
\pgfsetlinewidth{0.000000pt}%
\definecolor{currentstroke}{rgb}{0.000000,0.000000,0.000000}%
\pgfsetstrokecolor{currentstroke}%
\pgfsetdash{}{0pt}%
\pgfpathmoveto{\pgfqpoint{5.257671in}{1.941578in}}%
\pgfpathlineto{\pgfqpoint{5.271927in}{1.940332in}}%
\pgfpathlineto{\pgfqpoint{5.286193in}{1.939110in}}%
\pgfpathlineto{\pgfqpoint{5.300467in}{1.937913in}}%
\pgfpathlineto{\pgfqpoint{5.314750in}{1.936741in}}%
\pgfpathlineto{\pgfqpoint{5.307105in}{1.925828in}}%
\pgfpathlineto{\pgfqpoint{5.299454in}{1.914866in}}%
\pgfpathlineto{\pgfqpoint{5.291797in}{1.903857in}}%
\pgfpathlineto{\pgfqpoint{5.284134in}{1.892805in}}%
\pgfpathlineto{\pgfqpoint{5.269844in}{1.894116in}}%
\pgfpathlineto{\pgfqpoint{5.255563in}{1.895452in}}%
\pgfpathlineto{\pgfqpoint{5.241290in}{1.896812in}}%
\pgfpathlineto{\pgfqpoint{5.227026in}{1.898197in}}%
\pgfpathlineto{\pgfqpoint{5.234696in}{1.909105in}}%
\pgfpathlineto{\pgfqpoint{5.242360in}{1.919973in}}%
\pgfpathlineto{\pgfqpoint{5.250018in}{1.930799in}}%
\pgfpathlineto{\pgfqpoint{5.257671in}{1.941578in}}%
\pgfpathclose%
\pgfusepath{fill}%
\end{pgfscope}%
\begin{pgfscope}%
\pgfpathrectangle{\pgfqpoint{1.150000in}{0.150000in}}{\pgfqpoint{5.700000in}{5.700000in}}%
\pgfusepath{clip}%
\pgfsetbuttcap%
\pgfsetroundjoin%
\definecolor{currentfill}{rgb}{0.271305,0.019942,0.347269}%
\pgfsetfillcolor{currentfill}%
\pgfsetfillopacity{0.700000}%
\pgfsetlinewidth{0.000000pt}%
\definecolor{currentstroke}{rgb}{0.000000,0.000000,0.000000}%
\pgfsetstrokecolor{currentstroke}%
\pgfsetdash{}{0pt}%
\pgfpathmoveto{\pgfqpoint{4.531639in}{1.740529in}}%
\pgfpathlineto{\pgfqpoint{4.545671in}{1.737155in}}%
\pgfpathlineto{\pgfqpoint{4.559710in}{1.733805in}}%
\pgfpathlineto{\pgfqpoint{4.573756in}{1.730480in}}%
\pgfpathlineto{\pgfqpoint{4.587810in}{1.727179in}}%
\pgfpathlineto{\pgfqpoint{4.579951in}{1.718403in}}%
\pgfpathlineto{\pgfqpoint{4.572087in}{1.709730in}}%
\pgfpathlineto{\pgfqpoint{4.564219in}{1.701165in}}%
\pgfpathlineto{\pgfqpoint{4.556346in}{1.692714in}}%
\pgfpathlineto{\pgfqpoint{4.542282in}{1.696246in}}%
\pgfpathlineto{\pgfqpoint{4.528224in}{1.699802in}}%
\pgfpathlineto{\pgfqpoint{4.514174in}{1.703383in}}%
\pgfpathlineto{\pgfqpoint{4.500130in}{1.706989in}}%
\pgfpathlineto{\pgfqpoint{4.508015in}{1.715203in}}%
\pgfpathlineto{\pgfqpoint{4.515894in}{1.723535in}}%
\pgfpathlineto{\pgfqpoint{4.523769in}{1.731979in}}%
\pgfpathlineto{\pgfqpoint{4.531639in}{1.740529in}}%
\pgfpathclose%
\pgfusepath{fill}%
\end{pgfscope}%
\begin{pgfscope}%
\pgfpathrectangle{\pgfqpoint{1.150000in}{0.150000in}}{\pgfqpoint{5.700000in}{5.700000in}}%
\pgfusepath{clip}%
\pgfsetbuttcap%
\pgfsetroundjoin%
\definecolor{currentfill}{rgb}{0.274952,0.037752,0.364543}%
\pgfsetfillcolor{currentfill}%
\pgfsetfillopacity{0.700000}%
\pgfsetlinewidth{0.000000pt}%
\definecolor{currentstroke}{rgb}{0.000000,0.000000,0.000000}%
\pgfsetstrokecolor{currentstroke}%
\pgfsetdash{}{0pt}%
\pgfpathmoveto{\pgfqpoint{4.763038in}{1.778287in}}%
\pgfpathlineto{\pgfqpoint{4.777139in}{1.775630in}}%
\pgfpathlineto{\pgfqpoint{4.791246in}{1.772996in}}%
\pgfpathlineto{\pgfqpoint{4.805362in}{1.770388in}}%
\pgfpathlineto{\pgfqpoint{4.819485in}{1.767803in}}%
\pgfpathlineto{\pgfqpoint{4.811692in}{1.757801in}}%
\pgfpathlineto{\pgfqpoint{4.803895in}{1.747848in}}%
\pgfpathlineto{\pgfqpoint{4.796093in}{1.737952in}}%
\pgfpathlineto{\pgfqpoint{4.788287in}{1.728115in}}%
\pgfpathlineto{\pgfqpoint{4.774155in}{1.730905in}}%
\pgfpathlineto{\pgfqpoint{4.760030in}{1.733718in}}%
\pgfpathlineto{\pgfqpoint{4.745914in}{1.736557in}}%
\pgfpathlineto{\pgfqpoint{4.731804in}{1.739419in}}%
\pgfpathlineto{\pgfqpoint{4.739620in}{1.749045in}}%
\pgfpathlineto{\pgfqpoint{4.747430in}{1.758736in}}%
\pgfpathlineto{\pgfqpoint{4.755237in}{1.768485in}}%
\pgfpathlineto{\pgfqpoint{4.763038in}{1.778287in}}%
\pgfpathclose%
\pgfusepath{fill}%
\end{pgfscope}%
\begin{pgfscope}%
\pgfpathrectangle{\pgfqpoint{1.150000in}{0.150000in}}{\pgfqpoint{5.700000in}{5.700000in}}%
\pgfusepath{clip}%
\pgfsetbuttcap%
\pgfsetroundjoin%
\definecolor{currentfill}{rgb}{0.283091,0.110553,0.431554}%
\pgfsetfillcolor{currentfill}%
\pgfsetfillopacity{0.700000}%
\pgfsetlinewidth{0.000000pt}%
\definecolor{currentstroke}{rgb}{0.000000,0.000000,0.000000}%
\pgfsetstrokecolor{currentstroke}%
\pgfsetdash{}{0pt}%
\pgfpathmoveto{\pgfqpoint{5.170056in}{1.903979in}}%
\pgfpathlineto{\pgfqpoint{5.184286in}{1.902497in}}%
\pgfpathlineto{\pgfqpoint{5.198524in}{1.901039in}}%
\pgfpathlineto{\pgfqpoint{5.212771in}{1.899606in}}%
\pgfpathlineto{\pgfqpoint{5.227026in}{1.898197in}}%
\pgfpathlineto{\pgfqpoint{5.219351in}{1.887252in}}%
\pgfpathlineto{\pgfqpoint{5.211670in}{1.876274in}}%
\pgfpathlineto{\pgfqpoint{5.203984in}{1.865266in}}%
\pgfpathlineto{\pgfqpoint{5.196292in}{1.854232in}}%
\pgfpathlineto{\pgfqpoint{5.182030in}{1.855794in}}%
\pgfpathlineto{\pgfqpoint{5.167776in}{1.857379in}}%
\pgfpathlineto{\pgfqpoint{5.153531in}{1.858989in}}%
\pgfpathlineto{\pgfqpoint{5.139294in}{1.860623in}}%
\pgfpathlineto{\pgfqpoint{5.146992in}{1.871500in}}%
\pgfpathlineto{\pgfqpoint{5.154686in}{1.882354in}}%
\pgfpathlineto{\pgfqpoint{5.162374in}{1.893181in}}%
\pgfpathlineto{\pgfqpoint{5.170056in}{1.903979in}}%
\pgfpathclose%
\pgfusepath{fill}%
\end{pgfscope}%
\begin{pgfscope}%
\pgfpathrectangle{\pgfqpoint{1.150000in}{0.150000in}}{\pgfqpoint{5.700000in}{5.700000in}}%
\pgfusepath{clip}%
\pgfsetbuttcap%
\pgfsetroundjoin%
\definecolor{currentfill}{rgb}{0.271305,0.019942,0.347269}%
\pgfsetfillcolor{currentfill}%
\pgfsetfillopacity{0.700000}%
\pgfsetlinewidth{0.000000pt}%
\definecolor{currentstroke}{rgb}{0.000000,0.000000,0.000000}%
\pgfsetstrokecolor{currentstroke}%
\pgfsetdash{}{0pt}%
\pgfpathmoveto{\pgfqpoint{4.156869in}{1.733574in}}%
\pgfpathlineto{\pgfqpoint{4.170809in}{1.728970in}}%
\pgfpathlineto{\pgfqpoint{4.184755in}{1.724390in}}%
\pgfpathlineto{\pgfqpoint{4.198707in}{1.719836in}}%
\pgfpathlineto{\pgfqpoint{4.212666in}{1.715307in}}%
\pgfpathlineto{\pgfqpoint{4.204677in}{1.709491in}}%
\pgfpathlineto{\pgfqpoint{4.196681in}{1.703869in}}%
\pgfpathlineto{\pgfqpoint{4.188679in}{1.698448in}}%
\pgfpathlineto{\pgfqpoint{4.180671in}{1.693236in}}%
\pgfpathlineto{\pgfqpoint{4.166696in}{1.698036in}}%
\pgfpathlineto{\pgfqpoint{4.152727in}{1.702860in}}%
\pgfpathlineto{\pgfqpoint{4.138764in}{1.707711in}}%
\pgfpathlineto{\pgfqpoint{4.124807in}{1.712586in}}%
\pgfpathlineto{\pgfqpoint{4.132833in}{1.717523in}}%
\pgfpathlineto{\pgfqpoint{4.140852in}{1.722671in}}%
\pgfpathlineto{\pgfqpoint{4.148864in}{1.728024in}}%
\pgfpathlineto{\pgfqpoint{4.156869in}{1.733574in}}%
\pgfpathclose%
\pgfusepath{fill}%
\end{pgfscope}%
\begin{pgfscope}%
\pgfpathrectangle{\pgfqpoint{1.150000in}{0.150000in}}{\pgfqpoint{5.700000in}{5.700000in}}%
\pgfusepath{clip}%
\pgfsetbuttcap%
\pgfsetroundjoin%
\definecolor{currentfill}{rgb}{0.255645,0.260703,0.528312}%
\pgfsetfillcolor{currentfill}%
\pgfsetfillopacity{0.700000}%
\pgfsetlinewidth{0.000000pt}%
\definecolor{currentstroke}{rgb}{0.000000,0.000000,0.000000}%
\pgfsetstrokecolor{currentstroke}%
\pgfsetdash{}{0pt}%
\pgfpathmoveto{\pgfqpoint{2.962184in}{2.205395in}}%
\pgfpathlineto{\pgfqpoint{2.975940in}{2.196883in}}%
\pgfpathlineto{\pgfqpoint{2.989699in}{2.188406in}}%
\pgfpathlineto{\pgfqpoint{3.003462in}{2.179962in}}%
\pgfpathlineto{\pgfqpoint{3.017228in}{2.171553in}}%
\pgfpathlineto{\pgfqpoint{3.008474in}{2.178848in}}%
\pgfpathlineto{\pgfqpoint{2.999699in}{2.186601in}}%
\pgfpathlineto{\pgfqpoint{2.990903in}{2.194824in}}%
\pgfpathlineto{\pgfqpoint{2.982084in}{2.203526in}}%
\pgfpathlineto{\pgfqpoint{2.968279in}{2.212296in}}%
\pgfpathlineto{\pgfqpoint{2.954476in}{2.221100in}}%
\pgfpathlineto{\pgfqpoint{2.940677in}{2.229938in}}%
\pgfpathlineto{\pgfqpoint{2.926880in}{2.238810in}}%
\pgfpathlineto{\pgfqpoint{2.935740in}{2.229742in}}%
\pgfpathlineto{\pgfqpoint{2.944576in}{2.221157in}}%
\pgfpathlineto{\pgfqpoint{2.953391in}{2.213045in}}%
\pgfpathlineto{\pgfqpoint{2.962184in}{2.205395in}}%
\pgfpathclose%
\pgfusepath{fill}%
\end{pgfscope}%
\begin{pgfscope}%
\pgfpathrectangle{\pgfqpoint{1.150000in}{0.150000in}}{\pgfqpoint{5.700000in}{5.700000in}}%
\pgfusepath{clip}%
\pgfsetbuttcap%
\pgfsetroundjoin%
\definecolor{currentfill}{rgb}{0.281446,0.084320,0.407414}%
\pgfsetfillcolor{currentfill}%
\pgfsetfillopacity{0.700000}%
\pgfsetlinewidth{0.000000pt}%
\definecolor{currentstroke}{rgb}{0.000000,0.000000,0.000000}%
\pgfsetstrokecolor{currentstroke}%
\pgfsetdash{}{0pt}%
\pgfpathmoveto{\pgfqpoint{3.670733in}{1.844907in}}%
\pgfpathlineto{\pgfqpoint{3.684576in}{1.838716in}}%
\pgfpathlineto{\pgfqpoint{3.698425in}{1.832552in}}%
\pgfpathlineto{\pgfqpoint{3.712279in}{1.826415in}}%
\pgfpathlineto{\pgfqpoint{3.726137in}{1.820306in}}%
\pgfpathlineto{\pgfqpoint{3.717910in}{1.819448in}}%
\pgfpathlineto{\pgfqpoint{3.709671in}{1.818899in}}%
\pgfpathlineto{\pgfqpoint{3.701422in}{1.818667in}}%
\pgfpathlineto{\pgfqpoint{3.693161in}{1.818761in}}%
\pgfpathlineto{\pgfqpoint{3.679277in}{1.825183in}}%
\pgfpathlineto{\pgfqpoint{3.665398in}{1.831632in}}%
\pgfpathlineto{\pgfqpoint{3.651524in}{1.838108in}}%
\pgfpathlineto{\pgfqpoint{3.637655in}{1.844611in}}%
\pgfpathlineto{\pgfqpoint{3.645941in}{1.844200in}}%
\pgfpathlineto{\pgfqpoint{3.654217in}{1.844118in}}%
\pgfpathlineto{\pgfqpoint{3.662480in}{1.844356in}}%
\pgfpathlineto{\pgfqpoint{3.670733in}{1.844907in}}%
\pgfpathclose%
\pgfusepath{fill}%
\end{pgfscope}%
\begin{pgfscope}%
\pgfpathrectangle{\pgfqpoint{1.150000in}{0.150000in}}{\pgfqpoint{5.700000in}{5.700000in}}%
\pgfusepath{clip}%
\pgfsetbuttcap%
\pgfsetroundjoin%
\definecolor{currentfill}{rgb}{0.273809,0.031497,0.358853}%
\pgfsetfillcolor{currentfill}%
\pgfsetfillopacity{0.700000}%
\pgfsetlinewidth{0.000000pt}%
\definecolor{currentstroke}{rgb}{0.000000,0.000000,0.000000}%
\pgfsetstrokecolor{currentstroke}%
\pgfsetdash{}{0pt}%
\pgfpathmoveto{\pgfqpoint{4.013368in}{1.752505in}}%
\pgfpathlineto{\pgfqpoint{4.027278in}{1.747425in}}%
\pgfpathlineto{\pgfqpoint{4.041193in}{1.742372in}}%
\pgfpathlineto{\pgfqpoint{4.055114in}{1.737344in}}%
\pgfpathlineto{\pgfqpoint{4.069040in}{1.732341in}}%
\pgfpathlineto{\pgfqpoint{4.060990in}{1.727902in}}%
\pgfpathlineto{\pgfqpoint{4.052932in}{1.723693in}}%
\pgfpathlineto{\pgfqpoint{4.044866in}{1.719721in}}%
\pgfpathlineto{\pgfqpoint{4.036793in}{1.715993in}}%
\pgfpathlineto{\pgfqpoint{4.022848in}{1.721280in}}%
\pgfpathlineto{\pgfqpoint{4.008908in}{1.726592in}}%
\pgfpathlineto{\pgfqpoint{3.994974in}{1.731930in}}%
\pgfpathlineto{\pgfqpoint{3.981045in}{1.737293in}}%
\pgfpathlineto{\pgfqpoint{3.989138in}{1.740732in}}%
\pgfpathlineto{\pgfqpoint{3.997222in}{1.744418in}}%
\pgfpathlineto{\pgfqpoint{4.005299in}{1.748345in}}%
\pgfpathlineto{\pgfqpoint{4.013368in}{1.752505in}}%
\pgfpathclose%
\pgfusepath{fill}%
\end{pgfscope}%
\begin{pgfscope}%
\pgfpathrectangle{\pgfqpoint{1.150000in}{0.150000in}}{\pgfqpoint{5.700000in}{5.700000in}}%
\pgfusepath{clip}%
\pgfsetbuttcap%
\pgfsetroundjoin%
\definecolor{currentfill}{rgb}{0.283187,0.125848,0.444960}%
\pgfsetfillcolor{currentfill}%
\pgfsetfillopacity{0.700000}%
\pgfsetlinewidth{0.000000pt}%
\definecolor{currentstroke}{rgb}{0.000000,0.000000,0.000000}%
\pgfsetstrokecolor{currentstroke}%
\pgfsetdash{}{0pt}%
\pgfpathmoveto{\pgfqpoint{3.471594in}{1.924822in}}%
\pgfpathlineto{\pgfqpoint{3.485407in}{1.917983in}}%
\pgfpathlineto{\pgfqpoint{3.499224in}{1.911172in}}%
\pgfpathlineto{\pgfqpoint{3.513046in}{1.904389in}}%
\pgfpathlineto{\pgfqpoint{3.526873in}{1.897635in}}%
\pgfpathlineto{\pgfqpoint{3.518519in}{1.899027in}}%
\pgfpathlineto{\pgfqpoint{3.510151in}{1.900772in}}%
\pgfpathlineto{\pgfqpoint{3.501771in}{1.902880in}}%
\pgfpathlineto{\pgfqpoint{3.493376in}{1.905359in}}%
\pgfpathlineto{\pgfqpoint{3.479521in}{1.912440in}}%
\pgfpathlineto{\pgfqpoint{3.465669in}{1.919550in}}%
\pgfpathlineto{\pgfqpoint{3.451822in}{1.926688in}}%
\pgfpathlineto{\pgfqpoint{3.437979in}{1.933854in}}%
\pgfpathlineto{\pgfqpoint{3.446404in}{1.931043in}}%
\pgfpathlineto{\pgfqpoint{3.454815in}{1.928606in}}%
\pgfpathlineto{\pgfqpoint{3.463211in}{1.926535in}}%
\pgfpathlineto{\pgfqpoint{3.471594in}{1.924822in}}%
\pgfpathclose%
\pgfusepath{fill}%
\end{pgfscope}%
\begin{pgfscope}%
\pgfpathrectangle{\pgfqpoint{1.150000in}{0.150000in}}{\pgfqpoint{5.700000in}{5.700000in}}%
\pgfusepath{clip}%
\pgfsetbuttcap%
\pgfsetroundjoin%
\definecolor{currentfill}{rgb}{0.269944,0.014625,0.341379}%
\pgfsetfillcolor{currentfill}%
\pgfsetfillopacity{0.700000}%
\pgfsetlinewidth{0.000000pt}%
\definecolor{currentstroke}{rgb}{0.000000,0.000000,0.000000}%
\pgfsetstrokecolor{currentstroke}%
\pgfsetdash{}{0pt}%
\pgfpathmoveto{\pgfqpoint{4.300399in}{1.723545in}}%
\pgfpathlineto{\pgfqpoint{4.314374in}{1.719399in}}%
\pgfpathlineto{\pgfqpoint{4.328356in}{1.715277in}}%
\pgfpathlineto{\pgfqpoint{4.342345in}{1.711181in}}%
\pgfpathlineto{\pgfqpoint{4.356340in}{1.707109in}}%
\pgfpathlineto{\pgfqpoint{4.348404in}{1.700077in}}%
\pgfpathlineto{\pgfqpoint{4.340463in}{1.693207in}}%
\pgfpathlineto{\pgfqpoint{4.332516in}{1.686504in}}%
\pgfpathlineto{\pgfqpoint{4.324563in}{1.679975in}}%
\pgfpathlineto{\pgfqpoint{4.310553in}{1.684305in}}%
\pgfpathlineto{\pgfqpoint{4.296550in}{1.688659in}}%
\pgfpathlineto{\pgfqpoint{4.282554in}{1.693038in}}%
\pgfpathlineto{\pgfqpoint{4.268563in}{1.697442in}}%
\pgfpathlineto{\pgfqpoint{4.276531in}{1.703708in}}%
\pgfpathlineto{\pgfqpoint{4.284493in}{1.710152in}}%
\pgfpathlineto{\pgfqpoint{4.292449in}{1.716766in}}%
\pgfpathlineto{\pgfqpoint{4.300399in}{1.723545in}}%
\pgfpathclose%
\pgfusepath{fill}%
\end{pgfscope}%
\begin{pgfscope}%
\pgfpathrectangle{\pgfqpoint{1.150000in}{0.150000in}}{\pgfqpoint{5.700000in}{5.700000in}}%
\pgfusepath{clip}%
\pgfsetbuttcap%
\pgfsetroundjoin%
\definecolor{currentfill}{rgb}{0.281924,0.089666,0.412415}%
\pgfsetfillcolor{currentfill}%
\pgfsetfillopacity{0.700000}%
\pgfsetlinewidth{0.000000pt}%
\definecolor{currentstroke}{rgb}{0.000000,0.000000,0.000000}%
\pgfsetstrokecolor{currentstroke}%
\pgfsetdash{}{0pt}%
\pgfpathmoveto{\pgfqpoint{5.082430in}{1.867404in}}%
\pgfpathlineto{\pgfqpoint{5.096633in}{1.865672in}}%
\pgfpathlineto{\pgfqpoint{5.110845in}{1.863965in}}%
\pgfpathlineto{\pgfqpoint{5.125065in}{1.862282in}}%
\pgfpathlineto{\pgfqpoint{5.139294in}{1.860623in}}%
\pgfpathlineto{\pgfqpoint{5.131590in}{1.849728in}}%
\pgfpathlineto{\pgfqpoint{5.123881in}{1.838817in}}%
\pgfpathlineto{\pgfqpoint{5.116167in}{1.827894in}}%
\pgfpathlineto{\pgfqpoint{5.108447in}{1.816964in}}%
\pgfpathlineto{\pgfqpoint{5.094212in}{1.818789in}}%
\pgfpathlineto{\pgfqpoint{5.079985in}{1.820637in}}%
\pgfpathlineto{\pgfqpoint{5.065766in}{1.822510in}}%
\pgfpathlineto{\pgfqpoint{5.051555in}{1.824407in}}%
\pgfpathlineto{\pgfqpoint{5.059281in}{1.835167in}}%
\pgfpathlineto{\pgfqpoint{5.067002in}{1.845922in}}%
\pgfpathlineto{\pgfqpoint{5.074719in}{1.856669in}}%
\pgfpathlineto{\pgfqpoint{5.082430in}{1.867404in}}%
\pgfpathclose%
\pgfusepath{fill}%
\end{pgfscope}%
\begin{pgfscope}%
\pgfpathrectangle{\pgfqpoint{1.150000in}{0.150000in}}{\pgfqpoint{5.700000in}{5.700000in}}%
\pgfusepath{clip}%
\pgfsetbuttcap%
\pgfsetroundjoin%
\definecolor{currentfill}{rgb}{0.150476,0.504369,0.557430}%
\pgfsetfillcolor{currentfill}%
\pgfsetfillopacity{0.700000}%
\pgfsetlinewidth{0.000000pt}%
\definecolor{currentstroke}{rgb}{0.000000,0.000000,0.000000}%
\pgfsetstrokecolor{currentstroke}%
\pgfsetdash{}{0pt}%
\pgfpathmoveto{\pgfqpoint{2.174429in}{2.816183in}}%
\pgfpathlineto{\pgfqpoint{2.188181in}{2.804686in}}%
\pgfpathlineto{\pgfqpoint{2.201933in}{2.793243in}}%
\pgfpathlineto{\pgfqpoint{2.215685in}{2.781855in}}%
\pgfpathlineto{\pgfqpoint{2.229438in}{2.770519in}}%
\pgfpathlineto{\pgfqpoint{2.219855in}{2.786872in}}%
\pgfpathlineto{\pgfqpoint{2.210237in}{2.803825in}}%
\pgfpathlineto{\pgfqpoint{2.200582in}{2.821391in}}%
\pgfpathlineto{\pgfqpoint{2.190888in}{2.839581in}}%
\pgfpathlineto{\pgfqpoint{2.177079in}{2.851323in}}%
\pgfpathlineto{\pgfqpoint{2.163270in}{2.863118in}}%
\pgfpathlineto{\pgfqpoint{2.149460in}{2.874967in}}%
\pgfpathlineto{\pgfqpoint{2.135651in}{2.886871in}}%
\pgfpathlineto{\pgfqpoint{2.145403in}{2.868267in}}%
\pgfpathlineto{\pgfqpoint{2.155116in}{2.850293in}}%
\pgfpathlineto{\pgfqpoint{2.164791in}{2.832935in}}%
\pgfpathlineto{\pgfqpoint{2.174429in}{2.816183in}}%
\pgfpathclose%
\pgfusepath{fill}%
\end{pgfscope}%
\begin{pgfscope}%
\pgfpathrectangle{\pgfqpoint{1.150000in}{0.150000in}}{\pgfqpoint{5.700000in}{5.700000in}}%
\pgfusepath{clip}%
\pgfsetbuttcap%
\pgfsetroundjoin%
\definecolor{currentfill}{rgb}{0.214298,0.355619,0.551184}%
\pgfsetfillcolor{currentfill}%
\pgfsetfillopacity{0.700000}%
\pgfsetlinewidth{0.000000pt}%
\definecolor{currentstroke}{rgb}{0.000000,0.000000,0.000000}%
\pgfsetstrokecolor{currentstroke}%
\pgfsetdash{}{0pt}%
\pgfpathmoveto{\pgfqpoint{2.651528in}{2.423880in}}%
\pgfpathlineto{\pgfqpoint{2.665272in}{2.414264in}}%
\pgfpathlineto{\pgfqpoint{2.679019in}{2.404688in}}%
\pgfpathlineto{\pgfqpoint{2.692767in}{2.395151in}}%
\pgfpathlineto{\pgfqpoint{2.706518in}{2.385655in}}%
\pgfpathlineto{\pgfqpoint{2.697461in}{2.396704in}}%
\pgfpathlineto{\pgfqpoint{2.688376in}{2.408275in}}%
\pgfpathlineto{\pgfqpoint{2.679265in}{2.420377in}}%
\pgfpathlineto{\pgfqpoint{2.670126in}{2.433021in}}%
\pgfpathlineto{\pgfqpoint{2.656329in}{2.442898in}}%
\pgfpathlineto{\pgfqpoint{2.642533in}{2.452814in}}%
\pgfpathlineto{\pgfqpoint{2.628740in}{2.462770in}}%
\pgfpathlineto{\pgfqpoint{2.614949in}{2.472766in}}%
\pgfpathlineto{\pgfqpoint{2.624136in}{2.459735in}}%
\pgfpathlineto{\pgfqpoint{2.633295in}{2.447251in}}%
\pgfpathlineto{\pgfqpoint{2.642425in}{2.435303in}}%
\pgfpathlineto{\pgfqpoint{2.651528in}{2.423880in}}%
\pgfpathclose%
\pgfusepath{fill}%
\end{pgfscope}%
\begin{pgfscope}%
\pgfpathrectangle{\pgfqpoint{1.150000in}{0.150000in}}{\pgfqpoint{5.700000in}{5.700000in}}%
\pgfusepath{clip}%
\pgfsetbuttcap%
\pgfsetroundjoin%
\definecolor{currentfill}{rgb}{0.277018,0.050344,0.375715}%
\pgfsetfillcolor{currentfill}%
\pgfsetfillopacity{0.700000}%
\pgfsetlinewidth{0.000000pt}%
\definecolor{currentstroke}{rgb}{0.000000,0.000000,0.000000}%
\pgfsetstrokecolor{currentstroke}%
\pgfsetdash{}{0pt}%
\pgfpathmoveto{\pgfqpoint{3.869820in}{1.781135in}}%
\pgfpathlineto{\pgfqpoint{3.883704in}{1.775564in}}%
\pgfpathlineto{\pgfqpoint{3.897593in}{1.770018in}}%
\pgfpathlineto{\pgfqpoint{3.911487in}{1.764499in}}%
\pgfpathlineto{\pgfqpoint{3.925388in}{1.759006in}}%
\pgfpathlineto{\pgfqpoint{3.917266in}{1.756115in}}%
\pgfpathlineto{\pgfqpoint{3.909136in}{1.753491in}}%
\pgfpathlineto{\pgfqpoint{3.900997in}{1.751141in}}%
\pgfpathlineto{\pgfqpoint{3.892849in}{1.749073in}}%
\pgfpathlineto{\pgfqpoint{3.878927in}{1.754864in}}%
\pgfpathlineto{\pgfqpoint{3.865011in}{1.760681in}}%
\pgfpathlineto{\pgfqpoint{3.851100in}{1.766524in}}%
\pgfpathlineto{\pgfqpoint{3.837194in}{1.772394in}}%
\pgfpathlineto{\pgfqpoint{3.845364in}{1.774159in}}%
\pgfpathlineto{\pgfqpoint{3.853525in}{1.776209in}}%
\pgfpathlineto{\pgfqpoint{3.861677in}{1.778537in}}%
\pgfpathlineto{\pgfqpoint{3.869820in}{1.781135in}}%
\pgfpathclose%
\pgfusepath{fill}%
\end{pgfscope}%
\begin{pgfscope}%
\pgfpathrectangle{\pgfqpoint{1.150000in}{0.150000in}}{\pgfqpoint{5.700000in}{5.700000in}}%
\pgfusepath{clip}%
\pgfsetbuttcap%
\pgfsetroundjoin%
\definecolor{currentfill}{rgb}{0.272594,0.025563,0.353093}%
\pgfsetfillcolor{currentfill}%
\pgfsetfillopacity{0.700000}%
\pgfsetlinewidth{0.000000pt}%
\definecolor{currentstroke}{rgb}{0.000000,0.000000,0.000000}%
\pgfsetstrokecolor{currentstroke}%
\pgfsetdash{}{0pt}%
\pgfpathmoveto{\pgfqpoint{4.675441in}{1.751113in}}%
\pgfpathlineto{\pgfqpoint{4.689521in}{1.748153in}}%
\pgfpathlineto{\pgfqpoint{4.703608in}{1.745217in}}%
\pgfpathlineto{\pgfqpoint{4.717702in}{1.742306in}}%
\pgfpathlineto{\pgfqpoint{4.731804in}{1.739419in}}%
\pgfpathlineto{\pgfqpoint{4.723984in}{1.729862in}}%
\pgfpathlineto{\pgfqpoint{4.716160in}{1.720378in}}%
\pgfpathlineto{\pgfqpoint{4.708331in}{1.710974in}}%
\pgfpathlineto{\pgfqpoint{4.700497in}{1.701655in}}%
\pgfpathlineto{\pgfqpoint{4.686386in}{1.704760in}}%
\pgfpathlineto{\pgfqpoint{4.672282in}{1.707890in}}%
\pgfpathlineto{\pgfqpoint{4.658185in}{1.711043in}}%
\pgfpathlineto{\pgfqpoint{4.644095in}{1.714222in}}%
\pgfpathlineto{\pgfqpoint{4.651939in}{1.723318in}}%
\pgfpathlineto{\pgfqpoint{4.659778in}{1.732502in}}%
\pgfpathlineto{\pgfqpoint{4.667612in}{1.741769in}}%
\pgfpathlineto{\pgfqpoint{4.675441in}{1.751113in}}%
\pgfpathclose%
\pgfusepath{fill}%
\end{pgfscope}%
\begin{pgfscope}%
\pgfpathrectangle{\pgfqpoint{1.150000in}{0.150000in}}{\pgfqpoint{5.700000in}{5.700000in}}%
\pgfusepath{clip}%
\pgfsetbuttcap%
\pgfsetroundjoin%
\definecolor{currentfill}{rgb}{0.269944,0.014625,0.341379}%
\pgfsetfillcolor{currentfill}%
\pgfsetfillopacity{0.700000}%
\pgfsetlinewidth{0.000000pt}%
\definecolor{currentstroke}{rgb}{0.000000,0.000000,0.000000}%
\pgfsetstrokecolor{currentstroke}%
\pgfsetdash{}{0pt}%
\pgfpathmoveto{\pgfqpoint{4.444025in}{1.721656in}}%
\pgfpathlineto{\pgfqpoint{4.458041in}{1.717952in}}%
\pgfpathlineto{\pgfqpoint{4.472064in}{1.714273in}}%
\pgfpathlineto{\pgfqpoint{4.486094in}{1.710618in}}%
\pgfpathlineto{\pgfqpoint{4.500130in}{1.706989in}}%
\pgfpathlineto{\pgfqpoint{4.492241in}{1.698897in}}%
\pgfpathlineto{\pgfqpoint{4.484347in}{1.690935in}}%
\pgfpathlineto{\pgfqpoint{4.476447in}{1.683109in}}%
\pgfpathlineto{\pgfqpoint{4.468543in}{1.675424in}}%
\pgfpathlineto{\pgfqpoint{4.454494in}{1.679299in}}%
\pgfpathlineto{\pgfqpoint{4.440452in}{1.683198in}}%
\pgfpathlineto{\pgfqpoint{4.426416in}{1.687121in}}%
\pgfpathlineto{\pgfqpoint{4.412388in}{1.691069in}}%
\pgfpathlineto{\pgfqpoint{4.420305in}{1.698505in}}%
\pgfpathlineto{\pgfqpoint{4.428217in}{1.706085in}}%
\pgfpathlineto{\pgfqpoint{4.436124in}{1.713804in}}%
\pgfpathlineto{\pgfqpoint{4.444025in}{1.721656in}}%
\pgfpathclose%
\pgfusepath{fill}%
\end{pgfscope}%
\begin{pgfscope}%
\pgfpathrectangle{\pgfqpoint{1.150000in}{0.150000in}}{\pgfqpoint{5.700000in}{5.700000in}}%
\pgfusepath{clip}%
\pgfsetbuttcap%
\pgfsetroundjoin%
\definecolor{currentfill}{rgb}{0.277134,0.185228,0.489898}%
\pgfsetfillcolor{currentfill}%
\pgfsetfillopacity{0.700000}%
\pgfsetlinewidth{0.000000pt}%
\definecolor{currentstroke}{rgb}{0.000000,0.000000,0.000000}%
\pgfsetstrokecolor{currentstroke}%
\pgfsetdash{}{0pt}%
\pgfpathmoveto{\pgfqpoint{5.577817in}{2.054663in}}%
\pgfpathlineto{\pgfqpoint{5.592195in}{2.054144in}}%
\pgfpathlineto{\pgfqpoint{5.606582in}{2.053649in}}%
\pgfpathlineto{\pgfqpoint{5.620978in}{2.053179in}}%
\pgfpathlineto{\pgfqpoint{5.613442in}{2.042724in}}%
\pgfpathlineto{\pgfqpoint{5.605898in}{2.032177in}}%
\pgfpathlineto{\pgfqpoint{5.598346in}{2.021539in}}%
\pgfpathlineto{\pgfqpoint{5.590787in}{2.010812in}}%
\pgfpathlineto{\pgfqpoint{5.576383in}{2.011380in}}%
\pgfpathlineto{\pgfqpoint{5.561988in}{2.011973in}}%
\pgfpathlineto{\pgfqpoint{5.547603in}{2.012590in}}%
\pgfpathlineto{\pgfqpoint{5.555167in}{2.023240in}}%
\pgfpathlineto{\pgfqpoint{5.562725in}{2.033803in}}%
\pgfpathlineto{\pgfqpoint{5.570274in}{2.044278in}}%
\pgfpathlineto{\pgfqpoint{5.577817in}{2.054663in}}%
\pgfpathclose%
\pgfusepath{fill}%
\end{pgfscope}%
\begin{pgfscope}%
\pgfpathrectangle{\pgfqpoint{1.150000in}{0.150000in}}{\pgfqpoint{5.700000in}{5.700000in}}%
\pgfusepath{clip}%
\pgfsetbuttcap%
\pgfsetroundjoin%
\definecolor{currentfill}{rgb}{0.280267,0.073417,0.397163}%
\pgfsetfillcolor{currentfill}%
\pgfsetfillopacity{0.700000}%
\pgfsetlinewidth{0.000000pt}%
\definecolor{currentstroke}{rgb}{0.000000,0.000000,0.000000}%
\pgfsetstrokecolor{currentstroke}%
\pgfsetdash{}{0pt}%
\pgfpathmoveto{\pgfqpoint{4.994793in}{1.832240in}}%
\pgfpathlineto{\pgfqpoint{5.008971in}{1.830245in}}%
\pgfpathlineto{\pgfqpoint{5.023158in}{1.828275in}}%
\pgfpathlineto{\pgfqpoint{5.037352in}{1.826329in}}%
\pgfpathlineto{\pgfqpoint{5.051555in}{1.824407in}}%
\pgfpathlineto{\pgfqpoint{5.043823in}{1.813648in}}%
\pgfpathlineto{\pgfqpoint{5.036087in}{1.802893in}}%
\pgfpathlineto{\pgfqpoint{5.028346in}{1.792145in}}%
\pgfpathlineto{\pgfqpoint{5.020600in}{1.781411in}}%
\pgfpathlineto{\pgfqpoint{5.006391in}{1.783511in}}%
\pgfpathlineto{\pgfqpoint{4.992189in}{1.785636in}}%
\pgfpathlineto{\pgfqpoint{4.977995in}{1.787785in}}%
\pgfpathlineto{\pgfqpoint{4.963809in}{1.789958in}}%
\pgfpathlineto{\pgfqpoint{4.971562in}{1.800509in}}%
\pgfpathlineto{\pgfqpoint{4.979311in}{1.811076in}}%
\pgfpathlineto{\pgfqpoint{4.987055in}{1.821654in}}%
\pgfpathlineto{\pgfqpoint{4.994793in}{1.832240in}}%
\pgfpathclose%
\pgfusepath{fill}%
\end{pgfscope}%
\begin{pgfscope}%
\pgfpathrectangle{\pgfqpoint{1.150000in}{0.150000in}}{\pgfqpoint{5.700000in}{5.700000in}}%
\pgfusepath{clip}%
\pgfsetbuttcap%
\pgfsetroundjoin%
\definecolor{currentfill}{rgb}{0.278012,0.180367,0.486697}%
\pgfsetfillcolor{currentfill}%
\pgfsetfillopacity{0.700000}%
\pgfsetlinewidth{0.000000pt}%
\definecolor{currentstroke}{rgb}{0.000000,0.000000,0.000000}%
\pgfsetstrokecolor{currentstroke}%
\pgfsetdash{}{0pt}%
\pgfpathmoveto{\pgfqpoint{3.272199in}{2.022143in}}%
\pgfpathlineto{\pgfqpoint{3.285991in}{2.014622in}}%
\pgfpathlineto{\pgfqpoint{3.299788in}{2.007130in}}%
\pgfpathlineto{\pgfqpoint{3.313588in}{1.999669in}}%
\pgfpathlineto{\pgfqpoint{3.327393in}{1.992238in}}%
\pgfpathlineto{\pgfqpoint{3.318889in}{1.996107in}}%
\pgfpathlineto{\pgfqpoint{3.310370in}{2.000377in}}%
\pgfpathlineto{\pgfqpoint{3.301834in}{2.005057in}}%
\pgfpathlineto{\pgfqpoint{3.293282in}{2.010158in}}%
\pgfpathlineto{\pgfqpoint{3.279444in}{2.017931in}}%
\pgfpathlineto{\pgfqpoint{3.265609in}{2.025735in}}%
\pgfpathlineto{\pgfqpoint{3.251779in}{2.033569in}}%
\pgfpathlineto{\pgfqpoint{3.237953in}{2.041434in}}%
\pgfpathlineto{\pgfqpoint{3.246540in}{2.035985in}}%
\pgfpathlineto{\pgfqpoint{3.255110in}{2.030961in}}%
\pgfpathlineto{\pgfqpoint{3.263663in}{2.026350in}}%
\pgfpathlineto{\pgfqpoint{3.272199in}{2.022143in}}%
\pgfpathclose%
\pgfusepath{fill}%
\end{pgfscope}%
\begin{pgfscope}%
\pgfpathrectangle{\pgfqpoint{1.150000in}{0.150000in}}{\pgfqpoint{5.700000in}{5.700000in}}%
\pgfusepath{clip}%
\pgfsetbuttcap%
\pgfsetroundjoin%
\definecolor{currentfill}{rgb}{0.156270,0.489624,0.557936}%
\pgfsetfillcolor{currentfill}%
\pgfsetfillopacity{0.700000}%
\pgfsetlinewidth{0.000000pt}%
\definecolor{currentstroke}{rgb}{0.000000,0.000000,0.000000}%
\pgfsetstrokecolor{currentstroke}%
\pgfsetdash{}{0pt}%
\pgfpathmoveto{\pgfqpoint{2.229438in}{2.770519in}}%
\pgfpathlineto{\pgfqpoint{2.243191in}{2.759237in}}%
\pgfpathlineto{\pgfqpoint{2.256945in}{2.748006in}}%
\pgfpathlineto{\pgfqpoint{2.270700in}{2.736828in}}%
\pgfpathlineto{\pgfqpoint{2.284455in}{2.725700in}}%
\pgfpathlineto{\pgfqpoint{2.274928in}{2.741655in}}%
\pgfpathlineto{\pgfqpoint{2.265365in}{2.758205in}}%
\pgfpathlineto{\pgfqpoint{2.255766in}{2.775364in}}%
\pgfpathlineto{\pgfqpoint{2.246131in}{2.793142in}}%
\pgfpathlineto{\pgfqpoint{2.232319in}{2.804674in}}%
\pgfpathlineto{\pgfqpoint{2.218509in}{2.816258in}}%
\pgfpathlineto{\pgfqpoint{2.204698in}{2.827893in}}%
\pgfpathlineto{\pgfqpoint{2.190888in}{2.839581in}}%
\pgfpathlineto{\pgfqpoint{2.200582in}{2.821391in}}%
\pgfpathlineto{\pgfqpoint{2.210237in}{2.803825in}}%
\pgfpathlineto{\pgfqpoint{2.219855in}{2.786872in}}%
\pgfpathlineto{\pgfqpoint{2.229438in}{2.770519in}}%
\pgfpathclose%
\pgfusepath{fill}%
\end{pgfscope}%
\begin{pgfscope}%
\pgfpathrectangle{\pgfqpoint{1.150000in}{0.150000in}}{\pgfqpoint{5.700000in}{5.700000in}}%
\pgfusepath{clip}%
\pgfsetbuttcap%
\pgfsetroundjoin%
\definecolor{currentfill}{rgb}{0.258965,0.251537,0.524736}%
\pgfsetfillcolor{currentfill}%
\pgfsetfillopacity{0.700000}%
\pgfsetlinewidth{0.000000pt}%
\definecolor{currentstroke}{rgb}{0.000000,0.000000,0.000000}%
\pgfsetstrokecolor{currentstroke}%
\pgfsetdash{}{0pt}%
\pgfpathmoveto{\pgfqpoint{3.017228in}{2.171553in}}%
\pgfpathlineto{\pgfqpoint{3.030997in}{2.163177in}}%
\pgfpathlineto{\pgfqpoint{3.044769in}{2.154834in}}%
\pgfpathlineto{\pgfqpoint{3.058545in}{2.146524in}}%
\pgfpathlineto{\pgfqpoint{3.072324in}{2.138248in}}%
\pgfpathlineto{\pgfqpoint{3.063609in}{2.145189in}}%
\pgfpathlineto{\pgfqpoint{3.054873in}{2.152585in}}%
\pgfpathlineto{\pgfqpoint{3.046116in}{2.160446in}}%
\pgfpathlineto{\pgfqpoint{3.037338in}{2.168782in}}%
\pgfpathlineto{\pgfqpoint{3.023520in}{2.177418in}}%
\pgfpathlineto{\pgfqpoint{3.009705in}{2.186087in}}%
\pgfpathlineto{\pgfqpoint{2.995893in}{2.194790in}}%
\pgfpathlineto{\pgfqpoint{2.982084in}{2.203526in}}%
\pgfpathlineto{\pgfqpoint{2.990903in}{2.194824in}}%
\pgfpathlineto{\pgfqpoint{2.999699in}{2.186601in}}%
\pgfpathlineto{\pgfqpoint{3.008474in}{2.178848in}}%
\pgfpathlineto{\pgfqpoint{3.017228in}{2.171553in}}%
\pgfpathclose%
\pgfusepath{fill}%
\end{pgfscope}%
\begin{pgfscope}%
\pgfpathrectangle{\pgfqpoint{1.150000in}{0.150000in}}{\pgfqpoint{5.700000in}{5.700000in}}%
\pgfusepath{clip}%
\pgfsetbuttcap%
\pgfsetroundjoin%
\definecolor{currentfill}{rgb}{0.278791,0.062145,0.386592}%
\pgfsetfillcolor{currentfill}%
\pgfsetfillopacity{0.700000}%
\pgfsetlinewidth{0.000000pt}%
\definecolor{currentstroke}{rgb}{0.000000,0.000000,0.000000}%
\pgfsetstrokecolor{currentstroke}%
\pgfsetdash{}{0pt}%
\pgfpathmoveto{\pgfqpoint{4.907146in}{1.798895in}}%
\pgfpathlineto{\pgfqpoint{4.921300in}{1.796625in}}%
\pgfpathlineto{\pgfqpoint{4.935462in}{1.794378in}}%
\pgfpathlineto{\pgfqpoint{4.949631in}{1.792156in}}%
\pgfpathlineto{\pgfqpoint{4.963809in}{1.789958in}}%
\pgfpathlineto{\pgfqpoint{4.956051in}{1.779428in}}%
\pgfpathlineto{\pgfqpoint{4.948289in}{1.768923in}}%
\pgfpathlineto{\pgfqpoint{4.940521in}{1.758446in}}%
\pgfpathlineto{\pgfqpoint{4.932750in}{1.748004in}}%
\pgfpathlineto{\pgfqpoint{4.918564in}{1.750394in}}%
\pgfpathlineto{\pgfqpoint{4.904387in}{1.752808in}}%
\pgfpathlineto{\pgfqpoint{4.890217in}{1.755247in}}%
\pgfpathlineto{\pgfqpoint{4.876055in}{1.757709in}}%
\pgfpathlineto{\pgfqpoint{4.883835in}{1.767954in}}%
\pgfpathlineto{\pgfqpoint{4.891610in}{1.778237in}}%
\pgfpathlineto{\pgfqpoint{4.899380in}{1.788552in}}%
\pgfpathlineto{\pgfqpoint{4.907146in}{1.798895in}}%
\pgfpathclose%
\pgfusepath{fill}%
\end{pgfscope}%
\begin{pgfscope}%
\pgfpathrectangle{\pgfqpoint{1.150000in}{0.150000in}}{\pgfqpoint{5.700000in}{5.700000in}}%
\pgfusepath{clip}%
\pgfsetbuttcap%
\pgfsetroundjoin%
\definecolor{currentfill}{rgb}{0.279574,0.170599,0.479997}%
\pgfsetfillcolor{currentfill}%
\pgfsetfillopacity{0.700000}%
\pgfsetlinewidth{0.000000pt}%
\definecolor{currentstroke}{rgb}{0.000000,0.000000,0.000000}%
\pgfsetstrokecolor{currentstroke}%
\pgfsetdash{}{0pt}%
\pgfpathmoveto{\pgfqpoint{5.490153in}{2.015305in}}%
\pgfpathlineto{\pgfqpoint{5.504501in}{2.014589in}}%
\pgfpathlineto{\pgfqpoint{5.518859in}{2.013898in}}%
\pgfpathlineto{\pgfqpoint{5.533227in}{2.013232in}}%
\pgfpathlineto{\pgfqpoint{5.547603in}{2.012590in}}%
\pgfpathlineto{\pgfqpoint{5.540031in}{2.001857in}}%
\pgfpathlineto{\pgfqpoint{5.532453in}{1.991042in}}%
\pgfpathlineto{\pgfqpoint{5.524867in}{1.980147in}}%
\pgfpathlineto{\pgfqpoint{5.517275in}{1.969176in}}%
\pgfpathlineto{\pgfqpoint{5.502891in}{1.969930in}}%
\pgfpathlineto{\pgfqpoint{5.488517in}{1.970708in}}%
\pgfpathlineto{\pgfqpoint{5.474152in}{1.971510in}}%
\pgfpathlineto{\pgfqpoint{5.459795in}{1.972338in}}%
\pgfpathlineto{\pgfqpoint{5.467395in}{1.983192in}}%
\pgfpathlineto{\pgfqpoint{5.474988in}{1.993973in}}%
\pgfpathlineto{\pgfqpoint{5.482574in}{2.004678in}}%
\pgfpathlineto{\pgfqpoint{5.490153in}{2.015305in}}%
\pgfpathclose%
\pgfusepath{fill}%
\end{pgfscope}%
\begin{pgfscope}%
\pgfpathrectangle{\pgfqpoint{1.150000in}{0.150000in}}{\pgfqpoint{5.700000in}{5.700000in}}%
\pgfusepath{clip}%
\pgfsetbuttcap%
\pgfsetroundjoin%
\definecolor{currentfill}{rgb}{0.220057,0.343307,0.549413}%
\pgfsetfillcolor{currentfill}%
\pgfsetfillopacity{0.700000}%
\pgfsetlinewidth{0.000000pt}%
\definecolor{currentstroke}{rgb}{0.000000,0.000000,0.000000}%
\pgfsetstrokecolor{currentstroke}%
\pgfsetdash{}{0pt}%
\pgfpathmoveto{\pgfqpoint{2.706518in}{2.385655in}}%
\pgfpathlineto{\pgfqpoint{2.720271in}{2.376197in}}%
\pgfpathlineto{\pgfqpoint{2.734027in}{2.366777in}}%
\pgfpathlineto{\pgfqpoint{2.747785in}{2.357396in}}%
\pgfpathlineto{\pgfqpoint{2.761545in}{2.348054in}}%
\pgfpathlineto{\pgfqpoint{2.752533in}{2.358730in}}%
\pgfpathlineto{\pgfqpoint{2.743495in}{2.369924in}}%
\pgfpathlineto{\pgfqpoint{2.734430in}{2.381644in}}%
\pgfpathlineto{\pgfqpoint{2.725338in}{2.393903in}}%
\pgfpathlineto{\pgfqpoint{2.711531in}{2.403625in}}%
\pgfpathlineto{\pgfqpoint{2.697727in}{2.413385in}}%
\pgfpathlineto{\pgfqpoint{2.683926in}{2.423184in}}%
\pgfpathlineto{\pgfqpoint{2.670126in}{2.433021in}}%
\pgfpathlineto{\pgfqpoint{2.679265in}{2.420377in}}%
\pgfpathlineto{\pgfqpoint{2.688376in}{2.408275in}}%
\pgfpathlineto{\pgfqpoint{2.697461in}{2.396704in}}%
\pgfpathlineto{\pgfqpoint{2.706518in}{2.385655in}}%
\pgfpathclose%
\pgfusepath{fill}%
\end{pgfscope}%
\begin{pgfscope}%
\pgfpathrectangle{\pgfqpoint{1.150000in}{0.150000in}}{\pgfqpoint{5.700000in}{5.700000in}}%
\pgfusepath{clip}%
\pgfsetbuttcap%
\pgfsetroundjoin%
\definecolor{currentfill}{rgb}{0.283229,0.120777,0.440584}%
\pgfsetfillcolor{currentfill}%
\pgfsetfillopacity{0.700000}%
\pgfsetlinewidth{0.000000pt}%
\definecolor{currentstroke}{rgb}{0.000000,0.000000,0.000000}%
\pgfsetstrokecolor{currentstroke}%
\pgfsetdash{}{0pt}%
\pgfpathmoveto{\pgfqpoint{3.526873in}{1.897635in}}%
\pgfpathlineto{\pgfqpoint{3.540704in}{1.890909in}}%
\pgfpathlineto{\pgfqpoint{3.554540in}{1.884212in}}%
\pgfpathlineto{\pgfqpoint{3.568380in}{1.877542in}}%
\pgfpathlineto{\pgfqpoint{3.582226in}{1.870900in}}%
\pgfpathlineto{\pgfqpoint{3.573900in}{1.871971in}}%
\pgfpathlineto{\pgfqpoint{3.565561in}{1.873391in}}%
\pgfpathlineto{\pgfqpoint{3.557209in}{1.875170in}}%
\pgfpathlineto{\pgfqpoint{3.548845in}{1.877317in}}%
\pgfpathlineto{\pgfqpoint{3.534971in}{1.884285in}}%
\pgfpathlineto{\pgfqpoint{3.521101in}{1.891282in}}%
\pgfpathlineto{\pgfqpoint{3.507237in}{1.898306in}}%
\pgfpathlineto{\pgfqpoint{3.493376in}{1.905359in}}%
\pgfpathlineto{\pgfqpoint{3.501771in}{1.902880in}}%
\pgfpathlineto{\pgfqpoint{3.510151in}{1.900772in}}%
\pgfpathlineto{\pgfqpoint{3.518519in}{1.899027in}}%
\pgfpathlineto{\pgfqpoint{3.526873in}{1.897635in}}%
\pgfpathclose%
\pgfusepath{fill}%
\end{pgfscope}%
\begin{pgfscope}%
\pgfpathrectangle{\pgfqpoint{1.150000in}{0.150000in}}{\pgfqpoint{5.700000in}{5.700000in}}%
\pgfusepath{clip}%
\pgfsetbuttcap%
\pgfsetroundjoin%
\definecolor{currentfill}{rgb}{0.271305,0.019942,0.347269}%
\pgfsetfillcolor{currentfill}%
\pgfsetfillopacity{0.700000}%
\pgfsetlinewidth{0.000000pt}%
\definecolor{currentstroke}{rgb}{0.000000,0.000000,0.000000}%
\pgfsetstrokecolor{currentstroke}%
\pgfsetdash{}{0pt}%
\pgfpathmoveto{\pgfqpoint{4.587810in}{1.727179in}}%
\pgfpathlineto{\pgfqpoint{4.601870in}{1.723903in}}%
\pgfpathlineto{\pgfqpoint{4.615938in}{1.720652in}}%
\pgfpathlineto{\pgfqpoint{4.630013in}{1.717424in}}%
\pgfpathlineto{\pgfqpoint{4.644095in}{1.714222in}}%
\pgfpathlineto{\pgfqpoint{4.636247in}{1.705219in}}%
\pgfpathlineto{\pgfqpoint{4.628395in}{1.696317in}}%
\pgfpathlineto{\pgfqpoint{4.620537in}{1.687519in}}%
\pgfpathlineto{\pgfqpoint{4.612676in}{1.678831in}}%
\pgfpathlineto{\pgfqpoint{4.598583in}{1.682265in}}%
\pgfpathlineto{\pgfqpoint{4.584497in}{1.685724in}}%
\pgfpathlineto{\pgfqpoint{4.570418in}{1.689207in}}%
\pgfpathlineto{\pgfqpoint{4.556346in}{1.692714in}}%
\pgfpathlineto{\pgfqpoint{4.564219in}{1.701165in}}%
\pgfpathlineto{\pgfqpoint{4.572087in}{1.709730in}}%
\pgfpathlineto{\pgfqpoint{4.579951in}{1.718403in}}%
\pgfpathlineto{\pgfqpoint{4.587810in}{1.727179in}}%
\pgfpathclose%
\pgfusepath{fill}%
\end{pgfscope}%
\begin{pgfscope}%
\pgfpathrectangle{\pgfqpoint{1.150000in}{0.150000in}}{\pgfqpoint{5.700000in}{5.700000in}}%
\pgfusepath{clip}%
\pgfsetbuttcap%
\pgfsetroundjoin%
\definecolor{currentfill}{rgb}{0.280894,0.078907,0.402329}%
\pgfsetfillcolor{currentfill}%
\pgfsetfillopacity{0.700000}%
\pgfsetlinewidth{0.000000pt}%
\definecolor{currentstroke}{rgb}{0.000000,0.000000,0.000000}%
\pgfsetstrokecolor{currentstroke}%
\pgfsetdash{}{0pt}%
\pgfpathmoveto{\pgfqpoint{3.726137in}{1.820306in}}%
\pgfpathlineto{\pgfqpoint{3.740001in}{1.814223in}}%
\pgfpathlineto{\pgfqpoint{3.753870in}{1.808167in}}%
\pgfpathlineto{\pgfqpoint{3.767744in}{1.802138in}}%
\pgfpathlineto{\pgfqpoint{3.781624in}{1.796136in}}%
\pgfpathlineto{\pgfqpoint{3.773420in}{1.794972in}}%
\pgfpathlineto{\pgfqpoint{3.765206in}{1.794113in}}%
\pgfpathlineto{\pgfqpoint{3.756982in}{1.793567in}}%
\pgfpathlineto{\pgfqpoint{3.748747in}{1.793343in}}%
\pgfpathlineto{\pgfqpoint{3.734843in}{1.799658in}}%
\pgfpathlineto{\pgfqpoint{3.720944in}{1.805999in}}%
\pgfpathlineto{\pgfqpoint{3.707050in}{1.812366in}}%
\pgfpathlineto{\pgfqpoint{3.693161in}{1.818761in}}%
\pgfpathlineto{\pgfqpoint{3.701422in}{1.818667in}}%
\pgfpathlineto{\pgfqpoint{3.709671in}{1.818899in}}%
\pgfpathlineto{\pgfqpoint{3.717910in}{1.819448in}}%
\pgfpathlineto{\pgfqpoint{3.726137in}{1.820306in}}%
\pgfpathclose%
\pgfusepath{fill}%
\end{pgfscope}%
\begin{pgfscope}%
\pgfpathrectangle{\pgfqpoint{1.150000in}{0.150000in}}{\pgfqpoint{5.700000in}{5.700000in}}%
\pgfusepath{clip}%
\pgfsetbuttcap%
\pgfsetroundjoin%
\definecolor{currentfill}{rgb}{0.281887,0.150881,0.465405}%
\pgfsetfillcolor{currentfill}%
\pgfsetfillopacity{0.700000}%
\pgfsetlinewidth{0.000000pt}%
\definecolor{currentstroke}{rgb}{0.000000,0.000000,0.000000}%
\pgfsetstrokecolor{currentstroke}%
\pgfsetdash{}{0pt}%
\pgfpathmoveto{\pgfqpoint{5.402461in}{1.975891in}}%
\pgfpathlineto{\pgfqpoint{5.416781in}{1.974966in}}%
\pgfpathlineto{\pgfqpoint{5.431110in}{1.974065in}}%
\pgfpathlineto{\pgfqpoint{5.445448in}{1.973189in}}%
\pgfpathlineto{\pgfqpoint{5.459795in}{1.972338in}}%
\pgfpathlineto{\pgfqpoint{5.452189in}{1.961413in}}%
\pgfpathlineto{\pgfqpoint{5.444577in}{1.950420in}}%
\pgfpathlineto{\pgfqpoint{5.436958in}{1.939361in}}%
\pgfpathlineto{\pgfqpoint{5.429332in}{1.928240in}}%
\pgfpathlineto{\pgfqpoint{5.414978in}{1.929217in}}%
\pgfpathlineto{\pgfqpoint{5.400633in}{1.930219in}}%
\pgfpathlineto{\pgfqpoint{5.386297in}{1.931245in}}%
\pgfpathlineto{\pgfqpoint{5.371970in}{1.932295in}}%
\pgfpathlineto{\pgfqpoint{5.379602in}{1.943285in}}%
\pgfpathlineto{\pgfqpoint{5.387228in}{1.954217in}}%
\pgfpathlineto{\pgfqpoint{5.394848in}{1.965086in}}%
\pgfpathlineto{\pgfqpoint{5.402461in}{1.975891in}}%
\pgfpathclose%
\pgfusepath{fill}%
\end{pgfscope}%
\begin{pgfscope}%
\pgfpathrectangle{\pgfqpoint{1.150000in}{0.150000in}}{\pgfqpoint{5.700000in}{5.700000in}}%
\pgfusepath{clip}%
\pgfsetbuttcap%
\pgfsetroundjoin%
\definecolor{currentfill}{rgb}{0.269944,0.014625,0.341379}%
\pgfsetfillcolor{currentfill}%
\pgfsetfillopacity{0.700000}%
\pgfsetlinewidth{0.000000pt}%
\definecolor{currentstroke}{rgb}{0.000000,0.000000,0.000000}%
\pgfsetstrokecolor{currentstroke}%
\pgfsetdash{}{0pt}%
\pgfpathmoveto{\pgfqpoint{4.212666in}{1.715307in}}%
\pgfpathlineto{\pgfqpoint{4.226631in}{1.710803in}}%
\pgfpathlineto{\pgfqpoint{4.240602in}{1.706324in}}%
\pgfpathlineto{\pgfqpoint{4.254579in}{1.701870in}}%
\pgfpathlineto{\pgfqpoint{4.268563in}{1.697442in}}%
\pgfpathlineto{\pgfqpoint{4.260590in}{1.691360in}}%
\pgfpathlineto{\pgfqpoint{4.252610in}{1.685468in}}%
\pgfpathlineto{\pgfqpoint{4.244624in}{1.679775in}}%
\pgfpathlineto{\pgfqpoint{4.236633in}{1.674287in}}%
\pgfpathlineto{\pgfqpoint{4.222633in}{1.678987in}}%
\pgfpathlineto{\pgfqpoint{4.208639in}{1.683711in}}%
\pgfpathlineto{\pgfqpoint{4.194652in}{1.688461in}}%
\pgfpathlineto{\pgfqpoint{4.180671in}{1.693236in}}%
\pgfpathlineto{\pgfqpoint{4.188679in}{1.698448in}}%
\pgfpathlineto{\pgfqpoint{4.196681in}{1.703869in}}%
\pgfpathlineto{\pgfqpoint{4.204677in}{1.709491in}}%
\pgfpathlineto{\pgfqpoint{4.212666in}{1.715307in}}%
\pgfpathclose%
\pgfusepath{fill}%
\end{pgfscope}%
\begin{pgfscope}%
\pgfpathrectangle{\pgfqpoint{1.150000in}{0.150000in}}{\pgfqpoint{5.700000in}{5.700000in}}%
\pgfusepath{clip}%
\pgfsetbuttcap%
\pgfsetroundjoin%
\definecolor{currentfill}{rgb}{0.162142,0.474838,0.558140}%
\pgfsetfillcolor{currentfill}%
\pgfsetfillopacity{0.700000}%
\pgfsetlinewidth{0.000000pt}%
\definecolor{currentstroke}{rgb}{0.000000,0.000000,0.000000}%
\pgfsetstrokecolor{currentstroke}%
\pgfsetdash{}{0pt}%
\pgfpathmoveto{\pgfqpoint{2.284455in}{2.725700in}}%
\pgfpathlineto{\pgfqpoint{2.298212in}{2.714624in}}%
\pgfpathlineto{\pgfqpoint{2.311969in}{2.703597in}}%
\pgfpathlineto{\pgfqpoint{2.325727in}{2.692620in}}%
\pgfpathlineto{\pgfqpoint{2.339486in}{2.681693in}}%
\pgfpathlineto{\pgfqpoint{2.330012in}{2.697250in}}%
\pgfpathlineto{\pgfqpoint{2.320504in}{2.713399in}}%
\pgfpathlineto{\pgfqpoint{2.310961in}{2.730151in}}%
\pgfpathlineto{\pgfqpoint{2.301382in}{2.747519in}}%
\pgfpathlineto{\pgfqpoint{2.287568in}{2.758850in}}%
\pgfpathlineto{\pgfqpoint{2.273755in}{2.770231in}}%
\pgfpathlineto{\pgfqpoint{2.259943in}{2.781661in}}%
\pgfpathlineto{\pgfqpoint{2.246131in}{2.793142in}}%
\pgfpathlineto{\pgfqpoint{2.255766in}{2.775364in}}%
\pgfpathlineto{\pgfqpoint{2.265365in}{2.758205in}}%
\pgfpathlineto{\pgfqpoint{2.274928in}{2.741655in}}%
\pgfpathlineto{\pgfqpoint{2.284455in}{2.725700in}}%
\pgfpathclose%
\pgfusepath{fill}%
\end{pgfscope}%
\begin{pgfscope}%
\pgfpathrectangle{\pgfqpoint{1.150000in}{0.150000in}}{\pgfqpoint{5.700000in}{5.700000in}}%
\pgfusepath{clip}%
\pgfsetbuttcap%
\pgfsetroundjoin%
\definecolor{currentfill}{rgb}{0.272594,0.025563,0.353093}%
\pgfsetfillcolor{currentfill}%
\pgfsetfillopacity{0.700000}%
\pgfsetlinewidth{0.000000pt}%
\definecolor{currentstroke}{rgb}{0.000000,0.000000,0.000000}%
\pgfsetstrokecolor{currentstroke}%
\pgfsetdash{}{0pt}%
\pgfpathmoveto{\pgfqpoint{4.069040in}{1.732341in}}%
\pgfpathlineto{\pgfqpoint{4.082973in}{1.727364in}}%
\pgfpathlineto{\pgfqpoint{4.096912in}{1.722413in}}%
\pgfpathlineto{\pgfqpoint{4.110856in}{1.717487in}}%
\pgfpathlineto{\pgfqpoint{4.124807in}{1.712586in}}%
\pgfpathlineto{\pgfqpoint{4.116775in}{1.707868in}}%
\pgfpathlineto{\pgfqpoint{4.108735in}{1.703377in}}%
\pgfpathlineto{\pgfqpoint{4.100688in}{1.699118in}}%
\pgfpathlineto{\pgfqpoint{4.092634in}{1.695101in}}%
\pgfpathlineto{\pgfqpoint{4.078665in}{1.700286in}}%
\pgfpathlineto{\pgfqpoint{4.064702in}{1.705496in}}%
\pgfpathlineto{\pgfqpoint{4.050745in}{1.710732in}}%
\pgfpathlineto{\pgfqpoint{4.036793in}{1.715993in}}%
\pgfpathlineto{\pgfqpoint{4.044866in}{1.719721in}}%
\pgfpathlineto{\pgfqpoint{4.052932in}{1.723693in}}%
\pgfpathlineto{\pgfqpoint{4.060990in}{1.727902in}}%
\pgfpathlineto{\pgfqpoint{4.069040in}{1.732341in}}%
\pgfpathclose%
\pgfusepath{fill}%
\end{pgfscope}%
\begin{pgfscope}%
\pgfpathrectangle{\pgfqpoint{1.150000in}{0.150000in}}{\pgfqpoint{5.700000in}{5.700000in}}%
\pgfusepath{clip}%
\pgfsetbuttcap%
\pgfsetroundjoin%
\definecolor{currentfill}{rgb}{0.282884,0.135920,0.453427}%
\pgfsetfillcolor{currentfill}%
\pgfsetfillopacity{0.700000}%
\pgfsetlinewidth{0.000000pt}%
\definecolor{currentstroke}{rgb}{0.000000,0.000000,0.000000}%
\pgfsetstrokecolor{currentstroke}%
\pgfsetdash{}{0pt}%
\pgfpathmoveto{\pgfqpoint{5.314750in}{1.936741in}}%
\pgfpathlineto{\pgfqpoint{5.329042in}{1.935593in}}%
\pgfpathlineto{\pgfqpoint{5.343343in}{1.934469in}}%
\pgfpathlineto{\pgfqpoint{5.357652in}{1.933370in}}%
\pgfpathlineto{\pgfqpoint{5.371970in}{1.932295in}}%
\pgfpathlineto{\pgfqpoint{5.364332in}{1.921249in}}%
\pgfpathlineto{\pgfqpoint{5.356688in}{1.910149in}}%
\pgfpathlineto{\pgfqpoint{5.349037in}{1.899000in}}%
\pgfpathlineto{\pgfqpoint{5.341381in}{1.887804in}}%
\pgfpathlineto{\pgfqpoint{5.327056in}{1.889017in}}%
\pgfpathlineto{\pgfqpoint{5.312740in}{1.890256in}}%
\pgfpathlineto{\pgfqpoint{5.298433in}{1.891518in}}%
\pgfpathlineto{\pgfqpoint{5.284134in}{1.892805in}}%
\pgfpathlineto{\pgfqpoint{5.291797in}{1.903857in}}%
\pgfpathlineto{\pgfqpoint{5.299454in}{1.914866in}}%
\pgfpathlineto{\pgfqpoint{5.307105in}{1.925828in}}%
\pgfpathlineto{\pgfqpoint{5.314750in}{1.936741in}}%
\pgfpathclose%
\pgfusepath{fill}%
\end{pgfscope}%
\begin{pgfscope}%
\pgfpathrectangle{\pgfqpoint{1.150000in}{0.150000in}}{\pgfqpoint{5.700000in}{5.700000in}}%
\pgfusepath{clip}%
\pgfsetbuttcap%
\pgfsetroundjoin%
\definecolor{currentfill}{rgb}{0.276022,0.044167,0.370164}%
\pgfsetfillcolor{currentfill}%
\pgfsetfillopacity{0.700000}%
\pgfsetlinewidth{0.000000pt}%
\definecolor{currentstroke}{rgb}{0.000000,0.000000,0.000000}%
\pgfsetstrokecolor{currentstroke}%
\pgfsetdash{}{0pt}%
\pgfpathmoveto{\pgfqpoint{4.819485in}{1.767803in}}%
\pgfpathlineto{\pgfqpoint{4.833616in}{1.765243in}}%
\pgfpathlineto{\pgfqpoint{4.847754in}{1.762708in}}%
\pgfpathlineto{\pgfqpoint{4.861901in}{1.760196in}}%
\pgfpathlineto{\pgfqpoint{4.876055in}{1.757709in}}%
\pgfpathlineto{\pgfqpoint{4.868271in}{1.747506in}}%
\pgfpathlineto{\pgfqpoint{4.860482in}{1.737351in}}%
\pgfpathlineto{\pgfqpoint{4.852689in}{1.727247in}}%
\pgfpathlineto{\pgfqpoint{4.844891in}{1.717200in}}%
\pgfpathlineto{\pgfqpoint{4.830729in}{1.719892in}}%
\pgfpathlineto{\pgfqpoint{4.816574in}{1.722609in}}%
\pgfpathlineto{\pgfqpoint{4.802426in}{1.725350in}}%
\pgfpathlineto{\pgfqpoint{4.788287in}{1.728115in}}%
\pgfpathlineto{\pgfqpoint{4.796093in}{1.737952in}}%
\pgfpathlineto{\pgfqpoint{4.803895in}{1.747848in}}%
\pgfpathlineto{\pgfqpoint{4.811692in}{1.757801in}}%
\pgfpathlineto{\pgfqpoint{4.819485in}{1.767803in}}%
\pgfpathclose%
\pgfusepath{fill}%
\end{pgfscope}%
\begin{pgfscope}%
\pgfpathrectangle{\pgfqpoint{1.150000in}{0.150000in}}{\pgfqpoint{5.700000in}{5.700000in}}%
\pgfusepath{clip}%
\pgfsetbuttcap%
\pgfsetroundjoin%
\definecolor{currentfill}{rgb}{0.269944,0.014625,0.341379}%
\pgfsetfillcolor{currentfill}%
\pgfsetfillopacity{0.700000}%
\pgfsetlinewidth{0.000000pt}%
\definecolor{currentstroke}{rgb}{0.000000,0.000000,0.000000}%
\pgfsetstrokecolor{currentstroke}%
\pgfsetdash{}{0pt}%
\pgfpathmoveto{\pgfqpoint{4.356340in}{1.707109in}}%
\pgfpathlineto{\pgfqpoint{4.370342in}{1.703062in}}%
\pgfpathlineto{\pgfqpoint{4.384351in}{1.699040in}}%
\pgfpathlineto{\pgfqpoint{4.398366in}{1.695042in}}%
\pgfpathlineto{\pgfqpoint{4.412388in}{1.691069in}}%
\pgfpathlineto{\pgfqpoint{4.404465in}{1.683785in}}%
\pgfpathlineto{\pgfqpoint{4.396538in}{1.676659in}}%
\pgfpathlineto{\pgfqpoint{4.388605in}{1.669697in}}%
\pgfpathlineto{\pgfqpoint{4.380666in}{1.662905in}}%
\pgfpathlineto{\pgfqpoint{4.366631in}{1.667136in}}%
\pgfpathlineto{\pgfqpoint{4.352601in}{1.671391in}}%
\pgfpathlineto{\pgfqpoint{4.338579in}{1.675671in}}%
\pgfpathlineto{\pgfqpoint{4.324563in}{1.679975in}}%
\pgfpathlineto{\pgfqpoint{4.332516in}{1.686504in}}%
\pgfpathlineto{\pgfqpoint{4.340463in}{1.693207in}}%
\pgfpathlineto{\pgfqpoint{4.348404in}{1.700077in}}%
\pgfpathlineto{\pgfqpoint{4.356340in}{1.707109in}}%
\pgfpathclose%
\pgfusepath{fill}%
\end{pgfscope}%
\begin{pgfscope}%
\pgfpathrectangle{\pgfqpoint{1.150000in}{0.150000in}}{\pgfqpoint{5.700000in}{5.700000in}}%
\pgfusepath{clip}%
\pgfsetbuttcap%
\pgfsetroundjoin%
\definecolor{currentfill}{rgb}{0.279574,0.170599,0.479997}%
\pgfsetfillcolor{currentfill}%
\pgfsetfillopacity{0.700000}%
\pgfsetlinewidth{0.000000pt}%
\definecolor{currentstroke}{rgb}{0.000000,0.000000,0.000000}%
\pgfsetstrokecolor{currentstroke}%
\pgfsetdash{}{0pt}%
\pgfpathmoveto{\pgfqpoint{3.327393in}{1.992238in}}%
\pgfpathlineto{\pgfqpoint{3.341201in}{1.984837in}}%
\pgfpathlineto{\pgfqpoint{3.355014in}{1.977465in}}%
\pgfpathlineto{\pgfqpoint{3.368831in}{1.970124in}}%
\pgfpathlineto{\pgfqpoint{3.382652in}{1.962811in}}%
\pgfpathlineto{\pgfqpoint{3.374181in}{1.966344in}}%
\pgfpathlineto{\pgfqpoint{3.365695in}{1.970273in}}%
\pgfpathlineto{\pgfqpoint{3.357192in}{1.974609in}}%
\pgfpathlineto{\pgfqpoint{3.348674in}{1.979361in}}%
\pgfpathlineto{\pgfqpoint{3.334820in}{1.987016in}}%
\pgfpathlineto{\pgfqpoint{3.320970in}{1.994700in}}%
\pgfpathlineto{\pgfqpoint{3.307124in}{2.002414in}}%
\pgfpathlineto{\pgfqpoint{3.293282in}{2.010158in}}%
\pgfpathlineto{\pgfqpoint{3.301834in}{2.005057in}}%
\pgfpathlineto{\pgfqpoint{3.310370in}{2.000377in}}%
\pgfpathlineto{\pgfqpoint{3.318889in}{1.996107in}}%
\pgfpathlineto{\pgfqpoint{3.327393in}{1.992238in}}%
\pgfpathclose%
\pgfusepath{fill}%
\end{pgfscope}%
\begin{pgfscope}%
\pgfpathrectangle{\pgfqpoint{1.150000in}{0.150000in}}{\pgfqpoint{5.700000in}{5.700000in}}%
\pgfusepath{clip}%
\pgfsetbuttcap%
\pgfsetroundjoin%
\definecolor{currentfill}{rgb}{0.283197,0.115680,0.436115}%
\pgfsetfillcolor{currentfill}%
\pgfsetfillopacity{0.700000}%
\pgfsetlinewidth{0.000000pt}%
\definecolor{currentstroke}{rgb}{0.000000,0.000000,0.000000}%
\pgfsetstrokecolor{currentstroke}%
\pgfsetdash{}{0pt}%
\pgfpathmoveto{\pgfqpoint{5.227026in}{1.898197in}}%
\pgfpathlineto{\pgfqpoint{5.241290in}{1.896812in}}%
\pgfpathlineto{\pgfqpoint{5.255563in}{1.895452in}}%
\pgfpathlineto{\pgfqpoint{5.269844in}{1.894116in}}%
\pgfpathlineto{\pgfqpoint{5.284134in}{1.892805in}}%
\pgfpathlineto{\pgfqpoint{5.276466in}{1.881713in}}%
\pgfpathlineto{\pgfqpoint{5.268792in}{1.870584in}}%
\pgfpathlineto{\pgfqpoint{5.261112in}{1.859422in}}%
\pgfpathlineto{\pgfqpoint{5.253427in}{1.848231in}}%
\pgfpathlineto{\pgfqpoint{5.239130in}{1.849695in}}%
\pgfpathlineto{\pgfqpoint{5.224843in}{1.851183in}}%
\pgfpathlineto{\pgfqpoint{5.210563in}{1.852696in}}%
\pgfpathlineto{\pgfqpoint{5.196292in}{1.854232in}}%
\pgfpathlineto{\pgfqpoint{5.203984in}{1.865266in}}%
\pgfpathlineto{\pgfqpoint{5.211670in}{1.876274in}}%
\pgfpathlineto{\pgfqpoint{5.219351in}{1.887252in}}%
\pgfpathlineto{\pgfqpoint{5.227026in}{1.898197in}}%
\pgfpathclose%
\pgfusepath{fill}%
\end{pgfscope}%
\begin{pgfscope}%
\pgfpathrectangle{\pgfqpoint{1.150000in}{0.150000in}}{\pgfqpoint{5.700000in}{5.700000in}}%
\pgfusepath{clip}%
\pgfsetbuttcap%
\pgfsetroundjoin%
\definecolor{currentfill}{rgb}{0.276022,0.044167,0.370164}%
\pgfsetfillcolor{currentfill}%
\pgfsetfillopacity{0.700000}%
\pgfsetlinewidth{0.000000pt}%
\definecolor{currentstroke}{rgb}{0.000000,0.000000,0.000000}%
\pgfsetstrokecolor{currentstroke}%
\pgfsetdash{}{0pt}%
\pgfpathmoveto{\pgfqpoint{3.925388in}{1.759006in}}%
\pgfpathlineto{\pgfqpoint{3.939294in}{1.753539in}}%
\pgfpathlineto{\pgfqpoint{3.953205in}{1.748098in}}%
\pgfpathlineto{\pgfqpoint{3.967122in}{1.742683in}}%
\pgfpathlineto{\pgfqpoint{3.981045in}{1.737293in}}%
\pgfpathlineto{\pgfqpoint{3.972944in}{1.734110in}}%
\pgfpathlineto{\pgfqpoint{3.964835in}{1.731189in}}%
\pgfpathlineto{\pgfqpoint{3.956718in}{1.728539in}}%
\pgfpathlineto{\pgfqpoint{3.948591in}{1.726168in}}%
\pgfpathlineto{\pgfqpoint{3.934648in}{1.731855in}}%
\pgfpathlineto{\pgfqpoint{3.920709in}{1.737568in}}%
\pgfpathlineto{\pgfqpoint{3.906777in}{1.743308in}}%
\pgfpathlineto{\pgfqpoint{3.892849in}{1.749073in}}%
\pgfpathlineto{\pgfqpoint{3.900997in}{1.751141in}}%
\pgfpathlineto{\pgfqpoint{3.909136in}{1.753491in}}%
\pgfpathlineto{\pgfqpoint{3.917266in}{1.756115in}}%
\pgfpathlineto{\pgfqpoint{3.925388in}{1.759006in}}%
\pgfpathclose%
\pgfusepath{fill}%
\end{pgfscope}%
\begin{pgfscope}%
\pgfpathrectangle{\pgfqpoint{1.150000in}{0.150000in}}{\pgfqpoint{5.700000in}{5.700000in}}%
\pgfusepath{clip}%
\pgfsetbuttcap%
\pgfsetroundjoin%
\definecolor{currentfill}{rgb}{0.262138,0.242286,0.520837}%
\pgfsetfillcolor{currentfill}%
\pgfsetfillopacity{0.700000}%
\pgfsetlinewidth{0.000000pt}%
\definecolor{currentstroke}{rgb}{0.000000,0.000000,0.000000}%
\pgfsetstrokecolor{currentstroke}%
\pgfsetdash{}{0pt}%
\pgfpathmoveto{\pgfqpoint{3.072324in}{2.138248in}}%
\pgfpathlineto{\pgfqpoint{3.086107in}{2.130004in}}%
\pgfpathlineto{\pgfqpoint{3.099893in}{2.121793in}}%
\pgfpathlineto{\pgfqpoint{3.113683in}{2.113615in}}%
\pgfpathlineto{\pgfqpoint{3.127476in}{2.105468in}}%
\pgfpathlineto{\pgfqpoint{3.118798in}{2.112056in}}%
\pgfpathlineto{\pgfqpoint{3.110100in}{2.119094in}}%
\pgfpathlineto{\pgfqpoint{3.101382in}{2.126594in}}%
\pgfpathlineto{\pgfqpoint{3.092644in}{2.134565in}}%
\pgfpathlineto{\pgfqpoint{3.078813in}{2.143070in}}%
\pgfpathlineto{\pgfqpoint{3.064985in}{2.151608in}}%
\pgfpathlineto{\pgfqpoint{3.051160in}{2.160179in}}%
\pgfpathlineto{\pgfqpoint{3.037338in}{2.168782in}}%
\pgfpathlineto{\pgfqpoint{3.046116in}{2.160446in}}%
\pgfpathlineto{\pgfqpoint{3.054873in}{2.152585in}}%
\pgfpathlineto{\pgfqpoint{3.063609in}{2.145189in}}%
\pgfpathlineto{\pgfqpoint{3.072324in}{2.138248in}}%
\pgfpathclose%
\pgfusepath{fill}%
\end{pgfscope}%
\begin{pgfscope}%
\pgfpathrectangle{\pgfqpoint{1.150000in}{0.150000in}}{\pgfqpoint{5.700000in}{5.700000in}}%
\pgfusepath{clip}%
\pgfsetbuttcap%
\pgfsetroundjoin%
\definecolor{currentfill}{rgb}{0.223925,0.334994,0.548053}%
\pgfsetfillcolor{currentfill}%
\pgfsetfillopacity{0.700000}%
\pgfsetlinewidth{0.000000pt}%
\definecolor{currentstroke}{rgb}{0.000000,0.000000,0.000000}%
\pgfsetstrokecolor{currentstroke}%
\pgfsetdash{}{0pt}%
\pgfpathmoveto{\pgfqpoint{2.761545in}{2.348054in}}%
\pgfpathlineto{\pgfqpoint{2.775308in}{2.338749in}}%
\pgfpathlineto{\pgfqpoint{2.789074in}{2.329481in}}%
\pgfpathlineto{\pgfqpoint{2.802842in}{2.320251in}}%
\pgfpathlineto{\pgfqpoint{2.816613in}{2.311058in}}%
\pgfpathlineto{\pgfqpoint{2.807645in}{2.321362in}}%
\pgfpathlineto{\pgfqpoint{2.798651in}{2.332179in}}%
\pgfpathlineto{\pgfqpoint{2.789632in}{2.343519in}}%
\pgfpathlineto{\pgfqpoint{2.780587in}{2.355393in}}%
\pgfpathlineto{\pgfqpoint{2.766771in}{2.364965in}}%
\pgfpathlineto{\pgfqpoint{2.752958in}{2.374573in}}%
\pgfpathlineto{\pgfqpoint{2.739146in}{2.384219in}}%
\pgfpathlineto{\pgfqpoint{2.725338in}{2.393903in}}%
\pgfpathlineto{\pgfqpoint{2.734430in}{2.381644in}}%
\pgfpathlineto{\pgfqpoint{2.743495in}{2.369924in}}%
\pgfpathlineto{\pgfqpoint{2.752533in}{2.358730in}}%
\pgfpathlineto{\pgfqpoint{2.761545in}{2.348054in}}%
\pgfpathclose%
\pgfusepath{fill}%
\end{pgfscope}%
\begin{pgfscope}%
\pgfpathrectangle{\pgfqpoint{1.150000in}{0.150000in}}{\pgfqpoint{5.700000in}{5.700000in}}%
\pgfusepath{clip}%
\pgfsetbuttcap%
\pgfsetroundjoin%
\definecolor{currentfill}{rgb}{0.282656,0.100196,0.422160}%
\pgfsetfillcolor{currentfill}%
\pgfsetfillopacity{0.700000}%
\pgfsetlinewidth{0.000000pt}%
\definecolor{currentstroke}{rgb}{0.000000,0.000000,0.000000}%
\pgfsetstrokecolor{currentstroke}%
\pgfsetdash{}{0pt}%
\pgfpathmoveto{\pgfqpoint{5.139294in}{1.860623in}}%
\pgfpathlineto{\pgfqpoint{5.153531in}{1.858989in}}%
\pgfpathlineto{\pgfqpoint{5.167776in}{1.857379in}}%
\pgfpathlineto{\pgfqpoint{5.182030in}{1.855794in}}%
\pgfpathlineto{\pgfqpoint{5.196292in}{1.854232in}}%
\pgfpathlineto{\pgfqpoint{5.188595in}{1.843176in}}%
\pgfpathlineto{\pgfqpoint{5.180893in}{1.832101in}}%
\pgfpathlineto{\pgfqpoint{5.173186in}{1.821011in}}%
\pgfpathlineto{\pgfqpoint{5.165473in}{1.809910in}}%
\pgfpathlineto{\pgfqpoint{5.151204in}{1.811637in}}%
\pgfpathlineto{\pgfqpoint{5.136944in}{1.813389in}}%
\pgfpathlineto{\pgfqpoint{5.122691in}{1.815164in}}%
\pgfpathlineto{\pgfqpoint{5.108447in}{1.816964in}}%
\pgfpathlineto{\pgfqpoint{5.116167in}{1.827894in}}%
\pgfpathlineto{\pgfqpoint{5.123881in}{1.838817in}}%
\pgfpathlineto{\pgfqpoint{5.131590in}{1.849728in}}%
\pgfpathlineto{\pgfqpoint{5.139294in}{1.860623in}}%
\pgfpathclose%
\pgfusepath{fill}%
\end{pgfscope}%
\begin{pgfscope}%
\pgfpathrectangle{\pgfqpoint{1.150000in}{0.150000in}}{\pgfqpoint{5.700000in}{5.700000in}}%
\pgfusepath{clip}%
\pgfsetbuttcap%
\pgfsetroundjoin%
\definecolor{currentfill}{rgb}{0.168126,0.459988,0.558082}%
\pgfsetfillcolor{currentfill}%
\pgfsetfillopacity{0.700000}%
\pgfsetlinewidth{0.000000pt}%
\definecolor{currentstroke}{rgb}{0.000000,0.000000,0.000000}%
\pgfsetstrokecolor{currentstroke}%
\pgfsetdash{}{0pt}%
\pgfpathmoveto{\pgfqpoint{2.339486in}{2.681693in}}%
\pgfpathlineto{\pgfqpoint{2.353246in}{2.670814in}}%
\pgfpathlineto{\pgfqpoint{2.367007in}{2.659983in}}%
\pgfpathlineto{\pgfqpoint{2.380769in}{2.649200in}}%
\pgfpathlineto{\pgfqpoint{2.394532in}{2.638464in}}%
\pgfpathlineto{\pgfqpoint{2.385112in}{2.653625in}}%
\pgfpathlineto{\pgfqpoint{2.375658in}{2.669374in}}%
\pgfpathlineto{\pgfqpoint{2.366170in}{2.685722in}}%
\pgfpathlineto{\pgfqpoint{2.356647in}{2.702680in}}%
\pgfpathlineto{\pgfqpoint{2.342829in}{2.713818in}}%
\pgfpathlineto{\pgfqpoint{2.329013in}{2.725004in}}%
\pgfpathlineto{\pgfqpoint{2.315197in}{2.736237in}}%
\pgfpathlineto{\pgfqpoint{2.301382in}{2.747519in}}%
\pgfpathlineto{\pgfqpoint{2.310961in}{2.730151in}}%
\pgfpathlineto{\pgfqpoint{2.320504in}{2.713399in}}%
\pgfpathlineto{\pgfqpoint{2.330012in}{2.697250in}}%
\pgfpathlineto{\pgfqpoint{2.339486in}{2.681693in}}%
\pgfpathclose%
\pgfusepath{fill}%
\end{pgfscope}%
\begin{pgfscope}%
\pgfpathrectangle{\pgfqpoint{1.150000in}{0.150000in}}{\pgfqpoint{5.700000in}{5.700000in}}%
\pgfusepath{clip}%
\pgfsetbuttcap%
\pgfsetroundjoin%
\definecolor{currentfill}{rgb}{0.269944,0.014625,0.341379}%
\pgfsetfillcolor{currentfill}%
\pgfsetfillopacity{0.700000}%
\pgfsetlinewidth{0.000000pt}%
\definecolor{currentstroke}{rgb}{0.000000,0.000000,0.000000}%
\pgfsetstrokecolor{currentstroke}%
\pgfsetdash{}{0pt}%
\pgfpathmoveto{\pgfqpoint{4.500130in}{1.706989in}}%
\pgfpathlineto{\pgfqpoint{4.514174in}{1.703383in}}%
\pgfpathlineto{\pgfqpoint{4.528224in}{1.699802in}}%
\pgfpathlineto{\pgfqpoint{4.542282in}{1.696246in}}%
\pgfpathlineto{\pgfqpoint{4.556346in}{1.692714in}}%
\pgfpathlineto{\pgfqpoint{4.548469in}{1.684384in}}%
\pgfpathlineto{\pgfqpoint{4.540586in}{1.676179in}}%
\pgfpathlineto{\pgfqpoint{4.532699in}{1.668106in}}%
\pgfpathlineto{\pgfqpoint{4.524807in}{1.660172in}}%
\pgfpathlineto{\pgfqpoint{4.510731in}{1.663948in}}%
\pgfpathlineto{\pgfqpoint{4.496661in}{1.667749in}}%
\pgfpathlineto{\pgfqpoint{4.482599in}{1.671575in}}%
\pgfpathlineto{\pgfqpoint{4.468543in}{1.675424in}}%
\pgfpathlineto{\pgfqpoint{4.476447in}{1.683109in}}%
\pgfpathlineto{\pgfqpoint{4.484347in}{1.690935in}}%
\pgfpathlineto{\pgfqpoint{4.492241in}{1.698897in}}%
\pgfpathlineto{\pgfqpoint{4.500130in}{1.706989in}}%
\pgfpathclose%
\pgfusepath{fill}%
\end{pgfscope}%
\begin{pgfscope}%
\pgfpathrectangle{\pgfqpoint{1.150000in}{0.150000in}}{\pgfqpoint{5.700000in}{5.700000in}}%
\pgfusepath{clip}%
\pgfsetbuttcap%
\pgfsetroundjoin%
\definecolor{currentfill}{rgb}{0.273809,0.031497,0.358853}%
\pgfsetfillcolor{currentfill}%
\pgfsetfillopacity{0.700000}%
\pgfsetlinewidth{0.000000pt}%
\definecolor{currentstroke}{rgb}{0.000000,0.000000,0.000000}%
\pgfsetstrokecolor{currentstroke}%
\pgfsetdash{}{0pt}%
\pgfpathmoveto{\pgfqpoint{4.731804in}{1.739419in}}%
\pgfpathlineto{\pgfqpoint{4.745914in}{1.736557in}}%
\pgfpathlineto{\pgfqpoint{4.760030in}{1.733718in}}%
\pgfpathlineto{\pgfqpoint{4.774155in}{1.730905in}}%
\pgfpathlineto{\pgfqpoint{4.788287in}{1.728115in}}%
\pgfpathlineto{\pgfqpoint{4.780476in}{1.718344in}}%
\pgfpathlineto{\pgfqpoint{4.772661in}{1.708644in}}%
\pgfpathlineto{\pgfqpoint{4.764841in}{1.699020in}}%
\pgfpathlineto{\pgfqpoint{4.757018in}{1.689477in}}%
\pgfpathlineto{\pgfqpoint{4.742876in}{1.692485in}}%
\pgfpathlineto{\pgfqpoint{4.728743in}{1.695517in}}%
\pgfpathlineto{\pgfqpoint{4.714616in}{1.698574in}}%
\pgfpathlineto{\pgfqpoint{4.700497in}{1.701655in}}%
\pgfpathlineto{\pgfqpoint{4.708331in}{1.710974in}}%
\pgfpathlineto{\pgfqpoint{4.716160in}{1.720378in}}%
\pgfpathlineto{\pgfqpoint{4.723984in}{1.729862in}}%
\pgfpathlineto{\pgfqpoint{4.731804in}{1.739419in}}%
\pgfpathclose%
\pgfusepath{fill}%
\end{pgfscope}%
\begin{pgfscope}%
\pgfpathrectangle{\pgfqpoint{1.150000in}{0.150000in}}{\pgfqpoint{5.700000in}{5.700000in}}%
\pgfusepath{clip}%
\pgfsetbuttcap%
\pgfsetroundjoin%
\definecolor{currentfill}{rgb}{0.281446,0.084320,0.407414}%
\pgfsetfillcolor{currentfill}%
\pgfsetfillopacity{0.700000}%
\pgfsetlinewidth{0.000000pt}%
\definecolor{currentstroke}{rgb}{0.000000,0.000000,0.000000}%
\pgfsetstrokecolor{currentstroke}%
\pgfsetdash{}{0pt}%
\pgfpathmoveto{\pgfqpoint{5.051555in}{1.824407in}}%
\pgfpathlineto{\pgfqpoint{5.065766in}{1.822510in}}%
\pgfpathlineto{\pgfqpoint{5.079985in}{1.820637in}}%
\pgfpathlineto{\pgfqpoint{5.094212in}{1.818789in}}%
\pgfpathlineto{\pgfqpoint{5.108447in}{1.816964in}}%
\pgfpathlineto{\pgfqpoint{5.100723in}{1.806031in}}%
\pgfpathlineto{\pgfqpoint{5.092994in}{1.795098in}}%
\pgfpathlineto{\pgfqpoint{5.085260in}{1.784170in}}%
\pgfpathlineto{\pgfqpoint{5.077522in}{1.773251in}}%
\pgfpathlineto{\pgfqpoint{5.063279in}{1.775255in}}%
\pgfpathlineto{\pgfqpoint{5.049045in}{1.777283in}}%
\pgfpathlineto{\pgfqpoint{5.034819in}{1.779335in}}%
\pgfpathlineto{\pgfqpoint{5.020600in}{1.781411in}}%
\pgfpathlineto{\pgfqpoint{5.028346in}{1.792145in}}%
\pgfpathlineto{\pgfqpoint{5.036087in}{1.802893in}}%
\pgfpathlineto{\pgfqpoint{5.043823in}{1.813648in}}%
\pgfpathlineto{\pgfqpoint{5.051555in}{1.824407in}}%
\pgfpathclose%
\pgfusepath{fill}%
\end{pgfscope}%
\begin{pgfscope}%
\pgfpathrectangle{\pgfqpoint{1.150000in}{0.150000in}}{\pgfqpoint{5.700000in}{5.700000in}}%
\pgfusepath{clip}%
\pgfsetbuttcap%
\pgfsetroundjoin%
\definecolor{currentfill}{rgb}{0.283091,0.110553,0.431554}%
\pgfsetfillcolor{currentfill}%
\pgfsetfillopacity{0.700000}%
\pgfsetlinewidth{0.000000pt}%
\definecolor{currentstroke}{rgb}{0.000000,0.000000,0.000000}%
\pgfsetstrokecolor{currentstroke}%
\pgfsetdash{}{0pt}%
\pgfpathmoveto{\pgfqpoint{3.582226in}{1.870900in}}%
\pgfpathlineto{\pgfqpoint{3.596076in}{1.864287in}}%
\pgfpathlineto{\pgfqpoint{3.609931in}{1.857701in}}%
\pgfpathlineto{\pgfqpoint{3.623790in}{1.851142in}}%
\pgfpathlineto{\pgfqpoint{3.637655in}{1.844611in}}%
\pgfpathlineto{\pgfqpoint{3.629356in}{1.845360in}}%
\pgfpathlineto{\pgfqpoint{3.621045in}{1.846456in}}%
\pgfpathlineto{\pgfqpoint{3.612722in}{1.847906in}}%
\pgfpathlineto{\pgfqpoint{3.604386in}{1.849721in}}%
\pgfpathlineto{\pgfqpoint{3.590494in}{1.856578in}}%
\pgfpathlineto{\pgfqpoint{3.576606in}{1.863463in}}%
\pgfpathlineto{\pgfqpoint{3.562723in}{1.870376in}}%
\pgfpathlineto{\pgfqpoint{3.548845in}{1.877317in}}%
\pgfpathlineto{\pgfqpoint{3.557209in}{1.875170in}}%
\pgfpathlineto{\pgfqpoint{3.565561in}{1.873391in}}%
\pgfpathlineto{\pgfqpoint{3.573900in}{1.871971in}}%
\pgfpathlineto{\pgfqpoint{3.582226in}{1.870900in}}%
\pgfpathclose%
\pgfusepath{fill}%
\end{pgfscope}%
\begin{pgfscope}%
\pgfpathrectangle{\pgfqpoint{1.150000in}{0.150000in}}{\pgfqpoint{5.700000in}{5.700000in}}%
\pgfusepath{clip}%
\pgfsetbuttcap%
\pgfsetroundjoin%
\definecolor{currentfill}{rgb}{0.280267,0.073417,0.397163}%
\pgfsetfillcolor{currentfill}%
\pgfsetfillopacity{0.700000}%
\pgfsetlinewidth{0.000000pt}%
\definecolor{currentstroke}{rgb}{0.000000,0.000000,0.000000}%
\pgfsetstrokecolor{currentstroke}%
\pgfsetdash{}{0pt}%
\pgfpathmoveto{\pgfqpoint{3.781624in}{1.796136in}}%
\pgfpathlineto{\pgfqpoint{3.795508in}{1.790161in}}%
\pgfpathlineto{\pgfqpoint{3.809398in}{1.784212in}}%
\pgfpathlineto{\pgfqpoint{3.823293in}{1.778290in}}%
\pgfpathlineto{\pgfqpoint{3.837194in}{1.772394in}}%
\pgfpathlineto{\pgfqpoint{3.829014in}{1.770923in}}%
\pgfpathlineto{\pgfqpoint{3.820824in}{1.769753in}}%
\pgfpathlineto{\pgfqpoint{3.812624in}{1.768893in}}%
\pgfpathlineto{\pgfqpoint{3.804414in}{1.768352in}}%
\pgfpathlineto{\pgfqpoint{3.790490in}{1.774560in}}%
\pgfpathlineto{\pgfqpoint{3.776570in}{1.780795in}}%
\pgfpathlineto{\pgfqpoint{3.762656in}{1.787056in}}%
\pgfpathlineto{\pgfqpoint{3.748747in}{1.793343in}}%
\pgfpathlineto{\pgfqpoint{3.756982in}{1.793567in}}%
\pgfpathlineto{\pgfqpoint{3.765206in}{1.794113in}}%
\pgfpathlineto{\pgfqpoint{3.773420in}{1.794972in}}%
\pgfpathlineto{\pgfqpoint{3.781624in}{1.796136in}}%
\pgfpathclose%
\pgfusepath{fill}%
\end{pgfscope}%
\begin{pgfscope}%
\pgfpathrectangle{\pgfqpoint{1.150000in}{0.150000in}}{\pgfqpoint{5.700000in}{5.700000in}}%
\pgfusepath{clip}%
\pgfsetbuttcap%
\pgfsetroundjoin%
\definecolor{currentfill}{rgb}{0.172719,0.448791,0.557885}%
\pgfsetfillcolor{currentfill}%
\pgfsetfillopacity{0.700000}%
\pgfsetlinewidth{0.000000pt}%
\definecolor{currentstroke}{rgb}{0.000000,0.000000,0.000000}%
\pgfsetstrokecolor{currentstroke}%
\pgfsetdash{}{0pt}%
\pgfpathmoveto{\pgfqpoint{2.394532in}{2.638464in}}%
\pgfpathlineto{\pgfqpoint{2.408297in}{2.627775in}}%
\pgfpathlineto{\pgfqpoint{2.422063in}{2.617132in}}%
\pgfpathlineto{\pgfqpoint{2.435830in}{2.606535in}}%
\pgfpathlineto{\pgfqpoint{2.449599in}{2.595984in}}%
\pgfpathlineto{\pgfqpoint{2.440230in}{2.610750in}}%
\pgfpathlineto{\pgfqpoint{2.430830in}{2.626100in}}%
\pgfpathlineto{\pgfqpoint{2.421395in}{2.642044in}}%
\pgfpathlineto{\pgfqpoint{2.411928in}{2.658594in}}%
\pgfpathlineto{\pgfqpoint{2.398106in}{2.669547in}}%
\pgfpathlineto{\pgfqpoint{2.384285in}{2.680545in}}%
\pgfpathlineto{\pgfqpoint{2.370465in}{2.691589in}}%
\pgfpathlineto{\pgfqpoint{2.356647in}{2.702680in}}%
\pgfpathlineto{\pgfqpoint{2.366170in}{2.685722in}}%
\pgfpathlineto{\pgfqpoint{2.375658in}{2.669374in}}%
\pgfpathlineto{\pgfqpoint{2.385112in}{2.653625in}}%
\pgfpathlineto{\pgfqpoint{2.394532in}{2.638464in}}%
\pgfpathclose%
\pgfusepath{fill}%
\end{pgfscope}%
\begin{pgfscope}%
\pgfpathrectangle{\pgfqpoint{1.150000in}{0.150000in}}{\pgfqpoint{5.700000in}{5.700000in}}%
\pgfusepath{clip}%
\pgfsetbuttcap%
\pgfsetroundjoin%
\definecolor{currentfill}{rgb}{0.278826,0.175490,0.483397}%
\pgfsetfillcolor{currentfill}%
\pgfsetfillopacity{0.700000}%
\pgfsetlinewidth{0.000000pt}%
\definecolor{currentstroke}{rgb}{0.000000,0.000000,0.000000}%
\pgfsetstrokecolor{currentstroke}%
\pgfsetdash{}{0pt}%
\pgfpathmoveto{\pgfqpoint{5.547603in}{2.012590in}}%
\pgfpathlineto{\pgfqpoint{5.561988in}{2.011973in}}%
\pgfpathlineto{\pgfqpoint{5.576383in}{2.011380in}}%
\pgfpathlineto{\pgfqpoint{5.590787in}{2.010812in}}%
\pgfpathlineto{\pgfqpoint{5.583221in}{1.999999in}}%
\pgfpathlineto{\pgfqpoint{5.575648in}{1.989101in}}%
\pgfpathlineto{\pgfqpoint{5.568067in}{1.978122in}}%
\pgfpathlineto{\pgfqpoint{5.560480in}{1.967063in}}%
\pgfpathlineto{\pgfqpoint{5.546069in}{1.967743in}}%
\pgfpathlineto{\pgfqpoint{5.531667in}{1.968447in}}%
\pgfpathlineto{\pgfqpoint{5.517275in}{1.969176in}}%
\pgfpathlineto{\pgfqpoint{5.524867in}{1.980147in}}%
\pgfpathlineto{\pgfqpoint{5.532453in}{1.991042in}}%
\pgfpathlineto{\pgfqpoint{5.540031in}{2.001857in}}%
\pgfpathlineto{\pgfqpoint{5.547603in}{2.012590in}}%
\pgfpathclose%
\pgfusepath{fill}%
\end{pgfscope}%
\begin{pgfscope}%
\pgfpathrectangle{\pgfqpoint{1.150000in}{0.150000in}}{\pgfqpoint{5.700000in}{5.700000in}}%
\pgfusepath{clip}%
\pgfsetbuttcap%
\pgfsetroundjoin%
\definecolor{currentfill}{rgb}{0.279566,0.067836,0.391917}%
\pgfsetfillcolor{currentfill}%
\pgfsetfillopacity{0.700000}%
\pgfsetlinewidth{0.000000pt}%
\definecolor{currentstroke}{rgb}{0.000000,0.000000,0.000000}%
\pgfsetstrokecolor{currentstroke}%
\pgfsetdash{}{0pt}%
\pgfpathmoveto{\pgfqpoint{4.963809in}{1.789958in}}%
\pgfpathlineto{\pgfqpoint{4.977995in}{1.787785in}}%
\pgfpathlineto{\pgfqpoint{4.992189in}{1.785636in}}%
\pgfpathlineto{\pgfqpoint{5.006391in}{1.783511in}}%
\pgfpathlineto{\pgfqpoint{5.020600in}{1.781411in}}%
\pgfpathlineto{\pgfqpoint{5.012850in}{1.770693in}}%
\pgfpathlineto{\pgfqpoint{5.005095in}{1.759997in}}%
\pgfpathlineto{\pgfqpoint{4.997335in}{1.749327in}}%
\pgfpathlineto{\pgfqpoint{4.989571in}{1.738687in}}%
\pgfpathlineto{\pgfqpoint{4.975354in}{1.740980in}}%
\pgfpathlineto{\pgfqpoint{4.961145in}{1.743297in}}%
\pgfpathlineto{\pgfqpoint{4.946943in}{1.745639in}}%
\pgfpathlineto{\pgfqpoint{4.932750in}{1.748004in}}%
\pgfpathlineto{\pgfqpoint{4.940521in}{1.758446in}}%
\pgfpathlineto{\pgfqpoint{4.948289in}{1.768923in}}%
\pgfpathlineto{\pgfqpoint{4.956051in}{1.779428in}}%
\pgfpathlineto{\pgfqpoint{4.963809in}{1.789958in}}%
\pgfpathclose%
\pgfusepath{fill}%
\end{pgfscope}%
\begin{pgfscope}%
\pgfpathrectangle{\pgfqpoint{1.150000in}{0.150000in}}{\pgfqpoint{5.700000in}{5.700000in}}%
\pgfusepath{clip}%
\pgfsetbuttcap%
\pgfsetroundjoin%
\definecolor{currentfill}{rgb}{0.280868,0.160771,0.472899}%
\pgfsetfillcolor{currentfill}%
\pgfsetfillopacity{0.700000}%
\pgfsetlinewidth{0.000000pt}%
\definecolor{currentstroke}{rgb}{0.000000,0.000000,0.000000}%
\pgfsetstrokecolor{currentstroke}%
\pgfsetdash{}{0pt}%
\pgfpathmoveto{\pgfqpoint{3.382652in}{1.962811in}}%
\pgfpathlineto{\pgfqpoint{3.396478in}{1.955528in}}%
\pgfpathlineto{\pgfqpoint{3.410307in}{1.948275in}}%
\pgfpathlineto{\pgfqpoint{3.424141in}{1.941050in}}%
\pgfpathlineto{\pgfqpoint{3.437979in}{1.933854in}}%
\pgfpathlineto{\pgfqpoint{3.429540in}{1.937050in}}%
\pgfpathlineto{\pgfqpoint{3.421086in}{1.940640in}}%
\pgfpathlineto{\pgfqpoint{3.412616in}{1.944631in}}%
\pgfpathlineto{\pgfqpoint{3.404131in}{1.949035in}}%
\pgfpathlineto{\pgfqpoint{3.390260in}{1.956573in}}%
\pgfpathlineto{\pgfqpoint{3.376394in}{1.964140in}}%
\pgfpathlineto{\pgfqpoint{3.362532in}{1.971736in}}%
\pgfpathlineto{\pgfqpoint{3.348674in}{1.979361in}}%
\pgfpathlineto{\pgfqpoint{3.357192in}{1.974609in}}%
\pgfpathlineto{\pgfqpoint{3.365695in}{1.970273in}}%
\pgfpathlineto{\pgfqpoint{3.374181in}{1.966344in}}%
\pgfpathlineto{\pgfqpoint{3.382652in}{1.962811in}}%
\pgfpathclose%
\pgfusepath{fill}%
\end{pgfscope}%
\begin{pgfscope}%
\pgfpathrectangle{\pgfqpoint{1.150000in}{0.150000in}}{\pgfqpoint{5.700000in}{5.700000in}}%
\pgfusepath{clip}%
\pgfsetbuttcap%
\pgfsetroundjoin%
\definecolor{currentfill}{rgb}{0.272594,0.025563,0.353093}%
\pgfsetfillcolor{currentfill}%
\pgfsetfillopacity{0.700000}%
\pgfsetlinewidth{0.000000pt}%
\definecolor{currentstroke}{rgb}{0.000000,0.000000,0.000000}%
\pgfsetstrokecolor{currentstroke}%
\pgfsetdash{}{0pt}%
\pgfpathmoveto{\pgfqpoint{4.124807in}{1.712586in}}%
\pgfpathlineto{\pgfqpoint{4.138764in}{1.707711in}}%
\pgfpathlineto{\pgfqpoint{4.152727in}{1.702860in}}%
\pgfpathlineto{\pgfqpoint{4.166696in}{1.698036in}}%
\pgfpathlineto{\pgfqpoint{4.180671in}{1.693236in}}%
\pgfpathlineto{\pgfqpoint{4.172656in}{1.688239in}}%
\pgfpathlineto{\pgfqpoint{4.164634in}{1.683464in}}%
\pgfpathlineto{\pgfqpoint{4.156605in}{1.678920in}}%
\pgfpathlineto{\pgfqpoint{4.148570in}{1.674613in}}%
\pgfpathlineto{\pgfqpoint{4.134577in}{1.679697in}}%
\pgfpathlineto{\pgfqpoint{4.120590in}{1.684806in}}%
\pgfpathlineto{\pgfqpoint{4.106609in}{1.689941in}}%
\pgfpathlineto{\pgfqpoint{4.092634in}{1.695101in}}%
\pgfpathlineto{\pgfqpoint{4.100688in}{1.699118in}}%
\pgfpathlineto{\pgfqpoint{4.108735in}{1.703377in}}%
\pgfpathlineto{\pgfqpoint{4.116775in}{1.707868in}}%
\pgfpathlineto{\pgfqpoint{4.124807in}{1.712586in}}%
\pgfpathclose%
\pgfusepath{fill}%
\end{pgfscope}%
\begin{pgfscope}%
\pgfpathrectangle{\pgfqpoint{1.150000in}{0.150000in}}{\pgfqpoint{5.700000in}{5.700000in}}%
\pgfusepath{clip}%
\pgfsetbuttcap%
\pgfsetroundjoin%
\definecolor{currentfill}{rgb}{0.269944,0.014625,0.341379}%
\pgfsetfillcolor{currentfill}%
\pgfsetfillopacity{0.700000}%
\pgfsetlinewidth{0.000000pt}%
\definecolor{currentstroke}{rgb}{0.000000,0.000000,0.000000}%
\pgfsetstrokecolor{currentstroke}%
\pgfsetdash{}{0pt}%
\pgfpathmoveto{\pgfqpoint{4.268563in}{1.697442in}}%
\pgfpathlineto{\pgfqpoint{4.282554in}{1.693038in}}%
\pgfpathlineto{\pgfqpoint{4.296550in}{1.688659in}}%
\pgfpathlineto{\pgfqpoint{4.310553in}{1.684305in}}%
\pgfpathlineto{\pgfqpoint{4.324563in}{1.679975in}}%
\pgfpathlineto{\pgfqpoint{4.316604in}{1.673627in}}%
\pgfpathlineto{\pgfqpoint{4.308640in}{1.667467in}}%
\pgfpathlineto{\pgfqpoint{4.300670in}{1.661501in}}%
\pgfpathlineto{\pgfqpoint{4.292694in}{1.655737in}}%
\pgfpathlineto{\pgfqpoint{4.278670in}{1.660337in}}%
\pgfpathlineto{\pgfqpoint{4.264651in}{1.664962in}}%
\pgfpathlineto{\pgfqpoint{4.250639in}{1.669612in}}%
\pgfpathlineto{\pgfqpoint{4.236633in}{1.674287in}}%
\pgfpathlineto{\pgfqpoint{4.244624in}{1.679775in}}%
\pgfpathlineto{\pgfqpoint{4.252610in}{1.685468in}}%
\pgfpathlineto{\pgfqpoint{4.260590in}{1.691360in}}%
\pgfpathlineto{\pgfqpoint{4.268563in}{1.697442in}}%
\pgfpathclose%
\pgfusepath{fill}%
\end{pgfscope}%
\begin{pgfscope}%
\pgfpathrectangle{\pgfqpoint{1.150000in}{0.150000in}}{\pgfqpoint{5.700000in}{5.700000in}}%
\pgfusepath{clip}%
\pgfsetbuttcap%
\pgfsetroundjoin%
\definecolor{currentfill}{rgb}{0.266580,0.228262,0.514349}%
\pgfsetfillcolor{currentfill}%
\pgfsetfillopacity{0.700000}%
\pgfsetlinewidth{0.000000pt}%
\definecolor{currentstroke}{rgb}{0.000000,0.000000,0.000000}%
\pgfsetstrokecolor{currentstroke}%
\pgfsetdash{}{0pt}%
\pgfpathmoveto{\pgfqpoint{3.127476in}{2.105468in}}%
\pgfpathlineto{\pgfqpoint{3.141273in}{2.097354in}}%
\pgfpathlineto{\pgfqpoint{3.155073in}{2.089271in}}%
\pgfpathlineto{\pgfqpoint{3.168877in}{2.081220in}}%
\pgfpathlineto{\pgfqpoint{3.182685in}{2.073201in}}%
\pgfpathlineto{\pgfqpoint{3.174043in}{2.079435in}}%
\pgfpathlineto{\pgfqpoint{3.165383in}{2.086117in}}%
\pgfpathlineto{\pgfqpoint{3.156704in}{2.093256in}}%
\pgfpathlineto{\pgfqpoint{3.148004in}{2.100861in}}%
\pgfpathlineto{\pgfqpoint{3.134159in}{2.109240in}}%
\pgfpathlineto{\pgfqpoint{3.120317in}{2.117649in}}%
\pgfpathlineto{\pgfqpoint{3.106479in}{2.126091in}}%
\pgfpathlineto{\pgfqpoint{3.092644in}{2.134565in}}%
\pgfpathlineto{\pgfqpoint{3.101382in}{2.126594in}}%
\pgfpathlineto{\pgfqpoint{3.110100in}{2.119094in}}%
\pgfpathlineto{\pgfqpoint{3.118798in}{2.112056in}}%
\pgfpathlineto{\pgfqpoint{3.127476in}{2.105468in}}%
\pgfpathclose%
\pgfusepath{fill}%
\end{pgfscope}%
\begin{pgfscope}%
\pgfpathrectangle{\pgfqpoint{1.150000in}{0.150000in}}{\pgfqpoint{5.700000in}{5.700000in}}%
\pgfusepath{clip}%
\pgfsetbuttcap%
\pgfsetroundjoin%
\definecolor{currentfill}{rgb}{0.272594,0.025563,0.353093}%
\pgfsetfillcolor{currentfill}%
\pgfsetfillopacity{0.700000}%
\pgfsetlinewidth{0.000000pt}%
\definecolor{currentstroke}{rgb}{0.000000,0.000000,0.000000}%
\pgfsetstrokecolor{currentstroke}%
\pgfsetdash{}{0pt}%
\pgfpathmoveto{\pgfqpoint{4.644095in}{1.714222in}}%
\pgfpathlineto{\pgfqpoint{4.658185in}{1.711043in}}%
\pgfpathlineto{\pgfqpoint{4.672282in}{1.707890in}}%
\pgfpathlineto{\pgfqpoint{4.686386in}{1.704760in}}%
\pgfpathlineto{\pgfqpoint{4.700497in}{1.701655in}}%
\pgfpathlineto{\pgfqpoint{4.692659in}{1.692426in}}%
\pgfpathlineto{\pgfqpoint{4.684817in}{1.683293in}}%
\pgfpathlineto{\pgfqpoint{4.676970in}{1.674262in}}%
\pgfpathlineto{\pgfqpoint{4.669119in}{1.665338in}}%
\pgfpathlineto{\pgfqpoint{4.654997in}{1.668675in}}%
\pgfpathlineto{\pgfqpoint{4.640883in}{1.672036in}}%
\pgfpathlineto{\pgfqpoint{4.626776in}{1.675422in}}%
\pgfpathlineto{\pgfqpoint{4.612676in}{1.678831in}}%
\pgfpathlineto{\pgfqpoint{4.620537in}{1.687519in}}%
\pgfpathlineto{\pgfqpoint{4.628395in}{1.696317in}}%
\pgfpathlineto{\pgfqpoint{4.636247in}{1.705219in}}%
\pgfpathlineto{\pgfqpoint{4.644095in}{1.714222in}}%
\pgfpathclose%
\pgfusepath{fill}%
\end{pgfscope}%
\begin{pgfscope}%
\pgfpathrectangle{\pgfqpoint{1.150000in}{0.150000in}}{\pgfqpoint{5.700000in}{5.700000in}}%
\pgfusepath{clip}%
\pgfsetbuttcap%
\pgfsetroundjoin%
\definecolor{currentfill}{rgb}{0.229739,0.322361,0.545706}%
\pgfsetfillcolor{currentfill}%
\pgfsetfillopacity{0.700000}%
\pgfsetlinewidth{0.000000pt}%
\definecolor{currentstroke}{rgb}{0.000000,0.000000,0.000000}%
\pgfsetstrokecolor{currentstroke}%
\pgfsetdash{}{0pt}%
\pgfpathmoveto{\pgfqpoint{2.816613in}{2.311058in}}%
\pgfpathlineto{\pgfqpoint{2.830387in}{2.301901in}}%
\pgfpathlineto{\pgfqpoint{2.844163in}{2.292781in}}%
\pgfpathlineto{\pgfqpoint{2.857942in}{2.283697in}}%
\pgfpathlineto{\pgfqpoint{2.871724in}{2.274649in}}%
\pgfpathlineto{\pgfqpoint{2.862799in}{2.284582in}}%
\pgfpathlineto{\pgfqpoint{2.853850in}{2.295023in}}%
\pgfpathlineto{\pgfqpoint{2.844876in}{2.305983in}}%
\pgfpathlineto{\pgfqpoint{2.835876in}{2.317473in}}%
\pgfpathlineto{\pgfqpoint{2.822050in}{2.326898in}}%
\pgfpathlineto{\pgfqpoint{2.808226in}{2.336360in}}%
\pgfpathlineto{\pgfqpoint{2.794405in}{2.345858in}}%
\pgfpathlineto{\pgfqpoint{2.780587in}{2.355393in}}%
\pgfpathlineto{\pgfqpoint{2.789632in}{2.343519in}}%
\pgfpathlineto{\pgfqpoint{2.798651in}{2.332179in}}%
\pgfpathlineto{\pgfqpoint{2.807645in}{2.321362in}}%
\pgfpathlineto{\pgfqpoint{2.816613in}{2.311058in}}%
\pgfpathclose%
\pgfusepath{fill}%
\end{pgfscope}%
\begin{pgfscope}%
\pgfpathrectangle{\pgfqpoint{1.150000in}{0.150000in}}{\pgfqpoint{5.700000in}{5.700000in}}%
\pgfusepath{clip}%
\pgfsetbuttcap%
\pgfsetroundjoin%
\definecolor{currentfill}{rgb}{0.276022,0.044167,0.370164}%
\pgfsetfillcolor{currentfill}%
\pgfsetfillopacity{0.700000}%
\pgfsetlinewidth{0.000000pt}%
\definecolor{currentstroke}{rgb}{0.000000,0.000000,0.000000}%
\pgfsetstrokecolor{currentstroke}%
\pgfsetdash{}{0pt}%
\pgfpathmoveto{\pgfqpoint{3.981045in}{1.737293in}}%
\pgfpathlineto{\pgfqpoint{3.994974in}{1.731930in}}%
\pgfpathlineto{\pgfqpoint{4.008908in}{1.726592in}}%
\pgfpathlineto{\pgfqpoint{4.022848in}{1.721280in}}%
\pgfpathlineto{\pgfqpoint{4.036793in}{1.715993in}}%
\pgfpathlineto{\pgfqpoint{4.028713in}{1.712517in}}%
\pgfpathlineto{\pgfqpoint{4.020624in}{1.709300in}}%
\pgfpathlineto{\pgfqpoint{4.012527in}{1.706350in}}%
\pgfpathlineto{\pgfqpoint{4.004422in}{1.703675in}}%
\pgfpathlineto{\pgfqpoint{3.990456in}{1.709260in}}%
\pgfpathlineto{\pgfqpoint{3.976496in}{1.714870in}}%
\pgfpathlineto{\pgfqpoint{3.962541in}{1.720506in}}%
\pgfpathlineto{\pgfqpoint{3.948591in}{1.726168in}}%
\pgfpathlineto{\pgfqpoint{3.956718in}{1.728539in}}%
\pgfpathlineto{\pgfqpoint{3.964835in}{1.731189in}}%
\pgfpathlineto{\pgfqpoint{3.972944in}{1.734110in}}%
\pgfpathlineto{\pgfqpoint{3.981045in}{1.737293in}}%
\pgfpathclose%
\pgfusepath{fill}%
\end{pgfscope}%
\begin{pgfscope}%
\pgfpathrectangle{\pgfqpoint{1.150000in}{0.150000in}}{\pgfqpoint{5.700000in}{5.700000in}}%
\pgfusepath{clip}%
\pgfsetbuttcap%
\pgfsetroundjoin%
\definecolor{currentfill}{rgb}{0.269944,0.014625,0.341379}%
\pgfsetfillcolor{currentfill}%
\pgfsetfillopacity{0.700000}%
\pgfsetlinewidth{0.000000pt}%
\definecolor{currentstroke}{rgb}{0.000000,0.000000,0.000000}%
\pgfsetstrokecolor{currentstroke}%
\pgfsetdash{}{0pt}%
\pgfpathmoveto{\pgfqpoint{4.412388in}{1.691069in}}%
\pgfpathlineto{\pgfqpoint{4.426416in}{1.687121in}}%
\pgfpathlineto{\pgfqpoint{4.440452in}{1.683198in}}%
\pgfpathlineto{\pgfqpoint{4.454494in}{1.679299in}}%
\pgfpathlineto{\pgfqpoint{4.468543in}{1.675424in}}%
\pgfpathlineto{\pgfqpoint{4.460634in}{1.667888in}}%
\pgfpathlineto{\pgfqpoint{4.452719in}{1.660505in}}%
\pgfpathlineto{\pgfqpoint{4.444800in}{1.653283in}}%
\pgfpathlineto{\pgfqpoint{4.436875in}{1.646229in}}%
\pgfpathlineto{\pgfqpoint{4.422813in}{1.650361in}}%
\pgfpathlineto{\pgfqpoint{4.408757in}{1.654518in}}%
\pgfpathlineto{\pgfqpoint{4.394709in}{1.658699in}}%
\pgfpathlineto{\pgfqpoint{4.380666in}{1.662905in}}%
\pgfpathlineto{\pgfqpoint{4.388605in}{1.669697in}}%
\pgfpathlineto{\pgfqpoint{4.396538in}{1.676659in}}%
\pgfpathlineto{\pgfqpoint{4.404465in}{1.683785in}}%
\pgfpathlineto{\pgfqpoint{4.412388in}{1.691069in}}%
\pgfpathclose%
\pgfusepath{fill}%
\end{pgfscope}%
\begin{pgfscope}%
\pgfpathrectangle{\pgfqpoint{1.150000in}{0.150000in}}{\pgfqpoint{5.700000in}{5.700000in}}%
\pgfusepath{clip}%
\pgfsetbuttcap%
\pgfsetroundjoin%
\definecolor{currentfill}{rgb}{0.280868,0.160771,0.472899}%
\pgfsetfillcolor{currentfill}%
\pgfsetfillopacity{0.700000}%
\pgfsetlinewidth{0.000000pt}%
\definecolor{currentstroke}{rgb}{0.000000,0.000000,0.000000}%
\pgfsetstrokecolor{currentstroke}%
\pgfsetdash{}{0pt}%
\pgfpathmoveto{\pgfqpoint{5.459795in}{1.972338in}}%
\pgfpathlineto{\pgfqpoint{5.474152in}{1.971510in}}%
\pgfpathlineto{\pgfqpoint{5.488517in}{1.970708in}}%
\pgfpathlineto{\pgfqpoint{5.502891in}{1.969930in}}%
\pgfpathlineto{\pgfqpoint{5.517275in}{1.969176in}}%
\pgfpathlineto{\pgfqpoint{5.509675in}{1.958131in}}%
\pgfpathlineto{\pgfqpoint{5.502070in}{1.947014in}}%
\pgfpathlineto{\pgfqpoint{5.494457in}{1.935829in}}%
\pgfpathlineto{\pgfqpoint{5.486838in}{1.924577in}}%
\pgfpathlineto{\pgfqpoint{5.472448in}{1.925456in}}%
\pgfpathlineto{\pgfqpoint{5.458067in}{1.926360in}}%
\pgfpathlineto{\pgfqpoint{5.443695in}{1.927288in}}%
\pgfpathlineto{\pgfqpoint{5.429332in}{1.928240in}}%
\pgfpathlineto{\pgfqpoint{5.436958in}{1.939361in}}%
\pgfpathlineto{\pgfqpoint{5.444577in}{1.950420in}}%
\pgfpathlineto{\pgfqpoint{5.452189in}{1.961413in}}%
\pgfpathlineto{\pgfqpoint{5.459795in}{1.972338in}}%
\pgfpathclose%
\pgfusepath{fill}%
\end{pgfscope}%
\begin{pgfscope}%
\pgfpathrectangle{\pgfqpoint{1.150000in}{0.150000in}}{\pgfqpoint{5.700000in}{5.700000in}}%
\pgfusepath{clip}%
\pgfsetbuttcap%
\pgfsetroundjoin%
\definecolor{currentfill}{rgb}{0.277018,0.050344,0.375715}%
\pgfsetfillcolor{currentfill}%
\pgfsetfillopacity{0.700000}%
\pgfsetlinewidth{0.000000pt}%
\definecolor{currentstroke}{rgb}{0.000000,0.000000,0.000000}%
\pgfsetstrokecolor{currentstroke}%
\pgfsetdash{}{0pt}%
\pgfpathmoveto{\pgfqpoint{4.876055in}{1.757709in}}%
\pgfpathlineto{\pgfqpoint{4.890217in}{1.755247in}}%
\pgfpathlineto{\pgfqpoint{4.904387in}{1.752808in}}%
\pgfpathlineto{\pgfqpoint{4.918564in}{1.750394in}}%
\pgfpathlineto{\pgfqpoint{4.932750in}{1.748004in}}%
\pgfpathlineto{\pgfqpoint{4.924974in}{1.737601in}}%
\pgfpathlineto{\pgfqpoint{4.917193in}{1.727241in}}%
\pgfpathlineto{\pgfqpoint{4.909408in}{1.716930in}}%
\pgfpathlineto{\pgfqpoint{4.901619in}{1.706673in}}%
\pgfpathlineto{\pgfqpoint{4.887425in}{1.709268in}}%
\pgfpathlineto{\pgfqpoint{4.873239in}{1.711888in}}%
\pgfpathlineto{\pgfqpoint{4.859061in}{1.714532in}}%
\pgfpathlineto{\pgfqpoint{4.844891in}{1.717200in}}%
\pgfpathlineto{\pgfqpoint{4.852689in}{1.727247in}}%
\pgfpathlineto{\pgfqpoint{4.860482in}{1.737351in}}%
\pgfpathlineto{\pgfqpoint{4.868271in}{1.747506in}}%
\pgfpathlineto{\pgfqpoint{4.876055in}{1.757709in}}%
\pgfpathclose%
\pgfusepath{fill}%
\end{pgfscope}%
\begin{pgfscope}%
\pgfpathrectangle{\pgfqpoint{1.150000in}{0.150000in}}{\pgfqpoint{5.700000in}{5.700000in}}%
\pgfusepath{clip}%
\pgfsetbuttcap%
\pgfsetroundjoin%
\definecolor{currentfill}{rgb}{0.177423,0.437527,0.557565}%
\pgfsetfillcolor{currentfill}%
\pgfsetfillopacity{0.700000}%
\pgfsetlinewidth{0.000000pt}%
\definecolor{currentstroke}{rgb}{0.000000,0.000000,0.000000}%
\pgfsetstrokecolor{currentstroke}%
\pgfsetdash{}{0pt}%
\pgfpathmoveto{\pgfqpoint{2.449599in}{2.595984in}}%
\pgfpathlineto{\pgfqpoint{2.463369in}{2.585477in}}%
\pgfpathlineto{\pgfqpoint{2.477140in}{2.575015in}}%
\pgfpathlineto{\pgfqpoint{2.490913in}{2.564598in}}%
\pgfpathlineto{\pgfqpoint{2.504688in}{2.554224in}}%
\pgfpathlineto{\pgfqpoint{2.495371in}{2.568598in}}%
\pgfpathlineto{\pgfqpoint{2.486023in}{2.583549in}}%
\pgfpathlineto{\pgfqpoint{2.476642in}{2.599091in}}%
\pgfpathlineto{\pgfqpoint{2.467228in}{2.615234in}}%
\pgfpathlineto{\pgfqpoint{2.453401in}{2.626007in}}%
\pgfpathlineto{\pgfqpoint{2.439575in}{2.636825in}}%
\pgfpathlineto{\pgfqpoint{2.425751in}{2.647687in}}%
\pgfpathlineto{\pgfqpoint{2.411928in}{2.658594in}}%
\pgfpathlineto{\pgfqpoint{2.421395in}{2.642044in}}%
\pgfpathlineto{\pgfqpoint{2.430830in}{2.626100in}}%
\pgfpathlineto{\pgfqpoint{2.440230in}{2.610750in}}%
\pgfpathlineto{\pgfqpoint{2.449599in}{2.595984in}}%
\pgfpathclose%
\pgfusepath{fill}%
\end{pgfscope}%
\begin{pgfscope}%
\pgfpathrectangle{\pgfqpoint{1.150000in}{0.150000in}}{\pgfqpoint{5.700000in}{5.700000in}}%
\pgfusepath{clip}%
\pgfsetbuttcap%
\pgfsetroundjoin%
\definecolor{currentfill}{rgb}{0.282623,0.140926,0.457517}%
\pgfsetfillcolor{currentfill}%
\pgfsetfillopacity{0.700000}%
\pgfsetlinewidth{0.000000pt}%
\definecolor{currentstroke}{rgb}{0.000000,0.000000,0.000000}%
\pgfsetstrokecolor{currentstroke}%
\pgfsetdash{}{0pt}%
\pgfpathmoveto{\pgfqpoint{5.371970in}{1.932295in}}%
\pgfpathlineto{\pgfqpoint{5.386297in}{1.931245in}}%
\pgfpathlineto{\pgfqpoint{5.400633in}{1.930219in}}%
\pgfpathlineto{\pgfqpoint{5.414978in}{1.929217in}}%
\pgfpathlineto{\pgfqpoint{5.429332in}{1.928240in}}%
\pgfpathlineto{\pgfqpoint{5.421701in}{1.917060in}}%
\pgfpathlineto{\pgfqpoint{5.414063in}{1.905823in}}%
\pgfpathlineto{\pgfqpoint{5.406419in}{1.894533in}}%
\pgfpathlineto{\pgfqpoint{5.398769in}{1.883192in}}%
\pgfpathlineto{\pgfqpoint{5.384409in}{1.884309in}}%
\pgfpathlineto{\pgfqpoint{5.370057in}{1.885449in}}%
\pgfpathlineto{\pgfqpoint{5.355715in}{1.886614in}}%
\pgfpathlineto{\pgfqpoint{5.341381in}{1.887804in}}%
\pgfpathlineto{\pgfqpoint{5.349037in}{1.899000in}}%
\pgfpathlineto{\pgfqpoint{5.356688in}{1.910149in}}%
\pgfpathlineto{\pgfqpoint{5.364332in}{1.921249in}}%
\pgfpathlineto{\pgfqpoint{5.371970in}{1.932295in}}%
\pgfpathclose%
\pgfusepath{fill}%
\end{pgfscope}%
\begin{pgfscope}%
\pgfpathrectangle{\pgfqpoint{1.150000in}{0.150000in}}{\pgfqpoint{5.700000in}{5.700000in}}%
\pgfusepath{clip}%
\pgfsetbuttcap%
\pgfsetroundjoin%
\definecolor{currentfill}{rgb}{0.282910,0.105393,0.426902}%
\pgfsetfillcolor{currentfill}%
\pgfsetfillopacity{0.700000}%
\pgfsetlinewidth{0.000000pt}%
\definecolor{currentstroke}{rgb}{0.000000,0.000000,0.000000}%
\pgfsetstrokecolor{currentstroke}%
\pgfsetdash{}{0pt}%
\pgfpathmoveto{\pgfqpoint{3.637655in}{1.844611in}}%
\pgfpathlineto{\pgfqpoint{3.651524in}{1.838108in}}%
\pgfpathlineto{\pgfqpoint{3.665398in}{1.831632in}}%
\pgfpathlineto{\pgfqpoint{3.679277in}{1.825183in}}%
\pgfpathlineto{\pgfqpoint{3.693161in}{1.818761in}}%
\pgfpathlineto{\pgfqpoint{3.684889in}{1.819189in}}%
\pgfpathlineto{\pgfqpoint{3.676606in}{1.819959in}}%
\pgfpathlineto{\pgfqpoint{3.668310in}{1.821081in}}%
\pgfpathlineto{\pgfqpoint{3.660003in}{1.822564in}}%
\pgfpathlineto{\pgfqpoint{3.646091in}{1.829312in}}%
\pgfpathlineto{\pgfqpoint{3.632185in}{1.836088in}}%
\pgfpathlineto{\pgfqpoint{3.618283in}{1.842890in}}%
\pgfpathlineto{\pgfqpoint{3.604386in}{1.849721in}}%
\pgfpathlineto{\pgfqpoint{3.612722in}{1.847906in}}%
\pgfpathlineto{\pgfqpoint{3.621045in}{1.846456in}}%
\pgfpathlineto{\pgfqpoint{3.629356in}{1.845360in}}%
\pgfpathlineto{\pgfqpoint{3.637655in}{1.844611in}}%
\pgfpathclose%
\pgfusepath{fill}%
\end{pgfscope}%
\begin{pgfscope}%
\pgfpathrectangle{\pgfqpoint{1.150000in}{0.150000in}}{\pgfqpoint{5.700000in}{5.700000in}}%
\pgfusepath{clip}%
\pgfsetbuttcap%
\pgfsetroundjoin%
\definecolor{currentfill}{rgb}{0.283187,0.125848,0.444960}%
\pgfsetfillcolor{currentfill}%
\pgfsetfillopacity{0.700000}%
\pgfsetlinewidth{0.000000pt}%
\definecolor{currentstroke}{rgb}{0.000000,0.000000,0.000000}%
\pgfsetstrokecolor{currentstroke}%
\pgfsetdash{}{0pt}%
\pgfpathmoveto{\pgfqpoint{5.284134in}{1.892805in}}%
\pgfpathlineto{\pgfqpoint{5.298433in}{1.891518in}}%
\pgfpathlineto{\pgfqpoint{5.312740in}{1.890256in}}%
\pgfpathlineto{\pgfqpoint{5.327056in}{1.889017in}}%
\pgfpathlineto{\pgfqpoint{5.341381in}{1.887804in}}%
\pgfpathlineto{\pgfqpoint{5.333719in}{1.876564in}}%
\pgfpathlineto{\pgfqpoint{5.326052in}{1.865284in}}%
\pgfpathlineto{\pgfqpoint{5.318378in}{1.853968in}}%
\pgfpathlineto{\pgfqpoint{5.310700in}{1.842619in}}%
\pgfpathlineto{\pgfqpoint{5.296368in}{1.843986in}}%
\pgfpathlineto{\pgfqpoint{5.282046in}{1.845376in}}%
\pgfpathlineto{\pgfqpoint{5.267732in}{1.846792in}}%
\pgfpathlineto{\pgfqpoint{5.253427in}{1.848231in}}%
\pgfpathlineto{\pgfqpoint{5.261112in}{1.859422in}}%
\pgfpathlineto{\pgfqpoint{5.268792in}{1.870584in}}%
\pgfpathlineto{\pgfqpoint{5.276466in}{1.881713in}}%
\pgfpathlineto{\pgfqpoint{5.284134in}{1.892805in}}%
\pgfpathclose%
\pgfusepath{fill}%
\end{pgfscope}%
\begin{pgfscope}%
\pgfpathrectangle{\pgfqpoint{1.150000in}{0.150000in}}{\pgfqpoint{5.700000in}{5.700000in}}%
\pgfusepath{clip}%
\pgfsetbuttcap%
\pgfsetroundjoin%
\definecolor{currentfill}{rgb}{0.269944,0.014625,0.341379}%
\pgfsetfillcolor{currentfill}%
\pgfsetfillopacity{0.700000}%
\pgfsetlinewidth{0.000000pt}%
\definecolor{currentstroke}{rgb}{0.000000,0.000000,0.000000}%
\pgfsetstrokecolor{currentstroke}%
\pgfsetdash{}{0pt}%
\pgfpathmoveto{\pgfqpoint{4.556346in}{1.692714in}}%
\pgfpathlineto{\pgfqpoint{4.570418in}{1.689207in}}%
\pgfpathlineto{\pgfqpoint{4.584497in}{1.685724in}}%
\pgfpathlineto{\pgfqpoint{4.598583in}{1.682265in}}%
\pgfpathlineto{\pgfqpoint{4.612676in}{1.678831in}}%
\pgfpathlineto{\pgfqpoint{4.604809in}{1.670261in}}%
\pgfpathlineto{\pgfqpoint{4.596938in}{1.661813in}}%
\pgfpathlineto{\pgfqpoint{4.589063in}{1.653494in}}%
\pgfpathlineto{\pgfqpoint{4.581183in}{1.645310in}}%
\pgfpathlineto{\pgfqpoint{4.567078in}{1.648989in}}%
\pgfpathlineto{\pgfqpoint{4.552981in}{1.652692in}}%
\pgfpathlineto{\pgfqpoint{4.538891in}{1.656420in}}%
\pgfpathlineto{\pgfqpoint{4.524807in}{1.660172in}}%
\pgfpathlineto{\pgfqpoint{4.532699in}{1.668106in}}%
\pgfpathlineto{\pgfqpoint{4.540586in}{1.676179in}}%
\pgfpathlineto{\pgfqpoint{4.548469in}{1.684384in}}%
\pgfpathlineto{\pgfqpoint{4.556346in}{1.692714in}}%
\pgfpathclose%
\pgfusepath{fill}%
\end{pgfscope}%
\begin{pgfscope}%
\pgfpathrectangle{\pgfqpoint{1.150000in}{0.150000in}}{\pgfqpoint{5.700000in}{5.700000in}}%
\pgfusepath{clip}%
\pgfsetbuttcap%
\pgfsetroundjoin%
\definecolor{currentfill}{rgb}{0.235526,0.309527,0.542944}%
\pgfsetfillcolor{currentfill}%
\pgfsetfillopacity{0.700000}%
\pgfsetlinewidth{0.000000pt}%
\definecolor{currentstroke}{rgb}{0.000000,0.000000,0.000000}%
\pgfsetstrokecolor{currentstroke}%
\pgfsetdash{}{0pt}%
\pgfpathmoveto{\pgfqpoint{2.871724in}{2.274649in}}%
\pgfpathlineto{\pgfqpoint{2.885509in}{2.265636in}}%
\pgfpathlineto{\pgfqpoint{2.899296in}{2.256659in}}%
\pgfpathlineto{\pgfqpoint{2.913087in}{2.247717in}}%
\pgfpathlineto{\pgfqpoint{2.926880in}{2.238810in}}%
\pgfpathlineto{\pgfqpoint{2.917998in}{2.248372in}}%
\pgfpathlineto{\pgfqpoint{2.909092in}{2.258438in}}%
\pgfpathlineto{\pgfqpoint{2.900162in}{2.269019in}}%
\pgfpathlineto{\pgfqpoint{2.891207in}{2.280125in}}%
\pgfpathlineto{\pgfqpoint{2.877370in}{2.289409in}}%
\pgfpathlineto{\pgfqpoint{2.863536in}{2.298728in}}%
\pgfpathlineto{\pgfqpoint{2.849705in}{2.308083in}}%
\pgfpathlineto{\pgfqpoint{2.835876in}{2.317473in}}%
\pgfpathlineto{\pgfqpoint{2.844876in}{2.305983in}}%
\pgfpathlineto{\pgfqpoint{2.853850in}{2.295023in}}%
\pgfpathlineto{\pgfqpoint{2.862799in}{2.284582in}}%
\pgfpathlineto{\pgfqpoint{2.871724in}{2.274649in}}%
\pgfpathclose%
\pgfusepath{fill}%
\end{pgfscope}%
\begin{pgfscope}%
\pgfpathrectangle{\pgfqpoint{1.150000in}{0.150000in}}{\pgfqpoint{5.700000in}{5.700000in}}%
\pgfusepath{clip}%
\pgfsetbuttcap%
\pgfsetroundjoin%
\definecolor{currentfill}{rgb}{0.279566,0.067836,0.391917}%
\pgfsetfillcolor{currentfill}%
\pgfsetfillopacity{0.700000}%
\pgfsetlinewidth{0.000000pt}%
\definecolor{currentstroke}{rgb}{0.000000,0.000000,0.000000}%
\pgfsetstrokecolor{currentstroke}%
\pgfsetdash{}{0pt}%
\pgfpathmoveto{\pgfqpoint{3.837194in}{1.772394in}}%
\pgfpathlineto{\pgfqpoint{3.851100in}{1.766524in}}%
\pgfpathlineto{\pgfqpoint{3.865011in}{1.760681in}}%
\pgfpathlineto{\pgfqpoint{3.878927in}{1.754864in}}%
\pgfpathlineto{\pgfqpoint{3.892849in}{1.749073in}}%
\pgfpathlineto{\pgfqpoint{3.884692in}{1.747295in}}%
\pgfpathlineto{\pgfqpoint{3.876526in}{1.745815in}}%
\pgfpathlineto{\pgfqpoint{3.868349in}{1.744641in}}%
\pgfpathlineto{\pgfqpoint{3.860164in}{1.743783in}}%
\pgfpathlineto{\pgfqpoint{3.846218in}{1.749886in}}%
\pgfpathlineto{\pgfqpoint{3.832278in}{1.756015in}}%
\pgfpathlineto{\pgfqpoint{3.818344in}{1.762171in}}%
\pgfpathlineto{\pgfqpoint{3.804414in}{1.768352in}}%
\pgfpathlineto{\pgfqpoint{3.812624in}{1.768893in}}%
\pgfpathlineto{\pgfqpoint{3.820824in}{1.769753in}}%
\pgfpathlineto{\pgfqpoint{3.829014in}{1.770923in}}%
\pgfpathlineto{\pgfqpoint{3.837194in}{1.772394in}}%
\pgfpathclose%
\pgfusepath{fill}%
\end{pgfscope}%
\begin{pgfscope}%
\pgfpathrectangle{\pgfqpoint{1.150000in}{0.150000in}}{\pgfqpoint{5.700000in}{5.700000in}}%
\pgfusepath{clip}%
\pgfsetbuttcap%
\pgfsetroundjoin%
\definecolor{currentfill}{rgb}{0.274952,0.037752,0.364543}%
\pgfsetfillcolor{currentfill}%
\pgfsetfillopacity{0.700000}%
\pgfsetlinewidth{0.000000pt}%
\definecolor{currentstroke}{rgb}{0.000000,0.000000,0.000000}%
\pgfsetstrokecolor{currentstroke}%
\pgfsetdash{}{0pt}%
\pgfpathmoveto{\pgfqpoint{4.788287in}{1.728115in}}%
\pgfpathlineto{\pgfqpoint{4.802426in}{1.725350in}}%
\pgfpathlineto{\pgfqpoint{4.816574in}{1.722609in}}%
\pgfpathlineto{\pgfqpoint{4.830729in}{1.719892in}}%
\pgfpathlineto{\pgfqpoint{4.844891in}{1.717200in}}%
\pgfpathlineto{\pgfqpoint{4.837089in}{1.707215in}}%
\pgfpathlineto{\pgfqpoint{4.829283in}{1.697298in}}%
\pgfpathlineto{\pgfqpoint{4.821472in}{1.687454in}}%
\pgfpathlineto{\pgfqpoint{4.813658in}{1.677687in}}%
\pgfpathlineto{\pgfqpoint{4.799486in}{1.680598in}}%
\pgfpathlineto{\pgfqpoint{4.785323in}{1.683534in}}%
\pgfpathlineto{\pgfqpoint{4.771166in}{1.686493in}}%
\pgfpathlineto{\pgfqpoint{4.757018in}{1.689477in}}%
\pgfpathlineto{\pgfqpoint{4.764841in}{1.699020in}}%
\pgfpathlineto{\pgfqpoint{4.772661in}{1.708644in}}%
\pgfpathlineto{\pgfqpoint{4.780476in}{1.718344in}}%
\pgfpathlineto{\pgfqpoint{4.788287in}{1.728115in}}%
\pgfpathclose%
\pgfusepath{fill}%
\end{pgfscope}%
\begin{pgfscope}%
\pgfpathrectangle{\pgfqpoint{1.150000in}{0.150000in}}{\pgfqpoint{5.700000in}{5.700000in}}%
\pgfusepath{clip}%
\pgfsetbuttcap%
\pgfsetroundjoin%
\definecolor{currentfill}{rgb}{0.269308,0.218818,0.509577}%
\pgfsetfillcolor{currentfill}%
\pgfsetfillopacity{0.700000}%
\pgfsetlinewidth{0.000000pt}%
\definecolor{currentstroke}{rgb}{0.000000,0.000000,0.000000}%
\pgfsetstrokecolor{currentstroke}%
\pgfsetdash{}{0pt}%
\pgfpathmoveto{\pgfqpoint{3.182685in}{2.073201in}}%
\pgfpathlineto{\pgfqpoint{3.196496in}{2.065213in}}%
\pgfpathlineto{\pgfqpoint{3.210311in}{2.057255in}}%
\pgfpathlineto{\pgfqpoint{3.224130in}{2.049329in}}%
\pgfpathlineto{\pgfqpoint{3.237953in}{2.041434in}}%
\pgfpathlineto{\pgfqpoint{3.229348in}{2.047316in}}%
\pgfpathlineto{\pgfqpoint{3.220724in}{2.053641in}}%
\pgfpathlineto{\pgfqpoint{3.212082in}{2.060419in}}%
\pgfpathlineto{\pgfqpoint{3.203421in}{2.067660in}}%
\pgfpathlineto{\pgfqpoint{3.189561in}{2.075914in}}%
\pgfpathlineto{\pgfqpoint{3.175705in}{2.084199in}}%
\pgfpathlineto{\pgfqpoint{3.161853in}{2.092514in}}%
\pgfpathlineto{\pgfqpoint{3.148004in}{2.100861in}}%
\pgfpathlineto{\pgfqpoint{3.156704in}{2.093256in}}%
\pgfpathlineto{\pgfqpoint{3.165383in}{2.086117in}}%
\pgfpathlineto{\pgfqpoint{3.174043in}{2.079435in}}%
\pgfpathlineto{\pgfqpoint{3.182685in}{2.073201in}}%
\pgfpathclose%
\pgfusepath{fill}%
\end{pgfscope}%
\begin{pgfscope}%
\pgfpathrectangle{\pgfqpoint{1.150000in}{0.150000in}}{\pgfqpoint{5.700000in}{5.700000in}}%
\pgfusepath{clip}%
\pgfsetbuttcap%
\pgfsetroundjoin%
\definecolor{currentfill}{rgb}{0.282910,0.105393,0.426902}%
\pgfsetfillcolor{currentfill}%
\pgfsetfillopacity{0.700000}%
\pgfsetlinewidth{0.000000pt}%
\definecolor{currentstroke}{rgb}{0.000000,0.000000,0.000000}%
\pgfsetstrokecolor{currentstroke}%
\pgfsetdash{}{0pt}%
\pgfpathmoveto{\pgfqpoint{5.196292in}{1.854232in}}%
\pgfpathlineto{\pgfqpoint{5.210563in}{1.852696in}}%
\pgfpathlineto{\pgfqpoint{5.224843in}{1.851183in}}%
\pgfpathlineto{\pgfqpoint{5.239130in}{1.849695in}}%
\pgfpathlineto{\pgfqpoint{5.253427in}{1.848231in}}%
\pgfpathlineto{\pgfqpoint{5.245737in}{1.837014in}}%
\pgfpathlineto{\pgfqpoint{5.238041in}{1.825774in}}%
\pgfpathlineto{\pgfqpoint{5.230340in}{1.814517in}}%
\pgfpathlineto{\pgfqpoint{5.222634in}{1.803245in}}%
\pgfpathlineto{\pgfqpoint{5.208331in}{1.804875in}}%
\pgfpathlineto{\pgfqpoint{5.194037in}{1.806529in}}%
\pgfpathlineto{\pgfqpoint{5.179751in}{1.808207in}}%
\pgfpathlineto{\pgfqpoint{5.165473in}{1.809910in}}%
\pgfpathlineto{\pgfqpoint{5.173186in}{1.821011in}}%
\pgfpathlineto{\pgfqpoint{5.180893in}{1.832101in}}%
\pgfpathlineto{\pgfqpoint{5.188595in}{1.843176in}}%
\pgfpathlineto{\pgfqpoint{5.196292in}{1.854232in}}%
\pgfpathclose%
\pgfusepath{fill}%
\end{pgfscope}%
\begin{pgfscope}%
\pgfpathrectangle{\pgfqpoint{1.150000in}{0.150000in}}{\pgfqpoint{5.700000in}{5.700000in}}%
\pgfusepath{clip}%
\pgfsetbuttcap%
\pgfsetroundjoin%
\definecolor{currentfill}{rgb}{0.281412,0.155834,0.469201}%
\pgfsetfillcolor{currentfill}%
\pgfsetfillopacity{0.700000}%
\pgfsetlinewidth{0.000000pt}%
\definecolor{currentstroke}{rgb}{0.000000,0.000000,0.000000}%
\pgfsetstrokecolor{currentstroke}%
\pgfsetdash{}{0pt}%
\pgfpathmoveto{\pgfqpoint{3.437979in}{1.933854in}}%
\pgfpathlineto{\pgfqpoint{3.451822in}{1.926688in}}%
\pgfpathlineto{\pgfqpoint{3.465669in}{1.919550in}}%
\pgfpathlineto{\pgfqpoint{3.479521in}{1.912440in}}%
\pgfpathlineto{\pgfqpoint{3.493376in}{1.905359in}}%
\pgfpathlineto{\pgfqpoint{3.484968in}{1.908218in}}%
\pgfpathlineto{\pgfqpoint{3.476545in}{1.911468in}}%
\pgfpathlineto{\pgfqpoint{3.468107in}{1.915116in}}%
\pgfpathlineto{\pgfqpoint{3.459655in}{1.919172in}}%
\pgfpathlineto{\pgfqpoint{3.445767in}{1.926595in}}%
\pgfpathlineto{\pgfqpoint{3.431884in}{1.934046in}}%
\pgfpathlineto{\pgfqpoint{3.418006in}{1.941527in}}%
\pgfpathlineto{\pgfqpoint{3.404131in}{1.949035in}}%
\pgfpathlineto{\pgfqpoint{3.412616in}{1.944631in}}%
\pgfpathlineto{\pgfqpoint{3.421086in}{1.940640in}}%
\pgfpathlineto{\pgfqpoint{3.429540in}{1.937050in}}%
\pgfpathlineto{\pgfqpoint{3.437979in}{1.933854in}}%
\pgfpathclose%
\pgfusepath{fill}%
\end{pgfscope}%
\begin{pgfscope}%
\pgfpathrectangle{\pgfqpoint{1.150000in}{0.150000in}}{\pgfqpoint{5.700000in}{5.700000in}}%
\pgfusepath{clip}%
\pgfsetbuttcap%
\pgfsetroundjoin%
\definecolor{currentfill}{rgb}{0.183898,0.422383,0.556944}%
\pgfsetfillcolor{currentfill}%
\pgfsetfillopacity{0.700000}%
\pgfsetlinewidth{0.000000pt}%
\definecolor{currentstroke}{rgb}{0.000000,0.000000,0.000000}%
\pgfsetstrokecolor{currentstroke}%
\pgfsetdash{}{0pt}%
\pgfpathmoveto{\pgfqpoint{2.504688in}{2.554224in}}%
\pgfpathlineto{\pgfqpoint{2.518465in}{2.543894in}}%
\pgfpathlineto{\pgfqpoint{2.532243in}{2.533607in}}%
\pgfpathlineto{\pgfqpoint{2.546022in}{2.523362in}}%
\pgfpathlineto{\pgfqpoint{2.559804in}{2.513160in}}%
\pgfpathlineto{\pgfqpoint{2.550538in}{2.527141in}}%
\pgfpathlineto{\pgfqpoint{2.541241in}{2.541695in}}%
\pgfpathlineto{\pgfqpoint{2.531912in}{2.556835in}}%
\pgfpathlineto{\pgfqpoint{2.522552in}{2.572572in}}%
\pgfpathlineto{\pgfqpoint{2.508718in}{2.583173in}}%
\pgfpathlineto{\pgfqpoint{2.494887in}{2.593817in}}%
\pgfpathlineto{\pgfqpoint{2.481057in}{2.604504in}}%
\pgfpathlineto{\pgfqpoint{2.467228in}{2.615234in}}%
\pgfpathlineto{\pgfqpoint{2.476642in}{2.599091in}}%
\pgfpathlineto{\pgfqpoint{2.486023in}{2.583549in}}%
\pgfpathlineto{\pgfqpoint{2.495371in}{2.568598in}}%
\pgfpathlineto{\pgfqpoint{2.504688in}{2.554224in}}%
\pgfpathclose%
\pgfusepath{fill}%
\end{pgfscope}%
\begin{pgfscope}%
\pgfpathrectangle{\pgfqpoint{1.150000in}{0.150000in}}{\pgfqpoint{5.700000in}{5.700000in}}%
\pgfusepath{clip}%
\pgfsetbuttcap%
\pgfsetroundjoin%
\definecolor{currentfill}{rgb}{0.281924,0.089666,0.412415}%
\pgfsetfillcolor{currentfill}%
\pgfsetfillopacity{0.700000}%
\pgfsetlinewidth{0.000000pt}%
\definecolor{currentstroke}{rgb}{0.000000,0.000000,0.000000}%
\pgfsetstrokecolor{currentstroke}%
\pgfsetdash{}{0pt}%
\pgfpathmoveto{\pgfqpoint{5.108447in}{1.816964in}}%
\pgfpathlineto{\pgfqpoint{5.122691in}{1.815164in}}%
\pgfpathlineto{\pgfqpoint{5.136944in}{1.813389in}}%
\pgfpathlineto{\pgfqpoint{5.151204in}{1.811637in}}%
\pgfpathlineto{\pgfqpoint{5.165473in}{1.809910in}}%
\pgfpathlineto{\pgfqpoint{5.157756in}{1.798803in}}%
\pgfpathlineto{\pgfqpoint{5.150034in}{1.787692in}}%
\pgfpathlineto{\pgfqpoint{5.142306in}{1.776583in}}%
\pgfpathlineto{\pgfqpoint{5.134575in}{1.765480in}}%
\pgfpathlineto{\pgfqpoint{5.120299in}{1.767386in}}%
\pgfpathlineto{\pgfqpoint{5.106032in}{1.769317in}}%
\pgfpathlineto{\pgfqpoint{5.091772in}{1.771272in}}%
\pgfpathlineto{\pgfqpoint{5.077522in}{1.773251in}}%
\pgfpathlineto{\pgfqpoint{5.085260in}{1.784170in}}%
\pgfpathlineto{\pgfqpoint{5.092994in}{1.795098in}}%
\pgfpathlineto{\pgfqpoint{5.100723in}{1.806031in}}%
\pgfpathlineto{\pgfqpoint{5.108447in}{1.816964in}}%
\pgfpathclose%
\pgfusepath{fill}%
\end{pgfscope}%
\begin{pgfscope}%
\pgfpathrectangle{\pgfqpoint{1.150000in}{0.150000in}}{\pgfqpoint{5.700000in}{5.700000in}}%
\pgfusepath{clip}%
\pgfsetbuttcap%
\pgfsetroundjoin%
\definecolor{currentfill}{rgb}{0.272594,0.025563,0.353093}%
\pgfsetfillcolor{currentfill}%
\pgfsetfillopacity{0.700000}%
\pgfsetlinewidth{0.000000pt}%
\definecolor{currentstroke}{rgb}{0.000000,0.000000,0.000000}%
\pgfsetstrokecolor{currentstroke}%
\pgfsetdash{}{0pt}%
\pgfpathmoveto{\pgfqpoint{4.180671in}{1.693236in}}%
\pgfpathlineto{\pgfqpoint{4.194652in}{1.688461in}}%
\pgfpathlineto{\pgfqpoint{4.208639in}{1.683711in}}%
\pgfpathlineto{\pgfqpoint{4.222633in}{1.678987in}}%
\pgfpathlineto{\pgfqpoint{4.236633in}{1.674287in}}%
\pgfpathlineto{\pgfqpoint{4.228634in}{1.669011in}}%
\pgfpathlineto{\pgfqpoint{4.220630in}{1.663953in}}%
\pgfpathlineto{\pgfqpoint{4.212619in}{1.659123in}}%
\pgfpathlineto{\pgfqpoint{4.204601in}{1.654526in}}%
\pgfpathlineto{\pgfqpoint{4.190584in}{1.659510in}}%
\pgfpathlineto{\pgfqpoint{4.176574in}{1.664519in}}%
\pgfpathlineto{\pgfqpoint{4.162569in}{1.669554in}}%
\pgfpathlineto{\pgfqpoint{4.148570in}{1.674613in}}%
\pgfpathlineto{\pgfqpoint{4.156605in}{1.678920in}}%
\pgfpathlineto{\pgfqpoint{4.164634in}{1.683464in}}%
\pgfpathlineto{\pgfqpoint{4.172656in}{1.688239in}}%
\pgfpathlineto{\pgfqpoint{4.180671in}{1.693236in}}%
\pgfpathclose%
\pgfusepath{fill}%
\end{pgfscope}%
\begin{pgfscope}%
\pgfpathrectangle{\pgfqpoint{1.150000in}{0.150000in}}{\pgfqpoint{5.700000in}{5.700000in}}%
\pgfusepath{clip}%
\pgfsetbuttcap%
\pgfsetroundjoin%
\definecolor{currentfill}{rgb}{0.269944,0.014625,0.341379}%
\pgfsetfillcolor{currentfill}%
\pgfsetfillopacity{0.700000}%
\pgfsetlinewidth{0.000000pt}%
\definecolor{currentstroke}{rgb}{0.000000,0.000000,0.000000}%
\pgfsetstrokecolor{currentstroke}%
\pgfsetdash{}{0pt}%
\pgfpathmoveto{\pgfqpoint{4.324563in}{1.679975in}}%
\pgfpathlineto{\pgfqpoint{4.338579in}{1.675671in}}%
\pgfpathlineto{\pgfqpoint{4.352601in}{1.671391in}}%
\pgfpathlineto{\pgfqpoint{4.366631in}{1.667136in}}%
\pgfpathlineto{\pgfqpoint{4.380666in}{1.662905in}}%
\pgfpathlineto{\pgfqpoint{4.372722in}{1.656291in}}%
\pgfpathlineto{\pgfqpoint{4.364773in}{1.649861in}}%
\pgfpathlineto{\pgfqpoint{4.356818in}{1.643622in}}%
\pgfpathlineto{\pgfqpoint{4.348858in}{1.637582in}}%
\pgfpathlineto{\pgfqpoint{4.334807in}{1.642083in}}%
\pgfpathlineto{\pgfqpoint{4.320763in}{1.646610in}}%
\pgfpathlineto{\pgfqpoint{4.306726in}{1.651161in}}%
\pgfpathlineto{\pgfqpoint{4.292694in}{1.655737in}}%
\pgfpathlineto{\pgfqpoint{4.300670in}{1.661501in}}%
\pgfpathlineto{\pgfqpoint{4.308640in}{1.667467in}}%
\pgfpathlineto{\pgfqpoint{4.316604in}{1.673627in}}%
\pgfpathlineto{\pgfqpoint{4.324563in}{1.679975in}}%
\pgfpathclose%
\pgfusepath{fill}%
\end{pgfscope}%
\begin{pgfscope}%
\pgfpathrectangle{\pgfqpoint{1.150000in}{0.150000in}}{\pgfqpoint{5.700000in}{5.700000in}}%
\pgfusepath{clip}%
\pgfsetbuttcap%
\pgfsetroundjoin%
\definecolor{currentfill}{rgb}{0.274952,0.037752,0.364543}%
\pgfsetfillcolor{currentfill}%
\pgfsetfillopacity{0.700000}%
\pgfsetlinewidth{0.000000pt}%
\definecolor{currentstroke}{rgb}{0.000000,0.000000,0.000000}%
\pgfsetstrokecolor{currentstroke}%
\pgfsetdash{}{0pt}%
\pgfpathmoveto{\pgfqpoint{4.036793in}{1.715993in}}%
\pgfpathlineto{\pgfqpoint{4.050745in}{1.710732in}}%
\pgfpathlineto{\pgfqpoint{4.064702in}{1.705496in}}%
\pgfpathlineto{\pgfqpoint{4.078665in}{1.700286in}}%
\pgfpathlineto{\pgfqpoint{4.092634in}{1.695101in}}%
\pgfpathlineto{\pgfqpoint{4.084573in}{1.691331in}}%
\pgfpathlineto{\pgfqpoint{4.076504in}{1.687818in}}%
\pgfpathlineto{\pgfqpoint{4.068428in}{1.684568in}}%
\pgfpathlineto{\pgfqpoint{4.060344in}{1.681590in}}%
\pgfpathlineto{\pgfqpoint{4.046355in}{1.687073in}}%
\pgfpathlineto{\pgfqpoint{4.032372in}{1.692582in}}%
\pgfpathlineto{\pgfqpoint{4.018394in}{1.698116in}}%
\pgfpathlineto{\pgfqpoint{4.004422in}{1.703675in}}%
\pgfpathlineto{\pgfqpoint{4.012527in}{1.706350in}}%
\pgfpathlineto{\pgfqpoint{4.020624in}{1.709300in}}%
\pgfpathlineto{\pgfqpoint{4.028713in}{1.712517in}}%
\pgfpathlineto{\pgfqpoint{4.036793in}{1.715993in}}%
\pgfpathclose%
\pgfusepath{fill}%
\end{pgfscope}%
\begin{pgfscope}%
\pgfpathrectangle{\pgfqpoint{1.150000in}{0.150000in}}{\pgfqpoint{5.700000in}{5.700000in}}%
\pgfusepath{clip}%
\pgfsetbuttcap%
\pgfsetroundjoin%
\definecolor{currentfill}{rgb}{0.272594,0.025563,0.353093}%
\pgfsetfillcolor{currentfill}%
\pgfsetfillopacity{0.700000}%
\pgfsetlinewidth{0.000000pt}%
\definecolor{currentstroke}{rgb}{0.000000,0.000000,0.000000}%
\pgfsetstrokecolor{currentstroke}%
\pgfsetdash{}{0pt}%
\pgfpathmoveto{\pgfqpoint{4.700497in}{1.701655in}}%
\pgfpathlineto{\pgfqpoint{4.714616in}{1.698574in}}%
\pgfpathlineto{\pgfqpoint{4.728743in}{1.695517in}}%
\pgfpathlineto{\pgfqpoint{4.742876in}{1.692485in}}%
\pgfpathlineto{\pgfqpoint{4.757018in}{1.689477in}}%
\pgfpathlineto{\pgfqpoint{4.749189in}{1.680021in}}%
\pgfpathlineto{\pgfqpoint{4.741357in}{1.670658in}}%
\pgfpathlineto{\pgfqpoint{4.733520in}{1.661393in}}%
\pgfpathlineto{\pgfqpoint{4.725679in}{1.652232in}}%
\pgfpathlineto{\pgfqpoint{4.711528in}{1.655473in}}%
\pgfpathlineto{\pgfqpoint{4.697384in}{1.658737in}}%
\pgfpathlineto{\pgfqpoint{4.683248in}{1.662025in}}%
\pgfpathlineto{\pgfqpoint{4.669119in}{1.665338in}}%
\pgfpathlineto{\pgfqpoint{4.676970in}{1.674262in}}%
\pgfpathlineto{\pgfqpoint{4.684817in}{1.683293in}}%
\pgfpathlineto{\pgfqpoint{4.692659in}{1.692426in}}%
\pgfpathlineto{\pgfqpoint{4.700497in}{1.701655in}}%
\pgfpathclose%
\pgfusepath{fill}%
\end{pgfscope}%
\begin{pgfscope}%
\pgfpathrectangle{\pgfqpoint{1.150000in}{0.150000in}}{\pgfqpoint{5.700000in}{5.700000in}}%
\pgfusepath{clip}%
\pgfsetbuttcap%
\pgfsetroundjoin%
\definecolor{currentfill}{rgb}{0.280267,0.073417,0.397163}%
\pgfsetfillcolor{currentfill}%
\pgfsetfillopacity{0.700000}%
\pgfsetlinewidth{0.000000pt}%
\definecolor{currentstroke}{rgb}{0.000000,0.000000,0.000000}%
\pgfsetstrokecolor{currentstroke}%
\pgfsetdash{}{0pt}%
\pgfpathmoveto{\pgfqpoint{5.020600in}{1.781411in}}%
\pgfpathlineto{\pgfqpoint{5.034819in}{1.779335in}}%
\pgfpathlineto{\pgfqpoint{5.049045in}{1.777283in}}%
\pgfpathlineto{\pgfqpoint{5.063279in}{1.775255in}}%
\pgfpathlineto{\pgfqpoint{5.077522in}{1.773251in}}%
\pgfpathlineto{\pgfqpoint{5.069778in}{1.762346in}}%
\pgfpathlineto{\pgfqpoint{5.062030in}{1.751459in}}%
\pgfpathlineto{\pgfqpoint{5.054278in}{1.740594in}}%
\pgfpathlineto{\pgfqpoint{5.046521in}{1.729757in}}%
\pgfpathlineto{\pgfqpoint{5.032272in}{1.731953in}}%
\pgfpathlineto{\pgfqpoint{5.018030in}{1.734174in}}%
\pgfpathlineto{\pgfqpoint{5.003797in}{1.736418in}}%
\pgfpathlineto{\pgfqpoint{4.989571in}{1.738687in}}%
\pgfpathlineto{\pgfqpoint{4.997335in}{1.749327in}}%
\pgfpathlineto{\pgfqpoint{5.005095in}{1.759997in}}%
\pgfpathlineto{\pgfqpoint{5.012850in}{1.770693in}}%
\pgfpathlineto{\pgfqpoint{5.020600in}{1.781411in}}%
\pgfpathclose%
\pgfusepath{fill}%
\end{pgfscope}%
\begin{pgfscope}%
\pgfpathrectangle{\pgfqpoint{1.150000in}{0.150000in}}{\pgfqpoint{5.700000in}{5.700000in}}%
\pgfusepath{clip}%
\pgfsetbuttcap%
\pgfsetroundjoin%
\definecolor{currentfill}{rgb}{0.239346,0.300855,0.540844}%
\pgfsetfillcolor{currentfill}%
\pgfsetfillopacity{0.700000}%
\pgfsetlinewidth{0.000000pt}%
\definecolor{currentstroke}{rgb}{0.000000,0.000000,0.000000}%
\pgfsetstrokecolor{currentstroke}%
\pgfsetdash{}{0pt}%
\pgfpathmoveto{\pgfqpoint{2.926880in}{2.238810in}}%
\pgfpathlineto{\pgfqpoint{2.940677in}{2.229938in}}%
\pgfpathlineto{\pgfqpoint{2.954476in}{2.221100in}}%
\pgfpathlineto{\pgfqpoint{2.968279in}{2.212296in}}%
\pgfpathlineto{\pgfqpoint{2.982084in}{2.203526in}}%
\pgfpathlineto{\pgfqpoint{2.973244in}{2.212717in}}%
\pgfpathlineto{\pgfqpoint{2.964380in}{2.222409in}}%
\pgfpathlineto{\pgfqpoint{2.955494in}{2.232611in}}%
\pgfpathlineto{\pgfqpoint{2.946583in}{2.243334in}}%
\pgfpathlineto{\pgfqpoint{2.932735in}{2.252481in}}%
\pgfpathlineto{\pgfqpoint{2.918890in}{2.261661in}}%
\pgfpathlineto{\pgfqpoint{2.905047in}{2.270876in}}%
\pgfpathlineto{\pgfqpoint{2.891207in}{2.280125in}}%
\pgfpathlineto{\pgfqpoint{2.900162in}{2.269019in}}%
\pgfpathlineto{\pgfqpoint{2.909092in}{2.258438in}}%
\pgfpathlineto{\pgfqpoint{2.917998in}{2.248372in}}%
\pgfpathlineto{\pgfqpoint{2.926880in}{2.238810in}}%
\pgfpathclose%
\pgfusepath{fill}%
\end{pgfscope}%
\begin{pgfscope}%
\pgfpathrectangle{\pgfqpoint{1.150000in}{0.150000in}}{\pgfqpoint{5.700000in}{5.700000in}}%
\pgfusepath{clip}%
\pgfsetbuttcap%
\pgfsetroundjoin%
\definecolor{currentfill}{rgb}{0.269944,0.014625,0.341379}%
\pgfsetfillcolor{currentfill}%
\pgfsetfillopacity{0.700000}%
\pgfsetlinewidth{0.000000pt}%
\definecolor{currentstroke}{rgb}{0.000000,0.000000,0.000000}%
\pgfsetstrokecolor{currentstroke}%
\pgfsetdash{}{0pt}%
\pgfpathmoveto{\pgfqpoint{4.468543in}{1.675424in}}%
\pgfpathlineto{\pgfqpoint{4.482599in}{1.671575in}}%
\pgfpathlineto{\pgfqpoint{4.496661in}{1.667749in}}%
\pgfpathlineto{\pgfqpoint{4.510731in}{1.663948in}}%
\pgfpathlineto{\pgfqpoint{4.524807in}{1.660172in}}%
\pgfpathlineto{\pgfqpoint{4.516911in}{1.652382in}}%
\pgfpathlineto{\pgfqpoint{4.509009in}{1.644743in}}%
\pgfpathlineto{\pgfqpoint{4.501102in}{1.637262in}}%
\pgfpathlineto{\pgfqpoint{4.493191in}{1.629944in}}%
\pgfpathlineto{\pgfqpoint{4.479102in}{1.633979in}}%
\pgfpathlineto{\pgfqpoint{4.465020in}{1.638038in}}%
\pgfpathlineto{\pgfqpoint{4.450944in}{1.642121in}}%
\pgfpathlineto{\pgfqpoint{4.436875in}{1.646229in}}%
\pgfpathlineto{\pgfqpoint{4.444800in}{1.653283in}}%
\pgfpathlineto{\pgfqpoint{4.452719in}{1.660505in}}%
\pgfpathlineto{\pgfqpoint{4.460634in}{1.667888in}}%
\pgfpathlineto{\pgfqpoint{4.468543in}{1.675424in}}%
\pgfpathclose%
\pgfusepath{fill}%
\end{pgfscope}%
\begin{pgfscope}%
\pgfpathrectangle{\pgfqpoint{1.150000in}{0.150000in}}{\pgfqpoint{5.700000in}{5.700000in}}%
\pgfusepath{clip}%
\pgfsetbuttcap%
\pgfsetroundjoin%
\definecolor{currentfill}{rgb}{0.188923,0.410910,0.556326}%
\pgfsetfillcolor{currentfill}%
\pgfsetfillopacity{0.700000}%
\pgfsetlinewidth{0.000000pt}%
\definecolor{currentstroke}{rgb}{0.000000,0.000000,0.000000}%
\pgfsetstrokecolor{currentstroke}%
\pgfsetdash{}{0pt}%
\pgfpathmoveto{\pgfqpoint{2.559804in}{2.513160in}}%
\pgfpathlineto{\pgfqpoint{2.573587in}{2.503000in}}%
\pgfpathlineto{\pgfqpoint{2.587373in}{2.492881in}}%
\pgfpathlineto{\pgfqpoint{2.601160in}{2.482803in}}%
\pgfpathlineto{\pgfqpoint{2.614949in}{2.472766in}}%
\pgfpathlineto{\pgfqpoint{2.605732in}{2.486356in}}%
\pgfpathlineto{\pgfqpoint{2.596486in}{2.500514in}}%
\pgfpathlineto{\pgfqpoint{2.587209in}{2.515253in}}%
\pgfpathlineto{\pgfqpoint{2.577901in}{2.530586in}}%
\pgfpathlineto{\pgfqpoint{2.564061in}{2.541020in}}%
\pgfpathlineto{\pgfqpoint{2.550223in}{2.551496in}}%
\pgfpathlineto{\pgfqpoint{2.536386in}{2.562013in}}%
\pgfpathlineto{\pgfqpoint{2.522552in}{2.572572in}}%
\pgfpathlineto{\pgfqpoint{2.531912in}{2.556835in}}%
\pgfpathlineto{\pgfqpoint{2.541241in}{2.541695in}}%
\pgfpathlineto{\pgfqpoint{2.550538in}{2.527141in}}%
\pgfpathlineto{\pgfqpoint{2.559804in}{2.513160in}}%
\pgfpathclose%
\pgfusepath{fill}%
\end{pgfscope}%
\begin{pgfscope}%
\pgfpathrectangle{\pgfqpoint{1.150000in}{0.150000in}}{\pgfqpoint{5.700000in}{5.700000in}}%
\pgfusepath{clip}%
\pgfsetbuttcap%
\pgfsetroundjoin%
\definecolor{currentfill}{rgb}{0.282656,0.100196,0.422160}%
\pgfsetfillcolor{currentfill}%
\pgfsetfillopacity{0.700000}%
\pgfsetlinewidth{0.000000pt}%
\definecolor{currentstroke}{rgb}{0.000000,0.000000,0.000000}%
\pgfsetstrokecolor{currentstroke}%
\pgfsetdash{}{0pt}%
\pgfpathmoveto{\pgfqpoint{3.693161in}{1.818761in}}%
\pgfpathlineto{\pgfqpoint{3.707050in}{1.812366in}}%
\pgfpathlineto{\pgfqpoint{3.720944in}{1.805999in}}%
\pgfpathlineto{\pgfqpoint{3.734843in}{1.799658in}}%
\pgfpathlineto{\pgfqpoint{3.748747in}{1.793343in}}%
\pgfpathlineto{\pgfqpoint{3.740501in}{1.793450in}}%
\pgfpathlineto{\pgfqpoint{3.732244in}{1.793896in}}%
\pgfpathlineto{\pgfqpoint{3.723976in}{1.794689in}}%
\pgfpathlineto{\pgfqpoint{3.715696in}{1.795839in}}%
\pgfpathlineto{\pgfqpoint{3.701765in}{1.802480in}}%
\pgfpathlineto{\pgfqpoint{3.687839in}{1.809148in}}%
\pgfpathlineto{\pgfqpoint{3.673919in}{1.815842in}}%
\pgfpathlineto{\pgfqpoint{3.660003in}{1.822564in}}%
\pgfpathlineto{\pgfqpoint{3.668310in}{1.821081in}}%
\pgfpathlineto{\pgfqpoint{3.676606in}{1.819959in}}%
\pgfpathlineto{\pgfqpoint{3.684889in}{1.819189in}}%
\pgfpathlineto{\pgfqpoint{3.693161in}{1.818761in}}%
\pgfpathclose%
\pgfusepath{fill}%
\end{pgfscope}%
\begin{pgfscope}%
\pgfpathrectangle{\pgfqpoint{1.150000in}{0.150000in}}{\pgfqpoint{5.700000in}{5.700000in}}%
\pgfusepath{clip}%
\pgfsetbuttcap%
\pgfsetroundjoin%
\definecolor{currentfill}{rgb}{0.280255,0.165693,0.476498}%
\pgfsetfillcolor{currentfill}%
\pgfsetfillopacity{0.700000}%
\pgfsetlinewidth{0.000000pt}%
\definecolor{currentstroke}{rgb}{0.000000,0.000000,0.000000}%
\pgfsetstrokecolor{currentstroke}%
\pgfsetdash{}{0pt}%
\pgfpathmoveto{\pgfqpoint{5.517275in}{1.969176in}}%
\pgfpathlineto{\pgfqpoint{5.531667in}{1.968447in}}%
\pgfpathlineto{\pgfqpoint{5.546069in}{1.967743in}}%
\pgfpathlineto{\pgfqpoint{5.560480in}{1.967063in}}%
\pgfpathlineto{\pgfqpoint{5.552886in}{1.955927in}}%
\pgfpathlineto{\pgfqpoint{5.545285in}{1.944717in}}%
\pgfpathlineto{\pgfqpoint{5.537677in}{1.933436in}}%
\pgfpathlineto{\pgfqpoint{5.530063in}{1.922086in}}%
\pgfpathlineto{\pgfqpoint{5.515645in}{1.922892in}}%
\pgfpathlineto{\pgfqpoint{5.501237in}{1.923722in}}%
\pgfpathlineto{\pgfqpoint{5.486838in}{1.924577in}}%
\pgfpathlineto{\pgfqpoint{5.494457in}{1.935829in}}%
\pgfpathlineto{\pgfqpoint{5.502070in}{1.947014in}}%
\pgfpathlineto{\pgfqpoint{5.509675in}{1.958131in}}%
\pgfpathlineto{\pgfqpoint{5.517275in}{1.969176in}}%
\pgfpathclose%
\pgfusepath{fill}%
\end{pgfscope}%
\begin{pgfscope}%
\pgfpathrectangle{\pgfqpoint{1.150000in}{0.150000in}}{\pgfqpoint{5.700000in}{5.700000in}}%
\pgfusepath{clip}%
\pgfsetbuttcap%
\pgfsetroundjoin%
\definecolor{currentfill}{rgb}{0.271828,0.209303,0.504434}%
\pgfsetfillcolor{currentfill}%
\pgfsetfillopacity{0.700000}%
\pgfsetlinewidth{0.000000pt}%
\definecolor{currentstroke}{rgb}{0.000000,0.000000,0.000000}%
\pgfsetstrokecolor{currentstroke}%
\pgfsetdash{}{0pt}%
\pgfpathmoveto{\pgfqpoint{3.237953in}{2.041434in}}%
\pgfpathlineto{\pgfqpoint{3.251779in}{2.033569in}}%
\pgfpathlineto{\pgfqpoint{3.265609in}{2.025735in}}%
\pgfpathlineto{\pgfqpoint{3.279444in}{2.017931in}}%
\pgfpathlineto{\pgfqpoint{3.293282in}{2.010158in}}%
\pgfpathlineto{\pgfqpoint{3.284712in}{2.015687in}}%
\pgfpathlineto{\pgfqpoint{3.276125in}{2.021656in}}%
\pgfpathlineto{\pgfqpoint{3.267519in}{2.028074in}}%
\pgfpathlineto{\pgfqpoint{3.258895in}{2.034951in}}%
\pgfpathlineto{\pgfqpoint{3.245021in}{2.043083in}}%
\pgfpathlineto{\pgfqpoint{3.231151in}{2.051245in}}%
\pgfpathlineto{\pgfqpoint{3.217284in}{2.059437in}}%
\pgfpathlineto{\pgfqpoint{3.203421in}{2.067660in}}%
\pgfpathlineto{\pgfqpoint{3.212082in}{2.060419in}}%
\pgfpathlineto{\pgfqpoint{3.220724in}{2.053641in}}%
\pgfpathlineto{\pgfqpoint{3.229348in}{2.047316in}}%
\pgfpathlineto{\pgfqpoint{3.237953in}{2.041434in}}%
\pgfpathclose%
\pgfusepath{fill}%
\end{pgfscope}%
\begin{pgfscope}%
\pgfpathrectangle{\pgfqpoint{1.150000in}{0.150000in}}{\pgfqpoint{5.700000in}{5.700000in}}%
\pgfusepath{clip}%
\pgfsetbuttcap%
\pgfsetroundjoin%
\definecolor{currentfill}{rgb}{0.277941,0.056324,0.381191}%
\pgfsetfillcolor{currentfill}%
\pgfsetfillopacity{0.700000}%
\pgfsetlinewidth{0.000000pt}%
\definecolor{currentstroke}{rgb}{0.000000,0.000000,0.000000}%
\pgfsetstrokecolor{currentstroke}%
\pgfsetdash{}{0pt}%
\pgfpathmoveto{\pgfqpoint{4.932750in}{1.748004in}}%
\pgfpathlineto{\pgfqpoint{4.946943in}{1.745639in}}%
\pgfpathlineto{\pgfqpoint{4.961145in}{1.743297in}}%
\pgfpathlineto{\pgfqpoint{4.975354in}{1.740980in}}%
\pgfpathlineto{\pgfqpoint{4.989571in}{1.738687in}}%
\pgfpathlineto{\pgfqpoint{4.981803in}{1.728083in}}%
\pgfpathlineto{\pgfqpoint{4.974030in}{1.717519in}}%
\pgfpathlineto{\pgfqpoint{4.966253in}{1.707000in}}%
\pgfpathlineto{\pgfqpoint{4.958471in}{1.696532in}}%
\pgfpathlineto{\pgfqpoint{4.944246in}{1.699031in}}%
\pgfpathlineto{\pgfqpoint{4.930029in}{1.701554in}}%
\pgfpathlineto{\pgfqpoint{4.915820in}{1.704101in}}%
\pgfpathlineto{\pgfqpoint{4.901619in}{1.706673in}}%
\pgfpathlineto{\pgfqpoint{4.909408in}{1.716930in}}%
\pgfpathlineto{\pgfqpoint{4.917193in}{1.727241in}}%
\pgfpathlineto{\pgfqpoint{4.924974in}{1.737601in}}%
\pgfpathlineto{\pgfqpoint{4.932750in}{1.748004in}}%
\pgfpathclose%
\pgfusepath{fill}%
\end{pgfscope}%
\begin{pgfscope}%
\pgfpathrectangle{\pgfqpoint{1.150000in}{0.150000in}}{\pgfqpoint{5.700000in}{5.700000in}}%
\pgfusepath{clip}%
\pgfsetbuttcap%
\pgfsetroundjoin%
\definecolor{currentfill}{rgb}{0.282290,0.145912,0.461510}%
\pgfsetfillcolor{currentfill}%
\pgfsetfillopacity{0.700000}%
\pgfsetlinewidth{0.000000pt}%
\definecolor{currentstroke}{rgb}{0.000000,0.000000,0.000000}%
\pgfsetstrokecolor{currentstroke}%
\pgfsetdash{}{0pt}%
\pgfpathmoveto{\pgfqpoint{3.493376in}{1.905359in}}%
\pgfpathlineto{\pgfqpoint{3.507237in}{1.898306in}}%
\pgfpathlineto{\pgfqpoint{3.521101in}{1.891282in}}%
\pgfpathlineto{\pgfqpoint{3.534971in}{1.884285in}}%
\pgfpathlineto{\pgfqpoint{3.548845in}{1.877317in}}%
\pgfpathlineto{\pgfqpoint{3.540466in}{1.879840in}}%
\pgfpathlineto{\pgfqpoint{3.532074in}{1.882749in}}%
\pgfpathlineto{\pgfqpoint{3.523668in}{1.886054in}}%
\pgfpathlineto{\pgfqpoint{3.515247in}{1.889763in}}%
\pgfpathlineto{\pgfqpoint{3.501343in}{1.897073in}}%
\pgfpathlineto{\pgfqpoint{3.487442in}{1.904411in}}%
\pgfpathlineto{\pgfqpoint{3.473546in}{1.911777in}}%
\pgfpathlineto{\pgfqpoint{3.459655in}{1.919172in}}%
\pgfpathlineto{\pgfqpoint{3.468107in}{1.915116in}}%
\pgfpathlineto{\pgfqpoint{3.476545in}{1.911468in}}%
\pgfpathlineto{\pgfqpoint{3.484968in}{1.908218in}}%
\pgfpathlineto{\pgfqpoint{3.493376in}{1.905359in}}%
\pgfpathclose%
\pgfusepath{fill}%
\end{pgfscope}%
\begin{pgfscope}%
\pgfpathrectangle{\pgfqpoint{1.150000in}{0.150000in}}{\pgfqpoint{5.700000in}{5.700000in}}%
\pgfusepath{clip}%
\pgfsetbuttcap%
\pgfsetroundjoin%
\definecolor{currentfill}{rgb}{0.278791,0.062145,0.386592}%
\pgfsetfillcolor{currentfill}%
\pgfsetfillopacity{0.700000}%
\pgfsetlinewidth{0.000000pt}%
\definecolor{currentstroke}{rgb}{0.000000,0.000000,0.000000}%
\pgfsetstrokecolor{currentstroke}%
\pgfsetdash{}{0pt}%
\pgfpathmoveto{\pgfqpoint{3.892849in}{1.749073in}}%
\pgfpathlineto{\pgfqpoint{3.906777in}{1.743308in}}%
\pgfpathlineto{\pgfqpoint{3.920709in}{1.737568in}}%
\pgfpathlineto{\pgfqpoint{3.934648in}{1.731855in}}%
\pgfpathlineto{\pgfqpoint{3.948591in}{1.726168in}}%
\pgfpathlineto{\pgfqpoint{3.940457in}{1.724083in}}%
\pgfpathlineto{\pgfqpoint{3.932313in}{1.722293in}}%
\pgfpathlineto{\pgfqpoint{3.924160in}{1.720806in}}%
\pgfpathlineto{\pgfqpoint{3.915998in}{1.719629in}}%
\pgfpathlineto{\pgfqpoint{3.902031in}{1.725629in}}%
\pgfpathlineto{\pgfqpoint{3.888070in}{1.731654in}}%
\pgfpathlineto{\pgfqpoint{3.874114in}{1.737705in}}%
\pgfpathlineto{\pgfqpoint{3.860164in}{1.743783in}}%
\pgfpathlineto{\pgfqpoint{3.868349in}{1.744641in}}%
\pgfpathlineto{\pgfqpoint{3.876526in}{1.745815in}}%
\pgfpathlineto{\pgfqpoint{3.884692in}{1.747295in}}%
\pgfpathlineto{\pgfqpoint{3.892849in}{1.749073in}}%
\pgfpathclose%
\pgfusepath{fill}%
\end{pgfscope}%
\begin{pgfscope}%
\pgfpathrectangle{\pgfqpoint{1.150000in}{0.150000in}}{\pgfqpoint{5.700000in}{5.700000in}}%
\pgfusepath{clip}%
\pgfsetbuttcap%
\pgfsetroundjoin%
\definecolor{currentfill}{rgb}{0.281887,0.150881,0.465405}%
\pgfsetfillcolor{currentfill}%
\pgfsetfillopacity{0.700000}%
\pgfsetlinewidth{0.000000pt}%
\definecolor{currentstroke}{rgb}{0.000000,0.000000,0.000000}%
\pgfsetstrokecolor{currentstroke}%
\pgfsetdash{}{0pt}%
\pgfpathmoveto{\pgfqpoint{5.429332in}{1.928240in}}%
\pgfpathlineto{\pgfqpoint{5.443695in}{1.927288in}}%
\pgfpathlineto{\pgfqpoint{5.458067in}{1.926360in}}%
\pgfpathlineto{\pgfqpoint{5.472448in}{1.925456in}}%
\pgfpathlineto{\pgfqpoint{5.486838in}{1.924577in}}%
\pgfpathlineto{\pgfqpoint{5.479213in}{1.913263in}}%
\pgfpathlineto{\pgfqpoint{5.471581in}{1.901888in}}%
\pgfpathlineto{\pgfqpoint{5.463943in}{1.890456in}}%
\pgfpathlineto{\pgfqpoint{5.456300in}{1.878971in}}%
\pgfpathlineto{\pgfqpoint{5.441903in}{1.879990in}}%
\pgfpathlineto{\pgfqpoint{5.427516in}{1.881033in}}%
\pgfpathlineto{\pgfqpoint{5.413138in}{1.882101in}}%
\pgfpathlineto{\pgfqpoint{5.398769in}{1.883192in}}%
\pgfpathlineto{\pgfqpoint{5.406419in}{1.894533in}}%
\pgfpathlineto{\pgfqpoint{5.414063in}{1.905823in}}%
\pgfpathlineto{\pgfqpoint{5.421701in}{1.917060in}}%
\pgfpathlineto{\pgfqpoint{5.429332in}{1.928240in}}%
\pgfpathclose%
\pgfusepath{fill}%
\end{pgfscope}%
\begin{pgfscope}%
\pgfpathrectangle{\pgfqpoint{1.150000in}{0.150000in}}{\pgfqpoint{5.700000in}{5.700000in}}%
\pgfusepath{clip}%
\pgfsetbuttcap%
\pgfsetroundjoin%
\definecolor{currentfill}{rgb}{0.271305,0.019942,0.347269}%
\pgfsetfillcolor{currentfill}%
\pgfsetfillopacity{0.700000}%
\pgfsetlinewidth{0.000000pt}%
\definecolor{currentstroke}{rgb}{0.000000,0.000000,0.000000}%
\pgfsetstrokecolor{currentstroke}%
\pgfsetdash{}{0pt}%
\pgfpathmoveto{\pgfqpoint{4.612676in}{1.678831in}}%
\pgfpathlineto{\pgfqpoint{4.626776in}{1.675422in}}%
\pgfpathlineto{\pgfqpoint{4.640883in}{1.672036in}}%
\pgfpathlineto{\pgfqpoint{4.654997in}{1.668675in}}%
\pgfpathlineto{\pgfqpoint{4.669119in}{1.665338in}}%
\pgfpathlineto{\pgfqpoint{4.661264in}{1.656527in}}%
\pgfpathlineto{\pgfqpoint{4.653404in}{1.647836in}}%
\pgfpathlineto{\pgfqpoint{4.645539in}{1.639270in}}%
\pgfpathlineto{\pgfqpoint{4.637670in}{1.630836in}}%
\pgfpathlineto{\pgfqpoint{4.623538in}{1.634418in}}%
\pgfpathlineto{\pgfqpoint{4.609412in}{1.638024in}}%
\pgfpathlineto{\pgfqpoint{4.595294in}{1.641655in}}%
\pgfpathlineto{\pgfqpoint{4.581183in}{1.645310in}}%
\pgfpathlineto{\pgfqpoint{4.589063in}{1.653494in}}%
\pgfpathlineto{\pgfqpoint{4.596938in}{1.661813in}}%
\pgfpathlineto{\pgfqpoint{4.604809in}{1.670261in}}%
\pgfpathlineto{\pgfqpoint{4.612676in}{1.678831in}}%
\pgfpathclose%
\pgfusepath{fill}%
\end{pgfscope}%
\begin{pgfscope}%
\pgfpathrectangle{\pgfqpoint{1.150000in}{0.150000in}}{\pgfqpoint{5.700000in}{5.700000in}}%
\pgfusepath{clip}%
\pgfsetbuttcap%
\pgfsetroundjoin%
\definecolor{currentfill}{rgb}{0.283072,0.130895,0.449241}%
\pgfsetfillcolor{currentfill}%
\pgfsetfillopacity{0.700000}%
\pgfsetlinewidth{0.000000pt}%
\definecolor{currentstroke}{rgb}{0.000000,0.000000,0.000000}%
\pgfsetstrokecolor{currentstroke}%
\pgfsetdash{}{0pt}%
\pgfpathmoveto{\pgfqpoint{5.341381in}{1.887804in}}%
\pgfpathlineto{\pgfqpoint{5.355715in}{1.886614in}}%
\pgfpathlineto{\pgfqpoint{5.370057in}{1.885449in}}%
\pgfpathlineto{\pgfqpoint{5.384409in}{1.884309in}}%
\pgfpathlineto{\pgfqpoint{5.398769in}{1.883192in}}%
\pgfpathlineto{\pgfqpoint{5.391113in}{1.871805in}}%
\pgfpathlineto{\pgfqpoint{5.383452in}{1.860374in}}%
\pgfpathlineto{\pgfqpoint{5.375785in}{1.848903in}}%
\pgfpathlineto{\pgfqpoint{5.368112in}{1.837396in}}%
\pgfpathlineto{\pgfqpoint{5.353746in}{1.838665in}}%
\pgfpathlineto{\pgfqpoint{5.339388in}{1.839959in}}%
\pgfpathlineto{\pgfqpoint{5.325039in}{1.841277in}}%
\pgfpathlineto{\pgfqpoint{5.310700in}{1.842619in}}%
\pgfpathlineto{\pgfqpoint{5.318378in}{1.853968in}}%
\pgfpathlineto{\pgfqpoint{5.326052in}{1.865284in}}%
\pgfpathlineto{\pgfqpoint{5.333719in}{1.876564in}}%
\pgfpathlineto{\pgfqpoint{5.341381in}{1.887804in}}%
\pgfpathclose%
\pgfusepath{fill}%
\end{pgfscope}%
\begin{pgfscope}%
\pgfpathrectangle{\pgfqpoint{1.150000in}{0.150000in}}{\pgfqpoint{5.700000in}{5.700000in}}%
\pgfusepath{clip}%
\pgfsetbuttcap%
\pgfsetroundjoin%
\definecolor{currentfill}{rgb}{0.276022,0.044167,0.370164}%
\pgfsetfillcolor{currentfill}%
\pgfsetfillopacity{0.700000}%
\pgfsetlinewidth{0.000000pt}%
\definecolor{currentstroke}{rgb}{0.000000,0.000000,0.000000}%
\pgfsetstrokecolor{currentstroke}%
\pgfsetdash{}{0pt}%
\pgfpathmoveto{\pgfqpoint{4.844891in}{1.717200in}}%
\pgfpathlineto{\pgfqpoint{4.859061in}{1.714532in}}%
\pgfpathlineto{\pgfqpoint{4.873239in}{1.711888in}}%
\pgfpathlineto{\pgfqpoint{4.887425in}{1.709268in}}%
\pgfpathlineto{\pgfqpoint{4.901619in}{1.706673in}}%
\pgfpathlineto{\pgfqpoint{4.893825in}{1.696474in}}%
\pgfpathlineto{\pgfqpoint{4.886027in}{1.686339in}}%
\pgfpathlineto{\pgfqpoint{4.878225in}{1.676274in}}%
\pgfpathlineto{\pgfqpoint{4.870419in}{1.666283in}}%
\pgfpathlineto{\pgfqpoint{4.856217in}{1.669098in}}%
\pgfpathlineto{\pgfqpoint{4.842023in}{1.671937in}}%
\pgfpathlineto{\pgfqpoint{4.827837in}{1.674800in}}%
\pgfpathlineto{\pgfqpoint{4.813658in}{1.677687in}}%
\pgfpathlineto{\pgfqpoint{4.821472in}{1.687454in}}%
\pgfpathlineto{\pgfqpoint{4.829283in}{1.697298in}}%
\pgfpathlineto{\pgfqpoint{4.837089in}{1.707215in}}%
\pgfpathlineto{\pgfqpoint{4.844891in}{1.717200in}}%
\pgfpathclose%
\pgfusepath{fill}%
\end{pgfscope}%
\begin{pgfscope}%
\pgfpathrectangle{\pgfqpoint{1.150000in}{0.150000in}}{\pgfqpoint{5.700000in}{5.700000in}}%
\pgfusepath{clip}%
\pgfsetbuttcap%
\pgfsetroundjoin%
\definecolor{currentfill}{rgb}{0.244972,0.287675,0.537260}%
\pgfsetfillcolor{currentfill}%
\pgfsetfillopacity{0.700000}%
\pgfsetlinewidth{0.000000pt}%
\definecolor{currentstroke}{rgb}{0.000000,0.000000,0.000000}%
\pgfsetstrokecolor{currentstroke}%
\pgfsetdash{}{0pt}%
\pgfpathmoveto{\pgfqpoint{2.982084in}{2.203526in}}%
\pgfpathlineto{\pgfqpoint{2.995893in}{2.194790in}}%
\pgfpathlineto{\pgfqpoint{3.009705in}{2.186087in}}%
\pgfpathlineto{\pgfqpoint{3.023520in}{2.177418in}}%
\pgfpathlineto{\pgfqpoint{3.037338in}{2.168782in}}%
\pgfpathlineto{\pgfqpoint{3.028539in}{2.177603in}}%
\pgfpathlineto{\pgfqpoint{3.019717in}{2.186921in}}%
\pgfpathlineto{\pgfqpoint{3.010873in}{2.196745in}}%
\pgfpathlineto{\pgfqpoint{3.002006in}{2.207086in}}%
\pgfpathlineto{\pgfqpoint{2.988146in}{2.216098in}}%
\pgfpathlineto{\pgfqpoint{2.974289in}{2.225143in}}%
\pgfpathlineto{\pgfqpoint{2.960434in}{2.234222in}}%
\pgfpathlineto{\pgfqpoint{2.946583in}{2.243334in}}%
\pgfpathlineto{\pgfqpoint{2.955494in}{2.232611in}}%
\pgfpathlineto{\pgfqpoint{2.964380in}{2.222409in}}%
\pgfpathlineto{\pgfqpoint{2.973244in}{2.212717in}}%
\pgfpathlineto{\pgfqpoint{2.982084in}{2.203526in}}%
\pgfpathclose%
\pgfusepath{fill}%
\end{pgfscope}%
\begin{pgfscope}%
\pgfpathrectangle{\pgfqpoint{1.150000in}{0.150000in}}{\pgfqpoint{5.700000in}{5.700000in}}%
\pgfusepath{clip}%
\pgfsetbuttcap%
\pgfsetroundjoin%
\definecolor{currentfill}{rgb}{0.194100,0.399323,0.555565}%
\pgfsetfillcolor{currentfill}%
\pgfsetfillopacity{0.700000}%
\pgfsetlinewidth{0.000000pt}%
\definecolor{currentstroke}{rgb}{0.000000,0.000000,0.000000}%
\pgfsetstrokecolor{currentstroke}%
\pgfsetdash{}{0pt}%
\pgfpathmoveto{\pgfqpoint{2.614949in}{2.472766in}}%
\pgfpathlineto{\pgfqpoint{2.628740in}{2.462770in}}%
\pgfpathlineto{\pgfqpoint{2.642533in}{2.452814in}}%
\pgfpathlineto{\pgfqpoint{2.656329in}{2.442898in}}%
\pgfpathlineto{\pgfqpoint{2.670126in}{2.433021in}}%
\pgfpathlineto{\pgfqpoint{2.660958in}{2.446219in}}%
\pgfpathlineto{\pgfqpoint{2.651762in}{2.459983in}}%
\pgfpathlineto{\pgfqpoint{2.642536in}{2.474323in}}%
\pgfpathlineto{\pgfqpoint{2.633279in}{2.489251in}}%
\pgfpathlineto{\pgfqpoint{2.619432in}{2.499525in}}%
\pgfpathlineto{\pgfqpoint{2.605586in}{2.509838in}}%
\pgfpathlineto{\pgfqpoint{2.591743in}{2.520192in}}%
\pgfpathlineto{\pgfqpoint{2.577901in}{2.530586in}}%
\pgfpathlineto{\pgfqpoint{2.587209in}{2.515253in}}%
\pgfpathlineto{\pgfqpoint{2.596486in}{2.500514in}}%
\pgfpathlineto{\pgfqpoint{2.605732in}{2.486356in}}%
\pgfpathlineto{\pgfqpoint{2.614949in}{2.472766in}}%
\pgfpathclose%
\pgfusepath{fill}%
\end{pgfscope}%
\begin{pgfscope}%
\pgfpathrectangle{\pgfqpoint{1.150000in}{0.150000in}}{\pgfqpoint{5.700000in}{5.700000in}}%
\pgfusepath{clip}%
\pgfsetbuttcap%
\pgfsetroundjoin%
\definecolor{currentfill}{rgb}{0.271305,0.019942,0.347269}%
\pgfsetfillcolor{currentfill}%
\pgfsetfillopacity{0.700000}%
\pgfsetlinewidth{0.000000pt}%
\definecolor{currentstroke}{rgb}{0.000000,0.000000,0.000000}%
\pgfsetstrokecolor{currentstroke}%
\pgfsetdash{}{0pt}%
\pgfpathmoveto{\pgfqpoint{4.236633in}{1.674287in}}%
\pgfpathlineto{\pgfqpoint{4.250639in}{1.669612in}}%
\pgfpathlineto{\pgfqpoint{4.264651in}{1.664962in}}%
\pgfpathlineto{\pgfqpoint{4.278670in}{1.660337in}}%
\pgfpathlineto{\pgfqpoint{4.292694in}{1.655737in}}%
\pgfpathlineto{\pgfqpoint{4.284713in}{1.650181in}}%
\pgfpathlineto{\pgfqpoint{4.276725in}{1.644840in}}%
\pgfpathlineto{\pgfqpoint{4.268731in}{1.639723in}}%
\pgfpathlineto{\pgfqpoint{4.260730in}{1.634836in}}%
\pgfpathlineto{\pgfqpoint{4.246689in}{1.639722in}}%
\pgfpathlineto{\pgfqpoint{4.232654in}{1.644632in}}%
\pgfpathlineto{\pgfqpoint{4.218624in}{1.649566in}}%
\pgfpathlineto{\pgfqpoint{4.204601in}{1.654526in}}%
\pgfpathlineto{\pgfqpoint{4.212619in}{1.659123in}}%
\pgfpathlineto{\pgfqpoint{4.220630in}{1.663953in}}%
\pgfpathlineto{\pgfqpoint{4.228634in}{1.669011in}}%
\pgfpathlineto{\pgfqpoint{4.236633in}{1.674287in}}%
\pgfpathclose%
\pgfusepath{fill}%
\end{pgfscope}%
\begin{pgfscope}%
\pgfpathrectangle{\pgfqpoint{1.150000in}{0.150000in}}{\pgfqpoint{5.700000in}{5.700000in}}%
\pgfusepath{clip}%
\pgfsetbuttcap%
\pgfsetroundjoin%
\definecolor{currentfill}{rgb}{0.283197,0.115680,0.436115}%
\pgfsetfillcolor{currentfill}%
\pgfsetfillopacity{0.700000}%
\pgfsetlinewidth{0.000000pt}%
\definecolor{currentstroke}{rgb}{0.000000,0.000000,0.000000}%
\pgfsetstrokecolor{currentstroke}%
\pgfsetdash{}{0pt}%
\pgfpathmoveto{\pgfqpoint{5.253427in}{1.848231in}}%
\pgfpathlineto{\pgfqpoint{5.267732in}{1.846792in}}%
\pgfpathlineto{\pgfqpoint{5.282046in}{1.845376in}}%
\pgfpathlineto{\pgfqpoint{5.296368in}{1.843986in}}%
\pgfpathlineto{\pgfqpoint{5.310700in}{1.842619in}}%
\pgfpathlineto{\pgfqpoint{5.303015in}{1.831240in}}%
\pgfpathlineto{\pgfqpoint{5.295326in}{1.819836in}}%
\pgfpathlineto{\pgfqpoint{5.287631in}{1.808411in}}%
\pgfpathlineto{\pgfqpoint{5.279931in}{1.796967in}}%
\pgfpathlineto{\pgfqpoint{5.265594in}{1.798500in}}%
\pgfpathlineto{\pgfqpoint{5.251265in}{1.800057in}}%
\pgfpathlineto{\pgfqpoint{5.236945in}{1.801639in}}%
\pgfpathlineto{\pgfqpoint{5.222634in}{1.803245in}}%
\pgfpathlineto{\pgfqpoint{5.230340in}{1.814517in}}%
\pgfpathlineto{\pgfqpoint{5.238041in}{1.825774in}}%
\pgfpathlineto{\pgfqpoint{5.245737in}{1.837014in}}%
\pgfpathlineto{\pgfqpoint{5.253427in}{1.848231in}}%
\pgfpathclose%
\pgfusepath{fill}%
\end{pgfscope}%
\begin{pgfscope}%
\pgfpathrectangle{\pgfqpoint{1.150000in}{0.150000in}}{\pgfqpoint{5.700000in}{5.700000in}}%
\pgfusepath{clip}%
\pgfsetbuttcap%
\pgfsetroundjoin%
\definecolor{currentfill}{rgb}{0.269944,0.014625,0.341379}%
\pgfsetfillcolor{currentfill}%
\pgfsetfillopacity{0.700000}%
\pgfsetlinewidth{0.000000pt}%
\definecolor{currentstroke}{rgb}{0.000000,0.000000,0.000000}%
\pgfsetstrokecolor{currentstroke}%
\pgfsetdash{}{0pt}%
\pgfpathmoveto{\pgfqpoint{4.380666in}{1.662905in}}%
\pgfpathlineto{\pgfqpoint{4.394709in}{1.658699in}}%
\pgfpathlineto{\pgfqpoint{4.408757in}{1.654518in}}%
\pgfpathlineto{\pgfqpoint{4.422813in}{1.650361in}}%
\pgfpathlineto{\pgfqpoint{4.436875in}{1.646229in}}%
\pgfpathlineto{\pgfqpoint{4.428945in}{1.639349in}}%
\pgfpathlineto{\pgfqpoint{4.421010in}{1.632649in}}%
\pgfpathlineto{\pgfqpoint{4.413070in}{1.626137in}}%
\pgfpathlineto{\pgfqpoint{4.405125in}{1.619820in}}%
\pgfpathlineto{\pgfqpoint{4.391048in}{1.624223in}}%
\pgfpathlineto{\pgfqpoint{4.376978in}{1.628652in}}%
\pgfpathlineto{\pgfqpoint{4.362915in}{1.633104in}}%
\pgfpathlineto{\pgfqpoint{4.348858in}{1.637582in}}%
\pgfpathlineto{\pgfqpoint{4.356818in}{1.643622in}}%
\pgfpathlineto{\pgfqpoint{4.364773in}{1.649861in}}%
\pgfpathlineto{\pgfqpoint{4.372722in}{1.656291in}}%
\pgfpathlineto{\pgfqpoint{4.380666in}{1.662905in}}%
\pgfpathclose%
\pgfusepath{fill}%
\end{pgfscope}%
\begin{pgfscope}%
\pgfpathrectangle{\pgfqpoint{1.150000in}{0.150000in}}{\pgfqpoint{5.700000in}{5.700000in}}%
\pgfusepath{clip}%
\pgfsetbuttcap%
\pgfsetroundjoin%
\definecolor{currentfill}{rgb}{0.274952,0.037752,0.364543}%
\pgfsetfillcolor{currentfill}%
\pgfsetfillopacity{0.700000}%
\pgfsetlinewidth{0.000000pt}%
\definecolor{currentstroke}{rgb}{0.000000,0.000000,0.000000}%
\pgfsetstrokecolor{currentstroke}%
\pgfsetdash{}{0pt}%
\pgfpathmoveto{\pgfqpoint{4.092634in}{1.695101in}}%
\pgfpathlineto{\pgfqpoint{4.106609in}{1.689941in}}%
\pgfpathlineto{\pgfqpoint{4.120590in}{1.684806in}}%
\pgfpathlineto{\pgfqpoint{4.134577in}{1.679697in}}%
\pgfpathlineto{\pgfqpoint{4.148570in}{1.674613in}}%
\pgfpathlineto{\pgfqpoint{4.140527in}{1.670550in}}%
\pgfpathlineto{\pgfqpoint{4.132478in}{1.666740in}}%
\pgfpathlineto{\pgfqpoint{4.124421in}{1.663191in}}%
\pgfpathlineto{\pgfqpoint{4.116357in}{1.659909in}}%
\pgfpathlineto{\pgfqpoint{4.102345in}{1.665292in}}%
\pgfpathlineto{\pgfqpoint{4.088339in}{1.670699in}}%
\pgfpathlineto{\pgfqpoint{4.074338in}{1.676132in}}%
\pgfpathlineto{\pgfqpoint{4.060344in}{1.681590in}}%
\pgfpathlineto{\pgfqpoint{4.068428in}{1.684568in}}%
\pgfpathlineto{\pgfqpoint{4.076504in}{1.687818in}}%
\pgfpathlineto{\pgfqpoint{4.084573in}{1.691331in}}%
\pgfpathlineto{\pgfqpoint{4.092634in}{1.695101in}}%
\pgfpathclose%
\pgfusepath{fill}%
\end{pgfscope}%
\begin{pgfscope}%
\pgfpathrectangle{\pgfqpoint{1.150000in}{0.150000in}}{\pgfqpoint{5.700000in}{5.700000in}}%
\pgfusepath{clip}%
\pgfsetbuttcap%
\pgfsetroundjoin%
\definecolor{currentfill}{rgb}{0.274128,0.199721,0.498911}%
\pgfsetfillcolor{currentfill}%
\pgfsetfillopacity{0.700000}%
\pgfsetlinewidth{0.000000pt}%
\definecolor{currentstroke}{rgb}{0.000000,0.000000,0.000000}%
\pgfsetstrokecolor{currentstroke}%
\pgfsetdash{}{0pt}%
\pgfpathmoveto{\pgfqpoint{3.293282in}{2.010158in}}%
\pgfpathlineto{\pgfqpoint{3.307124in}{2.002414in}}%
\pgfpathlineto{\pgfqpoint{3.320970in}{1.994700in}}%
\pgfpathlineto{\pgfqpoint{3.334820in}{1.987016in}}%
\pgfpathlineto{\pgfqpoint{3.348674in}{1.979361in}}%
\pgfpathlineto{\pgfqpoint{3.340139in}{1.984539in}}%
\pgfpathlineto{\pgfqpoint{3.331587in}{1.990152in}}%
\pgfpathlineto{\pgfqpoint{3.323017in}{1.996210in}}%
\pgfpathlineto{\pgfqpoint{3.314430in}{2.002723in}}%
\pgfpathlineto{\pgfqpoint{3.300541in}{2.010735in}}%
\pgfpathlineto{\pgfqpoint{3.286655in}{2.018777in}}%
\pgfpathlineto{\pgfqpoint{3.272773in}{2.026849in}}%
\pgfpathlineto{\pgfqpoint{3.258895in}{2.034951in}}%
\pgfpathlineto{\pgfqpoint{3.267519in}{2.028074in}}%
\pgfpathlineto{\pgfqpoint{3.276125in}{2.021656in}}%
\pgfpathlineto{\pgfqpoint{3.284712in}{2.015687in}}%
\pgfpathlineto{\pgfqpoint{3.293282in}{2.010158in}}%
\pgfpathclose%
\pgfusepath{fill}%
\end{pgfscope}%
\begin{pgfscope}%
\pgfpathrectangle{\pgfqpoint{1.150000in}{0.150000in}}{\pgfqpoint{5.700000in}{5.700000in}}%
\pgfusepath{clip}%
\pgfsetbuttcap%
\pgfsetroundjoin%
\definecolor{currentfill}{rgb}{0.282327,0.094955,0.417331}%
\pgfsetfillcolor{currentfill}%
\pgfsetfillopacity{0.700000}%
\pgfsetlinewidth{0.000000pt}%
\definecolor{currentstroke}{rgb}{0.000000,0.000000,0.000000}%
\pgfsetstrokecolor{currentstroke}%
\pgfsetdash{}{0pt}%
\pgfpathmoveto{\pgfqpoint{5.165473in}{1.809910in}}%
\pgfpathlineto{\pgfqpoint{5.179751in}{1.808207in}}%
\pgfpathlineto{\pgfqpoint{5.194037in}{1.806529in}}%
\pgfpathlineto{\pgfqpoint{5.208331in}{1.804875in}}%
\pgfpathlineto{\pgfqpoint{5.222634in}{1.803245in}}%
\pgfpathlineto{\pgfqpoint{5.214923in}{1.791962in}}%
\pgfpathlineto{\pgfqpoint{5.207207in}{1.780674in}}%
\pgfpathlineto{\pgfqpoint{5.199486in}{1.769383in}}%
\pgfpathlineto{\pgfqpoint{5.191761in}{1.758095in}}%
\pgfpathlineto{\pgfqpoint{5.177452in}{1.759905in}}%
\pgfpathlineto{\pgfqpoint{5.163151in}{1.761739in}}%
\pgfpathlineto{\pgfqpoint{5.148858in}{1.763597in}}%
\pgfpathlineto{\pgfqpoint{5.134575in}{1.765480in}}%
\pgfpathlineto{\pgfqpoint{5.142306in}{1.776583in}}%
\pgfpathlineto{\pgfqpoint{5.150034in}{1.787692in}}%
\pgfpathlineto{\pgfqpoint{5.157756in}{1.798803in}}%
\pgfpathlineto{\pgfqpoint{5.165473in}{1.809910in}}%
\pgfpathclose%
\pgfusepath{fill}%
\end{pgfscope}%
\begin{pgfscope}%
\pgfpathrectangle{\pgfqpoint{1.150000in}{0.150000in}}{\pgfqpoint{5.700000in}{5.700000in}}%
\pgfusepath{clip}%
\pgfsetbuttcap%
\pgfsetroundjoin%
\definecolor{currentfill}{rgb}{0.282327,0.094955,0.417331}%
\pgfsetfillcolor{currentfill}%
\pgfsetfillopacity{0.700000}%
\pgfsetlinewidth{0.000000pt}%
\definecolor{currentstroke}{rgb}{0.000000,0.000000,0.000000}%
\pgfsetstrokecolor{currentstroke}%
\pgfsetdash{}{0pt}%
\pgfpathmoveto{\pgfqpoint{3.748747in}{1.793343in}}%
\pgfpathlineto{\pgfqpoint{3.762656in}{1.787056in}}%
\pgfpathlineto{\pgfqpoint{3.776570in}{1.780795in}}%
\pgfpathlineto{\pgfqpoint{3.790490in}{1.774560in}}%
\pgfpathlineto{\pgfqpoint{3.804414in}{1.768352in}}%
\pgfpathlineto{\pgfqpoint{3.796194in}{1.768138in}}%
\pgfpathlineto{\pgfqpoint{3.787962in}{1.768259in}}%
\pgfpathlineto{\pgfqpoint{3.779720in}{1.768724in}}%
\pgfpathlineto{\pgfqpoint{3.771467in}{1.769542in}}%
\pgfpathlineto{\pgfqpoint{3.757517in}{1.776077in}}%
\pgfpathlineto{\pgfqpoint{3.743572in}{1.782638in}}%
\pgfpathlineto{\pgfqpoint{3.729631in}{1.789225in}}%
\pgfpathlineto{\pgfqpoint{3.715696in}{1.795839in}}%
\pgfpathlineto{\pgfqpoint{3.723976in}{1.794689in}}%
\pgfpathlineto{\pgfqpoint{3.732244in}{1.793896in}}%
\pgfpathlineto{\pgfqpoint{3.740501in}{1.793450in}}%
\pgfpathlineto{\pgfqpoint{3.748747in}{1.793343in}}%
\pgfpathclose%
\pgfusepath{fill}%
\end{pgfscope}%
\begin{pgfscope}%
\pgfpathrectangle{\pgfqpoint{1.150000in}{0.150000in}}{\pgfqpoint{5.700000in}{5.700000in}}%
\pgfusepath{clip}%
\pgfsetbuttcap%
\pgfsetroundjoin%
\definecolor{currentfill}{rgb}{0.132444,0.552216,0.553018}%
\pgfsetfillcolor{currentfill}%
\pgfsetfillopacity{0.700000}%
\pgfsetlinewidth{0.000000pt}%
\definecolor{currentstroke}{rgb}{0.000000,0.000000,0.000000}%
\pgfsetstrokecolor{currentstroke}%
\pgfsetdash{}{0pt}%
\pgfpathmoveto{\pgfqpoint{2.135651in}{2.886871in}}%
\pgfpathlineto{\pgfqpoint{2.149460in}{2.874967in}}%
\pgfpathlineto{\pgfqpoint{2.163270in}{2.863118in}}%
\pgfpathlineto{\pgfqpoint{2.177079in}{2.851323in}}%
\pgfpathlineto{\pgfqpoint{2.190888in}{2.839581in}}%
\pgfpathlineto{\pgfqpoint{2.181157in}{2.858409in}}%
\pgfpathlineto{\pgfqpoint{2.171386in}{2.877885in}}%
\pgfpathlineto{\pgfqpoint{2.161575in}{2.898024in}}%
\pgfpathlineto{\pgfqpoint{2.151724in}{2.918837in}}%
\pgfpathlineto{\pgfqpoint{2.137853in}{2.931003in}}%
\pgfpathlineto{\pgfqpoint{2.123983in}{2.943224in}}%
\pgfpathlineto{\pgfqpoint{2.110113in}{2.955499in}}%
\pgfpathlineto{\pgfqpoint{2.096243in}{2.967829in}}%
\pgfpathlineto{\pgfqpoint{2.106157in}{2.946583in}}%
\pgfpathlineto{\pgfqpoint{2.116029in}{2.926017in}}%
\pgfpathlineto{\pgfqpoint{2.125860in}{2.906117in}}%
\pgfpathlineto{\pgfqpoint{2.135651in}{2.886871in}}%
\pgfpathclose%
\pgfusepath{fill}%
\end{pgfscope}%
\begin{pgfscope}%
\pgfpathrectangle{\pgfqpoint{1.150000in}{0.150000in}}{\pgfqpoint{5.700000in}{5.700000in}}%
\pgfusepath{clip}%
\pgfsetbuttcap%
\pgfsetroundjoin%
\definecolor{currentfill}{rgb}{0.273809,0.031497,0.358853}%
\pgfsetfillcolor{currentfill}%
\pgfsetfillopacity{0.700000}%
\pgfsetlinewidth{0.000000pt}%
\definecolor{currentstroke}{rgb}{0.000000,0.000000,0.000000}%
\pgfsetstrokecolor{currentstroke}%
\pgfsetdash{}{0pt}%
\pgfpathmoveto{\pgfqpoint{4.757018in}{1.689477in}}%
\pgfpathlineto{\pgfqpoint{4.771166in}{1.686493in}}%
\pgfpathlineto{\pgfqpoint{4.785323in}{1.683534in}}%
\pgfpathlineto{\pgfqpoint{4.799486in}{1.680598in}}%
\pgfpathlineto{\pgfqpoint{4.813658in}{1.677687in}}%
\pgfpathlineto{\pgfqpoint{4.805839in}{1.668004in}}%
\pgfpathlineto{\pgfqpoint{4.798016in}{1.658410in}}%
\pgfpathlineto{\pgfqpoint{4.790188in}{1.648912in}}%
\pgfpathlineto{\pgfqpoint{4.782357in}{1.639513in}}%
\pgfpathlineto{\pgfqpoint{4.768176in}{1.642657in}}%
\pgfpathlineto{\pgfqpoint{4.754003in}{1.645825in}}%
\pgfpathlineto{\pgfqpoint{4.739837in}{1.649017in}}%
\pgfpathlineto{\pgfqpoint{4.725679in}{1.652232in}}%
\pgfpathlineto{\pgfqpoint{4.733520in}{1.661393in}}%
\pgfpathlineto{\pgfqpoint{4.741357in}{1.670658in}}%
\pgfpathlineto{\pgfqpoint{4.749189in}{1.680021in}}%
\pgfpathlineto{\pgfqpoint{4.757018in}{1.689477in}}%
\pgfpathclose%
\pgfusepath{fill}%
\end{pgfscope}%
\begin{pgfscope}%
\pgfpathrectangle{\pgfqpoint{1.150000in}{0.150000in}}{\pgfqpoint{5.700000in}{5.700000in}}%
\pgfusepath{clip}%
\pgfsetbuttcap%
\pgfsetroundjoin%
\definecolor{currentfill}{rgb}{0.282623,0.140926,0.457517}%
\pgfsetfillcolor{currentfill}%
\pgfsetfillopacity{0.700000}%
\pgfsetlinewidth{0.000000pt}%
\definecolor{currentstroke}{rgb}{0.000000,0.000000,0.000000}%
\pgfsetstrokecolor{currentstroke}%
\pgfsetdash{}{0pt}%
\pgfpathmoveto{\pgfqpoint{3.548845in}{1.877317in}}%
\pgfpathlineto{\pgfqpoint{3.562723in}{1.870376in}}%
\pgfpathlineto{\pgfqpoint{3.576606in}{1.863463in}}%
\pgfpathlineto{\pgfqpoint{3.590494in}{1.856578in}}%
\pgfpathlineto{\pgfqpoint{3.604386in}{1.849721in}}%
\pgfpathlineto{\pgfqpoint{3.596037in}{1.851908in}}%
\pgfpathlineto{\pgfqpoint{3.587675in}{1.854477in}}%
\pgfpathlineto{\pgfqpoint{3.579300in}{1.857438in}}%
\pgfpathlineto{\pgfqpoint{3.570910in}{1.860800in}}%
\pgfpathlineto{\pgfqpoint{3.556988in}{1.867999in}}%
\pgfpathlineto{\pgfqpoint{3.543070in}{1.875226in}}%
\pgfpathlineto{\pgfqpoint{3.529156in}{1.882480in}}%
\pgfpathlineto{\pgfqpoint{3.515247in}{1.889763in}}%
\pgfpathlineto{\pgfqpoint{3.523668in}{1.886054in}}%
\pgfpathlineto{\pgfqpoint{3.532074in}{1.882749in}}%
\pgfpathlineto{\pgfqpoint{3.540466in}{1.879840in}}%
\pgfpathlineto{\pgfqpoint{3.548845in}{1.877317in}}%
\pgfpathclose%
\pgfusepath{fill}%
\end{pgfscope}%
\begin{pgfscope}%
\pgfpathrectangle{\pgfqpoint{1.150000in}{0.150000in}}{\pgfqpoint{5.700000in}{5.700000in}}%
\pgfusepath{clip}%
\pgfsetbuttcap%
\pgfsetroundjoin%
\definecolor{currentfill}{rgb}{0.269944,0.014625,0.341379}%
\pgfsetfillcolor{currentfill}%
\pgfsetfillopacity{0.700000}%
\pgfsetlinewidth{0.000000pt}%
\definecolor{currentstroke}{rgb}{0.000000,0.000000,0.000000}%
\pgfsetstrokecolor{currentstroke}%
\pgfsetdash{}{0pt}%
\pgfpathmoveto{\pgfqpoint{4.524807in}{1.660172in}}%
\pgfpathlineto{\pgfqpoint{4.538891in}{1.656420in}}%
\pgfpathlineto{\pgfqpoint{4.552981in}{1.652692in}}%
\pgfpathlineto{\pgfqpoint{4.567078in}{1.648989in}}%
\pgfpathlineto{\pgfqpoint{4.581183in}{1.645310in}}%
\pgfpathlineto{\pgfqpoint{4.573298in}{1.637266in}}%
\pgfpathlineto{\pgfqpoint{4.565409in}{1.629371in}}%
\pgfpathlineto{\pgfqpoint{4.557515in}{1.621629in}}%
\pgfpathlineto{\pgfqpoint{4.549616in}{1.614048in}}%
\pgfpathlineto{\pgfqpoint{4.535499in}{1.617986in}}%
\pgfpathlineto{\pgfqpoint{4.521390in}{1.621948in}}%
\pgfpathlineto{\pgfqpoint{4.507287in}{1.625934in}}%
\pgfpathlineto{\pgfqpoint{4.493191in}{1.629944in}}%
\pgfpathlineto{\pgfqpoint{4.501102in}{1.637262in}}%
\pgfpathlineto{\pgfqpoint{4.509009in}{1.644743in}}%
\pgfpathlineto{\pgfqpoint{4.516911in}{1.652382in}}%
\pgfpathlineto{\pgfqpoint{4.524807in}{1.660172in}}%
\pgfpathclose%
\pgfusepath{fill}%
\end{pgfscope}%
\begin{pgfscope}%
\pgfpathrectangle{\pgfqpoint{1.150000in}{0.150000in}}{\pgfqpoint{5.700000in}{5.700000in}}%
\pgfusepath{clip}%
\pgfsetbuttcap%
\pgfsetroundjoin%
\definecolor{currentfill}{rgb}{0.199430,0.387607,0.554642}%
\pgfsetfillcolor{currentfill}%
\pgfsetfillopacity{0.700000}%
\pgfsetlinewidth{0.000000pt}%
\definecolor{currentstroke}{rgb}{0.000000,0.000000,0.000000}%
\pgfsetstrokecolor{currentstroke}%
\pgfsetdash{}{0pt}%
\pgfpathmoveto{\pgfqpoint{2.670126in}{2.433021in}}%
\pgfpathlineto{\pgfqpoint{2.683926in}{2.423184in}}%
\pgfpathlineto{\pgfqpoint{2.697727in}{2.413385in}}%
\pgfpathlineto{\pgfqpoint{2.711531in}{2.403625in}}%
\pgfpathlineto{\pgfqpoint{2.725338in}{2.393903in}}%
\pgfpathlineto{\pgfqpoint{2.716218in}{2.406712in}}%
\pgfpathlineto{\pgfqpoint{2.707071in}{2.420081in}}%
\pgfpathlineto{\pgfqpoint{2.697894in}{2.434022in}}%
\pgfpathlineto{\pgfqpoint{2.688688in}{2.448547in}}%
\pgfpathlineto{\pgfqpoint{2.674833in}{2.458665in}}%
\pgfpathlineto{\pgfqpoint{2.660980in}{2.468821in}}%
\pgfpathlineto{\pgfqpoint{2.647128in}{2.479016in}}%
\pgfpathlineto{\pgfqpoint{2.633279in}{2.489251in}}%
\pgfpathlineto{\pgfqpoint{2.642536in}{2.474323in}}%
\pgfpathlineto{\pgfqpoint{2.651762in}{2.459983in}}%
\pgfpathlineto{\pgfqpoint{2.660958in}{2.446219in}}%
\pgfpathlineto{\pgfqpoint{2.670126in}{2.433021in}}%
\pgfpathclose%
\pgfusepath{fill}%
\end{pgfscope}%
\begin{pgfscope}%
\pgfpathrectangle{\pgfqpoint{1.150000in}{0.150000in}}{\pgfqpoint{5.700000in}{5.700000in}}%
\pgfusepath{clip}%
\pgfsetbuttcap%
\pgfsetroundjoin%
\definecolor{currentfill}{rgb}{0.280894,0.078907,0.402329}%
\pgfsetfillcolor{currentfill}%
\pgfsetfillopacity{0.700000}%
\pgfsetlinewidth{0.000000pt}%
\definecolor{currentstroke}{rgb}{0.000000,0.000000,0.000000}%
\pgfsetstrokecolor{currentstroke}%
\pgfsetdash{}{0pt}%
\pgfpathmoveto{\pgfqpoint{5.077522in}{1.773251in}}%
\pgfpathlineto{\pgfqpoint{5.091772in}{1.771272in}}%
\pgfpathlineto{\pgfqpoint{5.106032in}{1.769317in}}%
\pgfpathlineto{\pgfqpoint{5.120299in}{1.767386in}}%
\pgfpathlineto{\pgfqpoint{5.134575in}{1.765480in}}%
\pgfpathlineto{\pgfqpoint{5.126838in}{1.754387in}}%
\pgfpathlineto{\pgfqpoint{5.119097in}{1.743308in}}%
\pgfpathlineto{\pgfqpoint{5.111351in}{1.732249in}}%
\pgfpathlineto{\pgfqpoint{5.103601in}{1.721213in}}%
\pgfpathlineto{\pgfqpoint{5.089319in}{1.723313in}}%
\pgfpathlineto{\pgfqpoint{5.075045in}{1.725437in}}%
\pgfpathlineto{\pgfqpoint{5.060779in}{1.727585in}}%
\pgfpathlineto{\pgfqpoint{5.046521in}{1.729757in}}%
\pgfpathlineto{\pgfqpoint{5.054278in}{1.740594in}}%
\pgfpathlineto{\pgfqpoint{5.062030in}{1.751459in}}%
\pgfpathlineto{\pgfqpoint{5.069778in}{1.762346in}}%
\pgfpathlineto{\pgfqpoint{5.077522in}{1.773251in}}%
\pgfpathclose%
\pgfusepath{fill}%
\end{pgfscope}%
\begin{pgfscope}%
\pgfpathrectangle{\pgfqpoint{1.150000in}{0.150000in}}{\pgfqpoint{5.700000in}{5.700000in}}%
\pgfusepath{clip}%
\pgfsetbuttcap%
\pgfsetroundjoin%
\definecolor{currentfill}{rgb}{0.278791,0.062145,0.386592}%
\pgfsetfillcolor{currentfill}%
\pgfsetfillopacity{0.700000}%
\pgfsetlinewidth{0.000000pt}%
\definecolor{currentstroke}{rgb}{0.000000,0.000000,0.000000}%
\pgfsetstrokecolor{currentstroke}%
\pgfsetdash{}{0pt}%
\pgfpathmoveto{\pgfqpoint{3.948591in}{1.726168in}}%
\pgfpathlineto{\pgfqpoint{3.962541in}{1.720506in}}%
\pgfpathlineto{\pgfqpoint{3.976496in}{1.714870in}}%
\pgfpathlineto{\pgfqpoint{3.990456in}{1.709260in}}%
\pgfpathlineto{\pgfqpoint{4.004422in}{1.703675in}}%
\pgfpathlineto{\pgfqpoint{3.996309in}{1.701283in}}%
\pgfpathlineto{\pgfqpoint{3.988187in}{1.699183in}}%
\pgfpathlineto{\pgfqpoint{3.980057in}{1.697382in}}%
\pgfpathlineto{\pgfqpoint{3.971918in}{1.695888in}}%
\pgfpathlineto{\pgfqpoint{3.957930in}{1.701785in}}%
\pgfpathlineto{\pgfqpoint{3.943947in}{1.707707in}}%
\pgfpathlineto{\pgfqpoint{3.929970in}{1.713656in}}%
\pgfpathlineto{\pgfqpoint{3.915998in}{1.719629in}}%
\pgfpathlineto{\pgfqpoint{3.924160in}{1.720806in}}%
\pgfpathlineto{\pgfqpoint{3.932313in}{1.722293in}}%
\pgfpathlineto{\pgfqpoint{3.940457in}{1.724083in}}%
\pgfpathlineto{\pgfqpoint{3.948591in}{1.726168in}}%
\pgfpathclose%
\pgfusepath{fill}%
\end{pgfscope}%
\begin{pgfscope}%
\pgfpathrectangle{\pgfqpoint{1.150000in}{0.150000in}}{\pgfqpoint{5.700000in}{5.700000in}}%
\pgfusepath{clip}%
\pgfsetbuttcap%
\pgfsetroundjoin%
\definecolor{currentfill}{rgb}{0.248629,0.278775,0.534556}%
\pgfsetfillcolor{currentfill}%
\pgfsetfillopacity{0.700000}%
\pgfsetlinewidth{0.000000pt}%
\definecolor{currentstroke}{rgb}{0.000000,0.000000,0.000000}%
\pgfsetstrokecolor{currentstroke}%
\pgfsetdash{}{0pt}%
\pgfpathmoveto{\pgfqpoint{3.037338in}{2.168782in}}%
\pgfpathlineto{\pgfqpoint{3.051160in}{2.160179in}}%
\pgfpathlineto{\pgfqpoint{3.064985in}{2.151608in}}%
\pgfpathlineto{\pgfqpoint{3.078813in}{2.143070in}}%
\pgfpathlineto{\pgfqpoint{3.092644in}{2.134565in}}%
\pgfpathlineto{\pgfqpoint{3.083885in}{2.143017in}}%
\pgfpathlineto{\pgfqpoint{3.075105in}{2.151961in}}%
\pgfpathlineto{\pgfqpoint{3.066302in}{2.161407in}}%
\pgfpathlineto{\pgfqpoint{3.057478in}{2.171367in}}%
\pgfpathlineto{\pgfqpoint{3.043605in}{2.180248in}}%
\pgfpathlineto{\pgfqpoint{3.029736in}{2.189161in}}%
\pgfpathlineto{\pgfqpoint{3.015869in}{2.198107in}}%
\pgfpathlineto{\pgfqpoint{3.002006in}{2.207086in}}%
\pgfpathlineto{\pgfqpoint{3.010873in}{2.196745in}}%
\pgfpathlineto{\pgfqpoint{3.019717in}{2.186921in}}%
\pgfpathlineto{\pgfqpoint{3.028539in}{2.177603in}}%
\pgfpathlineto{\pgfqpoint{3.037338in}{2.168782in}}%
\pgfpathclose%
\pgfusepath{fill}%
\end{pgfscope}%
\begin{pgfscope}%
\pgfpathrectangle{\pgfqpoint{1.150000in}{0.150000in}}{\pgfqpoint{5.700000in}{5.700000in}}%
\pgfusepath{clip}%
\pgfsetbuttcap%
\pgfsetroundjoin%
\definecolor{currentfill}{rgb}{0.137770,0.537492,0.554906}%
\pgfsetfillcolor{currentfill}%
\pgfsetfillopacity{0.700000}%
\pgfsetlinewidth{0.000000pt}%
\definecolor{currentstroke}{rgb}{0.000000,0.000000,0.000000}%
\pgfsetstrokecolor{currentstroke}%
\pgfsetdash{}{0pt}%
\pgfpathmoveto{\pgfqpoint{2.190888in}{2.839581in}}%
\pgfpathlineto{\pgfqpoint{2.204698in}{2.827893in}}%
\pgfpathlineto{\pgfqpoint{2.218509in}{2.816258in}}%
\pgfpathlineto{\pgfqpoint{2.232319in}{2.804674in}}%
\pgfpathlineto{\pgfqpoint{2.246131in}{2.793142in}}%
\pgfpathlineto{\pgfqpoint{2.236458in}{2.811553in}}%
\pgfpathlineto{\pgfqpoint{2.226747in}{2.830608in}}%
\pgfpathlineto{\pgfqpoint{2.216996in}{2.850320in}}%
\pgfpathlineto{\pgfqpoint{2.207206in}{2.870702in}}%
\pgfpathlineto{\pgfqpoint{2.193335in}{2.882657in}}%
\pgfpathlineto{\pgfqpoint{2.179465in}{2.894665in}}%
\pgfpathlineto{\pgfqpoint{2.165594in}{2.906724in}}%
\pgfpathlineto{\pgfqpoint{2.151724in}{2.918837in}}%
\pgfpathlineto{\pgfqpoint{2.161575in}{2.898024in}}%
\pgfpathlineto{\pgfqpoint{2.171386in}{2.877885in}}%
\pgfpathlineto{\pgfqpoint{2.181157in}{2.858409in}}%
\pgfpathlineto{\pgfqpoint{2.190888in}{2.839581in}}%
\pgfpathclose%
\pgfusepath{fill}%
\end{pgfscope}%
\begin{pgfscope}%
\pgfpathrectangle{\pgfqpoint{1.150000in}{0.150000in}}{\pgfqpoint{5.700000in}{5.700000in}}%
\pgfusepath{clip}%
\pgfsetbuttcap%
\pgfsetroundjoin%
\definecolor{currentfill}{rgb}{0.278791,0.062145,0.386592}%
\pgfsetfillcolor{currentfill}%
\pgfsetfillopacity{0.700000}%
\pgfsetlinewidth{0.000000pt}%
\definecolor{currentstroke}{rgb}{0.000000,0.000000,0.000000}%
\pgfsetstrokecolor{currentstroke}%
\pgfsetdash{}{0pt}%
\pgfpathmoveto{\pgfqpoint{4.989571in}{1.738687in}}%
\pgfpathlineto{\pgfqpoint{5.003797in}{1.736418in}}%
\pgfpathlineto{\pgfqpoint{5.018030in}{1.734174in}}%
\pgfpathlineto{\pgfqpoint{5.032272in}{1.731953in}}%
\pgfpathlineto{\pgfqpoint{5.046521in}{1.729757in}}%
\pgfpathlineto{\pgfqpoint{5.038760in}{1.718952in}}%
\pgfpathlineto{\pgfqpoint{5.030994in}{1.708183in}}%
\pgfpathlineto{\pgfqpoint{5.023224in}{1.697457in}}%
\pgfpathlineto{\pgfqpoint{5.015450in}{1.686777in}}%
\pgfpathlineto{\pgfqpoint{5.001194in}{1.689180in}}%
\pgfpathlineto{\pgfqpoint{4.986945in}{1.691606in}}%
\pgfpathlineto{\pgfqpoint{4.972704in}{1.694057in}}%
\pgfpathlineto{\pgfqpoint{4.958471in}{1.696532in}}%
\pgfpathlineto{\pgfqpoint{4.966253in}{1.707000in}}%
\pgfpathlineto{\pgfqpoint{4.974030in}{1.717519in}}%
\pgfpathlineto{\pgfqpoint{4.981803in}{1.728083in}}%
\pgfpathlineto{\pgfqpoint{4.989571in}{1.738687in}}%
\pgfpathclose%
\pgfusepath{fill}%
\end{pgfscope}%
\begin{pgfscope}%
\pgfpathrectangle{\pgfqpoint{1.150000in}{0.150000in}}{\pgfqpoint{5.700000in}{5.700000in}}%
\pgfusepath{clip}%
\pgfsetbuttcap%
\pgfsetroundjoin%
\definecolor{currentfill}{rgb}{0.271305,0.019942,0.347269}%
\pgfsetfillcolor{currentfill}%
\pgfsetfillopacity{0.700000}%
\pgfsetlinewidth{0.000000pt}%
\definecolor{currentstroke}{rgb}{0.000000,0.000000,0.000000}%
\pgfsetstrokecolor{currentstroke}%
\pgfsetdash{}{0pt}%
\pgfpathmoveto{\pgfqpoint{4.669119in}{1.665338in}}%
\pgfpathlineto{\pgfqpoint{4.683248in}{1.662025in}}%
\pgfpathlineto{\pgfqpoint{4.697384in}{1.658737in}}%
\pgfpathlineto{\pgfqpoint{4.711528in}{1.655473in}}%
\pgfpathlineto{\pgfqpoint{4.725679in}{1.652232in}}%
\pgfpathlineto{\pgfqpoint{4.717834in}{1.643182in}}%
\pgfpathlineto{\pgfqpoint{4.709984in}{1.634247in}}%
\pgfpathlineto{\pgfqpoint{4.702130in}{1.625434in}}%
\pgfpathlineto{\pgfqpoint{4.694272in}{1.616749in}}%
\pgfpathlineto{\pgfqpoint{4.680111in}{1.620234in}}%
\pgfpathlineto{\pgfqpoint{4.665957in}{1.623744in}}%
\pgfpathlineto{\pgfqpoint{4.651810in}{1.627278in}}%
\pgfpathlineto{\pgfqpoint{4.637670in}{1.630836in}}%
\pgfpathlineto{\pgfqpoint{4.645539in}{1.639270in}}%
\pgfpathlineto{\pgfqpoint{4.653404in}{1.647836in}}%
\pgfpathlineto{\pgfqpoint{4.661264in}{1.656527in}}%
\pgfpathlineto{\pgfqpoint{4.669119in}{1.665338in}}%
\pgfpathclose%
\pgfusepath{fill}%
\end{pgfscope}%
\begin{pgfscope}%
\pgfpathrectangle{\pgfqpoint{1.150000in}{0.150000in}}{\pgfqpoint{5.700000in}{5.700000in}}%
\pgfusepath{clip}%
\pgfsetbuttcap%
\pgfsetroundjoin%
\definecolor{currentfill}{rgb}{0.275191,0.194905,0.496005}%
\pgfsetfillcolor{currentfill}%
\pgfsetfillopacity{0.700000}%
\pgfsetlinewidth{0.000000pt}%
\definecolor{currentstroke}{rgb}{0.000000,0.000000,0.000000}%
\pgfsetstrokecolor{currentstroke}%
\pgfsetdash{}{0pt}%
\pgfpathmoveto{\pgfqpoint{3.348674in}{1.979361in}}%
\pgfpathlineto{\pgfqpoint{3.362532in}{1.971736in}}%
\pgfpathlineto{\pgfqpoint{3.376394in}{1.964140in}}%
\pgfpathlineto{\pgfqpoint{3.390260in}{1.956573in}}%
\pgfpathlineto{\pgfqpoint{3.404131in}{1.949035in}}%
\pgfpathlineto{\pgfqpoint{3.395630in}{1.953861in}}%
\pgfpathlineto{\pgfqpoint{3.387112in}{1.959119in}}%
\pgfpathlineto{\pgfqpoint{3.378578in}{1.964817in}}%
\pgfpathlineto{\pgfqpoint{3.370027in}{1.970967in}}%
\pgfpathlineto{\pgfqpoint{3.356122in}{1.978862in}}%
\pgfpathlineto{\pgfqpoint{3.342221in}{1.986787in}}%
\pgfpathlineto{\pgfqpoint{3.328323in}{1.994740in}}%
\pgfpathlineto{\pgfqpoint{3.314430in}{2.002723in}}%
\pgfpathlineto{\pgfqpoint{3.323017in}{1.996210in}}%
\pgfpathlineto{\pgfqpoint{3.331587in}{1.990152in}}%
\pgfpathlineto{\pgfqpoint{3.340139in}{1.984539in}}%
\pgfpathlineto{\pgfqpoint{3.348674in}{1.979361in}}%
\pgfpathclose%
\pgfusepath{fill}%
\end{pgfscope}%
\begin{pgfscope}%
\pgfpathrectangle{\pgfqpoint{1.150000in}{0.150000in}}{\pgfqpoint{5.700000in}{5.700000in}}%
\pgfusepath{clip}%
\pgfsetbuttcap%
\pgfsetroundjoin%
\definecolor{currentfill}{rgb}{0.281412,0.155834,0.469201}%
\pgfsetfillcolor{currentfill}%
\pgfsetfillopacity{0.700000}%
\pgfsetlinewidth{0.000000pt}%
\definecolor{currentstroke}{rgb}{0.000000,0.000000,0.000000}%
\pgfsetstrokecolor{currentstroke}%
\pgfsetdash{}{0pt}%
\pgfpathmoveto{\pgfqpoint{5.486838in}{1.924577in}}%
\pgfpathlineto{\pgfqpoint{5.501237in}{1.923722in}}%
\pgfpathlineto{\pgfqpoint{5.515645in}{1.922892in}}%
\pgfpathlineto{\pgfqpoint{5.530063in}{1.922086in}}%
\pgfpathlineto{\pgfqpoint{5.522442in}{1.910671in}}%
\pgfpathlineto{\pgfqpoint{5.514815in}{1.899193in}}%
\pgfpathlineto{\pgfqpoint{5.507182in}{1.887655in}}%
\pgfpathlineto{\pgfqpoint{5.499542in}{1.876062in}}%
\pgfpathlineto{\pgfqpoint{5.485119in}{1.877007in}}%
\pgfpathlineto{\pgfqpoint{5.470705in}{1.877977in}}%
\pgfpathlineto{\pgfqpoint{5.456300in}{1.878971in}}%
\pgfpathlineto{\pgfqpoint{5.463943in}{1.890456in}}%
\pgfpathlineto{\pgfqpoint{5.471581in}{1.901888in}}%
\pgfpathlineto{\pgfqpoint{5.479213in}{1.913263in}}%
\pgfpathlineto{\pgfqpoint{5.486838in}{1.924577in}}%
\pgfpathclose%
\pgfusepath{fill}%
\end{pgfscope}%
\begin{pgfscope}%
\pgfpathrectangle{\pgfqpoint{1.150000in}{0.150000in}}{\pgfqpoint{5.700000in}{5.700000in}}%
\pgfusepath{clip}%
\pgfsetbuttcap%
\pgfsetroundjoin%
\definecolor{currentfill}{rgb}{0.204903,0.375746,0.553533}%
\pgfsetfillcolor{currentfill}%
\pgfsetfillopacity{0.700000}%
\pgfsetlinewidth{0.000000pt}%
\definecolor{currentstroke}{rgb}{0.000000,0.000000,0.000000}%
\pgfsetstrokecolor{currentstroke}%
\pgfsetdash{}{0pt}%
\pgfpathmoveto{\pgfqpoint{2.725338in}{2.393903in}}%
\pgfpathlineto{\pgfqpoint{2.739146in}{2.384219in}}%
\pgfpathlineto{\pgfqpoint{2.752958in}{2.374573in}}%
\pgfpathlineto{\pgfqpoint{2.766771in}{2.364965in}}%
\pgfpathlineto{\pgfqpoint{2.780587in}{2.355393in}}%
\pgfpathlineto{\pgfqpoint{2.771515in}{2.367812in}}%
\pgfpathlineto{\pgfqpoint{2.762415in}{2.380788in}}%
\pgfpathlineto{\pgfqpoint{2.753288in}{2.394331in}}%
\pgfpathlineto{\pgfqpoint{2.744132in}{2.408454in}}%
\pgfpathlineto{\pgfqpoint{2.730268in}{2.418421in}}%
\pgfpathlineto{\pgfqpoint{2.716406in}{2.428425in}}%
\pgfpathlineto{\pgfqpoint{2.702546in}{2.438467in}}%
\pgfpathlineto{\pgfqpoint{2.688688in}{2.448547in}}%
\pgfpathlineto{\pgfqpoint{2.697894in}{2.434022in}}%
\pgfpathlineto{\pgfqpoint{2.707071in}{2.420081in}}%
\pgfpathlineto{\pgfqpoint{2.716218in}{2.406712in}}%
\pgfpathlineto{\pgfqpoint{2.725338in}{2.393903in}}%
\pgfpathclose%
\pgfusepath{fill}%
\end{pgfscope}%
\begin{pgfscope}%
\pgfpathrectangle{\pgfqpoint{1.150000in}{0.150000in}}{\pgfqpoint{5.700000in}{5.700000in}}%
\pgfusepath{clip}%
\pgfsetbuttcap%
\pgfsetroundjoin%
\definecolor{currentfill}{rgb}{0.143343,0.522773,0.556295}%
\pgfsetfillcolor{currentfill}%
\pgfsetfillopacity{0.700000}%
\pgfsetlinewidth{0.000000pt}%
\definecolor{currentstroke}{rgb}{0.000000,0.000000,0.000000}%
\pgfsetstrokecolor{currentstroke}%
\pgfsetdash{}{0pt}%
\pgfpathmoveto{\pgfqpoint{2.246131in}{2.793142in}}%
\pgfpathlineto{\pgfqpoint{2.259943in}{2.781661in}}%
\pgfpathlineto{\pgfqpoint{2.273755in}{2.770231in}}%
\pgfpathlineto{\pgfqpoint{2.287568in}{2.758850in}}%
\pgfpathlineto{\pgfqpoint{2.301382in}{2.747519in}}%
\pgfpathlineto{\pgfqpoint{2.291767in}{2.765514in}}%
\pgfpathlineto{\pgfqpoint{2.282115in}{2.784149in}}%
\pgfpathlineto{\pgfqpoint{2.272424in}{2.803437in}}%
\pgfpathlineto{\pgfqpoint{2.262695in}{2.823389in}}%
\pgfpathlineto{\pgfqpoint{2.248822in}{2.835142in}}%
\pgfpathlineto{\pgfqpoint{2.234950in}{2.846945in}}%
\pgfpathlineto{\pgfqpoint{2.221078in}{2.858798in}}%
\pgfpathlineto{\pgfqpoint{2.207206in}{2.870702in}}%
\pgfpathlineto{\pgfqpoint{2.216996in}{2.850320in}}%
\pgfpathlineto{\pgfqpoint{2.226747in}{2.830608in}}%
\pgfpathlineto{\pgfqpoint{2.236458in}{2.811553in}}%
\pgfpathlineto{\pgfqpoint{2.246131in}{2.793142in}}%
\pgfpathclose%
\pgfusepath{fill}%
\end{pgfscope}%
\begin{pgfscope}%
\pgfpathrectangle{\pgfqpoint{1.150000in}{0.150000in}}{\pgfqpoint{5.700000in}{5.700000in}}%
\pgfusepath{clip}%
\pgfsetbuttcap%
\pgfsetroundjoin%
\definecolor{currentfill}{rgb}{0.271305,0.019942,0.347269}%
\pgfsetfillcolor{currentfill}%
\pgfsetfillopacity{0.700000}%
\pgfsetlinewidth{0.000000pt}%
\definecolor{currentstroke}{rgb}{0.000000,0.000000,0.000000}%
\pgfsetstrokecolor{currentstroke}%
\pgfsetdash{}{0pt}%
\pgfpathmoveto{\pgfqpoint{4.292694in}{1.655737in}}%
\pgfpathlineto{\pgfqpoint{4.306726in}{1.651161in}}%
\pgfpathlineto{\pgfqpoint{4.320763in}{1.646610in}}%
\pgfpathlineto{\pgfqpoint{4.334807in}{1.642083in}}%
\pgfpathlineto{\pgfqpoint{4.348858in}{1.637582in}}%
\pgfpathlineto{\pgfqpoint{4.340892in}{1.631746in}}%
\pgfpathlineto{\pgfqpoint{4.332920in}{1.626123in}}%
\pgfpathlineto{\pgfqpoint{4.324942in}{1.620719in}}%
\pgfpathlineto{\pgfqpoint{4.316959in}{1.615542in}}%
\pgfpathlineto{\pgfqpoint{4.302892in}{1.620329in}}%
\pgfpathlineto{\pgfqpoint{4.288832in}{1.625140in}}%
\pgfpathlineto{\pgfqpoint{4.274778in}{1.629976in}}%
\pgfpathlineto{\pgfqpoint{4.260730in}{1.634836in}}%
\pgfpathlineto{\pgfqpoint{4.268731in}{1.639723in}}%
\pgfpathlineto{\pgfqpoint{4.276725in}{1.644840in}}%
\pgfpathlineto{\pgfqpoint{4.284713in}{1.650181in}}%
\pgfpathlineto{\pgfqpoint{4.292694in}{1.655737in}}%
\pgfpathclose%
\pgfusepath{fill}%
\end{pgfscope}%
\begin{pgfscope}%
\pgfpathrectangle{\pgfqpoint{1.150000in}{0.150000in}}{\pgfqpoint{5.700000in}{5.700000in}}%
\pgfusepath{clip}%
\pgfsetbuttcap%
\pgfsetroundjoin%
\definecolor{currentfill}{rgb}{0.277018,0.050344,0.375715}%
\pgfsetfillcolor{currentfill}%
\pgfsetfillopacity{0.700000}%
\pgfsetlinewidth{0.000000pt}%
\definecolor{currentstroke}{rgb}{0.000000,0.000000,0.000000}%
\pgfsetstrokecolor{currentstroke}%
\pgfsetdash{}{0pt}%
\pgfpathmoveto{\pgfqpoint{4.901619in}{1.706673in}}%
\pgfpathlineto{\pgfqpoint{4.915820in}{1.704101in}}%
\pgfpathlineto{\pgfqpoint{4.930029in}{1.701554in}}%
\pgfpathlineto{\pgfqpoint{4.944246in}{1.699031in}}%
\pgfpathlineto{\pgfqpoint{4.958471in}{1.696532in}}%
\pgfpathlineto{\pgfqpoint{4.950686in}{1.686119in}}%
\pgfpathlineto{\pgfqpoint{4.942896in}{1.675767in}}%
\pgfpathlineto{\pgfqpoint{4.935102in}{1.665480in}}%
\pgfpathlineto{\pgfqpoint{4.927304in}{1.655265in}}%
\pgfpathlineto{\pgfqpoint{4.913071in}{1.657984in}}%
\pgfpathlineto{\pgfqpoint{4.898846in}{1.660726in}}%
\pgfpathlineto{\pgfqpoint{4.884629in}{1.663493in}}%
\pgfpathlineto{\pgfqpoint{4.870419in}{1.666283in}}%
\pgfpathlineto{\pgfqpoint{4.878225in}{1.676274in}}%
\pgfpathlineto{\pgfqpoint{4.886027in}{1.686339in}}%
\pgfpathlineto{\pgfqpoint{4.893825in}{1.696474in}}%
\pgfpathlineto{\pgfqpoint{4.901619in}{1.706673in}}%
\pgfpathclose%
\pgfusepath{fill}%
\end{pgfscope}%
\begin{pgfscope}%
\pgfpathrectangle{\pgfqpoint{1.150000in}{0.150000in}}{\pgfqpoint{5.700000in}{5.700000in}}%
\pgfusepath{clip}%
\pgfsetbuttcap%
\pgfsetroundjoin%
\definecolor{currentfill}{rgb}{0.273809,0.031497,0.358853}%
\pgfsetfillcolor{currentfill}%
\pgfsetfillopacity{0.700000}%
\pgfsetlinewidth{0.000000pt}%
\definecolor{currentstroke}{rgb}{0.000000,0.000000,0.000000}%
\pgfsetstrokecolor{currentstroke}%
\pgfsetdash{}{0pt}%
\pgfpathmoveto{\pgfqpoint{4.148570in}{1.674613in}}%
\pgfpathlineto{\pgfqpoint{4.162569in}{1.669554in}}%
\pgfpathlineto{\pgfqpoint{4.176574in}{1.664519in}}%
\pgfpathlineto{\pgfqpoint{4.190584in}{1.659510in}}%
\pgfpathlineto{\pgfqpoint{4.204601in}{1.654526in}}%
\pgfpathlineto{\pgfqpoint{4.196577in}{1.650170in}}%
\pgfpathlineto{\pgfqpoint{4.188546in}{1.646064in}}%
\pgfpathlineto{\pgfqpoint{4.180508in}{1.642214in}}%
\pgfpathlineto{\pgfqpoint{4.172463in}{1.638629in}}%
\pgfpathlineto{\pgfqpoint{4.158428in}{1.643911in}}%
\pgfpathlineto{\pgfqpoint{4.144398in}{1.649219in}}%
\pgfpathlineto{\pgfqpoint{4.130375in}{1.654552in}}%
\pgfpathlineto{\pgfqpoint{4.116357in}{1.659909in}}%
\pgfpathlineto{\pgfqpoint{4.124421in}{1.663191in}}%
\pgfpathlineto{\pgfqpoint{4.132478in}{1.666740in}}%
\pgfpathlineto{\pgfqpoint{4.140527in}{1.670550in}}%
\pgfpathlineto{\pgfqpoint{4.148570in}{1.674613in}}%
\pgfpathclose%
\pgfusepath{fill}%
\end{pgfscope}%
\begin{pgfscope}%
\pgfpathrectangle{\pgfqpoint{1.150000in}{0.150000in}}{\pgfqpoint{5.700000in}{5.700000in}}%
\pgfusepath{clip}%
\pgfsetbuttcap%
\pgfsetroundjoin%
\definecolor{currentfill}{rgb}{0.281924,0.089666,0.412415}%
\pgfsetfillcolor{currentfill}%
\pgfsetfillopacity{0.700000}%
\pgfsetlinewidth{0.000000pt}%
\definecolor{currentstroke}{rgb}{0.000000,0.000000,0.000000}%
\pgfsetstrokecolor{currentstroke}%
\pgfsetdash{}{0pt}%
\pgfpathmoveto{\pgfqpoint{3.804414in}{1.768352in}}%
\pgfpathlineto{\pgfqpoint{3.818344in}{1.762171in}}%
\pgfpathlineto{\pgfqpoint{3.832278in}{1.756015in}}%
\pgfpathlineto{\pgfqpoint{3.846218in}{1.749886in}}%
\pgfpathlineto{\pgfqpoint{3.860164in}{1.743783in}}%
\pgfpathlineto{\pgfqpoint{3.851968in}{1.743247in}}%
\pgfpathlineto{\pgfqpoint{3.843762in}{1.743044in}}%
\pgfpathlineto{\pgfqpoint{3.835546in}{1.743180in}}%
\pgfpathlineto{\pgfqpoint{3.827319in}{1.743667in}}%
\pgfpathlineto{\pgfqpoint{3.813348in}{1.750096in}}%
\pgfpathlineto{\pgfqpoint{3.799383in}{1.756552in}}%
\pgfpathlineto{\pgfqpoint{3.785423in}{1.763034in}}%
\pgfpathlineto{\pgfqpoint{3.771467in}{1.769542in}}%
\pgfpathlineto{\pgfqpoint{3.779720in}{1.768724in}}%
\pgfpathlineto{\pgfqpoint{3.787962in}{1.768259in}}%
\pgfpathlineto{\pgfqpoint{3.796194in}{1.768138in}}%
\pgfpathlineto{\pgfqpoint{3.804414in}{1.768352in}}%
\pgfpathclose%
\pgfusepath{fill}%
\end{pgfscope}%
\begin{pgfscope}%
\pgfpathrectangle{\pgfqpoint{1.150000in}{0.150000in}}{\pgfqpoint{5.700000in}{5.700000in}}%
\pgfusepath{clip}%
\pgfsetbuttcap%
\pgfsetroundjoin%
\definecolor{currentfill}{rgb}{0.269944,0.014625,0.341379}%
\pgfsetfillcolor{currentfill}%
\pgfsetfillopacity{0.700000}%
\pgfsetlinewidth{0.000000pt}%
\definecolor{currentstroke}{rgb}{0.000000,0.000000,0.000000}%
\pgfsetstrokecolor{currentstroke}%
\pgfsetdash{}{0pt}%
\pgfpathmoveto{\pgfqpoint{4.436875in}{1.646229in}}%
\pgfpathlineto{\pgfqpoint{4.450944in}{1.642121in}}%
\pgfpathlineto{\pgfqpoint{4.465020in}{1.638038in}}%
\pgfpathlineto{\pgfqpoint{4.479102in}{1.633979in}}%
\pgfpathlineto{\pgfqpoint{4.493191in}{1.629944in}}%
\pgfpathlineto{\pgfqpoint{4.485275in}{1.622797in}}%
\pgfpathlineto{\pgfqpoint{4.477354in}{1.615827in}}%
\pgfpathlineto{\pgfqpoint{4.469428in}{1.609042in}}%
\pgfpathlineto{\pgfqpoint{4.461496in}{1.602448in}}%
\pgfpathlineto{\pgfqpoint{4.447394in}{1.606754in}}%
\pgfpathlineto{\pgfqpoint{4.433297in}{1.611085in}}%
\pgfpathlineto{\pgfqpoint{4.419208in}{1.615440in}}%
\pgfpathlineto{\pgfqpoint{4.405125in}{1.619820in}}%
\pgfpathlineto{\pgfqpoint{4.413070in}{1.626137in}}%
\pgfpathlineto{\pgfqpoint{4.421010in}{1.632649in}}%
\pgfpathlineto{\pgfqpoint{4.428945in}{1.639349in}}%
\pgfpathlineto{\pgfqpoint{4.436875in}{1.646229in}}%
\pgfpathclose%
\pgfusepath{fill}%
\end{pgfscope}%
\begin{pgfscope}%
\pgfpathrectangle{\pgfqpoint{1.150000in}{0.150000in}}{\pgfqpoint{5.700000in}{5.700000in}}%
\pgfusepath{clip}%
\pgfsetbuttcap%
\pgfsetroundjoin%
\definecolor{currentfill}{rgb}{0.283072,0.130895,0.449241}%
\pgfsetfillcolor{currentfill}%
\pgfsetfillopacity{0.700000}%
\pgfsetlinewidth{0.000000pt}%
\definecolor{currentstroke}{rgb}{0.000000,0.000000,0.000000}%
\pgfsetstrokecolor{currentstroke}%
\pgfsetdash{}{0pt}%
\pgfpathmoveto{\pgfqpoint{3.604386in}{1.849721in}}%
\pgfpathlineto{\pgfqpoint{3.618283in}{1.842890in}}%
\pgfpathlineto{\pgfqpoint{3.632185in}{1.836088in}}%
\pgfpathlineto{\pgfqpoint{3.646091in}{1.829312in}}%
\pgfpathlineto{\pgfqpoint{3.660003in}{1.822564in}}%
\pgfpathlineto{\pgfqpoint{3.651683in}{1.824415in}}%
\pgfpathlineto{\pgfqpoint{3.643350in}{1.826645in}}%
\pgfpathlineto{\pgfqpoint{3.635004in}{1.829263in}}%
\pgfpathlineto{\pgfqpoint{3.626645in}{1.832277in}}%
\pgfpathlineto{\pgfqpoint{3.612705in}{1.839367in}}%
\pgfpathlineto{\pgfqpoint{3.598769in}{1.846484in}}%
\pgfpathlineto{\pgfqpoint{3.584837in}{1.853628in}}%
\pgfpathlineto{\pgfqpoint{3.570910in}{1.860800in}}%
\pgfpathlineto{\pgfqpoint{3.579300in}{1.857438in}}%
\pgfpathlineto{\pgfqpoint{3.587675in}{1.854477in}}%
\pgfpathlineto{\pgfqpoint{3.596037in}{1.851908in}}%
\pgfpathlineto{\pgfqpoint{3.604386in}{1.849721in}}%
\pgfpathclose%
\pgfusepath{fill}%
\end{pgfscope}%
\begin{pgfscope}%
\pgfpathrectangle{\pgfqpoint{1.150000in}{0.150000in}}{\pgfqpoint{5.700000in}{5.700000in}}%
\pgfusepath{clip}%
\pgfsetbuttcap%
\pgfsetroundjoin%
\definecolor{currentfill}{rgb}{0.282623,0.140926,0.457517}%
\pgfsetfillcolor{currentfill}%
\pgfsetfillopacity{0.700000}%
\pgfsetlinewidth{0.000000pt}%
\definecolor{currentstroke}{rgb}{0.000000,0.000000,0.000000}%
\pgfsetstrokecolor{currentstroke}%
\pgfsetdash{}{0pt}%
\pgfpathmoveto{\pgfqpoint{5.398769in}{1.883192in}}%
\pgfpathlineto{\pgfqpoint{5.413138in}{1.882101in}}%
\pgfpathlineto{\pgfqpoint{5.427516in}{1.881033in}}%
\pgfpathlineto{\pgfqpoint{5.441903in}{1.879990in}}%
\pgfpathlineto{\pgfqpoint{5.456300in}{1.878971in}}%
\pgfpathlineto{\pgfqpoint{5.448650in}{1.867436in}}%
\pgfpathlineto{\pgfqpoint{5.440994in}{1.855854in}}%
\pgfpathlineto{\pgfqpoint{5.433333in}{1.844228in}}%
\pgfpathlineto{\pgfqpoint{5.425666in}{1.832562in}}%
\pgfpathlineto{\pgfqpoint{5.411264in}{1.833734in}}%
\pgfpathlineto{\pgfqpoint{5.396871in}{1.834930in}}%
\pgfpathlineto{\pgfqpoint{5.382487in}{1.836151in}}%
\pgfpathlineto{\pgfqpoint{5.368112in}{1.837396in}}%
\pgfpathlineto{\pgfqpoint{5.375785in}{1.848903in}}%
\pgfpathlineto{\pgfqpoint{5.383452in}{1.860374in}}%
\pgfpathlineto{\pgfqpoint{5.391113in}{1.871805in}}%
\pgfpathlineto{\pgfqpoint{5.398769in}{1.883192in}}%
\pgfpathclose%
\pgfusepath{fill}%
\end{pgfscope}%
\begin{pgfscope}%
\pgfpathrectangle{\pgfqpoint{1.150000in}{0.150000in}}{\pgfqpoint{5.700000in}{5.700000in}}%
\pgfusepath{clip}%
\pgfsetbuttcap%
\pgfsetroundjoin%
\definecolor{currentfill}{rgb}{0.252194,0.269783,0.531579}%
\pgfsetfillcolor{currentfill}%
\pgfsetfillopacity{0.700000}%
\pgfsetlinewidth{0.000000pt}%
\definecolor{currentstroke}{rgb}{0.000000,0.000000,0.000000}%
\pgfsetstrokecolor{currentstroke}%
\pgfsetdash{}{0pt}%
\pgfpathmoveto{\pgfqpoint{3.092644in}{2.134565in}}%
\pgfpathlineto{\pgfqpoint{3.106479in}{2.126091in}}%
\pgfpathlineto{\pgfqpoint{3.120317in}{2.117649in}}%
\pgfpathlineto{\pgfqpoint{3.134159in}{2.109240in}}%
\pgfpathlineto{\pgfqpoint{3.148004in}{2.100861in}}%
\pgfpathlineto{\pgfqpoint{3.139285in}{2.108944in}}%
\pgfpathlineto{\pgfqpoint{3.130545in}{2.117515in}}%
\pgfpathlineto{\pgfqpoint{3.121784in}{2.126585in}}%
\pgfpathlineto{\pgfqpoint{3.113001in}{2.136163in}}%
\pgfpathlineto{\pgfqpoint{3.099115in}{2.144916in}}%
\pgfpathlineto{\pgfqpoint{3.085233in}{2.153701in}}%
\pgfpathlineto{\pgfqpoint{3.071354in}{2.162518in}}%
\pgfpathlineto{\pgfqpoint{3.057478in}{2.171367in}}%
\pgfpathlineto{\pgfqpoint{3.066302in}{2.161407in}}%
\pgfpathlineto{\pgfqpoint{3.075105in}{2.151961in}}%
\pgfpathlineto{\pgfqpoint{3.083885in}{2.143017in}}%
\pgfpathlineto{\pgfqpoint{3.092644in}{2.134565in}}%
\pgfpathclose%
\pgfusepath{fill}%
\end{pgfscope}%
\begin{pgfscope}%
\pgfpathrectangle{\pgfqpoint{1.150000in}{0.150000in}}{\pgfqpoint{5.700000in}{5.700000in}}%
\pgfusepath{clip}%
\pgfsetbuttcap%
\pgfsetroundjoin%
\definecolor{currentfill}{rgb}{0.283229,0.120777,0.440584}%
\pgfsetfillcolor{currentfill}%
\pgfsetfillopacity{0.700000}%
\pgfsetlinewidth{0.000000pt}%
\definecolor{currentstroke}{rgb}{0.000000,0.000000,0.000000}%
\pgfsetstrokecolor{currentstroke}%
\pgfsetdash{}{0pt}%
\pgfpathmoveto{\pgfqpoint{5.310700in}{1.842619in}}%
\pgfpathlineto{\pgfqpoint{5.325039in}{1.841277in}}%
\pgfpathlineto{\pgfqpoint{5.339388in}{1.839959in}}%
\pgfpathlineto{\pgfqpoint{5.353746in}{1.838665in}}%
\pgfpathlineto{\pgfqpoint{5.368112in}{1.837396in}}%
\pgfpathlineto{\pgfqpoint{5.360434in}{1.825856in}}%
\pgfpathlineto{\pgfqpoint{5.352750in}{1.814287in}}%
\pgfpathlineto{\pgfqpoint{5.345061in}{1.802693in}}%
\pgfpathlineto{\pgfqpoint{5.337367in}{1.791077in}}%
\pgfpathlineto{\pgfqpoint{5.322995in}{1.792513in}}%
\pgfpathlineto{\pgfqpoint{5.308632in}{1.793974in}}%
\pgfpathlineto{\pgfqpoint{5.294277in}{1.795458in}}%
\pgfpathlineto{\pgfqpoint{5.279931in}{1.796967in}}%
\pgfpathlineto{\pgfqpoint{5.287631in}{1.808411in}}%
\pgfpathlineto{\pgfqpoint{5.295326in}{1.819836in}}%
\pgfpathlineto{\pgfqpoint{5.303015in}{1.831240in}}%
\pgfpathlineto{\pgfqpoint{5.310700in}{1.842619in}}%
\pgfpathclose%
\pgfusepath{fill}%
\end{pgfscope}%
\begin{pgfscope}%
\pgfpathrectangle{\pgfqpoint{1.150000in}{0.150000in}}{\pgfqpoint{5.700000in}{5.700000in}}%
\pgfusepath{clip}%
\pgfsetbuttcap%
\pgfsetroundjoin%
\definecolor{currentfill}{rgb}{0.149039,0.508051,0.557250}%
\pgfsetfillcolor{currentfill}%
\pgfsetfillopacity{0.700000}%
\pgfsetlinewidth{0.000000pt}%
\definecolor{currentstroke}{rgb}{0.000000,0.000000,0.000000}%
\pgfsetstrokecolor{currentstroke}%
\pgfsetdash{}{0pt}%
\pgfpathmoveto{\pgfqpoint{2.301382in}{2.747519in}}%
\pgfpathlineto{\pgfqpoint{2.315197in}{2.736237in}}%
\pgfpathlineto{\pgfqpoint{2.329013in}{2.725004in}}%
\pgfpathlineto{\pgfqpoint{2.342829in}{2.713818in}}%
\pgfpathlineto{\pgfqpoint{2.356647in}{2.702680in}}%
\pgfpathlineto{\pgfqpoint{2.347088in}{2.720261in}}%
\pgfpathlineto{\pgfqpoint{2.337494in}{2.738477in}}%
\pgfpathlineto{\pgfqpoint{2.327862in}{2.757341in}}%
\pgfpathlineto{\pgfqpoint{2.318192in}{2.776865in}}%
\pgfpathlineto{\pgfqpoint{2.304317in}{2.788424in}}%
\pgfpathlineto{\pgfqpoint{2.290442in}{2.800030in}}%
\pgfpathlineto{\pgfqpoint{2.276568in}{2.811685in}}%
\pgfpathlineto{\pgfqpoint{2.262695in}{2.823389in}}%
\pgfpathlineto{\pgfqpoint{2.272424in}{2.803437in}}%
\pgfpathlineto{\pgfqpoint{2.282115in}{2.784149in}}%
\pgfpathlineto{\pgfqpoint{2.291767in}{2.765514in}}%
\pgfpathlineto{\pgfqpoint{2.301382in}{2.747519in}}%
\pgfpathclose%
\pgfusepath{fill}%
\end{pgfscope}%
\begin{pgfscope}%
\pgfpathrectangle{\pgfqpoint{1.150000in}{0.150000in}}{\pgfqpoint{5.700000in}{5.700000in}}%
\pgfusepath{clip}%
\pgfsetbuttcap%
\pgfsetroundjoin%
\definecolor{currentfill}{rgb}{0.277941,0.056324,0.381191}%
\pgfsetfillcolor{currentfill}%
\pgfsetfillopacity{0.700000}%
\pgfsetlinewidth{0.000000pt}%
\definecolor{currentstroke}{rgb}{0.000000,0.000000,0.000000}%
\pgfsetstrokecolor{currentstroke}%
\pgfsetdash{}{0pt}%
\pgfpathmoveto{\pgfqpoint{4.004422in}{1.703675in}}%
\pgfpathlineto{\pgfqpoint{4.018394in}{1.698116in}}%
\pgfpathlineto{\pgfqpoint{4.032372in}{1.692582in}}%
\pgfpathlineto{\pgfqpoint{4.046355in}{1.687073in}}%
\pgfpathlineto{\pgfqpoint{4.060344in}{1.681590in}}%
\pgfpathlineto{\pgfqpoint{4.052251in}{1.678892in}}%
\pgfpathlineto{\pgfqpoint{4.044151in}{1.676481in}}%
\pgfpathlineto{\pgfqpoint{4.036043in}{1.674365in}}%
\pgfpathlineto{\pgfqpoint{4.027926in}{1.672554in}}%
\pgfpathlineto{\pgfqpoint{4.013916in}{1.678350in}}%
\pgfpathlineto{\pgfqpoint{3.999911in}{1.684170in}}%
\pgfpathlineto{\pgfqpoint{3.985912in}{1.690016in}}%
\pgfpathlineto{\pgfqpoint{3.971918in}{1.695888in}}%
\pgfpathlineto{\pgfqpoint{3.980057in}{1.697382in}}%
\pgfpathlineto{\pgfqpoint{3.988187in}{1.699183in}}%
\pgfpathlineto{\pgfqpoint{3.996309in}{1.701283in}}%
\pgfpathlineto{\pgfqpoint{4.004422in}{1.703675in}}%
\pgfpathclose%
\pgfusepath{fill}%
\end{pgfscope}%
\begin{pgfscope}%
\pgfpathrectangle{\pgfqpoint{1.150000in}{0.150000in}}{\pgfqpoint{5.700000in}{5.700000in}}%
\pgfusepath{clip}%
\pgfsetbuttcap%
\pgfsetroundjoin%
\definecolor{currentfill}{rgb}{0.271305,0.019942,0.347269}%
\pgfsetfillcolor{currentfill}%
\pgfsetfillopacity{0.700000}%
\pgfsetlinewidth{0.000000pt}%
\definecolor{currentstroke}{rgb}{0.000000,0.000000,0.000000}%
\pgfsetstrokecolor{currentstroke}%
\pgfsetdash{}{0pt}%
\pgfpathmoveto{\pgfqpoint{4.581183in}{1.645310in}}%
\pgfpathlineto{\pgfqpoint{4.595294in}{1.641655in}}%
\pgfpathlineto{\pgfqpoint{4.609412in}{1.638024in}}%
\pgfpathlineto{\pgfqpoint{4.623538in}{1.634418in}}%
\pgfpathlineto{\pgfqpoint{4.637670in}{1.630836in}}%
\pgfpathlineto{\pgfqpoint{4.629797in}{1.622539in}}%
\pgfpathlineto{\pgfqpoint{4.621920in}{1.614387in}}%
\pgfpathlineto{\pgfqpoint{4.614038in}{1.606385in}}%
\pgfpathlineto{\pgfqpoint{4.606151in}{1.598540in}}%
\pgfpathlineto{\pgfqpoint{4.592007in}{1.602381in}}%
\pgfpathlineto{\pgfqpoint{4.577870in}{1.606246in}}%
\pgfpathlineto{\pgfqpoint{4.563739in}{1.610135in}}%
\pgfpathlineto{\pgfqpoint{4.549616in}{1.614048in}}%
\pgfpathlineto{\pgfqpoint{4.557515in}{1.621629in}}%
\pgfpathlineto{\pgfqpoint{4.565409in}{1.629371in}}%
\pgfpathlineto{\pgfqpoint{4.573298in}{1.637266in}}%
\pgfpathlineto{\pgfqpoint{4.581183in}{1.645310in}}%
\pgfpathclose%
\pgfusepath{fill}%
\end{pgfscope}%
\begin{pgfscope}%
\pgfpathrectangle{\pgfqpoint{1.150000in}{0.150000in}}{\pgfqpoint{5.700000in}{5.700000in}}%
\pgfusepath{clip}%
\pgfsetbuttcap%
\pgfsetroundjoin%
\definecolor{currentfill}{rgb}{0.274952,0.037752,0.364543}%
\pgfsetfillcolor{currentfill}%
\pgfsetfillopacity{0.700000}%
\pgfsetlinewidth{0.000000pt}%
\definecolor{currentstroke}{rgb}{0.000000,0.000000,0.000000}%
\pgfsetstrokecolor{currentstroke}%
\pgfsetdash{}{0pt}%
\pgfpathmoveto{\pgfqpoint{4.813658in}{1.677687in}}%
\pgfpathlineto{\pgfqpoint{4.827837in}{1.674800in}}%
\pgfpathlineto{\pgfqpoint{4.842023in}{1.671937in}}%
\pgfpathlineto{\pgfqpoint{4.856217in}{1.669098in}}%
\pgfpathlineto{\pgfqpoint{4.870419in}{1.666283in}}%
\pgfpathlineto{\pgfqpoint{4.862609in}{1.656373in}}%
\pgfpathlineto{\pgfqpoint{4.854795in}{1.646548in}}%
\pgfpathlineto{\pgfqpoint{4.846976in}{1.636815in}}%
\pgfpathlineto{\pgfqpoint{4.839154in}{1.627180in}}%
\pgfpathlineto{\pgfqpoint{4.824944in}{1.630227in}}%
\pgfpathlineto{\pgfqpoint{4.810740in}{1.633299in}}%
\pgfpathlineto{\pgfqpoint{4.796545in}{1.636394in}}%
\pgfpathlineto{\pgfqpoint{4.782357in}{1.639513in}}%
\pgfpathlineto{\pgfqpoint{4.790188in}{1.648912in}}%
\pgfpathlineto{\pgfqpoint{4.798016in}{1.658410in}}%
\pgfpathlineto{\pgfqpoint{4.805839in}{1.668004in}}%
\pgfpathlineto{\pgfqpoint{4.813658in}{1.677687in}}%
\pgfpathclose%
\pgfusepath{fill}%
\end{pgfscope}%
\begin{pgfscope}%
\pgfpathrectangle{\pgfqpoint{1.150000in}{0.150000in}}{\pgfqpoint{5.700000in}{5.700000in}}%
\pgfusepath{clip}%
\pgfsetbuttcap%
\pgfsetroundjoin%
\definecolor{currentfill}{rgb}{0.282910,0.105393,0.426902}%
\pgfsetfillcolor{currentfill}%
\pgfsetfillopacity{0.700000}%
\pgfsetlinewidth{0.000000pt}%
\definecolor{currentstroke}{rgb}{0.000000,0.000000,0.000000}%
\pgfsetstrokecolor{currentstroke}%
\pgfsetdash{}{0pt}%
\pgfpathmoveto{\pgfqpoint{5.222634in}{1.803245in}}%
\pgfpathlineto{\pgfqpoint{5.236945in}{1.801639in}}%
\pgfpathlineto{\pgfqpoint{5.251265in}{1.800057in}}%
\pgfpathlineto{\pgfqpoint{5.265594in}{1.798500in}}%
\pgfpathlineto{\pgfqpoint{5.279931in}{1.796967in}}%
\pgfpathlineto{\pgfqpoint{5.272226in}{1.785510in}}%
\pgfpathlineto{\pgfqpoint{5.264516in}{1.774043in}}%
\pgfpathlineto{\pgfqpoint{5.256802in}{1.762571in}}%
\pgfpathlineto{\pgfqpoint{5.249082in}{1.751097in}}%
\pgfpathlineto{\pgfqpoint{5.234739in}{1.752811in}}%
\pgfpathlineto{\pgfqpoint{5.220404in}{1.754548in}}%
\pgfpathlineto{\pgfqpoint{5.206078in}{1.756309in}}%
\pgfpathlineto{\pgfqpoint{5.191761in}{1.758095in}}%
\pgfpathlineto{\pgfqpoint{5.199486in}{1.769383in}}%
\pgfpathlineto{\pgfqpoint{5.207207in}{1.780674in}}%
\pgfpathlineto{\pgfqpoint{5.214923in}{1.791962in}}%
\pgfpathlineto{\pgfqpoint{5.222634in}{1.803245in}}%
\pgfpathclose%
\pgfusepath{fill}%
\end{pgfscope}%
\begin{pgfscope}%
\pgfpathrectangle{\pgfqpoint{1.150000in}{0.150000in}}{\pgfqpoint{5.700000in}{5.700000in}}%
\pgfusepath{clip}%
\pgfsetbuttcap%
\pgfsetroundjoin%
\definecolor{currentfill}{rgb}{0.210503,0.363727,0.552206}%
\pgfsetfillcolor{currentfill}%
\pgfsetfillopacity{0.700000}%
\pgfsetlinewidth{0.000000pt}%
\definecolor{currentstroke}{rgb}{0.000000,0.000000,0.000000}%
\pgfsetstrokecolor{currentstroke}%
\pgfsetdash{}{0pt}%
\pgfpathmoveto{\pgfqpoint{2.780587in}{2.355393in}}%
\pgfpathlineto{\pgfqpoint{2.794405in}{2.345858in}}%
\pgfpathlineto{\pgfqpoint{2.808226in}{2.336360in}}%
\pgfpathlineto{\pgfqpoint{2.822050in}{2.326898in}}%
\pgfpathlineto{\pgfqpoint{2.835876in}{2.317473in}}%
\pgfpathlineto{\pgfqpoint{2.826850in}{2.329503in}}%
\pgfpathlineto{\pgfqpoint{2.817798in}{2.342086in}}%
\pgfpathlineto{\pgfqpoint{2.808719in}{2.355233in}}%
\pgfpathlineto{\pgfqpoint{2.799612in}{2.368954in}}%
\pgfpathlineto{\pgfqpoint{2.785739in}{2.378775in}}%
\pgfpathlineto{\pgfqpoint{2.771867in}{2.388631in}}%
\pgfpathlineto{\pgfqpoint{2.757999in}{2.398524in}}%
\pgfpathlineto{\pgfqpoint{2.744132in}{2.408454in}}%
\pgfpathlineto{\pgfqpoint{2.753288in}{2.394331in}}%
\pgfpathlineto{\pgfqpoint{2.762415in}{2.380788in}}%
\pgfpathlineto{\pgfqpoint{2.771515in}{2.367812in}}%
\pgfpathlineto{\pgfqpoint{2.780587in}{2.355393in}}%
\pgfpathclose%
\pgfusepath{fill}%
\end{pgfscope}%
\begin{pgfscope}%
\pgfpathrectangle{\pgfqpoint{1.150000in}{0.150000in}}{\pgfqpoint{5.700000in}{5.700000in}}%
\pgfusepath{clip}%
\pgfsetbuttcap%
\pgfsetroundjoin%
\definecolor{currentfill}{rgb}{0.277134,0.185228,0.489898}%
\pgfsetfillcolor{currentfill}%
\pgfsetfillopacity{0.700000}%
\pgfsetlinewidth{0.000000pt}%
\definecolor{currentstroke}{rgb}{0.000000,0.000000,0.000000}%
\pgfsetstrokecolor{currentstroke}%
\pgfsetdash{}{0pt}%
\pgfpathmoveto{\pgfqpoint{3.404131in}{1.949035in}}%
\pgfpathlineto{\pgfqpoint{3.418006in}{1.941527in}}%
\pgfpathlineto{\pgfqpoint{3.431884in}{1.934046in}}%
\pgfpathlineto{\pgfqpoint{3.445767in}{1.926595in}}%
\pgfpathlineto{\pgfqpoint{3.459655in}{1.919172in}}%
\pgfpathlineto{\pgfqpoint{3.451187in}{1.923646in}}%
\pgfpathlineto{\pgfqpoint{3.442703in}{1.928548in}}%
\pgfpathlineto{\pgfqpoint{3.434204in}{1.933888in}}%
\pgfpathlineto{\pgfqpoint{3.425688in}{1.939675in}}%
\pgfpathlineto{\pgfqpoint{3.411767in}{1.947455in}}%
\pgfpathlineto{\pgfqpoint{3.397849in}{1.955264in}}%
\pgfpathlineto{\pgfqpoint{3.383936in}{1.963101in}}%
\pgfpathlineto{\pgfqpoint{3.370027in}{1.970967in}}%
\pgfpathlineto{\pgfqpoint{3.378578in}{1.964817in}}%
\pgfpathlineto{\pgfqpoint{3.387112in}{1.959119in}}%
\pgfpathlineto{\pgfqpoint{3.395630in}{1.953861in}}%
\pgfpathlineto{\pgfqpoint{3.404131in}{1.949035in}}%
\pgfpathclose%
\pgfusepath{fill}%
\end{pgfscope}%
\begin{pgfscope}%
\pgfpathrectangle{\pgfqpoint{1.150000in}{0.150000in}}{\pgfqpoint{5.700000in}{5.700000in}}%
\pgfusepath{clip}%
\pgfsetbuttcap%
\pgfsetroundjoin%
\definecolor{currentfill}{rgb}{0.281446,0.084320,0.407414}%
\pgfsetfillcolor{currentfill}%
\pgfsetfillopacity{0.700000}%
\pgfsetlinewidth{0.000000pt}%
\definecolor{currentstroke}{rgb}{0.000000,0.000000,0.000000}%
\pgfsetstrokecolor{currentstroke}%
\pgfsetdash{}{0pt}%
\pgfpathmoveto{\pgfqpoint{5.134575in}{1.765480in}}%
\pgfpathlineto{\pgfqpoint{5.148858in}{1.763597in}}%
\pgfpathlineto{\pgfqpoint{5.163151in}{1.761739in}}%
\pgfpathlineto{\pgfqpoint{5.177452in}{1.759905in}}%
\pgfpathlineto{\pgfqpoint{5.191761in}{1.758095in}}%
\pgfpathlineto{\pgfqpoint{5.184031in}{1.746814in}}%
\pgfpathlineto{\pgfqpoint{5.176296in}{1.735544in}}%
\pgfpathlineto{\pgfqpoint{5.168557in}{1.724289in}}%
\pgfpathlineto{\pgfqpoint{5.160813in}{1.713055in}}%
\pgfpathlineto{\pgfqpoint{5.146497in}{1.715059in}}%
\pgfpathlineto{\pgfqpoint{5.132190in}{1.717086in}}%
\pgfpathlineto{\pgfqpoint{5.117892in}{1.719138in}}%
\pgfpathlineto{\pgfqpoint{5.103601in}{1.721213in}}%
\pgfpathlineto{\pgfqpoint{5.111351in}{1.732249in}}%
\pgfpathlineto{\pgfqpoint{5.119097in}{1.743308in}}%
\pgfpathlineto{\pgfqpoint{5.126838in}{1.754387in}}%
\pgfpathlineto{\pgfqpoint{5.134575in}{1.765480in}}%
\pgfpathclose%
\pgfusepath{fill}%
\end{pgfscope}%
\begin{pgfscope}%
\pgfpathrectangle{\pgfqpoint{1.150000in}{0.150000in}}{\pgfqpoint{5.700000in}{5.700000in}}%
\pgfusepath{clip}%
\pgfsetbuttcap%
\pgfsetroundjoin%
\definecolor{currentfill}{rgb}{0.154815,0.493313,0.557840}%
\pgfsetfillcolor{currentfill}%
\pgfsetfillopacity{0.700000}%
\pgfsetlinewidth{0.000000pt}%
\definecolor{currentstroke}{rgb}{0.000000,0.000000,0.000000}%
\pgfsetstrokecolor{currentstroke}%
\pgfsetdash{}{0pt}%
\pgfpathmoveto{\pgfqpoint{2.356647in}{2.702680in}}%
\pgfpathlineto{\pgfqpoint{2.370465in}{2.691589in}}%
\pgfpathlineto{\pgfqpoint{2.384285in}{2.680545in}}%
\pgfpathlineto{\pgfqpoint{2.398106in}{2.669547in}}%
\pgfpathlineto{\pgfqpoint{2.411928in}{2.658594in}}%
\pgfpathlineto{\pgfqpoint{2.402425in}{2.675763in}}%
\pgfpathlineto{\pgfqpoint{2.392887in}{2.693561in}}%
\pgfpathlineto{\pgfqpoint{2.383313in}{2.712003in}}%
\pgfpathlineto{\pgfqpoint{2.373703in}{2.731100in}}%
\pgfpathlineto{\pgfqpoint{2.359824in}{2.742472in}}%
\pgfpathlineto{\pgfqpoint{2.345946in}{2.753889in}}%
\pgfpathlineto{\pgfqpoint{2.332068in}{2.765354in}}%
\pgfpathlineto{\pgfqpoint{2.318192in}{2.776865in}}%
\pgfpathlineto{\pgfqpoint{2.327862in}{2.757341in}}%
\pgfpathlineto{\pgfqpoint{2.337494in}{2.738477in}}%
\pgfpathlineto{\pgfqpoint{2.347088in}{2.720261in}}%
\pgfpathlineto{\pgfqpoint{2.356647in}{2.702680in}}%
\pgfpathclose%
\pgfusepath{fill}%
\end{pgfscope}%
\begin{pgfscope}%
\pgfpathrectangle{\pgfqpoint{1.150000in}{0.150000in}}{\pgfqpoint{5.700000in}{5.700000in}}%
\pgfusepath{clip}%
\pgfsetbuttcap%
\pgfsetroundjoin%
\definecolor{currentfill}{rgb}{0.257322,0.256130,0.526563}%
\pgfsetfillcolor{currentfill}%
\pgfsetfillopacity{0.700000}%
\pgfsetlinewidth{0.000000pt}%
\definecolor{currentstroke}{rgb}{0.000000,0.000000,0.000000}%
\pgfsetstrokecolor{currentstroke}%
\pgfsetdash{}{0pt}%
\pgfpathmoveto{\pgfqpoint{3.148004in}{2.100861in}}%
\pgfpathlineto{\pgfqpoint{3.161853in}{2.092514in}}%
\pgfpathlineto{\pgfqpoint{3.175705in}{2.084199in}}%
\pgfpathlineto{\pgfqpoint{3.189561in}{2.075914in}}%
\pgfpathlineto{\pgfqpoint{3.203421in}{2.067660in}}%
\pgfpathlineto{\pgfqpoint{3.194740in}{2.075375in}}%
\pgfpathlineto{\pgfqpoint{3.186040in}{2.083574in}}%
\pgfpathlineto{\pgfqpoint{3.177319in}{2.092266in}}%
\pgfpathlineto{\pgfqpoint{3.168577in}{2.101464in}}%
\pgfpathlineto{\pgfqpoint{3.154678in}{2.110092in}}%
\pgfpathlineto{\pgfqpoint{3.140782in}{2.118752in}}%
\pgfpathlineto{\pgfqpoint{3.126890in}{2.127442in}}%
\pgfpathlineto{\pgfqpoint{3.113001in}{2.136163in}}%
\pgfpathlineto{\pgfqpoint{3.121784in}{2.126585in}}%
\pgfpathlineto{\pgfqpoint{3.130545in}{2.117515in}}%
\pgfpathlineto{\pgfqpoint{3.139285in}{2.108944in}}%
\pgfpathlineto{\pgfqpoint{3.148004in}{2.100861in}}%
\pgfpathclose%
\pgfusepath{fill}%
\end{pgfscope}%
\begin{pgfscope}%
\pgfpathrectangle{\pgfqpoint{1.150000in}{0.150000in}}{\pgfqpoint{5.700000in}{5.700000in}}%
\pgfusepath{clip}%
\pgfsetbuttcap%
\pgfsetroundjoin%
\definecolor{currentfill}{rgb}{0.279566,0.067836,0.391917}%
\pgfsetfillcolor{currentfill}%
\pgfsetfillopacity{0.700000}%
\pgfsetlinewidth{0.000000pt}%
\definecolor{currentstroke}{rgb}{0.000000,0.000000,0.000000}%
\pgfsetstrokecolor{currentstroke}%
\pgfsetdash{}{0pt}%
\pgfpathmoveto{\pgfqpoint{5.046521in}{1.729757in}}%
\pgfpathlineto{\pgfqpoint{5.060779in}{1.727585in}}%
\pgfpathlineto{\pgfqpoint{5.075045in}{1.725437in}}%
\pgfpathlineto{\pgfqpoint{5.089319in}{1.723313in}}%
\pgfpathlineto{\pgfqpoint{5.103601in}{1.721213in}}%
\pgfpathlineto{\pgfqpoint{5.095847in}{1.710206in}}%
\pgfpathlineto{\pgfqpoint{5.088088in}{1.699233in}}%
\pgfpathlineto{\pgfqpoint{5.080325in}{1.688298in}}%
\pgfpathlineto{\pgfqpoint{5.072558in}{1.677407in}}%
\pgfpathlineto{\pgfqpoint{5.058269in}{1.679714in}}%
\pgfpathlineto{\pgfqpoint{5.043988in}{1.682044in}}%
\pgfpathlineto{\pgfqpoint{5.029715in}{1.684398in}}%
\pgfpathlineto{\pgfqpoint{5.015450in}{1.686777in}}%
\pgfpathlineto{\pgfqpoint{5.023224in}{1.697457in}}%
\pgfpathlineto{\pgfqpoint{5.030994in}{1.708183in}}%
\pgfpathlineto{\pgfqpoint{5.038760in}{1.718952in}}%
\pgfpathlineto{\pgfqpoint{5.046521in}{1.729757in}}%
\pgfpathclose%
\pgfusepath{fill}%
\end{pgfscope}%
\begin{pgfscope}%
\pgfpathrectangle{\pgfqpoint{1.150000in}{0.150000in}}{\pgfqpoint{5.700000in}{5.700000in}}%
\pgfusepath{clip}%
\pgfsetbuttcap%
\pgfsetroundjoin%
\definecolor{currentfill}{rgb}{0.281446,0.084320,0.407414}%
\pgfsetfillcolor{currentfill}%
\pgfsetfillopacity{0.700000}%
\pgfsetlinewidth{0.000000pt}%
\definecolor{currentstroke}{rgb}{0.000000,0.000000,0.000000}%
\pgfsetstrokecolor{currentstroke}%
\pgfsetdash{}{0pt}%
\pgfpathmoveto{\pgfqpoint{3.860164in}{1.743783in}}%
\pgfpathlineto{\pgfqpoint{3.874114in}{1.737705in}}%
\pgfpathlineto{\pgfqpoint{3.888070in}{1.731654in}}%
\pgfpathlineto{\pgfqpoint{3.902031in}{1.725629in}}%
\pgfpathlineto{\pgfqpoint{3.915998in}{1.719629in}}%
\pgfpathlineto{\pgfqpoint{3.907826in}{1.718773in}}%
\pgfpathlineto{\pgfqpoint{3.899645in}{1.718245in}}%
\pgfpathlineto{\pgfqpoint{3.891454in}{1.718053in}}%
\pgfpathlineto{\pgfqpoint{3.883252in}{1.718208in}}%
\pgfpathlineto{\pgfqpoint{3.869261in}{1.724534in}}%
\pgfpathlineto{\pgfqpoint{3.855275in}{1.730885in}}%
\pgfpathlineto{\pgfqpoint{3.841295in}{1.737263in}}%
\pgfpathlineto{\pgfqpoint{3.827319in}{1.743667in}}%
\pgfpathlineto{\pgfqpoint{3.835546in}{1.743180in}}%
\pgfpathlineto{\pgfqpoint{3.843762in}{1.743044in}}%
\pgfpathlineto{\pgfqpoint{3.851968in}{1.743247in}}%
\pgfpathlineto{\pgfqpoint{3.860164in}{1.743783in}}%
\pgfpathclose%
\pgfusepath{fill}%
\end{pgfscope}%
\begin{pgfscope}%
\pgfpathrectangle{\pgfqpoint{1.150000in}{0.150000in}}{\pgfqpoint{5.700000in}{5.700000in}}%
\pgfusepath{clip}%
\pgfsetbuttcap%
\pgfsetroundjoin%
\definecolor{currentfill}{rgb}{0.283187,0.125848,0.444960}%
\pgfsetfillcolor{currentfill}%
\pgfsetfillopacity{0.700000}%
\pgfsetlinewidth{0.000000pt}%
\definecolor{currentstroke}{rgb}{0.000000,0.000000,0.000000}%
\pgfsetstrokecolor{currentstroke}%
\pgfsetdash{}{0pt}%
\pgfpathmoveto{\pgfqpoint{3.660003in}{1.822564in}}%
\pgfpathlineto{\pgfqpoint{3.673919in}{1.815842in}}%
\pgfpathlineto{\pgfqpoint{3.687839in}{1.809148in}}%
\pgfpathlineto{\pgfqpoint{3.701765in}{1.802480in}}%
\pgfpathlineto{\pgfqpoint{3.715696in}{1.795839in}}%
\pgfpathlineto{\pgfqpoint{3.707404in}{1.797355in}}%
\pgfpathlineto{\pgfqpoint{3.699100in}{1.799246in}}%
\pgfpathlineto{\pgfqpoint{3.690784in}{1.801520in}}%
\pgfpathlineto{\pgfqpoint{3.682455in}{1.804188in}}%
\pgfpathlineto{\pgfqpoint{3.668495in}{1.811170in}}%
\pgfpathlineto{\pgfqpoint{3.654541in}{1.818179in}}%
\pgfpathlineto{\pgfqpoint{3.640591in}{1.825215in}}%
\pgfpathlineto{\pgfqpoint{3.626645in}{1.832277in}}%
\pgfpathlineto{\pgfqpoint{3.635004in}{1.829263in}}%
\pgfpathlineto{\pgfqpoint{3.643350in}{1.826645in}}%
\pgfpathlineto{\pgfqpoint{3.651683in}{1.824415in}}%
\pgfpathlineto{\pgfqpoint{3.660003in}{1.822564in}}%
\pgfpathclose%
\pgfusepath{fill}%
\end{pgfscope}%
\begin{pgfscope}%
\pgfpathrectangle{\pgfqpoint{1.150000in}{0.150000in}}{\pgfqpoint{5.700000in}{5.700000in}}%
\pgfusepath{clip}%
\pgfsetbuttcap%
\pgfsetroundjoin%
\definecolor{currentfill}{rgb}{0.272594,0.025563,0.353093}%
\pgfsetfillcolor{currentfill}%
\pgfsetfillopacity{0.700000}%
\pgfsetlinewidth{0.000000pt}%
\definecolor{currentstroke}{rgb}{0.000000,0.000000,0.000000}%
\pgfsetstrokecolor{currentstroke}%
\pgfsetdash{}{0pt}%
\pgfpathmoveto{\pgfqpoint{4.725679in}{1.652232in}}%
\pgfpathlineto{\pgfqpoint{4.739837in}{1.649017in}}%
\pgfpathlineto{\pgfqpoint{4.754003in}{1.645825in}}%
\pgfpathlineto{\pgfqpoint{4.768176in}{1.642657in}}%
\pgfpathlineto{\pgfqpoint{4.782357in}{1.639513in}}%
\pgfpathlineto{\pgfqpoint{4.774521in}{1.630222in}}%
\pgfpathlineto{\pgfqpoint{4.766682in}{1.621043in}}%
\pgfpathlineto{\pgfqpoint{4.758838in}{1.611983in}}%
\pgfpathlineto{\pgfqpoint{4.750990in}{1.603047in}}%
\pgfpathlineto{\pgfqpoint{4.736800in}{1.606436in}}%
\pgfpathlineto{\pgfqpoint{4.722617in}{1.609850in}}%
\pgfpathlineto{\pgfqpoint{4.708441in}{1.613287in}}%
\pgfpathlineto{\pgfqpoint{4.694272in}{1.616749in}}%
\pgfpathlineto{\pgfqpoint{4.702130in}{1.625434in}}%
\pgfpathlineto{\pgfqpoint{4.709984in}{1.634247in}}%
\pgfpathlineto{\pgfqpoint{4.717834in}{1.643182in}}%
\pgfpathlineto{\pgfqpoint{4.725679in}{1.652232in}}%
\pgfpathclose%
\pgfusepath{fill}%
\end{pgfscope}%
\begin{pgfscope}%
\pgfpathrectangle{\pgfqpoint{1.150000in}{0.150000in}}{\pgfqpoint{5.700000in}{5.700000in}}%
\pgfusepath{clip}%
\pgfsetbuttcap%
\pgfsetroundjoin%
\definecolor{currentfill}{rgb}{0.271305,0.019942,0.347269}%
\pgfsetfillcolor{currentfill}%
\pgfsetfillopacity{0.700000}%
\pgfsetlinewidth{0.000000pt}%
\definecolor{currentstroke}{rgb}{0.000000,0.000000,0.000000}%
\pgfsetstrokecolor{currentstroke}%
\pgfsetdash{}{0pt}%
\pgfpathmoveto{\pgfqpoint{4.348858in}{1.637582in}}%
\pgfpathlineto{\pgfqpoint{4.362915in}{1.633104in}}%
\pgfpathlineto{\pgfqpoint{4.376978in}{1.628652in}}%
\pgfpathlineto{\pgfqpoint{4.391048in}{1.624223in}}%
\pgfpathlineto{\pgfqpoint{4.405125in}{1.619820in}}%
\pgfpathlineto{\pgfqpoint{4.397174in}{1.613704in}}%
\pgfpathlineto{\pgfqpoint{4.389217in}{1.607797in}}%
\pgfpathlineto{\pgfqpoint{4.381256in}{1.602107in}}%
\pgfpathlineto{\pgfqpoint{4.373288in}{1.596640in}}%
\pgfpathlineto{\pgfqpoint{4.359196in}{1.601329in}}%
\pgfpathlineto{\pgfqpoint{4.345111in}{1.606042in}}%
\pgfpathlineto{\pgfqpoint{4.331032in}{1.610780in}}%
\pgfpathlineto{\pgfqpoint{4.316959in}{1.615542in}}%
\pgfpathlineto{\pgfqpoint{4.324942in}{1.620719in}}%
\pgfpathlineto{\pgfqpoint{4.332920in}{1.626123in}}%
\pgfpathlineto{\pgfqpoint{4.340892in}{1.631746in}}%
\pgfpathlineto{\pgfqpoint{4.348858in}{1.637582in}}%
\pgfpathclose%
\pgfusepath{fill}%
\end{pgfscope}%
\begin{pgfscope}%
\pgfpathrectangle{\pgfqpoint{1.150000in}{0.150000in}}{\pgfqpoint{5.700000in}{5.700000in}}%
\pgfusepath{clip}%
\pgfsetbuttcap%
\pgfsetroundjoin%
\definecolor{currentfill}{rgb}{0.273809,0.031497,0.358853}%
\pgfsetfillcolor{currentfill}%
\pgfsetfillopacity{0.700000}%
\pgfsetlinewidth{0.000000pt}%
\definecolor{currentstroke}{rgb}{0.000000,0.000000,0.000000}%
\pgfsetstrokecolor{currentstroke}%
\pgfsetdash{}{0pt}%
\pgfpathmoveto{\pgfqpoint{4.204601in}{1.654526in}}%
\pgfpathlineto{\pgfqpoint{4.218624in}{1.649566in}}%
\pgfpathlineto{\pgfqpoint{4.232654in}{1.644632in}}%
\pgfpathlineto{\pgfqpoint{4.246689in}{1.639722in}}%
\pgfpathlineto{\pgfqpoint{4.260730in}{1.634836in}}%
\pgfpathlineto{\pgfqpoint{4.252724in}{1.630188in}}%
\pgfpathlineto{\pgfqpoint{4.244711in}{1.625784in}}%
\pgfpathlineto{\pgfqpoint{4.236692in}{1.621634in}}%
\pgfpathlineto{\pgfqpoint{4.228665in}{1.617745in}}%
\pgfpathlineto{\pgfqpoint{4.214606in}{1.622929in}}%
\pgfpathlineto{\pgfqpoint{4.200552in}{1.628137in}}%
\pgfpathlineto{\pgfqpoint{4.186505in}{1.633371in}}%
\pgfpathlineto{\pgfqpoint{4.172463in}{1.638629in}}%
\pgfpathlineto{\pgfqpoint{4.180508in}{1.642214in}}%
\pgfpathlineto{\pgfqpoint{4.188546in}{1.646064in}}%
\pgfpathlineto{\pgfqpoint{4.196577in}{1.650170in}}%
\pgfpathlineto{\pgfqpoint{4.204601in}{1.654526in}}%
\pgfpathclose%
\pgfusepath{fill}%
\end{pgfscope}%
\begin{pgfscope}%
\pgfpathrectangle{\pgfqpoint{1.150000in}{0.150000in}}{\pgfqpoint{5.700000in}{5.700000in}}%
\pgfusepath{clip}%
\pgfsetbuttcap%
\pgfsetroundjoin%
\definecolor{currentfill}{rgb}{0.216210,0.351535,0.550627}%
\pgfsetfillcolor{currentfill}%
\pgfsetfillopacity{0.700000}%
\pgfsetlinewidth{0.000000pt}%
\definecolor{currentstroke}{rgb}{0.000000,0.000000,0.000000}%
\pgfsetstrokecolor{currentstroke}%
\pgfsetdash{}{0pt}%
\pgfpathmoveto{\pgfqpoint{2.835876in}{2.317473in}}%
\pgfpathlineto{\pgfqpoint{2.849705in}{2.308083in}}%
\pgfpathlineto{\pgfqpoint{2.863536in}{2.298728in}}%
\pgfpathlineto{\pgfqpoint{2.877370in}{2.289409in}}%
\pgfpathlineto{\pgfqpoint{2.891207in}{2.280125in}}%
\pgfpathlineto{\pgfqpoint{2.882227in}{2.291768in}}%
\pgfpathlineto{\pgfqpoint{2.873222in}{2.303959in}}%
\pgfpathlineto{\pgfqpoint{2.864190in}{2.316709in}}%
\pgfpathlineto{\pgfqpoint{2.855132in}{2.330030in}}%
\pgfpathlineto{\pgfqpoint{2.841248in}{2.339708in}}%
\pgfpathlineto{\pgfqpoint{2.827367in}{2.349421in}}%
\pgfpathlineto{\pgfqpoint{2.813488in}{2.359170in}}%
\pgfpathlineto{\pgfqpoint{2.799612in}{2.368954in}}%
\pgfpathlineto{\pgfqpoint{2.808719in}{2.355233in}}%
\pgfpathlineto{\pgfqpoint{2.817798in}{2.342086in}}%
\pgfpathlineto{\pgfqpoint{2.826850in}{2.329503in}}%
\pgfpathlineto{\pgfqpoint{2.835876in}{2.317473in}}%
\pgfpathclose%
\pgfusepath{fill}%
\end{pgfscope}%
\begin{pgfscope}%
\pgfpathrectangle{\pgfqpoint{1.150000in}{0.150000in}}{\pgfqpoint{5.700000in}{5.700000in}}%
\pgfusepath{clip}%
\pgfsetbuttcap%
\pgfsetroundjoin%
\definecolor{currentfill}{rgb}{0.269944,0.014625,0.341379}%
\pgfsetfillcolor{currentfill}%
\pgfsetfillopacity{0.700000}%
\pgfsetlinewidth{0.000000pt}%
\definecolor{currentstroke}{rgb}{0.000000,0.000000,0.000000}%
\pgfsetstrokecolor{currentstroke}%
\pgfsetdash{}{0pt}%
\pgfpathmoveto{\pgfqpoint{4.493191in}{1.629944in}}%
\pgfpathlineto{\pgfqpoint{4.507287in}{1.625934in}}%
\pgfpathlineto{\pgfqpoint{4.521390in}{1.621948in}}%
\pgfpathlineto{\pgfqpoint{4.535499in}{1.617986in}}%
\pgfpathlineto{\pgfqpoint{4.549616in}{1.614048in}}%
\pgfpathlineto{\pgfqpoint{4.541713in}{1.606635in}}%
\pgfpathlineto{\pgfqpoint{4.533805in}{1.599395in}}%
\pgfpathlineto{\pgfqpoint{4.525892in}{1.592336in}}%
\pgfpathlineto{\pgfqpoint{4.517975in}{1.585465in}}%
\pgfpathlineto{\pgfqpoint{4.503845in}{1.589674in}}%
\pgfpathlineto{\pgfqpoint{4.489722in}{1.593908in}}%
\pgfpathlineto{\pgfqpoint{4.475606in}{1.598166in}}%
\pgfpathlineto{\pgfqpoint{4.461496in}{1.602448in}}%
\pgfpathlineto{\pgfqpoint{4.469428in}{1.609042in}}%
\pgfpathlineto{\pgfqpoint{4.477354in}{1.615827in}}%
\pgfpathlineto{\pgfqpoint{4.485275in}{1.622797in}}%
\pgfpathlineto{\pgfqpoint{4.493191in}{1.629944in}}%
\pgfpathclose%
\pgfusepath{fill}%
\end{pgfscope}%
\begin{pgfscope}%
\pgfpathrectangle{\pgfqpoint{1.150000in}{0.150000in}}{\pgfqpoint{5.700000in}{5.700000in}}%
\pgfusepath{clip}%
\pgfsetbuttcap%
\pgfsetroundjoin%
\definecolor{currentfill}{rgb}{0.159194,0.482237,0.558073}%
\pgfsetfillcolor{currentfill}%
\pgfsetfillopacity{0.700000}%
\pgfsetlinewidth{0.000000pt}%
\definecolor{currentstroke}{rgb}{0.000000,0.000000,0.000000}%
\pgfsetstrokecolor{currentstroke}%
\pgfsetdash{}{0pt}%
\pgfpathmoveto{\pgfqpoint{2.411928in}{2.658594in}}%
\pgfpathlineto{\pgfqpoint{2.425751in}{2.647687in}}%
\pgfpathlineto{\pgfqpoint{2.439575in}{2.636825in}}%
\pgfpathlineto{\pgfqpoint{2.453401in}{2.626007in}}%
\pgfpathlineto{\pgfqpoint{2.467228in}{2.615234in}}%
\pgfpathlineto{\pgfqpoint{2.457781in}{2.631990in}}%
\pgfpathlineto{\pgfqpoint{2.448299in}{2.649373in}}%
\pgfpathlineto{\pgfqpoint{2.438782in}{2.667393in}}%
\pgfpathlineto{\pgfqpoint{2.429229in}{2.686065in}}%
\pgfpathlineto{\pgfqpoint{2.415346in}{2.697257in}}%
\pgfpathlineto{\pgfqpoint{2.401464in}{2.708493in}}%
\pgfpathlineto{\pgfqpoint{2.387583in}{2.719774in}}%
\pgfpathlineto{\pgfqpoint{2.373703in}{2.731100in}}%
\pgfpathlineto{\pgfqpoint{2.383313in}{2.712003in}}%
\pgfpathlineto{\pgfqpoint{2.392887in}{2.693561in}}%
\pgfpathlineto{\pgfqpoint{2.402425in}{2.675763in}}%
\pgfpathlineto{\pgfqpoint{2.411928in}{2.658594in}}%
\pgfpathclose%
\pgfusepath{fill}%
\end{pgfscope}%
\begin{pgfscope}%
\pgfpathrectangle{\pgfqpoint{1.150000in}{0.150000in}}{\pgfqpoint{5.700000in}{5.700000in}}%
\pgfusepath{clip}%
\pgfsetbuttcap%
\pgfsetroundjoin%
\definecolor{currentfill}{rgb}{0.277941,0.056324,0.381191}%
\pgfsetfillcolor{currentfill}%
\pgfsetfillopacity{0.700000}%
\pgfsetlinewidth{0.000000pt}%
\definecolor{currentstroke}{rgb}{0.000000,0.000000,0.000000}%
\pgfsetstrokecolor{currentstroke}%
\pgfsetdash{}{0pt}%
\pgfpathmoveto{\pgfqpoint{4.958471in}{1.696532in}}%
\pgfpathlineto{\pgfqpoint{4.972704in}{1.694057in}}%
\pgfpathlineto{\pgfqpoint{4.986945in}{1.691606in}}%
\pgfpathlineto{\pgfqpoint{5.001194in}{1.689180in}}%
\pgfpathlineto{\pgfqpoint{5.015450in}{1.686777in}}%
\pgfpathlineto{\pgfqpoint{5.007672in}{1.676149in}}%
\pgfpathlineto{\pgfqpoint{4.999890in}{1.665579in}}%
\pgfpathlineto{\pgfqpoint{4.992104in}{1.655071in}}%
\pgfpathlineto{\pgfqpoint{4.984313in}{1.644631in}}%
\pgfpathlineto{\pgfqpoint{4.970049in}{1.647254in}}%
\pgfpathlineto{\pgfqpoint{4.955793in}{1.649900in}}%
\pgfpathlineto{\pgfqpoint{4.941545in}{1.652571in}}%
\pgfpathlineto{\pgfqpoint{4.927304in}{1.655265in}}%
\pgfpathlineto{\pgfqpoint{4.935102in}{1.665480in}}%
\pgfpathlineto{\pgfqpoint{4.942896in}{1.675767in}}%
\pgfpathlineto{\pgfqpoint{4.950686in}{1.686119in}}%
\pgfpathlineto{\pgfqpoint{4.958471in}{1.696532in}}%
\pgfpathclose%
\pgfusepath{fill}%
\end{pgfscope}%
\begin{pgfscope}%
\pgfpathrectangle{\pgfqpoint{1.150000in}{0.150000in}}{\pgfqpoint{5.700000in}{5.700000in}}%
\pgfusepath{clip}%
\pgfsetbuttcap%
\pgfsetroundjoin%
\definecolor{currentfill}{rgb}{0.277018,0.050344,0.375715}%
\pgfsetfillcolor{currentfill}%
\pgfsetfillopacity{0.700000}%
\pgfsetlinewidth{0.000000pt}%
\definecolor{currentstroke}{rgb}{0.000000,0.000000,0.000000}%
\pgfsetstrokecolor{currentstroke}%
\pgfsetdash{}{0pt}%
\pgfpathmoveto{\pgfqpoint{4.060344in}{1.681590in}}%
\pgfpathlineto{\pgfqpoint{4.074338in}{1.676132in}}%
\pgfpathlineto{\pgfqpoint{4.088339in}{1.670699in}}%
\pgfpathlineto{\pgfqpoint{4.102345in}{1.665292in}}%
\pgfpathlineto{\pgfqpoint{4.116357in}{1.659909in}}%
\pgfpathlineto{\pgfqpoint{4.108285in}{1.656904in}}%
\pgfpathlineto{\pgfqpoint{4.100206in}{1.654182in}}%
\pgfpathlineto{\pgfqpoint{4.092118in}{1.651753in}}%
\pgfpathlineto{\pgfqpoint{4.084023in}{1.649624in}}%
\pgfpathlineto{\pgfqpoint{4.069990in}{1.655319in}}%
\pgfpathlineto{\pgfqpoint{4.055963in}{1.661039in}}%
\pgfpathlineto{\pgfqpoint{4.041942in}{1.666784in}}%
\pgfpathlineto{\pgfqpoint{4.027926in}{1.672554in}}%
\pgfpathlineto{\pgfqpoint{4.036043in}{1.674365in}}%
\pgfpathlineto{\pgfqpoint{4.044151in}{1.676481in}}%
\pgfpathlineto{\pgfqpoint{4.052251in}{1.678892in}}%
\pgfpathlineto{\pgfqpoint{4.060344in}{1.681590in}}%
\pgfpathclose%
\pgfusepath{fill}%
\end{pgfscope}%
\begin{pgfscope}%
\pgfpathrectangle{\pgfqpoint{1.150000in}{0.150000in}}{\pgfqpoint{5.700000in}{5.700000in}}%
\pgfusepath{clip}%
\pgfsetbuttcap%
\pgfsetroundjoin%
\definecolor{currentfill}{rgb}{0.282290,0.145912,0.461510}%
\pgfsetfillcolor{currentfill}%
\pgfsetfillopacity{0.700000}%
\pgfsetlinewidth{0.000000pt}%
\definecolor{currentstroke}{rgb}{0.000000,0.000000,0.000000}%
\pgfsetstrokecolor{currentstroke}%
\pgfsetdash{}{0pt}%
\pgfpathmoveto{\pgfqpoint{5.456300in}{1.878971in}}%
\pgfpathlineto{\pgfqpoint{5.470705in}{1.877977in}}%
\pgfpathlineto{\pgfqpoint{5.485119in}{1.877007in}}%
\pgfpathlineto{\pgfqpoint{5.499542in}{1.876062in}}%
\pgfpathlineto{\pgfqpoint{5.491897in}{1.864415in}}%
\pgfpathlineto{\pgfqpoint{5.484245in}{1.852719in}}%
\pgfpathlineto{\pgfqpoint{5.476588in}{1.840977in}}%
\pgfpathlineto{\pgfqpoint{5.468925in}{1.829192in}}%
\pgfpathlineto{\pgfqpoint{5.454496in}{1.830291in}}%
\pgfpathlineto{\pgfqpoint{5.440076in}{1.831414in}}%
\pgfpathlineto{\pgfqpoint{5.425666in}{1.832562in}}%
\pgfpathlineto{\pgfqpoint{5.433333in}{1.844228in}}%
\pgfpathlineto{\pgfqpoint{5.440994in}{1.855854in}}%
\pgfpathlineto{\pgfqpoint{5.448650in}{1.867436in}}%
\pgfpathlineto{\pgfqpoint{5.456300in}{1.878971in}}%
\pgfpathclose%
\pgfusepath{fill}%
\end{pgfscope}%
\begin{pgfscope}%
\pgfpathrectangle{\pgfqpoint{1.150000in}{0.150000in}}{\pgfqpoint{5.700000in}{5.700000in}}%
\pgfusepath{clip}%
\pgfsetbuttcap%
\pgfsetroundjoin%
\definecolor{currentfill}{rgb}{0.278826,0.175490,0.483397}%
\pgfsetfillcolor{currentfill}%
\pgfsetfillopacity{0.700000}%
\pgfsetlinewidth{0.000000pt}%
\definecolor{currentstroke}{rgb}{0.000000,0.000000,0.000000}%
\pgfsetstrokecolor{currentstroke}%
\pgfsetdash{}{0pt}%
\pgfpathmoveto{\pgfqpoint{3.459655in}{1.919172in}}%
\pgfpathlineto{\pgfqpoint{3.473546in}{1.911777in}}%
\pgfpathlineto{\pgfqpoint{3.487442in}{1.904411in}}%
\pgfpathlineto{\pgfqpoint{3.501343in}{1.897073in}}%
\pgfpathlineto{\pgfqpoint{3.515247in}{1.889763in}}%
\pgfpathlineto{\pgfqpoint{3.506812in}{1.893886in}}%
\pgfpathlineto{\pgfqpoint{3.498362in}{1.898433in}}%
\pgfpathlineto{\pgfqpoint{3.489896in}{1.903413in}}%
\pgfpathlineto{\pgfqpoint{3.481414in}{1.908837in}}%
\pgfpathlineto{\pgfqpoint{3.467476in}{1.916505in}}%
\pgfpathlineto{\pgfqpoint{3.453543in}{1.924200in}}%
\pgfpathlineto{\pgfqpoint{3.439613in}{1.931923in}}%
\pgfpathlineto{\pgfqpoint{3.425688in}{1.939675in}}%
\pgfpathlineto{\pgfqpoint{3.434204in}{1.933888in}}%
\pgfpathlineto{\pgfqpoint{3.442703in}{1.928548in}}%
\pgfpathlineto{\pgfqpoint{3.451187in}{1.923646in}}%
\pgfpathlineto{\pgfqpoint{3.459655in}{1.919172in}}%
\pgfpathclose%
\pgfusepath{fill}%
\end{pgfscope}%
\begin{pgfscope}%
\pgfpathrectangle{\pgfqpoint{1.150000in}{0.150000in}}{\pgfqpoint{5.700000in}{5.700000in}}%
\pgfusepath{clip}%
\pgfsetbuttcap%
\pgfsetroundjoin%
\definecolor{currentfill}{rgb}{0.260571,0.246922,0.522828}%
\pgfsetfillcolor{currentfill}%
\pgfsetfillopacity{0.700000}%
\pgfsetlinewidth{0.000000pt}%
\definecolor{currentstroke}{rgb}{0.000000,0.000000,0.000000}%
\pgfsetstrokecolor{currentstroke}%
\pgfsetdash{}{0pt}%
\pgfpathmoveto{\pgfqpoint{3.203421in}{2.067660in}}%
\pgfpathlineto{\pgfqpoint{3.217284in}{2.059437in}}%
\pgfpathlineto{\pgfqpoint{3.231151in}{2.051245in}}%
\pgfpathlineto{\pgfqpoint{3.245021in}{2.043083in}}%
\pgfpathlineto{\pgfqpoint{3.258895in}{2.034951in}}%
\pgfpathlineto{\pgfqpoint{3.250253in}{2.042298in}}%
\pgfpathlineto{\pgfqpoint{3.241591in}{2.050124in}}%
\pgfpathlineto{\pgfqpoint{3.232910in}{2.058441in}}%
\pgfpathlineto{\pgfqpoint{3.224208in}{2.067259in}}%
\pgfpathlineto{\pgfqpoint{3.210295in}{2.075764in}}%
\pgfpathlineto{\pgfqpoint{3.196386in}{2.084300in}}%
\pgfpathlineto{\pgfqpoint{3.182480in}{2.092867in}}%
\pgfpathlineto{\pgfqpoint{3.168577in}{2.101464in}}%
\pgfpathlineto{\pgfqpoint{3.177319in}{2.092266in}}%
\pgfpathlineto{\pgfqpoint{3.186040in}{2.083574in}}%
\pgfpathlineto{\pgfqpoint{3.194740in}{2.075375in}}%
\pgfpathlineto{\pgfqpoint{3.203421in}{2.067660in}}%
\pgfpathclose%
\pgfusepath{fill}%
\end{pgfscope}%
\begin{pgfscope}%
\pgfpathrectangle{\pgfqpoint{1.150000in}{0.150000in}}{\pgfqpoint{5.700000in}{5.700000in}}%
\pgfusepath{clip}%
\pgfsetbuttcap%
\pgfsetroundjoin%
\definecolor{currentfill}{rgb}{0.271305,0.019942,0.347269}%
\pgfsetfillcolor{currentfill}%
\pgfsetfillopacity{0.700000}%
\pgfsetlinewidth{0.000000pt}%
\definecolor{currentstroke}{rgb}{0.000000,0.000000,0.000000}%
\pgfsetstrokecolor{currentstroke}%
\pgfsetdash{}{0pt}%
\pgfpathmoveto{\pgfqpoint{4.637670in}{1.630836in}}%
\pgfpathlineto{\pgfqpoint{4.651810in}{1.627278in}}%
\pgfpathlineto{\pgfqpoint{4.665957in}{1.623744in}}%
\pgfpathlineto{\pgfqpoint{4.680111in}{1.620234in}}%
\pgfpathlineto{\pgfqpoint{4.694272in}{1.616749in}}%
\pgfpathlineto{\pgfqpoint{4.686410in}{1.608198in}}%
\pgfpathlineto{\pgfqpoint{4.678544in}{1.599789in}}%
\pgfpathlineto{\pgfqpoint{4.670673in}{1.591526in}}%
\pgfpathlineto{\pgfqpoint{4.662799in}{1.583418in}}%
\pgfpathlineto{\pgfqpoint{4.648626in}{1.587162in}}%
\pgfpathlineto{\pgfqpoint{4.634461in}{1.590931in}}%
\pgfpathlineto{\pgfqpoint{4.620302in}{1.594723in}}%
\pgfpathlineto{\pgfqpoint{4.606151in}{1.598540in}}%
\pgfpathlineto{\pgfqpoint{4.614038in}{1.606385in}}%
\pgfpathlineto{\pgfqpoint{4.621920in}{1.614387in}}%
\pgfpathlineto{\pgfqpoint{4.629797in}{1.622539in}}%
\pgfpathlineto{\pgfqpoint{4.637670in}{1.630836in}}%
\pgfpathclose%
\pgfusepath{fill}%
\end{pgfscope}%
\begin{pgfscope}%
\pgfpathrectangle{\pgfqpoint{1.150000in}{0.150000in}}{\pgfqpoint{5.700000in}{5.700000in}}%
\pgfusepath{clip}%
\pgfsetbuttcap%
\pgfsetroundjoin%
\definecolor{currentfill}{rgb}{0.283072,0.130895,0.449241}%
\pgfsetfillcolor{currentfill}%
\pgfsetfillopacity{0.700000}%
\pgfsetlinewidth{0.000000pt}%
\definecolor{currentstroke}{rgb}{0.000000,0.000000,0.000000}%
\pgfsetstrokecolor{currentstroke}%
\pgfsetdash{}{0pt}%
\pgfpathmoveto{\pgfqpoint{5.368112in}{1.837396in}}%
\pgfpathlineto{\pgfqpoint{5.382487in}{1.836151in}}%
\pgfpathlineto{\pgfqpoint{5.396871in}{1.834930in}}%
\pgfpathlineto{\pgfqpoint{5.411264in}{1.833734in}}%
\pgfpathlineto{\pgfqpoint{5.425666in}{1.832562in}}%
\pgfpathlineto{\pgfqpoint{5.417993in}{1.820860in}}%
\pgfpathlineto{\pgfqpoint{5.410315in}{1.809126in}}%
\pgfpathlineto{\pgfqpoint{5.402631in}{1.797363in}}%
\pgfpathlineto{\pgfqpoint{5.394942in}{1.785575in}}%
\pgfpathlineto{\pgfqpoint{5.380535in}{1.786915in}}%
\pgfpathlineto{\pgfqpoint{5.366137in}{1.788278in}}%
\pgfpathlineto{\pgfqpoint{5.351748in}{1.789666in}}%
\pgfpathlineto{\pgfqpoint{5.337367in}{1.791077in}}%
\pgfpathlineto{\pgfqpoint{5.345061in}{1.802693in}}%
\pgfpathlineto{\pgfqpoint{5.352750in}{1.814287in}}%
\pgfpathlineto{\pgfqpoint{5.360434in}{1.825856in}}%
\pgfpathlineto{\pgfqpoint{5.368112in}{1.837396in}}%
\pgfpathclose%
\pgfusepath{fill}%
\end{pgfscope}%
\begin{pgfscope}%
\pgfpathrectangle{\pgfqpoint{1.150000in}{0.150000in}}{\pgfqpoint{5.700000in}{5.700000in}}%
\pgfusepath{clip}%
\pgfsetbuttcap%
\pgfsetroundjoin%
\definecolor{currentfill}{rgb}{0.274952,0.037752,0.364543}%
\pgfsetfillcolor{currentfill}%
\pgfsetfillopacity{0.700000}%
\pgfsetlinewidth{0.000000pt}%
\definecolor{currentstroke}{rgb}{0.000000,0.000000,0.000000}%
\pgfsetstrokecolor{currentstroke}%
\pgfsetdash{}{0pt}%
\pgfpathmoveto{\pgfqpoint{4.870419in}{1.666283in}}%
\pgfpathlineto{\pgfqpoint{4.884629in}{1.663493in}}%
\pgfpathlineto{\pgfqpoint{4.898846in}{1.660726in}}%
\pgfpathlineto{\pgfqpoint{4.913071in}{1.657984in}}%
\pgfpathlineto{\pgfqpoint{4.927304in}{1.655265in}}%
\pgfpathlineto{\pgfqpoint{4.919502in}{1.645127in}}%
\pgfpathlineto{\pgfqpoint{4.911696in}{1.635071in}}%
\pgfpathlineto{\pgfqpoint{4.903886in}{1.625104in}}%
\pgfpathlineto{\pgfqpoint{4.896073in}{1.615230in}}%
\pgfpathlineto{\pgfqpoint{4.881832in}{1.618182in}}%
\pgfpathlineto{\pgfqpoint{4.867598in}{1.621157in}}%
\pgfpathlineto{\pgfqpoint{4.853372in}{1.624156in}}%
\pgfpathlineto{\pgfqpoint{4.839154in}{1.627180in}}%
\pgfpathlineto{\pgfqpoint{4.846976in}{1.636815in}}%
\pgfpathlineto{\pgfqpoint{4.854795in}{1.646548in}}%
\pgfpathlineto{\pgfqpoint{4.862609in}{1.656373in}}%
\pgfpathlineto{\pgfqpoint{4.870419in}{1.666283in}}%
\pgfpathclose%
\pgfusepath{fill}%
\end{pgfscope}%
\begin{pgfscope}%
\pgfpathrectangle{\pgfqpoint{1.150000in}{0.150000in}}{\pgfqpoint{5.700000in}{5.700000in}}%
\pgfusepath{clip}%
\pgfsetbuttcap%
\pgfsetroundjoin%
\definecolor{currentfill}{rgb}{0.221989,0.339161,0.548752}%
\pgfsetfillcolor{currentfill}%
\pgfsetfillopacity{0.700000}%
\pgfsetlinewidth{0.000000pt}%
\definecolor{currentstroke}{rgb}{0.000000,0.000000,0.000000}%
\pgfsetstrokecolor{currentstroke}%
\pgfsetdash{}{0pt}%
\pgfpathmoveto{\pgfqpoint{2.891207in}{2.280125in}}%
\pgfpathlineto{\pgfqpoint{2.905047in}{2.270876in}}%
\pgfpathlineto{\pgfqpoint{2.918890in}{2.261661in}}%
\pgfpathlineto{\pgfqpoint{2.932735in}{2.252481in}}%
\pgfpathlineto{\pgfqpoint{2.946583in}{2.243334in}}%
\pgfpathlineto{\pgfqpoint{2.937648in}{2.254590in}}%
\pgfpathlineto{\pgfqpoint{2.928689in}{2.266390in}}%
\pgfpathlineto{\pgfqpoint{2.919703in}{2.278745in}}%
\pgfpathlineto{\pgfqpoint{2.910692in}{2.291666in}}%
\pgfpathlineto{\pgfqpoint{2.896798in}{2.301205in}}%
\pgfpathlineto{\pgfqpoint{2.882907in}{2.310779in}}%
\pgfpathlineto{\pgfqpoint{2.869018in}{2.320387in}}%
\pgfpathlineto{\pgfqpoint{2.855132in}{2.330030in}}%
\pgfpathlineto{\pgfqpoint{2.864190in}{2.316709in}}%
\pgfpathlineto{\pgfqpoint{2.873222in}{2.303959in}}%
\pgfpathlineto{\pgfqpoint{2.882227in}{2.291768in}}%
\pgfpathlineto{\pgfqpoint{2.891207in}{2.280125in}}%
\pgfpathclose%
\pgfusepath{fill}%
\end{pgfscope}%
\begin{pgfscope}%
\pgfpathrectangle{\pgfqpoint{1.150000in}{0.150000in}}{\pgfqpoint{5.700000in}{5.700000in}}%
\pgfusepath{clip}%
\pgfsetbuttcap%
\pgfsetroundjoin%
\definecolor{currentfill}{rgb}{0.165117,0.467423,0.558141}%
\pgfsetfillcolor{currentfill}%
\pgfsetfillopacity{0.700000}%
\pgfsetlinewidth{0.000000pt}%
\definecolor{currentstroke}{rgb}{0.000000,0.000000,0.000000}%
\pgfsetstrokecolor{currentstroke}%
\pgfsetdash{}{0pt}%
\pgfpathmoveto{\pgfqpoint{2.467228in}{2.615234in}}%
\pgfpathlineto{\pgfqpoint{2.481057in}{2.604504in}}%
\pgfpathlineto{\pgfqpoint{2.494887in}{2.593817in}}%
\pgfpathlineto{\pgfqpoint{2.508718in}{2.583173in}}%
\pgfpathlineto{\pgfqpoint{2.522552in}{2.572572in}}%
\pgfpathlineto{\pgfqpoint{2.513158in}{2.588918in}}%
\pgfpathlineto{\pgfqpoint{2.503732in}{2.605886in}}%
\pgfpathlineto{\pgfqpoint{2.494271in}{2.623487in}}%
\pgfpathlineto{\pgfqpoint{2.484775in}{2.641734in}}%
\pgfpathlineto{\pgfqpoint{2.470887in}{2.652752in}}%
\pgfpathlineto{\pgfqpoint{2.456999in}{2.663813in}}%
\pgfpathlineto{\pgfqpoint{2.443114in}{2.674917in}}%
\pgfpathlineto{\pgfqpoint{2.429229in}{2.686065in}}%
\pgfpathlineto{\pgfqpoint{2.438782in}{2.667393in}}%
\pgfpathlineto{\pgfqpoint{2.448299in}{2.649373in}}%
\pgfpathlineto{\pgfqpoint{2.457781in}{2.631990in}}%
\pgfpathlineto{\pgfqpoint{2.467228in}{2.615234in}}%
\pgfpathclose%
\pgfusepath{fill}%
\end{pgfscope}%
\begin{pgfscope}%
\pgfpathrectangle{\pgfqpoint{1.150000in}{0.150000in}}{\pgfqpoint{5.700000in}{5.700000in}}%
\pgfusepath{clip}%
\pgfsetbuttcap%
\pgfsetroundjoin%
\definecolor{currentfill}{rgb}{0.283091,0.110553,0.431554}%
\pgfsetfillcolor{currentfill}%
\pgfsetfillopacity{0.700000}%
\pgfsetlinewidth{0.000000pt}%
\definecolor{currentstroke}{rgb}{0.000000,0.000000,0.000000}%
\pgfsetstrokecolor{currentstroke}%
\pgfsetdash{}{0pt}%
\pgfpathmoveto{\pgfqpoint{5.279931in}{1.796967in}}%
\pgfpathlineto{\pgfqpoint{5.294277in}{1.795458in}}%
\pgfpathlineto{\pgfqpoint{5.308632in}{1.793974in}}%
\pgfpathlineto{\pgfqpoint{5.322995in}{1.792513in}}%
\pgfpathlineto{\pgfqpoint{5.337367in}{1.791077in}}%
\pgfpathlineto{\pgfqpoint{5.329668in}{1.779445in}}%
\pgfpathlineto{\pgfqpoint{5.321963in}{1.767799in}}%
\pgfpathlineto{\pgfqpoint{5.314254in}{1.756145in}}%
\pgfpathlineto{\pgfqpoint{5.306540in}{1.744486in}}%
\pgfpathlineto{\pgfqpoint{5.292163in}{1.746103in}}%
\pgfpathlineto{\pgfqpoint{5.277794in}{1.747744in}}%
\pgfpathlineto{\pgfqpoint{5.263434in}{1.749408in}}%
\pgfpathlineto{\pgfqpoint{5.249082in}{1.751097in}}%
\pgfpathlineto{\pgfqpoint{5.256802in}{1.762571in}}%
\pgfpathlineto{\pgfqpoint{5.264516in}{1.774043in}}%
\pgfpathlineto{\pgfqpoint{5.272226in}{1.785510in}}%
\pgfpathlineto{\pgfqpoint{5.279931in}{1.796967in}}%
\pgfpathclose%
\pgfusepath{fill}%
\end{pgfscope}%
\begin{pgfscope}%
\pgfpathrectangle{\pgfqpoint{1.150000in}{0.150000in}}{\pgfqpoint{5.700000in}{5.700000in}}%
\pgfusepath{clip}%
\pgfsetbuttcap%
\pgfsetroundjoin%
\definecolor{currentfill}{rgb}{0.283197,0.115680,0.436115}%
\pgfsetfillcolor{currentfill}%
\pgfsetfillopacity{0.700000}%
\pgfsetlinewidth{0.000000pt}%
\definecolor{currentstroke}{rgb}{0.000000,0.000000,0.000000}%
\pgfsetstrokecolor{currentstroke}%
\pgfsetdash{}{0pt}%
\pgfpathmoveto{\pgfqpoint{3.715696in}{1.795839in}}%
\pgfpathlineto{\pgfqpoint{3.729631in}{1.789225in}}%
\pgfpathlineto{\pgfqpoint{3.743572in}{1.782638in}}%
\pgfpathlineto{\pgfqpoint{3.757517in}{1.776077in}}%
\pgfpathlineto{\pgfqpoint{3.771467in}{1.769542in}}%
\pgfpathlineto{\pgfqpoint{3.763203in}{1.770722in}}%
\pgfpathlineto{\pgfqpoint{3.754927in}{1.772274in}}%
\pgfpathlineto{\pgfqpoint{3.746639in}{1.774205in}}%
\pgfpathlineto{\pgfqpoint{3.738339in}{1.776526in}}%
\pgfpathlineto{\pgfqpoint{3.724361in}{1.783402in}}%
\pgfpathlineto{\pgfqpoint{3.710387in}{1.790304in}}%
\pgfpathlineto{\pgfqpoint{3.696419in}{1.797233in}}%
\pgfpathlineto{\pgfqpoint{3.682455in}{1.804188in}}%
\pgfpathlineto{\pgfqpoint{3.690784in}{1.801520in}}%
\pgfpathlineto{\pgfqpoint{3.699100in}{1.799246in}}%
\pgfpathlineto{\pgfqpoint{3.707404in}{1.797355in}}%
\pgfpathlineto{\pgfqpoint{3.715696in}{1.795839in}}%
\pgfpathclose%
\pgfusepath{fill}%
\end{pgfscope}%
\begin{pgfscope}%
\pgfpathrectangle{\pgfqpoint{1.150000in}{0.150000in}}{\pgfqpoint{5.700000in}{5.700000in}}%
\pgfusepath{clip}%
\pgfsetbuttcap%
\pgfsetroundjoin%
\definecolor{currentfill}{rgb}{0.280894,0.078907,0.402329}%
\pgfsetfillcolor{currentfill}%
\pgfsetfillopacity{0.700000}%
\pgfsetlinewidth{0.000000pt}%
\definecolor{currentstroke}{rgb}{0.000000,0.000000,0.000000}%
\pgfsetstrokecolor{currentstroke}%
\pgfsetdash{}{0pt}%
\pgfpathmoveto{\pgfqpoint{3.915998in}{1.719629in}}%
\pgfpathlineto{\pgfqpoint{3.929970in}{1.713656in}}%
\pgfpathlineto{\pgfqpoint{3.943947in}{1.707707in}}%
\pgfpathlineto{\pgfqpoint{3.957930in}{1.701785in}}%
\pgfpathlineto{\pgfqpoint{3.971918in}{1.695888in}}%
\pgfpathlineto{\pgfqpoint{3.963770in}{1.694710in}}%
\pgfpathlineto{\pgfqpoint{3.955612in}{1.693857in}}%
\pgfpathlineto{\pgfqpoint{3.947446in}{1.693338in}}%
\pgfpathlineto{\pgfqpoint{3.939269in}{1.693160in}}%
\pgfpathlineto{\pgfqpoint{3.925257in}{1.699384in}}%
\pgfpathlineto{\pgfqpoint{3.911250in}{1.705633in}}%
\pgfpathlineto{\pgfqpoint{3.897249in}{1.711907in}}%
\pgfpathlineto{\pgfqpoint{3.883252in}{1.718208in}}%
\pgfpathlineto{\pgfqpoint{3.891454in}{1.718053in}}%
\pgfpathlineto{\pgfqpoint{3.899645in}{1.718245in}}%
\pgfpathlineto{\pgfqpoint{3.907826in}{1.718773in}}%
\pgfpathlineto{\pgfqpoint{3.915998in}{1.719629in}}%
\pgfpathclose%
\pgfusepath{fill}%
\end{pgfscope}%
\begin{pgfscope}%
\pgfpathrectangle{\pgfqpoint{1.150000in}{0.150000in}}{\pgfqpoint{5.700000in}{5.700000in}}%
\pgfusepath{clip}%
\pgfsetbuttcap%
\pgfsetroundjoin%
\definecolor{currentfill}{rgb}{0.282327,0.094955,0.417331}%
\pgfsetfillcolor{currentfill}%
\pgfsetfillopacity{0.700000}%
\pgfsetlinewidth{0.000000pt}%
\definecolor{currentstroke}{rgb}{0.000000,0.000000,0.000000}%
\pgfsetstrokecolor{currentstroke}%
\pgfsetdash{}{0pt}%
\pgfpathmoveto{\pgfqpoint{5.191761in}{1.758095in}}%
\pgfpathlineto{\pgfqpoint{5.206078in}{1.756309in}}%
\pgfpathlineto{\pgfqpoint{5.220404in}{1.754548in}}%
\pgfpathlineto{\pgfqpoint{5.234739in}{1.752811in}}%
\pgfpathlineto{\pgfqpoint{5.249082in}{1.751097in}}%
\pgfpathlineto{\pgfqpoint{5.241358in}{1.739627in}}%
\pgfpathlineto{\pgfqpoint{5.233629in}{1.728165in}}%
\pgfpathlineto{\pgfqpoint{5.225896in}{1.716716in}}%
\pgfpathlineto{\pgfqpoint{5.218158in}{1.705283in}}%
\pgfpathlineto{\pgfqpoint{5.203809in}{1.707190in}}%
\pgfpathlineto{\pgfqpoint{5.189469in}{1.709121in}}%
\pgfpathlineto{\pgfqpoint{5.175137in}{1.711076in}}%
\pgfpathlineto{\pgfqpoint{5.160813in}{1.713055in}}%
\pgfpathlineto{\pgfqpoint{5.168557in}{1.724289in}}%
\pgfpathlineto{\pgfqpoint{5.176296in}{1.735544in}}%
\pgfpathlineto{\pgfqpoint{5.184031in}{1.746814in}}%
\pgfpathlineto{\pgfqpoint{5.191761in}{1.758095in}}%
\pgfpathclose%
\pgfusepath{fill}%
\end{pgfscope}%
\begin{pgfscope}%
\pgfpathrectangle{\pgfqpoint{1.150000in}{0.150000in}}{\pgfqpoint{5.700000in}{5.700000in}}%
\pgfusepath{clip}%
\pgfsetbuttcap%
\pgfsetroundjoin%
\definecolor{currentfill}{rgb}{0.273809,0.031497,0.358853}%
\pgfsetfillcolor{currentfill}%
\pgfsetfillopacity{0.700000}%
\pgfsetlinewidth{0.000000pt}%
\definecolor{currentstroke}{rgb}{0.000000,0.000000,0.000000}%
\pgfsetstrokecolor{currentstroke}%
\pgfsetdash{}{0pt}%
\pgfpathmoveto{\pgfqpoint{4.260730in}{1.634836in}}%
\pgfpathlineto{\pgfqpoint{4.274778in}{1.629976in}}%
\pgfpathlineto{\pgfqpoint{4.288832in}{1.625140in}}%
\pgfpathlineto{\pgfqpoint{4.302892in}{1.620329in}}%
\pgfpathlineto{\pgfqpoint{4.316959in}{1.615542in}}%
\pgfpathlineto{\pgfqpoint{4.308969in}{1.610600in}}%
\pgfpathlineto{\pgfqpoint{4.300974in}{1.605899in}}%
\pgfpathlineto{\pgfqpoint{4.292972in}{1.601449in}}%
\pgfpathlineto{\pgfqpoint{4.284964in}{1.597256in}}%
\pgfpathlineto{\pgfqpoint{4.270880in}{1.602342in}}%
\pgfpathlineto{\pgfqpoint{4.256803in}{1.607452in}}%
\pgfpathlineto{\pgfqpoint{4.242731in}{1.612586in}}%
\pgfpathlineto{\pgfqpoint{4.228665in}{1.617745in}}%
\pgfpathlineto{\pgfqpoint{4.236692in}{1.621634in}}%
\pgfpathlineto{\pgfqpoint{4.244711in}{1.625784in}}%
\pgfpathlineto{\pgfqpoint{4.252724in}{1.630188in}}%
\pgfpathlineto{\pgfqpoint{4.260730in}{1.634836in}}%
\pgfpathclose%
\pgfusepath{fill}%
\end{pgfscope}%
\begin{pgfscope}%
\pgfpathrectangle{\pgfqpoint{1.150000in}{0.150000in}}{\pgfqpoint{5.700000in}{5.700000in}}%
\pgfusepath{clip}%
\pgfsetbuttcap%
\pgfsetroundjoin%
\definecolor{currentfill}{rgb}{0.271305,0.019942,0.347269}%
\pgfsetfillcolor{currentfill}%
\pgfsetfillopacity{0.700000}%
\pgfsetlinewidth{0.000000pt}%
\definecolor{currentstroke}{rgb}{0.000000,0.000000,0.000000}%
\pgfsetstrokecolor{currentstroke}%
\pgfsetdash{}{0pt}%
\pgfpathmoveto{\pgfqpoint{4.405125in}{1.619820in}}%
\pgfpathlineto{\pgfqpoint{4.419208in}{1.615440in}}%
\pgfpathlineto{\pgfqpoint{4.433297in}{1.611085in}}%
\pgfpathlineto{\pgfqpoint{4.447394in}{1.606754in}}%
\pgfpathlineto{\pgfqpoint{4.461496in}{1.602448in}}%
\pgfpathlineto{\pgfqpoint{4.453560in}{1.596052in}}%
\pgfpathlineto{\pgfqpoint{4.445619in}{1.589862in}}%
\pgfpathlineto{\pgfqpoint{4.437672in}{1.583884in}}%
\pgfpathlineto{\pgfqpoint{4.429720in}{1.578127in}}%
\pgfpathlineto{\pgfqpoint{4.415603in}{1.582719in}}%
\pgfpathlineto{\pgfqpoint{4.401491in}{1.587335in}}%
\pgfpathlineto{\pgfqpoint{4.387387in}{1.591975in}}%
\pgfpathlineto{\pgfqpoint{4.373288in}{1.596640in}}%
\pgfpathlineto{\pgfqpoint{4.381256in}{1.602107in}}%
\pgfpathlineto{\pgfqpoint{4.389217in}{1.607797in}}%
\pgfpathlineto{\pgfqpoint{4.397174in}{1.613704in}}%
\pgfpathlineto{\pgfqpoint{4.405125in}{1.619820in}}%
\pgfpathclose%
\pgfusepath{fill}%
\end{pgfscope}%
\begin{pgfscope}%
\pgfpathrectangle{\pgfqpoint{1.150000in}{0.150000in}}{\pgfqpoint{5.700000in}{5.700000in}}%
\pgfusepath{clip}%
\pgfsetbuttcap%
\pgfsetroundjoin%
\definecolor{currentfill}{rgb}{0.273809,0.031497,0.358853}%
\pgfsetfillcolor{currentfill}%
\pgfsetfillopacity{0.700000}%
\pgfsetlinewidth{0.000000pt}%
\definecolor{currentstroke}{rgb}{0.000000,0.000000,0.000000}%
\pgfsetstrokecolor{currentstroke}%
\pgfsetdash{}{0pt}%
\pgfpathmoveto{\pgfqpoint{4.782357in}{1.639513in}}%
\pgfpathlineto{\pgfqpoint{4.796545in}{1.636394in}}%
\pgfpathlineto{\pgfqpoint{4.810740in}{1.633299in}}%
\pgfpathlineto{\pgfqpoint{4.824944in}{1.630227in}}%
\pgfpathlineto{\pgfqpoint{4.839154in}{1.627180in}}%
\pgfpathlineto{\pgfqpoint{4.831328in}{1.617647in}}%
\pgfpathlineto{\pgfqpoint{4.823498in}{1.608224in}}%
\pgfpathlineto{\pgfqpoint{4.815664in}{1.598916in}}%
\pgfpathlineto{\pgfqpoint{4.807826in}{1.589729in}}%
\pgfpathlineto{\pgfqpoint{4.793606in}{1.593023in}}%
\pgfpathlineto{\pgfqpoint{4.779393in}{1.596340in}}%
\pgfpathlineto{\pgfqpoint{4.765188in}{1.599682in}}%
\pgfpathlineto{\pgfqpoint{4.750990in}{1.603047in}}%
\pgfpathlineto{\pgfqpoint{4.758838in}{1.611983in}}%
\pgfpathlineto{\pgfqpoint{4.766682in}{1.621043in}}%
\pgfpathlineto{\pgfqpoint{4.774521in}{1.630222in}}%
\pgfpathlineto{\pgfqpoint{4.782357in}{1.639513in}}%
\pgfpathclose%
\pgfusepath{fill}%
\end{pgfscope}%
\begin{pgfscope}%
\pgfpathrectangle{\pgfqpoint{1.150000in}{0.150000in}}{\pgfqpoint{5.700000in}{5.700000in}}%
\pgfusepath{clip}%
\pgfsetbuttcap%
\pgfsetroundjoin%
\definecolor{currentfill}{rgb}{0.280255,0.165693,0.476498}%
\pgfsetfillcolor{currentfill}%
\pgfsetfillopacity{0.700000}%
\pgfsetlinewidth{0.000000pt}%
\definecolor{currentstroke}{rgb}{0.000000,0.000000,0.000000}%
\pgfsetstrokecolor{currentstroke}%
\pgfsetdash{}{0pt}%
\pgfpathmoveto{\pgfqpoint{3.515247in}{1.889763in}}%
\pgfpathlineto{\pgfqpoint{3.529156in}{1.882480in}}%
\pgfpathlineto{\pgfqpoint{3.543070in}{1.875226in}}%
\pgfpathlineto{\pgfqpoint{3.556988in}{1.867999in}}%
\pgfpathlineto{\pgfqpoint{3.570910in}{1.860800in}}%
\pgfpathlineto{\pgfqpoint{3.562507in}{1.864572in}}%
\pgfpathlineto{\pgfqpoint{3.554089in}{1.868764in}}%
\pgfpathlineto{\pgfqpoint{3.545656in}{1.873386in}}%
\pgfpathlineto{\pgfqpoint{3.537208in}{1.878448in}}%
\pgfpathlineto{\pgfqpoint{3.523253in}{1.886004in}}%
\pgfpathlineto{\pgfqpoint{3.509303in}{1.893587in}}%
\pgfpathlineto{\pgfqpoint{3.495356in}{1.901198in}}%
\pgfpathlineto{\pgfqpoint{3.481414in}{1.908837in}}%
\pgfpathlineto{\pgfqpoint{3.489896in}{1.903413in}}%
\pgfpathlineto{\pgfqpoint{3.498362in}{1.898433in}}%
\pgfpathlineto{\pgfqpoint{3.506812in}{1.893886in}}%
\pgfpathlineto{\pgfqpoint{3.515247in}{1.889763in}}%
\pgfpathclose%
\pgfusepath{fill}%
\end{pgfscope}%
\begin{pgfscope}%
\pgfpathrectangle{\pgfqpoint{1.150000in}{0.150000in}}{\pgfqpoint{5.700000in}{5.700000in}}%
\pgfusepath{clip}%
\pgfsetbuttcap%
\pgfsetroundjoin%
\definecolor{currentfill}{rgb}{0.169646,0.456262,0.558030}%
\pgfsetfillcolor{currentfill}%
\pgfsetfillopacity{0.700000}%
\pgfsetlinewidth{0.000000pt}%
\definecolor{currentstroke}{rgb}{0.000000,0.000000,0.000000}%
\pgfsetstrokecolor{currentstroke}%
\pgfsetdash{}{0pt}%
\pgfpathmoveto{\pgfqpoint{2.522552in}{2.572572in}}%
\pgfpathlineto{\pgfqpoint{2.536386in}{2.562013in}}%
\pgfpathlineto{\pgfqpoint{2.550223in}{2.551496in}}%
\pgfpathlineto{\pgfqpoint{2.564061in}{2.541020in}}%
\pgfpathlineto{\pgfqpoint{2.577901in}{2.530586in}}%
\pgfpathlineto{\pgfqpoint{2.568561in}{2.546522in}}%
\pgfpathlineto{\pgfqpoint{2.559188in}{2.563076in}}%
\pgfpathlineto{\pgfqpoint{2.549783in}{2.580258in}}%
\pgfpathlineto{\pgfqpoint{2.540343in}{2.598081in}}%
\pgfpathlineto{\pgfqpoint{2.526449in}{2.608932in}}%
\pgfpathlineto{\pgfqpoint{2.512556in}{2.619824in}}%
\pgfpathlineto{\pgfqpoint{2.498665in}{2.630758in}}%
\pgfpathlineto{\pgfqpoint{2.484775in}{2.641734in}}%
\pgfpathlineto{\pgfqpoint{2.494271in}{2.623487in}}%
\pgfpathlineto{\pgfqpoint{2.503732in}{2.605886in}}%
\pgfpathlineto{\pgfqpoint{2.513158in}{2.588918in}}%
\pgfpathlineto{\pgfqpoint{2.522552in}{2.572572in}}%
\pgfpathclose%
\pgfusepath{fill}%
\end{pgfscope}%
\begin{pgfscope}%
\pgfpathrectangle{\pgfqpoint{1.150000in}{0.150000in}}{\pgfqpoint{5.700000in}{5.700000in}}%
\pgfusepath{clip}%
\pgfsetbuttcap%
\pgfsetroundjoin%
\definecolor{currentfill}{rgb}{0.280267,0.073417,0.397163}%
\pgfsetfillcolor{currentfill}%
\pgfsetfillopacity{0.700000}%
\pgfsetlinewidth{0.000000pt}%
\definecolor{currentstroke}{rgb}{0.000000,0.000000,0.000000}%
\pgfsetstrokecolor{currentstroke}%
\pgfsetdash{}{0pt}%
\pgfpathmoveto{\pgfqpoint{5.103601in}{1.721213in}}%
\pgfpathlineto{\pgfqpoint{5.117892in}{1.719138in}}%
\pgfpathlineto{\pgfqpoint{5.132190in}{1.717086in}}%
\pgfpathlineto{\pgfqpoint{5.146497in}{1.715059in}}%
\pgfpathlineto{\pgfqpoint{5.160813in}{1.713055in}}%
\pgfpathlineto{\pgfqpoint{5.153065in}{1.701847in}}%
\pgfpathlineto{\pgfqpoint{5.145313in}{1.690668in}}%
\pgfpathlineto{\pgfqpoint{5.137556in}{1.679525in}}%
\pgfpathlineto{\pgfqpoint{5.129796in}{1.668422in}}%
\pgfpathlineto{\pgfqpoint{5.115474in}{1.670632in}}%
\pgfpathlineto{\pgfqpoint{5.101160in}{1.672867in}}%
\pgfpathlineto{\pgfqpoint{5.086855in}{1.675125in}}%
\pgfpathlineto{\pgfqpoint{5.072558in}{1.677407in}}%
\pgfpathlineto{\pgfqpoint{5.080325in}{1.688298in}}%
\pgfpathlineto{\pgfqpoint{5.088088in}{1.699233in}}%
\pgfpathlineto{\pgfqpoint{5.095847in}{1.710206in}}%
\pgfpathlineto{\pgfqpoint{5.103601in}{1.721213in}}%
\pgfpathclose%
\pgfusepath{fill}%
\end{pgfscope}%
\begin{pgfscope}%
\pgfpathrectangle{\pgfqpoint{1.150000in}{0.150000in}}{\pgfqpoint{5.700000in}{5.700000in}}%
\pgfusepath{clip}%
\pgfsetbuttcap%
\pgfsetroundjoin%
\definecolor{currentfill}{rgb}{0.263663,0.237631,0.518762}%
\pgfsetfillcolor{currentfill}%
\pgfsetfillopacity{0.700000}%
\pgfsetlinewidth{0.000000pt}%
\definecolor{currentstroke}{rgb}{0.000000,0.000000,0.000000}%
\pgfsetstrokecolor{currentstroke}%
\pgfsetdash{}{0pt}%
\pgfpathmoveto{\pgfqpoint{3.258895in}{2.034951in}}%
\pgfpathlineto{\pgfqpoint{3.272773in}{2.026849in}}%
\pgfpathlineto{\pgfqpoint{3.286655in}{2.018777in}}%
\pgfpathlineto{\pgfqpoint{3.300541in}{2.010735in}}%
\pgfpathlineto{\pgfqpoint{3.314430in}{2.002723in}}%
\pgfpathlineto{\pgfqpoint{3.305825in}{2.009702in}}%
\pgfpathlineto{\pgfqpoint{3.297201in}{2.017157in}}%
\pgfpathlineto{\pgfqpoint{3.288559in}{2.025098in}}%
\pgfpathlineto{\pgfqpoint{3.279897in}{2.033536in}}%
\pgfpathlineto{\pgfqpoint{3.265969in}{2.041922in}}%
\pgfpathlineto{\pgfqpoint{3.252045in}{2.050338in}}%
\pgfpathlineto{\pgfqpoint{3.238125in}{2.058783in}}%
\pgfpathlineto{\pgfqpoint{3.224208in}{2.067259in}}%
\pgfpathlineto{\pgfqpoint{3.232910in}{2.058441in}}%
\pgfpathlineto{\pgfqpoint{3.241591in}{2.050124in}}%
\pgfpathlineto{\pgfqpoint{3.250253in}{2.042298in}}%
\pgfpathlineto{\pgfqpoint{3.258895in}{2.034951in}}%
\pgfpathclose%
\pgfusepath{fill}%
\end{pgfscope}%
\begin{pgfscope}%
\pgfpathrectangle{\pgfqpoint{1.150000in}{0.150000in}}{\pgfqpoint{5.700000in}{5.700000in}}%
\pgfusepath{clip}%
\pgfsetbuttcap%
\pgfsetroundjoin%
\definecolor{currentfill}{rgb}{0.225863,0.330805,0.547314}%
\pgfsetfillcolor{currentfill}%
\pgfsetfillopacity{0.700000}%
\pgfsetlinewidth{0.000000pt}%
\definecolor{currentstroke}{rgb}{0.000000,0.000000,0.000000}%
\pgfsetstrokecolor{currentstroke}%
\pgfsetdash{}{0pt}%
\pgfpathmoveto{\pgfqpoint{2.946583in}{2.243334in}}%
\pgfpathlineto{\pgfqpoint{2.960434in}{2.234222in}}%
\pgfpathlineto{\pgfqpoint{2.974289in}{2.225143in}}%
\pgfpathlineto{\pgfqpoint{2.988146in}{2.216098in}}%
\pgfpathlineto{\pgfqpoint{3.002006in}{2.207086in}}%
\pgfpathlineto{\pgfqpoint{2.993115in}{2.217956in}}%
\pgfpathlineto{\pgfqpoint{2.984200in}{2.229365in}}%
\pgfpathlineto{\pgfqpoint{2.975261in}{2.241325in}}%
\pgfpathlineto{\pgfqpoint{2.966297in}{2.253846in}}%
\pgfpathlineto{\pgfqpoint{2.952391in}{2.263251in}}%
\pgfpathlineto{\pgfqpoint{2.938489in}{2.272689in}}%
\pgfpathlineto{\pgfqpoint{2.924589in}{2.282160in}}%
\pgfpathlineto{\pgfqpoint{2.910692in}{2.291666in}}%
\pgfpathlineto{\pgfqpoint{2.919703in}{2.278745in}}%
\pgfpathlineto{\pgfqpoint{2.928689in}{2.266390in}}%
\pgfpathlineto{\pgfqpoint{2.937648in}{2.254590in}}%
\pgfpathlineto{\pgfqpoint{2.946583in}{2.243334in}}%
\pgfpathclose%
\pgfusepath{fill}%
\end{pgfscope}%
\begin{pgfscope}%
\pgfpathrectangle{\pgfqpoint{1.150000in}{0.150000in}}{\pgfqpoint{5.700000in}{5.700000in}}%
\pgfusepath{clip}%
\pgfsetbuttcap%
\pgfsetroundjoin%
\definecolor{currentfill}{rgb}{0.277018,0.050344,0.375715}%
\pgfsetfillcolor{currentfill}%
\pgfsetfillopacity{0.700000}%
\pgfsetlinewidth{0.000000pt}%
\definecolor{currentstroke}{rgb}{0.000000,0.000000,0.000000}%
\pgfsetstrokecolor{currentstroke}%
\pgfsetdash{}{0pt}%
\pgfpathmoveto{\pgfqpoint{4.116357in}{1.659909in}}%
\pgfpathlineto{\pgfqpoint{4.130375in}{1.654552in}}%
\pgfpathlineto{\pgfqpoint{4.144398in}{1.649219in}}%
\pgfpathlineto{\pgfqpoint{4.158428in}{1.643911in}}%
\pgfpathlineto{\pgfqpoint{4.172463in}{1.638629in}}%
\pgfpathlineto{\pgfqpoint{4.164412in}{1.635316in}}%
\pgfpathlineto{\pgfqpoint{4.156352in}{1.632284in}}%
\pgfpathlineto{\pgfqpoint{4.148286in}{1.629540in}}%
\pgfpathlineto{\pgfqpoint{4.140212in}{1.627094in}}%
\pgfpathlineto{\pgfqpoint{4.126156in}{1.632689in}}%
\pgfpathlineto{\pgfqpoint{4.112106in}{1.638309in}}%
\pgfpathlineto{\pgfqpoint{4.098062in}{1.643954in}}%
\pgfpathlineto{\pgfqpoint{4.084023in}{1.649624in}}%
\pgfpathlineto{\pgfqpoint{4.092118in}{1.651753in}}%
\pgfpathlineto{\pgfqpoint{4.100206in}{1.654182in}}%
\pgfpathlineto{\pgfqpoint{4.108285in}{1.656904in}}%
\pgfpathlineto{\pgfqpoint{4.116357in}{1.659909in}}%
\pgfpathclose%
\pgfusepath{fill}%
\end{pgfscope}%
\begin{pgfscope}%
\pgfpathrectangle{\pgfqpoint{1.150000in}{0.150000in}}{\pgfqpoint{5.700000in}{5.700000in}}%
\pgfusepath{clip}%
\pgfsetbuttcap%
\pgfsetroundjoin%
\definecolor{currentfill}{rgb}{0.271305,0.019942,0.347269}%
\pgfsetfillcolor{currentfill}%
\pgfsetfillopacity{0.700000}%
\pgfsetlinewidth{0.000000pt}%
\definecolor{currentstroke}{rgb}{0.000000,0.000000,0.000000}%
\pgfsetstrokecolor{currentstroke}%
\pgfsetdash{}{0pt}%
\pgfpathmoveto{\pgfqpoint{4.549616in}{1.614048in}}%
\pgfpathlineto{\pgfqpoint{4.563739in}{1.610135in}}%
\pgfpathlineto{\pgfqpoint{4.577870in}{1.606246in}}%
\pgfpathlineto{\pgfqpoint{4.592007in}{1.602381in}}%
\pgfpathlineto{\pgfqpoint{4.606151in}{1.598540in}}%
\pgfpathlineto{\pgfqpoint{4.598260in}{1.590859in}}%
\pgfpathlineto{\pgfqpoint{4.590365in}{1.583349in}}%
\pgfpathlineto{\pgfqpoint{4.582465in}{1.576016in}}%
\pgfpathlineto{\pgfqpoint{4.574561in}{1.568868in}}%
\pgfpathlineto{\pgfqpoint{4.560404in}{1.572981in}}%
\pgfpathlineto{\pgfqpoint{4.546254in}{1.577118in}}%
\pgfpathlineto{\pgfqpoint{4.532111in}{1.581280in}}%
\pgfpathlineto{\pgfqpoint{4.517975in}{1.585465in}}%
\pgfpathlineto{\pgfqpoint{4.525892in}{1.592336in}}%
\pgfpathlineto{\pgfqpoint{4.533805in}{1.599395in}}%
\pgfpathlineto{\pgfqpoint{4.541713in}{1.606635in}}%
\pgfpathlineto{\pgfqpoint{4.549616in}{1.614048in}}%
\pgfpathclose%
\pgfusepath{fill}%
\end{pgfscope}%
\begin{pgfscope}%
\pgfpathrectangle{\pgfqpoint{1.150000in}{0.150000in}}{\pgfqpoint{5.700000in}{5.700000in}}%
\pgfusepath{clip}%
\pgfsetbuttcap%
\pgfsetroundjoin%
\definecolor{currentfill}{rgb}{0.278791,0.062145,0.386592}%
\pgfsetfillcolor{currentfill}%
\pgfsetfillopacity{0.700000}%
\pgfsetlinewidth{0.000000pt}%
\definecolor{currentstroke}{rgb}{0.000000,0.000000,0.000000}%
\pgfsetstrokecolor{currentstroke}%
\pgfsetdash{}{0pt}%
\pgfpathmoveto{\pgfqpoint{5.015450in}{1.686777in}}%
\pgfpathlineto{\pgfqpoint{5.029715in}{1.684398in}}%
\pgfpathlineto{\pgfqpoint{5.043988in}{1.682044in}}%
\pgfpathlineto{\pgfqpoint{5.058269in}{1.679714in}}%
\pgfpathlineto{\pgfqpoint{5.072558in}{1.677407in}}%
\pgfpathlineto{\pgfqpoint{5.064787in}{1.666565in}}%
\pgfpathlineto{\pgfqpoint{5.057012in}{1.655776in}}%
\pgfpathlineto{\pgfqpoint{5.049233in}{1.645046in}}%
\pgfpathlineto{\pgfqpoint{5.041450in}{1.634381in}}%
\pgfpathlineto{\pgfqpoint{5.027154in}{1.636908in}}%
\pgfpathlineto{\pgfqpoint{5.012866in}{1.639458in}}%
\pgfpathlineto{\pgfqpoint{4.998586in}{1.642033in}}%
\pgfpathlineto{\pgfqpoint{4.984313in}{1.644631in}}%
\pgfpathlineto{\pgfqpoint{4.992104in}{1.655071in}}%
\pgfpathlineto{\pgfqpoint{4.999890in}{1.665579in}}%
\pgfpathlineto{\pgfqpoint{5.007672in}{1.676149in}}%
\pgfpathlineto{\pgfqpoint{5.015450in}{1.686777in}}%
\pgfpathclose%
\pgfusepath{fill}%
\end{pgfscope}%
\begin{pgfscope}%
\pgfpathrectangle{\pgfqpoint{1.150000in}{0.150000in}}{\pgfqpoint{5.700000in}{5.700000in}}%
\pgfusepath{clip}%
\pgfsetbuttcap%
\pgfsetroundjoin%
\definecolor{currentfill}{rgb}{0.283091,0.110553,0.431554}%
\pgfsetfillcolor{currentfill}%
\pgfsetfillopacity{0.700000}%
\pgfsetlinewidth{0.000000pt}%
\definecolor{currentstroke}{rgb}{0.000000,0.000000,0.000000}%
\pgfsetstrokecolor{currentstroke}%
\pgfsetdash{}{0pt}%
\pgfpathmoveto{\pgfqpoint{3.771467in}{1.769542in}}%
\pgfpathlineto{\pgfqpoint{3.785423in}{1.763034in}}%
\pgfpathlineto{\pgfqpoint{3.799383in}{1.756552in}}%
\pgfpathlineto{\pgfqpoint{3.813348in}{1.750096in}}%
\pgfpathlineto{\pgfqpoint{3.827319in}{1.743667in}}%
\pgfpathlineto{\pgfqpoint{3.819081in}{1.744511in}}%
\pgfpathlineto{\pgfqpoint{3.810833in}{1.745723in}}%
\pgfpathlineto{\pgfqpoint{3.802573in}{1.747312in}}%
\pgfpathlineto{\pgfqpoint{3.794301in}{1.749286in}}%
\pgfpathlineto{\pgfqpoint{3.780303in}{1.756057in}}%
\pgfpathlineto{\pgfqpoint{3.766311in}{1.762854in}}%
\pgfpathlineto{\pgfqpoint{3.752322in}{1.769677in}}%
\pgfpathlineto{\pgfqpoint{3.738339in}{1.776526in}}%
\pgfpathlineto{\pgfqpoint{3.746639in}{1.774205in}}%
\pgfpathlineto{\pgfqpoint{3.754927in}{1.772274in}}%
\pgfpathlineto{\pgfqpoint{3.763203in}{1.770722in}}%
\pgfpathlineto{\pgfqpoint{3.771467in}{1.769542in}}%
\pgfpathclose%
\pgfusepath{fill}%
\end{pgfscope}%
\begin{pgfscope}%
\pgfpathrectangle{\pgfqpoint{1.150000in}{0.150000in}}{\pgfqpoint{5.700000in}{5.700000in}}%
\pgfusepath{clip}%
\pgfsetbuttcap%
\pgfsetroundjoin%
\definecolor{currentfill}{rgb}{0.174274,0.445044,0.557792}%
\pgfsetfillcolor{currentfill}%
\pgfsetfillopacity{0.700000}%
\pgfsetlinewidth{0.000000pt}%
\definecolor{currentstroke}{rgb}{0.000000,0.000000,0.000000}%
\pgfsetstrokecolor{currentstroke}%
\pgfsetdash{}{0pt}%
\pgfpathmoveto{\pgfqpoint{2.577901in}{2.530586in}}%
\pgfpathlineto{\pgfqpoint{2.591743in}{2.520192in}}%
\pgfpathlineto{\pgfqpoint{2.605586in}{2.509838in}}%
\pgfpathlineto{\pgfqpoint{2.619432in}{2.499525in}}%
\pgfpathlineto{\pgfqpoint{2.633279in}{2.489251in}}%
\pgfpathlineto{\pgfqpoint{2.623991in}{2.504779in}}%
\pgfpathlineto{\pgfqpoint{2.614672in}{2.520919in}}%
\pgfpathlineto{\pgfqpoint{2.605321in}{2.537684in}}%
\pgfpathlineto{\pgfqpoint{2.595937in}{2.555085in}}%
\pgfpathlineto{\pgfqpoint{2.582036in}{2.565774in}}%
\pgfpathlineto{\pgfqpoint{2.568137in}{2.576503in}}%
\pgfpathlineto{\pgfqpoint{2.554239in}{2.587272in}}%
\pgfpathlineto{\pgfqpoint{2.540343in}{2.598081in}}%
\pgfpathlineto{\pgfqpoint{2.549783in}{2.580258in}}%
\pgfpathlineto{\pgfqpoint{2.559188in}{2.563076in}}%
\pgfpathlineto{\pgfqpoint{2.568561in}{2.546522in}}%
\pgfpathlineto{\pgfqpoint{2.577901in}{2.530586in}}%
\pgfpathclose%
\pgfusepath{fill}%
\end{pgfscope}%
\begin{pgfscope}%
\pgfpathrectangle{\pgfqpoint{1.150000in}{0.150000in}}{\pgfqpoint{5.700000in}{5.700000in}}%
\pgfusepath{clip}%
\pgfsetbuttcap%
\pgfsetroundjoin%
\definecolor{currentfill}{rgb}{0.271305,0.019942,0.347269}%
\pgfsetfillcolor{currentfill}%
\pgfsetfillopacity{0.700000}%
\pgfsetlinewidth{0.000000pt}%
\definecolor{currentstroke}{rgb}{0.000000,0.000000,0.000000}%
\pgfsetstrokecolor{currentstroke}%
\pgfsetdash{}{0pt}%
\pgfpathmoveto{\pgfqpoint{4.694272in}{1.616749in}}%
\pgfpathlineto{\pgfqpoint{4.708441in}{1.613287in}}%
\pgfpathlineto{\pgfqpoint{4.722617in}{1.609850in}}%
\pgfpathlineto{\pgfqpoint{4.736800in}{1.606436in}}%
\pgfpathlineto{\pgfqpoint{4.750990in}{1.603047in}}%
\pgfpathlineto{\pgfqpoint{4.743139in}{1.594242in}}%
\pgfpathlineto{\pgfqpoint{4.735283in}{1.585575in}}%
\pgfpathlineto{\pgfqpoint{4.727423in}{1.577052in}}%
\pgfpathlineto{\pgfqpoint{4.719560in}{1.568679in}}%
\pgfpathlineto{\pgfqpoint{4.705359in}{1.572328in}}%
\pgfpathlineto{\pgfqpoint{4.691165in}{1.576000in}}%
\pgfpathlineto{\pgfqpoint{4.676978in}{1.579697in}}%
\pgfpathlineto{\pgfqpoint{4.662799in}{1.583418in}}%
\pgfpathlineto{\pgfqpoint{4.670673in}{1.591526in}}%
\pgfpathlineto{\pgfqpoint{4.678544in}{1.599789in}}%
\pgfpathlineto{\pgfqpoint{4.686410in}{1.608198in}}%
\pgfpathlineto{\pgfqpoint{4.694272in}{1.616749in}}%
\pgfpathclose%
\pgfusepath{fill}%
\end{pgfscope}%
\begin{pgfscope}%
\pgfpathrectangle{\pgfqpoint{1.150000in}{0.150000in}}{\pgfqpoint{5.700000in}{5.700000in}}%
\pgfusepath{clip}%
\pgfsetbuttcap%
\pgfsetroundjoin%
\definecolor{currentfill}{rgb}{0.280267,0.073417,0.397163}%
\pgfsetfillcolor{currentfill}%
\pgfsetfillopacity{0.700000}%
\pgfsetlinewidth{0.000000pt}%
\definecolor{currentstroke}{rgb}{0.000000,0.000000,0.000000}%
\pgfsetstrokecolor{currentstroke}%
\pgfsetdash{}{0pt}%
\pgfpathmoveto{\pgfqpoint{3.971918in}{1.695888in}}%
\pgfpathlineto{\pgfqpoint{3.985912in}{1.690016in}}%
\pgfpathlineto{\pgfqpoint{3.999911in}{1.684170in}}%
\pgfpathlineto{\pgfqpoint{4.013916in}{1.678350in}}%
\pgfpathlineto{\pgfqpoint{4.027926in}{1.672554in}}%
\pgfpathlineto{\pgfqpoint{4.019801in}{1.671055in}}%
\pgfpathlineto{\pgfqpoint{4.011666in}{1.669878in}}%
\pgfpathlineto{\pgfqpoint{4.003523in}{1.669030in}}%
\pgfpathlineto{\pgfqpoint{3.995371in}{1.668521in}}%
\pgfpathlineto{\pgfqpoint{3.981338in}{1.674643in}}%
\pgfpathlineto{\pgfqpoint{3.967310in}{1.680790in}}%
\pgfpathlineto{\pgfqpoint{3.953287in}{1.686962in}}%
\pgfpathlineto{\pgfqpoint{3.939269in}{1.693160in}}%
\pgfpathlineto{\pgfqpoint{3.947446in}{1.693338in}}%
\pgfpathlineto{\pgfqpoint{3.955612in}{1.693857in}}%
\pgfpathlineto{\pgfqpoint{3.963770in}{1.694710in}}%
\pgfpathlineto{\pgfqpoint{3.971918in}{1.695888in}}%
\pgfpathclose%
\pgfusepath{fill}%
\end{pgfscope}%
\begin{pgfscope}%
\pgfpathrectangle{\pgfqpoint{1.150000in}{0.150000in}}{\pgfqpoint{5.700000in}{5.700000in}}%
\pgfusepath{clip}%
\pgfsetbuttcap%
\pgfsetroundjoin%
\definecolor{currentfill}{rgb}{0.282884,0.135920,0.453427}%
\pgfsetfillcolor{currentfill}%
\pgfsetfillopacity{0.700000}%
\pgfsetlinewidth{0.000000pt}%
\definecolor{currentstroke}{rgb}{0.000000,0.000000,0.000000}%
\pgfsetstrokecolor{currentstroke}%
\pgfsetdash{}{0pt}%
\pgfpathmoveto{\pgfqpoint{5.425666in}{1.832562in}}%
\pgfpathlineto{\pgfqpoint{5.440076in}{1.831414in}}%
\pgfpathlineto{\pgfqpoint{5.454496in}{1.830291in}}%
\pgfpathlineto{\pgfqpoint{5.468925in}{1.829192in}}%
\pgfpathlineto{\pgfqpoint{5.461256in}{1.817368in}}%
\pgfpathlineto{\pgfqpoint{5.453582in}{1.805510in}}%
\pgfpathlineto{\pgfqpoint{5.445902in}{1.793620in}}%
\pgfpathlineto{\pgfqpoint{5.438217in}{1.781703in}}%
\pgfpathlineto{\pgfqpoint{5.423783in}{1.782970in}}%
\pgfpathlineto{\pgfqpoint{5.409359in}{1.784260in}}%
\pgfpathlineto{\pgfqpoint{5.394942in}{1.785575in}}%
\pgfpathlineto{\pgfqpoint{5.402631in}{1.797363in}}%
\pgfpathlineto{\pgfqpoint{5.410315in}{1.809126in}}%
\pgfpathlineto{\pgfqpoint{5.417993in}{1.820860in}}%
\pgfpathlineto{\pgfqpoint{5.425666in}{1.832562in}}%
\pgfpathclose%
\pgfusepath{fill}%
\end{pgfscope}%
\begin{pgfscope}%
\pgfpathrectangle{\pgfqpoint{1.150000in}{0.150000in}}{\pgfqpoint{5.700000in}{5.700000in}}%
\pgfusepath{clip}%
\pgfsetbuttcap%
\pgfsetroundjoin%
\definecolor{currentfill}{rgb}{0.276022,0.044167,0.370164}%
\pgfsetfillcolor{currentfill}%
\pgfsetfillopacity{0.700000}%
\pgfsetlinewidth{0.000000pt}%
\definecolor{currentstroke}{rgb}{0.000000,0.000000,0.000000}%
\pgfsetstrokecolor{currentstroke}%
\pgfsetdash{}{0pt}%
\pgfpathmoveto{\pgfqpoint{4.927304in}{1.655265in}}%
\pgfpathlineto{\pgfqpoint{4.941545in}{1.652571in}}%
\pgfpathlineto{\pgfqpoint{4.955793in}{1.649900in}}%
\pgfpathlineto{\pgfqpoint{4.970049in}{1.647254in}}%
\pgfpathlineto{\pgfqpoint{4.984313in}{1.644631in}}%
\pgfpathlineto{\pgfqpoint{4.976519in}{1.634265in}}%
\pgfpathlineto{\pgfqpoint{4.968721in}{1.623978in}}%
\pgfpathlineto{\pgfqpoint{4.960920in}{1.613776in}}%
\pgfpathlineto{\pgfqpoint{4.953114in}{1.603664in}}%
\pgfpathlineto{\pgfqpoint{4.938842in}{1.606519in}}%
\pgfpathlineto{\pgfqpoint{4.924578in}{1.609399in}}%
\pgfpathlineto{\pgfqpoint{4.910322in}{1.612303in}}%
\pgfpathlineto{\pgfqpoint{4.896073in}{1.615230in}}%
\pgfpathlineto{\pgfqpoint{4.903886in}{1.625104in}}%
\pgfpathlineto{\pgfqpoint{4.911696in}{1.635071in}}%
\pgfpathlineto{\pgfqpoint{4.919502in}{1.645127in}}%
\pgfpathlineto{\pgfqpoint{4.927304in}{1.655265in}}%
\pgfpathclose%
\pgfusepath{fill}%
\end{pgfscope}%
\begin{pgfscope}%
\pgfpathrectangle{\pgfqpoint{1.150000in}{0.150000in}}{\pgfqpoint{5.700000in}{5.700000in}}%
\pgfusepath{clip}%
\pgfsetbuttcap%
\pgfsetroundjoin%
\definecolor{currentfill}{rgb}{0.231674,0.318106,0.544834}%
\pgfsetfillcolor{currentfill}%
\pgfsetfillopacity{0.700000}%
\pgfsetlinewidth{0.000000pt}%
\definecolor{currentstroke}{rgb}{0.000000,0.000000,0.000000}%
\pgfsetstrokecolor{currentstroke}%
\pgfsetdash{}{0pt}%
\pgfpathmoveto{\pgfqpoint{3.002006in}{2.207086in}}%
\pgfpathlineto{\pgfqpoint{3.015869in}{2.198107in}}%
\pgfpathlineto{\pgfqpoint{3.029736in}{2.189161in}}%
\pgfpathlineto{\pgfqpoint{3.043605in}{2.180248in}}%
\pgfpathlineto{\pgfqpoint{3.057478in}{2.171367in}}%
\pgfpathlineto{\pgfqpoint{3.048630in}{2.181850in}}%
\pgfpathlineto{\pgfqpoint{3.039760in}{2.192870in}}%
\pgfpathlineto{\pgfqpoint{3.030865in}{2.204435in}}%
\pgfpathlineto{\pgfqpoint{3.021947in}{2.216558in}}%
\pgfpathlineto{\pgfqpoint{3.008030in}{2.225831in}}%
\pgfpathlineto{\pgfqpoint{2.994116in}{2.235137in}}%
\pgfpathlineto{\pgfqpoint{2.980205in}{2.244475in}}%
\pgfpathlineto{\pgfqpoint{2.966297in}{2.253846in}}%
\pgfpathlineto{\pgfqpoint{2.975261in}{2.241325in}}%
\pgfpathlineto{\pgfqpoint{2.984200in}{2.229365in}}%
\pgfpathlineto{\pgfqpoint{2.993115in}{2.217956in}}%
\pgfpathlineto{\pgfqpoint{3.002006in}{2.207086in}}%
\pgfpathclose%
\pgfusepath{fill}%
\end{pgfscope}%
\begin{pgfscope}%
\pgfpathrectangle{\pgfqpoint{1.150000in}{0.150000in}}{\pgfqpoint{5.700000in}{5.700000in}}%
\pgfusepath{clip}%
\pgfsetbuttcap%
\pgfsetroundjoin%
\definecolor{currentfill}{rgb}{0.266580,0.228262,0.514349}%
\pgfsetfillcolor{currentfill}%
\pgfsetfillopacity{0.700000}%
\pgfsetlinewidth{0.000000pt}%
\definecolor{currentstroke}{rgb}{0.000000,0.000000,0.000000}%
\pgfsetstrokecolor{currentstroke}%
\pgfsetdash{}{0pt}%
\pgfpathmoveto{\pgfqpoint{3.314430in}{2.002723in}}%
\pgfpathlineto{\pgfqpoint{3.328323in}{1.994740in}}%
\pgfpathlineto{\pgfqpoint{3.342221in}{1.986787in}}%
\pgfpathlineto{\pgfqpoint{3.356122in}{1.978862in}}%
\pgfpathlineto{\pgfqpoint{3.370027in}{1.970967in}}%
\pgfpathlineto{\pgfqpoint{3.361458in}{1.977579in}}%
\pgfpathlineto{\pgfqpoint{3.352872in}{1.984662in}}%
\pgfpathlineto{\pgfqpoint{3.344268in}{1.992228in}}%
\pgfpathlineto{\pgfqpoint{3.335644in}{2.000287in}}%
\pgfpathlineto{\pgfqpoint{3.321702in}{2.008555in}}%
\pgfpathlineto{\pgfqpoint{3.307763in}{2.016853in}}%
\pgfpathlineto{\pgfqpoint{3.293828in}{2.025180in}}%
\pgfpathlineto{\pgfqpoint{3.279897in}{2.033536in}}%
\pgfpathlineto{\pgfqpoint{3.288559in}{2.025098in}}%
\pgfpathlineto{\pgfqpoint{3.297201in}{2.017157in}}%
\pgfpathlineto{\pgfqpoint{3.305825in}{2.009702in}}%
\pgfpathlineto{\pgfqpoint{3.314430in}{2.002723in}}%
\pgfpathclose%
\pgfusepath{fill}%
\end{pgfscope}%
\begin{pgfscope}%
\pgfpathrectangle{\pgfqpoint{1.150000in}{0.150000in}}{\pgfqpoint{5.700000in}{5.700000in}}%
\pgfusepath{clip}%
\pgfsetbuttcap%
\pgfsetroundjoin%
\definecolor{currentfill}{rgb}{0.280868,0.160771,0.472899}%
\pgfsetfillcolor{currentfill}%
\pgfsetfillopacity{0.700000}%
\pgfsetlinewidth{0.000000pt}%
\definecolor{currentstroke}{rgb}{0.000000,0.000000,0.000000}%
\pgfsetstrokecolor{currentstroke}%
\pgfsetdash{}{0pt}%
\pgfpathmoveto{\pgfqpoint{3.570910in}{1.860800in}}%
\pgfpathlineto{\pgfqpoint{3.584837in}{1.853628in}}%
\pgfpathlineto{\pgfqpoint{3.598769in}{1.846484in}}%
\pgfpathlineto{\pgfqpoint{3.612705in}{1.839367in}}%
\pgfpathlineto{\pgfqpoint{3.626645in}{1.832277in}}%
\pgfpathlineto{\pgfqpoint{3.618273in}{1.835699in}}%
\pgfpathlineto{\pgfqpoint{3.609887in}{1.839536in}}%
\pgfpathlineto{\pgfqpoint{3.601487in}{1.843799in}}%
\pgfpathlineto{\pgfqpoint{3.593072in}{1.848499in}}%
\pgfpathlineto{\pgfqpoint{3.579099in}{1.855945in}}%
\pgfpathlineto{\pgfqpoint{3.565131in}{1.863419in}}%
\pgfpathlineto{\pgfqpoint{3.551168in}{1.870920in}}%
\pgfpathlineto{\pgfqpoint{3.537208in}{1.878448in}}%
\pgfpathlineto{\pgfqpoint{3.545656in}{1.873386in}}%
\pgfpathlineto{\pgfqpoint{3.554089in}{1.868764in}}%
\pgfpathlineto{\pgfqpoint{3.562507in}{1.864572in}}%
\pgfpathlineto{\pgfqpoint{3.570910in}{1.860800in}}%
\pgfpathclose%
\pgfusepath{fill}%
\end{pgfscope}%
\begin{pgfscope}%
\pgfpathrectangle{\pgfqpoint{1.150000in}{0.150000in}}{\pgfqpoint{5.700000in}{5.700000in}}%
\pgfusepath{clip}%
\pgfsetbuttcap%
\pgfsetroundjoin%
\definecolor{currentfill}{rgb}{0.272594,0.025563,0.353093}%
\pgfsetfillcolor{currentfill}%
\pgfsetfillopacity{0.700000}%
\pgfsetlinewidth{0.000000pt}%
\definecolor{currentstroke}{rgb}{0.000000,0.000000,0.000000}%
\pgfsetstrokecolor{currentstroke}%
\pgfsetdash{}{0pt}%
\pgfpathmoveto{\pgfqpoint{4.316959in}{1.615542in}}%
\pgfpathlineto{\pgfqpoint{4.331032in}{1.610780in}}%
\pgfpathlineto{\pgfqpoint{4.345111in}{1.606042in}}%
\pgfpathlineto{\pgfqpoint{4.359196in}{1.601329in}}%
\pgfpathlineto{\pgfqpoint{4.373288in}{1.596640in}}%
\pgfpathlineto{\pgfqpoint{4.365315in}{1.591404in}}%
\pgfpathlineto{\pgfqpoint{4.357337in}{1.586406in}}%
\pgfpathlineto{\pgfqpoint{4.349352in}{1.581655in}}%
\pgfpathlineto{\pgfqpoint{4.341362in}{1.577158in}}%
\pgfpathlineto{\pgfqpoint{4.327253in}{1.582146in}}%
\pgfpathlineto{\pgfqpoint{4.313151in}{1.587158in}}%
\pgfpathlineto{\pgfqpoint{4.299054in}{1.592195in}}%
\pgfpathlineto{\pgfqpoint{4.284964in}{1.597256in}}%
\pgfpathlineto{\pgfqpoint{4.292972in}{1.601449in}}%
\pgfpathlineto{\pgfqpoint{4.300974in}{1.605899in}}%
\pgfpathlineto{\pgfqpoint{4.308969in}{1.610600in}}%
\pgfpathlineto{\pgfqpoint{4.316959in}{1.615542in}}%
\pgfpathclose%
\pgfusepath{fill}%
\end{pgfscope}%
\begin{pgfscope}%
\pgfpathrectangle{\pgfqpoint{1.150000in}{0.150000in}}{\pgfqpoint{5.700000in}{5.700000in}}%
\pgfusepath{clip}%
\pgfsetbuttcap%
\pgfsetroundjoin%
\definecolor{currentfill}{rgb}{0.283197,0.115680,0.436115}%
\pgfsetfillcolor{currentfill}%
\pgfsetfillopacity{0.700000}%
\pgfsetlinewidth{0.000000pt}%
\definecolor{currentstroke}{rgb}{0.000000,0.000000,0.000000}%
\pgfsetstrokecolor{currentstroke}%
\pgfsetdash{}{0pt}%
\pgfpathmoveto{\pgfqpoint{5.337367in}{1.791077in}}%
\pgfpathlineto{\pgfqpoint{5.351748in}{1.789666in}}%
\pgfpathlineto{\pgfqpoint{5.366137in}{1.788278in}}%
\pgfpathlineto{\pgfqpoint{5.380535in}{1.786915in}}%
\pgfpathlineto{\pgfqpoint{5.394942in}{1.785575in}}%
\pgfpathlineto{\pgfqpoint{5.387249in}{1.773767in}}%
\pgfpathlineto{\pgfqpoint{5.379550in}{1.761943in}}%
\pgfpathlineto{\pgfqpoint{5.371846in}{1.750106in}}%
\pgfpathlineto{\pgfqpoint{5.364137in}{1.738261in}}%
\pgfpathlineto{\pgfqpoint{5.349725in}{1.739781in}}%
\pgfpathlineto{\pgfqpoint{5.335321in}{1.741326in}}%
\pgfpathlineto{\pgfqpoint{5.320926in}{1.742894in}}%
\pgfpathlineto{\pgfqpoint{5.306540in}{1.744486in}}%
\pgfpathlineto{\pgfqpoint{5.314254in}{1.756145in}}%
\pgfpathlineto{\pgfqpoint{5.321963in}{1.767799in}}%
\pgfpathlineto{\pgfqpoint{5.329668in}{1.779445in}}%
\pgfpathlineto{\pgfqpoint{5.337367in}{1.791077in}}%
\pgfpathclose%
\pgfusepath{fill}%
\end{pgfscope}%
\begin{pgfscope}%
\pgfpathrectangle{\pgfqpoint{1.150000in}{0.150000in}}{\pgfqpoint{5.700000in}{5.700000in}}%
\pgfusepath{clip}%
\pgfsetbuttcap%
\pgfsetroundjoin%
\definecolor{currentfill}{rgb}{0.271305,0.019942,0.347269}%
\pgfsetfillcolor{currentfill}%
\pgfsetfillopacity{0.700000}%
\pgfsetlinewidth{0.000000pt}%
\definecolor{currentstroke}{rgb}{0.000000,0.000000,0.000000}%
\pgfsetstrokecolor{currentstroke}%
\pgfsetdash{}{0pt}%
\pgfpathmoveto{\pgfqpoint{4.461496in}{1.602448in}}%
\pgfpathlineto{\pgfqpoint{4.475606in}{1.598166in}}%
\pgfpathlineto{\pgfqpoint{4.489722in}{1.593908in}}%
\pgfpathlineto{\pgfqpoint{4.503845in}{1.589674in}}%
\pgfpathlineto{\pgfqpoint{4.517975in}{1.585465in}}%
\pgfpathlineto{\pgfqpoint{4.510052in}{1.578789in}}%
\pgfpathlineto{\pgfqpoint{4.502125in}{1.572315in}}%
\pgfpathlineto{\pgfqpoint{4.494193in}{1.566050in}}%
\pgfpathlineto{\pgfqpoint{4.486256in}{1.560002in}}%
\pgfpathlineto{\pgfqpoint{4.472113in}{1.564497in}}%
\pgfpathlineto{\pgfqpoint{4.457975in}{1.569016in}}%
\pgfpathlineto{\pgfqpoint{4.443844in}{1.573559in}}%
\pgfpathlineto{\pgfqpoint{4.429720in}{1.578127in}}%
\pgfpathlineto{\pgfqpoint{4.437672in}{1.583884in}}%
\pgfpathlineto{\pgfqpoint{4.445619in}{1.589862in}}%
\pgfpathlineto{\pgfqpoint{4.453560in}{1.596052in}}%
\pgfpathlineto{\pgfqpoint{4.461496in}{1.602448in}}%
\pgfpathclose%
\pgfusepath{fill}%
\end{pgfscope}%
\begin{pgfscope}%
\pgfpathrectangle{\pgfqpoint{1.150000in}{0.150000in}}{\pgfqpoint{5.700000in}{5.700000in}}%
\pgfusepath{clip}%
\pgfsetbuttcap%
\pgfsetroundjoin%
\definecolor{currentfill}{rgb}{0.180629,0.429975,0.557282}%
\pgfsetfillcolor{currentfill}%
\pgfsetfillopacity{0.700000}%
\pgfsetlinewidth{0.000000pt}%
\definecolor{currentstroke}{rgb}{0.000000,0.000000,0.000000}%
\pgfsetstrokecolor{currentstroke}%
\pgfsetdash{}{0pt}%
\pgfpathmoveto{\pgfqpoint{2.633279in}{2.489251in}}%
\pgfpathlineto{\pgfqpoint{2.647128in}{2.479016in}}%
\pgfpathlineto{\pgfqpoint{2.660980in}{2.468821in}}%
\pgfpathlineto{\pgfqpoint{2.674833in}{2.458665in}}%
\pgfpathlineto{\pgfqpoint{2.688688in}{2.448547in}}%
\pgfpathlineto{\pgfqpoint{2.679453in}{2.463668in}}%
\pgfpathlineto{\pgfqpoint{2.670186in}{2.479396in}}%
\pgfpathlineto{\pgfqpoint{2.660888in}{2.495744in}}%
\pgfpathlineto{\pgfqpoint{2.651558in}{2.512724in}}%
\pgfpathlineto{\pgfqpoint{2.637650in}{2.523256in}}%
\pgfpathlineto{\pgfqpoint{2.623744in}{2.533827in}}%
\pgfpathlineto{\pgfqpoint{2.609839in}{2.544436in}}%
\pgfpathlineto{\pgfqpoint{2.595937in}{2.555085in}}%
\pgfpathlineto{\pgfqpoint{2.605321in}{2.537684in}}%
\pgfpathlineto{\pgfqpoint{2.614672in}{2.520919in}}%
\pgfpathlineto{\pgfqpoint{2.623991in}{2.504779in}}%
\pgfpathlineto{\pgfqpoint{2.633279in}{2.489251in}}%
\pgfpathclose%
\pgfusepath{fill}%
\end{pgfscope}%
\begin{pgfscope}%
\pgfpathrectangle{\pgfqpoint{1.150000in}{0.150000in}}{\pgfqpoint{5.700000in}{5.700000in}}%
\pgfusepath{clip}%
\pgfsetbuttcap%
\pgfsetroundjoin%
\definecolor{currentfill}{rgb}{0.282656,0.100196,0.422160}%
\pgfsetfillcolor{currentfill}%
\pgfsetfillopacity{0.700000}%
\pgfsetlinewidth{0.000000pt}%
\definecolor{currentstroke}{rgb}{0.000000,0.000000,0.000000}%
\pgfsetstrokecolor{currentstroke}%
\pgfsetdash{}{0pt}%
\pgfpathmoveto{\pgfqpoint{5.249082in}{1.751097in}}%
\pgfpathlineto{\pgfqpoint{5.263434in}{1.749408in}}%
\pgfpathlineto{\pgfqpoint{5.277794in}{1.747744in}}%
\pgfpathlineto{\pgfqpoint{5.292163in}{1.746103in}}%
\pgfpathlineto{\pgfqpoint{5.306540in}{1.744486in}}%
\pgfpathlineto{\pgfqpoint{5.298822in}{1.732827in}}%
\pgfpathlineto{\pgfqpoint{5.291099in}{1.721173in}}%
\pgfpathlineto{\pgfqpoint{5.283371in}{1.709528in}}%
\pgfpathlineto{\pgfqpoint{5.275639in}{1.697896in}}%
\pgfpathlineto{\pgfqpoint{5.261256in}{1.699707in}}%
\pgfpathlineto{\pgfqpoint{5.246881in}{1.701541in}}%
\pgfpathlineto{\pgfqpoint{5.232516in}{1.703400in}}%
\pgfpathlineto{\pgfqpoint{5.218158in}{1.705283in}}%
\pgfpathlineto{\pgfqpoint{5.225896in}{1.716716in}}%
\pgfpathlineto{\pgfqpoint{5.233629in}{1.728165in}}%
\pgfpathlineto{\pgfqpoint{5.241358in}{1.739627in}}%
\pgfpathlineto{\pgfqpoint{5.249082in}{1.751097in}}%
\pgfpathclose%
\pgfusepath{fill}%
\end{pgfscope}%
\begin{pgfscope}%
\pgfpathrectangle{\pgfqpoint{1.150000in}{0.150000in}}{\pgfqpoint{5.700000in}{5.700000in}}%
\pgfusepath{clip}%
\pgfsetbuttcap%
\pgfsetroundjoin%
\definecolor{currentfill}{rgb}{0.276022,0.044167,0.370164}%
\pgfsetfillcolor{currentfill}%
\pgfsetfillopacity{0.700000}%
\pgfsetlinewidth{0.000000pt}%
\definecolor{currentstroke}{rgb}{0.000000,0.000000,0.000000}%
\pgfsetstrokecolor{currentstroke}%
\pgfsetdash{}{0pt}%
\pgfpathmoveto{\pgfqpoint{4.172463in}{1.638629in}}%
\pgfpathlineto{\pgfqpoint{4.186505in}{1.633371in}}%
\pgfpathlineto{\pgfqpoint{4.200552in}{1.628137in}}%
\pgfpathlineto{\pgfqpoint{4.214606in}{1.622929in}}%
\pgfpathlineto{\pgfqpoint{4.228665in}{1.617745in}}%
\pgfpathlineto{\pgfqpoint{4.220633in}{1.614125in}}%
\pgfpathlineto{\pgfqpoint{4.212593in}{1.610782in}}%
\pgfpathlineto{\pgfqpoint{4.204546in}{1.607725in}}%
\pgfpathlineto{\pgfqpoint{4.196493in}{1.604961in}}%
\pgfpathlineto{\pgfqpoint{4.182414in}{1.610457in}}%
\pgfpathlineto{\pgfqpoint{4.168341in}{1.615978in}}%
\pgfpathlineto{\pgfqpoint{4.154273in}{1.621524in}}%
\pgfpathlineto{\pgfqpoint{4.140212in}{1.627094in}}%
\pgfpathlineto{\pgfqpoint{4.148286in}{1.629540in}}%
\pgfpathlineto{\pgfqpoint{4.156352in}{1.632284in}}%
\pgfpathlineto{\pgfqpoint{4.164412in}{1.635316in}}%
\pgfpathlineto{\pgfqpoint{4.172463in}{1.638629in}}%
\pgfpathclose%
\pgfusepath{fill}%
\end{pgfscope}%
\begin{pgfscope}%
\pgfpathrectangle{\pgfqpoint{1.150000in}{0.150000in}}{\pgfqpoint{5.700000in}{5.700000in}}%
\pgfusepath{clip}%
\pgfsetbuttcap%
\pgfsetroundjoin%
\definecolor{currentfill}{rgb}{0.273809,0.031497,0.358853}%
\pgfsetfillcolor{currentfill}%
\pgfsetfillopacity{0.700000}%
\pgfsetlinewidth{0.000000pt}%
\definecolor{currentstroke}{rgb}{0.000000,0.000000,0.000000}%
\pgfsetstrokecolor{currentstroke}%
\pgfsetdash{}{0pt}%
\pgfpathmoveto{\pgfqpoint{4.839154in}{1.627180in}}%
\pgfpathlineto{\pgfqpoint{4.853372in}{1.624156in}}%
\pgfpathlineto{\pgfqpoint{4.867598in}{1.621157in}}%
\pgfpathlineto{\pgfqpoint{4.881832in}{1.618182in}}%
\pgfpathlineto{\pgfqpoint{4.896073in}{1.615230in}}%
\pgfpathlineto{\pgfqpoint{4.888255in}{1.605456in}}%
\pgfpathlineto{\pgfqpoint{4.880434in}{1.595788in}}%
\pgfpathlineto{\pgfqpoint{4.872609in}{1.586232in}}%
\pgfpathlineto{\pgfqpoint{4.864781in}{1.576794in}}%
\pgfpathlineto{\pgfqpoint{4.850531in}{1.579992in}}%
\pgfpathlineto{\pgfqpoint{4.836288in}{1.583214in}}%
\pgfpathlineto{\pgfqpoint{4.822053in}{1.586460in}}%
\pgfpathlineto{\pgfqpoint{4.807826in}{1.589729in}}%
\pgfpathlineto{\pgfqpoint{4.815664in}{1.598916in}}%
\pgfpathlineto{\pgfqpoint{4.823498in}{1.608224in}}%
\pgfpathlineto{\pgfqpoint{4.831328in}{1.617647in}}%
\pgfpathlineto{\pgfqpoint{4.839154in}{1.627180in}}%
\pgfpathclose%
\pgfusepath{fill}%
\end{pgfscope}%
\begin{pgfscope}%
\pgfpathrectangle{\pgfqpoint{1.150000in}{0.150000in}}{\pgfqpoint{5.700000in}{5.700000in}}%
\pgfusepath{clip}%
\pgfsetbuttcap%
\pgfsetroundjoin%
\definecolor{currentfill}{rgb}{0.271305,0.019942,0.347269}%
\pgfsetfillcolor{currentfill}%
\pgfsetfillopacity{0.700000}%
\pgfsetlinewidth{0.000000pt}%
\definecolor{currentstroke}{rgb}{0.000000,0.000000,0.000000}%
\pgfsetstrokecolor{currentstroke}%
\pgfsetdash{}{0pt}%
\pgfpathmoveto{\pgfqpoint{4.606151in}{1.598540in}}%
\pgfpathlineto{\pgfqpoint{4.620302in}{1.594723in}}%
\pgfpathlineto{\pgfqpoint{4.634461in}{1.590931in}}%
\pgfpathlineto{\pgfqpoint{4.648626in}{1.587162in}}%
\pgfpathlineto{\pgfqpoint{4.662799in}{1.583418in}}%
\pgfpathlineto{\pgfqpoint{4.654920in}{1.575469in}}%
\pgfpathlineto{\pgfqpoint{4.647036in}{1.567689in}}%
\pgfpathlineto{\pgfqpoint{4.639149in}{1.560082in}}%
\pgfpathlineto{\pgfqpoint{4.631257in}{1.552656in}}%
\pgfpathlineto{\pgfqpoint{4.617073in}{1.556673in}}%
\pgfpathlineto{\pgfqpoint{4.602895in}{1.560714in}}%
\pgfpathlineto{\pgfqpoint{4.588725in}{1.564779in}}%
\pgfpathlineto{\pgfqpoint{4.574561in}{1.568868in}}%
\pgfpathlineto{\pgfqpoint{4.582465in}{1.576016in}}%
\pgfpathlineto{\pgfqpoint{4.590365in}{1.583349in}}%
\pgfpathlineto{\pgfqpoint{4.598260in}{1.590859in}}%
\pgfpathlineto{\pgfqpoint{4.606151in}{1.598540in}}%
\pgfpathclose%
\pgfusepath{fill}%
\end{pgfscope}%
\begin{pgfscope}%
\pgfpathrectangle{\pgfqpoint{1.150000in}{0.150000in}}{\pgfqpoint{5.700000in}{5.700000in}}%
\pgfusepath{clip}%
\pgfsetbuttcap%
\pgfsetroundjoin%
\definecolor{currentfill}{rgb}{0.281446,0.084320,0.407414}%
\pgfsetfillcolor{currentfill}%
\pgfsetfillopacity{0.700000}%
\pgfsetlinewidth{0.000000pt}%
\definecolor{currentstroke}{rgb}{0.000000,0.000000,0.000000}%
\pgfsetstrokecolor{currentstroke}%
\pgfsetdash{}{0pt}%
\pgfpathmoveto{\pgfqpoint{5.160813in}{1.713055in}}%
\pgfpathlineto{\pgfqpoint{5.175137in}{1.711076in}}%
\pgfpathlineto{\pgfqpoint{5.189469in}{1.709121in}}%
\pgfpathlineto{\pgfqpoint{5.203809in}{1.707190in}}%
\pgfpathlineto{\pgfqpoint{5.218158in}{1.705283in}}%
\pgfpathlineto{\pgfqpoint{5.210416in}{1.693872in}}%
\pgfpathlineto{\pgfqpoint{5.202670in}{1.682488in}}%
\pgfpathlineto{\pgfqpoint{5.194920in}{1.671136in}}%
\pgfpathlineto{\pgfqpoint{5.187165in}{1.659821in}}%
\pgfpathlineto{\pgfqpoint{5.172810in}{1.661935in}}%
\pgfpathlineto{\pgfqpoint{5.158464in}{1.664073in}}%
\pgfpathlineto{\pgfqpoint{5.144126in}{1.666236in}}%
\pgfpathlineto{\pgfqpoint{5.129796in}{1.668422in}}%
\pgfpathlineto{\pgfqpoint{5.137556in}{1.679525in}}%
\pgfpathlineto{\pgfqpoint{5.145313in}{1.690668in}}%
\pgfpathlineto{\pgfqpoint{5.153065in}{1.701847in}}%
\pgfpathlineto{\pgfqpoint{5.160813in}{1.713055in}}%
\pgfpathclose%
\pgfusepath{fill}%
\end{pgfscope}%
\begin{pgfscope}%
\pgfpathrectangle{\pgfqpoint{1.150000in}{0.150000in}}{\pgfqpoint{5.700000in}{5.700000in}}%
\pgfusepath{clip}%
\pgfsetbuttcap%
\pgfsetroundjoin%
\definecolor{currentfill}{rgb}{0.282910,0.105393,0.426902}%
\pgfsetfillcolor{currentfill}%
\pgfsetfillopacity{0.700000}%
\pgfsetlinewidth{0.000000pt}%
\definecolor{currentstroke}{rgb}{0.000000,0.000000,0.000000}%
\pgfsetstrokecolor{currentstroke}%
\pgfsetdash{}{0pt}%
\pgfpathmoveto{\pgfqpoint{3.827319in}{1.743667in}}%
\pgfpathlineto{\pgfqpoint{3.841295in}{1.737263in}}%
\pgfpathlineto{\pgfqpoint{3.855275in}{1.730885in}}%
\pgfpathlineto{\pgfqpoint{3.869261in}{1.724534in}}%
\pgfpathlineto{\pgfqpoint{3.883252in}{1.718208in}}%
\pgfpathlineto{\pgfqpoint{3.875041in}{1.718717in}}%
\pgfpathlineto{\pgfqpoint{3.866819in}{1.719589in}}%
\pgfpathlineto{\pgfqpoint{3.858586in}{1.720835in}}%
\pgfpathlineto{\pgfqpoint{3.850342in}{1.722463in}}%
\pgfpathlineto{\pgfqpoint{3.836325in}{1.729130in}}%
\pgfpathlineto{\pgfqpoint{3.822312in}{1.735823in}}%
\pgfpathlineto{\pgfqpoint{3.808304in}{1.742541in}}%
\pgfpathlineto{\pgfqpoint{3.794301in}{1.749286in}}%
\pgfpathlineto{\pgfqpoint{3.802573in}{1.747312in}}%
\pgfpathlineto{\pgfqpoint{3.810833in}{1.745723in}}%
\pgfpathlineto{\pgfqpoint{3.819081in}{1.744511in}}%
\pgfpathlineto{\pgfqpoint{3.827319in}{1.743667in}}%
\pgfpathclose%
\pgfusepath{fill}%
\end{pgfscope}%
\begin{pgfscope}%
\pgfpathrectangle{\pgfqpoint{1.150000in}{0.150000in}}{\pgfqpoint{5.700000in}{5.700000in}}%
\pgfusepath{clip}%
\pgfsetbuttcap%
\pgfsetroundjoin%
\definecolor{currentfill}{rgb}{0.119512,0.607464,0.540218}%
\pgfsetfillcolor{currentfill}%
\pgfsetfillopacity{0.700000}%
\pgfsetlinewidth{0.000000pt}%
\definecolor{currentstroke}{rgb}{0.000000,0.000000,0.000000}%
\pgfsetstrokecolor{currentstroke}%
\pgfsetdash{}{0pt}%
\pgfpathmoveto{\pgfqpoint{2.096243in}{2.967829in}}%
\pgfpathlineto{\pgfqpoint{2.110113in}{2.955499in}}%
\pgfpathlineto{\pgfqpoint{2.123983in}{2.943224in}}%
\pgfpathlineto{\pgfqpoint{2.137853in}{2.931003in}}%
\pgfpathlineto{\pgfqpoint{2.151724in}{2.918837in}}%
\pgfpathlineto{\pgfqpoint{2.141830in}{2.940338in}}%
\pgfpathlineto{\pgfqpoint{2.131894in}{2.962539in}}%
\pgfpathlineto{\pgfqpoint{2.121915in}{2.985455in}}%
\pgfpathlineto{\pgfqpoint{2.111891in}{3.009099in}}%
\pgfpathlineto{\pgfqpoint{2.097956in}{3.021710in}}%
\pgfpathlineto{\pgfqpoint{2.084021in}{3.034375in}}%
\pgfpathlineto{\pgfqpoint{2.070085in}{3.047096in}}%
\pgfpathlineto{\pgfqpoint{2.056149in}{3.059873in}}%
\pgfpathlineto{\pgfqpoint{2.066240in}{3.035776in}}%
\pgfpathlineto{\pgfqpoint{2.076285in}{3.012412in}}%
\pgfpathlineto{\pgfqpoint{2.086286in}{2.989768in}}%
\pgfpathlineto{\pgfqpoint{2.096243in}{2.967829in}}%
\pgfpathclose%
\pgfusepath{fill}%
\end{pgfscope}%
\begin{pgfscope}%
\pgfpathrectangle{\pgfqpoint{1.150000in}{0.150000in}}{\pgfqpoint{5.700000in}{5.700000in}}%
\pgfusepath{clip}%
\pgfsetbuttcap%
\pgfsetroundjoin%
\definecolor{currentfill}{rgb}{0.235526,0.309527,0.542944}%
\pgfsetfillcolor{currentfill}%
\pgfsetfillopacity{0.700000}%
\pgfsetlinewidth{0.000000pt}%
\definecolor{currentstroke}{rgb}{0.000000,0.000000,0.000000}%
\pgfsetstrokecolor{currentstroke}%
\pgfsetdash{}{0pt}%
\pgfpathmoveto{\pgfqpoint{3.057478in}{2.171367in}}%
\pgfpathlineto{\pgfqpoint{3.071354in}{2.162518in}}%
\pgfpathlineto{\pgfqpoint{3.085233in}{2.153701in}}%
\pgfpathlineto{\pgfqpoint{3.099115in}{2.144916in}}%
\pgfpathlineto{\pgfqpoint{3.113001in}{2.136163in}}%
\pgfpathlineto{\pgfqpoint{3.104196in}{2.146262in}}%
\pgfpathlineto{\pgfqpoint{3.095369in}{2.156892in}}%
\pgfpathlineto{\pgfqpoint{3.086519in}{2.168063in}}%
\pgfpathlineto{\pgfqpoint{3.077645in}{2.179789in}}%
\pgfpathlineto{\pgfqpoint{3.063716in}{2.188933in}}%
\pgfpathlineto{\pgfqpoint{3.049790in}{2.198110in}}%
\pgfpathlineto{\pgfqpoint{3.035867in}{2.207318in}}%
\pgfpathlineto{\pgfqpoint{3.021947in}{2.216558in}}%
\pgfpathlineto{\pgfqpoint{3.030865in}{2.204435in}}%
\pgfpathlineto{\pgfqpoint{3.039760in}{2.192870in}}%
\pgfpathlineto{\pgfqpoint{3.048630in}{2.181850in}}%
\pgfpathlineto{\pgfqpoint{3.057478in}{2.171367in}}%
\pgfpathclose%
\pgfusepath{fill}%
\end{pgfscope}%
\begin{pgfscope}%
\pgfpathrectangle{\pgfqpoint{1.150000in}{0.150000in}}{\pgfqpoint{5.700000in}{5.700000in}}%
\pgfusepath{clip}%
\pgfsetbuttcap%
\pgfsetroundjoin%
\definecolor{currentfill}{rgb}{0.269308,0.218818,0.509577}%
\pgfsetfillcolor{currentfill}%
\pgfsetfillopacity{0.700000}%
\pgfsetlinewidth{0.000000pt}%
\definecolor{currentstroke}{rgb}{0.000000,0.000000,0.000000}%
\pgfsetstrokecolor{currentstroke}%
\pgfsetdash{}{0pt}%
\pgfpathmoveto{\pgfqpoint{3.370027in}{1.970967in}}%
\pgfpathlineto{\pgfqpoint{3.383936in}{1.963101in}}%
\pgfpathlineto{\pgfqpoint{3.397849in}{1.955264in}}%
\pgfpathlineto{\pgfqpoint{3.411767in}{1.947455in}}%
\pgfpathlineto{\pgfqpoint{3.425688in}{1.939675in}}%
\pgfpathlineto{\pgfqpoint{3.417155in}{1.945919in}}%
\pgfpathlineto{\pgfqpoint{3.408605in}{1.952632in}}%
\pgfpathlineto{\pgfqpoint{3.400038in}{1.959823in}}%
\pgfpathlineto{\pgfqpoint{3.391453in}{1.967503in}}%
\pgfpathlineto{\pgfqpoint{3.377495in}{1.975656in}}%
\pgfpathlineto{\pgfqpoint{3.363541in}{1.983837in}}%
\pgfpathlineto{\pgfqpoint{3.349591in}{1.992048in}}%
\pgfpathlineto{\pgfqpoint{3.335644in}{2.000287in}}%
\pgfpathlineto{\pgfqpoint{3.344268in}{1.992228in}}%
\pgfpathlineto{\pgfqpoint{3.352872in}{1.984662in}}%
\pgfpathlineto{\pgfqpoint{3.361458in}{1.977579in}}%
\pgfpathlineto{\pgfqpoint{3.370027in}{1.970967in}}%
\pgfpathclose%
\pgfusepath{fill}%
\end{pgfscope}%
\begin{pgfscope}%
\pgfpathrectangle{\pgfqpoint{1.150000in}{0.150000in}}{\pgfqpoint{5.700000in}{5.700000in}}%
\pgfusepath{clip}%
\pgfsetbuttcap%
\pgfsetroundjoin%
\definecolor{currentfill}{rgb}{0.279566,0.067836,0.391917}%
\pgfsetfillcolor{currentfill}%
\pgfsetfillopacity{0.700000}%
\pgfsetlinewidth{0.000000pt}%
\definecolor{currentstroke}{rgb}{0.000000,0.000000,0.000000}%
\pgfsetstrokecolor{currentstroke}%
\pgfsetdash{}{0pt}%
\pgfpathmoveto{\pgfqpoint{4.027926in}{1.672554in}}%
\pgfpathlineto{\pgfqpoint{4.041942in}{1.666784in}}%
\pgfpathlineto{\pgfqpoint{4.055963in}{1.661039in}}%
\pgfpathlineto{\pgfqpoint{4.069990in}{1.655319in}}%
\pgfpathlineto{\pgfqpoint{4.084023in}{1.649624in}}%
\pgfpathlineto{\pgfqpoint{4.075920in}{1.647804in}}%
\pgfpathlineto{\pgfqpoint{4.067809in}{1.646302in}}%
\pgfpathlineto{\pgfqpoint{4.059689in}{1.645126in}}%
\pgfpathlineto{\pgfqpoint{4.051560in}{1.644285in}}%
\pgfpathlineto{\pgfqpoint{4.037505in}{1.650306in}}%
\pgfpathlineto{\pgfqpoint{4.023455in}{1.656352in}}%
\pgfpathlineto{\pgfqpoint{4.009410in}{1.662424in}}%
\pgfpathlineto{\pgfqpoint{3.995371in}{1.668521in}}%
\pgfpathlineto{\pgfqpoint{4.003523in}{1.669030in}}%
\pgfpathlineto{\pgfqpoint{4.011666in}{1.669878in}}%
\pgfpathlineto{\pgfqpoint{4.019801in}{1.671055in}}%
\pgfpathlineto{\pgfqpoint{4.027926in}{1.672554in}}%
\pgfpathclose%
\pgfusepath{fill}%
\end{pgfscope}%
\begin{pgfscope}%
\pgfpathrectangle{\pgfqpoint{1.150000in}{0.150000in}}{\pgfqpoint{5.700000in}{5.700000in}}%
\pgfusepath{clip}%
\pgfsetbuttcap%
\pgfsetroundjoin%
\definecolor{currentfill}{rgb}{0.185556,0.418570,0.556753}%
\pgfsetfillcolor{currentfill}%
\pgfsetfillopacity{0.700000}%
\pgfsetlinewidth{0.000000pt}%
\definecolor{currentstroke}{rgb}{0.000000,0.000000,0.000000}%
\pgfsetstrokecolor{currentstroke}%
\pgfsetdash{}{0pt}%
\pgfpathmoveto{\pgfqpoint{2.688688in}{2.448547in}}%
\pgfpathlineto{\pgfqpoint{2.702546in}{2.438467in}}%
\pgfpathlineto{\pgfqpoint{2.716406in}{2.428425in}}%
\pgfpathlineto{\pgfqpoint{2.730268in}{2.418421in}}%
\pgfpathlineto{\pgfqpoint{2.744132in}{2.408454in}}%
\pgfpathlineto{\pgfqpoint{2.734947in}{2.423168in}}%
\pgfpathlineto{\pgfqpoint{2.725732in}{2.438485in}}%
\pgfpathlineto{\pgfqpoint{2.716487in}{2.454418in}}%
\pgfpathlineto{\pgfqpoint{2.707211in}{2.470978in}}%
\pgfpathlineto{\pgfqpoint{2.693294in}{2.481358in}}%
\pgfpathlineto{\pgfqpoint{2.679380in}{2.491775in}}%
\pgfpathlineto{\pgfqpoint{2.665468in}{2.502231in}}%
\pgfpathlineto{\pgfqpoint{2.651558in}{2.512724in}}%
\pgfpathlineto{\pgfqpoint{2.660888in}{2.495744in}}%
\pgfpathlineto{\pgfqpoint{2.670186in}{2.479396in}}%
\pgfpathlineto{\pgfqpoint{2.679453in}{2.463668in}}%
\pgfpathlineto{\pgfqpoint{2.688688in}{2.448547in}}%
\pgfpathclose%
\pgfusepath{fill}%
\end{pgfscope}%
\begin{pgfscope}%
\pgfpathrectangle{\pgfqpoint{1.150000in}{0.150000in}}{\pgfqpoint{5.700000in}{5.700000in}}%
\pgfusepath{clip}%
\pgfsetbuttcap%
\pgfsetroundjoin%
\definecolor{currentfill}{rgb}{0.279566,0.067836,0.391917}%
\pgfsetfillcolor{currentfill}%
\pgfsetfillopacity{0.700000}%
\pgfsetlinewidth{0.000000pt}%
\definecolor{currentstroke}{rgb}{0.000000,0.000000,0.000000}%
\pgfsetstrokecolor{currentstroke}%
\pgfsetdash{}{0pt}%
\pgfpathmoveto{\pgfqpoint{5.072558in}{1.677407in}}%
\pgfpathlineto{\pgfqpoint{5.086855in}{1.675125in}}%
\pgfpathlineto{\pgfqpoint{5.101160in}{1.672867in}}%
\pgfpathlineto{\pgfqpoint{5.115474in}{1.670632in}}%
\pgfpathlineto{\pgfqpoint{5.129796in}{1.668422in}}%
\pgfpathlineto{\pgfqpoint{5.122031in}{1.657364in}}%
\pgfpathlineto{\pgfqpoint{5.114263in}{1.646357in}}%
\pgfpathlineto{\pgfqpoint{5.106490in}{1.635405in}}%
\pgfpathlineto{\pgfqpoint{5.098714in}{1.624515in}}%
\pgfpathlineto{\pgfqpoint{5.084386in}{1.626945in}}%
\pgfpathlineto{\pgfqpoint{5.070066in}{1.629400in}}%
\pgfpathlineto{\pgfqpoint{5.055754in}{1.631879in}}%
\pgfpathlineto{\pgfqpoint{5.041450in}{1.634381in}}%
\pgfpathlineto{\pgfqpoint{5.049233in}{1.645046in}}%
\pgfpathlineto{\pgfqpoint{5.057012in}{1.655776in}}%
\pgfpathlineto{\pgfqpoint{5.064787in}{1.666565in}}%
\pgfpathlineto{\pgfqpoint{5.072558in}{1.677407in}}%
\pgfpathclose%
\pgfusepath{fill}%
\end{pgfscope}%
\begin{pgfscope}%
\pgfpathrectangle{\pgfqpoint{1.150000in}{0.150000in}}{\pgfqpoint{5.700000in}{5.700000in}}%
\pgfusepath{clip}%
\pgfsetbuttcap%
\pgfsetroundjoin%
\definecolor{currentfill}{rgb}{0.281887,0.150881,0.465405}%
\pgfsetfillcolor{currentfill}%
\pgfsetfillopacity{0.700000}%
\pgfsetlinewidth{0.000000pt}%
\definecolor{currentstroke}{rgb}{0.000000,0.000000,0.000000}%
\pgfsetstrokecolor{currentstroke}%
\pgfsetdash{}{0pt}%
\pgfpathmoveto{\pgfqpoint{3.626645in}{1.832277in}}%
\pgfpathlineto{\pgfqpoint{3.640591in}{1.825215in}}%
\pgfpathlineto{\pgfqpoint{3.654541in}{1.818179in}}%
\pgfpathlineto{\pgfqpoint{3.668495in}{1.811170in}}%
\pgfpathlineto{\pgfqpoint{3.682455in}{1.804188in}}%
\pgfpathlineto{\pgfqpoint{3.674113in}{1.807258in}}%
\pgfpathlineto{\pgfqpoint{3.665757in}{1.810742in}}%
\pgfpathlineto{\pgfqpoint{3.657389in}{1.814647in}}%
\pgfpathlineto{\pgfqpoint{3.649006in}{1.818984in}}%
\pgfpathlineto{\pgfqpoint{3.635016in}{1.826322in}}%
\pgfpathlineto{\pgfqpoint{3.621030in}{1.833688in}}%
\pgfpathlineto{\pgfqpoint{3.607049in}{1.841080in}}%
\pgfpathlineto{\pgfqpoint{3.593072in}{1.848499in}}%
\pgfpathlineto{\pgfqpoint{3.601487in}{1.843799in}}%
\pgfpathlineto{\pgfqpoint{3.609887in}{1.839536in}}%
\pgfpathlineto{\pgfqpoint{3.618273in}{1.835699in}}%
\pgfpathlineto{\pgfqpoint{3.626645in}{1.832277in}}%
\pgfpathclose%
\pgfusepath{fill}%
\end{pgfscope}%
\begin{pgfscope}%
\pgfpathrectangle{\pgfqpoint{1.150000in}{0.150000in}}{\pgfqpoint{5.700000in}{5.700000in}}%
\pgfusepath{clip}%
\pgfsetbuttcap%
\pgfsetroundjoin%
\definecolor{currentfill}{rgb}{0.272594,0.025563,0.353093}%
\pgfsetfillcolor{currentfill}%
\pgfsetfillopacity{0.700000}%
\pgfsetlinewidth{0.000000pt}%
\definecolor{currentstroke}{rgb}{0.000000,0.000000,0.000000}%
\pgfsetstrokecolor{currentstroke}%
\pgfsetdash{}{0pt}%
\pgfpathmoveto{\pgfqpoint{4.750990in}{1.603047in}}%
\pgfpathlineto{\pgfqpoint{4.765188in}{1.599682in}}%
\pgfpathlineto{\pgfqpoint{4.779393in}{1.596340in}}%
\pgfpathlineto{\pgfqpoint{4.793606in}{1.593023in}}%
\pgfpathlineto{\pgfqpoint{4.807826in}{1.589729in}}%
\pgfpathlineto{\pgfqpoint{4.799984in}{1.580670in}}%
\pgfpathlineto{\pgfqpoint{4.792139in}{1.571745in}}%
\pgfpathlineto{\pgfqpoint{4.784289in}{1.562961in}}%
\pgfpathlineto{\pgfqpoint{4.776436in}{1.554324in}}%
\pgfpathlineto{\pgfqpoint{4.762206in}{1.557877in}}%
\pgfpathlineto{\pgfqpoint{4.747983in}{1.561454in}}%
\pgfpathlineto{\pgfqpoint{4.733768in}{1.565054in}}%
\pgfpathlineto{\pgfqpoint{4.719560in}{1.568679in}}%
\pgfpathlineto{\pgfqpoint{4.727423in}{1.577052in}}%
\pgfpathlineto{\pgfqpoint{4.735283in}{1.585575in}}%
\pgfpathlineto{\pgfqpoint{4.743139in}{1.594242in}}%
\pgfpathlineto{\pgfqpoint{4.750990in}{1.603047in}}%
\pgfpathclose%
\pgfusepath{fill}%
\end{pgfscope}%
\begin{pgfscope}%
\pgfpathrectangle{\pgfqpoint{1.150000in}{0.150000in}}{\pgfqpoint{5.700000in}{5.700000in}}%
\pgfusepath{clip}%
\pgfsetbuttcap%
\pgfsetroundjoin%
\definecolor{currentfill}{rgb}{0.121831,0.589055,0.545623}%
\pgfsetfillcolor{currentfill}%
\pgfsetfillopacity{0.700000}%
\pgfsetlinewidth{0.000000pt}%
\definecolor{currentstroke}{rgb}{0.000000,0.000000,0.000000}%
\pgfsetstrokecolor{currentstroke}%
\pgfsetdash{}{0pt}%
\pgfpathmoveto{\pgfqpoint{2.151724in}{2.918837in}}%
\pgfpathlineto{\pgfqpoint{2.165594in}{2.906724in}}%
\pgfpathlineto{\pgfqpoint{2.179465in}{2.894665in}}%
\pgfpathlineto{\pgfqpoint{2.193335in}{2.882657in}}%
\pgfpathlineto{\pgfqpoint{2.207206in}{2.870702in}}%
\pgfpathlineto{\pgfqpoint{2.197376in}{2.891767in}}%
\pgfpathlineto{\pgfqpoint{2.187503in}{2.913527in}}%
\pgfpathlineto{\pgfqpoint{2.177589in}{2.935997in}}%
\pgfpathlineto{\pgfqpoint{2.167631in}{2.959189in}}%
\pgfpathlineto{\pgfqpoint{2.153696in}{2.971588in}}%
\pgfpathlineto{\pgfqpoint{2.139761in}{2.984039in}}%
\pgfpathlineto{\pgfqpoint{2.125826in}{2.996542in}}%
\pgfpathlineto{\pgfqpoint{2.111891in}{3.009099in}}%
\pgfpathlineto{\pgfqpoint{2.121915in}{2.985455in}}%
\pgfpathlineto{\pgfqpoint{2.131894in}{2.962539in}}%
\pgfpathlineto{\pgfqpoint{2.141830in}{2.940338in}}%
\pgfpathlineto{\pgfqpoint{2.151724in}{2.918837in}}%
\pgfpathclose%
\pgfusepath{fill}%
\end{pgfscope}%
\begin{pgfscope}%
\pgfpathrectangle{\pgfqpoint{1.150000in}{0.150000in}}{\pgfqpoint{5.700000in}{5.700000in}}%
\pgfusepath{clip}%
\pgfsetbuttcap%
\pgfsetroundjoin%
\definecolor{currentfill}{rgb}{0.277018,0.050344,0.375715}%
\pgfsetfillcolor{currentfill}%
\pgfsetfillopacity{0.700000}%
\pgfsetlinewidth{0.000000pt}%
\definecolor{currentstroke}{rgb}{0.000000,0.000000,0.000000}%
\pgfsetstrokecolor{currentstroke}%
\pgfsetdash{}{0pt}%
\pgfpathmoveto{\pgfqpoint{4.984313in}{1.644631in}}%
\pgfpathlineto{\pgfqpoint{4.998586in}{1.642033in}}%
\pgfpathlineto{\pgfqpoint{5.012866in}{1.639458in}}%
\pgfpathlineto{\pgfqpoint{5.027154in}{1.636908in}}%
\pgfpathlineto{\pgfqpoint{5.041450in}{1.634381in}}%
\pgfpathlineto{\pgfqpoint{5.033663in}{1.623787in}}%
\pgfpathlineto{\pgfqpoint{5.025872in}{1.613268in}}%
\pgfpathlineto{\pgfqpoint{5.018078in}{1.602830in}}%
\pgfpathlineto{\pgfqpoint{5.010281in}{1.592480in}}%
\pgfpathlineto{\pgfqpoint{4.995977in}{1.595240in}}%
\pgfpathlineto{\pgfqpoint{4.981682in}{1.598024in}}%
\pgfpathlineto{\pgfqpoint{4.967394in}{1.600832in}}%
\pgfpathlineto{\pgfqpoint{4.953114in}{1.603664in}}%
\pgfpathlineto{\pgfqpoint{4.960920in}{1.613776in}}%
\pgfpathlineto{\pgfqpoint{4.968721in}{1.623978in}}%
\pgfpathlineto{\pgfqpoint{4.976519in}{1.634265in}}%
\pgfpathlineto{\pgfqpoint{4.984313in}{1.644631in}}%
\pgfpathclose%
\pgfusepath{fill}%
\end{pgfscope}%
\begin{pgfscope}%
\pgfpathrectangle{\pgfqpoint{1.150000in}{0.150000in}}{\pgfqpoint{5.700000in}{5.700000in}}%
\pgfusepath{clip}%
\pgfsetbuttcap%
\pgfsetroundjoin%
\definecolor{currentfill}{rgb}{0.272594,0.025563,0.353093}%
\pgfsetfillcolor{currentfill}%
\pgfsetfillopacity{0.700000}%
\pgfsetlinewidth{0.000000pt}%
\definecolor{currentstroke}{rgb}{0.000000,0.000000,0.000000}%
\pgfsetstrokecolor{currentstroke}%
\pgfsetdash{}{0pt}%
\pgfpathmoveto{\pgfqpoint{4.373288in}{1.596640in}}%
\pgfpathlineto{\pgfqpoint{4.387387in}{1.591975in}}%
\pgfpathlineto{\pgfqpoint{4.401491in}{1.587335in}}%
\pgfpathlineto{\pgfqpoint{4.415603in}{1.582719in}}%
\pgfpathlineto{\pgfqpoint{4.429720in}{1.578127in}}%
\pgfpathlineto{\pgfqpoint{4.421763in}{1.572597in}}%
\pgfpathlineto{\pgfqpoint{4.413801in}{1.567302in}}%
\pgfpathlineto{\pgfqpoint{4.405833in}{1.562251in}}%
\pgfpathlineto{\pgfqpoint{4.397859in}{1.557450in}}%
\pgfpathlineto{\pgfqpoint{4.383725in}{1.562341in}}%
\pgfpathlineto{\pgfqpoint{4.369598in}{1.567256in}}%
\pgfpathlineto{\pgfqpoint{4.355477in}{1.572195in}}%
\pgfpathlineto{\pgfqpoint{4.341362in}{1.577158in}}%
\pgfpathlineto{\pgfqpoint{4.349352in}{1.581655in}}%
\pgfpathlineto{\pgfqpoint{4.357337in}{1.586406in}}%
\pgfpathlineto{\pgfqpoint{4.365315in}{1.591404in}}%
\pgfpathlineto{\pgfqpoint{4.373288in}{1.596640in}}%
\pgfpathclose%
\pgfusepath{fill}%
\end{pgfscope}%
\begin{pgfscope}%
\pgfpathrectangle{\pgfqpoint{1.150000in}{0.150000in}}{\pgfqpoint{5.700000in}{5.700000in}}%
\pgfusepath{clip}%
\pgfsetbuttcap%
\pgfsetroundjoin%
\definecolor{currentfill}{rgb}{0.241237,0.296485,0.539709}%
\pgfsetfillcolor{currentfill}%
\pgfsetfillopacity{0.700000}%
\pgfsetlinewidth{0.000000pt}%
\definecolor{currentstroke}{rgb}{0.000000,0.000000,0.000000}%
\pgfsetstrokecolor{currentstroke}%
\pgfsetdash{}{0pt}%
\pgfpathmoveto{\pgfqpoint{3.113001in}{2.136163in}}%
\pgfpathlineto{\pgfqpoint{3.126890in}{2.127442in}}%
\pgfpathlineto{\pgfqpoint{3.140782in}{2.118752in}}%
\pgfpathlineto{\pgfqpoint{3.154678in}{2.110092in}}%
\pgfpathlineto{\pgfqpoint{3.168577in}{2.101464in}}%
\pgfpathlineto{\pgfqpoint{3.159814in}{2.111178in}}%
\pgfpathlineto{\pgfqpoint{3.151029in}{2.121419in}}%
\pgfpathlineto{\pgfqpoint{3.142223in}{2.132198in}}%
\pgfpathlineto{\pgfqpoint{3.133393in}{2.143526in}}%
\pgfpathlineto{\pgfqpoint{3.119451in}{2.152545in}}%
\pgfpathlineto{\pgfqpoint{3.105513in}{2.161595in}}%
\pgfpathlineto{\pgfqpoint{3.091577in}{2.170676in}}%
\pgfpathlineto{\pgfqpoint{3.077645in}{2.179789in}}%
\pgfpathlineto{\pgfqpoint{3.086519in}{2.168063in}}%
\pgfpathlineto{\pgfqpoint{3.095369in}{2.156892in}}%
\pgfpathlineto{\pgfqpoint{3.104196in}{2.146262in}}%
\pgfpathlineto{\pgfqpoint{3.113001in}{2.136163in}}%
\pgfpathclose%
\pgfusepath{fill}%
\end{pgfscope}%
\begin{pgfscope}%
\pgfpathrectangle{\pgfqpoint{1.150000in}{0.150000in}}{\pgfqpoint{5.700000in}{5.700000in}}%
\pgfusepath{clip}%
\pgfsetbuttcap%
\pgfsetroundjoin%
\definecolor{currentfill}{rgb}{0.271305,0.019942,0.347269}%
\pgfsetfillcolor{currentfill}%
\pgfsetfillopacity{0.700000}%
\pgfsetlinewidth{0.000000pt}%
\definecolor{currentstroke}{rgb}{0.000000,0.000000,0.000000}%
\pgfsetstrokecolor{currentstroke}%
\pgfsetdash{}{0pt}%
\pgfpathmoveto{\pgfqpoint{4.517975in}{1.585465in}}%
\pgfpathlineto{\pgfqpoint{4.532111in}{1.581280in}}%
\pgfpathlineto{\pgfqpoint{4.546254in}{1.577118in}}%
\pgfpathlineto{\pgfqpoint{4.560404in}{1.572981in}}%
\pgfpathlineto{\pgfqpoint{4.574561in}{1.568868in}}%
\pgfpathlineto{\pgfqpoint{4.566652in}{1.561911in}}%
\pgfpathlineto{\pgfqpoint{4.558739in}{1.555153in}}%
\pgfpathlineto{\pgfqpoint{4.550821in}{1.548601in}}%
\pgfpathlineto{\pgfqpoint{4.542898in}{1.542262in}}%
\pgfpathlineto{\pgfqpoint{4.528728in}{1.546661in}}%
\pgfpathlineto{\pgfqpoint{4.514564in}{1.551084in}}%
\pgfpathlineto{\pgfqpoint{4.500407in}{1.555531in}}%
\pgfpathlineto{\pgfqpoint{4.486256in}{1.560002in}}%
\pgfpathlineto{\pgfqpoint{4.494193in}{1.566050in}}%
\pgfpathlineto{\pgfqpoint{4.502125in}{1.572315in}}%
\pgfpathlineto{\pgfqpoint{4.510052in}{1.578789in}}%
\pgfpathlineto{\pgfqpoint{4.517975in}{1.585465in}}%
\pgfpathclose%
\pgfusepath{fill}%
\end{pgfscope}%
\begin{pgfscope}%
\pgfpathrectangle{\pgfqpoint{1.150000in}{0.150000in}}{\pgfqpoint{5.700000in}{5.700000in}}%
\pgfusepath{clip}%
\pgfsetbuttcap%
\pgfsetroundjoin%
\definecolor{currentfill}{rgb}{0.274952,0.037752,0.364543}%
\pgfsetfillcolor{currentfill}%
\pgfsetfillopacity{0.700000}%
\pgfsetlinewidth{0.000000pt}%
\definecolor{currentstroke}{rgb}{0.000000,0.000000,0.000000}%
\pgfsetstrokecolor{currentstroke}%
\pgfsetdash{}{0pt}%
\pgfpathmoveto{\pgfqpoint{4.228665in}{1.617745in}}%
\pgfpathlineto{\pgfqpoint{4.242731in}{1.612586in}}%
\pgfpathlineto{\pgfqpoint{4.256803in}{1.607452in}}%
\pgfpathlineto{\pgfqpoint{4.270880in}{1.602342in}}%
\pgfpathlineto{\pgfqpoint{4.284964in}{1.597256in}}%
\pgfpathlineto{\pgfqpoint{4.276950in}{1.593329in}}%
\pgfpathlineto{\pgfqpoint{4.268929in}{1.589675in}}%
\pgfpathlineto{\pgfqpoint{4.260902in}{1.586303in}}%
\pgfpathlineto{\pgfqpoint{4.252868in}{1.583221in}}%
\pgfpathlineto{\pgfqpoint{4.238765in}{1.588619in}}%
\pgfpathlineto{\pgfqpoint{4.224669in}{1.594042in}}%
\pgfpathlineto{\pgfqpoint{4.210578in}{1.599489in}}%
\pgfpathlineto{\pgfqpoint{4.196493in}{1.604961in}}%
\pgfpathlineto{\pgfqpoint{4.204546in}{1.607725in}}%
\pgfpathlineto{\pgfqpoint{4.212593in}{1.610782in}}%
\pgfpathlineto{\pgfqpoint{4.220633in}{1.614125in}}%
\pgfpathlineto{\pgfqpoint{4.228665in}{1.617745in}}%
\pgfpathclose%
\pgfusepath{fill}%
\end{pgfscope}%
\begin{pgfscope}%
\pgfpathrectangle{\pgfqpoint{1.150000in}{0.150000in}}{\pgfqpoint{5.700000in}{5.700000in}}%
\pgfusepath{clip}%
\pgfsetbuttcap%
\pgfsetroundjoin%
\definecolor{currentfill}{rgb}{0.125394,0.574318,0.549086}%
\pgfsetfillcolor{currentfill}%
\pgfsetfillopacity{0.700000}%
\pgfsetlinewidth{0.000000pt}%
\definecolor{currentstroke}{rgb}{0.000000,0.000000,0.000000}%
\pgfsetstrokecolor{currentstroke}%
\pgfsetdash{}{0pt}%
\pgfpathmoveto{\pgfqpoint{2.207206in}{2.870702in}}%
\pgfpathlineto{\pgfqpoint{2.221078in}{2.858798in}}%
\pgfpathlineto{\pgfqpoint{2.234950in}{2.846945in}}%
\pgfpathlineto{\pgfqpoint{2.248822in}{2.835142in}}%
\pgfpathlineto{\pgfqpoint{2.262695in}{2.823389in}}%
\pgfpathlineto{\pgfqpoint{2.252926in}{2.844019in}}%
\pgfpathlineto{\pgfqpoint{2.243116in}{2.865340in}}%
\pgfpathlineto{\pgfqpoint{2.233265in}{2.887365in}}%
\pgfpathlineto{\pgfqpoint{2.223372in}{2.910108in}}%
\pgfpathlineto{\pgfqpoint{2.209437in}{2.922303in}}%
\pgfpathlineto{\pgfqpoint{2.195501in}{2.934547in}}%
\pgfpathlineto{\pgfqpoint{2.181566in}{2.946843in}}%
\pgfpathlineto{\pgfqpoint{2.167631in}{2.959189in}}%
\pgfpathlineto{\pgfqpoint{2.177589in}{2.935997in}}%
\pgfpathlineto{\pgfqpoint{2.187503in}{2.913527in}}%
\pgfpathlineto{\pgfqpoint{2.197376in}{2.891767in}}%
\pgfpathlineto{\pgfqpoint{2.207206in}{2.870702in}}%
\pgfpathclose%
\pgfusepath{fill}%
\end{pgfscope}%
\begin{pgfscope}%
\pgfpathrectangle{\pgfqpoint{1.150000in}{0.150000in}}{\pgfqpoint{5.700000in}{5.700000in}}%
\pgfusepath{clip}%
\pgfsetbuttcap%
\pgfsetroundjoin%
\definecolor{currentfill}{rgb}{0.190631,0.407061,0.556089}%
\pgfsetfillcolor{currentfill}%
\pgfsetfillopacity{0.700000}%
\pgfsetlinewidth{0.000000pt}%
\definecolor{currentstroke}{rgb}{0.000000,0.000000,0.000000}%
\pgfsetstrokecolor{currentstroke}%
\pgfsetdash{}{0pt}%
\pgfpathmoveto{\pgfqpoint{2.744132in}{2.408454in}}%
\pgfpathlineto{\pgfqpoint{2.757999in}{2.398524in}}%
\pgfpathlineto{\pgfqpoint{2.771867in}{2.388631in}}%
\pgfpathlineto{\pgfqpoint{2.785739in}{2.378775in}}%
\pgfpathlineto{\pgfqpoint{2.799612in}{2.368954in}}%
\pgfpathlineto{\pgfqpoint{2.790477in}{2.383262in}}%
\pgfpathlineto{\pgfqpoint{2.781313in}{2.398169in}}%
\pgfpathlineto{\pgfqpoint{2.772120in}{2.413687in}}%
\pgfpathlineto{\pgfqpoint{2.762896in}{2.429828in}}%
\pgfpathlineto{\pgfqpoint{2.748971in}{2.440060in}}%
\pgfpathlineto{\pgfqpoint{2.735049in}{2.450329in}}%
\pgfpathlineto{\pgfqpoint{2.721129in}{2.460635in}}%
\pgfpathlineto{\pgfqpoint{2.707211in}{2.470978in}}%
\pgfpathlineto{\pgfqpoint{2.716487in}{2.454418in}}%
\pgfpathlineto{\pgfqpoint{2.725732in}{2.438485in}}%
\pgfpathlineto{\pgfqpoint{2.734947in}{2.423168in}}%
\pgfpathlineto{\pgfqpoint{2.744132in}{2.408454in}}%
\pgfpathclose%
\pgfusepath{fill}%
\end{pgfscope}%
\begin{pgfscope}%
\pgfpathrectangle{\pgfqpoint{1.150000in}{0.150000in}}{\pgfqpoint{5.700000in}{5.700000in}}%
\pgfusepath{clip}%
\pgfsetbuttcap%
\pgfsetroundjoin%
\definecolor{currentfill}{rgb}{0.283187,0.125848,0.444960}%
\pgfsetfillcolor{currentfill}%
\pgfsetfillopacity{0.700000}%
\pgfsetlinewidth{0.000000pt}%
\definecolor{currentstroke}{rgb}{0.000000,0.000000,0.000000}%
\pgfsetstrokecolor{currentstroke}%
\pgfsetdash{}{0pt}%
\pgfpathmoveto{\pgfqpoint{5.394942in}{1.785575in}}%
\pgfpathlineto{\pgfqpoint{5.409359in}{1.784260in}}%
\pgfpathlineto{\pgfqpoint{5.423783in}{1.782970in}}%
\pgfpathlineto{\pgfqpoint{5.438217in}{1.781703in}}%
\pgfpathlineto{\pgfqpoint{5.430527in}{1.769763in}}%
\pgfpathlineto{\pgfqpoint{5.422832in}{1.757804in}}%
\pgfpathlineto{\pgfqpoint{5.415132in}{1.745831in}}%
\pgfpathlineto{\pgfqpoint{5.407427in}{1.733846in}}%
\pgfpathlineto{\pgfqpoint{5.392988in}{1.735294in}}%
\pgfpathlineto{\pgfqpoint{5.378558in}{1.736766in}}%
\pgfpathlineto{\pgfqpoint{5.364137in}{1.738261in}}%
\pgfpathlineto{\pgfqpoint{5.371846in}{1.750106in}}%
\pgfpathlineto{\pgfqpoint{5.379550in}{1.761943in}}%
\pgfpathlineto{\pgfqpoint{5.387249in}{1.773767in}}%
\pgfpathlineto{\pgfqpoint{5.394942in}{1.785575in}}%
\pgfpathclose%
\pgfusepath{fill}%
\end{pgfscope}%
\begin{pgfscope}%
\pgfpathrectangle{\pgfqpoint{1.150000in}{0.150000in}}{\pgfqpoint{5.700000in}{5.700000in}}%
\pgfusepath{clip}%
\pgfsetbuttcap%
\pgfsetroundjoin%
\definecolor{currentfill}{rgb}{0.282656,0.100196,0.422160}%
\pgfsetfillcolor{currentfill}%
\pgfsetfillopacity{0.700000}%
\pgfsetlinewidth{0.000000pt}%
\definecolor{currentstroke}{rgb}{0.000000,0.000000,0.000000}%
\pgfsetstrokecolor{currentstroke}%
\pgfsetdash{}{0pt}%
\pgfpathmoveto{\pgfqpoint{3.883252in}{1.718208in}}%
\pgfpathlineto{\pgfqpoint{3.897249in}{1.711907in}}%
\pgfpathlineto{\pgfqpoint{3.911250in}{1.705633in}}%
\pgfpathlineto{\pgfqpoint{3.925257in}{1.699384in}}%
\pgfpathlineto{\pgfqpoint{3.939269in}{1.693160in}}%
\pgfpathlineto{\pgfqpoint{3.931083in}{1.693334in}}%
\pgfpathlineto{\pgfqpoint{3.922887in}{1.693867in}}%
\pgfpathlineto{\pgfqpoint{3.914681in}{1.694770in}}%
\pgfpathlineto{\pgfqpoint{3.906464in}{1.696052in}}%
\pgfpathlineto{\pgfqpoint{3.892426in}{1.702616in}}%
\pgfpathlineto{\pgfqpoint{3.878393in}{1.709206in}}%
\pgfpathlineto{\pgfqpoint{3.864365in}{1.715821in}}%
\pgfpathlineto{\pgfqpoint{3.850342in}{1.722463in}}%
\pgfpathlineto{\pgfqpoint{3.858586in}{1.720835in}}%
\pgfpathlineto{\pgfqpoint{3.866819in}{1.719589in}}%
\pgfpathlineto{\pgfqpoint{3.875041in}{1.718717in}}%
\pgfpathlineto{\pgfqpoint{3.883252in}{1.718208in}}%
\pgfpathclose%
\pgfusepath{fill}%
\end{pgfscope}%
\begin{pgfscope}%
\pgfpathrectangle{\pgfqpoint{1.150000in}{0.150000in}}{\pgfqpoint{5.700000in}{5.700000in}}%
\pgfusepath{clip}%
\pgfsetbuttcap%
\pgfsetroundjoin%
\definecolor{currentfill}{rgb}{0.271828,0.209303,0.504434}%
\pgfsetfillcolor{currentfill}%
\pgfsetfillopacity{0.700000}%
\pgfsetlinewidth{0.000000pt}%
\definecolor{currentstroke}{rgb}{0.000000,0.000000,0.000000}%
\pgfsetstrokecolor{currentstroke}%
\pgfsetdash{}{0pt}%
\pgfpathmoveto{\pgfqpoint{3.425688in}{1.939675in}}%
\pgfpathlineto{\pgfqpoint{3.439613in}{1.931923in}}%
\pgfpathlineto{\pgfqpoint{3.453543in}{1.924200in}}%
\pgfpathlineto{\pgfqpoint{3.467476in}{1.916505in}}%
\pgfpathlineto{\pgfqpoint{3.481414in}{1.908837in}}%
\pgfpathlineto{\pgfqpoint{3.472917in}{1.914715in}}%
\pgfpathlineto{\pgfqpoint{3.464403in}{1.921057in}}%
\pgfpathlineto{\pgfqpoint{3.455872in}{1.927874in}}%
\pgfpathlineto{\pgfqpoint{3.447324in}{1.935175in}}%
\pgfpathlineto{\pgfqpoint{3.433350in}{1.943215in}}%
\pgfpathlineto{\pgfqpoint{3.419381in}{1.951282in}}%
\pgfpathlineto{\pgfqpoint{3.405415in}{1.959378in}}%
\pgfpathlineto{\pgfqpoint{3.391453in}{1.967503in}}%
\pgfpathlineto{\pgfqpoint{3.400038in}{1.959823in}}%
\pgfpathlineto{\pgfqpoint{3.408605in}{1.952632in}}%
\pgfpathlineto{\pgfqpoint{3.417155in}{1.945919in}}%
\pgfpathlineto{\pgfqpoint{3.425688in}{1.939675in}}%
\pgfpathclose%
\pgfusepath{fill}%
\end{pgfscope}%
\begin{pgfscope}%
\pgfpathrectangle{\pgfqpoint{1.150000in}{0.150000in}}{\pgfqpoint{5.700000in}{5.700000in}}%
\pgfusepath{clip}%
\pgfsetbuttcap%
\pgfsetroundjoin%
\definecolor{currentfill}{rgb}{0.274952,0.037752,0.364543}%
\pgfsetfillcolor{currentfill}%
\pgfsetfillopacity{0.700000}%
\pgfsetlinewidth{0.000000pt}%
\definecolor{currentstroke}{rgb}{0.000000,0.000000,0.000000}%
\pgfsetstrokecolor{currentstroke}%
\pgfsetdash{}{0pt}%
\pgfpathmoveto{\pgfqpoint{4.896073in}{1.615230in}}%
\pgfpathlineto{\pgfqpoint{4.910322in}{1.612303in}}%
\pgfpathlineto{\pgfqpoint{4.924578in}{1.609399in}}%
\pgfpathlineto{\pgfqpoint{4.938842in}{1.606519in}}%
\pgfpathlineto{\pgfqpoint{4.953114in}{1.603664in}}%
\pgfpathlineto{\pgfqpoint{4.945305in}{1.593648in}}%
\pgfpathlineto{\pgfqpoint{4.937492in}{1.583735in}}%
\pgfpathlineto{\pgfqpoint{4.929676in}{1.573931in}}%
\pgfpathlineto{\pgfqpoint{4.921856in}{1.564241in}}%
\pgfpathlineto{\pgfqpoint{4.907576in}{1.567344in}}%
\pgfpathlineto{\pgfqpoint{4.893303in}{1.570470in}}%
\pgfpathlineto{\pgfqpoint{4.879038in}{1.573620in}}%
\pgfpathlineto{\pgfqpoint{4.864781in}{1.576794in}}%
\pgfpathlineto{\pgfqpoint{4.872609in}{1.586232in}}%
\pgfpathlineto{\pgfqpoint{4.880434in}{1.595788in}}%
\pgfpathlineto{\pgfqpoint{4.888255in}{1.605456in}}%
\pgfpathlineto{\pgfqpoint{4.896073in}{1.615230in}}%
\pgfpathclose%
\pgfusepath{fill}%
\end{pgfscope}%
\begin{pgfscope}%
\pgfpathrectangle{\pgfqpoint{1.150000in}{0.150000in}}{\pgfqpoint{5.700000in}{5.700000in}}%
\pgfusepath{clip}%
\pgfsetbuttcap%
\pgfsetroundjoin%
\definecolor{currentfill}{rgb}{0.271305,0.019942,0.347269}%
\pgfsetfillcolor{currentfill}%
\pgfsetfillopacity{0.700000}%
\pgfsetlinewidth{0.000000pt}%
\definecolor{currentstroke}{rgb}{0.000000,0.000000,0.000000}%
\pgfsetstrokecolor{currentstroke}%
\pgfsetdash{}{0pt}%
\pgfpathmoveto{\pgfqpoint{4.662799in}{1.583418in}}%
\pgfpathlineto{\pgfqpoint{4.676978in}{1.579697in}}%
\pgfpathlineto{\pgfqpoint{4.691165in}{1.576000in}}%
\pgfpathlineto{\pgfqpoint{4.705359in}{1.572328in}}%
\pgfpathlineto{\pgfqpoint{4.719560in}{1.568679in}}%
\pgfpathlineto{\pgfqpoint{4.711692in}{1.560464in}}%
\pgfpathlineto{\pgfqpoint{4.703820in}{1.552412in}}%
\pgfpathlineto{\pgfqpoint{4.695945in}{1.544530in}}%
\pgfpathlineto{\pgfqpoint{4.688065in}{1.536827in}}%
\pgfpathlineto{\pgfqpoint{4.673853in}{1.540748in}}%
\pgfpathlineto{\pgfqpoint{4.659647in}{1.544694in}}%
\pgfpathlineto{\pgfqpoint{4.645449in}{1.548663in}}%
\pgfpathlineto{\pgfqpoint{4.631257in}{1.552656in}}%
\pgfpathlineto{\pgfqpoint{4.639149in}{1.560082in}}%
\pgfpathlineto{\pgfqpoint{4.647036in}{1.567689in}}%
\pgfpathlineto{\pgfqpoint{4.654920in}{1.575469in}}%
\pgfpathlineto{\pgfqpoint{4.662799in}{1.583418in}}%
\pgfpathclose%
\pgfusepath{fill}%
\end{pgfscope}%
\begin{pgfscope}%
\pgfpathrectangle{\pgfqpoint{1.150000in}{0.150000in}}{\pgfqpoint{5.700000in}{5.700000in}}%
\pgfusepath{clip}%
\pgfsetbuttcap%
\pgfsetroundjoin%
\definecolor{currentfill}{rgb}{0.282910,0.105393,0.426902}%
\pgfsetfillcolor{currentfill}%
\pgfsetfillopacity{0.700000}%
\pgfsetlinewidth{0.000000pt}%
\definecolor{currentstroke}{rgb}{0.000000,0.000000,0.000000}%
\pgfsetstrokecolor{currentstroke}%
\pgfsetdash{}{0pt}%
\pgfpathmoveto{\pgfqpoint{5.306540in}{1.744486in}}%
\pgfpathlineto{\pgfqpoint{5.320926in}{1.742894in}}%
\pgfpathlineto{\pgfqpoint{5.335321in}{1.741326in}}%
\pgfpathlineto{\pgfqpoint{5.349725in}{1.739781in}}%
\pgfpathlineto{\pgfqpoint{5.364137in}{1.738261in}}%
\pgfpathlineto{\pgfqpoint{5.356424in}{1.726413in}}%
\pgfpathlineto{\pgfqpoint{5.348706in}{1.714566in}}%
\pgfpathlineto{\pgfqpoint{5.340983in}{1.702725in}}%
\pgfpathlineto{\pgfqpoint{5.333256in}{1.690894in}}%
\pgfpathlineto{\pgfqpoint{5.318839in}{1.692608in}}%
\pgfpathlineto{\pgfqpoint{5.304430in}{1.694347in}}%
\pgfpathlineto{\pgfqpoint{5.290030in}{1.696109in}}%
\pgfpathlineto{\pgfqpoint{5.275639in}{1.697896in}}%
\pgfpathlineto{\pgfqpoint{5.283371in}{1.709528in}}%
\pgfpathlineto{\pgfqpoint{5.291099in}{1.721173in}}%
\pgfpathlineto{\pgfqpoint{5.298822in}{1.732827in}}%
\pgfpathlineto{\pgfqpoint{5.306540in}{1.744486in}}%
\pgfpathclose%
\pgfusepath{fill}%
\end{pgfscope}%
\begin{pgfscope}%
\pgfpathrectangle{\pgfqpoint{1.150000in}{0.150000in}}{\pgfqpoint{5.700000in}{5.700000in}}%
\pgfusepath{clip}%
\pgfsetbuttcap%
\pgfsetroundjoin%
\definecolor{currentfill}{rgb}{0.282290,0.145912,0.461510}%
\pgfsetfillcolor{currentfill}%
\pgfsetfillopacity{0.700000}%
\pgfsetlinewidth{0.000000pt}%
\definecolor{currentstroke}{rgb}{0.000000,0.000000,0.000000}%
\pgfsetstrokecolor{currentstroke}%
\pgfsetdash{}{0pt}%
\pgfpathmoveto{\pgfqpoint{3.682455in}{1.804188in}}%
\pgfpathlineto{\pgfqpoint{3.696419in}{1.797233in}}%
\pgfpathlineto{\pgfqpoint{3.710387in}{1.790304in}}%
\pgfpathlineto{\pgfqpoint{3.724361in}{1.783402in}}%
\pgfpathlineto{\pgfqpoint{3.738339in}{1.776526in}}%
\pgfpathlineto{\pgfqpoint{3.730027in}{1.779246in}}%
\pgfpathlineto{\pgfqpoint{3.721702in}{1.782375in}}%
\pgfpathlineto{\pgfqpoint{3.713364in}{1.785922in}}%
\pgfpathlineto{\pgfqpoint{3.705014in}{1.789897in}}%
\pgfpathlineto{\pgfqpoint{3.691005in}{1.797129in}}%
\pgfpathlineto{\pgfqpoint{3.677001in}{1.804388in}}%
\pgfpathlineto{\pgfqpoint{3.663001in}{1.811673in}}%
\pgfpathlineto{\pgfqpoint{3.649006in}{1.818984in}}%
\pgfpathlineto{\pgfqpoint{3.657389in}{1.814647in}}%
\pgfpathlineto{\pgfqpoint{3.665757in}{1.810742in}}%
\pgfpathlineto{\pgfqpoint{3.674113in}{1.807258in}}%
\pgfpathlineto{\pgfqpoint{3.682455in}{1.804188in}}%
\pgfpathclose%
\pgfusepath{fill}%
\end{pgfscope}%
\begin{pgfscope}%
\pgfpathrectangle{\pgfqpoint{1.150000in}{0.150000in}}{\pgfqpoint{5.700000in}{5.700000in}}%
\pgfusepath{clip}%
\pgfsetbuttcap%
\pgfsetroundjoin%
\definecolor{currentfill}{rgb}{0.129933,0.559582,0.551864}%
\pgfsetfillcolor{currentfill}%
\pgfsetfillopacity{0.700000}%
\pgfsetlinewidth{0.000000pt}%
\definecolor{currentstroke}{rgb}{0.000000,0.000000,0.000000}%
\pgfsetstrokecolor{currentstroke}%
\pgfsetdash{}{0pt}%
\pgfpathmoveto{\pgfqpoint{2.262695in}{2.823389in}}%
\pgfpathlineto{\pgfqpoint{2.276568in}{2.811685in}}%
\pgfpathlineto{\pgfqpoint{2.290442in}{2.800030in}}%
\pgfpathlineto{\pgfqpoint{2.304317in}{2.788424in}}%
\pgfpathlineto{\pgfqpoint{2.318192in}{2.776865in}}%
\pgfpathlineto{\pgfqpoint{2.308484in}{2.797062in}}%
\pgfpathlineto{\pgfqpoint{2.298736in}{2.817945in}}%
\pgfpathlineto{\pgfqpoint{2.288948in}{2.839527in}}%
\pgfpathlineto{\pgfqpoint{2.279119in}{2.861822in}}%
\pgfpathlineto{\pgfqpoint{2.265182in}{2.873821in}}%
\pgfpathlineto{\pgfqpoint{2.251245in}{2.885868in}}%
\pgfpathlineto{\pgfqpoint{2.237309in}{2.897963in}}%
\pgfpathlineto{\pgfqpoint{2.223372in}{2.910108in}}%
\pgfpathlineto{\pgfqpoint{2.233265in}{2.887365in}}%
\pgfpathlineto{\pgfqpoint{2.243116in}{2.865340in}}%
\pgfpathlineto{\pgfqpoint{2.252926in}{2.844019in}}%
\pgfpathlineto{\pgfqpoint{2.262695in}{2.823389in}}%
\pgfpathclose%
\pgfusepath{fill}%
\end{pgfscope}%
\begin{pgfscope}%
\pgfpathrectangle{\pgfqpoint{1.150000in}{0.150000in}}{\pgfqpoint{5.700000in}{5.700000in}}%
\pgfusepath{clip}%
\pgfsetbuttcap%
\pgfsetroundjoin%
\definecolor{currentfill}{rgb}{0.278791,0.062145,0.386592}%
\pgfsetfillcolor{currentfill}%
\pgfsetfillopacity{0.700000}%
\pgfsetlinewidth{0.000000pt}%
\definecolor{currentstroke}{rgb}{0.000000,0.000000,0.000000}%
\pgfsetstrokecolor{currentstroke}%
\pgfsetdash{}{0pt}%
\pgfpathmoveto{\pgfqpoint{4.084023in}{1.649624in}}%
\pgfpathlineto{\pgfqpoint{4.098062in}{1.643954in}}%
\pgfpathlineto{\pgfqpoint{4.112106in}{1.638309in}}%
\pgfpathlineto{\pgfqpoint{4.126156in}{1.632689in}}%
\pgfpathlineto{\pgfqpoint{4.140212in}{1.627094in}}%
\pgfpathlineto{\pgfqpoint{4.132130in}{1.624953in}}%
\pgfpathlineto{\pgfqpoint{4.124040in}{1.623126in}}%
\pgfpathlineto{\pgfqpoint{4.115943in}{1.621622in}}%
\pgfpathlineto{\pgfqpoint{4.107837in}{1.620448in}}%
\pgfpathlineto{\pgfqpoint{4.093760in}{1.626370in}}%
\pgfpathlineto{\pgfqpoint{4.079688in}{1.632317in}}%
\pgfpathlineto{\pgfqpoint{4.065621in}{1.638288in}}%
\pgfpathlineto{\pgfqpoint{4.051560in}{1.644285in}}%
\pgfpathlineto{\pgfqpoint{4.059689in}{1.645126in}}%
\pgfpathlineto{\pgfqpoint{4.067809in}{1.646302in}}%
\pgfpathlineto{\pgfqpoint{4.075920in}{1.647804in}}%
\pgfpathlineto{\pgfqpoint{4.084023in}{1.649624in}}%
\pgfpathclose%
\pgfusepath{fill}%
\end{pgfscope}%
\begin{pgfscope}%
\pgfpathrectangle{\pgfqpoint{1.150000in}{0.150000in}}{\pgfqpoint{5.700000in}{5.700000in}}%
\pgfusepath{clip}%
\pgfsetbuttcap%
\pgfsetroundjoin%
\definecolor{currentfill}{rgb}{0.281924,0.089666,0.412415}%
\pgfsetfillcolor{currentfill}%
\pgfsetfillopacity{0.700000}%
\pgfsetlinewidth{0.000000pt}%
\definecolor{currentstroke}{rgb}{0.000000,0.000000,0.000000}%
\pgfsetstrokecolor{currentstroke}%
\pgfsetdash{}{0pt}%
\pgfpathmoveto{\pgfqpoint{5.218158in}{1.705283in}}%
\pgfpathlineto{\pgfqpoint{5.232516in}{1.703400in}}%
\pgfpathlineto{\pgfqpoint{5.246881in}{1.701541in}}%
\pgfpathlineto{\pgfqpoint{5.261256in}{1.699707in}}%
\pgfpathlineto{\pgfqpoint{5.275639in}{1.697896in}}%
\pgfpathlineto{\pgfqpoint{5.267902in}{1.686283in}}%
\pgfpathlineto{\pgfqpoint{5.260162in}{1.674693in}}%
\pgfpathlineto{\pgfqpoint{5.252417in}{1.663131in}}%
\pgfpathlineto{\pgfqpoint{5.244668in}{1.651603in}}%
\pgfpathlineto{\pgfqpoint{5.230280in}{1.653622in}}%
\pgfpathlineto{\pgfqpoint{5.215900in}{1.655664in}}%
\pgfpathlineto{\pgfqpoint{5.201528in}{1.657730in}}%
\pgfpathlineto{\pgfqpoint{5.187165in}{1.659821in}}%
\pgfpathlineto{\pgfqpoint{5.194920in}{1.671136in}}%
\pgfpathlineto{\pgfqpoint{5.202670in}{1.682488in}}%
\pgfpathlineto{\pgfqpoint{5.210416in}{1.693872in}}%
\pgfpathlineto{\pgfqpoint{5.218158in}{1.705283in}}%
\pgfpathclose%
\pgfusepath{fill}%
\end{pgfscope}%
\begin{pgfscope}%
\pgfpathrectangle{\pgfqpoint{1.150000in}{0.150000in}}{\pgfqpoint{5.700000in}{5.700000in}}%
\pgfusepath{clip}%
\pgfsetbuttcap%
\pgfsetroundjoin%
\definecolor{currentfill}{rgb}{0.195860,0.395433,0.555276}%
\pgfsetfillcolor{currentfill}%
\pgfsetfillopacity{0.700000}%
\pgfsetlinewidth{0.000000pt}%
\definecolor{currentstroke}{rgb}{0.000000,0.000000,0.000000}%
\pgfsetstrokecolor{currentstroke}%
\pgfsetdash{}{0pt}%
\pgfpathmoveto{\pgfqpoint{2.799612in}{2.368954in}}%
\pgfpathlineto{\pgfqpoint{2.813488in}{2.359170in}}%
\pgfpathlineto{\pgfqpoint{2.827367in}{2.349421in}}%
\pgfpathlineto{\pgfqpoint{2.841248in}{2.339708in}}%
\pgfpathlineto{\pgfqpoint{2.855132in}{2.330030in}}%
\pgfpathlineto{\pgfqpoint{2.846046in}{2.343933in}}%
\pgfpathlineto{\pgfqpoint{2.836932in}{2.358431in}}%
\pgfpathlineto{\pgfqpoint{2.827789in}{2.373535in}}%
\pgfpathlineto{\pgfqpoint{2.818617in}{2.389257in}}%
\pgfpathlineto{\pgfqpoint{2.804683in}{2.399346in}}%
\pgfpathlineto{\pgfqpoint{2.790752in}{2.409471in}}%
\pgfpathlineto{\pgfqpoint{2.776823in}{2.419632in}}%
\pgfpathlineto{\pgfqpoint{2.762896in}{2.429828in}}%
\pgfpathlineto{\pgfqpoint{2.772120in}{2.413687in}}%
\pgfpathlineto{\pgfqpoint{2.781313in}{2.398169in}}%
\pgfpathlineto{\pgfqpoint{2.790477in}{2.383262in}}%
\pgfpathlineto{\pgfqpoint{2.799612in}{2.368954in}}%
\pgfpathclose%
\pgfusepath{fill}%
\end{pgfscope}%
\begin{pgfscope}%
\pgfpathrectangle{\pgfqpoint{1.150000in}{0.150000in}}{\pgfqpoint{5.700000in}{5.700000in}}%
\pgfusepath{clip}%
\pgfsetbuttcap%
\pgfsetroundjoin%
\definecolor{currentfill}{rgb}{0.244972,0.287675,0.537260}%
\pgfsetfillcolor{currentfill}%
\pgfsetfillopacity{0.700000}%
\pgfsetlinewidth{0.000000pt}%
\definecolor{currentstroke}{rgb}{0.000000,0.000000,0.000000}%
\pgfsetstrokecolor{currentstroke}%
\pgfsetdash{}{0pt}%
\pgfpathmoveto{\pgfqpoint{3.168577in}{2.101464in}}%
\pgfpathlineto{\pgfqpoint{3.182480in}{2.092867in}}%
\pgfpathlineto{\pgfqpoint{3.196386in}{2.084300in}}%
\pgfpathlineto{\pgfqpoint{3.210295in}{2.075764in}}%
\pgfpathlineto{\pgfqpoint{3.224208in}{2.067259in}}%
\pgfpathlineto{\pgfqpoint{3.215486in}{2.076588in}}%
\pgfpathlineto{\pgfqpoint{3.206744in}{2.086441in}}%
\pgfpathlineto{\pgfqpoint{3.197979in}{2.096827in}}%
\pgfpathlineto{\pgfqpoint{3.189193in}{2.107758in}}%
\pgfpathlineto{\pgfqpoint{3.175238in}{2.116654in}}%
\pgfpathlineto{\pgfqpoint{3.161286in}{2.125581in}}%
\pgfpathlineto{\pgfqpoint{3.147338in}{2.134538in}}%
\pgfpathlineto{\pgfqpoint{3.133393in}{2.143526in}}%
\pgfpathlineto{\pgfqpoint{3.142223in}{2.132198in}}%
\pgfpathlineto{\pgfqpoint{3.151029in}{2.121419in}}%
\pgfpathlineto{\pgfqpoint{3.159814in}{2.111178in}}%
\pgfpathlineto{\pgfqpoint{3.168577in}{2.101464in}}%
\pgfpathclose%
\pgfusepath{fill}%
\end{pgfscope}%
\begin{pgfscope}%
\pgfpathrectangle{\pgfqpoint{1.150000in}{0.150000in}}{\pgfqpoint{5.700000in}{5.700000in}}%
\pgfusepath{clip}%
\pgfsetbuttcap%
\pgfsetroundjoin%
\definecolor{currentfill}{rgb}{0.280267,0.073417,0.397163}%
\pgfsetfillcolor{currentfill}%
\pgfsetfillopacity{0.700000}%
\pgfsetlinewidth{0.000000pt}%
\definecolor{currentstroke}{rgb}{0.000000,0.000000,0.000000}%
\pgfsetstrokecolor{currentstroke}%
\pgfsetdash{}{0pt}%
\pgfpathmoveto{\pgfqpoint{5.129796in}{1.668422in}}%
\pgfpathlineto{\pgfqpoint{5.144126in}{1.666236in}}%
\pgfpathlineto{\pgfqpoint{5.158464in}{1.664073in}}%
\pgfpathlineto{\pgfqpoint{5.172810in}{1.661935in}}%
\pgfpathlineto{\pgfqpoint{5.187165in}{1.659821in}}%
\pgfpathlineto{\pgfqpoint{5.179407in}{1.648547in}}%
\pgfpathlineto{\pgfqpoint{5.171644in}{1.637321in}}%
\pgfpathlineto{\pgfqpoint{5.163878in}{1.626147in}}%
\pgfpathlineto{\pgfqpoint{5.156109in}{1.615030in}}%
\pgfpathlineto{\pgfqpoint{5.141748in}{1.617366in}}%
\pgfpathlineto{\pgfqpoint{5.127395in}{1.619725in}}%
\pgfpathlineto{\pgfqpoint{5.113050in}{1.622108in}}%
\pgfpathlineto{\pgfqpoint{5.098714in}{1.624515in}}%
\pgfpathlineto{\pgfqpoint{5.106490in}{1.635405in}}%
\pgfpathlineto{\pgfqpoint{5.114263in}{1.646357in}}%
\pgfpathlineto{\pgfqpoint{5.122031in}{1.657364in}}%
\pgfpathlineto{\pgfqpoint{5.129796in}{1.668422in}}%
\pgfpathclose%
\pgfusepath{fill}%
\end{pgfscope}%
\begin{pgfscope}%
\pgfpathrectangle{\pgfqpoint{1.150000in}{0.150000in}}{\pgfqpoint{5.700000in}{5.700000in}}%
\pgfusepath{clip}%
\pgfsetbuttcap%
\pgfsetroundjoin%
\definecolor{currentfill}{rgb}{0.272594,0.025563,0.353093}%
\pgfsetfillcolor{currentfill}%
\pgfsetfillopacity{0.700000}%
\pgfsetlinewidth{0.000000pt}%
\definecolor{currentstroke}{rgb}{0.000000,0.000000,0.000000}%
\pgfsetstrokecolor{currentstroke}%
\pgfsetdash{}{0pt}%
\pgfpathmoveto{\pgfqpoint{4.807826in}{1.589729in}}%
\pgfpathlineto{\pgfqpoint{4.822053in}{1.586460in}}%
\pgfpathlineto{\pgfqpoint{4.836288in}{1.583214in}}%
\pgfpathlineto{\pgfqpoint{4.850531in}{1.579992in}}%
\pgfpathlineto{\pgfqpoint{4.864781in}{1.576794in}}%
\pgfpathlineto{\pgfqpoint{4.856948in}{1.567481in}}%
\pgfpathlineto{\pgfqpoint{4.849112in}{1.558298in}}%
\pgfpathlineto{\pgfqpoint{4.841273in}{1.549252in}}%
\pgfpathlineto{\pgfqpoint{4.833430in}{1.540350in}}%
\pgfpathlineto{\pgfqpoint{4.819170in}{1.543808in}}%
\pgfpathlineto{\pgfqpoint{4.804918in}{1.547289in}}%
\pgfpathlineto{\pgfqpoint{4.790674in}{1.550794in}}%
\pgfpathlineto{\pgfqpoint{4.776436in}{1.554324in}}%
\pgfpathlineto{\pgfqpoint{4.784289in}{1.562961in}}%
\pgfpathlineto{\pgfqpoint{4.792139in}{1.571745in}}%
\pgfpathlineto{\pgfqpoint{4.799984in}{1.580670in}}%
\pgfpathlineto{\pgfqpoint{4.807826in}{1.589729in}}%
\pgfpathclose%
\pgfusepath{fill}%
\end{pgfscope}%
\begin{pgfscope}%
\pgfpathrectangle{\pgfqpoint{1.150000in}{0.150000in}}{\pgfqpoint{5.700000in}{5.700000in}}%
\pgfusepath{clip}%
\pgfsetbuttcap%
\pgfsetroundjoin%
\definecolor{currentfill}{rgb}{0.135066,0.544853,0.554029}%
\pgfsetfillcolor{currentfill}%
\pgfsetfillopacity{0.700000}%
\pgfsetlinewidth{0.000000pt}%
\definecolor{currentstroke}{rgb}{0.000000,0.000000,0.000000}%
\pgfsetstrokecolor{currentstroke}%
\pgfsetdash{}{0pt}%
\pgfpathmoveto{\pgfqpoint{2.318192in}{2.776865in}}%
\pgfpathlineto{\pgfqpoint{2.332068in}{2.765354in}}%
\pgfpathlineto{\pgfqpoint{2.345946in}{2.753889in}}%
\pgfpathlineto{\pgfqpoint{2.359824in}{2.742472in}}%
\pgfpathlineto{\pgfqpoint{2.373703in}{2.731100in}}%
\pgfpathlineto{\pgfqpoint{2.364054in}{2.750865in}}%
\pgfpathlineto{\pgfqpoint{2.354367in}{2.771311in}}%
\pgfpathlineto{\pgfqpoint{2.344642in}{2.792452in}}%
\pgfpathlineto{\pgfqpoint{2.334876in}{2.814301in}}%
\pgfpathlineto{\pgfqpoint{2.320936in}{2.826111in}}%
\pgfpathlineto{\pgfqpoint{2.306996in}{2.837968in}}%
\pgfpathlineto{\pgfqpoint{2.293058in}{2.849871in}}%
\pgfpathlineto{\pgfqpoint{2.279119in}{2.861822in}}%
\pgfpathlineto{\pgfqpoint{2.288948in}{2.839527in}}%
\pgfpathlineto{\pgfqpoint{2.298736in}{2.817945in}}%
\pgfpathlineto{\pgfqpoint{2.308484in}{2.797062in}}%
\pgfpathlineto{\pgfqpoint{2.318192in}{2.776865in}}%
\pgfpathclose%
\pgfusepath{fill}%
\end{pgfscope}%
\begin{pgfscope}%
\pgfpathrectangle{\pgfqpoint{1.150000in}{0.150000in}}{\pgfqpoint{5.700000in}{5.700000in}}%
\pgfusepath{clip}%
\pgfsetbuttcap%
\pgfsetroundjoin%
\definecolor{currentfill}{rgb}{0.272594,0.025563,0.353093}%
\pgfsetfillcolor{currentfill}%
\pgfsetfillopacity{0.700000}%
\pgfsetlinewidth{0.000000pt}%
\definecolor{currentstroke}{rgb}{0.000000,0.000000,0.000000}%
\pgfsetstrokecolor{currentstroke}%
\pgfsetdash{}{0pt}%
\pgfpathmoveto{\pgfqpoint{4.429720in}{1.578127in}}%
\pgfpathlineto{\pgfqpoint{4.443844in}{1.573559in}}%
\pgfpathlineto{\pgfqpoint{4.457975in}{1.569016in}}%
\pgfpathlineto{\pgfqpoint{4.472113in}{1.564497in}}%
\pgfpathlineto{\pgfqpoint{4.486256in}{1.560002in}}%
\pgfpathlineto{\pgfqpoint{4.478315in}{1.554178in}}%
\pgfpathlineto{\pgfqpoint{4.470368in}{1.548586in}}%
\pgfpathlineto{\pgfqpoint{4.462416in}{1.543233in}}%
\pgfpathlineto{\pgfqpoint{4.454458in}{1.538128in}}%
\pgfpathlineto{\pgfqpoint{4.440299in}{1.542922in}}%
\pgfpathlineto{\pgfqpoint{4.426146in}{1.547741in}}%
\pgfpathlineto{\pgfqpoint{4.411999in}{1.552583in}}%
\pgfpathlineto{\pgfqpoint{4.397859in}{1.557450in}}%
\pgfpathlineto{\pgfqpoint{4.405833in}{1.562251in}}%
\pgfpathlineto{\pgfqpoint{4.413801in}{1.567302in}}%
\pgfpathlineto{\pgfqpoint{4.421763in}{1.572597in}}%
\pgfpathlineto{\pgfqpoint{4.429720in}{1.578127in}}%
\pgfpathclose%
\pgfusepath{fill}%
\end{pgfscope}%
\begin{pgfscope}%
\pgfpathrectangle{\pgfqpoint{1.150000in}{0.150000in}}{\pgfqpoint{5.700000in}{5.700000in}}%
\pgfusepath{clip}%
\pgfsetbuttcap%
\pgfsetroundjoin%
\definecolor{currentfill}{rgb}{0.274128,0.199721,0.498911}%
\pgfsetfillcolor{currentfill}%
\pgfsetfillopacity{0.700000}%
\pgfsetlinewidth{0.000000pt}%
\definecolor{currentstroke}{rgb}{0.000000,0.000000,0.000000}%
\pgfsetstrokecolor{currentstroke}%
\pgfsetdash{}{0pt}%
\pgfpathmoveto{\pgfqpoint{3.481414in}{1.908837in}}%
\pgfpathlineto{\pgfqpoint{3.495356in}{1.901198in}}%
\pgfpathlineto{\pgfqpoint{3.509303in}{1.893587in}}%
\pgfpathlineto{\pgfqpoint{3.523253in}{1.886004in}}%
\pgfpathlineto{\pgfqpoint{3.537208in}{1.878448in}}%
\pgfpathlineto{\pgfqpoint{3.528745in}{1.883959in}}%
\pgfpathlineto{\pgfqpoint{3.520266in}{1.889931in}}%
\pgfpathlineto{\pgfqpoint{3.511771in}{1.896373in}}%
\pgfpathlineto{\pgfqpoint{3.503260in}{1.903296in}}%
\pgfpathlineto{\pgfqpoint{3.489270in}{1.911224in}}%
\pgfpathlineto{\pgfqpoint{3.475284in}{1.919180in}}%
\pgfpathlineto{\pgfqpoint{3.461302in}{1.927164in}}%
\pgfpathlineto{\pgfqpoint{3.447324in}{1.935175in}}%
\pgfpathlineto{\pgfqpoint{3.455872in}{1.927874in}}%
\pgfpathlineto{\pgfqpoint{3.464403in}{1.921057in}}%
\pgfpathlineto{\pgfqpoint{3.472917in}{1.914715in}}%
\pgfpathlineto{\pgfqpoint{3.481414in}{1.908837in}}%
\pgfpathclose%
\pgfusepath{fill}%
\end{pgfscope}%
\begin{pgfscope}%
\pgfpathrectangle{\pgfqpoint{1.150000in}{0.150000in}}{\pgfqpoint{5.700000in}{5.700000in}}%
\pgfusepath{clip}%
\pgfsetbuttcap%
\pgfsetroundjoin%
\definecolor{currentfill}{rgb}{0.274952,0.037752,0.364543}%
\pgfsetfillcolor{currentfill}%
\pgfsetfillopacity{0.700000}%
\pgfsetlinewidth{0.000000pt}%
\definecolor{currentstroke}{rgb}{0.000000,0.000000,0.000000}%
\pgfsetstrokecolor{currentstroke}%
\pgfsetdash{}{0pt}%
\pgfpathmoveto{\pgfqpoint{4.284964in}{1.597256in}}%
\pgfpathlineto{\pgfqpoint{4.299054in}{1.592195in}}%
\pgfpathlineto{\pgfqpoint{4.313151in}{1.587158in}}%
\pgfpathlineto{\pgfqpoint{4.327253in}{1.582146in}}%
\pgfpathlineto{\pgfqpoint{4.341362in}{1.577158in}}%
\pgfpathlineto{\pgfqpoint{4.333365in}{1.572924in}}%
\pgfpathlineto{\pgfqpoint{4.325363in}{1.568959in}}%
\pgfpathlineto{\pgfqpoint{4.317354in}{1.565273in}}%
\pgfpathlineto{\pgfqpoint{4.309340in}{1.561873in}}%
\pgfpathlineto{\pgfqpoint{4.295213in}{1.567173in}}%
\pgfpathlineto{\pgfqpoint{4.281092in}{1.572498in}}%
\pgfpathlineto{\pgfqpoint{4.266977in}{1.577848in}}%
\pgfpathlineto{\pgfqpoint{4.252868in}{1.583221in}}%
\pgfpathlineto{\pgfqpoint{4.260902in}{1.586303in}}%
\pgfpathlineto{\pgfqpoint{4.268929in}{1.589675in}}%
\pgfpathlineto{\pgfqpoint{4.276950in}{1.593329in}}%
\pgfpathlineto{\pgfqpoint{4.284964in}{1.597256in}}%
\pgfpathclose%
\pgfusepath{fill}%
\end{pgfscope}%
\begin{pgfscope}%
\pgfpathrectangle{\pgfqpoint{1.150000in}{0.150000in}}{\pgfqpoint{5.700000in}{5.700000in}}%
\pgfusepath{clip}%
\pgfsetbuttcap%
\pgfsetroundjoin%
\definecolor{currentfill}{rgb}{0.271305,0.019942,0.347269}%
\pgfsetfillcolor{currentfill}%
\pgfsetfillopacity{0.700000}%
\pgfsetlinewidth{0.000000pt}%
\definecolor{currentstroke}{rgb}{0.000000,0.000000,0.000000}%
\pgfsetstrokecolor{currentstroke}%
\pgfsetdash{}{0pt}%
\pgfpathmoveto{\pgfqpoint{4.574561in}{1.568868in}}%
\pgfpathlineto{\pgfqpoint{4.588725in}{1.564779in}}%
\pgfpathlineto{\pgfqpoint{4.602895in}{1.560714in}}%
\pgfpathlineto{\pgfqpoint{4.617073in}{1.556673in}}%
\pgfpathlineto{\pgfqpoint{4.631257in}{1.552656in}}%
\pgfpathlineto{\pgfqpoint{4.623361in}{1.545418in}}%
\pgfpathlineto{\pgfqpoint{4.615461in}{1.538376in}}%
\pgfpathlineto{\pgfqpoint{4.607557in}{1.531536in}}%
\pgfpathlineto{\pgfqpoint{4.599648in}{1.524906in}}%
\pgfpathlineto{\pgfqpoint{4.585450in}{1.529209in}}%
\pgfpathlineto{\pgfqpoint{4.571260in}{1.533536in}}%
\pgfpathlineto{\pgfqpoint{4.557076in}{1.537887in}}%
\pgfpathlineto{\pgfqpoint{4.542898in}{1.542262in}}%
\pgfpathlineto{\pgfqpoint{4.550821in}{1.548601in}}%
\pgfpathlineto{\pgfqpoint{4.558739in}{1.555153in}}%
\pgfpathlineto{\pgfqpoint{4.566652in}{1.561911in}}%
\pgfpathlineto{\pgfqpoint{4.574561in}{1.568868in}}%
\pgfpathclose%
\pgfusepath{fill}%
\end{pgfscope}%
\begin{pgfscope}%
\pgfpathrectangle{\pgfqpoint{1.150000in}{0.150000in}}{\pgfqpoint{5.700000in}{5.700000in}}%
\pgfusepath{clip}%
\pgfsetbuttcap%
\pgfsetroundjoin%
\definecolor{currentfill}{rgb}{0.282327,0.094955,0.417331}%
\pgfsetfillcolor{currentfill}%
\pgfsetfillopacity{0.700000}%
\pgfsetlinewidth{0.000000pt}%
\definecolor{currentstroke}{rgb}{0.000000,0.000000,0.000000}%
\pgfsetstrokecolor{currentstroke}%
\pgfsetdash{}{0pt}%
\pgfpathmoveto{\pgfqpoint{3.939269in}{1.693160in}}%
\pgfpathlineto{\pgfqpoint{3.953287in}{1.686962in}}%
\pgfpathlineto{\pgfqpoint{3.967310in}{1.680790in}}%
\pgfpathlineto{\pgfqpoint{3.981338in}{1.674643in}}%
\pgfpathlineto{\pgfqpoint{3.995371in}{1.668521in}}%
\pgfpathlineto{\pgfqpoint{3.987210in}{1.668359in}}%
\pgfpathlineto{\pgfqpoint{3.979039in}{1.668553in}}%
\pgfpathlineto{\pgfqpoint{3.970859in}{1.669113in}}%
\pgfpathlineto{\pgfqpoint{3.962668in}{1.670048in}}%
\pgfpathlineto{\pgfqpoint{3.948610in}{1.676511in}}%
\pgfpathlineto{\pgfqpoint{3.934556in}{1.682999in}}%
\pgfpathlineto{\pgfqpoint{3.920508in}{1.689513in}}%
\pgfpathlineto{\pgfqpoint{3.906464in}{1.696052in}}%
\pgfpathlineto{\pgfqpoint{3.914681in}{1.694770in}}%
\pgfpathlineto{\pgfqpoint{3.922887in}{1.693867in}}%
\pgfpathlineto{\pgfqpoint{3.931083in}{1.693334in}}%
\pgfpathlineto{\pgfqpoint{3.939269in}{1.693160in}}%
\pgfpathclose%
\pgfusepath{fill}%
\end{pgfscope}%
\begin{pgfscope}%
\pgfpathrectangle{\pgfqpoint{1.150000in}{0.150000in}}{\pgfqpoint{5.700000in}{5.700000in}}%
\pgfusepath{clip}%
\pgfsetbuttcap%
\pgfsetroundjoin%
\definecolor{currentfill}{rgb}{0.277941,0.056324,0.381191}%
\pgfsetfillcolor{currentfill}%
\pgfsetfillopacity{0.700000}%
\pgfsetlinewidth{0.000000pt}%
\definecolor{currentstroke}{rgb}{0.000000,0.000000,0.000000}%
\pgfsetstrokecolor{currentstroke}%
\pgfsetdash{}{0pt}%
\pgfpathmoveto{\pgfqpoint{5.041450in}{1.634381in}}%
\pgfpathlineto{\pgfqpoint{5.055754in}{1.631879in}}%
\pgfpathlineto{\pgfqpoint{5.070066in}{1.629400in}}%
\pgfpathlineto{\pgfqpoint{5.084386in}{1.626945in}}%
\pgfpathlineto{\pgfqpoint{5.098714in}{1.624515in}}%
\pgfpathlineto{\pgfqpoint{5.090934in}{1.613691in}}%
\pgfpathlineto{\pgfqpoint{5.083151in}{1.602940in}}%
\pgfpathlineto{\pgfqpoint{5.075364in}{1.592267in}}%
\pgfpathlineto{\pgfqpoint{5.067573in}{1.581677in}}%
\pgfpathlineto{\pgfqpoint{5.053238in}{1.584342in}}%
\pgfpathlineto{\pgfqpoint{5.038911in}{1.587031in}}%
\pgfpathlineto{\pgfqpoint{5.024592in}{1.589743in}}%
\pgfpathlineto{\pgfqpoint{5.010281in}{1.592480in}}%
\pgfpathlineto{\pgfqpoint{5.018078in}{1.602830in}}%
\pgfpathlineto{\pgfqpoint{5.025872in}{1.613268in}}%
\pgfpathlineto{\pgfqpoint{5.033663in}{1.623787in}}%
\pgfpathlineto{\pgfqpoint{5.041450in}{1.634381in}}%
\pgfpathclose%
\pgfusepath{fill}%
\end{pgfscope}%
\begin{pgfscope}%
\pgfpathrectangle{\pgfqpoint{1.150000in}{0.150000in}}{\pgfqpoint{5.700000in}{5.700000in}}%
\pgfusepath{clip}%
\pgfsetbuttcap%
\pgfsetroundjoin%
\definecolor{currentfill}{rgb}{0.201239,0.383670,0.554294}%
\pgfsetfillcolor{currentfill}%
\pgfsetfillopacity{0.700000}%
\pgfsetlinewidth{0.000000pt}%
\definecolor{currentstroke}{rgb}{0.000000,0.000000,0.000000}%
\pgfsetstrokecolor{currentstroke}%
\pgfsetdash{}{0pt}%
\pgfpathmoveto{\pgfqpoint{2.855132in}{2.330030in}}%
\pgfpathlineto{\pgfqpoint{2.869018in}{2.320387in}}%
\pgfpathlineto{\pgfqpoint{2.882907in}{2.310779in}}%
\pgfpathlineto{\pgfqpoint{2.896798in}{2.301205in}}%
\pgfpathlineto{\pgfqpoint{2.910692in}{2.291666in}}%
\pgfpathlineto{\pgfqpoint{2.901654in}{2.305165in}}%
\pgfpathlineto{\pgfqpoint{2.892590in}{2.319254in}}%
\pgfpathlineto{\pgfqpoint{2.883497in}{2.333945in}}%
\pgfpathlineto{\pgfqpoint{2.874376in}{2.349250in}}%
\pgfpathlineto{\pgfqpoint{2.860433in}{2.359200in}}%
\pgfpathlineto{\pgfqpoint{2.846492in}{2.369184in}}%
\pgfpathlineto{\pgfqpoint{2.832553in}{2.379203in}}%
\pgfpathlineto{\pgfqpoint{2.818617in}{2.389257in}}%
\pgfpathlineto{\pgfqpoint{2.827789in}{2.373535in}}%
\pgfpathlineto{\pgfqpoint{2.836932in}{2.358431in}}%
\pgfpathlineto{\pgfqpoint{2.846046in}{2.343933in}}%
\pgfpathlineto{\pgfqpoint{2.855132in}{2.330030in}}%
\pgfpathclose%
\pgfusepath{fill}%
\end{pgfscope}%
\begin{pgfscope}%
\pgfpathrectangle{\pgfqpoint{1.150000in}{0.150000in}}{\pgfqpoint{5.700000in}{5.700000in}}%
\pgfusepath{clip}%
\pgfsetbuttcap%
\pgfsetroundjoin%
\definecolor{currentfill}{rgb}{0.282884,0.135920,0.453427}%
\pgfsetfillcolor{currentfill}%
\pgfsetfillopacity{0.700000}%
\pgfsetlinewidth{0.000000pt}%
\definecolor{currentstroke}{rgb}{0.000000,0.000000,0.000000}%
\pgfsetstrokecolor{currentstroke}%
\pgfsetdash{}{0pt}%
\pgfpathmoveto{\pgfqpoint{3.738339in}{1.776526in}}%
\pgfpathlineto{\pgfqpoint{3.752322in}{1.769677in}}%
\pgfpathlineto{\pgfqpoint{3.766311in}{1.762854in}}%
\pgfpathlineto{\pgfqpoint{3.780303in}{1.756057in}}%
\pgfpathlineto{\pgfqpoint{3.794301in}{1.749286in}}%
\pgfpathlineto{\pgfqpoint{3.786018in}{1.751656in}}%
\pgfpathlineto{\pgfqpoint{3.777723in}{1.754430in}}%
\pgfpathlineto{\pgfqpoint{3.769415in}{1.757620in}}%
\pgfpathlineto{\pgfqpoint{3.761095in}{1.761233in}}%
\pgfpathlineto{\pgfqpoint{3.747068in}{1.768360in}}%
\pgfpathlineto{\pgfqpoint{3.733045in}{1.775513in}}%
\pgfpathlineto{\pgfqpoint{3.719027in}{1.782692in}}%
\pgfpathlineto{\pgfqpoint{3.705014in}{1.789897in}}%
\pgfpathlineto{\pgfqpoint{3.713364in}{1.785922in}}%
\pgfpathlineto{\pgfqpoint{3.721702in}{1.782375in}}%
\pgfpathlineto{\pgfqpoint{3.730027in}{1.779246in}}%
\pgfpathlineto{\pgfqpoint{3.738339in}{1.776526in}}%
\pgfpathclose%
\pgfusepath{fill}%
\end{pgfscope}%
\begin{pgfscope}%
\pgfpathrectangle{\pgfqpoint{1.150000in}{0.150000in}}{\pgfqpoint{5.700000in}{5.700000in}}%
\pgfusepath{clip}%
\pgfsetbuttcap%
\pgfsetroundjoin%
\definecolor{currentfill}{rgb}{0.140536,0.530132,0.555659}%
\pgfsetfillcolor{currentfill}%
\pgfsetfillopacity{0.700000}%
\pgfsetlinewidth{0.000000pt}%
\definecolor{currentstroke}{rgb}{0.000000,0.000000,0.000000}%
\pgfsetstrokecolor{currentstroke}%
\pgfsetdash{}{0pt}%
\pgfpathmoveto{\pgfqpoint{2.373703in}{2.731100in}}%
\pgfpathlineto{\pgfqpoint{2.387583in}{2.719774in}}%
\pgfpathlineto{\pgfqpoint{2.401464in}{2.708493in}}%
\pgfpathlineto{\pgfqpoint{2.415346in}{2.697257in}}%
\pgfpathlineto{\pgfqpoint{2.429229in}{2.686065in}}%
\pgfpathlineto{\pgfqpoint{2.419640in}{2.705400in}}%
\pgfpathlineto{\pgfqpoint{2.410013in}{2.725411in}}%
\pgfpathlineto{\pgfqpoint{2.400348in}{2.746111in}}%
\pgfpathlineto{\pgfqpoint{2.390644in}{2.767515in}}%
\pgfpathlineto{\pgfqpoint{2.376701in}{2.779144in}}%
\pgfpathlineto{\pgfqpoint{2.362758in}{2.790817in}}%
\pgfpathlineto{\pgfqpoint{2.348816in}{2.802536in}}%
\pgfpathlineto{\pgfqpoint{2.334876in}{2.814301in}}%
\pgfpathlineto{\pgfqpoint{2.344642in}{2.792452in}}%
\pgfpathlineto{\pgfqpoint{2.354367in}{2.771311in}}%
\pgfpathlineto{\pgfqpoint{2.364054in}{2.750865in}}%
\pgfpathlineto{\pgfqpoint{2.373703in}{2.731100in}}%
\pgfpathclose%
\pgfusepath{fill}%
\end{pgfscope}%
\begin{pgfscope}%
\pgfpathrectangle{\pgfqpoint{1.150000in}{0.150000in}}{\pgfqpoint{5.700000in}{5.700000in}}%
\pgfusepath{clip}%
\pgfsetbuttcap%
\pgfsetroundjoin%
\definecolor{currentfill}{rgb}{0.248629,0.278775,0.534556}%
\pgfsetfillcolor{currentfill}%
\pgfsetfillopacity{0.700000}%
\pgfsetlinewidth{0.000000pt}%
\definecolor{currentstroke}{rgb}{0.000000,0.000000,0.000000}%
\pgfsetstrokecolor{currentstroke}%
\pgfsetdash{}{0pt}%
\pgfpathmoveto{\pgfqpoint{3.224208in}{2.067259in}}%
\pgfpathlineto{\pgfqpoint{3.238125in}{2.058783in}}%
\pgfpathlineto{\pgfqpoint{3.252045in}{2.050338in}}%
\pgfpathlineto{\pgfqpoint{3.265969in}{2.041922in}}%
\pgfpathlineto{\pgfqpoint{3.279897in}{2.033536in}}%
\pgfpathlineto{\pgfqpoint{3.271215in}{2.042482in}}%
\pgfpathlineto{\pgfqpoint{3.262513in}{2.051946in}}%
\pgfpathlineto{\pgfqpoint{3.253791in}{2.061941in}}%
\pgfpathlineto{\pgfqpoint{3.245047in}{2.072476in}}%
\pgfpathlineto{\pgfqpoint{3.231079in}{2.081252in}}%
\pgfpathlineto{\pgfqpoint{3.217113in}{2.090057in}}%
\pgfpathlineto{\pgfqpoint{3.203152in}{2.098893in}}%
\pgfpathlineto{\pgfqpoint{3.189193in}{2.107758in}}%
\pgfpathlineto{\pgfqpoint{3.197979in}{2.096827in}}%
\pgfpathlineto{\pgfqpoint{3.206744in}{2.086441in}}%
\pgfpathlineto{\pgfqpoint{3.215486in}{2.076588in}}%
\pgfpathlineto{\pgfqpoint{3.224208in}{2.067259in}}%
\pgfpathclose%
\pgfusepath{fill}%
\end{pgfscope}%
\begin{pgfscope}%
\pgfpathrectangle{\pgfqpoint{1.150000in}{0.150000in}}{\pgfqpoint{5.700000in}{5.700000in}}%
\pgfusepath{clip}%
\pgfsetbuttcap%
\pgfsetroundjoin%
\definecolor{currentfill}{rgb}{0.278791,0.062145,0.386592}%
\pgfsetfillcolor{currentfill}%
\pgfsetfillopacity{0.700000}%
\pgfsetlinewidth{0.000000pt}%
\definecolor{currentstroke}{rgb}{0.000000,0.000000,0.000000}%
\pgfsetstrokecolor{currentstroke}%
\pgfsetdash{}{0pt}%
\pgfpathmoveto{\pgfqpoint{4.140212in}{1.627094in}}%
\pgfpathlineto{\pgfqpoint{4.154273in}{1.621524in}}%
\pgfpathlineto{\pgfqpoint{4.168341in}{1.615978in}}%
\pgfpathlineto{\pgfqpoint{4.182414in}{1.610457in}}%
\pgfpathlineto{\pgfqpoint{4.196493in}{1.604961in}}%
\pgfpathlineto{\pgfqpoint{4.188432in}{1.602498in}}%
\pgfpathlineto{\pgfqpoint{4.180364in}{1.600347in}}%
\pgfpathlineto{\pgfqpoint{4.172288in}{1.598514in}}%
\pgfpathlineto{\pgfqpoint{4.164205in}{1.597009in}}%
\pgfpathlineto{\pgfqpoint{4.150104in}{1.602832in}}%
\pgfpathlineto{\pgfqpoint{4.136010in}{1.608679in}}%
\pgfpathlineto{\pgfqpoint{4.121921in}{1.614552in}}%
\pgfpathlineto{\pgfqpoint{4.107837in}{1.620448in}}%
\pgfpathlineto{\pgfqpoint{4.115943in}{1.621622in}}%
\pgfpathlineto{\pgfqpoint{4.124040in}{1.623126in}}%
\pgfpathlineto{\pgfqpoint{4.132130in}{1.624953in}}%
\pgfpathlineto{\pgfqpoint{4.140212in}{1.627094in}}%
\pgfpathclose%
\pgfusepath{fill}%
\end{pgfscope}%
\begin{pgfscope}%
\pgfpathrectangle{\pgfqpoint{1.150000in}{0.150000in}}{\pgfqpoint{5.700000in}{5.700000in}}%
\pgfusepath{clip}%
\pgfsetbuttcap%
\pgfsetroundjoin%
\definecolor{currentfill}{rgb}{0.271305,0.019942,0.347269}%
\pgfsetfillcolor{currentfill}%
\pgfsetfillopacity{0.700000}%
\pgfsetlinewidth{0.000000pt}%
\definecolor{currentstroke}{rgb}{0.000000,0.000000,0.000000}%
\pgfsetstrokecolor{currentstroke}%
\pgfsetdash{}{0pt}%
\pgfpathmoveto{\pgfqpoint{4.719560in}{1.568679in}}%
\pgfpathlineto{\pgfqpoint{4.733768in}{1.565054in}}%
\pgfpathlineto{\pgfqpoint{4.747983in}{1.561454in}}%
\pgfpathlineto{\pgfqpoint{4.762206in}{1.557877in}}%
\pgfpathlineto{\pgfqpoint{4.776436in}{1.554324in}}%
\pgfpathlineto{\pgfqpoint{4.768579in}{1.545840in}}%
\pgfpathlineto{\pgfqpoint{4.760719in}{1.537517in}}%
\pgfpathlineto{\pgfqpoint{4.752854in}{1.529361in}}%
\pgfpathlineto{\pgfqpoint{4.744986in}{1.521380in}}%
\pgfpathlineto{\pgfqpoint{4.730745in}{1.525206in}}%
\pgfpathlineto{\pgfqpoint{4.716511in}{1.529056in}}%
\pgfpathlineto{\pgfqpoint{4.702285in}{1.532929in}}%
\pgfpathlineto{\pgfqpoint{4.688065in}{1.536827in}}%
\pgfpathlineto{\pgfqpoint{4.695945in}{1.544530in}}%
\pgfpathlineto{\pgfqpoint{4.703820in}{1.552412in}}%
\pgfpathlineto{\pgfqpoint{4.711692in}{1.560464in}}%
\pgfpathlineto{\pgfqpoint{4.719560in}{1.568679in}}%
\pgfpathclose%
\pgfusepath{fill}%
\end{pgfscope}%
\begin{pgfscope}%
\pgfpathrectangle{\pgfqpoint{1.150000in}{0.150000in}}{\pgfqpoint{5.700000in}{5.700000in}}%
\pgfusepath{clip}%
\pgfsetbuttcap%
\pgfsetroundjoin%
\definecolor{currentfill}{rgb}{0.276022,0.044167,0.370164}%
\pgfsetfillcolor{currentfill}%
\pgfsetfillopacity{0.700000}%
\pgfsetlinewidth{0.000000pt}%
\definecolor{currentstroke}{rgb}{0.000000,0.000000,0.000000}%
\pgfsetstrokecolor{currentstroke}%
\pgfsetdash{}{0pt}%
\pgfpathmoveto{\pgfqpoint{4.953114in}{1.603664in}}%
\pgfpathlineto{\pgfqpoint{4.967394in}{1.600832in}}%
\pgfpathlineto{\pgfqpoint{4.981682in}{1.598024in}}%
\pgfpathlineto{\pgfqpoint{4.995977in}{1.595240in}}%
\pgfpathlineto{\pgfqpoint{5.010281in}{1.592480in}}%
\pgfpathlineto{\pgfqpoint{5.002479in}{1.582222in}}%
\pgfpathlineto{\pgfqpoint{4.994674in}{1.572064in}}%
\pgfpathlineto{\pgfqpoint{4.986866in}{1.562011in}}%
\pgfpathlineto{\pgfqpoint{4.979054in}{1.552069in}}%
\pgfpathlineto{\pgfqpoint{4.964743in}{1.555077in}}%
\pgfpathlineto{\pgfqpoint{4.950440in}{1.558108in}}%
\pgfpathlineto{\pgfqpoint{4.936144in}{1.561163in}}%
\pgfpathlineto{\pgfqpoint{4.921856in}{1.564241in}}%
\pgfpathlineto{\pgfqpoint{4.929676in}{1.573931in}}%
\pgfpathlineto{\pgfqpoint{4.937492in}{1.583735in}}%
\pgfpathlineto{\pgfqpoint{4.945305in}{1.593648in}}%
\pgfpathlineto{\pgfqpoint{4.953114in}{1.603664in}}%
\pgfpathclose%
\pgfusepath{fill}%
\end{pgfscope}%
\begin{pgfscope}%
\pgfpathrectangle{\pgfqpoint{1.150000in}{0.150000in}}{\pgfqpoint{5.700000in}{5.700000in}}%
\pgfusepath{clip}%
\pgfsetbuttcap%
\pgfsetroundjoin%
\definecolor{currentfill}{rgb}{0.283197,0.115680,0.436115}%
\pgfsetfillcolor{currentfill}%
\pgfsetfillopacity{0.700000}%
\pgfsetlinewidth{0.000000pt}%
\definecolor{currentstroke}{rgb}{0.000000,0.000000,0.000000}%
\pgfsetstrokecolor{currentstroke}%
\pgfsetdash{}{0pt}%
\pgfpathmoveto{\pgfqpoint{5.364137in}{1.738261in}}%
\pgfpathlineto{\pgfqpoint{5.378558in}{1.736766in}}%
\pgfpathlineto{\pgfqpoint{5.392988in}{1.735294in}}%
\pgfpathlineto{\pgfqpoint{5.407427in}{1.733846in}}%
\pgfpathlineto{\pgfqpoint{5.399717in}{1.721856in}}%
\pgfpathlineto{\pgfqpoint{5.392003in}{1.709864in}}%
\pgfpathlineto{\pgfqpoint{5.384284in}{1.697875in}}%
\pgfpathlineto{\pgfqpoint{5.376561in}{1.685895in}}%
\pgfpathlineto{\pgfqpoint{5.362117in}{1.687537in}}%
\pgfpathlineto{\pgfqpoint{5.347682in}{1.689203in}}%
\pgfpathlineto{\pgfqpoint{5.333256in}{1.690894in}}%
\pgfpathlineto{\pgfqpoint{5.340983in}{1.702725in}}%
\pgfpathlineto{\pgfqpoint{5.348706in}{1.714566in}}%
\pgfpathlineto{\pgfqpoint{5.356424in}{1.726413in}}%
\pgfpathlineto{\pgfqpoint{5.364137in}{1.738261in}}%
\pgfpathclose%
\pgfusepath{fill}%
\end{pgfscope}%
\begin{pgfscope}%
\pgfpathrectangle{\pgfqpoint{1.150000in}{0.150000in}}{\pgfqpoint{5.700000in}{5.700000in}}%
\pgfusepath{clip}%
\pgfsetbuttcap%
\pgfsetroundjoin%
\definecolor{currentfill}{rgb}{0.144759,0.519093,0.556572}%
\pgfsetfillcolor{currentfill}%
\pgfsetfillopacity{0.700000}%
\pgfsetlinewidth{0.000000pt}%
\definecolor{currentstroke}{rgb}{0.000000,0.000000,0.000000}%
\pgfsetstrokecolor{currentstroke}%
\pgfsetdash{}{0pt}%
\pgfpathmoveto{\pgfqpoint{2.429229in}{2.686065in}}%
\pgfpathlineto{\pgfqpoint{2.443114in}{2.674917in}}%
\pgfpathlineto{\pgfqpoint{2.456999in}{2.663813in}}%
\pgfpathlineto{\pgfqpoint{2.470887in}{2.652752in}}%
\pgfpathlineto{\pgfqpoint{2.484775in}{2.641734in}}%
\pgfpathlineto{\pgfqpoint{2.475244in}{2.660639in}}%
\pgfpathlineto{\pgfqpoint{2.465676in}{2.680216in}}%
\pgfpathlineto{\pgfqpoint{2.456071in}{2.700478in}}%
\pgfpathlineto{\pgfqpoint{2.446428in}{2.721438in}}%
\pgfpathlineto{\pgfqpoint{2.432481in}{2.732892in}}%
\pgfpathlineto{\pgfqpoint{2.418534in}{2.744389in}}%
\pgfpathlineto{\pgfqpoint{2.404589in}{2.755930in}}%
\pgfpathlineto{\pgfqpoint{2.390644in}{2.767515in}}%
\pgfpathlineto{\pgfqpoint{2.400348in}{2.746111in}}%
\pgfpathlineto{\pgfqpoint{2.410013in}{2.725411in}}%
\pgfpathlineto{\pgfqpoint{2.419640in}{2.705400in}}%
\pgfpathlineto{\pgfqpoint{2.429229in}{2.686065in}}%
\pgfpathclose%
\pgfusepath{fill}%
\end{pgfscope}%
\begin{pgfscope}%
\pgfpathrectangle{\pgfqpoint{1.150000in}{0.150000in}}{\pgfqpoint{5.700000in}{5.700000in}}%
\pgfusepath{clip}%
\pgfsetbuttcap%
\pgfsetroundjoin%
\definecolor{currentfill}{rgb}{0.276194,0.190074,0.493001}%
\pgfsetfillcolor{currentfill}%
\pgfsetfillopacity{0.700000}%
\pgfsetlinewidth{0.000000pt}%
\definecolor{currentstroke}{rgb}{0.000000,0.000000,0.000000}%
\pgfsetstrokecolor{currentstroke}%
\pgfsetdash{}{0pt}%
\pgfpathmoveto{\pgfqpoint{3.537208in}{1.878448in}}%
\pgfpathlineto{\pgfqpoint{3.551168in}{1.870920in}}%
\pgfpathlineto{\pgfqpoint{3.565131in}{1.863419in}}%
\pgfpathlineto{\pgfqpoint{3.579099in}{1.855945in}}%
\pgfpathlineto{\pgfqpoint{3.593072in}{1.848499in}}%
\pgfpathlineto{\pgfqpoint{3.584642in}{1.853644in}}%
\pgfpathlineto{\pgfqpoint{3.576198in}{1.859246in}}%
\pgfpathlineto{\pgfqpoint{3.567738in}{1.865314in}}%
\pgfpathlineto{\pgfqpoint{3.559262in}{1.871859in}}%
\pgfpathlineto{\pgfqpoint{3.545255in}{1.879678in}}%
\pgfpathlineto{\pgfqpoint{3.531253in}{1.887523in}}%
\pgfpathlineto{\pgfqpoint{3.517254in}{1.895396in}}%
\pgfpathlineto{\pgfqpoint{3.503260in}{1.903296in}}%
\pgfpathlineto{\pgfqpoint{3.511771in}{1.896373in}}%
\pgfpathlineto{\pgfqpoint{3.520266in}{1.889931in}}%
\pgfpathlineto{\pgfqpoint{3.528745in}{1.883959in}}%
\pgfpathlineto{\pgfqpoint{3.537208in}{1.878448in}}%
\pgfpathclose%
\pgfusepath{fill}%
\end{pgfscope}%
\begin{pgfscope}%
\pgfpathrectangle{\pgfqpoint{1.150000in}{0.150000in}}{\pgfqpoint{5.700000in}{5.700000in}}%
\pgfusepath{clip}%
\pgfsetbuttcap%
\pgfsetroundjoin%
\definecolor{currentfill}{rgb}{0.206756,0.371758,0.553117}%
\pgfsetfillcolor{currentfill}%
\pgfsetfillopacity{0.700000}%
\pgfsetlinewidth{0.000000pt}%
\definecolor{currentstroke}{rgb}{0.000000,0.000000,0.000000}%
\pgfsetstrokecolor{currentstroke}%
\pgfsetdash{}{0pt}%
\pgfpathmoveto{\pgfqpoint{2.910692in}{2.291666in}}%
\pgfpathlineto{\pgfqpoint{2.924589in}{2.282160in}}%
\pgfpathlineto{\pgfqpoint{2.938489in}{2.272689in}}%
\pgfpathlineto{\pgfqpoint{2.952391in}{2.263251in}}%
\pgfpathlineto{\pgfqpoint{2.966297in}{2.253846in}}%
\pgfpathlineto{\pgfqpoint{2.957306in}{2.266942in}}%
\pgfpathlineto{\pgfqpoint{2.948290in}{2.280623in}}%
\pgfpathlineto{\pgfqpoint{2.939246in}{2.294902in}}%
\pgfpathlineto{\pgfqpoint{2.930176in}{2.309790in}}%
\pgfpathlineto{\pgfqpoint{2.916222in}{2.319604in}}%
\pgfpathlineto{\pgfqpoint{2.902271in}{2.329452in}}%
\pgfpathlineto{\pgfqpoint{2.888322in}{2.339334in}}%
\pgfpathlineto{\pgfqpoint{2.874376in}{2.349250in}}%
\pgfpathlineto{\pgfqpoint{2.883497in}{2.333945in}}%
\pgfpathlineto{\pgfqpoint{2.892590in}{2.319254in}}%
\pgfpathlineto{\pgfqpoint{2.901654in}{2.305165in}}%
\pgfpathlineto{\pgfqpoint{2.910692in}{2.291666in}}%
\pgfpathclose%
\pgfusepath{fill}%
\end{pgfscope}%
\begin{pgfscope}%
\pgfpathrectangle{\pgfqpoint{1.150000in}{0.150000in}}{\pgfqpoint{5.700000in}{5.700000in}}%
\pgfusepath{clip}%
\pgfsetbuttcap%
\pgfsetroundjoin%
\definecolor{currentfill}{rgb}{0.282327,0.094955,0.417331}%
\pgfsetfillcolor{currentfill}%
\pgfsetfillopacity{0.700000}%
\pgfsetlinewidth{0.000000pt}%
\definecolor{currentstroke}{rgb}{0.000000,0.000000,0.000000}%
\pgfsetstrokecolor{currentstroke}%
\pgfsetdash{}{0pt}%
\pgfpathmoveto{\pgfqpoint{5.275639in}{1.697896in}}%
\pgfpathlineto{\pgfqpoint{5.290030in}{1.696109in}}%
\pgfpathlineto{\pgfqpoint{5.304430in}{1.694347in}}%
\pgfpathlineto{\pgfqpoint{5.318839in}{1.692608in}}%
\pgfpathlineto{\pgfqpoint{5.333256in}{1.690894in}}%
\pgfpathlineto{\pgfqpoint{5.325525in}{1.679078in}}%
\pgfpathlineto{\pgfqpoint{5.317790in}{1.667281in}}%
\pgfpathlineto{\pgfqpoint{5.310050in}{1.655510in}}%
\pgfpathlineto{\pgfqpoint{5.302307in}{1.643769in}}%
\pgfpathlineto{\pgfqpoint{5.287884in}{1.645692in}}%
\pgfpathlineto{\pgfqpoint{5.273470in}{1.647638in}}%
\pgfpathlineto{\pgfqpoint{5.259065in}{1.649609in}}%
\pgfpathlineto{\pgfqpoint{5.244668in}{1.651603in}}%
\pgfpathlineto{\pgfqpoint{5.252417in}{1.663131in}}%
\pgfpathlineto{\pgfqpoint{5.260162in}{1.674693in}}%
\pgfpathlineto{\pgfqpoint{5.267902in}{1.686283in}}%
\pgfpathlineto{\pgfqpoint{5.275639in}{1.697896in}}%
\pgfpathclose%
\pgfusepath{fill}%
\end{pgfscope}%
\begin{pgfscope}%
\pgfpathrectangle{\pgfqpoint{1.150000in}{0.150000in}}{\pgfqpoint{5.700000in}{5.700000in}}%
\pgfusepath{clip}%
\pgfsetbuttcap%
\pgfsetroundjoin%
\definecolor{currentfill}{rgb}{0.273809,0.031497,0.358853}%
\pgfsetfillcolor{currentfill}%
\pgfsetfillopacity{0.700000}%
\pgfsetlinewidth{0.000000pt}%
\definecolor{currentstroke}{rgb}{0.000000,0.000000,0.000000}%
\pgfsetstrokecolor{currentstroke}%
\pgfsetdash{}{0pt}%
\pgfpathmoveto{\pgfqpoint{4.864781in}{1.576794in}}%
\pgfpathlineto{\pgfqpoint{4.879038in}{1.573620in}}%
\pgfpathlineto{\pgfqpoint{4.893303in}{1.570470in}}%
\pgfpathlineto{\pgfqpoint{4.907576in}{1.567344in}}%
\pgfpathlineto{\pgfqpoint{4.921856in}{1.564241in}}%
\pgfpathlineto{\pgfqpoint{4.914033in}{1.554673in}}%
\pgfpathlineto{\pgfqpoint{4.906206in}{1.545231in}}%
\pgfpathlineto{\pgfqpoint{4.898376in}{1.535924in}}%
\pgfpathlineto{\pgfqpoint{4.890542in}{1.526756in}}%
\pgfpathlineto{\pgfqpoint{4.876253in}{1.530119in}}%
\pgfpathlineto{\pgfqpoint{4.861971in}{1.533506in}}%
\pgfpathlineto{\pgfqpoint{4.847697in}{1.536916in}}%
\pgfpathlineto{\pgfqpoint{4.833430in}{1.540350in}}%
\pgfpathlineto{\pgfqpoint{4.841273in}{1.549252in}}%
\pgfpathlineto{\pgfqpoint{4.849112in}{1.558298in}}%
\pgfpathlineto{\pgfqpoint{4.856948in}{1.567481in}}%
\pgfpathlineto{\pgfqpoint{4.864781in}{1.576794in}}%
\pgfpathclose%
\pgfusepath{fill}%
\end{pgfscope}%
\begin{pgfscope}%
\pgfpathrectangle{\pgfqpoint{1.150000in}{0.150000in}}{\pgfqpoint{5.700000in}{5.700000in}}%
\pgfusepath{clip}%
\pgfsetbuttcap%
\pgfsetroundjoin%
\definecolor{currentfill}{rgb}{0.281924,0.089666,0.412415}%
\pgfsetfillcolor{currentfill}%
\pgfsetfillopacity{0.700000}%
\pgfsetlinewidth{0.000000pt}%
\definecolor{currentstroke}{rgb}{0.000000,0.000000,0.000000}%
\pgfsetstrokecolor{currentstroke}%
\pgfsetdash{}{0pt}%
\pgfpathmoveto{\pgfqpoint{3.995371in}{1.668521in}}%
\pgfpathlineto{\pgfqpoint{4.009410in}{1.662424in}}%
\pgfpathlineto{\pgfqpoint{4.023455in}{1.656352in}}%
\pgfpathlineto{\pgfqpoint{4.037505in}{1.650306in}}%
\pgfpathlineto{\pgfqpoint{4.051560in}{1.644285in}}%
\pgfpathlineto{\pgfqpoint{4.043423in}{1.643787in}}%
\pgfpathlineto{\pgfqpoint{4.035277in}{1.643643in}}%
\pgfpathlineto{\pgfqpoint{4.027121in}{1.643860in}}%
\pgfpathlineto{\pgfqpoint{4.018957in}{1.644448in}}%
\pgfpathlineto{\pgfqpoint{4.004877in}{1.650811in}}%
\pgfpathlineto{\pgfqpoint{3.990802in}{1.657198in}}%
\pgfpathlineto{\pgfqpoint{3.976733in}{1.663611in}}%
\pgfpathlineto{\pgfqpoint{3.962668in}{1.670048in}}%
\pgfpathlineto{\pgfqpoint{3.970859in}{1.669113in}}%
\pgfpathlineto{\pgfqpoint{3.979039in}{1.668553in}}%
\pgfpathlineto{\pgfqpoint{3.987210in}{1.668359in}}%
\pgfpathlineto{\pgfqpoint{3.995371in}{1.668521in}}%
\pgfpathclose%
\pgfusepath{fill}%
\end{pgfscope}%
\begin{pgfscope}%
\pgfpathrectangle{\pgfqpoint{1.150000in}{0.150000in}}{\pgfqpoint{5.700000in}{5.700000in}}%
\pgfusepath{clip}%
\pgfsetbuttcap%
\pgfsetroundjoin%
\definecolor{currentfill}{rgb}{0.272594,0.025563,0.353093}%
\pgfsetfillcolor{currentfill}%
\pgfsetfillopacity{0.700000}%
\pgfsetlinewidth{0.000000pt}%
\definecolor{currentstroke}{rgb}{0.000000,0.000000,0.000000}%
\pgfsetstrokecolor{currentstroke}%
\pgfsetdash{}{0pt}%
\pgfpathmoveto{\pgfqpoint{4.486256in}{1.560002in}}%
\pgfpathlineto{\pgfqpoint{4.500407in}{1.555531in}}%
\pgfpathlineto{\pgfqpoint{4.514564in}{1.551084in}}%
\pgfpathlineto{\pgfqpoint{4.528728in}{1.546661in}}%
\pgfpathlineto{\pgfqpoint{4.542898in}{1.542262in}}%
\pgfpathlineto{\pgfqpoint{4.534971in}{1.536144in}}%
\pgfpathlineto{\pgfqpoint{4.527039in}{1.530254in}}%
\pgfpathlineto{\pgfqpoint{4.519103in}{1.524600in}}%
\pgfpathlineto{\pgfqpoint{4.511161in}{1.519190in}}%
\pgfpathlineto{\pgfqpoint{4.496976in}{1.523889in}}%
\pgfpathlineto{\pgfqpoint{4.482797in}{1.528611in}}%
\pgfpathlineto{\pgfqpoint{4.468624in}{1.533357in}}%
\pgfpathlineto{\pgfqpoint{4.454458in}{1.538128in}}%
\pgfpathlineto{\pgfqpoint{4.462416in}{1.543233in}}%
\pgfpathlineto{\pgfqpoint{4.470368in}{1.548586in}}%
\pgfpathlineto{\pgfqpoint{4.478315in}{1.554178in}}%
\pgfpathlineto{\pgfqpoint{4.486256in}{1.560002in}}%
\pgfpathclose%
\pgfusepath{fill}%
\end{pgfscope}%
\begin{pgfscope}%
\pgfpathrectangle{\pgfqpoint{1.150000in}{0.150000in}}{\pgfqpoint{5.700000in}{5.700000in}}%
\pgfusepath{clip}%
\pgfsetbuttcap%
\pgfsetroundjoin%
\definecolor{currentfill}{rgb}{0.253935,0.265254,0.529983}%
\pgfsetfillcolor{currentfill}%
\pgfsetfillopacity{0.700000}%
\pgfsetlinewidth{0.000000pt}%
\definecolor{currentstroke}{rgb}{0.000000,0.000000,0.000000}%
\pgfsetstrokecolor{currentstroke}%
\pgfsetdash{}{0pt}%
\pgfpathmoveto{\pgfqpoint{3.279897in}{2.033536in}}%
\pgfpathlineto{\pgfqpoint{3.293828in}{2.025180in}}%
\pgfpathlineto{\pgfqpoint{3.307763in}{2.016853in}}%
\pgfpathlineto{\pgfqpoint{3.321702in}{2.008555in}}%
\pgfpathlineto{\pgfqpoint{3.335644in}{2.000287in}}%
\pgfpathlineto{\pgfqpoint{3.327002in}{2.008850in}}%
\pgfpathlineto{\pgfqpoint{3.318341in}{2.017927in}}%
\pgfpathlineto{\pgfqpoint{3.309659in}{2.027529in}}%
\pgfpathlineto{\pgfqpoint{3.300958in}{2.037669in}}%
\pgfpathlineto{\pgfqpoint{3.286975in}{2.046327in}}%
\pgfpathlineto{\pgfqpoint{3.272995in}{2.055014in}}%
\pgfpathlineto{\pgfqpoint{3.259020in}{2.063730in}}%
\pgfpathlineto{\pgfqpoint{3.245047in}{2.072476in}}%
\pgfpathlineto{\pgfqpoint{3.253791in}{2.061941in}}%
\pgfpathlineto{\pgfqpoint{3.262513in}{2.051946in}}%
\pgfpathlineto{\pgfqpoint{3.271215in}{2.042482in}}%
\pgfpathlineto{\pgfqpoint{3.279897in}{2.033536in}}%
\pgfpathclose%
\pgfusepath{fill}%
\end{pgfscope}%
\begin{pgfscope}%
\pgfpathrectangle{\pgfqpoint{1.150000in}{0.150000in}}{\pgfqpoint{5.700000in}{5.700000in}}%
\pgfusepath{clip}%
\pgfsetbuttcap%
\pgfsetroundjoin%
\definecolor{currentfill}{rgb}{0.283072,0.130895,0.449241}%
\pgfsetfillcolor{currentfill}%
\pgfsetfillopacity{0.700000}%
\pgfsetlinewidth{0.000000pt}%
\definecolor{currentstroke}{rgb}{0.000000,0.000000,0.000000}%
\pgfsetstrokecolor{currentstroke}%
\pgfsetdash{}{0pt}%
\pgfpathmoveto{\pgfqpoint{3.794301in}{1.749286in}}%
\pgfpathlineto{\pgfqpoint{3.808304in}{1.742541in}}%
\pgfpathlineto{\pgfqpoint{3.822312in}{1.735823in}}%
\pgfpathlineto{\pgfqpoint{3.836325in}{1.729130in}}%
\pgfpathlineto{\pgfqpoint{3.850342in}{1.722463in}}%
\pgfpathlineto{\pgfqpoint{3.842088in}{1.724482in}}%
\pgfpathlineto{\pgfqpoint{3.833821in}{1.726903in}}%
\pgfpathlineto{\pgfqpoint{3.825543in}{1.729734in}}%
\pgfpathlineto{\pgfqpoint{3.817253in}{1.732987in}}%
\pgfpathlineto{\pgfqpoint{3.803206in}{1.740009in}}%
\pgfpathlineto{\pgfqpoint{3.789165in}{1.747058in}}%
\pgfpathlineto{\pgfqpoint{3.775128in}{1.754133in}}%
\pgfpathlineto{\pgfqpoint{3.761095in}{1.761233in}}%
\pgfpathlineto{\pgfqpoint{3.769415in}{1.757620in}}%
\pgfpathlineto{\pgfqpoint{3.777723in}{1.754430in}}%
\pgfpathlineto{\pgfqpoint{3.786018in}{1.751656in}}%
\pgfpathlineto{\pgfqpoint{3.794301in}{1.749286in}}%
\pgfpathclose%
\pgfusepath{fill}%
\end{pgfscope}%
\begin{pgfscope}%
\pgfpathrectangle{\pgfqpoint{1.150000in}{0.150000in}}{\pgfqpoint{5.700000in}{5.700000in}}%
\pgfusepath{clip}%
\pgfsetbuttcap%
\pgfsetroundjoin%
\definecolor{currentfill}{rgb}{0.274952,0.037752,0.364543}%
\pgfsetfillcolor{currentfill}%
\pgfsetfillopacity{0.700000}%
\pgfsetlinewidth{0.000000pt}%
\definecolor{currentstroke}{rgb}{0.000000,0.000000,0.000000}%
\pgfsetstrokecolor{currentstroke}%
\pgfsetdash{}{0pt}%
\pgfpathmoveto{\pgfqpoint{4.341362in}{1.577158in}}%
\pgfpathlineto{\pgfqpoint{4.355477in}{1.572195in}}%
\pgfpathlineto{\pgfqpoint{4.369598in}{1.567256in}}%
\pgfpathlineto{\pgfqpoint{4.383725in}{1.562341in}}%
\pgfpathlineto{\pgfqpoint{4.397859in}{1.557450in}}%
\pgfpathlineto{\pgfqpoint{4.389880in}{1.552907in}}%
\pgfpathlineto{\pgfqpoint{4.381895in}{1.548632in}}%
\pgfpathlineto{\pgfqpoint{4.373905in}{1.544631in}}%
\pgfpathlineto{\pgfqpoint{4.365909in}{1.540913in}}%
\pgfpathlineto{\pgfqpoint{4.351757in}{1.546116in}}%
\pgfpathlineto{\pgfqpoint{4.337612in}{1.551344in}}%
\pgfpathlineto{\pgfqpoint{4.323473in}{1.556596in}}%
\pgfpathlineto{\pgfqpoint{4.309340in}{1.561873in}}%
\pgfpathlineto{\pgfqpoint{4.317354in}{1.565273in}}%
\pgfpathlineto{\pgfqpoint{4.325363in}{1.568959in}}%
\pgfpathlineto{\pgfqpoint{4.333365in}{1.572924in}}%
\pgfpathlineto{\pgfqpoint{4.341362in}{1.577158in}}%
\pgfpathclose%
\pgfusepath{fill}%
\end{pgfscope}%
\begin{pgfscope}%
\pgfpathrectangle{\pgfqpoint{1.150000in}{0.150000in}}{\pgfqpoint{5.700000in}{5.700000in}}%
\pgfusepath{clip}%
\pgfsetbuttcap%
\pgfsetroundjoin%
\definecolor{currentfill}{rgb}{0.280894,0.078907,0.402329}%
\pgfsetfillcolor{currentfill}%
\pgfsetfillopacity{0.700000}%
\pgfsetlinewidth{0.000000pt}%
\definecolor{currentstroke}{rgb}{0.000000,0.000000,0.000000}%
\pgfsetstrokecolor{currentstroke}%
\pgfsetdash{}{0pt}%
\pgfpathmoveto{\pgfqpoint{5.187165in}{1.659821in}}%
\pgfpathlineto{\pgfqpoint{5.201528in}{1.657730in}}%
\pgfpathlineto{\pgfqpoint{5.215900in}{1.655664in}}%
\pgfpathlineto{\pgfqpoint{5.230280in}{1.653622in}}%
\pgfpathlineto{\pgfqpoint{5.244668in}{1.651603in}}%
\pgfpathlineto{\pgfqpoint{5.236915in}{1.640113in}}%
\pgfpathlineto{\pgfqpoint{5.229159in}{1.628668in}}%
\pgfpathlineto{\pgfqpoint{5.221399in}{1.617271in}}%
\pgfpathlineto{\pgfqpoint{5.213635in}{1.605928in}}%
\pgfpathlineto{\pgfqpoint{5.199241in}{1.608168in}}%
\pgfpathlineto{\pgfqpoint{5.184855in}{1.610432in}}%
\pgfpathlineto{\pgfqpoint{5.170478in}{1.612719in}}%
\pgfpathlineto{\pgfqpoint{5.156109in}{1.615030in}}%
\pgfpathlineto{\pgfqpoint{5.163878in}{1.626147in}}%
\pgfpathlineto{\pgfqpoint{5.171644in}{1.637321in}}%
\pgfpathlineto{\pgfqpoint{5.179407in}{1.648547in}}%
\pgfpathlineto{\pgfqpoint{5.187165in}{1.659821in}}%
\pgfpathclose%
\pgfusepath{fill}%
\end{pgfscope}%
\begin{pgfscope}%
\pgfpathrectangle{\pgfqpoint{1.150000in}{0.150000in}}{\pgfqpoint{5.700000in}{5.700000in}}%
\pgfusepath{clip}%
\pgfsetbuttcap%
\pgfsetroundjoin%
\definecolor{currentfill}{rgb}{0.150476,0.504369,0.557430}%
\pgfsetfillcolor{currentfill}%
\pgfsetfillopacity{0.700000}%
\pgfsetlinewidth{0.000000pt}%
\definecolor{currentstroke}{rgb}{0.000000,0.000000,0.000000}%
\pgfsetstrokecolor{currentstroke}%
\pgfsetdash{}{0pt}%
\pgfpathmoveto{\pgfqpoint{2.484775in}{2.641734in}}%
\pgfpathlineto{\pgfqpoint{2.498665in}{2.630758in}}%
\pgfpathlineto{\pgfqpoint{2.512556in}{2.619824in}}%
\pgfpathlineto{\pgfqpoint{2.526449in}{2.608932in}}%
\pgfpathlineto{\pgfqpoint{2.540343in}{2.598081in}}%
\pgfpathlineto{\pgfqpoint{2.530869in}{2.616559in}}%
\pgfpathlineto{\pgfqpoint{2.521360in}{2.635703in}}%
\pgfpathlineto{\pgfqpoint{2.511814in}{2.655528in}}%
\pgfpathlineto{\pgfqpoint{2.502231in}{2.676045in}}%
\pgfpathlineto{\pgfqpoint{2.488279in}{2.687330in}}%
\pgfpathlineto{\pgfqpoint{2.474327in}{2.698657in}}%
\pgfpathlineto{\pgfqpoint{2.460377in}{2.710026in}}%
\pgfpathlineto{\pgfqpoint{2.446428in}{2.721438in}}%
\pgfpathlineto{\pgfqpoint{2.456071in}{2.700478in}}%
\pgfpathlineto{\pgfqpoint{2.465676in}{2.680216in}}%
\pgfpathlineto{\pgfqpoint{2.475244in}{2.660639in}}%
\pgfpathlineto{\pgfqpoint{2.484775in}{2.641734in}}%
\pgfpathclose%
\pgfusepath{fill}%
\end{pgfscope}%
\begin{pgfscope}%
\pgfpathrectangle{\pgfqpoint{1.150000in}{0.150000in}}{\pgfqpoint{5.700000in}{5.700000in}}%
\pgfusepath{clip}%
\pgfsetbuttcap%
\pgfsetroundjoin%
\definecolor{currentfill}{rgb}{0.271305,0.019942,0.347269}%
\pgfsetfillcolor{currentfill}%
\pgfsetfillopacity{0.700000}%
\pgfsetlinewidth{0.000000pt}%
\definecolor{currentstroke}{rgb}{0.000000,0.000000,0.000000}%
\pgfsetstrokecolor{currentstroke}%
\pgfsetdash{}{0pt}%
\pgfpathmoveto{\pgfqpoint{4.631257in}{1.552656in}}%
\pgfpathlineto{\pgfqpoint{4.645449in}{1.548663in}}%
\pgfpathlineto{\pgfqpoint{4.659647in}{1.544694in}}%
\pgfpathlineto{\pgfqpoint{4.673853in}{1.540748in}}%
\pgfpathlineto{\pgfqpoint{4.688065in}{1.536827in}}%
\pgfpathlineto{\pgfqpoint{4.680182in}{1.529308in}}%
\pgfpathlineto{\pgfqpoint{4.672294in}{1.521981in}}%
\pgfpathlineto{\pgfqpoint{4.664402in}{1.514853in}}%
\pgfpathlineto{\pgfqpoint{4.656507in}{1.507932in}}%
\pgfpathlineto{\pgfqpoint{4.642282in}{1.512140in}}%
\pgfpathlineto{\pgfqpoint{4.628064in}{1.516371in}}%
\pgfpathlineto{\pgfqpoint{4.613852in}{1.520627in}}%
\pgfpathlineto{\pgfqpoint{4.599648in}{1.524906in}}%
\pgfpathlineto{\pgfqpoint{4.607557in}{1.531536in}}%
\pgfpathlineto{\pgfqpoint{4.615461in}{1.538376in}}%
\pgfpathlineto{\pgfqpoint{4.623361in}{1.545418in}}%
\pgfpathlineto{\pgfqpoint{4.631257in}{1.552656in}}%
\pgfpathclose%
\pgfusepath{fill}%
\end{pgfscope}%
\begin{pgfscope}%
\pgfpathrectangle{\pgfqpoint{1.150000in}{0.150000in}}{\pgfqpoint{5.700000in}{5.700000in}}%
\pgfusepath{clip}%
\pgfsetbuttcap%
\pgfsetroundjoin%
\definecolor{currentfill}{rgb}{0.277941,0.056324,0.381191}%
\pgfsetfillcolor{currentfill}%
\pgfsetfillopacity{0.700000}%
\pgfsetlinewidth{0.000000pt}%
\definecolor{currentstroke}{rgb}{0.000000,0.000000,0.000000}%
\pgfsetstrokecolor{currentstroke}%
\pgfsetdash{}{0pt}%
\pgfpathmoveto{\pgfqpoint{4.196493in}{1.604961in}}%
\pgfpathlineto{\pgfqpoint{4.210578in}{1.599489in}}%
\pgfpathlineto{\pgfqpoint{4.224669in}{1.594042in}}%
\pgfpathlineto{\pgfqpoint{4.238765in}{1.588619in}}%
\pgfpathlineto{\pgfqpoint{4.252868in}{1.583221in}}%
\pgfpathlineto{\pgfqpoint{4.244828in}{1.580438in}}%
\pgfpathlineto{\pgfqpoint{4.236780in}{1.577961in}}%
\pgfpathlineto{\pgfqpoint{4.228725in}{1.575800in}}%
\pgfpathlineto{\pgfqpoint{4.220663in}{1.573963in}}%
\pgfpathlineto{\pgfqpoint{4.206540in}{1.579687in}}%
\pgfpathlineto{\pgfqpoint{4.192422in}{1.585437in}}%
\pgfpathlineto{\pgfqpoint{4.178311in}{1.591211in}}%
\pgfpathlineto{\pgfqpoint{4.164205in}{1.597009in}}%
\pgfpathlineto{\pgfqpoint{4.172288in}{1.598514in}}%
\pgfpathlineto{\pgfqpoint{4.180364in}{1.600347in}}%
\pgfpathlineto{\pgfqpoint{4.188432in}{1.602498in}}%
\pgfpathlineto{\pgfqpoint{4.196493in}{1.604961in}}%
\pgfpathclose%
\pgfusepath{fill}%
\end{pgfscope}%
\begin{pgfscope}%
\pgfpathrectangle{\pgfqpoint{1.150000in}{0.150000in}}{\pgfqpoint{5.700000in}{5.700000in}}%
\pgfusepath{clip}%
\pgfsetbuttcap%
\pgfsetroundjoin%
\definecolor{currentfill}{rgb}{0.278791,0.062145,0.386592}%
\pgfsetfillcolor{currentfill}%
\pgfsetfillopacity{0.700000}%
\pgfsetlinewidth{0.000000pt}%
\definecolor{currentstroke}{rgb}{0.000000,0.000000,0.000000}%
\pgfsetstrokecolor{currentstroke}%
\pgfsetdash{}{0pt}%
\pgfpathmoveto{\pgfqpoint{5.098714in}{1.624515in}}%
\pgfpathlineto{\pgfqpoint{5.113050in}{1.622108in}}%
\pgfpathlineto{\pgfqpoint{5.127395in}{1.619725in}}%
\pgfpathlineto{\pgfqpoint{5.141748in}{1.617366in}}%
\pgfpathlineto{\pgfqpoint{5.156109in}{1.615030in}}%
\pgfpathlineto{\pgfqpoint{5.148335in}{1.603978in}}%
\pgfpathlineto{\pgfqpoint{5.140558in}{1.592994in}}%
\pgfpathlineto{\pgfqpoint{5.132778in}{1.582085in}}%
\pgfpathlineto{\pgfqpoint{5.124994in}{1.571256in}}%
\pgfpathlineto{\pgfqpoint{5.110626in}{1.573826in}}%
\pgfpathlineto{\pgfqpoint{5.096267in}{1.576419in}}%
\pgfpathlineto{\pgfqpoint{5.081916in}{1.579036in}}%
\pgfpathlineto{\pgfqpoint{5.067573in}{1.581677in}}%
\pgfpathlineto{\pgfqpoint{5.075364in}{1.592267in}}%
\pgfpathlineto{\pgfqpoint{5.083151in}{1.602940in}}%
\pgfpathlineto{\pgfqpoint{5.090934in}{1.613691in}}%
\pgfpathlineto{\pgfqpoint{5.098714in}{1.624515in}}%
\pgfpathclose%
\pgfusepath{fill}%
\end{pgfscope}%
\begin{pgfscope}%
\pgfpathrectangle{\pgfqpoint{1.150000in}{0.150000in}}{\pgfqpoint{5.700000in}{5.700000in}}%
\pgfusepath{clip}%
\pgfsetbuttcap%
\pgfsetroundjoin%
\definecolor{currentfill}{rgb}{0.210503,0.363727,0.552206}%
\pgfsetfillcolor{currentfill}%
\pgfsetfillopacity{0.700000}%
\pgfsetlinewidth{0.000000pt}%
\definecolor{currentstroke}{rgb}{0.000000,0.000000,0.000000}%
\pgfsetstrokecolor{currentstroke}%
\pgfsetdash{}{0pt}%
\pgfpathmoveto{\pgfqpoint{2.966297in}{2.253846in}}%
\pgfpathlineto{\pgfqpoint{2.980205in}{2.244475in}}%
\pgfpathlineto{\pgfqpoint{2.994116in}{2.235137in}}%
\pgfpathlineto{\pgfqpoint{3.008030in}{2.225831in}}%
\pgfpathlineto{\pgfqpoint{3.021947in}{2.216558in}}%
\pgfpathlineto{\pgfqpoint{3.013003in}{2.229251in}}%
\pgfpathlineto{\pgfqpoint{3.004034in}{2.242525in}}%
\pgfpathlineto{\pgfqpoint{2.995039in}{2.256392in}}%
\pgfpathlineto{\pgfqpoint{2.986018in}{2.270865in}}%
\pgfpathlineto{\pgfqpoint{2.972053in}{2.280547in}}%
\pgfpathlineto{\pgfqpoint{2.958091in}{2.290261in}}%
\pgfpathlineto{\pgfqpoint{2.944132in}{2.300009in}}%
\pgfpathlineto{\pgfqpoint{2.930176in}{2.309790in}}%
\pgfpathlineto{\pgfqpoint{2.939246in}{2.294902in}}%
\pgfpathlineto{\pgfqpoint{2.948290in}{2.280623in}}%
\pgfpathlineto{\pgfqpoint{2.957306in}{2.266942in}}%
\pgfpathlineto{\pgfqpoint{2.966297in}{2.253846in}}%
\pgfpathclose%
\pgfusepath{fill}%
\end{pgfscope}%
\begin{pgfscope}%
\pgfpathrectangle{\pgfqpoint{1.150000in}{0.150000in}}{\pgfqpoint{5.700000in}{5.700000in}}%
\pgfusepath{clip}%
\pgfsetbuttcap%
\pgfsetroundjoin%
\definecolor{currentfill}{rgb}{0.277134,0.185228,0.489898}%
\pgfsetfillcolor{currentfill}%
\pgfsetfillopacity{0.700000}%
\pgfsetlinewidth{0.000000pt}%
\definecolor{currentstroke}{rgb}{0.000000,0.000000,0.000000}%
\pgfsetstrokecolor{currentstroke}%
\pgfsetdash{}{0pt}%
\pgfpathmoveto{\pgfqpoint{3.593072in}{1.848499in}}%
\pgfpathlineto{\pgfqpoint{3.607049in}{1.841080in}}%
\pgfpathlineto{\pgfqpoint{3.621030in}{1.833688in}}%
\pgfpathlineto{\pgfqpoint{3.635016in}{1.826322in}}%
\pgfpathlineto{\pgfqpoint{3.649006in}{1.818984in}}%
\pgfpathlineto{\pgfqpoint{3.640610in}{1.823764in}}%
\pgfpathlineto{\pgfqpoint{3.632199in}{1.828996in}}%
\pgfpathlineto{\pgfqpoint{3.623773in}{1.834690in}}%
\pgfpathlineto{\pgfqpoint{3.615332in}{1.840858in}}%
\pgfpathlineto{\pgfqpoint{3.601308in}{1.848568in}}%
\pgfpathlineto{\pgfqpoint{3.587289in}{1.856305in}}%
\pgfpathlineto{\pgfqpoint{3.573273in}{1.864068in}}%
\pgfpathlineto{\pgfqpoint{3.559262in}{1.871859in}}%
\pgfpathlineto{\pgfqpoint{3.567738in}{1.865314in}}%
\pgfpathlineto{\pgfqpoint{3.576198in}{1.859246in}}%
\pgfpathlineto{\pgfqpoint{3.584642in}{1.853644in}}%
\pgfpathlineto{\pgfqpoint{3.593072in}{1.848499in}}%
\pgfpathclose%
\pgfusepath{fill}%
\end{pgfscope}%
\begin{pgfscope}%
\pgfpathrectangle{\pgfqpoint{1.150000in}{0.150000in}}{\pgfqpoint{5.700000in}{5.700000in}}%
\pgfusepath{clip}%
\pgfsetbuttcap%
\pgfsetroundjoin%
\definecolor{currentfill}{rgb}{0.154815,0.493313,0.557840}%
\pgfsetfillcolor{currentfill}%
\pgfsetfillopacity{0.700000}%
\pgfsetlinewidth{0.000000pt}%
\definecolor{currentstroke}{rgb}{0.000000,0.000000,0.000000}%
\pgfsetstrokecolor{currentstroke}%
\pgfsetdash{}{0pt}%
\pgfpathmoveto{\pgfqpoint{2.540343in}{2.598081in}}%
\pgfpathlineto{\pgfqpoint{2.554239in}{2.587272in}}%
\pgfpathlineto{\pgfqpoint{2.568137in}{2.576503in}}%
\pgfpathlineto{\pgfqpoint{2.582036in}{2.565774in}}%
\pgfpathlineto{\pgfqpoint{2.595937in}{2.555085in}}%
\pgfpathlineto{\pgfqpoint{2.586519in}{2.573136in}}%
\pgfpathlineto{\pgfqpoint{2.577067in}{2.591849in}}%
\pgfpathlineto{\pgfqpoint{2.567579in}{2.611237in}}%
\pgfpathlineto{\pgfqpoint{2.558056in}{2.631313in}}%
\pgfpathlineto{\pgfqpoint{2.544098in}{2.642435in}}%
\pgfpathlineto{\pgfqpoint{2.530141in}{2.653598in}}%
\pgfpathlineto{\pgfqpoint{2.516185in}{2.664801in}}%
\pgfpathlineto{\pgfqpoint{2.502231in}{2.676045in}}%
\pgfpathlineto{\pgfqpoint{2.511814in}{2.655528in}}%
\pgfpathlineto{\pgfqpoint{2.521360in}{2.635703in}}%
\pgfpathlineto{\pgfqpoint{2.530869in}{2.616559in}}%
\pgfpathlineto{\pgfqpoint{2.540343in}{2.598081in}}%
\pgfpathclose%
\pgfusepath{fill}%
\end{pgfscope}%
\begin{pgfscope}%
\pgfpathrectangle{\pgfqpoint{1.150000in}{0.150000in}}{\pgfqpoint{5.700000in}{5.700000in}}%
\pgfusepath{clip}%
\pgfsetbuttcap%
\pgfsetroundjoin%
\definecolor{currentfill}{rgb}{0.272594,0.025563,0.353093}%
\pgfsetfillcolor{currentfill}%
\pgfsetfillopacity{0.700000}%
\pgfsetlinewidth{0.000000pt}%
\definecolor{currentstroke}{rgb}{0.000000,0.000000,0.000000}%
\pgfsetstrokecolor{currentstroke}%
\pgfsetdash{}{0pt}%
\pgfpathmoveto{\pgfqpoint{4.776436in}{1.554324in}}%
\pgfpathlineto{\pgfqpoint{4.790674in}{1.550794in}}%
\pgfpathlineto{\pgfqpoint{4.804918in}{1.547289in}}%
\pgfpathlineto{\pgfqpoint{4.819170in}{1.543808in}}%
\pgfpathlineto{\pgfqpoint{4.833430in}{1.540350in}}%
\pgfpathlineto{\pgfqpoint{4.825583in}{1.531598in}}%
\pgfpathlineto{\pgfqpoint{4.817733in}{1.523003in}}%
\pgfpathlineto{\pgfqpoint{4.809879in}{1.514573in}}%
\pgfpathlineto{\pgfqpoint{4.802022in}{1.506313in}}%
\pgfpathlineto{\pgfqpoint{4.787752in}{1.510044in}}%
\pgfpathlineto{\pgfqpoint{4.773490in}{1.513799in}}%
\pgfpathlineto{\pgfqpoint{4.759234in}{1.517577in}}%
\pgfpathlineto{\pgfqpoint{4.744986in}{1.521380in}}%
\pgfpathlineto{\pgfqpoint{4.752854in}{1.529361in}}%
\pgfpathlineto{\pgfqpoint{4.760719in}{1.537517in}}%
\pgfpathlineto{\pgfqpoint{4.768579in}{1.545840in}}%
\pgfpathlineto{\pgfqpoint{4.776436in}{1.554324in}}%
\pgfpathclose%
\pgfusepath{fill}%
\end{pgfscope}%
\begin{pgfscope}%
\pgfpathrectangle{\pgfqpoint{1.150000in}{0.150000in}}{\pgfqpoint{5.700000in}{5.700000in}}%
\pgfusepath{clip}%
\pgfsetbuttcap%
\pgfsetroundjoin%
\definecolor{currentfill}{rgb}{0.276022,0.044167,0.370164}%
\pgfsetfillcolor{currentfill}%
\pgfsetfillopacity{0.700000}%
\pgfsetlinewidth{0.000000pt}%
\definecolor{currentstroke}{rgb}{0.000000,0.000000,0.000000}%
\pgfsetstrokecolor{currentstroke}%
\pgfsetdash{}{0pt}%
\pgfpathmoveto{\pgfqpoint{5.010281in}{1.592480in}}%
\pgfpathlineto{\pgfqpoint{5.024592in}{1.589743in}}%
\pgfpathlineto{\pgfqpoint{5.038911in}{1.587031in}}%
\pgfpathlineto{\pgfqpoint{5.053238in}{1.584342in}}%
\pgfpathlineto{\pgfqpoint{5.067573in}{1.581677in}}%
\pgfpathlineto{\pgfqpoint{5.059779in}{1.571178in}}%
\pgfpathlineto{\pgfqpoint{5.051982in}{1.560774in}}%
\pgfpathlineto{\pgfqpoint{5.044181in}{1.550472in}}%
\pgfpathlineto{\pgfqpoint{5.036377in}{1.540278in}}%
\pgfpathlineto{\pgfqpoint{5.022035in}{1.543190in}}%
\pgfpathlineto{\pgfqpoint{5.007700in}{1.546126in}}%
\pgfpathlineto{\pgfqpoint{4.993373in}{1.549086in}}%
\pgfpathlineto{\pgfqpoint{4.979054in}{1.552069in}}%
\pgfpathlineto{\pgfqpoint{4.986866in}{1.562011in}}%
\pgfpathlineto{\pgfqpoint{4.994674in}{1.572064in}}%
\pgfpathlineto{\pgfqpoint{5.002479in}{1.582222in}}%
\pgfpathlineto{\pgfqpoint{5.010281in}{1.592480in}}%
\pgfpathclose%
\pgfusepath{fill}%
\end{pgfscope}%
\begin{pgfscope}%
\pgfpathrectangle{\pgfqpoint{1.150000in}{0.150000in}}{\pgfqpoint{5.700000in}{5.700000in}}%
\pgfusepath{clip}%
\pgfsetbuttcap%
\pgfsetroundjoin%
\definecolor{currentfill}{rgb}{0.257322,0.256130,0.526563}%
\pgfsetfillcolor{currentfill}%
\pgfsetfillopacity{0.700000}%
\pgfsetlinewidth{0.000000pt}%
\definecolor{currentstroke}{rgb}{0.000000,0.000000,0.000000}%
\pgfsetstrokecolor{currentstroke}%
\pgfsetdash{}{0pt}%
\pgfpathmoveto{\pgfqpoint{3.335644in}{2.000287in}}%
\pgfpathlineto{\pgfqpoint{3.349591in}{1.992048in}}%
\pgfpathlineto{\pgfqpoint{3.363541in}{1.983837in}}%
\pgfpathlineto{\pgfqpoint{3.377495in}{1.975656in}}%
\pgfpathlineto{\pgfqpoint{3.391453in}{1.967503in}}%
\pgfpathlineto{\pgfqpoint{3.382849in}{1.975682in}}%
\pgfpathlineto{\pgfqpoint{3.374227in}{1.984372in}}%
\pgfpathlineto{\pgfqpoint{3.365586in}{1.993584in}}%
\pgfpathlineto{\pgfqpoint{3.356925in}{2.003329in}}%
\pgfpathlineto{\pgfqpoint{3.342928in}{2.011870in}}%
\pgfpathlineto{\pgfqpoint{3.328934in}{2.020441in}}%
\pgfpathlineto{\pgfqpoint{3.314944in}{2.029041in}}%
\pgfpathlineto{\pgfqpoint{3.300958in}{2.037669in}}%
\pgfpathlineto{\pgfqpoint{3.309659in}{2.027529in}}%
\pgfpathlineto{\pgfqpoint{3.318341in}{2.017927in}}%
\pgfpathlineto{\pgfqpoint{3.327002in}{2.008850in}}%
\pgfpathlineto{\pgfqpoint{3.335644in}{2.000287in}}%
\pgfpathclose%
\pgfusepath{fill}%
\end{pgfscope}%
\begin{pgfscope}%
\pgfpathrectangle{\pgfqpoint{1.150000in}{0.150000in}}{\pgfqpoint{5.700000in}{5.700000in}}%
\pgfusepath{clip}%
\pgfsetbuttcap%
\pgfsetroundjoin%
\definecolor{currentfill}{rgb}{0.283187,0.125848,0.444960}%
\pgfsetfillcolor{currentfill}%
\pgfsetfillopacity{0.700000}%
\pgfsetlinewidth{0.000000pt}%
\definecolor{currentstroke}{rgb}{0.000000,0.000000,0.000000}%
\pgfsetstrokecolor{currentstroke}%
\pgfsetdash{}{0pt}%
\pgfpathmoveto{\pgfqpoint{3.850342in}{1.722463in}}%
\pgfpathlineto{\pgfqpoint{3.864365in}{1.715821in}}%
\pgfpathlineto{\pgfqpoint{3.878393in}{1.709206in}}%
\pgfpathlineto{\pgfqpoint{3.892426in}{1.702616in}}%
\pgfpathlineto{\pgfqpoint{3.906464in}{1.696052in}}%
\pgfpathlineto{\pgfqpoint{3.898237in}{1.697721in}}%
\pgfpathlineto{\pgfqpoint{3.889999in}{1.699788in}}%
\pgfpathlineto{\pgfqpoint{3.881750in}{1.702262in}}%
\pgfpathlineto{\pgfqpoint{3.873489in}{1.705152in}}%
\pgfpathlineto{\pgfqpoint{3.859423in}{1.712073in}}%
\pgfpathlineto{\pgfqpoint{3.845361in}{1.719018in}}%
\pgfpathlineto{\pgfqpoint{3.831305in}{1.725990in}}%
\pgfpathlineto{\pgfqpoint{3.817253in}{1.732987in}}%
\pgfpathlineto{\pgfqpoint{3.825543in}{1.729734in}}%
\pgfpathlineto{\pgfqpoint{3.833821in}{1.726903in}}%
\pgfpathlineto{\pgfqpoint{3.842088in}{1.724482in}}%
\pgfpathlineto{\pgfqpoint{3.850342in}{1.722463in}}%
\pgfpathclose%
\pgfusepath{fill}%
\end{pgfscope}%
\begin{pgfscope}%
\pgfpathrectangle{\pgfqpoint{1.150000in}{0.150000in}}{\pgfqpoint{5.700000in}{5.700000in}}%
\pgfusepath{clip}%
\pgfsetbuttcap%
\pgfsetroundjoin%
\definecolor{currentfill}{rgb}{0.281446,0.084320,0.407414}%
\pgfsetfillcolor{currentfill}%
\pgfsetfillopacity{0.700000}%
\pgfsetlinewidth{0.000000pt}%
\definecolor{currentstroke}{rgb}{0.000000,0.000000,0.000000}%
\pgfsetstrokecolor{currentstroke}%
\pgfsetdash{}{0pt}%
\pgfpathmoveto{\pgfqpoint{4.051560in}{1.644285in}}%
\pgfpathlineto{\pgfqpoint{4.065621in}{1.638288in}}%
\pgfpathlineto{\pgfqpoint{4.079688in}{1.632317in}}%
\pgfpathlineto{\pgfqpoint{4.093760in}{1.626370in}}%
\pgfpathlineto{\pgfqpoint{4.107837in}{1.620448in}}%
\pgfpathlineto{\pgfqpoint{4.099723in}{1.619616in}}%
\pgfpathlineto{\pgfqpoint{4.091601in}{1.619132in}}%
\pgfpathlineto{\pgfqpoint{4.083470in}{1.619006in}}%
\pgfpathlineto{\pgfqpoint{4.075330in}{1.619249in}}%
\pgfpathlineto{\pgfqpoint{4.061229in}{1.625511in}}%
\pgfpathlineto{\pgfqpoint{4.047133in}{1.631799in}}%
\pgfpathlineto{\pgfqpoint{4.033042in}{1.638111in}}%
\pgfpathlineto{\pgfqpoint{4.018957in}{1.644448in}}%
\pgfpathlineto{\pgfqpoint{4.027121in}{1.643860in}}%
\pgfpathlineto{\pgfqpoint{4.035277in}{1.643643in}}%
\pgfpathlineto{\pgfqpoint{4.043423in}{1.643787in}}%
\pgfpathlineto{\pgfqpoint{4.051560in}{1.644285in}}%
\pgfpathclose%
\pgfusepath{fill}%
\end{pgfscope}%
\begin{pgfscope}%
\pgfpathrectangle{\pgfqpoint{1.150000in}{0.150000in}}{\pgfqpoint{5.700000in}{5.700000in}}%
\pgfusepath{clip}%
\pgfsetbuttcap%
\pgfsetroundjoin%
\definecolor{currentfill}{rgb}{0.216210,0.351535,0.550627}%
\pgfsetfillcolor{currentfill}%
\pgfsetfillopacity{0.700000}%
\pgfsetlinewidth{0.000000pt}%
\definecolor{currentstroke}{rgb}{0.000000,0.000000,0.000000}%
\pgfsetstrokecolor{currentstroke}%
\pgfsetdash{}{0pt}%
\pgfpathmoveto{\pgfqpoint{3.021947in}{2.216558in}}%
\pgfpathlineto{\pgfqpoint{3.035867in}{2.207318in}}%
\pgfpathlineto{\pgfqpoint{3.049790in}{2.198110in}}%
\pgfpathlineto{\pgfqpoint{3.063716in}{2.188933in}}%
\pgfpathlineto{\pgfqpoint{3.077645in}{2.179789in}}%
\pgfpathlineto{\pgfqpoint{3.068747in}{2.192080in}}%
\pgfpathlineto{\pgfqpoint{3.059825in}{2.204947in}}%
\pgfpathlineto{\pgfqpoint{3.050877in}{2.218404in}}%
\pgfpathlineto{\pgfqpoint{3.041904in}{2.232461in}}%
\pgfpathlineto{\pgfqpoint{3.027928in}{2.242014in}}%
\pgfpathlineto{\pgfqpoint{3.013955in}{2.251598in}}%
\pgfpathlineto{\pgfqpoint{2.999985in}{2.261215in}}%
\pgfpathlineto{\pgfqpoint{2.986018in}{2.270865in}}%
\pgfpathlineto{\pgfqpoint{2.995039in}{2.256392in}}%
\pgfpathlineto{\pgfqpoint{3.004034in}{2.242525in}}%
\pgfpathlineto{\pgfqpoint{3.013003in}{2.229251in}}%
\pgfpathlineto{\pgfqpoint{3.021947in}{2.216558in}}%
\pgfpathclose%
\pgfusepath{fill}%
\end{pgfscope}%
\begin{pgfscope}%
\pgfpathrectangle{\pgfqpoint{1.150000in}{0.150000in}}{\pgfqpoint{5.700000in}{5.700000in}}%
\pgfusepath{clip}%
\pgfsetbuttcap%
\pgfsetroundjoin%
\definecolor{currentfill}{rgb}{0.282656,0.100196,0.422160}%
\pgfsetfillcolor{currentfill}%
\pgfsetfillopacity{0.700000}%
\pgfsetlinewidth{0.000000pt}%
\definecolor{currentstroke}{rgb}{0.000000,0.000000,0.000000}%
\pgfsetstrokecolor{currentstroke}%
\pgfsetdash{}{0pt}%
\pgfpathmoveto{\pgfqpoint{5.333256in}{1.690894in}}%
\pgfpathlineto{\pgfqpoint{5.347682in}{1.689203in}}%
\pgfpathlineto{\pgfqpoint{5.362117in}{1.687537in}}%
\pgfpathlineto{\pgfqpoint{5.376561in}{1.685895in}}%
\pgfpathlineto{\pgfqpoint{5.368833in}{1.673926in}}%
\pgfpathlineto{\pgfqpoint{5.361101in}{1.661975in}}%
\pgfpathlineto{\pgfqpoint{5.353365in}{1.650046in}}%
\pgfpathlineto{\pgfqpoint{5.345625in}{1.638145in}}%
\pgfpathlineto{\pgfqpoint{5.331177in}{1.639996in}}%
\pgfpathlineto{\pgfqpoint{5.316738in}{1.641870in}}%
\pgfpathlineto{\pgfqpoint{5.302307in}{1.643769in}}%
\pgfpathlineto{\pgfqpoint{5.310050in}{1.655510in}}%
\pgfpathlineto{\pgfqpoint{5.317790in}{1.667281in}}%
\pgfpathlineto{\pgfqpoint{5.325525in}{1.679078in}}%
\pgfpathlineto{\pgfqpoint{5.333256in}{1.690894in}}%
\pgfpathclose%
\pgfusepath{fill}%
\end{pgfscope}%
\begin{pgfscope}%
\pgfpathrectangle{\pgfqpoint{1.150000in}{0.150000in}}{\pgfqpoint{5.700000in}{5.700000in}}%
\pgfusepath{clip}%
\pgfsetbuttcap%
\pgfsetroundjoin%
\definecolor{currentfill}{rgb}{0.160665,0.478540,0.558115}%
\pgfsetfillcolor{currentfill}%
\pgfsetfillopacity{0.700000}%
\pgfsetlinewidth{0.000000pt}%
\definecolor{currentstroke}{rgb}{0.000000,0.000000,0.000000}%
\pgfsetstrokecolor{currentstroke}%
\pgfsetdash{}{0pt}%
\pgfpathmoveto{\pgfqpoint{2.595937in}{2.555085in}}%
\pgfpathlineto{\pgfqpoint{2.609839in}{2.544436in}}%
\pgfpathlineto{\pgfqpoint{2.623744in}{2.533827in}}%
\pgfpathlineto{\pgfqpoint{2.637650in}{2.523256in}}%
\pgfpathlineto{\pgfqpoint{2.651558in}{2.512724in}}%
\pgfpathlineto{\pgfqpoint{2.642196in}{2.530349in}}%
\pgfpathlineto{\pgfqpoint{2.632800in}{2.548631in}}%
\pgfpathlineto{\pgfqpoint{2.623370in}{2.567584in}}%
\pgfpathlineto{\pgfqpoint{2.613905in}{2.587220in}}%
\pgfpathlineto{\pgfqpoint{2.599940in}{2.598185in}}%
\pgfpathlineto{\pgfqpoint{2.585977in}{2.609188in}}%
\pgfpathlineto{\pgfqpoint{2.572016in}{2.620231in}}%
\pgfpathlineto{\pgfqpoint{2.558056in}{2.631313in}}%
\pgfpathlineto{\pgfqpoint{2.567579in}{2.611237in}}%
\pgfpathlineto{\pgfqpoint{2.577067in}{2.591849in}}%
\pgfpathlineto{\pgfqpoint{2.586519in}{2.573136in}}%
\pgfpathlineto{\pgfqpoint{2.595937in}{2.555085in}}%
\pgfpathclose%
\pgfusepath{fill}%
\end{pgfscope}%
\begin{pgfscope}%
\pgfpathrectangle{\pgfqpoint{1.150000in}{0.150000in}}{\pgfqpoint{5.700000in}{5.700000in}}%
\pgfusepath{clip}%
\pgfsetbuttcap%
\pgfsetroundjoin%
\definecolor{currentfill}{rgb}{0.272594,0.025563,0.353093}%
\pgfsetfillcolor{currentfill}%
\pgfsetfillopacity{0.700000}%
\pgfsetlinewidth{0.000000pt}%
\definecolor{currentstroke}{rgb}{0.000000,0.000000,0.000000}%
\pgfsetstrokecolor{currentstroke}%
\pgfsetdash{}{0pt}%
\pgfpathmoveto{\pgfqpoint{4.542898in}{1.542262in}}%
\pgfpathlineto{\pgfqpoint{4.557076in}{1.537887in}}%
\pgfpathlineto{\pgfqpoint{4.571260in}{1.533536in}}%
\pgfpathlineto{\pgfqpoint{4.585450in}{1.529209in}}%
\pgfpathlineto{\pgfqpoint{4.599648in}{1.524906in}}%
\pgfpathlineto{\pgfqpoint{4.591735in}{1.518493in}}%
\pgfpathlineto{\pgfqpoint{4.583817in}{1.512306in}}%
\pgfpathlineto{\pgfqpoint{4.575895in}{1.506351in}}%
\pgfpathlineto{\pgfqpoint{4.567969in}{1.500636in}}%
\pgfpathlineto{\pgfqpoint{4.553757in}{1.505239in}}%
\pgfpathlineto{\pgfqpoint{4.539552in}{1.509865in}}%
\pgfpathlineto{\pgfqpoint{4.525353in}{1.514516in}}%
\pgfpathlineto{\pgfqpoint{4.511161in}{1.519190in}}%
\pgfpathlineto{\pgfqpoint{4.519103in}{1.524600in}}%
\pgfpathlineto{\pgfqpoint{4.527039in}{1.530254in}}%
\pgfpathlineto{\pgfqpoint{4.534971in}{1.536144in}}%
\pgfpathlineto{\pgfqpoint{4.542898in}{1.542262in}}%
\pgfpathclose%
\pgfusepath{fill}%
\end{pgfscope}%
\begin{pgfscope}%
\pgfpathrectangle{\pgfqpoint{1.150000in}{0.150000in}}{\pgfqpoint{5.700000in}{5.700000in}}%
\pgfusepath{clip}%
\pgfsetbuttcap%
\pgfsetroundjoin%
\definecolor{currentfill}{rgb}{0.273809,0.031497,0.358853}%
\pgfsetfillcolor{currentfill}%
\pgfsetfillopacity{0.700000}%
\pgfsetlinewidth{0.000000pt}%
\definecolor{currentstroke}{rgb}{0.000000,0.000000,0.000000}%
\pgfsetstrokecolor{currentstroke}%
\pgfsetdash{}{0pt}%
\pgfpathmoveto{\pgfqpoint{4.397859in}{1.557450in}}%
\pgfpathlineto{\pgfqpoint{4.411999in}{1.552583in}}%
\pgfpathlineto{\pgfqpoint{4.426146in}{1.547741in}}%
\pgfpathlineto{\pgfqpoint{4.440299in}{1.542922in}}%
\pgfpathlineto{\pgfqpoint{4.454458in}{1.538128in}}%
\pgfpathlineto{\pgfqpoint{4.446496in}{1.533278in}}%
\pgfpathlineto{\pgfqpoint{4.438528in}{1.528690in}}%
\pgfpathlineto{\pgfqpoint{4.430555in}{1.524375in}}%
\pgfpathlineto{\pgfqpoint{4.422577in}{1.520338in}}%
\pgfpathlineto{\pgfqpoint{4.408400in}{1.525446in}}%
\pgfpathlineto{\pgfqpoint{4.394230in}{1.530577in}}%
\pgfpathlineto{\pgfqpoint{4.380066in}{1.535733in}}%
\pgfpathlineto{\pgfqpoint{4.365909in}{1.540913in}}%
\pgfpathlineto{\pgfqpoint{4.373905in}{1.544631in}}%
\pgfpathlineto{\pgfqpoint{4.381895in}{1.548632in}}%
\pgfpathlineto{\pgfqpoint{4.389880in}{1.552907in}}%
\pgfpathlineto{\pgfqpoint{4.397859in}{1.557450in}}%
\pgfpathclose%
\pgfusepath{fill}%
\end{pgfscope}%
\begin{pgfscope}%
\pgfpathrectangle{\pgfqpoint{1.150000in}{0.150000in}}{\pgfqpoint{5.700000in}{5.700000in}}%
\pgfusepath{clip}%
\pgfsetbuttcap%
\pgfsetroundjoin%
\definecolor{currentfill}{rgb}{0.274952,0.037752,0.364543}%
\pgfsetfillcolor{currentfill}%
\pgfsetfillopacity{0.700000}%
\pgfsetlinewidth{0.000000pt}%
\definecolor{currentstroke}{rgb}{0.000000,0.000000,0.000000}%
\pgfsetstrokecolor{currentstroke}%
\pgfsetdash{}{0pt}%
\pgfpathmoveto{\pgfqpoint{4.921856in}{1.564241in}}%
\pgfpathlineto{\pgfqpoint{4.936144in}{1.561163in}}%
\pgfpathlineto{\pgfqpoint{4.950440in}{1.558108in}}%
\pgfpathlineto{\pgfqpoint{4.964743in}{1.555077in}}%
\pgfpathlineto{\pgfqpoint{4.979054in}{1.552069in}}%
\pgfpathlineto{\pgfqpoint{4.971239in}{1.542245in}}%
\pgfpathlineto{\pgfqpoint{4.963421in}{1.532545in}}%
\pgfpathlineto{\pgfqpoint{4.955600in}{1.522975in}}%
\pgfpathlineto{\pgfqpoint{4.947775in}{1.513543in}}%
\pgfpathlineto{\pgfqpoint{4.933455in}{1.516811in}}%
\pgfpathlineto{\pgfqpoint{4.919143in}{1.520102in}}%
\pgfpathlineto{\pgfqpoint{4.904839in}{1.523417in}}%
\pgfpathlineto{\pgfqpoint{4.890542in}{1.526756in}}%
\pgfpathlineto{\pgfqpoint{4.898376in}{1.535924in}}%
\pgfpathlineto{\pgfqpoint{4.906206in}{1.545231in}}%
\pgfpathlineto{\pgfqpoint{4.914033in}{1.554673in}}%
\pgfpathlineto{\pgfqpoint{4.921856in}{1.564241in}}%
\pgfpathclose%
\pgfusepath{fill}%
\end{pgfscope}%
\begin{pgfscope}%
\pgfpathrectangle{\pgfqpoint{1.150000in}{0.150000in}}{\pgfqpoint{5.700000in}{5.700000in}}%
\pgfusepath{clip}%
\pgfsetbuttcap%
\pgfsetroundjoin%
\definecolor{currentfill}{rgb}{0.271305,0.019942,0.347269}%
\pgfsetfillcolor{currentfill}%
\pgfsetfillopacity{0.700000}%
\pgfsetlinewidth{0.000000pt}%
\definecolor{currentstroke}{rgb}{0.000000,0.000000,0.000000}%
\pgfsetstrokecolor{currentstroke}%
\pgfsetdash{}{0pt}%
\pgfpathmoveto{\pgfqpoint{4.688065in}{1.536827in}}%
\pgfpathlineto{\pgfqpoint{4.702285in}{1.532929in}}%
\pgfpathlineto{\pgfqpoint{4.716511in}{1.529056in}}%
\pgfpathlineto{\pgfqpoint{4.730745in}{1.525206in}}%
\pgfpathlineto{\pgfqpoint{4.744986in}{1.521380in}}%
\pgfpathlineto{\pgfqpoint{4.737114in}{1.513579in}}%
\pgfpathlineto{\pgfqpoint{4.729239in}{1.505968in}}%
\pgfpathlineto{\pgfqpoint{4.721359in}{1.498552in}}%
\pgfpathlineto{\pgfqpoint{4.713476in}{1.491339in}}%
\pgfpathlineto{\pgfqpoint{4.699223in}{1.495452in}}%
\pgfpathlineto{\pgfqpoint{4.684977in}{1.499588in}}%
\pgfpathlineto{\pgfqpoint{4.670739in}{1.503748in}}%
\pgfpathlineto{\pgfqpoint{4.656507in}{1.507932in}}%
\pgfpathlineto{\pgfqpoint{4.664402in}{1.514853in}}%
\pgfpathlineto{\pgfqpoint{4.672294in}{1.521981in}}%
\pgfpathlineto{\pgfqpoint{4.680182in}{1.529308in}}%
\pgfpathlineto{\pgfqpoint{4.688065in}{1.536827in}}%
\pgfpathclose%
\pgfusepath{fill}%
\end{pgfscope}%
\begin{pgfscope}%
\pgfpathrectangle{\pgfqpoint{1.150000in}{0.150000in}}{\pgfqpoint{5.700000in}{5.700000in}}%
\pgfusepath{clip}%
\pgfsetbuttcap%
\pgfsetroundjoin%
\definecolor{currentfill}{rgb}{0.278826,0.175490,0.483397}%
\pgfsetfillcolor{currentfill}%
\pgfsetfillopacity{0.700000}%
\pgfsetlinewidth{0.000000pt}%
\definecolor{currentstroke}{rgb}{0.000000,0.000000,0.000000}%
\pgfsetstrokecolor{currentstroke}%
\pgfsetdash{}{0pt}%
\pgfpathmoveto{\pgfqpoint{3.649006in}{1.818984in}}%
\pgfpathlineto{\pgfqpoint{3.663001in}{1.811673in}}%
\pgfpathlineto{\pgfqpoint{3.677001in}{1.804388in}}%
\pgfpathlineto{\pgfqpoint{3.691005in}{1.797129in}}%
\pgfpathlineto{\pgfqpoint{3.705014in}{1.789897in}}%
\pgfpathlineto{\pgfqpoint{3.696649in}{1.794311in}}%
\pgfpathlineto{\pgfqpoint{3.688271in}{1.799174in}}%
\pgfpathlineto{\pgfqpoint{3.679879in}{1.804495in}}%
\pgfpathlineto{\pgfqpoint{3.671472in}{1.810285in}}%
\pgfpathlineto{\pgfqpoint{3.657431in}{1.817888in}}%
\pgfpathlineto{\pgfqpoint{3.643393in}{1.825518in}}%
\pgfpathlineto{\pgfqpoint{3.629361in}{1.833175in}}%
\pgfpathlineto{\pgfqpoint{3.615332in}{1.840858in}}%
\pgfpathlineto{\pgfqpoint{3.623773in}{1.834690in}}%
\pgfpathlineto{\pgfqpoint{3.632199in}{1.828996in}}%
\pgfpathlineto{\pgfqpoint{3.640610in}{1.823764in}}%
\pgfpathlineto{\pgfqpoint{3.649006in}{1.818984in}}%
\pgfpathclose%
\pgfusepath{fill}%
\end{pgfscope}%
\begin{pgfscope}%
\pgfpathrectangle{\pgfqpoint{1.150000in}{0.150000in}}{\pgfqpoint{5.700000in}{5.700000in}}%
\pgfusepath{clip}%
\pgfsetbuttcap%
\pgfsetroundjoin%
\definecolor{currentfill}{rgb}{0.277018,0.050344,0.375715}%
\pgfsetfillcolor{currentfill}%
\pgfsetfillopacity{0.700000}%
\pgfsetlinewidth{0.000000pt}%
\definecolor{currentstroke}{rgb}{0.000000,0.000000,0.000000}%
\pgfsetstrokecolor{currentstroke}%
\pgfsetdash{}{0pt}%
\pgfpathmoveto{\pgfqpoint{4.252868in}{1.583221in}}%
\pgfpathlineto{\pgfqpoint{4.266977in}{1.577848in}}%
\pgfpathlineto{\pgfqpoint{4.281092in}{1.572498in}}%
\pgfpathlineto{\pgfqpoint{4.295213in}{1.567173in}}%
\pgfpathlineto{\pgfqpoint{4.309340in}{1.561873in}}%
\pgfpathlineto{\pgfqpoint{4.301318in}{1.558768in}}%
\pgfpathlineto{\pgfqpoint{4.293291in}{1.555966in}}%
\pgfpathlineto{\pgfqpoint{4.285257in}{1.553476in}}%
\pgfpathlineto{\pgfqpoint{4.277216in}{1.551307in}}%
\pgfpathlineto{\pgfqpoint{4.263069in}{1.556934in}}%
\pgfpathlineto{\pgfqpoint{4.248928in}{1.562586in}}%
\pgfpathlineto{\pgfqpoint{4.234793in}{1.568262in}}%
\pgfpathlineto{\pgfqpoint{4.220663in}{1.573963in}}%
\pgfpathlineto{\pgfqpoint{4.228725in}{1.575800in}}%
\pgfpathlineto{\pgfqpoint{4.236780in}{1.577961in}}%
\pgfpathlineto{\pgfqpoint{4.244828in}{1.580438in}}%
\pgfpathlineto{\pgfqpoint{4.252868in}{1.583221in}}%
\pgfpathclose%
\pgfusepath{fill}%
\end{pgfscope}%
\begin{pgfscope}%
\pgfpathrectangle{\pgfqpoint{1.150000in}{0.150000in}}{\pgfqpoint{5.700000in}{5.700000in}}%
\pgfusepath{clip}%
\pgfsetbuttcap%
\pgfsetroundjoin%
\definecolor{currentfill}{rgb}{0.281446,0.084320,0.407414}%
\pgfsetfillcolor{currentfill}%
\pgfsetfillopacity{0.700000}%
\pgfsetlinewidth{0.000000pt}%
\definecolor{currentstroke}{rgb}{0.000000,0.000000,0.000000}%
\pgfsetstrokecolor{currentstroke}%
\pgfsetdash{}{0pt}%
\pgfpathmoveto{\pgfqpoint{5.244668in}{1.651603in}}%
\pgfpathlineto{\pgfqpoint{5.259065in}{1.649609in}}%
\pgfpathlineto{\pgfqpoint{5.273470in}{1.647638in}}%
\pgfpathlineto{\pgfqpoint{5.287884in}{1.645692in}}%
\pgfpathlineto{\pgfqpoint{5.302307in}{1.643769in}}%
\pgfpathlineto{\pgfqpoint{5.294559in}{1.632063in}}%
\pgfpathlineto{\pgfqpoint{5.286808in}{1.620397in}}%
\pgfpathlineto{\pgfqpoint{5.279053in}{1.608777in}}%
\pgfpathlineto{\pgfqpoint{5.271294in}{1.597208in}}%
\pgfpathlineto{\pgfqpoint{5.256867in}{1.599353in}}%
\pgfpathlineto{\pgfqpoint{5.242448in}{1.601521in}}%
\pgfpathlineto{\pgfqpoint{5.228037in}{1.603713in}}%
\pgfpathlineto{\pgfqpoint{5.213635in}{1.605928in}}%
\pgfpathlineto{\pgfqpoint{5.221399in}{1.617271in}}%
\pgfpathlineto{\pgfqpoint{5.229159in}{1.628668in}}%
\pgfpathlineto{\pgfqpoint{5.236915in}{1.640113in}}%
\pgfpathlineto{\pgfqpoint{5.244668in}{1.651603in}}%
\pgfpathclose%
\pgfusepath{fill}%
\end{pgfscope}%
\begin{pgfscope}%
\pgfpathrectangle{\pgfqpoint{1.150000in}{0.150000in}}{\pgfqpoint{5.700000in}{5.700000in}}%
\pgfusepath{clip}%
\pgfsetbuttcap%
\pgfsetroundjoin%
\definecolor{currentfill}{rgb}{0.260571,0.246922,0.522828}%
\pgfsetfillcolor{currentfill}%
\pgfsetfillopacity{0.700000}%
\pgfsetlinewidth{0.000000pt}%
\definecolor{currentstroke}{rgb}{0.000000,0.000000,0.000000}%
\pgfsetstrokecolor{currentstroke}%
\pgfsetdash{}{0pt}%
\pgfpathmoveto{\pgfqpoint{3.391453in}{1.967503in}}%
\pgfpathlineto{\pgfqpoint{3.405415in}{1.959378in}}%
\pgfpathlineto{\pgfqpoint{3.419381in}{1.951282in}}%
\pgfpathlineto{\pgfqpoint{3.433350in}{1.943215in}}%
\pgfpathlineto{\pgfqpoint{3.447324in}{1.935175in}}%
\pgfpathlineto{\pgfqpoint{3.438759in}{1.942972in}}%
\pgfpathlineto{\pgfqpoint{3.430175in}{1.951276in}}%
\pgfpathlineto{\pgfqpoint{3.421573in}{1.960097in}}%
\pgfpathlineto{\pgfqpoint{3.412953in}{1.969446in}}%
\pgfpathlineto{\pgfqpoint{3.398940in}{1.977874in}}%
\pgfpathlineto{\pgfqpoint{3.384932in}{1.986331in}}%
\pgfpathlineto{\pgfqpoint{3.370927in}{1.994815in}}%
\pgfpathlineto{\pgfqpoint{3.356925in}{2.003329in}}%
\pgfpathlineto{\pgfqpoint{3.365586in}{1.993584in}}%
\pgfpathlineto{\pgfqpoint{3.374227in}{1.984372in}}%
\pgfpathlineto{\pgfqpoint{3.382849in}{1.975682in}}%
\pgfpathlineto{\pgfqpoint{3.391453in}{1.967503in}}%
\pgfpathclose%
\pgfusepath{fill}%
\end{pgfscope}%
\begin{pgfscope}%
\pgfpathrectangle{\pgfqpoint{1.150000in}{0.150000in}}{\pgfqpoint{5.700000in}{5.700000in}}%
\pgfusepath{clip}%
\pgfsetbuttcap%
\pgfsetroundjoin%
\definecolor{currentfill}{rgb}{0.165117,0.467423,0.558141}%
\pgfsetfillcolor{currentfill}%
\pgfsetfillopacity{0.700000}%
\pgfsetlinewidth{0.000000pt}%
\definecolor{currentstroke}{rgb}{0.000000,0.000000,0.000000}%
\pgfsetstrokecolor{currentstroke}%
\pgfsetdash{}{0pt}%
\pgfpathmoveto{\pgfqpoint{2.651558in}{2.512724in}}%
\pgfpathlineto{\pgfqpoint{2.665468in}{2.502231in}}%
\pgfpathlineto{\pgfqpoint{2.679380in}{2.491775in}}%
\pgfpathlineto{\pgfqpoint{2.693294in}{2.481358in}}%
\pgfpathlineto{\pgfqpoint{2.707211in}{2.470978in}}%
\pgfpathlineto{\pgfqpoint{2.697902in}{2.488178in}}%
\pgfpathlineto{\pgfqpoint{2.688562in}{2.506031in}}%
\pgfpathlineto{\pgfqpoint{2.679188in}{2.524549in}}%
\pgfpathlineto{\pgfqpoint{2.669781in}{2.543747in}}%
\pgfpathlineto{\pgfqpoint{2.655809in}{2.554558in}}%
\pgfpathlineto{\pgfqpoint{2.641839in}{2.565407in}}%
\pgfpathlineto{\pgfqpoint{2.627871in}{2.576295in}}%
\pgfpathlineto{\pgfqpoint{2.613905in}{2.587220in}}%
\pgfpathlineto{\pgfqpoint{2.623370in}{2.567584in}}%
\pgfpathlineto{\pgfqpoint{2.632800in}{2.548631in}}%
\pgfpathlineto{\pgfqpoint{2.642196in}{2.530349in}}%
\pgfpathlineto{\pgfqpoint{2.651558in}{2.512724in}}%
\pgfpathclose%
\pgfusepath{fill}%
\end{pgfscope}%
\begin{pgfscope}%
\pgfpathrectangle{\pgfqpoint{1.150000in}{0.150000in}}{\pgfqpoint{5.700000in}{5.700000in}}%
\pgfusepath{clip}%
\pgfsetbuttcap%
\pgfsetroundjoin%
\definecolor{currentfill}{rgb}{0.221989,0.339161,0.548752}%
\pgfsetfillcolor{currentfill}%
\pgfsetfillopacity{0.700000}%
\pgfsetlinewidth{0.000000pt}%
\definecolor{currentstroke}{rgb}{0.000000,0.000000,0.000000}%
\pgfsetstrokecolor{currentstroke}%
\pgfsetdash{}{0pt}%
\pgfpathmoveto{\pgfqpoint{3.077645in}{2.179789in}}%
\pgfpathlineto{\pgfqpoint{3.091577in}{2.170676in}}%
\pgfpathlineto{\pgfqpoint{3.105513in}{2.161595in}}%
\pgfpathlineto{\pgfqpoint{3.119451in}{2.152545in}}%
\pgfpathlineto{\pgfqpoint{3.133393in}{2.143526in}}%
\pgfpathlineto{\pgfqpoint{3.124540in}{2.155415in}}%
\pgfpathlineto{\pgfqpoint{3.115664in}{2.167877in}}%
\pgfpathlineto{\pgfqpoint{3.106763in}{2.180923in}}%
\pgfpathlineto{\pgfqpoint{3.097838in}{2.194566in}}%
\pgfpathlineto{\pgfqpoint{3.083850in}{2.203992in}}%
\pgfpathlineto{\pgfqpoint{3.069865in}{2.213450in}}%
\pgfpathlineto{\pgfqpoint{3.055883in}{2.222940in}}%
\pgfpathlineto{\pgfqpoint{3.041904in}{2.232461in}}%
\pgfpathlineto{\pgfqpoint{3.050877in}{2.218404in}}%
\pgfpathlineto{\pgfqpoint{3.059825in}{2.204947in}}%
\pgfpathlineto{\pgfqpoint{3.068747in}{2.192080in}}%
\pgfpathlineto{\pgfqpoint{3.077645in}{2.179789in}}%
\pgfpathclose%
\pgfusepath{fill}%
\end{pgfscope}%
\begin{pgfscope}%
\pgfpathrectangle{\pgfqpoint{1.150000in}{0.150000in}}{\pgfqpoint{5.700000in}{5.700000in}}%
\pgfusepath{clip}%
\pgfsetbuttcap%
\pgfsetroundjoin%
\definecolor{currentfill}{rgb}{0.279566,0.067836,0.391917}%
\pgfsetfillcolor{currentfill}%
\pgfsetfillopacity{0.700000}%
\pgfsetlinewidth{0.000000pt}%
\definecolor{currentstroke}{rgb}{0.000000,0.000000,0.000000}%
\pgfsetstrokecolor{currentstroke}%
\pgfsetdash{}{0pt}%
\pgfpathmoveto{\pgfqpoint{5.156109in}{1.615030in}}%
\pgfpathlineto{\pgfqpoint{5.170478in}{1.612719in}}%
\pgfpathlineto{\pgfqpoint{5.184855in}{1.610432in}}%
\pgfpathlineto{\pgfqpoint{5.199241in}{1.608168in}}%
\pgfpathlineto{\pgfqpoint{5.213635in}{1.605928in}}%
\pgfpathlineto{\pgfqpoint{5.205867in}{1.594646in}}%
\pgfpathlineto{\pgfqpoint{5.198096in}{1.583429in}}%
\pgfpathlineto{\pgfqpoint{5.190322in}{1.572284in}}%
\pgfpathlineto{\pgfqpoint{5.182544in}{1.561216in}}%
\pgfpathlineto{\pgfqpoint{5.168144in}{1.563690in}}%
\pgfpathlineto{\pgfqpoint{5.153752in}{1.566188in}}%
\pgfpathlineto{\pgfqpoint{5.139369in}{1.568710in}}%
\pgfpathlineto{\pgfqpoint{5.124994in}{1.571256in}}%
\pgfpathlineto{\pgfqpoint{5.132778in}{1.582085in}}%
\pgfpathlineto{\pgfqpoint{5.140558in}{1.592994in}}%
\pgfpathlineto{\pgfqpoint{5.148335in}{1.603978in}}%
\pgfpathlineto{\pgfqpoint{5.156109in}{1.615030in}}%
\pgfpathclose%
\pgfusepath{fill}%
\end{pgfscope}%
\begin{pgfscope}%
\pgfpathrectangle{\pgfqpoint{1.150000in}{0.150000in}}{\pgfqpoint{5.700000in}{5.700000in}}%
\pgfusepath{clip}%
\pgfsetbuttcap%
\pgfsetroundjoin%
\definecolor{currentfill}{rgb}{0.272594,0.025563,0.353093}%
\pgfsetfillcolor{currentfill}%
\pgfsetfillopacity{0.700000}%
\pgfsetlinewidth{0.000000pt}%
\definecolor{currentstroke}{rgb}{0.000000,0.000000,0.000000}%
\pgfsetstrokecolor{currentstroke}%
\pgfsetdash{}{0pt}%
\pgfpathmoveto{\pgfqpoint{4.833430in}{1.540350in}}%
\pgfpathlineto{\pgfqpoint{4.847697in}{1.536916in}}%
\pgfpathlineto{\pgfqpoint{4.861971in}{1.533506in}}%
\pgfpathlineto{\pgfqpoint{4.876253in}{1.530119in}}%
\pgfpathlineto{\pgfqpoint{4.890542in}{1.526756in}}%
\pgfpathlineto{\pgfqpoint{4.882705in}{1.517736in}}%
\pgfpathlineto{\pgfqpoint{4.874865in}{1.508870in}}%
\pgfpathlineto{\pgfqpoint{4.867021in}{1.500163in}}%
\pgfpathlineto{\pgfqpoint{4.859174in}{1.491625in}}%
\pgfpathlineto{\pgfqpoint{4.844875in}{1.495261in}}%
\pgfpathlineto{\pgfqpoint{4.830583in}{1.498921in}}%
\pgfpathlineto{\pgfqpoint{4.816299in}{1.502605in}}%
\pgfpathlineto{\pgfqpoint{4.802022in}{1.506313in}}%
\pgfpathlineto{\pgfqpoint{4.809879in}{1.514573in}}%
\pgfpathlineto{\pgfqpoint{4.817733in}{1.523003in}}%
\pgfpathlineto{\pgfqpoint{4.825583in}{1.531598in}}%
\pgfpathlineto{\pgfqpoint{4.833430in}{1.540350in}}%
\pgfpathclose%
\pgfusepath{fill}%
\end{pgfscope}%
\begin{pgfscope}%
\pgfpathrectangle{\pgfqpoint{1.150000in}{0.150000in}}{\pgfqpoint{5.700000in}{5.700000in}}%
\pgfusepath{clip}%
\pgfsetbuttcap%
\pgfsetroundjoin%
\definecolor{currentfill}{rgb}{0.283229,0.120777,0.440584}%
\pgfsetfillcolor{currentfill}%
\pgfsetfillopacity{0.700000}%
\pgfsetlinewidth{0.000000pt}%
\definecolor{currentstroke}{rgb}{0.000000,0.000000,0.000000}%
\pgfsetstrokecolor{currentstroke}%
\pgfsetdash{}{0pt}%
\pgfpathmoveto{\pgfqpoint{3.906464in}{1.696052in}}%
\pgfpathlineto{\pgfqpoint{3.920508in}{1.689513in}}%
\pgfpathlineto{\pgfqpoint{3.934556in}{1.682999in}}%
\pgfpathlineto{\pgfqpoint{3.948610in}{1.676511in}}%
\pgfpathlineto{\pgfqpoint{3.962668in}{1.670048in}}%
\pgfpathlineto{\pgfqpoint{3.954468in}{1.671367in}}%
\pgfpathlineto{\pgfqpoint{3.946257in}{1.673080in}}%
\pgfpathlineto{\pgfqpoint{3.938036in}{1.675197in}}%
\pgfpathlineto{\pgfqpoint{3.929804in}{1.677726in}}%
\pgfpathlineto{\pgfqpoint{3.915718in}{1.684545in}}%
\pgfpathlineto{\pgfqpoint{3.901636in}{1.691389in}}%
\pgfpathlineto{\pgfqpoint{3.887560in}{1.698258in}}%
\pgfpathlineto{\pgfqpoint{3.873489in}{1.705152in}}%
\pgfpathlineto{\pgfqpoint{3.881750in}{1.702262in}}%
\pgfpathlineto{\pgfqpoint{3.889999in}{1.699788in}}%
\pgfpathlineto{\pgfqpoint{3.898237in}{1.697721in}}%
\pgfpathlineto{\pgfqpoint{3.906464in}{1.696052in}}%
\pgfpathclose%
\pgfusepath{fill}%
\end{pgfscope}%
\begin{pgfscope}%
\pgfpathrectangle{\pgfqpoint{1.150000in}{0.150000in}}{\pgfqpoint{5.700000in}{5.700000in}}%
\pgfusepath{clip}%
\pgfsetbuttcap%
\pgfsetroundjoin%
\definecolor{currentfill}{rgb}{0.280894,0.078907,0.402329}%
\pgfsetfillcolor{currentfill}%
\pgfsetfillopacity{0.700000}%
\pgfsetlinewidth{0.000000pt}%
\definecolor{currentstroke}{rgb}{0.000000,0.000000,0.000000}%
\pgfsetstrokecolor{currentstroke}%
\pgfsetdash{}{0pt}%
\pgfpathmoveto{\pgfqpoint{4.107837in}{1.620448in}}%
\pgfpathlineto{\pgfqpoint{4.121921in}{1.614552in}}%
\pgfpathlineto{\pgfqpoint{4.136010in}{1.608679in}}%
\pgfpathlineto{\pgfqpoint{4.150104in}{1.602832in}}%
\pgfpathlineto{\pgfqpoint{4.164205in}{1.597009in}}%
\pgfpathlineto{\pgfqpoint{4.156113in}{1.595840in}}%
\pgfpathlineto{\pgfqpoint{4.148014in}{1.595018in}}%
\pgfpathlineto{\pgfqpoint{4.139907in}{1.594549in}}%
\pgfpathlineto{\pgfqpoint{4.131791in}{1.594445in}}%
\pgfpathlineto{\pgfqpoint{4.117668in}{1.600609in}}%
\pgfpathlineto{\pgfqpoint{4.103550in}{1.606798in}}%
\pgfpathlineto{\pgfqpoint{4.089437in}{1.613011in}}%
\pgfpathlineto{\pgfqpoint{4.075330in}{1.619249in}}%
\pgfpathlineto{\pgfqpoint{4.083470in}{1.619006in}}%
\pgfpathlineto{\pgfqpoint{4.091601in}{1.619132in}}%
\pgfpathlineto{\pgfqpoint{4.099723in}{1.619616in}}%
\pgfpathlineto{\pgfqpoint{4.107837in}{1.620448in}}%
\pgfpathclose%
\pgfusepath{fill}%
\end{pgfscope}%
\begin{pgfscope}%
\pgfpathrectangle{\pgfqpoint{1.150000in}{0.150000in}}{\pgfqpoint{5.700000in}{5.700000in}}%
\pgfusepath{clip}%
\pgfsetbuttcap%
\pgfsetroundjoin%
\definecolor{currentfill}{rgb}{0.277018,0.050344,0.375715}%
\pgfsetfillcolor{currentfill}%
\pgfsetfillopacity{0.700000}%
\pgfsetlinewidth{0.000000pt}%
\definecolor{currentstroke}{rgb}{0.000000,0.000000,0.000000}%
\pgfsetstrokecolor{currentstroke}%
\pgfsetdash{}{0pt}%
\pgfpathmoveto{\pgfqpoint{5.067573in}{1.581677in}}%
\pgfpathlineto{\pgfqpoint{5.081916in}{1.579036in}}%
\pgfpathlineto{\pgfqpoint{5.096267in}{1.576419in}}%
\pgfpathlineto{\pgfqpoint{5.110626in}{1.573826in}}%
\pgfpathlineto{\pgfqpoint{5.124994in}{1.571256in}}%
\pgfpathlineto{\pgfqpoint{5.117206in}{1.560514in}}%
\pgfpathlineto{\pgfqpoint{5.109416in}{1.549864in}}%
\pgfpathlineto{\pgfqpoint{5.101622in}{1.539313in}}%
\pgfpathlineto{\pgfqpoint{5.093825in}{1.528866in}}%
\pgfpathlineto{\pgfqpoint{5.079451in}{1.531683in}}%
\pgfpathlineto{\pgfqpoint{5.065085in}{1.534524in}}%
\pgfpathlineto{\pgfqpoint{5.050727in}{1.537389in}}%
\pgfpathlineto{\pgfqpoint{5.036377in}{1.540278in}}%
\pgfpathlineto{\pgfqpoint{5.044181in}{1.550472in}}%
\pgfpathlineto{\pgfqpoint{5.051982in}{1.560774in}}%
\pgfpathlineto{\pgfqpoint{5.059779in}{1.571178in}}%
\pgfpathlineto{\pgfqpoint{5.067573in}{1.581677in}}%
\pgfpathclose%
\pgfusepath{fill}%
\end{pgfscope}%
\begin{pgfscope}%
\pgfpathrectangle{\pgfqpoint{1.150000in}{0.150000in}}{\pgfqpoint{5.700000in}{5.700000in}}%
\pgfusepath{clip}%
\pgfsetbuttcap%
\pgfsetroundjoin%
\definecolor{currentfill}{rgb}{0.140210,0.665859,0.513427}%
\pgfsetfillcolor{currentfill}%
\pgfsetfillopacity{0.700000}%
\pgfsetlinewidth{0.000000pt}%
\definecolor{currentstroke}{rgb}{0.000000,0.000000,0.000000}%
\pgfsetstrokecolor{currentstroke}%
\pgfsetdash{}{0pt}%
\pgfpathmoveto{\pgfqpoint{2.056149in}{3.059873in}}%
\pgfpathlineto{\pgfqpoint{2.070085in}{3.047096in}}%
\pgfpathlineto{\pgfqpoint{2.084021in}{3.034375in}}%
\pgfpathlineto{\pgfqpoint{2.097956in}{3.021710in}}%
\pgfpathlineto{\pgfqpoint{2.111891in}{3.009099in}}%
\pgfpathlineto{\pgfqpoint{2.101822in}{3.033485in}}%
\pgfpathlineto{\pgfqpoint{2.091707in}{3.058626in}}%
\pgfpathlineto{\pgfqpoint{2.081544in}{3.084537in}}%
\pgfpathlineto{\pgfqpoint{2.071334in}{3.111233in}}%
\pgfpathlineto{\pgfqpoint{2.057329in}{3.124310in}}%
\pgfpathlineto{\pgfqpoint{2.043324in}{3.137441in}}%
\pgfpathlineto{\pgfqpoint{2.029318in}{3.150628in}}%
\pgfpathlineto{\pgfqpoint{2.015312in}{3.163871in}}%
\pgfpathlineto{\pgfqpoint{2.025594in}{3.136701in}}%
\pgfpathlineto{\pgfqpoint{2.035827in}{3.110321in}}%
\pgfpathlineto{\pgfqpoint{2.046012in}{3.084716in}}%
\pgfpathlineto{\pgfqpoint{2.056149in}{3.059873in}}%
\pgfpathclose%
\pgfusepath{fill}%
\end{pgfscope}%
\begin{pgfscope}%
\pgfpathrectangle{\pgfqpoint{1.150000in}{0.150000in}}{\pgfqpoint{5.700000in}{5.700000in}}%
\pgfusepath{clip}%
\pgfsetbuttcap%
\pgfsetroundjoin%
\definecolor{currentfill}{rgb}{0.171176,0.452530,0.557965}%
\pgfsetfillcolor{currentfill}%
\pgfsetfillopacity{0.700000}%
\pgfsetlinewidth{0.000000pt}%
\definecolor{currentstroke}{rgb}{0.000000,0.000000,0.000000}%
\pgfsetstrokecolor{currentstroke}%
\pgfsetdash{}{0pt}%
\pgfpathmoveto{\pgfqpoint{2.707211in}{2.470978in}}%
\pgfpathlineto{\pgfqpoint{2.721129in}{2.460635in}}%
\pgfpathlineto{\pgfqpoint{2.735049in}{2.450329in}}%
\pgfpathlineto{\pgfqpoint{2.748971in}{2.440060in}}%
\pgfpathlineto{\pgfqpoint{2.762896in}{2.429828in}}%
\pgfpathlineto{\pgfqpoint{2.753642in}{2.446604in}}%
\pgfpathlineto{\pgfqpoint{2.744356in}{2.464029in}}%
\pgfpathlineto{\pgfqpoint{2.735038in}{2.482114in}}%
\pgfpathlineto{\pgfqpoint{2.725687in}{2.500874in}}%
\pgfpathlineto{\pgfqpoint{2.711708in}{2.511537in}}%
\pgfpathlineto{\pgfqpoint{2.697730in}{2.522236in}}%
\pgfpathlineto{\pgfqpoint{2.683755in}{2.532973in}}%
\pgfpathlineto{\pgfqpoint{2.669781in}{2.543747in}}%
\pgfpathlineto{\pgfqpoint{2.679188in}{2.524549in}}%
\pgfpathlineto{\pgfqpoint{2.688562in}{2.506031in}}%
\pgfpathlineto{\pgfqpoint{2.697902in}{2.488178in}}%
\pgfpathlineto{\pgfqpoint{2.707211in}{2.470978in}}%
\pgfpathclose%
\pgfusepath{fill}%
\end{pgfscope}%
\begin{pgfscope}%
\pgfpathrectangle{\pgfqpoint{1.150000in}{0.150000in}}{\pgfqpoint{5.700000in}{5.700000in}}%
\pgfusepath{clip}%
\pgfsetbuttcap%
\pgfsetroundjoin%
\definecolor{currentfill}{rgb}{0.273809,0.031497,0.358853}%
\pgfsetfillcolor{currentfill}%
\pgfsetfillopacity{0.700000}%
\pgfsetlinewidth{0.000000pt}%
\definecolor{currentstroke}{rgb}{0.000000,0.000000,0.000000}%
\pgfsetstrokecolor{currentstroke}%
\pgfsetdash{}{0pt}%
\pgfpathmoveto{\pgfqpoint{4.454458in}{1.538128in}}%
\pgfpathlineto{\pgfqpoint{4.468624in}{1.533357in}}%
\pgfpathlineto{\pgfqpoint{4.482797in}{1.528611in}}%
\pgfpathlineto{\pgfqpoint{4.496976in}{1.523889in}}%
\pgfpathlineto{\pgfqpoint{4.511161in}{1.519190in}}%
\pgfpathlineto{\pgfqpoint{4.503215in}{1.514032in}}%
\pgfpathlineto{\pgfqpoint{4.495263in}{1.509134in}}%
\pgfpathlineto{\pgfqpoint{4.487307in}{1.504503in}}%
\pgfpathlineto{\pgfqpoint{4.479346in}{1.500148in}}%
\pgfpathlineto{\pgfqpoint{4.465144in}{1.505160in}}%
\pgfpathlineto{\pgfqpoint{4.450948in}{1.510196in}}%
\pgfpathlineto{\pgfqpoint{4.436759in}{1.515255in}}%
\pgfpathlineto{\pgfqpoint{4.422577in}{1.520338in}}%
\pgfpathlineto{\pgfqpoint{4.430555in}{1.524375in}}%
\pgfpathlineto{\pgfqpoint{4.438528in}{1.528690in}}%
\pgfpathlineto{\pgfqpoint{4.446496in}{1.533278in}}%
\pgfpathlineto{\pgfqpoint{4.454458in}{1.538128in}}%
\pgfpathclose%
\pgfusepath{fill}%
\end{pgfscope}%
\begin{pgfscope}%
\pgfpathrectangle{\pgfqpoint{1.150000in}{0.150000in}}{\pgfqpoint{5.700000in}{5.700000in}}%
\pgfusepath{clip}%
\pgfsetbuttcap%
\pgfsetroundjoin%
\definecolor{currentfill}{rgb}{0.279574,0.170599,0.479997}%
\pgfsetfillcolor{currentfill}%
\pgfsetfillopacity{0.700000}%
\pgfsetlinewidth{0.000000pt}%
\definecolor{currentstroke}{rgb}{0.000000,0.000000,0.000000}%
\pgfsetstrokecolor{currentstroke}%
\pgfsetdash{}{0pt}%
\pgfpathmoveto{\pgfqpoint{3.705014in}{1.789897in}}%
\pgfpathlineto{\pgfqpoint{3.719027in}{1.782692in}}%
\pgfpathlineto{\pgfqpoint{3.733045in}{1.775513in}}%
\pgfpathlineto{\pgfqpoint{3.747068in}{1.768360in}}%
\pgfpathlineto{\pgfqpoint{3.761095in}{1.761233in}}%
\pgfpathlineto{\pgfqpoint{3.752762in}{1.765282in}}%
\pgfpathlineto{\pgfqpoint{3.744416in}{1.769775in}}%
\pgfpathlineto{\pgfqpoint{3.736057in}{1.774723in}}%
\pgfpathlineto{\pgfqpoint{3.727684in}{1.780136in}}%
\pgfpathlineto{\pgfqpoint{3.713624in}{1.787634in}}%
\pgfpathlineto{\pgfqpoint{3.699569in}{1.795158in}}%
\pgfpathlineto{\pgfqpoint{3.685518in}{1.802708in}}%
\pgfpathlineto{\pgfqpoint{3.671472in}{1.810285in}}%
\pgfpathlineto{\pgfqpoint{3.679879in}{1.804495in}}%
\pgfpathlineto{\pgfqpoint{3.688271in}{1.799174in}}%
\pgfpathlineto{\pgfqpoint{3.696649in}{1.794311in}}%
\pgfpathlineto{\pgfqpoint{3.705014in}{1.789897in}}%
\pgfpathclose%
\pgfusepath{fill}%
\end{pgfscope}%
\begin{pgfscope}%
\pgfpathrectangle{\pgfqpoint{1.150000in}{0.150000in}}{\pgfqpoint{5.700000in}{5.700000in}}%
\pgfusepath{clip}%
\pgfsetbuttcap%
\pgfsetroundjoin%
\definecolor{currentfill}{rgb}{0.272594,0.025563,0.353093}%
\pgfsetfillcolor{currentfill}%
\pgfsetfillopacity{0.700000}%
\pgfsetlinewidth{0.000000pt}%
\definecolor{currentstroke}{rgb}{0.000000,0.000000,0.000000}%
\pgfsetstrokecolor{currentstroke}%
\pgfsetdash{}{0pt}%
\pgfpathmoveto{\pgfqpoint{4.599648in}{1.524906in}}%
\pgfpathlineto{\pgfqpoint{4.613852in}{1.520627in}}%
\pgfpathlineto{\pgfqpoint{4.628064in}{1.516371in}}%
\pgfpathlineto{\pgfqpoint{4.642282in}{1.512140in}}%
\pgfpathlineto{\pgfqpoint{4.656507in}{1.507932in}}%
\pgfpathlineto{\pgfqpoint{4.648607in}{1.501225in}}%
\pgfpathlineto{\pgfqpoint{4.640703in}{1.494739in}}%
\pgfpathlineto{\pgfqpoint{4.632795in}{1.488482in}}%
\pgfpathlineto{\pgfqpoint{4.624883in}{1.482463in}}%
\pgfpathlineto{\pgfqpoint{4.610644in}{1.486970in}}%
\pgfpathlineto{\pgfqpoint{4.596412in}{1.491502in}}%
\pgfpathlineto{\pgfqpoint{4.582187in}{1.496057in}}%
\pgfpathlineto{\pgfqpoint{4.567969in}{1.500636in}}%
\pgfpathlineto{\pgfqpoint{4.575895in}{1.506351in}}%
\pgfpathlineto{\pgfqpoint{4.583817in}{1.512306in}}%
\pgfpathlineto{\pgfqpoint{4.591735in}{1.518493in}}%
\pgfpathlineto{\pgfqpoint{4.599648in}{1.524906in}}%
\pgfpathclose%
\pgfusepath{fill}%
\end{pgfscope}%
\begin{pgfscope}%
\pgfpathrectangle{\pgfqpoint{1.150000in}{0.150000in}}{\pgfqpoint{5.700000in}{5.700000in}}%
\pgfusepath{clip}%
\pgfsetbuttcap%
\pgfsetroundjoin%
\definecolor{currentfill}{rgb}{0.263663,0.237631,0.518762}%
\pgfsetfillcolor{currentfill}%
\pgfsetfillopacity{0.700000}%
\pgfsetlinewidth{0.000000pt}%
\definecolor{currentstroke}{rgb}{0.000000,0.000000,0.000000}%
\pgfsetstrokecolor{currentstroke}%
\pgfsetdash{}{0pt}%
\pgfpathmoveto{\pgfqpoint{3.447324in}{1.935175in}}%
\pgfpathlineto{\pgfqpoint{3.461302in}{1.927164in}}%
\pgfpathlineto{\pgfqpoint{3.475284in}{1.919180in}}%
\pgfpathlineto{\pgfqpoint{3.489270in}{1.911224in}}%
\pgfpathlineto{\pgfqpoint{3.503260in}{1.903296in}}%
\pgfpathlineto{\pgfqpoint{3.494732in}{1.910711in}}%
\pgfpathlineto{\pgfqpoint{3.486186in}{1.918628in}}%
\pgfpathlineto{\pgfqpoint{3.477623in}{1.927059in}}%
\pgfpathlineto{\pgfqpoint{3.469042in}{1.936015in}}%
\pgfpathlineto{\pgfqpoint{3.455014in}{1.944331in}}%
\pgfpathlineto{\pgfqpoint{3.440990in}{1.952675in}}%
\pgfpathlineto{\pgfqpoint{3.426969in}{1.961047in}}%
\pgfpathlineto{\pgfqpoint{3.412953in}{1.969446in}}%
\pgfpathlineto{\pgfqpoint{3.421573in}{1.960097in}}%
\pgfpathlineto{\pgfqpoint{3.430175in}{1.951276in}}%
\pgfpathlineto{\pgfqpoint{3.438759in}{1.942972in}}%
\pgfpathlineto{\pgfqpoint{3.447324in}{1.935175in}}%
\pgfpathclose%
\pgfusepath{fill}%
\end{pgfscope}%
\begin{pgfscope}%
\pgfpathrectangle{\pgfqpoint{1.150000in}{0.150000in}}{\pgfqpoint{5.700000in}{5.700000in}}%
\pgfusepath{clip}%
\pgfsetbuttcap%
\pgfsetroundjoin%
\definecolor{currentfill}{rgb}{0.225863,0.330805,0.547314}%
\pgfsetfillcolor{currentfill}%
\pgfsetfillopacity{0.700000}%
\pgfsetlinewidth{0.000000pt}%
\definecolor{currentstroke}{rgb}{0.000000,0.000000,0.000000}%
\pgfsetstrokecolor{currentstroke}%
\pgfsetdash{}{0pt}%
\pgfpathmoveto{\pgfqpoint{3.133393in}{2.143526in}}%
\pgfpathlineto{\pgfqpoint{3.147338in}{2.134538in}}%
\pgfpathlineto{\pgfqpoint{3.161286in}{2.125581in}}%
\pgfpathlineto{\pgfqpoint{3.175238in}{2.116654in}}%
\pgfpathlineto{\pgfqpoint{3.189193in}{2.107758in}}%
\pgfpathlineto{\pgfqpoint{3.180385in}{2.119247in}}%
\pgfpathlineto{\pgfqpoint{3.171553in}{2.131303in}}%
\pgfpathlineto{\pgfqpoint{3.162699in}{2.143940in}}%
\pgfpathlineto{\pgfqpoint{3.153820in}{2.157169in}}%
\pgfpathlineto{\pgfqpoint{3.139820in}{2.166472in}}%
\pgfpathlineto{\pgfqpoint{3.125823in}{2.175806in}}%
\pgfpathlineto{\pgfqpoint{3.111829in}{2.185170in}}%
\pgfpathlineto{\pgfqpoint{3.097838in}{2.194566in}}%
\pgfpathlineto{\pgfqpoint{3.106763in}{2.180923in}}%
\pgfpathlineto{\pgfqpoint{3.115664in}{2.167877in}}%
\pgfpathlineto{\pgfqpoint{3.124540in}{2.155415in}}%
\pgfpathlineto{\pgfqpoint{3.133393in}{2.143526in}}%
\pgfpathclose%
\pgfusepath{fill}%
\end{pgfscope}%
\begin{pgfscope}%
\pgfpathrectangle{\pgfqpoint{1.150000in}{0.150000in}}{\pgfqpoint{5.700000in}{5.700000in}}%
\pgfusepath{clip}%
\pgfsetbuttcap%
\pgfsetroundjoin%
\definecolor{currentfill}{rgb}{0.128087,0.647749,0.523491}%
\pgfsetfillcolor{currentfill}%
\pgfsetfillopacity{0.700000}%
\pgfsetlinewidth{0.000000pt}%
\definecolor{currentstroke}{rgb}{0.000000,0.000000,0.000000}%
\pgfsetstrokecolor{currentstroke}%
\pgfsetdash{}{0pt}%
\pgfpathmoveto{\pgfqpoint{2.111891in}{3.009099in}}%
\pgfpathlineto{\pgfqpoint{2.125826in}{2.996542in}}%
\pgfpathlineto{\pgfqpoint{2.139761in}{2.984039in}}%
\pgfpathlineto{\pgfqpoint{2.153696in}{2.971588in}}%
\pgfpathlineto{\pgfqpoint{2.167631in}{2.959189in}}%
\pgfpathlineto{\pgfqpoint{2.157629in}{2.983119in}}%
\pgfpathlineto{\pgfqpoint{2.147582in}{3.007798in}}%
\pgfpathlineto{\pgfqpoint{2.137489in}{3.033243in}}%
\pgfpathlineto{\pgfqpoint{2.127348in}{3.059467in}}%
\pgfpathlineto{\pgfqpoint{2.113345in}{3.072329in}}%
\pgfpathlineto{\pgfqpoint{2.099342in}{3.085243in}}%
\pgfpathlineto{\pgfqpoint{2.085338in}{3.098211in}}%
\pgfpathlineto{\pgfqpoint{2.071334in}{3.111233in}}%
\pgfpathlineto{\pgfqpoint{2.081544in}{3.084537in}}%
\pgfpathlineto{\pgfqpoint{2.091707in}{3.058626in}}%
\pgfpathlineto{\pgfqpoint{2.101822in}{3.033485in}}%
\pgfpathlineto{\pgfqpoint{2.111891in}{3.009099in}}%
\pgfpathclose%
\pgfusepath{fill}%
\end{pgfscope}%
\begin{pgfscope}%
\pgfpathrectangle{\pgfqpoint{1.150000in}{0.150000in}}{\pgfqpoint{5.700000in}{5.700000in}}%
\pgfusepath{clip}%
\pgfsetbuttcap%
\pgfsetroundjoin%
\definecolor{currentfill}{rgb}{0.277018,0.050344,0.375715}%
\pgfsetfillcolor{currentfill}%
\pgfsetfillopacity{0.700000}%
\pgfsetlinewidth{0.000000pt}%
\definecolor{currentstroke}{rgb}{0.000000,0.000000,0.000000}%
\pgfsetstrokecolor{currentstroke}%
\pgfsetdash{}{0pt}%
\pgfpathmoveto{\pgfqpoint{4.309340in}{1.561873in}}%
\pgfpathlineto{\pgfqpoint{4.323473in}{1.556596in}}%
\pgfpathlineto{\pgfqpoint{4.337612in}{1.551344in}}%
\pgfpathlineto{\pgfqpoint{4.351757in}{1.546116in}}%
\pgfpathlineto{\pgfqpoint{4.365909in}{1.540913in}}%
\pgfpathlineto{\pgfqpoint{4.357906in}{1.537486in}}%
\pgfpathlineto{\pgfqpoint{4.349898in}{1.534359in}}%
\pgfpathlineto{\pgfqpoint{4.341883in}{1.531541in}}%
\pgfpathlineto{\pgfqpoint{4.333863in}{1.529039in}}%
\pgfpathlineto{\pgfqpoint{4.319692in}{1.534570in}}%
\pgfpathlineto{\pgfqpoint{4.305527in}{1.540125in}}%
\pgfpathlineto{\pgfqpoint{4.291368in}{1.545704in}}%
\pgfpathlineto{\pgfqpoint{4.277216in}{1.551307in}}%
\pgfpathlineto{\pgfqpoint{4.285257in}{1.553476in}}%
\pgfpathlineto{\pgfqpoint{4.293291in}{1.555966in}}%
\pgfpathlineto{\pgfqpoint{4.301318in}{1.558768in}}%
\pgfpathlineto{\pgfqpoint{4.309340in}{1.561873in}}%
\pgfpathclose%
\pgfusepath{fill}%
\end{pgfscope}%
\begin{pgfscope}%
\pgfpathrectangle{\pgfqpoint{1.150000in}{0.150000in}}{\pgfqpoint{5.700000in}{5.700000in}}%
\pgfusepath{clip}%
\pgfsetbuttcap%
\pgfsetroundjoin%
\definecolor{currentfill}{rgb}{0.274952,0.037752,0.364543}%
\pgfsetfillcolor{currentfill}%
\pgfsetfillopacity{0.700000}%
\pgfsetlinewidth{0.000000pt}%
\definecolor{currentstroke}{rgb}{0.000000,0.000000,0.000000}%
\pgfsetstrokecolor{currentstroke}%
\pgfsetdash{}{0pt}%
\pgfpathmoveto{\pgfqpoint{4.979054in}{1.552069in}}%
\pgfpathlineto{\pgfqpoint{4.993373in}{1.549086in}}%
\pgfpathlineto{\pgfqpoint{5.007700in}{1.546126in}}%
\pgfpathlineto{\pgfqpoint{5.022035in}{1.543190in}}%
\pgfpathlineto{\pgfqpoint{5.036377in}{1.540278in}}%
\pgfpathlineto{\pgfqpoint{5.028570in}{1.530198in}}%
\pgfpathlineto{\pgfqpoint{5.020759in}{1.520239in}}%
\pgfpathlineto{\pgfqpoint{5.012946in}{1.510407in}}%
\pgfpathlineto{\pgfqpoint{5.005130in}{1.500708in}}%
\pgfpathlineto{\pgfqpoint{4.990779in}{1.503881in}}%
\pgfpathlineto{\pgfqpoint{4.976437in}{1.507078in}}%
\pgfpathlineto{\pgfqpoint{4.962102in}{1.510299in}}%
\pgfpathlineto{\pgfqpoint{4.947775in}{1.513543in}}%
\pgfpathlineto{\pgfqpoint{4.955600in}{1.522975in}}%
\pgfpathlineto{\pgfqpoint{4.963421in}{1.532545in}}%
\pgfpathlineto{\pgfqpoint{4.971239in}{1.542245in}}%
\pgfpathlineto{\pgfqpoint{4.979054in}{1.552069in}}%
\pgfpathclose%
\pgfusepath{fill}%
\end{pgfscope}%
\begin{pgfscope}%
\pgfpathrectangle{\pgfqpoint{1.150000in}{0.150000in}}{\pgfqpoint{5.700000in}{5.700000in}}%
\pgfusepath{clip}%
\pgfsetbuttcap%
\pgfsetroundjoin%
\definecolor{currentfill}{rgb}{0.272594,0.025563,0.353093}%
\pgfsetfillcolor{currentfill}%
\pgfsetfillopacity{0.700000}%
\pgfsetlinewidth{0.000000pt}%
\definecolor{currentstroke}{rgb}{0.000000,0.000000,0.000000}%
\pgfsetstrokecolor{currentstroke}%
\pgfsetdash{}{0pt}%
\pgfpathmoveto{\pgfqpoint{4.744986in}{1.521380in}}%
\pgfpathlineto{\pgfqpoint{4.759234in}{1.517577in}}%
\pgfpathlineto{\pgfqpoint{4.773490in}{1.513799in}}%
\pgfpathlineto{\pgfqpoint{4.787752in}{1.510044in}}%
\pgfpathlineto{\pgfqpoint{4.802022in}{1.506313in}}%
\pgfpathlineto{\pgfqpoint{4.794161in}{1.498231in}}%
\pgfpathlineto{\pgfqpoint{4.786297in}{1.490334in}}%
\pgfpathlineto{\pgfqpoint{4.778429in}{1.482629in}}%
\pgfpathlineto{\pgfqpoint{4.770558in}{1.475125in}}%
\pgfpathlineto{\pgfqpoint{4.756277in}{1.479143in}}%
\pgfpathlineto{\pgfqpoint{4.742003in}{1.483184in}}%
\pgfpathlineto{\pgfqpoint{4.727736in}{1.487250in}}%
\pgfpathlineto{\pgfqpoint{4.713476in}{1.491339in}}%
\pgfpathlineto{\pgfqpoint{4.721359in}{1.498552in}}%
\pgfpathlineto{\pgfqpoint{4.729239in}{1.505968in}}%
\pgfpathlineto{\pgfqpoint{4.737114in}{1.513579in}}%
\pgfpathlineto{\pgfqpoint{4.744986in}{1.521380in}}%
\pgfpathclose%
\pgfusepath{fill}%
\end{pgfscope}%
\begin{pgfscope}%
\pgfpathrectangle{\pgfqpoint{1.150000in}{0.150000in}}{\pgfqpoint{5.700000in}{5.700000in}}%
\pgfusepath{clip}%
\pgfsetbuttcap%
\pgfsetroundjoin%
\definecolor{currentfill}{rgb}{0.281924,0.089666,0.412415}%
\pgfsetfillcolor{currentfill}%
\pgfsetfillopacity{0.700000}%
\pgfsetlinewidth{0.000000pt}%
\definecolor{currentstroke}{rgb}{0.000000,0.000000,0.000000}%
\pgfsetstrokecolor{currentstroke}%
\pgfsetdash{}{0pt}%
\pgfpathmoveto{\pgfqpoint{5.302307in}{1.643769in}}%
\pgfpathlineto{\pgfqpoint{5.316738in}{1.641870in}}%
\pgfpathlineto{\pgfqpoint{5.331177in}{1.639996in}}%
\pgfpathlineto{\pgfqpoint{5.345625in}{1.638145in}}%
\pgfpathlineto{\pgfqpoint{5.337881in}{1.626276in}}%
\pgfpathlineto{\pgfqpoint{5.330134in}{1.614446in}}%
\pgfpathlineto{\pgfqpoint{5.322383in}{1.602658in}}%
\pgfpathlineto{\pgfqpoint{5.314628in}{1.590919in}}%
\pgfpathlineto{\pgfqpoint{5.300175in}{1.592991in}}%
\pgfpathlineto{\pgfqpoint{5.285730in}{1.595088in}}%
\pgfpathlineto{\pgfqpoint{5.271294in}{1.597208in}}%
\pgfpathlineto{\pgfqpoint{5.279053in}{1.608777in}}%
\pgfpathlineto{\pgfqpoint{5.286808in}{1.620397in}}%
\pgfpathlineto{\pgfqpoint{5.294559in}{1.632063in}}%
\pgfpathlineto{\pgfqpoint{5.302307in}{1.643769in}}%
\pgfpathclose%
\pgfusepath{fill}%
\end{pgfscope}%
\begin{pgfscope}%
\pgfpathrectangle{\pgfqpoint{1.150000in}{0.150000in}}{\pgfqpoint{5.700000in}{5.700000in}}%
\pgfusepath{clip}%
\pgfsetbuttcap%
\pgfsetroundjoin%
\definecolor{currentfill}{rgb}{0.175841,0.441290,0.557685}%
\pgfsetfillcolor{currentfill}%
\pgfsetfillopacity{0.700000}%
\pgfsetlinewidth{0.000000pt}%
\definecolor{currentstroke}{rgb}{0.000000,0.000000,0.000000}%
\pgfsetstrokecolor{currentstroke}%
\pgfsetdash{}{0pt}%
\pgfpathmoveto{\pgfqpoint{2.762896in}{2.429828in}}%
\pgfpathlineto{\pgfqpoint{2.776823in}{2.419632in}}%
\pgfpathlineto{\pgfqpoint{2.790752in}{2.409471in}}%
\pgfpathlineto{\pgfqpoint{2.804683in}{2.399346in}}%
\pgfpathlineto{\pgfqpoint{2.818617in}{2.389257in}}%
\pgfpathlineto{\pgfqpoint{2.809415in}{2.405611in}}%
\pgfpathlineto{\pgfqpoint{2.800183in}{2.422608in}}%
\pgfpathlineto{\pgfqpoint{2.790920in}{2.440261in}}%
\pgfpathlineto{\pgfqpoint{2.781625in}{2.458584in}}%
\pgfpathlineto{\pgfqpoint{2.767637in}{2.469102in}}%
\pgfpathlineto{\pgfqpoint{2.753652in}{2.479657in}}%
\pgfpathlineto{\pgfqpoint{2.739668in}{2.490247in}}%
\pgfpathlineto{\pgfqpoint{2.725687in}{2.500874in}}%
\pgfpathlineto{\pgfqpoint{2.735038in}{2.482114in}}%
\pgfpathlineto{\pgfqpoint{2.744356in}{2.464029in}}%
\pgfpathlineto{\pgfqpoint{2.753642in}{2.446604in}}%
\pgfpathlineto{\pgfqpoint{2.762896in}{2.429828in}}%
\pgfpathclose%
\pgfusepath{fill}%
\end{pgfscope}%
\begin{pgfscope}%
\pgfpathrectangle{\pgfqpoint{1.150000in}{0.150000in}}{\pgfqpoint{5.700000in}{5.700000in}}%
\pgfusepath{clip}%
\pgfsetbuttcap%
\pgfsetroundjoin%
\definecolor{currentfill}{rgb}{0.122312,0.633153,0.530398}%
\pgfsetfillcolor{currentfill}%
\pgfsetfillopacity{0.700000}%
\pgfsetlinewidth{0.000000pt}%
\definecolor{currentstroke}{rgb}{0.000000,0.000000,0.000000}%
\pgfsetstrokecolor{currentstroke}%
\pgfsetdash{}{0pt}%
\pgfpathmoveto{\pgfqpoint{2.167631in}{2.959189in}}%
\pgfpathlineto{\pgfqpoint{2.181566in}{2.946843in}}%
\pgfpathlineto{\pgfqpoint{2.195501in}{2.934547in}}%
\pgfpathlineto{\pgfqpoint{2.209437in}{2.922303in}}%
\pgfpathlineto{\pgfqpoint{2.223372in}{2.910108in}}%
\pgfpathlineto{\pgfqpoint{2.213436in}{2.933583in}}%
\pgfpathlineto{\pgfqpoint{2.203456in}{2.957803in}}%
\pgfpathlineto{\pgfqpoint{2.193432in}{2.982782in}}%
\pgfpathlineto{\pgfqpoint{2.183361in}{3.008536in}}%
\pgfpathlineto{\pgfqpoint{2.169358in}{3.021192in}}%
\pgfpathlineto{\pgfqpoint{2.155355in}{3.033899in}}%
\pgfpathlineto{\pgfqpoint{2.141352in}{3.046657in}}%
\pgfpathlineto{\pgfqpoint{2.127348in}{3.059467in}}%
\pgfpathlineto{\pgfqpoint{2.137489in}{3.033243in}}%
\pgfpathlineto{\pgfqpoint{2.147582in}{3.007798in}}%
\pgfpathlineto{\pgfqpoint{2.157629in}{2.983119in}}%
\pgfpathlineto{\pgfqpoint{2.167631in}{2.959189in}}%
\pgfpathclose%
\pgfusepath{fill}%
\end{pgfscope}%
\begin{pgfscope}%
\pgfpathrectangle{\pgfqpoint{1.150000in}{0.150000in}}{\pgfqpoint{5.700000in}{5.700000in}}%
\pgfusepath{clip}%
\pgfsetbuttcap%
\pgfsetroundjoin%
\definecolor{currentfill}{rgb}{0.283091,0.110553,0.431554}%
\pgfsetfillcolor{currentfill}%
\pgfsetfillopacity{0.700000}%
\pgfsetlinewidth{0.000000pt}%
\definecolor{currentstroke}{rgb}{0.000000,0.000000,0.000000}%
\pgfsetstrokecolor{currentstroke}%
\pgfsetdash{}{0pt}%
\pgfpathmoveto{\pgfqpoint{3.962668in}{1.670048in}}%
\pgfpathlineto{\pgfqpoint{3.976733in}{1.663611in}}%
\pgfpathlineto{\pgfqpoint{3.990802in}{1.657198in}}%
\pgfpathlineto{\pgfqpoint{4.004877in}{1.650811in}}%
\pgfpathlineto{\pgfqpoint{4.018957in}{1.644448in}}%
\pgfpathlineto{\pgfqpoint{4.010782in}{1.645418in}}%
\pgfpathlineto{\pgfqpoint{4.002598in}{1.646777in}}%
\pgfpathlineto{\pgfqpoint{3.994404in}{1.648536in}}%
\pgfpathlineto{\pgfqpoint{3.986200in}{1.650704in}}%
\pgfpathlineto{\pgfqpoint{3.972094in}{1.657422in}}%
\pgfpathlineto{\pgfqpoint{3.957992in}{1.664165in}}%
\pgfpathlineto{\pgfqpoint{3.943895in}{1.670933in}}%
\pgfpathlineto{\pgfqpoint{3.929804in}{1.677726in}}%
\pgfpathlineto{\pgfqpoint{3.938036in}{1.675197in}}%
\pgfpathlineto{\pgfqpoint{3.946257in}{1.673080in}}%
\pgfpathlineto{\pgfqpoint{3.954468in}{1.671367in}}%
\pgfpathlineto{\pgfqpoint{3.962668in}{1.670048in}}%
\pgfpathclose%
\pgfusepath{fill}%
\end{pgfscope}%
\begin{pgfscope}%
\pgfpathrectangle{\pgfqpoint{1.150000in}{0.150000in}}{\pgfqpoint{5.700000in}{5.700000in}}%
\pgfusepath{clip}%
\pgfsetbuttcap%
\pgfsetroundjoin%
\definecolor{currentfill}{rgb}{0.280267,0.073417,0.397163}%
\pgfsetfillcolor{currentfill}%
\pgfsetfillopacity{0.700000}%
\pgfsetlinewidth{0.000000pt}%
\definecolor{currentstroke}{rgb}{0.000000,0.000000,0.000000}%
\pgfsetstrokecolor{currentstroke}%
\pgfsetdash{}{0pt}%
\pgfpathmoveto{\pgfqpoint{4.164205in}{1.597009in}}%
\pgfpathlineto{\pgfqpoint{4.178311in}{1.591211in}}%
\pgfpathlineto{\pgfqpoint{4.192422in}{1.585437in}}%
\pgfpathlineto{\pgfqpoint{4.206540in}{1.579687in}}%
\pgfpathlineto{\pgfqpoint{4.220663in}{1.573963in}}%
\pgfpathlineto{\pgfqpoint{4.212594in}{1.572459in}}%
\pgfpathlineto{\pgfqpoint{4.204518in}{1.571297in}}%
\pgfpathlineto{\pgfqpoint{4.196433in}{1.570486in}}%
\pgfpathlineto{\pgfqpoint{4.188341in}{1.570035in}}%
\pgfpathlineto{\pgfqpoint{4.174195in}{1.576101in}}%
\pgfpathlineto{\pgfqpoint{4.160055in}{1.582191in}}%
\pgfpathlineto{\pgfqpoint{4.145920in}{1.588306in}}%
\pgfpathlineto{\pgfqpoint{4.131791in}{1.594445in}}%
\pgfpathlineto{\pgfqpoint{4.139907in}{1.594549in}}%
\pgfpathlineto{\pgfqpoint{4.148014in}{1.595018in}}%
\pgfpathlineto{\pgfqpoint{4.156113in}{1.595840in}}%
\pgfpathlineto{\pgfqpoint{4.164205in}{1.597009in}}%
\pgfpathclose%
\pgfusepath{fill}%
\end{pgfscope}%
\begin{pgfscope}%
\pgfpathrectangle{\pgfqpoint{1.150000in}{0.150000in}}{\pgfqpoint{5.700000in}{5.700000in}}%
\pgfusepath{clip}%
\pgfsetbuttcap%
\pgfsetroundjoin%
\definecolor{currentfill}{rgb}{0.231674,0.318106,0.544834}%
\pgfsetfillcolor{currentfill}%
\pgfsetfillopacity{0.700000}%
\pgfsetlinewidth{0.000000pt}%
\definecolor{currentstroke}{rgb}{0.000000,0.000000,0.000000}%
\pgfsetstrokecolor{currentstroke}%
\pgfsetdash{}{0pt}%
\pgfpathmoveto{\pgfqpoint{3.189193in}{2.107758in}}%
\pgfpathlineto{\pgfqpoint{3.203152in}{2.098893in}}%
\pgfpathlineto{\pgfqpoint{3.217113in}{2.090057in}}%
\pgfpathlineto{\pgfqpoint{3.231079in}{2.081252in}}%
\pgfpathlineto{\pgfqpoint{3.245047in}{2.072476in}}%
\pgfpathlineto{\pgfqpoint{3.236282in}{2.083564in}}%
\pgfpathlineto{\pgfqpoint{3.227495in}{2.095216in}}%
\pgfpathlineto{\pgfqpoint{3.218685in}{2.107444in}}%
\pgfpathlineto{\pgfqpoint{3.209853in}{2.120259in}}%
\pgfpathlineto{\pgfqpoint{3.195840in}{2.129441in}}%
\pgfpathlineto{\pgfqpoint{3.181830in}{2.138654in}}%
\pgfpathlineto{\pgfqpoint{3.167823in}{2.147896in}}%
\pgfpathlineto{\pgfqpoint{3.153820in}{2.157169in}}%
\pgfpathlineto{\pgfqpoint{3.162699in}{2.143940in}}%
\pgfpathlineto{\pgfqpoint{3.171553in}{2.131303in}}%
\pgfpathlineto{\pgfqpoint{3.180385in}{2.119247in}}%
\pgfpathlineto{\pgfqpoint{3.189193in}{2.107758in}}%
\pgfpathclose%
\pgfusepath{fill}%
\end{pgfscope}%
\begin{pgfscope}%
\pgfpathrectangle{\pgfqpoint{1.150000in}{0.150000in}}{\pgfqpoint{5.700000in}{5.700000in}}%
\pgfusepath{clip}%
\pgfsetbuttcap%
\pgfsetroundjoin%
\definecolor{currentfill}{rgb}{0.280267,0.073417,0.397163}%
\pgfsetfillcolor{currentfill}%
\pgfsetfillopacity{0.700000}%
\pgfsetlinewidth{0.000000pt}%
\definecolor{currentstroke}{rgb}{0.000000,0.000000,0.000000}%
\pgfsetstrokecolor{currentstroke}%
\pgfsetdash{}{0pt}%
\pgfpathmoveto{\pgfqpoint{5.213635in}{1.605928in}}%
\pgfpathlineto{\pgfqpoint{5.228037in}{1.603713in}}%
\pgfpathlineto{\pgfqpoint{5.242448in}{1.601521in}}%
\pgfpathlineto{\pgfqpoint{5.256867in}{1.599353in}}%
\pgfpathlineto{\pgfqpoint{5.271294in}{1.597208in}}%
\pgfpathlineto{\pgfqpoint{5.263532in}{1.585696in}}%
\pgfpathlineto{\pgfqpoint{5.255767in}{1.574246in}}%
\pgfpathlineto{\pgfqpoint{5.247998in}{1.562864in}}%
\pgfpathlineto{\pgfqpoint{5.240226in}{1.551556in}}%
\pgfpathlineto{\pgfqpoint{5.225793in}{1.553935in}}%
\pgfpathlineto{\pgfqpoint{5.211368in}{1.556338in}}%
\pgfpathlineto{\pgfqpoint{5.196952in}{1.558765in}}%
\pgfpathlineto{\pgfqpoint{5.182544in}{1.561216in}}%
\pgfpathlineto{\pgfqpoint{5.190322in}{1.572284in}}%
\pgfpathlineto{\pgfqpoint{5.198096in}{1.583429in}}%
\pgfpathlineto{\pgfqpoint{5.205867in}{1.594646in}}%
\pgfpathlineto{\pgfqpoint{5.213635in}{1.605928in}}%
\pgfpathclose%
\pgfusepath{fill}%
\end{pgfscope}%
\begin{pgfscope}%
\pgfpathrectangle{\pgfqpoint{1.150000in}{0.150000in}}{\pgfqpoint{5.700000in}{5.700000in}}%
\pgfusepath{clip}%
\pgfsetbuttcap%
\pgfsetroundjoin%
\definecolor{currentfill}{rgb}{0.273809,0.031497,0.358853}%
\pgfsetfillcolor{currentfill}%
\pgfsetfillopacity{0.700000}%
\pgfsetlinewidth{0.000000pt}%
\definecolor{currentstroke}{rgb}{0.000000,0.000000,0.000000}%
\pgfsetstrokecolor{currentstroke}%
\pgfsetdash{}{0pt}%
\pgfpathmoveto{\pgfqpoint{4.890542in}{1.526756in}}%
\pgfpathlineto{\pgfqpoint{4.904839in}{1.523417in}}%
\pgfpathlineto{\pgfqpoint{4.919143in}{1.520102in}}%
\pgfpathlineto{\pgfqpoint{4.933455in}{1.516811in}}%
\pgfpathlineto{\pgfqpoint{4.947775in}{1.513543in}}%
\pgfpathlineto{\pgfqpoint{4.939947in}{1.504254in}}%
\pgfpathlineto{\pgfqpoint{4.932116in}{1.495115in}}%
\pgfpathlineto{\pgfqpoint{4.924282in}{1.486133in}}%
\pgfpathlineto{\pgfqpoint{4.916444in}{1.477316in}}%
\pgfpathlineto{\pgfqpoint{4.902116in}{1.480858in}}%
\pgfpathlineto{\pgfqpoint{4.887794in}{1.484423in}}%
\pgfpathlineto{\pgfqpoint{4.873481in}{1.488012in}}%
\pgfpathlineto{\pgfqpoint{4.859174in}{1.491625in}}%
\pgfpathlineto{\pgfqpoint{4.867021in}{1.500163in}}%
\pgfpathlineto{\pgfqpoint{4.874865in}{1.508870in}}%
\pgfpathlineto{\pgfqpoint{4.882705in}{1.517736in}}%
\pgfpathlineto{\pgfqpoint{4.890542in}{1.526756in}}%
\pgfpathclose%
\pgfusepath{fill}%
\end{pgfscope}%
\begin{pgfscope}%
\pgfpathrectangle{\pgfqpoint{1.150000in}{0.150000in}}{\pgfqpoint{5.700000in}{5.700000in}}%
\pgfusepath{clip}%
\pgfsetbuttcap%
\pgfsetroundjoin%
\definecolor{currentfill}{rgb}{0.266580,0.228262,0.514349}%
\pgfsetfillcolor{currentfill}%
\pgfsetfillopacity{0.700000}%
\pgfsetlinewidth{0.000000pt}%
\definecolor{currentstroke}{rgb}{0.000000,0.000000,0.000000}%
\pgfsetstrokecolor{currentstroke}%
\pgfsetdash{}{0pt}%
\pgfpathmoveto{\pgfqpoint{3.503260in}{1.903296in}}%
\pgfpathlineto{\pgfqpoint{3.517254in}{1.895396in}}%
\pgfpathlineto{\pgfqpoint{3.531253in}{1.887523in}}%
\pgfpathlineto{\pgfqpoint{3.545255in}{1.879678in}}%
\pgfpathlineto{\pgfqpoint{3.559262in}{1.871859in}}%
\pgfpathlineto{\pgfqpoint{3.550770in}{1.878892in}}%
\pgfpathlineto{\pgfqpoint{3.542262in}{1.886424in}}%
\pgfpathlineto{\pgfqpoint{3.533737in}{1.894465in}}%
\pgfpathlineto{\pgfqpoint{3.525194in}{1.903026in}}%
\pgfpathlineto{\pgfqpoint{3.511150in}{1.911232in}}%
\pgfpathlineto{\pgfqpoint{3.497110in}{1.919465in}}%
\pgfpathlineto{\pgfqpoint{3.483074in}{1.927726in}}%
\pgfpathlineto{\pgfqpoint{3.469042in}{1.936015in}}%
\pgfpathlineto{\pgfqpoint{3.477623in}{1.927059in}}%
\pgfpathlineto{\pgfqpoint{3.486186in}{1.918628in}}%
\pgfpathlineto{\pgfqpoint{3.494732in}{1.910711in}}%
\pgfpathlineto{\pgfqpoint{3.503260in}{1.903296in}}%
\pgfpathclose%
\pgfusepath{fill}%
\end{pgfscope}%
\begin{pgfscope}%
\pgfpathrectangle{\pgfqpoint{1.150000in}{0.150000in}}{\pgfqpoint{5.700000in}{5.700000in}}%
\pgfusepath{clip}%
\pgfsetbuttcap%
\pgfsetroundjoin%
\definecolor{currentfill}{rgb}{0.280868,0.160771,0.472899}%
\pgfsetfillcolor{currentfill}%
\pgfsetfillopacity{0.700000}%
\pgfsetlinewidth{0.000000pt}%
\definecolor{currentstroke}{rgb}{0.000000,0.000000,0.000000}%
\pgfsetstrokecolor{currentstroke}%
\pgfsetdash{}{0pt}%
\pgfpathmoveto{\pgfqpoint{3.761095in}{1.761233in}}%
\pgfpathlineto{\pgfqpoint{3.775128in}{1.754133in}}%
\pgfpathlineto{\pgfqpoint{3.789165in}{1.747058in}}%
\pgfpathlineto{\pgfqpoint{3.803206in}{1.740009in}}%
\pgfpathlineto{\pgfqpoint{3.817253in}{1.732987in}}%
\pgfpathlineto{\pgfqpoint{3.808951in}{1.736670in}}%
\pgfpathlineto{\pgfqpoint{3.800636in}{1.740793in}}%
\pgfpathlineto{\pgfqpoint{3.792309in}{1.745368in}}%
\pgfpathlineto{\pgfqpoint{3.783968in}{1.750405in}}%
\pgfpathlineto{\pgfqpoint{3.769890in}{1.757798in}}%
\pgfpathlineto{\pgfqpoint{3.755817in}{1.765218in}}%
\pgfpathlineto{\pgfqpoint{3.741748in}{1.772664in}}%
\pgfpathlineto{\pgfqpoint{3.727684in}{1.780136in}}%
\pgfpathlineto{\pgfqpoint{3.736057in}{1.774723in}}%
\pgfpathlineto{\pgfqpoint{3.744416in}{1.769775in}}%
\pgfpathlineto{\pgfqpoint{3.752762in}{1.765282in}}%
\pgfpathlineto{\pgfqpoint{3.761095in}{1.761233in}}%
\pgfpathclose%
\pgfusepath{fill}%
\end{pgfscope}%
\begin{pgfscope}%
\pgfpathrectangle{\pgfqpoint{1.150000in}{0.150000in}}{\pgfqpoint{5.700000in}{5.700000in}}%
\pgfusepath{clip}%
\pgfsetbuttcap%
\pgfsetroundjoin%
\definecolor{currentfill}{rgb}{0.119699,0.618490,0.536347}%
\pgfsetfillcolor{currentfill}%
\pgfsetfillopacity{0.700000}%
\pgfsetlinewidth{0.000000pt}%
\definecolor{currentstroke}{rgb}{0.000000,0.000000,0.000000}%
\pgfsetstrokecolor{currentstroke}%
\pgfsetdash{}{0pt}%
\pgfpathmoveto{\pgfqpoint{2.223372in}{2.910108in}}%
\pgfpathlineto{\pgfqpoint{2.237309in}{2.897963in}}%
\pgfpathlineto{\pgfqpoint{2.251245in}{2.885868in}}%
\pgfpathlineto{\pgfqpoint{2.265182in}{2.873821in}}%
\pgfpathlineto{\pgfqpoint{2.279119in}{2.861822in}}%
\pgfpathlineto{\pgfqpoint{2.269249in}{2.884844in}}%
\pgfpathlineto{\pgfqpoint{2.259335in}{2.908605in}}%
\pgfpathlineto{\pgfqpoint{2.249377in}{2.933122in}}%
\pgfpathlineto{\pgfqpoint{2.239375in}{2.958406in}}%
\pgfpathlineto{\pgfqpoint{2.225371in}{2.970865in}}%
\pgfpathlineto{\pgfqpoint{2.211368in}{2.983373in}}%
\pgfpathlineto{\pgfqpoint{2.197364in}{2.995929in}}%
\pgfpathlineto{\pgfqpoint{2.183361in}{3.008536in}}%
\pgfpathlineto{\pgfqpoint{2.193432in}{2.982782in}}%
\pgfpathlineto{\pgfqpoint{2.203456in}{2.957803in}}%
\pgfpathlineto{\pgfqpoint{2.213436in}{2.933583in}}%
\pgfpathlineto{\pgfqpoint{2.223372in}{2.910108in}}%
\pgfpathclose%
\pgfusepath{fill}%
\end{pgfscope}%
\begin{pgfscope}%
\pgfpathrectangle{\pgfqpoint{1.150000in}{0.150000in}}{\pgfqpoint{5.700000in}{5.700000in}}%
\pgfusepath{clip}%
\pgfsetbuttcap%
\pgfsetroundjoin%
\definecolor{currentfill}{rgb}{0.180629,0.429975,0.557282}%
\pgfsetfillcolor{currentfill}%
\pgfsetfillopacity{0.700000}%
\pgfsetlinewidth{0.000000pt}%
\definecolor{currentstroke}{rgb}{0.000000,0.000000,0.000000}%
\pgfsetstrokecolor{currentstroke}%
\pgfsetdash{}{0pt}%
\pgfpathmoveto{\pgfqpoint{2.818617in}{2.389257in}}%
\pgfpathlineto{\pgfqpoint{2.832553in}{2.379203in}}%
\pgfpathlineto{\pgfqpoint{2.846492in}{2.369184in}}%
\pgfpathlineto{\pgfqpoint{2.860433in}{2.359200in}}%
\pgfpathlineto{\pgfqpoint{2.874376in}{2.349250in}}%
\pgfpathlineto{\pgfqpoint{2.865227in}{2.365181in}}%
\pgfpathlineto{\pgfqpoint{2.856047in}{2.381751in}}%
\pgfpathlineto{\pgfqpoint{2.846838in}{2.398973in}}%
\pgfpathlineto{\pgfqpoint{2.837597in}{2.416860in}}%
\pgfpathlineto{\pgfqpoint{2.823601in}{2.427239in}}%
\pgfpathlineto{\pgfqpoint{2.809607in}{2.437652in}}%
\pgfpathlineto{\pgfqpoint{2.795615in}{2.448100in}}%
\pgfpathlineto{\pgfqpoint{2.781625in}{2.458584in}}%
\pgfpathlineto{\pgfqpoint{2.790920in}{2.440261in}}%
\pgfpathlineto{\pgfqpoint{2.800183in}{2.422608in}}%
\pgfpathlineto{\pgfqpoint{2.809415in}{2.405611in}}%
\pgfpathlineto{\pgfqpoint{2.818617in}{2.389257in}}%
\pgfpathclose%
\pgfusepath{fill}%
\end{pgfscope}%
\begin{pgfscope}%
\pgfpathrectangle{\pgfqpoint{1.150000in}{0.150000in}}{\pgfqpoint{5.700000in}{5.700000in}}%
\pgfusepath{clip}%
\pgfsetbuttcap%
\pgfsetroundjoin%
\definecolor{currentfill}{rgb}{0.273809,0.031497,0.358853}%
\pgfsetfillcolor{currentfill}%
\pgfsetfillopacity{0.700000}%
\pgfsetlinewidth{0.000000pt}%
\definecolor{currentstroke}{rgb}{0.000000,0.000000,0.000000}%
\pgfsetstrokecolor{currentstroke}%
\pgfsetdash{}{0pt}%
\pgfpathmoveto{\pgfqpoint{4.511161in}{1.519190in}}%
\pgfpathlineto{\pgfqpoint{4.525353in}{1.514516in}}%
\pgfpathlineto{\pgfqpoint{4.539552in}{1.509865in}}%
\pgfpathlineto{\pgfqpoint{4.553757in}{1.505239in}}%
\pgfpathlineto{\pgfqpoint{4.567969in}{1.500636in}}%
\pgfpathlineto{\pgfqpoint{4.560038in}{1.495169in}}%
\pgfpathlineto{\pgfqpoint{4.552102in}{1.489959in}}%
\pgfpathlineto{\pgfqpoint{4.544162in}{1.485013in}}%
\pgfpathlineto{\pgfqpoint{4.536217in}{1.480340in}}%
\pgfpathlineto{\pgfqpoint{4.521990in}{1.485257in}}%
\pgfpathlineto{\pgfqpoint{4.507768in}{1.490197in}}%
\pgfpathlineto{\pgfqpoint{4.493554in}{1.495161in}}%
\pgfpathlineto{\pgfqpoint{4.479346in}{1.500148in}}%
\pgfpathlineto{\pgfqpoint{4.487307in}{1.504503in}}%
\pgfpathlineto{\pgfqpoint{4.495263in}{1.509134in}}%
\pgfpathlineto{\pgfqpoint{4.503215in}{1.514032in}}%
\pgfpathlineto{\pgfqpoint{4.511161in}{1.519190in}}%
\pgfpathclose%
\pgfusepath{fill}%
\end{pgfscope}%
\begin{pgfscope}%
\pgfpathrectangle{\pgfqpoint{1.150000in}{0.150000in}}{\pgfqpoint{5.700000in}{5.700000in}}%
\pgfusepath{clip}%
\pgfsetbuttcap%
\pgfsetroundjoin%
\definecolor{currentfill}{rgb}{0.277941,0.056324,0.381191}%
\pgfsetfillcolor{currentfill}%
\pgfsetfillopacity{0.700000}%
\pgfsetlinewidth{0.000000pt}%
\definecolor{currentstroke}{rgb}{0.000000,0.000000,0.000000}%
\pgfsetstrokecolor{currentstroke}%
\pgfsetdash{}{0pt}%
\pgfpathmoveto{\pgfqpoint{5.124994in}{1.571256in}}%
\pgfpathlineto{\pgfqpoint{5.139369in}{1.568710in}}%
\pgfpathlineto{\pgfqpoint{5.153752in}{1.566188in}}%
\pgfpathlineto{\pgfqpoint{5.168144in}{1.563690in}}%
\pgfpathlineto{\pgfqpoint{5.182544in}{1.561216in}}%
\pgfpathlineto{\pgfqpoint{5.174763in}{1.550230in}}%
\pgfpathlineto{\pgfqpoint{5.166979in}{1.539334in}}%
\pgfpathlineto{\pgfqpoint{5.159192in}{1.528533in}}%
\pgfpathlineto{\pgfqpoint{5.151402in}{1.517833in}}%
\pgfpathlineto{\pgfqpoint{5.136995in}{1.520556in}}%
\pgfpathlineto{\pgfqpoint{5.122597in}{1.523302in}}%
\pgfpathlineto{\pgfqpoint{5.108207in}{1.526072in}}%
\pgfpathlineto{\pgfqpoint{5.093825in}{1.528866in}}%
\pgfpathlineto{\pgfqpoint{5.101622in}{1.539313in}}%
\pgfpathlineto{\pgfqpoint{5.109416in}{1.549864in}}%
\pgfpathlineto{\pgfqpoint{5.117206in}{1.560514in}}%
\pgfpathlineto{\pgfqpoint{5.124994in}{1.571256in}}%
\pgfpathclose%
\pgfusepath{fill}%
\end{pgfscope}%
\begin{pgfscope}%
\pgfpathrectangle{\pgfqpoint{1.150000in}{0.150000in}}{\pgfqpoint{5.700000in}{5.700000in}}%
\pgfusepath{clip}%
\pgfsetbuttcap%
\pgfsetroundjoin%
\definecolor{currentfill}{rgb}{0.272594,0.025563,0.353093}%
\pgfsetfillcolor{currentfill}%
\pgfsetfillopacity{0.700000}%
\pgfsetlinewidth{0.000000pt}%
\definecolor{currentstroke}{rgb}{0.000000,0.000000,0.000000}%
\pgfsetstrokecolor{currentstroke}%
\pgfsetdash{}{0pt}%
\pgfpathmoveto{\pgfqpoint{4.656507in}{1.507932in}}%
\pgfpathlineto{\pgfqpoint{4.670739in}{1.503748in}}%
\pgfpathlineto{\pgfqpoint{4.684977in}{1.499588in}}%
\pgfpathlineto{\pgfqpoint{4.699223in}{1.495452in}}%
\pgfpathlineto{\pgfqpoint{4.713476in}{1.491339in}}%
\pgfpathlineto{\pgfqpoint{4.705589in}{1.484336in}}%
\pgfpathlineto{\pgfqpoint{4.697698in}{1.477552in}}%
\pgfpathlineto{\pgfqpoint{4.689804in}{1.470994in}}%
\pgfpathlineto{\pgfqpoint{4.681906in}{1.464669in}}%
\pgfpathlineto{\pgfqpoint{4.667640in}{1.469082in}}%
\pgfpathlineto{\pgfqpoint{4.653381in}{1.473518in}}%
\pgfpathlineto{\pgfqpoint{4.639129in}{1.477979in}}%
\pgfpathlineto{\pgfqpoint{4.624883in}{1.482463in}}%
\pgfpathlineto{\pgfqpoint{4.632795in}{1.488482in}}%
\pgfpathlineto{\pgfqpoint{4.640703in}{1.494739in}}%
\pgfpathlineto{\pgfqpoint{4.648607in}{1.501225in}}%
\pgfpathlineto{\pgfqpoint{4.656507in}{1.507932in}}%
\pgfpathclose%
\pgfusepath{fill}%
\end{pgfscope}%
\begin{pgfscope}%
\pgfpathrectangle{\pgfqpoint{1.150000in}{0.150000in}}{\pgfqpoint{5.700000in}{5.700000in}}%
\pgfusepath{clip}%
\pgfsetbuttcap%
\pgfsetroundjoin%
\definecolor{currentfill}{rgb}{0.276022,0.044167,0.370164}%
\pgfsetfillcolor{currentfill}%
\pgfsetfillopacity{0.700000}%
\pgfsetlinewidth{0.000000pt}%
\definecolor{currentstroke}{rgb}{0.000000,0.000000,0.000000}%
\pgfsetstrokecolor{currentstroke}%
\pgfsetdash{}{0pt}%
\pgfpathmoveto{\pgfqpoint{4.365909in}{1.540913in}}%
\pgfpathlineto{\pgfqpoint{4.380066in}{1.535733in}}%
\pgfpathlineto{\pgfqpoint{4.394230in}{1.530577in}}%
\pgfpathlineto{\pgfqpoint{4.408400in}{1.525446in}}%
\pgfpathlineto{\pgfqpoint{4.422577in}{1.520338in}}%
\pgfpathlineto{\pgfqpoint{4.414593in}{1.516590in}}%
\pgfpathlineto{\pgfqpoint{4.406603in}{1.513138in}}%
\pgfpathlineto{\pgfqpoint{4.398607in}{1.509991in}}%
\pgfpathlineto{\pgfqpoint{4.390606in}{1.507157in}}%
\pgfpathlineto{\pgfqpoint{4.376411in}{1.512592in}}%
\pgfpathlineto{\pgfqpoint{4.362222in}{1.518050in}}%
\pgfpathlineto{\pgfqpoint{4.348039in}{1.523533in}}%
\pgfpathlineto{\pgfqpoint{4.333863in}{1.529039in}}%
\pgfpathlineto{\pgfqpoint{4.341883in}{1.531541in}}%
\pgfpathlineto{\pgfqpoint{4.349898in}{1.534359in}}%
\pgfpathlineto{\pgfqpoint{4.357906in}{1.537486in}}%
\pgfpathlineto{\pgfqpoint{4.365909in}{1.540913in}}%
\pgfpathclose%
\pgfusepath{fill}%
\end{pgfscope}%
\begin{pgfscope}%
\pgfpathrectangle{\pgfqpoint{1.150000in}{0.150000in}}{\pgfqpoint{5.700000in}{5.700000in}}%
\pgfusepath{clip}%
\pgfsetbuttcap%
\pgfsetroundjoin%
\definecolor{currentfill}{rgb}{0.120092,0.600104,0.542530}%
\pgfsetfillcolor{currentfill}%
\pgfsetfillopacity{0.700000}%
\pgfsetlinewidth{0.000000pt}%
\definecolor{currentstroke}{rgb}{0.000000,0.000000,0.000000}%
\pgfsetstrokecolor{currentstroke}%
\pgfsetdash{}{0pt}%
\pgfpathmoveto{\pgfqpoint{2.279119in}{2.861822in}}%
\pgfpathlineto{\pgfqpoint{2.293058in}{2.849871in}}%
\pgfpathlineto{\pgfqpoint{2.306996in}{2.837968in}}%
\pgfpathlineto{\pgfqpoint{2.320936in}{2.826111in}}%
\pgfpathlineto{\pgfqpoint{2.334876in}{2.814301in}}%
\pgfpathlineto{\pgfqpoint{2.325069in}{2.836871in}}%
\pgfpathlineto{\pgfqpoint{2.315220in}{2.860176in}}%
\pgfpathlineto{\pgfqpoint{2.305329in}{2.884230in}}%
\pgfpathlineto{\pgfqpoint{2.295395in}{2.909048in}}%
\pgfpathlineto{\pgfqpoint{2.281389in}{2.921317in}}%
\pgfpathlineto{\pgfqpoint{2.267384in}{2.933633in}}%
\pgfpathlineto{\pgfqpoint{2.253380in}{2.945996in}}%
\pgfpathlineto{\pgfqpoint{2.239375in}{2.958406in}}%
\pgfpathlineto{\pgfqpoint{2.249377in}{2.933122in}}%
\pgfpathlineto{\pgfqpoint{2.259335in}{2.908605in}}%
\pgfpathlineto{\pgfqpoint{2.269249in}{2.884844in}}%
\pgfpathlineto{\pgfqpoint{2.279119in}{2.861822in}}%
\pgfpathclose%
\pgfusepath{fill}%
\end{pgfscope}%
\begin{pgfscope}%
\pgfpathrectangle{\pgfqpoint{1.150000in}{0.150000in}}{\pgfqpoint{5.700000in}{5.700000in}}%
\pgfusepath{clip}%
\pgfsetbuttcap%
\pgfsetroundjoin%
\definecolor{currentfill}{rgb}{0.282910,0.105393,0.426902}%
\pgfsetfillcolor{currentfill}%
\pgfsetfillopacity{0.700000}%
\pgfsetlinewidth{0.000000pt}%
\definecolor{currentstroke}{rgb}{0.000000,0.000000,0.000000}%
\pgfsetstrokecolor{currentstroke}%
\pgfsetdash{}{0pt}%
\pgfpathmoveto{\pgfqpoint{4.018957in}{1.644448in}}%
\pgfpathlineto{\pgfqpoint{4.033042in}{1.638111in}}%
\pgfpathlineto{\pgfqpoint{4.047133in}{1.631799in}}%
\pgfpathlineto{\pgfqpoint{4.061229in}{1.625511in}}%
\pgfpathlineto{\pgfqpoint{4.075330in}{1.619249in}}%
\pgfpathlineto{\pgfqpoint{4.067182in}{1.619868in}}%
\pgfpathlineto{\pgfqpoint{4.059024in}{1.620873in}}%
\pgfpathlineto{\pgfqpoint{4.050856in}{1.622275in}}%
\pgfpathlineto{\pgfqpoint{4.042679in}{1.624082in}}%
\pgfpathlineto{\pgfqpoint{4.028552in}{1.630700in}}%
\pgfpathlineto{\pgfqpoint{4.014429in}{1.637343in}}%
\pgfpathlineto{\pgfqpoint{4.000312in}{1.644011in}}%
\pgfpathlineto{\pgfqpoint{3.986200in}{1.650704in}}%
\pgfpathlineto{\pgfqpoint{3.994404in}{1.648536in}}%
\pgfpathlineto{\pgfqpoint{4.002598in}{1.646777in}}%
\pgfpathlineto{\pgfqpoint{4.010782in}{1.645418in}}%
\pgfpathlineto{\pgfqpoint{4.018957in}{1.644448in}}%
\pgfpathclose%
\pgfusepath{fill}%
\end{pgfscope}%
\begin{pgfscope}%
\pgfpathrectangle{\pgfqpoint{1.150000in}{0.150000in}}{\pgfqpoint{5.700000in}{5.700000in}}%
\pgfusepath{clip}%
\pgfsetbuttcap%
\pgfsetroundjoin%
\definecolor{currentfill}{rgb}{0.235526,0.309527,0.542944}%
\pgfsetfillcolor{currentfill}%
\pgfsetfillopacity{0.700000}%
\pgfsetlinewidth{0.000000pt}%
\definecolor{currentstroke}{rgb}{0.000000,0.000000,0.000000}%
\pgfsetstrokecolor{currentstroke}%
\pgfsetdash{}{0pt}%
\pgfpathmoveto{\pgfqpoint{3.245047in}{2.072476in}}%
\pgfpathlineto{\pgfqpoint{3.259020in}{2.063730in}}%
\pgfpathlineto{\pgfqpoint{3.272995in}{2.055014in}}%
\pgfpathlineto{\pgfqpoint{3.286975in}{2.046327in}}%
\pgfpathlineto{\pgfqpoint{3.300958in}{2.037669in}}%
\pgfpathlineto{\pgfqpoint{3.292235in}{2.048357in}}%
\pgfpathlineto{\pgfqpoint{3.283491in}{2.059605in}}%
\pgfpathlineto{\pgfqpoint{3.274726in}{2.071424in}}%
\pgfpathlineto{\pgfqpoint{3.265938in}{2.083827in}}%
\pgfpathlineto{\pgfqpoint{3.251912in}{2.092891in}}%
\pgfpathlineto{\pgfqpoint{3.237889in}{2.101984in}}%
\pgfpathlineto{\pgfqpoint{3.223869in}{2.111107in}}%
\pgfpathlineto{\pgfqpoint{3.209853in}{2.120259in}}%
\pgfpathlineto{\pgfqpoint{3.218685in}{2.107444in}}%
\pgfpathlineto{\pgfqpoint{3.227495in}{2.095216in}}%
\pgfpathlineto{\pgfqpoint{3.236282in}{2.083564in}}%
\pgfpathlineto{\pgfqpoint{3.245047in}{2.072476in}}%
\pgfpathclose%
\pgfusepath{fill}%
\end{pgfscope}%
\begin{pgfscope}%
\pgfpathrectangle{\pgfqpoint{1.150000in}{0.150000in}}{\pgfqpoint{5.700000in}{5.700000in}}%
\pgfusepath{clip}%
\pgfsetbuttcap%
\pgfsetroundjoin%
\definecolor{currentfill}{rgb}{0.272594,0.025563,0.353093}%
\pgfsetfillcolor{currentfill}%
\pgfsetfillopacity{0.700000}%
\pgfsetlinewidth{0.000000pt}%
\definecolor{currentstroke}{rgb}{0.000000,0.000000,0.000000}%
\pgfsetstrokecolor{currentstroke}%
\pgfsetdash{}{0pt}%
\pgfpathmoveto{\pgfqpoint{4.802022in}{1.506313in}}%
\pgfpathlineto{\pgfqpoint{4.816299in}{1.502605in}}%
\pgfpathlineto{\pgfqpoint{4.830583in}{1.498921in}}%
\pgfpathlineto{\pgfqpoint{4.844875in}{1.495261in}}%
\pgfpathlineto{\pgfqpoint{4.859174in}{1.491625in}}%
\pgfpathlineto{\pgfqpoint{4.851324in}{1.483261in}}%
\pgfpathlineto{\pgfqpoint{4.843470in}{1.475079in}}%
\pgfpathlineto{\pgfqpoint{4.835614in}{1.467086in}}%
\pgfpathlineto{\pgfqpoint{4.827754in}{1.459289in}}%
\pgfpathlineto{\pgfqpoint{4.813444in}{1.463212in}}%
\pgfpathlineto{\pgfqpoint{4.799142in}{1.467159in}}%
\pgfpathlineto{\pgfqpoint{4.784846in}{1.471130in}}%
\pgfpathlineto{\pgfqpoint{4.770558in}{1.475125in}}%
\pgfpathlineto{\pgfqpoint{4.778429in}{1.482629in}}%
\pgfpathlineto{\pgfqpoint{4.786297in}{1.490334in}}%
\pgfpathlineto{\pgfqpoint{4.794161in}{1.498231in}}%
\pgfpathlineto{\pgfqpoint{4.802022in}{1.506313in}}%
\pgfpathclose%
\pgfusepath{fill}%
\end{pgfscope}%
\begin{pgfscope}%
\pgfpathrectangle{\pgfqpoint{1.150000in}{0.150000in}}{\pgfqpoint{5.700000in}{5.700000in}}%
\pgfusepath{clip}%
\pgfsetbuttcap%
\pgfsetroundjoin%
\definecolor{currentfill}{rgb}{0.276022,0.044167,0.370164}%
\pgfsetfillcolor{currentfill}%
\pgfsetfillopacity{0.700000}%
\pgfsetlinewidth{0.000000pt}%
\definecolor{currentstroke}{rgb}{0.000000,0.000000,0.000000}%
\pgfsetstrokecolor{currentstroke}%
\pgfsetdash{}{0pt}%
\pgfpathmoveto{\pgfqpoint{5.036377in}{1.540278in}}%
\pgfpathlineto{\pgfqpoint{5.050727in}{1.537389in}}%
\pgfpathlineto{\pgfqpoint{5.065085in}{1.534524in}}%
\pgfpathlineto{\pgfqpoint{5.079451in}{1.531683in}}%
\pgfpathlineto{\pgfqpoint{5.093825in}{1.528866in}}%
\pgfpathlineto{\pgfqpoint{5.086025in}{1.518530in}}%
\pgfpathlineto{\pgfqpoint{5.078223in}{1.508311in}}%
\pgfpathlineto{\pgfqpoint{5.070417in}{1.498216in}}%
\pgfpathlineto{\pgfqpoint{5.062608in}{1.488251in}}%
\pgfpathlineto{\pgfqpoint{5.048227in}{1.491330in}}%
\pgfpathlineto{\pgfqpoint{5.033853in}{1.494432in}}%
\pgfpathlineto{\pgfqpoint{5.019488in}{1.497558in}}%
\pgfpathlineto{\pgfqpoint{5.005130in}{1.500708in}}%
\pgfpathlineto{\pgfqpoint{5.012946in}{1.510407in}}%
\pgfpathlineto{\pgfqpoint{5.020759in}{1.520239in}}%
\pgfpathlineto{\pgfqpoint{5.028570in}{1.530198in}}%
\pgfpathlineto{\pgfqpoint{5.036377in}{1.540278in}}%
\pgfpathclose%
\pgfusepath{fill}%
\end{pgfscope}%
\begin{pgfscope}%
\pgfpathrectangle{\pgfqpoint{1.150000in}{0.150000in}}{\pgfqpoint{5.700000in}{5.700000in}}%
\pgfusepath{clip}%
\pgfsetbuttcap%
\pgfsetroundjoin%
\definecolor{currentfill}{rgb}{0.185556,0.418570,0.556753}%
\pgfsetfillcolor{currentfill}%
\pgfsetfillopacity{0.700000}%
\pgfsetlinewidth{0.000000pt}%
\definecolor{currentstroke}{rgb}{0.000000,0.000000,0.000000}%
\pgfsetstrokecolor{currentstroke}%
\pgfsetdash{}{0pt}%
\pgfpathmoveto{\pgfqpoint{2.874376in}{2.349250in}}%
\pgfpathlineto{\pgfqpoint{2.888322in}{2.339334in}}%
\pgfpathlineto{\pgfqpoint{2.902271in}{2.329452in}}%
\pgfpathlineto{\pgfqpoint{2.916222in}{2.319604in}}%
\pgfpathlineto{\pgfqpoint{2.930176in}{2.309790in}}%
\pgfpathlineto{\pgfqpoint{2.921077in}{2.325300in}}%
\pgfpathlineto{\pgfqpoint{2.911950in}{2.341445in}}%
\pgfpathlineto{\pgfqpoint{2.902793in}{2.358237in}}%
\pgfpathlineto{\pgfqpoint{2.893607in}{2.375689in}}%
\pgfpathlineto{\pgfqpoint{2.879601in}{2.385931in}}%
\pgfpathlineto{\pgfqpoint{2.865597in}{2.396206in}}%
\pgfpathlineto{\pgfqpoint{2.851596in}{2.406516in}}%
\pgfpathlineto{\pgfqpoint{2.837597in}{2.416860in}}%
\pgfpathlineto{\pgfqpoint{2.846838in}{2.398973in}}%
\pgfpathlineto{\pgfqpoint{2.856047in}{2.381751in}}%
\pgfpathlineto{\pgfqpoint{2.865227in}{2.365181in}}%
\pgfpathlineto{\pgfqpoint{2.874376in}{2.349250in}}%
\pgfpathclose%
\pgfusepath{fill}%
\end{pgfscope}%
\begin{pgfscope}%
\pgfpathrectangle{\pgfqpoint{1.150000in}{0.150000in}}{\pgfqpoint{5.700000in}{5.700000in}}%
\pgfusepath{clip}%
\pgfsetbuttcap%
\pgfsetroundjoin%
\definecolor{currentfill}{rgb}{0.269308,0.218818,0.509577}%
\pgfsetfillcolor{currentfill}%
\pgfsetfillopacity{0.700000}%
\pgfsetlinewidth{0.000000pt}%
\definecolor{currentstroke}{rgb}{0.000000,0.000000,0.000000}%
\pgfsetstrokecolor{currentstroke}%
\pgfsetdash{}{0pt}%
\pgfpathmoveto{\pgfqpoint{3.559262in}{1.871859in}}%
\pgfpathlineto{\pgfqpoint{3.573273in}{1.864068in}}%
\pgfpathlineto{\pgfqpoint{3.587289in}{1.856305in}}%
\pgfpathlineto{\pgfqpoint{3.601308in}{1.848568in}}%
\pgfpathlineto{\pgfqpoint{3.615332in}{1.840858in}}%
\pgfpathlineto{\pgfqpoint{3.606876in}{1.847509in}}%
\pgfpathlineto{\pgfqpoint{3.598404in}{1.854655in}}%
\pgfpathlineto{\pgfqpoint{3.589916in}{1.862306in}}%
\pgfpathlineto{\pgfqpoint{3.581411in}{1.870474in}}%
\pgfpathlineto{\pgfqpoint{3.567351in}{1.878572in}}%
\pgfpathlineto{\pgfqpoint{3.553294in}{1.886696in}}%
\pgfpathlineto{\pgfqpoint{3.539242in}{1.894847in}}%
\pgfpathlineto{\pgfqpoint{3.525194in}{1.903026in}}%
\pgfpathlineto{\pgfqpoint{3.533737in}{1.894465in}}%
\pgfpathlineto{\pgfqpoint{3.542262in}{1.886424in}}%
\pgfpathlineto{\pgfqpoint{3.550770in}{1.878892in}}%
\pgfpathlineto{\pgfqpoint{3.559262in}{1.871859in}}%
\pgfpathclose%
\pgfusepath{fill}%
\end{pgfscope}%
\begin{pgfscope}%
\pgfpathrectangle{\pgfqpoint{1.150000in}{0.150000in}}{\pgfqpoint{5.700000in}{5.700000in}}%
\pgfusepath{clip}%
\pgfsetbuttcap%
\pgfsetroundjoin%
\definecolor{currentfill}{rgb}{0.122606,0.585371,0.546557}%
\pgfsetfillcolor{currentfill}%
\pgfsetfillopacity{0.700000}%
\pgfsetlinewidth{0.000000pt}%
\definecolor{currentstroke}{rgb}{0.000000,0.000000,0.000000}%
\pgfsetstrokecolor{currentstroke}%
\pgfsetdash{}{0pt}%
\pgfpathmoveto{\pgfqpoint{2.334876in}{2.814301in}}%
\pgfpathlineto{\pgfqpoint{2.348816in}{2.802536in}}%
\pgfpathlineto{\pgfqpoint{2.362758in}{2.790817in}}%
\pgfpathlineto{\pgfqpoint{2.376701in}{2.779144in}}%
\pgfpathlineto{\pgfqpoint{2.390644in}{2.767515in}}%
\pgfpathlineto{\pgfqpoint{2.380901in}{2.789635in}}%
\pgfpathlineto{\pgfqpoint{2.371116in}{2.812485in}}%
\pgfpathlineto{\pgfqpoint{2.361291in}{2.836079in}}%
\pgfpathlineto{\pgfqpoint{2.351422in}{2.860432in}}%
\pgfpathlineto{\pgfqpoint{2.337415in}{2.872518in}}%
\pgfpathlineto{\pgfqpoint{2.323407in}{2.884649in}}%
\pgfpathlineto{\pgfqpoint{2.309401in}{2.896826in}}%
\pgfpathlineto{\pgfqpoint{2.295395in}{2.909048in}}%
\pgfpathlineto{\pgfqpoint{2.305329in}{2.884230in}}%
\pgfpathlineto{\pgfqpoint{2.315220in}{2.860176in}}%
\pgfpathlineto{\pgfqpoint{2.325069in}{2.836871in}}%
\pgfpathlineto{\pgfqpoint{2.334876in}{2.814301in}}%
\pgfpathclose%
\pgfusepath{fill}%
\end{pgfscope}%
\begin{pgfscope}%
\pgfpathrectangle{\pgfqpoint{1.150000in}{0.150000in}}{\pgfqpoint{5.700000in}{5.700000in}}%
\pgfusepath{clip}%
\pgfsetbuttcap%
\pgfsetroundjoin%
\definecolor{currentfill}{rgb}{0.280267,0.073417,0.397163}%
\pgfsetfillcolor{currentfill}%
\pgfsetfillopacity{0.700000}%
\pgfsetlinewidth{0.000000pt}%
\definecolor{currentstroke}{rgb}{0.000000,0.000000,0.000000}%
\pgfsetstrokecolor{currentstroke}%
\pgfsetdash{}{0pt}%
\pgfpathmoveto{\pgfqpoint{4.220663in}{1.573963in}}%
\pgfpathlineto{\pgfqpoint{4.234793in}{1.568262in}}%
\pgfpathlineto{\pgfqpoint{4.248928in}{1.562586in}}%
\pgfpathlineto{\pgfqpoint{4.263069in}{1.556934in}}%
\pgfpathlineto{\pgfqpoint{4.277216in}{1.551307in}}%
\pgfpathlineto{\pgfqpoint{4.269168in}{1.549467in}}%
\pgfpathlineto{\pgfqpoint{4.261113in}{1.547966in}}%
\pgfpathlineto{\pgfqpoint{4.253051in}{1.546813in}}%
\pgfpathlineto{\pgfqpoint{4.244982in}{1.546016in}}%
\pgfpathlineto{\pgfqpoint{4.230813in}{1.551984in}}%
\pgfpathlineto{\pgfqpoint{4.216650in}{1.557977in}}%
\pgfpathlineto{\pgfqpoint{4.202493in}{1.563994in}}%
\pgfpathlineto{\pgfqpoint{4.188341in}{1.570035in}}%
\pgfpathlineto{\pgfqpoint{4.196433in}{1.570486in}}%
\pgfpathlineto{\pgfqpoint{4.204518in}{1.571297in}}%
\pgfpathlineto{\pgfqpoint{4.212594in}{1.572459in}}%
\pgfpathlineto{\pgfqpoint{4.220663in}{1.573963in}}%
\pgfpathclose%
\pgfusepath{fill}%
\end{pgfscope}%
\begin{pgfscope}%
\pgfpathrectangle{\pgfqpoint{1.150000in}{0.150000in}}{\pgfqpoint{5.700000in}{5.700000in}}%
\pgfusepath{clip}%
\pgfsetbuttcap%
\pgfsetroundjoin%
\definecolor{currentfill}{rgb}{0.281412,0.155834,0.469201}%
\pgfsetfillcolor{currentfill}%
\pgfsetfillopacity{0.700000}%
\pgfsetlinewidth{0.000000pt}%
\definecolor{currentstroke}{rgb}{0.000000,0.000000,0.000000}%
\pgfsetstrokecolor{currentstroke}%
\pgfsetdash{}{0pt}%
\pgfpathmoveto{\pgfqpoint{3.817253in}{1.732987in}}%
\pgfpathlineto{\pgfqpoint{3.831305in}{1.725990in}}%
\pgfpathlineto{\pgfqpoint{3.845361in}{1.719018in}}%
\pgfpathlineto{\pgfqpoint{3.859423in}{1.712073in}}%
\pgfpathlineto{\pgfqpoint{3.873489in}{1.705152in}}%
\pgfpathlineto{\pgfqpoint{3.865217in}{1.708470in}}%
\pgfpathlineto{\pgfqpoint{3.856933in}{1.712225in}}%
\pgfpathlineto{\pgfqpoint{3.848636in}{1.716427in}}%
\pgfpathlineto{\pgfqpoint{3.840328in}{1.721087in}}%
\pgfpathlineto{\pgfqpoint{3.826231in}{1.728378in}}%
\pgfpathlineto{\pgfqpoint{3.812138in}{1.735694in}}%
\pgfpathlineto{\pgfqpoint{3.798051in}{1.743037in}}%
\pgfpathlineto{\pgfqpoint{3.783968in}{1.750405in}}%
\pgfpathlineto{\pgfqpoint{3.792309in}{1.745368in}}%
\pgfpathlineto{\pgfqpoint{3.800636in}{1.740793in}}%
\pgfpathlineto{\pgfqpoint{3.808951in}{1.736670in}}%
\pgfpathlineto{\pgfqpoint{3.817253in}{1.732987in}}%
\pgfpathclose%
\pgfusepath{fill}%
\end{pgfscope}%
\begin{pgfscope}%
\pgfpathrectangle{\pgfqpoint{1.150000in}{0.150000in}}{\pgfqpoint{5.700000in}{5.700000in}}%
\pgfusepath{clip}%
\pgfsetbuttcap%
\pgfsetroundjoin%
\definecolor{currentfill}{rgb}{0.280894,0.078907,0.402329}%
\pgfsetfillcolor{currentfill}%
\pgfsetfillopacity{0.700000}%
\pgfsetlinewidth{0.000000pt}%
\definecolor{currentstroke}{rgb}{0.000000,0.000000,0.000000}%
\pgfsetstrokecolor{currentstroke}%
\pgfsetdash{}{0pt}%
\pgfpathmoveto{\pgfqpoint{5.271294in}{1.597208in}}%
\pgfpathlineto{\pgfqpoint{5.285730in}{1.595088in}}%
\pgfpathlineto{\pgfqpoint{5.300175in}{1.592991in}}%
\pgfpathlineto{\pgfqpoint{5.314628in}{1.590919in}}%
\pgfpathlineto{\pgfqpoint{5.306869in}{1.579234in}}%
\pgfpathlineto{\pgfqpoint{5.299108in}{1.567609in}}%
\pgfpathlineto{\pgfqpoint{5.291343in}{1.556049in}}%
\pgfpathlineto{\pgfqpoint{5.283575in}{1.544560in}}%
\pgfpathlineto{\pgfqpoint{5.269117in}{1.546868in}}%
\pgfpathlineto{\pgfqpoint{5.254667in}{1.549200in}}%
\pgfpathlineto{\pgfqpoint{5.240226in}{1.551556in}}%
\pgfpathlineto{\pgfqpoint{5.247998in}{1.562864in}}%
\pgfpathlineto{\pgfqpoint{5.255767in}{1.574246in}}%
\pgfpathlineto{\pgfqpoint{5.263532in}{1.585696in}}%
\pgfpathlineto{\pgfqpoint{5.271294in}{1.597208in}}%
\pgfpathclose%
\pgfusepath{fill}%
\end{pgfscope}%
\begin{pgfscope}%
\pgfpathrectangle{\pgfqpoint{1.150000in}{0.150000in}}{\pgfqpoint{5.700000in}{5.700000in}}%
\pgfusepath{clip}%
\pgfsetbuttcap%
\pgfsetroundjoin%
\definecolor{currentfill}{rgb}{0.273809,0.031497,0.358853}%
\pgfsetfillcolor{currentfill}%
\pgfsetfillopacity{0.700000}%
\pgfsetlinewidth{0.000000pt}%
\definecolor{currentstroke}{rgb}{0.000000,0.000000,0.000000}%
\pgfsetstrokecolor{currentstroke}%
\pgfsetdash{}{0pt}%
\pgfpathmoveto{\pgfqpoint{4.947775in}{1.513543in}}%
\pgfpathlineto{\pgfqpoint{4.962102in}{1.510299in}}%
\pgfpathlineto{\pgfqpoint{4.976437in}{1.507078in}}%
\pgfpathlineto{\pgfqpoint{4.990779in}{1.503881in}}%
\pgfpathlineto{\pgfqpoint{5.005130in}{1.500708in}}%
\pgfpathlineto{\pgfqpoint{4.997310in}{1.491150in}}%
\pgfpathlineto{\pgfqpoint{4.989488in}{1.481738in}}%
\pgfpathlineto{\pgfqpoint{4.981663in}{1.472481in}}%
\pgfpathlineto{\pgfqpoint{4.973835in}{1.463384in}}%
\pgfpathlineto{\pgfqpoint{4.959476in}{1.466832in}}%
\pgfpathlineto{\pgfqpoint{4.945124in}{1.470303in}}%
\pgfpathlineto{\pgfqpoint{4.930781in}{1.473797in}}%
\pgfpathlineto{\pgfqpoint{4.916444in}{1.477316in}}%
\pgfpathlineto{\pgfqpoint{4.924282in}{1.486133in}}%
\pgfpathlineto{\pgfqpoint{4.932116in}{1.495115in}}%
\pgfpathlineto{\pgfqpoint{4.939947in}{1.504254in}}%
\pgfpathlineto{\pgfqpoint{4.947775in}{1.513543in}}%
\pgfpathclose%
\pgfusepath{fill}%
\end{pgfscope}%
\begin{pgfscope}%
\pgfpathrectangle{\pgfqpoint{1.150000in}{0.150000in}}{\pgfqpoint{5.700000in}{5.700000in}}%
\pgfusepath{clip}%
\pgfsetbuttcap%
\pgfsetroundjoin%
\definecolor{currentfill}{rgb}{0.125394,0.574318,0.549086}%
\pgfsetfillcolor{currentfill}%
\pgfsetfillopacity{0.700000}%
\pgfsetlinewidth{0.000000pt}%
\definecolor{currentstroke}{rgb}{0.000000,0.000000,0.000000}%
\pgfsetstrokecolor{currentstroke}%
\pgfsetdash{}{0pt}%
\pgfpathmoveto{\pgfqpoint{2.390644in}{2.767515in}}%
\pgfpathlineto{\pgfqpoint{2.404589in}{2.755930in}}%
\pgfpathlineto{\pgfqpoint{2.418534in}{2.744389in}}%
\pgfpathlineto{\pgfqpoint{2.432481in}{2.732892in}}%
\pgfpathlineto{\pgfqpoint{2.446428in}{2.721438in}}%
\pgfpathlineto{\pgfqpoint{2.436747in}{2.743109in}}%
\pgfpathlineto{\pgfqpoint{2.427026in}{2.765505in}}%
\pgfpathlineto{\pgfqpoint{2.417265in}{2.788641in}}%
\pgfpathlineto{\pgfqpoint{2.407462in}{2.812530in}}%
\pgfpathlineto{\pgfqpoint{2.393451in}{2.824440in}}%
\pgfpathlineto{\pgfqpoint{2.379441in}{2.836393in}}%
\pgfpathlineto{\pgfqpoint{2.365431in}{2.848390in}}%
\pgfpathlineto{\pgfqpoint{2.351422in}{2.860432in}}%
\pgfpathlineto{\pgfqpoint{2.361291in}{2.836079in}}%
\pgfpathlineto{\pgfqpoint{2.371116in}{2.812485in}}%
\pgfpathlineto{\pgfqpoint{2.380901in}{2.789635in}}%
\pgfpathlineto{\pgfqpoint{2.390644in}{2.767515in}}%
\pgfpathclose%
\pgfusepath{fill}%
\end{pgfscope}%
\begin{pgfscope}%
\pgfpathrectangle{\pgfqpoint{1.150000in}{0.150000in}}{\pgfqpoint{5.700000in}{5.700000in}}%
\pgfusepath{clip}%
\pgfsetbuttcap%
\pgfsetroundjoin%
\definecolor{currentfill}{rgb}{0.273809,0.031497,0.358853}%
\pgfsetfillcolor{currentfill}%
\pgfsetfillopacity{0.700000}%
\pgfsetlinewidth{0.000000pt}%
\definecolor{currentstroke}{rgb}{0.000000,0.000000,0.000000}%
\pgfsetstrokecolor{currentstroke}%
\pgfsetdash{}{0pt}%
\pgfpathmoveto{\pgfqpoint{4.567969in}{1.500636in}}%
\pgfpathlineto{\pgfqpoint{4.582187in}{1.496057in}}%
\pgfpathlineto{\pgfqpoint{4.596412in}{1.491502in}}%
\pgfpathlineto{\pgfqpoint{4.610644in}{1.486970in}}%
\pgfpathlineto{\pgfqpoint{4.624883in}{1.482463in}}%
\pgfpathlineto{\pgfqpoint{4.616967in}{1.476688in}}%
\pgfpathlineto{\pgfqpoint{4.609046in}{1.471166in}}%
\pgfpathlineto{\pgfqpoint{4.601122in}{1.465904in}}%
\pgfpathlineto{\pgfqpoint{4.593193in}{1.460912in}}%
\pgfpathlineto{\pgfqpoint{4.578939in}{1.465734in}}%
\pgfpathlineto{\pgfqpoint{4.564692in}{1.470579in}}%
\pgfpathlineto{\pgfqpoint{4.550451in}{1.475448in}}%
\pgfpathlineto{\pgfqpoint{4.536217in}{1.480340in}}%
\pgfpathlineto{\pgfqpoint{4.544162in}{1.485013in}}%
\pgfpathlineto{\pgfqpoint{4.552102in}{1.489959in}}%
\pgfpathlineto{\pgfqpoint{4.560038in}{1.495169in}}%
\pgfpathlineto{\pgfqpoint{4.567969in}{1.500636in}}%
\pgfpathclose%
\pgfusepath{fill}%
\end{pgfscope}%
\begin{pgfscope}%
\pgfpathrectangle{\pgfqpoint{1.150000in}{0.150000in}}{\pgfqpoint{5.700000in}{5.700000in}}%
\pgfusepath{clip}%
\pgfsetbuttcap%
\pgfsetroundjoin%
\definecolor{currentfill}{rgb}{0.241237,0.296485,0.539709}%
\pgfsetfillcolor{currentfill}%
\pgfsetfillopacity{0.700000}%
\pgfsetlinewidth{0.000000pt}%
\definecolor{currentstroke}{rgb}{0.000000,0.000000,0.000000}%
\pgfsetstrokecolor{currentstroke}%
\pgfsetdash{}{0pt}%
\pgfpathmoveto{\pgfqpoint{3.300958in}{2.037669in}}%
\pgfpathlineto{\pgfqpoint{3.314944in}{2.029041in}}%
\pgfpathlineto{\pgfqpoint{3.328934in}{2.020441in}}%
\pgfpathlineto{\pgfqpoint{3.342928in}{2.011870in}}%
\pgfpathlineto{\pgfqpoint{3.356925in}{2.003329in}}%
\pgfpathlineto{\pgfqpoint{3.348245in}{2.013617in}}%
\pgfpathlineto{\pgfqpoint{3.339544in}{2.024462in}}%
\pgfpathlineto{\pgfqpoint{3.330822in}{2.035873in}}%
\pgfpathlineto{\pgfqpoint{3.322078in}{2.047864in}}%
\pgfpathlineto{\pgfqpoint{3.308038in}{2.056812in}}%
\pgfpathlineto{\pgfqpoint{3.294001in}{2.065788in}}%
\pgfpathlineto{\pgfqpoint{3.279968in}{2.074793in}}%
\pgfpathlineto{\pgfqpoint{3.265938in}{2.083827in}}%
\pgfpathlineto{\pgfqpoint{3.274726in}{2.071424in}}%
\pgfpathlineto{\pgfqpoint{3.283491in}{2.059605in}}%
\pgfpathlineto{\pgfqpoint{3.292235in}{2.048357in}}%
\pgfpathlineto{\pgfqpoint{3.300958in}{2.037669in}}%
\pgfpathclose%
\pgfusepath{fill}%
\end{pgfscope}%
\begin{pgfscope}%
\pgfpathrectangle{\pgfqpoint{1.150000in}{0.150000in}}{\pgfqpoint{5.700000in}{5.700000in}}%
\pgfusepath{clip}%
\pgfsetbuttcap%
\pgfsetroundjoin%
\definecolor{currentfill}{rgb}{0.190631,0.407061,0.556089}%
\pgfsetfillcolor{currentfill}%
\pgfsetfillopacity{0.700000}%
\pgfsetlinewidth{0.000000pt}%
\definecolor{currentstroke}{rgb}{0.000000,0.000000,0.000000}%
\pgfsetstrokecolor{currentstroke}%
\pgfsetdash{}{0pt}%
\pgfpathmoveto{\pgfqpoint{2.930176in}{2.309790in}}%
\pgfpathlineto{\pgfqpoint{2.944132in}{2.300009in}}%
\pgfpathlineto{\pgfqpoint{2.958091in}{2.290261in}}%
\pgfpathlineto{\pgfqpoint{2.972053in}{2.280547in}}%
\pgfpathlineto{\pgfqpoint{2.986018in}{2.270865in}}%
\pgfpathlineto{\pgfqpoint{2.976969in}{2.285955in}}%
\pgfpathlineto{\pgfqpoint{2.967893in}{2.301674in}}%
\pgfpathlineto{\pgfqpoint{2.958789in}{2.318037in}}%
\pgfpathlineto{\pgfqpoint{2.949655in}{2.335055in}}%
\pgfpathlineto{\pgfqpoint{2.935639in}{2.345164in}}%
\pgfpathlineto{\pgfqpoint{2.921626in}{2.355305in}}%
\pgfpathlineto{\pgfqpoint{2.907615in}{2.365480in}}%
\pgfpathlineto{\pgfqpoint{2.893607in}{2.375689in}}%
\pgfpathlineto{\pgfqpoint{2.902793in}{2.358237in}}%
\pgfpathlineto{\pgfqpoint{2.911950in}{2.341445in}}%
\pgfpathlineto{\pgfqpoint{2.921077in}{2.325300in}}%
\pgfpathlineto{\pgfqpoint{2.930176in}{2.309790in}}%
\pgfpathclose%
\pgfusepath{fill}%
\end{pgfscope}%
\begin{pgfscope}%
\pgfpathrectangle{\pgfqpoint{1.150000in}{0.150000in}}{\pgfqpoint{5.700000in}{5.700000in}}%
\pgfusepath{clip}%
\pgfsetbuttcap%
\pgfsetroundjoin%
\definecolor{currentfill}{rgb}{0.276022,0.044167,0.370164}%
\pgfsetfillcolor{currentfill}%
\pgfsetfillopacity{0.700000}%
\pgfsetlinewidth{0.000000pt}%
\definecolor{currentstroke}{rgb}{0.000000,0.000000,0.000000}%
\pgfsetstrokecolor{currentstroke}%
\pgfsetdash{}{0pt}%
\pgfpathmoveto{\pgfqpoint{4.422577in}{1.520338in}}%
\pgfpathlineto{\pgfqpoint{4.436759in}{1.515255in}}%
\pgfpathlineto{\pgfqpoint{4.450948in}{1.510196in}}%
\pgfpathlineto{\pgfqpoint{4.465144in}{1.505160in}}%
\pgfpathlineto{\pgfqpoint{4.479346in}{1.500148in}}%
\pgfpathlineto{\pgfqpoint{4.471379in}{1.496078in}}%
\pgfpathlineto{\pgfqpoint{4.463407in}{1.492301in}}%
\pgfpathlineto{\pgfqpoint{4.455430in}{1.488825in}}%
\pgfpathlineto{\pgfqpoint{4.447448in}{1.485659in}}%
\pgfpathlineto{\pgfqpoint{4.433228in}{1.490997in}}%
\pgfpathlineto{\pgfqpoint{4.419015in}{1.496360in}}%
\pgfpathlineto{\pgfqpoint{4.404807in}{1.501747in}}%
\pgfpathlineto{\pgfqpoint{4.390606in}{1.507157in}}%
\pgfpathlineto{\pgfqpoint{4.398607in}{1.509991in}}%
\pgfpathlineto{\pgfqpoint{4.406603in}{1.513138in}}%
\pgfpathlineto{\pgfqpoint{4.414593in}{1.516590in}}%
\pgfpathlineto{\pgfqpoint{4.422577in}{1.520338in}}%
\pgfpathclose%
\pgfusepath{fill}%
\end{pgfscope}%
\begin{pgfscope}%
\pgfpathrectangle{\pgfqpoint{1.150000in}{0.150000in}}{\pgfqpoint{5.700000in}{5.700000in}}%
\pgfusepath{clip}%
\pgfsetbuttcap%
\pgfsetroundjoin%
\definecolor{currentfill}{rgb}{0.272594,0.025563,0.353093}%
\pgfsetfillcolor{currentfill}%
\pgfsetfillopacity{0.700000}%
\pgfsetlinewidth{0.000000pt}%
\definecolor{currentstroke}{rgb}{0.000000,0.000000,0.000000}%
\pgfsetstrokecolor{currentstroke}%
\pgfsetdash{}{0pt}%
\pgfpathmoveto{\pgfqpoint{4.713476in}{1.491339in}}%
\pgfpathlineto{\pgfqpoint{4.727736in}{1.487250in}}%
\pgfpathlineto{\pgfqpoint{4.742003in}{1.483184in}}%
\pgfpathlineto{\pgfqpoint{4.756277in}{1.479143in}}%
\pgfpathlineto{\pgfqpoint{4.770558in}{1.475125in}}%
\pgfpathlineto{\pgfqpoint{4.762683in}{1.467827in}}%
\pgfpathlineto{\pgfqpoint{4.754805in}{1.460744in}}%
\pgfpathlineto{\pgfqpoint{4.746923in}{1.453883in}}%
\pgfpathlineto{\pgfqpoint{4.739038in}{1.447253in}}%
\pgfpathlineto{\pgfqpoint{4.724745in}{1.451572in}}%
\pgfpathlineto{\pgfqpoint{4.710458in}{1.455914in}}%
\pgfpathlineto{\pgfqpoint{4.696178in}{1.460279in}}%
\pgfpathlineto{\pgfqpoint{4.681906in}{1.464669in}}%
\pgfpathlineto{\pgfqpoint{4.689804in}{1.470994in}}%
\pgfpathlineto{\pgfqpoint{4.697698in}{1.477552in}}%
\pgfpathlineto{\pgfqpoint{4.705589in}{1.484336in}}%
\pgfpathlineto{\pgfqpoint{4.713476in}{1.491339in}}%
\pgfpathclose%
\pgfusepath{fill}%
\end{pgfscope}%
\begin{pgfscope}%
\pgfpathrectangle{\pgfqpoint{1.150000in}{0.150000in}}{\pgfqpoint{5.700000in}{5.700000in}}%
\pgfusepath{clip}%
\pgfsetbuttcap%
\pgfsetroundjoin%
\definecolor{currentfill}{rgb}{0.282656,0.100196,0.422160}%
\pgfsetfillcolor{currentfill}%
\pgfsetfillopacity{0.700000}%
\pgfsetlinewidth{0.000000pt}%
\definecolor{currentstroke}{rgb}{0.000000,0.000000,0.000000}%
\pgfsetstrokecolor{currentstroke}%
\pgfsetdash{}{0pt}%
\pgfpathmoveto{\pgfqpoint{4.075330in}{1.619249in}}%
\pgfpathlineto{\pgfqpoint{4.089437in}{1.613011in}}%
\pgfpathlineto{\pgfqpoint{4.103550in}{1.606798in}}%
\pgfpathlineto{\pgfqpoint{4.117668in}{1.600609in}}%
\pgfpathlineto{\pgfqpoint{4.131791in}{1.594445in}}%
\pgfpathlineto{\pgfqpoint{4.123667in}{1.594714in}}%
\pgfpathlineto{\pgfqpoint{4.115535in}{1.595366in}}%
\pgfpathlineto{\pgfqpoint{4.107393in}{1.596410in}}%
\pgfpathlineto{\pgfqpoint{4.099243in}{1.597856in}}%
\pgfpathlineto{\pgfqpoint{4.085094in}{1.604376in}}%
\pgfpathlineto{\pgfqpoint{4.070950in}{1.610920in}}%
\pgfpathlineto{\pgfqpoint{4.056812in}{1.617488in}}%
\pgfpathlineto{\pgfqpoint{4.042679in}{1.624082in}}%
\pgfpathlineto{\pgfqpoint{4.050856in}{1.622275in}}%
\pgfpathlineto{\pgfqpoint{4.059024in}{1.620873in}}%
\pgfpathlineto{\pgfqpoint{4.067182in}{1.619868in}}%
\pgfpathlineto{\pgfqpoint{4.075330in}{1.619249in}}%
\pgfpathclose%
\pgfusepath{fill}%
\end{pgfscope}%
\begin{pgfscope}%
\pgfpathrectangle{\pgfqpoint{1.150000in}{0.150000in}}{\pgfqpoint{5.700000in}{5.700000in}}%
\pgfusepath{clip}%
\pgfsetbuttcap%
\pgfsetroundjoin%
\definecolor{currentfill}{rgb}{0.271828,0.209303,0.504434}%
\pgfsetfillcolor{currentfill}%
\pgfsetfillopacity{0.700000}%
\pgfsetlinewidth{0.000000pt}%
\definecolor{currentstroke}{rgb}{0.000000,0.000000,0.000000}%
\pgfsetstrokecolor{currentstroke}%
\pgfsetdash{}{0pt}%
\pgfpathmoveto{\pgfqpoint{3.615332in}{1.840858in}}%
\pgfpathlineto{\pgfqpoint{3.629361in}{1.833175in}}%
\pgfpathlineto{\pgfqpoint{3.643393in}{1.825518in}}%
\pgfpathlineto{\pgfqpoint{3.657431in}{1.817888in}}%
\pgfpathlineto{\pgfqpoint{3.671472in}{1.810285in}}%
\pgfpathlineto{\pgfqpoint{3.663051in}{1.816555in}}%
\pgfpathlineto{\pgfqpoint{3.654614in}{1.823316in}}%
\pgfpathlineto{\pgfqpoint{3.646162in}{1.830578in}}%
\pgfpathlineto{\pgfqpoint{3.637695in}{1.838352in}}%
\pgfpathlineto{\pgfqpoint{3.623618in}{1.846343in}}%
\pgfpathlineto{\pgfqpoint{3.609544in}{1.854360in}}%
\pgfpathlineto{\pgfqpoint{3.595476in}{1.862404in}}%
\pgfpathlineto{\pgfqpoint{3.581411in}{1.870474in}}%
\pgfpathlineto{\pgfqpoint{3.589916in}{1.862306in}}%
\pgfpathlineto{\pgfqpoint{3.598404in}{1.854655in}}%
\pgfpathlineto{\pgfqpoint{3.606876in}{1.847509in}}%
\pgfpathlineto{\pgfqpoint{3.615332in}{1.840858in}}%
\pgfpathclose%
\pgfusepath{fill}%
\end{pgfscope}%
\begin{pgfscope}%
\pgfpathrectangle{\pgfqpoint{1.150000in}{0.150000in}}{\pgfqpoint{5.700000in}{5.700000in}}%
\pgfusepath{clip}%
\pgfsetbuttcap%
\pgfsetroundjoin%
\definecolor{currentfill}{rgb}{0.278791,0.062145,0.386592}%
\pgfsetfillcolor{currentfill}%
\pgfsetfillopacity{0.700000}%
\pgfsetlinewidth{0.000000pt}%
\definecolor{currentstroke}{rgb}{0.000000,0.000000,0.000000}%
\pgfsetstrokecolor{currentstroke}%
\pgfsetdash{}{0pt}%
\pgfpathmoveto{\pgfqpoint{5.182544in}{1.561216in}}%
\pgfpathlineto{\pgfqpoint{5.196952in}{1.558765in}}%
\pgfpathlineto{\pgfqpoint{5.211368in}{1.556338in}}%
\pgfpathlineto{\pgfqpoint{5.225793in}{1.553935in}}%
\pgfpathlineto{\pgfqpoint{5.240226in}{1.551556in}}%
\pgfpathlineto{\pgfqpoint{5.232451in}{1.540327in}}%
\pgfpathlineto{\pgfqpoint{5.224672in}{1.529184in}}%
\pgfpathlineto{\pgfqpoint{5.216891in}{1.518133in}}%
\pgfpathlineto{\pgfqpoint{5.209107in}{1.507179in}}%
\pgfpathlineto{\pgfqpoint{5.194668in}{1.509807in}}%
\pgfpathlineto{\pgfqpoint{5.180238in}{1.512459in}}%
\pgfpathlineto{\pgfqpoint{5.165816in}{1.515134in}}%
\pgfpathlineto{\pgfqpoint{5.151402in}{1.517833in}}%
\pgfpathlineto{\pgfqpoint{5.159192in}{1.528533in}}%
\pgfpathlineto{\pgfqpoint{5.166979in}{1.539334in}}%
\pgfpathlineto{\pgfqpoint{5.174763in}{1.550230in}}%
\pgfpathlineto{\pgfqpoint{5.182544in}{1.561216in}}%
\pgfpathclose%
\pgfusepath{fill}%
\end{pgfscope}%
\begin{pgfscope}%
\pgfpathrectangle{\pgfqpoint{1.150000in}{0.150000in}}{\pgfqpoint{5.700000in}{5.700000in}}%
\pgfusepath{clip}%
\pgfsetbuttcap%
\pgfsetroundjoin%
\definecolor{currentfill}{rgb}{0.129933,0.559582,0.551864}%
\pgfsetfillcolor{currentfill}%
\pgfsetfillopacity{0.700000}%
\pgfsetlinewidth{0.000000pt}%
\definecolor{currentstroke}{rgb}{0.000000,0.000000,0.000000}%
\pgfsetstrokecolor{currentstroke}%
\pgfsetdash{}{0pt}%
\pgfpathmoveto{\pgfqpoint{2.446428in}{2.721438in}}%
\pgfpathlineto{\pgfqpoint{2.460377in}{2.710026in}}%
\pgfpathlineto{\pgfqpoint{2.474327in}{2.698657in}}%
\pgfpathlineto{\pgfqpoint{2.488279in}{2.687330in}}%
\pgfpathlineto{\pgfqpoint{2.502231in}{2.676045in}}%
\pgfpathlineto{\pgfqpoint{2.492611in}{2.697269in}}%
\pgfpathlineto{\pgfqpoint{2.482953in}{2.719213in}}%
\pgfpathlineto{\pgfqpoint{2.473255in}{2.741891in}}%
\pgfpathlineto{\pgfqpoint{2.463517in}{2.765318in}}%
\pgfpathlineto{\pgfqpoint{2.449501in}{2.777057in}}%
\pgfpathlineto{\pgfqpoint{2.435487in}{2.788839in}}%
\pgfpathlineto{\pgfqpoint{2.421474in}{2.800663in}}%
\pgfpathlineto{\pgfqpoint{2.407462in}{2.812530in}}%
\pgfpathlineto{\pgfqpoint{2.417265in}{2.788641in}}%
\pgfpathlineto{\pgfqpoint{2.427026in}{2.765505in}}%
\pgfpathlineto{\pgfqpoint{2.436747in}{2.743109in}}%
\pgfpathlineto{\pgfqpoint{2.446428in}{2.721438in}}%
\pgfpathclose%
\pgfusepath{fill}%
\end{pgfscope}%
\begin{pgfscope}%
\pgfpathrectangle{\pgfqpoint{1.150000in}{0.150000in}}{\pgfqpoint{5.700000in}{5.700000in}}%
\pgfusepath{clip}%
\pgfsetbuttcap%
\pgfsetroundjoin%
\definecolor{currentfill}{rgb}{0.282290,0.145912,0.461510}%
\pgfsetfillcolor{currentfill}%
\pgfsetfillopacity{0.700000}%
\pgfsetlinewidth{0.000000pt}%
\definecolor{currentstroke}{rgb}{0.000000,0.000000,0.000000}%
\pgfsetstrokecolor{currentstroke}%
\pgfsetdash{}{0pt}%
\pgfpathmoveto{\pgfqpoint{3.873489in}{1.705152in}}%
\pgfpathlineto{\pgfqpoint{3.887560in}{1.698258in}}%
\pgfpathlineto{\pgfqpoint{3.901636in}{1.691389in}}%
\pgfpathlineto{\pgfqpoint{3.915718in}{1.684545in}}%
\pgfpathlineto{\pgfqpoint{3.929804in}{1.677726in}}%
\pgfpathlineto{\pgfqpoint{3.921561in}{1.680679in}}%
\pgfpathlineto{\pgfqpoint{3.913307in}{1.684065in}}%
\pgfpathlineto{\pgfqpoint{3.905041in}{1.687894in}}%
\pgfpathlineto{\pgfqpoint{3.896763in}{1.692177in}}%
\pgfpathlineto{\pgfqpoint{3.882647in}{1.699367in}}%
\pgfpathlineto{\pgfqpoint{3.868536in}{1.706581in}}%
\pgfpathlineto{\pgfqpoint{3.854429in}{1.713821in}}%
\pgfpathlineto{\pgfqpoint{3.840328in}{1.721087in}}%
\pgfpathlineto{\pgfqpoint{3.848636in}{1.716427in}}%
\pgfpathlineto{\pgfqpoint{3.856933in}{1.712225in}}%
\pgfpathlineto{\pgfqpoint{3.865217in}{1.708470in}}%
\pgfpathlineto{\pgfqpoint{3.873489in}{1.705152in}}%
\pgfpathclose%
\pgfusepath{fill}%
\end{pgfscope}%
\begin{pgfscope}%
\pgfpathrectangle{\pgfqpoint{1.150000in}{0.150000in}}{\pgfqpoint{5.700000in}{5.700000in}}%
\pgfusepath{clip}%
\pgfsetbuttcap%
\pgfsetroundjoin%
\definecolor{currentfill}{rgb}{0.279566,0.067836,0.391917}%
\pgfsetfillcolor{currentfill}%
\pgfsetfillopacity{0.700000}%
\pgfsetlinewidth{0.000000pt}%
\definecolor{currentstroke}{rgb}{0.000000,0.000000,0.000000}%
\pgfsetstrokecolor{currentstroke}%
\pgfsetdash{}{0pt}%
\pgfpathmoveto{\pgfqpoint{4.277216in}{1.551307in}}%
\pgfpathlineto{\pgfqpoint{4.291368in}{1.545704in}}%
\pgfpathlineto{\pgfqpoint{4.305527in}{1.540125in}}%
\pgfpathlineto{\pgfqpoint{4.319692in}{1.534570in}}%
\pgfpathlineto{\pgfqpoint{4.333863in}{1.529039in}}%
\pgfpathlineto{\pgfqpoint{4.325835in}{1.526864in}}%
\pgfpathlineto{\pgfqpoint{4.317802in}{1.525024in}}%
\pgfpathlineto{\pgfqpoint{4.309761in}{1.523527in}}%
\pgfpathlineto{\pgfqpoint{4.301714in}{1.522383in}}%
\pgfpathlineto{\pgfqpoint{4.287522in}{1.528255in}}%
\pgfpathlineto{\pgfqpoint{4.273336in}{1.534151in}}%
\pgfpathlineto{\pgfqpoint{4.259156in}{1.540071in}}%
\pgfpathlineto{\pgfqpoint{4.244982in}{1.546016in}}%
\pgfpathlineto{\pgfqpoint{4.253051in}{1.546813in}}%
\pgfpathlineto{\pgfqpoint{4.261113in}{1.547966in}}%
\pgfpathlineto{\pgfqpoint{4.269168in}{1.549467in}}%
\pgfpathlineto{\pgfqpoint{4.277216in}{1.551307in}}%
\pgfpathclose%
\pgfusepath{fill}%
\end{pgfscope}%
\begin{pgfscope}%
\pgfpathrectangle{\pgfqpoint{1.150000in}{0.150000in}}{\pgfqpoint{5.700000in}{5.700000in}}%
\pgfusepath{clip}%
\pgfsetbuttcap%
\pgfsetroundjoin%
\definecolor{currentfill}{rgb}{0.272594,0.025563,0.353093}%
\pgfsetfillcolor{currentfill}%
\pgfsetfillopacity{0.700000}%
\pgfsetlinewidth{0.000000pt}%
\definecolor{currentstroke}{rgb}{0.000000,0.000000,0.000000}%
\pgfsetstrokecolor{currentstroke}%
\pgfsetdash{}{0pt}%
\pgfpathmoveto{\pgfqpoint{4.859174in}{1.491625in}}%
\pgfpathlineto{\pgfqpoint{4.873481in}{1.488012in}}%
\pgfpathlineto{\pgfqpoint{4.887794in}{1.484423in}}%
\pgfpathlineto{\pgfqpoint{4.902116in}{1.480858in}}%
\pgfpathlineto{\pgfqpoint{4.916444in}{1.477316in}}%
\pgfpathlineto{\pgfqpoint{4.908604in}{1.468669in}}%
\pgfpathlineto{\pgfqpoint{4.900761in}{1.460202in}}%
\pgfpathlineto{\pgfqpoint{4.892915in}{1.451919in}}%
\pgfpathlineto{\pgfqpoint{4.885066in}{1.443830in}}%
\pgfpathlineto{\pgfqpoint{4.870727in}{1.447659in}}%
\pgfpathlineto{\pgfqpoint{4.856395in}{1.451512in}}%
\pgfpathlineto{\pgfqpoint{4.842071in}{1.455389in}}%
\pgfpathlineto{\pgfqpoint{4.827754in}{1.459289in}}%
\pgfpathlineto{\pgfqpoint{4.835614in}{1.467086in}}%
\pgfpathlineto{\pgfqpoint{4.843470in}{1.475079in}}%
\pgfpathlineto{\pgfqpoint{4.851324in}{1.483261in}}%
\pgfpathlineto{\pgfqpoint{4.859174in}{1.491625in}}%
\pgfpathclose%
\pgfusepath{fill}%
\end{pgfscope}%
\begin{pgfscope}%
\pgfpathrectangle{\pgfqpoint{1.150000in}{0.150000in}}{\pgfqpoint{5.700000in}{5.700000in}}%
\pgfusepath{clip}%
\pgfsetbuttcap%
\pgfsetroundjoin%
\definecolor{currentfill}{rgb}{0.277018,0.050344,0.375715}%
\pgfsetfillcolor{currentfill}%
\pgfsetfillopacity{0.700000}%
\pgfsetlinewidth{0.000000pt}%
\definecolor{currentstroke}{rgb}{0.000000,0.000000,0.000000}%
\pgfsetstrokecolor{currentstroke}%
\pgfsetdash{}{0pt}%
\pgfpathmoveto{\pgfqpoint{5.093825in}{1.528866in}}%
\pgfpathlineto{\pgfqpoint{5.108207in}{1.526072in}}%
\pgfpathlineto{\pgfqpoint{5.122597in}{1.523302in}}%
\pgfpathlineto{\pgfqpoint{5.136995in}{1.520556in}}%
\pgfpathlineto{\pgfqpoint{5.151402in}{1.517833in}}%
\pgfpathlineto{\pgfqpoint{5.143608in}{1.507241in}}%
\pgfpathlineto{\pgfqpoint{5.135812in}{1.496762in}}%
\pgfpathlineto{\pgfqpoint{5.128014in}{1.486404in}}%
\pgfpathlineto{\pgfqpoint{5.120212in}{1.476173in}}%
\pgfpathlineto{\pgfqpoint{5.105799in}{1.479157in}}%
\pgfpathlineto{\pgfqpoint{5.091394in}{1.482165in}}%
\pgfpathlineto{\pgfqpoint{5.076997in}{1.485196in}}%
\pgfpathlineto{\pgfqpoint{5.062608in}{1.488251in}}%
\pgfpathlineto{\pgfqpoint{5.070417in}{1.498216in}}%
\pgfpathlineto{\pgfqpoint{5.078223in}{1.508311in}}%
\pgfpathlineto{\pgfqpoint{5.086025in}{1.518530in}}%
\pgfpathlineto{\pgfqpoint{5.093825in}{1.528866in}}%
\pgfpathclose%
\pgfusepath{fill}%
\end{pgfscope}%
\begin{pgfscope}%
\pgfpathrectangle{\pgfqpoint{1.150000in}{0.150000in}}{\pgfqpoint{5.700000in}{5.700000in}}%
\pgfusepath{clip}%
\pgfsetbuttcap%
\pgfsetroundjoin%
\definecolor{currentfill}{rgb}{0.195860,0.395433,0.555276}%
\pgfsetfillcolor{currentfill}%
\pgfsetfillopacity{0.700000}%
\pgfsetlinewidth{0.000000pt}%
\definecolor{currentstroke}{rgb}{0.000000,0.000000,0.000000}%
\pgfsetstrokecolor{currentstroke}%
\pgfsetdash{}{0pt}%
\pgfpathmoveto{\pgfqpoint{2.986018in}{2.270865in}}%
\pgfpathlineto{\pgfqpoint{2.999985in}{2.261215in}}%
\pgfpathlineto{\pgfqpoint{3.013955in}{2.251598in}}%
\pgfpathlineto{\pgfqpoint{3.027928in}{2.242014in}}%
\pgfpathlineto{\pgfqpoint{3.041904in}{2.232461in}}%
\pgfpathlineto{\pgfqpoint{3.032905in}{2.247131in}}%
\pgfpathlineto{\pgfqpoint{3.023879in}{2.262427in}}%
\pgfpathlineto{\pgfqpoint{3.014826in}{2.278360in}}%
\pgfpathlineto{\pgfqpoint{3.005745in}{2.294945in}}%
\pgfpathlineto{\pgfqpoint{2.991718in}{2.304924in}}%
\pgfpathlineto{\pgfqpoint{2.977695in}{2.314935in}}%
\pgfpathlineto{\pgfqpoint{2.963674in}{2.324979in}}%
\pgfpathlineto{\pgfqpoint{2.949655in}{2.335055in}}%
\pgfpathlineto{\pgfqpoint{2.958789in}{2.318037in}}%
\pgfpathlineto{\pgfqpoint{2.967893in}{2.301674in}}%
\pgfpathlineto{\pgfqpoint{2.976969in}{2.285955in}}%
\pgfpathlineto{\pgfqpoint{2.986018in}{2.270865in}}%
\pgfpathclose%
\pgfusepath{fill}%
\end{pgfscope}%
\begin{pgfscope}%
\pgfpathrectangle{\pgfqpoint{1.150000in}{0.150000in}}{\pgfqpoint{5.700000in}{5.700000in}}%
\pgfusepath{clip}%
\pgfsetbuttcap%
\pgfsetroundjoin%
\definecolor{currentfill}{rgb}{0.244972,0.287675,0.537260}%
\pgfsetfillcolor{currentfill}%
\pgfsetfillopacity{0.700000}%
\pgfsetlinewidth{0.000000pt}%
\definecolor{currentstroke}{rgb}{0.000000,0.000000,0.000000}%
\pgfsetstrokecolor{currentstroke}%
\pgfsetdash{}{0pt}%
\pgfpathmoveto{\pgfqpoint{3.356925in}{2.003329in}}%
\pgfpathlineto{\pgfqpoint{3.370927in}{1.994815in}}%
\pgfpathlineto{\pgfqpoint{3.384932in}{1.986331in}}%
\pgfpathlineto{\pgfqpoint{3.398940in}{1.977874in}}%
\pgfpathlineto{\pgfqpoint{3.412953in}{1.969446in}}%
\pgfpathlineto{\pgfqpoint{3.404313in}{1.979336in}}%
\pgfpathlineto{\pgfqpoint{3.395654in}{1.989777in}}%
\pgfpathlineto{\pgfqpoint{3.386974in}{2.000782in}}%
\pgfpathlineto{\pgfqpoint{3.378275in}{2.012361in}}%
\pgfpathlineto{\pgfqpoint{3.364220in}{2.021194in}}%
\pgfpathlineto{\pgfqpoint{3.350169in}{2.030056in}}%
\pgfpathlineto{\pgfqpoint{3.336122in}{2.038946in}}%
\pgfpathlineto{\pgfqpoint{3.322078in}{2.047864in}}%
\pgfpathlineto{\pgfqpoint{3.330822in}{2.035873in}}%
\pgfpathlineto{\pgfqpoint{3.339544in}{2.024462in}}%
\pgfpathlineto{\pgfqpoint{3.348245in}{2.013617in}}%
\pgfpathlineto{\pgfqpoint{3.356925in}{2.003329in}}%
\pgfpathclose%
\pgfusepath{fill}%
\end{pgfscope}%
\begin{pgfscope}%
\pgfpathrectangle{\pgfqpoint{1.150000in}{0.150000in}}{\pgfqpoint{5.700000in}{5.700000in}}%
\pgfusepath{clip}%
\pgfsetbuttcap%
\pgfsetroundjoin%
\definecolor{currentfill}{rgb}{0.135066,0.544853,0.554029}%
\pgfsetfillcolor{currentfill}%
\pgfsetfillopacity{0.700000}%
\pgfsetlinewidth{0.000000pt}%
\definecolor{currentstroke}{rgb}{0.000000,0.000000,0.000000}%
\pgfsetstrokecolor{currentstroke}%
\pgfsetdash{}{0pt}%
\pgfpathmoveto{\pgfqpoint{2.502231in}{2.676045in}}%
\pgfpathlineto{\pgfqpoint{2.516185in}{2.664801in}}%
\pgfpathlineto{\pgfqpoint{2.530141in}{2.653598in}}%
\pgfpathlineto{\pgfqpoint{2.544098in}{2.642435in}}%
\pgfpathlineto{\pgfqpoint{2.558056in}{2.631313in}}%
\pgfpathlineto{\pgfqpoint{2.548496in}{2.652091in}}%
\pgfpathlineto{\pgfqpoint{2.538899in}{2.673584in}}%
\pgfpathlineto{\pgfqpoint{2.529263in}{2.695806in}}%
\pgfpathlineto{\pgfqpoint{2.519589in}{2.718772in}}%
\pgfpathlineto{\pgfqpoint{2.505569in}{2.730347in}}%
\pgfpathlineto{\pgfqpoint{2.491550in}{2.741963in}}%
\pgfpathlineto{\pgfqpoint{2.477533in}{2.753620in}}%
\pgfpathlineto{\pgfqpoint{2.463517in}{2.765318in}}%
\pgfpathlineto{\pgfqpoint{2.473255in}{2.741891in}}%
\pgfpathlineto{\pgfqpoint{2.482953in}{2.719213in}}%
\pgfpathlineto{\pgfqpoint{2.492611in}{2.697269in}}%
\pgfpathlineto{\pgfqpoint{2.502231in}{2.676045in}}%
\pgfpathclose%
\pgfusepath{fill}%
\end{pgfscope}%
\begin{pgfscope}%
\pgfpathrectangle{\pgfqpoint{1.150000in}{0.150000in}}{\pgfqpoint{5.700000in}{5.700000in}}%
\pgfusepath{clip}%
\pgfsetbuttcap%
\pgfsetroundjoin%
\definecolor{currentfill}{rgb}{0.273006,0.204520,0.501721}%
\pgfsetfillcolor{currentfill}%
\pgfsetfillopacity{0.700000}%
\pgfsetlinewidth{0.000000pt}%
\definecolor{currentstroke}{rgb}{0.000000,0.000000,0.000000}%
\pgfsetstrokecolor{currentstroke}%
\pgfsetdash{}{0pt}%
\pgfpathmoveto{\pgfqpoint{3.671472in}{1.810285in}}%
\pgfpathlineto{\pgfqpoint{3.685518in}{1.802708in}}%
\pgfpathlineto{\pgfqpoint{3.699569in}{1.795158in}}%
\pgfpathlineto{\pgfqpoint{3.713624in}{1.787634in}}%
\pgfpathlineto{\pgfqpoint{3.727684in}{1.780136in}}%
\pgfpathlineto{\pgfqpoint{3.719296in}{1.786025in}}%
\pgfpathlineto{\pgfqpoint{3.710895in}{1.792401in}}%
\pgfpathlineto{\pgfqpoint{3.702478in}{1.799274in}}%
\pgfpathlineto{\pgfqpoint{3.694047in}{1.806655in}}%
\pgfpathlineto{\pgfqpoint{3.679952in}{1.814540in}}%
\pgfpathlineto{\pgfqpoint{3.665862in}{1.822451in}}%
\pgfpathlineto{\pgfqpoint{3.651776in}{1.830389in}}%
\pgfpathlineto{\pgfqpoint{3.637695in}{1.838352in}}%
\pgfpathlineto{\pgfqpoint{3.646162in}{1.830578in}}%
\pgfpathlineto{\pgfqpoint{3.654614in}{1.823316in}}%
\pgfpathlineto{\pgfqpoint{3.663051in}{1.816555in}}%
\pgfpathlineto{\pgfqpoint{3.671472in}{1.810285in}}%
\pgfpathclose%
\pgfusepath{fill}%
\end{pgfscope}%
\begin{pgfscope}%
\pgfpathrectangle{\pgfqpoint{1.150000in}{0.150000in}}{\pgfqpoint{5.700000in}{5.700000in}}%
\pgfusepath{clip}%
\pgfsetbuttcap%
\pgfsetroundjoin%
\definecolor{currentfill}{rgb}{0.273809,0.031497,0.358853}%
\pgfsetfillcolor{currentfill}%
\pgfsetfillopacity{0.700000}%
\pgfsetlinewidth{0.000000pt}%
\definecolor{currentstroke}{rgb}{0.000000,0.000000,0.000000}%
\pgfsetstrokecolor{currentstroke}%
\pgfsetdash{}{0pt}%
\pgfpathmoveto{\pgfqpoint{4.624883in}{1.482463in}}%
\pgfpathlineto{\pgfqpoint{4.639129in}{1.477979in}}%
\pgfpathlineto{\pgfqpoint{4.653381in}{1.473518in}}%
\pgfpathlineto{\pgfqpoint{4.667640in}{1.469082in}}%
\pgfpathlineto{\pgfqpoint{4.681906in}{1.464669in}}%
\pgfpathlineto{\pgfqpoint{4.674004in}{1.458585in}}%
\pgfpathlineto{\pgfqpoint{4.666097in}{1.452751in}}%
\pgfpathlineto{\pgfqpoint{4.658188in}{1.447174in}}%
\pgfpathlineto{\pgfqpoint{4.650274in}{1.441863in}}%
\pgfpathlineto{\pgfqpoint{4.635993in}{1.446590in}}%
\pgfpathlineto{\pgfqpoint{4.621720in}{1.451340in}}%
\pgfpathlineto{\pgfqpoint{4.607453in}{1.456115in}}%
\pgfpathlineto{\pgfqpoint{4.593193in}{1.460912in}}%
\pgfpathlineto{\pgfqpoint{4.601122in}{1.465904in}}%
\pgfpathlineto{\pgfqpoint{4.609046in}{1.471166in}}%
\pgfpathlineto{\pgfqpoint{4.616967in}{1.476688in}}%
\pgfpathlineto{\pgfqpoint{4.624883in}{1.482463in}}%
\pgfpathclose%
\pgfusepath{fill}%
\end{pgfscope}%
\begin{pgfscope}%
\pgfpathrectangle{\pgfqpoint{1.150000in}{0.150000in}}{\pgfqpoint{5.700000in}{5.700000in}}%
\pgfusepath{clip}%
\pgfsetbuttcap%
\pgfsetroundjoin%
\definecolor{currentfill}{rgb}{0.274952,0.037752,0.364543}%
\pgfsetfillcolor{currentfill}%
\pgfsetfillopacity{0.700000}%
\pgfsetlinewidth{0.000000pt}%
\definecolor{currentstroke}{rgb}{0.000000,0.000000,0.000000}%
\pgfsetstrokecolor{currentstroke}%
\pgfsetdash{}{0pt}%
\pgfpathmoveto{\pgfqpoint{5.005130in}{1.500708in}}%
\pgfpathlineto{\pgfqpoint{5.019488in}{1.497558in}}%
\pgfpathlineto{\pgfqpoint{5.033853in}{1.494432in}}%
\pgfpathlineto{\pgfqpoint{5.048227in}{1.491330in}}%
\pgfpathlineto{\pgfqpoint{5.062608in}{1.488251in}}%
\pgfpathlineto{\pgfqpoint{5.054797in}{1.478424in}}%
\pgfpathlineto{\pgfqpoint{5.046983in}{1.468739in}}%
\pgfpathlineto{\pgfqpoint{5.039166in}{1.459205in}}%
\pgfpathlineto{\pgfqpoint{5.031346in}{1.449829in}}%
\pgfpathlineto{\pgfqpoint{5.016957in}{1.453182in}}%
\pgfpathlineto{\pgfqpoint{5.002575in}{1.456559in}}%
\pgfpathlineto{\pgfqpoint{4.988201in}{1.459960in}}%
\pgfpathlineto{\pgfqpoint{4.973835in}{1.463384in}}%
\pgfpathlineto{\pgfqpoint{4.981663in}{1.472481in}}%
\pgfpathlineto{\pgfqpoint{4.989488in}{1.481738in}}%
\pgfpathlineto{\pgfqpoint{4.997310in}{1.491150in}}%
\pgfpathlineto{\pgfqpoint{5.005130in}{1.500708in}}%
\pgfpathclose%
\pgfusepath{fill}%
\end{pgfscope}%
\begin{pgfscope}%
\pgfpathrectangle{\pgfqpoint{1.150000in}{0.150000in}}{\pgfqpoint{5.700000in}{5.700000in}}%
\pgfusepath{clip}%
\pgfsetbuttcap%
\pgfsetroundjoin%
\definecolor{currentfill}{rgb}{0.276022,0.044167,0.370164}%
\pgfsetfillcolor{currentfill}%
\pgfsetfillopacity{0.700000}%
\pgfsetlinewidth{0.000000pt}%
\definecolor{currentstroke}{rgb}{0.000000,0.000000,0.000000}%
\pgfsetstrokecolor{currentstroke}%
\pgfsetdash{}{0pt}%
\pgfpathmoveto{\pgfqpoint{4.479346in}{1.500148in}}%
\pgfpathlineto{\pgfqpoint{4.493554in}{1.495161in}}%
\pgfpathlineto{\pgfqpoint{4.507768in}{1.490197in}}%
\pgfpathlineto{\pgfqpoint{4.521990in}{1.485257in}}%
\pgfpathlineto{\pgfqpoint{4.536217in}{1.480340in}}%
\pgfpathlineto{\pgfqpoint{4.528267in}{1.475948in}}%
\pgfpathlineto{\pgfqpoint{4.520313in}{1.471845in}}%
\pgfpathlineto{\pgfqpoint{4.512354in}{1.468040in}}%
\pgfpathlineto{\pgfqpoint{4.504389in}{1.464541in}}%
\pgfpathlineto{\pgfqpoint{4.490144in}{1.469785in}}%
\pgfpathlineto{\pgfqpoint{4.475906in}{1.475052in}}%
\pgfpathlineto{\pgfqpoint{4.461674in}{1.480344in}}%
\pgfpathlineto{\pgfqpoint{4.447448in}{1.485659in}}%
\pgfpathlineto{\pgfqpoint{4.455430in}{1.488825in}}%
\pgfpathlineto{\pgfqpoint{4.463407in}{1.492301in}}%
\pgfpathlineto{\pgfqpoint{4.471379in}{1.496078in}}%
\pgfpathlineto{\pgfqpoint{4.479346in}{1.500148in}}%
\pgfpathclose%
\pgfusepath{fill}%
\end{pgfscope}%
\begin{pgfscope}%
\pgfpathrectangle{\pgfqpoint{1.150000in}{0.150000in}}{\pgfqpoint{5.700000in}{5.700000in}}%
\pgfusepath{clip}%
\pgfsetbuttcap%
\pgfsetroundjoin%
\definecolor{currentfill}{rgb}{0.282327,0.094955,0.417331}%
\pgfsetfillcolor{currentfill}%
\pgfsetfillopacity{0.700000}%
\pgfsetlinewidth{0.000000pt}%
\definecolor{currentstroke}{rgb}{0.000000,0.000000,0.000000}%
\pgfsetstrokecolor{currentstroke}%
\pgfsetdash{}{0pt}%
\pgfpathmoveto{\pgfqpoint{4.131791in}{1.594445in}}%
\pgfpathlineto{\pgfqpoint{4.145920in}{1.588306in}}%
\pgfpathlineto{\pgfqpoint{4.160055in}{1.582191in}}%
\pgfpathlineto{\pgfqpoint{4.174195in}{1.576101in}}%
\pgfpathlineto{\pgfqpoint{4.188341in}{1.570035in}}%
\pgfpathlineto{\pgfqpoint{4.180241in}{1.569954in}}%
\pgfpathlineto{\pgfqpoint{4.172133in}{1.570252in}}%
\pgfpathlineto{\pgfqpoint{4.164017in}{1.570939in}}%
\pgfpathlineto{\pgfqpoint{4.155892in}{1.572024in}}%
\pgfpathlineto{\pgfqpoint{4.141722in}{1.578446in}}%
\pgfpathlineto{\pgfqpoint{4.127556in}{1.584891in}}%
\pgfpathlineto{\pgfqpoint{4.113397in}{1.591362in}}%
\pgfpathlineto{\pgfqpoint{4.099243in}{1.597856in}}%
\pgfpathlineto{\pgfqpoint{4.107393in}{1.596410in}}%
\pgfpathlineto{\pgfqpoint{4.115535in}{1.595366in}}%
\pgfpathlineto{\pgfqpoint{4.123667in}{1.594714in}}%
\pgfpathlineto{\pgfqpoint{4.131791in}{1.594445in}}%
\pgfpathclose%
\pgfusepath{fill}%
\end{pgfscope}%
\begin{pgfscope}%
\pgfpathrectangle{\pgfqpoint{1.150000in}{0.150000in}}{\pgfqpoint{5.700000in}{5.700000in}}%
\pgfusepath{clip}%
\pgfsetbuttcap%
\pgfsetroundjoin%
\definecolor{currentfill}{rgb}{0.140536,0.530132,0.555659}%
\pgfsetfillcolor{currentfill}%
\pgfsetfillopacity{0.700000}%
\pgfsetlinewidth{0.000000pt}%
\definecolor{currentstroke}{rgb}{0.000000,0.000000,0.000000}%
\pgfsetstrokecolor{currentstroke}%
\pgfsetdash{}{0pt}%
\pgfpathmoveto{\pgfqpoint{2.558056in}{2.631313in}}%
\pgfpathlineto{\pgfqpoint{2.572016in}{2.620231in}}%
\pgfpathlineto{\pgfqpoint{2.585977in}{2.609188in}}%
\pgfpathlineto{\pgfqpoint{2.599940in}{2.598185in}}%
\pgfpathlineto{\pgfqpoint{2.613905in}{2.587220in}}%
\pgfpathlineto{\pgfqpoint{2.604405in}{2.607553in}}%
\pgfpathlineto{\pgfqpoint{2.594868in}{2.628597in}}%
\pgfpathlineto{\pgfqpoint{2.585294in}{2.650364in}}%
\pgfpathlineto{\pgfqpoint{2.575682in}{2.672870in}}%
\pgfpathlineto{\pgfqpoint{2.561657in}{2.684286in}}%
\pgfpathlineto{\pgfqpoint{2.547633in}{2.695742in}}%
\pgfpathlineto{\pgfqpoint{2.533610in}{2.707237in}}%
\pgfpathlineto{\pgfqpoint{2.519589in}{2.718772in}}%
\pgfpathlineto{\pgfqpoint{2.529263in}{2.695806in}}%
\pgfpathlineto{\pgfqpoint{2.538899in}{2.673584in}}%
\pgfpathlineto{\pgfqpoint{2.548496in}{2.652091in}}%
\pgfpathlineto{\pgfqpoint{2.558056in}{2.631313in}}%
\pgfpathclose%
\pgfusepath{fill}%
\end{pgfscope}%
\begin{pgfscope}%
\pgfpathrectangle{\pgfqpoint{1.150000in}{0.150000in}}{\pgfqpoint{5.700000in}{5.700000in}}%
\pgfusepath{clip}%
\pgfsetbuttcap%
\pgfsetroundjoin%
\definecolor{currentfill}{rgb}{0.201239,0.383670,0.554294}%
\pgfsetfillcolor{currentfill}%
\pgfsetfillopacity{0.700000}%
\pgfsetlinewidth{0.000000pt}%
\definecolor{currentstroke}{rgb}{0.000000,0.000000,0.000000}%
\pgfsetstrokecolor{currentstroke}%
\pgfsetdash{}{0pt}%
\pgfpathmoveto{\pgfqpoint{3.041904in}{2.232461in}}%
\pgfpathlineto{\pgfqpoint{3.055883in}{2.222940in}}%
\pgfpathlineto{\pgfqpoint{3.069865in}{2.213450in}}%
\pgfpathlineto{\pgfqpoint{3.083850in}{2.203992in}}%
\pgfpathlineto{\pgfqpoint{3.097838in}{2.194566in}}%
\pgfpathlineto{\pgfqpoint{3.088887in}{2.208817in}}%
\pgfpathlineto{\pgfqpoint{3.079911in}{2.223689in}}%
\pgfpathlineto{\pgfqpoint{3.070908in}{2.239195in}}%
\pgfpathlineto{\pgfqpoint{3.061878in}{2.255348in}}%
\pgfpathlineto{\pgfqpoint{3.047840in}{2.265200in}}%
\pgfpathlineto{\pgfqpoint{3.033806in}{2.275083in}}%
\pgfpathlineto{\pgfqpoint{3.019774in}{2.284998in}}%
\pgfpathlineto{\pgfqpoint{3.005745in}{2.294945in}}%
\pgfpathlineto{\pgfqpoint{3.014826in}{2.278360in}}%
\pgfpathlineto{\pgfqpoint{3.023879in}{2.262427in}}%
\pgfpathlineto{\pgfqpoint{3.032905in}{2.247131in}}%
\pgfpathlineto{\pgfqpoint{3.041904in}{2.232461in}}%
\pgfpathclose%
\pgfusepath{fill}%
\end{pgfscope}%
\begin{pgfscope}%
\pgfpathrectangle{\pgfqpoint{1.150000in}{0.150000in}}{\pgfqpoint{5.700000in}{5.700000in}}%
\pgfusepath{clip}%
\pgfsetbuttcap%
\pgfsetroundjoin%
\definecolor{currentfill}{rgb}{0.282623,0.140926,0.457517}%
\pgfsetfillcolor{currentfill}%
\pgfsetfillopacity{0.700000}%
\pgfsetlinewidth{0.000000pt}%
\definecolor{currentstroke}{rgb}{0.000000,0.000000,0.000000}%
\pgfsetstrokecolor{currentstroke}%
\pgfsetdash{}{0pt}%
\pgfpathmoveto{\pgfqpoint{3.929804in}{1.677726in}}%
\pgfpathlineto{\pgfqpoint{3.943895in}{1.670933in}}%
\pgfpathlineto{\pgfqpoint{3.957992in}{1.664165in}}%
\pgfpathlineto{\pgfqpoint{3.972094in}{1.657422in}}%
\pgfpathlineto{\pgfqpoint{3.986200in}{1.650704in}}%
\pgfpathlineto{\pgfqpoint{3.977986in}{1.653292in}}%
\pgfpathlineto{\pgfqpoint{3.969761in}{1.656309in}}%
\pgfpathlineto{\pgfqpoint{3.961525in}{1.659766in}}%
\pgfpathlineto{\pgfqpoint{3.953277in}{1.663673in}}%
\pgfpathlineto{\pgfqpoint{3.939141in}{1.670761in}}%
\pgfpathlineto{\pgfqpoint{3.925010in}{1.677875in}}%
\pgfpathlineto{\pgfqpoint{3.910884in}{1.685013in}}%
\pgfpathlineto{\pgfqpoint{3.896763in}{1.692177in}}%
\pgfpathlineto{\pgfqpoint{3.905041in}{1.687894in}}%
\pgfpathlineto{\pgfqpoint{3.913307in}{1.684065in}}%
\pgfpathlineto{\pgfqpoint{3.921561in}{1.680679in}}%
\pgfpathlineto{\pgfqpoint{3.929804in}{1.677726in}}%
\pgfpathclose%
\pgfusepath{fill}%
\end{pgfscope}%
\begin{pgfscope}%
\pgfpathrectangle{\pgfqpoint{1.150000in}{0.150000in}}{\pgfqpoint{5.700000in}{5.700000in}}%
\pgfusepath{clip}%
\pgfsetbuttcap%
\pgfsetroundjoin%
\definecolor{currentfill}{rgb}{0.272594,0.025563,0.353093}%
\pgfsetfillcolor{currentfill}%
\pgfsetfillopacity{0.700000}%
\pgfsetlinewidth{0.000000pt}%
\definecolor{currentstroke}{rgb}{0.000000,0.000000,0.000000}%
\pgfsetstrokecolor{currentstroke}%
\pgfsetdash{}{0pt}%
\pgfpathmoveto{\pgfqpoint{4.770558in}{1.475125in}}%
\pgfpathlineto{\pgfqpoint{4.784846in}{1.471130in}}%
\pgfpathlineto{\pgfqpoint{4.799142in}{1.467159in}}%
\pgfpathlineto{\pgfqpoint{4.813444in}{1.463212in}}%
\pgfpathlineto{\pgfqpoint{4.827754in}{1.459289in}}%
\pgfpathlineto{\pgfqpoint{4.819891in}{1.451695in}}%
\pgfpathlineto{\pgfqpoint{4.812024in}{1.444314in}}%
\pgfpathlineto{\pgfqpoint{4.804155in}{1.437151in}}%
\pgfpathlineto{\pgfqpoint{4.796282in}{1.430214in}}%
\pgfpathlineto{\pgfqpoint{4.781961in}{1.434439in}}%
\pgfpathlineto{\pgfqpoint{4.767646in}{1.438687in}}%
\pgfpathlineto{\pgfqpoint{4.753339in}{1.442958in}}%
\pgfpathlineto{\pgfqpoint{4.739038in}{1.447253in}}%
\pgfpathlineto{\pgfqpoint{4.746923in}{1.453883in}}%
\pgfpathlineto{\pgfqpoint{4.754805in}{1.460744in}}%
\pgfpathlineto{\pgfqpoint{4.762683in}{1.467827in}}%
\pgfpathlineto{\pgfqpoint{4.770558in}{1.475125in}}%
\pgfpathclose%
\pgfusepath{fill}%
\end{pgfscope}%
\begin{pgfscope}%
\pgfpathrectangle{\pgfqpoint{1.150000in}{0.150000in}}{\pgfqpoint{5.700000in}{5.700000in}}%
\pgfusepath{clip}%
\pgfsetbuttcap%
\pgfsetroundjoin%
\definecolor{currentfill}{rgb}{0.279566,0.067836,0.391917}%
\pgfsetfillcolor{currentfill}%
\pgfsetfillopacity{0.700000}%
\pgfsetlinewidth{0.000000pt}%
\definecolor{currentstroke}{rgb}{0.000000,0.000000,0.000000}%
\pgfsetstrokecolor{currentstroke}%
\pgfsetdash{}{0pt}%
\pgfpathmoveto{\pgfqpoint{5.240226in}{1.551556in}}%
\pgfpathlineto{\pgfqpoint{5.254667in}{1.549200in}}%
\pgfpathlineto{\pgfqpoint{5.269117in}{1.546868in}}%
\pgfpathlineto{\pgfqpoint{5.283575in}{1.544560in}}%
\pgfpathlineto{\pgfqpoint{5.275803in}{1.533149in}}%
\pgfpathlineto{\pgfqpoint{5.268029in}{1.521820in}}%
\pgfpathlineto{\pgfqpoint{5.260252in}{1.510581in}}%
\pgfpathlineto{\pgfqpoint{5.252472in}{1.499437in}}%
\pgfpathlineto{\pgfqpoint{5.238009in}{1.501994in}}%
\pgfpathlineto{\pgfqpoint{5.223554in}{1.504575in}}%
\pgfpathlineto{\pgfqpoint{5.209107in}{1.507179in}}%
\pgfpathlineto{\pgfqpoint{5.216891in}{1.518133in}}%
\pgfpathlineto{\pgfqpoint{5.224672in}{1.529184in}}%
\pgfpathlineto{\pgfqpoint{5.232451in}{1.540327in}}%
\pgfpathlineto{\pgfqpoint{5.240226in}{1.551556in}}%
\pgfpathclose%
\pgfusepath{fill}%
\end{pgfscope}%
\begin{pgfscope}%
\pgfpathrectangle{\pgfqpoint{1.150000in}{0.150000in}}{\pgfqpoint{5.700000in}{5.700000in}}%
\pgfusepath{clip}%
\pgfsetbuttcap%
\pgfsetroundjoin%
\definecolor{currentfill}{rgb}{0.248629,0.278775,0.534556}%
\pgfsetfillcolor{currentfill}%
\pgfsetfillopacity{0.700000}%
\pgfsetlinewidth{0.000000pt}%
\definecolor{currentstroke}{rgb}{0.000000,0.000000,0.000000}%
\pgfsetstrokecolor{currentstroke}%
\pgfsetdash{}{0pt}%
\pgfpathmoveto{\pgfqpoint{3.412953in}{1.969446in}}%
\pgfpathlineto{\pgfqpoint{3.426969in}{1.961047in}}%
\pgfpathlineto{\pgfqpoint{3.440990in}{1.952675in}}%
\pgfpathlineto{\pgfqpoint{3.455014in}{1.944331in}}%
\pgfpathlineto{\pgfqpoint{3.469042in}{1.936015in}}%
\pgfpathlineto{\pgfqpoint{3.460442in}{1.945506in}}%
\pgfpathlineto{\pgfqpoint{3.451824in}{1.955545in}}%
\pgfpathlineto{\pgfqpoint{3.443186in}{1.966142in}}%
\pgfpathlineto{\pgfqpoint{3.434529in}{1.977310in}}%
\pgfpathlineto{\pgfqpoint{3.420460in}{1.986031in}}%
\pgfpathlineto{\pgfqpoint{3.406394in}{1.994780in}}%
\pgfpathlineto{\pgfqpoint{3.392333in}{2.003556in}}%
\pgfpathlineto{\pgfqpoint{3.378275in}{2.012361in}}%
\pgfpathlineto{\pgfqpoint{3.386974in}{2.000782in}}%
\pgfpathlineto{\pgfqpoint{3.395654in}{1.989777in}}%
\pgfpathlineto{\pgfqpoint{3.404313in}{1.979336in}}%
\pgfpathlineto{\pgfqpoint{3.412953in}{1.969446in}}%
\pgfpathclose%
\pgfusepath{fill}%
\end{pgfscope}%
\begin{pgfscope}%
\pgfpathrectangle{\pgfqpoint{1.150000in}{0.150000in}}{\pgfqpoint{5.700000in}{5.700000in}}%
\pgfusepath{clip}%
\pgfsetbuttcap%
\pgfsetroundjoin%
\definecolor{currentfill}{rgb}{0.278791,0.062145,0.386592}%
\pgfsetfillcolor{currentfill}%
\pgfsetfillopacity{0.700000}%
\pgfsetlinewidth{0.000000pt}%
\definecolor{currentstroke}{rgb}{0.000000,0.000000,0.000000}%
\pgfsetstrokecolor{currentstroke}%
\pgfsetdash{}{0pt}%
\pgfpathmoveto{\pgfqpoint{4.333863in}{1.529039in}}%
\pgfpathlineto{\pgfqpoint{4.348039in}{1.523533in}}%
\pgfpathlineto{\pgfqpoint{4.362222in}{1.518050in}}%
\pgfpathlineto{\pgfqpoint{4.376411in}{1.512592in}}%
\pgfpathlineto{\pgfqpoint{4.390606in}{1.507157in}}%
\pgfpathlineto{\pgfqpoint{4.382599in}{1.504646in}}%
\pgfpathlineto{\pgfqpoint{4.374586in}{1.502466in}}%
\pgfpathlineto{\pgfqpoint{4.366566in}{1.500627in}}%
\pgfpathlineto{\pgfqpoint{4.358540in}{1.499137in}}%
\pgfpathlineto{\pgfqpoint{4.344325in}{1.504912in}}%
\pgfpathlineto{\pgfqpoint{4.330115in}{1.510712in}}%
\pgfpathlineto{\pgfqpoint{4.315912in}{1.516536in}}%
\pgfpathlineto{\pgfqpoint{4.301714in}{1.522383in}}%
\pgfpathlineto{\pgfqpoint{4.309761in}{1.523527in}}%
\pgfpathlineto{\pgfqpoint{4.317802in}{1.525024in}}%
\pgfpathlineto{\pgfqpoint{4.325835in}{1.526864in}}%
\pgfpathlineto{\pgfqpoint{4.333863in}{1.529039in}}%
\pgfpathclose%
\pgfusepath{fill}%
\end{pgfscope}%
\begin{pgfscope}%
\pgfpathrectangle{\pgfqpoint{1.150000in}{0.150000in}}{\pgfqpoint{5.700000in}{5.700000in}}%
\pgfusepath{clip}%
\pgfsetbuttcap%
\pgfsetroundjoin%
\definecolor{currentfill}{rgb}{0.144759,0.519093,0.556572}%
\pgfsetfillcolor{currentfill}%
\pgfsetfillopacity{0.700000}%
\pgfsetlinewidth{0.000000pt}%
\definecolor{currentstroke}{rgb}{0.000000,0.000000,0.000000}%
\pgfsetstrokecolor{currentstroke}%
\pgfsetdash{}{0pt}%
\pgfpathmoveto{\pgfqpoint{2.613905in}{2.587220in}}%
\pgfpathlineto{\pgfqpoint{2.627871in}{2.576295in}}%
\pgfpathlineto{\pgfqpoint{2.641839in}{2.565407in}}%
\pgfpathlineto{\pgfqpoint{2.655809in}{2.554558in}}%
\pgfpathlineto{\pgfqpoint{2.669781in}{2.543747in}}%
\pgfpathlineto{\pgfqpoint{2.660339in}{2.563636in}}%
\pgfpathlineto{\pgfqpoint{2.650862in}{2.584231in}}%
\pgfpathlineto{\pgfqpoint{2.641349in}{2.605545in}}%
\pgfpathlineto{\pgfqpoint{2.631799in}{2.627592in}}%
\pgfpathlineto{\pgfqpoint{2.617767in}{2.638854in}}%
\pgfpathlineto{\pgfqpoint{2.603737in}{2.650154in}}%
\pgfpathlineto{\pgfqpoint{2.589709in}{2.661493in}}%
\pgfpathlineto{\pgfqpoint{2.575682in}{2.672870in}}%
\pgfpathlineto{\pgfqpoint{2.585294in}{2.650364in}}%
\pgfpathlineto{\pgfqpoint{2.594868in}{2.628597in}}%
\pgfpathlineto{\pgfqpoint{2.604405in}{2.607553in}}%
\pgfpathlineto{\pgfqpoint{2.613905in}{2.587220in}}%
\pgfpathclose%
\pgfusepath{fill}%
\end{pgfscope}%
\begin{pgfscope}%
\pgfpathrectangle{\pgfqpoint{1.150000in}{0.150000in}}{\pgfqpoint{5.700000in}{5.700000in}}%
\pgfusepath{clip}%
\pgfsetbuttcap%
\pgfsetroundjoin%
\definecolor{currentfill}{rgb}{0.273809,0.031497,0.358853}%
\pgfsetfillcolor{currentfill}%
\pgfsetfillopacity{0.700000}%
\pgfsetlinewidth{0.000000pt}%
\definecolor{currentstroke}{rgb}{0.000000,0.000000,0.000000}%
\pgfsetstrokecolor{currentstroke}%
\pgfsetdash{}{0pt}%
\pgfpathmoveto{\pgfqpoint{4.916444in}{1.477316in}}%
\pgfpathlineto{\pgfqpoint{4.930781in}{1.473797in}}%
\pgfpathlineto{\pgfqpoint{4.945124in}{1.470303in}}%
\pgfpathlineto{\pgfqpoint{4.959476in}{1.466832in}}%
\pgfpathlineto{\pgfqpoint{4.973835in}{1.463384in}}%
\pgfpathlineto{\pgfqpoint{4.966004in}{1.454455in}}%
\pgfpathlineto{\pgfqpoint{4.958170in}{1.445701in}}%
\pgfpathlineto{\pgfqpoint{4.950334in}{1.437129in}}%
\pgfpathlineto{\pgfqpoint{4.942495in}{1.428747in}}%
\pgfpathlineto{\pgfqpoint{4.928126in}{1.432483in}}%
\pgfpathlineto{\pgfqpoint{4.913765in}{1.436241in}}%
\pgfpathlineto{\pgfqpoint{4.899412in}{1.440024in}}%
\pgfpathlineto{\pgfqpoint{4.885066in}{1.443830in}}%
\pgfpathlineto{\pgfqpoint{4.892915in}{1.451919in}}%
\pgfpathlineto{\pgfqpoint{4.900761in}{1.460202in}}%
\pgfpathlineto{\pgfqpoint{4.908604in}{1.468669in}}%
\pgfpathlineto{\pgfqpoint{4.916444in}{1.477316in}}%
\pgfpathclose%
\pgfusepath{fill}%
\end{pgfscope}%
\begin{pgfscope}%
\pgfpathrectangle{\pgfqpoint{1.150000in}{0.150000in}}{\pgfqpoint{5.700000in}{5.700000in}}%
\pgfusepath{clip}%
\pgfsetbuttcap%
\pgfsetroundjoin%
\definecolor{currentfill}{rgb}{0.275191,0.194905,0.496005}%
\pgfsetfillcolor{currentfill}%
\pgfsetfillopacity{0.700000}%
\pgfsetlinewidth{0.000000pt}%
\definecolor{currentstroke}{rgb}{0.000000,0.000000,0.000000}%
\pgfsetstrokecolor{currentstroke}%
\pgfsetdash{}{0pt}%
\pgfpathmoveto{\pgfqpoint{3.727684in}{1.780136in}}%
\pgfpathlineto{\pgfqpoint{3.741748in}{1.772664in}}%
\pgfpathlineto{\pgfqpoint{3.755817in}{1.765218in}}%
\pgfpathlineto{\pgfqpoint{3.769890in}{1.757798in}}%
\pgfpathlineto{\pgfqpoint{3.783968in}{1.750405in}}%
\pgfpathlineto{\pgfqpoint{3.775614in}{1.755913in}}%
\pgfpathlineto{\pgfqpoint{3.767247in}{1.761904in}}%
\pgfpathlineto{\pgfqpoint{3.758865in}{1.768388in}}%
\pgfpathlineto{\pgfqpoint{3.750469in}{1.775377in}}%
\pgfpathlineto{\pgfqpoint{3.736357in}{1.783158in}}%
\pgfpathlineto{\pgfqpoint{3.722249in}{1.790964in}}%
\pgfpathlineto{\pgfqpoint{3.708146in}{1.798797in}}%
\pgfpathlineto{\pgfqpoint{3.694047in}{1.806655in}}%
\pgfpathlineto{\pgfqpoint{3.702478in}{1.799274in}}%
\pgfpathlineto{\pgfqpoint{3.710895in}{1.792401in}}%
\pgfpathlineto{\pgfqpoint{3.719296in}{1.786025in}}%
\pgfpathlineto{\pgfqpoint{3.727684in}{1.780136in}}%
\pgfpathclose%
\pgfusepath{fill}%
\end{pgfscope}%
\begin{pgfscope}%
\pgfpathrectangle{\pgfqpoint{1.150000in}{0.150000in}}{\pgfqpoint{5.700000in}{5.700000in}}%
\pgfusepath{clip}%
\pgfsetbuttcap%
\pgfsetroundjoin%
\definecolor{currentfill}{rgb}{0.204903,0.375746,0.553533}%
\pgfsetfillcolor{currentfill}%
\pgfsetfillopacity{0.700000}%
\pgfsetlinewidth{0.000000pt}%
\definecolor{currentstroke}{rgb}{0.000000,0.000000,0.000000}%
\pgfsetstrokecolor{currentstroke}%
\pgfsetdash{}{0pt}%
\pgfpathmoveto{\pgfqpoint{3.097838in}{2.194566in}}%
\pgfpathlineto{\pgfqpoint{3.111829in}{2.185170in}}%
\pgfpathlineto{\pgfqpoint{3.125823in}{2.175806in}}%
\pgfpathlineto{\pgfqpoint{3.139820in}{2.166472in}}%
\pgfpathlineto{\pgfqpoint{3.153820in}{2.157169in}}%
\pgfpathlineto{\pgfqpoint{3.144917in}{2.171002in}}%
\pgfpathlineto{\pgfqpoint{3.135989in}{2.185452in}}%
\pgfpathlineto{\pgfqpoint{3.127035in}{2.200530in}}%
\pgfpathlineto{\pgfqpoint{3.118056in}{2.216251in}}%
\pgfpathlineto{\pgfqpoint{3.104007in}{2.225979in}}%
\pgfpathlineto{\pgfqpoint{3.089961in}{2.235737in}}%
\pgfpathlineto{\pgfqpoint{3.075918in}{2.245527in}}%
\pgfpathlineto{\pgfqpoint{3.061878in}{2.255348in}}%
\pgfpathlineto{\pgfqpoint{3.070908in}{2.239195in}}%
\pgfpathlineto{\pgfqpoint{3.079911in}{2.223689in}}%
\pgfpathlineto{\pgfqpoint{3.088887in}{2.208817in}}%
\pgfpathlineto{\pgfqpoint{3.097838in}{2.194566in}}%
\pgfpathclose%
\pgfusepath{fill}%
\end{pgfscope}%
\begin{pgfscope}%
\pgfpathrectangle{\pgfqpoint{1.150000in}{0.150000in}}{\pgfqpoint{5.700000in}{5.700000in}}%
\pgfusepath{clip}%
\pgfsetbuttcap%
\pgfsetroundjoin%
\definecolor{currentfill}{rgb}{0.277941,0.056324,0.381191}%
\pgfsetfillcolor{currentfill}%
\pgfsetfillopacity{0.700000}%
\pgfsetlinewidth{0.000000pt}%
\definecolor{currentstroke}{rgb}{0.000000,0.000000,0.000000}%
\pgfsetstrokecolor{currentstroke}%
\pgfsetdash{}{0pt}%
\pgfpathmoveto{\pgfqpoint{5.151402in}{1.517833in}}%
\pgfpathlineto{\pgfqpoint{5.165816in}{1.515134in}}%
\pgfpathlineto{\pgfqpoint{5.180238in}{1.512459in}}%
\pgfpathlineto{\pgfqpoint{5.194668in}{1.509807in}}%
\pgfpathlineto{\pgfqpoint{5.209107in}{1.507179in}}%
\pgfpathlineto{\pgfqpoint{5.201320in}{1.496330in}}%
\pgfpathlineto{\pgfqpoint{5.193531in}{1.485591in}}%
\pgfpathlineto{\pgfqpoint{5.185738in}{1.474969in}}%
\pgfpathlineto{\pgfqpoint{5.177944in}{1.464471in}}%
\pgfpathlineto{\pgfqpoint{5.163499in}{1.467361in}}%
\pgfpathlineto{\pgfqpoint{5.149062in}{1.470275in}}%
\pgfpathlineto{\pgfqpoint{5.134633in}{1.473212in}}%
\pgfpathlineto{\pgfqpoint{5.120212in}{1.476173in}}%
\pgfpathlineto{\pgfqpoint{5.128014in}{1.486404in}}%
\pgfpathlineto{\pgfqpoint{5.135812in}{1.496762in}}%
\pgfpathlineto{\pgfqpoint{5.143608in}{1.507241in}}%
\pgfpathlineto{\pgfqpoint{5.151402in}{1.517833in}}%
\pgfpathclose%
\pgfusepath{fill}%
\end{pgfscope}%
\begin{pgfscope}%
\pgfpathrectangle{\pgfqpoint{1.150000in}{0.150000in}}{\pgfqpoint{5.700000in}{5.700000in}}%
\pgfusepath{clip}%
\pgfsetbuttcap%
\pgfsetroundjoin%
\definecolor{currentfill}{rgb}{0.281924,0.089666,0.412415}%
\pgfsetfillcolor{currentfill}%
\pgfsetfillopacity{0.700000}%
\pgfsetlinewidth{0.000000pt}%
\definecolor{currentstroke}{rgb}{0.000000,0.000000,0.000000}%
\pgfsetstrokecolor{currentstroke}%
\pgfsetdash{}{0pt}%
\pgfpathmoveto{\pgfqpoint{4.188341in}{1.570035in}}%
\pgfpathlineto{\pgfqpoint{4.202493in}{1.563994in}}%
\pgfpathlineto{\pgfqpoint{4.216650in}{1.557977in}}%
\pgfpathlineto{\pgfqpoint{4.230813in}{1.551984in}}%
\pgfpathlineto{\pgfqpoint{4.244982in}{1.546016in}}%
\pgfpathlineto{\pgfqpoint{4.236905in}{1.545585in}}%
\pgfpathlineto{\pgfqpoint{4.228821in}{1.545529in}}%
\pgfpathlineto{\pgfqpoint{4.220729in}{1.545858in}}%
\pgfpathlineto{\pgfqpoint{4.212629in}{1.546582in}}%
\pgfpathlineto{\pgfqpoint{4.198436in}{1.552906in}}%
\pgfpathlineto{\pgfqpoint{4.184249in}{1.559255in}}%
\pgfpathlineto{\pgfqpoint{4.170068in}{1.565627in}}%
\pgfpathlineto{\pgfqpoint{4.155892in}{1.572024in}}%
\pgfpathlineto{\pgfqpoint{4.164017in}{1.570939in}}%
\pgfpathlineto{\pgfqpoint{4.172133in}{1.570252in}}%
\pgfpathlineto{\pgfqpoint{4.180241in}{1.569954in}}%
\pgfpathlineto{\pgfqpoint{4.188341in}{1.570035in}}%
\pgfpathclose%
\pgfusepath{fill}%
\end{pgfscope}%
\begin{pgfscope}%
\pgfpathrectangle{\pgfqpoint{1.150000in}{0.150000in}}{\pgfqpoint{5.700000in}{5.700000in}}%
\pgfusepath{clip}%
\pgfsetbuttcap%
\pgfsetroundjoin%
\definecolor{currentfill}{rgb}{0.252194,0.269783,0.531579}%
\pgfsetfillcolor{currentfill}%
\pgfsetfillopacity{0.700000}%
\pgfsetlinewidth{0.000000pt}%
\definecolor{currentstroke}{rgb}{0.000000,0.000000,0.000000}%
\pgfsetstrokecolor{currentstroke}%
\pgfsetdash{}{0pt}%
\pgfpathmoveto{\pgfqpoint{3.469042in}{1.936015in}}%
\pgfpathlineto{\pgfqpoint{3.483074in}{1.927726in}}%
\pgfpathlineto{\pgfqpoint{3.497110in}{1.919465in}}%
\pgfpathlineto{\pgfqpoint{3.511150in}{1.911232in}}%
\pgfpathlineto{\pgfqpoint{3.525194in}{1.903026in}}%
\pgfpathlineto{\pgfqpoint{3.516634in}{1.912119in}}%
\pgfpathlineto{\pgfqpoint{3.508056in}{1.921756in}}%
\pgfpathlineto{\pgfqpoint{3.499459in}{1.931947in}}%
\pgfpathlineto{\pgfqpoint{3.490843in}{1.942705in}}%
\pgfpathlineto{\pgfqpoint{3.476759in}{1.951315in}}%
\pgfpathlineto{\pgfqpoint{3.462679in}{1.959952in}}%
\pgfpathlineto{\pgfqpoint{3.448602in}{1.968618in}}%
\pgfpathlineto{\pgfqpoint{3.434529in}{1.977310in}}%
\pgfpathlineto{\pgfqpoint{3.443186in}{1.966142in}}%
\pgfpathlineto{\pgfqpoint{3.451824in}{1.955545in}}%
\pgfpathlineto{\pgfqpoint{3.460442in}{1.945506in}}%
\pgfpathlineto{\pgfqpoint{3.469042in}{1.936015in}}%
\pgfpathclose%
\pgfusepath{fill}%
\end{pgfscope}%
\begin{pgfscope}%
\pgfpathrectangle{\pgfqpoint{1.150000in}{0.150000in}}{\pgfqpoint{5.700000in}{5.700000in}}%
\pgfusepath{clip}%
\pgfsetbuttcap%
\pgfsetroundjoin%
\definecolor{currentfill}{rgb}{0.283072,0.130895,0.449241}%
\pgfsetfillcolor{currentfill}%
\pgfsetfillopacity{0.700000}%
\pgfsetlinewidth{0.000000pt}%
\definecolor{currentstroke}{rgb}{0.000000,0.000000,0.000000}%
\pgfsetstrokecolor{currentstroke}%
\pgfsetdash{}{0pt}%
\pgfpathmoveto{\pgfqpoint{3.986200in}{1.650704in}}%
\pgfpathlineto{\pgfqpoint{4.000312in}{1.644011in}}%
\pgfpathlineto{\pgfqpoint{4.014429in}{1.637343in}}%
\pgfpathlineto{\pgfqpoint{4.028552in}{1.630700in}}%
\pgfpathlineto{\pgfqpoint{4.042679in}{1.624082in}}%
\pgfpathlineto{\pgfqpoint{4.034493in}{1.626305in}}%
\pgfpathlineto{\pgfqpoint{4.026296in}{1.628953in}}%
\pgfpathlineto{\pgfqpoint{4.018089in}{1.632038in}}%
\pgfpathlineto{\pgfqpoint{4.009871in}{1.635568in}}%
\pgfpathlineto{\pgfqpoint{3.995715in}{1.642557in}}%
\pgfpathlineto{\pgfqpoint{3.981564in}{1.649571in}}%
\pgfpathlineto{\pgfqpoint{3.967418in}{1.656609in}}%
\pgfpathlineto{\pgfqpoint{3.953277in}{1.663673in}}%
\pgfpathlineto{\pgfqpoint{3.961525in}{1.659766in}}%
\pgfpathlineto{\pgfqpoint{3.969761in}{1.656309in}}%
\pgfpathlineto{\pgfqpoint{3.977986in}{1.653292in}}%
\pgfpathlineto{\pgfqpoint{3.986200in}{1.650704in}}%
\pgfpathclose%
\pgfusepath{fill}%
\end{pgfscope}%
\begin{pgfscope}%
\pgfpathrectangle{\pgfqpoint{1.150000in}{0.150000in}}{\pgfqpoint{5.700000in}{5.700000in}}%
\pgfusepath{clip}%
\pgfsetbuttcap%
\pgfsetroundjoin%
\definecolor{currentfill}{rgb}{0.150476,0.504369,0.557430}%
\pgfsetfillcolor{currentfill}%
\pgfsetfillopacity{0.700000}%
\pgfsetlinewidth{0.000000pt}%
\definecolor{currentstroke}{rgb}{0.000000,0.000000,0.000000}%
\pgfsetstrokecolor{currentstroke}%
\pgfsetdash{}{0pt}%
\pgfpathmoveto{\pgfqpoint{2.669781in}{2.543747in}}%
\pgfpathlineto{\pgfqpoint{2.683755in}{2.532973in}}%
\pgfpathlineto{\pgfqpoint{2.697730in}{2.522236in}}%
\pgfpathlineto{\pgfqpoint{2.711708in}{2.511537in}}%
\pgfpathlineto{\pgfqpoint{2.725687in}{2.500874in}}%
\pgfpathlineto{\pgfqpoint{2.716303in}{2.520320in}}%
\pgfpathlineto{\pgfqpoint{2.706884in}{2.540468in}}%
\pgfpathlineto{\pgfqpoint{2.697430in}{2.561330in}}%
\pgfpathlineto{\pgfqpoint{2.687941in}{2.582920in}}%
\pgfpathlineto{\pgfqpoint{2.673903in}{2.594032in}}%
\pgfpathlineto{\pgfqpoint{2.659867in}{2.605181in}}%
\pgfpathlineto{\pgfqpoint{2.645832in}{2.616368in}}%
\pgfpathlineto{\pgfqpoint{2.631799in}{2.627592in}}%
\pgfpathlineto{\pgfqpoint{2.641349in}{2.605545in}}%
\pgfpathlineto{\pgfqpoint{2.650862in}{2.584231in}}%
\pgfpathlineto{\pgfqpoint{2.660339in}{2.563636in}}%
\pgfpathlineto{\pgfqpoint{2.669781in}{2.543747in}}%
\pgfpathclose%
\pgfusepath{fill}%
\end{pgfscope}%
\begin{pgfscope}%
\pgfpathrectangle{\pgfqpoint{1.150000in}{0.150000in}}{\pgfqpoint{5.700000in}{5.700000in}}%
\pgfusepath{clip}%
\pgfsetbuttcap%
\pgfsetroundjoin%
\definecolor{currentfill}{rgb}{0.273809,0.031497,0.358853}%
\pgfsetfillcolor{currentfill}%
\pgfsetfillopacity{0.700000}%
\pgfsetlinewidth{0.000000pt}%
\definecolor{currentstroke}{rgb}{0.000000,0.000000,0.000000}%
\pgfsetstrokecolor{currentstroke}%
\pgfsetdash{}{0pt}%
\pgfpathmoveto{\pgfqpoint{4.681906in}{1.464669in}}%
\pgfpathlineto{\pgfqpoint{4.696178in}{1.460279in}}%
\pgfpathlineto{\pgfqpoint{4.710458in}{1.455914in}}%
\pgfpathlineto{\pgfqpoint{4.724745in}{1.451572in}}%
\pgfpathlineto{\pgfqpoint{4.739038in}{1.447253in}}%
\pgfpathlineto{\pgfqpoint{4.731150in}{1.440861in}}%
\pgfpathlineto{\pgfqpoint{4.723257in}{1.434714in}}%
\pgfpathlineto{\pgfqpoint{4.715361in}{1.428821in}}%
\pgfpathlineto{\pgfqpoint{4.707462in}{1.423191in}}%
\pgfpathlineto{\pgfqpoint{4.693155in}{1.427824in}}%
\pgfpathlineto{\pgfqpoint{4.678854in}{1.432480in}}%
\pgfpathlineto{\pgfqpoint{4.664561in}{1.437160in}}%
\pgfpathlineto{\pgfqpoint{4.650274in}{1.441863in}}%
\pgfpathlineto{\pgfqpoint{4.658188in}{1.447174in}}%
\pgfpathlineto{\pgfqpoint{4.666097in}{1.452751in}}%
\pgfpathlineto{\pgfqpoint{4.674004in}{1.458585in}}%
\pgfpathlineto{\pgfqpoint{4.681906in}{1.464669in}}%
\pgfpathclose%
\pgfusepath{fill}%
\end{pgfscope}%
\begin{pgfscope}%
\pgfpathrectangle{\pgfqpoint{1.150000in}{0.150000in}}{\pgfqpoint{5.700000in}{5.700000in}}%
\pgfusepath{clip}%
\pgfsetbuttcap%
\pgfsetroundjoin%
\definecolor{currentfill}{rgb}{0.274952,0.037752,0.364543}%
\pgfsetfillcolor{currentfill}%
\pgfsetfillopacity{0.700000}%
\pgfsetlinewidth{0.000000pt}%
\definecolor{currentstroke}{rgb}{0.000000,0.000000,0.000000}%
\pgfsetstrokecolor{currentstroke}%
\pgfsetdash{}{0pt}%
\pgfpathmoveto{\pgfqpoint{4.536217in}{1.480340in}}%
\pgfpathlineto{\pgfqpoint{4.550451in}{1.475448in}}%
\pgfpathlineto{\pgfqpoint{4.564692in}{1.470579in}}%
\pgfpathlineto{\pgfqpoint{4.578939in}{1.465734in}}%
\pgfpathlineto{\pgfqpoint{4.593193in}{1.460912in}}%
\pgfpathlineto{\pgfqpoint{4.585259in}{1.456198in}}%
\pgfpathlineto{\pgfqpoint{4.577321in}{1.451769in}}%
\pgfpathlineto{\pgfqpoint{4.569379in}{1.447635in}}%
\pgfpathlineto{\pgfqpoint{4.561432in}{1.443803in}}%
\pgfpathlineto{\pgfqpoint{4.547162in}{1.448952in}}%
\pgfpathlineto{\pgfqpoint{4.532898in}{1.454125in}}%
\pgfpathlineto{\pgfqpoint{4.518640in}{1.459321in}}%
\pgfpathlineto{\pgfqpoint{4.504389in}{1.464541in}}%
\pgfpathlineto{\pgfqpoint{4.512354in}{1.468040in}}%
\pgfpathlineto{\pgfqpoint{4.520313in}{1.471845in}}%
\pgfpathlineto{\pgfqpoint{4.528267in}{1.475948in}}%
\pgfpathlineto{\pgfqpoint{4.536217in}{1.480340in}}%
\pgfpathclose%
\pgfusepath{fill}%
\end{pgfscope}%
\begin{pgfscope}%
\pgfpathrectangle{\pgfqpoint{1.150000in}{0.150000in}}{\pgfqpoint{5.700000in}{5.700000in}}%
\pgfusepath{clip}%
\pgfsetbuttcap%
\pgfsetroundjoin%
\definecolor{currentfill}{rgb}{0.276022,0.044167,0.370164}%
\pgfsetfillcolor{currentfill}%
\pgfsetfillopacity{0.700000}%
\pgfsetlinewidth{0.000000pt}%
\definecolor{currentstroke}{rgb}{0.000000,0.000000,0.000000}%
\pgfsetstrokecolor{currentstroke}%
\pgfsetdash{}{0pt}%
\pgfpathmoveto{\pgfqpoint{5.062608in}{1.488251in}}%
\pgfpathlineto{\pgfqpoint{5.076997in}{1.485196in}}%
\pgfpathlineto{\pgfqpoint{5.091394in}{1.482165in}}%
\pgfpathlineto{\pgfqpoint{5.105799in}{1.479157in}}%
\pgfpathlineto{\pgfqpoint{5.120212in}{1.476173in}}%
\pgfpathlineto{\pgfqpoint{5.112408in}{1.466075in}}%
\pgfpathlineto{\pgfqpoint{5.104602in}{1.456117in}}%
\pgfpathlineto{\pgfqpoint{5.096793in}{1.446307in}}%
\pgfpathlineto{\pgfqpoint{5.088981in}{1.436650in}}%
\pgfpathlineto{\pgfqpoint{5.074561in}{1.439910in}}%
\pgfpathlineto{\pgfqpoint{5.060148in}{1.443193in}}%
\pgfpathlineto{\pgfqpoint{5.045743in}{1.446499in}}%
\pgfpathlineto{\pgfqpoint{5.031346in}{1.449829in}}%
\pgfpathlineto{\pgfqpoint{5.039166in}{1.459205in}}%
\pgfpathlineto{\pgfqpoint{5.046983in}{1.468739in}}%
\pgfpathlineto{\pgfqpoint{5.054797in}{1.478424in}}%
\pgfpathlineto{\pgfqpoint{5.062608in}{1.488251in}}%
\pgfpathclose%
\pgfusepath{fill}%
\end{pgfscope}%
\begin{pgfscope}%
\pgfpathrectangle{\pgfqpoint{1.150000in}{0.150000in}}{\pgfqpoint{5.700000in}{5.700000in}}%
\pgfusepath{clip}%
\pgfsetbuttcap%
\pgfsetroundjoin%
\definecolor{currentfill}{rgb}{0.272594,0.025563,0.353093}%
\pgfsetfillcolor{currentfill}%
\pgfsetfillopacity{0.700000}%
\pgfsetlinewidth{0.000000pt}%
\definecolor{currentstroke}{rgb}{0.000000,0.000000,0.000000}%
\pgfsetstrokecolor{currentstroke}%
\pgfsetdash{}{0pt}%
\pgfpathmoveto{\pgfqpoint{4.827754in}{1.459289in}}%
\pgfpathlineto{\pgfqpoint{4.842071in}{1.455389in}}%
\pgfpathlineto{\pgfqpoint{4.856395in}{1.451512in}}%
\pgfpathlineto{\pgfqpoint{4.870727in}{1.447659in}}%
\pgfpathlineto{\pgfqpoint{4.885066in}{1.443830in}}%
\pgfpathlineto{\pgfqpoint{4.877213in}{1.435941in}}%
\pgfpathlineto{\pgfqpoint{4.869358in}{1.428260in}}%
\pgfpathlineto{\pgfqpoint{4.861500in}{1.420794in}}%
\pgfpathlineto{\pgfqpoint{4.853640in}{1.413552in}}%
\pgfpathlineto{\pgfqpoint{4.839290in}{1.417682in}}%
\pgfpathlineto{\pgfqpoint{4.824947in}{1.421836in}}%
\pgfpathlineto{\pgfqpoint{4.810611in}{1.426013in}}%
\pgfpathlineto{\pgfqpoint{4.796282in}{1.430214in}}%
\pgfpathlineto{\pgfqpoint{4.804155in}{1.437151in}}%
\pgfpathlineto{\pgfqpoint{4.812024in}{1.444314in}}%
\pgfpathlineto{\pgfqpoint{4.819891in}{1.451695in}}%
\pgfpathlineto{\pgfqpoint{4.827754in}{1.459289in}}%
\pgfpathclose%
\pgfusepath{fill}%
\end{pgfscope}%
\begin{pgfscope}%
\pgfpathrectangle{\pgfqpoint{1.150000in}{0.150000in}}{\pgfqpoint{5.700000in}{5.700000in}}%
\pgfusepath{clip}%
\pgfsetbuttcap%
\pgfsetroundjoin%
\definecolor{currentfill}{rgb}{0.210503,0.363727,0.552206}%
\pgfsetfillcolor{currentfill}%
\pgfsetfillopacity{0.700000}%
\pgfsetlinewidth{0.000000pt}%
\definecolor{currentstroke}{rgb}{0.000000,0.000000,0.000000}%
\pgfsetstrokecolor{currentstroke}%
\pgfsetdash{}{0pt}%
\pgfpathmoveto{\pgfqpoint{3.153820in}{2.157169in}}%
\pgfpathlineto{\pgfqpoint{3.167823in}{2.147896in}}%
\pgfpathlineto{\pgfqpoint{3.181830in}{2.138654in}}%
\pgfpathlineto{\pgfqpoint{3.195840in}{2.129441in}}%
\pgfpathlineto{\pgfqpoint{3.209853in}{2.120259in}}%
\pgfpathlineto{\pgfqpoint{3.200997in}{2.133675in}}%
\pgfpathlineto{\pgfqpoint{3.192116in}{2.147702in}}%
\pgfpathlineto{\pgfqpoint{3.183211in}{2.162355in}}%
\pgfpathlineto{\pgfqpoint{3.174281in}{2.177645in}}%
\pgfpathlineto{\pgfqpoint{3.160220in}{2.187251in}}%
\pgfpathlineto{\pgfqpoint{3.146162in}{2.196887in}}%
\pgfpathlineto{\pgfqpoint{3.132108in}{2.206554in}}%
\pgfpathlineto{\pgfqpoint{3.118056in}{2.216251in}}%
\pgfpathlineto{\pgfqpoint{3.127035in}{2.200530in}}%
\pgfpathlineto{\pgfqpoint{3.135989in}{2.185452in}}%
\pgfpathlineto{\pgfqpoint{3.144917in}{2.171002in}}%
\pgfpathlineto{\pgfqpoint{3.153820in}{2.157169in}}%
\pgfpathclose%
\pgfusepath{fill}%
\end{pgfscope}%
\begin{pgfscope}%
\pgfpathrectangle{\pgfqpoint{1.150000in}{0.150000in}}{\pgfqpoint{5.700000in}{5.700000in}}%
\pgfusepath{clip}%
\pgfsetbuttcap%
\pgfsetroundjoin%
\definecolor{currentfill}{rgb}{0.278791,0.062145,0.386592}%
\pgfsetfillcolor{currentfill}%
\pgfsetfillopacity{0.700000}%
\pgfsetlinewidth{0.000000pt}%
\definecolor{currentstroke}{rgb}{0.000000,0.000000,0.000000}%
\pgfsetstrokecolor{currentstroke}%
\pgfsetdash{}{0pt}%
\pgfpathmoveto{\pgfqpoint{4.390606in}{1.507157in}}%
\pgfpathlineto{\pgfqpoint{4.404807in}{1.501747in}}%
\pgfpathlineto{\pgfqpoint{4.419015in}{1.496360in}}%
\pgfpathlineto{\pgfqpoint{4.433228in}{1.490997in}}%
\pgfpathlineto{\pgfqpoint{4.447448in}{1.485659in}}%
\pgfpathlineto{\pgfqpoint{4.439460in}{1.482811in}}%
\pgfpathlineto{\pgfqpoint{4.431466in}{1.480292in}}%
\pgfpathlineto{\pgfqpoint{4.423467in}{1.478109in}}%
\pgfpathlineto{\pgfqpoint{4.415462in}{1.476273in}}%
\pgfpathlineto{\pgfqpoint{4.401222in}{1.481953in}}%
\pgfpathlineto{\pgfqpoint{4.386989in}{1.487657in}}%
\pgfpathlineto{\pgfqpoint{4.372762in}{1.493385in}}%
\pgfpathlineto{\pgfqpoint{4.358540in}{1.499137in}}%
\pgfpathlineto{\pgfqpoint{4.366566in}{1.500627in}}%
\pgfpathlineto{\pgfqpoint{4.374586in}{1.502466in}}%
\pgfpathlineto{\pgfqpoint{4.382599in}{1.504646in}}%
\pgfpathlineto{\pgfqpoint{4.390606in}{1.507157in}}%
\pgfpathclose%
\pgfusepath{fill}%
\end{pgfscope}%
\begin{pgfscope}%
\pgfpathrectangle{\pgfqpoint{1.150000in}{0.150000in}}{\pgfqpoint{5.700000in}{5.700000in}}%
\pgfusepath{clip}%
\pgfsetbuttcap%
\pgfsetroundjoin%
\definecolor{currentfill}{rgb}{0.277134,0.185228,0.489898}%
\pgfsetfillcolor{currentfill}%
\pgfsetfillopacity{0.700000}%
\pgfsetlinewidth{0.000000pt}%
\definecolor{currentstroke}{rgb}{0.000000,0.000000,0.000000}%
\pgfsetstrokecolor{currentstroke}%
\pgfsetdash{}{0pt}%
\pgfpathmoveto{\pgfqpoint{3.783968in}{1.750405in}}%
\pgfpathlineto{\pgfqpoint{3.798051in}{1.743037in}}%
\pgfpathlineto{\pgfqpoint{3.812138in}{1.735694in}}%
\pgfpathlineto{\pgfqpoint{3.826231in}{1.728378in}}%
\pgfpathlineto{\pgfqpoint{3.840328in}{1.721087in}}%
\pgfpathlineto{\pgfqpoint{3.832006in}{1.726214in}}%
\pgfpathlineto{\pgfqpoint{3.823672in}{1.731821in}}%
\pgfpathlineto{\pgfqpoint{3.815324in}{1.737917in}}%
\pgfpathlineto{\pgfqpoint{3.806963in}{1.744513in}}%
\pgfpathlineto{\pgfqpoint{3.792832in}{1.752191in}}%
\pgfpathlineto{\pgfqpoint{3.778707in}{1.759894in}}%
\pgfpathlineto{\pgfqpoint{3.764586in}{1.767623in}}%
\pgfpathlineto{\pgfqpoint{3.750469in}{1.775377in}}%
\pgfpathlineto{\pgfqpoint{3.758865in}{1.768388in}}%
\pgfpathlineto{\pgfqpoint{3.767247in}{1.761904in}}%
\pgfpathlineto{\pgfqpoint{3.775614in}{1.755913in}}%
\pgfpathlineto{\pgfqpoint{3.783968in}{1.750405in}}%
\pgfpathclose%
\pgfusepath{fill}%
\end{pgfscope}%
\begin{pgfscope}%
\pgfpathrectangle{\pgfqpoint{1.150000in}{0.150000in}}{\pgfqpoint{5.700000in}{5.700000in}}%
\pgfusepath{clip}%
\pgfsetbuttcap%
\pgfsetroundjoin%
\definecolor{currentfill}{rgb}{0.154815,0.493313,0.557840}%
\pgfsetfillcolor{currentfill}%
\pgfsetfillopacity{0.700000}%
\pgfsetlinewidth{0.000000pt}%
\definecolor{currentstroke}{rgb}{0.000000,0.000000,0.000000}%
\pgfsetstrokecolor{currentstroke}%
\pgfsetdash{}{0pt}%
\pgfpathmoveto{\pgfqpoint{2.725687in}{2.500874in}}%
\pgfpathlineto{\pgfqpoint{2.739668in}{2.490247in}}%
\pgfpathlineto{\pgfqpoint{2.753652in}{2.479657in}}%
\pgfpathlineto{\pgfqpoint{2.767637in}{2.469102in}}%
\pgfpathlineto{\pgfqpoint{2.781625in}{2.458584in}}%
\pgfpathlineto{\pgfqpoint{2.772297in}{2.477589in}}%
\pgfpathlineto{\pgfqpoint{2.762936in}{2.497290in}}%
\pgfpathlineto{\pgfqpoint{2.753542in}{2.517700in}}%
\pgfpathlineto{\pgfqpoint{2.744112in}{2.538834in}}%
\pgfpathlineto{\pgfqpoint{2.730067in}{2.549802in}}%
\pgfpathlineto{\pgfqpoint{2.716023in}{2.560805in}}%
\pgfpathlineto{\pgfqpoint{2.701981in}{2.571844in}}%
\pgfpathlineto{\pgfqpoint{2.687941in}{2.582920in}}%
\pgfpathlineto{\pgfqpoint{2.697430in}{2.561330in}}%
\pgfpathlineto{\pgfqpoint{2.706884in}{2.540468in}}%
\pgfpathlineto{\pgfqpoint{2.716303in}{2.520320in}}%
\pgfpathlineto{\pgfqpoint{2.725687in}{2.500874in}}%
\pgfpathclose%
\pgfusepath{fill}%
\end{pgfscope}%
\begin{pgfscope}%
\pgfpathrectangle{\pgfqpoint{1.150000in}{0.150000in}}{\pgfqpoint{5.700000in}{5.700000in}}%
\pgfusepath{clip}%
\pgfsetbuttcap%
\pgfsetroundjoin%
\definecolor{currentfill}{rgb}{0.220124,0.725509,0.466226}%
\pgfsetfillcolor{currentfill}%
\pgfsetfillopacity{0.700000}%
\pgfsetlinewidth{0.000000pt}%
\definecolor{currentstroke}{rgb}{0.000000,0.000000,0.000000}%
\pgfsetstrokecolor{currentstroke}%
\pgfsetdash{}{0pt}%
\pgfpathmoveto{\pgfqpoint{2.015312in}{3.163871in}}%
\pgfpathlineto{\pgfqpoint{2.029318in}{3.150628in}}%
\pgfpathlineto{\pgfqpoint{2.043324in}{3.137441in}}%
\pgfpathlineto{\pgfqpoint{2.057329in}{3.124310in}}%
\pgfpathlineto{\pgfqpoint{2.071334in}{3.111233in}}%
\pgfpathlineto{\pgfqpoint{2.061074in}{3.138729in}}%
\pgfpathlineto{\pgfqpoint{2.050764in}{3.167039in}}%
\pgfpathlineto{\pgfqpoint{2.040403in}{3.196178in}}%
\pgfpathlineto{\pgfqpoint{2.029990in}{3.226163in}}%
\pgfpathlineto{\pgfqpoint{2.015911in}{3.239727in}}%
\pgfpathlineto{\pgfqpoint{2.001831in}{3.253346in}}%
\pgfpathlineto{\pgfqpoint{1.987751in}{3.267021in}}%
\pgfpathlineto{\pgfqpoint{1.973670in}{3.280753in}}%
\pgfpathlineto{\pgfqpoint{1.984160in}{3.250272in}}%
\pgfpathlineto{\pgfqpoint{1.994596in}{3.220642in}}%
\pgfpathlineto{\pgfqpoint{2.004980in}{3.191846in}}%
\pgfpathlineto{\pgfqpoint{2.015312in}{3.163871in}}%
\pgfpathclose%
\pgfusepath{fill}%
\end{pgfscope}%
\begin{pgfscope}%
\pgfpathrectangle{\pgfqpoint{1.150000in}{0.150000in}}{\pgfqpoint{5.700000in}{5.700000in}}%
\pgfusepath{clip}%
\pgfsetbuttcap%
\pgfsetroundjoin%
\definecolor{currentfill}{rgb}{0.255645,0.260703,0.528312}%
\pgfsetfillcolor{currentfill}%
\pgfsetfillopacity{0.700000}%
\pgfsetlinewidth{0.000000pt}%
\definecolor{currentstroke}{rgb}{0.000000,0.000000,0.000000}%
\pgfsetstrokecolor{currentstroke}%
\pgfsetdash{}{0pt}%
\pgfpathmoveto{\pgfqpoint{3.525194in}{1.903026in}}%
\pgfpathlineto{\pgfqpoint{3.539242in}{1.894847in}}%
\pgfpathlineto{\pgfqpoint{3.553294in}{1.886696in}}%
\pgfpathlineto{\pgfqpoint{3.567351in}{1.878572in}}%
\pgfpathlineto{\pgfqpoint{3.581411in}{1.870474in}}%
\pgfpathlineto{\pgfqpoint{3.572889in}{1.879170in}}%
\pgfpathlineto{\pgfqpoint{3.564351in}{1.888404in}}%
\pgfpathlineto{\pgfqpoint{3.555794in}{1.898190in}}%
\pgfpathlineto{\pgfqpoint{3.547220in}{1.908537in}}%
\pgfpathlineto{\pgfqpoint{3.533120in}{1.917038in}}%
\pgfpathlineto{\pgfqpoint{3.519024in}{1.925567in}}%
\pgfpathlineto{\pgfqpoint{3.504932in}{1.934122in}}%
\pgfpathlineto{\pgfqpoint{3.490843in}{1.942705in}}%
\pgfpathlineto{\pgfqpoint{3.499459in}{1.931947in}}%
\pgfpathlineto{\pgfqpoint{3.508056in}{1.921756in}}%
\pgfpathlineto{\pgfqpoint{3.516634in}{1.912119in}}%
\pgfpathlineto{\pgfqpoint{3.525194in}{1.903026in}}%
\pgfpathclose%
\pgfusepath{fill}%
\end{pgfscope}%
\begin{pgfscope}%
\pgfpathrectangle{\pgfqpoint{1.150000in}{0.150000in}}{\pgfqpoint{5.700000in}{5.700000in}}%
\pgfusepath{clip}%
\pgfsetbuttcap%
\pgfsetroundjoin%
\definecolor{currentfill}{rgb}{0.273809,0.031497,0.358853}%
\pgfsetfillcolor{currentfill}%
\pgfsetfillopacity{0.700000}%
\pgfsetlinewidth{0.000000pt}%
\definecolor{currentstroke}{rgb}{0.000000,0.000000,0.000000}%
\pgfsetstrokecolor{currentstroke}%
\pgfsetdash{}{0pt}%
\pgfpathmoveto{\pgfqpoint{4.973835in}{1.463384in}}%
\pgfpathlineto{\pgfqpoint{4.988201in}{1.459960in}}%
\pgfpathlineto{\pgfqpoint{5.002575in}{1.456559in}}%
\pgfpathlineto{\pgfqpoint{5.016957in}{1.453182in}}%
\pgfpathlineto{\pgfqpoint{5.031346in}{1.449829in}}%
\pgfpathlineto{\pgfqpoint{5.023524in}{1.440617in}}%
\pgfpathlineto{\pgfqpoint{5.015699in}{1.431577in}}%
\pgfpathlineto{\pgfqpoint{5.007872in}{1.422715in}}%
\pgfpathlineto{\pgfqpoint{5.000043in}{1.414040in}}%
\pgfpathlineto{\pgfqpoint{4.985644in}{1.417682in}}%
\pgfpathlineto{\pgfqpoint{4.971254in}{1.421347in}}%
\pgfpathlineto{\pgfqpoint{4.956870in}{1.425035in}}%
\pgfpathlineto{\pgfqpoint{4.942495in}{1.428747in}}%
\pgfpathlineto{\pgfqpoint{4.950334in}{1.437129in}}%
\pgfpathlineto{\pgfqpoint{4.958170in}{1.445701in}}%
\pgfpathlineto{\pgfqpoint{4.966004in}{1.454455in}}%
\pgfpathlineto{\pgfqpoint{4.973835in}{1.463384in}}%
\pgfpathclose%
\pgfusepath{fill}%
\end{pgfscope}%
\begin{pgfscope}%
\pgfpathrectangle{\pgfqpoint{1.150000in}{0.150000in}}{\pgfqpoint{5.700000in}{5.700000in}}%
\pgfusepath{clip}%
\pgfsetbuttcap%
\pgfsetroundjoin%
\definecolor{currentfill}{rgb}{0.277941,0.056324,0.381191}%
\pgfsetfillcolor{currentfill}%
\pgfsetfillopacity{0.700000}%
\pgfsetlinewidth{0.000000pt}%
\definecolor{currentstroke}{rgb}{0.000000,0.000000,0.000000}%
\pgfsetstrokecolor{currentstroke}%
\pgfsetdash{}{0pt}%
\pgfpathmoveto{\pgfqpoint{5.209107in}{1.507179in}}%
\pgfpathlineto{\pgfqpoint{5.223554in}{1.504575in}}%
\pgfpathlineto{\pgfqpoint{5.238009in}{1.501994in}}%
\pgfpathlineto{\pgfqpoint{5.252472in}{1.499437in}}%
\pgfpathlineto{\pgfqpoint{5.244690in}{1.488395in}}%
\pgfpathlineto{\pgfqpoint{5.236904in}{1.477461in}}%
\pgfpathlineto{\pgfqpoint{5.229117in}{1.466641in}}%
\pgfpathlineto{\pgfqpoint{5.221327in}{1.455942in}}%
\pgfpathlineto{\pgfqpoint{5.206858in}{1.458762in}}%
\pgfpathlineto{\pgfqpoint{5.192397in}{1.461605in}}%
\pgfpathlineto{\pgfqpoint{5.177944in}{1.464471in}}%
\pgfpathlineto{\pgfqpoint{5.185738in}{1.474969in}}%
\pgfpathlineto{\pgfqpoint{5.193531in}{1.485591in}}%
\pgfpathlineto{\pgfqpoint{5.201320in}{1.496330in}}%
\pgfpathlineto{\pgfqpoint{5.209107in}{1.507179in}}%
\pgfpathclose%
\pgfusepath{fill}%
\end{pgfscope}%
\begin{pgfscope}%
\pgfpathrectangle{\pgfqpoint{1.150000in}{0.150000in}}{\pgfqpoint{5.700000in}{5.700000in}}%
\pgfusepath{clip}%
\pgfsetbuttcap%
\pgfsetroundjoin%
\definecolor{currentfill}{rgb}{0.191090,0.708366,0.482284}%
\pgfsetfillcolor{currentfill}%
\pgfsetfillopacity{0.700000}%
\pgfsetlinewidth{0.000000pt}%
\definecolor{currentstroke}{rgb}{0.000000,0.000000,0.000000}%
\pgfsetstrokecolor{currentstroke}%
\pgfsetdash{}{0pt}%
\pgfpathmoveto{\pgfqpoint{2.071334in}{3.111233in}}%
\pgfpathlineto{\pgfqpoint{2.085338in}{3.098211in}}%
\pgfpathlineto{\pgfqpoint{2.099342in}{3.085243in}}%
\pgfpathlineto{\pgfqpoint{2.113345in}{3.072329in}}%
\pgfpathlineto{\pgfqpoint{2.127348in}{3.059467in}}%
\pgfpathlineto{\pgfqpoint{2.117160in}{3.086485in}}%
\pgfpathlineto{\pgfqpoint{2.106924in}{3.114312in}}%
\pgfpathlineto{\pgfqpoint{2.096637in}{3.142963in}}%
\pgfpathlineto{\pgfqpoint{2.086299in}{3.172453in}}%
\pgfpathlineto{\pgfqpoint{2.072222in}{3.185800in}}%
\pgfpathlineto{\pgfqpoint{2.058145in}{3.199201in}}%
\pgfpathlineto{\pgfqpoint{2.044068in}{3.212655in}}%
\pgfpathlineto{\pgfqpoint{2.029990in}{3.226163in}}%
\pgfpathlineto{\pgfqpoint{2.040403in}{3.196178in}}%
\pgfpathlineto{\pgfqpoint{2.050764in}{3.167039in}}%
\pgfpathlineto{\pgfqpoint{2.061074in}{3.138729in}}%
\pgfpathlineto{\pgfqpoint{2.071334in}{3.111233in}}%
\pgfpathclose%
\pgfusepath{fill}%
\end{pgfscope}%
\begin{pgfscope}%
\pgfpathrectangle{\pgfqpoint{1.150000in}{0.150000in}}{\pgfqpoint{5.700000in}{5.700000in}}%
\pgfusepath{clip}%
\pgfsetbuttcap%
\pgfsetroundjoin%
\definecolor{currentfill}{rgb}{0.283187,0.125848,0.444960}%
\pgfsetfillcolor{currentfill}%
\pgfsetfillopacity{0.700000}%
\pgfsetlinewidth{0.000000pt}%
\definecolor{currentstroke}{rgb}{0.000000,0.000000,0.000000}%
\pgfsetstrokecolor{currentstroke}%
\pgfsetdash{}{0pt}%
\pgfpathmoveto{\pgfqpoint{4.042679in}{1.624082in}}%
\pgfpathlineto{\pgfqpoint{4.056812in}{1.617488in}}%
\pgfpathlineto{\pgfqpoint{4.070950in}{1.610920in}}%
\pgfpathlineto{\pgfqpoint{4.085094in}{1.604376in}}%
\pgfpathlineto{\pgfqpoint{4.099243in}{1.597856in}}%
\pgfpathlineto{\pgfqpoint{4.091083in}{1.599715in}}%
\pgfpathlineto{\pgfqpoint{4.082914in}{1.601995in}}%
\pgfpathlineto{\pgfqpoint{4.074735in}{1.604707in}}%
\pgfpathlineto{\pgfqpoint{4.066546in}{1.607861in}}%
\pgfpathlineto{\pgfqpoint{4.052370in}{1.614751in}}%
\pgfpathlineto{\pgfqpoint{4.038198in}{1.621665in}}%
\pgfpathlineto{\pgfqpoint{4.024032in}{1.628604in}}%
\pgfpathlineto{\pgfqpoint{4.009871in}{1.635568in}}%
\pgfpathlineto{\pgfqpoint{4.018089in}{1.632038in}}%
\pgfpathlineto{\pgfqpoint{4.026296in}{1.628953in}}%
\pgfpathlineto{\pgfqpoint{4.034493in}{1.626305in}}%
\pgfpathlineto{\pgfqpoint{4.042679in}{1.624082in}}%
\pgfpathclose%
\pgfusepath{fill}%
\end{pgfscope}%
\begin{pgfscope}%
\pgfpathrectangle{\pgfqpoint{1.150000in}{0.150000in}}{\pgfqpoint{5.700000in}{5.700000in}}%
\pgfusepath{clip}%
\pgfsetbuttcap%
\pgfsetroundjoin%
\definecolor{currentfill}{rgb}{0.281446,0.084320,0.407414}%
\pgfsetfillcolor{currentfill}%
\pgfsetfillopacity{0.700000}%
\pgfsetlinewidth{0.000000pt}%
\definecolor{currentstroke}{rgb}{0.000000,0.000000,0.000000}%
\pgfsetstrokecolor{currentstroke}%
\pgfsetdash{}{0pt}%
\pgfpathmoveto{\pgfqpoint{4.244982in}{1.546016in}}%
\pgfpathlineto{\pgfqpoint{4.259156in}{1.540071in}}%
\pgfpathlineto{\pgfqpoint{4.273336in}{1.534151in}}%
\pgfpathlineto{\pgfqpoint{4.287522in}{1.528255in}}%
\pgfpathlineto{\pgfqpoint{4.301714in}{1.522383in}}%
\pgfpathlineto{\pgfqpoint{4.293660in}{1.521602in}}%
\pgfpathlineto{\pgfqpoint{4.285599in}{1.521193in}}%
\pgfpathlineto{\pgfqpoint{4.277531in}{1.521165in}}%
\pgfpathlineto{\pgfqpoint{4.269455in}{1.521528in}}%
\pgfpathlineto{\pgfqpoint{4.255240in}{1.527756in}}%
\pgfpathlineto{\pgfqpoint{4.241031in}{1.534007in}}%
\pgfpathlineto{\pgfqpoint{4.226827in}{1.540283in}}%
\pgfpathlineto{\pgfqpoint{4.212629in}{1.546582in}}%
\pgfpathlineto{\pgfqpoint{4.220729in}{1.545858in}}%
\pgfpathlineto{\pgfqpoint{4.228821in}{1.545529in}}%
\pgfpathlineto{\pgfqpoint{4.236905in}{1.545585in}}%
\pgfpathlineto{\pgfqpoint{4.244982in}{1.546016in}}%
\pgfpathclose%
\pgfusepath{fill}%
\end{pgfscope}%
\begin{pgfscope}%
\pgfpathrectangle{\pgfqpoint{1.150000in}{0.150000in}}{\pgfqpoint{5.700000in}{5.700000in}}%
\pgfusepath{clip}%
\pgfsetbuttcap%
\pgfsetroundjoin%
\definecolor{currentfill}{rgb}{0.216210,0.351535,0.550627}%
\pgfsetfillcolor{currentfill}%
\pgfsetfillopacity{0.700000}%
\pgfsetlinewidth{0.000000pt}%
\definecolor{currentstroke}{rgb}{0.000000,0.000000,0.000000}%
\pgfsetstrokecolor{currentstroke}%
\pgfsetdash{}{0pt}%
\pgfpathmoveto{\pgfqpoint{3.209853in}{2.120259in}}%
\pgfpathlineto{\pgfqpoint{3.223869in}{2.111107in}}%
\pgfpathlineto{\pgfqpoint{3.237889in}{2.101984in}}%
\pgfpathlineto{\pgfqpoint{3.251912in}{2.092891in}}%
\pgfpathlineto{\pgfqpoint{3.265938in}{2.083827in}}%
\pgfpathlineto{\pgfqpoint{3.257128in}{2.096826in}}%
\pgfpathlineto{\pgfqpoint{3.248294in}{2.110432in}}%
\pgfpathlineto{\pgfqpoint{3.239437in}{2.124659in}}%
\pgfpathlineto{\pgfqpoint{3.230556in}{2.139518in}}%
\pgfpathlineto{\pgfqpoint{3.216482in}{2.149005in}}%
\pgfpathlineto{\pgfqpoint{3.202412in}{2.158522in}}%
\pgfpathlineto{\pgfqpoint{3.188345in}{2.168068in}}%
\pgfpathlineto{\pgfqpoint{3.174281in}{2.177645in}}%
\pgfpathlineto{\pgfqpoint{3.183211in}{2.162355in}}%
\pgfpathlineto{\pgfqpoint{3.192116in}{2.147702in}}%
\pgfpathlineto{\pgfqpoint{3.200997in}{2.133675in}}%
\pgfpathlineto{\pgfqpoint{3.209853in}{2.120259in}}%
\pgfpathclose%
\pgfusepath{fill}%
\end{pgfscope}%
\begin{pgfscope}%
\pgfpathrectangle{\pgfqpoint{1.150000in}{0.150000in}}{\pgfqpoint{5.700000in}{5.700000in}}%
\pgfusepath{clip}%
\pgfsetbuttcap%
\pgfsetroundjoin%
\definecolor{currentfill}{rgb}{0.160665,0.478540,0.558115}%
\pgfsetfillcolor{currentfill}%
\pgfsetfillopacity{0.700000}%
\pgfsetlinewidth{0.000000pt}%
\definecolor{currentstroke}{rgb}{0.000000,0.000000,0.000000}%
\pgfsetstrokecolor{currentstroke}%
\pgfsetdash{}{0pt}%
\pgfpathmoveto{\pgfqpoint{2.781625in}{2.458584in}}%
\pgfpathlineto{\pgfqpoint{2.795615in}{2.448100in}}%
\pgfpathlineto{\pgfqpoint{2.809607in}{2.437652in}}%
\pgfpathlineto{\pgfqpoint{2.823601in}{2.427239in}}%
\pgfpathlineto{\pgfqpoint{2.837597in}{2.416860in}}%
\pgfpathlineto{\pgfqpoint{2.828326in}{2.435425in}}%
\pgfpathlineto{\pgfqpoint{2.819021in}{2.454681in}}%
\pgfpathlineto{\pgfqpoint{2.809685in}{2.474641in}}%
\pgfpathlineto{\pgfqpoint{2.800314in}{2.495320in}}%
\pgfpathlineto{\pgfqpoint{2.786260in}{2.506146in}}%
\pgfpathlineto{\pgfqpoint{2.772209in}{2.517007in}}%
\pgfpathlineto{\pgfqpoint{2.758160in}{2.527903in}}%
\pgfpathlineto{\pgfqpoint{2.744112in}{2.538834in}}%
\pgfpathlineto{\pgfqpoint{2.753542in}{2.517700in}}%
\pgfpathlineto{\pgfqpoint{2.762936in}{2.497290in}}%
\pgfpathlineto{\pgfqpoint{2.772297in}{2.477589in}}%
\pgfpathlineto{\pgfqpoint{2.781625in}{2.458584in}}%
\pgfpathclose%
\pgfusepath{fill}%
\end{pgfscope}%
\begin{pgfscope}%
\pgfpathrectangle{\pgfqpoint{1.150000in}{0.150000in}}{\pgfqpoint{5.700000in}{5.700000in}}%
\pgfusepath{clip}%
\pgfsetbuttcap%
\pgfsetroundjoin%
\definecolor{currentfill}{rgb}{0.274952,0.037752,0.364543}%
\pgfsetfillcolor{currentfill}%
\pgfsetfillopacity{0.700000}%
\pgfsetlinewidth{0.000000pt}%
\definecolor{currentstroke}{rgb}{0.000000,0.000000,0.000000}%
\pgfsetstrokecolor{currentstroke}%
\pgfsetdash{}{0pt}%
\pgfpathmoveto{\pgfqpoint{4.593193in}{1.460912in}}%
\pgfpathlineto{\pgfqpoint{4.607453in}{1.456115in}}%
\pgfpathlineto{\pgfqpoint{4.621720in}{1.451340in}}%
\pgfpathlineto{\pgfqpoint{4.635993in}{1.446590in}}%
\pgfpathlineto{\pgfqpoint{4.650274in}{1.441863in}}%
\pgfpathlineto{\pgfqpoint{4.642356in}{1.436826in}}%
\pgfpathlineto{\pgfqpoint{4.634434in}{1.432071in}}%
\pgfpathlineto{\pgfqpoint{4.626508in}{1.427608in}}%
\pgfpathlineto{\pgfqpoint{4.618578in}{1.423443in}}%
\pgfpathlineto{\pgfqpoint{4.604281in}{1.428498in}}%
\pgfpathlineto{\pgfqpoint{4.589992in}{1.433576in}}%
\pgfpathlineto{\pgfqpoint{4.575709in}{1.438678in}}%
\pgfpathlineto{\pgfqpoint{4.561432in}{1.443803in}}%
\pgfpathlineto{\pgfqpoint{4.569379in}{1.447635in}}%
\pgfpathlineto{\pgfqpoint{4.577321in}{1.451769in}}%
\pgfpathlineto{\pgfqpoint{4.585259in}{1.456198in}}%
\pgfpathlineto{\pgfqpoint{4.593193in}{1.460912in}}%
\pgfpathclose%
\pgfusepath{fill}%
\end{pgfscope}%
\begin{pgfscope}%
\pgfpathrectangle{\pgfqpoint{1.150000in}{0.150000in}}{\pgfqpoint{5.700000in}{5.700000in}}%
\pgfusepath{clip}%
\pgfsetbuttcap%
\pgfsetroundjoin%
\definecolor{currentfill}{rgb}{0.273809,0.031497,0.358853}%
\pgfsetfillcolor{currentfill}%
\pgfsetfillopacity{0.700000}%
\pgfsetlinewidth{0.000000pt}%
\definecolor{currentstroke}{rgb}{0.000000,0.000000,0.000000}%
\pgfsetstrokecolor{currentstroke}%
\pgfsetdash{}{0pt}%
\pgfpathmoveto{\pgfqpoint{4.739038in}{1.447253in}}%
\pgfpathlineto{\pgfqpoint{4.753339in}{1.442958in}}%
\pgfpathlineto{\pgfqpoint{4.767646in}{1.438687in}}%
\pgfpathlineto{\pgfqpoint{4.781961in}{1.434439in}}%
\pgfpathlineto{\pgfqpoint{4.796282in}{1.430214in}}%
\pgfpathlineto{\pgfqpoint{4.788406in}{1.423513in}}%
\pgfpathlineto{\pgfqpoint{4.780527in}{1.417054in}}%
\pgfpathlineto{\pgfqpoint{4.772645in}{1.410845in}}%
\pgfpathlineto{\pgfqpoint{4.764759in}{1.404895in}}%
\pgfpathlineto{\pgfqpoint{4.750425in}{1.409434in}}%
\pgfpathlineto{\pgfqpoint{4.736097in}{1.413996in}}%
\pgfpathlineto{\pgfqpoint{4.721776in}{1.418582in}}%
\pgfpathlineto{\pgfqpoint{4.707462in}{1.423191in}}%
\pgfpathlineto{\pgfqpoint{4.715361in}{1.428821in}}%
\pgfpathlineto{\pgfqpoint{4.723257in}{1.434714in}}%
\pgfpathlineto{\pgfqpoint{4.731150in}{1.440861in}}%
\pgfpathlineto{\pgfqpoint{4.739038in}{1.447253in}}%
\pgfpathclose%
\pgfusepath{fill}%
\end{pgfscope}%
\begin{pgfscope}%
\pgfpathrectangle{\pgfqpoint{1.150000in}{0.150000in}}{\pgfqpoint{5.700000in}{5.700000in}}%
\pgfusepath{clip}%
\pgfsetbuttcap%
\pgfsetroundjoin%
\definecolor{currentfill}{rgb}{0.170948,0.694384,0.493803}%
\pgfsetfillcolor{currentfill}%
\pgfsetfillopacity{0.700000}%
\pgfsetlinewidth{0.000000pt}%
\definecolor{currentstroke}{rgb}{0.000000,0.000000,0.000000}%
\pgfsetstrokecolor{currentstroke}%
\pgfsetdash{}{0pt}%
\pgfpathmoveto{\pgfqpoint{2.127348in}{3.059467in}}%
\pgfpathlineto{\pgfqpoint{2.141352in}{3.046657in}}%
\pgfpathlineto{\pgfqpoint{2.155355in}{3.033899in}}%
\pgfpathlineto{\pgfqpoint{2.169358in}{3.021192in}}%
\pgfpathlineto{\pgfqpoint{2.183361in}{3.008536in}}%
\pgfpathlineto{\pgfqpoint{2.173244in}{3.035078in}}%
\pgfpathlineto{\pgfqpoint{2.163079in}{3.062424in}}%
\pgfpathlineto{\pgfqpoint{2.152865in}{3.090588in}}%
\pgfpathlineto{\pgfqpoint{2.142602in}{3.119586in}}%
\pgfpathlineto{\pgfqpoint{2.128526in}{3.132726in}}%
\pgfpathlineto{\pgfqpoint{2.114451in}{3.145916in}}%
\pgfpathlineto{\pgfqpoint{2.100375in}{3.159159in}}%
\pgfpathlineto{\pgfqpoint{2.086299in}{3.172453in}}%
\pgfpathlineto{\pgfqpoint{2.096637in}{3.142963in}}%
\pgfpathlineto{\pgfqpoint{2.106924in}{3.114312in}}%
\pgfpathlineto{\pgfqpoint{2.117160in}{3.086485in}}%
\pgfpathlineto{\pgfqpoint{2.127348in}{3.059467in}}%
\pgfpathclose%
\pgfusepath{fill}%
\end{pgfscope}%
\begin{pgfscope}%
\pgfpathrectangle{\pgfqpoint{1.150000in}{0.150000in}}{\pgfqpoint{5.700000in}{5.700000in}}%
\pgfusepath{clip}%
\pgfsetbuttcap%
\pgfsetroundjoin%
\definecolor{currentfill}{rgb}{0.278012,0.180367,0.486697}%
\pgfsetfillcolor{currentfill}%
\pgfsetfillopacity{0.700000}%
\pgfsetlinewidth{0.000000pt}%
\definecolor{currentstroke}{rgb}{0.000000,0.000000,0.000000}%
\pgfsetstrokecolor{currentstroke}%
\pgfsetdash{}{0pt}%
\pgfpathmoveto{\pgfqpoint{3.840328in}{1.721087in}}%
\pgfpathlineto{\pgfqpoint{3.854429in}{1.713821in}}%
\pgfpathlineto{\pgfqpoint{3.868536in}{1.706581in}}%
\pgfpathlineto{\pgfqpoint{3.882647in}{1.699367in}}%
\pgfpathlineto{\pgfqpoint{3.896763in}{1.692177in}}%
\pgfpathlineto{\pgfqpoint{3.888474in}{1.696925in}}%
\pgfpathlineto{\pgfqpoint{3.880172in}{1.702147in}}%
\pgfpathlineto{\pgfqpoint{3.871857in}{1.707855in}}%
\pgfpathlineto{\pgfqpoint{3.863530in}{1.714059in}}%
\pgfpathlineto{\pgfqpoint{3.849381in}{1.721635in}}%
\pgfpathlineto{\pgfqpoint{3.835237in}{1.729235in}}%
\pgfpathlineto{\pgfqpoint{3.821098in}{1.736862in}}%
\pgfpathlineto{\pgfqpoint{3.806963in}{1.744513in}}%
\pgfpathlineto{\pgfqpoint{3.815324in}{1.737917in}}%
\pgfpathlineto{\pgfqpoint{3.823672in}{1.731821in}}%
\pgfpathlineto{\pgfqpoint{3.832006in}{1.726214in}}%
\pgfpathlineto{\pgfqpoint{3.840328in}{1.721087in}}%
\pgfpathclose%
\pgfusepath{fill}%
\end{pgfscope}%
\begin{pgfscope}%
\pgfpathrectangle{\pgfqpoint{1.150000in}{0.150000in}}{\pgfqpoint{5.700000in}{5.700000in}}%
\pgfusepath{clip}%
\pgfsetbuttcap%
\pgfsetroundjoin%
\definecolor{currentfill}{rgb}{0.277941,0.056324,0.381191}%
\pgfsetfillcolor{currentfill}%
\pgfsetfillopacity{0.700000}%
\pgfsetlinewidth{0.000000pt}%
\definecolor{currentstroke}{rgb}{0.000000,0.000000,0.000000}%
\pgfsetstrokecolor{currentstroke}%
\pgfsetdash{}{0pt}%
\pgfpathmoveto{\pgfqpoint{4.447448in}{1.485659in}}%
\pgfpathlineto{\pgfqpoint{4.461674in}{1.480344in}}%
\pgfpathlineto{\pgfqpoint{4.475906in}{1.475052in}}%
\pgfpathlineto{\pgfqpoint{4.490144in}{1.469785in}}%
\pgfpathlineto{\pgfqpoint{4.504389in}{1.464541in}}%
\pgfpathlineto{\pgfqpoint{4.496420in}{1.461358in}}%
\pgfpathlineto{\pgfqpoint{4.488445in}{1.458499in}}%
\pgfpathlineto{\pgfqpoint{4.480465in}{1.455973in}}%
\pgfpathlineto{\pgfqpoint{4.472480in}{1.453790in}}%
\pgfpathlineto{\pgfqpoint{4.458216in}{1.459375in}}%
\pgfpathlineto{\pgfqpoint{4.443959in}{1.464984in}}%
\pgfpathlineto{\pgfqpoint{4.429707in}{1.470616in}}%
\pgfpathlineto{\pgfqpoint{4.415462in}{1.476273in}}%
\pgfpathlineto{\pgfqpoint{4.423467in}{1.478109in}}%
\pgfpathlineto{\pgfqpoint{4.431466in}{1.480292in}}%
\pgfpathlineto{\pgfqpoint{4.439460in}{1.482811in}}%
\pgfpathlineto{\pgfqpoint{4.447448in}{1.485659in}}%
\pgfpathclose%
\pgfusepath{fill}%
\end{pgfscope}%
\begin{pgfscope}%
\pgfpathrectangle{\pgfqpoint{1.150000in}{0.150000in}}{\pgfqpoint{5.700000in}{5.700000in}}%
\pgfusepath{clip}%
\pgfsetbuttcap%
\pgfsetroundjoin%
\definecolor{currentfill}{rgb}{0.276022,0.044167,0.370164}%
\pgfsetfillcolor{currentfill}%
\pgfsetfillopacity{0.700000}%
\pgfsetlinewidth{0.000000pt}%
\definecolor{currentstroke}{rgb}{0.000000,0.000000,0.000000}%
\pgfsetstrokecolor{currentstroke}%
\pgfsetdash{}{0pt}%
\pgfpathmoveto{\pgfqpoint{5.120212in}{1.476173in}}%
\pgfpathlineto{\pgfqpoint{5.134633in}{1.473212in}}%
\pgfpathlineto{\pgfqpoint{5.149062in}{1.470275in}}%
\pgfpathlineto{\pgfqpoint{5.163499in}{1.467361in}}%
\pgfpathlineto{\pgfqpoint{5.177944in}{1.464471in}}%
\pgfpathlineto{\pgfqpoint{5.170146in}{1.454103in}}%
\pgfpathlineto{\pgfqpoint{5.162347in}{1.443872in}}%
\pgfpathlineto{\pgfqpoint{5.154545in}{1.433784in}}%
\pgfpathlineto{\pgfqpoint{5.146741in}{1.423848in}}%
\pgfpathlineto{\pgfqpoint{5.132289in}{1.427013in}}%
\pgfpathlineto{\pgfqpoint{5.117845in}{1.430202in}}%
\pgfpathlineto{\pgfqpoint{5.103409in}{1.433414in}}%
\pgfpathlineto{\pgfqpoint{5.088981in}{1.436650in}}%
\pgfpathlineto{\pgfqpoint{5.096793in}{1.446307in}}%
\pgfpathlineto{\pgfqpoint{5.104602in}{1.456117in}}%
\pgfpathlineto{\pgfqpoint{5.112408in}{1.466075in}}%
\pgfpathlineto{\pgfqpoint{5.120212in}{1.476173in}}%
\pgfpathclose%
\pgfusepath{fill}%
\end{pgfscope}%
\begin{pgfscope}%
\pgfpathrectangle{\pgfqpoint{1.150000in}{0.150000in}}{\pgfqpoint{5.700000in}{5.700000in}}%
\pgfusepath{clip}%
\pgfsetbuttcap%
\pgfsetroundjoin%
\definecolor{currentfill}{rgb}{0.272594,0.025563,0.353093}%
\pgfsetfillcolor{currentfill}%
\pgfsetfillopacity{0.700000}%
\pgfsetlinewidth{0.000000pt}%
\definecolor{currentstroke}{rgb}{0.000000,0.000000,0.000000}%
\pgfsetstrokecolor{currentstroke}%
\pgfsetdash{}{0pt}%
\pgfpathmoveto{\pgfqpoint{4.885066in}{1.443830in}}%
\pgfpathlineto{\pgfqpoint{4.899412in}{1.440024in}}%
\pgfpathlineto{\pgfqpoint{4.913765in}{1.436241in}}%
\pgfpathlineto{\pgfqpoint{4.928126in}{1.432483in}}%
\pgfpathlineto{\pgfqpoint{4.942495in}{1.428747in}}%
\pgfpathlineto{\pgfqpoint{4.934653in}{1.420562in}}%
\pgfpathlineto{\pgfqpoint{4.926808in}{1.412581in}}%
\pgfpathlineto{\pgfqpoint{4.918961in}{1.404813in}}%
\pgfpathlineto{\pgfqpoint{4.911112in}{1.397264in}}%
\pgfpathlineto{\pgfqpoint{4.896733in}{1.401301in}}%
\pgfpathlineto{\pgfqpoint{4.882361in}{1.405361in}}%
\pgfpathlineto{\pgfqpoint{4.867997in}{1.409445in}}%
\pgfpathlineto{\pgfqpoint{4.853640in}{1.413552in}}%
\pgfpathlineto{\pgfqpoint{4.861500in}{1.420794in}}%
\pgfpathlineto{\pgfqpoint{4.869358in}{1.428260in}}%
\pgfpathlineto{\pgfqpoint{4.877213in}{1.435941in}}%
\pgfpathlineto{\pgfqpoint{4.885066in}{1.443830in}}%
\pgfpathclose%
\pgfusepath{fill}%
\end{pgfscope}%
\begin{pgfscope}%
\pgfpathrectangle{\pgfqpoint{1.150000in}{0.150000in}}{\pgfqpoint{5.700000in}{5.700000in}}%
\pgfusepath{clip}%
\pgfsetbuttcap%
\pgfsetroundjoin%
\definecolor{currentfill}{rgb}{0.258965,0.251537,0.524736}%
\pgfsetfillcolor{currentfill}%
\pgfsetfillopacity{0.700000}%
\pgfsetlinewidth{0.000000pt}%
\definecolor{currentstroke}{rgb}{0.000000,0.000000,0.000000}%
\pgfsetstrokecolor{currentstroke}%
\pgfsetdash{}{0pt}%
\pgfpathmoveto{\pgfqpoint{3.581411in}{1.870474in}}%
\pgfpathlineto{\pgfqpoint{3.595476in}{1.862404in}}%
\pgfpathlineto{\pgfqpoint{3.609544in}{1.854360in}}%
\pgfpathlineto{\pgfqpoint{3.623618in}{1.846343in}}%
\pgfpathlineto{\pgfqpoint{3.637695in}{1.838352in}}%
\pgfpathlineto{\pgfqpoint{3.629211in}{1.846651in}}%
\pgfpathlineto{\pgfqpoint{3.620711in}{1.855484in}}%
\pgfpathlineto{\pgfqpoint{3.612193in}{1.864864in}}%
\pgfpathlineto{\pgfqpoint{3.603659in}{1.874802in}}%
\pgfpathlineto{\pgfqpoint{3.589543in}{1.883195in}}%
\pgfpathlineto{\pgfqpoint{3.575431in}{1.891616in}}%
\pgfpathlineto{\pgfqpoint{3.561323in}{1.900063in}}%
\pgfpathlineto{\pgfqpoint{3.547220in}{1.908537in}}%
\pgfpathlineto{\pgfqpoint{3.555794in}{1.898190in}}%
\pgfpathlineto{\pgfqpoint{3.564351in}{1.888404in}}%
\pgfpathlineto{\pgfqpoint{3.572889in}{1.879170in}}%
\pgfpathlineto{\pgfqpoint{3.581411in}{1.870474in}}%
\pgfpathclose%
\pgfusepath{fill}%
\end{pgfscope}%
\begin{pgfscope}%
\pgfpathrectangle{\pgfqpoint{1.150000in}{0.150000in}}{\pgfqpoint{5.700000in}{5.700000in}}%
\pgfusepath{clip}%
\pgfsetbuttcap%
\pgfsetroundjoin%
\definecolor{currentfill}{rgb}{0.150148,0.676631,0.506589}%
\pgfsetfillcolor{currentfill}%
\pgfsetfillopacity{0.700000}%
\pgfsetlinewidth{0.000000pt}%
\definecolor{currentstroke}{rgb}{0.000000,0.000000,0.000000}%
\pgfsetstrokecolor{currentstroke}%
\pgfsetdash{}{0pt}%
\pgfpathmoveto{\pgfqpoint{2.183361in}{3.008536in}}%
\pgfpathlineto{\pgfqpoint{2.197364in}{2.995929in}}%
\pgfpathlineto{\pgfqpoint{2.211368in}{2.983373in}}%
\pgfpathlineto{\pgfqpoint{2.225371in}{2.970865in}}%
\pgfpathlineto{\pgfqpoint{2.239375in}{2.958406in}}%
\pgfpathlineto{\pgfqpoint{2.229328in}{2.984475in}}%
\pgfpathlineto{\pgfqpoint{2.219234in}{3.011341in}}%
\pgfpathlineto{\pgfqpoint{2.209092in}{3.039021in}}%
\pgfpathlineto{\pgfqpoint{2.198902in}{3.067529in}}%
\pgfpathlineto{\pgfqpoint{2.184827in}{3.080469in}}%
\pgfpathlineto{\pgfqpoint{2.170752in}{3.093458in}}%
\pgfpathlineto{\pgfqpoint{2.156677in}{3.106497in}}%
\pgfpathlineto{\pgfqpoint{2.142602in}{3.119586in}}%
\pgfpathlineto{\pgfqpoint{2.152865in}{3.090588in}}%
\pgfpathlineto{\pgfqpoint{2.163079in}{3.062424in}}%
\pgfpathlineto{\pgfqpoint{2.173244in}{3.035078in}}%
\pgfpathlineto{\pgfqpoint{2.183361in}{3.008536in}}%
\pgfpathclose%
\pgfusepath{fill}%
\end{pgfscope}%
\begin{pgfscope}%
\pgfpathrectangle{\pgfqpoint{1.150000in}{0.150000in}}{\pgfqpoint{5.700000in}{5.700000in}}%
\pgfusepath{clip}%
\pgfsetbuttcap%
\pgfsetroundjoin%
\definecolor{currentfill}{rgb}{0.165117,0.467423,0.558141}%
\pgfsetfillcolor{currentfill}%
\pgfsetfillopacity{0.700000}%
\pgfsetlinewidth{0.000000pt}%
\definecolor{currentstroke}{rgb}{0.000000,0.000000,0.000000}%
\pgfsetstrokecolor{currentstroke}%
\pgfsetdash{}{0pt}%
\pgfpathmoveto{\pgfqpoint{2.837597in}{2.416860in}}%
\pgfpathlineto{\pgfqpoint{2.851596in}{2.406516in}}%
\pgfpathlineto{\pgfqpoint{2.865597in}{2.396206in}}%
\pgfpathlineto{\pgfqpoint{2.879601in}{2.385931in}}%
\pgfpathlineto{\pgfqpoint{2.893607in}{2.375689in}}%
\pgfpathlineto{\pgfqpoint{2.884390in}{2.393814in}}%
\pgfpathlineto{\pgfqpoint{2.875142in}{2.412625in}}%
\pgfpathlineto{\pgfqpoint{2.865862in}{2.432137in}}%
\pgfpathlineto{\pgfqpoint{2.856549in}{2.452362in}}%
\pgfpathlineto{\pgfqpoint{2.842487in}{2.463050in}}%
\pgfpathlineto{\pgfqpoint{2.828427in}{2.473772in}}%
\pgfpathlineto{\pgfqpoint{2.814370in}{2.484529in}}%
\pgfpathlineto{\pgfqpoint{2.800314in}{2.495320in}}%
\pgfpathlineto{\pgfqpoint{2.809685in}{2.474641in}}%
\pgfpathlineto{\pgfqpoint{2.819021in}{2.454681in}}%
\pgfpathlineto{\pgfqpoint{2.828326in}{2.435425in}}%
\pgfpathlineto{\pgfqpoint{2.837597in}{2.416860in}}%
\pgfpathclose%
\pgfusepath{fill}%
\end{pgfscope}%
\begin{pgfscope}%
\pgfpathrectangle{\pgfqpoint{1.150000in}{0.150000in}}{\pgfqpoint{5.700000in}{5.700000in}}%
\pgfusepath{clip}%
\pgfsetbuttcap%
\pgfsetroundjoin%
\definecolor{currentfill}{rgb}{0.220057,0.343307,0.549413}%
\pgfsetfillcolor{currentfill}%
\pgfsetfillopacity{0.700000}%
\pgfsetlinewidth{0.000000pt}%
\definecolor{currentstroke}{rgb}{0.000000,0.000000,0.000000}%
\pgfsetstrokecolor{currentstroke}%
\pgfsetdash{}{0pt}%
\pgfpathmoveto{\pgfqpoint{3.265938in}{2.083827in}}%
\pgfpathlineto{\pgfqpoint{3.279968in}{2.074793in}}%
\pgfpathlineto{\pgfqpoint{3.294001in}{2.065788in}}%
\pgfpathlineto{\pgfqpoint{3.308038in}{2.056812in}}%
\pgfpathlineto{\pgfqpoint{3.322078in}{2.047864in}}%
\pgfpathlineto{\pgfqpoint{3.313313in}{2.060446in}}%
\pgfpathlineto{\pgfqpoint{3.304526in}{2.073632in}}%
\pgfpathlineto{\pgfqpoint{3.295715in}{2.087434in}}%
\pgfpathlineto{\pgfqpoint{3.286882in}{2.101864in}}%
\pgfpathlineto{\pgfqpoint{3.272795in}{2.111234in}}%
\pgfpathlineto{\pgfqpoint{3.258712in}{2.120633in}}%
\pgfpathlineto{\pgfqpoint{3.244632in}{2.130061in}}%
\pgfpathlineto{\pgfqpoint{3.230556in}{2.139518in}}%
\pgfpathlineto{\pgfqpoint{3.239437in}{2.124659in}}%
\pgfpathlineto{\pgfqpoint{3.248294in}{2.110432in}}%
\pgfpathlineto{\pgfqpoint{3.257128in}{2.096826in}}%
\pgfpathlineto{\pgfqpoint{3.265938in}{2.083827in}}%
\pgfpathclose%
\pgfusepath{fill}%
\end{pgfscope}%
\begin{pgfscope}%
\pgfpathrectangle{\pgfqpoint{1.150000in}{0.150000in}}{\pgfqpoint{5.700000in}{5.700000in}}%
\pgfusepath{clip}%
\pgfsetbuttcap%
\pgfsetroundjoin%
\definecolor{currentfill}{rgb}{0.283229,0.120777,0.440584}%
\pgfsetfillcolor{currentfill}%
\pgfsetfillopacity{0.700000}%
\pgfsetlinewidth{0.000000pt}%
\definecolor{currentstroke}{rgb}{0.000000,0.000000,0.000000}%
\pgfsetstrokecolor{currentstroke}%
\pgfsetdash{}{0pt}%
\pgfpathmoveto{\pgfqpoint{4.099243in}{1.597856in}}%
\pgfpathlineto{\pgfqpoint{4.113397in}{1.591362in}}%
\pgfpathlineto{\pgfqpoint{4.127556in}{1.584891in}}%
\pgfpathlineto{\pgfqpoint{4.141722in}{1.578446in}}%
\pgfpathlineto{\pgfqpoint{4.155892in}{1.572024in}}%
\pgfpathlineto{\pgfqpoint{4.147758in}{1.573518in}}%
\pgfpathlineto{\pgfqpoint{4.139616in}{1.575429in}}%
\pgfpathlineto{\pgfqpoint{4.131465in}{1.577769in}}%
\pgfpathlineto{\pgfqpoint{4.123304in}{1.580547in}}%
\pgfpathlineto{\pgfqpoint{4.109107in}{1.587339in}}%
\pgfpathlineto{\pgfqpoint{4.094914in}{1.594155in}}%
\pgfpathlineto{\pgfqpoint{4.080728in}{1.600996in}}%
\pgfpathlineto{\pgfqpoint{4.066546in}{1.607861in}}%
\pgfpathlineto{\pgfqpoint{4.074735in}{1.604707in}}%
\pgfpathlineto{\pgfqpoint{4.082914in}{1.601995in}}%
\pgfpathlineto{\pgfqpoint{4.091083in}{1.599715in}}%
\pgfpathlineto{\pgfqpoint{4.099243in}{1.597856in}}%
\pgfpathclose%
\pgfusepath{fill}%
\end{pgfscope}%
\begin{pgfscope}%
\pgfpathrectangle{\pgfqpoint{1.150000in}{0.150000in}}{\pgfqpoint{5.700000in}{5.700000in}}%
\pgfusepath{clip}%
\pgfsetbuttcap%
\pgfsetroundjoin%
\definecolor{currentfill}{rgb}{0.137339,0.662252,0.515571}%
\pgfsetfillcolor{currentfill}%
\pgfsetfillopacity{0.700000}%
\pgfsetlinewidth{0.000000pt}%
\definecolor{currentstroke}{rgb}{0.000000,0.000000,0.000000}%
\pgfsetstrokecolor{currentstroke}%
\pgfsetdash{}{0pt}%
\pgfpathmoveto{\pgfqpoint{2.239375in}{2.958406in}}%
\pgfpathlineto{\pgfqpoint{2.253380in}{2.945996in}}%
\pgfpathlineto{\pgfqpoint{2.267384in}{2.933633in}}%
\pgfpathlineto{\pgfqpoint{2.281389in}{2.921317in}}%
\pgfpathlineto{\pgfqpoint{2.295395in}{2.909048in}}%
\pgfpathlineto{\pgfqpoint{2.285416in}{2.934644in}}%
\pgfpathlineto{\pgfqpoint{2.275391in}{2.961033in}}%
\pgfpathlineto{\pgfqpoint{2.265321in}{2.988230in}}%
\pgfpathlineto{\pgfqpoint{2.255203in}{3.016249in}}%
\pgfpathlineto{\pgfqpoint{2.241128in}{3.028998in}}%
\pgfpathlineto{\pgfqpoint{2.227052in}{3.041794in}}%
\pgfpathlineto{\pgfqpoint{2.212977in}{3.054637in}}%
\pgfpathlineto{\pgfqpoint{2.198902in}{3.067529in}}%
\pgfpathlineto{\pgfqpoint{2.209092in}{3.039021in}}%
\pgfpathlineto{\pgfqpoint{2.219234in}{3.011341in}}%
\pgfpathlineto{\pgfqpoint{2.229328in}{2.984475in}}%
\pgfpathlineto{\pgfqpoint{2.239375in}{2.958406in}}%
\pgfpathclose%
\pgfusepath{fill}%
\end{pgfscope}%
\begin{pgfscope}%
\pgfpathrectangle{\pgfqpoint{1.150000in}{0.150000in}}{\pgfqpoint{5.700000in}{5.700000in}}%
\pgfusepath{clip}%
\pgfsetbuttcap%
\pgfsetroundjoin%
\definecolor{currentfill}{rgb}{0.281446,0.084320,0.407414}%
\pgfsetfillcolor{currentfill}%
\pgfsetfillopacity{0.700000}%
\pgfsetlinewidth{0.000000pt}%
\definecolor{currentstroke}{rgb}{0.000000,0.000000,0.000000}%
\pgfsetstrokecolor{currentstroke}%
\pgfsetdash{}{0pt}%
\pgfpathmoveto{\pgfqpoint{4.301714in}{1.522383in}}%
\pgfpathlineto{\pgfqpoint{4.315912in}{1.516536in}}%
\pgfpathlineto{\pgfqpoint{4.330115in}{1.510712in}}%
\pgfpathlineto{\pgfqpoint{4.344325in}{1.504912in}}%
\pgfpathlineto{\pgfqpoint{4.358540in}{1.499137in}}%
\pgfpathlineto{\pgfqpoint{4.350508in}{1.498005in}}%
\pgfpathlineto{\pgfqpoint{4.342469in}{1.497242in}}%
\pgfpathlineto{\pgfqpoint{4.334424in}{1.496857in}}%
\pgfpathlineto{\pgfqpoint{4.326372in}{1.496859in}}%
\pgfpathlineto{\pgfqpoint{4.312134in}{1.502990in}}%
\pgfpathlineto{\pgfqpoint{4.297902in}{1.509146in}}%
\pgfpathlineto{\pgfqpoint{4.283676in}{1.515325in}}%
\pgfpathlineto{\pgfqpoint{4.269455in}{1.521528in}}%
\pgfpathlineto{\pgfqpoint{4.277531in}{1.521165in}}%
\pgfpathlineto{\pgfqpoint{4.285599in}{1.521193in}}%
\pgfpathlineto{\pgfqpoint{4.293660in}{1.521602in}}%
\pgfpathlineto{\pgfqpoint{4.301714in}{1.522383in}}%
\pgfpathclose%
\pgfusepath{fill}%
\end{pgfscope}%
\begin{pgfscope}%
\pgfpathrectangle{\pgfqpoint{1.150000in}{0.150000in}}{\pgfqpoint{5.700000in}{5.700000in}}%
\pgfusepath{clip}%
\pgfsetbuttcap%
\pgfsetroundjoin%
\definecolor{currentfill}{rgb}{0.274952,0.037752,0.364543}%
\pgfsetfillcolor{currentfill}%
\pgfsetfillopacity{0.700000}%
\pgfsetlinewidth{0.000000pt}%
\definecolor{currentstroke}{rgb}{0.000000,0.000000,0.000000}%
\pgfsetstrokecolor{currentstroke}%
\pgfsetdash{}{0pt}%
\pgfpathmoveto{\pgfqpoint{5.031346in}{1.449829in}}%
\pgfpathlineto{\pgfqpoint{5.045743in}{1.446499in}}%
\pgfpathlineto{\pgfqpoint{5.060148in}{1.443193in}}%
\pgfpathlineto{\pgfqpoint{5.074561in}{1.439910in}}%
\pgfpathlineto{\pgfqpoint{5.088981in}{1.436650in}}%
\pgfpathlineto{\pgfqpoint{5.081167in}{1.427155in}}%
\pgfpathlineto{\pgfqpoint{5.073351in}{1.417828in}}%
\pgfpathlineto{\pgfqpoint{5.065532in}{1.408676in}}%
\pgfpathlineto{\pgfqpoint{5.057712in}{1.399708in}}%
\pgfpathlineto{\pgfqpoint{5.043283in}{1.403256in}}%
\pgfpathlineto{\pgfqpoint{5.028862in}{1.406827in}}%
\pgfpathlineto{\pgfqpoint{5.014449in}{1.410422in}}%
\pgfpathlineto{\pgfqpoint{5.000043in}{1.414040in}}%
\pgfpathlineto{\pgfqpoint{5.007872in}{1.422715in}}%
\pgfpathlineto{\pgfqpoint{5.015699in}{1.431577in}}%
\pgfpathlineto{\pgfqpoint{5.023524in}{1.440617in}}%
\pgfpathlineto{\pgfqpoint{5.031346in}{1.449829in}}%
\pgfpathclose%
\pgfusepath{fill}%
\end{pgfscope}%
\begin{pgfscope}%
\pgfpathrectangle{\pgfqpoint{1.150000in}{0.150000in}}{\pgfqpoint{5.700000in}{5.700000in}}%
\pgfusepath{clip}%
\pgfsetbuttcap%
\pgfsetroundjoin%
\definecolor{currentfill}{rgb}{0.169646,0.456262,0.558030}%
\pgfsetfillcolor{currentfill}%
\pgfsetfillopacity{0.700000}%
\pgfsetlinewidth{0.000000pt}%
\definecolor{currentstroke}{rgb}{0.000000,0.000000,0.000000}%
\pgfsetstrokecolor{currentstroke}%
\pgfsetdash{}{0pt}%
\pgfpathmoveto{\pgfqpoint{2.893607in}{2.375689in}}%
\pgfpathlineto{\pgfqpoint{2.907615in}{2.365480in}}%
\pgfpathlineto{\pgfqpoint{2.921626in}{2.355305in}}%
\pgfpathlineto{\pgfqpoint{2.935639in}{2.345164in}}%
\pgfpathlineto{\pgfqpoint{2.949655in}{2.335055in}}%
\pgfpathlineto{\pgfqpoint{2.940492in}{2.352741in}}%
\pgfpathlineto{\pgfqpoint{2.931299in}{2.371109in}}%
\pgfpathlineto{\pgfqpoint{2.922075in}{2.390172in}}%
\pgfpathlineto{\pgfqpoint{2.912819in}{2.409944in}}%
\pgfpathlineto{\pgfqpoint{2.898748in}{2.420499in}}%
\pgfpathlineto{\pgfqpoint{2.884680in}{2.431086in}}%
\pgfpathlineto{\pgfqpoint{2.870613in}{2.441707in}}%
\pgfpathlineto{\pgfqpoint{2.856549in}{2.452362in}}%
\pgfpathlineto{\pgfqpoint{2.865862in}{2.432137in}}%
\pgfpathlineto{\pgfqpoint{2.875142in}{2.412625in}}%
\pgfpathlineto{\pgfqpoint{2.884390in}{2.393814in}}%
\pgfpathlineto{\pgfqpoint{2.893607in}{2.375689in}}%
\pgfpathclose%
\pgfusepath{fill}%
\end{pgfscope}%
\begin{pgfscope}%
\pgfpathrectangle{\pgfqpoint{1.150000in}{0.150000in}}{\pgfqpoint{5.700000in}{5.700000in}}%
\pgfusepath{clip}%
\pgfsetbuttcap%
\pgfsetroundjoin%
\definecolor{currentfill}{rgb}{0.279574,0.170599,0.479997}%
\pgfsetfillcolor{currentfill}%
\pgfsetfillopacity{0.700000}%
\pgfsetlinewidth{0.000000pt}%
\definecolor{currentstroke}{rgb}{0.000000,0.000000,0.000000}%
\pgfsetstrokecolor{currentstroke}%
\pgfsetdash{}{0pt}%
\pgfpathmoveto{\pgfqpoint{3.896763in}{1.692177in}}%
\pgfpathlineto{\pgfqpoint{3.910884in}{1.685013in}}%
\pgfpathlineto{\pgfqpoint{3.925010in}{1.677875in}}%
\pgfpathlineto{\pgfqpoint{3.939141in}{1.670761in}}%
\pgfpathlineto{\pgfqpoint{3.953277in}{1.663673in}}%
\pgfpathlineto{\pgfqpoint{3.945019in}{1.668040in}}%
\pgfpathlineto{\pgfqpoint{3.936748in}{1.672878in}}%
\pgfpathlineto{\pgfqpoint{3.928466in}{1.678198in}}%
\pgfpathlineto{\pgfqpoint{3.920172in}{1.684010in}}%
\pgfpathlineto{\pgfqpoint{3.906004in}{1.691485in}}%
\pgfpathlineto{\pgfqpoint{3.891841in}{1.698984in}}%
\pgfpathlineto{\pgfqpoint{3.877683in}{1.706509in}}%
\pgfpathlineto{\pgfqpoint{3.863530in}{1.714059in}}%
\pgfpathlineto{\pgfqpoint{3.871857in}{1.707855in}}%
\pgfpathlineto{\pgfqpoint{3.880172in}{1.702147in}}%
\pgfpathlineto{\pgfqpoint{3.888474in}{1.696925in}}%
\pgfpathlineto{\pgfqpoint{3.896763in}{1.692177in}}%
\pgfpathclose%
\pgfusepath{fill}%
\end{pgfscope}%
\begin{pgfscope}%
\pgfpathrectangle{\pgfqpoint{1.150000in}{0.150000in}}{\pgfqpoint{5.700000in}{5.700000in}}%
\pgfusepath{clip}%
\pgfsetbuttcap%
\pgfsetroundjoin%
\definecolor{currentfill}{rgb}{0.128087,0.647749,0.523491}%
\pgfsetfillcolor{currentfill}%
\pgfsetfillopacity{0.700000}%
\pgfsetlinewidth{0.000000pt}%
\definecolor{currentstroke}{rgb}{0.000000,0.000000,0.000000}%
\pgfsetstrokecolor{currentstroke}%
\pgfsetdash{}{0pt}%
\pgfpathmoveto{\pgfqpoint{2.295395in}{2.909048in}}%
\pgfpathlineto{\pgfqpoint{2.309401in}{2.896826in}}%
\pgfpathlineto{\pgfqpoint{2.323407in}{2.884649in}}%
\pgfpathlineto{\pgfqpoint{2.337415in}{2.872518in}}%
\pgfpathlineto{\pgfqpoint{2.351422in}{2.860432in}}%
\pgfpathlineto{\pgfqpoint{2.341511in}{2.885557in}}%
\pgfpathlineto{\pgfqpoint{2.331556in}{2.911470in}}%
\pgfpathlineto{\pgfqpoint{2.321555in}{2.938186in}}%
\pgfpathlineto{\pgfqpoint{2.311509in}{2.965719in}}%
\pgfpathlineto{\pgfqpoint{2.297432in}{2.978283in}}%
\pgfpathlineto{\pgfqpoint{2.283355in}{2.990892in}}%
\pgfpathlineto{\pgfqpoint{2.269279in}{3.003548in}}%
\pgfpathlineto{\pgfqpoint{2.255203in}{3.016249in}}%
\pgfpathlineto{\pgfqpoint{2.265321in}{2.988230in}}%
\pgfpathlineto{\pgfqpoint{2.275391in}{2.961033in}}%
\pgfpathlineto{\pgfqpoint{2.285416in}{2.934644in}}%
\pgfpathlineto{\pgfqpoint{2.295395in}{2.909048in}}%
\pgfpathclose%
\pgfusepath{fill}%
\end{pgfscope}%
\begin{pgfscope}%
\pgfpathrectangle{\pgfqpoint{1.150000in}{0.150000in}}{\pgfqpoint{5.700000in}{5.700000in}}%
\pgfusepath{clip}%
\pgfsetbuttcap%
\pgfsetroundjoin%
\definecolor{currentfill}{rgb}{0.274952,0.037752,0.364543}%
\pgfsetfillcolor{currentfill}%
\pgfsetfillopacity{0.700000}%
\pgfsetlinewidth{0.000000pt}%
\definecolor{currentstroke}{rgb}{0.000000,0.000000,0.000000}%
\pgfsetstrokecolor{currentstroke}%
\pgfsetdash{}{0pt}%
\pgfpathmoveto{\pgfqpoint{4.650274in}{1.441863in}}%
\pgfpathlineto{\pgfqpoint{4.664561in}{1.437160in}}%
\pgfpathlineto{\pgfqpoint{4.678854in}{1.432480in}}%
\pgfpathlineto{\pgfqpoint{4.693155in}{1.427824in}}%
\pgfpathlineto{\pgfqpoint{4.707462in}{1.423191in}}%
\pgfpathlineto{\pgfqpoint{4.699559in}{1.417831in}}%
\pgfpathlineto{\pgfqpoint{4.691652in}{1.412751in}}%
\pgfpathlineto{\pgfqpoint{4.683742in}{1.407957in}}%
\pgfpathlineto{\pgfqpoint{4.675828in}{1.403460in}}%
\pgfpathlineto{\pgfqpoint{4.661505in}{1.408421in}}%
\pgfpathlineto{\pgfqpoint{4.647190in}{1.413405in}}%
\pgfpathlineto{\pgfqpoint{4.632880in}{1.418412in}}%
\pgfpathlineto{\pgfqpoint{4.618578in}{1.423443in}}%
\pgfpathlineto{\pgfqpoint{4.626508in}{1.427608in}}%
\pgfpathlineto{\pgfqpoint{4.634434in}{1.432071in}}%
\pgfpathlineto{\pgfqpoint{4.642356in}{1.436826in}}%
\pgfpathlineto{\pgfqpoint{4.650274in}{1.441863in}}%
\pgfpathclose%
\pgfusepath{fill}%
\end{pgfscope}%
\begin{pgfscope}%
\pgfpathrectangle{\pgfqpoint{1.150000in}{0.150000in}}{\pgfqpoint{5.700000in}{5.700000in}}%
\pgfusepath{clip}%
\pgfsetbuttcap%
\pgfsetroundjoin%
\definecolor{currentfill}{rgb}{0.262138,0.242286,0.520837}%
\pgfsetfillcolor{currentfill}%
\pgfsetfillopacity{0.700000}%
\pgfsetlinewidth{0.000000pt}%
\definecolor{currentstroke}{rgb}{0.000000,0.000000,0.000000}%
\pgfsetstrokecolor{currentstroke}%
\pgfsetdash{}{0pt}%
\pgfpathmoveto{\pgfqpoint{3.637695in}{1.838352in}}%
\pgfpathlineto{\pgfqpoint{3.651776in}{1.830389in}}%
\pgfpathlineto{\pgfqpoint{3.665862in}{1.822451in}}%
\pgfpathlineto{\pgfqpoint{3.679952in}{1.814540in}}%
\pgfpathlineto{\pgfqpoint{3.694047in}{1.806655in}}%
\pgfpathlineto{\pgfqpoint{3.685600in}{1.814556in}}%
\pgfpathlineto{\pgfqpoint{3.677137in}{1.822989in}}%
\pgfpathlineto{\pgfqpoint{3.668659in}{1.831963in}}%
\pgfpathlineto{\pgfqpoint{3.660164in}{1.841492in}}%
\pgfpathlineto{\pgfqpoint{3.646031in}{1.849780in}}%
\pgfpathlineto{\pgfqpoint{3.631903in}{1.858094in}}%
\pgfpathlineto{\pgfqpoint{3.617779in}{1.866434in}}%
\pgfpathlineto{\pgfqpoint{3.603659in}{1.874802in}}%
\pgfpathlineto{\pgfqpoint{3.612193in}{1.864864in}}%
\pgfpathlineto{\pgfqpoint{3.620711in}{1.855484in}}%
\pgfpathlineto{\pgfqpoint{3.629211in}{1.846651in}}%
\pgfpathlineto{\pgfqpoint{3.637695in}{1.838352in}}%
\pgfpathclose%
\pgfusepath{fill}%
\end{pgfscope}%
\begin{pgfscope}%
\pgfpathrectangle{\pgfqpoint{1.150000in}{0.150000in}}{\pgfqpoint{5.700000in}{5.700000in}}%
\pgfusepath{clip}%
\pgfsetbuttcap%
\pgfsetroundjoin%
\definecolor{currentfill}{rgb}{0.273809,0.031497,0.358853}%
\pgfsetfillcolor{currentfill}%
\pgfsetfillopacity{0.700000}%
\pgfsetlinewidth{0.000000pt}%
\definecolor{currentstroke}{rgb}{0.000000,0.000000,0.000000}%
\pgfsetstrokecolor{currentstroke}%
\pgfsetdash{}{0pt}%
\pgfpathmoveto{\pgfqpoint{4.796282in}{1.430214in}}%
\pgfpathlineto{\pgfqpoint{4.810611in}{1.426013in}}%
\pgfpathlineto{\pgfqpoint{4.824947in}{1.421836in}}%
\pgfpathlineto{\pgfqpoint{4.839290in}{1.417682in}}%
\pgfpathlineto{\pgfqpoint{4.853640in}{1.413552in}}%
\pgfpathlineto{\pgfqpoint{4.845776in}{1.406540in}}%
\pgfpathlineto{\pgfqpoint{4.837909in}{1.399768in}}%
\pgfpathlineto{\pgfqpoint{4.830040in}{1.393244in}}%
\pgfpathlineto{\pgfqpoint{4.822167in}{1.386974in}}%
\pgfpathlineto{\pgfqpoint{4.807805in}{1.391419in}}%
\pgfpathlineto{\pgfqpoint{4.793449in}{1.395888in}}%
\pgfpathlineto{\pgfqpoint{4.779101in}{1.400380in}}%
\pgfpathlineto{\pgfqpoint{4.764759in}{1.404895in}}%
\pgfpathlineto{\pgfqpoint{4.772645in}{1.410845in}}%
\pgfpathlineto{\pgfqpoint{4.780527in}{1.417054in}}%
\pgfpathlineto{\pgfqpoint{4.788406in}{1.423513in}}%
\pgfpathlineto{\pgfqpoint{4.796282in}{1.430214in}}%
\pgfpathclose%
\pgfusepath{fill}%
\end{pgfscope}%
\begin{pgfscope}%
\pgfpathrectangle{\pgfqpoint{1.150000in}{0.150000in}}{\pgfqpoint{5.700000in}{5.700000in}}%
\pgfusepath{clip}%
\pgfsetbuttcap%
\pgfsetroundjoin%
\definecolor{currentfill}{rgb}{0.225863,0.330805,0.547314}%
\pgfsetfillcolor{currentfill}%
\pgfsetfillopacity{0.700000}%
\pgfsetlinewidth{0.000000pt}%
\definecolor{currentstroke}{rgb}{0.000000,0.000000,0.000000}%
\pgfsetstrokecolor{currentstroke}%
\pgfsetdash{}{0pt}%
\pgfpathmoveto{\pgfqpoint{3.322078in}{2.047864in}}%
\pgfpathlineto{\pgfqpoint{3.336122in}{2.038946in}}%
\pgfpathlineto{\pgfqpoint{3.350169in}{2.030056in}}%
\pgfpathlineto{\pgfqpoint{3.364220in}{2.021194in}}%
\pgfpathlineto{\pgfqpoint{3.378275in}{2.012361in}}%
\pgfpathlineto{\pgfqpoint{3.369554in}{2.024528in}}%
\pgfpathlineto{\pgfqpoint{3.360811in}{2.037293in}}%
\pgfpathlineto{\pgfqpoint{3.352047in}{2.050670in}}%
\pgfpathlineto{\pgfqpoint{3.343260in}{2.064671in}}%
\pgfpathlineto{\pgfqpoint{3.329160in}{2.073926in}}%
\pgfpathlineto{\pgfqpoint{3.315064in}{2.083210in}}%
\pgfpathlineto{\pgfqpoint{3.300971in}{2.092522in}}%
\pgfpathlineto{\pgfqpoint{3.286882in}{2.101864in}}%
\pgfpathlineto{\pgfqpoint{3.295715in}{2.087434in}}%
\pgfpathlineto{\pgfqpoint{3.304526in}{2.073632in}}%
\pgfpathlineto{\pgfqpoint{3.313313in}{2.060446in}}%
\pgfpathlineto{\pgfqpoint{3.322078in}{2.047864in}}%
\pgfpathclose%
\pgfusepath{fill}%
\end{pgfscope}%
\begin{pgfscope}%
\pgfpathrectangle{\pgfqpoint{1.150000in}{0.150000in}}{\pgfqpoint{5.700000in}{5.700000in}}%
\pgfusepath{clip}%
\pgfsetbuttcap%
\pgfsetroundjoin%
\definecolor{currentfill}{rgb}{0.277941,0.056324,0.381191}%
\pgfsetfillcolor{currentfill}%
\pgfsetfillopacity{0.700000}%
\pgfsetlinewidth{0.000000pt}%
\definecolor{currentstroke}{rgb}{0.000000,0.000000,0.000000}%
\pgfsetstrokecolor{currentstroke}%
\pgfsetdash{}{0pt}%
\pgfpathmoveto{\pgfqpoint{4.504389in}{1.464541in}}%
\pgfpathlineto{\pgfqpoint{4.518640in}{1.459321in}}%
\pgfpathlineto{\pgfqpoint{4.532898in}{1.454125in}}%
\pgfpathlineto{\pgfqpoint{4.547162in}{1.448952in}}%
\pgfpathlineto{\pgfqpoint{4.561432in}{1.443803in}}%
\pgfpathlineto{\pgfqpoint{4.553480in}{1.440284in}}%
\pgfpathlineto{\pgfqpoint{4.545524in}{1.437085in}}%
\pgfpathlineto{\pgfqpoint{4.537563in}{1.434216in}}%
\pgfpathlineto{\pgfqpoint{4.529597in}{1.431686in}}%
\pgfpathlineto{\pgfqpoint{4.515309in}{1.437176in}}%
\pgfpathlineto{\pgfqpoint{4.501026in}{1.442691in}}%
\pgfpathlineto{\pgfqpoint{4.486750in}{1.448228in}}%
\pgfpathlineto{\pgfqpoint{4.472480in}{1.453790in}}%
\pgfpathlineto{\pgfqpoint{4.480465in}{1.455973in}}%
\pgfpathlineto{\pgfqpoint{4.488445in}{1.458499in}}%
\pgfpathlineto{\pgfqpoint{4.496420in}{1.461358in}}%
\pgfpathlineto{\pgfqpoint{4.504389in}{1.464541in}}%
\pgfpathclose%
\pgfusepath{fill}%
\end{pgfscope}%
\begin{pgfscope}%
\pgfpathrectangle{\pgfqpoint{1.150000in}{0.150000in}}{\pgfqpoint{5.700000in}{5.700000in}}%
\pgfusepath{clip}%
\pgfsetbuttcap%
\pgfsetroundjoin%
\definecolor{currentfill}{rgb}{0.277018,0.050344,0.375715}%
\pgfsetfillcolor{currentfill}%
\pgfsetfillopacity{0.700000}%
\pgfsetlinewidth{0.000000pt}%
\definecolor{currentstroke}{rgb}{0.000000,0.000000,0.000000}%
\pgfsetstrokecolor{currentstroke}%
\pgfsetdash{}{0pt}%
\pgfpathmoveto{\pgfqpoint{5.177944in}{1.464471in}}%
\pgfpathlineto{\pgfqpoint{5.192397in}{1.461605in}}%
\pgfpathlineto{\pgfqpoint{5.206858in}{1.458762in}}%
\pgfpathlineto{\pgfqpoint{5.221327in}{1.455942in}}%
\pgfpathlineto{\pgfqpoint{5.213534in}{1.445371in}}%
\pgfpathlineto{\pgfqpoint{5.205739in}{1.434935in}}%
\pgfpathlineto{\pgfqpoint{5.197942in}{1.424639in}}%
\pgfpathlineto{\pgfqpoint{5.190143in}{1.414492in}}%
\pgfpathlineto{\pgfqpoint{5.175668in}{1.417587in}}%
\pgfpathlineto{\pgfqpoint{5.161200in}{1.420705in}}%
\pgfpathlineto{\pgfqpoint{5.146741in}{1.423848in}}%
\pgfpathlineto{\pgfqpoint{5.154545in}{1.433784in}}%
\pgfpathlineto{\pgfqpoint{5.162347in}{1.443872in}}%
\pgfpathlineto{\pgfqpoint{5.170146in}{1.454103in}}%
\pgfpathlineto{\pgfqpoint{5.177944in}{1.464471in}}%
\pgfpathclose%
\pgfusepath{fill}%
\end{pgfscope}%
\begin{pgfscope}%
\pgfpathrectangle{\pgfqpoint{1.150000in}{0.150000in}}{\pgfqpoint{5.700000in}{5.700000in}}%
\pgfusepath{clip}%
\pgfsetbuttcap%
\pgfsetroundjoin%
\definecolor{currentfill}{rgb}{0.122312,0.633153,0.530398}%
\pgfsetfillcolor{currentfill}%
\pgfsetfillopacity{0.700000}%
\pgfsetlinewidth{0.000000pt}%
\definecolor{currentstroke}{rgb}{0.000000,0.000000,0.000000}%
\pgfsetstrokecolor{currentstroke}%
\pgfsetdash{}{0pt}%
\pgfpathmoveto{\pgfqpoint{2.351422in}{2.860432in}}%
\pgfpathlineto{\pgfqpoint{2.365431in}{2.848390in}}%
\pgfpathlineto{\pgfqpoint{2.379441in}{2.836393in}}%
\pgfpathlineto{\pgfqpoint{2.393451in}{2.824440in}}%
\pgfpathlineto{\pgfqpoint{2.407462in}{2.812530in}}%
\pgfpathlineto{\pgfqpoint{2.397618in}{2.837186in}}%
\pgfpathlineto{\pgfqpoint{2.387730in}{2.862625in}}%
\pgfpathlineto{\pgfqpoint{2.377798in}{2.888861in}}%
\pgfpathlineto{\pgfqpoint{2.367822in}{2.915909in}}%
\pgfpathlineto{\pgfqpoint{2.353743in}{2.928295in}}%
\pgfpathlineto{\pgfqpoint{2.339664in}{2.940725in}}%
\pgfpathlineto{\pgfqpoint{2.325586in}{2.953200in}}%
\pgfpathlineto{\pgfqpoint{2.311509in}{2.965719in}}%
\pgfpathlineto{\pgfqpoint{2.321555in}{2.938186in}}%
\pgfpathlineto{\pgfqpoint{2.331556in}{2.911470in}}%
\pgfpathlineto{\pgfqpoint{2.341511in}{2.885557in}}%
\pgfpathlineto{\pgfqpoint{2.351422in}{2.860432in}}%
\pgfpathclose%
\pgfusepath{fill}%
\end{pgfscope}%
\begin{pgfscope}%
\pgfpathrectangle{\pgfqpoint{1.150000in}{0.150000in}}{\pgfqpoint{5.700000in}{5.700000in}}%
\pgfusepath{clip}%
\pgfsetbuttcap%
\pgfsetroundjoin%
\definecolor{currentfill}{rgb}{0.273809,0.031497,0.358853}%
\pgfsetfillcolor{currentfill}%
\pgfsetfillopacity{0.700000}%
\pgfsetlinewidth{0.000000pt}%
\definecolor{currentstroke}{rgb}{0.000000,0.000000,0.000000}%
\pgfsetstrokecolor{currentstroke}%
\pgfsetdash{}{0pt}%
\pgfpathmoveto{\pgfqpoint{4.942495in}{1.428747in}}%
\pgfpathlineto{\pgfqpoint{4.956870in}{1.425035in}}%
\pgfpathlineto{\pgfqpoint{4.971254in}{1.421347in}}%
\pgfpathlineto{\pgfqpoint{4.985644in}{1.417682in}}%
\pgfpathlineto{\pgfqpoint{5.000043in}{1.414040in}}%
\pgfpathlineto{\pgfqpoint{4.992211in}{1.405558in}}%
\pgfpathlineto{\pgfqpoint{4.984376in}{1.397278in}}%
\pgfpathlineto{\pgfqpoint{4.976540in}{1.389206in}}%
\pgfpathlineto{\pgfqpoint{4.968701in}{1.381350in}}%
\pgfpathlineto{\pgfqpoint{4.954292in}{1.385294in}}%
\pgfpathlineto{\pgfqpoint{4.939891in}{1.389260in}}%
\pgfpathlineto{\pgfqpoint{4.925498in}{1.393250in}}%
\pgfpathlineto{\pgfqpoint{4.911112in}{1.397264in}}%
\pgfpathlineto{\pgfqpoint{4.918961in}{1.404813in}}%
\pgfpathlineto{\pgfqpoint{4.926808in}{1.412581in}}%
\pgfpathlineto{\pgfqpoint{4.934653in}{1.420562in}}%
\pgfpathlineto{\pgfqpoint{4.942495in}{1.428747in}}%
\pgfpathclose%
\pgfusepath{fill}%
\end{pgfscope}%
\begin{pgfscope}%
\pgfpathrectangle{\pgfqpoint{1.150000in}{0.150000in}}{\pgfqpoint{5.700000in}{5.700000in}}%
\pgfusepath{clip}%
\pgfsetbuttcap%
\pgfsetroundjoin%
\definecolor{currentfill}{rgb}{0.174274,0.445044,0.557792}%
\pgfsetfillcolor{currentfill}%
\pgfsetfillopacity{0.700000}%
\pgfsetlinewidth{0.000000pt}%
\definecolor{currentstroke}{rgb}{0.000000,0.000000,0.000000}%
\pgfsetstrokecolor{currentstroke}%
\pgfsetdash{}{0pt}%
\pgfpathmoveto{\pgfqpoint{2.949655in}{2.335055in}}%
\pgfpathlineto{\pgfqpoint{2.963674in}{2.324979in}}%
\pgfpathlineto{\pgfqpoint{2.977695in}{2.314935in}}%
\pgfpathlineto{\pgfqpoint{2.991718in}{2.304924in}}%
\pgfpathlineto{\pgfqpoint{3.005745in}{2.294945in}}%
\pgfpathlineto{\pgfqpoint{2.996635in}{2.312194in}}%
\pgfpathlineto{\pgfqpoint{2.987496in}{2.330119in}}%
\pgfpathlineto{\pgfqpoint{2.978327in}{2.348735in}}%
\pgfpathlineto{\pgfqpoint{2.969127in}{2.368055in}}%
\pgfpathlineto{\pgfqpoint{2.955047in}{2.378479in}}%
\pgfpathlineto{\pgfqpoint{2.940969in}{2.388935in}}%
\pgfpathlineto{\pgfqpoint{2.926893in}{2.399423in}}%
\pgfpathlineto{\pgfqpoint{2.912819in}{2.409944in}}%
\pgfpathlineto{\pgfqpoint{2.922075in}{2.390172in}}%
\pgfpathlineto{\pgfqpoint{2.931299in}{2.371109in}}%
\pgfpathlineto{\pgfqpoint{2.940492in}{2.352741in}}%
\pgfpathlineto{\pgfqpoint{2.949655in}{2.335055in}}%
\pgfpathclose%
\pgfusepath{fill}%
\end{pgfscope}%
\begin{pgfscope}%
\pgfpathrectangle{\pgfqpoint{1.150000in}{0.150000in}}{\pgfqpoint{5.700000in}{5.700000in}}%
\pgfusepath{clip}%
\pgfsetbuttcap%
\pgfsetroundjoin%
\definecolor{currentfill}{rgb}{0.283197,0.115680,0.436115}%
\pgfsetfillcolor{currentfill}%
\pgfsetfillopacity{0.700000}%
\pgfsetlinewidth{0.000000pt}%
\definecolor{currentstroke}{rgb}{0.000000,0.000000,0.000000}%
\pgfsetstrokecolor{currentstroke}%
\pgfsetdash{}{0pt}%
\pgfpathmoveto{\pgfqpoint{4.155892in}{1.572024in}}%
\pgfpathlineto{\pgfqpoint{4.170068in}{1.565627in}}%
\pgfpathlineto{\pgfqpoint{4.184249in}{1.559255in}}%
\pgfpathlineto{\pgfqpoint{4.198436in}{1.552906in}}%
\pgfpathlineto{\pgfqpoint{4.212629in}{1.546582in}}%
\pgfpathlineto{\pgfqpoint{4.204521in}{1.547711in}}%
\pgfpathlineto{\pgfqpoint{4.196405in}{1.549254in}}%
\pgfpathlineto{\pgfqpoint{4.188280in}{1.551222in}}%
\pgfpathlineto{\pgfqpoint{4.180146in}{1.553624in}}%
\pgfpathlineto{\pgfqpoint{4.165928in}{1.560319in}}%
\pgfpathlineto{\pgfqpoint{4.151715in}{1.567037in}}%
\pgfpathlineto{\pgfqpoint{4.137507in}{1.573780in}}%
\pgfpathlineto{\pgfqpoint{4.123304in}{1.580547in}}%
\pgfpathlineto{\pgfqpoint{4.131465in}{1.577769in}}%
\pgfpathlineto{\pgfqpoint{4.139616in}{1.575429in}}%
\pgfpathlineto{\pgfqpoint{4.147758in}{1.573518in}}%
\pgfpathlineto{\pgfqpoint{4.155892in}{1.572024in}}%
\pgfpathclose%
\pgfusepath{fill}%
\end{pgfscope}%
\begin{pgfscope}%
\pgfpathrectangle{\pgfqpoint{1.150000in}{0.150000in}}{\pgfqpoint{5.700000in}{5.700000in}}%
\pgfusepath{clip}%
\pgfsetbuttcap%
\pgfsetroundjoin%
\definecolor{currentfill}{rgb}{0.280894,0.078907,0.402329}%
\pgfsetfillcolor{currentfill}%
\pgfsetfillopacity{0.700000}%
\pgfsetlinewidth{0.000000pt}%
\definecolor{currentstroke}{rgb}{0.000000,0.000000,0.000000}%
\pgfsetstrokecolor{currentstroke}%
\pgfsetdash{}{0pt}%
\pgfpathmoveto{\pgfqpoint{4.358540in}{1.499137in}}%
\pgfpathlineto{\pgfqpoint{4.372762in}{1.493385in}}%
\pgfpathlineto{\pgfqpoint{4.386989in}{1.487657in}}%
\pgfpathlineto{\pgfqpoint{4.401222in}{1.481953in}}%
\pgfpathlineto{\pgfqpoint{4.415462in}{1.476273in}}%
\pgfpathlineto{\pgfqpoint{4.407451in}{1.474791in}}%
\pgfpathlineto{\pgfqpoint{4.399434in}{1.473674in}}%
\pgfpathlineto{\pgfqpoint{4.391411in}{1.472932in}}%
\pgfpathlineto{\pgfqpoint{4.383381in}{1.472573in}}%
\pgfpathlineto{\pgfqpoint{4.369120in}{1.478608in}}%
\pgfpathlineto{\pgfqpoint{4.354865in}{1.484668in}}%
\pgfpathlineto{\pgfqpoint{4.340615in}{1.490752in}}%
\pgfpathlineto{\pgfqpoint{4.326372in}{1.496859in}}%
\pgfpathlineto{\pgfqpoint{4.334424in}{1.496857in}}%
\pgfpathlineto{\pgfqpoint{4.342469in}{1.497242in}}%
\pgfpathlineto{\pgfqpoint{4.350508in}{1.498005in}}%
\pgfpathlineto{\pgfqpoint{4.358540in}{1.499137in}}%
\pgfpathclose%
\pgfusepath{fill}%
\end{pgfscope}%
\begin{pgfscope}%
\pgfpathrectangle{\pgfqpoint{1.150000in}{0.150000in}}{\pgfqpoint{5.700000in}{5.700000in}}%
\pgfusepath{clip}%
\pgfsetbuttcap%
\pgfsetroundjoin%
\definecolor{currentfill}{rgb}{0.119699,0.618490,0.536347}%
\pgfsetfillcolor{currentfill}%
\pgfsetfillopacity{0.700000}%
\pgfsetlinewidth{0.000000pt}%
\definecolor{currentstroke}{rgb}{0.000000,0.000000,0.000000}%
\pgfsetstrokecolor{currentstroke}%
\pgfsetdash{}{0pt}%
\pgfpathmoveto{\pgfqpoint{2.407462in}{2.812530in}}%
\pgfpathlineto{\pgfqpoint{2.421474in}{2.800663in}}%
\pgfpathlineto{\pgfqpoint{2.435487in}{2.788839in}}%
\pgfpathlineto{\pgfqpoint{2.449501in}{2.777057in}}%
\pgfpathlineto{\pgfqpoint{2.463517in}{2.765318in}}%
\pgfpathlineto{\pgfqpoint{2.453738in}{2.789507in}}%
\pgfpathlineto{\pgfqpoint{2.443917in}{2.814473in}}%
\pgfpathlineto{\pgfqpoint{2.434054in}{2.840230in}}%
\pgfpathlineto{\pgfqpoint{2.424147in}{2.866795in}}%
\pgfpathlineto{\pgfqpoint{2.410064in}{2.879010in}}%
\pgfpathlineto{\pgfqpoint{2.395983in}{2.891267in}}%
\pgfpathlineto{\pgfqpoint{2.381902in}{2.903566in}}%
\pgfpathlineto{\pgfqpoint{2.367822in}{2.915909in}}%
\pgfpathlineto{\pgfqpoint{2.377798in}{2.888861in}}%
\pgfpathlineto{\pgfqpoint{2.387730in}{2.862625in}}%
\pgfpathlineto{\pgfqpoint{2.397618in}{2.837186in}}%
\pgfpathlineto{\pgfqpoint{2.407462in}{2.812530in}}%
\pgfpathclose%
\pgfusepath{fill}%
\end{pgfscope}%
\begin{pgfscope}%
\pgfpathrectangle{\pgfqpoint{1.150000in}{0.150000in}}{\pgfqpoint{5.700000in}{5.700000in}}%
\pgfusepath{clip}%
\pgfsetbuttcap%
\pgfsetroundjoin%
\definecolor{currentfill}{rgb}{0.229739,0.322361,0.545706}%
\pgfsetfillcolor{currentfill}%
\pgfsetfillopacity{0.700000}%
\pgfsetlinewidth{0.000000pt}%
\definecolor{currentstroke}{rgb}{0.000000,0.000000,0.000000}%
\pgfsetstrokecolor{currentstroke}%
\pgfsetdash{}{0pt}%
\pgfpathmoveto{\pgfqpoint{3.378275in}{2.012361in}}%
\pgfpathlineto{\pgfqpoint{3.392333in}{2.003556in}}%
\pgfpathlineto{\pgfqpoint{3.406394in}{1.994780in}}%
\pgfpathlineto{\pgfqpoint{3.420460in}{1.986031in}}%
\pgfpathlineto{\pgfqpoint{3.434529in}{1.977310in}}%
\pgfpathlineto{\pgfqpoint{3.425852in}{1.989062in}}%
\pgfpathlineto{\pgfqpoint{3.417154in}{2.001407in}}%
\pgfpathlineto{\pgfqpoint{3.408434in}{2.014360in}}%
\pgfpathlineto{\pgfqpoint{3.399694in}{2.027933in}}%
\pgfpathlineto{\pgfqpoint{3.385580in}{2.037075in}}%
\pgfpathlineto{\pgfqpoint{3.371470in}{2.046246in}}%
\pgfpathlineto{\pgfqpoint{3.357363in}{2.055444in}}%
\pgfpathlineto{\pgfqpoint{3.343260in}{2.064671in}}%
\pgfpathlineto{\pgfqpoint{3.352047in}{2.050670in}}%
\pgfpathlineto{\pgfqpoint{3.360811in}{2.037293in}}%
\pgfpathlineto{\pgfqpoint{3.369554in}{2.024528in}}%
\pgfpathlineto{\pgfqpoint{3.378275in}{2.012361in}}%
\pgfpathclose%
\pgfusepath{fill}%
\end{pgfscope}%
\begin{pgfscope}%
\pgfpathrectangle{\pgfqpoint{1.150000in}{0.150000in}}{\pgfqpoint{5.700000in}{5.700000in}}%
\pgfusepath{clip}%
\pgfsetbuttcap%
\pgfsetroundjoin%
\definecolor{currentfill}{rgb}{0.280255,0.165693,0.476498}%
\pgfsetfillcolor{currentfill}%
\pgfsetfillopacity{0.700000}%
\pgfsetlinewidth{0.000000pt}%
\definecolor{currentstroke}{rgb}{0.000000,0.000000,0.000000}%
\pgfsetstrokecolor{currentstroke}%
\pgfsetdash{}{0pt}%
\pgfpathmoveto{\pgfqpoint{3.953277in}{1.663673in}}%
\pgfpathlineto{\pgfqpoint{3.967418in}{1.656609in}}%
\pgfpathlineto{\pgfqpoint{3.981564in}{1.649571in}}%
\pgfpathlineto{\pgfqpoint{3.995715in}{1.642557in}}%
\pgfpathlineto{\pgfqpoint{4.009871in}{1.635568in}}%
\pgfpathlineto{\pgfqpoint{4.001643in}{1.639555in}}%
\pgfpathlineto{\pgfqpoint{3.993403in}{1.644010in}}%
\pgfpathlineto{\pgfqpoint{3.985153in}{1.648942in}}%
\pgfpathlineto{\pgfqpoint{3.976890in}{1.654362in}}%
\pgfpathlineto{\pgfqpoint{3.962704in}{1.661737in}}%
\pgfpathlineto{\pgfqpoint{3.948521in}{1.669136in}}%
\pgfpathlineto{\pgfqpoint{3.934344in}{1.676561in}}%
\pgfpathlineto{\pgfqpoint{3.920172in}{1.684010in}}%
\pgfpathlineto{\pgfqpoint{3.928466in}{1.678198in}}%
\pgfpathlineto{\pgfqpoint{3.936748in}{1.672878in}}%
\pgfpathlineto{\pgfqpoint{3.945019in}{1.668040in}}%
\pgfpathlineto{\pgfqpoint{3.953277in}{1.663673in}}%
\pgfpathclose%
\pgfusepath{fill}%
\end{pgfscope}%
\begin{pgfscope}%
\pgfpathrectangle{\pgfqpoint{1.150000in}{0.150000in}}{\pgfqpoint{5.700000in}{5.700000in}}%
\pgfusepath{clip}%
\pgfsetbuttcap%
\pgfsetroundjoin%
\definecolor{currentfill}{rgb}{0.265145,0.232956,0.516599}%
\pgfsetfillcolor{currentfill}%
\pgfsetfillopacity{0.700000}%
\pgfsetlinewidth{0.000000pt}%
\definecolor{currentstroke}{rgb}{0.000000,0.000000,0.000000}%
\pgfsetstrokecolor{currentstroke}%
\pgfsetdash{}{0pt}%
\pgfpathmoveto{\pgfqpoint{3.694047in}{1.806655in}}%
\pgfpathlineto{\pgfqpoint{3.708146in}{1.798797in}}%
\pgfpathlineto{\pgfqpoint{3.722249in}{1.790964in}}%
\pgfpathlineto{\pgfqpoint{3.736357in}{1.783158in}}%
\pgfpathlineto{\pgfqpoint{3.750469in}{1.775377in}}%
\pgfpathlineto{\pgfqpoint{3.742058in}{1.782882in}}%
\pgfpathlineto{\pgfqpoint{3.733633in}{1.790913in}}%
\pgfpathlineto{\pgfqpoint{3.725192in}{1.799483in}}%
\pgfpathlineto{\pgfqpoint{3.716735in}{1.808603in}}%
\pgfpathlineto{\pgfqpoint{3.702586in}{1.816786in}}%
\pgfpathlineto{\pgfqpoint{3.688441in}{1.824995in}}%
\pgfpathlineto{\pgfqpoint{3.674300in}{1.833230in}}%
\pgfpathlineto{\pgfqpoint{3.660164in}{1.841492in}}%
\pgfpathlineto{\pgfqpoint{3.668659in}{1.831963in}}%
\pgfpathlineto{\pgfqpoint{3.677137in}{1.822989in}}%
\pgfpathlineto{\pgfqpoint{3.685600in}{1.814556in}}%
\pgfpathlineto{\pgfqpoint{3.694047in}{1.806655in}}%
\pgfpathclose%
\pgfusepath{fill}%
\end{pgfscope}%
\begin{pgfscope}%
\pgfpathrectangle{\pgfqpoint{1.150000in}{0.150000in}}{\pgfqpoint{5.700000in}{5.700000in}}%
\pgfusepath{clip}%
\pgfsetbuttcap%
\pgfsetroundjoin%
\definecolor{currentfill}{rgb}{0.274952,0.037752,0.364543}%
\pgfsetfillcolor{currentfill}%
\pgfsetfillopacity{0.700000}%
\pgfsetlinewidth{0.000000pt}%
\definecolor{currentstroke}{rgb}{0.000000,0.000000,0.000000}%
\pgfsetstrokecolor{currentstroke}%
\pgfsetdash{}{0pt}%
\pgfpathmoveto{\pgfqpoint{5.088981in}{1.436650in}}%
\pgfpathlineto{\pgfqpoint{5.103409in}{1.433414in}}%
\pgfpathlineto{\pgfqpoint{5.117845in}{1.430202in}}%
\pgfpathlineto{\pgfqpoint{5.132289in}{1.427013in}}%
\pgfpathlineto{\pgfqpoint{5.146741in}{1.423848in}}%
\pgfpathlineto{\pgfqpoint{5.138934in}{1.414068in}}%
\pgfpathlineto{\pgfqpoint{5.131126in}{1.404454in}}%
\pgfpathlineto{\pgfqpoint{5.123315in}{1.395012in}}%
\pgfpathlineto{\pgfqpoint{5.115503in}{1.385750in}}%
\pgfpathlineto{\pgfqpoint{5.101043in}{1.389204in}}%
\pgfpathlineto{\pgfqpoint{5.086592in}{1.392682in}}%
\pgfpathlineto{\pgfqpoint{5.072148in}{1.396183in}}%
\pgfpathlineto{\pgfqpoint{5.057712in}{1.399708in}}%
\pgfpathlineto{\pgfqpoint{5.065532in}{1.408676in}}%
\pgfpathlineto{\pgfqpoint{5.073351in}{1.417828in}}%
\pgfpathlineto{\pgfqpoint{5.081167in}{1.427155in}}%
\pgfpathlineto{\pgfqpoint{5.088981in}{1.436650in}}%
\pgfpathclose%
\pgfusepath{fill}%
\end{pgfscope}%
\begin{pgfscope}%
\pgfpathrectangle{\pgfqpoint{1.150000in}{0.150000in}}{\pgfqpoint{5.700000in}{5.700000in}}%
\pgfusepath{clip}%
\pgfsetbuttcap%
\pgfsetroundjoin%
\definecolor{currentfill}{rgb}{0.179019,0.433756,0.557430}%
\pgfsetfillcolor{currentfill}%
\pgfsetfillopacity{0.700000}%
\pgfsetlinewidth{0.000000pt}%
\definecolor{currentstroke}{rgb}{0.000000,0.000000,0.000000}%
\pgfsetstrokecolor{currentstroke}%
\pgfsetdash{}{0pt}%
\pgfpathmoveto{\pgfqpoint{3.005745in}{2.294945in}}%
\pgfpathlineto{\pgfqpoint{3.019774in}{2.284998in}}%
\pgfpathlineto{\pgfqpoint{3.033806in}{2.275083in}}%
\pgfpathlineto{\pgfqpoint{3.047840in}{2.265200in}}%
\pgfpathlineto{\pgfqpoint{3.061878in}{2.255348in}}%
\pgfpathlineto{\pgfqpoint{3.052820in}{2.272159in}}%
\pgfpathlineto{\pgfqpoint{3.043734in}{2.289643in}}%
\pgfpathlineto{\pgfqpoint{3.034619in}{2.307813in}}%
\pgfpathlineto{\pgfqpoint{3.025475in}{2.326682in}}%
\pgfpathlineto{\pgfqpoint{3.011384in}{2.336978in}}%
\pgfpathlineto{\pgfqpoint{2.997296in}{2.347305in}}%
\pgfpathlineto{\pgfqpoint{2.983210in}{2.357664in}}%
\pgfpathlineto{\pgfqpoint{2.969127in}{2.368055in}}%
\pgfpathlineto{\pgfqpoint{2.978327in}{2.348735in}}%
\pgfpathlineto{\pgfqpoint{2.987496in}{2.330119in}}%
\pgfpathlineto{\pgfqpoint{2.996635in}{2.312194in}}%
\pgfpathlineto{\pgfqpoint{3.005745in}{2.294945in}}%
\pgfpathclose%
\pgfusepath{fill}%
\end{pgfscope}%
\begin{pgfscope}%
\pgfpathrectangle{\pgfqpoint{1.150000in}{0.150000in}}{\pgfqpoint{5.700000in}{5.700000in}}%
\pgfusepath{clip}%
\pgfsetbuttcap%
\pgfsetroundjoin%
\definecolor{currentfill}{rgb}{0.274952,0.037752,0.364543}%
\pgfsetfillcolor{currentfill}%
\pgfsetfillopacity{0.700000}%
\pgfsetlinewidth{0.000000pt}%
\definecolor{currentstroke}{rgb}{0.000000,0.000000,0.000000}%
\pgfsetstrokecolor{currentstroke}%
\pgfsetdash{}{0pt}%
\pgfpathmoveto{\pgfqpoint{4.707462in}{1.423191in}}%
\pgfpathlineto{\pgfqpoint{4.721776in}{1.418582in}}%
\pgfpathlineto{\pgfqpoint{4.736097in}{1.413996in}}%
\pgfpathlineto{\pgfqpoint{4.750425in}{1.409434in}}%
\pgfpathlineto{\pgfqpoint{4.764759in}{1.404895in}}%
\pgfpathlineto{\pgfqpoint{4.756871in}{1.399213in}}%
\pgfpathlineto{\pgfqpoint{4.748979in}{1.393805in}}%
\pgfpathlineto{\pgfqpoint{4.741083in}{1.388682in}}%
\pgfpathlineto{\pgfqpoint{4.733184in}{1.383851in}}%
\pgfpathlineto{\pgfqpoint{4.718835in}{1.388719in}}%
\pgfpathlineto{\pgfqpoint{4.704493in}{1.393609in}}%
\pgfpathlineto{\pgfqpoint{4.690157in}{1.398523in}}%
\pgfpathlineto{\pgfqpoint{4.675828in}{1.403460in}}%
\pgfpathlineto{\pgfqpoint{4.683742in}{1.407957in}}%
\pgfpathlineto{\pgfqpoint{4.691652in}{1.412751in}}%
\pgfpathlineto{\pgfqpoint{4.699559in}{1.417831in}}%
\pgfpathlineto{\pgfqpoint{4.707462in}{1.423191in}}%
\pgfpathclose%
\pgfusepath{fill}%
\end{pgfscope}%
\begin{pgfscope}%
\pgfpathrectangle{\pgfqpoint{1.150000in}{0.150000in}}{\pgfqpoint{5.700000in}{5.700000in}}%
\pgfusepath{clip}%
\pgfsetbuttcap%
\pgfsetroundjoin%
\definecolor{currentfill}{rgb}{0.119738,0.603785,0.541400}%
\pgfsetfillcolor{currentfill}%
\pgfsetfillopacity{0.700000}%
\pgfsetlinewidth{0.000000pt}%
\definecolor{currentstroke}{rgb}{0.000000,0.000000,0.000000}%
\pgfsetstrokecolor{currentstroke}%
\pgfsetdash{}{0pt}%
\pgfpathmoveto{\pgfqpoint{2.463517in}{2.765318in}}%
\pgfpathlineto{\pgfqpoint{2.477533in}{2.753620in}}%
\pgfpathlineto{\pgfqpoint{2.491550in}{2.741963in}}%
\pgfpathlineto{\pgfqpoint{2.505569in}{2.730347in}}%
\pgfpathlineto{\pgfqpoint{2.519589in}{2.718772in}}%
\pgfpathlineto{\pgfqpoint{2.509875in}{2.742494in}}%
\pgfpathlineto{\pgfqpoint{2.500120in}{2.766989in}}%
\pgfpathlineto{\pgfqpoint{2.490324in}{2.792270in}}%
\pgfpathlineto{\pgfqpoint{2.480485in}{2.818353in}}%
\pgfpathlineto{\pgfqpoint{2.466399in}{2.830402in}}%
\pgfpathlineto{\pgfqpoint{2.452314in}{2.842492in}}%
\pgfpathlineto{\pgfqpoint{2.438230in}{2.854623in}}%
\pgfpathlineto{\pgfqpoint{2.424147in}{2.866795in}}%
\pgfpathlineto{\pgfqpoint{2.434054in}{2.840230in}}%
\pgfpathlineto{\pgfqpoint{2.443917in}{2.814473in}}%
\pgfpathlineto{\pgfqpoint{2.453738in}{2.789507in}}%
\pgfpathlineto{\pgfqpoint{2.463517in}{2.765318in}}%
\pgfpathclose%
\pgfusepath{fill}%
\end{pgfscope}%
\begin{pgfscope}%
\pgfpathrectangle{\pgfqpoint{1.150000in}{0.150000in}}{\pgfqpoint{5.700000in}{5.700000in}}%
\pgfusepath{clip}%
\pgfsetbuttcap%
\pgfsetroundjoin%
\definecolor{currentfill}{rgb}{0.273809,0.031497,0.358853}%
\pgfsetfillcolor{currentfill}%
\pgfsetfillopacity{0.700000}%
\pgfsetlinewidth{0.000000pt}%
\definecolor{currentstroke}{rgb}{0.000000,0.000000,0.000000}%
\pgfsetstrokecolor{currentstroke}%
\pgfsetdash{}{0pt}%
\pgfpathmoveto{\pgfqpoint{4.853640in}{1.413552in}}%
\pgfpathlineto{\pgfqpoint{4.867997in}{1.409445in}}%
\pgfpathlineto{\pgfqpoint{4.882361in}{1.405361in}}%
\pgfpathlineto{\pgfqpoint{4.896733in}{1.401301in}}%
\pgfpathlineto{\pgfqpoint{4.911112in}{1.397264in}}%
\pgfpathlineto{\pgfqpoint{4.903260in}{1.389943in}}%
\pgfpathlineto{\pgfqpoint{4.895405in}{1.382858in}}%
\pgfpathlineto{\pgfqpoint{4.887547in}{1.376016in}}%
\pgfpathlineto{\pgfqpoint{4.879688in}{1.369427in}}%
\pgfpathlineto{\pgfqpoint{4.865297in}{1.373779in}}%
\pgfpathlineto{\pgfqpoint{4.850913in}{1.378154in}}%
\pgfpathlineto{\pgfqpoint{4.836537in}{1.382552in}}%
\pgfpathlineto{\pgfqpoint{4.822167in}{1.386974in}}%
\pgfpathlineto{\pgfqpoint{4.830040in}{1.393244in}}%
\pgfpathlineto{\pgfqpoint{4.837909in}{1.399768in}}%
\pgfpathlineto{\pgfqpoint{4.845776in}{1.406540in}}%
\pgfpathlineto{\pgfqpoint{4.853640in}{1.413552in}}%
\pgfpathclose%
\pgfusepath{fill}%
\end{pgfscope}%
\begin{pgfscope}%
\pgfpathrectangle{\pgfqpoint{1.150000in}{0.150000in}}{\pgfqpoint{5.700000in}{5.700000in}}%
\pgfusepath{clip}%
\pgfsetbuttcap%
\pgfsetroundjoin%
\definecolor{currentfill}{rgb}{0.277018,0.050344,0.375715}%
\pgfsetfillcolor{currentfill}%
\pgfsetfillopacity{0.700000}%
\pgfsetlinewidth{0.000000pt}%
\definecolor{currentstroke}{rgb}{0.000000,0.000000,0.000000}%
\pgfsetstrokecolor{currentstroke}%
\pgfsetdash{}{0pt}%
\pgfpathmoveto{\pgfqpoint{4.561432in}{1.443803in}}%
\pgfpathlineto{\pgfqpoint{4.575709in}{1.438678in}}%
\pgfpathlineto{\pgfqpoint{4.589992in}{1.433576in}}%
\pgfpathlineto{\pgfqpoint{4.604281in}{1.428498in}}%
\pgfpathlineto{\pgfqpoint{4.618578in}{1.423443in}}%
\pgfpathlineto{\pgfqpoint{4.610643in}{1.419588in}}%
\pgfpathlineto{\pgfqpoint{4.602705in}{1.416049in}}%
\pgfpathlineto{\pgfqpoint{4.594762in}{1.412836in}}%
\pgfpathlineto{\pgfqpoint{4.586814in}{1.409959in}}%
\pgfpathlineto{\pgfqpoint{4.572500in}{1.415355in}}%
\pgfpathlineto{\pgfqpoint{4.558193in}{1.420775in}}%
\pgfpathlineto{\pgfqpoint{4.543892in}{1.426219in}}%
\pgfpathlineto{\pgfqpoint{4.529597in}{1.431686in}}%
\pgfpathlineto{\pgfqpoint{4.537563in}{1.434216in}}%
\pgfpathlineto{\pgfqpoint{4.545524in}{1.437085in}}%
\pgfpathlineto{\pgfqpoint{4.553480in}{1.440284in}}%
\pgfpathlineto{\pgfqpoint{4.561432in}{1.443803in}}%
\pgfpathclose%
\pgfusepath{fill}%
\end{pgfscope}%
\begin{pgfscope}%
\pgfpathrectangle{\pgfqpoint{1.150000in}{0.150000in}}{\pgfqpoint{5.700000in}{5.700000in}}%
\pgfusepath{clip}%
\pgfsetbuttcap%
\pgfsetroundjoin%
\definecolor{currentfill}{rgb}{0.283091,0.110553,0.431554}%
\pgfsetfillcolor{currentfill}%
\pgfsetfillopacity{0.700000}%
\pgfsetlinewidth{0.000000pt}%
\definecolor{currentstroke}{rgb}{0.000000,0.000000,0.000000}%
\pgfsetstrokecolor{currentstroke}%
\pgfsetdash{}{0pt}%
\pgfpathmoveto{\pgfqpoint{4.212629in}{1.546582in}}%
\pgfpathlineto{\pgfqpoint{4.226827in}{1.540283in}}%
\pgfpathlineto{\pgfqpoint{4.241031in}{1.534007in}}%
\pgfpathlineto{\pgfqpoint{4.255240in}{1.527756in}}%
\pgfpathlineto{\pgfqpoint{4.269455in}{1.521528in}}%
\pgfpathlineto{\pgfqpoint{4.261372in}{1.522292in}}%
\pgfpathlineto{\pgfqpoint{4.253281in}{1.523467in}}%
\pgfpathlineto{\pgfqpoint{4.245182in}{1.525062in}}%
\pgfpathlineto{\pgfqpoint{4.237075in}{1.527089in}}%
\pgfpathlineto{\pgfqpoint{4.222835in}{1.533687in}}%
\pgfpathlineto{\pgfqpoint{4.208600in}{1.540308in}}%
\pgfpathlineto{\pgfqpoint{4.194370in}{1.546954in}}%
\pgfpathlineto{\pgfqpoint{4.180146in}{1.553624in}}%
\pgfpathlineto{\pgfqpoint{4.188280in}{1.551222in}}%
\pgfpathlineto{\pgfqpoint{4.196405in}{1.549254in}}%
\pgfpathlineto{\pgfqpoint{4.204521in}{1.547711in}}%
\pgfpathlineto{\pgfqpoint{4.212629in}{1.546582in}}%
\pgfpathclose%
\pgfusepath{fill}%
\end{pgfscope}%
\begin{pgfscope}%
\pgfpathrectangle{\pgfqpoint{1.150000in}{0.150000in}}{\pgfqpoint{5.700000in}{5.700000in}}%
\pgfusepath{clip}%
\pgfsetbuttcap%
\pgfsetroundjoin%
\definecolor{currentfill}{rgb}{0.235526,0.309527,0.542944}%
\pgfsetfillcolor{currentfill}%
\pgfsetfillopacity{0.700000}%
\pgfsetlinewidth{0.000000pt}%
\definecolor{currentstroke}{rgb}{0.000000,0.000000,0.000000}%
\pgfsetstrokecolor{currentstroke}%
\pgfsetdash{}{0pt}%
\pgfpathmoveto{\pgfqpoint{3.434529in}{1.977310in}}%
\pgfpathlineto{\pgfqpoint{3.448602in}{1.968618in}}%
\pgfpathlineto{\pgfqpoint{3.462679in}{1.959952in}}%
\pgfpathlineto{\pgfqpoint{3.476759in}{1.951315in}}%
\pgfpathlineto{\pgfqpoint{3.490843in}{1.942705in}}%
\pgfpathlineto{\pgfqpoint{3.482209in}{1.954041in}}%
\pgfpathlineto{\pgfqpoint{3.473554in}{1.965968in}}%
\pgfpathlineto{\pgfqpoint{3.464879in}{1.978498in}}%
\pgfpathlineto{\pgfqpoint{3.456184in}{1.991642in}}%
\pgfpathlineto{\pgfqpoint{3.442056in}{2.000674in}}%
\pgfpathlineto{\pgfqpoint{3.427931in}{2.009732in}}%
\pgfpathlineto{\pgfqpoint{3.413811in}{2.018819in}}%
\pgfpathlineto{\pgfqpoint{3.399694in}{2.027933in}}%
\pgfpathlineto{\pgfqpoint{3.408434in}{2.014360in}}%
\pgfpathlineto{\pgfqpoint{3.417154in}{2.001407in}}%
\pgfpathlineto{\pgfqpoint{3.425852in}{1.989062in}}%
\pgfpathlineto{\pgfqpoint{3.434529in}{1.977310in}}%
\pgfpathclose%
\pgfusepath{fill}%
\end{pgfscope}%
\begin{pgfscope}%
\pgfpathrectangle{\pgfqpoint{1.150000in}{0.150000in}}{\pgfqpoint{5.700000in}{5.700000in}}%
\pgfusepath{clip}%
\pgfsetbuttcap%
\pgfsetroundjoin%
\definecolor{currentfill}{rgb}{0.121831,0.589055,0.545623}%
\pgfsetfillcolor{currentfill}%
\pgfsetfillopacity{0.700000}%
\pgfsetlinewidth{0.000000pt}%
\definecolor{currentstroke}{rgb}{0.000000,0.000000,0.000000}%
\pgfsetstrokecolor{currentstroke}%
\pgfsetdash{}{0pt}%
\pgfpathmoveto{\pgfqpoint{2.519589in}{2.718772in}}%
\pgfpathlineto{\pgfqpoint{2.533610in}{2.707237in}}%
\pgfpathlineto{\pgfqpoint{2.547633in}{2.695742in}}%
\pgfpathlineto{\pgfqpoint{2.561657in}{2.684286in}}%
\pgfpathlineto{\pgfqpoint{2.575682in}{2.672870in}}%
\pgfpathlineto{\pgfqpoint{2.566032in}{2.696128in}}%
\pgfpathlineto{\pgfqpoint{2.556342in}{2.720153in}}%
\pgfpathlineto{\pgfqpoint{2.546612in}{2.744959in}}%
\pgfpathlineto{\pgfqpoint{2.536840in}{2.770562in}}%
\pgfpathlineto{\pgfqpoint{2.522749in}{2.782450in}}%
\pgfpathlineto{\pgfqpoint{2.508660in}{2.794378in}}%
\pgfpathlineto{\pgfqpoint{2.494572in}{2.806345in}}%
\pgfpathlineto{\pgfqpoint{2.480485in}{2.818353in}}%
\pgfpathlineto{\pgfqpoint{2.490324in}{2.792270in}}%
\pgfpathlineto{\pgfqpoint{2.500120in}{2.766989in}}%
\pgfpathlineto{\pgfqpoint{2.509875in}{2.742494in}}%
\pgfpathlineto{\pgfqpoint{2.519589in}{2.718772in}}%
\pgfpathclose%
\pgfusepath{fill}%
\end{pgfscope}%
\begin{pgfscope}%
\pgfpathrectangle{\pgfqpoint{1.150000in}{0.150000in}}{\pgfqpoint{5.700000in}{5.700000in}}%
\pgfusepath{clip}%
\pgfsetbuttcap%
\pgfsetroundjoin%
\definecolor{currentfill}{rgb}{0.273809,0.031497,0.358853}%
\pgfsetfillcolor{currentfill}%
\pgfsetfillopacity{0.700000}%
\pgfsetlinewidth{0.000000pt}%
\definecolor{currentstroke}{rgb}{0.000000,0.000000,0.000000}%
\pgfsetstrokecolor{currentstroke}%
\pgfsetdash{}{0pt}%
\pgfpathmoveto{\pgfqpoint{5.000043in}{1.414040in}}%
\pgfpathlineto{\pgfqpoint{5.014449in}{1.410422in}}%
\pgfpathlineto{\pgfqpoint{5.028862in}{1.406827in}}%
\pgfpathlineto{\pgfqpoint{5.043283in}{1.403256in}}%
\pgfpathlineto{\pgfqpoint{5.057712in}{1.399708in}}%
\pgfpathlineto{\pgfqpoint{5.049889in}{1.390929in}}%
\pgfpathlineto{\pgfqpoint{5.042064in}{1.382348in}}%
\pgfpathlineto{\pgfqpoint{5.034237in}{1.373973in}}%
\pgfpathlineto{\pgfqpoint{5.026407in}{1.365810in}}%
\pgfpathlineto{\pgfqpoint{5.011970in}{1.369660in}}%
\pgfpathlineto{\pgfqpoint{4.997539in}{1.373534in}}%
\pgfpathlineto{\pgfqpoint{4.983116in}{1.377430in}}%
\pgfpathlineto{\pgfqpoint{4.968701in}{1.381350in}}%
\pgfpathlineto{\pgfqpoint{4.976540in}{1.389206in}}%
\pgfpathlineto{\pgfqpoint{4.984376in}{1.397278in}}%
\pgfpathlineto{\pgfqpoint{4.992211in}{1.405558in}}%
\pgfpathlineto{\pgfqpoint{5.000043in}{1.414040in}}%
\pgfpathclose%
\pgfusepath{fill}%
\end{pgfscope}%
\begin{pgfscope}%
\pgfpathrectangle{\pgfqpoint{1.150000in}{0.150000in}}{\pgfqpoint{5.700000in}{5.700000in}}%
\pgfusepath{clip}%
\pgfsetbuttcap%
\pgfsetroundjoin%
\definecolor{currentfill}{rgb}{0.267968,0.223549,0.512008}%
\pgfsetfillcolor{currentfill}%
\pgfsetfillopacity{0.700000}%
\pgfsetlinewidth{0.000000pt}%
\definecolor{currentstroke}{rgb}{0.000000,0.000000,0.000000}%
\pgfsetstrokecolor{currentstroke}%
\pgfsetdash{}{0pt}%
\pgfpathmoveto{\pgfqpoint{3.750469in}{1.775377in}}%
\pgfpathlineto{\pgfqpoint{3.764586in}{1.767623in}}%
\pgfpathlineto{\pgfqpoint{3.778707in}{1.759894in}}%
\pgfpathlineto{\pgfqpoint{3.792832in}{1.752191in}}%
\pgfpathlineto{\pgfqpoint{3.806963in}{1.744513in}}%
\pgfpathlineto{\pgfqpoint{3.798587in}{1.751622in}}%
\pgfpathlineto{\pgfqpoint{3.790198in}{1.759253in}}%
\pgfpathlineto{\pgfqpoint{3.781794in}{1.767418in}}%
\pgfpathlineto{\pgfqpoint{3.773375in}{1.776129in}}%
\pgfpathlineto{\pgfqpoint{3.759208in}{1.784209in}}%
\pgfpathlineto{\pgfqpoint{3.745046in}{1.792314in}}%
\pgfpathlineto{\pgfqpoint{3.730888in}{1.800446in}}%
\pgfpathlineto{\pgfqpoint{3.716735in}{1.808603in}}%
\pgfpathlineto{\pgfqpoint{3.725192in}{1.799483in}}%
\pgfpathlineto{\pgfqpoint{3.733633in}{1.790913in}}%
\pgfpathlineto{\pgfqpoint{3.742058in}{1.782882in}}%
\pgfpathlineto{\pgfqpoint{3.750469in}{1.775377in}}%
\pgfpathclose%
\pgfusepath{fill}%
\end{pgfscope}%
\begin{pgfscope}%
\pgfpathrectangle{\pgfqpoint{1.150000in}{0.150000in}}{\pgfqpoint{5.700000in}{5.700000in}}%
\pgfusepath{clip}%
\pgfsetbuttcap%
\pgfsetroundjoin%
\definecolor{currentfill}{rgb}{0.183898,0.422383,0.556944}%
\pgfsetfillcolor{currentfill}%
\pgfsetfillopacity{0.700000}%
\pgfsetlinewidth{0.000000pt}%
\definecolor{currentstroke}{rgb}{0.000000,0.000000,0.000000}%
\pgfsetstrokecolor{currentstroke}%
\pgfsetdash{}{0pt}%
\pgfpathmoveto{\pgfqpoint{3.061878in}{2.255348in}}%
\pgfpathlineto{\pgfqpoint{3.075918in}{2.245527in}}%
\pgfpathlineto{\pgfqpoint{3.089961in}{2.235737in}}%
\pgfpathlineto{\pgfqpoint{3.104007in}{2.225979in}}%
\pgfpathlineto{\pgfqpoint{3.118056in}{2.216251in}}%
\pgfpathlineto{\pgfqpoint{3.109049in}{2.232626in}}%
\pgfpathlineto{\pgfqpoint{3.100016in}{2.249670in}}%
\pgfpathlineto{\pgfqpoint{3.090954in}{2.267394in}}%
\pgfpathlineto{\pgfqpoint{3.081864in}{2.285813in}}%
\pgfpathlineto{\pgfqpoint{3.067763in}{2.295983in}}%
\pgfpathlineto{\pgfqpoint{3.053664in}{2.306185in}}%
\pgfpathlineto{\pgfqpoint{3.039568in}{2.316418in}}%
\pgfpathlineto{\pgfqpoint{3.025475in}{2.326682in}}%
\pgfpathlineto{\pgfqpoint{3.034619in}{2.307813in}}%
\pgfpathlineto{\pgfqpoint{3.043734in}{2.289643in}}%
\pgfpathlineto{\pgfqpoint{3.052820in}{2.272159in}}%
\pgfpathlineto{\pgfqpoint{3.061878in}{2.255348in}}%
\pgfpathclose%
\pgfusepath{fill}%
\end{pgfscope}%
\begin{pgfscope}%
\pgfpathrectangle{\pgfqpoint{1.150000in}{0.150000in}}{\pgfqpoint{5.700000in}{5.700000in}}%
\pgfusepath{clip}%
\pgfsetbuttcap%
\pgfsetroundjoin%
\definecolor{currentfill}{rgb}{0.280267,0.073417,0.397163}%
\pgfsetfillcolor{currentfill}%
\pgfsetfillopacity{0.700000}%
\pgfsetlinewidth{0.000000pt}%
\definecolor{currentstroke}{rgb}{0.000000,0.000000,0.000000}%
\pgfsetstrokecolor{currentstroke}%
\pgfsetdash{}{0pt}%
\pgfpathmoveto{\pgfqpoint{4.415462in}{1.476273in}}%
\pgfpathlineto{\pgfqpoint{4.429707in}{1.470616in}}%
\pgfpathlineto{\pgfqpoint{4.443959in}{1.464984in}}%
\pgfpathlineto{\pgfqpoint{4.458216in}{1.459375in}}%
\pgfpathlineto{\pgfqpoint{4.472480in}{1.453790in}}%
\pgfpathlineto{\pgfqpoint{4.464490in}{1.451958in}}%
\pgfpathlineto{\pgfqpoint{4.456494in}{1.450487in}}%
\pgfpathlineto{\pgfqpoint{4.448492in}{1.449387in}}%
\pgfpathlineto{\pgfqpoint{4.440484in}{1.448667in}}%
\pgfpathlineto{\pgfqpoint{4.426200in}{1.454608in}}%
\pgfpathlineto{\pgfqpoint{4.411921in}{1.460572in}}%
\pgfpathlineto{\pgfqpoint{4.397648in}{1.466561in}}%
\pgfpathlineto{\pgfqpoint{4.383381in}{1.472573in}}%
\pgfpathlineto{\pgfqpoint{4.391411in}{1.472932in}}%
\pgfpathlineto{\pgfqpoint{4.399434in}{1.473674in}}%
\pgfpathlineto{\pgfqpoint{4.407451in}{1.474791in}}%
\pgfpathlineto{\pgfqpoint{4.415462in}{1.476273in}}%
\pgfpathclose%
\pgfusepath{fill}%
\end{pgfscope}%
\begin{pgfscope}%
\pgfpathrectangle{\pgfqpoint{1.150000in}{0.150000in}}{\pgfqpoint{5.700000in}{5.700000in}}%
\pgfusepath{clip}%
\pgfsetbuttcap%
\pgfsetroundjoin%
\definecolor{currentfill}{rgb}{0.281412,0.155834,0.469201}%
\pgfsetfillcolor{currentfill}%
\pgfsetfillopacity{0.700000}%
\pgfsetlinewidth{0.000000pt}%
\definecolor{currentstroke}{rgb}{0.000000,0.000000,0.000000}%
\pgfsetstrokecolor{currentstroke}%
\pgfsetdash{}{0pt}%
\pgfpathmoveto{\pgfqpoint{4.009871in}{1.635568in}}%
\pgfpathlineto{\pgfqpoint{4.024032in}{1.628604in}}%
\pgfpathlineto{\pgfqpoint{4.038198in}{1.621665in}}%
\pgfpathlineto{\pgfqpoint{4.052370in}{1.614751in}}%
\pgfpathlineto{\pgfqpoint{4.066546in}{1.607861in}}%
\pgfpathlineto{\pgfqpoint{4.058347in}{1.611468in}}%
\pgfpathlineto{\pgfqpoint{4.050138in}{1.615539in}}%
\pgfpathlineto{\pgfqpoint{4.041918in}{1.620083in}}%
\pgfpathlineto{\pgfqpoint{4.033687in}{1.625112in}}%
\pgfpathlineto{\pgfqpoint{4.019481in}{1.632388in}}%
\pgfpathlineto{\pgfqpoint{4.005279in}{1.639688in}}%
\pgfpathlineto{\pgfqpoint{3.991082in}{1.647013in}}%
\pgfpathlineto{\pgfqpoint{3.976890in}{1.654362in}}%
\pgfpathlineto{\pgfqpoint{3.985153in}{1.648942in}}%
\pgfpathlineto{\pgfqpoint{3.993403in}{1.644010in}}%
\pgfpathlineto{\pgfqpoint{4.001643in}{1.639555in}}%
\pgfpathlineto{\pgfqpoint{4.009871in}{1.635568in}}%
\pgfpathclose%
\pgfusepath{fill}%
\end{pgfscope}%
\begin{pgfscope}%
\pgfpathrectangle{\pgfqpoint{1.150000in}{0.150000in}}{\pgfqpoint{5.700000in}{5.700000in}}%
\pgfusepath{clip}%
\pgfsetbuttcap%
\pgfsetroundjoin%
\definecolor{currentfill}{rgb}{0.125394,0.574318,0.549086}%
\pgfsetfillcolor{currentfill}%
\pgfsetfillopacity{0.700000}%
\pgfsetlinewidth{0.000000pt}%
\definecolor{currentstroke}{rgb}{0.000000,0.000000,0.000000}%
\pgfsetstrokecolor{currentstroke}%
\pgfsetdash{}{0pt}%
\pgfpathmoveto{\pgfqpoint{2.575682in}{2.672870in}}%
\pgfpathlineto{\pgfqpoint{2.589709in}{2.661493in}}%
\pgfpathlineto{\pgfqpoint{2.603737in}{2.650154in}}%
\pgfpathlineto{\pgfqpoint{2.617767in}{2.638854in}}%
\pgfpathlineto{\pgfqpoint{2.631799in}{2.627592in}}%
\pgfpathlineto{\pgfqpoint{2.622211in}{2.650387in}}%
\pgfpathlineto{\pgfqpoint{2.612585in}{2.673943in}}%
\pgfpathlineto{\pgfqpoint{2.602920in}{2.698276in}}%
\pgfpathlineto{\pgfqpoint{2.593214in}{2.723399in}}%
\pgfpathlineto{\pgfqpoint{2.579119in}{2.735132in}}%
\pgfpathlineto{\pgfqpoint{2.565025in}{2.746903in}}%
\pgfpathlineto{\pgfqpoint{2.550932in}{2.758713in}}%
\pgfpathlineto{\pgfqpoint{2.536840in}{2.770562in}}%
\pgfpathlineto{\pgfqpoint{2.546612in}{2.744959in}}%
\pgfpathlineto{\pgfqpoint{2.556342in}{2.720153in}}%
\pgfpathlineto{\pgfqpoint{2.566032in}{2.696128in}}%
\pgfpathlineto{\pgfqpoint{2.575682in}{2.672870in}}%
\pgfpathclose%
\pgfusepath{fill}%
\end{pgfscope}%
\begin{pgfscope}%
\pgfpathrectangle{\pgfqpoint{1.150000in}{0.150000in}}{\pgfqpoint{5.700000in}{5.700000in}}%
\pgfusepath{clip}%
\pgfsetbuttcap%
\pgfsetroundjoin%
\definecolor{currentfill}{rgb}{0.276022,0.044167,0.370164}%
\pgfsetfillcolor{currentfill}%
\pgfsetfillopacity{0.700000}%
\pgfsetlinewidth{0.000000pt}%
\definecolor{currentstroke}{rgb}{0.000000,0.000000,0.000000}%
\pgfsetstrokecolor{currentstroke}%
\pgfsetdash{}{0pt}%
\pgfpathmoveto{\pgfqpoint{5.146741in}{1.423848in}}%
\pgfpathlineto{\pgfqpoint{5.161200in}{1.420705in}}%
\pgfpathlineto{\pgfqpoint{5.175668in}{1.417587in}}%
\pgfpathlineto{\pgfqpoint{5.190143in}{1.414492in}}%
\pgfpathlineto{\pgfqpoint{5.182342in}{1.404500in}}%
\pgfpathlineto{\pgfqpoint{5.174539in}{1.394670in}}%
\pgfpathlineto{\pgfqpoint{5.166734in}{1.385010in}}%
\pgfpathlineto{\pgfqpoint{5.158928in}{1.375526in}}%
\pgfpathlineto{\pgfqpoint{5.144445in}{1.378911in}}%
\pgfpathlineto{\pgfqpoint{5.129970in}{1.382318in}}%
\pgfpathlineto{\pgfqpoint{5.115503in}{1.385750in}}%
\pgfpathlineto{\pgfqpoint{5.123315in}{1.395012in}}%
\pgfpathlineto{\pgfqpoint{5.131126in}{1.404454in}}%
\pgfpathlineto{\pgfqpoint{5.138934in}{1.414068in}}%
\pgfpathlineto{\pgfqpoint{5.146741in}{1.423848in}}%
\pgfpathclose%
\pgfusepath{fill}%
\end{pgfscope}%
\begin{pgfscope}%
\pgfpathrectangle{\pgfqpoint{1.150000in}{0.150000in}}{\pgfqpoint{5.700000in}{5.700000in}}%
\pgfusepath{clip}%
\pgfsetbuttcap%
\pgfsetroundjoin%
\definecolor{currentfill}{rgb}{0.239346,0.300855,0.540844}%
\pgfsetfillcolor{currentfill}%
\pgfsetfillopacity{0.700000}%
\pgfsetlinewidth{0.000000pt}%
\definecolor{currentstroke}{rgb}{0.000000,0.000000,0.000000}%
\pgfsetstrokecolor{currentstroke}%
\pgfsetdash{}{0pt}%
\pgfpathmoveto{\pgfqpoint{3.490843in}{1.942705in}}%
\pgfpathlineto{\pgfqpoint{3.504932in}{1.934122in}}%
\pgfpathlineto{\pgfqpoint{3.519024in}{1.925567in}}%
\pgfpathlineto{\pgfqpoint{3.533120in}{1.917038in}}%
\pgfpathlineto{\pgfqpoint{3.547220in}{1.908537in}}%
\pgfpathlineto{\pgfqpoint{3.538626in}{1.919459in}}%
\pgfpathlineto{\pgfqpoint{3.530014in}{1.930967in}}%
\pgfpathlineto{\pgfqpoint{3.521383in}{1.943074in}}%
\pgfpathlineto{\pgfqpoint{3.512732in}{1.955792in}}%
\pgfpathlineto{\pgfqpoint{3.498589in}{1.964714in}}%
\pgfpathlineto{\pgfqpoint{3.484450in}{1.973663in}}%
\pgfpathlineto{\pgfqpoint{3.470315in}{1.982639in}}%
\pgfpathlineto{\pgfqpoint{3.456184in}{1.991642in}}%
\pgfpathlineto{\pgfqpoint{3.464879in}{1.978498in}}%
\pgfpathlineto{\pgfqpoint{3.473554in}{1.965968in}}%
\pgfpathlineto{\pgfqpoint{3.482209in}{1.954041in}}%
\pgfpathlineto{\pgfqpoint{3.490843in}{1.942705in}}%
\pgfpathclose%
\pgfusepath{fill}%
\end{pgfscope}%
\begin{pgfscope}%
\pgfpathrectangle{\pgfqpoint{1.150000in}{0.150000in}}{\pgfqpoint{5.700000in}{5.700000in}}%
\pgfusepath{clip}%
\pgfsetbuttcap%
\pgfsetroundjoin%
\definecolor{currentfill}{rgb}{0.274952,0.037752,0.364543}%
\pgfsetfillcolor{currentfill}%
\pgfsetfillopacity{0.700000}%
\pgfsetlinewidth{0.000000pt}%
\definecolor{currentstroke}{rgb}{0.000000,0.000000,0.000000}%
\pgfsetstrokecolor{currentstroke}%
\pgfsetdash{}{0pt}%
\pgfpathmoveto{\pgfqpoint{4.764759in}{1.404895in}}%
\pgfpathlineto{\pgfqpoint{4.779101in}{1.400380in}}%
\pgfpathlineto{\pgfqpoint{4.793449in}{1.395888in}}%
\pgfpathlineto{\pgfqpoint{4.807805in}{1.391419in}}%
\pgfpathlineto{\pgfqpoint{4.822167in}{1.386974in}}%
\pgfpathlineto{\pgfqpoint{4.814292in}{1.380968in}}%
\pgfpathlineto{\pgfqpoint{4.806414in}{1.375234in}}%
\pgfpathlineto{\pgfqpoint{4.798533in}{1.369781in}}%
\pgfpathlineto{\pgfqpoint{4.790649in}{1.364617in}}%
\pgfpathlineto{\pgfqpoint{4.776272in}{1.369391in}}%
\pgfpathlineto{\pgfqpoint{4.761903in}{1.374188in}}%
\pgfpathlineto{\pgfqpoint{4.747540in}{1.379008in}}%
\pgfpathlineto{\pgfqpoint{4.733184in}{1.383851in}}%
\pgfpathlineto{\pgfqpoint{4.741083in}{1.388682in}}%
\pgfpathlineto{\pgfqpoint{4.748979in}{1.393805in}}%
\pgfpathlineto{\pgfqpoint{4.756871in}{1.399213in}}%
\pgfpathlineto{\pgfqpoint{4.764759in}{1.404895in}}%
\pgfpathclose%
\pgfusepath{fill}%
\end{pgfscope}%
\begin{pgfscope}%
\pgfpathrectangle{\pgfqpoint{1.150000in}{0.150000in}}{\pgfqpoint{5.700000in}{5.700000in}}%
\pgfusepath{clip}%
\pgfsetbuttcap%
\pgfsetroundjoin%
\definecolor{currentfill}{rgb}{0.188923,0.410910,0.556326}%
\pgfsetfillcolor{currentfill}%
\pgfsetfillopacity{0.700000}%
\pgfsetlinewidth{0.000000pt}%
\definecolor{currentstroke}{rgb}{0.000000,0.000000,0.000000}%
\pgfsetstrokecolor{currentstroke}%
\pgfsetdash{}{0pt}%
\pgfpathmoveto{\pgfqpoint{3.118056in}{2.216251in}}%
\pgfpathlineto{\pgfqpoint{3.132108in}{2.206554in}}%
\pgfpathlineto{\pgfqpoint{3.146162in}{2.196887in}}%
\pgfpathlineto{\pgfqpoint{3.160220in}{2.187251in}}%
\pgfpathlineto{\pgfqpoint{3.174281in}{2.177645in}}%
\pgfpathlineto{\pgfqpoint{3.165325in}{2.193585in}}%
\pgfpathlineto{\pgfqpoint{3.156343in}{2.210188in}}%
\pgfpathlineto{\pgfqpoint{3.147334in}{2.227467in}}%
\pgfpathlineto{\pgfqpoint{3.138298in}{2.245437in}}%
\pgfpathlineto{\pgfqpoint{3.124185in}{2.255485in}}%
\pgfpathlineto{\pgfqpoint{3.110075in}{2.265564in}}%
\pgfpathlineto{\pgfqpoint{3.095969in}{2.275673in}}%
\pgfpathlineto{\pgfqpoint{3.081864in}{2.285813in}}%
\pgfpathlineto{\pgfqpoint{3.090954in}{2.267394in}}%
\pgfpathlineto{\pgfqpoint{3.100016in}{2.249670in}}%
\pgfpathlineto{\pgfqpoint{3.109049in}{2.232626in}}%
\pgfpathlineto{\pgfqpoint{3.118056in}{2.216251in}}%
\pgfpathclose%
\pgfusepath{fill}%
\end{pgfscope}%
\begin{pgfscope}%
\pgfpathrectangle{\pgfqpoint{1.150000in}{0.150000in}}{\pgfqpoint{5.700000in}{5.700000in}}%
\pgfusepath{clip}%
\pgfsetbuttcap%
\pgfsetroundjoin%
\definecolor{currentfill}{rgb}{0.277018,0.050344,0.375715}%
\pgfsetfillcolor{currentfill}%
\pgfsetfillopacity{0.700000}%
\pgfsetlinewidth{0.000000pt}%
\definecolor{currentstroke}{rgb}{0.000000,0.000000,0.000000}%
\pgfsetstrokecolor{currentstroke}%
\pgfsetdash{}{0pt}%
\pgfpathmoveto{\pgfqpoint{4.618578in}{1.423443in}}%
\pgfpathlineto{\pgfqpoint{4.632880in}{1.418412in}}%
\pgfpathlineto{\pgfqpoint{4.647190in}{1.413405in}}%
\pgfpathlineto{\pgfqpoint{4.661505in}{1.408421in}}%
\pgfpathlineto{\pgfqpoint{4.675828in}{1.403460in}}%
\pgfpathlineto{\pgfqpoint{4.667910in}{1.399267in}}%
\pgfpathlineto{\pgfqpoint{4.659988in}{1.395389in}}%
\pgfpathlineto{\pgfqpoint{4.652063in}{1.391832in}}%
\pgfpathlineto{\pgfqpoint{4.644133in}{1.388608in}}%
\pgfpathlineto{\pgfqpoint{4.629794in}{1.393911in}}%
\pgfpathlineto{\pgfqpoint{4.615461in}{1.399237in}}%
\pgfpathlineto{\pgfqpoint{4.601134in}{1.404586in}}%
\pgfpathlineto{\pgfqpoint{4.586814in}{1.409959in}}%
\pgfpathlineto{\pgfqpoint{4.594762in}{1.412836in}}%
\pgfpathlineto{\pgfqpoint{4.602705in}{1.416049in}}%
\pgfpathlineto{\pgfqpoint{4.610643in}{1.419588in}}%
\pgfpathlineto{\pgfqpoint{4.618578in}{1.423443in}}%
\pgfpathclose%
\pgfusepath{fill}%
\end{pgfscope}%
\begin{pgfscope}%
\pgfpathrectangle{\pgfqpoint{1.150000in}{0.150000in}}{\pgfqpoint{5.700000in}{5.700000in}}%
\pgfusepath{clip}%
\pgfsetbuttcap%
\pgfsetroundjoin%
\definecolor{currentfill}{rgb}{0.273809,0.031497,0.358853}%
\pgfsetfillcolor{currentfill}%
\pgfsetfillopacity{0.700000}%
\pgfsetlinewidth{0.000000pt}%
\definecolor{currentstroke}{rgb}{0.000000,0.000000,0.000000}%
\pgfsetstrokecolor{currentstroke}%
\pgfsetdash{}{0pt}%
\pgfpathmoveto{\pgfqpoint{4.911112in}{1.397264in}}%
\pgfpathlineto{\pgfqpoint{4.925498in}{1.393250in}}%
\pgfpathlineto{\pgfqpoint{4.939891in}{1.389260in}}%
\pgfpathlineto{\pgfqpoint{4.954292in}{1.385294in}}%
\pgfpathlineto{\pgfqpoint{4.968701in}{1.381350in}}%
\pgfpathlineto{\pgfqpoint{4.960859in}{1.373719in}}%
\pgfpathlineto{\pgfqpoint{4.953016in}{1.366321in}}%
\pgfpathlineto{\pgfqpoint{4.945170in}{1.359162in}}%
\pgfpathlineto{\pgfqpoint{4.937322in}{1.352253in}}%
\pgfpathlineto{\pgfqpoint{4.922902in}{1.356511in}}%
\pgfpathlineto{\pgfqpoint{4.908490in}{1.360793in}}%
\pgfpathlineto{\pgfqpoint{4.894085in}{1.365098in}}%
\pgfpathlineto{\pgfqpoint{4.879688in}{1.369427in}}%
\pgfpathlineto{\pgfqpoint{4.887547in}{1.376016in}}%
\pgfpathlineto{\pgfqpoint{4.895405in}{1.382858in}}%
\pgfpathlineto{\pgfqpoint{4.903260in}{1.389943in}}%
\pgfpathlineto{\pgfqpoint{4.911112in}{1.397264in}}%
\pgfpathclose%
\pgfusepath{fill}%
\end{pgfscope}%
\begin{pgfscope}%
\pgfpathrectangle{\pgfqpoint{1.150000in}{0.150000in}}{\pgfqpoint{5.700000in}{5.700000in}}%
\pgfusepath{clip}%
\pgfsetbuttcap%
\pgfsetroundjoin%
\definecolor{currentfill}{rgb}{0.282910,0.105393,0.426902}%
\pgfsetfillcolor{currentfill}%
\pgfsetfillopacity{0.700000}%
\pgfsetlinewidth{0.000000pt}%
\definecolor{currentstroke}{rgb}{0.000000,0.000000,0.000000}%
\pgfsetstrokecolor{currentstroke}%
\pgfsetdash{}{0pt}%
\pgfpathmoveto{\pgfqpoint{4.269455in}{1.521528in}}%
\pgfpathlineto{\pgfqpoint{4.283676in}{1.515325in}}%
\pgfpathlineto{\pgfqpoint{4.297902in}{1.509146in}}%
\pgfpathlineto{\pgfqpoint{4.312134in}{1.502990in}}%
\pgfpathlineto{\pgfqpoint{4.326372in}{1.496859in}}%
\pgfpathlineto{\pgfqpoint{4.318313in}{1.497258in}}%
\pgfpathlineto{\pgfqpoint{4.310246in}{1.498064in}}%
\pgfpathlineto{\pgfqpoint{4.302173in}{1.499288in}}%
\pgfpathlineto{\pgfqpoint{4.294091in}{1.500939in}}%
\pgfpathlineto{\pgfqpoint{4.279829in}{1.507440in}}%
\pgfpathlineto{\pgfqpoint{4.265572in}{1.513966in}}%
\pgfpathlineto{\pgfqpoint{4.251321in}{1.520515in}}%
\pgfpathlineto{\pgfqpoint{4.237075in}{1.527089in}}%
\pgfpathlineto{\pgfqpoint{4.245182in}{1.525062in}}%
\pgfpathlineto{\pgfqpoint{4.253281in}{1.523467in}}%
\pgfpathlineto{\pgfqpoint{4.261372in}{1.522292in}}%
\pgfpathlineto{\pgfqpoint{4.269455in}{1.521528in}}%
\pgfpathclose%
\pgfusepath{fill}%
\end{pgfscope}%
\begin{pgfscope}%
\pgfpathrectangle{\pgfqpoint{1.150000in}{0.150000in}}{\pgfqpoint{5.700000in}{5.700000in}}%
\pgfusepath{clip}%
\pgfsetbuttcap%
\pgfsetroundjoin%
\definecolor{currentfill}{rgb}{0.270595,0.214069,0.507052}%
\pgfsetfillcolor{currentfill}%
\pgfsetfillopacity{0.700000}%
\pgfsetlinewidth{0.000000pt}%
\definecolor{currentstroke}{rgb}{0.000000,0.000000,0.000000}%
\pgfsetstrokecolor{currentstroke}%
\pgfsetdash{}{0pt}%
\pgfpathmoveto{\pgfqpoint{3.806963in}{1.744513in}}%
\pgfpathlineto{\pgfqpoint{3.821098in}{1.736862in}}%
\pgfpathlineto{\pgfqpoint{3.835237in}{1.729235in}}%
\pgfpathlineto{\pgfqpoint{3.849381in}{1.721635in}}%
\pgfpathlineto{\pgfqpoint{3.863530in}{1.714059in}}%
\pgfpathlineto{\pgfqpoint{3.855189in}{1.720771in}}%
\pgfpathlineto{\pgfqpoint{3.846835in}{1.728002in}}%
\pgfpathlineto{\pgfqpoint{3.838467in}{1.735763in}}%
\pgfpathlineto{\pgfqpoint{3.830085in}{1.744066in}}%
\pgfpathlineto{\pgfqpoint{3.815900in}{1.752044in}}%
\pgfpathlineto{\pgfqpoint{3.801721in}{1.760047in}}%
\pgfpathlineto{\pgfqpoint{3.787546in}{1.768075in}}%
\pgfpathlineto{\pgfqpoint{3.773375in}{1.776129in}}%
\pgfpathlineto{\pgfqpoint{3.781794in}{1.767418in}}%
\pgfpathlineto{\pgfqpoint{3.790198in}{1.759253in}}%
\pgfpathlineto{\pgfqpoint{3.798587in}{1.751622in}}%
\pgfpathlineto{\pgfqpoint{3.806963in}{1.744513in}}%
\pgfpathclose%
\pgfusepath{fill}%
\end{pgfscope}%
\begin{pgfscope}%
\pgfpathrectangle{\pgfqpoint{1.150000in}{0.150000in}}{\pgfqpoint{5.700000in}{5.700000in}}%
\pgfusepath{clip}%
\pgfsetbuttcap%
\pgfsetroundjoin%
\definecolor{currentfill}{rgb}{0.129933,0.559582,0.551864}%
\pgfsetfillcolor{currentfill}%
\pgfsetfillopacity{0.700000}%
\pgfsetlinewidth{0.000000pt}%
\definecolor{currentstroke}{rgb}{0.000000,0.000000,0.000000}%
\pgfsetstrokecolor{currentstroke}%
\pgfsetdash{}{0pt}%
\pgfpathmoveto{\pgfqpoint{2.631799in}{2.627592in}}%
\pgfpathlineto{\pgfqpoint{2.645832in}{2.616368in}}%
\pgfpathlineto{\pgfqpoint{2.659867in}{2.605181in}}%
\pgfpathlineto{\pgfqpoint{2.673903in}{2.594032in}}%
\pgfpathlineto{\pgfqpoint{2.687941in}{2.582920in}}%
\pgfpathlineto{\pgfqpoint{2.678415in}{2.605252in}}%
\pgfpathlineto{\pgfqpoint{2.668852in}{2.628341in}}%
\pgfpathlineto{\pgfqpoint{2.659251in}{2.652201in}}%
\pgfpathlineto{\pgfqpoint{2.649611in}{2.676847in}}%
\pgfpathlineto{\pgfqpoint{2.635510in}{2.688429in}}%
\pgfpathlineto{\pgfqpoint{2.621410in}{2.700048in}}%
\pgfpathlineto{\pgfqpoint{2.607311in}{2.711705in}}%
\pgfpathlineto{\pgfqpoint{2.593214in}{2.723399in}}%
\pgfpathlineto{\pgfqpoint{2.602920in}{2.698276in}}%
\pgfpathlineto{\pgfqpoint{2.612585in}{2.673943in}}%
\pgfpathlineto{\pgfqpoint{2.622211in}{2.650387in}}%
\pgfpathlineto{\pgfqpoint{2.631799in}{2.627592in}}%
\pgfpathclose%
\pgfusepath{fill}%
\end{pgfscope}%
\begin{pgfscope}%
\pgfpathrectangle{\pgfqpoint{1.150000in}{0.150000in}}{\pgfqpoint{5.700000in}{5.700000in}}%
\pgfusepath{clip}%
\pgfsetbuttcap%
\pgfsetroundjoin%
\definecolor{currentfill}{rgb}{0.281887,0.150881,0.465405}%
\pgfsetfillcolor{currentfill}%
\pgfsetfillopacity{0.700000}%
\pgfsetlinewidth{0.000000pt}%
\definecolor{currentstroke}{rgb}{0.000000,0.000000,0.000000}%
\pgfsetstrokecolor{currentstroke}%
\pgfsetdash{}{0pt}%
\pgfpathmoveto{\pgfqpoint{4.066546in}{1.607861in}}%
\pgfpathlineto{\pgfqpoint{4.080728in}{1.600996in}}%
\pgfpathlineto{\pgfqpoint{4.094914in}{1.594155in}}%
\pgfpathlineto{\pgfqpoint{4.109107in}{1.587339in}}%
\pgfpathlineto{\pgfqpoint{4.123304in}{1.580547in}}%
\pgfpathlineto{\pgfqpoint{4.115134in}{1.583775in}}%
\pgfpathlineto{\pgfqpoint{4.106954in}{1.587461in}}%
\pgfpathlineto{\pgfqpoint{4.098764in}{1.591618in}}%
\pgfpathlineto{\pgfqpoint{4.090564in}{1.596256in}}%
\pgfpathlineto{\pgfqpoint{4.076337in}{1.603433in}}%
\pgfpathlineto{\pgfqpoint{4.062116in}{1.610635in}}%
\pgfpathlineto{\pgfqpoint{4.047899in}{1.617861in}}%
\pgfpathlineto{\pgfqpoint{4.033687in}{1.625112in}}%
\pgfpathlineto{\pgfqpoint{4.041918in}{1.620083in}}%
\pgfpathlineto{\pgfqpoint{4.050138in}{1.615539in}}%
\pgfpathlineto{\pgfqpoint{4.058347in}{1.611468in}}%
\pgfpathlineto{\pgfqpoint{4.066546in}{1.607861in}}%
\pgfpathclose%
\pgfusepath{fill}%
\end{pgfscope}%
\begin{pgfscope}%
\pgfpathrectangle{\pgfqpoint{1.150000in}{0.150000in}}{\pgfqpoint{5.700000in}{5.700000in}}%
\pgfusepath{clip}%
\pgfsetbuttcap%
\pgfsetroundjoin%
\definecolor{currentfill}{rgb}{0.279566,0.067836,0.391917}%
\pgfsetfillcolor{currentfill}%
\pgfsetfillopacity{0.700000}%
\pgfsetlinewidth{0.000000pt}%
\definecolor{currentstroke}{rgb}{0.000000,0.000000,0.000000}%
\pgfsetstrokecolor{currentstroke}%
\pgfsetdash{}{0pt}%
\pgfpathmoveto{\pgfqpoint{4.472480in}{1.453790in}}%
\pgfpathlineto{\pgfqpoint{4.486750in}{1.448228in}}%
\pgfpathlineto{\pgfqpoint{4.501026in}{1.442691in}}%
\pgfpathlineto{\pgfqpoint{4.515309in}{1.437176in}}%
\pgfpathlineto{\pgfqpoint{4.529597in}{1.431686in}}%
\pgfpathlineto{\pgfqpoint{4.521626in}{1.429504in}}%
\pgfpathlineto{\pgfqpoint{4.513650in}{1.427679in}}%
\pgfpathlineto{\pgfqpoint{4.505669in}{1.426221in}}%
\pgfpathlineto{\pgfqpoint{4.497683in}{1.425139in}}%
\pgfpathlineto{\pgfqpoint{4.483374in}{1.430986in}}%
\pgfpathlineto{\pgfqpoint{4.469072in}{1.436856in}}%
\pgfpathlineto{\pgfqpoint{4.454775in}{1.442749in}}%
\pgfpathlineto{\pgfqpoint{4.440484in}{1.448667in}}%
\pgfpathlineto{\pgfqpoint{4.448492in}{1.449387in}}%
\pgfpathlineto{\pgfqpoint{4.456494in}{1.450487in}}%
\pgfpathlineto{\pgfqpoint{4.464490in}{1.451958in}}%
\pgfpathlineto{\pgfqpoint{4.472480in}{1.453790in}}%
\pgfpathclose%
\pgfusepath{fill}%
\end{pgfscope}%
\begin{pgfscope}%
\pgfpathrectangle{\pgfqpoint{1.150000in}{0.150000in}}{\pgfqpoint{5.700000in}{5.700000in}}%
\pgfusepath{clip}%
\pgfsetbuttcap%
\pgfsetroundjoin%
\definecolor{currentfill}{rgb}{0.274952,0.037752,0.364543}%
\pgfsetfillcolor{currentfill}%
\pgfsetfillopacity{0.700000}%
\pgfsetlinewidth{0.000000pt}%
\definecolor{currentstroke}{rgb}{0.000000,0.000000,0.000000}%
\pgfsetstrokecolor{currentstroke}%
\pgfsetdash{}{0pt}%
\pgfpathmoveto{\pgfqpoint{5.057712in}{1.399708in}}%
\pgfpathlineto{\pgfqpoint{5.072148in}{1.396183in}}%
\pgfpathlineto{\pgfqpoint{5.086592in}{1.392682in}}%
\pgfpathlineto{\pgfqpoint{5.101043in}{1.389204in}}%
\pgfpathlineto{\pgfqpoint{5.115503in}{1.385750in}}%
\pgfpathlineto{\pgfqpoint{5.107688in}{1.376674in}}%
\pgfpathlineto{\pgfqpoint{5.099872in}{1.367792in}}%
\pgfpathlineto{\pgfqpoint{5.092054in}{1.359113in}}%
\pgfpathlineto{\pgfqpoint{5.084234in}{1.350643in}}%
\pgfpathlineto{\pgfqpoint{5.069766in}{1.354400in}}%
\pgfpathlineto{\pgfqpoint{5.055306in}{1.358180in}}%
\pgfpathlineto{\pgfqpoint{5.040853in}{1.361984in}}%
\pgfpathlineto{\pgfqpoint{5.026407in}{1.365810in}}%
\pgfpathlineto{\pgfqpoint{5.034237in}{1.373973in}}%
\pgfpathlineto{\pgfqpoint{5.042064in}{1.382348in}}%
\pgfpathlineto{\pgfqpoint{5.049889in}{1.390929in}}%
\pgfpathlineto{\pgfqpoint{5.057712in}{1.399708in}}%
\pgfpathclose%
\pgfusepath{fill}%
\end{pgfscope}%
\begin{pgfscope}%
\pgfpathrectangle{\pgfqpoint{1.150000in}{0.150000in}}{\pgfqpoint{5.700000in}{5.700000in}}%
\pgfusepath{clip}%
\pgfsetbuttcap%
\pgfsetroundjoin%
\definecolor{currentfill}{rgb}{0.194100,0.399323,0.555565}%
\pgfsetfillcolor{currentfill}%
\pgfsetfillopacity{0.700000}%
\pgfsetlinewidth{0.000000pt}%
\definecolor{currentstroke}{rgb}{0.000000,0.000000,0.000000}%
\pgfsetstrokecolor{currentstroke}%
\pgfsetdash{}{0pt}%
\pgfpathmoveto{\pgfqpoint{3.174281in}{2.177645in}}%
\pgfpathlineto{\pgfqpoint{3.188345in}{2.168068in}}%
\pgfpathlineto{\pgfqpoint{3.202412in}{2.158522in}}%
\pgfpathlineto{\pgfqpoint{3.216482in}{2.149005in}}%
\pgfpathlineto{\pgfqpoint{3.230556in}{2.139518in}}%
\pgfpathlineto{\pgfqpoint{3.221649in}{2.155024in}}%
\pgfpathlineto{\pgfqpoint{3.212718in}{2.171188in}}%
\pgfpathlineto{\pgfqpoint{3.203760in}{2.188023in}}%
\pgfpathlineto{\pgfqpoint{3.194776in}{2.205544in}}%
\pgfpathlineto{\pgfqpoint{3.180652in}{2.215473in}}%
\pgfpathlineto{\pgfqpoint{3.166531in}{2.225431in}}%
\pgfpathlineto{\pgfqpoint{3.152413in}{2.235419in}}%
\pgfpathlineto{\pgfqpoint{3.138298in}{2.245437in}}%
\pgfpathlineto{\pgfqpoint{3.147334in}{2.227467in}}%
\pgfpathlineto{\pgfqpoint{3.156343in}{2.210188in}}%
\pgfpathlineto{\pgfqpoint{3.165325in}{2.193585in}}%
\pgfpathlineto{\pgfqpoint{3.174281in}{2.177645in}}%
\pgfpathclose%
\pgfusepath{fill}%
\end{pgfscope}%
\begin{pgfscope}%
\pgfpathrectangle{\pgfqpoint{1.150000in}{0.150000in}}{\pgfqpoint{5.700000in}{5.700000in}}%
\pgfusepath{clip}%
\pgfsetbuttcap%
\pgfsetroundjoin%
\definecolor{currentfill}{rgb}{0.243113,0.292092,0.538516}%
\pgfsetfillcolor{currentfill}%
\pgfsetfillopacity{0.700000}%
\pgfsetlinewidth{0.000000pt}%
\definecolor{currentstroke}{rgb}{0.000000,0.000000,0.000000}%
\pgfsetstrokecolor{currentstroke}%
\pgfsetdash{}{0pt}%
\pgfpathmoveto{\pgfqpoint{3.547220in}{1.908537in}}%
\pgfpathlineto{\pgfqpoint{3.561323in}{1.900063in}}%
\pgfpathlineto{\pgfqpoint{3.575431in}{1.891616in}}%
\pgfpathlineto{\pgfqpoint{3.589543in}{1.883195in}}%
\pgfpathlineto{\pgfqpoint{3.603659in}{1.874802in}}%
\pgfpathlineto{\pgfqpoint{3.595107in}{1.885310in}}%
\pgfpathlineto{\pgfqpoint{3.586537in}{1.896400in}}%
\pgfpathlineto{\pgfqpoint{3.577948in}{1.908084in}}%
\pgfpathlineto{\pgfqpoint{3.569340in}{1.920375in}}%
\pgfpathlineto{\pgfqpoint{3.555182in}{1.929189in}}%
\pgfpathlineto{\pgfqpoint{3.541028in}{1.938030in}}%
\pgfpathlineto{\pgfqpoint{3.526878in}{1.946897in}}%
\pgfpathlineto{\pgfqpoint{3.512732in}{1.955792in}}%
\pgfpathlineto{\pgfqpoint{3.521383in}{1.943074in}}%
\pgfpathlineto{\pgfqpoint{3.530014in}{1.930967in}}%
\pgfpathlineto{\pgfqpoint{3.538626in}{1.919459in}}%
\pgfpathlineto{\pgfqpoint{3.547220in}{1.908537in}}%
\pgfpathclose%
\pgfusepath{fill}%
\end{pgfscope}%
\begin{pgfscope}%
\pgfpathrectangle{\pgfqpoint{1.150000in}{0.150000in}}{\pgfqpoint{5.700000in}{5.700000in}}%
\pgfusepath{clip}%
\pgfsetbuttcap%
\pgfsetroundjoin%
\definecolor{currentfill}{rgb}{0.133743,0.548535,0.553541}%
\pgfsetfillcolor{currentfill}%
\pgfsetfillopacity{0.700000}%
\pgfsetlinewidth{0.000000pt}%
\definecolor{currentstroke}{rgb}{0.000000,0.000000,0.000000}%
\pgfsetstrokecolor{currentstroke}%
\pgfsetdash{}{0pt}%
\pgfpathmoveto{\pgfqpoint{2.687941in}{2.582920in}}%
\pgfpathlineto{\pgfqpoint{2.701981in}{2.571844in}}%
\pgfpathlineto{\pgfqpoint{2.716023in}{2.560805in}}%
\pgfpathlineto{\pgfqpoint{2.730067in}{2.549802in}}%
\pgfpathlineto{\pgfqpoint{2.744112in}{2.538834in}}%
\pgfpathlineto{\pgfqpoint{2.734647in}{2.560706in}}%
\pgfpathlineto{\pgfqpoint{2.725146in}{2.583329in}}%
\pgfpathlineto{\pgfqpoint{2.715608in}{2.606718in}}%
\pgfpathlineto{\pgfqpoint{2.706032in}{2.630888in}}%
\pgfpathlineto{\pgfqpoint{2.691925in}{2.642323in}}%
\pgfpathlineto{\pgfqpoint{2.677819in}{2.653794in}}%
\pgfpathlineto{\pgfqpoint{2.663714in}{2.665302in}}%
\pgfpathlineto{\pgfqpoint{2.649611in}{2.676847in}}%
\pgfpathlineto{\pgfqpoint{2.659251in}{2.652201in}}%
\pgfpathlineto{\pgfqpoint{2.668852in}{2.628341in}}%
\pgfpathlineto{\pgfqpoint{2.678415in}{2.605252in}}%
\pgfpathlineto{\pgfqpoint{2.687941in}{2.582920in}}%
\pgfpathclose%
\pgfusepath{fill}%
\end{pgfscope}%
\begin{pgfscope}%
\pgfpathrectangle{\pgfqpoint{1.150000in}{0.150000in}}{\pgfqpoint{5.700000in}{5.700000in}}%
\pgfusepath{clip}%
\pgfsetbuttcap%
\pgfsetroundjoin%
\definecolor{currentfill}{rgb}{0.273006,0.204520,0.501721}%
\pgfsetfillcolor{currentfill}%
\pgfsetfillopacity{0.700000}%
\pgfsetlinewidth{0.000000pt}%
\definecolor{currentstroke}{rgb}{0.000000,0.000000,0.000000}%
\pgfsetstrokecolor{currentstroke}%
\pgfsetdash{}{0pt}%
\pgfpathmoveto{\pgfqpoint{3.863530in}{1.714059in}}%
\pgfpathlineto{\pgfqpoint{3.877683in}{1.706509in}}%
\pgfpathlineto{\pgfqpoint{3.891841in}{1.698984in}}%
\pgfpathlineto{\pgfqpoint{3.906004in}{1.691485in}}%
\pgfpathlineto{\pgfqpoint{3.920172in}{1.684010in}}%
\pgfpathlineto{\pgfqpoint{3.911865in}{1.690326in}}%
\pgfpathlineto{\pgfqpoint{3.903545in}{1.697157in}}%
\pgfpathlineto{\pgfqpoint{3.895212in}{1.704515in}}%
\pgfpathlineto{\pgfqpoint{3.886866in}{1.712410in}}%
\pgfpathlineto{\pgfqpoint{3.872664in}{1.720286in}}%
\pgfpathlineto{\pgfqpoint{3.858466in}{1.728187in}}%
\pgfpathlineto{\pgfqpoint{3.844273in}{1.736114in}}%
\pgfpathlineto{\pgfqpoint{3.830085in}{1.744066in}}%
\pgfpathlineto{\pgfqpoint{3.838467in}{1.735763in}}%
\pgfpathlineto{\pgfqpoint{3.846835in}{1.728002in}}%
\pgfpathlineto{\pgfqpoint{3.855189in}{1.720771in}}%
\pgfpathlineto{\pgfqpoint{3.863530in}{1.714059in}}%
\pgfpathclose%
\pgfusepath{fill}%
\end{pgfscope}%
\begin{pgfscope}%
\pgfpathrectangle{\pgfqpoint{1.150000in}{0.150000in}}{\pgfqpoint{5.700000in}{5.700000in}}%
\pgfusepath{clip}%
\pgfsetbuttcap%
\pgfsetroundjoin%
\definecolor{currentfill}{rgb}{0.273809,0.031497,0.358853}%
\pgfsetfillcolor{currentfill}%
\pgfsetfillopacity{0.700000}%
\pgfsetlinewidth{0.000000pt}%
\definecolor{currentstroke}{rgb}{0.000000,0.000000,0.000000}%
\pgfsetstrokecolor{currentstroke}%
\pgfsetdash{}{0pt}%
\pgfpathmoveto{\pgfqpoint{4.822167in}{1.386974in}}%
\pgfpathlineto{\pgfqpoint{4.836537in}{1.382552in}}%
\pgfpathlineto{\pgfqpoint{4.850913in}{1.378154in}}%
\pgfpathlineto{\pgfqpoint{4.865297in}{1.373779in}}%
\pgfpathlineto{\pgfqpoint{4.879688in}{1.369427in}}%
\pgfpathlineto{\pgfqpoint{4.871825in}{1.363098in}}%
\pgfpathlineto{\pgfqpoint{4.863960in}{1.357037in}}%
\pgfpathlineto{\pgfqpoint{4.856093in}{1.351253in}}%
\pgfpathlineto{\pgfqpoint{4.848222in}{1.345755in}}%
\pgfpathlineto{\pgfqpoint{4.833819in}{1.350436in}}%
\pgfpathlineto{\pgfqpoint{4.819422in}{1.355140in}}%
\pgfpathlineto{\pgfqpoint{4.805032in}{1.359867in}}%
\pgfpathlineto{\pgfqpoint{4.790649in}{1.364617in}}%
\pgfpathlineto{\pgfqpoint{4.798533in}{1.369781in}}%
\pgfpathlineto{\pgfqpoint{4.806414in}{1.375234in}}%
\pgfpathlineto{\pgfqpoint{4.814292in}{1.380968in}}%
\pgfpathlineto{\pgfqpoint{4.822167in}{1.386974in}}%
\pgfpathclose%
\pgfusepath{fill}%
\end{pgfscope}%
\begin{pgfscope}%
\pgfpathrectangle{\pgfqpoint{1.150000in}{0.150000in}}{\pgfqpoint{5.700000in}{5.700000in}}%
\pgfusepath{clip}%
\pgfsetbuttcap%
\pgfsetroundjoin%
\definecolor{currentfill}{rgb}{0.282656,0.100196,0.422160}%
\pgfsetfillcolor{currentfill}%
\pgfsetfillopacity{0.700000}%
\pgfsetlinewidth{0.000000pt}%
\definecolor{currentstroke}{rgb}{0.000000,0.000000,0.000000}%
\pgfsetstrokecolor{currentstroke}%
\pgfsetdash{}{0pt}%
\pgfpathmoveto{\pgfqpoint{4.326372in}{1.496859in}}%
\pgfpathlineto{\pgfqpoint{4.340615in}{1.490752in}}%
\pgfpathlineto{\pgfqpoint{4.354865in}{1.484668in}}%
\pgfpathlineto{\pgfqpoint{4.369120in}{1.478608in}}%
\pgfpathlineto{\pgfqpoint{4.383381in}{1.472573in}}%
\pgfpathlineto{\pgfqpoint{4.375345in}{1.472607in}}%
\pgfpathlineto{\pgfqpoint{4.367303in}{1.473045in}}%
\pgfpathlineto{\pgfqpoint{4.359253in}{1.473896in}}%
\pgfpathlineto{\pgfqpoint{4.351197in}{1.475171in}}%
\pgfpathlineto{\pgfqpoint{4.336912in}{1.481577in}}%
\pgfpathlineto{\pgfqpoint{4.322633in}{1.488007in}}%
\pgfpathlineto{\pgfqpoint{4.308359in}{1.494461in}}%
\pgfpathlineto{\pgfqpoint{4.294091in}{1.500939in}}%
\pgfpathlineto{\pgfqpoint{4.302173in}{1.499288in}}%
\pgfpathlineto{\pgfqpoint{4.310246in}{1.498064in}}%
\pgfpathlineto{\pgfqpoint{4.318313in}{1.497258in}}%
\pgfpathlineto{\pgfqpoint{4.326372in}{1.496859in}}%
\pgfpathclose%
\pgfusepath{fill}%
\end{pgfscope}%
\begin{pgfscope}%
\pgfpathrectangle{\pgfqpoint{1.150000in}{0.150000in}}{\pgfqpoint{5.700000in}{5.700000in}}%
\pgfusepath{clip}%
\pgfsetbuttcap%
\pgfsetroundjoin%
\definecolor{currentfill}{rgb}{0.276022,0.044167,0.370164}%
\pgfsetfillcolor{currentfill}%
\pgfsetfillopacity{0.700000}%
\pgfsetlinewidth{0.000000pt}%
\definecolor{currentstroke}{rgb}{0.000000,0.000000,0.000000}%
\pgfsetstrokecolor{currentstroke}%
\pgfsetdash{}{0pt}%
\pgfpathmoveto{\pgfqpoint{4.675828in}{1.403460in}}%
\pgfpathlineto{\pgfqpoint{4.690157in}{1.398523in}}%
\pgfpathlineto{\pgfqpoint{4.704493in}{1.393609in}}%
\pgfpathlineto{\pgfqpoint{4.718835in}{1.388719in}}%
\pgfpathlineto{\pgfqpoint{4.733184in}{1.383851in}}%
\pgfpathlineto{\pgfqpoint{4.725282in}{1.379322in}}%
\pgfpathlineto{\pgfqpoint{4.717377in}{1.375103in}}%
\pgfpathlineto{\pgfqpoint{4.709468in}{1.371203in}}%
\pgfpathlineto{\pgfqpoint{4.701555in}{1.367631in}}%
\pgfpathlineto{\pgfqpoint{4.687190in}{1.372841in}}%
\pgfpathlineto{\pgfqpoint{4.672831in}{1.378073in}}%
\pgfpathlineto{\pgfqpoint{4.658479in}{1.383329in}}%
\pgfpathlineto{\pgfqpoint{4.644133in}{1.388608in}}%
\pgfpathlineto{\pgfqpoint{4.652063in}{1.391832in}}%
\pgfpathlineto{\pgfqpoint{4.659988in}{1.395389in}}%
\pgfpathlineto{\pgfqpoint{4.667910in}{1.399267in}}%
\pgfpathlineto{\pgfqpoint{4.675828in}{1.403460in}}%
\pgfpathclose%
\pgfusepath{fill}%
\end{pgfscope}%
\begin{pgfscope}%
\pgfpathrectangle{\pgfqpoint{1.150000in}{0.150000in}}{\pgfqpoint{5.700000in}{5.700000in}}%
\pgfusepath{clip}%
\pgfsetbuttcap%
\pgfsetroundjoin%
\definecolor{currentfill}{rgb}{0.139147,0.533812,0.555298}%
\pgfsetfillcolor{currentfill}%
\pgfsetfillopacity{0.700000}%
\pgfsetlinewidth{0.000000pt}%
\definecolor{currentstroke}{rgb}{0.000000,0.000000,0.000000}%
\pgfsetstrokecolor{currentstroke}%
\pgfsetdash{}{0pt}%
\pgfpathmoveto{\pgfqpoint{2.744112in}{2.538834in}}%
\pgfpathlineto{\pgfqpoint{2.758160in}{2.527903in}}%
\pgfpathlineto{\pgfqpoint{2.772209in}{2.517007in}}%
\pgfpathlineto{\pgfqpoint{2.786260in}{2.506146in}}%
\pgfpathlineto{\pgfqpoint{2.800314in}{2.495320in}}%
\pgfpathlineto{\pgfqpoint{2.790909in}{2.516732in}}%
\pgfpathlineto{\pgfqpoint{2.781469in}{2.538890in}}%
\pgfpathlineto{\pgfqpoint{2.771993in}{2.561809in}}%
\pgfpathlineto{\pgfqpoint{2.762481in}{2.585504in}}%
\pgfpathlineto{\pgfqpoint{2.748366in}{2.596796in}}%
\pgfpathlineto{\pgfqpoint{2.734253in}{2.608125in}}%
\pgfpathlineto{\pgfqpoint{2.720142in}{2.619488in}}%
\pgfpathlineto{\pgfqpoint{2.706032in}{2.630888in}}%
\pgfpathlineto{\pgfqpoint{2.715608in}{2.606718in}}%
\pgfpathlineto{\pgfqpoint{2.725146in}{2.583329in}}%
\pgfpathlineto{\pgfqpoint{2.734647in}{2.560706in}}%
\pgfpathlineto{\pgfqpoint{2.744112in}{2.538834in}}%
\pgfpathclose%
\pgfusepath{fill}%
\end{pgfscope}%
\begin{pgfscope}%
\pgfpathrectangle{\pgfqpoint{1.150000in}{0.150000in}}{\pgfqpoint{5.700000in}{5.700000in}}%
\pgfusepath{clip}%
\pgfsetbuttcap%
\pgfsetroundjoin%
\definecolor{currentfill}{rgb}{0.282623,0.140926,0.457517}%
\pgfsetfillcolor{currentfill}%
\pgfsetfillopacity{0.700000}%
\pgfsetlinewidth{0.000000pt}%
\definecolor{currentstroke}{rgb}{0.000000,0.000000,0.000000}%
\pgfsetstrokecolor{currentstroke}%
\pgfsetdash{}{0pt}%
\pgfpathmoveto{\pgfqpoint{4.123304in}{1.580547in}}%
\pgfpathlineto{\pgfqpoint{4.137507in}{1.573780in}}%
\pgfpathlineto{\pgfqpoint{4.151715in}{1.567037in}}%
\pgfpathlineto{\pgfqpoint{4.165928in}{1.560319in}}%
\pgfpathlineto{\pgfqpoint{4.180146in}{1.553624in}}%
\pgfpathlineto{\pgfqpoint{4.172004in}{1.556472in}}%
\pgfpathlineto{\pgfqpoint{4.163853in}{1.559775in}}%
\pgfpathlineto{\pgfqpoint{4.155692in}{1.563545in}}%
\pgfpathlineto{\pgfqpoint{4.147522in}{1.567791in}}%
\pgfpathlineto{\pgfqpoint{4.133275in}{1.574871in}}%
\pgfpathlineto{\pgfqpoint{4.119033in}{1.581975in}}%
\pgfpathlineto{\pgfqpoint{4.104796in}{1.589103in}}%
\pgfpathlineto{\pgfqpoint{4.090564in}{1.596256in}}%
\pgfpathlineto{\pgfqpoint{4.098764in}{1.591618in}}%
\pgfpathlineto{\pgfqpoint{4.106954in}{1.587461in}}%
\pgfpathlineto{\pgfqpoint{4.115134in}{1.583775in}}%
\pgfpathlineto{\pgfqpoint{4.123304in}{1.580547in}}%
\pgfpathclose%
\pgfusepath{fill}%
\end{pgfscope}%
\begin{pgfscope}%
\pgfpathrectangle{\pgfqpoint{1.150000in}{0.150000in}}{\pgfqpoint{5.700000in}{5.700000in}}%
\pgfusepath{clip}%
\pgfsetbuttcap%
\pgfsetroundjoin%
\definecolor{currentfill}{rgb}{0.199430,0.387607,0.554642}%
\pgfsetfillcolor{currentfill}%
\pgfsetfillopacity{0.700000}%
\pgfsetlinewidth{0.000000pt}%
\definecolor{currentstroke}{rgb}{0.000000,0.000000,0.000000}%
\pgfsetstrokecolor{currentstroke}%
\pgfsetdash{}{0pt}%
\pgfpathmoveto{\pgfqpoint{3.230556in}{2.139518in}}%
\pgfpathlineto{\pgfqpoint{3.244632in}{2.130061in}}%
\pgfpathlineto{\pgfqpoint{3.258712in}{2.120633in}}%
\pgfpathlineto{\pgfqpoint{3.272795in}{2.111234in}}%
\pgfpathlineto{\pgfqpoint{3.286882in}{2.101864in}}%
\pgfpathlineto{\pgfqpoint{3.278024in}{2.116935in}}%
\pgfpathlineto{\pgfqpoint{3.269142in}{2.132660in}}%
\pgfpathlineto{\pgfqpoint{3.260235in}{2.149053in}}%
\pgfpathlineto{\pgfqpoint{3.251303in}{2.166126in}}%
\pgfpathlineto{\pgfqpoint{3.237166in}{2.175937in}}%
\pgfpathlineto{\pgfqpoint{3.223033in}{2.185776in}}%
\pgfpathlineto{\pgfqpoint{3.208903in}{2.195646in}}%
\pgfpathlineto{\pgfqpoint{3.194776in}{2.205544in}}%
\pgfpathlineto{\pgfqpoint{3.203760in}{2.188023in}}%
\pgfpathlineto{\pgfqpoint{3.212718in}{2.171188in}}%
\pgfpathlineto{\pgfqpoint{3.221649in}{2.155024in}}%
\pgfpathlineto{\pgfqpoint{3.230556in}{2.139518in}}%
\pgfpathclose%
\pgfusepath{fill}%
\end{pgfscope}%
\begin{pgfscope}%
\pgfpathrectangle{\pgfqpoint{1.150000in}{0.150000in}}{\pgfqpoint{5.700000in}{5.700000in}}%
\pgfusepath{clip}%
\pgfsetbuttcap%
\pgfsetroundjoin%
\definecolor{currentfill}{rgb}{0.273809,0.031497,0.358853}%
\pgfsetfillcolor{currentfill}%
\pgfsetfillopacity{0.700000}%
\pgfsetlinewidth{0.000000pt}%
\definecolor{currentstroke}{rgb}{0.000000,0.000000,0.000000}%
\pgfsetstrokecolor{currentstroke}%
\pgfsetdash{}{0pt}%
\pgfpathmoveto{\pgfqpoint{4.968701in}{1.381350in}}%
\pgfpathlineto{\pgfqpoint{4.983116in}{1.377430in}}%
\pgfpathlineto{\pgfqpoint{4.997539in}{1.373534in}}%
\pgfpathlineto{\pgfqpoint{5.011970in}{1.369660in}}%
\pgfpathlineto{\pgfqpoint{5.026407in}{1.365810in}}%
\pgfpathlineto{\pgfqpoint{5.018576in}{1.357869in}}%
\pgfpathlineto{\pgfqpoint{5.010743in}{1.350156in}}%
\pgfpathlineto{\pgfqpoint{5.002908in}{1.342681in}}%
\pgfpathlineto{\pgfqpoint{4.995071in}{1.335450in}}%
\pgfpathlineto{\pgfqpoint{4.980623in}{1.339616in}}%
\pgfpathlineto{\pgfqpoint{4.966182in}{1.343805in}}%
\pgfpathlineto{\pgfqpoint{4.951748in}{1.348017in}}%
\pgfpathlineto{\pgfqpoint{4.937322in}{1.352253in}}%
\pgfpathlineto{\pgfqpoint{4.945170in}{1.359162in}}%
\pgfpathlineto{\pgfqpoint{4.953016in}{1.366321in}}%
\pgfpathlineto{\pgfqpoint{4.960859in}{1.373719in}}%
\pgfpathlineto{\pgfqpoint{4.968701in}{1.381350in}}%
\pgfpathclose%
\pgfusepath{fill}%
\end{pgfscope}%
\begin{pgfscope}%
\pgfpathrectangle{\pgfqpoint{1.150000in}{0.150000in}}{\pgfqpoint{5.700000in}{5.700000in}}%
\pgfusepath{clip}%
\pgfsetbuttcap%
\pgfsetroundjoin%
\definecolor{currentfill}{rgb}{0.246811,0.283237,0.535941}%
\pgfsetfillcolor{currentfill}%
\pgfsetfillopacity{0.700000}%
\pgfsetlinewidth{0.000000pt}%
\definecolor{currentstroke}{rgb}{0.000000,0.000000,0.000000}%
\pgfsetstrokecolor{currentstroke}%
\pgfsetdash{}{0pt}%
\pgfpathmoveto{\pgfqpoint{3.603659in}{1.874802in}}%
\pgfpathlineto{\pgfqpoint{3.617779in}{1.866434in}}%
\pgfpathlineto{\pgfqpoint{3.631903in}{1.858094in}}%
\pgfpathlineto{\pgfqpoint{3.646031in}{1.849780in}}%
\pgfpathlineto{\pgfqpoint{3.660164in}{1.841492in}}%
\pgfpathlineto{\pgfqpoint{3.651652in}{1.851587in}}%
\pgfpathlineto{\pgfqpoint{3.643122in}{1.862259in}}%
\pgfpathlineto{\pgfqpoint{3.634575in}{1.873521in}}%
\pgfpathlineto{\pgfqpoint{3.626010in}{1.885386in}}%
\pgfpathlineto{\pgfqpoint{3.611837in}{1.894094in}}%
\pgfpathlineto{\pgfqpoint{3.597667in}{1.902828in}}%
\pgfpathlineto{\pgfqpoint{3.583502in}{1.911588in}}%
\pgfpathlineto{\pgfqpoint{3.569340in}{1.920375in}}%
\pgfpathlineto{\pgfqpoint{3.577948in}{1.908084in}}%
\pgfpathlineto{\pgfqpoint{3.586537in}{1.896400in}}%
\pgfpathlineto{\pgfqpoint{3.595107in}{1.885310in}}%
\pgfpathlineto{\pgfqpoint{3.603659in}{1.874802in}}%
\pgfpathclose%
\pgfusepath{fill}%
\end{pgfscope}%
\begin{pgfscope}%
\pgfpathrectangle{\pgfqpoint{1.150000in}{0.150000in}}{\pgfqpoint{5.700000in}{5.700000in}}%
\pgfusepath{clip}%
\pgfsetbuttcap%
\pgfsetroundjoin%
\definecolor{currentfill}{rgb}{0.279566,0.067836,0.391917}%
\pgfsetfillcolor{currentfill}%
\pgfsetfillopacity{0.700000}%
\pgfsetlinewidth{0.000000pt}%
\definecolor{currentstroke}{rgb}{0.000000,0.000000,0.000000}%
\pgfsetstrokecolor{currentstroke}%
\pgfsetdash{}{0pt}%
\pgfpathmoveto{\pgfqpoint{4.529597in}{1.431686in}}%
\pgfpathlineto{\pgfqpoint{4.543892in}{1.426219in}}%
\pgfpathlineto{\pgfqpoint{4.558193in}{1.420775in}}%
\pgfpathlineto{\pgfqpoint{4.572500in}{1.415355in}}%
\pgfpathlineto{\pgfqpoint{4.586814in}{1.409959in}}%
\pgfpathlineto{\pgfqpoint{4.578862in}{1.407426in}}%
\pgfpathlineto{\pgfqpoint{4.570906in}{1.405247in}}%
\pgfpathlineto{\pgfqpoint{4.562945in}{1.403432in}}%
\pgfpathlineto{\pgfqpoint{4.554979in}{1.401989in}}%
\pgfpathlineto{\pgfqpoint{4.540646in}{1.407741in}}%
\pgfpathlineto{\pgfqpoint{4.526319in}{1.413517in}}%
\pgfpathlineto{\pgfqpoint{4.511998in}{1.419317in}}%
\pgfpathlineto{\pgfqpoint{4.497683in}{1.425139in}}%
\pgfpathlineto{\pgfqpoint{4.505669in}{1.426221in}}%
\pgfpathlineto{\pgfqpoint{4.513650in}{1.427679in}}%
\pgfpathlineto{\pgfqpoint{4.521626in}{1.429504in}}%
\pgfpathlineto{\pgfqpoint{4.529597in}{1.431686in}}%
\pgfpathclose%
\pgfusepath{fill}%
\end{pgfscope}%
\begin{pgfscope}%
\pgfpathrectangle{\pgfqpoint{1.150000in}{0.150000in}}{\pgfqpoint{5.700000in}{5.700000in}}%
\pgfusepath{clip}%
\pgfsetbuttcap%
\pgfsetroundjoin%
\definecolor{currentfill}{rgb}{0.360741,0.785964,0.387814}%
\pgfsetfillcolor{currentfill}%
\pgfsetfillopacity{0.700000}%
\pgfsetlinewidth{0.000000pt}%
\definecolor{currentstroke}{rgb}{0.000000,0.000000,0.000000}%
\pgfsetstrokecolor{currentstroke}%
\pgfsetdash{}{0pt}%
\pgfpathmoveto{\pgfqpoint{1.973670in}{3.280753in}}%
\pgfpathlineto{\pgfqpoint{1.987751in}{3.267021in}}%
\pgfpathlineto{\pgfqpoint{2.001831in}{3.253346in}}%
\pgfpathlineto{\pgfqpoint{2.015911in}{3.239727in}}%
\pgfpathlineto{\pgfqpoint{2.029990in}{3.226163in}}%
\pgfpathlineto{\pgfqpoint{2.019523in}{3.257009in}}%
\pgfpathlineto{\pgfqpoint{2.009002in}{3.288732in}}%
\pgfpathlineto{\pgfqpoint{1.998425in}{3.321348in}}%
\pgfpathlineto{\pgfqpoint{1.987792in}{3.354875in}}%
\pgfpathlineto{\pgfqpoint{1.973634in}{3.368948in}}%
\pgfpathlineto{\pgfqpoint{1.959474in}{3.383078in}}%
\pgfpathlineto{\pgfqpoint{1.945314in}{3.397264in}}%
\pgfpathlineto{\pgfqpoint{1.931153in}{3.411508in}}%
\pgfpathlineto{\pgfqpoint{1.941868in}{3.377462in}}%
\pgfpathlineto{\pgfqpoint{1.952525in}{3.344332in}}%
\pgfpathlineto{\pgfqpoint{1.963125in}{3.312101in}}%
\pgfpathlineto{\pgfqpoint{1.973670in}{3.280753in}}%
\pgfpathclose%
\pgfusepath{fill}%
\end{pgfscope}%
\begin{pgfscope}%
\pgfpathrectangle{\pgfqpoint{1.150000in}{0.150000in}}{\pgfqpoint{5.700000in}{5.700000in}}%
\pgfusepath{clip}%
\pgfsetbuttcap%
\pgfsetroundjoin%
\definecolor{currentfill}{rgb}{0.274952,0.037752,0.364543}%
\pgfsetfillcolor{currentfill}%
\pgfsetfillopacity{0.700000}%
\pgfsetlinewidth{0.000000pt}%
\definecolor{currentstroke}{rgb}{0.000000,0.000000,0.000000}%
\pgfsetstrokecolor{currentstroke}%
\pgfsetdash{}{0pt}%
\pgfpathmoveto{\pgfqpoint{5.115503in}{1.385750in}}%
\pgfpathlineto{\pgfqpoint{5.129970in}{1.382318in}}%
\pgfpathlineto{\pgfqpoint{5.144445in}{1.378911in}}%
\pgfpathlineto{\pgfqpoint{5.158928in}{1.375526in}}%
\pgfpathlineto{\pgfqpoint{5.151119in}{1.366227in}}%
\pgfpathlineto{\pgfqpoint{5.143309in}{1.357120in}}%
\pgfpathlineto{\pgfqpoint{5.135497in}{1.348213in}}%
\pgfpathlineto{\pgfqpoint{5.127684in}{1.339512in}}%
\pgfpathlineto{\pgfqpoint{5.113193in}{1.343199in}}%
\pgfpathlineto{\pgfqpoint{5.098710in}{1.346909in}}%
\pgfpathlineto{\pgfqpoint{5.084234in}{1.350643in}}%
\pgfpathlineto{\pgfqpoint{5.092054in}{1.359113in}}%
\pgfpathlineto{\pgfqpoint{5.099872in}{1.367792in}}%
\pgfpathlineto{\pgfqpoint{5.107688in}{1.376674in}}%
\pgfpathlineto{\pgfqpoint{5.115503in}{1.385750in}}%
\pgfpathclose%
\pgfusepath{fill}%
\end{pgfscope}%
\begin{pgfscope}%
\pgfpathrectangle{\pgfqpoint{1.150000in}{0.150000in}}{\pgfqpoint{5.700000in}{5.700000in}}%
\pgfusepath{clip}%
\pgfsetbuttcap%
\pgfsetroundjoin%
\definecolor{currentfill}{rgb}{0.143343,0.522773,0.556295}%
\pgfsetfillcolor{currentfill}%
\pgfsetfillopacity{0.700000}%
\pgfsetlinewidth{0.000000pt}%
\definecolor{currentstroke}{rgb}{0.000000,0.000000,0.000000}%
\pgfsetstrokecolor{currentstroke}%
\pgfsetdash{}{0pt}%
\pgfpathmoveto{\pgfqpoint{2.800314in}{2.495320in}}%
\pgfpathlineto{\pgfqpoint{2.814370in}{2.484529in}}%
\pgfpathlineto{\pgfqpoint{2.828427in}{2.473772in}}%
\pgfpathlineto{\pgfqpoint{2.842487in}{2.463050in}}%
\pgfpathlineto{\pgfqpoint{2.856549in}{2.452362in}}%
\pgfpathlineto{\pgfqpoint{2.847203in}{2.473314in}}%
\pgfpathlineto{\pgfqpoint{2.837823in}{2.495008in}}%
\pgfpathlineto{\pgfqpoint{2.828409in}{2.517459in}}%
\pgfpathlineto{\pgfqpoint{2.818958in}{2.540680in}}%
\pgfpathlineto{\pgfqpoint{2.804836in}{2.551834in}}%
\pgfpathlineto{\pgfqpoint{2.790716in}{2.563022in}}%
\pgfpathlineto{\pgfqpoint{2.776597in}{2.574246in}}%
\pgfpathlineto{\pgfqpoint{2.762481in}{2.585504in}}%
\pgfpathlineto{\pgfqpoint{2.771993in}{2.561809in}}%
\pgfpathlineto{\pgfqpoint{2.781469in}{2.538890in}}%
\pgfpathlineto{\pgfqpoint{2.790909in}{2.516732in}}%
\pgfpathlineto{\pgfqpoint{2.800314in}{2.495320in}}%
\pgfpathclose%
\pgfusepath{fill}%
\end{pgfscope}%
\begin{pgfscope}%
\pgfpathrectangle{\pgfqpoint{1.150000in}{0.150000in}}{\pgfqpoint{5.700000in}{5.700000in}}%
\pgfusepath{clip}%
\pgfsetbuttcap%
\pgfsetroundjoin%
\definecolor{currentfill}{rgb}{0.319809,0.770914,0.411152}%
\pgfsetfillcolor{currentfill}%
\pgfsetfillopacity{0.700000}%
\pgfsetlinewidth{0.000000pt}%
\definecolor{currentstroke}{rgb}{0.000000,0.000000,0.000000}%
\pgfsetstrokecolor{currentstroke}%
\pgfsetdash{}{0pt}%
\pgfpathmoveto{\pgfqpoint{2.029990in}{3.226163in}}%
\pgfpathlineto{\pgfqpoint{2.044068in}{3.212655in}}%
\pgfpathlineto{\pgfqpoint{2.058145in}{3.199201in}}%
\pgfpathlineto{\pgfqpoint{2.072222in}{3.185800in}}%
\pgfpathlineto{\pgfqpoint{2.086299in}{3.172453in}}%
\pgfpathlineto{\pgfqpoint{2.075909in}{3.202799in}}%
\pgfpathlineto{\pgfqpoint{2.065466in}{3.234016in}}%
\pgfpathlineto{\pgfqpoint{2.054969in}{3.266121in}}%
\pgfpathlineto{\pgfqpoint{2.044416in}{3.299130in}}%
\pgfpathlineto{\pgfqpoint{2.030262in}{3.312985in}}%
\pgfpathlineto{\pgfqpoint{2.016106in}{3.326893in}}%
\pgfpathlineto{\pgfqpoint{2.001949in}{3.340857in}}%
\pgfpathlineto{\pgfqpoint{1.987792in}{3.354875in}}%
\pgfpathlineto{\pgfqpoint{1.998425in}{3.321348in}}%
\pgfpathlineto{\pgfqpoint{2.009002in}{3.288732in}}%
\pgfpathlineto{\pgfqpoint{2.019523in}{3.257009in}}%
\pgfpathlineto{\pgfqpoint{2.029990in}{3.226163in}}%
\pgfpathclose%
\pgfusepath{fill}%
\end{pgfscope}%
\begin{pgfscope}%
\pgfpathrectangle{\pgfqpoint{1.150000in}{0.150000in}}{\pgfqpoint{5.700000in}{5.700000in}}%
\pgfusepath{clip}%
\pgfsetbuttcap%
\pgfsetroundjoin%
\definecolor{currentfill}{rgb}{0.274128,0.199721,0.498911}%
\pgfsetfillcolor{currentfill}%
\pgfsetfillopacity{0.700000}%
\pgfsetlinewidth{0.000000pt}%
\definecolor{currentstroke}{rgb}{0.000000,0.000000,0.000000}%
\pgfsetstrokecolor{currentstroke}%
\pgfsetdash{}{0pt}%
\pgfpathmoveto{\pgfqpoint{3.920172in}{1.684010in}}%
\pgfpathlineto{\pgfqpoint{3.934344in}{1.676561in}}%
\pgfpathlineto{\pgfqpoint{3.948521in}{1.669136in}}%
\pgfpathlineto{\pgfqpoint{3.962704in}{1.661737in}}%
\pgfpathlineto{\pgfqpoint{3.976890in}{1.654362in}}%
\pgfpathlineto{\pgfqpoint{3.968617in}{1.660283in}}%
\pgfpathlineto{\pgfqpoint{3.960330in}{1.666714in}}%
\pgfpathlineto{\pgfqpoint{3.952032in}{1.673668in}}%
\pgfpathlineto{\pgfqpoint{3.943721in}{1.681155in}}%
\pgfpathlineto{\pgfqpoint{3.929500in}{1.688931in}}%
\pgfpathlineto{\pgfqpoint{3.915284in}{1.696732in}}%
\pgfpathlineto{\pgfqpoint{3.901073in}{1.704558in}}%
\pgfpathlineto{\pgfqpoint{3.886866in}{1.712410in}}%
\pgfpathlineto{\pgfqpoint{3.895212in}{1.704515in}}%
\pgfpathlineto{\pgfqpoint{3.903545in}{1.697157in}}%
\pgfpathlineto{\pgfqpoint{3.911865in}{1.690326in}}%
\pgfpathlineto{\pgfqpoint{3.920172in}{1.684010in}}%
\pgfpathclose%
\pgfusepath{fill}%
\end{pgfscope}%
\begin{pgfscope}%
\pgfpathrectangle{\pgfqpoint{1.150000in}{0.150000in}}{\pgfqpoint{5.700000in}{5.700000in}}%
\pgfusepath{clip}%
\pgfsetbuttcap%
\pgfsetroundjoin%
\definecolor{currentfill}{rgb}{0.203063,0.379716,0.553925}%
\pgfsetfillcolor{currentfill}%
\pgfsetfillopacity{0.700000}%
\pgfsetlinewidth{0.000000pt}%
\definecolor{currentstroke}{rgb}{0.000000,0.000000,0.000000}%
\pgfsetstrokecolor{currentstroke}%
\pgfsetdash{}{0pt}%
\pgfpathmoveto{\pgfqpoint{3.286882in}{2.101864in}}%
\pgfpathlineto{\pgfqpoint{3.300971in}{2.092522in}}%
\pgfpathlineto{\pgfqpoint{3.315064in}{2.083210in}}%
\pgfpathlineto{\pgfqpoint{3.329160in}{2.073926in}}%
\pgfpathlineto{\pgfqpoint{3.343260in}{2.064671in}}%
\pgfpathlineto{\pgfqpoint{3.334450in}{2.079309in}}%
\pgfpathlineto{\pgfqpoint{3.325617in}{2.094596in}}%
\pgfpathlineto{\pgfqpoint{3.316760in}{2.110547in}}%
\pgfpathlineto{\pgfqpoint{3.307878in}{2.127173in}}%
\pgfpathlineto{\pgfqpoint{3.293730in}{2.136868in}}%
\pgfpathlineto{\pgfqpoint{3.279584in}{2.146592in}}%
\pgfpathlineto{\pgfqpoint{3.265442in}{2.156344in}}%
\pgfpathlineto{\pgfqpoint{3.251303in}{2.166126in}}%
\pgfpathlineto{\pgfqpoint{3.260235in}{2.149053in}}%
\pgfpathlineto{\pgfqpoint{3.269142in}{2.132660in}}%
\pgfpathlineto{\pgfqpoint{3.278024in}{2.116935in}}%
\pgfpathlineto{\pgfqpoint{3.286882in}{2.101864in}}%
\pgfpathclose%
\pgfusepath{fill}%
\end{pgfscope}%
\begin{pgfscope}%
\pgfpathrectangle{\pgfqpoint{1.150000in}{0.150000in}}{\pgfqpoint{5.700000in}{5.700000in}}%
\pgfusepath{clip}%
\pgfsetbuttcap%
\pgfsetroundjoin%
\definecolor{currentfill}{rgb}{0.288921,0.758394,0.428426}%
\pgfsetfillcolor{currentfill}%
\pgfsetfillopacity{0.700000}%
\pgfsetlinewidth{0.000000pt}%
\definecolor{currentstroke}{rgb}{0.000000,0.000000,0.000000}%
\pgfsetstrokecolor{currentstroke}%
\pgfsetdash{}{0pt}%
\pgfpathmoveto{\pgfqpoint{2.086299in}{3.172453in}}%
\pgfpathlineto{\pgfqpoint{2.100375in}{3.159159in}}%
\pgfpathlineto{\pgfqpoint{2.114451in}{3.145916in}}%
\pgfpathlineto{\pgfqpoint{2.128526in}{3.132726in}}%
\pgfpathlineto{\pgfqpoint{2.142602in}{3.119586in}}%
\pgfpathlineto{\pgfqpoint{2.132288in}{3.149434in}}%
\pgfpathlineto{\pgfqpoint{2.121922in}{3.180148in}}%
\pgfpathlineto{\pgfqpoint{2.111503in}{3.211743in}}%
\pgfpathlineto{\pgfqpoint{2.101030in}{3.244237in}}%
\pgfpathlineto{\pgfqpoint{2.086877in}{3.257882in}}%
\pgfpathlineto{\pgfqpoint{2.072724in}{3.271579in}}%
\pgfpathlineto{\pgfqpoint{2.058571in}{3.285328in}}%
\pgfpathlineto{\pgfqpoint{2.044416in}{3.299130in}}%
\pgfpathlineto{\pgfqpoint{2.054969in}{3.266121in}}%
\pgfpathlineto{\pgfqpoint{2.065466in}{3.234016in}}%
\pgfpathlineto{\pgfqpoint{2.075909in}{3.202799in}}%
\pgfpathlineto{\pgfqpoint{2.086299in}{3.172453in}}%
\pgfpathclose%
\pgfusepath{fill}%
\end{pgfscope}%
\begin{pgfscope}%
\pgfpathrectangle{\pgfqpoint{1.150000in}{0.150000in}}{\pgfqpoint{5.700000in}{5.700000in}}%
\pgfusepath{clip}%
\pgfsetbuttcap%
\pgfsetroundjoin%
\definecolor{currentfill}{rgb}{0.282327,0.094955,0.417331}%
\pgfsetfillcolor{currentfill}%
\pgfsetfillopacity{0.700000}%
\pgfsetlinewidth{0.000000pt}%
\definecolor{currentstroke}{rgb}{0.000000,0.000000,0.000000}%
\pgfsetstrokecolor{currentstroke}%
\pgfsetdash{}{0pt}%
\pgfpathmoveto{\pgfqpoint{4.383381in}{1.472573in}}%
\pgfpathlineto{\pgfqpoint{4.397648in}{1.466561in}}%
\pgfpathlineto{\pgfqpoint{4.411921in}{1.460572in}}%
\pgfpathlineto{\pgfqpoint{4.426200in}{1.454608in}}%
\pgfpathlineto{\pgfqpoint{4.440484in}{1.448667in}}%
\pgfpathlineto{\pgfqpoint{4.432471in}{1.448336in}}%
\pgfpathlineto{\pgfqpoint{4.424451in}{1.448406in}}%
\pgfpathlineto{\pgfqpoint{4.416426in}{1.448885in}}%
\pgfpathlineto{\pgfqpoint{4.408394in}{1.449784in}}%
\pgfpathlineto{\pgfqpoint{4.394086in}{1.456095in}}%
\pgfpathlineto{\pgfqpoint{4.379784in}{1.462430in}}%
\pgfpathlineto{\pgfqpoint{4.365488in}{1.468789in}}%
\pgfpathlineto{\pgfqpoint{4.351197in}{1.475171in}}%
\pgfpathlineto{\pgfqpoint{4.359253in}{1.473896in}}%
\pgfpathlineto{\pgfqpoint{4.367303in}{1.473045in}}%
\pgfpathlineto{\pgfqpoint{4.375345in}{1.472607in}}%
\pgfpathlineto{\pgfqpoint{4.383381in}{1.472573in}}%
\pgfpathclose%
\pgfusepath{fill}%
\end{pgfscope}%
\begin{pgfscope}%
\pgfpathrectangle{\pgfqpoint{1.150000in}{0.150000in}}{\pgfqpoint{5.700000in}{5.700000in}}%
\pgfusepath{clip}%
\pgfsetbuttcap%
\pgfsetroundjoin%
\definecolor{currentfill}{rgb}{0.282884,0.135920,0.453427}%
\pgfsetfillcolor{currentfill}%
\pgfsetfillopacity{0.700000}%
\pgfsetlinewidth{0.000000pt}%
\definecolor{currentstroke}{rgb}{0.000000,0.000000,0.000000}%
\pgfsetstrokecolor{currentstroke}%
\pgfsetdash{}{0pt}%
\pgfpathmoveto{\pgfqpoint{4.180146in}{1.553624in}}%
\pgfpathlineto{\pgfqpoint{4.194370in}{1.546954in}}%
\pgfpathlineto{\pgfqpoint{4.208600in}{1.540308in}}%
\pgfpathlineto{\pgfqpoint{4.222835in}{1.533687in}}%
\pgfpathlineto{\pgfqpoint{4.237075in}{1.527089in}}%
\pgfpathlineto{\pgfqpoint{4.228960in}{1.529557in}}%
\pgfpathlineto{\pgfqpoint{4.220836in}{1.532477in}}%
\pgfpathlineto{\pgfqpoint{4.212704in}{1.535859in}}%
\pgfpathlineto{\pgfqpoint{4.204563in}{1.539714in}}%
\pgfpathlineto{\pgfqpoint{4.190295in}{1.546697in}}%
\pgfpathlineto{\pgfqpoint{4.176032in}{1.553704in}}%
\pgfpathlineto{\pgfqpoint{4.161774in}{1.560736in}}%
\pgfpathlineto{\pgfqpoint{4.147522in}{1.567791in}}%
\pgfpathlineto{\pgfqpoint{4.155692in}{1.563545in}}%
\pgfpathlineto{\pgfqpoint{4.163853in}{1.559775in}}%
\pgfpathlineto{\pgfqpoint{4.172004in}{1.556472in}}%
\pgfpathlineto{\pgfqpoint{4.180146in}{1.553624in}}%
\pgfpathclose%
\pgfusepath{fill}%
\end{pgfscope}%
\begin{pgfscope}%
\pgfpathrectangle{\pgfqpoint{1.150000in}{0.150000in}}{\pgfqpoint{5.700000in}{5.700000in}}%
\pgfusepath{clip}%
\pgfsetbuttcap%
\pgfsetroundjoin%
\definecolor{currentfill}{rgb}{0.250425,0.274290,0.533103}%
\pgfsetfillcolor{currentfill}%
\pgfsetfillopacity{0.700000}%
\pgfsetlinewidth{0.000000pt}%
\definecolor{currentstroke}{rgb}{0.000000,0.000000,0.000000}%
\pgfsetstrokecolor{currentstroke}%
\pgfsetdash{}{0pt}%
\pgfpathmoveto{\pgfqpoint{3.660164in}{1.841492in}}%
\pgfpathlineto{\pgfqpoint{3.674300in}{1.833230in}}%
\pgfpathlineto{\pgfqpoint{3.688441in}{1.824995in}}%
\pgfpathlineto{\pgfqpoint{3.702586in}{1.816786in}}%
\pgfpathlineto{\pgfqpoint{3.716735in}{1.808603in}}%
\pgfpathlineto{\pgfqpoint{3.708262in}{1.818284in}}%
\pgfpathlineto{\pgfqpoint{3.699773in}{1.828540in}}%
\pgfpathlineto{\pgfqpoint{3.691267in}{1.839381in}}%
\pgfpathlineto{\pgfqpoint{3.682744in}{1.850820in}}%
\pgfpathlineto{\pgfqpoint{3.668554in}{1.859422in}}%
\pgfpathlineto{\pgfqpoint{3.654369in}{1.868051in}}%
\pgfpathlineto{\pgfqpoint{3.640188in}{1.876705in}}%
\pgfpathlineto{\pgfqpoint{3.626010in}{1.885386in}}%
\pgfpathlineto{\pgfqpoint{3.634575in}{1.873521in}}%
\pgfpathlineto{\pgfqpoint{3.643122in}{1.862259in}}%
\pgfpathlineto{\pgfqpoint{3.651652in}{1.851587in}}%
\pgfpathlineto{\pgfqpoint{3.660164in}{1.841492in}}%
\pgfpathclose%
\pgfusepath{fill}%
\end{pgfscope}%
\begin{pgfscope}%
\pgfpathrectangle{\pgfqpoint{1.150000in}{0.150000in}}{\pgfqpoint{5.700000in}{5.700000in}}%
\pgfusepath{clip}%
\pgfsetbuttcap%
\pgfsetroundjoin%
\definecolor{currentfill}{rgb}{0.276022,0.044167,0.370164}%
\pgfsetfillcolor{currentfill}%
\pgfsetfillopacity{0.700000}%
\pgfsetlinewidth{0.000000pt}%
\definecolor{currentstroke}{rgb}{0.000000,0.000000,0.000000}%
\pgfsetstrokecolor{currentstroke}%
\pgfsetdash{}{0pt}%
\pgfpathmoveto{\pgfqpoint{4.733184in}{1.383851in}}%
\pgfpathlineto{\pgfqpoint{4.747540in}{1.379008in}}%
\pgfpathlineto{\pgfqpoint{4.761903in}{1.374188in}}%
\pgfpathlineto{\pgfqpoint{4.776272in}{1.369391in}}%
\pgfpathlineto{\pgfqpoint{4.790649in}{1.364617in}}%
\pgfpathlineto{\pgfqpoint{4.782762in}{1.359751in}}%
\pgfpathlineto{\pgfqpoint{4.774872in}{1.355191in}}%
\pgfpathlineto{\pgfqpoint{4.766978in}{1.350947in}}%
\pgfpathlineto{\pgfqpoint{4.759082in}{1.347028in}}%
\pgfpathlineto{\pgfqpoint{4.744691in}{1.352144in}}%
\pgfpathlineto{\pgfqpoint{4.730305in}{1.357283in}}%
\pgfpathlineto{\pgfqpoint{4.715927in}{1.362446in}}%
\pgfpathlineto{\pgfqpoint{4.701555in}{1.367631in}}%
\pgfpathlineto{\pgfqpoint{4.709468in}{1.371203in}}%
\pgfpathlineto{\pgfqpoint{4.717377in}{1.375103in}}%
\pgfpathlineto{\pgfqpoint{4.725282in}{1.379322in}}%
\pgfpathlineto{\pgfqpoint{4.733184in}{1.383851in}}%
\pgfpathclose%
\pgfusepath{fill}%
\end{pgfscope}%
\begin{pgfscope}%
\pgfpathrectangle{\pgfqpoint{1.150000in}{0.150000in}}{\pgfqpoint{5.700000in}{5.700000in}}%
\pgfusepath{clip}%
\pgfsetbuttcap%
\pgfsetroundjoin%
\definecolor{currentfill}{rgb}{0.149039,0.508051,0.557250}%
\pgfsetfillcolor{currentfill}%
\pgfsetfillopacity{0.700000}%
\pgfsetlinewidth{0.000000pt}%
\definecolor{currentstroke}{rgb}{0.000000,0.000000,0.000000}%
\pgfsetstrokecolor{currentstroke}%
\pgfsetdash{}{0pt}%
\pgfpathmoveto{\pgfqpoint{2.856549in}{2.452362in}}%
\pgfpathlineto{\pgfqpoint{2.870613in}{2.441707in}}%
\pgfpathlineto{\pgfqpoint{2.884680in}{2.431086in}}%
\pgfpathlineto{\pgfqpoint{2.898748in}{2.420499in}}%
\pgfpathlineto{\pgfqpoint{2.912819in}{2.409944in}}%
\pgfpathlineto{\pgfqpoint{2.903532in}{2.430439in}}%
\pgfpathlineto{\pgfqpoint{2.894211in}{2.451671in}}%
\pgfpathlineto{\pgfqpoint{2.884856in}{2.473653in}}%
\pgfpathlineto{\pgfqpoint{2.875468in}{2.496402in}}%
\pgfpathlineto{\pgfqpoint{2.861337in}{2.507421in}}%
\pgfpathlineto{\pgfqpoint{2.847209in}{2.518473in}}%
\pgfpathlineto{\pgfqpoint{2.833083in}{2.529560in}}%
\pgfpathlineto{\pgfqpoint{2.818958in}{2.540680in}}%
\pgfpathlineto{\pgfqpoint{2.828409in}{2.517459in}}%
\pgfpathlineto{\pgfqpoint{2.837823in}{2.495008in}}%
\pgfpathlineto{\pgfqpoint{2.847203in}{2.473314in}}%
\pgfpathlineto{\pgfqpoint{2.856549in}{2.452362in}}%
\pgfpathclose%
\pgfusepath{fill}%
\end{pgfscope}%
\begin{pgfscope}%
\pgfpathrectangle{\pgfqpoint{1.150000in}{0.150000in}}{\pgfqpoint{5.700000in}{5.700000in}}%
\pgfusepath{clip}%
\pgfsetbuttcap%
\pgfsetroundjoin%
\definecolor{currentfill}{rgb}{0.273809,0.031497,0.358853}%
\pgfsetfillcolor{currentfill}%
\pgfsetfillopacity{0.700000}%
\pgfsetlinewidth{0.000000pt}%
\definecolor{currentstroke}{rgb}{0.000000,0.000000,0.000000}%
\pgfsetstrokecolor{currentstroke}%
\pgfsetdash{}{0pt}%
\pgfpathmoveto{\pgfqpoint{4.879688in}{1.369427in}}%
\pgfpathlineto{\pgfqpoint{4.894085in}{1.365098in}}%
\pgfpathlineto{\pgfqpoint{4.908490in}{1.360793in}}%
\pgfpathlineto{\pgfqpoint{4.922902in}{1.356511in}}%
\pgfpathlineto{\pgfqpoint{4.937322in}{1.352253in}}%
\pgfpathlineto{\pgfqpoint{4.929471in}{1.345599in}}%
\pgfpathlineto{\pgfqpoint{4.921619in}{1.339212in}}%
\pgfpathlineto{\pgfqpoint{4.913764in}{1.333097in}}%
\pgfpathlineto{\pgfqpoint{4.905907in}{1.327265in}}%
\pgfpathlineto{\pgfqpoint{4.891476in}{1.331853in}}%
\pgfpathlineto{\pgfqpoint{4.877051in}{1.336464in}}%
\pgfpathlineto{\pgfqpoint{4.862633in}{1.341098in}}%
\pgfpathlineto{\pgfqpoint{4.848222in}{1.345755in}}%
\pgfpathlineto{\pgfqpoint{4.856093in}{1.351253in}}%
\pgfpathlineto{\pgfqpoint{4.863960in}{1.357037in}}%
\pgfpathlineto{\pgfqpoint{4.871825in}{1.363098in}}%
\pgfpathlineto{\pgfqpoint{4.879688in}{1.369427in}}%
\pgfpathclose%
\pgfusepath{fill}%
\end{pgfscope}%
\begin{pgfscope}%
\pgfpathrectangle{\pgfqpoint{1.150000in}{0.150000in}}{\pgfqpoint{5.700000in}{5.700000in}}%
\pgfusepath{clip}%
\pgfsetbuttcap%
\pgfsetroundjoin%
\definecolor{currentfill}{rgb}{0.252899,0.742211,0.448284}%
\pgfsetfillcolor{currentfill}%
\pgfsetfillopacity{0.700000}%
\pgfsetlinewidth{0.000000pt}%
\definecolor{currentstroke}{rgb}{0.000000,0.000000,0.000000}%
\pgfsetstrokecolor{currentstroke}%
\pgfsetdash{}{0pt}%
\pgfpathmoveto{\pgfqpoint{2.142602in}{3.119586in}}%
\pgfpathlineto{\pgfqpoint{2.156677in}{3.106497in}}%
\pgfpathlineto{\pgfqpoint{2.170752in}{3.093458in}}%
\pgfpathlineto{\pgfqpoint{2.184827in}{3.080469in}}%
\pgfpathlineto{\pgfqpoint{2.198902in}{3.067529in}}%
\pgfpathlineto{\pgfqpoint{2.188662in}{3.096881in}}%
\pgfpathlineto{\pgfqpoint{2.178372in}{3.127093in}}%
\pgfpathlineto{\pgfqpoint{2.168031in}{3.158181in}}%
\pgfpathlineto{\pgfqpoint{2.157636in}{3.190162in}}%
\pgfpathlineto{\pgfqpoint{2.143485in}{3.203606in}}%
\pgfpathlineto{\pgfqpoint{2.129334in}{3.217099in}}%
\pgfpathlineto{\pgfqpoint{2.115182in}{3.230643in}}%
\pgfpathlineto{\pgfqpoint{2.101030in}{3.244237in}}%
\pgfpathlineto{\pgfqpoint{2.111503in}{3.211743in}}%
\pgfpathlineto{\pgfqpoint{2.121922in}{3.180148in}}%
\pgfpathlineto{\pgfqpoint{2.132288in}{3.149434in}}%
\pgfpathlineto{\pgfqpoint{2.142602in}{3.119586in}}%
\pgfpathclose%
\pgfusepath{fill}%
\end{pgfscope}%
\begin{pgfscope}%
\pgfpathrectangle{\pgfqpoint{1.150000in}{0.150000in}}{\pgfqpoint{5.700000in}{5.700000in}}%
\pgfusepath{clip}%
\pgfsetbuttcap%
\pgfsetroundjoin%
\definecolor{currentfill}{rgb}{0.278791,0.062145,0.386592}%
\pgfsetfillcolor{currentfill}%
\pgfsetfillopacity{0.700000}%
\pgfsetlinewidth{0.000000pt}%
\definecolor{currentstroke}{rgb}{0.000000,0.000000,0.000000}%
\pgfsetstrokecolor{currentstroke}%
\pgfsetdash{}{0pt}%
\pgfpathmoveto{\pgfqpoint{4.586814in}{1.409959in}}%
\pgfpathlineto{\pgfqpoint{4.601134in}{1.404586in}}%
\pgfpathlineto{\pgfqpoint{4.615461in}{1.399237in}}%
\pgfpathlineto{\pgfqpoint{4.629794in}{1.393911in}}%
\pgfpathlineto{\pgfqpoint{4.644133in}{1.388608in}}%
\pgfpathlineto{\pgfqpoint{4.636199in}{1.385725in}}%
\pgfpathlineto{\pgfqpoint{4.628262in}{1.383191in}}%
\pgfpathlineto{\pgfqpoint{4.620320in}{1.381018in}}%
\pgfpathlineto{\pgfqpoint{4.612374in}{1.379214in}}%
\pgfpathlineto{\pgfqpoint{4.598016in}{1.384873in}}%
\pgfpathlineto{\pgfqpoint{4.583664in}{1.390555in}}%
\pgfpathlineto{\pgfqpoint{4.569318in}{1.396260in}}%
\pgfpathlineto{\pgfqpoint{4.554979in}{1.401989in}}%
\pgfpathlineto{\pgfqpoint{4.562945in}{1.403432in}}%
\pgfpathlineto{\pgfqpoint{4.570906in}{1.405247in}}%
\pgfpathlineto{\pgfqpoint{4.578862in}{1.407426in}}%
\pgfpathlineto{\pgfqpoint{4.586814in}{1.409959in}}%
\pgfpathclose%
\pgfusepath{fill}%
\end{pgfscope}%
\begin{pgfscope}%
\pgfpathrectangle{\pgfqpoint{1.150000in}{0.150000in}}{\pgfqpoint{5.700000in}{5.700000in}}%
\pgfusepath{clip}%
\pgfsetbuttcap%
\pgfsetroundjoin%
\definecolor{currentfill}{rgb}{0.273809,0.031497,0.358853}%
\pgfsetfillcolor{currentfill}%
\pgfsetfillopacity{0.700000}%
\pgfsetlinewidth{0.000000pt}%
\definecolor{currentstroke}{rgb}{0.000000,0.000000,0.000000}%
\pgfsetstrokecolor{currentstroke}%
\pgfsetdash{}{0pt}%
\pgfpathmoveto{\pgfqpoint{5.026407in}{1.365810in}}%
\pgfpathlineto{\pgfqpoint{5.040853in}{1.361984in}}%
\pgfpathlineto{\pgfqpoint{5.055306in}{1.358180in}}%
\pgfpathlineto{\pgfqpoint{5.069766in}{1.354400in}}%
\pgfpathlineto{\pgfqpoint{5.084234in}{1.350643in}}%
\pgfpathlineto{\pgfqpoint{5.076413in}{1.342391in}}%
\pgfpathlineto{\pgfqpoint{5.068589in}{1.334364in}}%
\pgfpathlineto{\pgfqpoint{5.060764in}{1.326571in}}%
\pgfpathlineto{\pgfqpoint{5.052938in}{1.319020in}}%
\pgfpathlineto{\pgfqpoint{5.038460in}{1.323093in}}%
\pgfpathlineto{\pgfqpoint{5.023990in}{1.327189in}}%
\pgfpathlineto{\pgfqpoint{5.009527in}{1.331308in}}%
\pgfpathlineto{\pgfqpoint{4.995071in}{1.335450in}}%
\pgfpathlineto{\pgfqpoint{5.002908in}{1.342681in}}%
\pgfpathlineto{\pgfqpoint{5.010743in}{1.350156in}}%
\pgfpathlineto{\pgfqpoint{5.018576in}{1.357869in}}%
\pgfpathlineto{\pgfqpoint{5.026407in}{1.365810in}}%
\pgfpathclose%
\pgfusepath{fill}%
\end{pgfscope}%
\begin{pgfscope}%
\pgfpathrectangle{\pgfqpoint{1.150000in}{0.150000in}}{\pgfqpoint{5.700000in}{5.700000in}}%
\pgfusepath{clip}%
\pgfsetbuttcap%
\pgfsetroundjoin%
\definecolor{currentfill}{rgb}{0.208623,0.367752,0.552675}%
\pgfsetfillcolor{currentfill}%
\pgfsetfillopacity{0.700000}%
\pgfsetlinewidth{0.000000pt}%
\definecolor{currentstroke}{rgb}{0.000000,0.000000,0.000000}%
\pgfsetstrokecolor{currentstroke}%
\pgfsetdash{}{0pt}%
\pgfpathmoveto{\pgfqpoint{3.343260in}{2.064671in}}%
\pgfpathlineto{\pgfqpoint{3.357363in}{2.055444in}}%
\pgfpathlineto{\pgfqpoint{3.371470in}{2.046246in}}%
\pgfpathlineto{\pgfqpoint{3.385580in}{2.037075in}}%
\pgfpathlineto{\pgfqpoint{3.399694in}{2.027933in}}%
\pgfpathlineto{\pgfqpoint{3.390931in}{2.042138in}}%
\pgfpathlineto{\pgfqpoint{3.382146in}{2.056988in}}%
\pgfpathlineto{\pgfqpoint{3.373337in}{2.072497in}}%
\pgfpathlineto{\pgfqpoint{3.364505in}{2.088677in}}%
\pgfpathlineto{\pgfqpoint{3.350344in}{2.098258in}}%
\pgfpathlineto{\pgfqpoint{3.336185in}{2.107868in}}%
\pgfpathlineto{\pgfqpoint{3.322030in}{2.117506in}}%
\pgfpathlineto{\pgfqpoint{3.307878in}{2.127173in}}%
\pgfpathlineto{\pgfqpoint{3.316760in}{2.110547in}}%
\pgfpathlineto{\pgfqpoint{3.325617in}{2.094596in}}%
\pgfpathlineto{\pgfqpoint{3.334450in}{2.079309in}}%
\pgfpathlineto{\pgfqpoint{3.343260in}{2.064671in}}%
\pgfpathclose%
\pgfusepath{fill}%
\end{pgfscope}%
\begin{pgfscope}%
\pgfpathrectangle{\pgfqpoint{1.150000in}{0.150000in}}{\pgfqpoint{5.700000in}{5.700000in}}%
\pgfusepath{clip}%
\pgfsetbuttcap%
\pgfsetroundjoin%
\definecolor{currentfill}{rgb}{0.220124,0.725509,0.466226}%
\pgfsetfillcolor{currentfill}%
\pgfsetfillopacity{0.700000}%
\pgfsetlinewidth{0.000000pt}%
\definecolor{currentstroke}{rgb}{0.000000,0.000000,0.000000}%
\pgfsetstrokecolor{currentstroke}%
\pgfsetdash{}{0pt}%
\pgfpathmoveto{\pgfqpoint{2.198902in}{3.067529in}}%
\pgfpathlineto{\pgfqpoint{2.212977in}{3.054637in}}%
\pgfpathlineto{\pgfqpoint{2.227052in}{3.041794in}}%
\pgfpathlineto{\pgfqpoint{2.241128in}{3.028998in}}%
\pgfpathlineto{\pgfqpoint{2.255203in}{3.016249in}}%
\pgfpathlineto{\pgfqpoint{2.245037in}{3.045108in}}%
\pgfpathlineto{\pgfqpoint{2.234822in}{3.074820in}}%
\pgfpathlineto{\pgfqpoint{2.224556in}{3.105403in}}%
\pgfpathlineto{\pgfqpoint{2.214240in}{3.136872in}}%
\pgfpathlineto{\pgfqpoint{2.200089in}{3.150123in}}%
\pgfpathlineto{\pgfqpoint{2.185938in}{3.163421in}}%
\pgfpathlineto{\pgfqpoint{2.171787in}{3.176767in}}%
\pgfpathlineto{\pgfqpoint{2.157636in}{3.190162in}}%
\pgfpathlineto{\pgfqpoint{2.168031in}{3.158181in}}%
\pgfpathlineto{\pgfqpoint{2.178372in}{3.127093in}}%
\pgfpathlineto{\pgfqpoint{2.188662in}{3.096881in}}%
\pgfpathlineto{\pgfqpoint{2.198902in}{3.067529in}}%
\pgfpathclose%
\pgfusepath{fill}%
\end{pgfscope}%
\begin{pgfscope}%
\pgfpathrectangle{\pgfqpoint{1.150000in}{0.150000in}}{\pgfqpoint{5.700000in}{5.700000in}}%
\pgfusepath{clip}%
\pgfsetbuttcap%
\pgfsetroundjoin%
\definecolor{currentfill}{rgb}{0.276194,0.190074,0.493001}%
\pgfsetfillcolor{currentfill}%
\pgfsetfillopacity{0.700000}%
\pgfsetlinewidth{0.000000pt}%
\definecolor{currentstroke}{rgb}{0.000000,0.000000,0.000000}%
\pgfsetstrokecolor{currentstroke}%
\pgfsetdash{}{0pt}%
\pgfpathmoveto{\pgfqpoint{3.976890in}{1.654362in}}%
\pgfpathlineto{\pgfqpoint{3.991082in}{1.647013in}}%
\pgfpathlineto{\pgfqpoint{4.005279in}{1.639688in}}%
\pgfpathlineto{\pgfqpoint{4.019481in}{1.632388in}}%
\pgfpathlineto{\pgfqpoint{4.033687in}{1.625112in}}%
\pgfpathlineto{\pgfqpoint{4.025446in}{1.630637in}}%
\pgfpathlineto{\pgfqpoint{4.017192in}{1.636669in}}%
\pgfpathlineto{\pgfqpoint{4.008928in}{1.643219in}}%
\pgfpathlineto{\pgfqpoint{4.000651in}{1.650299in}}%
\pgfpathlineto{\pgfqpoint{3.986411in}{1.657976in}}%
\pgfpathlineto{\pgfqpoint{3.972176in}{1.665678in}}%
\pgfpathlineto{\pgfqpoint{3.957946in}{1.673404in}}%
\pgfpathlineto{\pgfqpoint{3.943721in}{1.681155in}}%
\pgfpathlineto{\pgfqpoint{3.952032in}{1.673668in}}%
\pgfpathlineto{\pgfqpoint{3.960330in}{1.666714in}}%
\pgfpathlineto{\pgfqpoint{3.968617in}{1.660283in}}%
\pgfpathlineto{\pgfqpoint{3.976890in}{1.654362in}}%
\pgfpathclose%
\pgfusepath{fill}%
\end{pgfscope}%
\begin{pgfscope}%
\pgfpathrectangle{\pgfqpoint{1.150000in}{0.150000in}}{\pgfqpoint{5.700000in}{5.700000in}}%
\pgfusepath{clip}%
\pgfsetbuttcap%
\pgfsetroundjoin%
\definecolor{currentfill}{rgb}{0.153364,0.497000,0.557724}%
\pgfsetfillcolor{currentfill}%
\pgfsetfillopacity{0.700000}%
\pgfsetlinewidth{0.000000pt}%
\definecolor{currentstroke}{rgb}{0.000000,0.000000,0.000000}%
\pgfsetstrokecolor{currentstroke}%
\pgfsetdash{}{0pt}%
\pgfpathmoveto{\pgfqpoint{2.912819in}{2.409944in}}%
\pgfpathlineto{\pgfqpoint{2.926893in}{2.399423in}}%
\pgfpathlineto{\pgfqpoint{2.940969in}{2.388935in}}%
\pgfpathlineto{\pgfqpoint{2.955047in}{2.378479in}}%
\pgfpathlineto{\pgfqpoint{2.969127in}{2.368055in}}%
\pgfpathlineto{\pgfqpoint{2.959897in}{2.388093in}}%
\pgfpathlineto{\pgfqpoint{2.950634in}{2.408863in}}%
\pgfpathlineto{\pgfqpoint{2.941339in}{2.430379in}}%
\pgfpathlineto{\pgfqpoint{2.932011in}{2.452656in}}%
\pgfpathlineto{\pgfqpoint{2.917872in}{2.463543in}}%
\pgfpathlineto{\pgfqpoint{2.903735in}{2.474463in}}%
\pgfpathlineto{\pgfqpoint{2.889600in}{2.485416in}}%
\pgfpathlineto{\pgfqpoint{2.875468in}{2.496402in}}%
\pgfpathlineto{\pgfqpoint{2.884856in}{2.473653in}}%
\pgfpathlineto{\pgfqpoint{2.894211in}{2.451671in}}%
\pgfpathlineto{\pgfqpoint{2.903532in}{2.430439in}}%
\pgfpathlineto{\pgfqpoint{2.912819in}{2.409944in}}%
\pgfpathclose%
\pgfusepath{fill}%
\end{pgfscope}%
\begin{pgfscope}%
\pgfpathrectangle{\pgfqpoint{1.150000in}{0.150000in}}{\pgfqpoint{5.700000in}{5.700000in}}%
\pgfusepath{clip}%
\pgfsetbuttcap%
\pgfsetroundjoin%
\definecolor{currentfill}{rgb}{0.253935,0.265254,0.529983}%
\pgfsetfillcolor{currentfill}%
\pgfsetfillopacity{0.700000}%
\pgfsetlinewidth{0.000000pt}%
\definecolor{currentstroke}{rgb}{0.000000,0.000000,0.000000}%
\pgfsetstrokecolor{currentstroke}%
\pgfsetdash{}{0pt}%
\pgfpathmoveto{\pgfqpoint{3.716735in}{1.808603in}}%
\pgfpathlineto{\pgfqpoint{3.730888in}{1.800446in}}%
\pgfpathlineto{\pgfqpoint{3.745046in}{1.792314in}}%
\pgfpathlineto{\pgfqpoint{3.759208in}{1.784209in}}%
\pgfpathlineto{\pgfqpoint{3.773375in}{1.776129in}}%
\pgfpathlineto{\pgfqpoint{3.764940in}{1.785398in}}%
\pgfpathlineto{\pgfqpoint{3.756490in}{1.795236in}}%
\pgfpathlineto{\pgfqpoint{3.748024in}{1.805656in}}%
\pgfpathlineto{\pgfqpoint{3.739542in}{1.816670in}}%
\pgfpathlineto{\pgfqpoint{3.725336in}{1.825169in}}%
\pgfpathlineto{\pgfqpoint{3.711135in}{1.833693in}}%
\pgfpathlineto{\pgfqpoint{3.696937in}{1.842243in}}%
\pgfpathlineto{\pgfqpoint{3.682744in}{1.850820in}}%
\pgfpathlineto{\pgfqpoint{3.691267in}{1.839381in}}%
\pgfpathlineto{\pgfqpoint{3.699773in}{1.828540in}}%
\pgfpathlineto{\pgfqpoint{3.708262in}{1.818284in}}%
\pgfpathlineto{\pgfqpoint{3.716735in}{1.808603in}}%
\pgfpathclose%
\pgfusepath{fill}%
\end{pgfscope}%
\begin{pgfscope}%
\pgfpathrectangle{\pgfqpoint{1.150000in}{0.150000in}}{\pgfqpoint{5.700000in}{5.700000in}}%
\pgfusepath{clip}%
\pgfsetbuttcap%
\pgfsetroundjoin%
\definecolor{currentfill}{rgb}{0.196571,0.711827,0.479221}%
\pgfsetfillcolor{currentfill}%
\pgfsetfillopacity{0.700000}%
\pgfsetlinewidth{0.000000pt}%
\definecolor{currentstroke}{rgb}{0.000000,0.000000,0.000000}%
\pgfsetstrokecolor{currentstroke}%
\pgfsetdash{}{0pt}%
\pgfpathmoveto{\pgfqpoint{2.255203in}{3.016249in}}%
\pgfpathlineto{\pgfqpoint{2.269279in}{3.003548in}}%
\pgfpathlineto{\pgfqpoint{2.283355in}{2.990892in}}%
\pgfpathlineto{\pgfqpoint{2.297432in}{2.978283in}}%
\pgfpathlineto{\pgfqpoint{2.311509in}{2.965719in}}%
\pgfpathlineto{\pgfqpoint{2.301415in}{2.994085in}}%
\pgfpathlineto{\pgfqpoint{2.291274in}{3.023299in}}%
\pgfpathlineto{\pgfqpoint{2.281084in}{3.053379in}}%
\pgfpathlineto{\pgfqpoint{2.270843in}{3.084339in}}%
\pgfpathlineto{\pgfqpoint{2.256692in}{3.097403in}}%
\pgfpathlineto{\pgfqpoint{2.242541in}{3.110513in}}%
\pgfpathlineto{\pgfqpoint{2.228390in}{3.123669in}}%
\pgfpathlineto{\pgfqpoint{2.214240in}{3.136872in}}%
\pgfpathlineto{\pgfqpoint{2.224556in}{3.105403in}}%
\pgfpathlineto{\pgfqpoint{2.234822in}{3.074820in}}%
\pgfpathlineto{\pgfqpoint{2.245037in}{3.045108in}}%
\pgfpathlineto{\pgfqpoint{2.255203in}{3.016249in}}%
\pgfpathclose%
\pgfusepath{fill}%
\end{pgfscope}%
\begin{pgfscope}%
\pgfpathrectangle{\pgfqpoint{1.150000in}{0.150000in}}{\pgfqpoint{5.700000in}{5.700000in}}%
\pgfusepath{clip}%
\pgfsetbuttcap%
\pgfsetroundjoin%
\definecolor{currentfill}{rgb}{0.283072,0.130895,0.449241}%
\pgfsetfillcolor{currentfill}%
\pgfsetfillopacity{0.700000}%
\pgfsetlinewidth{0.000000pt}%
\definecolor{currentstroke}{rgb}{0.000000,0.000000,0.000000}%
\pgfsetstrokecolor{currentstroke}%
\pgfsetdash{}{0pt}%
\pgfpathmoveto{\pgfqpoint{4.237075in}{1.527089in}}%
\pgfpathlineto{\pgfqpoint{4.251321in}{1.520515in}}%
\pgfpathlineto{\pgfqpoint{4.265572in}{1.513966in}}%
\pgfpathlineto{\pgfqpoint{4.279829in}{1.507440in}}%
\pgfpathlineto{\pgfqpoint{4.294091in}{1.500939in}}%
\pgfpathlineto{\pgfqpoint{4.286003in}{1.503027in}}%
\pgfpathlineto{\pgfqpoint{4.277906in}{1.505563in}}%
\pgfpathlineto{\pgfqpoint{4.269801in}{1.508558in}}%
\pgfpathlineto{\pgfqpoint{4.261689in}{1.512023in}}%
\pgfpathlineto{\pgfqpoint{4.247399in}{1.518910in}}%
\pgfpathlineto{\pgfqpoint{4.233115in}{1.525820in}}%
\pgfpathlineto{\pgfqpoint{4.218836in}{1.532755in}}%
\pgfpathlineto{\pgfqpoint{4.204563in}{1.539714in}}%
\pgfpathlineto{\pgfqpoint{4.212704in}{1.535859in}}%
\pgfpathlineto{\pgfqpoint{4.220836in}{1.532477in}}%
\pgfpathlineto{\pgfqpoint{4.228960in}{1.529557in}}%
\pgfpathlineto{\pgfqpoint{4.237075in}{1.527089in}}%
\pgfpathclose%
\pgfusepath{fill}%
\end{pgfscope}%
\begin{pgfscope}%
\pgfpathrectangle{\pgfqpoint{1.150000in}{0.150000in}}{\pgfqpoint{5.700000in}{5.700000in}}%
\pgfusepath{clip}%
\pgfsetbuttcap%
\pgfsetroundjoin%
\definecolor{currentfill}{rgb}{0.281924,0.089666,0.412415}%
\pgfsetfillcolor{currentfill}%
\pgfsetfillopacity{0.700000}%
\pgfsetlinewidth{0.000000pt}%
\definecolor{currentstroke}{rgb}{0.000000,0.000000,0.000000}%
\pgfsetstrokecolor{currentstroke}%
\pgfsetdash{}{0pt}%
\pgfpathmoveto{\pgfqpoint{4.440484in}{1.448667in}}%
\pgfpathlineto{\pgfqpoint{4.454775in}{1.442749in}}%
\pgfpathlineto{\pgfqpoint{4.469072in}{1.436856in}}%
\pgfpathlineto{\pgfqpoint{4.483374in}{1.430986in}}%
\pgfpathlineto{\pgfqpoint{4.497683in}{1.425139in}}%
\pgfpathlineto{\pgfqpoint{4.489691in}{1.424444in}}%
\pgfpathlineto{\pgfqpoint{4.481694in}{1.424145in}}%
\pgfpathlineto{\pgfqpoint{4.473691in}{1.424252in}}%
\pgfpathlineto{\pgfqpoint{4.465683in}{1.424775in}}%
\pgfpathlineto{\pgfqpoint{4.451352in}{1.430992in}}%
\pgfpathlineto{\pgfqpoint{4.437027in}{1.437232in}}%
\pgfpathlineto{\pgfqpoint{4.422707in}{1.443496in}}%
\pgfpathlineto{\pgfqpoint{4.408394in}{1.449784in}}%
\pgfpathlineto{\pgfqpoint{4.416426in}{1.448885in}}%
\pgfpathlineto{\pgfqpoint{4.424451in}{1.448406in}}%
\pgfpathlineto{\pgfqpoint{4.432471in}{1.448336in}}%
\pgfpathlineto{\pgfqpoint{4.440484in}{1.448667in}}%
\pgfpathclose%
\pgfusepath{fill}%
\end{pgfscope}%
\begin{pgfscope}%
\pgfpathrectangle{\pgfqpoint{1.150000in}{0.150000in}}{\pgfqpoint{5.700000in}{5.700000in}}%
\pgfusepath{clip}%
\pgfsetbuttcap%
\pgfsetroundjoin%
\definecolor{currentfill}{rgb}{0.214298,0.355619,0.551184}%
\pgfsetfillcolor{currentfill}%
\pgfsetfillopacity{0.700000}%
\pgfsetlinewidth{0.000000pt}%
\definecolor{currentstroke}{rgb}{0.000000,0.000000,0.000000}%
\pgfsetstrokecolor{currentstroke}%
\pgfsetdash{}{0pt}%
\pgfpathmoveto{\pgfqpoint{3.399694in}{2.027933in}}%
\pgfpathlineto{\pgfqpoint{3.413811in}{2.018819in}}%
\pgfpathlineto{\pgfqpoint{3.427931in}{2.009732in}}%
\pgfpathlineto{\pgfqpoint{3.442056in}{2.000674in}}%
\pgfpathlineto{\pgfqpoint{3.456184in}{1.991642in}}%
\pgfpathlineto{\pgfqpoint{3.447467in}{2.005415in}}%
\pgfpathlineto{\pgfqpoint{3.438729in}{2.019829in}}%
\pgfpathlineto{\pgfqpoint{3.429968in}{2.034896in}}%
\pgfpathlineto{\pgfqpoint{3.421185in}{2.050631in}}%
\pgfpathlineto{\pgfqpoint{3.407010in}{2.060101in}}%
\pgfpathlineto{\pgfqpoint{3.392838in}{2.069598in}}%
\pgfpathlineto{\pgfqpoint{3.378670in}{2.079124in}}%
\pgfpathlineto{\pgfqpoint{3.364505in}{2.088677in}}%
\pgfpathlineto{\pgfqpoint{3.373337in}{2.072497in}}%
\pgfpathlineto{\pgfqpoint{3.382146in}{2.056988in}}%
\pgfpathlineto{\pgfqpoint{3.390931in}{2.042138in}}%
\pgfpathlineto{\pgfqpoint{3.399694in}{2.027933in}}%
\pgfpathclose%
\pgfusepath{fill}%
\end{pgfscope}%
\begin{pgfscope}%
\pgfpathrectangle{\pgfqpoint{1.150000in}{0.150000in}}{\pgfqpoint{5.700000in}{5.700000in}}%
\pgfusepath{clip}%
\pgfsetbuttcap%
\pgfsetroundjoin%
\definecolor{currentfill}{rgb}{0.157729,0.485932,0.558013}%
\pgfsetfillcolor{currentfill}%
\pgfsetfillopacity{0.700000}%
\pgfsetlinewidth{0.000000pt}%
\definecolor{currentstroke}{rgb}{0.000000,0.000000,0.000000}%
\pgfsetstrokecolor{currentstroke}%
\pgfsetdash{}{0pt}%
\pgfpathmoveto{\pgfqpoint{2.969127in}{2.368055in}}%
\pgfpathlineto{\pgfqpoint{2.983210in}{2.357664in}}%
\pgfpathlineto{\pgfqpoint{2.997296in}{2.347305in}}%
\pgfpathlineto{\pgfqpoint{3.011384in}{2.336978in}}%
\pgfpathlineto{\pgfqpoint{3.025475in}{2.326682in}}%
\pgfpathlineto{\pgfqpoint{3.016301in}{2.346264in}}%
\pgfpathlineto{\pgfqpoint{3.007095in}{2.366573in}}%
\pgfpathlineto{\pgfqpoint{2.997859in}{2.387624in}}%
\pgfpathlineto{\pgfqpoint{2.988590in}{2.409430in}}%
\pgfpathlineto{\pgfqpoint{2.974442in}{2.420188in}}%
\pgfpathlineto{\pgfqpoint{2.960296in}{2.430979in}}%
\pgfpathlineto{\pgfqpoint{2.946152in}{2.441801in}}%
\pgfpathlineto{\pgfqpoint{2.932011in}{2.452656in}}%
\pgfpathlineto{\pgfqpoint{2.941339in}{2.430379in}}%
\pgfpathlineto{\pgfqpoint{2.950634in}{2.408863in}}%
\pgfpathlineto{\pgfqpoint{2.959897in}{2.388093in}}%
\pgfpathlineto{\pgfqpoint{2.969127in}{2.368055in}}%
\pgfpathclose%
\pgfusepath{fill}%
\end{pgfscope}%
\begin{pgfscope}%
\pgfpathrectangle{\pgfqpoint{1.150000in}{0.150000in}}{\pgfqpoint{5.700000in}{5.700000in}}%
\pgfusepath{clip}%
\pgfsetbuttcap%
\pgfsetroundjoin%
\definecolor{currentfill}{rgb}{0.170948,0.694384,0.493803}%
\pgfsetfillcolor{currentfill}%
\pgfsetfillopacity{0.700000}%
\pgfsetlinewidth{0.000000pt}%
\definecolor{currentstroke}{rgb}{0.000000,0.000000,0.000000}%
\pgfsetstrokecolor{currentstroke}%
\pgfsetdash{}{0pt}%
\pgfpathmoveto{\pgfqpoint{2.311509in}{2.965719in}}%
\pgfpathlineto{\pgfqpoint{2.325586in}{2.953200in}}%
\pgfpathlineto{\pgfqpoint{2.339664in}{2.940725in}}%
\pgfpathlineto{\pgfqpoint{2.353743in}{2.928295in}}%
\pgfpathlineto{\pgfqpoint{2.367822in}{2.915909in}}%
\pgfpathlineto{\pgfqpoint{2.357800in}{2.943785in}}%
\pgfpathlineto{\pgfqpoint{2.347732in}{2.972503in}}%
\pgfpathlineto{\pgfqpoint{2.337615in}{3.002081in}}%
\pgfpathlineto{\pgfqpoint{2.327450in}{3.032534in}}%
\pgfpathlineto{\pgfqpoint{2.313298in}{3.045419in}}%
\pgfpathlineto{\pgfqpoint{2.299146in}{3.058347in}}%
\pgfpathlineto{\pgfqpoint{2.284994in}{3.071321in}}%
\pgfpathlineto{\pgfqpoint{2.270843in}{3.084339in}}%
\pgfpathlineto{\pgfqpoint{2.281084in}{3.053379in}}%
\pgfpathlineto{\pgfqpoint{2.291274in}{3.023299in}}%
\pgfpathlineto{\pgfqpoint{2.301415in}{2.994085in}}%
\pgfpathlineto{\pgfqpoint{2.311509in}{2.965719in}}%
\pgfpathclose%
\pgfusepath{fill}%
\end{pgfscope}%
\begin{pgfscope}%
\pgfpathrectangle{\pgfqpoint{1.150000in}{0.150000in}}{\pgfqpoint{5.700000in}{5.700000in}}%
\pgfusepath{clip}%
\pgfsetbuttcap%
\pgfsetroundjoin%
\definecolor{currentfill}{rgb}{0.276022,0.044167,0.370164}%
\pgfsetfillcolor{currentfill}%
\pgfsetfillopacity{0.700000}%
\pgfsetlinewidth{0.000000pt}%
\definecolor{currentstroke}{rgb}{0.000000,0.000000,0.000000}%
\pgfsetstrokecolor{currentstroke}%
\pgfsetdash{}{0pt}%
\pgfpathmoveto{\pgfqpoint{4.790649in}{1.364617in}}%
\pgfpathlineto{\pgfqpoint{4.805032in}{1.359867in}}%
\pgfpathlineto{\pgfqpoint{4.819422in}{1.355140in}}%
\pgfpathlineto{\pgfqpoint{4.833819in}{1.350436in}}%
\pgfpathlineto{\pgfqpoint{4.848222in}{1.345755in}}%
\pgfpathlineto{\pgfqpoint{4.840350in}{1.340552in}}%
\pgfpathlineto{\pgfqpoint{4.832474in}{1.335651in}}%
\pgfpathlineto{\pgfqpoint{4.824597in}{1.331063in}}%
\pgfpathlineto{\pgfqpoint{4.816716in}{1.326796in}}%
\pgfpathlineto{\pgfqpoint{4.802297in}{1.331819in}}%
\pgfpathlineto{\pgfqpoint{4.787886in}{1.336866in}}%
\pgfpathlineto{\pgfqpoint{4.773481in}{1.341935in}}%
\pgfpathlineto{\pgfqpoint{4.759082in}{1.347028in}}%
\pgfpathlineto{\pgfqpoint{4.766978in}{1.350947in}}%
\pgfpathlineto{\pgfqpoint{4.774872in}{1.355191in}}%
\pgfpathlineto{\pgfqpoint{4.782762in}{1.359751in}}%
\pgfpathlineto{\pgfqpoint{4.790649in}{1.364617in}}%
\pgfpathclose%
\pgfusepath{fill}%
\end{pgfscope}%
\begin{pgfscope}%
\pgfpathrectangle{\pgfqpoint{1.150000in}{0.150000in}}{\pgfqpoint{5.700000in}{5.700000in}}%
\pgfusepath{clip}%
\pgfsetbuttcap%
\pgfsetroundjoin%
\definecolor{currentfill}{rgb}{0.274952,0.037752,0.364543}%
\pgfsetfillcolor{currentfill}%
\pgfsetfillopacity{0.700000}%
\pgfsetlinewidth{0.000000pt}%
\definecolor{currentstroke}{rgb}{0.000000,0.000000,0.000000}%
\pgfsetstrokecolor{currentstroke}%
\pgfsetdash{}{0pt}%
\pgfpathmoveto{\pgfqpoint{4.937322in}{1.352253in}}%
\pgfpathlineto{\pgfqpoint{4.951748in}{1.348017in}}%
\pgfpathlineto{\pgfqpoint{4.966182in}{1.343805in}}%
\pgfpathlineto{\pgfqpoint{4.980623in}{1.339616in}}%
\pgfpathlineto{\pgfqpoint{4.995071in}{1.335450in}}%
\pgfpathlineto{\pgfqpoint{4.987232in}{1.328473in}}%
\pgfpathlineto{\pgfqpoint{4.979392in}{1.321758in}}%
\pgfpathlineto{\pgfqpoint{4.971549in}{1.315313in}}%
\pgfpathlineto{\pgfqpoint{4.963705in}{1.309147in}}%
\pgfpathlineto{\pgfqpoint{4.949245in}{1.313642in}}%
\pgfpathlineto{\pgfqpoint{4.934792in}{1.318160in}}%
\pgfpathlineto{\pgfqpoint{4.920346in}{1.322701in}}%
\pgfpathlineto{\pgfqpoint{4.905907in}{1.327265in}}%
\pgfpathlineto{\pgfqpoint{4.913764in}{1.333097in}}%
\pgfpathlineto{\pgfqpoint{4.921619in}{1.339212in}}%
\pgfpathlineto{\pgfqpoint{4.929471in}{1.345599in}}%
\pgfpathlineto{\pgfqpoint{4.937322in}{1.352253in}}%
\pgfpathclose%
\pgfusepath{fill}%
\end{pgfscope}%
\begin{pgfscope}%
\pgfpathrectangle{\pgfqpoint{1.150000in}{0.150000in}}{\pgfqpoint{5.700000in}{5.700000in}}%
\pgfusepath{clip}%
\pgfsetbuttcap%
\pgfsetroundjoin%
\definecolor{currentfill}{rgb}{0.278012,0.180367,0.486697}%
\pgfsetfillcolor{currentfill}%
\pgfsetfillopacity{0.700000}%
\pgfsetlinewidth{0.000000pt}%
\definecolor{currentstroke}{rgb}{0.000000,0.000000,0.000000}%
\pgfsetstrokecolor{currentstroke}%
\pgfsetdash{}{0pt}%
\pgfpathmoveto{\pgfqpoint{4.033687in}{1.625112in}}%
\pgfpathlineto{\pgfqpoint{4.047899in}{1.617861in}}%
\pgfpathlineto{\pgfqpoint{4.062116in}{1.610635in}}%
\pgfpathlineto{\pgfqpoint{4.076337in}{1.603433in}}%
\pgfpathlineto{\pgfqpoint{4.090564in}{1.596256in}}%
\pgfpathlineto{\pgfqpoint{4.082354in}{1.601386in}}%
\pgfpathlineto{\pgfqpoint{4.074132in}{1.607019in}}%
\pgfpathlineto{\pgfqpoint{4.065900in}{1.613166in}}%
\pgfpathlineto{\pgfqpoint{4.057657in}{1.619838in}}%
\pgfpathlineto{\pgfqpoint{4.043398in}{1.627417in}}%
\pgfpathlineto{\pgfqpoint{4.029144in}{1.635020in}}%
\pgfpathlineto{\pgfqpoint{4.014895in}{1.642647in}}%
\pgfpathlineto{\pgfqpoint{4.000651in}{1.650299in}}%
\pgfpathlineto{\pgfqpoint{4.008928in}{1.643219in}}%
\pgfpathlineto{\pgfqpoint{4.017192in}{1.636669in}}%
\pgfpathlineto{\pgfqpoint{4.025446in}{1.630637in}}%
\pgfpathlineto{\pgfqpoint{4.033687in}{1.625112in}}%
\pgfpathclose%
\pgfusepath{fill}%
\end{pgfscope}%
\begin{pgfscope}%
\pgfpathrectangle{\pgfqpoint{1.150000in}{0.150000in}}{\pgfqpoint{5.700000in}{5.700000in}}%
\pgfusepath{clip}%
\pgfsetbuttcap%
\pgfsetroundjoin%
\definecolor{currentfill}{rgb}{0.278791,0.062145,0.386592}%
\pgfsetfillcolor{currentfill}%
\pgfsetfillopacity{0.700000}%
\pgfsetlinewidth{0.000000pt}%
\definecolor{currentstroke}{rgb}{0.000000,0.000000,0.000000}%
\pgfsetstrokecolor{currentstroke}%
\pgfsetdash{}{0pt}%
\pgfpathmoveto{\pgfqpoint{4.644133in}{1.388608in}}%
\pgfpathlineto{\pgfqpoint{4.658479in}{1.383329in}}%
\pgfpathlineto{\pgfqpoint{4.672831in}{1.378073in}}%
\pgfpathlineto{\pgfqpoint{4.687190in}{1.372841in}}%
\pgfpathlineto{\pgfqpoint{4.701555in}{1.367631in}}%
\pgfpathlineto{\pgfqpoint{4.693639in}{1.364397in}}%
\pgfpathlineto{\pgfqpoint{4.685719in}{1.361510in}}%
\pgfpathlineto{\pgfqpoint{4.677796in}{1.358978in}}%
\pgfpathlineto{\pgfqpoint{4.669869in}{1.356812in}}%
\pgfpathlineto{\pgfqpoint{4.655486in}{1.362378in}}%
\pgfpathlineto{\pgfqpoint{4.641109in}{1.367966in}}%
\pgfpathlineto{\pgfqpoint{4.626738in}{1.373578in}}%
\pgfpathlineto{\pgfqpoint{4.612374in}{1.379214in}}%
\pgfpathlineto{\pgfqpoint{4.620320in}{1.381018in}}%
\pgfpathlineto{\pgfqpoint{4.628262in}{1.383191in}}%
\pgfpathlineto{\pgfqpoint{4.636199in}{1.385725in}}%
\pgfpathlineto{\pgfqpoint{4.644133in}{1.388608in}}%
\pgfpathclose%
\pgfusepath{fill}%
\end{pgfscope}%
\begin{pgfscope}%
\pgfpathrectangle{\pgfqpoint{1.150000in}{0.150000in}}{\pgfqpoint{5.700000in}{5.700000in}}%
\pgfusepath{clip}%
\pgfsetbuttcap%
\pgfsetroundjoin%
\definecolor{currentfill}{rgb}{0.274952,0.037752,0.364543}%
\pgfsetfillcolor{currentfill}%
\pgfsetfillopacity{0.700000}%
\pgfsetlinewidth{0.000000pt}%
\definecolor{currentstroke}{rgb}{0.000000,0.000000,0.000000}%
\pgfsetstrokecolor{currentstroke}%
\pgfsetdash{}{0pt}%
\pgfpathmoveto{\pgfqpoint{5.084234in}{1.350643in}}%
\pgfpathlineto{\pgfqpoint{5.098710in}{1.346909in}}%
\pgfpathlineto{\pgfqpoint{5.113193in}{1.343199in}}%
\pgfpathlineto{\pgfqpoint{5.127684in}{1.339512in}}%
\pgfpathlineto{\pgfqpoint{5.119869in}{1.331027in}}%
\pgfpathlineto{\pgfqpoint{5.112053in}{1.322764in}}%
\pgfpathlineto{\pgfqpoint{5.104235in}{1.314733in}}%
\pgfpathlineto{\pgfqpoint{5.096416in}{1.306941in}}%
\pgfpathlineto{\pgfqpoint{5.081916in}{1.310944in}}%
\pgfpathlineto{\pgfqpoint{5.067423in}{1.314970in}}%
\pgfpathlineto{\pgfqpoint{5.052938in}{1.319020in}}%
\pgfpathlineto{\pgfqpoint{5.060764in}{1.326571in}}%
\pgfpathlineto{\pgfqpoint{5.068589in}{1.334364in}}%
\pgfpathlineto{\pgfqpoint{5.076413in}{1.342391in}}%
\pgfpathlineto{\pgfqpoint{5.084234in}{1.350643in}}%
\pgfpathclose%
\pgfusepath{fill}%
\end{pgfscope}%
\begin{pgfscope}%
\pgfpathrectangle{\pgfqpoint{1.150000in}{0.150000in}}{\pgfqpoint{5.700000in}{5.700000in}}%
\pgfusepath{clip}%
\pgfsetbuttcap%
\pgfsetroundjoin%
\definecolor{currentfill}{rgb}{0.153894,0.680203,0.504172}%
\pgfsetfillcolor{currentfill}%
\pgfsetfillopacity{0.700000}%
\pgfsetlinewidth{0.000000pt}%
\definecolor{currentstroke}{rgb}{0.000000,0.000000,0.000000}%
\pgfsetstrokecolor{currentstroke}%
\pgfsetdash{}{0pt}%
\pgfpathmoveto{\pgfqpoint{2.367822in}{2.915909in}}%
\pgfpathlineto{\pgfqpoint{2.381902in}{2.903566in}}%
\pgfpathlineto{\pgfqpoint{2.395983in}{2.891267in}}%
\pgfpathlineto{\pgfqpoint{2.410064in}{2.879010in}}%
\pgfpathlineto{\pgfqpoint{2.424147in}{2.866795in}}%
\pgfpathlineto{\pgfqpoint{2.414195in}{2.894182in}}%
\pgfpathlineto{\pgfqpoint{2.404198in}{2.922407in}}%
\pgfpathlineto{\pgfqpoint{2.394155in}{2.951485in}}%
\pgfpathlineto{\pgfqpoint{2.384064in}{2.981432in}}%
\pgfpathlineto{\pgfqpoint{2.369910in}{2.994143in}}%
\pgfpathlineto{\pgfqpoint{2.355756in}{3.006897in}}%
\pgfpathlineto{\pgfqpoint{2.341603in}{3.019694in}}%
\pgfpathlineto{\pgfqpoint{2.327450in}{3.032534in}}%
\pgfpathlineto{\pgfqpoint{2.337615in}{3.002081in}}%
\pgfpathlineto{\pgfqpoint{2.347732in}{2.972503in}}%
\pgfpathlineto{\pgfqpoint{2.357800in}{2.943785in}}%
\pgfpathlineto{\pgfqpoint{2.367822in}{2.915909in}}%
\pgfpathclose%
\pgfusepath{fill}%
\end{pgfscope}%
\begin{pgfscope}%
\pgfpathrectangle{\pgfqpoint{1.150000in}{0.150000in}}{\pgfqpoint{5.700000in}{5.700000in}}%
\pgfusepath{clip}%
\pgfsetbuttcap%
\pgfsetroundjoin%
\definecolor{currentfill}{rgb}{0.257322,0.256130,0.526563}%
\pgfsetfillcolor{currentfill}%
\pgfsetfillopacity{0.700000}%
\pgfsetlinewidth{0.000000pt}%
\definecolor{currentstroke}{rgb}{0.000000,0.000000,0.000000}%
\pgfsetstrokecolor{currentstroke}%
\pgfsetdash{}{0pt}%
\pgfpathmoveto{\pgfqpoint{3.773375in}{1.776129in}}%
\pgfpathlineto{\pgfqpoint{3.787546in}{1.768075in}}%
\pgfpathlineto{\pgfqpoint{3.801721in}{1.760047in}}%
\pgfpathlineto{\pgfqpoint{3.815900in}{1.752044in}}%
\pgfpathlineto{\pgfqpoint{3.830085in}{1.744066in}}%
\pgfpathlineto{\pgfqpoint{3.821688in}{1.752923in}}%
\pgfpathlineto{\pgfqpoint{3.813276in}{1.762344in}}%
\pgfpathlineto{\pgfqpoint{3.804849in}{1.772344in}}%
\pgfpathlineto{\pgfqpoint{3.796407in}{1.782933in}}%
\pgfpathlineto{\pgfqpoint{3.782184in}{1.791329in}}%
\pgfpathlineto{\pgfqpoint{3.767966in}{1.799750in}}%
\pgfpathlineto{\pgfqpoint{3.753752in}{1.808197in}}%
\pgfpathlineto{\pgfqpoint{3.739542in}{1.816670in}}%
\pgfpathlineto{\pgfqpoint{3.748024in}{1.805656in}}%
\pgfpathlineto{\pgfqpoint{3.756490in}{1.795236in}}%
\pgfpathlineto{\pgfqpoint{3.764940in}{1.785398in}}%
\pgfpathlineto{\pgfqpoint{3.773375in}{1.776129in}}%
\pgfpathclose%
\pgfusepath{fill}%
\end{pgfscope}%
\begin{pgfscope}%
\pgfpathrectangle{\pgfqpoint{1.150000in}{0.150000in}}{\pgfqpoint{5.700000in}{5.700000in}}%
\pgfusepath{clip}%
\pgfsetbuttcap%
\pgfsetroundjoin%
\definecolor{currentfill}{rgb}{0.163625,0.471133,0.558148}%
\pgfsetfillcolor{currentfill}%
\pgfsetfillopacity{0.700000}%
\pgfsetlinewidth{0.000000pt}%
\definecolor{currentstroke}{rgb}{0.000000,0.000000,0.000000}%
\pgfsetstrokecolor{currentstroke}%
\pgfsetdash{}{0pt}%
\pgfpathmoveto{\pgfqpoint{3.025475in}{2.326682in}}%
\pgfpathlineto{\pgfqpoint{3.039568in}{2.316418in}}%
\pgfpathlineto{\pgfqpoint{3.053664in}{2.306185in}}%
\pgfpathlineto{\pgfqpoint{3.067763in}{2.295983in}}%
\pgfpathlineto{\pgfqpoint{3.081864in}{2.285813in}}%
\pgfpathlineto{\pgfqpoint{3.072745in}{2.304940in}}%
\pgfpathlineto{\pgfqpoint{3.063597in}{2.324789in}}%
\pgfpathlineto{\pgfqpoint{3.054417in}{2.345375in}}%
\pgfpathlineto{\pgfqpoint{3.045207in}{2.366712in}}%
\pgfpathlineto{\pgfqpoint{3.031049in}{2.377344in}}%
\pgfpathlineto{\pgfqpoint{3.016894in}{2.388008in}}%
\pgfpathlineto{\pgfqpoint{3.002741in}{2.398703in}}%
\pgfpathlineto{\pgfqpoint{2.988590in}{2.409430in}}%
\pgfpathlineto{\pgfqpoint{2.997859in}{2.387624in}}%
\pgfpathlineto{\pgfqpoint{3.007095in}{2.366573in}}%
\pgfpathlineto{\pgfqpoint{3.016301in}{2.346264in}}%
\pgfpathlineto{\pgfqpoint{3.025475in}{2.326682in}}%
\pgfpathclose%
\pgfusepath{fill}%
\end{pgfscope}%
\begin{pgfscope}%
\pgfpathrectangle{\pgfqpoint{1.150000in}{0.150000in}}{\pgfqpoint{5.700000in}{5.700000in}}%
\pgfusepath{clip}%
\pgfsetbuttcap%
\pgfsetroundjoin%
\definecolor{currentfill}{rgb}{0.218130,0.347432,0.550038}%
\pgfsetfillcolor{currentfill}%
\pgfsetfillopacity{0.700000}%
\pgfsetlinewidth{0.000000pt}%
\definecolor{currentstroke}{rgb}{0.000000,0.000000,0.000000}%
\pgfsetstrokecolor{currentstroke}%
\pgfsetdash{}{0pt}%
\pgfpathmoveto{\pgfqpoint{3.456184in}{1.991642in}}%
\pgfpathlineto{\pgfqpoint{3.470315in}{1.982639in}}%
\pgfpathlineto{\pgfqpoint{3.484450in}{1.973663in}}%
\pgfpathlineto{\pgfqpoint{3.498589in}{1.964714in}}%
\pgfpathlineto{\pgfqpoint{3.512732in}{1.955792in}}%
\pgfpathlineto{\pgfqpoint{3.504060in}{1.969133in}}%
\pgfpathlineto{\pgfqpoint{3.495368in}{1.983111in}}%
\pgfpathlineto{\pgfqpoint{3.486655in}{1.997738in}}%
\pgfpathlineto{\pgfqpoint{3.477920in}{2.013027in}}%
\pgfpathlineto{\pgfqpoint{3.463731in}{2.022387in}}%
\pgfpathlineto{\pgfqpoint{3.449546in}{2.031774in}}%
\pgfpathlineto{\pgfqpoint{3.435364in}{2.041189in}}%
\pgfpathlineto{\pgfqpoint{3.421185in}{2.050631in}}%
\pgfpathlineto{\pgfqpoint{3.429968in}{2.034896in}}%
\pgfpathlineto{\pgfqpoint{3.438729in}{2.019829in}}%
\pgfpathlineto{\pgfqpoint{3.447467in}{2.005415in}}%
\pgfpathlineto{\pgfqpoint{3.456184in}{1.991642in}}%
\pgfpathclose%
\pgfusepath{fill}%
\end{pgfscope}%
\begin{pgfscope}%
\pgfpathrectangle{\pgfqpoint{1.150000in}{0.150000in}}{\pgfqpoint{5.700000in}{5.700000in}}%
\pgfusepath{clip}%
\pgfsetbuttcap%
\pgfsetroundjoin%
\definecolor{currentfill}{rgb}{0.283187,0.125848,0.444960}%
\pgfsetfillcolor{currentfill}%
\pgfsetfillopacity{0.700000}%
\pgfsetlinewidth{0.000000pt}%
\definecolor{currentstroke}{rgb}{0.000000,0.000000,0.000000}%
\pgfsetstrokecolor{currentstroke}%
\pgfsetdash{}{0pt}%
\pgfpathmoveto{\pgfqpoint{4.294091in}{1.500939in}}%
\pgfpathlineto{\pgfqpoint{4.308359in}{1.494461in}}%
\pgfpathlineto{\pgfqpoint{4.322633in}{1.488007in}}%
\pgfpathlineto{\pgfqpoint{4.336912in}{1.481577in}}%
\pgfpathlineto{\pgfqpoint{4.351197in}{1.475171in}}%
\pgfpathlineto{\pgfqpoint{4.343134in}{1.476880in}}%
\pgfpathlineto{\pgfqpoint{4.335064in}{1.479033in}}%
\pgfpathlineto{\pgfqpoint{4.326986in}{1.481641in}}%
\pgfpathlineto{\pgfqpoint{4.318900in}{1.484714in}}%
\pgfpathlineto{\pgfqpoint{4.304589in}{1.491505in}}%
\pgfpathlineto{\pgfqpoint{4.290284in}{1.498321in}}%
\pgfpathlineto{\pgfqpoint{4.275984in}{1.505160in}}%
\pgfpathlineto{\pgfqpoint{4.261689in}{1.512023in}}%
\pgfpathlineto{\pgfqpoint{4.269801in}{1.508558in}}%
\pgfpathlineto{\pgfqpoint{4.277906in}{1.505563in}}%
\pgfpathlineto{\pgfqpoint{4.286003in}{1.503027in}}%
\pgfpathlineto{\pgfqpoint{4.294091in}{1.500939in}}%
\pgfpathclose%
\pgfusepath{fill}%
\end{pgfscope}%
\begin{pgfscope}%
\pgfpathrectangle{\pgfqpoint{1.150000in}{0.150000in}}{\pgfqpoint{5.700000in}{5.700000in}}%
\pgfusepath{clip}%
\pgfsetbuttcap%
\pgfsetroundjoin%
\definecolor{currentfill}{rgb}{0.140210,0.665859,0.513427}%
\pgfsetfillcolor{currentfill}%
\pgfsetfillopacity{0.700000}%
\pgfsetlinewidth{0.000000pt}%
\definecolor{currentstroke}{rgb}{0.000000,0.000000,0.000000}%
\pgfsetstrokecolor{currentstroke}%
\pgfsetdash{}{0pt}%
\pgfpathmoveto{\pgfqpoint{2.424147in}{2.866795in}}%
\pgfpathlineto{\pgfqpoint{2.438230in}{2.854623in}}%
\pgfpathlineto{\pgfqpoint{2.452314in}{2.842492in}}%
\pgfpathlineto{\pgfqpoint{2.466399in}{2.830402in}}%
\pgfpathlineto{\pgfqpoint{2.480485in}{2.818353in}}%
\pgfpathlineto{\pgfqpoint{2.470603in}{2.845253in}}%
\pgfpathlineto{\pgfqpoint{2.460677in}{2.872985in}}%
\pgfpathlineto{\pgfqpoint{2.450705in}{2.901565in}}%
\pgfpathlineto{\pgfqpoint{2.440687in}{2.931009in}}%
\pgfpathlineto{\pgfqpoint{2.426531in}{2.943552in}}%
\pgfpathlineto{\pgfqpoint{2.412374in}{2.956137in}}%
\pgfpathlineto{\pgfqpoint{2.398219in}{2.968764in}}%
\pgfpathlineto{\pgfqpoint{2.384064in}{2.981432in}}%
\pgfpathlineto{\pgfqpoint{2.394155in}{2.951485in}}%
\pgfpathlineto{\pgfqpoint{2.404198in}{2.922407in}}%
\pgfpathlineto{\pgfqpoint{2.414195in}{2.894182in}}%
\pgfpathlineto{\pgfqpoint{2.424147in}{2.866795in}}%
\pgfpathclose%
\pgfusepath{fill}%
\end{pgfscope}%
\begin{pgfscope}%
\pgfpathrectangle{\pgfqpoint{1.150000in}{0.150000in}}{\pgfqpoint{5.700000in}{5.700000in}}%
\pgfusepath{clip}%
\pgfsetbuttcap%
\pgfsetroundjoin%
\definecolor{currentfill}{rgb}{0.281446,0.084320,0.407414}%
\pgfsetfillcolor{currentfill}%
\pgfsetfillopacity{0.700000}%
\pgfsetlinewidth{0.000000pt}%
\definecolor{currentstroke}{rgb}{0.000000,0.000000,0.000000}%
\pgfsetstrokecolor{currentstroke}%
\pgfsetdash{}{0pt}%
\pgfpathmoveto{\pgfqpoint{4.497683in}{1.425139in}}%
\pgfpathlineto{\pgfqpoint{4.511998in}{1.419317in}}%
\pgfpathlineto{\pgfqpoint{4.526319in}{1.413517in}}%
\pgfpathlineto{\pgfqpoint{4.540646in}{1.407741in}}%
\pgfpathlineto{\pgfqpoint{4.554979in}{1.401989in}}%
\pgfpathlineto{\pgfqpoint{4.547008in}{1.400929in}}%
\pgfpathlineto{\pgfqpoint{4.539033in}{1.400261in}}%
\pgfpathlineto{\pgfqpoint{4.531052in}{1.399996in}}%
\pgfpathlineto{\pgfqpoint{4.523066in}{1.400143in}}%
\pgfpathlineto{\pgfqpoint{4.508711in}{1.406266in}}%
\pgfpathlineto{\pgfqpoint{4.494363in}{1.412412in}}%
\pgfpathlineto{\pgfqpoint{4.480020in}{1.418582in}}%
\pgfpathlineto{\pgfqpoint{4.465683in}{1.424775in}}%
\pgfpathlineto{\pgfqpoint{4.473691in}{1.424252in}}%
\pgfpathlineto{\pgfqpoint{4.481694in}{1.424145in}}%
\pgfpathlineto{\pgfqpoint{4.489691in}{1.424444in}}%
\pgfpathlineto{\pgfqpoint{4.497683in}{1.425139in}}%
\pgfpathclose%
\pgfusepath{fill}%
\end{pgfscope}%
\begin{pgfscope}%
\pgfpathrectangle{\pgfqpoint{1.150000in}{0.150000in}}{\pgfqpoint{5.700000in}{5.700000in}}%
\pgfusepath{clip}%
\pgfsetbuttcap%
\pgfsetroundjoin%
\definecolor{currentfill}{rgb}{0.278826,0.175490,0.483397}%
\pgfsetfillcolor{currentfill}%
\pgfsetfillopacity{0.700000}%
\pgfsetlinewidth{0.000000pt}%
\definecolor{currentstroke}{rgb}{0.000000,0.000000,0.000000}%
\pgfsetstrokecolor{currentstroke}%
\pgfsetdash{}{0pt}%
\pgfpathmoveto{\pgfqpoint{4.090564in}{1.596256in}}%
\pgfpathlineto{\pgfqpoint{4.104796in}{1.589103in}}%
\pgfpathlineto{\pgfqpoint{4.119033in}{1.581975in}}%
\pgfpathlineto{\pgfqpoint{4.133275in}{1.574871in}}%
\pgfpathlineto{\pgfqpoint{4.147522in}{1.567791in}}%
\pgfpathlineto{\pgfqpoint{4.139342in}{1.572526in}}%
\pgfpathlineto{\pgfqpoint{4.131152in}{1.577759in}}%
\pgfpathlineto{\pgfqpoint{4.122952in}{1.583503in}}%
\pgfpathlineto{\pgfqpoint{4.114742in}{1.589769in}}%
\pgfpathlineto{\pgfqpoint{4.100463in}{1.597250in}}%
\pgfpathlineto{\pgfqpoint{4.086190in}{1.604755in}}%
\pgfpathlineto{\pgfqpoint{4.071921in}{1.612284in}}%
\pgfpathlineto{\pgfqpoint{4.057657in}{1.619838in}}%
\pgfpathlineto{\pgfqpoint{4.065900in}{1.613166in}}%
\pgfpathlineto{\pgfqpoint{4.074132in}{1.607019in}}%
\pgfpathlineto{\pgfqpoint{4.082354in}{1.601386in}}%
\pgfpathlineto{\pgfqpoint{4.090564in}{1.596256in}}%
\pgfpathclose%
\pgfusepath{fill}%
\end{pgfscope}%
\begin{pgfscope}%
\pgfpathrectangle{\pgfqpoint{1.150000in}{0.150000in}}{\pgfqpoint{5.700000in}{5.700000in}}%
\pgfusepath{clip}%
\pgfsetbuttcap%
\pgfsetroundjoin%
\definecolor{currentfill}{rgb}{0.168126,0.459988,0.558082}%
\pgfsetfillcolor{currentfill}%
\pgfsetfillopacity{0.700000}%
\pgfsetlinewidth{0.000000pt}%
\definecolor{currentstroke}{rgb}{0.000000,0.000000,0.000000}%
\pgfsetstrokecolor{currentstroke}%
\pgfsetdash{}{0pt}%
\pgfpathmoveto{\pgfqpoint{3.081864in}{2.285813in}}%
\pgfpathlineto{\pgfqpoint{3.095969in}{2.275673in}}%
\pgfpathlineto{\pgfqpoint{3.110075in}{2.265564in}}%
\pgfpathlineto{\pgfqpoint{3.124185in}{2.255485in}}%
\pgfpathlineto{\pgfqpoint{3.138298in}{2.245437in}}%
\pgfpathlineto{\pgfqpoint{3.129233in}{2.264110in}}%
\pgfpathlineto{\pgfqpoint{3.120140in}{2.283501in}}%
\pgfpathlineto{\pgfqpoint{3.111017in}{2.303623in}}%
\pgfpathlineto{\pgfqpoint{3.101864in}{2.324491in}}%
\pgfpathlineto{\pgfqpoint{3.087696in}{2.335000in}}%
\pgfpathlineto{\pgfqpoint{3.073530in}{2.345540in}}%
\pgfpathlineto{\pgfqpoint{3.059367in}{2.356110in}}%
\pgfpathlineto{\pgfqpoint{3.045207in}{2.366712in}}%
\pgfpathlineto{\pgfqpoint{3.054417in}{2.345375in}}%
\pgfpathlineto{\pgfqpoint{3.063597in}{2.324789in}}%
\pgfpathlineto{\pgfqpoint{3.072745in}{2.304940in}}%
\pgfpathlineto{\pgfqpoint{3.081864in}{2.285813in}}%
\pgfpathclose%
\pgfusepath{fill}%
\end{pgfscope}%
\begin{pgfscope}%
\pgfpathrectangle{\pgfqpoint{1.150000in}{0.150000in}}{\pgfqpoint{5.700000in}{5.700000in}}%
\pgfusepath{clip}%
\pgfsetbuttcap%
\pgfsetroundjoin%
\definecolor{currentfill}{rgb}{0.130067,0.651384,0.521608}%
\pgfsetfillcolor{currentfill}%
\pgfsetfillopacity{0.700000}%
\pgfsetlinewidth{0.000000pt}%
\definecolor{currentstroke}{rgb}{0.000000,0.000000,0.000000}%
\pgfsetstrokecolor{currentstroke}%
\pgfsetdash{}{0pt}%
\pgfpathmoveto{\pgfqpoint{2.480485in}{2.818353in}}%
\pgfpathlineto{\pgfqpoint{2.494572in}{2.806345in}}%
\pgfpathlineto{\pgfqpoint{2.508660in}{2.794378in}}%
\pgfpathlineto{\pgfqpoint{2.522749in}{2.782450in}}%
\pgfpathlineto{\pgfqpoint{2.536840in}{2.770562in}}%
\pgfpathlineto{\pgfqpoint{2.527026in}{2.796976in}}%
\pgfpathlineto{\pgfqpoint{2.517170in}{2.824217in}}%
\pgfpathlineto{\pgfqpoint{2.507269in}{2.852300in}}%
\pgfpathlineto{\pgfqpoint{2.497324in}{2.881242in}}%
\pgfpathlineto{\pgfqpoint{2.483163in}{2.893623in}}%
\pgfpathlineto{\pgfqpoint{2.469004in}{2.906045in}}%
\pgfpathlineto{\pgfqpoint{2.454845in}{2.918507in}}%
\pgfpathlineto{\pgfqpoint{2.440687in}{2.931009in}}%
\pgfpathlineto{\pgfqpoint{2.450705in}{2.901565in}}%
\pgfpathlineto{\pgfqpoint{2.460677in}{2.872985in}}%
\pgfpathlineto{\pgfqpoint{2.470603in}{2.845253in}}%
\pgfpathlineto{\pgfqpoint{2.480485in}{2.818353in}}%
\pgfpathclose%
\pgfusepath{fill}%
\end{pgfscope}%
\begin{pgfscope}%
\pgfpathrectangle{\pgfqpoint{1.150000in}{0.150000in}}{\pgfqpoint{5.700000in}{5.700000in}}%
\pgfusepath{clip}%
\pgfsetbuttcap%
\pgfsetroundjoin%
\definecolor{currentfill}{rgb}{0.276022,0.044167,0.370164}%
\pgfsetfillcolor{currentfill}%
\pgfsetfillopacity{0.700000}%
\pgfsetlinewidth{0.000000pt}%
\definecolor{currentstroke}{rgb}{0.000000,0.000000,0.000000}%
\pgfsetstrokecolor{currentstroke}%
\pgfsetdash{}{0pt}%
\pgfpathmoveto{\pgfqpoint{4.848222in}{1.345755in}}%
\pgfpathlineto{\pgfqpoint{4.862633in}{1.341098in}}%
\pgfpathlineto{\pgfqpoint{4.877051in}{1.336464in}}%
\pgfpathlineto{\pgfqpoint{4.891476in}{1.331853in}}%
\pgfpathlineto{\pgfqpoint{4.905907in}{1.327265in}}%
\pgfpathlineto{\pgfqpoint{4.898048in}{1.321724in}}%
\pgfpathlineto{\pgfqpoint{4.890187in}{1.316483in}}%
\pgfpathlineto{\pgfqpoint{4.882324in}{1.311550in}}%
\pgfpathlineto{\pgfqpoint{4.874458in}{1.306936in}}%
\pgfpathlineto{\pgfqpoint{4.860012in}{1.311866in}}%
\pgfpathlineto{\pgfqpoint{4.845573in}{1.316820in}}%
\pgfpathlineto{\pgfqpoint{4.831141in}{1.321796in}}%
\pgfpathlineto{\pgfqpoint{4.816716in}{1.326796in}}%
\pgfpathlineto{\pgfqpoint{4.824597in}{1.331063in}}%
\pgfpathlineto{\pgfqpoint{4.832474in}{1.335651in}}%
\pgfpathlineto{\pgfqpoint{4.840350in}{1.340552in}}%
\pgfpathlineto{\pgfqpoint{4.848222in}{1.345755in}}%
\pgfpathclose%
\pgfusepath{fill}%
\end{pgfscope}%
\begin{pgfscope}%
\pgfpathrectangle{\pgfqpoint{1.150000in}{0.150000in}}{\pgfqpoint{5.700000in}{5.700000in}}%
\pgfusepath{clip}%
\pgfsetbuttcap%
\pgfsetroundjoin%
\definecolor{currentfill}{rgb}{0.260571,0.246922,0.522828}%
\pgfsetfillcolor{currentfill}%
\pgfsetfillopacity{0.700000}%
\pgfsetlinewidth{0.000000pt}%
\definecolor{currentstroke}{rgb}{0.000000,0.000000,0.000000}%
\pgfsetstrokecolor{currentstroke}%
\pgfsetdash{}{0pt}%
\pgfpathmoveto{\pgfqpoint{3.830085in}{1.744066in}}%
\pgfpathlineto{\pgfqpoint{3.844273in}{1.736114in}}%
\pgfpathlineto{\pgfqpoint{3.858466in}{1.728187in}}%
\pgfpathlineto{\pgfqpoint{3.872664in}{1.720286in}}%
\pgfpathlineto{\pgfqpoint{3.886866in}{1.712410in}}%
\pgfpathlineto{\pgfqpoint{3.878506in}{1.720854in}}%
\pgfpathlineto{\pgfqpoint{3.870132in}{1.729860in}}%
\pgfpathlineto{\pgfqpoint{3.861744in}{1.739439in}}%
\pgfpathlineto{\pgfqpoint{3.853340in}{1.749603in}}%
\pgfpathlineto{\pgfqpoint{3.839101in}{1.757897in}}%
\pgfpathlineto{\pgfqpoint{3.824865in}{1.766217in}}%
\pgfpathlineto{\pgfqpoint{3.810634in}{1.774562in}}%
\pgfpathlineto{\pgfqpoint{3.796407in}{1.782933in}}%
\pgfpathlineto{\pgfqpoint{3.804849in}{1.772344in}}%
\pgfpathlineto{\pgfqpoint{3.813276in}{1.762344in}}%
\pgfpathlineto{\pgfqpoint{3.821688in}{1.752923in}}%
\pgfpathlineto{\pgfqpoint{3.830085in}{1.744066in}}%
\pgfpathclose%
\pgfusepath{fill}%
\end{pgfscope}%
\begin{pgfscope}%
\pgfpathrectangle{\pgfqpoint{1.150000in}{0.150000in}}{\pgfqpoint{5.700000in}{5.700000in}}%
\pgfusepath{clip}%
\pgfsetbuttcap%
\pgfsetroundjoin%
\definecolor{currentfill}{rgb}{0.274952,0.037752,0.364543}%
\pgfsetfillcolor{currentfill}%
\pgfsetfillopacity{0.700000}%
\pgfsetlinewidth{0.000000pt}%
\definecolor{currentstroke}{rgb}{0.000000,0.000000,0.000000}%
\pgfsetstrokecolor{currentstroke}%
\pgfsetdash{}{0pt}%
\pgfpathmoveto{\pgfqpoint{4.995071in}{1.335450in}}%
\pgfpathlineto{\pgfqpoint{5.009527in}{1.331308in}}%
\pgfpathlineto{\pgfqpoint{5.023990in}{1.327189in}}%
\pgfpathlineto{\pgfqpoint{5.038460in}{1.323093in}}%
\pgfpathlineto{\pgfqpoint{5.052938in}{1.319020in}}%
\pgfpathlineto{\pgfqpoint{5.045110in}{1.311719in}}%
\pgfpathlineto{\pgfqpoint{5.037280in}{1.304676in}}%
\pgfpathlineto{\pgfqpoint{5.029449in}{1.297899in}}%
\pgfpathlineto{\pgfqpoint{5.021617in}{1.291398in}}%
\pgfpathlineto{\pgfqpoint{5.007128in}{1.295801in}}%
\pgfpathlineto{\pgfqpoint{4.992647in}{1.300226in}}%
\pgfpathlineto{\pgfqpoint{4.978172in}{1.304675in}}%
\pgfpathlineto{\pgfqpoint{4.963705in}{1.309147in}}%
\pgfpathlineto{\pgfqpoint{4.971549in}{1.315313in}}%
\pgfpathlineto{\pgfqpoint{4.979392in}{1.321758in}}%
\pgfpathlineto{\pgfqpoint{4.987232in}{1.328473in}}%
\pgfpathlineto{\pgfqpoint{4.995071in}{1.335450in}}%
\pgfpathclose%
\pgfusepath{fill}%
\end{pgfscope}%
\begin{pgfscope}%
\pgfpathrectangle{\pgfqpoint{1.150000in}{0.150000in}}{\pgfqpoint{5.700000in}{5.700000in}}%
\pgfusepath{clip}%
\pgfsetbuttcap%
\pgfsetroundjoin%
\definecolor{currentfill}{rgb}{0.223925,0.334994,0.548053}%
\pgfsetfillcolor{currentfill}%
\pgfsetfillopacity{0.700000}%
\pgfsetlinewidth{0.000000pt}%
\definecolor{currentstroke}{rgb}{0.000000,0.000000,0.000000}%
\pgfsetstrokecolor{currentstroke}%
\pgfsetdash{}{0pt}%
\pgfpathmoveto{\pgfqpoint{3.512732in}{1.955792in}}%
\pgfpathlineto{\pgfqpoint{3.526878in}{1.946897in}}%
\pgfpathlineto{\pgfqpoint{3.541028in}{1.938030in}}%
\pgfpathlineto{\pgfqpoint{3.555182in}{1.929189in}}%
\pgfpathlineto{\pgfqpoint{3.569340in}{1.920375in}}%
\pgfpathlineto{\pgfqpoint{3.560713in}{1.933286in}}%
\pgfpathlineto{\pgfqpoint{3.552066in}{1.946828in}}%
\pgfpathlineto{\pgfqpoint{3.543399in}{1.961015in}}%
\pgfpathlineto{\pgfqpoint{3.534711in}{1.975860in}}%
\pgfpathlineto{\pgfqpoint{3.520508in}{1.985111in}}%
\pgfpathlineto{\pgfqpoint{3.506308in}{1.994389in}}%
\pgfpathlineto{\pgfqpoint{3.492112in}{2.003695in}}%
\pgfpathlineto{\pgfqpoint{3.477920in}{2.013027in}}%
\pgfpathlineto{\pgfqpoint{3.486655in}{1.997738in}}%
\pgfpathlineto{\pgfqpoint{3.495368in}{1.983111in}}%
\pgfpathlineto{\pgfqpoint{3.504060in}{1.969133in}}%
\pgfpathlineto{\pgfqpoint{3.512732in}{1.955792in}}%
\pgfpathclose%
\pgfusepath{fill}%
\end{pgfscope}%
\begin{pgfscope}%
\pgfpathrectangle{\pgfqpoint{1.150000in}{0.150000in}}{\pgfqpoint{5.700000in}{5.700000in}}%
\pgfusepath{clip}%
\pgfsetbuttcap%
\pgfsetroundjoin%
\definecolor{currentfill}{rgb}{0.277941,0.056324,0.381191}%
\pgfsetfillcolor{currentfill}%
\pgfsetfillopacity{0.700000}%
\pgfsetlinewidth{0.000000pt}%
\definecolor{currentstroke}{rgb}{0.000000,0.000000,0.000000}%
\pgfsetstrokecolor{currentstroke}%
\pgfsetdash{}{0pt}%
\pgfpathmoveto{\pgfqpoint{4.701555in}{1.367631in}}%
\pgfpathlineto{\pgfqpoint{4.715927in}{1.362446in}}%
\pgfpathlineto{\pgfqpoint{4.730305in}{1.357283in}}%
\pgfpathlineto{\pgfqpoint{4.744691in}{1.352144in}}%
\pgfpathlineto{\pgfqpoint{4.759082in}{1.347028in}}%
\pgfpathlineto{\pgfqpoint{4.751183in}{1.343442in}}%
\pgfpathlineto{\pgfqpoint{4.743281in}{1.340200in}}%
\pgfpathlineto{\pgfqpoint{4.735375in}{1.337311in}}%
\pgfpathlineto{\pgfqpoint{4.727466in}{1.334783in}}%
\pgfpathlineto{\pgfqpoint{4.713057in}{1.340256in}}%
\pgfpathlineto{\pgfqpoint{4.698655in}{1.345751in}}%
\pgfpathlineto{\pgfqpoint{4.684259in}{1.351270in}}%
\pgfpathlineto{\pgfqpoint{4.669869in}{1.356812in}}%
\pgfpathlineto{\pgfqpoint{4.677796in}{1.358978in}}%
\pgfpathlineto{\pgfqpoint{4.685719in}{1.361510in}}%
\pgfpathlineto{\pgfqpoint{4.693639in}{1.364397in}}%
\pgfpathlineto{\pgfqpoint{4.701555in}{1.367631in}}%
\pgfpathclose%
\pgfusepath{fill}%
\end{pgfscope}%
\begin{pgfscope}%
\pgfpathrectangle{\pgfqpoint{1.150000in}{0.150000in}}{\pgfqpoint{5.700000in}{5.700000in}}%
\pgfusepath{clip}%
\pgfsetbuttcap%
\pgfsetroundjoin%
\definecolor{currentfill}{rgb}{0.123444,0.636809,0.528763}%
\pgfsetfillcolor{currentfill}%
\pgfsetfillopacity{0.700000}%
\pgfsetlinewidth{0.000000pt}%
\definecolor{currentstroke}{rgb}{0.000000,0.000000,0.000000}%
\pgfsetstrokecolor{currentstroke}%
\pgfsetdash{}{0pt}%
\pgfpathmoveto{\pgfqpoint{2.536840in}{2.770562in}}%
\pgfpathlineto{\pgfqpoint{2.550932in}{2.758713in}}%
\pgfpathlineto{\pgfqpoint{2.565025in}{2.746903in}}%
\pgfpathlineto{\pgfqpoint{2.579119in}{2.735132in}}%
\pgfpathlineto{\pgfqpoint{2.593214in}{2.723399in}}%
\pgfpathlineto{\pgfqpoint{2.583468in}{2.749329in}}%
\pgfpathlineto{\pgfqpoint{2.573680in}{2.776081in}}%
\pgfpathlineto{\pgfqpoint{2.563850in}{2.803669in}}%
\pgfpathlineto{\pgfqpoint{2.553975in}{2.832110in}}%
\pgfpathlineto{\pgfqpoint{2.539811in}{2.844335in}}%
\pgfpathlineto{\pgfqpoint{2.525647in}{2.856598in}}%
\pgfpathlineto{\pgfqpoint{2.511485in}{2.868900in}}%
\pgfpathlineto{\pgfqpoint{2.497324in}{2.881242in}}%
\pgfpathlineto{\pgfqpoint{2.507269in}{2.852300in}}%
\pgfpathlineto{\pgfqpoint{2.517170in}{2.824217in}}%
\pgfpathlineto{\pgfqpoint{2.527026in}{2.796976in}}%
\pgfpathlineto{\pgfqpoint{2.536840in}{2.770562in}}%
\pgfpathclose%
\pgfusepath{fill}%
\end{pgfscope}%
\begin{pgfscope}%
\pgfpathrectangle{\pgfqpoint{1.150000in}{0.150000in}}{\pgfqpoint{5.700000in}{5.700000in}}%
\pgfusepath{clip}%
\pgfsetbuttcap%
\pgfsetroundjoin%
\definecolor{currentfill}{rgb}{0.283229,0.120777,0.440584}%
\pgfsetfillcolor{currentfill}%
\pgfsetfillopacity{0.700000}%
\pgfsetlinewidth{0.000000pt}%
\definecolor{currentstroke}{rgb}{0.000000,0.000000,0.000000}%
\pgfsetstrokecolor{currentstroke}%
\pgfsetdash{}{0pt}%
\pgfpathmoveto{\pgfqpoint{4.351197in}{1.475171in}}%
\pgfpathlineto{\pgfqpoint{4.365488in}{1.468789in}}%
\pgfpathlineto{\pgfqpoint{4.379784in}{1.462430in}}%
\pgfpathlineto{\pgfqpoint{4.394086in}{1.456095in}}%
\pgfpathlineto{\pgfqpoint{4.408394in}{1.449784in}}%
\pgfpathlineto{\pgfqpoint{4.400355in}{1.451113in}}%
\pgfpathlineto{\pgfqpoint{4.392310in}{1.452883in}}%
\pgfpathlineto{\pgfqpoint{4.384259in}{1.455104in}}%
\pgfpathlineto{\pgfqpoint{4.376200in}{1.457786in}}%
\pgfpathlineto{\pgfqpoint{4.361867in}{1.464483in}}%
\pgfpathlineto{\pgfqpoint{4.347539in}{1.471203in}}%
\pgfpathlineto{\pgfqpoint{4.333217in}{1.477947in}}%
\pgfpathlineto{\pgfqpoint{4.318900in}{1.484714in}}%
\pgfpathlineto{\pgfqpoint{4.326986in}{1.481641in}}%
\pgfpathlineto{\pgfqpoint{4.335064in}{1.479033in}}%
\pgfpathlineto{\pgfqpoint{4.343134in}{1.476880in}}%
\pgfpathlineto{\pgfqpoint{4.351197in}{1.475171in}}%
\pgfpathclose%
\pgfusepath{fill}%
\end{pgfscope}%
\begin{pgfscope}%
\pgfpathrectangle{\pgfqpoint{1.150000in}{0.150000in}}{\pgfqpoint{5.700000in}{5.700000in}}%
\pgfusepath{clip}%
\pgfsetbuttcap%
\pgfsetroundjoin%
\definecolor{currentfill}{rgb}{0.172719,0.448791,0.557885}%
\pgfsetfillcolor{currentfill}%
\pgfsetfillopacity{0.700000}%
\pgfsetlinewidth{0.000000pt}%
\definecolor{currentstroke}{rgb}{0.000000,0.000000,0.000000}%
\pgfsetstrokecolor{currentstroke}%
\pgfsetdash{}{0pt}%
\pgfpathmoveto{\pgfqpoint{3.138298in}{2.245437in}}%
\pgfpathlineto{\pgfqpoint{3.152413in}{2.235419in}}%
\pgfpathlineto{\pgfqpoint{3.166531in}{2.225431in}}%
\pgfpathlineto{\pgfqpoint{3.180652in}{2.215473in}}%
\pgfpathlineto{\pgfqpoint{3.194776in}{2.205544in}}%
\pgfpathlineto{\pgfqpoint{3.185765in}{2.223764in}}%
\pgfpathlineto{\pgfqpoint{3.176726in}{2.242697in}}%
\pgfpathlineto{\pgfqpoint{3.167659in}{2.262356in}}%
\pgfpathlineto{\pgfqpoint{3.158563in}{2.282756in}}%
\pgfpathlineto{\pgfqpoint{3.144384in}{2.293145in}}%
\pgfpathlineto{\pgfqpoint{3.130208in}{2.303563in}}%
\pgfpathlineto{\pgfqpoint{3.116035in}{2.314012in}}%
\pgfpathlineto{\pgfqpoint{3.101864in}{2.324491in}}%
\pgfpathlineto{\pgfqpoint{3.111017in}{2.303623in}}%
\pgfpathlineto{\pgfqpoint{3.120140in}{2.283501in}}%
\pgfpathlineto{\pgfqpoint{3.129233in}{2.264110in}}%
\pgfpathlineto{\pgfqpoint{3.138298in}{2.245437in}}%
\pgfpathclose%
\pgfusepath{fill}%
\end{pgfscope}%
\begin{pgfscope}%
\pgfpathrectangle{\pgfqpoint{1.150000in}{0.150000in}}{\pgfqpoint{5.700000in}{5.700000in}}%
\pgfusepath{clip}%
\pgfsetbuttcap%
\pgfsetroundjoin%
\definecolor{currentfill}{rgb}{0.281446,0.084320,0.407414}%
\pgfsetfillcolor{currentfill}%
\pgfsetfillopacity{0.700000}%
\pgfsetlinewidth{0.000000pt}%
\definecolor{currentstroke}{rgb}{0.000000,0.000000,0.000000}%
\pgfsetstrokecolor{currentstroke}%
\pgfsetdash{}{0pt}%
\pgfpathmoveto{\pgfqpoint{4.554979in}{1.401989in}}%
\pgfpathlineto{\pgfqpoint{4.569318in}{1.396260in}}%
\pgfpathlineto{\pgfqpoint{4.583664in}{1.390555in}}%
\pgfpathlineto{\pgfqpoint{4.598016in}{1.384873in}}%
\pgfpathlineto{\pgfqpoint{4.612374in}{1.379214in}}%
\pgfpathlineto{\pgfqpoint{4.604423in}{1.377789in}}%
\pgfpathlineto{\pgfqpoint{4.596469in}{1.376752in}}%
\pgfpathlineto{\pgfqpoint{4.588509in}{1.376115in}}%
\pgfpathlineto{\pgfqpoint{4.580545in}{1.375886in}}%
\pgfpathlineto{\pgfqpoint{4.566166in}{1.381915in}}%
\pgfpathlineto{\pgfqpoint{4.551794in}{1.387968in}}%
\pgfpathlineto{\pgfqpoint{4.537427in}{1.394044in}}%
\pgfpathlineto{\pgfqpoint{4.523066in}{1.400143in}}%
\pgfpathlineto{\pgfqpoint{4.531052in}{1.399996in}}%
\pgfpathlineto{\pgfqpoint{4.539033in}{1.400261in}}%
\pgfpathlineto{\pgfqpoint{4.547008in}{1.400929in}}%
\pgfpathlineto{\pgfqpoint{4.554979in}{1.401989in}}%
\pgfpathclose%
\pgfusepath{fill}%
\end{pgfscope}%
\begin{pgfscope}%
\pgfpathrectangle{\pgfqpoint{1.150000in}{0.150000in}}{\pgfqpoint{5.700000in}{5.700000in}}%
\pgfusepath{clip}%
\pgfsetbuttcap%
\pgfsetroundjoin%
\definecolor{currentfill}{rgb}{0.120081,0.622161,0.534946}%
\pgfsetfillcolor{currentfill}%
\pgfsetfillopacity{0.700000}%
\pgfsetlinewidth{0.000000pt}%
\definecolor{currentstroke}{rgb}{0.000000,0.000000,0.000000}%
\pgfsetstrokecolor{currentstroke}%
\pgfsetdash{}{0pt}%
\pgfpathmoveto{\pgfqpoint{2.593214in}{2.723399in}}%
\pgfpathlineto{\pgfqpoint{2.607311in}{2.711705in}}%
\pgfpathlineto{\pgfqpoint{2.621410in}{2.700048in}}%
\pgfpathlineto{\pgfqpoint{2.635510in}{2.688429in}}%
\pgfpathlineto{\pgfqpoint{2.649611in}{2.676847in}}%
\pgfpathlineto{\pgfqpoint{2.639931in}{2.702294in}}%
\pgfpathlineto{\pgfqpoint{2.630211in}{2.728558in}}%
\pgfpathlineto{\pgfqpoint{2.620449in}{2.755652in}}%
\pgfpathlineto{\pgfqpoint{2.610645in}{2.783595in}}%
\pgfpathlineto{\pgfqpoint{2.596476in}{2.795667in}}%
\pgfpathlineto{\pgfqpoint{2.582308in}{2.807777in}}%
\pgfpathlineto{\pgfqpoint{2.568141in}{2.819924in}}%
\pgfpathlineto{\pgfqpoint{2.553975in}{2.832110in}}%
\pgfpathlineto{\pgfqpoint{2.563850in}{2.803669in}}%
\pgfpathlineto{\pgfqpoint{2.573680in}{2.776081in}}%
\pgfpathlineto{\pgfqpoint{2.583468in}{2.749329in}}%
\pgfpathlineto{\pgfqpoint{2.593214in}{2.723399in}}%
\pgfpathclose%
\pgfusepath{fill}%
\end{pgfscope}%
\begin{pgfscope}%
\pgfpathrectangle{\pgfqpoint{1.150000in}{0.150000in}}{\pgfqpoint{5.700000in}{5.700000in}}%
\pgfusepath{clip}%
\pgfsetbuttcap%
\pgfsetroundjoin%
\definecolor{currentfill}{rgb}{0.280255,0.165693,0.476498}%
\pgfsetfillcolor{currentfill}%
\pgfsetfillopacity{0.700000}%
\pgfsetlinewidth{0.000000pt}%
\definecolor{currentstroke}{rgb}{0.000000,0.000000,0.000000}%
\pgfsetstrokecolor{currentstroke}%
\pgfsetdash{}{0pt}%
\pgfpathmoveto{\pgfqpoint{4.147522in}{1.567791in}}%
\pgfpathlineto{\pgfqpoint{4.161774in}{1.560736in}}%
\pgfpathlineto{\pgfqpoint{4.176032in}{1.553704in}}%
\pgfpathlineto{\pgfqpoint{4.190295in}{1.546697in}}%
\pgfpathlineto{\pgfqpoint{4.204563in}{1.539714in}}%
\pgfpathlineto{\pgfqpoint{4.196413in}{1.544054in}}%
\pgfpathlineto{\pgfqpoint{4.188254in}{1.548889in}}%
\pgfpathlineto{\pgfqpoint{4.180085in}{1.554230in}}%
\pgfpathlineto{\pgfqpoint{4.171906in}{1.560089in}}%
\pgfpathlineto{\pgfqpoint{4.157607in}{1.567473in}}%
\pgfpathlineto{\pgfqpoint{4.143314in}{1.574881in}}%
\pgfpathlineto{\pgfqpoint{4.129025in}{1.582313in}}%
\pgfpathlineto{\pgfqpoint{4.114742in}{1.589769in}}%
\pgfpathlineto{\pgfqpoint{4.122952in}{1.583503in}}%
\pgfpathlineto{\pgfqpoint{4.131152in}{1.577759in}}%
\pgfpathlineto{\pgfqpoint{4.139342in}{1.572526in}}%
\pgfpathlineto{\pgfqpoint{4.147522in}{1.567791in}}%
\pgfpathclose%
\pgfusepath{fill}%
\end{pgfscope}%
\begin{pgfscope}%
\pgfpathrectangle{\pgfqpoint{1.150000in}{0.150000in}}{\pgfqpoint{5.700000in}{5.700000in}}%
\pgfusepath{clip}%
\pgfsetbuttcap%
\pgfsetroundjoin%
\definecolor{currentfill}{rgb}{0.227802,0.326594,0.546532}%
\pgfsetfillcolor{currentfill}%
\pgfsetfillopacity{0.700000}%
\pgfsetlinewidth{0.000000pt}%
\definecolor{currentstroke}{rgb}{0.000000,0.000000,0.000000}%
\pgfsetstrokecolor{currentstroke}%
\pgfsetdash{}{0pt}%
\pgfpathmoveto{\pgfqpoint{3.569340in}{1.920375in}}%
\pgfpathlineto{\pgfqpoint{3.583502in}{1.911588in}}%
\pgfpathlineto{\pgfqpoint{3.597667in}{1.902828in}}%
\pgfpathlineto{\pgfqpoint{3.611837in}{1.894094in}}%
\pgfpathlineto{\pgfqpoint{3.626010in}{1.885386in}}%
\pgfpathlineto{\pgfqpoint{3.617427in}{1.897866in}}%
\pgfpathlineto{\pgfqpoint{3.608824in}{1.910973in}}%
\pgfpathlineto{\pgfqpoint{3.600202in}{1.924721in}}%
\pgfpathlineto{\pgfqpoint{3.591560in}{1.939122in}}%
\pgfpathlineto{\pgfqpoint{3.577342in}{1.948267in}}%
\pgfpathlineto{\pgfqpoint{3.563128in}{1.957438in}}%
\pgfpathlineto{\pgfqpoint{3.548918in}{1.966635in}}%
\pgfpathlineto{\pgfqpoint{3.534711in}{1.975860in}}%
\pgfpathlineto{\pgfqpoint{3.543399in}{1.961015in}}%
\pgfpathlineto{\pgfqpoint{3.552066in}{1.946828in}}%
\pgfpathlineto{\pgfqpoint{3.560713in}{1.933286in}}%
\pgfpathlineto{\pgfqpoint{3.569340in}{1.920375in}}%
\pgfpathclose%
\pgfusepath{fill}%
\end{pgfscope}%
\begin{pgfscope}%
\pgfpathrectangle{\pgfqpoint{1.150000in}{0.150000in}}{\pgfqpoint{5.700000in}{5.700000in}}%
\pgfusepath{clip}%
\pgfsetbuttcap%
\pgfsetroundjoin%
\definecolor{currentfill}{rgb}{0.263663,0.237631,0.518762}%
\pgfsetfillcolor{currentfill}%
\pgfsetfillopacity{0.700000}%
\pgfsetlinewidth{0.000000pt}%
\definecolor{currentstroke}{rgb}{0.000000,0.000000,0.000000}%
\pgfsetstrokecolor{currentstroke}%
\pgfsetdash{}{0pt}%
\pgfpathmoveto{\pgfqpoint{3.886866in}{1.712410in}}%
\pgfpathlineto{\pgfqpoint{3.901073in}{1.704558in}}%
\pgfpathlineto{\pgfqpoint{3.915284in}{1.696732in}}%
\pgfpathlineto{\pgfqpoint{3.929500in}{1.688931in}}%
\pgfpathlineto{\pgfqpoint{3.943721in}{1.681155in}}%
\pgfpathlineto{\pgfqpoint{3.935397in}{1.689188in}}%
\pgfpathlineto{\pgfqpoint{3.927060in}{1.697778in}}%
\pgfpathlineto{\pgfqpoint{3.918709in}{1.706937in}}%
\pgfpathlineto{\pgfqpoint{3.910344in}{1.716677in}}%
\pgfpathlineto{\pgfqpoint{3.896086in}{1.724871in}}%
\pgfpathlineto{\pgfqpoint{3.881833in}{1.733090in}}%
\pgfpathlineto{\pgfqpoint{3.867585in}{1.741334in}}%
\pgfpathlineto{\pgfqpoint{3.853340in}{1.749603in}}%
\pgfpathlineto{\pgfqpoint{3.861744in}{1.739439in}}%
\pgfpathlineto{\pgfqpoint{3.870132in}{1.729860in}}%
\pgfpathlineto{\pgfqpoint{3.878506in}{1.720854in}}%
\pgfpathlineto{\pgfqpoint{3.886866in}{1.712410in}}%
\pgfpathclose%
\pgfusepath{fill}%
\end{pgfscope}%
\begin{pgfscope}%
\pgfpathrectangle{\pgfqpoint{1.150000in}{0.150000in}}{\pgfqpoint{5.700000in}{5.700000in}}%
\pgfusepath{clip}%
\pgfsetbuttcap%
\pgfsetroundjoin%
\definecolor{currentfill}{rgb}{0.276022,0.044167,0.370164}%
\pgfsetfillcolor{currentfill}%
\pgfsetfillopacity{0.700000}%
\pgfsetlinewidth{0.000000pt}%
\definecolor{currentstroke}{rgb}{0.000000,0.000000,0.000000}%
\pgfsetstrokecolor{currentstroke}%
\pgfsetdash{}{0pt}%
\pgfpathmoveto{\pgfqpoint{4.905907in}{1.327265in}}%
\pgfpathlineto{\pgfqpoint{4.920346in}{1.322701in}}%
\pgfpathlineto{\pgfqpoint{4.934792in}{1.318160in}}%
\pgfpathlineto{\pgfqpoint{4.949245in}{1.313642in}}%
\pgfpathlineto{\pgfqpoint{4.963705in}{1.309147in}}%
\pgfpathlineto{\pgfqpoint{4.955859in}{1.303268in}}%
\pgfpathlineto{\pgfqpoint{4.948011in}{1.297685in}}%
\pgfpathlineto{\pgfqpoint{4.940161in}{1.292408in}}%
\pgfpathlineto{\pgfqpoint{4.932310in}{1.287445in}}%
\pgfpathlineto{\pgfqpoint{4.917836in}{1.292283in}}%
\pgfpathlineto{\pgfqpoint{4.903370in}{1.297144in}}%
\pgfpathlineto{\pgfqpoint{4.888910in}{1.302028in}}%
\pgfpathlineto{\pgfqpoint{4.874458in}{1.306936in}}%
\pgfpathlineto{\pgfqpoint{4.882324in}{1.311550in}}%
\pgfpathlineto{\pgfqpoint{4.890187in}{1.316483in}}%
\pgfpathlineto{\pgfqpoint{4.898048in}{1.321724in}}%
\pgfpathlineto{\pgfqpoint{4.905907in}{1.327265in}}%
\pgfpathclose%
\pgfusepath{fill}%
\end{pgfscope}%
\begin{pgfscope}%
\pgfpathrectangle{\pgfqpoint{1.150000in}{0.150000in}}{\pgfqpoint{5.700000in}{5.700000in}}%
\pgfusepath{clip}%
\pgfsetbuttcap%
\pgfsetroundjoin%
\definecolor{currentfill}{rgb}{0.274952,0.037752,0.364543}%
\pgfsetfillcolor{currentfill}%
\pgfsetfillopacity{0.700000}%
\pgfsetlinewidth{0.000000pt}%
\definecolor{currentstroke}{rgb}{0.000000,0.000000,0.000000}%
\pgfsetstrokecolor{currentstroke}%
\pgfsetdash{}{0pt}%
\pgfpathmoveto{\pgfqpoint{5.052938in}{1.319020in}}%
\pgfpathlineto{\pgfqpoint{5.067423in}{1.314970in}}%
\pgfpathlineto{\pgfqpoint{5.081916in}{1.310944in}}%
\pgfpathlineto{\pgfqpoint{5.096416in}{1.306941in}}%
\pgfpathlineto{\pgfqpoint{5.088596in}{1.299396in}}%
\pgfpathlineto{\pgfqpoint{5.080774in}{1.292107in}}%
\pgfpathlineto{\pgfqpoint{5.072951in}{1.285082in}}%
\pgfpathlineto{\pgfqpoint{5.065127in}{1.278330in}}%
\pgfpathlineto{\pgfqpoint{5.050616in}{1.282663in}}%
\pgfpathlineto{\pgfqpoint{5.036113in}{1.287019in}}%
\pgfpathlineto{\pgfqpoint{5.021617in}{1.291398in}}%
\pgfpathlineto{\pgfqpoint{5.029449in}{1.297899in}}%
\pgfpathlineto{\pgfqpoint{5.037280in}{1.304676in}}%
\pgfpathlineto{\pgfqpoint{5.045110in}{1.311719in}}%
\pgfpathlineto{\pgfqpoint{5.052938in}{1.319020in}}%
\pgfpathclose%
\pgfusepath{fill}%
\end{pgfscope}%
\begin{pgfscope}%
\pgfpathrectangle{\pgfqpoint{1.150000in}{0.150000in}}{\pgfqpoint{5.700000in}{5.700000in}}%
\pgfusepath{clip}%
\pgfsetbuttcap%
\pgfsetroundjoin%
\definecolor{currentfill}{rgb}{0.177423,0.437527,0.557565}%
\pgfsetfillcolor{currentfill}%
\pgfsetfillopacity{0.700000}%
\pgfsetlinewidth{0.000000pt}%
\definecolor{currentstroke}{rgb}{0.000000,0.000000,0.000000}%
\pgfsetstrokecolor{currentstroke}%
\pgfsetdash{}{0pt}%
\pgfpathmoveto{\pgfqpoint{3.194776in}{2.205544in}}%
\pgfpathlineto{\pgfqpoint{3.208903in}{2.195646in}}%
\pgfpathlineto{\pgfqpoint{3.223033in}{2.185776in}}%
\pgfpathlineto{\pgfqpoint{3.237166in}{2.175937in}}%
\pgfpathlineto{\pgfqpoint{3.251303in}{2.166126in}}%
\pgfpathlineto{\pgfqpoint{3.242344in}{2.183893in}}%
\pgfpathlineto{\pgfqpoint{3.233359in}{2.202369in}}%
\pgfpathlineto{\pgfqpoint{3.224346in}{2.221566in}}%
\pgfpathlineto{\pgfqpoint{3.215306in}{2.241500in}}%
\pgfpathlineto{\pgfqpoint{3.201116in}{2.251770in}}%
\pgfpathlineto{\pgfqpoint{3.186929in}{2.262069in}}%
\pgfpathlineto{\pgfqpoint{3.172745in}{2.272398in}}%
\pgfpathlineto{\pgfqpoint{3.158563in}{2.282756in}}%
\pgfpathlineto{\pgfqpoint{3.167659in}{2.262356in}}%
\pgfpathlineto{\pgfqpoint{3.176726in}{2.242697in}}%
\pgfpathlineto{\pgfqpoint{3.185765in}{2.223764in}}%
\pgfpathlineto{\pgfqpoint{3.194776in}{2.205544in}}%
\pgfpathclose%
\pgfusepath{fill}%
\end{pgfscope}%
\begin{pgfscope}%
\pgfpathrectangle{\pgfqpoint{1.150000in}{0.150000in}}{\pgfqpoint{5.700000in}{5.700000in}}%
\pgfusepath{clip}%
\pgfsetbuttcap%
\pgfsetroundjoin%
\definecolor{currentfill}{rgb}{0.277941,0.056324,0.381191}%
\pgfsetfillcolor{currentfill}%
\pgfsetfillopacity{0.700000}%
\pgfsetlinewidth{0.000000pt}%
\definecolor{currentstroke}{rgb}{0.000000,0.000000,0.000000}%
\pgfsetstrokecolor{currentstroke}%
\pgfsetdash{}{0pt}%
\pgfpathmoveto{\pgfqpoint{4.759082in}{1.347028in}}%
\pgfpathlineto{\pgfqpoint{4.773481in}{1.341935in}}%
\pgfpathlineto{\pgfqpoint{4.787886in}{1.336866in}}%
\pgfpathlineto{\pgfqpoint{4.802297in}{1.331819in}}%
\pgfpathlineto{\pgfqpoint{4.816716in}{1.326796in}}%
\pgfpathlineto{\pgfqpoint{4.808833in}{1.322860in}}%
\pgfpathlineto{\pgfqpoint{4.800947in}{1.319263in}}%
\pgfpathlineto{\pgfqpoint{4.793058in}{1.316015in}}%
\pgfpathlineto{\pgfqpoint{4.785167in}{1.313126in}}%
\pgfpathlineto{\pgfqpoint{4.770732in}{1.318505in}}%
\pgfpathlineto{\pgfqpoint{4.756303in}{1.323908in}}%
\pgfpathlineto{\pgfqpoint{4.741881in}{1.329334in}}%
\pgfpathlineto{\pgfqpoint{4.727466in}{1.334783in}}%
\pgfpathlineto{\pgfqpoint{4.735375in}{1.337311in}}%
\pgfpathlineto{\pgfqpoint{4.743281in}{1.340200in}}%
\pgfpathlineto{\pgfqpoint{4.751183in}{1.343442in}}%
\pgfpathlineto{\pgfqpoint{4.759082in}{1.347028in}}%
\pgfpathclose%
\pgfusepath{fill}%
\end{pgfscope}%
\begin{pgfscope}%
\pgfpathrectangle{\pgfqpoint{1.150000in}{0.150000in}}{\pgfqpoint{5.700000in}{5.700000in}}%
\pgfusepath{clip}%
\pgfsetbuttcap%
\pgfsetroundjoin%
\definecolor{currentfill}{rgb}{0.119512,0.607464,0.540218}%
\pgfsetfillcolor{currentfill}%
\pgfsetfillopacity{0.700000}%
\pgfsetlinewidth{0.000000pt}%
\definecolor{currentstroke}{rgb}{0.000000,0.000000,0.000000}%
\pgfsetstrokecolor{currentstroke}%
\pgfsetdash{}{0pt}%
\pgfpathmoveto{\pgfqpoint{2.649611in}{2.676847in}}%
\pgfpathlineto{\pgfqpoint{2.663714in}{2.665302in}}%
\pgfpathlineto{\pgfqpoint{2.677819in}{2.653794in}}%
\pgfpathlineto{\pgfqpoint{2.691925in}{2.642323in}}%
\pgfpathlineto{\pgfqpoint{2.706032in}{2.630888in}}%
\pgfpathlineto{\pgfqpoint{2.696418in}{2.655853in}}%
\pgfpathlineto{\pgfqpoint{2.686764in}{2.681630in}}%
\pgfpathlineto{\pgfqpoint{2.677070in}{2.708232in}}%
\pgfpathlineto{\pgfqpoint{2.667335in}{2.735677in}}%
\pgfpathlineto{\pgfqpoint{2.653160in}{2.747602in}}%
\pgfpathlineto{\pgfqpoint{2.638987in}{2.759562in}}%
\pgfpathlineto{\pgfqpoint{2.624815in}{2.771560in}}%
\pgfpathlineto{\pgfqpoint{2.610645in}{2.783595in}}%
\pgfpathlineto{\pgfqpoint{2.620449in}{2.755652in}}%
\pgfpathlineto{\pgfqpoint{2.630211in}{2.728558in}}%
\pgfpathlineto{\pgfqpoint{2.639931in}{2.702294in}}%
\pgfpathlineto{\pgfqpoint{2.649611in}{2.676847in}}%
\pgfpathclose%
\pgfusepath{fill}%
\end{pgfscope}%
\begin{pgfscope}%
\pgfpathrectangle{\pgfqpoint{1.150000in}{0.150000in}}{\pgfqpoint{5.700000in}{5.700000in}}%
\pgfusepath{clip}%
\pgfsetbuttcap%
\pgfsetroundjoin%
\definecolor{currentfill}{rgb}{0.283197,0.115680,0.436115}%
\pgfsetfillcolor{currentfill}%
\pgfsetfillopacity{0.700000}%
\pgfsetlinewidth{0.000000pt}%
\definecolor{currentstroke}{rgb}{0.000000,0.000000,0.000000}%
\pgfsetstrokecolor{currentstroke}%
\pgfsetdash{}{0pt}%
\pgfpathmoveto{\pgfqpoint{4.408394in}{1.449784in}}%
\pgfpathlineto{\pgfqpoint{4.422707in}{1.443496in}}%
\pgfpathlineto{\pgfqpoint{4.437027in}{1.437232in}}%
\pgfpathlineto{\pgfqpoint{4.451352in}{1.430992in}}%
\pgfpathlineto{\pgfqpoint{4.465683in}{1.424775in}}%
\pgfpathlineto{\pgfqpoint{4.457669in}{1.425725in}}%
\pgfpathlineto{\pgfqpoint{4.449648in}{1.427111in}}%
\pgfpathlineto{\pgfqpoint{4.441622in}{1.428945in}}%
\pgfpathlineto{\pgfqpoint{4.433589in}{1.431237in}}%
\pgfpathlineto{\pgfqpoint{4.419233in}{1.437839in}}%
\pgfpathlineto{\pgfqpoint{4.404883in}{1.444464in}}%
\pgfpathlineto{\pgfqpoint{4.390539in}{1.451113in}}%
\pgfpathlineto{\pgfqpoint{4.376200in}{1.457786in}}%
\pgfpathlineto{\pgfqpoint{4.384259in}{1.455104in}}%
\pgfpathlineto{\pgfqpoint{4.392310in}{1.452883in}}%
\pgfpathlineto{\pgfqpoint{4.400355in}{1.451113in}}%
\pgfpathlineto{\pgfqpoint{4.408394in}{1.449784in}}%
\pgfpathclose%
\pgfusepath{fill}%
\end{pgfscope}%
\begin{pgfscope}%
\pgfpathrectangle{\pgfqpoint{1.150000in}{0.150000in}}{\pgfqpoint{5.700000in}{5.700000in}}%
\pgfusepath{clip}%
\pgfsetbuttcap%
\pgfsetroundjoin%
\definecolor{currentfill}{rgb}{0.231674,0.318106,0.544834}%
\pgfsetfillcolor{currentfill}%
\pgfsetfillopacity{0.700000}%
\pgfsetlinewidth{0.000000pt}%
\definecolor{currentstroke}{rgb}{0.000000,0.000000,0.000000}%
\pgfsetstrokecolor{currentstroke}%
\pgfsetdash{}{0pt}%
\pgfpathmoveto{\pgfqpoint{3.626010in}{1.885386in}}%
\pgfpathlineto{\pgfqpoint{3.640188in}{1.876705in}}%
\pgfpathlineto{\pgfqpoint{3.654369in}{1.868051in}}%
\pgfpathlineto{\pgfqpoint{3.668554in}{1.859422in}}%
\pgfpathlineto{\pgfqpoint{3.682744in}{1.850820in}}%
\pgfpathlineto{\pgfqpoint{3.674203in}{1.862869in}}%
\pgfpathlineto{\pgfqpoint{3.665644in}{1.875542in}}%
\pgfpathlineto{\pgfqpoint{3.657066in}{1.888851in}}%
\pgfpathlineto{\pgfqpoint{3.648469in}{1.902809in}}%
\pgfpathlineto{\pgfqpoint{3.634236in}{1.911848in}}%
\pgfpathlineto{\pgfqpoint{3.620007in}{1.920913in}}%
\pgfpathlineto{\pgfqpoint{3.605782in}{1.930004in}}%
\pgfpathlineto{\pgfqpoint{3.591560in}{1.939122in}}%
\pgfpathlineto{\pgfqpoint{3.600202in}{1.924721in}}%
\pgfpathlineto{\pgfqpoint{3.608824in}{1.910973in}}%
\pgfpathlineto{\pgfqpoint{3.617427in}{1.897866in}}%
\pgfpathlineto{\pgfqpoint{3.626010in}{1.885386in}}%
\pgfpathclose%
\pgfusepath{fill}%
\end{pgfscope}%
\begin{pgfscope}%
\pgfpathrectangle{\pgfqpoint{1.150000in}{0.150000in}}{\pgfqpoint{5.700000in}{5.700000in}}%
\pgfusepath{clip}%
\pgfsetbuttcap%
\pgfsetroundjoin%
\definecolor{currentfill}{rgb}{0.280894,0.078907,0.402329}%
\pgfsetfillcolor{currentfill}%
\pgfsetfillopacity{0.700000}%
\pgfsetlinewidth{0.000000pt}%
\definecolor{currentstroke}{rgb}{0.000000,0.000000,0.000000}%
\pgfsetstrokecolor{currentstroke}%
\pgfsetdash{}{0pt}%
\pgfpathmoveto{\pgfqpoint{4.612374in}{1.379214in}}%
\pgfpathlineto{\pgfqpoint{4.626738in}{1.373578in}}%
\pgfpathlineto{\pgfqpoint{4.641109in}{1.367966in}}%
\pgfpathlineto{\pgfqpoint{4.655486in}{1.362378in}}%
\pgfpathlineto{\pgfqpoint{4.669869in}{1.356812in}}%
\pgfpathlineto{\pgfqpoint{4.661938in}{1.355022in}}%
\pgfpathlineto{\pgfqpoint{4.654003in}{1.353617in}}%
\pgfpathlineto{\pgfqpoint{4.646064in}{1.352607in}}%
\pgfpathlineto{\pgfqpoint{4.638121in}{1.352002in}}%
\pgfpathlineto{\pgfqpoint{4.623718in}{1.357938in}}%
\pgfpathlineto{\pgfqpoint{4.609321in}{1.363898in}}%
\pgfpathlineto{\pgfqpoint{4.594930in}{1.369880in}}%
\pgfpathlineto{\pgfqpoint{4.580545in}{1.375886in}}%
\pgfpathlineto{\pgfqpoint{4.588509in}{1.376115in}}%
\pgfpathlineto{\pgfqpoint{4.596469in}{1.376752in}}%
\pgfpathlineto{\pgfqpoint{4.604423in}{1.377789in}}%
\pgfpathlineto{\pgfqpoint{4.612374in}{1.379214in}}%
\pgfpathclose%
\pgfusepath{fill}%
\end{pgfscope}%
\begin{pgfscope}%
\pgfpathrectangle{\pgfqpoint{1.150000in}{0.150000in}}{\pgfqpoint{5.700000in}{5.700000in}}%
\pgfusepath{clip}%
\pgfsetbuttcap%
\pgfsetroundjoin%
\definecolor{currentfill}{rgb}{0.266580,0.228262,0.514349}%
\pgfsetfillcolor{currentfill}%
\pgfsetfillopacity{0.700000}%
\pgfsetlinewidth{0.000000pt}%
\definecolor{currentstroke}{rgb}{0.000000,0.000000,0.000000}%
\pgfsetstrokecolor{currentstroke}%
\pgfsetdash{}{0pt}%
\pgfpathmoveto{\pgfqpoint{3.943721in}{1.681155in}}%
\pgfpathlineto{\pgfqpoint{3.957946in}{1.673404in}}%
\pgfpathlineto{\pgfqpoint{3.972176in}{1.665678in}}%
\pgfpathlineto{\pgfqpoint{3.986411in}{1.657976in}}%
\pgfpathlineto{\pgfqpoint{4.000651in}{1.650299in}}%
\pgfpathlineto{\pgfqpoint{3.992362in}{1.657920in}}%
\pgfpathlineto{\pgfqpoint{3.984060in}{1.666095in}}%
\pgfpathlineto{\pgfqpoint{3.975746in}{1.674834in}}%
\pgfpathlineto{\pgfqpoint{3.967419in}{1.684150in}}%
\pgfpathlineto{\pgfqpoint{3.953143in}{1.692245in}}%
\pgfpathlineto{\pgfqpoint{3.938872in}{1.700364in}}%
\pgfpathlineto{\pgfqpoint{3.924606in}{1.708508in}}%
\pgfpathlineto{\pgfqpoint{3.910344in}{1.716677in}}%
\pgfpathlineto{\pgfqpoint{3.918709in}{1.706937in}}%
\pgfpathlineto{\pgfqpoint{3.927060in}{1.697778in}}%
\pgfpathlineto{\pgfqpoint{3.935397in}{1.689188in}}%
\pgfpathlineto{\pgfqpoint{3.943721in}{1.681155in}}%
\pgfpathclose%
\pgfusepath{fill}%
\end{pgfscope}%
\begin{pgfscope}%
\pgfpathrectangle{\pgfqpoint{1.150000in}{0.150000in}}{\pgfqpoint{5.700000in}{5.700000in}}%
\pgfusepath{clip}%
\pgfsetbuttcap%
\pgfsetroundjoin%
\definecolor{currentfill}{rgb}{0.121148,0.592739,0.544641}%
\pgfsetfillcolor{currentfill}%
\pgfsetfillopacity{0.700000}%
\pgfsetlinewidth{0.000000pt}%
\definecolor{currentstroke}{rgb}{0.000000,0.000000,0.000000}%
\pgfsetstrokecolor{currentstroke}%
\pgfsetdash{}{0pt}%
\pgfpathmoveto{\pgfqpoint{2.706032in}{2.630888in}}%
\pgfpathlineto{\pgfqpoint{2.720142in}{2.619488in}}%
\pgfpathlineto{\pgfqpoint{2.734253in}{2.608125in}}%
\pgfpathlineto{\pgfqpoint{2.748366in}{2.596796in}}%
\pgfpathlineto{\pgfqpoint{2.762481in}{2.585504in}}%
\pgfpathlineto{\pgfqpoint{2.752931in}{2.609989in}}%
\pgfpathlineto{\pgfqpoint{2.743343in}{2.635280in}}%
\pgfpathlineto{\pgfqpoint{2.733716in}{2.661392in}}%
\pgfpathlineto{\pgfqpoint{2.724048in}{2.688341in}}%
\pgfpathlineto{\pgfqpoint{2.709868in}{2.700121in}}%
\pgfpathlineto{\pgfqpoint{2.695689in}{2.711937in}}%
\pgfpathlineto{\pgfqpoint{2.681511in}{2.723789in}}%
\pgfpathlineto{\pgfqpoint{2.667335in}{2.735677in}}%
\pgfpathlineto{\pgfqpoint{2.677070in}{2.708232in}}%
\pgfpathlineto{\pgfqpoint{2.686764in}{2.681630in}}%
\pgfpathlineto{\pgfqpoint{2.696418in}{2.655853in}}%
\pgfpathlineto{\pgfqpoint{2.706032in}{2.630888in}}%
\pgfpathclose%
\pgfusepath{fill}%
\end{pgfscope}%
\begin{pgfscope}%
\pgfpathrectangle{\pgfqpoint{1.150000in}{0.150000in}}{\pgfqpoint{5.700000in}{5.700000in}}%
\pgfusepath{clip}%
\pgfsetbuttcap%
\pgfsetroundjoin%
\definecolor{currentfill}{rgb}{0.280868,0.160771,0.472899}%
\pgfsetfillcolor{currentfill}%
\pgfsetfillopacity{0.700000}%
\pgfsetlinewidth{0.000000pt}%
\definecolor{currentstroke}{rgb}{0.000000,0.000000,0.000000}%
\pgfsetstrokecolor{currentstroke}%
\pgfsetdash{}{0pt}%
\pgfpathmoveto{\pgfqpoint{4.204563in}{1.539714in}}%
\pgfpathlineto{\pgfqpoint{4.218836in}{1.532755in}}%
\pgfpathlineto{\pgfqpoint{4.233115in}{1.525820in}}%
\pgfpathlineto{\pgfqpoint{4.247399in}{1.518910in}}%
\pgfpathlineto{\pgfqpoint{4.261689in}{1.512023in}}%
\pgfpathlineto{\pgfqpoint{4.253567in}{1.515967in}}%
\pgfpathlineto{\pgfqpoint{4.245438in}{1.520403in}}%
\pgfpathlineto{\pgfqpoint{4.237299in}{1.525342in}}%
\pgfpathlineto{\pgfqpoint{4.229151in}{1.530794in}}%
\pgfpathlineto{\pgfqpoint{4.214832in}{1.538082in}}%
\pgfpathlineto{\pgfqpoint{4.200518in}{1.545394in}}%
\pgfpathlineto{\pgfqpoint{4.186210in}{1.552729in}}%
\pgfpathlineto{\pgfqpoint{4.171906in}{1.560089in}}%
\pgfpathlineto{\pgfqpoint{4.180085in}{1.554230in}}%
\pgfpathlineto{\pgfqpoint{4.188254in}{1.548889in}}%
\pgfpathlineto{\pgfqpoint{4.196413in}{1.544054in}}%
\pgfpathlineto{\pgfqpoint{4.204563in}{1.539714in}}%
\pgfpathclose%
\pgfusepath{fill}%
\end{pgfscope}%
\begin{pgfscope}%
\pgfpathrectangle{\pgfqpoint{1.150000in}{0.150000in}}{\pgfqpoint{5.700000in}{5.700000in}}%
\pgfusepath{clip}%
\pgfsetbuttcap%
\pgfsetroundjoin%
\definecolor{currentfill}{rgb}{0.182256,0.426184,0.557120}%
\pgfsetfillcolor{currentfill}%
\pgfsetfillopacity{0.700000}%
\pgfsetlinewidth{0.000000pt}%
\definecolor{currentstroke}{rgb}{0.000000,0.000000,0.000000}%
\pgfsetstrokecolor{currentstroke}%
\pgfsetdash{}{0pt}%
\pgfpathmoveto{\pgfqpoint{3.251303in}{2.166126in}}%
\pgfpathlineto{\pgfqpoint{3.265442in}{2.156344in}}%
\pgfpathlineto{\pgfqpoint{3.279584in}{2.146592in}}%
\pgfpathlineto{\pgfqpoint{3.293730in}{2.136868in}}%
\pgfpathlineto{\pgfqpoint{3.307878in}{2.127173in}}%
\pgfpathlineto{\pgfqpoint{3.298971in}{2.144489in}}%
\pgfpathlineto{\pgfqpoint{3.290039in}{2.162508in}}%
\pgfpathlineto{\pgfqpoint{3.281080in}{2.181244in}}%
\pgfpathlineto{\pgfqpoint{3.272094in}{2.200712in}}%
\pgfpathlineto{\pgfqpoint{3.257893in}{2.210865in}}%
\pgfpathlineto{\pgfqpoint{3.243694in}{2.221048in}}%
\pgfpathlineto{\pgfqpoint{3.229499in}{2.231259in}}%
\pgfpathlineto{\pgfqpoint{3.215306in}{2.241500in}}%
\pgfpathlineto{\pgfqpoint{3.224346in}{2.221566in}}%
\pgfpathlineto{\pgfqpoint{3.233359in}{2.202369in}}%
\pgfpathlineto{\pgfqpoint{3.242344in}{2.183893in}}%
\pgfpathlineto{\pgfqpoint{3.251303in}{2.166126in}}%
\pgfpathclose%
\pgfusepath{fill}%
\end{pgfscope}%
\begin{pgfscope}%
\pgfpathrectangle{\pgfqpoint{1.150000in}{0.150000in}}{\pgfqpoint{5.700000in}{5.700000in}}%
\pgfusepath{clip}%
\pgfsetbuttcap%
\pgfsetroundjoin%
\definecolor{currentfill}{rgb}{0.124395,0.578002,0.548287}%
\pgfsetfillcolor{currentfill}%
\pgfsetfillopacity{0.700000}%
\pgfsetlinewidth{0.000000pt}%
\definecolor{currentstroke}{rgb}{0.000000,0.000000,0.000000}%
\pgfsetstrokecolor{currentstroke}%
\pgfsetdash{}{0pt}%
\pgfpathmoveto{\pgfqpoint{2.762481in}{2.585504in}}%
\pgfpathlineto{\pgfqpoint{2.776597in}{2.574246in}}%
\pgfpathlineto{\pgfqpoint{2.790716in}{2.563022in}}%
\pgfpathlineto{\pgfqpoint{2.804836in}{2.551834in}}%
\pgfpathlineto{\pgfqpoint{2.818958in}{2.540680in}}%
\pgfpathlineto{\pgfqpoint{2.809472in}{2.564686in}}%
\pgfpathlineto{\pgfqpoint{2.799949in}{2.589493in}}%
\pgfpathlineto{\pgfqpoint{2.790387in}{2.615116in}}%
\pgfpathlineto{\pgfqpoint{2.780787in}{2.641570in}}%
\pgfpathlineto{\pgfqpoint{2.766600in}{2.653210in}}%
\pgfpathlineto{\pgfqpoint{2.752414in}{2.664885in}}%
\pgfpathlineto{\pgfqpoint{2.738231in}{2.676596in}}%
\pgfpathlineto{\pgfqpoint{2.724048in}{2.688341in}}%
\pgfpathlineto{\pgfqpoint{2.733716in}{2.661392in}}%
\pgfpathlineto{\pgfqpoint{2.743343in}{2.635280in}}%
\pgfpathlineto{\pgfqpoint{2.752931in}{2.609989in}}%
\pgfpathlineto{\pgfqpoint{2.762481in}{2.585504in}}%
\pgfpathclose%
\pgfusepath{fill}%
\end{pgfscope}%
\begin{pgfscope}%
\pgfpathrectangle{\pgfqpoint{1.150000in}{0.150000in}}{\pgfqpoint{5.700000in}{5.700000in}}%
\pgfusepath{clip}%
\pgfsetbuttcap%
\pgfsetroundjoin%
\definecolor{currentfill}{rgb}{0.276022,0.044167,0.370164}%
\pgfsetfillcolor{currentfill}%
\pgfsetfillopacity{0.700000}%
\pgfsetlinewidth{0.000000pt}%
\definecolor{currentstroke}{rgb}{0.000000,0.000000,0.000000}%
\pgfsetstrokecolor{currentstroke}%
\pgfsetdash{}{0pt}%
\pgfpathmoveto{\pgfqpoint{4.963705in}{1.309147in}}%
\pgfpathlineto{\pgfqpoint{4.978172in}{1.304675in}}%
\pgfpathlineto{\pgfqpoint{4.992647in}{1.300226in}}%
\pgfpathlineto{\pgfqpoint{5.007128in}{1.295801in}}%
\pgfpathlineto{\pgfqpoint{5.021617in}{1.291398in}}%
\pgfpathlineto{\pgfqpoint{5.013783in}{1.285181in}}%
\pgfpathlineto{\pgfqpoint{5.005948in}{1.279257in}}%
\pgfpathlineto{\pgfqpoint{4.998111in}{1.273635in}}%
\pgfpathlineto{\pgfqpoint{4.990273in}{1.268324in}}%
\pgfpathlineto{\pgfqpoint{4.975772in}{1.273069in}}%
\pgfpathlineto{\pgfqpoint{4.961277in}{1.277838in}}%
\pgfpathlineto{\pgfqpoint{4.946790in}{1.282630in}}%
\pgfpathlineto{\pgfqpoint{4.932310in}{1.287445in}}%
\pgfpathlineto{\pgfqpoint{4.940161in}{1.292408in}}%
\pgfpathlineto{\pgfqpoint{4.948011in}{1.297685in}}%
\pgfpathlineto{\pgfqpoint{4.955859in}{1.303268in}}%
\pgfpathlineto{\pgfqpoint{4.963705in}{1.309147in}}%
\pgfpathclose%
\pgfusepath{fill}%
\end{pgfscope}%
\begin{pgfscope}%
\pgfpathrectangle{\pgfqpoint{1.150000in}{0.150000in}}{\pgfqpoint{5.700000in}{5.700000in}}%
\pgfusepath{clip}%
\pgfsetbuttcap%
\pgfsetroundjoin%
\definecolor{currentfill}{rgb}{0.237441,0.305202,0.541921}%
\pgfsetfillcolor{currentfill}%
\pgfsetfillopacity{0.700000}%
\pgfsetlinewidth{0.000000pt}%
\definecolor{currentstroke}{rgb}{0.000000,0.000000,0.000000}%
\pgfsetstrokecolor{currentstroke}%
\pgfsetdash{}{0pt}%
\pgfpathmoveto{\pgfqpoint{3.682744in}{1.850820in}}%
\pgfpathlineto{\pgfqpoint{3.696937in}{1.842243in}}%
\pgfpathlineto{\pgfqpoint{3.711135in}{1.833693in}}%
\pgfpathlineto{\pgfqpoint{3.725336in}{1.825169in}}%
\pgfpathlineto{\pgfqpoint{3.739542in}{1.816670in}}%
\pgfpathlineto{\pgfqpoint{3.731043in}{1.828290in}}%
\pgfpathlineto{\pgfqpoint{3.722526in}{1.840529in}}%
\pgfpathlineto{\pgfqpoint{3.713992in}{1.853400in}}%
\pgfpathlineto{\pgfqpoint{3.705440in}{1.866915in}}%
\pgfpathlineto{\pgfqpoint{3.691192in}{1.875849in}}%
\pgfpathlineto{\pgfqpoint{3.676947in}{1.884810in}}%
\pgfpathlineto{\pgfqpoint{3.662706in}{1.893796in}}%
\pgfpathlineto{\pgfqpoint{3.648469in}{1.902809in}}%
\pgfpathlineto{\pgfqpoint{3.657066in}{1.888851in}}%
\pgfpathlineto{\pgfqpoint{3.665644in}{1.875542in}}%
\pgfpathlineto{\pgfqpoint{3.674203in}{1.862869in}}%
\pgfpathlineto{\pgfqpoint{3.682744in}{1.850820in}}%
\pgfpathclose%
\pgfusepath{fill}%
\end{pgfscope}%
\begin{pgfscope}%
\pgfpathrectangle{\pgfqpoint{1.150000in}{0.150000in}}{\pgfqpoint{5.700000in}{5.700000in}}%
\pgfusepath{clip}%
\pgfsetbuttcap%
\pgfsetroundjoin%
\definecolor{currentfill}{rgb}{0.277941,0.056324,0.381191}%
\pgfsetfillcolor{currentfill}%
\pgfsetfillopacity{0.700000}%
\pgfsetlinewidth{0.000000pt}%
\definecolor{currentstroke}{rgb}{0.000000,0.000000,0.000000}%
\pgfsetstrokecolor{currentstroke}%
\pgfsetdash{}{0pt}%
\pgfpathmoveto{\pgfqpoint{4.816716in}{1.326796in}}%
\pgfpathlineto{\pgfqpoint{4.831141in}{1.321796in}}%
\pgfpathlineto{\pgfqpoint{4.845573in}{1.316820in}}%
\pgfpathlineto{\pgfqpoint{4.860012in}{1.311866in}}%
\pgfpathlineto{\pgfqpoint{4.874458in}{1.306936in}}%
\pgfpathlineto{\pgfqpoint{4.866590in}{1.302648in}}%
\pgfpathlineto{\pgfqpoint{4.858720in}{1.298696in}}%
\pgfpathlineto{\pgfqpoint{4.850847in}{1.295089in}}%
\pgfpathlineto{\pgfqpoint{4.842973in}{1.291838in}}%
\pgfpathlineto{\pgfqpoint{4.828511in}{1.297125in}}%
\pgfpathlineto{\pgfqpoint{4.814056in}{1.302436in}}%
\pgfpathlineto{\pgfqpoint{4.799608in}{1.307769in}}%
\pgfpathlineto{\pgfqpoint{4.785167in}{1.313126in}}%
\pgfpathlineto{\pgfqpoint{4.793058in}{1.316015in}}%
\pgfpathlineto{\pgfqpoint{4.800947in}{1.319263in}}%
\pgfpathlineto{\pgfqpoint{4.808833in}{1.322860in}}%
\pgfpathlineto{\pgfqpoint{4.816716in}{1.326796in}}%
\pgfpathclose%
\pgfusepath{fill}%
\end{pgfscope}%
\begin{pgfscope}%
\pgfpathrectangle{\pgfqpoint{1.150000in}{0.150000in}}{\pgfqpoint{5.700000in}{5.700000in}}%
\pgfusepath{clip}%
\pgfsetbuttcap%
\pgfsetroundjoin%
\definecolor{currentfill}{rgb}{0.187231,0.414746,0.556547}%
\pgfsetfillcolor{currentfill}%
\pgfsetfillopacity{0.700000}%
\pgfsetlinewidth{0.000000pt}%
\definecolor{currentstroke}{rgb}{0.000000,0.000000,0.000000}%
\pgfsetstrokecolor{currentstroke}%
\pgfsetdash{}{0pt}%
\pgfpathmoveto{\pgfqpoint{3.307878in}{2.127173in}}%
\pgfpathlineto{\pgfqpoint{3.322030in}{2.117506in}}%
\pgfpathlineto{\pgfqpoint{3.336185in}{2.107868in}}%
\pgfpathlineto{\pgfqpoint{3.350344in}{2.098258in}}%
\pgfpathlineto{\pgfqpoint{3.364505in}{2.088677in}}%
\pgfpathlineto{\pgfqpoint{3.355649in}{2.105542in}}%
\pgfpathlineto{\pgfqpoint{3.346768in}{2.123105in}}%
\pgfpathlineto{\pgfqpoint{3.337862in}{2.141381in}}%
\pgfpathlineto{\pgfqpoint{3.328930in}{2.160384in}}%
\pgfpathlineto{\pgfqpoint{3.314717in}{2.170423in}}%
\pgfpathlineto{\pgfqpoint{3.300506in}{2.180491in}}%
\pgfpathlineto{\pgfqpoint{3.286299in}{2.190587in}}%
\pgfpathlineto{\pgfqpoint{3.272094in}{2.200712in}}%
\pgfpathlineto{\pgfqpoint{3.281080in}{2.181244in}}%
\pgfpathlineto{\pgfqpoint{3.290039in}{2.162508in}}%
\pgfpathlineto{\pgfqpoint{3.298971in}{2.144489in}}%
\pgfpathlineto{\pgfqpoint{3.307878in}{2.127173in}}%
\pgfpathclose%
\pgfusepath{fill}%
\end{pgfscope}%
\begin{pgfscope}%
\pgfpathrectangle{\pgfqpoint{1.150000in}{0.150000in}}{\pgfqpoint{5.700000in}{5.700000in}}%
\pgfusepath{clip}%
\pgfsetbuttcap%
\pgfsetroundjoin%
\definecolor{currentfill}{rgb}{0.283091,0.110553,0.431554}%
\pgfsetfillcolor{currentfill}%
\pgfsetfillopacity{0.700000}%
\pgfsetlinewidth{0.000000pt}%
\definecolor{currentstroke}{rgb}{0.000000,0.000000,0.000000}%
\pgfsetstrokecolor{currentstroke}%
\pgfsetdash{}{0pt}%
\pgfpathmoveto{\pgfqpoint{4.465683in}{1.424775in}}%
\pgfpathlineto{\pgfqpoint{4.480020in}{1.418582in}}%
\pgfpathlineto{\pgfqpoint{4.494363in}{1.412412in}}%
\pgfpathlineto{\pgfqpoint{4.508711in}{1.406266in}}%
\pgfpathlineto{\pgfqpoint{4.523066in}{1.400143in}}%
\pgfpathlineto{\pgfqpoint{4.515075in}{1.400713in}}%
\pgfpathlineto{\pgfqpoint{4.507079in}{1.401716in}}%
\pgfpathlineto{\pgfqpoint{4.499076in}{1.403163in}}%
\pgfpathlineto{\pgfqpoint{4.491069in}{1.405064in}}%
\pgfpathlineto{\pgfqpoint{4.476690in}{1.411572in}}%
\pgfpathlineto{\pgfqpoint{4.462317in}{1.418103in}}%
\pgfpathlineto{\pgfqpoint{4.447950in}{1.424658in}}%
\pgfpathlineto{\pgfqpoint{4.433589in}{1.431237in}}%
\pgfpathlineto{\pgfqpoint{4.441622in}{1.428945in}}%
\pgfpathlineto{\pgfqpoint{4.449648in}{1.427111in}}%
\pgfpathlineto{\pgfqpoint{4.457669in}{1.425725in}}%
\pgfpathlineto{\pgfqpoint{4.465683in}{1.424775in}}%
\pgfpathclose%
\pgfusepath{fill}%
\end{pgfscope}%
\begin{pgfscope}%
\pgfpathrectangle{\pgfqpoint{1.150000in}{0.150000in}}{\pgfqpoint{5.700000in}{5.700000in}}%
\pgfusepath{clip}%
\pgfsetbuttcap%
\pgfsetroundjoin%
\definecolor{currentfill}{rgb}{0.269308,0.218818,0.509577}%
\pgfsetfillcolor{currentfill}%
\pgfsetfillopacity{0.700000}%
\pgfsetlinewidth{0.000000pt}%
\definecolor{currentstroke}{rgb}{0.000000,0.000000,0.000000}%
\pgfsetstrokecolor{currentstroke}%
\pgfsetdash{}{0pt}%
\pgfpathmoveto{\pgfqpoint{4.000651in}{1.650299in}}%
\pgfpathlineto{\pgfqpoint{4.014895in}{1.642647in}}%
\pgfpathlineto{\pgfqpoint{4.029144in}{1.635020in}}%
\pgfpathlineto{\pgfqpoint{4.043398in}{1.627417in}}%
\pgfpathlineto{\pgfqpoint{4.057657in}{1.619838in}}%
\pgfpathlineto{\pgfqpoint{4.049402in}{1.627048in}}%
\pgfpathlineto{\pgfqpoint{4.041136in}{1.634807in}}%
\pgfpathlineto{\pgfqpoint{4.032858in}{1.643127in}}%
\pgfpathlineto{\pgfqpoint{4.024567in}{1.652020in}}%
\pgfpathlineto{\pgfqpoint{4.010273in}{1.660016in}}%
\pgfpathlineto{\pgfqpoint{3.995983in}{1.668036in}}%
\pgfpathlineto{\pgfqpoint{3.981699in}{1.676081in}}%
\pgfpathlineto{\pgfqpoint{3.967419in}{1.684150in}}%
\pgfpathlineto{\pgfqpoint{3.975746in}{1.674834in}}%
\pgfpathlineto{\pgfqpoint{3.984060in}{1.666095in}}%
\pgfpathlineto{\pgfqpoint{3.992362in}{1.657920in}}%
\pgfpathlineto{\pgfqpoint{4.000651in}{1.650299in}}%
\pgfpathclose%
\pgfusepath{fill}%
\end{pgfscope}%
\begin{pgfscope}%
\pgfpathrectangle{\pgfqpoint{1.150000in}{0.150000in}}{\pgfqpoint{5.700000in}{5.700000in}}%
\pgfusepath{clip}%
\pgfsetbuttcap%
\pgfsetroundjoin%
\definecolor{currentfill}{rgb}{0.281412,0.155834,0.469201}%
\pgfsetfillcolor{currentfill}%
\pgfsetfillopacity{0.700000}%
\pgfsetlinewidth{0.000000pt}%
\definecolor{currentstroke}{rgb}{0.000000,0.000000,0.000000}%
\pgfsetstrokecolor{currentstroke}%
\pgfsetdash{}{0pt}%
\pgfpathmoveto{\pgfqpoint{4.261689in}{1.512023in}}%
\pgfpathlineto{\pgfqpoint{4.275984in}{1.505160in}}%
\pgfpathlineto{\pgfqpoint{4.290284in}{1.498321in}}%
\pgfpathlineto{\pgfqpoint{4.304589in}{1.491505in}}%
\pgfpathlineto{\pgfqpoint{4.318900in}{1.484714in}}%
\pgfpathlineto{\pgfqpoint{4.310807in}{1.488264in}}%
\pgfpathlineto{\pgfqpoint{4.302706in}{1.492301in}}%
\pgfpathlineto{\pgfqpoint{4.294597in}{1.496837in}}%
\pgfpathlineto{\pgfqpoint{4.286480in}{1.501883in}}%
\pgfpathlineto{\pgfqpoint{4.272140in}{1.509075in}}%
\pgfpathlineto{\pgfqpoint{4.257805in}{1.516291in}}%
\pgfpathlineto{\pgfqpoint{4.243476in}{1.523531in}}%
\pgfpathlineto{\pgfqpoint{4.229151in}{1.530794in}}%
\pgfpathlineto{\pgfqpoint{4.237299in}{1.525342in}}%
\pgfpathlineto{\pgfqpoint{4.245438in}{1.520403in}}%
\pgfpathlineto{\pgfqpoint{4.253567in}{1.515967in}}%
\pgfpathlineto{\pgfqpoint{4.261689in}{1.512023in}}%
\pgfpathclose%
\pgfusepath{fill}%
\end{pgfscope}%
\begin{pgfscope}%
\pgfpathrectangle{\pgfqpoint{1.150000in}{0.150000in}}{\pgfqpoint{5.700000in}{5.700000in}}%
\pgfusepath{clip}%
\pgfsetbuttcap%
\pgfsetroundjoin%
\definecolor{currentfill}{rgb}{0.127568,0.566949,0.550556}%
\pgfsetfillcolor{currentfill}%
\pgfsetfillopacity{0.700000}%
\pgfsetlinewidth{0.000000pt}%
\definecolor{currentstroke}{rgb}{0.000000,0.000000,0.000000}%
\pgfsetstrokecolor{currentstroke}%
\pgfsetdash{}{0pt}%
\pgfpathmoveto{\pgfqpoint{2.818958in}{2.540680in}}%
\pgfpathlineto{\pgfqpoint{2.833083in}{2.529560in}}%
\pgfpathlineto{\pgfqpoint{2.847209in}{2.518473in}}%
\pgfpathlineto{\pgfqpoint{2.861337in}{2.507421in}}%
\pgfpathlineto{\pgfqpoint{2.875468in}{2.496402in}}%
\pgfpathlineto{\pgfqpoint{2.866044in}{2.519930in}}%
\pgfpathlineto{\pgfqpoint{2.856584in}{2.544254in}}%
\pgfpathlineto{\pgfqpoint{2.847088in}{2.569389in}}%
\pgfpathlineto{\pgfqpoint{2.837554in}{2.595349in}}%
\pgfpathlineto{\pgfqpoint{2.823359in}{2.606853in}}%
\pgfpathlineto{\pgfqpoint{2.809167in}{2.618391in}}%
\pgfpathlineto{\pgfqpoint{2.794976in}{2.629963in}}%
\pgfpathlineto{\pgfqpoint{2.780787in}{2.641570in}}%
\pgfpathlineto{\pgfqpoint{2.790387in}{2.615116in}}%
\pgfpathlineto{\pgfqpoint{2.799949in}{2.589493in}}%
\pgfpathlineto{\pgfqpoint{2.809472in}{2.564686in}}%
\pgfpathlineto{\pgfqpoint{2.818958in}{2.540680in}}%
\pgfpathclose%
\pgfusepath{fill}%
\end{pgfscope}%
\begin{pgfscope}%
\pgfpathrectangle{\pgfqpoint{1.150000in}{0.150000in}}{\pgfqpoint{5.700000in}{5.700000in}}%
\pgfusepath{clip}%
\pgfsetbuttcap%
\pgfsetroundjoin%
\definecolor{currentfill}{rgb}{0.280267,0.073417,0.397163}%
\pgfsetfillcolor{currentfill}%
\pgfsetfillopacity{0.700000}%
\pgfsetlinewidth{0.000000pt}%
\definecolor{currentstroke}{rgb}{0.000000,0.000000,0.000000}%
\pgfsetstrokecolor{currentstroke}%
\pgfsetdash{}{0pt}%
\pgfpathmoveto{\pgfqpoint{4.669869in}{1.356812in}}%
\pgfpathlineto{\pgfqpoint{4.684259in}{1.351270in}}%
\pgfpathlineto{\pgfqpoint{4.698655in}{1.345751in}}%
\pgfpathlineto{\pgfqpoint{4.713057in}{1.340256in}}%
\pgfpathlineto{\pgfqpoint{4.727466in}{1.334783in}}%
\pgfpathlineto{\pgfqpoint{4.719554in}{1.332628in}}%
\pgfpathlineto{\pgfqpoint{4.711638in}{1.330854in}}%
\pgfpathlineto{\pgfqpoint{4.703719in}{1.329471in}}%
\pgfpathlineto{\pgfqpoint{4.695797in}{1.328491in}}%
\pgfpathlineto{\pgfqpoint{4.681368in}{1.334334in}}%
\pgfpathlineto{\pgfqpoint{4.666947in}{1.340200in}}%
\pgfpathlineto{\pgfqpoint{4.652531in}{1.346090in}}%
\pgfpathlineto{\pgfqpoint{4.638121in}{1.352002in}}%
\pgfpathlineto{\pgfqpoint{4.646064in}{1.352607in}}%
\pgfpathlineto{\pgfqpoint{4.654003in}{1.353617in}}%
\pgfpathlineto{\pgfqpoint{4.661938in}{1.355022in}}%
\pgfpathlineto{\pgfqpoint{4.669869in}{1.356812in}}%
\pgfpathclose%
\pgfusepath{fill}%
\end{pgfscope}%
\begin{pgfscope}%
\pgfpathrectangle{\pgfqpoint{1.150000in}{0.150000in}}{\pgfqpoint{5.700000in}{5.700000in}}%
\pgfusepath{clip}%
\pgfsetbuttcap%
\pgfsetroundjoin%
\definecolor{currentfill}{rgb}{0.555484,0.840254,0.269281}%
\pgfsetfillcolor{currentfill}%
\pgfsetfillopacity{0.700000}%
\pgfsetlinewidth{0.000000pt}%
\definecolor{currentstroke}{rgb}{0.000000,0.000000,0.000000}%
\pgfsetstrokecolor{currentstroke}%
\pgfsetdash{}{0pt}%
\pgfpathmoveto{\pgfqpoint{1.931153in}{3.411508in}}%
\pgfpathlineto{\pgfqpoint{1.945314in}{3.397264in}}%
\pgfpathlineto{\pgfqpoint{1.959474in}{3.383078in}}%
\pgfpathlineto{\pgfqpoint{1.973634in}{3.368948in}}%
\pgfpathlineto{\pgfqpoint{1.987792in}{3.354875in}}%
\pgfpathlineto{\pgfqpoint{1.977101in}{3.389328in}}%
\pgfpathlineto{\pgfqpoint{1.966351in}{3.424725in}}%
\pgfpathlineto{\pgfqpoint{1.955540in}{3.461083in}}%
\pgfpathlineto{\pgfqpoint{1.944668in}{3.498421in}}%
\pgfpathlineto{\pgfqpoint{1.930425in}{3.513029in}}%
\pgfpathlineto{\pgfqpoint{1.916180in}{3.527694in}}%
\pgfpathlineto{\pgfqpoint{1.901934in}{3.542416in}}%
\pgfpathlineto{\pgfqpoint{1.887687in}{3.557196in}}%
\pgfpathlineto{\pgfqpoint{1.898646in}{3.519314in}}%
\pgfpathlineto{\pgfqpoint{1.909543in}{3.482417in}}%
\pgfpathlineto{\pgfqpoint{1.920378in}{3.446487in}}%
\pgfpathlineto{\pgfqpoint{1.931153in}{3.411508in}}%
\pgfpathclose%
\pgfusepath{fill}%
\end{pgfscope}%
\begin{pgfscope}%
\pgfpathrectangle{\pgfqpoint{1.150000in}{0.150000in}}{\pgfqpoint{5.700000in}{5.700000in}}%
\pgfusepath{clip}%
\pgfsetbuttcap%
\pgfsetroundjoin%
\definecolor{currentfill}{rgb}{0.190631,0.407061,0.556089}%
\pgfsetfillcolor{currentfill}%
\pgfsetfillopacity{0.700000}%
\pgfsetlinewidth{0.000000pt}%
\definecolor{currentstroke}{rgb}{0.000000,0.000000,0.000000}%
\pgfsetstrokecolor{currentstroke}%
\pgfsetdash{}{0pt}%
\pgfpathmoveto{\pgfqpoint{3.364505in}{2.088677in}}%
\pgfpathlineto{\pgfqpoint{3.378670in}{2.079124in}}%
\pgfpathlineto{\pgfqpoint{3.392838in}{2.069598in}}%
\pgfpathlineto{\pgfqpoint{3.407010in}{2.060101in}}%
\pgfpathlineto{\pgfqpoint{3.421185in}{2.050631in}}%
\pgfpathlineto{\pgfqpoint{3.412379in}{2.067045in}}%
\pgfpathlineto{\pgfqpoint{3.403549in}{2.084154in}}%
\pgfpathlineto{\pgfqpoint{3.394695in}{2.101971in}}%
\pgfpathlineto{\pgfqpoint{3.385816in}{2.120509in}}%
\pgfpathlineto{\pgfqpoint{3.371590in}{2.130436in}}%
\pgfpathlineto{\pgfqpoint{3.357367in}{2.140390in}}%
\pgfpathlineto{\pgfqpoint{3.343147in}{2.150373in}}%
\pgfpathlineto{\pgfqpoint{3.328930in}{2.160384in}}%
\pgfpathlineto{\pgfqpoint{3.337862in}{2.141381in}}%
\pgfpathlineto{\pgfqpoint{3.346768in}{2.123105in}}%
\pgfpathlineto{\pgfqpoint{3.355649in}{2.105542in}}%
\pgfpathlineto{\pgfqpoint{3.364505in}{2.088677in}}%
\pgfpathclose%
\pgfusepath{fill}%
\end{pgfscope}%
\begin{pgfscope}%
\pgfpathrectangle{\pgfqpoint{1.150000in}{0.150000in}}{\pgfqpoint{5.700000in}{5.700000in}}%
\pgfusepath{clip}%
\pgfsetbuttcap%
\pgfsetroundjoin%
\definecolor{currentfill}{rgb}{0.241237,0.296485,0.539709}%
\pgfsetfillcolor{currentfill}%
\pgfsetfillopacity{0.700000}%
\pgfsetlinewidth{0.000000pt}%
\definecolor{currentstroke}{rgb}{0.000000,0.000000,0.000000}%
\pgfsetstrokecolor{currentstroke}%
\pgfsetdash{}{0pt}%
\pgfpathmoveto{\pgfqpoint{3.739542in}{1.816670in}}%
\pgfpathlineto{\pgfqpoint{3.753752in}{1.808197in}}%
\pgfpathlineto{\pgfqpoint{3.767966in}{1.799750in}}%
\pgfpathlineto{\pgfqpoint{3.782184in}{1.791329in}}%
\pgfpathlineto{\pgfqpoint{3.796407in}{1.782933in}}%
\pgfpathlineto{\pgfqpoint{3.787949in}{1.794123in}}%
\pgfpathlineto{\pgfqpoint{3.779474in}{1.805929in}}%
\pgfpathlineto{\pgfqpoint{3.770983in}{1.818362in}}%
\pgfpathlineto{\pgfqpoint{3.762474in}{1.831435in}}%
\pgfpathlineto{\pgfqpoint{3.748210in}{1.840266in}}%
\pgfpathlineto{\pgfqpoint{3.733949in}{1.849123in}}%
\pgfpathlineto{\pgfqpoint{3.719693in}{1.858006in}}%
\pgfpathlineto{\pgfqpoint{3.705440in}{1.866915in}}%
\pgfpathlineto{\pgfqpoint{3.713992in}{1.853400in}}%
\pgfpathlineto{\pgfqpoint{3.722526in}{1.840529in}}%
\pgfpathlineto{\pgfqpoint{3.731043in}{1.828290in}}%
\pgfpathlineto{\pgfqpoint{3.739542in}{1.816670in}}%
\pgfpathclose%
\pgfusepath{fill}%
\end{pgfscope}%
\begin{pgfscope}%
\pgfpathrectangle{\pgfqpoint{1.150000in}{0.150000in}}{\pgfqpoint{5.700000in}{5.700000in}}%
\pgfusepath{clip}%
\pgfsetbuttcap%
\pgfsetroundjoin%
\definecolor{currentfill}{rgb}{0.506271,0.828786,0.300362}%
\pgfsetfillcolor{currentfill}%
\pgfsetfillopacity{0.700000}%
\pgfsetlinewidth{0.000000pt}%
\definecolor{currentstroke}{rgb}{0.000000,0.000000,0.000000}%
\pgfsetstrokecolor{currentstroke}%
\pgfsetdash{}{0pt}%
\pgfpathmoveto{\pgfqpoint{1.987792in}{3.354875in}}%
\pgfpathlineto{\pgfqpoint{2.001949in}{3.340857in}}%
\pgfpathlineto{\pgfqpoint{2.016106in}{3.326893in}}%
\pgfpathlineto{\pgfqpoint{2.030262in}{3.312985in}}%
\pgfpathlineto{\pgfqpoint{2.044416in}{3.299130in}}%
\pgfpathlineto{\pgfqpoint{2.033808in}{3.333059in}}%
\pgfpathlineto{\pgfqpoint{2.023141in}{3.367927in}}%
\pgfpathlineto{\pgfqpoint{2.012416in}{3.403750in}}%
\pgfpathlineto{\pgfqpoint{2.001630in}{3.440545in}}%
\pgfpathlineto{\pgfqpoint{1.987391in}{3.454932in}}%
\pgfpathlineto{\pgfqpoint{1.973151in}{3.469373in}}%
\pgfpathlineto{\pgfqpoint{1.958910in}{3.483870in}}%
\pgfpathlineto{\pgfqpoint{1.944668in}{3.498421in}}%
\pgfpathlineto{\pgfqpoint{1.955540in}{3.461083in}}%
\pgfpathlineto{\pgfqpoint{1.966351in}{3.424725in}}%
\pgfpathlineto{\pgfqpoint{1.977101in}{3.389328in}}%
\pgfpathlineto{\pgfqpoint{1.987792in}{3.354875in}}%
\pgfpathclose%
\pgfusepath{fill}%
\end{pgfscope}%
\begin{pgfscope}%
\pgfpathrectangle{\pgfqpoint{1.150000in}{0.150000in}}{\pgfqpoint{5.700000in}{5.700000in}}%
\pgfusepath{clip}%
\pgfsetbuttcap%
\pgfsetroundjoin%
\definecolor{currentfill}{rgb}{0.132444,0.552216,0.553018}%
\pgfsetfillcolor{currentfill}%
\pgfsetfillopacity{0.700000}%
\pgfsetlinewidth{0.000000pt}%
\definecolor{currentstroke}{rgb}{0.000000,0.000000,0.000000}%
\pgfsetstrokecolor{currentstroke}%
\pgfsetdash{}{0pt}%
\pgfpathmoveto{\pgfqpoint{2.875468in}{2.496402in}}%
\pgfpathlineto{\pgfqpoint{2.889600in}{2.485416in}}%
\pgfpathlineto{\pgfqpoint{2.903735in}{2.474463in}}%
\pgfpathlineto{\pgfqpoint{2.917872in}{2.463543in}}%
\pgfpathlineto{\pgfqpoint{2.932011in}{2.452656in}}%
\pgfpathlineto{\pgfqpoint{2.922648in}{2.475708in}}%
\pgfpathlineto{\pgfqpoint{2.913251in}{2.499550in}}%
\pgfpathlineto{\pgfqpoint{2.903819in}{2.524197in}}%
\pgfpathlineto{\pgfqpoint{2.894350in}{2.549666in}}%
\pgfpathlineto{\pgfqpoint{2.880148in}{2.561037in}}%
\pgfpathlineto{\pgfqpoint{2.865948in}{2.572441in}}%
\pgfpathlineto{\pgfqpoint{2.851750in}{2.583878in}}%
\pgfpathlineto{\pgfqpoint{2.837554in}{2.595349in}}%
\pgfpathlineto{\pgfqpoint{2.847088in}{2.569389in}}%
\pgfpathlineto{\pgfqpoint{2.856584in}{2.544254in}}%
\pgfpathlineto{\pgfqpoint{2.866044in}{2.519930in}}%
\pgfpathlineto{\pgfqpoint{2.875468in}{2.496402in}}%
\pgfpathclose%
\pgfusepath{fill}%
\end{pgfscope}%
\begin{pgfscope}%
\pgfpathrectangle{\pgfqpoint{1.150000in}{0.150000in}}{\pgfqpoint{5.700000in}{5.700000in}}%
\pgfusepath{clip}%
\pgfsetbuttcap%
\pgfsetroundjoin%
\definecolor{currentfill}{rgb}{0.274952,0.037752,0.364543}%
\pgfsetfillcolor{currentfill}%
\pgfsetfillopacity{0.700000}%
\pgfsetlinewidth{0.000000pt}%
\definecolor{currentstroke}{rgb}{0.000000,0.000000,0.000000}%
\pgfsetstrokecolor{currentstroke}%
\pgfsetdash{}{0pt}%
\pgfpathmoveto{\pgfqpoint{5.021617in}{1.291398in}}%
\pgfpathlineto{\pgfqpoint{5.036113in}{1.287019in}}%
\pgfpathlineto{\pgfqpoint{5.050616in}{1.282663in}}%
\pgfpathlineto{\pgfqpoint{5.065127in}{1.278330in}}%
\pgfpathlineto{\pgfqpoint{5.057302in}{1.271859in}}%
\pgfpathlineto{\pgfqpoint{5.049475in}{1.265679in}}%
\pgfpathlineto{\pgfqpoint{5.041648in}{1.259798in}}%
\pgfpathlineto{\pgfqpoint{5.033819in}{1.254225in}}%
\pgfpathlineto{\pgfqpoint{5.019297in}{1.258901in}}%
\pgfpathlineto{\pgfqpoint{5.004781in}{1.263601in}}%
\pgfpathlineto{\pgfqpoint{4.990273in}{1.268324in}}%
\pgfpathlineto{\pgfqpoint{4.998111in}{1.273635in}}%
\pgfpathlineto{\pgfqpoint{5.005948in}{1.279257in}}%
\pgfpathlineto{\pgfqpoint{5.013783in}{1.285181in}}%
\pgfpathlineto{\pgfqpoint{5.021617in}{1.291398in}}%
\pgfpathclose%
\pgfusepath{fill}%
\end{pgfscope}%
\begin{pgfscope}%
\pgfpathrectangle{\pgfqpoint{1.150000in}{0.150000in}}{\pgfqpoint{5.700000in}{5.700000in}}%
\pgfusepath{clip}%
\pgfsetbuttcap%
\pgfsetroundjoin%
\definecolor{currentfill}{rgb}{0.271828,0.209303,0.504434}%
\pgfsetfillcolor{currentfill}%
\pgfsetfillopacity{0.700000}%
\pgfsetlinewidth{0.000000pt}%
\definecolor{currentstroke}{rgb}{0.000000,0.000000,0.000000}%
\pgfsetstrokecolor{currentstroke}%
\pgfsetdash{}{0pt}%
\pgfpathmoveto{\pgfqpoint{4.057657in}{1.619838in}}%
\pgfpathlineto{\pgfqpoint{4.071921in}{1.612284in}}%
\pgfpathlineto{\pgfqpoint{4.086190in}{1.604755in}}%
\pgfpathlineto{\pgfqpoint{4.100463in}{1.597250in}}%
\pgfpathlineto{\pgfqpoint{4.114742in}{1.589769in}}%
\pgfpathlineto{\pgfqpoint{4.106520in}{1.596568in}}%
\pgfpathlineto{\pgfqpoint{4.098288in}{1.603912in}}%
\pgfpathlineto{\pgfqpoint{4.090045in}{1.611813in}}%
\pgfpathlineto{\pgfqpoint{4.081789in}{1.620283in}}%
\pgfpathlineto{\pgfqpoint{4.067477in}{1.628180in}}%
\pgfpathlineto{\pgfqpoint{4.053169in}{1.636103in}}%
\pgfpathlineto{\pgfqpoint{4.038865in}{1.644049in}}%
\pgfpathlineto{\pgfqpoint{4.024567in}{1.652020in}}%
\pgfpathlineto{\pgfqpoint{4.032858in}{1.643127in}}%
\pgfpathlineto{\pgfqpoint{4.041136in}{1.634807in}}%
\pgfpathlineto{\pgfqpoint{4.049402in}{1.627048in}}%
\pgfpathlineto{\pgfqpoint{4.057657in}{1.619838in}}%
\pgfpathclose%
\pgfusepath{fill}%
\end{pgfscope}%
\begin{pgfscope}%
\pgfpathrectangle{\pgfqpoint{1.150000in}{0.150000in}}{\pgfqpoint{5.700000in}{5.700000in}}%
\pgfusepath{clip}%
\pgfsetbuttcap%
\pgfsetroundjoin%
\definecolor{currentfill}{rgb}{0.458674,0.816363,0.329727}%
\pgfsetfillcolor{currentfill}%
\pgfsetfillopacity{0.700000}%
\pgfsetlinewidth{0.000000pt}%
\definecolor{currentstroke}{rgb}{0.000000,0.000000,0.000000}%
\pgfsetstrokecolor{currentstroke}%
\pgfsetdash{}{0pt}%
\pgfpathmoveto{\pgfqpoint{2.044416in}{3.299130in}}%
\pgfpathlineto{\pgfqpoint{2.058571in}{3.285328in}}%
\pgfpathlineto{\pgfqpoint{2.072724in}{3.271579in}}%
\pgfpathlineto{\pgfqpoint{2.086877in}{3.257882in}}%
\pgfpathlineto{\pgfqpoint{2.101030in}{3.244237in}}%
\pgfpathlineto{\pgfqpoint{2.090502in}{3.277645in}}%
\pgfpathlineto{\pgfqpoint{2.079918in}{3.311985in}}%
\pgfpathlineto{\pgfqpoint{2.069276in}{3.347275in}}%
\pgfpathlineto{\pgfqpoint{2.058576in}{3.383531in}}%
\pgfpathlineto{\pgfqpoint{2.044341in}{3.397706in}}%
\pgfpathlineto{\pgfqpoint{2.030105in}{3.411933in}}%
\pgfpathlineto{\pgfqpoint{2.015868in}{3.426212in}}%
\pgfpathlineto{\pgfqpoint{2.001630in}{3.440545in}}%
\pgfpathlineto{\pgfqpoint{2.012416in}{3.403750in}}%
\pgfpathlineto{\pgfqpoint{2.023141in}{3.367927in}}%
\pgfpathlineto{\pgfqpoint{2.033808in}{3.333059in}}%
\pgfpathlineto{\pgfqpoint{2.044416in}{3.299130in}}%
\pgfpathclose%
\pgfusepath{fill}%
\end{pgfscope}%
\begin{pgfscope}%
\pgfpathrectangle{\pgfqpoint{1.150000in}{0.150000in}}{\pgfqpoint{5.700000in}{5.700000in}}%
\pgfusepath{clip}%
\pgfsetbuttcap%
\pgfsetroundjoin%
\definecolor{currentfill}{rgb}{0.282910,0.105393,0.426902}%
\pgfsetfillcolor{currentfill}%
\pgfsetfillopacity{0.700000}%
\pgfsetlinewidth{0.000000pt}%
\definecolor{currentstroke}{rgb}{0.000000,0.000000,0.000000}%
\pgfsetstrokecolor{currentstroke}%
\pgfsetdash{}{0pt}%
\pgfpathmoveto{\pgfqpoint{4.523066in}{1.400143in}}%
\pgfpathlineto{\pgfqpoint{4.537427in}{1.394044in}}%
\pgfpathlineto{\pgfqpoint{4.551794in}{1.387968in}}%
\pgfpathlineto{\pgfqpoint{4.566166in}{1.381915in}}%
\pgfpathlineto{\pgfqpoint{4.580545in}{1.375886in}}%
\pgfpathlineto{\pgfqpoint{4.572577in}{1.376076in}}%
\pgfpathlineto{\pgfqpoint{4.564603in}{1.376696in}}%
\pgfpathlineto{\pgfqpoint{4.556625in}{1.377756in}}%
\pgfpathlineto{\pgfqpoint{4.548641in}{1.379266in}}%
\pgfpathlineto{\pgfqpoint{4.534239in}{1.385680in}}%
\pgfpathlineto{\pgfqpoint{4.519843in}{1.392118in}}%
\pgfpathlineto{\pgfqpoint{4.505453in}{1.398579in}}%
\pgfpathlineto{\pgfqpoint{4.491069in}{1.405064in}}%
\pgfpathlineto{\pgfqpoint{4.499076in}{1.403163in}}%
\pgfpathlineto{\pgfqpoint{4.507079in}{1.401716in}}%
\pgfpathlineto{\pgfqpoint{4.515075in}{1.400713in}}%
\pgfpathlineto{\pgfqpoint{4.523066in}{1.400143in}}%
\pgfpathclose%
\pgfusepath{fill}%
\end{pgfscope}%
\begin{pgfscope}%
\pgfpathrectangle{\pgfqpoint{1.150000in}{0.150000in}}{\pgfqpoint{5.700000in}{5.700000in}}%
\pgfusepath{clip}%
\pgfsetbuttcap%
\pgfsetroundjoin%
\definecolor{currentfill}{rgb}{0.277018,0.050344,0.375715}%
\pgfsetfillcolor{currentfill}%
\pgfsetfillopacity{0.700000}%
\pgfsetlinewidth{0.000000pt}%
\definecolor{currentstroke}{rgb}{0.000000,0.000000,0.000000}%
\pgfsetstrokecolor{currentstroke}%
\pgfsetdash{}{0pt}%
\pgfpathmoveto{\pgfqpoint{4.874458in}{1.306936in}}%
\pgfpathlineto{\pgfqpoint{4.888910in}{1.302028in}}%
\pgfpathlineto{\pgfqpoint{4.903370in}{1.297144in}}%
\pgfpathlineto{\pgfqpoint{4.917836in}{1.292283in}}%
\pgfpathlineto{\pgfqpoint{4.932310in}{1.287445in}}%
\pgfpathlineto{\pgfqpoint{4.924456in}{1.282805in}}%
\pgfpathlineto{\pgfqpoint{4.916601in}{1.278498in}}%
\pgfpathlineto{\pgfqpoint{4.908744in}{1.274533in}}%
\pgfpathlineto{\pgfqpoint{4.900885in}{1.270920in}}%
\pgfpathlineto{\pgfqpoint{4.886397in}{1.276115in}}%
\pgfpathlineto{\pgfqpoint{4.871916in}{1.281333in}}%
\pgfpathlineto{\pgfqpoint{4.857441in}{1.286574in}}%
\pgfpathlineto{\pgfqpoint{4.842973in}{1.291838in}}%
\pgfpathlineto{\pgfqpoint{4.850847in}{1.295089in}}%
\pgfpathlineto{\pgfqpoint{4.858720in}{1.298696in}}%
\pgfpathlineto{\pgfqpoint{4.866590in}{1.302648in}}%
\pgfpathlineto{\pgfqpoint{4.874458in}{1.306936in}}%
\pgfpathclose%
\pgfusepath{fill}%
\end{pgfscope}%
\begin{pgfscope}%
\pgfpathrectangle{\pgfqpoint{1.150000in}{0.150000in}}{\pgfqpoint{5.700000in}{5.700000in}}%
\pgfusepath{clip}%
\pgfsetbuttcap%
\pgfsetroundjoin%
\definecolor{currentfill}{rgb}{0.282290,0.145912,0.461510}%
\pgfsetfillcolor{currentfill}%
\pgfsetfillopacity{0.700000}%
\pgfsetlinewidth{0.000000pt}%
\definecolor{currentstroke}{rgb}{0.000000,0.000000,0.000000}%
\pgfsetstrokecolor{currentstroke}%
\pgfsetdash{}{0pt}%
\pgfpathmoveto{\pgfqpoint{4.318900in}{1.484714in}}%
\pgfpathlineto{\pgfqpoint{4.333217in}{1.477947in}}%
\pgfpathlineto{\pgfqpoint{4.347539in}{1.471203in}}%
\pgfpathlineto{\pgfqpoint{4.361867in}{1.464483in}}%
\pgfpathlineto{\pgfqpoint{4.376200in}{1.457786in}}%
\pgfpathlineto{\pgfqpoint{4.368134in}{1.460941in}}%
\pgfpathlineto{\pgfqpoint{4.360061in}{1.464580in}}%
\pgfpathlineto{\pgfqpoint{4.351981in}{1.468714in}}%
\pgfpathlineto{\pgfqpoint{4.343893in}{1.473353in}}%
\pgfpathlineto{\pgfqpoint{4.329531in}{1.480450in}}%
\pgfpathlineto{\pgfqpoint{4.315176in}{1.487571in}}%
\pgfpathlineto{\pgfqpoint{4.300825in}{1.494715in}}%
\pgfpathlineto{\pgfqpoint{4.286480in}{1.501883in}}%
\pgfpathlineto{\pgfqpoint{4.294597in}{1.496837in}}%
\pgfpathlineto{\pgfqpoint{4.302706in}{1.492301in}}%
\pgfpathlineto{\pgfqpoint{4.310807in}{1.488264in}}%
\pgfpathlineto{\pgfqpoint{4.318900in}{1.484714in}}%
\pgfpathclose%
\pgfusepath{fill}%
\end{pgfscope}%
\begin{pgfscope}%
\pgfpathrectangle{\pgfqpoint{1.150000in}{0.150000in}}{\pgfqpoint{5.700000in}{5.700000in}}%
\pgfusepath{clip}%
\pgfsetbuttcap%
\pgfsetroundjoin%
\definecolor{currentfill}{rgb}{0.412913,0.803041,0.357269}%
\pgfsetfillcolor{currentfill}%
\pgfsetfillopacity{0.700000}%
\pgfsetlinewidth{0.000000pt}%
\definecolor{currentstroke}{rgb}{0.000000,0.000000,0.000000}%
\pgfsetstrokecolor{currentstroke}%
\pgfsetdash{}{0pt}%
\pgfpathmoveto{\pgfqpoint{2.101030in}{3.244237in}}%
\pgfpathlineto{\pgfqpoint{2.115182in}{3.230643in}}%
\pgfpathlineto{\pgfqpoint{2.129334in}{3.217099in}}%
\pgfpathlineto{\pgfqpoint{2.143485in}{3.203606in}}%
\pgfpathlineto{\pgfqpoint{2.157636in}{3.190162in}}%
\pgfpathlineto{\pgfqpoint{2.147189in}{3.223051in}}%
\pgfpathlineto{\pgfqpoint{2.136686in}{3.256866in}}%
\pgfpathlineto{\pgfqpoint{2.126127in}{3.291624in}}%
\pgfpathlineto{\pgfqpoint{2.115511in}{3.327343in}}%
\pgfpathlineto{\pgfqpoint{2.101278in}{3.341315in}}%
\pgfpathlineto{\pgfqpoint{2.087045in}{3.355336in}}%
\pgfpathlineto{\pgfqpoint{2.072811in}{3.369408in}}%
\pgfpathlineto{\pgfqpoint{2.058576in}{3.383531in}}%
\pgfpathlineto{\pgfqpoint{2.069276in}{3.347275in}}%
\pgfpathlineto{\pgfqpoint{2.079918in}{3.311985in}}%
\pgfpathlineto{\pgfqpoint{2.090502in}{3.277645in}}%
\pgfpathlineto{\pgfqpoint{2.101030in}{3.244237in}}%
\pgfpathclose%
\pgfusepath{fill}%
\end{pgfscope}%
\begin{pgfscope}%
\pgfpathrectangle{\pgfqpoint{1.150000in}{0.150000in}}{\pgfqpoint{5.700000in}{5.700000in}}%
\pgfusepath{clip}%
\pgfsetbuttcap%
\pgfsetroundjoin%
\definecolor{currentfill}{rgb}{0.136408,0.541173,0.554483}%
\pgfsetfillcolor{currentfill}%
\pgfsetfillopacity{0.700000}%
\pgfsetlinewidth{0.000000pt}%
\definecolor{currentstroke}{rgb}{0.000000,0.000000,0.000000}%
\pgfsetstrokecolor{currentstroke}%
\pgfsetdash{}{0pt}%
\pgfpathmoveto{\pgfqpoint{2.932011in}{2.452656in}}%
\pgfpathlineto{\pgfqpoint{2.946152in}{2.441801in}}%
\pgfpathlineto{\pgfqpoint{2.960296in}{2.430979in}}%
\pgfpathlineto{\pgfqpoint{2.974442in}{2.420188in}}%
\pgfpathlineto{\pgfqpoint{2.988590in}{2.409430in}}%
\pgfpathlineto{\pgfqpoint{2.979288in}{2.432006in}}%
\pgfpathlineto{\pgfqpoint{2.969953in}{2.455367in}}%
\pgfpathlineto{\pgfqpoint{2.960583in}{2.479529in}}%
\pgfpathlineto{\pgfqpoint{2.951178in}{2.504506in}}%
\pgfpathlineto{\pgfqpoint{2.936968in}{2.515748in}}%
\pgfpathlineto{\pgfqpoint{2.922760in}{2.527021in}}%
\pgfpathlineto{\pgfqpoint{2.908554in}{2.538327in}}%
\pgfpathlineto{\pgfqpoint{2.894350in}{2.549666in}}%
\pgfpathlineto{\pgfqpoint{2.903819in}{2.524197in}}%
\pgfpathlineto{\pgfqpoint{2.913251in}{2.499550in}}%
\pgfpathlineto{\pgfqpoint{2.922648in}{2.475708in}}%
\pgfpathlineto{\pgfqpoint{2.932011in}{2.452656in}}%
\pgfpathclose%
\pgfusepath{fill}%
\end{pgfscope}%
\begin{pgfscope}%
\pgfpathrectangle{\pgfqpoint{1.150000in}{0.150000in}}{\pgfqpoint{5.700000in}{5.700000in}}%
\pgfusepath{clip}%
\pgfsetbuttcap%
\pgfsetroundjoin%
\definecolor{currentfill}{rgb}{0.195860,0.395433,0.555276}%
\pgfsetfillcolor{currentfill}%
\pgfsetfillopacity{0.700000}%
\pgfsetlinewidth{0.000000pt}%
\definecolor{currentstroke}{rgb}{0.000000,0.000000,0.000000}%
\pgfsetstrokecolor{currentstroke}%
\pgfsetdash{}{0pt}%
\pgfpathmoveto{\pgfqpoint{3.421185in}{2.050631in}}%
\pgfpathlineto{\pgfqpoint{3.435364in}{2.041189in}}%
\pgfpathlineto{\pgfqpoint{3.449546in}{2.031774in}}%
\pgfpathlineto{\pgfqpoint{3.463731in}{2.022387in}}%
\pgfpathlineto{\pgfqpoint{3.477920in}{2.013027in}}%
\pgfpathlineto{\pgfqpoint{3.469163in}{2.028992in}}%
\pgfpathlineto{\pgfqpoint{3.460382in}{2.045647in}}%
\pgfpathlineto{\pgfqpoint{3.451579in}{2.063004in}}%
\pgfpathlineto{\pgfqpoint{3.442752in}{2.081079in}}%
\pgfpathlineto{\pgfqpoint{3.428513in}{2.090895in}}%
\pgfpathlineto{\pgfqpoint{3.414277in}{2.100739in}}%
\pgfpathlineto{\pgfqpoint{3.400045in}{2.110610in}}%
\pgfpathlineto{\pgfqpoint{3.385816in}{2.120509in}}%
\pgfpathlineto{\pgfqpoint{3.394695in}{2.101971in}}%
\pgfpathlineto{\pgfqpoint{3.403549in}{2.084154in}}%
\pgfpathlineto{\pgfqpoint{3.412379in}{2.067045in}}%
\pgfpathlineto{\pgfqpoint{3.421185in}{2.050631in}}%
\pgfpathclose%
\pgfusepath{fill}%
\end{pgfscope}%
\begin{pgfscope}%
\pgfpathrectangle{\pgfqpoint{1.150000in}{0.150000in}}{\pgfqpoint{5.700000in}{5.700000in}}%
\pgfusepath{clip}%
\pgfsetbuttcap%
\pgfsetroundjoin%
\definecolor{currentfill}{rgb}{0.280267,0.073417,0.397163}%
\pgfsetfillcolor{currentfill}%
\pgfsetfillopacity{0.700000}%
\pgfsetlinewidth{0.000000pt}%
\definecolor{currentstroke}{rgb}{0.000000,0.000000,0.000000}%
\pgfsetstrokecolor{currentstroke}%
\pgfsetdash{}{0pt}%
\pgfpathmoveto{\pgfqpoint{4.727466in}{1.334783in}}%
\pgfpathlineto{\pgfqpoint{4.741881in}{1.329334in}}%
\pgfpathlineto{\pgfqpoint{4.756303in}{1.323908in}}%
\pgfpathlineto{\pgfqpoint{4.770732in}{1.318505in}}%
\pgfpathlineto{\pgfqpoint{4.785167in}{1.313126in}}%
\pgfpathlineto{\pgfqpoint{4.777272in}{1.310605in}}%
\pgfpathlineto{\pgfqpoint{4.769375in}{1.308462in}}%
\pgfpathlineto{\pgfqpoint{4.761475in}{1.306707in}}%
\pgfpathlineto{\pgfqpoint{4.753572in}{1.305349in}}%
\pgfpathlineto{\pgfqpoint{4.739119in}{1.311100in}}%
\pgfpathlineto{\pgfqpoint{4.724672in}{1.316874in}}%
\pgfpathlineto{\pgfqpoint{4.710231in}{1.322671in}}%
\pgfpathlineto{\pgfqpoint{4.695797in}{1.328491in}}%
\pgfpathlineto{\pgfqpoint{4.703719in}{1.329471in}}%
\pgfpathlineto{\pgfqpoint{4.711638in}{1.330854in}}%
\pgfpathlineto{\pgfqpoint{4.719554in}{1.332628in}}%
\pgfpathlineto{\pgfqpoint{4.727466in}{1.334783in}}%
\pgfpathclose%
\pgfusepath{fill}%
\end{pgfscope}%
\begin{pgfscope}%
\pgfpathrectangle{\pgfqpoint{1.150000in}{0.150000in}}{\pgfqpoint{5.700000in}{5.700000in}}%
\pgfusepath{clip}%
\pgfsetbuttcap%
\pgfsetroundjoin%
\definecolor{currentfill}{rgb}{0.244972,0.287675,0.537260}%
\pgfsetfillcolor{currentfill}%
\pgfsetfillopacity{0.700000}%
\pgfsetlinewidth{0.000000pt}%
\definecolor{currentstroke}{rgb}{0.000000,0.000000,0.000000}%
\pgfsetstrokecolor{currentstroke}%
\pgfsetdash{}{0pt}%
\pgfpathmoveto{\pgfqpoint{3.796407in}{1.782933in}}%
\pgfpathlineto{\pgfqpoint{3.810634in}{1.774562in}}%
\pgfpathlineto{\pgfqpoint{3.824865in}{1.766217in}}%
\pgfpathlineto{\pgfqpoint{3.839101in}{1.757897in}}%
\pgfpathlineto{\pgfqpoint{3.853340in}{1.749603in}}%
\pgfpathlineto{\pgfqpoint{3.844922in}{1.760365in}}%
\pgfpathlineto{\pgfqpoint{3.836488in}{1.771738in}}%
\pgfpathlineto{\pgfqpoint{3.828039in}{1.783733in}}%
\pgfpathlineto{\pgfqpoint{3.819573in}{1.796364in}}%
\pgfpathlineto{\pgfqpoint{3.805292in}{1.805094in}}%
\pgfpathlineto{\pgfqpoint{3.791015in}{1.813849in}}%
\pgfpathlineto{\pgfqpoint{3.776743in}{1.822629in}}%
\pgfpathlineto{\pgfqpoint{3.762474in}{1.831435in}}%
\pgfpathlineto{\pgfqpoint{3.770983in}{1.818362in}}%
\pgfpathlineto{\pgfqpoint{3.779474in}{1.805929in}}%
\pgfpathlineto{\pgfqpoint{3.787949in}{1.794123in}}%
\pgfpathlineto{\pgfqpoint{3.796407in}{1.782933in}}%
\pgfpathclose%
\pgfusepath{fill}%
\end{pgfscope}%
\begin{pgfscope}%
\pgfpathrectangle{\pgfqpoint{1.150000in}{0.150000in}}{\pgfqpoint{5.700000in}{5.700000in}}%
\pgfusepath{clip}%
\pgfsetbuttcap%
\pgfsetroundjoin%
\definecolor{currentfill}{rgb}{0.369214,0.788888,0.382914}%
\pgfsetfillcolor{currentfill}%
\pgfsetfillopacity{0.700000}%
\pgfsetlinewidth{0.000000pt}%
\definecolor{currentstroke}{rgb}{0.000000,0.000000,0.000000}%
\pgfsetstrokecolor{currentstroke}%
\pgfsetdash{}{0pt}%
\pgfpathmoveto{\pgfqpoint{2.157636in}{3.190162in}}%
\pgfpathlineto{\pgfqpoint{2.171787in}{3.176767in}}%
\pgfpathlineto{\pgfqpoint{2.185938in}{3.163421in}}%
\pgfpathlineto{\pgfqpoint{2.200089in}{3.150123in}}%
\pgfpathlineto{\pgfqpoint{2.214240in}{3.136872in}}%
\pgfpathlineto{\pgfqpoint{2.203871in}{3.169245in}}%
\pgfpathlineto{\pgfqpoint{2.193448in}{3.202537in}}%
\pgfpathlineto{\pgfqpoint{2.182971in}{3.236767in}}%
\pgfpathlineto{\pgfqpoint{2.172438in}{3.271950in}}%
\pgfpathlineto{\pgfqpoint{2.158207in}{3.285726in}}%
\pgfpathlineto{\pgfqpoint{2.143975in}{3.299549in}}%
\pgfpathlineto{\pgfqpoint{2.129743in}{3.313422in}}%
\pgfpathlineto{\pgfqpoint{2.115511in}{3.327343in}}%
\pgfpathlineto{\pgfqpoint{2.126127in}{3.291624in}}%
\pgfpathlineto{\pgfqpoint{2.136686in}{3.256866in}}%
\pgfpathlineto{\pgfqpoint{2.147189in}{3.223051in}}%
\pgfpathlineto{\pgfqpoint{2.157636in}{3.190162in}}%
\pgfpathclose%
\pgfusepath{fill}%
\end{pgfscope}%
\begin{pgfscope}%
\pgfpathrectangle{\pgfqpoint{1.150000in}{0.150000in}}{\pgfqpoint{5.700000in}{5.700000in}}%
\pgfusepath{clip}%
\pgfsetbuttcap%
\pgfsetroundjoin%
\definecolor{currentfill}{rgb}{0.327796,0.773980,0.406640}%
\pgfsetfillcolor{currentfill}%
\pgfsetfillopacity{0.700000}%
\pgfsetlinewidth{0.000000pt}%
\definecolor{currentstroke}{rgb}{0.000000,0.000000,0.000000}%
\pgfsetstrokecolor{currentstroke}%
\pgfsetdash{}{0pt}%
\pgfpathmoveto{\pgfqpoint{2.214240in}{3.136872in}}%
\pgfpathlineto{\pgfqpoint{2.228390in}{3.123669in}}%
\pgfpathlineto{\pgfqpoint{2.242541in}{3.110513in}}%
\pgfpathlineto{\pgfqpoint{2.256692in}{3.097403in}}%
\pgfpathlineto{\pgfqpoint{2.270843in}{3.084339in}}%
\pgfpathlineto{\pgfqpoint{2.260552in}{3.116197in}}%
\pgfpathlineto{\pgfqpoint{2.250208in}{3.148968in}}%
\pgfpathlineto{\pgfqpoint{2.239812in}{3.182671in}}%
\pgfpathlineto{\pgfqpoint{2.229361in}{3.217322in}}%
\pgfpathlineto{\pgfqpoint{2.215130in}{3.230909in}}%
\pgfpathlineto{\pgfqpoint{2.200900in}{3.244542in}}%
\pgfpathlineto{\pgfqpoint{2.186669in}{3.258223in}}%
\pgfpathlineto{\pgfqpoint{2.172438in}{3.271950in}}%
\pgfpathlineto{\pgfqpoint{2.182971in}{3.236767in}}%
\pgfpathlineto{\pgfqpoint{2.193448in}{3.202537in}}%
\pgfpathlineto{\pgfqpoint{2.203871in}{3.169245in}}%
\pgfpathlineto{\pgfqpoint{2.214240in}{3.136872in}}%
\pgfpathclose%
\pgfusepath{fill}%
\end{pgfscope}%
\begin{pgfscope}%
\pgfpathrectangle{\pgfqpoint{1.150000in}{0.150000in}}{\pgfqpoint{5.700000in}{5.700000in}}%
\pgfusepath{clip}%
\pgfsetbuttcap%
\pgfsetroundjoin%
\definecolor{currentfill}{rgb}{0.141935,0.526453,0.555991}%
\pgfsetfillcolor{currentfill}%
\pgfsetfillopacity{0.700000}%
\pgfsetlinewidth{0.000000pt}%
\definecolor{currentstroke}{rgb}{0.000000,0.000000,0.000000}%
\pgfsetstrokecolor{currentstroke}%
\pgfsetdash{}{0pt}%
\pgfpathmoveto{\pgfqpoint{2.988590in}{2.409430in}}%
\pgfpathlineto{\pgfqpoint{3.002741in}{2.398703in}}%
\pgfpathlineto{\pgfqpoint{3.016894in}{2.388008in}}%
\pgfpathlineto{\pgfqpoint{3.031049in}{2.377344in}}%
\pgfpathlineto{\pgfqpoint{3.045207in}{2.366712in}}%
\pgfpathlineto{\pgfqpoint{3.035965in}{2.388813in}}%
\pgfpathlineto{\pgfqpoint{3.026690in}{2.411695in}}%
\pgfpathlineto{\pgfqpoint{3.017382in}{2.435372in}}%
\pgfpathlineto{\pgfqpoint{3.008040in}{2.459860in}}%
\pgfpathlineto{\pgfqpoint{2.993822in}{2.470974in}}%
\pgfpathlineto{\pgfqpoint{2.979605in}{2.482120in}}%
\pgfpathlineto{\pgfqpoint{2.965390in}{2.493297in}}%
\pgfpathlineto{\pgfqpoint{2.951178in}{2.504506in}}%
\pgfpathlineto{\pgfqpoint{2.960583in}{2.479529in}}%
\pgfpathlineto{\pgfqpoint{2.969953in}{2.455367in}}%
\pgfpathlineto{\pgfqpoint{2.979288in}{2.432006in}}%
\pgfpathlineto{\pgfqpoint{2.988590in}{2.409430in}}%
\pgfpathclose%
\pgfusepath{fill}%
\end{pgfscope}%
\begin{pgfscope}%
\pgfpathrectangle{\pgfqpoint{1.150000in}{0.150000in}}{\pgfqpoint{5.700000in}{5.700000in}}%
\pgfusepath{clip}%
\pgfsetbuttcap%
\pgfsetroundjoin%
\definecolor{currentfill}{rgb}{0.273006,0.204520,0.501721}%
\pgfsetfillcolor{currentfill}%
\pgfsetfillopacity{0.700000}%
\pgfsetlinewidth{0.000000pt}%
\definecolor{currentstroke}{rgb}{0.000000,0.000000,0.000000}%
\pgfsetstrokecolor{currentstroke}%
\pgfsetdash{}{0pt}%
\pgfpathmoveto{\pgfqpoint{4.114742in}{1.589769in}}%
\pgfpathlineto{\pgfqpoint{4.129025in}{1.582313in}}%
\pgfpathlineto{\pgfqpoint{4.143314in}{1.574881in}}%
\pgfpathlineto{\pgfqpoint{4.157607in}{1.567473in}}%
\pgfpathlineto{\pgfqpoint{4.171906in}{1.560089in}}%
\pgfpathlineto{\pgfqpoint{4.163717in}{1.566477in}}%
\pgfpathlineto{\pgfqpoint{4.155518in}{1.573406in}}%
\pgfpathlineto{\pgfqpoint{4.147309in}{1.580888in}}%
\pgfpathlineto{\pgfqpoint{4.139089in}{1.588935in}}%
\pgfpathlineto{\pgfqpoint{4.124757in}{1.596736in}}%
\pgfpathlineto{\pgfqpoint{4.110429in}{1.604560in}}%
\pgfpathlineto{\pgfqpoint{4.096107in}{1.612409in}}%
\pgfpathlineto{\pgfqpoint{4.081789in}{1.620283in}}%
\pgfpathlineto{\pgfqpoint{4.090045in}{1.611813in}}%
\pgfpathlineto{\pgfqpoint{4.098288in}{1.603912in}}%
\pgfpathlineto{\pgfqpoint{4.106520in}{1.596568in}}%
\pgfpathlineto{\pgfqpoint{4.114742in}{1.589769in}}%
\pgfpathclose%
\pgfusepath{fill}%
\end{pgfscope}%
\begin{pgfscope}%
\pgfpathrectangle{\pgfqpoint{1.150000in}{0.150000in}}{\pgfqpoint{5.700000in}{5.700000in}}%
\pgfusepath{clip}%
\pgfsetbuttcap%
\pgfsetroundjoin%
\definecolor{currentfill}{rgb}{0.201239,0.383670,0.554294}%
\pgfsetfillcolor{currentfill}%
\pgfsetfillopacity{0.700000}%
\pgfsetlinewidth{0.000000pt}%
\definecolor{currentstroke}{rgb}{0.000000,0.000000,0.000000}%
\pgfsetstrokecolor{currentstroke}%
\pgfsetdash{}{0pt}%
\pgfpathmoveto{\pgfqpoint{3.477920in}{2.013027in}}%
\pgfpathlineto{\pgfqpoint{3.492112in}{2.003695in}}%
\pgfpathlineto{\pgfqpoint{3.506308in}{1.994389in}}%
\pgfpathlineto{\pgfqpoint{3.520508in}{1.985111in}}%
\pgfpathlineto{\pgfqpoint{3.534711in}{1.975860in}}%
\pgfpathlineto{\pgfqpoint{3.526002in}{1.991376in}}%
\pgfpathlineto{\pgfqpoint{3.517271in}{2.007577in}}%
\pgfpathlineto{\pgfqpoint{3.508517in}{2.024476in}}%
\pgfpathlineto{\pgfqpoint{3.499741in}{2.042088in}}%
\pgfpathlineto{\pgfqpoint{3.485489in}{2.051795in}}%
\pgfpathlineto{\pgfqpoint{3.471240in}{2.061529in}}%
\pgfpathlineto{\pgfqpoint{3.456994in}{2.071291in}}%
\pgfpathlineto{\pgfqpoint{3.442752in}{2.081079in}}%
\pgfpathlineto{\pgfqpoint{3.451579in}{2.063004in}}%
\pgfpathlineto{\pgfqpoint{3.460382in}{2.045647in}}%
\pgfpathlineto{\pgfqpoint{3.469163in}{2.028992in}}%
\pgfpathlineto{\pgfqpoint{3.477920in}{2.013027in}}%
\pgfpathclose%
\pgfusepath{fill}%
\end{pgfscope}%
\begin{pgfscope}%
\pgfpathrectangle{\pgfqpoint{1.150000in}{0.150000in}}{\pgfqpoint{5.700000in}{5.700000in}}%
\pgfusepath{clip}%
\pgfsetbuttcap%
\pgfsetroundjoin%
\definecolor{currentfill}{rgb}{0.282656,0.100196,0.422160}%
\pgfsetfillcolor{currentfill}%
\pgfsetfillopacity{0.700000}%
\pgfsetlinewidth{0.000000pt}%
\definecolor{currentstroke}{rgb}{0.000000,0.000000,0.000000}%
\pgfsetstrokecolor{currentstroke}%
\pgfsetdash{}{0pt}%
\pgfpathmoveto{\pgfqpoint{4.580545in}{1.375886in}}%
\pgfpathlineto{\pgfqpoint{4.594930in}{1.369880in}}%
\pgfpathlineto{\pgfqpoint{4.609321in}{1.363898in}}%
\pgfpathlineto{\pgfqpoint{4.623718in}{1.357938in}}%
\pgfpathlineto{\pgfqpoint{4.638121in}{1.352002in}}%
\pgfpathlineto{\pgfqpoint{4.630174in}{1.351813in}}%
\pgfpathlineto{\pgfqpoint{4.622223in}{1.352049in}}%
\pgfpathlineto{\pgfqpoint{4.614268in}{1.352722in}}%
\pgfpathlineto{\pgfqpoint{4.606307in}{1.353841in}}%
\pgfpathlineto{\pgfqpoint{4.591882in}{1.360162in}}%
\pgfpathlineto{\pgfqpoint{4.577462in}{1.366507in}}%
\pgfpathlineto{\pgfqpoint{4.563049in}{1.372875in}}%
\pgfpathlineto{\pgfqpoint{4.548641in}{1.379266in}}%
\pgfpathlineto{\pgfqpoint{4.556625in}{1.377756in}}%
\pgfpathlineto{\pgfqpoint{4.564603in}{1.376696in}}%
\pgfpathlineto{\pgfqpoint{4.572577in}{1.376076in}}%
\pgfpathlineto{\pgfqpoint{4.580545in}{1.375886in}}%
\pgfpathclose%
\pgfusepath{fill}%
\end{pgfscope}%
\begin{pgfscope}%
\pgfpathrectangle{\pgfqpoint{1.150000in}{0.150000in}}{\pgfqpoint{5.700000in}{5.700000in}}%
\pgfusepath{clip}%
\pgfsetbuttcap%
\pgfsetroundjoin%
\definecolor{currentfill}{rgb}{0.296479,0.761561,0.424223}%
\pgfsetfillcolor{currentfill}%
\pgfsetfillopacity{0.700000}%
\pgfsetlinewidth{0.000000pt}%
\definecolor{currentstroke}{rgb}{0.000000,0.000000,0.000000}%
\pgfsetstrokecolor{currentstroke}%
\pgfsetdash{}{0pt}%
\pgfpathmoveto{\pgfqpoint{2.270843in}{3.084339in}}%
\pgfpathlineto{\pgfqpoint{2.284994in}{3.071321in}}%
\pgfpathlineto{\pgfqpoint{2.299146in}{3.058347in}}%
\pgfpathlineto{\pgfqpoint{2.313298in}{3.045419in}}%
\pgfpathlineto{\pgfqpoint{2.327450in}{3.032534in}}%
\pgfpathlineto{\pgfqpoint{2.317235in}{3.063879in}}%
\pgfpathlineto{\pgfqpoint{2.306970in}{3.096132in}}%
\pgfpathlineto{\pgfqpoint{2.296653in}{3.129310in}}%
\pgfpathlineto{\pgfqpoint{2.286282in}{3.163430in}}%
\pgfpathlineto{\pgfqpoint{2.272052in}{3.176835in}}%
\pgfpathlineto{\pgfqpoint{2.257821in}{3.190286in}}%
\pgfpathlineto{\pgfqpoint{2.243591in}{3.203781in}}%
\pgfpathlineto{\pgfqpoint{2.229361in}{3.217322in}}%
\pgfpathlineto{\pgfqpoint{2.239812in}{3.182671in}}%
\pgfpathlineto{\pgfqpoint{2.250208in}{3.148968in}}%
\pgfpathlineto{\pgfqpoint{2.260552in}{3.116197in}}%
\pgfpathlineto{\pgfqpoint{2.270843in}{3.084339in}}%
\pgfpathclose%
\pgfusepath{fill}%
\end{pgfscope}%
\begin{pgfscope}%
\pgfpathrectangle{\pgfqpoint{1.150000in}{0.150000in}}{\pgfqpoint{5.700000in}{5.700000in}}%
\pgfusepath{clip}%
\pgfsetbuttcap%
\pgfsetroundjoin%
\definecolor{currentfill}{rgb}{0.282623,0.140926,0.457517}%
\pgfsetfillcolor{currentfill}%
\pgfsetfillopacity{0.700000}%
\pgfsetlinewidth{0.000000pt}%
\definecolor{currentstroke}{rgb}{0.000000,0.000000,0.000000}%
\pgfsetstrokecolor{currentstroke}%
\pgfsetdash{}{0pt}%
\pgfpathmoveto{\pgfqpoint{4.376200in}{1.457786in}}%
\pgfpathlineto{\pgfqpoint{4.390539in}{1.451113in}}%
\pgfpathlineto{\pgfqpoint{4.404883in}{1.444464in}}%
\pgfpathlineto{\pgfqpoint{4.419233in}{1.437839in}}%
\pgfpathlineto{\pgfqpoint{4.433589in}{1.431237in}}%
\pgfpathlineto{\pgfqpoint{4.425550in}{1.433997in}}%
\pgfpathlineto{\pgfqpoint{4.417504in}{1.437238in}}%
\pgfpathlineto{\pgfqpoint{4.409451in}{1.440969in}}%
\pgfpathlineto{\pgfqpoint{4.401391in}{1.445202in}}%
\pgfpathlineto{\pgfqpoint{4.387009in}{1.452205in}}%
\pgfpathlineto{\pgfqpoint{4.372631in}{1.459231in}}%
\pgfpathlineto{\pgfqpoint{4.358259in}{1.466280in}}%
\pgfpathlineto{\pgfqpoint{4.343893in}{1.473353in}}%
\pgfpathlineto{\pgfqpoint{4.351981in}{1.468714in}}%
\pgfpathlineto{\pgfqpoint{4.360061in}{1.464580in}}%
\pgfpathlineto{\pgfqpoint{4.368134in}{1.460941in}}%
\pgfpathlineto{\pgfqpoint{4.376200in}{1.457786in}}%
\pgfpathclose%
\pgfusepath{fill}%
\end{pgfscope}%
\begin{pgfscope}%
\pgfpathrectangle{\pgfqpoint{1.150000in}{0.150000in}}{\pgfqpoint{5.700000in}{5.700000in}}%
\pgfusepath{clip}%
\pgfsetbuttcap%
\pgfsetroundjoin%
\definecolor{currentfill}{rgb}{0.277018,0.050344,0.375715}%
\pgfsetfillcolor{currentfill}%
\pgfsetfillopacity{0.700000}%
\pgfsetlinewidth{0.000000pt}%
\definecolor{currentstroke}{rgb}{0.000000,0.000000,0.000000}%
\pgfsetstrokecolor{currentstroke}%
\pgfsetdash{}{0pt}%
\pgfpathmoveto{\pgfqpoint{4.932310in}{1.287445in}}%
\pgfpathlineto{\pgfqpoint{4.946790in}{1.282630in}}%
\pgfpathlineto{\pgfqpoint{4.961277in}{1.277838in}}%
\pgfpathlineto{\pgfqpoint{4.975772in}{1.273069in}}%
\pgfpathlineto{\pgfqpoint{4.990273in}{1.268324in}}%
\pgfpathlineto{\pgfqpoint{4.982433in}{1.263332in}}%
\pgfpathlineto{\pgfqpoint{4.974593in}{1.258670in}}%
\pgfpathlineto{\pgfqpoint{4.966750in}{1.254346in}}%
\pgfpathlineto{\pgfqpoint{4.958907in}{1.250370in}}%
\pgfpathlineto{\pgfqpoint{4.944391in}{1.255473in}}%
\pgfpathlineto{\pgfqpoint{4.929882in}{1.260599in}}%
\pgfpathlineto{\pgfqpoint{4.915381in}{1.265748in}}%
\pgfpathlineto{\pgfqpoint{4.900885in}{1.270920in}}%
\pgfpathlineto{\pgfqpoint{4.908744in}{1.274533in}}%
\pgfpathlineto{\pgfqpoint{4.916601in}{1.278498in}}%
\pgfpathlineto{\pgfqpoint{4.924456in}{1.282805in}}%
\pgfpathlineto{\pgfqpoint{4.932310in}{1.287445in}}%
\pgfpathclose%
\pgfusepath{fill}%
\end{pgfscope}%
\begin{pgfscope}%
\pgfpathrectangle{\pgfqpoint{1.150000in}{0.150000in}}{\pgfqpoint{5.700000in}{5.700000in}}%
\pgfusepath{clip}%
\pgfsetbuttcap%
\pgfsetroundjoin%
\definecolor{currentfill}{rgb}{0.248629,0.278775,0.534556}%
\pgfsetfillcolor{currentfill}%
\pgfsetfillopacity{0.700000}%
\pgfsetlinewidth{0.000000pt}%
\definecolor{currentstroke}{rgb}{0.000000,0.000000,0.000000}%
\pgfsetstrokecolor{currentstroke}%
\pgfsetdash{}{0pt}%
\pgfpathmoveto{\pgfqpoint{3.853340in}{1.749603in}}%
\pgfpathlineto{\pgfqpoint{3.867585in}{1.741334in}}%
\pgfpathlineto{\pgfqpoint{3.881833in}{1.733090in}}%
\pgfpathlineto{\pgfqpoint{3.896086in}{1.724871in}}%
\pgfpathlineto{\pgfqpoint{3.910344in}{1.716677in}}%
\pgfpathlineto{\pgfqpoint{3.901965in}{1.727011in}}%
\pgfpathlineto{\pgfqpoint{3.893571in}{1.737950in}}%
\pgfpathlineto{\pgfqpoint{3.885162in}{1.749509in}}%
\pgfpathlineto{\pgfqpoint{3.876738in}{1.761699in}}%
\pgfpathlineto{\pgfqpoint{3.862440in}{1.770328in}}%
\pgfpathlineto{\pgfqpoint{3.848147in}{1.778981in}}%
\pgfpathlineto{\pgfqpoint{3.833858in}{1.787660in}}%
\pgfpathlineto{\pgfqpoint{3.819573in}{1.796364in}}%
\pgfpathlineto{\pgfqpoint{3.828039in}{1.783733in}}%
\pgfpathlineto{\pgfqpoint{3.836488in}{1.771738in}}%
\pgfpathlineto{\pgfqpoint{3.844922in}{1.760365in}}%
\pgfpathlineto{\pgfqpoint{3.853340in}{1.749603in}}%
\pgfpathclose%
\pgfusepath{fill}%
\end{pgfscope}%
\begin{pgfscope}%
\pgfpathrectangle{\pgfqpoint{1.150000in}{0.150000in}}{\pgfqpoint{5.700000in}{5.700000in}}%
\pgfusepath{clip}%
\pgfsetbuttcap%
\pgfsetroundjoin%
\definecolor{currentfill}{rgb}{0.146180,0.515413,0.556823}%
\pgfsetfillcolor{currentfill}%
\pgfsetfillopacity{0.700000}%
\pgfsetlinewidth{0.000000pt}%
\definecolor{currentstroke}{rgb}{0.000000,0.000000,0.000000}%
\pgfsetstrokecolor{currentstroke}%
\pgfsetdash{}{0pt}%
\pgfpathmoveto{\pgfqpoint{3.045207in}{2.366712in}}%
\pgfpathlineto{\pgfqpoint{3.059367in}{2.356110in}}%
\pgfpathlineto{\pgfqpoint{3.073530in}{2.345540in}}%
\pgfpathlineto{\pgfqpoint{3.087696in}{2.335000in}}%
\pgfpathlineto{\pgfqpoint{3.101864in}{2.324491in}}%
\pgfpathlineto{\pgfqpoint{3.092681in}{2.346119in}}%
\pgfpathlineto{\pgfqpoint{3.083466in}{2.368522in}}%
\pgfpathlineto{\pgfqpoint{3.074219in}{2.391715in}}%
\pgfpathlineto{\pgfqpoint{3.064939in}{2.415714in}}%
\pgfpathlineto{\pgfqpoint{3.050711in}{2.426704in}}%
\pgfpathlineto{\pgfqpoint{3.036485in}{2.437725in}}%
\pgfpathlineto{\pgfqpoint{3.022262in}{2.448777in}}%
\pgfpathlineto{\pgfqpoint{3.008040in}{2.459860in}}%
\pgfpathlineto{\pgfqpoint{3.017382in}{2.435372in}}%
\pgfpathlineto{\pgfqpoint{3.026690in}{2.411695in}}%
\pgfpathlineto{\pgfqpoint{3.035965in}{2.388813in}}%
\pgfpathlineto{\pgfqpoint{3.045207in}{2.366712in}}%
\pgfpathclose%
\pgfusepath{fill}%
\end{pgfscope}%
\begin{pgfscope}%
\pgfpathrectangle{\pgfqpoint{1.150000in}{0.150000in}}{\pgfqpoint{5.700000in}{5.700000in}}%
\pgfusepath{clip}%
\pgfsetbuttcap%
\pgfsetroundjoin%
\definecolor{currentfill}{rgb}{0.259857,0.745492,0.444467}%
\pgfsetfillcolor{currentfill}%
\pgfsetfillopacity{0.700000}%
\pgfsetlinewidth{0.000000pt}%
\definecolor{currentstroke}{rgb}{0.000000,0.000000,0.000000}%
\pgfsetstrokecolor{currentstroke}%
\pgfsetdash{}{0pt}%
\pgfpathmoveto{\pgfqpoint{2.327450in}{3.032534in}}%
\pgfpathlineto{\pgfqpoint{2.341603in}{3.019694in}}%
\pgfpathlineto{\pgfqpoint{2.355756in}{3.006897in}}%
\pgfpathlineto{\pgfqpoint{2.369910in}{2.994143in}}%
\pgfpathlineto{\pgfqpoint{2.384064in}{2.981432in}}%
\pgfpathlineto{\pgfqpoint{2.373925in}{3.012266in}}%
\pgfpathlineto{\pgfqpoint{2.363736in}{3.044002in}}%
\pgfpathlineto{\pgfqpoint{2.353497in}{3.076657in}}%
\pgfpathlineto{\pgfqpoint{2.343206in}{3.110248in}}%
\pgfpathlineto{\pgfqpoint{2.328975in}{3.123479in}}%
\pgfpathlineto{\pgfqpoint{2.314744in}{3.136752in}}%
\pgfpathlineto{\pgfqpoint{2.300513in}{3.150069in}}%
\pgfpathlineto{\pgfqpoint{2.286282in}{3.163430in}}%
\pgfpathlineto{\pgfqpoint{2.296653in}{3.129310in}}%
\pgfpathlineto{\pgfqpoint{2.306970in}{3.096132in}}%
\pgfpathlineto{\pgfqpoint{2.317235in}{3.063879in}}%
\pgfpathlineto{\pgfqpoint{2.327450in}{3.032534in}}%
\pgfpathclose%
\pgfusepath{fill}%
\end{pgfscope}%
\begin{pgfscope}%
\pgfpathrectangle{\pgfqpoint{1.150000in}{0.150000in}}{\pgfqpoint{5.700000in}{5.700000in}}%
\pgfusepath{clip}%
\pgfsetbuttcap%
\pgfsetroundjoin%
\definecolor{currentfill}{rgb}{0.279566,0.067836,0.391917}%
\pgfsetfillcolor{currentfill}%
\pgfsetfillopacity{0.700000}%
\pgfsetlinewidth{0.000000pt}%
\definecolor{currentstroke}{rgb}{0.000000,0.000000,0.000000}%
\pgfsetstrokecolor{currentstroke}%
\pgfsetdash{}{0pt}%
\pgfpathmoveto{\pgfqpoint{4.785167in}{1.313126in}}%
\pgfpathlineto{\pgfqpoint{4.799608in}{1.307769in}}%
\pgfpathlineto{\pgfqpoint{4.814056in}{1.302436in}}%
\pgfpathlineto{\pgfqpoint{4.828511in}{1.297125in}}%
\pgfpathlineto{\pgfqpoint{4.842973in}{1.291838in}}%
\pgfpathlineto{\pgfqpoint{4.835096in}{1.288951in}}%
\pgfpathlineto{\pgfqpoint{4.827216in}{1.286439in}}%
\pgfpathlineto{\pgfqpoint{4.819334in}{1.284312in}}%
\pgfpathlineto{\pgfqpoint{4.811450in}{1.282578in}}%
\pgfpathlineto{\pgfqpoint{4.796971in}{1.288236in}}%
\pgfpathlineto{\pgfqpoint{4.782498in}{1.293918in}}%
\pgfpathlineto{\pgfqpoint{4.768032in}{1.299622in}}%
\pgfpathlineto{\pgfqpoint{4.753572in}{1.305349in}}%
\pgfpathlineto{\pgfqpoint{4.761475in}{1.306707in}}%
\pgfpathlineto{\pgfqpoint{4.769375in}{1.308462in}}%
\pgfpathlineto{\pgfqpoint{4.777272in}{1.310605in}}%
\pgfpathlineto{\pgfqpoint{4.785167in}{1.313126in}}%
\pgfpathclose%
\pgfusepath{fill}%
\end{pgfscope}%
\begin{pgfscope}%
\pgfpathrectangle{\pgfqpoint{1.150000in}{0.150000in}}{\pgfqpoint{5.700000in}{5.700000in}}%
\pgfusepath{clip}%
\pgfsetbuttcap%
\pgfsetroundjoin%
\definecolor{currentfill}{rgb}{0.204903,0.375746,0.553533}%
\pgfsetfillcolor{currentfill}%
\pgfsetfillopacity{0.700000}%
\pgfsetlinewidth{0.000000pt}%
\definecolor{currentstroke}{rgb}{0.000000,0.000000,0.000000}%
\pgfsetstrokecolor{currentstroke}%
\pgfsetdash{}{0pt}%
\pgfpathmoveto{\pgfqpoint{3.534711in}{1.975860in}}%
\pgfpathlineto{\pgfqpoint{3.548918in}{1.966635in}}%
\pgfpathlineto{\pgfqpoint{3.563128in}{1.957438in}}%
\pgfpathlineto{\pgfqpoint{3.577342in}{1.948267in}}%
\pgfpathlineto{\pgfqpoint{3.591560in}{1.939122in}}%
\pgfpathlineto{\pgfqpoint{3.582898in}{1.954190in}}%
\pgfpathlineto{\pgfqpoint{3.574215in}{1.969938in}}%
\pgfpathlineto{\pgfqpoint{3.565511in}{1.986380in}}%
\pgfpathlineto{\pgfqpoint{3.556785in}{2.003530in}}%
\pgfpathlineto{\pgfqpoint{3.542519in}{2.013130in}}%
\pgfpathlineto{\pgfqpoint{3.528256in}{2.022756in}}%
\pgfpathlineto{\pgfqpoint{3.513997in}{2.032409in}}%
\pgfpathlineto{\pgfqpoint{3.499741in}{2.042088in}}%
\pgfpathlineto{\pgfqpoint{3.508517in}{2.024476in}}%
\pgfpathlineto{\pgfqpoint{3.517271in}{2.007577in}}%
\pgfpathlineto{\pgfqpoint{3.526002in}{1.991376in}}%
\pgfpathlineto{\pgfqpoint{3.534711in}{1.975860in}}%
\pgfpathclose%
\pgfusepath{fill}%
\end{pgfscope}%
\begin{pgfscope}%
\pgfpathrectangle{\pgfqpoint{1.150000in}{0.150000in}}{\pgfqpoint{5.700000in}{5.700000in}}%
\pgfusepath{clip}%
\pgfsetbuttcap%
\pgfsetroundjoin%
\definecolor{currentfill}{rgb}{0.275191,0.194905,0.496005}%
\pgfsetfillcolor{currentfill}%
\pgfsetfillopacity{0.700000}%
\pgfsetlinewidth{0.000000pt}%
\definecolor{currentstroke}{rgb}{0.000000,0.000000,0.000000}%
\pgfsetstrokecolor{currentstroke}%
\pgfsetdash{}{0pt}%
\pgfpathmoveto{\pgfqpoint{4.171906in}{1.560089in}}%
\pgfpathlineto{\pgfqpoint{4.186210in}{1.552729in}}%
\pgfpathlineto{\pgfqpoint{4.200518in}{1.545394in}}%
\pgfpathlineto{\pgfqpoint{4.214832in}{1.538082in}}%
\pgfpathlineto{\pgfqpoint{4.229151in}{1.530794in}}%
\pgfpathlineto{\pgfqpoint{4.220994in}{1.536772in}}%
\pgfpathlineto{\pgfqpoint{4.212828in}{1.543287in}}%
\pgfpathlineto{\pgfqpoint{4.204652in}{1.550350in}}%
\pgfpathlineto{\pgfqpoint{4.196466in}{1.557974in}}%
\pgfpathlineto{\pgfqpoint{4.182114in}{1.565678in}}%
\pgfpathlineto{\pgfqpoint{4.167767in}{1.573406in}}%
\pgfpathlineto{\pgfqpoint{4.153425in}{1.581159in}}%
\pgfpathlineto{\pgfqpoint{4.139089in}{1.588935in}}%
\pgfpathlineto{\pgfqpoint{4.147309in}{1.580888in}}%
\pgfpathlineto{\pgfqpoint{4.155518in}{1.573406in}}%
\pgfpathlineto{\pgfqpoint{4.163717in}{1.566477in}}%
\pgfpathlineto{\pgfqpoint{4.171906in}{1.560089in}}%
\pgfpathclose%
\pgfusepath{fill}%
\end{pgfscope}%
\begin{pgfscope}%
\pgfpathrectangle{\pgfqpoint{1.150000in}{0.150000in}}{\pgfqpoint{5.700000in}{5.700000in}}%
\pgfusepath{clip}%
\pgfsetbuttcap%
\pgfsetroundjoin%
\definecolor{currentfill}{rgb}{0.232815,0.732247,0.459277}%
\pgfsetfillcolor{currentfill}%
\pgfsetfillopacity{0.700000}%
\pgfsetlinewidth{0.000000pt}%
\definecolor{currentstroke}{rgb}{0.000000,0.000000,0.000000}%
\pgfsetstrokecolor{currentstroke}%
\pgfsetdash{}{0pt}%
\pgfpathmoveto{\pgfqpoint{2.384064in}{2.981432in}}%
\pgfpathlineto{\pgfqpoint{2.398219in}{2.968764in}}%
\pgfpathlineto{\pgfqpoint{2.412374in}{2.956137in}}%
\pgfpathlineto{\pgfqpoint{2.426531in}{2.943552in}}%
\pgfpathlineto{\pgfqpoint{2.440687in}{2.931009in}}%
\pgfpathlineto{\pgfqpoint{2.430623in}{2.961333in}}%
\pgfpathlineto{\pgfqpoint{2.420510in}{2.992554in}}%
\pgfpathlineto{\pgfqpoint{2.410348in}{3.024688in}}%
\pgfpathlineto{\pgfqpoint{2.400136in}{3.057753in}}%
\pgfpathlineto{\pgfqpoint{2.385903in}{3.070814in}}%
\pgfpathlineto{\pgfqpoint{2.371670in}{3.083916in}}%
\pgfpathlineto{\pgfqpoint{2.357438in}{3.097061in}}%
\pgfpathlineto{\pgfqpoint{2.343206in}{3.110248in}}%
\pgfpathlineto{\pgfqpoint{2.353497in}{3.076657in}}%
\pgfpathlineto{\pgfqpoint{2.363736in}{3.044002in}}%
\pgfpathlineto{\pgfqpoint{2.373925in}{3.012266in}}%
\pgfpathlineto{\pgfqpoint{2.384064in}{2.981432in}}%
\pgfpathclose%
\pgfusepath{fill}%
\end{pgfscope}%
\begin{pgfscope}%
\pgfpathrectangle{\pgfqpoint{1.150000in}{0.150000in}}{\pgfqpoint{5.700000in}{5.700000in}}%
\pgfusepath{clip}%
\pgfsetbuttcap%
\pgfsetroundjoin%
\definecolor{currentfill}{rgb}{0.150476,0.504369,0.557430}%
\pgfsetfillcolor{currentfill}%
\pgfsetfillopacity{0.700000}%
\pgfsetlinewidth{0.000000pt}%
\definecolor{currentstroke}{rgb}{0.000000,0.000000,0.000000}%
\pgfsetstrokecolor{currentstroke}%
\pgfsetdash{}{0pt}%
\pgfpathmoveto{\pgfqpoint{3.101864in}{2.324491in}}%
\pgfpathlineto{\pgfqpoint{3.116035in}{2.314012in}}%
\pgfpathlineto{\pgfqpoint{3.130208in}{2.303563in}}%
\pgfpathlineto{\pgfqpoint{3.144384in}{2.293145in}}%
\pgfpathlineto{\pgfqpoint{3.158563in}{2.282756in}}%
\pgfpathlineto{\pgfqpoint{3.149437in}{2.303912in}}%
\pgfpathlineto{\pgfqpoint{3.140281in}{2.325838in}}%
\pgfpathlineto{\pgfqpoint{3.131094in}{2.348549in}}%
\pgfpathlineto{\pgfqpoint{3.121875in}{2.372059in}}%
\pgfpathlineto{\pgfqpoint{3.107638in}{2.382928in}}%
\pgfpathlineto{\pgfqpoint{3.093402in}{2.393826in}}%
\pgfpathlineto{\pgfqpoint{3.079169in}{2.404755in}}%
\pgfpathlineto{\pgfqpoint{3.064939in}{2.415714in}}%
\pgfpathlineto{\pgfqpoint{3.074219in}{2.391715in}}%
\pgfpathlineto{\pgfqpoint{3.083466in}{2.368522in}}%
\pgfpathlineto{\pgfqpoint{3.092681in}{2.346119in}}%
\pgfpathlineto{\pgfqpoint{3.101864in}{2.324491in}}%
\pgfpathclose%
\pgfusepath{fill}%
\end{pgfscope}%
\begin{pgfscope}%
\pgfpathrectangle{\pgfqpoint{1.150000in}{0.150000in}}{\pgfqpoint{5.700000in}{5.700000in}}%
\pgfusepath{clip}%
\pgfsetbuttcap%
\pgfsetroundjoin%
\definecolor{currentfill}{rgb}{0.252194,0.269783,0.531579}%
\pgfsetfillcolor{currentfill}%
\pgfsetfillopacity{0.700000}%
\pgfsetlinewidth{0.000000pt}%
\definecolor{currentstroke}{rgb}{0.000000,0.000000,0.000000}%
\pgfsetstrokecolor{currentstroke}%
\pgfsetdash{}{0pt}%
\pgfpathmoveto{\pgfqpoint{3.910344in}{1.716677in}}%
\pgfpathlineto{\pgfqpoint{3.924606in}{1.708508in}}%
\pgfpathlineto{\pgfqpoint{3.938872in}{1.700364in}}%
\pgfpathlineto{\pgfqpoint{3.953143in}{1.692245in}}%
\pgfpathlineto{\pgfqpoint{3.967419in}{1.684150in}}%
\pgfpathlineto{\pgfqpoint{3.959078in}{1.694056in}}%
\pgfpathlineto{\pgfqpoint{3.950723in}{1.704564in}}%
\pgfpathlineto{\pgfqpoint{3.942354in}{1.715686in}}%
\pgfpathlineto{\pgfqpoint{3.933971in}{1.727435in}}%
\pgfpathlineto{\pgfqpoint{3.919656in}{1.735964in}}%
\pgfpathlineto{\pgfqpoint{3.905346in}{1.744517in}}%
\pgfpathlineto{\pgfqpoint{3.891040in}{1.753096in}}%
\pgfpathlineto{\pgfqpoint{3.876738in}{1.761699in}}%
\pgfpathlineto{\pgfqpoint{3.885162in}{1.749509in}}%
\pgfpathlineto{\pgfqpoint{3.893571in}{1.737950in}}%
\pgfpathlineto{\pgfqpoint{3.901965in}{1.727011in}}%
\pgfpathlineto{\pgfqpoint{3.910344in}{1.716677in}}%
\pgfpathclose%
\pgfusepath{fill}%
\end{pgfscope}%
\begin{pgfscope}%
\pgfpathrectangle{\pgfqpoint{1.150000in}{0.150000in}}{\pgfqpoint{5.700000in}{5.700000in}}%
\pgfusepath{clip}%
\pgfsetbuttcap%
\pgfsetroundjoin%
\definecolor{currentfill}{rgb}{0.282327,0.094955,0.417331}%
\pgfsetfillcolor{currentfill}%
\pgfsetfillopacity{0.700000}%
\pgfsetlinewidth{0.000000pt}%
\definecolor{currentstroke}{rgb}{0.000000,0.000000,0.000000}%
\pgfsetstrokecolor{currentstroke}%
\pgfsetdash{}{0pt}%
\pgfpathmoveto{\pgfqpoint{4.638121in}{1.352002in}}%
\pgfpathlineto{\pgfqpoint{4.652531in}{1.346090in}}%
\pgfpathlineto{\pgfqpoint{4.666947in}{1.340200in}}%
\pgfpathlineto{\pgfqpoint{4.681368in}{1.334334in}}%
\pgfpathlineto{\pgfqpoint{4.695797in}{1.328491in}}%
\pgfpathlineto{\pgfqpoint{4.687870in}{1.327921in}}%
\pgfpathlineto{\pgfqpoint{4.679941in}{1.327774in}}%
\pgfpathlineto{\pgfqpoint{4.672007in}{1.328060in}}%
\pgfpathlineto{\pgfqpoint{4.664070in}{1.328788in}}%
\pgfpathlineto{\pgfqpoint{4.649620in}{1.335017in}}%
\pgfpathlineto{\pgfqpoint{4.635176in}{1.341268in}}%
\pgfpathlineto{\pgfqpoint{4.620739in}{1.347543in}}%
\pgfpathlineto{\pgfqpoint{4.606307in}{1.353841in}}%
\pgfpathlineto{\pgfqpoint{4.614268in}{1.352722in}}%
\pgfpathlineto{\pgfqpoint{4.622223in}{1.352049in}}%
\pgfpathlineto{\pgfqpoint{4.630174in}{1.351813in}}%
\pgfpathlineto{\pgfqpoint{4.638121in}{1.352002in}}%
\pgfpathclose%
\pgfusepath{fill}%
\end{pgfscope}%
\begin{pgfscope}%
\pgfpathrectangle{\pgfqpoint{1.150000in}{0.150000in}}{\pgfqpoint{5.700000in}{5.700000in}}%
\pgfusepath{clip}%
\pgfsetbuttcap%
\pgfsetroundjoin%
\definecolor{currentfill}{rgb}{0.282884,0.135920,0.453427}%
\pgfsetfillcolor{currentfill}%
\pgfsetfillopacity{0.700000}%
\pgfsetlinewidth{0.000000pt}%
\definecolor{currentstroke}{rgb}{0.000000,0.000000,0.000000}%
\pgfsetstrokecolor{currentstroke}%
\pgfsetdash{}{0pt}%
\pgfpathmoveto{\pgfqpoint{4.433589in}{1.431237in}}%
\pgfpathlineto{\pgfqpoint{4.447950in}{1.424658in}}%
\pgfpathlineto{\pgfqpoint{4.462317in}{1.418103in}}%
\pgfpathlineto{\pgfqpoint{4.476690in}{1.411572in}}%
\pgfpathlineto{\pgfqpoint{4.491069in}{1.405064in}}%
\pgfpathlineto{\pgfqpoint{4.483055in}{1.407430in}}%
\pgfpathlineto{\pgfqpoint{4.475036in}{1.410272in}}%
\pgfpathlineto{\pgfqpoint{4.467010in}{1.413601in}}%
\pgfpathlineto{\pgfqpoint{4.458978in}{1.417428in}}%
\pgfpathlineto{\pgfqpoint{4.444573in}{1.424336in}}%
\pgfpathlineto{\pgfqpoint{4.430174in}{1.431268in}}%
\pgfpathlineto{\pgfqpoint{4.415780in}{1.438223in}}%
\pgfpathlineto{\pgfqpoint{4.401391in}{1.445202in}}%
\pgfpathlineto{\pgfqpoint{4.409451in}{1.440969in}}%
\pgfpathlineto{\pgfqpoint{4.417504in}{1.437238in}}%
\pgfpathlineto{\pgfqpoint{4.425550in}{1.433997in}}%
\pgfpathlineto{\pgfqpoint{4.433589in}{1.431237in}}%
\pgfpathclose%
\pgfusepath{fill}%
\end{pgfscope}%
\begin{pgfscope}%
\pgfpathrectangle{\pgfqpoint{1.150000in}{0.150000in}}{\pgfqpoint{5.700000in}{5.700000in}}%
\pgfusepath{clip}%
\pgfsetbuttcap%
\pgfsetroundjoin%
\definecolor{currentfill}{rgb}{0.202219,0.715272,0.476084}%
\pgfsetfillcolor{currentfill}%
\pgfsetfillopacity{0.700000}%
\pgfsetlinewidth{0.000000pt}%
\definecolor{currentstroke}{rgb}{0.000000,0.000000,0.000000}%
\pgfsetstrokecolor{currentstroke}%
\pgfsetdash{}{0pt}%
\pgfpathmoveto{\pgfqpoint{2.440687in}{2.931009in}}%
\pgfpathlineto{\pgfqpoint{2.454845in}{2.918507in}}%
\pgfpathlineto{\pgfqpoint{2.469004in}{2.906045in}}%
\pgfpathlineto{\pgfqpoint{2.483163in}{2.893623in}}%
\pgfpathlineto{\pgfqpoint{2.497324in}{2.881242in}}%
\pgfpathlineto{\pgfqpoint{2.487332in}{2.911058in}}%
\pgfpathlineto{\pgfqpoint{2.477294in}{2.941766in}}%
\pgfpathlineto{\pgfqpoint{2.467209in}{2.973381in}}%
\pgfpathlineto{\pgfqpoint{2.457074in}{3.005921in}}%
\pgfpathlineto{\pgfqpoint{2.442839in}{3.018818in}}%
\pgfpathlineto{\pgfqpoint{2.428604in}{3.031755in}}%
\pgfpathlineto{\pgfqpoint{2.414370in}{3.044733in}}%
\pgfpathlineto{\pgfqpoint{2.400136in}{3.057753in}}%
\pgfpathlineto{\pgfqpoint{2.410348in}{3.024688in}}%
\pgfpathlineto{\pgfqpoint{2.420510in}{2.992554in}}%
\pgfpathlineto{\pgfqpoint{2.430623in}{2.961333in}}%
\pgfpathlineto{\pgfqpoint{2.440687in}{2.931009in}}%
\pgfpathclose%
\pgfusepath{fill}%
\end{pgfscope}%
\begin{pgfscope}%
\pgfpathrectangle{\pgfqpoint{1.150000in}{0.150000in}}{\pgfqpoint{5.700000in}{5.700000in}}%
\pgfusepath{clip}%
\pgfsetbuttcap%
\pgfsetroundjoin%
\definecolor{currentfill}{rgb}{0.277018,0.050344,0.375715}%
\pgfsetfillcolor{currentfill}%
\pgfsetfillopacity{0.700000}%
\pgfsetlinewidth{0.000000pt}%
\definecolor{currentstroke}{rgb}{0.000000,0.000000,0.000000}%
\pgfsetstrokecolor{currentstroke}%
\pgfsetdash{}{0pt}%
\pgfpathmoveto{\pgfqpoint{4.990273in}{1.268324in}}%
\pgfpathlineto{\pgfqpoint{5.004781in}{1.263601in}}%
\pgfpathlineto{\pgfqpoint{5.019297in}{1.258901in}}%
\pgfpathlineto{\pgfqpoint{5.033819in}{1.254225in}}%
\pgfpathlineto{\pgfqpoint{5.025990in}{1.248969in}}%
\pgfpathlineto{\pgfqpoint{5.018159in}{1.244040in}}%
\pgfpathlineto{\pgfqpoint{5.010328in}{1.239446in}}%
\pgfpathlineto{\pgfqpoint{5.002495in}{1.235199in}}%
\pgfpathlineto{\pgfqpoint{4.987959in}{1.240233in}}%
\pgfpathlineto{\pgfqpoint{4.973429in}{1.245290in}}%
\pgfpathlineto{\pgfqpoint{4.958907in}{1.250370in}}%
\pgfpathlineto{\pgfqpoint{4.966750in}{1.254346in}}%
\pgfpathlineto{\pgfqpoint{4.974593in}{1.258670in}}%
\pgfpathlineto{\pgfqpoint{4.982433in}{1.263332in}}%
\pgfpathlineto{\pgfqpoint{4.990273in}{1.268324in}}%
\pgfpathclose%
\pgfusepath{fill}%
\end{pgfscope}%
\begin{pgfscope}%
\pgfpathrectangle{\pgfqpoint{1.150000in}{0.150000in}}{\pgfqpoint{5.700000in}{5.700000in}}%
\pgfusepath{clip}%
\pgfsetbuttcap%
\pgfsetroundjoin%
\definecolor{currentfill}{rgb}{0.210503,0.363727,0.552206}%
\pgfsetfillcolor{currentfill}%
\pgfsetfillopacity{0.700000}%
\pgfsetlinewidth{0.000000pt}%
\definecolor{currentstroke}{rgb}{0.000000,0.000000,0.000000}%
\pgfsetstrokecolor{currentstroke}%
\pgfsetdash{}{0pt}%
\pgfpathmoveto{\pgfqpoint{3.591560in}{1.939122in}}%
\pgfpathlineto{\pgfqpoint{3.605782in}{1.930004in}}%
\pgfpathlineto{\pgfqpoint{3.620007in}{1.920913in}}%
\pgfpathlineto{\pgfqpoint{3.634236in}{1.911848in}}%
\pgfpathlineto{\pgfqpoint{3.648469in}{1.902809in}}%
\pgfpathlineto{\pgfqpoint{3.639854in}{1.917429in}}%
\pgfpathlineto{\pgfqpoint{3.631218in}{1.932725in}}%
\pgfpathlineto{\pgfqpoint{3.622562in}{1.948711in}}%
\pgfpathlineto{\pgfqpoint{3.613885in}{1.965399in}}%
\pgfpathlineto{\pgfqpoint{3.599605in}{1.974892in}}%
\pgfpathlineto{\pgfqpoint{3.585328in}{1.984412in}}%
\pgfpathlineto{\pgfqpoint{3.571055in}{1.993958in}}%
\pgfpathlineto{\pgfqpoint{3.556785in}{2.003530in}}%
\pgfpathlineto{\pgfqpoint{3.565511in}{1.986380in}}%
\pgfpathlineto{\pgfqpoint{3.574215in}{1.969938in}}%
\pgfpathlineto{\pgfqpoint{3.582898in}{1.954190in}}%
\pgfpathlineto{\pgfqpoint{3.591560in}{1.939122in}}%
\pgfpathclose%
\pgfusepath{fill}%
\end{pgfscope}%
\begin{pgfscope}%
\pgfpathrectangle{\pgfqpoint{1.150000in}{0.150000in}}{\pgfqpoint{5.700000in}{5.700000in}}%
\pgfusepath{clip}%
\pgfsetbuttcap%
\pgfsetroundjoin%
\definecolor{currentfill}{rgb}{0.180653,0.701402,0.488189}%
\pgfsetfillcolor{currentfill}%
\pgfsetfillopacity{0.700000}%
\pgfsetlinewidth{0.000000pt}%
\definecolor{currentstroke}{rgb}{0.000000,0.000000,0.000000}%
\pgfsetstrokecolor{currentstroke}%
\pgfsetdash{}{0pt}%
\pgfpathmoveto{\pgfqpoint{2.497324in}{2.881242in}}%
\pgfpathlineto{\pgfqpoint{2.511485in}{2.868900in}}%
\pgfpathlineto{\pgfqpoint{2.525647in}{2.856598in}}%
\pgfpathlineto{\pgfqpoint{2.539811in}{2.844335in}}%
\pgfpathlineto{\pgfqpoint{2.553975in}{2.832110in}}%
\pgfpathlineto{\pgfqpoint{2.544056in}{2.861421in}}%
\pgfpathlineto{\pgfqpoint{2.534092in}{2.891616in}}%
\pgfpathlineto{\pgfqpoint{2.524081in}{2.922714in}}%
\pgfpathlineto{\pgfqpoint{2.514023in}{2.954731in}}%
\pgfpathlineto{\pgfqpoint{2.499785in}{2.967469in}}%
\pgfpathlineto{\pgfqpoint{2.485547in}{2.980247in}}%
\pgfpathlineto{\pgfqpoint{2.471310in}{2.993064in}}%
\pgfpathlineto{\pgfqpoint{2.457074in}{3.005921in}}%
\pgfpathlineto{\pgfqpoint{2.467209in}{2.973381in}}%
\pgfpathlineto{\pgfqpoint{2.477294in}{2.941766in}}%
\pgfpathlineto{\pgfqpoint{2.487332in}{2.911058in}}%
\pgfpathlineto{\pgfqpoint{2.497324in}{2.881242in}}%
\pgfpathclose%
\pgfusepath{fill}%
\end{pgfscope}%
\begin{pgfscope}%
\pgfpathrectangle{\pgfqpoint{1.150000in}{0.150000in}}{\pgfqpoint{5.700000in}{5.700000in}}%
\pgfusepath{clip}%
\pgfsetbuttcap%
\pgfsetroundjoin%
\definecolor{currentfill}{rgb}{0.154815,0.493313,0.557840}%
\pgfsetfillcolor{currentfill}%
\pgfsetfillopacity{0.700000}%
\pgfsetlinewidth{0.000000pt}%
\definecolor{currentstroke}{rgb}{0.000000,0.000000,0.000000}%
\pgfsetstrokecolor{currentstroke}%
\pgfsetdash{}{0pt}%
\pgfpathmoveto{\pgfqpoint{3.158563in}{2.282756in}}%
\pgfpathlineto{\pgfqpoint{3.172745in}{2.272398in}}%
\pgfpathlineto{\pgfqpoint{3.186929in}{2.262069in}}%
\pgfpathlineto{\pgfqpoint{3.201116in}{2.251770in}}%
\pgfpathlineto{\pgfqpoint{3.215306in}{2.241500in}}%
\pgfpathlineto{\pgfqpoint{3.206237in}{2.262184in}}%
\pgfpathlineto{\pgfqpoint{3.197139in}{2.283633in}}%
\pgfpathlineto{\pgfqpoint{3.188011in}{2.305862in}}%
\pgfpathlineto{\pgfqpoint{3.178852in}{2.328886in}}%
\pgfpathlineto{\pgfqpoint{3.164604in}{2.339635in}}%
\pgfpathlineto{\pgfqpoint{3.150359in}{2.350413in}}%
\pgfpathlineto{\pgfqpoint{3.136116in}{2.361221in}}%
\pgfpathlineto{\pgfqpoint{3.121875in}{2.372059in}}%
\pgfpathlineto{\pgfqpoint{3.131094in}{2.348549in}}%
\pgfpathlineto{\pgfqpoint{3.140281in}{2.325838in}}%
\pgfpathlineto{\pgfqpoint{3.149437in}{2.303912in}}%
\pgfpathlineto{\pgfqpoint{3.158563in}{2.282756in}}%
\pgfpathclose%
\pgfusepath{fill}%
\end{pgfscope}%
\begin{pgfscope}%
\pgfpathrectangle{\pgfqpoint{1.150000in}{0.150000in}}{\pgfqpoint{5.700000in}{5.700000in}}%
\pgfusepath{clip}%
\pgfsetbuttcap%
\pgfsetroundjoin%
\definecolor{currentfill}{rgb}{0.279566,0.067836,0.391917}%
\pgfsetfillcolor{currentfill}%
\pgfsetfillopacity{0.700000}%
\pgfsetlinewidth{0.000000pt}%
\definecolor{currentstroke}{rgb}{0.000000,0.000000,0.000000}%
\pgfsetstrokecolor{currentstroke}%
\pgfsetdash{}{0pt}%
\pgfpathmoveto{\pgfqpoint{4.842973in}{1.291838in}}%
\pgfpathlineto{\pgfqpoint{4.857441in}{1.286574in}}%
\pgfpathlineto{\pgfqpoint{4.871916in}{1.281333in}}%
\pgfpathlineto{\pgfqpoint{4.886397in}{1.276115in}}%
\pgfpathlineto{\pgfqpoint{4.900885in}{1.270920in}}%
\pgfpathlineto{\pgfqpoint{4.893025in}{1.267667in}}%
\pgfpathlineto{\pgfqpoint{4.885162in}{1.264786in}}%
\pgfpathlineto{\pgfqpoint{4.877298in}{1.262285in}}%
\pgfpathlineto{\pgfqpoint{4.869431in}{1.260175in}}%
\pgfpathlineto{\pgfqpoint{4.854926in}{1.265741in}}%
\pgfpathlineto{\pgfqpoint{4.840428in}{1.271331in}}%
\pgfpathlineto{\pgfqpoint{4.825936in}{1.276943in}}%
\pgfpathlineto{\pgfqpoint{4.811450in}{1.282578in}}%
\pgfpathlineto{\pgfqpoint{4.819334in}{1.284312in}}%
\pgfpathlineto{\pgfqpoint{4.827216in}{1.286439in}}%
\pgfpathlineto{\pgfqpoint{4.835096in}{1.288951in}}%
\pgfpathlineto{\pgfqpoint{4.842973in}{1.291838in}}%
\pgfpathclose%
\pgfusepath{fill}%
\end{pgfscope}%
\begin{pgfscope}%
\pgfpathrectangle{\pgfqpoint{1.150000in}{0.150000in}}{\pgfqpoint{5.700000in}{5.700000in}}%
\pgfusepath{clip}%
\pgfsetbuttcap%
\pgfsetroundjoin%
\definecolor{currentfill}{rgb}{0.276194,0.190074,0.493001}%
\pgfsetfillcolor{currentfill}%
\pgfsetfillopacity{0.700000}%
\pgfsetlinewidth{0.000000pt}%
\definecolor{currentstroke}{rgb}{0.000000,0.000000,0.000000}%
\pgfsetstrokecolor{currentstroke}%
\pgfsetdash{}{0pt}%
\pgfpathmoveto{\pgfqpoint{4.229151in}{1.530794in}}%
\pgfpathlineto{\pgfqpoint{4.243476in}{1.523531in}}%
\pgfpathlineto{\pgfqpoint{4.257805in}{1.516291in}}%
\pgfpathlineto{\pgfqpoint{4.272140in}{1.509075in}}%
\pgfpathlineto{\pgfqpoint{4.286480in}{1.501883in}}%
\pgfpathlineto{\pgfqpoint{4.278354in}{1.507451in}}%
\pgfpathlineto{\pgfqpoint{4.270219in}{1.513551in}}%
\pgfpathlineto{\pgfqpoint{4.262075in}{1.520196in}}%
\pgfpathlineto{\pgfqpoint{4.253922in}{1.527397in}}%
\pgfpathlineto{\pgfqpoint{4.239550in}{1.535006in}}%
\pgfpathlineto{\pgfqpoint{4.225184in}{1.542638in}}%
\pgfpathlineto{\pgfqpoint{4.210822in}{1.550294in}}%
\pgfpathlineto{\pgfqpoint{4.196466in}{1.557974in}}%
\pgfpathlineto{\pgfqpoint{4.204652in}{1.550350in}}%
\pgfpathlineto{\pgfqpoint{4.212828in}{1.543287in}}%
\pgfpathlineto{\pgfqpoint{4.220994in}{1.536772in}}%
\pgfpathlineto{\pgfqpoint{4.229151in}{1.530794in}}%
\pgfpathclose%
\pgfusepath{fill}%
\end{pgfscope}%
\begin{pgfscope}%
\pgfpathrectangle{\pgfqpoint{1.150000in}{0.150000in}}{\pgfqpoint{5.700000in}{5.700000in}}%
\pgfusepath{clip}%
\pgfsetbuttcap%
\pgfsetroundjoin%
\definecolor{currentfill}{rgb}{0.255645,0.260703,0.528312}%
\pgfsetfillcolor{currentfill}%
\pgfsetfillopacity{0.700000}%
\pgfsetlinewidth{0.000000pt}%
\definecolor{currentstroke}{rgb}{0.000000,0.000000,0.000000}%
\pgfsetstrokecolor{currentstroke}%
\pgfsetdash{}{0pt}%
\pgfpathmoveto{\pgfqpoint{3.967419in}{1.684150in}}%
\pgfpathlineto{\pgfqpoint{3.981699in}{1.676081in}}%
\pgfpathlineto{\pgfqpoint{3.995983in}{1.668036in}}%
\pgfpathlineto{\pgfqpoint{4.010273in}{1.660016in}}%
\pgfpathlineto{\pgfqpoint{4.024567in}{1.652020in}}%
\pgfpathlineto{\pgfqpoint{4.016263in}{1.661498in}}%
\pgfpathlineto{\pgfqpoint{4.007947in}{1.671574in}}%
\pgfpathlineto{\pgfqpoint{3.999617in}{1.682260in}}%
\pgfpathlineto{\pgfqpoint{3.991274in}{1.693569in}}%
\pgfpathlineto{\pgfqpoint{3.976941in}{1.701998in}}%
\pgfpathlineto{\pgfqpoint{3.962613in}{1.710453in}}%
\pgfpathlineto{\pgfqpoint{3.948290in}{1.718932in}}%
\pgfpathlineto{\pgfqpoint{3.933971in}{1.727435in}}%
\pgfpathlineto{\pgfqpoint{3.942354in}{1.715686in}}%
\pgfpathlineto{\pgfqpoint{3.950723in}{1.704564in}}%
\pgfpathlineto{\pgfqpoint{3.959078in}{1.694056in}}%
\pgfpathlineto{\pgfqpoint{3.967419in}{1.684150in}}%
\pgfpathclose%
\pgfusepath{fill}%
\end{pgfscope}%
\begin{pgfscope}%
\pgfpathrectangle{\pgfqpoint{1.150000in}{0.150000in}}{\pgfqpoint{5.700000in}{5.700000in}}%
\pgfusepath{clip}%
\pgfsetbuttcap%
\pgfsetroundjoin%
\definecolor{currentfill}{rgb}{0.162016,0.687316,0.499129}%
\pgfsetfillcolor{currentfill}%
\pgfsetfillopacity{0.700000}%
\pgfsetlinewidth{0.000000pt}%
\definecolor{currentstroke}{rgb}{0.000000,0.000000,0.000000}%
\pgfsetstrokecolor{currentstroke}%
\pgfsetdash{}{0pt}%
\pgfpathmoveto{\pgfqpoint{2.553975in}{2.832110in}}%
\pgfpathlineto{\pgfqpoint{2.568141in}{2.819924in}}%
\pgfpathlineto{\pgfqpoint{2.582308in}{2.807777in}}%
\pgfpathlineto{\pgfqpoint{2.596476in}{2.795667in}}%
\pgfpathlineto{\pgfqpoint{2.610645in}{2.783595in}}%
\pgfpathlineto{\pgfqpoint{2.600797in}{2.812401in}}%
\pgfpathlineto{\pgfqpoint{2.590906in}{2.842086in}}%
\pgfpathlineto{\pgfqpoint{2.580969in}{2.872668in}}%
\pgfpathlineto{\pgfqpoint{2.570986in}{2.904163in}}%
\pgfpathlineto{\pgfqpoint{2.556744in}{2.916748in}}%
\pgfpathlineto{\pgfqpoint{2.542503in}{2.929370in}}%
\pgfpathlineto{\pgfqpoint{2.528262in}{2.942031in}}%
\pgfpathlineto{\pgfqpoint{2.514023in}{2.954731in}}%
\pgfpathlineto{\pgfqpoint{2.524081in}{2.922714in}}%
\pgfpathlineto{\pgfqpoint{2.534092in}{2.891616in}}%
\pgfpathlineto{\pgfqpoint{2.544056in}{2.861421in}}%
\pgfpathlineto{\pgfqpoint{2.553975in}{2.832110in}}%
\pgfpathclose%
\pgfusepath{fill}%
\end{pgfscope}%
\begin{pgfscope}%
\pgfpathrectangle{\pgfqpoint{1.150000in}{0.150000in}}{\pgfqpoint{5.700000in}{5.700000in}}%
\pgfusepath{clip}%
\pgfsetbuttcap%
\pgfsetroundjoin%
\definecolor{currentfill}{rgb}{0.283072,0.130895,0.449241}%
\pgfsetfillcolor{currentfill}%
\pgfsetfillopacity{0.700000}%
\pgfsetlinewidth{0.000000pt}%
\definecolor{currentstroke}{rgb}{0.000000,0.000000,0.000000}%
\pgfsetstrokecolor{currentstroke}%
\pgfsetdash{}{0pt}%
\pgfpathmoveto{\pgfqpoint{4.491069in}{1.405064in}}%
\pgfpathlineto{\pgfqpoint{4.505453in}{1.398579in}}%
\pgfpathlineto{\pgfqpoint{4.519843in}{1.392118in}}%
\pgfpathlineto{\pgfqpoint{4.534239in}{1.385680in}}%
\pgfpathlineto{\pgfqpoint{4.548641in}{1.379266in}}%
\pgfpathlineto{\pgfqpoint{4.540652in}{1.381237in}}%
\pgfpathlineto{\pgfqpoint{4.532658in}{1.383681in}}%
\pgfpathlineto{\pgfqpoint{4.524659in}{1.386608in}}%
\pgfpathlineto{\pgfqpoint{4.516654in}{1.390029in}}%
\pgfpathlineto{\pgfqpoint{4.502226in}{1.396844in}}%
\pgfpathlineto{\pgfqpoint{4.487805in}{1.403682in}}%
\pgfpathlineto{\pgfqpoint{4.473388in}{1.410543in}}%
\pgfpathlineto{\pgfqpoint{4.458978in}{1.417428in}}%
\pgfpathlineto{\pgfqpoint{4.467010in}{1.413601in}}%
\pgfpathlineto{\pgfqpoint{4.475036in}{1.410272in}}%
\pgfpathlineto{\pgfqpoint{4.483055in}{1.407430in}}%
\pgfpathlineto{\pgfqpoint{4.491069in}{1.405064in}}%
\pgfpathclose%
\pgfusepath{fill}%
\end{pgfscope}%
\begin{pgfscope}%
\pgfpathrectangle{\pgfqpoint{1.150000in}{0.150000in}}{\pgfqpoint{5.700000in}{5.700000in}}%
\pgfusepath{clip}%
\pgfsetbuttcap%
\pgfsetroundjoin%
\definecolor{currentfill}{rgb}{0.214298,0.355619,0.551184}%
\pgfsetfillcolor{currentfill}%
\pgfsetfillopacity{0.700000}%
\pgfsetlinewidth{0.000000pt}%
\definecolor{currentstroke}{rgb}{0.000000,0.000000,0.000000}%
\pgfsetstrokecolor{currentstroke}%
\pgfsetdash{}{0pt}%
\pgfpathmoveto{\pgfqpoint{3.648469in}{1.902809in}}%
\pgfpathlineto{\pgfqpoint{3.662706in}{1.893796in}}%
\pgfpathlineto{\pgfqpoint{3.676947in}{1.884810in}}%
\pgfpathlineto{\pgfqpoint{3.691192in}{1.875849in}}%
\pgfpathlineto{\pgfqpoint{3.705440in}{1.866915in}}%
\pgfpathlineto{\pgfqpoint{3.696870in}{1.881088in}}%
\pgfpathlineto{\pgfqpoint{3.688280in}{1.895932in}}%
\pgfpathlineto{\pgfqpoint{3.679671in}{1.911461in}}%
\pgfpathlineto{\pgfqpoint{3.671043in}{1.927689in}}%
\pgfpathlineto{\pgfqpoint{3.656748in}{1.937077in}}%
\pgfpathlineto{\pgfqpoint{3.642457in}{1.946492in}}%
\pgfpathlineto{\pgfqpoint{3.628169in}{1.955932in}}%
\pgfpathlineto{\pgfqpoint{3.613885in}{1.965399in}}%
\pgfpathlineto{\pgfqpoint{3.622562in}{1.948711in}}%
\pgfpathlineto{\pgfqpoint{3.631218in}{1.932725in}}%
\pgfpathlineto{\pgfqpoint{3.639854in}{1.917429in}}%
\pgfpathlineto{\pgfqpoint{3.648469in}{1.902809in}}%
\pgfpathclose%
\pgfusepath{fill}%
\end{pgfscope}%
\begin{pgfscope}%
\pgfpathrectangle{\pgfqpoint{1.150000in}{0.150000in}}{\pgfqpoint{5.700000in}{5.700000in}}%
\pgfusepath{clip}%
\pgfsetbuttcap%
\pgfsetroundjoin%
\definecolor{currentfill}{rgb}{0.160665,0.478540,0.558115}%
\pgfsetfillcolor{currentfill}%
\pgfsetfillopacity{0.700000}%
\pgfsetlinewidth{0.000000pt}%
\definecolor{currentstroke}{rgb}{0.000000,0.000000,0.000000}%
\pgfsetstrokecolor{currentstroke}%
\pgfsetdash{}{0pt}%
\pgfpathmoveto{\pgfqpoint{3.215306in}{2.241500in}}%
\pgfpathlineto{\pgfqpoint{3.229499in}{2.231259in}}%
\pgfpathlineto{\pgfqpoint{3.243694in}{2.221048in}}%
\pgfpathlineto{\pgfqpoint{3.257893in}{2.210865in}}%
\pgfpathlineto{\pgfqpoint{3.272094in}{2.200712in}}%
\pgfpathlineto{\pgfqpoint{3.263081in}{2.220925in}}%
\pgfpathlineto{\pgfqpoint{3.254040in}{2.241898in}}%
\pgfpathlineto{\pgfqpoint{3.244970in}{2.263647in}}%
\pgfpathlineto{\pgfqpoint{3.235870in}{2.286185in}}%
\pgfpathlineto{\pgfqpoint{3.221612in}{2.296817in}}%
\pgfpathlineto{\pgfqpoint{3.207356in}{2.307477in}}%
\pgfpathlineto{\pgfqpoint{3.193103in}{2.318167in}}%
\pgfpathlineto{\pgfqpoint{3.178852in}{2.328886in}}%
\pgfpathlineto{\pgfqpoint{3.188011in}{2.305862in}}%
\pgfpathlineto{\pgfqpoint{3.197139in}{2.283633in}}%
\pgfpathlineto{\pgfqpoint{3.206237in}{2.262184in}}%
\pgfpathlineto{\pgfqpoint{3.215306in}{2.241500in}}%
\pgfpathclose%
\pgfusepath{fill}%
\end{pgfscope}%
\begin{pgfscope}%
\pgfpathrectangle{\pgfqpoint{1.150000in}{0.150000in}}{\pgfqpoint{5.700000in}{5.700000in}}%
\pgfusepath{clip}%
\pgfsetbuttcap%
\pgfsetroundjoin%
\definecolor{currentfill}{rgb}{0.281924,0.089666,0.412415}%
\pgfsetfillcolor{currentfill}%
\pgfsetfillopacity{0.700000}%
\pgfsetlinewidth{0.000000pt}%
\definecolor{currentstroke}{rgb}{0.000000,0.000000,0.000000}%
\pgfsetstrokecolor{currentstroke}%
\pgfsetdash{}{0pt}%
\pgfpathmoveto{\pgfqpoint{4.695797in}{1.328491in}}%
\pgfpathlineto{\pgfqpoint{4.710231in}{1.322671in}}%
\pgfpathlineto{\pgfqpoint{4.724672in}{1.316874in}}%
\pgfpathlineto{\pgfqpoint{4.739119in}{1.311100in}}%
\pgfpathlineto{\pgfqpoint{4.753572in}{1.305349in}}%
\pgfpathlineto{\pgfqpoint{4.745666in}{1.304400in}}%
\pgfpathlineto{\pgfqpoint{4.737757in}{1.303870in}}%
\pgfpathlineto{\pgfqpoint{4.729845in}{1.303768in}}%
\pgfpathlineto{\pgfqpoint{4.721929in}{1.304106in}}%
\pgfpathlineto{\pgfqpoint{4.707455in}{1.310242in}}%
\pgfpathlineto{\pgfqpoint{4.692987in}{1.316401in}}%
\pgfpathlineto{\pgfqpoint{4.678525in}{1.322583in}}%
\pgfpathlineto{\pgfqpoint{4.664070in}{1.328788in}}%
\pgfpathlineto{\pgfqpoint{4.672007in}{1.328060in}}%
\pgfpathlineto{\pgfqpoint{4.679941in}{1.327774in}}%
\pgfpathlineto{\pgfqpoint{4.687870in}{1.327921in}}%
\pgfpathlineto{\pgfqpoint{4.695797in}{1.328491in}}%
\pgfpathclose%
\pgfusepath{fill}%
\end{pgfscope}%
\begin{pgfscope}%
\pgfpathrectangle{\pgfqpoint{1.150000in}{0.150000in}}{\pgfqpoint{5.700000in}{5.700000in}}%
\pgfusepath{clip}%
\pgfsetbuttcap%
\pgfsetroundjoin%
\definecolor{currentfill}{rgb}{0.146616,0.673050,0.508936}%
\pgfsetfillcolor{currentfill}%
\pgfsetfillopacity{0.700000}%
\pgfsetlinewidth{0.000000pt}%
\definecolor{currentstroke}{rgb}{0.000000,0.000000,0.000000}%
\pgfsetstrokecolor{currentstroke}%
\pgfsetdash{}{0pt}%
\pgfpathmoveto{\pgfqpoint{2.610645in}{2.783595in}}%
\pgfpathlineto{\pgfqpoint{2.624815in}{2.771560in}}%
\pgfpathlineto{\pgfqpoint{2.638987in}{2.759562in}}%
\pgfpathlineto{\pgfqpoint{2.653160in}{2.747602in}}%
\pgfpathlineto{\pgfqpoint{2.667335in}{2.735677in}}%
\pgfpathlineto{\pgfqpoint{2.657558in}{2.763980in}}%
\pgfpathlineto{\pgfqpoint{2.647738in}{2.793157in}}%
\pgfpathlineto{\pgfqpoint{2.637874in}{2.823225in}}%
\pgfpathlineto{\pgfqpoint{2.627965in}{2.854200in}}%
\pgfpathlineto{\pgfqpoint{2.613719in}{2.866635in}}%
\pgfpathlineto{\pgfqpoint{2.599473in}{2.879107in}}%
\pgfpathlineto{\pgfqpoint{2.585229in}{2.891617in}}%
\pgfpathlineto{\pgfqpoint{2.570986in}{2.904163in}}%
\pgfpathlineto{\pgfqpoint{2.580969in}{2.872668in}}%
\pgfpathlineto{\pgfqpoint{2.590906in}{2.842086in}}%
\pgfpathlineto{\pgfqpoint{2.600797in}{2.812401in}}%
\pgfpathlineto{\pgfqpoint{2.610645in}{2.783595in}}%
\pgfpathclose%
\pgfusepath{fill}%
\end{pgfscope}%
\begin{pgfscope}%
\pgfpathrectangle{\pgfqpoint{1.150000in}{0.150000in}}{\pgfqpoint{5.700000in}{5.700000in}}%
\pgfusepath{clip}%
\pgfsetbuttcap%
\pgfsetroundjoin%
\definecolor{currentfill}{rgb}{0.278012,0.180367,0.486697}%
\pgfsetfillcolor{currentfill}%
\pgfsetfillopacity{0.700000}%
\pgfsetlinewidth{0.000000pt}%
\definecolor{currentstroke}{rgb}{0.000000,0.000000,0.000000}%
\pgfsetstrokecolor{currentstroke}%
\pgfsetdash{}{0pt}%
\pgfpathmoveto{\pgfqpoint{4.286480in}{1.501883in}}%
\pgfpathlineto{\pgfqpoint{4.300825in}{1.494715in}}%
\pgfpathlineto{\pgfqpoint{4.315176in}{1.487571in}}%
\pgfpathlineto{\pgfqpoint{4.329531in}{1.480450in}}%
\pgfpathlineto{\pgfqpoint{4.343893in}{1.473353in}}%
\pgfpathlineto{\pgfqpoint{4.335797in}{1.478510in}}%
\pgfpathlineto{\pgfqpoint{4.327693in}{1.484196in}}%
\pgfpathlineto{\pgfqpoint{4.319580in}{1.490423in}}%
\pgfpathlineto{\pgfqpoint{4.311459in}{1.497203in}}%
\pgfpathlineto{\pgfqpoint{4.297067in}{1.504716in}}%
\pgfpathlineto{\pgfqpoint{4.282680in}{1.512252in}}%
\pgfpathlineto{\pgfqpoint{4.268299in}{1.519813in}}%
\pgfpathlineto{\pgfqpoint{4.253922in}{1.527397in}}%
\pgfpathlineto{\pgfqpoint{4.262075in}{1.520196in}}%
\pgfpathlineto{\pgfqpoint{4.270219in}{1.513551in}}%
\pgfpathlineto{\pgfqpoint{4.278354in}{1.507451in}}%
\pgfpathlineto{\pgfqpoint{4.286480in}{1.501883in}}%
\pgfpathclose%
\pgfusepath{fill}%
\end{pgfscope}%
\begin{pgfscope}%
\pgfpathrectangle{\pgfqpoint{1.150000in}{0.150000in}}{\pgfqpoint{5.700000in}{5.700000in}}%
\pgfusepath{clip}%
\pgfsetbuttcap%
\pgfsetroundjoin%
\definecolor{currentfill}{rgb}{0.258965,0.251537,0.524736}%
\pgfsetfillcolor{currentfill}%
\pgfsetfillopacity{0.700000}%
\pgfsetlinewidth{0.000000pt}%
\definecolor{currentstroke}{rgb}{0.000000,0.000000,0.000000}%
\pgfsetstrokecolor{currentstroke}%
\pgfsetdash{}{0pt}%
\pgfpathmoveto{\pgfqpoint{4.024567in}{1.652020in}}%
\pgfpathlineto{\pgfqpoint{4.038865in}{1.644049in}}%
\pgfpathlineto{\pgfqpoint{4.053169in}{1.636103in}}%
\pgfpathlineto{\pgfqpoint{4.067477in}{1.628180in}}%
\pgfpathlineto{\pgfqpoint{4.081789in}{1.620283in}}%
\pgfpathlineto{\pgfqpoint{4.073523in}{1.629333in}}%
\pgfpathlineto{\pgfqpoint{4.065243in}{1.638978in}}%
\pgfpathlineto{\pgfqpoint{4.056952in}{1.649228in}}%
\pgfpathlineto{\pgfqpoint{4.048648in}{1.660097in}}%
\pgfpathlineto{\pgfqpoint{4.034297in}{1.668428in}}%
\pgfpathlineto{\pgfqpoint{4.019952in}{1.676784in}}%
\pgfpathlineto{\pgfqpoint{4.005610in}{1.685164in}}%
\pgfpathlineto{\pgfqpoint{3.991274in}{1.693569in}}%
\pgfpathlineto{\pgfqpoint{3.999617in}{1.682260in}}%
\pgfpathlineto{\pgfqpoint{4.007947in}{1.671574in}}%
\pgfpathlineto{\pgfqpoint{4.016263in}{1.661498in}}%
\pgfpathlineto{\pgfqpoint{4.024567in}{1.652020in}}%
\pgfpathclose%
\pgfusepath{fill}%
\end{pgfscope}%
\begin{pgfscope}%
\pgfpathrectangle{\pgfqpoint{1.150000in}{0.150000in}}{\pgfqpoint{5.700000in}{5.700000in}}%
\pgfusepath{clip}%
\pgfsetbuttcap%
\pgfsetroundjoin%
\definecolor{currentfill}{rgb}{0.278791,0.062145,0.386592}%
\pgfsetfillcolor{currentfill}%
\pgfsetfillopacity{0.700000}%
\pgfsetlinewidth{0.000000pt}%
\definecolor{currentstroke}{rgb}{0.000000,0.000000,0.000000}%
\pgfsetstrokecolor{currentstroke}%
\pgfsetdash{}{0pt}%
\pgfpathmoveto{\pgfqpoint{4.900885in}{1.270920in}}%
\pgfpathlineto{\pgfqpoint{4.915381in}{1.265748in}}%
\pgfpathlineto{\pgfqpoint{4.929882in}{1.260599in}}%
\pgfpathlineto{\pgfqpoint{4.944391in}{1.255473in}}%
\pgfpathlineto{\pgfqpoint{4.958907in}{1.250370in}}%
\pgfpathlineto{\pgfqpoint{4.951062in}{1.246751in}}%
\pgfpathlineto{\pgfqpoint{4.943215in}{1.243500in}}%
\pgfpathlineto{\pgfqpoint{4.935368in}{1.240627in}}%
\pgfpathlineto{\pgfqpoint{4.927518in}{1.238140in}}%
\pgfpathlineto{\pgfqpoint{4.912987in}{1.243614in}}%
\pgfpathlineto{\pgfqpoint{4.898462in}{1.249112in}}%
\pgfpathlineto{\pgfqpoint{4.883943in}{1.254632in}}%
\pgfpathlineto{\pgfqpoint{4.869431in}{1.260175in}}%
\pgfpathlineto{\pgfqpoint{4.877298in}{1.262285in}}%
\pgfpathlineto{\pgfqpoint{4.885162in}{1.264786in}}%
\pgfpathlineto{\pgfqpoint{4.893025in}{1.267667in}}%
\pgfpathlineto{\pgfqpoint{4.900885in}{1.270920in}}%
\pgfpathclose%
\pgfusepath{fill}%
\end{pgfscope}%
\begin{pgfscope}%
\pgfpathrectangle{\pgfqpoint{1.150000in}{0.150000in}}{\pgfqpoint{5.700000in}{5.700000in}}%
\pgfusepath{clip}%
\pgfsetbuttcap%
\pgfsetroundjoin%
\definecolor{currentfill}{rgb}{0.165117,0.467423,0.558141}%
\pgfsetfillcolor{currentfill}%
\pgfsetfillopacity{0.700000}%
\pgfsetlinewidth{0.000000pt}%
\definecolor{currentstroke}{rgb}{0.000000,0.000000,0.000000}%
\pgfsetstrokecolor{currentstroke}%
\pgfsetdash{}{0pt}%
\pgfpathmoveto{\pgfqpoint{3.272094in}{2.200712in}}%
\pgfpathlineto{\pgfqpoint{3.286299in}{2.190587in}}%
\pgfpathlineto{\pgfqpoint{3.300506in}{2.180491in}}%
\pgfpathlineto{\pgfqpoint{3.314717in}{2.170423in}}%
\pgfpathlineto{\pgfqpoint{3.328930in}{2.160384in}}%
\pgfpathlineto{\pgfqpoint{3.319972in}{2.180127in}}%
\pgfpathlineto{\pgfqpoint{3.310987in}{2.200626in}}%
\pgfpathlineto{\pgfqpoint{3.301974in}{2.221895in}}%
\pgfpathlineto{\pgfqpoint{3.292932in}{2.243948in}}%
\pgfpathlineto{\pgfqpoint{3.278663in}{2.254465in}}%
\pgfpathlineto{\pgfqpoint{3.264396in}{2.265009in}}%
\pgfpathlineto{\pgfqpoint{3.250132in}{2.275583in}}%
\pgfpathlineto{\pgfqpoint{3.235870in}{2.286185in}}%
\pgfpathlineto{\pgfqpoint{3.244970in}{2.263647in}}%
\pgfpathlineto{\pgfqpoint{3.254040in}{2.241898in}}%
\pgfpathlineto{\pgfqpoint{3.263081in}{2.220925in}}%
\pgfpathlineto{\pgfqpoint{3.272094in}{2.200712in}}%
\pgfpathclose%
\pgfusepath{fill}%
\end{pgfscope}%
\begin{pgfscope}%
\pgfpathrectangle{\pgfqpoint{1.150000in}{0.150000in}}{\pgfqpoint{5.700000in}{5.700000in}}%
\pgfusepath{clip}%
\pgfsetbuttcap%
\pgfsetroundjoin%
\definecolor{currentfill}{rgb}{0.134692,0.658636,0.517649}%
\pgfsetfillcolor{currentfill}%
\pgfsetfillopacity{0.700000}%
\pgfsetlinewidth{0.000000pt}%
\definecolor{currentstroke}{rgb}{0.000000,0.000000,0.000000}%
\pgfsetstrokecolor{currentstroke}%
\pgfsetdash{}{0pt}%
\pgfpathmoveto{\pgfqpoint{2.667335in}{2.735677in}}%
\pgfpathlineto{\pgfqpoint{2.681511in}{2.723789in}}%
\pgfpathlineto{\pgfqpoint{2.695689in}{2.711937in}}%
\pgfpathlineto{\pgfqpoint{2.709868in}{2.700121in}}%
\pgfpathlineto{\pgfqpoint{2.724048in}{2.688341in}}%
\pgfpathlineto{\pgfqpoint{2.714341in}{2.716142in}}%
\pgfpathlineto{\pgfqpoint{2.704591in}{2.744812in}}%
\pgfpathlineto{\pgfqpoint{2.694799in}{2.774367in}}%
\pgfpathlineto{\pgfqpoint{2.684963in}{2.804824in}}%
\pgfpathlineto{\pgfqpoint{2.670712in}{2.817114in}}%
\pgfpathlineto{\pgfqpoint{2.656462in}{2.829440in}}%
\pgfpathlineto{\pgfqpoint{2.642213in}{2.841802in}}%
\pgfpathlineto{\pgfqpoint{2.627965in}{2.854200in}}%
\pgfpathlineto{\pgfqpoint{2.637874in}{2.823225in}}%
\pgfpathlineto{\pgfqpoint{2.647738in}{2.793157in}}%
\pgfpathlineto{\pgfqpoint{2.657558in}{2.763980in}}%
\pgfpathlineto{\pgfqpoint{2.667335in}{2.735677in}}%
\pgfpathclose%
\pgfusepath{fill}%
\end{pgfscope}%
\begin{pgfscope}%
\pgfpathrectangle{\pgfqpoint{1.150000in}{0.150000in}}{\pgfqpoint{5.700000in}{5.700000in}}%
\pgfusepath{clip}%
\pgfsetbuttcap%
\pgfsetroundjoin%
\definecolor{currentfill}{rgb}{0.220057,0.343307,0.549413}%
\pgfsetfillcolor{currentfill}%
\pgfsetfillopacity{0.700000}%
\pgfsetlinewidth{0.000000pt}%
\definecolor{currentstroke}{rgb}{0.000000,0.000000,0.000000}%
\pgfsetstrokecolor{currentstroke}%
\pgfsetdash{}{0pt}%
\pgfpathmoveto{\pgfqpoint{3.705440in}{1.866915in}}%
\pgfpathlineto{\pgfqpoint{3.719693in}{1.858006in}}%
\pgfpathlineto{\pgfqpoint{3.733949in}{1.849123in}}%
\pgfpathlineto{\pgfqpoint{3.748210in}{1.840266in}}%
\pgfpathlineto{\pgfqpoint{3.762474in}{1.831435in}}%
\pgfpathlineto{\pgfqpoint{3.753948in}{1.845161in}}%
\pgfpathlineto{\pgfqpoint{3.745404in}{1.859555in}}%
\pgfpathlineto{\pgfqpoint{3.736841in}{1.874628in}}%
\pgfpathlineto{\pgfqpoint{3.728260in}{1.890396in}}%
\pgfpathlineto{\pgfqpoint{3.713950in}{1.899680in}}%
\pgfpathlineto{\pgfqpoint{3.699644in}{1.908991in}}%
\pgfpathlineto{\pgfqpoint{3.685341in}{1.918327in}}%
\pgfpathlineto{\pgfqpoint{3.671043in}{1.927689in}}%
\pgfpathlineto{\pgfqpoint{3.679671in}{1.911461in}}%
\pgfpathlineto{\pgfqpoint{3.688280in}{1.895932in}}%
\pgfpathlineto{\pgfqpoint{3.696870in}{1.881088in}}%
\pgfpathlineto{\pgfqpoint{3.705440in}{1.866915in}}%
\pgfpathclose%
\pgfusepath{fill}%
\end{pgfscope}%
\begin{pgfscope}%
\pgfpathrectangle{\pgfqpoint{1.150000in}{0.150000in}}{\pgfqpoint{5.700000in}{5.700000in}}%
\pgfusepath{clip}%
\pgfsetbuttcap%
\pgfsetroundjoin%
\definecolor{currentfill}{rgb}{0.283187,0.125848,0.444960}%
\pgfsetfillcolor{currentfill}%
\pgfsetfillopacity{0.700000}%
\pgfsetlinewidth{0.000000pt}%
\definecolor{currentstroke}{rgb}{0.000000,0.000000,0.000000}%
\pgfsetstrokecolor{currentstroke}%
\pgfsetdash{}{0pt}%
\pgfpathmoveto{\pgfqpoint{4.548641in}{1.379266in}}%
\pgfpathlineto{\pgfqpoint{4.563049in}{1.372875in}}%
\pgfpathlineto{\pgfqpoint{4.577462in}{1.366507in}}%
\pgfpathlineto{\pgfqpoint{4.591882in}{1.360162in}}%
\pgfpathlineto{\pgfqpoint{4.606307in}{1.353841in}}%
\pgfpathlineto{\pgfqpoint{4.598343in}{1.355418in}}%
\pgfpathlineto{\pgfqpoint{4.590373in}{1.357463in}}%
\pgfpathlineto{\pgfqpoint{4.582399in}{1.359988in}}%
\pgfpathlineto{\pgfqpoint{4.574420in}{1.363003in}}%
\pgfpathlineto{\pgfqpoint{4.559970in}{1.369725in}}%
\pgfpathlineto{\pgfqpoint{4.545525in}{1.376469in}}%
\pgfpathlineto{\pgfqpoint{4.531087in}{1.383237in}}%
\pgfpathlineto{\pgfqpoint{4.516654in}{1.390029in}}%
\pgfpathlineto{\pgfqpoint{4.524659in}{1.386608in}}%
\pgfpathlineto{\pgfqpoint{4.532658in}{1.383681in}}%
\pgfpathlineto{\pgfqpoint{4.540652in}{1.381237in}}%
\pgfpathlineto{\pgfqpoint{4.548641in}{1.379266in}}%
\pgfpathclose%
\pgfusepath{fill}%
\end{pgfscope}%
\begin{pgfscope}%
\pgfpathrectangle{\pgfqpoint{1.150000in}{0.150000in}}{\pgfqpoint{5.700000in}{5.700000in}}%
\pgfusepath{clip}%
\pgfsetbuttcap%
\pgfsetroundjoin%
\definecolor{currentfill}{rgb}{0.126326,0.644107,0.525311}%
\pgfsetfillcolor{currentfill}%
\pgfsetfillopacity{0.700000}%
\pgfsetlinewidth{0.000000pt}%
\definecolor{currentstroke}{rgb}{0.000000,0.000000,0.000000}%
\pgfsetstrokecolor{currentstroke}%
\pgfsetdash{}{0pt}%
\pgfpathmoveto{\pgfqpoint{2.724048in}{2.688341in}}%
\pgfpathlineto{\pgfqpoint{2.738231in}{2.676596in}}%
\pgfpathlineto{\pgfqpoint{2.752414in}{2.664885in}}%
\pgfpathlineto{\pgfqpoint{2.766600in}{2.653210in}}%
\pgfpathlineto{\pgfqpoint{2.780787in}{2.641570in}}%
\pgfpathlineto{\pgfqpoint{2.771148in}{2.668871in}}%
\pgfpathlineto{\pgfqpoint{2.761468in}{2.697035in}}%
\pgfpathlineto{\pgfqpoint{2.751746in}{2.726079in}}%
\pgfpathlineto{\pgfqpoint{2.741983in}{2.756019in}}%
\pgfpathlineto{\pgfqpoint{2.727726in}{2.768168in}}%
\pgfpathlineto{\pgfqpoint{2.713470in}{2.780351in}}%
\pgfpathlineto{\pgfqpoint{2.699216in}{2.792570in}}%
\pgfpathlineto{\pgfqpoint{2.684963in}{2.804824in}}%
\pgfpathlineto{\pgfqpoint{2.694799in}{2.774367in}}%
\pgfpathlineto{\pgfqpoint{2.704591in}{2.744812in}}%
\pgfpathlineto{\pgfqpoint{2.714341in}{2.716142in}}%
\pgfpathlineto{\pgfqpoint{2.724048in}{2.688341in}}%
\pgfpathclose%
\pgfusepath{fill}%
\end{pgfscope}%
\begin{pgfscope}%
\pgfpathrectangle{\pgfqpoint{1.150000in}{0.150000in}}{\pgfqpoint{5.700000in}{5.700000in}}%
\pgfusepath{clip}%
\pgfsetbuttcap%
\pgfsetroundjoin%
\definecolor{currentfill}{rgb}{0.281924,0.089666,0.412415}%
\pgfsetfillcolor{currentfill}%
\pgfsetfillopacity{0.700000}%
\pgfsetlinewidth{0.000000pt}%
\definecolor{currentstroke}{rgb}{0.000000,0.000000,0.000000}%
\pgfsetstrokecolor{currentstroke}%
\pgfsetdash{}{0pt}%
\pgfpathmoveto{\pgfqpoint{4.753572in}{1.305349in}}%
\pgfpathlineto{\pgfqpoint{4.768032in}{1.299622in}}%
\pgfpathlineto{\pgfqpoint{4.782498in}{1.293918in}}%
\pgfpathlineto{\pgfqpoint{4.796971in}{1.288236in}}%
\pgfpathlineto{\pgfqpoint{4.811450in}{1.282578in}}%
\pgfpathlineto{\pgfqpoint{4.803563in}{1.281249in}}%
\pgfpathlineto{\pgfqpoint{4.795674in}{1.280335in}}%
\pgfpathlineto{\pgfqpoint{4.787782in}{1.279846in}}%
\pgfpathlineto{\pgfqpoint{4.779887in}{1.279792in}}%
\pgfpathlineto{\pgfqpoint{4.765388in}{1.285836in}}%
\pgfpathlineto{\pgfqpoint{4.750896in}{1.291903in}}%
\pgfpathlineto{\pgfqpoint{4.736409in}{1.297993in}}%
\pgfpathlineto{\pgfqpoint{4.721929in}{1.304106in}}%
\pgfpathlineto{\pgfqpoint{4.729845in}{1.303768in}}%
\pgfpathlineto{\pgfqpoint{4.737757in}{1.303870in}}%
\pgfpathlineto{\pgfqpoint{4.745666in}{1.304400in}}%
\pgfpathlineto{\pgfqpoint{4.753572in}{1.305349in}}%
\pgfpathclose%
\pgfusepath{fill}%
\end{pgfscope}%
\begin{pgfscope}%
\pgfpathrectangle{\pgfqpoint{1.150000in}{0.150000in}}{\pgfqpoint{5.700000in}{5.700000in}}%
\pgfusepath{clip}%
\pgfsetbuttcap%
\pgfsetroundjoin%
\definecolor{currentfill}{rgb}{0.169646,0.456262,0.558030}%
\pgfsetfillcolor{currentfill}%
\pgfsetfillopacity{0.700000}%
\pgfsetlinewidth{0.000000pt}%
\definecolor{currentstroke}{rgb}{0.000000,0.000000,0.000000}%
\pgfsetstrokecolor{currentstroke}%
\pgfsetdash{}{0pt}%
\pgfpathmoveto{\pgfqpoint{3.328930in}{2.160384in}}%
\pgfpathlineto{\pgfqpoint{3.343147in}{2.150373in}}%
\pgfpathlineto{\pgfqpoint{3.357367in}{2.140390in}}%
\pgfpathlineto{\pgfqpoint{3.371590in}{2.130436in}}%
\pgfpathlineto{\pgfqpoint{3.385816in}{2.120509in}}%
\pgfpathlineto{\pgfqpoint{3.376911in}{2.139783in}}%
\pgfpathlineto{\pgfqpoint{3.367981in}{2.159808in}}%
\pgfpathlineto{\pgfqpoint{3.359024in}{2.180598in}}%
\pgfpathlineto{\pgfqpoint{3.350040in}{2.202168in}}%
\pgfpathlineto{\pgfqpoint{3.335759in}{2.212570in}}%
\pgfpathlineto{\pgfqpoint{3.321480in}{2.223001in}}%
\pgfpathlineto{\pgfqpoint{3.307205in}{2.233461in}}%
\pgfpathlineto{\pgfqpoint{3.292932in}{2.243948in}}%
\pgfpathlineto{\pgfqpoint{3.301974in}{2.221895in}}%
\pgfpathlineto{\pgfqpoint{3.310987in}{2.200626in}}%
\pgfpathlineto{\pgfqpoint{3.319972in}{2.180127in}}%
\pgfpathlineto{\pgfqpoint{3.328930in}{2.160384in}}%
\pgfpathclose%
\pgfusepath{fill}%
\end{pgfscope}%
\begin{pgfscope}%
\pgfpathrectangle{\pgfqpoint{1.150000in}{0.150000in}}{\pgfqpoint{5.700000in}{5.700000in}}%
\pgfusepath{clip}%
\pgfsetbuttcap%
\pgfsetroundjoin%
\definecolor{currentfill}{rgb}{0.278826,0.175490,0.483397}%
\pgfsetfillcolor{currentfill}%
\pgfsetfillopacity{0.700000}%
\pgfsetlinewidth{0.000000pt}%
\definecolor{currentstroke}{rgb}{0.000000,0.000000,0.000000}%
\pgfsetstrokecolor{currentstroke}%
\pgfsetdash{}{0pt}%
\pgfpathmoveto{\pgfqpoint{4.343893in}{1.473353in}}%
\pgfpathlineto{\pgfqpoint{4.358259in}{1.466280in}}%
\pgfpathlineto{\pgfqpoint{4.372631in}{1.459231in}}%
\pgfpathlineto{\pgfqpoint{4.387009in}{1.452205in}}%
\pgfpathlineto{\pgfqpoint{4.401391in}{1.445202in}}%
\pgfpathlineto{\pgfqpoint{4.393325in}{1.449949in}}%
\pgfpathlineto{\pgfqpoint{4.385251in}{1.455221in}}%
\pgfpathlineto{\pgfqpoint{4.377169in}{1.461030in}}%
\pgfpathlineto{\pgfqpoint{4.369080in}{1.467387in}}%
\pgfpathlineto{\pgfqpoint{4.354667in}{1.474806in}}%
\pgfpathlineto{\pgfqpoint{4.340259in}{1.482248in}}%
\pgfpathlineto{\pgfqpoint{4.325857in}{1.489713in}}%
\pgfpathlineto{\pgfqpoint{4.311459in}{1.497203in}}%
\pgfpathlineto{\pgfqpoint{4.319580in}{1.490423in}}%
\pgfpathlineto{\pgfqpoint{4.327693in}{1.484196in}}%
\pgfpathlineto{\pgfqpoint{4.335797in}{1.478510in}}%
\pgfpathlineto{\pgfqpoint{4.343893in}{1.473353in}}%
\pgfpathclose%
\pgfusepath{fill}%
\end{pgfscope}%
\begin{pgfscope}%
\pgfpathrectangle{\pgfqpoint{1.150000in}{0.150000in}}{\pgfqpoint{5.700000in}{5.700000in}}%
\pgfusepath{clip}%
\pgfsetbuttcap%
\pgfsetroundjoin%
\definecolor{currentfill}{rgb}{0.262138,0.242286,0.520837}%
\pgfsetfillcolor{currentfill}%
\pgfsetfillopacity{0.700000}%
\pgfsetlinewidth{0.000000pt}%
\definecolor{currentstroke}{rgb}{0.000000,0.000000,0.000000}%
\pgfsetstrokecolor{currentstroke}%
\pgfsetdash{}{0pt}%
\pgfpathmoveto{\pgfqpoint{4.081789in}{1.620283in}}%
\pgfpathlineto{\pgfqpoint{4.096107in}{1.612409in}}%
\pgfpathlineto{\pgfqpoint{4.110429in}{1.604560in}}%
\pgfpathlineto{\pgfqpoint{4.124757in}{1.596736in}}%
\pgfpathlineto{\pgfqpoint{4.139089in}{1.588935in}}%
\pgfpathlineto{\pgfqpoint{4.130857in}{1.597559in}}%
\pgfpathlineto{\pgfqpoint{4.122615in}{1.606772in}}%
\pgfpathlineto{\pgfqpoint{4.114360in}{1.616586in}}%
\pgfpathlineto{\pgfqpoint{4.106094in}{1.627016in}}%
\pgfpathlineto{\pgfqpoint{4.091726in}{1.635250in}}%
\pgfpathlineto{\pgfqpoint{4.077362in}{1.643508in}}%
\pgfpathlineto{\pgfqpoint{4.063002in}{1.651790in}}%
\pgfpathlineto{\pgfqpoint{4.048648in}{1.660097in}}%
\pgfpathlineto{\pgfqpoint{4.056952in}{1.649228in}}%
\pgfpathlineto{\pgfqpoint{4.065243in}{1.638978in}}%
\pgfpathlineto{\pgfqpoint{4.073523in}{1.629333in}}%
\pgfpathlineto{\pgfqpoint{4.081789in}{1.620283in}}%
\pgfpathclose%
\pgfusepath{fill}%
\end{pgfscope}%
\begin{pgfscope}%
\pgfpathrectangle{\pgfqpoint{1.150000in}{0.150000in}}{\pgfqpoint{5.700000in}{5.700000in}}%
\pgfusepath{clip}%
\pgfsetbuttcap%
\pgfsetroundjoin%
\definecolor{currentfill}{rgb}{0.223925,0.334994,0.548053}%
\pgfsetfillcolor{currentfill}%
\pgfsetfillopacity{0.700000}%
\pgfsetlinewidth{0.000000pt}%
\definecolor{currentstroke}{rgb}{0.000000,0.000000,0.000000}%
\pgfsetstrokecolor{currentstroke}%
\pgfsetdash{}{0pt}%
\pgfpathmoveto{\pgfqpoint{3.762474in}{1.831435in}}%
\pgfpathlineto{\pgfqpoint{3.776743in}{1.822629in}}%
\pgfpathlineto{\pgfqpoint{3.791015in}{1.813849in}}%
\pgfpathlineto{\pgfqpoint{3.805292in}{1.805094in}}%
\pgfpathlineto{\pgfqpoint{3.819573in}{1.796364in}}%
\pgfpathlineto{\pgfqpoint{3.811090in}{1.809645in}}%
\pgfpathlineto{\pgfqpoint{3.802591in}{1.823587in}}%
\pgfpathlineto{\pgfqpoint{3.794074in}{1.838206in}}%
\pgfpathlineto{\pgfqpoint{3.785539in}{1.853514in}}%
\pgfpathlineto{\pgfqpoint{3.771213in}{1.862696in}}%
\pgfpathlineto{\pgfqpoint{3.756892in}{1.871903in}}%
\pgfpathlineto{\pgfqpoint{3.742574in}{1.881137in}}%
\pgfpathlineto{\pgfqpoint{3.728260in}{1.890396in}}%
\pgfpathlineto{\pgfqpoint{3.736841in}{1.874628in}}%
\pgfpathlineto{\pgfqpoint{3.745404in}{1.859555in}}%
\pgfpathlineto{\pgfqpoint{3.753948in}{1.845161in}}%
\pgfpathlineto{\pgfqpoint{3.762474in}{1.831435in}}%
\pgfpathclose%
\pgfusepath{fill}%
\end{pgfscope}%
\begin{pgfscope}%
\pgfpathrectangle{\pgfqpoint{1.150000in}{0.150000in}}{\pgfqpoint{5.700000in}{5.700000in}}%
\pgfusepath{clip}%
\pgfsetbuttcap%
\pgfsetroundjoin%
\definecolor{currentfill}{rgb}{0.121380,0.629492,0.531973}%
\pgfsetfillcolor{currentfill}%
\pgfsetfillopacity{0.700000}%
\pgfsetlinewidth{0.000000pt}%
\definecolor{currentstroke}{rgb}{0.000000,0.000000,0.000000}%
\pgfsetstrokecolor{currentstroke}%
\pgfsetdash{}{0pt}%
\pgfpathmoveto{\pgfqpoint{2.780787in}{2.641570in}}%
\pgfpathlineto{\pgfqpoint{2.794976in}{2.629963in}}%
\pgfpathlineto{\pgfqpoint{2.809167in}{2.618391in}}%
\pgfpathlineto{\pgfqpoint{2.823359in}{2.606853in}}%
\pgfpathlineto{\pgfqpoint{2.837554in}{2.595349in}}%
\pgfpathlineto{\pgfqpoint{2.827981in}{2.622151in}}%
\pgfpathlineto{\pgfqpoint{2.818370in}{2.649812in}}%
\pgfpathlineto{\pgfqpoint{2.808718in}{2.678346in}}%
\pgfpathlineto{\pgfqpoint{2.799026in}{2.707770in}}%
\pgfpathlineto{\pgfqpoint{2.784763in}{2.719781in}}%
\pgfpathlineto{\pgfqpoint{2.770501in}{2.731826in}}%
\pgfpathlineto{\pgfqpoint{2.756241in}{2.743905in}}%
\pgfpathlineto{\pgfqpoint{2.741983in}{2.756019in}}%
\pgfpathlineto{\pgfqpoint{2.751746in}{2.726079in}}%
\pgfpathlineto{\pgfqpoint{2.761468in}{2.697035in}}%
\pgfpathlineto{\pgfqpoint{2.771148in}{2.668871in}}%
\pgfpathlineto{\pgfqpoint{2.780787in}{2.641570in}}%
\pgfpathclose%
\pgfusepath{fill}%
\end{pgfscope}%
\begin{pgfscope}%
\pgfpathrectangle{\pgfqpoint{1.150000in}{0.150000in}}{\pgfqpoint{5.700000in}{5.700000in}}%
\pgfusepath{clip}%
\pgfsetbuttcap%
\pgfsetroundjoin%
\definecolor{currentfill}{rgb}{0.278791,0.062145,0.386592}%
\pgfsetfillcolor{currentfill}%
\pgfsetfillopacity{0.700000}%
\pgfsetlinewidth{0.000000pt}%
\definecolor{currentstroke}{rgb}{0.000000,0.000000,0.000000}%
\pgfsetstrokecolor{currentstroke}%
\pgfsetdash{}{0pt}%
\pgfpathmoveto{\pgfqpoint{4.958907in}{1.250370in}}%
\pgfpathlineto{\pgfqpoint{4.973429in}{1.245290in}}%
\pgfpathlineto{\pgfqpoint{4.987959in}{1.240233in}}%
\pgfpathlineto{\pgfqpoint{5.002495in}{1.235199in}}%
\pgfpathlineto{\pgfqpoint{4.994661in}{1.231306in}}%
\pgfpathlineto{\pgfqpoint{4.986826in}{1.227777in}}%
\pgfpathlineto{\pgfqpoint{4.978990in}{1.224624in}}%
\pgfpathlineto{\pgfqpoint{4.971153in}{1.221854in}}%
\pgfpathlineto{\pgfqpoint{4.956602in}{1.227260in}}%
\pgfpathlineto{\pgfqpoint{4.942057in}{1.232689in}}%
\pgfpathlineto{\pgfqpoint{4.927518in}{1.238140in}}%
\pgfpathlineto{\pgfqpoint{4.935368in}{1.240627in}}%
\pgfpathlineto{\pgfqpoint{4.943215in}{1.243500in}}%
\pgfpathlineto{\pgfqpoint{4.951062in}{1.246751in}}%
\pgfpathlineto{\pgfqpoint{4.958907in}{1.250370in}}%
\pgfpathclose%
\pgfusepath{fill}%
\end{pgfscope}%
\begin{pgfscope}%
\pgfpathrectangle{\pgfqpoint{1.150000in}{0.150000in}}{\pgfqpoint{5.700000in}{5.700000in}}%
\pgfusepath{clip}%
\pgfsetbuttcap%
\pgfsetroundjoin%
\definecolor{currentfill}{rgb}{0.174274,0.445044,0.557792}%
\pgfsetfillcolor{currentfill}%
\pgfsetfillopacity{0.700000}%
\pgfsetlinewidth{0.000000pt}%
\definecolor{currentstroke}{rgb}{0.000000,0.000000,0.000000}%
\pgfsetstrokecolor{currentstroke}%
\pgfsetdash{}{0pt}%
\pgfpathmoveto{\pgfqpoint{3.385816in}{2.120509in}}%
\pgfpathlineto{\pgfqpoint{3.400045in}{2.110610in}}%
\pgfpathlineto{\pgfqpoint{3.414277in}{2.100739in}}%
\pgfpathlineto{\pgfqpoint{3.428513in}{2.090895in}}%
\pgfpathlineto{\pgfqpoint{3.442752in}{2.081079in}}%
\pgfpathlineto{\pgfqpoint{3.433901in}{2.099885in}}%
\pgfpathlineto{\pgfqpoint{3.425024in}{2.119437in}}%
\pgfpathlineto{\pgfqpoint{3.416122in}{2.139749in}}%
\pgfpathlineto{\pgfqpoint{3.407195in}{2.160836in}}%
\pgfpathlineto{\pgfqpoint{3.392901in}{2.171127in}}%
\pgfpathlineto{\pgfqpoint{3.378611in}{2.181446in}}%
\pgfpathlineto{\pgfqpoint{3.364324in}{2.191793in}}%
\pgfpathlineto{\pgfqpoint{3.350040in}{2.202168in}}%
\pgfpathlineto{\pgfqpoint{3.359024in}{2.180598in}}%
\pgfpathlineto{\pgfqpoint{3.367981in}{2.159808in}}%
\pgfpathlineto{\pgfqpoint{3.376911in}{2.139783in}}%
\pgfpathlineto{\pgfqpoint{3.385816in}{2.120509in}}%
\pgfpathclose%
\pgfusepath{fill}%
\end{pgfscope}%
\begin{pgfscope}%
\pgfpathrectangle{\pgfqpoint{1.150000in}{0.150000in}}{\pgfqpoint{5.700000in}{5.700000in}}%
\pgfusepath{clip}%
\pgfsetbuttcap%
\pgfsetroundjoin%
\definecolor{currentfill}{rgb}{0.283229,0.120777,0.440584}%
\pgfsetfillcolor{currentfill}%
\pgfsetfillopacity{0.700000}%
\pgfsetlinewidth{0.000000pt}%
\definecolor{currentstroke}{rgb}{0.000000,0.000000,0.000000}%
\pgfsetstrokecolor{currentstroke}%
\pgfsetdash{}{0pt}%
\pgfpathmoveto{\pgfqpoint{4.606307in}{1.353841in}}%
\pgfpathlineto{\pgfqpoint{4.620739in}{1.347543in}}%
\pgfpathlineto{\pgfqpoint{4.635176in}{1.341268in}}%
\pgfpathlineto{\pgfqpoint{4.649620in}{1.335017in}}%
\pgfpathlineto{\pgfqpoint{4.664070in}{1.328788in}}%
\pgfpathlineto{\pgfqpoint{4.656128in}{1.329970in}}%
\pgfpathlineto{\pgfqpoint{4.648183in}{1.331617in}}%
\pgfpathlineto{\pgfqpoint{4.640233in}{1.333740in}}%
\pgfpathlineto{\pgfqpoint{4.632279in}{1.336349in}}%
\pgfpathlineto{\pgfqpoint{4.617805in}{1.342978in}}%
\pgfpathlineto{\pgfqpoint{4.603338in}{1.349630in}}%
\pgfpathlineto{\pgfqpoint{4.588876in}{1.356305in}}%
\pgfpathlineto{\pgfqpoint{4.574420in}{1.363003in}}%
\pgfpathlineto{\pgfqpoint{4.582399in}{1.359988in}}%
\pgfpathlineto{\pgfqpoint{4.590373in}{1.357463in}}%
\pgfpathlineto{\pgfqpoint{4.598343in}{1.355418in}}%
\pgfpathlineto{\pgfqpoint{4.606307in}{1.353841in}}%
\pgfpathclose%
\pgfusepath{fill}%
\end{pgfscope}%
\begin{pgfscope}%
\pgfpathrectangle{\pgfqpoint{1.150000in}{0.150000in}}{\pgfqpoint{5.700000in}{5.700000in}}%
\pgfusepath{clip}%
\pgfsetbuttcap%
\pgfsetroundjoin%
\definecolor{currentfill}{rgb}{0.119483,0.614817,0.537692}%
\pgfsetfillcolor{currentfill}%
\pgfsetfillopacity{0.700000}%
\pgfsetlinewidth{0.000000pt}%
\definecolor{currentstroke}{rgb}{0.000000,0.000000,0.000000}%
\pgfsetstrokecolor{currentstroke}%
\pgfsetdash{}{0pt}%
\pgfpathmoveto{\pgfqpoint{2.837554in}{2.595349in}}%
\pgfpathlineto{\pgfqpoint{2.851750in}{2.583878in}}%
\pgfpathlineto{\pgfqpoint{2.865948in}{2.572441in}}%
\pgfpathlineto{\pgfqpoint{2.880148in}{2.561037in}}%
\pgfpathlineto{\pgfqpoint{2.894350in}{2.549666in}}%
\pgfpathlineto{\pgfqpoint{2.884844in}{2.575970in}}%
\pgfpathlineto{\pgfqpoint{2.875300in}{2.603128in}}%
\pgfpathlineto{\pgfqpoint{2.865717in}{2.631153in}}%
\pgfpathlineto{\pgfqpoint{2.856095in}{2.660064in}}%
\pgfpathlineto{\pgfqpoint{2.841825in}{2.671941in}}%
\pgfpathlineto{\pgfqpoint{2.827557in}{2.683850in}}%
\pgfpathlineto{\pgfqpoint{2.813291in}{2.695793in}}%
\pgfpathlineto{\pgfqpoint{2.799026in}{2.707770in}}%
\pgfpathlineto{\pgfqpoint{2.808718in}{2.678346in}}%
\pgfpathlineto{\pgfqpoint{2.818370in}{2.649812in}}%
\pgfpathlineto{\pgfqpoint{2.827981in}{2.622151in}}%
\pgfpathlineto{\pgfqpoint{2.837554in}{2.595349in}}%
\pgfpathclose%
\pgfusepath{fill}%
\end{pgfscope}%
\begin{pgfscope}%
\pgfpathrectangle{\pgfqpoint{1.150000in}{0.150000in}}{\pgfqpoint{5.700000in}{5.700000in}}%
\pgfusepath{clip}%
\pgfsetbuttcap%
\pgfsetroundjoin%
\definecolor{currentfill}{rgb}{0.281446,0.084320,0.407414}%
\pgfsetfillcolor{currentfill}%
\pgfsetfillopacity{0.700000}%
\pgfsetlinewidth{0.000000pt}%
\definecolor{currentstroke}{rgb}{0.000000,0.000000,0.000000}%
\pgfsetstrokecolor{currentstroke}%
\pgfsetdash{}{0pt}%
\pgfpathmoveto{\pgfqpoint{4.811450in}{1.282578in}}%
\pgfpathlineto{\pgfqpoint{4.825936in}{1.276943in}}%
\pgfpathlineto{\pgfqpoint{4.840428in}{1.271331in}}%
\pgfpathlineto{\pgfqpoint{4.854926in}{1.265741in}}%
\pgfpathlineto{\pgfqpoint{4.869431in}{1.260175in}}%
\pgfpathlineto{\pgfqpoint{4.861563in}{1.258466in}}%
\pgfpathlineto{\pgfqpoint{4.853693in}{1.257168in}}%
\pgfpathlineto{\pgfqpoint{4.845820in}{1.256292in}}%
\pgfpathlineto{\pgfqpoint{4.837946in}{1.255847in}}%
\pgfpathlineto{\pgfqpoint{4.823422in}{1.261799in}}%
\pgfpathlineto{\pgfqpoint{4.808904in}{1.267774in}}%
\pgfpathlineto{\pgfqpoint{4.794392in}{1.273772in}}%
\pgfpathlineto{\pgfqpoint{4.779887in}{1.279792in}}%
\pgfpathlineto{\pgfqpoint{4.787782in}{1.279846in}}%
\pgfpathlineto{\pgfqpoint{4.795674in}{1.280335in}}%
\pgfpathlineto{\pgfqpoint{4.803563in}{1.281249in}}%
\pgfpathlineto{\pgfqpoint{4.811450in}{1.282578in}}%
\pgfpathclose%
\pgfusepath{fill}%
\end{pgfscope}%
\begin{pgfscope}%
\pgfpathrectangle{\pgfqpoint{1.150000in}{0.150000in}}{\pgfqpoint{5.700000in}{5.700000in}}%
\pgfusepath{clip}%
\pgfsetbuttcap%
\pgfsetroundjoin%
\definecolor{currentfill}{rgb}{0.229739,0.322361,0.545706}%
\pgfsetfillcolor{currentfill}%
\pgfsetfillopacity{0.700000}%
\pgfsetlinewidth{0.000000pt}%
\definecolor{currentstroke}{rgb}{0.000000,0.000000,0.000000}%
\pgfsetstrokecolor{currentstroke}%
\pgfsetdash{}{0pt}%
\pgfpathmoveto{\pgfqpoint{3.819573in}{1.796364in}}%
\pgfpathlineto{\pgfqpoint{3.833858in}{1.787660in}}%
\pgfpathlineto{\pgfqpoint{3.848147in}{1.778981in}}%
\pgfpathlineto{\pgfqpoint{3.862440in}{1.770328in}}%
\pgfpathlineto{\pgfqpoint{3.876738in}{1.761699in}}%
\pgfpathlineto{\pgfqpoint{3.868298in}{1.774534in}}%
\pgfpathlineto{\pgfqpoint{3.859842in}{1.788027in}}%
\pgfpathlineto{\pgfqpoint{3.851369in}{1.802190in}}%
\pgfpathlineto{\pgfqpoint{3.842880in}{1.817039in}}%
\pgfpathlineto{\pgfqpoint{3.828538in}{1.826120in}}%
\pgfpathlineto{\pgfqpoint{3.814201in}{1.835226in}}%
\pgfpathlineto{\pgfqpoint{3.799868in}{1.844357in}}%
\pgfpathlineto{\pgfqpoint{3.785539in}{1.853514in}}%
\pgfpathlineto{\pgfqpoint{3.794074in}{1.838206in}}%
\pgfpathlineto{\pgfqpoint{3.802591in}{1.823587in}}%
\pgfpathlineto{\pgfqpoint{3.811090in}{1.809645in}}%
\pgfpathlineto{\pgfqpoint{3.819573in}{1.796364in}}%
\pgfpathclose%
\pgfusepath{fill}%
\end{pgfscope}%
\begin{pgfscope}%
\pgfpathrectangle{\pgfqpoint{1.150000in}{0.150000in}}{\pgfqpoint{5.700000in}{5.700000in}}%
\pgfusepath{clip}%
\pgfsetbuttcap%
\pgfsetroundjoin%
\definecolor{currentfill}{rgb}{0.265145,0.232956,0.516599}%
\pgfsetfillcolor{currentfill}%
\pgfsetfillopacity{0.700000}%
\pgfsetlinewidth{0.000000pt}%
\definecolor{currentstroke}{rgb}{0.000000,0.000000,0.000000}%
\pgfsetstrokecolor{currentstroke}%
\pgfsetdash{}{0pt}%
\pgfpathmoveto{\pgfqpoint{4.139089in}{1.588935in}}%
\pgfpathlineto{\pgfqpoint{4.153425in}{1.581159in}}%
\pgfpathlineto{\pgfqpoint{4.167767in}{1.573406in}}%
\pgfpathlineto{\pgfqpoint{4.182114in}{1.565678in}}%
\pgfpathlineto{\pgfqpoint{4.196466in}{1.557974in}}%
\pgfpathlineto{\pgfqpoint{4.188269in}{1.566171in}}%
\pgfpathlineto{\pgfqpoint{4.180062in}{1.574953in}}%
\pgfpathlineto{\pgfqpoint{4.171844in}{1.584333in}}%
\pgfpathlineto{\pgfqpoint{4.163615in}{1.594323in}}%
\pgfpathlineto{\pgfqpoint{4.149228in}{1.602460in}}%
\pgfpathlineto{\pgfqpoint{4.134845in}{1.610621in}}%
\pgfpathlineto{\pgfqpoint{4.120467in}{1.618806in}}%
\pgfpathlineto{\pgfqpoint{4.106094in}{1.627016in}}%
\pgfpathlineto{\pgfqpoint{4.114360in}{1.616586in}}%
\pgfpathlineto{\pgfqpoint{4.122615in}{1.606772in}}%
\pgfpathlineto{\pgfqpoint{4.130857in}{1.597559in}}%
\pgfpathlineto{\pgfqpoint{4.139089in}{1.588935in}}%
\pgfpathclose%
\pgfusepath{fill}%
\end{pgfscope}%
\begin{pgfscope}%
\pgfpathrectangle{\pgfqpoint{1.150000in}{0.150000in}}{\pgfqpoint{5.700000in}{5.700000in}}%
\pgfusepath{clip}%
\pgfsetbuttcap%
\pgfsetroundjoin%
\definecolor{currentfill}{rgb}{0.280255,0.165693,0.476498}%
\pgfsetfillcolor{currentfill}%
\pgfsetfillopacity{0.700000}%
\pgfsetlinewidth{0.000000pt}%
\definecolor{currentstroke}{rgb}{0.000000,0.000000,0.000000}%
\pgfsetstrokecolor{currentstroke}%
\pgfsetdash{}{0pt}%
\pgfpathmoveto{\pgfqpoint{4.401391in}{1.445202in}}%
\pgfpathlineto{\pgfqpoint{4.415780in}{1.438223in}}%
\pgfpathlineto{\pgfqpoint{4.430174in}{1.431268in}}%
\pgfpathlineto{\pgfqpoint{4.444573in}{1.424336in}}%
\pgfpathlineto{\pgfqpoint{4.458978in}{1.417428in}}%
\pgfpathlineto{\pgfqpoint{4.450940in}{1.421765in}}%
\pgfpathlineto{\pgfqpoint{4.442894in}{1.426623in}}%
\pgfpathlineto{\pgfqpoint{4.434843in}{1.432014in}}%
\pgfpathlineto{\pgfqpoint{4.426784in}{1.437949in}}%
\pgfpathlineto{\pgfqpoint{4.412350in}{1.445273in}}%
\pgfpathlineto{\pgfqpoint{4.397921in}{1.452621in}}%
\pgfpathlineto{\pgfqpoint{4.383498in}{1.459992in}}%
\pgfpathlineto{\pgfqpoint{4.369080in}{1.467387in}}%
\pgfpathlineto{\pgfqpoint{4.377169in}{1.461030in}}%
\pgfpathlineto{\pgfqpoint{4.385251in}{1.455221in}}%
\pgfpathlineto{\pgfqpoint{4.393325in}{1.449949in}}%
\pgfpathlineto{\pgfqpoint{4.401391in}{1.445202in}}%
\pgfpathclose%
\pgfusepath{fill}%
\end{pgfscope}%
\begin{pgfscope}%
\pgfpathrectangle{\pgfqpoint{1.150000in}{0.150000in}}{\pgfqpoint{5.700000in}{5.700000in}}%
\pgfusepath{clip}%
\pgfsetbuttcap%
\pgfsetroundjoin%
\definecolor{currentfill}{rgb}{0.119738,0.603785,0.541400}%
\pgfsetfillcolor{currentfill}%
\pgfsetfillopacity{0.700000}%
\pgfsetlinewidth{0.000000pt}%
\definecolor{currentstroke}{rgb}{0.000000,0.000000,0.000000}%
\pgfsetstrokecolor{currentstroke}%
\pgfsetdash{}{0pt}%
\pgfpathmoveto{\pgfqpoint{2.894350in}{2.549666in}}%
\pgfpathlineto{\pgfqpoint{2.908554in}{2.538327in}}%
\pgfpathlineto{\pgfqpoint{2.922760in}{2.527021in}}%
\pgfpathlineto{\pgfqpoint{2.936968in}{2.515748in}}%
\pgfpathlineto{\pgfqpoint{2.951178in}{2.504506in}}%
\pgfpathlineto{\pgfqpoint{2.941737in}{2.530315in}}%
\pgfpathlineto{\pgfqpoint{2.932260in}{2.556970in}}%
\pgfpathlineto{\pgfqpoint{2.922745in}{2.584489in}}%
\pgfpathlineto{\pgfqpoint{2.913191in}{2.612887in}}%
\pgfpathlineto{\pgfqpoint{2.898914in}{2.624632in}}%
\pgfpathlineto{\pgfqpoint{2.884639in}{2.636410in}}%
\pgfpathlineto{\pgfqpoint{2.870366in}{2.648221in}}%
\pgfpathlineto{\pgfqpoint{2.856095in}{2.660064in}}%
\pgfpathlineto{\pgfqpoint{2.865717in}{2.631153in}}%
\pgfpathlineto{\pgfqpoint{2.875300in}{2.603128in}}%
\pgfpathlineto{\pgfqpoint{2.884844in}{2.575970in}}%
\pgfpathlineto{\pgfqpoint{2.894350in}{2.549666in}}%
\pgfpathclose%
\pgfusepath{fill}%
\end{pgfscope}%
\begin{pgfscope}%
\pgfpathrectangle{\pgfqpoint{1.150000in}{0.150000in}}{\pgfqpoint{5.700000in}{5.700000in}}%
\pgfusepath{clip}%
\pgfsetbuttcap%
\pgfsetroundjoin%
\definecolor{currentfill}{rgb}{0.179019,0.433756,0.557430}%
\pgfsetfillcolor{currentfill}%
\pgfsetfillopacity{0.700000}%
\pgfsetlinewidth{0.000000pt}%
\definecolor{currentstroke}{rgb}{0.000000,0.000000,0.000000}%
\pgfsetstrokecolor{currentstroke}%
\pgfsetdash{}{0pt}%
\pgfpathmoveto{\pgfqpoint{3.442752in}{2.081079in}}%
\pgfpathlineto{\pgfqpoint{3.456994in}{2.071291in}}%
\pgfpathlineto{\pgfqpoint{3.471240in}{2.061529in}}%
\pgfpathlineto{\pgfqpoint{3.485489in}{2.051795in}}%
\pgfpathlineto{\pgfqpoint{3.499741in}{2.042088in}}%
\pgfpathlineto{\pgfqpoint{3.490942in}{2.060427in}}%
\pgfpathlineto{\pgfqpoint{3.482119in}{2.079506in}}%
\pgfpathlineto{\pgfqpoint{3.473271in}{2.099341in}}%
\pgfpathlineto{\pgfqpoint{3.464398in}{2.119946in}}%
\pgfpathlineto{\pgfqpoint{3.450093in}{2.130127in}}%
\pgfpathlineto{\pgfqpoint{3.435790in}{2.140336in}}%
\pgfpathlineto{\pgfqpoint{3.421491in}{2.150572in}}%
\pgfpathlineto{\pgfqpoint{3.407195in}{2.160836in}}%
\pgfpathlineto{\pgfqpoint{3.416122in}{2.139749in}}%
\pgfpathlineto{\pgfqpoint{3.425024in}{2.119437in}}%
\pgfpathlineto{\pgfqpoint{3.433901in}{2.099885in}}%
\pgfpathlineto{\pgfqpoint{3.442752in}{2.081079in}}%
\pgfpathclose%
\pgfusepath{fill}%
\end{pgfscope}%
\begin{pgfscope}%
\pgfpathrectangle{\pgfqpoint{1.150000in}{0.150000in}}{\pgfqpoint{5.700000in}{5.700000in}}%
\pgfusepath{clip}%
\pgfsetbuttcap%
\pgfsetroundjoin%
\definecolor{currentfill}{rgb}{0.783315,0.879285,0.125405}%
\pgfsetfillcolor{currentfill}%
\pgfsetfillopacity{0.700000}%
\pgfsetlinewidth{0.000000pt}%
\definecolor{currentstroke}{rgb}{0.000000,0.000000,0.000000}%
\pgfsetstrokecolor{currentstroke}%
\pgfsetdash{}{0pt}%
\pgfpathmoveto{\pgfqpoint{1.887687in}{3.557196in}}%
\pgfpathlineto{\pgfqpoint{1.901934in}{3.542416in}}%
\pgfpathlineto{\pgfqpoint{1.916180in}{3.527694in}}%
\pgfpathlineto{\pgfqpoint{1.930425in}{3.513029in}}%
\pgfpathlineto{\pgfqpoint{1.944668in}{3.498421in}}%
\pgfpathlineto{\pgfqpoint{1.933734in}{3.536757in}}%
\pgfpathlineto{\pgfqpoint{1.922735in}{3.576108in}}%
\pgfpathlineto{\pgfqpoint{1.911671in}{3.616493in}}%
\pgfpathlineto{\pgfqpoint{1.900540in}{3.657932in}}%
\pgfpathlineto{\pgfqpoint{1.886205in}{3.673100in}}%
\pgfpathlineto{\pgfqpoint{1.871869in}{3.688326in}}%
\pgfpathlineto{\pgfqpoint{1.857531in}{3.703609in}}%
\pgfpathlineto{\pgfqpoint{1.843192in}{3.718951in}}%
\pgfpathlineto{\pgfqpoint{1.854417in}{3.676941in}}%
\pgfpathlineto{\pgfqpoint{1.865573in}{3.635991in}}%
\pgfpathlineto{\pgfqpoint{1.876663in}{3.596083in}}%
\pgfpathlineto{\pgfqpoint{1.887687in}{3.557196in}}%
\pgfpathclose%
\pgfusepath{fill}%
\end{pgfscope}%
\begin{pgfscope}%
\pgfpathrectangle{\pgfqpoint{1.150000in}{0.150000in}}{\pgfqpoint{5.700000in}{5.700000in}}%
\pgfusepath{clip}%
\pgfsetbuttcap%
\pgfsetroundjoin%
\definecolor{currentfill}{rgb}{0.121831,0.589055,0.545623}%
\pgfsetfillcolor{currentfill}%
\pgfsetfillopacity{0.700000}%
\pgfsetlinewidth{0.000000pt}%
\definecolor{currentstroke}{rgb}{0.000000,0.000000,0.000000}%
\pgfsetstrokecolor{currentstroke}%
\pgfsetdash{}{0pt}%
\pgfpathmoveto{\pgfqpoint{2.951178in}{2.504506in}}%
\pgfpathlineto{\pgfqpoint{2.965390in}{2.493297in}}%
\pgfpathlineto{\pgfqpoint{2.979605in}{2.482120in}}%
\pgfpathlineto{\pgfqpoint{2.993822in}{2.470974in}}%
\pgfpathlineto{\pgfqpoint{3.008040in}{2.459860in}}%
\pgfpathlineto{\pgfqpoint{2.998664in}{2.485173in}}%
\pgfpathlineto{\pgfqpoint{2.989252in}{2.511328in}}%
\pgfpathlineto{\pgfqpoint{2.979803in}{2.538341in}}%
\pgfpathlineto{\pgfqpoint{2.970318in}{2.566228in}}%
\pgfpathlineto{\pgfqpoint{2.956033in}{2.577845in}}%
\pgfpathlineto{\pgfqpoint{2.941751in}{2.589493in}}%
\pgfpathlineto{\pgfqpoint{2.927470in}{2.601174in}}%
\pgfpathlineto{\pgfqpoint{2.913191in}{2.612887in}}%
\pgfpathlineto{\pgfqpoint{2.922745in}{2.584489in}}%
\pgfpathlineto{\pgfqpoint{2.932260in}{2.556970in}}%
\pgfpathlineto{\pgfqpoint{2.941737in}{2.530315in}}%
\pgfpathlineto{\pgfqpoint{2.951178in}{2.504506in}}%
\pgfpathclose%
\pgfusepath{fill}%
\end{pgfscope}%
\begin{pgfscope}%
\pgfpathrectangle{\pgfqpoint{1.150000in}{0.150000in}}{\pgfqpoint{5.700000in}{5.700000in}}%
\pgfusepath{clip}%
\pgfsetbuttcap%
\pgfsetroundjoin%
\definecolor{currentfill}{rgb}{0.730889,0.871916,0.156029}%
\pgfsetfillcolor{currentfill}%
\pgfsetfillopacity{0.700000}%
\pgfsetlinewidth{0.000000pt}%
\definecolor{currentstroke}{rgb}{0.000000,0.000000,0.000000}%
\pgfsetstrokecolor{currentstroke}%
\pgfsetdash{}{0pt}%
\pgfpathmoveto{\pgfqpoint{1.944668in}{3.498421in}}%
\pgfpathlineto{\pgfqpoint{1.958910in}{3.483870in}}%
\pgfpathlineto{\pgfqpoint{1.973151in}{3.469373in}}%
\pgfpathlineto{\pgfqpoint{1.987391in}{3.454932in}}%
\pgfpathlineto{\pgfqpoint{2.001630in}{3.440545in}}%
\pgfpathlineto{\pgfqpoint{1.990783in}{3.478332in}}%
\pgfpathlineto{\pgfqpoint{1.979874in}{3.517128in}}%
\pgfpathlineto{\pgfqpoint{1.968901in}{3.556952in}}%
\pgfpathlineto{\pgfqpoint{1.957863in}{3.597823in}}%
\pgfpathlineto{\pgfqpoint{1.943534in}{3.612767in}}%
\pgfpathlineto{\pgfqpoint{1.929204in}{3.627766in}}%
\pgfpathlineto{\pgfqpoint{1.914873in}{3.642821in}}%
\pgfpathlineto{\pgfqpoint{1.900540in}{3.657932in}}%
\pgfpathlineto{\pgfqpoint{1.911671in}{3.616493in}}%
\pgfpathlineto{\pgfqpoint{1.922735in}{3.576108in}}%
\pgfpathlineto{\pgfqpoint{1.933734in}{3.536757in}}%
\pgfpathlineto{\pgfqpoint{1.944668in}{3.498421in}}%
\pgfpathclose%
\pgfusepath{fill}%
\end{pgfscope}%
\begin{pgfscope}%
\pgfpathrectangle{\pgfqpoint{1.150000in}{0.150000in}}{\pgfqpoint{5.700000in}{5.700000in}}%
\pgfusepath{clip}%
\pgfsetbuttcap%
\pgfsetroundjoin%
\definecolor{currentfill}{rgb}{0.283197,0.115680,0.436115}%
\pgfsetfillcolor{currentfill}%
\pgfsetfillopacity{0.700000}%
\pgfsetlinewidth{0.000000pt}%
\definecolor{currentstroke}{rgb}{0.000000,0.000000,0.000000}%
\pgfsetstrokecolor{currentstroke}%
\pgfsetdash{}{0pt}%
\pgfpathmoveto{\pgfqpoint{4.664070in}{1.328788in}}%
\pgfpathlineto{\pgfqpoint{4.678525in}{1.322583in}}%
\pgfpathlineto{\pgfqpoint{4.692987in}{1.316401in}}%
\pgfpathlineto{\pgfqpoint{4.707455in}{1.310242in}}%
\pgfpathlineto{\pgfqpoint{4.721929in}{1.304106in}}%
\pgfpathlineto{\pgfqpoint{4.714010in}{1.304893in}}%
\pgfpathlineto{\pgfqpoint{4.706087in}{1.306142in}}%
\pgfpathlineto{\pgfqpoint{4.698161in}{1.307862in}}%
\pgfpathlineto{\pgfqpoint{4.690232in}{1.310065in}}%
\pgfpathlineto{\pgfqpoint{4.675734in}{1.316602in}}%
\pgfpathlineto{\pgfqpoint{4.661243in}{1.323161in}}%
\pgfpathlineto{\pgfqpoint{4.646758in}{1.329743in}}%
\pgfpathlineto{\pgfqpoint{4.632279in}{1.336349in}}%
\pgfpathlineto{\pgfqpoint{4.640233in}{1.333740in}}%
\pgfpathlineto{\pgfqpoint{4.648183in}{1.331617in}}%
\pgfpathlineto{\pgfqpoint{4.656128in}{1.329970in}}%
\pgfpathlineto{\pgfqpoint{4.664070in}{1.328788in}}%
\pgfpathclose%
\pgfusepath{fill}%
\end{pgfscope}%
\begin{pgfscope}%
\pgfpathrectangle{\pgfqpoint{1.150000in}{0.150000in}}{\pgfqpoint{5.700000in}{5.700000in}}%
\pgfusepath{clip}%
\pgfsetbuttcap%
\pgfsetroundjoin%
\definecolor{currentfill}{rgb}{0.233603,0.313828,0.543914}%
\pgfsetfillcolor{currentfill}%
\pgfsetfillopacity{0.700000}%
\pgfsetlinewidth{0.000000pt}%
\definecolor{currentstroke}{rgb}{0.000000,0.000000,0.000000}%
\pgfsetstrokecolor{currentstroke}%
\pgfsetdash{}{0pt}%
\pgfpathmoveto{\pgfqpoint{3.876738in}{1.761699in}}%
\pgfpathlineto{\pgfqpoint{3.891040in}{1.753096in}}%
\pgfpathlineto{\pgfqpoint{3.905346in}{1.744517in}}%
\pgfpathlineto{\pgfqpoint{3.919656in}{1.735964in}}%
\pgfpathlineto{\pgfqpoint{3.933971in}{1.727435in}}%
\pgfpathlineto{\pgfqpoint{3.925573in}{1.739825in}}%
\pgfpathlineto{\pgfqpoint{3.917159in}{1.752868in}}%
\pgfpathlineto{\pgfqpoint{3.908730in}{1.766578in}}%
\pgfpathlineto{\pgfqpoint{3.900285in}{1.780968in}}%
\pgfpathlineto{\pgfqpoint{3.885928in}{1.789948in}}%
\pgfpathlineto{\pgfqpoint{3.871574in}{1.798953in}}%
\pgfpathlineto{\pgfqpoint{3.857225in}{1.807984in}}%
\pgfpathlineto{\pgfqpoint{3.842880in}{1.817039in}}%
\pgfpathlineto{\pgfqpoint{3.851369in}{1.802190in}}%
\pgfpathlineto{\pgfqpoint{3.859842in}{1.788027in}}%
\pgfpathlineto{\pgfqpoint{3.868298in}{1.774534in}}%
\pgfpathlineto{\pgfqpoint{3.876738in}{1.761699in}}%
\pgfpathclose%
\pgfusepath{fill}%
\end{pgfscope}%
\begin{pgfscope}%
\pgfpathrectangle{\pgfqpoint{1.150000in}{0.150000in}}{\pgfqpoint{5.700000in}{5.700000in}}%
\pgfusepath{clip}%
\pgfsetbuttcap%
\pgfsetroundjoin%
\definecolor{currentfill}{rgb}{0.266580,0.228262,0.514349}%
\pgfsetfillcolor{currentfill}%
\pgfsetfillopacity{0.700000}%
\pgfsetlinewidth{0.000000pt}%
\definecolor{currentstroke}{rgb}{0.000000,0.000000,0.000000}%
\pgfsetstrokecolor{currentstroke}%
\pgfsetdash{}{0pt}%
\pgfpathmoveto{\pgfqpoint{4.196466in}{1.557974in}}%
\pgfpathlineto{\pgfqpoint{4.210822in}{1.550294in}}%
\pgfpathlineto{\pgfqpoint{4.225184in}{1.542638in}}%
\pgfpathlineto{\pgfqpoint{4.239550in}{1.535006in}}%
\pgfpathlineto{\pgfqpoint{4.253922in}{1.527397in}}%
\pgfpathlineto{\pgfqpoint{4.245759in}{1.535168in}}%
\pgfpathlineto{\pgfqpoint{4.237587in}{1.543519in}}%
\pgfpathlineto{\pgfqpoint{4.229405in}{1.552464in}}%
\pgfpathlineto{\pgfqpoint{4.221212in}{1.562015in}}%
\pgfpathlineto{\pgfqpoint{4.206806in}{1.570056in}}%
\pgfpathlineto{\pgfqpoint{4.192404in}{1.578121in}}%
\pgfpathlineto{\pgfqpoint{4.178007in}{1.586210in}}%
\pgfpathlineto{\pgfqpoint{4.163615in}{1.594323in}}%
\pgfpathlineto{\pgfqpoint{4.171844in}{1.584333in}}%
\pgfpathlineto{\pgfqpoint{4.180062in}{1.574953in}}%
\pgfpathlineto{\pgfqpoint{4.188269in}{1.566171in}}%
\pgfpathlineto{\pgfqpoint{4.196466in}{1.557974in}}%
\pgfpathclose%
\pgfusepath{fill}%
\end{pgfscope}%
\begin{pgfscope}%
\pgfpathrectangle{\pgfqpoint{1.150000in}{0.150000in}}{\pgfqpoint{5.700000in}{5.700000in}}%
\pgfusepath{clip}%
\pgfsetbuttcap%
\pgfsetroundjoin%
\definecolor{currentfill}{rgb}{0.182256,0.426184,0.557120}%
\pgfsetfillcolor{currentfill}%
\pgfsetfillopacity{0.700000}%
\pgfsetlinewidth{0.000000pt}%
\definecolor{currentstroke}{rgb}{0.000000,0.000000,0.000000}%
\pgfsetstrokecolor{currentstroke}%
\pgfsetdash{}{0pt}%
\pgfpathmoveto{\pgfqpoint{3.499741in}{2.042088in}}%
\pgfpathlineto{\pgfqpoint{3.513997in}{2.032409in}}%
\pgfpathlineto{\pgfqpoint{3.528256in}{2.022756in}}%
\pgfpathlineto{\pgfqpoint{3.542519in}{2.013130in}}%
\pgfpathlineto{\pgfqpoint{3.556785in}{2.003530in}}%
\pgfpathlineto{\pgfqpoint{3.548037in}{2.021402in}}%
\pgfpathlineto{\pgfqpoint{3.539266in}{2.040010in}}%
\pgfpathlineto{\pgfqpoint{3.530471in}{2.059368in}}%
\pgfpathlineto{\pgfqpoint{3.521653in}{2.079492in}}%
\pgfpathlineto{\pgfqpoint{3.507334in}{2.089565in}}%
\pgfpathlineto{\pgfqpoint{3.493019in}{2.099665in}}%
\pgfpathlineto{\pgfqpoint{3.478707in}{2.109792in}}%
\pgfpathlineto{\pgfqpoint{3.464398in}{2.119946in}}%
\pgfpathlineto{\pgfqpoint{3.473271in}{2.099341in}}%
\pgfpathlineto{\pgfqpoint{3.482119in}{2.079506in}}%
\pgfpathlineto{\pgfqpoint{3.490942in}{2.060427in}}%
\pgfpathlineto{\pgfqpoint{3.499741in}{2.042088in}}%
\pgfpathclose%
\pgfusepath{fill}%
\end{pgfscope}%
\begin{pgfscope}%
\pgfpathrectangle{\pgfqpoint{1.150000in}{0.150000in}}{\pgfqpoint{5.700000in}{5.700000in}}%
\pgfusepath{clip}%
\pgfsetbuttcap%
\pgfsetroundjoin%
\definecolor{currentfill}{rgb}{0.678489,0.863742,0.189503}%
\pgfsetfillcolor{currentfill}%
\pgfsetfillopacity{0.700000}%
\pgfsetlinewidth{0.000000pt}%
\definecolor{currentstroke}{rgb}{0.000000,0.000000,0.000000}%
\pgfsetstrokecolor{currentstroke}%
\pgfsetdash{}{0pt}%
\pgfpathmoveto{\pgfqpoint{2.001630in}{3.440545in}}%
\pgfpathlineto{\pgfqpoint{2.015868in}{3.426212in}}%
\pgfpathlineto{\pgfqpoint{2.030105in}{3.411933in}}%
\pgfpathlineto{\pgfqpoint{2.044341in}{3.397706in}}%
\pgfpathlineto{\pgfqpoint{2.058576in}{3.383531in}}%
\pgfpathlineto{\pgfqpoint{2.047816in}{3.420772in}}%
\pgfpathlineto{\pgfqpoint{2.036995in}{3.459015in}}%
\pgfpathlineto{\pgfqpoint{2.026112in}{3.498280in}}%
\pgfpathlineto{\pgfqpoint{2.015165in}{3.538586in}}%
\pgfpathlineto{\pgfqpoint{2.000841in}{3.553315in}}%
\pgfpathlineto{\pgfqpoint{1.986516in}{3.568098in}}%
\pgfpathlineto{\pgfqpoint{1.972190in}{3.582933in}}%
\pgfpathlineto{\pgfqpoint{1.957863in}{3.597823in}}%
\pgfpathlineto{\pgfqpoint{1.968901in}{3.556952in}}%
\pgfpathlineto{\pgfqpoint{1.979874in}{3.517128in}}%
\pgfpathlineto{\pgfqpoint{1.990783in}{3.478332in}}%
\pgfpathlineto{\pgfqpoint{2.001630in}{3.440545in}}%
\pgfpathclose%
\pgfusepath{fill}%
\end{pgfscope}%
\begin{pgfscope}%
\pgfpathrectangle{\pgfqpoint{1.150000in}{0.150000in}}{\pgfqpoint{5.700000in}{5.700000in}}%
\pgfusepath{clip}%
\pgfsetbuttcap%
\pgfsetroundjoin%
\definecolor{currentfill}{rgb}{0.280868,0.160771,0.472899}%
\pgfsetfillcolor{currentfill}%
\pgfsetfillopacity{0.700000}%
\pgfsetlinewidth{0.000000pt}%
\definecolor{currentstroke}{rgb}{0.000000,0.000000,0.000000}%
\pgfsetstrokecolor{currentstroke}%
\pgfsetdash{}{0pt}%
\pgfpathmoveto{\pgfqpoint{4.458978in}{1.417428in}}%
\pgfpathlineto{\pgfqpoint{4.473388in}{1.410543in}}%
\pgfpathlineto{\pgfqpoint{4.487805in}{1.403682in}}%
\pgfpathlineto{\pgfqpoint{4.502226in}{1.396844in}}%
\pgfpathlineto{\pgfqpoint{4.516654in}{1.390029in}}%
\pgfpathlineto{\pgfqpoint{4.508643in}{1.393956in}}%
\pgfpathlineto{\pgfqpoint{4.500626in}{1.398400in}}%
\pgfpathlineto{\pgfqpoint{4.492603in}{1.403373in}}%
\pgfpathlineto{\pgfqpoint{4.484574in}{1.408887in}}%
\pgfpathlineto{\pgfqpoint{4.470118in}{1.416117in}}%
\pgfpathlineto{\pgfqpoint{4.455668in}{1.423371in}}%
\pgfpathlineto{\pgfqpoint{4.441223in}{1.430648in}}%
\pgfpathlineto{\pgfqpoint{4.426784in}{1.437949in}}%
\pgfpathlineto{\pgfqpoint{4.434843in}{1.432014in}}%
\pgfpathlineto{\pgfqpoint{4.442894in}{1.426623in}}%
\pgfpathlineto{\pgfqpoint{4.450940in}{1.421765in}}%
\pgfpathlineto{\pgfqpoint{4.458978in}{1.417428in}}%
\pgfpathclose%
\pgfusepath{fill}%
\end{pgfscope}%
\begin{pgfscope}%
\pgfpathrectangle{\pgfqpoint{1.150000in}{0.150000in}}{\pgfqpoint{5.700000in}{5.700000in}}%
\pgfusepath{clip}%
\pgfsetbuttcap%
\pgfsetroundjoin%
\definecolor{currentfill}{rgb}{0.280894,0.078907,0.402329}%
\pgfsetfillcolor{currentfill}%
\pgfsetfillopacity{0.700000}%
\pgfsetlinewidth{0.000000pt}%
\definecolor{currentstroke}{rgb}{0.000000,0.000000,0.000000}%
\pgfsetstrokecolor{currentstroke}%
\pgfsetdash{}{0pt}%
\pgfpathmoveto{\pgfqpoint{4.869431in}{1.260175in}}%
\pgfpathlineto{\pgfqpoint{4.883943in}{1.254632in}}%
\pgfpathlineto{\pgfqpoint{4.898462in}{1.249112in}}%
\pgfpathlineto{\pgfqpoint{4.912987in}{1.243614in}}%
\pgfpathlineto{\pgfqpoint{4.927518in}{1.238140in}}%
\pgfpathlineto{\pgfqpoint{4.919668in}{1.236051in}}%
\pgfpathlineto{\pgfqpoint{4.911815in}{1.234369in}}%
\pgfpathlineto{\pgfqpoint{4.903962in}{1.233105in}}%
\pgfpathlineto{\pgfqpoint{4.896107in}{1.232269in}}%
\pgfpathlineto{\pgfqpoint{4.881557in}{1.238129in}}%
\pgfpathlineto{\pgfqpoint{4.867013in}{1.244012in}}%
\pgfpathlineto{\pgfqpoint{4.852476in}{1.249918in}}%
\pgfpathlineto{\pgfqpoint{4.837946in}{1.255847in}}%
\pgfpathlineto{\pgfqpoint{4.845820in}{1.256292in}}%
\pgfpathlineto{\pgfqpoint{4.853693in}{1.257168in}}%
\pgfpathlineto{\pgfqpoint{4.861563in}{1.258466in}}%
\pgfpathlineto{\pgfqpoint{4.869431in}{1.260175in}}%
\pgfpathclose%
\pgfusepath{fill}%
\end{pgfscope}%
\begin{pgfscope}%
\pgfpathrectangle{\pgfqpoint{1.150000in}{0.150000in}}{\pgfqpoint{5.700000in}{5.700000in}}%
\pgfusepath{clip}%
\pgfsetbuttcap%
\pgfsetroundjoin%
\definecolor{currentfill}{rgb}{0.626579,0.854645,0.223353}%
\pgfsetfillcolor{currentfill}%
\pgfsetfillopacity{0.700000}%
\pgfsetlinewidth{0.000000pt}%
\definecolor{currentstroke}{rgb}{0.000000,0.000000,0.000000}%
\pgfsetstrokecolor{currentstroke}%
\pgfsetdash{}{0pt}%
\pgfpathmoveto{\pgfqpoint{2.058576in}{3.383531in}}%
\pgfpathlineto{\pgfqpoint{2.072811in}{3.369408in}}%
\pgfpathlineto{\pgfqpoint{2.087045in}{3.355336in}}%
\pgfpathlineto{\pgfqpoint{2.101278in}{3.341315in}}%
\pgfpathlineto{\pgfqpoint{2.115511in}{3.327343in}}%
\pgfpathlineto{\pgfqpoint{2.104837in}{3.364041in}}%
\pgfpathlineto{\pgfqpoint{2.094103in}{3.401734in}}%
\pgfpathlineto{\pgfqpoint{2.083308in}{3.440443in}}%
\pgfpathlineto{\pgfqpoint{2.072452in}{3.480186in}}%
\pgfpathlineto{\pgfqpoint{2.058131in}{3.494709in}}%
\pgfpathlineto{\pgfqpoint{2.043810in}{3.509283in}}%
\pgfpathlineto{\pgfqpoint{2.029488in}{3.523909in}}%
\pgfpathlineto{\pgfqpoint{2.015165in}{3.538586in}}%
\pgfpathlineto{\pgfqpoint{2.026112in}{3.498280in}}%
\pgfpathlineto{\pgfqpoint{2.036995in}{3.459015in}}%
\pgfpathlineto{\pgfqpoint{2.047816in}{3.420772in}}%
\pgfpathlineto{\pgfqpoint{2.058576in}{3.383531in}}%
\pgfpathclose%
\pgfusepath{fill}%
\end{pgfscope}%
\begin{pgfscope}%
\pgfpathrectangle{\pgfqpoint{1.150000in}{0.150000in}}{\pgfqpoint{5.700000in}{5.700000in}}%
\pgfusepath{clip}%
\pgfsetbuttcap%
\pgfsetroundjoin%
\definecolor{currentfill}{rgb}{0.125394,0.574318,0.549086}%
\pgfsetfillcolor{currentfill}%
\pgfsetfillopacity{0.700000}%
\pgfsetlinewidth{0.000000pt}%
\definecolor{currentstroke}{rgb}{0.000000,0.000000,0.000000}%
\pgfsetstrokecolor{currentstroke}%
\pgfsetdash{}{0pt}%
\pgfpathmoveto{\pgfqpoint{3.008040in}{2.459860in}}%
\pgfpathlineto{\pgfqpoint{3.022262in}{2.448777in}}%
\pgfpathlineto{\pgfqpoint{3.036485in}{2.437725in}}%
\pgfpathlineto{\pgfqpoint{3.050711in}{2.426704in}}%
\pgfpathlineto{\pgfqpoint{3.064939in}{2.415714in}}%
\pgfpathlineto{\pgfqpoint{3.055626in}{2.440533in}}%
\pgfpathlineto{\pgfqpoint{3.046278in}{2.466189in}}%
\pgfpathlineto{\pgfqpoint{3.036895in}{2.492697in}}%
\pgfpathlineto{\pgfqpoint{3.027477in}{2.520074in}}%
\pgfpathlineto{\pgfqpoint{3.013184in}{2.531566in}}%
\pgfpathlineto{\pgfqpoint{2.998893in}{2.543088in}}%
\pgfpathlineto{\pgfqpoint{2.984605in}{2.554642in}}%
\pgfpathlineto{\pgfqpoint{2.970318in}{2.566228in}}%
\pgfpathlineto{\pgfqpoint{2.979803in}{2.538341in}}%
\pgfpathlineto{\pgfqpoint{2.989252in}{2.511328in}}%
\pgfpathlineto{\pgfqpoint{2.998664in}{2.485173in}}%
\pgfpathlineto{\pgfqpoint{3.008040in}{2.459860in}}%
\pgfpathclose%
\pgfusepath{fill}%
\end{pgfscope}%
\begin{pgfscope}%
\pgfpathrectangle{\pgfqpoint{1.150000in}{0.150000in}}{\pgfqpoint{5.700000in}{5.700000in}}%
\pgfusepath{clip}%
\pgfsetbuttcap%
\pgfsetroundjoin%
\definecolor{currentfill}{rgb}{0.575563,0.844566,0.256415}%
\pgfsetfillcolor{currentfill}%
\pgfsetfillopacity{0.700000}%
\pgfsetlinewidth{0.000000pt}%
\definecolor{currentstroke}{rgb}{0.000000,0.000000,0.000000}%
\pgfsetstrokecolor{currentstroke}%
\pgfsetdash{}{0pt}%
\pgfpathmoveto{\pgfqpoint{2.115511in}{3.327343in}}%
\pgfpathlineto{\pgfqpoint{2.129743in}{3.313422in}}%
\pgfpathlineto{\pgfqpoint{2.143975in}{3.299549in}}%
\pgfpathlineto{\pgfqpoint{2.158207in}{3.285726in}}%
\pgfpathlineto{\pgfqpoint{2.172438in}{3.271950in}}%
\pgfpathlineto{\pgfqpoint{2.161848in}{3.308106in}}%
\pgfpathlineto{\pgfqpoint{2.151200in}{3.345253in}}%
\pgfpathlineto{\pgfqpoint{2.140493in}{3.383408in}}%
\pgfpathlineto{\pgfqpoint{2.129725in}{3.422590in}}%
\pgfpathlineto{\pgfqpoint{2.115408in}{3.436915in}}%
\pgfpathlineto{\pgfqpoint{2.101090in}{3.451289in}}%
\pgfpathlineto{\pgfqpoint{2.086771in}{3.465712in}}%
\pgfpathlineto{\pgfqpoint{2.072452in}{3.480186in}}%
\pgfpathlineto{\pgfqpoint{2.083308in}{3.440443in}}%
\pgfpathlineto{\pgfqpoint{2.094103in}{3.401734in}}%
\pgfpathlineto{\pgfqpoint{2.104837in}{3.364041in}}%
\pgfpathlineto{\pgfqpoint{2.115511in}{3.327343in}}%
\pgfpathclose%
\pgfusepath{fill}%
\end{pgfscope}%
\begin{pgfscope}%
\pgfpathrectangle{\pgfqpoint{1.150000in}{0.150000in}}{\pgfqpoint{5.700000in}{5.700000in}}%
\pgfusepath{clip}%
\pgfsetbuttcap%
\pgfsetroundjoin%
\definecolor{currentfill}{rgb}{0.237441,0.305202,0.541921}%
\pgfsetfillcolor{currentfill}%
\pgfsetfillopacity{0.700000}%
\pgfsetlinewidth{0.000000pt}%
\definecolor{currentstroke}{rgb}{0.000000,0.000000,0.000000}%
\pgfsetstrokecolor{currentstroke}%
\pgfsetdash{}{0pt}%
\pgfpathmoveto{\pgfqpoint{3.933971in}{1.727435in}}%
\pgfpathlineto{\pgfqpoint{3.948290in}{1.718932in}}%
\pgfpathlineto{\pgfqpoint{3.962613in}{1.710453in}}%
\pgfpathlineto{\pgfqpoint{3.976941in}{1.701998in}}%
\pgfpathlineto{\pgfqpoint{3.991274in}{1.693569in}}%
\pgfpathlineto{\pgfqpoint{3.982916in}{1.705514in}}%
\pgfpathlineto{\pgfqpoint{3.974545in}{1.718108in}}%
\pgfpathlineto{\pgfqpoint{3.966158in}{1.731364in}}%
\pgfpathlineto{\pgfqpoint{3.957757in}{1.745296in}}%
\pgfpathlineto{\pgfqpoint{3.943383in}{1.754177in}}%
\pgfpathlineto{\pgfqpoint{3.929013in}{1.763083in}}%
\pgfpathlineto{\pgfqpoint{3.914647in}{1.772013in}}%
\pgfpathlineto{\pgfqpoint{3.900285in}{1.780968in}}%
\pgfpathlineto{\pgfqpoint{3.908730in}{1.766578in}}%
\pgfpathlineto{\pgfqpoint{3.917159in}{1.752868in}}%
\pgfpathlineto{\pgfqpoint{3.925573in}{1.739825in}}%
\pgfpathlineto{\pgfqpoint{3.933971in}{1.727435in}}%
\pgfpathclose%
\pgfusepath{fill}%
\end{pgfscope}%
\begin{pgfscope}%
\pgfpathrectangle{\pgfqpoint{1.150000in}{0.150000in}}{\pgfqpoint{5.700000in}{5.700000in}}%
\pgfusepath{clip}%
\pgfsetbuttcap%
\pgfsetroundjoin%
\definecolor{currentfill}{rgb}{0.187231,0.414746,0.556547}%
\pgfsetfillcolor{currentfill}%
\pgfsetfillopacity{0.700000}%
\pgfsetlinewidth{0.000000pt}%
\definecolor{currentstroke}{rgb}{0.000000,0.000000,0.000000}%
\pgfsetstrokecolor{currentstroke}%
\pgfsetdash{}{0pt}%
\pgfpathmoveto{\pgfqpoint{3.556785in}{2.003530in}}%
\pgfpathlineto{\pgfqpoint{3.571055in}{1.993958in}}%
\pgfpathlineto{\pgfqpoint{3.585328in}{1.984412in}}%
\pgfpathlineto{\pgfqpoint{3.599605in}{1.974892in}}%
\pgfpathlineto{\pgfqpoint{3.613885in}{1.965399in}}%
\pgfpathlineto{\pgfqpoint{3.605187in}{1.982804in}}%
\pgfpathlineto{\pgfqpoint{3.596467in}{2.000941in}}%
\pgfpathlineto{\pgfqpoint{3.587725in}{2.019824in}}%
\pgfpathlineto{\pgfqpoint{3.578960in}{2.039467in}}%
\pgfpathlineto{\pgfqpoint{3.564628in}{2.049433in}}%
\pgfpathlineto{\pgfqpoint{3.550299in}{2.059426in}}%
\pgfpathlineto{\pgfqpoint{3.535974in}{2.069445in}}%
\pgfpathlineto{\pgfqpoint{3.521653in}{2.079492in}}%
\pgfpathlineto{\pgfqpoint{3.530471in}{2.059368in}}%
\pgfpathlineto{\pgfqpoint{3.539266in}{2.040010in}}%
\pgfpathlineto{\pgfqpoint{3.548037in}{2.021402in}}%
\pgfpathlineto{\pgfqpoint{3.556785in}{2.003530in}}%
\pgfpathclose%
\pgfusepath{fill}%
\end{pgfscope}%
\begin{pgfscope}%
\pgfpathrectangle{\pgfqpoint{1.150000in}{0.150000in}}{\pgfqpoint{5.700000in}{5.700000in}}%
\pgfusepath{clip}%
\pgfsetbuttcap%
\pgfsetroundjoin%
\definecolor{currentfill}{rgb}{0.525776,0.833491,0.288127}%
\pgfsetfillcolor{currentfill}%
\pgfsetfillopacity{0.700000}%
\pgfsetlinewidth{0.000000pt}%
\definecolor{currentstroke}{rgb}{0.000000,0.000000,0.000000}%
\pgfsetstrokecolor{currentstroke}%
\pgfsetdash{}{0pt}%
\pgfpathmoveto{\pgfqpoint{2.172438in}{3.271950in}}%
\pgfpathlineto{\pgfqpoint{2.186669in}{3.258223in}}%
\pgfpathlineto{\pgfqpoint{2.200900in}{3.244542in}}%
\pgfpathlineto{\pgfqpoint{2.215130in}{3.230909in}}%
\pgfpathlineto{\pgfqpoint{2.229361in}{3.217322in}}%
\pgfpathlineto{\pgfqpoint{2.218854in}{3.252939in}}%
\pgfpathlineto{\pgfqpoint{2.208291in}{3.289540in}}%
\pgfpathlineto{\pgfqpoint{2.197670in}{3.327144in}}%
\pgfpathlineto{\pgfqpoint{2.186990in}{3.365768in}}%
\pgfpathlineto{\pgfqpoint{2.172675in}{3.379902in}}%
\pgfpathlineto{\pgfqpoint{2.158359in}{3.394084in}}%
\pgfpathlineto{\pgfqpoint{2.144042in}{3.408313in}}%
\pgfpathlineto{\pgfqpoint{2.129725in}{3.422590in}}%
\pgfpathlineto{\pgfqpoint{2.140493in}{3.383408in}}%
\pgfpathlineto{\pgfqpoint{2.151200in}{3.345253in}}%
\pgfpathlineto{\pgfqpoint{2.161848in}{3.308106in}}%
\pgfpathlineto{\pgfqpoint{2.172438in}{3.271950in}}%
\pgfpathclose%
\pgfusepath{fill}%
\end{pgfscope}%
\begin{pgfscope}%
\pgfpathrectangle{\pgfqpoint{1.150000in}{0.150000in}}{\pgfqpoint{5.700000in}{5.700000in}}%
\pgfusepath{clip}%
\pgfsetbuttcap%
\pgfsetroundjoin%
\definecolor{currentfill}{rgb}{0.269308,0.218818,0.509577}%
\pgfsetfillcolor{currentfill}%
\pgfsetfillopacity{0.700000}%
\pgfsetlinewidth{0.000000pt}%
\definecolor{currentstroke}{rgb}{0.000000,0.000000,0.000000}%
\pgfsetstrokecolor{currentstroke}%
\pgfsetdash{}{0pt}%
\pgfpathmoveto{\pgfqpoint{4.253922in}{1.527397in}}%
\pgfpathlineto{\pgfqpoint{4.268299in}{1.519813in}}%
\pgfpathlineto{\pgfqpoint{4.282680in}{1.512252in}}%
\pgfpathlineto{\pgfqpoint{4.297067in}{1.504716in}}%
\pgfpathlineto{\pgfqpoint{4.311459in}{1.497203in}}%
\pgfpathlineto{\pgfqpoint{4.303330in}{1.504547in}}%
\pgfpathlineto{\pgfqpoint{4.295191in}{1.512468in}}%
\pgfpathlineto{\pgfqpoint{4.287044in}{1.520978in}}%
\pgfpathlineto{\pgfqpoint{4.278887in}{1.530090in}}%
\pgfpathlineto{\pgfqpoint{4.264461in}{1.538036in}}%
\pgfpathlineto{\pgfqpoint{4.250040in}{1.546005in}}%
\pgfpathlineto{\pgfqpoint{4.235624in}{1.553998in}}%
\pgfpathlineto{\pgfqpoint{4.221212in}{1.562015in}}%
\pgfpathlineto{\pgfqpoint{4.229405in}{1.552464in}}%
\pgfpathlineto{\pgfqpoint{4.237587in}{1.543519in}}%
\pgfpathlineto{\pgfqpoint{4.245759in}{1.535168in}}%
\pgfpathlineto{\pgfqpoint{4.253922in}{1.527397in}}%
\pgfpathclose%
\pgfusepath{fill}%
\end{pgfscope}%
\begin{pgfscope}%
\pgfpathrectangle{\pgfqpoint{1.150000in}{0.150000in}}{\pgfqpoint{5.700000in}{5.700000in}}%
\pgfusepath{clip}%
\pgfsetbuttcap%
\pgfsetroundjoin%
\definecolor{currentfill}{rgb}{0.283091,0.110553,0.431554}%
\pgfsetfillcolor{currentfill}%
\pgfsetfillopacity{0.700000}%
\pgfsetlinewidth{0.000000pt}%
\definecolor{currentstroke}{rgb}{0.000000,0.000000,0.000000}%
\pgfsetstrokecolor{currentstroke}%
\pgfsetdash{}{0pt}%
\pgfpathmoveto{\pgfqpoint{4.721929in}{1.304106in}}%
\pgfpathlineto{\pgfqpoint{4.736409in}{1.297993in}}%
\pgfpathlineto{\pgfqpoint{4.750896in}{1.291903in}}%
\pgfpathlineto{\pgfqpoint{4.765388in}{1.285836in}}%
\pgfpathlineto{\pgfqpoint{4.779887in}{1.279792in}}%
\pgfpathlineto{\pgfqpoint{4.771990in}{1.280185in}}%
\pgfpathlineto{\pgfqpoint{4.764089in}{1.281035in}}%
\pgfpathlineto{\pgfqpoint{4.756186in}{1.282353in}}%
\pgfpathlineto{\pgfqpoint{4.748280in}{1.284151in}}%
\pgfpathlineto{\pgfqpoint{4.733759in}{1.290595in}}%
\pgfpathlineto{\pgfqpoint{4.719244in}{1.297062in}}%
\pgfpathlineto{\pgfqpoint{4.704735in}{1.303552in}}%
\pgfpathlineto{\pgfqpoint{4.690232in}{1.310065in}}%
\pgfpathlineto{\pgfqpoint{4.698161in}{1.307862in}}%
\pgfpathlineto{\pgfqpoint{4.706087in}{1.306142in}}%
\pgfpathlineto{\pgfqpoint{4.714010in}{1.304893in}}%
\pgfpathlineto{\pgfqpoint{4.721929in}{1.304106in}}%
\pgfpathclose%
\pgfusepath{fill}%
\end{pgfscope}%
\begin{pgfscope}%
\pgfpathrectangle{\pgfqpoint{1.150000in}{0.150000in}}{\pgfqpoint{5.700000in}{5.700000in}}%
\pgfusepath{clip}%
\pgfsetbuttcap%
\pgfsetroundjoin%
\definecolor{currentfill}{rgb}{0.128729,0.563265,0.551229}%
\pgfsetfillcolor{currentfill}%
\pgfsetfillopacity{0.700000}%
\pgfsetlinewidth{0.000000pt}%
\definecolor{currentstroke}{rgb}{0.000000,0.000000,0.000000}%
\pgfsetstrokecolor{currentstroke}%
\pgfsetdash{}{0pt}%
\pgfpathmoveto{\pgfqpoint{3.064939in}{2.415714in}}%
\pgfpathlineto{\pgfqpoint{3.079169in}{2.404755in}}%
\pgfpathlineto{\pgfqpoint{3.093402in}{2.393826in}}%
\pgfpathlineto{\pgfqpoint{3.107638in}{2.382928in}}%
\pgfpathlineto{\pgfqpoint{3.121875in}{2.372059in}}%
\pgfpathlineto{\pgfqpoint{3.112624in}{2.396386in}}%
\pgfpathlineto{\pgfqpoint{3.103340in}{2.421544in}}%
\pgfpathlineto{\pgfqpoint{3.094022in}{2.447548in}}%
\pgfpathlineto{\pgfqpoint{3.084670in}{2.474416in}}%
\pgfpathlineto{\pgfqpoint{3.070368in}{2.485785in}}%
\pgfpathlineto{\pgfqpoint{3.056069in}{2.497184in}}%
\pgfpathlineto{\pgfqpoint{3.041772in}{2.508614in}}%
\pgfpathlineto{\pgfqpoint{3.027477in}{2.520074in}}%
\pgfpathlineto{\pgfqpoint{3.036895in}{2.492697in}}%
\pgfpathlineto{\pgfqpoint{3.046278in}{2.466189in}}%
\pgfpathlineto{\pgfqpoint{3.055626in}{2.440533in}}%
\pgfpathlineto{\pgfqpoint{3.064939in}{2.415714in}}%
\pgfpathclose%
\pgfusepath{fill}%
\end{pgfscope}%
\begin{pgfscope}%
\pgfpathrectangle{\pgfqpoint{1.150000in}{0.150000in}}{\pgfqpoint{5.700000in}{5.700000in}}%
\pgfusepath{clip}%
\pgfsetbuttcap%
\pgfsetroundjoin%
\definecolor{currentfill}{rgb}{0.477504,0.821444,0.318195}%
\pgfsetfillcolor{currentfill}%
\pgfsetfillopacity{0.700000}%
\pgfsetlinewidth{0.000000pt}%
\definecolor{currentstroke}{rgb}{0.000000,0.000000,0.000000}%
\pgfsetstrokecolor{currentstroke}%
\pgfsetdash{}{0pt}%
\pgfpathmoveto{\pgfqpoint{2.229361in}{3.217322in}}%
\pgfpathlineto{\pgfqpoint{2.243591in}{3.203781in}}%
\pgfpathlineto{\pgfqpoint{2.257821in}{3.190286in}}%
\pgfpathlineto{\pgfqpoint{2.272052in}{3.176835in}}%
\pgfpathlineto{\pgfqpoint{2.286282in}{3.163430in}}%
\pgfpathlineto{\pgfqpoint{2.275858in}{3.198510in}}%
\pgfpathlineto{\pgfqpoint{2.265379in}{3.234568in}}%
\pgfpathlineto{\pgfqpoint{2.254843in}{3.271622in}}%
\pgfpathlineto{\pgfqpoint{2.244249in}{3.309691in}}%
\pgfpathlineto{\pgfqpoint{2.229935in}{3.323642in}}%
\pgfpathlineto{\pgfqpoint{2.215620in}{3.337638in}}%
\pgfpathlineto{\pgfqpoint{2.201305in}{3.351680in}}%
\pgfpathlineto{\pgfqpoint{2.186990in}{3.365768in}}%
\pgfpathlineto{\pgfqpoint{2.197670in}{3.327144in}}%
\pgfpathlineto{\pgfqpoint{2.208291in}{3.289540in}}%
\pgfpathlineto{\pgfqpoint{2.218854in}{3.252939in}}%
\pgfpathlineto{\pgfqpoint{2.229361in}{3.217322in}}%
\pgfpathclose%
\pgfusepath{fill}%
\end{pgfscope}%
\begin{pgfscope}%
\pgfpathrectangle{\pgfqpoint{1.150000in}{0.150000in}}{\pgfqpoint{5.700000in}{5.700000in}}%
\pgfusepath{clip}%
\pgfsetbuttcap%
\pgfsetroundjoin%
\definecolor{currentfill}{rgb}{0.281412,0.155834,0.469201}%
\pgfsetfillcolor{currentfill}%
\pgfsetfillopacity{0.700000}%
\pgfsetlinewidth{0.000000pt}%
\definecolor{currentstroke}{rgb}{0.000000,0.000000,0.000000}%
\pgfsetstrokecolor{currentstroke}%
\pgfsetdash{}{0pt}%
\pgfpathmoveto{\pgfqpoint{4.516654in}{1.390029in}}%
\pgfpathlineto{\pgfqpoint{4.531087in}{1.383237in}}%
\pgfpathlineto{\pgfqpoint{4.545525in}{1.376469in}}%
\pgfpathlineto{\pgfqpoint{4.559970in}{1.369725in}}%
\pgfpathlineto{\pgfqpoint{4.574420in}{1.363003in}}%
\pgfpathlineto{\pgfqpoint{4.566436in}{1.366520in}}%
\pgfpathlineto{\pgfqpoint{4.558446in}{1.370550in}}%
\pgfpathlineto{\pgfqpoint{4.550452in}{1.375106in}}%
\pgfpathlineto{\pgfqpoint{4.542451in}{1.380198in}}%
\pgfpathlineto{\pgfqpoint{4.527974in}{1.387335in}}%
\pgfpathlineto{\pgfqpoint{4.513501in}{1.394496in}}%
\pgfpathlineto{\pgfqpoint{4.499035in}{1.401680in}}%
\pgfpathlineto{\pgfqpoint{4.484574in}{1.408887in}}%
\pgfpathlineto{\pgfqpoint{4.492603in}{1.403373in}}%
\pgfpathlineto{\pgfqpoint{4.500626in}{1.398400in}}%
\pgfpathlineto{\pgfqpoint{4.508643in}{1.393956in}}%
\pgfpathlineto{\pgfqpoint{4.516654in}{1.390029in}}%
\pgfpathclose%
\pgfusepath{fill}%
\end{pgfscope}%
\begin{pgfscope}%
\pgfpathrectangle{\pgfqpoint{1.150000in}{0.150000in}}{\pgfqpoint{5.700000in}{5.700000in}}%
\pgfusepath{clip}%
\pgfsetbuttcap%
\pgfsetroundjoin%
\definecolor{currentfill}{rgb}{0.280894,0.078907,0.402329}%
\pgfsetfillcolor{currentfill}%
\pgfsetfillopacity{0.700000}%
\pgfsetlinewidth{0.000000pt}%
\definecolor{currentstroke}{rgb}{0.000000,0.000000,0.000000}%
\pgfsetstrokecolor{currentstroke}%
\pgfsetdash{}{0pt}%
\pgfpathmoveto{\pgfqpoint{4.927518in}{1.238140in}}%
\pgfpathlineto{\pgfqpoint{4.942057in}{1.232689in}}%
\pgfpathlineto{\pgfqpoint{4.956602in}{1.227260in}}%
\pgfpathlineto{\pgfqpoint{4.971153in}{1.221854in}}%
\pgfpathlineto{\pgfqpoint{4.963316in}{1.219480in}}%
\pgfpathlineto{\pgfqpoint{4.955476in}{1.217510in}}%
\pgfpathlineto{\pgfqpoint{4.947636in}{1.215955in}}%
\pgfpathlineto{\pgfqpoint{4.939795in}{1.214826in}}%
\pgfpathlineto{\pgfqpoint{4.925226in}{1.220617in}}%
\pgfpathlineto{\pgfqpoint{4.910663in}{1.226432in}}%
\pgfpathlineto{\pgfqpoint{4.896107in}{1.232269in}}%
\pgfpathlineto{\pgfqpoint{4.903962in}{1.233105in}}%
\pgfpathlineto{\pgfqpoint{4.911815in}{1.234369in}}%
\pgfpathlineto{\pgfqpoint{4.919668in}{1.236051in}}%
\pgfpathlineto{\pgfqpoint{4.927518in}{1.238140in}}%
\pgfpathclose%
\pgfusepath{fill}%
\end{pgfscope}%
\begin{pgfscope}%
\pgfpathrectangle{\pgfqpoint{1.150000in}{0.150000in}}{\pgfqpoint{5.700000in}{5.700000in}}%
\pgfusepath{clip}%
\pgfsetbuttcap%
\pgfsetroundjoin%
\definecolor{currentfill}{rgb}{0.430983,0.808473,0.346476}%
\pgfsetfillcolor{currentfill}%
\pgfsetfillopacity{0.700000}%
\pgfsetlinewidth{0.000000pt}%
\definecolor{currentstroke}{rgb}{0.000000,0.000000,0.000000}%
\pgfsetstrokecolor{currentstroke}%
\pgfsetdash{}{0pt}%
\pgfpathmoveto{\pgfqpoint{2.286282in}{3.163430in}}%
\pgfpathlineto{\pgfqpoint{2.300513in}{3.150069in}}%
\pgfpathlineto{\pgfqpoint{2.314744in}{3.136752in}}%
\pgfpathlineto{\pgfqpoint{2.328975in}{3.123479in}}%
\pgfpathlineto{\pgfqpoint{2.343206in}{3.110248in}}%
\pgfpathlineto{\pgfqpoint{2.332863in}{3.144794in}}%
\pgfpathlineto{\pgfqpoint{2.322466in}{3.180311in}}%
\pgfpathlineto{\pgfqpoint{2.312014in}{3.216818in}}%
\pgfpathlineto{\pgfqpoint{2.301507in}{3.254333in}}%
\pgfpathlineto{\pgfqpoint{2.287192in}{3.268107in}}%
\pgfpathlineto{\pgfqpoint{2.272878in}{3.281924in}}%
\pgfpathlineto{\pgfqpoint{2.258564in}{3.295785in}}%
\pgfpathlineto{\pgfqpoint{2.244249in}{3.309691in}}%
\pgfpathlineto{\pgfqpoint{2.254843in}{3.271622in}}%
\pgfpathlineto{\pgfqpoint{2.265379in}{3.234568in}}%
\pgfpathlineto{\pgfqpoint{2.275858in}{3.198510in}}%
\pgfpathlineto{\pgfqpoint{2.286282in}{3.163430in}}%
\pgfpathclose%
\pgfusepath{fill}%
\end{pgfscope}%
\begin{pgfscope}%
\pgfpathrectangle{\pgfqpoint{1.150000in}{0.150000in}}{\pgfqpoint{5.700000in}{5.700000in}}%
\pgfusepath{clip}%
\pgfsetbuttcap%
\pgfsetroundjoin%
\definecolor{currentfill}{rgb}{0.192357,0.403199,0.555836}%
\pgfsetfillcolor{currentfill}%
\pgfsetfillopacity{0.700000}%
\pgfsetlinewidth{0.000000pt}%
\definecolor{currentstroke}{rgb}{0.000000,0.000000,0.000000}%
\pgfsetstrokecolor{currentstroke}%
\pgfsetdash{}{0pt}%
\pgfpathmoveto{\pgfqpoint{3.613885in}{1.965399in}}%
\pgfpathlineto{\pgfqpoint{3.628169in}{1.955932in}}%
\pgfpathlineto{\pgfqpoint{3.642457in}{1.946492in}}%
\pgfpathlineto{\pgfqpoint{3.656748in}{1.937077in}}%
\pgfpathlineto{\pgfqpoint{3.671043in}{1.927689in}}%
\pgfpathlineto{\pgfqpoint{3.662394in}{1.944629in}}%
\pgfpathlineto{\pgfqpoint{3.653724in}{1.962296in}}%
\pgfpathlineto{\pgfqpoint{3.645033in}{1.980704in}}%
\pgfpathlineto{\pgfqpoint{3.636320in}{1.999867in}}%
\pgfpathlineto{\pgfqpoint{3.621975in}{2.009728in}}%
\pgfpathlineto{\pgfqpoint{3.607633in}{2.019614in}}%
\pgfpathlineto{\pgfqpoint{3.593295in}{2.029528in}}%
\pgfpathlineto{\pgfqpoint{3.578960in}{2.039467in}}%
\pgfpathlineto{\pgfqpoint{3.587725in}{2.019824in}}%
\pgfpathlineto{\pgfqpoint{3.596467in}{2.000941in}}%
\pgfpathlineto{\pgfqpoint{3.605187in}{1.982804in}}%
\pgfpathlineto{\pgfqpoint{3.613885in}{1.965399in}}%
\pgfpathclose%
\pgfusepath{fill}%
\end{pgfscope}%
\begin{pgfscope}%
\pgfpathrectangle{\pgfqpoint{1.150000in}{0.150000in}}{\pgfqpoint{5.700000in}{5.700000in}}%
\pgfusepath{clip}%
\pgfsetbuttcap%
\pgfsetroundjoin%
\definecolor{currentfill}{rgb}{0.133743,0.548535,0.553541}%
\pgfsetfillcolor{currentfill}%
\pgfsetfillopacity{0.700000}%
\pgfsetlinewidth{0.000000pt}%
\definecolor{currentstroke}{rgb}{0.000000,0.000000,0.000000}%
\pgfsetstrokecolor{currentstroke}%
\pgfsetdash{}{0pt}%
\pgfpathmoveto{\pgfqpoint{3.121875in}{2.372059in}}%
\pgfpathlineto{\pgfqpoint{3.136116in}{2.361221in}}%
\pgfpathlineto{\pgfqpoint{3.150359in}{2.350413in}}%
\pgfpathlineto{\pgfqpoint{3.164604in}{2.339635in}}%
\pgfpathlineto{\pgfqpoint{3.178852in}{2.328886in}}%
\pgfpathlineto{\pgfqpoint{3.169662in}{2.352721in}}%
\pgfpathlineto{\pgfqpoint{3.160440in}{2.377381in}}%
\pgfpathlineto{\pgfqpoint{3.151186in}{2.402884in}}%
\pgfpathlineto{\pgfqpoint{3.141898in}{2.429244in}}%
\pgfpathlineto{\pgfqpoint{3.127588in}{2.440492in}}%
\pgfpathlineto{\pgfqpoint{3.113279in}{2.451770in}}%
\pgfpathlineto{\pgfqpoint{3.098973in}{2.463078in}}%
\pgfpathlineto{\pgfqpoint{3.084670in}{2.474416in}}%
\pgfpathlineto{\pgfqpoint{3.094022in}{2.447548in}}%
\pgfpathlineto{\pgfqpoint{3.103340in}{2.421544in}}%
\pgfpathlineto{\pgfqpoint{3.112624in}{2.396386in}}%
\pgfpathlineto{\pgfqpoint{3.121875in}{2.372059in}}%
\pgfpathclose%
\pgfusepath{fill}%
\end{pgfscope}%
\begin{pgfscope}%
\pgfpathrectangle{\pgfqpoint{1.150000in}{0.150000in}}{\pgfqpoint{5.700000in}{5.700000in}}%
\pgfusepath{clip}%
\pgfsetbuttcap%
\pgfsetroundjoin%
\definecolor{currentfill}{rgb}{0.241237,0.296485,0.539709}%
\pgfsetfillcolor{currentfill}%
\pgfsetfillopacity{0.700000}%
\pgfsetlinewidth{0.000000pt}%
\definecolor{currentstroke}{rgb}{0.000000,0.000000,0.000000}%
\pgfsetstrokecolor{currentstroke}%
\pgfsetdash{}{0pt}%
\pgfpathmoveto{\pgfqpoint{3.991274in}{1.693569in}}%
\pgfpathlineto{\pgfqpoint{4.005610in}{1.685164in}}%
\pgfpathlineto{\pgfqpoint{4.019952in}{1.676784in}}%
\pgfpathlineto{\pgfqpoint{4.034297in}{1.668428in}}%
\pgfpathlineto{\pgfqpoint{4.048648in}{1.660097in}}%
\pgfpathlineto{\pgfqpoint{4.040330in}{1.671597in}}%
\pgfpathlineto{\pgfqpoint{4.031999in}{1.683743in}}%
\pgfpathlineto{\pgfqpoint{4.023655in}{1.696546in}}%
\pgfpathlineto{\pgfqpoint{4.015296in}{1.710021in}}%
\pgfpathlineto{\pgfqpoint{4.000905in}{1.718803in}}%
\pgfpathlineto{\pgfqpoint{3.986518in}{1.727609in}}%
\pgfpathlineto{\pgfqpoint{3.972135in}{1.736441in}}%
\pgfpathlineto{\pgfqpoint{3.957757in}{1.745296in}}%
\pgfpathlineto{\pgfqpoint{3.966158in}{1.731364in}}%
\pgfpathlineto{\pgfqpoint{3.974545in}{1.718108in}}%
\pgfpathlineto{\pgfqpoint{3.982916in}{1.705514in}}%
\pgfpathlineto{\pgfqpoint{3.991274in}{1.693569in}}%
\pgfpathclose%
\pgfusepath{fill}%
\end{pgfscope}%
\begin{pgfscope}%
\pgfpathrectangle{\pgfqpoint{1.150000in}{0.150000in}}{\pgfqpoint{5.700000in}{5.700000in}}%
\pgfusepath{clip}%
\pgfsetbuttcap%
\pgfsetroundjoin%
\definecolor{currentfill}{rgb}{0.386433,0.794644,0.372886}%
\pgfsetfillcolor{currentfill}%
\pgfsetfillopacity{0.700000}%
\pgfsetlinewidth{0.000000pt}%
\definecolor{currentstroke}{rgb}{0.000000,0.000000,0.000000}%
\pgfsetstrokecolor{currentstroke}%
\pgfsetdash{}{0pt}%
\pgfpathmoveto{\pgfqpoint{2.343206in}{3.110248in}}%
\pgfpathlineto{\pgfqpoint{2.357438in}{3.097061in}}%
\pgfpathlineto{\pgfqpoint{2.371670in}{3.083916in}}%
\pgfpathlineto{\pgfqpoint{2.385903in}{3.070814in}}%
\pgfpathlineto{\pgfqpoint{2.400136in}{3.057753in}}%
\pgfpathlineto{\pgfqpoint{2.389873in}{3.091766in}}%
\pgfpathlineto{\pgfqpoint{2.379557in}{3.126744in}}%
\pgfpathlineto{\pgfqpoint{2.369188in}{3.162706in}}%
\pgfpathlineto{\pgfqpoint{2.358765in}{3.199670in}}%
\pgfpathlineto{\pgfqpoint{2.344450in}{3.213272in}}%
\pgfpathlineto{\pgfqpoint{2.330135in}{3.226916in}}%
\pgfpathlineto{\pgfqpoint{2.315821in}{3.240603in}}%
\pgfpathlineto{\pgfqpoint{2.301507in}{3.254333in}}%
\pgfpathlineto{\pgfqpoint{2.312014in}{3.216818in}}%
\pgfpathlineto{\pgfqpoint{2.322466in}{3.180311in}}%
\pgfpathlineto{\pgfqpoint{2.332863in}{3.144794in}}%
\pgfpathlineto{\pgfqpoint{2.343206in}{3.110248in}}%
\pgfpathclose%
\pgfusepath{fill}%
\end{pgfscope}%
\begin{pgfscope}%
\pgfpathrectangle{\pgfqpoint{1.150000in}{0.150000in}}{\pgfqpoint{5.700000in}{5.700000in}}%
\pgfusepath{clip}%
\pgfsetbuttcap%
\pgfsetroundjoin%
\definecolor{currentfill}{rgb}{0.271828,0.209303,0.504434}%
\pgfsetfillcolor{currentfill}%
\pgfsetfillopacity{0.700000}%
\pgfsetlinewidth{0.000000pt}%
\definecolor{currentstroke}{rgb}{0.000000,0.000000,0.000000}%
\pgfsetstrokecolor{currentstroke}%
\pgfsetdash{}{0pt}%
\pgfpathmoveto{\pgfqpoint{4.311459in}{1.497203in}}%
\pgfpathlineto{\pgfqpoint{4.325857in}{1.489713in}}%
\pgfpathlineto{\pgfqpoint{4.340259in}{1.482248in}}%
\pgfpathlineto{\pgfqpoint{4.354667in}{1.474806in}}%
\pgfpathlineto{\pgfqpoint{4.369080in}{1.467387in}}%
\pgfpathlineto{\pgfqpoint{4.360982in}{1.474305in}}%
\pgfpathlineto{\pgfqpoint{4.352877in}{1.481796in}}%
\pgfpathlineto{\pgfqpoint{4.344763in}{1.489872in}}%
\pgfpathlineto{\pgfqpoint{4.336640in}{1.498546in}}%
\pgfpathlineto{\pgfqpoint{4.322194in}{1.506396in}}%
\pgfpathlineto{\pgfqpoint{4.307754in}{1.514271in}}%
\pgfpathlineto{\pgfqpoint{4.293318in}{1.522169in}}%
\pgfpathlineto{\pgfqpoint{4.278887in}{1.530090in}}%
\pgfpathlineto{\pgfqpoint{4.287044in}{1.520978in}}%
\pgfpathlineto{\pgfqpoint{4.295191in}{1.512468in}}%
\pgfpathlineto{\pgfqpoint{4.303330in}{1.504547in}}%
\pgfpathlineto{\pgfqpoint{4.311459in}{1.497203in}}%
\pgfpathclose%
\pgfusepath{fill}%
\end{pgfscope}%
\begin{pgfscope}%
\pgfpathrectangle{\pgfqpoint{1.150000in}{0.150000in}}{\pgfqpoint{5.700000in}{5.700000in}}%
\pgfusepath{clip}%
\pgfsetbuttcap%
\pgfsetroundjoin%
\definecolor{currentfill}{rgb}{0.352360,0.783011,0.392636}%
\pgfsetfillcolor{currentfill}%
\pgfsetfillopacity{0.700000}%
\pgfsetlinewidth{0.000000pt}%
\definecolor{currentstroke}{rgb}{0.000000,0.000000,0.000000}%
\pgfsetstrokecolor{currentstroke}%
\pgfsetdash{}{0pt}%
\pgfpathmoveto{\pgfqpoint{2.400136in}{3.057753in}}%
\pgfpathlineto{\pgfqpoint{2.414370in}{3.044733in}}%
\pgfpathlineto{\pgfqpoint{2.428604in}{3.031755in}}%
\pgfpathlineto{\pgfqpoint{2.442839in}{3.018818in}}%
\pgfpathlineto{\pgfqpoint{2.457074in}{3.005921in}}%
\pgfpathlineto{\pgfqpoint{2.446890in}{3.039402in}}%
\pgfpathlineto{\pgfqpoint{2.436654in}{3.073844in}}%
\pgfpathlineto{\pgfqpoint{2.426367in}{3.109263in}}%
\pgfpathlineto{\pgfqpoint{2.416026in}{3.145677in}}%
\pgfpathlineto{\pgfqpoint{2.401711in}{3.159114in}}%
\pgfpathlineto{\pgfqpoint{2.387395in}{3.172591in}}%
\pgfpathlineto{\pgfqpoint{2.373080in}{3.186109in}}%
\pgfpathlineto{\pgfqpoint{2.358765in}{3.199670in}}%
\pgfpathlineto{\pgfqpoint{2.369188in}{3.162706in}}%
\pgfpathlineto{\pgfqpoint{2.379557in}{3.126744in}}%
\pgfpathlineto{\pgfqpoint{2.389873in}{3.091766in}}%
\pgfpathlineto{\pgfqpoint{2.400136in}{3.057753in}}%
\pgfpathclose%
\pgfusepath{fill}%
\end{pgfscope}%
\begin{pgfscope}%
\pgfpathrectangle{\pgfqpoint{1.150000in}{0.150000in}}{\pgfqpoint{5.700000in}{5.700000in}}%
\pgfusepath{clip}%
\pgfsetbuttcap%
\pgfsetroundjoin%
\definecolor{currentfill}{rgb}{0.282910,0.105393,0.426902}%
\pgfsetfillcolor{currentfill}%
\pgfsetfillopacity{0.700000}%
\pgfsetlinewidth{0.000000pt}%
\definecolor{currentstroke}{rgb}{0.000000,0.000000,0.000000}%
\pgfsetstrokecolor{currentstroke}%
\pgfsetdash{}{0pt}%
\pgfpathmoveto{\pgfqpoint{4.779887in}{1.279792in}}%
\pgfpathlineto{\pgfqpoint{4.794392in}{1.273772in}}%
\pgfpathlineto{\pgfqpoint{4.808904in}{1.267774in}}%
\pgfpathlineto{\pgfqpoint{4.823422in}{1.261799in}}%
\pgfpathlineto{\pgfqpoint{4.837946in}{1.255847in}}%
\pgfpathlineto{\pgfqpoint{4.830069in}{1.255845in}}%
\pgfpathlineto{\pgfqpoint{4.822190in}{1.256297in}}%
\pgfpathlineto{\pgfqpoint{4.814309in}{1.257213in}}%
\pgfpathlineto{\pgfqpoint{4.806425in}{1.258604in}}%
\pgfpathlineto{\pgfqpoint{4.791880in}{1.264956in}}%
\pgfpathlineto{\pgfqpoint{4.777340in}{1.271331in}}%
\pgfpathlineto{\pgfqpoint{4.762807in}{1.277730in}}%
\pgfpathlineto{\pgfqpoint{4.748280in}{1.284151in}}%
\pgfpathlineto{\pgfqpoint{4.756186in}{1.282353in}}%
\pgfpathlineto{\pgfqpoint{4.764089in}{1.281035in}}%
\pgfpathlineto{\pgfqpoint{4.771990in}{1.280185in}}%
\pgfpathlineto{\pgfqpoint{4.779887in}{1.279792in}}%
\pgfpathclose%
\pgfusepath{fill}%
\end{pgfscope}%
\begin{pgfscope}%
\pgfpathrectangle{\pgfqpoint{1.150000in}{0.150000in}}{\pgfqpoint{5.700000in}{5.700000in}}%
\pgfusepath{clip}%
\pgfsetbuttcap%
\pgfsetroundjoin%
\definecolor{currentfill}{rgb}{0.137770,0.537492,0.554906}%
\pgfsetfillcolor{currentfill}%
\pgfsetfillopacity{0.700000}%
\pgfsetlinewidth{0.000000pt}%
\definecolor{currentstroke}{rgb}{0.000000,0.000000,0.000000}%
\pgfsetstrokecolor{currentstroke}%
\pgfsetdash{}{0pt}%
\pgfpathmoveto{\pgfqpoint{3.178852in}{2.328886in}}%
\pgfpathlineto{\pgfqpoint{3.193103in}{2.318167in}}%
\pgfpathlineto{\pgfqpoint{3.207356in}{2.307477in}}%
\pgfpathlineto{\pgfqpoint{3.221612in}{2.296817in}}%
\pgfpathlineto{\pgfqpoint{3.235870in}{2.286185in}}%
\pgfpathlineto{\pgfqpoint{3.226741in}{2.309529in}}%
\pgfpathlineto{\pgfqpoint{3.217580in}{2.333694in}}%
\pgfpathlineto{\pgfqpoint{3.208389in}{2.358695in}}%
\pgfpathlineto{\pgfqpoint{3.199165in}{2.384549in}}%
\pgfpathlineto{\pgfqpoint{3.184844in}{2.395679in}}%
\pgfpathlineto{\pgfqpoint{3.170527in}{2.406838in}}%
\pgfpathlineto{\pgfqpoint{3.156211in}{2.418026in}}%
\pgfpathlineto{\pgfqpoint{3.141898in}{2.429244in}}%
\pgfpathlineto{\pgfqpoint{3.151186in}{2.402884in}}%
\pgfpathlineto{\pgfqpoint{3.160440in}{2.377381in}}%
\pgfpathlineto{\pgfqpoint{3.169662in}{2.352721in}}%
\pgfpathlineto{\pgfqpoint{3.178852in}{2.328886in}}%
\pgfpathclose%
\pgfusepath{fill}%
\end{pgfscope}%
\begin{pgfscope}%
\pgfpathrectangle{\pgfqpoint{1.150000in}{0.150000in}}{\pgfqpoint{5.700000in}{5.700000in}}%
\pgfusepath{clip}%
\pgfsetbuttcap%
\pgfsetroundjoin%
\definecolor{currentfill}{rgb}{0.282290,0.145912,0.461510}%
\pgfsetfillcolor{currentfill}%
\pgfsetfillopacity{0.700000}%
\pgfsetlinewidth{0.000000pt}%
\definecolor{currentstroke}{rgb}{0.000000,0.000000,0.000000}%
\pgfsetstrokecolor{currentstroke}%
\pgfsetdash{}{0pt}%
\pgfpathmoveto{\pgfqpoint{4.574420in}{1.363003in}}%
\pgfpathlineto{\pgfqpoint{4.588876in}{1.356305in}}%
\pgfpathlineto{\pgfqpoint{4.603338in}{1.349630in}}%
\pgfpathlineto{\pgfqpoint{4.617805in}{1.342978in}}%
\pgfpathlineto{\pgfqpoint{4.632279in}{1.336349in}}%
\pgfpathlineto{\pgfqpoint{4.624320in}{1.339456in}}%
\pgfpathlineto{\pgfqpoint{4.616357in}{1.343073in}}%
\pgfpathlineto{\pgfqpoint{4.608390in}{1.347211in}}%
\pgfpathlineto{\pgfqpoint{4.600417in}{1.351881in}}%
\pgfpathlineto{\pgfqpoint{4.585917in}{1.358926in}}%
\pgfpathlineto{\pgfqpoint{4.571423in}{1.365993in}}%
\pgfpathlineto{\pgfqpoint{4.556934in}{1.373084in}}%
\pgfpathlineto{\pgfqpoint{4.542451in}{1.380198in}}%
\pgfpathlineto{\pgfqpoint{4.550452in}{1.375106in}}%
\pgfpathlineto{\pgfqpoint{4.558446in}{1.370550in}}%
\pgfpathlineto{\pgfqpoint{4.566436in}{1.366520in}}%
\pgfpathlineto{\pgfqpoint{4.574420in}{1.363003in}}%
\pgfpathclose%
\pgfusepath{fill}%
\end{pgfscope}%
\begin{pgfscope}%
\pgfpathrectangle{\pgfqpoint{1.150000in}{0.150000in}}{\pgfqpoint{5.700000in}{5.700000in}}%
\pgfusepath{clip}%
\pgfsetbuttcap%
\pgfsetroundjoin%
\definecolor{currentfill}{rgb}{0.197636,0.391528,0.554969}%
\pgfsetfillcolor{currentfill}%
\pgfsetfillopacity{0.700000}%
\pgfsetlinewidth{0.000000pt}%
\definecolor{currentstroke}{rgb}{0.000000,0.000000,0.000000}%
\pgfsetstrokecolor{currentstroke}%
\pgfsetdash{}{0pt}%
\pgfpathmoveto{\pgfqpoint{3.671043in}{1.927689in}}%
\pgfpathlineto{\pgfqpoint{3.685341in}{1.918327in}}%
\pgfpathlineto{\pgfqpoint{3.699644in}{1.908991in}}%
\pgfpathlineto{\pgfqpoint{3.713950in}{1.899680in}}%
\pgfpathlineto{\pgfqpoint{3.728260in}{1.890396in}}%
\pgfpathlineto{\pgfqpoint{3.719659in}{1.906871in}}%
\pgfpathlineto{\pgfqpoint{3.711039in}{1.924068in}}%
\pgfpathlineto{\pgfqpoint{3.702398in}{1.942002in}}%
\pgfpathlineto{\pgfqpoint{3.693737in}{1.960686in}}%
\pgfpathlineto{\pgfqpoint{3.679377in}{1.970442in}}%
\pgfpathlineto{\pgfqpoint{3.665022in}{1.980224in}}%
\pgfpathlineto{\pgfqpoint{3.650669in}{1.990033in}}%
\pgfpathlineto{\pgfqpoint{3.636320in}{1.999867in}}%
\pgfpathlineto{\pgfqpoint{3.645033in}{1.980704in}}%
\pgfpathlineto{\pgfqpoint{3.653724in}{1.962296in}}%
\pgfpathlineto{\pgfqpoint{3.662394in}{1.944629in}}%
\pgfpathlineto{\pgfqpoint{3.671043in}{1.927689in}}%
\pgfpathclose%
\pgfusepath{fill}%
\end{pgfscope}%
\begin{pgfscope}%
\pgfpathrectangle{\pgfqpoint{1.150000in}{0.150000in}}{\pgfqpoint{5.700000in}{5.700000in}}%
\pgfusepath{clip}%
\pgfsetbuttcap%
\pgfsetroundjoin%
\definecolor{currentfill}{rgb}{0.311925,0.767822,0.415586}%
\pgfsetfillcolor{currentfill}%
\pgfsetfillopacity{0.700000}%
\pgfsetlinewidth{0.000000pt}%
\definecolor{currentstroke}{rgb}{0.000000,0.000000,0.000000}%
\pgfsetstrokecolor{currentstroke}%
\pgfsetdash{}{0pt}%
\pgfpathmoveto{\pgfqpoint{2.457074in}{3.005921in}}%
\pgfpathlineto{\pgfqpoint{2.471310in}{2.993064in}}%
\pgfpathlineto{\pgfqpoint{2.485547in}{2.980247in}}%
\pgfpathlineto{\pgfqpoint{2.499785in}{2.967469in}}%
\pgfpathlineto{\pgfqpoint{2.514023in}{2.954731in}}%
\pgfpathlineto{\pgfqpoint{2.503916in}{2.987683in}}%
\pgfpathlineto{\pgfqpoint{2.493760in}{3.021590in}}%
\pgfpathlineto{\pgfqpoint{2.483553in}{3.056467in}}%
\pgfpathlineto{\pgfqpoint{2.473295in}{3.092335in}}%
\pgfpathlineto{\pgfqpoint{2.458977in}{3.105610in}}%
\pgfpathlineto{\pgfqpoint{2.444660in}{3.118926in}}%
\pgfpathlineto{\pgfqpoint{2.430343in}{3.132281in}}%
\pgfpathlineto{\pgfqpoint{2.416026in}{3.145677in}}%
\pgfpathlineto{\pgfqpoint{2.426367in}{3.109263in}}%
\pgfpathlineto{\pgfqpoint{2.436654in}{3.073844in}}%
\pgfpathlineto{\pgfqpoint{2.446890in}{3.039402in}}%
\pgfpathlineto{\pgfqpoint{2.457074in}{3.005921in}}%
\pgfpathclose%
\pgfusepath{fill}%
\end{pgfscope}%
\begin{pgfscope}%
\pgfpathrectangle{\pgfqpoint{1.150000in}{0.150000in}}{\pgfqpoint{5.700000in}{5.700000in}}%
\pgfusepath{clip}%
\pgfsetbuttcap%
\pgfsetroundjoin%
\definecolor{currentfill}{rgb}{0.244972,0.287675,0.537260}%
\pgfsetfillcolor{currentfill}%
\pgfsetfillopacity{0.700000}%
\pgfsetlinewidth{0.000000pt}%
\definecolor{currentstroke}{rgb}{0.000000,0.000000,0.000000}%
\pgfsetstrokecolor{currentstroke}%
\pgfsetdash{}{0pt}%
\pgfpathmoveto{\pgfqpoint{4.048648in}{1.660097in}}%
\pgfpathlineto{\pgfqpoint{4.063002in}{1.651790in}}%
\pgfpathlineto{\pgfqpoint{4.077362in}{1.643508in}}%
\pgfpathlineto{\pgfqpoint{4.091726in}{1.635250in}}%
\pgfpathlineto{\pgfqpoint{4.106094in}{1.627016in}}%
\pgfpathlineto{\pgfqpoint{4.097816in}{1.638072in}}%
\pgfpathlineto{\pgfqpoint{4.089525in}{1.649769in}}%
\pgfpathlineto{\pgfqpoint{4.081221in}{1.662120in}}%
\pgfpathlineto{\pgfqpoint{4.072904in}{1.675138in}}%
\pgfpathlineto{\pgfqpoint{4.058495in}{1.683822in}}%
\pgfpathlineto{\pgfqpoint{4.044091in}{1.692530in}}%
\pgfpathlineto{\pgfqpoint{4.029691in}{1.701263in}}%
\pgfpathlineto{\pgfqpoint{4.015296in}{1.710021in}}%
\pgfpathlineto{\pgfqpoint{4.023655in}{1.696546in}}%
\pgfpathlineto{\pgfqpoint{4.031999in}{1.683743in}}%
\pgfpathlineto{\pgfqpoint{4.040330in}{1.671597in}}%
\pgfpathlineto{\pgfqpoint{4.048648in}{1.660097in}}%
\pgfpathclose%
\pgfusepath{fill}%
\end{pgfscope}%
\begin{pgfscope}%
\pgfpathrectangle{\pgfqpoint{1.150000in}{0.150000in}}{\pgfqpoint{5.700000in}{5.700000in}}%
\pgfusepath{clip}%
\pgfsetbuttcap%
\pgfsetroundjoin%
\definecolor{currentfill}{rgb}{0.141935,0.526453,0.555991}%
\pgfsetfillcolor{currentfill}%
\pgfsetfillopacity{0.700000}%
\pgfsetlinewidth{0.000000pt}%
\definecolor{currentstroke}{rgb}{0.000000,0.000000,0.000000}%
\pgfsetstrokecolor{currentstroke}%
\pgfsetdash{}{0pt}%
\pgfpathmoveto{\pgfqpoint{3.235870in}{2.286185in}}%
\pgfpathlineto{\pgfqpoint{3.250132in}{2.275583in}}%
\pgfpathlineto{\pgfqpoint{3.264396in}{2.265009in}}%
\pgfpathlineto{\pgfqpoint{3.278663in}{2.254465in}}%
\pgfpathlineto{\pgfqpoint{3.292932in}{2.243948in}}%
\pgfpathlineto{\pgfqpoint{3.283862in}{2.266802in}}%
\pgfpathlineto{\pgfqpoint{3.274762in}{2.290472in}}%
\pgfpathlineto{\pgfqpoint{3.265632in}{2.314973in}}%
\pgfpathlineto{\pgfqpoint{3.256471in}{2.340322in}}%
\pgfpathlineto{\pgfqpoint{3.242141in}{2.351335in}}%
\pgfpathlineto{\pgfqpoint{3.227813in}{2.362377in}}%
\pgfpathlineto{\pgfqpoint{3.213488in}{2.373449in}}%
\pgfpathlineto{\pgfqpoint{3.199165in}{2.384549in}}%
\pgfpathlineto{\pgfqpoint{3.208389in}{2.358695in}}%
\pgfpathlineto{\pgfqpoint{3.217580in}{2.333694in}}%
\pgfpathlineto{\pgfqpoint{3.226741in}{2.309529in}}%
\pgfpathlineto{\pgfqpoint{3.235870in}{2.286185in}}%
\pgfpathclose%
\pgfusepath{fill}%
\end{pgfscope}%
\begin{pgfscope}%
\pgfpathrectangle{\pgfqpoint{1.150000in}{0.150000in}}{\pgfqpoint{5.700000in}{5.700000in}}%
\pgfusepath{clip}%
\pgfsetbuttcap%
\pgfsetroundjoin%
\definecolor{currentfill}{rgb}{0.281477,0.755203,0.432552}%
\pgfsetfillcolor{currentfill}%
\pgfsetfillopacity{0.700000}%
\pgfsetlinewidth{0.000000pt}%
\definecolor{currentstroke}{rgb}{0.000000,0.000000,0.000000}%
\pgfsetstrokecolor{currentstroke}%
\pgfsetdash{}{0pt}%
\pgfpathmoveto{\pgfqpoint{2.514023in}{2.954731in}}%
\pgfpathlineto{\pgfqpoint{2.528262in}{2.942031in}}%
\pgfpathlineto{\pgfqpoint{2.542503in}{2.929370in}}%
\pgfpathlineto{\pgfqpoint{2.556744in}{2.916748in}}%
\pgfpathlineto{\pgfqpoint{2.570986in}{2.904163in}}%
\pgfpathlineto{\pgfqpoint{2.560956in}{2.936589in}}%
\pgfpathlineto{\pgfqpoint{2.550878in}{2.969962in}}%
\pgfpathlineto{\pgfqpoint{2.540750in}{3.004300in}}%
\pgfpathlineto{\pgfqpoint{2.530573in}{3.039622in}}%
\pgfpathlineto{\pgfqpoint{2.516252in}{3.052742in}}%
\pgfpathlineto{\pgfqpoint{2.501933in}{3.065901in}}%
\pgfpathlineto{\pgfqpoint{2.487614in}{3.079098in}}%
\pgfpathlineto{\pgfqpoint{2.473295in}{3.092335in}}%
\pgfpathlineto{\pgfqpoint{2.483553in}{3.056467in}}%
\pgfpathlineto{\pgfqpoint{2.493760in}{3.021590in}}%
\pgfpathlineto{\pgfqpoint{2.503916in}{2.987683in}}%
\pgfpathlineto{\pgfqpoint{2.514023in}{2.954731in}}%
\pgfpathclose%
\pgfusepath{fill}%
\end{pgfscope}%
\begin{pgfscope}%
\pgfpathrectangle{\pgfqpoint{1.150000in}{0.150000in}}{\pgfqpoint{5.700000in}{5.700000in}}%
\pgfusepath{clip}%
\pgfsetbuttcap%
\pgfsetroundjoin%
\definecolor{currentfill}{rgb}{0.273006,0.204520,0.501721}%
\pgfsetfillcolor{currentfill}%
\pgfsetfillopacity{0.700000}%
\pgfsetlinewidth{0.000000pt}%
\definecolor{currentstroke}{rgb}{0.000000,0.000000,0.000000}%
\pgfsetstrokecolor{currentstroke}%
\pgfsetdash{}{0pt}%
\pgfpathmoveto{\pgfqpoint{4.369080in}{1.467387in}}%
\pgfpathlineto{\pgfqpoint{4.383498in}{1.459992in}}%
\pgfpathlineto{\pgfqpoint{4.397921in}{1.452621in}}%
\pgfpathlineto{\pgfqpoint{4.412350in}{1.445273in}}%
\pgfpathlineto{\pgfqpoint{4.426784in}{1.437949in}}%
\pgfpathlineto{\pgfqpoint{4.418718in}{1.444441in}}%
\pgfpathlineto{\pgfqpoint{4.410644in}{1.451502in}}%
\pgfpathlineto{\pgfqpoint{4.402563in}{1.459144in}}%
\pgfpathlineto{\pgfqpoint{4.394474in}{1.467380in}}%
\pgfpathlineto{\pgfqpoint{4.380008in}{1.475136in}}%
\pgfpathlineto{\pgfqpoint{4.365547in}{1.482916in}}%
\pgfpathlineto{\pgfqpoint{4.351091in}{1.490719in}}%
\pgfpathlineto{\pgfqpoint{4.336640in}{1.498546in}}%
\pgfpathlineto{\pgfqpoint{4.344763in}{1.489872in}}%
\pgfpathlineto{\pgfqpoint{4.352877in}{1.481796in}}%
\pgfpathlineto{\pgfqpoint{4.360982in}{1.474305in}}%
\pgfpathlineto{\pgfqpoint{4.369080in}{1.467387in}}%
\pgfpathclose%
\pgfusepath{fill}%
\end{pgfscope}%
\begin{pgfscope}%
\pgfpathrectangle{\pgfqpoint{1.150000in}{0.150000in}}{\pgfqpoint{5.700000in}{5.700000in}}%
\pgfusepath{clip}%
\pgfsetbuttcap%
\pgfsetroundjoin%
\definecolor{currentfill}{rgb}{0.201239,0.383670,0.554294}%
\pgfsetfillcolor{currentfill}%
\pgfsetfillopacity{0.700000}%
\pgfsetlinewidth{0.000000pt}%
\definecolor{currentstroke}{rgb}{0.000000,0.000000,0.000000}%
\pgfsetstrokecolor{currentstroke}%
\pgfsetdash{}{0pt}%
\pgfpathmoveto{\pgfqpoint{3.728260in}{1.890396in}}%
\pgfpathlineto{\pgfqpoint{3.742574in}{1.881137in}}%
\pgfpathlineto{\pgfqpoint{3.756892in}{1.871903in}}%
\pgfpathlineto{\pgfqpoint{3.771213in}{1.862696in}}%
\pgfpathlineto{\pgfqpoint{3.785539in}{1.853514in}}%
\pgfpathlineto{\pgfqpoint{3.776985in}{1.869525in}}%
\pgfpathlineto{\pgfqpoint{3.768413in}{1.886253in}}%
\pgfpathlineto{\pgfqpoint{3.759822in}{1.903713in}}%
\pgfpathlineto{\pgfqpoint{3.751211in}{1.921919in}}%
\pgfpathlineto{\pgfqpoint{3.736837in}{1.931572in}}%
\pgfpathlineto{\pgfqpoint{3.722467in}{1.941251in}}%
\pgfpathlineto{\pgfqpoint{3.708100in}{1.950956in}}%
\pgfpathlineto{\pgfqpoint{3.693737in}{1.960686in}}%
\pgfpathlineto{\pgfqpoint{3.702398in}{1.942002in}}%
\pgfpathlineto{\pgfqpoint{3.711039in}{1.924068in}}%
\pgfpathlineto{\pgfqpoint{3.719659in}{1.906871in}}%
\pgfpathlineto{\pgfqpoint{3.728260in}{1.890396in}}%
\pgfpathclose%
\pgfusepath{fill}%
\end{pgfscope}%
\begin{pgfscope}%
\pgfpathrectangle{\pgfqpoint{1.150000in}{0.150000in}}{\pgfqpoint{5.700000in}{5.700000in}}%
\pgfusepath{clip}%
\pgfsetbuttcap%
\pgfsetroundjoin%
\definecolor{currentfill}{rgb}{0.246070,0.738910,0.452024}%
\pgfsetfillcolor{currentfill}%
\pgfsetfillopacity{0.700000}%
\pgfsetlinewidth{0.000000pt}%
\definecolor{currentstroke}{rgb}{0.000000,0.000000,0.000000}%
\pgfsetstrokecolor{currentstroke}%
\pgfsetdash{}{0pt}%
\pgfpathmoveto{\pgfqpoint{2.570986in}{2.904163in}}%
\pgfpathlineto{\pgfqpoint{2.585229in}{2.891617in}}%
\pgfpathlineto{\pgfqpoint{2.599473in}{2.879107in}}%
\pgfpathlineto{\pgfqpoint{2.613719in}{2.866635in}}%
\pgfpathlineto{\pgfqpoint{2.627965in}{2.854200in}}%
\pgfpathlineto{\pgfqpoint{2.618011in}{2.886100in}}%
\pgfpathlineto{\pgfqpoint{2.608010in}{2.918941in}}%
\pgfpathlineto{\pgfqpoint{2.597961in}{2.952742in}}%
\pgfpathlineto{\pgfqpoint{2.587863in}{2.987520in}}%
\pgfpathlineto{\pgfqpoint{2.573539in}{3.000489in}}%
\pgfpathlineto{\pgfqpoint{2.559216in}{3.013496in}}%
\pgfpathlineto{\pgfqpoint{2.544894in}{3.026540in}}%
\pgfpathlineto{\pgfqpoint{2.530573in}{3.039622in}}%
\pgfpathlineto{\pgfqpoint{2.540750in}{3.004300in}}%
\pgfpathlineto{\pgfqpoint{2.550878in}{2.969962in}}%
\pgfpathlineto{\pgfqpoint{2.560956in}{2.936589in}}%
\pgfpathlineto{\pgfqpoint{2.570986in}{2.904163in}}%
\pgfpathclose%
\pgfusepath{fill}%
\end{pgfscope}%
\begin{pgfscope}%
\pgfpathrectangle{\pgfqpoint{1.150000in}{0.150000in}}{\pgfqpoint{5.700000in}{5.700000in}}%
\pgfusepath{clip}%
\pgfsetbuttcap%
\pgfsetroundjoin%
\definecolor{currentfill}{rgb}{0.282656,0.100196,0.422160}%
\pgfsetfillcolor{currentfill}%
\pgfsetfillopacity{0.700000}%
\pgfsetlinewidth{0.000000pt}%
\definecolor{currentstroke}{rgb}{0.000000,0.000000,0.000000}%
\pgfsetstrokecolor{currentstroke}%
\pgfsetdash{}{0pt}%
\pgfpathmoveto{\pgfqpoint{4.837946in}{1.255847in}}%
\pgfpathlineto{\pgfqpoint{4.852476in}{1.249918in}}%
\pgfpathlineto{\pgfqpoint{4.867013in}{1.244012in}}%
\pgfpathlineto{\pgfqpoint{4.881557in}{1.238129in}}%
\pgfpathlineto{\pgfqpoint{4.896107in}{1.232269in}}%
\pgfpathlineto{\pgfqpoint{4.888250in}{1.231872in}}%
\pgfpathlineto{\pgfqpoint{4.880391in}{1.231925in}}%
\pgfpathlineto{\pgfqpoint{4.872531in}{1.232439in}}%
\pgfpathlineto{\pgfqpoint{4.864669in}{1.233424in}}%
\pgfpathlineto{\pgfqpoint{4.850099in}{1.239685in}}%
\pgfpathlineto{\pgfqpoint{4.835535in}{1.245968in}}%
\pgfpathlineto{\pgfqpoint{4.820977in}{1.252275in}}%
\pgfpathlineto{\pgfqpoint{4.806425in}{1.258604in}}%
\pgfpathlineto{\pgfqpoint{4.814309in}{1.257213in}}%
\pgfpathlineto{\pgfqpoint{4.822190in}{1.256297in}}%
\pgfpathlineto{\pgfqpoint{4.830069in}{1.255845in}}%
\pgfpathlineto{\pgfqpoint{4.837946in}{1.255847in}}%
\pgfpathclose%
\pgfusepath{fill}%
\end{pgfscope}%
\begin{pgfscope}%
\pgfpathrectangle{\pgfqpoint{1.150000in}{0.150000in}}{\pgfqpoint{5.700000in}{5.700000in}}%
\pgfusepath{clip}%
\pgfsetbuttcap%
\pgfsetroundjoin%
\definecolor{currentfill}{rgb}{0.282623,0.140926,0.457517}%
\pgfsetfillcolor{currentfill}%
\pgfsetfillopacity{0.700000}%
\pgfsetlinewidth{0.000000pt}%
\definecolor{currentstroke}{rgb}{0.000000,0.000000,0.000000}%
\pgfsetstrokecolor{currentstroke}%
\pgfsetdash{}{0pt}%
\pgfpathmoveto{\pgfqpoint{4.632279in}{1.336349in}}%
\pgfpathlineto{\pgfqpoint{4.646758in}{1.329743in}}%
\pgfpathlineto{\pgfqpoint{4.661243in}{1.323161in}}%
\pgfpathlineto{\pgfqpoint{4.675734in}{1.316602in}}%
\pgfpathlineto{\pgfqpoint{4.690232in}{1.310065in}}%
\pgfpathlineto{\pgfqpoint{4.682298in}{1.312763in}}%
\pgfpathlineto{\pgfqpoint{4.674361in}{1.315966in}}%
\pgfpathlineto{\pgfqpoint{4.666419in}{1.319686in}}%
\pgfpathlineto{\pgfqpoint{4.658474in}{1.323935in}}%
\pgfpathlineto{\pgfqpoint{4.643951in}{1.330887in}}%
\pgfpathlineto{\pgfqpoint{4.629434in}{1.337862in}}%
\pgfpathlineto{\pgfqpoint{4.614923in}{1.344860in}}%
\pgfpathlineto{\pgfqpoint{4.600417in}{1.351881in}}%
\pgfpathlineto{\pgfqpoint{4.608390in}{1.347211in}}%
\pgfpathlineto{\pgfqpoint{4.616357in}{1.343073in}}%
\pgfpathlineto{\pgfqpoint{4.624320in}{1.339456in}}%
\pgfpathlineto{\pgfqpoint{4.632279in}{1.336349in}}%
\pgfpathclose%
\pgfusepath{fill}%
\end{pgfscope}%
\begin{pgfscope}%
\pgfpathrectangle{\pgfqpoint{1.150000in}{0.150000in}}{\pgfqpoint{5.700000in}{5.700000in}}%
\pgfusepath{clip}%
\pgfsetbuttcap%
\pgfsetroundjoin%
\definecolor{currentfill}{rgb}{0.248629,0.278775,0.534556}%
\pgfsetfillcolor{currentfill}%
\pgfsetfillopacity{0.700000}%
\pgfsetlinewidth{0.000000pt}%
\definecolor{currentstroke}{rgb}{0.000000,0.000000,0.000000}%
\pgfsetstrokecolor{currentstroke}%
\pgfsetdash{}{0pt}%
\pgfpathmoveto{\pgfqpoint{4.106094in}{1.627016in}}%
\pgfpathlineto{\pgfqpoint{4.120467in}{1.618806in}}%
\pgfpathlineto{\pgfqpoint{4.134845in}{1.610621in}}%
\pgfpathlineto{\pgfqpoint{4.149228in}{1.602460in}}%
\pgfpathlineto{\pgfqpoint{4.163615in}{1.594323in}}%
\pgfpathlineto{\pgfqpoint{4.155375in}{1.604936in}}%
\pgfpathlineto{\pgfqpoint{4.147123in}{1.616185in}}%
\pgfpathlineto{\pgfqpoint{4.138859in}{1.628083in}}%
\pgfpathlineto{\pgfqpoint{4.130583in}{1.640645in}}%
\pgfpathlineto{\pgfqpoint{4.116156in}{1.649231in}}%
\pgfpathlineto{\pgfqpoint{4.101734in}{1.657843in}}%
\pgfpathlineto{\pgfqpoint{4.087317in}{1.666478in}}%
\pgfpathlineto{\pgfqpoint{4.072904in}{1.675138in}}%
\pgfpathlineto{\pgfqpoint{4.081221in}{1.662120in}}%
\pgfpathlineto{\pgfqpoint{4.089525in}{1.649769in}}%
\pgfpathlineto{\pgfqpoint{4.097816in}{1.638072in}}%
\pgfpathlineto{\pgfqpoint{4.106094in}{1.627016in}}%
\pgfpathclose%
\pgfusepath{fill}%
\end{pgfscope}%
\begin{pgfscope}%
\pgfpathrectangle{\pgfqpoint{1.150000in}{0.150000in}}{\pgfqpoint{5.700000in}{5.700000in}}%
\pgfusepath{clip}%
\pgfsetbuttcap%
\pgfsetroundjoin%
\definecolor{currentfill}{rgb}{0.147607,0.511733,0.557049}%
\pgfsetfillcolor{currentfill}%
\pgfsetfillopacity{0.700000}%
\pgfsetlinewidth{0.000000pt}%
\definecolor{currentstroke}{rgb}{0.000000,0.000000,0.000000}%
\pgfsetstrokecolor{currentstroke}%
\pgfsetdash{}{0pt}%
\pgfpathmoveto{\pgfqpoint{3.292932in}{2.243948in}}%
\pgfpathlineto{\pgfqpoint{3.307205in}{2.233461in}}%
\pgfpathlineto{\pgfqpoint{3.321480in}{2.223001in}}%
\pgfpathlineto{\pgfqpoint{3.335759in}{2.212570in}}%
\pgfpathlineto{\pgfqpoint{3.350040in}{2.202168in}}%
\pgfpathlineto{\pgfqpoint{3.341028in}{2.224533in}}%
\pgfpathlineto{\pgfqpoint{3.331987in}{2.247709in}}%
\pgfpathlineto{\pgfqpoint{3.322918in}{2.271710in}}%
\pgfpathlineto{\pgfqpoint{3.313819in}{2.296555in}}%
\pgfpathlineto{\pgfqpoint{3.299478in}{2.307454in}}%
\pgfpathlineto{\pgfqpoint{3.285140in}{2.318381in}}%
\pgfpathlineto{\pgfqpoint{3.270804in}{2.329337in}}%
\pgfpathlineto{\pgfqpoint{3.256471in}{2.340322in}}%
\pgfpathlineto{\pgfqpoint{3.265632in}{2.314973in}}%
\pgfpathlineto{\pgfqpoint{3.274762in}{2.290472in}}%
\pgfpathlineto{\pgfqpoint{3.283862in}{2.266802in}}%
\pgfpathlineto{\pgfqpoint{3.292932in}{2.243948in}}%
\pgfpathclose%
\pgfusepath{fill}%
\end{pgfscope}%
\begin{pgfscope}%
\pgfpathrectangle{\pgfqpoint{1.150000in}{0.150000in}}{\pgfqpoint{5.700000in}{5.700000in}}%
\pgfusepath{clip}%
\pgfsetbuttcap%
\pgfsetroundjoin%
\definecolor{currentfill}{rgb}{0.220124,0.725509,0.466226}%
\pgfsetfillcolor{currentfill}%
\pgfsetfillopacity{0.700000}%
\pgfsetlinewidth{0.000000pt}%
\definecolor{currentstroke}{rgb}{0.000000,0.000000,0.000000}%
\pgfsetstrokecolor{currentstroke}%
\pgfsetdash{}{0pt}%
\pgfpathmoveto{\pgfqpoint{2.627965in}{2.854200in}}%
\pgfpathlineto{\pgfqpoint{2.642213in}{2.841802in}}%
\pgfpathlineto{\pgfqpoint{2.656462in}{2.829440in}}%
\pgfpathlineto{\pgfqpoint{2.670712in}{2.817114in}}%
\pgfpathlineto{\pgfqpoint{2.684963in}{2.804824in}}%
\pgfpathlineto{\pgfqpoint{2.675083in}{2.836200in}}%
\pgfpathlineto{\pgfqpoint{2.665158in}{2.868511in}}%
\pgfpathlineto{\pgfqpoint{2.655186in}{2.901776in}}%
\pgfpathlineto{\pgfqpoint{2.645167in}{2.936013in}}%
\pgfpathlineto{\pgfqpoint{2.630840in}{2.948835in}}%
\pgfpathlineto{\pgfqpoint{2.616513in}{2.961693in}}%
\pgfpathlineto{\pgfqpoint{2.602188in}{2.974588in}}%
\pgfpathlineto{\pgfqpoint{2.587863in}{2.987520in}}%
\pgfpathlineto{\pgfqpoint{2.597961in}{2.952742in}}%
\pgfpathlineto{\pgfqpoint{2.608010in}{2.918941in}}%
\pgfpathlineto{\pgfqpoint{2.618011in}{2.886100in}}%
\pgfpathlineto{\pgfqpoint{2.627965in}{2.854200in}}%
\pgfpathclose%
\pgfusepath{fill}%
\end{pgfscope}%
\begin{pgfscope}%
\pgfpathrectangle{\pgfqpoint{1.150000in}{0.150000in}}{\pgfqpoint{5.700000in}{5.700000in}}%
\pgfusepath{clip}%
\pgfsetbuttcap%
\pgfsetroundjoin%
\definecolor{currentfill}{rgb}{0.206756,0.371758,0.553117}%
\pgfsetfillcolor{currentfill}%
\pgfsetfillopacity{0.700000}%
\pgfsetlinewidth{0.000000pt}%
\definecolor{currentstroke}{rgb}{0.000000,0.000000,0.000000}%
\pgfsetstrokecolor{currentstroke}%
\pgfsetdash{}{0pt}%
\pgfpathmoveto{\pgfqpoint{3.785539in}{1.853514in}}%
\pgfpathlineto{\pgfqpoint{3.799868in}{1.844357in}}%
\pgfpathlineto{\pgfqpoint{3.814201in}{1.835226in}}%
\pgfpathlineto{\pgfqpoint{3.828538in}{1.826120in}}%
\pgfpathlineto{\pgfqpoint{3.842880in}{1.817039in}}%
\pgfpathlineto{\pgfqpoint{3.834373in}{1.832587in}}%
\pgfpathlineto{\pgfqpoint{3.825848in}{1.848847in}}%
\pgfpathlineto{\pgfqpoint{3.817305in}{1.865834in}}%
\pgfpathlineto{\pgfqpoint{3.808744in}{1.883563in}}%
\pgfpathlineto{\pgfqpoint{3.794355in}{1.893114in}}%
\pgfpathlineto{\pgfqpoint{3.779970in}{1.902690in}}%
\pgfpathlineto{\pgfqpoint{3.765588in}{1.912292in}}%
\pgfpathlineto{\pgfqpoint{3.751211in}{1.921919in}}%
\pgfpathlineto{\pgfqpoint{3.759822in}{1.903713in}}%
\pgfpathlineto{\pgfqpoint{3.768413in}{1.886253in}}%
\pgfpathlineto{\pgfqpoint{3.776985in}{1.869525in}}%
\pgfpathlineto{\pgfqpoint{3.785539in}{1.853514in}}%
\pgfpathclose%
\pgfusepath{fill}%
\end{pgfscope}%
\begin{pgfscope}%
\pgfpathrectangle{\pgfqpoint{1.150000in}{0.150000in}}{\pgfqpoint{5.700000in}{5.700000in}}%
\pgfusepath{clip}%
\pgfsetbuttcap%
\pgfsetroundjoin%
\definecolor{currentfill}{rgb}{0.275191,0.194905,0.496005}%
\pgfsetfillcolor{currentfill}%
\pgfsetfillopacity{0.700000}%
\pgfsetlinewidth{0.000000pt}%
\definecolor{currentstroke}{rgb}{0.000000,0.000000,0.000000}%
\pgfsetstrokecolor{currentstroke}%
\pgfsetdash{}{0pt}%
\pgfpathmoveto{\pgfqpoint{4.426784in}{1.437949in}}%
\pgfpathlineto{\pgfqpoint{4.441223in}{1.430648in}}%
\pgfpathlineto{\pgfqpoint{4.455668in}{1.423371in}}%
\pgfpathlineto{\pgfqpoint{4.470118in}{1.416117in}}%
\pgfpathlineto{\pgfqpoint{4.484574in}{1.408887in}}%
\pgfpathlineto{\pgfqpoint{4.476538in}{1.414953in}}%
\pgfpathlineto{\pgfqpoint{4.468496in}{1.421584in}}%
\pgfpathlineto{\pgfqpoint{4.460447in}{1.428792in}}%
\pgfpathlineto{\pgfqpoint{4.452391in}{1.436590in}}%
\pgfpathlineto{\pgfqpoint{4.437904in}{1.444252in}}%
\pgfpathlineto{\pgfqpoint{4.423422in}{1.451938in}}%
\pgfpathlineto{\pgfqpoint{4.408946in}{1.459647in}}%
\pgfpathlineto{\pgfqpoint{4.394474in}{1.467380in}}%
\pgfpathlineto{\pgfqpoint{4.402563in}{1.459144in}}%
\pgfpathlineto{\pgfqpoint{4.410644in}{1.451502in}}%
\pgfpathlineto{\pgfqpoint{4.418718in}{1.444441in}}%
\pgfpathlineto{\pgfqpoint{4.426784in}{1.437949in}}%
\pgfpathclose%
\pgfusepath{fill}%
\end{pgfscope}%
\begin{pgfscope}%
\pgfpathrectangle{\pgfqpoint{1.150000in}{0.150000in}}{\pgfqpoint{5.700000in}{5.700000in}}%
\pgfusepath{clip}%
\pgfsetbuttcap%
\pgfsetroundjoin%
\definecolor{currentfill}{rgb}{0.151918,0.500685,0.557587}%
\pgfsetfillcolor{currentfill}%
\pgfsetfillopacity{0.700000}%
\pgfsetlinewidth{0.000000pt}%
\definecolor{currentstroke}{rgb}{0.000000,0.000000,0.000000}%
\pgfsetstrokecolor{currentstroke}%
\pgfsetdash{}{0pt}%
\pgfpathmoveto{\pgfqpoint{3.350040in}{2.202168in}}%
\pgfpathlineto{\pgfqpoint{3.364324in}{2.191793in}}%
\pgfpathlineto{\pgfqpoint{3.378611in}{2.181446in}}%
\pgfpathlineto{\pgfqpoint{3.392901in}{2.171127in}}%
\pgfpathlineto{\pgfqpoint{3.407195in}{2.160836in}}%
\pgfpathlineto{\pgfqpoint{3.398240in}{2.182713in}}%
\pgfpathlineto{\pgfqpoint{3.389258in}{2.205396in}}%
\pgfpathlineto{\pgfqpoint{3.380248in}{2.228899in}}%
\pgfpathlineto{\pgfqpoint{3.371210in}{2.253240in}}%
\pgfpathlineto{\pgfqpoint{3.356858in}{2.264027in}}%
\pgfpathlineto{\pgfqpoint{3.342509in}{2.274841in}}%
\pgfpathlineto{\pgfqpoint{3.328162in}{2.285684in}}%
\pgfpathlineto{\pgfqpoint{3.313819in}{2.296555in}}%
\pgfpathlineto{\pgfqpoint{3.322918in}{2.271710in}}%
\pgfpathlineto{\pgfqpoint{3.331987in}{2.247709in}}%
\pgfpathlineto{\pgfqpoint{3.341028in}{2.224533in}}%
\pgfpathlineto{\pgfqpoint{3.350040in}{2.202168in}}%
\pgfpathclose%
\pgfusepath{fill}%
\end{pgfscope}%
\begin{pgfscope}%
\pgfpathrectangle{\pgfqpoint{1.150000in}{0.150000in}}{\pgfqpoint{5.700000in}{5.700000in}}%
\pgfusepath{clip}%
\pgfsetbuttcap%
\pgfsetroundjoin%
\definecolor{currentfill}{rgb}{0.196571,0.711827,0.479221}%
\pgfsetfillcolor{currentfill}%
\pgfsetfillopacity{0.700000}%
\pgfsetlinewidth{0.000000pt}%
\definecolor{currentstroke}{rgb}{0.000000,0.000000,0.000000}%
\pgfsetstrokecolor{currentstroke}%
\pgfsetdash{}{0pt}%
\pgfpathmoveto{\pgfqpoint{2.684963in}{2.804824in}}%
\pgfpathlineto{\pgfqpoint{2.699216in}{2.792570in}}%
\pgfpathlineto{\pgfqpoint{2.713470in}{2.780351in}}%
\pgfpathlineto{\pgfqpoint{2.727726in}{2.768168in}}%
\pgfpathlineto{\pgfqpoint{2.741983in}{2.756019in}}%
\pgfpathlineto{\pgfqpoint{2.732176in}{2.786872in}}%
\pgfpathlineto{\pgfqpoint{2.722326in}{2.818655in}}%
\pgfpathlineto{\pgfqpoint{2.712430in}{2.851386in}}%
\pgfpathlineto{\pgfqpoint{2.702489in}{2.885083in}}%
\pgfpathlineto{\pgfqpoint{2.688157in}{2.897762in}}%
\pgfpathlineto{\pgfqpoint{2.673826in}{2.910476in}}%
\pgfpathlineto{\pgfqpoint{2.659496in}{2.923227in}}%
\pgfpathlineto{\pgfqpoint{2.645167in}{2.936013in}}%
\pgfpathlineto{\pgfqpoint{2.655186in}{2.901776in}}%
\pgfpathlineto{\pgfqpoint{2.665158in}{2.868511in}}%
\pgfpathlineto{\pgfqpoint{2.675083in}{2.836200in}}%
\pgfpathlineto{\pgfqpoint{2.684963in}{2.804824in}}%
\pgfpathclose%
\pgfusepath{fill}%
\end{pgfscope}%
\begin{pgfscope}%
\pgfpathrectangle{\pgfqpoint{1.150000in}{0.150000in}}{\pgfqpoint{5.700000in}{5.700000in}}%
\pgfusepath{clip}%
\pgfsetbuttcap%
\pgfsetroundjoin%
\definecolor{currentfill}{rgb}{0.252194,0.269783,0.531579}%
\pgfsetfillcolor{currentfill}%
\pgfsetfillopacity{0.700000}%
\pgfsetlinewidth{0.000000pt}%
\definecolor{currentstroke}{rgb}{0.000000,0.000000,0.000000}%
\pgfsetstrokecolor{currentstroke}%
\pgfsetdash{}{0pt}%
\pgfpathmoveto{\pgfqpoint{4.163615in}{1.594323in}}%
\pgfpathlineto{\pgfqpoint{4.178007in}{1.586210in}}%
\pgfpathlineto{\pgfqpoint{4.192404in}{1.578121in}}%
\pgfpathlineto{\pgfqpoint{4.206806in}{1.570056in}}%
\pgfpathlineto{\pgfqpoint{4.221212in}{1.562015in}}%
\pgfpathlineto{\pgfqpoint{4.213009in}{1.572185in}}%
\pgfpathlineto{\pgfqpoint{4.204795in}{1.582987in}}%
\pgfpathlineto{\pgfqpoint{4.196570in}{1.594433in}}%
\pgfpathlineto{\pgfqpoint{4.188334in}{1.606538in}}%
\pgfpathlineto{\pgfqpoint{4.173889in}{1.615029in}}%
\pgfpathlineto{\pgfqpoint{4.159449in}{1.623543in}}%
\pgfpathlineto{\pgfqpoint{4.145014in}{1.632082in}}%
\pgfpathlineto{\pgfqpoint{4.130583in}{1.640645in}}%
\pgfpathlineto{\pgfqpoint{4.138859in}{1.628083in}}%
\pgfpathlineto{\pgfqpoint{4.147123in}{1.616185in}}%
\pgfpathlineto{\pgfqpoint{4.155375in}{1.604936in}}%
\pgfpathlineto{\pgfqpoint{4.163615in}{1.594323in}}%
\pgfpathclose%
\pgfusepath{fill}%
\end{pgfscope}%
\begin{pgfscope}%
\pgfpathrectangle{\pgfqpoint{1.150000in}{0.150000in}}{\pgfqpoint{5.700000in}{5.700000in}}%
\pgfusepath{clip}%
\pgfsetbuttcap%
\pgfsetroundjoin%
\definecolor{currentfill}{rgb}{0.282656,0.100196,0.422160}%
\pgfsetfillcolor{currentfill}%
\pgfsetfillopacity{0.700000}%
\pgfsetlinewidth{0.000000pt}%
\definecolor{currentstroke}{rgb}{0.000000,0.000000,0.000000}%
\pgfsetstrokecolor{currentstroke}%
\pgfsetdash{}{0pt}%
\pgfpathmoveto{\pgfqpoint{4.896107in}{1.232269in}}%
\pgfpathlineto{\pgfqpoint{4.910663in}{1.226432in}}%
\pgfpathlineto{\pgfqpoint{4.925226in}{1.220617in}}%
\pgfpathlineto{\pgfqpoint{4.939795in}{1.214826in}}%
\pgfpathlineto{\pgfqpoint{4.931953in}{1.214133in}}%
\pgfpathlineto{\pgfqpoint{4.924109in}{1.213887in}}%
\pgfpathlineto{\pgfqpoint{4.916264in}{1.214099in}}%
\pgfpathlineto{\pgfqpoint{4.908418in}{1.214780in}}%
\pgfpathlineto{\pgfqpoint{4.893829in}{1.220972in}}%
\pgfpathlineto{\pgfqpoint{4.879246in}{1.227187in}}%
\pgfpathlineto{\pgfqpoint{4.864669in}{1.233424in}}%
\pgfpathlineto{\pgfqpoint{4.872531in}{1.232439in}}%
\pgfpathlineto{\pgfqpoint{4.880391in}{1.231925in}}%
\pgfpathlineto{\pgfqpoint{4.888250in}{1.231872in}}%
\pgfpathlineto{\pgfqpoint{4.896107in}{1.232269in}}%
\pgfpathclose%
\pgfusepath{fill}%
\end{pgfscope}%
\begin{pgfscope}%
\pgfpathrectangle{\pgfqpoint{1.150000in}{0.150000in}}{\pgfqpoint{5.700000in}{5.700000in}}%
\pgfusepath{clip}%
\pgfsetbuttcap%
\pgfsetroundjoin%
\definecolor{currentfill}{rgb}{0.282884,0.135920,0.453427}%
\pgfsetfillcolor{currentfill}%
\pgfsetfillopacity{0.700000}%
\pgfsetlinewidth{0.000000pt}%
\definecolor{currentstroke}{rgb}{0.000000,0.000000,0.000000}%
\pgfsetstrokecolor{currentstroke}%
\pgfsetdash{}{0pt}%
\pgfpathmoveto{\pgfqpoint{4.690232in}{1.310065in}}%
\pgfpathlineto{\pgfqpoint{4.704735in}{1.303552in}}%
\pgfpathlineto{\pgfqpoint{4.719244in}{1.297062in}}%
\pgfpathlineto{\pgfqpoint{4.733759in}{1.290595in}}%
\pgfpathlineto{\pgfqpoint{4.748280in}{1.284151in}}%
\pgfpathlineto{\pgfqpoint{4.740370in}{1.286438in}}%
\pgfpathlineto{\pgfqpoint{4.732458in}{1.289228in}}%
\pgfpathlineto{\pgfqpoint{4.724542in}{1.292531in}}%
\pgfpathlineto{\pgfqpoint{4.716623in}{1.296359in}}%
\pgfpathlineto{\pgfqpoint{4.702077in}{1.303218in}}%
\pgfpathlineto{\pgfqpoint{4.687537in}{1.310101in}}%
\pgfpathlineto{\pgfqpoint{4.673003in}{1.317007in}}%
\pgfpathlineto{\pgfqpoint{4.658474in}{1.323935in}}%
\pgfpathlineto{\pgfqpoint{4.666419in}{1.319686in}}%
\pgfpathlineto{\pgfqpoint{4.674361in}{1.315966in}}%
\pgfpathlineto{\pgfqpoint{4.682298in}{1.312763in}}%
\pgfpathlineto{\pgfqpoint{4.690232in}{1.310065in}}%
\pgfpathclose%
\pgfusepath{fill}%
\end{pgfscope}%
\begin{pgfscope}%
\pgfpathrectangle{\pgfqpoint{1.150000in}{0.150000in}}{\pgfqpoint{5.700000in}{5.700000in}}%
\pgfusepath{clip}%
\pgfsetbuttcap%
\pgfsetroundjoin%
\definecolor{currentfill}{rgb}{0.175707,0.697900,0.491033}%
\pgfsetfillcolor{currentfill}%
\pgfsetfillopacity{0.700000}%
\pgfsetlinewidth{0.000000pt}%
\definecolor{currentstroke}{rgb}{0.000000,0.000000,0.000000}%
\pgfsetstrokecolor{currentstroke}%
\pgfsetdash{}{0pt}%
\pgfpathmoveto{\pgfqpoint{2.741983in}{2.756019in}}%
\pgfpathlineto{\pgfqpoint{2.756241in}{2.743905in}}%
\pgfpathlineto{\pgfqpoint{2.770501in}{2.731826in}}%
\pgfpathlineto{\pgfqpoint{2.784763in}{2.719781in}}%
\pgfpathlineto{\pgfqpoint{2.799026in}{2.707770in}}%
\pgfpathlineto{\pgfqpoint{2.789292in}{2.738102in}}%
\pgfpathlineto{\pgfqpoint{2.779515in}{2.769359in}}%
\pgfpathlineto{\pgfqpoint{2.769694in}{2.801557in}}%
\pgfpathlineto{\pgfqpoint{2.759829in}{2.834715in}}%
\pgfpathlineto{\pgfqpoint{2.745492in}{2.847255in}}%
\pgfpathlineto{\pgfqpoint{2.731157in}{2.859829in}}%
\pgfpathlineto{\pgfqpoint{2.716822in}{2.872438in}}%
\pgfpathlineto{\pgfqpoint{2.702489in}{2.885083in}}%
\pgfpathlineto{\pgfqpoint{2.712430in}{2.851386in}}%
\pgfpathlineto{\pgfqpoint{2.722326in}{2.818655in}}%
\pgfpathlineto{\pgfqpoint{2.732176in}{2.786872in}}%
\pgfpathlineto{\pgfqpoint{2.741983in}{2.756019in}}%
\pgfpathclose%
\pgfusepath{fill}%
\end{pgfscope}%
\begin{pgfscope}%
\pgfpathrectangle{\pgfqpoint{1.150000in}{0.150000in}}{\pgfqpoint{5.700000in}{5.700000in}}%
\pgfusepath{clip}%
\pgfsetbuttcap%
\pgfsetroundjoin%
\definecolor{currentfill}{rgb}{0.210503,0.363727,0.552206}%
\pgfsetfillcolor{currentfill}%
\pgfsetfillopacity{0.700000}%
\pgfsetlinewidth{0.000000pt}%
\definecolor{currentstroke}{rgb}{0.000000,0.000000,0.000000}%
\pgfsetstrokecolor{currentstroke}%
\pgfsetdash{}{0pt}%
\pgfpathmoveto{\pgfqpoint{3.842880in}{1.817039in}}%
\pgfpathlineto{\pgfqpoint{3.857225in}{1.807984in}}%
\pgfpathlineto{\pgfqpoint{3.871574in}{1.798953in}}%
\pgfpathlineto{\pgfqpoint{3.885928in}{1.789948in}}%
\pgfpathlineto{\pgfqpoint{3.900285in}{1.780968in}}%
\pgfpathlineto{\pgfqpoint{3.891824in}{1.796053in}}%
\pgfpathlineto{\pgfqpoint{3.883346in}{1.811845in}}%
\pgfpathlineto{\pgfqpoint{3.874850in}{1.828360in}}%
\pgfpathlineto{\pgfqpoint{3.866337in}{1.845612in}}%
\pgfpathlineto{\pgfqpoint{3.851933in}{1.855062in}}%
\pgfpathlineto{\pgfqpoint{3.837533in}{1.864537in}}%
\pgfpathlineto{\pgfqpoint{3.823136in}{1.874037in}}%
\pgfpathlineto{\pgfqpoint{3.808744in}{1.883563in}}%
\pgfpathlineto{\pgfqpoint{3.817305in}{1.865834in}}%
\pgfpathlineto{\pgfqpoint{3.825848in}{1.848847in}}%
\pgfpathlineto{\pgfqpoint{3.834373in}{1.832587in}}%
\pgfpathlineto{\pgfqpoint{3.842880in}{1.817039in}}%
\pgfpathclose%
\pgfusepath{fill}%
\end{pgfscope}%
\begin{pgfscope}%
\pgfpathrectangle{\pgfqpoint{1.150000in}{0.150000in}}{\pgfqpoint{5.700000in}{5.700000in}}%
\pgfusepath{clip}%
\pgfsetbuttcap%
\pgfsetroundjoin%
\definecolor{currentfill}{rgb}{0.156270,0.489624,0.557936}%
\pgfsetfillcolor{currentfill}%
\pgfsetfillopacity{0.700000}%
\pgfsetlinewidth{0.000000pt}%
\definecolor{currentstroke}{rgb}{0.000000,0.000000,0.000000}%
\pgfsetstrokecolor{currentstroke}%
\pgfsetdash{}{0pt}%
\pgfpathmoveto{\pgfqpoint{3.407195in}{2.160836in}}%
\pgfpathlineto{\pgfqpoint{3.421491in}{2.150572in}}%
\pgfpathlineto{\pgfqpoint{3.435790in}{2.140336in}}%
\pgfpathlineto{\pgfqpoint{3.450093in}{2.130127in}}%
\pgfpathlineto{\pgfqpoint{3.464398in}{2.119946in}}%
\pgfpathlineto{\pgfqpoint{3.455500in}{2.141336in}}%
\pgfpathlineto{\pgfqpoint{3.446576in}{2.163527in}}%
\pgfpathlineto{\pgfqpoint{3.437624in}{2.186533in}}%
\pgfpathlineto{\pgfqpoint{3.428646in}{2.210372in}}%
\pgfpathlineto{\pgfqpoint{3.414282in}{2.221047in}}%
\pgfpathlineto{\pgfqpoint{3.399922in}{2.231751in}}%
\pgfpathlineto{\pgfqpoint{3.385564in}{2.242481in}}%
\pgfpathlineto{\pgfqpoint{3.371210in}{2.253240in}}%
\pgfpathlineto{\pgfqpoint{3.380248in}{2.228899in}}%
\pgfpathlineto{\pgfqpoint{3.389258in}{2.205396in}}%
\pgfpathlineto{\pgfqpoint{3.398240in}{2.182713in}}%
\pgfpathlineto{\pgfqpoint{3.407195in}{2.160836in}}%
\pgfpathclose%
\pgfusepath{fill}%
\end{pgfscope}%
\begin{pgfscope}%
\pgfpathrectangle{\pgfqpoint{1.150000in}{0.150000in}}{\pgfqpoint{5.700000in}{5.700000in}}%
\pgfusepath{clip}%
\pgfsetbuttcap%
\pgfsetroundjoin%
\definecolor{currentfill}{rgb}{0.276194,0.190074,0.493001}%
\pgfsetfillcolor{currentfill}%
\pgfsetfillopacity{0.700000}%
\pgfsetlinewidth{0.000000pt}%
\definecolor{currentstroke}{rgb}{0.000000,0.000000,0.000000}%
\pgfsetstrokecolor{currentstroke}%
\pgfsetdash{}{0pt}%
\pgfpathmoveto{\pgfqpoint{4.484574in}{1.408887in}}%
\pgfpathlineto{\pgfqpoint{4.499035in}{1.401680in}}%
\pgfpathlineto{\pgfqpoint{4.513501in}{1.394496in}}%
\pgfpathlineto{\pgfqpoint{4.527974in}{1.387335in}}%
\pgfpathlineto{\pgfqpoint{4.542451in}{1.380198in}}%
\pgfpathlineto{\pgfqpoint{4.534445in}{1.385839in}}%
\pgfpathlineto{\pgfqpoint{4.526433in}{1.392040in}}%
\pgfpathlineto{\pgfqpoint{4.518415in}{1.398815in}}%
\pgfpathlineto{\pgfqpoint{4.510391in}{1.406175in}}%
\pgfpathlineto{\pgfqpoint{4.495883in}{1.413744in}}%
\pgfpathlineto{\pgfqpoint{4.481380in}{1.421336in}}%
\pgfpathlineto{\pgfqpoint{4.466883in}{1.428951in}}%
\pgfpathlineto{\pgfqpoint{4.452391in}{1.436590in}}%
\pgfpathlineto{\pgfqpoint{4.460447in}{1.428792in}}%
\pgfpathlineto{\pgfqpoint{4.468496in}{1.421584in}}%
\pgfpathlineto{\pgfqpoint{4.476538in}{1.414953in}}%
\pgfpathlineto{\pgfqpoint{4.484574in}{1.408887in}}%
\pgfpathclose%
\pgfusepath{fill}%
\end{pgfscope}%
\begin{pgfscope}%
\pgfpathrectangle{\pgfqpoint{1.150000in}{0.150000in}}{\pgfqpoint{5.700000in}{5.700000in}}%
\pgfusepath{clip}%
\pgfsetbuttcap%
\pgfsetroundjoin%
\definecolor{currentfill}{rgb}{0.157851,0.683765,0.501686}%
\pgfsetfillcolor{currentfill}%
\pgfsetfillopacity{0.700000}%
\pgfsetlinewidth{0.000000pt}%
\definecolor{currentstroke}{rgb}{0.000000,0.000000,0.000000}%
\pgfsetstrokecolor{currentstroke}%
\pgfsetdash{}{0pt}%
\pgfpathmoveto{\pgfqpoint{2.799026in}{2.707770in}}%
\pgfpathlineto{\pgfqpoint{2.813291in}{2.695793in}}%
\pgfpathlineto{\pgfqpoint{2.827557in}{2.683850in}}%
\pgfpathlineto{\pgfqpoint{2.841825in}{2.671941in}}%
\pgfpathlineto{\pgfqpoint{2.856095in}{2.660064in}}%
\pgfpathlineto{\pgfqpoint{2.846432in}{2.689877in}}%
\pgfpathlineto{\pgfqpoint{2.836728in}{2.720608in}}%
\pgfpathlineto{\pgfqpoint{2.826981in}{2.752275in}}%
\pgfpathlineto{\pgfqpoint{2.817192in}{2.784896in}}%
\pgfpathlineto{\pgfqpoint{2.802849in}{2.797300in}}%
\pgfpathlineto{\pgfqpoint{2.788508in}{2.809738in}}%
\pgfpathlineto{\pgfqpoint{2.774168in}{2.822209in}}%
\pgfpathlineto{\pgfqpoint{2.759829in}{2.834715in}}%
\pgfpathlineto{\pgfqpoint{2.769694in}{2.801557in}}%
\pgfpathlineto{\pgfqpoint{2.779515in}{2.769359in}}%
\pgfpathlineto{\pgfqpoint{2.789292in}{2.738102in}}%
\pgfpathlineto{\pgfqpoint{2.799026in}{2.707770in}}%
\pgfpathclose%
\pgfusepath{fill}%
\end{pgfscope}%
\begin{pgfscope}%
\pgfpathrectangle{\pgfqpoint{1.150000in}{0.150000in}}{\pgfqpoint{5.700000in}{5.700000in}}%
\pgfusepath{clip}%
\pgfsetbuttcap%
\pgfsetroundjoin%
\definecolor{currentfill}{rgb}{0.255645,0.260703,0.528312}%
\pgfsetfillcolor{currentfill}%
\pgfsetfillopacity{0.700000}%
\pgfsetlinewidth{0.000000pt}%
\definecolor{currentstroke}{rgb}{0.000000,0.000000,0.000000}%
\pgfsetstrokecolor{currentstroke}%
\pgfsetdash{}{0pt}%
\pgfpathmoveto{\pgfqpoint{4.221212in}{1.562015in}}%
\pgfpathlineto{\pgfqpoint{4.235624in}{1.553998in}}%
\pgfpathlineto{\pgfqpoint{4.250040in}{1.546005in}}%
\pgfpathlineto{\pgfqpoint{4.264461in}{1.538036in}}%
\pgfpathlineto{\pgfqpoint{4.278887in}{1.530090in}}%
\pgfpathlineto{\pgfqpoint{4.270720in}{1.539817in}}%
\pgfpathlineto{\pgfqpoint{4.262543in}{1.550172in}}%
\pgfpathlineto{\pgfqpoint{4.254356in}{1.561167in}}%
\pgfpathlineto{\pgfqpoint{4.246159in}{1.572816in}}%
\pgfpathlineto{\pgfqpoint{4.231696in}{1.581211in}}%
\pgfpathlineto{\pgfqpoint{4.217237in}{1.589629in}}%
\pgfpathlineto{\pgfqpoint{4.202783in}{1.598072in}}%
\pgfpathlineto{\pgfqpoint{4.188334in}{1.606538in}}%
\pgfpathlineto{\pgfqpoint{4.196570in}{1.594433in}}%
\pgfpathlineto{\pgfqpoint{4.204795in}{1.582987in}}%
\pgfpathlineto{\pgfqpoint{4.213009in}{1.572185in}}%
\pgfpathlineto{\pgfqpoint{4.221212in}{1.562015in}}%
\pgfpathclose%
\pgfusepath{fill}%
\end{pgfscope}%
\begin{pgfscope}%
\pgfpathrectangle{\pgfqpoint{1.150000in}{0.150000in}}{\pgfqpoint{5.700000in}{5.700000in}}%
\pgfusepath{clip}%
\pgfsetbuttcap%
\pgfsetroundjoin%
\definecolor{currentfill}{rgb}{0.160665,0.478540,0.558115}%
\pgfsetfillcolor{currentfill}%
\pgfsetfillopacity{0.700000}%
\pgfsetlinewidth{0.000000pt}%
\definecolor{currentstroke}{rgb}{0.000000,0.000000,0.000000}%
\pgfsetstrokecolor{currentstroke}%
\pgfsetdash{}{0pt}%
\pgfpathmoveto{\pgfqpoint{3.464398in}{2.119946in}}%
\pgfpathlineto{\pgfqpoint{3.478707in}{2.109792in}}%
\pgfpathlineto{\pgfqpoint{3.493019in}{2.099665in}}%
\pgfpathlineto{\pgfqpoint{3.507334in}{2.089565in}}%
\pgfpathlineto{\pgfqpoint{3.521653in}{2.079492in}}%
\pgfpathlineto{\pgfqpoint{3.512810in}{2.100396in}}%
\pgfpathlineto{\pgfqpoint{3.503942in}{2.122095in}}%
\pgfpathlineto{\pgfqpoint{3.495048in}{2.144605in}}%
\pgfpathlineto{\pgfqpoint{3.486128in}{2.167942in}}%
\pgfpathlineto{\pgfqpoint{3.471753in}{2.178509in}}%
\pgfpathlineto{\pgfqpoint{3.457381in}{2.189102in}}%
\pgfpathlineto{\pgfqpoint{3.443012in}{2.199723in}}%
\pgfpathlineto{\pgfqpoint{3.428646in}{2.210372in}}%
\pgfpathlineto{\pgfqpoint{3.437624in}{2.186533in}}%
\pgfpathlineto{\pgfqpoint{3.446576in}{2.163527in}}%
\pgfpathlineto{\pgfqpoint{3.455500in}{2.141336in}}%
\pgfpathlineto{\pgfqpoint{3.464398in}{2.119946in}}%
\pgfpathclose%
\pgfusepath{fill}%
\end{pgfscope}%
\begin{pgfscope}%
\pgfpathrectangle{\pgfqpoint{1.150000in}{0.150000in}}{\pgfqpoint{5.700000in}{5.700000in}}%
\pgfusepath{clip}%
\pgfsetbuttcap%
\pgfsetroundjoin%
\definecolor{currentfill}{rgb}{0.216210,0.351535,0.550627}%
\pgfsetfillcolor{currentfill}%
\pgfsetfillopacity{0.700000}%
\pgfsetlinewidth{0.000000pt}%
\definecolor{currentstroke}{rgb}{0.000000,0.000000,0.000000}%
\pgfsetstrokecolor{currentstroke}%
\pgfsetdash{}{0pt}%
\pgfpathmoveto{\pgfqpoint{3.900285in}{1.780968in}}%
\pgfpathlineto{\pgfqpoint{3.914647in}{1.772013in}}%
\pgfpathlineto{\pgfqpoint{3.929013in}{1.763083in}}%
\pgfpathlineto{\pgfqpoint{3.943383in}{1.754177in}}%
\pgfpathlineto{\pgfqpoint{3.957757in}{1.745296in}}%
\pgfpathlineto{\pgfqpoint{3.949340in}{1.759918in}}%
\pgfpathlineto{\pgfqpoint{3.940907in}{1.775244in}}%
\pgfpathlineto{\pgfqpoint{3.932458in}{1.791287in}}%
\pgfpathlineto{\pgfqpoint{3.923993in}{1.808062in}}%
\pgfpathlineto{\pgfqpoint{3.909573in}{1.817412in}}%
\pgfpathlineto{\pgfqpoint{3.895157in}{1.826787in}}%
\pgfpathlineto{\pgfqpoint{3.880745in}{1.836187in}}%
\pgfpathlineto{\pgfqpoint{3.866337in}{1.845612in}}%
\pgfpathlineto{\pgfqpoint{3.874850in}{1.828360in}}%
\pgfpathlineto{\pgfqpoint{3.883346in}{1.811845in}}%
\pgfpathlineto{\pgfqpoint{3.891824in}{1.796053in}}%
\pgfpathlineto{\pgfqpoint{3.900285in}{1.780968in}}%
\pgfpathclose%
\pgfusepath{fill}%
\end{pgfscope}%
\begin{pgfscope}%
\pgfpathrectangle{\pgfqpoint{1.150000in}{0.150000in}}{\pgfqpoint{5.700000in}{5.700000in}}%
\pgfusepath{clip}%
\pgfsetbuttcap%
\pgfsetroundjoin%
\definecolor{currentfill}{rgb}{0.283072,0.130895,0.449241}%
\pgfsetfillcolor{currentfill}%
\pgfsetfillopacity{0.700000}%
\pgfsetlinewidth{0.000000pt}%
\definecolor{currentstroke}{rgb}{0.000000,0.000000,0.000000}%
\pgfsetstrokecolor{currentstroke}%
\pgfsetdash{}{0pt}%
\pgfpathmoveto{\pgfqpoint{4.748280in}{1.284151in}}%
\pgfpathlineto{\pgfqpoint{4.762807in}{1.277730in}}%
\pgfpathlineto{\pgfqpoint{4.777340in}{1.271331in}}%
\pgfpathlineto{\pgfqpoint{4.791880in}{1.264956in}}%
\pgfpathlineto{\pgfqpoint{4.806425in}{1.258604in}}%
\pgfpathlineto{\pgfqpoint{4.798539in}{1.260482in}}%
\pgfpathlineto{\pgfqpoint{4.790650in}{1.262858in}}%
\pgfpathlineto{\pgfqpoint{4.782759in}{1.265743in}}%
\pgfpathlineto{\pgfqpoint{4.774865in}{1.269150in}}%
\pgfpathlineto{\pgfqpoint{4.760296in}{1.275918in}}%
\pgfpathlineto{\pgfqpoint{4.745732in}{1.282708in}}%
\pgfpathlineto{\pgfqpoint{4.731175in}{1.289522in}}%
\pgfpathlineto{\pgfqpoint{4.716623in}{1.296359in}}%
\pgfpathlineto{\pgfqpoint{4.724542in}{1.292531in}}%
\pgfpathlineto{\pgfqpoint{4.732458in}{1.289228in}}%
\pgfpathlineto{\pgfqpoint{4.740370in}{1.286438in}}%
\pgfpathlineto{\pgfqpoint{4.748280in}{1.284151in}}%
\pgfpathclose%
\pgfusepath{fill}%
\end{pgfscope}%
\begin{pgfscope}%
\pgfpathrectangle{\pgfqpoint{1.150000in}{0.150000in}}{\pgfqpoint{5.700000in}{5.700000in}}%
\pgfusepath{clip}%
\pgfsetbuttcap%
\pgfsetroundjoin%
\definecolor{currentfill}{rgb}{0.143303,0.669459,0.511215}%
\pgfsetfillcolor{currentfill}%
\pgfsetfillopacity{0.700000}%
\pgfsetlinewidth{0.000000pt}%
\definecolor{currentstroke}{rgb}{0.000000,0.000000,0.000000}%
\pgfsetstrokecolor{currentstroke}%
\pgfsetdash{}{0pt}%
\pgfpathmoveto{\pgfqpoint{2.856095in}{2.660064in}}%
\pgfpathlineto{\pgfqpoint{2.870366in}{2.648221in}}%
\pgfpathlineto{\pgfqpoint{2.884639in}{2.636410in}}%
\pgfpathlineto{\pgfqpoint{2.898914in}{2.624632in}}%
\pgfpathlineto{\pgfqpoint{2.913191in}{2.612887in}}%
\pgfpathlineto{\pgfqpoint{2.903599in}{2.642181in}}%
\pgfpathlineto{\pgfqpoint{2.893966in}{2.672389in}}%
\pgfpathlineto{\pgfqpoint{2.884293in}{2.703526in}}%
\pgfpathlineto{\pgfqpoint{2.874577in}{2.735612in}}%
\pgfpathlineto{\pgfqpoint{2.860228in}{2.747883in}}%
\pgfpathlineto{\pgfqpoint{2.845881in}{2.760188in}}%
\pgfpathlineto{\pgfqpoint{2.831536in}{2.772525in}}%
\pgfpathlineto{\pgfqpoint{2.817192in}{2.784896in}}%
\pgfpathlineto{\pgfqpoint{2.826981in}{2.752275in}}%
\pgfpathlineto{\pgfqpoint{2.836728in}{2.720608in}}%
\pgfpathlineto{\pgfqpoint{2.846432in}{2.689877in}}%
\pgfpathlineto{\pgfqpoint{2.856095in}{2.660064in}}%
\pgfpathclose%
\pgfusepath{fill}%
\end{pgfscope}%
\begin{pgfscope}%
\pgfpathrectangle{\pgfqpoint{1.150000in}{0.150000in}}{\pgfqpoint{5.700000in}{5.700000in}}%
\pgfusepath{clip}%
\pgfsetbuttcap%
\pgfsetroundjoin%
\definecolor{currentfill}{rgb}{0.278012,0.180367,0.486697}%
\pgfsetfillcolor{currentfill}%
\pgfsetfillopacity{0.700000}%
\pgfsetlinewidth{0.000000pt}%
\definecolor{currentstroke}{rgb}{0.000000,0.000000,0.000000}%
\pgfsetstrokecolor{currentstroke}%
\pgfsetdash{}{0pt}%
\pgfpathmoveto{\pgfqpoint{4.542451in}{1.380198in}}%
\pgfpathlineto{\pgfqpoint{4.556934in}{1.373084in}}%
\pgfpathlineto{\pgfqpoint{4.571423in}{1.365993in}}%
\pgfpathlineto{\pgfqpoint{4.585917in}{1.358926in}}%
\pgfpathlineto{\pgfqpoint{4.600417in}{1.351881in}}%
\pgfpathlineto{\pgfqpoint{4.592440in}{1.357097in}}%
\pgfpathlineto{\pgfqpoint{4.584457in}{1.362869in}}%
\pgfpathlineto{\pgfqpoint{4.576470in}{1.369210in}}%
\pgfpathlineto{\pgfqpoint{4.568476in}{1.376132in}}%
\pgfpathlineto{\pgfqpoint{4.553947in}{1.383608in}}%
\pgfpathlineto{\pgfqpoint{4.539423in}{1.391107in}}%
\pgfpathlineto{\pgfqpoint{4.524904in}{1.398629in}}%
\pgfpathlineto{\pgfqpoint{4.510391in}{1.406175in}}%
\pgfpathlineto{\pgfqpoint{4.518415in}{1.398815in}}%
\pgfpathlineto{\pgfqpoint{4.526433in}{1.392040in}}%
\pgfpathlineto{\pgfqpoint{4.534445in}{1.385839in}}%
\pgfpathlineto{\pgfqpoint{4.542451in}{1.380198in}}%
\pgfpathclose%
\pgfusepath{fill}%
\end{pgfscope}%
\begin{pgfscope}%
\pgfpathrectangle{\pgfqpoint{1.150000in}{0.150000in}}{\pgfqpoint{5.700000in}{5.700000in}}%
\pgfusepath{clip}%
\pgfsetbuttcap%
\pgfsetroundjoin%
\definecolor{currentfill}{rgb}{0.132268,0.655014,0.519661}%
\pgfsetfillcolor{currentfill}%
\pgfsetfillopacity{0.700000}%
\pgfsetlinewidth{0.000000pt}%
\definecolor{currentstroke}{rgb}{0.000000,0.000000,0.000000}%
\pgfsetstrokecolor{currentstroke}%
\pgfsetdash{}{0pt}%
\pgfpathmoveto{\pgfqpoint{2.913191in}{2.612887in}}%
\pgfpathlineto{\pgfqpoint{2.927470in}{2.601174in}}%
\pgfpathlineto{\pgfqpoint{2.941751in}{2.589493in}}%
\pgfpathlineto{\pgfqpoint{2.956033in}{2.577845in}}%
\pgfpathlineto{\pgfqpoint{2.970318in}{2.566228in}}%
\pgfpathlineto{\pgfqpoint{2.960795in}{2.595005in}}%
\pgfpathlineto{\pgfqpoint{2.951233in}{2.624690in}}%
\pgfpathlineto{\pgfqpoint{2.941631in}{2.655300in}}%
\pgfpathlineto{\pgfqpoint{2.931989in}{2.686851in}}%
\pgfpathlineto{\pgfqpoint{2.917633in}{2.698993in}}%
\pgfpathlineto{\pgfqpoint{2.903280in}{2.711167in}}%
\pgfpathlineto{\pgfqpoint{2.888928in}{2.723373in}}%
\pgfpathlineto{\pgfqpoint{2.874577in}{2.735612in}}%
\pgfpathlineto{\pgfqpoint{2.884293in}{2.703526in}}%
\pgfpathlineto{\pgfqpoint{2.893966in}{2.672389in}}%
\pgfpathlineto{\pgfqpoint{2.903599in}{2.642181in}}%
\pgfpathlineto{\pgfqpoint{2.913191in}{2.612887in}}%
\pgfpathclose%
\pgfusepath{fill}%
\end{pgfscope}%
\begin{pgfscope}%
\pgfpathrectangle{\pgfqpoint{1.150000in}{0.150000in}}{\pgfqpoint{5.700000in}{5.700000in}}%
\pgfusepath{clip}%
\pgfsetbuttcap%
\pgfsetroundjoin%
\definecolor{currentfill}{rgb}{0.165117,0.467423,0.558141}%
\pgfsetfillcolor{currentfill}%
\pgfsetfillopacity{0.700000}%
\pgfsetlinewidth{0.000000pt}%
\definecolor{currentstroke}{rgb}{0.000000,0.000000,0.000000}%
\pgfsetstrokecolor{currentstroke}%
\pgfsetdash{}{0pt}%
\pgfpathmoveto{\pgfqpoint{3.521653in}{2.079492in}}%
\pgfpathlineto{\pgfqpoint{3.535974in}{2.069445in}}%
\pgfpathlineto{\pgfqpoint{3.550299in}{2.059426in}}%
\pgfpathlineto{\pgfqpoint{3.564628in}{2.049433in}}%
\pgfpathlineto{\pgfqpoint{3.578960in}{2.039467in}}%
\pgfpathlineto{\pgfqpoint{3.570171in}{2.059886in}}%
\pgfpathlineto{\pgfqpoint{3.561358in}{2.081095in}}%
\pgfpathlineto{\pgfqpoint{3.552521in}{2.103110in}}%
\pgfpathlineto{\pgfqpoint{3.543659in}{2.125946in}}%
\pgfpathlineto{\pgfqpoint{3.529272in}{2.136405in}}%
\pgfpathlineto{\pgfqpoint{3.514888in}{2.146891in}}%
\pgfpathlineto{\pgfqpoint{3.500506in}{2.157403in}}%
\pgfpathlineto{\pgfqpoint{3.486128in}{2.167942in}}%
\pgfpathlineto{\pgfqpoint{3.495048in}{2.144605in}}%
\pgfpathlineto{\pgfqpoint{3.503942in}{2.122095in}}%
\pgfpathlineto{\pgfqpoint{3.512810in}{2.100396in}}%
\pgfpathlineto{\pgfqpoint{3.521653in}{2.079492in}}%
\pgfpathclose%
\pgfusepath{fill}%
\end{pgfscope}%
\begin{pgfscope}%
\pgfpathrectangle{\pgfqpoint{1.150000in}{0.150000in}}{\pgfqpoint{5.700000in}{5.700000in}}%
\pgfusepath{clip}%
\pgfsetbuttcap%
\pgfsetroundjoin%
\definecolor{currentfill}{rgb}{0.258965,0.251537,0.524736}%
\pgfsetfillcolor{currentfill}%
\pgfsetfillopacity{0.700000}%
\pgfsetlinewidth{0.000000pt}%
\definecolor{currentstroke}{rgb}{0.000000,0.000000,0.000000}%
\pgfsetstrokecolor{currentstroke}%
\pgfsetdash{}{0pt}%
\pgfpathmoveto{\pgfqpoint{4.278887in}{1.530090in}}%
\pgfpathlineto{\pgfqpoint{4.293318in}{1.522169in}}%
\pgfpathlineto{\pgfqpoint{4.307754in}{1.514271in}}%
\pgfpathlineto{\pgfqpoint{4.322194in}{1.506396in}}%
\pgfpathlineto{\pgfqpoint{4.336640in}{1.498546in}}%
\pgfpathlineto{\pgfqpoint{4.328509in}{1.507830in}}%
\pgfpathlineto{\pgfqpoint{4.320368in}{1.517738in}}%
\pgfpathlineto{\pgfqpoint{4.312219in}{1.528282in}}%
\pgfpathlineto{\pgfqpoint{4.304059in}{1.539476in}}%
\pgfpathlineto{\pgfqpoint{4.289577in}{1.547775in}}%
\pgfpathlineto{\pgfqpoint{4.275099in}{1.556098in}}%
\pgfpathlineto{\pgfqpoint{4.260627in}{1.564445in}}%
\pgfpathlineto{\pgfqpoint{4.246159in}{1.572816in}}%
\pgfpathlineto{\pgfqpoint{4.254356in}{1.561167in}}%
\pgfpathlineto{\pgfqpoint{4.262543in}{1.550172in}}%
\pgfpathlineto{\pgfqpoint{4.270720in}{1.539817in}}%
\pgfpathlineto{\pgfqpoint{4.278887in}{1.530090in}}%
\pgfpathclose%
\pgfusepath{fill}%
\end{pgfscope}%
\begin{pgfscope}%
\pgfpathrectangle{\pgfqpoint{1.150000in}{0.150000in}}{\pgfqpoint{5.700000in}{5.700000in}}%
\pgfusepath{clip}%
\pgfsetbuttcap%
\pgfsetroundjoin%
\definecolor{currentfill}{rgb}{0.993248,0.906157,0.143936}%
\pgfsetfillcolor{currentfill}%
\pgfsetfillopacity{0.700000}%
\pgfsetlinewidth{0.000000pt}%
\definecolor{currentstroke}{rgb}{0.000000,0.000000,0.000000}%
\pgfsetstrokecolor{currentstroke}%
\pgfsetdash{}{0pt}%
\pgfpathmoveto{\pgfqpoint{1.843192in}{3.718951in}}%
\pgfpathlineto{\pgfqpoint{1.857531in}{3.703609in}}%
\pgfpathlineto{\pgfqpoint{1.871869in}{3.688326in}}%
\pgfpathlineto{\pgfqpoint{1.886205in}{3.673100in}}%
\pgfpathlineto{\pgfqpoint{1.900540in}{3.657932in}}%
\pgfpathlineto{\pgfqpoint{1.889340in}{3.700444in}}%
\pgfpathlineto{\pgfqpoint{1.878072in}{3.744049in}}%
\pgfpathlineto{\pgfqpoint{1.866732in}{3.788767in}}%
\pgfpathlineto{\pgfqpoint{1.852325in}{3.804373in}}%
\pgfpathlineto{\pgfqpoint{1.837916in}{3.820037in}}%
\pgfpathlineto{\pgfqpoint{1.823506in}{3.835759in}}%
\pgfpathlineto{\pgfqpoint{1.809093in}{3.851541in}}%
\pgfpathlineto{\pgfqpoint{1.820531in}{3.806231in}}%
\pgfpathlineto{\pgfqpoint{1.831897in}{3.762041in}}%
\pgfpathlineto{\pgfqpoint{1.843192in}{3.718951in}}%
\pgfpathclose%
\pgfusepath{fill}%
\end{pgfscope}%
\begin{pgfscope}%
\pgfpathrectangle{\pgfqpoint{1.150000in}{0.150000in}}{\pgfqpoint{5.700000in}{5.700000in}}%
\pgfusepath{clip}%
\pgfsetbuttcap%
\pgfsetroundjoin%
\definecolor{currentfill}{rgb}{0.220057,0.343307,0.549413}%
\pgfsetfillcolor{currentfill}%
\pgfsetfillopacity{0.700000}%
\pgfsetlinewidth{0.000000pt}%
\definecolor{currentstroke}{rgb}{0.000000,0.000000,0.000000}%
\pgfsetstrokecolor{currentstroke}%
\pgfsetdash{}{0pt}%
\pgfpathmoveto{\pgfqpoint{3.957757in}{1.745296in}}%
\pgfpathlineto{\pgfqpoint{3.972135in}{1.736441in}}%
\pgfpathlineto{\pgfqpoint{3.986518in}{1.727609in}}%
\pgfpathlineto{\pgfqpoint{4.000905in}{1.718803in}}%
\pgfpathlineto{\pgfqpoint{4.015296in}{1.710021in}}%
\pgfpathlineto{\pgfqpoint{4.006923in}{1.724181in}}%
\pgfpathlineto{\pgfqpoint{3.998534in}{1.739040in}}%
\pgfpathlineto{\pgfqpoint{3.990131in}{1.754612in}}%
\pgfpathlineto{\pgfqpoint{3.981712in}{1.770911in}}%
\pgfpathlineto{\pgfqpoint{3.967276in}{1.780162in}}%
\pgfpathlineto{\pgfqpoint{3.952844in}{1.789437in}}%
\pgfpathlineto{\pgfqpoint{3.938416in}{1.798737in}}%
\pgfpathlineto{\pgfqpoint{3.923993in}{1.808062in}}%
\pgfpathlineto{\pgfqpoint{3.932458in}{1.791287in}}%
\pgfpathlineto{\pgfqpoint{3.940907in}{1.775244in}}%
\pgfpathlineto{\pgfqpoint{3.949340in}{1.759918in}}%
\pgfpathlineto{\pgfqpoint{3.957757in}{1.745296in}}%
\pgfpathclose%
\pgfusepath{fill}%
\end{pgfscope}%
\begin{pgfscope}%
\pgfpathrectangle{\pgfqpoint{1.150000in}{0.150000in}}{\pgfqpoint{5.700000in}{5.700000in}}%
\pgfusepath{clip}%
\pgfsetbuttcap%
\pgfsetroundjoin%
\definecolor{currentfill}{rgb}{0.945636,0.899815,0.112838}%
\pgfsetfillcolor{currentfill}%
\pgfsetfillopacity{0.700000}%
\pgfsetlinewidth{0.000000pt}%
\definecolor{currentstroke}{rgb}{0.000000,0.000000,0.000000}%
\pgfsetstrokecolor{currentstroke}%
\pgfsetdash{}{0pt}%
\pgfpathmoveto{\pgfqpoint{1.900540in}{3.657932in}}%
\pgfpathlineto{\pgfqpoint{1.914873in}{3.642821in}}%
\pgfpathlineto{\pgfqpoint{1.929204in}{3.627766in}}%
\pgfpathlineto{\pgfqpoint{1.943534in}{3.612767in}}%
\pgfpathlineto{\pgfqpoint{1.957863in}{3.597823in}}%
\pgfpathlineto{\pgfqpoint{1.946758in}{3.639760in}}%
\pgfpathlineto{\pgfqpoint{1.935585in}{3.682783in}}%
\pgfpathlineto{\pgfqpoint{1.924343in}{3.726912in}}%
\pgfpathlineto{\pgfqpoint{1.909943in}{3.742291in}}%
\pgfpathlineto{\pgfqpoint{1.895541in}{3.757727in}}%
\pgfpathlineto{\pgfqpoint{1.881137in}{3.773218in}}%
\pgfpathlineto{\pgfqpoint{1.866732in}{3.788767in}}%
\pgfpathlineto{\pgfqpoint{1.878072in}{3.744049in}}%
\pgfpathlineto{\pgfqpoint{1.889340in}{3.700444in}}%
\pgfpathlineto{\pgfqpoint{1.900540in}{3.657932in}}%
\pgfpathclose%
\pgfusepath{fill}%
\end{pgfscope}%
\begin{pgfscope}%
\pgfpathrectangle{\pgfqpoint{1.150000in}{0.150000in}}{\pgfqpoint{5.700000in}{5.700000in}}%
\pgfusepath{clip}%
\pgfsetbuttcap%
\pgfsetroundjoin%
\definecolor{currentfill}{rgb}{0.886271,0.892374,0.095374}%
\pgfsetfillcolor{currentfill}%
\pgfsetfillopacity{0.700000}%
\pgfsetlinewidth{0.000000pt}%
\definecolor{currentstroke}{rgb}{0.000000,0.000000,0.000000}%
\pgfsetstrokecolor{currentstroke}%
\pgfsetdash{}{0pt}%
\pgfpathmoveto{\pgfqpoint{1.957863in}{3.597823in}}%
\pgfpathlineto{\pgfqpoint{1.972190in}{3.582933in}}%
\pgfpathlineto{\pgfqpoint{1.986516in}{3.568098in}}%
\pgfpathlineto{\pgfqpoint{2.000841in}{3.553315in}}%
\pgfpathlineto{\pgfqpoint{2.015165in}{3.538586in}}%
\pgfpathlineto{\pgfqpoint{2.004153in}{3.579951in}}%
\pgfpathlineto{\pgfqpoint{1.993075in}{3.622394in}}%
\pgfpathlineto{\pgfqpoint{1.981929in}{3.665937in}}%
\pgfpathlineto{\pgfqpoint{1.967535in}{3.681100in}}%
\pgfpathlineto{\pgfqpoint{1.953139in}{3.696317in}}%
\pgfpathlineto{\pgfqpoint{1.938742in}{3.711587in}}%
\pgfpathlineto{\pgfqpoint{1.924343in}{3.726912in}}%
\pgfpathlineto{\pgfqpoint{1.935585in}{3.682783in}}%
\pgfpathlineto{\pgfqpoint{1.946758in}{3.639760in}}%
\pgfpathlineto{\pgfqpoint{1.957863in}{3.597823in}}%
\pgfpathclose%
\pgfusepath{fill}%
\end{pgfscope}%
\begin{pgfscope}%
\pgfpathrectangle{\pgfqpoint{1.150000in}{0.150000in}}{\pgfqpoint{5.700000in}{5.700000in}}%
\pgfusepath{clip}%
\pgfsetbuttcap%
\pgfsetroundjoin%
\definecolor{currentfill}{rgb}{0.124780,0.640461,0.527068}%
\pgfsetfillcolor{currentfill}%
\pgfsetfillopacity{0.700000}%
\pgfsetlinewidth{0.000000pt}%
\definecolor{currentstroke}{rgb}{0.000000,0.000000,0.000000}%
\pgfsetstrokecolor{currentstroke}%
\pgfsetdash{}{0pt}%
\pgfpathmoveto{\pgfqpoint{2.970318in}{2.566228in}}%
\pgfpathlineto{\pgfqpoint{2.984605in}{2.554642in}}%
\pgfpathlineto{\pgfqpoint{2.998893in}{2.543088in}}%
\pgfpathlineto{\pgfqpoint{3.013184in}{2.531566in}}%
\pgfpathlineto{\pgfqpoint{3.027477in}{2.520074in}}%
\pgfpathlineto{\pgfqpoint{3.018022in}{2.548336in}}%
\pgfpathlineto{\pgfqpoint{3.008529in}{2.577500in}}%
\pgfpathlineto{\pgfqpoint{2.998998in}{2.607583in}}%
\pgfpathlineto{\pgfqpoint{2.989428in}{2.638602in}}%
\pgfpathlineto{\pgfqpoint{2.975066in}{2.650617in}}%
\pgfpathlineto{\pgfqpoint{2.960705in}{2.662663in}}%
\pgfpathlineto{\pgfqpoint{2.946346in}{2.674741in}}%
\pgfpathlineto{\pgfqpoint{2.931989in}{2.686851in}}%
\pgfpathlineto{\pgfqpoint{2.941631in}{2.655300in}}%
\pgfpathlineto{\pgfqpoint{2.951233in}{2.624690in}}%
\pgfpathlineto{\pgfqpoint{2.960795in}{2.595005in}}%
\pgfpathlineto{\pgfqpoint{2.970318in}{2.566228in}}%
\pgfpathclose%
\pgfusepath{fill}%
\end{pgfscope}%
\begin{pgfscope}%
\pgfpathrectangle{\pgfqpoint{1.150000in}{0.150000in}}{\pgfqpoint{5.700000in}{5.700000in}}%
\pgfusepath{clip}%
\pgfsetbuttcap%
\pgfsetroundjoin%
\definecolor{currentfill}{rgb}{0.283187,0.125848,0.444960}%
\pgfsetfillcolor{currentfill}%
\pgfsetfillopacity{0.700000}%
\pgfsetlinewidth{0.000000pt}%
\definecolor{currentstroke}{rgb}{0.000000,0.000000,0.000000}%
\pgfsetstrokecolor{currentstroke}%
\pgfsetdash{}{0pt}%
\pgfpathmoveto{\pgfqpoint{4.806425in}{1.258604in}}%
\pgfpathlineto{\pgfqpoint{4.820977in}{1.252275in}}%
\pgfpathlineto{\pgfqpoint{4.835535in}{1.245968in}}%
\pgfpathlineto{\pgfqpoint{4.850099in}{1.239685in}}%
\pgfpathlineto{\pgfqpoint{4.864669in}{1.233424in}}%
\pgfpathlineto{\pgfqpoint{4.856805in}{1.234892in}}%
\pgfpathlineto{\pgfqpoint{4.848940in}{1.236855in}}%
\pgfpathlineto{\pgfqpoint{4.841072in}{1.239323in}}%
\pgfpathlineto{\pgfqpoint{4.833203in}{1.242308in}}%
\pgfpathlineto{\pgfqpoint{4.818609in}{1.248984in}}%
\pgfpathlineto{\pgfqpoint{4.804022in}{1.255683in}}%
\pgfpathlineto{\pgfqpoint{4.789441in}{1.262405in}}%
\pgfpathlineto{\pgfqpoint{4.774865in}{1.269150in}}%
\pgfpathlineto{\pgfqpoint{4.782759in}{1.265743in}}%
\pgfpathlineto{\pgfqpoint{4.790650in}{1.262858in}}%
\pgfpathlineto{\pgfqpoint{4.798539in}{1.260482in}}%
\pgfpathlineto{\pgfqpoint{4.806425in}{1.258604in}}%
\pgfpathclose%
\pgfusepath{fill}%
\end{pgfscope}%
\begin{pgfscope}%
\pgfpathrectangle{\pgfqpoint{1.150000in}{0.150000in}}{\pgfqpoint{5.700000in}{5.700000in}}%
\pgfusepath{clip}%
\pgfsetbuttcap%
\pgfsetroundjoin%
\definecolor{currentfill}{rgb}{0.835270,0.886029,0.102646}%
\pgfsetfillcolor{currentfill}%
\pgfsetfillopacity{0.700000}%
\pgfsetlinewidth{0.000000pt}%
\definecolor{currentstroke}{rgb}{0.000000,0.000000,0.000000}%
\pgfsetstrokecolor{currentstroke}%
\pgfsetdash{}{0pt}%
\pgfpathmoveto{\pgfqpoint{2.015165in}{3.538586in}}%
\pgfpathlineto{\pgfqpoint{2.029488in}{3.523909in}}%
\pgfpathlineto{\pgfqpoint{2.043810in}{3.509283in}}%
\pgfpathlineto{\pgfqpoint{2.058131in}{3.494709in}}%
\pgfpathlineto{\pgfqpoint{2.072452in}{3.480186in}}%
\pgfpathlineto{\pgfqpoint{2.061531in}{3.520981in}}%
\pgfpathlineto{\pgfqpoint{2.050547in}{3.562849in}}%
\pgfpathlineto{\pgfqpoint{2.039496in}{3.605808in}}%
\pgfpathlineto{\pgfqpoint{2.025106in}{3.620763in}}%
\pgfpathlineto{\pgfqpoint{2.010715in}{3.635769in}}%
\pgfpathlineto{\pgfqpoint{1.996323in}{3.650827in}}%
\pgfpathlineto{\pgfqpoint{1.981929in}{3.665937in}}%
\pgfpathlineto{\pgfqpoint{1.993075in}{3.622394in}}%
\pgfpathlineto{\pgfqpoint{2.004153in}{3.579951in}}%
\pgfpathlineto{\pgfqpoint{2.015165in}{3.538586in}}%
\pgfpathclose%
\pgfusepath{fill}%
\end{pgfscope}%
\begin{pgfscope}%
\pgfpathrectangle{\pgfqpoint{1.150000in}{0.150000in}}{\pgfqpoint{5.700000in}{5.700000in}}%
\pgfusepath{clip}%
\pgfsetbuttcap%
\pgfsetroundjoin%
\definecolor{currentfill}{rgb}{0.169646,0.456262,0.558030}%
\pgfsetfillcolor{currentfill}%
\pgfsetfillopacity{0.700000}%
\pgfsetlinewidth{0.000000pt}%
\definecolor{currentstroke}{rgb}{0.000000,0.000000,0.000000}%
\pgfsetstrokecolor{currentstroke}%
\pgfsetdash{}{0pt}%
\pgfpathmoveto{\pgfqpoint{3.578960in}{2.039467in}}%
\pgfpathlineto{\pgfqpoint{3.593295in}{2.029528in}}%
\pgfpathlineto{\pgfqpoint{3.607633in}{2.019614in}}%
\pgfpathlineto{\pgfqpoint{3.621975in}{2.009728in}}%
\pgfpathlineto{\pgfqpoint{3.636320in}{1.999867in}}%
\pgfpathlineto{\pgfqpoint{3.627585in}{2.019801in}}%
\pgfpathlineto{\pgfqpoint{3.618827in}{2.040521in}}%
\pgfpathlineto{\pgfqpoint{3.610046in}{2.062041in}}%
\pgfpathlineto{\pgfqpoint{3.601240in}{2.084378in}}%
\pgfpathlineto{\pgfqpoint{3.586840in}{2.094730in}}%
\pgfpathlineto{\pgfqpoint{3.572444in}{2.105109in}}%
\pgfpathlineto{\pgfqpoint{3.558050in}{2.115514in}}%
\pgfpathlineto{\pgfqpoint{3.543659in}{2.125946in}}%
\pgfpathlineto{\pgfqpoint{3.552521in}{2.103110in}}%
\pgfpathlineto{\pgfqpoint{3.561358in}{2.081095in}}%
\pgfpathlineto{\pgfqpoint{3.570171in}{2.059886in}}%
\pgfpathlineto{\pgfqpoint{3.578960in}{2.039467in}}%
\pgfpathclose%
\pgfusepath{fill}%
\end{pgfscope}%
\begin{pgfscope}%
\pgfpathrectangle{\pgfqpoint{1.150000in}{0.150000in}}{\pgfqpoint{5.700000in}{5.700000in}}%
\pgfusepath{clip}%
\pgfsetbuttcap%
\pgfsetroundjoin%
\definecolor{currentfill}{rgb}{0.783315,0.879285,0.125405}%
\pgfsetfillcolor{currentfill}%
\pgfsetfillopacity{0.700000}%
\pgfsetlinewidth{0.000000pt}%
\definecolor{currentstroke}{rgb}{0.000000,0.000000,0.000000}%
\pgfsetstrokecolor{currentstroke}%
\pgfsetdash{}{0pt}%
\pgfpathmoveto{\pgfqpoint{2.072452in}{3.480186in}}%
\pgfpathlineto{\pgfqpoint{2.086771in}{3.465712in}}%
\pgfpathlineto{\pgfqpoint{2.101090in}{3.451289in}}%
\pgfpathlineto{\pgfqpoint{2.115408in}{3.436915in}}%
\pgfpathlineto{\pgfqpoint{2.129725in}{3.422590in}}%
\pgfpathlineto{\pgfqpoint{2.118896in}{3.462818in}}%
\pgfpathlineto{\pgfqpoint{2.108003in}{3.504112in}}%
\pgfpathlineto{\pgfqpoint{2.097046in}{3.546491in}}%
\pgfpathlineto{\pgfqpoint{2.082660in}{3.561246in}}%
\pgfpathlineto{\pgfqpoint{2.068273in}{3.576050in}}%
\pgfpathlineto{\pgfqpoint{2.053885in}{3.590904in}}%
\pgfpathlineto{\pgfqpoint{2.039496in}{3.605808in}}%
\pgfpathlineto{\pgfqpoint{2.050547in}{3.562849in}}%
\pgfpathlineto{\pgfqpoint{2.061531in}{3.520981in}}%
\pgfpathlineto{\pgfqpoint{2.072452in}{3.480186in}}%
\pgfpathclose%
\pgfusepath{fill}%
\end{pgfscope}%
\begin{pgfscope}%
\pgfpathrectangle{\pgfqpoint{1.150000in}{0.150000in}}{\pgfqpoint{5.700000in}{5.700000in}}%
\pgfusepath{clip}%
\pgfsetbuttcap%
\pgfsetroundjoin%
\definecolor{currentfill}{rgb}{0.278826,0.175490,0.483397}%
\pgfsetfillcolor{currentfill}%
\pgfsetfillopacity{0.700000}%
\pgfsetlinewidth{0.000000pt}%
\definecolor{currentstroke}{rgb}{0.000000,0.000000,0.000000}%
\pgfsetstrokecolor{currentstroke}%
\pgfsetdash{}{0pt}%
\pgfpathmoveto{\pgfqpoint{4.600417in}{1.351881in}}%
\pgfpathlineto{\pgfqpoint{4.614923in}{1.344860in}}%
\pgfpathlineto{\pgfqpoint{4.629434in}{1.337862in}}%
\pgfpathlineto{\pgfqpoint{4.643951in}{1.330887in}}%
\pgfpathlineto{\pgfqpoint{4.658474in}{1.323935in}}%
\pgfpathlineto{\pgfqpoint{4.650524in}{1.328725in}}%
\pgfpathlineto{\pgfqpoint{4.642570in}{1.334068in}}%
\pgfpathlineto{\pgfqpoint{4.634612in}{1.339976in}}%
\pgfpathlineto{\pgfqpoint{4.626648in}{1.346461in}}%
\pgfpathlineto{\pgfqpoint{4.612097in}{1.353844in}}%
\pgfpathlineto{\pgfqpoint{4.597551in}{1.361250in}}%
\pgfpathlineto{\pgfqpoint{4.583011in}{1.368680in}}%
\pgfpathlineto{\pgfqpoint{4.568476in}{1.376132in}}%
\pgfpathlineto{\pgfqpoint{4.576470in}{1.369210in}}%
\pgfpathlineto{\pgfqpoint{4.584457in}{1.362869in}}%
\pgfpathlineto{\pgfqpoint{4.592440in}{1.357097in}}%
\pgfpathlineto{\pgfqpoint{4.600417in}{1.351881in}}%
\pgfpathclose%
\pgfusepath{fill}%
\end{pgfscope}%
\begin{pgfscope}%
\pgfpathrectangle{\pgfqpoint{1.150000in}{0.150000in}}{\pgfqpoint{5.700000in}{5.700000in}}%
\pgfusepath{clip}%
\pgfsetbuttcap%
\pgfsetroundjoin%
\definecolor{currentfill}{rgb}{0.121380,0.629492,0.531973}%
\pgfsetfillcolor{currentfill}%
\pgfsetfillopacity{0.700000}%
\pgfsetlinewidth{0.000000pt}%
\definecolor{currentstroke}{rgb}{0.000000,0.000000,0.000000}%
\pgfsetstrokecolor{currentstroke}%
\pgfsetdash{}{0pt}%
\pgfpathmoveto{\pgfqpoint{3.027477in}{2.520074in}}%
\pgfpathlineto{\pgfqpoint{3.041772in}{2.508614in}}%
\pgfpathlineto{\pgfqpoint{3.056069in}{2.497184in}}%
\pgfpathlineto{\pgfqpoint{3.070368in}{2.485785in}}%
\pgfpathlineto{\pgfqpoint{3.084670in}{2.474416in}}%
\pgfpathlineto{\pgfqpoint{3.075282in}{2.502164in}}%
\pgfpathlineto{\pgfqpoint{3.065857in}{2.530808in}}%
\pgfpathlineto{\pgfqpoint{3.056396in}{2.560365in}}%
\pgfpathlineto{\pgfqpoint{3.046897in}{2.590853in}}%
\pgfpathlineto{\pgfqpoint{3.032527in}{2.602744in}}%
\pgfpathlineto{\pgfqpoint{3.018159in}{2.614666in}}%
\pgfpathlineto{\pgfqpoint{3.003792in}{2.626618in}}%
\pgfpathlineto{\pgfqpoint{2.989428in}{2.638602in}}%
\pgfpathlineto{\pgfqpoint{2.998998in}{2.607583in}}%
\pgfpathlineto{\pgfqpoint{3.008529in}{2.577500in}}%
\pgfpathlineto{\pgfqpoint{3.018022in}{2.548336in}}%
\pgfpathlineto{\pgfqpoint{3.027477in}{2.520074in}}%
\pgfpathclose%
\pgfusepath{fill}%
\end{pgfscope}%
\begin{pgfscope}%
\pgfpathrectangle{\pgfqpoint{1.150000in}{0.150000in}}{\pgfqpoint{5.700000in}{5.700000in}}%
\pgfusepath{clip}%
\pgfsetbuttcap%
\pgfsetroundjoin%
\definecolor{currentfill}{rgb}{0.262138,0.242286,0.520837}%
\pgfsetfillcolor{currentfill}%
\pgfsetfillopacity{0.700000}%
\pgfsetlinewidth{0.000000pt}%
\definecolor{currentstroke}{rgb}{0.000000,0.000000,0.000000}%
\pgfsetstrokecolor{currentstroke}%
\pgfsetdash{}{0pt}%
\pgfpathmoveto{\pgfqpoint{4.336640in}{1.498546in}}%
\pgfpathlineto{\pgfqpoint{4.351091in}{1.490719in}}%
\pgfpathlineto{\pgfqpoint{4.365547in}{1.482916in}}%
\pgfpathlineto{\pgfqpoint{4.380008in}{1.475136in}}%
\pgfpathlineto{\pgfqpoint{4.394474in}{1.467380in}}%
\pgfpathlineto{\pgfqpoint{4.386378in}{1.476222in}}%
\pgfpathlineto{\pgfqpoint{4.378272in}{1.485683in}}%
\pgfpathlineto{\pgfqpoint{4.370159in}{1.495776in}}%
\pgfpathlineto{\pgfqpoint{4.362036in}{1.506515in}}%
\pgfpathlineto{\pgfqpoint{4.347535in}{1.514720in}}%
\pgfpathlineto{\pgfqpoint{4.333038in}{1.522948in}}%
\pgfpathlineto{\pgfqpoint{4.318546in}{1.531200in}}%
\pgfpathlineto{\pgfqpoint{4.304059in}{1.539476in}}%
\pgfpathlineto{\pgfqpoint{4.312219in}{1.528282in}}%
\pgfpathlineto{\pgfqpoint{4.320368in}{1.517738in}}%
\pgfpathlineto{\pgfqpoint{4.328509in}{1.507830in}}%
\pgfpathlineto{\pgfqpoint{4.336640in}{1.498546in}}%
\pgfpathclose%
\pgfusepath{fill}%
\end{pgfscope}%
\begin{pgfscope}%
\pgfpathrectangle{\pgfqpoint{1.150000in}{0.150000in}}{\pgfqpoint{5.700000in}{5.700000in}}%
\pgfusepath{clip}%
\pgfsetbuttcap%
\pgfsetroundjoin%
\definecolor{currentfill}{rgb}{0.730889,0.871916,0.156029}%
\pgfsetfillcolor{currentfill}%
\pgfsetfillopacity{0.700000}%
\pgfsetlinewidth{0.000000pt}%
\definecolor{currentstroke}{rgb}{0.000000,0.000000,0.000000}%
\pgfsetstrokecolor{currentstroke}%
\pgfsetdash{}{0pt}%
\pgfpathmoveto{\pgfqpoint{2.129725in}{3.422590in}}%
\pgfpathlineto{\pgfqpoint{2.144042in}{3.408313in}}%
\pgfpathlineto{\pgfqpoint{2.158359in}{3.394084in}}%
\pgfpathlineto{\pgfqpoint{2.172675in}{3.379902in}}%
\pgfpathlineto{\pgfqpoint{2.186990in}{3.365768in}}%
\pgfpathlineto{\pgfqpoint{2.176250in}{3.405432in}}%
\pgfpathlineto{\pgfqpoint{2.165448in}{3.446154in}}%
\pgfpathlineto{\pgfqpoint{2.154584in}{3.487956in}}%
\pgfpathlineto{\pgfqpoint{2.140200in}{3.502518in}}%
\pgfpathlineto{\pgfqpoint{2.125816in}{3.517128in}}%
\pgfpathlineto{\pgfqpoint{2.111432in}{3.531785in}}%
\pgfpathlineto{\pgfqpoint{2.097046in}{3.546491in}}%
\pgfpathlineto{\pgfqpoint{2.108003in}{3.504112in}}%
\pgfpathlineto{\pgfqpoint{2.118896in}{3.462818in}}%
\pgfpathlineto{\pgfqpoint{2.129725in}{3.422590in}}%
\pgfpathclose%
\pgfusepath{fill}%
\end{pgfscope}%
\begin{pgfscope}%
\pgfpathrectangle{\pgfqpoint{1.150000in}{0.150000in}}{\pgfqpoint{5.700000in}{5.700000in}}%
\pgfusepath{clip}%
\pgfsetbuttcap%
\pgfsetroundjoin%
\definecolor{currentfill}{rgb}{0.223925,0.334994,0.548053}%
\pgfsetfillcolor{currentfill}%
\pgfsetfillopacity{0.700000}%
\pgfsetlinewidth{0.000000pt}%
\definecolor{currentstroke}{rgb}{0.000000,0.000000,0.000000}%
\pgfsetstrokecolor{currentstroke}%
\pgfsetdash{}{0pt}%
\pgfpathmoveto{\pgfqpoint{4.015296in}{1.710021in}}%
\pgfpathlineto{\pgfqpoint{4.029691in}{1.701263in}}%
\pgfpathlineto{\pgfqpoint{4.044091in}{1.692530in}}%
\pgfpathlineto{\pgfqpoint{4.058495in}{1.683822in}}%
\pgfpathlineto{\pgfqpoint{4.072904in}{1.675138in}}%
\pgfpathlineto{\pgfqpoint{4.064573in}{1.688836in}}%
\pgfpathlineto{\pgfqpoint{4.056229in}{1.703229in}}%
\pgfpathlineto{\pgfqpoint{4.047870in}{1.718331in}}%
\pgfpathlineto{\pgfqpoint{4.039496in}{1.734155in}}%
\pgfpathlineto{\pgfqpoint{4.025044in}{1.743307in}}%
\pgfpathlineto{\pgfqpoint{4.010596in}{1.752484in}}%
\pgfpathlineto{\pgfqpoint{3.996152in}{1.761685in}}%
\pgfpathlineto{\pgfqpoint{3.981712in}{1.770911in}}%
\pgfpathlineto{\pgfqpoint{3.990131in}{1.754612in}}%
\pgfpathlineto{\pgfqpoint{3.998534in}{1.739040in}}%
\pgfpathlineto{\pgfqpoint{4.006923in}{1.724181in}}%
\pgfpathlineto{\pgfqpoint{4.015296in}{1.710021in}}%
\pgfpathclose%
\pgfusepath{fill}%
\end{pgfscope}%
\begin{pgfscope}%
\pgfpathrectangle{\pgfqpoint{1.150000in}{0.150000in}}{\pgfqpoint{5.700000in}{5.700000in}}%
\pgfusepath{clip}%
\pgfsetbuttcap%
\pgfsetroundjoin%
\definecolor{currentfill}{rgb}{0.678489,0.863742,0.189503}%
\pgfsetfillcolor{currentfill}%
\pgfsetfillopacity{0.700000}%
\pgfsetlinewidth{0.000000pt}%
\definecolor{currentstroke}{rgb}{0.000000,0.000000,0.000000}%
\pgfsetstrokecolor{currentstroke}%
\pgfsetdash{}{0pt}%
\pgfpathmoveto{\pgfqpoint{2.186990in}{3.365768in}}%
\pgfpathlineto{\pgfqpoint{2.201305in}{3.351680in}}%
\pgfpathlineto{\pgfqpoint{2.215620in}{3.337638in}}%
\pgfpathlineto{\pgfqpoint{2.229935in}{3.323642in}}%
\pgfpathlineto{\pgfqpoint{2.244249in}{3.309691in}}%
\pgfpathlineto{\pgfqpoint{2.233597in}{3.348793in}}%
\pgfpathlineto{\pgfqpoint{2.222885in}{3.388947in}}%
\pgfpathlineto{\pgfqpoint{2.212112in}{3.430173in}}%
\pgfpathlineto{\pgfqpoint{2.197731in}{3.444550in}}%
\pgfpathlineto{\pgfqpoint{2.183349in}{3.458972in}}%
\pgfpathlineto{\pgfqpoint{2.168967in}{3.473441in}}%
\pgfpathlineto{\pgfqpoint{2.154584in}{3.487956in}}%
\pgfpathlineto{\pgfqpoint{2.165448in}{3.446154in}}%
\pgfpathlineto{\pgfqpoint{2.176250in}{3.405432in}}%
\pgfpathlineto{\pgfqpoint{2.186990in}{3.365768in}}%
\pgfpathclose%
\pgfusepath{fill}%
\end{pgfscope}%
\begin{pgfscope}%
\pgfpathrectangle{\pgfqpoint{1.150000in}{0.150000in}}{\pgfqpoint{5.700000in}{5.700000in}}%
\pgfusepath{clip}%
\pgfsetbuttcap%
\pgfsetroundjoin%
\definecolor{currentfill}{rgb}{0.174274,0.445044,0.557792}%
\pgfsetfillcolor{currentfill}%
\pgfsetfillopacity{0.700000}%
\pgfsetlinewidth{0.000000pt}%
\definecolor{currentstroke}{rgb}{0.000000,0.000000,0.000000}%
\pgfsetstrokecolor{currentstroke}%
\pgfsetdash{}{0pt}%
\pgfpathmoveto{\pgfqpoint{3.636320in}{1.999867in}}%
\pgfpathlineto{\pgfqpoint{3.650669in}{1.990033in}}%
\pgfpathlineto{\pgfqpoint{3.665022in}{1.980224in}}%
\pgfpathlineto{\pgfqpoint{3.679377in}{1.970442in}}%
\pgfpathlineto{\pgfqpoint{3.693737in}{1.960686in}}%
\pgfpathlineto{\pgfqpoint{3.685054in}{1.980136in}}%
\pgfpathlineto{\pgfqpoint{3.676350in}{2.000367in}}%
\pgfpathlineto{\pgfqpoint{3.667623in}{2.021394in}}%
\pgfpathlineto{\pgfqpoint{3.658873in}{2.043232in}}%
\pgfpathlineto{\pgfqpoint{3.644460in}{2.053479in}}%
\pgfpathlineto{\pgfqpoint{3.630050in}{2.063753in}}%
\pgfpathlineto{\pgfqpoint{3.615644in}{2.074052in}}%
\pgfpathlineto{\pgfqpoint{3.601240in}{2.084378in}}%
\pgfpathlineto{\pgfqpoint{3.610046in}{2.062041in}}%
\pgfpathlineto{\pgfqpoint{3.618827in}{2.040521in}}%
\pgfpathlineto{\pgfqpoint{3.627585in}{2.019801in}}%
\pgfpathlineto{\pgfqpoint{3.636320in}{1.999867in}}%
\pgfpathclose%
\pgfusepath{fill}%
\end{pgfscope}%
\begin{pgfscope}%
\pgfpathrectangle{\pgfqpoint{1.150000in}{0.150000in}}{\pgfqpoint{5.700000in}{5.700000in}}%
\pgfusepath{clip}%
\pgfsetbuttcap%
\pgfsetroundjoin%
\definecolor{currentfill}{rgb}{0.119483,0.614817,0.537692}%
\pgfsetfillcolor{currentfill}%
\pgfsetfillopacity{0.700000}%
\pgfsetlinewidth{0.000000pt}%
\definecolor{currentstroke}{rgb}{0.000000,0.000000,0.000000}%
\pgfsetstrokecolor{currentstroke}%
\pgfsetdash{}{0pt}%
\pgfpathmoveto{\pgfqpoint{3.084670in}{2.474416in}}%
\pgfpathlineto{\pgfqpoint{3.098973in}{2.463078in}}%
\pgfpathlineto{\pgfqpoint{3.113279in}{2.451770in}}%
\pgfpathlineto{\pgfqpoint{3.127588in}{2.440492in}}%
\pgfpathlineto{\pgfqpoint{3.141898in}{2.429244in}}%
\pgfpathlineto{\pgfqpoint{3.132576in}{2.456479in}}%
\pgfpathlineto{\pgfqpoint{3.123220in}{2.484604in}}%
\pgfpathlineto{\pgfqpoint{3.113827in}{2.513638in}}%
\pgfpathlineto{\pgfqpoint{3.104398in}{2.543596in}}%
\pgfpathlineto{\pgfqpoint{3.090020in}{2.555365in}}%
\pgfpathlineto{\pgfqpoint{3.075644in}{2.567164in}}%
\pgfpathlineto{\pgfqpoint{3.061269in}{2.578993in}}%
\pgfpathlineto{\pgfqpoint{3.046897in}{2.590853in}}%
\pgfpathlineto{\pgfqpoint{3.056396in}{2.560365in}}%
\pgfpathlineto{\pgfqpoint{3.065857in}{2.530808in}}%
\pgfpathlineto{\pgfqpoint{3.075282in}{2.502164in}}%
\pgfpathlineto{\pgfqpoint{3.084670in}{2.474416in}}%
\pgfpathclose%
\pgfusepath{fill}%
\end{pgfscope}%
\begin{pgfscope}%
\pgfpathrectangle{\pgfqpoint{1.150000in}{0.150000in}}{\pgfqpoint{5.700000in}{5.700000in}}%
\pgfusepath{clip}%
\pgfsetbuttcap%
\pgfsetroundjoin%
\definecolor{currentfill}{rgb}{0.283229,0.120777,0.440584}%
\pgfsetfillcolor{currentfill}%
\pgfsetfillopacity{0.700000}%
\pgfsetlinewidth{0.000000pt}%
\definecolor{currentstroke}{rgb}{0.000000,0.000000,0.000000}%
\pgfsetstrokecolor{currentstroke}%
\pgfsetdash{}{0pt}%
\pgfpathmoveto{\pgfqpoint{4.864669in}{1.233424in}}%
\pgfpathlineto{\pgfqpoint{4.879246in}{1.227187in}}%
\pgfpathlineto{\pgfqpoint{4.893829in}{1.220972in}}%
\pgfpathlineto{\pgfqpoint{4.908418in}{1.214780in}}%
\pgfpathlineto{\pgfqpoint{4.900570in}{1.215940in}}%
\pgfpathlineto{\pgfqpoint{4.892722in}{1.217593in}}%
\pgfpathlineto{\pgfqpoint{4.884871in}{1.219748in}}%
\pgfpathlineto{\pgfqpoint{4.877019in}{1.222417in}}%
\pgfpathlineto{\pgfqpoint{4.862408in}{1.229024in}}%
\pgfpathlineto{\pgfqpoint{4.847802in}{1.235655in}}%
\pgfpathlineto{\pgfqpoint{4.833203in}{1.242308in}}%
\pgfpathlineto{\pgfqpoint{4.841072in}{1.239323in}}%
\pgfpathlineto{\pgfqpoint{4.848940in}{1.236855in}}%
\pgfpathlineto{\pgfqpoint{4.856805in}{1.234892in}}%
\pgfpathlineto{\pgfqpoint{4.864669in}{1.233424in}}%
\pgfpathclose%
\pgfusepath{fill}%
\end{pgfscope}%
\begin{pgfscope}%
\pgfpathrectangle{\pgfqpoint{1.150000in}{0.150000in}}{\pgfqpoint{5.700000in}{5.700000in}}%
\pgfusepath{clip}%
\pgfsetbuttcap%
\pgfsetroundjoin%
\definecolor{currentfill}{rgb}{0.626579,0.854645,0.223353}%
\pgfsetfillcolor{currentfill}%
\pgfsetfillopacity{0.700000}%
\pgfsetlinewidth{0.000000pt}%
\definecolor{currentstroke}{rgb}{0.000000,0.000000,0.000000}%
\pgfsetstrokecolor{currentstroke}%
\pgfsetdash{}{0pt}%
\pgfpathmoveto{\pgfqpoint{2.244249in}{3.309691in}}%
\pgfpathlineto{\pgfqpoint{2.258564in}{3.295785in}}%
\pgfpathlineto{\pgfqpoint{2.272878in}{3.281924in}}%
\pgfpathlineto{\pgfqpoint{2.287192in}{3.268107in}}%
\pgfpathlineto{\pgfqpoint{2.301507in}{3.254333in}}%
\pgfpathlineto{\pgfqpoint{2.290942in}{3.292875in}}%
\pgfpathlineto{\pgfqpoint{2.280318in}{3.332463in}}%
\pgfpathlineto{\pgfqpoint{2.269635in}{3.373116in}}%
\pgfpathlineto{\pgfqpoint{2.255255in}{3.387313in}}%
\pgfpathlineto{\pgfqpoint{2.240874in}{3.401555in}}%
\pgfpathlineto{\pgfqpoint{2.226493in}{3.415842in}}%
\pgfpathlineto{\pgfqpoint{2.212112in}{3.430173in}}%
\pgfpathlineto{\pgfqpoint{2.222885in}{3.388947in}}%
\pgfpathlineto{\pgfqpoint{2.233597in}{3.348793in}}%
\pgfpathlineto{\pgfqpoint{2.244249in}{3.309691in}}%
\pgfpathclose%
\pgfusepath{fill}%
\end{pgfscope}%
\begin{pgfscope}%
\pgfpathrectangle{\pgfqpoint{1.150000in}{0.150000in}}{\pgfqpoint{5.700000in}{5.700000in}}%
\pgfusepath{clip}%
\pgfsetbuttcap%
\pgfsetroundjoin%
\definecolor{currentfill}{rgb}{0.575563,0.844566,0.256415}%
\pgfsetfillcolor{currentfill}%
\pgfsetfillopacity{0.700000}%
\pgfsetlinewidth{0.000000pt}%
\definecolor{currentstroke}{rgb}{0.000000,0.000000,0.000000}%
\pgfsetstrokecolor{currentstroke}%
\pgfsetdash{}{0pt}%
\pgfpathmoveto{\pgfqpoint{2.301507in}{3.254333in}}%
\pgfpathlineto{\pgfqpoint{2.315821in}{3.240603in}}%
\pgfpathlineto{\pgfqpoint{2.330135in}{3.226916in}}%
\pgfpathlineto{\pgfqpoint{2.344450in}{3.213272in}}%
\pgfpathlineto{\pgfqpoint{2.358765in}{3.199670in}}%
\pgfpathlineto{\pgfqpoint{2.348285in}{3.237654in}}%
\pgfpathlineto{\pgfqpoint{2.337749in}{3.276677in}}%
\pgfpathlineto{\pgfqpoint{2.327155in}{3.316760in}}%
\pgfpathlineto{\pgfqpoint{2.312775in}{3.330784in}}%
\pgfpathlineto{\pgfqpoint{2.298395in}{3.344852in}}%
\pgfpathlineto{\pgfqpoint{2.284015in}{3.358962in}}%
\pgfpathlineto{\pgfqpoint{2.269635in}{3.373116in}}%
\pgfpathlineto{\pgfqpoint{2.280318in}{3.332463in}}%
\pgfpathlineto{\pgfqpoint{2.290942in}{3.292875in}}%
\pgfpathlineto{\pgfqpoint{2.301507in}{3.254333in}}%
\pgfpathclose%
\pgfusepath{fill}%
\end{pgfscope}%
\begin{pgfscope}%
\pgfpathrectangle{\pgfqpoint{1.150000in}{0.150000in}}{\pgfqpoint{5.700000in}{5.700000in}}%
\pgfusepath{clip}%
\pgfsetbuttcap%
\pgfsetroundjoin%
\definecolor{currentfill}{rgb}{0.229739,0.322361,0.545706}%
\pgfsetfillcolor{currentfill}%
\pgfsetfillopacity{0.700000}%
\pgfsetlinewidth{0.000000pt}%
\definecolor{currentstroke}{rgb}{0.000000,0.000000,0.000000}%
\pgfsetstrokecolor{currentstroke}%
\pgfsetdash{}{0pt}%
\pgfpathmoveto{\pgfqpoint{4.072904in}{1.675138in}}%
\pgfpathlineto{\pgfqpoint{4.087317in}{1.666478in}}%
\pgfpathlineto{\pgfqpoint{4.101734in}{1.657843in}}%
\pgfpathlineto{\pgfqpoint{4.116156in}{1.649231in}}%
\pgfpathlineto{\pgfqpoint{4.130583in}{1.640645in}}%
\pgfpathlineto{\pgfqpoint{4.122294in}{1.653882in}}%
\pgfpathlineto{\pgfqpoint{4.113992in}{1.667809in}}%
\pgfpathlineto{\pgfqpoint{4.105677in}{1.682441in}}%
\pgfpathlineto{\pgfqpoint{4.097348in}{1.697791in}}%
\pgfpathlineto{\pgfqpoint{4.082879in}{1.706845in}}%
\pgfpathlineto{\pgfqpoint{4.068414in}{1.715924in}}%
\pgfpathlineto{\pgfqpoint{4.053953in}{1.725027in}}%
\pgfpathlineto{\pgfqpoint{4.039496in}{1.734155in}}%
\pgfpathlineto{\pgfqpoint{4.047870in}{1.718331in}}%
\pgfpathlineto{\pgfqpoint{4.056229in}{1.703229in}}%
\pgfpathlineto{\pgfqpoint{4.064573in}{1.688836in}}%
\pgfpathlineto{\pgfqpoint{4.072904in}{1.675138in}}%
\pgfpathclose%
\pgfusepath{fill}%
\end{pgfscope}%
\begin{pgfscope}%
\pgfpathrectangle{\pgfqpoint{1.150000in}{0.150000in}}{\pgfqpoint{5.700000in}{5.700000in}}%
\pgfusepath{clip}%
\pgfsetbuttcap%
\pgfsetroundjoin%
\definecolor{currentfill}{rgb}{0.265145,0.232956,0.516599}%
\pgfsetfillcolor{currentfill}%
\pgfsetfillopacity{0.700000}%
\pgfsetlinewidth{0.000000pt}%
\definecolor{currentstroke}{rgb}{0.000000,0.000000,0.000000}%
\pgfsetstrokecolor{currentstroke}%
\pgfsetdash{}{0pt}%
\pgfpathmoveto{\pgfqpoint{4.394474in}{1.467380in}}%
\pgfpathlineto{\pgfqpoint{4.408946in}{1.459647in}}%
\pgfpathlineto{\pgfqpoint{4.423422in}{1.451938in}}%
\pgfpathlineto{\pgfqpoint{4.437904in}{1.444252in}}%
\pgfpathlineto{\pgfqpoint{4.452391in}{1.436590in}}%
\pgfpathlineto{\pgfqpoint{4.444327in}{1.444990in}}%
\pgfpathlineto{\pgfqpoint{4.436257in}{1.454005in}}%
\pgfpathlineto{\pgfqpoint{4.428178in}{1.463647in}}%
\pgfpathlineto{\pgfqpoint{4.420092in}{1.473931in}}%
\pgfpathlineto{\pgfqpoint{4.405571in}{1.482042in}}%
\pgfpathlineto{\pgfqpoint{4.391054in}{1.490176in}}%
\pgfpathlineto{\pgfqpoint{4.376543in}{1.498334in}}%
\pgfpathlineto{\pgfqpoint{4.362036in}{1.506515in}}%
\pgfpathlineto{\pgfqpoint{4.370159in}{1.495776in}}%
\pgfpathlineto{\pgfqpoint{4.378272in}{1.485683in}}%
\pgfpathlineto{\pgfqpoint{4.386378in}{1.476222in}}%
\pgfpathlineto{\pgfqpoint{4.394474in}{1.467380in}}%
\pgfpathclose%
\pgfusepath{fill}%
\end{pgfscope}%
\begin{pgfscope}%
\pgfpathrectangle{\pgfqpoint{1.150000in}{0.150000in}}{\pgfqpoint{5.700000in}{5.700000in}}%
\pgfusepath{clip}%
\pgfsetbuttcap%
\pgfsetroundjoin%
\definecolor{currentfill}{rgb}{0.280255,0.165693,0.476498}%
\pgfsetfillcolor{currentfill}%
\pgfsetfillopacity{0.700000}%
\pgfsetlinewidth{0.000000pt}%
\definecolor{currentstroke}{rgb}{0.000000,0.000000,0.000000}%
\pgfsetstrokecolor{currentstroke}%
\pgfsetdash{}{0pt}%
\pgfpathmoveto{\pgfqpoint{4.658474in}{1.323935in}}%
\pgfpathlineto{\pgfqpoint{4.673003in}{1.317007in}}%
\pgfpathlineto{\pgfqpoint{4.687537in}{1.310101in}}%
\pgfpathlineto{\pgfqpoint{4.702077in}{1.303218in}}%
\pgfpathlineto{\pgfqpoint{4.716623in}{1.296359in}}%
\pgfpathlineto{\pgfqpoint{4.708700in}{1.300723in}}%
\pgfpathlineto{\pgfqpoint{4.700774in}{1.305637in}}%
\pgfpathlineto{\pgfqpoint{4.692843in}{1.311111in}}%
\pgfpathlineto{\pgfqpoint{4.684909in}{1.317159in}}%
\pgfpathlineto{\pgfqpoint{4.670336in}{1.324450in}}%
\pgfpathlineto{\pgfqpoint{4.655768in}{1.331764in}}%
\pgfpathlineto{\pgfqpoint{4.641205in}{1.339101in}}%
\pgfpathlineto{\pgfqpoint{4.626648in}{1.346461in}}%
\pgfpathlineto{\pgfqpoint{4.634612in}{1.339976in}}%
\pgfpathlineto{\pgfqpoint{4.642570in}{1.334068in}}%
\pgfpathlineto{\pgfqpoint{4.650524in}{1.328725in}}%
\pgfpathlineto{\pgfqpoint{4.658474in}{1.323935in}}%
\pgfpathclose%
\pgfusepath{fill}%
\end{pgfscope}%
\begin{pgfscope}%
\pgfpathrectangle{\pgfqpoint{1.150000in}{0.150000in}}{\pgfqpoint{5.700000in}{5.700000in}}%
\pgfusepath{clip}%
\pgfsetbuttcap%
\pgfsetroundjoin%
\definecolor{currentfill}{rgb}{0.120092,0.600104,0.542530}%
\pgfsetfillcolor{currentfill}%
\pgfsetfillopacity{0.700000}%
\pgfsetlinewidth{0.000000pt}%
\definecolor{currentstroke}{rgb}{0.000000,0.000000,0.000000}%
\pgfsetstrokecolor{currentstroke}%
\pgfsetdash{}{0pt}%
\pgfpathmoveto{\pgfqpoint{3.141898in}{2.429244in}}%
\pgfpathlineto{\pgfqpoint{3.156211in}{2.418026in}}%
\pgfpathlineto{\pgfqpoint{3.170527in}{2.406838in}}%
\pgfpathlineto{\pgfqpoint{3.184844in}{2.395679in}}%
\pgfpathlineto{\pgfqpoint{3.199165in}{2.384549in}}%
\pgfpathlineto{\pgfqpoint{3.189908in}{2.411272in}}%
\pgfpathlineto{\pgfqpoint{3.180617in}{2.438880in}}%
\pgfpathlineto{\pgfqpoint{3.171293in}{2.467390in}}%
\pgfpathlineto{\pgfqpoint{3.161933in}{2.496820in}}%
\pgfpathlineto{\pgfqpoint{3.147546in}{2.508469in}}%
\pgfpathlineto{\pgfqpoint{3.133161in}{2.520148in}}%
\pgfpathlineto{\pgfqpoint{3.118778in}{2.531857in}}%
\pgfpathlineto{\pgfqpoint{3.104398in}{2.543596in}}%
\pgfpathlineto{\pgfqpoint{3.113827in}{2.513638in}}%
\pgfpathlineto{\pgfqpoint{3.123220in}{2.484604in}}%
\pgfpathlineto{\pgfqpoint{3.132576in}{2.456479in}}%
\pgfpathlineto{\pgfqpoint{3.141898in}{2.429244in}}%
\pgfpathclose%
\pgfusepath{fill}%
\end{pgfscope}%
\begin{pgfscope}%
\pgfpathrectangle{\pgfqpoint{1.150000in}{0.150000in}}{\pgfqpoint{5.700000in}{5.700000in}}%
\pgfusepath{clip}%
\pgfsetbuttcap%
\pgfsetroundjoin%
\definecolor{currentfill}{rgb}{0.525776,0.833491,0.288127}%
\pgfsetfillcolor{currentfill}%
\pgfsetfillopacity{0.700000}%
\pgfsetlinewidth{0.000000pt}%
\definecolor{currentstroke}{rgb}{0.000000,0.000000,0.000000}%
\pgfsetstrokecolor{currentstroke}%
\pgfsetdash{}{0pt}%
\pgfpathmoveto{\pgfqpoint{2.358765in}{3.199670in}}%
\pgfpathlineto{\pgfqpoint{2.373080in}{3.186109in}}%
\pgfpathlineto{\pgfqpoint{2.387395in}{3.172591in}}%
\pgfpathlineto{\pgfqpoint{2.401711in}{3.159114in}}%
\pgfpathlineto{\pgfqpoint{2.416026in}{3.145677in}}%
\pgfpathlineto{\pgfqpoint{2.405632in}{3.183106in}}%
\pgfpathlineto{\pgfqpoint{2.395182in}{3.221568in}}%
\pgfpathlineto{\pgfqpoint{2.384675in}{3.261081in}}%
\pgfpathlineto{\pgfqpoint{2.370295in}{3.274939in}}%
\pgfpathlineto{\pgfqpoint{2.355915in}{3.288837in}}%
\pgfpathlineto{\pgfqpoint{2.341535in}{3.302778in}}%
\pgfpathlineto{\pgfqpoint{2.327155in}{3.316760in}}%
\pgfpathlineto{\pgfqpoint{2.337749in}{3.276677in}}%
\pgfpathlineto{\pgfqpoint{2.348285in}{3.237654in}}%
\pgfpathlineto{\pgfqpoint{2.358765in}{3.199670in}}%
\pgfpathclose%
\pgfusepath{fill}%
\end{pgfscope}%
\begin{pgfscope}%
\pgfpathrectangle{\pgfqpoint{1.150000in}{0.150000in}}{\pgfqpoint{5.700000in}{5.700000in}}%
\pgfusepath{clip}%
\pgfsetbuttcap%
\pgfsetroundjoin%
\definecolor{currentfill}{rgb}{0.179019,0.433756,0.557430}%
\pgfsetfillcolor{currentfill}%
\pgfsetfillopacity{0.700000}%
\pgfsetlinewidth{0.000000pt}%
\definecolor{currentstroke}{rgb}{0.000000,0.000000,0.000000}%
\pgfsetstrokecolor{currentstroke}%
\pgfsetdash{}{0pt}%
\pgfpathmoveto{\pgfqpoint{3.693737in}{1.960686in}}%
\pgfpathlineto{\pgfqpoint{3.708100in}{1.950956in}}%
\pgfpathlineto{\pgfqpoint{3.722467in}{1.941251in}}%
\pgfpathlineto{\pgfqpoint{3.736837in}{1.931572in}}%
\pgfpathlineto{\pgfqpoint{3.751211in}{1.921919in}}%
\pgfpathlineto{\pgfqpoint{3.742580in}{1.940886in}}%
\pgfpathlineto{\pgfqpoint{3.733928in}{1.960629in}}%
\pgfpathlineto{\pgfqpoint{3.725255in}{1.981163in}}%
\pgfpathlineto{\pgfqpoint{3.716560in}{2.002504in}}%
\pgfpathlineto{\pgfqpoint{3.702133in}{2.012647in}}%
\pgfpathlineto{\pgfqpoint{3.687710in}{2.022816in}}%
\pgfpathlineto{\pgfqpoint{3.673290in}{2.033011in}}%
\pgfpathlineto{\pgfqpoint{3.658873in}{2.043232in}}%
\pgfpathlineto{\pgfqpoint{3.667623in}{2.021394in}}%
\pgfpathlineto{\pgfqpoint{3.676350in}{2.000367in}}%
\pgfpathlineto{\pgfqpoint{3.685054in}{1.980136in}}%
\pgfpathlineto{\pgfqpoint{3.693737in}{1.960686in}}%
\pgfpathclose%
\pgfusepath{fill}%
\end{pgfscope}%
\begin{pgfscope}%
\pgfpathrectangle{\pgfqpoint{1.150000in}{0.150000in}}{\pgfqpoint{5.700000in}{5.700000in}}%
\pgfusepath{clip}%
\pgfsetbuttcap%
\pgfsetroundjoin%
\definecolor{currentfill}{rgb}{0.477504,0.821444,0.318195}%
\pgfsetfillcolor{currentfill}%
\pgfsetfillopacity{0.700000}%
\pgfsetlinewidth{0.000000pt}%
\definecolor{currentstroke}{rgb}{0.000000,0.000000,0.000000}%
\pgfsetstrokecolor{currentstroke}%
\pgfsetdash{}{0pt}%
\pgfpathmoveto{\pgfqpoint{2.416026in}{3.145677in}}%
\pgfpathlineto{\pgfqpoint{2.430343in}{3.132281in}}%
\pgfpathlineto{\pgfqpoint{2.444660in}{3.118926in}}%
\pgfpathlineto{\pgfqpoint{2.458977in}{3.105610in}}%
\pgfpathlineto{\pgfqpoint{2.473295in}{3.092335in}}%
\pgfpathlineto{\pgfqpoint{2.462984in}{3.129210in}}%
\pgfpathlineto{\pgfqpoint{2.452619in}{3.167112in}}%
\pgfpathlineto{\pgfqpoint{2.442199in}{3.206059in}}%
\pgfpathlineto{\pgfqpoint{2.427818in}{3.219754in}}%
\pgfpathlineto{\pgfqpoint{2.413437in}{3.233490in}}%
\pgfpathlineto{\pgfqpoint{2.399056in}{3.247265in}}%
\pgfpathlineto{\pgfqpoint{2.384675in}{3.261081in}}%
\pgfpathlineto{\pgfqpoint{2.395182in}{3.221568in}}%
\pgfpathlineto{\pgfqpoint{2.405632in}{3.183106in}}%
\pgfpathlineto{\pgfqpoint{2.416026in}{3.145677in}}%
\pgfpathclose%
\pgfusepath{fill}%
\end{pgfscope}%
\begin{pgfscope}%
\pgfpathrectangle{\pgfqpoint{1.150000in}{0.150000in}}{\pgfqpoint{5.700000in}{5.700000in}}%
\pgfusepath{clip}%
\pgfsetbuttcap%
\pgfsetroundjoin%
\definecolor{currentfill}{rgb}{0.121831,0.589055,0.545623}%
\pgfsetfillcolor{currentfill}%
\pgfsetfillopacity{0.700000}%
\pgfsetlinewidth{0.000000pt}%
\definecolor{currentstroke}{rgb}{0.000000,0.000000,0.000000}%
\pgfsetstrokecolor{currentstroke}%
\pgfsetdash{}{0pt}%
\pgfpathmoveto{\pgfqpoint{3.199165in}{2.384549in}}%
\pgfpathlineto{\pgfqpoint{3.213488in}{2.373449in}}%
\pgfpathlineto{\pgfqpoint{3.227813in}{2.362377in}}%
\pgfpathlineto{\pgfqpoint{3.242141in}{2.351335in}}%
\pgfpathlineto{\pgfqpoint{3.256471in}{2.340322in}}%
\pgfpathlineto{\pgfqpoint{3.247278in}{2.366534in}}%
\pgfpathlineto{\pgfqpoint{3.238053in}{2.393626in}}%
\pgfpathlineto{\pgfqpoint{3.228795in}{2.421614in}}%
\pgfpathlineto{\pgfqpoint{3.219502in}{2.450517in}}%
\pgfpathlineto{\pgfqpoint{3.205107in}{2.462049in}}%
\pgfpathlineto{\pgfqpoint{3.190713in}{2.473610in}}%
\pgfpathlineto{\pgfqpoint{3.176322in}{2.485200in}}%
\pgfpathlineto{\pgfqpoint{3.161933in}{2.496820in}}%
\pgfpathlineto{\pgfqpoint{3.171293in}{2.467390in}}%
\pgfpathlineto{\pgfqpoint{3.180617in}{2.438880in}}%
\pgfpathlineto{\pgfqpoint{3.189908in}{2.411272in}}%
\pgfpathlineto{\pgfqpoint{3.199165in}{2.384549in}}%
\pgfpathclose%
\pgfusepath{fill}%
\end{pgfscope}%
\begin{pgfscope}%
\pgfpathrectangle{\pgfqpoint{1.150000in}{0.150000in}}{\pgfqpoint{5.700000in}{5.700000in}}%
\pgfusepath{clip}%
\pgfsetbuttcap%
\pgfsetroundjoin%
\definecolor{currentfill}{rgb}{0.233603,0.313828,0.543914}%
\pgfsetfillcolor{currentfill}%
\pgfsetfillopacity{0.700000}%
\pgfsetlinewidth{0.000000pt}%
\definecolor{currentstroke}{rgb}{0.000000,0.000000,0.000000}%
\pgfsetstrokecolor{currentstroke}%
\pgfsetdash{}{0pt}%
\pgfpathmoveto{\pgfqpoint{4.130583in}{1.640645in}}%
\pgfpathlineto{\pgfqpoint{4.145014in}{1.632082in}}%
\pgfpathlineto{\pgfqpoint{4.159449in}{1.623543in}}%
\pgfpathlineto{\pgfqpoint{4.173889in}{1.615029in}}%
\pgfpathlineto{\pgfqpoint{4.188334in}{1.606538in}}%
\pgfpathlineto{\pgfqpoint{4.180086in}{1.619315in}}%
\pgfpathlineto{\pgfqpoint{4.171826in}{1.632777in}}%
\pgfpathlineto{\pgfqpoint{4.163553in}{1.646939in}}%
\pgfpathlineto{\pgfqpoint{4.155268in}{1.661815in}}%
\pgfpathlineto{\pgfqpoint{4.140781in}{1.670773in}}%
\pgfpathlineto{\pgfqpoint{4.126299in}{1.679754in}}%
\pgfpathlineto{\pgfqpoint{4.111821in}{1.688760in}}%
\pgfpathlineto{\pgfqpoint{4.097348in}{1.697791in}}%
\pgfpathlineto{\pgfqpoint{4.105677in}{1.682441in}}%
\pgfpathlineto{\pgfqpoint{4.113992in}{1.667809in}}%
\pgfpathlineto{\pgfqpoint{4.122294in}{1.653882in}}%
\pgfpathlineto{\pgfqpoint{4.130583in}{1.640645in}}%
\pgfpathclose%
\pgfusepath{fill}%
\end{pgfscope}%
\begin{pgfscope}%
\pgfpathrectangle{\pgfqpoint{1.150000in}{0.150000in}}{\pgfqpoint{5.700000in}{5.700000in}}%
\pgfusepath{clip}%
\pgfsetbuttcap%
\pgfsetroundjoin%
\definecolor{currentfill}{rgb}{0.440137,0.811138,0.340967}%
\pgfsetfillcolor{currentfill}%
\pgfsetfillopacity{0.700000}%
\pgfsetlinewidth{0.000000pt}%
\definecolor{currentstroke}{rgb}{0.000000,0.000000,0.000000}%
\pgfsetstrokecolor{currentstroke}%
\pgfsetdash{}{0pt}%
\pgfpathmoveto{\pgfqpoint{2.473295in}{3.092335in}}%
\pgfpathlineto{\pgfqpoint{2.487614in}{3.079098in}}%
\pgfpathlineto{\pgfqpoint{2.501933in}{3.065901in}}%
\pgfpathlineto{\pgfqpoint{2.516252in}{3.052742in}}%
\pgfpathlineto{\pgfqpoint{2.530573in}{3.039622in}}%
\pgfpathlineto{\pgfqpoint{2.520344in}{3.075945in}}%
\pgfpathlineto{\pgfqpoint{2.510063in}{3.113289in}}%
\pgfpathlineto{\pgfqpoint{2.499728in}{3.151673in}}%
\pgfpathlineto{\pgfqpoint{2.485345in}{3.165211in}}%
\pgfpathlineto{\pgfqpoint{2.470963in}{3.178788in}}%
\pgfpathlineto{\pgfqpoint{2.456581in}{3.192404in}}%
\pgfpathlineto{\pgfqpoint{2.442199in}{3.206059in}}%
\pgfpathlineto{\pgfqpoint{2.452619in}{3.167112in}}%
\pgfpathlineto{\pgfqpoint{2.462984in}{3.129210in}}%
\pgfpathlineto{\pgfqpoint{2.473295in}{3.092335in}}%
\pgfpathclose%
\pgfusepath{fill}%
\end{pgfscope}%
\begin{pgfscope}%
\pgfpathrectangle{\pgfqpoint{1.150000in}{0.150000in}}{\pgfqpoint{5.700000in}{5.700000in}}%
\pgfusepath{clip}%
\pgfsetbuttcap%
\pgfsetroundjoin%
\definecolor{currentfill}{rgb}{0.182256,0.426184,0.557120}%
\pgfsetfillcolor{currentfill}%
\pgfsetfillopacity{0.700000}%
\pgfsetlinewidth{0.000000pt}%
\definecolor{currentstroke}{rgb}{0.000000,0.000000,0.000000}%
\pgfsetstrokecolor{currentstroke}%
\pgfsetdash{}{0pt}%
\pgfpathmoveto{\pgfqpoint{3.751211in}{1.921919in}}%
\pgfpathlineto{\pgfqpoint{3.765588in}{1.912292in}}%
\pgfpathlineto{\pgfqpoint{3.779970in}{1.902690in}}%
\pgfpathlineto{\pgfqpoint{3.794355in}{1.893114in}}%
\pgfpathlineto{\pgfqpoint{3.808744in}{1.883563in}}%
\pgfpathlineto{\pgfqpoint{3.800163in}{1.902047in}}%
\pgfpathlineto{\pgfqpoint{3.791563in}{1.921303in}}%
\pgfpathlineto{\pgfqpoint{3.782942in}{1.941345in}}%
\pgfpathlineto{\pgfqpoint{3.774301in}{1.962188in}}%
\pgfpathlineto{\pgfqpoint{3.759860in}{1.972228in}}%
\pgfpathlineto{\pgfqpoint{3.745423in}{1.982294in}}%
\pgfpathlineto{\pgfqpoint{3.730990in}{1.992386in}}%
\pgfpathlineto{\pgfqpoint{3.716560in}{2.002504in}}%
\pgfpathlineto{\pgfqpoint{3.725255in}{1.981163in}}%
\pgfpathlineto{\pgfqpoint{3.733928in}{1.960629in}}%
\pgfpathlineto{\pgfqpoint{3.742580in}{1.940886in}}%
\pgfpathlineto{\pgfqpoint{3.751211in}{1.921919in}}%
\pgfpathclose%
\pgfusepath{fill}%
\end{pgfscope}%
\begin{pgfscope}%
\pgfpathrectangle{\pgfqpoint{1.150000in}{0.150000in}}{\pgfqpoint{5.700000in}{5.700000in}}%
\pgfusepath{clip}%
\pgfsetbuttcap%
\pgfsetroundjoin%
\definecolor{currentfill}{rgb}{0.266580,0.228262,0.514349}%
\pgfsetfillcolor{currentfill}%
\pgfsetfillopacity{0.700000}%
\pgfsetlinewidth{0.000000pt}%
\definecolor{currentstroke}{rgb}{0.000000,0.000000,0.000000}%
\pgfsetstrokecolor{currentstroke}%
\pgfsetdash{}{0pt}%
\pgfpathmoveto{\pgfqpoint{4.452391in}{1.436590in}}%
\pgfpathlineto{\pgfqpoint{4.466883in}{1.428951in}}%
\pgfpathlineto{\pgfqpoint{4.481380in}{1.421336in}}%
\pgfpathlineto{\pgfqpoint{4.495883in}{1.413744in}}%
\pgfpathlineto{\pgfqpoint{4.510391in}{1.406175in}}%
\pgfpathlineto{\pgfqpoint{4.502360in}{1.414133in}}%
\pgfpathlineto{\pgfqpoint{4.494323in}{1.422702in}}%
\pgfpathlineto{\pgfqpoint{4.486279in}{1.431894in}}%
\pgfpathlineto{\pgfqpoint{4.478228in}{1.441724in}}%
\pgfpathlineto{\pgfqpoint{4.463686in}{1.449740in}}%
\pgfpathlineto{\pgfqpoint{4.449150in}{1.457781in}}%
\pgfpathlineto{\pgfqpoint{4.434618in}{1.465844in}}%
\pgfpathlineto{\pgfqpoint{4.420092in}{1.473931in}}%
\pgfpathlineto{\pgfqpoint{4.428178in}{1.463647in}}%
\pgfpathlineto{\pgfqpoint{4.436257in}{1.454005in}}%
\pgfpathlineto{\pgfqpoint{4.444327in}{1.444990in}}%
\pgfpathlineto{\pgfqpoint{4.452391in}{1.436590in}}%
\pgfpathclose%
\pgfusepath{fill}%
\end{pgfscope}%
\begin{pgfscope}%
\pgfpathrectangle{\pgfqpoint{1.150000in}{0.150000in}}{\pgfqpoint{5.700000in}{5.700000in}}%
\pgfusepath{clip}%
\pgfsetbuttcap%
\pgfsetroundjoin%
\definecolor{currentfill}{rgb}{0.280868,0.160771,0.472899}%
\pgfsetfillcolor{currentfill}%
\pgfsetfillopacity{0.700000}%
\pgfsetlinewidth{0.000000pt}%
\definecolor{currentstroke}{rgb}{0.000000,0.000000,0.000000}%
\pgfsetstrokecolor{currentstroke}%
\pgfsetdash{}{0pt}%
\pgfpathmoveto{\pgfqpoint{4.716623in}{1.296359in}}%
\pgfpathlineto{\pgfqpoint{4.731175in}{1.289522in}}%
\pgfpathlineto{\pgfqpoint{4.745732in}{1.282708in}}%
\pgfpathlineto{\pgfqpoint{4.760296in}{1.275918in}}%
\pgfpathlineto{\pgfqpoint{4.774865in}{1.269150in}}%
\pgfpathlineto{\pgfqpoint{4.766968in}{1.273089in}}%
\pgfpathlineto{\pgfqpoint{4.759069in}{1.277574in}}%
\pgfpathlineto{\pgfqpoint{4.751166in}{1.282615in}}%
\pgfpathlineto{\pgfqpoint{4.743260in}{1.288226in}}%
\pgfpathlineto{\pgfqpoint{4.728664in}{1.295425in}}%
\pgfpathlineto{\pgfqpoint{4.714073in}{1.302646in}}%
\pgfpathlineto{\pgfqpoint{4.699488in}{1.309891in}}%
\pgfpathlineto{\pgfqpoint{4.684909in}{1.317159in}}%
\pgfpathlineto{\pgfqpoint{4.692843in}{1.311111in}}%
\pgfpathlineto{\pgfqpoint{4.700774in}{1.305637in}}%
\pgfpathlineto{\pgfqpoint{4.708700in}{1.300723in}}%
\pgfpathlineto{\pgfqpoint{4.716623in}{1.296359in}}%
\pgfpathclose%
\pgfusepath{fill}%
\end{pgfscope}%
\begin{pgfscope}%
\pgfpathrectangle{\pgfqpoint{1.150000in}{0.150000in}}{\pgfqpoint{5.700000in}{5.700000in}}%
\pgfusepath{clip}%
\pgfsetbuttcap%
\pgfsetroundjoin%
\definecolor{currentfill}{rgb}{0.395174,0.797475,0.367757}%
\pgfsetfillcolor{currentfill}%
\pgfsetfillopacity{0.700000}%
\pgfsetlinewidth{0.000000pt}%
\definecolor{currentstroke}{rgb}{0.000000,0.000000,0.000000}%
\pgfsetstrokecolor{currentstroke}%
\pgfsetdash{}{0pt}%
\pgfpathmoveto{\pgfqpoint{2.530573in}{3.039622in}}%
\pgfpathlineto{\pgfqpoint{2.544894in}{3.026540in}}%
\pgfpathlineto{\pgfqpoint{2.559216in}{3.013496in}}%
\pgfpathlineto{\pgfqpoint{2.573539in}{3.000489in}}%
\pgfpathlineto{\pgfqpoint{2.587863in}{2.987520in}}%
\pgfpathlineto{\pgfqpoint{2.577715in}{3.023294in}}%
\pgfpathlineto{\pgfqpoint{2.567517in}{3.060082in}}%
\pgfpathlineto{\pgfqpoint{2.557267in}{3.097903in}}%
\pgfpathlineto{\pgfqpoint{2.542881in}{3.111289in}}%
\pgfpathlineto{\pgfqpoint{2.528496in}{3.124712in}}%
\pgfpathlineto{\pgfqpoint{2.514112in}{3.138173in}}%
\pgfpathlineto{\pgfqpoint{2.499728in}{3.151673in}}%
\pgfpathlineto{\pgfqpoint{2.510063in}{3.113289in}}%
\pgfpathlineto{\pgfqpoint{2.520344in}{3.075945in}}%
\pgfpathlineto{\pgfqpoint{2.530573in}{3.039622in}}%
\pgfpathclose%
\pgfusepath{fill}%
\end{pgfscope}%
\begin{pgfscope}%
\pgfpathrectangle{\pgfqpoint{1.150000in}{0.150000in}}{\pgfqpoint{5.700000in}{5.700000in}}%
\pgfusepath{clip}%
\pgfsetbuttcap%
\pgfsetroundjoin%
\definecolor{currentfill}{rgb}{0.125394,0.574318,0.549086}%
\pgfsetfillcolor{currentfill}%
\pgfsetfillopacity{0.700000}%
\pgfsetlinewidth{0.000000pt}%
\definecolor{currentstroke}{rgb}{0.000000,0.000000,0.000000}%
\pgfsetstrokecolor{currentstroke}%
\pgfsetdash{}{0pt}%
\pgfpathmoveto{\pgfqpoint{3.256471in}{2.340322in}}%
\pgfpathlineto{\pgfqpoint{3.270804in}{2.329337in}}%
\pgfpathlineto{\pgfqpoint{3.285140in}{2.318381in}}%
\pgfpathlineto{\pgfqpoint{3.299478in}{2.307454in}}%
\pgfpathlineto{\pgfqpoint{3.313819in}{2.296555in}}%
\pgfpathlineto{\pgfqpoint{3.304689in}{2.322257in}}%
\pgfpathlineto{\pgfqpoint{3.295528in}{2.348834in}}%
\pgfpathlineto{\pgfqpoint{3.286335in}{2.376302in}}%
\pgfpathlineto{\pgfqpoint{3.277110in}{2.404678in}}%
\pgfpathlineto{\pgfqpoint{3.262704in}{2.416095in}}%
\pgfpathlineto{\pgfqpoint{3.248301in}{2.427540in}}%
\pgfpathlineto{\pgfqpoint{3.233901in}{2.439014in}}%
\pgfpathlineto{\pgfqpoint{3.219502in}{2.450517in}}%
\pgfpathlineto{\pgfqpoint{3.228795in}{2.421614in}}%
\pgfpathlineto{\pgfqpoint{3.238053in}{2.393626in}}%
\pgfpathlineto{\pgfqpoint{3.247278in}{2.366534in}}%
\pgfpathlineto{\pgfqpoint{3.256471in}{2.340322in}}%
\pgfpathclose%
\pgfusepath{fill}%
\end{pgfscope}%
\begin{pgfscope}%
\pgfpathrectangle{\pgfqpoint{1.150000in}{0.150000in}}{\pgfqpoint{5.700000in}{5.700000in}}%
\pgfusepath{clip}%
\pgfsetbuttcap%
\pgfsetroundjoin%
\definecolor{currentfill}{rgb}{0.360741,0.785964,0.387814}%
\pgfsetfillcolor{currentfill}%
\pgfsetfillopacity{0.700000}%
\pgfsetlinewidth{0.000000pt}%
\definecolor{currentstroke}{rgb}{0.000000,0.000000,0.000000}%
\pgfsetstrokecolor{currentstroke}%
\pgfsetdash{}{0pt}%
\pgfpathmoveto{\pgfqpoint{2.587863in}{2.987520in}}%
\pgfpathlineto{\pgfqpoint{2.602188in}{2.974588in}}%
\pgfpathlineto{\pgfqpoint{2.616513in}{2.961693in}}%
\pgfpathlineto{\pgfqpoint{2.630840in}{2.948835in}}%
\pgfpathlineto{\pgfqpoint{2.645167in}{2.936013in}}%
\pgfpathlineto{\pgfqpoint{2.635100in}{2.971238in}}%
\pgfpathlineto{\pgfqpoint{2.624983in}{3.007472in}}%
\pgfpathlineto{\pgfqpoint{2.614816in}{3.044733in}}%
\pgfpathlineto{\pgfqpoint{2.600427in}{3.057970in}}%
\pgfpathlineto{\pgfqpoint{2.586040in}{3.071244in}}%
\pgfpathlineto{\pgfqpoint{2.571653in}{3.084555in}}%
\pgfpathlineto{\pgfqpoint{2.557267in}{3.097903in}}%
\pgfpathlineto{\pgfqpoint{2.567517in}{3.060082in}}%
\pgfpathlineto{\pgfqpoint{2.577715in}{3.023294in}}%
\pgfpathlineto{\pgfqpoint{2.587863in}{2.987520in}}%
\pgfpathclose%
\pgfusepath{fill}%
\end{pgfscope}%
\begin{pgfscope}%
\pgfpathrectangle{\pgfqpoint{1.150000in}{0.150000in}}{\pgfqpoint{5.700000in}{5.700000in}}%
\pgfusepath{clip}%
\pgfsetbuttcap%
\pgfsetroundjoin%
\definecolor{currentfill}{rgb}{0.237441,0.305202,0.541921}%
\pgfsetfillcolor{currentfill}%
\pgfsetfillopacity{0.700000}%
\pgfsetlinewidth{0.000000pt}%
\definecolor{currentstroke}{rgb}{0.000000,0.000000,0.000000}%
\pgfsetstrokecolor{currentstroke}%
\pgfsetdash{}{0pt}%
\pgfpathmoveto{\pgfqpoint{4.188334in}{1.606538in}}%
\pgfpathlineto{\pgfqpoint{4.202783in}{1.598072in}}%
\pgfpathlineto{\pgfqpoint{4.217237in}{1.589629in}}%
\pgfpathlineto{\pgfqpoint{4.231696in}{1.581211in}}%
\pgfpathlineto{\pgfqpoint{4.246159in}{1.572816in}}%
\pgfpathlineto{\pgfqpoint{4.237950in}{1.585133in}}%
\pgfpathlineto{\pgfqpoint{4.229731in}{1.598130in}}%
\pgfpathlineto{\pgfqpoint{4.221500in}{1.611823in}}%
\pgfpathlineto{\pgfqpoint{4.213258in}{1.626225in}}%
\pgfpathlineto{\pgfqpoint{4.198753in}{1.635087in}}%
\pgfpathlineto{\pgfqpoint{4.184254in}{1.643972in}}%
\pgfpathlineto{\pgfqpoint{4.169759in}{1.652881in}}%
\pgfpathlineto{\pgfqpoint{4.155268in}{1.661815in}}%
\pgfpathlineto{\pgfqpoint{4.163553in}{1.646939in}}%
\pgfpathlineto{\pgfqpoint{4.171826in}{1.632777in}}%
\pgfpathlineto{\pgfqpoint{4.180086in}{1.619315in}}%
\pgfpathlineto{\pgfqpoint{4.188334in}{1.606538in}}%
\pgfpathclose%
\pgfusepath{fill}%
\end{pgfscope}%
\begin{pgfscope}%
\pgfpathrectangle{\pgfqpoint{1.150000in}{0.150000in}}{\pgfqpoint{5.700000in}{5.700000in}}%
\pgfusepath{clip}%
\pgfsetbuttcap%
\pgfsetroundjoin%
\definecolor{currentfill}{rgb}{0.187231,0.414746,0.556547}%
\pgfsetfillcolor{currentfill}%
\pgfsetfillopacity{0.700000}%
\pgfsetlinewidth{0.000000pt}%
\definecolor{currentstroke}{rgb}{0.000000,0.000000,0.000000}%
\pgfsetstrokecolor{currentstroke}%
\pgfsetdash{}{0pt}%
\pgfpathmoveto{\pgfqpoint{3.808744in}{1.883563in}}%
\pgfpathlineto{\pgfqpoint{3.823136in}{1.874037in}}%
\pgfpathlineto{\pgfqpoint{3.837533in}{1.864537in}}%
\pgfpathlineto{\pgfqpoint{3.851933in}{1.855062in}}%
\pgfpathlineto{\pgfqpoint{3.866337in}{1.845612in}}%
\pgfpathlineto{\pgfqpoint{3.857806in}{1.863615in}}%
\pgfpathlineto{\pgfqpoint{3.849256in}{1.882384in}}%
\pgfpathlineto{\pgfqpoint{3.840687in}{1.901934in}}%
\pgfpathlineto{\pgfqpoint{3.832099in}{1.922281in}}%
\pgfpathlineto{\pgfqpoint{3.817644in}{1.932220in}}%
\pgfpathlineto{\pgfqpoint{3.803193in}{1.942184in}}%
\pgfpathlineto{\pgfqpoint{3.788745in}{1.952173in}}%
\pgfpathlineto{\pgfqpoint{3.774301in}{1.962188in}}%
\pgfpathlineto{\pgfqpoint{3.782942in}{1.941345in}}%
\pgfpathlineto{\pgfqpoint{3.791563in}{1.921303in}}%
\pgfpathlineto{\pgfqpoint{3.800163in}{1.902047in}}%
\pgfpathlineto{\pgfqpoint{3.808744in}{1.883563in}}%
\pgfpathclose%
\pgfusepath{fill}%
\end{pgfscope}%
\begin{pgfscope}%
\pgfpathrectangle{\pgfqpoint{1.150000in}{0.150000in}}{\pgfqpoint{5.700000in}{5.700000in}}%
\pgfusepath{clip}%
\pgfsetbuttcap%
\pgfsetroundjoin%
\definecolor{currentfill}{rgb}{0.128729,0.563265,0.551229}%
\pgfsetfillcolor{currentfill}%
\pgfsetfillopacity{0.700000}%
\pgfsetlinewidth{0.000000pt}%
\definecolor{currentstroke}{rgb}{0.000000,0.000000,0.000000}%
\pgfsetstrokecolor{currentstroke}%
\pgfsetdash{}{0pt}%
\pgfpathmoveto{\pgfqpoint{3.313819in}{2.296555in}}%
\pgfpathlineto{\pgfqpoint{3.328162in}{2.285684in}}%
\pgfpathlineto{\pgfqpoint{3.342509in}{2.274841in}}%
\pgfpathlineto{\pgfqpoint{3.356858in}{2.264027in}}%
\pgfpathlineto{\pgfqpoint{3.371210in}{2.253240in}}%
\pgfpathlineto{\pgfqpoint{3.362142in}{2.278434in}}%
\pgfpathlineto{\pgfqpoint{3.353044in}{2.304497in}}%
\pgfpathlineto{\pgfqpoint{3.343916in}{2.331446in}}%
\pgfpathlineto{\pgfqpoint{3.334756in}{2.359297in}}%
\pgfpathlineto{\pgfqpoint{3.320341in}{2.370600in}}%
\pgfpathlineto{\pgfqpoint{3.305928in}{2.381931in}}%
\pgfpathlineto{\pgfqpoint{3.291518in}{2.393291in}}%
\pgfpathlineto{\pgfqpoint{3.277110in}{2.404678in}}%
\pgfpathlineto{\pgfqpoint{3.286335in}{2.376302in}}%
\pgfpathlineto{\pgfqpoint{3.295528in}{2.348834in}}%
\pgfpathlineto{\pgfqpoint{3.304689in}{2.322257in}}%
\pgfpathlineto{\pgfqpoint{3.313819in}{2.296555in}}%
\pgfpathclose%
\pgfusepath{fill}%
\end{pgfscope}%
\begin{pgfscope}%
\pgfpathrectangle{\pgfqpoint{1.150000in}{0.150000in}}{\pgfqpoint{5.700000in}{5.700000in}}%
\pgfusepath{clip}%
\pgfsetbuttcap%
\pgfsetroundjoin%
\definecolor{currentfill}{rgb}{0.269308,0.218818,0.509577}%
\pgfsetfillcolor{currentfill}%
\pgfsetfillopacity{0.700000}%
\pgfsetlinewidth{0.000000pt}%
\definecolor{currentstroke}{rgb}{0.000000,0.000000,0.000000}%
\pgfsetstrokecolor{currentstroke}%
\pgfsetdash{}{0pt}%
\pgfpathmoveto{\pgfqpoint{4.510391in}{1.406175in}}%
\pgfpathlineto{\pgfqpoint{4.524904in}{1.398629in}}%
\pgfpathlineto{\pgfqpoint{4.539423in}{1.391107in}}%
\pgfpathlineto{\pgfqpoint{4.553947in}{1.383608in}}%
\pgfpathlineto{\pgfqpoint{4.568476in}{1.376132in}}%
\pgfpathlineto{\pgfqpoint{4.560477in}{1.383648in}}%
\pgfpathlineto{\pgfqpoint{4.552473in}{1.391772in}}%
\pgfpathlineto{\pgfqpoint{4.544462in}{1.400514in}}%
\pgfpathlineto{\pgfqpoint{4.536445in}{1.409889in}}%
\pgfpathlineto{\pgfqpoint{4.521883in}{1.417813in}}%
\pgfpathlineto{\pgfqpoint{4.507326in}{1.425760in}}%
\pgfpathlineto{\pgfqpoint{4.492774in}{1.433730in}}%
\pgfpathlineto{\pgfqpoint{4.478228in}{1.441724in}}%
\pgfpathlineto{\pgfqpoint{4.486279in}{1.431894in}}%
\pgfpathlineto{\pgfqpoint{4.494323in}{1.422702in}}%
\pgfpathlineto{\pgfqpoint{4.502360in}{1.414133in}}%
\pgfpathlineto{\pgfqpoint{4.510391in}{1.406175in}}%
\pgfpathclose%
\pgfusepath{fill}%
\end{pgfscope}%
\begin{pgfscope}%
\pgfpathrectangle{\pgfqpoint{1.150000in}{0.150000in}}{\pgfqpoint{5.700000in}{5.700000in}}%
\pgfusepath{clip}%
\pgfsetbuttcap%
\pgfsetroundjoin%
\definecolor{currentfill}{rgb}{0.319809,0.770914,0.411152}%
\pgfsetfillcolor{currentfill}%
\pgfsetfillopacity{0.700000}%
\pgfsetlinewidth{0.000000pt}%
\definecolor{currentstroke}{rgb}{0.000000,0.000000,0.000000}%
\pgfsetstrokecolor{currentstroke}%
\pgfsetdash{}{0pt}%
\pgfpathmoveto{\pgfqpoint{2.645167in}{2.936013in}}%
\pgfpathlineto{\pgfqpoint{2.659496in}{2.923227in}}%
\pgfpathlineto{\pgfqpoint{2.673826in}{2.910476in}}%
\pgfpathlineto{\pgfqpoint{2.688157in}{2.897762in}}%
\pgfpathlineto{\pgfqpoint{2.702489in}{2.885083in}}%
\pgfpathlineto{\pgfqpoint{2.692500in}{2.919762in}}%
\pgfpathlineto{\pgfqpoint{2.682464in}{2.955444in}}%
\pgfpathlineto{\pgfqpoint{2.672378in}{2.992146in}}%
\pgfpathlineto{\pgfqpoint{2.657986in}{3.005239in}}%
\pgfpathlineto{\pgfqpoint{2.643595in}{3.018367in}}%
\pgfpathlineto{\pgfqpoint{2.629205in}{3.031532in}}%
\pgfpathlineto{\pgfqpoint{2.614816in}{3.044733in}}%
\pgfpathlineto{\pgfqpoint{2.624983in}{3.007472in}}%
\pgfpathlineto{\pgfqpoint{2.635100in}{2.971238in}}%
\pgfpathlineto{\pgfqpoint{2.645167in}{2.936013in}}%
\pgfpathclose%
\pgfusepath{fill}%
\end{pgfscope}%
\begin{pgfscope}%
\pgfpathrectangle{\pgfqpoint{1.150000in}{0.150000in}}{\pgfqpoint{5.700000in}{5.700000in}}%
\pgfusepath{clip}%
\pgfsetbuttcap%
\pgfsetroundjoin%
\definecolor{currentfill}{rgb}{0.281412,0.155834,0.469201}%
\pgfsetfillcolor{currentfill}%
\pgfsetfillopacity{0.700000}%
\pgfsetlinewidth{0.000000pt}%
\definecolor{currentstroke}{rgb}{0.000000,0.000000,0.000000}%
\pgfsetstrokecolor{currentstroke}%
\pgfsetdash{}{0pt}%
\pgfpathmoveto{\pgfqpoint{4.774865in}{1.269150in}}%
\pgfpathlineto{\pgfqpoint{4.789441in}{1.262405in}}%
\pgfpathlineto{\pgfqpoint{4.804022in}{1.255683in}}%
\pgfpathlineto{\pgfqpoint{4.818609in}{1.248984in}}%
\pgfpathlineto{\pgfqpoint{4.833203in}{1.242308in}}%
\pgfpathlineto{\pgfqpoint{4.825331in}{1.245822in}}%
\pgfpathlineto{\pgfqpoint{4.817457in}{1.249878in}}%
\pgfpathlineto{\pgfqpoint{4.809581in}{1.254486in}}%
\pgfpathlineto{\pgfqpoint{4.801702in}{1.259660in}}%
\pgfpathlineto{\pgfqpoint{4.787083in}{1.266767in}}%
\pgfpathlineto{\pgfqpoint{4.772470in}{1.273897in}}%
\pgfpathlineto{\pgfqpoint{4.757862in}{1.281050in}}%
\pgfpathlineto{\pgfqpoint{4.743260in}{1.288226in}}%
\pgfpathlineto{\pgfqpoint{4.751166in}{1.282615in}}%
\pgfpathlineto{\pgfqpoint{4.759069in}{1.277574in}}%
\pgfpathlineto{\pgfqpoint{4.766968in}{1.273089in}}%
\pgfpathlineto{\pgfqpoint{4.774865in}{1.269150in}}%
\pgfpathclose%
\pgfusepath{fill}%
\end{pgfscope}%
\begin{pgfscope}%
\pgfpathrectangle{\pgfqpoint{1.150000in}{0.150000in}}{\pgfqpoint{5.700000in}{5.700000in}}%
\pgfusepath{clip}%
\pgfsetbuttcap%
\pgfsetroundjoin%
\definecolor{currentfill}{rgb}{0.288921,0.758394,0.428426}%
\pgfsetfillcolor{currentfill}%
\pgfsetfillopacity{0.700000}%
\pgfsetlinewidth{0.000000pt}%
\definecolor{currentstroke}{rgb}{0.000000,0.000000,0.000000}%
\pgfsetstrokecolor{currentstroke}%
\pgfsetdash{}{0pt}%
\pgfpathmoveto{\pgfqpoint{2.702489in}{2.885083in}}%
\pgfpathlineto{\pgfqpoint{2.716822in}{2.872438in}}%
\pgfpathlineto{\pgfqpoint{2.731157in}{2.859829in}}%
\pgfpathlineto{\pgfqpoint{2.745492in}{2.847255in}}%
\pgfpathlineto{\pgfqpoint{2.759829in}{2.834715in}}%
\pgfpathlineto{\pgfqpoint{2.749919in}{2.868850in}}%
\pgfpathlineto{\pgfqpoint{2.739962in}{2.903981in}}%
\pgfpathlineto{\pgfqpoint{2.729957in}{2.940126in}}%
\pgfpathlineto{\pgfqpoint{2.715561in}{2.953079in}}%
\pgfpathlineto{\pgfqpoint{2.701166in}{2.966066in}}%
\pgfpathlineto{\pgfqpoint{2.686772in}{2.979088in}}%
\pgfpathlineto{\pgfqpoint{2.672378in}{2.992146in}}%
\pgfpathlineto{\pgfqpoint{2.682464in}{2.955444in}}%
\pgfpathlineto{\pgfqpoint{2.692500in}{2.919762in}}%
\pgfpathlineto{\pgfqpoint{2.702489in}{2.885083in}}%
\pgfpathclose%
\pgfusepath{fill}%
\end{pgfscope}%
\begin{pgfscope}%
\pgfpathrectangle{\pgfqpoint{1.150000in}{0.150000in}}{\pgfqpoint{5.700000in}{5.700000in}}%
\pgfusepath{clip}%
\pgfsetbuttcap%
\pgfsetroundjoin%
\definecolor{currentfill}{rgb}{0.192357,0.403199,0.555836}%
\pgfsetfillcolor{currentfill}%
\pgfsetfillopacity{0.700000}%
\pgfsetlinewidth{0.000000pt}%
\definecolor{currentstroke}{rgb}{0.000000,0.000000,0.000000}%
\pgfsetstrokecolor{currentstroke}%
\pgfsetdash{}{0pt}%
\pgfpathmoveto{\pgfqpoint{3.866337in}{1.845612in}}%
\pgfpathlineto{\pgfqpoint{3.880745in}{1.836187in}}%
\pgfpathlineto{\pgfqpoint{3.895157in}{1.826787in}}%
\pgfpathlineto{\pgfqpoint{3.909573in}{1.817412in}}%
\pgfpathlineto{\pgfqpoint{3.923993in}{1.808062in}}%
\pgfpathlineto{\pgfqpoint{3.915510in}{1.825584in}}%
\pgfpathlineto{\pgfqpoint{3.907010in}{1.843868in}}%
\pgfpathlineto{\pgfqpoint{3.898491in}{1.862927in}}%
\pgfpathlineto{\pgfqpoint{3.889955in}{1.882778in}}%
\pgfpathlineto{\pgfqpoint{3.875485in}{1.892616in}}%
\pgfpathlineto{\pgfqpoint{3.861019in}{1.902479in}}%
\pgfpathlineto{\pgfqpoint{3.846557in}{1.912367in}}%
\pgfpathlineto{\pgfqpoint{3.832099in}{1.922281in}}%
\pgfpathlineto{\pgfqpoint{3.840687in}{1.901934in}}%
\pgfpathlineto{\pgfqpoint{3.849256in}{1.882384in}}%
\pgfpathlineto{\pgfqpoint{3.857806in}{1.863615in}}%
\pgfpathlineto{\pgfqpoint{3.866337in}{1.845612in}}%
\pgfpathclose%
\pgfusepath{fill}%
\end{pgfscope}%
\begin{pgfscope}%
\pgfpathrectangle{\pgfqpoint{1.150000in}{0.150000in}}{\pgfqpoint{5.700000in}{5.700000in}}%
\pgfusepath{clip}%
\pgfsetbuttcap%
\pgfsetroundjoin%
\definecolor{currentfill}{rgb}{0.133743,0.548535,0.553541}%
\pgfsetfillcolor{currentfill}%
\pgfsetfillopacity{0.700000}%
\pgfsetlinewidth{0.000000pt}%
\definecolor{currentstroke}{rgb}{0.000000,0.000000,0.000000}%
\pgfsetstrokecolor{currentstroke}%
\pgfsetdash{}{0pt}%
\pgfpathmoveto{\pgfqpoint{3.371210in}{2.253240in}}%
\pgfpathlineto{\pgfqpoint{3.385564in}{2.242481in}}%
\pgfpathlineto{\pgfqpoint{3.399922in}{2.231751in}}%
\pgfpathlineto{\pgfqpoint{3.414282in}{2.221047in}}%
\pgfpathlineto{\pgfqpoint{3.428646in}{2.210372in}}%
\pgfpathlineto{\pgfqpoint{3.419639in}{2.235058in}}%
\pgfpathlineto{\pgfqpoint{3.410603in}{2.260608in}}%
\pgfpathlineto{\pgfqpoint{3.401538in}{2.287038in}}%
\pgfpathlineto{\pgfqpoint{3.392443in}{2.314366in}}%
\pgfpathlineto{\pgfqpoint{3.378018in}{2.325557in}}%
\pgfpathlineto{\pgfqpoint{3.363595in}{2.336776in}}%
\pgfpathlineto{\pgfqpoint{3.349174in}{2.348023in}}%
\pgfpathlineto{\pgfqpoint{3.334756in}{2.359297in}}%
\pgfpathlineto{\pgfqpoint{3.343916in}{2.331446in}}%
\pgfpathlineto{\pgfqpoint{3.353044in}{2.304497in}}%
\pgfpathlineto{\pgfqpoint{3.362142in}{2.278434in}}%
\pgfpathlineto{\pgfqpoint{3.371210in}{2.253240in}}%
\pgfpathclose%
\pgfusepath{fill}%
\end{pgfscope}%
\begin{pgfscope}%
\pgfpathrectangle{\pgfqpoint{1.150000in}{0.150000in}}{\pgfqpoint{5.700000in}{5.700000in}}%
\pgfusepath{clip}%
\pgfsetbuttcap%
\pgfsetroundjoin%
\definecolor{currentfill}{rgb}{0.241237,0.296485,0.539709}%
\pgfsetfillcolor{currentfill}%
\pgfsetfillopacity{0.700000}%
\pgfsetlinewidth{0.000000pt}%
\definecolor{currentstroke}{rgb}{0.000000,0.000000,0.000000}%
\pgfsetstrokecolor{currentstroke}%
\pgfsetdash{}{0pt}%
\pgfpathmoveto{\pgfqpoint{4.246159in}{1.572816in}}%
\pgfpathlineto{\pgfqpoint{4.260627in}{1.564445in}}%
\pgfpathlineto{\pgfqpoint{4.275099in}{1.556098in}}%
\pgfpathlineto{\pgfqpoint{4.289577in}{1.547775in}}%
\pgfpathlineto{\pgfqpoint{4.304059in}{1.539476in}}%
\pgfpathlineto{\pgfqpoint{4.295890in}{1.551332in}}%
\pgfpathlineto{\pgfqpoint{4.287710in}{1.563866in}}%
\pgfpathlineto{\pgfqpoint{4.279520in}{1.577090in}}%
\pgfpathlineto{\pgfqpoint{4.271319in}{1.591019in}}%
\pgfpathlineto{\pgfqpoint{4.256797in}{1.599785in}}%
\pgfpathlineto{\pgfqpoint{4.242279in}{1.608574in}}%
\pgfpathlineto{\pgfqpoint{4.227766in}{1.617388in}}%
\pgfpathlineto{\pgfqpoint{4.213258in}{1.626225in}}%
\pgfpathlineto{\pgfqpoint{4.221500in}{1.611823in}}%
\pgfpathlineto{\pgfqpoint{4.229731in}{1.598130in}}%
\pgfpathlineto{\pgfqpoint{4.237950in}{1.585133in}}%
\pgfpathlineto{\pgfqpoint{4.246159in}{1.572816in}}%
\pgfpathclose%
\pgfusepath{fill}%
\end{pgfscope}%
\begin{pgfscope}%
\pgfpathrectangle{\pgfqpoint{1.150000in}{0.150000in}}{\pgfqpoint{5.700000in}{5.700000in}}%
\pgfusepath{clip}%
\pgfsetbuttcap%
\pgfsetroundjoin%
\definecolor{currentfill}{rgb}{0.259857,0.745492,0.444467}%
\pgfsetfillcolor{currentfill}%
\pgfsetfillopacity{0.700000}%
\pgfsetlinewidth{0.000000pt}%
\definecolor{currentstroke}{rgb}{0.000000,0.000000,0.000000}%
\pgfsetstrokecolor{currentstroke}%
\pgfsetdash{}{0pt}%
\pgfpathmoveto{\pgfqpoint{2.759829in}{2.834715in}}%
\pgfpathlineto{\pgfqpoint{2.774168in}{2.822209in}}%
\pgfpathlineto{\pgfqpoint{2.788508in}{2.809738in}}%
\pgfpathlineto{\pgfqpoint{2.802849in}{2.797300in}}%
\pgfpathlineto{\pgfqpoint{2.817192in}{2.784896in}}%
\pgfpathlineto{\pgfqpoint{2.807358in}{2.818488in}}%
\pgfpathlineto{\pgfqpoint{2.797479in}{2.853070in}}%
\pgfpathlineto{\pgfqpoint{2.787554in}{2.888660in}}%
\pgfpathlineto{\pgfqpoint{2.773153in}{2.901476in}}%
\pgfpathlineto{\pgfqpoint{2.758753in}{2.914325in}}%
\pgfpathlineto{\pgfqpoint{2.744354in}{2.927208in}}%
\pgfpathlineto{\pgfqpoint{2.729957in}{2.940126in}}%
\pgfpathlineto{\pgfqpoint{2.739962in}{2.903981in}}%
\pgfpathlineto{\pgfqpoint{2.749919in}{2.868850in}}%
\pgfpathlineto{\pgfqpoint{2.759829in}{2.834715in}}%
\pgfpathclose%
\pgfusepath{fill}%
\end{pgfscope}%
\begin{pgfscope}%
\pgfpathrectangle{\pgfqpoint{1.150000in}{0.150000in}}{\pgfqpoint{5.700000in}{5.700000in}}%
\pgfusepath{clip}%
\pgfsetbuttcap%
\pgfsetroundjoin%
\definecolor{currentfill}{rgb}{0.271828,0.209303,0.504434}%
\pgfsetfillcolor{currentfill}%
\pgfsetfillopacity{0.700000}%
\pgfsetlinewidth{0.000000pt}%
\definecolor{currentstroke}{rgb}{0.000000,0.000000,0.000000}%
\pgfsetstrokecolor{currentstroke}%
\pgfsetdash{}{0pt}%
\pgfpathmoveto{\pgfqpoint{4.568476in}{1.376132in}}%
\pgfpathlineto{\pgfqpoint{4.583011in}{1.368680in}}%
\pgfpathlineto{\pgfqpoint{4.597551in}{1.361250in}}%
\pgfpathlineto{\pgfqpoint{4.612097in}{1.353844in}}%
\pgfpathlineto{\pgfqpoint{4.626648in}{1.346461in}}%
\pgfpathlineto{\pgfqpoint{4.618680in}{1.353536in}}%
\pgfpathlineto{\pgfqpoint{4.610707in}{1.361213in}}%
\pgfpathlineto{\pgfqpoint{4.602729in}{1.369506in}}%
\pgfpathlineto{\pgfqpoint{4.594745in}{1.378427in}}%
\pgfpathlineto{\pgfqpoint{4.580162in}{1.386258in}}%
\pgfpathlineto{\pgfqpoint{4.565585in}{1.394112in}}%
\pgfpathlineto{\pgfqpoint{4.551012in}{1.401989in}}%
\pgfpathlineto{\pgfqpoint{4.536445in}{1.409889in}}%
\pgfpathlineto{\pgfqpoint{4.544462in}{1.400514in}}%
\pgfpathlineto{\pgfqpoint{4.552473in}{1.391772in}}%
\pgfpathlineto{\pgfqpoint{4.560477in}{1.383648in}}%
\pgfpathlineto{\pgfqpoint{4.568476in}{1.376132in}}%
\pgfpathclose%
\pgfusepath{fill}%
\end{pgfscope}%
\begin{pgfscope}%
\pgfpathrectangle{\pgfqpoint{1.150000in}{0.150000in}}{\pgfqpoint{5.700000in}{5.700000in}}%
\pgfusepath{clip}%
\pgfsetbuttcap%
\pgfsetroundjoin%
\definecolor{currentfill}{rgb}{0.281887,0.150881,0.465405}%
\pgfsetfillcolor{currentfill}%
\pgfsetfillopacity{0.700000}%
\pgfsetlinewidth{0.000000pt}%
\definecolor{currentstroke}{rgb}{0.000000,0.000000,0.000000}%
\pgfsetstrokecolor{currentstroke}%
\pgfsetdash{}{0pt}%
\pgfpathmoveto{\pgfqpoint{4.833203in}{1.242308in}}%
\pgfpathlineto{\pgfqpoint{4.847802in}{1.235655in}}%
\pgfpathlineto{\pgfqpoint{4.862408in}{1.229024in}}%
\pgfpathlineto{\pgfqpoint{4.877019in}{1.222417in}}%
\pgfpathlineto{\pgfqpoint{4.869166in}{1.225612in}}%
\pgfpathlineto{\pgfqpoint{4.861311in}{1.229346in}}%
\pgfpathlineto{\pgfqpoint{4.853454in}{1.233629in}}%
\pgfpathlineto{\pgfqpoint{4.845595in}{1.238476in}}%
\pgfpathlineto{\pgfqpoint{4.830958in}{1.245514in}}%
\pgfpathlineto{\pgfqpoint{4.816327in}{1.252575in}}%
\pgfpathlineto{\pgfqpoint{4.801702in}{1.259660in}}%
\pgfpathlineto{\pgfqpoint{4.809581in}{1.254486in}}%
\pgfpathlineto{\pgfqpoint{4.817457in}{1.249878in}}%
\pgfpathlineto{\pgfqpoint{4.825331in}{1.245822in}}%
\pgfpathlineto{\pgfqpoint{4.833203in}{1.242308in}}%
\pgfpathclose%
\pgfusepath{fill}%
\end{pgfscope}%
\begin{pgfscope}%
\pgfpathrectangle{\pgfqpoint{1.150000in}{0.150000in}}{\pgfqpoint{5.700000in}{5.700000in}}%
\pgfusepath{clip}%
\pgfsetbuttcap%
\pgfsetroundjoin%
\definecolor{currentfill}{rgb}{0.137770,0.537492,0.554906}%
\pgfsetfillcolor{currentfill}%
\pgfsetfillopacity{0.700000}%
\pgfsetlinewidth{0.000000pt}%
\definecolor{currentstroke}{rgb}{0.000000,0.000000,0.000000}%
\pgfsetstrokecolor{currentstroke}%
\pgfsetdash{}{0pt}%
\pgfpathmoveto{\pgfqpoint{3.428646in}{2.210372in}}%
\pgfpathlineto{\pgfqpoint{3.443012in}{2.199723in}}%
\pgfpathlineto{\pgfqpoint{3.457381in}{2.189102in}}%
\pgfpathlineto{\pgfqpoint{3.471753in}{2.178509in}}%
\pgfpathlineto{\pgfqpoint{3.486128in}{2.167942in}}%
\pgfpathlineto{\pgfqpoint{3.477181in}{2.192122in}}%
\pgfpathlineto{\pgfqpoint{3.468207in}{2.217160in}}%
\pgfpathlineto{\pgfqpoint{3.459205in}{2.243073in}}%
\pgfpathlineto{\pgfqpoint{3.450173in}{2.269879in}}%
\pgfpathlineto{\pgfqpoint{3.435737in}{2.280959in}}%
\pgfpathlineto{\pgfqpoint{3.421303in}{2.292067in}}%
\pgfpathlineto{\pgfqpoint{3.406872in}{2.303203in}}%
\pgfpathlineto{\pgfqpoint{3.392443in}{2.314366in}}%
\pgfpathlineto{\pgfqpoint{3.401538in}{2.287038in}}%
\pgfpathlineto{\pgfqpoint{3.410603in}{2.260608in}}%
\pgfpathlineto{\pgfqpoint{3.419639in}{2.235058in}}%
\pgfpathlineto{\pgfqpoint{3.428646in}{2.210372in}}%
\pgfpathclose%
\pgfusepath{fill}%
\end{pgfscope}%
\begin{pgfscope}%
\pgfpathrectangle{\pgfqpoint{1.150000in}{0.150000in}}{\pgfqpoint{5.700000in}{5.700000in}}%
\pgfusepath{clip}%
\pgfsetbuttcap%
\pgfsetroundjoin%
\definecolor{currentfill}{rgb}{0.226397,0.728888,0.462789}%
\pgfsetfillcolor{currentfill}%
\pgfsetfillopacity{0.700000}%
\pgfsetlinewidth{0.000000pt}%
\definecolor{currentstroke}{rgb}{0.000000,0.000000,0.000000}%
\pgfsetstrokecolor{currentstroke}%
\pgfsetdash{}{0pt}%
\pgfpathmoveto{\pgfqpoint{2.817192in}{2.784896in}}%
\pgfpathlineto{\pgfqpoint{2.831536in}{2.772525in}}%
\pgfpathlineto{\pgfqpoint{2.845881in}{2.760188in}}%
\pgfpathlineto{\pgfqpoint{2.860228in}{2.747883in}}%
\pgfpathlineto{\pgfqpoint{2.874577in}{2.735612in}}%
\pgfpathlineto{\pgfqpoint{2.864819in}{2.768663in}}%
\pgfpathlineto{\pgfqpoint{2.855017in}{2.802698in}}%
\pgfpathlineto{\pgfqpoint{2.845171in}{2.837735in}}%
\pgfpathlineto{\pgfqpoint{2.830764in}{2.850416in}}%
\pgfpathlineto{\pgfqpoint{2.816360in}{2.863131in}}%
\pgfpathlineto{\pgfqpoint{2.801956in}{2.875879in}}%
\pgfpathlineto{\pgfqpoint{2.787554in}{2.888660in}}%
\pgfpathlineto{\pgfqpoint{2.797479in}{2.853070in}}%
\pgfpathlineto{\pgfqpoint{2.807358in}{2.818488in}}%
\pgfpathlineto{\pgfqpoint{2.817192in}{2.784896in}}%
\pgfpathclose%
\pgfusepath{fill}%
\end{pgfscope}%
\begin{pgfscope}%
\pgfpathrectangle{\pgfqpoint{1.150000in}{0.150000in}}{\pgfqpoint{5.700000in}{5.700000in}}%
\pgfusepath{clip}%
\pgfsetbuttcap%
\pgfsetroundjoin%
\definecolor{currentfill}{rgb}{0.195860,0.395433,0.555276}%
\pgfsetfillcolor{currentfill}%
\pgfsetfillopacity{0.700000}%
\pgfsetlinewidth{0.000000pt}%
\definecolor{currentstroke}{rgb}{0.000000,0.000000,0.000000}%
\pgfsetstrokecolor{currentstroke}%
\pgfsetdash{}{0pt}%
\pgfpathmoveto{\pgfqpoint{3.923993in}{1.808062in}}%
\pgfpathlineto{\pgfqpoint{3.938416in}{1.798737in}}%
\pgfpathlineto{\pgfqpoint{3.952844in}{1.789437in}}%
\pgfpathlineto{\pgfqpoint{3.967276in}{1.780162in}}%
\pgfpathlineto{\pgfqpoint{3.981712in}{1.770911in}}%
\pgfpathlineto{\pgfqpoint{3.973277in}{1.787953in}}%
\pgfpathlineto{\pgfqpoint{3.964825in}{1.805751in}}%
\pgfpathlineto{\pgfqpoint{3.956357in}{1.824321in}}%
\pgfpathlineto{\pgfqpoint{3.947871in}{1.843677in}}%
\pgfpathlineto{\pgfqpoint{3.933386in}{1.853415in}}%
\pgfpathlineto{\pgfqpoint{3.918905in}{1.863178in}}%
\pgfpathlineto{\pgfqpoint{3.904428in}{1.872966in}}%
\pgfpathlineto{\pgfqpoint{3.889955in}{1.882778in}}%
\pgfpathlineto{\pgfqpoint{3.898491in}{1.862927in}}%
\pgfpathlineto{\pgfqpoint{3.907010in}{1.843868in}}%
\pgfpathlineto{\pgfqpoint{3.915510in}{1.825584in}}%
\pgfpathlineto{\pgfqpoint{3.923993in}{1.808062in}}%
\pgfpathclose%
\pgfusepath{fill}%
\end{pgfscope}%
\begin{pgfscope}%
\pgfpathrectangle{\pgfqpoint{1.150000in}{0.150000in}}{\pgfqpoint{5.700000in}{5.700000in}}%
\pgfusepath{clip}%
\pgfsetbuttcap%
\pgfsetroundjoin%
\definecolor{currentfill}{rgb}{0.244972,0.287675,0.537260}%
\pgfsetfillcolor{currentfill}%
\pgfsetfillopacity{0.700000}%
\pgfsetlinewidth{0.000000pt}%
\definecolor{currentstroke}{rgb}{0.000000,0.000000,0.000000}%
\pgfsetstrokecolor{currentstroke}%
\pgfsetdash{}{0pt}%
\pgfpathmoveto{\pgfqpoint{4.304059in}{1.539476in}}%
\pgfpathlineto{\pgfqpoint{4.318546in}{1.531200in}}%
\pgfpathlineto{\pgfqpoint{4.333038in}{1.522948in}}%
\pgfpathlineto{\pgfqpoint{4.347535in}{1.514720in}}%
\pgfpathlineto{\pgfqpoint{4.362036in}{1.506515in}}%
\pgfpathlineto{\pgfqpoint{4.353905in}{1.517912in}}%
\pgfpathlineto{\pgfqpoint{4.345764in}{1.529982in}}%
\pgfpathlineto{\pgfqpoint{4.337614in}{1.542738in}}%
\pgfpathlineto{\pgfqpoint{4.329454in}{1.556194in}}%
\pgfpathlineto{\pgfqpoint{4.314913in}{1.564865in}}%
\pgfpathlineto{\pgfqpoint{4.300377in}{1.573559in}}%
\pgfpathlineto{\pgfqpoint{4.285846in}{1.582277in}}%
\pgfpathlineto{\pgfqpoint{4.271319in}{1.591019in}}%
\pgfpathlineto{\pgfqpoint{4.279520in}{1.577090in}}%
\pgfpathlineto{\pgfqpoint{4.287710in}{1.563866in}}%
\pgfpathlineto{\pgfqpoint{4.295890in}{1.551332in}}%
\pgfpathlineto{\pgfqpoint{4.304059in}{1.539476in}}%
\pgfpathclose%
\pgfusepath{fill}%
\end{pgfscope}%
\begin{pgfscope}%
\pgfpathrectangle{\pgfqpoint{1.150000in}{0.150000in}}{\pgfqpoint{5.700000in}{5.700000in}}%
\pgfusepath{clip}%
\pgfsetbuttcap%
\pgfsetroundjoin%
\definecolor{currentfill}{rgb}{0.202219,0.715272,0.476084}%
\pgfsetfillcolor{currentfill}%
\pgfsetfillopacity{0.700000}%
\pgfsetlinewidth{0.000000pt}%
\definecolor{currentstroke}{rgb}{0.000000,0.000000,0.000000}%
\pgfsetstrokecolor{currentstroke}%
\pgfsetdash{}{0pt}%
\pgfpathmoveto{\pgfqpoint{2.874577in}{2.735612in}}%
\pgfpathlineto{\pgfqpoint{2.888928in}{2.723373in}}%
\pgfpathlineto{\pgfqpoint{2.903280in}{2.711167in}}%
\pgfpathlineto{\pgfqpoint{2.917633in}{2.698993in}}%
\pgfpathlineto{\pgfqpoint{2.931989in}{2.686851in}}%
\pgfpathlineto{\pgfqpoint{2.922305in}{2.719362in}}%
\pgfpathlineto{\pgfqpoint{2.912579in}{2.752851in}}%
\pgfpathlineto{\pgfqpoint{2.902810in}{2.787336in}}%
\pgfpathlineto{\pgfqpoint{2.888398in}{2.799887in}}%
\pgfpathlineto{\pgfqpoint{2.873988in}{2.812470in}}%
\pgfpathlineto{\pgfqpoint{2.859578in}{2.825086in}}%
\pgfpathlineto{\pgfqpoint{2.845171in}{2.837735in}}%
\pgfpathlineto{\pgfqpoint{2.855017in}{2.802698in}}%
\pgfpathlineto{\pgfqpoint{2.864819in}{2.768663in}}%
\pgfpathlineto{\pgfqpoint{2.874577in}{2.735612in}}%
\pgfpathclose%
\pgfusepath{fill}%
\end{pgfscope}%
\begin{pgfscope}%
\pgfpathrectangle{\pgfqpoint{1.150000in}{0.150000in}}{\pgfqpoint{5.700000in}{5.700000in}}%
\pgfusepath{clip}%
\pgfsetbuttcap%
\pgfsetroundjoin%
\definecolor{currentfill}{rgb}{0.141935,0.526453,0.555991}%
\pgfsetfillcolor{currentfill}%
\pgfsetfillopacity{0.700000}%
\pgfsetlinewidth{0.000000pt}%
\definecolor{currentstroke}{rgb}{0.000000,0.000000,0.000000}%
\pgfsetstrokecolor{currentstroke}%
\pgfsetdash{}{0pt}%
\pgfpathmoveto{\pgfqpoint{3.486128in}{2.167942in}}%
\pgfpathlineto{\pgfqpoint{3.500506in}{2.157403in}}%
\pgfpathlineto{\pgfqpoint{3.514888in}{2.146891in}}%
\pgfpathlineto{\pgfqpoint{3.529272in}{2.136405in}}%
\pgfpathlineto{\pgfqpoint{3.543659in}{2.125946in}}%
\pgfpathlineto{\pgfqpoint{3.534771in}{2.149620in}}%
\pgfpathlineto{\pgfqpoint{3.525857in}{2.174148in}}%
\pgfpathlineto{\pgfqpoint{3.516916in}{2.199545in}}%
\pgfpathlineto{\pgfqpoint{3.507947in}{2.225829in}}%
\pgfpathlineto{\pgfqpoint{3.493500in}{2.236800in}}%
\pgfpathlineto{\pgfqpoint{3.479055in}{2.247799in}}%
\pgfpathlineto{\pgfqpoint{3.464613in}{2.258825in}}%
\pgfpathlineto{\pgfqpoint{3.450173in}{2.269879in}}%
\pgfpathlineto{\pgfqpoint{3.459205in}{2.243073in}}%
\pgfpathlineto{\pgfqpoint{3.468207in}{2.217160in}}%
\pgfpathlineto{\pgfqpoint{3.477181in}{2.192122in}}%
\pgfpathlineto{\pgfqpoint{3.486128in}{2.167942in}}%
\pgfpathclose%
\pgfusepath{fill}%
\end{pgfscope}%
\begin{pgfscope}%
\pgfpathrectangle{\pgfqpoint{1.150000in}{0.150000in}}{\pgfqpoint{5.700000in}{5.700000in}}%
\pgfusepath{clip}%
\pgfsetbuttcap%
\pgfsetroundjoin%
\definecolor{currentfill}{rgb}{0.273006,0.204520,0.501721}%
\pgfsetfillcolor{currentfill}%
\pgfsetfillopacity{0.700000}%
\pgfsetlinewidth{0.000000pt}%
\definecolor{currentstroke}{rgb}{0.000000,0.000000,0.000000}%
\pgfsetstrokecolor{currentstroke}%
\pgfsetdash{}{0pt}%
\pgfpathmoveto{\pgfqpoint{4.626648in}{1.346461in}}%
\pgfpathlineto{\pgfqpoint{4.641205in}{1.339101in}}%
\pgfpathlineto{\pgfqpoint{4.655768in}{1.331764in}}%
\pgfpathlineto{\pgfqpoint{4.670336in}{1.324450in}}%
\pgfpathlineto{\pgfqpoint{4.684909in}{1.317159in}}%
\pgfpathlineto{\pgfqpoint{4.676971in}{1.323793in}}%
\pgfpathlineto{\pgfqpoint{4.669029in}{1.331025in}}%
\pgfpathlineto{\pgfqpoint{4.661082in}{1.338868in}}%
\pgfpathlineto{\pgfqpoint{4.653131in}{1.347336in}}%
\pgfpathlineto{\pgfqpoint{4.638526in}{1.355074in}}%
\pgfpathlineto{\pgfqpoint{4.623927in}{1.362835in}}%
\pgfpathlineto{\pgfqpoint{4.609334in}{1.370620in}}%
\pgfpathlineto{\pgfqpoint{4.594745in}{1.378427in}}%
\pgfpathlineto{\pgfqpoint{4.602729in}{1.369506in}}%
\pgfpathlineto{\pgfqpoint{4.610707in}{1.361213in}}%
\pgfpathlineto{\pgfqpoint{4.618680in}{1.353536in}}%
\pgfpathlineto{\pgfqpoint{4.626648in}{1.346461in}}%
\pgfpathclose%
\pgfusepath{fill}%
\end{pgfscope}%
\begin{pgfscope}%
\pgfpathrectangle{\pgfqpoint{1.150000in}{0.150000in}}{\pgfqpoint{5.700000in}{5.700000in}}%
\pgfusepath{clip}%
\pgfsetbuttcap%
\pgfsetroundjoin%
\definecolor{currentfill}{rgb}{0.201239,0.383670,0.554294}%
\pgfsetfillcolor{currentfill}%
\pgfsetfillopacity{0.700000}%
\pgfsetlinewidth{0.000000pt}%
\definecolor{currentstroke}{rgb}{0.000000,0.000000,0.000000}%
\pgfsetstrokecolor{currentstroke}%
\pgfsetdash{}{0pt}%
\pgfpathmoveto{\pgfqpoint{3.981712in}{1.770911in}}%
\pgfpathlineto{\pgfqpoint{3.996152in}{1.761685in}}%
\pgfpathlineto{\pgfqpoint{4.010596in}{1.752484in}}%
\pgfpathlineto{\pgfqpoint{4.025044in}{1.743307in}}%
\pgfpathlineto{\pgfqpoint{4.039496in}{1.734155in}}%
\pgfpathlineto{\pgfqpoint{4.031108in}{1.750717in}}%
\pgfpathlineto{\pgfqpoint{4.022704in}{1.768030in}}%
\pgfpathlineto{\pgfqpoint{4.014284in}{1.786111in}}%
\pgfpathlineto{\pgfqpoint{4.005848in}{1.804973in}}%
\pgfpathlineto{\pgfqpoint{3.991348in}{1.814612in}}%
\pgfpathlineto{\pgfqpoint{3.976852in}{1.824275in}}%
\pgfpathlineto{\pgfqpoint{3.962359in}{1.833964in}}%
\pgfpathlineto{\pgfqpoint{3.947871in}{1.843677in}}%
\pgfpathlineto{\pgfqpoint{3.956357in}{1.824321in}}%
\pgfpathlineto{\pgfqpoint{3.964825in}{1.805751in}}%
\pgfpathlineto{\pgfqpoint{3.973277in}{1.787953in}}%
\pgfpathlineto{\pgfqpoint{3.981712in}{1.770911in}}%
\pgfpathclose%
\pgfusepath{fill}%
\end{pgfscope}%
\begin{pgfscope}%
\pgfpathrectangle{\pgfqpoint{1.150000in}{0.150000in}}{\pgfqpoint{5.700000in}{5.700000in}}%
\pgfusepath{clip}%
\pgfsetbuttcap%
\pgfsetroundjoin%
\definecolor{currentfill}{rgb}{0.180653,0.701402,0.488189}%
\pgfsetfillcolor{currentfill}%
\pgfsetfillopacity{0.700000}%
\pgfsetlinewidth{0.000000pt}%
\definecolor{currentstroke}{rgb}{0.000000,0.000000,0.000000}%
\pgfsetstrokecolor{currentstroke}%
\pgfsetdash{}{0pt}%
\pgfpathmoveto{\pgfqpoint{2.931989in}{2.686851in}}%
\pgfpathlineto{\pgfqpoint{2.946346in}{2.674741in}}%
\pgfpathlineto{\pgfqpoint{2.960705in}{2.662663in}}%
\pgfpathlineto{\pgfqpoint{2.975066in}{2.650617in}}%
\pgfpathlineto{\pgfqpoint{2.989428in}{2.638602in}}%
\pgfpathlineto{\pgfqpoint{2.979818in}{2.670575in}}%
\pgfpathlineto{\pgfqpoint{2.970167in}{2.703520in}}%
\pgfpathlineto{\pgfqpoint{2.960474in}{2.737454in}}%
\pgfpathlineto{\pgfqpoint{2.946056in}{2.749877in}}%
\pgfpathlineto{\pgfqpoint{2.931639in}{2.762332in}}%
\pgfpathlineto{\pgfqpoint{2.917224in}{2.774818in}}%
\pgfpathlineto{\pgfqpoint{2.902810in}{2.787336in}}%
\pgfpathlineto{\pgfqpoint{2.912579in}{2.752851in}}%
\pgfpathlineto{\pgfqpoint{2.922305in}{2.719362in}}%
\pgfpathlineto{\pgfqpoint{2.931989in}{2.686851in}}%
\pgfpathclose%
\pgfusepath{fill}%
\end{pgfscope}%
\begin{pgfscope}%
\pgfpathrectangle{\pgfqpoint{1.150000in}{0.150000in}}{\pgfqpoint{5.700000in}{5.700000in}}%
\pgfusepath{clip}%
\pgfsetbuttcap%
\pgfsetroundjoin%
\definecolor{currentfill}{rgb}{0.248629,0.278775,0.534556}%
\pgfsetfillcolor{currentfill}%
\pgfsetfillopacity{0.700000}%
\pgfsetlinewidth{0.000000pt}%
\definecolor{currentstroke}{rgb}{0.000000,0.000000,0.000000}%
\pgfsetstrokecolor{currentstroke}%
\pgfsetdash{}{0pt}%
\pgfpathmoveto{\pgfqpoint{4.362036in}{1.506515in}}%
\pgfpathlineto{\pgfqpoint{4.376543in}{1.498334in}}%
\pgfpathlineto{\pgfqpoint{4.391054in}{1.490176in}}%
\pgfpathlineto{\pgfqpoint{4.405571in}{1.482042in}}%
\pgfpathlineto{\pgfqpoint{4.420092in}{1.473931in}}%
\pgfpathlineto{\pgfqpoint{4.411997in}{1.484870in}}%
\pgfpathlineto{\pgfqpoint{4.403895in}{1.496476in}}%
\pgfpathlineto{\pgfqpoint{4.395783in}{1.508764in}}%
\pgfpathlineto{\pgfqpoint{4.387663in}{1.521748in}}%
\pgfpathlineto{\pgfqpoint{4.373103in}{1.530324in}}%
\pgfpathlineto{\pgfqpoint{4.358549in}{1.538924in}}%
\pgfpathlineto{\pgfqpoint{4.343999in}{1.547547in}}%
\pgfpathlineto{\pgfqpoint{4.329454in}{1.556194in}}%
\pgfpathlineto{\pgfqpoint{4.337614in}{1.542738in}}%
\pgfpathlineto{\pgfqpoint{4.345764in}{1.529982in}}%
\pgfpathlineto{\pgfqpoint{4.353905in}{1.517912in}}%
\pgfpathlineto{\pgfqpoint{4.362036in}{1.506515in}}%
\pgfpathclose%
\pgfusepath{fill}%
\end{pgfscope}%
\begin{pgfscope}%
\pgfpathrectangle{\pgfqpoint{1.150000in}{0.150000in}}{\pgfqpoint{5.700000in}{5.700000in}}%
\pgfusepath{clip}%
\pgfsetbuttcap%
\pgfsetroundjoin%
\definecolor{currentfill}{rgb}{0.146180,0.515413,0.556823}%
\pgfsetfillcolor{currentfill}%
\pgfsetfillopacity{0.700000}%
\pgfsetlinewidth{0.000000pt}%
\definecolor{currentstroke}{rgb}{0.000000,0.000000,0.000000}%
\pgfsetstrokecolor{currentstroke}%
\pgfsetdash{}{0pt}%
\pgfpathmoveto{\pgfqpoint{3.543659in}{2.125946in}}%
\pgfpathlineto{\pgfqpoint{3.558050in}{2.115514in}}%
\pgfpathlineto{\pgfqpoint{3.572444in}{2.105109in}}%
\pgfpathlineto{\pgfqpoint{3.586840in}{2.094730in}}%
\pgfpathlineto{\pgfqpoint{3.601240in}{2.084378in}}%
\pgfpathlineto{\pgfqpoint{3.592411in}{2.107547in}}%
\pgfpathlineto{\pgfqpoint{3.583555in}{2.131565in}}%
\pgfpathlineto{\pgfqpoint{3.574675in}{2.156447in}}%
\pgfpathlineto{\pgfqpoint{3.565767in}{2.182210in}}%
\pgfpathlineto{\pgfqpoint{3.551308in}{2.193074in}}%
\pgfpathlineto{\pgfqpoint{3.536852in}{2.203966in}}%
\pgfpathlineto{\pgfqpoint{3.522398in}{2.214884in}}%
\pgfpathlineto{\pgfqpoint{3.507947in}{2.225829in}}%
\pgfpathlineto{\pgfqpoint{3.516916in}{2.199545in}}%
\pgfpathlineto{\pgfqpoint{3.525857in}{2.174148in}}%
\pgfpathlineto{\pgfqpoint{3.534771in}{2.149620in}}%
\pgfpathlineto{\pgfqpoint{3.543659in}{2.125946in}}%
\pgfpathclose%
\pgfusepath{fill}%
\end{pgfscope}%
\begin{pgfscope}%
\pgfpathrectangle{\pgfqpoint{1.150000in}{0.150000in}}{\pgfqpoint{5.700000in}{5.700000in}}%
\pgfusepath{clip}%
\pgfsetbuttcap%
\pgfsetroundjoin%
\definecolor{currentfill}{rgb}{0.162016,0.687316,0.499129}%
\pgfsetfillcolor{currentfill}%
\pgfsetfillopacity{0.700000}%
\pgfsetlinewidth{0.000000pt}%
\definecolor{currentstroke}{rgb}{0.000000,0.000000,0.000000}%
\pgfsetstrokecolor{currentstroke}%
\pgfsetdash{}{0pt}%
\pgfpathmoveto{\pgfqpoint{2.989428in}{2.638602in}}%
\pgfpathlineto{\pgfqpoint{3.003792in}{2.626618in}}%
\pgfpathlineto{\pgfqpoint{3.018159in}{2.614666in}}%
\pgfpathlineto{\pgfqpoint{3.032527in}{2.602744in}}%
\pgfpathlineto{\pgfqpoint{3.046897in}{2.590853in}}%
\pgfpathlineto{\pgfqpoint{3.037360in}{2.622289in}}%
\pgfpathlineto{\pgfqpoint{3.027782in}{2.654691in}}%
\pgfpathlineto{\pgfqpoint{3.018165in}{2.688078in}}%
\pgfpathlineto{\pgfqpoint{3.003740in}{2.700375in}}%
\pgfpathlineto{\pgfqpoint{2.989316in}{2.712703in}}%
\pgfpathlineto{\pgfqpoint{2.974894in}{2.725063in}}%
\pgfpathlineto{\pgfqpoint{2.960474in}{2.737454in}}%
\pgfpathlineto{\pgfqpoint{2.970167in}{2.703520in}}%
\pgfpathlineto{\pgfqpoint{2.979818in}{2.670575in}}%
\pgfpathlineto{\pgfqpoint{2.989428in}{2.638602in}}%
\pgfpathclose%
\pgfusepath{fill}%
\end{pgfscope}%
\begin{pgfscope}%
\pgfpathrectangle{\pgfqpoint{1.150000in}{0.150000in}}{\pgfqpoint{5.700000in}{5.700000in}}%
\pgfusepath{clip}%
\pgfsetbuttcap%
\pgfsetroundjoin%
\definecolor{currentfill}{rgb}{0.204903,0.375746,0.553533}%
\pgfsetfillcolor{currentfill}%
\pgfsetfillopacity{0.700000}%
\pgfsetlinewidth{0.000000pt}%
\definecolor{currentstroke}{rgb}{0.000000,0.000000,0.000000}%
\pgfsetstrokecolor{currentstroke}%
\pgfsetdash{}{0pt}%
\pgfpathmoveto{\pgfqpoint{4.039496in}{1.734155in}}%
\pgfpathlineto{\pgfqpoint{4.053953in}{1.725027in}}%
\pgfpathlineto{\pgfqpoint{4.068414in}{1.715924in}}%
\pgfpathlineto{\pgfqpoint{4.082879in}{1.706845in}}%
\pgfpathlineto{\pgfqpoint{4.097348in}{1.697791in}}%
\pgfpathlineto{\pgfqpoint{4.089005in}{1.713873in}}%
\pgfpathlineto{\pgfqpoint{4.080648in}{1.730703in}}%
\pgfpathlineto{\pgfqpoint{4.072275in}{1.748294in}}%
\pgfpathlineto{\pgfqpoint{4.063888in}{1.766663in}}%
\pgfpathlineto{\pgfqpoint{4.049372in}{1.776204in}}%
\pgfpathlineto{\pgfqpoint{4.034860in}{1.785769in}}%
\pgfpathlineto{\pgfqpoint{4.020352in}{1.795359in}}%
\pgfpathlineto{\pgfqpoint{4.005848in}{1.804973in}}%
\pgfpathlineto{\pgfqpoint{4.014284in}{1.786111in}}%
\pgfpathlineto{\pgfqpoint{4.022704in}{1.768030in}}%
\pgfpathlineto{\pgfqpoint{4.031108in}{1.750717in}}%
\pgfpathlineto{\pgfqpoint{4.039496in}{1.734155in}}%
\pgfpathclose%
\pgfusepath{fill}%
\end{pgfscope}%
\begin{pgfscope}%
\pgfpathrectangle{\pgfqpoint{1.150000in}{0.150000in}}{\pgfqpoint{5.700000in}{5.700000in}}%
\pgfusepath{clip}%
\pgfsetbuttcap%
\pgfsetroundjoin%
\definecolor{currentfill}{rgb}{0.275191,0.194905,0.496005}%
\pgfsetfillcolor{currentfill}%
\pgfsetfillopacity{0.700000}%
\pgfsetlinewidth{0.000000pt}%
\definecolor{currentstroke}{rgb}{0.000000,0.000000,0.000000}%
\pgfsetstrokecolor{currentstroke}%
\pgfsetdash{}{0pt}%
\pgfpathmoveto{\pgfqpoint{4.684909in}{1.317159in}}%
\pgfpathlineto{\pgfqpoint{4.699488in}{1.309891in}}%
\pgfpathlineto{\pgfqpoint{4.714073in}{1.302646in}}%
\pgfpathlineto{\pgfqpoint{4.728664in}{1.295425in}}%
\pgfpathlineto{\pgfqpoint{4.743260in}{1.288226in}}%
\pgfpathlineto{\pgfqpoint{4.735351in}{1.294418in}}%
\pgfpathlineto{\pgfqpoint{4.727438in}{1.301205in}}%
\pgfpathlineto{\pgfqpoint{4.719522in}{1.308599in}}%
\pgfpathlineto{\pgfqpoint{4.711602in}{1.316613in}}%
\pgfpathlineto{\pgfqpoint{4.696976in}{1.324259in}}%
\pgfpathlineto{\pgfqpoint{4.682356in}{1.331929in}}%
\pgfpathlineto{\pgfqpoint{4.667740in}{1.339621in}}%
\pgfpathlineto{\pgfqpoint{4.653131in}{1.347336in}}%
\pgfpathlineto{\pgfqpoint{4.661082in}{1.338868in}}%
\pgfpathlineto{\pgfqpoint{4.669029in}{1.331025in}}%
\pgfpathlineto{\pgfqpoint{4.676971in}{1.323793in}}%
\pgfpathlineto{\pgfqpoint{4.684909in}{1.317159in}}%
\pgfpathclose%
\pgfusepath{fill}%
\end{pgfscope}%
\begin{pgfscope}%
\pgfpathrectangle{\pgfqpoint{1.150000in}{0.150000in}}{\pgfqpoint{5.700000in}{5.700000in}}%
\pgfusepath{clip}%
\pgfsetbuttcap%
\pgfsetroundjoin%
\definecolor{currentfill}{rgb}{0.151918,0.500685,0.557587}%
\pgfsetfillcolor{currentfill}%
\pgfsetfillopacity{0.700000}%
\pgfsetlinewidth{0.000000pt}%
\definecolor{currentstroke}{rgb}{0.000000,0.000000,0.000000}%
\pgfsetstrokecolor{currentstroke}%
\pgfsetdash{}{0pt}%
\pgfpathmoveto{\pgfqpoint{3.601240in}{2.084378in}}%
\pgfpathlineto{\pgfqpoint{3.615644in}{2.074052in}}%
\pgfpathlineto{\pgfqpoint{3.630050in}{2.063753in}}%
\pgfpathlineto{\pgfqpoint{3.644460in}{2.053479in}}%
\pgfpathlineto{\pgfqpoint{3.658873in}{2.043232in}}%
\pgfpathlineto{\pgfqpoint{3.650100in}{2.065897in}}%
\pgfpathlineto{\pgfqpoint{3.641303in}{2.089406in}}%
\pgfpathlineto{\pgfqpoint{3.632482in}{2.113774in}}%
\pgfpathlineto{\pgfqpoint{3.623635in}{2.139017in}}%
\pgfpathlineto{\pgfqpoint{3.609164in}{2.149776in}}%
\pgfpathlineto{\pgfqpoint{3.594695in}{2.160561in}}%
\pgfpathlineto{\pgfqpoint{3.580230in}{2.171372in}}%
\pgfpathlineto{\pgfqpoint{3.565767in}{2.182210in}}%
\pgfpathlineto{\pgfqpoint{3.574675in}{2.156447in}}%
\pgfpathlineto{\pgfqpoint{3.583555in}{2.131565in}}%
\pgfpathlineto{\pgfqpoint{3.592411in}{2.107547in}}%
\pgfpathlineto{\pgfqpoint{3.601240in}{2.084378in}}%
\pgfpathclose%
\pgfusepath{fill}%
\end{pgfscope}%
\begin{pgfscope}%
\pgfpathrectangle{\pgfqpoint{1.150000in}{0.150000in}}{\pgfqpoint{5.700000in}{5.700000in}}%
\pgfusepath{clip}%
\pgfsetbuttcap%
\pgfsetroundjoin%
\definecolor{currentfill}{rgb}{0.146616,0.673050,0.508936}%
\pgfsetfillcolor{currentfill}%
\pgfsetfillopacity{0.700000}%
\pgfsetlinewidth{0.000000pt}%
\definecolor{currentstroke}{rgb}{0.000000,0.000000,0.000000}%
\pgfsetstrokecolor{currentstroke}%
\pgfsetdash{}{0pt}%
\pgfpathmoveto{\pgfqpoint{3.046897in}{2.590853in}}%
\pgfpathlineto{\pgfqpoint{3.061269in}{2.578993in}}%
\pgfpathlineto{\pgfqpoint{3.075644in}{2.567164in}}%
\pgfpathlineto{\pgfqpoint{3.090020in}{2.555365in}}%
\pgfpathlineto{\pgfqpoint{3.104398in}{2.543596in}}%
\pgfpathlineto{\pgfqpoint{3.094932in}{2.574496in}}%
\pgfpathlineto{\pgfqpoint{3.085427in}{2.606357in}}%
\pgfpathlineto{\pgfqpoint{3.075884in}{2.639196in}}%
\pgfpathlineto{\pgfqpoint{3.061451in}{2.651371in}}%
\pgfpathlineto{\pgfqpoint{3.047021in}{2.663576in}}%
\pgfpathlineto{\pgfqpoint{3.032592in}{2.675811in}}%
\pgfpathlineto{\pgfqpoint{3.018165in}{2.688078in}}%
\pgfpathlineto{\pgfqpoint{3.027782in}{2.654691in}}%
\pgfpathlineto{\pgfqpoint{3.037360in}{2.622289in}}%
\pgfpathlineto{\pgfqpoint{3.046897in}{2.590853in}}%
\pgfpathclose%
\pgfusepath{fill}%
\end{pgfscope}%
\begin{pgfscope}%
\pgfpathrectangle{\pgfqpoint{1.150000in}{0.150000in}}{\pgfqpoint{5.700000in}{5.700000in}}%
\pgfusepath{clip}%
\pgfsetbuttcap%
\pgfsetroundjoin%
\definecolor{currentfill}{rgb}{0.252194,0.269783,0.531579}%
\pgfsetfillcolor{currentfill}%
\pgfsetfillopacity{0.700000}%
\pgfsetlinewidth{0.000000pt}%
\definecolor{currentstroke}{rgb}{0.000000,0.000000,0.000000}%
\pgfsetstrokecolor{currentstroke}%
\pgfsetdash{}{0pt}%
\pgfpathmoveto{\pgfqpoint{4.420092in}{1.473931in}}%
\pgfpathlineto{\pgfqpoint{4.434618in}{1.465844in}}%
\pgfpathlineto{\pgfqpoint{4.449150in}{1.457781in}}%
\pgfpathlineto{\pgfqpoint{4.463686in}{1.449740in}}%
\pgfpathlineto{\pgfqpoint{4.478228in}{1.441724in}}%
\pgfpathlineto{\pgfqpoint{4.470169in}{1.452203in}}%
\pgfpathlineto{\pgfqpoint{4.462103in}{1.463346in}}%
\pgfpathlineto{\pgfqpoint{4.454030in}{1.475167in}}%
\pgfpathlineto{\pgfqpoint{4.445948in}{1.487679in}}%
\pgfpathlineto{\pgfqpoint{4.431370in}{1.496161in}}%
\pgfpathlineto{\pgfqpoint{4.416796in}{1.504667in}}%
\pgfpathlineto{\pgfqpoint{4.402227in}{1.513196in}}%
\pgfpathlineto{\pgfqpoint{4.387663in}{1.521748in}}%
\pgfpathlineto{\pgfqpoint{4.395783in}{1.508764in}}%
\pgfpathlineto{\pgfqpoint{4.403895in}{1.496476in}}%
\pgfpathlineto{\pgfqpoint{4.411997in}{1.484870in}}%
\pgfpathlineto{\pgfqpoint{4.420092in}{1.473931in}}%
\pgfpathclose%
\pgfusepath{fill}%
\end{pgfscope}%
\begin{pgfscope}%
\pgfpathrectangle{\pgfqpoint{1.150000in}{0.150000in}}{\pgfqpoint{5.700000in}{5.700000in}}%
\pgfusepath{clip}%
\pgfsetbuttcap%
\pgfsetroundjoin%
\definecolor{currentfill}{rgb}{0.210503,0.363727,0.552206}%
\pgfsetfillcolor{currentfill}%
\pgfsetfillopacity{0.700000}%
\pgfsetlinewidth{0.000000pt}%
\definecolor{currentstroke}{rgb}{0.000000,0.000000,0.000000}%
\pgfsetstrokecolor{currentstroke}%
\pgfsetdash{}{0pt}%
\pgfpathmoveto{\pgfqpoint{4.097348in}{1.697791in}}%
\pgfpathlineto{\pgfqpoint{4.111821in}{1.688760in}}%
\pgfpathlineto{\pgfqpoint{4.126299in}{1.679754in}}%
\pgfpathlineto{\pgfqpoint{4.140781in}{1.670773in}}%
\pgfpathlineto{\pgfqpoint{4.155268in}{1.661815in}}%
\pgfpathlineto{\pgfqpoint{4.146969in}{1.677419in}}%
\pgfpathlineto{\pgfqpoint{4.138658in}{1.693765in}}%
\pgfpathlineto{\pgfqpoint{4.130332in}{1.710868in}}%
\pgfpathlineto{\pgfqpoint{4.121993in}{1.728744in}}%
\pgfpathlineto{\pgfqpoint{4.107460in}{1.738187in}}%
\pgfpathlineto{\pgfqpoint{4.092932in}{1.747655in}}%
\pgfpathlineto{\pgfqpoint{4.078408in}{1.757147in}}%
\pgfpathlineto{\pgfqpoint{4.063888in}{1.766663in}}%
\pgfpathlineto{\pgfqpoint{4.072275in}{1.748294in}}%
\pgfpathlineto{\pgfqpoint{4.080648in}{1.730703in}}%
\pgfpathlineto{\pgfqpoint{4.089005in}{1.713873in}}%
\pgfpathlineto{\pgfqpoint{4.097348in}{1.697791in}}%
\pgfpathclose%
\pgfusepath{fill}%
\end{pgfscope}%
\begin{pgfscope}%
\pgfpathrectangle{\pgfqpoint{1.150000in}{0.150000in}}{\pgfqpoint{5.700000in}{5.700000in}}%
\pgfusepath{clip}%
\pgfsetbuttcap%
\pgfsetroundjoin%
\definecolor{currentfill}{rgb}{0.137339,0.662252,0.515571}%
\pgfsetfillcolor{currentfill}%
\pgfsetfillopacity{0.700000}%
\pgfsetlinewidth{0.000000pt}%
\definecolor{currentstroke}{rgb}{0.000000,0.000000,0.000000}%
\pgfsetstrokecolor{currentstroke}%
\pgfsetdash{}{0pt}%
\pgfpathmoveto{\pgfqpoint{3.104398in}{2.543596in}}%
\pgfpathlineto{\pgfqpoint{3.118778in}{2.531857in}}%
\pgfpathlineto{\pgfqpoint{3.133161in}{2.520148in}}%
\pgfpathlineto{\pgfqpoint{3.147546in}{2.508469in}}%
\pgfpathlineto{\pgfqpoint{3.161933in}{2.496820in}}%
\pgfpathlineto{\pgfqpoint{3.152536in}{2.527186in}}%
\pgfpathlineto{\pgfqpoint{3.143104in}{2.558507in}}%
\pgfpathlineto{\pgfqpoint{3.133633in}{2.590800in}}%
\pgfpathlineto{\pgfqpoint{3.119193in}{2.602854in}}%
\pgfpathlineto{\pgfqpoint{3.104754in}{2.614938in}}%
\pgfpathlineto{\pgfqpoint{3.090318in}{2.627052in}}%
\pgfpathlineto{\pgfqpoint{3.075884in}{2.639196in}}%
\pgfpathlineto{\pgfqpoint{3.085427in}{2.606357in}}%
\pgfpathlineto{\pgfqpoint{3.094932in}{2.574496in}}%
\pgfpathlineto{\pgfqpoint{3.104398in}{2.543596in}}%
\pgfpathclose%
\pgfusepath{fill}%
\end{pgfscope}%
\begin{pgfscope}%
\pgfpathrectangle{\pgfqpoint{1.150000in}{0.150000in}}{\pgfqpoint{5.700000in}{5.700000in}}%
\pgfusepath{clip}%
\pgfsetbuttcap%
\pgfsetroundjoin%
\definecolor{currentfill}{rgb}{0.156270,0.489624,0.557936}%
\pgfsetfillcolor{currentfill}%
\pgfsetfillopacity{0.700000}%
\pgfsetlinewidth{0.000000pt}%
\definecolor{currentstroke}{rgb}{0.000000,0.000000,0.000000}%
\pgfsetstrokecolor{currentstroke}%
\pgfsetdash{}{0pt}%
\pgfpathmoveto{\pgfqpoint{3.658873in}{2.043232in}}%
\pgfpathlineto{\pgfqpoint{3.673290in}{2.033011in}}%
\pgfpathlineto{\pgfqpoint{3.687710in}{2.022816in}}%
\pgfpathlineto{\pgfqpoint{3.702133in}{2.012647in}}%
\pgfpathlineto{\pgfqpoint{3.716560in}{2.002504in}}%
\pgfpathlineto{\pgfqpoint{3.707843in}{2.024666in}}%
\pgfpathlineto{\pgfqpoint{3.699103in}{2.047666in}}%
\pgfpathlineto{\pgfqpoint{3.690339in}{2.071521in}}%
\pgfpathlineto{\pgfqpoint{3.681552in}{2.096246in}}%
\pgfpathlineto{\pgfqpoint{3.667068in}{2.106900in}}%
\pgfpathlineto{\pgfqpoint{3.652587in}{2.117579in}}%
\pgfpathlineto{\pgfqpoint{3.638110in}{2.128285in}}%
\pgfpathlineto{\pgfqpoint{3.623635in}{2.139017in}}%
\pgfpathlineto{\pgfqpoint{3.632482in}{2.113774in}}%
\pgfpathlineto{\pgfqpoint{3.641303in}{2.089406in}}%
\pgfpathlineto{\pgfqpoint{3.650100in}{2.065897in}}%
\pgfpathlineto{\pgfqpoint{3.658873in}{2.043232in}}%
\pgfpathclose%
\pgfusepath{fill}%
\end{pgfscope}%
\begin{pgfscope}%
\pgfpathrectangle{\pgfqpoint{1.150000in}{0.150000in}}{\pgfqpoint{5.700000in}{5.700000in}}%
\pgfusepath{clip}%
\pgfsetbuttcap%
\pgfsetroundjoin%
\definecolor{currentfill}{rgb}{0.276194,0.190074,0.493001}%
\pgfsetfillcolor{currentfill}%
\pgfsetfillopacity{0.700000}%
\pgfsetlinewidth{0.000000pt}%
\definecolor{currentstroke}{rgb}{0.000000,0.000000,0.000000}%
\pgfsetstrokecolor{currentstroke}%
\pgfsetdash{}{0pt}%
\pgfpathmoveto{\pgfqpoint{4.743260in}{1.288226in}}%
\pgfpathlineto{\pgfqpoint{4.757862in}{1.281050in}}%
\pgfpathlineto{\pgfqpoint{4.772470in}{1.273897in}}%
\pgfpathlineto{\pgfqpoint{4.787083in}{1.266767in}}%
\pgfpathlineto{\pgfqpoint{4.801702in}{1.259660in}}%
\pgfpathlineto{\pgfqpoint{4.793821in}{1.265411in}}%
\pgfpathlineto{\pgfqpoint{4.785937in}{1.271753in}}%
\pgfpathlineto{\pgfqpoint{4.778051in}{1.278698in}}%
\pgfpathlineto{\pgfqpoint{4.770162in}{1.286259in}}%
\pgfpathlineto{\pgfqpoint{4.755513in}{1.293813in}}%
\pgfpathlineto{\pgfqpoint{4.740871in}{1.301390in}}%
\pgfpathlineto{\pgfqpoint{4.726234in}{1.308990in}}%
\pgfpathlineto{\pgfqpoint{4.711602in}{1.316613in}}%
\pgfpathlineto{\pgfqpoint{4.719522in}{1.308599in}}%
\pgfpathlineto{\pgfqpoint{4.727438in}{1.301205in}}%
\pgfpathlineto{\pgfqpoint{4.735351in}{1.294418in}}%
\pgfpathlineto{\pgfqpoint{4.743260in}{1.288226in}}%
\pgfpathclose%
\pgfusepath{fill}%
\end{pgfscope}%
\begin{pgfscope}%
\pgfpathrectangle{\pgfqpoint{1.150000in}{0.150000in}}{\pgfqpoint{5.700000in}{5.700000in}}%
\pgfusepath{clip}%
\pgfsetbuttcap%
\pgfsetroundjoin%
\definecolor{currentfill}{rgb}{0.128087,0.647749,0.523491}%
\pgfsetfillcolor{currentfill}%
\pgfsetfillopacity{0.700000}%
\pgfsetlinewidth{0.000000pt}%
\definecolor{currentstroke}{rgb}{0.000000,0.000000,0.000000}%
\pgfsetstrokecolor{currentstroke}%
\pgfsetdash{}{0pt}%
\pgfpathmoveto{\pgfqpoint{3.161933in}{2.496820in}}%
\pgfpathlineto{\pgfqpoint{3.176322in}{2.485200in}}%
\pgfpathlineto{\pgfqpoint{3.190713in}{2.473610in}}%
\pgfpathlineto{\pgfqpoint{3.205107in}{2.462049in}}%
\pgfpathlineto{\pgfqpoint{3.219502in}{2.450517in}}%
\pgfpathlineto{\pgfqpoint{3.210176in}{2.480350in}}%
\pgfpathlineto{\pgfqpoint{3.200813in}{2.511132in}}%
\pgfpathlineto{\pgfqpoint{3.191415in}{2.542881in}}%
\pgfpathlineto{\pgfqpoint{3.176966in}{2.554817in}}%
\pgfpathlineto{\pgfqpoint{3.162520in}{2.566782in}}%
\pgfpathlineto{\pgfqpoint{3.148075in}{2.578776in}}%
\pgfpathlineto{\pgfqpoint{3.133633in}{2.590800in}}%
\pgfpathlineto{\pgfqpoint{3.143104in}{2.558507in}}%
\pgfpathlineto{\pgfqpoint{3.152536in}{2.527186in}}%
\pgfpathlineto{\pgfqpoint{3.161933in}{2.496820in}}%
\pgfpathclose%
\pgfusepath{fill}%
\end{pgfscope}%
\begin{pgfscope}%
\pgfpathrectangle{\pgfqpoint{1.150000in}{0.150000in}}{\pgfqpoint{5.700000in}{5.700000in}}%
\pgfusepath{clip}%
\pgfsetbuttcap%
\pgfsetroundjoin%
\definecolor{currentfill}{rgb}{0.255645,0.260703,0.528312}%
\pgfsetfillcolor{currentfill}%
\pgfsetfillopacity{0.700000}%
\pgfsetlinewidth{0.000000pt}%
\definecolor{currentstroke}{rgb}{0.000000,0.000000,0.000000}%
\pgfsetstrokecolor{currentstroke}%
\pgfsetdash{}{0pt}%
\pgfpathmoveto{\pgfqpoint{4.478228in}{1.441724in}}%
\pgfpathlineto{\pgfqpoint{4.492774in}{1.433730in}}%
\pgfpathlineto{\pgfqpoint{4.507326in}{1.425760in}}%
\pgfpathlineto{\pgfqpoint{4.521883in}{1.417813in}}%
\pgfpathlineto{\pgfqpoint{4.536445in}{1.409889in}}%
\pgfpathlineto{\pgfqpoint{4.528422in}{1.419911in}}%
\pgfpathlineto{\pgfqpoint{4.520392in}{1.430591in}}%
\pgfpathlineto{\pgfqpoint{4.512355in}{1.441945in}}%
\pgfpathlineto{\pgfqpoint{4.504311in}{1.453985in}}%
\pgfpathlineto{\pgfqpoint{4.489713in}{1.462374in}}%
\pgfpathlineto{\pgfqpoint{4.475120in}{1.470786in}}%
\pgfpathlineto{\pgfqpoint{4.460532in}{1.479221in}}%
\pgfpathlineto{\pgfqpoint{4.445948in}{1.487679in}}%
\pgfpathlineto{\pgfqpoint{4.454030in}{1.475167in}}%
\pgfpathlineto{\pgfqpoint{4.462103in}{1.463346in}}%
\pgfpathlineto{\pgfqpoint{4.470169in}{1.452203in}}%
\pgfpathlineto{\pgfqpoint{4.478228in}{1.441724in}}%
\pgfpathclose%
\pgfusepath{fill}%
\end{pgfscope}%
\begin{pgfscope}%
\pgfpathrectangle{\pgfqpoint{1.150000in}{0.150000in}}{\pgfqpoint{5.700000in}{5.700000in}}%
\pgfusepath{clip}%
\pgfsetbuttcap%
\pgfsetroundjoin%
\definecolor{currentfill}{rgb}{0.160665,0.478540,0.558115}%
\pgfsetfillcolor{currentfill}%
\pgfsetfillopacity{0.700000}%
\pgfsetlinewidth{0.000000pt}%
\definecolor{currentstroke}{rgb}{0.000000,0.000000,0.000000}%
\pgfsetstrokecolor{currentstroke}%
\pgfsetdash{}{0pt}%
\pgfpathmoveto{\pgfqpoint{3.716560in}{2.002504in}}%
\pgfpathlineto{\pgfqpoint{3.730990in}{1.992386in}}%
\pgfpathlineto{\pgfqpoint{3.745423in}{1.982294in}}%
\pgfpathlineto{\pgfqpoint{3.759860in}{1.972228in}}%
\pgfpathlineto{\pgfqpoint{3.774301in}{1.962188in}}%
\pgfpathlineto{\pgfqpoint{3.765639in}{1.983848in}}%
\pgfpathlineto{\pgfqpoint{3.756955in}{2.006341in}}%
\pgfpathlineto{\pgfqpoint{3.748249in}{2.029684in}}%
\pgfpathlineto{\pgfqpoint{3.739520in}{2.053891in}}%
\pgfpathlineto{\pgfqpoint{3.725023in}{2.064441in}}%
\pgfpathlineto{\pgfqpoint{3.710530in}{2.075017in}}%
\pgfpathlineto{\pgfqpoint{3.696039in}{2.085618in}}%
\pgfpathlineto{\pgfqpoint{3.681552in}{2.096246in}}%
\pgfpathlineto{\pgfqpoint{3.690339in}{2.071521in}}%
\pgfpathlineto{\pgfqpoint{3.699103in}{2.047666in}}%
\pgfpathlineto{\pgfqpoint{3.707843in}{2.024666in}}%
\pgfpathlineto{\pgfqpoint{3.716560in}{2.002504in}}%
\pgfpathclose%
\pgfusepath{fill}%
\end{pgfscope}%
\begin{pgfscope}%
\pgfpathrectangle{\pgfqpoint{1.150000in}{0.150000in}}{\pgfqpoint{5.700000in}{5.700000in}}%
\pgfusepath{clip}%
\pgfsetbuttcap%
\pgfsetroundjoin%
\definecolor{currentfill}{rgb}{0.214298,0.355619,0.551184}%
\pgfsetfillcolor{currentfill}%
\pgfsetfillopacity{0.700000}%
\pgfsetlinewidth{0.000000pt}%
\definecolor{currentstroke}{rgb}{0.000000,0.000000,0.000000}%
\pgfsetstrokecolor{currentstroke}%
\pgfsetdash{}{0pt}%
\pgfpathmoveto{\pgfqpoint{4.155268in}{1.661815in}}%
\pgfpathlineto{\pgfqpoint{4.169759in}{1.652881in}}%
\pgfpathlineto{\pgfqpoint{4.184254in}{1.643972in}}%
\pgfpathlineto{\pgfqpoint{4.198753in}{1.635087in}}%
\pgfpathlineto{\pgfqpoint{4.213258in}{1.626225in}}%
\pgfpathlineto{\pgfqpoint{4.205003in}{1.641351in}}%
\pgfpathlineto{\pgfqpoint{4.196736in}{1.657214in}}%
\pgfpathlineto{\pgfqpoint{4.188456in}{1.673830in}}%
\pgfpathlineto{\pgfqpoint{4.180163in}{1.691214in}}%
\pgfpathlineto{\pgfqpoint{4.165614in}{1.700560in}}%
\pgfpathlineto{\pgfqpoint{4.151070in}{1.709931in}}%
\pgfpathlineto{\pgfqpoint{4.136529in}{1.719325in}}%
\pgfpathlineto{\pgfqpoint{4.121993in}{1.728744in}}%
\pgfpathlineto{\pgfqpoint{4.130332in}{1.710868in}}%
\pgfpathlineto{\pgfqpoint{4.138658in}{1.693765in}}%
\pgfpathlineto{\pgfqpoint{4.146969in}{1.677419in}}%
\pgfpathlineto{\pgfqpoint{4.155268in}{1.661815in}}%
\pgfpathclose%
\pgfusepath{fill}%
\end{pgfscope}%
\begin{pgfscope}%
\pgfpathrectangle{\pgfqpoint{1.150000in}{0.150000in}}{\pgfqpoint{5.700000in}{5.700000in}}%
\pgfusepath{clip}%
\pgfsetbuttcap%
\pgfsetroundjoin%
\definecolor{currentfill}{rgb}{0.122312,0.633153,0.530398}%
\pgfsetfillcolor{currentfill}%
\pgfsetfillopacity{0.700000}%
\pgfsetlinewidth{0.000000pt}%
\definecolor{currentstroke}{rgb}{0.000000,0.000000,0.000000}%
\pgfsetstrokecolor{currentstroke}%
\pgfsetdash{}{0pt}%
\pgfpathmoveto{\pgfqpoint{3.219502in}{2.450517in}}%
\pgfpathlineto{\pgfqpoint{3.233901in}{2.439014in}}%
\pgfpathlineto{\pgfqpoint{3.248301in}{2.427540in}}%
\pgfpathlineto{\pgfqpoint{3.262704in}{2.416095in}}%
\pgfpathlineto{\pgfqpoint{3.277110in}{2.404678in}}%
\pgfpathlineto{\pgfqpoint{3.267851in}{2.433980in}}%
\pgfpathlineto{\pgfqpoint{3.258558in}{2.464225in}}%
\pgfpathlineto{\pgfqpoint{3.249230in}{2.495431in}}%
\pgfpathlineto{\pgfqpoint{3.234773in}{2.507250in}}%
\pgfpathlineto{\pgfqpoint{3.220318in}{2.519098in}}%
\pgfpathlineto{\pgfqpoint{3.205865in}{2.530975in}}%
\pgfpathlineto{\pgfqpoint{3.191415in}{2.542881in}}%
\pgfpathlineto{\pgfqpoint{3.200813in}{2.511132in}}%
\pgfpathlineto{\pgfqpoint{3.210176in}{2.480350in}}%
\pgfpathlineto{\pgfqpoint{3.219502in}{2.450517in}}%
\pgfpathclose%
\pgfusepath{fill}%
\end{pgfscope}%
\begin{pgfscope}%
\pgfpathrectangle{\pgfqpoint{1.150000in}{0.150000in}}{\pgfqpoint{5.700000in}{5.700000in}}%
\pgfusepath{clip}%
\pgfsetbuttcap%
\pgfsetroundjoin%
\definecolor{currentfill}{rgb}{0.277134,0.185228,0.489898}%
\pgfsetfillcolor{currentfill}%
\pgfsetfillopacity{0.700000}%
\pgfsetlinewidth{0.000000pt}%
\definecolor{currentstroke}{rgb}{0.000000,0.000000,0.000000}%
\pgfsetstrokecolor{currentstroke}%
\pgfsetdash{}{0pt}%
\pgfpathmoveto{\pgfqpoint{4.801702in}{1.259660in}}%
\pgfpathlineto{\pgfqpoint{4.816327in}{1.252575in}}%
\pgfpathlineto{\pgfqpoint{4.830958in}{1.245514in}}%
\pgfpathlineto{\pgfqpoint{4.845595in}{1.238476in}}%
\pgfpathlineto{\pgfqpoint{4.837734in}{1.243896in}}%
\pgfpathlineto{\pgfqpoint{4.829872in}{1.249904in}}%
\pgfpathlineto{\pgfqpoint{4.822007in}{1.256513in}}%
\pgfpathlineto{\pgfqpoint{4.814140in}{1.263734in}}%
\pgfpathlineto{\pgfqpoint{4.799475in}{1.271219in}}%
\pgfpathlineto{\pgfqpoint{4.784815in}{1.278728in}}%
\pgfpathlineto{\pgfqpoint{4.770162in}{1.286259in}}%
\pgfpathlineto{\pgfqpoint{4.778051in}{1.278698in}}%
\pgfpathlineto{\pgfqpoint{4.785937in}{1.271753in}}%
\pgfpathlineto{\pgfqpoint{4.793821in}{1.265411in}}%
\pgfpathlineto{\pgfqpoint{4.801702in}{1.259660in}}%
\pgfpathclose%
\pgfusepath{fill}%
\end{pgfscope}%
\begin{pgfscope}%
\pgfpathrectangle{\pgfqpoint{1.150000in}{0.150000in}}{\pgfqpoint{5.700000in}{5.700000in}}%
\pgfusepath{clip}%
\pgfsetbuttcap%
\pgfsetroundjoin%
\definecolor{currentfill}{rgb}{0.165117,0.467423,0.558141}%
\pgfsetfillcolor{currentfill}%
\pgfsetfillopacity{0.700000}%
\pgfsetlinewidth{0.000000pt}%
\definecolor{currentstroke}{rgb}{0.000000,0.000000,0.000000}%
\pgfsetstrokecolor{currentstroke}%
\pgfsetdash{}{0pt}%
\pgfpathmoveto{\pgfqpoint{3.774301in}{1.962188in}}%
\pgfpathlineto{\pgfqpoint{3.788745in}{1.952173in}}%
\pgfpathlineto{\pgfqpoint{3.803193in}{1.942184in}}%
\pgfpathlineto{\pgfqpoint{3.817644in}{1.932220in}}%
\pgfpathlineto{\pgfqpoint{3.832099in}{1.922281in}}%
\pgfpathlineto{\pgfqpoint{3.823490in}{1.943440in}}%
\pgfpathlineto{\pgfqpoint{3.814862in}{1.965427in}}%
\pgfpathlineto{\pgfqpoint{3.806212in}{1.988257in}}%
\pgfpathlineto{\pgfqpoint{3.797540in}{2.011948in}}%
\pgfpathlineto{\pgfqpoint{3.783030in}{2.022396in}}%
\pgfpathlineto{\pgfqpoint{3.768523in}{2.032868in}}%
\pgfpathlineto{\pgfqpoint{3.754020in}{2.043367in}}%
\pgfpathlineto{\pgfqpoint{3.739520in}{2.053891in}}%
\pgfpathlineto{\pgfqpoint{3.748249in}{2.029684in}}%
\pgfpathlineto{\pgfqpoint{3.756955in}{2.006341in}}%
\pgfpathlineto{\pgfqpoint{3.765639in}{1.983848in}}%
\pgfpathlineto{\pgfqpoint{3.774301in}{1.962188in}}%
\pgfpathclose%
\pgfusepath{fill}%
\end{pgfscope}%
\begin{pgfscope}%
\pgfpathrectangle{\pgfqpoint{1.150000in}{0.150000in}}{\pgfqpoint{5.700000in}{5.700000in}}%
\pgfusepath{clip}%
\pgfsetbuttcap%
\pgfsetroundjoin%
\definecolor{currentfill}{rgb}{0.258965,0.251537,0.524736}%
\pgfsetfillcolor{currentfill}%
\pgfsetfillopacity{0.700000}%
\pgfsetlinewidth{0.000000pt}%
\definecolor{currentstroke}{rgb}{0.000000,0.000000,0.000000}%
\pgfsetstrokecolor{currentstroke}%
\pgfsetdash{}{0pt}%
\pgfpathmoveto{\pgfqpoint{4.536445in}{1.409889in}}%
\pgfpathlineto{\pgfqpoint{4.551012in}{1.401989in}}%
\pgfpathlineto{\pgfqpoint{4.565585in}{1.394112in}}%
\pgfpathlineto{\pgfqpoint{4.580162in}{1.386258in}}%
\pgfpathlineto{\pgfqpoint{4.594745in}{1.378427in}}%
\pgfpathlineto{\pgfqpoint{4.586756in}{1.387990in}}%
\pgfpathlineto{\pgfqpoint{4.578761in}{1.398209in}}%
\pgfpathlineto{\pgfqpoint{4.570761in}{1.409095in}}%
\pgfpathlineto{\pgfqpoint{4.562754in}{1.420665in}}%
\pgfpathlineto{\pgfqpoint{4.548136in}{1.428960in}}%
\pgfpathlineto{\pgfqpoint{4.533523in}{1.437279in}}%
\pgfpathlineto{\pgfqpoint{4.518914in}{1.445620in}}%
\pgfpathlineto{\pgfqpoint{4.504311in}{1.453985in}}%
\pgfpathlineto{\pgfqpoint{4.512355in}{1.441945in}}%
\pgfpathlineto{\pgfqpoint{4.520392in}{1.430591in}}%
\pgfpathlineto{\pgfqpoint{4.528422in}{1.419911in}}%
\pgfpathlineto{\pgfqpoint{4.536445in}{1.409889in}}%
\pgfpathclose%
\pgfusepath{fill}%
\end{pgfscope}%
\begin{pgfscope}%
\pgfpathrectangle{\pgfqpoint{1.150000in}{0.150000in}}{\pgfqpoint{5.700000in}{5.700000in}}%
\pgfusepath{clip}%
\pgfsetbuttcap%
\pgfsetroundjoin%
\definecolor{currentfill}{rgb}{0.119699,0.618490,0.536347}%
\pgfsetfillcolor{currentfill}%
\pgfsetfillopacity{0.700000}%
\pgfsetlinewidth{0.000000pt}%
\definecolor{currentstroke}{rgb}{0.000000,0.000000,0.000000}%
\pgfsetstrokecolor{currentstroke}%
\pgfsetdash{}{0pt}%
\pgfpathmoveto{\pgfqpoint{3.277110in}{2.404678in}}%
\pgfpathlineto{\pgfqpoint{3.291518in}{2.393291in}}%
\pgfpathlineto{\pgfqpoint{3.305928in}{2.381931in}}%
\pgfpathlineto{\pgfqpoint{3.320341in}{2.370600in}}%
\pgfpathlineto{\pgfqpoint{3.334756in}{2.359297in}}%
\pgfpathlineto{\pgfqpoint{3.325564in}{2.388069in}}%
\pgfpathlineto{\pgfqpoint{3.316340in}{2.417778in}}%
\pgfpathlineto{\pgfqpoint{3.307082in}{2.448441in}}%
\pgfpathlineto{\pgfqpoint{3.292615in}{2.460146in}}%
\pgfpathlineto{\pgfqpoint{3.278151in}{2.471879in}}%
\pgfpathlineto{\pgfqpoint{3.263690in}{2.483641in}}%
\pgfpathlineto{\pgfqpoint{3.249230in}{2.495431in}}%
\pgfpathlineto{\pgfqpoint{3.258558in}{2.464225in}}%
\pgfpathlineto{\pgfqpoint{3.267851in}{2.433980in}}%
\pgfpathlineto{\pgfqpoint{3.277110in}{2.404678in}}%
\pgfpathclose%
\pgfusepath{fill}%
\end{pgfscope}%
\begin{pgfscope}%
\pgfpathrectangle{\pgfqpoint{1.150000in}{0.150000in}}{\pgfqpoint{5.700000in}{5.700000in}}%
\pgfusepath{clip}%
\pgfsetbuttcap%
\pgfsetroundjoin%
\definecolor{currentfill}{rgb}{0.220057,0.343307,0.549413}%
\pgfsetfillcolor{currentfill}%
\pgfsetfillopacity{0.700000}%
\pgfsetlinewidth{0.000000pt}%
\definecolor{currentstroke}{rgb}{0.000000,0.000000,0.000000}%
\pgfsetstrokecolor{currentstroke}%
\pgfsetdash{}{0pt}%
\pgfpathmoveto{\pgfqpoint{4.213258in}{1.626225in}}%
\pgfpathlineto{\pgfqpoint{4.227766in}{1.617388in}}%
\pgfpathlineto{\pgfqpoint{4.242279in}{1.608574in}}%
\pgfpathlineto{\pgfqpoint{4.256797in}{1.599785in}}%
\pgfpathlineto{\pgfqpoint{4.271319in}{1.591019in}}%
\pgfpathlineto{\pgfqpoint{4.263107in}{1.605667in}}%
\pgfpathlineto{\pgfqpoint{4.254884in}{1.621048in}}%
\pgfpathlineto{\pgfqpoint{4.246649in}{1.637177in}}%
\pgfpathlineto{\pgfqpoint{4.238401in}{1.654069in}}%
\pgfpathlineto{\pgfqpoint{4.223835in}{1.663319in}}%
\pgfpathlineto{\pgfqpoint{4.209274in}{1.672593in}}%
\pgfpathlineto{\pgfqpoint{4.194716in}{1.681891in}}%
\pgfpathlineto{\pgfqpoint{4.180163in}{1.691214in}}%
\pgfpathlineto{\pgfqpoint{4.188456in}{1.673830in}}%
\pgfpathlineto{\pgfqpoint{4.196736in}{1.657214in}}%
\pgfpathlineto{\pgfqpoint{4.205003in}{1.641351in}}%
\pgfpathlineto{\pgfqpoint{4.213258in}{1.626225in}}%
\pgfpathclose%
\pgfusepath{fill}%
\end{pgfscope}%
\begin{pgfscope}%
\pgfpathrectangle{\pgfqpoint{1.150000in}{0.150000in}}{\pgfqpoint{5.700000in}{5.700000in}}%
\pgfusepath{clip}%
\pgfsetbuttcap%
\pgfsetroundjoin%
\definecolor{currentfill}{rgb}{0.119512,0.607464,0.540218}%
\pgfsetfillcolor{currentfill}%
\pgfsetfillopacity{0.700000}%
\pgfsetlinewidth{0.000000pt}%
\definecolor{currentstroke}{rgb}{0.000000,0.000000,0.000000}%
\pgfsetstrokecolor{currentstroke}%
\pgfsetdash{}{0pt}%
\pgfpathmoveto{\pgfqpoint{3.334756in}{2.359297in}}%
\pgfpathlineto{\pgfqpoint{3.349174in}{2.348023in}}%
\pgfpathlineto{\pgfqpoint{3.363595in}{2.336776in}}%
\pgfpathlineto{\pgfqpoint{3.378018in}{2.325557in}}%
\pgfpathlineto{\pgfqpoint{3.392443in}{2.314366in}}%
\pgfpathlineto{\pgfqpoint{3.383318in}{2.342609in}}%
\pgfpathlineto{\pgfqpoint{3.374160in}{2.371782in}}%
\pgfpathlineto{\pgfqpoint{3.364971in}{2.401906in}}%
\pgfpathlineto{\pgfqpoint{3.350495in}{2.413498in}}%
\pgfpathlineto{\pgfqpoint{3.336022in}{2.425117in}}%
\pgfpathlineto{\pgfqpoint{3.321550in}{2.436765in}}%
\pgfpathlineto{\pgfqpoint{3.307082in}{2.448441in}}%
\pgfpathlineto{\pgfqpoint{3.316340in}{2.417778in}}%
\pgfpathlineto{\pgfqpoint{3.325564in}{2.388069in}}%
\pgfpathlineto{\pgfqpoint{3.334756in}{2.359297in}}%
\pgfpathclose%
\pgfusepath{fill}%
\end{pgfscope}%
\begin{pgfscope}%
\pgfpathrectangle{\pgfqpoint{1.150000in}{0.150000in}}{\pgfqpoint{5.700000in}{5.700000in}}%
\pgfusepath{clip}%
\pgfsetbuttcap%
\pgfsetroundjoin%
\definecolor{currentfill}{rgb}{0.168126,0.459988,0.558082}%
\pgfsetfillcolor{currentfill}%
\pgfsetfillopacity{0.700000}%
\pgfsetlinewidth{0.000000pt}%
\definecolor{currentstroke}{rgb}{0.000000,0.000000,0.000000}%
\pgfsetstrokecolor{currentstroke}%
\pgfsetdash{}{0pt}%
\pgfpathmoveto{\pgfqpoint{3.832099in}{1.922281in}}%
\pgfpathlineto{\pgfqpoint{3.846557in}{1.912367in}}%
\pgfpathlineto{\pgfqpoint{3.861019in}{1.902479in}}%
\pgfpathlineto{\pgfqpoint{3.875485in}{1.892616in}}%
\pgfpathlineto{\pgfqpoint{3.889955in}{1.882778in}}%
\pgfpathlineto{\pgfqpoint{3.881399in}{1.903437in}}%
\pgfpathlineto{\pgfqpoint{3.872825in}{1.924918in}}%
\pgfpathlineto{\pgfqpoint{3.864230in}{1.947238in}}%
\pgfpathlineto{\pgfqpoint{3.855615in}{1.970413in}}%
\pgfpathlineto{\pgfqpoint{3.841091in}{1.980759in}}%
\pgfpathlineto{\pgfqpoint{3.826571in}{1.991130in}}%
\pgfpathlineto{\pgfqpoint{3.812054in}{2.001526in}}%
\pgfpathlineto{\pgfqpoint{3.797540in}{2.011948in}}%
\pgfpathlineto{\pgfqpoint{3.806212in}{1.988257in}}%
\pgfpathlineto{\pgfqpoint{3.814862in}{1.965427in}}%
\pgfpathlineto{\pgfqpoint{3.823490in}{1.943440in}}%
\pgfpathlineto{\pgfqpoint{3.832099in}{1.922281in}}%
\pgfpathclose%
\pgfusepath{fill}%
\end{pgfscope}%
\begin{pgfscope}%
\pgfpathrectangle{\pgfqpoint{1.150000in}{0.150000in}}{\pgfqpoint{5.700000in}{5.700000in}}%
\pgfusepath{clip}%
\pgfsetbuttcap%
\pgfsetroundjoin%
\definecolor{currentfill}{rgb}{0.260571,0.246922,0.522828}%
\pgfsetfillcolor{currentfill}%
\pgfsetfillopacity{0.700000}%
\pgfsetlinewidth{0.000000pt}%
\definecolor{currentstroke}{rgb}{0.000000,0.000000,0.000000}%
\pgfsetstrokecolor{currentstroke}%
\pgfsetdash{}{0pt}%
\pgfpathmoveto{\pgfqpoint{4.594745in}{1.378427in}}%
\pgfpathlineto{\pgfqpoint{4.609334in}{1.370620in}}%
\pgfpathlineto{\pgfqpoint{4.623927in}{1.362835in}}%
\pgfpathlineto{\pgfqpoint{4.638526in}{1.355074in}}%
\pgfpathlineto{\pgfqpoint{4.653131in}{1.347336in}}%
\pgfpathlineto{\pgfqpoint{4.645175in}{1.356441in}}%
\pgfpathlineto{\pgfqpoint{4.637214in}{1.366197in}}%
\pgfpathlineto{\pgfqpoint{4.629248in}{1.376617in}}%
\pgfpathlineto{\pgfqpoint{4.621277in}{1.387716in}}%
\pgfpathlineto{\pgfqpoint{4.606639in}{1.395919in}}%
\pgfpathlineto{\pgfqpoint{4.592005in}{1.404144in}}%
\pgfpathlineto{\pgfqpoint{4.577377in}{1.412393in}}%
\pgfpathlineto{\pgfqpoint{4.562754in}{1.420665in}}%
\pgfpathlineto{\pgfqpoint{4.570761in}{1.409095in}}%
\pgfpathlineto{\pgfqpoint{4.578761in}{1.398209in}}%
\pgfpathlineto{\pgfqpoint{4.586756in}{1.387990in}}%
\pgfpathlineto{\pgfqpoint{4.594745in}{1.378427in}}%
\pgfpathclose%
\pgfusepath{fill}%
\end{pgfscope}%
\begin{pgfscope}%
\pgfpathrectangle{\pgfqpoint{1.150000in}{0.150000in}}{\pgfqpoint{5.700000in}{5.700000in}}%
\pgfusepath{clip}%
\pgfsetbuttcap%
\pgfsetroundjoin%
\definecolor{currentfill}{rgb}{0.223925,0.334994,0.548053}%
\pgfsetfillcolor{currentfill}%
\pgfsetfillopacity{0.700000}%
\pgfsetlinewidth{0.000000pt}%
\definecolor{currentstroke}{rgb}{0.000000,0.000000,0.000000}%
\pgfsetstrokecolor{currentstroke}%
\pgfsetdash{}{0pt}%
\pgfpathmoveto{\pgfqpoint{4.271319in}{1.591019in}}%
\pgfpathlineto{\pgfqpoint{4.285846in}{1.582277in}}%
\pgfpathlineto{\pgfqpoint{4.300377in}{1.573559in}}%
\pgfpathlineto{\pgfqpoint{4.314913in}{1.564865in}}%
\pgfpathlineto{\pgfqpoint{4.329454in}{1.556194in}}%
\pgfpathlineto{\pgfqpoint{4.321283in}{1.570365in}}%
\pgfpathlineto{\pgfqpoint{4.313103in}{1.585264in}}%
\pgfpathlineto{\pgfqpoint{4.304911in}{1.600907in}}%
\pgfpathlineto{\pgfqpoint{4.296709in}{1.617307in}}%
\pgfpathlineto{\pgfqpoint{4.282125in}{1.626462in}}%
\pgfpathlineto{\pgfqpoint{4.267546in}{1.635640in}}%
\pgfpathlineto{\pgfqpoint{4.252972in}{1.644843in}}%
\pgfpathlineto{\pgfqpoint{4.238401in}{1.654069in}}%
\pgfpathlineto{\pgfqpoint{4.246649in}{1.637177in}}%
\pgfpathlineto{\pgfqpoint{4.254884in}{1.621048in}}%
\pgfpathlineto{\pgfqpoint{4.263107in}{1.605667in}}%
\pgfpathlineto{\pgfqpoint{4.271319in}{1.591019in}}%
\pgfpathclose%
\pgfusepath{fill}%
\end{pgfscope}%
\begin{pgfscope}%
\pgfpathrectangle{\pgfqpoint{1.150000in}{0.150000in}}{\pgfqpoint{5.700000in}{5.700000in}}%
\pgfusepath{clip}%
\pgfsetbuttcap%
\pgfsetroundjoin%
\definecolor{currentfill}{rgb}{0.121148,0.592739,0.544641}%
\pgfsetfillcolor{currentfill}%
\pgfsetfillopacity{0.700000}%
\pgfsetlinewidth{0.000000pt}%
\definecolor{currentstroke}{rgb}{0.000000,0.000000,0.000000}%
\pgfsetstrokecolor{currentstroke}%
\pgfsetdash{}{0pt}%
\pgfpathmoveto{\pgfqpoint{3.392443in}{2.314366in}}%
\pgfpathlineto{\pgfqpoint{3.406872in}{2.303203in}}%
\pgfpathlineto{\pgfqpoint{3.421303in}{2.292067in}}%
\pgfpathlineto{\pgfqpoint{3.435737in}{2.280959in}}%
\pgfpathlineto{\pgfqpoint{3.450173in}{2.269879in}}%
\pgfpathlineto{\pgfqpoint{3.441112in}{2.297593in}}%
\pgfpathlineto{\pgfqpoint{3.432021in}{2.326233in}}%
\pgfpathlineto{\pgfqpoint{3.422899in}{2.355817in}}%
\pgfpathlineto{\pgfqpoint{3.408414in}{2.367298in}}%
\pgfpathlineto{\pgfqpoint{3.393930in}{2.378806in}}%
\pgfpathlineto{\pgfqpoint{3.379449in}{2.390342in}}%
\pgfpathlineto{\pgfqpoint{3.364971in}{2.401906in}}%
\pgfpathlineto{\pgfqpoint{3.374160in}{2.371782in}}%
\pgfpathlineto{\pgfqpoint{3.383318in}{2.342609in}}%
\pgfpathlineto{\pgfqpoint{3.392443in}{2.314366in}}%
\pgfpathclose%
\pgfusepath{fill}%
\end{pgfscope}%
\begin{pgfscope}%
\pgfpathrectangle{\pgfqpoint{1.150000in}{0.150000in}}{\pgfqpoint{5.700000in}{5.700000in}}%
\pgfusepath{clip}%
\pgfsetbuttcap%
\pgfsetroundjoin%
\definecolor{currentfill}{rgb}{0.172719,0.448791,0.557885}%
\pgfsetfillcolor{currentfill}%
\pgfsetfillopacity{0.700000}%
\pgfsetlinewidth{0.000000pt}%
\definecolor{currentstroke}{rgb}{0.000000,0.000000,0.000000}%
\pgfsetstrokecolor{currentstroke}%
\pgfsetdash{}{0pt}%
\pgfpathmoveto{\pgfqpoint{3.889955in}{1.882778in}}%
\pgfpathlineto{\pgfqpoint{3.904428in}{1.872966in}}%
\pgfpathlineto{\pgfqpoint{3.918905in}{1.863178in}}%
\pgfpathlineto{\pgfqpoint{3.933386in}{1.853415in}}%
\pgfpathlineto{\pgfqpoint{3.947871in}{1.843677in}}%
\pgfpathlineto{\pgfqpoint{3.939367in}{1.863836in}}%
\pgfpathlineto{\pgfqpoint{3.930845in}{1.884812in}}%
\pgfpathlineto{\pgfqpoint{3.922305in}{1.906622in}}%
\pgfpathlineto{\pgfqpoint{3.913745in}{1.929282in}}%
\pgfpathlineto{\pgfqpoint{3.899207in}{1.939527in}}%
\pgfpathlineto{\pgfqpoint{3.884673in}{1.949797in}}%
\pgfpathlineto{\pgfqpoint{3.870142in}{1.960093in}}%
\pgfpathlineto{\pgfqpoint{3.855615in}{1.970413in}}%
\pgfpathlineto{\pgfqpoint{3.864230in}{1.947238in}}%
\pgfpathlineto{\pgfqpoint{3.872825in}{1.924918in}}%
\pgfpathlineto{\pgfqpoint{3.881399in}{1.903437in}}%
\pgfpathlineto{\pgfqpoint{3.889955in}{1.882778in}}%
\pgfpathclose%
\pgfusepath{fill}%
\end{pgfscope}%
\begin{pgfscope}%
\pgfpathrectangle{\pgfqpoint{1.150000in}{0.150000in}}{\pgfqpoint{5.700000in}{5.700000in}}%
\pgfusepath{clip}%
\pgfsetbuttcap%
\pgfsetroundjoin%
\definecolor{currentfill}{rgb}{0.123463,0.581687,0.547445}%
\pgfsetfillcolor{currentfill}%
\pgfsetfillopacity{0.700000}%
\pgfsetlinewidth{0.000000pt}%
\definecolor{currentstroke}{rgb}{0.000000,0.000000,0.000000}%
\pgfsetstrokecolor{currentstroke}%
\pgfsetdash{}{0pt}%
\pgfpathmoveto{\pgfqpoint{3.450173in}{2.269879in}}%
\pgfpathlineto{\pgfqpoint{3.464613in}{2.258825in}}%
\pgfpathlineto{\pgfqpoint{3.479055in}{2.247799in}}%
\pgfpathlineto{\pgfqpoint{3.493500in}{2.236800in}}%
\pgfpathlineto{\pgfqpoint{3.507947in}{2.225829in}}%
\pgfpathlineto{\pgfqpoint{3.498950in}{2.253016in}}%
\pgfpathlineto{\pgfqpoint{3.489925in}{2.281123in}}%
\pgfpathlineto{\pgfqpoint{3.480869in}{2.310169in}}%
\pgfpathlineto{\pgfqpoint{3.466373in}{2.321540in}}%
\pgfpathlineto{\pgfqpoint{3.451879in}{2.332939in}}%
\pgfpathlineto{\pgfqpoint{3.437388in}{2.344364in}}%
\pgfpathlineto{\pgfqpoint{3.422899in}{2.355817in}}%
\pgfpathlineto{\pgfqpoint{3.432021in}{2.326233in}}%
\pgfpathlineto{\pgfqpoint{3.441112in}{2.297593in}}%
\pgfpathlineto{\pgfqpoint{3.450173in}{2.269879in}}%
\pgfpathclose%
\pgfusepath{fill}%
\end{pgfscope}%
\begin{pgfscope}%
\pgfpathrectangle{\pgfqpoint{1.150000in}{0.150000in}}{\pgfqpoint{5.700000in}{5.700000in}}%
\pgfusepath{clip}%
\pgfsetbuttcap%
\pgfsetroundjoin%
\definecolor{currentfill}{rgb}{0.227802,0.326594,0.546532}%
\pgfsetfillcolor{currentfill}%
\pgfsetfillopacity{0.700000}%
\pgfsetlinewidth{0.000000pt}%
\definecolor{currentstroke}{rgb}{0.000000,0.000000,0.000000}%
\pgfsetstrokecolor{currentstroke}%
\pgfsetdash{}{0pt}%
\pgfpathmoveto{\pgfqpoint{4.329454in}{1.556194in}}%
\pgfpathlineto{\pgfqpoint{4.343999in}{1.547547in}}%
\pgfpathlineto{\pgfqpoint{4.358549in}{1.538924in}}%
\pgfpathlineto{\pgfqpoint{4.373103in}{1.530324in}}%
\pgfpathlineto{\pgfqpoint{4.387663in}{1.521748in}}%
\pgfpathlineto{\pgfqpoint{4.379533in}{1.535442in}}%
\pgfpathlineto{\pgfqpoint{4.371394in}{1.549860in}}%
\pgfpathlineto{\pgfqpoint{4.363246in}{1.565017in}}%
\pgfpathlineto{\pgfqpoint{4.355087in}{1.580927in}}%
\pgfpathlineto{\pgfqpoint{4.340486in}{1.589986in}}%
\pgfpathlineto{\pgfqpoint{4.325889in}{1.599070in}}%
\pgfpathlineto{\pgfqpoint{4.311297in}{1.608177in}}%
\pgfpathlineto{\pgfqpoint{4.296709in}{1.617307in}}%
\pgfpathlineto{\pgfqpoint{4.304911in}{1.600907in}}%
\pgfpathlineto{\pgfqpoint{4.313103in}{1.585264in}}%
\pgfpathlineto{\pgfqpoint{4.321283in}{1.570365in}}%
\pgfpathlineto{\pgfqpoint{4.329454in}{1.556194in}}%
\pgfpathclose%
\pgfusepath{fill}%
\end{pgfscope}%
\begin{pgfscope}%
\pgfpathrectangle{\pgfqpoint{1.150000in}{0.150000in}}{\pgfqpoint{5.700000in}{5.700000in}}%
\pgfusepath{clip}%
\pgfsetbuttcap%
\pgfsetroundjoin%
\definecolor{currentfill}{rgb}{0.263663,0.237631,0.518762}%
\pgfsetfillcolor{currentfill}%
\pgfsetfillopacity{0.700000}%
\pgfsetlinewidth{0.000000pt}%
\definecolor{currentstroke}{rgb}{0.000000,0.000000,0.000000}%
\pgfsetstrokecolor{currentstroke}%
\pgfsetdash{}{0pt}%
\pgfpathmoveto{\pgfqpoint{4.653131in}{1.347336in}}%
\pgfpathlineto{\pgfqpoint{4.667740in}{1.339621in}}%
\pgfpathlineto{\pgfqpoint{4.682356in}{1.331929in}}%
\pgfpathlineto{\pgfqpoint{4.696976in}{1.324259in}}%
\pgfpathlineto{\pgfqpoint{4.711602in}{1.316613in}}%
\pgfpathlineto{\pgfqpoint{4.703679in}{1.325261in}}%
\pgfpathlineto{\pgfqpoint{4.695751in}{1.334555in}}%
\pgfpathlineto{\pgfqpoint{4.687819in}{1.344509in}}%
\pgfpathlineto{\pgfqpoint{4.679883in}{1.355138in}}%
\pgfpathlineto{\pgfqpoint{4.665224in}{1.363248in}}%
\pgfpathlineto{\pgfqpoint{4.650570in}{1.371381in}}%
\pgfpathlineto{\pgfqpoint{4.635921in}{1.379537in}}%
\pgfpathlineto{\pgfqpoint{4.621277in}{1.387716in}}%
\pgfpathlineto{\pgfqpoint{4.629248in}{1.376617in}}%
\pgfpathlineto{\pgfqpoint{4.637214in}{1.366197in}}%
\pgfpathlineto{\pgfqpoint{4.645175in}{1.356441in}}%
\pgfpathlineto{\pgfqpoint{4.653131in}{1.347336in}}%
\pgfpathclose%
\pgfusepath{fill}%
\end{pgfscope}%
\begin{pgfscope}%
\pgfpathrectangle{\pgfqpoint{1.150000in}{0.150000in}}{\pgfqpoint{5.700000in}{5.700000in}}%
\pgfusepath{clip}%
\pgfsetbuttcap%
\pgfsetroundjoin%
\definecolor{currentfill}{rgb}{0.177423,0.437527,0.557565}%
\pgfsetfillcolor{currentfill}%
\pgfsetfillopacity{0.700000}%
\pgfsetlinewidth{0.000000pt}%
\definecolor{currentstroke}{rgb}{0.000000,0.000000,0.000000}%
\pgfsetstrokecolor{currentstroke}%
\pgfsetdash{}{0pt}%
\pgfpathmoveto{\pgfqpoint{3.947871in}{1.843677in}}%
\pgfpathlineto{\pgfqpoint{3.962359in}{1.833964in}}%
\pgfpathlineto{\pgfqpoint{3.976852in}{1.824275in}}%
\pgfpathlineto{\pgfqpoint{3.991348in}{1.814612in}}%
\pgfpathlineto{\pgfqpoint{4.005848in}{1.804973in}}%
\pgfpathlineto{\pgfqpoint{3.997395in}{1.824632in}}%
\pgfpathlineto{\pgfqpoint{3.988925in}{1.845105in}}%
\pgfpathlineto{\pgfqpoint{3.980438in}{1.866406in}}%
\pgfpathlineto{\pgfqpoint{3.971932in}{1.888552in}}%
\pgfpathlineto{\pgfqpoint{3.957380in}{1.898697in}}%
\pgfpathlineto{\pgfqpoint{3.942831in}{1.908867in}}%
\pgfpathlineto{\pgfqpoint{3.928286in}{1.919062in}}%
\pgfpathlineto{\pgfqpoint{3.913745in}{1.929282in}}%
\pgfpathlineto{\pgfqpoint{3.922305in}{1.906622in}}%
\pgfpathlineto{\pgfqpoint{3.930845in}{1.884812in}}%
\pgfpathlineto{\pgfqpoint{3.939367in}{1.863836in}}%
\pgfpathlineto{\pgfqpoint{3.947871in}{1.843677in}}%
\pgfpathclose%
\pgfusepath{fill}%
\end{pgfscope}%
\begin{pgfscope}%
\pgfpathrectangle{\pgfqpoint{1.150000in}{0.150000in}}{\pgfqpoint{5.700000in}{5.700000in}}%
\pgfusepath{clip}%
\pgfsetbuttcap%
\pgfsetroundjoin%
\definecolor{currentfill}{rgb}{0.126453,0.570633,0.549841}%
\pgfsetfillcolor{currentfill}%
\pgfsetfillopacity{0.700000}%
\pgfsetlinewidth{0.000000pt}%
\definecolor{currentstroke}{rgb}{0.000000,0.000000,0.000000}%
\pgfsetstrokecolor{currentstroke}%
\pgfsetdash{}{0pt}%
\pgfpathmoveto{\pgfqpoint{3.507947in}{2.225829in}}%
\pgfpathlineto{\pgfqpoint{3.522398in}{2.214884in}}%
\pgfpathlineto{\pgfqpoint{3.536852in}{2.203966in}}%
\pgfpathlineto{\pgfqpoint{3.551308in}{2.193074in}}%
\pgfpathlineto{\pgfqpoint{3.565767in}{2.182210in}}%
\pgfpathlineto{\pgfqpoint{3.556833in}{2.208871in}}%
\pgfpathlineto{\pgfqpoint{3.547871in}{2.236448in}}%
\pgfpathlineto{\pgfqpoint{3.538881in}{2.264956in}}%
\pgfpathlineto{\pgfqpoint{3.524374in}{2.276219in}}%
\pgfpathlineto{\pgfqpoint{3.509870in}{2.287509in}}%
\pgfpathlineto{\pgfqpoint{3.495368in}{2.298826in}}%
\pgfpathlineto{\pgfqpoint{3.480869in}{2.310169in}}%
\pgfpathlineto{\pgfqpoint{3.489925in}{2.281123in}}%
\pgfpathlineto{\pgfqpoint{3.498950in}{2.253016in}}%
\pgfpathlineto{\pgfqpoint{3.507947in}{2.225829in}}%
\pgfpathclose%
\pgfusepath{fill}%
\end{pgfscope}%
\begin{pgfscope}%
\pgfpathrectangle{\pgfqpoint{1.150000in}{0.150000in}}{\pgfqpoint{5.700000in}{5.700000in}}%
\pgfusepath{clip}%
\pgfsetbuttcap%
\pgfsetroundjoin%
\definecolor{currentfill}{rgb}{0.231674,0.318106,0.544834}%
\pgfsetfillcolor{currentfill}%
\pgfsetfillopacity{0.700000}%
\pgfsetlinewidth{0.000000pt}%
\definecolor{currentstroke}{rgb}{0.000000,0.000000,0.000000}%
\pgfsetstrokecolor{currentstroke}%
\pgfsetdash{}{0pt}%
\pgfpathmoveto{\pgfqpoint{4.387663in}{1.521748in}}%
\pgfpathlineto{\pgfqpoint{4.402227in}{1.513196in}}%
\pgfpathlineto{\pgfqpoint{4.416796in}{1.504667in}}%
\pgfpathlineto{\pgfqpoint{4.431370in}{1.496161in}}%
\pgfpathlineto{\pgfqpoint{4.445948in}{1.487679in}}%
\pgfpathlineto{\pgfqpoint{4.437858in}{1.500897in}}%
\pgfpathlineto{\pgfqpoint{4.429760in}{1.514834in}}%
\pgfpathlineto{\pgfqpoint{4.421653in}{1.529505in}}%
\pgfpathlineto{\pgfqpoint{4.413537in}{1.544925in}}%
\pgfpathlineto{\pgfqpoint{4.398918in}{1.553890in}}%
\pgfpathlineto{\pgfqpoint{4.384303in}{1.562879in}}%
\pgfpathlineto{\pgfqpoint{4.369693in}{1.571891in}}%
\pgfpathlineto{\pgfqpoint{4.355087in}{1.580927in}}%
\pgfpathlineto{\pgfqpoint{4.363246in}{1.565017in}}%
\pgfpathlineto{\pgfqpoint{4.371394in}{1.549860in}}%
\pgfpathlineto{\pgfqpoint{4.379533in}{1.535442in}}%
\pgfpathlineto{\pgfqpoint{4.387663in}{1.521748in}}%
\pgfpathclose%
\pgfusepath{fill}%
\end{pgfscope}%
\begin{pgfscope}%
\pgfpathrectangle{\pgfqpoint{1.150000in}{0.150000in}}{\pgfqpoint{5.700000in}{5.700000in}}%
\pgfusepath{clip}%
\pgfsetbuttcap%
\pgfsetroundjoin%
\definecolor{currentfill}{rgb}{0.266580,0.228262,0.514349}%
\pgfsetfillcolor{currentfill}%
\pgfsetfillopacity{0.700000}%
\pgfsetlinewidth{0.000000pt}%
\definecolor{currentstroke}{rgb}{0.000000,0.000000,0.000000}%
\pgfsetstrokecolor{currentstroke}%
\pgfsetdash{}{0pt}%
\pgfpathmoveto{\pgfqpoint{4.711602in}{1.316613in}}%
\pgfpathlineto{\pgfqpoint{4.726234in}{1.308990in}}%
\pgfpathlineto{\pgfqpoint{4.740871in}{1.301390in}}%
\pgfpathlineto{\pgfqpoint{4.755513in}{1.293813in}}%
\pgfpathlineto{\pgfqpoint{4.770162in}{1.286259in}}%
\pgfpathlineto{\pgfqpoint{4.762269in}{1.294449in}}%
\pgfpathlineto{\pgfqpoint{4.754374in}{1.303282in}}%
\pgfpathlineto{\pgfqpoint{4.746475in}{1.312770in}}%
\pgfpathlineto{\pgfqpoint{4.738573in}{1.322928in}}%
\pgfpathlineto{\pgfqpoint{4.723892in}{1.330946in}}%
\pgfpathlineto{\pgfqpoint{4.709217in}{1.338987in}}%
\pgfpathlineto{\pgfqpoint{4.694548in}{1.347051in}}%
\pgfpathlineto{\pgfqpoint{4.679883in}{1.355138in}}%
\pgfpathlineto{\pgfqpoint{4.687819in}{1.344509in}}%
\pgfpathlineto{\pgfqpoint{4.695751in}{1.334555in}}%
\pgfpathlineto{\pgfqpoint{4.703679in}{1.325261in}}%
\pgfpathlineto{\pgfqpoint{4.711602in}{1.316613in}}%
\pgfpathclose%
\pgfusepath{fill}%
\end{pgfscope}%
\begin{pgfscope}%
\pgfpathrectangle{\pgfqpoint{1.150000in}{0.150000in}}{\pgfqpoint{5.700000in}{5.700000in}}%
\pgfusepath{clip}%
\pgfsetbuttcap%
\pgfsetroundjoin%
\definecolor{currentfill}{rgb}{0.182256,0.426184,0.557120}%
\pgfsetfillcolor{currentfill}%
\pgfsetfillopacity{0.700000}%
\pgfsetlinewidth{0.000000pt}%
\definecolor{currentstroke}{rgb}{0.000000,0.000000,0.000000}%
\pgfsetstrokecolor{currentstroke}%
\pgfsetdash{}{0pt}%
\pgfpathmoveto{\pgfqpoint{4.005848in}{1.804973in}}%
\pgfpathlineto{\pgfqpoint{4.020352in}{1.795359in}}%
\pgfpathlineto{\pgfqpoint{4.034860in}{1.785769in}}%
\pgfpathlineto{\pgfqpoint{4.049372in}{1.776204in}}%
\pgfpathlineto{\pgfqpoint{4.063888in}{1.766663in}}%
\pgfpathlineto{\pgfqpoint{4.055485in}{1.785824in}}%
\pgfpathlineto{\pgfqpoint{4.047066in}{1.805793in}}%
\pgfpathlineto{\pgfqpoint{4.038631in}{1.826586in}}%
\pgfpathlineto{\pgfqpoint{4.030178in}{1.848219in}}%
\pgfpathlineto{\pgfqpoint{4.015611in}{1.858265in}}%
\pgfpathlineto{\pgfqpoint{4.001048in}{1.868336in}}%
\pgfpathlineto{\pgfqpoint{3.986488in}{1.878431in}}%
\pgfpathlineto{\pgfqpoint{3.971932in}{1.888552in}}%
\pgfpathlineto{\pgfqpoint{3.980438in}{1.866406in}}%
\pgfpathlineto{\pgfqpoint{3.988925in}{1.845105in}}%
\pgfpathlineto{\pgfqpoint{3.997395in}{1.824632in}}%
\pgfpathlineto{\pgfqpoint{4.005848in}{1.804973in}}%
\pgfpathclose%
\pgfusepath{fill}%
\end{pgfscope}%
\begin{pgfscope}%
\pgfpathrectangle{\pgfqpoint{1.150000in}{0.150000in}}{\pgfqpoint{5.700000in}{5.700000in}}%
\pgfusepath{clip}%
\pgfsetbuttcap%
\pgfsetroundjoin%
\definecolor{currentfill}{rgb}{0.131172,0.555899,0.552459}%
\pgfsetfillcolor{currentfill}%
\pgfsetfillopacity{0.700000}%
\pgfsetlinewidth{0.000000pt}%
\definecolor{currentstroke}{rgb}{0.000000,0.000000,0.000000}%
\pgfsetstrokecolor{currentstroke}%
\pgfsetdash{}{0pt}%
\pgfpathmoveto{\pgfqpoint{3.565767in}{2.182210in}}%
\pgfpathlineto{\pgfqpoint{3.580230in}{2.171372in}}%
\pgfpathlineto{\pgfqpoint{3.594695in}{2.160561in}}%
\pgfpathlineto{\pgfqpoint{3.609164in}{2.149776in}}%
\pgfpathlineto{\pgfqpoint{3.623635in}{2.139017in}}%
\pgfpathlineto{\pgfqpoint{3.614763in}{2.165154in}}%
\pgfpathlineto{\pgfqpoint{3.605864in}{2.192200in}}%
\pgfpathlineto{\pgfqpoint{3.596938in}{2.220173in}}%
\pgfpathlineto{\pgfqpoint{3.582420in}{2.231329in}}%
\pgfpathlineto{\pgfqpoint{3.567904in}{2.242511in}}%
\pgfpathlineto{\pgfqpoint{3.553391in}{2.253720in}}%
\pgfpathlineto{\pgfqpoint{3.538881in}{2.264956in}}%
\pgfpathlineto{\pgfqpoint{3.547871in}{2.236448in}}%
\pgfpathlineto{\pgfqpoint{3.556833in}{2.208871in}}%
\pgfpathlineto{\pgfqpoint{3.565767in}{2.182210in}}%
\pgfpathclose%
\pgfusepath{fill}%
\end{pgfscope}%
\begin{pgfscope}%
\pgfpathrectangle{\pgfqpoint{1.150000in}{0.150000in}}{\pgfqpoint{5.700000in}{5.700000in}}%
\pgfusepath{clip}%
\pgfsetbuttcap%
\pgfsetroundjoin%
\definecolor{currentfill}{rgb}{0.235526,0.309527,0.542944}%
\pgfsetfillcolor{currentfill}%
\pgfsetfillopacity{0.700000}%
\pgfsetlinewidth{0.000000pt}%
\definecolor{currentstroke}{rgb}{0.000000,0.000000,0.000000}%
\pgfsetstrokecolor{currentstroke}%
\pgfsetdash{}{0pt}%
\pgfpathmoveto{\pgfqpoint{4.445948in}{1.487679in}}%
\pgfpathlineto{\pgfqpoint{4.460532in}{1.479221in}}%
\pgfpathlineto{\pgfqpoint{4.475120in}{1.470786in}}%
\pgfpathlineto{\pgfqpoint{4.489713in}{1.462374in}}%
\pgfpathlineto{\pgfqpoint{4.504311in}{1.453985in}}%
\pgfpathlineto{\pgfqpoint{4.496260in}{1.466727in}}%
\pgfpathlineto{\pgfqpoint{4.488202in}{1.480184in}}%
\pgfpathlineto{\pgfqpoint{4.480136in}{1.494370in}}%
\pgfpathlineto{\pgfqpoint{4.472061in}{1.509301in}}%
\pgfpathlineto{\pgfqpoint{4.457423in}{1.518172in}}%
\pgfpathlineto{\pgfqpoint{4.442790in}{1.527066in}}%
\pgfpathlineto{\pgfqpoint{4.428161in}{1.535984in}}%
\pgfpathlineto{\pgfqpoint{4.413537in}{1.544925in}}%
\pgfpathlineto{\pgfqpoint{4.421653in}{1.529505in}}%
\pgfpathlineto{\pgfqpoint{4.429760in}{1.514834in}}%
\pgfpathlineto{\pgfqpoint{4.437858in}{1.500897in}}%
\pgfpathlineto{\pgfqpoint{4.445948in}{1.487679in}}%
\pgfpathclose%
\pgfusepath{fill}%
\end{pgfscope}%
\begin{pgfscope}%
\pgfpathrectangle{\pgfqpoint{1.150000in}{0.150000in}}{\pgfqpoint{5.700000in}{5.700000in}}%
\pgfusepath{clip}%
\pgfsetbuttcap%
\pgfsetroundjoin%
\definecolor{currentfill}{rgb}{0.185556,0.418570,0.556753}%
\pgfsetfillcolor{currentfill}%
\pgfsetfillopacity{0.700000}%
\pgfsetlinewidth{0.000000pt}%
\definecolor{currentstroke}{rgb}{0.000000,0.000000,0.000000}%
\pgfsetstrokecolor{currentstroke}%
\pgfsetdash{}{0pt}%
\pgfpathmoveto{\pgfqpoint{4.063888in}{1.766663in}}%
\pgfpathlineto{\pgfqpoint{4.078408in}{1.757147in}}%
\pgfpathlineto{\pgfqpoint{4.092932in}{1.747655in}}%
\pgfpathlineto{\pgfqpoint{4.107460in}{1.738187in}}%
\pgfpathlineto{\pgfqpoint{4.121993in}{1.728744in}}%
\pgfpathlineto{\pgfqpoint{4.113638in}{1.747408in}}%
\pgfpathlineto{\pgfqpoint{4.105269in}{1.766874in}}%
\pgfpathlineto{\pgfqpoint{4.096885in}{1.787159in}}%
\pgfpathlineto{\pgfqpoint{4.088485in}{1.808279in}}%
\pgfpathlineto{\pgfqpoint{4.073902in}{1.818227in}}%
\pgfpathlineto{\pgfqpoint{4.059324in}{1.828200in}}%
\pgfpathlineto{\pgfqpoint{4.044749in}{1.838197in}}%
\pgfpathlineto{\pgfqpoint{4.030178in}{1.848219in}}%
\pgfpathlineto{\pgfqpoint{4.038631in}{1.826586in}}%
\pgfpathlineto{\pgfqpoint{4.047066in}{1.805793in}}%
\pgfpathlineto{\pgfqpoint{4.055485in}{1.785824in}}%
\pgfpathlineto{\pgfqpoint{4.063888in}{1.766663in}}%
\pgfpathclose%
\pgfusepath{fill}%
\end{pgfscope}%
\begin{pgfscope}%
\pgfpathrectangle{\pgfqpoint{1.150000in}{0.150000in}}{\pgfqpoint{5.700000in}{5.700000in}}%
\pgfusepath{clip}%
\pgfsetbuttcap%
\pgfsetroundjoin%
\definecolor{currentfill}{rgb}{0.135066,0.544853,0.554029}%
\pgfsetfillcolor{currentfill}%
\pgfsetfillopacity{0.700000}%
\pgfsetlinewidth{0.000000pt}%
\definecolor{currentstroke}{rgb}{0.000000,0.000000,0.000000}%
\pgfsetstrokecolor{currentstroke}%
\pgfsetdash{}{0pt}%
\pgfpathmoveto{\pgfqpoint{3.623635in}{2.139017in}}%
\pgfpathlineto{\pgfqpoint{3.638110in}{2.128285in}}%
\pgfpathlineto{\pgfqpoint{3.652587in}{2.117579in}}%
\pgfpathlineto{\pgfqpoint{3.667068in}{2.106900in}}%
\pgfpathlineto{\pgfqpoint{3.681552in}{2.096246in}}%
\pgfpathlineto{\pgfqpoint{3.672740in}{2.121859in}}%
\pgfpathlineto{\pgfqpoint{3.663904in}{2.148375in}}%
\pgfpathlineto{\pgfqpoint{3.655041in}{2.175813in}}%
\pgfpathlineto{\pgfqpoint{3.640511in}{2.186864in}}%
\pgfpathlineto{\pgfqpoint{3.625984in}{2.197940in}}%
\pgfpathlineto{\pgfqpoint{3.611459in}{2.209043in}}%
\pgfpathlineto{\pgfqpoint{3.596938in}{2.220173in}}%
\pgfpathlineto{\pgfqpoint{3.605864in}{2.192200in}}%
\pgfpathlineto{\pgfqpoint{3.614763in}{2.165154in}}%
\pgfpathlineto{\pgfqpoint{3.623635in}{2.139017in}}%
\pgfpathclose%
\pgfusepath{fill}%
\end{pgfscope}%
\begin{pgfscope}%
\pgfpathrectangle{\pgfqpoint{1.150000in}{0.150000in}}{\pgfqpoint{5.700000in}{5.700000in}}%
\pgfusepath{clip}%
\pgfsetbuttcap%
\pgfsetroundjoin%
\definecolor{currentfill}{rgb}{0.267968,0.223549,0.512008}%
\pgfsetfillcolor{currentfill}%
\pgfsetfillopacity{0.700000}%
\pgfsetlinewidth{0.000000pt}%
\definecolor{currentstroke}{rgb}{0.000000,0.000000,0.000000}%
\pgfsetstrokecolor{currentstroke}%
\pgfsetdash{}{0pt}%
\pgfpathmoveto{\pgfqpoint{4.770162in}{1.286259in}}%
\pgfpathlineto{\pgfqpoint{4.784815in}{1.278728in}}%
\pgfpathlineto{\pgfqpoint{4.799475in}{1.271219in}}%
\pgfpathlineto{\pgfqpoint{4.814140in}{1.263734in}}%
\pgfpathlineto{\pgfqpoint{4.806270in}{1.271581in}}%
\pgfpathlineto{\pgfqpoint{4.798398in}{1.280067in}}%
\pgfpathlineto{\pgfqpoint{4.790524in}{1.289207in}}%
\pgfpathlineto{\pgfqpoint{4.782646in}{1.299012in}}%
\pgfpathlineto{\pgfqpoint{4.767950in}{1.306961in}}%
\pgfpathlineto{\pgfqpoint{4.753259in}{1.314933in}}%
\pgfpathlineto{\pgfqpoint{4.738573in}{1.322928in}}%
\pgfpathlineto{\pgfqpoint{4.746475in}{1.312770in}}%
\pgfpathlineto{\pgfqpoint{4.754374in}{1.303282in}}%
\pgfpathlineto{\pgfqpoint{4.762269in}{1.294449in}}%
\pgfpathlineto{\pgfqpoint{4.770162in}{1.286259in}}%
\pgfpathclose%
\pgfusepath{fill}%
\end{pgfscope}%
\begin{pgfscope}%
\pgfpathrectangle{\pgfqpoint{1.150000in}{0.150000in}}{\pgfqpoint{5.700000in}{5.700000in}}%
\pgfusepath{clip}%
\pgfsetbuttcap%
\pgfsetroundjoin%
\definecolor{currentfill}{rgb}{0.139147,0.533812,0.555298}%
\pgfsetfillcolor{currentfill}%
\pgfsetfillopacity{0.700000}%
\pgfsetlinewidth{0.000000pt}%
\definecolor{currentstroke}{rgb}{0.000000,0.000000,0.000000}%
\pgfsetstrokecolor{currentstroke}%
\pgfsetdash{}{0pt}%
\pgfpathmoveto{\pgfqpoint{3.681552in}{2.096246in}}%
\pgfpathlineto{\pgfqpoint{3.696039in}{2.085618in}}%
\pgfpathlineto{\pgfqpoint{3.710530in}{2.075017in}}%
\pgfpathlineto{\pgfqpoint{3.725023in}{2.064441in}}%
\pgfpathlineto{\pgfqpoint{3.739520in}{2.053891in}}%
\pgfpathlineto{\pgfqpoint{3.730768in}{2.078981in}}%
\pgfpathlineto{\pgfqpoint{3.721992in}{2.104969in}}%
\pgfpathlineto{\pgfqpoint{3.713191in}{2.131874in}}%
\pgfpathlineto{\pgfqpoint{3.698649in}{2.142820in}}%
\pgfpathlineto{\pgfqpoint{3.684110in}{2.153791in}}%
\pgfpathlineto{\pgfqpoint{3.669574in}{2.164789in}}%
\pgfpathlineto{\pgfqpoint{3.655041in}{2.175813in}}%
\pgfpathlineto{\pgfqpoint{3.663904in}{2.148375in}}%
\pgfpathlineto{\pgfqpoint{3.672740in}{2.121859in}}%
\pgfpathlineto{\pgfqpoint{3.681552in}{2.096246in}}%
\pgfpathclose%
\pgfusepath{fill}%
\end{pgfscope}%
\begin{pgfscope}%
\pgfpathrectangle{\pgfqpoint{1.150000in}{0.150000in}}{\pgfqpoint{5.700000in}{5.700000in}}%
\pgfusepath{clip}%
\pgfsetbuttcap%
\pgfsetroundjoin%
\definecolor{currentfill}{rgb}{0.190631,0.407061,0.556089}%
\pgfsetfillcolor{currentfill}%
\pgfsetfillopacity{0.700000}%
\pgfsetlinewidth{0.000000pt}%
\definecolor{currentstroke}{rgb}{0.000000,0.000000,0.000000}%
\pgfsetstrokecolor{currentstroke}%
\pgfsetdash{}{0pt}%
\pgfpathmoveto{\pgfqpoint{4.121993in}{1.728744in}}%
\pgfpathlineto{\pgfqpoint{4.136529in}{1.719325in}}%
\pgfpathlineto{\pgfqpoint{4.151070in}{1.709931in}}%
\pgfpathlineto{\pgfqpoint{4.165614in}{1.700560in}}%
\pgfpathlineto{\pgfqpoint{4.180163in}{1.691214in}}%
\pgfpathlineto{\pgfqpoint{4.171857in}{1.709380in}}%
\pgfpathlineto{\pgfqpoint{4.163537in}{1.728345in}}%
\pgfpathlineto{\pgfqpoint{4.155202in}{1.748123in}}%
\pgfpathlineto{\pgfqpoint{4.146853in}{1.768731in}}%
\pgfpathlineto{\pgfqpoint{4.132255in}{1.778582in}}%
\pgfpathlineto{\pgfqpoint{4.117661in}{1.788457in}}%
\pgfpathlineto{\pgfqpoint{4.103071in}{1.798356in}}%
\pgfpathlineto{\pgfqpoint{4.088485in}{1.808279in}}%
\pgfpathlineto{\pgfqpoint{4.096885in}{1.787159in}}%
\pgfpathlineto{\pgfqpoint{4.105269in}{1.766874in}}%
\pgfpathlineto{\pgfqpoint{4.113638in}{1.747408in}}%
\pgfpathlineto{\pgfqpoint{4.121993in}{1.728744in}}%
\pgfpathclose%
\pgfusepath{fill}%
\end{pgfscope}%
\begin{pgfscope}%
\pgfpathrectangle{\pgfqpoint{1.150000in}{0.150000in}}{\pgfqpoint{5.700000in}{5.700000in}}%
\pgfusepath{clip}%
\pgfsetbuttcap%
\pgfsetroundjoin%
\definecolor{currentfill}{rgb}{0.239346,0.300855,0.540844}%
\pgfsetfillcolor{currentfill}%
\pgfsetfillopacity{0.700000}%
\pgfsetlinewidth{0.000000pt}%
\definecolor{currentstroke}{rgb}{0.000000,0.000000,0.000000}%
\pgfsetstrokecolor{currentstroke}%
\pgfsetdash{}{0pt}%
\pgfpathmoveto{\pgfqpoint{4.504311in}{1.453985in}}%
\pgfpathlineto{\pgfqpoint{4.518914in}{1.445620in}}%
\pgfpathlineto{\pgfqpoint{4.533523in}{1.437279in}}%
\pgfpathlineto{\pgfqpoint{4.548136in}{1.428960in}}%
\pgfpathlineto{\pgfqpoint{4.562754in}{1.420665in}}%
\pgfpathlineto{\pgfqpoint{4.554741in}{1.432931in}}%
\pgfpathlineto{\pgfqpoint{4.546721in}{1.445908in}}%
\pgfpathlineto{\pgfqpoint{4.538694in}{1.459609in}}%
\pgfpathlineto{\pgfqpoint{4.530661in}{1.474051in}}%
\pgfpathlineto{\pgfqpoint{4.516004in}{1.482828in}}%
\pgfpathlineto{\pgfqpoint{4.501351in}{1.491629in}}%
\pgfpathlineto{\pgfqpoint{4.486704in}{1.500453in}}%
\pgfpathlineto{\pgfqpoint{4.472061in}{1.509301in}}%
\pgfpathlineto{\pgfqpoint{4.480136in}{1.494370in}}%
\pgfpathlineto{\pgfqpoint{4.488202in}{1.480184in}}%
\pgfpathlineto{\pgfqpoint{4.496260in}{1.466727in}}%
\pgfpathlineto{\pgfqpoint{4.504311in}{1.453985in}}%
\pgfpathclose%
\pgfusepath{fill}%
\end{pgfscope}%
\begin{pgfscope}%
\pgfpathrectangle{\pgfqpoint{1.150000in}{0.150000in}}{\pgfqpoint{5.700000in}{5.700000in}}%
\pgfusepath{clip}%
\pgfsetbuttcap%
\pgfsetroundjoin%
\definecolor{currentfill}{rgb}{0.144759,0.519093,0.556572}%
\pgfsetfillcolor{currentfill}%
\pgfsetfillopacity{0.700000}%
\pgfsetlinewidth{0.000000pt}%
\definecolor{currentstroke}{rgb}{0.000000,0.000000,0.000000}%
\pgfsetstrokecolor{currentstroke}%
\pgfsetdash{}{0pt}%
\pgfpathmoveto{\pgfqpoint{3.739520in}{2.053891in}}%
\pgfpathlineto{\pgfqpoint{3.754020in}{2.043367in}}%
\pgfpathlineto{\pgfqpoint{3.768523in}{2.032868in}}%
\pgfpathlineto{\pgfqpoint{3.783030in}{2.022396in}}%
\pgfpathlineto{\pgfqpoint{3.797540in}{2.011948in}}%
\pgfpathlineto{\pgfqpoint{3.788847in}{2.036516in}}%
\pgfpathlineto{\pgfqpoint{3.780131in}{2.061977in}}%
\pgfpathlineto{\pgfqpoint{3.771391in}{2.088349in}}%
\pgfpathlineto{\pgfqpoint{3.756837in}{2.099191in}}%
\pgfpathlineto{\pgfqpoint{3.742285in}{2.110060in}}%
\pgfpathlineto{\pgfqpoint{3.727737in}{2.120954in}}%
\pgfpathlineto{\pgfqpoint{3.713191in}{2.131874in}}%
\pgfpathlineto{\pgfqpoint{3.721992in}{2.104969in}}%
\pgfpathlineto{\pgfqpoint{3.730768in}{2.078981in}}%
\pgfpathlineto{\pgfqpoint{3.739520in}{2.053891in}}%
\pgfpathclose%
\pgfusepath{fill}%
\end{pgfscope}%
\begin{pgfscope}%
\pgfpathrectangle{\pgfqpoint{1.150000in}{0.150000in}}{\pgfqpoint{5.700000in}{5.700000in}}%
\pgfusepath{clip}%
\pgfsetbuttcap%
\pgfsetroundjoin%
\definecolor{currentfill}{rgb}{0.195860,0.395433,0.555276}%
\pgfsetfillcolor{currentfill}%
\pgfsetfillopacity{0.700000}%
\pgfsetlinewidth{0.000000pt}%
\definecolor{currentstroke}{rgb}{0.000000,0.000000,0.000000}%
\pgfsetstrokecolor{currentstroke}%
\pgfsetdash{}{0pt}%
\pgfpathmoveto{\pgfqpoint{4.180163in}{1.691214in}}%
\pgfpathlineto{\pgfqpoint{4.194716in}{1.681891in}}%
\pgfpathlineto{\pgfqpoint{4.209274in}{1.672593in}}%
\pgfpathlineto{\pgfqpoint{4.223835in}{1.663319in}}%
\pgfpathlineto{\pgfqpoint{4.238401in}{1.654069in}}%
\pgfpathlineto{\pgfqpoint{4.230142in}{1.671739in}}%
\pgfpathlineto{\pgfqpoint{4.221870in}{1.690202in}}%
\pgfpathlineto{\pgfqpoint{4.213584in}{1.709474in}}%
\pgfpathlineto{\pgfqpoint{4.205285in}{1.729572in}}%
\pgfpathlineto{\pgfqpoint{4.190671in}{1.739325in}}%
\pgfpathlineto{\pgfqpoint{4.176061in}{1.749103in}}%
\pgfpathlineto{\pgfqpoint{4.161455in}{1.758905in}}%
\pgfpathlineto{\pgfqpoint{4.146853in}{1.768731in}}%
\pgfpathlineto{\pgfqpoint{4.155202in}{1.748123in}}%
\pgfpathlineto{\pgfqpoint{4.163537in}{1.728345in}}%
\pgfpathlineto{\pgfqpoint{4.171857in}{1.709380in}}%
\pgfpathlineto{\pgfqpoint{4.180163in}{1.691214in}}%
\pgfpathclose%
\pgfusepath{fill}%
\end{pgfscope}%
\begin{pgfscope}%
\pgfpathrectangle{\pgfqpoint{1.150000in}{0.150000in}}{\pgfqpoint{5.700000in}{5.700000in}}%
\pgfusepath{clip}%
\pgfsetbuttcap%
\pgfsetroundjoin%
\definecolor{currentfill}{rgb}{0.243113,0.292092,0.538516}%
\pgfsetfillcolor{currentfill}%
\pgfsetfillopacity{0.700000}%
\pgfsetlinewidth{0.000000pt}%
\definecolor{currentstroke}{rgb}{0.000000,0.000000,0.000000}%
\pgfsetstrokecolor{currentstroke}%
\pgfsetdash{}{0pt}%
\pgfpathmoveto{\pgfqpoint{4.562754in}{1.420665in}}%
\pgfpathlineto{\pgfqpoint{4.577377in}{1.412393in}}%
\pgfpathlineto{\pgfqpoint{4.592005in}{1.404144in}}%
\pgfpathlineto{\pgfqpoint{4.606639in}{1.395919in}}%
\pgfpathlineto{\pgfqpoint{4.621277in}{1.387716in}}%
\pgfpathlineto{\pgfqpoint{4.613301in}{1.399507in}}%
\pgfpathlineto{\pgfqpoint{4.605319in}{1.412004in}}%
\pgfpathlineto{\pgfqpoint{4.597331in}{1.425222in}}%
\pgfpathlineto{\pgfqpoint{4.589337in}{1.439175in}}%
\pgfpathlineto{\pgfqpoint{4.574661in}{1.447859in}}%
\pgfpathlineto{\pgfqpoint{4.559989in}{1.456566in}}%
\pgfpathlineto{\pgfqpoint{4.545323in}{1.465297in}}%
\pgfpathlineto{\pgfqpoint{4.530661in}{1.474051in}}%
\pgfpathlineto{\pgfqpoint{4.538694in}{1.459609in}}%
\pgfpathlineto{\pgfqpoint{4.546721in}{1.445908in}}%
\pgfpathlineto{\pgfqpoint{4.554741in}{1.432931in}}%
\pgfpathlineto{\pgfqpoint{4.562754in}{1.420665in}}%
\pgfpathclose%
\pgfusepath{fill}%
\end{pgfscope}%
\begin{pgfscope}%
\pgfpathrectangle{\pgfqpoint{1.150000in}{0.150000in}}{\pgfqpoint{5.700000in}{5.700000in}}%
\pgfusepath{clip}%
\pgfsetbuttcap%
\pgfsetroundjoin%
\definecolor{currentfill}{rgb}{0.149039,0.508051,0.557250}%
\pgfsetfillcolor{currentfill}%
\pgfsetfillopacity{0.700000}%
\pgfsetlinewidth{0.000000pt}%
\definecolor{currentstroke}{rgb}{0.000000,0.000000,0.000000}%
\pgfsetstrokecolor{currentstroke}%
\pgfsetdash{}{0pt}%
\pgfpathmoveto{\pgfqpoint{3.797540in}{2.011948in}}%
\pgfpathlineto{\pgfqpoint{3.812054in}{2.001526in}}%
\pgfpathlineto{\pgfqpoint{3.826571in}{1.991130in}}%
\pgfpathlineto{\pgfqpoint{3.841091in}{1.980759in}}%
\pgfpathlineto{\pgfqpoint{3.855615in}{1.970413in}}%
\pgfpathlineto{\pgfqpoint{3.846979in}{1.994460in}}%
\pgfpathlineto{\pgfqpoint{3.838321in}{2.019395in}}%
\pgfpathlineto{\pgfqpoint{3.829642in}{2.045235in}}%
\pgfpathlineto{\pgfqpoint{3.815074in}{2.055975in}}%
\pgfpathlineto{\pgfqpoint{3.800510in}{2.066741in}}%
\pgfpathlineto{\pgfqpoint{3.785949in}{2.077532in}}%
\pgfpathlineto{\pgfqpoint{3.771391in}{2.088349in}}%
\pgfpathlineto{\pgfqpoint{3.780131in}{2.061977in}}%
\pgfpathlineto{\pgfqpoint{3.788847in}{2.036516in}}%
\pgfpathlineto{\pgfqpoint{3.797540in}{2.011948in}}%
\pgfpathclose%
\pgfusepath{fill}%
\end{pgfscope}%
\begin{pgfscope}%
\pgfpathrectangle{\pgfqpoint{1.150000in}{0.150000in}}{\pgfqpoint{5.700000in}{5.700000in}}%
\pgfusepath{clip}%
\pgfsetbuttcap%
\pgfsetroundjoin%
\definecolor{currentfill}{rgb}{0.199430,0.387607,0.554642}%
\pgfsetfillcolor{currentfill}%
\pgfsetfillopacity{0.700000}%
\pgfsetlinewidth{0.000000pt}%
\definecolor{currentstroke}{rgb}{0.000000,0.000000,0.000000}%
\pgfsetstrokecolor{currentstroke}%
\pgfsetdash{}{0pt}%
\pgfpathmoveto{\pgfqpoint{4.238401in}{1.654069in}}%
\pgfpathlineto{\pgfqpoint{4.252972in}{1.644843in}}%
\pgfpathlineto{\pgfqpoint{4.267546in}{1.635640in}}%
\pgfpathlineto{\pgfqpoint{4.282125in}{1.626462in}}%
\pgfpathlineto{\pgfqpoint{4.296709in}{1.617307in}}%
\pgfpathlineto{\pgfqpoint{4.288495in}{1.634482in}}%
\pgfpathlineto{\pgfqpoint{4.280270in}{1.652444in}}%
\pgfpathlineto{\pgfqpoint{4.272032in}{1.671211in}}%
\pgfpathlineto{\pgfqpoint{4.263782in}{1.690798in}}%
\pgfpathlineto{\pgfqpoint{4.249152in}{1.700455in}}%
\pgfpathlineto{\pgfqpoint{4.234525in}{1.710136in}}%
\pgfpathlineto{\pgfqpoint{4.219903in}{1.719842in}}%
\pgfpathlineto{\pgfqpoint{4.205285in}{1.729572in}}%
\pgfpathlineto{\pgfqpoint{4.213584in}{1.709474in}}%
\pgfpathlineto{\pgfqpoint{4.221870in}{1.690202in}}%
\pgfpathlineto{\pgfqpoint{4.230142in}{1.671739in}}%
\pgfpathlineto{\pgfqpoint{4.238401in}{1.654069in}}%
\pgfpathclose%
\pgfusepath{fill}%
\end{pgfscope}%
\begin{pgfscope}%
\pgfpathrectangle{\pgfqpoint{1.150000in}{0.150000in}}{\pgfqpoint{5.700000in}{5.700000in}}%
\pgfusepath{clip}%
\pgfsetbuttcap%
\pgfsetroundjoin%
\definecolor{currentfill}{rgb}{0.246811,0.283237,0.535941}%
\pgfsetfillcolor{currentfill}%
\pgfsetfillopacity{0.700000}%
\pgfsetlinewidth{0.000000pt}%
\definecolor{currentstroke}{rgb}{0.000000,0.000000,0.000000}%
\pgfsetstrokecolor{currentstroke}%
\pgfsetdash{}{0pt}%
\pgfpathmoveto{\pgfqpoint{4.621277in}{1.387716in}}%
\pgfpathlineto{\pgfqpoint{4.635921in}{1.379537in}}%
\pgfpathlineto{\pgfqpoint{4.650570in}{1.371381in}}%
\pgfpathlineto{\pgfqpoint{4.665224in}{1.363248in}}%
\pgfpathlineto{\pgfqpoint{4.679883in}{1.355138in}}%
\pgfpathlineto{\pgfqpoint{4.671942in}{1.366454in}}%
\pgfpathlineto{\pgfqpoint{4.663997in}{1.378471in}}%
\pgfpathlineto{\pgfqpoint{4.656047in}{1.391205in}}%
\pgfpathlineto{\pgfqpoint{4.648092in}{1.404670in}}%
\pgfpathlineto{\pgfqpoint{4.633396in}{1.413262in}}%
\pgfpathlineto{\pgfqpoint{4.618704in}{1.421876in}}%
\pgfpathlineto{\pgfqpoint{4.604018in}{1.430514in}}%
\pgfpathlineto{\pgfqpoint{4.589337in}{1.439175in}}%
\pgfpathlineto{\pgfqpoint{4.597331in}{1.425222in}}%
\pgfpathlineto{\pgfqpoint{4.605319in}{1.412004in}}%
\pgfpathlineto{\pgfqpoint{4.613301in}{1.399507in}}%
\pgfpathlineto{\pgfqpoint{4.621277in}{1.387716in}}%
\pgfpathclose%
\pgfusepath{fill}%
\end{pgfscope}%
\begin{pgfscope}%
\pgfpathrectangle{\pgfqpoint{1.150000in}{0.150000in}}{\pgfqpoint{5.700000in}{5.700000in}}%
\pgfusepath{clip}%
\pgfsetbuttcap%
\pgfsetroundjoin%
\definecolor{currentfill}{rgb}{0.153364,0.497000,0.557724}%
\pgfsetfillcolor{currentfill}%
\pgfsetfillopacity{0.700000}%
\pgfsetlinewidth{0.000000pt}%
\definecolor{currentstroke}{rgb}{0.000000,0.000000,0.000000}%
\pgfsetstrokecolor{currentstroke}%
\pgfsetdash{}{0pt}%
\pgfpathmoveto{\pgfqpoint{3.855615in}{1.970413in}}%
\pgfpathlineto{\pgfqpoint{3.870142in}{1.960093in}}%
\pgfpathlineto{\pgfqpoint{3.884673in}{1.949797in}}%
\pgfpathlineto{\pgfqpoint{3.899207in}{1.939527in}}%
\pgfpathlineto{\pgfqpoint{3.913745in}{1.929282in}}%
\pgfpathlineto{\pgfqpoint{3.905165in}{1.952809in}}%
\pgfpathlineto{\pgfqpoint{3.896565in}{1.977218in}}%
\pgfpathlineto{\pgfqpoint{3.887945in}{2.002527in}}%
\pgfpathlineto{\pgfqpoint{3.873364in}{2.013166in}}%
\pgfpathlineto{\pgfqpoint{3.858787in}{2.023830in}}%
\pgfpathlineto{\pgfqpoint{3.844213in}{2.034520in}}%
\pgfpathlineto{\pgfqpoint{3.829642in}{2.045235in}}%
\pgfpathlineto{\pgfqpoint{3.838321in}{2.019395in}}%
\pgfpathlineto{\pgfqpoint{3.846979in}{1.994460in}}%
\pgfpathlineto{\pgfqpoint{3.855615in}{1.970413in}}%
\pgfpathclose%
\pgfusepath{fill}%
\end{pgfscope}%
\begin{pgfscope}%
\pgfpathrectangle{\pgfqpoint{1.150000in}{0.150000in}}{\pgfqpoint{5.700000in}{5.700000in}}%
\pgfusepath{clip}%
\pgfsetbuttcap%
\pgfsetroundjoin%
\definecolor{currentfill}{rgb}{0.204903,0.375746,0.553533}%
\pgfsetfillcolor{currentfill}%
\pgfsetfillopacity{0.700000}%
\pgfsetlinewidth{0.000000pt}%
\definecolor{currentstroke}{rgb}{0.000000,0.000000,0.000000}%
\pgfsetstrokecolor{currentstroke}%
\pgfsetdash{}{0pt}%
\pgfpathmoveto{\pgfqpoint{4.296709in}{1.617307in}}%
\pgfpathlineto{\pgfqpoint{4.311297in}{1.608177in}}%
\pgfpathlineto{\pgfqpoint{4.325889in}{1.599070in}}%
\pgfpathlineto{\pgfqpoint{4.340486in}{1.589986in}}%
\pgfpathlineto{\pgfqpoint{4.355087in}{1.580927in}}%
\pgfpathlineto{\pgfqpoint{4.346918in}{1.597606in}}%
\pgfpathlineto{\pgfqpoint{4.338738in}{1.615068in}}%
\pgfpathlineto{\pgfqpoint{4.330548in}{1.633330in}}%
\pgfpathlineto{\pgfqpoint{4.322346in}{1.652407in}}%
\pgfpathlineto{\pgfqpoint{4.307698in}{1.661969in}}%
\pgfpathlineto{\pgfqpoint{4.293055in}{1.671555in}}%
\pgfpathlineto{\pgfqpoint{4.278417in}{1.681164in}}%
\pgfpathlineto{\pgfqpoint{4.263782in}{1.690798in}}%
\pgfpathlineto{\pgfqpoint{4.272032in}{1.671211in}}%
\pgfpathlineto{\pgfqpoint{4.280270in}{1.652444in}}%
\pgfpathlineto{\pgfqpoint{4.288495in}{1.634482in}}%
\pgfpathlineto{\pgfqpoint{4.296709in}{1.617307in}}%
\pgfpathclose%
\pgfusepath{fill}%
\end{pgfscope}%
\begin{pgfscope}%
\pgfpathrectangle{\pgfqpoint{1.150000in}{0.150000in}}{\pgfqpoint{5.700000in}{5.700000in}}%
\pgfusepath{clip}%
\pgfsetbuttcap%
\pgfsetroundjoin%
\definecolor{currentfill}{rgb}{0.157729,0.485932,0.558013}%
\pgfsetfillcolor{currentfill}%
\pgfsetfillopacity{0.700000}%
\pgfsetlinewidth{0.000000pt}%
\definecolor{currentstroke}{rgb}{0.000000,0.000000,0.000000}%
\pgfsetstrokecolor{currentstroke}%
\pgfsetdash{}{0pt}%
\pgfpathmoveto{\pgfqpoint{3.913745in}{1.929282in}}%
\pgfpathlineto{\pgfqpoint{3.928286in}{1.919062in}}%
\pgfpathlineto{\pgfqpoint{3.942831in}{1.908867in}}%
\pgfpathlineto{\pgfqpoint{3.957380in}{1.898697in}}%
\pgfpathlineto{\pgfqpoint{3.971932in}{1.888552in}}%
\pgfpathlineto{\pgfqpoint{3.963408in}{1.911559in}}%
\pgfpathlineto{\pgfqpoint{3.954865in}{1.935444in}}%
\pgfpathlineto{\pgfqpoint{3.946302in}{1.960223in}}%
\pgfpathlineto{\pgfqpoint{3.931708in}{1.970762in}}%
\pgfpathlineto{\pgfqpoint{3.917117in}{1.981325in}}%
\pgfpathlineto{\pgfqpoint{3.902529in}{1.991914in}}%
\pgfpathlineto{\pgfqpoint{3.887945in}{2.002527in}}%
\pgfpathlineto{\pgfqpoint{3.896565in}{1.977218in}}%
\pgfpathlineto{\pgfqpoint{3.905165in}{1.952809in}}%
\pgfpathlineto{\pgfqpoint{3.913745in}{1.929282in}}%
\pgfpathclose%
\pgfusepath{fill}%
\end{pgfscope}%
\begin{pgfscope}%
\pgfpathrectangle{\pgfqpoint{1.150000in}{0.150000in}}{\pgfqpoint{5.700000in}{5.700000in}}%
\pgfusepath{clip}%
\pgfsetbuttcap%
\pgfsetroundjoin%
\definecolor{currentfill}{rgb}{0.250425,0.274290,0.533103}%
\pgfsetfillcolor{currentfill}%
\pgfsetfillopacity{0.700000}%
\pgfsetlinewidth{0.000000pt}%
\definecolor{currentstroke}{rgb}{0.000000,0.000000,0.000000}%
\pgfsetstrokecolor{currentstroke}%
\pgfsetdash{}{0pt}%
\pgfpathmoveto{\pgfqpoint{4.679883in}{1.355138in}}%
\pgfpathlineto{\pgfqpoint{4.694548in}{1.347051in}}%
\pgfpathlineto{\pgfqpoint{4.709217in}{1.338987in}}%
\pgfpathlineto{\pgfqpoint{4.723892in}{1.330946in}}%
\pgfpathlineto{\pgfqpoint{4.738573in}{1.322928in}}%
\pgfpathlineto{\pgfqpoint{4.730667in}{1.333769in}}%
\pgfpathlineto{\pgfqpoint{4.722757in}{1.345308in}}%
\pgfpathlineto{\pgfqpoint{4.714844in}{1.357559in}}%
\pgfpathlineto{\pgfqpoint{4.706926in}{1.370536in}}%
\pgfpathlineto{\pgfqpoint{4.692210in}{1.379035in}}%
\pgfpathlineto{\pgfqpoint{4.677499in}{1.387557in}}%
\pgfpathlineto{\pgfqpoint{4.662793in}{1.396102in}}%
\pgfpathlineto{\pgfqpoint{4.648092in}{1.404670in}}%
\pgfpathlineto{\pgfqpoint{4.656047in}{1.391205in}}%
\pgfpathlineto{\pgfqpoint{4.663997in}{1.378471in}}%
\pgfpathlineto{\pgfqpoint{4.671942in}{1.366454in}}%
\pgfpathlineto{\pgfqpoint{4.679883in}{1.355138in}}%
\pgfpathclose%
\pgfusepath{fill}%
\end{pgfscope}%
\begin{pgfscope}%
\pgfpathrectangle{\pgfqpoint{1.150000in}{0.150000in}}{\pgfqpoint{5.700000in}{5.700000in}}%
\pgfusepath{clip}%
\pgfsetbuttcap%
\pgfsetroundjoin%
\definecolor{currentfill}{rgb}{0.208623,0.367752,0.552675}%
\pgfsetfillcolor{currentfill}%
\pgfsetfillopacity{0.700000}%
\pgfsetlinewidth{0.000000pt}%
\definecolor{currentstroke}{rgb}{0.000000,0.000000,0.000000}%
\pgfsetstrokecolor{currentstroke}%
\pgfsetdash{}{0pt}%
\pgfpathmoveto{\pgfqpoint{4.355087in}{1.580927in}}%
\pgfpathlineto{\pgfqpoint{4.369693in}{1.571891in}}%
\pgfpathlineto{\pgfqpoint{4.384303in}{1.562879in}}%
\pgfpathlineto{\pgfqpoint{4.398918in}{1.553890in}}%
\pgfpathlineto{\pgfqpoint{4.413537in}{1.544925in}}%
\pgfpathlineto{\pgfqpoint{4.405412in}{1.561109in}}%
\pgfpathlineto{\pgfqpoint{4.397277in}{1.578073in}}%
\pgfpathlineto{\pgfqpoint{4.389132in}{1.595830in}}%
\pgfpathlineto{\pgfqpoint{4.380977in}{1.614398in}}%
\pgfpathlineto{\pgfqpoint{4.366313in}{1.623865in}}%
\pgfpathlineto{\pgfqpoint{4.351653in}{1.633355in}}%
\pgfpathlineto{\pgfqpoint{4.336997in}{1.642869in}}%
\pgfpathlineto{\pgfqpoint{4.322346in}{1.652407in}}%
\pgfpathlineto{\pgfqpoint{4.330548in}{1.633330in}}%
\pgfpathlineto{\pgfqpoint{4.338738in}{1.615068in}}%
\pgfpathlineto{\pgfqpoint{4.346918in}{1.597606in}}%
\pgfpathlineto{\pgfqpoint{4.355087in}{1.580927in}}%
\pgfpathclose%
\pgfusepath{fill}%
\end{pgfscope}%
\begin{pgfscope}%
\pgfpathrectangle{\pgfqpoint{1.150000in}{0.150000in}}{\pgfqpoint{5.700000in}{5.700000in}}%
\pgfusepath{clip}%
\pgfsetbuttcap%
\pgfsetroundjoin%
\definecolor{currentfill}{rgb}{0.162142,0.474838,0.558140}%
\pgfsetfillcolor{currentfill}%
\pgfsetfillopacity{0.700000}%
\pgfsetlinewidth{0.000000pt}%
\definecolor{currentstroke}{rgb}{0.000000,0.000000,0.000000}%
\pgfsetstrokecolor{currentstroke}%
\pgfsetdash{}{0pt}%
\pgfpathmoveto{\pgfqpoint{3.971932in}{1.888552in}}%
\pgfpathlineto{\pgfqpoint{3.986488in}{1.878431in}}%
\pgfpathlineto{\pgfqpoint{4.001048in}{1.868336in}}%
\pgfpathlineto{\pgfqpoint{4.015611in}{1.858265in}}%
\pgfpathlineto{\pgfqpoint{4.030178in}{1.848219in}}%
\pgfpathlineto{\pgfqpoint{4.021708in}{1.870707in}}%
\pgfpathlineto{\pgfqpoint{4.013221in}{1.894068in}}%
\pgfpathlineto{\pgfqpoint{4.004715in}{1.918319in}}%
\pgfpathlineto{\pgfqpoint{3.990107in}{1.928758in}}%
\pgfpathlineto{\pgfqpoint{3.975502in}{1.939221in}}%
\pgfpathlineto{\pgfqpoint{3.960900in}{1.949710in}}%
\pgfpathlineto{\pgfqpoint{3.946302in}{1.960223in}}%
\pgfpathlineto{\pgfqpoint{3.954865in}{1.935444in}}%
\pgfpathlineto{\pgfqpoint{3.963408in}{1.911559in}}%
\pgfpathlineto{\pgfqpoint{3.971932in}{1.888552in}}%
\pgfpathclose%
\pgfusepath{fill}%
\end{pgfscope}%
\begin{pgfscope}%
\pgfpathrectangle{\pgfqpoint{1.150000in}{0.150000in}}{\pgfqpoint{5.700000in}{5.700000in}}%
\pgfusepath{clip}%
\pgfsetbuttcap%
\pgfsetroundjoin%
\definecolor{currentfill}{rgb}{0.253935,0.265254,0.529983}%
\pgfsetfillcolor{currentfill}%
\pgfsetfillopacity{0.700000}%
\pgfsetlinewidth{0.000000pt}%
\definecolor{currentstroke}{rgb}{0.000000,0.000000,0.000000}%
\pgfsetstrokecolor{currentstroke}%
\pgfsetdash{}{0pt}%
\pgfpathmoveto{\pgfqpoint{4.738573in}{1.322928in}}%
\pgfpathlineto{\pgfqpoint{4.753259in}{1.314933in}}%
\pgfpathlineto{\pgfqpoint{4.767950in}{1.306961in}}%
\pgfpathlineto{\pgfqpoint{4.782646in}{1.299012in}}%
\pgfpathlineto{\pgfqpoint{4.774766in}{1.309498in}}%
\pgfpathlineto{\pgfqpoint{4.766882in}{1.320678in}}%
\pgfpathlineto{\pgfqpoint{4.758996in}{1.332566in}}%
\pgfpathlineto{\pgfqpoint{4.751106in}{1.345178in}}%
\pgfpathlineto{\pgfqpoint{4.736374in}{1.353607in}}%
\pgfpathlineto{\pgfqpoint{4.721648in}{1.362060in}}%
\pgfpathlineto{\pgfqpoint{4.706926in}{1.370536in}}%
\pgfpathlineto{\pgfqpoint{4.714844in}{1.357559in}}%
\pgfpathlineto{\pgfqpoint{4.722757in}{1.345308in}}%
\pgfpathlineto{\pgfqpoint{4.730667in}{1.333769in}}%
\pgfpathlineto{\pgfqpoint{4.738573in}{1.322928in}}%
\pgfpathclose%
\pgfusepath{fill}%
\end{pgfscope}%
\begin{pgfscope}%
\pgfpathrectangle{\pgfqpoint{1.150000in}{0.150000in}}{\pgfqpoint{5.700000in}{5.700000in}}%
\pgfusepath{clip}%
\pgfsetbuttcap%
\pgfsetroundjoin%
\definecolor{currentfill}{rgb}{0.166617,0.463708,0.558119}%
\pgfsetfillcolor{currentfill}%
\pgfsetfillopacity{0.700000}%
\pgfsetlinewidth{0.000000pt}%
\definecolor{currentstroke}{rgb}{0.000000,0.000000,0.000000}%
\pgfsetstrokecolor{currentstroke}%
\pgfsetdash{}{0pt}%
\pgfpathmoveto{\pgfqpoint{4.030178in}{1.848219in}}%
\pgfpathlineto{\pgfqpoint{4.044749in}{1.838197in}}%
\pgfpathlineto{\pgfqpoint{4.059324in}{1.828200in}}%
\pgfpathlineto{\pgfqpoint{4.073902in}{1.818227in}}%
\pgfpathlineto{\pgfqpoint{4.088485in}{1.808279in}}%
\pgfpathlineto{\pgfqpoint{4.080068in}{1.830250in}}%
\pgfpathlineto{\pgfqpoint{4.071635in}{1.853089in}}%
\pgfpathlineto{\pgfqpoint{4.063185in}{1.876811in}}%
\pgfpathlineto{\pgfqpoint{4.048562in}{1.887151in}}%
\pgfpathlineto{\pgfqpoint{4.033943in}{1.897516in}}%
\pgfpathlineto{\pgfqpoint{4.019327in}{1.907905in}}%
\pgfpathlineto{\pgfqpoint{4.004715in}{1.918319in}}%
\pgfpathlineto{\pgfqpoint{4.013221in}{1.894068in}}%
\pgfpathlineto{\pgfqpoint{4.021708in}{1.870707in}}%
\pgfpathlineto{\pgfqpoint{4.030178in}{1.848219in}}%
\pgfpathclose%
\pgfusepath{fill}%
\end{pgfscope}%
\begin{pgfscope}%
\pgfpathrectangle{\pgfqpoint{1.150000in}{0.150000in}}{\pgfqpoint{5.700000in}{5.700000in}}%
\pgfusepath{clip}%
\pgfsetbuttcap%
\pgfsetroundjoin%
\definecolor{currentfill}{rgb}{0.212395,0.359683,0.551710}%
\pgfsetfillcolor{currentfill}%
\pgfsetfillopacity{0.700000}%
\pgfsetlinewidth{0.000000pt}%
\definecolor{currentstroke}{rgb}{0.000000,0.000000,0.000000}%
\pgfsetstrokecolor{currentstroke}%
\pgfsetdash{}{0pt}%
\pgfpathmoveto{\pgfqpoint{4.413537in}{1.544925in}}%
\pgfpathlineto{\pgfqpoint{4.428161in}{1.535984in}}%
\pgfpathlineto{\pgfqpoint{4.442790in}{1.527066in}}%
\pgfpathlineto{\pgfqpoint{4.457423in}{1.518172in}}%
\pgfpathlineto{\pgfqpoint{4.472061in}{1.509301in}}%
\pgfpathlineto{\pgfqpoint{4.463979in}{1.524991in}}%
\pgfpathlineto{\pgfqpoint{4.455888in}{1.541455in}}%
\pgfpathlineto{\pgfqpoint{4.447788in}{1.558709in}}%
\pgfpathlineto{\pgfqpoint{4.439679in}{1.576768in}}%
\pgfpathlineto{\pgfqpoint{4.424997in}{1.586140in}}%
\pgfpathlineto{\pgfqpoint{4.410319in}{1.595536in}}%
\pgfpathlineto{\pgfqpoint{4.395646in}{1.604955in}}%
\pgfpathlineto{\pgfqpoint{4.380977in}{1.614398in}}%
\pgfpathlineto{\pgfqpoint{4.389132in}{1.595830in}}%
\pgfpathlineto{\pgfqpoint{4.397277in}{1.578073in}}%
\pgfpathlineto{\pgfqpoint{4.405412in}{1.561109in}}%
\pgfpathlineto{\pgfqpoint{4.413537in}{1.544925in}}%
\pgfpathclose%
\pgfusepath{fill}%
\end{pgfscope}%
\begin{pgfscope}%
\pgfpathrectangle{\pgfqpoint{1.150000in}{0.150000in}}{\pgfqpoint{5.700000in}{5.700000in}}%
\pgfusepath{clip}%
\pgfsetbuttcap%
\pgfsetroundjoin%
\definecolor{currentfill}{rgb}{0.169646,0.456262,0.558030}%
\pgfsetfillcolor{currentfill}%
\pgfsetfillopacity{0.700000}%
\pgfsetlinewidth{0.000000pt}%
\definecolor{currentstroke}{rgb}{0.000000,0.000000,0.000000}%
\pgfsetstrokecolor{currentstroke}%
\pgfsetdash{}{0pt}%
\pgfpathmoveto{\pgfqpoint{4.088485in}{1.808279in}}%
\pgfpathlineto{\pgfqpoint{4.103071in}{1.798356in}}%
\pgfpathlineto{\pgfqpoint{4.117661in}{1.788457in}}%
\pgfpathlineto{\pgfqpoint{4.132255in}{1.778582in}}%
\pgfpathlineto{\pgfqpoint{4.146853in}{1.768731in}}%
\pgfpathlineto{\pgfqpoint{4.138489in}{1.790185in}}%
\pgfpathlineto{\pgfqpoint{4.130109in}{1.812502in}}%
\pgfpathlineto{\pgfqpoint{4.121714in}{1.835697in}}%
\pgfpathlineto{\pgfqpoint{4.107076in}{1.845939in}}%
\pgfpathlineto{\pgfqpoint{4.092442in}{1.856205in}}%
\pgfpathlineto{\pgfqpoint{4.077812in}{1.866496in}}%
\pgfpathlineto{\pgfqpoint{4.063185in}{1.876811in}}%
\pgfpathlineto{\pgfqpoint{4.071635in}{1.853089in}}%
\pgfpathlineto{\pgfqpoint{4.080068in}{1.830250in}}%
\pgfpathlineto{\pgfqpoint{4.088485in}{1.808279in}}%
\pgfpathclose%
\pgfusepath{fill}%
\end{pgfscope}%
\begin{pgfscope}%
\pgfpathrectangle{\pgfqpoint{1.150000in}{0.150000in}}{\pgfqpoint{5.700000in}{5.700000in}}%
\pgfusepath{clip}%
\pgfsetbuttcap%
\pgfsetroundjoin%
\definecolor{currentfill}{rgb}{0.218130,0.347432,0.550038}%
\pgfsetfillcolor{currentfill}%
\pgfsetfillopacity{0.700000}%
\pgfsetlinewidth{0.000000pt}%
\definecolor{currentstroke}{rgb}{0.000000,0.000000,0.000000}%
\pgfsetstrokecolor{currentstroke}%
\pgfsetdash{}{0pt}%
\pgfpathmoveto{\pgfqpoint{4.472061in}{1.509301in}}%
\pgfpathlineto{\pgfqpoint{4.486704in}{1.500453in}}%
\pgfpathlineto{\pgfqpoint{4.501351in}{1.491629in}}%
\pgfpathlineto{\pgfqpoint{4.516004in}{1.482828in}}%
\pgfpathlineto{\pgfqpoint{4.530661in}{1.474051in}}%
\pgfpathlineto{\pgfqpoint{4.522620in}{1.489247in}}%
\pgfpathlineto{\pgfqpoint{4.514571in}{1.505213in}}%
\pgfpathlineto{\pgfqpoint{4.506515in}{1.521964in}}%
\pgfpathlineto{\pgfqpoint{4.498451in}{1.539515in}}%
\pgfpathlineto{\pgfqpoint{4.483751in}{1.548793in}}%
\pgfpathlineto{\pgfqpoint{4.469056in}{1.558094in}}%
\pgfpathlineto{\pgfqpoint{4.454365in}{1.567419in}}%
\pgfpathlineto{\pgfqpoint{4.439679in}{1.576768in}}%
\pgfpathlineto{\pgfqpoint{4.447788in}{1.558709in}}%
\pgfpathlineto{\pgfqpoint{4.455888in}{1.541455in}}%
\pgfpathlineto{\pgfqpoint{4.463979in}{1.524991in}}%
\pgfpathlineto{\pgfqpoint{4.472061in}{1.509301in}}%
\pgfpathclose%
\pgfusepath{fill}%
\end{pgfscope}%
\begin{pgfscope}%
\pgfpathrectangle{\pgfqpoint{1.150000in}{0.150000in}}{\pgfqpoint{5.700000in}{5.700000in}}%
\pgfusepath{clip}%
\pgfsetbuttcap%
\pgfsetroundjoin%
\definecolor{currentfill}{rgb}{0.174274,0.445044,0.557792}%
\pgfsetfillcolor{currentfill}%
\pgfsetfillopacity{0.700000}%
\pgfsetlinewidth{0.000000pt}%
\definecolor{currentstroke}{rgb}{0.000000,0.000000,0.000000}%
\pgfsetstrokecolor{currentstroke}%
\pgfsetdash{}{0pt}%
\pgfpathmoveto{\pgfqpoint{4.146853in}{1.768731in}}%
\pgfpathlineto{\pgfqpoint{4.161455in}{1.758905in}}%
\pgfpathlineto{\pgfqpoint{4.176061in}{1.749103in}}%
\pgfpathlineto{\pgfqpoint{4.190671in}{1.739325in}}%
\pgfpathlineto{\pgfqpoint{4.205285in}{1.729572in}}%
\pgfpathlineto{\pgfqpoint{4.196972in}{1.750509in}}%
\pgfpathlineto{\pgfqpoint{4.188645in}{1.772304in}}%
\pgfpathlineto{\pgfqpoint{4.180303in}{1.794973in}}%
\pgfpathlineto{\pgfqpoint{4.165650in}{1.805118in}}%
\pgfpathlineto{\pgfqpoint{4.151001in}{1.815286in}}%
\pgfpathlineto{\pgfqpoint{4.136356in}{1.825479in}}%
\pgfpathlineto{\pgfqpoint{4.121714in}{1.835697in}}%
\pgfpathlineto{\pgfqpoint{4.130109in}{1.812502in}}%
\pgfpathlineto{\pgfqpoint{4.138489in}{1.790185in}}%
\pgfpathlineto{\pgfqpoint{4.146853in}{1.768731in}}%
\pgfpathclose%
\pgfusepath{fill}%
\end{pgfscope}%
\begin{pgfscope}%
\pgfpathrectangle{\pgfqpoint{1.150000in}{0.150000in}}{\pgfqpoint{5.700000in}{5.700000in}}%
\pgfusepath{clip}%
\pgfsetbuttcap%
\pgfsetroundjoin%
\definecolor{currentfill}{rgb}{0.221989,0.339161,0.548752}%
\pgfsetfillcolor{currentfill}%
\pgfsetfillopacity{0.700000}%
\pgfsetlinewidth{0.000000pt}%
\definecolor{currentstroke}{rgb}{0.000000,0.000000,0.000000}%
\pgfsetstrokecolor{currentstroke}%
\pgfsetdash{}{0pt}%
\pgfpathmoveto{\pgfqpoint{4.530661in}{1.474051in}}%
\pgfpathlineto{\pgfqpoint{4.545323in}{1.465297in}}%
\pgfpathlineto{\pgfqpoint{4.559989in}{1.456566in}}%
\pgfpathlineto{\pgfqpoint{4.574661in}{1.447859in}}%
\pgfpathlineto{\pgfqpoint{4.589337in}{1.439175in}}%
\pgfpathlineto{\pgfqpoint{4.581337in}{1.453878in}}%
\pgfpathlineto{\pgfqpoint{4.573330in}{1.469346in}}%
\pgfpathlineto{\pgfqpoint{4.565317in}{1.485594in}}%
\pgfpathlineto{\pgfqpoint{4.557296in}{1.502638in}}%
\pgfpathlineto{\pgfqpoint{4.542578in}{1.511822in}}%
\pgfpathlineto{\pgfqpoint{4.527865in}{1.521030in}}%
\pgfpathlineto{\pgfqpoint{4.513156in}{1.530261in}}%
\pgfpathlineto{\pgfqpoint{4.498451in}{1.539515in}}%
\pgfpathlineto{\pgfqpoint{4.506515in}{1.521964in}}%
\pgfpathlineto{\pgfqpoint{4.514571in}{1.505213in}}%
\pgfpathlineto{\pgfqpoint{4.522620in}{1.489247in}}%
\pgfpathlineto{\pgfqpoint{4.530661in}{1.474051in}}%
\pgfpathclose%
\pgfusepath{fill}%
\end{pgfscope}%
\begin{pgfscope}%
\pgfpathrectangle{\pgfqpoint{1.150000in}{0.150000in}}{\pgfqpoint{5.700000in}{5.700000in}}%
\pgfusepath{clip}%
\pgfsetbuttcap%
\pgfsetroundjoin%
\definecolor{currentfill}{rgb}{0.179019,0.433756,0.557430}%
\pgfsetfillcolor{currentfill}%
\pgfsetfillopacity{0.700000}%
\pgfsetlinewidth{0.000000pt}%
\definecolor{currentstroke}{rgb}{0.000000,0.000000,0.000000}%
\pgfsetstrokecolor{currentstroke}%
\pgfsetdash{}{0pt}%
\pgfpathmoveto{\pgfqpoint{4.205285in}{1.729572in}}%
\pgfpathlineto{\pgfqpoint{4.219903in}{1.719842in}}%
\pgfpathlineto{\pgfqpoint{4.234525in}{1.710136in}}%
\pgfpathlineto{\pgfqpoint{4.249152in}{1.700455in}}%
\pgfpathlineto{\pgfqpoint{4.263782in}{1.690798in}}%
\pgfpathlineto{\pgfqpoint{4.255519in}{1.711220in}}%
\pgfpathlineto{\pgfqpoint{4.247243in}{1.732495in}}%
\pgfpathlineto{\pgfqpoint{4.238954in}{1.754638in}}%
\pgfpathlineto{\pgfqpoint{4.224285in}{1.764685in}}%
\pgfpathlineto{\pgfqpoint{4.209621in}{1.774757in}}%
\pgfpathlineto{\pgfqpoint{4.194960in}{1.784853in}}%
\pgfpathlineto{\pgfqpoint{4.180303in}{1.794973in}}%
\pgfpathlineto{\pgfqpoint{4.188645in}{1.772304in}}%
\pgfpathlineto{\pgfqpoint{4.196972in}{1.750509in}}%
\pgfpathlineto{\pgfqpoint{4.205285in}{1.729572in}}%
\pgfpathclose%
\pgfusepath{fill}%
\end{pgfscope}%
\begin{pgfscope}%
\pgfpathrectangle{\pgfqpoint{1.150000in}{0.150000in}}{\pgfqpoint{5.700000in}{5.700000in}}%
\pgfusepath{clip}%
\pgfsetbuttcap%
\pgfsetroundjoin%
\definecolor{currentfill}{rgb}{0.225863,0.330805,0.547314}%
\pgfsetfillcolor{currentfill}%
\pgfsetfillopacity{0.700000}%
\pgfsetlinewidth{0.000000pt}%
\definecolor{currentstroke}{rgb}{0.000000,0.000000,0.000000}%
\pgfsetstrokecolor{currentstroke}%
\pgfsetdash{}{0pt}%
\pgfpathmoveto{\pgfqpoint{4.589337in}{1.439175in}}%
\pgfpathlineto{\pgfqpoint{4.604018in}{1.430514in}}%
\pgfpathlineto{\pgfqpoint{4.618704in}{1.421876in}}%
\pgfpathlineto{\pgfqpoint{4.633396in}{1.413262in}}%
\pgfpathlineto{\pgfqpoint{4.648092in}{1.404670in}}%
\pgfpathlineto{\pgfqpoint{4.640131in}{1.418880in}}%
\pgfpathlineto{\pgfqpoint{4.632165in}{1.433851in}}%
\pgfpathlineto{\pgfqpoint{4.624194in}{1.449597in}}%
\pgfpathlineto{\pgfqpoint{4.616216in}{1.466134in}}%
\pgfpathlineto{\pgfqpoint{4.601479in}{1.475225in}}%
\pgfpathlineto{\pgfqpoint{4.586747in}{1.484339in}}%
\pgfpathlineto{\pgfqpoint{4.572019in}{1.493477in}}%
\pgfpathlineto{\pgfqpoint{4.557296in}{1.502638in}}%
\pgfpathlineto{\pgfqpoint{4.565317in}{1.485594in}}%
\pgfpathlineto{\pgfqpoint{4.573330in}{1.469346in}}%
\pgfpathlineto{\pgfqpoint{4.581337in}{1.453878in}}%
\pgfpathlineto{\pgfqpoint{4.589337in}{1.439175in}}%
\pgfpathclose%
\pgfusepath{fill}%
\end{pgfscope}%
\begin{pgfscope}%
\pgfpathrectangle{\pgfqpoint{1.150000in}{0.150000in}}{\pgfqpoint{5.700000in}{5.700000in}}%
\pgfusepath{clip}%
\pgfsetbuttcap%
\pgfsetroundjoin%
\definecolor{currentfill}{rgb}{0.183898,0.422383,0.556944}%
\pgfsetfillcolor{currentfill}%
\pgfsetfillopacity{0.700000}%
\pgfsetlinewidth{0.000000pt}%
\definecolor{currentstroke}{rgb}{0.000000,0.000000,0.000000}%
\pgfsetstrokecolor{currentstroke}%
\pgfsetdash{}{0pt}%
\pgfpathmoveto{\pgfqpoint{4.263782in}{1.690798in}}%
\pgfpathlineto{\pgfqpoint{4.278417in}{1.681164in}}%
\pgfpathlineto{\pgfqpoint{4.293055in}{1.671555in}}%
\pgfpathlineto{\pgfqpoint{4.307698in}{1.661969in}}%
\pgfpathlineto{\pgfqpoint{4.322346in}{1.652407in}}%
\pgfpathlineto{\pgfqpoint{4.314132in}{1.672315in}}%
\pgfpathlineto{\pgfqpoint{4.305906in}{1.693070in}}%
\pgfpathlineto{\pgfqpoint{4.297668in}{1.714688in}}%
\pgfpathlineto{\pgfqpoint{4.282984in}{1.724639in}}%
\pgfpathlineto{\pgfqpoint{4.268303in}{1.734615in}}%
\pgfpathlineto{\pgfqpoint{4.253627in}{1.744614in}}%
\pgfpathlineto{\pgfqpoint{4.238954in}{1.754638in}}%
\pgfpathlineto{\pgfqpoint{4.247243in}{1.732495in}}%
\pgfpathlineto{\pgfqpoint{4.255519in}{1.711220in}}%
\pgfpathlineto{\pgfqpoint{4.263782in}{1.690798in}}%
\pgfpathclose%
\pgfusepath{fill}%
\end{pgfscope}%
\begin{pgfscope}%
\pgfpathrectangle{\pgfqpoint{1.150000in}{0.150000in}}{\pgfqpoint{5.700000in}{5.700000in}}%
\pgfusepath{clip}%
\pgfsetbuttcap%
\pgfsetroundjoin%
\definecolor{currentfill}{rgb}{0.229739,0.322361,0.545706}%
\pgfsetfillcolor{currentfill}%
\pgfsetfillopacity{0.700000}%
\pgfsetlinewidth{0.000000pt}%
\definecolor{currentstroke}{rgb}{0.000000,0.000000,0.000000}%
\pgfsetstrokecolor{currentstroke}%
\pgfsetdash{}{0pt}%
\pgfpathmoveto{\pgfqpoint{4.648092in}{1.404670in}}%
\pgfpathlineto{\pgfqpoint{4.662793in}{1.396102in}}%
\pgfpathlineto{\pgfqpoint{4.677499in}{1.387557in}}%
\pgfpathlineto{\pgfqpoint{4.692210in}{1.379035in}}%
\pgfpathlineto{\pgfqpoint{4.706926in}{1.370536in}}%
\pgfpathlineto{\pgfqpoint{4.699005in}{1.384254in}}%
\pgfpathlineto{\pgfqpoint{4.691078in}{1.398727in}}%
\pgfpathlineto{\pgfqpoint{4.683147in}{1.413972in}}%
\pgfpathlineto{\pgfqpoint{4.675211in}{1.430003in}}%
\pgfpathlineto{\pgfqpoint{4.660455in}{1.439001in}}%
\pgfpathlineto{\pgfqpoint{4.645704in}{1.448022in}}%
\pgfpathlineto{\pgfqpoint{4.630958in}{1.457066in}}%
\pgfpathlineto{\pgfqpoint{4.616216in}{1.466134in}}%
\pgfpathlineto{\pgfqpoint{4.624194in}{1.449597in}}%
\pgfpathlineto{\pgfqpoint{4.632165in}{1.433851in}}%
\pgfpathlineto{\pgfqpoint{4.640131in}{1.418880in}}%
\pgfpathlineto{\pgfqpoint{4.648092in}{1.404670in}}%
\pgfpathclose%
\pgfusepath{fill}%
\end{pgfscope}%
\begin{pgfscope}%
\pgfpathrectangle{\pgfqpoint{1.150000in}{0.150000in}}{\pgfqpoint{5.700000in}{5.700000in}}%
\pgfusepath{clip}%
\pgfsetbuttcap%
\pgfsetroundjoin%
\definecolor{currentfill}{rgb}{0.187231,0.414746,0.556547}%
\pgfsetfillcolor{currentfill}%
\pgfsetfillopacity{0.700000}%
\pgfsetlinewidth{0.000000pt}%
\definecolor{currentstroke}{rgb}{0.000000,0.000000,0.000000}%
\pgfsetstrokecolor{currentstroke}%
\pgfsetdash{}{0pt}%
\pgfpathmoveto{\pgfqpoint{4.322346in}{1.652407in}}%
\pgfpathlineto{\pgfqpoint{4.336997in}{1.642869in}}%
\pgfpathlineto{\pgfqpoint{4.351653in}{1.633355in}}%
\pgfpathlineto{\pgfqpoint{4.366313in}{1.623865in}}%
\pgfpathlineto{\pgfqpoint{4.380977in}{1.614398in}}%
\pgfpathlineto{\pgfqpoint{4.372812in}{1.633792in}}%
\pgfpathlineto{\pgfqpoint{4.364635in}{1.654027in}}%
\pgfpathlineto{\pgfqpoint{4.356448in}{1.675121in}}%
\pgfpathlineto{\pgfqpoint{4.341747in}{1.684977in}}%
\pgfpathlineto{\pgfqpoint{4.327050in}{1.694857in}}%
\pgfpathlineto{\pgfqpoint{4.312357in}{1.704760in}}%
\pgfpathlineto{\pgfqpoint{4.297668in}{1.714688in}}%
\pgfpathlineto{\pgfqpoint{4.305906in}{1.693070in}}%
\pgfpathlineto{\pgfqpoint{4.314132in}{1.672315in}}%
\pgfpathlineto{\pgfqpoint{4.322346in}{1.652407in}}%
\pgfpathclose%
\pgfusepath{fill}%
\end{pgfscope}%
\begin{pgfscope}%
\pgfpathrectangle{\pgfqpoint{1.150000in}{0.150000in}}{\pgfqpoint{5.700000in}{5.700000in}}%
\pgfusepath{clip}%
\pgfsetbuttcap%
\pgfsetroundjoin%
\definecolor{currentfill}{rgb}{0.233603,0.313828,0.543914}%
\pgfsetfillcolor{currentfill}%
\pgfsetfillopacity{0.700000}%
\pgfsetlinewidth{0.000000pt}%
\definecolor{currentstroke}{rgb}{0.000000,0.000000,0.000000}%
\pgfsetstrokecolor{currentstroke}%
\pgfsetdash{}{0pt}%
\pgfpathmoveto{\pgfqpoint{4.706926in}{1.370536in}}%
\pgfpathlineto{\pgfqpoint{4.721648in}{1.362060in}}%
\pgfpathlineto{\pgfqpoint{4.736374in}{1.353607in}}%
\pgfpathlineto{\pgfqpoint{4.751106in}{1.345178in}}%
\pgfpathlineto{\pgfqpoint{4.743212in}{1.358526in}}%
\pgfpathlineto{\pgfqpoint{4.735315in}{1.372628in}}%
\pgfpathlineto{\pgfqpoint{4.727414in}{1.387496in}}%
\pgfpathlineto{\pgfqpoint{4.719509in}{1.403148in}}%
\pgfpathlineto{\pgfqpoint{4.704738in}{1.412076in}}%
\pgfpathlineto{\pgfqpoint{4.689972in}{1.421028in}}%
\pgfpathlineto{\pgfqpoint{4.675211in}{1.430003in}}%
\pgfpathlineto{\pgfqpoint{4.683147in}{1.413972in}}%
\pgfpathlineto{\pgfqpoint{4.691078in}{1.398727in}}%
\pgfpathlineto{\pgfqpoint{4.699005in}{1.384254in}}%
\pgfpathlineto{\pgfqpoint{4.706926in}{1.370536in}}%
\pgfpathclose%
\pgfusepath{fill}%
\end{pgfscope}%
\begin{pgfscope}%
\pgfpathrectangle{\pgfqpoint{1.150000in}{0.150000in}}{\pgfqpoint{5.700000in}{5.700000in}}%
\pgfusepath{clip}%
\pgfsetbuttcap%
\pgfsetroundjoin%
\definecolor{currentfill}{rgb}{0.192357,0.403199,0.555836}%
\pgfsetfillcolor{currentfill}%
\pgfsetfillopacity{0.700000}%
\pgfsetlinewidth{0.000000pt}%
\definecolor{currentstroke}{rgb}{0.000000,0.000000,0.000000}%
\pgfsetstrokecolor{currentstroke}%
\pgfsetdash{}{0pt}%
\pgfpathmoveto{\pgfqpoint{4.380977in}{1.614398in}}%
\pgfpathlineto{\pgfqpoint{4.395646in}{1.604955in}}%
\pgfpathlineto{\pgfqpoint{4.410319in}{1.595536in}}%
\pgfpathlineto{\pgfqpoint{4.424997in}{1.586140in}}%
\pgfpathlineto{\pgfqpoint{4.439679in}{1.576768in}}%
\pgfpathlineto{\pgfqpoint{4.431560in}{1.595648in}}%
\pgfpathlineto{\pgfqpoint{4.423432in}{1.615365in}}%
\pgfpathlineto{\pgfqpoint{4.415293in}{1.635936in}}%
\pgfpathlineto{\pgfqpoint{4.400576in}{1.645697in}}%
\pgfpathlineto{\pgfqpoint{4.385862in}{1.655481in}}%
\pgfpathlineto{\pgfqpoint{4.371153in}{1.665289in}}%
\pgfpathlineto{\pgfqpoint{4.356448in}{1.675121in}}%
\pgfpathlineto{\pgfqpoint{4.364635in}{1.654027in}}%
\pgfpathlineto{\pgfqpoint{4.372812in}{1.633792in}}%
\pgfpathlineto{\pgfqpoint{4.380977in}{1.614398in}}%
\pgfpathclose%
\pgfusepath{fill}%
\end{pgfscope}%
\begin{pgfscope}%
\pgfpathrectangle{\pgfqpoint{1.150000in}{0.150000in}}{\pgfqpoint{5.700000in}{5.700000in}}%
\pgfusepath{clip}%
\pgfsetbuttcap%
\pgfsetroundjoin%
\definecolor{currentfill}{rgb}{0.195860,0.395433,0.555276}%
\pgfsetfillcolor{currentfill}%
\pgfsetfillopacity{0.700000}%
\pgfsetlinewidth{0.000000pt}%
\definecolor{currentstroke}{rgb}{0.000000,0.000000,0.000000}%
\pgfsetstrokecolor{currentstroke}%
\pgfsetdash{}{0pt}%
\pgfpathmoveto{\pgfqpoint{4.439679in}{1.576768in}}%
\pgfpathlineto{\pgfqpoint{4.454365in}{1.567419in}}%
\pgfpathlineto{\pgfqpoint{4.469056in}{1.558094in}}%
\pgfpathlineto{\pgfqpoint{4.483751in}{1.548793in}}%
\pgfpathlineto{\pgfqpoint{4.498451in}{1.539515in}}%
\pgfpathlineto{\pgfqpoint{4.490379in}{1.557882in}}%
\pgfpathlineto{\pgfqpoint{4.482297in}{1.577082in}}%
\pgfpathlineto{\pgfqpoint{4.474207in}{1.597130in}}%
\pgfpathlineto{\pgfqpoint{4.459472in}{1.606796in}}%
\pgfpathlineto{\pgfqpoint{4.444742in}{1.616486in}}%
\pgfpathlineto{\pgfqpoint{4.430015in}{1.626199in}}%
\pgfpathlineto{\pgfqpoint{4.415293in}{1.635936in}}%
\pgfpathlineto{\pgfqpoint{4.423432in}{1.615365in}}%
\pgfpathlineto{\pgfqpoint{4.431560in}{1.595648in}}%
\pgfpathlineto{\pgfqpoint{4.439679in}{1.576768in}}%
\pgfpathclose%
\pgfusepath{fill}%
\end{pgfscope}%
\begin{pgfscope}%
\pgfpathrectangle{\pgfqpoint{1.150000in}{0.150000in}}{\pgfqpoint{5.700000in}{5.700000in}}%
\pgfusepath{clip}%
\pgfsetbuttcap%
\pgfsetroundjoin%
\definecolor{currentfill}{rgb}{0.201239,0.383670,0.554294}%
\pgfsetfillcolor{currentfill}%
\pgfsetfillopacity{0.700000}%
\pgfsetlinewidth{0.000000pt}%
\definecolor{currentstroke}{rgb}{0.000000,0.000000,0.000000}%
\pgfsetstrokecolor{currentstroke}%
\pgfsetdash{}{0pt}%
\pgfpathmoveto{\pgfqpoint{4.498451in}{1.539515in}}%
\pgfpathlineto{\pgfqpoint{4.513156in}{1.530261in}}%
\pgfpathlineto{\pgfqpoint{4.527865in}{1.521030in}}%
\pgfpathlineto{\pgfqpoint{4.542578in}{1.511822in}}%
\pgfpathlineto{\pgfqpoint{4.557296in}{1.502638in}}%
\pgfpathlineto{\pgfqpoint{4.549269in}{1.520493in}}%
\pgfpathlineto{\pgfqpoint{4.541234in}{1.539175in}}%
\pgfpathlineto{\pgfqpoint{4.533191in}{1.558702in}}%
\pgfpathlineto{\pgfqpoint{4.518438in}{1.568273in}}%
\pgfpathlineto{\pgfqpoint{4.503690in}{1.577869in}}%
\pgfpathlineto{\pgfqpoint{4.488946in}{1.587488in}}%
\pgfpathlineto{\pgfqpoint{4.474207in}{1.597130in}}%
\pgfpathlineto{\pgfqpoint{4.482297in}{1.577082in}}%
\pgfpathlineto{\pgfqpoint{4.490379in}{1.557882in}}%
\pgfpathlineto{\pgfqpoint{4.498451in}{1.539515in}}%
\pgfpathclose%
\pgfusepath{fill}%
\end{pgfscope}%
\begin{pgfscope}%
\pgfpathrectangle{\pgfqpoint{1.150000in}{0.150000in}}{\pgfqpoint{5.700000in}{5.700000in}}%
\pgfusepath{clip}%
\pgfsetbuttcap%
\pgfsetroundjoin%
\definecolor{currentfill}{rgb}{0.204903,0.375746,0.553533}%
\pgfsetfillcolor{currentfill}%
\pgfsetfillopacity{0.700000}%
\pgfsetlinewidth{0.000000pt}%
\definecolor{currentstroke}{rgb}{0.000000,0.000000,0.000000}%
\pgfsetstrokecolor{currentstroke}%
\pgfsetdash{}{0pt}%
\pgfpathmoveto{\pgfqpoint{4.557296in}{1.502638in}}%
\pgfpathlineto{\pgfqpoint{4.572019in}{1.493477in}}%
\pgfpathlineto{\pgfqpoint{4.586747in}{1.484339in}}%
\pgfpathlineto{\pgfqpoint{4.601479in}{1.475225in}}%
\pgfpathlineto{\pgfqpoint{4.616216in}{1.466134in}}%
\pgfpathlineto{\pgfqpoint{4.608232in}{1.483478in}}%
\pgfpathlineto{\pgfqpoint{4.600242in}{1.501644in}}%
\pgfpathlineto{\pgfqpoint{4.592245in}{1.520648in}}%
\pgfpathlineto{\pgfqpoint{4.577475in}{1.530127in}}%
\pgfpathlineto{\pgfqpoint{4.562709in}{1.539628in}}%
\pgfpathlineto{\pgfqpoint{4.547947in}{1.549153in}}%
\pgfpathlineto{\pgfqpoint{4.533191in}{1.558702in}}%
\pgfpathlineto{\pgfqpoint{4.541234in}{1.539175in}}%
\pgfpathlineto{\pgfqpoint{4.549269in}{1.520493in}}%
\pgfpathlineto{\pgfqpoint{4.557296in}{1.502638in}}%
\pgfpathclose%
\pgfusepath{fill}%
\end{pgfscope}%
\begin{pgfscope}%
\pgfpathrectangle{\pgfqpoint{1.150000in}{0.150000in}}{\pgfqpoint{5.700000in}{5.700000in}}%
\pgfusepath{clip}%
\pgfsetbuttcap%
\pgfsetroundjoin%
\definecolor{currentfill}{rgb}{0.208623,0.367752,0.552675}%
\pgfsetfillcolor{currentfill}%
\pgfsetfillopacity{0.700000}%
\pgfsetlinewidth{0.000000pt}%
\definecolor{currentstroke}{rgb}{0.000000,0.000000,0.000000}%
\pgfsetstrokecolor{currentstroke}%
\pgfsetdash{}{0pt}%
\pgfpathmoveto{\pgfqpoint{4.616216in}{1.466134in}}%
\pgfpathlineto{\pgfqpoint{4.630958in}{1.457066in}}%
\pgfpathlineto{\pgfqpoint{4.645704in}{1.448022in}}%
\pgfpathlineto{\pgfqpoint{4.660455in}{1.439001in}}%
\pgfpathlineto{\pgfqpoint{4.675211in}{1.430003in}}%
\pgfpathlineto{\pgfqpoint{4.667270in}{1.446835in}}%
\pgfpathlineto{\pgfqpoint{4.659324in}{1.464486in}}%
\pgfpathlineto{\pgfqpoint{4.651372in}{1.482969in}}%
\pgfpathlineto{\pgfqpoint{4.636583in}{1.492354in}}%
\pgfpathlineto{\pgfqpoint{4.621799in}{1.501762in}}%
\pgfpathlineto{\pgfqpoint{4.607020in}{1.511194in}}%
\pgfpathlineto{\pgfqpoint{4.592245in}{1.520648in}}%
\pgfpathlineto{\pgfqpoint{4.600242in}{1.501644in}}%
\pgfpathlineto{\pgfqpoint{4.608232in}{1.483478in}}%
\pgfpathlineto{\pgfqpoint{4.616216in}{1.466134in}}%
\pgfpathclose%
\pgfusepath{fill}%
\end{pgfscope}%
\begin{pgfscope}%
\pgfpathrectangle{\pgfqpoint{1.150000in}{0.150000in}}{\pgfqpoint{5.700000in}{5.700000in}}%
\pgfusepath{clip}%
\pgfsetbuttcap%
\pgfsetroundjoin%
\definecolor{currentfill}{rgb}{0.212395,0.359683,0.551710}%
\pgfsetfillcolor{currentfill}%
\pgfsetfillopacity{0.700000}%
\pgfsetlinewidth{0.000000pt}%
\definecolor{currentstroke}{rgb}{0.000000,0.000000,0.000000}%
\pgfsetstrokecolor{currentstroke}%
\pgfsetdash{}{0pt}%
\pgfpathmoveto{\pgfqpoint{4.675211in}{1.430003in}}%
\pgfpathlineto{\pgfqpoint{4.689972in}{1.421028in}}%
\pgfpathlineto{\pgfqpoint{4.704738in}{1.412076in}}%
\pgfpathlineto{\pgfqpoint{4.719509in}{1.403148in}}%
\pgfpathlineto{\pgfqpoint{4.711599in}{1.419597in}}%
\pgfpathlineto{\pgfqpoint{4.703685in}{1.436861in}}%
\pgfpathlineto{\pgfqpoint{4.695766in}{1.454955in}}%
\pgfpathlineto{\pgfqpoint{4.680963in}{1.464270in}}%
\pgfpathlineto{\pgfqpoint{4.666165in}{1.473608in}}%
\pgfpathlineto{\pgfqpoint{4.651372in}{1.482969in}}%
\pgfpathlineto{\pgfqpoint{4.659324in}{1.464486in}}%
\pgfpathlineto{\pgfqpoint{4.667270in}{1.446835in}}%
\pgfpathlineto{\pgfqpoint{4.675211in}{1.430003in}}%
\pgfpathclose%
\pgfusepath{fill}%
\end{pgfscope}%
\end{pgfpicture}%
\makeatother%
\endgroup%
}
        \caption{3D graf funkcie}
        \label{fig:newton_vpravo}
    \end{subfigure}
    
    \label{fig:newton_komplet}
\end{figure}


Ďalej si už ukážeme tabuľku súhrnných výsledkov Newtonovej metódy pre rôzne počiatočné body.




\newpage

\begin{table}[h!]
\centering
% \resizebox{šírka}{výška (zachovať pomer)}{obsah}
\resizebox{\textwidth}{!}{%
\begin{tabular}{|c|c|c|c|c|}
\hline
\textbf{Počiatočný bod} $x^{[0]}$ & \textbf{Iterácie} ($k$) & \textbf{Nájdené minimum} $\tilde{x}$ & \textbf{Hodnota} $f(\tilde{x})$ & \textbf{Rozdiel} $|f_k - f_{k-1}|$ \\ \hline
$[0.0; 0.0]$     & 3 & $[0.4003; -0.7596]$ & $1,569493$ & $0,000496$ \\ \hline
$[1.5; 0.5]$     & 6 & $[0.3796; -0.7386]$ & $1,569051$ & $0,000055$ \\ \hline
$[2.0; -2.0]$    & 4 & $[0.3921; -0.7485]$ & $1,569088$ & $0,000092$ \\ \hline
$[-1.0; -1.0]$   & 4 & $[0.3902; -0.7437]$ & $1,569006$ & $0,000028$ \\ \hline
$[-1.0; 1.0]$    & 5 & $[0.3826; -0.7352]$ & $1,569075$ & $0,000227$ \\ \hline
$[2.0; 2.0]$     & 6 & $[0.3829; -0.7354]$ & $1,569070$ & $0,000204$ \\ \hline
$[-2.0; -2.0]$   & 5 & $[0.3890; -0.7420]$ & $1,569001$ & $0,000010$ \\ \hline
$[0.5; -1.0]$    & 2 & $[0.3851; -0.7368]$ & $1,569022$ & $0,000063$ \\ \hline
$[3.0; 0.0]$     & 5 & $[0.3832; -0.7359]$ & $1,569064$ & $0,000188$ \\ \hline
$[0.0; 2.0]$     & 6 & $[0.3818; -0.7341]$ & $1,569092$ & $0,000277$ \\ \hline
$[5.0; 5.0]$     & 8 & $[0.3822; -0.7348]$ & $1,569080$ & $0,000150$ \\ \hline
$[-5.0; -5.0]$   & 7 & $[0.3895; -0.7429]$ & $1,569003$ & $0,000012$ \\ \hline
$[10.0; 0.0]$    & 9 & $[0.3835; -0.7362]$ & $1,569061$ & $0,000210$ \\ \hline
\end{tabular}%
}
\caption{Súhrnné výsledky Newtonovej metódy pre blízke aj vzdialené body}
\label{tab:newton_results_extended}
\end{table}

Je zrejmé, že aj pri extrémnom počiatočnom bode $[10; 0]$ metóda skonvergovala už po 9 iteráciách. To naznačuje, že Newtonova metóda je pre túto funkciu veľmi stabilná a dokáže efektívne pracovať aj pri veľkej vzdialenosti od optima, najmä vďaka presnej informácii o zakrivení poskytovanej Hessovou maticou.

Naskytuje sa otázka, ako moc by sa zmenili iterácie, keby sme zvolili  $\epsilon = 10^{-6}$, teda sprísnené ukončovacie kritérium. Výsledky môžeme vidieť v nasledujúcej tabuľke

\begin{table}[H]
\centering
% \resizebox{šírka}{výška (zachovať pomer)}{obsah}
\resizebox{\textwidth}{!}{%
\begin{tabular}{|c|c|c|c|}
\hline
\textbf{Počiatočný bod} $x^{[0]}$ & \textbf{Iterácie} ($k$) & \textbf{Nájdené minimum} $\tilde{x}$ & \textbf{Hodnota} $f(\tilde{x})$  \\ \hline
$[0.0; 0.0]$     & 4 & $[0.3881; -0.7409]$ & $1,568996$ \\ \hline
$[1.5; 0.5]$     & 7 & $[0.3881; -0.7409]$ & $1,568996$  \\ \hline
$[2.0; -2.0]$    & 5 & $[0.3881; -0.7409]$ & $1,568996$  \\ \hline
$[-1.0; -1.0]$   & 5 & $[0.3881; -0.7409]$ & $1,568996$  \\ \hline
$[-1.0; 1.0]$    & 6 & $[0.3881; -0.7409]$ & $1,568996$  \\ \hline
$[2.0; 2.0]$     & 7 & $[0.3881; -0.7409]$ & $1,568996$  \\ \hline
$[-2.0; -2.0]$   & 6 & $[0.3881; -0.7409]$ & $1,568996$  \\ \hline
$[0.5; -1.0]$    & 3 & $[0.3881; -0.7409]$ & $1,568996$  \\ \hline
$[3.0; 0.0]$     & 6 & $[0.3881; -0.7409]$ & $1,568996$  \\ \hline
$[0.0; 2.0]$     & 7 & $[0.3881; -0.7409]$ & $1,568996$  \\ \hline
$[5.0; 5.0]$     & 9 & $[0.3881; -0.7409]$ & $1,568996$  \\ \hline
$[-5.0; -5.0]$   & 8 & $[0.3881; -0.7409]$ & $1,568996$  \\ \hline
$[10.0; 0.0]$    & 10 & $[0.3881; -0.7409]$ & $1,568996$  \\ \hline
\end{tabular}%
}
\caption{Súhrnné výsledky Newtonovej metódy pre sprísnené kritérium $\epsilon = 10^{-6}$}
\label{tab:newton_results_strict}
\end{table}

Vidíme, že pre Newtonovu metódu je nastavenie $\epsilon = 0{,}001$ pomerne benevolentné. Vzhľadom na jej obrovskú rýchlosť by nastavenie $\epsilon = 10^{-6}$ stálo len zanedbateľný výpočtový čas navyše (maximálne 1-2 iterácie navyše), no prinieslo by oveľa presnejšie výsledky. Pri pomalších metódach (ako metoda najväčšieho spádu) by však sprísnenie kritéria mohlo znamenať stovky iterácií navyše.

\newpage

\subsection{Metóda združených gradientov (MSG)}

Táto metóda, v skriptách označovaná ako MSG, bola pôvodne navrhnutá pre riešenie sústav lineárnych rovníc s pozitívne definitnou maticou (čo zodpovedá minimalizácii kvadratickej funkcie). Pre kvadratické funkcie v $\mathbb{R}^n$ nájde MSG presné minimum najviac v $n$ krokoch.

Pre všeobecné nekvadratické funkcie (náš prípad) sa metóda modifikuje tak, že nový smer hľadania $h_k$ vhodne konštruujeme ako lineárnu kombináciu aktuálneho záporu gradientu a predchádzajúceho smeru
$$ h_k = - \nabla f(x^{[k]}) + \beta_{k-1} h_{k-1}, $$
kde pre prvý krok volíme $h_0 = -\nabla f(x^{[0]})$.
Koeficient $\beta_{k-1}$ zabezpečuje, aby boli smery združené. V našej analýze využijeme \textit{Fletcherov-Reevesov} vzorec
$$ \beta_{k-1}^{FR} = \frac{\|\nabla f(x^{[k]})\|^2}{\|\nabla f(x^{[k-1]})\|^2}. $$
Metóda sa pri nekvadratických funkciách zvyčajne po $n$ krokoch reštartuje (položí sa $\beta = 0$), aby sa eliminovali kumulované chyby a zachovala konvergencia. Štandardne sa volí dĺžka cyklu pred resetom $\beta$ tak, aby zodpovedala počtu premenných minimalizovanej funkcie. Hodnota $\beta_{k-1}$ sa teda bude rovnať $0$ pri každej párnej iterácii, t. j. pre $k = 2n, n \in \mathbb{N}_0$.

\vspace{0.3cm}

\noindent {\Large \textbf{Analýza metódy pri rôznych hodnotách počiatočnej aproximácie}}

\vspace{0.3cm}

Teraz sa zameriame na analýzu metódy združených gradientov v závislosti od rôznych počiatočných bodov iteračného procesu. Súbežne budeme skúmať variant MSG bez resetu, ako aj MSG s resetovaním hodnoty $\beta$ s dĺžkou cyklu 2, čo predstavuje štandardnú dĺžku vzhľadom na počet premenných funkcie $f(x, y)$.
Porovnanie vykonáme pre rovnaké počiatočné body ako pri Newtonovej metóde.
Opäť volíme ukončovaciu podmienku vzhľadom na zmenu funkčnej hodnoty
$$ |f(x^{[k]}) - f(x^{[k-1]})|=\epsilon < 0{,}001. $$
Naším cieľom bude sledovať počet krokov (iterácií), ktoré metóda potrebuje na dosiahnutie tohto kritéria pre rôzne zvolené body $x^{[0]}$.

%!!!!!!!!!!!!!!!!!!!!!!!!!!TREBA SPRAVIŤ AJ S TÝM RESETOM BETA!!!!!!!!!!!!!)))))))))))

\newpage
\subsubsection{MSG s nulováním $\beta$ po dvou iteracích}

\noindent \textbf{Počiatočný bod} $x^{[0]} = [0; 0]$ \\

\begin{table}[H]
    \centering
    \begin{tabular}{cccc}
        \toprule
        \textbf{Iterácia} & \textbf{Bod } $x^{[k]} = [x;y]$ & \textbf{Hodnota } $f(x^{[k]})$ & \textbf{Norma } $\|\nabla f\|$ \\
        \midrule
        0  & $[0.000000;\; 0.000000]$   & 2.000000 & -- \\
        1  & $[0.000000;\; -0.500000]$  & 1.669031 & 1.000000 \\
        2  & $[0.250000;\; -0.553265]$  & 1.607017 & 0.511223 \\
        \dots & \dots & \dots & \dots \\
        13 & $[0.388215;\; -0.741333]$  & 1.568997 & 0.015403 \\
        14 & $[0.388461;\; -0.740832]$  & 1.568997 & 0.002233 \\
        15 & $[0.388004;\; -0.740889]$  & 1.568996 & 0.001887 \\
        \bottomrule
    \end{tabular}
    \caption{Priebeh MSG pre $x^{[0]} = [0;0]$.}
\end{table}

Pri tomto počiatočnom bode vidíme výrazný rozdiel oproti Newtonovej metóde – konvergencia tu trvala až 15 iterácií, zatiaľ čo Newton to zvládol za 3. Graf ukazuje, ako sa metóda postupne približuje k minimu jemnejším cik-cak pohybom, čo je daň za to, že nepoužívame informácie z druhých derivácií (Hessovu maticu). Pravidelný reštart parametra $\beta$ každé dva kroky tu zjavne funguje a udržuje metódu stabilnú, aj keď je pomalšia.

\begin{figure}[H]
    \centering

    \begin{subfigure}{0.48\textwidth}
        \centering
        \resizebox{\linewidth}{!}{%% Creator: Matplotlib, PGF backend
%%
%% To include the figure in your LaTeX document, write
%%   \input{<filename>.pgf}
%%
%% Make sure the required packages are loaded in your preamble
%%   \usepackage{pgf}
%%
%% Also ensure that all the required font packages are loaded; for instance,
%% the lmodern package is sometimes necessary when using math font.
%%   \usepackage{lmodern}
%%
%% Figures using additional raster images can only be included by \input if
%% they are in the same directory as the main LaTeX file. For loading figures
%% from other directories you can use the `import` package
%%   \usepackage{import}
%%
%% and then include the figures with
%%   \import{<path to file>}{<filename>.pgf}
%%
%% Matplotlib used the following preamble
%%   
%%   \usepackage{fontspec}
%%   \setmainfont{DejaVuSerif.ttf}[Path=\detokenize{/home/radimek/Documents/projekt_mat_prog/mat_prog_kernel/lib/python3.12/site-packages/matplotlib/mpl-data/fonts/ttf/}]
%%   \setsansfont{DejaVuSans.ttf}[Path=\detokenize{/home/radimek/Documents/projekt_mat_prog/mat_prog_kernel/lib/python3.12/site-packages/matplotlib/mpl-data/fonts/ttf/}]
%%   \setmonofont{DejaVuSansMono.ttf}[Path=\detokenize{/home/radimek/Documents/projekt_mat_prog/mat_prog_kernel/lib/python3.12/site-packages/matplotlib/mpl-data/fonts/ttf/}]
%%   \makeatletter\@ifpackageloaded{underscore}{}{\usepackage[strings]{underscore}}\makeatother
%%
\begingroup%
\makeatletter%
\begin{pgfpicture}%
\pgfpathrectangle{\pgfpointorigin}{\pgfqpoint{7.000000in}{6.000000in}}%
\pgfusepath{use as bounding box, clip}%
\begin{pgfscope}%
\pgfsetbuttcap%
\pgfsetmiterjoin%
\definecolor{currentfill}{rgb}{1.000000,1.000000,1.000000}%
\pgfsetfillcolor{currentfill}%
\pgfsetlinewidth{0.000000pt}%
\definecolor{currentstroke}{rgb}{1.000000,1.000000,1.000000}%
\pgfsetstrokecolor{currentstroke}%
\pgfsetdash{}{0pt}%
\pgfpathmoveto{\pgfqpoint{0.000000in}{0.000000in}}%
\pgfpathlineto{\pgfqpoint{7.000000in}{0.000000in}}%
\pgfpathlineto{\pgfqpoint{7.000000in}{6.000000in}}%
\pgfpathlineto{\pgfqpoint{0.000000in}{6.000000in}}%
\pgfpathlineto{\pgfqpoint{0.000000in}{0.000000in}}%
\pgfpathclose%
\pgfusepath{fill}%
\end{pgfscope}%
\begin{pgfscope}%
\pgfsetbuttcap%
\pgfsetmiterjoin%
\definecolor{currentfill}{rgb}{1.000000,1.000000,1.000000}%
\pgfsetfillcolor{currentfill}%
\pgfsetlinewidth{0.000000pt}%
\definecolor{currentstroke}{rgb}{0.000000,0.000000,0.000000}%
\pgfsetstrokecolor{currentstroke}%
\pgfsetstrokeopacity{0.000000}%
\pgfsetdash{}{0pt}%
\pgfpathmoveto{\pgfqpoint{0.766095in}{0.571603in}}%
\pgfpathlineto{\pgfqpoint{6.695378in}{0.571603in}}%
\pgfpathlineto{\pgfqpoint{6.695378in}{5.640039in}}%
\pgfpathlineto{\pgfqpoint{0.766095in}{5.640039in}}%
\pgfpathlineto{\pgfqpoint{0.766095in}{0.571603in}}%
\pgfpathclose%
\pgfusepath{fill}%
\end{pgfscope}%
\begin{pgfscope}%
\pgfpathrectangle{\pgfqpoint{0.766095in}{0.571603in}}{\pgfqpoint{5.929283in}{5.068436in}}%
\pgfusepath{clip}%
\pgfsetbuttcap%
\pgfsetroundjoin%
\definecolor{currentfill}{rgb}{0.000000,0.000000,1.000000}%
\pgfsetfillcolor{currentfill}%
\pgfsetlinewidth{1.003750pt}%
\definecolor{currentstroke}{rgb}{0.000000,0.000000,1.000000}%
\pgfsetstrokecolor{currentstroke}%
\pgfsetdash{}{0pt}%
\pgfsys@defobject{currentmarker}{\pgfqpoint{-0.069444in}{-0.069444in}}{\pgfqpoint{0.069444in}{0.069444in}}{%
\pgfpathmoveto{\pgfqpoint{0.000000in}{-0.069444in}}%
\pgfpathcurveto{\pgfqpoint{0.018417in}{-0.069444in}}{\pgfqpoint{0.036082in}{-0.062127in}}{\pgfqpoint{0.049105in}{-0.049105in}}%
\pgfpathcurveto{\pgfqpoint{0.062127in}{-0.036082in}}{\pgfqpoint{0.069444in}{-0.018417in}}{\pgfqpoint{0.069444in}{0.000000in}}%
\pgfpathcurveto{\pgfqpoint{0.069444in}{0.018417in}}{\pgfqpoint{0.062127in}{0.036082in}}{\pgfqpoint{0.049105in}{0.049105in}}%
\pgfpathcurveto{\pgfqpoint{0.036082in}{0.062127in}}{\pgfqpoint{0.018417in}{0.069444in}}{\pgfqpoint{0.000000in}{0.069444in}}%
\pgfpathcurveto{\pgfqpoint{-0.018417in}{0.069444in}}{\pgfqpoint{-0.036082in}{0.062127in}}{\pgfqpoint{-0.049105in}{0.049105in}}%
\pgfpathcurveto{\pgfqpoint{-0.062127in}{0.036082in}}{\pgfqpoint{-0.069444in}{0.018417in}}{\pgfqpoint{-0.069444in}{0.000000in}}%
\pgfpathcurveto{\pgfqpoint{-0.069444in}{-0.018417in}}{\pgfqpoint{-0.062127in}{-0.036082in}}{\pgfqpoint{-0.049105in}{-0.049105in}}%
\pgfpathcurveto{\pgfqpoint{-0.036082in}{-0.062127in}}{\pgfqpoint{-0.018417in}{-0.069444in}}{\pgfqpoint{0.000000in}{-0.069444in}}%
\pgfpathlineto{\pgfqpoint{0.000000in}{-0.069444in}}%
\pgfpathclose%
\pgfusepath{stroke,fill}%
}%
\begin{pgfscope}%
\pgfsys@transformshift{4.881029in}{1.447129in}%
\pgfsys@useobject{currentmarker}{}%
\end{pgfscope}%
\end{pgfscope}%
\begin{pgfscope}%
\pgfsetbuttcap%
\pgfsetroundjoin%
\definecolor{currentfill}{rgb}{0.000000,0.000000,0.000000}%
\pgfsetfillcolor{currentfill}%
\pgfsetlinewidth{0.803000pt}%
\definecolor{currentstroke}{rgb}{0.000000,0.000000,0.000000}%
\pgfsetstrokecolor{currentstroke}%
\pgfsetdash{}{0pt}%
\pgfsys@defobject{currentmarker}{\pgfqpoint{0.000000in}{-0.048611in}}{\pgfqpoint{0.000000in}{0.000000in}}{%
\pgfpathmoveto{\pgfqpoint{0.000000in}{0.000000in}}%
\pgfpathlineto{\pgfqpoint{0.000000in}{-0.048611in}}%
\pgfusepath{stroke,fill}%
}%
\begin{pgfscope}%
\pgfsys@transformshift{0.766095in}{0.571603in}%
\pgfsys@useobject{currentmarker}{}%
\end{pgfscope}%
\end{pgfscope}%
\begin{pgfscope}%
\definecolor{textcolor}{rgb}{0.000000,0.000000,0.000000}%
\pgfsetstrokecolor{textcolor}%
\pgfsetfillcolor{textcolor}%
\pgftext[x=0.766095in,y=0.474381in,,top]{\color{textcolor}\sffamily\fontsize{10.000000}{12.000000}\selectfont \ensuremath{-}1.00}%
\end{pgfscope}%
\begin{pgfscope}%
\pgfsetbuttcap%
\pgfsetroundjoin%
\definecolor{currentfill}{rgb}{0.000000,0.000000,0.000000}%
\pgfsetfillcolor{currentfill}%
\pgfsetlinewidth{0.803000pt}%
\definecolor{currentstroke}{rgb}{0.000000,0.000000,0.000000}%
\pgfsetstrokecolor{currentstroke}%
\pgfsetdash{}{0pt}%
\pgfsys@defobject{currentmarker}{\pgfqpoint{0.000000in}{-0.048611in}}{\pgfqpoint{0.000000in}{0.000000in}}{%
\pgfpathmoveto{\pgfqpoint{0.000000in}{0.000000in}}%
\pgfpathlineto{\pgfqpoint{0.000000in}{-0.048611in}}%
\pgfusepath{stroke,fill}%
}%
\begin{pgfscope}%
\pgfsys@transformshift{1.507255in}{0.571603in}%
\pgfsys@useobject{currentmarker}{}%
\end{pgfscope}%
\end{pgfscope}%
\begin{pgfscope}%
\definecolor{textcolor}{rgb}{0.000000,0.000000,0.000000}%
\pgfsetstrokecolor{textcolor}%
\pgfsetfillcolor{textcolor}%
\pgftext[x=1.507255in,y=0.474381in,,top]{\color{textcolor}\sffamily\fontsize{10.000000}{12.000000}\selectfont \ensuremath{-}0.75}%
\end{pgfscope}%
\begin{pgfscope}%
\pgfsetbuttcap%
\pgfsetroundjoin%
\definecolor{currentfill}{rgb}{0.000000,0.000000,0.000000}%
\pgfsetfillcolor{currentfill}%
\pgfsetlinewidth{0.803000pt}%
\definecolor{currentstroke}{rgb}{0.000000,0.000000,0.000000}%
\pgfsetstrokecolor{currentstroke}%
\pgfsetdash{}{0pt}%
\pgfsys@defobject{currentmarker}{\pgfqpoint{0.000000in}{-0.048611in}}{\pgfqpoint{0.000000in}{0.000000in}}{%
\pgfpathmoveto{\pgfqpoint{0.000000in}{0.000000in}}%
\pgfpathlineto{\pgfqpoint{0.000000in}{-0.048611in}}%
\pgfusepath{stroke,fill}%
}%
\begin{pgfscope}%
\pgfsys@transformshift{2.248416in}{0.571603in}%
\pgfsys@useobject{currentmarker}{}%
\end{pgfscope}%
\end{pgfscope}%
\begin{pgfscope}%
\definecolor{textcolor}{rgb}{0.000000,0.000000,0.000000}%
\pgfsetstrokecolor{textcolor}%
\pgfsetfillcolor{textcolor}%
\pgftext[x=2.248416in,y=0.474381in,,top]{\color{textcolor}\sffamily\fontsize{10.000000}{12.000000}\selectfont \ensuremath{-}0.50}%
\end{pgfscope}%
\begin{pgfscope}%
\pgfsetbuttcap%
\pgfsetroundjoin%
\definecolor{currentfill}{rgb}{0.000000,0.000000,0.000000}%
\pgfsetfillcolor{currentfill}%
\pgfsetlinewidth{0.803000pt}%
\definecolor{currentstroke}{rgb}{0.000000,0.000000,0.000000}%
\pgfsetstrokecolor{currentstroke}%
\pgfsetdash{}{0pt}%
\pgfsys@defobject{currentmarker}{\pgfqpoint{0.000000in}{-0.048611in}}{\pgfqpoint{0.000000in}{0.000000in}}{%
\pgfpathmoveto{\pgfqpoint{0.000000in}{0.000000in}}%
\pgfpathlineto{\pgfqpoint{0.000000in}{-0.048611in}}%
\pgfusepath{stroke,fill}%
}%
\begin{pgfscope}%
\pgfsys@transformshift{2.989576in}{0.571603in}%
\pgfsys@useobject{currentmarker}{}%
\end{pgfscope}%
\end{pgfscope}%
\begin{pgfscope}%
\definecolor{textcolor}{rgb}{0.000000,0.000000,0.000000}%
\pgfsetstrokecolor{textcolor}%
\pgfsetfillcolor{textcolor}%
\pgftext[x=2.989576in,y=0.474381in,,top]{\color{textcolor}\sffamily\fontsize{10.000000}{12.000000}\selectfont \ensuremath{-}0.25}%
\end{pgfscope}%
\begin{pgfscope}%
\pgfsetbuttcap%
\pgfsetroundjoin%
\definecolor{currentfill}{rgb}{0.000000,0.000000,0.000000}%
\pgfsetfillcolor{currentfill}%
\pgfsetlinewidth{0.803000pt}%
\definecolor{currentstroke}{rgb}{0.000000,0.000000,0.000000}%
\pgfsetstrokecolor{currentstroke}%
\pgfsetdash{}{0pt}%
\pgfsys@defobject{currentmarker}{\pgfqpoint{0.000000in}{-0.048611in}}{\pgfqpoint{0.000000in}{0.000000in}}{%
\pgfpathmoveto{\pgfqpoint{0.000000in}{0.000000in}}%
\pgfpathlineto{\pgfqpoint{0.000000in}{-0.048611in}}%
\pgfusepath{stroke,fill}%
}%
\begin{pgfscope}%
\pgfsys@transformshift{3.730736in}{0.571603in}%
\pgfsys@useobject{currentmarker}{}%
\end{pgfscope}%
\end{pgfscope}%
\begin{pgfscope}%
\definecolor{textcolor}{rgb}{0.000000,0.000000,0.000000}%
\pgfsetstrokecolor{textcolor}%
\pgfsetfillcolor{textcolor}%
\pgftext[x=3.730736in,y=0.474381in,,top]{\color{textcolor}\sffamily\fontsize{10.000000}{12.000000}\selectfont 0.00}%
\end{pgfscope}%
\begin{pgfscope}%
\pgfsetbuttcap%
\pgfsetroundjoin%
\definecolor{currentfill}{rgb}{0.000000,0.000000,0.000000}%
\pgfsetfillcolor{currentfill}%
\pgfsetlinewidth{0.803000pt}%
\definecolor{currentstroke}{rgb}{0.000000,0.000000,0.000000}%
\pgfsetstrokecolor{currentstroke}%
\pgfsetdash{}{0pt}%
\pgfsys@defobject{currentmarker}{\pgfqpoint{0.000000in}{-0.048611in}}{\pgfqpoint{0.000000in}{0.000000in}}{%
\pgfpathmoveto{\pgfqpoint{0.000000in}{0.000000in}}%
\pgfpathlineto{\pgfqpoint{0.000000in}{-0.048611in}}%
\pgfusepath{stroke,fill}%
}%
\begin{pgfscope}%
\pgfsys@transformshift{4.471897in}{0.571603in}%
\pgfsys@useobject{currentmarker}{}%
\end{pgfscope}%
\end{pgfscope}%
\begin{pgfscope}%
\definecolor{textcolor}{rgb}{0.000000,0.000000,0.000000}%
\pgfsetstrokecolor{textcolor}%
\pgfsetfillcolor{textcolor}%
\pgftext[x=4.471897in,y=0.474381in,,top]{\color{textcolor}\sffamily\fontsize{10.000000}{12.000000}\selectfont 0.25}%
\end{pgfscope}%
\begin{pgfscope}%
\pgfsetbuttcap%
\pgfsetroundjoin%
\definecolor{currentfill}{rgb}{0.000000,0.000000,0.000000}%
\pgfsetfillcolor{currentfill}%
\pgfsetlinewidth{0.803000pt}%
\definecolor{currentstroke}{rgb}{0.000000,0.000000,0.000000}%
\pgfsetstrokecolor{currentstroke}%
\pgfsetdash{}{0pt}%
\pgfsys@defobject{currentmarker}{\pgfqpoint{0.000000in}{-0.048611in}}{\pgfqpoint{0.000000in}{0.000000in}}{%
\pgfpathmoveto{\pgfqpoint{0.000000in}{0.000000in}}%
\pgfpathlineto{\pgfqpoint{0.000000in}{-0.048611in}}%
\pgfusepath{stroke,fill}%
}%
\begin{pgfscope}%
\pgfsys@transformshift{5.213057in}{0.571603in}%
\pgfsys@useobject{currentmarker}{}%
\end{pgfscope}%
\end{pgfscope}%
\begin{pgfscope}%
\definecolor{textcolor}{rgb}{0.000000,0.000000,0.000000}%
\pgfsetstrokecolor{textcolor}%
\pgfsetfillcolor{textcolor}%
\pgftext[x=5.213057in,y=0.474381in,,top]{\color{textcolor}\sffamily\fontsize{10.000000}{12.000000}\selectfont 0.50}%
\end{pgfscope}%
\begin{pgfscope}%
\pgfsetbuttcap%
\pgfsetroundjoin%
\definecolor{currentfill}{rgb}{0.000000,0.000000,0.000000}%
\pgfsetfillcolor{currentfill}%
\pgfsetlinewidth{0.803000pt}%
\definecolor{currentstroke}{rgb}{0.000000,0.000000,0.000000}%
\pgfsetstrokecolor{currentstroke}%
\pgfsetdash{}{0pt}%
\pgfsys@defobject{currentmarker}{\pgfqpoint{0.000000in}{-0.048611in}}{\pgfqpoint{0.000000in}{0.000000in}}{%
\pgfpathmoveto{\pgfqpoint{0.000000in}{0.000000in}}%
\pgfpathlineto{\pgfqpoint{0.000000in}{-0.048611in}}%
\pgfusepath{stroke,fill}%
}%
\begin{pgfscope}%
\pgfsys@transformshift{5.954217in}{0.571603in}%
\pgfsys@useobject{currentmarker}{}%
\end{pgfscope}%
\end{pgfscope}%
\begin{pgfscope}%
\definecolor{textcolor}{rgb}{0.000000,0.000000,0.000000}%
\pgfsetstrokecolor{textcolor}%
\pgfsetfillcolor{textcolor}%
\pgftext[x=5.954217in,y=0.474381in,,top]{\color{textcolor}\sffamily\fontsize{10.000000}{12.000000}\selectfont 0.75}%
\end{pgfscope}%
\begin{pgfscope}%
\pgfsetbuttcap%
\pgfsetroundjoin%
\definecolor{currentfill}{rgb}{0.000000,0.000000,0.000000}%
\pgfsetfillcolor{currentfill}%
\pgfsetlinewidth{0.803000pt}%
\definecolor{currentstroke}{rgb}{0.000000,0.000000,0.000000}%
\pgfsetstrokecolor{currentstroke}%
\pgfsetdash{}{0pt}%
\pgfsys@defobject{currentmarker}{\pgfqpoint{0.000000in}{-0.048611in}}{\pgfqpoint{0.000000in}{0.000000in}}{%
\pgfpathmoveto{\pgfqpoint{0.000000in}{0.000000in}}%
\pgfpathlineto{\pgfqpoint{0.000000in}{-0.048611in}}%
\pgfusepath{stroke,fill}%
}%
\begin{pgfscope}%
\pgfsys@transformshift{6.695378in}{0.571603in}%
\pgfsys@useobject{currentmarker}{}%
\end{pgfscope}%
\end{pgfscope}%
\begin{pgfscope}%
\definecolor{textcolor}{rgb}{0.000000,0.000000,0.000000}%
\pgfsetstrokecolor{textcolor}%
\pgfsetfillcolor{textcolor}%
\pgftext[x=6.695378in,y=0.474381in,,top]{\color{textcolor}\sffamily\fontsize{10.000000}{12.000000}\selectfont 1.00}%
\end{pgfscope}%
\begin{pgfscope}%
\definecolor{textcolor}{rgb}{0.000000,0.000000,0.000000}%
\pgfsetstrokecolor{textcolor}%
\pgfsetfillcolor{textcolor}%
\pgftext[x=3.730736in,y=0.284413in,,top]{\color{textcolor}\sffamily\fontsize{10.000000}{12.000000}\selectfont x}%
\end{pgfscope}%
\begin{pgfscope}%
\pgfsetbuttcap%
\pgfsetroundjoin%
\definecolor{currentfill}{rgb}{0.000000,0.000000,0.000000}%
\pgfsetfillcolor{currentfill}%
\pgfsetlinewidth{0.803000pt}%
\definecolor{currentstroke}{rgb}{0.000000,0.000000,0.000000}%
\pgfsetstrokecolor{currentstroke}%
\pgfsetdash{}{0pt}%
\pgfsys@defobject{currentmarker}{\pgfqpoint{-0.048611in}{0.000000in}}{\pgfqpoint{-0.000000in}{0.000000in}}{%
\pgfpathmoveto{\pgfqpoint{-0.000000in}{0.000000in}}%
\pgfpathlineto{\pgfqpoint{-0.048611in}{0.000000in}}%
\pgfusepath{stroke,fill}%
}%
\begin{pgfscope}%
\pgfsys@transformshift{0.766095in}{0.571603in}%
\pgfsys@useobject{currentmarker}{}%
\end{pgfscope}%
\end{pgfscope}%
\begin{pgfscope}%
\definecolor{textcolor}{rgb}{0.000000,0.000000,0.000000}%
\pgfsetstrokecolor{textcolor}%
\pgfsetfillcolor{textcolor}%
\pgftext[x=0.339968in, y=0.518842in, left, base]{\color{textcolor}\sffamily\fontsize{10.000000}{12.000000}\selectfont \ensuremath{-}1.0}%
\end{pgfscope}%
\begin{pgfscope}%
\pgfsetbuttcap%
\pgfsetroundjoin%
\definecolor{currentfill}{rgb}{0.000000,0.000000,0.000000}%
\pgfsetfillcolor{currentfill}%
\pgfsetlinewidth{0.803000pt}%
\definecolor{currentstroke}{rgb}{0.000000,0.000000,0.000000}%
\pgfsetstrokecolor{currentstroke}%
\pgfsetdash{}{0pt}%
\pgfsys@defobject{currentmarker}{\pgfqpoint{-0.048611in}{0.000000in}}{\pgfqpoint{-0.000000in}{0.000000in}}{%
\pgfpathmoveto{\pgfqpoint{-0.000000in}{0.000000in}}%
\pgfpathlineto{\pgfqpoint{-0.048611in}{0.000000in}}%
\pgfusepath{stroke,fill}%
}%
\begin{pgfscope}%
\pgfsys@transformshift{0.766095in}{1.247395in}%
\pgfsys@useobject{currentmarker}{}%
\end{pgfscope}%
\end{pgfscope}%
\begin{pgfscope}%
\definecolor{textcolor}{rgb}{0.000000,0.000000,0.000000}%
\pgfsetstrokecolor{textcolor}%
\pgfsetfillcolor{textcolor}%
\pgftext[x=0.339968in, y=1.194633in, left, base]{\color{textcolor}\sffamily\fontsize{10.000000}{12.000000}\selectfont \ensuremath{-}0.8}%
\end{pgfscope}%
\begin{pgfscope}%
\pgfsetbuttcap%
\pgfsetroundjoin%
\definecolor{currentfill}{rgb}{0.000000,0.000000,0.000000}%
\pgfsetfillcolor{currentfill}%
\pgfsetlinewidth{0.803000pt}%
\definecolor{currentstroke}{rgb}{0.000000,0.000000,0.000000}%
\pgfsetstrokecolor{currentstroke}%
\pgfsetdash{}{0pt}%
\pgfsys@defobject{currentmarker}{\pgfqpoint{-0.048611in}{0.000000in}}{\pgfqpoint{-0.000000in}{0.000000in}}{%
\pgfpathmoveto{\pgfqpoint{-0.000000in}{0.000000in}}%
\pgfpathlineto{\pgfqpoint{-0.048611in}{0.000000in}}%
\pgfusepath{stroke,fill}%
}%
\begin{pgfscope}%
\pgfsys@transformshift{0.766095in}{1.923186in}%
\pgfsys@useobject{currentmarker}{}%
\end{pgfscope}%
\end{pgfscope}%
\begin{pgfscope}%
\definecolor{textcolor}{rgb}{0.000000,0.000000,0.000000}%
\pgfsetstrokecolor{textcolor}%
\pgfsetfillcolor{textcolor}%
\pgftext[x=0.339968in, y=1.870425in, left, base]{\color{textcolor}\sffamily\fontsize{10.000000}{12.000000}\selectfont \ensuremath{-}0.6}%
\end{pgfscope}%
\begin{pgfscope}%
\pgfsetbuttcap%
\pgfsetroundjoin%
\definecolor{currentfill}{rgb}{0.000000,0.000000,0.000000}%
\pgfsetfillcolor{currentfill}%
\pgfsetlinewidth{0.803000pt}%
\definecolor{currentstroke}{rgb}{0.000000,0.000000,0.000000}%
\pgfsetstrokecolor{currentstroke}%
\pgfsetdash{}{0pt}%
\pgfsys@defobject{currentmarker}{\pgfqpoint{-0.048611in}{0.000000in}}{\pgfqpoint{-0.000000in}{0.000000in}}{%
\pgfpathmoveto{\pgfqpoint{-0.000000in}{0.000000in}}%
\pgfpathlineto{\pgfqpoint{-0.048611in}{0.000000in}}%
\pgfusepath{stroke,fill}%
}%
\begin{pgfscope}%
\pgfsys@transformshift{0.766095in}{2.598978in}%
\pgfsys@useobject{currentmarker}{}%
\end{pgfscope}%
\end{pgfscope}%
\begin{pgfscope}%
\definecolor{textcolor}{rgb}{0.000000,0.000000,0.000000}%
\pgfsetstrokecolor{textcolor}%
\pgfsetfillcolor{textcolor}%
\pgftext[x=0.339968in, y=2.546216in, left, base]{\color{textcolor}\sffamily\fontsize{10.000000}{12.000000}\selectfont \ensuremath{-}0.4}%
\end{pgfscope}%
\begin{pgfscope}%
\pgfsetbuttcap%
\pgfsetroundjoin%
\definecolor{currentfill}{rgb}{0.000000,0.000000,0.000000}%
\pgfsetfillcolor{currentfill}%
\pgfsetlinewidth{0.803000pt}%
\definecolor{currentstroke}{rgb}{0.000000,0.000000,0.000000}%
\pgfsetstrokecolor{currentstroke}%
\pgfsetdash{}{0pt}%
\pgfsys@defobject{currentmarker}{\pgfqpoint{-0.048611in}{0.000000in}}{\pgfqpoint{-0.000000in}{0.000000in}}{%
\pgfpathmoveto{\pgfqpoint{-0.000000in}{0.000000in}}%
\pgfpathlineto{\pgfqpoint{-0.048611in}{0.000000in}}%
\pgfusepath{stroke,fill}%
}%
\begin{pgfscope}%
\pgfsys@transformshift{0.766095in}{3.274769in}%
\pgfsys@useobject{currentmarker}{}%
\end{pgfscope}%
\end{pgfscope}%
\begin{pgfscope}%
\definecolor{textcolor}{rgb}{0.000000,0.000000,0.000000}%
\pgfsetstrokecolor{textcolor}%
\pgfsetfillcolor{textcolor}%
\pgftext[x=0.339968in, y=3.222008in, left, base]{\color{textcolor}\sffamily\fontsize{10.000000}{12.000000}\selectfont \ensuremath{-}0.2}%
\end{pgfscope}%
\begin{pgfscope}%
\pgfsetbuttcap%
\pgfsetroundjoin%
\definecolor{currentfill}{rgb}{0.000000,0.000000,0.000000}%
\pgfsetfillcolor{currentfill}%
\pgfsetlinewidth{0.803000pt}%
\definecolor{currentstroke}{rgb}{0.000000,0.000000,0.000000}%
\pgfsetstrokecolor{currentstroke}%
\pgfsetdash{}{0pt}%
\pgfsys@defobject{currentmarker}{\pgfqpoint{-0.048611in}{0.000000in}}{\pgfqpoint{-0.000000in}{0.000000in}}{%
\pgfpathmoveto{\pgfqpoint{-0.000000in}{0.000000in}}%
\pgfpathlineto{\pgfqpoint{-0.048611in}{0.000000in}}%
\pgfusepath{stroke,fill}%
}%
\begin{pgfscope}%
\pgfsys@transformshift{0.766095in}{3.950560in}%
\pgfsys@useobject{currentmarker}{}%
\end{pgfscope}%
\end{pgfscope}%
\begin{pgfscope}%
\definecolor{textcolor}{rgb}{0.000000,0.000000,0.000000}%
\pgfsetstrokecolor{textcolor}%
\pgfsetfillcolor{textcolor}%
\pgftext[x=0.447993in, y=3.897799in, left, base]{\color{textcolor}\sffamily\fontsize{10.000000}{12.000000}\selectfont 0.0}%
\end{pgfscope}%
\begin{pgfscope}%
\pgfsetbuttcap%
\pgfsetroundjoin%
\definecolor{currentfill}{rgb}{0.000000,0.000000,0.000000}%
\pgfsetfillcolor{currentfill}%
\pgfsetlinewidth{0.803000pt}%
\definecolor{currentstroke}{rgb}{0.000000,0.000000,0.000000}%
\pgfsetstrokecolor{currentstroke}%
\pgfsetdash{}{0pt}%
\pgfsys@defobject{currentmarker}{\pgfqpoint{-0.048611in}{0.000000in}}{\pgfqpoint{-0.000000in}{0.000000in}}{%
\pgfpathmoveto{\pgfqpoint{-0.000000in}{0.000000in}}%
\pgfpathlineto{\pgfqpoint{-0.048611in}{0.000000in}}%
\pgfusepath{stroke,fill}%
}%
\begin{pgfscope}%
\pgfsys@transformshift{0.766095in}{4.626352in}%
\pgfsys@useobject{currentmarker}{}%
\end{pgfscope}%
\end{pgfscope}%
\begin{pgfscope}%
\definecolor{textcolor}{rgb}{0.000000,0.000000,0.000000}%
\pgfsetstrokecolor{textcolor}%
\pgfsetfillcolor{textcolor}%
\pgftext[x=0.447993in, y=4.573590in, left, base]{\color{textcolor}\sffamily\fontsize{10.000000}{12.000000}\selectfont 0.2}%
\end{pgfscope}%
\begin{pgfscope}%
\pgfsetbuttcap%
\pgfsetroundjoin%
\definecolor{currentfill}{rgb}{0.000000,0.000000,0.000000}%
\pgfsetfillcolor{currentfill}%
\pgfsetlinewidth{0.803000pt}%
\definecolor{currentstroke}{rgb}{0.000000,0.000000,0.000000}%
\pgfsetstrokecolor{currentstroke}%
\pgfsetdash{}{0pt}%
\pgfsys@defobject{currentmarker}{\pgfqpoint{-0.048611in}{0.000000in}}{\pgfqpoint{-0.000000in}{0.000000in}}{%
\pgfpathmoveto{\pgfqpoint{-0.000000in}{0.000000in}}%
\pgfpathlineto{\pgfqpoint{-0.048611in}{0.000000in}}%
\pgfusepath{stroke,fill}%
}%
\begin{pgfscope}%
\pgfsys@transformshift{0.766095in}{5.302143in}%
\pgfsys@useobject{currentmarker}{}%
\end{pgfscope}%
\end{pgfscope}%
\begin{pgfscope}%
\definecolor{textcolor}{rgb}{0.000000,0.000000,0.000000}%
\pgfsetstrokecolor{textcolor}%
\pgfsetfillcolor{textcolor}%
\pgftext[x=0.447993in, y=5.249382in, left, base]{\color{textcolor}\sffamily\fontsize{10.000000}{12.000000}\selectfont 0.4}%
\end{pgfscope}%
\begin{pgfscope}%
\definecolor{textcolor}{rgb}{0.000000,0.000000,0.000000}%
\pgfsetstrokecolor{textcolor}%
\pgfsetfillcolor{textcolor}%
\pgftext[x=0.284413in,y=3.105821in,,bottom,rotate=90.000000]{\color{textcolor}\sffamily\fontsize{10.000000}{12.000000}\selectfont y}%
\end{pgfscope}%
\begin{pgfscope}%
\pgfpathrectangle{\pgfqpoint{0.766095in}{0.571603in}}{\pgfqpoint{5.929283in}{5.068436in}}%
\pgfusepath{clip}%
\pgfsetbuttcap%
\pgfsetroundjoin%
\pgfsetlinewidth{1.505625pt}%
\definecolor{currentstroke}{rgb}{0.273809,0.031497,0.358853}%
\pgfsetstrokecolor{currentstroke}%
\pgfsetdash{}{0pt}%
\pgfpathmoveto{\pgfqpoint{5.324790in}{0.924907in}}%
\pgfpathlineto{\pgfqpoint{5.311613in}{0.928177in}}%
\pgfpathlineto{\pgfqpoint{5.294994in}{0.932481in}}%
\pgfpathlineto{\pgfqpoint{5.265199in}{0.940990in}}%
\pgfpathlineto{\pgfqpoint{5.235403in}{0.949954in}}%
\pgfpathlineto{\pgfqpoint{5.224037in}{0.953646in}}%
\pgfpathlineto{\pgfqpoint{5.205608in}{0.959873in}}%
\pgfpathlineto{\pgfqpoint{5.175813in}{0.970443in}}%
\pgfpathlineto{\pgfqpoint{5.152123in}{0.979116in}}%
\pgfpathlineto{\pgfqpoint{5.146017in}{0.981438in}}%
\pgfpathlineto{\pgfqpoint{5.116222in}{0.993423in}}%
\pgfpathlineto{\pgfqpoint{5.088770in}{1.004585in}}%
\pgfpathlineto{\pgfqpoint{5.086427in}{1.005574in}}%
\pgfpathlineto{\pgfqpoint{5.056631in}{1.018805in}}%
\pgfpathlineto{\pgfqpoint{5.031435in}{1.030055in}}%
\pgfpathlineto{\pgfqpoint{5.026836in}{1.032183in}}%
\pgfpathlineto{\pgfqpoint{4.997040in}{1.046509in}}%
\pgfpathlineto{\pgfqpoint{4.978484in}{1.055524in}}%
\pgfpathlineto{\pgfqpoint{4.967245in}{1.061178in}}%
\pgfpathlineto{\pgfqpoint{4.937450in}{1.076464in}}%
\pgfpathlineto{\pgfqpoint{4.928815in}{1.080994in}}%
\pgfpathlineto{\pgfqpoint{4.907654in}{1.092477in}}%
\pgfpathlineto{\pgfqpoint{4.881995in}{1.106463in}}%
\pgfpathlineto{\pgfqpoint{4.877859in}{1.108796in}}%
\pgfpathlineto{\pgfqpoint{4.848063in}{1.126011in}}%
\pgfpathlineto{\pgfqpoint{4.837937in}{1.131933in}}%
\pgfpathlineto{\pgfqpoint{4.818268in}{1.143823in}}%
\pgfpathlineto{\pgfqpoint{4.795939in}{1.157402in}}%
\pgfpathlineto{\pgfqpoint{4.788473in}{1.162098in}}%
\pgfpathlineto{\pgfqpoint{4.758677in}{1.181061in}}%
\pgfpathlineto{\pgfqpoint{4.755886in}{1.182872in}}%
\pgfpathlineto{\pgfqpoint{4.728882in}{1.200970in}}%
\pgfpathlineto{\pgfqpoint{4.717962in}{1.208341in}}%
\pgfpathlineto{\pgfqpoint{4.699086in}{1.221510in}}%
\pgfpathlineto{\pgfqpoint{4.681595in}{1.233811in}}%
\pgfpathlineto{\pgfqpoint{4.669291in}{1.242756in}}%
\pgfpathlineto{\pgfqpoint{4.646760in}{1.259281in}}%
\pgfpathlineto{\pgfqpoint{4.639496in}{1.264790in}}%
\pgfpathlineto{\pgfqpoint{4.613425in}{1.284750in}}%
\pgfpathlineto{\pgfqpoint{4.609700in}{1.287700in}}%
\pgfpathlineto{\pgfqpoint{4.581554in}{1.310220in}}%
\pgfpathlineto{\pgfqpoint{4.579905in}{1.311585in}}%
\pgfpathlineto{\pgfqpoint{4.551107in}{1.335689in}}%
\pgfpathlineto{\pgfqpoint{4.550110in}{1.336553in}}%
\pgfpathlineto{\pgfqpoint{4.522037in}{1.361159in}}%
\pgfpathlineto{\pgfqpoint{4.520314in}{1.362722in}}%
\pgfpathlineto{\pgfqpoint{4.494295in}{1.386628in}}%
\pgfpathlineto{\pgfqpoint{4.490519in}{1.390223in}}%
\pgfpathlineto{\pgfqpoint{4.467830in}{1.412098in}}%
\pgfpathlineto{\pgfqpoint{4.460723in}{1.419200in}}%
\pgfpathlineto{\pgfqpoint{4.442587in}{1.437567in}}%
\pgfpathlineto{\pgfqpoint{4.430928in}{1.449812in}}%
\pgfpathlineto{\pgfqpoint{4.418509in}{1.463037in}}%
\pgfpathlineto{\pgfqpoint{4.401133in}{1.482235in}}%
\pgfpathlineto{\pgfqpoint{4.395537in}{1.488506in}}%
\pgfpathlineto{\pgfqpoint{4.373774in}{1.513976in}}%
\pgfpathlineto{\pgfqpoint{4.371337in}{1.516987in}}%
\pgfpathlineto{\pgfqpoint{4.353453in}{1.539445in}}%
\pgfpathlineto{\pgfqpoint{4.341542in}{1.555025in}}%
\pgfpathlineto{\pgfqpoint{4.334107in}{1.564915in}}%
\pgfpathlineto{\pgfqpoint{4.315952in}{1.590384in}}%
\pgfpathlineto{\pgfqpoint{4.311746in}{1.596697in}}%
\pgfpathlineto{\pgfqpoint{4.299220in}{1.615854in}}%
\pgfpathlineto{\pgfqpoint{4.283394in}{1.641323in}}%
\pgfpathlineto{\pgfqpoint{4.281951in}{1.643878in}}%
\pgfpathlineto{\pgfqpoint{4.269276in}{1.666793in}}%
\pgfpathlineto{\pgfqpoint{4.256121in}{1.692262in}}%
\pgfpathlineto{\pgfqpoint{4.252156in}{1.700878in}}%
\pgfpathlineto{\pgfqpoint{4.244573in}{1.717732in}}%
\pgfpathlineto{\pgfqpoint{4.234451in}{1.743202in}}%
\pgfpathlineto{\pgfqpoint{4.225633in}{1.768671in}}%
\pgfpathlineto{\pgfqpoint{4.222360in}{1.780443in}}%
\pgfpathlineto{\pgfqpoint{4.218655in}{1.794141in}}%
\pgfpathlineto{\pgfqpoint{4.213603in}{1.819610in}}%
\pgfpathlineto{\pgfqpoint{4.210483in}{1.845080in}}%
\pgfpathlineto{\pgfqpoint{4.209617in}{1.870549in}}%
\pgfpathlineto{\pgfqpoint{4.211411in}{1.896019in}}%
\pgfpathlineto{\pgfqpoint{4.216373in}{1.921488in}}%
\pgfpathlineto{\pgfqpoint{4.222360in}{1.939254in}}%
\pgfpathlineto{\pgfqpoint{4.225548in}{1.946958in}}%
\pgfpathlineto{\pgfqpoint{4.241102in}{1.972427in}}%
\pgfpathlineto{\pgfqpoint{4.252156in}{1.985423in}}%
\pgfpathlineto{\pgfqpoint{4.266241in}{1.997897in}}%
\pgfpathlineto{\pgfqpoint{4.281951in}{2.008694in}}%
\pgfpathlineto{\pgfqpoint{4.311746in}{2.022586in}}%
\pgfpathlineto{\pgfqpoint{4.314385in}{2.023366in}}%
\pgfpathlineto{\pgfqpoint{4.341542in}{2.030075in}}%
\pgfpathlineto{\pgfqpoint{4.371337in}{2.033550in}}%
\pgfpathlineto{\pgfqpoint{4.401133in}{2.033858in}}%
\pgfpathlineto{\pgfqpoint{4.430928in}{2.031583in}}%
\pgfpathlineto{\pgfqpoint{4.460723in}{2.027177in}}%
\pgfpathlineto{\pgfqpoint{4.478681in}{2.023366in}}%
\pgfpathlineto{\pgfqpoint{4.490519in}{2.020834in}}%
\pgfpathlineto{\pgfqpoint{4.520314in}{2.012687in}}%
\pgfpathlineto{\pgfqpoint{4.550110in}{2.003252in}}%
\pgfpathlineto{\pgfqpoint{4.564888in}{1.997897in}}%
\pgfpathlineto{\pgfqpoint{4.579905in}{1.992422in}}%
\pgfpathlineto{\pgfqpoint{4.609700in}{1.980352in}}%
\pgfpathlineto{\pgfqpoint{4.627745in}{1.972427in}}%
\pgfpathlineto{\pgfqpoint{4.639496in}{1.967238in}}%
\pgfpathlineto{\pgfqpoint{4.669291in}{1.953045in}}%
\pgfpathlineto{\pgfqpoint{4.681280in}{1.946958in}}%
\pgfpathlineto{\pgfqpoint{4.699086in}{1.937872in}}%
\pgfpathlineto{\pgfqpoint{4.728882in}{1.921982in}}%
\pgfpathlineto{\pgfqpoint{4.729749in}{1.921488in}}%
\pgfpathlineto{\pgfqpoint{4.758677in}{1.904958in}}%
\pgfpathlineto{\pgfqpoint{4.773782in}{1.896019in}}%
\pgfpathlineto{\pgfqpoint{4.788473in}{1.887281in}}%
\pgfpathlineto{\pgfqpoint{4.815695in}{1.870549in}}%
\pgfpathlineto{\pgfqpoint{4.818268in}{1.868959in}}%
\pgfpathlineto{\pgfqpoint{4.848063in}{1.849740in}}%
\pgfpathlineto{\pgfqpoint{4.855063in}{1.845080in}}%
\pgfpathlineto{\pgfqpoint{4.877859in}{1.829827in}}%
\pgfpathlineto{\pgfqpoint{4.892737in}{1.819610in}}%
\pgfpathlineto{\pgfqpoint{4.907654in}{1.809310in}}%
\pgfpathlineto{\pgfqpoint{4.929094in}{1.794141in}}%
\pgfpathlineto{\pgfqpoint{4.937450in}{1.788194in}}%
\pgfpathlineto{\pgfqpoint{4.964257in}{1.768671in}}%
\pgfpathlineto{\pgfqpoint{4.967245in}{1.766481in}}%
\pgfpathlineto{\pgfqpoint{4.997040in}{1.744120in}}%
\pgfpathlineto{\pgfqpoint{4.998235in}{1.743202in}}%
\pgfpathlineto{\pgfqpoint{5.026836in}{1.721071in}}%
\pgfpathlineto{\pgfqpoint{5.031072in}{1.717732in}}%
\pgfpathlineto{\pgfqpoint{5.056631in}{1.697451in}}%
\pgfpathlineto{\pgfqpoint{5.063057in}{1.692262in}}%
\pgfpathlineto{\pgfqpoint{5.086427in}{1.673258in}}%
\pgfpathlineto{\pgfqpoint{5.094247in}{1.666793in}}%
\pgfpathlineto{\pgfqpoint{5.116222in}{1.648486in}}%
\pgfpathlineto{\pgfqpoint{5.124688in}{1.641323in}}%
\pgfpathlineto{\pgfqpoint{5.146017in}{1.623128in}}%
\pgfpathlineto{\pgfqpoint{5.154422in}{1.615854in}}%
\pgfpathlineto{\pgfqpoint{5.175813in}{1.597173in}}%
\pgfpathlineto{\pgfqpoint{5.183482in}{1.590384in}}%
\pgfpathlineto{\pgfqpoint{5.205608in}{1.570608in}}%
\pgfpathlineto{\pgfqpoint{5.211899in}{1.564915in}}%
\pgfpathlineto{\pgfqpoint{5.235403in}{1.543417in}}%
\pgfpathlineto{\pgfqpoint{5.239695in}{1.539445in}}%
\pgfpathlineto{\pgfqpoint{5.265199in}{1.515578in}}%
\pgfpathlineto{\pgfqpoint{5.266893in}{1.513976in}}%
\pgfpathlineto{\pgfqpoint{5.293407in}{1.488506in}}%
\pgfpathlineto{\pgfqpoint{5.294994in}{1.486954in}}%
\pgfpathlineto{\pgfqpoint{5.319203in}{1.463037in}}%
\pgfpathlineto{\pgfqpoint{5.324790in}{1.457432in}}%
\pgfpathlineto{\pgfqpoint{5.344406in}{1.437567in}}%
\pgfpathlineto{\pgfqpoint{5.354585in}{1.427088in}}%
\pgfpathlineto{\pgfqpoint{5.369022in}{1.412098in}}%
\pgfpathlineto{\pgfqpoint{5.384380in}{1.395867in}}%
\pgfpathlineto{\pgfqpoint{5.393055in}{1.386628in}}%
\pgfpathlineto{\pgfqpoint{5.414176in}{1.363705in}}%
\pgfpathlineto{\pgfqpoint{5.416505in}{1.361159in}}%
\pgfpathlineto{\pgfqpoint{5.439061in}{1.335689in}}%
\pgfpathlineto{\pgfqpoint{5.443971in}{1.329973in}}%
\pgfpathlineto{\pgfqpoint{5.460837in}{1.310220in}}%
\pgfpathlineto{\pgfqpoint{5.473767in}{1.294675in}}%
\pgfpathlineto{\pgfqpoint{5.481980in}{1.284750in}}%
\pgfpathlineto{\pgfqpoint{5.502401in}{1.259281in}}%
\pgfpathlineto{\pgfqpoint{5.503562in}{1.257742in}}%
\pgfpathlineto{\pgfqpoint{5.521543in}{1.233811in}}%
\pgfpathlineto{\pgfqpoint{5.533357in}{1.217497in}}%
\pgfpathlineto{\pgfqpoint{5.539966in}{1.208341in}}%
\pgfpathlineto{\pgfqpoint{5.557234in}{1.182872in}}%
\pgfpathlineto{\pgfqpoint{5.563153in}{1.173434in}}%
\pgfpathlineto{\pgfqpoint{5.573186in}{1.157402in}}%
\pgfpathlineto{\pgfqpoint{5.587905in}{1.131933in}}%
\pgfpathlineto{\pgfqpoint{5.592948in}{1.122019in}}%
\pgfpathlineto{\pgfqpoint{5.600856in}{1.106463in}}%
\pgfpathlineto{\pgfqpoint{5.611926in}{1.080994in}}%
\pgfpathlineto{\pgfqpoint{5.620993in}{1.055524in}}%
\pgfpathlineto{\pgfqpoint{5.622744in}{1.047965in}}%
\pgfpathlineto{\pgfqpoint{5.626899in}{1.030055in}}%
\pgfpathlineto{\pgfqpoint{5.629119in}{1.004585in}}%
\pgfpathlineto{\pgfqpoint{5.626395in}{0.979116in}}%
\pgfpathlineto{\pgfqpoint{5.622744in}{0.968928in}}%
\pgfpathlineto{\pgfqpoint{5.615509in}{0.953646in}}%
\pgfpathlineto{\pgfqpoint{5.592948in}{0.930267in}}%
\pgfpathlineto{\pgfqpoint{5.589969in}{0.928177in}}%
\pgfpathlineto{\pgfqpoint{5.563153in}{0.915790in}}%
\pgfpathlineto{\pgfqpoint{5.533357in}{0.907925in}}%
\pgfpathlineto{\pgfqpoint{5.503562in}{0.904247in}}%
\pgfpathlineto{\pgfqpoint{5.473767in}{0.903563in}}%
\pgfpathlineto{\pgfqpoint{5.443971in}{0.905098in}}%
\pgfpathlineto{\pgfqpoint{5.414176in}{0.908324in}}%
\pgfpathlineto{\pgfqpoint{5.384380in}{0.912868in}}%
\pgfpathlineto{\pgfqpoint{5.354585in}{0.918463in}}%
\pgfpathlineto{\pgfqpoint{5.324790in}{0.924907in}}%
\pgfpathclose%
\pgfusepath{stroke}%
\end{pgfscope}%
\begin{pgfscope}%
\pgfpathrectangle{\pgfqpoint{0.766095in}{0.571603in}}{\pgfqpoint{5.929283in}{5.068436in}}%
\pgfusepath{clip}%
\pgfsetbuttcap%
\pgfsetroundjoin%
\pgfsetlinewidth{1.505625pt}%
\definecolor{currentstroke}{rgb}{0.279566,0.067836,0.391917}%
\pgfsetstrokecolor{currentstroke}%
\pgfsetdash{}{0pt}%
\pgfpathmoveto{\pgfqpoint{5.345103in}{0.571603in}}%
\pgfpathlineto{\pgfqpoint{5.235403in}{0.615549in}}%
\pgfpathlineto{\pgfqpoint{5.146017in}{0.653747in}}%
\pgfpathlineto{\pgfqpoint{5.046402in}{0.698951in}}%
\pgfpathlineto{\pgfqpoint{4.937450in}{0.751738in}}%
\pgfpathlineto{\pgfqpoint{4.842358in}{0.800829in}}%
\pgfpathlineto{\pgfqpoint{4.749288in}{0.851768in}}%
\pgfpathlineto{\pgfqpoint{4.661315in}{0.902707in}}%
\pgfpathlineto{\pgfqpoint{4.577997in}{0.953646in}}%
\pgfpathlineto{\pgfqpoint{4.490519in}{1.010404in}}%
\pgfpathlineto{\pgfqpoint{4.424424in}{1.055524in}}%
\pgfpathlineto{\pgfqpoint{4.341542in}{1.115421in}}%
\pgfpathlineto{\pgfqpoint{4.281951in}{1.160791in}}%
\pgfpathlineto{\pgfqpoint{4.222360in}{1.208445in}}%
\pgfpathlineto{\pgfqpoint{4.162137in}{1.259281in}}%
\pgfpathlineto{\pgfqpoint{4.103179in}{1.311947in}}%
\pgfpathlineto{\pgfqpoint{4.043588in}{1.368572in}}%
\pgfpathlineto{\pgfqpoint{3.983997in}{1.429043in}}%
\pgfpathlineto{\pgfqpoint{3.952211in}{1.463037in}}%
\pgfpathlineto{\pgfqpoint{3.907271in}{1.513976in}}%
\pgfpathlineto{\pgfqpoint{3.864816in}{1.565114in}}%
\pgfpathlineto{\pgfqpoint{3.825722in}{1.615854in}}%
\pgfpathlineto{\pgfqpoint{3.789163in}{1.666793in}}%
\pgfpathlineto{\pgfqpoint{3.755334in}{1.717732in}}%
\pgfpathlineto{\pgfqpoint{3.724261in}{1.768671in}}%
\pgfpathlineto{\pgfqpoint{3.695958in}{1.819610in}}%
\pgfpathlineto{\pgfqpoint{3.670422in}{1.870549in}}%
\pgfpathlineto{\pgfqpoint{3.647636in}{1.921488in}}%
\pgfpathlineto{\pgfqpoint{3.626452in}{1.975729in}}%
\pgfpathlineto{\pgfqpoint{3.610694in}{2.023366in}}%
\pgfpathlineto{\pgfqpoint{3.596484in}{2.074305in}}%
\pgfpathlineto{\pgfqpoint{3.585646in}{2.125244in}}%
\pgfpathlineto{\pgfqpoint{3.577903in}{2.176183in}}%
\pgfpathlineto{\pgfqpoint{3.573548in}{2.227123in}}%
\pgfpathlineto{\pgfqpoint{3.572904in}{2.278062in}}%
\pgfpathlineto{\pgfqpoint{3.576335in}{2.329001in}}%
\pgfpathlineto{\pgfqpoint{3.579705in}{2.354470in}}%
\pgfpathlineto{\pgfqpoint{3.584249in}{2.379940in}}%
\pgfpathlineto{\pgfqpoint{3.590027in}{2.405409in}}%
\pgfpathlineto{\pgfqpoint{3.597125in}{2.430879in}}%
\pgfpathlineto{\pgfqpoint{3.606061in}{2.456348in}}%
\pgfpathlineto{\pgfqpoint{3.616540in}{2.481818in}}%
\pgfpathlineto{\pgfqpoint{3.628793in}{2.507287in}}%
\pgfpathlineto{\pgfqpoint{3.643574in}{2.532757in}}%
\pgfpathlineto{\pgfqpoint{3.660665in}{2.558226in}}%
\pgfpathlineto{\pgfqpoint{3.686043in}{2.589370in}}%
\pgfpathlineto{\pgfqpoint{3.715839in}{2.618712in}}%
\pgfpathlineto{\pgfqpoint{3.745634in}{2.642390in}}%
\pgfpathlineto{\pgfqpoint{3.775429in}{2.661664in}}%
\pgfpathlineto{\pgfqpoint{3.805225in}{2.677136in}}%
\pgfpathlineto{\pgfqpoint{3.835020in}{2.689631in}}%
\pgfpathlineto{\pgfqpoint{3.864816in}{2.699365in}}%
\pgfpathlineto{\pgfqpoint{3.894611in}{2.706815in}}%
\pgfpathlineto{\pgfqpoint{3.924406in}{2.712133in}}%
\pgfpathlineto{\pgfqpoint{3.954202in}{2.715456in}}%
\pgfpathlineto{\pgfqpoint{3.983997in}{2.717033in}}%
\pgfpathlineto{\pgfqpoint{4.013792in}{2.716990in}}%
\pgfpathlineto{\pgfqpoint{4.043588in}{2.715438in}}%
\pgfpathlineto{\pgfqpoint{4.083272in}{2.711044in}}%
\pgfpathlineto{\pgfqpoint{4.103179in}{2.708155in}}%
\pgfpathlineto{\pgfqpoint{4.132974in}{2.702557in}}%
\pgfpathlineto{\pgfqpoint{4.192565in}{2.687963in}}%
\pgfpathlineto{\pgfqpoint{4.252156in}{2.669040in}}%
\pgfpathlineto{\pgfqpoint{4.311746in}{2.646290in}}%
\pgfpathlineto{\pgfqpoint{4.371337in}{2.620047in}}%
\pgfpathlineto{\pgfqpoint{4.430928in}{2.590632in}}%
\pgfpathlineto{\pgfqpoint{4.490734in}{2.558226in}}%
\pgfpathlineto{\pgfqpoint{4.550110in}{2.523354in}}%
\pgfpathlineto{\pgfqpoint{4.616098in}{2.481818in}}%
\pgfpathlineto{\pgfqpoint{4.691958in}{2.430879in}}%
\pgfpathlineto{\pgfqpoint{4.728882in}{2.405047in}}%
\pgfpathlineto{\pgfqpoint{4.818268in}{2.339484in}}%
\pgfpathlineto{\pgfqpoint{4.907654in}{2.270523in}}%
\pgfpathlineto{\pgfqpoint{4.997040in}{2.198619in}}%
\pgfpathlineto{\pgfqpoint{5.086427in}{2.124168in}}%
\pgfpathlineto{\pgfqpoint{5.205608in}{2.021519in}}%
\pgfpathlineto{\pgfqpoint{5.324790in}{1.915595in}}%
\pgfpathlineto{\pgfqpoint{5.457718in}{1.794141in}}%
\pgfpathlineto{\pgfqpoint{5.566775in}{1.692262in}}%
\pgfpathlineto{\pgfqpoint{5.699926in}{1.564915in}}%
\pgfpathlineto{\pgfqpoint{5.804135in}{1.463037in}}%
\pgfpathlineto{\pgfqpoint{5.920697in}{1.346047in}}%
\pgfpathlineto{\pgfqpoint{6.010084in}{1.253843in}}%
\pgfpathlineto{\pgfqpoint{6.100681in}{1.157402in}}%
\pgfpathlineto{\pgfqpoint{6.188856in}{1.059327in}}%
\pgfpathlineto{\pgfqpoint{6.236010in}{1.004585in}}%
\pgfpathlineto{\pgfqpoint{6.298843in}{0.928177in}}%
\pgfpathlineto{\pgfqpoint{6.338455in}{0.877238in}}%
\pgfpathlineto{\pgfqpoint{6.375579in}{0.826299in}}%
\pgfpathlineto{\pgfqpoint{6.410020in}{0.775360in}}%
\pgfpathlineto{\pgfqpoint{6.441142in}{0.724420in}}%
\pgfpathlineto{\pgfqpoint{6.457014in}{0.695571in}}%
\pgfpathlineto{\pgfqpoint{6.479748in}{0.648012in}}%
\pgfpathlineto{\pgfqpoint{6.489890in}{0.622542in}}%
\pgfpathlineto{\pgfqpoint{6.498130in}{0.597073in}}%
\pgfpathlineto{\pgfqpoint{6.504415in}{0.571603in}}%
\pgfpathlineto{\pgfqpoint{6.504415in}{0.571603in}}%
\pgfusepath{stroke}%
\end{pgfscope}%
\begin{pgfscope}%
\pgfpathrectangle{\pgfqpoint{0.766095in}{0.571603in}}{\pgfqpoint{5.929283in}{5.068436in}}%
\pgfusepath{clip}%
\pgfsetbuttcap%
\pgfsetroundjoin%
\pgfsetlinewidth{1.505625pt}%
\definecolor{currentstroke}{rgb}{0.282327,0.094955,0.417331}%
\pgfsetstrokecolor{currentstroke}%
\pgfsetdash{}{0pt}%
\pgfpathmoveto{\pgfqpoint{4.910519in}{0.571603in}}%
\pgfpathlineto{\pgfqpoint{4.808940in}{0.622542in}}%
\pgfpathlineto{\pgfqpoint{4.699086in}{0.680512in}}%
\pgfpathlineto{\pgfqpoint{4.609700in}{0.729984in}}%
\pgfpathlineto{\pgfqpoint{4.520314in}{0.781765in}}%
\pgfpathlineto{\pgfqpoint{4.430928in}{0.836037in}}%
\pgfpathlineto{\pgfqpoint{4.341542in}{0.892989in}}%
\pgfpathlineto{\pgfqpoint{4.281951in}{0.932544in}}%
\pgfpathlineto{\pgfqpoint{4.192565in}{0.994686in}}%
\pgfpathlineto{\pgfqpoint{4.132974in}{1.038031in}}%
\pgfpathlineto{\pgfqpoint{4.073383in}{1.083041in}}%
\pgfpathlineto{\pgfqpoint{4.011385in}{1.131933in}}%
\pgfpathlineto{\pgfqpoint{3.949674in}{1.182872in}}%
\pgfpathlineto{\pgfqpoint{3.890749in}{1.233811in}}%
\pgfpathlineto{\pgfqpoint{3.834525in}{1.284750in}}%
\pgfpathlineto{\pgfqpoint{3.775429in}{1.341225in}}%
\pgfpathlineto{\pgfqpoint{3.715839in}{1.401469in}}%
\pgfpathlineto{\pgfqpoint{3.681724in}{1.437567in}}%
\pgfpathlineto{\pgfqpoint{3.626452in}{1.499328in}}%
\pgfpathlineto{\pgfqpoint{3.592342in}{1.539445in}}%
\pgfpathlineto{\pgfqpoint{3.537066in}{1.608759in}}%
\pgfpathlineto{\pgfqpoint{3.507271in}{1.648546in}}%
\pgfpathlineto{\pgfqpoint{3.476085in}{1.692262in}}%
\pgfpathlineto{\pgfqpoint{3.442004in}{1.743202in}}%
\pgfpathlineto{\pgfqpoint{3.410158in}{1.794141in}}%
\pgfpathlineto{\pgfqpoint{3.380537in}{1.845080in}}%
\pgfpathlineto{\pgfqpoint{3.353125in}{1.896019in}}%
\pgfpathlineto{\pgfqpoint{3.327895in}{1.946958in}}%
\pgfpathlineto{\pgfqpoint{3.298703in}{2.012811in}}%
\pgfpathlineto{\pgfqpoint{3.284262in}{2.048836in}}%
\pgfpathlineto{\pgfqpoint{3.265650in}{2.099775in}}%
\pgfpathlineto{\pgfqpoint{3.249385in}{2.150714in}}%
\pgfpathlineto{\pgfqpoint{3.235259in}{2.201653in}}%
\pgfpathlineto{\pgfqpoint{3.223518in}{2.252592in}}%
\pgfpathlineto{\pgfqpoint{3.213970in}{2.303531in}}%
\pgfpathlineto{\pgfqpoint{3.206806in}{2.354470in}}%
\pgfpathlineto{\pgfqpoint{3.202139in}{2.405409in}}%
\pgfpathlineto{\pgfqpoint{3.199933in}{2.456348in}}%
\pgfpathlineto{\pgfqpoint{3.200299in}{2.507287in}}%
\pgfpathlineto{\pgfqpoint{3.203350in}{2.558226in}}%
\pgfpathlineto{\pgfqpoint{3.209317in}{2.609916in}}%
\pgfpathlineto{\pgfqpoint{3.218303in}{2.660104in}}%
\pgfpathlineto{\pgfqpoint{3.230577in}{2.711044in}}%
\pgfpathlineto{\pgfqpoint{3.246445in}{2.761983in}}%
\pgfpathlineto{\pgfqpoint{3.266271in}{2.812922in}}%
\pgfpathlineto{\pgfqpoint{3.277982in}{2.838391in}}%
\pgfpathlineto{\pgfqpoint{3.298703in}{2.878415in}}%
\pgfpathlineto{\pgfqpoint{3.320683in}{2.914800in}}%
\pgfpathlineto{\pgfqpoint{3.337999in}{2.940269in}}%
\pgfpathlineto{\pgfqpoint{3.358294in}{2.967406in}}%
\pgfpathlineto{\pgfqpoint{3.388089in}{3.002126in}}%
\pgfpathlineto{\pgfqpoint{3.417885in}{3.032308in}}%
\pgfpathlineto{\pgfqpoint{3.447680in}{3.058657in}}%
\pgfpathlineto{\pgfqpoint{3.477475in}{3.081721in}}%
\pgfpathlineto{\pgfqpoint{3.507271in}{3.101934in}}%
\pgfpathlineto{\pgfqpoint{3.537066in}{3.119637in}}%
\pgfpathlineto{\pgfqpoint{3.566862in}{3.134938in}}%
\pgfpathlineto{\pgfqpoint{3.596657in}{3.148201in}}%
\pgfpathlineto{\pgfqpoint{3.626452in}{3.159472in}}%
\pgfpathlineto{\pgfqpoint{3.658118in}{3.169495in}}%
\pgfpathlineto{\pgfqpoint{3.686043in}{3.176734in}}%
\pgfpathlineto{\pgfqpoint{3.715839in}{3.182894in}}%
\pgfpathlineto{\pgfqpoint{3.745634in}{3.187544in}}%
\pgfpathlineto{\pgfqpoint{3.775429in}{3.190743in}}%
\pgfpathlineto{\pgfqpoint{3.805225in}{3.192548in}}%
\pgfpathlineto{\pgfqpoint{3.835020in}{3.193011in}}%
\pgfpathlineto{\pgfqpoint{3.864816in}{3.192184in}}%
\pgfpathlineto{\pgfqpoint{3.894611in}{3.190114in}}%
\pgfpathlineto{\pgfqpoint{3.924406in}{3.186848in}}%
\pgfpathlineto{\pgfqpoint{3.983997in}{3.176902in}}%
\pgfpathlineto{\pgfqpoint{4.043588in}{3.162612in}}%
\pgfpathlineto{\pgfqpoint{4.103833in}{3.144025in}}%
\pgfpathlineto{\pgfqpoint{4.162769in}{3.122039in}}%
\pgfpathlineto{\pgfqpoint{4.228936in}{3.093086in}}%
\pgfpathlineto{\pgfqpoint{4.281951in}{3.067056in}}%
\pgfpathlineto{\pgfqpoint{4.341542in}{3.034714in}}%
\pgfpathlineto{\pgfqpoint{4.401133in}{2.999470in}}%
\pgfpathlineto{\pgfqpoint{4.460723in}{2.961538in}}%
\pgfpathlineto{\pgfqpoint{4.529210in}{2.914800in}}%
\pgfpathlineto{\pgfqpoint{4.599508in}{2.863861in}}%
\pgfpathlineto{\pgfqpoint{4.639496in}{2.833735in}}%
\pgfpathlineto{\pgfqpoint{4.730454in}{2.761983in}}%
\pgfpathlineto{\pgfqpoint{4.822621in}{2.685574in}}%
\pgfpathlineto{\pgfqpoint{4.911343in}{2.609165in}}%
\pgfpathlineto{\pgfqpoint{5.025771in}{2.507287in}}%
\pgfpathlineto{\pgfqpoint{5.086427in}{2.452037in}}%
\pgfpathlineto{\pgfqpoint{5.246061in}{2.303531in}}%
\pgfpathlineto{\pgfqpoint{5.384380in}{2.172495in}}%
\pgfpathlineto{\pgfqpoint{5.682334in}{1.885857in}}%
\pgfpathlineto{\pgfqpoint{6.129265in}{1.450611in}}%
\pgfpathlineto{\pgfqpoint{6.348371in}{1.233811in}}%
\pgfpathlineto{\pgfqpoint{6.499650in}{1.080994in}}%
\pgfpathlineto{\pgfqpoint{6.622409in}{0.953646in}}%
\pgfpathlineto{\pgfqpoint{6.695378in}{0.875832in}}%
\pgfpathlineto{\pgfqpoint{6.695378in}{0.875832in}}%
\pgfusepath{stroke}%
\end{pgfscope}%
\begin{pgfscope}%
\pgfpathrectangle{\pgfqpoint{0.766095in}{0.571603in}}{\pgfqpoint{5.929283in}{5.068436in}}%
\pgfusepath{clip}%
\pgfsetbuttcap%
\pgfsetroundjoin%
\pgfsetlinewidth{1.505625pt}%
\definecolor{currentstroke}{rgb}{0.283187,0.125848,0.444960}%
\pgfsetstrokecolor{currentstroke}%
\pgfsetdash{}{0pt}%
\pgfpathmoveto{\pgfqpoint{4.616814in}{0.571603in}}%
\pgfpathlineto{\pgfqpoint{4.520314in}{0.624320in}}%
\pgfpathlineto{\pgfqpoint{4.401133in}{0.692676in}}%
\pgfpathlineto{\pgfqpoint{4.311746in}{0.746390in}}%
\pgfpathlineto{\pgfqpoint{4.252156in}{0.783563in}}%
\pgfpathlineto{\pgfqpoint{4.162769in}{0.841376in}}%
\pgfpathlineto{\pgfqpoint{4.103179in}{0.881377in}}%
\pgfpathlineto{\pgfqpoint{4.013792in}{0.943973in}}%
\pgfpathlineto{\pgfqpoint{3.954202in}{0.987467in}}%
\pgfpathlineto{\pgfqpoint{3.894611in}{1.032482in}}%
\pgfpathlineto{\pgfqpoint{3.832837in}{1.080994in}}%
\pgfpathlineto{\pgfqpoint{3.745634in}{1.153026in}}%
\pgfpathlineto{\pgfqpoint{3.682070in}{1.208341in}}%
\pgfpathlineto{\pgfqpoint{3.626130in}{1.259281in}}%
\pgfpathlineto{\pgfqpoint{3.566862in}{1.315956in}}%
\pgfpathlineto{\pgfqpoint{3.507271in}{1.375933in}}%
\pgfpathlineto{\pgfqpoint{3.449235in}{1.437567in}}%
\pgfpathlineto{\pgfqpoint{3.403905in}{1.488506in}}%
\pgfpathlineto{\pgfqpoint{3.358294in}{1.542329in}}%
\pgfpathlineto{\pgfqpoint{3.299907in}{1.615854in}}%
\pgfpathlineto{\pgfqpoint{3.262108in}{1.666793in}}%
\pgfpathlineto{\pgfqpoint{3.226363in}{1.717732in}}%
\pgfpathlineto{\pgfqpoint{3.192645in}{1.768671in}}%
\pgfpathlineto{\pgfqpoint{3.160942in}{1.819610in}}%
\pgfpathlineto{\pgfqpoint{3.131241in}{1.870549in}}%
\pgfpathlineto{\pgfqpoint{3.103518in}{1.921488in}}%
\pgfpathlineto{\pgfqpoint{3.077746in}{1.972427in}}%
\pgfpathlineto{\pgfqpoint{3.053889in}{2.023366in}}%
\pgfpathlineto{\pgfqpoint{3.030545in}{2.077763in}}%
\pgfpathlineto{\pgfqpoint{3.012023in}{2.125244in}}%
\pgfpathlineto{\pgfqpoint{2.993926in}{2.176183in}}%
\pgfpathlineto{\pgfqpoint{2.977758in}{2.227123in}}%
\pgfpathlineto{\pgfqpoint{2.963502in}{2.278062in}}%
\pgfpathlineto{\pgfqpoint{2.951169in}{2.329001in}}%
\pgfpathlineto{\pgfqpoint{2.940663in}{2.379940in}}%
\pgfpathlineto{\pgfqpoint{2.932248in}{2.430879in}}%
\pgfpathlineto{\pgfqpoint{2.925727in}{2.481818in}}%
\pgfpathlineto{\pgfqpoint{2.921158in}{2.532757in}}%
\pgfpathlineto{\pgfqpoint{2.918595in}{2.583696in}}%
\pgfpathlineto{\pgfqpoint{2.918093in}{2.634635in}}%
\pgfpathlineto{\pgfqpoint{2.919705in}{2.685574in}}%
\pgfpathlineto{\pgfqpoint{2.923479in}{2.736513in}}%
\pgfpathlineto{\pgfqpoint{2.929464in}{2.787452in}}%
\pgfpathlineto{\pgfqpoint{2.937701in}{2.838391in}}%
\pgfpathlineto{\pgfqpoint{2.948459in}{2.889330in}}%
\pgfpathlineto{\pgfqpoint{2.961741in}{2.940269in}}%
\pgfpathlineto{\pgfqpoint{2.977693in}{2.991208in}}%
\pgfpathlineto{\pgfqpoint{3.000749in}{3.052530in}}%
\pgfpathlineto{\pgfqpoint{3.018611in}{3.093086in}}%
\pgfpathlineto{\pgfqpoint{3.044082in}{3.144025in}}%
\pgfpathlineto{\pgfqpoint{3.073405in}{3.194965in}}%
\pgfpathlineto{\pgfqpoint{3.090135in}{3.221350in}}%
\pgfpathlineto{\pgfqpoint{3.125805in}{3.271373in}}%
\pgfpathlineto{\pgfqpoint{3.149726in}{3.301516in}}%
\pgfpathlineto{\pgfqpoint{3.191113in}{3.347782in}}%
\pgfpathlineto{\pgfqpoint{3.216461in}{3.373251in}}%
\pgfpathlineto{\pgfqpoint{3.244084in}{3.398721in}}%
\pgfpathlineto{\pgfqpoint{3.274423in}{3.424190in}}%
\pgfpathlineto{\pgfqpoint{3.308045in}{3.449660in}}%
\pgfpathlineto{\pgfqpoint{3.345693in}{3.475129in}}%
\pgfpathlineto{\pgfqpoint{3.358294in}{3.483107in}}%
\pgfpathlineto{\pgfqpoint{3.388374in}{3.500599in}}%
\pgfpathlineto{\pgfqpoint{3.439163in}{3.526068in}}%
\pgfpathlineto{\pgfqpoint{3.447680in}{3.530018in}}%
\pgfpathlineto{\pgfqpoint{3.501914in}{3.551538in}}%
\pgfpathlineto{\pgfqpoint{3.507271in}{3.553491in}}%
\pgfpathlineto{\pgfqpoint{3.537066in}{3.563060in}}%
\pgfpathlineto{\pgfqpoint{3.591242in}{3.577007in}}%
\pgfpathlineto{\pgfqpoint{3.596657in}{3.578253in}}%
\pgfpathlineto{\pgfqpoint{3.626452in}{3.583883in}}%
\pgfpathlineto{\pgfqpoint{3.656248in}{3.588286in}}%
\pgfpathlineto{\pgfqpoint{3.686043in}{3.591483in}}%
\pgfpathlineto{\pgfqpoint{3.715839in}{3.593494in}}%
\pgfpathlineto{\pgfqpoint{3.745634in}{3.594342in}}%
\pgfpathlineto{\pgfqpoint{3.775429in}{3.594046in}}%
\pgfpathlineto{\pgfqpoint{3.835020in}{3.590103in}}%
\pgfpathlineto{\pgfqpoint{3.894611in}{3.581822in}}%
\pgfpathlineto{\pgfqpoint{3.954202in}{3.569304in}}%
\pgfpathlineto{\pgfqpoint{4.013792in}{3.552703in}}%
\pgfpathlineto{\pgfqpoint{4.073383in}{3.532104in}}%
\pgfpathlineto{\pgfqpoint{4.132974in}{3.507669in}}%
\pgfpathlineto{\pgfqpoint{4.192565in}{3.479535in}}%
\pgfpathlineto{\pgfqpoint{4.252156in}{3.447856in}}%
\pgfpathlineto{\pgfqpoint{4.311746in}{3.412794in}}%
\pgfpathlineto{\pgfqpoint{4.373257in}{3.373251in}}%
\pgfpathlineto{\pgfqpoint{4.430928in}{3.333338in}}%
\pgfpathlineto{\pgfqpoint{4.490519in}{3.289358in}}%
\pgfpathlineto{\pgfqpoint{4.550110in}{3.242844in}}%
\pgfpathlineto{\pgfqpoint{4.609700in}{3.194027in}}%
\pgfpathlineto{\pgfqpoint{4.699086in}{3.116998in}}%
\pgfpathlineto{\pgfqpoint{4.788473in}{3.036100in}}%
\pgfpathlineto{\pgfqpoint{4.890104in}{2.940269in}}%
\pgfpathlineto{\pgfqpoint{4.969024in}{2.863861in}}%
\pgfpathlineto{\pgfqpoint{5.097400in}{2.736513in}}%
\pgfpathlineto{\pgfqpoint{5.235403in}{2.597269in}}%
\pgfpathlineto{\pgfqpoint{5.652539in}{2.172650in}}%
\pgfpathlineto{\pgfqpoint{5.861107in}{1.962562in}}%
\pgfpathlineto{\pgfqpoint{6.039879in}{1.784597in}}%
\pgfpathlineto{\pgfqpoint{6.248447in}{1.579326in}}%
\pgfpathlineto{\pgfqpoint{6.516605in}{1.318066in}}%
\pgfpathlineto{\pgfqpoint{6.695378in}{1.144310in}}%
\pgfpathlineto{\pgfqpoint{6.695378in}{1.144310in}}%
\pgfusepath{stroke}%
\end{pgfscope}%
\begin{pgfscope}%
\pgfpathrectangle{\pgfqpoint{0.766095in}{0.571603in}}{\pgfqpoint{5.929283in}{5.068436in}}%
\pgfusepath{clip}%
\pgfsetbuttcap%
\pgfsetroundjoin%
\pgfsetlinewidth{1.505625pt}%
\definecolor{currentstroke}{rgb}{0.281887,0.150881,0.465405}%
\pgfsetstrokecolor{currentstroke}%
\pgfsetdash{}{0pt}%
\pgfpathmoveto{\pgfqpoint{4.380410in}{0.571603in}}%
\pgfpathlineto{\pgfqpoint{4.281951in}{0.628375in}}%
\pgfpathlineto{\pgfqpoint{4.192565in}{0.682010in}}%
\pgfpathlineto{\pgfqpoint{4.103179in}{0.737813in}}%
\pgfpathlineto{\pgfqpoint{4.013792in}{0.795937in}}%
\pgfpathlineto{\pgfqpoint{3.931469in}{0.851768in}}%
\pgfpathlineto{\pgfqpoint{3.835020in}{0.920340in}}%
\pgfpathlineto{\pgfqpoint{3.756096in}{0.979116in}}%
\pgfpathlineto{\pgfqpoint{3.686043in}{1.033606in}}%
\pgfpathlineto{\pgfqpoint{3.626452in}{1.081790in}}%
\pgfpathlineto{\pgfqpoint{3.566801in}{1.131933in}}%
\pgfpathlineto{\pgfqpoint{3.507271in}{1.184070in}}%
\pgfpathlineto{\pgfqpoint{3.447680in}{1.238573in}}%
\pgfpathlineto{\pgfqpoint{3.388089in}{1.295619in}}%
\pgfpathlineto{\pgfqpoint{3.322956in}{1.361159in}}%
\pgfpathlineto{\pgfqpoint{3.268908in}{1.418600in}}%
\pgfpathlineto{\pgfqpoint{3.206657in}{1.488506in}}%
\pgfpathlineto{\pgfqpoint{3.149726in}{1.556768in}}%
\pgfpathlineto{\pgfqpoint{3.103304in}{1.615854in}}%
\pgfpathlineto{\pgfqpoint{3.060340in}{1.673824in}}%
\pgfpathlineto{\pgfqpoint{3.012213in}{1.743202in}}%
\pgfpathlineto{\pgfqpoint{2.970954in}{1.807182in}}%
\pgfpathlineto{\pgfqpoint{2.932949in}{1.870549in}}%
\pgfpathlineto{\pgfqpoint{2.904464in}{1.921488in}}%
\pgfpathlineto{\pgfqpoint{2.877771in}{1.972427in}}%
\pgfpathlineto{\pgfqpoint{2.851772in}{2.025713in}}%
\pgfpathlineto{\pgfqpoint{2.818865in}{2.099775in}}%
\pgfpathlineto{\pgfqpoint{2.788781in}{2.176183in}}%
\pgfpathlineto{\pgfqpoint{2.762386in}{2.253126in}}%
\pgfpathlineto{\pgfqpoint{2.740274in}{2.329001in}}%
\pgfpathlineto{\pgfqpoint{2.727500in}{2.379940in}}%
\pgfpathlineto{\pgfqpoint{2.716483in}{2.430879in}}%
\pgfpathlineto{\pgfqpoint{2.707094in}{2.481818in}}%
\pgfpathlineto{\pgfqpoint{2.699450in}{2.532757in}}%
\pgfpathlineto{\pgfqpoint{2.693572in}{2.583696in}}%
\pgfpathlineto{\pgfqpoint{2.689394in}{2.634635in}}%
\pgfpathlineto{\pgfqpoint{2.686948in}{2.685574in}}%
\pgfpathlineto{\pgfqpoint{2.686262in}{2.736513in}}%
\pgfpathlineto{\pgfqpoint{2.687363in}{2.787452in}}%
\pgfpathlineto{\pgfqpoint{2.690276in}{2.838391in}}%
\pgfpathlineto{\pgfqpoint{2.695020in}{2.889330in}}%
\pgfpathlineto{\pgfqpoint{2.702795in}{2.947840in}}%
\pgfpathlineto{\pgfqpoint{2.710269in}{2.991208in}}%
\pgfpathlineto{\pgfqpoint{2.720887in}{3.042147in}}%
\pgfpathlineto{\pgfqpoint{2.733467in}{3.093086in}}%
\pgfpathlineto{\pgfqpoint{2.748382in}{3.144025in}}%
\pgfpathlineto{\pgfqpoint{2.765387in}{3.194965in}}%
\pgfpathlineto{\pgfqpoint{2.784876in}{3.245904in}}%
\pgfpathlineto{\pgfqpoint{2.806868in}{3.296843in}}%
\pgfpathlineto{\pgfqpoint{2.831502in}{3.347782in}}%
\pgfpathlineto{\pgfqpoint{2.859035in}{3.398721in}}%
\pgfpathlineto{\pgfqpoint{2.889758in}{3.449660in}}%
\pgfpathlineto{\pgfqpoint{2.923999in}{3.500599in}}%
\pgfpathlineto{\pgfqpoint{2.962131in}{3.551538in}}%
\pgfpathlineto{\pgfqpoint{2.982877in}{3.577007in}}%
\pgfpathlineto{\pgfqpoint{3.004743in}{3.602477in}}%
\pgfpathlineto{\pgfqpoint{3.030545in}{3.630690in}}%
\pgfpathlineto{\pgfqpoint{3.079136in}{3.678886in}}%
\pgfpathlineto{\pgfqpoint{3.107283in}{3.704355in}}%
\pgfpathlineto{\pgfqpoint{3.137473in}{3.729825in}}%
\pgfpathlineto{\pgfqpoint{3.179522in}{3.762451in}}%
\pgfpathlineto{\pgfqpoint{3.209317in}{3.783574in}}%
\pgfpathlineto{\pgfqpoint{3.244071in}{3.806233in}}%
\pgfpathlineto{\pgfqpoint{3.298703in}{3.838143in}}%
\pgfpathlineto{\pgfqpoint{3.335806in}{3.857172in}}%
\pgfpathlineto{\pgfqpoint{3.388089in}{3.880905in}}%
\pgfpathlineto{\pgfqpoint{3.417885in}{3.892704in}}%
\pgfpathlineto{\pgfqpoint{3.477475in}{3.912913in}}%
\pgfpathlineto{\pgfqpoint{3.537066in}{3.928663in}}%
\pgfpathlineto{\pgfqpoint{3.596657in}{3.940100in}}%
\pgfpathlineto{\pgfqpoint{3.656248in}{3.947324in}}%
\pgfpathlineto{\pgfqpoint{3.715839in}{3.950410in}}%
\pgfpathlineto{\pgfqpoint{3.775429in}{3.949389in}}%
\pgfpathlineto{\pgfqpoint{3.835020in}{3.944288in}}%
\pgfpathlineto{\pgfqpoint{3.894611in}{3.935136in}}%
\pgfpathlineto{\pgfqpoint{3.954202in}{3.921899in}}%
\pgfpathlineto{\pgfqpoint{4.013792in}{3.904637in}}%
\pgfpathlineto{\pgfqpoint{4.075106in}{3.882642in}}%
\pgfpathlineto{\pgfqpoint{4.134820in}{3.857172in}}%
\pgfpathlineto{\pgfqpoint{4.192565in}{3.828803in}}%
\pgfpathlineto{\pgfqpoint{4.252156in}{3.795689in}}%
\pgfpathlineto{\pgfqpoint{4.317045in}{3.755294in}}%
\pgfpathlineto{\pgfqpoint{4.371337in}{3.718317in}}%
\pgfpathlineto{\pgfqpoint{4.430928in}{3.674363in}}%
\pgfpathlineto{\pgfqpoint{4.490519in}{3.627154in}}%
\pgfpathlineto{\pgfqpoint{4.550110in}{3.576925in}}%
\pgfpathlineto{\pgfqpoint{4.609700in}{3.523934in}}%
\pgfpathlineto{\pgfqpoint{4.688843in}{3.449660in}}%
\pgfpathlineto{\pgfqpoint{4.740981in}{3.398721in}}%
\pgfpathlineto{\pgfqpoint{4.818268in}{3.320776in}}%
\pgfpathlineto{\pgfqpoint{4.914184in}{3.220434in}}%
\pgfpathlineto{\pgfqpoint{5.026836in}{3.099300in}}%
\pgfpathlineto{\pgfqpoint{5.658953in}{2.405409in}}%
\pgfpathlineto{\pgfqpoint{5.850290in}{2.201653in}}%
\pgfpathlineto{\pgfqpoint{6.046331in}{1.997897in}}%
\pgfpathlineto{\pgfqpoint{6.221902in}{1.819610in}}%
\pgfpathlineto{\pgfqpoint{6.401127in}{1.641323in}}%
\pgfpathlineto{\pgfqpoint{6.605991in}{1.441484in}}%
\pgfpathlineto{\pgfqpoint{6.695378in}{1.355361in}}%
\pgfpathlineto{\pgfqpoint{6.695378in}{1.355361in}}%
\pgfusepath{stroke}%
\end{pgfscope}%
\begin{pgfscope}%
\pgfpathrectangle{\pgfqpoint{0.766095in}{0.571603in}}{\pgfqpoint{5.929283in}{5.068436in}}%
\pgfusepath{clip}%
\pgfsetbuttcap%
\pgfsetroundjoin%
\pgfsetlinewidth{1.505625pt}%
\definecolor{currentstroke}{rgb}{0.278012,0.180367,0.486697}%
\pgfsetstrokecolor{currentstroke}%
\pgfsetdash{}{0pt}%
\pgfpathmoveto{\pgfqpoint{4.177568in}{0.571603in}}%
\pgfpathlineto{\pgfqpoint{4.073383in}{0.634071in}}%
\pgfpathlineto{\pgfqpoint{3.983997in}{0.689841in}}%
\pgfpathlineto{\pgfqpoint{3.891442in}{0.749890in}}%
\pgfpathlineto{\pgfqpoint{3.779286in}{0.826299in}}%
\pgfpathlineto{\pgfqpoint{3.686043in}{0.893183in}}%
\pgfpathlineto{\pgfqpoint{3.605365in}{0.953646in}}%
\pgfpathlineto{\pgfqpoint{3.537066in}{1.007021in}}%
\pgfpathlineto{\pgfqpoint{3.447680in}{1.080195in}}%
\pgfpathlineto{\pgfqpoint{3.387381in}{1.131933in}}%
\pgfpathlineto{\pgfqpoint{3.328499in}{1.184527in}}%
\pgfpathlineto{\pgfqpoint{3.248849in}{1.259281in}}%
\pgfpathlineto{\pgfqpoint{3.179522in}{1.328169in}}%
\pgfpathlineto{\pgfqpoint{3.123602in}{1.386628in}}%
\pgfpathlineto{\pgfqpoint{3.060340in}{1.456541in}}%
\pgfpathlineto{\pgfqpoint{3.011117in}{1.513976in}}%
\pgfpathlineto{\pgfqpoint{2.969482in}{1.564915in}}%
\pgfpathlineto{\pgfqpoint{2.910651in}{1.641323in}}%
\pgfpathlineto{\pgfqpoint{2.856023in}{1.717732in}}%
\pgfpathlineto{\pgfqpoint{2.821819in}{1.768671in}}%
\pgfpathlineto{\pgfqpoint{2.773957in}{1.845080in}}%
\pgfpathlineto{\pgfqpoint{2.732591in}{1.916660in}}%
\pgfpathlineto{\pgfqpoint{2.702676in}{1.972427in}}%
\pgfpathlineto{\pgfqpoint{2.665077in}{2.048836in}}%
\pgfpathlineto{\pgfqpoint{2.631160in}{2.125244in}}%
\pgfpathlineto{\pgfqpoint{2.610534in}{2.176183in}}%
\pgfpathlineto{\pgfqpoint{2.582667in}{2.252592in}}%
\pgfpathlineto{\pgfqpoint{2.558434in}{2.329001in}}%
\pgfpathlineto{\pgfqpoint{2.544262in}{2.379940in}}%
\pgfpathlineto{\pgfqpoint{2.525857in}{2.456348in}}%
\pgfpathlineto{\pgfqpoint{2.515618in}{2.507287in}}%
\pgfpathlineto{\pgfqpoint{2.506907in}{2.558226in}}%
\pgfpathlineto{\pgfqpoint{2.499712in}{2.609165in}}%
\pgfpathlineto{\pgfqpoint{2.494059in}{2.660104in}}%
\pgfpathlineto{\pgfqpoint{2.490052in}{2.711044in}}%
\pgfpathlineto{\pgfqpoint{2.487595in}{2.761983in}}%
\pgfpathlineto{\pgfqpoint{2.486704in}{2.812922in}}%
\pgfpathlineto{\pgfqpoint{2.487395in}{2.863861in}}%
\pgfpathlineto{\pgfqpoint{2.489680in}{2.914800in}}%
\pgfpathlineto{\pgfqpoint{2.494228in}{2.972359in}}%
\pgfpathlineto{\pgfqpoint{2.499182in}{3.016678in}}%
\pgfpathlineto{\pgfqpoint{2.506469in}{3.067617in}}%
\pgfpathlineto{\pgfqpoint{2.515414in}{3.118556in}}%
\pgfpathlineto{\pgfqpoint{2.526066in}{3.169495in}}%
\pgfpathlineto{\pgfqpoint{2.538630in}{3.220434in}}%
\pgfpathlineto{\pgfqpoint{2.553818in}{3.274453in}}%
\pgfpathlineto{\pgfqpoint{2.569210in}{3.322312in}}%
\pgfpathlineto{\pgfqpoint{2.587369in}{3.373251in}}%
\pgfpathlineto{\pgfqpoint{2.613409in}{3.437787in}}%
\pgfpathlineto{\pgfqpoint{2.630121in}{3.475129in}}%
\pgfpathlineto{\pgfqpoint{2.654810in}{3.526068in}}%
\pgfpathlineto{\pgfqpoint{2.681891in}{3.577007in}}%
\pgfpathlineto{\pgfqpoint{2.711548in}{3.627946in}}%
\pgfpathlineto{\pgfqpoint{2.743978in}{3.678886in}}%
\pgfpathlineto{\pgfqpoint{2.779397in}{3.729825in}}%
\pgfpathlineto{\pgfqpoint{2.818040in}{3.780764in}}%
\pgfpathlineto{\pgfqpoint{2.851772in}{3.821702in}}%
\pgfpathlineto{\pgfqpoint{2.883093in}{3.857172in}}%
\pgfpathlineto{\pgfqpoint{2.932032in}{3.908111in}}%
\pgfpathlineto{\pgfqpoint{2.970954in}{3.945385in}}%
\pgfpathlineto{\pgfqpoint{3.015445in}{3.984520in}}%
\pgfpathlineto{\pgfqpoint{3.060340in}{4.020937in}}%
\pgfpathlineto{\pgfqpoint{3.114768in}{4.060928in}}%
\pgfpathlineto{\pgfqpoint{3.152647in}{4.086398in}}%
\pgfpathlineto{\pgfqpoint{3.209317in}{4.121010in}}%
\pgfpathlineto{\pgfqpoint{3.268908in}{4.153179in}}%
\pgfpathlineto{\pgfqpoint{3.328499in}{4.181260in}}%
\pgfpathlineto{\pgfqpoint{3.388089in}{4.205412in}}%
\pgfpathlineto{\pgfqpoint{3.447680in}{4.225767in}}%
\pgfpathlineto{\pgfqpoint{3.507271in}{4.242432in}}%
\pgfpathlineto{\pgfqpoint{3.566862in}{4.255403in}}%
\pgfpathlineto{\pgfqpoint{3.626452in}{4.264778in}}%
\pgfpathlineto{\pgfqpoint{3.686043in}{4.270463in}}%
\pgfpathlineto{\pgfqpoint{3.745634in}{4.272498in}}%
\pgfpathlineto{\pgfqpoint{3.805225in}{4.270829in}}%
\pgfpathlineto{\pgfqpoint{3.864816in}{4.265402in}}%
\pgfpathlineto{\pgfqpoint{3.924406in}{4.256104in}}%
\pgfpathlineto{\pgfqpoint{3.983997in}{4.242914in}}%
\pgfpathlineto{\pgfqpoint{4.043588in}{4.225708in}}%
\pgfpathlineto{\pgfqpoint{4.103179in}{4.204436in}}%
\pgfpathlineto{\pgfqpoint{4.162769in}{4.179019in}}%
\pgfpathlineto{\pgfqpoint{4.222360in}{4.149405in}}%
\pgfpathlineto{\pgfqpoint{4.281951in}{4.115576in}}%
\pgfpathlineto{\pgfqpoint{4.341542in}{4.077529in}}%
\pgfpathlineto{\pgfqpoint{4.401133in}{4.035329in}}%
\pgfpathlineto{\pgfqpoint{4.466233in}{3.984520in}}%
\pgfpathlineto{\pgfqpoint{4.526305in}{3.933581in}}%
\pgfpathlineto{\pgfqpoint{4.582453in}{3.882642in}}%
\pgfpathlineto{\pgfqpoint{4.639496in}{3.827793in}}%
\pgfpathlineto{\pgfqpoint{4.710614in}{3.755294in}}%
\pgfpathlineto{\pgfqpoint{4.758677in}{3.704273in}}%
\pgfpathlineto{\pgfqpoint{4.850445in}{3.602477in}}%
\pgfpathlineto{\pgfqpoint{4.938681in}{3.500599in}}%
\pgfpathlineto{\pgfqpoint{5.066940in}{3.347782in}}%
\pgfpathlineto{\pgfqpoint{5.213989in}{3.169495in}}%
\pgfpathlineto{\pgfqpoint{5.488485in}{2.838391in}}%
\pgfpathlineto{\pgfqpoint{5.662094in}{2.634635in}}%
\pgfpathlineto{\pgfqpoint{5.818294in}{2.456348in}}%
\pgfpathlineto{\pgfqpoint{5.955854in}{2.303531in}}%
\pgfpathlineto{\pgfqpoint{6.099470in}{2.148076in}}%
\pgfpathlineto{\pgfqpoint{6.218651in}{2.022064in}}%
\pgfpathlineto{\pgfqpoint{6.340355in}{1.896019in}}%
\pgfpathlineto{\pgfqpoint{6.491223in}{1.743202in}}%
\pgfpathlineto{\pgfqpoint{6.645566in}{1.590384in}}%
\pgfpathlineto{\pgfqpoint{6.695378in}{1.541796in}}%
\pgfpathlineto{\pgfqpoint{6.695378in}{1.541796in}}%
\pgfusepath{stroke}%
\end{pgfscope}%
\begin{pgfscope}%
\pgfpathrectangle{\pgfqpoint{0.766095in}{0.571603in}}{\pgfqpoint{5.929283in}{5.068436in}}%
\pgfusepath{clip}%
\pgfsetbuttcap%
\pgfsetroundjoin%
\pgfsetlinewidth{1.505625pt}%
\definecolor{currentstroke}{rgb}{0.273006,0.204520,0.501721}%
\pgfsetstrokecolor{currentstroke}%
\pgfsetdash{}{0pt}%
\pgfpathmoveto{\pgfqpoint{3.997464in}{0.571603in}}%
\pgfpathlineto{\pgfqpoint{3.894611in}{0.635084in}}%
\pgfpathlineto{\pgfqpoint{3.795257in}{0.698951in}}%
\pgfpathlineto{\pgfqpoint{3.681919in}{0.775360in}}%
\pgfpathlineto{\pgfqpoint{3.574348in}{0.851768in}}%
\pgfpathlineto{\pgfqpoint{3.477475in}{0.924196in}}%
\pgfpathlineto{\pgfqpoint{3.407202in}{0.979116in}}%
\pgfpathlineto{\pgfqpoint{3.328499in}{1.043326in}}%
\pgfpathlineto{\pgfqpoint{3.254486in}{1.106463in}}%
\pgfpathlineto{\pgfqpoint{3.179522in}{1.173621in}}%
\pgfpathlineto{\pgfqpoint{3.115379in}{1.233811in}}%
\pgfpathlineto{\pgfqpoint{3.060340in}{1.287827in}}%
\pgfpathlineto{\pgfqpoint{2.989259in}{1.361159in}}%
\pgfpathlineto{\pgfqpoint{2.941158in}{1.413287in}}%
\pgfpathlineto{\pgfqpoint{2.875386in}{1.488506in}}%
\pgfpathlineto{\pgfqpoint{2.821977in}{1.553364in}}%
\pgfpathlineto{\pgfqpoint{2.773265in}{1.615854in}}%
\pgfpathlineto{\pgfqpoint{2.732591in}{1.670904in}}%
\pgfpathlineto{\pgfqpoint{2.682324in}{1.743202in}}%
\pgfpathlineto{\pgfqpoint{2.643205in}{1.803256in}}%
\pgfpathlineto{\pgfqpoint{2.602135in}{1.870549in}}%
\pgfpathlineto{\pgfqpoint{2.572972in}{1.921488in}}%
\pgfpathlineto{\pgfqpoint{2.545433in}{1.972427in}}%
\pgfpathlineto{\pgfqpoint{2.507154in}{2.048836in}}%
\pgfpathlineto{\pgfqpoint{2.483599in}{2.099775in}}%
\pgfpathlineto{\pgfqpoint{2.451194in}{2.176183in}}%
\pgfpathlineto{\pgfqpoint{2.431466in}{2.227123in}}%
\pgfpathlineto{\pgfqpoint{2.404750in}{2.303531in}}%
\pgfpathlineto{\pgfqpoint{2.381474in}{2.379940in}}%
\pgfpathlineto{\pgfqpoint{2.367782in}{2.430879in}}%
\pgfpathlineto{\pgfqpoint{2.349977in}{2.507287in}}%
\pgfpathlineto{\pgfqpoint{2.339946in}{2.558226in}}%
\pgfpathlineto{\pgfqpoint{2.327643in}{2.634635in}}%
\pgfpathlineto{\pgfqpoint{2.318526in}{2.711044in}}%
\pgfpathlineto{\pgfqpoint{2.314269in}{2.761983in}}%
\pgfpathlineto{\pgfqpoint{2.310651in}{2.838391in}}%
\pgfpathlineto{\pgfqpoint{2.310278in}{2.914800in}}%
\pgfpathlineto{\pgfqpoint{2.313177in}{2.991208in}}%
\pgfpathlineto{\pgfqpoint{2.316967in}{3.042147in}}%
\pgfpathlineto{\pgfqpoint{2.322300in}{3.093086in}}%
\pgfpathlineto{\pgfqpoint{2.329131in}{3.144025in}}%
\pgfpathlineto{\pgfqpoint{2.342175in}{3.220434in}}%
\pgfpathlineto{\pgfqpoint{2.352898in}{3.271373in}}%
\pgfpathlineto{\pgfqpoint{2.371920in}{3.347782in}}%
\pgfpathlineto{\pgfqpoint{2.386738in}{3.398721in}}%
\pgfpathlineto{\pgfqpoint{2.404841in}{3.454600in}}%
\pgfpathlineto{\pgfqpoint{2.431175in}{3.526068in}}%
\pgfpathlineto{\pgfqpoint{2.452193in}{3.577007in}}%
\pgfpathlineto{\pgfqpoint{2.475070in}{3.627946in}}%
\pgfpathlineto{\pgfqpoint{2.499921in}{3.678886in}}%
\pgfpathlineto{\pgfqpoint{2.526865in}{3.729825in}}%
\pgfpathlineto{\pgfqpoint{2.556032in}{3.780764in}}%
\pgfpathlineto{\pgfqpoint{2.587559in}{3.831703in}}%
\pgfpathlineto{\pgfqpoint{2.621592in}{3.882642in}}%
\pgfpathlineto{\pgfqpoint{2.658283in}{3.933581in}}%
\pgfpathlineto{\pgfqpoint{2.697794in}{3.984520in}}%
\pgfpathlineto{\pgfqpoint{2.732591in}{4.026331in}}%
\pgfpathlineto{\pgfqpoint{2.763108in}{4.060928in}}%
\pgfpathlineto{\pgfqpoint{2.811334in}{4.111867in}}%
\pgfpathlineto{\pgfqpoint{2.851772in}{4.151692in}}%
\pgfpathlineto{\pgfqpoint{2.891555in}{4.188276in}}%
\pgfpathlineto{\pgfqpoint{2.941158in}{4.230895in}}%
\pgfpathlineto{\pgfqpoint{2.983625in}{4.264685in}}%
\pgfpathlineto{\pgfqpoint{3.030545in}{4.299616in}}%
\pgfpathlineto{\pgfqpoint{3.091375in}{4.341093in}}%
\pgfpathlineto{\pgfqpoint{3.149726in}{4.377120in}}%
\pgfpathlineto{\pgfqpoint{3.209317in}{4.410395in}}%
\pgfpathlineto{\pgfqpoint{3.268908in}{4.440258in}}%
\pgfpathlineto{\pgfqpoint{3.332505in}{4.468441in}}%
\pgfpathlineto{\pgfqpoint{3.388089in}{4.490143in}}%
\pgfpathlineto{\pgfqpoint{3.447680in}{4.510318in}}%
\pgfpathlineto{\pgfqpoint{3.507271in}{4.527394in}}%
\pgfpathlineto{\pgfqpoint{3.566862in}{4.541372in}}%
\pgfpathlineto{\pgfqpoint{3.626452in}{4.552201in}}%
\pgfpathlineto{\pgfqpoint{3.686043in}{4.559879in}}%
\pgfpathlineto{\pgfqpoint{3.745634in}{4.564355in}}%
\pgfpathlineto{\pgfqpoint{3.805225in}{4.565537in}}%
\pgfpathlineto{\pgfqpoint{3.864816in}{4.563328in}}%
\pgfpathlineto{\pgfqpoint{3.924406in}{4.557624in}}%
\pgfpathlineto{\pgfqpoint{3.983997in}{4.548317in}}%
\pgfpathlineto{\pgfqpoint{4.043588in}{4.535221in}}%
\pgfpathlineto{\pgfqpoint{4.103179in}{4.518236in}}%
\pgfpathlineto{\pgfqpoint{4.162769in}{4.497150in}}%
\pgfpathlineto{\pgfqpoint{4.192565in}{4.485021in}}%
\pgfpathlineto{\pgfqpoint{4.252156in}{4.457521in}}%
\pgfpathlineto{\pgfqpoint{4.311746in}{4.425555in}}%
\pgfpathlineto{\pgfqpoint{4.366577in}{4.392032in}}%
\pgfpathlineto{\pgfqpoint{4.404526in}{4.366563in}}%
\pgfpathlineto{\pgfqpoint{4.460723in}{4.325347in}}%
\pgfpathlineto{\pgfqpoint{4.520314in}{4.277013in}}%
\pgfpathlineto{\pgfqpoint{4.579905in}{4.224021in}}%
\pgfpathlineto{\pgfqpoint{4.643139in}{4.162807in}}%
\pgfpathlineto{\pgfqpoint{4.699086in}{4.104740in}}%
\pgfpathlineto{\pgfqpoint{4.761770in}{4.035459in}}%
\pgfpathlineto{\pgfqpoint{4.827162in}{3.959050in}}%
\pgfpathlineto{\pgfqpoint{4.889702in}{3.882642in}}%
\pgfpathlineto{\pgfqpoint{4.950152in}{3.806233in}}%
\pgfpathlineto{\pgfqpoint{5.028565in}{3.704355in}}%
\pgfpathlineto{\pgfqpoint{5.181179in}{3.500599in}}%
\pgfpathlineto{\pgfqpoint{5.414176in}{3.188664in}}%
\pgfpathlineto{\pgfqpoint{5.506401in}{3.067617in}}%
\pgfpathlineto{\pgfqpoint{5.625349in}{2.914800in}}%
\pgfpathlineto{\pgfqpoint{5.726935in}{2.787452in}}%
\pgfpathlineto{\pgfqpoint{5.861107in}{2.624112in}}%
\pgfpathlineto{\pgfqpoint{5.950493in}{2.518194in}}%
\pgfpathlineto{\pgfqpoint{6.039879in}{2.414567in}}%
\pgfpathlineto{\pgfqpoint{6.129265in}{2.313141in}}%
\pgfpathlineto{\pgfqpoint{6.218651in}{2.213814in}}%
\pgfpathlineto{\pgfqpoint{6.323509in}{2.099775in}}%
\pgfpathlineto{\pgfqpoint{6.457014in}{1.958357in}}%
\pgfpathlineto{\pgfqpoint{6.566560in}{1.845080in}}%
\pgfpathlineto{\pgfqpoint{6.695378in}{1.714917in}}%
\pgfpathlineto{\pgfqpoint{6.695378in}{1.714917in}}%
\pgfusepath{stroke}%
\end{pgfscope}%
\begin{pgfscope}%
\pgfpathrectangle{\pgfqpoint{0.766095in}{0.571603in}}{\pgfqpoint{5.929283in}{5.068436in}}%
\pgfusepath{clip}%
\pgfsetbuttcap%
\pgfsetroundjoin%
\pgfsetlinewidth{1.505625pt}%
\definecolor{currentstroke}{rgb}{0.265145,0.232956,0.516599}%
\pgfsetstrokecolor{currentstroke}%
\pgfsetdash{}{0pt}%
\pgfpathmoveto{\pgfqpoint{3.834052in}{0.571603in}}%
\pgfpathlineto{\pgfqpoint{3.713550in}{0.648012in}}%
\pgfpathlineto{\pgfqpoint{3.596657in}{0.725929in}}%
\pgfpathlineto{\pgfqpoint{3.489803in}{0.800829in}}%
\pgfpathlineto{\pgfqpoint{3.386077in}{0.877238in}}%
\pgfpathlineto{\pgfqpoint{3.287653in}{0.953646in}}%
\pgfpathlineto{\pgfqpoint{3.194242in}{1.030055in}}%
\pgfpathlineto{\pgfqpoint{3.119931in}{1.093942in}}%
\pgfpathlineto{\pgfqpoint{3.049318in}{1.157402in}}%
\pgfpathlineto{\pgfqpoint{2.968545in}{1.233811in}}%
\pgfpathlineto{\pgfqpoint{2.892333in}{1.310220in}}%
\pgfpathlineto{\pgfqpoint{2.821977in}{1.384909in}}%
\pgfpathlineto{\pgfqpoint{2.774869in}{1.437567in}}%
\pgfpathlineto{\pgfqpoint{2.731093in}{1.488506in}}%
\pgfpathlineto{\pgfqpoint{2.668899in}{1.564915in}}%
\pgfpathlineto{\pgfqpoint{2.610641in}{1.641323in}}%
\pgfpathlineto{\pgfqpoint{2.553818in}{1.721298in}}%
\pgfpathlineto{\pgfqpoint{2.505659in}{1.794141in}}%
\pgfpathlineto{\pgfqpoint{2.464432in}{1.860909in}}%
\pgfpathlineto{\pgfqpoint{2.429373in}{1.921488in}}%
\pgfpathlineto{\pgfqpoint{2.388376in}{1.997897in}}%
\pgfpathlineto{\pgfqpoint{2.362952in}{2.048836in}}%
\pgfpathlineto{\pgfqpoint{2.327658in}{2.125244in}}%
\pgfpathlineto{\pgfqpoint{2.305986in}{2.176183in}}%
\pgfpathlineto{\pgfqpoint{2.276241in}{2.252592in}}%
\pgfpathlineto{\pgfqpoint{2.249738in}{2.329001in}}%
\pgfpathlineto{\pgfqpoint{2.226069in}{2.406652in}}%
\pgfpathlineto{\pgfqpoint{2.206353in}{2.481818in}}%
\pgfpathlineto{\pgfqpoint{2.189354in}{2.558226in}}%
\pgfpathlineto{\pgfqpoint{2.175476in}{2.634635in}}%
\pgfpathlineto{\pgfqpoint{2.164626in}{2.711044in}}%
\pgfpathlineto{\pgfqpoint{2.156927in}{2.787452in}}%
\pgfpathlineto{\pgfqpoint{2.152250in}{2.863861in}}%
\pgfpathlineto{\pgfqpoint{2.150622in}{2.940269in}}%
\pgfpathlineto{\pgfqpoint{2.152062in}{3.016678in}}%
\pgfpathlineto{\pgfqpoint{2.156581in}{3.093086in}}%
\pgfpathlineto{\pgfqpoint{2.164180in}{3.169495in}}%
\pgfpathlineto{\pgfqpoint{2.171044in}{3.220434in}}%
\pgfpathlineto{\pgfqpoint{2.184013in}{3.296843in}}%
\pgfpathlineto{\pgfqpoint{2.200162in}{3.373251in}}%
\pgfpathlineto{\pgfqpoint{2.219701in}{3.449660in}}%
\pgfpathlineto{\pgfqpoint{2.234626in}{3.500599in}}%
\pgfpathlineto{\pgfqpoint{2.259898in}{3.577007in}}%
\pgfpathlineto{\pgfqpoint{2.288830in}{3.653416in}}%
\pgfpathlineto{\pgfqpoint{2.321591in}{3.729825in}}%
\pgfpathlineto{\pgfqpoint{2.345568in}{3.780764in}}%
\pgfpathlineto{\pgfqpoint{2.375046in}{3.838470in}}%
\pgfpathlineto{\pgfqpoint{2.413945in}{3.908111in}}%
\pgfpathlineto{\pgfqpoint{2.444788in}{3.959050in}}%
\pgfpathlineto{\pgfqpoint{2.477800in}{4.009989in}}%
\pgfpathlineto{\pgfqpoint{2.513088in}{4.060928in}}%
\pgfpathlineto{\pgfqpoint{2.553818in}{4.115866in}}%
\pgfpathlineto{\pgfqpoint{2.591145in}{4.162807in}}%
\pgfpathlineto{\pgfqpoint{2.643205in}{4.223871in}}%
\pgfpathlineto{\pgfqpoint{2.680533in}{4.264685in}}%
\pgfpathlineto{\pgfqpoint{2.732591in}{4.318124in}}%
\pgfpathlineto{\pgfqpoint{2.792182in}{4.374559in}}%
\pgfpathlineto{\pgfqpoint{2.851772in}{4.426577in}}%
\pgfpathlineto{\pgfqpoint{2.911363in}{4.474586in}}%
\pgfpathlineto{\pgfqpoint{2.971601in}{4.519380in}}%
\pgfpathlineto{\pgfqpoint{3.030545in}{4.559761in}}%
\pgfpathlineto{\pgfqpoint{3.090135in}{4.597451in}}%
\pgfpathlineto{\pgfqpoint{3.149726in}{4.632024in}}%
\pgfpathlineto{\pgfqpoint{3.209317in}{4.663713in}}%
\pgfpathlineto{\pgfqpoint{3.268908in}{4.692590in}}%
\pgfpathlineto{\pgfqpoint{3.328499in}{4.718718in}}%
\pgfpathlineto{\pgfqpoint{3.388089in}{4.742150in}}%
\pgfpathlineto{\pgfqpoint{3.447680in}{4.762928in}}%
\pgfpathlineto{\pgfqpoint{3.507271in}{4.781083in}}%
\pgfpathlineto{\pgfqpoint{3.566862in}{4.796602in}}%
\pgfpathlineto{\pgfqpoint{3.626452in}{4.809420in}}%
\pgfpathlineto{\pgfqpoint{3.686043in}{4.819564in}}%
\pgfpathlineto{\pgfqpoint{3.745634in}{4.826949in}}%
\pgfpathlineto{\pgfqpoint{3.805225in}{4.831461in}}%
\pgfpathlineto{\pgfqpoint{3.864816in}{4.833053in}}%
\pgfpathlineto{\pgfqpoint{3.924406in}{4.831602in}}%
\pgfpathlineto{\pgfqpoint{3.983997in}{4.826977in}}%
\pgfpathlineto{\pgfqpoint{4.043588in}{4.818981in}}%
\pgfpathlineto{\pgfqpoint{4.103179in}{4.807474in}}%
\pgfpathlineto{\pgfqpoint{4.162769in}{4.792254in}}%
\pgfpathlineto{\pgfqpoint{4.222360in}{4.773112in}}%
\pgfpathlineto{\pgfqpoint{4.281951in}{4.749772in}}%
\pgfpathlineto{\pgfqpoint{4.311746in}{4.736439in}}%
\pgfpathlineto{\pgfqpoint{4.341542in}{4.722006in}}%
\pgfpathlineto{\pgfqpoint{4.401133in}{4.689541in}}%
\pgfpathlineto{\pgfqpoint{4.430928in}{4.671494in}}%
\pgfpathlineto{\pgfqpoint{4.490519in}{4.631536in}}%
\pgfpathlineto{\pgfqpoint{4.538080in}{4.595788in}}%
\pgfpathlineto{\pgfqpoint{4.579905in}{4.561695in}}%
\pgfpathlineto{\pgfqpoint{4.627515in}{4.519380in}}%
\pgfpathlineto{\pgfqpoint{4.669291in}{4.479550in}}%
\pgfpathlineto{\pgfqpoint{4.705223in}{4.442971in}}%
\pgfpathlineto{\pgfqpoint{4.758677in}{4.385125in}}%
\pgfpathlineto{\pgfqpoint{4.818294in}{4.315624in}}%
\pgfpathlineto{\pgfqpoint{4.879549in}{4.239215in}}%
\pgfpathlineto{\pgfqpoint{4.937537in}{4.162807in}}%
\pgfpathlineto{\pgfqpoint{5.011098in}{4.060928in}}%
\pgfpathlineto{\pgfqpoint{5.086427in}{3.952616in}}%
\pgfpathlineto{\pgfqpoint{5.236332in}{3.729825in}}%
\pgfpathlineto{\pgfqpoint{5.389757in}{3.500599in}}%
\pgfpathlineto{\pgfqpoint{5.503562in}{3.334037in}}%
\pgfpathlineto{\pgfqpoint{5.582902in}{3.220434in}}%
\pgfpathlineto{\pgfqpoint{5.692629in}{3.067617in}}%
\pgfpathlineto{\pgfqpoint{5.801516in}{2.921041in}}%
\pgfpathlineto{\pgfqpoint{5.864535in}{2.838391in}}%
\pgfpathlineto{\pgfqpoint{5.964175in}{2.711044in}}%
\pgfpathlineto{\pgfqpoint{6.069674in}{2.580536in}}%
\pgfpathlineto{\pgfqpoint{6.159061in}{2.473108in}}%
\pgfpathlineto{\pgfqpoint{6.248447in}{2.368430in}}%
\pgfpathlineto{\pgfqpoint{6.337833in}{2.266332in}}%
\pgfpathlineto{\pgfqpoint{6.427219in}{2.166643in}}%
\pgfpathlineto{\pgfqpoint{6.516605in}{2.069202in}}%
\pgfpathlineto{\pgfqpoint{6.607339in}{1.972427in}}%
\pgfpathlineto{\pgfqpoint{6.695378in}{1.880443in}}%
\pgfpathlineto{\pgfqpoint{6.695378in}{1.880443in}}%
\pgfusepath{stroke}%
\end{pgfscope}%
\begin{pgfscope}%
\pgfpathrectangle{\pgfqpoint{0.766095in}{0.571603in}}{\pgfqpoint{5.929283in}{5.068436in}}%
\pgfusepath{clip}%
\pgfsetbuttcap%
\pgfsetroundjoin%
\pgfsetlinewidth{1.505625pt}%
\definecolor{currentstroke}{rgb}{0.255645,0.260703,0.528312}%
\pgfsetstrokecolor{currentstroke}%
\pgfsetdash{}{0pt}%
\pgfpathmoveto{\pgfqpoint{3.683700in}{0.571603in}}%
\pgfpathlineto{\pgfqpoint{3.565565in}{0.648012in}}%
\pgfpathlineto{\pgfqpoint{3.453043in}{0.724420in}}%
\pgfpathlineto{\pgfqpoint{3.358294in}{0.791840in}}%
\pgfpathlineto{\pgfqpoint{3.277375in}{0.851768in}}%
\pgfpathlineto{\pgfqpoint{3.178775in}{0.928177in}}%
\pgfpathlineto{\pgfqpoint{3.085170in}{1.004585in}}%
\pgfpathlineto{\pgfqpoint{2.996312in}{1.080994in}}%
\pgfpathlineto{\pgfqpoint{2.911363in}{1.158055in}}%
\pgfpathlineto{\pgfqpoint{2.832360in}{1.233811in}}%
\pgfpathlineto{\pgfqpoint{2.762386in}{1.304586in}}%
\pgfpathlineto{\pgfqpoint{2.709042in}{1.361159in}}%
\pgfpathlineto{\pgfqpoint{2.643205in}{1.434589in}}%
\pgfpathlineto{\pgfqpoint{2.597265in}{1.488506in}}%
\pgfpathlineto{\pgfqpoint{2.553818in}{1.541694in}}%
\pgfpathlineto{\pgfqpoint{2.494228in}{1.618780in}}%
\pgfpathlineto{\pgfqpoint{2.441004in}{1.692262in}}%
\pgfpathlineto{\pgfqpoint{2.404841in}{1.745060in}}%
\pgfpathlineto{\pgfqpoint{2.356780in}{1.819610in}}%
\pgfpathlineto{\pgfqpoint{2.315455in}{1.888272in}}%
\pgfpathlineto{\pgfqpoint{2.282322in}{1.946958in}}%
\pgfpathlineto{\pgfqpoint{2.242257in}{2.023366in}}%
\pgfpathlineto{\pgfqpoint{2.205475in}{2.099775in}}%
\pgfpathlineto{\pgfqpoint{2.182783in}{2.150714in}}%
\pgfpathlineto{\pgfqpoint{2.151408in}{2.227123in}}%
\pgfpathlineto{\pgfqpoint{2.123193in}{2.303531in}}%
\pgfpathlineto{\pgfqpoint{2.098075in}{2.379940in}}%
\pgfpathlineto{\pgfqpoint{2.075985in}{2.456348in}}%
\pgfpathlineto{\pgfqpoint{2.057003in}{2.532757in}}%
\pgfpathlineto{\pgfqpoint{2.040945in}{2.609165in}}%
\pgfpathlineto{\pgfqpoint{2.027869in}{2.685574in}}%
\pgfpathlineto{\pgfqpoint{2.017501in}{2.763363in}}%
\pgfpathlineto{\pgfqpoint{2.010448in}{2.838391in}}%
\pgfpathlineto{\pgfqpoint{2.006116in}{2.914800in}}%
\pgfpathlineto{\pgfqpoint{2.004671in}{2.991208in}}%
\pgfpathlineto{\pgfqpoint{2.006123in}{3.067617in}}%
\pgfpathlineto{\pgfqpoint{2.010476in}{3.144025in}}%
\pgfpathlineto{\pgfqpoint{2.017729in}{3.220434in}}%
\pgfpathlineto{\pgfqpoint{2.028054in}{3.296843in}}%
\pgfpathlineto{\pgfqpoint{2.041301in}{3.373251in}}%
\pgfpathlineto{\pgfqpoint{2.051828in}{3.424190in}}%
\pgfpathlineto{\pgfqpoint{2.070229in}{3.500599in}}%
\pgfpathlineto{\pgfqpoint{2.084241in}{3.551538in}}%
\pgfpathlineto{\pgfqpoint{2.107904in}{3.627946in}}%
\pgfpathlineto{\pgfqpoint{2.136683in}{3.708972in}}%
\pgfpathlineto{\pgfqpoint{2.166478in}{3.783329in}}%
\pgfpathlineto{\pgfqpoint{2.199448in}{3.857172in}}%
\pgfpathlineto{\pgfqpoint{2.237273in}{3.933581in}}%
\pgfpathlineto{\pgfqpoint{2.264625in}{3.984520in}}%
\pgfpathlineto{\pgfqpoint{2.293802in}{4.035459in}}%
\pgfpathlineto{\pgfqpoint{2.324882in}{4.086398in}}%
\pgfpathlineto{\pgfqpoint{2.357948in}{4.137337in}}%
\pgfpathlineto{\pgfqpoint{2.393082in}{4.188276in}}%
\pgfpathlineto{\pgfqpoint{2.434637in}{4.244866in}}%
\pgfpathlineto{\pgfqpoint{2.470037in}{4.290154in}}%
\pgfpathlineto{\pgfqpoint{2.524023in}{4.354899in}}%
\pgfpathlineto{\pgfqpoint{2.556931in}{4.392032in}}%
\pgfpathlineto{\pgfqpoint{2.613409in}{4.452060in}}%
\pgfpathlineto{\pgfqpoint{2.673000in}{4.510874in}}%
\pgfpathlineto{\pgfqpoint{2.732591in}{4.565538in}}%
\pgfpathlineto{\pgfqpoint{2.792182in}{4.616424in}}%
\pgfpathlineto{\pgfqpoint{2.851772in}{4.663852in}}%
\pgfpathlineto{\pgfqpoint{2.911363in}{4.708102in}}%
\pgfpathlineto{\pgfqpoint{2.970954in}{4.749419in}}%
\pgfpathlineto{\pgfqpoint{3.030545in}{4.787857in}}%
\pgfpathlineto{\pgfqpoint{3.092416in}{4.825014in}}%
\pgfpathlineto{\pgfqpoint{3.149726in}{4.856959in}}%
\pgfpathlineto{\pgfqpoint{3.209317in}{4.887771in}}%
\pgfpathlineto{\pgfqpoint{3.268908in}{4.916218in}}%
\pgfpathlineto{\pgfqpoint{3.328499in}{4.942343in}}%
\pgfpathlineto{\pgfqpoint{3.388089in}{4.966182in}}%
\pgfpathlineto{\pgfqpoint{3.447680in}{4.987766in}}%
\pgfpathlineto{\pgfqpoint{3.507271in}{5.007120in}}%
\pgfpathlineto{\pgfqpoint{3.566862in}{5.024198in}}%
\pgfpathlineto{\pgfqpoint{3.626452in}{5.038993in}}%
\pgfpathlineto{\pgfqpoint{3.686043in}{5.051525in}}%
\pgfpathlineto{\pgfqpoint{3.745634in}{5.061662in}}%
\pgfpathlineto{\pgfqpoint{3.805225in}{5.069400in}}%
\pgfpathlineto{\pgfqpoint{3.864816in}{5.074664in}}%
\pgfpathlineto{\pgfqpoint{3.924406in}{5.077337in}}%
\pgfpathlineto{\pgfqpoint{3.983997in}{5.077293in}}%
\pgfpathlineto{\pgfqpoint{4.043588in}{5.074394in}}%
\pgfpathlineto{\pgfqpoint{4.103179in}{5.068492in}}%
\pgfpathlineto{\pgfqpoint{4.162769in}{5.059428in}}%
\pgfpathlineto{\pgfqpoint{4.222360in}{5.046944in}}%
\pgfpathlineto{\pgfqpoint{4.281951in}{5.030841in}}%
\pgfpathlineto{\pgfqpoint{4.311746in}{5.021311in}}%
\pgfpathlineto{\pgfqpoint{4.371337in}{4.999212in}}%
\pgfpathlineto{\pgfqpoint{4.430928in}{4.972712in}}%
\pgfpathlineto{\pgfqpoint{4.490519in}{4.941469in}}%
\pgfpathlineto{\pgfqpoint{4.520314in}{4.923975in}}%
\pgfpathlineto{\pgfqpoint{4.555665in}{4.901423in}}%
\pgfpathlineto{\pgfqpoint{4.592378in}{4.875953in}}%
\pgfpathlineto{\pgfqpoint{4.639496in}{4.840185in}}%
\pgfpathlineto{\pgfqpoint{4.687650in}{4.799545in}}%
\pgfpathlineto{\pgfqpoint{4.728882in}{4.761647in}}%
\pgfpathlineto{\pgfqpoint{4.767533in}{4.723136in}}%
\pgfpathlineto{\pgfqpoint{4.818268in}{4.668580in}}%
\pgfpathlineto{\pgfqpoint{4.858891in}{4.621258in}}%
\pgfpathlineto{\pgfqpoint{4.907654in}{4.560694in}}%
\pgfpathlineto{\pgfqpoint{4.957728in}{4.493910in}}%
\pgfpathlineto{\pgfqpoint{4.997040in}{4.438766in}}%
\pgfpathlineto{\pgfqpoint{5.062837in}{4.341093in}}%
\pgfpathlineto{\pgfqpoint{5.127967in}{4.239215in}}%
\pgfpathlineto{\pgfqpoint{5.205608in}{4.113033in}}%
\pgfpathlineto{\pgfqpoint{5.496922in}{3.627946in}}%
\pgfpathlineto{\pgfqpoint{5.622744in}{3.426687in}}%
\pgfpathlineto{\pgfqpoint{5.682334in}{3.334404in}}%
\pgfpathlineto{\pgfqpoint{5.741925in}{3.244169in}}%
\pgfpathlineto{\pgfqpoint{5.809712in}{3.144025in}}%
\pgfpathlineto{\pgfqpoint{5.898711in}{3.016678in}}%
\pgfpathlineto{\pgfqpoint{5.990949in}{2.889330in}}%
\pgfpathlineto{\pgfqpoint{6.086593in}{2.761983in}}%
\pgfpathlineto{\pgfqpoint{6.188856in}{2.630793in}}%
\pgfpathlineto{\pgfqpoint{6.278242in}{2.519895in}}%
\pgfpathlineto{\pgfqpoint{6.351941in}{2.430879in}}%
\pgfpathlineto{\pgfqpoint{6.460607in}{2.303531in}}%
\pgfpathlineto{\pgfqpoint{6.550180in}{2.201653in}}%
\pgfpathlineto{\pgfqpoint{6.665582in}{2.074205in}}%
\pgfpathlineto{\pgfqpoint{6.695378in}{2.041936in}}%
\pgfpathlineto{\pgfqpoint{6.695378in}{2.041936in}}%
\pgfusepath{stroke}%
\end{pgfscope}%
\begin{pgfscope}%
\pgfpathrectangle{\pgfqpoint{0.766095in}{0.571603in}}{\pgfqpoint{5.929283in}{5.068436in}}%
\pgfusepath{clip}%
\pgfsetbuttcap%
\pgfsetroundjoin%
\pgfsetlinewidth{1.505625pt}%
\definecolor{currentstroke}{rgb}{0.246811,0.283237,0.535941}%
\pgfsetstrokecolor{currentstroke}%
\pgfsetdash{}{0pt}%
\pgfpathmoveto{\pgfqpoint{3.543832in}{0.571603in}}%
\pgfpathlineto{\pgfqpoint{3.447680in}{0.634687in}}%
\pgfpathlineto{\pgfqpoint{3.353365in}{0.698951in}}%
\pgfpathlineto{\pgfqpoint{3.246190in}{0.775360in}}%
\pgfpathlineto{\pgfqpoint{3.149726in}{0.847482in}}%
\pgfpathlineto{\pgfqpoint{3.078827in}{0.902707in}}%
\pgfpathlineto{\pgfqpoint{2.984904in}{0.979116in}}%
\pgfpathlineto{\pgfqpoint{2.895665in}{1.055524in}}%
\pgfpathlineto{\pgfqpoint{2.810968in}{1.131933in}}%
\pgfpathlineto{\pgfqpoint{2.730644in}{1.208341in}}%
\pgfpathlineto{\pgfqpoint{2.654671in}{1.284750in}}%
\pgfpathlineto{\pgfqpoint{2.582767in}{1.361159in}}%
\pgfpathlineto{\pgfqpoint{2.514981in}{1.437567in}}%
\pgfpathlineto{\pgfqpoint{2.451085in}{1.513976in}}%
\pgfpathlineto{\pgfqpoint{2.390994in}{1.590384in}}%
\pgfpathlineto{\pgfqpoint{2.334610in}{1.666793in}}%
\pgfpathlineto{\pgfqpoint{2.281821in}{1.743202in}}%
\pgfpathlineto{\pgfqpoint{2.232594in}{1.819610in}}%
\pgfpathlineto{\pgfqpoint{2.196274in}{1.879760in}}%
\pgfpathlineto{\pgfqpoint{2.158137in}{1.946958in}}%
\pgfpathlineto{\pgfqpoint{2.117898in}{2.023366in}}%
\pgfpathlineto{\pgfqpoint{2.080859in}{2.099775in}}%
\pgfpathlineto{\pgfqpoint{2.046993in}{2.176183in}}%
\pgfpathlineto{\pgfqpoint{2.016261in}{2.252592in}}%
\pgfpathlineto{\pgfqpoint{1.987706in}{2.331555in}}%
\pgfpathlineto{\pgfqpoint{1.963906in}{2.405409in}}%
\pgfpathlineto{\pgfqpoint{1.942183in}{2.481818in}}%
\pgfpathlineto{\pgfqpoint{1.923318in}{2.558226in}}%
\pgfpathlineto{\pgfqpoint{1.907371in}{2.634635in}}%
\pgfpathlineto{\pgfqpoint{1.894221in}{2.711044in}}%
\pgfpathlineto{\pgfqpoint{1.883940in}{2.787452in}}%
\pgfpathlineto{\pgfqpoint{1.876407in}{2.863861in}}%
\pgfpathlineto{\pgfqpoint{1.871641in}{2.940269in}}%
\pgfpathlineto{\pgfqpoint{1.869657in}{3.016678in}}%
\pgfpathlineto{\pgfqpoint{1.870462in}{3.093086in}}%
\pgfpathlineto{\pgfqpoint{1.874059in}{3.169495in}}%
\pgfpathlineto{\pgfqpoint{1.880443in}{3.245904in}}%
\pgfpathlineto{\pgfqpoint{1.889601in}{3.322312in}}%
\pgfpathlineto{\pgfqpoint{1.901573in}{3.398721in}}%
\pgfpathlineto{\pgfqpoint{1.916484in}{3.475129in}}%
\pgfpathlineto{\pgfqpoint{1.934255in}{3.551538in}}%
\pgfpathlineto{\pgfqpoint{1.955001in}{3.627946in}}%
\pgfpathlineto{\pgfqpoint{1.970578in}{3.678886in}}%
\pgfpathlineto{\pgfqpoint{1.996456in}{3.755294in}}%
\pgfpathlineto{\pgfqpoint{2.025535in}{3.831703in}}%
\pgfpathlineto{\pgfqpoint{2.057928in}{3.908111in}}%
\pgfpathlineto{\pgfqpoint{2.093749in}{3.984520in}}%
\pgfpathlineto{\pgfqpoint{2.119605in}{4.035459in}}%
\pgfpathlineto{\pgfqpoint{2.147099in}{4.086398in}}%
\pgfpathlineto{\pgfqpoint{2.176296in}{4.137337in}}%
\pgfpathlineto{\pgfqpoint{2.207262in}{4.188276in}}%
\pgfpathlineto{\pgfqpoint{2.240065in}{4.239215in}}%
\pgfpathlineto{\pgfqpoint{2.285660in}{4.305612in}}%
\pgfpathlineto{\pgfqpoint{2.330637in}{4.366563in}}%
\pgfpathlineto{\pgfqpoint{2.375046in}{4.423125in}}%
\pgfpathlineto{\pgfqpoint{2.434682in}{4.493910in}}%
\pgfpathlineto{\pgfqpoint{2.494228in}{4.559386in}}%
\pgfpathlineto{\pgfqpoint{2.554575in}{4.621258in}}%
\pgfpathlineto{\pgfqpoint{2.613409in}{4.677562in}}%
\pgfpathlineto{\pgfqpoint{2.673000in}{4.731014in}}%
\pgfpathlineto{\pgfqpoint{2.732591in}{4.781180in}}%
\pgfpathlineto{\pgfqpoint{2.792182in}{4.828323in}}%
\pgfpathlineto{\pgfqpoint{2.856427in}{4.875953in}}%
\pgfpathlineto{\pgfqpoint{2.930212in}{4.926892in}}%
\pgfpathlineto{\pgfqpoint{2.970954in}{4.953479in}}%
\pgfpathlineto{\pgfqpoint{3.030545in}{4.990250in}}%
\pgfpathlineto{\pgfqpoint{3.097252in}{5.028770in}}%
\pgfpathlineto{\pgfqpoint{3.149726in}{5.057207in}}%
\pgfpathlineto{\pgfqpoint{3.209317in}{5.087466in}}%
\pgfpathlineto{\pgfqpoint{3.268908in}{5.115697in}}%
\pgfpathlineto{\pgfqpoint{3.328499in}{5.141936in}}%
\pgfpathlineto{\pgfqpoint{3.388089in}{5.166211in}}%
\pgfpathlineto{\pgfqpoint{3.447680in}{5.188549in}}%
\pgfpathlineto{\pgfqpoint{3.507271in}{5.208970in}}%
\pgfpathlineto{\pgfqpoint{3.584702in}{5.232527in}}%
\pgfpathlineto{\pgfqpoint{3.626452in}{5.243922in}}%
\pgfpathlineto{\pgfqpoint{3.686043in}{5.258517in}}%
\pgfpathlineto{\pgfqpoint{3.745634in}{5.271029in}}%
\pgfpathlineto{\pgfqpoint{3.817822in}{5.283466in}}%
\pgfpathlineto{\pgfqpoint{3.864816in}{5.289955in}}%
\pgfpathlineto{\pgfqpoint{3.924406in}{5.296174in}}%
\pgfpathlineto{\pgfqpoint{3.983997in}{5.300144in}}%
\pgfpathlineto{\pgfqpoint{4.043588in}{5.301745in}}%
\pgfpathlineto{\pgfqpoint{4.103179in}{5.300843in}}%
\pgfpathlineto{\pgfqpoint{4.162769in}{5.297298in}}%
\pgfpathlineto{\pgfqpoint{4.222360in}{5.290953in}}%
\pgfpathlineto{\pgfqpoint{4.281951in}{5.281612in}}%
\pgfpathlineto{\pgfqpoint{4.341542in}{5.268972in}}%
\pgfpathlineto{\pgfqpoint{4.401133in}{5.252863in}}%
\pgfpathlineto{\pgfqpoint{4.461818in}{5.232527in}}%
\pgfpathlineto{\pgfqpoint{4.520314in}{5.208808in}}%
\pgfpathlineto{\pgfqpoint{4.550110in}{5.195026in}}%
\pgfpathlineto{\pgfqpoint{4.579905in}{5.180101in}}%
\pgfpathlineto{\pgfqpoint{4.623032in}{5.156118in}}%
\pgfpathlineto{\pgfqpoint{4.669291in}{5.127422in}}%
\pgfpathlineto{\pgfqpoint{4.701683in}{5.105179in}}%
\pgfpathlineto{\pgfqpoint{4.735879in}{5.079709in}}%
\pgfpathlineto{\pgfqpoint{4.767565in}{5.054240in}}%
\pgfpathlineto{\pgfqpoint{4.797127in}{5.028770in}}%
\pgfpathlineto{\pgfqpoint{4.824875in}{5.003301in}}%
\pgfpathlineto{\pgfqpoint{4.851068in}{4.977831in}}%
\pgfpathlineto{\pgfqpoint{4.899381in}{4.926892in}}%
\pgfpathlineto{\pgfqpoint{4.937450in}{4.883241in}}%
\pgfpathlineto{\pgfqpoint{4.967245in}{4.846725in}}%
\pgfpathlineto{\pgfqpoint{5.003437in}{4.799545in}}%
\pgfpathlineto{\pgfqpoint{5.040187in}{4.748606in}}%
\pgfpathlineto{\pgfqpoint{5.086427in}{4.680396in}}%
\pgfpathlineto{\pgfqpoint{5.139825in}{4.595788in}}%
\pgfpathlineto{\pgfqpoint{5.175813in}{4.535845in}}%
\pgfpathlineto{\pgfqpoint{5.243343in}{4.417502in}}%
\pgfpathlineto{\pgfqpoint{5.312874in}{4.290154in}}%
\pgfpathlineto{\pgfqpoint{5.394280in}{4.137337in}}%
\pgfpathlineto{\pgfqpoint{5.543959in}{3.857172in}}%
\pgfpathlineto{\pgfqpoint{5.628326in}{3.704355in}}%
\pgfpathlineto{\pgfqpoint{5.700962in}{3.577007in}}%
\pgfpathlineto{\pgfqpoint{5.791591in}{3.424190in}}%
\pgfpathlineto{\pgfqpoint{5.870357in}{3.296843in}}%
\pgfpathlineto{\pgfqpoint{5.952372in}{3.169495in}}%
\pgfpathlineto{\pgfqpoint{6.020382in}{3.067617in}}%
\pgfpathlineto{\pgfqpoint{6.108645in}{2.940269in}}%
\pgfpathlineto{\pgfqpoint{6.200585in}{2.812922in}}%
\pgfpathlineto{\pgfqpoint{6.296315in}{2.685574in}}%
\pgfpathlineto{\pgfqpoint{6.397424in}{2.556375in}}%
\pgfpathlineto{\pgfqpoint{6.486810in}{2.446172in}}%
\pgfpathlineto{\pgfqpoint{6.563381in}{2.354470in}}%
\pgfpathlineto{\pgfqpoint{6.673183in}{2.227123in}}%
\pgfpathlineto{\pgfqpoint{6.695378in}{2.201957in}}%
\pgfpathlineto{\pgfqpoint{6.695378in}{2.201957in}}%
\pgfusepath{stroke}%
\end{pgfscope}%
\begin{pgfscope}%
\pgfpathrectangle{\pgfqpoint{0.766095in}{0.571603in}}{\pgfqpoint{5.929283in}{5.068436in}}%
\pgfusepath{clip}%
\pgfsetbuttcap%
\pgfsetroundjoin%
\pgfsetlinewidth{1.505625pt}%
\definecolor{currentstroke}{rgb}{0.235526,0.309527,0.542944}%
\pgfsetstrokecolor{currentstroke}%
\pgfsetdash{}{0pt}%
\pgfpathmoveto{\pgfqpoint{3.412610in}{0.571603in}}%
\pgfpathlineto{\pgfqpoint{3.298301in}{0.648012in}}%
\pgfpathlineto{\pgfqpoint{3.189303in}{0.724420in}}%
\pgfpathlineto{\pgfqpoint{3.085321in}{0.800829in}}%
\pgfpathlineto{\pgfqpoint{2.986295in}{0.877238in}}%
\pgfpathlineto{\pgfqpoint{2.892004in}{0.953646in}}%
\pgfpathlineto{\pgfqpoint{2.802335in}{1.030055in}}%
\pgfpathlineto{\pgfqpoint{2.717149in}{1.106463in}}%
\pgfpathlineto{\pgfqpoint{2.636287in}{1.182872in}}%
\pgfpathlineto{\pgfqpoint{2.559652in}{1.259281in}}%
\pgfpathlineto{\pgfqpoint{2.494228in}{1.328040in}}%
\pgfpathlineto{\pgfqpoint{2.440974in}{1.386628in}}%
\pgfpathlineto{\pgfqpoint{2.374969in}{1.463037in}}%
\pgfpathlineto{\pgfqpoint{2.312808in}{1.539445in}}%
\pgfpathlineto{\pgfqpoint{2.254316in}{1.615854in}}%
\pgfpathlineto{\pgfqpoint{2.196274in}{1.696843in}}%
\pgfpathlineto{\pgfqpoint{2.148074in}{1.768671in}}%
\pgfpathlineto{\pgfqpoint{2.100085in}{1.845080in}}%
\pgfpathlineto{\pgfqpoint{2.055445in}{1.921488in}}%
\pgfpathlineto{\pgfqpoint{2.014031in}{1.997897in}}%
\pgfpathlineto{\pgfqpoint{1.975847in}{2.074305in}}%
\pgfpathlineto{\pgfqpoint{1.940746in}{2.150714in}}%
\pgfpathlineto{\pgfqpoint{1.908686in}{2.227123in}}%
\pgfpathlineto{\pgfqpoint{1.879620in}{2.303531in}}%
\pgfpathlineto{\pgfqpoint{1.853489in}{2.379940in}}%
\pgfpathlineto{\pgfqpoint{1.830228in}{2.456348in}}%
\pgfpathlineto{\pgfqpoint{1.808934in}{2.536208in}}%
\pgfpathlineto{\pgfqpoint{1.792211in}{2.609165in}}%
\pgfpathlineto{\pgfqpoint{1.777317in}{2.685574in}}%
\pgfpathlineto{\pgfqpoint{1.765245in}{2.761983in}}%
\pgfpathlineto{\pgfqpoint{1.755817in}{2.838391in}}%
\pgfpathlineto{\pgfqpoint{1.749057in}{2.914800in}}%
\pgfpathlineto{\pgfqpoint{1.745037in}{2.991208in}}%
\pgfpathlineto{\pgfqpoint{1.743661in}{3.067617in}}%
\pgfpathlineto{\pgfqpoint{1.744932in}{3.144025in}}%
\pgfpathlineto{\pgfqpoint{1.749343in}{3.227105in}}%
\pgfpathlineto{\pgfqpoint{1.755503in}{3.296843in}}%
\pgfpathlineto{\pgfqpoint{1.764833in}{3.373251in}}%
\pgfpathlineto{\pgfqpoint{1.776810in}{3.449660in}}%
\pgfpathlineto{\pgfqpoint{1.791622in}{3.526068in}}%
\pgfpathlineto{\pgfqpoint{1.809114in}{3.602477in}}%
\pgfpathlineto{\pgfqpoint{1.829558in}{3.678886in}}%
\pgfpathlineto{\pgfqpoint{1.852851in}{3.755294in}}%
\pgfpathlineto{\pgfqpoint{1.879093in}{3.831703in}}%
\pgfpathlineto{\pgfqpoint{1.908387in}{3.908111in}}%
\pgfpathlineto{\pgfqpoint{1.940832in}{3.984520in}}%
\pgfpathlineto{\pgfqpoint{1.976525in}{4.060928in}}%
\pgfpathlineto{\pgfqpoint{2.002200in}{4.111867in}}%
\pgfpathlineto{\pgfqpoint{2.029416in}{4.162807in}}%
\pgfpathlineto{\pgfqpoint{2.058230in}{4.213746in}}%
\pgfpathlineto{\pgfqpoint{2.088698in}{4.264685in}}%
\pgfpathlineto{\pgfqpoint{2.120877in}{4.315624in}}%
\pgfpathlineto{\pgfqpoint{2.166478in}{4.383494in}}%
\pgfpathlineto{\pgfqpoint{2.209247in}{4.442971in}}%
\pgfpathlineto{\pgfqpoint{2.255864in}{4.504015in}}%
\pgfpathlineto{\pgfqpoint{2.309957in}{4.570319in}}%
\pgfpathlineto{\pgfqpoint{2.354093in}{4.621258in}}%
\pgfpathlineto{\pgfqpoint{2.404841in}{4.676776in}}%
\pgfpathlineto{\pgfqpoint{2.464432in}{4.738053in}}%
\pgfpathlineto{\pgfqpoint{2.528218in}{4.799545in}}%
\pgfpathlineto{\pgfqpoint{2.584340in}{4.850484in}}%
\pgfpathlineto{\pgfqpoint{2.643778in}{4.901423in}}%
\pgfpathlineto{\pgfqpoint{2.706852in}{4.952362in}}%
\pgfpathlineto{\pgfqpoint{2.773919in}{5.003301in}}%
\pgfpathlineto{\pgfqpoint{2.845369in}{5.054240in}}%
\pgfpathlineto{\pgfqpoint{2.882909in}{5.079709in}}%
\pgfpathlineto{\pgfqpoint{2.962352in}{5.130649in}}%
\pgfpathlineto{\pgfqpoint{3.004310in}{5.156118in}}%
\pgfpathlineto{\pgfqpoint{3.060340in}{5.188547in}}%
\pgfpathlineto{\pgfqpoint{3.141515in}{5.232527in}}%
\pgfpathlineto{\pgfqpoint{3.209317in}{5.266668in}}%
\pgfpathlineto{\pgfqpoint{3.268908in}{5.294753in}}%
\pgfpathlineto{\pgfqpoint{3.328499in}{5.321100in}}%
\pgfpathlineto{\pgfqpoint{3.388089in}{5.345734in}}%
\pgfpathlineto{\pgfqpoint{3.447680in}{5.368678in}}%
\pgfpathlineto{\pgfqpoint{3.507271in}{5.389954in}}%
\pgfpathlineto{\pgfqpoint{3.570935in}{5.410813in}}%
\pgfpathlineto{\pgfqpoint{3.656248in}{5.435786in}}%
\pgfpathlineto{\pgfqpoint{3.715839in}{5.451100in}}%
\pgfpathlineto{\pgfqpoint{3.775429in}{5.464758in}}%
\pgfpathlineto{\pgfqpoint{3.835020in}{5.476599in}}%
\pgfpathlineto{\pgfqpoint{3.898179in}{5.487222in}}%
\pgfpathlineto{\pgfqpoint{3.954202in}{5.494860in}}%
\pgfpathlineto{\pgfqpoint{4.013792in}{5.501115in}}%
\pgfpathlineto{\pgfqpoint{4.073383in}{5.505371in}}%
\pgfpathlineto{\pgfqpoint{4.132974in}{5.507513in}}%
\pgfpathlineto{\pgfqpoint{4.192565in}{5.507416in}}%
\pgfpathlineto{\pgfqpoint{4.252156in}{5.504944in}}%
\pgfpathlineto{\pgfqpoint{4.311746in}{5.499949in}}%
\pgfpathlineto{\pgfqpoint{4.371337in}{5.492268in}}%
\pgfpathlineto{\pgfqpoint{4.430928in}{5.481635in}}%
\pgfpathlineto{\pgfqpoint{4.490519in}{5.467815in}}%
\pgfpathlineto{\pgfqpoint{4.550110in}{5.450495in}}%
\pgfpathlineto{\pgfqpoint{4.609700in}{5.429356in}}%
\pgfpathlineto{\pgfqpoint{4.669291in}{5.403952in}}%
\pgfpathlineto{\pgfqpoint{4.707157in}{5.385344in}}%
\pgfpathlineto{\pgfqpoint{4.758677in}{5.356802in}}%
\pgfpathlineto{\pgfqpoint{4.794534in}{5.334405in}}%
\pgfpathlineto{\pgfqpoint{4.831646in}{5.308935in}}%
\pgfpathlineto{\pgfqpoint{4.865604in}{5.283466in}}%
\pgfpathlineto{\pgfqpoint{4.907654in}{5.248859in}}%
\pgfpathlineto{\pgfqpoint{4.953140in}{5.207057in}}%
\pgfpathlineto{\pgfqpoint{4.978662in}{5.181588in}}%
\pgfpathlineto{\pgfqpoint{5.002773in}{5.156118in}}%
\pgfpathlineto{\pgfqpoint{5.047250in}{5.105179in}}%
\pgfpathlineto{\pgfqpoint{5.067935in}{5.079709in}}%
\pgfpathlineto{\pgfqpoint{5.106741in}{5.028770in}}%
\pgfpathlineto{\pgfqpoint{5.142681in}{4.977831in}}%
\pgfpathlineto{\pgfqpoint{5.176236in}{4.926892in}}%
\pgfpathlineto{\pgfqpoint{5.222902in}{4.850484in}}%
\pgfpathlineto{\pgfqpoint{5.266384in}{4.774075in}}%
\pgfpathlineto{\pgfqpoint{5.320488in}{4.672197in}}%
\pgfpathlineto{\pgfqpoint{5.359058in}{4.595788in}}%
\pgfpathlineto{\pgfqpoint{5.420861in}{4.468441in}}%
\pgfpathlineto{\pgfqpoint{5.540746in}{4.213746in}}%
\pgfpathlineto{\pgfqpoint{5.637776in}{4.009989in}}%
\pgfpathlineto{\pgfqpoint{5.713261in}{3.857172in}}%
\pgfpathlineto{\pgfqpoint{5.778575in}{3.729825in}}%
\pgfpathlineto{\pgfqpoint{5.846592in}{3.602477in}}%
\pgfpathlineto{\pgfqpoint{5.920697in}{3.469879in}}%
\pgfpathlineto{\pgfqpoint{5.980288in}{3.367515in}}%
\pgfpathlineto{\pgfqpoint{6.039879in}{3.268787in}}%
\pgfpathlineto{\pgfqpoint{6.101990in}{3.169495in}}%
\pgfpathlineto{\pgfqpoint{6.168080in}{3.067617in}}%
\pgfpathlineto{\pgfqpoint{6.248447in}{2.948662in}}%
\pgfpathlineto{\pgfqpoint{6.308038in}{2.863506in}}%
\pgfpathlineto{\pgfqpoint{6.381396in}{2.761983in}}%
\pgfpathlineto{\pgfqpoint{6.457683in}{2.660104in}}%
\pgfpathlineto{\pgfqpoint{6.536522in}{2.558226in}}%
\pgfpathlineto{\pgfqpoint{6.638846in}{2.430879in}}%
\pgfpathlineto{\pgfqpoint{6.695378in}{2.362654in}}%
\pgfpathlineto{\pgfqpoint{6.695378in}{2.362654in}}%
\pgfusepath{stroke}%
\end{pgfscope}%
\begin{pgfscope}%
\pgfpathrectangle{\pgfqpoint{0.766095in}{0.571603in}}{\pgfqpoint{5.929283in}{5.068436in}}%
\pgfusepath{clip}%
\pgfsetbuttcap%
\pgfsetroundjoin%
\pgfsetlinewidth{1.505625pt}%
\definecolor{currentstroke}{rgb}{0.225863,0.330805,0.547314}%
\pgfsetstrokecolor{currentstroke}%
\pgfsetdash{}{0pt}%
\pgfpathmoveto{\pgfqpoint{3.288750in}{0.571603in}}%
\pgfpathlineto{\pgfqpoint{3.176054in}{0.648012in}}%
\pgfpathlineto{\pgfqpoint{3.068508in}{0.724420in}}%
\pgfpathlineto{\pgfqpoint{2.965884in}{0.800829in}}%
\pgfpathlineto{\pgfqpoint{2.868091in}{0.877238in}}%
\pgfpathlineto{\pgfqpoint{2.774929in}{0.953646in}}%
\pgfpathlineto{\pgfqpoint{2.686281in}{1.030055in}}%
\pgfpathlineto{\pgfqpoint{2.602010in}{1.106463in}}%
\pgfpathlineto{\pgfqpoint{2.521955in}{1.182872in}}%
\pgfpathlineto{\pgfqpoint{2.446094in}{1.259281in}}%
\pgfpathlineto{\pgfqpoint{2.374165in}{1.335689in}}%
\pgfpathlineto{\pgfqpoint{2.306206in}{1.412098in}}%
\pgfpathlineto{\pgfqpoint{2.242003in}{1.488506in}}%
\pgfpathlineto{\pgfqpoint{2.181474in}{1.564915in}}%
\pgfpathlineto{\pgfqpoint{2.124525in}{1.641323in}}%
\pgfpathlineto{\pgfqpoint{2.071049in}{1.717732in}}%
\pgfpathlineto{\pgfqpoint{2.017501in}{1.799672in}}%
\pgfpathlineto{\pgfqpoint{1.974260in}{1.870549in}}%
\pgfpathlineto{\pgfqpoint{1.930712in}{1.946958in}}%
\pgfpathlineto{\pgfqpoint{1.890382in}{2.023366in}}%
\pgfpathlineto{\pgfqpoint{1.853125in}{2.099775in}}%
\pgfpathlineto{\pgfqpoint{1.818872in}{2.176183in}}%
\pgfpathlineto{\pgfqpoint{1.787579in}{2.252592in}}%
\pgfpathlineto{\pgfqpoint{1.759194in}{2.329001in}}%
\pgfpathlineto{\pgfqpoint{1.733659in}{2.405409in}}%
\pgfpathlineto{\pgfqpoint{1.710907in}{2.481818in}}%
\pgfpathlineto{\pgfqpoint{1.689752in}{2.562950in}}%
\pgfpathlineto{\pgfqpoint{1.673654in}{2.634635in}}%
\pgfpathlineto{\pgfqpoint{1.659011in}{2.711044in}}%
\pgfpathlineto{\pgfqpoint{1.647114in}{2.787452in}}%
\pgfpathlineto{\pgfqpoint{1.637781in}{2.863861in}}%
\pgfpathlineto{\pgfqpoint{1.631025in}{2.940269in}}%
\pgfpathlineto{\pgfqpoint{1.626909in}{3.016678in}}%
\pgfpathlineto{\pgfqpoint{1.625363in}{3.093086in}}%
\pgfpathlineto{\pgfqpoint{1.626373in}{3.169495in}}%
\pgfpathlineto{\pgfqpoint{1.630161in}{3.249175in}}%
\pgfpathlineto{\pgfqpoint{1.636139in}{3.322312in}}%
\pgfpathlineto{\pgfqpoint{1.644917in}{3.398721in}}%
\pgfpathlineto{\pgfqpoint{1.656249in}{3.475129in}}%
\pgfpathlineto{\pgfqpoint{1.670279in}{3.551538in}}%
\pgfpathlineto{\pgfqpoint{1.686913in}{3.627946in}}%
\pgfpathlineto{\pgfqpoint{1.706332in}{3.704355in}}%
\pgfpathlineto{\pgfqpoint{1.728461in}{3.780764in}}%
\pgfpathlineto{\pgfqpoint{1.753393in}{3.857172in}}%
\pgfpathlineto{\pgfqpoint{1.781222in}{3.933581in}}%
\pgfpathlineto{\pgfqpoint{1.812041in}{4.009989in}}%
\pgfpathlineto{\pgfqpoint{1.845937in}{4.086398in}}%
\pgfpathlineto{\pgfqpoint{1.882997in}{4.162807in}}%
\pgfpathlineto{\pgfqpoint{1.923300in}{4.239215in}}%
\pgfpathlineto{\pgfqpoint{1.957911in}{4.300137in}}%
\pgfpathlineto{\pgfqpoint{1.998263in}{4.366563in}}%
\pgfpathlineto{\pgfqpoint{2.047297in}{4.441826in}}%
\pgfpathlineto{\pgfqpoint{2.083530in}{4.493910in}}%
\pgfpathlineto{\pgfqpoint{2.136683in}{4.565923in}}%
\pgfpathlineto{\pgfqpoint{2.180202in}{4.621258in}}%
\pgfpathlineto{\pgfqpoint{2.226069in}{4.676664in}}%
\pgfpathlineto{\pgfqpoint{2.289489in}{4.748606in}}%
\pgfpathlineto{\pgfqpoint{2.345251in}{4.807955in}}%
\pgfpathlineto{\pgfqpoint{2.404841in}{4.867785in}}%
\pgfpathlineto{\pgfqpoint{2.467289in}{4.926892in}}%
\pgfpathlineto{\pgfqpoint{2.524186in}{4.977831in}}%
\pgfpathlineto{\pgfqpoint{2.584217in}{5.028770in}}%
\pgfpathlineto{\pgfqpoint{2.647648in}{5.079709in}}%
\pgfpathlineto{\pgfqpoint{2.714769in}{5.130649in}}%
\pgfpathlineto{\pgfqpoint{2.785890in}{5.181588in}}%
\pgfpathlineto{\pgfqpoint{2.851772in}{5.226110in}}%
\pgfpathlineto{\pgfqpoint{2.911363in}{5.264301in}}%
\pgfpathlineto{\pgfqpoint{2.985082in}{5.308935in}}%
\pgfpathlineto{\pgfqpoint{3.060340in}{5.351749in}}%
\pgfpathlineto{\pgfqpoint{3.123027in}{5.385344in}}%
\pgfpathlineto{\pgfqpoint{3.209317in}{5.428590in}}%
\pgfpathlineto{\pgfqpoint{3.280452in}{5.461752in}}%
\pgfpathlineto{\pgfqpoint{3.358294in}{5.495569in}}%
\pgfpathlineto{\pgfqpoint{3.447680in}{5.531231in}}%
\pgfpathlineto{\pgfqpoint{3.507271in}{5.553143in}}%
\pgfpathlineto{\pgfqpoint{3.566862in}{5.573595in}}%
\pgfpathlineto{\pgfqpoint{3.656248in}{5.601514in}}%
\pgfpathlineto{\pgfqpoint{3.715839in}{5.618307in}}%
\pgfpathlineto{\pgfqpoint{3.802548in}{5.640039in}}%
\pgfpathlineto{\pgfqpoint{3.802548in}{5.640039in}}%
\pgfusepath{stroke}%
\end{pgfscope}%
\begin{pgfscope}%
\pgfpathrectangle{\pgfqpoint{0.766095in}{0.571603in}}{\pgfqpoint{5.929283in}{5.068436in}}%
\pgfusepath{clip}%
\pgfsetbuttcap%
\pgfsetroundjoin%
\pgfsetlinewidth{1.505625pt}%
\definecolor{currentstroke}{rgb}{0.225863,0.330805,0.547314}%
\pgfsetstrokecolor{currentstroke}%
\pgfsetdash{}{0pt}%
\pgfpathmoveto{\pgfqpoint{4.680831in}{5.640039in}}%
\pgfpathlineto{\pgfqpoint{4.728882in}{5.623748in}}%
\pgfpathlineto{\pgfqpoint{4.758677in}{5.612359in}}%
\pgfpathlineto{\pgfqpoint{4.818268in}{5.586234in}}%
\pgfpathlineto{\pgfqpoint{4.862598in}{5.563630in}}%
\pgfpathlineto{\pgfqpoint{4.907654in}{5.537791in}}%
\pgfpathlineto{\pgfqpoint{4.946484in}{5.512691in}}%
\pgfpathlineto{\pgfqpoint{4.982246in}{5.487222in}}%
\pgfpathlineto{\pgfqpoint{5.014949in}{5.461752in}}%
\pgfpathlineto{\pgfqpoint{5.045072in}{5.436283in}}%
\pgfpathlineto{\pgfqpoint{5.073004in}{5.410813in}}%
\pgfpathlineto{\pgfqpoint{5.099062in}{5.385344in}}%
\pgfpathlineto{\pgfqpoint{5.123509in}{5.359874in}}%
\pgfpathlineto{\pgfqpoint{5.168218in}{5.308935in}}%
\pgfpathlineto{\pgfqpoint{5.188794in}{5.283466in}}%
\pgfpathlineto{\pgfqpoint{5.227090in}{5.232527in}}%
\pgfpathlineto{\pgfqpoint{5.262142in}{5.181588in}}%
\pgfpathlineto{\pgfqpoint{5.294509in}{5.130649in}}%
\pgfpathlineto{\pgfqpoint{5.324790in}{5.079469in}}%
\pgfpathlineto{\pgfqpoint{5.366458in}{5.003301in}}%
\pgfpathlineto{\pgfqpoint{5.405210in}{4.926892in}}%
\pgfpathlineto{\pgfqpoint{5.443971in}{4.845436in}}%
\pgfpathlineto{\pgfqpoint{5.498517in}{4.723136in}}%
\pgfpathlineto{\pgfqpoint{5.541902in}{4.621258in}}%
\pgfpathlineto{\pgfqpoint{5.652539in}{4.354196in}}%
\pgfpathlineto{\pgfqpoint{5.712130in}{4.211785in}}%
\pgfpathlineto{\pgfqpoint{5.771720in}{4.073595in}}%
\pgfpathlineto{\pgfqpoint{5.822872in}{3.959050in}}%
\pgfpathlineto{\pgfqpoint{5.890902in}{3.813595in}}%
\pgfpathlineto{\pgfqpoint{5.931686in}{3.729825in}}%
\pgfpathlineto{\pgfqpoint{5.996394in}{3.602477in}}%
\pgfpathlineto{\pgfqpoint{6.064518in}{3.475129in}}%
\pgfpathlineto{\pgfqpoint{6.136230in}{3.347782in}}%
\pgfpathlineto{\pgfqpoint{6.196327in}{3.245904in}}%
\pgfpathlineto{\pgfqpoint{6.258925in}{3.144025in}}%
\pgfpathlineto{\pgfqpoint{6.324095in}{3.042147in}}%
\pgfpathlineto{\pgfqpoint{6.397424in}{2.932209in}}%
\pgfpathlineto{\pgfqpoint{6.462371in}{2.838391in}}%
\pgfpathlineto{\pgfqpoint{6.535522in}{2.736513in}}%
\pgfpathlineto{\pgfqpoint{6.611427in}{2.634635in}}%
\pgfpathlineto{\pgfqpoint{6.695378in}{2.526042in}}%
\pgfpathlineto{\pgfqpoint{6.695378in}{2.526042in}}%
\pgfusepath{stroke}%
\end{pgfscope}%
\begin{pgfscope}%
\pgfpathrectangle{\pgfqpoint{0.766095in}{0.571603in}}{\pgfqpoint{5.929283in}{5.068436in}}%
\pgfusepath{clip}%
\pgfsetbuttcap%
\pgfsetroundjoin%
\pgfsetlinewidth{1.505625pt}%
\definecolor{currentstroke}{rgb}{0.214298,0.355619,0.551184}%
\pgfsetstrokecolor{currentstroke}%
\pgfsetdash{}{0pt}%
\pgfpathmoveto{\pgfqpoint{3.171192in}{0.571603in}}%
\pgfpathlineto{\pgfqpoint{3.059941in}{0.648012in}}%
\pgfpathlineto{\pgfqpoint{2.953762in}{0.724420in}}%
\pgfpathlineto{\pgfqpoint{2.851772in}{0.801270in}}%
\pgfpathlineto{\pgfqpoint{2.755688in}{0.877238in}}%
\pgfpathlineto{\pgfqpoint{2.663571in}{0.953646in}}%
\pgfpathlineto{\pgfqpoint{2.575862in}{1.030055in}}%
\pgfpathlineto{\pgfqpoint{2.492422in}{1.106463in}}%
\pgfpathlineto{\pgfqpoint{2.413200in}{1.182872in}}%
\pgfpathlineto{\pgfqpoint{2.337993in}{1.259281in}}%
\pgfpathlineto{\pgfqpoint{2.266727in}{1.335689in}}%
\pgfpathlineto{\pgfqpoint{2.199244in}{1.412098in}}%
\pgfpathlineto{\pgfqpoint{2.135490in}{1.488506in}}%
\pgfpathlineto{\pgfqpoint{2.075360in}{1.564915in}}%
\pgfpathlineto{\pgfqpoint{2.017501in}{1.643056in}}%
\pgfpathlineto{\pgfqpoint{1.965569in}{1.717732in}}%
\pgfpathlineto{\pgfqpoint{1.915715in}{1.794141in}}%
\pgfpathlineto{\pgfqpoint{1.868524in}{1.871459in}}%
\pgfpathlineto{\pgfqpoint{1.825659in}{1.946958in}}%
\pgfpathlineto{\pgfqpoint{1.785295in}{2.023366in}}%
\pgfpathlineto{\pgfqpoint{1.747950in}{2.099775in}}%
\pgfpathlineto{\pgfqpoint{1.713611in}{2.176183in}}%
\pgfpathlineto{\pgfqpoint{1.682154in}{2.252592in}}%
\pgfpathlineto{\pgfqpoint{1.653526in}{2.329001in}}%
\pgfpathlineto{\pgfqpoint{1.627669in}{2.405409in}}%
\pgfpathlineto{\pgfqpoint{1.604580in}{2.481818in}}%
\pgfpathlineto{\pgfqpoint{1.584201in}{2.558226in}}%
\pgfpathlineto{\pgfqpoint{1.566420in}{2.634635in}}%
\pgfpathlineto{\pgfqpoint{1.551306in}{2.711044in}}%
\pgfpathlineto{\pgfqpoint{1.538718in}{2.787452in}}%
\pgfpathlineto{\pgfqpoint{1.528763in}{2.863861in}}%
\pgfpathlineto{\pgfqpoint{1.521300in}{2.940269in}}%
\pgfpathlineto{\pgfqpoint{1.516341in}{3.016678in}}%
\pgfpathlineto{\pgfqpoint{1.513890in}{3.093086in}}%
\pgfpathlineto{\pgfqpoint{1.513948in}{3.169495in}}%
\pgfpathlineto{\pgfqpoint{1.516514in}{3.245904in}}%
\pgfpathlineto{\pgfqpoint{1.521578in}{3.322312in}}%
\pgfpathlineto{\pgfqpoint{1.529130in}{3.398721in}}%
\pgfpathlineto{\pgfqpoint{1.539153in}{3.475129in}}%
\pgfpathlineto{\pgfqpoint{1.551799in}{3.551538in}}%
\pgfpathlineto{\pgfqpoint{1.566936in}{3.627946in}}%
\pgfpathlineto{\pgfqpoint{1.584733in}{3.704355in}}%
\pgfpathlineto{\pgfqpoint{1.605104in}{3.780764in}}%
\pgfpathlineto{\pgfqpoint{1.630161in}{3.863335in}}%
\pgfpathlineto{\pgfqpoint{1.654024in}{3.933581in}}%
\pgfpathlineto{\pgfqpoint{1.682667in}{4.009989in}}%
\pgfpathlineto{\pgfqpoint{1.714182in}{4.086398in}}%
\pgfpathlineto{\pgfqpoint{1.736868in}{4.137337in}}%
\pgfpathlineto{\pgfqpoint{1.773382in}{4.213746in}}%
\pgfpathlineto{\pgfqpoint{1.808934in}{4.282494in}}%
\pgfpathlineto{\pgfqpoint{1.841318in}{4.341093in}}%
\pgfpathlineto{\pgfqpoint{1.886586in}{4.417502in}}%
\pgfpathlineto{\pgfqpoint{1.928115in}{4.482982in}}%
\pgfpathlineto{\pgfqpoint{1.969885in}{4.544849in}}%
\pgfpathlineto{\pgfqpoint{2.017501in}{4.611305in}}%
\pgfpathlineto{\pgfqpoint{2.063860in}{4.672197in}}%
\pgfpathlineto{\pgfqpoint{2.106888in}{4.725896in}}%
\pgfpathlineto{\pgfqpoint{2.169562in}{4.799545in}}%
\pgfpathlineto{\pgfqpoint{2.226069in}{4.861902in}}%
\pgfpathlineto{\pgfqpoint{2.288496in}{4.926892in}}%
\pgfpathlineto{\pgfqpoint{2.345251in}{4.982713in}}%
\pgfpathlineto{\pgfqpoint{2.404841in}{5.038302in}}%
\pgfpathlineto{\pgfqpoint{2.464432in}{5.091082in}}%
\pgfpathlineto{\pgfqpoint{2.524023in}{5.141264in}}%
\pgfpathlineto{\pgfqpoint{2.583614in}{5.189043in}}%
\pgfpathlineto{\pgfqpoint{2.643205in}{5.234595in}}%
\pgfpathlineto{\pgfqpoint{2.710524in}{5.283466in}}%
\pgfpathlineto{\pgfqpoint{2.792182in}{5.339446in}}%
\pgfpathlineto{\pgfqpoint{2.863282in}{5.385344in}}%
\pgfpathlineto{\pgfqpoint{2.941158in}{5.432866in}}%
\pgfpathlineto{\pgfqpoint{3.000749in}{5.467304in}}%
\pgfpathlineto{\pgfqpoint{3.090135in}{5.516068in}}%
\pgfpathlineto{\pgfqpoint{3.183953in}{5.563630in}}%
\pgfpathlineto{\pgfqpoint{3.268908in}{5.603636in}}%
\pgfpathlineto{\pgfqpoint{3.351877in}{5.640039in}}%
\pgfpathlineto{\pgfqpoint{3.351877in}{5.640039in}}%
\pgfusepath{stroke}%
\end{pgfscope}%
\begin{pgfscope}%
\pgfpathrectangle{\pgfqpoint{0.766095in}{0.571603in}}{\pgfqpoint{5.929283in}{5.068436in}}%
\pgfusepath{clip}%
\pgfsetbuttcap%
\pgfsetroundjoin%
\pgfsetlinewidth{1.505625pt}%
\definecolor{currentstroke}{rgb}{0.214298,0.355619,0.551184}%
\pgfsetstrokecolor{currentstroke}%
\pgfsetdash{}{0pt}%
\pgfpathmoveto{\pgfqpoint{5.152260in}{5.640039in}}%
\pgfpathlineto{\pgfqpoint{5.182614in}{5.614570in}}%
\pgfpathlineto{\pgfqpoint{5.210658in}{5.589100in}}%
\pgfpathlineto{\pgfqpoint{5.236721in}{5.563630in}}%
\pgfpathlineto{\pgfqpoint{5.265199in}{5.533568in}}%
\pgfpathlineto{\pgfqpoint{5.305084in}{5.487222in}}%
\pgfpathlineto{\pgfqpoint{5.344409in}{5.436283in}}%
\pgfpathlineto{\pgfqpoint{5.362580in}{5.410813in}}%
\pgfpathlineto{\pgfqpoint{5.396358in}{5.359874in}}%
\pgfpathlineto{\pgfqpoint{5.427285in}{5.308935in}}%
\pgfpathlineto{\pgfqpoint{5.455843in}{5.257996in}}%
\pgfpathlineto{\pgfqpoint{5.482434in}{5.207057in}}%
\pgfpathlineto{\pgfqpoint{5.519284in}{5.130649in}}%
\pgfpathlineto{\pgfqpoint{5.553304in}{5.054240in}}%
\pgfpathlineto{\pgfqpoint{5.585157in}{4.977831in}}%
\pgfpathlineto{\pgfqpoint{5.622744in}{4.882450in}}%
\pgfpathlineto{\pgfqpoint{5.682030in}{4.723136in}}%
\pgfpathlineto{\pgfqpoint{5.737163in}{4.570319in}}%
\pgfpathlineto{\pgfqpoint{5.839892in}{4.290154in}}%
\pgfpathlineto{\pgfqpoint{5.890902in}{4.158165in}}%
\pgfpathlineto{\pgfqpoint{5.940438in}{4.035459in}}%
\pgfpathlineto{\pgfqpoint{5.994570in}{3.908111in}}%
\pgfpathlineto{\pgfqpoint{6.051777in}{3.780764in}}%
\pgfpathlineto{\pgfqpoint{6.099987in}{3.678886in}}%
\pgfpathlineto{\pgfqpoint{6.150403in}{3.577007in}}%
\pgfpathlineto{\pgfqpoint{6.203231in}{3.475129in}}%
\pgfpathlineto{\pgfqpoint{6.258551in}{3.373251in}}%
\pgfpathlineto{\pgfqpoint{6.316430in}{3.271373in}}%
\pgfpathlineto{\pgfqpoint{6.376932in}{3.169495in}}%
\pgfpathlineto{\pgfqpoint{6.440118in}{3.067617in}}%
\pgfpathlineto{\pgfqpoint{6.506044in}{2.965739in}}%
\pgfpathlineto{\pgfqpoint{6.576196in}{2.861804in}}%
\pgfpathlineto{\pgfqpoint{6.646241in}{2.761983in}}%
\pgfpathlineto{\pgfqpoint{6.695378in}{2.694265in}}%
\pgfpathlineto{\pgfqpoint{6.695378in}{2.694265in}}%
\pgfusepath{stroke}%
\end{pgfscope}%
\begin{pgfscope}%
\pgfpathrectangle{\pgfqpoint{0.766095in}{0.571603in}}{\pgfqpoint{5.929283in}{5.068436in}}%
\pgfusepath{clip}%
\pgfsetbuttcap%
\pgfsetroundjoin%
\pgfsetlinewidth{1.505625pt}%
\definecolor{currentstroke}{rgb}{0.204903,0.375746,0.553533}%
\pgfsetstrokecolor{currentstroke}%
\pgfsetdash{}{0pt}%
\pgfpathmoveto{\pgfqpoint{3.059125in}{0.571603in}}%
\pgfpathlineto{\pgfqpoint{2.949273in}{0.648012in}}%
\pgfpathlineto{\pgfqpoint{2.844283in}{0.724420in}}%
\pgfpathlineto{\pgfqpoint{2.744052in}{0.800829in}}%
\pgfpathlineto{\pgfqpoint{2.648401in}{0.877238in}}%
\pgfpathlineto{\pgfqpoint{2.557229in}{0.953646in}}%
\pgfpathlineto{\pgfqpoint{2.470415in}{1.030055in}}%
\pgfpathlineto{\pgfqpoint{2.387817in}{1.106463in}}%
\pgfpathlineto{\pgfqpoint{2.309279in}{1.182872in}}%
\pgfpathlineto{\pgfqpoint{2.234736in}{1.259281in}}%
\pgfpathlineto{\pgfqpoint{2.164020in}{1.335689in}}%
\pgfpathlineto{\pgfqpoint{2.097107in}{1.412098in}}%
\pgfpathlineto{\pgfqpoint{2.033819in}{1.488506in}}%
\pgfpathlineto{\pgfqpoint{1.974073in}{1.564915in}}%
\pgfpathlineto{\pgfqpoint{1.917775in}{1.641323in}}%
\pgfpathlineto{\pgfqpoint{1.864820in}{1.717732in}}%
\pgfpathlineto{\pgfqpoint{1.815174in}{1.794141in}}%
\pgfpathlineto{\pgfqpoint{1.768721in}{1.870549in}}%
\pgfpathlineto{\pgfqpoint{1.725356in}{1.946958in}}%
\pgfpathlineto{\pgfqpoint{1.685041in}{2.023366in}}%
\pgfpathlineto{\pgfqpoint{1.647723in}{2.099775in}}%
\pgfpathlineto{\pgfqpoint{1.613289in}{2.176183in}}%
\pgfpathlineto{\pgfqpoint{1.581696in}{2.252592in}}%
\pgfpathlineto{\pgfqpoint{1.552893in}{2.329001in}}%
\pgfpathlineto{\pgfqpoint{1.526823in}{2.405409in}}%
\pgfpathlineto{\pgfqpoint{1.503423in}{2.481818in}}%
\pgfpathlineto{\pgfqpoint{1.481184in}{2.564070in}}%
\pgfpathlineto{\pgfqpoint{1.464536in}{2.634635in}}%
\pgfpathlineto{\pgfqpoint{1.448910in}{2.711044in}}%
\pgfpathlineto{\pgfqpoint{1.435886in}{2.787452in}}%
\pgfpathlineto{\pgfqpoint{1.425295in}{2.863861in}}%
\pgfpathlineto{\pgfqpoint{1.417215in}{2.940269in}}%
\pgfpathlineto{\pgfqpoint{1.411609in}{3.016678in}}%
\pgfpathlineto{\pgfqpoint{1.408429in}{3.093086in}}%
\pgfpathlineto{\pgfqpoint{1.407677in}{3.169495in}}%
\pgfpathlineto{\pgfqpoint{1.409349in}{3.245904in}}%
\pgfpathlineto{\pgfqpoint{1.413439in}{3.322312in}}%
\pgfpathlineto{\pgfqpoint{1.419935in}{3.398721in}}%
\pgfpathlineto{\pgfqpoint{1.428935in}{3.475129in}}%
\pgfpathlineto{\pgfqpoint{1.440371in}{3.551538in}}%
\pgfpathlineto{\pgfqpoint{1.454239in}{3.627946in}}%
\pgfpathlineto{\pgfqpoint{1.470678in}{3.704355in}}%
\pgfpathlineto{\pgfqpoint{1.489617in}{3.780764in}}%
\pgfpathlineto{\pgfqpoint{1.511094in}{3.857172in}}%
\pgfpathlineto{\pgfqpoint{1.535281in}{3.933581in}}%
\pgfpathlineto{\pgfqpoint{1.562131in}{4.009989in}}%
\pgfpathlineto{\pgfqpoint{1.591720in}{4.086398in}}%
\pgfpathlineto{\pgfqpoint{1.624121in}{4.162807in}}%
\pgfpathlineto{\pgfqpoint{1.659409in}{4.239215in}}%
\pgfpathlineto{\pgfqpoint{1.689752in}{4.300138in}}%
\pgfpathlineto{\pgfqpoint{1.725051in}{4.366563in}}%
\pgfpathlineto{\pgfqpoint{1.768666in}{4.442971in}}%
\pgfpathlineto{\pgfqpoint{1.808934in}{4.508900in}}%
\pgfpathlineto{\pgfqpoint{1.848807in}{4.570319in}}%
\pgfpathlineto{\pgfqpoint{1.901609in}{4.646728in}}%
\pgfpathlineto{\pgfqpoint{1.958209in}{4.723136in}}%
\pgfpathlineto{\pgfqpoint{2.018847in}{4.799545in}}%
\pgfpathlineto{\pgfqpoint{2.083765in}{4.875953in}}%
\pgfpathlineto{\pgfqpoint{2.136683in}{4.934669in}}%
\pgfpathlineto{\pgfqpoint{2.202144in}{5.003301in}}%
\pgfpathlineto{\pgfqpoint{2.255864in}{5.056676in}}%
\pgfpathlineto{\pgfqpoint{2.315455in}{5.112977in}}%
\pgfpathlineto{\pgfqpoint{2.375046in}{5.166550in}}%
\pgfpathlineto{\pgfqpoint{2.434637in}{5.217599in}}%
\pgfpathlineto{\pgfqpoint{2.494228in}{5.266308in}}%
\pgfpathlineto{\pgfqpoint{2.553818in}{5.312850in}}%
\pgfpathlineto{\pgfqpoint{2.616895in}{5.359874in}}%
\pgfpathlineto{\pgfqpoint{2.702795in}{5.420467in}}%
\pgfpathlineto{\pgfqpoint{2.764485in}{5.461752in}}%
\pgfpathlineto{\pgfqpoint{2.851772in}{5.517068in}}%
\pgfpathlineto{\pgfqpoint{2.941158in}{5.570246in}}%
\pgfpathlineto{\pgfqpoint{3.030545in}{5.620136in}}%
\pgfpathlineto{\pgfqpoint{3.067892in}{5.640039in}}%
\pgfpathlineto{\pgfqpoint{3.067892in}{5.640039in}}%
\pgfusepath{stroke}%
\end{pgfscope}%
\begin{pgfscope}%
\pgfpathrectangle{\pgfqpoint{0.766095in}{0.571603in}}{\pgfqpoint{5.929283in}{5.068436in}}%
\pgfusepath{clip}%
\pgfsetbuttcap%
\pgfsetroundjoin%
\pgfsetlinewidth{1.505625pt}%
\definecolor{currentstroke}{rgb}{0.204903,0.375746,0.553533}%
\pgfsetstrokecolor{currentstroke}%
\pgfsetdash{}{0pt}%
\pgfpathmoveto{\pgfqpoint{5.456810in}{5.640039in}}%
\pgfpathlineto{\pgfqpoint{5.494359in}{5.589100in}}%
\pgfpathlineto{\pgfqpoint{5.511618in}{5.563630in}}%
\pgfpathlineto{\pgfqpoint{5.543556in}{5.512691in}}%
\pgfpathlineto{\pgfqpoint{5.572592in}{5.461752in}}%
\pgfpathlineto{\pgfqpoint{5.599216in}{5.410813in}}%
\pgfpathlineto{\pgfqpoint{5.623834in}{5.359874in}}%
\pgfpathlineto{\pgfqpoint{5.657576in}{5.283466in}}%
\pgfpathlineto{\pgfqpoint{5.688383in}{5.207057in}}%
\pgfpathlineto{\pgfqpoint{5.716938in}{5.130649in}}%
\pgfpathlineto{\pgfqpoint{5.752401in}{5.028770in}}%
\pgfpathlineto{\pgfqpoint{5.794047in}{4.901423in}}%
\pgfpathlineto{\pgfqpoint{5.849945in}{4.723136in}}%
\pgfpathlineto{\pgfqpoint{5.930380in}{4.468441in}}%
\pgfpathlineto{\pgfqpoint{5.981295in}{4.315624in}}%
\pgfpathlineto{\pgfqpoint{6.026054in}{4.188276in}}%
\pgfpathlineto{\pgfqpoint{6.073533in}{4.060928in}}%
\pgfpathlineto{\pgfqpoint{6.113639in}{3.959050in}}%
\pgfpathlineto{\pgfqpoint{6.159061in}{3.849819in}}%
\pgfpathlineto{\pgfqpoint{6.200339in}{3.755294in}}%
\pgfpathlineto{\pgfqpoint{6.248447in}{3.650805in}}%
\pgfpathlineto{\pgfqpoint{6.296463in}{3.551538in}}%
\pgfpathlineto{\pgfqpoint{6.348316in}{3.449660in}}%
\pgfpathlineto{\pgfqpoint{6.402815in}{3.347782in}}%
\pgfpathlineto{\pgfqpoint{6.460011in}{3.245904in}}%
\pgfpathlineto{\pgfqpoint{6.519957in}{3.144025in}}%
\pgfpathlineto{\pgfqpoint{6.582700in}{3.042147in}}%
\pgfpathlineto{\pgfqpoint{6.648290in}{2.940269in}}%
\pgfpathlineto{\pgfqpoint{6.695378in}{2.869828in}}%
\pgfpathlineto{\pgfqpoint{6.695378in}{2.869828in}}%
\pgfusepath{stroke}%
\end{pgfscope}%
\begin{pgfscope}%
\pgfpathrectangle{\pgfqpoint{0.766095in}{0.571603in}}{\pgfqpoint{5.929283in}{5.068436in}}%
\pgfusepath{clip}%
\pgfsetbuttcap%
\pgfsetroundjoin%
\pgfsetlinewidth{1.505625pt}%
\definecolor{currentstroke}{rgb}{0.194100,0.399323,0.555565}%
\pgfsetstrokecolor{currentstroke}%
\pgfsetdash{}{0pt}%
\pgfpathmoveto{\pgfqpoint{2.951994in}{0.571603in}}%
\pgfpathlineto{\pgfqpoint{2.843342in}{0.648012in}}%
\pgfpathlineto{\pgfqpoint{2.739494in}{0.724420in}}%
\pgfpathlineto{\pgfqpoint{2.640297in}{0.800829in}}%
\pgfpathlineto{\pgfqpoint{2.545660in}{0.877238in}}%
\pgfpathlineto{\pgfqpoint{2.455398in}{0.953646in}}%
\pgfpathlineto{\pgfqpoint{2.369389in}{1.030055in}}%
\pgfpathlineto{\pgfqpoint{2.285660in}{1.108247in}}%
\pgfpathlineto{\pgfqpoint{2.209710in}{1.182872in}}%
\pgfpathlineto{\pgfqpoint{2.135741in}{1.259281in}}%
\pgfpathlineto{\pgfqpoint{2.065644in}{1.335689in}}%
\pgfpathlineto{\pgfqpoint{1.999200in}{1.412098in}}%
\pgfpathlineto{\pgfqpoint{1.936336in}{1.488506in}}%
\pgfpathlineto{\pgfqpoint{1.876969in}{1.564915in}}%
\pgfpathlineto{\pgfqpoint{1.821005in}{1.641323in}}%
\pgfpathlineto{\pgfqpoint{1.768338in}{1.717732in}}%
\pgfpathlineto{\pgfqpoint{1.718855in}{1.794141in}}%
\pgfpathlineto{\pgfqpoint{1.672594in}{1.870549in}}%
\pgfpathlineto{\pgfqpoint{1.629311in}{1.946958in}}%
\pgfpathlineto{\pgfqpoint{1.589100in}{2.023366in}}%
\pgfpathlineto{\pgfqpoint{1.551770in}{2.099775in}}%
\pgfpathlineto{\pgfqpoint{1.517282in}{2.176183in}}%
\pgfpathlineto{\pgfqpoint{1.485594in}{2.252592in}}%
\pgfpathlineto{\pgfqpoint{1.456656in}{2.329001in}}%
\pgfpathlineto{\pgfqpoint{1.430411in}{2.405409in}}%
\pgfpathlineto{\pgfqpoint{1.406797in}{2.481818in}}%
\pgfpathlineto{\pgfqpoint{1.385747in}{2.558226in}}%
\pgfpathlineto{\pgfqpoint{1.367260in}{2.634635in}}%
\pgfpathlineto{\pgfqpoint{1.351307in}{2.711044in}}%
\pgfpathlineto{\pgfqpoint{1.337799in}{2.787452in}}%
\pgfpathlineto{\pgfqpoint{1.326756in}{2.863861in}}%
\pgfpathlineto{\pgfqpoint{1.318159in}{2.940269in}}%
\pgfpathlineto{\pgfqpoint{1.311940in}{3.016678in}}%
\pgfpathlineto{\pgfqpoint{1.308102in}{3.093086in}}%
\pgfpathlineto{\pgfqpoint{1.306646in}{3.169495in}}%
\pgfpathlineto{\pgfqpoint{1.307571in}{3.245904in}}%
\pgfpathlineto{\pgfqpoint{1.310869in}{3.322312in}}%
\pgfpathlineto{\pgfqpoint{1.316533in}{3.398721in}}%
\pgfpathlineto{\pgfqpoint{1.324549in}{3.475129in}}%
\pgfpathlineto{\pgfqpoint{1.334941in}{3.551538in}}%
\pgfpathlineto{\pgfqpoint{1.347798in}{3.627946in}}%
\pgfpathlineto{\pgfqpoint{1.362997in}{3.704355in}}%
\pgfpathlineto{\pgfqpoint{1.380746in}{3.780764in}}%
\pgfpathlineto{\pgfqpoint{1.400939in}{3.857172in}}%
\pgfpathlineto{\pgfqpoint{1.423636in}{3.933581in}}%
\pgfpathlineto{\pgfqpoint{1.451389in}{4.016926in}}%
\pgfpathlineto{\pgfqpoint{1.481184in}{4.097439in}}%
\pgfpathlineto{\pgfqpoint{1.510980in}{4.170876in}}%
\pgfpathlineto{\pgfqpoint{1.540995in}{4.239215in}}%
\pgfpathlineto{\pgfqpoint{1.577342in}{4.315624in}}%
\pgfpathlineto{\pgfqpoint{1.616625in}{4.392032in}}%
\pgfpathlineto{\pgfqpoint{1.659957in}{4.470268in}}%
\pgfpathlineto{\pgfqpoint{1.704476in}{4.544849in}}%
\pgfpathlineto{\pgfqpoint{1.753282in}{4.621258in}}%
\pgfpathlineto{\pgfqpoint{1.808934in}{4.702379in}}%
\pgfpathlineto{\pgfqpoint{1.861570in}{4.774075in}}%
\pgfpathlineto{\pgfqpoint{1.900970in}{4.825014in}}%
\pgfpathlineto{\pgfqpoint{1.963466in}{4.901423in}}%
\pgfpathlineto{\pgfqpoint{2.017501in}{4.963695in}}%
\pgfpathlineto{\pgfqpoint{2.077092in}{5.028791in}}%
\pgfpathlineto{\pgfqpoint{2.136683in}{5.090358in}}%
\pgfpathlineto{\pgfqpoint{2.203860in}{5.156118in}}%
\pgfpathlineto{\pgfqpoint{2.258559in}{5.207057in}}%
\pgfpathlineto{\pgfqpoint{2.315763in}{5.257996in}}%
\pgfpathlineto{\pgfqpoint{2.375634in}{5.308935in}}%
\pgfpathlineto{\pgfqpoint{2.438341in}{5.359874in}}%
\pgfpathlineto{\pgfqpoint{2.504059in}{5.410813in}}%
\pgfpathlineto{\pgfqpoint{2.583614in}{5.469414in}}%
\pgfpathlineto{\pgfqpoint{2.645299in}{5.512691in}}%
\pgfpathlineto{\pgfqpoint{2.732591in}{5.570831in}}%
\pgfpathlineto{\pgfqpoint{2.821977in}{5.626938in}}%
\pgfpathlineto{\pgfqpoint{2.843623in}{5.640039in}}%
\pgfpathlineto{\pgfqpoint{2.843623in}{5.640039in}}%
\pgfusepath{stroke}%
\end{pgfscope}%
\begin{pgfscope}%
\pgfpathrectangle{\pgfqpoint{0.766095in}{0.571603in}}{\pgfqpoint{5.929283in}{5.068436in}}%
\pgfusepath{clip}%
\pgfsetbuttcap%
\pgfsetroundjoin%
\pgfsetlinewidth{1.505625pt}%
\definecolor{currentstroke}{rgb}{0.194100,0.399323,0.555565}%
\pgfsetstrokecolor{currentstroke}%
\pgfsetdash{}{0pt}%
\pgfpathmoveto{\pgfqpoint{5.701662in}{5.640039in}}%
\pgfpathlineto{\pgfqpoint{5.728173in}{5.589100in}}%
\pgfpathlineto{\pgfqpoint{5.752349in}{5.538161in}}%
\pgfpathlineto{\pgfqpoint{5.774582in}{5.487222in}}%
\pgfpathlineto{\pgfqpoint{5.804836in}{5.410813in}}%
\pgfpathlineto{\pgfqpoint{5.832201in}{5.334405in}}%
\pgfpathlineto{\pgfqpoint{5.865255in}{5.232527in}}%
\pgfpathlineto{\pgfqpoint{5.895573in}{5.130649in}}%
\pgfpathlineto{\pgfqpoint{5.937918in}{4.977831in}}%
\pgfpathlineto{\pgfqpoint{6.062271in}{4.519380in}}%
\pgfpathlineto{\pgfqpoint{6.107553in}{4.366563in}}%
\pgfpathlineto{\pgfqpoint{6.147965in}{4.239215in}}%
\pgfpathlineto{\pgfqpoint{6.191286in}{4.111867in}}%
\pgfpathlineto{\pgfqpoint{6.228199in}{4.009989in}}%
\pgfpathlineto{\pgfqpoint{6.267344in}{3.908111in}}%
\pgfpathlineto{\pgfqpoint{6.308862in}{3.806233in}}%
\pgfpathlineto{\pgfqpoint{6.352766in}{3.704355in}}%
\pgfpathlineto{\pgfqpoint{6.399287in}{3.602477in}}%
\pgfpathlineto{\pgfqpoint{6.448395in}{3.500599in}}%
\pgfpathlineto{\pgfqpoint{6.500232in}{3.398721in}}%
\pgfpathlineto{\pgfqpoint{6.554860in}{3.296843in}}%
\pgfpathlineto{\pgfqpoint{6.612318in}{3.194965in}}%
\pgfpathlineto{\pgfqpoint{6.672649in}{3.093086in}}%
\pgfpathlineto{\pgfqpoint{6.695378in}{3.056004in}}%
\pgfpathlineto{\pgfqpoint{6.695378in}{3.056004in}}%
\pgfusepath{stroke}%
\end{pgfscope}%
\begin{pgfscope}%
\pgfpathrectangle{\pgfqpoint{0.766095in}{0.571603in}}{\pgfqpoint{5.929283in}{5.068436in}}%
\pgfusepath{clip}%
\pgfsetbuttcap%
\pgfsetroundjoin%
\pgfsetlinewidth{1.505625pt}%
\definecolor{currentstroke}{rgb}{0.183898,0.422383,0.556944}%
\pgfsetstrokecolor{currentstroke}%
\pgfsetdash{}{0pt}%
\pgfpathmoveto{\pgfqpoint{2.849146in}{0.571603in}}%
\pgfpathlineto{\pgfqpoint{2.741671in}{0.648012in}}%
\pgfpathlineto{\pgfqpoint{2.638881in}{0.724420in}}%
\pgfpathlineto{\pgfqpoint{2.540712in}{0.800829in}}%
\pgfpathlineto{\pgfqpoint{2.446973in}{0.877238in}}%
\pgfpathlineto{\pgfqpoint{2.357557in}{0.953646in}}%
\pgfpathlineto{\pgfqpoint{2.272339in}{1.030055in}}%
\pgfpathlineto{\pgfqpoint{2.191177in}{1.106463in}}%
\pgfpathlineto{\pgfqpoint{2.114000in}{1.182872in}}%
\pgfpathlineto{\pgfqpoint{2.040661in}{1.259281in}}%
\pgfpathlineto{\pgfqpoint{1.971074in}{1.335689in}}%
\pgfpathlineto{\pgfqpoint{1.905095in}{1.412098in}}%
\pgfpathlineto{\pgfqpoint{1.842650in}{1.488506in}}%
\pgfpathlineto{\pgfqpoint{1.783656in}{1.564915in}}%
\pgfpathlineto{\pgfqpoint{1.728017in}{1.641323in}}%
\pgfpathlineto{\pgfqpoint{1.675629in}{1.717732in}}%
\pgfpathlineto{\pgfqpoint{1.626379in}{1.794141in}}%
\pgfpathlineto{\pgfqpoint{1.580266in}{1.870549in}}%
\pgfpathlineto{\pgfqpoint{1.537128in}{1.946958in}}%
\pgfpathlineto{\pgfqpoint{1.496981in}{2.023366in}}%
\pgfpathlineto{\pgfqpoint{1.459673in}{2.099775in}}%
\pgfpathlineto{\pgfqpoint{1.425167in}{2.176183in}}%
\pgfpathlineto{\pgfqpoint{1.391798in}{2.256710in}}%
\pgfpathlineto{\pgfqpoint{1.362003in}{2.335623in}}%
\pgfpathlineto{\pgfqpoint{1.337988in}{2.405409in}}%
\pgfpathlineto{\pgfqpoint{1.314191in}{2.481818in}}%
\pgfpathlineto{\pgfqpoint{1.292920in}{2.558226in}}%
\pgfpathlineto{\pgfqpoint{1.272617in}{2.641338in}}%
\pgfpathlineto{\pgfqpoint{1.257870in}{2.711044in}}%
\pgfpathlineto{\pgfqpoint{1.242821in}{2.794566in}}%
\pgfpathlineto{\pgfqpoint{1.232550in}{2.863861in}}%
\pgfpathlineto{\pgfqpoint{1.223473in}{2.940269in}}%
\pgfpathlineto{\pgfqpoint{1.216730in}{3.016678in}}%
\pgfpathlineto{\pgfqpoint{1.212334in}{3.093086in}}%
\pgfpathlineto{\pgfqpoint{1.210294in}{3.169495in}}%
\pgfpathlineto{\pgfqpoint{1.210558in}{3.245904in}}%
\pgfpathlineto{\pgfqpoint{1.213123in}{3.322312in}}%
\pgfpathlineto{\pgfqpoint{1.218044in}{3.398721in}}%
\pgfpathlineto{\pgfqpoint{1.225277in}{3.475129in}}%
\pgfpathlineto{\pgfqpoint{1.234807in}{3.551538in}}%
\pgfpathlineto{\pgfqpoint{1.246670in}{3.627946in}}%
\pgfpathlineto{\pgfqpoint{1.260941in}{3.704355in}}%
\pgfpathlineto{\pgfqpoint{1.277551in}{3.780764in}}%
\pgfpathlineto{\pgfqpoint{1.296605in}{3.857172in}}%
\pgfpathlineto{\pgfqpoint{1.318131in}{3.933581in}}%
\pgfpathlineto{\pgfqpoint{1.342130in}{4.009989in}}%
\pgfpathlineto{\pgfqpoint{1.368674in}{4.086398in}}%
\pgfpathlineto{\pgfqpoint{1.397835in}{4.162807in}}%
\pgfpathlineto{\pgfqpoint{1.429680in}{4.239215in}}%
\pgfpathlineto{\pgfqpoint{1.464277in}{4.315624in}}%
\pgfpathlineto{\pgfqpoint{1.501688in}{4.392032in}}%
\pgfpathlineto{\pgfqpoint{1.541991in}{4.468441in}}%
\pgfpathlineto{\pgfqpoint{1.585417in}{4.544849in}}%
\pgfpathlineto{\pgfqpoint{1.631896in}{4.621258in}}%
\pgfpathlineto{\pgfqpoint{1.681737in}{4.697667in}}%
\pgfpathlineto{\pgfqpoint{1.719547in}{4.752487in}}%
\pgfpathlineto{\pgfqpoint{1.772442in}{4.825014in}}%
\pgfpathlineto{\pgfqpoint{1.811546in}{4.875953in}}%
\pgfpathlineto{\pgfqpoint{1.873474in}{4.952362in}}%
\pgfpathlineto{\pgfqpoint{1.939478in}{5.028770in}}%
\pgfpathlineto{\pgfqpoint{1.987706in}{5.081762in}}%
\pgfpathlineto{\pgfqpoint{2.059164in}{5.156118in}}%
\pgfpathlineto{\pgfqpoint{2.110658in}{5.207057in}}%
\pgfpathlineto{\pgfqpoint{2.166478in}{5.259913in}}%
\pgfpathlineto{\pgfqpoint{2.226069in}{5.313830in}}%
\pgfpathlineto{\pgfqpoint{2.285660in}{5.365400in}}%
\pgfpathlineto{\pgfqpoint{2.345251in}{5.414790in}}%
\pgfpathlineto{\pgfqpoint{2.404841in}{5.462156in}}%
\pgfpathlineto{\pgfqpoint{2.494228in}{5.529588in}}%
\pgfpathlineto{\pgfqpoint{2.577750in}{5.589100in}}%
\pgfpathlineto{\pgfqpoint{2.653084in}{5.640039in}}%
\pgfpathlineto{\pgfqpoint{2.653084in}{5.640039in}}%
\pgfusepath{stroke}%
\end{pgfscope}%
\begin{pgfscope}%
\pgfpathrectangle{\pgfqpoint{0.766095in}{0.571603in}}{\pgfqpoint{5.929283in}{5.068436in}}%
\pgfusepath{clip}%
\pgfsetbuttcap%
\pgfsetroundjoin%
\pgfsetlinewidth{1.505625pt}%
\definecolor{currentstroke}{rgb}{0.183898,0.422383,0.556944}%
\pgfsetstrokecolor{currentstroke}%
\pgfsetdash{}{0pt}%
\pgfpathmoveto{\pgfqpoint{5.912776in}{5.640039in}}%
\pgfpathlineto{\pgfqpoint{5.920697in}{5.619850in}}%
\pgfpathlineto{\pgfqpoint{5.922715in}{5.614570in}}%
\pgfpathlineto{\pgfqpoint{5.932162in}{5.589100in}}%
\pgfpathlineto{\pgfqpoint{5.941271in}{5.563630in}}%
\pgfpathlineto{\pgfqpoint{5.950067in}{5.538161in}}%
\pgfpathlineto{\pgfqpoint{5.950493in}{5.536896in}}%
\pgfpathlineto{\pgfqpoint{5.958447in}{5.512691in}}%
\pgfpathlineto{\pgfqpoint{5.966558in}{5.487222in}}%
\pgfpathlineto{\pgfqpoint{5.974426in}{5.461752in}}%
\pgfpathlineto{\pgfqpoint{5.980288in}{5.442258in}}%
\pgfpathlineto{\pgfqpoint{5.982044in}{5.436283in}}%
\pgfpathlineto{\pgfqpoint{5.989372in}{5.410813in}}%
\pgfpathlineto{\pgfqpoint{5.996516in}{5.385344in}}%
\pgfpathlineto{\pgfqpoint{6.003490in}{5.359874in}}%
\pgfpathlineto{\pgfqpoint{6.010084in}{5.335255in}}%
\pgfpathlineto{\pgfqpoint{6.010307in}{5.334405in}}%
\pgfpathlineto{\pgfqpoint{6.016893in}{5.308935in}}%
\pgfpathlineto{\pgfqpoint{6.023356in}{5.283466in}}%
\pgfpathlineto{\pgfqpoint{6.029709in}{5.257996in}}%
\pgfpathlineto{\pgfqpoint{6.035964in}{5.232527in}}%
\pgfpathlineto{\pgfqpoint{6.039879in}{5.216399in}}%
\pgfpathlineto{\pgfqpoint{6.042102in}{5.207057in}}%
\pgfpathlineto{\pgfqpoint{6.048114in}{5.181588in}}%
\pgfpathlineto{\pgfqpoint{6.054063in}{5.156118in}}%
\pgfpathlineto{\pgfqpoint{6.059960in}{5.130649in}}%
\pgfpathlineto{\pgfqpoint{6.065815in}{5.105179in}}%
\pgfpathlineto{\pgfqpoint{6.069674in}{5.088320in}}%
\pgfpathlineto{\pgfqpoint{6.071610in}{5.079709in}}%
\pgfpathlineto{\pgfqpoint{6.077332in}{5.054240in}}%
\pgfpathlineto{\pgfqpoint{6.083039in}{5.028770in}}%
\pgfpathlineto{\pgfqpoint{6.088740in}{5.003301in}}%
\pgfpathlineto{\pgfqpoint{6.094442in}{4.977831in}}%
\pgfpathlineto{\pgfqpoint{6.099470in}{4.955422in}}%
\pgfpathlineto{\pgfqpoint{6.100145in}{4.952362in}}%
\pgfpathlineto{\pgfqpoint{6.105801in}{4.926892in}}%
\pgfpathlineto{\pgfqpoint{6.111482in}{4.901423in}}%
\pgfpathlineto{\pgfqpoint{6.117193in}{4.875953in}}%
\pgfpathlineto{\pgfqpoint{6.122941in}{4.850484in}}%
\pgfpathlineto{\pgfqpoint{6.128732in}{4.825014in}}%
\pgfpathlineto{\pgfqpoint{6.129265in}{4.822698in}}%
\pgfpathlineto{\pgfqpoint{6.134510in}{4.799545in}}%
\pgfpathlineto{\pgfqpoint{6.140337in}{4.774075in}}%
\pgfpathlineto{\pgfqpoint{6.146225in}{4.748606in}}%
\pgfpathlineto{\pgfqpoint{6.152180in}{4.723136in}}%
\pgfpathlineto{\pgfqpoint{6.158207in}{4.697667in}}%
\pgfpathlineto{\pgfqpoint{6.159061in}{4.694116in}}%
\pgfpathlineto{\pgfqpoint{6.164251in}{4.672197in}}%
\pgfpathlineto{\pgfqpoint{6.170367in}{4.646728in}}%
\pgfpathlineto{\pgfqpoint{6.176571in}{4.621258in}}%
\pgfpathlineto{\pgfqpoint{6.182866in}{4.595788in}}%
\pgfpathlineto{\pgfqpoint{6.188856in}{4.571925in}}%
\pgfpathlineto{\pgfqpoint{6.189253in}{4.570319in}}%
\pgfpathlineto{\pgfqpoint{6.195674in}{4.544849in}}%
\pgfpathlineto{\pgfqpoint{6.202199in}{4.519380in}}%
\pgfpathlineto{\pgfqpoint{6.208833in}{4.493910in}}%
\pgfpathlineto{\pgfqpoint{6.215582in}{4.468441in}}%
\pgfpathlineto{\pgfqpoint{6.218651in}{4.457068in}}%
\pgfpathlineto{\pgfqpoint{6.222406in}{4.442971in}}%
\pgfpathlineto{\pgfqpoint{6.229318in}{4.417502in}}%
\pgfpathlineto{\pgfqpoint{6.236355in}{4.392032in}}%
\pgfpathlineto{\pgfqpoint{6.243520in}{4.366563in}}%
\pgfpathlineto{\pgfqpoint{6.248447in}{4.349381in}}%
\pgfpathlineto{\pgfqpoint{6.250793in}{4.341093in}}%
\pgfpathlineto{\pgfqpoint{6.258149in}{4.315624in}}%
\pgfpathlineto{\pgfqpoint{6.265643in}{4.290154in}}%
\pgfpathlineto{\pgfqpoint{6.273281in}{4.264685in}}%
\pgfpathlineto{\pgfqpoint{6.278242in}{4.248458in}}%
\pgfpathlineto{\pgfqpoint{6.281035in}{4.239215in}}%
\pgfpathlineto{\pgfqpoint{6.288885in}{4.213746in}}%
\pgfpathlineto{\pgfqpoint{6.296888in}{4.188276in}}%
\pgfpathlineto{\pgfqpoint{6.305046in}{4.162807in}}%
\pgfpathlineto{\pgfqpoint{6.308038in}{4.153652in}}%
\pgfpathlineto{\pgfqpoint{6.313309in}{4.137337in}}%
\pgfpathlineto{\pgfqpoint{6.321701in}{4.111867in}}%
\pgfpathlineto{\pgfqpoint{6.330258in}{4.086398in}}%
\pgfpathlineto{\pgfqpoint{6.337833in}{4.064287in}}%
\pgfpathlineto{\pgfqpoint{6.338971in}{4.060928in}}%
\pgfpathlineto{\pgfqpoint{6.347776in}{4.035459in}}%
\pgfpathlineto{\pgfqpoint{6.356753in}{4.009989in}}%
\pgfpathlineto{\pgfqpoint{6.365907in}{3.984520in}}%
\pgfpathlineto{\pgfqpoint{6.367628in}{3.979823in}}%
\pgfpathlineto{\pgfqpoint{6.375163in}{3.959050in}}%
\pgfpathlineto{\pgfqpoint{6.384582in}{3.933581in}}%
\pgfpathlineto{\pgfqpoint{6.394186in}{3.908111in}}%
\pgfpathlineto{\pgfqpoint{6.397424in}{3.899687in}}%
\pgfpathlineto{\pgfqpoint{6.403911in}{3.882642in}}%
\pgfpathlineto{\pgfqpoint{6.413792in}{3.857172in}}%
\pgfpathlineto{\pgfqpoint{6.423865in}{3.831703in}}%
\pgfpathlineto{\pgfqpoint{6.427219in}{3.823380in}}%
\pgfpathlineto{\pgfqpoint{6.434064in}{3.806233in}}%
\pgfpathlineto{\pgfqpoint{6.444426in}{3.780764in}}%
\pgfpathlineto{\pgfqpoint{6.454987in}{3.755294in}}%
\pgfpathlineto{\pgfqpoint{6.457014in}{3.750495in}}%
\pgfpathlineto{\pgfqpoint{6.465665in}{3.729825in}}%
\pgfpathlineto{\pgfqpoint{6.476526in}{3.704355in}}%
\pgfpathlineto{\pgfqpoint{6.486810in}{3.680689in}}%
\pgfpathlineto{\pgfqpoint{6.487587in}{3.678886in}}%
\pgfpathlineto{\pgfqpoint{6.498755in}{3.653416in}}%
\pgfpathlineto{\pgfqpoint{6.510133in}{3.627946in}}%
\pgfpathlineto{\pgfqpoint{6.516605in}{3.613719in}}%
\pgfpathlineto{\pgfqpoint{6.521676in}{3.602477in}}%
\pgfpathlineto{\pgfqpoint{6.533371in}{3.577007in}}%
\pgfpathlineto{\pgfqpoint{6.545285in}{3.551538in}}%
\pgfpathlineto{\pgfqpoint{6.546401in}{3.549193in}}%
\pgfpathlineto{\pgfqpoint{6.557314in}{3.526068in}}%
\pgfpathlineto{\pgfqpoint{6.569554in}{3.500599in}}%
\pgfpathlineto{\pgfqpoint{6.576196in}{3.487017in}}%
\pgfpathlineto{\pgfqpoint{6.581964in}{3.475129in}}%
\pgfpathlineto{\pgfqpoint{6.594537in}{3.449660in}}%
\pgfpathlineto{\pgfqpoint{6.605991in}{3.426867in}}%
\pgfpathlineto{\pgfqpoint{6.607326in}{3.424190in}}%
\pgfpathlineto{\pgfqpoint{6.620238in}{3.398721in}}%
\pgfpathlineto{\pgfqpoint{6.633382in}{3.373251in}}%
\pgfpathlineto{\pgfqpoint{6.635787in}{3.368667in}}%
\pgfpathlineto{\pgfqpoint{6.646660in}{3.347782in}}%
\pgfpathlineto{\pgfqpoint{6.660153in}{3.322312in}}%
\pgfpathlineto{\pgfqpoint{6.665582in}{3.312230in}}%
\pgfpathlineto{\pgfqpoint{6.673809in}{3.296843in}}%
\pgfpathlineto{\pgfqpoint{6.687656in}{3.271373in}}%
\pgfpathlineto{\pgfqpoint{6.695378in}{3.257401in}}%
\pgfusepath{stroke}%
\end{pgfscope}%
\begin{pgfscope}%
\pgfpathrectangle{\pgfqpoint{0.766095in}{0.571603in}}{\pgfqpoint{5.929283in}{5.068436in}}%
\pgfusepath{clip}%
\pgfsetbuttcap%
\pgfsetroundjoin%
\pgfsetlinewidth{1.505625pt}%
\definecolor{currentstroke}{rgb}{0.175841,0.441290,0.557685}%
\pgfsetstrokecolor{currentstroke}%
\pgfsetdash{}{0pt}%
\pgfpathmoveto{\pgfqpoint{2.750257in}{0.571603in}}%
\pgfpathlineto{\pgfqpoint{2.643205in}{0.648452in}}%
\pgfpathlineto{\pgfqpoint{2.542067in}{0.724420in}}%
\pgfpathlineto{\pgfqpoint{2.444808in}{0.800829in}}%
\pgfpathlineto{\pgfqpoint{2.351928in}{0.877238in}}%
\pgfpathlineto{\pgfqpoint{2.263316in}{0.953646in}}%
\pgfpathlineto{\pgfqpoint{2.178847in}{1.030055in}}%
\pgfpathlineto{\pgfqpoint{2.098378in}{1.106463in}}%
\pgfpathlineto{\pgfqpoint{2.017501in}{1.187301in}}%
\pgfpathlineto{\pgfqpoint{1.949049in}{1.259281in}}%
\pgfpathlineto{\pgfqpoint{1.879968in}{1.335689in}}%
\pgfpathlineto{\pgfqpoint{1.814448in}{1.412098in}}%
\pgfpathlineto{\pgfqpoint{1.752416in}{1.488506in}}%
\pgfpathlineto{\pgfqpoint{1.693786in}{1.564915in}}%
\pgfpathlineto{\pgfqpoint{1.638463in}{1.641323in}}%
\pgfpathlineto{\pgfqpoint{1.586344in}{1.717732in}}%
\pgfpathlineto{\pgfqpoint{1.537315in}{1.794141in}}%
\pgfpathlineto{\pgfqpoint{1.491383in}{1.870549in}}%
\pgfpathlineto{\pgfqpoint{1.448376in}{1.946958in}}%
\pgfpathlineto{\pgfqpoint{1.408322in}{2.023366in}}%
\pgfpathlineto{\pgfqpoint{1.371066in}{2.099775in}}%
\pgfpathlineto{\pgfqpoint{1.336570in}{2.176183in}}%
\pgfpathlineto{\pgfqpoint{1.304788in}{2.252592in}}%
\pgfpathlineto{\pgfqpoint{1.275673in}{2.329001in}}%
\pgfpathlineto{\pgfqpoint{1.249167in}{2.405409in}}%
\pgfpathlineto{\pgfqpoint{1.225213in}{2.481818in}}%
\pgfpathlineto{\pgfqpoint{1.203746in}{2.558226in}}%
\pgfpathlineto{\pgfqpoint{1.183230in}{2.641141in}}%
\pgfpathlineto{\pgfqpoint{1.168186in}{2.711044in}}%
\pgfpathlineto{\pgfqpoint{1.153435in}{2.790644in}}%
\pgfpathlineto{\pgfqpoint{1.142193in}{2.863861in}}%
\pgfpathlineto{\pgfqpoint{1.132713in}{2.940269in}}%
\pgfpathlineto{\pgfqpoint{1.125523in}{3.016678in}}%
\pgfpathlineto{\pgfqpoint{1.120666in}{3.093086in}}%
\pgfpathlineto{\pgfqpoint{1.118098in}{3.169495in}}%
\pgfpathlineto{\pgfqpoint{1.117793in}{3.245904in}}%
\pgfpathlineto{\pgfqpoint{1.119744in}{3.322312in}}%
\pgfpathlineto{\pgfqpoint{1.123948in}{3.398721in}}%
\pgfpathlineto{\pgfqpoint{1.130477in}{3.475129in}}%
\pgfpathlineto{\pgfqpoint{1.139262in}{3.551538in}}%
\pgfpathlineto{\pgfqpoint{1.150288in}{3.627946in}}%
\pgfpathlineto{\pgfqpoint{1.163679in}{3.704355in}}%
\pgfpathlineto{\pgfqpoint{1.179346in}{3.780764in}}%
\pgfpathlineto{\pgfqpoint{1.197422in}{3.857172in}}%
\pgfpathlineto{\pgfqpoint{1.217832in}{3.933581in}}%
\pgfpathlineto{\pgfqpoint{1.242821in}{4.016720in}}%
\pgfpathlineto{\pgfqpoint{1.266038in}{4.086398in}}%
\pgfpathlineto{\pgfqpoint{1.293907in}{4.162807in}}%
\pgfpathlineto{\pgfqpoint{1.324351in}{4.239215in}}%
\pgfpathlineto{\pgfqpoint{1.357436in}{4.315624in}}%
\pgfpathlineto{\pgfqpoint{1.393243in}{4.392032in}}%
\pgfpathlineto{\pgfqpoint{1.431921in}{4.468441in}}%
\pgfpathlineto{\pgfqpoint{1.473457in}{4.544849in}}%
\pgfpathlineto{\pgfqpoint{1.510980in}{4.609554in}}%
\pgfpathlineto{\pgfqpoint{1.549448in}{4.672197in}}%
\pgfpathlineto{\pgfqpoint{1.600366in}{4.750264in}}%
\pgfpathlineto{\pgfqpoint{1.652473in}{4.825014in}}%
\pgfpathlineto{\pgfqpoint{1.689828in}{4.875953in}}%
\pgfpathlineto{\pgfqpoint{1.749343in}{4.952839in}}%
\pgfpathlineto{\pgfqpoint{1.811921in}{5.028770in}}%
\pgfpathlineto{\pgfqpoint{1.878914in}{5.105179in}}%
\pgfpathlineto{\pgfqpoint{1.928115in}{5.158508in}}%
\pgfpathlineto{\pgfqpoint{2.000106in}{5.232527in}}%
\pgfpathlineto{\pgfqpoint{2.052149in}{5.283466in}}%
\pgfpathlineto{\pgfqpoint{2.106888in}{5.334867in}}%
\pgfpathlineto{\pgfqpoint{2.166478in}{5.388432in}}%
\pgfpathlineto{\pgfqpoint{2.226069in}{5.439750in}}%
\pgfpathlineto{\pgfqpoint{2.285660in}{5.488981in}}%
\pgfpathlineto{\pgfqpoint{2.347723in}{5.538161in}}%
\pgfpathlineto{\pgfqpoint{2.434637in}{5.603697in}}%
\pgfpathlineto{\pgfqpoint{2.484983in}{5.640039in}}%
\pgfpathlineto{\pgfqpoint{2.484983in}{5.640039in}}%
\pgfusepath{stroke}%
\end{pgfscope}%
\begin{pgfscope}%
\pgfpathrectangle{\pgfqpoint{0.766095in}{0.571603in}}{\pgfqpoint{5.929283in}{5.068436in}}%
\pgfusepath{clip}%
\pgfsetbuttcap%
\pgfsetroundjoin%
\pgfsetlinewidth{1.505625pt}%
\definecolor{currentstroke}{rgb}{0.175841,0.441290,0.557685}%
\pgfsetstrokecolor{currentstroke}%
\pgfsetdash{}{0pt}%
\pgfpathmoveto{\pgfqpoint{6.101447in}{5.640039in}}%
\pgfpathlineto{\pgfqpoint{6.108703in}{5.614570in}}%
\pgfpathlineto{\pgfqpoint{6.115699in}{5.589100in}}%
\pgfpathlineto{\pgfqpoint{6.122453in}{5.563630in}}%
\pgfpathlineto{\pgfqpoint{6.128985in}{5.538161in}}%
\pgfpathlineto{\pgfqpoint{6.129265in}{5.537040in}}%
\pgfpathlineto{\pgfqpoint{6.135226in}{5.512691in}}%
\pgfpathlineto{\pgfqpoint{6.141279in}{5.487222in}}%
\pgfpathlineto{\pgfqpoint{6.147163in}{5.461752in}}%
\pgfpathlineto{\pgfqpoint{6.152893in}{5.436283in}}%
\pgfpathlineto{\pgfqpoint{6.158483in}{5.410813in}}%
\pgfpathlineto{\pgfqpoint{6.159061in}{5.408138in}}%
\pgfpathlineto{\pgfqpoint{6.163881in}{5.385344in}}%
\pgfpathlineto{\pgfqpoint{6.169161in}{5.359874in}}%
\pgfpathlineto{\pgfqpoint{6.174341in}{5.334405in}}%
\pgfpathlineto{\pgfqpoint{6.179434in}{5.308935in}}%
\pgfpathlineto{\pgfqpoint{6.184450in}{5.283466in}}%
\pgfpathlineto{\pgfqpoint{6.188856in}{5.260811in}}%
\pgfpathlineto{\pgfqpoint{6.189393in}{5.257996in}}%
\pgfpathlineto{\pgfqpoint{6.194226in}{5.232527in}}%
\pgfpathlineto{\pgfqpoint{6.199014in}{5.207057in}}%
\pgfpathlineto{\pgfqpoint{6.203766in}{5.181588in}}%
\pgfpathlineto{\pgfqpoint{6.208491in}{5.156118in}}%
\pgfpathlineto{\pgfqpoint{6.213198in}{5.130649in}}%
\pgfpathlineto{\pgfqpoint{6.217895in}{5.105179in}}%
\pgfpathlineto{\pgfqpoint{6.218651in}{5.101090in}}%
\pgfpathlineto{\pgfqpoint{6.222542in}{5.079709in}}%
\pgfpathlineto{\pgfqpoint{6.227185in}{5.054240in}}%
\pgfpathlineto{\pgfqpoint{6.231842in}{5.028770in}}%
\pgfpathlineto{\pgfqpoint{6.236519in}{5.003301in}}%
\pgfpathlineto{\pgfqpoint{6.241222in}{4.977831in}}%
\pgfpathlineto{\pgfqpoint{6.245959in}{4.952362in}}%
\pgfpathlineto{\pgfqpoint{6.248447in}{4.939123in}}%
\pgfpathlineto{\pgfqpoint{6.250709in}{4.926892in}}%
\pgfpathlineto{\pgfqpoint{6.255475in}{4.901423in}}%
\pgfpathlineto{\pgfqpoint{6.260293in}{4.875953in}}%
\pgfpathlineto{\pgfqpoint{6.265168in}{4.850484in}}%
\pgfpathlineto{\pgfqpoint{6.270106in}{4.825014in}}%
\pgfpathlineto{\pgfqpoint{6.275112in}{4.799545in}}%
\pgfpathlineto{\pgfqpoint{6.278242in}{4.783865in}}%
\pgfpathlineto{\pgfqpoint{6.280168in}{4.774075in}}%
\pgfpathlineto{\pgfqpoint{6.285268in}{4.748606in}}%
\pgfpathlineto{\pgfqpoint{6.290450in}{4.723136in}}%
\pgfpathlineto{\pgfqpoint{6.295720in}{4.697667in}}%
\pgfpathlineto{\pgfqpoint{6.301081in}{4.672197in}}%
\pgfpathlineto{\pgfqpoint{6.306539in}{4.646728in}}%
\pgfpathlineto{\pgfqpoint{6.308038in}{4.639877in}}%
\pgfpathlineto{\pgfqpoint{6.312054in}{4.621258in}}%
\pgfpathlineto{\pgfqpoint{6.317657in}{4.595788in}}%
\pgfpathlineto{\pgfqpoint{6.323369in}{4.570319in}}%
\pgfpathlineto{\pgfqpoint{6.329193in}{4.544849in}}%
\pgfpathlineto{\pgfqpoint{6.335134in}{4.519380in}}%
\pgfpathlineto{\pgfqpoint{6.337833in}{4.508055in}}%
\pgfpathlineto{\pgfqpoint{6.341160in}{4.493910in}}%
\pgfpathlineto{\pgfqpoint{6.347282in}{4.468441in}}%
\pgfpathlineto{\pgfqpoint{6.353530in}{4.442971in}}%
\pgfpathlineto{\pgfqpoint{6.359909in}{4.417502in}}%
\pgfpathlineto{\pgfqpoint{6.366423in}{4.392032in}}%
\pgfpathlineto{\pgfqpoint{6.367628in}{4.387422in}}%
\pgfpathlineto{\pgfqpoint{6.373018in}{4.366563in}}%
\pgfpathlineto{\pgfqpoint{6.379742in}{4.341093in}}%
\pgfpathlineto{\pgfqpoint{6.386609in}{4.315624in}}%
\pgfpathlineto{\pgfqpoint{6.393624in}{4.290154in}}%
\pgfpathlineto{\pgfqpoint{6.397424in}{4.276657in}}%
\pgfpathlineto{\pgfqpoint{6.400756in}{4.264685in}}%
\pgfpathlineto{\pgfqpoint{6.408001in}{4.239215in}}%
\pgfpathlineto{\pgfqpoint{6.415402in}{4.213746in}}%
\pgfpathlineto{\pgfqpoint{6.422963in}{4.188276in}}%
\pgfpathlineto{\pgfqpoint{6.427219in}{4.174243in}}%
\pgfpathlineto{\pgfqpoint{6.430651in}{4.162807in}}%
\pgfpathlineto{\pgfqpoint{6.438460in}{4.137337in}}%
\pgfpathlineto{\pgfqpoint{6.446437in}{4.111867in}}%
\pgfpathlineto{\pgfqpoint{6.454584in}{4.086398in}}%
\pgfpathlineto{\pgfqpoint{6.457014in}{4.078961in}}%
\pgfpathlineto{\pgfqpoint{6.462847in}{4.060928in}}%
\pgfpathlineto{\pgfqpoint{6.471260in}{4.035459in}}%
\pgfpathlineto{\pgfqpoint{6.479851in}{4.009989in}}%
\pgfpathlineto{\pgfqpoint{6.486810in}{3.989786in}}%
\pgfpathlineto{\pgfqpoint{6.488606in}{3.984520in}}%
\pgfpathlineto{\pgfqpoint{6.497475in}{3.959050in}}%
\pgfpathlineto{\pgfqpoint{6.506529in}{3.933581in}}%
\pgfpathlineto{\pgfqpoint{6.515771in}{3.908111in}}%
\pgfpathlineto{\pgfqpoint{6.516605in}{3.905859in}}%
\pgfpathlineto{\pgfqpoint{6.525122in}{3.882642in}}%
\pgfpathlineto{\pgfqpoint{6.534657in}{3.857172in}}%
\pgfpathlineto{\pgfqpoint{6.544387in}{3.831703in}}%
\pgfpathlineto{\pgfqpoint{6.546401in}{3.826536in}}%
\pgfpathlineto{\pgfqpoint{6.554241in}{3.806233in}}%
\pgfpathlineto{\pgfqpoint{6.564274in}{3.780764in}}%
\pgfpathlineto{\pgfqpoint{6.574509in}{3.755294in}}%
\pgfpathlineto{\pgfqpoint{6.576196in}{3.751177in}}%
\pgfpathlineto{\pgfqpoint{6.584867in}{3.729825in}}%
\pgfpathlineto{\pgfqpoint{6.595415in}{3.704355in}}%
\pgfpathlineto{\pgfqpoint{6.605991in}{3.679314in}}%
\pgfpathlineto{\pgfqpoint{6.606171in}{3.678886in}}%
\pgfpathlineto{\pgfqpoint{6.617038in}{3.653416in}}%
\pgfpathlineto{\pgfqpoint{6.628117in}{3.627946in}}%
\pgfpathlineto{\pgfqpoint{6.635787in}{3.610646in}}%
\pgfpathlineto{\pgfqpoint{6.639379in}{3.602477in}}%
\pgfpathlineto{\pgfqpoint{6.650786in}{3.577007in}}%
\pgfpathlineto{\pgfqpoint{6.662413in}{3.551538in}}%
\pgfpathlineto{\pgfqpoint{6.665582in}{3.544720in}}%
\pgfpathlineto{\pgfqpoint{6.674183in}{3.526068in}}%
\pgfpathlineto{\pgfqpoint{6.686147in}{3.500599in}}%
\pgfpathlineto{\pgfqpoint{6.695378in}{3.481306in}}%
\pgfusepath{stroke}%
\end{pgfscope}%
\begin{pgfscope}%
\pgfpathrectangle{\pgfqpoint{0.766095in}{0.571603in}}{\pgfqpoint{5.929283in}{5.068436in}}%
\pgfusepath{clip}%
\pgfsetbuttcap%
\pgfsetroundjoin%
\pgfsetlinewidth{1.505625pt}%
\definecolor{currentstroke}{rgb}{0.166617,0.463708,0.558119}%
\pgfsetstrokecolor{currentstroke}%
\pgfsetdash{}{0pt}%
\pgfpathmoveto{\pgfqpoint{2.654877in}{0.571603in}}%
\pgfpathlineto{\pgfqpoint{2.549455in}{0.648012in}}%
\pgfpathlineto{\pgfqpoint{2.448653in}{0.724420in}}%
\pgfpathlineto{\pgfqpoint{2.352270in}{0.800829in}}%
\pgfpathlineto{\pgfqpoint{2.260212in}{0.877238in}}%
\pgfpathlineto{\pgfqpoint{2.172367in}{0.953646in}}%
\pgfpathlineto{\pgfqpoint{2.088607in}{1.030055in}}%
\pgfpathlineto{\pgfqpoint{2.008790in}{1.106463in}}%
\pgfpathlineto{\pgfqpoint{1.928115in}{1.187747in}}%
\pgfpathlineto{\pgfqpoint{1.860601in}{1.259281in}}%
\pgfpathlineto{\pgfqpoint{1.792020in}{1.335689in}}%
\pgfpathlineto{\pgfqpoint{1.726954in}{1.412098in}}%
\pgfpathlineto{\pgfqpoint{1.665326in}{1.488506in}}%
\pgfpathlineto{\pgfqpoint{1.607051in}{1.564915in}}%
\pgfpathlineto{\pgfqpoint{1.552035in}{1.641323in}}%
\pgfpathlineto{\pgfqpoint{1.500173in}{1.717732in}}%
\pgfpathlineto{\pgfqpoint{1.451356in}{1.794141in}}%
\pgfpathlineto{\pgfqpoint{1.405629in}{1.870549in}}%
\pgfpathlineto{\pgfqpoint{1.362003in}{1.948346in}}%
\pgfpathlineto{\pgfqpoint{1.322805in}{2.023366in}}%
\pgfpathlineto{\pgfqpoint{1.285625in}{2.099775in}}%
\pgfpathlineto{\pgfqpoint{1.251161in}{2.176183in}}%
\pgfpathlineto{\pgfqpoint{1.219371in}{2.252592in}}%
\pgfpathlineto{\pgfqpoint{1.190203in}{2.329001in}}%
\pgfpathlineto{\pgfqpoint{1.163604in}{2.405409in}}%
\pgfpathlineto{\pgfqpoint{1.139516in}{2.481818in}}%
\pgfpathlineto{\pgfqpoint{1.117874in}{2.558226in}}%
\pgfpathlineto{\pgfqpoint{1.098674in}{2.634635in}}%
\pgfpathlineto{\pgfqpoint{1.081887in}{2.711044in}}%
\pgfpathlineto{\pgfqpoint{1.067411in}{2.787452in}}%
\pgfpathlineto{\pgfqpoint{1.055307in}{2.863861in}}%
\pgfpathlineto{\pgfqpoint{1.045489in}{2.940269in}}%
\pgfpathlineto{\pgfqpoint{1.037917in}{3.016678in}}%
\pgfpathlineto{\pgfqpoint{1.032618in}{3.093086in}}%
\pgfpathlineto{\pgfqpoint{1.029588in}{3.169495in}}%
\pgfpathlineto{\pgfqpoint{1.028778in}{3.245904in}}%
\pgfpathlineto{\pgfqpoint{1.030184in}{3.322312in}}%
\pgfpathlineto{\pgfqpoint{1.034254in}{3.405567in}}%
\pgfpathlineto{\pgfqpoint{1.039684in}{3.475129in}}%
\pgfpathlineto{\pgfqpoint{1.047793in}{3.551538in}}%
\pgfpathlineto{\pgfqpoint{1.058105in}{3.627946in}}%
\pgfpathlineto{\pgfqpoint{1.070691in}{3.704355in}}%
\pgfpathlineto{\pgfqpoint{1.085556in}{3.780764in}}%
\pgfpathlineto{\pgfqpoint{1.102717in}{3.857172in}}%
\pgfpathlineto{\pgfqpoint{1.123640in}{3.938949in}}%
\pgfpathlineto{\pgfqpoint{1.144061in}{4.009989in}}%
\pgfpathlineto{\pgfqpoint{1.168330in}{4.086398in}}%
\pgfpathlineto{\pgfqpoint{1.195044in}{4.162807in}}%
\pgfpathlineto{\pgfqpoint{1.224269in}{4.239215in}}%
\pgfpathlineto{\pgfqpoint{1.256067in}{4.315624in}}%
\pgfpathlineto{\pgfqpoint{1.290500in}{4.392032in}}%
\pgfpathlineto{\pgfqpoint{1.327626in}{4.468441in}}%
\pgfpathlineto{\pgfqpoint{1.367582in}{4.544849in}}%
\pgfpathlineto{\pgfqpoint{1.410447in}{4.621258in}}%
\pgfpathlineto{\pgfqpoint{1.456274in}{4.697667in}}%
\pgfpathlineto{\pgfqpoint{1.505245in}{4.774075in}}%
\pgfpathlineto{\pgfqpoint{1.540775in}{4.826630in}}%
\pgfpathlineto{\pgfqpoint{1.600366in}{4.909858in}}%
\pgfpathlineto{\pgfqpoint{1.659957in}{4.987839in}}%
\pgfpathlineto{\pgfqpoint{1.719547in}{5.061242in}}%
\pgfpathlineto{\pgfqpoint{1.779149in}{5.130649in}}%
\pgfpathlineto{\pgfqpoint{1.848744in}{5.207057in}}%
\pgfpathlineto{\pgfqpoint{1.898320in}{5.258900in}}%
\pgfpathlineto{\pgfqpoint{1.974360in}{5.334405in}}%
\pgfpathlineto{\pgfqpoint{2.028238in}{5.385344in}}%
\pgfpathlineto{\pgfqpoint{2.084327in}{5.436283in}}%
\pgfpathlineto{\pgfqpoint{2.142748in}{5.487222in}}%
\pgfpathlineto{\pgfqpoint{2.226069in}{5.556425in}}%
\pgfpathlineto{\pgfqpoint{2.299808in}{5.614570in}}%
\pgfpathlineto{\pgfqpoint{2.333232in}{5.640039in}}%
\pgfpathlineto{\pgfqpoint{2.333232in}{5.640039in}}%
\pgfusepath{stroke}%
\end{pgfscope}%
\begin{pgfscope}%
\pgfpathrectangle{\pgfqpoint{0.766095in}{0.571603in}}{\pgfqpoint{5.929283in}{5.068436in}}%
\pgfusepath{clip}%
\pgfsetbuttcap%
\pgfsetroundjoin%
\pgfsetlinewidth{1.505625pt}%
\definecolor{currentstroke}{rgb}{0.166617,0.463708,0.558119}%
\pgfsetstrokecolor{currentstroke}%
\pgfsetdash{}{0pt}%
\pgfpathmoveto{\pgfqpoint{6.273720in}{5.640039in}}%
\pgfpathlineto{\pgfqpoint{6.278242in}{5.618450in}}%
\pgfpathlineto{\pgfqpoint{6.279037in}{5.614570in}}%
\pgfpathlineto{\pgfqpoint{6.284108in}{5.589100in}}%
\pgfpathlineto{\pgfqpoint{6.289007in}{5.563630in}}%
\pgfpathlineto{\pgfqpoint{6.293750in}{5.538161in}}%
\pgfpathlineto{\pgfqpoint{6.298350in}{5.512691in}}%
\pgfpathlineto{\pgfqpoint{6.302822in}{5.487222in}}%
\pgfpathlineto{\pgfqpoint{6.307177in}{5.461752in}}%
\pgfpathlineto{\pgfqpoint{6.308038in}{5.456625in}}%
\pgfpathlineto{\pgfqpoint{6.311386in}{5.436283in}}%
\pgfpathlineto{\pgfqpoint{6.315493in}{5.410813in}}%
\pgfpathlineto{\pgfqpoint{6.319522in}{5.385344in}}%
\pgfpathlineto{\pgfqpoint{6.323481in}{5.359874in}}%
\pgfpathlineto{\pgfqpoint{6.327381in}{5.334405in}}%
\pgfpathlineto{\pgfqpoint{6.331232in}{5.308935in}}%
\pgfpathlineto{\pgfqpoint{6.335043in}{5.283466in}}%
\pgfpathlineto{\pgfqpoint{6.337833in}{5.264698in}}%
\pgfpathlineto{\pgfqpoint{6.338812in}{5.257996in}}%
\pgfpathlineto{\pgfqpoint{6.342525in}{5.232527in}}%
\pgfpathlineto{\pgfqpoint{6.346224in}{5.207057in}}%
\pgfpathlineto{\pgfqpoint{6.349918in}{5.181588in}}%
\pgfpathlineto{\pgfqpoint{6.353613in}{5.156118in}}%
\pgfpathlineto{\pgfqpoint{6.357318in}{5.130649in}}%
\pgfpathlineto{\pgfqpoint{6.361037in}{5.105179in}}%
\pgfpathlineto{\pgfqpoint{6.364780in}{5.079709in}}%
\pgfpathlineto{\pgfqpoint{6.367628in}{5.060495in}}%
\pgfpathlineto{\pgfqpoint{6.368541in}{5.054240in}}%
\pgfpathlineto{\pgfqpoint{6.372305in}{5.028770in}}%
\pgfpathlineto{\pgfqpoint{6.376111in}{5.003301in}}%
\pgfpathlineto{\pgfqpoint{6.379964in}{4.977831in}}%
\pgfpathlineto{\pgfqpoint{6.383870in}{4.952362in}}%
\pgfpathlineto{\pgfqpoint{6.387833in}{4.926892in}}%
\pgfpathlineto{\pgfqpoint{6.391861in}{4.901423in}}%
\pgfpathlineto{\pgfqpoint{6.395957in}{4.875953in}}%
\pgfpathlineto{\pgfqpoint{6.397424in}{4.867016in}}%
\pgfpathlineto{\pgfqpoint{6.400098in}{4.850484in}}%
\pgfpathlineto{\pgfqpoint{6.404301in}{4.825014in}}%
\pgfpathlineto{\pgfqpoint{6.408587in}{4.799545in}}%
\pgfpathlineto{\pgfqpoint{6.412960in}{4.774075in}}%
\pgfpathlineto{\pgfqpoint{6.417425in}{4.748606in}}%
\pgfpathlineto{\pgfqpoint{6.421986in}{4.723136in}}%
\pgfpathlineto{\pgfqpoint{6.426647in}{4.697667in}}%
\pgfpathlineto{\pgfqpoint{6.427219in}{4.694618in}}%
\pgfpathlineto{\pgfqpoint{6.431370in}{4.672197in}}%
\pgfpathlineto{\pgfqpoint{6.436194in}{4.646728in}}%
\pgfpathlineto{\pgfqpoint{6.441129in}{4.621258in}}%
\pgfpathlineto{\pgfqpoint{6.446180in}{4.595788in}}%
\pgfpathlineto{\pgfqpoint{6.451351in}{4.570319in}}%
\pgfpathlineto{\pgfqpoint{6.456644in}{4.544849in}}%
\pgfpathlineto{\pgfqpoint{6.457014in}{4.543110in}}%
\pgfpathlineto{\pgfqpoint{6.462012in}{4.519380in}}%
\pgfpathlineto{\pgfqpoint{6.467506in}{4.493910in}}%
\pgfpathlineto{\pgfqpoint{6.473133in}{4.468441in}}%
\pgfpathlineto{\pgfqpoint{6.478895in}{4.442971in}}%
\pgfpathlineto{\pgfqpoint{6.484797in}{4.417502in}}%
\pgfpathlineto{\pgfqpoint{6.486810in}{4.409027in}}%
\pgfpathlineto{\pgfqpoint{6.490801in}{4.392032in}}%
\pgfpathlineto{\pgfqpoint{6.496930in}{4.366563in}}%
\pgfpathlineto{\pgfqpoint{6.503207in}{4.341093in}}%
\pgfpathlineto{\pgfqpoint{6.509634in}{4.315624in}}%
\pgfpathlineto{\pgfqpoint{6.516216in}{4.290154in}}%
\pgfpathlineto{\pgfqpoint{6.516605in}{4.288685in}}%
\pgfpathlineto{\pgfqpoint{6.522893in}{4.264685in}}%
\pgfpathlineto{\pgfqpoint{6.529726in}{4.239215in}}%
\pgfpathlineto{\pgfqpoint{6.536720in}{4.213746in}}%
\pgfpathlineto{\pgfqpoint{6.543879in}{4.188276in}}%
\pgfpathlineto{\pgfqpoint{6.546401in}{4.179516in}}%
\pgfpathlineto{\pgfqpoint{6.551160in}{4.162807in}}%
\pgfpathlineto{\pgfqpoint{6.558587in}{4.137337in}}%
\pgfpathlineto{\pgfqpoint{6.566185in}{4.111867in}}%
\pgfpathlineto{\pgfqpoint{6.573958in}{4.086398in}}%
\pgfpathlineto{\pgfqpoint{6.576196in}{4.079232in}}%
\pgfpathlineto{\pgfqpoint{6.581856in}{4.060928in}}%
\pgfpathlineto{\pgfqpoint{6.589911in}{4.035459in}}%
\pgfpathlineto{\pgfqpoint{6.598149in}{4.009989in}}%
\pgfpathlineto{\pgfqpoint{6.605991in}{3.986272in}}%
\pgfpathlineto{\pgfqpoint{6.606565in}{3.984520in}}%
\pgfpathlineto{\pgfqpoint{6.615095in}{3.959050in}}%
\pgfpathlineto{\pgfqpoint{6.623814in}{3.933581in}}%
\pgfpathlineto{\pgfqpoint{6.632723in}{3.908111in}}%
\pgfpathlineto{\pgfqpoint{6.635787in}{3.899540in}}%
\pgfpathlineto{\pgfqpoint{6.641772in}{3.882642in}}%
\pgfpathlineto{\pgfqpoint{6.650988in}{3.857172in}}%
\pgfpathlineto{\pgfqpoint{6.660401in}{3.831703in}}%
\pgfpathlineto{\pgfqpoint{6.665582in}{3.817975in}}%
\pgfpathlineto{\pgfqpoint{6.669975in}{3.806233in}}%
\pgfpathlineto{\pgfqpoint{6.679704in}{3.780764in}}%
\pgfpathlineto{\pgfqpoint{6.689637in}{3.755294in}}%
\pgfpathlineto{\pgfqpoint{6.695378in}{3.740871in}}%
\pgfusepath{stroke}%
\end{pgfscope}%
\begin{pgfscope}%
\pgfpathrectangle{\pgfqpoint{0.766095in}{0.571603in}}{\pgfqpoint{5.929283in}{5.068436in}}%
\pgfusepath{clip}%
\pgfsetbuttcap%
\pgfsetroundjoin%
\pgfsetlinewidth{1.505625pt}%
\definecolor{currentstroke}{rgb}{0.159194,0.482237,0.558073}%
\pgfsetstrokecolor{currentstroke}%
\pgfsetdash{}{0pt}%
\pgfpathmoveto{\pgfqpoint{2.562715in}{0.571603in}}%
\pgfpathlineto{\pgfqpoint{2.458267in}{0.648012in}}%
\pgfpathlineto{\pgfqpoint{2.358359in}{0.724420in}}%
\pgfpathlineto{\pgfqpoint{2.262817in}{0.800829in}}%
\pgfpathlineto{\pgfqpoint{2.171546in}{0.877238in}}%
\pgfpathlineto{\pgfqpoint{2.084431in}{0.953646in}}%
\pgfpathlineto{\pgfqpoint{2.001343in}{1.030055in}}%
\pgfpathlineto{\pgfqpoint{1.922141in}{1.106463in}}%
\pgfpathlineto{\pgfqpoint{1.846759in}{1.182872in}}%
\pgfpathlineto{\pgfqpoint{1.775048in}{1.259281in}}%
\pgfpathlineto{\pgfqpoint{1.706960in}{1.335689in}}%
\pgfpathlineto{\pgfqpoint{1.642337in}{1.412098in}}%
\pgfpathlineto{\pgfqpoint{1.581105in}{1.488506in}}%
\pgfpathlineto{\pgfqpoint{1.523176in}{1.564915in}}%
\pgfpathlineto{\pgfqpoint{1.468456in}{1.641323in}}%
\pgfpathlineto{\pgfqpoint{1.416843in}{1.717732in}}%
\pgfpathlineto{\pgfqpoint{1.368298in}{1.794141in}}%
\pgfpathlineto{\pgfqpoint{1.322724in}{1.870549in}}%
\pgfpathlineto{\pgfqpoint{1.280028in}{1.946958in}}%
\pgfpathlineto{\pgfqpoint{1.240143in}{2.023366in}}%
\pgfpathlineto{\pgfqpoint{1.203059in}{2.099775in}}%
\pgfpathlineto{\pgfqpoint{1.168649in}{2.176183in}}%
\pgfpathlineto{\pgfqpoint{1.136868in}{2.252592in}}%
\pgfpathlineto{\pgfqpoint{1.107668in}{2.329001in}}%
\pgfpathlineto{\pgfqpoint{1.080995in}{2.405409in}}%
\pgfpathlineto{\pgfqpoint{1.056790in}{2.481818in}}%
\pgfpathlineto{\pgfqpoint{1.034254in}{2.561051in}}%
\pgfpathlineto{\pgfqpoint{1.015676in}{2.634635in}}%
\pgfpathlineto{\pgfqpoint{0.998652in}{2.711044in}}%
\pgfpathlineto{\pgfqpoint{0.983976in}{2.787452in}}%
\pgfpathlineto{\pgfqpoint{0.971558in}{2.863861in}}%
\pgfpathlineto{\pgfqpoint{0.961457in}{2.940269in}}%
\pgfpathlineto{\pgfqpoint{0.953561in}{3.016678in}}%
\pgfpathlineto{\pgfqpoint{0.947873in}{3.093086in}}%
\pgfpathlineto{\pgfqpoint{0.944400in}{3.169495in}}%
\pgfpathlineto{\pgfqpoint{0.943144in}{3.245904in}}%
\pgfpathlineto{\pgfqpoint{0.944063in}{3.322312in}}%
\pgfpathlineto{\pgfqpoint{0.947180in}{3.398721in}}%
\pgfpathlineto{\pgfqpoint{0.952495in}{3.475129in}}%
\pgfpathlineto{\pgfqpoint{0.959988in}{3.551538in}}%
\pgfpathlineto{\pgfqpoint{0.969644in}{3.627946in}}%
\pgfpathlineto{\pgfqpoint{0.981539in}{3.704355in}}%
\pgfpathlineto{\pgfqpoint{0.995662in}{3.780764in}}%
\pgfpathlineto{\pgfqpoint{1.012025in}{3.857172in}}%
\pgfpathlineto{\pgfqpoint{1.034254in}{3.947305in}}%
\pgfpathlineto{\pgfqpoint{1.051637in}{4.009989in}}%
\pgfpathlineto{\pgfqpoint{1.074940in}{4.086398in}}%
\pgfpathlineto{\pgfqpoint{1.100626in}{4.162807in}}%
\pgfpathlineto{\pgfqpoint{1.128760in}{4.239215in}}%
\pgfpathlineto{\pgfqpoint{1.159404in}{4.315624in}}%
\pgfpathlineto{\pgfqpoint{1.192620in}{4.392032in}}%
\pgfpathlineto{\pgfqpoint{1.228464in}{4.468441in}}%
\pgfpathlineto{\pgfqpoint{1.266992in}{4.544849in}}%
\pgfpathlineto{\pgfqpoint{1.308341in}{4.621258in}}%
\pgfpathlineto{\pgfqpoint{1.352596in}{4.697667in}}%
\pgfpathlineto{\pgfqpoint{1.399836in}{4.774075in}}%
\pgfpathlineto{\pgfqpoint{1.451389in}{4.852280in}}%
\pgfpathlineto{\pgfqpoint{1.510980in}{4.936768in}}%
\pgfpathlineto{\pgfqpoint{1.560819in}{5.003301in}}%
\pgfpathlineto{\pgfqpoint{1.600736in}{5.054240in}}%
\pgfpathlineto{\pgfqpoint{1.663774in}{5.130649in}}%
\pgfpathlineto{\pgfqpoint{1.730670in}{5.207057in}}%
\pgfpathlineto{\pgfqpoint{1.779138in}{5.259795in}}%
\pgfpathlineto{\pgfqpoint{1.851261in}{5.334405in}}%
\pgfpathlineto{\pgfqpoint{1.902878in}{5.385344in}}%
\pgfpathlineto{\pgfqpoint{1.957911in}{5.437531in}}%
\pgfpathlineto{\pgfqpoint{2.041238in}{5.512691in}}%
\pgfpathlineto{\pgfqpoint{2.106888in}{5.568975in}}%
\pgfpathlineto{\pgfqpoint{2.194015in}{5.640039in}}%
\pgfpathlineto{\pgfqpoint{2.194015in}{5.640039in}}%
\pgfusepath{stroke}%
\end{pgfscope}%
\begin{pgfscope}%
\pgfpathrectangle{\pgfqpoint{0.766095in}{0.571603in}}{\pgfqpoint{5.929283in}{5.068436in}}%
\pgfusepath{clip}%
\pgfsetbuttcap%
\pgfsetroundjoin%
\pgfsetlinewidth{1.505625pt}%
\definecolor{currentstroke}{rgb}{0.159194,0.482237,0.558073}%
\pgfsetstrokecolor{currentstroke}%
\pgfsetdash{}{0pt}%
\pgfpathmoveto{\pgfqpoint{6.433401in}{5.640039in}}%
\pgfpathlineto{\pgfqpoint{6.437102in}{5.614570in}}%
\pgfpathlineto{\pgfqpoint{6.440669in}{5.589100in}}%
\pgfpathlineto{\pgfqpoint{6.444116in}{5.563630in}}%
\pgfpathlineto{\pgfqpoint{6.447455in}{5.538161in}}%
\pgfpathlineto{\pgfqpoint{6.450697in}{5.512691in}}%
\pgfpathlineto{\pgfqpoint{6.453854in}{5.487222in}}%
\pgfpathlineto{\pgfqpoint{6.456937in}{5.461752in}}%
\pgfpathlineto{\pgfqpoint{6.457014in}{5.461102in}}%
\pgfpathlineto{\pgfqpoint{6.459921in}{5.436283in}}%
\pgfpathlineto{\pgfqpoint{6.462851in}{5.410813in}}%
\pgfpathlineto{\pgfqpoint{6.465736in}{5.385344in}}%
\pgfpathlineto{\pgfqpoint{6.468586in}{5.359874in}}%
\pgfpathlineto{\pgfqpoint{6.471410in}{5.334405in}}%
\pgfpathlineto{\pgfqpoint{6.474214in}{5.308935in}}%
\pgfpathlineto{\pgfqpoint{6.477008in}{5.283466in}}%
\pgfpathlineto{\pgfqpoint{6.479799in}{5.257996in}}%
\pgfpathlineto{\pgfqpoint{6.482594in}{5.232527in}}%
\pgfpathlineto{\pgfqpoint{6.485401in}{5.207057in}}%
\pgfpathlineto{\pgfqpoint{6.486810in}{5.194380in}}%
\pgfpathlineto{\pgfqpoint{6.488209in}{5.181588in}}%
\pgfpathlineto{\pgfqpoint{6.491027in}{5.156118in}}%
\pgfpathlineto{\pgfqpoint{6.493876in}{5.130649in}}%
\pgfpathlineto{\pgfqpoint{6.496761in}{5.105179in}}%
\pgfpathlineto{\pgfqpoint{6.499690in}{5.079709in}}%
\pgfpathlineto{\pgfqpoint{6.502666in}{5.054240in}}%
\pgfpathlineto{\pgfqpoint{6.505696in}{5.028770in}}%
\pgfpathlineto{\pgfqpoint{6.508785in}{5.003301in}}%
\pgfpathlineto{\pgfqpoint{6.511939in}{4.977831in}}%
\pgfpathlineto{\pgfqpoint{6.515161in}{4.952362in}}%
\pgfpathlineto{\pgfqpoint{6.516605in}{4.941230in}}%
\pgfpathlineto{\pgfqpoint{6.518439in}{4.926892in}}%
\pgfpathlineto{\pgfqpoint{6.521780in}{4.901423in}}%
\pgfpathlineto{\pgfqpoint{6.525203in}{4.875953in}}%
\pgfpathlineto{\pgfqpoint{6.528713in}{4.850484in}}%
\pgfpathlineto{\pgfqpoint{6.532315in}{4.825014in}}%
\pgfpathlineto{\pgfqpoint{6.536013in}{4.799545in}}%
\pgfpathlineto{\pgfqpoint{6.539810in}{4.774075in}}%
\pgfpathlineto{\pgfqpoint{6.543710in}{4.748606in}}%
\pgfpathlineto{\pgfqpoint{6.546401in}{4.731528in}}%
\pgfpathlineto{\pgfqpoint{6.547706in}{4.723136in}}%
\pgfpathlineto{\pgfqpoint{6.551785in}{4.697667in}}%
\pgfpathlineto{\pgfqpoint{6.555978in}{4.672197in}}%
\pgfpathlineto{\pgfqpoint{6.560288in}{4.646728in}}%
\pgfpathlineto{\pgfqpoint{6.564721in}{4.621258in}}%
\pgfpathlineto{\pgfqpoint{6.569277in}{4.595788in}}%
\pgfpathlineto{\pgfqpoint{6.573963in}{4.570319in}}%
\pgfpathlineto{\pgfqpoint{6.576196in}{4.558523in}}%
\pgfpathlineto{\pgfqpoint{6.578754in}{4.544849in}}%
\pgfpathlineto{\pgfqpoint{6.583658in}{4.519380in}}%
\pgfpathlineto{\pgfqpoint{6.588699in}{4.493910in}}%
\pgfpathlineto{\pgfqpoint{6.593880in}{4.468441in}}%
\pgfpathlineto{\pgfqpoint{6.599205in}{4.442971in}}%
\pgfpathlineto{\pgfqpoint{6.604675in}{4.417502in}}%
\pgfpathlineto{\pgfqpoint{6.605991in}{4.411544in}}%
\pgfpathlineto{\pgfqpoint{6.610255in}{4.392032in}}%
\pgfpathlineto{\pgfqpoint{6.615973in}{4.366563in}}%
\pgfpathlineto{\pgfqpoint{6.621845in}{4.341093in}}%
\pgfpathlineto{\pgfqpoint{6.627874in}{4.315624in}}%
\pgfpathlineto{\pgfqpoint{6.634063in}{4.290154in}}%
\pgfpathlineto{\pgfqpoint{6.635787in}{4.283246in}}%
\pgfpathlineto{\pgfqpoint{6.640371in}{4.264685in}}%
\pgfpathlineto{\pgfqpoint{6.646828in}{4.239215in}}%
\pgfpathlineto{\pgfqpoint{6.653452in}{4.213746in}}%
\pgfpathlineto{\pgfqpoint{6.660245in}{4.188276in}}%
\pgfpathlineto{\pgfqpoint{6.665582in}{4.168765in}}%
\pgfpathlineto{\pgfqpoint{6.667196in}{4.162807in}}%
\pgfpathlineto{\pgfqpoint{6.674272in}{4.137337in}}%
\pgfpathlineto{\pgfqpoint{6.681524in}{4.111867in}}%
\pgfpathlineto{\pgfqpoint{6.688955in}{4.086398in}}%
\pgfpathlineto{\pgfqpoint{6.695378in}{4.064910in}}%
\pgfusepath{stroke}%
\end{pgfscope}%
\begin{pgfscope}%
\pgfpathrectangle{\pgfqpoint{0.766095in}{0.571603in}}{\pgfqpoint{5.929283in}{5.068436in}}%
\pgfusepath{clip}%
\pgfsetbuttcap%
\pgfsetroundjoin%
\pgfsetlinewidth{1.505625pt}%
\definecolor{currentstroke}{rgb}{0.150476,0.504369,0.557430}%
\pgfsetstrokecolor{currentstroke}%
\pgfsetdash{}{0pt}%
\pgfpathmoveto{\pgfqpoint{2.473513in}{0.571603in}}%
\pgfpathlineto{\pgfqpoint{2.369976in}{0.648012in}}%
\pgfpathlineto{\pgfqpoint{2.270930in}{0.724420in}}%
\pgfpathlineto{\pgfqpoint{2.176197in}{0.800829in}}%
\pgfpathlineto{\pgfqpoint{2.085678in}{0.877238in}}%
\pgfpathlineto{\pgfqpoint{1.999259in}{0.953646in}}%
\pgfpathlineto{\pgfqpoint{1.916809in}{1.030055in}}%
\pgfpathlineto{\pgfqpoint{1.838187in}{1.106463in}}%
\pgfpathlineto{\pgfqpoint{1.763389in}{1.182872in}}%
\pgfpathlineto{\pgfqpoint{1.689752in}{1.261960in}}%
\pgfpathlineto{\pgfqpoint{1.624541in}{1.335689in}}%
\pgfpathlineto{\pgfqpoint{1.560354in}{1.412098in}}%
\pgfpathlineto{\pgfqpoint{1.499507in}{1.488506in}}%
\pgfpathlineto{\pgfqpoint{1.441914in}{1.564915in}}%
\pgfpathlineto{\pgfqpoint{1.387481in}{1.641323in}}%
\pgfpathlineto{\pgfqpoint{1.332207in}{1.723818in}}%
\pgfpathlineto{\pgfqpoint{1.287849in}{1.794141in}}%
\pgfpathlineto{\pgfqpoint{1.242416in}{1.870549in}}%
\pgfpathlineto{\pgfqpoint{1.199921in}{1.946958in}}%
\pgfpathlineto{\pgfqpoint{1.160162in}{2.023366in}}%
\pgfpathlineto{\pgfqpoint{1.123111in}{2.099775in}}%
\pgfpathlineto{\pgfqpoint{1.088770in}{2.176183in}}%
\pgfpathlineto{\pgfqpoint{1.057015in}{2.252592in}}%
\pgfpathlineto{\pgfqpoint{1.027799in}{2.329001in}}%
\pgfpathlineto{\pgfqpoint{1.001067in}{2.405409in}}%
\pgfpathlineto{\pgfqpoint{0.974663in}{2.488954in}}%
\pgfpathlineto{\pgfqpoint{0.954947in}{2.558226in}}%
\pgfpathlineto{\pgfqpoint{0.935438in}{2.634635in}}%
\pgfpathlineto{\pgfqpoint{0.918232in}{2.711044in}}%
\pgfpathlineto{\pgfqpoint{0.903365in}{2.787452in}}%
\pgfpathlineto{\pgfqpoint{0.890715in}{2.863861in}}%
\pgfpathlineto{\pgfqpoint{0.880313in}{2.940269in}}%
\pgfpathlineto{\pgfqpoint{0.872142in}{3.016678in}}%
\pgfpathlineto{\pgfqpoint{0.866137in}{3.093086in}}%
\pgfpathlineto{\pgfqpoint{0.862300in}{3.169495in}}%
\pgfpathlineto{\pgfqpoint{0.860628in}{3.245904in}}%
\pgfpathlineto{\pgfqpoint{0.861117in}{3.322312in}}%
\pgfpathlineto{\pgfqpoint{0.863761in}{3.398721in}}%
\pgfpathlineto{\pgfqpoint{0.868553in}{3.475129in}}%
\pgfpathlineto{\pgfqpoint{0.875482in}{3.551538in}}%
\pgfpathlineto{\pgfqpoint{0.885277in}{3.633358in}}%
\pgfpathlineto{\pgfqpoint{0.895838in}{3.704355in}}%
\pgfpathlineto{\pgfqpoint{0.909272in}{3.780764in}}%
\pgfpathlineto{\pgfqpoint{0.924939in}{3.857172in}}%
\pgfpathlineto{\pgfqpoint{0.944867in}{3.941809in}}%
\pgfpathlineto{\pgfqpoint{0.962977in}{4.009989in}}%
\pgfpathlineto{\pgfqpoint{0.985422in}{4.086398in}}%
\pgfpathlineto{\pgfqpoint{1.010192in}{4.162807in}}%
\pgfpathlineto{\pgfqpoint{1.037349in}{4.239215in}}%
\pgfpathlineto{\pgfqpoint{1.066955in}{4.315624in}}%
\pgfpathlineto{\pgfqpoint{1.099069in}{4.392032in}}%
\pgfpathlineto{\pgfqpoint{1.133751in}{4.468441in}}%
\pgfpathlineto{\pgfqpoint{1.171055in}{4.544849in}}%
\pgfpathlineto{\pgfqpoint{1.213026in}{4.624922in}}%
\pgfpathlineto{\pgfqpoint{1.253888in}{4.697667in}}%
\pgfpathlineto{\pgfqpoint{1.302412in}{4.778645in}}%
\pgfpathlineto{\pgfqpoint{1.348346in}{4.850484in}}%
\pgfpathlineto{\pgfqpoint{1.400197in}{4.926892in}}%
\pgfpathlineto{\pgfqpoint{1.455298in}{5.003301in}}%
\pgfpathlineto{\pgfqpoint{1.513816in}{5.079709in}}%
\pgfpathlineto{\pgfqpoint{1.575913in}{5.156118in}}%
\pgfpathlineto{\pgfqpoint{1.641754in}{5.232527in}}%
\pgfpathlineto{\pgfqpoint{1.689752in}{5.285617in}}%
\pgfpathlineto{\pgfqpoint{1.760294in}{5.359874in}}%
\pgfpathlineto{\pgfqpoint{1.810956in}{5.410813in}}%
\pgfpathlineto{\pgfqpoint{1.890786in}{5.487222in}}%
\pgfpathlineto{\pgfqpoint{1.957911in}{5.548231in}}%
\pgfpathlineto{\pgfqpoint{2.034512in}{5.614570in}}%
\pgfpathlineto{\pgfqpoint{2.064967in}{5.640039in}}%
\pgfpathlineto{\pgfqpoint{2.064967in}{5.640039in}}%
\pgfusepath{stroke}%
\end{pgfscope}%
\begin{pgfscope}%
\pgfpathrectangle{\pgfqpoint{0.766095in}{0.571603in}}{\pgfqpoint{5.929283in}{5.068436in}}%
\pgfusepath{clip}%
\pgfsetbuttcap%
\pgfsetroundjoin%
\pgfsetlinewidth{1.505625pt}%
\definecolor{currentstroke}{rgb}{0.150476,0.504369,0.557430}%
\pgfsetstrokecolor{currentstroke}%
\pgfsetdash{}{0pt}%
\pgfpathmoveto{\pgfqpoint{6.582991in}{5.640039in}}%
\pgfpathlineto{\pgfqpoint{6.585391in}{5.614570in}}%
\pgfpathlineto{\pgfqpoint{6.587701in}{5.589100in}}%
\pgfpathlineto{\pgfqpoint{6.589931in}{5.563630in}}%
\pgfpathlineto{\pgfqpoint{6.592091in}{5.538161in}}%
\pgfpathlineto{\pgfqpoint{6.594191in}{5.512691in}}%
\pgfpathlineto{\pgfqpoint{6.596242in}{5.487222in}}%
\pgfpathlineto{\pgfqpoint{6.598251in}{5.461752in}}%
\pgfpathlineto{\pgfqpoint{6.600227in}{5.436283in}}%
\pgfpathlineto{\pgfqpoint{6.602180in}{5.410813in}}%
\pgfpathlineto{\pgfqpoint{6.604116in}{5.385344in}}%
\pgfpathlineto{\pgfqpoint{6.605991in}{5.360575in}}%
\pgfpathlineto{\pgfqpoint{6.606044in}{5.359874in}}%
\pgfpathlineto{\pgfqpoint{6.607950in}{5.334405in}}%
\pgfpathlineto{\pgfqpoint{6.609862in}{5.308935in}}%
\pgfpathlineto{\pgfqpoint{6.611788in}{5.283466in}}%
\pgfpathlineto{\pgfqpoint{6.613733in}{5.257996in}}%
\pgfpathlineto{\pgfqpoint{6.615704in}{5.232527in}}%
\pgfpathlineto{\pgfqpoint{6.617707in}{5.207057in}}%
\pgfpathlineto{\pgfqpoint{6.619748in}{5.181588in}}%
\pgfpathlineto{\pgfqpoint{6.621832in}{5.156118in}}%
\pgfpathlineto{\pgfqpoint{6.623966in}{5.130649in}}%
\pgfpathlineto{\pgfqpoint{6.626154in}{5.105179in}}%
\pgfpathlineto{\pgfqpoint{6.628401in}{5.079709in}}%
\pgfpathlineto{\pgfqpoint{6.630713in}{5.054240in}}%
\pgfpathlineto{\pgfqpoint{6.633095in}{5.028770in}}%
\pgfpathlineto{\pgfqpoint{6.635550in}{5.003301in}}%
\pgfpathlineto{\pgfqpoint{6.635787in}{5.000929in}}%
\pgfpathlineto{\pgfqpoint{6.638061in}{4.977831in}}%
\pgfpathlineto{\pgfqpoint{6.640653in}{4.952362in}}%
\pgfpathlineto{\pgfqpoint{6.643331in}{4.926892in}}%
\pgfpathlineto{\pgfqpoint{6.646100in}{4.901423in}}%
\pgfpathlineto{\pgfqpoint{6.648964in}{4.875953in}}%
\pgfpathlineto{\pgfqpoint{6.651927in}{4.850484in}}%
\pgfpathlineto{\pgfqpoint{6.654993in}{4.825014in}}%
\pgfpathlineto{\pgfqpoint{6.658165in}{4.799545in}}%
\pgfpathlineto{\pgfqpoint{6.661447in}{4.774075in}}%
\pgfpathlineto{\pgfqpoint{6.664844in}{4.748606in}}%
\pgfpathlineto{\pgfqpoint{6.665582in}{4.743264in}}%
\pgfpathlineto{\pgfqpoint{6.668331in}{4.723136in}}%
\pgfpathlineto{\pgfqpoint{6.671931in}{4.697667in}}%
\pgfpathlineto{\pgfqpoint{6.675654in}{4.672197in}}%
\pgfpathlineto{\pgfqpoint{6.679504in}{4.646728in}}%
\pgfpathlineto{\pgfqpoint{6.683484in}{4.621258in}}%
\pgfpathlineto{\pgfqpoint{6.687596in}{4.595788in}}%
\pgfpathlineto{\pgfqpoint{6.691844in}{4.570319in}}%
\pgfpathlineto{\pgfqpoint{6.695378in}{4.549813in}}%
\pgfusepath{stroke}%
\end{pgfscope}%
\begin{pgfscope}%
\pgfpathrectangle{\pgfqpoint{0.766095in}{0.571603in}}{\pgfqpoint{5.929283in}{5.068436in}}%
\pgfusepath{clip}%
\pgfsetbuttcap%
\pgfsetroundjoin%
\pgfsetlinewidth{1.505625pt}%
\definecolor{currentstroke}{rgb}{0.143343,0.522773,0.556295}%
\pgfsetstrokecolor{currentstroke}%
\pgfsetdash{}{0pt}%
\pgfpathmoveto{\pgfqpoint{2.387035in}{0.571603in}}%
\pgfpathlineto{\pgfqpoint{2.284352in}{0.648012in}}%
\pgfpathlineto{\pgfqpoint{2.186137in}{0.724420in}}%
\pgfpathlineto{\pgfqpoint{2.092181in}{0.800829in}}%
\pgfpathlineto{\pgfqpoint{2.002383in}{0.877238in}}%
\pgfpathlineto{\pgfqpoint{1.916627in}{0.953646in}}%
\pgfpathlineto{\pgfqpoint{1.834783in}{1.030055in}}%
\pgfpathlineto{\pgfqpoint{1.756786in}{1.106463in}}%
\pgfpathlineto{\pgfqpoint{1.682485in}{1.182872in}}%
\pgfpathlineto{\pgfqpoint{1.611798in}{1.259281in}}%
\pgfpathlineto{\pgfqpoint{1.540775in}{1.340161in}}%
\pgfpathlineto{\pgfqpoint{1.480782in}{1.412098in}}%
\pgfpathlineto{\pgfqpoint{1.420311in}{1.488506in}}%
\pgfpathlineto{\pgfqpoint{1.362003in}{1.566368in}}%
\pgfpathlineto{\pgfqpoint{1.302412in}{1.650899in}}%
\pgfpathlineto{\pgfqpoint{1.257903in}{1.717732in}}%
\pgfpathlineto{\pgfqpoint{1.209782in}{1.794141in}}%
\pgfpathlineto{\pgfqpoint{1.164606in}{1.870549in}}%
\pgfpathlineto{\pgfqpoint{1.122198in}{1.946958in}}%
\pgfpathlineto{\pgfqpoint{1.082611in}{2.023366in}}%
\pgfpathlineto{\pgfqpoint{1.045682in}{2.099775in}}%
\pgfpathlineto{\pgfqpoint{1.011371in}{2.176183in}}%
\pgfpathlineto{\pgfqpoint{0.979633in}{2.252592in}}%
\pgfpathlineto{\pgfqpoint{0.950421in}{2.329001in}}%
\pgfpathlineto{\pgfqpoint{0.923681in}{2.405409in}}%
\pgfpathlineto{\pgfqpoint{0.899355in}{2.481818in}}%
\pgfpathlineto{\pgfqpoint{0.877386in}{2.558226in}}%
\pgfpathlineto{\pgfqpoint{0.855481in}{2.644110in}}%
\pgfpathlineto{\pgfqpoint{0.840433in}{2.711044in}}%
\pgfpathlineto{\pgfqpoint{0.825317in}{2.787452in}}%
\pgfpathlineto{\pgfqpoint{0.812516in}{2.863861in}}%
\pgfpathlineto{\pgfqpoint{0.801860in}{2.940269in}}%
\pgfpathlineto{\pgfqpoint{0.793383in}{3.016678in}}%
\pgfpathlineto{\pgfqpoint{0.787106in}{3.093086in}}%
\pgfpathlineto{\pgfqpoint{0.782955in}{3.169495in}}%
\pgfpathlineto{\pgfqpoint{0.780927in}{3.245904in}}%
\pgfpathlineto{\pgfqpoint{0.781020in}{3.322312in}}%
\pgfpathlineto{\pgfqpoint{0.783228in}{3.398721in}}%
\pgfpathlineto{\pgfqpoint{0.787543in}{3.475129in}}%
\pgfpathlineto{\pgfqpoint{0.793956in}{3.551538in}}%
\pgfpathlineto{\pgfqpoint{0.802539in}{3.627946in}}%
\pgfpathlineto{\pgfqpoint{0.813247in}{3.704355in}}%
\pgfpathlineto{\pgfqpoint{0.826046in}{3.780764in}}%
\pgfpathlineto{\pgfqpoint{0.841101in}{3.857172in}}%
\pgfpathlineto{\pgfqpoint{0.858269in}{3.933581in}}%
\pgfpathlineto{\pgfqpoint{0.877707in}{4.009989in}}%
\pgfpathlineto{\pgfqpoint{0.899388in}{4.086398in}}%
\pgfpathlineto{\pgfqpoint{0.923338in}{4.162807in}}%
\pgfpathlineto{\pgfqpoint{0.949617in}{4.239215in}}%
\pgfpathlineto{\pgfqpoint{0.978286in}{4.315624in}}%
\pgfpathlineto{\pgfqpoint{1.009404in}{4.392032in}}%
\pgfpathlineto{\pgfqpoint{1.043029in}{4.468441in}}%
\pgfpathlineto{\pgfqpoint{1.079215in}{4.544849in}}%
\pgfpathlineto{\pgfqpoint{1.118016in}{4.621258in}}%
\pgfpathlineto{\pgfqpoint{1.159564in}{4.697667in}}%
\pgfpathlineto{\pgfqpoint{1.203936in}{4.774075in}}%
\pgfpathlineto{\pgfqpoint{1.251212in}{4.850484in}}%
\pgfpathlineto{\pgfqpoint{1.302412in}{4.928261in}}%
\pgfpathlineto{\pgfqpoint{1.362003in}{5.013030in}}%
\pgfpathlineto{\pgfqpoint{1.411703in}{5.079709in}}%
\pgfpathlineto{\pgfqpoint{1.451389in}{5.130673in}}%
\pgfpathlineto{\pgfqpoint{1.513913in}{5.207057in}}%
\pgfpathlineto{\pgfqpoint{1.580163in}{5.283466in}}%
\pgfpathlineto{\pgfqpoint{1.630161in}{5.338420in}}%
\pgfpathlineto{\pgfqpoint{1.699271in}{5.410813in}}%
\pgfpathlineto{\pgfqpoint{1.750114in}{5.461752in}}%
\pgfpathlineto{\pgfqpoint{1.830174in}{5.538161in}}%
\pgfpathlineto{\pgfqpoint{1.898320in}{5.599971in}}%
\pgfpathlineto{\pgfqpoint{1.944195in}{5.640039in}}%
\pgfpathlineto{\pgfqpoint{1.944195in}{5.640039in}}%
\pgfusepath{stroke}%
\end{pgfscope}%
\begin{pgfscope}%
\pgfpathrectangle{\pgfqpoint{0.766095in}{0.571603in}}{\pgfqpoint{5.929283in}{5.068436in}}%
\pgfusepath{clip}%
\pgfsetbuttcap%
\pgfsetroundjoin%
\pgfsetlinewidth{1.505625pt}%
\definecolor{currentstroke}{rgb}{0.135066,0.544853,0.554029}%
\pgfsetstrokecolor{currentstroke}%
\pgfsetdash{}{0pt}%
\pgfpathmoveto{\pgfqpoint{2.303068in}{0.571603in}}%
\pgfpathlineto{\pgfqpoint{2.196274in}{0.651839in}}%
\pgfpathlineto{\pgfqpoint{2.103773in}{0.724420in}}%
\pgfpathlineto{\pgfqpoint{2.010564in}{0.800829in}}%
\pgfpathlineto{\pgfqpoint{1.921457in}{0.877238in}}%
\pgfpathlineto{\pgfqpoint{1.836333in}{0.953646in}}%
\pgfpathlineto{\pgfqpoint{1.749343in}{1.035653in}}%
\pgfpathlineto{\pgfqpoint{1.677687in}{1.106463in}}%
\pgfpathlineto{\pgfqpoint{1.600366in}{1.186626in}}%
\pgfpathlineto{\pgfqpoint{1.533678in}{1.259281in}}%
\pgfpathlineto{\pgfqpoint{1.466936in}{1.335689in}}%
\pgfpathlineto{\pgfqpoint{1.403549in}{1.412098in}}%
\pgfpathlineto{\pgfqpoint{1.343438in}{1.488506in}}%
\pgfpathlineto{\pgfqpoint{1.286515in}{1.564915in}}%
\pgfpathlineto{\pgfqpoint{1.232686in}{1.641323in}}%
\pgfpathlineto{\pgfqpoint{1.181848in}{1.717732in}}%
\pgfpathlineto{\pgfqpoint{1.134010in}{1.794141in}}%
\pgfpathlineto{\pgfqpoint{1.088990in}{1.870549in}}%
\pgfpathlineto{\pgfqpoint{1.046795in}{1.946958in}}%
\pgfpathlineto{\pgfqpoint{1.004458in}{2.029064in}}%
\pgfpathlineto{\pgfqpoint{0.970457in}{2.099775in}}%
\pgfpathlineto{\pgfqpoint{0.936242in}{2.176183in}}%
\pgfpathlineto{\pgfqpoint{0.904557in}{2.252592in}}%
\pgfpathlineto{\pgfqpoint{0.875353in}{2.329001in}}%
\pgfpathlineto{\pgfqpoint{0.848579in}{2.405409in}}%
\pgfpathlineto{\pgfqpoint{0.824178in}{2.481818in}}%
\pgfpathlineto{\pgfqpoint{0.802165in}{2.558226in}}%
\pgfpathlineto{\pgfqpoint{0.782450in}{2.634635in}}%
\pgfpathlineto{\pgfqpoint{0.766095in}{2.705781in}}%
\pgfpathlineto{\pgfqpoint{0.766095in}{2.705781in}}%
\pgfusepath{stroke}%
\end{pgfscope}%
\begin{pgfscope}%
\pgfpathrectangle{\pgfqpoint{0.766095in}{0.571603in}}{\pgfqpoint{5.929283in}{5.068436in}}%
\pgfusepath{clip}%
\pgfsetbuttcap%
\pgfsetroundjoin%
\pgfsetlinewidth{1.505625pt}%
\definecolor{currentstroke}{rgb}{0.135066,0.544853,0.554029}%
\pgfsetstrokecolor{currentstroke}%
\pgfsetdash{}{0pt}%
\pgfpathmoveto{\pgfqpoint{0.766095in}{3.885621in}}%
\pgfpathlineto{\pgfqpoint{0.771000in}{3.908111in}}%
\pgfpathlineto{\pgfqpoint{0.776783in}{3.933581in}}%
\pgfpathlineto{\pgfqpoint{0.782793in}{3.959050in}}%
\pgfpathlineto{\pgfqpoint{0.789031in}{3.984520in}}%
\pgfpathlineto{\pgfqpoint{0.795494in}{4.009989in}}%
\pgfpathlineto{\pgfqpoint{0.795890in}{4.011498in}}%
\pgfpathlineto{\pgfqpoint{0.802264in}{4.035459in}}%
\pgfpathlineto{\pgfqpoint{0.809266in}{4.060928in}}%
\pgfpathlineto{\pgfqpoint{0.816495in}{4.086398in}}%
\pgfpathlineto{\pgfqpoint{0.823949in}{4.111867in}}%
\pgfpathlineto{\pgfqpoint{0.825686in}{4.117628in}}%
\pgfpathlineto{\pgfqpoint{0.831704in}{4.137337in}}%
\pgfpathlineto{\pgfqpoint{0.839709in}{4.162807in}}%
\pgfpathlineto{\pgfqpoint{0.847940in}{4.188276in}}%
\pgfpathlineto{\pgfqpoint{0.855481in}{4.210993in}}%
\pgfpathlineto{\pgfqpoint{0.856406in}{4.213746in}}%
\pgfpathlineto{\pgfqpoint{0.865198in}{4.239215in}}%
\pgfpathlineto{\pgfqpoint{0.874215in}{4.264685in}}%
\pgfpathlineto{\pgfqpoint{0.883455in}{4.290154in}}%
\pgfpathlineto{\pgfqpoint{0.885277in}{4.295052in}}%
\pgfpathlineto{\pgfqpoint{0.893020in}{4.315624in}}%
\pgfpathlineto{\pgfqpoint{0.902832in}{4.341093in}}%
\pgfpathlineto{\pgfqpoint{0.912868in}{4.366563in}}%
\pgfpathlineto{\pgfqpoint{0.915072in}{4.372029in}}%
\pgfpathlineto{\pgfqpoint{0.923233in}{4.392032in}}%
\pgfpathlineto{\pgfqpoint{0.933850in}{4.417502in}}%
\pgfpathlineto{\pgfqpoint{0.944690in}{4.442971in}}%
\pgfpathlineto{\pgfqpoint{0.944867in}{4.443380in}}%
\pgfpathlineto{\pgfqpoint{0.955894in}{4.468441in}}%
\pgfpathlineto{\pgfqpoint{0.967324in}{4.493910in}}%
\pgfpathlineto{\pgfqpoint{0.974663in}{4.509946in}}%
\pgfpathlineto{\pgfqpoint{0.979031in}{4.519380in}}%
\pgfpathlineto{\pgfqpoint{0.991059in}{4.544849in}}%
\pgfpathlineto{\pgfqpoint{1.003307in}{4.570319in}}%
\pgfpathlineto{\pgfqpoint{1.004458in}{4.572667in}}%
\pgfpathlineto{\pgfqpoint{1.015925in}{4.595788in}}%
\pgfpathlineto{\pgfqpoint{1.028779in}{4.621258in}}%
\pgfpathlineto{\pgfqpoint{1.034254in}{4.631913in}}%
\pgfpathlineto{\pgfqpoint{1.041953in}{4.646728in}}%
\pgfpathlineto{\pgfqpoint{1.055420in}{4.672197in}}%
\pgfpathlineto{\pgfqpoint{1.064049in}{4.688246in}}%
\pgfpathlineto{\pgfqpoint{1.069172in}{4.697667in}}%
\pgfpathlineto{\pgfqpoint{1.083259in}{4.723136in}}%
\pgfpathlineto{\pgfqpoint{1.093844in}{4.741974in}}%
\pgfpathlineto{\pgfqpoint{1.097613in}{4.748606in}}%
\pgfpathlineto{\pgfqpoint{1.112327in}{4.774075in}}%
\pgfpathlineto{\pgfqpoint{1.123640in}{4.793365in}}%
\pgfpathlineto{\pgfqpoint{1.127305in}{4.799545in}}%
\pgfpathlineto{\pgfqpoint{1.142652in}{4.825014in}}%
\pgfpathlineto{\pgfqpoint{1.153435in}{4.842651in}}%
\pgfpathlineto{\pgfqpoint{1.158277in}{4.850484in}}%
\pgfpathlineto{\pgfqpoint{1.174263in}{4.875953in}}%
\pgfpathlineto{\pgfqpoint{1.183230in}{4.890037in}}%
\pgfpathlineto{\pgfqpoint{1.190559in}{4.901423in}}%
\pgfpathlineto{\pgfqpoint{1.207190in}{4.926892in}}%
\pgfpathlineto{\pgfqpoint{1.213026in}{4.935704in}}%
\pgfpathlineto{\pgfqpoint{1.224180in}{4.952362in}}%
\pgfpathlineto{\pgfqpoint{1.241458in}{4.977831in}}%
\pgfpathlineto{\pgfqpoint{1.242821in}{4.979811in}}%
\pgfpathlineto{\pgfqpoint{1.259165in}{5.003301in}}%
\pgfpathlineto{\pgfqpoint{1.272617in}{5.022392in}}%
\pgfpathlineto{\pgfqpoint{1.277159in}{5.028770in}}%
\pgfpathlineto{\pgfqpoint{1.295544in}{5.054240in}}%
\pgfpathlineto{\pgfqpoint{1.302412in}{5.063632in}}%
\pgfpathlineto{\pgfqpoint{1.314293in}{5.079709in}}%
\pgfpathlineto{\pgfqpoint{1.332207in}{5.103660in}}%
\pgfpathlineto{\pgfqpoint{1.333356in}{5.105179in}}%
\pgfpathlineto{\pgfqpoint{1.352869in}{5.130649in}}%
\pgfpathlineto{\pgfqpoint{1.362003in}{5.142429in}}%
\pgfpathlineto{\pgfqpoint{1.372727in}{5.156118in}}%
\pgfpathlineto{\pgfqpoint{1.391798in}{5.180180in}}%
\pgfpathlineto{\pgfqpoint{1.392925in}{5.181588in}}%
\pgfpathlineto{\pgfqpoint{1.413586in}{5.207057in}}%
\pgfpathlineto{\pgfqpoint{1.421594in}{5.216814in}}%
\pgfpathlineto{\pgfqpoint{1.434619in}{5.232527in}}%
\pgfpathlineto{\pgfqpoint{1.451389in}{5.252528in}}%
\pgfpathlineto{\pgfqpoint{1.456020in}{5.257996in}}%
\pgfpathlineto{\pgfqpoint{1.477846in}{5.283466in}}%
\pgfpathlineto{\pgfqpoint{1.481184in}{5.287316in}}%
\pgfpathlineto{\pgfqpoint{1.500120in}{5.308935in}}%
\pgfpathlineto{\pgfqpoint{1.510980in}{5.321200in}}%
\pgfpathlineto{\pgfqpoint{1.522789in}{5.334405in}}%
\pgfpathlineto{\pgfqpoint{1.540775in}{5.354300in}}%
\pgfpathlineto{\pgfqpoint{1.545864in}{5.359874in}}%
\pgfpathlineto{\pgfqpoint{1.569373in}{5.385344in}}%
\pgfpathlineto{\pgfqpoint{1.570571in}{5.386626in}}%
\pgfpathlineto{\pgfqpoint{1.593376in}{5.410813in}}%
\pgfpathlineto{\pgfqpoint{1.600366in}{5.418148in}}%
\pgfpathlineto{\pgfqpoint{1.617814in}{5.436283in}}%
\pgfpathlineto{\pgfqpoint{1.630161in}{5.448983in}}%
\pgfpathlineto{\pgfqpoint{1.642696in}{5.461752in}}%
\pgfpathlineto{\pgfqpoint{1.659957in}{5.479154in}}%
\pgfpathlineto{\pgfqpoint{1.668035in}{5.487222in}}%
\pgfpathlineto{\pgfqpoint{1.689752in}{5.508687in}}%
\pgfpathlineto{\pgfqpoint{1.693842in}{5.512691in}}%
\pgfpathlineto{\pgfqpoint{1.719547in}{5.537603in}}%
\pgfpathlineto{\pgfqpoint{1.720128in}{5.538161in}}%
\pgfpathlineto{\pgfqpoint{1.746940in}{5.563630in}}%
\pgfpathlineto{\pgfqpoint{1.749343in}{5.565889in}}%
\pgfpathlineto{\pgfqpoint{1.774257in}{5.589100in}}%
\pgfpathlineto{\pgfqpoint{1.779138in}{5.593602in}}%
\pgfpathlineto{\pgfqpoint{1.802081in}{5.614570in}}%
\pgfpathlineto{\pgfqpoint{1.808934in}{5.620770in}}%
\pgfpathlineto{\pgfqpoint{1.830424in}{5.640039in}}%
\pgfusepath{stroke}%
\end{pgfscope}%
\begin{pgfscope}%
\pgfpathrectangle{\pgfqpoint{0.766095in}{0.571603in}}{\pgfqpoint{5.929283in}{5.068436in}}%
\pgfusepath{clip}%
\pgfsetbuttcap%
\pgfsetroundjoin%
\pgfsetlinewidth{1.505625pt}%
\definecolor{currentstroke}{rgb}{0.127568,0.566949,0.550556}%
\pgfsetstrokecolor{currentstroke}%
\pgfsetdash{}{0pt}%
\pgfpathmoveto{\pgfqpoint{2.221421in}{0.571603in}}%
\pgfpathlineto{\pgfqpoint{2.196274in}{0.590352in}}%
\pgfpathlineto{\pgfqpoint{2.187276in}{0.597073in}}%
\pgfpathlineto{\pgfqpoint{2.166478in}{0.612796in}}%
\pgfpathlineto{\pgfqpoint{2.153612in}{0.622542in}}%
\pgfpathlineto{\pgfqpoint{2.136683in}{0.635522in}}%
\pgfpathlineto{\pgfqpoint{2.120426in}{0.648012in}}%
\pgfpathlineto{\pgfqpoint{2.106888in}{0.658541in}}%
\pgfpathlineto{\pgfqpoint{2.087717in}{0.673481in}}%
\pgfpathlineto{\pgfqpoint{2.077092in}{0.681863in}}%
\pgfpathlineto{\pgfqpoint{2.055480in}{0.698951in}}%
\pgfpathlineto{\pgfqpoint{2.047297in}{0.705501in}}%
\pgfpathlineto{\pgfqpoint{2.023713in}{0.724420in}}%
\pgfpathlineto{\pgfqpoint{2.017501in}{0.729465in}}%
\pgfpathlineto{\pgfqpoint{1.992411in}{0.749890in}}%
\pgfpathlineto{\pgfqpoint{1.987706in}{0.753768in}}%
\pgfpathlineto{\pgfqpoint{1.961572in}{0.775360in}}%
\pgfpathlineto{\pgfqpoint{1.957911in}{0.778422in}}%
\pgfpathlineto{\pgfqpoint{1.931191in}{0.800829in}}%
\pgfpathlineto{\pgfqpoint{1.928115in}{0.803440in}}%
\pgfpathlineto{\pgfqpoint{1.901263in}{0.826299in}}%
\pgfpathlineto{\pgfqpoint{1.898320in}{0.828835in}}%
\pgfpathlineto{\pgfqpoint{1.871786in}{0.851768in}}%
\pgfpathlineto{\pgfqpoint{1.868524in}{0.854622in}}%
\pgfpathlineto{\pgfqpoint{1.842754in}{0.877238in}}%
\pgfpathlineto{\pgfqpoint{1.838729in}{0.880814in}}%
\pgfpathlineto{\pgfqpoint{1.814163in}{0.902707in}}%
\pgfpathlineto{\pgfqpoint{1.808934in}{0.907425in}}%
\pgfpathlineto{\pgfqpoint{1.786008in}{0.928177in}}%
\pgfpathlineto{\pgfqpoint{1.779138in}{0.934472in}}%
\pgfpathlineto{\pgfqpoint{1.758283in}{0.953646in}}%
\pgfpathlineto{\pgfqpoint{1.749343in}{0.961969in}}%
\pgfpathlineto{\pgfqpoint{1.730986in}{0.979116in}}%
\pgfpathlineto{\pgfqpoint{1.719547in}{0.989933in}}%
\pgfpathlineto{\pgfqpoint{1.704108in}{1.004585in}}%
\pgfpathlineto{\pgfqpoint{1.689752in}{1.018381in}}%
\pgfpathlineto{\pgfqpoint{1.677647in}{1.030055in}}%
\pgfpathlineto{\pgfqpoint{1.659957in}{1.047329in}}%
\pgfpathlineto{\pgfqpoint{1.651596in}{1.055524in}}%
\pgfpathlineto{\pgfqpoint{1.630161in}{1.076797in}}%
\pgfpathlineto{\pgfqpoint{1.625949in}{1.080994in}}%
\pgfpathlineto{\pgfqpoint{1.600704in}{1.106463in}}%
\pgfpathlineto{\pgfqpoint{1.600366in}{1.106809in}}%
\pgfpathlineto{\pgfqpoint{1.575901in}{1.131933in}}%
\pgfpathlineto{\pgfqpoint{1.570571in}{1.137477in}}%
\pgfpathlineto{\pgfqpoint{1.551489in}{1.157402in}}%
\pgfpathlineto{\pgfqpoint{1.540775in}{1.168733in}}%
\pgfpathlineto{\pgfqpoint{1.527462in}{1.182872in}}%
\pgfpathlineto{\pgfqpoint{1.510980in}{1.200601in}}%
\pgfpathlineto{\pgfqpoint{1.503814in}{1.208341in}}%
\pgfpathlineto{\pgfqpoint{1.481184in}{1.233102in}}%
\pgfpathlineto{\pgfqpoint{1.480540in}{1.233811in}}%
\pgfpathlineto{\pgfqpoint{1.457697in}{1.259281in}}%
\pgfpathlineto{\pgfqpoint{1.451389in}{1.266407in}}%
\pgfpathlineto{\pgfqpoint{1.435226in}{1.284750in}}%
\pgfpathlineto{\pgfqpoint{1.421594in}{1.300423in}}%
\pgfpathlineto{\pgfqpoint{1.413111in}{1.310220in}}%
\pgfpathlineto{\pgfqpoint{1.391798in}{1.335160in}}%
\pgfpathlineto{\pgfqpoint{1.391348in}{1.335689in}}%
\pgfpathlineto{\pgfqpoint{1.370013in}{1.361159in}}%
\pgfpathlineto{\pgfqpoint{1.362003in}{1.370850in}}%
\pgfpathlineto{\pgfqpoint{1.349025in}{1.386628in}}%
\pgfpathlineto{\pgfqpoint{1.332207in}{1.407347in}}%
\pgfpathlineto{\pgfqpoint{1.328370in}{1.412098in}}%
\pgfpathlineto{\pgfqpoint{1.308103in}{1.437567in}}%
\pgfpathlineto{\pgfqpoint{1.302412in}{1.444824in}}%
\pgfpathlineto{\pgfqpoint{1.288202in}{1.463037in}}%
\pgfpathlineto{\pgfqpoint{1.272617in}{1.483284in}}%
\pgfpathlineto{\pgfqpoint{1.268617in}{1.488506in}}%
\pgfpathlineto{\pgfqpoint{1.249413in}{1.513976in}}%
\pgfpathlineto{\pgfqpoint{1.242821in}{1.522849in}}%
\pgfpathlineto{\pgfqpoint{1.230558in}{1.539445in}}%
\pgfpathlineto{\pgfqpoint{1.213026in}{1.563502in}}%
\pgfpathlineto{\pgfqpoint{1.212002in}{1.564915in}}%
\pgfpathlineto{\pgfqpoint{1.193852in}{1.590384in}}%
\pgfpathlineto{\pgfqpoint{1.183230in}{1.605504in}}%
\pgfpathlineto{\pgfqpoint{1.176001in}{1.615854in}}%
\pgfpathlineto{\pgfqpoint{1.158484in}{1.641323in}}%
\pgfpathlineto{\pgfqpoint{1.153435in}{1.648786in}}%
\pgfpathlineto{\pgfqpoint{1.141323in}{1.666793in}}%
\pgfpathlineto{\pgfqpoint{1.124439in}{1.692262in}}%
\pgfpathlineto{\pgfqpoint{1.123640in}{1.693490in}}%
\pgfpathlineto{\pgfqpoint{1.107952in}{1.717732in}}%
\pgfpathlineto{\pgfqpoint{1.093844in}{1.739854in}}%
\pgfpathlineto{\pgfqpoint{1.091722in}{1.743202in}}%
\pgfpathlineto{\pgfqpoint{1.075871in}{1.768671in}}%
\pgfpathlineto{\pgfqpoint{1.064049in}{1.787963in}}%
\pgfpathlineto{\pgfqpoint{1.060287in}{1.794141in}}%
\pgfpathlineto{\pgfqpoint{1.045059in}{1.819610in}}%
\pgfpathlineto{\pgfqpoint{1.034254in}{1.837982in}}%
\pgfpathlineto{\pgfqpoint{1.030105in}{1.845080in}}%
\pgfpathlineto{\pgfqpoint{1.015498in}{1.870549in}}%
\pgfpathlineto{\pgfqpoint{1.004458in}{1.890125in}}%
\pgfpathlineto{\pgfqpoint{1.001156in}{1.896019in}}%
\pgfpathlineto{\pgfqpoint{0.987165in}{1.921488in}}%
\pgfpathlineto{\pgfqpoint{0.974663in}{1.944639in}}%
\pgfpathlineto{\pgfqpoint{0.973419in}{1.946958in}}%
\pgfpathlineto{\pgfqpoint{0.960040in}{1.972427in}}%
\pgfpathlineto{\pgfqpoint{0.946892in}{1.997897in}}%
\pgfpathlineto{\pgfqpoint{0.944867in}{2.001904in}}%
\pgfpathlineto{\pgfqpoint{0.934098in}{2.023366in}}%
\pgfpathlineto{\pgfqpoint{0.921558in}{2.048836in}}%
\pgfpathlineto{\pgfqpoint{0.915072in}{2.062284in}}%
\pgfpathlineto{\pgfqpoint{0.909315in}{2.074305in}}%
\pgfpathlineto{\pgfqpoint{0.897378in}{2.099775in}}%
\pgfpathlineto{\pgfqpoint{0.885673in}{2.125244in}}%
\pgfpathlineto{\pgfqpoint{0.885277in}{2.126127in}}%
\pgfpathlineto{\pgfqpoint{0.874326in}{2.150714in}}%
\pgfpathlineto{\pgfqpoint{0.863217in}{2.176183in}}%
\pgfpathlineto{\pgfqpoint{0.855481in}{2.194322in}}%
\pgfpathlineto{\pgfqpoint{0.852378in}{2.201653in}}%
\pgfpathlineto{\pgfqpoint{0.841858in}{2.227123in}}%
\pgfpathlineto{\pgfqpoint{0.831573in}{2.252592in}}%
\pgfpathlineto{\pgfqpoint{0.825686in}{2.267532in}}%
\pgfpathlineto{\pgfqpoint{0.821569in}{2.278062in}}%
\pgfpathlineto{\pgfqpoint{0.811865in}{2.303531in}}%
\pgfpathlineto{\pgfqpoint{0.802395in}{2.329001in}}%
\pgfpathlineto{\pgfqpoint{0.795890in}{2.346958in}}%
\pgfpathlineto{\pgfqpoint{0.793191in}{2.354470in}}%
\pgfpathlineto{\pgfqpoint{0.784294in}{2.379940in}}%
\pgfpathlineto{\pgfqpoint{0.775631in}{2.405409in}}%
\pgfpathlineto{\pgfqpoint{0.767204in}{2.430879in}}%
\pgfpathlineto{\pgfqpoint{0.766095in}{2.434335in}}%
\pgfusepath{stroke}%
\end{pgfscope}%
\begin{pgfscope}%
\pgfpathrectangle{\pgfqpoint{0.766095in}{0.571603in}}{\pgfqpoint{5.929283in}{5.068436in}}%
\pgfusepath{clip}%
\pgfsetbuttcap%
\pgfsetroundjoin%
\pgfsetlinewidth{1.505625pt}%
\definecolor{currentstroke}{rgb}{0.127568,0.566949,0.550556}%
\pgfsetstrokecolor{currentstroke}%
\pgfsetdash{}{0pt}%
\pgfpathmoveto{\pgfqpoint{0.766095in}{4.185519in}}%
\pgfpathlineto{\pgfqpoint{0.766968in}{4.188276in}}%
\pgfpathlineto{\pgfqpoint{0.775257in}{4.213746in}}%
\pgfpathlineto{\pgfqpoint{0.783769in}{4.239215in}}%
\pgfpathlineto{\pgfqpoint{0.792503in}{4.264685in}}%
\pgfpathlineto{\pgfqpoint{0.795890in}{4.274313in}}%
\pgfpathlineto{\pgfqpoint{0.801530in}{4.290154in}}%
\pgfpathlineto{\pgfqpoint{0.810823in}{4.315624in}}%
\pgfpathlineto{\pgfqpoint{0.820338in}{4.341093in}}%
\pgfpathlineto{\pgfqpoint{0.825686in}{4.355079in}}%
\pgfpathlineto{\pgfqpoint{0.830130in}{4.366563in}}%
\pgfpathlineto{\pgfqpoint{0.840213in}{4.392032in}}%
\pgfpathlineto{\pgfqpoint{0.850517in}{4.417502in}}%
\pgfpathlineto{\pgfqpoint{0.855481in}{4.429509in}}%
\pgfpathlineto{\pgfqpoint{0.861112in}{4.442971in}}%
\pgfpathlineto{\pgfqpoint{0.871994in}{4.468441in}}%
\pgfpathlineto{\pgfqpoint{0.883095in}{4.493910in}}%
\pgfpathlineto{\pgfqpoint{0.885277in}{4.498813in}}%
\pgfpathlineto{\pgfqpoint{0.894534in}{4.519380in}}%
\pgfpathlineto{\pgfqpoint{0.906221in}{4.544849in}}%
\pgfpathlineto{\pgfqpoint{0.915072in}{4.563776in}}%
\pgfpathlineto{\pgfqpoint{0.918167in}{4.570319in}}%
\pgfpathlineto{\pgfqpoint{0.930448in}{4.595788in}}%
\pgfpathlineto{\pgfqpoint{0.942947in}{4.621258in}}%
\pgfpathlineto{\pgfqpoint{0.944867in}{4.625099in}}%
\pgfpathlineto{\pgfqpoint{0.955804in}{4.646728in}}%
\pgfpathlineto{\pgfqpoint{0.968904in}{4.672197in}}%
\pgfpathlineto{\pgfqpoint{0.974663in}{4.683199in}}%
\pgfpathlineto{\pgfqpoint{0.982320in}{4.697667in}}%
\pgfpathlineto{\pgfqpoint{0.996028in}{4.723136in}}%
\pgfpathlineto{\pgfqpoint{1.004458in}{4.738544in}}%
\pgfpathlineto{\pgfqpoint{1.010025in}{4.748606in}}%
\pgfpathlineto{\pgfqpoint{1.024348in}{4.774075in}}%
\pgfpathlineto{\pgfqpoint{1.034254in}{4.791417in}}%
\pgfpathlineto{\pgfqpoint{1.038947in}{4.799545in}}%
\pgfpathlineto{\pgfqpoint{1.053892in}{4.825014in}}%
\pgfpathlineto{\pgfqpoint{1.064049in}{4.842067in}}%
\pgfpathlineto{\pgfqpoint{1.069117in}{4.850484in}}%
\pgfpathlineto{\pgfqpoint{1.084689in}{4.875953in}}%
\pgfpathlineto{\pgfqpoint{1.093844in}{4.890711in}}%
\pgfpathlineto{\pgfqpoint{1.100561in}{4.901423in}}%
\pgfpathlineto{\pgfqpoint{1.116767in}{4.926892in}}%
\pgfpathlineto{\pgfqpoint{1.123640in}{4.937540in}}%
\pgfpathlineto{\pgfqpoint{1.133310in}{4.952362in}}%
\pgfpathlineto{\pgfqpoint{1.150153in}{4.977831in}}%
\pgfpathlineto{\pgfqpoint{1.153435in}{4.982722in}}%
\pgfpathlineto{\pgfqpoint{1.167389in}{5.003301in}}%
\pgfpathlineto{\pgfqpoint{1.183230in}{5.026369in}}%
\pgfpathlineto{\pgfqpoint{1.184897in}{5.028770in}}%
\pgfpathlineto{\pgfqpoint{1.202825in}{5.054240in}}%
\pgfpathlineto{\pgfqpoint{1.213026in}{5.068549in}}%
\pgfpathlineto{\pgfqpoint{1.221065in}{5.079709in}}%
\pgfpathlineto{\pgfqpoint{1.239645in}{5.105179in}}%
\pgfpathlineto{\pgfqpoint{1.242821in}{5.109474in}}%
\pgfpathlineto{\pgfqpoint{1.258638in}{5.130649in}}%
\pgfpathlineto{\pgfqpoint{1.272617in}{5.149141in}}%
\pgfpathlineto{\pgfqpoint{1.277945in}{5.156118in}}%
\pgfpathlineto{\pgfqpoint{1.297641in}{5.181588in}}%
\pgfpathlineto{\pgfqpoint{1.302412in}{5.187679in}}%
\pgfpathlineto{\pgfqpoint{1.317740in}{5.207057in}}%
\pgfpathlineto{\pgfqpoint{1.332207in}{5.225138in}}%
\pgfpathlineto{\pgfqpoint{1.338179in}{5.232527in}}%
\pgfpathlineto{\pgfqpoint{1.359010in}{5.257996in}}%
\pgfpathlineto{\pgfqpoint{1.362003in}{5.261611in}}%
\pgfpathlineto{\pgfqpoint{1.380278in}{5.283466in}}%
\pgfpathlineto{\pgfqpoint{1.391798in}{5.297090in}}%
\pgfpathlineto{\pgfqpoint{1.401913in}{5.308935in}}%
\pgfpathlineto{\pgfqpoint{1.421594in}{5.331729in}}%
\pgfpathlineto{\pgfqpoint{1.423926in}{5.334405in}}%
\pgfpathlineto{\pgfqpoint{1.446396in}{5.359874in}}%
\pgfpathlineto{\pgfqpoint{1.451389in}{5.365470in}}%
\pgfpathlineto{\pgfqpoint{1.469291in}{5.385344in}}%
\pgfpathlineto{\pgfqpoint{1.481184in}{5.398405in}}%
\pgfpathlineto{\pgfqpoint{1.492591in}{5.410813in}}%
\pgfpathlineto{\pgfqpoint{1.510980in}{5.430605in}}%
\pgfpathlineto{\pgfqpoint{1.516305in}{5.436283in}}%
\pgfpathlineto{\pgfqpoint{1.540450in}{5.461752in}}%
\pgfpathlineto{\pgfqpoint{1.540775in}{5.462091in}}%
\pgfpathlineto{\pgfqpoint{1.565100in}{5.487222in}}%
\pgfpathlineto{\pgfqpoint{1.570571in}{5.492815in}}%
\pgfpathlineto{\pgfqpoint{1.590191in}{5.512691in}}%
\pgfpathlineto{\pgfqpoint{1.600366in}{5.522892in}}%
\pgfpathlineto{\pgfqpoint{1.615735in}{5.538161in}}%
\pgfpathlineto{\pgfqpoint{1.630161in}{5.552346in}}%
\pgfpathlineto{\pgfqpoint{1.641743in}{5.563630in}}%
\pgfpathlineto{\pgfqpoint{1.659957in}{5.581197in}}%
\pgfpathlineto{\pgfqpoint{1.668226in}{5.589100in}}%
\pgfpathlineto{\pgfqpoint{1.689752in}{5.609467in}}%
\pgfpathlineto{\pgfqpoint{1.695194in}{5.614570in}}%
\pgfpathlineto{\pgfqpoint{1.719547in}{5.637176in}}%
\pgfpathlineto{\pgfqpoint{1.722660in}{5.640039in}}%
\pgfusepath{stroke}%
\end{pgfscope}%
\begin{pgfscope}%
\pgfpathrectangle{\pgfqpoint{0.766095in}{0.571603in}}{\pgfqpoint{5.929283in}{5.068436in}}%
\pgfusepath{clip}%
\pgfsetbuttcap%
\pgfsetroundjoin%
\pgfsetlinewidth{1.505625pt}%
\definecolor{currentstroke}{rgb}{0.122606,0.585371,0.546557}%
\pgfsetstrokecolor{currentstroke}%
\pgfsetdash{}{0pt}%
\pgfpathmoveto{\pgfqpoint{2.141970in}{0.571603in}}%
\pgfpathlineto{\pgfqpoint{2.136683in}{0.575568in}}%
\pgfpathlineto{\pgfqpoint{2.108060in}{0.597073in}}%
\pgfpathlineto{\pgfqpoint{2.106888in}{0.597964in}}%
\pgfpathlineto{\pgfqpoint{2.077092in}{0.620679in}}%
\pgfpathlineto{\pgfqpoint{2.074654in}{0.622542in}}%
\pgfpathlineto{\pgfqpoint{2.047297in}{0.643699in}}%
\pgfpathlineto{\pgfqpoint{2.041732in}{0.648012in}}%
\pgfpathlineto{\pgfqpoint{2.017501in}{0.667019in}}%
\pgfpathlineto{\pgfqpoint{2.009281in}{0.673481in}}%
\pgfpathlineto{\pgfqpoint{1.987706in}{0.690648in}}%
\pgfpathlineto{\pgfqpoint{1.977296in}{0.698951in}}%
\pgfpathlineto{\pgfqpoint{1.957911in}{0.714599in}}%
\pgfpathlineto{\pgfqpoint{1.945773in}{0.724420in}}%
\pgfpathlineto{\pgfqpoint{1.928115in}{0.738883in}}%
\pgfpathlineto{\pgfqpoint{1.914711in}{0.749890in}}%
\pgfpathlineto{\pgfqpoint{1.898320in}{0.763513in}}%
\pgfpathlineto{\pgfqpoint{1.884104in}{0.775360in}}%
\pgfpathlineto{\pgfqpoint{1.868524in}{0.788501in}}%
\pgfpathlineto{\pgfqpoint{1.853949in}{0.800829in}}%
\pgfpathlineto{\pgfqpoint{1.838729in}{0.813860in}}%
\pgfpathlineto{\pgfqpoint{1.824242in}{0.826299in}}%
\pgfpathlineto{\pgfqpoint{1.808934in}{0.839603in}}%
\pgfpathlineto{\pgfqpoint{1.794977in}{0.851768in}}%
\pgfpathlineto{\pgfqpoint{1.779138in}{0.865744in}}%
\pgfpathlineto{\pgfqpoint{1.766152in}{0.877238in}}%
\pgfpathlineto{\pgfqpoint{1.749343in}{0.892298in}}%
\pgfpathlineto{\pgfqpoint{1.737761in}{0.902707in}}%
\pgfpathlineto{\pgfqpoint{1.719547in}{0.919278in}}%
\pgfpathlineto{\pgfqpoint{1.709799in}{0.928177in}}%
\pgfpathlineto{\pgfqpoint{1.689752in}{0.946701in}}%
\pgfpathlineto{\pgfqpoint{1.682262in}{0.953646in}}%
\pgfpathlineto{\pgfqpoint{1.659957in}{0.974583in}}%
\pgfpathlineto{\pgfqpoint{1.655144in}{0.979116in}}%
\pgfpathlineto{\pgfqpoint{1.630161in}{1.002939in}}%
\pgfpathlineto{\pgfqpoint{1.628441in}{1.004585in}}%
\pgfpathlineto{\pgfqpoint{1.602164in}{1.030055in}}%
\pgfpathlineto{\pgfqpoint{1.600366in}{1.031820in}}%
\pgfpathlineto{\pgfqpoint{1.576313in}{1.055524in}}%
\pgfpathlineto{\pgfqpoint{1.570571in}{1.061255in}}%
\pgfpathlineto{\pgfqpoint{1.550865in}{1.080994in}}%
\pgfpathlineto{\pgfqpoint{1.540775in}{1.091227in}}%
\pgfpathlineto{\pgfqpoint{1.525812in}{1.106463in}}%
\pgfpathlineto{\pgfqpoint{1.510980in}{1.121757in}}%
\pgfpathlineto{\pgfqpoint{1.501151in}{1.131933in}}%
\pgfpathlineto{\pgfqpoint{1.481184in}{1.152865in}}%
\pgfpathlineto{\pgfqpoint{1.476874in}{1.157402in}}%
\pgfpathlineto{\pgfqpoint{1.452992in}{1.182872in}}%
\pgfpathlineto{\pgfqpoint{1.451389in}{1.184605in}}%
\pgfpathlineto{\pgfqpoint{1.429531in}{1.208341in}}%
\pgfpathlineto{\pgfqpoint{1.421594in}{1.217072in}}%
\pgfpathlineto{\pgfqpoint{1.406441in}{1.233811in}}%
\pgfpathlineto{\pgfqpoint{1.391798in}{1.250195in}}%
\pgfpathlineto{\pgfqpoint{1.383714in}{1.259281in}}%
\pgfpathlineto{\pgfqpoint{1.362003in}{1.283999in}}%
\pgfpathlineto{\pgfqpoint{1.361346in}{1.284750in}}%
\pgfpathlineto{\pgfqpoint{1.339403in}{1.310220in}}%
\pgfpathlineto{\pgfqpoint{1.332207in}{1.318682in}}%
\pgfpathlineto{\pgfqpoint{1.317815in}{1.335689in}}%
\pgfpathlineto{\pgfqpoint{1.302412in}{1.354129in}}%
\pgfpathlineto{\pgfqpoint{1.296569in}{1.361159in}}%
\pgfpathlineto{\pgfqpoint{1.275690in}{1.386628in}}%
\pgfpathlineto{\pgfqpoint{1.272617in}{1.390433in}}%
\pgfpathlineto{\pgfqpoint{1.255204in}{1.412098in}}%
\pgfpathlineto{\pgfqpoint{1.242821in}{1.427711in}}%
\pgfpathlineto{\pgfqpoint{1.235043in}{1.437567in}}%
\pgfpathlineto{\pgfqpoint{1.215224in}{1.463037in}}%
\pgfpathlineto{\pgfqpoint{1.213026in}{1.465906in}}%
\pgfpathlineto{\pgfqpoint{1.195801in}{1.488506in}}%
\pgfpathlineto{\pgfqpoint{1.183230in}{1.505225in}}%
\pgfpathlineto{\pgfqpoint{1.176686in}{1.513976in}}%
\pgfpathlineto{\pgfqpoint{1.157919in}{1.539445in}}%
\pgfpathlineto{\pgfqpoint{1.153435in}{1.545626in}}%
\pgfpathlineto{\pgfqpoint{1.139518in}{1.564915in}}%
\pgfpathlineto{\pgfqpoint{1.123640in}{1.587229in}}%
\pgfpathlineto{\pgfqpoint{1.121407in}{1.590384in}}%
\pgfpathlineto{\pgfqpoint{1.103682in}{1.615854in}}%
\pgfpathlineto{\pgfqpoint{1.093844in}{1.630200in}}%
\pgfpathlineto{\pgfqpoint{1.086260in}{1.641323in}}%
\pgfpathlineto{\pgfqpoint{1.069163in}{1.666793in}}%
\pgfpathlineto{\pgfqpoint{1.064049in}{1.674539in}}%
\pgfpathlineto{\pgfqpoint{1.052415in}{1.692262in}}%
\pgfpathlineto{\pgfqpoint{1.035944in}{1.717732in}}%
\pgfpathlineto{\pgfqpoint{1.034254in}{1.720393in}}%
\pgfpathlineto{\pgfqpoint{1.019854in}{1.743202in}}%
\pgfpathlineto{\pgfqpoint{1.004458in}{1.767948in}}%
\pgfpathlineto{\pgfqpoint{1.004011in}{1.768671in}}%
\pgfpathlineto{\pgfqpoint{0.988558in}{1.794141in}}%
\pgfpathlineto{\pgfqpoint{0.974663in}{1.817393in}}%
\pgfpathlineto{\pgfqpoint{0.973346in}{1.819610in}}%
\pgfpathlineto{\pgfqpoint{0.958509in}{1.845080in}}%
\pgfpathlineto{\pgfqpoint{0.944867in}{1.868868in}}%
\pgfpathlineto{\pgfqpoint{0.943910in}{1.870549in}}%
\pgfpathlineto{\pgfqpoint{0.929685in}{1.896019in}}%
\pgfpathlineto{\pgfqpoint{0.915688in}{1.921488in}}%
\pgfpathlineto{\pgfqpoint{0.915072in}{1.922633in}}%
\pgfpathlineto{\pgfqpoint{0.902065in}{1.946958in}}%
\pgfpathlineto{\pgfqpoint{0.888677in}{1.972427in}}%
\pgfpathlineto{\pgfqpoint{0.885277in}{1.979029in}}%
\pgfpathlineto{\pgfqpoint{0.875626in}{1.997897in}}%
\pgfpathlineto{\pgfqpoint{0.862841in}{2.023366in}}%
\pgfpathlineto{\pgfqpoint{0.855481in}{2.038324in}}%
\pgfpathlineto{\pgfqpoint{0.850345in}{2.048836in}}%
\pgfpathlineto{\pgfqpoint{0.838158in}{2.074305in}}%
\pgfpathlineto{\pgfqpoint{0.826203in}{2.099775in}}%
\pgfpathlineto{\pgfqpoint{0.825686in}{2.100903in}}%
\pgfpathlineto{\pgfqpoint{0.814603in}{2.125244in}}%
\pgfpathlineto{\pgfqpoint{0.803240in}{2.150714in}}%
\pgfpathlineto{\pgfqpoint{0.795890in}{2.167553in}}%
\pgfpathlineto{\pgfqpoint{0.792151in}{2.176183in}}%
\pgfpathlineto{\pgfqpoint{0.781373in}{2.201653in}}%
\pgfpathlineto{\pgfqpoint{0.770828in}{2.227123in}}%
\pgfpathlineto{\pgfqpoint{0.766095in}{2.238833in}}%
\pgfusepath{stroke}%
\end{pgfscope}%
\begin{pgfscope}%
\pgfpathrectangle{\pgfqpoint{0.766095in}{0.571603in}}{\pgfqpoint{5.929283in}{5.068436in}}%
\pgfusepath{clip}%
\pgfsetbuttcap%
\pgfsetroundjoin%
\pgfsetlinewidth{1.505625pt}%
\definecolor{currentstroke}{rgb}{0.122606,0.585371,0.546557}%
\pgfsetstrokecolor{currentstroke}%
\pgfsetdash{}{0pt}%
\pgfpathmoveto{\pgfqpoint{0.766095in}{4.407443in}}%
\pgfpathlineto{\pgfqpoint{0.770094in}{4.417502in}}%
\pgfpathlineto{\pgfqpoint{0.780445in}{4.442971in}}%
\pgfpathlineto{\pgfqpoint{0.791014in}{4.468441in}}%
\pgfpathlineto{\pgfqpoint{0.795890in}{4.479946in}}%
\pgfpathlineto{\pgfqpoint{0.801877in}{4.493910in}}%
\pgfpathlineto{\pgfqpoint{0.813020in}{4.519380in}}%
\pgfpathlineto{\pgfqpoint{0.824380in}{4.544849in}}%
\pgfpathlineto{\pgfqpoint{0.825686in}{4.547718in}}%
\pgfpathlineto{\pgfqpoint{0.836089in}{4.570319in}}%
\pgfpathlineto{\pgfqpoint{0.848031in}{4.595788in}}%
\pgfpathlineto{\pgfqpoint{0.855481in}{4.611386in}}%
\pgfpathlineto{\pgfqpoint{0.860249in}{4.621258in}}%
\pgfpathlineto{\pgfqpoint{0.872781in}{4.646728in}}%
\pgfpathlineto{\pgfqpoint{0.885277in}{4.671693in}}%
\pgfpathlineto{\pgfqpoint{0.885532in}{4.672197in}}%
\pgfpathlineto{\pgfqpoint{0.898660in}{4.697667in}}%
\pgfpathlineto{\pgfqpoint{0.912004in}{4.723136in}}%
\pgfpathlineto{\pgfqpoint{0.915072in}{4.728891in}}%
\pgfpathlineto{\pgfqpoint{0.925697in}{4.748606in}}%
\pgfpathlineto{\pgfqpoint{0.939644in}{4.774075in}}%
\pgfpathlineto{\pgfqpoint{0.944867in}{4.783458in}}%
\pgfpathlineto{\pgfqpoint{0.953920in}{4.799545in}}%
\pgfpathlineto{\pgfqpoint{0.968477in}{4.825014in}}%
\pgfpathlineto{\pgfqpoint{0.974663in}{4.835669in}}%
\pgfpathlineto{\pgfqpoint{0.983357in}{4.850484in}}%
\pgfpathlineto{\pgfqpoint{0.998530in}{4.875953in}}%
\pgfpathlineto{\pgfqpoint{1.004458in}{4.885754in}}%
\pgfpathlineto{\pgfqpoint{1.014037in}{4.901423in}}%
\pgfpathlineto{\pgfqpoint{1.029831in}{4.926892in}}%
\pgfpathlineto{\pgfqpoint{1.034254in}{4.933917in}}%
\pgfpathlineto{\pgfqpoint{1.045986in}{4.952362in}}%
\pgfpathlineto{\pgfqpoint{1.062408in}{4.977831in}}%
\pgfpathlineto{\pgfqpoint{1.064049in}{4.980338in}}%
\pgfpathlineto{\pgfqpoint{1.079234in}{5.003301in}}%
\pgfpathlineto{\pgfqpoint{1.093844in}{5.025113in}}%
\pgfpathlineto{\pgfqpoint{1.096320in}{5.028770in}}%
\pgfpathlineto{\pgfqpoint{1.113805in}{5.054240in}}%
\pgfpathlineto{\pgfqpoint{1.123640in}{5.068381in}}%
\pgfpathlineto{\pgfqpoint{1.131598in}{5.079709in}}%
\pgfpathlineto{\pgfqpoint{1.149726in}{5.105179in}}%
\pgfpathlineto{\pgfqpoint{1.153435in}{5.110321in}}%
\pgfpathlineto{\pgfqpoint{1.168248in}{5.130649in}}%
\pgfpathlineto{\pgfqpoint{1.183230in}{5.150962in}}%
\pgfpathlineto{\pgfqpoint{1.187071in}{5.156118in}}%
\pgfpathlineto{\pgfqpoint{1.206294in}{5.181588in}}%
\pgfpathlineto{\pgfqpoint{1.213026in}{5.190397in}}%
\pgfpathlineto{\pgfqpoint{1.225883in}{5.207057in}}%
\pgfpathlineto{\pgfqpoint{1.242821in}{5.228751in}}%
\pgfpathlineto{\pgfqpoint{1.245798in}{5.232527in}}%
\pgfpathlineto{\pgfqpoint{1.266136in}{5.257996in}}%
\pgfpathlineto{\pgfqpoint{1.272617in}{5.266016in}}%
\pgfpathlineto{\pgfqpoint{1.286854in}{5.283466in}}%
\pgfpathlineto{\pgfqpoint{1.302412in}{5.302319in}}%
\pgfpathlineto{\pgfqpoint{1.307924in}{5.308935in}}%
\pgfpathlineto{\pgfqpoint{1.329393in}{5.334405in}}%
\pgfpathlineto{\pgfqpoint{1.332207in}{5.337703in}}%
\pgfpathlineto{\pgfqpoint{1.351302in}{5.359874in}}%
\pgfpathlineto{\pgfqpoint{1.362003in}{5.372163in}}%
\pgfpathlineto{\pgfqpoint{1.373589in}{5.385344in}}%
\pgfpathlineto{\pgfqpoint{1.391798in}{5.405835in}}%
\pgfpathlineto{\pgfqpoint{1.396263in}{5.410813in}}%
\pgfpathlineto{\pgfqpoint{1.419365in}{5.436283in}}%
\pgfpathlineto{\pgfqpoint{1.421594in}{5.438711in}}%
\pgfpathlineto{\pgfqpoint{1.442930in}{5.461752in}}%
\pgfpathlineto{\pgfqpoint{1.451389in}{5.470791in}}%
\pgfpathlineto{\pgfqpoint{1.466908in}{5.487222in}}%
\pgfpathlineto{\pgfqpoint{1.481184in}{5.502178in}}%
\pgfpathlineto{\pgfqpoint{1.491310in}{5.512691in}}%
\pgfpathlineto{\pgfqpoint{1.510980in}{5.532900in}}%
\pgfpathlineto{\pgfqpoint{1.516147in}{5.538161in}}%
\pgfpathlineto{\pgfqpoint{1.540775in}{5.562978in}}%
\pgfpathlineto{\pgfqpoint{1.541428in}{5.563630in}}%
\pgfpathlineto{\pgfqpoint{1.567211in}{5.589100in}}%
\pgfpathlineto{\pgfqpoint{1.570571in}{5.592384in}}%
\pgfpathlineto{\pgfqpoint{1.593463in}{5.614570in}}%
\pgfpathlineto{\pgfqpoint{1.600366in}{5.621191in}}%
\pgfpathlineto{\pgfqpoint{1.620186in}{5.640039in}}%
\pgfusepath{stroke}%
\end{pgfscope}%
\begin{pgfscope}%
\pgfpathrectangle{\pgfqpoint{0.766095in}{0.571603in}}{\pgfqpoint{5.929283in}{5.068436in}}%
\pgfusepath{clip}%
\pgfsetbuttcap%
\pgfsetroundjoin%
\pgfsetlinewidth{1.505625pt}%
\definecolor{currentstroke}{rgb}{0.119512,0.607464,0.540218}%
\pgfsetstrokecolor{currentstroke}%
\pgfsetdash{}{0pt}%
\pgfpathmoveto{\pgfqpoint{2.064567in}{0.571603in}}%
\pgfpathlineto{\pgfqpoint{2.047297in}{0.584666in}}%
\pgfpathlineto{\pgfqpoint{2.030927in}{0.597073in}}%
\pgfpathlineto{\pgfqpoint{2.017501in}{0.607370in}}%
\pgfpathlineto{\pgfqpoint{1.997761in}{0.622542in}}%
\pgfpathlineto{\pgfqpoint{1.987706in}{0.630363in}}%
\pgfpathlineto{\pgfqpoint{1.965066in}{0.648012in}}%
\pgfpathlineto{\pgfqpoint{1.957911in}{0.653657in}}%
\pgfpathlineto{\pgfqpoint{1.932839in}{0.673481in}}%
\pgfpathlineto{\pgfqpoint{1.928115in}{0.677262in}}%
\pgfpathlineto{\pgfqpoint{1.901077in}{0.698951in}}%
\pgfpathlineto{\pgfqpoint{1.898320in}{0.701189in}}%
\pgfpathlineto{\pgfqpoint{1.869777in}{0.724420in}}%
\pgfpathlineto{\pgfqpoint{1.868524in}{0.725452in}}%
\pgfpathlineto{\pgfqpoint{1.838934in}{0.749890in}}%
\pgfpathlineto{\pgfqpoint{1.838729in}{0.750061in}}%
\pgfpathlineto{\pgfqpoint{1.808934in}{0.775036in}}%
\pgfpathlineto{\pgfqpoint{1.808548in}{0.775360in}}%
\pgfpathlineto{\pgfqpoint{1.779138in}{0.800379in}}%
\pgfpathlineto{\pgfqpoint{1.778610in}{0.800829in}}%
\pgfpathlineto{\pgfqpoint{1.749343in}{0.826100in}}%
\pgfpathlineto{\pgfqpoint{1.749113in}{0.826299in}}%
\pgfpathlineto{\pgfqpoint{1.720058in}{0.851768in}}%
\pgfpathlineto{\pgfqpoint{1.719547in}{0.852221in}}%
\pgfpathlineto{\pgfqpoint{1.691442in}{0.877238in}}%
\pgfpathlineto{\pgfqpoint{1.689752in}{0.878760in}}%
\pgfpathlineto{\pgfqpoint{1.663258in}{0.902707in}}%
\pgfpathlineto{\pgfqpoint{1.659957in}{0.905727in}}%
\pgfpathlineto{\pgfqpoint{1.635501in}{0.928177in}}%
\pgfpathlineto{\pgfqpoint{1.630161in}{0.933138in}}%
\pgfpathlineto{\pgfqpoint{1.608166in}{0.953646in}}%
\pgfpathlineto{\pgfqpoint{1.600366in}{0.961008in}}%
\pgfpathlineto{\pgfqpoint{1.581248in}{0.979116in}}%
\pgfpathlineto{\pgfqpoint{1.570571in}{0.989354in}}%
\pgfpathlineto{\pgfqpoint{1.554742in}{1.004585in}}%
\pgfpathlineto{\pgfqpoint{1.540775in}{1.018192in}}%
\pgfpathlineto{\pgfqpoint{1.528643in}{1.030055in}}%
\pgfpathlineto{\pgfqpoint{1.510980in}{1.047540in}}%
\pgfpathlineto{\pgfqpoint{1.502945in}{1.055524in}}%
\pgfpathlineto{\pgfqpoint{1.481184in}{1.077416in}}%
\pgfpathlineto{\pgfqpoint{1.477642in}{1.080994in}}%
\pgfpathlineto{\pgfqpoint{1.452742in}{1.106463in}}%
\pgfpathlineto{\pgfqpoint{1.451389in}{1.107867in}}%
\pgfpathlineto{\pgfqpoint{1.428267in}{1.131933in}}%
\pgfpathlineto{\pgfqpoint{1.421594in}{1.138967in}}%
\pgfpathlineto{\pgfqpoint{1.404174in}{1.157402in}}%
\pgfpathlineto{\pgfqpoint{1.391798in}{1.170666in}}%
\pgfpathlineto{\pgfqpoint{1.380457in}{1.182872in}}%
\pgfpathlineto{\pgfqpoint{1.362003in}{1.202986in}}%
\pgfpathlineto{\pgfqpoint{1.357110in}{1.208341in}}%
\pgfpathlineto{\pgfqpoint{1.334147in}{1.233811in}}%
\pgfpathlineto{\pgfqpoint{1.332207in}{1.235992in}}%
\pgfpathlineto{\pgfqpoint{1.311595in}{1.259281in}}%
\pgfpathlineto{\pgfqpoint{1.302412in}{1.269789in}}%
\pgfpathlineto{\pgfqpoint{1.289398in}{1.284750in}}%
\pgfpathlineto{\pgfqpoint{1.272617in}{1.304292in}}%
\pgfpathlineto{\pgfqpoint{1.267550in}{1.310220in}}%
\pgfpathlineto{\pgfqpoint{1.246077in}{1.335689in}}%
\pgfpathlineto{\pgfqpoint{1.242821in}{1.339607in}}%
\pgfpathlineto{\pgfqpoint{1.224996in}{1.361159in}}%
\pgfpathlineto{\pgfqpoint{1.213026in}{1.375822in}}%
\pgfpathlineto{\pgfqpoint{1.204248in}{1.386628in}}%
\pgfpathlineto{\pgfqpoint{1.183831in}{1.412098in}}%
\pgfpathlineto{\pgfqpoint{1.183230in}{1.412858in}}%
\pgfpathlineto{\pgfqpoint{1.163828in}{1.437567in}}%
\pgfpathlineto{\pgfqpoint{1.153435in}{1.450980in}}%
\pgfpathlineto{\pgfqpoint{1.144140in}{1.463037in}}%
\pgfpathlineto{\pgfqpoint{1.124773in}{1.488506in}}%
\pgfpathlineto{\pgfqpoint{1.123640in}{1.490021in}}%
\pgfpathlineto{\pgfqpoint{1.105809in}{1.513976in}}%
\pgfpathlineto{\pgfqpoint{1.093844in}{1.530270in}}%
\pgfpathlineto{\pgfqpoint{1.087143in}{1.539445in}}%
\pgfpathlineto{\pgfqpoint{1.068817in}{1.564915in}}%
\pgfpathlineto{\pgfqpoint{1.064049in}{1.571647in}}%
\pgfpathlineto{\pgfqpoint{1.050850in}{1.590384in}}%
\pgfpathlineto{\pgfqpoint{1.034254in}{1.614274in}}%
\pgfpathlineto{\pgfqpoint{1.033162in}{1.615854in}}%
\pgfpathlineto{\pgfqpoint{1.015867in}{1.641323in}}%
\pgfpathlineto{\pgfqpoint{1.004458in}{1.658369in}}%
\pgfpathlineto{\pgfqpoint{0.998852in}{1.666793in}}%
\pgfpathlineto{\pgfqpoint{0.982178in}{1.692262in}}%
\pgfpathlineto{\pgfqpoint{0.974663in}{1.703928in}}%
\pgfpathlineto{\pgfqpoint{0.965821in}{1.717732in}}%
\pgfpathlineto{\pgfqpoint{0.949766in}{1.743202in}}%
\pgfpathlineto{\pgfqpoint{0.944867in}{1.751108in}}%
\pgfpathlineto{\pgfqpoint{0.934051in}{1.768671in}}%
\pgfpathlineto{\pgfqpoint{0.918612in}{1.794141in}}%
\pgfpathlineto{\pgfqpoint{0.915072in}{1.800088in}}%
\pgfpathlineto{\pgfqpoint{0.903523in}{1.819610in}}%
\pgfpathlineto{\pgfqpoint{0.888696in}{1.845080in}}%
\pgfpathlineto{\pgfqpoint{0.885277in}{1.851065in}}%
\pgfpathlineto{\pgfqpoint{0.874216in}{1.870549in}}%
\pgfpathlineto{\pgfqpoint{0.859998in}{1.896019in}}%
\pgfpathlineto{\pgfqpoint{0.855481in}{1.904265in}}%
\pgfpathlineto{\pgfqpoint{0.846109in}{1.921488in}}%
\pgfpathlineto{\pgfqpoint{0.832495in}{1.946958in}}%
\pgfpathlineto{\pgfqpoint{0.825686in}{1.959941in}}%
\pgfpathlineto{\pgfqpoint{0.819180in}{1.972427in}}%
\pgfpathlineto{\pgfqpoint{0.806166in}{1.997897in}}%
\pgfpathlineto{\pgfqpoint{0.795890in}{2.018382in}}%
\pgfpathlineto{\pgfqpoint{0.793408in}{2.023366in}}%
\pgfpathlineto{\pgfqpoint{0.780987in}{2.048836in}}%
\pgfpathlineto{\pgfqpoint{0.768796in}{2.074305in}}%
\pgfpathlineto{\pgfqpoint{0.766095in}{2.080073in}}%
\pgfusepath{stroke}%
\end{pgfscope}%
\begin{pgfscope}%
\pgfpathrectangle{\pgfqpoint{0.766095in}{0.571603in}}{\pgfqpoint{5.929283in}{5.068436in}}%
\pgfusepath{clip}%
\pgfsetbuttcap%
\pgfsetroundjoin%
\pgfsetlinewidth{1.505625pt}%
\definecolor{currentstroke}{rgb}{0.119512,0.607464,0.540218}%
\pgfsetstrokecolor{currentstroke}%
\pgfsetdash{}{0pt}%
\pgfpathmoveto{\pgfqpoint{0.766095in}{4.590767in}}%
\pgfpathlineto{\pgfqpoint{0.768411in}{4.595788in}}%
\pgfpathlineto{\pgfqpoint{0.780391in}{4.621258in}}%
\pgfpathlineto{\pgfqpoint{0.792585in}{4.646728in}}%
\pgfpathlineto{\pgfqpoint{0.795890in}{4.653505in}}%
\pgfpathlineto{\pgfqpoint{0.805107in}{4.672197in}}%
\pgfpathlineto{\pgfqpoint{0.817886in}{4.697667in}}%
\pgfpathlineto{\pgfqpoint{0.825686in}{4.712945in}}%
\pgfpathlineto{\pgfqpoint{0.830945in}{4.723136in}}%
\pgfpathlineto{\pgfqpoint{0.844316in}{4.748606in}}%
\pgfpathlineto{\pgfqpoint{0.855481in}{4.769533in}}%
\pgfpathlineto{\pgfqpoint{0.857930in}{4.774075in}}%
\pgfpathlineto{\pgfqpoint{0.871901in}{4.799545in}}%
\pgfpathlineto{\pgfqpoint{0.885277in}{4.823563in}}%
\pgfpathlineto{\pgfqpoint{0.886093in}{4.825014in}}%
\pgfpathlineto{\pgfqpoint{0.900669in}{4.850484in}}%
\pgfpathlineto{\pgfqpoint{0.915072in}{4.875291in}}%
\pgfpathlineto{\pgfqpoint{0.915461in}{4.875953in}}%
\pgfpathlineto{\pgfqpoint{0.930648in}{4.901423in}}%
\pgfpathlineto{\pgfqpoint{0.944867in}{4.924939in}}%
\pgfpathlineto{\pgfqpoint{0.946060in}{4.926892in}}%
\pgfpathlineto{\pgfqpoint{0.961865in}{4.952362in}}%
\pgfpathlineto{\pgfqpoint{0.974663in}{4.972706in}}%
\pgfpathlineto{\pgfqpoint{0.977920in}{4.977831in}}%
\pgfpathlineto{\pgfqpoint{0.994348in}{5.003301in}}%
\pgfpathlineto{\pgfqpoint{1.004458in}{5.018764in}}%
\pgfpathlineto{\pgfqpoint{1.011066in}{5.028770in}}%
\pgfpathlineto{\pgfqpoint{1.028122in}{5.054240in}}%
\pgfpathlineto{\pgfqpoint{1.034254in}{5.063272in}}%
\pgfpathlineto{\pgfqpoint{1.045525in}{5.079709in}}%
\pgfpathlineto{\pgfqpoint{1.063213in}{5.105179in}}%
\pgfpathlineto{\pgfqpoint{1.064049in}{5.106366in}}%
\pgfpathlineto{\pgfqpoint{1.081322in}{5.130649in}}%
\pgfpathlineto{\pgfqpoint{1.093844in}{5.148038in}}%
\pgfpathlineto{\pgfqpoint{1.099720in}{5.156118in}}%
\pgfpathlineto{\pgfqpoint{1.118482in}{5.181588in}}%
\pgfpathlineto{\pgfqpoint{1.123640in}{5.188499in}}%
\pgfpathlineto{\pgfqpoint{1.137623in}{5.207057in}}%
\pgfpathlineto{\pgfqpoint{1.153435in}{5.227797in}}%
\pgfpathlineto{\pgfqpoint{1.157075in}{5.232527in}}%
\pgfpathlineto{\pgfqpoint{1.176932in}{5.257996in}}%
\pgfpathlineto{\pgfqpoint{1.183230in}{5.265977in}}%
\pgfpathlineto{\pgfqpoint{1.197164in}{5.283466in}}%
\pgfpathlineto{\pgfqpoint{1.213026in}{5.303150in}}%
\pgfpathlineto{\pgfqpoint{1.217732in}{5.308935in}}%
\pgfpathlineto{\pgfqpoint{1.238701in}{5.334405in}}%
\pgfpathlineto{\pgfqpoint{1.242821in}{5.339349in}}%
\pgfpathlineto{\pgfqpoint{1.260084in}{5.359874in}}%
\pgfpathlineto{\pgfqpoint{1.272617in}{5.374611in}}%
\pgfpathlineto{\pgfqpoint{1.281829in}{5.385344in}}%
\pgfpathlineto{\pgfqpoint{1.302412in}{5.409062in}}%
\pgfpathlineto{\pgfqpoint{1.303945in}{5.410813in}}%
\pgfpathlineto{\pgfqpoint{1.326519in}{5.436283in}}%
\pgfpathlineto{\pgfqpoint{1.332207in}{5.442630in}}%
\pgfpathlineto{\pgfqpoint{1.349500in}{5.461752in}}%
\pgfpathlineto{\pgfqpoint{1.362003in}{5.475430in}}%
\pgfpathlineto{\pgfqpoint{1.372879in}{5.487222in}}%
\pgfpathlineto{\pgfqpoint{1.391798in}{5.507518in}}%
\pgfpathlineto{\pgfqpoint{1.396664in}{5.512691in}}%
\pgfpathlineto{\pgfqpoint{1.420877in}{5.538161in}}%
\pgfpathlineto{\pgfqpoint{1.421594in}{5.538907in}}%
\pgfpathlineto{\pgfqpoint{1.445575in}{5.563630in}}%
\pgfpathlineto{\pgfqpoint{1.451389in}{5.569562in}}%
\pgfpathlineto{\pgfqpoint{1.470706in}{5.589100in}}%
\pgfpathlineto{\pgfqpoint{1.481184in}{5.599589in}}%
\pgfpathlineto{\pgfqpoint{1.496279in}{5.614570in}}%
\pgfpathlineto{\pgfqpoint{1.510980in}{5.629009in}}%
\pgfpathlineto{\pgfqpoint{1.522305in}{5.640039in}}%
\pgfusepath{stroke}%
\end{pgfscope}%
\begin{pgfscope}%
\pgfpathrectangle{\pgfqpoint{0.766095in}{0.571603in}}{\pgfqpoint{5.929283in}{5.068436in}}%
\pgfusepath{clip}%
\pgfsetbuttcap%
\pgfsetroundjoin%
\pgfsetlinewidth{1.505625pt}%
\definecolor{currentstroke}{rgb}{0.120638,0.625828,0.533488}%
\pgfsetstrokecolor{currentstroke}%
\pgfsetdash{}{0pt}%
\pgfpathmoveto{\pgfqpoint{1.989030in}{0.571603in}}%
\pgfpathlineto{\pgfqpoint{1.987706in}{0.572611in}}%
\pgfpathlineto{\pgfqpoint{1.957911in}{0.595336in}}%
\pgfpathlineto{\pgfqpoint{1.955638in}{0.597073in}}%
\pgfpathlineto{\pgfqpoint{1.928115in}{0.618365in}}%
\pgfpathlineto{\pgfqpoint{1.922727in}{0.622542in}}%
\pgfpathlineto{\pgfqpoint{1.898320in}{0.641691in}}%
\pgfpathlineto{\pgfqpoint{1.890281in}{0.648012in}}%
\pgfpathlineto{\pgfqpoint{1.868524in}{0.665324in}}%
\pgfpathlineto{\pgfqpoint{1.858297in}{0.673481in}}%
\pgfpathlineto{\pgfqpoint{1.838729in}{0.689276in}}%
\pgfpathlineto{\pgfqpoint{1.826771in}{0.698951in}}%
\pgfpathlineto{\pgfqpoint{1.808934in}{0.713557in}}%
\pgfpathlineto{\pgfqpoint{1.795701in}{0.724420in}}%
\pgfpathlineto{\pgfqpoint{1.779138in}{0.738181in}}%
\pgfpathlineto{\pgfqpoint{1.765082in}{0.749890in}}%
\pgfpathlineto{\pgfqpoint{1.749343in}{0.763158in}}%
\pgfpathlineto{\pgfqpoint{1.734910in}{0.775360in}}%
\pgfpathlineto{\pgfqpoint{1.719547in}{0.788502in}}%
\pgfpathlineto{\pgfqpoint{1.705181in}{0.800829in}}%
\pgfpathlineto{\pgfqpoint{1.689752in}{0.814226in}}%
\pgfpathlineto{\pgfqpoint{1.675891in}{0.826299in}}%
\pgfpathlineto{\pgfqpoint{1.659957in}{0.840344in}}%
\pgfpathlineto{\pgfqpoint{1.647035in}{0.851768in}}%
\pgfpathlineto{\pgfqpoint{1.630161in}{0.866868in}}%
\pgfpathlineto{\pgfqpoint{1.618610in}{0.877238in}}%
\pgfpathlineto{\pgfqpoint{1.600366in}{0.893814in}}%
\pgfpathlineto{\pgfqpoint{1.590610in}{0.902707in}}%
\pgfpathlineto{\pgfqpoint{1.570571in}{0.921196in}}%
\pgfpathlineto{\pgfqpoint{1.563030in}{0.928177in}}%
\pgfpathlineto{\pgfqpoint{1.540775in}{0.949030in}}%
\pgfpathlineto{\pgfqpoint{1.535866in}{0.953646in}}%
\pgfpathlineto{\pgfqpoint{1.510980in}{0.977332in}}%
\pgfpathlineto{\pgfqpoint{1.509112in}{0.979116in}}%
\pgfpathlineto{\pgfqpoint{1.482780in}{1.004585in}}%
\pgfpathlineto{\pgfqpoint{1.481184in}{1.006148in}}%
\pgfpathlineto{\pgfqpoint{1.456869in}{1.030055in}}%
\pgfpathlineto{\pgfqpoint{1.451389in}{1.035510in}}%
\pgfpathlineto{\pgfqpoint{1.431358in}{1.055524in}}%
\pgfpathlineto{\pgfqpoint{1.421594in}{1.065401in}}%
\pgfpathlineto{\pgfqpoint{1.406239in}{1.080994in}}%
\pgfpathlineto{\pgfqpoint{1.391798in}{1.095841in}}%
\pgfpathlineto{\pgfqpoint{1.381507in}{1.106463in}}%
\pgfpathlineto{\pgfqpoint{1.362003in}{1.126848in}}%
\pgfpathlineto{\pgfqpoint{1.357157in}{1.131933in}}%
\pgfpathlineto{\pgfqpoint{1.333192in}{1.157402in}}%
\pgfpathlineto{\pgfqpoint{1.332207in}{1.158464in}}%
\pgfpathlineto{\pgfqpoint{1.309650in}{1.182872in}}%
\pgfpathlineto{\pgfqpoint{1.302412in}{1.190803in}}%
\pgfpathlineto{\pgfqpoint{1.286475in}{1.208341in}}%
\pgfpathlineto{\pgfqpoint{1.272617in}{1.223786in}}%
\pgfpathlineto{\pgfqpoint{1.263661in}{1.233811in}}%
\pgfpathlineto{\pgfqpoint{1.242821in}{1.257436in}}%
\pgfpathlineto{\pgfqpoint{1.241202in}{1.259281in}}%
\pgfpathlineto{\pgfqpoint{1.219153in}{1.284750in}}%
\pgfpathlineto{\pgfqpoint{1.213026in}{1.291922in}}%
\pgfpathlineto{\pgfqpoint{1.197467in}{1.310220in}}%
\pgfpathlineto{\pgfqpoint{1.183230in}{1.327179in}}%
\pgfpathlineto{\pgfqpoint{1.176120in}{1.335689in}}%
\pgfpathlineto{\pgfqpoint{1.155123in}{1.361159in}}%
\pgfpathlineto{\pgfqpoint{1.153435in}{1.363237in}}%
\pgfpathlineto{\pgfqpoint{1.134529in}{1.386628in}}%
\pgfpathlineto{\pgfqpoint{1.123640in}{1.400279in}}%
\pgfpathlineto{\pgfqpoint{1.114258in}{1.412098in}}%
\pgfpathlineto{\pgfqpoint{1.094308in}{1.437567in}}%
\pgfpathlineto{\pgfqpoint{1.093844in}{1.438169in}}%
\pgfpathlineto{\pgfqpoint{1.074768in}{1.463037in}}%
\pgfpathlineto{\pgfqpoint{1.064049in}{1.477198in}}%
\pgfpathlineto{\pgfqpoint{1.055534in}{1.488506in}}%
\pgfpathlineto{\pgfqpoint{1.036623in}{1.513976in}}%
\pgfpathlineto{\pgfqpoint{1.034254in}{1.517218in}}%
\pgfpathlineto{\pgfqpoint{1.018097in}{1.539445in}}%
\pgfpathlineto{\pgfqpoint{1.004458in}{1.558465in}}%
\pgfpathlineto{\pgfqpoint{0.999859in}{1.564915in}}%
\pgfpathlineto{\pgfqpoint{0.981978in}{1.590384in}}%
\pgfpathlineto{\pgfqpoint{0.974663in}{1.600962in}}%
\pgfpathlineto{\pgfqpoint{0.964423in}{1.615854in}}%
\pgfpathlineto{\pgfqpoint{0.947162in}{1.641323in}}%
\pgfpathlineto{\pgfqpoint{0.944867in}{1.644767in}}%
\pgfpathlineto{\pgfqpoint{0.930275in}{1.666793in}}%
\pgfpathlineto{\pgfqpoint{0.915072in}{1.690068in}}%
\pgfpathlineto{\pgfqpoint{0.913647in}{1.692262in}}%
\pgfpathlineto{\pgfqpoint{0.897398in}{1.717732in}}%
\pgfpathlineto{\pgfqpoint{0.885277in}{1.737015in}}%
\pgfpathlineto{\pgfqpoint{0.881411in}{1.743202in}}%
\pgfpathlineto{\pgfqpoint{0.865773in}{1.768671in}}%
\pgfpathlineto{\pgfqpoint{0.855481in}{1.785701in}}%
\pgfpathlineto{\pgfqpoint{0.850412in}{1.794141in}}%
\pgfpathlineto{\pgfqpoint{0.835382in}{1.819610in}}%
\pgfpathlineto{\pgfqpoint{0.825686in}{1.836315in}}%
\pgfpathlineto{\pgfqpoint{0.820630in}{1.845080in}}%
\pgfpathlineto{\pgfqpoint{0.806205in}{1.870549in}}%
\pgfpathlineto{\pgfqpoint{0.795890in}{1.889069in}}%
\pgfpathlineto{\pgfqpoint{0.792045in}{1.896019in}}%
\pgfpathlineto{\pgfqpoint{0.778219in}{1.921488in}}%
\pgfpathlineto{\pgfqpoint{0.766095in}{1.944203in}}%
\pgfusepath{stroke}%
\end{pgfscope}%
\begin{pgfscope}%
\pgfpathrectangle{\pgfqpoint{0.766095in}{0.571603in}}{\pgfqpoint{5.929283in}{5.068436in}}%
\pgfusepath{clip}%
\pgfsetbuttcap%
\pgfsetroundjoin%
\pgfsetlinewidth{1.505625pt}%
\definecolor{currentstroke}{rgb}{0.120638,0.625828,0.533488}%
\pgfsetstrokecolor{currentstroke}%
\pgfsetdash{}{0pt}%
\pgfpathmoveto{\pgfqpoint{0.766095in}{4.749552in}}%
\pgfpathlineto{\pgfqpoint{0.778995in}{4.774075in}}%
\pgfpathlineto{\pgfqpoint{0.792605in}{4.799545in}}%
\pgfpathlineto{\pgfqpoint{0.795890in}{4.805590in}}%
\pgfpathlineto{\pgfqpoint{0.806557in}{4.825014in}}%
\pgfpathlineto{\pgfqpoint{0.820761in}{4.850484in}}%
\pgfpathlineto{\pgfqpoint{0.825686in}{4.859174in}}%
\pgfpathlineto{\pgfqpoint{0.835293in}{4.875953in}}%
\pgfpathlineto{\pgfqpoint{0.850097in}{4.901423in}}%
\pgfpathlineto{\pgfqpoint{0.855481in}{4.910543in}}%
\pgfpathlineto{\pgfqpoint{0.865231in}{4.926892in}}%
\pgfpathlineto{\pgfqpoint{0.880641in}{4.952362in}}%
\pgfpathlineto{\pgfqpoint{0.885277in}{4.959908in}}%
\pgfpathlineto{\pgfqpoint{0.896397in}{4.977831in}}%
\pgfpathlineto{\pgfqpoint{0.912419in}{5.003301in}}%
\pgfpathlineto{\pgfqpoint{0.915072in}{5.007455in}}%
\pgfpathlineto{\pgfqpoint{0.928818in}{5.028770in}}%
\pgfpathlineto{\pgfqpoint{0.944867in}{5.053335in}}%
\pgfpathlineto{\pgfqpoint{0.945465in}{5.054240in}}%
\pgfpathlineto{\pgfqpoint{0.962520in}{5.079709in}}%
\pgfpathlineto{\pgfqpoint{0.974663in}{5.097612in}}%
\pgfpathlineto{\pgfqpoint{0.979845in}{5.105179in}}%
\pgfpathlineto{\pgfqpoint{0.997528in}{5.130649in}}%
\pgfpathlineto{\pgfqpoint{1.004458in}{5.140499in}}%
\pgfpathlineto{\pgfqpoint{1.015553in}{5.156118in}}%
\pgfpathlineto{\pgfqpoint{1.033867in}{5.181588in}}%
\pgfpathlineto{\pgfqpoint{1.034254in}{5.182118in}}%
\pgfpathlineto{\pgfqpoint{1.052611in}{5.207057in}}%
\pgfpathlineto{\pgfqpoint{1.064049in}{5.222411in}}%
\pgfpathlineto{\pgfqpoint{1.071656in}{5.232527in}}%
\pgfpathlineto{\pgfqpoint{1.091045in}{5.257996in}}%
\pgfpathlineto{\pgfqpoint{1.093844in}{5.261626in}}%
\pgfpathlineto{\pgfqpoint{1.110844in}{5.283466in}}%
\pgfpathlineto{\pgfqpoint{1.123640in}{5.299715in}}%
\pgfpathlineto{\pgfqpoint{1.130968in}{5.308935in}}%
\pgfpathlineto{\pgfqpoint{1.151450in}{5.334405in}}%
\pgfpathlineto{\pgfqpoint{1.153435in}{5.336842in}}%
\pgfpathlineto{\pgfqpoint{1.172363in}{5.359874in}}%
\pgfpathlineto{\pgfqpoint{1.183230in}{5.372950in}}%
\pgfpathlineto{\pgfqpoint{1.193625in}{5.385344in}}%
\pgfpathlineto{\pgfqpoint{1.213026in}{5.408221in}}%
\pgfpathlineto{\pgfqpoint{1.215245in}{5.410813in}}%
\pgfpathlineto{\pgfqpoint{1.237302in}{5.436283in}}%
\pgfpathlineto{\pgfqpoint{1.242821in}{5.442584in}}%
\pgfpathlineto{\pgfqpoint{1.259761in}{5.461752in}}%
\pgfpathlineto{\pgfqpoint{1.272617in}{5.476143in}}%
\pgfpathlineto{\pgfqpoint{1.282602in}{5.487222in}}%
\pgfpathlineto{\pgfqpoint{1.302412in}{5.508968in}}%
\pgfpathlineto{\pgfqpoint{1.305834in}{5.512691in}}%
\pgfpathlineto{\pgfqpoint{1.329504in}{5.538161in}}%
\pgfpathlineto{\pgfqpoint{1.332207in}{5.541038in}}%
\pgfpathlineto{\pgfqpoint{1.353623in}{5.563630in}}%
\pgfpathlineto{\pgfqpoint{1.362003in}{5.572378in}}%
\pgfpathlineto{\pgfqpoint{1.378159in}{5.589100in}}%
\pgfpathlineto{\pgfqpoint{1.391798in}{5.603070in}}%
\pgfpathlineto{\pgfqpoint{1.403121in}{5.614570in}}%
\pgfpathlineto{\pgfqpoint{1.421594in}{5.633136in}}%
\pgfpathlineto{\pgfqpoint{1.428519in}{5.640039in}}%
\pgfusepath{stroke}%
\end{pgfscope}%
\begin{pgfscope}%
\pgfpathrectangle{\pgfqpoint{0.766095in}{0.571603in}}{\pgfqpoint{5.929283in}{5.068436in}}%
\pgfusepath{clip}%
\pgfsetbuttcap%
\pgfsetroundjoin%
\pgfsetlinewidth{1.505625pt}%
\definecolor{currentstroke}{rgb}{0.128087,0.647749,0.523491}%
\pgfsetstrokecolor{currentstroke}%
\pgfsetdash{}{0pt}%
\pgfpathmoveto{\pgfqpoint{1.915346in}{0.571603in}}%
\pgfpathlineto{\pgfqpoint{1.898320in}{0.584668in}}%
\pgfpathlineto{\pgfqpoint{1.882188in}{0.597073in}}%
\pgfpathlineto{\pgfqpoint{1.868524in}{0.607704in}}%
\pgfpathlineto{\pgfqpoint{1.849496in}{0.622542in}}%
\pgfpathlineto{\pgfqpoint{1.838729in}{0.631038in}}%
\pgfpathlineto{\pgfqpoint{1.817267in}{0.648012in}}%
\pgfpathlineto{\pgfqpoint{1.808934in}{0.654681in}}%
\pgfpathlineto{\pgfqpoint{1.785498in}{0.673481in}}%
\pgfpathlineto{\pgfqpoint{1.779138in}{0.678644in}}%
\pgfpathlineto{\pgfqpoint{1.754186in}{0.698951in}}%
\pgfpathlineto{\pgfqpoint{1.749343in}{0.702940in}}%
\pgfpathlineto{\pgfqpoint{1.723327in}{0.724420in}}%
\pgfpathlineto{\pgfqpoint{1.719547in}{0.727578in}}%
\pgfpathlineto{\pgfqpoint{1.692917in}{0.749890in}}%
\pgfpathlineto{\pgfqpoint{1.689752in}{0.752573in}}%
\pgfpathlineto{\pgfqpoint{1.662952in}{0.775360in}}%
\pgfpathlineto{\pgfqpoint{1.659957in}{0.777936in}}%
\pgfpathlineto{\pgfqpoint{1.633427in}{0.800829in}}%
\pgfpathlineto{\pgfqpoint{1.630161in}{0.803681in}}%
\pgfpathlineto{\pgfqpoint{1.604340in}{0.826299in}}%
\pgfpathlineto{\pgfqpoint{1.600366in}{0.829821in}}%
\pgfpathlineto{\pgfqpoint{1.575684in}{0.851768in}}%
\pgfpathlineto{\pgfqpoint{1.570571in}{0.856370in}}%
\pgfpathlineto{\pgfqpoint{1.547456in}{0.877238in}}%
\pgfpathlineto{\pgfqpoint{1.540775in}{0.883342in}}%
\pgfpathlineto{\pgfqpoint{1.519651in}{0.902707in}}%
\pgfpathlineto{\pgfqpoint{1.510980in}{0.910753in}}%
\pgfpathlineto{\pgfqpoint{1.492264in}{0.928177in}}%
\pgfpathlineto{\pgfqpoint{1.481184in}{0.938617in}}%
\pgfpathlineto{\pgfqpoint{1.465290in}{0.953646in}}%
\pgfpathlineto{\pgfqpoint{1.451389in}{0.966951in}}%
\pgfpathlineto{\pgfqpoint{1.438724in}{0.979116in}}%
\pgfpathlineto{\pgfqpoint{1.421594in}{0.995770in}}%
\pgfpathlineto{\pgfqpoint{1.412560in}{1.004585in}}%
\pgfpathlineto{\pgfqpoint{1.391798in}{1.025093in}}%
\pgfpathlineto{\pgfqpoint{1.386793in}{1.030055in}}%
\pgfpathlineto{\pgfqpoint{1.362003in}{1.054936in}}%
\pgfpathlineto{\pgfqpoint{1.361418in}{1.055524in}}%
\pgfpathlineto{\pgfqpoint{1.336472in}{1.080994in}}%
\pgfpathlineto{\pgfqpoint{1.332207in}{1.085402in}}%
\pgfpathlineto{\pgfqpoint{1.311916in}{1.106463in}}%
\pgfpathlineto{\pgfqpoint{1.302412in}{1.116450in}}%
\pgfpathlineto{\pgfqpoint{1.287738in}{1.131933in}}%
\pgfpathlineto{\pgfqpoint{1.272617in}{1.148087in}}%
\pgfpathlineto{\pgfqpoint{1.263933in}{1.157402in}}%
\pgfpathlineto{\pgfqpoint{1.242821in}{1.180335in}}%
\pgfpathlineto{\pgfqpoint{1.240495in}{1.182872in}}%
\pgfpathlineto{\pgfqpoint{1.217462in}{1.208341in}}%
\pgfpathlineto{\pgfqpoint{1.213026in}{1.213312in}}%
\pgfpathlineto{\pgfqpoint{1.194811in}{1.233811in}}%
\pgfpathlineto{\pgfqpoint{1.183230in}{1.247009in}}%
\pgfpathlineto{\pgfqpoint{1.172511in}{1.259281in}}%
\pgfpathlineto{\pgfqpoint{1.153435in}{1.281399in}}%
\pgfpathlineto{\pgfqpoint{1.150558in}{1.284750in}}%
\pgfpathlineto{\pgfqpoint{1.128997in}{1.310220in}}%
\pgfpathlineto{\pgfqpoint{1.123640in}{1.316635in}}%
\pgfpathlineto{\pgfqpoint{1.107803in}{1.335689in}}%
\pgfpathlineto{\pgfqpoint{1.093844in}{1.352701in}}%
\pgfpathlineto{\pgfqpoint{1.086938in}{1.361159in}}%
\pgfpathlineto{\pgfqpoint{1.066421in}{1.386628in}}%
\pgfpathlineto{\pgfqpoint{1.064049in}{1.389617in}}%
\pgfpathlineto{\pgfqpoint{1.046295in}{1.412098in}}%
\pgfpathlineto{\pgfqpoint{1.034254in}{1.427545in}}%
\pgfpathlineto{\pgfqpoint{1.026482in}{1.437567in}}%
\pgfpathlineto{\pgfqpoint{1.007001in}{1.463037in}}%
\pgfpathlineto{\pgfqpoint{1.004458in}{1.466413in}}%
\pgfpathlineto{\pgfqpoint{0.987904in}{1.488506in}}%
\pgfpathlineto{\pgfqpoint{0.974663in}{1.506416in}}%
\pgfpathlineto{\pgfqpoint{0.969103in}{1.513976in}}%
\pgfpathlineto{\pgfqpoint{0.950651in}{1.539445in}}%
\pgfpathlineto{\pgfqpoint{0.944867in}{1.547550in}}%
\pgfpathlineto{\pgfqpoint{0.932542in}{1.564915in}}%
\pgfpathlineto{\pgfqpoint{0.915072in}{1.589868in}}%
\pgfpathlineto{\pgfqpoint{0.914712in}{1.590384in}}%
\pgfpathlineto{\pgfqpoint{0.897277in}{1.615854in}}%
\pgfpathlineto{\pgfqpoint{0.885277in}{1.633632in}}%
\pgfpathlineto{\pgfqpoint{0.880114in}{1.641323in}}%
\pgfpathlineto{\pgfqpoint{0.863293in}{1.666793in}}%
\pgfpathlineto{\pgfqpoint{0.855481in}{1.678805in}}%
\pgfpathlineto{\pgfqpoint{0.846781in}{1.692262in}}%
\pgfpathlineto{\pgfqpoint{0.830570in}{1.717732in}}%
\pgfpathlineto{\pgfqpoint{0.825686in}{1.725536in}}%
\pgfpathlineto{\pgfqpoint{0.814696in}{1.743202in}}%
\pgfpathlineto{\pgfqpoint{0.799092in}{1.768671in}}%
\pgfpathlineto{\pgfqpoint{0.795890in}{1.773992in}}%
\pgfpathlineto{\pgfqpoint{0.783838in}{1.794141in}}%
\pgfpathlineto{\pgfqpoint{0.768840in}{1.819610in}}%
\pgfpathlineto{\pgfqpoint{0.766095in}{1.824357in}}%
\pgfusepath{stroke}%
\end{pgfscope}%
\begin{pgfscope}%
\pgfpathrectangle{\pgfqpoint{0.766095in}{0.571603in}}{\pgfqpoint{5.929283in}{5.068436in}}%
\pgfusepath{clip}%
\pgfsetbuttcap%
\pgfsetroundjoin%
\pgfsetlinewidth{1.505625pt}%
\definecolor{currentstroke}{rgb}{0.128087,0.647749,0.523491}%
\pgfsetstrokecolor{currentstroke}%
\pgfsetdash{}{0pt}%
\pgfpathmoveto{\pgfqpoint{0.766095in}{4.890780in}}%
\pgfpathlineto{\pgfqpoint{0.772193in}{4.901423in}}%
\pgfpathlineto{\pgfqpoint{0.787014in}{4.926892in}}%
\pgfpathlineto{\pgfqpoint{0.795890in}{4.941921in}}%
\pgfpathlineto{\pgfqpoint{0.802119in}{4.952362in}}%
\pgfpathlineto{\pgfqpoint{0.817541in}{4.977831in}}%
\pgfpathlineto{\pgfqpoint{0.825686in}{4.991088in}}%
\pgfpathlineto{\pgfqpoint{0.833263in}{5.003301in}}%
\pgfpathlineto{\pgfqpoint{0.849293in}{5.028770in}}%
\pgfpathlineto{\pgfqpoint{0.855481in}{5.038463in}}%
\pgfpathlineto{\pgfqpoint{0.865653in}{5.054240in}}%
\pgfpathlineto{\pgfqpoint{0.882295in}{5.079709in}}%
\pgfpathlineto{\pgfqpoint{0.885277in}{5.084208in}}%
\pgfpathlineto{\pgfqpoint{0.899312in}{5.105179in}}%
\pgfpathlineto{\pgfqpoint{0.915072in}{5.128431in}}%
\pgfpathlineto{\pgfqpoint{0.916590in}{5.130649in}}%
\pgfpathlineto{\pgfqpoint{0.934267in}{5.156118in}}%
\pgfpathlineto{\pgfqpoint{0.944867in}{5.171200in}}%
\pgfpathlineto{\pgfqpoint{0.952238in}{5.181588in}}%
\pgfpathlineto{\pgfqpoint{0.970542in}{5.207057in}}%
\pgfpathlineto{\pgfqpoint{0.974663in}{5.212716in}}%
\pgfpathlineto{\pgfqpoint{0.989225in}{5.232527in}}%
\pgfpathlineto{\pgfqpoint{1.004458in}{5.253004in}}%
\pgfpathlineto{\pgfqpoint{1.008206in}{5.257996in}}%
\pgfpathlineto{\pgfqpoint{1.027576in}{5.283466in}}%
\pgfpathlineto{\pgfqpoint{1.034254in}{5.292138in}}%
\pgfpathlineto{\pgfqpoint{1.047304in}{5.308935in}}%
\pgfpathlineto{\pgfqpoint{1.064049in}{5.330239in}}%
\pgfpathlineto{\pgfqpoint{1.067353in}{5.334405in}}%
\pgfpathlineto{\pgfqpoint{1.087808in}{5.359874in}}%
\pgfpathlineto{\pgfqpoint{1.093844in}{5.367302in}}%
\pgfpathlineto{\pgfqpoint{1.108638in}{5.385344in}}%
\pgfpathlineto{\pgfqpoint{1.123640in}{5.403434in}}%
\pgfpathlineto{\pgfqpoint{1.129813in}{5.410813in}}%
\pgfpathlineto{\pgfqpoint{1.151368in}{5.436283in}}%
\pgfpathlineto{\pgfqpoint{1.153435in}{5.438696in}}%
\pgfpathlineto{\pgfqpoint{1.173358in}{5.461752in}}%
\pgfpathlineto{\pgfqpoint{1.183230in}{5.473053in}}%
\pgfpathlineto{\pgfqpoint{1.195716in}{5.487222in}}%
\pgfpathlineto{\pgfqpoint{1.213026in}{5.506652in}}%
\pgfpathlineto{\pgfqpoint{1.218452in}{5.512691in}}%
\pgfpathlineto{\pgfqpoint{1.241591in}{5.538161in}}%
\pgfpathlineto{\pgfqpoint{1.242821in}{5.539500in}}%
\pgfpathlineto{\pgfqpoint{1.265189in}{5.563630in}}%
\pgfpathlineto{\pgfqpoint{1.272617in}{5.571559in}}%
\pgfpathlineto{\pgfqpoint{1.289189in}{5.589100in}}%
\pgfpathlineto{\pgfqpoint{1.302412in}{5.602950in}}%
\pgfpathlineto{\pgfqpoint{1.313599in}{5.614570in}}%
\pgfpathlineto{\pgfqpoint{1.332207in}{5.633696in}}%
\pgfpathlineto{\pgfqpoint{1.338430in}{5.640039in}}%
\pgfusepath{stroke}%
\end{pgfscope}%
\begin{pgfscope}%
\pgfpathrectangle{\pgfqpoint{0.766095in}{0.571603in}}{\pgfqpoint{5.929283in}{5.068436in}}%
\pgfusepath{clip}%
\pgfsetbuttcap%
\pgfsetroundjoin%
\pgfsetlinewidth{1.505625pt}%
\definecolor{currentstroke}{rgb}{0.140210,0.665859,0.513427}%
\pgfsetstrokecolor{currentstroke}%
\pgfsetdash{}{0pt}%
\pgfpathmoveto{\pgfqpoint{1.843309in}{0.571603in}}%
\pgfpathlineto{\pgfqpoint{1.838729in}{0.575138in}}%
\pgfpathlineto{\pgfqpoint{1.810368in}{0.597073in}}%
\pgfpathlineto{\pgfqpoint{1.808934in}{0.598196in}}%
\pgfpathlineto{\pgfqpoint{1.779138in}{0.621571in}}%
\pgfpathlineto{\pgfqpoint{1.777904in}{0.622542in}}%
\pgfpathlineto{\pgfqpoint{1.749343in}{0.645273in}}%
\pgfpathlineto{\pgfqpoint{1.745910in}{0.648012in}}%
\pgfpathlineto{\pgfqpoint{1.719547in}{0.669291in}}%
\pgfpathlineto{\pgfqpoint{1.714369in}{0.673481in}}%
\pgfpathlineto{\pgfqpoint{1.689752in}{0.693637in}}%
\pgfpathlineto{\pgfqpoint{1.683279in}{0.698951in}}%
\pgfpathlineto{\pgfqpoint{1.659957in}{0.718323in}}%
\pgfpathlineto{\pgfqpoint{1.652635in}{0.724420in}}%
\pgfpathlineto{\pgfqpoint{1.630161in}{0.743360in}}%
\pgfpathlineto{\pgfqpoint{1.622434in}{0.749890in}}%
\pgfpathlineto{\pgfqpoint{1.600366in}{0.768761in}}%
\pgfpathlineto{\pgfqpoint{1.592671in}{0.775360in}}%
\pgfpathlineto{\pgfqpoint{1.570571in}{0.794538in}}%
\pgfpathlineto{\pgfqpoint{1.563343in}{0.800829in}}%
\pgfpathlineto{\pgfqpoint{1.540775in}{0.820706in}}%
\pgfpathlineto{\pgfqpoint{1.534444in}{0.826299in}}%
\pgfpathlineto{\pgfqpoint{1.510980in}{0.847276in}}%
\pgfpathlineto{\pgfqpoint{1.505971in}{0.851768in}}%
\pgfpathlineto{\pgfqpoint{1.481184in}{0.874265in}}%
\pgfpathlineto{\pgfqpoint{1.477919in}{0.877238in}}%
\pgfpathlineto{\pgfqpoint{1.451389in}{0.901685in}}%
\pgfpathlineto{\pgfqpoint{1.450283in}{0.902707in}}%
\pgfpathlineto{\pgfqpoint{1.423073in}{0.928177in}}%
\pgfpathlineto{\pgfqpoint{1.421594in}{0.929578in}}%
\pgfpathlineto{\pgfqpoint{1.396283in}{0.953646in}}%
\pgfpathlineto{\pgfqpoint{1.391798in}{0.957963in}}%
\pgfpathlineto{\pgfqpoint{1.369899in}{0.979116in}}%
\pgfpathlineto{\pgfqpoint{1.362003in}{0.986836in}}%
\pgfpathlineto{\pgfqpoint{1.343915in}{1.004585in}}%
\pgfpathlineto{\pgfqpoint{1.332207in}{1.016214in}}%
\pgfpathlineto{\pgfqpoint{1.318325in}{1.030055in}}%
\pgfpathlineto{\pgfqpoint{1.302412in}{1.046114in}}%
\pgfpathlineto{\pgfqpoint{1.293124in}{1.055524in}}%
\pgfpathlineto{\pgfqpoint{1.272617in}{1.076556in}}%
\pgfpathlineto{\pgfqpoint{1.268307in}{1.080994in}}%
\pgfpathlineto{\pgfqpoint{1.243877in}{1.106463in}}%
\pgfpathlineto{\pgfqpoint{1.242821in}{1.107579in}}%
\pgfpathlineto{\pgfqpoint{1.219866in}{1.131933in}}%
\pgfpathlineto{\pgfqpoint{1.213026in}{1.139280in}}%
\pgfpathlineto{\pgfqpoint{1.196225in}{1.157402in}}%
\pgfpathlineto{\pgfqpoint{1.183230in}{1.171594in}}%
\pgfpathlineto{\pgfqpoint{1.172948in}{1.182872in}}%
\pgfpathlineto{\pgfqpoint{1.153435in}{1.204542in}}%
\pgfpathlineto{\pgfqpoint{1.150029in}{1.208341in}}%
\pgfpathlineto{\pgfqpoint{1.127500in}{1.233811in}}%
\pgfpathlineto{\pgfqpoint{1.123640in}{1.238234in}}%
\pgfpathlineto{\pgfqpoint{1.105354in}{1.259281in}}%
\pgfpathlineto{\pgfqpoint{1.093844in}{1.272696in}}%
\pgfpathlineto{\pgfqpoint{1.083550in}{1.284750in}}%
\pgfpathlineto{\pgfqpoint{1.064049in}{1.307878in}}%
\pgfpathlineto{\pgfqpoint{1.062084in}{1.310220in}}%
\pgfpathlineto{\pgfqpoint{1.041015in}{1.335689in}}%
\pgfpathlineto{\pgfqpoint{1.034254in}{1.343972in}}%
\pgfpathlineto{\pgfqpoint{1.020293in}{1.361159in}}%
\pgfpathlineto{\pgfqpoint{1.004458in}{1.380905in}}%
\pgfpathlineto{\pgfqpoint{0.999892in}{1.386628in}}%
\pgfpathlineto{\pgfqpoint{0.979856in}{1.412098in}}%
\pgfpathlineto{\pgfqpoint{0.974663in}{1.418794in}}%
\pgfpathlineto{\pgfqpoint{0.960178in}{1.437567in}}%
\pgfpathlineto{\pgfqpoint{0.944867in}{1.457673in}}%
\pgfpathlineto{\pgfqpoint{0.940804in}{1.463037in}}%
\pgfpathlineto{\pgfqpoint{0.921795in}{1.488506in}}%
\pgfpathlineto{\pgfqpoint{0.915072in}{1.497644in}}%
\pgfpathlineto{\pgfqpoint{0.903121in}{1.513976in}}%
\pgfpathlineto{\pgfqpoint{0.885277in}{1.538689in}}%
\pgfpathlineto{\pgfqpoint{0.884733in}{1.539445in}}%
\pgfpathlineto{\pgfqpoint{0.866741in}{1.564915in}}%
\pgfpathlineto{\pgfqpoint{0.855481in}{1.581074in}}%
\pgfpathlineto{\pgfqpoint{0.849029in}{1.590384in}}%
\pgfpathlineto{\pgfqpoint{0.831648in}{1.615854in}}%
\pgfpathlineto{\pgfqpoint{0.825686in}{1.624728in}}%
\pgfpathlineto{\pgfqpoint{0.814599in}{1.641323in}}%
\pgfpathlineto{\pgfqpoint{0.797828in}{1.666793in}}%
\pgfpathlineto{\pgfqpoint{0.795890in}{1.669786in}}%
\pgfpathlineto{\pgfqpoint{0.781425in}{1.692262in}}%
\pgfpathlineto{\pgfqpoint{0.766095in}{1.716424in}}%
\pgfusepath{stroke}%
\end{pgfscope}%
\begin{pgfscope}%
\pgfpathrectangle{\pgfqpoint{0.766095in}{0.571603in}}{\pgfqpoint{5.929283in}{5.068436in}}%
\pgfusepath{clip}%
\pgfsetbuttcap%
\pgfsetroundjoin%
\pgfsetlinewidth{1.505625pt}%
\definecolor{currentstroke}{rgb}{0.140210,0.665859,0.513427}%
\pgfsetstrokecolor{currentstroke}%
\pgfsetdash{}{0pt}%
\pgfpathmoveto{\pgfqpoint{0.766095in}{5.018752in}}%
\pgfpathlineto{\pgfqpoint{0.772313in}{5.028770in}}%
\pgfpathlineto{\pgfqpoint{0.788349in}{5.054240in}}%
\pgfpathlineto{\pgfqpoint{0.795890in}{5.066050in}}%
\pgfpathlineto{\pgfqpoint{0.804696in}{5.079709in}}%
\pgfpathlineto{\pgfqpoint{0.821340in}{5.105179in}}%
\pgfpathlineto{\pgfqpoint{0.825686in}{5.111735in}}%
\pgfpathlineto{\pgfqpoint{0.838340in}{5.130649in}}%
\pgfpathlineto{\pgfqpoint{0.855481in}{5.155948in}}%
\pgfpathlineto{\pgfqpoint{0.855598in}{5.156118in}}%
\pgfpathlineto{\pgfqpoint{0.873269in}{5.181588in}}%
\pgfpathlineto{\pgfqpoint{0.885277in}{5.198683in}}%
\pgfpathlineto{\pgfqpoint{0.891213in}{5.207057in}}%
\pgfpathlineto{\pgfqpoint{0.909506in}{5.232527in}}%
\pgfpathlineto{\pgfqpoint{0.915072in}{5.240176in}}%
\pgfpathlineto{\pgfqpoint{0.928157in}{5.257996in}}%
\pgfpathlineto{\pgfqpoint{0.944867in}{5.280485in}}%
\pgfpathlineto{\pgfqpoint{0.947102in}{5.283466in}}%
\pgfpathlineto{\pgfqpoint{0.966452in}{5.308935in}}%
\pgfpathlineto{\pgfqpoint{0.974663in}{5.319613in}}%
\pgfpathlineto{\pgfqpoint{0.986138in}{5.334405in}}%
\pgfpathlineto{\pgfqpoint{1.004458in}{5.357748in}}%
\pgfpathlineto{\pgfqpoint{1.006142in}{5.359874in}}%
\pgfpathlineto{\pgfqpoint{1.026567in}{5.385344in}}%
\pgfpathlineto{\pgfqpoint{1.034254in}{5.394819in}}%
\pgfpathlineto{\pgfqpoint{1.047344in}{5.410813in}}%
\pgfpathlineto{\pgfqpoint{1.064049in}{5.430997in}}%
\pgfpathlineto{\pgfqpoint{1.068462in}{5.436283in}}%
\pgfpathlineto{\pgfqpoint{1.089978in}{5.461752in}}%
\pgfpathlineto{\pgfqpoint{1.093844in}{5.466275in}}%
\pgfpathlineto{\pgfqpoint{1.111903in}{5.487222in}}%
\pgfpathlineto{\pgfqpoint{1.123640in}{5.500687in}}%
\pgfpathlineto{\pgfqpoint{1.134192in}{5.512691in}}%
\pgfpathlineto{\pgfqpoint{1.153435in}{5.534345in}}%
\pgfpathlineto{\pgfqpoint{1.156855in}{5.538161in}}%
\pgfpathlineto{\pgfqpoint{1.179940in}{5.563630in}}%
\pgfpathlineto{\pgfqpoint{1.183230in}{5.567219in}}%
\pgfpathlineto{\pgfqpoint{1.203456in}{5.589100in}}%
\pgfpathlineto{\pgfqpoint{1.213026in}{5.599344in}}%
\pgfpathlineto{\pgfqpoint{1.227367in}{5.614570in}}%
\pgfpathlineto{\pgfqpoint{1.242821in}{5.630803in}}%
\pgfpathlineto{\pgfqpoint{1.251685in}{5.640039in}}%
\pgfusepath{stroke}%
\end{pgfscope}%
\begin{pgfscope}%
\pgfpathrectangle{\pgfqpoint{0.766095in}{0.571603in}}{\pgfqpoint{5.929283in}{5.068436in}}%
\pgfusepath{clip}%
\pgfsetbuttcap%
\pgfsetroundjoin%
\pgfsetlinewidth{1.505625pt}%
\definecolor{currentstroke}{rgb}{0.162016,0.687316,0.499129}%
\pgfsetstrokecolor{currentstroke}%
\pgfsetdash{}{0pt}%
\pgfpathmoveto{\pgfqpoint{1.772892in}{0.571603in}}%
\pgfpathlineto{\pgfqpoint{1.749343in}{0.589934in}}%
\pgfpathlineto{\pgfqpoint{1.740193in}{0.597073in}}%
\pgfpathlineto{\pgfqpoint{1.719547in}{0.613368in}}%
\pgfpathlineto{\pgfqpoint{1.707952in}{0.622542in}}%
\pgfpathlineto{\pgfqpoint{1.689752in}{0.637110in}}%
\pgfpathlineto{\pgfqpoint{1.676166in}{0.648012in}}%
\pgfpathlineto{\pgfqpoint{1.659957in}{0.661171in}}%
\pgfpathlineto{\pgfqpoint{1.644832in}{0.673481in}}%
\pgfpathlineto{\pgfqpoint{1.630161in}{0.685562in}}%
\pgfpathlineto{\pgfqpoint{1.613945in}{0.698951in}}%
\pgfpathlineto{\pgfqpoint{1.600366in}{0.710295in}}%
\pgfpathlineto{\pgfqpoint{1.583503in}{0.724420in}}%
\pgfpathlineto{\pgfqpoint{1.570571in}{0.735381in}}%
\pgfpathlineto{\pgfqpoint{1.553501in}{0.749890in}}%
\pgfpathlineto{\pgfqpoint{1.540775in}{0.760834in}}%
\pgfpathlineto{\pgfqpoint{1.523935in}{0.775360in}}%
\pgfpathlineto{\pgfqpoint{1.510980in}{0.786666in}}%
\pgfpathlineto{\pgfqpoint{1.494800in}{0.800829in}}%
\pgfpathlineto{\pgfqpoint{1.481184in}{0.812890in}}%
\pgfpathlineto{\pgfqpoint{1.466093in}{0.826299in}}%
\pgfpathlineto{\pgfqpoint{1.451389in}{0.839519in}}%
\pgfpathlineto{\pgfqpoint{1.437809in}{0.851768in}}%
\pgfpathlineto{\pgfqpoint{1.421594in}{0.866569in}}%
\pgfpathlineto{\pgfqpoint{1.409944in}{0.877238in}}%
\pgfpathlineto{\pgfqpoint{1.391798in}{0.894053in}}%
\pgfpathlineto{\pgfqpoint{1.382491in}{0.902707in}}%
\pgfpathlineto{\pgfqpoint{1.362003in}{0.921987in}}%
\pgfpathlineto{\pgfqpoint{1.355448in}{0.928177in}}%
\pgfpathlineto{\pgfqpoint{1.332207in}{0.950386in}}%
\pgfpathlineto{\pgfqpoint{1.328808in}{0.953646in}}%
\pgfpathlineto{\pgfqpoint{1.302567in}{0.979116in}}%
\pgfpathlineto{\pgfqpoint{1.302412in}{0.979269in}}%
\pgfpathlineto{\pgfqpoint{1.276757in}{1.004585in}}%
\pgfpathlineto{\pgfqpoint{1.272617in}{1.008721in}}%
\pgfpathlineto{\pgfqpoint{1.251339in}{1.030055in}}%
\pgfpathlineto{\pgfqpoint{1.242821in}{1.038699in}}%
\pgfpathlineto{\pgfqpoint{1.226306in}{1.055524in}}%
\pgfpathlineto{\pgfqpoint{1.213026in}{1.069220in}}%
\pgfpathlineto{\pgfqpoint{1.201654in}{1.080994in}}%
\pgfpathlineto{\pgfqpoint{1.183230in}{1.100304in}}%
\pgfpathlineto{\pgfqpoint{1.177377in}{1.106463in}}%
\pgfpathlineto{\pgfqpoint{1.153470in}{1.131933in}}%
\pgfpathlineto{\pgfqpoint{1.153435in}{1.131970in}}%
\pgfpathlineto{\pgfqpoint{1.129986in}{1.157402in}}%
\pgfpathlineto{\pgfqpoint{1.123640in}{1.164372in}}%
\pgfpathlineto{\pgfqpoint{1.106864in}{1.182872in}}%
\pgfpathlineto{\pgfqpoint{1.093844in}{1.197410in}}%
\pgfpathlineto{\pgfqpoint{1.084097in}{1.208341in}}%
\pgfpathlineto{\pgfqpoint{1.064049in}{1.231108in}}%
\pgfpathlineto{\pgfqpoint{1.061679in}{1.233811in}}%
\pgfpathlineto{\pgfqpoint{1.039657in}{1.259281in}}%
\pgfpathlineto{\pgfqpoint{1.034254in}{1.265612in}}%
\pgfpathlineto{\pgfqpoint{1.017998in}{1.284750in}}%
\pgfpathlineto{\pgfqpoint{1.004458in}{1.300893in}}%
\pgfpathlineto{\pgfqpoint{0.996672in}{1.310220in}}%
\pgfpathlineto{\pgfqpoint{0.975684in}{1.335689in}}%
\pgfpathlineto{\pgfqpoint{0.974663in}{1.336947in}}%
\pgfpathlineto{\pgfqpoint{0.955098in}{1.361159in}}%
\pgfpathlineto{\pgfqpoint{0.944867in}{1.373983in}}%
\pgfpathlineto{\pgfqpoint{0.934830in}{1.386628in}}%
\pgfpathlineto{\pgfqpoint{0.915072in}{1.411843in}}%
\pgfpathlineto{\pgfqpoint{0.914873in}{1.412098in}}%
\pgfpathlineto{\pgfqpoint{0.895321in}{1.437567in}}%
\pgfpathlineto{\pgfqpoint{0.885277in}{1.450824in}}%
\pgfpathlineto{\pgfqpoint{0.876071in}{1.463037in}}%
\pgfpathlineto{\pgfqpoint{0.857133in}{1.488506in}}%
\pgfpathlineto{\pgfqpoint{0.855481in}{1.490763in}}%
\pgfpathlineto{\pgfqpoint{0.838579in}{1.513976in}}%
\pgfpathlineto{\pgfqpoint{0.825686in}{1.531920in}}%
\pgfpathlineto{\pgfqpoint{0.820309in}{1.539445in}}%
\pgfpathlineto{\pgfqpoint{0.802381in}{1.564915in}}%
\pgfpathlineto{\pgfqpoint{0.795890in}{1.574275in}}%
\pgfpathlineto{\pgfqpoint{0.784781in}{1.590384in}}%
\pgfpathlineto{\pgfqpoint{0.767463in}{1.615854in}}%
\pgfpathlineto{\pgfqpoint{0.766095in}{1.617899in}}%
\pgfusepath{stroke}%
\end{pgfscope}%
\begin{pgfscope}%
\pgfpathrectangle{\pgfqpoint{0.766095in}{0.571603in}}{\pgfqpoint{5.929283in}{5.068436in}}%
\pgfusepath{clip}%
\pgfsetbuttcap%
\pgfsetroundjoin%
\pgfsetlinewidth{1.505625pt}%
\definecolor{currentstroke}{rgb}{0.162016,0.687316,0.499129}%
\pgfsetstrokecolor{currentstroke}%
\pgfsetdash{}{0pt}%
\pgfpathmoveto{\pgfqpoint{0.766095in}{5.136143in}}%
\pgfpathlineto{\pgfqpoint{0.779459in}{5.156118in}}%
\pgfpathlineto{\pgfqpoint{0.795890in}{5.180372in}}%
\pgfpathlineto{\pgfqpoint{0.796721in}{5.181588in}}%
\pgfpathlineto{\pgfqpoint{0.814385in}{5.207057in}}%
\pgfpathlineto{\pgfqpoint{0.825686in}{5.223152in}}%
\pgfpathlineto{\pgfqpoint{0.832328in}{5.232527in}}%
\pgfpathlineto{\pgfqpoint{0.850608in}{5.257996in}}%
\pgfpathlineto{\pgfqpoint{0.855481in}{5.264697in}}%
\pgfpathlineto{\pgfqpoint{0.869252in}{5.283466in}}%
\pgfpathlineto{\pgfqpoint{0.885277in}{5.305049in}}%
\pgfpathlineto{\pgfqpoint{0.888188in}{5.308935in}}%
\pgfpathlineto{\pgfqpoint{0.907516in}{5.334405in}}%
\pgfpathlineto{\pgfqpoint{0.915072in}{5.344242in}}%
\pgfpathlineto{\pgfqpoint{0.927185in}{5.359874in}}%
\pgfpathlineto{\pgfqpoint{0.944867in}{5.382432in}}%
\pgfpathlineto{\pgfqpoint{0.947170in}{5.385344in}}%
\pgfpathlineto{\pgfqpoint{0.967564in}{5.410813in}}%
\pgfpathlineto{\pgfqpoint{0.974663in}{5.419576in}}%
\pgfpathlineto{\pgfqpoint{0.988314in}{5.436283in}}%
\pgfpathlineto{\pgfqpoint{1.004458in}{5.455820in}}%
\pgfpathlineto{\pgfqpoint{1.009403in}{5.461752in}}%
\pgfpathlineto{\pgfqpoint{1.030878in}{5.487222in}}%
\pgfpathlineto{\pgfqpoint{1.034254in}{5.491177in}}%
\pgfpathlineto{\pgfqpoint{1.052765in}{5.512691in}}%
\pgfpathlineto{\pgfqpoint{1.064049in}{5.525663in}}%
\pgfpathlineto{\pgfqpoint{1.075012in}{5.538161in}}%
\pgfpathlineto{\pgfqpoint{1.093844in}{5.559398in}}%
\pgfpathlineto{\pgfqpoint{1.097629in}{5.563630in}}%
\pgfpathlineto{\pgfqpoint{1.120660in}{5.589100in}}%
\pgfpathlineto{\pgfqpoint{1.123640in}{5.592357in}}%
\pgfpathlineto{\pgfqpoint{1.144121in}{5.614570in}}%
\pgfpathlineto{\pgfqpoint{1.153435in}{5.624564in}}%
\pgfpathlineto{\pgfqpoint{1.167973in}{5.640039in}}%
\pgfusepath{stroke}%
\end{pgfscope}%
\begin{pgfscope}%
\pgfpathrectangle{\pgfqpoint{0.766095in}{0.571603in}}{\pgfqpoint{5.929283in}{5.068436in}}%
\pgfusepath{clip}%
\pgfsetbuttcap%
\pgfsetroundjoin%
\pgfsetlinewidth{1.505625pt}%
\definecolor{currentstroke}{rgb}{0.185783,0.704891,0.485273}%
\pgfsetstrokecolor{currentstroke}%
\pgfsetdash{}{0pt}%
\pgfpathmoveto{\pgfqpoint{1.703995in}{0.571603in}}%
\pgfpathlineto{\pgfqpoint{1.689752in}{0.582754in}}%
\pgfpathlineto{\pgfqpoint{1.671505in}{0.597073in}}%
\pgfpathlineto{\pgfqpoint{1.659957in}{0.606240in}}%
\pgfpathlineto{\pgfqpoint{1.639471in}{0.622542in}}%
\pgfpathlineto{\pgfqpoint{1.630161in}{0.630037in}}%
\pgfpathlineto{\pgfqpoint{1.607890in}{0.648012in}}%
\pgfpathlineto{\pgfqpoint{1.600366in}{0.654155in}}%
\pgfpathlineto{\pgfqpoint{1.576757in}{0.673481in}}%
\pgfpathlineto{\pgfqpoint{1.570571in}{0.678606in}}%
\pgfpathlineto{\pgfqpoint{1.546071in}{0.698951in}}%
\pgfpathlineto{\pgfqpoint{1.540775in}{0.703400in}}%
\pgfpathlineto{\pgfqpoint{1.515826in}{0.724420in}}%
\pgfpathlineto{\pgfqpoint{1.510980in}{0.728552in}}%
\pgfpathlineto{\pgfqpoint{1.486019in}{0.749890in}}%
\pgfpathlineto{\pgfqpoint{1.481184in}{0.754072in}}%
\pgfpathlineto{\pgfqpoint{1.456645in}{0.775360in}}%
\pgfpathlineto{\pgfqpoint{1.451389in}{0.779973in}}%
\pgfpathlineto{\pgfqpoint{1.427701in}{0.800829in}}%
\pgfpathlineto{\pgfqpoint{1.421594in}{0.806269in}}%
\pgfpathlineto{\pgfqpoint{1.399181in}{0.826299in}}%
\pgfpathlineto{\pgfqpoint{1.391798in}{0.832974in}}%
\pgfpathlineto{\pgfqpoint{1.371081in}{0.851768in}}%
\pgfpathlineto{\pgfqpoint{1.362003in}{0.860102in}}%
\pgfpathlineto{\pgfqpoint{1.343398in}{0.877238in}}%
\pgfpathlineto{\pgfqpoint{1.332207in}{0.887667in}}%
\pgfpathlineto{\pgfqpoint{1.316124in}{0.902707in}}%
\pgfpathlineto{\pgfqpoint{1.302412in}{0.915684in}}%
\pgfpathlineto{\pgfqpoint{1.289257in}{0.928177in}}%
\pgfpathlineto{\pgfqpoint{1.272617in}{0.944169in}}%
\pgfpathlineto{\pgfqpoint{1.262790in}{0.953646in}}%
\pgfpathlineto{\pgfqpoint{1.242821in}{0.973138in}}%
\pgfpathlineto{\pgfqpoint{1.236719in}{0.979116in}}%
\pgfpathlineto{\pgfqpoint{1.213026in}{1.002607in}}%
\pgfpathlineto{\pgfqpoint{1.211038in}{1.004585in}}%
\pgfpathlineto{\pgfqpoint{1.185767in}{1.030055in}}%
\pgfpathlineto{\pgfqpoint{1.183230in}{1.032643in}}%
\pgfpathlineto{\pgfqpoint{1.160897in}{1.055524in}}%
\pgfpathlineto{\pgfqpoint{1.153435in}{1.063263in}}%
\pgfpathlineto{\pgfqpoint{1.136406in}{1.080994in}}%
\pgfpathlineto{\pgfqpoint{1.123640in}{1.094448in}}%
\pgfpathlineto{\pgfqpoint{1.112286in}{1.106463in}}%
\pgfpathlineto{\pgfqpoint{1.093844in}{1.126219in}}%
\pgfpathlineto{\pgfqpoint{1.088532in}{1.131933in}}%
\pgfpathlineto{\pgfqpoint{1.065150in}{1.157402in}}%
\pgfpathlineto{\pgfqpoint{1.064049in}{1.158618in}}%
\pgfpathlineto{\pgfqpoint{1.042177in}{1.182872in}}%
\pgfpathlineto{\pgfqpoint{1.034254in}{1.191767in}}%
\pgfpathlineto{\pgfqpoint{1.019556in}{1.208341in}}%
\pgfpathlineto{\pgfqpoint{1.004458in}{1.225579in}}%
\pgfpathlineto{\pgfqpoint{0.997281in}{1.233811in}}%
\pgfpathlineto{\pgfqpoint{0.975353in}{1.259281in}}%
\pgfpathlineto{\pgfqpoint{0.974663in}{1.260093in}}%
\pgfpathlineto{\pgfqpoint{0.953832in}{1.284750in}}%
\pgfpathlineto{\pgfqpoint{0.944867in}{1.295495in}}%
\pgfpathlineto{\pgfqpoint{0.932641in}{1.310220in}}%
\pgfpathlineto{\pgfqpoint{0.915072in}{1.331649in}}%
\pgfpathlineto{\pgfqpoint{0.911775in}{1.335689in}}%
\pgfpathlineto{\pgfqpoint{0.891287in}{1.361159in}}%
\pgfpathlineto{\pgfqpoint{0.885277in}{1.368732in}}%
\pgfpathlineto{\pgfqpoint{0.871145in}{1.386628in}}%
\pgfpathlineto{\pgfqpoint{0.855481in}{1.406722in}}%
\pgfpathlineto{\pgfqpoint{0.851312in}{1.412098in}}%
\pgfpathlineto{\pgfqpoint{0.831841in}{1.437567in}}%
\pgfpathlineto{\pgfqpoint{0.825686in}{1.445733in}}%
\pgfpathlineto{\pgfqpoint{0.812710in}{1.463037in}}%
\pgfpathlineto{\pgfqpoint{0.795890in}{1.485764in}}%
\pgfpathlineto{\pgfqpoint{0.793871in}{1.488506in}}%
\pgfpathlineto{\pgfqpoint{0.775410in}{1.513976in}}%
\pgfpathlineto{\pgfqpoint{0.766095in}{1.527005in}}%
\pgfusepath{stroke}%
\end{pgfscope}%
\begin{pgfscope}%
\pgfpathrectangle{\pgfqpoint{0.766095in}{0.571603in}}{\pgfqpoint{5.929283in}{5.068436in}}%
\pgfusepath{clip}%
\pgfsetbuttcap%
\pgfsetroundjoin%
\pgfsetlinewidth{1.505625pt}%
\definecolor{currentstroke}{rgb}{0.185783,0.704891,0.485273}%
\pgfsetstrokecolor{currentstroke}%
\pgfsetdash{}{0pt}%
\pgfpathmoveto{\pgfqpoint{0.766095in}{5.244799in}}%
\pgfpathlineto{\pgfqpoint{0.775444in}{5.257996in}}%
\pgfpathlineto{\pgfqpoint{0.793710in}{5.283466in}}%
\pgfpathlineto{\pgfqpoint{0.795890in}{5.286465in}}%
\pgfpathlineto{\pgfqpoint{0.812369in}{5.308935in}}%
\pgfpathlineto{\pgfqpoint{0.825686in}{5.326881in}}%
\pgfpathlineto{\pgfqpoint{0.831318in}{5.334405in}}%
\pgfpathlineto{\pgfqpoint{0.850623in}{5.359874in}}%
\pgfpathlineto{\pgfqpoint{0.855481in}{5.366204in}}%
\pgfpathlineto{\pgfqpoint{0.870299in}{5.385344in}}%
\pgfpathlineto{\pgfqpoint{0.885277in}{5.404470in}}%
\pgfpathlineto{\pgfqpoint{0.890287in}{5.410813in}}%
\pgfpathlineto{\pgfqpoint{0.910649in}{5.436283in}}%
\pgfpathlineto{\pgfqpoint{0.915072in}{5.441748in}}%
\pgfpathlineto{\pgfqpoint{0.931397in}{5.461752in}}%
\pgfpathlineto{\pgfqpoint{0.944867in}{5.478076in}}%
\pgfpathlineto{\pgfqpoint{0.952478in}{5.487222in}}%
\pgfpathlineto{\pgfqpoint{0.973913in}{5.512691in}}%
\pgfpathlineto{\pgfqpoint{0.974663in}{5.513572in}}%
\pgfpathlineto{\pgfqpoint{0.995785in}{5.538161in}}%
\pgfpathlineto{\pgfqpoint{1.004458in}{5.548147in}}%
\pgfpathlineto{\pgfqpoint{1.018015in}{5.563630in}}%
\pgfpathlineto{\pgfqpoint{1.034254in}{5.581976in}}%
\pgfpathlineto{\pgfqpoint{1.040611in}{5.589100in}}%
\pgfpathlineto{\pgfqpoint{1.063586in}{5.614570in}}%
\pgfpathlineto{\pgfqpoint{1.064049in}{5.615076in}}%
\pgfpathlineto{\pgfqpoint{1.087016in}{5.640039in}}%
\pgfusepath{stroke}%
\end{pgfscope}%
\begin{pgfscope}%
\pgfpathrectangle{\pgfqpoint{0.766095in}{0.571603in}}{\pgfqpoint{5.929283in}{5.068436in}}%
\pgfusepath{clip}%
\pgfsetbuttcap%
\pgfsetroundjoin%
\pgfsetlinewidth{1.505625pt}%
\definecolor{currentstroke}{rgb}{0.220124,0.725509,0.466226}%
\pgfsetstrokecolor{currentstroke}%
\pgfsetdash{}{0pt}%
\pgfpathmoveto{\pgfqpoint{1.636511in}{0.571603in}}%
\pgfpathlineto{\pgfqpoint{1.630161in}{0.576603in}}%
\pgfpathlineto{\pgfqpoint{1.604227in}{0.597073in}}%
\pgfpathlineto{\pgfqpoint{1.600366in}{0.600156in}}%
\pgfpathlineto{\pgfqpoint{1.572396in}{0.622542in}}%
\pgfpathlineto{\pgfqpoint{1.570571in}{0.624021in}}%
\pgfpathlineto{\pgfqpoint{1.541016in}{0.648012in}}%
\pgfpathlineto{\pgfqpoint{1.540775in}{0.648210in}}%
\pgfpathlineto{\pgfqpoint{1.510980in}{0.672747in}}%
\pgfpathlineto{\pgfqpoint{1.510091in}{0.673481in}}%
\pgfpathlineto{\pgfqpoint{1.481184in}{0.697629in}}%
\pgfpathlineto{\pgfqpoint{1.479607in}{0.698951in}}%
\pgfpathlineto{\pgfqpoint{1.451389in}{0.722864in}}%
\pgfpathlineto{\pgfqpoint{1.449558in}{0.724420in}}%
\pgfpathlineto{\pgfqpoint{1.421594in}{0.748464in}}%
\pgfpathlineto{\pgfqpoint{1.419940in}{0.749890in}}%
\pgfpathlineto{\pgfqpoint{1.391798in}{0.774441in}}%
\pgfpathlineto{\pgfqpoint{1.390749in}{0.775360in}}%
\pgfpathlineto{\pgfqpoint{1.362003in}{0.800809in}}%
\pgfpathlineto{\pgfqpoint{1.361980in}{0.800829in}}%
\pgfpathlineto{\pgfqpoint{1.333643in}{0.826299in}}%
\pgfpathlineto{\pgfqpoint{1.332207in}{0.827605in}}%
\pgfpathlineto{\pgfqpoint{1.305724in}{0.851768in}}%
\pgfpathlineto{\pgfqpoint{1.302412in}{0.854826in}}%
\pgfpathlineto{\pgfqpoint{1.278217in}{0.877238in}}%
\pgfpathlineto{\pgfqpoint{1.272617in}{0.882487in}}%
\pgfpathlineto{\pgfqpoint{1.251119in}{0.902707in}}%
\pgfpathlineto{\pgfqpoint{1.242821in}{0.910604in}}%
\pgfpathlineto{\pgfqpoint{1.224423in}{0.928177in}}%
\pgfpathlineto{\pgfqpoint{1.213026in}{0.939192in}}%
\pgfpathlineto{\pgfqpoint{1.198125in}{0.953646in}}%
\pgfpathlineto{\pgfqpoint{1.183230in}{0.968267in}}%
\pgfpathlineto{\pgfqpoint{1.172219in}{0.979116in}}%
\pgfpathlineto{\pgfqpoint{1.153435in}{0.997846in}}%
\pgfpathlineto{\pgfqpoint{1.146701in}{1.004585in}}%
\pgfpathlineto{\pgfqpoint{1.123640in}{1.027946in}}%
\pgfpathlineto{\pgfqpoint{1.121565in}{1.030055in}}%
\pgfpathlineto{\pgfqpoint{1.096834in}{1.055524in}}%
\pgfpathlineto{\pgfqpoint{1.093844in}{1.058642in}}%
\pgfpathlineto{\pgfqpoint{1.072497in}{1.080994in}}%
\pgfpathlineto{\pgfqpoint{1.064049in}{1.089948in}}%
\pgfpathlineto{\pgfqpoint{1.048530in}{1.106463in}}%
\pgfpathlineto{\pgfqpoint{1.034254in}{1.121842in}}%
\pgfpathlineto{\pgfqpoint{1.024925in}{1.131933in}}%
\pgfpathlineto{\pgfqpoint{1.004458in}{1.154345in}}%
\pgfpathlineto{\pgfqpoint{1.001678in}{1.157402in}}%
\pgfpathlineto{\pgfqpoint{0.978822in}{1.182872in}}%
\pgfpathlineto{\pgfqpoint{0.974663in}{1.187567in}}%
\pgfpathlineto{\pgfqpoint{0.956341in}{1.208341in}}%
\pgfpathlineto{\pgfqpoint{0.944867in}{1.221514in}}%
\pgfpathlineto{\pgfqpoint{0.934204in}{1.233811in}}%
\pgfpathlineto{\pgfqpoint{0.915072in}{1.256150in}}%
\pgfpathlineto{\pgfqpoint{0.912404in}{1.259281in}}%
\pgfpathlineto{\pgfqpoint{0.890989in}{1.284750in}}%
\pgfpathlineto{\pgfqpoint{0.885277in}{1.291634in}}%
\pgfpathlineto{\pgfqpoint{0.869928in}{1.310220in}}%
\pgfpathlineto{\pgfqpoint{0.855481in}{1.327935in}}%
\pgfpathlineto{\pgfqpoint{0.849188in}{1.335689in}}%
\pgfpathlineto{\pgfqpoint{0.828794in}{1.361159in}}%
\pgfpathlineto{\pgfqpoint{0.825686in}{1.365096in}}%
\pgfpathlineto{\pgfqpoint{0.808773in}{1.386628in}}%
\pgfpathlineto{\pgfqpoint{0.795890in}{1.403241in}}%
\pgfpathlineto{\pgfqpoint{0.789057in}{1.412098in}}%
\pgfpathlineto{\pgfqpoint{0.769676in}{1.437567in}}%
\pgfpathlineto{\pgfqpoint{0.766095in}{1.442342in}}%
\pgfusepath{stroke}%
\end{pgfscope}%
\begin{pgfscope}%
\pgfpathrectangle{\pgfqpoint{0.766095in}{0.571603in}}{\pgfqpoint{5.929283in}{5.068436in}}%
\pgfusepath{clip}%
\pgfsetbuttcap%
\pgfsetroundjoin%
\pgfsetlinewidth{1.505625pt}%
\definecolor{currentstroke}{rgb}{0.220124,0.725509,0.466226}%
\pgfsetstrokecolor{currentstroke}%
\pgfsetdash{}{0pt}%
\pgfpathmoveto{\pgfqpoint{0.766095in}{5.346156in}}%
\pgfpathlineto{\pgfqpoint{0.776360in}{5.359874in}}%
\pgfpathlineto{\pgfqpoint{0.795641in}{5.385344in}}%
\pgfpathlineto{\pgfqpoint{0.795890in}{5.385669in}}%
\pgfpathlineto{\pgfqpoint{0.815344in}{5.410813in}}%
\pgfpathlineto{\pgfqpoint{0.825686in}{5.424029in}}%
\pgfpathlineto{\pgfqpoint{0.835357in}{5.436283in}}%
\pgfpathlineto{\pgfqpoint{0.855481in}{5.461496in}}%
\pgfpathlineto{\pgfqpoint{0.855688in}{5.461752in}}%
\pgfpathlineto{\pgfqpoint{0.876450in}{5.487222in}}%
\pgfpathlineto{\pgfqpoint{0.885277in}{5.497928in}}%
\pgfpathlineto{\pgfqpoint{0.897547in}{5.512691in}}%
\pgfpathlineto{\pgfqpoint{0.915072in}{5.533543in}}%
\pgfpathlineto{\pgfqpoint{0.918985in}{5.538161in}}%
\pgfpathlineto{\pgfqpoint{0.940819in}{5.563630in}}%
\pgfpathlineto{\pgfqpoint{0.944867in}{5.568299in}}%
\pgfpathlineto{\pgfqpoint{0.963053in}{5.589100in}}%
\pgfpathlineto{\pgfqpoint{0.974663in}{5.602237in}}%
\pgfpathlineto{\pgfqpoint{0.985649in}{5.614570in}}%
\pgfpathlineto{\pgfqpoint{1.004458in}{5.635458in}}%
\pgfpathlineto{\pgfqpoint{1.008616in}{5.640039in}}%
\pgfusepath{stroke}%
\end{pgfscope}%
\begin{pgfscope}%
\pgfpathrectangle{\pgfqpoint{0.766095in}{0.571603in}}{\pgfqpoint{5.929283in}{5.068436in}}%
\pgfusepath{clip}%
\pgfsetbuttcap%
\pgfsetroundjoin%
\pgfsetlinewidth{1.505625pt}%
\definecolor{currentstroke}{rgb}{0.259857,0.745492,0.444467}%
\pgfsetstrokecolor{currentstroke}%
\pgfsetdash{}{0pt}%
\pgfpathmoveto{\pgfqpoint{1.570382in}{0.571603in}}%
\pgfpathlineto{\pgfqpoint{1.540775in}{0.595119in}}%
\pgfpathlineto{\pgfqpoint{1.538320in}{0.597073in}}%
\pgfpathlineto{\pgfqpoint{1.510980in}{0.619092in}}%
\pgfpathlineto{\pgfqpoint{1.506706in}{0.622542in}}%
\pgfpathlineto{\pgfqpoint{1.481184in}{0.643387in}}%
\pgfpathlineto{\pgfqpoint{1.475536in}{0.648012in}}%
\pgfpathlineto{\pgfqpoint{1.451389in}{0.668014in}}%
\pgfpathlineto{\pgfqpoint{1.444806in}{0.673481in}}%
\pgfpathlineto{\pgfqpoint{1.421594in}{0.692985in}}%
\pgfpathlineto{\pgfqpoint{1.414513in}{0.698951in}}%
\pgfpathlineto{\pgfqpoint{1.391798in}{0.718312in}}%
\pgfpathlineto{\pgfqpoint{1.384652in}{0.724420in}}%
\pgfpathlineto{\pgfqpoint{1.362003in}{0.744007in}}%
\pgfpathlineto{\pgfqpoint{1.355220in}{0.749890in}}%
\pgfpathlineto{\pgfqpoint{1.332207in}{0.770082in}}%
\pgfpathlineto{\pgfqpoint{1.326211in}{0.775360in}}%
\pgfpathlineto{\pgfqpoint{1.302412in}{0.796551in}}%
\pgfpathlineto{\pgfqpoint{1.297623in}{0.800829in}}%
\pgfpathlineto{\pgfqpoint{1.272617in}{0.823427in}}%
\pgfpathlineto{\pgfqpoint{1.269450in}{0.826299in}}%
\pgfpathlineto{\pgfqpoint{1.242821in}{0.850725in}}%
\pgfpathlineto{\pgfqpoint{1.241687in}{0.851768in}}%
\pgfpathlineto{\pgfqpoint{1.214343in}{0.877238in}}%
\pgfpathlineto{\pgfqpoint{1.213026in}{0.878479in}}%
\pgfpathlineto{\pgfqpoint{1.187414in}{0.902707in}}%
\pgfpathlineto{\pgfqpoint{1.183230in}{0.906712in}}%
\pgfpathlineto{\pgfqpoint{1.160885in}{0.928177in}}%
\pgfpathlineto{\pgfqpoint{1.153435in}{0.935419in}}%
\pgfpathlineto{\pgfqpoint{1.134751in}{0.953646in}}%
\pgfpathlineto{\pgfqpoint{1.123640in}{0.964616in}}%
\pgfpathlineto{\pgfqpoint{1.109007in}{0.979116in}}%
\pgfpathlineto{\pgfqpoint{1.093844in}{0.994321in}}%
\pgfpathlineto{\pgfqpoint{1.083647in}{1.004585in}}%
\pgfpathlineto{\pgfqpoint{1.064049in}{1.024550in}}%
\pgfpathlineto{\pgfqpoint{1.058666in}{1.030055in}}%
\pgfpathlineto{\pgfqpoint{1.034254in}{1.055322in}}%
\pgfpathlineto{\pgfqpoint{1.034058in}{1.055524in}}%
\pgfpathlineto{\pgfqpoint{1.009870in}{1.080994in}}%
\pgfpathlineto{\pgfqpoint{1.004458in}{1.086761in}}%
\pgfpathlineto{\pgfqpoint{0.986049in}{1.106463in}}%
\pgfpathlineto{\pgfqpoint{0.974663in}{1.118797in}}%
\pgfpathlineto{\pgfqpoint{0.962588in}{1.131933in}}%
\pgfpathlineto{\pgfqpoint{0.944867in}{1.151446in}}%
\pgfpathlineto{\pgfqpoint{0.939481in}{1.157402in}}%
\pgfpathlineto{\pgfqpoint{0.916739in}{1.182872in}}%
\pgfpathlineto{\pgfqpoint{0.915072in}{1.184764in}}%
\pgfpathlineto{\pgfqpoint{0.894394in}{1.208341in}}%
\pgfpathlineto{\pgfqpoint{0.885277in}{1.218866in}}%
\pgfpathlineto{\pgfqpoint{0.872388in}{1.233811in}}%
\pgfpathlineto{\pgfqpoint{0.855481in}{1.253661in}}%
\pgfpathlineto{\pgfqpoint{0.850717in}{1.259281in}}%
\pgfpathlineto{\pgfqpoint{0.829409in}{1.284750in}}%
\pgfpathlineto{\pgfqpoint{0.825686in}{1.289261in}}%
\pgfpathlineto{\pgfqpoint{0.808471in}{1.310220in}}%
\pgfpathlineto{\pgfqpoint{0.795890in}{1.325730in}}%
\pgfpathlineto{\pgfqpoint{0.787852in}{1.335689in}}%
\pgfpathlineto{\pgfqpoint{0.767559in}{1.361159in}}%
\pgfpathlineto{\pgfqpoint{0.766095in}{1.363023in}}%
\pgfusepath{stroke}%
\end{pgfscope}%
\begin{pgfscope}%
\pgfpathrectangle{\pgfqpoint{0.766095in}{0.571603in}}{\pgfqpoint{5.929283in}{5.068436in}}%
\pgfusepath{clip}%
\pgfsetbuttcap%
\pgfsetroundjoin%
\pgfsetlinewidth{1.505625pt}%
\definecolor{currentstroke}{rgb}{0.259857,0.745492,0.444467}%
\pgfsetstrokecolor{currentstroke}%
\pgfsetdash{}{0pt}%
\pgfpathmoveto{\pgfqpoint{0.766095in}{5.441269in}}%
\pgfpathlineto{\pgfqpoint{0.782251in}{5.461752in}}%
\pgfpathlineto{\pgfqpoint{0.795890in}{5.478851in}}%
\pgfpathlineto{\pgfqpoint{0.802623in}{5.487222in}}%
\pgfpathlineto{\pgfqpoint{0.823347in}{5.512691in}}%
\pgfpathlineto{\pgfqpoint{0.825686in}{5.515531in}}%
\pgfpathlineto{\pgfqpoint{0.844477in}{5.538161in}}%
\pgfpathlineto{\pgfqpoint{0.855481in}{5.551267in}}%
\pgfpathlineto{\pgfqpoint{0.865946in}{5.563630in}}%
\pgfpathlineto{\pgfqpoint{0.885277in}{5.586220in}}%
\pgfpathlineto{\pgfqpoint{0.887760in}{5.589100in}}%
\pgfpathlineto{\pgfqpoint{0.909989in}{5.614570in}}%
\pgfpathlineto{\pgfqpoint{0.915072in}{5.620328in}}%
\pgfpathlineto{\pgfqpoint{0.932606in}{5.640039in}}%
\pgfusepath{stroke}%
\end{pgfscope}%
\begin{pgfscope}%
\pgfpathrectangle{\pgfqpoint{0.766095in}{0.571603in}}{\pgfqpoint{5.929283in}{5.068436in}}%
\pgfusepath{clip}%
\pgfsetbuttcap%
\pgfsetroundjoin%
\pgfsetlinewidth{1.505625pt}%
\definecolor{currentstroke}{rgb}{0.296479,0.761561,0.424223}%
\pgfsetstrokecolor{currentstroke}%
\pgfsetdash{}{0pt}%
\pgfpathmoveto{\pgfqpoint{1.505593in}{0.571603in}}%
\pgfpathlineto{\pgfqpoint{1.481184in}{0.591102in}}%
\pgfpathlineto{\pgfqpoint{1.473729in}{0.597073in}}%
\pgfpathlineto{\pgfqpoint{1.451389in}{0.615169in}}%
\pgfpathlineto{\pgfqpoint{1.442309in}{0.622542in}}%
\pgfpathlineto{\pgfqpoint{1.421594in}{0.639559in}}%
\pgfpathlineto{\pgfqpoint{1.411331in}{0.648012in}}%
\pgfpathlineto{\pgfqpoint{1.391798in}{0.664286in}}%
\pgfpathlineto{\pgfqpoint{1.380791in}{0.673481in}}%
\pgfpathlineto{\pgfqpoint{1.362003in}{0.689359in}}%
\pgfpathlineto{\pgfqpoint{1.350685in}{0.698951in}}%
\pgfpathlineto{\pgfqpoint{1.332207in}{0.714791in}}%
\pgfpathlineto{\pgfqpoint{1.321008in}{0.724420in}}%
\pgfpathlineto{\pgfqpoint{1.302412in}{0.740595in}}%
\pgfpathlineto{\pgfqpoint{1.291757in}{0.749890in}}%
\pgfpathlineto{\pgfqpoint{1.272617in}{0.766782in}}%
\pgfpathlineto{\pgfqpoint{1.262927in}{0.775360in}}%
\pgfpathlineto{\pgfqpoint{1.242821in}{0.793366in}}%
\pgfpathlineto{\pgfqpoint{1.234515in}{0.800829in}}%
\pgfpathlineto{\pgfqpoint{1.213026in}{0.820361in}}%
\pgfpathlineto{\pgfqpoint{1.206515in}{0.826299in}}%
\pgfpathlineto{\pgfqpoint{1.183230in}{0.847780in}}%
\pgfpathlineto{\pgfqpoint{1.178922in}{0.851768in}}%
\pgfpathlineto{\pgfqpoint{1.153435in}{0.875638in}}%
\pgfpathlineto{\pgfqpoint{1.151733in}{0.877238in}}%
\pgfpathlineto{\pgfqpoint{1.124954in}{0.902707in}}%
\pgfpathlineto{\pgfqpoint{1.123640in}{0.903973in}}%
\pgfpathlineto{\pgfqpoint{1.098588in}{0.928177in}}%
\pgfpathlineto{\pgfqpoint{1.093844in}{0.932814in}}%
\pgfpathlineto{\pgfqpoint{1.072613in}{0.953646in}}%
\pgfpathlineto{\pgfqpoint{1.064049in}{0.962149in}}%
\pgfpathlineto{\pgfqpoint{1.047025in}{0.979116in}}%
\pgfpathlineto{\pgfqpoint{1.034254in}{0.991996in}}%
\pgfpathlineto{\pgfqpoint{1.021819in}{1.004585in}}%
\pgfpathlineto{\pgfqpoint{1.004458in}{1.022371in}}%
\pgfpathlineto{\pgfqpoint{0.996988in}{1.030055in}}%
\pgfpathlineto{\pgfqpoint{0.974663in}{1.053293in}}%
\pgfpathlineto{\pgfqpoint{0.972527in}{1.055524in}}%
\pgfpathlineto{\pgfqpoint{0.948465in}{1.080994in}}%
\pgfpathlineto{\pgfqpoint{0.944867in}{1.084850in}}%
\pgfpathlineto{\pgfqpoint{0.924786in}{1.106463in}}%
\pgfpathlineto{\pgfqpoint{0.915072in}{1.117045in}}%
\pgfpathlineto{\pgfqpoint{0.901464in}{1.131933in}}%
\pgfpathlineto{\pgfqpoint{0.885277in}{1.149858in}}%
\pgfpathlineto{\pgfqpoint{0.878493in}{1.157402in}}%
\pgfpathlineto{\pgfqpoint{0.855871in}{1.182872in}}%
\pgfpathlineto{\pgfqpoint{0.855481in}{1.183316in}}%
\pgfpathlineto{\pgfqpoint{0.833656in}{1.208341in}}%
\pgfpathlineto{\pgfqpoint{0.825686in}{1.217593in}}%
\pgfpathlineto{\pgfqpoint{0.811777in}{1.233811in}}%
\pgfpathlineto{\pgfqpoint{0.795890in}{1.252566in}}%
\pgfpathlineto{\pgfqpoint{0.790230in}{1.259281in}}%
\pgfpathlineto{\pgfqpoint{0.769034in}{1.284750in}}%
\pgfpathlineto{\pgfqpoint{0.766095in}{1.288331in}}%
\pgfusepath{stroke}%
\end{pgfscope}%
\begin{pgfscope}%
\pgfpathrectangle{\pgfqpoint{0.766095in}{0.571603in}}{\pgfqpoint{5.929283in}{5.068436in}}%
\pgfusepath{clip}%
\pgfsetbuttcap%
\pgfsetroundjoin%
\pgfsetlinewidth{1.505625pt}%
\definecolor{currentstroke}{rgb}{0.296479,0.761561,0.424223}%
\pgfsetstrokecolor{currentstroke}%
\pgfsetdash{}{0pt}%
\pgfpathmoveto{\pgfqpoint{0.766095in}{5.530914in}}%
\pgfpathlineto{\pgfqpoint{0.772034in}{5.538161in}}%
\pgfpathlineto{\pgfqpoint{0.793148in}{5.563630in}}%
\pgfpathlineto{\pgfqpoint{0.795890in}{5.566899in}}%
\pgfpathlineto{\pgfqpoint{0.814664in}{5.589100in}}%
\pgfpathlineto{\pgfqpoint{0.825686in}{5.601991in}}%
\pgfpathlineto{\pgfqpoint{0.836524in}{5.614570in}}%
\pgfpathlineto{\pgfqpoint{0.855481in}{5.636331in}}%
\pgfpathlineto{\pgfqpoint{0.858736in}{5.640039in}}%
\pgfusepath{stroke}%
\end{pgfscope}%
\begin{pgfscope}%
\pgfpathrectangle{\pgfqpoint{0.766095in}{0.571603in}}{\pgfqpoint{5.929283in}{5.068436in}}%
\pgfusepath{clip}%
\pgfsetbuttcap%
\pgfsetroundjoin%
\pgfsetlinewidth{1.505625pt}%
\definecolor{currentstroke}{rgb}{0.344074,0.780029,0.397381}%
\pgfsetstrokecolor{currentstroke}%
\pgfsetdash{}{0pt}%
\pgfpathmoveto{\pgfqpoint{1.442033in}{0.571603in}}%
\pgfpathlineto{\pgfqpoint{1.421594in}{0.588026in}}%
\pgfpathlineto{\pgfqpoint{1.410362in}{0.597073in}}%
\pgfpathlineto{\pgfqpoint{1.391798in}{0.612197in}}%
\pgfpathlineto{\pgfqpoint{1.379134in}{0.622542in}}%
\pgfpathlineto{\pgfqpoint{1.362003in}{0.636697in}}%
\pgfpathlineto{\pgfqpoint{1.348345in}{0.648012in}}%
\pgfpathlineto{\pgfqpoint{1.332207in}{0.661535in}}%
\pgfpathlineto{\pgfqpoint{1.317991in}{0.673481in}}%
\pgfpathlineto{\pgfqpoint{1.302412in}{0.686723in}}%
\pgfpathlineto{\pgfqpoint{1.288067in}{0.698951in}}%
\pgfpathlineto{\pgfqpoint{1.272617in}{0.712274in}}%
\pgfpathlineto{\pgfqpoint{1.258571in}{0.724420in}}%
\pgfpathlineto{\pgfqpoint{1.242821in}{0.738199in}}%
\pgfpathlineto{\pgfqpoint{1.229498in}{0.749890in}}%
\pgfpathlineto{\pgfqpoint{1.213026in}{0.764512in}}%
\pgfpathlineto{\pgfqpoint{1.200843in}{0.775360in}}%
\pgfpathlineto{\pgfqpoint{1.183230in}{0.791225in}}%
\pgfpathlineto{\pgfqpoint{1.172603in}{0.800829in}}%
\pgfpathlineto{\pgfqpoint{1.153435in}{0.818352in}}%
\pgfpathlineto{\pgfqpoint{1.144771in}{0.826299in}}%
\pgfpathlineto{\pgfqpoint{1.123640in}{0.845907in}}%
\pgfpathlineto{\pgfqpoint{1.117345in}{0.851768in}}%
\pgfpathlineto{\pgfqpoint{1.093844in}{0.873905in}}%
\pgfpathlineto{\pgfqpoint{1.090319in}{0.877238in}}%
\pgfpathlineto{\pgfqpoint{1.064049in}{0.902360in}}%
\pgfpathlineto{\pgfqpoint{1.063688in}{0.902707in}}%
\pgfpathlineto{\pgfqpoint{1.037476in}{0.928177in}}%
\pgfpathlineto{\pgfqpoint{1.034254in}{0.931345in}}%
\pgfpathlineto{\pgfqpoint{1.011657in}{0.953646in}}%
\pgfpathlineto{\pgfqpoint{1.004458in}{0.960834in}}%
\pgfpathlineto{\pgfqpoint{0.986220in}{0.979116in}}%
\pgfpathlineto{\pgfqpoint{0.974663in}{0.990839in}}%
\pgfpathlineto{\pgfqpoint{0.961162in}{1.004585in}}%
\pgfpathlineto{\pgfqpoint{0.944867in}{1.021375in}}%
\pgfpathlineto{\pgfqpoint{0.936477in}{1.030055in}}%
\pgfpathlineto{\pgfqpoint{0.915072in}{1.052463in}}%
\pgfpathlineto{\pgfqpoint{0.912159in}{1.055524in}}%
\pgfpathlineto{\pgfqpoint{0.888230in}{1.080994in}}%
\pgfpathlineto{\pgfqpoint{0.885277in}{1.084178in}}%
\pgfpathlineto{\pgfqpoint{0.864688in}{1.106463in}}%
\pgfpathlineto{\pgfqpoint{0.855481in}{1.116550in}}%
\pgfpathlineto{\pgfqpoint{0.841500in}{1.131933in}}%
\pgfpathlineto{\pgfqpoint{0.825686in}{1.149543in}}%
\pgfpathlineto{\pgfqpoint{0.818659in}{1.157402in}}%
\pgfpathlineto{\pgfqpoint{0.796163in}{1.182872in}}%
\pgfpathlineto{\pgfqpoint{0.795890in}{1.183185in}}%
\pgfpathlineto{\pgfqpoint{0.774073in}{1.208341in}}%
\pgfpathlineto{\pgfqpoint{0.766095in}{1.217654in}}%
\pgfusepath{stroke}%
\end{pgfscope}%
\begin{pgfscope}%
\pgfpathrectangle{\pgfqpoint{0.766095in}{0.571603in}}{\pgfqpoint{5.929283in}{5.068436in}}%
\pgfusepath{clip}%
\pgfsetbuttcap%
\pgfsetroundjoin%
\pgfsetlinewidth{1.505625pt}%
\definecolor{currentstroke}{rgb}{0.344074,0.780029,0.397381}%
\pgfsetstrokecolor{currentstroke}%
\pgfsetdash{}{0pt}%
\pgfpathmoveto{\pgfqpoint{0.766095in}{5.615820in}}%
\pgfpathlineto{\pgfqpoint{0.786947in}{5.640039in}}%
\pgfusepath{stroke}%
\end{pgfscope}%
\begin{pgfscope}%
\pgfpathrectangle{\pgfqpoint{0.766095in}{0.571603in}}{\pgfqpoint{5.929283in}{5.068436in}}%
\pgfusepath{clip}%
\pgfsetbuttcap%
\pgfsetroundjoin%
\pgfsetlinewidth{1.505625pt}%
\definecolor{currentstroke}{rgb}{0.386433,0.794644,0.372886}%
\pgfsetstrokecolor{currentstroke}%
\pgfsetdash{}{0pt}%
\pgfpathmoveto{\pgfqpoint{1.379650in}{0.571603in}}%
\pgfpathlineto{\pgfqpoint{1.362003in}{0.585865in}}%
\pgfpathlineto{\pgfqpoint{1.348170in}{0.597073in}}%
\pgfpathlineto{\pgfqpoint{1.332207in}{0.610154in}}%
\pgfpathlineto{\pgfqpoint{1.317130in}{0.622542in}}%
\pgfpathlineto{\pgfqpoint{1.302412in}{0.634773in}}%
\pgfpathlineto{\pgfqpoint{1.286526in}{0.648012in}}%
\pgfpathlineto{\pgfqpoint{1.272617in}{0.659735in}}%
\pgfpathlineto{\pgfqpoint{1.256354in}{0.673481in}}%
\pgfpathlineto{\pgfqpoint{1.242821in}{0.685051in}}%
\pgfpathlineto{\pgfqpoint{1.226610in}{0.698951in}}%
\pgfpathlineto{\pgfqpoint{1.213026in}{0.710733in}}%
\pgfpathlineto{\pgfqpoint{1.197291in}{0.724420in}}%
\pgfpathlineto{\pgfqpoint{1.183230in}{0.736793in}}%
\pgfpathlineto{\pgfqpoint{1.168392in}{0.749890in}}%
\pgfpathlineto{\pgfqpoint{1.153435in}{0.763244in}}%
\pgfpathlineto{\pgfqpoint{1.139908in}{0.775360in}}%
\pgfpathlineto{\pgfqpoint{1.123640in}{0.790099in}}%
\pgfpathlineto{\pgfqpoint{1.111835in}{0.800829in}}%
\pgfpathlineto{\pgfqpoint{1.093844in}{0.817372in}}%
\pgfpathlineto{\pgfqpoint{1.084169in}{0.826299in}}%
\pgfpathlineto{\pgfqpoint{1.064049in}{0.845077in}}%
\pgfpathlineto{\pgfqpoint{1.056905in}{0.851768in}}%
\pgfpathlineto{\pgfqpoint{1.034254in}{0.873229in}}%
\pgfpathlineto{\pgfqpoint{1.030037in}{0.877238in}}%
\pgfpathlineto{\pgfqpoint{1.004458in}{0.901842in}}%
\pgfpathlineto{\pgfqpoint{1.003562in}{0.902707in}}%
\pgfpathlineto{\pgfqpoint{0.977499in}{0.928177in}}%
\pgfpathlineto{\pgfqpoint{0.974663in}{0.930982in}}%
\pgfpathlineto{\pgfqpoint{0.951830in}{0.953646in}}%
\pgfpathlineto{\pgfqpoint{0.944867in}{0.960639in}}%
\pgfpathlineto{\pgfqpoint{0.926542in}{0.979116in}}%
\pgfpathlineto{\pgfqpoint{0.915072in}{0.990817in}}%
\pgfpathlineto{\pgfqpoint{0.901628in}{1.004585in}}%
\pgfpathlineto{\pgfqpoint{0.885277in}{1.021530in}}%
\pgfpathlineto{\pgfqpoint{0.877084in}{1.030055in}}%
\pgfpathlineto{\pgfqpoint{0.855481in}{1.052799in}}%
\pgfpathlineto{\pgfqpoint{0.852904in}{1.055524in}}%
\pgfpathlineto{\pgfqpoint{0.829114in}{1.080994in}}%
\pgfpathlineto{\pgfqpoint{0.825686in}{1.084710in}}%
\pgfpathlineto{\pgfqpoint{0.805704in}{1.106463in}}%
\pgfpathlineto{\pgfqpoint{0.795890in}{1.117276in}}%
\pgfpathlineto{\pgfqpoint{0.782644in}{1.131933in}}%
\pgfpathlineto{\pgfqpoint{0.766095in}{1.150467in}}%
\pgfusepath{stroke}%
\end{pgfscope}%
\begin{pgfscope}%
\pgfpathrectangle{\pgfqpoint{0.766095in}{0.571603in}}{\pgfqpoint{5.929283in}{5.068436in}}%
\pgfusepath{clip}%
\pgfsetbuttcap%
\pgfsetroundjoin%
\pgfsetlinewidth{1.505625pt}%
\definecolor{currentstroke}{rgb}{0.440137,0.811138,0.340967}%
\pgfsetstrokecolor{currentstroke}%
\pgfsetdash{}{0pt}%
\pgfpathmoveto{\pgfqpoint{1.318396in}{0.571603in}}%
\pgfpathlineto{\pgfqpoint{1.302412in}{0.584596in}}%
\pgfpathlineto{\pgfqpoint{1.287104in}{0.597073in}}%
\pgfpathlineto{\pgfqpoint{1.272617in}{0.609014in}}%
\pgfpathlineto{\pgfqpoint{1.256248in}{0.622542in}}%
\pgfpathlineto{\pgfqpoint{1.242821in}{0.633766in}}%
\pgfpathlineto{\pgfqpoint{1.225826in}{0.648012in}}%
\pgfpathlineto{\pgfqpoint{1.213026in}{0.658863in}}%
\pgfpathlineto{\pgfqpoint{1.195833in}{0.673481in}}%
\pgfpathlineto{\pgfqpoint{1.183230in}{0.684319in}}%
\pgfpathlineto{\pgfqpoint{1.166265in}{0.698951in}}%
\pgfpathlineto{\pgfqpoint{1.153435in}{0.710144in}}%
\pgfpathlineto{\pgfqpoint{1.137119in}{0.724420in}}%
\pgfpathlineto{\pgfqpoint{1.123640in}{0.736351in}}%
\pgfpathlineto{\pgfqpoint{1.108390in}{0.749890in}}%
\pgfpathlineto{\pgfqpoint{1.093844in}{0.762953in}}%
\pgfpathlineto{\pgfqpoint{1.080073in}{0.775360in}}%
\pgfpathlineto{\pgfqpoint{1.064049in}{0.789963in}}%
\pgfpathlineto{\pgfqpoint{1.052165in}{0.800829in}}%
\pgfpathlineto{\pgfqpoint{1.034254in}{0.817395in}}%
\pgfpathlineto{\pgfqpoint{1.024659in}{0.826299in}}%
\pgfpathlineto{\pgfqpoint{1.004458in}{0.845263in}}%
\pgfpathlineto{\pgfqpoint{0.997553in}{0.851768in}}%
\pgfpathlineto{\pgfqpoint{0.974663in}{0.873582in}}%
\pgfpathlineto{\pgfqpoint{0.970840in}{0.877238in}}%
\pgfpathlineto{\pgfqpoint{0.944867in}{0.902366in}}%
\pgfpathlineto{\pgfqpoint{0.944516in}{0.902707in}}%
\pgfpathlineto{\pgfqpoint{0.918608in}{0.928177in}}%
\pgfpathlineto{\pgfqpoint{0.915072in}{0.931694in}}%
\pgfpathlineto{\pgfqpoint{0.893086in}{0.953646in}}%
\pgfpathlineto{\pgfqpoint{0.885277in}{0.961535in}}%
\pgfpathlineto{\pgfqpoint{0.867940in}{0.979116in}}%
\pgfpathlineto{\pgfqpoint{0.855481in}{0.991900in}}%
\pgfpathlineto{\pgfqpoint{0.843167in}{1.004585in}}%
\pgfpathlineto{\pgfqpoint{0.825686in}{1.022806in}}%
\pgfpathlineto{\pgfqpoint{0.818759in}{1.030055in}}%
\pgfpathlineto{\pgfqpoint{0.795890in}{1.054272in}}%
\pgfpathlineto{\pgfqpoint{0.794712in}{1.055524in}}%
\pgfpathlineto{\pgfqpoint{0.771066in}{1.080994in}}%
\pgfpathlineto{\pgfqpoint{0.766095in}{1.086414in}}%
\pgfusepath{stroke}%
\end{pgfscope}%
\begin{pgfscope}%
\pgfpathrectangle{\pgfqpoint{0.766095in}{0.571603in}}{\pgfqpoint{5.929283in}{5.068436in}}%
\pgfusepath{clip}%
\pgfsetbuttcap%
\pgfsetroundjoin%
\pgfsetlinewidth{1.505625pt}%
\definecolor{currentstroke}{rgb}{0.487026,0.823929,0.312321}%
\pgfsetstrokecolor{currentstroke}%
\pgfsetdash{}{0pt}%
\pgfpathmoveto{\pgfqpoint{1.258226in}{0.571603in}}%
\pgfpathlineto{\pgfqpoint{1.242821in}{0.584198in}}%
\pgfpathlineto{\pgfqpoint{1.227117in}{0.597073in}}%
\pgfpathlineto{\pgfqpoint{1.213026in}{0.608755in}}%
\pgfpathlineto{\pgfqpoint{1.196442in}{0.622542in}}%
\pgfpathlineto{\pgfqpoint{1.183230in}{0.633651in}}%
\pgfpathlineto{\pgfqpoint{1.166198in}{0.648012in}}%
\pgfpathlineto{\pgfqpoint{1.153435in}{0.658896in}}%
\pgfpathlineto{\pgfqpoint{1.136381in}{0.673481in}}%
\pgfpathlineto{\pgfqpoint{1.123640in}{0.684502in}}%
\pgfpathlineto{\pgfqpoint{1.106986in}{0.698951in}}%
\pgfpathlineto{\pgfqpoint{1.093844in}{0.710482in}}%
\pgfpathlineto{\pgfqpoint{1.078009in}{0.724420in}}%
\pgfpathlineto{\pgfqpoint{1.064049in}{0.736848in}}%
\pgfpathlineto{\pgfqpoint{1.049446in}{0.749890in}}%
\pgfpathlineto{\pgfqpoint{1.034254in}{0.763614in}}%
\pgfpathlineto{\pgfqpoint{1.021293in}{0.775360in}}%
\pgfpathlineto{\pgfqpoint{1.004458in}{0.790791in}}%
\pgfpathlineto{\pgfqpoint{0.993544in}{0.800829in}}%
\pgfpathlineto{\pgfqpoint{0.974663in}{0.818395in}}%
\pgfpathlineto{\pgfqpoint{0.966196in}{0.826299in}}%
\pgfpathlineto{\pgfqpoint{0.944867in}{0.846439in}}%
\pgfpathlineto{\pgfqpoint{0.939244in}{0.851768in}}%
\pgfpathlineto{\pgfqpoint{0.915072in}{0.874938in}}%
\pgfpathlineto{\pgfqpoint{0.912682in}{0.877238in}}%
\pgfpathlineto{\pgfqpoint{0.886517in}{0.902707in}}%
\pgfpathlineto{\pgfqpoint{0.885277in}{0.903929in}}%
\pgfpathlineto{\pgfqpoint{0.860758in}{0.928177in}}%
\pgfpathlineto{\pgfqpoint{0.855481in}{0.933456in}}%
\pgfpathlineto{\pgfqpoint{0.835377in}{0.953646in}}%
\pgfpathlineto{\pgfqpoint{0.825686in}{0.963494in}}%
\pgfpathlineto{\pgfqpoint{0.810371in}{0.979116in}}%
\pgfpathlineto{\pgfqpoint{0.795890in}{0.994060in}}%
\pgfpathlineto{\pgfqpoint{0.785733in}{1.004585in}}%
\pgfpathlineto{\pgfqpoint{0.766095in}{1.025173in}}%
\pgfusepath{stroke}%
\end{pgfscope}%
\begin{pgfscope}%
\pgfpathrectangle{\pgfqpoint{0.766095in}{0.571603in}}{\pgfqpoint{5.929283in}{5.068436in}}%
\pgfusepath{clip}%
\pgfsetbuttcap%
\pgfsetroundjoin%
\pgfsetlinewidth{1.505625pt}%
\definecolor{currentstroke}{rgb}{0.545524,0.838039,0.275626}%
\pgfsetstrokecolor{currentstroke}%
\pgfsetdash{}{0pt}%
\pgfpathmoveto{\pgfqpoint{1.199093in}{0.571603in}}%
\pgfpathlineto{\pgfqpoint{1.183230in}{0.584649in}}%
\pgfpathlineto{\pgfqpoint{1.168165in}{0.597073in}}%
\pgfpathlineto{\pgfqpoint{1.153435in}{0.609357in}}%
\pgfpathlineto{\pgfqpoint{1.137669in}{0.622542in}}%
\pgfpathlineto{\pgfqpoint{1.123640in}{0.634407in}}%
\pgfpathlineto{\pgfqpoint{1.107600in}{0.648012in}}%
\pgfpathlineto{\pgfqpoint{1.093844in}{0.659811in}}%
\pgfpathlineto{\pgfqpoint{1.077954in}{0.673481in}}%
\pgfpathlineto{\pgfqpoint{1.064049in}{0.685580in}}%
\pgfpathlineto{\pgfqpoint{1.048728in}{0.698951in}}%
\pgfpathlineto{\pgfqpoint{1.034254in}{0.711727in}}%
\pgfpathlineto{\pgfqpoint{1.019917in}{0.724420in}}%
\pgfpathlineto{\pgfqpoint{1.004458in}{0.738264in}}%
\pgfpathlineto{\pgfqpoint{0.991517in}{0.749890in}}%
\pgfpathlineto{\pgfqpoint{0.974663in}{0.765204in}}%
\pgfpathlineto{\pgfqpoint{0.963523in}{0.775360in}}%
\pgfpathlineto{\pgfqpoint{0.944867in}{0.792561in}}%
\pgfpathlineto{\pgfqpoint{0.935931in}{0.800829in}}%
\pgfpathlineto{\pgfqpoint{0.915072in}{0.820349in}}%
\pgfpathlineto{\pgfqpoint{0.908736in}{0.826299in}}%
\pgfpathlineto{\pgfqpoint{0.885277in}{0.848582in}}%
\pgfpathlineto{\pgfqpoint{0.881934in}{0.851768in}}%
\pgfpathlineto{\pgfqpoint{0.855519in}{0.877238in}}%
\pgfpathlineto{\pgfqpoint{0.855481in}{0.877275in}}%
\pgfpathlineto{\pgfqpoint{0.829520in}{0.902707in}}%
\pgfpathlineto{\pgfqpoint{0.825686in}{0.906507in}}%
\pgfpathlineto{\pgfqpoint{0.803903in}{0.928177in}}%
\pgfpathlineto{\pgfqpoint{0.795890in}{0.936240in}}%
\pgfpathlineto{\pgfqpoint{0.778660in}{0.953646in}}%
\pgfpathlineto{\pgfqpoint{0.766095in}{0.966489in}}%
\pgfusepath{stroke}%
\end{pgfscope}%
\begin{pgfscope}%
\pgfpathrectangle{\pgfqpoint{0.766095in}{0.571603in}}{\pgfqpoint{5.929283in}{5.068436in}}%
\pgfusepath{clip}%
\pgfsetbuttcap%
\pgfsetroundjoin%
\pgfsetlinewidth{1.505625pt}%
\definecolor{currentstroke}{rgb}{0.606045,0.850733,0.236712}%
\pgfsetstrokecolor{currentstroke}%
\pgfsetdash{}{0pt}%
\pgfpathmoveto{\pgfqpoint{1.140958in}{0.571603in}}%
\pgfpathlineto{\pgfqpoint{1.123640in}{0.585930in}}%
\pgfpathlineto{\pgfqpoint{1.110207in}{0.597073in}}%
\pgfpathlineto{\pgfqpoint{1.093844in}{0.610799in}}%
\pgfpathlineto{\pgfqpoint{1.079885in}{0.622542in}}%
\pgfpathlineto{\pgfqpoint{1.064049in}{0.636015in}}%
\pgfpathlineto{\pgfqpoint{1.049988in}{0.648012in}}%
\pgfpathlineto{\pgfqpoint{1.034254in}{0.661588in}}%
\pgfpathlineto{\pgfqpoint{1.020510in}{0.673481in}}%
\pgfpathlineto{\pgfqpoint{1.004458in}{0.687530in}}%
\pgfpathlineto{\pgfqpoint{0.991450in}{0.698951in}}%
\pgfpathlineto{\pgfqpoint{0.974663in}{0.713855in}}%
\pgfpathlineto{\pgfqpoint{0.962801in}{0.724420in}}%
\pgfpathlineto{\pgfqpoint{0.944867in}{0.740575in}}%
\pgfpathlineto{\pgfqpoint{0.934560in}{0.749890in}}%
\pgfpathlineto{\pgfqpoint{0.915072in}{0.767702in}}%
\pgfpathlineto{\pgfqpoint{0.906722in}{0.775360in}}%
\pgfpathlineto{\pgfqpoint{0.885277in}{0.795251in}}%
\pgfpathlineto{\pgfqpoint{0.879283in}{0.800829in}}%
\pgfpathlineto{\pgfqpoint{0.855481in}{0.823235in}}%
\pgfpathlineto{\pgfqpoint{0.852238in}{0.826299in}}%
\pgfpathlineto{\pgfqpoint{0.825686in}{0.851668in}}%
\pgfpathlineto{\pgfqpoint{0.825582in}{0.851768in}}%
\pgfpathlineto{\pgfqpoint{0.799340in}{0.877238in}}%
\pgfpathlineto{\pgfqpoint{0.795890in}{0.880624in}}%
\pgfpathlineto{\pgfqpoint{0.773482in}{0.902707in}}%
\pgfpathlineto{\pgfqpoint{0.766095in}{0.910071in}}%
\pgfusepath{stroke}%
\end{pgfscope}%
\begin{pgfscope}%
\pgfpathrectangle{\pgfqpoint{0.766095in}{0.571603in}}{\pgfqpoint{5.929283in}{5.068436in}}%
\pgfusepath{clip}%
\pgfsetbuttcap%
\pgfsetroundjoin%
\pgfsetlinewidth{1.505625pt}%
\definecolor{currentstroke}{rgb}{0.657642,0.860219,0.203082}%
\pgfsetstrokecolor{currentstroke}%
\pgfsetdash{}{0pt}%
\pgfpathmoveto{\pgfqpoint{1.083779in}{0.571603in}}%
\pgfpathlineto{\pgfqpoint{1.064049in}{0.588022in}}%
\pgfpathlineto{\pgfqpoint{1.053203in}{0.597073in}}%
\pgfpathlineto{\pgfqpoint{1.034254in}{0.613063in}}%
\pgfpathlineto{\pgfqpoint{1.023052in}{0.622542in}}%
\pgfpathlineto{\pgfqpoint{1.004458in}{0.638454in}}%
\pgfpathlineto{\pgfqpoint{0.993322in}{0.648012in}}%
\pgfpathlineto{\pgfqpoint{0.974663in}{0.664207in}}%
\pgfpathlineto{\pgfqpoint{0.964010in}{0.673481in}}%
\pgfpathlineto{\pgfqpoint{0.944867in}{0.690334in}}%
\pgfpathlineto{\pgfqpoint{0.935111in}{0.698951in}}%
\pgfpathlineto{\pgfqpoint{0.915072in}{0.716848in}}%
\pgfpathlineto{\pgfqpoint{0.906621in}{0.724420in}}%
\pgfpathlineto{\pgfqpoint{0.885277in}{0.743762in}}%
\pgfpathlineto{\pgfqpoint{0.878536in}{0.749890in}}%
\pgfpathlineto{\pgfqpoint{0.855481in}{0.771087in}}%
\pgfpathlineto{\pgfqpoint{0.850850in}{0.775360in}}%
\pgfpathlineto{\pgfqpoint{0.825686in}{0.798839in}}%
\pgfpathlineto{\pgfqpoint{0.823560in}{0.800829in}}%
\pgfpathlineto{\pgfqpoint{0.796667in}{0.826299in}}%
\pgfpathlineto{\pgfqpoint{0.795890in}{0.827043in}}%
\pgfpathlineto{\pgfqpoint{0.770183in}{0.851768in}}%
\pgfpathlineto{\pgfqpoint{0.766095in}{0.855744in}}%
\pgfusepath{stroke}%
\end{pgfscope}%
\begin{pgfscope}%
\pgfpathrectangle{\pgfqpoint{0.766095in}{0.571603in}}{\pgfqpoint{5.929283in}{5.068436in}}%
\pgfusepath{clip}%
\pgfsetbuttcap%
\pgfsetroundjoin%
\pgfsetlinewidth{1.505625pt}%
\definecolor{currentstroke}{rgb}{0.720391,0.870350,0.162603}%
\pgfsetstrokecolor{currentstroke}%
\pgfsetdash{}{0pt}%
\pgfpathmoveto{\pgfqpoint{1.027519in}{0.571603in}}%
\pgfpathlineto{\pgfqpoint{1.004458in}{0.590907in}}%
\pgfpathlineto{\pgfqpoint{0.997113in}{0.597073in}}%
\pgfpathlineto{\pgfqpoint{0.974663in}{0.616129in}}%
\pgfpathlineto{\pgfqpoint{0.967130in}{0.622542in}}%
\pgfpathlineto{\pgfqpoint{0.944867in}{0.641707in}}%
\pgfpathlineto{\pgfqpoint{0.937565in}{0.648012in}}%
\pgfpathlineto{\pgfqpoint{0.915072in}{0.667651in}}%
\pgfpathlineto{\pgfqpoint{0.908414in}{0.673481in}}%
\pgfpathlineto{\pgfqpoint{0.885277in}{0.693973in}}%
\pgfpathlineto{\pgfqpoint{0.879674in}{0.698951in}}%
\pgfpathlineto{\pgfqpoint{0.855481in}{0.720687in}}%
\pgfpathlineto{\pgfqpoint{0.851339in}{0.724420in}}%
\pgfpathlineto{\pgfqpoint{0.825686in}{0.747805in}}%
\pgfpathlineto{\pgfqpoint{0.823406in}{0.749890in}}%
\pgfpathlineto{\pgfqpoint{0.795890in}{0.775340in}}%
\pgfpathlineto{\pgfqpoint{0.795869in}{0.775360in}}%
\pgfpathlineto{\pgfqpoint{0.768747in}{0.800829in}}%
\pgfpathlineto{\pgfqpoint{0.766095in}{0.803348in}}%
\pgfusepath{stroke}%
\end{pgfscope}%
\begin{pgfscope}%
\pgfpathrectangle{\pgfqpoint{0.766095in}{0.571603in}}{\pgfqpoint{5.929283in}{5.068436in}}%
\pgfusepath{clip}%
\pgfsetbuttcap%
\pgfsetroundjoin%
\pgfsetlinewidth{1.505625pt}%
\definecolor{currentstroke}{rgb}{0.772852,0.877868,0.131109}%
\pgfsetstrokecolor{currentstroke}%
\pgfsetdash{}{0pt}%
\pgfpathmoveto{\pgfqpoint{0.972140in}{0.571603in}}%
\pgfpathlineto{\pgfqpoint{0.944867in}{0.594568in}}%
\pgfpathlineto{\pgfqpoint{0.941902in}{0.597073in}}%
\pgfpathlineto{\pgfqpoint{0.915072in}{0.619983in}}%
\pgfpathlineto{\pgfqpoint{0.912083in}{0.622542in}}%
\pgfpathlineto{\pgfqpoint{0.885277in}{0.645756in}}%
\pgfpathlineto{\pgfqpoint{0.882680in}{0.648012in}}%
\pgfpathlineto{\pgfqpoint{0.855481in}{0.671901in}}%
\pgfpathlineto{\pgfqpoint{0.853687in}{0.673481in}}%
\pgfpathlineto{\pgfqpoint{0.825686in}{0.698429in}}%
\pgfpathlineto{\pgfqpoint{0.825102in}{0.698951in}}%
\pgfpathlineto{\pgfqpoint{0.796928in}{0.724420in}}%
\pgfpathlineto{\pgfqpoint{0.795890in}{0.725369in}}%
\pgfpathlineto{\pgfqpoint{0.769160in}{0.749890in}}%
\pgfpathlineto{\pgfqpoint{0.766095in}{0.752733in}}%
\pgfusepath{stroke}%
\end{pgfscope}%
\begin{pgfscope}%
\pgfpathrectangle{\pgfqpoint{0.766095in}{0.571603in}}{\pgfqpoint{5.929283in}{5.068436in}}%
\pgfusepath{clip}%
\pgfsetbuttcap%
\pgfsetroundjoin%
\pgfsetlinewidth{1.505625pt}%
\definecolor{currentstroke}{rgb}{0.835270,0.886029,0.102646}%
\pgfsetstrokecolor{currentstroke}%
\pgfsetdash{}{0pt}%
\pgfpathmoveto{\pgfqpoint{0.917629in}{0.571603in}}%
\pgfpathlineto{\pgfqpoint{0.915072in}{0.573763in}}%
\pgfpathlineto{\pgfqpoint{0.887553in}{0.597073in}}%
\pgfpathlineto{\pgfqpoint{0.885277in}{0.599023in}}%
\pgfpathlineto{\pgfqpoint{0.857897in}{0.622542in}}%
\pgfpathlineto{\pgfqpoint{0.855481in}{0.624641in}}%
\pgfpathlineto{\pgfqpoint{0.828656in}{0.648012in}}%
\pgfpathlineto{\pgfqpoint{0.825686in}{0.650629in}}%
\pgfpathlineto{\pgfqpoint{0.799827in}{0.673481in}}%
\pgfpathlineto{\pgfqpoint{0.795890in}{0.676999in}}%
\pgfpathlineto{\pgfqpoint{0.771405in}{0.698951in}}%
\pgfpathlineto{\pgfqpoint{0.766095in}{0.703764in}}%
\pgfusepath{stroke}%
\end{pgfscope}%
\begin{pgfscope}%
\pgfpathrectangle{\pgfqpoint{0.766095in}{0.571603in}}{\pgfqpoint{5.929283in}{5.068436in}}%
\pgfusepath{clip}%
\pgfsetbuttcap%
\pgfsetroundjoin%
\pgfsetlinewidth{1.505625pt}%
\definecolor{currentstroke}{rgb}{0.886271,0.892374,0.095374}%
\pgfsetstrokecolor{currentstroke}%
\pgfsetdash{}{0pt}%
\pgfpathmoveto{\pgfqpoint{0.863959in}{0.571603in}}%
\pgfpathlineto{\pgfqpoint{0.855481in}{0.578807in}}%
\pgfpathlineto{\pgfqpoint{0.834046in}{0.597073in}}%
\pgfpathlineto{\pgfqpoint{0.825686in}{0.604276in}}%
\pgfpathlineto{\pgfqpoint{0.804549in}{0.622542in}}%
\pgfpathlineto{\pgfqpoint{0.795890in}{0.630108in}}%
\pgfpathlineto{\pgfqpoint{0.775465in}{0.648012in}}%
\pgfpathlineto{\pgfqpoint{0.766095in}{0.656316in}}%
\pgfusepath{stroke}%
\end{pgfscope}%
\begin{pgfscope}%
\pgfpathrectangle{\pgfqpoint{0.766095in}{0.571603in}}{\pgfqpoint{5.929283in}{5.068436in}}%
\pgfusepath{clip}%
\pgfsetbuttcap%
\pgfsetroundjoin%
\pgfsetlinewidth{1.505625pt}%
\definecolor{currentstroke}{rgb}{0.945636,0.899815,0.112838}%
\pgfsetstrokecolor{currentstroke}%
\pgfsetdash{}{0pt}%
\pgfpathmoveto{\pgfqpoint{0.811077in}{0.571603in}}%
\pgfpathlineto{\pgfqpoint{0.795890in}{0.584584in}}%
\pgfpathlineto{\pgfqpoint{0.781323in}{0.597073in}}%
\pgfpathlineto{\pgfqpoint{0.766095in}{0.610272in}}%
\pgfusepath{stroke}%
\end{pgfscope}%
\begin{pgfscope}%
\pgfpathrectangle{\pgfqpoint{0.766095in}{0.571603in}}{\pgfqpoint{5.929283in}{5.068436in}}%
\pgfusepath{clip}%
\pgfsetrectcap%
\pgfsetroundjoin%
\pgfsetlinewidth{1.505625pt}%
\definecolor{currentstroke}{rgb}{1.000000,0.000000,0.000000}%
\pgfsetstrokecolor{currentstroke}%
\pgfsetdash{}{0pt}%
\pgfpathmoveto{\pgfqpoint{3.730736in}{3.950560in}}%
\pgfpathlineto{\pgfqpoint{3.730736in}{2.261082in}}%
\pgfpathlineto{\pgfqpoint{4.471897in}{2.081101in}}%
\pgfpathlineto{\pgfqpoint{4.920619in}{1.171096in}}%
\pgfpathlineto{\pgfqpoint{5.083653in}{1.565137in}}%
\pgfpathlineto{\pgfqpoint{4.786737in}{1.603683in}}%
\pgfpathlineto{\pgfqpoint{4.751720in}{1.478109in}}%
\pgfpathlineto{\pgfqpoint{4.938218in}{1.446199in}}%
\pgfpathlineto{\pgfqpoint{4.900795in}{1.399325in}}%
\pgfpathlineto{\pgfqpoint{4.862830in}{1.461929in}}%
\pgfpathlineto{\pgfqpoint{4.873938in}{1.442640in}}%
\pgfpathlineto{\pgfqpoint{4.875641in}{1.441306in}}%
\pgfpathlineto{\pgfqpoint{4.883213in}{1.454066in}}%
\pgfpathlineto{\pgfqpoint{4.881655in}{1.445629in}}%
\pgfpathlineto{\pgfqpoint{4.882384in}{1.447322in}}%
\pgfpathlineto{\pgfqpoint{4.881029in}{1.447129in}}%
\pgfusepath{stroke}%
\end{pgfscope}%
\begin{pgfscope}%
\pgfpathrectangle{\pgfqpoint{0.766095in}{0.571603in}}{\pgfqpoint{5.929283in}{5.068436in}}%
\pgfusepath{clip}%
\pgfsetbuttcap%
\pgfsetroundjoin%
\definecolor{currentfill}{rgb}{1.000000,0.000000,0.000000}%
\pgfsetfillcolor{currentfill}%
\pgfsetlinewidth{1.003750pt}%
\definecolor{currentstroke}{rgb}{1.000000,0.000000,0.000000}%
\pgfsetstrokecolor{currentstroke}%
\pgfsetdash{}{0pt}%
\pgfsys@defobject{currentmarker}{\pgfqpoint{-0.041667in}{-0.041667in}}{\pgfqpoint{0.041667in}{0.041667in}}{%
\pgfpathmoveto{\pgfqpoint{0.000000in}{-0.041667in}}%
\pgfpathcurveto{\pgfqpoint{0.011050in}{-0.041667in}}{\pgfqpoint{0.021649in}{-0.037276in}}{\pgfqpoint{0.029463in}{-0.029463in}}%
\pgfpathcurveto{\pgfqpoint{0.037276in}{-0.021649in}}{\pgfqpoint{0.041667in}{-0.011050in}}{\pgfqpoint{0.041667in}{0.000000in}}%
\pgfpathcurveto{\pgfqpoint{0.041667in}{0.011050in}}{\pgfqpoint{0.037276in}{0.021649in}}{\pgfqpoint{0.029463in}{0.029463in}}%
\pgfpathcurveto{\pgfqpoint{0.021649in}{0.037276in}}{\pgfqpoint{0.011050in}{0.041667in}}{\pgfqpoint{0.000000in}{0.041667in}}%
\pgfpathcurveto{\pgfqpoint{-0.011050in}{0.041667in}}{\pgfqpoint{-0.021649in}{0.037276in}}{\pgfqpoint{-0.029463in}{0.029463in}}%
\pgfpathcurveto{\pgfqpoint{-0.037276in}{0.021649in}}{\pgfqpoint{-0.041667in}{0.011050in}}{\pgfqpoint{-0.041667in}{0.000000in}}%
\pgfpathcurveto{\pgfqpoint{-0.041667in}{-0.011050in}}{\pgfqpoint{-0.037276in}{-0.021649in}}{\pgfqpoint{-0.029463in}{-0.029463in}}%
\pgfpathcurveto{\pgfqpoint{-0.021649in}{-0.037276in}}{\pgfqpoint{-0.011050in}{-0.041667in}}{\pgfqpoint{0.000000in}{-0.041667in}}%
\pgfpathlineto{\pgfqpoint{0.000000in}{-0.041667in}}%
\pgfpathclose%
\pgfusepath{stroke,fill}%
}%
\begin{pgfscope}%
\pgfsys@transformshift{3.730736in}{3.950560in}%
\pgfsys@useobject{currentmarker}{}%
\end{pgfscope}%
\begin{pgfscope}%
\pgfsys@transformshift{3.730736in}{2.261082in}%
\pgfsys@useobject{currentmarker}{}%
\end{pgfscope}%
\begin{pgfscope}%
\pgfsys@transformshift{4.471897in}{2.081101in}%
\pgfsys@useobject{currentmarker}{}%
\end{pgfscope}%
\begin{pgfscope}%
\pgfsys@transformshift{4.920619in}{1.171096in}%
\pgfsys@useobject{currentmarker}{}%
\end{pgfscope}%
\begin{pgfscope}%
\pgfsys@transformshift{5.083653in}{1.565137in}%
\pgfsys@useobject{currentmarker}{}%
\end{pgfscope}%
\begin{pgfscope}%
\pgfsys@transformshift{4.786737in}{1.603683in}%
\pgfsys@useobject{currentmarker}{}%
\end{pgfscope}%
\begin{pgfscope}%
\pgfsys@transformshift{4.751720in}{1.478109in}%
\pgfsys@useobject{currentmarker}{}%
\end{pgfscope}%
\begin{pgfscope}%
\pgfsys@transformshift{4.938218in}{1.446199in}%
\pgfsys@useobject{currentmarker}{}%
\end{pgfscope}%
\begin{pgfscope}%
\pgfsys@transformshift{4.900795in}{1.399325in}%
\pgfsys@useobject{currentmarker}{}%
\end{pgfscope}%
\begin{pgfscope}%
\pgfsys@transformshift{4.862830in}{1.461929in}%
\pgfsys@useobject{currentmarker}{}%
\end{pgfscope}%
\begin{pgfscope}%
\pgfsys@transformshift{4.873938in}{1.442640in}%
\pgfsys@useobject{currentmarker}{}%
\end{pgfscope}%
\begin{pgfscope}%
\pgfsys@transformshift{4.875641in}{1.441306in}%
\pgfsys@useobject{currentmarker}{}%
\end{pgfscope}%
\begin{pgfscope}%
\pgfsys@transformshift{4.883213in}{1.454066in}%
\pgfsys@useobject{currentmarker}{}%
\end{pgfscope}%
\begin{pgfscope}%
\pgfsys@transformshift{4.881655in}{1.445629in}%
\pgfsys@useobject{currentmarker}{}%
\end{pgfscope}%
\begin{pgfscope}%
\pgfsys@transformshift{4.882384in}{1.447322in}%
\pgfsys@useobject{currentmarker}{}%
\end{pgfscope}%
\begin{pgfscope}%
\pgfsys@transformshift{4.881029in}{1.447129in}%
\pgfsys@useobject{currentmarker}{}%
\end{pgfscope}%
\end{pgfscope}%
\begin{pgfscope}%
\pgfsetrectcap%
\pgfsetmiterjoin%
\pgfsetlinewidth{0.803000pt}%
\definecolor{currentstroke}{rgb}{0.000000,0.000000,0.000000}%
\pgfsetstrokecolor{currentstroke}%
\pgfsetdash{}{0pt}%
\pgfpathmoveto{\pgfqpoint{0.766095in}{0.571603in}}%
\pgfpathlineto{\pgfqpoint{0.766095in}{5.640039in}}%
\pgfusepath{stroke}%
\end{pgfscope}%
\begin{pgfscope}%
\pgfsetrectcap%
\pgfsetmiterjoin%
\pgfsetlinewidth{0.803000pt}%
\definecolor{currentstroke}{rgb}{0.000000,0.000000,0.000000}%
\pgfsetstrokecolor{currentstroke}%
\pgfsetdash{}{0pt}%
\pgfpathmoveto{\pgfqpoint{6.695378in}{0.571603in}}%
\pgfpathlineto{\pgfqpoint{6.695378in}{5.640039in}}%
\pgfusepath{stroke}%
\end{pgfscope}%
\begin{pgfscope}%
\pgfsetrectcap%
\pgfsetmiterjoin%
\pgfsetlinewidth{0.803000pt}%
\definecolor{currentstroke}{rgb}{0.000000,0.000000,0.000000}%
\pgfsetstrokecolor{currentstroke}%
\pgfsetdash{}{0pt}%
\pgfpathmoveto{\pgfqpoint{0.766095in}{0.571603in}}%
\pgfpathlineto{\pgfqpoint{6.695378in}{0.571603in}}%
\pgfusepath{stroke}%
\end{pgfscope}%
\begin{pgfscope}%
\pgfsetrectcap%
\pgfsetmiterjoin%
\pgfsetlinewidth{0.803000pt}%
\definecolor{currentstroke}{rgb}{0.000000,0.000000,0.000000}%
\pgfsetstrokecolor{currentstroke}%
\pgfsetdash{}{0pt}%
\pgfpathmoveto{\pgfqpoint{0.766095in}{5.640039in}}%
\pgfpathlineto{\pgfqpoint{6.695378in}{5.640039in}}%
\pgfusepath{stroke}%
\end{pgfscope}%
\begin{pgfscope}%
\definecolor{textcolor}{rgb}{0.000000,0.000000,0.000000}%
\pgfsetstrokecolor{textcolor}%
\pgfsetfillcolor{textcolor}%
\pgftext[x=3.730736in,y=5.723372in,,base]{\color{textcolor}\sffamily\fontsize{12.000000}{14.400000}\selectfont 2D Contour Plot}%
\end{pgfscope}%
\begin{pgfscope}%
\pgfsetbuttcap%
\pgfsetmiterjoin%
\definecolor{currentfill}{rgb}{1.000000,1.000000,1.000000}%
\pgfsetfillcolor{currentfill}%
\pgfsetfillopacity{0.800000}%
\pgfsetlinewidth{1.003750pt}%
\definecolor{currentstroke}{rgb}{0.800000,0.800000,0.800000}%
\pgfsetstrokecolor{currentstroke}%
\pgfsetstrokeopacity{0.800000}%
\pgfsetdash{}{0pt}%
\pgfpathmoveto{\pgfqpoint{5.492225in}{5.121213in}}%
\pgfpathlineto{\pgfqpoint{6.598155in}{5.121213in}}%
\pgfpathquadraticcurveto{\pgfqpoint{6.625933in}{5.121213in}}{\pgfqpoint{6.625933in}{5.148991in}}%
\pgfpathlineto{\pgfqpoint{6.625933in}{5.542817in}}%
\pgfpathquadraticcurveto{\pgfqpoint{6.625933in}{5.570595in}}{\pgfqpoint{6.598155in}{5.570595in}}%
\pgfpathlineto{\pgfqpoint{5.492225in}{5.570595in}}%
\pgfpathquadraticcurveto{\pgfqpoint{5.464448in}{5.570595in}}{\pgfqpoint{5.464448in}{5.542817in}}%
\pgfpathlineto{\pgfqpoint{5.464448in}{5.148991in}}%
\pgfpathquadraticcurveto{\pgfqpoint{5.464448in}{5.121213in}}{\pgfqpoint{5.492225in}{5.121213in}}%
\pgfpathlineto{\pgfqpoint{5.492225in}{5.121213in}}%
\pgfpathclose%
\pgfusepath{stroke,fill}%
\end{pgfscope}%
\begin{pgfscope}%
\pgfsetrectcap%
\pgfsetroundjoin%
\pgfsetlinewidth{1.505625pt}%
\definecolor{currentstroke}{rgb}{1.000000,0.000000,0.000000}%
\pgfsetstrokecolor{currentstroke}%
\pgfsetdash{}{0pt}%
\pgfpathmoveto{\pgfqpoint{5.520003in}{5.458127in}}%
\pgfpathlineto{\pgfqpoint{5.658892in}{5.458127in}}%
\pgfpathlineto{\pgfqpoint{5.797781in}{5.458127in}}%
\pgfusepath{stroke}%
\end{pgfscope}%
\begin{pgfscope}%
\pgfsetbuttcap%
\pgfsetroundjoin%
\definecolor{currentfill}{rgb}{1.000000,0.000000,0.000000}%
\pgfsetfillcolor{currentfill}%
\pgfsetlinewidth{1.003750pt}%
\definecolor{currentstroke}{rgb}{1.000000,0.000000,0.000000}%
\pgfsetstrokecolor{currentstroke}%
\pgfsetdash{}{0pt}%
\pgfsys@defobject{currentmarker}{\pgfqpoint{-0.041667in}{-0.041667in}}{\pgfqpoint{0.041667in}{0.041667in}}{%
\pgfpathmoveto{\pgfqpoint{0.000000in}{-0.041667in}}%
\pgfpathcurveto{\pgfqpoint{0.011050in}{-0.041667in}}{\pgfqpoint{0.021649in}{-0.037276in}}{\pgfqpoint{0.029463in}{-0.029463in}}%
\pgfpathcurveto{\pgfqpoint{0.037276in}{-0.021649in}}{\pgfqpoint{0.041667in}{-0.011050in}}{\pgfqpoint{0.041667in}{0.000000in}}%
\pgfpathcurveto{\pgfqpoint{0.041667in}{0.011050in}}{\pgfqpoint{0.037276in}{0.021649in}}{\pgfqpoint{0.029463in}{0.029463in}}%
\pgfpathcurveto{\pgfqpoint{0.021649in}{0.037276in}}{\pgfqpoint{0.011050in}{0.041667in}}{\pgfqpoint{0.000000in}{0.041667in}}%
\pgfpathcurveto{\pgfqpoint{-0.011050in}{0.041667in}}{\pgfqpoint{-0.021649in}{0.037276in}}{\pgfqpoint{-0.029463in}{0.029463in}}%
\pgfpathcurveto{\pgfqpoint{-0.037276in}{0.021649in}}{\pgfqpoint{-0.041667in}{0.011050in}}{\pgfqpoint{-0.041667in}{0.000000in}}%
\pgfpathcurveto{\pgfqpoint{-0.041667in}{-0.011050in}}{\pgfqpoint{-0.037276in}{-0.021649in}}{\pgfqpoint{-0.029463in}{-0.029463in}}%
\pgfpathcurveto{\pgfqpoint{-0.021649in}{-0.037276in}}{\pgfqpoint{-0.011050in}{-0.041667in}}{\pgfqpoint{0.000000in}{-0.041667in}}%
\pgfpathlineto{\pgfqpoint{0.000000in}{-0.041667in}}%
\pgfpathclose%
\pgfusepath{stroke,fill}%
}%
\begin{pgfscope}%
\pgfsys@transformshift{5.658892in}{5.458127in}%
\pgfsys@useobject{currentmarker}{}%
\end{pgfscope}%
\end{pgfscope}%
\begin{pgfscope}%
\definecolor{textcolor}{rgb}{0.000000,0.000000,0.000000}%
\pgfsetstrokecolor{textcolor}%
\pgfsetfillcolor{textcolor}%
\pgftext[x=5.908892in,y=5.409516in,left,base]{\color{textcolor}\sffamily\fontsize{10.000000}{12.000000}\selectfont Iterations}%
\end{pgfscope}%
\begin{pgfscope}%
\pgfsetbuttcap%
\pgfsetroundjoin%
\definecolor{currentfill}{rgb}{0.000000,0.000000,1.000000}%
\pgfsetfillcolor{currentfill}%
\pgfsetlinewidth{1.003750pt}%
\definecolor{currentstroke}{rgb}{0.000000,0.000000,1.000000}%
\pgfsetstrokecolor{currentstroke}%
\pgfsetdash{}{0pt}%
\pgfsys@defobject{currentmarker}{\pgfqpoint{-0.069444in}{-0.069444in}}{\pgfqpoint{0.069444in}{0.069444in}}{%
\pgfpathmoveto{\pgfqpoint{0.000000in}{-0.069444in}}%
\pgfpathcurveto{\pgfqpoint{0.018417in}{-0.069444in}}{\pgfqpoint{0.036082in}{-0.062127in}}{\pgfqpoint{0.049105in}{-0.049105in}}%
\pgfpathcurveto{\pgfqpoint{0.062127in}{-0.036082in}}{\pgfqpoint{0.069444in}{-0.018417in}}{\pgfqpoint{0.069444in}{0.000000in}}%
\pgfpathcurveto{\pgfqpoint{0.069444in}{0.018417in}}{\pgfqpoint{0.062127in}{0.036082in}}{\pgfqpoint{0.049105in}{0.049105in}}%
\pgfpathcurveto{\pgfqpoint{0.036082in}{0.062127in}}{\pgfqpoint{0.018417in}{0.069444in}}{\pgfqpoint{0.000000in}{0.069444in}}%
\pgfpathcurveto{\pgfqpoint{-0.018417in}{0.069444in}}{\pgfqpoint{-0.036082in}{0.062127in}}{\pgfqpoint{-0.049105in}{0.049105in}}%
\pgfpathcurveto{\pgfqpoint{-0.062127in}{0.036082in}}{\pgfqpoint{-0.069444in}{0.018417in}}{\pgfqpoint{-0.069444in}{0.000000in}}%
\pgfpathcurveto{\pgfqpoint{-0.069444in}{-0.018417in}}{\pgfqpoint{-0.062127in}{-0.036082in}}{\pgfqpoint{-0.049105in}{-0.049105in}}%
\pgfpathcurveto{\pgfqpoint{-0.036082in}{-0.062127in}}{\pgfqpoint{-0.018417in}{-0.069444in}}{\pgfqpoint{0.000000in}{-0.069444in}}%
\pgfpathlineto{\pgfqpoint{0.000000in}{-0.069444in}}%
\pgfpathclose%
\pgfusepath{stroke,fill}%
}%
\begin{pgfscope}%
\pgfsys@transformshift{5.658892in}{5.242117in}%
\pgfsys@useobject{currentmarker}{}%
\end{pgfscope}%
\end{pgfscope}%
\begin{pgfscope}%
\definecolor{textcolor}{rgb}{0.000000,0.000000,0.000000}%
\pgfsetstrokecolor{textcolor}%
\pgfsetfillcolor{textcolor}%
\pgftext[x=5.908892in,y=5.205659in,left,base]{\color{textcolor}\sffamily\fontsize{10.000000}{12.000000}\selectfont Minimum}%
\end{pgfscope}%
\end{pgfpicture}%
\makeatother%
\endgroup%
}
        \caption{Pohľad zhora (Vrstevnice)}
        \label{fig:newton_vlavo}
    \end{subfigure}
    \hfill
    \begin{subfigure}{0.48\textwidth}
        \centering
        \resizebox{\linewidth}{!}{\input{grafy/cg_surface_1.pgf}}
        \caption{3D graf funkcie}
        \label{fig:newton_vpravo}
    \end{subfigure}

    \label{fig:newton_komplet}
\end{figure}



\newpage
\noindent \textbf{Počiatočný bod} $x^{[0]} = [1.5; 0.5]$ \\
V tomto prípade je funkcia strmšia (vplyvom exponenciály), čo môže ovplyvniť veľkosť kroku v prvej iterácii.

\begin{table}[h!]
    \centering
    \begin{tabular}{cccc}
        \toprule
        \textbf{Iterácia} & \textbf{Bod } $x^{[k]} = [x;y]$ & \textbf{Hodnota } $f(x^{[k]})$ & \textbf{Norma } $\|\nabla f\|$ \\
        \midrule
        0  & $[1.500000;\; 0.500000]$   & 4.934351 & -- \\
        1  & $[-0.138435;\; 0.925639]$  & 4.524030 & 3.385639 \\
        2  & $[0.396305;\; -0.626467]$  & 1.603598 & 6.566560 \\
        \dots & \dots & \dots & \dots \\
        9  & $[0.387814;\; -0.741050]$  & 1.568997 & 0.015270 \\
        10 & $[0.388119;\; -0.740641]$  & 1.568997 & 0.002043 \\
        11 & $[0.388183;\; -0.740998]$  & 1.568996 & 0.001725 \\
        \bottomrule
    \end{tabular}
    \caption{Priebeh MSG pre $x^{[0]} = [1.5;0.5]$.}
\end{table}

\begin{figure}[H]
    \centering

    \begin{subfigure}{0.48\textwidth}
        \centering
        \resizebox{\linewidth}{!}{%% Creator: Matplotlib, PGF backend
%%
%% To include the figure in your LaTeX document, write
%%   \input{<filename>.pgf}
%%
%% Make sure the required packages are loaded in your preamble
%%   \usepackage{pgf}
%%
%% Also ensure that all the required font packages are loaded; for instance,
%% the lmodern package is sometimes necessary when using math font.
%%   \usepackage{lmodern}
%%
%% Figures using additional raster images can only be included by \input if
%% they are in the same directory as the main LaTeX file. For loading figures
%% from other directories you can use the `import` package
%%   \usepackage{import}
%%
%% and then include the figures with
%%   \import{<path to file>}{<filename>.pgf}
%%
%% Matplotlib used the following preamble
%%   
%%   \usepackage{fontspec}
%%   \setmainfont{DejaVuSerif.ttf}[Path=\detokenize{/home/radimek/Documents/projekt_mat_prog/mat_prog_kernel/lib/python3.12/site-packages/matplotlib/mpl-data/fonts/ttf/}]
%%   \setsansfont{DejaVuSans.ttf}[Path=\detokenize{/home/radimek/Documents/projekt_mat_prog/mat_prog_kernel/lib/python3.12/site-packages/matplotlib/mpl-data/fonts/ttf/}]
%%   \setmonofont{DejaVuSansMono.ttf}[Path=\detokenize{/home/radimek/Documents/projekt_mat_prog/mat_prog_kernel/lib/python3.12/site-packages/matplotlib/mpl-data/fonts/ttf/}]
%%   \makeatletter\@ifpackageloaded{underscore}{}{\usepackage[strings]{underscore}}\makeatother
%%
\begingroup%
\makeatletter%
\begin{pgfpicture}%
\pgfpathrectangle{\pgfpointorigin}{\pgfqpoint{7.000000in}{6.000000in}}%
\pgfusepath{use as bounding box, clip}%
\begin{pgfscope}%
\pgfsetbuttcap%
\pgfsetmiterjoin%
\definecolor{currentfill}{rgb}{1.000000,1.000000,1.000000}%
\pgfsetfillcolor{currentfill}%
\pgfsetlinewidth{0.000000pt}%
\definecolor{currentstroke}{rgb}{1.000000,1.000000,1.000000}%
\pgfsetstrokecolor{currentstroke}%
\pgfsetdash{}{0pt}%
\pgfpathmoveto{\pgfqpoint{0.000000in}{0.000000in}}%
\pgfpathlineto{\pgfqpoint{7.000000in}{0.000000in}}%
\pgfpathlineto{\pgfqpoint{7.000000in}{6.000000in}}%
\pgfpathlineto{\pgfqpoint{0.000000in}{6.000000in}}%
\pgfpathlineto{\pgfqpoint{0.000000in}{0.000000in}}%
\pgfpathclose%
\pgfusepath{fill}%
\end{pgfscope}%
\begin{pgfscope}%
\pgfsetbuttcap%
\pgfsetmiterjoin%
\definecolor{currentfill}{rgb}{1.000000,1.000000,1.000000}%
\pgfsetfillcolor{currentfill}%
\pgfsetlinewidth{0.000000pt}%
\definecolor{currentstroke}{rgb}{0.000000,0.000000,0.000000}%
\pgfsetstrokecolor{currentstroke}%
\pgfsetstrokeopacity{0.000000}%
\pgfsetdash{}{0pt}%
\pgfpathmoveto{\pgfqpoint{0.854460in}{0.571603in}}%
\pgfpathlineto{\pgfqpoint{6.739560in}{0.571603in}}%
\pgfpathlineto{\pgfqpoint{6.739560in}{5.640039in}}%
\pgfpathlineto{\pgfqpoint{0.854460in}{5.640039in}}%
\pgfpathlineto{\pgfqpoint{0.854460in}{0.571603in}}%
\pgfpathclose%
\pgfusepath{fill}%
\end{pgfscope}%
\begin{pgfscope}%
\pgfpathrectangle{\pgfqpoint{0.854460in}{0.571603in}}{\pgfqpoint{5.885100in}{5.068436in}}%
\pgfusepath{clip}%
\pgfsetbuttcap%
\pgfsetroundjoin%
\definecolor{currentfill}{rgb}{0.000000,0.000000,1.000000}%
\pgfsetfillcolor{currentfill}%
\pgfsetlinewidth{1.003750pt}%
\definecolor{currentstroke}{rgb}{0.000000,0.000000,1.000000}%
\pgfsetstrokecolor{currentstroke}%
\pgfsetdash{}{0pt}%
\pgfsys@defobject{currentmarker}{\pgfqpoint{-0.069444in}{-0.069444in}}{\pgfqpoint{0.069444in}{0.069444in}}{%
\pgfpathmoveto{\pgfqpoint{0.000000in}{-0.069444in}}%
\pgfpathcurveto{\pgfqpoint{0.018417in}{-0.069444in}}{\pgfqpoint{0.036082in}{-0.062127in}}{\pgfqpoint{0.049105in}{-0.049105in}}%
\pgfpathcurveto{\pgfqpoint{0.062127in}{-0.036082in}}{\pgfqpoint{0.069444in}{-0.018417in}}{\pgfqpoint{0.069444in}{0.000000in}}%
\pgfpathcurveto{\pgfqpoint{0.069444in}{0.018417in}}{\pgfqpoint{0.062127in}{0.036082in}}{\pgfqpoint{0.049105in}{0.049105in}}%
\pgfpathcurveto{\pgfqpoint{0.036082in}{0.062127in}}{\pgfqpoint{0.018417in}{0.069444in}}{\pgfqpoint{0.000000in}{0.069444in}}%
\pgfpathcurveto{\pgfqpoint{-0.018417in}{0.069444in}}{\pgfqpoint{-0.036082in}{0.062127in}}{\pgfqpoint{-0.049105in}{0.049105in}}%
\pgfpathcurveto{\pgfqpoint{-0.062127in}{0.036082in}}{\pgfqpoint{-0.069444in}{0.018417in}}{\pgfqpoint{-0.069444in}{0.000000in}}%
\pgfpathcurveto{\pgfqpoint{-0.069444in}{-0.018417in}}{\pgfqpoint{-0.062127in}{-0.036082in}}{\pgfqpoint{-0.049105in}{-0.049105in}}%
\pgfpathcurveto{\pgfqpoint{-0.036082in}{-0.062127in}}{\pgfqpoint{-0.018417in}{-0.069444in}}{\pgfqpoint{0.000000in}{-0.069444in}}%
\pgfpathlineto{\pgfqpoint{0.000000in}{-0.069444in}}%
\pgfpathclose%
\pgfusepath{stroke,fill}%
}%
\begin{pgfscope}%
\pgfsys@transformshift{3.577658in}{1.227971in}%
\pgfsys@useobject{currentmarker}{}%
\end{pgfscope}%
\end{pgfscope}%
\begin{pgfscope}%
\pgfsetbuttcap%
\pgfsetroundjoin%
\definecolor{currentfill}{rgb}{0.000000,0.000000,0.000000}%
\pgfsetfillcolor{currentfill}%
\pgfsetlinewidth{0.803000pt}%
\definecolor{currentstroke}{rgb}{0.000000,0.000000,0.000000}%
\pgfsetstrokecolor{currentstroke}%
\pgfsetdash{}{0pt}%
\pgfsys@defobject{currentmarker}{\pgfqpoint{0.000000in}{-0.048611in}}{\pgfqpoint{0.000000in}{0.000000in}}{%
\pgfpathmoveto{\pgfqpoint{0.000000in}{0.000000in}}%
\pgfpathlineto{\pgfqpoint{0.000000in}{-0.048611in}}%
\pgfusepath{stroke,fill}%
}%
\begin{pgfscope}%
\pgfsys@transformshift{0.854460in}{0.571603in}%
\pgfsys@useobject{currentmarker}{}%
\end{pgfscope}%
\end{pgfscope}%
\begin{pgfscope}%
\definecolor{textcolor}{rgb}{0.000000,0.000000,0.000000}%
\pgfsetstrokecolor{textcolor}%
\pgfsetfillcolor{textcolor}%
\pgftext[x=0.854460in,y=0.474381in,,top]{\color{textcolor}\sffamily\fontsize{10.000000}{12.000000}\selectfont \ensuremath{-}1.0}%
\end{pgfscope}%
\begin{pgfscope}%
\pgfsetbuttcap%
\pgfsetroundjoin%
\definecolor{currentfill}{rgb}{0.000000,0.000000,0.000000}%
\pgfsetfillcolor{currentfill}%
\pgfsetlinewidth{0.803000pt}%
\definecolor{currentstroke}{rgb}{0.000000,0.000000,0.000000}%
\pgfsetstrokecolor{currentstroke}%
\pgfsetdash{}{0pt}%
\pgfsys@defobject{currentmarker}{\pgfqpoint{0.000000in}{-0.048611in}}{\pgfqpoint{0.000000in}{0.000000in}}{%
\pgfpathmoveto{\pgfqpoint{0.000000in}{0.000000in}}%
\pgfpathlineto{\pgfqpoint{0.000000in}{-0.048611in}}%
\pgfusepath{stroke,fill}%
}%
\begin{pgfscope}%
\pgfsys@transformshift{1.835310in}{0.571603in}%
\pgfsys@useobject{currentmarker}{}%
\end{pgfscope}%
\end{pgfscope}%
\begin{pgfscope}%
\definecolor{textcolor}{rgb}{0.000000,0.000000,0.000000}%
\pgfsetstrokecolor{textcolor}%
\pgfsetfillcolor{textcolor}%
\pgftext[x=1.835310in,y=0.474381in,,top]{\color{textcolor}\sffamily\fontsize{10.000000}{12.000000}\selectfont \ensuremath{-}0.5}%
\end{pgfscope}%
\begin{pgfscope}%
\pgfsetbuttcap%
\pgfsetroundjoin%
\definecolor{currentfill}{rgb}{0.000000,0.000000,0.000000}%
\pgfsetfillcolor{currentfill}%
\pgfsetlinewidth{0.803000pt}%
\definecolor{currentstroke}{rgb}{0.000000,0.000000,0.000000}%
\pgfsetstrokecolor{currentstroke}%
\pgfsetdash{}{0pt}%
\pgfsys@defobject{currentmarker}{\pgfqpoint{0.000000in}{-0.048611in}}{\pgfqpoint{0.000000in}{0.000000in}}{%
\pgfpathmoveto{\pgfqpoint{0.000000in}{0.000000in}}%
\pgfpathlineto{\pgfqpoint{0.000000in}{-0.048611in}}%
\pgfusepath{stroke,fill}%
}%
\begin{pgfscope}%
\pgfsys@transformshift{2.816160in}{0.571603in}%
\pgfsys@useobject{currentmarker}{}%
\end{pgfscope}%
\end{pgfscope}%
\begin{pgfscope}%
\definecolor{textcolor}{rgb}{0.000000,0.000000,0.000000}%
\pgfsetstrokecolor{textcolor}%
\pgfsetfillcolor{textcolor}%
\pgftext[x=2.816160in,y=0.474381in,,top]{\color{textcolor}\sffamily\fontsize{10.000000}{12.000000}\selectfont 0.0}%
\end{pgfscope}%
\begin{pgfscope}%
\pgfsetbuttcap%
\pgfsetroundjoin%
\definecolor{currentfill}{rgb}{0.000000,0.000000,0.000000}%
\pgfsetfillcolor{currentfill}%
\pgfsetlinewidth{0.803000pt}%
\definecolor{currentstroke}{rgb}{0.000000,0.000000,0.000000}%
\pgfsetstrokecolor{currentstroke}%
\pgfsetdash{}{0pt}%
\pgfsys@defobject{currentmarker}{\pgfqpoint{0.000000in}{-0.048611in}}{\pgfqpoint{0.000000in}{0.000000in}}{%
\pgfpathmoveto{\pgfqpoint{0.000000in}{0.000000in}}%
\pgfpathlineto{\pgfqpoint{0.000000in}{-0.048611in}}%
\pgfusepath{stroke,fill}%
}%
\begin{pgfscope}%
\pgfsys@transformshift{3.797010in}{0.571603in}%
\pgfsys@useobject{currentmarker}{}%
\end{pgfscope}%
\end{pgfscope}%
\begin{pgfscope}%
\definecolor{textcolor}{rgb}{0.000000,0.000000,0.000000}%
\pgfsetstrokecolor{textcolor}%
\pgfsetfillcolor{textcolor}%
\pgftext[x=3.797010in,y=0.474381in,,top]{\color{textcolor}\sffamily\fontsize{10.000000}{12.000000}\selectfont 0.5}%
\end{pgfscope}%
\begin{pgfscope}%
\pgfsetbuttcap%
\pgfsetroundjoin%
\definecolor{currentfill}{rgb}{0.000000,0.000000,0.000000}%
\pgfsetfillcolor{currentfill}%
\pgfsetlinewidth{0.803000pt}%
\definecolor{currentstroke}{rgb}{0.000000,0.000000,0.000000}%
\pgfsetstrokecolor{currentstroke}%
\pgfsetdash{}{0pt}%
\pgfsys@defobject{currentmarker}{\pgfqpoint{0.000000in}{-0.048611in}}{\pgfqpoint{0.000000in}{0.000000in}}{%
\pgfpathmoveto{\pgfqpoint{0.000000in}{0.000000in}}%
\pgfpathlineto{\pgfqpoint{0.000000in}{-0.048611in}}%
\pgfusepath{stroke,fill}%
}%
\begin{pgfscope}%
\pgfsys@transformshift{4.777860in}{0.571603in}%
\pgfsys@useobject{currentmarker}{}%
\end{pgfscope}%
\end{pgfscope}%
\begin{pgfscope}%
\definecolor{textcolor}{rgb}{0.000000,0.000000,0.000000}%
\pgfsetstrokecolor{textcolor}%
\pgfsetfillcolor{textcolor}%
\pgftext[x=4.777860in,y=0.474381in,,top]{\color{textcolor}\sffamily\fontsize{10.000000}{12.000000}\selectfont 1.0}%
\end{pgfscope}%
\begin{pgfscope}%
\pgfsetbuttcap%
\pgfsetroundjoin%
\definecolor{currentfill}{rgb}{0.000000,0.000000,0.000000}%
\pgfsetfillcolor{currentfill}%
\pgfsetlinewidth{0.803000pt}%
\definecolor{currentstroke}{rgb}{0.000000,0.000000,0.000000}%
\pgfsetstrokecolor{currentstroke}%
\pgfsetdash{}{0pt}%
\pgfsys@defobject{currentmarker}{\pgfqpoint{0.000000in}{-0.048611in}}{\pgfqpoint{0.000000in}{0.000000in}}{%
\pgfpathmoveto{\pgfqpoint{0.000000in}{0.000000in}}%
\pgfpathlineto{\pgfqpoint{0.000000in}{-0.048611in}}%
\pgfusepath{stroke,fill}%
}%
\begin{pgfscope}%
\pgfsys@transformshift{5.758710in}{0.571603in}%
\pgfsys@useobject{currentmarker}{}%
\end{pgfscope}%
\end{pgfscope}%
\begin{pgfscope}%
\definecolor{textcolor}{rgb}{0.000000,0.000000,0.000000}%
\pgfsetstrokecolor{textcolor}%
\pgfsetfillcolor{textcolor}%
\pgftext[x=5.758710in,y=0.474381in,,top]{\color{textcolor}\sffamily\fontsize{10.000000}{12.000000}\selectfont 1.5}%
\end{pgfscope}%
\begin{pgfscope}%
\pgfsetbuttcap%
\pgfsetroundjoin%
\definecolor{currentfill}{rgb}{0.000000,0.000000,0.000000}%
\pgfsetfillcolor{currentfill}%
\pgfsetlinewidth{0.803000pt}%
\definecolor{currentstroke}{rgb}{0.000000,0.000000,0.000000}%
\pgfsetstrokecolor{currentstroke}%
\pgfsetdash{}{0pt}%
\pgfsys@defobject{currentmarker}{\pgfqpoint{0.000000in}{-0.048611in}}{\pgfqpoint{0.000000in}{0.000000in}}{%
\pgfpathmoveto{\pgfqpoint{0.000000in}{0.000000in}}%
\pgfpathlineto{\pgfqpoint{0.000000in}{-0.048611in}}%
\pgfusepath{stroke,fill}%
}%
\begin{pgfscope}%
\pgfsys@transformshift{6.739560in}{0.571603in}%
\pgfsys@useobject{currentmarker}{}%
\end{pgfscope}%
\end{pgfscope}%
\begin{pgfscope}%
\definecolor{textcolor}{rgb}{0.000000,0.000000,0.000000}%
\pgfsetstrokecolor{textcolor}%
\pgfsetfillcolor{textcolor}%
\pgftext[x=6.739560in,y=0.474381in,,top]{\color{textcolor}\sffamily\fontsize{10.000000}{12.000000}\selectfont 2.0}%
\end{pgfscope}%
\begin{pgfscope}%
\definecolor{textcolor}{rgb}{0.000000,0.000000,0.000000}%
\pgfsetstrokecolor{textcolor}%
\pgfsetfillcolor{textcolor}%
\pgftext[x=3.797010in,y=0.284413in,,top]{\color{textcolor}\sffamily\fontsize{10.000000}{12.000000}\selectfont x}%
\end{pgfscope}%
\begin{pgfscope}%
\pgfsetbuttcap%
\pgfsetroundjoin%
\definecolor{currentfill}{rgb}{0.000000,0.000000,0.000000}%
\pgfsetfillcolor{currentfill}%
\pgfsetlinewidth{0.803000pt}%
\definecolor{currentstroke}{rgb}{0.000000,0.000000,0.000000}%
\pgfsetstrokecolor{currentstroke}%
\pgfsetdash{}{0pt}%
\pgfsys@defobject{currentmarker}{\pgfqpoint{-0.048611in}{0.000000in}}{\pgfqpoint{-0.000000in}{0.000000in}}{%
\pgfpathmoveto{\pgfqpoint{-0.000000in}{0.000000in}}%
\pgfpathlineto{\pgfqpoint{-0.048611in}{0.000000in}}%
\pgfusepath{stroke,fill}%
}%
\begin{pgfscope}%
\pgfsys@transformshift{0.854460in}{0.571603in}%
\pgfsys@useobject{currentmarker}{}%
\end{pgfscope}%
\end{pgfscope}%
\begin{pgfscope}%
\definecolor{textcolor}{rgb}{0.000000,0.000000,0.000000}%
\pgfsetstrokecolor{textcolor}%
\pgfsetfillcolor{textcolor}%
\pgftext[x=0.339968in, y=0.518842in, left, base]{\color{textcolor}\sffamily\fontsize{10.000000}{12.000000}\selectfont \ensuremath{-}1.00}%
\end{pgfscope}%
\begin{pgfscope}%
\pgfsetbuttcap%
\pgfsetroundjoin%
\definecolor{currentfill}{rgb}{0.000000,0.000000,0.000000}%
\pgfsetfillcolor{currentfill}%
\pgfsetlinewidth{0.803000pt}%
\definecolor{currentstroke}{rgb}{0.000000,0.000000,0.000000}%
\pgfsetstrokecolor{currentstroke}%
\pgfsetdash{}{0pt}%
\pgfsys@defobject{currentmarker}{\pgfqpoint{-0.048611in}{0.000000in}}{\pgfqpoint{-0.000000in}{0.000000in}}{%
\pgfpathmoveto{\pgfqpoint{-0.000000in}{0.000000in}}%
\pgfpathlineto{\pgfqpoint{-0.048611in}{0.000000in}}%
\pgfusepath{stroke,fill}%
}%
\begin{pgfscope}%
\pgfsys@transformshift{0.854460in}{1.205158in}%
\pgfsys@useobject{currentmarker}{}%
\end{pgfscope}%
\end{pgfscope}%
\begin{pgfscope}%
\definecolor{textcolor}{rgb}{0.000000,0.000000,0.000000}%
\pgfsetstrokecolor{textcolor}%
\pgfsetfillcolor{textcolor}%
\pgftext[x=0.339968in, y=1.152396in, left, base]{\color{textcolor}\sffamily\fontsize{10.000000}{12.000000}\selectfont \ensuremath{-}0.75}%
\end{pgfscope}%
\begin{pgfscope}%
\pgfsetbuttcap%
\pgfsetroundjoin%
\definecolor{currentfill}{rgb}{0.000000,0.000000,0.000000}%
\pgfsetfillcolor{currentfill}%
\pgfsetlinewidth{0.803000pt}%
\definecolor{currentstroke}{rgb}{0.000000,0.000000,0.000000}%
\pgfsetstrokecolor{currentstroke}%
\pgfsetdash{}{0pt}%
\pgfsys@defobject{currentmarker}{\pgfqpoint{-0.048611in}{0.000000in}}{\pgfqpoint{-0.000000in}{0.000000in}}{%
\pgfpathmoveto{\pgfqpoint{-0.000000in}{0.000000in}}%
\pgfpathlineto{\pgfqpoint{-0.048611in}{0.000000in}}%
\pgfusepath{stroke,fill}%
}%
\begin{pgfscope}%
\pgfsys@transformshift{0.854460in}{1.838712in}%
\pgfsys@useobject{currentmarker}{}%
\end{pgfscope}%
\end{pgfscope}%
\begin{pgfscope}%
\definecolor{textcolor}{rgb}{0.000000,0.000000,0.000000}%
\pgfsetstrokecolor{textcolor}%
\pgfsetfillcolor{textcolor}%
\pgftext[x=0.339968in, y=1.785951in, left, base]{\color{textcolor}\sffamily\fontsize{10.000000}{12.000000}\selectfont \ensuremath{-}0.50}%
\end{pgfscope}%
\begin{pgfscope}%
\pgfsetbuttcap%
\pgfsetroundjoin%
\definecolor{currentfill}{rgb}{0.000000,0.000000,0.000000}%
\pgfsetfillcolor{currentfill}%
\pgfsetlinewidth{0.803000pt}%
\definecolor{currentstroke}{rgb}{0.000000,0.000000,0.000000}%
\pgfsetstrokecolor{currentstroke}%
\pgfsetdash{}{0pt}%
\pgfsys@defobject{currentmarker}{\pgfqpoint{-0.048611in}{0.000000in}}{\pgfqpoint{-0.000000in}{0.000000in}}{%
\pgfpathmoveto{\pgfqpoint{-0.000000in}{0.000000in}}%
\pgfpathlineto{\pgfqpoint{-0.048611in}{0.000000in}}%
\pgfusepath{stroke,fill}%
}%
\begin{pgfscope}%
\pgfsys@transformshift{0.854460in}{2.472267in}%
\pgfsys@useobject{currentmarker}{}%
\end{pgfscope}%
\end{pgfscope}%
\begin{pgfscope}%
\definecolor{textcolor}{rgb}{0.000000,0.000000,0.000000}%
\pgfsetstrokecolor{textcolor}%
\pgfsetfillcolor{textcolor}%
\pgftext[x=0.339968in, y=2.419505in, left, base]{\color{textcolor}\sffamily\fontsize{10.000000}{12.000000}\selectfont \ensuremath{-}0.25}%
\end{pgfscope}%
\begin{pgfscope}%
\pgfsetbuttcap%
\pgfsetroundjoin%
\definecolor{currentfill}{rgb}{0.000000,0.000000,0.000000}%
\pgfsetfillcolor{currentfill}%
\pgfsetlinewidth{0.803000pt}%
\definecolor{currentstroke}{rgb}{0.000000,0.000000,0.000000}%
\pgfsetstrokecolor{currentstroke}%
\pgfsetdash{}{0pt}%
\pgfsys@defobject{currentmarker}{\pgfqpoint{-0.048611in}{0.000000in}}{\pgfqpoint{-0.000000in}{0.000000in}}{%
\pgfpathmoveto{\pgfqpoint{-0.000000in}{0.000000in}}%
\pgfpathlineto{\pgfqpoint{-0.048611in}{0.000000in}}%
\pgfusepath{stroke,fill}%
}%
\begin{pgfscope}%
\pgfsys@transformshift{0.854460in}{3.105821in}%
\pgfsys@useobject{currentmarker}{}%
\end{pgfscope}%
\end{pgfscope}%
\begin{pgfscope}%
\definecolor{textcolor}{rgb}{0.000000,0.000000,0.000000}%
\pgfsetstrokecolor{textcolor}%
\pgfsetfillcolor{textcolor}%
\pgftext[x=0.447993in, y=3.053060in, left, base]{\color{textcolor}\sffamily\fontsize{10.000000}{12.000000}\selectfont 0.00}%
\end{pgfscope}%
\begin{pgfscope}%
\pgfsetbuttcap%
\pgfsetroundjoin%
\definecolor{currentfill}{rgb}{0.000000,0.000000,0.000000}%
\pgfsetfillcolor{currentfill}%
\pgfsetlinewidth{0.803000pt}%
\definecolor{currentstroke}{rgb}{0.000000,0.000000,0.000000}%
\pgfsetstrokecolor{currentstroke}%
\pgfsetdash{}{0pt}%
\pgfsys@defobject{currentmarker}{\pgfqpoint{-0.048611in}{0.000000in}}{\pgfqpoint{-0.000000in}{0.000000in}}{%
\pgfpathmoveto{\pgfqpoint{-0.000000in}{0.000000in}}%
\pgfpathlineto{\pgfqpoint{-0.048611in}{0.000000in}}%
\pgfusepath{stroke,fill}%
}%
\begin{pgfscope}%
\pgfsys@transformshift{0.854460in}{3.739376in}%
\pgfsys@useobject{currentmarker}{}%
\end{pgfscope}%
\end{pgfscope}%
\begin{pgfscope}%
\definecolor{textcolor}{rgb}{0.000000,0.000000,0.000000}%
\pgfsetstrokecolor{textcolor}%
\pgfsetfillcolor{textcolor}%
\pgftext[x=0.447993in, y=3.686614in, left, base]{\color{textcolor}\sffamily\fontsize{10.000000}{12.000000}\selectfont 0.25}%
\end{pgfscope}%
\begin{pgfscope}%
\pgfsetbuttcap%
\pgfsetroundjoin%
\definecolor{currentfill}{rgb}{0.000000,0.000000,0.000000}%
\pgfsetfillcolor{currentfill}%
\pgfsetlinewidth{0.803000pt}%
\definecolor{currentstroke}{rgb}{0.000000,0.000000,0.000000}%
\pgfsetstrokecolor{currentstroke}%
\pgfsetdash{}{0pt}%
\pgfsys@defobject{currentmarker}{\pgfqpoint{-0.048611in}{0.000000in}}{\pgfqpoint{-0.000000in}{0.000000in}}{%
\pgfpathmoveto{\pgfqpoint{-0.000000in}{0.000000in}}%
\pgfpathlineto{\pgfqpoint{-0.048611in}{0.000000in}}%
\pgfusepath{stroke,fill}%
}%
\begin{pgfscope}%
\pgfsys@transformshift{0.854460in}{4.372930in}%
\pgfsys@useobject{currentmarker}{}%
\end{pgfscope}%
\end{pgfscope}%
\begin{pgfscope}%
\definecolor{textcolor}{rgb}{0.000000,0.000000,0.000000}%
\pgfsetstrokecolor{textcolor}%
\pgfsetfillcolor{textcolor}%
\pgftext[x=0.447993in, y=4.320169in, left, base]{\color{textcolor}\sffamily\fontsize{10.000000}{12.000000}\selectfont 0.50}%
\end{pgfscope}%
\begin{pgfscope}%
\pgfsetbuttcap%
\pgfsetroundjoin%
\definecolor{currentfill}{rgb}{0.000000,0.000000,0.000000}%
\pgfsetfillcolor{currentfill}%
\pgfsetlinewidth{0.803000pt}%
\definecolor{currentstroke}{rgb}{0.000000,0.000000,0.000000}%
\pgfsetstrokecolor{currentstroke}%
\pgfsetdash{}{0pt}%
\pgfsys@defobject{currentmarker}{\pgfqpoint{-0.048611in}{0.000000in}}{\pgfqpoint{-0.000000in}{0.000000in}}{%
\pgfpathmoveto{\pgfqpoint{-0.000000in}{0.000000in}}%
\pgfpathlineto{\pgfqpoint{-0.048611in}{0.000000in}}%
\pgfusepath{stroke,fill}%
}%
\begin{pgfscope}%
\pgfsys@transformshift{0.854460in}{5.006485in}%
\pgfsys@useobject{currentmarker}{}%
\end{pgfscope}%
\end{pgfscope}%
\begin{pgfscope}%
\definecolor{textcolor}{rgb}{0.000000,0.000000,0.000000}%
\pgfsetstrokecolor{textcolor}%
\pgfsetfillcolor{textcolor}%
\pgftext[x=0.447993in, y=4.953723in, left, base]{\color{textcolor}\sffamily\fontsize{10.000000}{12.000000}\selectfont 0.75}%
\end{pgfscope}%
\begin{pgfscope}%
\pgfsetbuttcap%
\pgfsetroundjoin%
\definecolor{currentfill}{rgb}{0.000000,0.000000,0.000000}%
\pgfsetfillcolor{currentfill}%
\pgfsetlinewidth{0.803000pt}%
\definecolor{currentstroke}{rgb}{0.000000,0.000000,0.000000}%
\pgfsetstrokecolor{currentstroke}%
\pgfsetdash{}{0pt}%
\pgfsys@defobject{currentmarker}{\pgfqpoint{-0.048611in}{0.000000in}}{\pgfqpoint{-0.000000in}{0.000000in}}{%
\pgfpathmoveto{\pgfqpoint{-0.000000in}{0.000000in}}%
\pgfpathlineto{\pgfqpoint{-0.048611in}{0.000000in}}%
\pgfusepath{stroke,fill}%
}%
\begin{pgfscope}%
\pgfsys@transformshift{0.854460in}{5.640039in}%
\pgfsys@useobject{currentmarker}{}%
\end{pgfscope}%
\end{pgfscope}%
\begin{pgfscope}%
\definecolor{textcolor}{rgb}{0.000000,0.000000,0.000000}%
\pgfsetstrokecolor{textcolor}%
\pgfsetfillcolor{textcolor}%
\pgftext[x=0.447993in, y=5.587277in, left, base]{\color{textcolor}\sffamily\fontsize{10.000000}{12.000000}\selectfont 1.00}%
\end{pgfscope}%
\begin{pgfscope}%
\definecolor{textcolor}{rgb}{0.000000,0.000000,0.000000}%
\pgfsetstrokecolor{textcolor}%
\pgfsetfillcolor{textcolor}%
\pgftext[x=0.284413in,y=3.105821in,,bottom,rotate=90.000000]{\color{textcolor}\sffamily\fontsize{10.000000}{12.000000}\selectfont y}%
\end{pgfscope}%
\begin{pgfscope}%
\pgfpathrectangle{\pgfqpoint{0.854460in}{0.571603in}}{\pgfqpoint{5.885100in}{5.068436in}}%
\pgfusepath{clip}%
\pgfsetbuttcap%
\pgfsetroundjoin%
\pgfsetlinewidth{1.505625pt}%
\definecolor{currentstroke}{rgb}{0.273809,0.031497,0.358853}%
\pgfsetstrokecolor{currentstroke}%
\pgfsetdash{}{0pt}%
\pgfpathmoveto{\pgfqpoint{4.137104in}{0.594156in}}%
\pgfpathlineto{\pgfqpoint{4.107531in}{0.600389in}}%
\pgfpathlineto{\pgfqpoint{4.048384in}{0.615614in}}%
\pgfpathlineto{\pgfqpoint{4.018811in}{0.624181in}}%
\pgfpathlineto{\pgfqpoint{3.948066in}{0.648012in}}%
\pgfpathlineto{\pgfqpoint{3.881868in}{0.673481in}}%
\pgfpathlineto{\pgfqpoint{3.822205in}{0.698951in}}%
\pgfpathlineto{\pgfqpoint{3.767335in}{0.724420in}}%
\pgfpathlineto{\pgfqpoint{3.716076in}{0.749890in}}%
\pgfpathlineto{\pgfqpoint{3.634357in}{0.794152in}}%
\pgfpathlineto{\pgfqpoint{3.575210in}{0.828820in}}%
\pgfpathlineto{\pgfqpoint{3.499258in}{0.877238in}}%
\pgfpathlineto{\pgfqpoint{3.456917in}{0.906046in}}%
\pgfpathlineto{\pgfqpoint{3.391535in}{0.953646in}}%
\pgfpathlineto{\pgfqpoint{3.327111in}{1.004585in}}%
\pgfpathlineto{\pgfqpoint{3.267775in}{1.055524in}}%
\pgfpathlineto{\pgfqpoint{3.213147in}{1.106463in}}%
\pgfpathlineto{\pgfqpoint{3.161183in}{1.159266in}}%
\pgfpathlineto{\pgfqpoint{3.117242in}{1.208341in}}%
\pgfpathlineto{\pgfqpoint{3.075310in}{1.259281in}}%
\pgfpathlineto{\pgfqpoint{3.042890in}{1.302771in}}%
\pgfpathlineto{\pgfqpoint{3.020151in}{1.335689in}}%
\pgfpathlineto{\pgfqpoint{2.988135in}{1.386628in}}%
\pgfpathlineto{\pgfqpoint{2.973705in}{1.412098in}}%
\pgfpathlineto{\pgfqpoint{2.947533in}{1.463037in}}%
\pgfpathlineto{\pgfqpoint{2.924596in}{1.515674in}}%
\pgfpathlineto{\pgfqpoint{2.907476in}{1.564915in}}%
\pgfpathlineto{\pgfqpoint{2.893639in}{1.615854in}}%
\pgfpathlineto{\pgfqpoint{2.888732in}{1.641323in}}%
\pgfpathlineto{\pgfqpoint{2.885009in}{1.666793in}}%
\pgfpathlineto{\pgfqpoint{2.882585in}{1.692262in}}%
\pgfpathlineto{\pgfqpoint{2.881588in}{1.717732in}}%
\pgfpathlineto{\pgfqpoint{2.882167in}{1.743202in}}%
\pgfpathlineto{\pgfqpoint{2.884491in}{1.768671in}}%
\pgfpathlineto{\pgfqpoint{2.888754in}{1.794141in}}%
\pgfpathlineto{\pgfqpoint{2.895198in}{1.819610in}}%
\pgfpathlineto{\pgfqpoint{2.905163in}{1.845080in}}%
\pgfpathlineto{\pgfqpoint{2.924596in}{1.880337in}}%
\pgfpathlineto{\pgfqpoint{2.936985in}{1.896019in}}%
\pgfpathlineto{\pgfqpoint{2.954169in}{1.913874in}}%
\pgfpathlineto{\pgfqpoint{2.963745in}{1.921488in}}%
\pgfpathlineto{\pgfqpoint{2.983743in}{1.935053in}}%
\pgfpathlineto{\pgfqpoint{3.013316in}{1.948767in}}%
\pgfpathlineto{\pgfqpoint{3.042890in}{1.956682in}}%
\pgfpathlineto{\pgfqpoint{3.072463in}{1.960424in}}%
\pgfpathlineto{\pgfqpoint{3.102036in}{1.960653in}}%
\pgfpathlineto{\pgfqpoint{3.131610in}{1.957897in}}%
\pgfpathlineto{\pgfqpoint{3.161183in}{1.952588in}}%
\pgfpathlineto{\pgfqpoint{3.190756in}{1.945003in}}%
\pgfpathlineto{\pgfqpoint{3.220330in}{1.935236in}}%
\pgfpathlineto{\pgfqpoint{3.255016in}{1.921488in}}%
\pgfpathlineto{\pgfqpoint{3.279476in}{1.910526in}}%
\pgfpathlineto{\pgfqpoint{3.309050in}{1.895913in}}%
\pgfpathlineto{\pgfqpoint{3.368197in}{1.862272in}}%
\pgfpathlineto{\pgfqpoint{3.427343in}{1.823984in}}%
\pgfpathlineto{\pgfqpoint{3.486490in}{1.781515in}}%
\pgfpathlineto{\pgfqpoint{3.545637in}{1.735571in}}%
\pgfpathlineto{\pgfqpoint{3.604783in}{1.686548in}}%
\pgfpathlineto{\pgfqpoint{3.684793in}{1.615854in}}%
\pgfpathlineto{\pgfqpoint{3.739690in}{1.564915in}}%
\pgfpathlineto{\pgfqpoint{3.792774in}{1.513976in}}%
\pgfpathlineto{\pgfqpoint{3.870944in}{1.436234in}}%
\pgfpathlineto{\pgfqpoint{3.967054in}{1.335689in}}%
\pgfpathlineto{\pgfqpoint{4.036956in}{1.259281in}}%
\pgfpathlineto{\pgfqpoint{4.107531in}{1.178977in}}%
\pgfpathlineto{\pgfqpoint{4.168246in}{1.106463in}}%
\pgfpathlineto{\pgfqpoint{4.228536in}{1.030055in}}%
\pgfpathlineto{\pgfqpoint{4.284278in}{0.953646in}}%
\pgfpathlineto{\pgfqpoint{4.318017in}{0.902707in}}%
\pgfpathlineto{\pgfqpoint{4.348371in}{0.851768in}}%
\pgfpathlineto{\pgfqpoint{4.374350in}{0.800829in}}%
\pgfpathlineto{\pgfqpoint{4.384843in}{0.775360in}}%
\pgfpathlineto{\pgfqpoint{4.393739in}{0.749890in}}%
\pgfpathlineto{\pgfqpoint{4.400495in}{0.724420in}}%
\pgfpathlineto{\pgfqpoint{4.404221in}{0.698951in}}%
\pgfpathlineto{\pgfqpoint{4.403264in}{0.670073in}}%
\pgfpathlineto{\pgfqpoint{4.397683in}{0.648012in}}%
\pgfpathlineto{\pgfqpoint{4.382086in}{0.622542in}}%
\pgfpathlineto{\pgfqpoint{4.373691in}{0.614624in}}%
\pgfpathlineto{\pgfqpoint{4.344118in}{0.596246in}}%
\pgfpathlineto{\pgfqpoint{4.314544in}{0.587674in}}%
\pgfpathlineto{\pgfqpoint{4.284971in}{0.583249in}}%
\pgfpathlineto{\pgfqpoint{4.255398in}{0.581830in}}%
\pgfpathlineto{\pgfqpoint{4.225824in}{0.582661in}}%
\pgfpathlineto{\pgfqpoint{4.196251in}{0.585223in}}%
\pgfpathlineto{\pgfqpoint{4.137104in}{0.594156in}}%
\pgfpathlineto{\pgfqpoint{4.137104in}{0.594156in}}%
\pgfusepath{stroke}%
\end{pgfscope}%
\begin{pgfscope}%
\pgfpathrectangle{\pgfqpoint{0.854460in}{0.571603in}}{\pgfqpoint{5.885100in}{5.068436in}}%
\pgfusepath{clip}%
\pgfsetbuttcap%
\pgfsetroundjoin%
\pgfsetlinewidth{1.505625pt}%
\definecolor{currentstroke}{rgb}{0.278791,0.062145,0.386592}%
\pgfsetstrokecolor{currentstroke}%
\pgfsetdash{}{0pt}%
\pgfpathmoveto{\pgfqpoint{3.596928in}{0.571603in}}%
\pgfpathlineto{\pgfqpoint{3.507962in}{0.622542in}}%
\pgfpathlineto{\pgfqpoint{3.424344in}{0.673481in}}%
\pgfpathlineto{\pgfqpoint{3.338623in}{0.729256in}}%
\pgfpathlineto{\pgfqpoint{3.271786in}{0.775360in}}%
\pgfpathlineto{\pgfqpoint{3.190756in}{0.834882in}}%
\pgfpathlineto{\pgfqpoint{3.131610in}{0.880969in}}%
\pgfpathlineto{\pgfqpoint{3.072463in}{0.929707in}}%
\pgfpathlineto{\pgfqpoint{3.013316in}{0.981490in}}%
\pgfpathlineto{\pgfqpoint{2.954169in}{1.036752in}}%
\pgfpathlineto{\pgfqpoint{2.895023in}{1.095981in}}%
\pgfpathlineto{\pgfqpoint{2.861102in}{1.131933in}}%
\pgfpathlineto{\pgfqpoint{2.806303in}{1.194084in}}%
\pgfpathlineto{\pgfqpoint{2.773448in}{1.233811in}}%
\pgfpathlineto{\pgfqpoint{2.734109in}{1.284750in}}%
\pgfpathlineto{\pgfqpoint{2.697497in}{1.335689in}}%
\pgfpathlineto{\pgfqpoint{2.663625in}{1.386628in}}%
\pgfpathlineto{\pgfqpoint{2.628862in}{1.443863in}}%
\pgfpathlineto{\pgfqpoint{2.599289in}{1.497617in}}%
\pgfpathlineto{\pgfqpoint{2.578304in}{1.539445in}}%
\pgfpathlineto{\pgfqpoint{2.555142in}{1.590384in}}%
\pgfpathlineto{\pgfqpoint{2.534494in}{1.641323in}}%
\pgfpathlineto{\pgfqpoint{2.516507in}{1.692262in}}%
\pgfpathlineto{\pgfqpoint{2.501163in}{1.743202in}}%
\pgfpathlineto{\pgfqpoint{2.488396in}{1.794141in}}%
\pgfpathlineto{\pgfqpoint{2.478251in}{1.845080in}}%
\pgfpathlineto{\pgfqpoint{2.471010in}{1.896019in}}%
\pgfpathlineto{\pgfqpoint{2.466512in}{1.946958in}}%
\pgfpathlineto{\pgfqpoint{2.464931in}{1.997897in}}%
\pgfpathlineto{\pgfqpoint{2.466447in}{2.048836in}}%
\pgfpathlineto{\pgfqpoint{2.471242in}{2.099775in}}%
\pgfpathlineto{\pgfqpoint{2.479501in}{2.150714in}}%
\pgfpathlineto{\pgfqpoint{2.485234in}{2.176183in}}%
\pgfpathlineto{\pgfqpoint{2.499924in}{2.227123in}}%
\pgfpathlineto{\pgfqpoint{2.510569in}{2.257053in}}%
\pgfpathlineto{\pgfqpoint{2.531363in}{2.303531in}}%
\pgfpathlineto{\pgfqpoint{2.544884in}{2.329001in}}%
\pgfpathlineto{\pgfqpoint{2.569716in}{2.368332in}}%
\pgfpathlineto{\pgfqpoint{2.578291in}{2.379940in}}%
\pgfpathlineto{\pgfqpoint{2.599289in}{2.406256in}}%
\pgfpathlineto{\pgfqpoint{2.628862in}{2.436872in}}%
\pgfpathlineto{\pgfqpoint{2.658436in}{2.462009in}}%
\pgfpathlineto{\pgfqpoint{2.688009in}{2.482626in}}%
\pgfpathlineto{\pgfqpoint{2.717582in}{2.499232in}}%
\pgfpathlineto{\pgfqpoint{2.747156in}{2.512497in}}%
\pgfpathlineto{\pgfqpoint{2.776729in}{2.522657in}}%
\pgfpathlineto{\pgfqpoint{2.806303in}{2.530082in}}%
\pgfpathlineto{\pgfqpoint{2.835876in}{2.534897in}}%
\pgfpathlineto{\pgfqpoint{2.865449in}{2.537291in}}%
\pgfpathlineto{\pgfqpoint{2.895023in}{2.537447in}}%
\pgfpathlineto{\pgfqpoint{2.924596in}{2.535485in}}%
\pgfpathlineto{\pgfqpoint{2.954169in}{2.531502in}}%
\pgfpathlineto{\pgfqpoint{2.983743in}{2.525562in}}%
\pgfpathlineto{\pgfqpoint{3.013316in}{2.517813in}}%
\pgfpathlineto{\pgfqpoint{3.045689in}{2.507287in}}%
\pgfpathlineto{\pgfqpoint{3.072463in}{2.497161in}}%
\pgfpathlineto{\pgfqpoint{3.107404in}{2.481818in}}%
\pgfpathlineto{\pgfqpoint{3.131610in}{2.470132in}}%
\pgfpathlineto{\pgfqpoint{3.161183in}{2.454434in}}%
\pgfpathlineto{\pgfqpoint{3.201057in}{2.430879in}}%
\pgfpathlineto{\pgfqpoint{3.249903in}{2.399300in}}%
\pgfpathlineto{\pgfqpoint{3.311829in}{2.354470in}}%
\pgfpathlineto{\pgfqpoint{3.368197in}{2.309881in}}%
\pgfpathlineto{\pgfqpoint{3.435131in}{2.252592in}}%
\pgfpathlineto{\pgfqpoint{3.491374in}{2.201653in}}%
\pgfpathlineto{\pgfqpoint{3.545637in}{2.150413in}}%
\pgfpathlineto{\pgfqpoint{3.634357in}{2.062754in}}%
\pgfpathlineto{\pgfqpoint{3.752650in}{1.940697in}}%
\pgfpathlineto{\pgfqpoint{3.930090in}{1.750905in}}%
\pgfpathlineto{\pgfqpoint{4.354678in}{1.284750in}}%
\pgfpathlineto{\pgfqpoint{4.491984in}{1.131763in}}%
\pgfpathlineto{\pgfqpoint{4.648299in}{0.953646in}}%
\pgfpathlineto{\pgfqpoint{4.734868in}{0.851768in}}%
\pgfpathlineto{\pgfqpoint{4.797548in}{0.775360in}}%
\pgfpathlineto{\pgfqpoint{4.857619in}{0.698951in}}%
\pgfpathlineto{\pgfqpoint{4.906012in}{0.633775in}}%
\pgfpathlineto{\pgfqpoint{4.935585in}{0.591734in}}%
\pgfpathlineto{\pgfqpoint{4.949135in}{0.571603in}}%
\pgfpathlineto{\pgfqpoint{4.949135in}{0.571603in}}%
\pgfusepath{stroke}%
\end{pgfscope}%
\begin{pgfscope}%
\pgfpathrectangle{\pgfqpoint{0.854460in}{0.571603in}}{\pgfqpoint{5.885100in}{5.068436in}}%
\pgfusepath{clip}%
\pgfsetbuttcap%
\pgfsetroundjoin%
\pgfsetlinewidth{1.505625pt}%
\definecolor{currentstroke}{rgb}{0.282327,0.094955,0.417331}%
\pgfsetstrokecolor{currentstroke}%
\pgfsetdash{}{0pt}%
\pgfpathmoveto{\pgfqpoint{3.320897in}{0.571603in}}%
\pgfpathlineto{\pgfqpoint{3.241386in}{0.622542in}}%
\pgfpathlineto{\pgfqpoint{3.161183in}{0.676713in}}%
\pgfpathlineto{\pgfqpoint{3.094084in}{0.724420in}}%
\pgfpathlineto{\pgfqpoint{3.013316in}{0.785118in}}%
\pgfpathlineto{\pgfqpoint{2.954169in}{0.831977in}}%
\pgfpathlineto{\pgfqpoint{2.895023in}{0.881216in}}%
\pgfpathlineto{\pgfqpoint{2.835876in}{0.933148in}}%
\pgfpathlineto{\pgfqpoint{2.776729in}{0.988119in}}%
\pgfpathlineto{\pgfqpoint{2.717582in}{1.046510in}}%
\pgfpathlineto{\pgfqpoint{2.684320in}{1.080994in}}%
\pgfpathlineto{\pgfqpoint{2.628862in}{1.141900in}}%
\pgfpathlineto{\pgfqpoint{2.593597in}{1.182872in}}%
\pgfpathlineto{\pgfqpoint{2.552204in}{1.233811in}}%
\pgfpathlineto{\pgfqpoint{2.510569in}{1.288383in}}%
\pgfpathlineto{\pgfqpoint{2.476815in}{1.335689in}}%
\pgfpathlineto{\pgfqpoint{2.442837in}{1.386628in}}%
\pgfpathlineto{\pgfqpoint{2.411190in}{1.437567in}}%
\pgfpathlineto{\pgfqpoint{2.381843in}{1.488506in}}%
\pgfpathlineto{\pgfqpoint{2.354752in}{1.539445in}}%
\pgfpathlineto{\pgfqpoint{2.329861in}{1.590384in}}%
\pgfpathlineto{\pgfqpoint{2.303555in}{1.650142in}}%
\pgfpathlineto{\pgfqpoint{2.286822in}{1.692262in}}%
\pgfpathlineto{\pgfqpoint{2.268455in}{1.743202in}}%
\pgfpathlineto{\pgfqpoint{2.252290in}{1.794141in}}%
\pgfpathlineto{\pgfqpoint{2.238208in}{1.845080in}}%
\pgfpathlineto{\pgfqpoint{2.226295in}{1.896019in}}%
\pgfpathlineto{\pgfqpoint{2.214835in}{1.956276in}}%
\pgfpathlineto{\pgfqpoint{2.208712in}{1.997897in}}%
\pgfpathlineto{\pgfqpoint{2.203184in}{2.048836in}}%
\pgfpathlineto{\pgfqpoint{2.199788in}{2.099775in}}%
\pgfpathlineto{\pgfqpoint{2.198587in}{2.150714in}}%
\pgfpathlineto{\pgfqpoint{2.199638in}{2.201653in}}%
\pgfpathlineto{\pgfqpoint{2.202997in}{2.252592in}}%
\pgfpathlineto{\pgfqpoint{2.208708in}{2.303531in}}%
\pgfpathlineto{\pgfqpoint{2.216899in}{2.354470in}}%
\pgfpathlineto{\pgfqpoint{2.227905in}{2.405409in}}%
\pgfpathlineto{\pgfqpoint{2.244409in}{2.466027in}}%
\pgfpathlineto{\pgfqpoint{2.258291in}{2.507287in}}%
\pgfpathlineto{\pgfqpoint{2.278191in}{2.558226in}}%
\pgfpathlineto{\pgfqpoint{2.303555in}{2.612650in}}%
\pgfpathlineto{\pgfqpoint{2.333129in}{2.665897in}}%
\pgfpathlineto{\pgfqpoint{2.362702in}{2.711442in}}%
\pgfpathlineto{\pgfqpoint{2.392275in}{2.750860in}}%
\pgfpathlineto{\pgfqpoint{2.423604in}{2.787452in}}%
\pgfpathlineto{\pgfqpoint{2.451422in}{2.816138in}}%
\pgfpathlineto{\pgfqpoint{2.480996in}{2.843164in}}%
\pgfpathlineto{\pgfqpoint{2.510569in}{2.867070in}}%
\pgfpathlineto{\pgfqpoint{2.542015in}{2.889330in}}%
\pgfpathlineto{\pgfqpoint{2.584806in}{2.914800in}}%
\pgfpathlineto{\pgfqpoint{2.599289in}{2.922556in}}%
\pgfpathlineto{\pgfqpoint{2.638922in}{2.940269in}}%
\pgfpathlineto{\pgfqpoint{2.658436in}{2.947832in}}%
\pgfpathlineto{\pgfqpoint{2.688009in}{2.957297in}}%
\pgfpathlineto{\pgfqpoint{2.722650in}{2.965739in}}%
\pgfpathlineto{\pgfqpoint{2.747156in}{2.970275in}}%
\pgfpathlineto{\pgfqpoint{2.776729in}{2.973870in}}%
\pgfpathlineto{\pgfqpoint{2.806303in}{2.975610in}}%
\pgfpathlineto{\pgfqpoint{2.835876in}{2.975522in}}%
\pgfpathlineto{\pgfqpoint{2.865449in}{2.973632in}}%
\pgfpathlineto{\pgfqpoint{2.895023in}{2.969967in}}%
\pgfpathlineto{\pgfqpoint{2.924596in}{2.964542in}}%
\pgfpathlineto{\pgfqpoint{2.954169in}{2.957349in}}%
\pgfpathlineto{\pgfqpoint{2.983743in}{2.948454in}}%
\pgfpathlineto{\pgfqpoint{3.013316in}{2.937870in}}%
\pgfpathlineto{\pgfqpoint{3.042890in}{2.925587in}}%
\pgfpathlineto{\pgfqpoint{3.072463in}{2.911675in}}%
\pgfpathlineto{\pgfqpoint{3.113769in}{2.889330in}}%
\pgfpathlineto{\pgfqpoint{3.155599in}{2.863861in}}%
\pgfpathlineto{\pgfqpoint{3.161183in}{2.860344in}}%
\pgfpathlineto{\pgfqpoint{3.193200in}{2.838391in}}%
\pgfpathlineto{\pgfqpoint{3.227570in}{2.812922in}}%
\pgfpathlineto{\pgfqpoint{3.279476in}{2.771264in}}%
\pgfpathlineto{\pgfqpoint{3.338623in}{2.718945in}}%
\pgfpathlineto{\pgfqpoint{3.399841in}{2.660104in}}%
\pgfpathlineto{\pgfqpoint{3.456917in}{2.601708in}}%
\pgfpathlineto{\pgfqpoint{3.545637in}{2.505300in}}%
\pgfpathlineto{\pgfqpoint{3.634357in}{2.403961in}}%
\pgfpathlineto{\pgfqpoint{3.783692in}{2.227123in}}%
\pgfpathlineto{\pgfqpoint{4.060099in}{1.896019in}}%
\pgfpathlineto{\pgfqpoint{4.255874in}{1.666793in}}%
\pgfpathlineto{\pgfqpoint{4.389314in}{1.513976in}}%
\pgfpathlineto{\pgfqpoint{4.580705in}{1.299774in}}%
\pgfpathlineto{\pgfqpoint{4.733419in}{1.131933in}}%
\pgfpathlineto{\pgfqpoint{5.113025in}{0.716429in}}%
\pgfpathlineto{\pgfqpoint{5.218518in}{0.597073in}}%
\pgfpathlineto{\pgfqpoint{5.240530in}{0.571603in}}%
\pgfpathlineto{\pgfqpoint{5.240530in}{0.571603in}}%
\pgfusepath{stroke}%
\end{pgfscope}%
\begin{pgfscope}%
\pgfpathrectangle{\pgfqpoint{0.854460in}{0.571603in}}{\pgfqpoint{5.885100in}{5.068436in}}%
\pgfusepath{clip}%
\pgfsetbuttcap%
\pgfsetroundjoin%
\pgfsetlinewidth{1.505625pt}%
\definecolor{currentstroke}{rgb}{0.283229,0.120777,0.440584}%
\pgfsetstrokecolor{currentstroke}%
\pgfsetdash{}{0pt}%
\pgfpathmoveto{\pgfqpoint{3.111874in}{0.571603in}}%
\pgfpathlineto{\pgfqpoint{3.036974in}{0.622542in}}%
\pgfpathlineto{\pgfqpoint{2.954169in}{0.681864in}}%
\pgfpathlineto{\pgfqpoint{2.895023in}{0.726197in}}%
\pgfpathlineto{\pgfqpoint{2.832351in}{0.775360in}}%
\pgfpathlineto{\pgfqpoint{2.770439in}{0.826299in}}%
\pgfpathlineto{\pgfqpoint{2.711451in}{0.877238in}}%
\pgfpathlineto{\pgfqpoint{2.655289in}{0.928177in}}%
\pgfpathlineto{\pgfqpoint{2.599289in}{0.981703in}}%
\pgfpathlineto{\pgfqpoint{2.540142in}{1.041645in}}%
\pgfpathlineto{\pgfqpoint{2.503165in}{1.080994in}}%
\pgfpathlineto{\pgfqpoint{2.451422in}{1.139073in}}%
\pgfpathlineto{\pgfqpoint{2.414477in}{1.182872in}}%
\pgfpathlineto{\pgfqpoint{2.362702in}{1.248197in}}%
\pgfpathlineto{\pgfqpoint{2.333129in}{1.287748in}}%
\pgfpathlineto{\pgfqpoint{2.299196in}{1.335689in}}%
\pgfpathlineto{\pgfqpoint{2.265319in}{1.386628in}}%
\pgfpathlineto{\pgfqpoint{2.233586in}{1.437567in}}%
\pgfpathlineto{\pgfqpoint{2.203964in}{1.488506in}}%
\pgfpathlineto{\pgfqpoint{2.176410in}{1.539445in}}%
\pgfpathlineto{\pgfqpoint{2.150874in}{1.590384in}}%
\pgfpathlineto{\pgfqpoint{2.126115in}{1.644110in}}%
\pgfpathlineto{\pgfqpoint{2.095749in}{1.717732in}}%
\pgfpathlineto{\pgfqpoint{2.068520in}{1.794141in}}%
\pgfpathlineto{\pgfqpoint{2.052817in}{1.845080in}}%
\pgfpathlineto{\pgfqpoint{2.037395in}{1.901705in}}%
\pgfpathlineto{\pgfqpoint{2.026824in}{1.946958in}}%
\pgfpathlineto{\pgfqpoint{2.016594in}{1.997897in}}%
\pgfpathlineto{\pgfqpoint{2.007822in}{2.050970in}}%
\pgfpathlineto{\pgfqpoint{2.001639in}{2.099775in}}%
\pgfpathlineto{\pgfqpoint{1.996955in}{2.150714in}}%
\pgfpathlineto{\pgfqpoint{1.994088in}{2.201653in}}%
\pgfpathlineto{\pgfqpoint{1.993068in}{2.252592in}}%
\pgfpathlineto{\pgfqpoint{1.993917in}{2.303531in}}%
\pgfpathlineto{\pgfqpoint{1.996657in}{2.354470in}}%
\pgfpathlineto{\pgfqpoint{2.001300in}{2.405409in}}%
\pgfpathlineto{\pgfqpoint{2.007856in}{2.456348in}}%
\pgfpathlineto{\pgfqpoint{2.016639in}{2.507287in}}%
\pgfpathlineto{\pgfqpoint{2.027408in}{2.558226in}}%
\pgfpathlineto{\pgfqpoint{2.040259in}{2.609165in}}%
\pgfpathlineto{\pgfqpoint{2.055541in}{2.660104in}}%
\pgfpathlineto{\pgfqpoint{2.073079in}{2.711044in}}%
\pgfpathlineto{\pgfqpoint{2.096542in}{2.769995in}}%
\pgfpathlineto{\pgfqpoint{2.116068in}{2.812922in}}%
\pgfpathlineto{\pgfqpoint{2.141871in}{2.863861in}}%
\pgfpathlineto{\pgfqpoint{2.170909in}{2.914800in}}%
\pgfpathlineto{\pgfqpoint{2.186612in}{2.940269in}}%
\pgfpathlineto{\pgfqpoint{2.221387in}{2.991208in}}%
\pgfpathlineto{\pgfqpoint{2.260633in}{3.042147in}}%
\pgfpathlineto{\pgfqpoint{2.282138in}{3.067617in}}%
\pgfpathlineto{\pgfqpoint{2.305035in}{3.093086in}}%
\pgfpathlineto{\pgfqpoint{2.333129in}{3.121925in}}%
\pgfpathlineto{\pgfqpoint{2.362702in}{3.149813in}}%
\pgfpathlineto{\pgfqpoint{2.392275in}{3.175419in}}%
\pgfpathlineto{\pgfqpoint{2.421849in}{3.198907in}}%
\pgfpathlineto{\pgfqpoint{2.451447in}{3.220434in}}%
\pgfpathlineto{\pgfqpoint{2.490766in}{3.245904in}}%
\pgfpathlineto{\pgfqpoint{2.510569in}{3.257762in}}%
\pgfpathlineto{\pgfqpoint{2.540142in}{3.273866in}}%
\pgfpathlineto{\pgfqpoint{2.589522in}{3.296843in}}%
\pgfpathlineto{\pgfqpoint{2.599289in}{3.301038in}}%
\pgfpathlineto{\pgfqpoint{2.628862in}{3.312208in}}%
\pgfpathlineto{\pgfqpoint{2.660050in}{3.322312in}}%
\pgfpathlineto{\pgfqpoint{2.688009in}{3.329912in}}%
\pgfpathlineto{\pgfqpoint{2.717582in}{3.336443in}}%
\pgfpathlineto{\pgfqpoint{2.747156in}{3.341456in}}%
\pgfpathlineto{\pgfqpoint{2.776729in}{3.344936in}}%
\pgfpathlineto{\pgfqpoint{2.806303in}{3.346868in}}%
\pgfpathlineto{\pgfqpoint{2.835876in}{3.347237in}}%
\pgfpathlineto{\pgfqpoint{2.865449in}{3.346028in}}%
\pgfpathlineto{\pgfqpoint{2.895023in}{3.343225in}}%
\pgfpathlineto{\pgfqpoint{2.924596in}{3.338811in}}%
\pgfpathlineto{\pgfqpoint{2.954169in}{3.332769in}}%
\pgfpathlineto{\pgfqpoint{2.992540in}{3.322312in}}%
\pgfpathlineto{\pgfqpoint{3.013316in}{3.315686in}}%
\pgfpathlineto{\pgfqpoint{3.060808in}{3.296843in}}%
\pgfpathlineto{\pgfqpoint{3.072463in}{3.291753in}}%
\pgfpathlineto{\pgfqpoint{3.112557in}{3.271373in}}%
\pgfpathlineto{\pgfqpoint{3.131610in}{3.260798in}}%
\pgfpathlineto{\pgfqpoint{3.161183in}{3.242665in}}%
\pgfpathlineto{\pgfqpoint{3.193920in}{3.220434in}}%
\pgfpathlineto{\pgfqpoint{3.228022in}{3.194965in}}%
\pgfpathlineto{\pgfqpoint{3.259437in}{3.169495in}}%
\pgfpathlineto{\pgfqpoint{3.288731in}{3.144025in}}%
\pgfpathlineto{\pgfqpoint{3.316328in}{3.118556in}}%
\pgfpathlineto{\pgfqpoint{3.367628in}{3.067617in}}%
\pgfpathlineto{\pgfqpoint{3.397770in}{3.035531in}}%
\pgfpathlineto{\pgfqpoint{3.459181in}{2.965739in}}%
\pgfpathlineto{\pgfqpoint{3.521898in}{2.889330in}}%
\pgfpathlineto{\pgfqpoint{3.581555in}{2.812922in}}%
\pgfpathlineto{\pgfqpoint{3.663930in}{2.703320in}}%
\pgfpathlineto{\pgfqpoint{4.018811in}{2.218676in}}%
\pgfpathlineto{\pgfqpoint{4.089233in}{2.125244in}}%
\pgfpathlineto{\pgfqpoint{4.207448in}{1.972427in}}%
\pgfpathlineto{\pgfqpoint{4.329657in}{1.819610in}}%
\pgfpathlineto{\pgfqpoint{4.456168in}{1.666793in}}%
\pgfpathlineto{\pgfqpoint{4.586978in}{1.513976in}}%
\pgfpathlineto{\pgfqpoint{4.721985in}{1.361159in}}%
\pgfpathlineto{\pgfqpoint{4.860855in}{1.208341in}}%
\pgfpathlineto{\pgfqpoint{5.024305in}{1.033084in}}%
\pgfpathlineto{\pgfqpoint{5.172387in}{0.877238in}}%
\pgfpathlineto{\pgfqpoint{5.464941in}{0.571603in}}%
\pgfpathlineto{\pgfqpoint{5.464941in}{0.571603in}}%
\pgfusepath{stroke}%
\end{pgfscope}%
\begin{pgfscope}%
\pgfpathrectangle{\pgfqpoint{0.854460in}{0.571603in}}{\pgfqpoint{5.885100in}{5.068436in}}%
\pgfusepath{clip}%
\pgfsetbuttcap%
\pgfsetroundjoin%
\pgfsetlinewidth{1.505625pt}%
\definecolor{currentstroke}{rgb}{0.281887,0.150881,0.465405}%
\pgfsetstrokecolor{currentstroke}%
\pgfsetdash{}{0pt}%
\pgfpathmoveto{\pgfqpoint{2.937449in}{0.571603in}}%
\pgfpathlineto{\pgfqpoint{2.865449in}{0.622567in}}%
\pgfpathlineto{\pgfqpoint{2.776729in}{0.688851in}}%
\pgfpathlineto{\pgfqpoint{2.717582in}{0.735223in}}%
\pgfpathlineto{\pgfqpoint{2.658436in}{0.783590in}}%
\pgfpathlineto{\pgfqpoint{2.599289in}{0.834187in}}%
\pgfpathlineto{\pgfqpoint{2.540142in}{0.887274in}}%
\pgfpathlineto{\pgfqpoint{2.480996in}{0.943131in}}%
\pgfpathlineto{\pgfqpoint{2.421849in}{1.002067in}}%
\pgfpathlineto{\pgfqpoint{2.392275in}{1.032902in}}%
\pgfpathlineto{\pgfqpoint{2.333129in}{1.097625in}}%
\pgfpathlineto{\pgfqpoint{2.303200in}{1.131933in}}%
\pgfpathlineto{\pgfqpoint{2.244409in}{1.203456in}}%
\pgfpathlineto{\pgfqpoint{2.214835in}{1.241616in}}%
\pgfpathlineto{\pgfqpoint{2.182802in}{1.284750in}}%
\pgfpathlineto{\pgfqpoint{2.147058in}{1.335689in}}%
\pgfpathlineto{\pgfqpoint{2.113369in}{1.386628in}}%
\pgfpathlineto{\pgfqpoint{2.081708in}{1.437567in}}%
\pgfpathlineto{\pgfqpoint{2.052039in}{1.488506in}}%
\pgfpathlineto{\pgfqpoint{2.024323in}{1.539445in}}%
\pgfpathlineto{\pgfqpoint{1.998509in}{1.590384in}}%
\pgfpathlineto{\pgfqpoint{1.974542in}{1.641323in}}%
\pgfpathlineto{\pgfqpoint{1.948675in}{1.701494in}}%
\pgfpathlineto{\pgfqpoint{1.922716in}{1.768671in}}%
\pgfpathlineto{\pgfqpoint{1.905159in}{1.819610in}}%
\pgfpathlineto{\pgfqpoint{1.889196in}{1.870549in}}%
\pgfpathlineto{\pgfqpoint{1.868691in}{1.946958in}}%
\pgfpathlineto{\pgfqpoint{1.857039in}{1.997897in}}%
\pgfpathlineto{\pgfqpoint{1.847176in}{2.048836in}}%
\pgfpathlineto{\pgfqpoint{1.838897in}{2.099775in}}%
\pgfpathlineto{\pgfqpoint{1.830381in}{2.168032in}}%
\pgfpathlineto{\pgfqpoint{1.827285in}{2.201653in}}%
\pgfpathlineto{\pgfqpoint{1.824006in}{2.252592in}}%
\pgfpathlineto{\pgfqpoint{1.822355in}{2.303531in}}%
\pgfpathlineto{\pgfqpoint{1.822344in}{2.354470in}}%
\pgfpathlineto{\pgfqpoint{1.823983in}{2.405409in}}%
\pgfpathlineto{\pgfqpoint{1.827276in}{2.456348in}}%
\pgfpathlineto{\pgfqpoint{1.832281in}{2.507287in}}%
\pgfpathlineto{\pgfqpoint{1.839089in}{2.558226in}}%
\pgfpathlineto{\pgfqpoint{1.847595in}{2.609165in}}%
\pgfpathlineto{\pgfqpoint{1.859955in}{2.669787in}}%
\pgfpathlineto{\pgfqpoint{1.869968in}{2.711044in}}%
\pgfpathlineto{\pgfqpoint{1.889528in}{2.780754in}}%
\pgfpathlineto{\pgfqpoint{1.899907in}{2.812922in}}%
\pgfpathlineto{\pgfqpoint{1.919102in}{2.867381in}}%
\pgfpathlineto{\pgfqpoint{1.937976in}{2.914800in}}%
\pgfpathlineto{\pgfqpoint{1.960306in}{2.965739in}}%
\pgfpathlineto{\pgfqpoint{1.984961in}{3.016678in}}%
\pgfpathlineto{\pgfqpoint{2.012147in}{3.067617in}}%
\pgfpathlineto{\pgfqpoint{2.042089in}{3.118556in}}%
\pgfpathlineto{\pgfqpoint{2.075035in}{3.169495in}}%
\pgfpathlineto{\pgfqpoint{2.111252in}{3.220434in}}%
\pgfpathlineto{\pgfqpoint{2.130609in}{3.245904in}}%
\pgfpathlineto{\pgfqpoint{2.172569in}{3.296843in}}%
\pgfpathlineto{\pgfqpoint{2.195145in}{3.322312in}}%
\pgfpathlineto{\pgfqpoint{2.218884in}{3.347782in}}%
\pgfpathlineto{\pgfqpoint{2.244409in}{3.373699in}}%
\pgfpathlineto{\pgfqpoint{2.298991in}{3.424190in}}%
\pgfpathlineto{\pgfqpoint{2.333129in}{3.452949in}}%
\pgfpathlineto{\pgfqpoint{2.362702in}{3.476123in}}%
\pgfpathlineto{\pgfqpoint{2.396372in}{3.500599in}}%
\pgfpathlineto{\pgfqpoint{2.451422in}{3.536843in}}%
\pgfpathlineto{\pgfqpoint{2.480996in}{3.554408in}}%
\pgfpathlineto{\pgfqpoint{2.540142in}{3.585598in}}%
\pgfpathlineto{\pgfqpoint{2.577067in}{3.602477in}}%
\pgfpathlineto{\pgfqpoint{2.628862in}{3.623151in}}%
\pgfpathlineto{\pgfqpoint{2.658436in}{3.633234in}}%
\pgfpathlineto{\pgfqpoint{2.688009in}{3.642096in}}%
\pgfpathlineto{\pgfqpoint{2.734196in}{3.653416in}}%
\pgfpathlineto{\pgfqpoint{2.747156in}{3.656234in}}%
\pgfpathlineto{\pgfqpoint{2.776729in}{3.661421in}}%
\pgfpathlineto{\pgfqpoint{2.806303in}{3.665373in}}%
\pgfpathlineto{\pgfqpoint{2.835876in}{3.668056in}}%
\pgfpathlineto{\pgfqpoint{2.865449in}{3.669438in}}%
\pgfpathlineto{\pgfqpoint{2.895023in}{3.669484in}}%
\pgfpathlineto{\pgfqpoint{2.924596in}{3.668158in}}%
\pgfpathlineto{\pgfqpoint{2.954169in}{3.665422in}}%
\pgfpathlineto{\pgfqpoint{2.983743in}{3.661237in}}%
\pgfpathlineto{\pgfqpoint{3.022177in}{3.653416in}}%
\pgfpathlineto{\pgfqpoint{3.042890in}{3.648290in}}%
\pgfpathlineto{\pgfqpoint{3.072463in}{3.639403in}}%
\pgfpathlineto{\pgfqpoint{3.104313in}{3.627946in}}%
\pgfpathlineto{\pgfqpoint{3.131610in}{3.616554in}}%
\pgfpathlineto{\pgfqpoint{3.161186in}{3.602477in}}%
\pgfpathlineto{\pgfqpoint{3.206521in}{3.577007in}}%
\pgfpathlineto{\pgfqpoint{3.220330in}{3.568564in}}%
\pgfpathlineto{\pgfqpoint{3.249903in}{3.548662in}}%
\pgfpathlineto{\pgfqpoint{3.280268in}{3.526068in}}%
\pgfpathlineto{\pgfqpoint{3.311382in}{3.500599in}}%
\pgfpathlineto{\pgfqpoint{3.340018in}{3.475129in}}%
\pgfpathlineto{\pgfqpoint{3.368197in}{3.448117in}}%
\pgfpathlineto{\pgfqpoint{3.414998in}{3.398721in}}%
\pgfpathlineto{\pgfqpoint{3.437464in}{3.373251in}}%
\pgfpathlineto{\pgfqpoint{3.479636in}{3.322312in}}%
\pgfpathlineto{\pgfqpoint{3.516063in}{3.275320in}}%
\pgfpathlineto{\pgfqpoint{3.575210in}{3.193331in}}%
\pgfpathlineto{\pgfqpoint{3.642602in}{3.093086in}}%
\pgfpathlineto{\pgfqpoint{3.708005in}{2.991208in}}%
\pgfpathlineto{\pgfqpoint{3.787723in}{2.863861in}}%
\pgfpathlineto{\pgfqpoint{3.947075in}{2.609165in}}%
\pgfpathlineto{\pgfqpoint{4.062620in}{2.430879in}}%
\pgfpathlineto{\pgfqpoint{4.166677in}{2.277063in}}%
\pgfpathlineto{\pgfqpoint{4.237321in}{2.176183in}}%
\pgfpathlineto{\pgfqpoint{4.329717in}{2.048836in}}%
\pgfpathlineto{\pgfqpoint{4.432838in}{1.912407in}}%
\pgfpathlineto{\pgfqpoint{4.505314in}{1.819610in}}%
\pgfpathlineto{\pgfqpoint{4.610278in}{1.689776in}}%
\pgfpathlineto{\pgfqpoint{4.698998in}{1.583570in}}%
\pgfpathlineto{\pgfqpoint{4.787718in}{1.480338in}}%
\pgfpathlineto{\pgfqpoint{4.876438in}{1.379776in}}%
\pgfpathlineto{\pgfqpoint{4.965158in}{1.281608in}}%
\pgfpathlineto{\pgfqpoint{5.056368in}{1.182872in}}%
\pgfpathlineto{\pgfqpoint{5.176738in}{1.055524in}}%
\pgfpathlineto{\pgfqpoint{5.324784in}{0.902707in}}%
\pgfpathlineto{\pgfqpoint{5.501385in}{0.724420in}}%
\pgfpathlineto{\pgfqpoint{5.654718in}{0.571603in}}%
\pgfpathlineto{\pgfqpoint{5.654718in}{0.571603in}}%
\pgfusepath{stroke}%
\end{pgfscope}%
\begin{pgfscope}%
\pgfpathrectangle{\pgfqpoint{0.854460in}{0.571603in}}{\pgfqpoint{5.885100in}{5.068436in}}%
\pgfusepath{clip}%
\pgfsetbuttcap%
\pgfsetroundjoin%
\pgfsetlinewidth{1.505625pt}%
\definecolor{currentstroke}{rgb}{0.278826,0.175490,0.483397}%
\pgfsetstrokecolor{currentstroke}%
\pgfsetdash{}{0pt}%
\pgfpathmoveto{\pgfqpoint{2.785093in}{0.571603in}}%
\pgfpathlineto{\pgfqpoint{2.715383in}{0.622542in}}%
\pgfpathlineto{\pgfqpoint{2.628862in}{0.689104in}}%
\pgfpathlineto{\pgfqpoint{2.569716in}{0.736834in}}%
\pgfpathlineto{\pgfqpoint{2.510569in}{0.786620in}}%
\pgfpathlineto{\pgfqpoint{2.451422in}{0.838695in}}%
\pgfpathlineto{\pgfqpoint{2.392275in}{0.893309in}}%
\pgfpathlineto{\pgfqpoint{2.330203in}{0.953646in}}%
\pgfpathlineto{\pgfqpoint{2.273982in}{1.011526in}}%
\pgfpathlineto{\pgfqpoint{2.214835in}{1.075996in}}%
\pgfpathlineto{\pgfqpoint{2.185262in}{1.109811in}}%
\pgfpathlineto{\pgfqpoint{2.124704in}{1.182872in}}%
\pgfpathlineto{\pgfqpoint{2.066968in}{1.258131in}}%
\pgfpathlineto{\pgfqpoint{2.029709in}{1.310220in}}%
\pgfpathlineto{\pgfqpoint{1.995281in}{1.361159in}}%
\pgfpathlineto{\pgfqpoint{1.962806in}{1.412098in}}%
\pgfpathlineto{\pgfqpoint{1.932252in}{1.463037in}}%
\pgfpathlineto{\pgfqpoint{1.903583in}{1.513976in}}%
\pgfpathlineto{\pgfqpoint{1.876752in}{1.564915in}}%
\pgfpathlineto{\pgfqpoint{1.851712in}{1.615854in}}%
\pgfpathlineto{\pgfqpoint{1.828407in}{1.666793in}}%
\pgfpathlineto{\pgfqpoint{1.796785in}{1.743202in}}%
\pgfpathlineto{\pgfqpoint{1.768930in}{1.819610in}}%
\pgfpathlineto{\pgfqpoint{1.744809in}{1.896019in}}%
\pgfpathlineto{\pgfqpoint{1.730795in}{1.946958in}}%
\pgfpathlineto{\pgfqpoint{1.712599in}{2.023366in}}%
\pgfpathlineto{\pgfqpoint{1.702558in}{2.074305in}}%
\pgfpathlineto{\pgfqpoint{1.690277in}{2.150714in}}%
\pgfpathlineto{\pgfqpoint{1.682515in}{2.215869in}}%
\pgfpathlineto{\pgfqpoint{1.679240in}{2.252592in}}%
\pgfpathlineto{\pgfqpoint{1.676049in}{2.303531in}}%
\pgfpathlineto{\pgfqpoint{1.674356in}{2.354470in}}%
\pgfpathlineto{\pgfqpoint{1.674170in}{2.405409in}}%
\pgfpathlineto{\pgfqpoint{1.675492in}{2.456348in}}%
\pgfpathlineto{\pgfqpoint{1.678324in}{2.507287in}}%
\pgfpathlineto{\pgfqpoint{1.682666in}{2.558226in}}%
\pgfpathlineto{\pgfqpoint{1.688672in}{2.609165in}}%
\pgfpathlineto{\pgfqpoint{1.696214in}{2.660104in}}%
\pgfpathlineto{\pgfqpoint{1.705280in}{2.711044in}}%
\pgfpathlineto{\pgfqpoint{1.715965in}{2.761983in}}%
\pgfpathlineto{\pgfqpoint{1.735165in}{2.838391in}}%
\pgfpathlineto{\pgfqpoint{1.750092in}{2.889330in}}%
\pgfpathlineto{\pgfqpoint{1.771235in}{2.953201in}}%
\pgfpathlineto{\pgfqpoint{1.795185in}{3.016678in}}%
\pgfpathlineto{\pgfqpoint{1.816583in}{3.067617in}}%
\pgfpathlineto{\pgfqpoint{1.839933in}{3.118556in}}%
\pgfpathlineto{\pgfqpoint{1.865373in}{3.169495in}}%
\pgfpathlineto{\pgfqpoint{1.893055in}{3.220434in}}%
\pgfpathlineto{\pgfqpoint{1.923135in}{3.271373in}}%
\pgfpathlineto{\pgfqpoint{1.955780in}{3.322312in}}%
\pgfpathlineto{\pgfqpoint{1.991163in}{3.373251in}}%
\pgfpathlineto{\pgfqpoint{2.029464in}{3.424190in}}%
\pgfpathlineto{\pgfqpoint{2.066968in}{3.470383in}}%
\pgfpathlineto{\pgfqpoint{2.096542in}{3.504413in}}%
\pgfpathlineto{\pgfqpoint{2.140567in}{3.551538in}}%
\pgfpathlineto{\pgfqpoint{2.185262in}{3.595902in}}%
\pgfpathlineto{\pgfqpoint{2.220068in}{3.627946in}}%
\pgfpathlineto{\pgfqpoint{2.273982in}{3.673972in}}%
\pgfpathlineto{\pgfqpoint{2.312775in}{3.704355in}}%
\pgfpathlineto{\pgfqpoint{2.362702in}{3.740667in}}%
\pgfpathlineto{\pgfqpoint{2.423815in}{3.780764in}}%
\pgfpathlineto{\pgfqpoint{2.480996in}{3.814210in}}%
\pgfpathlineto{\pgfqpoint{2.540142in}{3.844894in}}%
\pgfpathlineto{\pgfqpoint{2.599289in}{3.871718in}}%
\pgfpathlineto{\pgfqpoint{2.658436in}{3.894733in}}%
\pgfpathlineto{\pgfqpoint{2.717582in}{3.913965in}}%
\pgfpathlineto{\pgfqpoint{2.776729in}{3.929344in}}%
\pgfpathlineto{\pgfqpoint{2.835876in}{3.940752in}}%
\pgfpathlineto{\pgfqpoint{2.895023in}{3.948059in}}%
\pgfpathlineto{\pgfqpoint{2.924596in}{3.950115in}}%
\pgfpathlineto{\pgfqpoint{2.954169in}{3.951053in}}%
\pgfpathlineto{\pgfqpoint{2.983743in}{3.950834in}}%
\pgfpathlineto{\pgfqpoint{3.013316in}{3.949413in}}%
\pgfpathlineto{\pgfqpoint{3.042890in}{3.946745in}}%
\pgfpathlineto{\pgfqpoint{3.072463in}{3.942779in}}%
\pgfpathlineto{\pgfqpoint{3.119233in}{3.933581in}}%
\pgfpathlineto{\pgfqpoint{3.131610in}{3.930700in}}%
\pgfpathlineto{\pgfqpoint{3.161183in}{3.922392in}}%
\pgfpathlineto{\pgfqpoint{3.202287in}{3.908111in}}%
\pgfpathlineto{\pgfqpoint{3.220330in}{3.900974in}}%
\pgfpathlineto{\pgfqpoint{3.259804in}{3.882642in}}%
\pgfpathlineto{\pgfqpoint{3.279476in}{3.872411in}}%
\pgfpathlineto{\pgfqpoint{3.309050in}{3.855217in}}%
\pgfpathlineto{\pgfqpoint{3.344457in}{3.831703in}}%
\pgfpathlineto{\pgfqpoint{3.378316in}{3.806233in}}%
\pgfpathlineto{\pgfqpoint{3.408674in}{3.780764in}}%
\pgfpathlineto{\pgfqpoint{3.436309in}{3.755294in}}%
\pgfpathlineto{\pgfqpoint{3.461792in}{3.729825in}}%
\pgfpathlineto{\pgfqpoint{3.486490in}{3.703280in}}%
\pgfpathlineto{\pgfqpoint{3.528436in}{3.653416in}}%
\pgfpathlineto{\pgfqpoint{3.567260in}{3.602477in}}%
\pgfpathlineto{\pgfqpoint{3.585399in}{3.577007in}}%
\pgfpathlineto{\pgfqpoint{3.619737in}{3.526068in}}%
\pgfpathlineto{\pgfqpoint{3.652007in}{3.475129in}}%
\pgfpathlineto{\pgfqpoint{3.693504in}{3.405884in}}%
\pgfpathlineto{\pgfqpoint{3.754717in}{3.296843in}}%
\pgfpathlineto{\pgfqpoint{3.841370in}{3.134652in}}%
\pgfpathlineto{\pgfqpoint{3.930348in}{2.965739in}}%
\pgfpathlineto{\pgfqpoint{3.998607in}{2.838391in}}%
\pgfpathlineto{\pgfqpoint{4.069081in}{2.711044in}}%
\pgfpathlineto{\pgfqpoint{4.157421in}{2.558226in}}%
\pgfpathlineto{\pgfqpoint{4.234760in}{2.430879in}}%
\pgfpathlineto{\pgfqpoint{4.299312in}{2.329001in}}%
\pgfpathlineto{\pgfqpoint{4.383608in}{2.201653in}}%
\pgfpathlineto{\pgfqpoint{4.462411in}{2.087994in}}%
\pgfpathlineto{\pgfqpoint{4.527177in}{1.997897in}}%
\pgfpathlineto{\pgfqpoint{4.610278in}{1.886599in}}%
\pgfpathlineto{\pgfqpoint{4.681731in}{1.794141in}}%
\pgfpathlineto{\pgfqpoint{4.763219in}{1.692262in}}%
\pgfpathlineto{\pgfqpoint{4.847472in}{1.590384in}}%
\pgfpathlineto{\pgfqpoint{4.935585in}{1.487159in}}%
\pgfpathlineto{\pgfqpoint{5.024305in}{1.386317in}}%
\pgfpathlineto{\pgfqpoint{5.116201in}{1.284750in}}%
\pgfpathlineto{\pgfqpoint{5.231319in}{1.161234in}}%
\pgfpathlineto{\pgfqpoint{5.332531in}{1.055524in}}%
\pgfpathlineto{\pgfqpoint{5.467906in}{0.917862in}}%
\pgfpathlineto{\pgfqpoint{5.586199in}{0.800446in}}%
\pgfpathlineto{\pgfqpoint{5.742861in}{0.648012in}}%
\pgfpathlineto{\pgfqpoint{5.822537in}{0.571603in}}%
\pgfpathlineto{\pgfqpoint{5.822537in}{0.571603in}}%
\pgfusepath{stroke}%
\end{pgfscope}%
\begin{pgfscope}%
\pgfpathrectangle{\pgfqpoint{0.854460in}{0.571603in}}{\pgfqpoint{5.885100in}{5.068436in}}%
\pgfusepath{clip}%
\pgfsetbuttcap%
\pgfsetroundjoin%
\pgfsetlinewidth{1.505625pt}%
\definecolor{currentstroke}{rgb}{0.273006,0.204520,0.501721}%
\pgfsetstrokecolor{currentstroke}%
\pgfsetdash{}{0pt}%
\pgfpathmoveto{\pgfqpoint{2.648461in}{0.571603in}}%
\pgfpathlineto{\pgfqpoint{2.569716in}{0.630911in}}%
\pgfpathlineto{\pgfqpoint{2.510569in}{0.677376in}}%
\pgfpathlineto{\pgfqpoint{2.451422in}{0.725724in}}%
\pgfpathlineto{\pgfqpoint{2.392275in}{0.776164in}}%
\pgfpathlineto{\pgfqpoint{2.333129in}{0.828923in}}%
\pgfpathlineto{\pgfqpoint{2.273982in}{0.884248in}}%
\pgfpathlineto{\pgfqpoint{2.214835in}{0.942409in}}%
\pgfpathlineto{\pgfqpoint{2.154856in}{1.004585in}}%
\pgfpathlineto{\pgfqpoint{2.096542in}{1.068903in}}%
\pgfpathlineto{\pgfqpoint{2.042639in}{1.131933in}}%
\pgfpathlineto{\pgfqpoint{2.001497in}{1.182872in}}%
\pgfpathlineto{\pgfqpoint{1.948675in}{1.252391in}}%
\pgfpathlineto{\pgfqpoint{1.907566in}{1.310220in}}%
\pgfpathlineto{\pgfqpoint{1.873426in}{1.361159in}}%
\pgfpathlineto{\pgfqpoint{1.830381in}{1.429815in}}%
\pgfpathlineto{\pgfqpoint{1.796225in}{1.488506in}}%
\pgfpathlineto{\pgfqpoint{1.755378in}{1.564915in}}%
\pgfpathlineto{\pgfqpoint{1.730293in}{1.615854in}}%
\pgfpathlineto{\pgfqpoint{1.706880in}{1.666793in}}%
\pgfpathlineto{\pgfqpoint{1.674934in}{1.743202in}}%
\pgfpathlineto{\pgfqpoint{1.646612in}{1.819610in}}%
\pgfpathlineto{\pgfqpoint{1.621809in}{1.896019in}}%
\pgfpathlineto{\pgfqpoint{1.600559in}{1.972427in}}%
\pgfpathlineto{\pgfqpoint{1.588247in}{2.023366in}}%
\pgfpathlineto{\pgfqpoint{1.572598in}{2.099775in}}%
\pgfpathlineto{\pgfqpoint{1.563922in}{2.150714in}}%
\pgfpathlineto{\pgfqpoint{1.553806in}{2.227123in}}%
\pgfpathlineto{\pgfqpoint{1.546851in}{2.303531in}}%
\pgfpathlineto{\pgfqpoint{1.543083in}{2.379940in}}%
\pgfpathlineto{\pgfqpoint{1.542519in}{2.456348in}}%
\pgfpathlineto{\pgfqpoint{1.545165in}{2.532757in}}%
\pgfpathlineto{\pgfqpoint{1.551013in}{2.609165in}}%
\pgfpathlineto{\pgfqpoint{1.560041in}{2.685574in}}%
\pgfpathlineto{\pgfqpoint{1.567906in}{2.736513in}}%
\pgfpathlineto{\pgfqpoint{1.582535in}{2.812922in}}%
\pgfpathlineto{\pgfqpoint{1.594046in}{2.863861in}}%
\pgfpathlineto{\pgfqpoint{1.614425in}{2.940269in}}%
\pgfpathlineto{\pgfqpoint{1.629920in}{2.991208in}}%
\pgfpathlineto{\pgfqpoint{1.656227in}{3.067617in}}%
\pgfpathlineto{\pgfqpoint{1.686409in}{3.144025in}}%
\pgfpathlineto{\pgfqpoint{1.712088in}{3.202247in}}%
\pgfpathlineto{\pgfqpoint{1.745810in}{3.271373in}}%
\pgfpathlineto{\pgfqpoint{1.772956in}{3.322312in}}%
\pgfpathlineto{\pgfqpoint{1.802225in}{3.373251in}}%
\pgfpathlineto{\pgfqpoint{1.833741in}{3.424190in}}%
\pgfpathlineto{\pgfqpoint{1.867632in}{3.475129in}}%
\pgfpathlineto{\pgfqpoint{1.904024in}{3.526068in}}%
\pgfpathlineto{\pgfqpoint{1.948675in}{3.584097in}}%
\pgfpathlineto{\pgfqpoint{1.985067in}{3.627946in}}%
\pgfpathlineto{\pgfqpoint{2.037395in}{3.686595in}}%
\pgfpathlineto{\pgfqpoint{2.079126in}{3.729825in}}%
\pgfpathlineto{\pgfqpoint{2.126115in}{3.775451in}}%
\pgfpathlineto{\pgfqpoint{2.159921in}{3.806233in}}%
\pgfpathlineto{\pgfqpoint{2.219853in}{3.857172in}}%
\pgfpathlineto{\pgfqpoint{2.273982in}{3.899486in}}%
\pgfpathlineto{\pgfqpoint{2.333129in}{3.941988in}}%
\pgfpathlineto{\pgfqpoint{2.398148in}{3.984520in}}%
\pgfpathlineto{\pgfqpoint{2.451422in}{4.016351in}}%
\pgfpathlineto{\pgfqpoint{2.510569in}{4.048549in}}%
\pgfpathlineto{\pgfqpoint{2.569716in}{4.077624in}}%
\pgfpathlineto{\pgfqpoint{2.628862in}{4.103621in}}%
\pgfpathlineto{\pgfqpoint{2.688009in}{4.126571in}}%
\pgfpathlineto{\pgfqpoint{2.747156in}{4.146491in}}%
\pgfpathlineto{\pgfqpoint{2.806303in}{4.163386in}}%
\pgfpathlineto{\pgfqpoint{2.865449in}{4.177019in}}%
\pgfpathlineto{\pgfqpoint{2.924596in}{4.187497in}}%
\pgfpathlineto{\pgfqpoint{2.983743in}{4.194449in}}%
\pgfpathlineto{\pgfqpoint{3.042890in}{4.197780in}}%
\pgfpathlineto{\pgfqpoint{3.102036in}{4.197193in}}%
\pgfpathlineto{\pgfqpoint{3.131610in}{4.195322in}}%
\pgfpathlineto{\pgfqpoint{3.161183in}{4.192335in}}%
\pgfpathlineto{\pgfqpoint{3.190756in}{4.188179in}}%
\pgfpathlineto{\pgfqpoint{3.220330in}{4.182695in}}%
\pgfpathlineto{\pgfqpoint{3.249903in}{4.175903in}}%
\pgfpathlineto{\pgfqpoint{3.294791in}{4.162807in}}%
\pgfpathlineto{\pgfqpoint{3.309050in}{4.158043in}}%
\pgfpathlineto{\pgfqpoint{3.360115in}{4.137337in}}%
\pgfpathlineto{\pgfqpoint{3.368197in}{4.133682in}}%
\pgfpathlineto{\pgfqpoint{3.410022in}{4.111867in}}%
\pgfpathlineto{\pgfqpoint{3.427343in}{4.101790in}}%
\pgfpathlineto{\pgfqpoint{3.456917in}{4.082688in}}%
\pgfpathlineto{\pgfqpoint{3.486860in}{4.060928in}}%
\pgfpathlineto{\pgfqpoint{3.517996in}{4.035459in}}%
\pgfpathlineto{\pgfqpoint{3.545997in}{4.009989in}}%
\pgfpathlineto{\pgfqpoint{3.575210in}{3.980505in}}%
\pgfpathlineto{\pgfqpoint{3.616258in}{3.933581in}}%
\pgfpathlineto{\pgfqpoint{3.636564in}{3.908111in}}%
\pgfpathlineto{\pgfqpoint{3.673376in}{3.857172in}}%
\pgfpathlineto{\pgfqpoint{3.706598in}{3.806233in}}%
\pgfpathlineto{\pgfqpoint{3.737052in}{3.755294in}}%
\pgfpathlineto{\pgfqpoint{3.765369in}{3.704355in}}%
\pgfpathlineto{\pgfqpoint{3.804883in}{3.627946in}}%
\pgfpathlineto{\pgfqpoint{3.841370in}{3.552664in}}%
\pgfpathlineto{\pgfqpoint{3.900517in}{3.423225in}}%
\pgfpathlineto{\pgfqpoint{4.058482in}{3.067617in}}%
\pgfpathlineto{\pgfqpoint{4.117793in}{2.940269in}}%
\pgfpathlineto{\pgfqpoint{4.179930in}{2.812922in}}%
\pgfpathlineto{\pgfqpoint{4.245475in}{2.685574in}}%
\pgfpathlineto{\pgfqpoint{4.314893in}{2.558226in}}%
\pgfpathlineto{\pgfqpoint{4.358413in}{2.481818in}}%
\pgfpathlineto{\pgfqpoint{4.418943in}{2.379940in}}%
\pgfpathlineto{\pgfqpoint{4.491984in}{2.263062in}}%
\pgfpathlineto{\pgfqpoint{4.531788in}{2.201653in}}%
\pgfpathlineto{\pgfqpoint{4.600337in}{2.099775in}}%
\pgfpathlineto{\pgfqpoint{4.671938in}{1.997897in}}%
\pgfpathlineto{\pgfqpoint{4.746501in}{1.896019in}}%
\pgfpathlineto{\pgfqpoint{4.824166in}{1.794141in}}%
\pgfpathlineto{\pgfqpoint{4.906012in}{1.690880in}}%
\pgfpathlineto{\pgfqpoint{4.988557in}{1.590384in}}%
\pgfpathlineto{\pgfqpoint{5.083452in}{1.479058in}}%
\pgfpathlineto{\pgfqpoint{5.164829in}{1.386628in}}%
\pgfpathlineto{\pgfqpoint{5.260892in}{1.280871in}}%
\pgfpathlineto{\pgfqpoint{5.352528in}{1.182872in}}%
\pgfpathlineto{\pgfqpoint{5.467906in}{1.063063in}}%
\pgfpathlineto{\pgfqpoint{5.576303in}{0.953646in}}%
\pgfpathlineto{\pgfqpoint{5.705939in}{0.826299in}}%
\pgfpathlineto{\pgfqpoint{5.852359in}{0.686115in}}%
\pgfpathlineto{\pgfqpoint{5.974598in}{0.571603in}}%
\pgfpathlineto{\pgfqpoint{5.974598in}{0.571603in}}%
\pgfusepath{stroke}%
\end{pgfscope}%
\begin{pgfscope}%
\pgfpathrectangle{\pgfqpoint{0.854460in}{0.571603in}}{\pgfqpoint{5.885100in}{5.068436in}}%
\pgfusepath{clip}%
\pgfsetbuttcap%
\pgfsetroundjoin%
\pgfsetlinewidth{1.505625pt}%
\definecolor{currentstroke}{rgb}{0.266580,0.228262,0.514349}%
\pgfsetstrokecolor{currentstroke}%
\pgfsetdash{}{0pt}%
\pgfpathmoveto{\pgfqpoint{2.523741in}{0.571603in}}%
\pgfpathlineto{\pgfqpoint{2.451422in}{0.627154in}}%
\pgfpathlineto{\pgfqpoint{2.392275in}{0.674470in}}%
\pgfpathlineto{\pgfqpoint{2.332323in}{0.724420in}}%
\pgfpathlineto{\pgfqpoint{2.273700in}{0.775360in}}%
\pgfpathlineto{\pgfqpoint{2.214835in}{0.828849in}}%
\pgfpathlineto{\pgfqpoint{2.155689in}{0.885203in}}%
\pgfpathlineto{\pgfqpoint{2.096542in}{0.944431in}}%
\pgfpathlineto{\pgfqpoint{2.039514in}{1.004585in}}%
\pgfpathlineto{\pgfqpoint{1.993724in}{1.055524in}}%
\pgfpathlineto{\pgfqpoint{1.948675in}{1.107994in}}%
\pgfpathlineto{\pgfqpoint{1.888250in}{1.182872in}}%
\pgfpathlineto{\pgfqpoint{1.830381in}{1.260203in}}%
\pgfpathlineto{\pgfqpoint{1.778215in}{1.335689in}}%
\pgfpathlineto{\pgfqpoint{1.741661in}{1.392427in}}%
\pgfpathlineto{\pgfqpoint{1.699324in}{1.463037in}}%
\pgfpathlineto{\pgfqpoint{1.670859in}{1.513976in}}%
\pgfpathlineto{\pgfqpoint{1.644100in}{1.564915in}}%
\pgfpathlineto{\pgfqpoint{1.619002in}{1.615854in}}%
\pgfpathlineto{\pgfqpoint{1.584491in}{1.692262in}}%
\pgfpathlineto{\pgfqpoint{1.553544in}{1.768671in}}%
\pgfpathlineto{\pgfqpoint{1.526073in}{1.845080in}}%
\pgfpathlineto{\pgfqpoint{1.501972in}{1.921488in}}%
\pgfpathlineto{\pgfqpoint{1.481244in}{1.997897in}}%
\pgfpathlineto{\pgfqpoint{1.469223in}{2.048836in}}%
\pgfpathlineto{\pgfqpoint{1.453860in}{2.125244in}}%
\pgfpathlineto{\pgfqpoint{1.441611in}{2.201653in}}%
\pgfpathlineto{\pgfqpoint{1.432530in}{2.278062in}}%
\pgfpathlineto{\pgfqpoint{1.426468in}{2.354470in}}%
\pgfpathlineto{\pgfqpoint{1.423441in}{2.430879in}}%
\pgfpathlineto{\pgfqpoint{1.423456in}{2.507287in}}%
\pgfpathlineto{\pgfqpoint{1.426511in}{2.583696in}}%
\pgfpathlineto{\pgfqpoint{1.432594in}{2.660104in}}%
\pgfpathlineto{\pgfqpoint{1.441680in}{2.736513in}}%
\pgfpathlineto{\pgfqpoint{1.449476in}{2.787452in}}%
\pgfpathlineto{\pgfqpoint{1.463840in}{2.863861in}}%
\pgfpathlineto{\pgfqpoint{1.481311in}{2.940269in}}%
\pgfpathlineto{\pgfqpoint{1.502055in}{3.016678in}}%
\pgfpathlineto{\pgfqpoint{1.517825in}{3.067617in}}%
\pgfpathlineto{\pgfqpoint{1.534977in}{3.118556in}}%
\pgfpathlineto{\pgfqpoint{1.564221in}{3.196000in}}%
\pgfpathlineto{\pgfqpoint{1.596357in}{3.271373in}}%
\pgfpathlineto{\pgfqpoint{1.632833in}{3.347782in}}%
\pgfpathlineto{\pgfqpoint{1.659367in}{3.398721in}}%
\pgfpathlineto{\pgfqpoint{1.687824in}{3.449660in}}%
\pgfpathlineto{\pgfqpoint{1.718303in}{3.500599in}}%
\pgfpathlineto{\pgfqpoint{1.750902in}{3.551538in}}%
\pgfpathlineto{\pgfqpoint{1.785723in}{3.602477in}}%
\pgfpathlineto{\pgfqpoint{1.830381in}{3.663373in}}%
\pgfpathlineto{\pgfqpoint{1.862502in}{3.704355in}}%
\pgfpathlineto{\pgfqpoint{1.904983in}{3.755294in}}%
\pgfpathlineto{\pgfqpoint{1.950213in}{3.806233in}}%
\pgfpathlineto{\pgfqpoint{2.007822in}{3.866299in}}%
\pgfpathlineto{\pgfqpoint{2.066968in}{3.923220in}}%
\pgfpathlineto{\pgfqpoint{2.126115in}{3.975839in}}%
\pgfpathlineto{\pgfqpoint{2.185262in}{4.024562in}}%
\pgfpathlineto{\pgfqpoint{2.244409in}{4.069748in}}%
\pgfpathlineto{\pgfqpoint{2.303790in}{4.111867in}}%
\pgfpathlineto{\pgfqpoint{2.362702in}{4.150489in}}%
\pgfpathlineto{\pgfqpoint{2.424939in}{4.188276in}}%
\pgfpathlineto{\pgfqpoint{2.480996in}{4.219676in}}%
\pgfpathlineto{\pgfqpoint{2.540142in}{4.250227in}}%
\pgfpathlineto{\pgfqpoint{2.599289in}{4.278229in}}%
\pgfpathlineto{\pgfqpoint{2.658436in}{4.303721in}}%
\pgfpathlineto{\pgfqpoint{2.717582in}{4.326729in}}%
\pgfpathlineto{\pgfqpoint{2.776729in}{4.347272in}}%
\pgfpathlineto{\pgfqpoint{2.840371in}{4.366563in}}%
\pgfpathlineto{\pgfqpoint{2.895023in}{4.380757in}}%
\pgfpathlineto{\pgfqpoint{2.954169in}{4.393661in}}%
\pgfpathlineto{\pgfqpoint{3.013316in}{4.403729in}}%
\pgfpathlineto{\pgfqpoint{3.072463in}{4.410989in}}%
\pgfpathlineto{\pgfqpoint{3.131610in}{4.415199in}}%
\pgfpathlineto{\pgfqpoint{3.190756in}{4.416079in}}%
\pgfpathlineto{\pgfqpoint{3.249903in}{4.413310in}}%
\pgfpathlineto{\pgfqpoint{3.309050in}{4.406528in}}%
\pgfpathlineto{\pgfqpoint{3.338623in}{4.401502in}}%
\pgfpathlineto{\pgfqpoint{3.381388in}{4.392032in}}%
\pgfpathlineto{\pgfqpoint{3.397770in}{4.387790in}}%
\pgfpathlineto{\pgfqpoint{3.427343in}{4.378866in}}%
\pgfpathlineto{\pgfqpoint{3.461941in}{4.366563in}}%
\pgfpathlineto{\pgfqpoint{3.486490in}{4.356496in}}%
\pgfpathlineto{\pgfqpoint{3.519392in}{4.341093in}}%
\pgfpathlineto{\pgfqpoint{3.565002in}{4.315624in}}%
\pgfpathlineto{\pgfqpoint{3.575210in}{4.309320in}}%
\pgfpathlineto{\pgfqpoint{3.604783in}{4.289267in}}%
\pgfpathlineto{\pgfqpoint{3.636667in}{4.264685in}}%
\pgfpathlineto{\pgfqpoint{3.665953in}{4.239215in}}%
\pgfpathlineto{\pgfqpoint{3.693504in}{4.212438in}}%
\pgfpathlineto{\pgfqpoint{3.723077in}{4.180220in}}%
\pgfpathlineto{\pgfqpoint{3.757812in}{4.137337in}}%
\pgfpathlineto{\pgfqpoint{3.793851in}{4.086398in}}%
\pgfpathlineto{\pgfqpoint{3.825657in}{4.035459in}}%
\pgfpathlineto{\pgfqpoint{3.854213in}{3.984520in}}%
\pgfpathlineto{\pgfqpoint{3.880260in}{3.933581in}}%
\pgfpathlineto{\pgfqpoint{3.904370in}{3.882642in}}%
\pgfpathlineto{\pgfqpoint{3.937655in}{3.806233in}}%
\pgfpathlineto{\pgfqpoint{3.968576in}{3.729825in}}%
\pgfpathlineto{\pgfqpoint{4.016975in}{3.602477in}}%
\pgfpathlineto{\pgfqpoint{4.054229in}{3.500599in}}%
\pgfpathlineto{\pgfqpoint{4.129014in}{3.296843in}}%
\pgfpathlineto{\pgfqpoint{4.187788in}{3.144025in}}%
\pgfpathlineto{\pgfqpoint{4.239728in}{3.016678in}}%
\pgfpathlineto{\pgfqpoint{4.284971in}{2.912096in}}%
\pgfpathlineto{\pgfqpoint{4.330071in}{2.812922in}}%
\pgfpathlineto{\pgfqpoint{4.379087in}{2.711044in}}%
\pgfpathlineto{\pgfqpoint{4.430876in}{2.609165in}}%
\pgfpathlineto{\pgfqpoint{4.491984in}{2.495726in}}%
\pgfpathlineto{\pgfqpoint{4.528506in}{2.430879in}}%
\pgfpathlineto{\pgfqpoint{4.588514in}{2.329001in}}%
\pgfpathlineto{\pgfqpoint{4.651661in}{2.227123in}}%
\pgfpathlineto{\pgfqpoint{4.718018in}{2.125244in}}%
\pgfpathlineto{\pgfqpoint{4.787718in}{2.023276in}}%
\pgfpathlineto{\pgfqpoint{4.846865in}{1.940209in}}%
\pgfpathlineto{\pgfqpoint{4.906012in}{1.860012in}}%
\pgfpathlineto{\pgfqpoint{4.965158in}{1.782409in}}%
\pgfpathlineto{\pgfqpoint{5.036166in}{1.692262in}}%
\pgfpathlineto{\pgfqpoint{5.119525in}{1.590384in}}%
\pgfpathlineto{\pgfqpoint{5.206055in}{1.488506in}}%
\pgfpathlineto{\pgfqpoint{5.295698in}{1.386628in}}%
\pgfpathlineto{\pgfqpoint{5.388396in}{1.284750in}}%
\pgfpathlineto{\pgfqpoint{5.497479in}{1.168901in}}%
\pgfpathlineto{\pgfqpoint{5.586199in}{1.077473in}}%
\pgfpathlineto{\pgfqpoint{5.684106in}{0.979116in}}%
\pgfpathlineto{\pgfqpoint{5.814594in}{0.851768in}}%
\pgfpathlineto{\pgfqpoint{5.941079in}{0.731792in}}%
\pgfpathlineto{\pgfqpoint{6.059373in}{0.622232in}}%
\pgfpathlineto{\pgfqpoint{6.114810in}{0.571603in}}%
\pgfpathlineto{\pgfqpoint{6.114810in}{0.571603in}}%
\pgfusepath{stroke}%
\end{pgfscope}%
\begin{pgfscope}%
\pgfpathrectangle{\pgfqpoint{0.854460in}{0.571603in}}{\pgfqpoint{5.885100in}{5.068436in}}%
\pgfusepath{clip}%
\pgfsetbuttcap%
\pgfsetroundjoin%
\pgfsetlinewidth{1.505625pt}%
\definecolor{currentstroke}{rgb}{0.257322,0.256130,0.526563}%
\pgfsetstrokecolor{currentstroke}%
\pgfsetdash{}{0pt}%
\pgfpathmoveto{\pgfqpoint{2.408459in}{0.571603in}}%
\pgfpathlineto{\pgfqpoint{2.333129in}{0.630694in}}%
\pgfpathlineto{\pgfqpoint{2.273982in}{0.679082in}}%
\pgfpathlineto{\pgfqpoint{2.214835in}{0.729455in}}%
\pgfpathlineto{\pgfqpoint{2.155689in}{0.782021in}}%
\pgfpathlineto{\pgfqpoint{2.096542in}{0.837003in}}%
\pgfpathlineto{\pgfqpoint{2.029326in}{0.902707in}}%
\pgfpathlineto{\pgfqpoint{1.978248in}{0.955263in}}%
\pgfpathlineto{\pgfqpoint{1.909625in}{1.030055in}}%
\pgfpathlineto{\pgfqpoint{1.859955in}{1.087518in}}%
\pgfpathlineto{\pgfqpoint{1.802978in}{1.157402in}}%
\pgfpathlineto{\pgfqpoint{1.763855in}{1.208341in}}%
\pgfpathlineto{\pgfqpoint{1.712088in}{1.279872in}}%
\pgfpathlineto{\pgfqpoint{1.674253in}{1.335689in}}%
\pgfpathlineto{\pgfqpoint{1.641616in}{1.386628in}}%
\pgfpathlineto{\pgfqpoint{1.610735in}{1.437567in}}%
\pgfpathlineto{\pgfqpoint{1.581573in}{1.488506in}}%
\pgfpathlineto{\pgfqpoint{1.554090in}{1.539445in}}%
\pgfpathlineto{\pgfqpoint{1.528242in}{1.590384in}}%
\pgfpathlineto{\pgfqpoint{1.492531in}{1.666793in}}%
\pgfpathlineto{\pgfqpoint{1.470637in}{1.717732in}}%
\pgfpathlineto{\pgfqpoint{1.440711in}{1.794141in}}%
\pgfpathlineto{\pgfqpoint{1.414113in}{1.870549in}}%
\pgfpathlineto{\pgfqpoint{1.390820in}{1.946958in}}%
\pgfpathlineto{\pgfqpoint{1.377088in}{1.997897in}}%
\pgfpathlineto{\pgfqpoint{1.359004in}{2.074305in}}%
\pgfpathlineto{\pgfqpoint{1.348741in}{2.125244in}}%
\pgfpathlineto{\pgfqpoint{1.335805in}{2.201653in}}%
\pgfpathlineto{\pgfqpoint{1.325816in}{2.278062in}}%
\pgfpathlineto{\pgfqpoint{1.318875in}{2.354470in}}%
\pgfpathlineto{\pgfqpoint{1.314819in}{2.430879in}}%
\pgfpathlineto{\pgfqpoint{1.313656in}{2.507287in}}%
\pgfpathlineto{\pgfqpoint{1.315382in}{2.583696in}}%
\pgfpathlineto{\pgfqpoint{1.319987in}{2.660104in}}%
\pgfpathlineto{\pgfqpoint{1.327634in}{2.738012in}}%
\pgfpathlineto{\pgfqpoint{1.337984in}{2.812922in}}%
\pgfpathlineto{\pgfqpoint{1.351383in}{2.889330in}}%
\pgfpathlineto{\pgfqpoint{1.361996in}{2.940269in}}%
\pgfpathlineto{\pgfqpoint{1.380493in}{3.016678in}}%
\pgfpathlineto{\pgfqpoint{1.394562in}{3.067617in}}%
\pgfpathlineto{\pgfqpoint{1.418274in}{3.144025in}}%
\pgfpathlineto{\pgfqpoint{1.445928in}{3.222040in}}%
\pgfpathlineto{\pgfqpoint{1.475808in}{3.296843in}}%
\pgfpathlineto{\pgfqpoint{1.509930in}{3.373251in}}%
\pgfpathlineto{\pgfqpoint{1.547812in}{3.449660in}}%
\pgfpathlineto{\pgfqpoint{1.575228in}{3.500599in}}%
\pgfpathlineto{\pgfqpoint{1.604490in}{3.551538in}}%
\pgfpathlineto{\pgfqpoint{1.635680in}{3.602477in}}%
\pgfpathlineto{\pgfqpoint{1.668884in}{3.653416in}}%
\pgfpathlineto{\pgfqpoint{1.712088in}{3.715381in}}%
\pgfpathlineto{\pgfqpoint{1.761415in}{3.780764in}}%
\pgfpathlineto{\pgfqpoint{1.802519in}{3.831703in}}%
\pgfpathlineto{\pgfqpoint{1.859955in}{3.897851in}}%
\pgfpathlineto{\pgfqpoint{1.892930in}{3.933581in}}%
\pgfpathlineto{\pgfqpoint{1.948675in}{3.990568in}}%
\pgfpathlineto{\pgfqpoint{2.007822in}{4.046822in}}%
\pgfpathlineto{\pgfqpoint{2.066968in}{4.099226in}}%
\pgfpathlineto{\pgfqpoint{2.126115in}{4.148134in}}%
\pgfpathlineto{\pgfqpoint{2.185262in}{4.193863in}}%
\pgfpathlineto{\pgfqpoint{2.248121in}{4.239215in}}%
\pgfpathlineto{\pgfqpoint{2.303555in}{4.276611in}}%
\pgfpathlineto{\pgfqpoint{2.365231in}{4.315624in}}%
\pgfpathlineto{\pgfqpoint{2.421849in}{4.349035in}}%
\pgfpathlineto{\pgfqpoint{2.480996in}{4.381681in}}%
\pgfpathlineto{\pgfqpoint{2.551170in}{4.417502in}}%
\pgfpathlineto{\pgfqpoint{2.605036in}{4.442971in}}%
\pgfpathlineto{\pgfqpoint{2.663107in}{4.468441in}}%
\pgfpathlineto{\pgfqpoint{2.726489in}{4.493910in}}%
\pgfpathlineto{\pgfqpoint{2.796708in}{4.519380in}}%
\pgfpathlineto{\pgfqpoint{2.835876in}{4.532325in}}%
\pgfpathlineto{\pgfqpoint{2.895023in}{4.550153in}}%
\pgfpathlineto{\pgfqpoint{2.972915in}{4.570319in}}%
\pgfpathlineto{\pgfqpoint{3.013316in}{4.579317in}}%
\pgfpathlineto{\pgfqpoint{3.072463in}{4.590580in}}%
\pgfpathlineto{\pgfqpoint{3.131610in}{4.599469in}}%
\pgfpathlineto{\pgfqpoint{3.190756in}{4.605810in}}%
\pgfpathlineto{\pgfqpoint{3.249903in}{4.609506in}}%
\pgfpathlineto{\pgfqpoint{3.309050in}{4.610308in}}%
\pgfpathlineto{\pgfqpoint{3.368197in}{4.607929in}}%
\pgfpathlineto{\pgfqpoint{3.427343in}{4.602040in}}%
\pgfpathlineto{\pgfqpoint{3.486490in}{4.592165in}}%
\pgfpathlineto{\pgfqpoint{3.516063in}{4.585507in}}%
\pgfpathlineto{\pgfqpoint{3.569663in}{4.570319in}}%
\pgfpathlineto{\pgfqpoint{3.575210in}{4.568541in}}%
\pgfpathlineto{\pgfqpoint{3.604783in}{4.557858in}}%
\pgfpathlineto{\pgfqpoint{3.636289in}{4.544849in}}%
\pgfpathlineto{\pgfqpoint{3.663930in}{4.531712in}}%
\pgfpathlineto{\pgfqpoint{3.693504in}{4.515893in}}%
\pgfpathlineto{\pgfqpoint{3.729076in}{4.493910in}}%
\pgfpathlineto{\pgfqpoint{3.764577in}{4.468441in}}%
\pgfpathlineto{\pgfqpoint{3.795522in}{4.442971in}}%
\pgfpathlineto{\pgfqpoint{3.822915in}{4.417502in}}%
\pgfpathlineto{\pgfqpoint{3.847496in}{4.392032in}}%
\pgfpathlineto{\pgfqpoint{3.870944in}{4.365164in}}%
\pgfpathlineto{\pgfqpoint{3.900517in}{4.326960in}}%
\pgfpathlineto{\pgfqpoint{3.930090in}{4.283549in}}%
\pgfpathlineto{\pgfqpoint{3.959664in}{4.234050in}}%
\pgfpathlineto{\pgfqpoint{3.989237in}{4.177585in}}%
\pgfpathlineto{\pgfqpoint{4.018811in}{4.113390in}}%
\pgfpathlineto{\pgfqpoint{4.040435in}{4.060928in}}%
\pgfpathlineto{\pgfqpoint{4.059863in}{4.009989in}}%
\pgfpathlineto{\pgfqpoint{4.086762in}{3.933581in}}%
\pgfpathlineto{\pgfqpoint{4.119682in}{3.831703in}}%
\pgfpathlineto{\pgfqpoint{4.166677in}{3.676284in}}%
\pgfpathlineto{\pgfqpoint{4.226620in}{3.475129in}}%
\pgfpathlineto{\pgfqpoint{4.266281in}{3.347782in}}%
\pgfpathlineto{\pgfqpoint{4.308516in}{3.220434in}}%
\pgfpathlineto{\pgfqpoint{4.353909in}{3.093086in}}%
\pgfpathlineto{\pgfqpoint{4.392902in}{2.991208in}}%
\pgfpathlineto{\pgfqpoint{4.434557in}{2.889330in}}%
\pgfpathlineto{\pgfqpoint{4.467572in}{2.812922in}}%
\pgfpathlineto{\pgfqpoint{4.514173in}{2.711044in}}%
\pgfpathlineto{\pgfqpoint{4.563818in}{2.609165in}}%
\pgfpathlineto{\pgfqpoint{4.616645in}{2.507287in}}%
\pgfpathlineto{\pgfqpoint{4.669425in}{2.411282in}}%
\pgfpathlineto{\pgfqpoint{4.701965in}{2.354470in}}%
\pgfpathlineto{\pgfqpoint{4.758145in}{2.260528in}}%
\pgfpathlineto{\pgfqpoint{4.794731in}{2.201653in}}%
\pgfpathlineto{\pgfqpoint{4.860797in}{2.099775in}}%
\pgfpathlineto{\pgfqpoint{4.935585in}{1.990419in}}%
\pgfpathlineto{\pgfqpoint{4.994732in}{1.907685in}}%
\pgfpathlineto{\pgfqpoint{5.053878in}{1.827927in}}%
\pgfpathlineto{\pgfqpoint{5.113025in}{1.750846in}}%
\pgfpathlineto{\pgfqpoint{5.172172in}{1.676177in}}%
\pgfpathlineto{\pgfqpoint{5.242315in}{1.590384in}}%
\pgfpathlineto{\pgfqpoint{5.328754in}{1.488506in}}%
\pgfpathlineto{\pgfqpoint{5.418481in}{1.386628in}}%
\pgfpathlineto{\pgfqpoint{5.511446in}{1.284750in}}%
\pgfpathlineto{\pgfqpoint{5.615772in}{1.174389in}}%
\pgfpathlineto{\pgfqpoint{5.706866in}{1.080994in}}%
\pgfpathlineto{\pgfqpoint{5.822786in}{0.965687in}}%
\pgfpathlineto{\pgfqpoint{5.914235in}{0.877238in}}%
\pgfpathlineto{\pgfqpoint{6.049494in}{0.749890in}}%
\pgfpathlineto{\pgfqpoint{6.177666in}{0.632596in}}%
\pgfpathlineto{\pgfqpoint{6.245586in}{0.571603in}}%
\pgfpathlineto{\pgfqpoint{6.245586in}{0.571603in}}%
\pgfusepath{stroke}%
\end{pgfscope}%
\begin{pgfscope}%
\pgfpathrectangle{\pgfqpoint{0.854460in}{0.571603in}}{\pgfqpoint{5.885100in}{5.068436in}}%
\pgfusepath{clip}%
\pgfsetbuttcap%
\pgfsetroundjoin%
\pgfsetlinewidth{1.505625pt}%
\definecolor{currentstroke}{rgb}{0.248629,0.278775,0.534556}%
\pgfsetstrokecolor{currentstroke}%
\pgfsetdash{}{0pt}%
\pgfpathmoveto{\pgfqpoint{2.300862in}{0.571603in}}%
\pgfpathlineto{\pgfqpoint{2.214835in}{0.640546in}}%
\pgfpathlineto{\pgfqpoint{2.145448in}{0.698951in}}%
\pgfpathlineto{\pgfqpoint{2.066968in}{0.768552in}}%
\pgfpathlineto{\pgfqpoint{2.005096in}{0.826299in}}%
\pgfpathlineto{\pgfqpoint{1.948675in}{0.881587in}}%
\pgfpathlineto{\pgfqpoint{1.878991in}{0.953646in}}%
\pgfpathlineto{\pgfqpoint{1.830381in}{1.006687in}}%
\pgfpathlineto{\pgfqpoint{1.766099in}{1.080994in}}%
\pgfpathlineto{\pgfqpoint{1.712088in}{1.147627in}}%
\pgfpathlineto{\pgfqpoint{1.665705in}{1.208341in}}%
\pgfpathlineto{\pgfqpoint{1.623368in}{1.267075in}}%
\pgfpathlineto{\pgfqpoint{1.576976in}{1.335689in}}%
\pgfpathlineto{\pgfqpoint{1.534648in}{1.402849in}}%
\pgfpathlineto{\pgfqpoint{1.499224in}{1.463037in}}%
\pgfpathlineto{\pgfqpoint{1.457630in}{1.539445in}}%
\pgfpathlineto{\pgfqpoint{1.431903in}{1.590384in}}%
\pgfpathlineto{\pgfqpoint{1.407732in}{1.641323in}}%
\pgfpathlineto{\pgfqpoint{1.374381in}{1.717732in}}%
\pgfpathlineto{\pgfqpoint{1.344379in}{1.794141in}}%
\pgfpathlineto{\pgfqpoint{1.317635in}{1.870549in}}%
\pgfpathlineto{\pgfqpoint{1.294045in}{1.946958in}}%
\pgfpathlineto{\pgfqpoint{1.273593in}{2.023366in}}%
\pgfpathlineto{\pgfqpoint{1.261653in}{2.074305in}}%
\pgfpathlineto{\pgfqpoint{1.246213in}{2.150714in}}%
\pgfpathlineto{\pgfqpoint{1.233692in}{2.227123in}}%
\pgfpathlineto{\pgfqpoint{1.224100in}{2.303531in}}%
\pgfpathlineto{\pgfqpoint{1.217308in}{2.379940in}}%
\pgfpathlineto{\pgfqpoint{1.213324in}{2.456348in}}%
\pgfpathlineto{\pgfqpoint{1.212153in}{2.532757in}}%
\pgfpathlineto{\pgfqpoint{1.213789in}{2.609165in}}%
\pgfpathlineto{\pgfqpoint{1.218224in}{2.685574in}}%
\pgfpathlineto{\pgfqpoint{1.225437in}{2.761983in}}%
\pgfpathlineto{\pgfqpoint{1.235403in}{2.838391in}}%
\pgfpathlineto{\pgfqpoint{1.248296in}{2.914800in}}%
\pgfpathlineto{\pgfqpoint{1.264009in}{2.991208in}}%
\pgfpathlineto{\pgfqpoint{1.276157in}{3.042147in}}%
\pgfpathlineto{\pgfqpoint{1.298061in}{3.122887in}}%
\pgfpathlineto{\pgfqpoint{1.320592in}{3.194965in}}%
\pgfpathlineto{\pgfqpoint{1.338197in}{3.245904in}}%
\pgfpathlineto{\pgfqpoint{1.367289in}{3.322312in}}%
\pgfpathlineto{\pgfqpoint{1.399761in}{3.398721in}}%
\pgfpathlineto{\pgfqpoint{1.423327in}{3.449660in}}%
\pgfpathlineto{\pgfqpoint{1.461784in}{3.526068in}}%
\pgfpathlineto{\pgfqpoint{1.489523in}{3.577007in}}%
\pgfpathlineto{\pgfqpoint{1.519033in}{3.627946in}}%
\pgfpathlineto{\pgfqpoint{1.550387in}{3.678886in}}%
\pgfpathlineto{\pgfqpoint{1.593795in}{3.744814in}}%
\pgfpathlineto{\pgfqpoint{1.637408in}{3.806233in}}%
\pgfpathlineto{\pgfqpoint{1.682515in}{3.865648in}}%
\pgfpathlineto{\pgfqpoint{1.738024in}{3.933581in}}%
\pgfpathlineto{\pgfqpoint{1.782622in}{3.984520in}}%
\pgfpathlineto{\pgfqpoint{1.830381in}{4.036106in}}%
\pgfpathlineto{\pgfqpoint{1.889528in}{4.095809in}}%
\pgfpathlineto{\pgfqpoint{1.948675in}{4.151606in}}%
\pgfpathlineto{\pgfqpoint{2.007822in}{4.203858in}}%
\pgfpathlineto{\pgfqpoint{2.066968in}{4.252887in}}%
\pgfpathlineto{\pgfqpoint{2.126115in}{4.298981in}}%
\pgfpathlineto{\pgfqpoint{2.185262in}{4.342401in}}%
\pgfpathlineto{\pgfqpoint{2.257714in}{4.392032in}}%
\pgfpathlineto{\pgfqpoint{2.303555in}{4.421704in}}%
\pgfpathlineto{\pgfqpoint{2.380565in}{4.468441in}}%
\pgfpathlineto{\pgfqpoint{2.425097in}{4.493910in}}%
\pgfpathlineto{\pgfqpoint{2.480996in}{4.524196in}}%
\pgfpathlineto{\pgfqpoint{2.540142in}{4.554342in}}%
\pgfpathlineto{\pgfqpoint{2.599289in}{4.582624in}}%
\pgfpathlineto{\pgfqpoint{2.658436in}{4.609080in}}%
\pgfpathlineto{\pgfqpoint{2.717582in}{4.633741in}}%
\pgfpathlineto{\pgfqpoint{2.776729in}{4.656636in}}%
\pgfpathlineto{\pgfqpoint{2.835876in}{4.677789in}}%
\pgfpathlineto{\pgfqpoint{2.896511in}{4.697667in}}%
\pgfpathlineto{\pgfqpoint{2.983743in}{4.722952in}}%
\pgfpathlineto{\pgfqpoint{3.042890in}{4.737776in}}%
\pgfpathlineto{\pgfqpoint{3.102036in}{4.750858in}}%
\pgfpathlineto{\pgfqpoint{3.161183in}{4.761905in}}%
\pgfpathlineto{\pgfqpoint{3.220330in}{4.771031in}}%
\pgfpathlineto{\pgfqpoint{3.279476in}{4.777972in}}%
\pgfpathlineto{\pgfqpoint{3.338623in}{4.782620in}}%
\pgfpathlineto{\pgfqpoint{3.397770in}{4.784844in}}%
\pgfpathlineto{\pgfqpoint{3.456917in}{4.784407in}}%
\pgfpathlineto{\pgfqpoint{3.516063in}{4.781039in}}%
\pgfpathlineto{\pgfqpoint{3.577472in}{4.774075in}}%
\pgfpathlineto{\pgfqpoint{3.634357in}{4.763918in}}%
\pgfpathlineto{\pgfqpoint{3.693504in}{4.749238in}}%
\pgfpathlineto{\pgfqpoint{3.723077in}{4.739932in}}%
\pgfpathlineto{\pgfqpoint{3.768080in}{4.723136in}}%
\pgfpathlineto{\pgfqpoint{3.782224in}{4.717168in}}%
\pgfpathlineto{\pgfqpoint{3.822770in}{4.697667in}}%
\pgfpathlineto{\pgfqpoint{3.841370in}{4.687559in}}%
\pgfpathlineto{\pgfqpoint{3.870944in}{4.669758in}}%
\pgfpathlineto{\pgfqpoint{3.904254in}{4.646728in}}%
\pgfpathlineto{\pgfqpoint{3.936265in}{4.621258in}}%
\pgfpathlineto{\pgfqpoint{3.964398in}{4.595788in}}%
\pgfpathlineto{\pgfqpoint{3.989449in}{4.570319in}}%
\pgfpathlineto{\pgfqpoint{4.018811in}{4.536316in}}%
\pgfpathlineto{\pgfqpoint{4.050572in}{4.493910in}}%
\pgfpathlineto{\pgfqpoint{4.083043in}{4.442971in}}%
\pgfpathlineto{\pgfqpoint{4.110873in}{4.392032in}}%
\pgfpathlineto{\pgfqpoint{4.137104in}{4.336514in}}%
\pgfpathlineto{\pgfqpoint{4.156342in}{4.290154in}}%
\pgfpathlineto{\pgfqpoint{4.175426in}{4.239215in}}%
\pgfpathlineto{\pgfqpoint{4.200808in}{4.162807in}}%
\pgfpathlineto{\pgfqpoint{4.225824in}{4.077342in}}%
\pgfpathlineto{\pgfqpoint{4.250214in}{3.984520in}}%
\pgfpathlineto{\pgfqpoint{4.275099in}{3.882642in}}%
\pgfpathlineto{\pgfqpoint{4.367028in}{3.500599in}}%
\pgfpathlineto{\pgfqpoint{4.403264in}{3.365873in}}%
\pgfpathlineto{\pgfqpoint{4.430674in}{3.271373in}}%
\pgfpathlineto{\pgfqpoint{4.470805in}{3.144025in}}%
\pgfpathlineto{\pgfqpoint{4.505842in}{3.042147in}}%
\pgfpathlineto{\pgfqpoint{4.543712in}{2.940269in}}%
\pgfpathlineto{\pgfqpoint{4.584560in}{2.838391in}}%
\pgfpathlineto{\pgfqpoint{4.617202in}{2.761983in}}%
\pgfpathlineto{\pgfqpoint{4.651645in}{2.685574in}}%
\pgfpathlineto{\pgfqpoint{4.700500in}{2.583696in}}%
\pgfpathlineto{\pgfqpoint{4.739268in}{2.507287in}}%
\pgfpathlineto{\pgfqpoint{4.794010in}{2.405409in}}%
\pgfpathlineto{\pgfqpoint{4.846865in}{2.312717in}}%
\pgfpathlineto{\pgfqpoint{4.882606in}{2.252592in}}%
\pgfpathlineto{\pgfqpoint{4.935585in}{2.167227in}}%
\pgfpathlineto{\pgfqpoint{4.979094in}{2.099775in}}%
\pgfpathlineto{\pgfqpoint{5.047879in}{1.997897in}}%
\pgfpathlineto{\pgfqpoint{5.120193in}{1.896019in}}%
\pgfpathlineto{\pgfqpoint{5.196066in}{1.794141in}}%
\pgfpathlineto{\pgfqpoint{5.275448in}{1.692262in}}%
\pgfpathlineto{\pgfqpoint{5.358370in}{1.590384in}}%
\pgfpathlineto{\pgfqpoint{5.444771in}{1.488506in}}%
\pgfpathlineto{\pgfqpoint{5.534601in}{1.386628in}}%
\pgfpathlineto{\pgfqpoint{5.627812in}{1.284750in}}%
\pgfpathlineto{\pgfqpoint{5.734066in}{1.172855in}}%
\pgfpathlineto{\pgfqpoint{5.824208in}{1.080994in}}%
\pgfpathlineto{\pgfqpoint{5.927151in}{0.979116in}}%
\pgfpathlineto{\pgfqpoint{6.033240in}{0.877238in}}%
\pgfpathlineto{\pgfqpoint{6.148093in}{0.769988in}}%
\pgfpathlineto{\pgfqpoint{6.266386in}{0.662457in}}%
\pgfpathlineto{\pgfqpoint{6.368678in}{0.571603in}}%
\pgfpathlineto{\pgfqpoint{6.368678in}{0.571603in}}%
\pgfusepath{stroke}%
\end{pgfscope}%
\begin{pgfscope}%
\pgfpathrectangle{\pgfqpoint{0.854460in}{0.571603in}}{\pgfqpoint{5.885100in}{5.068436in}}%
\pgfusepath{clip}%
\pgfsetbuttcap%
\pgfsetroundjoin%
\pgfsetlinewidth{1.505625pt}%
\definecolor{currentstroke}{rgb}{0.239346,0.300855,0.540844}%
\pgfsetstrokecolor{currentstroke}%
\pgfsetdash{}{0pt}%
\pgfpathmoveto{\pgfqpoint{2.199834in}{0.571603in}}%
\pgfpathlineto{\pgfqpoint{2.126115in}{0.631384in}}%
\pgfpathlineto{\pgfqpoint{2.046762in}{0.698951in}}%
\pgfpathlineto{\pgfqpoint{1.978248in}{0.760437in}}%
\pgfpathlineto{\pgfqpoint{1.908475in}{0.826299in}}%
\pgfpathlineto{\pgfqpoint{1.857047in}{0.877238in}}%
\pgfpathlineto{\pgfqpoint{1.800808in}{0.935659in}}%
\pgfpathlineto{\pgfqpoint{1.737889in}{1.004585in}}%
\pgfpathlineto{\pgfqpoint{1.682515in}{1.069018in}}%
\pgfpathlineto{\pgfqpoint{1.631389in}{1.131933in}}%
\pgfpathlineto{\pgfqpoint{1.592106in}{1.182872in}}%
\pgfpathlineto{\pgfqpoint{1.536696in}{1.259281in}}%
\pgfpathlineto{\pgfqpoint{1.502001in}{1.310220in}}%
\pgfpathlineto{\pgfqpoint{1.453222in}{1.386628in}}%
\pgfpathlineto{\pgfqpoint{1.416354in}{1.448740in}}%
\pgfpathlineto{\pgfqpoint{1.380218in}{1.513976in}}%
\pgfpathlineto{\pgfqpoint{1.341235in}{1.590384in}}%
\pgfpathlineto{\pgfqpoint{1.317154in}{1.641323in}}%
\pgfpathlineto{\pgfqpoint{1.283855in}{1.717732in}}%
\pgfpathlineto{\pgfqpoint{1.253832in}{1.794141in}}%
\pgfpathlineto{\pgfqpoint{1.226994in}{1.870549in}}%
\pgfpathlineto{\pgfqpoint{1.203240in}{1.946958in}}%
\pgfpathlineto{\pgfqpoint{1.182508in}{2.023366in}}%
\pgfpathlineto{\pgfqpoint{1.170386in}{2.074305in}}%
\pgfpathlineto{\pgfqpoint{1.154541in}{2.150714in}}%
\pgfpathlineto{\pgfqpoint{1.141602in}{2.227123in}}%
\pgfpathlineto{\pgfqpoint{1.131455in}{2.303531in}}%
\pgfpathlineto{\pgfqpoint{1.124030in}{2.379940in}}%
\pgfpathlineto{\pgfqpoint{1.119360in}{2.456348in}}%
\pgfpathlineto{\pgfqpoint{1.117439in}{2.532757in}}%
\pgfpathlineto{\pgfqpoint{1.118192in}{2.609165in}}%
\pgfpathlineto{\pgfqpoint{1.121631in}{2.685574in}}%
\pgfpathlineto{\pgfqpoint{1.127827in}{2.761983in}}%
\pgfpathlineto{\pgfqpoint{1.136707in}{2.838391in}}%
\pgfpathlineto{\pgfqpoint{1.148241in}{2.914800in}}%
\pgfpathlineto{\pgfqpoint{1.157546in}{2.965739in}}%
\pgfpathlineto{\pgfqpoint{1.173771in}{3.042147in}}%
\pgfpathlineto{\pgfqpoint{1.186178in}{3.093086in}}%
\pgfpathlineto{\pgfqpoint{1.209341in}{3.176792in}}%
\pgfpathlineto{\pgfqpoint{1.231212in}{3.245904in}}%
\pgfpathlineto{\pgfqpoint{1.248919in}{3.296843in}}%
\pgfpathlineto{\pgfqpoint{1.278088in}{3.373251in}}%
\pgfpathlineto{\pgfqpoint{1.310523in}{3.449660in}}%
\pgfpathlineto{\pgfqpoint{1.333993in}{3.500599in}}%
\pgfpathlineto{\pgfqpoint{1.372200in}{3.577007in}}%
\pgfpathlineto{\pgfqpoint{1.399685in}{3.627946in}}%
\pgfpathlineto{\pgfqpoint{1.428867in}{3.678886in}}%
\pgfpathlineto{\pgfqpoint{1.459814in}{3.729825in}}%
\pgfpathlineto{\pgfqpoint{1.492593in}{3.780764in}}%
\pgfpathlineto{\pgfqpoint{1.534648in}{3.842202in}}%
\pgfpathlineto{\pgfqpoint{1.583121in}{3.908111in}}%
\pgfpathlineto{\pgfqpoint{1.623368in}{3.959583in}}%
\pgfpathlineto{\pgfqpoint{1.682515in}{4.030171in}}%
\pgfpathlineto{\pgfqpoint{1.733009in}{4.086398in}}%
\pgfpathlineto{\pgfqpoint{1.781493in}{4.137337in}}%
\pgfpathlineto{\pgfqpoint{1.832725in}{4.188276in}}%
\pgfpathlineto{\pgfqpoint{1.889528in}{4.241483in}}%
\pgfpathlineto{\pgfqpoint{1.948675in}{4.293624in}}%
\pgfpathlineto{\pgfqpoint{2.007822in}{4.342777in}}%
\pgfpathlineto{\pgfqpoint{2.070772in}{4.392032in}}%
\pgfpathlineto{\pgfqpoint{2.140132in}{4.442971in}}%
\pgfpathlineto{\pgfqpoint{2.185262in}{4.474444in}}%
\pgfpathlineto{\pgfqpoint{2.253239in}{4.519380in}}%
\pgfpathlineto{\pgfqpoint{2.333129in}{4.568798in}}%
\pgfpathlineto{\pgfqpoint{2.392275in}{4.603044in}}%
\pgfpathlineto{\pgfqpoint{2.472648in}{4.646728in}}%
\pgfpathlineto{\pgfqpoint{2.540142in}{4.680940in}}%
\pgfpathlineto{\pgfqpoint{2.599289in}{4.709142in}}%
\pgfpathlineto{\pgfqpoint{2.658436in}{4.735739in}}%
\pgfpathlineto{\pgfqpoint{2.717582in}{4.760764in}}%
\pgfpathlineto{\pgfqpoint{2.776729in}{4.784247in}}%
\pgfpathlineto{\pgfqpoint{2.835876in}{4.806215in}}%
\pgfpathlineto{\pgfqpoint{2.924596in}{4.836287in}}%
\pgfpathlineto{\pgfqpoint{2.983743in}{4.854483in}}%
\pgfpathlineto{\pgfqpoint{3.072463in}{4.878854in}}%
\pgfpathlineto{\pgfqpoint{3.161183in}{4.899683in}}%
\pgfpathlineto{\pgfqpoint{3.220330in}{4.911462in}}%
\pgfpathlineto{\pgfqpoint{3.279476in}{4.921616in}}%
\pgfpathlineto{\pgfqpoint{3.338623in}{4.929977in}}%
\pgfpathlineto{\pgfqpoint{3.397770in}{4.936389in}}%
\pgfpathlineto{\pgfqpoint{3.456917in}{4.940826in}}%
\pgfpathlineto{\pgfqpoint{3.516063in}{4.943102in}}%
\pgfpathlineto{\pgfqpoint{3.575210in}{4.943010in}}%
\pgfpathlineto{\pgfqpoint{3.634357in}{4.940307in}}%
\pgfpathlineto{\pgfqpoint{3.693504in}{4.934713in}}%
\pgfpathlineto{\pgfqpoint{3.752650in}{4.925877in}}%
\pgfpathlineto{\pgfqpoint{3.811797in}{4.913099in}}%
\pgfpathlineto{\pgfqpoint{3.853806in}{4.901423in}}%
\pgfpathlineto{\pgfqpoint{3.900517in}{4.885433in}}%
\pgfpathlineto{\pgfqpoint{3.930090in}{4.873507in}}%
\pgfpathlineto{\pgfqpoint{3.977959in}{4.850484in}}%
\pgfpathlineto{\pgfqpoint{3.989237in}{4.844388in}}%
\pgfpathlineto{\pgfqpoint{4.021760in}{4.825014in}}%
\pgfpathlineto{\pgfqpoint{4.058440in}{4.799545in}}%
\pgfpathlineto{\pgfqpoint{4.090096in}{4.774075in}}%
\pgfpathlineto{\pgfqpoint{4.117805in}{4.748606in}}%
\pgfpathlineto{\pgfqpoint{4.142367in}{4.723136in}}%
\pgfpathlineto{\pgfqpoint{4.166677in}{4.694740in}}%
\pgfpathlineto{\pgfqpoint{4.196251in}{4.654960in}}%
\pgfpathlineto{\pgfqpoint{4.218060in}{4.621258in}}%
\pgfpathlineto{\pgfqpoint{4.232928in}{4.595788in}}%
\pgfpathlineto{\pgfqpoint{4.259187in}{4.544849in}}%
\pgfpathlineto{\pgfqpoint{4.284971in}{4.485642in}}%
\pgfpathlineto{\pgfqpoint{4.301007in}{4.442971in}}%
\pgfpathlineto{\pgfqpoint{4.318101in}{4.392032in}}%
\pgfpathlineto{\pgfqpoint{4.340179in}{4.315624in}}%
\pgfpathlineto{\pgfqpoint{4.353074in}{4.264685in}}%
\pgfpathlineto{\pgfqpoint{4.373691in}{4.173452in}}%
\pgfpathlineto{\pgfqpoint{4.395967in}{4.060928in}}%
\pgfpathlineto{\pgfqpoint{4.423776in}{3.908111in}}%
\pgfpathlineto{\pgfqpoint{4.466010in}{3.678886in}}%
\pgfpathlineto{\pgfqpoint{4.491984in}{3.551359in}}%
\pgfpathlineto{\pgfqpoint{4.514644in}{3.449660in}}%
\pgfpathlineto{\pgfqpoint{4.539528in}{3.347782in}}%
\pgfpathlineto{\pgfqpoint{4.566859in}{3.245904in}}%
\pgfpathlineto{\pgfqpoint{4.596864in}{3.144025in}}%
\pgfpathlineto{\pgfqpoint{4.629743in}{3.042147in}}%
\pgfpathlineto{\pgfqpoint{4.665675in}{2.940269in}}%
\pgfpathlineto{\pgfqpoint{4.704746in}{2.838391in}}%
\pgfpathlineto{\pgfqpoint{4.736204in}{2.761983in}}%
\pgfpathlineto{\pgfqpoint{4.769553in}{2.685574in}}%
\pgfpathlineto{\pgfqpoint{4.804840in}{2.609165in}}%
\pgfpathlineto{\pgfqpoint{4.854954in}{2.507287in}}%
\pgfpathlineto{\pgfqpoint{4.894848in}{2.430879in}}%
\pgfpathlineto{\pgfqpoint{4.951184in}{2.329001in}}%
\pgfpathlineto{\pgfqpoint{5.011142in}{2.227123in}}%
\pgfpathlineto{\pgfqpoint{5.074752in}{2.125244in}}%
\pgfpathlineto{\pgfqpoint{5.142599in}{2.022568in}}%
\pgfpathlineto{\pgfqpoint{5.201745in}{1.937307in}}%
\pgfpathlineto{\pgfqpoint{5.260892in}{1.855507in}}%
\pgfpathlineto{\pgfqpoint{5.320039in}{1.776784in}}%
\pgfpathlineto{\pgfqpoint{5.379185in}{1.700812in}}%
\pgfpathlineto{\pgfqpoint{5.438332in}{1.627308in}}%
\pgfpathlineto{\pgfqpoint{5.497479in}{1.556034in}}%
\pgfpathlineto{\pgfqpoint{5.556626in}{1.486786in}}%
\pgfpathlineto{\pgfqpoint{5.645083in}{1.386628in}}%
\pgfpathlineto{\pgfqpoint{5.738526in}{1.284750in}}%
\pgfpathlineto{\pgfqpoint{5.835431in}{1.182872in}}%
\pgfpathlineto{\pgfqpoint{5.941079in}{1.075714in}}%
\pgfpathlineto{\pgfqpoint{6.039406in}{0.979116in}}%
\pgfpathlineto{\pgfqpoint{6.148093in}{0.875578in}}%
\pgfpathlineto{\pgfqpoint{6.266386in}{0.766185in}}%
\pgfpathlineto{\pgfqpoint{6.384680in}{0.659889in}}%
\pgfpathlineto{\pgfqpoint{6.485327in}{0.571603in}}%
\pgfpathlineto{\pgfqpoint{6.485327in}{0.571603in}}%
\pgfusepath{stroke}%
\end{pgfscope}%
\begin{pgfscope}%
\pgfpathrectangle{\pgfqpoint{0.854460in}{0.571603in}}{\pgfqpoint{5.885100in}{5.068436in}}%
\pgfusepath{clip}%
\pgfsetbuttcap%
\pgfsetroundjoin%
\pgfsetlinewidth{1.505625pt}%
\definecolor{currentstroke}{rgb}{0.227802,0.326594,0.546532}%
\pgfsetstrokecolor{currentstroke}%
\pgfsetdash{}{0pt}%
\pgfpathmoveto{\pgfqpoint{2.104286in}{0.571603in}}%
\pgfpathlineto{\pgfqpoint{2.037395in}{0.626521in}}%
\pgfpathlineto{\pgfqpoint{1.978248in}{0.677095in}}%
\pgfpathlineto{\pgfqpoint{1.919102in}{0.729783in}}%
\pgfpathlineto{\pgfqpoint{1.843163in}{0.800829in}}%
\pgfpathlineto{\pgfqpoint{1.771235in}{0.872141in}}%
\pgfpathlineto{\pgfqpoint{1.717617in}{0.928177in}}%
\pgfpathlineto{\pgfqpoint{1.671078in}{0.979116in}}%
\pgfpathlineto{\pgfqpoint{1.623368in}{1.033725in}}%
\pgfpathlineto{\pgfqpoint{1.563287in}{1.106463in}}%
\pgfpathlineto{\pgfqpoint{1.504338in}{1.182872in}}%
\pgfpathlineto{\pgfqpoint{1.449491in}{1.259281in}}%
\pgfpathlineto{\pgfqpoint{1.415088in}{1.310220in}}%
\pgfpathlineto{\pgfqpoint{1.366741in}{1.386628in}}%
\pgfpathlineto{\pgfqpoint{1.327634in}{1.453142in}}%
\pgfpathlineto{\pgfqpoint{1.294223in}{1.513976in}}%
\pgfpathlineto{\pgfqpoint{1.255463in}{1.590384in}}%
\pgfpathlineto{\pgfqpoint{1.220046in}{1.666793in}}%
\pgfpathlineto{\pgfqpoint{1.198266in}{1.717732in}}%
\pgfpathlineto{\pgfqpoint{1.168264in}{1.794141in}}%
\pgfpathlineto{\pgfqpoint{1.141372in}{1.870549in}}%
\pgfpathlineto{\pgfqpoint{1.117491in}{1.946958in}}%
\pgfpathlineto{\pgfqpoint{1.096614in}{2.023366in}}%
\pgfpathlineto{\pgfqpoint{1.078636in}{2.099775in}}%
\pgfpathlineto{\pgfqpoint{1.063407in}{2.176183in}}%
\pgfpathlineto{\pgfqpoint{1.051049in}{2.252592in}}%
\pgfpathlineto{\pgfqpoint{1.041366in}{2.329001in}}%
\pgfpathlineto{\pgfqpoint{1.034333in}{2.405409in}}%
\pgfpathlineto{\pgfqpoint{1.029993in}{2.481818in}}%
\pgfpathlineto{\pgfqpoint{1.028305in}{2.558226in}}%
\pgfpathlineto{\pgfqpoint{1.029217in}{2.634635in}}%
\pgfpathlineto{\pgfqpoint{1.032734in}{2.711044in}}%
\pgfpathlineto{\pgfqpoint{1.038933in}{2.787452in}}%
\pgfpathlineto{\pgfqpoint{1.047737in}{2.863861in}}%
\pgfpathlineto{\pgfqpoint{1.059120in}{2.940269in}}%
\pgfpathlineto{\pgfqpoint{1.073293in}{3.016678in}}%
\pgfpathlineto{\pgfqpoint{1.091047in}{3.097090in}}%
\pgfpathlineto{\pgfqpoint{1.109776in}{3.169495in}}%
\pgfpathlineto{\pgfqpoint{1.132262in}{3.245904in}}%
\pgfpathlineto{\pgfqpoint{1.157660in}{3.322312in}}%
\pgfpathlineto{\pgfqpoint{1.186092in}{3.398721in}}%
\pgfpathlineto{\pgfqpoint{1.217675in}{3.475129in}}%
\pgfpathlineto{\pgfqpoint{1.252523in}{3.551538in}}%
\pgfpathlineto{\pgfqpoint{1.277617in}{3.602477in}}%
\pgfpathlineto{\pgfqpoint{1.318222in}{3.678886in}}%
\pgfpathlineto{\pgfqpoint{1.357208in}{3.746509in}}%
\pgfpathlineto{\pgfqpoint{1.394215in}{3.806233in}}%
\pgfpathlineto{\pgfqpoint{1.427737in}{3.857172in}}%
\pgfpathlineto{\pgfqpoint{1.475501in}{3.925339in}}%
\pgfpathlineto{\pgfqpoint{1.519941in}{3.984520in}}%
\pgfpathlineto{\pgfqpoint{1.564221in}{4.040151in}}%
\pgfpathlineto{\pgfqpoint{1.625315in}{4.111867in}}%
\pgfpathlineto{\pgfqpoint{1.682515in}{4.174390in}}%
\pgfpathlineto{\pgfqpoint{1.741661in}{4.235043in}}%
\pgfpathlineto{\pgfqpoint{1.800808in}{4.292002in}}%
\pgfpathlineto{\pgfqpoint{1.859955in}{4.345618in}}%
\pgfpathlineto{\pgfqpoint{1.919102in}{4.396265in}}%
\pgfpathlineto{\pgfqpoint{1.978248in}{4.444205in}}%
\pgfpathlineto{\pgfqpoint{2.043210in}{4.493910in}}%
\pgfpathlineto{\pgfqpoint{2.113884in}{4.544849in}}%
\pgfpathlineto{\pgfqpoint{2.185262in}{4.593298in}}%
\pgfpathlineto{\pgfqpoint{2.244409in}{4.631190in}}%
\pgfpathlineto{\pgfqpoint{2.311840in}{4.672197in}}%
\pgfpathlineto{\pgfqpoint{2.392275in}{4.718163in}}%
\pgfpathlineto{\pgfqpoint{2.451422in}{4.750047in}}%
\pgfpathlineto{\pgfqpoint{2.540142in}{4.794887in}}%
\pgfpathlineto{\pgfqpoint{2.603777in}{4.825014in}}%
\pgfpathlineto{\pgfqpoint{2.688009in}{4.862294in}}%
\pgfpathlineto{\pgfqpoint{2.776729in}{4.898607in}}%
\pgfpathlineto{\pgfqpoint{2.851808in}{4.926892in}}%
\pgfpathlineto{\pgfqpoint{2.924607in}{4.952362in}}%
\pgfpathlineto{\pgfqpoint{3.013316in}{4.980581in}}%
\pgfpathlineto{\pgfqpoint{3.102036in}{5.005839in}}%
\pgfpathlineto{\pgfqpoint{3.193847in}{5.028770in}}%
\pgfpathlineto{\pgfqpoint{3.279476in}{5.047071in}}%
\pgfpathlineto{\pgfqpoint{3.338623in}{5.057983in}}%
\pgfpathlineto{\pgfqpoint{3.427343in}{5.071405in}}%
\pgfpathlineto{\pgfqpoint{3.500161in}{5.079709in}}%
\pgfpathlineto{\pgfqpoint{3.545637in}{5.083564in}}%
\pgfpathlineto{\pgfqpoint{3.604783in}{5.086865in}}%
\pgfpathlineto{\pgfqpoint{3.663930in}{5.088166in}}%
\pgfpathlineto{\pgfqpoint{3.723077in}{5.087271in}}%
\pgfpathlineto{\pgfqpoint{3.782224in}{5.083956in}}%
\pgfpathlineto{\pgfqpoint{3.841370in}{5.077906in}}%
\pgfpathlineto{\pgfqpoint{3.900517in}{5.068645in}}%
\pgfpathlineto{\pgfqpoint{3.959664in}{5.055929in}}%
\pgfpathlineto{\pgfqpoint{3.989237in}{5.047941in}}%
\pgfpathlineto{\pgfqpoint{4.048384in}{5.028555in}}%
\pgfpathlineto{\pgfqpoint{4.077957in}{5.016674in}}%
\pgfpathlineto{\pgfqpoint{4.107787in}{5.003301in}}%
\pgfpathlineto{\pgfqpoint{4.155426in}{4.977831in}}%
\pgfpathlineto{\pgfqpoint{4.166677in}{4.971066in}}%
\pgfpathlineto{\pgfqpoint{4.196251in}{4.951678in}}%
\pgfpathlineto{\pgfqpoint{4.229038in}{4.926892in}}%
\pgfpathlineto{\pgfqpoint{4.258340in}{4.901423in}}%
\pgfpathlineto{\pgfqpoint{4.284971in}{4.875012in}}%
\pgfpathlineto{\pgfqpoint{4.314544in}{4.841084in}}%
\pgfpathlineto{\pgfqpoint{4.344118in}{4.801478in}}%
\pgfpathlineto{\pgfqpoint{4.361842in}{4.774075in}}%
\pgfpathlineto{\pgfqpoint{4.376852in}{4.748606in}}%
\pgfpathlineto{\pgfqpoint{4.403264in}{4.696895in}}%
\pgfpathlineto{\pgfqpoint{4.424655in}{4.646728in}}%
\pgfpathlineto{\pgfqpoint{4.443142in}{4.595788in}}%
\pgfpathlineto{\pgfqpoint{4.462411in}{4.532870in}}%
\pgfpathlineto{\pgfqpoint{4.472658in}{4.493910in}}%
\pgfpathlineto{\pgfqpoint{4.490084in}{4.417502in}}%
\pgfpathlineto{\pgfqpoint{4.499991in}{4.366563in}}%
\pgfpathlineto{\pgfqpoint{4.513101in}{4.290154in}}%
\pgfpathlineto{\pgfqpoint{4.528253in}{4.188276in}}%
\pgfpathlineto{\pgfqpoint{4.551739in}{4.009989in}}%
\pgfpathlineto{\pgfqpoint{4.575227in}{3.831703in}}%
\pgfpathlineto{\pgfqpoint{4.593905in}{3.704355in}}%
\pgfpathlineto{\pgfqpoint{4.610752in}{3.602477in}}%
\pgfpathlineto{\pgfqpoint{4.629524in}{3.500599in}}%
\pgfpathlineto{\pgfqpoint{4.650674in}{3.398721in}}%
\pgfpathlineto{\pgfqpoint{4.674431in}{3.296843in}}%
\pgfpathlineto{\pgfqpoint{4.698998in}{3.202361in}}%
\pgfpathlineto{\pgfqpoint{4.715330in}{3.144025in}}%
\pgfpathlineto{\pgfqpoint{4.746413in}{3.042147in}}%
\pgfpathlineto{\pgfqpoint{4.780720in}{2.940269in}}%
\pgfpathlineto{\pgfqpoint{4.818378in}{2.838391in}}%
\pgfpathlineto{\pgfqpoint{4.848845in}{2.761983in}}%
\pgfpathlineto{\pgfqpoint{4.881272in}{2.685574in}}%
\pgfpathlineto{\pgfqpoint{4.915700in}{2.609165in}}%
\pgfpathlineto{\pgfqpoint{4.952168in}{2.532757in}}%
\pgfpathlineto{\pgfqpoint{5.004000in}{2.430879in}}%
\pgfpathlineto{\pgfqpoint{5.053878in}{2.339160in}}%
\pgfpathlineto{\pgfqpoint{5.088685in}{2.278062in}}%
\pgfpathlineto{\pgfqpoint{5.142599in}{2.187948in}}%
\pgfpathlineto{\pgfqpoint{5.181750in}{2.125244in}}%
\pgfpathlineto{\pgfqpoint{5.231500in}{2.048836in}}%
\pgfpathlineto{\pgfqpoint{5.290465in}{1.962171in}}%
\pgfpathlineto{\pgfqpoint{5.349612in}{1.878979in}}%
\pgfpathlineto{\pgfqpoint{5.393299in}{1.819610in}}%
\pgfpathlineto{\pgfqpoint{5.471361in}{1.717732in}}%
\pgfpathlineto{\pgfqpoint{5.532333in}{1.641323in}}%
\pgfpathlineto{\pgfqpoint{5.616910in}{1.539445in}}%
\pgfpathlineto{\pgfqpoint{5.704492in}{1.438323in}}%
\pgfpathlineto{\pgfqpoint{5.773700in}{1.361159in}}%
\pgfpathlineto{\pgfqpoint{5.868295in}{1.259281in}}%
\pgfpathlineto{\pgfqpoint{5.970653in}{1.153167in}}%
\pgfpathlineto{\pgfqpoint{6.068131in}{1.055524in}}%
\pgfpathlineto{\pgfqpoint{6.177666in}{0.949442in}}%
\pgfpathlineto{\pgfqpoint{6.281681in}{0.851768in}}%
\pgfpathlineto{\pgfqpoint{6.393411in}{0.749890in}}%
\pgfpathlineto{\pgfqpoint{6.508309in}{0.648012in}}%
\pgfpathlineto{\pgfqpoint{6.596482in}{0.571603in}}%
\pgfpathlineto{\pgfqpoint{6.596482in}{0.571603in}}%
\pgfusepath{stroke}%
\end{pgfscope}%
\begin{pgfscope}%
\pgfpathrectangle{\pgfqpoint{0.854460in}{0.571603in}}{\pgfqpoint{5.885100in}{5.068436in}}%
\pgfusepath{clip}%
\pgfsetbuttcap%
\pgfsetroundjoin%
\pgfsetlinewidth{1.505625pt}%
\definecolor{currentstroke}{rgb}{0.218130,0.347432,0.550038}%
\pgfsetstrokecolor{currentstroke}%
\pgfsetdash{}{0pt}%
\pgfpathmoveto{\pgfqpoint{2.013550in}{0.571603in}}%
\pgfpathlineto{\pgfqpoint{1.948675in}{0.625575in}}%
\pgfpathlineto{\pgfqpoint{1.889528in}{0.676811in}}%
\pgfpathlineto{\pgfqpoint{1.809166in}{0.749890in}}%
\pgfpathlineto{\pgfqpoint{1.741661in}{0.814778in}}%
\pgfpathlineto{\pgfqpoint{1.679854in}{0.877238in}}%
\pgfpathlineto{\pgfqpoint{1.623368in}{0.937421in}}%
\pgfpathlineto{\pgfqpoint{1.563505in}{1.004585in}}%
\pgfpathlineto{\pgfqpoint{1.499613in}{1.080994in}}%
\pgfpathlineto{\pgfqpoint{1.439902in}{1.157402in}}%
\pgfpathlineto{\pgfqpoint{1.384219in}{1.233811in}}%
\pgfpathlineto{\pgfqpoint{1.332476in}{1.310220in}}%
\pgfpathlineto{\pgfqpoint{1.298061in}{1.364395in}}%
\pgfpathlineto{\pgfqpoint{1.254544in}{1.437567in}}%
\pgfpathlineto{\pgfqpoint{1.226152in}{1.488506in}}%
\pgfpathlineto{\pgfqpoint{1.186426in}{1.564915in}}%
\pgfpathlineto{\pgfqpoint{1.161842in}{1.615854in}}%
\pgfpathlineto{\pgfqpoint{1.138700in}{1.666793in}}%
\pgfpathlineto{\pgfqpoint{1.106659in}{1.743202in}}%
\pgfpathlineto{\pgfqpoint{1.077734in}{1.819610in}}%
\pgfpathlineto{\pgfqpoint{1.051830in}{1.896019in}}%
\pgfpathlineto{\pgfqpoint{1.028848in}{1.972427in}}%
\pgfpathlineto{\pgfqpoint{1.008794in}{2.048836in}}%
\pgfpathlineto{\pgfqpoint{0.991544in}{2.125244in}}%
\pgfpathlineto{\pgfqpoint{0.976994in}{2.201653in}}%
\pgfpathlineto{\pgfqpoint{0.965190in}{2.278062in}}%
\pgfpathlineto{\pgfqpoint{0.956035in}{2.354470in}}%
\pgfpathlineto{\pgfqpoint{0.949458in}{2.430879in}}%
\pgfpathlineto{\pgfqpoint{0.945462in}{2.507287in}}%
\pgfpathlineto{\pgfqpoint{0.944046in}{2.583696in}}%
\pgfpathlineto{\pgfqpoint{0.945205in}{2.660104in}}%
\pgfpathlineto{\pgfqpoint{0.948927in}{2.736513in}}%
\pgfpathlineto{\pgfqpoint{0.955198in}{2.812922in}}%
\pgfpathlineto{\pgfqpoint{0.963998in}{2.889330in}}%
\pgfpathlineto{\pgfqpoint{0.975354in}{2.965739in}}%
\pgfpathlineto{\pgfqpoint{0.989412in}{3.042147in}}%
\pgfpathlineto{\pgfqpoint{1.006036in}{3.118556in}}%
\pgfpathlineto{\pgfqpoint{1.025427in}{3.194965in}}%
\pgfpathlineto{\pgfqpoint{1.047595in}{3.271373in}}%
\pgfpathlineto{\pgfqpoint{1.072580in}{3.347782in}}%
\pgfpathlineto{\pgfqpoint{1.100498in}{3.424190in}}%
\pgfpathlineto{\pgfqpoint{1.131463in}{3.500599in}}%
\pgfpathlineto{\pgfqpoint{1.165583in}{3.577007in}}%
\pgfpathlineto{\pgfqpoint{1.190138in}{3.627946in}}%
\pgfpathlineto{\pgfqpoint{1.229814in}{3.704355in}}%
\pgfpathlineto{\pgfqpoint{1.268488in}{3.773045in}}%
\pgfpathlineto{\pgfqpoint{1.303931in}{3.831703in}}%
\pgfpathlineto{\pgfqpoint{1.353521in}{3.908111in}}%
\pgfpathlineto{\pgfqpoint{1.388855in}{3.959050in}}%
\pgfpathlineto{\pgfqpoint{1.445928in}{4.036048in}}%
\pgfpathlineto{\pgfqpoint{1.506726in}{4.111867in}}%
\pgfpathlineto{\pgfqpoint{1.564221in}{4.178531in}}%
\pgfpathlineto{\pgfqpoint{1.619995in}{4.239215in}}%
\pgfpathlineto{\pgfqpoint{1.669622in}{4.290154in}}%
\pgfpathlineto{\pgfqpoint{1.721849in}{4.341093in}}%
\pgfpathlineto{\pgfqpoint{1.776889in}{4.392032in}}%
\pgfpathlineto{\pgfqpoint{1.834963in}{4.442971in}}%
\pgfpathlineto{\pgfqpoint{1.896301in}{4.493910in}}%
\pgfpathlineto{\pgfqpoint{1.961135in}{4.544849in}}%
\pgfpathlineto{\pgfqpoint{2.029706in}{4.595788in}}%
\pgfpathlineto{\pgfqpoint{2.096542in}{4.642736in}}%
\pgfpathlineto{\pgfqpoint{2.155689in}{4.682169in}}%
\pgfpathlineto{\pgfqpoint{2.220144in}{4.723136in}}%
\pgfpathlineto{\pgfqpoint{2.305151in}{4.774075in}}%
\pgfpathlineto{\pgfqpoint{2.396281in}{4.825014in}}%
\pgfpathlineto{\pgfqpoint{2.480996in}{4.869213in}}%
\pgfpathlineto{\pgfqpoint{2.546417in}{4.901423in}}%
\pgfpathlineto{\pgfqpoint{2.628862in}{4.939639in}}%
\pgfpathlineto{\pgfqpoint{2.717582in}{4.978090in}}%
\pgfpathlineto{\pgfqpoint{2.806303in}{5.013622in}}%
\pgfpathlineto{\pgfqpoint{2.895023in}{5.046528in}}%
\pgfpathlineto{\pgfqpoint{2.983743in}{5.076803in}}%
\pgfpathlineto{\pgfqpoint{3.075024in}{5.105179in}}%
\pgfpathlineto{\pgfqpoint{3.165999in}{5.130649in}}%
\pgfpathlineto{\pgfqpoint{3.249903in}{5.151631in}}%
\pgfpathlineto{\pgfqpoint{3.338623in}{5.171121in}}%
\pgfpathlineto{\pgfqpoint{3.427343in}{5.187793in}}%
\pgfpathlineto{\pgfqpoint{3.516063in}{5.201424in}}%
\pgfpathlineto{\pgfqpoint{3.575210in}{5.208760in}}%
\pgfpathlineto{\pgfqpoint{3.634357in}{5.214484in}}%
\pgfpathlineto{\pgfqpoint{3.693504in}{5.218619in}}%
\pgfpathlineto{\pgfqpoint{3.752650in}{5.221024in}}%
\pgfpathlineto{\pgfqpoint{3.811797in}{5.221534in}}%
\pgfpathlineto{\pgfqpoint{3.870944in}{5.219963in}}%
\pgfpathlineto{\pgfqpoint{3.930090in}{5.216096in}}%
\pgfpathlineto{\pgfqpoint{3.989237in}{5.209683in}}%
\pgfpathlineto{\pgfqpoint{4.048384in}{5.200219in}}%
\pgfpathlineto{\pgfqpoint{4.107531in}{5.187378in}}%
\pgfpathlineto{\pgfqpoint{4.137104in}{5.179525in}}%
\pgfpathlineto{\pgfqpoint{4.196251in}{5.160270in}}%
\pgfpathlineto{\pgfqpoint{4.225824in}{5.148659in}}%
\pgfpathlineto{\pgfqpoint{4.265634in}{5.130649in}}%
\pgfpathlineto{\pgfqpoint{4.284971in}{5.120731in}}%
\pgfpathlineto{\pgfqpoint{4.314544in}{5.104053in}}%
\pgfpathlineto{\pgfqpoint{4.351651in}{5.079709in}}%
\pgfpathlineto{\pgfqpoint{4.384920in}{5.054240in}}%
\pgfpathlineto{\pgfqpoint{4.413745in}{5.028770in}}%
\pgfpathlineto{\pgfqpoint{4.439011in}{5.003301in}}%
\pgfpathlineto{\pgfqpoint{4.462411in}{4.976540in}}%
\pgfpathlineto{\pgfqpoint{4.491984in}{4.937024in}}%
\pgfpathlineto{\pgfqpoint{4.514721in}{4.901423in}}%
\pgfpathlineto{\pgfqpoint{4.529049in}{4.875953in}}%
\pgfpathlineto{\pgfqpoint{4.553787in}{4.825014in}}%
\pgfpathlineto{\pgfqpoint{4.574124in}{4.774075in}}%
\pgfpathlineto{\pgfqpoint{4.591037in}{4.723136in}}%
\pgfpathlineto{\pgfqpoint{4.605228in}{4.672197in}}%
\pgfpathlineto{\pgfqpoint{4.617135in}{4.621258in}}%
\pgfpathlineto{\pgfqpoint{4.627209in}{4.570319in}}%
\pgfpathlineto{\pgfqpoint{4.639851in}{4.493070in}}%
\pgfpathlineto{\pgfqpoint{4.649695in}{4.417502in}}%
\pgfpathlineto{\pgfqpoint{4.660514in}{4.315624in}}%
\pgfpathlineto{\pgfqpoint{4.673836in}{4.162807in}}%
\pgfpathlineto{\pgfqpoint{4.695557in}{3.908111in}}%
\pgfpathlineto{\pgfqpoint{4.708738in}{3.780764in}}%
\pgfpathlineto{\pgfqpoint{4.721254in}{3.678886in}}%
\pgfpathlineto{\pgfqpoint{4.735866in}{3.577007in}}%
\pgfpathlineto{\pgfqpoint{4.752885in}{3.475129in}}%
\pgfpathlineto{\pgfqpoint{4.772520in}{3.373251in}}%
\pgfpathlineto{\pgfqpoint{4.795050in}{3.271373in}}%
\pgfpathlineto{\pgfqpoint{4.817291in}{3.182280in}}%
\pgfpathlineto{\pgfqpoint{4.834564in}{3.118556in}}%
\pgfpathlineto{\pgfqpoint{4.857055in}{3.042147in}}%
\pgfpathlineto{\pgfqpoint{4.881449in}{2.965739in}}%
\pgfpathlineto{\pgfqpoint{4.907779in}{2.889330in}}%
\pgfpathlineto{\pgfqpoint{4.936077in}{2.812922in}}%
\pgfpathlineto{\pgfqpoint{4.966376in}{2.736513in}}%
\pgfpathlineto{\pgfqpoint{4.998709in}{2.660104in}}%
\pgfpathlineto{\pgfqpoint{5.033111in}{2.583696in}}%
\pgfpathlineto{\pgfqpoint{5.069618in}{2.507287in}}%
\pgfpathlineto{\pgfqpoint{5.113025in}{2.421837in}}%
\pgfpathlineto{\pgfqpoint{5.149017in}{2.354470in}}%
\pgfpathlineto{\pgfqpoint{5.201745in}{2.261167in}}%
\pgfpathlineto{\pgfqpoint{5.236975in}{2.201653in}}%
\pgfpathlineto{\pgfqpoint{5.290465in}{2.115418in}}%
\pgfpathlineto{\pgfqpoint{5.333553in}{2.048836in}}%
\pgfpathlineto{\pgfqpoint{5.402781in}{1.946958in}}%
\pgfpathlineto{\pgfqpoint{5.475839in}{1.845080in}}%
\pgfpathlineto{\pgfqpoint{5.552762in}{1.743202in}}%
\pgfpathlineto{\pgfqpoint{5.633479in}{1.641323in}}%
\pgfpathlineto{\pgfqpoint{5.718023in}{1.539445in}}%
\pgfpathlineto{\pgfqpoint{5.806348in}{1.437567in}}%
\pgfpathlineto{\pgfqpoint{5.898417in}{1.335689in}}%
\pgfpathlineto{\pgfqpoint{5.994203in}{1.233811in}}%
\pgfpathlineto{\pgfqpoint{6.093634in}{1.131933in}}%
\pgfpathlineto{\pgfqpoint{6.196642in}{1.030055in}}%
\pgfpathlineto{\pgfqpoint{6.303207in}{0.928177in}}%
\pgfpathlineto{\pgfqpoint{6.414253in}{0.825386in}}%
\pgfpathlineto{\pgfqpoint{6.532547in}{0.719185in}}%
\pgfpathlineto{\pgfqpoint{6.650840in}{0.616053in}}%
\pgfpathlineto{\pgfqpoint{6.702832in}{0.571603in}}%
\pgfpathlineto{\pgfqpoint{6.702832in}{0.571603in}}%
\pgfusepath{stroke}%
\end{pgfscope}%
\begin{pgfscope}%
\pgfpathrectangle{\pgfqpoint{0.854460in}{0.571603in}}{\pgfqpoint{5.885100in}{5.068436in}}%
\pgfusepath{clip}%
\pgfsetbuttcap%
\pgfsetroundjoin%
\pgfsetlinewidth{1.505625pt}%
\definecolor{currentstroke}{rgb}{0.206756,0.371758,0.553117}%
\pgfsetstrokecolor{currentstroke}%
\pgfsetdash{}{0pt}%
\pgfpathmoveto{\pgfqpoint{1.927055in}{0.571603in}}%
\pgfpathlineto{\pgfqpoint{1.859955in}{0.628218in}}%
\pgfpathlineto{\pgfqpoint{1.780068in}{0.698951in}}%
\pgfpathlineto{\pgfqpoint{1.712088in}{0.762430in}}%
\pgfpathlineto{\pgfqpoint{1.646978in}{0.826299in}}%
\pgfpathlineto{\pgfqpoint{1.593795in}{0.881083in}}%
\pgfpathlineto{\pgfqpoint{1.526955in}{0.953646in}}%
\pgfpathlineto{\pgfqpoint{1.475501in}{1.012717in}}%
\pgfpathlineto{\pgfqpoint{1.416354in}{1.084495in}}%
\pgfpathlineto{\pgfqpoint{1.357208in}{1.161196in}}%
\pgfpathlineto{\pgfqpoint{1.304912in}{1.233811in}}%
\pgfpathlineto{\pgfqpoint{1.268488in}{1.287451in}}%
\pgfpathlineto{\pgfqpoint{1.221495in}{1.361159in}}%
\pgfpathlineto{\pgfqpoint{1.179767in}{1.431440in}}%
\pgfpathlineto{\pgfqpoint{1.148054in}{1.488506in}}%
\pgfpathlineto{\pgfqpoint{1.108637in}{1.564915in}}%
\pgfpathlineto{\pgfqpoint{1.072470in}{1.641323in}}%
\pgfpathlineto{\pgfqpoint{1.050140in}{1.692262in}}%
\pgfpathlineto{\pgfqpoint{1.019239in}{1.768671in}}%
\pgfpathlineto{\pgfqpoint{0.991365in}{1.845080in}}%
\pgfpathlineto{\pgfqpoint{0.966422in}{1.921488in}}%
\pgfpathlineto{\pgfqpoint{0.943181in}{2.002221in}}%
\pgfpathlineto{\pgfqpoint{0.925123in}{2.074305in}}%
\pgfpathlineto{\pgfqpoint{0.908570in}{2.150714in}}%
\pgfpathlineto{\pgfqpoint{0.894743in}{2.227123in}}%
\pgfpathlineto{\pgfqpoint{0.883463in}{2.303531in}}%
\pgfpathlineto{\pgfqpoint{0.874877in}{2.379940in}}%
\pgfpathlineto{\pgfqpoint{0.868800in}{2.456348in}}%
\pgfpathlineto{\pgfqpoint{0.865230in}{2.532757in}}%
\pgfpathlineto{\pgfqpoint{0.864167in}{2.609165in}}%
\pgfpathlineto{\pgfqpoint{0.865601in}{2.685574in}}%
\pgfpathlineto{\pgfqpoint{0.869523in}{2.761983in}}%
\pgfpathlineto{\pgfqpoint{0.875918in}{2.838391in}}%
\pgfpathlineto{\pgfqpoint{0.884781in}{2.914800in}}%
\pgfpathlineto{\pgfqpoint{0.896280in}{2.991208in}}%
\pgfpathlineto{\pgfqpoint{0.910232in}{3.067617in}}%
\pgfpathlineto{\pgfqpoint{0.926860in}{3.144025in}}%
\pgfpathlineto{\pgfqpoint{0.946045in}{3.220434in}}%
\pgfpathlineto{\pgfqpoint{0.967994in}{3.296843in}}%
\pgfpathlineto{\pgfqpoint{0.984185in}{3.347782in}}%
\pgfpathlineto{\pgfqpoint{1.010784in}{3.424190in}}%
\pgfpathlineto{\pgfqpoint{1.040316in}{3.500599in}}%
\pgfpathlineto{\pgfqpoint{1.072887in}{3.577007in}}%
\pgfpathlineto{\pgfqpoint{1.108602in}{3.653416in}}%
\pgfpathlineto{\pgfqpoint{1.134230in}{3.704355in}}%
\pgfpathlineto{\pgfqpoint{1.161360in}{3.755294in}}%
\pgfpathlineto{\pgfqpoint{1.204960in}{3.831703in}}%
\pgfpathlineto{\pgfqpoint{1.238914in}{3.887159in}}%
\pgfpathlineto{\pgfqpoint{1.286013in}{3.959050in}}%
\pgfpathlineto{\pgfqpoint{1.327634in}{4.018563in}}%
\pgfpathlineto{\pgfqpoint{1.378277in}{4.086398in}}%
\pgfpathlineto{\pgfqpoint{1.418566in}{4.137337in}}%
\pgfpathlineto{\pgfqpoint{1.475501in}{4.205093in}}%
\pgfpathlineto{\pgfqpoint{1.528796in}{4.264685in}}%
\pgfpathlineto{\pgfqpoint{1.576939in}{4.315624in}}%
\pgfpathlineto{\pgfqpoint{1.627503in}{4.366563in}}%
\pgfpathlineto{\pgfqpoint{1.682515in}{4.419173in}}%
\pgfpathlineto{\pgfqpoint{1.741661in}{4.472755in}}%
\pgfpathlineto{\pgfqpoint{1.800808in}{4.523589in}}%
\pgfpathlineto{\pgfqpoint{1.859955in}{4.571912in}}%
\pgfpathlineto{\pgfqpoint{1.923579in}{4.621258in}}%
\pgfpathlineto{\pgfqpoint{1.992897in}{4.672197in}}%
\pgfpathlineto{\pgfqpoint{2.066968in}{4.723742in}}%
\pgfpathlineto{\pgfqpoint{2.155689in}{4.781643in}}%
\pgfpathlineto{\pgfqpoint{2.226165in}{4.825014in}}%
\pgfpathlineto{\pgfqpoint{2.313890in}{4.875953in}}%
\pgfpathlineto{\pgfqpoint{2.392275in}{4.918782in}}%
\pgfpathlineto{\pgfqpoint{2.456952in}{4.952362in}}%
\pgfpathlineto{\pgfqpoint{2.540142in}{4.993224in}}%
\pgfpathlineto{\pgfqpoint{2.628862in}{5.034188in}}%
\pgfpathlineto{\pgfqpoint{2.717582in}{5.072491in}}%
\pgfpathlineto{\pgfqpoint{2.806303in}{5.108300in}}%
\pgfpathlineto{\pgfqpoint{2.895023in}{5.141589in}}%
\pgfpathlineto{\pgfqpoint{2.983743in}{5.172526in}}%
\pgfpathlineto{\pgfqpoint{3.072463in}{5.201109in}}%
\pgfpathlineto{\pgfqpoint{3.161183in}{5.227331in}}%
\pgfpathlineto{\pgfqpoint{3.249903in}{5.251172in}}%
\pgfpathlineto{\pgfqpoint{3.338623in}{5.272606in}}%
\pgfpathlineto{\pgfqpoint{3.427343in}{5.291601in}}%
\pgfpathlineto{\pgfqpoint{3.521176in}{5.308935in}}%
\pgfpathlineto{\pgfqpoint{3.604783in}{5.321741in}}%
\pgfpathlineto{\pgfqpoint{3.693504in}{5.332657in}}%
\pgfpathlineto{\pgfqpoint{3.752650in}{5.338153in}}%
\pgfpathlineto{\pgfqpoint{3.811797in}{5.342164in}}%
\pgfpathlineto{\pgfqpoint{3.870944in}{5.344611in}}%
\pgfpathlineto{\pgfqpoint{3.930090in}{5.345351in}}%
\pgfpathlineto{\pgfqpoint{3.989237in}{5.344218in}}%
\pgfpathlineto{\pgfqpoint{4.048384in}{5.341023in}}%
\pgfpathlineto{\pgfqpoint{4.116872in}{5.334405in}}%
\pgfpathlineto{\pgfqpoint{4.166677in}{5.327313in}}%
\pgfpathlineto{\pgfqpoint{4.225824in}{5.316103in}}%
\pgfpathlineto{\pgfqpoint{4.284971in}{5.301330in}}%
\pgfpathlineto{\pgfqpoint{4.344118in}{5.282366in}}%
\pgfpathlineto{\pgfqpoint{4.373691in}{5.270851in}}%
\pgfpathlineto{\pgfqpoint{4.403459in}{5.257996in}}%
\pgfpathlineto{\pgfqpoint{4.452909in}{5.232527in}}%
\pgfpathlineto{\pgfqpoint{4.462411in}{5.227013in}}%
\pgfpathlineto{\pgfqpoint{4.494010in}{5.207057in}}%
\pgfpathlineto{\pgfqpoint{4.528663in}{5.181588in}}%
\pgfpathlineto{\pgfqpoint{4.558542in}{5.156118in}}%
\pgfpathlineto{\pgfqpoint{4.584594in}{5.130649in}}%
\pgfpathlineto{\pgfqpoint{4.610278in}{5.101774in}}%
\pgfpathlineto{\pgfqpoint{4.639851in}{5.062627in}}%
\pgfpathlineto{\pgfqpoint{4.645565in}{5.054240in}}%
\pgfpathlineto{\pgfqpoint{4.669425in}{5.015103in}}%
\pgfpathlineto{\pgfqpoint{4.688728in}{4.977831in}}%
\pgfpathlineto{\pgfqpoint{4.700349in}{4.952362in}}%
\pgfpathlineto{\pgfqpoint{4.720066in}{4.901423in}}%
\pgfpathlineto{\pgfqpoint{4.736157in}{4.850484in}}%
\pgfpathlineto{\pgfqpoint{4.749260in}{4.799545in}}%
\pgfpathlineto{\pgfqpoint{4.759995in}{4.748606in}}%
\pgfpathlineto{\pgfqpoint{4.768657in}{4.697667in}}%
\pgfpathlineto{\pgfqpoint{4.778797in}{4.621258in}}%
\pgfpathlineto{\pgfqpoint{4.786301in}{4.544849in}}%
\pgfpathlineto{\pgfqpoint{4.791822in}{4.468441in}}%
\pgfpathlineto{\pgfqpoint{4.798351in}{4.341093in}}%
\pgfpathlineto{\pgfqpoint{4.816272in}{3.933581in}}%
\pgfpathlineto{\pgfqpoint{4.825305in}{3.806233in}}%
\pgfpathlineto{\pgfqpoint{4.834760in}{3.704355in}}%
\pgfpathlineto{\pgfqpoint{4.846865in}{3.600279in}}%
\pgfpathlineto{\pgfqpoint{4.860864in}{3.500599in}}%
\pgfpathlineto{\pgfqpoint{4.878034in}{3.398721in}}%
\pgfpathlineto{\pgfqpoint{4.892798in}{3.322312in}}%
\pgfpathlineto{\pgfqpoint{4.909381in}{3.245904in}}%
\pgfpathlineto{\pgfqpoint{4.927738in}{3.169495in}}%
\pgfpathlineto{\pgfqpoint{4.947980in}{3.093086in}}%
\pgfpathlineto{\pgfqpoint{4.970176in}{3.016678in}}%
\pgfpathlineto{\pgfqpoint{4.994732in}{2.939132in}}%
\pgfpathlineto{\pgfqpoint{5.024305in}{2.853342in}}%
\pgfpathlineto{\pgfqpoint{5.048701in}{2.787452in}}%
\pgfpathlineto{\pgfqpoint{5.083452in}{2.700352in}}%
\pgfpathlineto{\pgfqpoint{5.113025in}{2.630996in}}%
\pgfpathlineto{\pgfqpoint{5.145926in}{2.558226in}}%
\pgfpathlineto{\pgfqpoint{5.182583in}{2.481818in}}%
\pgfpathlineto{\pgfqpoint{5.221429in}{2.405409in}}%
\pgfpathlineto{\pgfqpoint{5.262485in}{2.329001in}}%
\pgfpathlineto{\pgfqpoint{5.305670in}{2.252592in}}%
\pgfpathlineto{\pgfqpoint{5.351123in}{2.176183in}}%
\pgfpathlineto{\pgfqpoint{5.398719in}{2.099775in}}%
\pgfpathlineto{\pgfqpoint{5.465683in}{1.997897in}}%
\pgfpathlineto{\pgfqpoint{5.536528in}{1.896019in}}%
\pgfpathlineto{\pgfqpoint{5.592266in}{1.819610in}}%
\pgfpathlineto{\pgfqpoint{5.650203in}{1.743202in}}%
\pgfpathlineto{\pgfqpoint{5.710332in}{1.666793in}}%
\pgfpathlineto{\pgfqpoint{5.793213in}{1.565781in}}%
\pgfpathlineto{\pgfqpoint{5.852359in}{1.496368in}}%
\pgfpathlineto{\pgfqpoint{5.911506in}{1.428955in}}%
\pgfpathlineto{\pgfqpoint{5.972667in}{1.361159in}}%
\pgfpathlineto{\pgfqpoint{6.067736in}{1.259281in}}%
\pgfpathlineto{\pgfqpoint{6.166582in}{1.157402in}}%
\pgfpathlineto{\pgfqpoint{6.269166in}{1.055524in}}%
\pgfpathlineto{\pgfqpoint{6.375358in}{0.953646in}}%
\pgfpathlineto{\pgfqpoint{6.485156in}{0.851768in}}%
\pgfpathlineto{\pgfqpoint{6.598484in}{0.749890in}}%
\pgfpathlineto{\pgfqpoint{6.715240in}{0.648012in}}%
\pgfpathlineto{\pgfqpoint{6.739560in}{0.627156in}}%
\pgfpathlineto{\pgfqpoint{6.739560in}{0.627156in}}%
\pgfusepath{stroke}%
\end{pgfscope}%
\begin{pgfscope}%
\pgfpathrectangle{\pgfqpoint{0.854460in}{0.571603in}}{\pgfqpoint{5.885100in}{5.068436in}}%
\pgfusepath{clip}%
\pgfsetbuttcap%
\pgfsetroundjoin%
\pgfsetlinewidth{1.505625pt}%
\definecolor{currentstroke}{rgb}{0.197636,0.391528,0.554969}%
\pgfsetstrokecolor{currentstroke}%
\pgfsetdash{}{0pt}%
\pgfpathmoveto{\pgfqpoint{1.844312in}{0.571603in}}%
\pgfpathlineto{\pgfqpoint{1.830381in}{0.583332in}}%
\pgfpathlineto{\pgfqpoint{1.814129in}{0.597073in}}%
\pgfpathlineto{\pgfqpoint{1.800808in}{0.608504in}}%
\pgfpathlineto{\pgfqpoint{1.784522in}{0.622542in}}%
\pgfpathlineto{\pgfqpoint{1.771235in}{0.634167in}}%
\pgfpathlineto{\pgfqpoint{1.755483in}{0.648012in}}%
\pgfpathlineto{\pgfqpoint{1.741661in}{0.660344in}}%
\pgfpathlineto{\pgfqpoint{1.727006in}{0.673481in}}%
\pgfpathlineto{\pgfqpoint{1.712088in}{0.687056in}}%
\pgfpathlineto{\pgfqpoint{1.699081in}{0.698951in}}%
\pgfpathlineto{\pgfqpoint{1.682515in}{0.714330in}}%
\pgfpathlineto{\pgfqpoint{1.671701in}{0.724420in}}%
\pgfpathlineto{\pgfqpoint{1.652941in}{0.742191in}}%
\pgfpathlineto{\pgfqpoint{1.644857in}{0.749890in}}%
\pgfpathlineto{\pgfqpoint{1.623368in}{0.770666in}}%
\pgfpathlineto{\pgfqpoint{1.618539in}{0.775360in}}%
\pgfpathlineto{\pgfqpoint{1.593795in}{0.799781in}}%
\pgfpathlineto{\pgfqpoint{1.592739in}{0.800829in}}%
\pgfpathlineto{\pgfqpoint{1.567496in}{0.826299in}}%
\pgfpathlineto{\pgfqpoint{1.564221in}{0.829655in}}%
\pgfpathlineto{\pgfqpoint{1.542776in}{0.851768in}}%
\pgfpathlineto{\pgfqpoint{1.534648in}{0.860278in}}%
\pgfpathlineto{\pgfqpoint{1.518549in}{0.877238in}}%
\pgfpathlineto{\pgfqpoint{1.505074in}{0.891653in}}%
\pgfpathlineto{\pgfqpoint{1.494807in}{0.902707in}}%
\pgfpathlineto{\pgfqpoint{1.475501in}{0.923815in}}%
\pgfpathlineto{\pgfqpoint{1.471538in}{0.928177in}}%
\pgfpathlineto{\pgfqpoint{1.448775in}{0.953646in}}%
\pgfpathlineto{\pgfqpoint{1.445928in}{0.956887in}}%
\pgfpathlineto{\pgfqpoint{1.426533in}{0.979116in}}%
\pgfpathlineto{\pgfqpoint{1.416354in}{0.990965in}}%
\pgfpathlineto{\pgfqpoint{1.404737in}{1.004585in}}%
\pgfpathlineto{\pgfqpoint{1.386781in}{1.025969in}}%
\pgfpathlineto{\pgfqpoint{1.383375in}{1.030055in}}%
\pgfpathlineto{\pgfqpoint{1.362518in}{1.055524in}}%
\pgfpathlineto{\pgfqpoint{1.357208in}{1.062122in}}%
\pgfpathlineto{\pgfqpoint{1.342133in}{1.080994in}}%
\pgfpathlineto{\pgfqpoint{1.327634in}{1.099436in}}%
\pgfpathlineto{\pgfqpoint{1.322152in}{1.106463in}}%
\pgfpathlineto{\pgfqpoint{1.302636in}{1.131933in}}%
\pgfpathlineto{\pgfqpoint{1.298061in}{1.138014in}}%
\pgfpathlineto{\pgfqpoint{1.283593in}{1.157402in}}%
\pgfpathlineto{\pgfqpoint{1.268488in}{1.177977in}}%
\pgfpathlineto{\pgfqpoint{1.264923in}{1.182872in}}%
\pgfpathlineto{\pgfqpoint{1.246738in}{1.208341in}}%
\pgfpathlineto{\pgfqpoint{1.238914in}{1.219498in}}%
\pgfpathlineto{\pgfqpoint{1.228963in}{1.233811in}}%
\pgfpathlineto{\pgfqpoint{1.211563in}{1.259281in}}%
\pgfpathlineto{\pgfqpoint{1.209341in}{1.262603in}}%
\pgfpathlineto{\pgfqpoint{1.194658in}{1.284750in}}%
\pgfpathlineto{\pgfqpoint{1.179767in}{1.307593in}}%
\pgfpathlineto{\pgfqpoint{1.178071in}{1.310220in}}%
\pgfpathlineto{\pgfqpoint{1.161980in}{1.335689in}}%
\pgfpathlineto{\pgfqpoint{1.150194in}{1.354685in}}%
\pgfpathlineto{\pgfqpoint{1.146215in}{1.361159in}}%
\pgfpathlineto{\pgfqpoint{1.130900in}{1.386628in}}%
\pgfpathlineto{\pgfqpoint{1.120621in}{1.404059in}}%
\pgfpathlineto{\pgfqpoint{1.115926in}{1.412098in}}%
\pgfpathlineto{\pgfqpoint{1.101382in}{1.437567in}}%
\pgfpathlineto{\pgfqpoint{1.091047in}{1.456037in}}%
\pgfpathlineto{\pgfqpoint{1.087169in}{1.463037in}}%
\pgfpathlineto{\pgfqpoint{1.073392in}{1.488506in}}%
\pgfpathlineto{\pgfqpoint{1.061474in}{1.510992in}}%
\pgfpathlineto{\pgfqpoint{1.059908in}{1.513976in}}%
\pgfpathlineto{\pgfqpoint{1.046889in}{1.539445in}}%
\pgfpathlineto{\pgfqpoint{1.034140in}{1.564915in}}%
\pgfpathlineto{\pgfqpoint{1.031901in}{1.569510in}}%
\pgfpathlineto{\pgfqpoint{1.021834in}{1.590384in}}%
\pgfpathlineto{\pgfqpoint{1.009837in}{1.615854in}}%
\pgfpathlineto{\pgfqpoint{1.002327in}{1.632202in}}%
\pgfpathlineto{\pgfqpoint{0.998184in}{1.641323in}}%
\pgfpathlineto{\pgfqpoint{0.986928in}{1.666793in}}%
\pgfpathlineto{\pgfqpoint{0.975946in}{1.692262in}}%
\pgfpathlineto{\pgfqpoint{0.972754in}{1.699887in}}%
\pgfpathlineto{\pgfqpoint{0.965367in}{1.717732in}}%
\pgfpathlineto{\pgfqpoint{0.955117in}{1.743202in}}%
\pgfpathlineto{\pgfqpoint{0.945142in}{1.768671in}}%
\pgfpathlineto{\pgfqpoint{0.943181in}{1.773843in}}%
\pgfpathlineto{\pgfqpoint{0.935576in}{1.794141in}}%
\pgfpathlineto{\pgfqpoint{0.926318in}{1.819610in}}%
\pgfpathlineto{\pgfqpoint{0.917338in}{1.845080in}}%
\pgfpathlineto{\pgfqpoint{0.913607in}{1.856029in}}%
\pgfpathlineto{\pgfqpoint{0.908720in}{1.870549in}}%
\pgfpathlineto{\pgfqpoint{0.900442in}{1.896019in}}%
\pgfpathlineto{\pgfqpoint{0.892441in}{1.921488in}}%
\pgfpathlineto{\pgfqpoint{0.884718in}{1.946958in}}%
\pgfpathlineto{\pgfqpoint{0.884034in}{1.949307in}}%
\pgfpathlineto{\pgfqpoint{0.877390in}{1.972427in}}%
\pgfpathlineto{\pgfqpoint{0.870350in}{1.997897in}}%
\pgfpathlineto{\pgfqpoint{0.863588in}{2.023366in}}%
\pgfpathlineto{\pgfqpoint{0.857103in}{2.048836in}}%
\pgfpathlineto{\pgfqpoint{0.854460in}{2.059705in}}%
\pgfusepath{stroke}%
\end{pgfscope}%
\begin{pgfscope}%
\pgfpathrectangle{\pgfqpoint{0.854460in}{0.571603in}}{\pgfqpoint{5.885100in}{5.068436in}}%
\pgfusepath{clip}%
\pgfsetbuttcap%
\pgfsetroundjoin%
\pgfsetlinewidth{1.505625pt}%
\definecolor{currentstroke}{rgb}{0.197636,0.391528,0.554969}%
\pgfsetstrokecolor{currentstroke}%
\pgfsetdash{}{0pt}%
\pgfpathmoveto{\pgfqpoint{0.854460in}{3.180525in}}%
\pgfpathlineto{\pgfqpoint{0.871070in}{3.245904in}}%
\pgfpathlineto{\pgfqpoint{0.892890in}{3.322312in}}%
\pgfpathlineto{\pgfqpoint{0.917409in}{3.398721in}}%
\pgfpathlineto{\pgfqpoint{0.944736in}{3.475129in}}%
\pgfpathlineto{\pgfqpoint{0.974979in}{3.551538in}}%
\pgfpathlineto{\pgfqpoint{1.008241in}{3.627946in}}%
\pgfpathlineto{\pgfqpoint{1.044619in}{3.704355in}}%
\pgfpathlineto{\pgfqpoint{1.084208in}{3.780764in}}%
\pgfpathlineto{\pgfqpoint{1.120621in}{3.845852in}}%
\pgfpathlineto{\pgfqpoint{1.157885in}{3.908111in}}%
\pgfpathlineto{\pgfqpoint{1.206943in}{3.984520in}}%
\pgfpathlineto{\pgfqpoint{1.241838in}{4.035459in}}%
\pgfpathlineto{\pgfqpoint{1.298061in}{4.112515in}}%
\pgfpathlineto{\pgfqpoint{1.357653in}{4.188276in}}%
\pgfpathlineto{\pgfqpoint{1.416354in}{4.257778in}}%
\pgfpathlineto{\pgfqpoint{1.468280in}{4.315624in}}%
\pgfpathlineto{\pgfqpoint{1.516473in}{4.366563in}}%
\pgfpathlineto{\pgfqpoint{1.567031in}{4.417502in}}%
\pgfpathlineto{\pgfqpoint{1.623368in}{4.471396in}}%
\pgfpathlineto{\pgfqpoint{1.682515in}{4.525066in}}%
\pgfpathlineto{\pgfqpoint{1.741661in}{4.576057in}}%
\pgfpathlineto{\pgfqpoint{1.800808in}{4.624600in}}%
\pgfpathlineto{\pgfqpoint{1.861687in}{4.672197in}}%
\pgfpathlineto{\pgfqpoint{1.930308in}{4.723136in}}%
\pgfpathlineto{\pgfqpoint{2.007822in}{4.777652in}}%
\pgfpathlineto{\pgfqpoint{2.096542in}{4.836291in}}%
\pgfpathlineto{\pgfqpoint{2.159813in}{4.875953in}}%
\pgfpathlineto{\pgfqpoint{2.245485in}{4.926892in}}%
\pgfpathlineto{\pgfqpoint{2.336615in}{4.977831in}}%
\pgfpathlineto{\pgfqpoint{2.433832in}{5.028770in}}%
\pgfpathlineto{\pgfqpoint{2.510569in}{5.066663in}}%
\pgfpathlineto{\pgfqpoint{2.599289in}{5.108130in}}%
\pgfpathlineto{\pgfqpoint{2.688009in}{5.147058in}}%
\pgfpathlineto{\pgfqpoint{2.776729in}{5.183712in}}%
\pgfpathlineto{\pgfqpoint{2.865449in}{5.218001in}}%
\pgfpathlineto{\pgfqpoint{2.954169in}{5.250143in}}%
\pgfpathlineto{\pgfqpoint{3.053137in}{5.283466in}}%
\pgfpathlineto{\pgfqpoint{3.134672in}{5.308935in}}%
\pgfpathlineto{\pgfqpoint{3.222709in}{5.334405in}}%
\pgfpathlineto{\pgfqpoint{3.319161in}{5.359874in}}%
\pgfpathlineto{\pgfqpoint{3.397770in}{5.378751in}}%
\pgfpathlineto{\pgfqpoint{3.486490in}{5.397981in}}%
\pgfpathlineto{\pgfqpoint{3.575210in}{5.414993in}}%
\pgfpathlineto{\pgfqpoint{3.663930in}{5.429569in}}%
\pgfpathlineto{\pgfqpoint{3.752650in}{5.441654in}}%
\pgfpathlineto{\pgfqpoint{3.841370in}{5.450999in}}%
\pgfpathlineto{\pgfqpoint{3.930090in}{5.457464in}}%
\pgfpathlineto{\pgfqpoint{3.989237in}{5.459989in}}%
\pgfpathlineto{\pgfqpoint{4.048384in}{5.460932in}}%
\pgfpathlineto{\pgfqpoint{4.107531in}{5.460144in}}%
\pgfpathlineto{\pgfqpoint{4.166677in}{5.457450in}}%
\pgfpathlineto{\pgfqpoint{4.225824in}{5.452652in}}%
\pgfpathlineto{\pgfqpoint{4.284971in}{5.445521in}}%
\pgfpathlineto{\pgfqpoint{4.344118in}{5.435771in}}%
\pgfpathlineto{\pgfqpoint{4.403264in}{5.422664in}}%
\pgfpathlineto{\pgfqpoint{4.462411in}{5.405951in}}%
\pgfpathlineto{\pgfqpoint{4.521558in}{5.384687in}}%
\pgfpathlineto{\pgfqpoint{4.576240in}{5.359874in}}%
\pgfpathlineto{\pgfqpoint{4.610278in}{5.341340in}}%
\pgfpathlineto{\pgfqpoint{4.639851in}{5.322959in}}%
\pgfpathlineto{\pgfqpoint{4.669425in}{5.302085in}}%
\pgfpathlineto{\pgfqpoint{4.698998in}{5.278135in}}%
\pgfpathlineto{\pgfqpoint{4.728571in}{5.250365in}}%
\pgfpathlineto{\pgfqpoint{4.758145in}{5.217810in}}%
\pgfpathlineto{\pgfqpoint{4.766937in}{5.207057in}}%
\pgfpathlineto{\pgfqpoint{4.787718in}{5.179069in}}%
\pgfpathlineto{\pgfqpoint{4.802723in}{5.156118in}}%
\pgfpathlineto{\pgfqpoint{4.817754in}{5.130649in}}%
\pgfpathlineto{\pgfqpoint{4.830972in}{5.105179in}}%
\pgfpathlineto{\pgfqpoint{4.846865in}{5.070370in}}%
\pgfpathlineto{\pgfqpoint{4.862875in}{5.028770in}}%
\pgfpathlineto{\pgfqpoint{4.878890in}{4.977831in}}%
\pgfpathlineto{\pgfqpoint{4.891506in}{4.926892in}}%
\pgfpathlineto{\pgfqpoint{4.901533in}{4.875953in}}%
\pgfpathlineto{\pgfqpoint{4.909337in}{4.825014in}}%
\pgfpathlineto{\pgfqpoint{4.915300in}{4.774075in}}%
\pgfpathlineto{\pgfqpoint{4.921600in}{4.697667in}}%
\pgfpathlineto{\pgfqpoint{4.925464in}{4.621258in}}%
\pgfpathlineto{\pgfqpoint{4.927987in}{4.519380in}}%
\pgfpathlineto{\pgfqpoint{4.928668in}{4.392032in}}%
\pgfpathlineto{\pgfqpoint{4.929266in}{4.060928in}}%
\pgfpathlineto{\pgfqpoint{4.932351in}{3.933581in}}%
\pgfpathlineto{\pgfqpoint{4.936869in}{3.831703in}}%
\pgfpathlineto{\pgfqpoint{4.943531in}{3.729825in}}%
\pgfpathlineto{\pgfqpoint{4.952681in}{3.627946in}}%
\pgfpathlineto{\pgfqpoint{4.964590in}{3.526068in}}%
\pgfpathlineto{\pgfqpoint{4.979334in}{3.424190in}}%
\pgfpathlineto{\pgfqpoint{4.994732in}{3.335848in}}%
\pgfpathlineto{\pgfqpoint{5.007375in}{3.271373in}}%
\pgfpathlineto{\pgfqpoint{5.024305in}{3.194587in}}%
\pgfpathlineto{\pgfqpoint{5.042897in}{3.118556in}}%
\pgfpathlineto{\pgfqpoint{5.063595in}{3.042147in}}%
\pgfpathlineto{\pgfqpoint{5.086330in}{2.965739in}}%
\pgfpathlineto{\pgfqpoint{5.113025in}{2.883746in}}%
\pgfpathlineto{\pgfqpoint{5.137947in}{2.812922in}}%
\pgfpathlineto{\pgfqpoint{5.172172in}{2.723376in}}%
\pgfpathlineto{\pgfqpoint{5.201745in}{2.651468in}}%
\pgfpathlineto{\pgfqpoint{5.231354in}{2.583696in}}%
\pgfpathlineto{\pgfqpoint{5.266804in}{2.507287in}}%
\pgfpathlineto{\pgfqpoint{5.304469in}{2.430879in}}%
\pgfpathlineto{\pgfqpoint{5.349612in}{2.344853in}}%
\pgfpathlineto{\pgfqpoint{5.386504in}{2.278062in}}%
\pgfpathlineto{\pgfqpoint{5.438332in}{2.189255in}}%
\pgfpathlineto{\pgfqpoint{5.477471in}{2.125244in}}%
\pgfpathlineto{\pgfqpoint{5.527052in}{2.047785in}}%
\pgfpathlineto{\pgfqpoint{5.586199in}{1.959727in}}%
\pgfpathlineto{\pgfqpoint{5.630782in}{1.896019in}}%
\pgfpathlineto{\pgfqpoint{5.704492in}{1.795428in}}%
\pgfpathlineto{\pgfqpoint{5.744235in}{1.743202in}}%
\pgfpathlineto{\pgfqpoint{5.824870in}{1.641323in}}%
\pgfpathlineto{\pgfqpoint{5.887912in}{1.564915in}}%
\pgfpathlineto{\pgfqpoint{5.975441in}{1.463037in}}%
\pgfpathlineto{\pgfqpoint{6.066872in}{1.361159in}}%
\pgfpathlineto{\pgfqpoint{6.162185in}{1.259281in}}%
\pgfpathlineto{\pgfqpoint{6.261365in}{1.157402in}}%
\pgfpathlineto{\pgfqpoint{6.364315in}{1.055524in}}%
\pgfpathlineto{\pgfqpoint{6.473400in}{0.951441in}}%
\pgfpathlineto{\pgfqpoint{6.581411in}{0.851768in}}%
\pgfpathlineto{\pgfqpoint{6.695420in}{0.749890in}}%
\pgfpathlineto{\pgfqpoint{6.739560in}{0.711291in}}%
\pgfpathlineto{\pgfqpoint{6.739560in}{0.711291in}}%
\pgfusepath{stroke}%
\end{pgfscope}%
\begin{pgfscope}%
\pgfpathrectangle{\pgfqpoint{0.854460in}{0.571603in}}{\pgfqpoint{5.885100in}{5.068436in}}%
\pgfusepath{clip}%
\pgfsetbuttcap%
\pgfsetroundjoin%
\pgfsetlinewidth{1.505625pt}%
\definecolor{currentstroke}{rgb}{0.187231,0.414746,0.556547}%
\pgfsetstrokecolor{currentstroke}%
\pgfsetdash{}{0pt}%
\pgfpathmoveto{\pgfqpoint{1.764898in}{0.571603in}}%
\pgfpathlineto{\pgfqpoint{1.741661in}{0.591424in}}%
\pgfpathlineto{\pgfqpoint{1.735067in}{0.597073in}}%
\pgfpathlineto{\pgfqpoint{1.712088in}{0.617050in}}%
\pgfpathlineto{\pgfqpoint{1.705799in}{0.622542in}}%
\pgfpathlineto{\pgfqpoint{1.682515in}{0.643180in}}%
\pgfpathlineto{\pgfqpoint{1.677089in}{0.648012in}}%
\pgfpathlineto{\pgfqpoint{1.652941in}{0.669836in}}%
\pgfpathlineto{\pgfqpoint{1.648927in}{0.673481in}}%
\pgfpathlineto{\pgfqpoint{1.623368in}{0.697041in}}%
\pgfpathlineto{\pgfqpoint{1.621306in}{0.698951in}}%
\pgfpathlineto{\pgfqpoint{1.594224in}{0.724420in}}%
\pgfpathlineto{\pgfqpoint{1.593795in}{0.724831in}}%
\pgfpathlineto{\pgfqpoint{1.567705in}{0.749890in}}%
\pgfpathlineto{\pgfqpoint{1.564221in}{0.753287in}}%
\pgfpathlineto{\pgfqpoint{1.541709in}{0.775360in}}%
\pgfpathlineto{\pgfqpoint{1.534648in}{0.782387in}}%
\pgfpathlineto{\pgfqpoint{1.516225in}{0.800829in}}%
\pgfpathlineto{\pgfqpoint{1.505074in}{0.812161in}}%
\pgfpathlineto{\pgfqpoint{1.491244in}{0.826299in}}%
\pgfpathlineto{\pgfqpoint{1.475501in}{0.842637in}}%
\pgfpathlineto{\pgfqpoint{1.466757in}{0.851768in}}%
\pgfpathlineto{\pgfqpoint{1.445928in}{0.873849in}}%
\pgfpathlineto{\pgfqpoint{1.442751in}{0.877238in}}%
\pgfpathlineto{\pgfqpoint{1.419262in}{0.902707in}}%
\pgfpathlineto{\pgfqpoint{1.416354in}{0.905912in}}%
\pgfpathlineto{\pgfqpoint{1.396290in}{0.928177in}}%
\pgfpathlineto{\pgfqpoint{1.386781in}{0.938891in}}%
\pgfpathlineto{\pgfqpoint{1.373775in}{0.953646in}}%
\pgfpathlineto{\pgfqpoint{1.357208in}{0.972733in}}%
\pgfpathlineto{\pgfqpoint{1.351706in}{0.979116in}}%
\pgfpathlineto{\pgfqpoint{1.330109in}{1.004585in}}%
\pgfpathlineto{\pgfqpoint{1.327634in}{1.007556in}}%
\pgfpathlineto{\pgfqpoint{1.309028in}{1.030055in}}%
\pgfpathlineto{\pgfqpoint{1.298061in}{1.043524in}}%
\pgfpathlineto{\pgfqpoint{1.288363in}{1.055524in}}%
\pgfpathlineto{\pgfqpoint{1.268488in}{1.080508in}}%
\pgfpathlineto{\pgfqpoint{1.268104in}{1.080994in}}%
\pgfpathlineto{\pgfqpoint{1.248385in}{1.106463in}}%
\pgfpathlineto{\pgfqpoint{1.238914in}{1.118895in}}%
\pgfpathlineto{\pgfqpoint{1.229059in}{1.131933in}}%
\pgfpathlineto{\pgfqpoint{1.210120in}{1.157402in}}%
\pgfpathlineto{\pgfqpoint{1.209341in}{1.158472in}}%
\pgfpathlineto{\pgfqpoint{1.191705in}{1.182872in}}%
\pgfpathlineto{\pgfqpoint{1.179767in}{1.199660in}}%
\pgfpathlineto{\pgfqpoint{1.173645in}{1.208341in}}%
\pgfpathlineto{\pgfqpoint{1.156020in}{1.233811in}}%
\pgfpathlineto{\pgfqpoint{1.150194in}{1.242391in}}%
\pgfpathlineto{\pgfqpoint{1.138827in}{1.259281in}}%
\pgfpathlineto{\pgfqpoint{1.121977in}{1.284750in}}%
\pgfpathlineto{\pgfqpoint{1.120621in}{1.286846in}}%
\pgfpathlineto{\pgfqpoint{1.105625in}{1.310220in}}%
\pgfpathlineto{\pgfqpoint{1.091047in}{1.333331in}}%
\pgfpathlineto{\pgfqpoint{1.089574in}{1.335689in}}%
\pgfpathlineto{\pgfqpoint{1.074010in}{1.361159in}}%
\pgfpathlineto{\pgfqpoint{1.061474in}{1.382047in}}%
\pgfpathlineto{\pgfqpoint{1.058751in}{1.386628in}}%
\pgfpathlineto{\pgfqpoint{1.043950in}{1.412098in}}%
\pgfpathlineto{\pgfqpoint{1.031901in}{1.433231in}}%
\pgfpathlineto{\pgfqpoint{1.029452in}{1.437567in}}%
\pgfpathlineto{\pgfqpoint{1.015410in}{1.463037in}}%
\pgfpathlineto{\pgfqpoint{1.002327in}{1.487228in}}%
\pgfpathlineto{\pgfqpoint{1.001643in}{1.488506in}}%
\pgfpathlineto{\pgfqpoint{0.988351in}{1.513976in}}%
\pgfpathlineto{\pgfqpoint{0.975326in}{1.539445in}}%
\pgfpathlineto{\pgfqpoint{0.972754in}{1.544605in}}%
\pgfpathlineto{\pgfqpoint{0.962735in}{1.564915in}}%
\pgfpathlineto{\pgfqpoint{0.950454in}{1.590384in}}%
\pgfpathlineto{\pgfqpoint{0.943181in}{1.605842in}}%
\pgfpathlineto{\pgfqpoint{0.938520in}{1.615854in}}%
\pgfpathlineto{\pgfqpoint{0.926974in}{1.641323in}}%
\pgfpathlineto{\pgfqpoint{0.915697in}{1.666793in}}%
\pgfpathlineto{\pgfqpoint{0.913607in}{1.671652in}}%
\pgfpathlineto{\pgfqpoint{0.904840in}{1.692262in}}%
\pgfpathlineto{\pgfqpoint{0.894288in}{1.717732in}}%
\pgfpathlineto{\pgfqpoint{0.884034in}{1.743136in}}%
\pgfpathlineto{\pgfqpoint{0.884008in}{1.743202in}}%
\pgfpathlineto{\pgfqpoint{0.874167in}{1.768671in}}%
\pgfpathlineto{\pgfqpoint{0.864598in}{1.794141in}}%
\pgfpathlineto{\pgfqpoint{0.855302in}{1.819610in}}%
\pgfpathlineto{\pgfqpoint{0.854460in}{1.821997in}}%
\pgfusepath{stroke}%
\end{pgfscope}%
\begin{pgfscope}%
\pgfpathrectangle{\pgfqpoint{0.854460in}{0.571603in}}{\pgfqpoint{5.885100in}{5.068436in}}%
\pgfusepath{clip}%
\pgfsetbuttcap%
\pgfsetroundjoin%
\pgfsetlinewidth{1.505625pt}%
\definecolor{currentstroke}{rgb}{0.187231,0.414746,0.556547}%
\pgfsetstrokecolor{currentstroke}%
\pgfsetdash{}{0pt}%
\pgfpathmoveto{\pgfqpoint{0.854460in}{3.448520in}}%
\pgfpathlineto{\pgfqpoint{0.873372in}{3.500599in}}%
\pgfpathlineto{\pgfqpoint{0.903426in}{3.577007in}}%
\pgfpathlineto{\pgfqpoint{0.936405in}{3.653416in}}%
\pgfpathlineto{\pgfqpoint{0.960110in}{3.704355in}}%
\pgfpathlineto{\pgfqpoint{0.998230in}{3.780764in}}%
\pgfpathlineto{\pgfqpoint{1.031901in}{3.843297in}}%
\pgfpathlineto{\pgfqpoint{1.069215in}{3.908111in}}%
\pgfpathlineto{\pgfqpoint{1.116462in}{3.984520in}}%
\pgfpathlineto{\pgfqpoint{1.150194in}{4.035732in}}%
\pgfpathlineto{\pgfqpoint{1.203682in}{4.111867in}}%
\pgfpathlineto{\pgfqpoint{1.241686in}{4.162807in}}%
\pgfpathlineto{\pgfqpoint{1.302331in}{4.239215in}}%
\pgfpathlineto{\pgfqpoint{1.357208in}{4.303862in}}%
\pgfpathlineto{\pgfqpoint{1.416354in}{4.369441in}}%
\pgfpathlineto{\pgfqpoint{1.475501in}{4.431120in}}%
\pgfpathlineto{\pgfqpoint{1.539295in}{4.493910in}}%
\pgfpathlineto{\pgfqpoint{1.593896in}{4.544849in}}%
\pgfpathlineto{\pgfqpoint{1.652941in}{4.597230in}}%
\pgfpathlineto{\pgfqpoint{1.712088in}{4.647200in}}%
\pgfpathlineto{\pgfqpoint{1.774853in}{4.697667in}}%
\pgfpathlineto{\pgfqpoint{1.841465in}{4.748606in}}%
\pgfpathlineto{\pgfqpoint{1.919102in}{4.804913in}}%
\pgfpathlineto{\pgfqpoint{1.985299in}{4.850484in}}%
\pgfpathlineto{\pgfqpoint{2.066968in}{4.903881in}}%
\pgfpathlineto{\pgfqpoint{2.155689in}{4.958565in}}%
\pgfpathlineto{\pgfqpoint{2.244409in}{5.010128in}}%
\pgfpathlineto{\pgfqpoint{2.333129in}{5.058798in}}%
\pgfpathlineto{\pgfqpoint{2.422646in}{5.105179in}}%
\pgfpathlineto{\pgfqpoint{2.527450in}{5.156118in}}%
\pgfpathlineto{\pgfqpoint{2.599289in}{5.189120in}}%
\pgfpathlineto{\pgfqpoint{2.699077in}{5.232527in}}%
\pgfpathlineto{\pgfqpoint{2.806303in}{5.276074in}}%
\pgfpathlineto{\pgfqpoint{2.895023in}{5.309871in}}%
\pgfpathlineto{\pgfqpoint{3.013316in}{5.351686in}}%
\pgfpathlineto{\pgfqpoint{3.116367in}{5.385344in}}%
\pgfpathlineto{\pgfqpoint{3.220330in}{5.416626in}}%
\pgfpathlineto{\pgfqpoint{3.309050in}{5.441253in}}%
\pgfpathlineto{\pgfqpoint{3.397770in}{5.463970in}}%
\pgfpathlineto{\pgfqpoint{3.516063in}{5.491161in}}%
\pgfpathlineto{\pgfqpoint{3.623094in}{5.512691in}}%
\pgfpathlineto{\pgfqpoint{3.693504in}{5.525143in}}%
\pgfpathlineto{\pgfqpoint{3.782224in}{5.538994in}}%
\pgfpathlineto{\pgfqpoint{3.870944in}{5.550368in}}%
\pgfpathlineto{\pgfqpoint{3.959664in}{5.559367in}}%
\pgfpathlineto{\pgfqpoint{4.048384in}{5.565637in}}%
\pgfpathlineto{\pgfqpoint{4.137104in}{5.568847in}}%
\pgfpathlineto{\pgfqpoint{4.196251in}{5.569158in}}%
\pgfpathlineto{\pgfqpoint{4.255398in}{5.567847in}}%
\pgfpathlineto{\pgfqpoint{4.329285in}{5.563630in}}%
\pgfpathlineto{\pgfqpoint{4.373691in}{5.559558in}}%
\pgfpathlineto{\pgfqpoint{4.432838in}{5.552065in}}%
\pgfpathlineto{\pgfqpoint{4.491984in}{5.542051in}}%
\pgfpathlineto{\pgfqpoint{4.521558in}{5.535926in}}%
\pgfpathlineto{\pgfqpoint{4.580705in}{5.520993in}}%
\pgfpathlineto{\pgfqpoint{4.610278in}{5.512204in}}%
\pgfpathlineto{\pgfqpoint{4.669425in}{5.490929in}}%
\pgfpathlineto{\pgfqpoint{4.698998in}{5.478253in}}%
\pgfpathlineto{\pgfqpoint{4.733190in}{5.461752in}}%
\pgfpathlineto{\pgfqpoint{4.777773in}{5.436283in}}%
\pgfpathlineto{\pgfqpoint{4.787718in}{5.429891in}}%
\pgfpathlineto{\pgfqpoint{4.817291in}{5.409300in}}%
\pgfpathlineto{\pgfqpoint{4.847183in}{5.385344in}}%
\pgfpathlineto{\pgfqpoint{4.876438in}{5.358092in}}%
\pgfpathlineto{\pgfqpoint{4.906012in}{5.325655in}}%
\pgfpathlineto{\pgfqpoint{4.919488in}{5.308935in}}%
\pgfpathlineto{\pgfqpoint{4.937987in}{5.283466in}}%
\pgfpathlineto{\pgfqpoint{4.965158in}{5.238848in}}%
\pgfpathlineto{\pgfqpoint{4.981282in}{5.207057in}}%
\pgfpathlineto{\pgfqpoint{4.994732in}{5.176335in}}%
\pgfpathlineto{\pgfqpoint{5.011351in}{5.130649in}}%
\pgfpathlineto{\pgfqpoint{5.026062in}{5.079709in}}%
\pgfpathlineto{\pgfqpoint{5.037303in}{5.028770in}}%
\pgfpathlineto{\pgfqpoint{5.045871in}{4.977831in}}%
\pgfpathlineto{\pgfqpoint{5.052208in}{4.926892in}}%
\pgfpathlineto{\pgfqpoint{5.056640in}{4.875953in}}%
\pgfpathlineto{\pgfqpoint{5.060495in}{4.799545in}}%
\pgfpathlineto{\pgfqpoint{5.061808in}{4.723136in}}%
\pgfpathlineto{\pgfqpoint{5.060792in}{4.621258in}}%
\pgfpathlineto{\pgfqpoint{5.056856in}{4.493910in}}%
\pgfpathlineto{\pgfqpoint{5.043871in}{4.137337in}}%
\pgfpathlineto{\pgfqpoint{5.042193in}{4.009989in}}%
\pgfpathlineto{\pgfqpoint{5.042923in}{3.908111in}}%
\pgfpathlineto{\pgfqpoint{5.045854in}{3.806233in}}%
\pgfpathlineto{\pgfqpoint{5.051277in}{3.704355in}}%
\pgfpathlineto{\pgfqpoint{5.059383in}{3.602477in}}%
\pgfpathlineto{\pgfqpoint{5.067367in}{3.526068in}}%
\pgfpathlineto{\pgfqpoint{5.077108in}{3.449660in}}%
\pgfpathlineto{\pgfqpoint{5.088635in}{3.373251in}}%
\pgfpathlineto{\pgfqpoint{5.102002in}{3.296843in}}%
\pgfpathlineto{\pgfqpoint{5.117319in}{3.220434in}}%
\pgfpathlineto{\pgfqpoint{5.134573in}{3.144025in}}%
\pgfpathlineto{\pgfqpoint{5.153847in}{3.067617in}}%
\pgfpathlineto{\pgfqpoint{5.175207in}{2.991208in}}%
\pgfpathlineto{\pgfqpoint{5.201745in}{2.905277in}}%
\pgfpathlineto{\pgfqpoint{5.224170in}{2.838391in}}%
\pgfpathlineto{\pgfqpoint{5.251871in}{2.761983in}}%
\pgfpathlineto{\pgfqpoint{5.281756in}{2.685574in}}%
\pgfpathlineto{\pgfqpoint{5.320039in}{2.595128in}}%
\pgfpathlineto{\pgfqpoint{5.349612in}{2.529697in}}%
\pgfpathlineto{\pgfqpoint{5.384698in}{2.456348in}}%
\pgfpathlineto{\pgfqpoint{5.423475in}{2.379940in}}%
\pgfpathlineto{\pgfqpoint{5.467906in}{2.297534in}}%
\pgfpathlineto{\pgfqpoint{5.507858in}{2.227123in}}%
\pgfpathlineto{\pgfqpoint{5.556626in}{2.145630in}}%
\pgfpathlineto{\pgfqpoint{5.601340in}{2.074305in}}%
\pgfpathlineto{\pgfqpoint{5.651521in}{1.997897in}}%
\pgfpathlineto{\pgfqpoint{5.704492in}{1.920787in}}%
\pgfpathlineto{\pgfqpoint{5.763639in}{1.838404in}}%
\pgfpathlineto{\pgfqpoint{5.822786in}{1.759471in}}%
\pgfpathlineto{\pgfqpoint{5.881933in}{1.683589in}}%
\pgfpathlineto{\pgfqpoint{5.941079in}{1.610423in}}%
\pgfpathlineto{\pgfqpoint{6.000434in}{1.539445in}}%
\pgfpathlineto{\pgfqpoint{6.066486in}{1.463037in}}%
\pgfpathlineto{\pgfqpoint{6.158069in}{1.361159in}}%
\pgfpathlineto{\pgfqpoint{6.253608in}{1.259281in}}%
\pgfpathlineto{\pgfqpoint{6.355107in}{1.155391in}}%
\pgfpathlineto{\pgfqpoint{6.456404in}{1.055524in}}%
\pgfpathlineto{\pgfqpoint{6.563597in}{0.953646in}}%
\pgfpathlineto{\pgfqpoint{6.680414in}{0.846461in}}%
\pgfpathlineto{\pgfqpoint{6.739560in}{0.793581in}}%
\pgfpathlineto{\pgfqpoint{6.739560in}{0.793581in}}%
\pgfusepath{stroke}%
\end{pgfscope}%
\begin{pgfscope}%
\pgfpathrectangle{\pgfqpoint{0.854460in}{0.571603in}}{\pgfqpoint{5.885100in}{5.068436in}}%
\pgfusepath{clip}%
\pgfsetbuttcap%
\pgfsetroundjoin%
\pgfsetlinewidth{1.505625pt}%
\definecolor{currentstroke}{rgb}{0.179019,0.433756,0.557430}%
\pgfsetstrokecolor{currentstroke}%
\pgfsetdash{}{0pt}%
\pgfpathmoveto{\pgfqpoint{1.688535in}{0.571603in}}%
\pgfpathlineto{\pgfqpoint{1.682515in}{0.576783in}}%
\pgfpathlineto{\pgfqpoint{1.659034in}{0.597073in}}%
\pgfpathlineto{\pgfqpoint{1.652941in}{0.602416in}}%
\pgfpathlineto{\pgfqpoint{1.630094in}{0.622542in}}%
\pgfpathlineto{\pgfqpoint{1.623368in}{0.628555in}}%
\pgfpathlineto{\pgfqpoint{1.601706in}{0.648012in}}%
\pgfpathlineto{\pgfqpoint{1.593795in}{0.655224in}}%
\pgfpathlineto{\pgfqpoint{1.573864in}{0.673481in}}%
\pgfpathlineto{\pgfqpoint{1.564221in}{0.682446in}}%
\pgfpathlineto{\pgfqpoint{1.546558in}{0.698951in}}%
\pgfpathlineto{\pgfqpoint{1.534648in}{0.710247in}}%
\pgfpathlineto{\pgfqpoint{1.519781in}{0.724420in}}%
\pgfpathlineto{\pgfqpoint{1.505074in}{0.738651in}}%
\pgfpathlineto{\pgfqpoint{1.493523in}{0.749890in}}%
\pgfpathlineto{\pgfqpoint{1.475501in}{0.767687in}}%
\pgfpathlineto{\pgfqpoint{1.467775in}{0.775360in}}%
\pgfpathlineto{\pgfqpoint{1.445928in}{0.797381in}}%
\pgfpathlineto{\pgfqpoint{1.442527in}{0.800829in}}%
\pgfpathlineto{\pgfqpoint{1.417792in}{0.826299in}}%
\pgfpathlineto{\pgfqpoint{1.416354in}{0.827803in}}%
\pgfpathlineto{\pgfqpoint{1.393595in}{0.851768in}}%
\pgfpathlineto{\pgfqpoint{1.386781in}{0.859052in}}%
\pgfpathlineto{\pgfqpoint{1.369876in}{0.877238in}}%
\pgfpathlineto{\pgfqpoint{1.357208in}{0.891073in}}%
\pgfpathlineto{\pgfqpoint{1.346624in}{0.902707in}}%
\pgfpathlineto{\pgfqpoint{1.327634in}{0.923900in}}%
\pgfpathlineto{\pgfqpoint{1.323828in}{0.928177in}}%
\pgfpathlineto{\pgfqpoint{1.301529in}{0.953646in}}%
\pgfpathlineto{\pgfqpoint{1.298061in}{0.957675in}}%
\pgfpathlineto{\pgfqpoint{1.279729in}{0.979116in}}%
\pgfpathlineto{\pgfqpoint{1.268488in}{0.992468in}}%
\pgfpathlineto{\pgfqpoint{1.258358in}{1.004585in}}%
\pgfpathlineto{\pgfqpoint{1.238914in}{1.028208in}}%
\pgfpathlineto{\pgfqpoint{1.237405in}{1.030055in}}%
\pgfpathlineto{\pgfqpoint{1.216973in}{1.055524in}}%
\pgfpathlineto{\pgfqpoint{1.209341in}{1.065195in}}%
\pgfpathlineto{\pgfqpoint{1.196965in}{1.080994in}}%
\pgfpathlineto{\pgfqpoint{1.179767in}{1.103297in}}%
\pgfpathlineto{\pgfqpoint{1.177345in}{1.106463in}}%
\pgfpathlineto{\pgfqpoint{1.158224in}{1.131933in}}%
\pgfpathlineto{\pgfqpoint{1.150194in}{1.142811in}}%
\pgfpathlineto{\pgfqpoint{1.139510in}{1.157402in}}%
\pgfpathlineto{\pgfqpoint{1.121162in}{1.182872in}}%
\pgfpathlineto{\pgfqpoint{1.120621in}{1.183638in}}%
\pgfpathlineto{\pgfqpoint{1.103330in}{1.208341in}}%
\pgfpathlineto{\pgfqpoint{1.091047in}{1.226179in}}%
\pgfpathlineto{\pgfqpoint{1.085837in}{1.233811in}}%
\pgfpathlineto{\pgfqpoint{1.068782in}{1.259281in}}%
\pgfpathlineto{\pgfqpoint{1.061474in}{1.270400in}}%
\pgfpathlineto{\pgfqpoint{1.052126in}{1.284750in}}%
\pgfpathlineto{\pgfqpoint{1.035836in}{1.310220in}}%
\pgfpathlineto{\pgfqpoint{1.031901in}{1.316503in}}%
\pgfpathlineto{\pgfqpoint{1.019992in}{1.335689in}}%
\pgfpathlineto{\pgfqpoint{1.004464in}{1.361159in}}%
\pgfpathlineto{\pgfqpoint{1.002327in}{1.364743in}}%
\pgfpathlineto{\pgfqpoint{0.989402in}{1.386628in}}%
\pgfpathlineto{\pgfqpoint{0.974633in}{1.412098in}}%
\pgfpathlineto{\pgfqpoint{0.972754in}{1.415414in}}%
\pgfpathlineto{\pgfqpoint{0.960324in}{1.437567in}}%
\pgfpathlineto{\pgfqpoint{0.946306in}{1.463037in}}%
\pgfpathlineto{\pgfqpoint{0.943181in}{1.468851in}}%
\pgfpathlineto{\pgfqpoint{0.932720in}{1.488506in}}%
\pgfpathlineto{\pgfqpoint{0.919447in}{1.513976in}}%
\pgfpathlineto{\pgfqpoint{0.913607in}{1.525447in}}%
\pgfpathlineto{\pgfqpoint{0.906553in}{1.539445in}}%
\pgfpathlineto{\pgfqpoint{0.894016in}{1.564915in}}%
\pgfpathlineto{\pgfqpoint{0.884034in}{1.585656in}}%
\pgfpathlineto{\pgfqpoint{0.881782in}{1.590384in}}%
\pgfpathlineto{\pgfqpoint{0.869972in}{1.615854in}}%
\pgfpathlineto{\pgfqpoint{0.858429in}{1.641323in}}%
\pgfpathlineto{\pgfqpoint{0.854460in}{1.650320in}}%
\pgfusepath{stroke}%
\end{pgfscope}%
\begin{pgfscope}%
\pgfpathrectangle{\pgfqpoint{0.854460in}{0.571603in}}{\pgfqpoint{5.885100in}{5.068436in}}%
\pgfusepath{clip}%
\pgfsetbuttcap%
\pgfsetroundjoin%
\pgfsetlinewidth{1.505625pt}%
\definecolor{currentstroke}{rgb}{0.179019,0.433756,0.557430}%
\pgfsetstrokecolor{currentstroke}%
\pgfsetdash{}{0pt}%
\pgfpathmoveto{\pgfqpoint{0.854460in}{3.647376in}}%
\pgfpathlineto{\pgfqpoint{0.884034in}{3.713075in}}%
\pgfpathlineto{\pgfqpoint{0.916851in}{3.780764in}}%
\pgfpathlineto{\pgfqpoint{0.956944in}{3.857172in}}%
\pgfpathlineto{\pgfqpoint{1.000263in}{3.933581in}}%
\pgfpathlineto{\pgfqpoint{1.031901in}{3.985823in}}%
\pgfpathlineto{\pgfqpoint{1.080423in}{4.060928in}}%
\pgfpathlineto{\pgfqpoint{1.120621in}{4.119342in}}%
\pgfpathlineto{\pgfqpoint{1.171091in}{4.188276in}}%
\pgfpathlineto{\pgfqpoint{1.210498in}{4.239215in}}%
\pgfpathlineto{\pgfqpoint{1.273359in}{4.315624in}}%
\pgfpathlineto{\pgfqpoint{1.327634in}{4.377493in}}%
\pgfpathlineto{\pgfqpoint{1.388507in}{4.442971in}}%
\pgfpathlineto{\pgfqpoint{1.445928in}{4.501217in}}%
\pgfpathlineto{\pgfqpoint{1.505074in}{4.558118in}}%
\pgfpathlineto{\pgfqpoint{1.574462in}{4.621258in}}%
\pgfpathlineto{\pgfqpoint{1.633514in}{4.672197in}}%
\pgfpathlineto{\pgfqpoint{1.695484in}{4.723136in}}%
\pgfpathlineto{\pgfqpoint{1.771235in}{4.782220in}}%
\pgfpathlineto{\pgfqpoint{1.830381in}{4.826159in}}%
\pgfpathlineto{\pgfqpoint{1.919102in}{4.888582in}}%
\pgfpathlineto{\pgfqpoint{1.978248in}{4.928217in}}%
\pgfpathlineto{\pgfqpoint{2.066968in}{4.984695in}}%
\pgfpathlineto{\pgfqpoint{2.155689in}{5.038033in}}%
\pgfpathlineto{\pgfqpoint{2.244409in}{5.088461in}}%
\pgfpathlineto{\pgfqpoint{2.333129in}{5.136193in}}%
\pgfpathlineto{\pgfqpoint{2.422174in}{5.181588in}}%
\pgfpathlineto{\pgfqpoint{2.528393in}{5.232527in}}%
\pgfpathlineto{\pgfqpoint{2.628862in}{5.277808in}}%
\pgfpathlineto{\pgfqpoint{2.717582in}{5.315554in}}%
\pgfpathlineto{\pgfqpoint{2.828275in}{5.359874in}}%
\pgfpathlineto{\pgfqpoint{2.924596in}{5.395948in}}%
\pgfpathlineto{\pgfqpoint{3.013316in}{5.427282in}}%
\pgfpathlineto{\pgfqpoint{3.117372in}{5.461752in}}%
\pgfpathlineto{\pgfqpoint{3.220330in}{5.493412in}}%
\pgfpathlineto{\pgfqpoint{3.338623in}{5.526868in}}%
\pgfpathlineto{\pgfqpoint{3.427343in}{5.549928in}}%
\pgfpathlineto{\pgfqpoint{3.516063in}{5.571232in}}%
\pgfpathlineto{\pgfqpoint{3.604783in}{5.590774in}}%
\pgfpathlineto{\pgfqpoint{3.723077in}{5.613898in}}%
\pgfpathlineto{\pgfqpoint{3.811797in}{5.628907in}}%
\pgfpathlineto{\pgfqpoint{3.886176in}{5.640039in}}%
\pgfpathlineto{\pgfqpoint{3.886176in}{5.640039in}}%
\pgfusepath{stroke}%
\end{pgfscope}%
\begin{pgfscope}%
\pgfpathrectangle{\pgfqpoint{0.854460in}{0.571603in}}{\pgfqpoint{5.885100in}{5.068436in}}%
\pgfusepath{clip}%
\pgfsetbuttcap%
\pgfsetroundjoin%
\pgfsetlinewidth{1.505625pt}%
\definecolor{currentstroke}{rgb}{0.179019,0.433756,0.557430}%
\pgfsetstrokecolor{currentstroke}%
\pgfsetdash{}{0pt}%
\pgfpathmoveto{\pgfqpoint{4.649720in}{5.640039in}}%
\pgfpathlineto{\pgfqpoint{4.698998in}{5.628676in}}%
\pgfpathlineto{\pgfqpoint{4.758145in}{5.611895in}}%
\pgfpathlineto{\pgfqpoint{4.817291in}{5.590698in}}%
\pgfpathlineto{\pgfqpoint{4.846865in}{5.578015in}}%
\pgfpathlineto{\pgfqpoint{4.877169in}{5.563630in}}%
\pgfpathlineto{\pgfqpoint{4.922579in}{5.538161in}}%
\pgfpathlineto{\pgfqpoint{4.960719in}{5.512691in}}%
\pgfpathlineto{\pgfqpoint{4.965158in}{5.509403in}}%
\pgfpathlineto{\pgfqpoint{4.994732in}{5.485842in}}%
\pgfpathlineto{\pgfqpoint{5.024305in}{5.458392in}}%
\pgfpathlineto{\pgfqpoint{5.053878in}{5.425965in}}%
\pgfpathlineto{\pgfqpoint{5.066120in}{5.410813in}}%
\pgfpathlineto{\pgfqpoint{5.084649in}{5.385344in}}%
\pgfpathlineto{\pgfqpoint{5.100776in}{5.359874in}}%
\pgfpathlineto{\pgfqpoint{5.115087in}{5.334405in}}%
\pgfpathlineto{\pgfqpoint{5.127560in}{5.308935in}}%
\pgfpathlineto{\pgfqpoint{5.142599in}{5.273272in}}%
\pgfpathlineto{\pgfqpoint{5.156753in}{5.232527in}}%
\pgfpathlineto{\pgfqpoint{5.170667in}{5.181588in}}%
\pgfpathlineto{\pgfqpoint{5.180976in}{5.130649in}}%
\pgfpathlineto{\pgfqpoint{5.188457in}{5.079709in}}%
\pgfpathlineto{\pgfqpoint{5.193598in}{5.028770in}}%
\pgfpathlineto{\pgfqpoint{5.196789in}{4.977831in}}%
\pgfpathlineto{\pgfqpoint{5.198625in}{4.901423in}}%
\pgfpathlineto{\pgfqpoint{5.197722in}{4.825014in}}%
\pgfpathlineto{\pgfqpoint{5.193512in}{4.723136in}}%
\pgfpathlineto{\pgfqpoint{5.185294in}{4.595788in}}%
\pgfpathlineto{\pgfqpoint{5.156138in}{4.188276in}}%
\pgfpathlineto{\pgfqpoint{5.150426in}{4.060928in}}%
\pgfpathlineto{\pgfqpoint{5.148021in}{3.959050in}}%
\pgfpathlineto{\pgfqpoint{5.147877in}{3.857172in}}%
\pgfpathlineto{\pgfqpoint{5.150262in}{3.755294in}}%
\pgfpathlineto{\pgfqpoint{5.155417in}{3.653416in}}%
\pgfpathlineto{\pgfqpoint{5.161236in}{3.577007in}}%
\pgfpathlineto{\pgfqpoint{5.168826in}{3.500599in}}%
\pgfpathlineto{\pgfqpoint{5.178203in}{3.424190in}}%
\pgfpathlineto{\pgfqpoint{5.189459in}{3.347782in}}%
\pgfpathlineto{\pgfqpoint{5.202701in}{3.271373in}}%
\pgfpathlineto{\pgfqpoint{5.217862in}{3.194965in}}%
\pgfpathlineto{\pgfqpoint{5.235115in}{3.118556in}}%
\pgfpathlineto{\pgfqpoint{5.254413in}{3.042147in}}%
\pgfpathlineto{\pgfqpoint{5.275820in}{2.965739in}}%
\pgfpathlineto{\pgfqpoint{5.299392in}{2.889330in}}%
\pgfpathlineto{\pgfqpoint{5.325143in}{2.812922in}}%
\pgfpathlineto{\pgfqpoint{5.353090in}{2.736513in}}%
\pgfpathlineto{\pgfqpoint{5.383253in}{2.660104in}}%
\pgfpathlineto{\pgfqpoint{5.415653in}{2.583696in}}%
\pgfpathlineto{\pgfqpoint{5.450313in}{2.507287in}}%
\pgfpathlineto{\pgfqpoint{5.487261in}{2.430879in}}%
\pgfpathlineto{\pgfqpoint{5.527052in}{2.353485in}}%
\pgfpathlineto{\pgfqpoint{5.568022in}{2.278062in}}%
\pgfpathlineto{\pgfqpoint{5.615772in}{2.195110in}}%
\pgfpathlineto{\pgfqpoint{5.658002in}{2.125244in}}%
\pgfpathlineto{\pgfqpoint{5.706489in}{2.048836in}}%
\pgfpathlineto{\pgfqpoint{5.763639in}{1.963118in}}%
\pgfpathlineto{\pgfqpoint{5.810332in}{1.896019in}}%
\pgfpathlineto{\pgfqpoint{5.881933in}{1.797925in}}%
\pgfpathlineto{\pgfqpoint{5.923462in}{1.743202in}}%
\pgfpathlineto{\pgfqpoint{6.000226in}{1.646052in}}%
\pgfpathlineto{\pgfqpoint{6.059373in}{1.574188in}}%
\pgfpathlineto{\pgfqpoint{6.118520in}{1.504662in}}%
\pgfpathlineto{\pgfqpoint{6.177666in}{1.437249in}}%
\pgfpathlineto{\pgfqpoint{6.266386in}{1.339716in}}%
\pgfpathlineto{\pgfqpoint{6.325533in}{1.276837in}}%
\pgfpathlineto{\pgfqpoint{6.414253in}{1.185399in}}%
\pgfpathlineto{\pgfqpoint{6.493379in}{1.106463in}}%
\pgfpathlineto{\pgfqpoint{6.599030in}{1.004585in}}%
\pgfpathlineto{\pgfqpoint{6.709987in}{0.901421in}}%
\pgfpathlineto{\pgfqpoint{6.739560in}{0.874519in}}%
\pgfpathlineto{\pgfqpoint{6.739560in}{0.874519in}}%
\pgfusepath{stroke}%
\end{pgfscope}%
\begin{pgfscope}%
\pgfpathrectangle{\pgfqpoint{0.854460in}{0.571603in}}{\pgfqpoint{5.885100in}{5.068436in}}%
\pgfusepath{clip}%
\pgfsetbuttcap%
\pgfsetroundjoin%
\pgfsetlinewidth{1.505625pt}%
\definecolor{currentstroke}{rgb}{0.169646,0.456262,0.558030}%
\pgfsetstrokecolor{currentstroke}%
\pgfsetdash{}{0pt}%
\pgfpathmoveto{\pgfqpoint{1.614939in}{0.571603in}}%
\pgfpathlineto{\pgfqpoint{1.593795in}{0.590032in}}%
\pgfpathlineto{\pgfqpoint{1.585752in}{0.597073in}}%
\pgfpathlineto{\pgfqpoint{1.564221in}{0.616199in}}%
\pgfpathlineto{\pgfqpoint{1.557113in}{0.622542in}}%
\pgfpathlineto{\pgfqpoint{1.534648in}{0.642887in}}%
\pgfpathlineto{\pgfqpoint{1.529015in}{0.648012in}}%
\pgfpathlineto{\pgfqpoint{1.505074in}{0.670118in}}%
\pgfpathlineto{\pgfqpoint{1.501451in}{0.673481in}}%
\pgfpathlineto{\pgfqpoint{1.475501in}{0.697918in}}%
\pgfpathlineto{\pgfqpoint{1.474410in}{0.698951in}}%
\pgfpathlineto{\pgfqpoint{1.447914in}{0.724420in}}%
\pgfpathlineto{\pgfqpoint{1.445928in}{0.726359in}}%
\pgfpathlineto{\pgfqpoint{1.421949in}{0.749890in}}%
\pgfpathlineto{\pgfqpoint{1.416354in}{0.755462in}}%
\pgfpathlineto{\pgfqpoint{1.396488in}{0.775360in}}%
\pgfpathlineto{\pgfqpoint{1.386781in}{0.785228in}}%
\pgfpathlineto{\pgfqpoint{1.371524in}{0.800829in}}%
\pgfpathlineto{\pgfqpoint{1.357208in}{0.815687in}}%
\pgfpathlineto{\pgfqpoint{1.347045in}{0.826299in}}%
\pgfpathlineto{\pgfqpoint{1.327634in}{0.846870in}}%
\pgfpathlineto{\pgfqpoint{1.323041in}{0.851768in}}%
\pgfpathlineto{\pgfqpoint{1.299525in}{0.877238in}}%
\pgfpathlineto{\pgfqpoint{1.298061in}{0.878849in}}%
\pgfpathlineto{\pgfqpoint{1.276538in}{0.902707in}}%
\pgfpathlineto{\pgfqpoint{1.268488in}{0.911766in}}%
\pgfpathlineto{\pgfqpoint{1.254002in}{0.928177in}}%
\pgfpathlineto{\pgfqpoint{1.238914in}{0.945530in}}%
\pgfpathlineto{\pgfqpoint{1.231906in}{0.953646in}}%
\pgfpathlineto{\pgfqpoint{1.210252in}{0.979116in}}%
\pgfpathlineto{\pgfqpoint{1.209341in}{0.980207in}}%
\pgfpathlineto{\pgfqpoint{1.189128in}{1.004585in}}%
\pgfpathlineto{\pgfqpoint{1.179767in}{1.016050in}}%
\pgfpathlineto{\pgfqpoint{1.168416in}{1.030055in}}%
\pgfpathlineto{\pgfqpoint{1.150194in}{1.052886in}}%
\pgfpathlineto{\pgfqpoint{1.148105in}{1.055524in}}%
\pgfpathlineto{\pgfqpoint{1.128297in}{1.080994in}}%
\pgfpathlineto{\pgfqpoint{1.120621in}{1.091028in}}%
\pgfpathlineto{\pgfqpoint{1.108904in}{1.106463in}}%
\pgfpathlineto{\pgfqpoint{1.091047in}{1.130361in}}%
\pgfpathlineto{\pgfqpoint{1.089882in}{1.131933in}}%
\pgfpathlineto{\pgfqpoint{1.071368in}{1.157402in}}%
\pgfpathlineto{\pgfqpoint{1.061474in}{1.171239in}}%
\pgfpathlineto{\pgfqpoint{1.053224in}{1.182872in}}%
\pgfpathlineto{\pgfqpoint{1.035474in}{1.208341in}}%
\pgfpathlineto{\pgfqpoint{1.031901in}{1.213570in}}%
\pgfpathlineto{\pgfqpoint{1.018184in}{1.233811in}}%
\pgfpathlineto{\pgfqpoint{1.002327in}{1.257597in}}%
\pgfpathlineto{\pgfqpoint{1.001215in}{1.259281in}}%
\pgfpathlineto{\pgfqpoint{0.984736in}{1.284750in}}%
\pgfpathlineto{\pgfqpoint{0.972754in}{1.303589in}}%
\pgfpathlineto{\pgfqpoint{0.968574in}{1.310220in}}%
\pgfpathlineto{\pgfqpoint{0.952849in}{1.335689in}}%
\pgfpathlineto{\pgfqpoint{0.943181in}{1.351645in}}%
\pgfpathlineto{\pgfqpoint{0.937469in}{1.361159in}}%
\pgfpathlineto{\pgfqpoint{0.922493in}{1.386628in}}%
\pgfpathlineto{\pgfqpoint{0.913607in}{1.402042in}}%
\pgfpathlineto{\pgfqpoint{0.907865in}{1.412098in}}%
\pgfpathlineto{\pgfqpoint{0.893633in}{1.437567in}}%
\pgfpathlineto{\pgfqpoint{0.884034in}{1.455096in}}%
\pgfpathlineto{\pgfqpoint{0.879728in}{1.463037in}}%
\pgfpathlineto{\pgfqpoint{0.866232in}{1.488506in}}%
\pgfpathlineto{\pgfqpoint{0.854460in}{1.511174in}}%
\pgfusepath{stroke}%
\end{pgfscope}%
\begin{pgfscope}%
\pgfpathrectangle{\pgfqpoint{0.854460in}{0.571603in}}{\pgfqpoint{5.885100in}{5.068436in}}%
\pgfusepath{clip}%
\pgfsetbuttcap%
\pgfsetroundjoin%
\pgfsetlinewidth{1.505625pt}%
\definecolor{currentstroke}{rgb}{0.169646,0.456262,0.558030}%
\pgfsetstrokecolor{currentstroke}%
\pgfsetdash{}{0pt}%
\pgfpathmoveto{\pgfqpoint{0.854460in}{3.810981in}}%
\pgfpathlineto{\pgfqpoint{0.891977in}{3.882642in}}%
\pgfpathlineto{\pgfqpoint{0.935116in}{3.959050in}}%
\pgfpathlineto{\pgfqpoint{0.972754in}{4.021268in}}%
\pgfpathlineto{\pgfqpoint{1.014719in}{4.086398in}}%
\pgfpathlineto{\pgfqpoint{1.061474in}{4.154536in}}%
\pgfpathlineto{\pgfqpoint{1.104669in}{4.213746in}}%
\pgfpathlineto{\pgfqpoint{1.150194in}{4.272858in}}%
\pgfpathlineto{\pgfqpoint{1.209341in}{4.345115in}}%
\pgfpathlineto{\pgfqpoint{1.272681in}{4.417502in}}%
\pgfpathlineto{\pgfqpoint{1.327634in}{4.476630in}}%
\pgfpathlineto{\pgfqpoint{1.394801in}{4.544849in}}%
\pgfpathlineto{\pgfqpoint{1.447707in}{4.595788in}}%
\pgfpathlineto{\pgfqpoint{1.505074in}{4.648428in}}%
\pgfpathlineto{\pgfqpoint{1.564221in}{4.700158in}}%
\pgfpathlineto{\pgfqpoint{1.623368in}{4.749551in}}%
\pgfpathlineto{\pgfqpoint{1.686103in}{4.799545in}}%
\pgfpathlineto{\pgfqpoint{1.771235in}{4.863814in}}%
\pgfpathlineto{\pgfqpoint{1.830381in}{4.906309in}}%
\pgfpathlineto{\pgfqpoint{1.919102in}{4.966887in}}%
\pgfpathlineto{\pgfqpoint{1.978248in}{5.005415in}}%
\pgfpathlineto{\pgfqpoint{2.066968in}{5.060458in}}%
\pgfpathlineto{\pgfqpoint{2.155689in}{5.112570in}}%
\pgfpathlineto{\pgfqpoint{2.244409in}{5.161967in}}%
\pgfpathlineto{\pgfqpoint{2.333129in}{5.208850in}}%
\pgfpathlineto{\pgfqpoint{2.431459in}{5.257996in}}%
\pgfpathlineto{\pgfqpoint{2.540142in}{5.309215in}}%
\pgfpathlineto{\pgfqpoint{2.658436in}{5.361392in}}%
\pgfpathlineto{\pgfqpoint{2.778412in}{5.410813in}}%
\pgfpathlineto{\pgfqpoint{2.895023in}{5.455587in}}%
\pgfpathlineto{\pgfqpoint{2.983743in}{5.487703in}}%
\pgfpathlineto{\pgfqpoint{3.102036in}{5.527759in}}%
\pgfpathlineto{\pgfqpoint{3.190756in}{5.555930in}}%
\pgfpathlineto{\pgfqpoint{3.302396in}{5.589100in}}%
\pgfpathlineto{\pgfqpoint{3.397770in}{5.615401in}}%
\pgfpathlineto{\pgfqpoint{3.494076in}{5.640039in}}%
\pgfpathlineto{\pgfqpoint{3.494076in}{5.640039in}}%
\pgfusepath{stroke}%
\end{pgfscope}%
\begin{pgfscope}%
\pgfpathrectangle{\pgfqpoint{0.854460in}{0.571603in}}{\pgfqpoint{5.885100in}{5.068436in}}%
\pgfusepath{clip}%
\pgfsetbuttcap%
\pgfsetroundjoin%
\pgfsetlinewidth{1.505625pt}%
\definecolor{currentstroke}{rgb}{0.169646,0.456262,0.558030}%
\pgfsetstrokecolor{currentstroke}%
\pgfsetdash{}{0pt}%
\pgfpathmoveto{\pgfqpoint{5.057130in}{5.640039in}}%
\pgfpathlineto{\pgfqpoint{5.097244in}{5.614570in}}%
\pgfpathlineto{\pgfqpoint{5.131244in}{5.589100in}}%
\pgfpathlineto{\pgfqpoint{5.160345in}{5.563630in}}%
\pgfpathlineto{\pgfqpoint{5.185468in}{5.538161in}}%
\pgfpathlineto{\pgfqpoint{5.207326in}{5.512691in}}%
\pgfpathlineto{\pgfqpoint{5.231319in}{5.479870in}}%
\pgfpathlineto{\pgfqpoint{5.242946in}{5.461752in}}%
\pgfpathlineto{\pgfqpoint{5.260892in}{5.429783in}}%
\pgfpathlineto{\pgfqpoint{5.270228in}{5.410813in}}%
\pgfpathlineto{\pgfqpoint{5.281351in}{5.385344in}}%
\pgfpathlineto{\pgfqpoint{5.291106in}{5.359874in}}%
\pgfpathlineto{\pgfqpoint{5.306748in}{5.308935in}}%
\pgfpathlineto{\pgfqpoint{5.318399in}{5.257996in}}%
\pgfpathlineto{\pgfqpoint{5.326602in}{5.207057in}}%
\pgfpathlineto{\pgfqpoint{5.332056in}{5.156118in}}%
\pgfpathlineto{\pgfqpoint{5.335238in}{5.105179in}}%
\pgfpathlineto{\pgfqpoint{5.336523in}{5.054240in}}%
\pgfpathlineto{\pgfqpoint{5.335585in}{4.977831in}}%
\pgfpathlineto{\pgfqpoint{5.331986in}{4.901423in}}%
\pgfpathlineto{\pgfqpoint{5.324268in}{4.799545in}}%
\pgfpathlineto{\pgfqpoint{5.311646in}{4.672197in}}%
\pgfpathlineto{\pgfqpoint{5.268436in}{4.264685in}}%
\pgfpathlineto{\pgfqpoint{5.258302in}{4.137337in}}%
\pgfpathlineto{\pgfqpoint{5.252307in}{4.035459in}}%
\pgfpathlineto{\pgfqpoint{5.248580in}{3.933581in}}%
\pgfpathlineto{\pgfqpoint{5.247385in}{3.831703in}}%
\pgfpathlineto{\pgfqpoint{5.248952in}{3.729825in}}%
\pgfpathlineto{\pgfqpoint{5.252069in}{3.653416in}}%
\pgfpathlineto{\pgfqpoint{5.256942in}{3.577007in}}%
\pgfpathlineto{\pgfqpoint{5.263623in}{3.500599in}}%
\pgfpathlineto{\pgfqpoint{5.272155in}{3.424190in}}%
\pgfpathlineto{\pgfqpoint{5.282655in}{3.347782in}}%
\pgfpathlineto{\pgfqpoint{5.295149in}{3.271373in}}%
\pgfpathlineto{\pgfqpoint{5.309656in}{3.194965in}}%
\pgfpathlineto{\pgfqpoint{5.326267in}{3.118556in}}%
\pgfpathlineto{\pgfqpoint{5.344993in}{3.042147in}}%
\pgfpathlineto{\pgfqpoint{5.365848in}{2.965739in}}%
\pgfpathlineto{\pgfqpoint{5.388909in}{2.889330in}}%
\pgfpathlineto{\pgfqpoint{5.414187in}{2.812922in}}%
\pgfpathlineto{\pgfqpoint{5.441698in}{2.736513in}}%
\pgfpathlineto{\pgfqpoint{5.471459in}{2.660104in}}%
\pgfpathlineto{\pgfqpoint{5.503487in}{2.583696in}}%
\pgfpathlineto{\pgfqpoint{5.537806in}{2.507287in}}%
\pgfpathlineto{\pgfqpoint{5.574441in}{2.430879in}}%
\pgfpathlineto{\pgfqpoint{5.615772in}{2.350043in}}%
\pgfpathlineto{\pgfqpoint{5.654677in}{2.278062in}}%
\pgfpathlineto{\pgfqpoint{5.704492in}{2.191206in}}%
\pgfpathlineto{\pgfqpoint{5.744244in}{2.125244in}}%
\pgfpathlineto{\pgfqpoint{5.793213in}{2.047856in}}%
\pgfpathlineto{\pgfqpoint{5.852359in}{1.959033in}}%
\pgfpathlineto{\pgfqpoint{5.896172in}{1.896019in}}%
\pgfpathlineto{\pgfqpoint{5.951509in}{1.819610in}}%
\pgfpathlineto{\pgfqpoint{6.009188in}{1.743202in}}%
\pgfpathlineto{\pgfqpoint{6.069200in}{1.666793in}}%
\pgfpathlineto{\pgfqpoint{6.148093in}{1.570618in}}%
\pgfpathlineto{\pgfqpoint{6.207240in}{1.501266in}}%
\pgfpathlineto{\pgfqpoint{6.266386in}{1.434038in}}%
\pgfpathlineto{\pgfqpoint{6.332500in}{1.361159in}}%
\pgfpathlineto{\pgfqpoint{6.428449in}{1.259281in}}%
\pgfpathlineto{\pgfqpoint{6.528474in}{1.157402in}}%
\pgfpathlineto{\pgfqpoint{6.632478in}{1.055524in}}%
\pgfpathlineto{\pgfqpoint{6.739560in}{0.954520in}}%
\pgfpathlineto{\pgfqpoint{6.739560in}{0.954520in}}%
\pgfusepath{stroke}%
\end{pgfscope}%
\begin{pgfscope}%
\pgfpathrectangle{\pgfqpoint{0.854460in}{0.571603in}}{\pgfqpoint{5.885100in}{5.068436in}}%
\pgfusepath{clip}%
\pgfsetbuttcap%
\pgfsetroundjoin%
\pgfsetlinewidth{1.505625pt}%
\definecolor{currentstroke}{rgb}{0.162142,0.474838,0.558140}%
\pgfsetstrokecolor{currentstroke}%
\pgfsetdash{}{0pt}%
\pgfpathmoveto{\pgfqpoint{1.543857in}{0.571603in}}%
\pgfpathlineto{\pgfqpoint{1.534648in}{0.579699in}}%
\pgfpathlineto{\pgfqpoint{1.514974in}{0.597073in}}%
\pgfpathlineto{\pgfqpoint{1.505074in}{0.605943in}}%
\pgfpathlineto{\pgfqpoint{1.486635in}{0.622542in}}%
\pgfpathlineto{\pgfqpoint{1.475501in}{0.632712in}}%
\pgfpathlineto{\pgfqpoint{1.458833in}{0.648012in}}%
\pgfpathlineto{\pgfqpoint{1.445928in}{0.660031in}}%
\pgfpathlineto{\pgfqpoint{1.431558in}{0.673481in}}%
\pgfpathlineto{\pgfqpoint{1.416354in}{0.687922in}}%
\pgfpathlineto{\pgfqpoint{1.404804in}{0.698951in}}%
\pgfpathlineto{\pgfqpoint{1.386781in}{0.716413in}}%
\pgfpathlineto{\pgfqpoint{1.378561in}{0.724420in}}%
\pgfpathlineto{\pgfqpoint{1.357208in}{0.745528in}}%
\pgfpathlineto{\pgfqpoint{1.352819in}{0.749890in}}%
\pgfpathlineto{\pgfqpoint{1.327634in}{0.775295in}}%
\pgfpathlineto{\pgfqpoint{1.327571in}{0.775360in}}%
\pgfpathlineto{\pgfqpoint{1.302874in}{0.800829in}}%
\pgfpathlineto{\pgfqpoint{1.298061in}{0.805867in}}%
\pgfpathlineto{\pgfqpoint{1.278659in}{0.826299in}}%
\pgfpathlineto{\pgfqpoint{1.268488in}{0.837170in}}%
\pgfpathlineto{\pgfqpoint{1.254914in}{0.851768in}}%
\pgfpathlineto{\pgfqpoint{1.238914in}{0.869234in}}%
\pgfpathlineto{\pgfqpoint{1.231629in}{0.877238in}}%
\pgfpathlineto{\pgfqpoint{1.209341in}{0.902093in}}%
\pgfpathlineto{\pgfqpoint{1.208794in}{0.902707in}}%
\pgfpathlineto{\pgfqpoint{1.186494in}{0.928177in}}%
\pgfpathlineto{\pgfqpoint{1.179767in}{0.935978in}}%
\pgfpathlineto{\pgfqpoint{1.164638in}{0.953646in}}%
\pgfpathlineto{\pgfqpoint{1.150194in}{0.970770in}}%
\pgfpathlineto{\pgfqpoint{1.143204in}{0.979116in}}%
\pgfpathlineto{\pgfqpoint{1.122206in}{1.004585in}}%
\pgfpathlineto{\pgfqpoint{1.120621in}{1.006543in}}%
\pgfpathlineto{\pgfqpoint{1.101719in}{1.030055in}}%
\pgfpathlineto{\pgfqpoint{1.091047in}{1.043535in}}%
\pgfpathlineto{\pgfqpoint{1.081627in}{1.055524in}}%
\pgfpathlineto{\pgfqpoint{1.061925in}{1.080994in}}%
\pgfpathlineto{\pgfqpoint{1.061474in}{1.081588in}}%
\pgfpathlineto{\pgfqpoint{1.042742in}{1.106463in}}%
\pgfpathlineto{\pgfqpoint{1.031901in}{1.121087in}}%
\pgfpathlineto{\pgfqpoint{1.023924in}{1.131933in}}%
\pgfpathlineto{\pgfqpoint{1.005507in}{1.157402in}}%
\pgfpathlineto{\pgfqpoint{1.002327in}{1.161884in}}%
\pgfpathlineto{\pgfqpoint{0.987558in}{1.182872in}}%
\pgfpathlineto{\pgfqpoint{0.972754in}{1.204249in}}%
\pgfpathlineto{\pgfqpoint{0.969944in}{1.208341in}}%
\pgfpathlineto{\pgfqpoint{0.952796in}{1.233811in}}%
\pgfpathlineto{\pgfqpoint{0.943181in}{1.248343in}}%
\pgfpathlineto{\pgfqpoint{0.936006in}{1.259281in}}%
\pgfpathlineto{\pgfqpoint{0.919610in}{1.284750in}}%
\pgfpathlineto{\pgfqpoint{0.913607in}{1.294258in}}%
\pgfpathlineto{\pgfqpoint{0.903619in}{1.310220in}}%
\pgfpathlineto{\pgfqpoint{0.887971in}{1.335689in}}%
\pgfpathlineto{\pgfqpoint{0.884034in}{1.342232in}}%
\pgfpathlineto{\pgfqpoint{0.872751in}{1.361159in}}%
\pgfpathlineto{\pgfqpoint{0.857846in}{1.386628in}}%
\pgfpathlineto{\pgfqpoint{0.854460in}{1.392541in}}%
\pgfusepath{stroke}%
\end{pgfscope}%
\begin{pgfscope}%
\pgfpathrectangle{\pgfqpoint{0.854460in}{0.571603in}}{\pgfqpoint{5.885100in}{5.068436in}}%
\pgfusepath{clip}%
\pgfsetbuttcap%
\pgfsetroundjoin%
\pgfsetlinewidth{1.505625pt}%
\definecolor{currentstroke}{rgb}{0.162142,0.474838,0.558140}%
\pgfsetstrokecolor{currentstroke}%
\pgfsetdash{}{0pt}%
\pgfpathmoveto{\pgfqpoint{0.854460in}{3.951854in}}%
\pgfpathlineto{\pgfqpoint{0.888318in}{4.009989in}}%
\pgfpathlineto{\pgfqpoint{0.935865in}{4.086398in}}%
\pgfpathlineto{\pgfqpoint{0.972754in}{4.142100in}}%
\pgfpathlineto{\pgfqpoint{1.023191in}{4.213746in}}%
\pgfpathlineto{\pgfqpoint{1.061474in}{4.265207in}}%
\pgfpathlineto{\pgfqpoint{1.121413in}{4.341093in}}%
\pgfpathlineto{\pgfqpoint{1.186126in}{4.417502in}}%
\pgfpathlineto{\pgfqpoint{1.238914in}{4.476180in}}%
\pgfpathlineto{\pgfqpoint{1.304317in}{4.544849in}}%
\pgfpathlineto{\pgfqpoint{1.357208in}{4.597470in}}%
\pgfpathlineto{\pgfqpoint{1.416354in}{4.653449in}}%
\pgfpathlineto{\pgfqpoint{1.475501in}{4.706766in}}%
\pgfpathlineto{\pgfqpoint{1.534648in}{4.757643in}}%
\pgfpathlineto{\pgfqpoint{1.593795in}{4.806283in}}%
\pgfpathlineto{\pgfqpoint{1.652941in}{4.852873in}}%
\pgfpathlineto{\pgfqpoint{1.717389in}{4.901423in}}%
\pgfpathlineto{\pgfqpoint{1.800808in}{4.961153in}}%
\pgfpathlineto{\pgfqpoint{1.862472in}{5.003301in}}%
\pgfpathlineto{\pgfqpoint{1.948675in}{5.059420in}}%
\pgfpathlineto{\pgfqpoint{2.037395in}{5.114142in}}%
\pgfpathlineto{\pgfqpoint{2.126115in}{5.166037in}}%
\pgfpathlineto{\pgfqpoint{2.214835in}{5.215314in}}%
\pgfpathlineto{\pgfqpoint{2.303555in}{5.262167in}}%
\pgfpathlineto{\pgfqpoint{2.396774in}{5.308935in}}%
\pgfpathlineto{\pgfqpoint{2.510569in}{5.362819in}}%
\pgfpathlineto{\pgfqpoint{2.628862in}{5.415370in}}%
\pgfpathlineto{\pgfqpoint{2.747156in}{5.464650in}}%
\pgfpathlineto{\pgfqpoint{2.870417in}{5.512691in}}%
\pgfpathlineto{\pgfqpoint{2.983743in}{5.553976in}}%
\pgfpathlineto{\pgfqpoint{3.086081in}{5.589100in}}%
\pgfpathlineto{\pgfqpoint{3.190756in}{5.622830in}}%
\pgfpathlineto{\pgfqpoint{3.246629in}{5.640039in}}%
\pgfpathlineto{\pgfqpoint{3.246629in}{5.640039in}}%
\pgfusepath{stroke}%
\end{pgfscope}%
\begin{pgfscope}%
\pgfpathrectangle{\pgfqpoint{0.854460in}{0.571603in}}{\pgfqpoint{5.885100in}{5.068436in}}%
\pgfusepath{clip}%
\pgfsetbuttcap%
\pgfsetroundjoin%
\pgfsetlinewidth{1.505625pt}%
\definecolor{currentstroke}{rgb}{0.162142,0.474838,0.558140}%
\pgfsetstrokecolor{currentstroke}%
\pgfsetdash{}{0pt}%
\pgfpathmoveto{\pgfqpoint{5.319889in}{5.640039in}}%
\pgfpathlineto{\pgfqpoint{5.343080in}{5.614570in}}%
\pgfpathlineto{\pgfqpoint{5.363234in}{5.589100in}}%
\pgfpathlineto{\pgfqpoint{5.380860in}{5.563630in}}%
\pgfpathlineto{\pgfqpoint{5.396085in}{5.538161in}}%
\pgfpathlineto{\pgfqpoint{5.409467in}{5.512691in}}%
\pgfpathlineto{\pgfqpoint{5.420967in}{5.487222in}}%
\pgfpathlineto{\pgfqpoint{5.438332in}{5.440450in}}%
\pgfpathlineto{\pgfqpoint{5.447080in}{5.410813in}}%
\pgfpathlineto{\pgfqpoint{5.458713in}{5.359874in}}%
\pgfpathlineto{\pgfqpoint{5.466774in}{5.308935in}}%
\pgfpathlineto{\pgfqpoint{5.471784in}{5.257996in}}%
\pgfpathlineto{\pgfqpoint{5.474329in}{5.207057in}}%
\pgfpathlineto{\pgfqpoint{5.474831in}{5.156118in}}%
\pgfpathlineto{\pgfqpoint{5.473630in}{5.105179in}}%
\pgfpathlineto{\pgfqpoint{5.469261in}{5.028770in}}%
\pgfpathlineto{\pgfqpoint{5.462445in}{4.952362in}}%
\pgfpathlineto{\pgfqpoint{5.450740in}{4.850484in}}%
\pgfpathlineto{\pgfqpoint{5.433594in}{4.723136in}}%
\pgfpathlineto{\pgfqpoint{5.383680in}{4.366563in}}%
\pgfpathlineto{\pgfqpoint{5.368694in}{4.239215in}}%
\pgfpathlineto{\pgfqpoint{5.358744in}{4.137337in}}%
\pgfpathlineto{\pgfqpoint{5.350973in}{4.035459in}}%
\pgfpathlineto{\pgfqpoint{5.345599in}{3.933581in}}%
\pgfpathlineto{\pgfqpoint{5.342913in}{3.831703in}}%
\pgfpathlineto{\pgfqpoint{5.342799in}{3.755294in}}%
\pgfpathlineto{\pgfqpoint{5.344404in}{3.678886in}}%
\pgfpathlineto{\pgfqpoint{5.347804in}{3.602477in}}%
\pgfpathlineto{\pgfqpoint{5.353038in}{3.526068in}}%
\pgfpathlineto{\pgfqpoint{5.360175in}{3.449660in}}%
\pgfpathlineto{\pgfqpoint{5.369303in}{3.373251in}}%
\pgfpathlineto{\pgfqpoint{5.380478in}{3.296843in}}%
\pgfpathlineto{\pgfqpoint{5.393661in}{3.220434in}}%
\pgfpathlineto{\pgfqpoint{5.409018in}{3.144025in}}%
\pgfpathlineto{\pgfqpoint{5.426450in}{3.067617in}}%
\pgfpathlineto{\pgfqpoint{5.446103in}{2.991208in}}%
\pgfpathlineto{\pgfqpoint{5.467985in}{2.914800in}}%
\pgfpathlineto{\pgfqpoint{5.492053in}{2.838391in}}%
\pgfpathlineto{\pgfqpoint{5.518394in}{2.761983in}}%
\pgfpathlineto{\pgfqpoint{5.547023in}{2.685574in}}%
\pgfpathlineto{\pgfqpoint{5.577957in}{2.609165in}}%
\pgfpathlineto{\pgfqpoint{5.615772in}{2.522783in}}%
\pgfpathlineto{\pgfqpoint{5.646802in}{2.456348in}}%
\pgfpathlineto{\pgfqpoint{5.684696in}{2.379940in}}%
\pgfpathlineto{\pgfqpoint{5.724967in}{2.303531in}}%
\pgfpathlineto{\pgfqpoint{5.767609in}{2.227123in}}%
\pgfpathlineto{\pgfqpoint{5.812588in}{2.150714in}}%
\pgfpathlineto{\pgfqpoint{5.859957in}{2.074305in}}%
\pgfpathlineto{\pgfqpoint{5.911506in}{1.995231in}}%
\pgfpathlineto{\pgfqpoint{5.970653in}{1.908913in}}%
\pgfpathlineto{\pgfqpoint{6.016274in}{1.845080in}}%
\pgfpathlineto{\pgfqpoint{6.088946in}{1.748015in}}%
\pgfpathlineto{\pgfqpoint{6.132344in}{1.692262in}}%
\pgfpathlineto{\pgfqpoint{6.207240in}{1.599764in}}%
\pgfpathlineto{\pgfqpoint{6.266386in}{1.529518in}}%
\pgfpathlineto{\pgfqpoint{6.325533in}{1.461507in}}%
\pgfpathlineto{\pgfqpoint{6.392778in}{1.386628in}}%
\pgfpathlineto{\pgfqpoint{6.487894in}{1.284750in}}%
\pgfpathlineto{\pgfqpoint{6.587147in}{1.182872in}}%
\pgfpathlineto{\pgfqpoint{6.690454in}{1.080994in}}%
\pgfpathlineto{\pgfqpoint{6.739560in}{1.033950in}}%
\pgfpathlineto{\pgfqpoint{6.739560in}{1.033950in}}%
\pgfusepath{stroke}%
\end{pgfscope}%
\begin{pgfscope}%
\pgfpathrectangle{\pgfqpoint{0.854460in}{0.571603in}}{\pgfqpoint{5.885100in}{5.068436in}}%
\pgfusepath{clip}%
\pgfsetbuttcap%
\pgfsetroundjoin%
\pgfsetlinewidth{1.505625pt}%
\definecolor{currentstroke}{rgb}{0.154815,0.493313,0.557840}%
\pgfsetstrokecolor{currentstroke}%
\pgfsetdash{}{0pt}%
\pgfpathmoveto{\pgfqpoint{1.475048in}{0.571603in}}%
\pgfpathlineto{\pgfqpoint{1.446453in}{0.597073in}}%
\pgfpathlineto{\pgfqpoint{1.445928in}{0.597547in}}%
\pgfpathlineto{\pgfqpoint{1.418403in}{0.622542in}}%
\pgfpathlineto{\pgfqpoint{1.416354in}{0.624430in}}%
\pgfpathlineto{\pgfqpoint{1.390884in}{0.648012in}}%
\pgfpathlineto{\pgfqpoint{1.386781in}{0.651867in}}%
\pgfpathlineto{\pgfqpoint{1.363889in}{0.673481in}}%
\pgfpathlineto{\pgfqpoint{1.357208in}{0.679883in}}%
\pgfpathlineto{\pgfqpoint{1.337409in}{0.698951in}}%
\pgfpathlineto{\pgfqpoint{1.327634in}{0.708503in}}%
\pgfpathlineto{\pgfqpoint{1.311435in}{0.724420in}}%
\pgfpathlineto{\pgfqpoint{1.298061in}{0.737754in}}%
\pgfpathlineto{\pgfqpoint{1.285957in}{0.749890in}}%
\pgfpathlineto{\pgfqpoint{1.268488in}{0.767664in}}%
\pgfpathlineto{\pgfqpoint{1.260967in}{0.775360in}}%
\pgfpathlineto{\pgfqpoint{1.238914in}{0.798260in}}%
\pgfpathlineto{\pgfqpoint{1.236455in}{0.800829in}}%
\pgfpathlineto{\pgfqpoint{1.212455in}{0.826299in}}%
\pgfpathlineto{\pgfqpoint{1.209341in}{0.829655in}}%
\pgfpathlineto{\pgfqpoint{1.188955in}{0.851768in}}%
\pgfpathlineto{\pgfqpoint{1.179767in}{0.861883in}}%
\pgfpathlineto{\pgfqpoint{1.165909in}{0.877238in}}%
\pgfpathlineto{\pgfqpoint{1.150194in}{0.894911in}}%
\pgfpathlineto{\pgfqpoint{1.143308in}{0.902707in}}%
\pgfpathlineto{\pgfqpoint{1.121147in}{0.928177in}}%
\pgfpathlineto{\pgfqpoint{1.120621in}{0.928792in}}%
\pgfpathlineto{\pgfqpoint{1.099515in}{0.953646in}}%
\pgfpathlineto{\pgfqpoint{1.091047in}{0.963769in}}%
\pgfpathlineto{\pgfqpoint{1.078301in}{0.979116in}}%
\pgfpathlineto{\pgfqpoint{1.061474in}{0.999684in}}%
\pgfpathlineto{\pgfqpoint{1.057493in}{1.004585in}}%
\pgfpathlineto{\pgfqpoint{1.037156in}{1.030055in}}%
\pgfpathlineto{\pgfqpoint{1.031901in}{1.036748in}}%
\pgfpathlineto{\pgfqpoint{1.017268in}{1.055524in}}%
\pgfpathlineto{\pgfqpoint{1.002327in}{1.074993in}}%
\pgfpathlineto{\pgfqpoint{0.997757in}{1.080994in}}%
\pgfpathlineto{\pgfqpoint{0.978700in}{1.106463in}}%
\pgfpathlineto{\pgfqpoint{0.972754in}{1.114549in}}%
\pgfpathlineto{\pgfqpoint{0.960071in}{1.131933in}}%
\pgfpathlineto{\pgfqpoint{0.943181in}{1.155450in}}%
\pgfpathlineto{\pgfqpoint{0.941790in}{1.157402in}}%
\pgfpathlineto{\pgfqpoint{0.923998in}{1.182872in}}%
\pgfpathlineto{\pgfqpoint{0.913607in}{1.197993in}}%
\pgfpathlineto{\pgfqpoint{0.906556in}{1.208341in}}%
\pgfpathlineto{\pgfqpoint{0.889513in}{1.233811in}}%
\pgfpathlineto{\pgfqpoint{0.884034in}{1.242155in}}%
\pgfpathlineto{\pgfqpoint{0.872886in}{1.259281in}}%
\pgfpathlineto{\pgfqpoint{0.856588in}{1.284750in}}%
\pgfpathlineto{\pgfqpoint{0.854460in}{1.288145in}}%
\pgfusepath{stroke}%
\end{pgfscope}%
\begin{pgfscope}%
\pgfpathrectangle{\pgfqpoint{0.854460in}{0.571603in}}{\pgfqpoint{5.885100in}{5.068436in}}%
\pgfusepath{clip}%
\pgfsetbuttcap%
\pgfsetroundjoin%
\pgfsetlinewidth{1.505625pt}%
\definecolor{currentstroke}{rgb}{0.154815,0.493313,0.557840}%
\pgfsetstrokecolor{currentstroke}%
\pgfsetdash{}{0pt}%
\pgfpathmoveto{\pgfqpoint{0.854460in}{4.076472in}}%
\pgfpathlineto{\pgfqpoint{0.893449in}{4.137337in}}%
\pgfpathlineto{\pgfqpoint{0.945599in}{4.213746in}}%
\pgfpathlineto{\pgfqpoint{1.002327in}{4.291133in}}%
\pgfpathlineto{\pgfqpoint{1.061603in}{4.366563in}}%
\pgfpathlineto{\pgfqpoint{1.125931in}{4.442971in}}%
\pgfpathlineto{\pgfqpoint{1.179767in}{4.503162in}}%
\pgfpathlineto{\pgfqpoint{1.243284in}{4.570319in}}%
\pgfpathlineto{\pgfqpoint{1.298061in}{4.625175in}}%
\pgfpathlineto{\pgfqpoint{1.357208in}{4.681536in}}%
\pgfpathlineto{\pgfqpoint{1.416354in}{4.735238in}}%
\pgfpathlineto{\pgfqpoint{1.475501in}{4.786499in}}%
\pgfpathlineto{\pgfqpoint{1.534648in}{4.835521in}}%
\pgfpathlineto{\pgfqpoint{1.593795in}{4.882491in}}%
\pgfpathlineto{\pgfqpoint{1.652941in}{4.927581in}}%
\pgfpathlineto{\pgfqpoint{1.741661in}{4.991807in}}%
\pgfpathlineto{\pgfqpoint{1.800808in}{5.032673in}}%
\pgfpathlineto{\pgfqpoint{1.889528in}{5.091116in}}%
\pgfpathlineto{\pgfqpoint{1.978248in}{5.146520in}}%
\pgfpathlineto{\pgfqpoint{2.066968in}{5.199105in}}%
\pgfpathlineto{\pgfqpoint{2.155689in}{5.249079in}}%
\pgfpathlineto{\pgfqpoint{2.244409in}{5.296637in}}%
\pgfpathlineto{\pgfqpoint{2.333129in}{5.341960in}}%
\pgfpathlineto{\pgfqpoint{2.422111in}{5.385344in}}%
\pgfpathlineto{\pgfqpoint{2.540142in}{5.439724in}}%
\pgfpathlineto{\pgfqpoint{2.658436in}{5.490978in}}%
\pgfpathlineto{\pgfqpoint{2.776729in}{5.539188in}}%
\pgfpathlineto{\pgfqpoint{2.907587in}{5.589100in}}%
\pgfpathlineto{\pgfqpoint{3.013316in}{5.626946in}}%
\pgfpathlineto{\pgfqpoint{3.051169in}{5.640039in}}%
\pgfpathlineto{\pgfqpoint{3.051169in}{5.640039in}}%
\pgfusepath{stroke}%
\end{pgfscope}%
\begin{pgfscope}%
\pgfpathrectangle{\pgfqpoint{0.854460in}{0.571603in}}{\pgfqpoint{5.885100in}{5.068436in}}%
\pgfusepath{clip}%
\pgfsetbuttcap%
\pgfsetroundjoin%
\pgfsetlinewidth{1.505625pt}%
\definecolor{currentstroke}{rgb}{0.154815,0.493313,0.557840}%
\pgfsetstrokecolor{currentstroke}%
\pgfsetdash{}{0pt}%
\pgfpathmoveto{\pgfqpoint{5.530714in}{5.640039in}}%
\pgfpathlineto{\pgfqpoint{5.545001in}{5.614570in}}%
\pgfpathlineto{\pgfqpoint{5.557469in}{5.589100in}}%
\pgfpathlineto{\pgfqpoint{5.568107in}{5.563630in}}%
\pgfpathlineto{\pgfqpoint{5.577306in}{5.538161in}}%
\pgfpathlineto{\pgfqpoint{5.586199in}{5.508947in}}%
\pgfpathlineto{\pgfqpoint{5.597305in}{5.461752in}}%
\pgfpathlineto{\pgfqpoint{5.605513in}{5.410813in}}%
\pgfpathlineto{\pgfqpoint{5.610481in}{5.359874in}}%
\pgfpathlineto{\pgfqpoint{5.612741in}{5.308935in}}%
\pgfpathlineto{\pgfqpoint{5.612738in}{5.257996in}}%
\pgfpathlineto{\pgfqpoint{5.610849in}{5.207057in}}%
\pgfpathlineto{\pgfqpoint{5.605162in}{5.130649in}}%
\pgfpathlineto{\pgfqpoint{5.596829in}{5.054240in}}%
\pgfpathlineto{\pgfqpoint{5.582769in}{4.952362in}}%
\pgfpathlineto{\pgfqpoint{5.562141in}{4.825014in}}%
\pgfpathlineto{\pgfqpoint{5.484748in}{4.366563in}}%
\pgfpathlineto{\pgfqpoint{5.467906in}{4.244394in}}%
\pgfpathlineto{\pgfqpoint{5.458170in}{4.162807in}}%
\pgfpathlineto{\pgfqpoint{5.448127in}{4.060928in}}%
\pgfpathlineto{\pgfqpoint{5.440645in}{3.959050in}}%
\pgfpathlineto{\pgfqpoint{5.435910in}{3.857172in}}%
\pgfpathlineto{\pgfqpoint{5.434295in}{3.780764in}}%
\pgfpathlineto{\pgfqpoint{5.434440in}{3.704355in}}%
\pgfpathlineto{\pgfqpoint{5.436415in}{3.627946in}}%
\pgfpathlineto{\pgfqpoint{5.440270in}{3.551538in}}%
\pgfpathlineto{\pgfqpoint{5.446050in}{3.475129in}}%
\pgfpathlineto{\pgfqpoint{5.453842in}{3.398721in}}%
\pgfpathlineto{\pgfqpoint{5.463709in}{3.322312in}}%
\pgfpathlineto{\pgfqpoint{5.475639in}{3.245904in}}%
\pgfpathlineto{\pgfqpoint{5.489712in}{3.169495in}}%
\pgfpathlineto{\pgfqpoint{5.505952in}{3.093086in}}%
\pgfpathlineto{\pgfqpoint{5.527052in}{3.006607in}}%
\pgfpathlineto{\pgfqpoint{5.545066in}{2.940269in}}%
\pgfpathlineto{\pgfqpoint{5.567999in}{2.863861in}}%
\pgfpathlineto{\pgfqpoint{5.593221in}{2.787452in}}%
\pgfpathlineto{\pgfqpoint{5.620744in}{2.711044in}}%
\pgfpathlineto{\pgfqpoint{5.650582in}{2.634635in}}%
\pgfpathlineto{\pgfqpoint{5.682752in}{2.558226in}}%
\pgfpathlineto{\pgfqpoint{5.717274in}{2.481818in}}%
\pgfpathlineto{\pgfqpoint{5.754170in}{2.405409in}}%
\pgfpathlineto{\pgfqpoint{5.793464in}{2.329001in}}%
\pgfpathlineto{\pgfqpoint{5.835080in}{2.252592in}}%
\pgfpathlineto{\pgfqpoint{5.881933in}{2.171515in}}%
\pgfpathlineto{\pgfqpoint{5.925544in}{2.099775in}}%
\pgfpathlineto{\pgfqpoint{5.974388in}{2.023366in}}%
\pgfpathlineto{\pgfqpoint{6.029799in}{1.940909in}}%
\pgfpathlineto{\pgfqpoint{6.088946in}{1.857110in}}%
\pgfpathlineto{\pgfqpoint{6.135230in}{1.794141in}}%
\pgfpathlineto{\pgfqpoint{6.207240in}{1.700440in}}%
\pgfpathlineto{\pgfqpoint{6.266386in}{1.626715in}}%
\pgfpathlineto{\pgfqpoint{6.325533in}{1.555607in}}%
\pgfpathlineto{\pgfqpoint{6.384680in}{1.486844in}}%
\pgfpathlineto{\pgfqpoint{6.451135in}{1.412098in}}%
\pgfpathlineto{\pgfqpoint{6.545379in}{1.310220in}}%
\pgfpathlineto{\pgfqpoint{6.643814in}{1.208341in}}%
\pgfpathlineto{\pgfqpoint{6.739560in}{1.113142in}}%
\pgfpathlineto{\pgfqpoint{6.739560in}{1.113142in}}%
\pgfusepath{stroke}%
\end{pgfscope}%
\begin{pgfscope}%
\pgfpathrectangle{\pgfqpoint{0.854460in}{0.571603in}}{\pgfqpoint{5.885100in}{5.068436in}}%
\pgfusepath{clip}%
\pgfsetbuttcap%
\pgfsetroundjoin%
\pgfsetlinewidth{1.505625pt}%
\definecolor{currentstroke}{rgb}{0.146180,0.515413,0.556823}%
\pgfsetstrokecolor{currentstroke}%
\pgfsetdash{}{0pt}%
\pgfpathmoveto{\pgfqpoint{1.408457in}{0.571603in}}%
\pgfpathlineto{\pgfqpoint{1.386781in}{0.591073in}}%
\pgfpathlineto{\pgfqpoint{1.380134in}{0.597073in}}%
\pgfpathlineto{\pgfqpoint{1.357208in}{0.618062in}}%
\pgfpathlineto{\pgfqpoint{1.352338in}{0.622542in}}%
\pgfpathlineto{\pgfqpoint{1.327634in}{0.645598in}}%
\pgfpathlineto{\pgfqpoint{1.325061in}{0.648012in}}%
\pgfpathlineto{\pgfqpoint{1.298298in}{0.673481in}}%
\pgfpathlineto{\pgfqpoint{1.298061in}{0.673711in}}%
\pgfpathlineto{\pgfqpoint{1.272081in}{0.698951in}}%
\pgfpathlineto{\pgfqpoint{1.268488in}{0.702493in}}%
\pgfpathlineto{\pgfqpoint{1.246364in}{0.724420in}}%
\pgfpathlineto{\pgfqpoint{1.238914in}{0.731912in}}%
\pgfpathlineto{\pgfqpoint{1.221138in}{0.749890in}}%
\pgfpathlineto{\pgfqpoint{1.209341in}{0.761996in}}%
\pgfpathlineto{\pgfqpoint{1.196395in}{0.775360in}}%
\pgfpathlineto{\pgfqpoint{1.179767in}{0.792774in}}%
\pgfpathlineto{\pgfqpoint{1.172124in}{0.800829in}}%
\pgfpathlineto{\pgfqpoint{1.150194in}{0.824277in}}%
\pgfpathlineto{\pgfqpoint{1.148315in}{0.826299in}}%
\pgfpathlineto{\pgfqpoint{1.125020in}{0.851768in}}%
\pgfpathlineto{\pgfqpoint{1.120621in}{0.856653in}}%
\pgfpathlineto{\pgfqpoint{1.102201in}{0.877238in}}%
\pgfpathlineto{\pgfqpoint{1.091047in}{0.889888in}}%
\pgfpathlineto{\pgfqpoint{1.079820in}{0.902707in}}%
\pgfpathlineto{\pgfqpoint{1.061474in}{0.923967in}}%
\pgfpathlineto{\pgfqpoint{1.057866in}{0.928177in}}%
\pgfpathlineto{\pgfqpoint{1.036393in}{0.953646in}}%
\pgfpathlineto{\pgfqpoint{1.031901in}{0.959062in}}%
\pgfpathlineto{\pgfqpoint{1.015384in}{0.979116in}}%
\pgfpathlineto{\pgfqpoint{1.002327in}{0.995209in}}%
\pgfpathlineto{\pgfqpoint{0.994776in}{1.004585in}}%
\pgfpathlineto{\pgfqpoint{0.974583in}{1.030055in}}%
\pgfpathlineto{\pgfqpoint{0.972754in}{1.032403in}}%
\pgfpathlineto{\pgfqpoint{0.954883in}{1.055524in}}%
\pgfpathlineto{\pgfqpoint{0.943181in}{1.070900in}}%
\pgfpathlineto{\pgfqpoint{0.935556in}{1.080994in}}%
\pgfpathlineto{\pgfqpoint{0.916634in}{1.106463in}}%
\pgfpathlineto{\pgfqpoint{0.913607in}{1.110613in}}%
\pgfpathlineto{\pgfqpoint{0.898178in}{1.131933in}}%
\pgfpathlineto{\pgfqpoint{0.884034in}{1.151785in}}%
\pgfpathlineto{\pgfqpoint{0.880064in}{1.157402in}}%
\pgfpathlineto{\pgfqpoint{0.862397in}{1.182872in}}%
\pgfpathlineto{\pgfqpoint{0.854460in}{1.194513in}}%
\pgfusepath{stroke}%
\end{pgfscope}%
\begin{pgfscope}%
\pgfpathrectangle{\pgfqpoint{0.854460in}{0.571603in}}{\pgfqpoint{5.885100in}{5.068436in}}%
\pgfusepath{clip}%
\pgfsetbuttcap%
\pgfsetroundjoin%
\pgfsetlinewidth{1.505625pt}%
\definecolor{currentstroke}{rgb}{0.146180,0.515413,0.556823}%
\pgfsetstrokecolor{currentstroke}%
\pgfsetdash{}{0pt}%
\pgfpathmoveto{\pgfqpoint{0.854460in}{4.188757in}}%
\pgfpathlineto{\pgfqpoint{0.871553in}{4.213746in}}%
\pgfpathlineto{\pgfqpoint{0.884034in}{4.231730in}}%
\pgfpathlineto{\pgfqpoint{0.889296in}{4.239215in}}%
\pgfpathlineto{\pgfqpoint{0.907487in}{4.264685in}}%
\pgfpathlineto{\pgfqpoint{0.913607in}{4.273122in}}%
\pgfpathlineto{\pgfqpoint{0.926120in}{4.290154in}}%
\pgfpathlineto{\pgfqpoint{0.943181in}{4.313056in}}%
\pgfpathlineto{\pgfqpoint{0.945118in}{4.315624in}}%
\pgfpathlineto{\pgfqpoint{0.964640in}{4.341093in}}%
\pgfpathlineto{\pgfqpoint{0.972754in}{4.351530in}}%
\pgfpathlineto{\pgfqpoint{0.984587in}{4.366563in}}%
\pgfpathlineto{\pgfqpoint{1.002327in}{4.388798in}}%
\pgfpathlineto{\pgfqpoint{1.004940in}{4.392032in}}%
\pgfpathlineto{\pgfqpoint{1.025823in}{4.417502in}}%
\pgfpathlineto{\pgfqpoint{1.031901in}{4.424812in}}%
\pgfpathlineto{\pgfqpoint{1.047183in}{4.442971in}}%
\pgfpathlineto{\pgfqpoint{1.061474in}{4.459731in}}%
\pgfpathlineto{\pgfqpoint{1.068990in}{4.468441in}}%
\pgfpathlineto{\pgfqpoint{1.091047in}{4.493669in}}%
\pgfpathlineto{\pgfqpoint{1.091261in}{4.493910in}}%
\pgfpathlineto{\pgfqpoint{1.114127in}{4.519380in}}%
\pgfpathlineto{\pgfqpoint{1.120621in}{4.526520in}}%
\pgfpathlineto{\pgfqpoint{1.137486in}{4.544849in}}%
\pgfpathlineto{\pgfqpoint{1.150194in}{4.558486in}}%
\pgfpathlineto{\pgfqpoint{1.161349in}{4.570319in}}%
\pgfpathlineto{\pgfqpoint{1.179767in}{4.589610in}}%
\pgfpathlineto{\pgfqpoint{1.185733in}{4.595788in}}%
\pgfpathlineto{\pgfqpoint{1.209341in}{4.619929in}}%
\pgfpathlineto{\pgfqpoint{1.210655in}{4.621258in}}%
\pgfpathlineto{\pgfqpoint{1.236180in}{4.646728in}}%
\pgfpathlineto{\pgfqpoint{1.238914in}{4.649421in}}%
\pgfpathlineto{\pgfqpoint{1.262289in}{4.672197in}}%
\pgfpathlineto{\pgfqpoint{1.268488in}{4.678162in}}%
\pgfpathlineto{\pgfqpoint{1.288976in}{4.697667in}}%
\pgfpathlineto{\pgfqpoint{1.298061in}{4.706209in}}%
\pgfpathlineto{\pgfqpoint{1.316256in}{4.723136in}}%
\pgfpathlineto{\pgfqpoint{1.327634in}{4.733591in}}%
\pgfpathlineto{\pgfqpoint{1.344146in}{4.748606in}}%
\pgfpathlineto{\pgfqpoint{1.357208in}{4.760337in}}%
\pgfpathlineto{\pgfqpoint{1.372663in}{4.774075in}}%
\pgfpathlineto{\pgfqpoint{1.386781in}{4.786472in}}%
\pgfpathlineto{\pgfqpoint{1.401822in}{4.799545in}}%
\pgfpathlineto{\pgfqpoint{1.416354in}{4.812022in}}%
\pgfpathlineto{\pgfqpoint{1.431639in}{4.825014in}}%
\pgfpathlineto{\pgfqpoint{1.445928in}{4.837012in}}%
\pgfpathlineto{\pgfqpoint{1.462130in}{4.850484in}}%
\pgfpathlineto{\pgfqpoint{1.475501in}{4.861466in}}%
\pgfpathlineto{\pgfqpoint{1.493312in}{4.875953in}}%
\pgfpathlineto{\pgfqpoint{1.505074in}{4.885405in}}%
\pgfpathlineto{\pgfqpoint{1.525198in}{4.901423in}}%
\pgfpathlineto{\pgfqpoint{1.534648in}{4.908853in}}%
\pgfpathlineto{\pgfqpoint{1.557806in}{4.926892in}}%
\pgfpathlineto{\pgfqpoint{1.564221in}{4.931830in}}%
\pgfpathlineto{\pgfqpoint{1.591148in}{4.952362in}}%
\pgfpathlineto{\pgfqpoint{1.593795in}{4.954355in}}%
\pgfpathlineto{\pgfqpoint{1.623368in}{4.976422in}}%
\pgfpathlineto{\pgfqpoint{1.625276in}{4.977831in}}%
\pgfpathlineto{\pgfqpoint{1.652941in}{4.998027in}}%
\pgfpathlineto{\pgfqpoint{1.660230in}{5.003301in}}%
\pgfpathlineto{\pgfqpoint{1.682515in}{5.019230in}}%
\pgfpathlineto{\pgfqpoint{1.695978in}{5.028770in}}%
\pgfpathlineto{\pgfqpoint{1.712088in}{5.040049in}}%
\pgfpathlineto{\pgfqpoint{1.732531in}{5.054240in}}%
\pgfpathlineto{\pgfqpoint{1.741661in}{5.060501in}}%
\pgfpathlineto{\pgfqpoint{1.769903in}{5.079709in}}%
\pgfpathlineto{\pgfqpoint{1.771235in}{5.080604in}}%
\pgfpathlineto{\pgfqpoint{1.800808in}{5.100274in}}%
\pgfpathlineto{\pgfqpoint{1.808245in}{5.105179in}}%
\pgfpathlineto{\pgfqpoint{1.830381in}{5.119602in}}%
\pgfpathlineto{\pgfqpoint{1.847472in}{5.130649in}}%
\pgfpathlineto{\pgfqpoint{1.859955in}{5.138620in}}%
\pgfpathlineto{\pgfqpoint{1.887569in}{5.156118in}}%
\pgfpathlineto{\pgfqpoint{1.889528in}{5.157345in}}%
\pgfpathlineto{\pgfqpoint{1.919102in}{5.175670in}}%
\pgfpathlineto{\pgfqpoint{1.928728in}{5.181588in}}%
\pgfpathlineto{\pgfqpoint{1.948675in}{5.193700in}}%
\pgfpathlineto{\pgfqpoint{1.970834in}{5.207057in}}%
\pgfpathlineto{\pgfqpoint{1.978248in}{5.211472in}}%
\pgfpathlineto{\pgfqpoint{2.007822in}{5.228928in}}%
\pgfpathlineto{\pgfqpoint{2.013974in}{5.232527in}}%
\pgfpathlineto{\pgfqpoint{2.037395in}{5.246057in}}%
\pgfpathlineto{\pgfqpoint{2.058203in}{5.257996in}}%
\pgfpathlineto{\pgfqpoint{2.066968in}{5.262964in}}%
\pgfpathlineto{\pgfqpoint{2.096542in}{5.279582in}}%
\pgfpathlineto{\pgfqpoint{2.103518in}{5.283466in}}%
\pgfpathlineto{\pgfqpoint{2.126115in}{5.295893in}}%
\pgfpathlineto{\pgfqpoint{2.149982in}{5.308935in}}%
\pgfpathlineto{\pgfqpoint{2.155689in}{5.312015in}}%
\pgfpathlineto{\pgfqpoint{2.185262in}{5.327823in}}%
\pgfpathlineto{\pgfqpoint{2.197671in}{5.334405in}}%
\pgfpathlineto{\pgfqpoint{2.214835in}{5.343396in}}%
\pgfpathlineto{\pgfqpoint{2.244409in}{5.358787in}}%
\pgfpathlineto{\pgfqpoint{2.246521in}{5.359874in}}%
\pgfpathlineto{\pgfqpoint{2.273982in}{5.373835in}}%
\pgfpathlineto{\pgfqpoint{2.296747in}{5.385344in}}%
\pgfpathlineto{\pgfqpoint{2.303555in}{5.388743in}}%
\pgfpathlineto{\pgfqpoint{2.333129in}{5.403364in}}%
\pgfpathlineto{\pgfqpoint{2.348308in}{5.410813in}}%
\pgfpathlineto{\pgfqpoint{2.362702in}{5.417789in}}%
\pgfpathlineto{\pgfqpoint{2.392275in}{5.432011in}}%
\pgfpathlineto{\pgfqpoint{2.401242in}{5.436283in}}%
\pgfpathlineto{\pgfqpoint{2.421849in}{5.445977in}}%
\pgfpathlineto{\pgfqpoint{2.451422in}{5.459805in}}%
\pgfpathlineto{\pgfqpoint{2.455632in}{5.461752in}}%
\pgfpathlineto{\pgfqpoint{2.480996in}{5.473335in}}%
\pgfpathlineto{\pgfqpoint{2.510569in}{5.486774in}}%
\pgfpathlineto{\pgfqpoint{2.511566in}{5.487222in}}%
\pgfpathlineto{\pgfqpoint{2.540142in}{5.499889in}}%
\pgfpathlineto{\pgfqpoint{2.569147in}{5.512691in}}%
\pgfpathlineto{\pgfqpoint{2.569716in}{5.512939in}}%
\pgfpathlineto{\pgfqpoint{2.599289in}{5.525666in}}%
\pgfpathlineto{\pgfqpoint{2.628444in}{5.538161in}}%
\pgfpathlineto{\pgfqpoint{2.628862in}{5.538338in}}%
\pgfpathlineto{\pgfqpoint{2.658436in}{5.550691in}}%
\pgfpathlineto{\pgfqpoint{2.688009in}{5.562991in}}%
\pgfpathlineto{\pgfqpoint{2.689565in}{5.563630in}}%
\pgfpathlineto{\pgfqpoint{2.717582in}{5.574988in}}%
\pgfpathlineto{\pgfqpoint{2.747156in}{5.586917in}}%
\pgfpathlineto{\pgfqpoint{2.752632in}{5.589100in}}%
\pgfpathlineto{\pgfqpoint{2.776729in}{5.598581in}}%
\pgfpathlineto{\pgfqpoint{2.806303in}{5.610143in}}%
\pgfpathlineto{\pgfqpoint{2.817744in}{5.614570in}}%
\pgfpathlineto{\pgfqpoint{2.835876in}{5.621494in}}%
\pgfpathlineto{\pgfqpoint{2.865449in}{5.632692in}}%
\pgfpathlineto{\pgfqpoint{2.885014in}{5.640039in}}%
\pgfusepath{stroke}%
\end{pgfscope}%
\begin{pgfscope}%
\pgfpathrectangle{\pgfqpoint{0.854460in}{0.571603in}}{\pgfqpoint{5.885100in}{5.068436in}}%
\pgfusepath{clip}%
\pgfsetbuttcap%
\pgfsetroundjoin%
\pgfsetlinewidth{1.505625pt}%
\definecolor{currentstroke}{rgb}{0.146180,0.515413,0.556823}%
\pgfsetstrokecolor{currentstroke}%
\pgfsetdash{}{0pt}%
\pgfpathmoveto{\pgfqpoint{5.712248in}{5.640039in}}%
\pgfpathlineto{\pgfqpoint{5.720805in}{5.614570in}}%
\pgfpathlineto{\pgfqpoint{5.728054in}{5.589100in}}%
\pgfpathlineto{\pgfqpoint{5.734110in}{5.563630in}}%
\pgfpathlineto{\pgfqpoint{5.742908in}{5.512691in}}%
\pgfpathlineto{\pgfqpoint{5.748150in}{5.461752in}}%
\pgfpathlineto{\pgfqpoint{5.750421in}{5.410813in}}%
\pgfpathlineto{\pgfqpoint{5.750206in}{5.359874in}}%
\pgfpathlineto{\pgfqpoint{5.747915in}{5.308935in}}%
\pgfpathlineto{\pgfqpoint{5.743893in}{5.257996in}}%
\pgfpathlineto{\pgfqpoint{5.735248in}{5.181588in}}%
\pgfpathlineto{\pgfqpoint{5.724056in}{5.105179in}}%
\pgfpathlineto{\pgfqpoint{5.706472in}{5.003301in}}%
\pgfpathlineto{\pgfqpoint{5.681692in}{4.875953in}}%
\pgfpathlineto{\pgfqpoint{5.595356in}{4.442971in}}%
\pgfpathlineto{\pgfqpoint{5.573702in}{4.315624in}}%
\pgfpathlineto{\pgfqpoint{5.558628in}{4.213746in}}%
\pgfpathlineto{\pgfqpoint{5.545798in}{4.111867in}}%
\pgfpathlineto{\pgfqpoint{5.535562in}{4.009989in}}%
\pgfpathlineto{\pgfqpoint{5.528140in}{3.908111in}}%
\pgfpathlineto{\pgfqpoint{5.524504in}{3.831703in}}%
\pgfpathlineto{\pgfqpoint{5.522630in}{3.755294in}}%
\pgfpathlineto{\pgfqpoint{5.522596in}{3.678886in}}%
\pgfpathlineto{\pgfqpoint{5.524466in}{3.602477in}}%
\pgfpathlineto{\pgfqpoint{5.528291in}{3.526068in}}%
\pgfpathlineto{\pgfqpoint{5.534097in}{3.449660in}}%
\pgfpathlineto{\pgfqpoint{5.541972in}{3.373251in}}%
\pgfpathlineto{\pgfqpoint{5.551975in}{3.296843in}}%
\pgfpathlineto{\pgfqpoint{5.564095in}{3.220434in}}%
\pgfpathlineto{\pgfqpoint{5.578401in}{3.144025in}}%
\pgfpathlineto{\pgfqpoint{5.594919in}{3.067617in}}%
\pgfpathlineto{\pgfqpoint{5.615772in}{2.983436in}}%
\pgfpathlineto{\pgfqpoint{5.634710in}{2.914800in}}%
\pgfpathlineto{\pgfqpoint{5.658033in}{2.838391in}}%
\pgfpathlineto{\pgfqpoint{5.683680in}{2.761983in}}%
\pgfpathlineto{\pgfqpoint{5.711661in}{2.685574in}}%
\pgfpathlineto{\pgfqpoint{5.741988in}{2.609165in}}%
\pgfpathlineto{\pgfqpoint{5.774680in}{2.532757in}}%
\pgfpathlineto{\pgfqpoint{5.809755in}{2.456348in}}%
\pgfpathlineto{\pgfqpoint{5.852359in}{2.369950in}}%
\pgfpathlineto{\pgfqpoint{5.887104in}{2.303531in}}%
\pgfpathlineto{\pgfqpoint{5.929366in}{2.227123in}}%
\pgfpathlineto{\pgfqpoint{5.974072in}{2.150714in}}%
\pgfpathlineto{\pgfqpoint{6.021164in}{2.074305in}}%
\pgfpathlineto{\pgfqpoint{6.070685in}{1.997897in}}%
\pgfpathlineto{\pgfqpoint{6.122646in}{1.921488in}}%
\pgfpathlineto{\pgfqpoint{6.177666in}{1.844207in}}%
\pgfpathlineto{\pgfqpoint{6.236813in}{1.764739in}}%
\pgfpathlineto{\pgfqpoint{6.295960in}{1.688557in}}%
\pgfpathlineto{\pgfqpoint{6.355107in}{1.615282in}}%
\pgfpathlineto{\pgfqpoint{6.418642in}{1.539445in}}%
\pgfpathlineto{\pgfqpoint{6.502973in}{1.442926in}}%
\pgfpathlineto{\pgfqpoint{6.562120in}{1.377689in}}%
\pgfpathlineto{\pgfqpoint{6.625093in}{1.310220in}}%
\pgfpathlineto{\pgfqpoint{6.723709in}{1.208341in}}%
\pgfpathlineto{\pgfqpoint{6.739560in}{1.192399in}}%
\pgfpathlineto{\pgfqpoint{6.739560in}{1.192399in}}%
\pgfusepath{stroke}%
\end{pgfscope}%
\begin{pgfscope}%
\pgfpathrectangle{\pgfqpoint{0.854460in}{0.571603in}}{\pgfqpoint{5.885100in}{5.068436in}}%
\pgfusepath{clip}%
\pgfsetbuttcap%
\pgfsetroundjoin%
\pgfsetlinewidth{1.505625pt}%
\definecolor{currentstroke}{rgb}{0.139147,0.533812,0.555298}%
\pgfsetstrokecolor{currentstroke}%
\pgfsetdash{}{0pt}%
\pgfpathmoveto{\pgfqpoint{1.343822in}{0.571603in}}%
\pgfpathlineto{\pgfqpoint{1.327634in}{0.586270in}}%
\pgfpathlineto{\pgfqpoint{1.315769in}{0.597073in}}%
\pgfpathlineto{\pgfqpoint{1.298061in}{0.613425in}}%
\pgfpathlineto{\pgfqpoint{1.288238in}{0.622542in}}%
\pgfpathlineto{\pgfqpoint{1.268488in}{0.641135in}}%
\pgfpathlineto{\pgfqpoint{1.261221in}{0.648012in}}%
\pgfpathlineto{\pgfqpoint{1.238914in}{0.669423in}}%
\pgfpathlineto{\pgfqpoint{1.234709in}{0.673481in}}%
\pgfpathlineto{\pgfqpoint{1.209341in}{0.698315in}}%
\pgfpathlineto{\pgfqpoint{1.208695in}{0.698951in}}%
\pgfpathlineto{\pgfqpoint{1.183215in}{0.724420in}}%
\pgfpathlineto{\pgfqpoint{1.179767in}{0.727917in}}%
\pgfpathlineto{\pgfqpoint{1.158230in}{0.749890in}}%
\pgfpathlineto{\pgfqpoint{1.150194in}{0.758208in}}%
\pgfpathlineto{\pgfqpoint{1.133721in}{0.775360in}}%
\pgfpathlineto{\pgfqpoint{1.120621in}{0.789200in}}%
\pgfpathlineto{\pgfqpoint{1.109680in}{0.800829in}}%
\pgfpathlineto{\pgfqpoint{1.091047in}{0.820923in}}%
\pgfpathlineto{\pgfqpoint{1.086094in}{0.826299in}}%
\pgfpathlineto{\pgfqpoint{1.062977in}{0.851768in}}%
\pgfpathlineto{\pgfqpoint{1.061474in}{0.853451in}}%
\pgfpathlineto{\pgfqpoint{1.040371in}{0.877238in}}%
\pgfpathlineto{\pgfqpoint{1.031901in}{0.886927in}}%
\pgfpathlineto{\pgfqpoint{1.018198in}{0.902707in}}%
\pgfpathlineto{\pgfqpoint{1.002327in}{0.921257in}}%
\pgfpathlineto{\pgfqpoint{0.996447in}{0.928177in}}%
\pgfpathlineto{\pgfqpoint{0.975140in}{0.953646in}}%
\pgfpathlineto{\pgfqpoint{0.972754in}{0.956547in}}%
\pgfpathlineto{\pgfqpoint{0.954323in}{0.979116in}}%
\pgfpathlineto{\pgfqpoint{0.943181in}{0.992965in}}%
\pgfpathlineto{\pgfqpoint{0.933900in}{1.004585in}}%
\pgfpathlineto{\pgfqpoint{0.913865in}{1.030055in}}%
\pgfpathlineto{\pgfqpoint{0.913607in}{1.030389in}}%
\pgfpathlineto{\pgfqpoint{0.894341in}{1.055524in}}%
\pgfpathlineto{\pgfqpoint{0.884034in}{1.069179in}}%
\pgfpathlineto{\pgfqpoint{0.875184in}{1.080994in}}%
\pgfpathlineto{\pgfqpoint{0.856410in}{1.106463in}}%
\pgfpathlineto{\pgfqpoint{0.854460in}{1.109158in}}%
\pgfusepath{stroke}%
\end{pgfscope}%
\begin{pgfscope}%
\pgfpathrectangle{\pgfqpoint{0.854460in}{0.571603in}}{\pgfqpoint{5.885100in}{5.068436in}}%
\pgfusepath{clip}%
\pgfsetbuttcap%
\pgfsetroundjoin%
\pgfsetlinewidth{1.505625pt}%
\definecolor{currentstroke}{rgb}{0.139147,0.533812,0.555298}%
\pgfsetstrokecolor{currentstroke}%
\pgfsetdash{}{0pt}%
\pgfpathmoveto{\pgfqpoint{0.854460in}{4.291107in}}%
\pgfpathlineto{\pgfqpoint{0.872412in}{4.315624in}}%
\pgfpathlineto{\pgfqpoint{0.884034in}{4.331276in}}%
\pgfpathlineto{\pgfqpoint{0.891414in}{4.341093in}}%
\pgfpathlineto{\pgfqpoint{0.910840in}{4.366563in}}%
\pgfpathlineto{\pgfqpoint{0.913607in}{4.370136in}}%
\pgfpathlineto{\pgfqpoint{0.930776in}{4.392032in}}%
\pgfpathlineto{\pgfqpoint{0.943181in}{4.407641in}}%
\pgfpathlineto{\pgfqpoint{0.951112in}{4.417502in}}%
\pgfpathlineto{\pgfqpoint{0.971880in}{4.442971in}}%
\pgfpathlineto{\pgfqpoint{0.972754in}{4.444027in}}%
\pgfpathlineto{\pgfqpoint{0.993207in}{4.468441in}}%
\pgfpathlineto{\pgfqpoint{1.002327in}{4.479184in}}%
\pgfpathlineto{\pgfqpoint{1.014975in}{4.493910in}}%
\pgfpathlineto{\pgfqpoint{1.031901in}{4.513362in}}%
\pgfpathlineto{\pgfqpoint{1.037199in}{4.519380in}}%
\pgfpathlineto{\pgfqpoint{1.059921in}{4.544849in}}%
\pgfpathlineto{\pgfqpoint{1.061474in}{4.546566in}}%
\pgfpathlineto{\pgfqpoint{1.083214in}{4.570319in}}%
\pgfpathlineto{\pgfqpoint{1.091047in}{4.578769in}}%
\pgfpathlineto{\pgfqpoint{1.107002in}{4.595788in}}%
\pgfpathlineto{\pgfqpoint{1.120621in}{4.610131in}}%
\pgfpathlineto{\pgfqpoint{1.131303in}{4.621258in}}%
\pgfpathlineto{\pgfqpoint{1.150194in}{4.640688in}}%
\pgfpathlineto{\pgfqpoint{1.156130in}{4.646728in}}%
\pgfpathlineto{\pgfqpoint{1.179767in}{4.670474in}}%
\pgfpathlineto{\pgfqpoint{1.181502in}{4.672197in}}%
\pgfpathlineto{\pgfqpoint{1.207465in}{4.697667in}}%
\pgfpathlineto{\pgfqpoint{1.209341in}{4.699483in}}%
\pgfpathlineto{\pgfqpoint{1.234025in}{4.723136in}}%
\pgfpathlineto{\pgfqpoint{1.238914in}{4.727763in}}%
\pgfpathlineto{\pgfqpoint{1.261166in}{4.748606in}}%
\pgfpathlineto{\pgfqpoint{1.268488in}{4.755378in}}%
\pgfpathlineto{\pgfqpoint{1.288905in}{4.774075in}}%
\pgfpathlineto{\pgfqpoint{1.298061in}{4.782356in}}%
\pgfpathlineto{\pgfqpoint{1.317257in}{4.799545in}}%
\pgfpathlineto{\pgfqpoint{1.327634in}{4.808723in}}%
\pgfpathlineto{\pgfqpoint{1.346238in}{4.825014in}}%
\pgfpathlineto{\pgfqpoint{1.357208in}{4.834504in}}%
\pgfpathlineto{\pgfqpoint{1.375862in}{4.850484in}}%
\pgfpathlineto{\pgfqpoint{1.386781in}{4.859723in}}%
\pgfpathlineto{\pgfqpoint{1.406146in}{4.875953in}}%
\pgfpathlineto{\pgfqpoint{1.416354in}{4.884405in}}%
\pgfpathlineto{\pgfqpoint{1.437104in}{4.901423in}}%
\pgfpathlineto{\pgfqpoint{1.445928in}{4.908572in}}%
\pgfpathlineto{\pgfqpoint{1.468751in}{4.926892in}}%
\pgfpathlineto{\pgfqpoint{1.475501in}{4.932245in}}%
\pgfpathlineto{\pgfqpoint{1.501101in}{4.952362in}}%
\pgfpathlineto{\pgfqpoint{1.505074in}{4.955446in}}%
\pgfpathlineto{\pgfqpoint{1.534170in}{4.977831in}}%
\pgfpathlineto{\pgfqpoint{1.534648in}{4.978195in}}%
\pgfpathlineto{\pgfqpoint{1.564221in}{5.000454in}}%
\pgfpathlineto{\pgfqpoint{1.568038in}{5.003301in}}%
\pgfpathlineto{\pgfqpoint{1.593795in}{5.022283in}}%
\pgfpathlineto{\pgfqpoint{1.602673in}{5.028770in}}%
\pgfpathlineto{\pgfqpoint{1.623368in}{5.043709in}}%
\pgfpathlineto{\pgfqpoint{1.638081in}{5.054240in}}%
\pgfpathlineto{\pgfqpoint{1.652941in}{5.064748in}}%
\pgfpathlineto{\pgfqpoint{1.674274in}{5.079709in}}%
\pgfpathlineto{\pgfqpoint{1.682515in}{5.085419in}}%
\pgfpathlineto{\pgfqpoint{1.711265in}{5.105179in}}%
\pgfpathlineto{\pgfqpoint{1.712088in}{5.105738in}}%
\pgfpathlineto{\pgfqpoint{1.741661in}{5.125620in}}%
\pgfpathlineto{\pgfqpoint{1.749201in}{5.130649in}}%
\pgfpathlineto{\pgfqpoint{1.771235in}{5.145165in}}%
\pgfpathlineto{\pgfqpoint{1.787988in}{5.156118in}}%
\pgfpathlineto{\pgfqpoint{1.800808in}{5.164398in}}%
\pgfpathlineto{\pgfqpoint{1.827622in}{5.181588in}}%
\pgfpathlineto{\pgfqpoint{1.830381in}{5.183335in}}%
\pgfpathlineto{\pgfqpoint{1.859955in}{5.201887in}}%
\pgfpathlineto{\pgfqpoint{1.868264in}{5.207057in}}%
\pgfpathlineto{\pgfqpoint{1.889528in}{5.220129in}}%
\pgfpathlineto{\pgfqpoint{1.909840in}{5.232527in}}%
\pgfpathlineto{\pgfqpoint{1.919102in}{5.238111in}}%
\pgfpathlineto{\pgfqpoint{1.948675in}{5.255804in}}%
\pgfpathlineto{\pgfqpoint{1.952375in}{5.257996in}}%
\pgfpathlineto{\pgfqpoint{1.978248in}{5.273143in}}%
\pgfpathlineto{\pgfqpoint{1.996000in}{5.283466in}}%
\pgfpathlineto{\pgfqpoint{2.007822in}{5.290257in}}%
\pgfpathlineto{\pgfqpoint{2.037395in}{5.307121in}}%
\pgfpathlineto{\pgfqpoint{2.040607in}{5.308935in}}%
\pgfpathlineto{\pgfqpoint{2.066968in}{5.323640in}}%
\pgfpathlineto{\pgfqpoint{2.086386in}{5.334405in}}%
\pgfpathlineto{\pgfqpoint{2.096542in}{5.339966in}}%
\pgfpathlineto{\pgfqpoint{2.126115in}{5.356032in}}%
\pgfpathlineto{\pgfqpoint{2.133251in}{5.359874in}}%
\pgfpathlineto{\pgfqpoint{2.155689in}{5.371808in}}%
\pgfpathlineto{\pgfqpoint{2.181285in}{5.385344in}}%
\pgfpathlineto{\pgfqpoint{2.185262in}{5.387421in}}%
\pgfpathlineto{\pgfqpoint{2.214835in}{5.402716in}}%
\pgfpathlineto{\pgfqpoint{2.230596in}{5.410813in}}%
\pgfpathlineto{\pgfqpoint{2.244409in}{5.417822in}}%
\pgfpathlineto{\pgfqpoint{2.273982in}{5.432721in}}%
\pgfpathlineto{\pgfqpoint{2.281118in}{5.436283in}}%
\pgfpathlineto{\pgfqpoint{2.303555in}{5.447344in}}%
\pgfpathlineto{\pgfqpoint{2.332928in}{5.461752in}}%
\pgfpathlineto{\pgfqpoint{2.333129in}{5.461850in}}%
\pgfpathlineto{\pgfqpoint{2.362702in}{5.476015in}}%
\pgfpathlineto{\pgfqpoint{2.386210in}{5.487222in}}%
\pgfpathlineto{\pgfqpoint{2.392275in}{5.490077in}}%
\pgfpathlineto{\pgfqpoint{2.421849in}{5.503863in}}%
\pgfpathlineto{\pgfqpoint{2.440904in}{5.512691in}}%
\pgfpathlineto{\pgfqpoint{2.451422in}{5.517503in}}%
\pgfpathlineto{\pgfqpoint{2.480996in}{5.530915in}}%
\pgfpathlineto{\pgfqpoint{2.497088in}{5.538161in}}%
\pgfpathlineto{\pgfqpoint{2.510569in}{5.544155in}}%
\pgfpathlineto{\pgfqpoint{2.540142in}{5.557197in}}%
\pgfpathlineto{\pgfqpoint{2.554842in}{5.563630in}}%
\pgfpathlineto{\pgfqpoint{2.569716in}{5.570057in}}%
\pgfpathlineto{\pgfqpoint{2.599289in}{5.582735in}}%
\pgfpathlineto{\pgfqpoint{2.614253in}{5.589100in}}%
\pgfpathlineto{\pgfqpoint{2.628862in}{5.595235in}}%
\pgfpathlineto{\pgfqpoint{2.658436in}{5.607553in}}%
\pgfpathlineto{\pgfqpoint{2.675409in}{5.614570in}}%
\pgfpathlineto{\pgfqpoint{2.688009in}{5.619712in}}%
\pgfpathlineto{\pgfqpoint{2.717582in}{5.631674in}}%
\pgfpathlineto{\pgfqpoint{2.738400in}{5.640039in}}%
\pgfusepath{stroke}%
\end{pgfscope}%
\begin{pgfscope}%
\pgfpathrectangle{\pgfqpoint{0.854460in}{0.571603in}}{\pgfqpoint{5.885100in}{5.068436in}}%
\pgfusepath{clip}%
\pgfsetbuttcap%
\pgfsetroundjoin%
\pgfsetlinewidth{1.505625pt}%
\definecolor{currentstroke}{rgb}{0.139147,0.533812,0.555298}%
\pgfsetstrokecolor{currentstroke}%
\pgfsetdash{}{0pt}%
\pgfpathmoveto{\pgfqpoint{5.874289in}{5.640039in}}%
\pgfpathlineto{\pgfqpoint{5.882237in}{5.589100in}}%
\pgfpathlineto{\pgfqpoint{5.886455in}{5.538161in}}%
\pgfpathlineto{\pgfqpoint{5.887663in}{5.487222in}}%
\pgfpathlineto{\pgfqpoint{5.886352in}{5.436283in}}%
\pgfpathlineto{\pgfqpoint{5.881933in}{5.374756in}}%
\pgfpathlineto{\pgfqpoint{5.877700in}{5.334405in}}%
\pgfpathlineto{\pgfqpoint{5.867141in}{5.257996in}}%
\pgfpathlineto{\pgfqpoint{5.852359in}{5.172379in}}%
\pgfpathlineto{\pgfqpoint{5.833778in}{5.079709in}}%
\pgfpathlineto{\pgfqpoint{5.805527in}{4.952362in}}%
\pgfpathlineto{\pgfqpoint{5.701027in}{4.493910in}}%
\pgfpathlineto{\pgfqpoint{5.680705in}{4.392032in}}%
\pgfpathlineto{\pgfqpoint{5.662326in}{4.290154in}}%
\pgfpathlineto{\pgfqpoint{5.645346in}{4.181980in}}%
\pgfpathlineto{\pgfqpoint{5.632620in}{4.086398in}}%
\pgfpathlineto{\pgfqpoint{5.621796in}{3.984520in}}%
\pgfpathlineto{\pgfqpoint{5.615626in}{3.908111in}}%
\pgfpathlineto{\pgfqpoint{5.611160in}{3.831703in}}%
\pgfpathlineto{\pgfqpoint{5.608523in}{3.755294in}}%
\pgfpathlineto{\pgfqpoint{5.607775in}{3.678886in}}%
\pgfpathlineto{\pgfqpoint{5.608976in}{3.602477in}}%
\pgfpathlineto{\pgfqpoint{5.612181in}{3.526068in}}%
\pgfpathlineto{\pgfqpoint{5.617433in}{3.449660in}}%
\pgfpathlineto{\pgfqpoint{5.624751in}{3.373251in}}%
\pgfpathlineto{\pgfqpoint{5.634224in}{3.296843in}}%
\pgfpathlineto{\pgfqpoint{5.645905in}{3.220434in}}%
\pgfpathlineto{\pgfqpoint{5.659734in}{3.144025in}}%
\pgfpathlineto{\pgfqpoint{5.675867in}{3.067617in}}%
\pgfpathlineto{\pgfqpoint{5.694214in}{2.991208in}}%
\pgfpathlineto{\pgfqpoint{5.714884in}{2.914800in}}%
\pgfpathlineto{\pgfqpoint{5.737890in}{2.838391in}}%
\pgfpathlineto{\pgfqpoint{5.763639in}{2.760836in}}%
\pgfpathlineto{\pgfqpoint{5.793213in}{2.679602in}}%
\pgfpathlineto{\pgfqpoint{5.822786in}{2.604832in}}%
\pgfpathlineto{\pgfqpoint{5.853442in}{2.532757in}}%
\pgfpathlineto{\pgfqpoint{5.888283in}{2.456348in}}%
\pgfpathlineto{\pgfqpoint{5.925544in}{2.379940in}}%
\pgfpathlineto{\pgfqpoint{5.970653in}{2.293566in}}%
\pgfpathlineto{\pgfqpoint{6.007368in}{2.227123in}}%
\pgfpathlineto{\pgfqpoint{6.059373in}{2.138418in}}%
\pgfpathlineto{\pgfqpoint{6.098915in}{2.074305in}}%
\pgfpathlineto{\pgfqpoint{6.148378in}{1.997897in}}%
\pgfpathlineto{\pgfqpoint{6.207240in}{1.911506in}}%
\pgfpathlineto{\pgfqpoint{6.254551in}{1.845080in}}%
\pgfpathlineto{\pgfqpoint{6.325533in}{1.750091in}}%
\pgfpathlineto{\pgfqpoint{6.370517in}{1.692262in}}%
\pgfpathlineto{\pgfqpoint{6.443827in}{1.601775in}}%
\pgfpathlineto{\pgfqpoint{6.502973in}{1.531614in}}%
\pgfpathlineto{\pgfqpoint{6.562749in}{1.463037in}}%
\pgfpathlineto{\pgfqpoint{6.631636in}{1.386628in}}%
\pgfpathlineto{\pgfqpoint{6.709987in}{1.302865in}}%
\pgfpathlineto{\pgfqpoint{6.739560in}{1.272014in}}%
\pgfpathlineto{\pgfqpoint{6.739560in}{1.272014in}}%
\pgfusepath{stroke}%
\end{pgfscope}%
\begin{pgfscope}%
\pgfpathrectangle{\pgfqpoint{0.854460in}{0.571603in}}{\pgfqpoint{5.885100in}{5.068436in}}%
\pgfusepath{clip}%
\pgfsetbuttcap%
\pgfsetroundjoin%
\pgfsetlinewidth{1.505625pt}%
\definecolor{currentstroke}{rgb}{0.131172,0.555899,0.552459}%
\pgfsetstrokecolor{currentstroke}%
\pgfsetdash{}{0pt}%
\pgfpathmoveto{\pgfqpoint{1.281021in}{0.571603in}}%
\pgfpathlineto{\pgfqpoint{1.268488in}{0.583058in}}%
\pgfpathlineto{\pgfqpoint{1.253228in}{0.597073in}}%
\pgfpathlineto{\pgfqpoint{1.238914in}{0.610407in}}%
\pgfpathlineto{\pgfqpoint{1.225952in}{0.622542in}}%
\pgfpathlineto{\pgfqpoint{1.209341in}{0.638317in}}%
\pgfpathlineto{\pgfqpoint{1.199185in}{0.648012in}}%
\pgfpathlineto{\pgfqpoint{1.179767in}{0.666813in}}%
\pgfpathlineto{\pgfqpoint{1.172918in}{0.673481in}}%
\pgfpathlineto{\pgfqpoint{1.150194in}{0.695921in}}%
\pgfpathlineto{\pgfqpoint{1.147142in}{0.698951in}}%
\pgfpathlineto{\pgfqpoint{1.121866in}{0.724420in}}%
\pgfpathlineto{\pgfqpoint{1.120621in}{0.725694in}}%
\pgfpathlineto{\pgfqpoint{1.097111in}{0.749890in}}%
\pgfpathlineto{\pgfqpoint{1.091047in}{0.756220in}}%
\pgfpathlineto{\pgfqpoint{1.072826in}{0.775360in}}%
\pgfpathlineto{\pgfqpoint{1.061474in}{0.787456in}}%
\pgfpathlineto{\pgfqpoint{1.049002in}{0.800829in}}%
\pgfpathlineto{\pgfqpoint{1.031901in}{0.819433in}}%
\pgfpathlineto{\pgfqpoint{1.025629in}{0.826299in}}%
\pgfpathlineto{\pgfqpoint{1.002702in}{0.851768in}}%
\pgfpathlineto{\pgfqpoint{1.002327in}{0.852191in}}%
\pgfpathlineto{\pgfqpoint{0.980299in}{0.877238in}}%
\pgfpathlineto{\pgfqpoint{0.972754in}{0.885943in}}%
\pgfpathlineto{\pgfqpoint{0.958322in}{0.902707in}}%
\pgfpathlineto{\pgfqpoint{0.943181in}{0.920556in}}%
\pgfpathlineto{\pgfqpoint{0.936761in}{0.928177in}}%
\pgfpathlineto{\pgfqpoint{0.915633in}{0.953646in}}%
\pgfpathlineto{\pgfqpoint{0.913607in}{0.956130in}}%
\pgfpathlineto{\pgfqpoint{0.894995in}{0.979116in}}%
\pgfpathlineto{\pgfqpoint{0.884034in}{0.992856in}}%
\pgfpathlineto{\pgfqpoint{0.874746in}{1.004585in}}%
\pgfpathlineto{\pgfqpoint{0.854881in}{1.030055in}}%
\pgfpathlineto{\pgfqpoint{0.854460in}{1.030604in}}%
\pgfusepath{stroke}%
\end{pgfscope}%
\begin{pgfscope}%
\pgfpathrectangle{\pgfqpoint{0.854460in}{0.571603in}}{\pgfqpoint{5.885100in}{5.068436in}}%
\pgfusepath{clip}%
\pgfsetbuttcap%
\pgfsetroundjoin%
\pgfsetlinewidth{1.505625pt}%
\definecolor{currentstroke}{rgb}{0.131172,0.555899,0.552459}%
\pgfsetstrokecolor{currentstroke}%
\pgfsetdash{}{0pt}%
\pgfpathmoveto{\pgfqpoint{0.854460in}{4.385335in}}%
\pgfpathlineto{\pgfqpoint{0.859611in}{4.392032in}}%
\pgfpathlineto{\pgfqpoint{0.879488in}{4.417502in}}%
\pgfpathlineto{\pgfqpoint{0.884034in}{4.423242in}}%
\pgfpathlineto{\pgfqpoint{0.899846in}{4.442971in}}%
\pgfpathlineto{\pgfqpoint{0.913607in}{4.459914in}}%
\pgfpathlineto{\pgfqpoint{0.920614in}{4.468441in}}%
\pgfpathlineto{\pgfqpoint{0.941830in}{4.493910in}}%
\pgfpathlineto{\pgfqpoint{0.943181in}{4.495507in}}%
\pgfpathlineto{\pgfqpoint{0.963598in}{4.519380in}}%
\pgfpathlineto{\pgfqpoint{0.972754in}{4.529947in}}%
\pgfpathlineto{\pgfqpoint{0.985814in}{4.544849in}}%
\pgfpathlineto{\pgfqpoint{1.002327in}{4.563449in}}%
\pgfpathlineto{\pgfqpoint{1.008495in}{4.570319in}}%
\pgfpathlineto{\pgfqpoint{1.031660in}{4.595788in}}%
\pgfpathlineto{\pgfqpoint{1.031901in}{4.596049in}}%
\pgfpathlineto{\pgfqpoint{1.055416in}{4.621258in}}%
\pgfpathlineto{\pgfqpoint{1.061474in}{4.627670in}}%
\pgfpathlineto{\pgfqpoint{1.079675in}{4.646728in}}%
\pgfpathlineto{\pgfqpoint{1.091047in}{4.658485in}}%
\pgfpathlineto{\pgfqpoint{1.104452in}{4.672197in}}%
\pgfpathlineto{\pgfqpoint{1.120621in}{4.688529in}}%
\pgfpathlineto{\pgfqpoint{1.129763in}{4.697667in}}%
\pgfpathlineto{\pgfqpoint{1.150194in}{4.717832in}}%
\pgfpathlineto{\pgfqpoint{1.155624in}{4.723136in}}%
\pgfpathlineto{\pgfqpoint{1.179767in}{4.746427in}}%
\pgfpathlineto{\pgfqpoint{1.182049in}{4.748606in}}%
\pgfpathlineto{\pgfqpoint{1.209060in}{4.774075in}}%
\pgfpathlineto{\pgfqpoint{1.209341in}{4.774337in}}%
\pgfpathlineto{\pgfqpoint{1.236695in}{4.799545in}}%
\pgfpathlineto{\pgfqpoint{1.238914in}{4.801565in}}%
\pgfpathlineto{\pgfqpoint{1.264931in}{4.825014in}}%
\pgfpathlineto{\pgfqpoint{1.268488in}{4.828180in}}%
\pgfpathlineto{\pgfqpoint{1.293783in}{4.850484in}}%
\pgfpathlineto{\pgfqpoint{1.298061in}{4.854209in}}%
\pgfpathlineto{\pgfqpoint{1.323266in}{4.875953in}}%
\pgfpathlineto{\pgfqpoint{1.327634in}{4.879676in}}%
\pgfpathlineto{\pgfqpoint{1.353394in}{4.901423in}}%
\pgfpathlineto{\pgfqpoint{1.357208in}{4.904603in}}%
\pgfpathlineto{\pgfqpoint{1.384183in}{4.926892in}}%
\pgfpathlineto{\pgfqpoint{1.386781in}{4.929013in}}%
\pgfpathlineto{\pgfqpoint{1.415645in}{4.952362in}}%
\pgfpathlineto{\pgfqpoint{1.416354in}{4.952929in}}%
\pgfpathlineto{\pgfqpoint{1.445928in}{4.976340in}}%
\pgfpathlineto{\pgfqpoint{1.447828in}{4.977831in}}%
\pgfpathlineto{\pgfqpoint{1.475501in}{4.999277in}}%
\pgfpathlineto{\pgfqpoint{1.480738in}{5.003301in}}%
\pgfpathlineto{\pgfqpoint{1.505074in}{5.021771in}}%
\pgfpathlineto{\pgfqpoint{1.514376in}{5.028770in}}%
\pgfpathlineto{\pgfqpoint{1.534648in}{5.043840in}}%
\pgfpathlineto{\pgfqpoint{1.548755in}{5.054240in}}%
\pgfpathlineto{\pgfqpoint{1.564221in}{5.065504in}}%
\pgfpathlineto{\pgfqpoint{1.583887in}{5.079709in}}%
\pgfpathlineto{\pgfqpoint{1.593795in}{5.086779in}}%
\pgfpathlineto{\pgfqpoint{1.619787in}{5.105179in}}%
\pgfpathlineto{\pgfqpoint{1.623368in}{5.107684in}}%
\pgfpathlineto{\pgfqpoint{1.652941in}{5.128185in}}%
\pgfpathlineto{\pgfqpoint{1.656526in}{5.130649in}}%
\pgfpathlineto{\pgfqpoint{1.682515in}{5.148291in}}%
\pgfpathlineto{\pgfqpoint{1.694133in}{5.156118in}}%
\pgfpathlineto{\pgfqpoint{1.712088in}{5.168067in}}%
\pgfpathlineto{\pgfqpoint{1.732555in}{5.181588in}}%
\pgfpathlineto{\pgfqpoint{1.741661in}{5.187530in}}%
\pgfpathlineto{\pgfqpoint{1.771235in}{5.206687in}}%
\pgfpathlineto{\pgfqpoint{1.771812in}{5.207057in}}%
\pgfpathlineto{\pgfqpoint{1.800808in}{5.225433in}}%
\pgfpathlineto{\pgfqpoint{1.812082in}{5.232527in}}%
\pgfpathlineto{\pgfqpoint{1.830381in}{5.243902in}}%
\pgfpathlineto{\pgfqpoint{1.853210in}{5.257996in}}%
\pgfpathlineto{\pgfqpoint{1.859955in}{5.262109in}}%
\pgfpathlineto{\pgfqpoint{1.889528in}{5.279998in}}%
\pgfpathlineto{\pgfqpoint{1.895309in}{5.283466in}}%
\pgfpathlineto{\pgfqpoint{1.919102in}{5.297562in}}%
\pgfpathlineto{\pgfqpoint{1.938421in}{5.308935in}}%
\pgfpathlineto{\pgfqpoint{1.948675in}{5.314898in}}%
\pgfpathlineto{\pgfqpoint{1.978248in}{5.331969in}}%
\pgfpathlineto{\pgfqpoint{1.982507in}{5.334405in}}%
\pgfpathlineto{\pgfqpoint{2.007822in}{5.348709in}}%
\pgfpathlineto{\pgfqpoint{2.027700in}{5.359874in}}%
\pgfpathlineto{\pgfqpoint{2.037395in}{5.365253in}}%
\pgfpathlineto{\pgfqpoint{2.066968in}{5.381536in}}%
\pgfpathlineto{\pgfqpoint{2.073944in}{5.385344in}}%
\pgfpathlineto{\pgfqpoint{2.096542in}{5.397529in}}%
\pgfpathlineto{\pgfqpoint{2.121313in}{5.410813in}}%
\pgfpathlineto{\pgfqpoint{2.126115in}{5.413357in}}%
\pgfpathlineto{\pgfqpoint{2.155689in}{5.428879in}}%
\pgfpathlineto{\pgfqpoint{2.169890in}{5.436283in}}%
\pgfpathlineto{\pgfqpoint{2.185262in}{5.444199in}}%
\pgfpathlineto{\pgfqpoint{2.214835in}{5.459331in}}%
\pgfpathlineto{\pgfqpoint{2.219612in}{5.461752in}}%
\pgfpathlineto{\pgfqpoint{2.244409in}{5.474167in}}%
\pgfpathlineto{\pgfqpoint{2.270609in}{5.487222in}}%
\pgfpathlineto{\pgfqpoint{2.273982in}{5.488882in}}%
\pgfpathlineto{\pgfqpoint{2.303555in}{5.503291in}}%
\pgfpathlineto{\pgfqpoint{2.322953in}{5.512691in}}%
\pgfpathlineto{\pgfqpoint{2.333129in}{5.517561in}}%
\pgfpathlineto{\pgfqpoint{2.362702in}{5.531599in}}%
\pgfpathlineto{\pgfqpoint{2.376624in}{5.538161in}}%
\pgfpathlineto{\pgfqpoint{2.392275in}{5.545446in}}%
\pgfpathlineto{\pgfqpoint{2.421849in}{5.559118in}}%
\pgfpathlineto{\pgfqpoint{2.431692in}{5.563630in}}%
\pgfpathlineto{\pgfqpoint{2.451422in}{5.572562in}}%
\pgfpathlineto{\pgfqpoint{2.480996in}{5.585873in}}%
\pgfpathlineto{\pgfqpoint{2.488231in}{5.589100in}}%
\pgfpathlineto{\pgfqpoint{2.510569in}{5.598935in}}%
\pgfpathlineto{\pgfqpoint{2.540142in}{5.611891in}}%
\pgfpathlineto{\pgfqpoint{2.546317in}{5.614570in}}%
\pgfpathlineto{\pgfqpoint{2.569716in}{5.624591in}}%
\pgfpathlineto{\pgfqpoint{2.599289in}{5.637195in}}%
\pgfpathlineto{\pgfqpoint{2.606029in}{5.640039in}}%
\pgfusepath{stroke}%
\end{pgfscope}%
\begin{pgfscope}%
\pgfpathrectangle{\pgfqpoint{0.854460in}{0.571603in}}{\pgfqpoint{5.885100in}{5.068436in}}%
\pgfusepath{clip}%
\pgfsetbuttcap%
\pgfsetroundjoin%
\pgfsetlinewidth{1.505625pt}%
\definecolor{currentstroke}{rgb}{0.131172,0.555899,0.552459}%
\pgfsetstrokecolor{currentstroke}%
\pgfsetdash{}{0pt}%
\pgfpathmoveto{\pgfqpoint{6.022114in}{5.640039in}}%
\pgfpathlineto{\pgfqpoint{6.023966in}{5.589100in}}%
\pgfpathlineto{\pgfqpoint{6.022961in}{5.538161in}}%
\pgfpathlineto{\pgfqpoint{6.019566in}{5.487222in}}%
\pgfpathlineto{\pgfqpoint{6.014176in}{5.436283in}}%
\pgfpathlineto{\pgfqpoint{6.003066in}{5.359874in}}%
\pgfpathlineto{\pgfqpoint{5.988991in}{5.283466in}}%
\pgfpathlineto{\pgfqpoint{5.970653in}{5.197681in}}%
\pgfpathlineto{\pgfqpoint{5.948707in}{5.105179in}}%
\pgfpathlineto{\pgfqpoint{5.916297in}{4.977831in}}%
\pgfpathlineto{\pgfqpoint{5.822786in}{4.615830in}}%
\pgfpathlineto{\pgfqpoint{5.793213in}{4.490231in}}%
\pgfpathlineto{\pgfqpoint{5.771943in}{4.392032in}}%
\pgfpathlineto{\pgfqpoint{5.751996in}{4.290154in}}%
\pgfpathlineto{\pgfqpoint{5.734066in}{4.185517in}}%
\pgfpathlineto{\pgfqpoint{5.719585in}{4.086398in}}%
\pgfpathlineto{\pgfqpoint{5.710316in}{4.009989in}}%
\pgfpathlineto{\pgfqpoint{5.702748in}{3.933581in}}%
\pgfpathlineto{\pgfqpoint{5.696923in}{3.857172in}}%
\pgfpathlineto{\pgfqpoint{5.692956in}{3.780764in}}%
\pgfpathlineto{\pgfqpoint{5.690903in}{3.704355in}}%
\pgfpathlineto{\pgfqpoint{5.690819in}{3.627946in}}%
\pgfpathlineto{\pgfqpoint{5.692758in}{3.551538in}}%
\pgfpathlineto{\pgfqpoint{5.696771in}{3.475129in}}%
\pgfpathlineto{\pgfqpoint{5.702912in}{3.398721in}}%
\pgfpathlineto{\pgfqpoint{5.711173in}{3.322312in}}%
\pgfpathlineto{\pgfqpoint{5.721636in}{3.245904in}}%
\pgfpathlineto{\pgfqpoint{5.734366in}{3.169495in}}%
\pgfpathlineto{\pgfqpoint{5.749293in}{3.093086in}}%
\pgfpathlineto{\pgfqpoint{5.766561in}{3.016678in}}%
\pgfpathlineto{\pgfqpoint{5.786108in}{2.940269in}}%
\pgfpathlineto{\pgfqpoint{5.807996in}{2.863861in}}%
\pgfpathlineto{\pgfqpoint{5.832254in}{2.787452in}}%
\pgfpathlineto{\pgfqpoint{5.858889in}{2.711044in}}%
\pgfpathlineto{\pgfqpoint{5.887913in}{2.634635in}}%
\pgfpathlineto{\pgfqpoint{5.919338in}{2.558226in}}%
\pgfpathlineto{\pgfqpoint{5.953180in}{2.481818in}}%
\pgfpathlineto{\pgfqpoint{5.989460in}{2.405409in}}%
\pgfpathlineto{\pgfqpoint{6.029799in}{2.325983in}}%
\pgfpathlineto{\pgfqpoint{6.069345in}{2.252592in}}%
\pgfpathlineto{\pgfqpoint{6.118520in}{2.166840in}}%
\pgfpathlineto{\pgfqpoint{6.159038in}{2.099775in}}%
\pgfpathlineto{\pgfqpoint{6.207603in}{2.023366in}}%
\pgfpathlineto{\pgfqpoint{6.266386in}{1.935647in}}%
\pgfpathlineto{\pgfqpoint{6.312036in}{1.870549in}}%
\pgfpathlineto{\pgfqpoint{6.367962in}{1.794141in}}%
\pgfpathlineto{\pgfqpoint{6.426349in}{1.717732in}}%
\pgfpathlineto{\pgfqpoint{6.502973in}{1.622050in}}%
\pgfpathlineto{\pgfqpoint{6.562120in}{1.551248in}}%
\pgfpathlineto{\pgfqpoint{6.621267in}{1.482827in}}%
\pgfpathlineto{\pgfqpoint{6.684450in}{1.412098in}}%
\pgfpathlineto{\pgfqpoint{6.739560in}{1.352280in}}%
\pgfpathlineto{\pgfqpoint{6.739560in}{1.352280in}}%
\pgfusepath{stroke}%
\end{pgfscope}%
\begin{pgfscope}%
\pgfpathrectangle{\pgfqpoint{0.854460in}{0.571603in}}{\pgfqpoint{5.885100in}{5.068436in}}%
\pgfusepath{clip}%
\pgfsetbuttcap%
\pgfsetroundjoin%
\pgfsetlinewidth{1.505625pt}%
\definecolor{currentstroke}{rgb}{0.125394,0.574318,0.549086}%
\pgfsetstrokecolor{currentstroke}%
\pgfsetdash{}{0pt}%
\pgfpathmoveto{\pgfqpoint{1.219941in}{0.571603in}}%
\pgfpathlineto{\pgfqpoint{1.209341in}{0.581376in}}%
\pgfpathlineto{\pgfqpoint{1.192400in}{0.597073in}}%
\pgfpathlineto{\pgfqpoint{1.179767in}{0.608943in}}%
\pgfpathlineto{\pgfqpoint{1.165370in}{0.622542in}}%
\pgfpathlineto{\pgfqpoint{1.150194in}{0.637080in}}%
\pgfpathlineto{\pgfqpoint{1.138843in}{0.648012in}}%
\pgfpathlineto{\pgfqpoint{1.120621in}{0.665810in}}%
\pgfpathlineto{\pgfqpoint{1.112810in}{0.673481in}}%
\pgfpathlineto{\pgfqpoint{1.091047in}{0.695160in}}%
\pgfpathlineto{\pgfqpoint{1.087263in}{0.698951in}}%
\pgfpathlineto{\pgfqpoint{1.062202in}{0.724420in}}%
\pgfpathlineto{\pgfqpoint{1.061474in}{0.725172in}}%
\pgfpathlineto{\pgfqpoint{1.037667in}{0.749890in}}%
\pgfpathlineto{\pgfqpoint{1.031901in}{0.755963in}}%
\pgfpathlineto{\pgfqpoint{1.013595in}{0.775360in}}%
\pgfpathlineto{\pgfqpoint{1.002327in}{0.787472in}}%
\pgfpathlineto{\pgfqpoint{0.989979in}{0.800829in}}%
\pgfpathlineto{\pgfqpoint{0.972754in}{0.819731in}}%
\pgfpathlineto{\pgfqpoint{0.966807in}{0.826299in}}%
\pgfpathlineto{\pgfqpoint{0.944083in}{0.851768in}}%
\pgfpathlineto{\pgfqpoint{0.943181in}{0.852796in}}%
\pgfpathlineto{\pgfqpoint{0.921870in}{0.877238in}}%
\pgfpathlineto{\pgfqpoint{0.913607in}{0.886854in}}%
\pgfpathlineto{\pgfqpoint{0.900078in}{0.902707in}}%
\pgfpathlineto{\pgfqpoint{0.884034in}{0.921785in}}%
\pgfpathlineto{\pgfqpoint{0.878696in}{0.928177in}}%
\pgfpathlineto{\pgfqpoint{0.857758in}{0.953646in}}%
\pgfpathlineto{\pgfqpoint{0.854460in}{0.957724in}}%
\pgfusepath{stroke}%
\end{pgfscope}%
\begin{pgfscope}%
\pgfpathrectangle{\pgfqpoint{0.854460in}{0.571603in}}{\pgfqpoint{5.885100in}{5.068436in}}%
\pgfusepath{clip}%
\pgfsetbuttcap%
\pgfsetroundjoin%
\pgfsetlinewidth{1.505625pt}%
\definecolor{currentstroke}{rgb}{0.125394,0.574318,0.549086}%
\pgfsetstrokecolor{currentstroke}%
\pgfsetdash{}{0pt}%
\pgfpathmoveto{\pgfqpoint{0.854460in}{4.472844in}}%
\pgfpathlineto{\pgfqpoint{0.871713in}{4.493910in}}%
\pgfpathlineto{\pgfqpoint{0.884034in}{4.508758in}}%
\pgfpathlineto{\pgfqpoint{0.892949in}{4.519380in}}%
\pgfpathlineto{\pgfqpoint{0.913607in}{4.543671in}}%
\pgfpathlineto{\pgfqpoint{0.914620in}{4.544849in}}%
\pgfpathlineto{\pgfqpoint{0.936847in}{4.570319in}}%
\pgfpathlineto{\pgfqpoint{0.943181in}{4.577481in}}%
\pgfpathlineto{\pgfqpoint{0.959548in}{4.595788in}}%
\pgfpathlineto{\pgfqpoint{0.972754in}{4.610370in}}%
\pgfpathlineto{\pgfqpoint{0.982722in}{4.621258in}}%
\pgfpathlineto{\pgfqpoint{1.002327in}{4.642399in}}%
\pgfpathlineto{\pgfqpoint{1.006384in}{4.646728in}}%
\pgfpathlineto{\pgfqpoint{1.030571in}{4.672197in}}%
\pgfpathlineto{\pgfqpoint{1.031901in}{4.673578in}}%
\pgfpathlineto{\pgfqpoint{1.055335in}{4.697667in}}%
\pgfpathlineto{\pgfqpoint{1.061474in}{4.703898in}}%
\pgfpathlineto{\pgfqpoint{1.080624in}{4.723136in}}%
\pgfpathlineto{\pgfqpoint{1.091047in}{4.733477in}}%
\pgfpathlineto{\pgfqpoint{1.106453in}{4.748606in}}%
\pgfpathlineto{\pgfqpoint{1.120621in}{4.762346in}}%
\pgfpathlineto{\pgfqpoint{1.132836in}{4.774075in}}%
\pgfpathlineto{\pgfqpoint{1.150194in}{4.790535in}}%
\pgfpathlineto{\pgfqpoint{1.159790in}{4.799545in}}%
\pgfpathlineto{\pgfqpoint{1.179767in}{4.818070in}}%
\pgfpathlineto{\pgfqpoint{1.187329in}{4.825014in}}%
\pgfpathlineto{\pgfqpoint{1.209341in}{4.844980in}}%
\pgfpathlineto{\pgfqpoint{1.215467in}{4.850484in}}%
\pgfpathlineto{\pgfqpoint{1.238914in}{4.871290in}}%
\pgfpathlineto{\pgfqpoint{1.244220in}{4.875953in}}%
\pgfpathlineto{\pgfqpoint{1.268488in}{4.897024in}}%
\pgfpathlineto{\pgfqpoint{1.273602in}{4.901423in}}%
\pgfpathlineto{\pgfqpoint{1.298061in}{4.922206in}}%
\pgfpathlineto{\pgfqpoint{1.303627in}{4.926892in}}%
\pgfpathlineto{\pgfqpoint{1.327634in}{4.946859in}}%
\pgfpathlineto{\pgfqpoint{1.334310in}{4.952362in}}%
\pgfpathlineto{\pgfqpoint{1.357208in}{4.971006in}}%
\pgfpathlineto{\pgfqpoint{1.365664in}{4.977831in}}%
\pgfpathlineto{\pgfqpoint{1.386781in}{4.994667in}}%
\pgfpathlineto{\pgfqpoint{1.397704in}{5.003301in}}%
\pgfpathlineto{\pgfqpoint{1.416354in}{5.017863in}}%
\pgfpathlineto{\pgfqpoint{1.430443in}{5.028770in}}%
\pgfpathlineto{\pgfqpoint{1.445928in}{5.040613in}}%
\pgfpathlineto{\pgfqpoint{1.463894in}{5.054240in}}%
\pgfpathlineto{\pgfqpoint{1.475501in}{5.062937in}}%
\pgfpathlineto{\pgfqpoint{1.498069in}{5.079709in}}%
\pgfpathlineto{\pgfqpoint{1.505074in}{5.084853in}}%
\pgfpathlineto{\pgfqpoint{1.532982in}{5.105179in}}%
\pgfpathlineto{\pgfqpoint{1.534648in}{5.106378in}}%
\pgfpathlineto{\pgfqpoint{1.564221in}{5.127467in}}%
\pgfpathlineto{\pgfqpoint{1.568719in}{5.130649in}}%
\pgfpathlineto{\pgfqpoint{1.593795in}{5.148169in}}%
\pgfpathlineto{\pgfqpoint{1.605258in}{5.156118in}}%
\pgfpathlineto{\pgfqpoint{1.623368in}{5.168523in}}%
\pgfpathlineto{\pgfqpoint{1.642582in}{5.181588in}}%
\pgfpathlineto{\pgfqpoint{1.652941in}{5.188546in}}%
\pgfpathlineto{\pgfqpoint{1.680701in}{5.207057in}}%
\pgfpathlineto{\pgfqpoint{1.682515in}{5.208252in}}%
\pgfpathlineto{\pgfqpoint{1.712088in}{5.227558in}}%
\pgfpathlineto{\pgfqpoint{1.719756in}{5.232527in}}%
\pgfpathlineto{\pgfqpoint{1.741661in}{5.246548in}}%
\pgfpathlineto{\pgfqpoint{1.759671in}{5.257996in}}%
\pgfpathlineto{\pgfqpoint{1.771235in}{5.265258in}}%
\pgfpathlineto{\pgfqpoint{1.800423in}{5.283466in}}%
\pgfpathlineto{\pgfqpoint{1.800808in}{5.283703in}}%
\pgfpathlineto{\pgfqpoint{1.830381in}{5.301754in}}%
\pgfpathlineto{\pgfqpoint{1.842224in}{5.308935in}}%
\pgfpathlineto{\pgfqpoint{1.859955in}{5.319556in}}%
\pgfpathlineto{\pgfqpoint{1.884900in}{5.334405in}}%
\pgfpathlineto{\pgfqpoint{1.889528in}{5.337126in}}%
\pgfpathlineto{\pgfqpoint{1.919102in}{5.354368in}}%
\pgfpathlineto{\pgfqpoint{1.928614in}{5.359874in}}%
\pgfpathlineto{\pgfqpoint{1.948675in}{5.371343in}}%
\pgfpathlineto{\pgfqpoint{1.973306in}{5.385344in}}%
\pgfpathlineto{\pgfqpoint{1.978248in}{5.388119in}}%
\pgfpathlineto{\pgfqpoint{2.007822in}{5.404581in}}%
\pgfpathlineto{\pgfqpoint{2.019094in}{5.410813in}}%
\pgfpathlineto{\pgfqpoint{2.037395in}{5.420806in}}%
\pgfpathlineto{\pgfqpoint{2.065890in}{5.436283in}}%
\pgfpathlineto{\pgfqpoint{2.066968in}{5.436861in}}%
\pgfpathlineto{\pgfqpoint{2.096542in}{5.452573in}}%
\pgfpathlineto{\pgfqpoint{2.113915in}{5.461752in}}%
\pgfpathlineto{\pgfqpoint{2.126115in}{5.468119in}}%
\pgfpathlineto{\pgfqpoint{2.155689in}{5.483448in}}%
\pgfpathlineto{\pgfqpoint{2.163029in}{5.487222in}}%
\pgfpathlineto{\pgfqpoint{2.185262in}{5.498510in}}%
\pgfpathlineto{\pgfqpoint{2.213322in}{5.512691in}}%
\pgfpathlineto{\pgfqpoint{2.214835in}{5.513447in}}%
\pgfpathlineto{\pgfqpoint{2.244409in}{5.528062in}}%
\pgfpathlineto{\pgfqpoint{2.264940in}{5.538161in}}%
\pgfpathlineto{\pgfqpoint{2.273982in}{5.542553in}}%
\pgfpathlineto{\pgfqpoint{2.303555in}{5.556804in}}%
\pgfpathlineto{\pgfqpoint{2.317815in}{5.563630in}}%
\pgfpathlineto{\pgfqpoint{2.333129in}{5.570870in}}%
\pgfpathlineto{\pgfqpoint{2.362702in}{5.584762in}}%
\pgfpathlineto{\pgfqpoint{2.372011in}{5.589100in}}%
\pgfpathlineto{\pgfqpoint{2.392275in}{5.598425in}}%
\pgfpathlineto{\pgfqpoint{2.421849in}{5.611963in}}%
\pgfpathlineto{\pgfqpoint{2.427594in}{5.614570in}}%
\pgfpathlineto{\pgfqpoint{2.451422in}{5.625242in}}%
\pgfpathlineto{\pgfqpoint{2.480996in}{5.638432in}}%
\pgfpathlineto{\pgfqpoint{2.484635in}{5.640039in}}%
\pgfusepath{stroke}%
\end{pgfscope}%
\begin{pgfscope}%
\pgfpathrectangle{\pgfqpoint{0.854460in}{0.571603in}}{\pgfqpoint{5.885100in}{5.068436in}}%
\pgfusepath{clip}%
\pgfsetbuttcap%
\pgfsetroundjoin%
\pgfsetlinewidth{1.505625pt}%
\definecolor{currentstroke}{rgb}{0.125394,0.574318,0.549086}%
\pgfsetstrokecolor{currentstroke}%
\pgfsetdash{}{0pt}%
\pgfpathmoveto{\pgfqpoint{6.158977in}{5.640039in}}%
\pgfpathlineto{\pgfqpoint{6.155981in}{5.589100in}}%
\pgfpathlineto{\pgfqpoint{6.150734in}{5.538161in}}%
\pgfpathlineto{\pgfqpoint{6.143547in}{5.487222in}}%
\pgfpathlineto{\pgfqpoint{6.129836in}{5.410813in}}%
\pgfpathlineto{\pgfqpoint{6.113391in}{5.334405in}}%
\pgfpathlineto{\pgfqpoint{6.088313in}{5.232527in}}%
\pgfpathlineto{\pgfqpoint{6.053614in}{5.105179in}}%
\pgfpathlineto{\pgfqpoint{5.987486in}{4.875953in}}%
\pgfpathlineto{\pgfqpoint{5.937015in}{4.697667in}}%
\pgfpathlineto{\pgfqpoint{5.903129in}{4.570319in}}%
\pgfpathlineto{\pgfqpoint{5.877958in}{4.468441in}}%
\pgfpathlineto{\pgfqpoint{5.854844in}{4.366563in}}%
\pgfpathlineto{\pgfqpoint{5.834054in}{4.264685in}}%
\pgfpathlineto{\pgfqpoint{5.815902in}{4.162807in}}%
\pgfpathlineto{\pgfqpoint{5.800570in}{4.060928in}}%
\pgfpathlineto{\pgfqpoint{5.791066in}{3.984520in}}%
\pgfpathlineto{\pgfqpoint{5.783304in}{3.908111in}}%
\pgfpathlineto{\pgfqpoint{5.777409in}{3.831703in}}%
\pgfpathlineto{\pgfqpoint{5.773434in}{3.755294in}}%
\pgfpathlineto{\pgfqpoint{5.771429in}{3.678886in}}%
\pgfpathlineto{\pgfqpoint{5.771446in}{3.602477in}}%
\pgfpathlineto{\pgfqpoint{5.773532in}{3.526068in}}%
\pgfpathlineto{\pgfqpoint{5.777737in}{3.449660in}}%
\pgfpathlineto{\pgfqpoint{5.784112in}{3.373251in}}%
\pgfpathlineto{\pgfqpoint{5.793213in}{3.293009in}}%
\pgfpathlineto{\pgfqpoint{5.803480in}{3.220434in}}%
\pgfpathlineto{\pgfqpoint{5.816555in}{3.144025in}}%
\pgfpathlineto{\pgfqpoint{5.831910in}{3.067617in}}%
\pgfpathlineto{\pgfqpoint{5.849608in}{2.991208in}}%
\pgfpathlineto{\pgfqpoint{5.869614in}{2.914800in}}%
\pgfpathlineto{\pgfqpoint{5.892009in}{2.838391in}}%
\pgfpathlineto{\pgfqpoint{5.916796in}{2.761983in}}%
\pgfpathlineto{\pgfqpoint{5.943984in}{2.685574in}}%
\pgfpathlineto{\pgfqpoint{5.973582in}{2.609165in}}%
\pgfpathlineto{\pgfqpoint{6.005604in}{2.532757in}}%
\pgfpathlineto{\pgfqpoint{6.040067in}{2.456348in}}%
\pgfpathlineto{\pgfqpoint{6.076990in}{2.379940in}}%
\pgfpathlineto{\pgfqpoint{6.118520in}{2.299590in}}%
\pgfpathlineto{\pgfqpoint{6.158233in}{2.227123in}}%
\pgfpathlineto{\pgfqpoint{6.207240in}{2.142975in}}%
\pgfpathlineto{\pgfqpoint{6.249365in}{2.074305in}}%
\pgfpathlineto{\pgfqpoint{6.298671in}{1.997897in}}%
\pgfpathlineto{\pgfqpoint{6.355107in}{1.914817in}}%
\pgfpathlineto{\pgfqpoint{6.414253in}{1.832009in}}%
\pgfpathlineto{\pgfqpoint{6.461406in}{1.768671in}}%
\pgfpathlineto{\pgfqpoint{6.532547in}{1.677307in}}%
\pgfpathlineto{\pgfqpoint{6.591693in}{1.604550in}}%
\pgfpathlineto{\pgfqpoint{6.650840in}{1.534396in}}%
\pgfpathlineto{\pgfqpoint{6.713123in}{1.463037in}}%
\pgfpathlineto{\pgfqpoint{6.739560in}{1.433500in}}%
\pgfpathlineto{\pgfqpoint{6.739560in}{1.433500in}}%
\pgfusepath{stroke}%
\end{pgfscope}%
\begin{pgfscope}%
\pgfpathrectangle{\pgfqpoint{0.854460in}{0.571603in}}{\pgfqpoint{5.885100in}{5.068436in}}%
\pgfusepath{clip}%
\pgfsetbuttcap%
\pgfsetroundjoin%
\pgfsetlinewidth{1.505625pt}%
\definecolor{currentstroke}{rgb}{0.120565,0.596422,0.543611}%
\pgfsetstrokecolor{currentstroke}%
\pgfsetdash{}{0pt}%
\pgfpathmoveto{\pgfqpoint{1.160477in}{0.571603in}}%
\pgfpathlineto{\pgfqpoint{1.150194in}{0.581167in}}%
\pgfpathlineto{\pgfqpoint{1.133178in}{0.597073in}}%
\pgfpathlineto{\pgfqpoint{1.120621in}{0.608977in}}%
\pgfpathlineto{\pgfqpoint{1.106385in}{0.622542in}}%
\pgfpathlineto{\pgfqpoint{1.091047in}{0.637364in}}%
\pgfpathlineto{\pgfqpoint{1.080089in}{0.648012in}}%
\pgfpathlineto{\pgfqpoint{1.061474in}{0.666354in}}%
\pgfpathlineto{\pgfqpoint{1.054281in}{0.673481in}}%
\pgfpathlineto{\pgfqpoint{1.031901in}{0.695972in}}%
\pgfpathlineto{\pgfqpoint{1.028953in}{0.698951in}}%
\pgfpathlineto{\pgfqpoint{1.004120in}{0.724420in}}%
\pgfpathlineto{\pgfqpoint{1.002327in}{0.726287in}}%
\pgfpathlineto{\pgfqpoint{0.979793in}{0.749890in}}%
\pgfpathlineto{\pgfqpoint{0.972754in}{0.757369in}}%
\pgfpathlineto{\pgfqpoint{0.955925in}{0.775360in}}%
\pgfpathlineto{\pgfqpoint{0.943181in}{0.789180in}}%
\pgfpathlineto{\pgfqpoint{0.932506in}{0.800829in}}%
\pgfpathlineto{\pgfqpoint{0.913607in}{0.821751in}}%
\pgfpathlineto{\pgfqpoint{0.909525in}{0.826299in}}%
\pgfpathlineto{\pgfqpoint{0.887013in}{0.851768in}}%
\pgfpathlineto{\pgfqpoint{0.884034in}{0.855193in}}%
\pgfpathlineto{\pgfqpoint{0.864981in}{0.877238in}}%
\pgfpathlineto{\pgfqpoint{0.854460in}{0.889589in}}%
\pgfusepath{stroke}%
\end{pgfscope}%
\begin{pgfscope}%
\pgfpathrectangle{\pgfqpoint{0.854460in}{0.571603in}}{\pgfqpoint{5.885100in}{5.068436in}}%
\pgfusepath{clip}%
\pgfsetbuttcap%
\pgfsetroundjoin%
\pgfsetlinewidth{1.505625pt}%
\definecolor{currentstroke}{rgb}{0.120565,0.596422,0.543611}%
\pgfsetstrokecolor{currentstroke}%
\pgfsetdash{}{0pt}%
\pgfpathmoveto{\pgfqpoint{0.854460in}{4.554539in}}%
\pgfpathlineto{\pgfqpoint{0.867984in}{4.570319in}}%
\pgfpathlineto{\pgfqpoint{0.884034in}{4.588803in}}%
\pgfpathlineto{\pgfqpoint{0.890165in}{4.595788in}}%
\pgfpathlineto{\pgfqpoint{0.912817in}{4.621258in}}%
\pgfpathlineto{\pgfqpoint{0.913607in}{4.622133in}}%
\pgfpathlineto{\pgfqpoint{0.936035in}{4.646728in}}%
\pgfpathlineto{\pgfqpoint{0.943181in}{4.654464in}}%
\pgfpathlineto{\pgfqpoint{0.959732in}{4.672197in}}%
\pgfpathlineto{\pgfqpoint{0.972754in}{4.685971in}}%
\pgfpathlineto{\pgfqpoint{0.983926in}{4.697667in}}%
\pgfpathlineto{\pgfqpoint{1.002327in}{4.716688in}}%
\pgfpathlineto{\pgfqpoint{1.008629in}{4.723136in}}%
\pgfpathlineto{\pgfqpoint{1.031901in}{4.746648in}}%
\pgfpathlineto{\pgfqpoint{1.033858in}{4.748606in}}%
\pgfpathlineto{\pgfqpoint{1.059656in}{4.774075in}}%
\pgfpathlineto{\pgfqpoint{1.061474in}{4.775847in}}%
\pgfpathlineto{\pgfqpoint{1.086032in}{4.799545in}}%
\pgfpathlineto{\pgfqpoint{1.091047in}{4.804324in}}%
\pgfpathlineto{\pgfqpoint{1.112968in}{4.825014in}}%
\pgfpathlineto{\pgfqpoint{1.120621in}{4.832148in}}%
\pgfpathlineto{\pgfqpoint{1.140479in}{4.850484in}}%
\pgfpathlineto{\pgfqpoint{1.150194in}{4.859344in}}%
\pgfpathlineto{\pgfqpoint{1.168578in}{4.875953in}}%
\pgfpathlineto{\pgfqpoint{1.179767in}{4.885938in}}%
\pgfpathlineto{\pgfqpoint{1.197281in}{4.901423in}}%
\pgfpathlineto{\pgfqpoint{1.209341in}{4.911955in}}%
\pgfpathlineto{\pgfqpoint{1.226602in}{4.926892in}}%
\pgfpathlineto{\pgfqpoint{1.238914in}{4.937418in}}%
\pgfpathlineto{\pgfqpoint{1.256553in}{4.952362in}}%
\pgfpathlineto{\pgfqpoint{1.268488in}{4.962350in}}%
\pgfpathlineto{\pgfqpoint{1.287149in}{4.977831in}}%
\pgfpathlineto{\pgfqpoint{1.298061in}{4.986773in}}%
\pgfpathlineto{\pgfqpoint{1.318404in}{5.003301in}}%
\pgfpathlineto{\pgfqpoint{1.327634in}{5.010709in}}%
\pgfpathlineto{\pgfqpoint{1.350330in}{5.028770in}}%
\pgfpathlineto{\pgfqpoint{1.357208in}{5.034177in}}%
\pgfpathlineto{\pgfqpoint{1.382941in}{5.054240in}}%
\pgfpathlineto{\pgfqpoint{1.386781in}{5.057197in}}%
\pgfpathlineto{\pgfqpoint{1.416249in}{5.079709in}}%
\pgfpathlineto{\pgfqpoint{1.416354in}{5.079789in}}%
\pgfpathlineto{\pgfqpoint{1.445928in}{5.101906in}}%
\pgfpathlineto{\pgfqpoint{1.450339in}{5.105179in}}%
\pgfpathlineto{\pgfqpoint{1.475501in}{5.123621in}}%
\pgfpathlineto{\pgfqpoint{1.485164in}{5.130649in}}%
\pgfpathlineto{\pgfqpoint{1.505074in}{5.144953in}}%
\pgfpathlineto{\pgfqpoint{1.520734in}{5.156118in}}%
\pgfpathlineto{\pgfqpoint{1.534648in}{5.165919in}}%
\pgfpathlineto{\pgfqpoint{1.557060in}{5.181588in}}%
\pgfpathlineto{\pgfqpoint{1.564221in}{5.186534in}}%
\pgfpathlineto{\pgfqpoint{1.593795in}{5.206810in}}%
\pgfpathlineto{\pgfqpoint{1.594158in}{5.207057in}}%
\pgfpathlineto{\pgfqpoint{1.623368in}{5.226662in}}%
\pgfpathlineto{\pgfqpoint{1.632168in}{5.232527in}}%
\pgfpathlineto{\pgfqpoint{1.652941in}{5.246203in}}%
\pgfpathlineto{\pgfqpoint{1.670978in}{5.257996in}}%
\pgfpathlineto{\pgfqpoint{1.682515in}{5.265448in}}%
\pgfpathlineto{\pgfqpoint{1.710596in}{5.283466in}}%
\pgfpathlineto{\pgfqpoint{1.712088in}{5.284412in}}%
\pgfpathlineto{\pgfqpoint{1.741661in}{5.302990in}}%
\pgfpathlineto{\pgfqpoint{1.751190in}{5.308935in}}%
\pgfpathlineto{\pgfqpoint{1.771235in}{5.321289in}}%
\pgfpathlineto{\pgfqpoint{1.792649in}{5.334405in}}%
\pgfpathlineto{\pgfqpoint{1.800808in}{5.339341in}}%
\pgfpathlineto{\pgfqpoint{1.830381in}{5.357104in}}%
\pgfpathlineto{\pgfqpoint{1.835032in}{5.359874in}}%
\pgfpathlineto{\pgfqpoint{1.859955in}{5.374540in}}%
\pgfpathlineto{\pgfqpoint{1.878421in}{5.385344in}}%
\pgfpathlineto{\pgfqpoint{1.889528in}{5.391762in}}%
\pgfpathlineto{\pgfqpoint{1.919102in}{5.408740in}}%
\pgfpathlineto{\pgfqpoint{1.922744in}{5.410813in}}%
\pgfpathlineto{\pgfqpoint{1.948675in}{5.425392in}}%
\pgfpathlineto{\pgfqpoint{1.968151in}{5.436283in}}%
\pgfpathlineto{\pgfqpoint{1.978248in}{5.441860in}}%
\pgfpathlineto{\pgfqpoint{2.007822in}{5.458081in}}%
\pgfpathlineto{\pgfqpoint{2.014567in}{5.461752in}}%
\pgfpathlineto{\pgfqpoint{2.037395in}{5.474023in}}%
\pgfpathlineto{\pgfqpoint{2.062073in}{5.487222in}}%
\pgfpathlineto{\pgfqpoint{2.066968in}{5.489808in}}%
\pgfpathlineto{\pgfqpoint{2.096542in}{5.505300in}}%
\pgfpathlineto{\pgfqpoint{2.110737in}{5.512691in}}%
\pgfpathlineto{\pgfqpoint{2.126115in}{5.520600in}}%
\pgfpathlineto{\pgfqpoint{2.155689in}{5.535721in}}%
\pgfpathlineto{\pgfqpoint{2.160500in}{5.538161in}}%
\pgfpathlineto{\pgfqpoint{2.185262in}{5.550558in}}%
\pgfpathlineto{\pgfqpoint{2.211481in}{5.563630in}}%
\pgfpathlineto{\pgfqpoint{2.214835in}{5.565282in}}%
\pgfpathlineto{\pgfqpoint{2.244409in}{5.579711in}}%
\pgfpathlineto{\pgfqpoint{2.263744in}{5.589100in}}%
\pgfpathlineto{\pgfqpoint{2.273982in}{5.594009in}}%
\pgfpathlineto{\pgfqpoint{2.303555in}{5.608086in}}%
\pgfpathlineto{\pgfqpoint{2.317263in}{5.614570in}}%
\pgfpathlineto{\pgfqpoint{2.333129in}{5.621979in}}%
\pgfpathlineto{\pgfqpoint{2.362702in}{5.635709in}}%
\pgfpathlineto{\pgfqpoint{2.372101in}{5.640039in}}%
\pgfusepath{stroke}%
\end{pgfscope}%
\begin{pgfscope}%
\pgfpathrectangle{\pgfqpoint{0.854460in}{0.571603in}}{\pgfqpoint{5.885100in}{5.068436in}}%
\pgfusepath{clip}%
\pgfsetbuttcap%
\pgfsetroundjoin%
\pgfsetlinewidth{1.505625pt}%
\definecolor{currentstroke}{rgb}{0.120565,0.596422,0.543611}%
\pgfsetstrokecolor{currentstroke}%
\pgfsetdash{}{0pt}%
\pgfpathmoveto{\pgfqpoint{6.287072in}{5.640039in}}%
\pgfpathlineto{\pgfqpoint{6.280061in}{5.589100in}}%
\pgfpathlineto{\pgfqpoint{6.271247in}{5.538161in}}%
\pgfpathlineto{\pgfqpoint{6.255161in}{5.461752in}}%
\pgfpathlineto{\pgfqpoint{6.236465in}{5.385344in}}%
\pgfpathlineto{\pgfqpoint{6.207240in}{5.279337in}}%
\pgfpathlineto{\pgfqpoint{6.170238in}{5.156118in}}%
\pgfpathlineto{\pgfqpoint{6.082057in}{4.875953in}}%
\pgfpathlineto{\pgfqpoint{6.035356in}{4.723136in}}%
\pgfpathlineto{\pgfqpoint{5.998739in}{4.595788in}}%
\pgfpathlineto{\pgfqpoint{5.970653in}{4.490730in}}%
\pgfpathlineto{\pgfqpoint{5.946318in}{4.392032in}}%
\pgfpathlineto{\pgfqpoint{5.923601in}{4.290154in}}%
\pgfpathlineto{\pgfqpoint{5.903574in}{4.188276in}}%
\pgfpathlineto{\pgfqpoint{5.890431in}{4.111867in}}%
\pgfpathlineto{\pgfqpoint{5.879011in}{4.035459in}}%
\pgfpathlineto{\pgfqpoint{5.869335in}{3.959050in}}%
\pgfpathlineto{\pgfqpoint{5.861535in}{3.882642in}}%
\pgfpathlineto{\pgfqpoint{5.855661in}{3.806233in}}%
\pgfpathlineto{\pgfqpoint{5.851754in}{3.729825in}}%
\pgfpathlineto{\pgfqpoint{5.849856in}{3.653416in}}%
\pgfpathlineto{\pgfqpoint{5.850042in}{3.577007in}}%
\pgfpathlineto{\pgfqpoint{5.852359in}{3.500591in}}%
\pgfpathlineto{\pgfqpoint{5.856814in}{3.424190in}}%
\pgfpathlineto{\pgfqpoint{5.863473in}{3.347782in}}%
\pgfpathlineto{\pgfqpoint{5.872386in}{3.271373in}}%
\pgfpathlineto{\pgfqpoint{5.883588in}{3.194965in}}%
\pgfpathlineto{\pgfqpoint{5.897044in}{3.118556in}}%
\pgfpathlineto{\pgfqpoint{5.912876in}{3.042147in}}%
\pgfpathlineto{\pgfqpoint{5.931012in}{2.965739in}}%
\pgfpathlineto{\pgfqpoint{5.951543in}{2.889330in}}%
\pgfpathlineto{\pgfqpoint{5.974482in}{2.812922in}}%
\pgfpathlineto{\pgfqpoint{6.000226in}{2.735393in}}%
\pgfpathlineto{\pgfqpoint{6.029799in}{2.654369in}}%
\pgfpathlineto{\pgfqpoint{6.059373in}{2.579917in}}%
\pgfpathlineto{\pgfqpoint{6.090452in}{2.507287in}}%
\pgfpathlineto{\pgfqpoint{6.125559in}{2.430879in}}%
\pgfpathlineto{\pgfqpoint{6.163148in}{2.354470in}}%
\pgfpathlineto{\pgfqpoint{6.207240in}{2.270763in}}%
\pgfpathlineto{\pgfqpoint{6.245799in}{2.201653in}}%
\pgfpathlineto{\pgfqpoint{6.295960in}{2.116934in}}%
\pgfpathlineto{\pgfqpoint{6.338412in}{2.048836in}}%
\pgfpathlineto{\pgfqpoint{6.388483in}{1.972427in}}%
\pgfpathlineto{\pgfqpoint{6.443827in}{1.892103in}}%
\pgfpathlineto{\pgfqpoint{6.502973in}{1.810351in}}%
\pgfpathlineto{\pgfqpoint{6.562120in}{1.732280in}}%
\pgfpathlineto{\pgfqpoint{6.621267in}{1.657437in}}%
\pgfpathlineto{\pgfqpoint{6.680414in}{1.585446in}}%
\pgfpathlineto{\pgfqpoint{6.739560in}{1.516005in}}%
\pgfpathlineto{\pgfqpoint{6.739560in}{1.516005in}}%
\pgfusepath{stroke}%
\end{pgfscope}%
\begin{pgfscope}%
\pgfpathrectangle{\pgfqpoint{0.854460in}{0.571603in}}{\pgfqpoint{5.885100in}{5.068436in}}%
\pgfusepath{clip}%
\pgfsetbuttcap%
\pgfsetroundjoin%
\pgfsetlinewidth{1.505625pt}%
\definecolor{currentstroke}{rgb}{0.119483,0.614817,0.537692}%
\pgfsetstrokecolor{currentstroke}%
\pgfsetdash{}{0pt}%
\pgfpathmoveto{\pgfqpoint{1.102532in}{0.571603in}}%
\pgfpathlineto{\pgfqpoint{1.091047in}{0.582378in}}%
\pgfpathlineto{\pgfqpoint{1.075466in}{0.597073in}}%
\pgfpathlineto{\pgfqpoint{1.061474in}{0.610454in}}%
\pgfpathlineto{\pgfqpoint{1.048901in}{0.622542in}}%
\pgfpathlineto{\pgfqpoint{1.031901in}{0.639117in}}%
\pgfpathlineto{\pgfqpoint{1.022827in}{0.648012in}}%
\pgfpathlineto{\pgfqpoint{1.002327in}{0.668391in}}%
\pgfpathlineto{\pgfqpoint{0.997235in}{0.673481in}}%
\pgfpathlineto{\pgfqpoint{0.972754in}{0.698301in}}%
\pgfpathlineto{\pgfqpoint{0.972117in}{0.698951in}}%
\pgfpathlineto{\pgfqpoint{0.947520in}{0.724420in}}%
\pgfpathlineto{\pgfqpoint{0.943181in}{0.728979in}}%
\pgfpathlineto{\pgfqpoint{0.923394in}{0.749890in}}%
\pgfpathlineto{\pgfqpoint{0.913607in}{0.760380in}}%
\pgfpathlineto{\pgfqpoint{0.899719in}{0.775360in}}%
\pgfpathlineto{\pgfqpoint{0.884034in}{0.792520in}}%
\pgfpathlineto{\pgfqpoint{0.876487in}{0.800829in}}%
\pgfpathlineto{\pgfqpoint{0.854460in}{0.825430in}}%
\pgfusepath{stroke}%
\end{pgfscope}%
\begin{pgfscope}%
\pgfpathrectangle{\pgfqpoint{0.854460in}{0.571603in}}{\pgfqpoint{5.885100in}{5.068436in}}%
\pgfusepath{clip}%
\pgfsetbuttcap%
\pgfsetroundjoin%
\pgfsetlinewidth{1.505625pt}%
\definecolor{currentstroke}{rgb}{0.119483,0.614817,0.537692}%
\pgfsetstrokecolor{currentstroke}%
\pgfsetdash{}{0pt}%
\pgfpathmoveto{\pgfqpoint{0.854460in}{4.631264in}}%
\pgfpathlineto{\pgfqpoint{0.868311in}{4.646728in}}%
\pgfpathlineto{\pgfqpoint{0.884034in}{4.664057in}}%
\pgfpathlineto{\pgfqpoint{0.891496in}{4.672197in}}%
\pgfpathlineto{\pgfqpoint{0.913607in}{4.696008in}}%
\pgfpathlineto{\pgfqpoint{0.915163in}{4.697667in}}%
\pgfpathlineto{\pgfqpoint{0.939387in}{4.723136in}}%
\pgfpathlineto{\pgfqpoint{0.943181in}{4.727073in}}%
\pgfpathlineto{\pgfqpoint{0.964138in}{4.748606in}}%
\pgfpathlineto{\pgfqpoint{0.972754in}{4.757347in}}%
\pgfpathlineto{\pgfqpoint{0.989407in}{4.774075in}}%
\pgfpathlineto{\pgfqpoint{1.002327in}{4.786893in}}%
\pgfpathlineto{\pgfqpoint{1.015206in}{4.799545in}}%
\pgfpathlineto{\pgfqpoint{1.031901in}{4.815741in}}%
\pgfpathlineto{\pgfqpoint{1.041552in}{4.825014in}}%
\pgfpathlineto{\pgfqpoint{1.061474in}{4.843919in}}%
\pgfpathlineto{\pgfqpoint{1.068457in}{4.850484in}}%
\pgfpathlineto{\pgfqpoint{1.091047in}{4.871456in}}%
\pgfpathlineto{\pgfqpoint{1.095937in}{4.875953in}}%
\pgfpathlineto{\pgfqpoint{1.120621in}{4.898376in}}%
\pgfpathlineto{\pgfqpoint{1.124005in}{4.901423in}}%
\pgfpathlineto{\pgfqpoint{1.150194in}{4.924706in}}%
\pgfpathlineto{\pgfqpoint{1.152676in}{4.926892in}}%
\pgfpathlineto{\pgfqpoint{1.179767in}{4.950469in}}%
\pgfpathlineto{\pgfqpoint{1.181962in}{4.952362in}}%
\pgfpathlineto{\pgfqpoint{1.209341in}{4.975688in}}%
\pgfpathlineto{\pgfqpoint{1.211878in}{4.977831in}}%
\pgfpathlineto{\pgfqpoint{1.238914in}{5.000386in}}%
\pgfpathlineto{\pgfqpoint{1.242438in}{5.003301in}}%
\pgfpathlineto{\pgfqpoint{1.268488in}{5.024585in}}%
\pgfpathlineto{\pgfqpoint{1.273654in}{5.028770in}}%
\pgfpathlineto{\pgfqpoint{1.298061in}{5.048304in}}%
\pgfpathlineto{\pgfqpoint{1.305539in}{5.054240in}}%
\pgfpathlineto{\pgfqpoint{1.327634in}{5.071564in}}%
\pgfpathlineto{\pgfqpoint{1.338107in}{5.079709in}}%
\pgfpathlineto{\pgfqpoint{1.357208in}{5.094385in}}%
\pgfpathlineto{\pgfqpoint{1.371369in}{5.105179in}}%
\pgfpathlineto{\pgfqpoint{1.386781in}{5.116784in}}%
\pgfpathlineto{\pgfqpoint{1.405338in}{5.130649in}}%
\pgfpathlineto{\pgfqpoint{1.416354in}{5.138779in}}%
\pgfpathlineto{\pgfqpoint{1.440025in}{5.156118in}}%
\pgfpathlineto{\pgfqpoint{1.445928in}{5.160389in}}%
\pgfpathlineto{\pgfqpoint{1.475441in}{5.181588in}}%
\pgfpathlineto{\pgfqpoint{1.475501in}{5.181630in}}%
\pgfpathlineto{\pgfqpoint{1.505074in}{5.202428in}}%
\pgfpathlineto{\pgfqpoint{1.511704in}{5.207057in}}%
\pgfpathlineto{\pgfqpoint{1.534648in}{5.222882in}}%
\pgfpathlineto{\pgfqpoint{1.548731in}{5.232527in}}%
\pgfpathlineto{\pgfqpoint{1.564221in}{5.243007in}}%
\pgfpathlineto{\pgfqpoint{1.586530in}{5.257996in}}%
\pgfpathlineto{\pgfqpoint{1.593795in}{5.262818in}}%
\pgfpathlineto{\pgfqpoint{1.623368in}{5.282308in}}%
\pgfpathlineto{\pgfqpoint{1.625140in}{5.283466in}}%
\pgfpathlineto{\pgfqpoint{1.652941in}{5.301411in}}%
\pgfpathlineto{\pgfqpoint{1.664673in}{5.308935in}}%
\pgfpathlineto{\pgfqpoint{1.682515in}{5.320238in}}%
\pgfpathlineto{\pgfqpoint{1.705018in}{5.334405in}}%
\pgfpathlineto{\pgfqpoint{1.712088in}{5.338801in}}%
\pgfpathlineto{\pgfqpoint{1.741661in}{5.357059in}}%
\pgfpathlineto{\pgfqpoint{1.746257in}{5.359874in}}%
\pgfpathlineto{\pgfqpoint{1.771235in}{5.374987in}}%
\pgfpathlineto{\pgfqpoint{1.788453in}{5.385344in}}%
\pgfpathlineto{\pgfqpoint{1.800808in}{5.392684in}}%
\pgfpathlineto{\pgfqpoint{1.830381in}{5.410150in}}%
\pgfpathlineto{\pgfqpoint{1.831514in}{5.410813in}}%
\pgfpathlineto{\pgfqpoint{1.859955in}{5.427259in}}%
\pgfpathlineto{\pgfqpoint{1.875646in}{5.436283in}}%
\pgfpathlineto{\pgfqpoint{1.889528in}{5.444168in}}%
\pgfpathlineto{\pgfqpoint{1.919102in}{5.460872in}}%
\pgfpathlineto{\pgfqpoint{1.920673in}{5.461752in}}%
\pgfpathlineto{\pgfqpoint{1.948675in}{5.477237in}}%
\pgfpathlineto{\pgfqpoint{1.966821in}{5.487222in}}%
\pgfpathlineto{\pgfqpoint{1.978248in}{5.493432in}}%
\pgfpathlineto{\pgfqpoint{2.007822in}{5.509401in}}%
\pgfpathlineto{\pgfqpoint{2.013962in}{5.512691in}}%
\pgfpathlineto{\pgfqpoint{2.037395in}{5.525093in}}%
\pgfpathlineto{\pgfqpoint{2.062201in}{5.538161in}}%
\pgfpathlineto{\pgfqpoint{2.066968in}{5.540641in}}%
\pgfpathlineto{\pgfqpoint{2.096542in}{5.555903in}}%
\pgfpathlineto{\pgfqpoint{2.111600in}{5.563630in}}%
\pgfpathlineto{\pgfqpoint{2.126115in}{5.570987in}}%
\pgfpathlineto{\pgfqpoint{2.155689in}{5.585890in}}%
\pgfpathlineto{\pgfqpoint{2.162109in}{5.589100in}}%
\pgfpathlineto{\pgfqpoint{2.185262in}{5.600532in}}%
\pgfpathlineto{\pgfqpoint{2.213800in}{5.614570in}}%
\pgfpathlineto{\pgfqpoint{2.214835in}{5.615072in}}%
\pgfpathlineto{\pgfqpoint{2.244409in}{5.629303in}}%
\pgfpathlineto{\pgfqpoint{2.266804in}{5.640039in}}%
\pgfusepath{stroke}%
\end{pgfscope}%
\begin{pgfscope}%
\pgfpathrectangle{\pgfqpoint{0.854460in}{0.571603in}}{\pgfqpoint{5.885100in}{5.068436in}}%
\pgfusepath{clip}%
\pgfsetbuttcap%
\pgfsetroundjoin%
\pgfsetlinewidth{1.505625pt}%
\definecolor{currentstroke}{rgb}{0.119483,0.614817,0.537692}%
\pgfsetstrokecolor{currentstroke}%
\pgfsetdash{}{0pt}%
\pgfpathmoveto{\pgfqpoint{6.407925in}{5.640039in}}%
\pgfpathlineto{\pgfqpoint{6.397530in}{5.589100in}}%
\pgfpathlineto{\pgfqpoint{6.379250in}{5.512691in}}%
\pgfpathlineto{\pgfqpoint{6.355107in}{5.425093in}}%
\pgfpathlineto{\pgfqpoint{6.327588in}{5.334405in}}%
\pgfpathlineto{\pgfqpoint{6.286015in}{5.207057in}}%
\pgfpathlineto{\pgfqpoint{6.118520in}{4.705892in}}%
\pgfpathlineto{\pgfqpoint{6.085166in}{4.595788in}}%
\pgfpathlineto{\pgfqpoint{6.056395in}{4.493910in}}%
\pgfpathlineto{\pgfqpoint{6.029799in}{4.391505in}}%
\pgfpathlineto{\pgfqpoint{6.005982in}{4.290154in}}%
\pgfpathlineto{\pgfqpoint{5.984826in}{4.188276in}}%
\pgfpathlineto{\pgfqpoint{5.970653in}{4.110091in}}%
\pgfpathlineto{\pgfqpoint{5.958767in}{4.035459in}}%
\pgfpathlineto{\pgfqpoint{5.948450in}{3.959050in}}%
\pgfpathlineto{\pgfqpoint{5.940040in}{3.882642in}}%
\pgfpathlineto{\pgfqpoint{5.933545in}{3.806233in}}%
\pgfpathlineto{\pgfqpoint{5.929072in}{3.729825in}}%
\pgfpathlineto{\pgfqpoint{5.926667in}{3.653416in}}%
\pgfpathlineto{\pgfqpoint{5.926370in}{3.577007in}}%
\pgfpathlineto{\pgfqpoint{5.928226in}{3.500599in}}%
\pgfpathlineto{\pgfqpoint{5.932278in}{3.424190in}}%
\pgfpathlineto{\pgfqpoint{5.938571in}{3.347782in}}%
\pgfpathlineto{\pgfqpoint{5.947099in}{3.271373in}}%
\pgfpathlineto{\pgfqpoint{5.957922in}{3.194965in}}%
\pgfpathlineto{\pgfqpoint{5.971105in}{3.118556in}}%
\pgfpathlineto{\pgfqpoint{5.986580in}{3.042147in}}%
\pgfpathlineto{\pgfqpoint{6.004471in}{2.965739in}}%
\pgfpathlineto{\pgfqpoint{6.024737in}{2.889330in}}%
\pgfpathlineto{\pgfqpoint{6.047406in}{2.812922in}}%
\pgfpathlineto{\pgfqpoint{6.072516in}{2.736513in}}%
\pgfpathlineto{\pgfqpoint{6.100075in}{2.660104in}}%
\pgfpathlineto{\pgfqpoint{6.130090in}{2.583696in}}%
\pgfpathlineto{\pgfqpoint{6.162577in}{2.507287in}}%
\pgfpathlineto{\pgfqpoint{6.197550in}{2.430879in}}%
\pgfpathlineto{\pgfqpoint{6.236813in}{2.350997in}}%
\pgfpathlineto{\pgfqpoint{6.274972in}{2.278062in}}%
\pgfpathlineto{\pgfqpoint{6.317437in}{2.201653in}}%
\pgfpathlineto{\pgfqpoint{6.362413in}{2.125244in}}%
\pgfpathlineto{\pgfqpoint{6.414253in}{2.042095in}}%
\pgfpathlineto{\pgfqpoint{6.459898in}{1.972427in}}%
\pgfpathlineto{\pgfqpoint{6.512423in}{1.896019in}}%
\pgfpathlineto{\pgfqpoint{6.567471in}{1.819610in}}%
\pgfpathlineto{\pgfqpoint{6.625032in}{1.743202in}}%
\pgfpathlineto{\pgfqpoint{6.685103in}{1.666793in}}%
\pgfpathlineto{\pgfqpoint{6.739560in}{1.600178in}}%
\pgfpathlineto{\pgfqpoint{6.739560in}{1.600178in}}%
\pgfusepath{stroke}%
\end{pgfscope}%
\begin{pgfscope}%
\pgfpathrectangle{\pgfqpoint{0.854460in}{0.571603in}}{\pgfqpoint{5.885100in}{5.068436in}}%
\pgfusepath{clip}%
\pgfsetbuttcap%
\pgfsetroundjoin%
\pgfsetlinewidth{1.505625pt}%
\definecolor{currentstroke}{rgb}{0.123444,0.636809,0.528763}%
\pgfsetstrokecolor{currentstroke}%
\pgfsetdash{}{0pt}%
\pgfpathmoveto{\pgfqpoint{1.046014in}{0.571603in}}%
\pgfpathlineto{\pgfqpoint{1.031901in}{0.584962in}}%
\pgfpathlineto{\pgfqpoint{1.019173in}{0.597073in}}%
\pgfpathlineto{\pgfqpoint{1.002327in}{0.613327in}}%
\pgfpathlineto{\pgfqpoint{0.992827in}{0.622542in}}%
\pgfpathlineto{\pgfqpoint{0.972754in}{0.642287in}}%
\pgfpathlineto{\pgfqpoint{0.966966in}{0.648012in}}%
\pgfpathlineto{\pgfqpoint{0.943181in}{0.671868in}}%
\pgfpathlineto{\pgfqpoint{0.941581in}{0.673481in}}%
\pgfpathlineto{\pgfqpoint{0.916704in}{0.698951in}}%
\pgfpathlineto{\pgfqpoint{0.913607in}{0.702168in}}%
\pgfpathlineto{\pgfqpoint{0.892313in}{0.724420in}}%
\pgfpathlineto{\pgfqpoint{0.884034in}{0.733194in}}%
\pgfpathlineto{\pgfqpoint{0.868376in}{0.749890in}}%
\pgfpathlineto{\pgfqpoint{0.854460in}{0.764940in}}%
\pgfusepath{stroke}%
\end{pgfscope}%
\begin{pgfscope}%
\pgfpathrectangle{\pgfqpoint{0.854460in}{0.571603in}}{\pgfqpoint{5.885100in}{5.068436in}}%
\pgfusepath{clip}%
\pgfsetbuttcap%
\pgfsetroundjoin%
\pgfsetlinewidth{1.505625pt}%
\definecolor{currentstroke}{rgb}{0.123444,0.636809,0.528763}%
\pgfsetstrokecolor{currentstroke}%
\pgfsetdash{}{0pt}%
\pgfpathmoveto{\pgfqpoint{0.854460in}{4.703637in}}%
\pgfpathlineto{\pgfqpoint{0.872690in}{4.723136in}}%
\pgfpathlineto{\pgfqpoint{0.884034in}{4.735116in}}%
\pgfpathlineto{\pgfqpoint{0.896934in}{4.748606in}}%
\pgfpathlineto{\pgfqpoint{0.913607in}{4.765820in}}%
\pgfpathlineto{\pgfqpoint{0.921682in}{4.774075in}}%
\pgfpathlineto{\pgfqpoint{0.943181in}{4.795779in}}%
\pgfpathlineto{\pgfqpoint{0.946947in}{4.799545in}}%
\pgfpathlineto{\pgfqpoint{0.972744in}{4.825014in}}%
\pgfpathlineto{\pgfqpoint{0.972754in}{4.825024in}}%
\pgfpathlineto{\pgfqpoint{0.999137in}{4.850484in}}%
\pgfpathlineto{\pgfqpoint{1.002327in}{4.853525in}}%
\pgfpathlineto{\pgfqpoint{1.026081in}{4.875953in}}%
\pgfpathlineto{\pgfqpoint{1.031901in}{4.881381in}}%
\pgfpathlineto{\pgfqpoint{1.053590in}{4.901423in}}%
\pgfpathlineto{\pgfqpoint{1.061474in}{4.908618in}}%
\pgfpathlineto{\pgfqpoint{1.081678in}{4.926892in}}%
\pgfpathlineto{\pgfqpoint{1.091047in}{4.935263in}}%
\pgfpathlineto{\pgfqpoint{1.110358in}{4.952362in}}%
\pgfpathlineto{\pgfqpoint{1.120621in}{4.961338in}}%
\pgfpathlineto{\pgfqpoint{1.139643in}{4.977831in}}%
\pgfpathlineto{\pgfqpoint{1.150194in}{4.986868in}}%
\pgfpathlineto{\pgfqpoint{1.169547in}{5.003301in}}%
\pgfpathlineto{\pgfqpoint{1.179767in}{5.011873in}}%
\pgfpathlineto{\pgfqpoint{1.200083in}{5.028770in}}%
\pgfpathlineto{\pgfqpoint{1.209341in}{5.036377in}}%
\pgfpathlineto{\pgfqpoint{1.231263in}{5.054240in}}%
\pgfpathlineto{\pgfqpoint{1.238914in}{5.060399in}}%
\pgfpathlineto{\pgfqpoint{1.263100in}{5.079709in}}%
\pgfpathlineto{\pgfqpoint{1.268488in}{5.083959in}}%
\pgfpathlineto{\pgfqpoint{1.295606in}{5.105179in}}%
\pgfpathlineto{\pgfqpoint{1.298061in}{5.107076in}}%
\pgfpathlineto{\pgfqpoint{1.327634in}{5.129753in}}%
\pgfpathlineto{\pgfqpoint{1.328812in}{5.130649in}}%
\pgfpathlineto{\pgfqpoint{1.357208in}{5.151978in}}%
\pgfpathlineto{\pgfqpoint{1.362761in}{5.156118in}}%
\pgfpathlineto{\pgfqpoint{1.386781in}{5.173807in}}%
\pgfpathlineto{\pgfqpoint{1.397426in}{5.181588in}}%
\pgfpathlineto{\pgfqpoint{1.416354in}{5.195256in}}%
\pgfpathlineto{\pgfqpoint{1.432816in}{5.207057in}}%
\pgfpathlineto{\pgfqpoint{1.445928in}{5.216343in}}%
\pgfpathlineto{\pgfqpoint{1.468943in}{5.232527in}}%
\pgfpathlineto{\pgfqpoint{1.475501in}{5.237083in}}%
\pgfpathlineto{\pgfqpoint{1.505074in}{5.257481in}}%
\pgfpathlineto{\pgfqpoint{1.505828in}{5.257996in}}%
\pgfpathlineto{\pgfqpoint{1.534648in}{5.277467in}}%
\pgfpathlineto{\pgfqpoint{1.543587in}{5.283466in}}%
\pgfpathlineto{\pgfqpoint{1.564221in}{5.297145in}}%
\pgfpathlineto{\pgfqpoint{1.582123in}{5.308935in}}%
\pgfpathlineto{\pgfqpoint{1.593795in}{5.316529in}}%
\pgfpathlineto{\pgfqpoint{1.621445in}{5.334405in}}%
\pgfpathlineto{\pgfqpoint{1.623368in}{5.335633in}}%
\pgfpathlineto{\pgfqpoint{1.652941in}{5.354364in}}%
\pgfpathlineto{\pgfqpoint{1.661700in}{5.359874in}}%
\pgfpathlineto{\pgfqpoint{1.682515in}{5.372811in}}%
\pgfpathlineto{\pgfqpoint{1.702799in}{5.385344in}}%
\pgfpathlineto{\pgfqpoint{1.712088in}{5.391013in}}%
\pgfpathlineto{\pgfqpoint{1.741661in}{5.408944in}}%
\pgfpathlineto{\pgfqpoint{1.744769in}{5.410813in}}%
\pgfpathlineto{\pgfqpoint{1.771235in}{5.426538in}}%
\pgfpathlineto{\pgfqpoint{1.787729in}{5.436283in}}%
\pgfpathlineto{\pgfqpoint{1.800808in}{5.443916in}}%
\pgfpathlineto{\pgfqpoint{1.830381in}{5.461079in}}%
\pgfpathlineto{\pgfqpoint{1.831551in}{5.461752in}}%
\pgfpathlineto{\pgfqpoint{1.859955in}{5.477895in}}%
\pgfpathlineto{\pgfqpoint{1.876450in}{5.487222in}}%
\pgfpathlineto{\pgfqpoint{1.889528in}{5.494526in}}%
\pgfpathlineto{\pgfqpoint{1.919102in}{5.510949in}}%
\pgfpathlineto{\pgfqpoint{1.922265in}{5.512691in}}%
\pgfpathlineto{\pgfqpoint{1.948675in}{5.527059in}}%
\pgfpathlineto{\pgfqpoint{1.969178in}{5.538161in}}%
\pgfpathlineto{\pgfqpoint{1.978248in}{5.543012in}}%
\pgfpathlineto{\pgfqpoint{2.007822in}{5.558722in}}%
\pgfpathlineto{\pgfqpoint{2.017125in}{5.563630in}}%
\pgfpathlineto{\pgfqpoint{2.037395in}{5.574194in}}%
\pgfpathlineto{\pgfqpoint{2.066119in}{5.589100in}}%
\pgfpathlineto{\pgfqpoint{2.066968in}{5.589535in}}%
\pgfpathlineto{\pgfqpoint{2.096542in}{5.604558in}}%
\pgfpathlineto{\pgfqpoint{2.116334in}{5.614570in}}%
\pgfpathlineto{\pgfqpoint{2.126115in}{5.619456in}}%
\pgfpathlineto{\pgfqpoint{2.155689in}{5.634130in}}%
\pgfpathlineto{\pgfqpoint{2.167670in}{5.640039in}}%
\pgfusepath{stroke}%
\end{pgfscope}%
\begin{pgfscope}%
\pgfpathrectangle{\pgfqpoint{0.854460in}{0.571603in}}{\pgfqpoint{5.885100in}{5.068436in}}%
\pgfusepath{clip}%
\pgfsetbuttcap%
\pgfsetroundjoin%
\pgfsetlinewidth{1.505625pt}%
\definecolor{currentstroke}{rgb}{0.123444,0.636809,0.528763}%
\pgfsetstrokecolor{currentstroke}%
\pgfsetdash{}{0pt}%
\pgfpathmoveto{\pgfqpoint{6.522648in}{5.640039in}}%
\pgfpathlineto{\pgfqpoint{6.502284in}{5.563630in}}%
\pgfpathlineto{\pgfqpoint{6.473400in}{5.468523in}}%
\pgfpathlineto{\pgfqpoint{6.445907in}{5.385344in}}%
\pgfpathlineto{\pgfqpoint{6.401092in}{5.257996in}}%
\pgfpathlineto{\pgfqpoint{6.227004in}{4.774075in}}%
\pgfpathlineto{\pgfqpoint{6.185146in}{4.646728in}}%
\pgfpathlineto{\pgfqpoint{6.148093in}{4.525059in}}%
\pgfpathlineto{\pgfqpoint{6.124906in}{4.442971in}}%
\pgfpathlineto{\pgfqpoint{6.098477in}{4.341093in}}%
\pgfpathlineto{\pgfqpoint{6.074830in}{4.239215in}}%
\pgfpathlineto{\pgfqpoint{6.059075in}{4.162807in}}%
\pgfpathlineto{\pgfqpoint{6.045015in}{4.086398in}}%
\pgfpathlineto{\pgfqpoint{6.032833in}{4.009989in}}%
\pgfpathlineto{\pgfqpoint{6.022511in}{3.933581in}}%
\pgfpathlineto{\pgfqpoint{6.014146in}{3.857172in}}%
\pgfpathlineto{\pgfqpoint{6.007806in}{3.780764in}}%
\pgfpathlineto{\pgfqpoint{6.003531in}{3.704355in}}%
\pgfpathlineto{\pgfqpoint{6.001361in}{3.627946in}}%
\pgfpathlineto{\pgfqpoint{6.001335in}{3.551538in}}%
\pgfpathlineto{\pgfqpoint{6.003495in}{3.475129in}}%
\pgfpathlineto{\pgfqpoint{6.007881in}{3.398721in}}%
\pgfpathlineto{\pgfqpoint{6.014537in}{3.322312in}}%
\pgfpathlineto{\pgfqpoint{6.023506in}{3.245904in}}%
\pgfpathlineto{\pgfqpoint{6.034794in}{3.169495in}}%
\pgfpathlineto{\pgfqpoint{6.048420in}{3.093086in}}%
\pgfpathlineto{\pgfqpoint{6.064444in}{3.016678in}}%
\pgfpathlineto{\pgfqpoint{6.082854in}{2.940269in}}%
\pgfpathlineto{\pgfqpoint{6.103674in}{2.863861in}}%
\pgfpathlineto{\pgfqpoint{6.126947in}{2.787452in}}%
\pgfpathlineto{\pgfqpoint{6.152676in}{2.711044in}}%
\pgfpathlineto{\pgfqpoint{6.180867in}{2.634635in}}%
\pgfpathlineto{\pgfqpoint{6.211533in}{2.558226in}}%
\pgfpathlineto{\pgfqpoint{6.244684in}{2.481818in}}%
\pgfpathlineto{\pgfqpoint{6.280339in}{2.405409in}}%
\pgfpathlineto{\pgfqpoint{6.318519in}{2.329001in}}%
\pgfpathlineto{\pgfqpoint{6.359214in}{2.252592in}}%
\pgfpathlineto{\pgfqpoint{6.402411in}{2.176183in}}%
\pgfpathlineto{\pgfqpoint{6.448163in}{2.099775in}}%
\pgfpathlineto{\pgfqpoint{6.502973in}{2.013371in}}%
\pgfpathlineto{\pgfqpoint{6.547219in}{1.946958in}}%
\pgfpathlineto{\pgfqpoint{6.600560in}{1.870549in}}%
\pgfpathlineto{\pgfqpoint{6.656437in}{1.794141in}}%
\pgfpathlineto{\pgfqpoint{6.714843in}{1.717732in}}%
\pgfpathlineto{\pgfqpoint{6.739560in}{1.686393in}}%
\pgfpathlineto{\pgfqpoint{6.739560in}{1.686393in}}%
\pgfusepath{stroke}%
\end{pgfscope}%
\begin{pgfscope}%
\pgfpathrectangle{\pgfqpoint{0.854460in}{0.571603in}}{\pgfqpoint{5.885100in}{5.068436in}}%
\pgfusepath{clip}%
\pgfsetbuttcap%
\pgfsetroundjoin%
\pgfsetlinewidth{1.505625pt}%
\definecolor{currentstroke}{rgb}{0.132268,0.655014,0.519661}%
\pgfsetstrokecolor{currentstroke}%
\pgfsetdash{}{0pt}%
\pgfpathmoveto{\pgfqpoint{0.990838in}{0.571603in}}%
\pgfpathlineto{\pgfqpoint{0.972754in}{0.588875in}}%
\pgfpathlineto{\pgfqpoint{0.964215in}{0.597073in}}%
\pgfpathlineto{\pgfqpoint{0.943181in}{0.617549in}}%
\pgfpathlineto{\pgfqpoint{0.938079in}{0.622542in}}%
\pgfpathlineto{\pgfqpoint{0.913607in}{0.646830in}}%
\pgfpathlineto{\pgfqpoint{0.912423in}{0.648012in}}%
\pgfpathlineto{\pgfqpoint{0.887278in}{0.673481in}}%
\pgfpathlineto{\pgfqpoint{0.884034in}{0.676815in}}%
\pgfpathlineto{\pgfqpoint{0.862616in}{0.698951in}}%
\pgfpathlineto{\pgfqpoint{0.854460in}{0.707499in}}%
\pgfusepath{stroke}%
\end{pgfscope}%
\begin{pgfscope}%
\pgfpathrectangle{\pgfqpoint{0.854460in}{0.571603in}}{\pgfqpoint{5.885100in}{5.068436in}}%
\pgfusepath{clip}%
\pgfsetbuttcap%
\pgfsetroundjoin%
\pgfsetlinewidth{1.505625pt}%
\definecolor{currentstroke}{rgb}{0.132268,0.655014,0.519661}%
\pgfsetstrokecolor{currentstroke}%
\pgfsetdash{}{0pt}%
\pgfpathmoveto{\pgfqpoint{0.854460in}{4.772166in}}%
\pgfpathlineto{\pgfqpoint{0.856296in}{4.774075in}}%
\pgfpathlineto{\pgfqpoint{0.881110in}{4.799545in}}%
\pgfpathlineto{\pgfqpoint{0.884034in}{4.802507in}}%
\pgfpathlineto{\pgfqpoint{0.906462in}{4.825014in}}%
\pgfpathlineto{\pgfqpoint{0.913607in}{4.832095in}}%
\pgfpathlineto{\pgfqpoint{0.932338in}{4.850484in}}%
\pgfpathlineto{\pgfqpoint{0.943181in}{4.860996in}}%
\pgfpathlineto{\pgfqpoint{0.958751in}{4.875953in}}%
\pgfpathlineto{\pgfqpoint{0.972754in}{4.889237in}}%
\pgfpathlineto{\pgfqpoint{0.985716in}{4.901423in}}%
\pgfpathlineto{\pgfqpoint{1.002327in}{4.916845in}}%
\pgfpathlineto{\pgfqpoint{1.013246in}{4.926892in}}%
\pgfpathlineto{\pgfqpoint{1.031901in}{4.943846in}}%
\pgfpathlineto{\pgfqpoint{1.041354in}{4.952362in}}%
\pgfpathlineto{\pgfqpoint{1.061474in}{4.970264in}}%
\pgfpathlineto{\pgfqpoint{1.070054in}{4.977831in}}%
\pgfpathlineto{\pgfqpoint{1.091047in}{4.996122in}}%
\pgfpathlineto{\pgfqpoint{1.099358in}{5.003301in}}%
\pgfpathlineto{\pgfqpoint{1.120621in}{5.021443in}}%
\pgfpathlineto{\pgfqpoint{1.129280in}{5.028770in}}%
\pgfpathlineto{\pgfqpoint{1.150194in}{5.046250in}}%
\pgfpathlineto{\pgfqpoint{1.159833in}{5.054240in}}%
\pgfpathlineto{\pgfqpoint{1.179767in}{5.070563in}}%
\pgfpathlineto{\pgfqpoint{1.191029in}{5.079709in}}%
\pgfpathlineto{\pgfqpoint{1.209341in}{5.094402in}}%
\pgfpathlineto{\pgfqpoint{1.222880in}{5.105179in}}%
\pgfpathlineto{\pgfqpoint{1.238914in}{5.117787in}}%
\pgfpathlineto{\pgfqpoint{1.255398in}{5.130649in}}%
\pgfpathlineto{\pgfqpoint{1.268488in}{5.140737in}}%
\pgfpathlineto{\pgfqpoint{1.288595in}{5.156118in}}%
\pgfpathlineto{\pgfqpoint{1.298061in}{5.163271in}}%
\pgfpathlineto{\pgfqpoint{1.322483in}{5.181588in}}%
\pgfpathlineto{\pgfqpoint{1.327634in}{5.185404in}}%
\pgfpathlineto{\pgfqpoint{1.357072in}{5.207057in}}%
\pgfpathlineto{\pgfqpoint{1.357208in}{5.207156in}}%
\pgfpathlineto{\pgfqpoint{1.386781in}{5.228464in}}%
\pgfpathlineto{\pgfqpoint{1.392460in}{5.232527in}}%
\pgfpathlineto{\pgfqpoint{1.416354in}{5.249413in}}%
\pgfpathlineto{\pgfqpoint{1.428584in}{5.257996in}}%
\pgfpathlineto{\pgfqpoint{1.445928in}{5.270021in}}%
\pgfpathlineto{\pgfqpoint{1.465450in}{5.283466in}}%
\pgfpathlineto{\pgfqpoint{1.475501in}{5.290303in}}%
\pgfpathlineto{\pgfqpoint{1.503069in}{5.308935in}}%
\pgfpathlineto{\pgfqpoint{1.505074in}{5.310274in}}%
\pgfpathlineto{\pgfqpoint{1.534648in}{5.329860in}}%
\pgfpathlineto{\pgfqpoint{1.541556in}{5.334405in}}%
\pgfpathlineto{\pgfqpoint{1.564221in}{5.349132in}}%
\pgfpathlineto{\pgfqpoint{1.580856in}{5.359874in}}%
\pgfpathlineto{\pgfqpoint{1.593795in}{5.368128in}}%
\pgfpathlineto{\pgfqpoint{1.620945in}{5.385344in}}%
\pgfpathlineto{\pgfqpoint{1.623368in}{5.386862in}}%
\pgfpathlineto{\pgfqpoint{1.652941in}{5.405239in}}%
\pgfpathlineto{\pgfqpoint{1.661970in}{5.410813in}}%
\pgfpathlineto{\pgfqpoint{1.682515in}{5.423344in}}%
\pgfpathlineto{\pgfqpoint{1.703848in}{5.436283in}}%
\pgfpathlineto{\pgfqpoint{1.712088in}{5.441220in}}%
\pgfpathlineto{\pgfqpoint{1.741661in}{5.458821in}}%
\pgfpathlineto{\pgfqpoint{1.746624in}{5.461752in}}%
\pgfpathlineto{\pgfqpoint{1.771235in}{5.476114in}}%
\pgfpathlineto{\pgfqpoint{1.790369in}{5.487222in}}%
\pgfpathlineto{\pgfqpoint{1.800808in}{5.493208in}}%
\pgfpathlineto{\pgfqpoint{1.830381in}{5.510063in}}%
\pgfpathlineto{\pgfqpoint{1.835028in}{5.512691in}}%
\pgfpathlineto{\pgfqpoint{1.859955in}{5.526618in}}%
\pgfpathlineto{\pgfqpoint{1.880714in}{5.538161in}}%
\pgfpathlineto{\pgfqpoint{1.889528in}{5.543002in}}%
\pgfpathlineto{\pgfqpoint{1.919102in}{5.559138in}}%
\pgfpathlineto{\pgfqpoint{1.927391in}{5.563630in}}%
\pgfpathlineto{\pgfqpoint{1.948675in}{5.575024in}}%
\pgfpathlineto{\pgfqpoint{1.975084in}{5.589100in}}%
\pgfpathlineto{\pgfqpoint{1.978248in}{5.590766in}}%
\pgfpathlineto{\pgfqpoint{2.007822in}{5.606209in}}%
\pgfpathlineto{\pgfqpoint{2.023909in}{5.614570in}}%
\pgfpathlineto{\pgfqpoint{2.037395in}{5.621491in}}%
\pgfpathlineto{\pgfqpoint{2.066968in}{5.636586in}}%
\pgfpathlineto{\pgfqpoint{2.073783in}{5.640039in}}%
\pgfusepath{stroke}%
\end{pgfscope}%
\begin{pgfscope}%
\pgfpathrectangle{\pgfqpoint{0.854460in}{0.571603in}}{\pgfqpoint{5.885100in}{5.068436in}}%
\pgfusepath{clip}%
\pgfsetbuttcap%
\pgfsetroundjoin%
\pgfsetlinewidth{1.505625pt}%
\definecolor{currentstroke}{rgb}{0.132268,0.655014,0.519661}%
\pgfsetstrokecolor{currentstroke}%
\pgfsetdash{}{0pt}%
\pgfpathmoveto{\pgfqpoint{6.632130in}{5.640039in}}%
\pgfpathlineto{\pgfqpoint{6.608014in}{5.563630in}}%
\pgfpathlineto{\pgfqpoint{6.581840in}{5.487222in}}%
\pgfpathlineto{\pgfqpoint{6.544640in}{5.385344in}}%
\pgfpathlineto{\pgfqpoint{6.495906in}{5.257996in}}%
\pgfpathlineto{\pgfqpoint{6.355107in}{4.893584in}}%
\pgfpathlineto{\pgfqpoint{6.311512in}{4.774075in}}%
\pgfpathlineto{\pgfqpoint{6.267783in}{4.646728in}}%
\pgfpathlineto{\pgfqpoint{6.235165in}{4.544849in}}%
\pgfpathlineto{\pgfqpoint{6.204968in}{4.442971in}}%
\pgfpathlineto{\pgfqpoint{6.177445in}{4.341093in}}%
\pgfpathlineto{\pgfqpoint{6.158647in}{4.264685in}}%
\pgfpathlineto{\pgfqpoint{6.136280in}{4.162807in}}%
\pgfpathlineto{\pgfqpoint{6.118520in}{4.068753in}}%
\pgfpathlineto{\pgfqpoint{6.108787in}{4.009989in}}%
\pgfpathlineto{\pgfqpoint{6.097906in}{3.933581in}}%
\pgfpathlineto{\pgfqpoint{6.088946in}{3.856225in}}%
\pgfpathlineto{\pgfqpoint{6.082168in}{3.780764in}}%
\pgfpathlineto{\pgfqpoint{6.077403in}{3.704355in}}%
\pgfpathlineto{\pgfqpoint{6.074781in}{3.627946in}}%
\pgfpathlineto{\pgfqpoint{6.074341in}{3.551538in}}%
\pgfpathlineto{\pgfqpoint{6.076119in}{3.475129in}}%
\pgfpathlineto{\pgfqpoint{6.080156in}{3.398721in}}%
\pgfpathlineto{\pgfqpoint{6.086492in}{3.322312in}}%
\pgfpathlineto{\pgfqpoint{6.095120in}{3.245904in}}%
\pgfpathlineto{\pgfqpoint{6.106096in}{3.169495in}}%
\pgfpathlineto{\pgfqpoint{6.119477in}{3.093086in}}%
\pgfpathlineto{\pgfqpoint{6.135204in}{3.016678in}}%
\pgfpathlineto{\pgfqpoint{6.153385in}{2.940269in}}%
\pgfpathlineto{\pgfqpoint{6.173994in}{2.863861in}}%
\pgfpathlineto{\pgfqpoint{6.197040in}{2.787452in}}%
\pgfpathlineto{\pgfqpoint{6.222566in}{2.711044in}}%
\pgfpathlineto{\pgfqpoint{6.250578in}{2.634635in}}%
\pgfpathlineto{\pgfqpoint{6.281087in}{2.558226in}}%
\pgfpathlineto{\pgfqpoint{6.314104in}{2.481818in}}%
\pgfpathlineto{\pgfqpoint{6.355107in}{2.394229in}}%
\pgfpathlineto{\pgfqpoint{6.387712in}{2.329001in}}%
\pgfpathlineto{\pgfqpoint{6.428285in}{2.252592in}}%
\pgfpathlineto{\pgfqpoint{6.473400in}{2.172840in}}%
\pgfpathlineto{\pgfqpoint{6.517084in}{2.099775in}}%
\pgfpathlineto{\pgfqpoint{6.565318in}{2.023366in}}%
\pgfpathlineto{\pgfqpoint{6.621267in}{1.939412in}}%
\pgfpathlineto{\pgfqpoint{6.669382in}{1.870549in}}%
\pgfpathlineto{\pgfqpoint{6.725245in}{1.794141in}}%
\pgfpathlineto{\pgfqpoint{6.739560in}{1.775164in}}%
\pgfpathlineto{\pgfqpoint{6.739560in}{1.775164in}}%
\pgfusepath{stroke}%
\end{pgfscope}%
\begin{pgfscope}%
\pgfpathrectangle{\pgfqpoint{0.854460in}{0.571603in}}{\pgfqpoint{5.885100in}{5.068436in}}%
\pgfusepath{clip}%
\pgfsetbuttcap%
\pgfsetroundjoin%
\pgfsetlinewidth{1.505625pt}%
\definecolor{currentstroke}{rgb}{0.146616,0.673050,0.508936}%
\pgfsetstrokecolor{currentstroke}%
\pgfsetdash{}{0pt}%
\pgfpathmoveto{\pgfqpoint{0.936926in}{0.571603in}}%
\pgfpathlineto{\pgfqpoint{0.913607in}{0.594074in}}%
\pgfpathlineto{\pgfqpoint{0.910512in}{0.597073in}}%
\pgfpathlineto{\pgfqpoint{0.884585in}{0.622542in}}%
\pgfpathlineto{\pgfqpoint{0.884034in}{0.623092in}}%
\pgfpathlineto{\pgfqpoint{0.859178in}{0.648012in}}%
\pgfpathlineto{\pgfqpoint{0.854460in}{0.652807in}}%
\pgfusepath{stroke}%
\end{pgfscope}%
\begin{pgfscope}%
\pgfpathrectangle{\pgfqpoint{0.854460in}{0.571603in}}{\pgfqpoint{5.885100in}{5.068436in}}%
\pgfusepath{clip}%
\pgfsetbuttcap%
\pgfsetroundjoin%
\pgfsetlinewidth{1.505625pt}%
\definecolor{currentstroke}{rgb}{0.146616,0.673050,0.508936}%
\pgfsetstrokecolor{currentstroke}%
\pgfsetdash{}{0pt}%
\pgfpathmoveto{\pgfqpoint{0.854460in}{4.837189in}}%
\pgfpathlineto{\pgfqpoint{0.867775in}{4.850484in}}%
\pgfpathlineto{\pgfqpoint{0.884034in}{4.866516in}}%
\pgfpathlineto{\pgfqpoint{0.893694in}{4.875953in}}%
\pgfpathlineto{\pgfqpoint{0.913607in}{4.895167in}}%
\pgfpathlineto{\pgfqpoint{0.920150in}{4.901423in}}%
\pgfpathlineto{\pgfqpoint{0.943181in}{4.923170in}}%
\pgfpathlineto{\pgfqpoint{0.947158in}{4.926892in}}%
\pgfpathlineto{\pgfqpoint{0.972754in}{4.950551in}}%
\pgfpathlineto{\pgfqpoint{0.974730in}{4.952362in}}%
\pgfpathlineto{\pgfqpoint{1.002327in}{4.977335in}}%
\pgfpathlineto{\pgfqpoint{1.002881in}{4.977831in}}%
\pgfpathlineto{\pgfqpoint{1.031627in}{5.003301in}}%
\pgfpathlineto{\pgfqpoint{1.031901in}{5.003540in}}%
\pgfpathlineto{\pgfqpoint{1.060976in}{5.028770in}}%
\pgfpathlineto{\pgfqpoint{1.061474in}{5.029197in}}%
\pgfpathlineto{\pgfqpoint{1.090933in}{5.054240in}}%
\pgfpathlineto{\pgfqpoint{1.091047in}{5.054336in}}%
\pgfpathlineto{\pgfqpoint{1.120621in}{5.078963in}}%
\pgfpathlineto{\pgfqpoint{1.121524in}{5.079709in}}%
\pgfpathlineto{\pgfqpoint{1.150194in}{5.103104in}}%
\pgfpathlineto{\pgfqpoint{1.152758in}{5.105179in}}%
\pgfpathlineto{\pgfqpoint{1.179767in}{5.126778in}}%
\pgfpathlineto{\pgfqpoint{1.184645in}{5.130649in}}%
\pgfpathlineto{\pgfqpoint{1.209341in}{5.150006in}}%
\pgfpathlineto{\pgfqpoint{1.217198in}{5.156118in}}%
\pgfpathlineto{\pgfqpoint{1.238914in}{5.172805in}}%
\pgfpathlineto{\pgfqpoint{1.250428in}{5.181588in}}%
\pgfpathlineto{\pgfqpoint{1.268488in}{5.195195in}}%
\pgfpathlineto{\pgfqpoint{1.284346in}{5.207057in}}%
\pgfpathlineto{\pgfqpoint{1.298061in}{5.217191in}}%
\pgfpathlineto{\pgfqpoint{1.318963in}{5.232527in}}%
\pgfpathlineto{\pgfqpoint{1.327634in}{5.238811in}}%
\pgfpathlineto{\pgfqpoint{1.354289in}{5.257996in}}%
\pgfpathlineto{\pgfqpoint{1.357208in}{5.260072in}}%
\pgfpathlineto{\pgfqpoint{1.386781in}{5.280939in}}%
\pgfpathlineto{\pgfqpoint{1.390388in}{5.283466in}}%
\pgfpathlineto{\pgfqpoint{1.416354in}{5.301431in}}%
\pgfpathlineto{\pgfqpoint{1.427272in}{5.308935in}}%
\pgfpathlineto{\pgfqpoint{1.445928in}{5.321602in}}%
\pgfpathlineto{\pgfqpoint{1.464904in}{5.334405in}}%
\pgfpathlineto{\pgfqpoint{1.475501in}{5.341467in}}%
\pgfpathlineto{\pgfqpoint{1.503293in}{5.359874in}}%
\pgfpathlineto{\pgfqpoint{1.505074in}{5.361039in}}%
\pgfpathlineto{\pgfqpoint{1.534648in}{5.380236in}}%
\pgfpathlineto{\pgfqpoint{1.542568in}{5.385344in}}%
\pgfpathlineto{\pgfqpoint{1.564221in}{5.399138in}}%
\pgfpathlineto{\pgfqpoint{1.582655in}{5.410813in}}%
\pgfpathlineto{\pgfqpoint{1.593795in}{5.417783in}}%
\pgfpathlineto{\pgfqpoint{1.623368in}{5.436180in}}%
\pgfpathlineto{\pgfqpoint{1.623534in}{5.436283in}}%
\pgfpathlineto{\pgfqpoint{1.652941in}{5.454204in}}%
\pgfpathlineto{\pgfqpoint{1.665396in}{5.461752in}}%
\pgfpathlineto{\pgfqpoint{1.682515in}{5.472001in}}%
\pgfpathlineto{\pgfqpoint{1.708073in}{5.487222in}}%
\pgfpathlineto{\pgfqpoint{1.712088in}{5.489584in}}%
\pgfpathlineto{\pgfqpoint{1.741661in}{5.506852in}}%
\pgfpathlineto{\pgfqpoint{1.751722in}{5.512691in}}%
\pgfpathlineto{\pgfqpoint{1.771235in}{5.523878in}}%
\pgfpathlineto{\pgfqpoint{1.796270in}{5.538161in}}%
\pgfpathlineto{\pgfqpoint{1.800808in}{5.540718in}}%
\pgfpathlineto{\pgfqpoint{1.830381in}{5.557261in}}%
\pgfpathlineto{\pgfqpoint{1.841833in}{5.563630in}}%
\pgfpathlineto{\pgfqpoint{1.859955in}{5.573586in}}%
\pgfpathlineto{\pgfqpoint{1.888320in}{5.589100in}}%
\pgfpathlineto{\pgfqpoint{1.889528in}{5.589753in}}%
\pgfpathlineto{\pgfqpoint{1.919102in}{5.605598in}}%
\pgfpathlineto{\pgfqpoint{1.935922in}{5.614570in}}%
\pgfpathlineto{\pgfqpoint{1.948675in}{5.621288in}}%
\pgfpathlineto{\pgfqpoint{1.978248in}{5.636783in}}%
\pgfpathlineto{\pgfqpoint{1.984508in}{5.640039in}}%
\pgfusepath{stroke}%
\end{pgfscope}%
\begin{pgfscope}%
\pgfpathrectangle{\pgfqpoint{0.854460in}{0.571603in}}{\pgfqpoint{5.885100in}{5.068436in}}%
\pgfusepath{clip}%
\pgfsetbuttcap%
\pgfsetroundjoin%
\pgfsetlinewidth{1.505625pt}%
\definecolor{currentstroke}{rgb}{0.146616,0.673050,0.508936}%
\pgfsetstrokecolor{currentstroke}%
\pgfsetdash{}{0pt}%
\pgfpathmoveto{\pgfqpoint{6.737069in}{5.640039in}}%
\pgfpathlineto{\pgfqpoint{6.709679in}{5.563630in}}%
\pgfpathlineto{\pgfqpoint{6.670588in}{5.461752in}}%
\pgfpathlineto{\pgfqpoint{6.619235in}{5.334405in}}%
\pgfpathlineto{\pgfqpoint{6.452348in}{4.926892in}}%
\pgfpathlineto{\pgfqpoint{6.403327in}{4.799545in}}%
\pgfpathlineto{\pgfqpoint{6.366140in}{4.697667in}}%
\pgfpathlineto{\pgfqpoint{6.325533in}{4.578852in}}%
\pgfpathlineto{\pgfqpoint{6.298482in}{4.493910in}}%
\pgfpathlineto{\pgfqpoint{6.268507in}{4.392032in}}%
\pgfpathlineto{\pgfqpoint{6.241394in}{4.290154in}}%
\pgfpathlineto{\pgfqpoint{6.217340in}{4.188276in}}%
\pgfpathlineto{\pgfqpoint{6.201432in}{4.111867in}}%
\pgfpathlineto{\pgfqpoint{6.187395in}{4.035459in}}%
\pgfpathlineto{\pgfqpoint{6.175332in}{3.959050in}}%
\pgfpathlineto{\pgfqpoint{6.165235in}{3.882642in}}%
\pgfpathlineto{\pgfqpoint{6.157219in}{3.806233in}}%
\pgfpathlineto{\pgfqpoint{6.151320in}{3.729825in}}%
\pgfpathlineto{\pgfqpoint{6.147566in}{3.653416in}}%
\pgfpathlineto{\pgfqpoint{6.145989in}{3.577007in}}%
\pgfpathlineto{\pgfqpoint{6.146651in}{3.500599in}}%
\pgfpathlineto{\pgfqpoint{6.149576in}{3.424190in}}%
\pgfpathlineto{\pgfqpoint{6.154786in}{3.347782in}}%
\pgfpathlineto{\pgfqpoint{6.162331in}{3.271373in}}%
\pgfpathlineto{\pgfqpoint{6.172253in}{3.194965in}}%
\pgfpathlineto{\pgfqpoint{6.184543in}{3.118556in}}%
\pgfpathlineto{\pgfqpoint{6.199240in}{3.042147in}}%
\pgfpathlineto{\pgfqpoint{6.216364in}{2.965739in}}%
\pgfpathlineto{\pgfqpoint{6.236813in}{2.886264in}}%
\pgfpathlineto{\pgfqpoint{6.257977in}{2.812922in}}%
\pgfpathlineto{\pgfqpoint{6.282490in}{2.736513in}}%
\pgfpathlineto{\pgfqpoint{6.309503in}{2.660104in}}%
\pgfpathlineto{\pgfqpoint{6.339024in}{2.583696in}}%
\pgfpathlineto{\pgfqpoint{6.371063in}{2.507287in}}%
\pgfpathlineto{\pgfqpoint{6.405636in}{2.430879in}}%
\pgfpathlineto{\pgfqpoint{6.443827in}{2.352375in}}%
\pgfpathlineto{\pgfqpoint{6.482390in}{2.278062in}}%
\pgfpathlineto{\pgfqpoint{6.524592in}{2.201653in}}%
\pgfpathlineto{\pgfqpoint{6.569350in}{2.125244in}}%
\pgfpathlineto{\pgfqpoint{6.621267in}{2.041685in}}%
\pgfpathlineto{\pgfqpoint{6.666536in}{1.972427in}}%
\pgfpathlineto{\pgfqpoint{6.718981in}{1.896019in}}%
\pgfpathlineto{\pgfqpoint{6.739560in}{1.867062in}}%
\pgfpathlineto{\pgfqpoint{6.739560in}{1.867062in}}%
\pgfusepath{stroke}%
\end{pgfscope}%
\begin{pgfscope}%
\pgfpathrectangle{\pgfqpoint{0.854460in}{0.571603in}}{\pgfqpoint{5.885100in}{5.068436in}}%
\pgfusepath{clip}%
\pgfsetbuttcap%
\pgfsetroundjoin%
\pgfsetlinewidth{1.505625pt}%
\definecolor{currentstroke}{rgb}{0.170948,0.694384,0.493803}%
\pgfsetstrokecolor{currentstroke}%
\pgfsetdash{}{0pt}%
\pgfpathmoveto{\pgfqpoint{0.884205in}{0.571603in}}%
\pgfpathlineto{\pgfqpoint{0.884034in}{0.571769in}}%
\pgfpathlineto{\pgfqpoint{0.858034in}{0.597073in}}%
\pgfpathlineto{\pgfqpoint{0.854460in}{0.600599in}}%
\pgfusepath{stroke}%
\end{pgfscope}%
\begin{pgfscope}%
\pgfpathrectangle{\pgfqpoint{0.854460in}{0.571603in}}{\pgfqpoint{5.885100in}{5.068436in}}%
\pgfusepath{clip}%
\pgfsetbuttcap%
\pgfsetroundjoin%
\pgfsetlinewidth{1.505625pt}%
\definecolor{currentstroke}{rgb}{0.170948,0.694384,0.493803}%
\pgfsetstrokecolor{currentstroke}%
\pgfsetdash{}{0pt}%
\pgfpathmoveto{\pgfqpoint{0.854460in}{4.899220in}}%
\pgfpathlineto{\pgfqpoint{0.856726in}{4.901423in}}%
\pgfpathlineto{\pgfqpoint{0.883257in}{4.926892in}}%
\pgfpathlineto{\pgfqpoint{0.884034in}{4.927628in}}%
\pgfpathlineto{\pgfqpoint{0.910365in}{4.952362in}}%
\pgfpathlineto{\pgfqpoint{0.913607in}{4.955369in}}%
\pgfpathlineto{\pgfqpoint{0.938030in}{4.977831in}}%
\pgfpathlineto{\pgfqpoint{0.943181in}{4.982510in}}%
\pgfpathlineto{\pgfqpoint{0.966264in}{5.003301in}}%
\pgfpathlineto{\pgfqpoint{0.972754in}{5.009075in}}%
\pgfpathlineto{\pgfqpoint{0.995080in}{5.028770in}}%
\pgfpathlineto{\pgfqpoint{1.002327in}{5.035086in}}%
\pgfpathlineto{\pgfqpoint{1.024490in}{5.054240in}}%
\pgfpathlineto{\pgfqpoint{1.031901in}{5.060566in}}%
\pgfpathlineto{\pgfqpoint{1.054507in}{5.079709in}}%
\pgfpathlineto{\pgfqpoint{1.061474in}{5.085537in}}%
\pgfpathlineto{\pgfqpoint{1.085143in}{5.105179in}}%
\pgfpathlineto{\pgfqpoint{1.091047in}{5.110019in}}%
\pgfpathlineto{\pgfqpoint{1.116410in}{5.130649in}}%
\pgfpathlineto{\pgfqpoint{1.120621in}{5.134032in}}%
\pgfpathlineto{\pgfqpoint{1.148319in}{5.156118in}}%
\pgfpathlineto{\pgfqpoint{1.150194in}{5.157595in}}%
\pgfpathlineto{\pgfqpoint{1.179767in}{5.180710in}}%
\pgfpathlineto{\pgfqpoint{1.180900in}{5.181588in}}%
\pgfpathlineto{\pgfqpoint{1.209341in}{5.203375in}}%
\pgfpathlineto{\pgfqpoint{1.214183in}{5.207057in}}%
\pgfpathlineto{\pgfqpoint{1.238914in}{5.225636in}}%
\pgfpathlineto{\pgfqpoint{1.248152in}{5.232527in}}%
\pgfpathlineto{\pgfqpoint{1.268488in}{5.247511in}}%
\pgfpathlineto{\pgfqpoint{1.282818in}{5.257996in}}%
\pgfpathlineto{\pgfqpoint{1.298061in}{5.269015in}}%
\pgfpathlineto{\pgfqpoint{1.318189in}{5.283466in}}%
\pgfpathlineto{\pgfqpoint{1.327634in}{5.290164in}}%
\pgfpathlineto{\pgfqpoint{1.354277in}{5.308935in}}%
\pgfpathlineto{\pgfqpoint{1.357208in}{5.310975in}}%
\pgfpathlineto{\pgfqpoint{1.386781in}{5.331405in}}%
\pgfpathlineto{\pgfqpoint{1.391155in}{5.334405in}}%
\pgfpathlineto{\pgfqpoint{1.416354in}{5.351479in}}%
\pgfpathlineto{\pgfqpoint{1.428823in}{5.359874in}}%
\pgfpathlineto{\pgfqpoint{1.445928in}{5.371252in}}%
\pgfpathlineto{\pgfqpoint{1.467243in}{5.385344in}}%
\pgfpathlineto{\pgfqpoint{1.475501in}{5.390737in}}%
\pgfpathlineto{\pgfqpoint{1.505074in}{5.409931in}}%
\pgfpathlineto{\pgfqpoint{1.506444in}{5.410813in}}%
\pgfpathlineto{\pgfqpoint{1.534648in}{5.428754in}}%
\pgfpathlineto{\pgfqpoint{1.546551in}{5.436283in}}%
\pgfpathlineto{\pgfqpoint{1.564221in}{5.447323in}}%
\pgfpathlineto{\pgfqpoint{1.587443in}{5.461752in}}%
\pgfpathlineto{\pgfqpoint{1.593795in}{5.465651in}}%
\pgfpathlineto{\pgfqpoint{1.623368in}{5.483681in}}%
\pgfpathlineto{\pgfqpoint{1.629215in}{5.487222in}}%
\pgfpathlineto{\pgfqpoint{1.652941in}{5.501414in}}%
\pgfpathlineto{\pgfqpoint{1.671893in}{5.512691in}}%
\pgfpathlineto{\pgfqpoint{1.682515in}{5.518935in}}%
\pgfpathlineto{\pgfqpoint{1.712088in}{5.536220in}}%
\pgfpathlineto{\pgfqpoint{1.715434in}{5.538161in}}%
\pgfpathlineto{\pgfqpoint{1.741661in}{5.553190in}}%
\pgfpathlineto{\pgfqpoint{1.759967in}{5.563630in}}%
\pgfpathlineto{\pgfqpoint{1.771235in}{5.569978in}}%
\pgfpathlineto{\pgfqpoint{1.800808in}{5.586546in}}%
\pgfpathlineto{\pgfqpoint{1.805401in}{5.589100in}}%
\pgfpathlineto{\pgfqpoint{1.830381in}{5.602824in}}%
\pgfpathlineto{\pgfqpoint{1.851854in}{5.614570in}}%
\pgfpathlineto{\pgfqpoint{1.859955in}{5.618946in}}%
\pgfpathlineto{\pgfqpoint{1.889528in}{5.634823in}}%
\pgfpathlineto{\pgfqpoint{1.899303in}{5.640039in}}%
\pgfusepath{stroke}%
\end{pgfscope}%
\begin{pgfscope}%
\pgfpathrectangle{\pgfqpoint{0.854460in}{0.571603in}}{\pgfqpoint{5.885100in}{5.068436in}}%
\pgfusepath{clip}%
\pgfsetbuttcap%
\pgfsetroundjoin%
\pgfsetlinewidth{1.505625pt}%
\definecolor{currentstroke}{rgb}{0.170948,0.694384,0.493803}%
\pgfsetstrokecolor{currentstroke}%
\pgfsetdash{}{0pt}%
\pgfpathmoveto{\pgfqpoint{6.739560in}{5.402128in}}%
\pgfpathlineto{\pgfqpoint{6.532547in}{4.921016in}}%
\pgfpathlineto{\pgfqpoint{6.483999in}{4.799545in}}%
\pgfpathlineto{\pgfqpoint{6.443827in}{4.693151in}}%
\pgfpathlineto{\pgfqpoint{6.409186in}{4.595788in}}%
\pgfpathlineto{\pgfqpoint{6.375428in}{4.493910in}}%
\pgfpathlineto{\pgfqpoint{6.344433in}{4.392032in}}%
\pgfpathlineto{\pgfqpoint{6.316403in}{4.290154in}}%
\pgfpathlineto{\pgfqpoint{6.295960in}{4.207315in}}%
\pgfpathlineto{\pgfqpoint{6.280296in}{4.137337in}}%
\pgfpathlineto{\pgfqpoint{6.265092in}{4.060928in}}%
\pgfpathlineto{\pgfqpoint{6.251802in}{3.984520in}}%
\pgfpathlineto{\pgfqpoint{6.240572in}{3.908111in}}%
\pgfpathlineto{\pgfqpoint{6.231391in}{3.831703in}}%
\pgfpathlineto{\pgfqpoint{6.224323in}{3.755294in}}%
\pgfpathlineto{\pgfqpoint{6.219429in}{3.678886in}}%
\pgfpathlineto{\pgfqpoint{6.216740in}{3.602477in}}%
\pgfpathlineto{\pgfqpoint{6.216291in}{3.526068in}}%
\pgfpathlineto{\pgfqpoint{6.218116in}{3.449660in}}%
\pgfpathlineto{\pgfqpoint{6.222251in}{3.373251in}}%
\pgfpathlineto{\pgfqpoint{6.228732in}{3.296843in}}%
\pgfpathlineto{\pgfqpoint{6.237594in}{3.220434in}}%
\pgfpathlineto{\pgfqpoint{6.248805in}{3.144025in}}%
\pgfpathlineto{\pgfqpoint{6.262470in}{3.067617in}}%
\pgfpathlineto{\pgfqpoint{6.278544in}{2.991208in}}%
\pgfpathlineto{\pgfqpoint{6.297112in}{2.914800in}}%
\pgfpathlineto{\pgfqpoint{6.318114in}{2.838391in}}%
\pgfpathlineto{\pgfqpoint{6.341618in}{2.761983in}}%
\pgfpathlineto{\pgfqpoint{6.367633in}{2.685574in}}%
\pgfpathlineto{\pgfqpoint{6.396167in}{2.609165in}}%
\pgfpathlineto{\pgfqpoint{6.427229in}{2.532757in}}%
\pgfpathlineto{\pgfqpoint{6.460831in}{2.456348in}}%
\pgfpathlineto{\pgfqpoint{6.502973in}{2.367892in}}%
\pgfpathlineto{\pgfqpoint{6.535705in}{2.303531in}}%
\pgfpathlineto{\pgfqpoint{6.576957in}{2.227123in}}%
\pgfpathlineto{\pgfqpoint{6.621267in}{2.149966in}}%
\pgfpathlineto{\pgfqpoint{6.667204in}{2.074305in}}%
\pgfpathlineto{\pgfqpoint{6.716197in}{1.997897in}}%
\pgfpathlineto{\pgfqpoint{6.739560in}{1.962865in}}%
\pgfpathlineto{\pgfqpoint{6.739560in}{1.962865in}}%
\pgfusepath{stroke}%
\end{pgfscope}%
\begin{pgfscope}%
\pgfpathrectangle{\pgfqpoint{0.854460in}{0.571603in}}{\pgfqpoint{5.885100in}{5.068436in}}%
\pgfusepath{clip}%
\pgfsetbuttcap%
\pgfsetroundjoin%
\pgfsetlinewidth{1.505625pt}%
\definecolor{currentstroke}{rgb}{0.196571,0.711827,0.479221}%
\pgfsetstrokecolor{currentstroke}%
\pgfsetdash{}{0pt}%
\pgfpathmoveto{\pgfqpoint{0.854460in}{4.958405in}}%
\pgfpathlineto{\pgfqpoint{0.875240in}{4.977831in}}%
\pgfpathlineto{\pgfqpoint{0.884034in}{4.985951in}}%
\pgfpathlineto{\pgfqpoint{0.902984in}{5.003301in}}%
\pgfpathlineto{\pgfqpoint{0.913607in}{5.012907in}}%
\pgfpathlineto{\pgfqpoint{0.931298in}{5.028770in}}%
\pgfpathlineto{\pgfqpoint{0.943181in}{5.039295in}}%
\pgfpathlineto{\pgfqpoint{0.960193in}{5.054240in}}%
\pgfpathlineto{\pgfqpoint{0.972754in}{5.065139in}}%
\pgfpathlineto{\pgfqpoint{0.989683in}{5.079709in}}%
\pgfpathlineto{\pgfqpoint{1.002327in}{5.090460in}}%
\pgfpathlineto{\pgfqpoint{1.019778in}{5.105179in}}%
\pgfpathlineto{\pgfqpoint{1.031901in}{5.115279in}}%
\pgfpathlineto{\pgfqpoint{1.050492in}{5.130649in}}%
\pgfpathlineto{\pgfqpoint{1.061474in}{5.139617in}}%
\pgfpathlineto{\pgfqpoint{1.081835in}{5.156118in}}%
\pgfpathlineto{\pgfqpoint{1.091047in}{5.163493in}}%
\pgfpathlineto{\pgfqpoint{1.113820in}{5.181588in}}%
\pgfpathlineto{\pgfqpoint{1.120621in}{5.186926in}}%
\pgfpathlineto{\pgfqpoint{1.146457in}{5.207057in}}%
\pgfpathlineto{\pgfqpoint{1.150194in}{5.209934in}}%
\pgfpathlineto{\pgfqpoint{1.179757in}{5.232527in}}%
\pgfpathlineto{\pgfqpoint{1.179767in}{5.232534in}}%
\pgfpathlineto{\pgfqpoint{1.209341in}{5.254683in}}%
\pgfpathlineto{\pgfqpoint{1.213796in}{5.257996in}}%
\pgfpathlineto{\pgfqpoint{1.238914in}{5.276450in}}%
\pgfpathlineto{\pgfqpoint{1.248529in}{5.283466in}}%
\pgfpathlineto{\pgfqpoint{1.268488in}{5.297852in}}%
\pgfpathlineto{\pgfqpoint{1.283966in}{5.308935in}}%
\pgfpathlineto{\pgfqpoint{1.298061in}{5.318905in}}%
\pgfpathlineto{\pgfqpoint{1.320116in}{5.334405in}}%
\pgfpathlineto{\pgfqpoint{1.327634in}{5.339625in}}%
\pgfpathlineto{\pgfqpoint{1.356987in}{5.359874in}}%
\pgfpathlineto{\pgfqpoint{1.357208in}{5.360024in}}%
\pgfpathlineto{\pgfqpoint{1.386781in}{5.380019in}}%
\pgfpathlineto{\pgfqpoint{1.394706in}{5.385344in}}%
\pgfpathlineto{\pgfqpoint{1.416354in}{5.399714in}}%
\pgfpathlineto{\pgfqpoint{1.433177in}{5.410813in}}%
\pgfpathlineto{\pgfqpoint{1.445928in}{5.419125in}}%
\pgfpathlineto{\pgfqpoint{1.472404in}{5.436283in}}%
\pgfpathlineto{\pgfqpoint{1.475501in}{5.438265in}}%
\pgfpathlineto{\pgfqpoint{1.505074in}{5.457060in}}%
\pgfpathlineto{\pgfqpoint{1.512505in}{5.461752in}}%
\pgfpathlineto{\pgfqpoint{1.534648in}{5.475566in}}%
\pgfpathlineto{\pgfqpoint{1.553433in}{5.487222in}}%
\pgfpathlineto{\pgfqpoint{1.564221in}{5.493834in}}%
\pgfpathlineto{\pgfqpoint{1.593795in}{5.511861in}}%
\pgfpathlineto{\pgfqpoint{1.595168in}{5.512691in}}%
\pgfpathlineto{\pgfqpoint{1.623368in}{5.529541in}}%
\pgfpathlineto{\pgfqpoint{1.637867in}{5.538161in}}%
\pgfpathlineto{\pgfqpoint{1.652941in}{5.547013in}}%
\pgfpathlineto{\pgfqpoint{1.681374in}{5.563630in}}%
\pgfpathlineto{\pgfqpoint{1.682515in}{5.564289in}}%
\pgfpathlineto{\pgfqpoint{1.712088in}{5.581229in}}%
\pgfpathlineto{\pgfqpoint{1.725897in}{5.589100in}}%
\pgfpathlineto{\pgfqpoint{1.741661in}{5.597976in}}%
\pgfpathlineto{\pgfqpoint{1.771235in}{5.614555in}}%
\pgfpathlineto{\pgfqpoint{1.771262in}{5.614570in}}%
\pgfpathlineto{\pgfqpoint{1.800808in}{5.630798in}}%
\pgfpathlineto{\pgfqpoint{1.817704in}{5.640039in}}%
\pgfusepath{stroke}%
\end{pgfscope}%
\begin{pgfscope}%
\pgfpathrectangle{\pgfqpoint{0.854460in}{0.571603in}}{\pgfqpoint{5.885100in}{5.068436in}}%
\pgfusepath{clip}%
\pgfsetbuttcap%
\pgfsetroundjoin%
\pgfsetlinewidth{1.505625pt}%
\definecolor{currentstroke}{rgb}{0.196571,0.711827,0.479221}%
\pgfsetstrokecolor{currentstroke}%
\pgfsetdash{}{0pt}%
\pgfpathmoveto{\pgfqpoint{6.739560in}{5.206321in}}%
\pgfpathlineto{\pgfqpoint{6.670888in}{5.054240in}}%
\pgfpathlineto{\pgfqpoint{6.615500in}{4.926892in}}%
\pgfpathlineto{\pgfqpoint{6.573075in}{4.825014in}}%
\pgfpathlineto{\pgfqpoint{6.532547in}{4.722658in}}%
\pgfpathlineto{\pgfqpoint{6.494655in}{4.621258in}}%
\pgfpathlineto{\pgfqpoint{6.459174in}{4.519380in}}%
\pgfpathlineto{\pgfqpoint{6.426496in}{4.417502in}}%
\pgfpathlineto{\pgfqpoint{6.396813in}{4.315624in}}%
\pgfpathlineto{\pgfqpoint{6.376631in}{4.239215in}}%
\pgfpathlineto{\pgfqpoint{6.355107in}{4.148425in}}%
\pgfpathlineto{\pgfqpoint{6.341880in}{4.086398in}}%
\pgfpathlineto{\pgfqpoint{6.325533in}{3.998682in}}%
\pgfpathlineto{\pgfqpoint{6.315043in}{3.933581in}}%
\pgfpathlineto{\pgfqpoint{6.304725in}{3.857172in}}%
\pgfpathlineto{\pgfqpoint{6.295960in}{3.773990in}}%
\pgfpathlineto{\pgfqpoint{6.290525in}{3.704355in}}%
\pgfpathlineto{\pgfqpoint{6.286716in}{3.627946in}}%
\pgfpathlineto{\pgfqpoint{6.285164in}{3.551538in}}%
\pgfpathlineto{\pgfqpoint{6.285901in}{3.475129in}}%
\pgfpathlineto{\pgfqpoint{6.288961in}{3.398721in}}%
\pgfpathlineto{\pgfqpoint{6.294380in}{3.322312in}}%
\pgfpathlineto{\pgfqpoint{6.302147in}{3.245904in}}%
\pgfpathlineto{\pgfqpoint{6.312320in}{3.169495in}}%
\pgfpathlineto{\pgfqpoint{6.325533in}{3.089980in}}%
\pgfpathlineto{\pgfqpoint{6.339980in}{3.016678in}}%
\pgfpathlineto{\pgfqpoint{6.357519in}{2.940269in}}%
\pgfpathlineto{\pgfqpoint{6.377510in}{2.863861in}}%
\pgfpathlineto{\pgfqpoint{6.400006in}{2.787452in}}%
\pgfpathlineto{\pgfqpoint{6.425027in}{2.711044in}}%
\pgfpathlineto{\pgfqpoint{6.452575in}{2.634635in}}%
\pgfpathlineto{\pgfqpoint{6.482660in}{2.558226in}}%
\pgfpathlineto{\pgfqpoint{6.515292in}{2.481818in}}%
\pgfpathlineto{\pgfqpoint{6.550487in}{2.405409in}}%
\pgfpathlineto{\pgfqpoint{6.591693in}{2.322376in}}%
\pgfpathlineto{\pgfqpoint{6.628590in}{2.252592in}}%
\pgfpathlineto{\pgfqpoint{6.671502in}{2.176183in}}%
\pgfpathlineto{\pgfqpoint{6.717003in}{2.099775in}}%
\pgfpathlineto{\pgfqpoint{6.739560in}{2.063494in}}%
\pgfpathlineto{\pgfqpoint{6.739560in}{2.063494in}}%
\pgfusepath{stroke}%
\end{pgfscope}%
\begin{pgfscope}%
\pgfpathrectangle{\pgfqpoint{0.854460in}{0.571603in}}{\pgfqpoint{5.885100in}{5.068436in}}%
\pgfusepath{clip}%
\pgfsetbuttcap%
\pgfsetroundjoin%
\pgfsetlinewidth{1.505625pt}%
\definecolor{currentstroke}{rgb}{0.232815,0.732247,0.459277}%
\pgfsetstrokecolor{currentstroke}%
\pgfsetdash{}{0pt}%
\pgfpathmoveto{\pgfqpoint{0.854460in}{5.015079in}}%
\pgfpathlineto{\pgfqpoint{0.869485in}{5.028770in}}%
\pgfpathlineto{\pgfqpoint{0.884034in}{5.041866in}}%
\pgfpathlineto{\pgfqpoint{0.897895in}{5.054240in}}%
\pgfpathlineto{\pgfqpoint{0.913607in}{5.068095in}}%
\pgfpathlineto{\pgfqpoint{0.926887in}{5.079709in}}%
\pgfpathlineto{\pgfqpoint{0.943181in}{5.093787in}}%
\pgfpathlineto{\pgfqpoint{0.956472in}{5.105179in}}%
\pgfpathlineto{\pgfqpoint{0.972754in}{5.118965in}}%
\pgfpathlineto{\pgfqpoint{0.986662in}{5.130649in}}%
\pgfpathlineto{\pgfqpoint{1.002327in}{5.143648in}}%
\pgfpathlineto{\pgfqpoint{1.017470in}{5.156118in}}%
\pgfpathlineto{\pgfqpoint{1.031901in}{5.167857in}}%
\pgfpathlineto{\pgfqpoint{1.048907in}{5.181588in}}%
\pgfpathlineto{\pgfqpoint{1.061474in}{5.191611in}}%
\pgfpathlineto{\pgfqpoint{1.080983in}{5.207057in}}%
\pgfpathlineto{\pgfqpoint{1.091047in}{5.214928in}}%
\pgfpathlineto{\pgfqpoint{1.113711in}{5.232527in}}%
\pgfpathlineto{\pgfqpoint{1.120621in}{5.237827in}}%
\pgfpathlineto{\pgfqpoint{1.147100in}{5.257996in}}%
\pgfpathlineto{\pgfqpoint{1.150194in}{5.260325in}}%
\pgfpathlineto{\pgfqpoint{1.179767in}{5.282418in}}%
\pgfpathlineto{\pgfqpoint{1.181181in}{5.283466in}}%
\pgfpathlineto{\pgfqpoint{1.209341in}{5.304092in}}%
\pgfpathlineto{\pgfqpoint{1.215998in}{5.308935in}}%
\pgfpathlineto{\pgfqpoint{1.238914in}{5.325406in}}%
\pgfpathlineto{\pgfqpoint{1.251517in}{5.334405in}}%
\pgfpathlineto{\pgfqpoint{1.268488in}{5.346376in}}%
\pgfpathlineto{\pgfqpoint{1.287746in}{5.359874in}}%
\pgfpathlineto{\pgfqpoint{1.298061in}{5.367017in}}%
\pgfpathlineto{\pgfqpoint{1.324694in}{5.385344in}}%
\pgfpathlineto{\pgfqpoint{1.327634in}{5.387343in}}%
\pgfpathlineto{\pgfqpoint{1.357208in}{5.407302in}}%
\pgfpathlineto{\pgfqpoint{1.362445in}{5.410813in}}%
\pgfpathlineto{\pgfqpoint{1.386781in}{5.426931in}}%
\pgfpathlineto{\pgfqpoint{1.400984in}{5.436283in}}%
\pgfpathlineto{\pgfqpoint{1.416354in}{5.446281in}}%
\pgfpathlineto{\pgfqpoint{1.440276in}{5.461752in}}%
\pgfpathlineto{\pgfqpoint{1.445928in}{5.465364in}}%
\pgfpathlineto{\pgfqpoint{1.475501in}{5.484135in}}%
\pgfpathlineto{\pgfqpoint{1.480398in}{5.487222in}}%
\pgfpathlineto{\pgfqpoint{1.505074in}{5.502590in}}%
\pgfpathlineto{\pgfqpoint{1.521379in}{5.512691in}}%
\pgfpathlineto{\pgfqpoint{1.534648in}{5.520812in}}%
\pgfpathlineto{\pgfqpoint{1.563143in}{5.538161in}}%
\pgfpathlineto{\pgfqpoint{1.564221in}{5.538810in}}%
\pgfpathlineto{\pgfqpoint{1.593795in}{5.556461in}}%
\pgfpathlineto{\pgfqpoint{1.605867in}{5.563630in}}%
\pgfpathlineto{\pgfqpoint{1.623368in}{5.573896in}}%
\pgfpathlineto{\pgfqpoint{1.649410in}{5.589100in}}%
\pgfpathlineto{\pgfqpoint{1.652941in}{5.591137in}}%
\pgfpathlineto{\pgfqpoint{1.682515in}{5.608071in}}%
\pgfpathlineto{\pgfqpoint{1.693923in}{5.614570in}}%
\pgfpathlineto{\pgfqpoint{1.712088in}{5.624790in}}%
\pgfpathlineto{\pgfqpoint{1.739307in}{5.640039in}}%
\pgfusepath{stroke}%
\end{pgfscope}%
\begin{pgfscope}%
\pgfpathrectangle{\pgfqpoint{0.854460in}{0.571603in}}{\pgfqpoint{5.885100in}{5.068436in}}%
\pgfusepath{clip}%
\pgfsetbuttcap%
\pgfsetroundjoin%
\pgfsetlinewidth{1.505625pt}%
\definecolor{currentstroke}{rgb}{0.232815,0.732247,0.459277}%
\pgfsetstrokecolor{currentstroke}%
\pgfsetdash{}{0pt}%
\pgfpathmoveto{\pgfqpoint{6.739560in}{5.028158in}}%
\pgfpathlineto{\pgfqpoint{6.683068in}{4.901423in}}%
\pgfpathlineto{\pgfqpoint{6.639800in}{4.799545in}}%
\pgfpathlineto{\pgfqpoint{6.598797in}{4.697667in}}%
\pgfpathlineto{\pgfqpoint{6.560301in}{4.595788in}}%
\pgfpathlineto{\pgfqpoint{6.524503in}{4.493910in}}%
\pgfpathlineto{\pgfqpoint{6.491651in}{4.392032in}}%
\pgfpathlineto{\pgfqpoint{6.469066in}{4.315624in}}%
\pgfpathlineto{\pgfqpoint{6.443827in}{4.221688in}}%
\pgfpathlineto{\pgfqpoint{6.429400in}{4.162807in}}%
\pgfpathlineto{\pgfqpoint{6.412495in}{4.086398in}}%
\pgfpathlineto{\pgfqpoint{6.397556in}{4.009989in}}%
\pgfpathlineto{\pgfqpoint{6.384680in}{3.933262in}}%
\pgfpathlineto{\pgfqpoint{6.373956in}{3.857172in}}%
\pgfpathlineto{\pgfqpoint{6.365367in}{3.780764in}}%
\pgfpathlineto{\pgfqpoint{6.358988in}{3.704355in}}%
\pgfpathlineto{\pgfqpoint{6.354844in}{3.627946in}}%
\pgfpathlineto{\pgfqpoint{6.352956in}{3.551538in}}%
\pgfpathlineto{\pgfqpoint{6.353382in}{3.475129in}}%
\pgfpathlineto{\pgfqpoint{6.356147in}{3.398721in}}%
\pgfpathlineto{\pgfqpoint{6.361263in}{3.322312in}}%
\pgfpathlineto{\pgfqpoint{6.368778in}{3.245904in}}%
\pgfpathlineto{\pgfqpoint{6.378729in}{3.169495in}}%
\pgfpathlineto{\pgfqpoint{6.391111in}{3.093086in}}%
\pgfpathlineto{\pgfqpoint{6.405953in}{3.016678in}}%
\pgfpathlineto{\pgfqpoint{6.423277in}{2.940269in}}%
\pgfpathlineto{\pgfqpoint{6.443827in}{2.861362in}}%
\pgfpathlineto{\pgfqpoint{6.465441in}{2.787452in}}%
\pgfpathlineto{\pgfqpoint{6.490302in}{2.711044in}}%
\pgfpathlineto{\pgfqpoint{6.517709in}{2.634635in}}%
\pgfpathlineto{\pgfqpoint{6.547669in}{2.558226in}}%
\pgfpathlineto{\pgfqpoint{6.580193in}{2.481818in}}%
\pgfpathlineto{\pgfqpoint{6.621267in}{2.393046in}}%
\pgfpathlineto{\pgfqpoint{6.652981in}{2.329001in}}%
\pgfpathlineto{\pgfqpoint{6.693223in}{2.252592in}}%
\pgfpathlineto{\pgfqpoint{6.739560in}{2.170249in}}%
\pgfpathlineto{\pgfqpoint{6.739560in}{2.170249in}}%
\pgfusepath{stroke}%
\end{pgfscope}%
\begin{pgfscope}%
\pgfpathrectangle{\pgfqpoint{0.854460in}{0.571603in}}{\pgfqpoint{5.885100in}{5.068436in}}%
\pgfusepath{clip}%
\pgfsetbuttcap%
\pgfsetroundjoin%
\pgfsetlinewidth{1.505625pt}%
\definecolor{currentstroke}{rgb}{0.266941,0.748751,0.440573}%
\pgfsetstrokecolor{currentstroke}%
\pgfsetdash{}{0pt}%
\pgfpathmoveto{\pgfqpoint{6.739560in}{4.855943in}}%
\pgfpathlineto{\pgfqpoint{6.737162in}{4.850484in}}%
\pgfpathlineto{\pgfqpoint{6.726088in}{4.825014in}}%
\pgfpathlineto{\pgfqpoint{6.715190in}{4.799545in}}%
\pgfpathlineto{\pgfqpoint{6.709987in}{4.787224in}}%
\pgfpathlineto{\pgfqpoint{6.704426in}{4.774075in}}%
\pgfpathlineto{\pgfqpoint{6.693803in}{4.748606in}}%
\pgfpathlineto{\pgfqpoint{6.683365in}{4.723136in}}%
\pgfpathlineto{\pgfqpoint{6.680414in}{4.715835in}}%
\pgfpathlineto{\pgfqpoint{6.673056in}{4.697667in}}%
\pgfpathlineto{\pgfqpoint{6.662914in}{4.672197in}}%
\pgfpathlineto{\pgfqpoint{6.652966in}{4.646728in}}%
\pgfpathlineto{\pgfqpoint{6.650840in}{4.641202in}}%
\pgfpathlineto{\pgfqpoint{6.643152in}{4.621258in}}%
\pgfpathlineto{\pgfqpoint{6.633520in}{4.595788in}}%
\pgfpathlineto{\pgfqpoint{6.624088in}{4.570319in}}%
\pgfpathlineto{\pgfqpoint{6.621267in}{4.562564in}}%
\pgfpathlineto{\pgfqpoint{6.614807in}{4.544849in}}%
\pgfpathlineto{\pgfqpoint{6.605710in}{4.519380in}}%
\pgfpathlineto{\pgfqpoint{6.596820in}{4.493910in}}%
\pgfpathlineto{\pgfqpoint{6.591693in}{4.478904in}}%
\pgfpathlineto{\pgfqpoint{6.588110in}{4.468441in}}%
\pgfpathlineto{\pgfqpoint{6.579572in}{4.442971in}}%
\pgfpathlineto{\pgfqpoint{6.571246in}{4.417502in}}%
\pgfpathlineto{\pgfqpoint{6.563136in}{4.392032in}}%
\pgfpathlineto{\pgfqpoint{6.562120in}{4.388770in}}%
\pgfpathlineto{\pgfqpoint{6.555187in}{4.366563in}}%
\pgfpathlineto{\pgfqpoint{6.547448in}{4.341093in}}%
\pgfpathlineto{\pgfqpoint{6.539930in}{4.315624in}}%
\pgfpathlineto{\pgfqpoint{6.532633in}{4.290154in}}%
\pgfpathlineto{\pgfqpoint{6.532547in}{4.289846in}}%
\pgfpathlineto{\pgfqpoint{6.525503in}{4.264685in}}%
\pgfpathlineto{\pgfqpoint{6.518598in}{4.239215in}}%
\pgfpathlineto{\pgfqpoint{6.511920in}{4.213746in}}%
\pgfpathlineto{\pgfqpoint{6.505468in}{4.188276in}}%
\pgfpathlineto{\pgfqpoint{6.502973in}{4.178092in}}%
\pgfpathlineto{\pgfqpoint{6.499216in}{4.162807in}}%
\pgfpathlineto{\pgfqpoint{6.493175in}{4.137337in}}%
\pgfpathlineto{\pgfqpoint{6.487367in}{4.111867in}}%
\pgfpathlineto{\pgfqpoint{6.481791in}{4.086398in}}%
\pgfpathlineto{\pgfqpoint{6.476450in}{4.060928in}}%
\pgfpathlineto{\pgfqpoint{6.473400in}{4.045741in}}%
\pgfpathlineto{\pgfqpoint{6.471327in}{4.035459in}}%
\pgfpathlineto{\pgfqpoint{6.466419in}{4.009989in}}%
\pgfpathlineto{\pgfqpoint{6.461749in}{3.984520in}}%
\pgfpathlineto{\pgfqpoint{6.457319in}{3.959050in}}%
\pgfpathlineto{\pgfqpoint{6.453129in}{3.933581in}}%
\pgfpathlineto{\pgfqpoint{6.449181in}{3.908111in}}%
\pgfpathlineto{\pgfqpoint{6.445474in}{3.882642in}}%
\pgfpathlineto{\pgfqpoint{6.443827in}{3.870541in}}%
\pgfpathlineto{\pgfqpoint{6.441998in}{3.857172in}}%
\pgfpathlineto{\pgfqpoint{6.438754in}{3.831703in}}%
\pgfpathlineto{\pgfqpoint{6.435758in}{3.806233in}}%
\pgfpathlineto{\pgfqpoint{6.433009in}{3.780764in}}%
\pgfpathlineto{\pgfqpoint{6.430509in}{3.755294in}}%
\pgfpathlineto{\pgfqpoint{6.428259in}{3.729825in}}%
\pgfpathlineto{\pgfqpoint{6.426259in}{3.704355in}}%
\pgfpathlineto{\pgfqpoint{6.424512in}{3.678886in}}%
\pgfpathlineto{\pgfqpoint{6.423016in}{3.653416in}}%
\pgfpathlineto{\pgfqpoint{6.421775in}{3.627946in}}%
\pgfpathlineto{\pgfqpoint{6.420788in}{3.602477in}}%
\pgfpathlineto{\pgfqpoint{6.420057in}{3.577007in}}%
\pgfpathlineto{\pgfqpoint{6.419582in}{3.551538in}}%
\pgfpathlineto{\pgfqpoint{6.419366in}{3.526068in}}%
\pgfpathlineto{\pgfqpoint{6.419409in}{3.500599in}}%
\pgfpathlineto{\pgfqpoint{6.419711in}{3.475129in}}%
\pgfpathlineto{\pgfqpoint{6.420275in}{3.449660in}}%
\pgfpathlineto{\pgfqpoint{6.421102in}{3.424190in}}%
\pgfpathlineto{\pgfqpoint{6.422192in}{3.398721in}}%
\pgfpathlineto{\pgfqpoint{6.423547in}{3.373251in}}%
\pgfpathlineto{\pgfqpoint{6.425168in}{3.347782in}}%
\pgfpathlineto{\pgfqpoint{6.427056in}{3.322312in}}%
\pgfpathlineto{\pgfqpoint{6.429213in}{3.296843in}}%
\pgfpathlineto{\pgfqpoint{6.431640in}{3.271373in}}%
\pgfpathlineto{\pgfqpoint{6.434339in}{3.245904in}}%
\pgfpathlineto{\pgfqpoint{6.437310in}{3.220434in}}%
\pgfpathlineto{\pgfqpoint{6.440556in}{3.194965in}}%
\pgfpathlineto{\pgfqpoint{6.443827in}{3.171305in}}%
\pgfpathlineto{\pgfqpoint{6.444075in}{3.169495in}}%
\pgfpathlineto{\pgfqpoint{6.447846in}{3.144025in}}%
\pgfpathlineto{\pgfqpoint{6.451894in}{3.118556in}}%
\pgfpathlineto{\pgfqpoint{6.456220in}{3.093086in}}%
\pgfpathlineto{\pgfqpoint{6.460826in}{3.067617in}}%
\pgfpathlineto{\pgfqpoint{6.465714in}{3.042147in}}%
\pgfpathlineto{\pgfqpoint{6.470884in}{3.016678in}}%
\pgfpathlineto{\pgfqpoint{6.473400in}{3.004930in}}%
\pgfpathlineto{\pgfqpoint{6.476319in}{2.991208in}}%
\pgfpathlineto{\pgfqpoint{6.482020in}{2.965739in}}%
\pgfpathlineto{\pgfqpoint{6.488008in}{2.940269in}}%
\pgfpathlineto{\pgfqpoint{6.494283in}{2.914800in}}%
\pgfpathlineto{\pgfqpoint{6.500849in}{2.889330in}}%
\pgfpathlineto{\pgfqpoint{6.502973in}{2.881436in}}%
\pgfpathlineto{\pgfqpoint{6.507673in}{2.863861in}}%
\pgfpathlineto{\pgfqpoint{6.514774in}{2.838391in}}%
\pgfpathlineto{\pgfqpoint{6.522169in}{2.812922in}}%
\pgfpathlineto{\pgfqpoint{6.529859in}{2.787452in}}%
\pgfpathlineto{\pgfqpoint{6.532547in}{2.778878in}}%
\pgfpathlineto{\pgfqpoint{6.537811in}{2.761983in}}%
\pgfpathlineto{\pgfqpoint{6.546041in}{2.736513in}}%
\pgfpathlineto{\pgfqpoint{6.554571in}{2.711044in}}%
\pgfpathlineto{\pgfqpoint{6.562120in}{2.689268in}}%
\pgfpathlineto{\pgfqpoint{6.563394in}{2.685574in}}%
\pgfpathlineto{\pgfqpoint{6.572466in}{2.660104in}}%
\pgfpathlineto{\pgfqpoint{6.581844in}{2.634635in}}%
\pgfpathlineto{\pgfqpoint{6.591528in}{2.609165in}}%
\pgfpathlineto{\pgfqpoint{6.591693in}{2.608742in}}%
\pgfpathlineto{\pgfqpoint{6.601452in}{2.583696in}}%
\pgfpathlineto{\pgfqpoint{6.611686in}{2.558226in}}%
\pgfpathlineto{\pgfqpoint{6.621267in}{2.535081in}}%
\pgfpathlineto{\pgfqpoint{6.622224in}{2.532757in}}%
\pgfpathlineto{\pgfqpoint{6.633008in}{2.507287in}}%
\pgfpathlineto{\pgfqpoint{6.644107in}{2.481818in}}%
\pgfpathlineto{\pgfqpoint{6.650840in}{2.466783in}}%
\pgfpathlineto{\pgfqpoint{6.655491in}{2.456348in}}%
\pgfpathlineto{\pgfqpoint{6.667146in}{2.430879in}}%
\pgfpathlineto{\pgfqpoint{6.679122in}{2.405409in}}%
\pgfpathlineto{\pgfqpoint{6.680414in}{2.402728in}}%
\pgfpathlineto{\pgfqpoint{6.691345in}{2.379940in}}%
\pgfpathlineto{\pgfqpoint{6.703883in}{2.354470in}}%
\pgfpathlineto{\pgfqpoint{6.709987in}{2.342369in}}%
\pgfpathlineto{\pgfqpoint{6.716701in}{2.329001in}}%
\pgfpathlineto{\pgfqpoint{6.729804in}{2.303531in}}%
\pgfpathlineto{\pgfqpoint{6.739560in}{2.285021in}}%
\pgfusepath{stroke}%
\end{pgfscope}%
\begin{pgfscope}%
\pgfpathrectangle{\pgfqpoint{0.854460in}{0.571603in}}{\pgfqpoint{5.885100in}{5.068436in}}%
\pgfusepath{clip}%
\pgfsetbuttcap%
\pgfsetroundjoin%
\pgfsetlinewidth{1.505625pt}%
\definecolor{currentstroke}{rgb}{0.266941,0.748751,0.440573}%
\pgfsetstrokecolor{currentstroke}%
\pgfsetdash{}{0pt}%
\pgfpathmoveto{\pgfqpoint{0.854460in}{5.069472in}}%
\pgfpathlineto{\pgfqpoint{0.865981in}{5.079709in}}%
\pgfpathlineto{\pgfqpoint{0.884034in}{5.095557in}}%
\pgfpathlineto{\pgfqpoint{0.895083in}{5.105179in}}%
\pgfpathlineto{\pgfqpoint{0.913607in}{5.121113in}}%
\pgfpathlineto{\pgfqpoint{0.924779in}{5.130649in}}%
\pgfpathlineto{\pgfqpoint{0.943181in}{5.146162in}}%
\pgfpathlineto{\pgfqpoint{0.955080in}{5.156118in}}%
\pgfpathlineto{\pgfqpoint{0.972754in}{5.170725in}}%
\pgfpathlineto{\pgfqpoint{0.985997in}{5.181588in}}%
\pgfpathlineto{\pgfqpoint{1.002327in}{5.194819in}}%
\pgfpathlineto{\pgfqpoint{1.017542in}{5.207057in}}%
\pgfpathlineto{\pgfqpoint{1.031901in}{5.218466in}}%
\pgfpathlineto{\pgfqpoint{1.049726in}{5.232527in}}%
\pgfpathlineto{\pgfqpoint{1.061474in}{5.241681in}}%
\pgfpathlineto{\pgfqpoint{1.082559in}{5.257996in}}%
\pgfpathlineto{\pgfqpoint{1.091047in}{5.264485in}}%
\pgfpathlineto{\pgfqpoint{1.116052in}{5.283466in}}%
\pgfpathlineto{\pgfqpoint{1.120621in}{5.286892in}}%
\pgfpathlineto{\pgfqpoint{1.150194in}{5.308920in}}%
\pgfpathlineto{\pgfqpoint{1.150214in}{5.308935in}}%
\pgfpathlineto{\pgfqpoint{1.179767in}{5.330513in}}%
\pgfpathlineto{\pgfqpoint{1.185132in}{5.334405in}}%
\pgfpathlineto{\pgfqpoint{1.209341in}{5.351752in}}%
\pgfpathlineto{\pgfqpoint{1.220749in}{5.359874in}}%
\pgfpathlineto{\pgfqpoint{1.238914in}{5.372651in}}%
\pgfpathlineto{\pgfqpoint{1.257073in}{5.385344in}}%
\pgfpathlineto{\pgfqpoint{1.268488in}{5.393226in}}%
\pgfpathlineto{\pgfqpoint{1.294112in}{5.410813in}}%
\pgfpathlineto{\pgfqpoint{1.298061in}{5.413491in}}%
\pgfpathlineto{\pgfqpoint{1.327634in}{5.433405in}}%
\pgfpathlineto{\pgfqpoint{1.331937in}{5.436283in}}%
\pgfpathlineto{\pgfqpoint{1.357208in}{5.452980in}}%
\pgfpathlineto{\pgfqpoint{1.370560in}{5.461752in}}%
\pgfpathlineto{\pgfqpoint{1.386781in}{5.472280in}}%
\pgfpathlineto{\pgfqpoint{1.409933in}{5.487222in}}%
\pgfpathlineto{\pgfqpoint{1.416354in}{5.491316in}}%
\pgfpathlineto{\pgfqpoint{1.445928in}{5.510054in}}%
\pgfpathlineto{\pgfqpoint{1.450120in}{5.512691in}}%
\pgfpathlineto{\pgfqpoint{1.475501in}{5.528470in}}%
\pgfpathlineto{\pgfqpoint{1.491171in}{5.538161in}}%
\pgfpathlineto{\pgfqpoint{1.505074in}{5.546655in}}%
\pgfpathlineto{\pgfqpoint{1.533000in}{5.563630in}}%
\pgfpathlineto{\pgfqpoint{1.534648in}{5.564620in}}%
\pgfpathlineto{\pgfqpoint{1.564221in}{5.582249in}}%
\pgfpathlineto{\pgfqpoint{1.575774in}{5.589100in}}%
\pgfpathlineto{\pgfqpoint{1.593795in}{5.599657in}}%
\pgfpathlineto{\pgfqpoint{1.619369in}{5.614570in}}%
\pgfpathlineto{\pgfqpoint{1.623368in}{5.616873in}}%
\pgfpathlineto{\pgfqpoint{1.652941in}{5.633792in}}%
\pgfpathlineto{\pgfqpoint{1.663920in}{5.640039in}}%
\pgfusepath{stroke}%
\end{pgfscope}%
\begin{pgfscope}%
\pgfpathrectangle{\pgfqpoint{0.854460in}{0.571603in}}{\pgfqpoint{5.885100in}{5.068436in}}%
\pgfusepath{clip}%
\pgfsetbuttcap%
\pgfsetroundjoin%
\pgfsetlinewidth{1.505625pt}%
\definecolor{currentstroke}{rgb}{0.311925,0.767822,0.415586}%
\pgfsetstrokecolor{currentstroke}%
\pgfsetdash{}{0pt}%
\pgfpathmoveto{\pgfqpoint{6.739560in}{4.682235in}}%
\pgfpathlineto{\pgfqpoint{6.735466in}{4.672197in}}%
\pgfpathlineto{\pgfqpoint{6.725239in}{4.646728in}}%
\pgfpathlineto{\pgfqpoint{6.715214in}{4.621258in}}%
\pgfpathlineto{\pgfqpoint{6.709987in}{4.607740in}}%
\pgfpathlineto{\pgfqpoint{6.705357in}{4.595788in}}%
\pgfpathlineto{\pgfqpoint{6.695666in}{4.570319in}}%
\pgfpathlineto{\pgfqpoint{6.686184in}{4.544849in}}%
\pgfpathlineto{\pgfqpoint{6.680414in}{4.529032in}}%
\pgfpathlineto{\pgfqpoint{6.676884in}{4.519380in}}%
\pgfpathlineto{\pgfqpoint{6.667753in}{4.493910in}}%
\pgfpathlineto{\pgfqpoint{6.658837in}{4.468441in}}%
\pgfpathlineto{\pgfqpoint{6.650840in}{4.445042in}}%
\pgfpathlineto{\pgfqpoint{6.650131in}{4.442971in}}%
\pgfpathlineto{\pgfqpoint{6.641581in}{4.417502in}}%
\pgfpathlineto{\pgfqpoint{6.633252in}{4.392032in}}%
\pgfpathlineto{\pgfqpoint{6.625144in}{4.366563in}}%
\pgfpathlineto{\pgfqpoint{6.621267in}{4.354071in}}%
\pgfpathlineto{\pgfqpoint{6.617227in}{4.341093in}}%
\pgfpathlineto{\pgfqpoint{6.609506in}{4.315624in}}%
\pgfpathlineto{\pgfqpoint{6.602010in}{4.290154in}}%
\pgfpathlineto{\pgfqpoint{6.594741in}{4.264685in}}%
\pgfpathlineto{\pgfqpoint{6.591693in}{4.253686in}}%
\pgfpathlineto{\pgfqpoint{6.587670in}{4.239215in}}%
\pgfpathlineto{\pgfqpoint{6.580807in}{4.213746in}}%
\pgfpathlineto{\pgfqpoint{6.574175in}{4.188276in}}%
\pgfpathlineto{\pgfqpoint{6.567775in}{4.162807in}}%
\pgfpathlineto{\pgfqpoint{6.562120in}{4.139456in}}%
\pgfpathlineto{\pgfqpoint{6.561605in}{4.137337in}}%
\pgfpathlineto{\pgfqpoint{6.555629in}{4.111867in}}%
\pgfpathlineto{\pgfqpoint{6.549889in}{4.086398in}}%
\pgfpathlineto{\pgfqpoint{6.544388in}{4.060928in}}%
\pgfpathlineto{\pgfqpoint{6.539125in}{4.035459in}}%
\pgfpathlineto{\pgfqpoint{6.534101in}{4.009989in}}%
\pgfpathlineto{\pgfqpoint{6.532547in}{4.001732in}}%
\pgfpathlineto{\pgfqpoint{6.529294in}{3.984520in}}%
\pgfpathlineto{\pgfqpoint{6.524718in}{3.959050in}}%
\pgfpathlineto{\pgfqpoint{6.520386in}{3.933581in}}%
\pgfpathlineto{\pgfqpoint{6.516299in}{3.908111in}}%
\pgfpathlineto{\pgfqpoint{6.512457in}{3.882642in}}%
\pgfpathlineto{\pgfqpoint{6.508862in}{3.857172in}}%
\pgfpathlineto{\pgfqpoint{6.505514in}{3.831703in}}%
\pgfpathlineto{\pgfqpoint{6.502973in}{3.810842in}}%
\pgfpathlineto{\pgfqpoint{6.502409in}{3.806233in}}%
\pgfpathlineto{\pgfqpoint{6.499537in}{3.780764in}}%
\pgfpathlineto{\pgfqpoint{6.496917in}{3.755294in}}%
\pgfpathlineto{\pgfqpoint{6.494550in}{3.729825in}}%
\pgfpathlineto{\pgfqpoint{6.492436in}{3.704355in}}%
\pgfpathlineto{\pgfqpoint{6.490576in}{3.678886in}}%
\pgfpathlineto{\pgfqpoint{6.488972in}{3.653416in}}%
\pgfpathlineto{\pgfqpoint{6.487624in}{3.627946in}}%
\pgfpathlineto{\pgfqpoint{6.486533in}{3.602477in}}%
\pgfpathlineto{\pgfqpoint{6.485701in}{3.577007in}}%
\pgfpathlineto{\pgfqpoint{6.485128in}{3.551538in}}%
\pgfpathlineto{\pgfqpoint{6.484816in}{3.526068in}}%
\pgfpathlineto{\pgfqpoint{6.484764in}{3.500599in}}%
\pgfpathlineto{\pgfqpoint{6.484976in}{3.475129in}}%
\pgfpathlineto{\pgfqpoint{6.485450in}{3.449660in}}%
\pgfpathlineto{\pgfqpoint{6.486190in}{3.424190in}}%
\pgfpathlineto{\pgfqpoint{6.487195in}{3.398721in}}%
\pgfpathlineto{\pgfqpoint{6.488468in}{3.373251in}}%
\pgfpathlineto{\pgfqpoint{6.490008in}{3.347782in}}%
\pgfpathlineto{\pgfqpoint{6.491818in}{3.322312in}}%
\pgfpathlineto{\pgfqpoint{6.493899in}{3.296843in}}%
\pgfpathlineto{\pgfqpoint{6.496252in}{3.271373in}}%
\pgfpathlineto{\pgfqpoint{6.498878in}{3.245904in}}%
\pgfpathlineto{\pgfqpoint{6.501779in}{3.220434in}}%
\pgfpathlineto{\pgfqpoint{6.502973in}{3.210859in}}%
\pgfpathlineto{\pgfqpoint{6.504942in}{3.194965in}}%
\pgfpathlineto{\pgfqpoint{6.508372in}{3.169495in}}%
\pgfpathlineto{\pgfqpoint{6.512079in}{3.144025in}}%
\pgfpathlineto{\pgfqpoint{6.516064in}{3.118556in}}%
\pgfpathlineto{\pgfqpoint{6.520329in}{3.093086in}}%
\pgfpathlineto{\pgfqpoint{6.524876in}{3.067617in}}%
\pgfpathlineto{\pgfqpoint{6.529706in}{3.042147in}}%
\pgfpathlineto{\pgfqpoint{6.532547in}{3.028000in}}%
\pgfpathlineto{\pgfqpoint{6.534805in}{3.016678in}}%
\pgfpathlineto{\pgfqpoint{6.540168in}{2.991208in}}%
\pgfpathlineto{\pgfqpoint{6.545818in}{2.965739in}}%
\pgfpathlineto{\pgfqpoint{6.551755in}{2.940269in}}%
\pgfpathlineto{\pgfqpoint{6.557982in}{2.914800in}}%
\pgfpathlineto{\pgfqpoint{6.562120in}{2.898626in}}%
\pgfpathlineto{\pgfqpoint{6.564484in}{2.889330in}}%
\pgfpathlineto{\pgfqpoint{6.571248in}{2.863861in}}%
\pgfpathlineto{\pgfqpoint{6.578306in}{2.838391in}}%
\pgfpathlineto{\pgfqpoint{6.585658in}{2.812922in}}%
\pgfpathlineto{\pgfqpoint{6.591693in}{2.792822in}}%
\pgfpathlineto{\pgfqpoint{6.593296in}{2.787452in}}%
\pgfpathlineto{\pgfqpoint{6.601190in}{2.761983in}}%
\pgfpathlineto{\pgfqpoint{6.609383in}{2.736513in}}%
\pgfpathlineto{\pgfqpoint{6.617877in}{2.711044in}}%
\pgfpathlineto{\pgfqpoint{6.621267in}{2.701221in}}%
\pgfpathlineto{\pgfqpoint{6.626637in}{2.685574in}}%
\pgfpathlineto{\pgfqpoint{6.635677in}{2.660104in}}%
\pgfpathlineto{\pgfqpoint{6.645023in}{2.634635in}}%
\pgfpathlineto{\pgfqpoint{6.650840in}{2.619278in}}%
\pgfpathlineto{\pgfqpoint{6.654650in}{2.609165in}}%
\pgfpathlineto{\pgfqpoint{6.664546in}{2.583696in}}%
\pgfpathlineto{\pgfqpoint{6.674753in}{2.558226in}}%
\pgfpathlineto{\pgfqpoint{6.680414in}{2.544508in}}%
\pgfpathlineto{\pgfqpoint{6.685239in}{2.532757in}}%
\pgfpathlineto{\pgfqpoint{6.695999in}{2.507287in}}%
\pgfpathlineto{\pgfqpoint{6.707076in}{2.481818in}}%
\pgfpathlineto{\pgfqpoint{6.709987in}{2.475300in}}%
\pgfpathlineto{\pgfqpoint{6.718413in}{2.456348in}}%
\pgfpathlineto{\pgfqpoint{6.730049in}{2.430879in}}%
\pgfpathlineto{\pgfqpoint{6.739560in}{2.410609in}}%
\pgfusepath{stroke}%
\end{pgfscope}%
\begin{pgfscope}%
\pgfpathrectangle{\pgfqpoint{0.854460in}{0.571603in}}{\pgfqpoint{5.885100in}{5.068436in}}%
\pgfusepath{clip}%
\pgfsetbuttcap%
\pgfsetroundjoin%
\pgfsetlinewidth{1.505625pt}%
\definecolor{currentstroke}{rgb}{0.311925,0.767822,0.415586}%
\pgfsetstrokecolor{currentstroke}%
\pgfsetdash{}{0pt}%
\pgfpathmoveto{\pgfqpoint{0.854460in}{5.121761in}}%
\pgfpathlineto{\pgfqpoint{0.864712in}{5.130649in}}%
\pgfpathlineto{\pgfqpoint{0.884034in}{5.147196in}}%
\pgfpathlineto{\pgfqpoint{0.894533in}{5.156118in}}%
\pgfpathlineto{\pgfqpoint{0.913607in}{5.172131in}}%
\pgfpathlineto{\pgfqpoint{0.924957in}{5.181588in}}%
\pgfpathlineto{\pgfqpoint{0.943181in}{5.196586in}}%
\pgfpathlineto{\pgfqpoint{0.955998in}{5.207057in}}%
\pgfpathlineto{\pgfqpoint{0.972754in}{5.220580in}}%
\pgfpathlineto{\pgfqpoint{0.987665in}{5.232527in}}%
\pgfpathlineto{\pgfqpoint{1.002327in}{5.244132in}}%
\pgfpathlineto{\pgfqpoint{1.019969in}{5.257996in}}%
\pgfpathlineto{\pgfqpoint{1.031901in}{5.267259in}}%
\pgfpathlineto{\pgfqpoint{1.052922in}{5.283466in}}%
\pgfpathlineto{\pgfqpoint{1.061474in}{5.289979in}}%
\pgfpathlineto{\pgfqpoint{1.086532in}{5.308935in}}%
\pgfpathlineto{\pgfqpoint{1.091047in}{5.312310in}}%
\pgfpathlineto{\pgfqpoint{1.120621in}{5.334263in}}%
\pgfpathlineto{\pgfqpoint{1.120813in}{5.334405in}}%
\pgfpathlineto{\pgfqpoint{1.150194in}{5.355788in}}%
\pgfpathlineto{\pgfqpoint{1.155845in}{5.359874in}}%
\pgfpathlineto{\pgfqpoint{1.179767in}{5.376963in}}%
\pgfpathlineto{\pgfqpoint{1.191573in}{5.385344in}}%
\pgfpathlineto{\pgfqpoint{1.209341in}{5.397804in}}%
\pgfpathlineto{\pgfqpoint{1.228006in}{5.410813in}}%
\pgfpathlineto{\pgfqpoint{1.238914in}{5.418324in}}%
\pgfpathlineto{\pgfqpoint{1.265151in}{5.436283in}}%
\pgfpathlineto{\pgfqpoint{1.268488in}{5.438539in}}%
\pgfpathlineto{\pgfqpoint{1.298061in}{5.458398in}}%
\pgfpathlineto{\pgfqpoint{1.303088in}{5.461752in}}%
\pgfpathlineto{\pgfqpoint{1.327634in}{5.477931in}}%
\pgfpathlineto{\pgfqpoint{1.341810in}{5.487222in}}%
\pgfpathlineto{\pgfqpoint{1.357208in}{5.497191in}}%
\pgfpathlineto{\pgfqpoint{1.381277in}{5.512691in}}%
\pgfpathlineto{\pgfqpoint{1.386781in}{5.516193in}}%
\pgfpathlineto{\pgfqpoint{1.416354in}{5.534886in}}%
\pgfpathlineto{\pgfqpoint{1.421568in}{5.538161in}}%
\pgfpathlineto{\pgfqpoint{1.445928in}{5.553274in}}%
\pgfpathlineto{\pgfqpoint{1.462705in}{5.563630in}}%
\pgfpathlineto{\pgfqpoint{1.475501in}{5.571434in}}%
\pgfpathlineto{\pgfqpoint{1.504613in}{5.589100in}}%
\pgfpathlineto{\pgfqpoint{1.505074in}{5.589376in}}%
\pgfpathlineto{\pgfqpoint{1.534648in}{5.606972in}}%
\pgfpathlineto{\pgfqpoint{1.547478in}{5.614570in}}%
\pgfpathlineto{\pgfqpoint{1.564221in}{5.624363in}}%
\pgfpathlineto{\pgfqpoint{1.591141in}{5.640039in}}%
\pgfusepath{stroke}%
\end{pgfscope}%
\begin{pgfscope}%
\pgfpathrectangle{\pgfqpoint{0.854460in}{0.571603in}}{\pgfqpoint{5.885100in}{5.068436in}}%
\pgfusepath{clip}%
\pgfsetbuttcap%
\pgfsetroundjoin%
\pgfsetlinewidth{1.505625pt}%
\definecolor{currentstroke}{rgb}{0.352360,0.783011,0.392636}%
\pgfsetstrokecolor{currentstroke}%
\pgfsetdash{}{0pt}%
\pgfpathmoveto{\pgfqpoint{6.739560in}{4.499731in}}%
\pgfpathlineto{\pgfqpoint{6.737430in}{4.493910in}}%
\pgfpathlineto{\pgfqpoint{6.728292in}{4.468441in}}%
\pgfpathlineto{\pgfqpoint{6.719376in}{4.442971in}}%
\pgfpathlineto{\pgfqpoint{6.710681in}{4.417502in}}%
\pgfpathlineto{\pgfqpoint{6.709987in}{4.415424in}}%
\pgfpathlineto{\pgfqpoint{6.702151in}{4.392032in}}%
\pgfpathlineto{\pgfqpoint{6.693842in}{4.366563in}}%
\pgfpathlineto{\pgfqpoint{6.685759in}{4.341093in}}%
\pgfpathlineto{\pgfqpoint{6.680414in}{4.323784in}}%
\pgfpathlineto{\pgfqpoint{6.677886in}{4.315624in}}%
\pgfpathlineto{\pgfqpoint{6.670203in}{4.290154in}}%
\pgfpathlineto{\pgfqpoint{6.662751in}{4.264685in}}%
\pgfpathlineto{\pgfqpoint{6.655532in}{4.239215in}}%
\pgfpathlineto{\pgfqpoint{6.650840in}{4.222133in}}%
\pgfpathlineto{\pgfqpoint{6.648529in}{4.213746in}}%
\pgfpathlineto{\pgfqpoint{6.641726in}{4.188276in}}%
\pgfpathlineto{\pgfqpoint{6.635161in}{4.162807in}}%
\pgfpathlineto{\pgfqpoint{6.628832in}{4.137337in}}%
\pgfpathlineto{\pgfqpoint{6.622741in}{4.111867in}}%
\pgfpathlineto{\pgfqpoint{6.621267in}{4.105466in}}%
\pgfpathlineto{\pgfqpoint{6.616858in}{4.086398in}}%
\pgfpathlineto{\pgfqpoint{6.611204in}{4.060928in}}%
\pgfpathlineto{\pgfqpoint{6.605794in}{4.035459in}}%
\pgfpathlineto{\pgfqpoint{6.600626in}{4.009989in}}%
\pgfpathlineto{\pgfqpoint{6.595701in}{3.984520in}}%
\pgfpathlineto{\pgfqpoint{6.591693in}{3.962717in}}%
\pgfpathlineto{\pgfqpoint{6.591017in}{3.959050in}}%
\pgfpathlineto{\pgfqpoint{6.586550in}{3.933581in}}%
\pgfpathlineto{\pgfqpoint{6.582331in}{3.908111in}}%
\pgfpathlineto{\pgfqpoint{6.578361in}{3.882642in}}%
\pgfpathlineto{\pgfqpoint{6.574640in}{3.857172in}}%
\pgfpathlineto{\pgfqpoint{6.571169in}{3.831703in}}%
\pgfpathlineto{\pgfqpoint{6.567950in}{3.806233in}}%
\pgfpathlineto{\pgfqpoint{6.564982in}{3.780764in}}%
\pgfpathlineto{\pgfqpoint{6.562267in}{3.755294in}}%
\pgfpathlineto{\pgfqpoint{6.562120in}{3.753778in}}%
\pgfpathlineto{\pgfqpoint{6.559789in}{3.729825in}}%
\pgfpathlineto{\pgfqpoint{6.557566in}{3.704355in}}%
\pgfpathlineto{\pgfqpoint{6.555599in}{3.678886in}}%
\pgfpathlineto{\pgfqpoint{6.553891in}{3.653416in}}%
\pgfpathlineto{\pgfqpoint{6.552442in}{3.627946in}}%
\pgfpathlineto{\pgfqpoint{6.551253in}{3.602477in}}%
\pgfpathlineto{\pgfqpoint{6.550324in}{3.577007in}}%
\pgfpathlineto{\pgfqpoint{6.549657in}{3.551538in}}%
\pgfpathlineto{\pgfqpoint{6.549252in}{3.526068in}}%
\pgfpathlineto{\pgfqpoint{6.549111in}{3.500599in}}%
\pgfpathlineto{\pgfqpoint{6.549235in}{3.475129in}}%
\pgfpathlineto{\pgfqpoint{6.549624in}{3.449660in}}%
\pgfpathlineto{\pgfqpoint{6.550281in}{3.424190in}}%
\pgfpathlineto{\pgfqpoint{6.551205in}{3.398721in}}%
\pgfpathlineto{\pgfqpoint{6.552398in}{3.373251in}}%
\pgfpathlineto{\pgfqpoint{6.553861in}{3.347782in}}%
\pgfpathlineto{\pgfqpoint{6.555596in}{3.322312in}}%
\pgfpathlineto{\pgfqpoint{6.557603in}{3.296843in}}%
\pgfpathlineto{\pgfqpoint{6.559884in}{3.271373in}}%
\pgfpathlineto{\pgfqpoint{6.562120in}{3.249098in}}%
\pgfpathlineto{\pgfqpoint{6.562438in}{3.245904in}}%
\pgfpathlineto{\pgfqpoint{6.565251in}{3.220434in}}%
\pgfpathlineto{\pgfqpoint{6.568340in}{3.194965in}}%
\pgfpathlineto{\pgfqpoint{6.571707in}{3.169495in}}%
\pgfpathlineto{\pgfqpoint{6.575351in}{3.144025in}}%
\pgfpathlineto{\pgfqpoint{6.579276in}{3.118556in}}%
\pgfpathlineto{\pgfqpoint{6.583483in}{3.093086in}}%
\pgfpathlineto{\pgfqpoint{6.587972in}{3.067617in}}%
\pgfpathlineto{\pgfqpoint{6.591693in}{3.047762in}}%
\pgfpathlineto{\pgfqpoint{6.592739in}{3.042147in}}%
\pgfpathlineto{\pgfqpoint{6.597764in}{3.016678in}}%
\pgfpathlineto{\pgfqpoint{6.603076in}{2.991208in}}%
\pgfpathlineto{\pgfqpoint{6.608675in}{2.965739in}}%
\pgfpathlineto{\pgfqpoint{6.614563in}{2.940269in}}%
\pgfpathlineto{\pgfqpoint{6.620742in}{2.914800in}}%
\pgfpathlineto{\pgfqpoint{6.621267in}{2.912731in}}%
\pgfpathlineto{\pgfqpoint{6.627173in}{2.889330in}}%
\pgfpathlineto{\pgfqpoint{6.633893in}{2.863861in}}%
\pgfpathlineto{\pgfqpoint{6.640907in}{2.838391in}}%
\pgfpathlineto{\pgfqpoint{6.648218in}{2.812922in}}%
\pgfpathlineto{\pgfqpoint{6.650840in}{2.804138in}}%
\pgfpathlineto{\pgfqpoint{6.655793in}{2.787452in}}%
\pgfpathlineto{\pgfqpoint{6.663649in}{2.761983in}}%
\pgfpathlineto{\pgfqpoint{6.671804in}{2.736513in}}%
\pgfpathlineto{\pgfqpoint{6.680262in}{2.711044in}}%
\pgfpathlineto{\pgfqpoint{6.680414in}{2.710603in}}%
\pgfpathlineto{\pgfqpoint{6.688966in}{2.685574in}}%
\pgfpathlineto{\pgfqpoint{6.697974in}{2.660104in}}%
\pgfpathlineto{\pgfqpoint{6.707288in}{2.634635in}}%
\pgfpathlineto{\pgfqpoint{6.709987in}{2.627483in}}%
\pgfpathlineto{\pgfqpoint{6.716864in}{2.609165in}}%
\pgfpathlineto{\pgfqpoint{6.726732in}{2.583696in}}%
\pgfpathlineto{\pgfqpoint{6.736911in}{2.558226in}}%
\pgfpathlineto{\pgfqpoint{6.739560in}{2.551786in}}%
\pgfusepath{stroke}%
\end{pgfscope}%
\begin{pgfscope}%
\pgfpathrectangle{\pgfqpoint{0.854460in}{0.571603in}}{\pgfqpoint{5.885100in}{5.068436in}}%
\pgfusepath{clip}%
\pgfsetbuttcap%
\pgfsetroundjoin%
\pgfsetlinewidth{1.505625pt}%
\definecolor{currentstroke}{rgb}{0.352360,0.783011,0.392636}%
\pgfsetstrokecolor{currentstroke}%
\pgfsetdash{}{0pt}%
\pgfpathmoveto{\pgfqpoint{0.854460in}{5.172109in}}%
\pgfpathlineto{\pgfqpoint{0.865663in}{5.181588in}}%
\pgfpathlineto{\pgfqpoint{0.884034in}{5.196943in}}%
\pgfpathlineto{\pgfqpoint{0.896224in}{5.207057in}}%
\pgfpathlineto{\pgfqpoint{0.913607in}{5.221304in}}%
\pgfpathlineto{\pgfqpoint{0.927400in}{5.232527in}}%
\pgfpathlineto{\pgfqpoint{0.943181in}{5.245210in}}%
\pgfpathlineto{\pgfqpoint{0.959202in}{5.257996in}}%
\pgfpathlineto{\pgfqpoint{0.972754in}{5.268680in}}%
\pgfpathlineto{\pgfqpoint{0.991640in}{5.283466in}}%
\pgfpathlineto{\pgfqpoint{1.002327in}{5.291731in}}%
\pgfpathlineto{\pgfqpoint{1.024724in}{5.308935in}}%
\pgfpathlineto{\pgfqpoint{1.031901in}{5.314381in}}%
\pgfpathlineto{\pgfqpoint{1.058465in}{5.334405in}}%
\pgfpathlineto{\pgfqpoint{1.061474in}{5.336646in}}%
\pgfpathlineto{\pgfqpoint{1.091047in}{5.358516in}}%
\pgfpathlineto{\pgfqpoint{1.092897in}{5.359874in}}%
\pgfpathlineto{\pgfqpoint{1.120621in}{5.379984in}}%
\pgfpathlineto{\pgfqpoint{1.128056in}{5.385344in}}%
\pgfpathlineto{\pgfqpoint{1.150194in}{5.401108in}}%
\pgfpathlineto{\pgfqpoint{1.163909in}{5.410813in}}%
\pgfpathlineto{\pgfqpoint{1.179767in}{5.421901in}}%
\pgfpathlineto{\pgfqpoint{1.200463in}{5.436283in}}%
\pgfpathlineto{\pgfqpoint{1.209341in}{5.442378in}}%
\pgfpathlineto{\pgfqpoint{1.237727in}{5.461752in}}%
\pgfpathlineto{\pgfqpoint{1.238914in}{5.462553in}}%
\pgfpathlineto{\pgfqpoint{1.268488in}{5.482349in}}%
\pgfpathlineto{\pgfqpoint{1.275810in}{5.487222in}}%
\pgfpathlineto{\pgfqpoint{1.298061in}{5.501850in}}%
\pgfpathlineto{\pgfqpoint{1.314644in}{5.512691in}}%
\pgfpathlineto{\pgfqpoint{1.327634in}{5.521082in}}%
\pgfpathlineto{\pgfqpoint{1.354219in}{5.538161in}}%
\pgfpathlineto{\pgfqpoint{1.357208in}{5.540058in}}%
\pgfpathlineto{\pgfqpoint{1.386781in}{5.558699in}}%
\pgfpathlineto{\pgfqpoint{1.394649in}{5.563630in}}%
\pgfpathlineto{\pgfqpoint{1.416354in}{5.577069in}}%
\pgfpathlineto{\pgfqpoint{1.435884in}{5.589100in}}%
\pgfpathlineto{\pgfqpoint{1.445928in}{5.595212in}}%
\pgfpathlineto{\pgfqpoint{1.475501in}{5.613115in}}%
\pgfpathlineto{\pgfqpoint{1.477920in}{5.614570in}}%
\pgfpathlineto{\pgfqpoint{1.505074in}{5.630696in}}%
\pgfpathlineto{\pgfqpoint{1.520878in}{5.640039in}}%
\pgfusepath{stroke}%
\end{pgfscope}%
\begin{pgfscope}%
\pgfpathrectangle{\pgfqpoint{0.854460in}{0.571603in}}{\pgfqpoint{5.885100in}{5.068436in}}%
\pgfusepath{clip}%
\pgfsetbuttcap%
\pgfsetroundjoin%
\pgfsetlinewidth{1.505625pt}%
\definecolor{currentstroke}{rgb}{0.404001,0.800275,0.362552}%
\pgfsetstrokecolor{currentstroke}%
\pgfsetdash{}{0pt}%
\pgfpathmoveto{\pgfqpoint{6.739560in}{4.297611in}}%
\pgfpathlineto{\pgfqpoint{6.737266in}{4.290154in}}%
\pgfpathlineto{\pgfqpoint{6.729642in}{4.264685in}}%
\pgfpathlineto{\pgfqpoint{6.722254in}{4.239215in}}%
\pgfpathlineto{\pgfqpoint{6.715102in}{4.213746in}}%
\pgfpathlineto{\pgfqpoint{6.709987in}{4.194919in}}%
\pgfpathlineto{\pgfqpoint{6.708176in}{4.188276in}}%
\pgfpathlineto{\pgfqpoint{6.701453in}{4.162807in}}%
\pgfpathlineto{\pgfqpoint{6.694970in}{4.137337in}}%
\pgfpathlineto{\pgfqpoint{6.688730in}{4.111867in}}%
\pgfpathlineto{\pgfqpoint{6.682731in}{4.086398in}}%
\pgfpathlineto{\pgfqpoint{6.680414in}{4.076163in}}%
\pgfpathlineto{\pgfqpoint{6.676951in}{4.060928in}}%
\pgfpathlineto{\pgfqpoint{6.671400in}{4.035459in}}%
\pgfpathlineto{\pgfqpoint{6.666095in}{4.009989in}}%
\pgfpathlineto{\pgfqpoint{6.661037in}{3.984520in}}%
\pgfpathlineto{\pgfqpoint{6.656226in}{3.959050in}}%
\pgfpathlineto{\pgfqpoint{6.651664in}{3.933581in}}%
\pgfpathlineto{\pgfqpoint{6.650840in}{3.928726in}}%
\pgfpathlineto{\pgfqpoint{6.647326in}{3.908111in}}%
\pgfpathlineto{\pgfqpoint{6.643234in}{3.882642in}}%
\pgfpathlineto{\pgfqpoint{6.639394in}{3.857172in}}%
\pgfpathlineto{\pgfqpoint{6.635807in}{3.831703in}}%
\pgfpathlineto{\pgfqpoint{6.632474in}{3.806233in}}%
\pgfpathlineto{\pgfqpoint{6.629395in}{3.780764in}}%
\pgfpathlineto{\pgfqpoint{6.626571in}{3.755294in}}%
\pgfpathlineto{\pgfqpoint{6.624003in}{3.729825in}}%
\pgfpathlineto{\pgfqpoint{6.621693in}{3.704355in}}%
\pgfpathlineto{\pgfqpoint{6.621267in}{3.699081in}}%
\pgfpathlineto{\pgfqpoint{6.619628in}{3.678886in}}%
\pgfpathlineto{\pgfqpoint{6.617821in}{3.653416in}}%
\pgfpathlineto{\pgfqpoint{6.616275in}{3.627946in}}%
\pgfpathlineto{\pgfqpoint{6.614991in}{3.602477in}}%
\pgfpathlineto{\pgfqpoint{6.613970in}{3.577007in}}%
\pgfpathlineto{\pgfqpoint{6.613212in}{3.551538in}}%
\pgfpathlineto{\pgfqpoint{6.612720in}{3.526068in}}%
\pgfpathlineto{\pgfqpoint{6.612493in}{3.500599in}}%
\pgfpathlineto{\pgfqpoint{6.612533in}{3.475129in}}%
\pgfpathlineto{\pgfqpoint{6.612840in}{3.449660in}}%
\pgfpathlineto{\pgfqpoint{6.613417in}{3.424190in}}%
\pgfpathlineto{\pgfqpoint{6.614263in}{3.398721in}}%
\pgfpathlineto{\pgfqpoint{6.615380in}{3.373251in}}%
\pgfpathlineto{\pgfqpoint{6.616769in}{3.347782in}}%
\pgfpathlineto{\pgfqpoint{6.618431in}{3.322312in}}%
\pgfpathlineto{\pgfqpoint{6.620368in}{3.296843in}}%
\pgfpathlineto{\pgfqpoint{6.621267in}{3.286497in}}%
\pgfpathlineto{\pgfqpoint{6.622571in}{3.271373in}}%
\pgfpathlineto{\pgfqpoint{6.625043in}{3.245904in}}%
\pgfpathlineto{\pgfqpoint{6.627791in}{3.220434in}}%
\pgfpathlineto{\pgfqpoint{6.630817in}{3.194965in}}%
\pgfpathlineto{\pgfqpoint{6.634121in}{3.169495in}}%
\pgfpathlineto{\pgfqpoint{6.637706in}{3.144025in}}%
\pgfpathlineto{\pgfqpoint{6.641572in}{3.118556in}}%
\pgfpathlineto{\pgfqpoint{6.645721in}{3.093086in}}%
\pgfpathlineto{\pgfqpoint{6.650155in}{3.067617in}}%
\pgfpathlineto{\pgfqpoint{6.650840in}{3.063917in}}%
\pgfpathlineto{\pgfqpoint{6.654847in}{3.042147in}}%
\pgfpathlineto{\pgfqpoint{6.659820in}{3.016678in}}%
\pgfpathlineto{\pgfqpoint{6.665081in}{2.991208in}}%
\pgfpathlineto{\pgfqpoint{6.670630in}{2.965739in}}%
\pgfpathlineto{\pgfqpoint{6.676470in}{2.940269in}}%
\pgfpathlineto{\pgfqpoint{6.680414in}{2.923887in}}%
\pgfpathlineto{\pgfqpoint{6.682587in}{2.914800in}}%
\pgfpathlineto{\pgfqpoint{6.688970in}{2.889330in}}%
\pgfpathlineto{\pgfqpoint{6.695647in}{2.863861in}}%
\pgfpathlineto{\pgfqpoint{6.702619in}{2.838391in}}%
\pgfpathlineto{\pgfqpoint{6.709888in}{2.812922in}}%
\pgfpathlineto{\pgfqpoint{6.709987in}{2.812589in}}%
\pgfpathlineto{\pgfqpoint{6.717407in}{2.787452in}}%
\pgfpathlineto{\pgfqpoint{6.725225in}{2.761983in}}%
\pgfpathlineto{\pgfqpoint{6.733343in}{2.736513in}}%
\pgfpathlineto{\pgfqpoint{6.739560in}{2.717706in}}%
\pgfusepath{stroke}%
\end{pgfscope}%
\begin{pgfscope}%
\pgfpathrectangle{\pgfqpoint{0.854460in}{0.571603in}}{\pgfqpoint{5.885100in}{5.068436in}}%
\pgfusepath{clip}%
\pgfsetbuttcap%
\pgfsetroundjoin%
\pgfsetlinewidth{1.505625pt}%
\definecolor{currentstroke}{rgb}{0.404001,0.800275,0.362552}%
\pgfsetstrokecolor{currentstroke}%
\pgfsetdash{}{0pt}%
\pgfpathmoveto{\pgfqpoint{0.854460in}{5.220668in}}%
\pgfpathlineto{\pgfqpoint{0.868815in}{5.232527in}}%
\pgfpathlineto{\pgfqpoint{0.884034in}{5.244947in}}%
\pgfpathlineto{\pgfqpoint{0.900138in}{5.257996in}}%
\pgfpathlineto{\pgfqpoint{0.913607in}{5.268778in}}%
\pgfpathlineto{\pgfqpoint{0.932086in}{5.283466in}}%
\pgfpathlineto{\pgfqpoint{0.943181in}{5.292178in}}%
\pgfpathlineto{\pgfqpoint{0.964669in}{5.308935in}}%
\pgfpathlineto{\pgfqpoint{0.972754in}{5.315164in}}%
\pgfpathlineto{\pgfqpoint{0.997896in}{5.334405in}}%
\pgfpathlineto{\pgfqpoint{1.002327in}{5.337755in}}%
\pgfpathlineto{\pgfqpoint{1.031779in}{5.359874in}}%
\pgfpathlineto{\pgfqpoint{1.031901in}{5.359965in}}%
\pgfpathlineto{\pgfqpoint{1.061474in}{5.381745in}}%
\pgfpathlineto{\pgfqpoint{1.066392in}{5.385344in}}%
\pgfpathlineto{\pgfqpoint{1.091047in}{5.403167in}}%
\pgfpathlineto{\pgfqpoint{1.101690in}{5.410813in}}%
\pgfpathlineto{\pgfqpoint{1.120621in}{5.424250in}}%
\pgfpathlineto{\pgfqpoint{1.137678in}{5.436283in}}%
\pgfpathlineto{\pgfqpoint{1.150194in}{5.445006in}}%
\pgfpathlineto{\pgfqpoint{1.174366in}{5.461752in}}%
\pgfpathlineto{\pgfqpoint{1.179767in}{5.465450in}}%
\pgfpathlineto{\pgfqpoint{1.209341in}{5.485565in}}%
\pgfpathlineto{\pgfqpoint{1.211794in}{5.487222in}}%
\pgfpathlineto{\pgfqpoint{1.238914in}{5.505321in}}%
\pgfpathlineto{\pgfqpoint{1.250021in}{5.512691in}}%
\pgfpathlineto{\pgfqpoint{1.268488in}{5.524799in}}%
\pgfpathlineto{\pgfqpoint{1.288978in}{5.538161in}}%
\pgfpathlineto{\pgfqpoint{1.298061in}{5.544012in}}%
\pgfpathlineto{\pgfqpoint{1.327634in}{5.562961in}}%
\pgfpathlineto{\pgfqpoint{1.328687in}{5.563630in}}%
\pgfpathlineto{\pgfqpoint{1.357208in}{5.581554in}}%
\pgfpathlineto{\pgfqpoint{1.369275in}{5.589100in}}%
\pgfpathlineto{\pgfqpoint{1.386781in}{5.599914in}}%
\pgfpathlineto{\pgfqpoint{1.410620in}{5.614570in}}%
\pgfpathlineto{\pgfqpoint{1.416354in}{5.618052in}}%
\pgfpathlineto{\pgfqpoint{1.445928in}{5.635902in}}%
\pgfpathlineto{\pgfqpoint{1.452822in}{5.640039in}}%
\pgfusepath{stroke}%
\end{pgfscope}%
\begin{pgfscope}%
\pgfpathrectangle{\pgfqpoint{0.854460in}{0.571603in}}{\pgfqpoint{5.885100in}{5.068436in}}%
\pgfusepath{clip}%
\pgfsetbuttcap%
\pgfsetroundjoin%
\pgfsetlinewidth{1.505625pt}%
\definecolor{currentstroke}{rgb}{0.449368,0.813768,0.335384}%
\pgfsetstrokecolor{currentstroke}%
\pgfsetdash{}{0pt}%
\pgfpathmoveto{\pgfqpoint{6.739560in}{4.051473in}}%
\pgfpathlineto{\pgfqpoint{6.735991in}{4.035459in}}%
\pgfpathlineto{\pgfqpoint{6.730556in}{4.009989in}}%
\pgfpathlineto{\pgfqpoint{6.725371in}{3.984520in}}%
\pgfpathlineto{\pgfqpoint{6.720436in}{3.959050in}}%
\pgfpathlineto{\pgfqpoint{6.715753in}{3.933581in}}%
\pgfpathlineto{\pgfqpoint{6.711321in}{3.908111in}}%
\pgfpathlineto{\pgfqpoint{6.709987in}{3.899997in}}%
\pgfpathlineto{\pgfqpoint{6.707122in}{3.882642in}}%
\pgfpathlineto{\pgfqpoint{6.703168in}{3.857172in}}%
\pgfpathlineto{\pgfqpoint{6.699470in}{3.831703in}}%
\pgfpathlineto{\pgfqpoint{6.696029in}{3.806233in}}%
\pgfpathlineto{\pgfqpoint{6.692844in}{3.780764in}}%
\pgfpathlineto{\pgfqpoint{6.689917in}{3.755294in}}%
\pgfpathlineto{\pgfqpoint{6.687249in}{3.729825in}}%
\pgfpathlineto{\pgfqpoint{6.684839in}{3.704355in}}%
\pgfpathlineto{\pgfqpoint{6.682690in}{3.678886in}}%
\pgfpathlineto{\pgfqpoint{6.680801in}{3.653416in}}%
\pgfpathlineto{\pgfqpoint{6.680414in}{3.647365in}}%
\pgfpathlineto{\pgfqpoint{6.679165in}{3.627946in}}%
\pgfpathlineto{\pgfqpoint{6.677790in}{3.602477in}}%
\pgfpathlineto{\pgfqpoint{6.676681in}{3.577007in}}%
\pgfpathlineto{\pgfqpoint{6.675838in}{3.551538in}}%
\pgfpathlineto{\pgfqpoint{6.675261in}{3.526068in}}%
\pgfpathlineto{\pgfqpoint{6.674952in}{3.500599in}}%
\pgfpathlineto{\pgfqpoint{6.674911in}{3.475129in}}%
\pgfpathlineto{\pgfqpoint{6.675140in}{3.449660in}}%
\pgfpathlineto{\pgfqpoint{6.675640in}{3.424190in}}%
\pgfpathlineto{\pgfqpoint{6.676411in}{3.398721in}}%
\pgfpathlineto{\pgfqpoint{6.677455in}{3.373251in}}%
\pgfpathlineto{\pgfqpoint{6.678773in}{3.347782in}}%
\pgfpathlineto{\pgfqpoint{6.680365in}{3.322312in}}%
\pgfpathlineto{\pgfqpoint{6.680414in}{3.321654in}}%
\pgfpathlineto{\pgfqpoint{6.682221in}{3.296843in}}%
\pgfpathlineto{\pgfqpoint{6.684353in}{3.271373in}}%
\pgfpathlineto{\pgfqpoint{6.686760in}{3.245904in}}%
\pgfpathlineto{\pgfqpoint{6.689446in}{3.220434in}}%
\pgfpathlineto{\pgfqpoint{6.692410in}{3.194965in}}%
\pgfpathlineto{\pgfqpoint{6.695655in}{3.169495in}}%
\pgfpathlineto{\pgfqpoint{6.699181in}{3.144025in}}%
\pgfpathlineto{\pgfqpoint{6.702990in}{3.118556in}}%
\pgfpathlineto{\pgfqpoint{6.707083in}{3.093086in}}%
\pgfpathlineto{\pgfqpoint{6.709987in}{3.076200in}}%
\pgfpathlineto{\pgfqpoint{6.711453in}{3.067617in}}%
\pgfpathlineto{\pgfqpoint{6.716089in}{3.042147in}}%
\pgfpathlineto{\pgfqpoint{6.721011in}{3.016678in}}%
\pgfpathlineto{\pgfqpoint{6.726222in}{2.991208in}}%
\pgfpathlineto{\pgfqpoint{6.731723in}{2.965739in}}%
\pgfpathlineto{\pgfqpoint{6.737516in}{2.940269in}}%
\pgfpathlineto{\pgfqpoint{6.739560in}{2.931709in}}%
\pgfusepath{stroke}%
\end{pgfscope}%
\begin{pgfscope}%
\pgfpathrectangle{\pgfqpoint{0.854460in}{0.571603in}}{\pgfqpoint{5.885100in}{5.068436in}}%
\pgfusepath{clip}%
\pgfsetbuttcap%
\pgfsetroundjoin%
\pgfsetlinewidth{1.505625pt}%
\definecolor{currentstroke}{rgb}{0.449368,0.813768,0.335384}%
\pgfsetstrokecolor{currentstroke}%
\pgfsetdash{}{0pt}%
\pgfpathmoveto{\pgfqpoint{0.854460in}{5.267580in}}%
\pgfpathlineto{\pgfqpoint{0.874148in}{5.283466in}}%
\pgfpathlineto{\pgfqpoint{0.884034in}{5.291346in}}%
\pgfpathlineto{\pgfqpoint{0.906252in}{5.308935in}}%
\pgfpathlineto{\pgfqpoint{0.913607in}{5.314687in}}%
\pgfpathlineto{\pgfqpoint{0.938990in}{5.334405in}}%
\pgfpathlineto{\pgfqpoint{0.943181in}{5.337620in}}%
\pgfpathlineto{\pgfqpoint{0.972372in}{5.359874in}}%
\pgfpathlineto{\pgfqpoint{0.972754in}{5.360162in}}%
\pgfpathlineto{\pgfqpoint{1.002327in}{5.382272in}}%
\pgfpathlineto{\pgfqpoint{1.006462in}{5.385344in}}%
\pgfpathlineto{\pgfqpoint{1.031901in}{5.404011in}}%
\pgfpathlineto{\pgfqpoint{1.041229in}{5.410813in}}%
\pgfpathlineto{\pgfqpoint{1.061474in}{5.425398in}}%
\pgfpathlineto{\pgfqpoint{1.076676in}{5.436283in}}%
\pgfpathlineto{\pgfqpoint{1.091047in}{5.446449in}}%
\pgfpathlineto{\pgfqpoint{1.112811in}{5.461752in}}%
\pgfpathlineto{\pgfqpoint{1.120621in}{5.467178in}}%
\pgfpathlineto{\pgfqpoint{1.149642in}{5.487222in}}%
\pgfpathlineto{\pgfqpoint{1.150194in}{5.487598in}}%
\pgfpathlineto{\pgfqpoint{1.179767in}{5.507632in}}%
\pgfpathlineto{\pgfqpoint{1.187279in}{5.512691in}}%
\pgfpathlineto{\pgfqpoint{1.209341in}{5.527372in}}%
\pgfpathlineto{\pgfqpoint{1.225644in}{5.538161in}}%
\pgfpathlineto{\pgfqpoint{1.238914in}{5.546838in}}%
\pgfpathlineto{\pgfqpoint{1.264735in}{5.563630in}}%
\pgfpathlineto{\pgfqpoint{1.268488in}{5.566041in}}%
\pgfpathlineto{\pgfqpoint{1.298061in}{5.584919in}}%
\pgfpathlineto{\pgfqpoint{1.304649in}{5.589100in}}%
\pgfpathlineto{\pgfqpoint{1.327634in}{5.603510in}}%
\pgfpathlineto{\pgfqpoint{1.345363in}{5.614570in}}%
\pgfpathlineto{\pgfqpoint{1.357208in}{5.621870in}}%
\pgfpathlineto{\pgfqpoint{1.386781in}{5.640009in}}%
\pgfpathlineto{\pgfqpoint{1.386830in}{5.640039in}}%
\pgfusepath{stroke}%
\end{pgfscope}%
\begin{pgfscope}%
\pgfpathrectangle{\pgfqpoint{0.854460in}{0.571603in}}{\pgfqpoint{5.885100in}{5.068436in}}%
\pgfusepath{clip}%
\pgfsetbuttcap%
\pgfsetroundjoin%
\pgfsetlinewidth{1.505625pt}%
\definecolor{currentstroke}{rgb}{0.496615,0.826376,0.306377}%
\pgfsetstrokecolor{currentstroke}%
\pgfsetdash{}{0pt}%
\pgfpathmoveto{\pgfqpoint{6.739560in}{3.599678in}}%
\pgfpathlineto{\pgfqpoint{6.738498in}{3.577007in}}%
\pgfpathlineto{\pgfqpoint{6.737572in}{3.551538in}}%
\pgfpathlineto{\pgfqpoint{6.736914in}{3.526068in}}%
\pgfpathlineto{\pgfqpoint{6.736526in}{3.500599in}}%
\pgfpathlineto{\pgfqpoint{6.736409in}{3.475129in}}%
\pgfpathlineto{\pgfqpoint{6.736562in}{3.449660in}}%
\pgfpathlineto{\pgfqpoint{6.736988in}{3.424190in}}%
\pgfpathlineto{\pgfqpoint{6.737688in}{3.398721in}}%
\pgfpathlineto{\pgfqpoint{6.738661in}{3.373251in}}%
\pgfpathlineto{\pgfqpoint{6.739560in}{3.354921in}}%
\pgfusepath{stroke}%
\end{pgfscope}%
\begin{pgfscope}%
\pgfpathrectangle{\pgfqpoint{0.854460in}{0.571603in}}{\pgfqpoint{5.885100in}{5.068436in}}%
\pgfusepath{clip}%
\pgfsetbuttcap%
\pgfsetroundjoin%
\pgfsetlinewidth{1.505625pt}%
\definecolor{currentstroke}{rgb}{0.496615,0.826376,0.306377}%
\pgfsetstrokecolor{currentstroke}%
\pgfsetdash{}{0pt}%
\pgfpathmoveto{\pgfqpoint{0.854460in}{5.312975in}}%
\pgfpathlineto{\pgfqpoint{0.881642in}{5.334405in}}%
\pgfpathlineto{\pgfqpoint{0.884034in}{5.336268in}}%
\pgfpathlineto{\pgfqpoint{0.913607in}{5.359145in}}%
\pgfpathlineto{\pgfqpoint{0.914557in}{5.359874in}}%
\pgfpathlineto{\pgfqpoint{0.943181in}{5.381593in}}%
\pgfpathlineto{\pgfqpoint{0.948156in}{5.385344in}}%
\pgfpathlineto{\pgfqpoint{0.972754in}{5.403664in}}%
\pgfpathlineto{\pgfqpoint{0.982415in}{5.410813in}}%
\pgfpathlineto{\pgfqpoint{1.002327in}{5.425373in}}%
\pgfpathlineto{\pgfqpoint{1.017342in}{5.436283in}}%
\pgfpathlineto{\pgfqpoint{1.031901in}{5.446734in}}%
\pgfpathlineto{\pgfqpoint{1.052947in}{5.461752in}}%
\pgfpathlineto{\pgfqpoint{1.061474in}{5.467764in}}%
\pgfpathlineto{\pgfqpoint{1.089238in}{5.487222in}}%
\pgfpathlineto{\pgfqpoint{1.091047in}{5.488474in}}%
\pgfpathlineto{\pgfqpoint{1.120621in}{5.508810in}}%
\pgfpathlineto{\pgfqpoint{1.126299in}{5.512691in}}%
\pgfpathlineto{\pgfqpoint{1.150194in}{5.528827in}}%
\pgfpathlineto{\pgfqpoint{1.164094in}{5.538161in}}%
\pgfpathlineto{\pgfqpoint{1.179767in}{5.548559in}}%
\pgfpathlineto{\pgfqpoint{1.202607in}{5.563630in}}%
\pgfpathlineto{\pgfqpoint{1.209341in}{5.568021in}}%
\pgfpathlineto{\pgfqpoint{1.238914in}{5.587189in}}%
\pgfpathlineto{\pgfqpoint{1.241882in}{5.589100in}}%
\pgfpathlineto{\pgfqpoint{1.268488in}{5.606025in}}%
\pgfpathlineto{\pgfqpoint{1.281987in}{5.614570in}}%
\pgfpathlineto{\pgfqpoint{1.298061in}{5.624621in}}%
\pgfpathlineto{\pgfqpoint{1.322837in}{5.640039in}}%
\pgfusepath{stroke}%
\end{pgfscope}%
\begin{pgfscope}%
\pgfpathrectangle{\pgfqpoint{0.854460in}{0.571603in}}{\pgfqpoint{5.885100in}{5.068436in}}%
\pgfusepath{clip}%
\pgfsetbuttcap%
\pgfsetroundjoin%
\pgfsetlinewidth{1.505625pt}%
\definecolor{currentstroke}{rgb}{0.555484,0.840254,0.269281}%
\pgfsetstrokecolor{currentstroke}%
\pgfsetdash{}{0pt}%
\pgfpathmoveto{\pgfqpoint{0.854460in}{5.356923in}}%
\pgfpathlineto{\pgfqpoint{0.858247in}{5.359874in}}%
\pgfpathlineto{\pgfqpoint{0.884034in}{5.379731in}}%
\pgfpathlineto{\pgfqpoint{0.891370in}{5.385344in}}%
\pgfpathlineto{\pgfqpoint{0.913607in}{5.402151in}}%
\pgfpathlineto{\pgfqpoint{0.925143in}{5.410813in}}%
\pgfpathlineto{\pgfqpoint{0.943181in}{5.424197in}}%
\pgfpathlineto{\pgfqpoint{0.959573in}{5.436283in}}%
\pgfpathlineto{\pgfqpoint{0.972754in}{5.445885in}}%
\pgfpathlineto{\pgfqpoint{0.994670in}{5.461752in}}%
\pgfpathlineto{\pgfqpoint{1.002327in}{5.467230in}}%
\pgfpathlineto{\pgfqpoint{1.030443in}{5.487222in}}%
\pgfpathlineto{\pgfqpoint{1.031901in}{5.488246in}}%
\pgfpathlineto{\pgfqpoint{1.061474in}{5.508879in}}%
\pgfpathlineto{\pgfqpoint{1.066971in}{5.512691in}}%
\pgfpathlineto{\pgfqpoint{1.091047in}{5.529187in}}%
\pgfpathlineto{\pgfqpoint{1.104219in}{5.538161in}}%
\pgfpathlineto{\pgfqpoint{1.120621in}{5.549201in}}%
\pgfpathlineto{\pgfqpoint{1.142174in}{5.563630in}}%
\pgfpathlineto{\pgfqpoint{1.150194in}{5.568935in}}%
\pgfpathlineto{\pgfqpoint{1.179767in}{5.588388in}}%
\pgfpathlineto{\pgfqpoint{1.180857in}{5.589100in}}%
\pgfpathlineto{\pgfqpoint{1.209341in}{5.607483in}}%
\pgfpathlineto{\pgfqpoint{1.220377in}{5.614570in}}%
\pgfpathlineto{\pgfqpoint{1.238914in}{5.626330in}}%
\pgfpathlineto{\pgfqpoint{1.260631in}{5.640039in}}%
\pgfusepath{stroke}%
\end{pgfscope}%
\begin{pgfscope}%
\pgfpathrectangle{\pgfqpoint{0.854460in}{0.571603in}}{\pgfqpoint{5.885100in}{5.068436in}}%
\pgfusepath{clip}%
\pgfsetbuttcap%
\pgfsetroundjoin%
\pgfsetlinewidth{1.505625pt}%
\definecolor{currentstroke}{rgb}{0.606045,0.850733,0.236712}%
\pgfsetstrokecolor{currentstroke}%
\pgfsetdash{}{0pt}%
\pgfpathmoveto{\pgfqpoint{0.854460in}{5.399493in}}%
\pgfpathlineto{\pgfqpoint{0.869317in}{5.410813in}}%
\pgfpathlineto{\pgfqpoint{0.884034in}{5.421892in}}%
\pgfpathlineto{\pgfqpoint{0.903271in}{5.436283in}}%
\pgfpathlineto{\pgfqpoint{0.913607in}{5.443922in}}%
\pgfpathlineto{\pgfqpoint{0.937881in}{5.461752in}}%
\pgfpathlineto{\pgfqpoint{0.943181in}{5.465599in}}%
\pgfpathlineto{\pgfqpoint{0.972754in}{5.486931in}}%
\pgfpathlineto{\pgfqpoint{0.973161in}{5.487222in}}%
\pgfpathlineto{\pgfqpoint{1.002327in}{5.507861in}}%
\pgfpathlineto{\pgfqpoint{1.009193in}{5.512691in}}%
\pgfpathlineto{\pgfqpoint{1.031901in}{5.528475in}}%
\pgfpathlineto{\pgfqpoint{1.045915in}{5.538161in}}%
\pgfpathlineto{\pgfqpoint{1.061474in}{5.548786in}}%
\pgfpathlineto{\pgfqpoint{1.083333in}{5.563630in}}%
\pgfpathlineto{\pgfqpoint{1.091047in}{5.568806in}}%
\pgfpathlineto{\pgfqpoint{1.120621in}{5.588539in}}%
\pgfpathlineto{\pgfqpoint{1.121467in}{5.589100in}}%
\pgfpathlineto{\pgfqpoint{1.150194in}{5.607908in}}%
\pgfpathlineto{\pgfqpoint{1.160422in}{5.614570in}}%
\pgfpathlineto{\pgfqpoint{1.179767in}{5.627018in}}%
\pgfpathlineto{\pgfqpoint{1.200103in}{5.640039in}}%
\pgfusepath{stroke}%
\end{pgfscope}%
\begin{pgfscope}%
\pgfpathrectangle{\pgfqpoint{0.854460in}{0.571603in}}{\pgfqpoint{5.885100in}{5.068436in}}%
\pgfusepath{clip}%
\pgfsetbuttcap%
\pgfsetroundjoin%
\pgfsetlinewidth{1.505625pt}%
\definecolor{currentstroke}{rgb}{0.668054,0.861999,0.196293}%
\pgfsetstrokecolor{currentstroke}%
\pgfsetdash{}{0pt}%
\pgfpathmoveto{\pgfqpoint{0.854460in}{5.440868in}}%
\pgfpathlineto{\pgfqpoint{0.882488in}{5.461752in}}%
\pgfpathlineto{\pgfqpoint{0.884034in}{5.462891in}}%
\pgfpathlineto{\pgfqpoint{0.913607in}{5.484515in}}%
\pgfpathlineto{\pgfqpoint{0.917332in}{5.487222in}}%
\pgfpathlineto{\pgfqpoint{0.943181in}{5.505777in}}%
\pgfpathlineto{\pgfqpoint{0.952869in}{5.512691in}}%
\pgfpathlineto{\pgfqpoint{0.972754in}{5.526712in}}%
\pgfpathlineto{\pgfqpoint{0.989085in}{5.538161in}}%
\pgfpathlineto{\pgfqpoint{1.002327in}{5.547334in}}%
\pgfpathlineto{\pgfqpoint{1.025987in}{5.563630in}}%
\pgfpathlineto{\pgfqpoint{1.031901in}{5.567655in}}%
\pgfpathlineto{\pgfqpoint{1.061474in}{5.587663in}}%
\pgfpathlineto{\pgfqpoint{1.063612in}{5.589100in}}%
\pgfpathlineto{\pgfqpoint{1.091047in}{5.607319in}}%
\pgfpathlineto{\pgfqpoint{1.102023in}{5.614570in}}%
\pgfpathlineto{\pgfqpoint{1.120621in}{5.626708in}}%
\pgfpathlineto{\pgfqpoint{1.141151in}{5.640039in}}%
\pgfusepath{stroke}%
\end{pgfscope}%
\begin{pgfscope}%
\pgfpathrectangle{\pgfqpoint{0.854460in}{0.571603in}}{\pgfqpoint{5.885100in}{5.068436in}}%
\pgfusepath{clip}%
\pgfsetbuttcap%
\pgfsetroundjoin%
\pgfsetlinewidth{1.505625pt}%
\definecolor{currentstroke}{rgb}{0.720391,0.870350,0.162603}%
\pgfsetstrokecolor{currentstroke}%
\pgfsetdash{}{0pt}%
\pgfpathmoveto{\pgfqpoint{0.854460in}{5.481038in}}%
\pgfpathlineto{\pgfqpoint{0.862851in}{5.487222in}}%
\pgfpathlineto{\pgfqpoint{0.884034in}{5.502647in}}%
\pgfpathlineto{\pgfqpoint{0.897911in}{5.512691in}}%
\pgfpathlineto{\pgfqpoint{0.913607in}{5.523917in}}%
\pgfpathlineto{\pgfqpoint{0.933640in}{5.538161in}}%
\pgfpathlineto{\pgfqpoint{0.943181in}{5.544863in}}%
\pgfpathlineto{\pgfqpoint{0.970046in}{5.563630in}}%
\pgfpathlineto{\pgfqpoint{0.972754in}{5.565500in}}%
\pgfpathlineto{\pgfqpoint{1.002327in}{5.585780in}}%
\pgfpathlineto{\pgfqpoint{1.007198in}{5.589100in}}%
\pgfpathlineto{\pgfqpoint{1.031901in}{5.605737in}}%
\pgfpathlineto{\pgfqpoint{1.045086in}{5.614570in}}%
\pgfpathlineto{\pgfqpoint{1.061474in}{5.625417in}}%
\pgfpathlineto{\pgfqpoint{1.083680in}{5.640039in}}%
\pgfusepath{stroke}%
\end{pgfscope}%
\begin{pgfscope}%
\pgfpathrectangle{\pgfqpoint{0.854460in}{0.571603in}}{\pgfqpoint{5.885100in}{5.068436in}}%
\pgfusepath{clip}%
\pgfsetbuttcap%
\pgfsetroundjoin%
\pgfsetlinewidth{1.505625pt}%
\definecolor{currentstroke}{rgb}{0.783315,0.879285,0.125405}%
\pgfsetstrokecolor{currentstroke}%
\pgfsetdash{}{0pt}%
\pgfpathmoveto{\pgfqpoint{0.854460in}{5.520108in}}%
\pgfpathlineto{\pgfqpoint{0.879497in}{5.538161in}}%
\pgfpathlineto{\pgfqpoint{0.884034in}{5.541393in}}%
\pgfpathlineto{\pgfqpoint{0.913607in}{5.562335in}}%
\pgfpathlineto{\pgfqpoint{0.915448in}{5.563630in}}%
\pgfpathlineto{\pgfqpoint{0.943181in}{5.582907in}}%
\pgfpathlineto{\pgfqpoint{0.952140in}{5.589100in}}%
\pgfpathlineto{\pgfqpoint{0.972754in}{5.603179in}}%
\pgfpathlineto{\pgfqpoint{0.989523in}{5.614570in}}%
\pgfpathlineto{\pgfqpoint{1.002327in}{5.623164in}}%
\pgfpathlineto{\pgfqpoint{1.027602in}{5.640039in}}%
\pgfusepath{stroke}%
\end{pgfscope}%
\begin{pgfscope}%
\pgfpathrectangle{\pgfqpoint{0.854460in}{0.571603in}}{\pgfqpoint{5.885100in}{5.068436in}}%
\pgfusepath{clip}%
\pgfsetbuttcap%
\pgfsetroundjoin%
\pgfsetlinewidth{1.505625pt}%
\definecolor{currentstroke}{rgb}{0.835270,0.886029,0.102646}%
\pgfsetstrokecolor{currentstroke}%
\pgfsetdash{}{0pt}%
\pgfpathmoveto{\pgfqpoint{0.854460in}{5.558151in}}%
\pgfpathlineto{\pgfqpoint{0.862142in}{5.563630in}}%
\pgfpathlineto{\pgfqpoint{0.884034in}{5.579061in}}%
\pgfpathlineto{\pgfqpoint{0.898357in}{5.589100in}}%
\pgfpathlineto{\pgfqpoint{0.913607in}{5.599662in}}%
\pgfpathlineto{\pgfqpoint{0.935252in}{5.614570in}}%
\pgfpathlineto{\pgfqpoint{0.943181in}{5.619966in}}%
\pgfpathlineto{\pgfqpoint{0.972754in}{5.639985in}}%
\pgfpathlineto{\pgfqpoint{0.972835in}{5.640039in}}%
\pgfusepath{stroke}%
\end{pgfscope}%
\begin{pgfscope}%
\pgfpathrectangle{\pgfqpoint{0.854460in}{0.571603in}}{\pgfqpoint{5.885100in}{5.068436in}}%
\pgfusepath{clip}%
\pgfsetbuttcap%
\pgfsetroundjoin%
\pgfsetlinewidth{1.505625pt}%
\definecolor{currentstroke}{rgb}{0.896320,0.893616,0.096335}%
\pgfsetstrokecolor{currentstroke}%
\pgfsetdash{}{0pt}%
\pgfpathmoveto{\pgfqpoint{0.854460in}{5.595201in}}%
\pgfpathlineto{\pgfqpoint{0.882197in}{5.614570in}}%
\pgfpathlineto{\pgfqpoint{0.884034in}{5.615837in}}%
\pgfpathlineto{\pgfqpoint{0.913607in}{5.636110in}}%
\pgfpathlineto{\pgfqpoint{0.919371in}{5.640039in}}%
\pgfusepath{stroke}%
\end{pgfscope}%
\begin{pgfscope}%
\pgfpathrectangle{\pgfqpoint{0.854460in}{0.571603in}}{\pgfqpoint{5.885100in}{5.068436in}}%
\pgfusepath{clip}%
\pgfsetbuttcap%
\pgfsetroundjoin%
\pgfsetlinewidth{1.505625pt}%
\definecolor{currentstroke}{rgb}{0.945636,0.899815,0.112838}%
\pgfsetstrokecolor{currentstroke}%
\pgfsetdash{}{0pt}%
\pgfpathmoveto{\pgfqpoint{0.854460in}{5.631318in}}%
\pgfpathlineto{\pgfqpoint{0.867082in}{5.640039in}}%
\pgfusepath{stroke}%
\end{pgfscope}%
\begin{pgfscope}%
\pgfpathrectangle{\pgfqpoint{0.854460in}{0.571603in}}{\pgfqpoint{5.885100in}{5.068436in}}%
\pgfusepath{clip}%
\pgfsetrectcap%
\pgfsetroundjoin%
\pgfsetlinewidth{1.505625pt}%
\definecolor{currentstroke}{rgb}{1.000000,0.000000,0.000000}%
\pgfsetstrokecolor{currentstroke}%
\pgfsetdash{}{0pt}%
\pgfpathmoveto{\pgfqpoint{5.758710in}{4.372930in}}%
\pgfpathlineto{\pgfqpoint{2.544593in}{5.451593in}}%
\pgfpathlineto{\pgfqpoint{3.593593in}{1.518217in}}%
\pgfpathlineto{\pgfqpoint{3.439286in}{1.136257in}}%
\pgfpathlineto{\pgfqpoint{3.586714in}{1.409049in}}%
\pgfpathlineto{\pgfqpoint{3.547290in}{1.191602in}}%
\pgfpathlineto{\pgfqpoint{3.588765in}{1.272658in}}%
\pgfpathlineto{\pgfqpoint{3.575674in}{1.222389in}}%
\pgfpathlineto{\pgfqpoint{3.580248in}{1.232175in}}%
\pgfpathlineto{\pgfqpoint{3.576934in}{1.227839in}}%
\pgfpathlineto{\pgfqpoint{3.577534in}{1.228875in}}%
\pgfpathlineto{\pgfqpoint{3.577658in}{1.227971in}}%
\pgfusepath{stroke}%
\end{pgfscope}%
\begin{pgfscope}%
\pgfpathrectangle{\pgfqpoint{0.854460in}{0.571603in}}{\pgfqpoint{5.885100in}{5.068436in}}%
\pgfusepath{clip}%
\pgfsetbuttcap%
\pgfsetroundjoin%
\definecolor{currentfill}{rgb}{1.000000,0.000000,0.000000}%
\pgfsetfillcolor{currentfill}%
\pgfsetlinewidth{1.003750pt}%
\definecolor{currentstroke}{rgb}{1.000000,0.000000,0.000000}%
\pgfsetstrokecolor{currentstroke}%
\pgfsetdash{}{0pt}%
\pgfsys@defobject{currentmarker}{\pgfqpoint{-0.041667in}{-0.041667in}}{\pgfqpoint{0.041667in}{0.041667in}}{%
\pgfpathmoveto{\pgfqpoint{0.000000in}{-0.041667in}}%
\pgfpathcurveto{\pgfqpoint{0.011050in}{-0.041667in}}{\pgfqpoint{0.021649in}{-0.037276in}}{\pgfqpoint{0.029463in}{-0.029463in}}%
\pgfpathcurveto{\pgfqpoint{0.037276in}{-0.021649in}}{\pgfqpoint{0.041667in}{-0.011050in}}{\pgfqpoint{0.041667in}{0.000000in}}%
\pgfpathcurveto{\pgfqpoint{0.041667in}{0.011050in}}{\pgfqpoint{0.037276in}{0.021649in}}{\pgfqpoint{0.029463in}{0.029463in}}%
\pgfpathcurveto{\pgfqpoint{0.021649in}{0.037276in}}{\pgfqpoint{0.011050in}{0.041667in}}{\pgfqpoint{0.000000in}{0.041667in}}%
\pgfpathcurveto{\pgfqpoint{-0.011050in}{0.041667in}}{\pgfqpoint{-0.021649in}{0.037276in}}{\pgfqpoint{-0.029463in}{0.029463in}}%
\pgfpathcurveto{\pgfqpoint{-0.037276in}{0.021649in}}{\pgfqpoint{-0.041667in}{0.011050in}}{\pgfqpoint{-0.041667in}{0.000000in}}%
\pgfpathcurveto{\pgfqpoint{-0.041667in}{-0.011050in}}{\pgfqpoint{-0.037276in}{-0.021649in}}{\pgfqpoint{-0.029463in}{-0.029463in}}%
\pgfpathcurveto{\pgfqpoint{-0.021649in}{-0.037276in}}{\pgfqpoint{-0.011050in}{-0.041667in}}{\pgfqpoint{0.000000in}{-0.041667in}}%
\pgfpathlineto{\pgfqpoint{0.000000in}{-0.041667in}}%
\pgfpathclose%
\pgfusepath{stroke,fill}%
}%
\begin{pgfscope}%
\pgfsys@transformshift{5.758710in}{4.372930in}%
\pgfsys@useobject{currentmarker}{}%
\end{pgfscope}%
\begin{pgfscope}%
\pgfsys@transformshift{2.544593in}{5.451593in}%
\pgfsys@useobject{currentmarker}{}%
\end{pgfscope}%
\begin{pgfscope}%
\pgfsys@transformshift{3.593593in}{1.518217in}%
\pgfsys@useobject{currentmarker}{}%
\end{pgfscope}%
\begin{pgfscope}%
\pgfsys@transformshift{3.439286in}{1.136257in}%
\pgfsys@useobject{currentmarker}{}%
\end{pgfscope}%
\begin{pgfscope}%
\pgfsys@transformshift{3.586714in}{1.409049in}%
\pgfsys@useobject{currentmarker}{}%
\end{pgfscope}%
\begin{pgfscope}%
\pgfsys@transformshift{3.547290in}{1.191602in}%
\pgfsys@useobject{currentmarker}{}%
\end{pgfscope}%
\begin{pgfscope}%
\pgfsys@transformshift{3.588765in}{1.272658in}%
\pgfsys@useobject{currentmarker}{}%
\end{pgfscope}%
\begin{pgfscope}%
\pgfsys@transformshift{3.575674in}{1.222389in}%
\pgfsys@useobject{currentmarker}{}%
\end{pgfscope}%
\begin{pgfscope}%
\pgfsys@transformshift{3.580248in}{1.232175in}%
\pgfsys@useobject{currentmarker}{}%
\end{pgfscope}%
\begin{pgfscope}%
\pgfsys@transformshift{3.576934in}{1.227839in}%
\pgfsys@useobject{currentmarker}{}%
\end{pgfscope}%
\begin{pgfscope}%
\pgfsys@transformshift{3.577534in}{1.228875in}%
\pgfsys@useobject{currentmarker}{}%
\end{pgfscope}%
\begin{pgfscope}%
\pgfsys@transformshift{3.577658in}{1.227971in}%
\pgfsys@useobject{currentmarker}{}%
\end{pgfscope}%
\end{pgfscope}%
\begin{pgfscope}%
\pgfsetrectcap%
\pgfsetmiterjoin%
\pgfsetlinewidth{0.803000pt}%
\definecolor{currentstroke}{rgb}{0.000000,0.000000,0.000000}%
\pgfsetstrokecolor{currentstroke}%
\pgfsetdash{}{0pt}%
\pgfpathmoveto{\pgfqpoint{0.854460in}{0.571603in}}%
\pgfpathlineto{\pgfqpoint{0.854460in}{5.640039in}}%
\pgfusepath{stroke}%
\end{pgfscope}%
\begin{pgfscope}%
\pgfsetrectcap%
\pgfsetmiterjoin%
\pgfsetlinewidth{0.803000pt}%
\definecolor{currentstroke}{rgb}{0.000000,0.000000,0.000000}%
\pgfsetstrokecolor{currentstroke}%
\pgfsetdash{}{0pt}%
\pgfpathmoveto{\pgfqpoint{6.739560in}{0.571603in}}%
\pgfpathlineto{\pgfqpoint{6.739560in}{5.640039in}}%
\pgfusepath{stroke}%
\end{pgfscope}%
\begin{pgfscope}%
\pgfsetrectcap%
\pgfsetmiterjoin%
\pgfsetlinewidth{0.803000pt}%
\definecolor{currentstroke}{rgb}{0.000000,0.000000,0.000000}%
\pgfsetstrokecolor{currentstroke}%
\pgfsetdash{}{0pt}%
\pgfpathmoveto{\pgfqpoint{0.854460in}{0.571603in}}%
\pgfpathlineto{\pgfqpoint{6.739560in}{0.571603in}}%
\pgfusepath{stroke}%
\end{pgfscope}%
\begin{pgfscope}%
\pgfsetrectcap%
\pgfsetmiterjoin%
\pgfsetlinewidth{0.803000pt}%
\definecolor{currentstroke}{rgb}{0.000000,0.000000,0.000000}%
\pgfsetstrokecolor{currentstroke}%
\pgfsetdash{}{0pt}%
\pgfpathmoveto{\pgfqpoint{0.854460in}{5.640039in}}%
\pgfpathlineto{\pgfqpoint{6.739560in}{5.640039in}}%
\pgfusepath{stroke}%
\end{pgfscope}%
\begin{pgfscope}%
\definecolor{textcolor}{rgb}{0.000000,0.000000,0.000000}%
\pgfsetstrokecolor{textcolor}%
\pgfsetfillcolor{textcolor}%
\pgftext[x=3.797010in,y=5.723372in,,base]{\color{textcolor}\sffamily\fontsize{12.000000}{14.400000}\selectfont 2D Contour Plot}%
\end{pgfscope}%
\begin{pgfscope}%
\pgfsetbuttcap%
\pgfsetmiterjoin%
\definecolor{currentfill}{rgb}{1.000000,1.000000,1.000000}%
\pgfsetfillcolor{currentfill}%
\pgfsetfillopacity{0.800000}%
\pgfsetlinewidth{1.003750pt}%
\definecolor{currentstroke}{rgb}{0.800000,0.800000,0.800000}%
\pgfsetstrokecolor{currentstroke}%
\pgfsetstrokeopacity{0.800000}%
\pgfsetdash{}{0pt}%
\pgfpathmoveto{\pgfqpoint{5.536408in}{5.121213in}}%
\pgfpathlineto{\pgfqpoint{6.642338in}{5.121213in}}%
\pgfpathquadraticcurveto{\pgfqpoint{6.670116in}{5.121213in}}{\pgfqpoint{6.670116in}{5.148991in}}%
\pgfpathlineto{\pgfqpoint{6.670116in}{5.542817in}}%
\pgfpathquadraticcurveto{\pgfqpoint{6.670116in}{5.570595in}}{\pgfqpoint{6.642338in}{5.570595in}}%
\pgfpathlineto{\pgfqpoint{5.536408in}{5.570595in}}%
\pgfpathquadraticcurveto{\pgfqpoint{5.508630in}{5.570595in}}{\pgfqpoint{5.508630in}{5.542817in}}%
\pgfpathlineto{\pgfqpoint{5.508630in}{5.148991in}}%
\pgfpathquadraticcurveto{\pgfqpoint{5.508630in}{5.121213in}}{\pgfqpoint{5.536408in}{5.121213in}}%
\pgfpathlineto{\pgfqpoint{5.536408in}{5.121213in}}%
\pgfpathclose%
\pgfusepath{stroke,fill}%
\end{pgfscope}%
\begin{pgfscope}%
\pgfsetrectcap%
\pgfsetroundjoin%
\pgfsetlinewidth{1.505625pt}%
\definecolor{currentstroke}{rgb}{1.000000,0.000000,0.000000}%
\pgfsetstrokecolor{currentstroke}%
\pgfsetdash{}{0pt}%
\pgfpathmoveto{\pgfqpoint{5.564186in}{5.458127in}}%
\pgfpathlineto{\pgfqpoint{5.703075in}{5.458127in}}%
\pgfpathlineto{\pgfqpoint{5.841964in}{5.458127in}}%
\pgfusepath{stroke}%
\end{pgfscope}%
\begin{pgfscope}%
\pgfsetbuttcap%
\pgfsetroundjoin%
\definecolor{currentfill}{rgb}{1.000000,0.000000,0.000000}%
\pgfsetfillcolor{currentfill}%
\pgfsetlinewidth{1.003750pt}%
\definecolor{currentstroke}{rgb}{1.000000,0.000000,0.000000}%
\pgfsetstrokecolor{currentstroke}%
\pgfsetdash{}{0pt}%
\pgfsys@defobject{currentmarker}{\pgfqpoint{-0.041667in}{-0.041667in}}{\pgfqpoint{0.041667in}{0.041667in}}{%
\pgfpathmoveto{\pgfqpoint{0.000000in}{-0.041667in}}%
\pgfpathcurveto{\pgfqpoint{0.011050in}{-0.041667in}}{\pgfqpoint{0.021649in}{-0.037276in}}{\pgfqpoint{0.029463in}{-0.029463in}}%
\pgfpathcurveto{\pgfqpoint{0.037276in}{-0.021649in}}{\pgfqpoint{0.041667in}{-0.011050in}}{\pgfqpoint{0.041667in}{0.000000in}}%
\pgfpathcurveto{\pgfqpoint{0.041667in}{0.011050in}}{\pgfqpoint{0.037276in}{0.021649in}}{\pgfqpoint{0.029463in}{0.029463in}}%
\pgfpathcurveto{\pgfqpoint{0.021649in}{0.037276in}}{\pgfqpoint{0.011050in}{0.041667in}}{\pgfqpoint{0.000000in}{0.041667in}}%
\pgfpathcurveto{\pgfqpoint{-0.011050in}{0.041667in}}{\pgfqpoint{-0.021649in}{0.037276in}}{\pgfqpoint{-0.029463in}{0.029463in}}%
\pgfpathcurveto{\pgfqpoint{-0.037276in}{0.021649in}}{\pgfqpoint{-0.041667in}{0.011050in}}{\pgfqpoint{-0.041667in}{0.000000in}}%
\pgfpathcurveto{\pgfqpoint{-0.041667in}{-0.011050in}}{\pgfqpoint{-0.037276in}{-0.021649in}}{\pgfqpoint{-0.029463in}{-0.029463in}}%
\pgfpathcurveto{\pgfqpoint{-0.021649in}{-0.037276in}}{\pgfqpoint{-0.011050in}{-0.041667in}}{\pgfqpoint{0.000000in}{-0.041667in}}%
\pgfpathlineto{\pgfqpoint{0.000000in}{-0.041667in}}%
\pgfpathclose%
\pgfusepath{stroke,fill}%
}%
\begin{pgfscope}%
\pgfsys@transformshift{5.703075in}{5.458127in}%
\pgfsys@useobject{currentmarker}{}%
\end{pgfscope}%
\end{pgfscope}%
\begin{pgfscope}%
\definecolor{textcolor}{rgb}{0.000000,0.000000,0.000000}%
\pgfsetstrokecolor{textcolor}%
\pgfsetfillcolor{textcolor}%
\pgftext[x=5.953075in,y=5.409516in,left,base]{\color{textcolor}\sffamily\fontsize{10.000000}{12.000000}\selectfont Iterations}%
\end{pgfscope}%
\begin{pgfscope}%
\pgfsetbuttcap%
\pgfsetroundjoin%
\definecolor{currentfill}{rgb}{0.000000,0.000000,1.000000}%
\pgfsetfillcolor{currentfill}%
\pgfsetlinewidth{1.003750pt}%
\definecolor{currentstroke}{rgb}{0.000000,0.000000,1.000000}%
\pgfsetstrokecolor{currentstroke}%
\pgfsetdash{}{0pt}%
\pgfsys@defobject{currentmarker}{\pgfqpoint{-0.069444in}{-0.069444in}}{\pgfqpoint{0.069444in}{0.069444in}}{%
\pgfpathmoveto{\pgfqpoint{0.000000in}{-0.069444in}}%
\pgfpathcurveto{\pgfqpoint{0.018417in}{-0.069444in}}{\pgfqpoint{0.036082in}{-0.062127in}}{\pgfqpoint{0.049105in}{-0.049105in}}%
\pgfpathcurveto{\pgfqpoint{0.062127in}{-0.036082in}}{\pgfqpoint{0.069444in}{-0.018417in}}{\pgfqpoint{0.069444in}{0.000000in}}%
\pgfpathcurveto{\pgfqpoint{0.069444in}{0.018417in}}{\pgfqpoint{0.062127in}{0.036082in}}{\pgfqpoint{0.049105in}{0.049105in}}%
\pgfpathcurveto{\pgfqpoint{0.036082in}{0.062127in}}{\pgfqpoint{0.018417in}{0.069444in}}{\pgfqpoint{0.000000in}{0.069444in}}%
\pgfpathcurveto{\pgfqpoint{-0.018417in}{0.069444in}}{\pgfqpoint{-0.036082in}{0.062127in}}{\pgfqpoint{-0.049105in}{0.049105in}}%
\pgfpathcurveto{\pgfqpoint{-0.062127in}{0.036082in}}{\pgfqpoint{-0.069444in}{0.018417in}}{\pgfqpoint{-0.069444in}{0.000000in}}%
\pgfpathcurveto{\pgfqpoint{-0.069444in}{-0.018417in}}{\pgfqpoint{-0.062127in}{-0.036082in}}{\pgfqpoint{-0.049105in}{-0.049105in}}%
\pgfpathcurveto{\pgfqpoint{-0.036082in}{-0.062127in}}{\pgfqpoint{-0.018417in}{-0.069444in}}{\pgfqpoint{0.000000in}{-0.069444in}}%
\pgfpathlineto{\pgfqpoint{0.000000in}{-0.069444in}}%
\pgfpathclose%
\pgfusepath{stroke,fill}%
}%
\begin{pgfscope}%
\pgfsys@transformshift{5.703075in}{5.242117in}%
\pgfsys@useobject{currentmarker}{}%
\end{pgfscope}%
\end{pgfscope}%
\begin{pgfscope}%
\definecolor{textcolor}{rgb}{0.000000,0.000000,0.000000}%
\pgfsetstrokecolor{textcolor}%
\pgfsetfillcolor{textcolor}%
\pgftext[x=5.953075in,y=5.205659in,left,base]{\color{textcolor}\sffamily\fontsize{10.000000}{12.000000}\selectfont Minimum}%
\end{pgfscope}%
\end{pgfpicture}%
\makeatother%
\endgroup%
}
        \caption{Pohľad zhora (Vrstevnice)}
        \label{fig:newton_vlavo}
    \end{subfigure}
    \hfill
    \begin{subfigure}{0.48\textwidth}
        \centering
        \resizebox{\linewidth}{!}{%% Creator: Matplotlib, PGF backend
%%
%% To include the figure in your LaTeX document, write
%%   \input{<filename>.pgf}
%%
%% Make sure the required packages are loaded in your preamble
%%   \usepackage{pgf}
%%
%% Also ensure that all the required font packages are loaded; for instance,
%% the lmodern package is sometimes necessary when using math font.
%%   \usepackage{lmodern}
%%
%% Figures using additional raster images can only be included by \input if
%% they are in the same directory as the main LaTeX file. For loading figures
%% from other directories you can use the `import` package
%%   \usepackage{import}
%%
%% and then include the figures with
%%   \import{<path to file>}{<filename>.pgf}
%%
%% Matplotlib used the following preamble
%%   
%%   \usepackage{fontspec}
%%   \setmainfont{DejaVuSerif.ttf}[Path=\detokenize{/home/radimek/Documents/projekt_mat_prog/mat_prog_kernel/lib/python3.12/site-packages/matplotlib/mpl-data/fonts/ttf/}]
%%   \setsansfont{DejaVuSans.ttf}[Path=\detokenize{/home/radimek/Documents/projekt_mat_prog/mat_prog_kernel/lib/python3.12/site-packages/matplotlib/mpl-data/fonts/ttf/}]
%%   \setmonofont{DejaVuSansMono.ttf}[Path=\detokenize{/home/radimek/Documents/projekt_mat_prog/mat_prog_kernel/lib/python3.12/site-packages/matplotlib/mpl-data/fonts/ttf/}]
%%   \makeatletter\@ifpackageloaded{underscore}{}{\usepackage[strings]{underscore}}\makeatother
%%
\begingroup%
\makeatletter%
\begin{pgfpicture}%
\pgfpathrectangle{\pgfpointorigin}{\pgfqpoint{8.000000in}{6.000000in}}%
\pgfusepath{use as bounding box, clip}%
\begin{pgfscope}%
\pgfsetbuttcap%
\pgfsetmiterjoin%
\definecolor{currentfill}{rgb}{1.000000,1.000000,1.000000}%
\pgfsetfillcolor{currentfill}%
\pgfsetlinewidth{0.000000pt}%
\definecolor{currentstroke}{rgb}{1.000000,1.000000,1.000000}%
\pgfsetstrokecolor{currentstroke}%
\pgfsetdash{}{0pt}%
\pgfpathmoveto{\pgfqpoint{0.000000in}{0.000000in}}%
\pgfpathlineto{\pgfqpoint{8.000000in}{0.000000in}}%
\pgfpathlineto{\pgfqpoint{8.000000in}{6.000000in}}%
\pgfpathlineto{\pgfqpoint{0.000000in}{6.000000in}}%
\pgfpathlineto{\pgfqpoint{0.000000in}{0.000000in}}%
\pgfpathclose%
\pgfusepath{fill}%
\end{pgfscope}%
\begin{pgfscope}%
\pgfsetbuttcap%
\pgfsetmiterjoin%
\definecolor{currentfill}{rgb}{1.000000,1.000000,1.000000}%
\pgfsetfillcolor{currentfill}%
\pgfsetlinewidth{0.000000pt}%
\definecolor{currentstroke}{rgb}{0.000000,0.000000,0.000000}%
\pgfsetstrokecolor{currentstroke}%
\pgfsetstrokeopacity{0.000000}%
\pgfsetdash{}{0pt}%
\pgfpathmoveto{\pgfqpoint{1.254980in}{0.150000in}}%
\pgfpathlineto{\pgfqpoint{6.745020in}{0.150000in}}%
\pgfpathlineto{\pgfqpoint{6.745020in}{5.640039in}}%
\pgfpathlineto{\pgfqpoint{1.254980in}{5.640039in}}%
\pgfpathlineto{\pgfqpoint{1.254980in}{0.150000in}}%
\pgfpathclose%
\pgfusepath{fill}%
\end{pgfscope}%
\begin{pgfscope}%
\pgfsetbuttcap%
\pgfsetmiterjoin%
\definecolor{currentfill}{rgb}{0.950000,0.950000,0.950000}%
\pgfsetfillcolor{currentfill}%
\pgfsetfillopacity{0.500000}%
\pgfsetlinewidth{1.003750pt}%
\definecolor{currentstroke}{rgb}{0.950000,0.950000,0.950000}%
\pgfsetstrokecolor{currentstroke}%
\pgfsetstrokeopacity{0.500000}%
\pgfsetdash{}{0pt}%
\pgfpathmoveto{\pgfqpoint{1.669516in}{1.503668in}}%
\pgfpathlineto{\pgfqpoint{3.482506in}{3.023352in}}%
\pgfpathlineto{\pgfqpoint{3.457304in}{5.215008in}}%
\pgfpathlineto{\pgfqpoint{1.557553in}{3.828657in}}%
\pgfusepath{stroke,fill}%
\end{pgfscope}%
\begin{pgfscope}%
\pgfsetbuttcap%
\pgfsetmiterjoin%
\definecolor{currentfill}{rgb}{0.900000,0.900000,0.900000}%
\pgfsetfillcolor{currentfill}%
\pgfsetfillopacity{0.500000}%
\pgfsetlinewidth{1.003750pt}%
\definecolor{currentstroke}{rgb}{0.900000,0.900000,0.900000}%
\pgfsetstrokecolor{currentstroke}%
\pgfsetstrokeopacity{0.500000}%
\pgfsetdash{}{0pt}%
\pgfpathmoveto{\pgfqpoint{3.482506in}{3.023352in}}%
\pgfpathlineto{\pgfqpoint{6.391709in}{2.177762in}}%
\pgfpathlineto{\pgfqpoint{6.495528in}{4.444907in}}%
\pgfpathlineto{\pgfqpoint{3.457304in}{5.215008in}}%
\pgfusepath{stroke,fill}%
\end{pgfscope}%
\begin{pgfscope}%
\pgfsetbuttcap%
\pgfsetmiterjoin%
\definecolor{currentfill}{rgb}{0.925000,0.925000,0.925000}%
\pgfsetfillcolor{currentfill}%
\pgfsetfillopacity{0.500000}%
\pgfsetlinewidth{1.003750pt}%
\definecolor{currentstroke}{rgb}{0.925000,0.925000,0.925000}%
\pgfsetstrokecolor{currentstroke}%
\pgfsetstrokeopacity{0.500000}%
\pgfsetdash{}{0pt}%
\pgfpathmoveto{\pgfqpoint{1.669516in}{1.503668in}}%
\pgfpathlineto{\pgfqpoint{4.753413in}{0.496467in}}%
\pgfpathlineto{\pgfqpoint{6.391709in}{2.177762in}}%
\pgfpathlineto{\pgfqpoint{3.482506in}{3.023352in}}%
\pgfusepath{stroke,fill}%
\end{pgfscope}%
\begin{pgfscope}%
\pgfsetrectcap%
\pgfsetroundjoin%
\pgfsetlinewidth{0.803000pt}%
\definecolor{currentstroke}{rgb}{0.000000,0.000000,0.000000}%
\pgfsetstrokecolor{currentstroke}%
\pgfsetdash{}{0pt}%
\pgfpathmoveto{\pgfqpoint{1.669516in}{1.503668in}}%
\pgfpathlineto{\pgfqpoint{4.753413in}{0.496467in}}%
\pgfusepath{stroke}%
\end{pgfscope}%
\begin{pgfscope}%
\definecolor{textcolor}{rgb}{0.000000,0.000000,0.000000}%
\pgfsetstrokecolor{textcolor}%
\pgfsetfillcolor{textcolor}%
\pgftext[x=2.945156in,y=0.524780in,,]{\color{textcolor}\sffamily\fontsize{10.000000}{12.000000}\selectfont x}%
\end{pgfscope}%
\begin{pgfscope}%
\pgfsetbuttcap%
\pgfsetroundjoin%
\pgfsetlinewidth{0.803000pt}%
\definecolor{currentstroke}{rgb}{0.690196,0.690196,0.690196}%
\pgfsetstrokecolor{currentstroke}%
\pgfsetdash{}{0pt}%
\pgfpathmoveto{\pgfqpoint{1.856293in}{1.442666in}}%
\pgfpathlineto{\pgfqpoint{3.659435in}{2.971926in}}%
\pgfpathlineto{\pgfqpoint{3.641714in}{5.168266in}}%
\pgfusepath{stroke}%
\end{pgfscope}%
\begin{pgfscope}%
\pgfsetbuttcap%
\pgfsetroundjoin%
\pgfsetlinewidth{0.803000pt}%
\definecolor{currentstroke}{rgb}{0.690196,0.690196,0.690196}%
\pgfsetstrokecolor{currentstroke}%
\pgfsetdash{}{0pt}%
\pgfpathmoveto{\pgfqpoint{2.287848in}{1.301721in}}%
\pgfpathlineto{\pgfqpoint{4.067873in}{2.853209in}}%
\pgfpathlineto{\pgfqpoint{4.067601in}{5.060316in}}%
\pgfusepath{stroke}%
\end{pgfscope}%
\begin{pgfscope}%
\pgfsetbuttcap%
\pgfsetroundjoin%
\pgfsetlinewidth{0.803000pt}%
\definecolor{currentstroke}{rgb}{0.690196,0.690196,0.690196}%
\pgfsetstrokecolor{currentstroke}%
\pgfsetdash{}{0pt}%
\pgfpathmoveto{\pgfqpoint{2.725971in}{1.158630in}}%
\pgfpathlineto{\pgfqpoint{4.482010in}{2.732836in}}%
\pgfpathlineto{\pgfqpoint{4.499690in}{4.950794in}}%
\pgfusepath{stroke}%
\end{pgfscope}%
\begin{pgfscope}%
\pgfsetbuttcap%
\pgfsetroundjoin%
\pgfsetlinewidth{0.803000pt}%
\definecolor{currentstroke}{rgb}{0.690196,0.690196,0.690196}%
\pgfsetstrokecolor{currentstroke}%
\pgfsetdash{}{0pt}%
\pgfpathmoveto{\pgfqpoint{3.170814in}{1.013344in}}%
\pgfpathlineto{\pgfqpoint{4.901969in}{2.610770in}}%
\pgfpathlineto{\pgfqpoint{4.938117in}{4.839666in}}%
\pgfusepath{stroke}%
\end{pgfscope}%
\begin{pgfscope}%
\pgfsetbuttcap%
\pgfsetroundjoin%
\pgfsetlinewidth{0.803000pt}%
\definecolor{currentstroke}{rgb}{0.690196,0.690196,0.690196}%
\pgfsetstrokecolor{currentstroke}%
\pgfsetdash{}{0pt}%
\pgfpathmoveto{\pgfqpoint{3.622534in}{0.865812in}}%
\pgfpathlineto{\pgfqpoint{5.327872in}{2.486977in}}%
\pgfpathlineto{\pgfqpoint{5.383022in}{4.726895in}}%
\pgfusepath{stroke}%
\end{pgfscope}%
\begin{pgfscope}%
\pgfsetbuttcap%
\pgfsetroundjoin%
\pgfsetlinewidth{0.803000pt}%
\definecolor{currentstroke}{rgb}{0.690196,0.690196,0.690196}%
\pgfsetstrokecolor{currentstroke}%
\pgfsetdash{}{0pt}%
\pgfpathmoveto{\pgfqpoint{4.081290in}{0.715983in}}%
\pgfpathlineto{\pgfqpoint{5.759846in}{2.361419in}}%
\pgfpathlineto{\pgfqpoint{5.834551in}{4.612446in}}%
\pgfusepath{stroke}%
\end{pgfscope}%
\begin{pgfscope}%
\pgfsetbuttcap%
\pgfsetroundjoin%
\pgfsetlinewidth{0.803000pt}%
\definecolor{currentstroke}{rgb}{0.690196,0.690196,0.690196}%
\pgfsetstrokecolor{currentstroke}%
\pgfsetdash{}{0pt}%
\pgfpathmoveto{\pgfqpoint{4.547248in}{0.563801in}}%
\pgfpathlineto{\pgfqpoint{6.198022in}{2.234059in}}%
\pgfpathlineto{\pgfqpoint{6.292853in}{4.496280in}}%
\pgfusepath{stroke}%
\end{pgfscope}%
\begin{pgfscope}%
\pgfsetrectcap%
\pgfsetroundjoin%
\pgfsetlinewidth{0.803000pt}%
\definecolor{currentstroke}{rgb}{0.000000,0.000000,0.000000}%
\pgfsetstrokecolor{currentstroke}%
\pgfsetdash{}{0pt}%
\pgfpathmoveto{\pgfqpoint{1.871995in}{1.455983in}}%
\pgfpathlineto{\pgfqpoint{1.824823in}{1.415976in}}%
\pgfusepath{stroke}%
\end{pgfscope}%
\begin{pgfscope}%
\definecolor{textcolor}{rgb}{0.000000,0.000000,0.000000}%
\pgfsetstrokecolor{textcolor}%
\pgfsetfillcolor{textcolor}%
\pgftext[x=1.751850in,y=1.224727in,,top]{\color{textcolor}\sffamily\fontsize{10.000000}{12.000000}\selectfont \ensuremath{-}1.0}%
\end{pgfscope}%
\begin{pgfscope}%
\pgfsetrectcap%
\pgfsetroundjoin%
\pgfsetlinewidth{0.803000pt}%
\definecolor{currentstroke}{rgb}{0.000000,0.000000,0.000000}%
\pgfsetstrokecolor{currentstroke}%
\pgfsetdash{}{0pt}%
\pgfpathmoveto{\pgfqpoint{2.303357in}{1.315239in}}%
\pgfpathlineto{\pgfqpoint{2.256761in}{1.274625in}}%
\pgfusepath{stroke}%
\end{pgfscope}%
\begin{pgfscope}%
\definecolor{textcolor}{rgb}{0.000000,0.000000,0.000000}%
\pgfsetstrokecolor{textcolor}%
\pgfsetfillcolor{textcolor}%
\pgftext[x=2.183700in,y=1.081776in,,top]{\color{textcolor}\sffamily\fontsize{10.000000}{12.000000}\selectfont \ensuremath{-}0.5}%
\end{pgfscope}%
\begin{pgfscope}%
\pgfsetrectcap%
\pgfsetroundjoin%
\pgfsetlinewidth{0.803000pt}%
\definecolor{currentstroke}{rgb}{0.000000,0.000000,0.000000}%
\pgfsetstrokecolor{currentstroke}%
\pgfsetdash{}{0pt}%
\pgfpathmoveto{\pgfqpoint{2.741281in}{1.172355in}}%
\pgfpathlineto{\pgfqpoint{2.695283in}{1.131120in}}%
\pgfusepath{stroke}%
\end{pgfscope}%
\begin{pgfscope}%
\definecolor{textcolor}{rgb}{0.000000,0.000000,0.000000}%
\pgfsetstrokecolor{textcolor}%
\pgfsetfillcolor{textcolor}%
\pgftext[x=2.622141in,y=0.936643in,,top]{\color{textcolor}\sffamily\fontsize{10.000000}{12.000000}\selectfont 0.0}%
\end{pgfscope}%
\begin{pgfscope}%
\pgfsetrectcap%
\pgfsetroundjoin%
\pgfsetlinewidth{0.803000pt}%
\definecolor{currentstroke}{rgb}{0.000000,0.000000,0.000000}%
\pgfsetstrokecolor{currentstroke}%
\pgfsetdash{}{0pt}%
\pgfpathmoveto{\pgfqpoint{3.185917in}{1.027280in}}%
\pgfpathlineto{\pgfqpoint{3.140542in}{0.985410in}}%
\pgfusepath{stroke}%
\end{pgfscope}%
\begin{pgfscope}%
\definecolor{textcolor}{rgb}{0.000000,0.000000,0.000000}%
\pgfsetstrokecolor{textcolor}%
\pgfsetfillcolor{textcolor}%
\pgftext[x=3.067323in,y=0.789279in,,top]{\color{textcolor}\sffamily\fontsize{10.000000}{12.000000}\selectfont 0.5}%
\end{pgfscope}%
\begin{pgfscope}%
\pgfsetrectcap%
\pgfsetroundjoin%
\pgfsetlinewidth{0.803000pt}%
\definecolor{currentstroke}{rgb}{0.000000,0.000000,0.000000}%
\pgfsetstrokecolor{currentstroke}%
\pgfsetdash{}{0pt}%
\pgfpathmoveto{\pgfqpoint{3.637421in}{0.879965in}}%
\pgfpathlineto{\pgfqpoint{3.592693in}{0.837445in}}%
\pgfusepath{stroke}%
\end{pgfscope}%
\begin{pgfscope}%
\definecolor{textcolor}{rgb}{0.000000,0.000000,0.000000}%
\pgfsetstrokecolor{textcolor}%
\pgfsetfillcolor{textcolor}%
\pgftext[x=3.519405in,y=0.639631in,,top]{\color{textcolor}\sffamily\fontsize{10.000000}{12.000000}\selectfont 1.0}%
\end{pgfscope}%
\begin{pgfscope}%
\pgfsetrectcap%
\pgfsetroundjoin%
\pgfsetlinewidth{0.803000pt}%
\definecolor{currentstroke}{rgb}{0.000000,0.000000,0.000000}%
\pgfsetstrokecolor{currentstroke}%
\pgfsetdash{}{0pt}%
\pgfpathmoveto{\pgfqpoint{4.095953in}{0.730356in}}%
\pgfpathlineto{\pgfqpoint{4.051898in}{0.687171in}}%
\pgfusepath{stroke}%
\end{pgfscope}%
\begin{pgfscope}%
\definecolor{textcolor}{rgb}{0.000000,0.000000,0.000000}%
\pgfsetstrokecolor{textcolor}%
\pgfsetfillcolor{textcolor}%
\pgftext[x=3.978547in,y=0.487646in,,top]{\color{textcolor}\sffamily\fontsize{10.000000}{12.000000}\selectfont 1.5}%
\end{pgfscope}%
\begin{pgfscope}%
\pgfsetrectcap%
\pgfsetroundjoin%
\pgfsetlinewidth{0.803000pt}%
\definecolor{currentstroke}{rgb}{0.000000,0.000000,0.000000}%
\pgfsetstrokecolor{currentstroke}%
\pgfsetdash{}{0pt}%
\pgfpathmoveto{\pgfqpoint{4.561678in}{0.578401in}}%
\pgfpathlineto{\pgfqpoint{4.518323in}{0.534535in}}%
\pgfusepath{stroke}%
\end{pgfscope}%
\begin{pgfscope}%
\definecolor{textcolor}{rgb}{0.000000,0.000000,0.000000}%
\pgfsetstrokecolor{textcolor}%
\pgfsetfillcolor{textcolor}%
\pgftext[x=4.444916in,y=0.333269in,,top]{\color{textcolor}\sffamily\fontsize{10.000000}{12.000000}\selectfont 2.0}%
\end{pgfscope}%
\begin{pgfscope}%
\pgfsetrectcap%
\pgfsetroundjoin%
\pgfsetlinewidth{0.803000pt}%
\definecolor{currentstroke}{rgb}{0.000000,0.000000,0.000000}%
\pgfsetstrokecolor{currentstroke}%
\pgfsetdash{}{0pt}%
\pgfpathmoveto{\pgfqpoint{6.391709in}{2.177762in}}%
\pgfpathlineto{\pgfqpoint{4.753413in}{0.496467in}}%
\pgfusepath{stroke}%
\end{pgfscope}%
\begin{pgfscope}%
\definecolor{textcolor}{rgb}{0.000000,0.000000,0.000000}%
\pgfsetstrokecolor{textcolor}%
\pgfsetfillcolor{textcolor}%
\pgftext[x=5.983676in,y=0.985873in,,]{\color{textcolor}\sffamily\fontsize{10.000000}{12.000000}\selectfont y}%
\end{pgfscope}%
\begin{pgfscope}%
\pgfsetbuttcap%
\pgfsetroundjoin%
\pgfsetlinewidth{0.803000pt}%
\definecolor{currentstroke}{rgb}{0.690196,0.690196,0.690196}%
\pgfsetstrokecolor{currentstroke}%
\pgfsetdash{}{0pt}%
\pgfpathmoveto{\pgfqpoint{1.688926in}{3.924526in}}%
\pgfpathlineto{\pgfqpoint{1.794447in}{1.608387in}}%
\pgfpathlineto{\pgfqpoint{4.866770in}{0.612800in}}%
\pgfusepath{stroke}%
\end{pgfscope}%
\begin{pgfscope}%
\pgfsetbuttcap%
\pgfsetroundjoin%
\pgfsetlinewidth{0.803000pt}%
\definecolor{currentstroke}{rgb}{0.690196,0.690196,0.690196}%
\pgfsetstrokecolor{currentstroke}%
\pgfsetdash{}{0pt}%
\pgfpathmoveto{\pgfqpoint{1.910570in}{4.086273in}}%
\pgfpathlineto{\pgfqpoint{2.005371in}{1.785188in}}%
\pgfpathlineto{\pgfqpoint{5.057999in}{0.809047in}}%
\pgfusepath{stroke}%
\end{pgfscope}%
\begin{pgfscope}%
\pgfsetbuttcap%
\pgfsetroundjoin%
\pgfsetlinewidth{0.803000pt}%
\definecolor{currentstroke}{rgb}{0.690196,0.690196,0.690196}%
\pgfsetstrokecolor{currentstroke}%
\pgfsetdash{}{0pt}%
\pgfpathmoveto{\pgfqpoint{2.127827in}{4.244817in}}%
\pgfpathlineto{\pgfqpoint{2.212301in}{1.958641in}}%
\pgfpathlineto{\pgfqpoint{5.245415in}{1.001383in}}%
\pgfusepath{stroke}%
\end{pgfscope}%
\begin{pgfscope}%
\pgfsetbuttcap%
\pgfsetroundjoin%
\pgfsetlinewidth{0.803000pt}%
\definecolor{currentstroke}{rgb}{0.690196,0.690196,0.690196}%
\pgfsetstrokecolor{currentstroke}%
\pgfsetdash{}{0pt}%
\pgfpathmoveto{\pgfqpoint{2.340825in}{4.400253in}}%
\pgfpathlineto{\pgfqpoint{2.415349in}{2.128839in}}%
\pgfpathlineto{\pgfqpoint{5.429132in}{1.189921in}}%
\pgfusepath{stroke}%
\end{pgfscope}%
\begin{pgfscope}%
\pgfsetbuttcap%
\pgfsetroundjoin%
\pgfsetlinewidth{0.803000pt}%
\definecolor{currentstroke}{rgb}{0.690196,0.690196,0.690196}%
\pgfsetstrokecolor{currentstroke}%
\pgfsetdash{}{0pt}%
\pgfpathmoveto{\pgfqpoint{2.549688in}{4.552671in}}%
\pgfpathlineto{\pgfqpoint{2.614623in}{2.295875in}}%
\pgfpathlineto{\pgfqpoint{5.609257in}{1.374773in}}%
\pgfusepath{stroke}%
\end{pgfscope}%
\begin{pgfscope}%
\pgfsetbuttcap%
\pgfsetroundjoin%
\pgfsetlinewidth{0.803000pt}%
\definecolor{currentstroke}{rgb}{0.690196,0.690196,0.690196}%
\pgfsetstrokecolor{currentstroke}%
\pgfsetdash{}{0pt}%
\pgfpathmoveto{\pgfqpoint{2.754535in}{4.702159in}}%
\pgfpathlineto{\pgfqpoint{2.810227in}{2.459834in}}%
\pgfpathlineto{\pgfqpoint{5.785895in}{1.556047in}}%
\pgfusepath{stroke}%
\end{pgfscope}%
\begin{pgfscope}%
\pgfsetbuttcap%
\pgfsetroundjoin%
\pgfsetlinewidth{0.803000pt}%
\definecolor{currentstroke}{rgb}{0.690196,0.690196,0.690196}%
\pgfsetstrokecolor{currentstroke}%
\pgfsetdash{}{0pt}%
\pgfpathmoveto{\pgfqpoint{2.955481in}{4.848801in}}%
\pgfpathlineto{\pgfqpoint{3.002262in}{2.620801in}}%
\pgfpathlineto{\pgfqpoint{5.959146in}{1.733846in}}%
\pgfusepath{stroke}%
\end{pgfscope}%
\begin{pgfscope}%
\pgfsetbuttcap%
\pgfsetroundjoin%
\pgfsetlinewidth{0.803000pt}%
\definecolor{currentstroke}{rgb}{0.690196,0.690196,0.690196}%
\pgfsetstrokecolor{currentstroke}%
\pgfsetdash{}{0pt}%
\pgfpathmoveto{\pgfqpoint{3.152636in}{4.992676in}}%
\pgfpathlineto{\pgfqpoint{3.190824in}{2.778858in}}%
\pgfpathlineto{\pgfqpoint{6.129107in}{1.908268in}}%
\pgfusepath{stroke}%
\end{pgfscope}%
\begin{pgfscope}%
\pgfsetbuttcap%
\pgfsetroundjoin%
\pgfsetlinewidth{0.803000pt}%
\definecolor{currentstroke}{rgb}{0.690196,0.690196,0.690196}%
\pgfsetstrokecolor{currentstroke}%
\pgfsetdash{}{0pt}%
\pgfpathmoveto{\pgfqpoint{3.346107in}{5.133862in}}%
\pgfpathlineto{\pgfqpoint{3.376007in}{2.934082in}}%
\pgfpathlineto{\pgfqpoint{6.295871in}{2.079408in}}%
\pgfusepath{stroke}%
\end{pgfscope}%
\begin{pgfscope}%
\pgfsetrectcap%
\pgfsetroundjoin%
\pgfsetlinewidth{0.803000pt}%
\definecolor{currentstroke}{rgb}{0.000000,0.000000,0.000000}%
\pgfsetstrokecolor{currentstroke}%
\pgfsetdash{}{0pt}%
\pgfpathmoveto{\pgfqpoint{4.840880in}{0.621189in}}%
\pgfpathlineto{\pgfqpoint{4.918618in}{0.595998in}}%
\pgfusepath{stroke}%
\end{pgfscope}%
\begin{pgfscope}%
\definecolor{textcolor}{rgb}{0.000000,0.000000,0.000000}%
\pgfsetstrokecolor{textcolor}%
\pgfsetfillcolor{textcolor}%
\pgftext[x=5.045633in,y=0.426401in,,top]{\color{textcolor}\sffamily\fontsize{10.000000}{12.000000}\selectfont \ensuremath{-}1.00}%
\end{pgfscope}%
\begin{pgfscope}%
\pgfsetrectcap%
\pgfsetroundjoin%
\pgfsetlinewidth{0.803000pt}%
\definecolor{currentstroke}{rgb}{0.000000,0.000000,0.000000}%
\pgfsetstrokecolor{currentstroke}%
\pgfsetdash{}{0pt}%
\pgfpathmoveto{\pgfqpoint{5.032288in}{0.817269in}}%
\pgfpathlineto{\pgfqpoint{5.109488in}{0.792583in}}%
\pgfusepath{stroke}%
\end{pgfscope}%
\begin{pgfscope}%
\definecolor{textcolor}{rgb}{0.000000,0.000000,0.000000}%
\pgfsetstrokecolor{textcolor}%
\pgfsetfillcolor{textcolor}%
\pgftext[x=5.235136in,y=0.624605in,,top]{\color{textcolor}\sffamily\fontsize{10.000000}{12.000000}\selectfont \ensuremath{-}0.75}%
\end{pgfscope}%
\begin{pgfscope}%
\pgfsetrectcap%
\pgfsetroundjoin%
\pgfsetlinewidth{0.803000pt}%
\definecolor{currentstroke}{rgb}{0.000000,0.000000,0.000000}%
\pgfsetstrokecolor{currentstroke}%
\pgfsetdash{}{0pt}%
\pgfpathmoveto{\pgfqpoint{5.219881in}{1.009441in}}%
\pgfpathlineto{\pgfqpoint{5.296549in}{0.985245in}}%
\pgfusepath{stroke}%
\end{pgfscope}%
\begin{pgfscope}%
\definecolor{textcolor}{rgb}{0.000000,0.000000,0.000000}%
\pgfsetstrokecolor{textcolor}%
\pgfsetfillcolor{textcolor}%
\pgftext[x=5.420860in,y=0.818856in,,top]{\color{textcolor}\sffamily\fontsize{10.000000}{12.000000}\selectfont \ensuremath{-}0.50}%
\end{pgfscope}%
\begin{pgfscope}%
\pgfsetrectcap%
\pgfsetroundjoin%
\pgfsetlinewidth{0.803000pt}%
\definecolor{currentstroke}{rgb}{0.000000,0.000000,0.000000}%
\pgfsetstrokecolor{currentstroke}%
\pgfsetdash{}{0pt}%
\pgfpathmoveto{\pgfqpoint{5.403772in}{1.197821in}}%
\pgfpathlineto{\pgfqpoint{5.479914in}{1.174100in}}%
\pgfusepath{stroke}%
\end{pgfscope}%
\begin{pgfscope}%
\definecolor{textcolor}{rgb}{0.000000,0.000000,0.000000}%
\pgfsetstrokecolor{textcolor}%
\pgfsetfillcolor{textcolor}%
\pgftext[x=5.602916in,y=1.009270in,,top]{\color{textcolor}\sffamily\fontsize{10.000000}{12.000000}\selectfont \ensuremath{-}0.25}%
\end{pgfscope}%
\begin{pgfscope}%
\pgfsetrectcap%
\pgfsetroundjoin%
\pgfsetlinewidth{0.803000pt}%
\definecolor{currentstroke}{rgb}{0.000000,0.000000,0.000000}%
\pgfsetstrokecolor{currentstroke}%
\pgfsetdash{}{0pt}%
\pgfpathmoveto{\pgfqpoint{5.584071in}{1.382520in}}%
\pgfpathlineto{\pgfqpoint{5.659691in}{1.359261in}}%
\pgfusepath{stroke}%
\end{pgfscope}%
\begin{pgfscope}%
\definecolor{textcolor}{rgb}{0.000000,0.000000,0.000000}%
\pgfsetstrokecolor{textcolor}%
\pgfsetfillcolor{textcolor}%
\pgftext[x=5.781411in,y=1.195961in,,top]{\color{textcolor}\sffamily\fontsize{10.000000}{12.000000}\selectfont 0.00}%
\end{pgfscope}%
\begin{pgfscope}%
\pgfsetrectcap%
\pgfsetroundjoin%
\pgfsetlinewidth{0.803000pt}%
\definecolor{currentstroke}{rgb}{0.000000,0.000000,0.000000}%
\pgfsetstrokecolor{currentstroke}%
\pgfsetdash{}{0pt}%
\pgfpathmoveto{\pgfqpoint{5.760880in}{1.563645in}}%
\pgfpathlineto{\pgfqpoint{5.835985in}{1.540834in}}%
\pgfusepath{stroke}%
\end{pgfscope}%
\begin{pgfscope}%
\definecolor{textcolor}{rgb}{0.000000,0.000000,0.000000}%
\pgfsetstrokecolor{textcolor}%
\pgfsetfillcolor{textcolor}%
\pgftext[x=5.956450in,y=1.379036in,,top]{\color{textcolor}\sffamily\fontsize{10.000000}{12.000000}\selectfont 0.25}%
\end{pgfscope}%
\begin{pgfscope}%
\pgfsetrectcap%
\pgfsetroundjoin%
\pgfsetlinewidth{0.803000pt}%
\definecolor{currentstroke}{rgb}{0.000000,0.000000,0.000000}%
\pgfsetstrokecolor{currentstroke}%
\pgfsetdash{}{0pt}%
\pgfpathmoveto{\pgfqpoint{5.934301in}{1.741299in}}%
\pgfpathlineto{\pgfqpoint{6.008897in}{1.718923in}}%
\pgfusepath{stroke}%
\end{pgfscope}%
\begin{pgfscope}%
\definecolor{textcolor}{rgb}{0.000000,0.000000,0.000000}%
\pgfsetstrokecolor{textcolor}%
\pgfsetfillcolor{textcolor}%
\pgftext[x=6.128132in,y=1.558599in,,top]{\color{textcolor}\sffamily\fontsize{10.000000}{12.000000}\selectfont 0.50}%
\end{pgfscope}%
\begin{pgfscope}%
\pgfsetrectcap%
\pgfsetroundjoin%
\pgfsetlinewidth{0.803000pt}%
\definecolor{currentstroke}{rgb}{0.000000,0.000000,0.000000}%
\pgfsetstrokecolor{currentstroke}%
\pgfsetdash{}{0pt}%
\pgfpathmoveto{\pgfqpoint{6.104430in}{1.915580in}}%
\pgfpathlineto{\pgfqpoint{6.178522in}{1.893627in}}%
\pgfusepath{stroke}%
\end{pgfscope}%
\begin{pgfscope}%
\definecolor{textcolor}{rgb}{0.000000,0.000000,0.000000}%
\pgfsetstrokecolor{textcolor}%
\pgfsetfillcolor{textcolor}%
\pgftext[x=6.296552in,y=1.734751in,,top]{\color{textcolor}\sffamily\fontsize{10.000000}{12.000000}\selectfont 0.75}%
\end{pgfscope}%
\begin{pgfscope}%
\pgfsetrectcap%
\pgfsetroundjoin%
\pgfsetlinewidth{0.803000pt}%
\definecolor{currentstroke}{rgb}{0.000000,0.000000,0.000000}%
\pgfsetstrokecolor{currentstroke}%
\pgfsetdash{}{0pt}%
\pgfpathmoveto{\pgfqpoint{6.271359in}{2.086583in}}%
\pgfpathlineto{\pgfqpoint{6.344953in}{2.065041in}}%
\pgfusepath{stroke}%
\end{pgfscope}%
\begin{pgfscope}%
\definecolor{textcolor}{rgb}{0.000000,0.000000,0.000000}%
\pgfsetstrokecolor{textcolor}%
\pgfsetfillcolor{textcolor}%
\pgftext[x=6.461802in,y=1.907589in,,top]{\color{textcolor}\sffamily\fontsize{10.000000}{12.000000}\selectfont 1.00}%
\end{pgfscope}%
\begin{pgfscope}%
\pgfsetrectcap%
\pgfsetroundjoin%
\pgfsetlinewidth{0.803000pt}%
\definecolor{currentstroke}{rgb}{0.000000,0.000000,0.000000}%
\pgfsetstrokecolor{currentstroke}%
\pgfsetdash{}{0pt}%
\pgfpathmoveto{\pgfqpoint{6.391709in}{2.177762in}}%
\pgfpathlineto{\pgfqpoint{6.495528in}{4.444907in}}%
\pgfusepath{stroke}%
\end{pgfscope}%
\begin{pgfscope}%
\definecolor{textcolor}{rgb}{0.000000,0.000000,0.000000}%
\pgfsetstrokecolor{textcolor}%
\pgfsetfillcolor{textcolor}%
\pgftext[x=7.004475in,y=3.361793in,,,rotate=87.378092]{\color{textcolor}\sffamily\fontsize{10.000000}{12.000000}\selectfont f(x,y)}%
\end{pgfscope}%
\begin{pgfscope}%
\pgfsetbuttcap%
\pgfsetroundjoin%
\pgfsetlinewidth{0.803000pt}%
\definecolor{currentstroke}{rgb}{0.690196,0.690196,0.690196}%
\pgfsetstrokecolor{currentstroke}%
\pgfsetdash{}{0pt}%
\pgfpathmoveto{\pgfqpoint{6.403535in}{2.436011in}}%
\pgfpathlineto{\pgfqpoint{3.479630in}{3.273470in}}%
\pgfpathlineto{\pgfqpoint{1.656782in}{1.768110in}}%
\pgfusepath{stroke}%
\end{pgfscope}%
\begin{pgfscope}%
\pgfsetbuttcap%
\pgfsetroundjoin%
\pgfsetlinewidth{0.803000pt}%
\definecolor{currentstroke}{rgb}{0.690196,0.690196,0.690196}%
\pgfsetstrokecolor{currentstroke}%
\pgfsetdash{}{0pt}%
\pgfpathmoveto{\pgfqpoint{6.416362in}{2.716121in}}%
\pgfpathlineto{\pgfqpoint{3.476512in}{3.544624in}}%
\pgfpathlineto{\pgfqpoint{1.642964in}{2.055053in}}%
\pgfusepath{stroke}%
\end{pgfscope}%
\begin{pgfscope}%
\pgfsetbuttcap%
\pgfsetroundjoin%
\pgfsetlinewidth{0.803000pt}%
\definecolor{currentstroke}{rgb}{0.690196,0.690196,0.690196}%
\pgfsetstrokecolor{currentstroke}%
\pgfsetdash{}{0pt}%
\pgfpathmoveto{\pgfqpoint{6.429331in}{2.999333in}}%
\pgfpathlineto{\pgfqpoint{3.473361in}{3.818637in}}%
\pgfpathlineto{\pgfqpoint{1.628986in}{2.345295in}}%
\pgfusepath{stroke}%
\end{pgfscope}%
\begin{pgfscope}%
\pgfsetbuttcap%
\pgfsetroundjoin%
\pgfsetlinewidth{0.803000pt}%
\definecolor{currentstroke}{rgb}{0.690196,0.690196,0.690196}%
\pgfsetstrokecolor{currentstroke}%
\pgfsetdash{}{0pt}%
\pgfpathmoveto{\pgfqpoint{6.442445in}{3.285700in}}%
\pgfpathlineto{\pgfqpoint{3.470177in}{4.095555in}}%
\pgfpathlineto{\pgfqpoint{1.614848in}{2.638894in}}%
\pgfusepath{stroke}%
\end{pgfscope}%
\begin{pgfscope}%
\pgfsetbuttcap%
\pgfsetroundjoin%
\pgfsetlinewidth{0.803000pt}%
\definecolor{currentstroke}{rgb}{0.690196,0.690196,0.690196}%
\pgfsetstrokecolor{currentstroke}%
\pgfsetdash{}{0pt}%
\pgfpathmoveto{\pgfqpoint{6.455705in}{3.575273in}}%
\pgfpathlineto{\pgfqpoint{3.466959in}{4.375424in}}%
\pgfpathlineto{\pgfqpoint{1.600545in}{2.935907in}}%
\pgfusepath{stroke}%
\end{pgfscope}%
\begin{pgfscope}%
\pgfsetbuttcap%
\pgfsetroundjoin%
\pgfsetlinewidth{0.803000pt}%
\definecolor{currentstroke}{rgb}{0.690196,0.690196,0.690196}%
\pgfsetstrokecolor{currentstroke}%
\pgfsetdash{}{0pt}%
\pgfpathmoveto{\pgfqpoint{6.469115in}{3.868107in}}%
\pgfpathlineto{\pgfqpoint{3.463706in}{4.658291in}}%
\pgfpathlineto{\pgfqpoint{1.586074in}{3.236396in}}%
\pgfusepath{stroke}%
\end{pgfscope}%
\begin{pgfscope}%
\pgfsetbuttcap%
\pgfsetroundjoin%
\pgfsetlinewidth{0.803000pt}%
\definecolor{currentstroke}{rgb}{0.690196,0.690196,0.690196}%
\pgfsetstrokecolor{currentstroke}%
\pgfsetdash{}{0pt}%
\pgfpathmoveto{\pgfqpoint{6.482676in}{4.164257in}}%
\pgfpathlineto{\pgfqpoint{3.460418in}{4.944204in}}%
\pgfpathlineto{\pgfqpoint{1.571434in}{3.540420in}}%
\pgfusepath{stroke}%
\end{pgfscope}%
\begin{pgfscope}%
\pgfsetrectcap%
\pgfsetroundjoin%
\pgfsetlinewidth{0.803000pt}%
\definecolor{currentstroke}{rgb}{0.000000,0.000000,0.000000}%
\pgfsetstrokecolor{currentstroke}%
\pgfsetdash{}{0pt}%
\pgfpathmoveto{\pgfqpoint{6.378989in}{2.443041in}}%
\pgfpathlineto{\pgfqpoint{6.452684in}{2.421934in}}%
\pgfusepath{stroke}%
\end{pgfscope}%
\begin{pgfscope}%
\definecolor{textcolor}{rgb}{0.000000,0.000000,0.000000}%
\pgfsetstrokecolor{textcolor}%
\pgfsetfillcolor{textcolor}%
\pgftext[x=6.658505in,y=2.471887in,,top]{\color{textcolor}\sffamily\fontsize{10.000000}{12.000000}\selectfont 2}%
\end{pgfscope}%
\begin{pgfscope}%
\pgfsetrectcap%
\pgfsetroundjoin%
\pgfsetlinewidth{0.803000pt}%
\definecolor{currentstroke}{rgb}{0.000000,0.000000,0.000000}%
\pgfsetstrokecolor{currentstroke}%
\pgfsetdash{}{0pt}%
\pgfpathmoveto{\pgfqpoint{6.391676in}{2.723078in}}%
\pgfpathlineto{\pgfqpoint{6.465793in}{2.702191in}}%
\pgfusepath{stroke}%
\end{pgfscope}%
\begin{pgfscope}%
\definecolor{textcolor}{rgb}{0.000000,0.000000,0.000000}%
\pgfsetstrokecolor{textcolor}%
\pgfsetfillcolor{textcolor}%
\pgftext[x=6.672707in,y=2.751622in,,top]{\color{textcolor}\sffamily\fontsize{10.000000}{12.000000}\selectfont 3}%
\end{pgfscope}%
\begin{pgfscope}%
\pgfsetrectcap%
\pgfsetroundjoin%
\pgfsetlinewidth{0.803000pt}%
\definecolor{currentstroke}{rgb}{0.000000,0.000000,0.000000}%
\pgfsetstrokecolor{currentstroke}%
\pgfsetdash{}{0pt}%
\pgfpathmoveto{\pgfqpoint{6.404503in}{3.006215in}}%
\pgfpathlineto{\pgfqpoint{6.479046in}{2.985554in}}%
\pgfusepath{stroke}%
\end{pgfscope}%
\begin{pgfscope}%
\definecolor{textcolor}{rgb}{0.000000,0.000000,0.000000}%
\pgfsetstrokecolor{textcolor}%
\pgfsetfillcolor{textcolor}%
\pgftext[x=6.687066in,y=3.034449in,,top]{\color{textcolor}\sffamily\fontsize{10.000000}{12.000000}\selectfont 4}%
\end{pgfscope}%
\begin{pgfscope}%
\pgfsetrectcap%
\pgfsetroundjoin%
\pgfsetlinewidth{0.803000pt}%
\definecolor{currentstroke}{rgb}{0.000000,0.000000,0.000000}%
\pgfsetstrokecolor{currentstroke}%
\pgfsetdash{}{0pt}%
\pgfpathmoveto{\pgfqpoint{6.417473in}{3.292503in}}%
\pgfpathlineto{\pgfqpoint{6.492447in}{3.272075in}}%
\pgfusepath{stroke}%
\end{pgfscope}%
\begin{pgfscope}%
\definecolor{textcolor}{rgb}{0.000000,0.000000,0.000000}%
\pgfsetstrokecolor{textcolor}%
\pgfsetfillcolor{textcolor}%
\pgftext[x=6.701584in,y=3.320419in,,top]{\color{textcolor}\sffamily\fontsize{10.000000}{12.000000}\selectfont 5}%
\end{pgfscope}%
\begin{pgfscope}%
\pgfsetrectcap%
\pgfsetroundjoin%
\pgfsetlinewidth{0.803000pt}%
\definecolor{currentstroke}{rgb}{0.000000,0.000000,0.000000}%
\pgfsetstrokecolor{currentstroke}%
\pgfsetdash{}{0pt}%
\pgfpathmoveto{\pgfqpoint{6.430589in}{3.581997in}}%
\pgfpathlineto{\pgfqpoint{6.505999in}{3.561808in}}%
\pgfusepath{stroke}%
\end{pgfscope}%
\begin{pgfscope}%
\definecolor{textcolor}{rgb}{0.000000,0.000000,0.000000}%
\pgfsetstrokecolor{textcolor}%
\pgfsetfillcolor{textcolor}%
\pgftext[x=6.716265in,y=3.609585in,,top]{\color{textcolor}\sffamily\fontsize{10.000000}{12.000000}\selectfont 6}%
\end{pgfscope}%
\begin{pgfscope}%
\pgfsetrectcap%
\pgfsetroundjoin%
\pgfsetlinewidth{0.803000pt}%
\definecolor{currentstroke}{rgb}{0.000000,0.000000,0.000000}%
\pgfsetstrokecolor{currentstroke}%
\pgfsetdash{}{0pt}%
\pgfpathmoveto{\pgfqpoint{6.443852in}{3.874749in}}%
\pgfpathlineto{\pgfqpoint{6.519703in}{3.854806in}}%
\pgfusepath{stroke}%
\end{pgfscope}%
\begin{pgfscope}%
\definecolor{textcolor}{rgb}{0.000000,0.000000,0.000000}%
\pgfsetstrokecolor{textcolor}%
\pgfsetfillcolor{textcolor}%
\pgftext[x=6.731111in,y=3.902001in,,top]{\color{textcolor}\sffamily\fontsize{10.000000}{12.000000}\selectfont 7}%
\end{pgfscope}%
\begin{pgfscope}%
\pgfsetrectcap%
\pgfsetroundjoin%
\pgfsetlinewidth{0.803000pt}%
\definecolor{currentstroke}{rgb}{0.000000,0.000000,0.000000}%
\pgfsetstrokecolor{currentstroke}%
\pgfsetdash{}{0pt}%
\pgfpathmoveto{\pgfqpoint{6.457265in}{4.170815in}}%
\pgfpathlineto{\pgfqpoint{6.533562in}{4.151125in}}%
\pgfusepath{stroke}%
\end{pgfscope}%
\begin{pgfscope}%
\definecolor{textcolor}{rgb}{0.000000,0.000000,0.000000}%
\pgfsetstrokecolor{textcolor}%
\pgfsetfillcolor{textcolor}%
\pgftext[x=6.746125in,y=4.197721in,,top]{\color{textcolor}\sffamily\fontsize{10.000000}{12.000000}\selectfont 8}%
\end{pgfscope}%
\begin{pgfscope}%
\pgfpathrectangle{\pgfqpoint{1.254980in}{0.150000in}}{\pgfqpoint{5.490039in}{5.490039in}}%
\pgfusepath{clip}%
\pgfsetrectcap%
\pgfsetroundjoin%
\pgfsetlinewidth{1.505625pt}%
\definecolor{currentstroke}{rgb}{1.000000,0.000000,0.000000}%
\pgfsetstrokecolor{currentstroke}%
\pgfsetdash{}{0pt}%
\pgfpathmoveto{\pgfqpoint{5.343452in}{3.025813in}}%
\pgfpathlineto{\pgfqpoint{4.212739in}{3.581910in}}%
\pgfpathlineto{\pgfqpoint{3.496955in}{1.581500in}}%
\pgfpathlineto{\pgfqpoint{3.306558in}{1.489184in}}%
\pgfpathlineto{\pgfqpoint{3.459568in}{1.545357in}}%
\pgfpathlineto{\pgfqpoint{3.373020in}{1.484904in}}%
\pgfpathlineto{\pgfqpoint{3.417461in}{1.502213in}}%
\pgfpathlineto{\pgfqpoint{3.395624in}{1.489273in}}%
\pgfpathlineto{\pgfqpoint{3.400795in}{1.491437in}}%
\pgfpathlineto{\pgfqpoint{3.397922in}{1.490657in}}%
\pgfpathlineto{\pgfqpoint{3.398522in}{1.490870in}}%
\pgfpathlineto{\pgfqpoint{3.398291in}{1.490592in}}%
\pgfusepath{stroke}%
\end{pgfscope}%
\begin{pgfscope}%
\pgfpathrectangle{\pgfqpoint{1.254980in}{0.150000in}}{\pgfqpoint{5.490039in}{5.490039in}}%
\pgfusepath{clip}%
\pgfsetbuttcap%
\pgfsetroundjoin%
\definecolor{currentfill}{rgb}{1.000000,0.000000,0.000000}%
\pgfsetfillcolor{currentfill}%
\pgfsetlinewidth{1.003750pt}%
\definecolor{currentstroke}{rgb}{1.000000,0.000000,0.000000}%
\pgfsetstrokecolor{currentstroke}%
\pgfsetdash{}{0pt}%
\pgfsys@defobject{currentmarker}{\pgfqpoint{-0.041667in}{-0.041667in}}{\pgfqpoint{0.041667in}{0.041667in}}{%
\pgfpathmoveto{\pgfqpoint{0.000000in}{-0.041667in}}%
\pgfpathcurveto{\pgfqpoint{0.011050in}{-0.041667in}}{\pgfqpoint{0.021649in}{-0.037276in}}{\pgfqpoint{0.029463in}{-0.029463in}}%
\pgfpathcurveto{\pgfqpoint{0.037276in}{-0.021649in}}{\pgfqpoint{0.041667in}{-0.011050in}}{\pgfqpoint{0.041667in}{0.000000in}}%
\pgfpathcurveto{\pgfqpoint{0.041667in}{0.011050in}}{\pgfqpoint{0.037276in}{0.021649in}}{\pgfqpoint{0.029463in}{0.029463in}}%
\pgfpathcurveto{\pgfqpoint{0.021649in}{0.037276in}}{\pgfqpoint{0.011050in}{0.041667in}}{\pgfqpoint{0.000000in}{0.041667in}}%
\pgfpathcurveto{\pgfqpoint{-0.011050in}{0.041667in}}{\pgfqpoint{-0.021649in}{0.037276in}}{\pgfqpoint{-0.029463in}{0.029463in}}%
\pgfpathcurveto{\pgfqpoint{-0.037276in}{0.021649in}}{\pgfqpoint{-0.041667in}{0.011050in}}{\pgfqpoint{-0.041667in}{0.000000in}}%
\pgfpathcurveto{\pgfqpoint{-0.041667in}{-0.011050in}}{\pgfqpoint{-0.037276in}{-0.021649in}}{\pgfqpoint{-0.029463in}{-0.029463in}}%
\pgfpathcurveto{\pgfqpoint{-0.021649in}{-0.037276in}}{\pgfqpoint{-0.011050in}{-0.041667in}}{\pgfqpoint{0.000000in}{-0.041667in}}%
\pgfpathlineto{\pgfqpoint{0.000000in}{-0.041667in}}%
\pgfpathclose%
\pgfusepath{stroke,fill}%
}%
\begin{pgfscope}%
\pgfsys@transformshift{5.343452in}{3.025813in}%
\pgfsys@useobject{currentmarker}{}%
\end{pgfscope}%
\begin{pgfscope}%
\pgfsys@transformshift{4.212739in}{3.581910in}%
\pgfsys@useobject{currentmarker}{}%
\end{pgfscope}%
\begin{pgfscope}%
\pgfsys@transformshift{3.496955in}{1.581500in}%
\pgfsys@useobject{currentmarker}{}%
\end{pgfscope}%
\begin{pgfscope}%
\pgfsys@transformshift{3.306558in}{1.489184in}%
\pgfsys@useobject{currentmarker}{}%
\end{pgfscope}%
\begin{pgfscope}%
\pgfsys@transformshift{3.459568in}{1.545357in}%
\pgfsys@useobject{currentmarker}{}%
\end{pgfscope}%
\begin{pgfscope}%
\pgfsys@transformshift{3.373020in}{1.484904in}%
\pgfsys@useobject{currentmarker}{}%
\end{pgfscope}%
\begin{pgfscope}%
\pgfsys@transformshift{3.417461in}{1.502213in}%
\pgfsys@useobject{currentmarker}{}%
\end{pgfscope}%
\begin{pgfscope}%
\pgfsys@transformshift{3.395624in}{1.489273in}%
\pgfsys@useobject{currentmarker}{}%
\end{pgfscope}%
\begin{pgfscope}%
\pgfsys@transformshift{3.400795in}{1.491437in}%
\pgfsys@useobject{currentmarker}{}%
\end{pgfscope}%
\begin{pgfscope}%
\pgfsys@transformshift{3.397922in}{1.490657in}%
\pgfsys@useobject{currentmarker}{}%
\end{pgfscope}%
\begin{pgfscope}%
\pgfsys@transformshift{3.398522in}{1.490870in}%
\pgfsys@useobject{currentmarker}{}%
\end{pgfscope}%
\begin{pgfscope}%
\pgfsys@transformshift{3.398291in}{1.490592in}%
\pgfsys@useobject{currentmarker}{}%
\end{pgfscope}%
\end{pgfscope}%
\begin{pgfscope}%
\pgfpathrectangle{\pgfqpoint{1.254980in}{0.150000in}}{\pgfqpoint{5.490039in}{5.490039in}}%
\pgfusepath{clip}%
\pgfsetbuttcap%
\pgfsetroundjoin%
\definecolor{currentfill}{rgb}{0.000000,0.000000,1.000000}%
\pgfsetfillcolor{currentfill}%
\pgfsetlinewidth{1.003750pt}%
\definecolor{currentstroke}{rgb}{0.000000,0.000000,1.000000}%
\pgfsetstrokecolor{currentstroke}%
\pgfsetdash{}{0pt}%
\pgfsys@defobject{currentmarker}{\pgfqpoint{-0.069444in}{-0.069444in}}{\pgfqpoint{0.069444in}{0.069444in}}{%
\pgfpathmoveto{\pgfqpoint{0.000000in}{-0.069444in}}%
\pgfpathcurveto{\pgfqpoint{0.018417in}{-0.069444in}}{\pgfqpoint{0.036082in}{-0.062127in}}{\pgfqpoint{0.049105in}{-0.049105in}}%
\pgfpathcurveto{\pgfqpoint{0.062127in}{-0.036082in}}{\pgfqpoint{0.069444in}{-0.018417in}}{\pgfqpoint{0.069444in}{0.000000in}}%
\pgfpathcurveto{\pgfqpoint{0.069444in}{0.018417in}}{\pgfqpoint{0.062127in}{0.036082in}}{\pgfqpoint{0.049105in}{0.049105in}}%
\pgfpathcurveto{\pgfqpoint{0.036082in}{0.062127in}}{\pgfqpoint{0.018417in}{0.069444in}}{\pgfqpoint{0.000000in}{0.069444in}}%
\pgfpathcurveto{\pgfqpoint{-0.018417in}{0.069444in}}{\pgfqpoint{-0.036082in}{0.062127in}}{\pgfqpoint{-0.049105in}{0.049105in}}%
\pgfpathcurveto{\pgfqpoint{-0.062127in}{0.036082in}}{\pgfqpoint{-0.069444in}{0.018417in}}{\pgfqpoint{-0.069444in}{0.000000in}}%
\pgfpathcurveto{\pgfqpoint{-0.069444in}{-0.018417in}}{\pgfqpoint{-0.062127in}{-0.036082in}}{\pgfqpoint{-0.049105in}{-0.049105in}}%
\pgfpathcurveto{\pgfqpoint{-0.036082in}{-0.062127in}}{\pgfqpoint{-0.018417in}{-0.069444in}}{\pgfqpoint{0.000000in}{-0.069444in}}%
\pgfpathlineto{\pgfqpoint{0.000000in}{-0.069444in}}%
\pgfpathclose%
\pgfusepath{stroke,fill}%
}%
\begin{pgfscope}%
\pgfsys@transformshift{3.398291in}{1.490592in}%
\pgfsys@useobject{currentmarker}{}%
\end{pgfscope}%
\end{pgfscope}%
\begin{pgfscope}%
\definecolor{textcolor}{rgb}{0.000000,0.000000,0.000000}%
\pgfsetstrokecolor{textcolor}%
\pgfsetfillcolor{textcolor}%
\pgftext[x=4.000000in,y=5.723372in,,base]{\color{textcolor}\sffamily\fontsize{12.000000}{14.400000}\selectfont 3D Surface Plot}%
\end{pgfscope}%
\begin{pgfscope}%
\pgfpathrectangle{\pgfqpoint{1.254980in}{0.150000in}}{\pgfqpoint{5.490039in}{5.490039in}}%
\pgfusepath{clip}%
\pgfsetbuttcap%
\pgfsetroundjoin%
\definecolor{currentfill}{rgb}{0.124395,0.578002,0.548287}%
\pgfsetfillcolor{currentfill}%
\pgfsetfillopacity{0.700000}%
\pgfsetlinewidth{0.000000pt}%
\definecolor{currentstroke}{rgb}{0.000000,0.000000,0.000000}%
\pgfsetstrokecolor{currentstroke}%
\pgfsetdash{}{0pt}%
\pgfpathmoveto{\pgfqpoint{4.098565in}{3.716184in}}%
\pgfpathlineto{\pgfqpoint{4.111361in}{3.701701in}}%
\pgfpathlineto{\pgfqpoint{4.124158in}{3.687428in}}%
\pgfpathlineto{\pgfqpoint{4.136955in}{3.673366in}}%
\pgfpathlineto{\pgfqpoint{4.149753in}{3.659511in}}%
\pgfpathlineto{\pgfqpoint{4.157140in}{3.684405in}}%
\pgfpathlineto{\pgfqpoint{4.164528in}{3.709697in}}%
\pgfpathlineto{\pgfqpoint{4.171916in}{3.735396in}}%
\pgfpathlineto{\pgfqpoint{4.179305in}{3.761509in}}%
\pgfpathlineto{\pgfqpoint{4.166508in}{3.776012in}}%
\pgfpathlineto{\pgfqpoint{4.153712in}{3.790724in}}%
\pgfpathlineto{\pgfqpoint{4.140915in}{3.805646in}}%
\pgfpathlineto{\pgfqpoint{4.128119in}{3.820781in}}%
\pgfpathlineto{\pgfqpoint{4.120730in}{3.794006in}}%
\pgfpathlineto{\pgfqpoint{4.113341in}{3.767653in}}%
\pgfpathlineto{\pgfqpoint{4.105953in}{3.741715in}}%
\pgfpathlineto{\pgfqpoint{4.098565in}{3.716184in}}%
\pgfpathclose%
\pgfusepath{fill}%
\end{pgfscope}%
\begin{pgfscope}%
\pgfpathrectangle{\pgfqpoint{1.254980in}{0.150000in}}{\pgfqpoint{5.490039in}{5.490039in}}%
\pgfusepath{clip}%
\pgfsetbuttcap%
\pgfsetroundjoin%
\definecolor{currentfill}{rgb}{0.128729,0.563265,0.551229}%
\pgfsetfillcolor{currentfill}%
\pgfsetfillopacity{0.700000}%
\pgfsetlinewidth{0.000000pt}%
\definecolor{currentstroke}{rgb}{0.000000,0.000000,0.000000}%
\pgfsetstrokecolor{currentstroke}%
\pgfsetdash{}{0pt}%
\pgfpathmoveto{\pgfqpoint{4.017826in}{3.675567in}}%
\pgfpathlineto{\pgfqpoint{4.030623in}{3.660846in}}%
\pgfpathlineto{\pgfqpoint{4.043420in}{3.646341in}}%
\pgfpathlineto{\pgfqpoint{4.056216in}{3.632051in}}%
\pgfpathlineto{\pgfqpoint{4.069013in}{3.617974in}}%
\pgfpathlineto{\pgfqpoint{4.076402in}{3.641955in}}%
\pgfpathlineto{\pgfqpoint{4.083790in}{3.666312in}}%
\pgfpathlineto{\pgfqpoint{4.091177in}{3.691052in}}%
\pgfpathlineto{\pgfqpoint{4.098565in}{3.716184in}}%
\pgfpathlineto{\pgfqpoint{4.085769in}{3.730881in}}%
\pgfpathlineto{\pgfqpoint{4.072973in}{3.745791in}}%
\pgfpathlineto{\pgfqpoint{4.060177in}{3.760917in}}%
\pgfpathlineto{\pgfqpoint{4.047380in}{3.776261in}}%
\pgfpathlineto{\pgfqpoint{4.039992in}{3.750496in}}%
\pgfpathlineto{\pgfqpoint{4.032604in}{3.725130in}}%
\pgfpathlineto{\pgfqpoint{4.025215in}{3.700157in}}%
\pgfpathlineto{\pgfqpoint{4.017826in}{3.675567in}}%
\pgfpathclose%
\pgfusepath{fill}%
\end{pgfscope}%
\begin{pgfscope}%
\pgfpathrectangle{\pgfqpoint{1.254980in}{0.150000in}}{\pgfqpoint{5.490039in}{5.490039in}}%
\pgfusepath{clip}%
\pgfsetbuttcap%
\pgfsetroundjoin%
\definecolor{currentfill}{rgb}{0.119738,0.603785,0.541400}%
\pgfsetfillcolor{currentfill}%
\pgfsetfillopacity{0.700000}%
\pgfsetlinewidth{0.000000pt}%
\definecolor{currentstroke}{rgb}{0.000000,0.000000,0.000000}%
\pgfsetstrokecolor{currentstroke}%
\pgfsetdash{}{0pt}%
\pgfpathmoveto{\pgfqpoint{4.047380in}{3.776261in}}%
\pgfpathlineto{\pgfqpoint{4.060177in}{3.760917in}}%
\pgfpathlineto{\pgfqpoint{4.072973in}{3.745791in}}%
\pgfpathlineto{\pgfqpoint{4.085769in}{3.730881in}}%
\pgfpathlineto{\pgfqpoint{4.098565in}{3.716184in}}%
\pgfpathlineto{\pgfqpoint{4.105953in}{3.741715in}}%
\pgfpathlineto{\pgfqpoint{4.113341in}{3.767653in}}%
\pgfpathlineto{\pgfqpoint{4.120730in}{3.794006in}}%
\pgfpathlineto{\pgfqpoint{4.128119in}{3.820781in}}%
\pgfpathlineto{\pgfqpoint{4.115323in}{3.836129in}}%
\pgfpathlineto{\pgfqpoint{4.102526in}{3.851691in}}%
\pgfpathlineto{\pgfqpoint{4.089729in}{3.867471in}}%
\pgfpathlineto{\pgfqpoint{4.076932in}{3.883468in}}%
\pgfpathlineto{\pgfqpoint{4.069544in}{3.856028in}}%
\pgfpathlineto{\pgfqpoint{4.062156in}{3.829019in}}%
\pgfpathlineto{\pgfqpoint{4.054768in}{3.802432in}}%
\pgfpathlineto{\pgfqpoint{4.047380in}{3.776261in}}%
\pgfpathclose%
\pgfusepath{fill}%
\end{pgfscope}%
\begin{pgfscope}%
\pgfpathrectangle{\pgfqpoint{1.254980in}{0.150000in}}{\pgfqpoint{5.490039in}{5.490039in}}%
\pgfusepath{clip}%
\pgfsetbuttcap%
\pgfsetroundjoin%
\definecolor{currentfill}{rgb}{0.122606,0.585371,0.546557}%
\pgfsetfillcolor{currentfill}%
\pgfsetfillopacity{0.700000}%
\pgfsetlinewidth{0.000000pt}%
\definecolor{currentstroke}{rgb}{0.000000,0.000000,0.000000}%
\pgfsetstrokecolor{currentstroke}%
\pgfsetdash{}{0pt}%
\pgfpathmoveto{\pgfqpoint{3.966635in}{3.736646in}}%
\pgfpathlineto{\pgfqpoint{3.979434in}{3.721044in}}%
\pgfpathlineto{\pgfqpoint{3.992232in}{3.705665in}}%
\pgfpathlineto{\pgfqpoint{4.005029in}{3.690506in}}%
\pgfpathlineto{\pgfqpoint{4.017826in}{3.675567in}}%
\pgfpathlineto{\pgfqpoint{4.025215in}{3.700157in}}%
\pgfpathlineto{\pgfqpoint{4.032604in}{3.725130in}}%
\pgfpathlineto{\pgfqpoint{4.039992in}{3.750496in}}%
\pgfpathlineto{\pgfqpoint{4.047380in}{3.776261in}}%
\pgfpathlineto{\pgfqpoint{4.034583in}{3.791823in}}%
\pgfpathlineto{\pgfqpoint{4.021785in}{3.807605in}}%
\pgfpathlineto{\pgfqpoint{4.008987in}{3.823609in}}%
\pgfpathlineto{\pgfqpoint{3.996188in}{3.839836in}}%
\pgfpathlineto{\pgfqpoint{3.988800in}{3.813435in}}%
\pgfpathlineto{\pgfqpoint{3.981413in}{3.787442in}}%
\pgfpathlineto{\pgfqpoint{3.974024in}{3.761848in}}%
\pgfpathlineto{\pgfqpoint{3.966635in}{3.736646in}}%
\pgfpathclose%
\pgfusepath{fill}%
\end{pgfscope}%
\begin{pgfscope}%
\pgfpathrectangle{\pgfqpoint{1.254980in}{0.150000in}}{\pgfqpoint{5.490039in}{5.490039in}}%
\pgfusepath{clip}%
\pgfsetbuttcap%
\pgfsetroundjoin%
\definecolor{currentfill}{rgb}{0.136408,0.541173,0.554483}%
\pgfsetfillcolor{currentfill}%
\pgfsetfillopacity{0.700000}%
\pgfsetlinewidth{0.000000pt}%
\definecolor{currentstroke}{rgb}{0.000000,0.000000,0.000000}%
\pgfsetstrokecolor{currentstroke}%
\pgfsetdash{}{0pt}%
\pgfpathmoveto{\pgfqpoint{4.069013in}{3.617974in}}%
\pgfpathlineto{\pgfqpoint{4.081811in}{3.604110in}}%
\pgfpathlineto{\pgfqpoint{4.094608in}{3.590456in}}%
\pgfpathlineto{\pgfqpoint{4.107407in}{3.577011in}}%
\pgfpathlineto{\pgfqpoint{4.120206in}{3.563774in}}%
\pgfpathlineto{\pgfqpoint{4.127593in}{3.587148in}}%
\pgfpathlineto{\pgfqpoint{4.134980in}{3.610890in}}%
\pgfpathlineto{\pgfqpoint{4.142366in}{3.635009in}}%
\pgfpathlineto{\pgfqpoint{4.149753in}{3.659511in}}%
\pgfpathlineto{\pgfqpoint{4.136955in}{3.673366in}}%
\pgfpathlineto{\pgfqpoint{4.124158in}{3.687428in}}%
\pgfpathlineto{\pgfqpoint{4.111361in}{3.701701in}}%
\pgfpathlineto{\pgfqpoint{4.098565in}{3.716184in}}%
\pgfpathlineto{\pgfqpoint{4.091177in}{3.691052in}}%
\pgfpathlineto{\pgfqpoint{4.083790in}{3.666312in}}%
\pgfpathlineto{\pgfqpoint{4.076402in}{3.641955in}}%
\pgfpathlineto{\pgfqpoint{4.069013in}{3.617974in}}%
\pgfpathclose%
\pgfusepath{fill}%
\end{pgfscope}%
\begin{pgfscope}%
\pgfpathrectangle{\pgfqpoint{1.254980in}{0.150000in}}{\pgfqpoint{5.490039in}{5.490039in}}%
\pgfusepath{clip}%
\pgfsetbuttcap%
\pgfsetroundjoin%
\definecolor{currentfill}{rgb}{0.131172,0.555899,0.552459}%
\pgfsetfillcolor{currentfill}%
\pgfsetfillopacity{0.700000}%
\pgfsetlinewidth{0.000000pt}%
\definecolor{currentstroke}{rgb}{0.000000,0.000000,0.000000}%
\pgfsetstrokecolor{currentstroke}%
\pgfsetdash{}{0pt}%
\pgfpathmoveto{\pgfqpoint{4.149753in}{3.659511in}}%
\pgfpathlineto{\pgfqpoint{4.162552in}{3.645864in}}%
\pgfpathlineto{\pgfqpoint{4.175352in}{3.632423in}}%
\pgfpathlineto{\pgfqpoint{4.188152in}{3.619186in}}%
\pgfpathlineto{\pgfqpoint{4.200954in}{3.606152in}}%
\pgfpathlineto{\pgfqpoint{4.208340in}{3.630411in}}%
\pgfpathlineto{\pgfqpoint{4.215726in}{3.655060in}}%
\pgfpathlineto{\pgfqpoint{4.223113in}{3.680109in}}%
\pgfpathlineto{\pgfqpoint{4.230501in}{3.705564in}}%
\pgfpathlineto{\pgfqpoint{4.217701in}{3.719243in}}%
\pgfpathlineto{\pgfqpoint{4.204902in}{3.733126in}}%
\pgfpathlineto{\pgfqpoint{4.192103in}{3.747215in}}%
\pgfpathlineto{\pgfqpoint{4.179305in}{3.761509in}}%
\pgfpathlineto{\pgfqpoint{4.171916in}{3.735396in}}%
\pgfpathlineto{\pgfqpoint{4.164528in}{3.709697in}}%
\pgfpathlineto{\pgfqpoint{4.157140in}{3.684405in}}%
\pgfpathlineto{\pgfqpoint{4.149753in}{3.659511in}}%
\pgfpathclose%
\pgfusepath{fill}%
\end{pgfscope}%
\begin{pgfscope}%
\pgfpathrectangle{\pgfqpoint{1.254980in}{0.150000in}}{\pgfqpoint{5.490039in}{5.490039in}}%
\pgfusepath{clip}%
\pgfsetbuttcap%
\pgfsetroundjoin%
\definecolor{currentfill}{rgb}{0.119483,0.614817,0.537692}%
\pgfsetfillcolor{currentfill}%
\pgfsetfillopacity{0.700000}%
\pgfsetlinewidth{0.000000pt}%
\definecolor{currentstroke}{rgb}{0.000000,0.000000,0.000000}%
\pgfsetstrokecolor{currentstroke}%
\pgfsetdash{}{0pt}%
\pgfpathmoveto{\pgfqpoint{4.128119in}{3.820781in}}%
\pgfpathlineto{\pgfqpoint{4.140915in}{3.805646in}}%
\pgfpathlineto{\pgfqpoint{4.153712in}{3.790724in}}%
\pgfpathlineto{\pgfqpoint{4.166508in}{3.776012in}}%
\pgfpathlineto{\pgfqpoint{4.179305in}{3.761509in}}%
\pgfpathlineto{\pgfqpoint{4.186696in}{3.788045in}}%
\pgfpathlineto{\pgfqpoint{4.194087in}{3.815012in}}%
\pgfpathlineto{\pgfqpoint{4.201480in}{3.842418in}}%
\pgfpathlineto{\pgfqpoint{4.188683in}{3.857428in}}%
\pgfpathlineto{\pgfqpoint{4.175886in}{3.872648in}}%
\pgfpathlineto{\pgfqpoint{4.163089in}{3.888079in}}%
\pgfpathlineto{\pgfqpoint{4.150292in}{3.903723in}}%
\pgfpathlineto{\pgfqpoint{4.142900in}{3.875631in}}%
\pgfpathlineto{\pgfqpoint{4.135509in}{3.847986in}}%
\pgfpathlineto{\pgfqpoint{4.128119in}{3.820781in}}%
\pgfpathclose%
\pgfusepath{fill}%
\end{pgfscope}%
\begin{pgfscope}%
\pgfpathrectangle{\pgfqpoint{1.254980in}{0.150000in}}{\pgfqpoint{5.490039in}{5.490039in}}%
\pgfusepath{clip}%
\pgfsetbuttcap%
\pgfsetroundjoin%
\definecolor{currentfill}{rgb}{0.121148,0.592739,0.544641}%
\pgfsetfillcolor{currentfill}%
\pgfsetfillopacity{0.700000}%
\pgfsetlinewidth{0.000000pt}%
\definecolor{currentstroke}{rgb}{0.000000,0.000000,0.000000}%
\pgfsetstrokecolor{currentstroke}%
\pgfsetdash{}{0pt}%
\pgfpathmoveto{\pgfqpoint{4.179305in}{3.761509in}}%
\pgfpathlineto{\pgfqpoint{4.192103in}{3.747215in}}%
\pgfpathlineto{\pgfqpoint{4.204902in}{3.733126in}}%
\pgfpathlineto{\pgfqpoint{4.217701in}{3.719243in}}%
\pgfpathlineto{\pgfqpoint{4.230501in}{3.705564in}}%
\pgfpathlineto{\pgfqpoint{4.237891in}{3.731433in}}%
\pgfpathlineto{\pgfqpoint{4.245282in}{3.757725in}}%
\pgfpathlineto{\pgfqpoint{4.252674in}{3.784449in}}%
\pgfpathlineto{\pgfqpoint{4.239875in}{3.798633in}}%
\pgfpathlineto{\pgfqpoint{4.227076in}{3.813022in}}%
\pgfpathlineto{\pgfqpoint{4.214277in}{3.827616in}}%
\pgfpathlineto{\pgfqpoint{4.201480in}{3.842418in}}%
\pgfpathlineto{\pgfqpoint{4.194087in}{3.815012in}}%
\pgfpathlineto{\pgfqpoint{4.186696in}{3.788045in}}%
\pgfpathlineto{\pgfqpoint{4.179305in}{3.761509in}}%
\pgfpathclose%
\pgfusepath{fill}%
\end{pgfscope}%
\begin{pgfscope}%
\pgfpathrectangle{\pgfqpoint{1.254980in}{0.150000in}}{\pgfqpoint{5.490039in}{5.490039in}}%
\pgfusepath{clip}%
\pgfsetbuttcap%
\pgfsetroundjoin%
\definecolor{currentfill}{rgb}{0.120638,0.625828,0.533488}%
\pgfsetfillcolor{currentfill}%
\pgfsetfillopacity{0.700000}%
\pgfsetlinewidth{0.000000pt}%
\definecolor{currentstroke}{rgb}{0.000000,0.000000,0.000000}%
\pgfsetstrokecolor{currentstroke}%
\pgfsetdash{}{0pt}%
\pgfpathmoveto{\pgfqpoint{3.996188in}{3.839836in}}%
\pgfpathlineto{\pgfqpoint{4.008987in}{3.823609in}}%
\pgfpathlineto{\pgfqpoint{4.021785in}{3.807605in}}%
\pgfpathlineto{\pgfqpoint{4.034583in}{3.791823in}}%
\pgfpathlineto{\pgfqpoint{4.047380in}{3.776261in}}%
\pgfpathlineto{\pgfqpoint{4.054768in}{3.802432in}}%
\pgfpathlineto{\pgfqpoint{4.062156in}{3.829019in}}%
\pgfpathlineto{\pgfqpoint{4.069544in}{3.856028in}}%
\pgfpathlineto{\pgfqpoint{4.076932in}{3.883468in}}%
\pgfpathlineto{\pgfqpoint{4.064134in}{3.899685in}}%
\pgfpathlineto{\pgfqpoint{4.051336in}{3.916123in}}%
\pgfpathlineto{\pgfqpoint{4.038536in}{3.932784in}}%
\pgfpathlineto{\pgfqpoint{4.025735in}{3.949669in}}%
\pgfpathlineto{\pgfqpoint{4.018348in}{3.921561in}}%
\pgfpathlineto{\pgfqpoint{4.010962in}{3.893891in}}%
\pgfpathlineto{\pgfqpoint{4.003575in}{3.866652in}}%
\pgfpathlineto{\pgfqpoint{3.996188in}{3.839836in}}%
\pgfpathclose%
\pgfusepath{fill}%
\end{pgfscope}%
\begin{pgfscope}%
\pgfpathrectangle{\pgfqpoint{1.254980in}{0.150000in}}{\pgfqpoint{5.490039in}{5.490039in}}%
\pgfusepath{clip}%
\pgfsetbuttcap%
\pgfsetroundjoin%
\definecolor{currentfill}{rgb}{0.133743,0.548535,0.553541}%
\pgfsetfillcolor{currentfill}%
\pgfsetfillopacity{0.700000}%
\pgfsetlinewidth{0.000000pt}%
\definecolor{currentstroke}{rgb}{0.000000,0.000000,0.000000}%
\pgfsetstrokecolor{currentstroke}%
\pgfsetdash{}{0pt}%
\pgfpathmoveto{\pgfqpoint{3.937067in}{3.639615in}}%
\pgfpathlineto{\pgfqpoint{3.949866in}{3.624607in}}%
\pgfpathlineto{\pgfqpoint{3.962664in}{3.609820in}}%
\pgfpathlineto{\pgfqpoint{3.975463in}{3.595253in}}%
\pgfpathlineto{\pgfqpoint{3.988260in}{3.580906in}}%
\pgfpathlineto{\pgfqpoint{3.995654in}{3.604031in}}%
\pgfpathlineto{\pgfqpoint{4.003045in}{3.627511in}}%
\pgfpathlineto{\pgfqpoint{4.010436in}{3.651354in}}%
\pgfpathlineto{\pgfqpoint{4.017826in}{3.675567in}}%
\pgfpathlineto{\pgfqpoint{4.005029in}{3.690506in}}%
\pgfpathlineto{\pgfqpoint{3.992232in}{3.705665in}}%
\pgfpathlineto{\pgfqpoint{3.979434in}{3.721044in}}%
\pgfpathlineto{\pgfqpoint{3.966635in}{3.736646in}}%
\pgfpathlineto{\pgfqpoint{3.959245in}{3.711830in}}%
\pgfpathlineto{\pgfqpoint{3.951853in}{3.687390in}}%
\pgfpathlineto{\pgfqpoint{3.944461in}{3.663321in}}%
\pgfpathlineto{\pgfqpoint{3.937067in}{3.639615in}}%
\pgfpathclose%
\pgfusepath{fill}%
\end{pgfscope}%
\begin{pgfscope}%
\pgfpathrectangle{\pgfqpoint{1.254980in}{0.150000in}}{\pgfqpoint{5.490039in}{5.490039in}}%
\pgfusepath{clip}%
\pgfsetbuttcap%
\pgfsetroundjoin%
\definecolor{currentfill}{rgb}{0.124780,0.640461,0.527068}%
\pgfsetfillcolor{currentfill}%
\pgfsetfillopacity{0.700000}%
\pgfsetlinewidth{0.000000pt}%
\definecolor{currentstroke}{rgb}{0.000000,0.000000,0.000000}%
\pgfsetstrokecolor{currentstroke}%
\pgfsetdash{}{0pt}%
\pgfpathmoveto{\pgfqpoint{4.076932in}{3.883468in}}%
\pgfpathlineto{\pgfqpoint{4.089729in}{3.867471in}}%
\pgfpathlineto{\pgfqpoint{4.102526in}{3.851691in}}%
\pgfpathlineto{\pgfqpoint{4.115323in}{3.836129in}}%
\pgfpathlineto{\pgfqpoint{4.128119in}{3.820781in}}%
\pgfpathlineto{\pgfqpoint{4.135509in}{3.847986in}}%
\pgfpathlineto{\pgfqpoint{4.142900in}{3.875631in}}%
\pgfpathlineto{\pgfqpoint{4.150292in}{3.903723in}}%
\pgfpathlineto{\pgfqpoint{4.137495in}{3.919580in}}%
\pgfpathlineto{\pgfqpoint{4.124698in}{3.935654in}}%
\pgfpathlineto{\pgfqpoint{4.111901in}{3.951945in}}%
\pgfpathlineto{\pgfqpoint{4.099102in}{3.968454in}}%
\pgfpathlineto{\pgfqpoint{4.091711in}{3.939673in}}%
\pgfpathlineto{\pgfqpoint{4.084321in}{3.911347in}}%
\pgfpathlineto{\pgfqpoint{4.076932in}{3.883468in}}%
\pgfpathclose%
\pgfusepath{fill}%
\end{pgfscope}%
\begin{pgfscope}%
\pgfpathrectangle{\pgfqpoint{1.254980in}{0.150000in}}{\pgfqpoint{5.490039in}{5.490039in}}%
\pgfusepath{clip}%
\pgfsetbuttcap%
\pgfsetroundjoin%
\definecolor{currentfill}{rgb}{0.141935,0.526453,0.555991}%
\pgfsetfillcolor{currentfill}%
\pgfsetfillopacity{0.700000}%
\pgfsetlinewidth{0.000000pt}%
\definecolor{currentstroke}{rgb}{0.000000,0.000000,0.000000}%
\pgfsetstrokecolor{currentstroke}%
\pgfsetdash{}{0pt}%
\pgfpathmoveto{\pgfqpoint{3.988260in}{3.580906in}}%
\pgfpathlineto{\pgfqpoint{4.001058in}{3.566775in}}%
\pgfpathlineto{\pgfqpoint{4.013856in}{3.552861in}}%
\pgfpathlineto{\pgfqpoint{4.026654in}{3.539160in}}%
\pgfpathlineto{\pgfqpoint{4.039453in}{3.525672in}}%
\pgfpathlineto{\pgfqpoint{4.046844in}{3.548219in}}%
\pgfpathlineto{\pgfqpoint{4.054235in}{3.571114in}}%
\pgfpathlineto{\pgfqpoint{4.061624in}{3.594363in}}%
\pgfpathlineto{\pgfqpoint{4.069013in}{3.617974in}}%
\pgfpathlineto{\pgfqpoint{4.056216in}{3.632051in}}%
\pgfpathlineto{\pgfqpoint{4.043420in}{3.646341in}}%
\pgfpathlineto{\pgfqpoint{4.030623in}{3.660846in}}%
\pgfpathlineto{\pgfqpoint{4.017826in}{3.675567in}}%
\pgfpathlineto{\pgfqpoint{4.010436in}{3.651354in}}%
\pgfpathlineto{\pgfqpoint{4.003045in}{3.627511in}}%
\pgfpathlineto{\pgfqpoint{3.995654in}{3.604031in}}%
\pgfpathlineto{\pgfqpoint{3.988260in}{3.580906in}}%
\pgfpathclose%
\pgfusepath{fill}%
\end{pgfscope}%
\begin{pgfscope}%
\pgfpathrectangle{\pgfqpoint{1.254980in}{0.150000in}}{\pgfqpoint{5.490039in}{5.490039in}}%
\pgfusepath{clip}%
\pgfsetbuttcap%
\pgfsetroundjoin%
\definecolor{currentfill}{rgb}{0.119423,0.611141,0.538982}%
\pgfsetfillcolor{currentfill}%
\pgfsetfillopacity{0.700000}%
\pgfsetlinewidth{0.000000pt}%
\definecolor{currentstroke}{rgb}{0.000000,0.000000,0.000000}%
\pgfsetstrokecolor{currentstroke}%
\pgfsetdash{}{0pt}%
\pgfpathmoveto{\pgfqpoint{3.915429in}{3.801315in}}%
\pgfpathlineto{\pgfqpoint{3.928232in}{3.784806in}}%
\pgfpathlineto{\pgfqpoint{3.941034in}{3.768525in}}%
\pgfpathlineto{\pgfqpoint{3.953835in}{3.752473in}}%
\pgfpathlineto{\pgfqpoint{3.966635in}{3.736646in}}%
\pgfpathlineto{\pgfqpoint{3.974024in}{3.761848in}}%
\pgfpathlineto{\pgfqpoint{3.981413in}{3.787442in}}%
\pgfpathlineto{\pgfqpoint{3.988800in}{3.813435in}}%
\pgfpathlineto{\pgfqpoint{3.996188in}{3.839836in}}%
\pgfpathlineto{\pgfqpoint{3.983387in}{3.856289in}}%
\pgfpathlineto{\pgfqpoint{3.970586in}{3.872968in}}%
\pgfpathlineto{\pgfqpoint{3.957783in}{3.889876in}}%
\pgfpathlineto{\pgfqpoint{3.944978in}{3.907015in}}%
\pgfpathlineto{\pgfqpoint{3.937592in}{3.879974in}}%
\pgfpathlineto{\pgfqpoint{3.930205in}{3.853349in}}%
\pgfpathlineto{\pgfqpoint{3.922817in}{3.827132in}}%
\pgfpathlineto{\pgfqpoint{3.915429in}{3.801315in}}%
\pgfpathclose%
\pgfusepath{fill}%
\end{pgfscope}%
\begin{pgfscope}%
\pgfpathrectangle{\pgfqpoint{1.254980in}{0.150000in}}{\pgfqpoint{5.490039in}{5.490039in}}%
\pgfusepath{clip}%
\pgfsetbuttcap%
\pgfsetroundjoin%
\definecolor{currentfill}{rgb}{0.126453,0.570633,0.549841}%
\pgfsetfillcolor{currentfill}%
\pgfsetfillopacity{0.700000}%
\pgfsetlinewidth{0.000000pt}%
\definecolor{currentstroke}{rgb}{0.000000,0.000000,0.000000}%
\pgfsetstrokecolor{currentstroke}%
\pgfsetdash{}{0pt}%
\pgfpathmoveto{\pgfqpoint{4.230501in}{3.705564in}}%
\pgfpathlineto{\pgfqpoint{4.243303in}{3.692087in}}%
\pgfpathlineto{\pgfqpoint{4.256106in}{3.678811in}}%
\pgfpathlineto{\pgfqpoint{4.268911in}{3.665735in}}%
\pgfpathlineto{\pgfqpoint{4.281717in}{3.652859in}}%
\pgfpathlineto{\pgfqpoint{4.289105in}{3.678065in}}%
\pgfpathlineto{\pgfqpoint{4.296494in}{3.703685in}}%
\pgfpathlineto{\pgfqpoint{4.303886in}{3.729729in}}%
\pgfpathlineto{\pgfqpoint{4.291081in}{3.743109in}}%
\pgfpathlineto{\pgfqpoint{4.278277in}{3.756688in}}%
\pgfpathlineto{\pgfqpoint{4.265475in}{3.770467in}}%
\pgfpathlineto{\pgfqpoint{4.252674in}{3.784449in}}%
\pgfpathlineto{\pgfqpoint{4.245282in}{3.757725in}}%
\pgfpathlineto{\pgfqpoint{4.237891in}{3.731433in}}%
\pgfpathlineto{\pgfqpoint{4.230501in}{3.705564in}}%
\pgfpathclose%
\pgfusepath{fill}%
\end{pgfscope}%
\begin{pgfscope}%
\pgfpathrectangle{\pgfqpoint{1.254980in}{0.150000in}}{\pgfqpoint{5.490039in}{5.490039in}}%
\pgfusepath{clip}%
\pgfsetbuttcap%
\pgfsetroundjoin%
\definecolor{currentfill}{rgb}{0.144759,0.519093,0.556572}%
\pgfsetfillcolor{currentfill}%
\pgfsetfillopacity{0.700000}%
\pgfsetlinewidth{0.000000pt}%
\definecolor{currentstroke}{rgb}{0.000000,0.000000,0.000000}%
\pgfsetstrokecolor{currentstroke}%
\pgfsetdash{}{0pt}%
\pgfpathmoveto{\pgfqpoint{4.120206in}{3.563774in}}%
\pgfpathlineto{\pgfqpoint{4.133007in}{3.550743in}}%
\pgfpathlineto{\pgfqpoint{4.145808in}{3.537917in}}%
\pgfpathlineto{\pgfqpoint{4.158611in}{3.525296in}}%
\pgfpathlineto{\pgfqpoint{4.171415in}{3.512876in}}%
\pgfpathlineto{\pgfqpoint{4.178800in}{3.535646in}}%
\pgfpathlineto{\pgfqpoint{4.186184in}{3.558778in}}%
\pgfpathlineto{\pgfqpoint{4.193569in}{3.582277in}}%
\pgfpathlineto{\pgfqpoint{4.200954in}{3.606152in}}%
\pgfpathlineto{\pgfqpoint{4.188152in}{3.619186in}}%
\pgfpathlineto{\pgfqpoint{4.175352in}{3.632423in}}%
\pgfpathlineto{\pgfqpoint{4.162552in}{3.645864in}}%
\pgfpathlineto{\pgfqpoint{4.149753in}{3.659511in}}%
\pgfpathlineto{\pgfqpoint{4.142366in}{3.635009in}}%
\pgfpathlineto{\pgfqpoint{4.134980in}{3.610890in}}%
\pgfpathlineto{\pgfqpoint{4.127593in}{3.587148in}}%
\pgfpathlineto{\pgfqpoint{4.120206in}{3.563774in}}%
\pgfpathclose%
\pgfusepath{fill}%
\end{pgfscope}%
\begin{pgfscope}%
\pgfpathrectangle{\pgfqpoint{1.254980in}{0.150000in}}{\pgfqpoint{5.490039in}{5.490039in}}%
\pgfusepath{clip}%
\pgfsetbuttcap%
\pgfsetroundjoin%
\definecolor{currentfill}{rgb}{0.125394,0.574318,0.549086}%
\pgfsetfillcolor{currentfill}%
\pgfsetfillopacity{0.700000}%
\pgfsetlinewidth{0.000000pt}%
\definecolor{currentstroke}{rgb}{0.000000,0.000000,0.000000}%
\pgfsetstrokecolor{currentstroke}%
\pgfsetdash{}{0pt}%
\pgfpathmoveto{\pgfqpoint{3.885861in}{3.701902in}}%
\pgfpathlineto{\pgfqpoint{3.898664in}{3.685989in}}%
\pgfpathlineto{\pgfqpoint{3.911466in}{3.670305in}}%
\pgfpathlineto{\pgfqpoint{3.924267in}{3.654848in}}%
\pgfpathlineto{\pgfqpoint{3.937067in}{3.639615in}}%
\pgfpathlineto{\pgfqpoint{3.944461in}{3.663321in}}%
\pgfpathlineto{\pgfqpoint{3.951853in}{3.687390in}}%
\pgfpathlineto{\pgfqpoint{3.959245in}{3.711830in}}%
\pgfpathlineto{\pgfqpoint{3.966635in}{3.736646in}}%
\pgfpathlineto{\pgfqpoint{3.953835in}{3.752473in}}%
\pgfpathlineto{\pgfqpoint{3.941034in}{3.768525in}}%
\pgfpathlineto{\pgfqpoint{3.928232in}{3.784806in}}%
\pgfpathlineto{\pgfqpoint{3.915429in}{3.801315in}}%
\pgfpathlineto{\pgfqpoint{3.908039in}{3.775891in}}%
\pgfpathlineto{\pgfqpoint{3.900648in}{3.750852in}}%
\pgfpathlineto{\pgfqpoint{3.893255in}{3.726191in}}%
\pgfpathlineto{\pgfqpoint{3.885861in}{3.701902in}}%
\pgfpathclose%
\pgfusepath{fill}%
\end{pgfscope}%
\begin{pgfscope}%
\pgfpathrectangle{\pgfqpoint{1.254980in}{0.150000in}}{\pgfqpoint{5.490039in}{5.490039in}}%
\pgfusepath{clip}%
\pgfsetbuttcap%
\pgfsetroundjoin%
\definecolor{currentfill}{rgb}{0.137770,0.537492,0.554906}%
\pgfsetfillcolor{currentfill}%
\pgfsetfillopacity{0.700000}%
\pgfsetlinewidth{0.000000pt}%
\definecolor{currentstroke}{rgb}{0.000000,0.000000,0.000000}%
\pgfsetstrokecolor{currentstroke}%
\pgfsetdash{}{0pt}%
\pgfpathmoveto{\pgfqpoint{4.200954in}{3.606152in}}%
\pgfpathlineto{\pgfqpoint{4.213758in}{3.593320in}}%
\pgfpathlineto{\pgfqpoint{4.226563in}{3.580689in}}%
\pgfpathlineto{\pgfqpoint{4.239369in}{3.568257in}}%
\pgfpathlineto{\pgfqpoint{4.252178in}{3.556023in}}%
\pgfpathlineto{\pgfqpoint{4.259561in}{3.579649in}}%
\pgfpathlineto{\pgfqpoint{4.266945in}{3.603659in}}%
\pgfpathlineto{\pgfqpoint{4.274330in}{3.628059in}}%
\pgfpathlineto{\pgfqpoint{4.281717in}{3.652859in}}%
\pgfpathlineto{\pgfqpoint{4.268911in}{3.665735in}}%
\pgfpathlineto{\pgfqpoint{4.256106in}{3.678811in}}%
\pgfpathlineto{\pgfqpoint{4.243303in}{3.692087in}}%
\pgfpathlineto{\pgfqpoint{4.230501in}{3.705564in}}%
\pgfpathlineto{\pgfqpoint{4.223113in}{3.680109in}}%
\pgfpathlineto{\pgfqpoint{4.215726in}{3.655060in}}%
\pgfpathlineto{\pgfqpoint{4.208340in}{3.630411in}}%
\pgfpathlineto{\pgfqpoint{4.200954in}{3.606152in}}%
\pgfpathclose%
\pgfusepath{fill}%
\end{pgfscope}%
\begin{pgfscope}%
\pgfpathrectangle{\pgfqpoint{1.254980in}{0.150000in}}{\pgfqpoint{5.490039in}{5.490039in}}%
\pgfusepath{clip}%
\pgfsetbuttcap%
\pgfsetroundjoin%
\definecolor{currentfill}{rgb}{0.150476,0.504369,0.557430}%
\pgfsetfillcolor{currentfill}%
\pgfsetfillopacity{0.700000}%
\pgfsetlinewidth{0.000000pt}%
\definecolor{currentstroke}{rgb}{0.000000,0.000000,0.000000}%
\pgfsetstrokecolor{currentstroke}%
\pgfsetdash{}{0pt}%
\pgfpathmoveto{\pgfqpoint{4.039453in}{3.525672in}}%
\pgfpathlineto{\pgfqpoint{4.052252in}{3.512396in}}%
\pgfpathlineto{\pgfqpoint{4.065051in}{3.499329in}}%
\pgfpathlineto{\pgfqpoint{4.077852in}{3.486471in}}%
\pgfpathlineto{\pgfqpoint{4.090654in}{3.473820in}}%
\pgfpathlineto{\pgfqpoint{4.098043in}{3.495792in}}%
\pgfpathlineto{\pgfqpoint{4.105431in}{3.518103in}}%
\pgfpathlineto{\pgfqpoint{4.112819in}{3.540761in}}%
\pgfpathlineto{\pgfqpoint{4.120206in}{3.563774in}}%
\pgfpathlineto{\pgfqpoint{4.107407in}{3.577011in}}%
\pgfpathlineto{\pgfqpoint{4.094608in}{3.590456in}}%
\pgfpathlineto{\pgfqpoint{4.081811in}{3.604110in}}%
\pgfpathlineto{\pgfqpoint{4.069013in}{3.617974in}}%
\pgfpathlineto{\pgfqpoint{4.061624in}{3.594363in}}%
\pgfpathlineto{\pgfqpoint{4.054235in}{3.571114in}}%
\pgfpathlineto{\pgfqpoint{4.046844in}{3.548219in}}%
\pgfpathlineto{\pgfqpoint{4.039453in}{3.525672in}}%
\pgfpathclose%
\pgfusepath{fill}%
\end{pgfscope}%
\begin{pgfscope}%
\pgfpathrectangle{\pgfqpoint{1.254980in}{0.150000in}}{\pgfqpoint{5.490039in}{5.490039in}}%
\pgfusepath{clip}%
\pgfsetbuttcap%
\pgfsetroundjoin%
\definecolor{currentfill}{rgb}{0.132268,0.655014,0.519661}%
\pgfsetfillcolor{currentfill}%
\pgfsetfillopacity{0.700000}%
\pgfsetlinewidth{0.000000pt}%
\definecolor{currentstroke}{rgb}{0.000000,0.000000,0.000000}%
\pgfsetstrokecolor{currentstroke}%
\pgfsetdash{}{0pt}%
\pgfpathmoveto{\pgfqpoint{3.944978in}{3.907015in}}%
\pgfpathlineto{\pgfqpoint{3.957783in}{3.889876in}}%
\pgfpathlineto{\pgfqpoint{3.970586in}{3.872968in}}%
\pgfpathlineto{\pgfqpoint{3.983387in}{3.856289in}}%
\pgfpathlineto{\pgfqpoint{3.996188in}{3.839836in}}%
\pgfpathlineto{\pgfqpoint{4.003575in}{3.866652in}}%
\pgfpathlineto{\pgfqpoint{4.010962in}{3.893891in}}%
\pgfpathlineto{\pgfqpoint{4.018348in}{3.921561in}}%
\pgfpathlineto{\pgfqpoint{4.025735in}{3.949669in}}%
\pgfpathlineto{\pgfqpoint{4.012933in}{3.966781in}}%
\pgfpathlineto{\pgfqpoint{4.000130in}{3.984120in}}%
\pgfpathlineto{\pgfqpoint{3.987325in}{4.001689in}}%
\pgfpathlineto{\pgfqpoint{3.974518in}{4.019489in}}%
\pgfpathlineto{\pgfqpoint{3.967133in}{3.990708in}}%
\pgfpathlineto{\pgfqpoint{3.959748in}{3.962373in}}%
\pgfpathlineto{\pgfqpoint{3.952363in}{3.934479in}}%
\pgfpathlineto{\pgfqpoint{3.944978in}{3.907015in}}%
\pgfpathclose%
\pgfusepath{fill}%
\end{pgfscope}%
\begin{pgfscope}%
\pgfpathrectangle{\pgfqpoint{1.254980in}{0.150000in}}{\pgfqpoint{5.490039in}{5.490039in}}%
\pgfusepath{clip}%
\pgfsetbuttcap%
\pgfsetroundjoin%
\definecolor{currentfill}{rgb}{0.140210,0.665859,0.513427}%
\pgfsetfillcolor{currentfill}%
\pgfsetfillopacity{0.700000}%
\pgfsetlinewidth{0.000000pt}%
\definecolor{currentstroke}{rgb}{0.000000,0.000000,0.000000}%
\pgfsetstrokecolor{currentstroke}%
\pgfsetdash{}{0pt}%
\pgfpathmoveto{\pgfqpoint{4.025735in}{3.949669in}}%
\pgfpathlineto{\pgfqpoint{4.038536in}{3.932784in}}%
\pgfpathlineto{\pgfqpoint{4.051336in}{3.916123in}}%
\pgfpathlineto{\pgfqpoint{4.064134in}{3.899685in}}%
\pgfpathlineto{\pgfqpoint{4.076932in}{3.883468in}}%
\pgfpathlineto{\pgfqpoint{4.084321in}{3.911347in}}%
\pgfpathlineto{\pgfqpoint{4.091711in}{3.939673in}}%
\pgfpathlineto{\pgfqpoint{4.099102in}{3.968454in}}%
\pgfpathlineto{\pgfqpoint{4.086303in}{3.985184in}}%
\pgfpathlineto{\pgfqpoint{4.073503in}{4.002136in}}%
\pgfpathlineto{\pgfqpoint{4.060702in}{4.019311in}}%
\pgfpathlineto{\pgfqpoint{4.047899in}{4.036711in}}%
\pgfpathlineto{\pgfqpoint{4.040511in}{4.007236in}}%
\pgfpathlineto{\pgfqpoint{4.033123in}{3.978225in}}%
\pgfpathlineto{\pgfqpoint{4.025735in}{3.949669in}}%
\pgfpathclose%
\pgfusepath{fill}%
\end{pgfscope}%
\begin{pgfscope}%
\pgfpathrectangle{\pgfqpoint{1.254980in}{0.150000in}}{\pgfqpoint{5.490039in}{5.490039in}}%
\pgfusepath{clip}%
\pgfsetbuttcap%
\pgfsetroundjoin%
\definecolor{currentfill}{rgb}{0.120092,0.600104,0.542530}%
\pgfsetfillcolor{currentfill}%
\pgfsetfillopacity{0.700000}%
\pgfsetlinewidth{0.000000pt}%
\definecolor{currentstroke}{rgb}{0.000000,0.000000,0.000000}%
\pgfsetstrokecolor{currentstroke}%
\pgfsetdash{}{0pt}%
\pgfpathmoveto{\pgfqpoint{3.834632in}{3.767874in}}%
\pgfpathlineto{\pgfqpoint{3.847442in}{3.751029in}}%
\pgfpathlineto{\pgfqpoint{3.860250in}{3.734420in}}%
\pgfpathlineto{\pgfqpoint{3.873056in}{3.718045in}}%
\pgfpathlineto{\pgfqpoint{3.885861in}{3.701902in}}%
\pgfpathlineto{\pgfqpoint{3.893255in}{3.726191in}}%
\pgfpathlineto{\pgfqpoint{3.900648in}{3.750852in}}%
\pgfpathlineto{\pgfqpoint{3.908039in}{3.775891in}}%
\pgfpathlineto{\pgfqpoint{3.915429in}{3.801315in}}%
\pgfpathlineto{\pgfqpoint{3.902624in}{3.818056in}}%
\pgfpathlineto{\pgfqpoint{3.889817in}{3.835029in}}%
\pgfpathlineto{\pgfqpoint{3.877008in}{3.852238in}}%
\pgfpathlineto{\pgfqpoint{3.864197in}{3.869682in}}%
\pgfpathlineto{\pgfqpoint{3.856808in}{3.843647in}}%
\pgfpathlineto{\pgfqpoint{3.849418in}{3.818006in}}%
\pgfpathlineto{\pgfqpoint{3.842026in}{3.792750in}}%
\pgfpathlineto{\pgfqpoint{3.834632in}{3.767874in}}%
\pgfpathclose%
\pgfusepath{fill}%
\end{pgfscope}%
\begin{pgfscope}%
\pgfpathrectangle{\pgfqpoint{1.254980in}{0.150000in}}{\pgfqpoint{5.490039in}{5.490039in}}%
\pgfusepath{clip}%
\pgfsetbuttcap%
\pgfsetroundjoin%
\definecolor{currentfill}{rgb}{0.132444,0.552216,0.553018}%
\pgfsetfillcolor{currentfill}%
\pgfsetfillopacity{0.700000}%
\pgfsetlinewidth{0.000000pt}%
\definecolor{currentstroke}{rgb}{0.000000,0.000000,0.000000}%
\pgfsetstrokecolor{currentstroke}%
\pgfsetdash{}{0pt}%
\pgfpathmoveto{\pgfqpoint{4.281717in}{3.652859in}}%
\pgfpathlineto{\pgfqpoint{4.294525in}{3.640179in}}%
\pgfpathlineto{\pgfqpoint{4.307334in}{3.627697in}}%
\pgfpathlineto{\pgfqpoint{4.320146in}{3.615409in}}%
\pgfpathlineto{\pgfqpoint{4.332960in}{3.603316in}}%
\pgfpathlineto{\pgfqpoint{4.340346in}{3.627861in}}%
\pgfpathlineto{\pgfqpoint{4.347733in}{3.652813in}}%
\pgfpathlineto{\pgfqpoint{4.355123in}{3.678180in}}%
\pgfpathlineto{\pgfqpoint{4.342310in}{3.690774in}}%
\pgfpathlineto{\pgfqpoint{4.329500in}{3.703563in}}%
\pgfpathlineto{\pgfqpoint{4.316692in}{3.716548in}}%
\pgfpathlineto{\pgfqpoint{4.303886in}{3.729729in}}%
\pgfpathlineto{\pgfqpoint{4.296494in}{3.703685in}}%
\pgfpathlineto{\pgfqpoint{4.289105in}{3.678065in}}%
\pgfpathlineto{\pgfqpoint{4.281717in}{3.652859in}}%
\pgfpathclose%
\pgfusepath{fill}%
\end{pgfscope}%
\begin{pgfscope}%
\pgfpathrectangle{\pgfqpoint{1.254980in}{0.150000in}}{\pgfqpoint{5.490039in}{5.490039in}}%
\pgfusepath{clip}%
\pgfsetbuttcap%
\pgfsetroundjoin%
\definecolor{currentfill}{rgb}{0.147607,0.511733,0.557049}%
\pgfsetfillcolor{currentfill}%
\pgfsetfillopacity{0.700000}%
\pgfsetlinewidth{0.000000pt}%
\definecolor{currentstroke}{rgb}{0.000000,0.000000,0.000000}%
\pgfsetstrokecolor{currentstroke}%
\pgfsetdash{}{0pt}%
\pgfpathmoveto{\pgfqpoint{3.907474in}{3.548285in}}%
\pgfpathlineto{\pgfqpoint{3.920274in}{3.533839in}}%
\pgfpathlineto{\pgfqpoint{3.933074in}{3.519614in}}%
\pgfpathlineto{\pgfqpoint{3.945874in}{3.505609in}}%
\pgfpathlineto{\pgfqpoint{3.958674in}{3.491822in}}%
\pgfpathlineto{\pgfqpoint{3.966073in}{3.513594in}}%
\pgfpathlineto{\pgfqpoint{3.973470in}{3.535694in}}%
\pgfpathlineto{\pgfqpoint{3.980866in}{3.558129in}}%
\pgfpathlineto{\pgfqpoint{3.988260in}{3.580906in}}%
\pgfpathlineto{\pgfqpoint{3.975463in}{3.595253in}}%
\pgfpathlineto{\pgfqpoint{3.962664in}{3.609820in}}%
\pgfpathlineto{\pgfqpoint{3.949866in}{3.624607in}}%
\pgfpathlineto{\pgfqpoint{3.937067in}{3.639615in}}%
\pgfpathlineto{\pgfqpoint{3.929671in}{3.616266in}}%
\pgfpathlineto{\pgfqpoint{3.922274in}{3.593265in}}%
\pgfpathlineto{\pgfqpoint{3.914875in}{3.570607in}}%
\pgfpathlineto{\pgfqpoint{3.907474in}{3.548285in}}%
\pgfpathclose%
\pgfusepath{fill}%
\end{pgfscope}%
\begin{pgfscope}%
\pgfpathrectangle{\pgfqpoint{1.254980in}{0.150000in}}{\pgfqpoint{5.490039in}{5.490039in}}%
\pgfusepath{clip}%
\pgfsetbuttcap%
\pgfsetroundjoin%
\definecolor{currentfill}{rgb}{0.137770,0.537492,0.554906}%
\pgfsetfillcolor{currentfill}%
\pgfsetfillopacity{0.700000}%
\pgfsetlinewidth{0.000000pt}%
\definecolor{currentstroke}{rgb}{0.000000,0.000000,0.000000}%
\pgfsetstrokecolor{currentstroke}%
\pgfsetdash{}{0pt}%
\pgfpathmoveto{\pgfqpoint{3.856265in}{3.608315in}}%
\pgfpathlineto{\pgfqpoint{3.869069in}{3.592967in}}%
\pgfpathlineto{\pgfqpoint{3.881871in}{3.577848in}}%
\pgfpathlineto{\pgfqpoint{3.894673in}{3.562954in}}%
\pgfpathlineto{\pgfqpoint{3.907474in}{3.548285in}}%
\pgfpathlineto{\pgfqpoint{3.914875in}{3.570607in}}%
\pgfpathlineto{\pgfqpoint{3.922274in}{3.593265in}}%
\pgfpathlineto{\pgfqpoint{3.929671in}{3.616266in}}%
\pgfpathlineto{\pgfqpoint{3.937067in}{3.639615in}}%
\pgfpathlineto{\pgfqpoint{3.924267in}{3.654848in}}%
\pgfpathlineto{\pgfqpoint{3.911466in}{3.670305in}}%
\pgfpathlineto{\pgfqpoint{3.898664in}{3.685989in}}%
\pgfpathlineto{\pgfqpoint{3.885861in}{3.701902in}}%
\pgfpathlineto{\pgfqpoint{3.878465in}{3.677976in}}%
\pgfpathlineto{\pgfqpoint{3.871067in}{3.654408in}}%
\pgfpathlineto{\pgfqpoint{3.863667in}{3.631190in}}%
\pgfpathlineto{\pgfqpoint{3.856265in}{3.608315in}}%
\pgfpathclose%
\pgfusepath{fill}%
\end{pgfscope}%
\begin{pgfscope}%
\pgfpathrectangle{\pgfqpoint{1.254980in}{0.150000in}}{\pgfqpoint{5.490039in}{5.490039in}}%
\pgfusepath{clip}%
\pgfsetbuttcap%
\pgfsetroundjoin%
\definecolor{currentfill}{rgb}{0.124780,0.640461,0.527068}%
\pgfsetfillcolor{currentfill}%
\pgfsetfillopacity{0.700000}%
\pgfsetlinewidth{0.000000pt}%
\definecolor{currentstroke}{rgb}{0.000000,0.000000,0.000000}%
\pgfsetstrokecolor{currentstroke}%
\pgfsetdash{}{0pt}%
\pgfpathmoveto{\pgfqpoint{3.864197in}{3.869682in}}%
\pgfpathlineto{\pgfqpoint{3.877008in}{3.852238in}}%
\pgfpathlineto{\pgfqpoint{3.889817in}{3.835029in}}%
\pgfpathlineto{\pgfqpoint{3.902624in}{3.818056in}}%
\pgfpathlineto{\pgfqpoint{3.915429in}{3.801315in}}%
\pgfpathlineto{\pgfqpoint{3.922817in}{3.827132in}}%
\pgfpathlineto{\pgfqpoint{3.930205in}{3.853349in}}%
\pgfpathlineto{\pgfqpoint{3.937592in}{3.879974in}}%
\pgfpathlineto{\pgfqpoint{3.944978in}{3.907015in}}%
\pgfpathlineto{\pgfqpoint{3.932171in}{3.924385in}}%
\pgfpathlineto{\pgfqpoint{3.919363in}{3.941990in}}%
\pgfpathlineto{\pgfqpoint{3.906552in}{3.959830in}}%
\pgfpathlineto{\pgfqpoint{3.893739in}{3.977907in}}%
\pgfpathlineto{\pgfqpoint{3.886355in}{3.950223in}}%
\pgfpathlineto{\pgfqpoint{3.878971in}{3.922963in}}%
\pgfpathlineto{\pgfqpoint{3.871584in}{3.896118in}}%
\pgfpathlineto{\pgfqpoint{3.864197in}{3.869682in}}%
\pgfpathclose%
\pgfusepath{fill}%
\end{pgfscope}%
\begin{pgfscope}%
\pgfpathrectangle{\pgfqpoint{1.254980in}{0.150000in}}{\pgfqpoint{5.490039in}{5.490039in}}%
\pgfusepath{clip}%
\pgfsetbuttcap%
\pgfsetroundjoin%
\definecolor{currentfill}{rgb}{0.151918,0.500685,0.557587}%
\pgfsetfillcolor{currentfill}%
\pgfsetfillopacity{0.700000}%
\pgfsetlinewidth{0.000000pt}%
\definecolor{currentstroke}{rgb}{0.000000,0.000000,0.000000}%
\pgfsetstrokecolor{currentstroke}%
\pgfsetdash{}{0pt}%
\pgfpathmoveto{\pgfqpoint{4.171415in}{3.512876in}}%
\pgfpathlineto{\pgfqpoint{4.184221in}{3.500658in}}%
\pgfpathlineto{\pgfqpoint{4.197029in}{3.488640in}}%
\pgfpathlineto{\pgfqpoint{4.209838in}{3.476821in}}%
\pgfpathlineto{\pgfqpoint{4.222650in}{3.465200in}}%
\pgfpathlineto{\pgfqpoint{4.230031in}{3.487368in}}%
\pgfpathlineto{\pgfqpoint{4.237413in}{3.509890in}}%
\pgfpathlineto{\pgfqpoint{4.244795in}{3.532772in}}%
\pgfpathlineto{\pgfqpoint{4.252178in}{3.556023in}}%
\pgfpathlineto{\pgfqpoint{4.239369in}{3.568257in}}%
\pgfpathlineto{\pgfqpoint{4.226563in}{3.580689in}}%
\pgfpathlineto{\pgfqpoint{4.213758in}{3.593320in}}%
\pgfpathlineto{\pgfqpoint{4.200954in}{3.606152in}}%
\pgfpathlineto{\pgfqpoint{4.193569in}{3.582277in}}%
\pgfpathlineto{\pgfqpoint{4.186184in}{3.558778in}}%
\pgfpathlineto{\pgfqpoint{4.178800in}{3.535646in}}%
\pgfpathlineto{\pgfqpoint{4.171415in}{3.512876in}}%
\pgfpathclose%
\pgfusepath{fill}%
\end{pgfscope}%
\begin{pgfscope}%
\pgfpathrectangle{\pgfqpoint{1.254980in}{0.150000in}}{\pgfqpoint{5.490039in}{5.490039in}}%
\pgfusepath{clip}%
\pgfsetbuttcap%
\pgfsetroundjoin%
\definecolor{currentfill}{rgb}{0.154815,0.493313,0.557840}%
\pgfsetfillcolor{currentfill}%
\pgfsetfillopacity{0.700000}%
\pgfsetlinewidth{0.000000pt}%
\definecolor{currentstroke}{rgb}{0.000000,0.000000,0.000000}%
\pgfsetstrokecolor{currentstroke}%
\pgfsetdash{}{0pt}%
\pgfpathmoveto{\pgfqpoint{3.958674in}{3.491822in}}%
\pgfpathlineto{\pgfqpoint{3.971473in}{3.478252in}}%
\pgfpathlineto{\pgfqpoint{3.984273in}{3.464896in}}%
\pgfpathlineto{\pgfqpoint{3.997073in}{3.451755in}}%
\pgfpathlineto{\pgfqpoint{4.009874in}{3.438825in}}%
\pgfpathlineto{\pgfqpoint{4.017271in}{3.460049in}}%
\pgfpathlineto{\pgfqpoint{4.024666in}{3.481594in}}%
\pgfpathlineto{\pgfqpoint{4.032060in}{3.503466in}}%
\pgfpathlineto{\pgfqpoint{4.039453in}{3.525672in}}%
\pgfpathlineto{\pgfqpoint{4.026654in}{3.539160in}}%
\pgfpathlineto{\pgfqpoint{4.013856in}{3.552861in}}%
\pgfpathlineto{\pgfqpoint{4.001058in}{3.566775in}}%
\pgfpathlineto{\pgfqpoint{3.988260in}{3.580906in}}%
\pgfpathlineto{\pgfqpoint{3.980866in}{3.558129in}}%
\pgfpathlineto{\pgfqpoint{3.973470in}{3.535694in}}%
\pgfpathlineto{\pgfqpoint{3.966073in}{3.513594in}}%
\pgfpathlineto{\pgfqpoint{3.958674in}{3.491822in}}%
\pgfpathclose%
\pgfusepath{fill}%
\end{pgfscope}%
\begin{pgfscope}%
\pgfpathrectangle{\pgfqpoint{1.254980in}{0.150000in}}{\pgfqpoint{5.490039in}{5.490039in}}%
\pgfusepath{clip}%
\pgfsetbuttcap%
\pgfsetroundjoin%
\definecolor{currentfill}{rgb}{0.144759,0.519093,0.556572}%
\pgfsetfillcolor{currentfill}%
\pgfsetfillopacity{0.700000}%
\pgfsetlinewidth{0.000000pt}%
\definecolor{currentstroke}{rgb}{0.000000,0.000000,0.000000}%
\pgfsetstrokecolor{currentstroke}%
\pgfsetdash{}{0pt}%
\pgfpathmoveto{\pgfqpoint{4.252178in}{3.556023in}}%
\pgfpathlineto{\pgfqpoint{4.264988in}{3.543986in}}%
\pgfpathlineto{\pgfqpoint{4.277801in}{3.532145in}}%
\pgfpathlineto{\pgfqpoint{4.290616in}{3.520499in}}%
\pgfpathlineto{\pgfqpoint{4.303433in}{3.509046in}}%
\pgfpathlineto{\pgfqpoint{4.310813in}{3.532042in}}%
\pgfpathlineto{\pgfqpoint{4.318194in}{3.555414in}}%
\pgfpathlineto{\pgfqpoint{4.325577in}{3.579169in}}%
\pgfpathlineto{\pgfqpoint{4.332960in}{3.603316in}}%
\pgfpathlineto{\pgfqpoint{4.320146in}{3.615409in}}%
\pgfpathlineto{\pgfqpoint{4.307334in}{3.627697in}}%
\pgfpathlineto{\pgfqpoint{4.294525in}{3.640179in}}%
\pgfpathlineto{\pgfqpoint{4.281717in}{3.652859in}}%
\pgfpathlineto{\pgfqpoint{4.274330in}{3.628059in}}%
\pgfpathlineto{\pgfqpoint{4.266945in}{3.603659in}}%
\pgfpathlineto{\pgfqpoint{4.259561in}{3.579649in}}%
\pgfpathlineto{\pgfqpoint{4.252178in}{3.556023in}}%
\pgfpathclose%
\pgfusepath{fill}%
\end{pgfscope}%
\begin{pgfscope}%
\pgfpathrectangle{\pgfqpoint{1.254980in}{0.150000in}}{\pgfqpoint{5.490039in}{5.490039in}}%
\pgfusepath{clip}%
\pgfsetbuttcap%
\pgfsetroundjoin%
\definecolor{currentfill}{rgb}{0.157729,0.485932,0.558013}%
\pgfsetfillcolor{currentfill}%
\pgfsetfillopacity{0.700000}%
\pgfsetlinewidth{0.000000pt}%
\definecolor{currentstroke}{rgb}{0.000000,0.000000,0.000000}%
\pgfsetstrokecolor{currentstroke}%
\pgfsetdash{}{0pt}%
\pgfpathmoveto{\pgfqpoint{4.090654in}{3.473820in}}%
\pgfpathlineto{\pgfqpoint{4.103456in}{3.461376in}}%
\pgfpathlineto{\pgfqpoint{4.116261in}{3.449135in}}%
\pgfpathlineto{\pgfqpoint{4.129066in}{3.437098in}}%
\pgfpathlineto{\pgfqpoint{4.141873in}{3.425263in}}%
\pgfpathlineto{\pgfqpoint{4.149260in}{3.446660in}}%
\pgfpathlineto{\pgfqpoint{4.156645in}{3.468391in}}%
\pgfpathlineto{\pgfqpoint{4.164030in}{3.490460in}}%
\pgfpathlineto{\pgfqpoint{4.171415in}{3.512876in}}%
\pgfpathlineto{\pgfqpoint{4.158611in}{3.525296in}}%
\pgfpathlineto{\pgfqpoint{4.145808in}{3.537917in}}%
\pgfpathlineto{\pgfqpoint{4.133007in}{3.550743in}}%
\pgfpathlineto{\pgfqpoint{4.120206in}{3.563774in}}%
\pgfpathlineto{\pgfqpoint{4.112819in}{3.540761in}}%
\pgfpathlineto{\pgfqpoint{4.105431in}{3.518103in}}%
\pgfpathlineto{\pgfqpoint{4.098043in}{3.495792in}}%
\pgfpathlineto{\pgfqpoint{4.090654in}{3.473820in}}%
\pgfpathclose%
\pgfusepath{fill}%
\end{pgfscope}%
\begin{pgfscope}%
\pgfpathrectangle{\pgfqpoint{1.254980in}{0.150000in}}{\pgfqpoint{5.490039in}{5.490039in}}%
\pgfusepath{clip}%
\pgfsetbuttcap%
\pgfsetroundjoin%
\definecolor{currentfill}{rgb}{0.129933,0.559582,0.551864}%
\pgfsetfillcolor{currentfill}%
\pgfsetfillopacity{0.700000}%
\pgfsetlinewidth{0.000000pt}%
\definecolor{currentstroke}{rgb}{0.000000,0.000000,0.000000}%
\pgfsetstrokecolor{currentstroke}%
\pgfsetdash{}{0pt}%
\pgfpathmoveto{\pgfqpoint{3.805035in}{3.672018in}}%
\pgfpathlineto{\pgfqpoint{3.817845in}{3.655742in}}%
\pgfpathlineto{\pgfqpoint{3.830653in}{3.639700in}}%
\pgfpathlineto{\pgfqpoint{3.843460in}{3.623892in}}%
\pgfpathlineto{\pgfqpoint{3.856265in}{3.608315in}}%
\pgfpathlineto{\pgfqpoint{3.863667in}{3.631190in}}%
\pgfpathlineto{\pgfqpoint{3.871067in}{3.654408in}}%
\pgfpathlineto{\pgfqpoint{3.878465in}{3.677976in}}%
\pgfpathlineto{\pgfqpoint{3.885861in}{3.701902in}}%
\pgfpathlineto{\pgfqpoint{3.873056in}{3.718045in}}%
\pgfpathlineto{\pgfqpoint{3.860250in}{3.734420in}}%
\pgfpathlineto{\pgfqpoint{3.847442in}{3.751029in}}%
\pgfpathlineto{\pgfqpoint{3.834632in}{3.767874in}}%
\pgfpathlineto{\pgfqpoint{3.827236in}{3.743369in}}%
\pgfpathlineto{\pgfqpoint{3.819838in}{3.719230in}}%
\pgfpathlineto{\pgfqpoint{3.812438in}{3.695448in}}%
\pgfpathlineto{\pgfqpoint{3.805035in}{3.672018in}}%
\pgfpathclose%
\pgfusepath{fill}%
\end{pgfscope}%
\begin{pgfscope}%
\pgfpathrectangle{\pgfqpoint{1.254980in}{0.150000in}}{\pgfqpoint{5.490039in}{5.490039in}}%
\pgfusepath{clip}%
\pgfsetbuttcap%
\pgfsetroundjoin%
\definecolor{currentfill}{rgb}{0.170948,0.694384,0.493803}%
\pgfsetfillcolor{currentfill}%
\pgfsetfillopacity{0.700000}%
\pgfsetlinewidth{0.000000pt}%
\definecolor{currentstroke}{rgb}{0.000000,0.000000,0.000000}%
\pgfsetstrokecolor{currentstroke}%
\pgfsetdash{}{0pt}%
\pgfpathmoveto{\pgfqpoint{3.974518in}{4.019489in}}%
\pgfpathlineto{\pgfqpoint{3.987325in}{4.001689in}}%
\pgfpathlineto{\pgfqpoint{4.000130in}{3.984120in}}%
\pgfpathlineto{\pgfqpoint{4.012933in}{3.966781in}}%
\pgfpathlineto{\pgfqpoint{4.025735in}{3.949669in}}%
\pgfpathlineto{\pgfqpoint{4.033123in}{3.978225in}}%
\pgfpathlineto{\pgfqpoint{4.040511in}{4.007236in}}%
\pgfpathlineto{\pgfqpoint{4.047899in}{4.036711in}}%
\pgfpathlineto{\pgfqpoint{4.035095in}{4.054339in}}%
\pgfpathlineto{\pgfqpoint{4.022290in}{4.072195in}}%
\pgfpathlineto{\pgfqpoint{4.009483in}{4.090281in}}%
\pgfpathlineto{\pgfqpoint{3.996673in}{4.108600in}}%
\pgfpathlineto{\pgfqpoint{3.989288in}{4.078427in}}%
\pgfpathlineto{\pgfqpoint{3.981903in}{4.048726in}}%
\pgfpathlineto{\pgfqpoint{3.974518in}{4.019489in}}%
\pgfpathclose%
\pgfusepath{fill}%
\end{pgfscope}%
\begin{pgfscope}%
\pgfpathrectangle{\pgfqpoint{1.254980in}{0.150000in}}{\pgfqpoint{5.490039in}{5.490039in}}%
\pgfusepath{clip}%
\pgfsetbuttcap%
\pgfsetroundjoin%
\definecolor{currentfill}{rgb}{0.157851,0.683765,0.501686}%
\pgfsetfillcolor{currentfill}%
\pgfsetfillopacity{0.700000}%
\pgfsetlinewidth{0.000000pt}%
\definecolor{currentstroke}{rgb}{0.000000,0.000000,0.000000}%
\pgfsetstrokecolor{currentstroke}%
\pgfsetdash{}{0pt}%
\pgfpathmoveto{\pgfqpoint{3.893739in}{3.977907in}}%
\pgfpathlineto{\pgfqpoint{3.906552in}{3.959830in}}%
\pgfpathlineto{\pgfqpoint{3.919363in}{3.941990in}}%
\pgfpathlineto{\pgfqpoint{3.932171in}{3.924385in}}%
\pgfpathlineto{\pgfqpoint{3.944978in}{3.907015in}}%
\pgfpathlineto{\pgfqpoint{3.952363in}{3.934479in}}%
\pgfpathlineto{\pgfqpoint{3.959748in}{3.962373in}}%
\pgfpathlineto{\pgfqpoint{3.967133in}{3.990708in}}%
\pgfpathlineto{\pgfqpoint{3.974518in}{4.019489in}}%
\pgfpathlineto{\pgfqpoint{3.961709in}{4.037522in}}%
\pgfpathlineto{\pgfqpoint{3.948898in}{4.055790in}}%
\pgfpathlineto{\pgfqpoint{3.936085in}{4.074295in}}%
\pgfpathlineto{\pgfqpoint{3.923269in}{4.093039in}}%
\pgfpathlineto{\pgfqpoint{3.915887in}{4.063580in}}%
\pgfpathlineto{\pgfqpoint{3.908505in}{4.034578in}}%
\pgfpathlineto{\pgfqpoint{3.901123in}{4.006023in}}%
\pgfpathlineto{\pgfqpoint{3.893739in}{3.977907in}}%
\pgfpathclose%
\pgfusepath{fill}%
\end{pgfscope}%
\begin{pgfscope}%
\pgfpathrectangle{\pgfqpoint{1.254980in}{0.150000in}}{\pgfqpoint{5.490039in}{5.490039in}}%
\pgfusepath{clip}%
\pgfsetbuttcap%
\pgfsetroundjoin%
\definecolor{currentfill}{rgb}{0.120638,0.625828,0.533488}%
\pgfsetfillcolor{currentfill}%
\pgfsetfillopacity{0.700000}%
\pgfsetlinewidth{0.000000pt}%
\definecolor{currentstroke}{rgb}{0.000000,0.000000,0.000000}%
\pgfsetstrokecolor{currentstroke}%
\pgfsetdash{}{0pt}%
\pgfpathmoveto{\pgfqpoint{3.783368in}{3.837648in}}%
\pgfpathlineto{\pgfqpoint{3.796188in}{3.819841in}}%
\pgfpathlineto{\pgfqpoint{3.809005in}{3.802278in}}%
\pgfpathlineto{\pgfqpoint{3.821820in}{3.784956in}}%
\pgfpathlineto{\pgfqpoint{3.834632in}{3.767874in}}%
\pgfpathlineto{\pgfqpoint{3.842026in}{3.792750in}}%
\pgfpathlineto{\pgfqpoint{3.849418in}{3.818006in}}%
\pgfpathlineto{\pgfqpoint{3.856808in}{3.843647in}}%
\pgfpathlineto{\pgfqpoint{3.864197in}{3.869682in}}%
\pgfpathlineto{\pgfqpoint{3.851384in}{3.887366in}}%
\pgfpathlineto{\pgfqpoint{3.838568in}{3.905290in}}%
\pgfpathlineto{\pgfqpoint{3.825750in}{3.923456in}}%
\pgfpathlineto{\pgfqpoint{3.812928in}{3.941866in}}%
\pgfpathlineto{\pgfqpoint{3.805541in}{3.915216in}}%
\pgfpathlineto{\pgfqpoint{3.798152in}{3.888968in}}%
\pgfpathlineto{\pgfqpoint{3.790761in}{3.863114in}}%
\pgfpathlineto{\pgfqpoint{3.783368in}{3.837648in}}%
\pgfpathclose%
\pgfusepath{fill}%
\end{pgfscope}%
\begin{pgfscope}%
\pgfpathrectangle{\pgfqpoint{1.254980in}{0.150000in}}{\pgfqpoint{5.490039in}{5.490039in}}%
\pgfusepath{clip}%
\pgfsetbuttcap%
\pgfsetroundjoin%
\definecolor{currentfill}{rgb}{0.139147,0.533812,0.555298}%
\pgfsetfillcolor{currentfill}%
\pgfsetfillopacity{0.700000}%
\pgfsetlinewidth{0.000000pt}%
\definecolor{currentstroke}{rgb}{0.000000,0.000000,0.000000}%
\pgfsetstrokecolor{currentstroke}%
\pgfsetdash{}{0pt}%
\pgfpathmoveto{\pgfqpoint{4.332960in}{3.603316in}}%
\pgfpathlineto{\pgfqpoint{4.345777in}{3.591415in}}%
\pgfpathlineto{\pgfqpoint{4.358596in}{3.579707in}}%
\pgfpathlineto{\pgfqpoint{4.371417in}{3.568189in}}%
\pgfpathlineto{\pgfqpoint{4.384241in}{3.556862in}}%
\pgfpathlineto{\pgfqpoint{4.391624in}{3.580748in}}%
\pgfpathlineto{\pgfqpoint{4.399008in}{3.605034in}}%
\pgfpathlineto{\pgfqpoint{4.406394in}{3.629727in}}%
\pgfpathlineto{\pgfqpoint{4.393573in}{3.641554in}}%
\pgfpathlineto{\pgfqpoint{4.380753in}{3.653571in}}%
\pgfpathlineto{\pgfqpoint{4.367937in}{3.665779in}}%
\pgfpathlineto{\pgfqpoint{4.355123in}{3.678180in}}%
\pgfpathlineto{\pgfqpoint{4.347733in}{3.652813in}}%
\pgfpathlineto{\pgfqpoint{4.340346in}{3.627861in}}%
\pgfpathlineto{\pgfqpoint{4.332960in}{3.603316in}}%
\pgfpathclose%
\pgfusepath{fill}%
\end{pgfscope}%
\begin{pgfscope}%
\pgfpathrectangle{\pgfqpoint{1.254980in}{0.150000in}}{\pgfqpoint{5.490039in}{5.490039in}}%
\pgfusepath{clip}%
\pgfsetbuttcap%
\pgfsetroundjoin%
\definecolor{currentfill}{rgb}{0.163625,0.471133,0.558148}%
\pgfsetfillcolor{currentfill}%
\pgfsetfillopacity{0.700000}%
\pgfsetlinewidth{0.000000pt}%
\definecolor{currentstroke}{rgb}{0.000000,0.000000,0.000000}%
\pgfsetstrokecolor{currentstroke}%
\pgfsetdash{}{0pt}%
\pgfpathmoveto{\pgfqpoint{4.009874in}{3.438825in}}%
\pgfpathlineto{\pgfqpoint{4.022675in}{3.426106in}}%
\pgfpathlineto{\pgfqpoint{4.035478in}{3.413597in}}%
\pgfpathlineto{\pgfqpoint{4.048281in}{3.401296in}}%
\pgfpathlineto{\pgfqpoint{4.061085in}{3.389201in}}%
\pgfpathlineto{\pgfqpoint{4.068479in}{3.409880in}}%
\pgfpathlineto{\pgfqpoint{4.075872in}{3.430871in}}%
\pgfpathlineto{\pgfqpoint{4.083263in}{3.452183in}}%
\pgfpathlineto{\pgfqpoint{4.090654in}{3.473820in}}%
\pgfpathlineto{\pgfqpoint{4.077852in}{3.486471in}}%
\pgfpathlineto{\pgfqpoint{4.065051in}{3.499329in}}%
\pgfpathlineto{\pgfqpoint{4.052252in}{3.512396in}}%
\pgfpathlineto{\pgfqpoint{4.039453in}{3.525672in}}%
\pgfpathlineto{\pgfqpoint{4.032060in}{3.503466in}}%
\pgfpathlineto{\pgfqpoint{4.024666in}{3.481594in}}%
\pgfpathlineto{\pgfqpoint{4.017271in}{3.460049in}}%
\pgfpathlineto{\pgfqpoint{4.009874in}{3.438825in}}%
\pgfpathclose%
\pgfusepath{fill}%
\end{pgfscope}%
\begin{pgfscope}%
\pgfpathrectangle{\pgfqpoint{1.254980in}{0.150000in}}{\pgfqpoint{5.490039in}{5.490039in}}%
\pgfusepath{clip}%
\pgfsetbuttcap%
\pgfsetroundjoin%
\definecolor{currentfill}{rgb}{0.121831,0.589055,0.545623}%
\pgfsetfillcolor{currentfill}%
\pgfsetfillopacity{0.700000}%
\pgfsetlinewidth{0.000000pt}%
\definecolor{currentstroke}{rgb}{0.000000,0.000000,0.000000}%
\pgfsetstrokecolor{currentstroke}%
\pgfsetdash{}{0pt}%
\pgfpathmoveto{\pgfqpoint{3.753773in}{3.739510in}}%
\pgfpathlineto{\pgfqpoint{3.766592in}{3.722275in}}%
\pgfpathlineto{\pgfqpoint{3.779409in}{3.705283in}}%
\pgfpathlineto{\pgfqpoint{3.792223in}{3.688531in}}%
\pgfpathlineto{\pgfqpoint{3.805035in}{3.672018in}}%
\pgfpathlineto{\pgfqpoint{3.812438in}{3.695448in}}%
\pgfpathlineto{\pgfqpoint{3.819838in}{3.719230in}}%
\pgfpathlineto{\pgfqpoint{3.827236in}{3.743369in}}%
\pgfpathlineto{\pgfqpoint{3.834632in}{3.767874in}}%
\pgfpathlineto{\pgfqpoint{3.821820in}{3.784956in}}%
\pgfpathlineto{\pgfqpoint{3.809005in}{3.802278in}}%
\pgfpathlineto{\pgfqpoint{3.796188in}{3.819841in}}%
\pgfpathlineto{\pgfqpoint{3.783368in}{3.837648in}}%
\pgfpathlineto{\pgfqpoint{3.775973in}{3.812561in}}%
\pgfpathlineto{\pgfqpoint{3.768576in}{3.787847in}}%
\pgfpathlineto{\pgfqpoint{3.761176in}{3.763499in}}%
\pgfpathlineto{\pgfqpoint{3.753773in}{3.739510in}}%
\pgfpathclose%
\pgfusepath{fill}%
\end{pgfscope}%
\begin{pgfscope}%
\pgfpathrectangle{\pgfqpoint{1.254980in}{0.150000in}}{\pgfqpoint{5.490039in}{5.490039in}}%
\pgfusepath{clip}%
\pgfsetbuttcap%
\pgfsetroundjoin%
\definecolor{currentfill}{rgb}{0.150476,0.504369,0.557430}%
\pgfsetfillcolor{currentfill}%
\pgfsetfillopacity{0.700000}%
\pgfsetlinewidth{0.000000pt}%
\definecolor{currentstroke}{rgb}{0.000000,0.000000,0.000000}%
\pgfsetstrokecolor{currentstroke}%
\pgfsetdash{}{0pt}%
\pgfpathmoveto{\pgfqpoint{3.826633in}{3.520120in}}%
\pgfpathlineto{\pgfqpoint{3.839439in}{3.505307in}}%
\pgfpathlineto{\pgfqpoint{3.852243in}{3.490721in}}%
\pgfpathlineto{\pgfqpoint{3.865046in}{3.476361in}}%
\pgfpathlineto{\pgfqpoint{3.877849in}{3.462225in}}%
\pgfpathlineto{\pgfqpoint{3.885259in}{3.483269in}}%
\pgfpathlineto{\pgfqpoint{3.892666in}{3.504622in}}%
\pgfpathlineto{\pgfqpoint{3.900071in}{3.526292in}}%
\pgfpathlineto{\pgfqpoint{3.907474in}{3.548285in}}%
\pgfpathlineto{\pgfqpoint{3.894673in}{3.562954in}}%
\pgfpathlineto{\pgfqpoint{3.881871in}{3.577848in}}%
\pgfpathlineto{\pgfqpoint{3.869069in}{3.592967in}}%
\pgfpathlineto{\pgfqpoint{3.856265in}{3.608315in}}%
\pgfpathlineto{\pgfqpoint{3.848861in}{3.585777in}}%
\pgfpathlineto{\pgfqpoint{3.841454in}{3.563569in}}%
\pgfpathlineto{\pgfqpoint{3.834045in}{3.541686in}}%
\pgfpathlineto{\pgfqpoint{3.826633in}{3.520120in}}%
\pgfpathclose%
\pgfusepath{fill}%
\end{pgfscope}%
\begin{pgfscope}%
\pgfpathrectangle{\pgfqpoint{1.254980in}{0.150000in}}{\pgfqpoint{5.490039in}{5.490039in}}%
\pgfusepath{clip}%
\pgfsetbuttcap%
\pgfsetroundjoin%
\definecolor{currentfill}{rgb}{0.143303,0.669459,0.511215}%
\pgfsetfillcolor{currentfill}%
\pgfsetfillopacity{0.700000}%
\pgfsetlinewidth{0.000000pt}%
\definecolor{currentstroke}{rgb}{0.000000,0.000000,0.000000}%
\pgfsetstrokecolor{currentstroke}%
\pgfsetdash{}{0pt}%
\pgfpathmoveto{\pgfqpoint{3.812928in}{3.941866in}}%
\pgfpathlineto{\pgfqpoint{3.825750in}{3.923456in}}%
\pgfpathlineto{\pgfqpoint{3.838568in}{3.905290in}}%
\pgfpathlineto{\pgfqpoint{3.851384in}{3.887366in}}%
\pgfpathlineto{\pgfqpoint{3.864197in}{3.869682in}}%
\pgfpathlineto{\pgfqpoint{3.871584in}{3.896118in}}%
\pgfpathlineto{\pgfqpoint{3.878971in}{3.922963in}}%
\pgfpathlineto{\pgfqpoint{3.886355in}{3.950223in}}%
\pgfpathlineto{\pgfqpoint{3.893739in}{3.977907in}}%
\pgfpathlineto{\pgfqpoint{3.880924in}{3.996224in}}%
\pgfpathlineto{\pgfqpoint{3.868106in}{4.014783in}}%
\pgfpathlineto{\pgfqpoint{3.855285in}{4.033585in}}%
\pgfpathlineto{\pgfqpoint{3.842461in}{4.052632in}}%
\pgfpathlineto{\pgfqpoint{3.835080in}{4.024300in}}%
\pgfpathlineto{\pgfqpoint{3.827698in}{3.996401in}}%
\pgfpathlineto{\pgfqpoint{3.820314in}{3.968925in}}%
\pgfpathlineto{\pgfqpoint{3.812928in}{3.941866in}}%
\pgfpathclose%
\pgfusepath{fill}%
\end{pgfscope}%
\begin{pgfscope}%
\pgfpathrectangle{\pgfqpoint{1.254980in}{0.150000in}}{\pgfqpoint{5.490039in}{5.490039in}}%
\pgfusepath{clip}%
\pgfsetbuttcap%
\pgfsetroundjoin%
\definecolor{currentfill}{rgb}{0.157729,0.485932,0.558013}%
\pgfsetfillcolor{currentfill}%
\pgfsetfillopacity{0.700000}%
\pgfsetlinewidth{0.000000pt}%
\definecolor{currentstroke}{rgb}{0.000000,0.000000,0.000000}%
\pgfsetstrokecolor{currentstroke}%
\pgfsetdash{}{0pt}%
\pgfpathmoveto{\pgfqpoint{4.222650in}{3.465200in}}%
\pgfpathlineto{\pgfqpoint{4.235463in}{3.453775in}}%
\pgfpathlineto{\pgfqpoint{4.248279in}{3.442545in}}%
\pgfpathlineto{\pgfqpoint{4.261098in}{3.431510in}}%
\pgfpathlineto{\pgfqpoint{4.273919in}{3.420668in}}%
\pgfpathlineto{\pgfqpoint{4.281297in}{3.442236in}}%
\pgfpathlineto{\pgfqpoint{4.288675in}{3.464151in}}%
\pgfpathlineto{\pgfqpoint{4.296054in}{3.486418in}}%
\pgfpathlineto{\pgfqpoint{4.303433in}{3.509046in}}%
\pgfpathlineto{\pgfqpoint{4.290616in}{3.520499in}}%
\pgfpathlineto{\pgfqpoint{4.277801in}{3.532145in}}%
\pgfpathlineto{\pgfqpoint{4.264988in}{3.543986in}}%
\pgfpathlineto{\pgfqpoint{4.252178in}{3.556023in}}%
\pgfpathlineto{\pgfqpoint{4.244795in}{3.532772in}}%
\pgfpathlineto{\pgfqpoint{4.237413in}{3.509890in}}%
\pgfpathlineto{\pgfqpoint{4.230031in}{3.487368in}}%
\pgfpathlineto{\pgfqpoint{4.222650in}{3.465200in}}%
\pgfpathclose%
\pgfusepath{fill}%
\end{pgfscope}%
\begin{pgfscope}%
\pgfpathrectangle{\pgfqpoint{1.254980in}{0.150000in}}{\pgfqpoint{5.490039in}{5.490039in}}%
\pgfusepath{clip}%
\pgfsetbuttcap%
\pgfsetroundjoin%
\definecolor{currentfill}{rgb}{0.141935,0.526453,0.555991}%
\pgfsetfillcolor{currentfill}%
\pgfsetfillopacity{0.700000}%
\pgfsetlinewidth{0.000000pt}%
\definecolor{currentstroke}{rgb}{0.000000,0.000000,0.000000}%
\pgfsetstrokecolor{currentstroke}%
\pgfsetdash{}{0pt}%
\pgfpathmoveto{\pgfqpoint{3.775399in}{3.581678in}}%
\pgfpathlineto{\pgfqpoint{3.788210in}{3.565939in}}%
\pgfpathlineto{\pgfqpoint{3.801019in}{3.550434in}}%
\pgfpathlineto{\pgfqpoint{3.813827in}{3.535162in}}%
\pgfpathlineto{\pgfqpoint{3.826633in}{3.520120in}}%
\pgfpathlineto{\pgfqpoint{3.834045in}{3.541686in}}%
\pgfpathlineto{\pgfqpoint{3.841454in}{3.563569in}}%
\pgfpathlineto{\pgfqpoint{3.848861in}{3.585777in}}%
\pgfpathlineto{\pgfqpoint{3.856265in}{3.608315in}}%
\pgfpathlineto{\pgfqpoint{3.843460in}{3.623892in}}%
\pgfpathlineto{\pgfqpoint{3.830653in}{3.639700in}}%
\pgfpathlineto{\pgfqpoint{3.817845in}{3.655742in}}%
\pgfpathlineto{\pgfqpoint{3.805035in}{3.672018in}}%
\pgfpathlineto{\pgfqpoint{3.797630in}{3.648932in}}%
\pgfpathlineto{\pgfqpoint{3.790223in}{3.626185in}}%
\pgfpathlineto{\pgfqpoint{3.782812in}{3.603769in}}%
\pgfpathlineto{\pgfqpoint{3.775399in}{3.581678in}}%
\pgfpathclose%
\pgfusepath{fill}%
\end{pgfscope}%
\begin{pgfscope}%
\pgfpathrectangle{\pgfqpoint{1.254980in}{0.150000in}}{\pgfqpoint{5.490039in}{5.490039in}}%
\pgfusepath{clip}%
\pgfsetbuttcap%
\pgfsetroundjoin%
\definecolor{currentfill}{rgb}{0.165117,0.467423,0.558141}%
\pgfsetfillcolor{currentfill}%
\pgfsetfillopacity{0.700000}%
\pgfsetlinewidth{0.000000pt}%
\definecolor{currentstroke}{rgb}{0.000000,0.000000,0.000000}%
\pgfsetstrokecolor{currentstroke}%
\pgfsetdash{}{0pt}%
\pgfpathmoveto{\pgfqpoint{4.141873in}{3.425263in}}%
\pgfpathlineto{\pgfqpoint{4.154682in}{3.413629in}}%
\pgfpathlineto{\pgfqpoint{4.167493in}{3.402194in}}%
\pgfpathlineto{\pgfqpoint{4.180306in}{3.390958in}}%
\pgfpathlineto{\pgfqpoint{4.193121in}{3.379919in}}%
\pgfpathlineto{\pgfqpoint{4.200504in}{3.400744in}}%
\pgfpathlineto{\pgfqpoint{4.207886in}{3.421895in}}%
\pgfpathlineto{\pgfqpoint{4.215268in}{3.443378in}}%
\pgfpathlineto{\pgfqpoint{4.222650in}{3.465200in}}%
\pgfpathlineto{\pgfqpoint{4.209838in}{3.476821in}}%
\pgfpathlineto{\pgfqpoint{4.197029in}{3.488640in}}%
\pgfpathlineto{\pgfqpoint{4.184221in}{3.500658in}}%
\pgfpathlineto{\pgfqpoint{4.171415in}{3.512876in}}%
\pgfpathlineto{\pgfqpoint{4.164030in}{3.490460in}}%
\pgfpathlineto{\pgfqpoint{4.156645in}{3.468391in}}%
\pgfpathlineto{\pgfqpoint{4.149260in}{3.446660in}}%
\pgfpathlineto{\pgfqpoint{4.141873in}{3.425263in}}%
\pgfpathclose%
\pgfusepath{fill}%
\end{pgfscope}%
\begin{pgfscope}%
\pgfpathrectangle{\pgfqpoint{1.254980in}{0.150000in}}{\pgfqpoint{5.490039in}{5.490039in}}%
\pgfusepath{clip}%
\pgfsetbuttcap%
\pgfsetroundjoin%
\definecolor{currentfill}{rgb}{0.159194,0.482237,0.558073}%
\pgfsetfillcolor{currentfill}%
\pgfsetfillopacity{0.700000}%
\pgfsetlinewidth{0.000000pt}%
\definecolor{currentstroke}{rgb}{0.000000,0.000000,0.000000}%
\pgfsetstrokecolor{currentstroke}%
\pgfsetdash{}{0pt}%
\pgfpathmoveto{\pgfqpoint{3.877849in}{3.462225in}}%
\pgfpathlineto{\pgfqpoint{3.890651in}{3.448311in}}%
\pgfpathlineto{\pgfqpoint{3.903453in}{3.434617in}}%
\pgfpathlineto{\pgfqpoint{3.916255in}{3.421143in}}%
\pgfpathlineto{\pgfqpoint{3.929057in}{3.407887in}}%
\pgfpathlineto{\pgfqpoint{3.936464in}{3.428410in}}%
\pgfpathlineto{\pgfqpoint{3.943869in}{3.449237in}}%
\pgfpathlineto{\pgfqpoint{3.951272in}{3.470372in}}%
\pgfpathlineto{\pgfqpoint{3.958674in}{3.491822in}}%
\pgfpathlineto{\pgfqpoint{3.945874in}{3.505609in}}%
\pgfpathlineto{\pgfqpoint{3.933074in}{3.519614in}}%
\pgfpathlineto{\pgfqpoint{3.920274in}{3.533839in}}%
\pgfpathlineto{\pgfqpoint{3.907474in}{3.548285in}}%
\pgfpathlineto{\pgfqpoint{3.900071in}{3.526292in}}%
\pgfpathlineto{\pgfqpoint{3.892666in}{3.504622in}}%
\pgfpathlineto{\pgfqpoint{3.885259in}{3.483269in}}%
\pgfpathlineto{\pgfqpoint{3.877849in}{3.462225in}}%
\pgfpathclose%
\pgfusepath{fill}%
\end{pgfscope}%
\begin{pgfscope}%
\pgfpathrectangle{\pgfqpoint{1.254980in}{0.150000in}}{\pgfqpoint{5.490039in}{5.490039in}}%
\pgfusepath{clip}%
\pgfsetbuttcap%
\pgfsetroundjoin%
\definecolor{currentfill}{rgb}{0.150476,0.504369,0.557430}%
\pgfsetfillcolor{currentfill}%
\pgfsetfillopacity{0.700000}%
\pgfsetlinewidth{0.000000pt}%
\definecolor{currentstroke}{rgb}{0.000000,0.000000,0.000000}%
\pgfsetstrokecolor{currentstroke}%
\pgfsetdash{}{0pt}%
\pgfpathmoveto{\pgfqpoint{4.303433in}{3.509046in}}%
\pgfpathlineto{\pgfqpoint{4.316253in}{3.497786in}}%
\pgfpathlineto{\pgfqpoint{4.329075in}{3.486717in}}%
\pgfpathlineto{\pgfqpoint{4.341900in}{3.475839in}}%
\pgfpathlineto{\pgfqpoint{4.354729in}{3.465150in}}%
\pgfpathlineto{\pgfqpoint{4.362105in}{3.487518in}}%
\pgfpathlineto{\pgfqpoint{4.369482in}{3.510254in}}%
\pgfpathlineto{\pgfqpoint{4.376861in}{3.533366in}}%
\pgfpathlineto{\pgfqpoint{4.384241in}{3.556862in}}%
\pgfpathlineto{\pgfqpoint{4.371417in}{3.568189in}}%
\pgfpathlineto{\pgfqpoint{4.358596in}{3.579707in}}%
\pgfpathlineto{\pgfqpoint{4.345777in}{3.591415in}}%
\pgfpathlineto{\pgfqpoint{4.332960in}{3.603316in}}%
\pgfpathlineto{\pgfqpoint{4.325577in}{3.579169in}}%
\pgfpathlineto{\pgfqpoint{4.318194in}{3.555414in}}%
\pgfpathlineto{\pgfqpoint{4.310813in}{3.532042in}}%
\pgfpathlineto{\pgfqpoint{4.303433in}{3.509046in}}%
\pgfpathclose%
\pgfusepath{fill}%
\end{pgfscope}%
\begin{pgfscope}%
\pgfpathrectangle{\pgfqpoint{1.254980in}{0.150000in}}{\pgfqpoint{5.490039in}{5.490039in}}%
\pgfusepath{clip}%
\pgfsetbuttcap%
\pgfsetroundjoin%
\definecolor{currentfill}{rgb}{0.132444,0.552216,0.553018}%
\pgfsetfillcolor{currentfill}%
\pgfsetfillopacity{0.700000}%
\pgfsetlinewidth{0.000000pt}%
\definecolor{currentstroke}{rgb}{0.000000,0.000000,0.000000}%
\pgfsetstrokecolor{currentstroke}%
\pgfsetdash{}{0pt}%
\pgfpathmoveto{\pgfqpoint{3.724135in}{3.647014in}}%
\pgfpathlineto{\pgfqpoint{3.736955in}{3.630319in}}%
\pgfpathlineto{\pgfqpoint{3.749772in}{3.613866in}}%
\pgfpathlineto{\pgfqpoint{3.762586in}{3.597653in}}%
\pgfpathlineto{\pgfqpoint{3.775399in}{3.581678in}}%
\pgfpathlineto{\pgfqpoint{3.782812in}{3.603769in}}%
\pgfpathlineto{\pgfqpoint{3.790223in}{3.626185in}}%
\pgfpathlineto{\pgfqpoint{3.797630in}{3.648932in}}%
\pgfpathlineto{\pgfqpoint{3.805035in}{3.672018in}}%
\pgfpathlineto{\pgfqpoint{3.792223in}{3.688531in}}%
\pgfpathlineto{\pgfqpoint{3.779409in}{3.705283in}}%
\pgfpathlineto{\pgfqpoint{3.766592in}{3.722275in}}%
\pgfpathlineto{\pgfqpoint{3.753773in}{3.739510in}}%
\pgfpathlineto{\pgfqpoint{3.746368in}{3.715874in}}%
\pgfpathlineto{\pgfqpoint{3.738960in}{3.692583in}}%
\pgfpathlineto{\pgfqpoint{3.731549in}{3.669632in}}%
\pgfpathlineto{\pgfqpoint{3.724135in}{3.647014in}}%
\pgfpathclose%
\pgfusepath{fill}%
\end{pgfscope}%
\begin{pgfscope}%
\pgfpathrectangle{\pgfqpoint{1.254980in}{0.150000in}}{\pgfqpoint{5.490039in}{5.490039in}}%
\pgfusepath{clip}%
\pgfsetbuttcap%
\pgfsetroundjoin%
\definecolor{currentfill}{rgb}{0.171176,0.452530,0.557965}%
\pgfsetfillcolor{currentfill}%
\pgfsetfillopacity{0.700000}%
\pgfsetlinewidth{0.000000pt}%
\definecolor{currentstroke}{rgb}{0.000000,0.000000,0.000000}%
\pgfsetstrokecolor{currentstroke}%
\pgfsetdash{}{0pt}%
\pgfpathmoveto{\pgfqpoint{4.061085in}{3.389201in}}%
\pgfpathlineto{\pgfqpoint{4.073891in}{3.377312in}}%
\pgfpathlineto{\pgfqpoint{4.086699in}{3.365627in}}%
\pgfpathlineto{\pgfqpoint{4.099508in}{3.354145in}}%
\pgfpathlineto{\pgfqpoint{4.112318in}{3.342864in}}%
\pgfpathlineto{\pgfqpoint{4.119709in}{3.362999in}}%
\pgfpathlineto{\pgfqpoint{4.127098in}{3.383439in}}%
\pgfpathlineto{\pgfqpoint{4.134486in}{3.404191in}}%
\pgfpathlineto{\pgfqpoint{4.141873in}{3.425263in}}%
\pgfpathlineto{\pgfqpoint{4.129066in}{3.437098in}}%
\pgfpathlineto{\pgfqpoint{4.116261in}{3.449135in}}%
\pgfpathlineto{\pgfqpoint{4.103456in}{3.461376in}}%
\pgfpathlineto{\pgfqpoint{4.090654in}{3.473820in}}%
\pgfpathlineto{\pgfqpoint{4.083263in}{3.452183in}}%
\pgfpathlineto{\pgfqpoint{4.075872in}{3.430871in}}%
\pgfpathlineto{\pgfqpoint{4.068479in}{3.409880in}}%
\pgfpathlineto{\pgfqpoint{4.061085in}{3.389201in}}%
\pgfpathclose%
\pgfusepath{fill}%
\end{pgfscope}%
\begin{pgfscope}%
\pgfpathrectangle{\pgfqpoint{1.254980in}{0.150000in}}{\pgfqpoint{5.490039in}{5.490039in}}%
\pgfusepath{clip}%
\pgfsetbuttcap%
\pgfsetroundjoin%
\definecolor{currentfill}{rgb}{0.168126,0.459988,0.558082}%
\pgfsetfillcolor{currentfill}%
\pgfsetfillopacity{0.700000}%
\pgfsetlinewidth{0.000000pt}%
\definecolor{currentstroke}{rgb}{0.000000,0.000000,0.000000}%
\pgfsetstrokecolor{currentstroke}%
\pgfsetdash{}{0pt}%
\pgfpathmoveto{\pgfqpoint{3.929057in}{3.407887in}}%
\pgfpathlineto{\pgfqpoint{3.941859in}{3.394846in}}%
\pgfpathlineto{\pgfqpoint{3.954661in}{3.382020in}}%
\pgfpathlineto{\pgfqpoint{3.967464in}{3.369407in}}%
\pgfpathlineto{\pgfqpoint{3.980268in}{3.357006in}}%
\pgfpathlineto{\pgfqpoint{3.987672in}{3.377012in}}%
\pgfpathlineto{\pgfqpoint{3.995075in}{3.397313in}}%
\pgfpathlineto{\pgfqpoint{4.002475in}{3.417915in}}%
\pgfpathlineto{\pgfqpoint{4.009874in}{3.438825in}}%
\pgfpathlineto{\pgfqpoint{3.997073in}{3.451755in}}%
\pgfpathlineto{\pgfqpoint{3.984273in}{3.464896in}}%
\pgfpathlineto{\pgfqpoint{3.971473in}{3.478252in}}%
\pgfpathlineto{\pgfqpoint{3.958674in}{3.491822in}}%
\pgfpathlineto{\pgfqpoint{3.951272in}{3.470372in}}%
\pgfpathlineto{\pgfqpoint{3.943869in}{3.449237in}}%
\pgfpathlineto{\pgfqpoint{3.936464in}{3.428410in}}%
\pgfpathlineto{\pgfqpoint{3.929057in}{3.407887in}}%
\pgfpathclose%
\pgfusepath{fill}%
\end{pgfscope}%
\begin{pgfscope}%
\pgfpathrectangle{\pgfqpoint{1.254980in}{0.150000in}}{\pgfqpoint{5.490039in}{5.490039in}}%
\pgfusepath{clip}%
\pgfsetbuttcap%
\pgfsetroundjoin%
\definecolor{currentfill}{rgb}{0.214000,0.722114,0.469588}%
\pgfsetfillcolor{currentfill}%
\pgfsetfillopacity{0.700000}%
\pgfsetlinewidth{0.000000pt}%
\definecolor{currentstroke}{rgb}{0.000000,0.000000,0.000000}%
\pgfsetstrokecolor{currentstroke}%
\pgfsetdash{}{0pt}%
\pgfpathmoveto{\pgfqpoint{3.923269in}{4.093039in}}%
\pgfpathlineto{\pgfqpoint{3.936085in}{4.074295in}}%
\pgfpathlineto{\pgfqpoint{3.948898in}{4.055790in}}%
\pgfpathlineto{\pgfqpoint{3.961709in}{4.037522in}}%
\pgfpathlineto{\pgfqpoint{3.974518in}{4.019489in}}%
\pgfpathlineto{\pgfqpoint{3.981903in}{4.048726in}}%
\pgfpathlineto{\pgfqpoint{3.989288in}{4.078427in}}%
\pgfpathlineto{\pgfqpoint{3.996673in}{4.108600in}}%
\pgfpathlineto{\pgfqpoint{3.983862in}{4.127152in}}%
\pgfpathlineto{\pgfqpoint{3.971048in}{4.145941in}}%
\pgfpathlineto{\pgfqpoint{3.958232in}{4.164967in}}%
\pgfpathlineto{\pgfqpoint{3.945413in}{4.184233in}}%
\pgfpathlineto{\pgfqpoint{3.938032in}{4.153356in}}%
\pgfpathlineto{\pgfqpoint{3.930650in}{4.122961in}}%
\pgfpathlineto{\pgfqpoint{3.923269in}{4.093039in}}%
\pgfpathclose%
\pgfusepath{fill}%
\end{pgfscope}%
\begin{pgfscope}%
\pgfpathrectangle{\pgfqpoint{1.254980in}{0.150000in}}{\pgfqpoint{5.490039in}{5.490039in}}%
\pgfusepath{clip}%
\pgfsetbuttcap%
\pgfsetroundjoin%
\definecolor{currentfill}{rgb}{0.132268,0.655014,0.519661}%
\pgfsetfillcolor{currentfill}%
\pgfsetfillopacity{0.700000}%
\pgfsetlinewidth{0.000000pt}%
\definecolor{currentstroke}{rgb}{0.000000,0.000000,0.000000}%
\pgfsetstrokecolor{currentstroke}%
\pgfsetdash{}{0pt}%
\pgfpathmoveto{\pgfqpoint{3.732058in}{3.911348in}}%
\pgfpathlineto{\pgfqpoint{3.744891in}{3.892548in}}%
\pgfpathlineto{\pgfqpoint{3.757720in}{3.873999in}}%
\pgfpathlineto{\pgfqpoint{3.770546in}{3.855700in}}%
\pgfpathlineto{\pgfqpoint{3.783368in}{3.837648in}}%
\pgfpathlineto{\pgfqpoint{3.790761in}{3.863114in}}%
\pgfpathlineto{\pgfqpoint{3.798152in}{3.888968in}}%
\pgfpathlineto{\pgfqpoint{3.805541in}{3.915216in}}%
\pgfpathlineto{\pgfqpoint{3.812928in}{3.941866in}}%
\pgfpathlineto{\pgfqpoint{3.800104in}{3.960523in}}%
\pgfpathlineto{\pgfqpoint{3.787276in}{3.979428in}}%
\pgfpathlineto{\pgfqpoint{3.774445in}{3.998584in}}%
\pgfpathlineto{\pgfqpoint{3.761611in}{4.017993in}}%
\pgfpathlineto{\pgfqpoint{3.754226in}{3.990724in}}%
\pgfpathlineto{\pgfqpoint{3.746839in}{3.963865in}}%
\pgfpathlineto{\pgfqpoint{3.739450in}{3.937409in}}%
\pgfpathlineto{\pgfqpoint{3.732058in}{3.911348in}}%
\pgfpathclose%
\pgfusepath{fill}%
\end{pgfscope}%
\begin{pgfscope}%
\pgfpathrectangle{\pgfqpoint{1.254980in}{0.150000in}}{\pgfqpoint{5.490039in}{5.490039in}}%
\pgfusepath{clip}%
\pgfsetbuttcap%
\pgfsetroundjoin%
\definecolor{currentfill}{rgb}{0.119699,0.618490,0.536347}%
\pgfsetfillcolor{currentfill}%
\pgfsetfillopacity{0.700000}%
\pgfsetlinewidth{0.000000pt}%
\definecolor{currentstroke}{rgb}{0.000000,0.000000,0.000000}%
\pgfsetstrokecolor{currentstroke}%
\pgfsetdash{}{0pt}%
\pgfpathmoveto{\pgfqpoint{3.702467in}{3.810915in}}%
\pgfpathlineto{\pgfqpoint{3.715298in}{3.792690in}}%
\pgfpathlineto{\pgfqpoint{3.728126in}{3.774715in}}%
\pgfpathlineto{\pgfqpoint{3.740951in}{3.756989in}}%
\pgfpathlineto{\pgfqpoint{3.753773in}{3.739510in}}%
\pgfpathlineto{\pgfqpoint{3.761176in}{3.763499in}}%
\pgfpathlineto{\pgfqpoint{3.768576in}{3.787847in}}%
\pgfpathlineto{\pgfqpoint{3.775973in}{3.812561in}}%
\pgfpathlineto{\pgfqpoint{3.783368in}{3.837648in}}%
\pgfpathlineto{\pgfqpoint{3.770546in}{3.855700in}}%
\pgfpathlineto{\pgfqpoint{3.757720in}{3.873999in}}%
\pgfpathlineto{\pgfqpoint{3.744891in}{3.892548in}}%
\pgfpathlineto{\pgfqpoint{3.732058in}{3.911348in}}%
\pgfpathlineto{\pgfqpoint{3.724665in}{3.885675in}}%
\pgfpathlineto{\pgfqpoint{3.717268in}{3.860383in}}%
\pgfpathlineto{\pgfqpoint{3.709869in}{3.835466in}}%
\pgfpathlineto{\pgfqpoint{3.702467in}{3.810915in}}%
\pgfpathclose%
\pgfusepath{fill}%
\end{pgfscope}%
\begin{pgfscope}%
\pgfpathrectangle{\pgfqpoint{1.254980in}{0.150000in}}{\pgfqpoint{5.490039in}{5.490039in}}%
\pgfusepath{clip}%
\pgfsetbuttcap%
\pgfsetroundjoin%
\definecolor{currentfill}{rgb}{0.144759,0.519093,0.556572}%
\pgfsetfillcolor{currentfill}%
\pgfsetfillopacity{0.700000}%
\pgfsetlinewidth{0.000000pt}%
\definecolor{currentstroke}{rgb}{0.000000,0.000000,0.000000}%
\pgfsetstrokecolor{currentstroke}%
\pgfsetdash{}{0pt}%
\pgfpathmoveto{\pgfqpoint{4.384241in}{3.556862in}}%
\pgfpathlineto{\pgfqpoint{4.397069in}{3.545723in}}%
\pgfpathlineto{\pgfqpoint{4.409899in}{3.534772in}}%
\pgfpathlineto{\pgfqpoint{4.422732in}{3.524007in}}%
\pgfpathlineto{\pgfqpoint{4.435569in}{3.513428in}}%
\pgfpathlineto{\pgfqpoint{4.442947in}{3.536659in}}%
\pgfpathlineto{\pgfqpoint{4.450327in}{3.560281in}}%
\pgfpathlineto{\pgfqpoint{4.457709in}{3.584302in}}%
\pgfpathlineto{\pgfqpoint{4.444876in}{3.595378in}}%
\pgfpathlineto{\pgfqpoint{4.432046in}{3.606640in}}%
\pgfpathlineto{\pgfqpoint{4.419219in}{3.618090in}}%
\pgfpathlineto{\pgfqpoint{4.406394in}{3.629727in}}%
\pgfpathlineto{\pgfqpoint{4.399008in}{3.605034in}}%
\pgfpathlineto{\pgfqpoint{4.391624in}{3.580748in}}%
\pgfpathlineto{\pgfqpoint{4.384241in}{3.556862in}}%
\pgfpathclose%
\pgfusepath{fill}%
\end{pgfscope}%
\begin{pgfscope}%
\pgfpathrectangle{\pgfqpoint{1.254980in}{0.150000in}}{\pgfqpoint{5.490039in}{5.490039in}}%
\pgfusepath{clip}%
\pgfsetbuttcap%
\pgfsetroundjoin%
\definecolor{currentfill}{rgb}{0.196571,0.711827,0.479221}%
\pgfsetfillcolor{currentfill}%
\pgfsetfillopacity{0.700000}%
\pgfsetlinewidth{0.000000pt}%
\definecolor{currentstroke}{rgb}{0.000000,0.000000,0.000000}%
\pgfsetstrokecolor{currentstroke}%
\pgfsetdash{}{0pt}%
\pgfpathmoveto{\pgfqpoint{3.842461in}{4.052632in}}%
\pgfpathlineto{\pgfqpoint{3.855285in}{4.033585in}}%
\pgfpathlineto{\pgfqpoint{3.868106in}{4.014783in}}%
\pgfpathlineto{\pgfqpoint{3.880924in}{3.996224in}}%
\pgfpathlineto{\pgfqpoint{3.893739in}{3.977907in}}%
\pgfpathlineto{\pgfqpoint{3.901123in}{4.006023in}}%
\pgfpathlineto{\pgfqpoint{3.908505in}{4.034578in}}%
\pgfpathlineto{\pgfqpoint{3.915887in}{4.063580in}}%
\pgfpathlineto{\pgfqpoint{3.923269in}{4.093039in}}%
\pgfpathlineto{\pgfqpoint{3.910450in}{4.112023in}}%
\pgfpathlineto{\pgfqpoint{3.897629in}{4.131250in}}%
\pgfpathlineto{\pgfqpoint{3.884804in}{4.150722in}}%
\pgfpathlineto{\pgfqpoint{3.871977in}{4.170440in}}%
\pgfpathlineto{\pgfqpoint{3.864599in}{4.140300in}}%
\pgfpathlineto{\pgfqpoint{3.857221in}{4.110624in}}%
\pgfpathlineto{\pgfqpoint{3.849841in}{4.081404in}}%
\pgfpathlineto{\pgfqpoint{3.842461in}{4.052632in}}%
\pgfpathclose%
\pgfusepath{fill}%
\end{pgfscope}%
\begin{pgfscope}%
\pgfpathrectangle{\pgfqpoint{1.254980in}{0.150000in}}{\pgfqpoint{5.490039in}{5.490039in}}%
\pgfusepath{clip}%
\pgfsetbuttcap%
\pgfsetroundjoin%
\definecolor{currentfill}{rgb}{0.163625,0.471133,0.558148}%
\pgfsetfillcolor{currentfill}%
\pgfsetfillopacity{0.700000}%
\pgfsetlinewidth{0.000000pt}%
\definecolor{currentstroke}{rgb}{0.000000,0.000000,0.000000}%
\pgfsetstrokecolor{currentstroke}%
\pgfsetdash{}{0pt}%
\pgfpathmoveto{\pgfqpoint{4.273919in}{3.420668in}}%
\pgfpathlineto{\pgfqpoint{4.286743in}{3.410017in}}%
\pgfpathlineto{\pgfqpoint{4.299569in}{3.399558in}}%
\pgfpathlineto{\pgfqpoint{4.312399in}{3.389289in}}%
\pgfpathlineto{\pgfqpoint{4.325232in}{3.379208in}}%
\pgfpathlineto{\pgfqpoint{4.332605in}{3.400179in}}%
\pgfpathlineto{\pgfqpoint{4.339979in}{3.421488in}}%
\pgfpathlineto{\pgfqpoint{4.347354in}{3.443142in}}%
\pgfpathlineto{\pgfqpoint{4.354729in}{3.465150in}}%
\pgfpathlineto{\pgfqpoint{4.341900in}{3.475839in}}%
\pgfpathlineto{\pgfqpoint{4.329075in}{3.486717in}}%
\pgfpathlineto{\pgfqpoint{4.316253in}{3.497786in}}%
\pgfpathlineto{\pgfqpoint{4.303433in}{3.509046in}}%
\pgfpathlineto{\pgfqpoint{4.296054in}{3.486418in}}%
\pgfpathlineto{\pgfqpoint{4.288675in}{3.464151in}}%
\pgfpathlineto{\pgfqpoint{4.281297in}{3.442236in}}%
\pgfpathlineto{\pgfqpoint{4.273919in}{3.420668in}}%
\pgfpathclose%
\pgfusepath{fill}%
\end{pgfscope}%
\begin{pgfscope}%
\pgfpathrectangle{\pgfqpoint{1.254980in}{0.150000in}}{\pgfqpoint{5.490039in}{5.490039in}}%
\pgfusepath{clip}%
\pgfsetbuttcap%
\pgfsetroundjoin%
\definecolor{currentfill}{rgb}{0.171176,0.452530,0.557965}%
\pgfsetfillcolor{currentfill}%
\pgfsetfillopacity{0.700000}%
\pgfsetlinewidth{0.000000pt}%
\definecolor{currentstroke}{rgb}{0.000000,0.000000,0.000000}%
\pgfsetstrokecolor{currentstroke}%
\pgfsetdash{}{0pt}%
\pgfpathmoveto{\pgfqpoint{4.193121in}{3.379919in}}%
\pgfpathlineto{\pgfqpoint{4.205939in}{3.369076in}}%
\pgfpathlineto{\pgfqpoint{4.218759in}{3.358427in}}%
\pgfpathlineto{\pgfqpoint{4.231581in}{3.347973in}}%
\pgfpathlineto{\pgfqpoint{4.244407in}{3.337711in}}%
\pgfpathlineto{\pgfqpoint{4.251785in}{3.357967in}}%
\pgfpathlineto{\pgfqpoint{4.259163in}{3.378540in}}%
\pgfpathlineto{\pgfqpoint{4.266541in}{3.399438in}}%
\pgfpathlineto{\pgfqpoint{4.273919in}{3.420668in}}%
\pgfpathlineto{\pgfqpoint{4.261098in}{3.431510in}}%
\pgfpathlineto{\pgfqpoint{4.248279in}{3.442545in}}%
\pgfpathlineto{\pgfqpoint{4.235463in}{3.453775in}}%
\pgfpathlineto{\pgfqpoint{4.222650in}{3.465200in}}%
\pgfpathlineto{\pgfqpoint{4.215268in}{3.443378in}}%
\pgfpathlineto{\pgfqpoint{4.207886in}{3.421895in}}%
\pgfpathlineto{\pgfqpoint{4.200504in}{3.400744in}}%
\pgfpathlineto{\pgfqpoint{4.193121in}{3.379919in}}%
\pgfpathclose%
\pgfusepath{fill}%
\end{pgfscope}%
\begin{pgfscope}%
\pgfpathrectangle{\pgfqpoint{1.254980in}{0.150000in}}{\pgfqpoint{5.490039in}{5.490039in}}%
\pgfusepath{clip}%
\pgfsetbuttcap%
\pgfsetroundjoin%
\definecolor{currentfill}{rgb}{0.123463,0.581687,0.547445}%
\pgfsetfillcolor{currentfill}%
\pgfsetfillopacity{0.700000}%
\pgfsetlinewidth{0.000000pt}%
\definecolor{currentstroke}{rgb}{0.000000,0.000000,0.000000}%
\pgfsetstrokecolor{currentstroke}%
\pgfsetdash{}{0pt}%
\pgfpathmoveto{\pgfqpoint{3.672829in}{3.716250in}}%
\pgfpathlineto{\pgfqpoint{3.685660in}{3.698568in}}%
\pgfpathlineto{\pgfqpoint{3.698488in}{3.681136in}}%
\pgfpathlineto{\pgfqpoint{3.711313in}{3.663952in}}%
\pgfpathlineto{\pgfqpoint{3.724135in}{3.647014in}}%
\pgfpathlineto{\pgfqpoint{3.731549in}{3.669632in}}%
\pgfpathlineto{\pgfqpoint{3.738960in}{3.692583in}}%
\pgfpathlineto{\pgfqpoint{3.746368in}{3.715874in}}%
\pgfpathlineto{\pgfqpoint{3.753773in}{3.739510in}}%
\pgfpathlineto{\pgfqpoint{3.740951in}{3.756989in}}%
\pgfpathlineto{\pgfqpoint{3.728126in}{3.774715in}}%
\pgfpathlineto{\pgfqpoint{3.715298in}{3.792690in}}%
\pgfpathlineto{\pgfqpoint{3.702467in}{3.810915in}}%
\pgfpathlineto{\pgfqpoint{3.695062in}{3.786724in}}%
\pgfpathlineto{\pgfqpoint{3.687654in}{3.762888in}}%
\pgfpathlineto{\pgfqpoint{3.680243in}{3.739399in}}%
\pgfpathlineto{\pgfqpoint{3.672829in}{3.716250in}}%
\pgfpathclose%
\pgfusepath{fill}%
\end{pgfscope}%
\begin{pgfscope}%
\pgfpathrectangle{\pgfqpoint{1.254980in}{0.150000in}}{\pgfqpoint{5.490039in}{5.490039in}}%
\pgfusepath{clip}%
\pgfsetbuttcap%
\pgfsetroundjoin%
\definecolor{currentfill}{rgb}{0.175707,0.697900,0.491033}%
\pgfsetfillcolor{currentfill}%
\pgfsetfillopacity{0.700000}%
\pgfsetlinewidth{0.000000pt}%
\definecolor{currentstroke}{rgb}{0.000000,0.000000,0.000000}%
\pgfsetstrokecolor{currentstroke}%
\pgfsetdash{}{0pt}%
\pgfpathmoveto{\pgfqpoint{3.761611in}{4.017993in}}%
\pgfpathlineto{\pgfqpoint{3.774445in}{3.998584in}}%
\pgfpathlineto{\pgfqpoint{3.787276in}{3.979428in}}%
\pgfpathlineto{\pgfqpoint{3.800104in}{3.960523in}}%
\pgfpathlineto{\pgfqpoint{3.812928in}{3.941866in}}%
\pgfpathlineto{\pgfqpoint{3.820314in}{3.968925in}}%
\pgfpathlineto{\pgfqpoint{3.827698in}{3.996401in}}%
\pgfpathlineto{\pgfqpoint{3.835080in}{4.024300in}}%
\pgfpathlineto{\pgfqpoint{3.842461in}{4.052632in}}%
\pgfpathlineto{\pgfqpoint{3.829634in}{4.071927in}}%
\pgfpathlineto{\pgfqpoint{3.816804in}{4.091471in}}%
\pgfpathlineto{\pgfqpoint{3.803969in}{4.111268in}}%
\pgfpathlineto{\pgfqpoint{3.791131in}{4.131318in}}%
\pgfpathlineto{\pgfqpoint{3.783754in}{4.102333in}}%
\pgfpathlineto{\pgfqpoint{3.776374in}{4.073790in}}%
\pgfpathlineto{\pgfqpoint{3.768993in}{4.045679in}}%
\pgfpathlineto{\pgfqpoint{3.761611in}{4.017993in}}%
\pgfpathclose%
\pgfusepath{fill}%
\end{pgfscope}%
\begin{pgfscope}%
\pgfpathrectangle{\pgfqpoint{1.254980in}{0.150000in}}{\pgfqpoint{5.490039in}{5.490039in}}%
\pgfusepath{clip}%
\pgfsetbuttcap%
\pgfsetroundjoin%
\definecolor{currentfill}{rgb}{0.175841,0.441290,0.557685}%
\pgfsetfillcolor{currentfill}%
\pgfsetfillopacity{0.700000}%
\pgfsetlinewidth{0.000000pt}%
\definecolor{currentstroke}{rgb}{0.000000,0.000000,0.000000}%
\pgfsetstrokecolor{currentstroke}%
\pgfsetdash{}{0pt}%
\pgfpathmoveto{\pgfqpoint{3.980268in}{3.357006in}}%
\pgfpathlineto{\pgfqpoint{3.993073in}{3.344816in}}%
\pgfpathlineto{\pgfqpoint{4.005878in}{3.332834in}}%
\pgfpathlineto{\pgfqpoint{4.018685in}{3.321060in}}%
\pgfpathlineto{\pgfqpoint{4.031493in}{3.309492in}}%
\pgfpathlineto{\pgfqpoint{4.038894in}{3.328981in}}%
\pgfpathlineto{\pgfqpoint{4.046293in}{3.348759in}}%
\pgfpathlineto{\pgfqpoint{4.053690in}{3.368830in}}%
\pgfpathlineto{\pgfqpoint{4.061085in}{3.389201in}}%
\pgfpathlineto{\pgfqpoint{4.048281in}{3.401296in}}%
\pgfpathlineto{\pgfqpoint{4.035478in}{3.413597in}}%
\pgfpathlineto{\pgfqpoint{4.022675in}{3.426106in}}%
\pgfpathlineto{\pgfqpoint{4.009874in}{3.438825in}}%
\pgfpathlineto{\pgfqpoint{4.002475in}{3.417915in}}%
\pgfpathlineto{\pgfqpoint{3.995075in}{3.397313in}}%
\pgfpathlineto{\pgfqpoint{3.987672in}{3.377012in}}%
\pgfpathlineto{\pgfqpoint{3.980268in}{3.357006in}}%
\pgfpathclose%
\pgfusepath{fill}%
\end{pgfscope}%
\begin{pgfscope}%
\pgfpathrectangle{\pgfqpoint{1.254980in}{0.150000in}}{\pgfqpoint{5.490039in}{5.490039in}}%
\pgfusepath{clip}%
\pgfsetbuttcap%
\pgfsetroundjoin%
\definecolor{currentfill}{rgb}{0.154815,0.493313,0.557840}%
\pgfsetfillcolor{currentfill}%
\pgfsetfillopacity{0.700000}%
\pgfsetlinewidth{0.000000pt}%
\definecolor{currentstroke}{rgb}{0.000000,0.000000,0.000000}%
\pgfsetstrokecolor{currentstroke}%
\pgfsetdash{}{0pt}%
\pgfpathmoveto{\pgfqpoint{3.745717in}{3.496443in}}%
\pgfpathlineto{\pgfqpoint{3.758530in}{3.481211in}}%
\pgfpathlineto{\pgfqpoint{3.771341in}{3.466212in}}%
\pgfpathlineto{\pgfqpoint{3.784150in}{3.451445in}}%
\pgfpathlineto{\pgfqpoint{3.796959in}{3.436908in}}%
\pgfpathlineto{\pgfqpoint{3.804382in}{3.457265in}}%
\pgfpathlineto{\pgfqpoint{3.811802in}{3.477916in}}%
\pgfpathlineto{\pgfqpoint{3.819219in}{3.498865in}}%
\pgfpathlineto{\pgfqpoint{3.826633in}{3.520120in}}%
\pgfpathlineto{\pgfqpoint{3.813827in}{3.535162in}}%
\pgfpathlineto{\pgfqpoint{3.801019in}{3.550434in}}%
\pgfpathlineto{\pgfqpoint{3.788210in}{3.565939in}}%
\pgfpathlineto{\pgfqpoint{3.775399in}{3.581678in}}%
\pgfpathlineto{\pgfqpoint{3.767983in}{3.559906in}}%
\pgfpathlineto{\pgfqpoint{3.760564in}{3.538447in}}%
\pgfpathlineto{\pgfqpoint{3.753142in}{3.517295in}}%
\pgfpathlineto{\pgfqpoint{3.745717in}{3.496443in}}%
\pgfpathclose%
\pgfusepath{fill}%
\end{pgfscope}%
\begin{pgfscope}%
\pgfpathrectangle{\pgfqpoint{1.254980in}{0.150000in}}{\pgfqpoint{5.490039in}{5.490039in}}%
\pgfusepath{clip}%
\pgfsetbuttcap%
\pgfsetroundjoin%
\definecolor{currentfill}{rgb}{0.163625,0.471133,0.558148}%
\pgfsetfillcolor{currentfill}%
\pgfsetfillopacity{0.700000}%
\pgfsetlinewidth{0.000000pt}%
\definecolor{currentstroke}{rgb}{0.000000,0.000000,0.000000}%
\pgfsetstrokecolor{currentstroke}%
\pgfsetdash{}{0pt}%
\pgfpathmoveto{\pgfqpoint{3.796959in}{3.436908in}}%
\pgfpathlineto{\pgfqpoint{3.809766in}{3.422599in}}%
\pgfpathlineto{\pgfqpoint{3.822572in}{3.408518in}}%
\pgfpathlineto{\pgfqpoint{3.835378in}{3.394661in}}%
\pgfpathlineto{\pgfqpoint{3.848183in}{3.381027in}}%
\pgfpathlineto{\pgfqpoint{3.855604in}{3.400892in}}%
\pgfpathlineto{\pgfqpoint{3.863022in}{3.421043in}}%
\pgfpathlineto{\pgfqpoint{3.870437in}{3.441485in}}%
\pgfpathlineto{\pgfqpoint{3.877849in}{3.462225in}}%
\pgfpathlineto{\pgfqpoint{3.865046in}{3.476361in}}%
\pgfpathlineto{\pgfqpoint{3.852243in}{3.490721in}}%
\pgfpathlineto{\pgfqpoint{3.839439in}{3.505307in}}%
\pgfpathlineto{\pgfqpoint{3.826633in}{3.520120in}}%
\pgfpathlineto{\pgfqpoint{3.819219in}{3.498865in}}%
\pgfpathlineto{\pgfqpoint{3.811802in}{3.477916in}}%
\pgfpathlineto{\pgfqpoint{3.804382in}{3.457265in}}%
\pgfpathlineto{\pgfqpoint{3.796959in}{3.436908in}}%
\pgfpathclose%
\pgfusepath{fill}%
\end{pgfscope}%
\begin{pgfscope}%
\pgfpathrectangle{\pgfqpoint{1.254980in}{0.150000in}}{\pgfqpoint{5.490039in}{5.490039in}}%
\pgfusepath{clip}%
\pgfsetbuttcap%
\pgfsetroundjoin%
\definecolor{currentfill}{rgb}{0.156270,0.489624,0.557936}%
\pgfsetfillcolor{currentfill}%
\pgfsetfillopacity{0.700000}%
\pgfsetlinewidth{0.000000pt}%
\definecolor{currentstroke}{rgb}{0.000000,0.000000,0.000000}%
\pgfsetstrokecolor{currentstroke}%
\pgfsetdash{}{0pt}%
\pgfpathmoveto{\pgfqpoint{4.354729in}{3.465150in}}%
\pgfpathlineto{\pgfqpoint{4.367560in}{3.454649in}}%
\pgfpathlineto{\pgfqpoint{4.380395in}{3.444336in}}%
\pgfpathlineto{\pgfqpoint{4.393233in}{3.434209in}}%
\pgfpathlineto{\pgfqpoint{4.406074in}{3.424267in}}%
\pgfpathlineto{\pgfqpoint{4.413446in}{3.446009in}}%
\pgfpathlineto{\pgfqpoint{4.420818in}{3.468111in}}%
\pgfpathlineto{\pgfqpoint{4.428193in}{3.490582in}}%
\pgfpathlineto{\pgfqpoint{4.435569in}{3.513428in}}%
\pgfpathlineto{\pgfqpoint{4.422732in}{3.524007in}}%
\pgfpathlineto{\pgfqpoint{4.409899in}{3.534772in}}%
\pgfpathlineto{\pgfqpoint{4.397069in}{3.545723in}}%
\pgfpathlineto{\pgfqpoint{4.384241in}{3.556862in}}%
\pgfpathlineto{\pgfqpoint{4.376861in}{3.533366in}}%
\pgfpathlineto{\pgfqpoint{4.369482in}{3.510254in}}%
\pgfpathlineto{\pgfqpoint{4.362105in}{3.487518in}}%
\pgfpathlineto{\pgfqpoint{4.354729in}{3.465150in}}%
\pgfpathclose%
\pgfusepath{fill}%
\end{pgfscope}%
\begin{pgfscope}%
\pgfpathrectangle{\pgfqpoint{1.254980in}{0.150000in}}{\pgfqpoint{5.490039in}{5.490039in}}%
\pgfusepath{clip}%
\pgfsetbuttcap%
\pgfsetroundjoin%
\definecolor{currentfill}{rgb}{0.177423,0.437527,0.557565}%
\pgfsetfillcolor{currentfill}%
\pgfsetfillopacity{0.700000}%
\pgfsetlinewidth{0.000000pt}%
\definecolor{currentstroke}{rgb}{0.000000,0.000000,0.000000}%
\pgfsetstrokecolor{currentstroke}%
\pgfsetdash{}{0pt}%
\pgfpathmoveto{\pgfqpoint{4.112318in}{3.342864in}}%
\pgfpathlineto{\pgfqpoint{4.125131in}{3.331784in}}%
\pgfpathlineto{\pgfqpoint{4.137946in}{3.320903in}}%
\pgfpathlineto{\pgfqpoint{4.150763in}{3.310220in}}%
\pgfpathlineto{\pgfqpoint{4.163582in}{3.299734in}}%
\pgfpathlineto{\pgfqpoint{4.170968in}{3.319325in}}%
\pgfpathlineto{\pgfqpoint{4.178354in}{3.339216in}}%
\pgfpathlineto{\pgfqpoint{4.185738in}{3.359411in}}%
\pgfpathlineto{\pgfqpoint{4.193121in}{3.379919in}}%
\pgfpathlineto{\pgfqpoint{4.180306in}{3.390958in}}%
\pgfpathlineto{\pgfqpoint{4.167493in}{3.402194in}}%
\pgfpathlineto{\pgfqpoint{4.154682in}{3.413629in}}%
\pgfpathlineto{\pgfqpoint{4.141873in}{3.425263in}}%
\pgfpathlineto{\pgfqpoint{4.134486in}{3.404191in}}%
\pgfpathlineto{\pgfqpoint{4.127098in}{3.383439in}}%
\pgfpathlineto{\pgfqpoint{4.119709in}{3.362999in}}%
\pgfpathlineto{\pgfqpoint{4.112318in}{3.342864in}}%
\pgfpathclose%
\pgfusepath{fill}%
\end{pgfscope}%
\begin{pgfscope}%
\pgfpathrectangle{\pgfqpoint{1.254980in}{0.150000in}}{\pgfqpoint{5.490039in}{5.490039in}}%
\pgfusepath{clip}%
\pgfsetbuttcap%
\pgfsetroundjoin%
\definecolor{currentfill}{rgb}{0.144759,0.519093,0.556572}%
\pgfsetfillcolor{currentfill}%
\pgfsetfillopacity{0.700000}%
\pgfsetlinewidth{0.000000pt}%
\definecolor{currentstroke}{rgb}{0.000000,0.000000,0.000000}%
\pgfsetstrokecolor{currentstroke}%
\pgfsetdash{}{0pt}%
\pgfpathmoveto{\pgfqpoint{3.694447in}{3.559746in}}%
\pgfpathlineto{\pgfqpoint{3.707268in}{3.543561in}}%
\pgfpathlineto{\pgfqpoint{3.720086in}{3.527616in}}%
\pgfpathlineto{\pgfqpoint{3.732903in}{3.511911in}}%
\pgfpathlineto{\pgfqpoint{3.745717in}{3.496443in}}%
\pgfpathlineto{\pgfqpoint{3.753142in}{3.517295in}}%
\pgfpathlineto{\pgfqpoint{3.760564in}{3.538447in}}%
\pgfpathlineto{\pgfqpoint{3.767983in}{3.559906in}}%
\pgfpathlineto{\pgfqpoint{3.775399in}{3.581678in}}%
\pgfpathlineto{\pgfqpoint{3.762586in}{3.597653in}}%
\pgfpathlineto{\pgfqpoint{3.749772in}{3.613866in}}%
\pgfpathlineto{\pgfqpoint{3.736955in}{3.630319in}}%
\pgfpathlineto{\pgfqpoint{3.724135in}{3.647014in}}%
\pgfpathlineto{\pgfqpoint{3.716718in}{3.624722in}}%
\pgfpathlineto{\pgfqpoint{3.709298in}{3.602751in}}%
\pgfpathlineto{\pgfqpoint{3.701874in}{3.581094in}}%
\pgfpathlineto{\pgfqpoint{3.694447in}{3.559746in}}%
\pgfpathclose%
\pgfusepath{fill}%
\end{pgfscope}%
\begin{pgfscope}%
\pgfpathrectangle{\pgfqpoint{1.254980in}{0.150000in}}{\pgfqpoint{5.490039in}{5.490039in}}%
\pgfusepath{clip}%
\pgfsetbuttcap%
\pgfsetroundjoin%
\definecolor{currentfill}{rgb}{0.171176,0.452530,0.557965}%
\pgfsetfillcolor{currentfill}%
\pgfsetfillopacity{0.700000}%
\pgfsetlinewidth{0.000000pt}%
\definecolor{currentstroke}{rgb}{0.000000,0.000000,0.000000}%
\pgfsetstrokecolor{currentstroke}%
\pgfsetdash{}{0pt}%
\pgfpathmoveto{\pgfqpoint{3.848183in}{3.381027in}}%
\pgfpathlineto{\pgfqpoint{3.860988in}{3.367615in}}%
\pgfpathlineto{\pgfqpoint{3.873793in}{3.354424in}}%
\pgfpathlineto{\pgfqpoint{3.886598in}{3.341451in}}%
\pgfpathlineto{\pgfqpoint{3.899403in}{3.328695in}}%
\pgfpathlineto{\pgfqpoint{3.906820in}{3.348069in}}%
\pgfpathlineto{\pgfqpoint{3.914235in}{3.367722in}}%
\pgfpathlineto{\pgfqpoint{3.921647in}{3.387659in}}%
\pgfpathlineto{\pgfqpoint{3.929057in}{3.407887in}}%
\pgfpathlineto{\pgfqpoint{3.916255in}{3.421143in}}%
\pgfpathlineto{\pgfqpoint{3.903453in}{3.434617in}}%
\pgfpathlineto{\pgfqpoint{3.890651in}{3.448311in}}%
\pgfpathlineto{\pgfqpoint{3.877849in}{3.462225in}}%
\pgfpathlineto{\pgfqpoint{3.870437in}{3.441485in}}%
\pgfpathlineto{\pgfqpoint{3.863022in}{3.421043in}}%
\pgfpathlineto{\pgfqpoint{3.855604in}{3.400892in}}%
\pgfpathlineto{\pgfqpoint{3.848183in}{3.381027in}}%
\pgfpathclose%
\pgfusepath{fill}%
\end{pgfscope}%
\begin{pgfscope}%
\pgfpathrectangle{\pgfqpoint{1.254980in}{0.150000in}}{\pgfqpoint{5.490039in}{5.490039in}}%
\pgfusepath{clip}%
\pgfsetbuttcap%
\pgfsetroundjoin%
\definecolor{currentfill}{rgb}{0.266941,0.748751,0.440573}%
\pgfsetfillcolor{currentfill}%
\pgfsetfillopacity{0.700000}%
\pgfsetlinewidth{0.000000pt}%
\definecolor{currentstroke}{rgb}{0.000000,0.000000,0.000000}%
\pgfsetstrokecolor{currentstroke}%
\pgfsetdash{}{0pt}%
\pgfpathmoveto{\pgfqpoint{3.871977in}{4.170440in}}%
\pgfpathlineto{\pgfqpoint{3.884804in}{4.150722in}}%
\pgfpathlineto{\pgfqpoint{3.897629in}{4.131250in}}%
\pgfpathlineto{\pgfqpoint{3.910450in}{4.112023in}}%
\pgfpathlineto{\pgfqpoint{3.923269in}{4.093039in}}%
\pgfpathlineto{\pgfqpoint{3.930650in}{4.122961in}}%
\pgfpathlineto{\pgfqpoint{3.938032in}{4.153356in}}%
\pgfpathlineto{\pgfqpoint{3.945413in}{4.184233in}}%
\pgfpathlineto{\pgfqpoint{3.932592in}{4.203740in}}%
\pgfpathlineto{\pgfqpoint{3.919767in}{4.223491in}}%
\pgfpathlineto{\pgfqpoint{3.906939in}{4.243487in}}%
\pgfpathlineto{\pgfqpoint{3.894108in}{4.263731in}}%
\pgfpathlineto{\pgfqpoint{3.886731in}{4.232147in}}%
\pgfpathlineto{\pgfqpoint{3.879354in}{4.201053in}}%
\pgfpathlineto{\pgfqpoint{3.871977in}{4.170440in}}%
\pgfpathclose%
\pgfusepath{fill}%
\end{pgfscope}%
\begin{pgfscope}%
\pgfpathrectangle{\pgfqpoint{1.254980in}{0.150000in}}{\pgfqpoint{5.490039in}{5.490039in}}%
\pgfusepath{clip}%
\pgfsetbuttcap%
\pgfsetroundjoin%
\definecolor{currentfill}{rgb}{0.128087,0.647749,0.523491}%
\pgfsetfillcolor{currentfill}%
\pgfsetfillopacity{0.700000}%
\pgfsetlinewidth{0.000000pt}%
\definecolor{currentstroke}{rgb}{0.000000,0.000000,0.000000}%
\pgfsetstrokecolor{currentstroke}%
\pgfsetdash{}{0pt}%
\pgfpathmoveto{\pgfqpoint{3.651104in}{3.886365in}}%
\pgfpathlineto{\pgfqpoint{3.663951in}{3.867116in}}%
\pgfpathlineto{\pgfqpoint{3.676793in}{3.848126in}}%
\pgfpathlineto{\pgfqpoint{3.689632in}{3.829393in}}%
\pgfpathlineto{\pgfqpoint{3.702467in}{3.810915in}}%
\pgfpathlineto{\pgfqpoint{3.709869in}{3.835466in}}%
\pgfpathlineto{\pgfqpoint{3.717268in}{3.860383in}}%
\pgfpathlineto{\pgfqpoint{3.724665in}{3.885675in}}%
\pgfpathlineto{\pgfqpoint{3.732058in}{3.911348in}}%
\pgfpathlineto{\pgfqpoint{3.719222in}{3.930403in}}%
\pgfpathlineto{\pgfqpoint{3.706382in}{3.949713in}}%
\pgfpathlineto{\pgfqpoint{3.693538in}{3.969281in}}%
\pgfpathlineto{\pgfqpoint{3.680690in}{3.989110in}}%
\pgfpathlineto{\pgfqpoint{3.673298in}{3.962847in}}%
\pgfpathlineto{\pgfqpoint{3.665903in}{3.936973in}}%
\pgfpathlineto{\pgfqpoint{3.658505in}{3.911482in}}%
\pgfpathlineto{\pgfqpoint{3.651104in}{3.886365in}}%
\pgfpathclose%
\pgfusepath{fill}%
\end{pgfscope}%
\begin{pgfscope}%
\pgfpathrectangle{\pgfqpoint{1.254980in}{0.150000in}}{\pgfqpoint{5.490039in}{5.490039in}}%
\pgfusepath{clip}%
\pgfsetbuttcap%
\pgfsetroundjoin%
\definecolor{currentfill}{rgb}{0.150476,0.504369,0.557430}%
\pgfsetfillcolor{currentfill}%
\pgfsetfillopacity{0.700000}%
\pgfsetlinewidth{0.000000pt}%
\definecolor{currentstroke}{rgb}{0.000000,0.000000,0.000000}%
\pgfsetstrokecolor{currentstroke}%
\pgfsetdash{}{0pt}%
\pgfpathmoveto{\pgfqpoint{4.435569in}{3.513428in}}%
\pgfpathlineto{\pgfqpoint{4.448409in}{3.503035in}}%
\pgfpathlineto{\pgfqpoint{4.461252in}{3.492825in}}%
\pgfpathlineto{\pgfqpoint{4.474100in}{3.482798in}}%
\pgfpathlineto{\pgfqpoint{4.486951in}{3.472953in}}%
\pgfpathlineto{\pgfqpoint{4.494324in}{3.495529in}}%
\pgfpathlineto{\pgfqpoint{4.501699in}{3.518488in}}%
\pgfpathlineto{\pgfqpoint{4.509077in}{3.541840in}}%
\pgfpathlineto{\pgfqpoint{4.496229in}{3.552181in}}%
\pgfpathlineto{\pgfqpoint{4.483386in}{3.562704in}}%
\pgfpathlineto{\pgfqpoint{4.470546in}{3.573411in}}%
\pgfpathlineto{\pgfqpoint{4.457709in}{3.584302in}}%
\pgfpathlineto{\pgfqpoint{4.450327in}{3.560281in}}%
\pgfpathlineto{\pgfqpoint{4.442947in}{3.536659in}}%
\pgfpathlineto{\pgfqpoint{4.435569in}{3.513428in}}%
\pgfpathclose%
\pgfusepath{fill}%
\end{pgfscope}%
\begin{pgfscope}%
\pgfpathrectangle{\pgfqpoint{1.254980in}{0.150000in}}{\pgfqpoint{5.490039in}{5.490039in}}%
\pgfusepath{clip}%
\pgfsetbuttcap%
\pgfsetroundjoin%
\definecolor{currentfill}{rgb}{0.162016,0.687316,0.499129}%
\pgfsetfillcolor{currentfill}%
\pgfsetfillopacity{0.700000}%
\pgfsetlinewidth{0.000000pt}%
\definecolor{currentstroke}{rgb}{0.000000,0.000000,0.000000}%
\pgfsetstrokecolor{currentstroke}%
\pgfsetdash{}{0pt}%
\pgfpathmoveto{\pgfqpoint{3.680690in}{3.989110in}}%
\pgfpathlineto{\pgfqpoint{3.693538in}{3.969281in}}%
\pgfpathlineto{\pgfqpoint{3.706382in}{3.949713in}}%
\pgfpathlineto{\pgfqpoint{3.719222in}{3.930403in}}%
\pgfpathlineto{\pgfqpoint{3.732058in}{3.911348in}}%
\pgfpathlineto{\pgfqpoint{3.739450in}{3.937409in}}%
\pgfpathlineto{\pgfqpoint{3.746839in}{3.963865in}}%
\pgfpathlineto{\pgfqpoint{3.754226in}{3.990724in}}%
\pgfpathlineto{\pgfqpoint{3.761611in}{4.017993in}}%
\pgfpathlineto{\pgfqpoint{3.748772in}{4.037656in}}%
\pgfpathlineto{\pgfqpoint{3.735930in}{4.057576in}}%
\pgfpathlineto{\pgfqpoint{3.723083in}{4.077756in}}%
\pgfpathlineto{\pgfqpoint{3.710232in}{4.098198in}}%
\pgfpathlineto{\pgfqpoint{3.702850in}{4.070306in}}%
\pgfpathlineto{\pgfqpoint{3.695466in}{4.042832in}}%
\pgfpathlineto{\pgfqpoint{3.688079in}{4.015769in}}%
\pgfpathlineto{\pgfqpoint{3.680690in}{3.989110in}}%
\pgfpathclose%
\pgfusepath{fill}%
\end{pgfscope}%
\begin{pgfscope}%
\pgfpathrectangle{\pgfqpoint{1.254980in}{0.150000in}}{\pgfqpoint{5.490039in}{5.490039in}}%
\pgfusepath{clip}%
\pgfsetbuttcap%
\pgfsetroundjoin%
\definecolor{currentfill}{rgb}{0.135066,0.544853,0.554029}%
\pgfsetfillcolor{currentfill}%
\pgfsetfillopacity{0.700000}%
\pgfsetlinewidth{0.000000pt}%
\definecolor{currentstroke}{rgb}{0.000000,0.000000,0.000000}%
\pgfsetstrokecolor{currentstroke}%
\pgfsetdash{}{0pt}%
\pgfpathmoveto{\pgfqpoint{3.643137in}{3.626938in}}%
\pgfpathlineto{\pgfqpoint{3.655969in}{3.609768in}}%
\pgfpathlineto{\pgfqpoint{3.668798in}{3.592848in}}%
\pgfpathlineto{\pgfqpoint{3.681624in}{3.576174in}}%
\pgfpathlineto{\pgfqpoint{3.694447in}{3.559746in}}%
\pgfpathlineto{\pgfqpoint{3.701874in}{3.581094in}}%
\pgfpathlineto{\pgfqpoint{3.709298in}{3.602751in}}%
\pgfpathlineto{\pgfqpoint{3.716718in}{3.624722in}}%
\pgfpathlineto{\pgfqpoint{3.724135in}{3.647014in}}%
\pgfpathlineto{\pgfqpoint{3.711313in}{3.663952in}}%
\pgfpathlineto{\pgfqpoint{3.698488in}{3.681136in}}%
\pgfpathlineto{\pgfqpoint{3.685660in}{3.698568in}}%
\pgfpathlineto{\pgfqpoint{3.672829in}{3.716250in}}%
\pgfpathlineto{\pgfqpoint{3.665411in}{3.693436in}}%
\pgfpathlineto{\pgfqpoint{3.657990in}{3.670950in}}%
\pgfpathlineto{\pgfqpoint{3.650565in}{3.648786in}}%
\pgfpathlineto{\pgfqpoint{3.643137in}{3.626938in}}%
\pgfpathclose%
\pgfusepath{fill}%
\end{pgfscope}%
\begin{pgfscope}%
\pgfpathrectangle{\pgfqpoint{1.254980in}{0.150000in}}{\pgfqpoint{5.490039in}{5.490039in}}%
\pgfusepath{clip}%
\pgfsetbuttcap%
\pgfsetroundjoin%
\definecolor{currentfill}{rgb}{0.182256,0.426184,0.557120}%
\pgfsetfillcolor{currentfill}%
\pgfsetfillopacity{0.700000}%
\pgfsetlinewidth{0.000000pt}%
\definecolor{currentstroke}{rgb}{0.000000,0.000000,0.000000}%
\pgfsetstrokecolor{currentstroke}%
\pgfsetdash{}{0pt}%
\pgfpathmoveto{\pgfqpoint{4.031493in}{3.309492in}}%
\pgfpathlineto{\pgfqpoint{4.044302in}{3.298129in}}%
\pgfpathlineto{\pgfqpoint{4.057113in}{3.286970in}}%
\pgfpathlineto{\pgfqpoint{4.069926in}{3.276013in}}%
\pgfpathlineto{\pgfqpoint{4.082741in}{3.265258in}}%
\pgfpathlineto{\pgfqpoint{4.090138in}{3.284232in}}%
\pgfpathlineto{\pgfqpoint{4.097533in}{3.303488in}}%
\pgfpathlineto{\pgfqpoint{4.104926in}{3.323029in}}%
\pgfpathlineto{\pgfqpoint{4.112318in}{3.342864in}}%
\pgfpathlineto{\pgfqpoint{4.099508in}{3.354145in}}%
\pgfpathlineto{\pgfqpoint{4.086699in}{3.365627in}}%
\pgfpathlineto{\pgfqpoint{4.073891in}{3.377312in}}%
\pgfpathlineto{\pgfqpoint{4.061085in}{3.389201in}}%
\pgfpathlineto{\pgfqpoint{4.053690in}{3.368830in}}%
\pgfpathlineto{\pgfqpoint{4.046293in}{3.348759in}}%
\pgfpathlineto{\pgfqpoint{4.038894in}{3.328981in}}%
\pgfpathlineto{\pgfqpoint{4.031493in}{3.309492in}}%
\pgfpathclose%
\pgfusepath{fill}%
\end{pgfscope}%
\begin{pgfscope}%
\pgfpathrectangle{\pgfqpoint{1.254980in}{0.150000in}}{\pgfqpoint{5.490039in}{5.490039in}}%
\pgfusepath{clip}%
\pgfsetbuttcap%
\pgfsetroundjoin%
\definecolor{currentfill}{rgb}{0.252899,0.742211,0.448284}%
\pgfsetfillcolor{currentfill}%
\pgfsetfillopacity{0.700000}%
\pgfsetlinewidth{0.000000pt}%
\definecolor{currentstroke}{rgb}{0.000000,0.000000,0.000000}%
\pgfsetstrokecolor{currentstroke}%
\pgfsetdash{}{0pt}%
\pgfpathmoveto{\pgfqpoint{3.791131in}{4.131318in}}%
\pgfpathlineto{\pgfqpoint{3.803969in}{4.111268in}}%
\pgfpathlineto{\pgfqpoint{3.816804in}{4.091471in}}%
\pgfpathlineto{\pgfqpoint{3.829634in}{4.071927in}}%
\pgfpathlineto{\pgfqpoint{3.842461in}{4.052632in}}%
\pgfpathlineto{\pgfqpoint{3.849841in}{4.081404in}}%
\pgfpathlineto{\pgfqpoint{3.857221in}{4.110624in}}%
\pgfpathlineto{\pgfqpoint{3.864599in}{4.140300in}}%
\pgfpathlineto{\pgfqpoint{3.871977in}{4.170440in}}%
\pgfpathlineto{\pgfqpoint{3.859146in}{4.190407in}}%
\pgfpathlineto{\pgfqpoint{3.846311in}{4.210624in}}%
\pgfpathlineto{\pgfqpoint{3.833473in}{4.231095in}}%
\pgfpathlineto{\pgfqpoint{3.820630in}{4.251821in}}%
\pgfpathlineto{\pgfqpoint{3.813257in}{4.220994in}}%
\pgfpathlineto{\pgfqpoint{3.805883in}{4.190640in}}%
\pgfpathlineto{\pgfqpoint{3.798508in}{4.160750in}}%
\pgfpathlineto{\pgfqpoint{3.791131in}{4.131318in}}%
\pgfpathclose%
\pgfusepath{fill}%
\end{pgfscope}%
\begin{pgfscope}%
\pgfpathrectangle{\pgfqpoint{1.254980in}{0.150000in}}{\pgfqpoint{5.490039in}{5.490039in}}%
\pgfusepath{clip}%
\pgfsetbuttcap%
\pgfsetroundjoin%
\definecolor{currentfill}{rgb}{0.119423,0.611141,0.538982}%
\pgfsetfillcolor{currentfill}%
\pgfsetfillopacity{0.700000}%
\pgfsetlinewidth{0.000000pt}%
\definecolor{currentstroke}{rgb}{0.000000,0.000000,0.000000}%
\pgfsetstrokecolor{currentstroke}%
\pgfsetdash{}{0pt}%
\pgfpathmoveto{\pgfqpoint{3.621468in}{3.789518in}}%
\pgfpathlineto{\pgfqpoint{3.634314in}{3.770815in}}%
\pgfpathlineto{\pgfqpoint{3.647156in}{3.752371in}}%
\pgfpathlineto{\pgfqpoint{3.659994in}{3.734184in}}%
\pgfpathlineto{\pgfqpoint{3.672829in}{3.716250in}}%
\pgfpathlineto{\pgfqpoint{3.680243in}{3.739399in}}%
\pgfpathlineto{\pgfqpoint{3.687654in}{3.762888in}}%
\pgfpathlineto{\pgfqpoint{3.695062in}{3.786724in}}%
\pgfpathlineto{\pgfqpoint{3.702467in}{3.810915in}}%
\pgfpathlineto{\pgfqpoint{3.689632in}{3.829393in}}%
\pgfpathlineto{\pgfqpoint{3.676793in}{3.848126in}}%
\pgfpathlineto{\pgfqpoint{3.663951in}{3.867116in}}%
\pgfpathlineto{\pgfqpoint{3.651104in}{3.886365in}}%
\pgfpathlineto{\pgfqpoint{3.643700in}{3.861617in}}%
\pgfpathlineto{\pgfqpoint{3.636293in}{3.837231in}}%
\pgfpathlineto{\pgfqpoint{3.628882in}{3.813200in}}%
\pgfpathlineto{\pgfqpoint{3.621468in}{3.789518in}}%
\pgfpathclose%
\pgfusepath{fill}%
\end{pgfscope}%
\begin{pgfscope}%
\pgfpathrectangle{\pgfqpoint{1.254980in}{0.150000in}}{\pgfqpoint{5.490039in}{5.490039in}}%
\pgfusepath{clip}%
\pgfsetbuttcap%
\pgfsetroundjoin%
\definecolor{currentfill}{rgb}{0.180629,0.429975,0.557282}%
\pgfsetfillcolor{currentfill}%
\pgfsetfillopacity{0.700000}%
\pgfsetlinewidth{0.000000pt}%
\definecolor{currentstroke}{rgb}{0.000000,0.000000,0.000000}%
\pgfsetstrokecolor{currentstroke}%
\pgfsetdash{}{0pt}%
\pgfpathmoveto{\pgfqpoint{3.899403in}{3.328695in}}%
\pgfpathlineto{\pgfqpoint{3.912208in}{3.316154in}}%
\pgfpathlineto{\pgfqpoint{3.925014in}{3.303828in}}%
\pgfpathlineto{\pgfqpoint{3.937820in}{3.291715in}}%
\pgfpathlineto{\pgfqpoint{3.950627in}{3.279813in}}%
\pgfpathlineto{\pgfqpoint{3.958041in}{3.298699in}}%
\pgfpathlineto{\pgfqpoint{3.965453in}{3.317856in}}%
\pgfpathlineto{\pgfqpoint{3.972861in}{3.337290in}}%
\pgfpathlineto{\pgfqpoint{3.980268in}{3.357006in}}%
\pgfpathlineto{\pgfqpoint{3.967464in}{3.369407in}}%
\pgfpathlineto{\pgfqpoint{3.954661in}{3.382020in}}%
\pgfpathlineto{\pgfqpoint{3.941859in}{3.394846in}}%
\pgfpathlineto{\pgfqpoint{3.929057in}{3.407887in}}%
\pgfpathlineto{\pgfqpoint{3.921647in}{3.387659in}}%
\pgfpathlineto{\pgfqpoint{3.914235in}{3.367722in}}%
\pgfpathlineto{\pgfqpoint{3.906820in}{3.348069in}}%
\pgfpathlineto{\pgfqpoint{3.899403in}{3.328695in}}%
\pgfpathclose%
\pgfusepath{fill}%
\end{pgfscope}%
\begin{pgfscope}%
\pgfpathrectangle{\pgfqpoint{1.254980in}{0.150000in}}{\pgfqpoint{5.490039in}{5.490039in}}%
\pgfusepath{clip}%
\pgfsetbuttcap%
\pgfsetroundjoin%
\definecolor{currentfill}{rgb}{0.169646,0.456262,0.558030}%
\pgfsetfillcolor{currentfill}%
\pgfsetfillopacity{0.700000}%
\pgfsetlinewidth{0.000000pt}%
\definecolor{currentstroke}{rgb}{0.000000,0.000000,0.000000}%
\pgfsetstrokecolor{currentstroke}%
\pgfsetdash{}{0pt}%
\pgfpathmoveto{\pgfqpoint{4.325232in}{3.379208in}}%
\pgfpathlineto{\pgfqpoint{4.338068in}{3.369316in}}%
\pgfpathlineto{\pgfqpoint{4.350908in}{3.359610in}}%
\pgfpathlineto{\pgfqpoint{4.363751in}{3.350091in}}%
\pgfpathlineto{\pgfqpoint{4.376598in}{3.340756in}}%
\pgfpathlineto{\pgfqpoint{4.383966in}{3.361130in}}%
\pgfpathlineto{\pgfqpoint{4.391335in}{3.381835in}}%
\pgfpathlineto{\pgfqpoint{4.398704in}{3.402878in}}%
\pgfpathlineto{\pgfqpoint{4.406074in}{3.424267in}}%
\pgfpathlineto{\pgfqpoint{4.393233in}{3.434209in}}%
\pgfpathlineto{\pgfqpoint{4.380395in}{3.444336in}}%
\pgfpathlineto{\pgfqpoint{4.367560in}{3.454649in}}%
\pgfpathlineto{\pgfqpoint{4.354729in}{3.465150in}}%
\pgfpathlineto{\pgfqpoint{4.347354in}{3.443142in}}%
\pgfpathlineto{\pgfqpoint{4.339979in}{3.421488in}}%
\pgfpathlineto{\pgfqpoint{4.332605in}{3.400179in}}%
\pgfpathlineto{\pgfqpoint{4.325232in}{3.379208in}}%
\pgfpathclose%
\pgfusepath{fill}%
\end{pgfscope}%
\begin{pgfscope}%
\pgfpathrectangle{\pgfqpoint{1.254980in}{0.150000in}}{\pgfqpoint{5.490039in}{5.490039in}}%
\pgfusepath{clip}%
\pgfsetbuttcap%
\pgfsetroundjoin%
\definecolor{currentfill}{rgb}{0.177423,0.437527,0.557565}%
\pgfsetfillcolor{currentfill}%
\pgfsetfillopacity{0.700000}%
\pgfsetlinewidth{0.000000pt}%
\definecolor{currentstroke}{rgb}{0.000000,0.000000,0.000000}%
\pgfsetstrokecolor{currentstroke}%
\pgfsetdash{}{0pt}%
\pgfpathmoveto{\pgfqpoint{4.244407in}{3.337711in}}%
\pgfpathlineto{\pgfqpoint{4.257235in}{3.327641in}}%
\pgfpathlineto{\pgfqpoint{4.270067in}{3.317762in}}%
\pgfpathlineto{\pgfqpoint{4.282901in}{3.308072in}}%
\pgfpathlineto{\pgfqpoint{4.295739in}{3.298571in}}%
\pgfpathlineto{\pgfqpoint{4.303113in}{3.318258in}}%
\pgfpathlineto{\pgfqpoint{4.310486in}{3.338255in}}%
\pgfpathlineto{\pgfqpoint{4.317859in}{3.358570in}}%
\pgfpathlineto{\pgfqpoint{4.325232in}{3.379208in}}%
\pgfpathlineto{\pgfqpoint{4.312399in}{3.389289in}}%
\pgfpathlineto{\pgfqpoint{4.299569in}{3.399558in}}%
\pgfpathlineto{\pgfqpoint{4.286743in}{3.410017in}}%
\pgfpathlineto{\pgfqpoint{4.273919in}{3.420668in}}%
\pgfpathlineto{\pgfqpoint{4.266541in}{3.399438in}}%
\pgfpathlineto{\pgfqpoint{4.259163in}{3.378540in}}%
\pgfpathlineto{\pgfqpoint{4.251785in}{3.357967in}}%
\pgfpathlineto{\pgfqpoint{4.244407in}{3.337711in}}%
\pgfpathclose%
\pgfusepath{fill}%
\end{pgfscope}%
\begin{pgfscope}%
\pgfpathrectangle{\pgfqpoint{1.254980in}{0.150000in}}{\pgfqpoint{5.490039in}{5.490039in}}%
\pgfusepath{clip}%
\pgfsetbuttcap%
\pgfsetroundjoin%
\definecolor{currentfill}{rgb}{0.226397,0.728888,0.462789}%
\pgfsetfillcolor{currentfill}%
\pgfsetfillopacity{0.700000}%
\pgfsetlinewidth{0.000000pt}%
\definecolor{currentstroke}{rgb}{0.000000,0.000000,0.000000}%
\pgfsetstrokecolor{currentstroke}%
\pgfsetdash{}{0pt}%
\pgfpathmoveto{\pgfqpoint{3.710232in}{4.098198in}}%
\pgfpathlineto{\pgfqpoint{3.723083in}{4.077756in}}%
\pgfpathlineto{\pgfqpoint{3.735930in}{4.057576in}}%
\pgfpathlineto{\pgfqpoint{3.748772in}{4.037656in}}%
\pgfpathlineto{\pgfqpoint{3.761611in}{4.017993in}}%
\pgfpathlineto{\pgfqpoint{3.768993in}{4.045679in}}%
\pgfpathlineto{\pgfqpoint{3.776374in}{4.073790in}}%
\pgfpathlineto{\pgfqpoint{3.783754in}{4.102333in}}%
\pgfpathlineto{\pgfqpoint{3.791131in}{4.131318in}}%
\pgfpathlineto{\pgfqpoint{3.778290in}{4.151624in}}%
\pgfpathlineto{\pgfqpoint{3.765443in}{4.172188in}}%
\pgfpathlineto{\pgfqpoint{3.752593in}{4.193013in}}%
\pgfpathlineto{\pgfqpoint{3.739738in}{4.214100in}}%
\pgfpathlineto{\pgfqpoint{3.732364in}{4.184459in}}%
\pgfpathlineto{\pgfqpoint{3.724989in}{4.155266in}}%
\pgfpathlineto{\pgfqpoint{3.717611in}{4.126515in}}%
\pgfpathlineto{\pgfqpoint{3.710232in}{4.098198in}}%
\pgfpathclose%
\pgfusepath{fill}%
\end{pgfscope}%
\begin{pgfscope}%
\pgfpathrectangle{\pgfqpoint{1.254980in}{0.150000in}}{\pgfqpoint{5.490039in}{5.490039in}}%
\pgfusepath{clip}%
\pgfsetbuttcap%
\pgfsetroundjoin%
\definecolor{currentfill}{rgb}{0.162142,0.474838,0.558140}%
\pgfsetfillcolor{currentfill}%
\pgfsetfillopacity{0.700000}%
\pgfsetlinewidth{0.000000pt}%
\definecolor{currentstroke}{rgb}{0.000000,0.000000,0.000000}%
\pgfsetstrokecolor{currentstroke}%
\pgfsetdash{}{0pt}%
\pgfpathmoveto{\pgfqpoint{4.406074in}{3.424267in}}%
\pgfpathlineto{\pgfqpoint{4.418919in}{3.414509in}}%
\pgfpathlineto{\pgfqpoint{4.431768in}{3.404936in}}%
\pgfpathlineto{\pgfqpoint{4.444621in}{3.395544in}}%
\pgfpathlineto{\pgfqpoint{4.457478in}{3.386335in}}%
\pgfpathlineto{\pgfqpoint{4.464844in}{3.407452in}}%
\pgfpathlineto{\pgfqpoint{4.472211in}{3.428922in}}%
\pgfpathlineto{\pgfqpoint{4.479580in}{3.450753in}}%
\pgfpathlineto{\pgfqpoint{4.486951in}{3.472953in}}%
\pgfpathlineto{\pgfqpoint{4.474100in}{3.482798in}}%
\pgfpathlineto{\pgfqpoint{4.461252in}{3.492825in}}%
\pgfpathlineto{\pgfqpoint{4.448409in}{3.503035in}}%
\pgfpathlineto{\pgfqpoint{4.435569in}{3.513428in}}%
\pgfpathlineto{\pgfqpoint{4.428193in}{3.490582in}}%
\pgfpathlineto{\pgfqpoint{4.420818in}{3.468111in}}%
\pgfpathlineto{\pgfqpoint{4.413446in}{3.446009in}}%
\pgfpathlineto{\pgfqpoint{4.406074in}{3.424267in}}%
\pgfpathclose%
\pgfusepath{fill}%
\end{pgfscope}%
\begin{pgfscope}%
\pgfpathrectangle{\pgfqpoint{1.254980in}{0.150000in}}{\pgfqpoint{5.490039in}{5.490039in}}%
\pgfusepath{clip}%
\pgfsetbuttcap%
\pgfsetroundjoin%
\definecolor{currentfill}{rgb}{0.183898,0.422383,0.556944}%
\pgfsetfillcolor{currentfill}%
\pgfsetfillopacity{0.700000}%
\pgfsetlinewidth{0.000000pt}%
\definecolor{currentstroke}{rgb}{0.000000,0.000000,0.000000}%
\pgfsetstrokecolor{currentstroke}%
\pgfsetdash{}{0pt}%
\pgfpathmoveto{\pgfqpoint{4.163582in}{3.299734in}}%
\pgfpathlineto{\pgfqpoint{4.176404in}{3.289443in}}%
\pgfpathlineto{\pgfqpoint{4.189229in}{3.279347in}}%
\pgfpathlineto{\pgfqpoint{4.202056in}{3.269444in}}%
\pgfpathlineto{\pgfqpoint{4.214886in}{3.259734in}}%
\pgfpathlineto{\pgfqpoint{4.222268in}{3.278785in}}%
\pgfpathlineto{\pgfqpoint{4.229648in}{3.298127in}}%
\pgfpathlineto{\pgfqpoint{4.237028in}{3.317767in}}%
\pgfpathlineto{\pgfqpoint{4.244407in}{3.337711in}}%
\pgfpathlineto{\pgfqpoint{4.231581in}{3.347973in}}%
\pgfpathlineto{\pgfqpoint{4.218759in}{3.358427in}}%
\pgfpathlineto{\pgfqpoint{4.205939in}{3.369076in}}%
\pgfpathlineto{\pgfqpoint{4.193121in}{3.379919in}}%
\pgfpathlineto{\pgfqpoint{4.185738in}{3.359411in}}%
\pgfpathlineto{\pgfqpoint{4.178354in}{3.339216in}}%
\pgfpathlineto{\pgfqpoint{4.170968in}{3.319325in}}%
\pgfpathlineto{\pgfqpoint{4.163582in}{3.299734in}}%
\pgfpathclose%
\pgfusepath{fill}%
\end{pgfscope}%
\begin{pgfscope}%
\pgfpathrectangle{\pgfqpoint{1.254980in}{0.150000in}}{\pgfqpoint{5.490039in}{5.490039in}}%
\pgfusepath{clip}%
\pgfsetbuttcap%
\pgfsetroundjoin%
\definecolor{currentfill}{rgb}{0.165117,0.467423,0.558141}%
\pgfsetfillcolor{currentfill}%
\pgfsetfillopacity{0.700000}%
\pgfsetlinewidth{0.000000pt}%
\definecolor{currentstroke}{rgb}{0.000000,0.000000,0.000000}%
\pgfsetstrokecolor{currentstroke}%
\pgfsetdash{}{0pt}%
\pgfpathmoveto{\pgfqpoint{3.715982in}{3.415926in}}%
\pgfpathlineto{\pgfqpoint{3.728797in}{3.401171in}}%
\pgfpathlineto{\pgfqpoint{3.741611in}{3.386648in}}%
\pgfpathlineto{\pgfqpoint{3.754423in}{3.372356in}}%
\pgfpathlineto{\pgfqpoint{3.767234in}{3.358295in}}%
\pgfpathlineto{\pgfqpoint{3.774670in}{3.377537in}}%
\pgfpathlineto{\pgfqpoint{3.782103in}{3.397050in}}%
\pgfpathlineto{\pgfqpoint{3.789532in}{3.416838in}}%
\pgfpathlineto{\pgfqpoint{3.796959in}{3.436908in}}%
\pgfpathlineto{\pgfqpoint{3.784150in}{3.451445in}}%
\pgfpathlineto{\pgfqpoint{3.771341in}{3.466212in}}%
\pgfpathlineto{\pgfqpoint{3.758530in}{3.481211in}}%
\pgfpathlineto{\pgfqpoint{3.745717in}{3.496443in}}%
\pgfpathlineto{\pgfqpoint{3.738288in}{3.475886in}}%
\pgfpathlineto{\pgfqpoint{3.730856in}{3.455618in}}%
\pgfpathlineto{\pgfqpoint{3.723421in}{3.435634in}}%
\pgfpathlineto{\pgfqpoint{3.715982in}{3.415926in}}%
\pgfpathclose%
\pgfusepath{fill}%
\end{pgfscope}%
\begin{pgfscope}%
\pgfpathrectangle{\pgfqpoint{1.254980in}{0.150000in}}{\pgfqpoint{5.490039in}{5.490039in}}%
\pgfusepath{clip}%
\pgfsetbuttcap%
\pgfsetroundjoin%
\definecolor{currentfill}{rgb}{0.125394,0.574318,0.549086}%
\pgfsetfillcolor{currentfill}%
\pgfsetfillopacity{0.700000}%
\pgfsetlinewidth{0.000000pt}%
\definecolor{currentstroke}{rgb}{0.000000,0.000000,0.000000}%
\pgfsetstrokecolor{currentstroke}%
\pgfsetdash{}{0pt}%
\pgfpathmoveto{\pgfqpoint{3.591774in}{3.698149in}}%
\pgfpathlineto{\pgfqpoint{3.604621in}{3.679962in}}%
\pgfpathlineto{\pgfqpoint{3.617463in}{3.662032in}}%
\pgfpathlineto{\pgfqpoint{3.630302in}{3.644358in}}%
\pgfpathlineto{\pgfqpoint{3.643137in}{3.626938in}}%
\pgfpathlineto{\pgfqpoint{3.650565in}{3.648786in}}%
\pgfpathlineto{\pgfqpoint{3.657990in}{3.670950in}}%
\pgfpathlineto{\pgfqpoint{3.665411in}{3.693436in}}%
\pgfpathlineto{\pgfqpoint{3.672829in}{3.716250in}}%
\pgfpathlineto{\pgfqpoint{3.659994in}{3.734184in}}%
\pgfpathlineto{\pgfqpoint{3.647156in}{3.752371in}}%
\pgfpathlineto{\pgfqpoint{3.634314in}{3.770815in}}%
\pgfpathlineto{\pgfqpoint{3.621468in}{3.789518in}}%
\pgfpathlineto{\pgfqpoint{3.614050in}{3.766178in}}%
\pgfpathlineto{\pgfqpoint{3.606629in}{3.743174in}}%
\pgfpathlineto{\pgfqpoint{3.599203in}{3.720500in}}%
\pgfpathlineto{\pgfqpoint{3.591774in}{3.698149in}}%
\pgfpathclose%
\pgfusepath{fill}%
\end{pgfscope}%
\begin{pgfscope}%
\pgfpathrectangle{\pgfqpoint{1.254980in}{0.150000in}}{\pgfqpoint{5.490039in}{5.490039in}}%
\pgfusepath{clip}%
\pgfsetbuttcap%
\pgfsetroundjoin%
\definecolor{currentfill}{rgb}{0.156270,0.489624,0.557936}%
\pgfsetfillcolor{currentfill}%
\pgfsetfillopacity{0.700000}%
\pgfsetlinewidth{0.000000pt}%
\definecolor{currentstroke}{rgb}{0.000000,0.000000,0.000000}%
\pgfsetstrokecolor{currentstroke}%
\pgfsetdash{}{0pt}%
\pgfpathmoveto{\pgfqpoint{3.664703in}{3.477317in}}%
\pgfpathlineto{\pgfqpoint{3.677526in}{3.461610in}}%
\pgfpathlineto{\pgfqpoint{3.690347in}{3.446144in}}%
\pgfpathlineto{\pgfqpoint{3.703165in}{3.430917in}}%
\pgfpathlineto{\pgfqpoint{3.715982in}{3.415926in}}%
\pgfpathlineto{\pgfqpoint{3.723421in}{3.435634in}}%
\pgfpathlineto{\pgfqpoint{3.730856in}{3.455618in}}%
\pgfpathlineto{\pgfqpoint{3.738288in}{3.475886in}}%
\pgfpathlineto{\pgfqpoint{3.745717in}{3.496443in}}%
\pgfpathlineto{\pgfqpoint{3.732903in}{3.511911in}}%
\pgfpathlineto{\pgfqpoint{3.720086in}{3.527616in}}%
\pgfpathlineto{\pgfqpoint{3.707268in}{3.543561in}}%
\pgfpathlineto{\pgfqpoint{3.694447in}{3.559746in}}%
\pgfpathlineto{\pgfqpoint{3.687017in}{3.538700in}}%
\pgfpathlineto{\pgfqpoint{3.679583in}{3.517950in}}%
\pgfpathlineto{\pgfqpoint{3.672145in}{3.497491in}}%
\pgfpathlineto{\pgfqpoint{3.664703in}{3.477317in}}%
\pgfpathclose%
\pgfusepath{fill}%
\end{pgfscope}%
\begin{pgfscope}%
\pgfpathrectangle{\pgfqpoint{1.254980in}{0.150000in}}{\pgfqpoint{5.490039in}{5.490039in}}%
\pgfusepath{clip}%
\pgfsetbuttcap%
\pgfsetroundjoin%
\definecolor{currentfill}{rgb}{0.174274,0.445044,0.557792}%
\pgfsetfillcolor{currentfill}%
\pgfsetfillopacity{0.700000}%
\pgfsetlinewidth{0.000000pt}%
\definecolor{currentstroke}{rgb}{0.000000,0.000000,0.000000}%
\pgfsetstrokecolor{currentstroke}%
\pgfsetdash{}{0pt}%
\pgfpathmoveto{\pgfqpoint{3.767234in}{3.358295in}}%
\pgfpathlineto{\pgfqpoint{3.780044in}{3.344460in}}%
\pgfpathlineto{\pgfqpoint{3.792854in}{3.330853in}}%
\pgfpathlineto{\pgfqpoint{3.805663in}{3.317469in}}%
\pgfpathlineto{\pgfqpoint{3.818471in}{3.304309in}}%
\pgfpathlineto{\pgfqpoint{3.825904in}{3.323089in}}%
\pgfpathlineto{\pgfqpoint{3.833334in}{3.342131in}}%
\pgfpathlineto{\pgfqpoint{3.840760in}{3.361442in}}%
\pgfpathlineto{\pgfqpoint{3.848183in}{3.381027in}}%
\pgfpathlineto{\pgfqpoint{3.835378in}{3.394661in}}%
\pgfpathlineto{\pgfqpoint{3.822572in}{3.408518in}}%
\pgfpathlineto{\pgfqpoint{3.809766in}{3.422599in}}%
\pgfpathlineto{\pgfqpoint{3.796959in}{3.436908in}}%
\pgfpathlineto{\pgfqpoint{3.789532in}{3.416838in}}%
\pgfpathlineto{\pgfqpoint{3.782103in}{3.397050in}}%
\pgfpathlineto{\pgfqpoint{3.774670in}{3.377537in}}%
\pgfpathlineto{\pgfqpoint{3.767234in}{3.358295in}}%
\pgfpathclose%
\pgfusepath{fill}%
\end{pgfscope}%
\begin{pgfscope}%
\pgfpathrectangle{\pgfqpoint{1.254980in}{0.150000in}}{\pgfqpoint{5.490039in}{5.490039in}}%
\pgfusepath{clip}%
\pgfsetbuttcap%
\pgfsetroundjoin%
\definecolor{currentfill}{rgb}{0.187231,0.414746,0.556547}%
\pgfsetfillcolor{currentfill}%
\pgfsetfillopacity{0.700000}%
\pgfsetlinewidth{0.000000pt}%
\definecolor{currentstroke}{rgb}{0.000000,0.000000,0.000000}%
\pgfsetstrokecolor{currentstroke}%
\pgfsetdash{}{0pt}%
\pgfpathmoveto{\pgfqpoint{3.950627in}{3.279813in}}%
\pgfpathlineto{\pgfqpoint{3.963436in}{3.268121in}}%
\pgfpathlineto{\pgfqpoint{3.976245in}{3.256637in}}%
\pgfpathlineto{\pgfqpoint{3.989056in}{3.245361in}}%
\pgfpathlineto{\pgfqpoint{4.001868in}{3.234291in}}%
\pgfpathlineto{\pgfqpoint{4.009277in}{3.252690in}}%
\pgfpathlineto{\pgfqpoint{4.016685in}{3.271352in}}%
\pgfpathlineto{\pgfqpoint{4.024090in}{3.290284in}}%
\pgfpathlineto{\pgfqpoint{4.031493in}{3.309492in}}%
\pgfpathlineto{\pgfqpoint{4.018685in}{3.321060in}}%
\pgfpathlineto{\pgfqpoint{4.005878in}{3.332834in}}%
\pgfpathlineto{\pgfqpoint{3.993073in}{3.344816in}}%
\pgfpathlineto{\pgfqpoint{3.980268in}{3.357006in}}%
\pgfpathlineto{\pgfqpoint{3.972861in}{3.337290in}}%
\pgfpathlineto{\pgfqpoint{3.965453in}{3.317856in}}%
\pgfpathlineto{\pgfqpoint{3.958041in}{3.298699in}}%
\pgfpathlineto{\pgfqpoint{3.950627in}{3.279813in}}%
\pgfpathclose%
\pgfusepath{fill}%
\end{pgfscope}%
\begin{pgfscope}%
\pgfpathrectangle{\pgfqpoint{1.254980in}{0.150000in}}{\pgfqpoint{5.490039in}{5.490039in}}%
\pgfusepath{clip}%
\pgfsetbuttcap%
\pgfsetroundjoin%
\definecolor{currentfill}{rgb}{0.188923,0.410910,0.556326}%
\pgfsetfillcolor{currentfill}%
\pgfsetfillopacity{0.700000}%
\pgfsetlinewidth{0.000000pt}%
\definecolor{currentstroke}{rgb}{0.000000,0.000000,0.000000}%
\pgfsetstrokecolor{currentstroke}%
\pgfsetdash{}{0pt}%
\pgfpathmoveto{\pgfqpoint{4.082741in}{3.265258in}}%
\pgfpathlineto{\pgfqpoint{4.095558in}{3.254702in}}%
\pgfpathlineto{\pgfqpoint{4.108377in}{3.244345in}}%
\pgfpathlineto{\pgfqpoint{4.121199in}{3.234186in}}%
\pgfpathlineto{\pgfqpoint{4.134023in}{3.224224in}}%
\pgfpathlineto{\pgfqpoint{4.141415in}{3.242685in}}%
\pgfpathlineto{\pgfqpoint{4.148806in}{3.261420in}}%
\pgfpathlineto{\pgfqpoint{4.156195in}{3.280434in}}%
\pgfpathlineto{\pgfqpoint{4.163582in}{3.299734in}}%
\pgfpathlineto{\pgfqpoint{4.150763in}{3.310220in}}%
\pgfpathlineto{\pgfqpoint{4.137946in}{3.320903in}}%
\pgfpathlineto{\pgfqpoint{4.125131in}{3.331784in}}%
\pgfpathlineto{\pgfqpoint{4.112318in}{3.342864in}}%
\pgfpathlineto{\pgfqpoint{4.104926in}{3.323029in}}%
\pgfpathlineto{\pgfqpoint{4.097533in}{3.303488in}}%
\pgfpathlineto{\pgfqpoint{4.090138in}{3.284232in}}%
\pgfpathlineto{\pgfqpoint{4.082741in}{3.265258in}}%
\pgfpathclose%
\pgfusepath{fill}%
\end{pgfscope}%
\begin{pgfscope}%
\pgfpathrectangle{\pgfqpoint{1.254980in}{0.150000in}}{\pgfqpoint{5.490039in}{5.490039in}}%
\pgfusepath{clip}%
\pgfsetbuttcap%
\pgfsetroundjoin%
\definecolor{currentfill}{rgb}{0.146180,0.515413,0.556823}%
\pgfsetfillcolor{currentfill}%
\pgfsetfillopacity{0.700000}%
\pgfsetlinewidth{0.000000pt}%
\definecolor{currentstroke}{rgb}{0.000000,0.000000,0.000000}%
\pgfsetstrokecolor{currentstroke}%
\pgfsetdash{}{0pt}%
\pgfpathmoveto{\pgfqpoint{3.613386in}{3.542587in}}%
\pgfpathlineto{\pgfqpoint{3.626220in}{3.525899in}}%
\pgfpathlineto{\pgfqpoint{3.639050in}{3.509459in}}%
\pgfpathlineto{\pgfqpoint{3.651878in}{3.493266in}}%
\pgfpathlineto{\pgfqpoint{3.664703in}{3.477317in}}%
\pgfpathlineto{\pgfqpoint{3.672145in}{3.497491in}}%
\pgfpathlineto{\pgfqpoint{3.679583in}{3.517950in}}%
\pgfpathlineto{\pgfqpoint{3.687017in}{3.538700in}}%
\pgfpathlineto{\pgfqpoint{3.694447in}{3.559746in}}%
\pgfpathlineto{\pgfqpoint{3.681624in}{3.576174in}}%
\pgfpathlineto{\pgfqpoint{3.668798in}{3.592848in}}%
\pgfpathlineto{\pgfqpoint{3.655969in}{3.609768in}}%
\pgfpathlineto{\pgfqpoint{3.643137in}{3.626938in}}%
\pgfpathlineto{\pgfqpoint{3.635705in}{3.605400in}}%
\pgfpathlineto{\pgfqpoint{3.628269in}{3.584165in}}%
\pgfpathlineto{\pgfqpoint{3.620830in}{3.563230in}}%
\pgfpathlineto{\pgfqpoint{3.613386in}{3.542587in}}%
\pgfpathclose%
\pgfusepath{fill}%
\end{pgfscope}%
\begin{pgfscope}%
\pgfpathrectangle{\pgfqpoint{1.254980in}{0.150000in}}{\pgfqpoint{5.490039in}{5.490039in}}%
\pgfusepath{clip}%
\pgfsetbuttcap%
\pgfsetroundjoin%
\definecolor{currentfill}{rgb}{0.153894,0.680203,0.504172}%
\pgfsetfillcolor{currentfill}%
\pgfsetfillopacity{0.700000}%
\pgfsetlinewidth{0.000000pt}%
\definecolor{currentstroke}{rgb}{0.000000,0.000000,0.000000}%
\pgfsetstrokecolor{currentstroke}%
\pgfsetdash{}{0pt}%
\pgfpathmoveto{\pgfqpoint{3.599672in}{3.966004in}}%
\pgfpathlineto{\pgfqpoint{3.612537in}{3.945694in}}%
\pgfpathlineto{\pgfqpoint{3.625398in}{3.925652in}}%
\pgfpathlineto{\pgfqpoint{3.638253in}{3.905877in}}%
\pgfpathlineto{\pgfqpoint{3.651104in}{3.886365in}}%
\pgfpathlineto{\pgfqpoint{3.658505in}{3.911482in}}%
\pgfpathlineto{\pgfqpoint{3.665903in}{3.936973in}}%
\pgfpathlineto{\pgfqpoint{3.673298in}{3.962847in}}%
\pgfpathlineto{\pgfqpoint{3.680690in}{3.989110in}}%
\pgfpathlineto{\pgfqpoint{3.667837in}{4.009202in}}%
\pgfpathlineto{\pgfqpoint{3.654979in}{4.029559in}}%
\pgfpathlineto{\pgfqpoint{3.642117in}{4.050183in}}%
\pgfpathlineto{\pgfqpoint{3.629249in}{4.071077in}}%
\pgfpathlineto{\pgfqpoint{3.621860in}{4.044220in}}%
\pgfpathlineto{\pgfqpoint{3.614467in}{4.017760in}}%
\pgfpathlineto{\pgfqpoint{3.607071in}{3.991690in}}%
\pgfpathlineto{\pgfqpoint{3.599672in}{3.966004in}}%
\pgfpathclose%
\pgfusepath{fill}%
\end{pgfscope}%
\begin{pgfscope}%
\pgfpathrectangle{\pgfqpoint{1.254980in}{0.150000in}}{\pgfqpoint{5.490039in}{5.490039in}}%
\pgfusepath{clip}%
\pgfsetbuttcap%
\pgfsetroundjoin%
\definecolor{currentfill}{rgb}{0.156270,0.489624,0.557936}%
\pgfsetfillcolor{currentfill}%
\pgfsetfillopacity{0.700000}%
\pgfsetlinewidth{0.000000pt}%
\definecolor{currentstroke}{rgb}{0.000000,0.000000,0.000000}%
\pgfsetstrokecolor{currentstroke}%
\pgfsetdash{}{0pt}%
\pgfpathmoveto{\pgfqpoint{4.486951in}{3.472953in}}%
\pgfpathlineto{\pgfqpoint{4.499806in}{3.463289in}}%
\pgfpathlineto{\pgfqpoint{4.512665in}{3.453806in}}%
\pgfpathlineto{\pgfqpoint{4.525528in}{3.444502in}}%
\pgfpathlineto{\pgfqpoint{4.538396in}{3.435376in}}%
\pgfpathlineto{\pgfqpoint{4.545763in}{3.457299in}}%
\pgfpathlineto{\pgfqpoint{4.553132in}{3.479598in}}%
\pgfpathlineto{\pgfqpoint{4.560504in}{3.502282in}}%
\pgfpathlineto{\pgfqpoint{4.547641in}{3.511902in}}%
\pgfpathlineto{\pgfqpoint{4.534782in}{3.521701in}}%
\pgfpathlineto{\pgfqpoint{4.521927in}{3.531680in}}%
\pgfpathlineto{\pgfqpoint{4.509077in}{3.541840in}}%
\pgfpathlineto{\pgfqpoint{4.501699in}{3.518488in}}%
\pgfpathlineto{\pgfqpoint{4.494324in}{3.495529in}}%
\pgfpathlineto{\pgfqpoint{4.486951in}{3.472953in}}%
\pgfpathclose%
\pgfusepath{fill}%
\end{pgfscope}%
\begin{pgfscope}%
\pgfpathrectangle{\pgfqpoint{1.254980in}{0.150000in}}{\pgfqpoint{5.490039in}{5.490039in}}%
\pgfusepath{clip}%
\pgfsetbuttcap%
\pgfsetroundjoin%
\definecolor{currentfill}{rgb}{0.335885,0.777018,0.402049}%
\pgfsetfillcolor{currentfill}%
\pgfsetfillopacity{0.700000}%
\pgfsetlinewidth{0.000000pt}%
\definecolor{currentstroke}{rgb}{0.000000,0.000000,0.000000}%
\pgfsetstrokecolor{currentstroke}%
\pgfsetdash{}{0pt}%
\pgfpathmoveto{\pgfqpoint{3.820630in}{4.251821in}}%
\pgfpathlineto{\pgfqpoint{3.833473in}{4.231095in}}%
\pgfpathlineto{\pgfqpoint{3.846311in}{4.210624in}}%
\pgfpathlineto{\pgfqpoint{3.859146in}{4.190407in}}%
\pgfpathlineto{\pgfqpoint{3.871977in}{4.170440in}}%
\pgfpathlineto{\pgfqpoint{3.879354in}{4.201053in}}%
\pgfpathlineto{\pgfqpoint{3.886731in}{4.232147in}}%
\pgfpathlineto{\pgfqpoint{3.894108in}{4.263731in}}%
\pgfpathlineto{\pgfqpoint{3.881273in}{4.284225in}}%
\pgfpathlineto{\pgfqpoint{3.868435in}{4.304971in}}%
\pgfpathlineto{\pgfqpoint{3.855592in}{4.325970in}}%
\pgfpathlineto{\pgfqpoint{3.842746in}{4.347226in}}%
\pgfpathlineto{\pgfqpoint{3.835374in}{4.314928in}}%
\pgfpathlineto{\pgfqpoint{3.828003in}{4.283130in}}%
\pgfpathlineto{\pgfqpoint{3.820630in}{4.251821in}}%
\pgfpathclose%
\pgfusepath{fill}%
\end{pgfscope}%
\begin{pgfscope}%
\pgfpathrectangle{\pgfqpoint{1.254980in}{0.150000in}}{\pgfqpoint{5.490039in}{5.490039in}}%
\pgfusepath{clip}%
\pgfsetbuttcap%
\pgfsetroundjoin%
\definecolor{currentfill}{rgb}{0.124780,0.640461,0.527068}%
\pgfsetfillcolor{currentfill}%
\pgfsetfillopacity{0.700000}%
\pgfsetlinewidth{0.000000pt}%
\definecolor{currentstroke}{rgb}{0.000000,0.000000,0.000000}%
\pgfsetstrokecolor{currentstroke}%
\pgfsetdash{}{0pt}%
\pgfpathmoveto{\pgfqpoint{3.570039in}{3.866959in}}%
\pgfpathlineto{\pgfqpoint{3.582904in}{3.847200in}}%
\pgfpathlineto{\pgfqpoint{3.595763in}{3.827708in}}%
\pgfpathlineto{\pgfqpoint{3.608618in}{3.808481in}}%
\pgfpathlineto{\pgfqpoint{3.621468in}{3.789518in}}%
\pgfpathlineto{\pgfqpoint{3.628882in}{3.813200in}}%
\pgfpathlineto{\pgfqpoint{3.636293in}{3.837231in}}%
\pgfpathlineto{\pgfqpoint{3.643700in}{3.861617in}}%
\pgfpathlineto{\pgfqpoint{3.651104in}{3.886365in}}%
\pgfpathlineto{\pgfqpoint{3.638253in}{3.905877in}}%
\pgfpathlineto{\pgfqpoint{3.625398in}{3.925652in}}%
\pgfpathlineto{\pgfqpoint{3.612537in}{3.945694in}}%
\pgfpathlineto{\pgfqpoint{3.599672in}{3.966004in}}%
\pgfpathlineto{\pgfqpoint{3.592269in}{3.940695in}}%
\pgfpathlineto{\pgfqpoint{3.584863in}{3.915755in}}%
\pgfpathlineto{\pgfqpoint{3.577453in}{3.891179in}}%
\pgfpathlineto{\pgfqpoint{3.570039in}{3.866959in}}%
\pgfpathclose%
\pgfusepath{fill}%
\end{pgfscope}%
\begin{pgfscope}%
\pgfpathrectangle{\pgfqpoint{1.254980in}{0.150000in}}{\pgfqpoint{5.490039in}{5.490039in}}%
\pgfusepath{clip}%
\pgfsetbuttcap%
\pgfsetroundjoin%
\definecolor{currentfill}{rgb}{0.183898,0.422383,0.556944}%
\pgfsetfillcolor{currentfill}%
\pgfsetfillopacity{0.700000}%
\pgfsetlinewidth{0.000000pt}%
\definecolor{currentstroke}{rgb}{0.000000,0.000000,0.000000}%
\pgfsetstrokecolor{currentstroke}%
\pgfsetdash{}{0pt}%
\pgfpathmoveto{\pgfqpoint{3.818471in}{3.304309in}}%
\pgfpathlineto{\pgfqpoint{3.831280in}{3.291370in}}%
\pgfpathlineto{\pgfqpoint{3.844088in}{3.278650in}}%
\pgfpathlineto{\pgfqpoint{3.856896in}{3.266149in}}%
\pgfpathlineto{\pgfqpoint{3.869705in}{3.253865in}}%
\pgfpathlineto{\pgfqpoint{3.877134in}{3.272183in}}%
\pgfpathlineto{\pgfqpoint{3.884560in}{3.290757in}}%
\pgfpathlineto{\pgfqpoint{3.891983in}{3.309593in}}%
\pgfpathlineto{\pgfqpoint{3.899403in}{3.328695in}}%
\pgfpathlineto{\pgfqpoint{3.886598in}{3.341451in}}%
\pgfpathlineto{\pgfqpoint{3.873793in}{3.354424in}}%
\pgfpathlineto{\pgfqpoint{3.860988in}{3.367615in}}%
\pgfpathlineto{\pgfqpoint{3.848183in}{3.381027in}}%
\pgfpathlineto{\pgfqpoint{3.840760in}{3.361442in}}%
\pgfpathlineto{\pgfqpoint{3.833334in}{3.342131in}}%
\pgfpathlineto{\pgfqpoint{3.825904in}{3.323089in}}%
\pgfpathlineto{\pgfqpoint{3.818471in}{3.304309in}}%
\pgfpathclose%
\pgfusepath{fill}%
\end{pgfscope}%
\begin{pgfscope}%
\pgfpathrectangle{\pgfqpoint{1.254980in}{0.150000in}}{\pgfqpoint{5.490039in}{5.490039in}}%
\pgfusepath{clip}%
\pgfsetbuttcap%
\pgfsetroundjoin%
\definecolor{currentfill}{rgb}{0.208030,0.718701,0.472873}%
\pgfsetfillcolor{currentfill}%
\pgfsetfillopacity{0.700000}%
\pgfsetlinewidth{0.000000pt}%
\definecolor{currentstroke}{rgb}{0.000000,0.000000,0.000000}%
\pgfsetstrokecolor{currentstroke}%
\pgfsetdash{}{0pt}%
\pgfpathmoveto{\pgfqpoint{3.629249in}{4.071077in}}%
\pgfpathlineto{\pgfqpoint{3.642117in}{4.050183in}}%
\pgfpathlineto{\pgfqpoint{3.654979in}{4.029559in}}%
\pgfpathlineto{\pgfqpoint{3.667837in}{4.009202in}}%
\pgfpathlineto{\pgfqpoint{3.680690in}{3.989110in}}%
\pgfpathlineto{\pgfqpoint{3.688079in}{4.015769in}}%
\pgfpathlineto{\pgfqpoint{3.695466in}{4.042832in}}%
\pgfpathlineto{\pgfqpoint{3.702850in}{4.070306in}}%
\pgfpathlineto{\pgfqpoint{3.710232in}{4.098198in}}%
\pgfpathlineto{\pgfqpoint{3.697376in}{4.118903in}}%
\pgfpathlineto{\pgfqpoint{3.684515in}{4.139874in}}%
\pgfpathlineto{\pgfqpoint{3.671650in}{4.161115in}}%
\pgfpathlineto{\pgfqpoint{3.658779in}{4.182626in}}%
\pgfpathlineto{\pgfqpoint{3.651401in}{4.154106in}}%
\pgfpathlineto{\pgfqpoint{3.644020in}{4.126013in}}%
\pgfpathlineto{\pgfqpoint{3.636636in}{4.098339in}}%
\pgfpathlineto{\pgfqpoint{3.629249in}{4.071077in}}%
\pgfpathclose%
\pgfusepath{fill}%
\end{pgfscope}%
\begin{pgfscope}%
\pgfpathrectangle{\pgfqpoint{1.254980in}{0.150000in}}{\pgfqpoint{5.490039in}{5.490039in}}%
\pgfusepath{clip}%
\pgfsetbuttcap%
\pgfsetroundjoin%
\definecolor{currentfill}{rgb}{0.319809,0.770914,0.411152}%
\pgfsetfillcolor{currentfill}%
\pgfsetfillopacity{0.700000}%
\pgfsetlinewidth{0.000000pt}%
\definecolor{currentstroke}{rgb}{0.000000,0.000000,0.000000}%
\pgfsetstrokecolor{currentstroke}%
\pgfsetdash{}{0pt}%
\pgfpathmoveto{\pgfqpoint{3.739738in}{4.214100in}}%
\pgfpathlineto{\pgfqpoint{3.752593in}{4.193013in}}%
\pgfpathlineto{\pgfqpoint{3.765443in}{4.172188in}}%
\pgfpathlineto{\pgfqpoint{3.778290in}{4.151624in}}%
\pgfpathlineto{\pgfqpoint{3.791131in}{4.131318in}}%
\pgfpathlineto{\pgfqpoint{3.798508in}{4.160750in}}%
\pgfpathlineto{\pgfqpoint{3.805883in}{4.190640in}}%
\pgfpathlineto{\pgfqpoint{3.813257in}{4.220994in}}%
\pgfpathlineto{\pgfqpoint{3.820630in}{4.251821in}}%
\pgfpathlineto{\pgfqpoint{3.807784in}{4.272804in}}%
\pgfpathlineto{\pgfqpoint{3.794933in}{4.294047in}}%
\pgfpathlineto{\pgfqpoint{3.782077in}{4.315551in}}%
\pgfpathlineto{\pgfqpoint{3.769217in}{4.337320in}}%
\pgfpathlineto{\pgfqpoint{3.761849in}{4.305801in}}%
\pgfpathlineto{\pgfqpoint{3.754480in}{4.274763in}}%
\pgfpathlineto{\pgfqpoint{3.747110in}{4.244199in}}%
\pgfpathlineto{\pgfqpoint{3.739738in}{4.214100in}}%
\pgfpathclose%
\pgfusepath{fill}%
\end{pgfscope}%
\begin{pgfscope}%
\pgfpathrectangle{\pgfqpoint{1.254980in}{0.150000in}}{\pgfqpoint{5.490039in}{5.490039in}}%
\pgfusepath{clip}%
\pgfsetbuttcap%
\pgfsetroundjoin%
\definecolor{currentfill}{rgb}{0.136408,0.541173,0.554483}%
\pgfsetfillcolor{currentfill}%
\pgfsetfillopacity{0.700000}%
\pgfsetlinewidth{0.000000pt}%
\definecolor{currentstroke}{rgb}{0.000000,0.000000,0.000000}%
\pgfsetstrokecolor{currentstroke}%
\pgfsetdash{}{0pt}%
\pgfpathmoveto{\pgfqpoint{3.562017in}{3.611865in}}%
\pgfpathlineto{\pgfqpoint{3.574865in}{3.594162in}}%
\pgfpathlineto{\pgfqpoint{3.587709in}{3.576716in}}%
\pgfpathlineto{\pgfqpoint{3.600549in}{3.559525in}}%
\pgfpathlineto{\pgfqpoint{3.613386in}{3.542587in}}%
\pgfpathlineto{\pgfqpoint{3.620830in}{3.563230in}}%
\pgfpathlineto{\pgfqpoint{3.628269in}{3.584165in}}%
\pgfpathlineto{\pgfqpoint{3.635705in}{3.605400in}}%
\pgfpathlineto{\pgfqpoint{3.643137in}{3.626938in}}%
\pgfpathlineto{\pgfqpoint{3.630302in}{3.644358in}}%
\pgfpathlineto{\pgfqpoint{3.617463in}{3.662032in}}%
\pgfpathlineto{\pgfqpoint{3.604621in}{3.679962in}}%
\pgfpathlineto{\pgfqpoint{3.591774in}{3.698149in}}%
\pgfpathlineto{\pgfqpoint{3.584341in}{3.676116in}}%
\pgfpathlineto{\pgfqpoint{3.576904in}{3.654395in}}%
\pgfpathlineto{\pgfqpoint{3.569463in}{3.632980in}}%
\pgfpathlineto{\pgfqpoint{3.562017in}{3.611865in}}%
\pgfpathclose%
\pgfusepath{fill}%
\end{pgfscope}%
\begin{pgfscope}%
\pgfpathrectangle{\pgfqpoint{1.254980in}{0.150000in}}{\pgfqpoint{5.490039in}{5.490039in}}%
\pgfusepath{clip}%
\pgfsetbuttcap%
\pgfsetroundjoin%
\definecolor{currentfill}{rgb}{0.119512,0.607464,0.540218}%
\pgfsetfillcolor{currentfill}%
\pgfsetfillopacity{0.700000}%
\pgfsetlinewidth{0.000000pt}%
\definecolor{currentstroke}{rgb}{0.000000,0.000000,0.000000}%
\pgfsetstrokecolor{currentstroke}%
\pgfsetdash{}{0pt}%
\pgfpathmoveto{\pgfqpoint{3.540345in}{3.773520in}}%
\pgfpathlineto{\pgfqpoint{3.553210in}{3.754280in}}%
\pgfpathlineto{\pgfqpoint{3.566069in}{3.735306in}}%
\pgfpathlineto{\pgfqpoint{3.578924in}{3.716596in}}%
\pgfpathlineto{\pgfqpoint{3.591774in}{3.698149in}}%
\pgfpathlineto{\pgfqpoint{3.599203in}{3.720500in}}%
\pgfpathlineto{\pgfqpoint{3.606629in}{3.743174in}}%
\pgfpathlineto{\pgfqpoint{3.614050in}{3.766178in}}%
\pgfpathlineto{\pgfqpoint{3.621468in}{3.789518in}}%
\pgfpathlineto{\pgfqpoint{3.608618in}{3.808481in}}%
\pgfpathlineto{\pgfqpoint{3.595763in}{3.827708in}}%
\pgfpathlineto{\pgfqpoint{3.582904in}{3.847200in}}%
\pgfpathlineto{\pgfqpoint{3.570039in}{3.866959in}}%
\pgfpathlineto{\pgfqpoint{3.562622in}{3.843090in}}%
\pgfpathlineto{\pgfqpoint{3.555200in}{3.819564in}}%
\pgfpathlineto{\pgfqpoint{3.547775in}{3.796377in}}%
\pgfpathlineto{\pgfqpoint{3.540345in}{3.773520in}}%
\pgfpathclose%
\pgfusepath{fill}%
\end{pgfscope}%
\begin{pgfscope}%
\pgfpathrectangle{\pgfqpoint{1.254980in}{0.150000in}}{\pgfqpoint{5.490039in}{5.490039in}}%
\pgfusepath{clip}%
\pgfsetbuttcap%
\pgfsetroundjoin%
\definecolor{currentfill}{rgb}{0.182256,0.426184,0.557120}%
\pgfsetfillcolor{currentfill}%
\pgfsetfillopacity{0.700000}%
\pgfsetlinewidth{0.000000pt}%
\definecolor{currentstroke}{rgb}{0.000000,0.000000,0.000000}%
\pgfsetstrokecolor{currentstroke}%
\pgfsetdash{}{0pt}%
\pgfpathmoveto{\pgfqpoint{4.295739in}{3.298571in}}%
\pgfpathlineto{\pgfqpoint{4.308581in}{3.289258in}}%
\pgfpathlineto{\pgfqpoint{4.321426in}{3.280131in}}%
\pgfpathlineto{\pgfqpoint{4.334275in}{3.271189in}}%
\pgfpathlineto{\pgfqpoint{4.347128in}{3.262433in}}%
\pgfpathlineto{\pgfqpoint{4.354495in}{3.281552in}}%
\pgfpathlineto{\pgfqpoint{4.361863in}{3.300974in}}%
\pgfpathlineto{\pgfqpoint{4.369230in}{3.320707in}}%
\pgfpathlineto{\pgfqpoint{4.376598in}{3.340756in}}%
\pgfpathlineto{\pgfqpoint{4.363751in}{3.350091in}}%
\pgfpathlineto{\pgfqpoint{4.350908in}{3.359610in}}%
\pgfpathlineto{\pgfqpoint{4.338068in}{3.369316in}}%
\pgfpathlineto{\pgfqpoint{4.325232in}{3.379208in}}%
\pgfpathlineto{\pgfqpoint{4.317859in}{3.358570in}}%
\pgfpathlineto{\pgfqpoint{4.310486in}{3.338255in}}%
\pgfpathlineto{\pgfqpoint{4.303113in}{3.318258in}}%
\pgfpathlineto{\pgfqpoint{4.295739in}{3.298571in}}%
\pgfpathclose%
\pgfusepath{fill}%
\end{pgfscope}%
\begin{pgfscope}%
\pgfpathrectangle{\pgfqpoint{1.254980in}{0.150000in}}{\pgfqpoint{5.490039in}{5.490039in}}%
\pgfusepath{clip}%
\pgfsetbuttcap%
\pgfsetroundjoin%
\definecolor{currentfill}{rgb}{0.174274,0.445044,0.557792}%
\pgfsetfillcolor{currentfill}%
\pgfsetfillopacity{0.700000}%
\pgfsetlinewidth{0.000000pt}%
\definecolor{currentstroke}{rgb}{0.000000,0.000000,0.000000}%
\pgfsetstrokecolor{currentstroke}%
\pgfsetdash{}{0pt}%
\pgfpathmoveto{\pgfqpoint{4.376598in}{3.340756in}}%
\pgfpathlineto{\pgfqpoint{4.389449in}{3.331606in}}%
\pgfpathlineto{\pgfqpoint{4.402303in}{3.322638in}}%
\pgfpathlineto{\pgfqpoint{4.415162in}{3.313853in}}%
\pgfpathlineto{\pgfqpoint{4.428025in}{3.305249in}}%
\pgfpathlineto{\pgfqpoint{4.435387in}{3.325028in}}%
\pgfpathlineto{\pgfqpoint{4.442750in}{3.345130in}}%
\pgfpathlineto{\pgfqpoint{4.450113in}{3.365563in}}%
\pgfpathlineto{\pgfqpoint{4.457478in}{3.386335in}}%
\pgfpathlineto{\pgfqpoint{4.444621in}{3.395544in}}%
\pgfpathlineto{\pgfqpoint{4.431768in}{3.404936in}}%
\pgfpathlineto{\pgfqpoint{4.418919in}{3.414509in}}%
\pgfpathlineto{\pgfqpoint{4.406074in}{3.424267in}}%
\pgfpathlineto{\pgfqpoint{4.398704in}{3.402878in}}%
\pgfpathlineto{\pgfqpoint{4.391335in}{3.381835in}}%
\pgfpathlineto{\pgfqpoint{4.383966in}{3.361130in}}%
\pgfpathlineto{\pgfqpoint{4.376598in}{3.340756in}}%
\pgfpathclose%
\pgfusepath{fill}%
\end{pgfscope}%
\begin{pgfscope}%
\pgfpathrectangle{\pgfqpoint{1.254980in}{0.150000in}}{\pgfqpoint{5.490039in}{5.490039in}}%
\pgfusepath{clip}%
\pgfsetbuttcap%
\pgfsetroundjoin%
\definecolor{currentfill}{rgb}{0.194100,0.399323,0.555565}%
\pgfsetfillcolor{currentfill}%
\pgfsetfillopacity{0.700000}%
\pgfsetlinewidth{0.000000pt}%
\definecolor{currentstroke}{rgb}{0.000000,0.000000,0.000000}%
\pgfsetstrokecolor{currentstroke}%
\pgfsetdash{}{0pt}%
\pgfpathmoveto{\pgfqpoint{4.001868in}{3.234291in}}%
\pgfpathlineto{\pgfqpoint{4.014682in}{3.223425in}}%
\pgfpathlineto{\pgfqpoint{4.027497in}{3.212763in}}%
\pgfpathlineto{\pgfqpoint{4.040314in}{3.202303in}}%
\pgfpathlineto{\pgfqpoint{4.053134in}{3.192043in}}%
\pgfpathlineto{\pgfqpoint{4.060539in}{3.209956in}}%
\pgfpathlineto{\pgfqpoint{4.067942in}{3.228125in}}%
\pgfpathlineto{\pgfqpoint{4.075343in}{3.246557in}}%
\pgfpathlineto{\pgfqpoint{4.082741in}{3.265258in}}%
\pgfpathlineto{\pgfqpoint{4.069926in}{3.276013in}}%
\pgfpathlineto{\pgfqpoint{4.057113in}{3.286970in}}%
\pgfpathlineto{\pgfqpoint{4.044302in}{3.298129in}}%
\pgfpathlineto{\pgfqpoint{4.031493in}{3.309492in}}%
\pgfpathlineto{\pgfqpoint{4.024090in}{3.290284in}}%
\pgfpathlineto{\pgfqpoint{4.016685in}{3.271352in}}%
\pgfpathlineto{\pgfqpoint{4.009277in}{3.252690in}}%
\pgfpathlineto{\pgfqpoint{4.001868in}{3.234291in}}%
\pgfpathclose%
\pgfusepath{fill}%
\end{pgfscope}%
\begin{pgfscope}%
\pgfpathrectangle{\pgfqpoint{1.254980in}{0.150000in}}{\pgfqpoint{5.490039in}{5.490039in}}%
\pgfusepath{clip}%
\pgfsetbuttcap%
\pgfsetroundjoin%
\definecolor{currentfill}{rgb}{0.192357,0.403199,0.555836}%
\pgfsetfillcolor{currentfill}%
\pgfsetfillopacity{0.700000}%
\pgfsetlinewidth{0.000000pt}%
\definecolor{currentstroke}{rgb}{0.000000,0.000000,0.000000}%
\pgfsetstrokecolor{currentstroke}%
\pgfsetdash{}{0pt}%
\pgfpathmoveto{\pgfqpoint{3.869705in}{3.253865in}}%
\pgfpathlineto{\pgfqpoint{3.882514in}{3.241796in}}%
\pgfpathlineto{\pgfqpoint{3.895323in}{3.229941in}}%
\pgfpathlineto{\pgfqpoint{3.908134in}{3.218298in}}%
\pgfpathlineto{\pgfqpoint{3.920945in}{3.206866in}}%
\pgfpathlineto{\pgfqpoint{3.928370in}{3.224724in}}%
\pgfpathlineto{\pgfqpoint{3.935792in}{3.242831in}}%
\pgfpathlineto{\pgfqpoint{3.943211in}{3.261192in}}%
\pgfpathlineto{\pgfqpoint{3.950627in}{3.279813in}}%
\pgfpathlineto{\pgfqpoint{3.937820in}{3.291715in}}%
\pgfpathlineto{\pgfqpoint{3.925014in}{3.303828in}}%
\pgfpathlineto{\pgfqpoint{3.912208in}{3.316154in}}%
\pgfpathlineto{\pgfqpoint{3.899403in}{3.328695in}}%
\pgfpathlineto{\pgfqpoint{3.891983in}{3.309593in}}%
\pgfpathlineto{\pgfqpoint{3.884560in}{3.290757in}}%
\pgfpathlineto{\pgfqpoint{3.877134in}{3.272183in}}%
\pgfpathlineto{\pgfqpoint{3.869705in}{3.253865in}}%
\pgfpathclose%
\pgfusepath{fill}%
\end{pgfscope}%
\begin{pgfscope}%
\pgfpathrectangle{\pgfqpoint{1.254980in}{0.150000in}}{\pgfqpoint{5.490039in}{5.490039in}}%
\pgfusepath{clip}%
\pgfsetbuttcap%
\pgfsetroundjoin%
\definecolor{currentfill}{rgb}{0.188923,0.410910,0.556326}%
\pgfsetfillcolor{currentfill}%
\pgfsetfillopacity{0.700000}%
\pgfsetlinewidth{0.000000pt}%
\definecolor{currentstroke}{rgb}{0.000000,0.000000,0.000000}%
\pgfsetstrokecolor{currentstroke}%
\pgfsetdash{}{0pt}%
\pgfpathmoveto{\pgfqpoint{4.214886in}{3.259734in}}%
\pgfpathlineto{\pgfqpoint{4.227720in}{3.250215in}}%
\pgfpathlineto{\pgfqpoint{4.240556in}{3.240887in}}%
\pgfpathlineto{\pgfqpoint{4.253396in}{3.231748in}}%
\pgfpathlineto{\pgfqpoint{4.266240in}{3.222797in}}%
\pgfpathlineto{\pgfqpoint{4.273616in}{3.241308in}}%
\pgfpathlineto{\pgfqpoint{4.280991in}{3.260103in}}%
\pgfpathlineto{\pgfqpoint{4.288365in}{3.279188in}}%
\pgfpathlineto{\pgfqpoint{4.295739in}{3.298571in}}%
\pgfpathlineto{\pgfqpoint{4.282901in}{3.308072in}}%
\pgfpathlineto{\pgfqpoint{4.270067in}{3.317762in}}%
\pgfpathlineto{\pgfqpoint{4.257235in}{3.327641in}}%
\pgfpathlineto{\pgfqpoint{4.244407in}{3.337711in}}%
\pgfpathlineto{\pgfqpoint{4.237028in}{3.317767in}}%
\pgfpathlineto{\pgfqpoint{4.229648in}{3.298127in}}%
\pgfpathlineto{\pgfqpoint{4.222268in}{3.278785in}}%
\pgfpathlineto{\pgfqpoint{4.214886in}{3.259734in}}%
\pgfpathclose%
\pgfusepath{fill}%
\end{pgfscope}%
\begin{pgfscope}%
\pgfpathrectangle{\pgfqpoint{1.254980in}{0.150000in}}{\pgfqpoint{5.490039in}{5.490039in}}%
\pgfusepath{clip}%
\pgfsetbuttcap%
\pgfsetroundjoin%
\definecolor{currentfill}{rgb}{0.166617,0.463708,0.558119}%
\pgfsetfillcolor{currentfill}%
\pgfsetfillopacity{0.700000}%
\pgfsetlinewidth{0.000000pt}%
\definecolor{currentstroke}{rgb}{0.000000,0.000000,0.000000}%
\pgfsetstrokecolor{currentstroke}%
\pgfsetdash{}{0pt}%
\pgfpathmoveto{\pgfqpoint{4.457478in}{3.386335in}}%
\pgfpathlineto{\pgfqpoint{4.470339in}{3.377306in}}%
\pgfpathlineto{\pgfqpoint{4.483204in}{3.368457in}}%
\pgfpathlineto{\pgfqpoint{4.496074in}{3.359787in}}%
\pgfpathlineto{\pgfqpoint{4.508948in}{3.351295in}}%
\pgfpathlineto{\pgfqpoint{4.516308in}{3.371789in}}%
\pgfpathlineto{\pgfqpoint{4.523669in}{3.392629in}}%
\pgfpathlineto{\pgfqpoint{4.531031in}{3.413822in}}%
\pgfpathlineto{\pgfqpoint{4.538396in}{3.435376in}}%
\pgfpathlineto{\pgfqpoint{4.525528in}{3.444502in}}%
\pgfpathlineto{\pgfqpoint{4.512665in}{3.453806in}}%
\pgfpathlineto{\pgfqpoint{4.499806in}{3.463289in}}%
\pgfpathlineto{\pgfqpoint{4.486951in}{3.472953in}}%
\pgfpathlineto{\pgfqpoint{4.479580in}{3.450753in}}%
\pgfpathlineto{\pgfqpoint{4.472211in}{3.428922in}}%
\pgfpathlineto{\pgfqpoint{4.464844in}{3.407452in}}%
\pgfpathlineto{\pgfqpoint{4.457478in}{3.386335in}}%
\pgfpathclose%
\pgfusepath{fill}%
\end{pgfscope}%
\begin{pgfscope}%
\pgfpathrectangle{\pgfqpoint{1.254980in}{0.150000in}}{\pgfqpoint{5.490039in}{5.490039in}}%
\pgfusepath{clip}%
\pgfsetbuttcap%
\pgfsetroundjoin%
\definecolor{currentfill}{rgb}{0.296479,0.761561,0.424223}%
\pgfsetfillcolor{currentfill}%
\pgfsetfillopacity{0.700000}%
\pgfsetlinewidth{0.000000pt}%
\definecolor{currentstroke}{rgb}{0.000000,0.000000,0.000000}%
\pgfsetstrokecolor{currentstroke}%
\pgfsetdash{}{0pt}%
\pgfpathmoveto{\pgfqpoint{3.658779in}{4.182626in}}%
\pgfpathlineto{\pgfqpoint{3.671650in}{4.161115in}}%
\pgfpathlineto{\pgfqpoint{3.684515in}{4.139874in}}%
\pgfpathlineto{\pgfqpoint{3.697376in}{4.118903in}}%
\pgfpathlineto{\pgfqpoint{3.710232in}{4.098198in}}%
\pgfpathlineto{\pgfqpoint{3.717611in}{4.126515in}}%
\pgfpathlineto{\pgfqpoint{3.724989in}{4.155266in}}%
\pgfpathlineto{\pgfqpoint{3.732364in}{4.184459in}}%
\pgfpathlineto{\pgfqpoint{3.739738in}{4.214100in}}%
\pgfpathlineto{\pgfqpoint{3.726878in}{4.235453in}}%
\pgfpathlineto{\pgfqpoint{3.714013in}{4.257074in}}%
\pgfpathlineto{\pgfqpoint{3.701143in}{4.278964in}}%
\pgfpathlineto{\pgfqpoint{3.688268in}{4.301127in}}%
\pgfpathlineto{\pgfqpoint{3.680899in}{4.270823in}}%
\pgfpathlineto{\pgfqpoint{3.673528in}{4.240977in}}%
\pgfpathlineto{\pgfqpoint{3.666154in}{4.211580in}}%
\pgfpathlineto{\pgfqpoint{3.658779in}{4.182626in}}%
\pgfpathclose%
\pgfusepath{fill}%
\end{pgfscope}%
\begin{pgfscope}%
\pgfpathrectangle{\pgfqpoint{1.254980in}{0.150000in}}{\pgfqpoint{5.490039in}{5.490039in}}%
\pgfusepath{clip}%
\pgfsetbuttcap%
\pgfsetroundjoin%
\definecolor{currentfill}{rgb}{0.168126,0.459988,0.558082}%
\pgfsetfillcolor{currentfill}%
\pgfsetfillopacity{0.700000}%
\pgfsetlinewidth{0.000000pt}%
\definecolor{currentstroke}{rgb}{0.000000,0.000000,0.000000}%
\pgfsetstrokecolor{currentstroke}%
\pgfsetdash{}{0pt}%
\pgfpathmoveto{\pgfqpoint{3.634898in}{3.399362in}}%
\pgfpathlineto{\pgfqpoint{3.647724in}{3.384104in}}%
\pgfpathlineto{\pgfqpoint{3.660547in}{3.369087in}}%
\pgfpathlineto{\pgfqpoint{3.673369in}{3.354307in}}%
\pgfpathlineto{\pgfqpoint{3.686189in}{3.339764in}}%
\pgfpathlineto{\pgfqpoint{3.693643in}{3.358416in}}%
\pgfpathlineto{\pgfqpoint{3.701093in}{3.377323in}}%
\pgfpathlineto{\pgfqpoint{3.708539in}{3.396491in}}%
\pgfpathlineto{\pgfqpoint{3.715982in}{3.415926in}}%
\pgfpathlineto{\pgfqpoint{3.703165in}{3.430917in}}%
\pgfpathlineto{\pgfqpoint{3.690347in}{3.446144in}}%
\pgfpathlineto{\pgfqpoint{3.677526in}{3.461610in}}%
\pgfpathlineto{\pgfqpoint{3.664703in}{3.477317in}}%
\pgfpathlineto{\pgfqpoint{3.657258in}{3.457422in}}%
\pgfpathlineto{\pgfqpoint{3.649809in}{3.437802in}}%
\pgfpathlineto{\pgfqpoint{3.642356in}{3.418450in}}%
\pgfpathlineto{\pgfqpoint{3.634898in}{3.399362in}}%
\pgfpathclose%
\pgfusepath{fill}%
\end{pgfscope}%
\begin{pgfscope}%
\pgfpathrectangle{\pgfqpoint{1.254980in}{0.150000in}}{\pgfqpoint{5.490039in}{5.490039in}}%
\pgfusepath{clip}%
\pgfsetbuttcap%
\pgfsetroundjoin%
\definecolor{currentfill}{rgb}{0.195860,0.395433,0.555276}%
\pgfsetfillcolor{currentfill}%
\pgfsetfillopacity{0.700000}%
\pgfsetlinewidth{0.000000pt}%
\definecolor{currentstroke}{rgb}{0.000000,0.000000,0.000000}%
\pgfsetstrokecolor{currentstroke}%
\pgfsetdash{}{0pt}%
\pgfpathmoveto{\pgfqpoint{4.134023in}{3.224224in}}%
\pgfpathlineto{\pgfqpoint{4.146850in}{3.214457in}}%
\pgfpathlineto{\pgfqpoint{4.159679in}{3.204884in}}%
\pgfpathlineto{\pgfqpoint{4.172512in}{3.195504in}}%
\pgfpathlineto{\pgfqpoint{4.185348in}{3.186317in}}%
\pgfpathlineto{\pgfqpoint{4.192734in}{3.204266in}}%
\pgfpathlineto{\pgfqpoint{4.200120in}{3.222481in}}%
\pgfpathlineto{\pgfqpoint{4.207504in}{3.240968in}}%
\pgfpathlineto{\pgfqpoint{4.214886in}{3.259734in}}%
\pgfpathlineto{\pgfqpoint{4.202056in}{3.269444in}}%
\pgfpathlineto{\pgfqpoint{4.189229in}{3.279347in}}%
\pgfpathlineto{\pgfqpoint{4.176404in}{3.289443in}}%
\pgfpathlineto{\pgfqpoint{4.163582in}{3.299734in}}%
\pgfpathlineto{\pgfqpoint{4.156195in}{3.280434in}}%
\pgfpathlineto{\pgfqpoint{4.148806in}{3.261420in}}%
\pgfpathlineto{\pgfqpoint{4.141415in}{3.242685in}}%
\pgfpathlineto{\pgfqpoint{4.134023in}{3.224224in}}%
\pgfpathclose%
\pgfusepath{fill}%
\end{pgfscope}%
\begin{pgfscope}%
\pgfpathrectangle{\pgfqpoint{1.254980in}{0.150000in}}{\pgfqpoint{5.490039in}{5.490039in}}%
\pgfusepath{clip}%
\pgfsetbuttcap%
\pgfsetroundjoin%
\definecolor{currentfill}{rgb}{0.177423,0.437527,0.557565}%
\pgfsetfillcolor{currentfill}%
\pgfsetfillopacity{0.700000}%
\pgfsetlinewidth{0.000000pt}%
\definecolor{currentstroke}{rgb}{0.000000,0.000000,0.000000}%
\pgfsetstrokecolor{currentstroke}%
\pgfsetdash{}{0pt}%
\pgfpathmoveto{\pgfqpoint{3.686189in}{3.339764in}}%
\pgfpathlineto{\pgfqpoint{3.699007in}{3.325455in}}%
\pgfpathlineto{\pgfqpoint{3.711824in}{3.311379in}}%
\pgfpathlineto{\pgfqpoint{3.724640in}{3.297534in}}%
\pgfpathlineto{\pgfqpoint{3.737454in}{3.283918in}}%
\pgfpathlineto{\pgfqpoint{3.744905in}{3.302134in}}%
\pgfpathlineto{\pgfqpoint{3.752351in}{3.320598in}}%
\pgfpathlineto{\pgfqpoint{3.759795in}{3.339317in}}%
\pgfpathlineto{\pgfqpoint{3.767234in}{3.358295in}}%
\pgfpathlineto{\pgfqpoint{3.754423in}{3.372356in}}%
\pgfpathlineto{\pgfqpoint{3.741611in}{3.386648in}}%
\pgfpathlineto{\pgfqpoint{3.728797in}{3.401171in}}%
\pgfpathlineto{\pgfqpoint{3.715982in}{3.415926in}}%
\pgfpathlineto{\pgfqpoint{3.708539in}{3.396491in}}%
\pgfpathlineto{\pgfqpoint{3.701093in}{3.377323in}}%
\pgfpathlineto{\pgfqpoint{3.693643in}{3.358416in}}%
\pgfpathlineto{\pgfqpoint{3.686189in}{3.339764in}}%
\pgfpathclose%
\pgfusepath{fill}%
\end{pgfscope}%
\begin{pgfscope}%
\pgfpathrectangle{\pgfqpoint{1.254980in}{0.150000in}}{\pgfqpoint{5.490039in}{5.490039in}}%
\pgfusepath{clip}%
\pgfsetbuttcap%
\pgfsetroundjoin%
\definecolor{currentfill}{rgb}{0.157729,0.485932,0.558013}%
\pgfsetfillcolor{currentfill}%
\pgfsetfillopacity{0.700000}%
\pgfsetlinewidth{0.000000pt}%
\definecolor{currentstroke}{rgb}{0.000000,0.000000,0.000000}%
\pgfsetstrokecolor{currentstroke}%
\pgfsetdash{}{0pt}%
\pgfpathmoveto{\pgfqpoint{3.583570in}{3.462830in}}%
\pgfpathlineto{\pgfqpoint{3.596406in}{3.446593in}}%
\pgfpathlineto{\pgfqpoint{3.609240in}{3.430604in}}%
\pgfpathlineto{\pgfqpoint{3.622070in}{3.414861in}}%
\pgfpathlineto{\pgfqpoint{3.634898in}{3.399362in}}%
\pgfpathlineto{\pgfqpoint{3.642356in}{3.418450in}}%
\pgfpathlineto{\pgfqpoint{3.649809in}{3.437802in}}%
\pgfpathlineto{\pgfqpoint{3.657258in}{3.457422in}}%
\pgfpathlineto{\pgfqpoint{3.664703in}{3.477317in}}%
\pgfpathlineto{\pgfqpoint{3.651878in}{3.493266in}}%
\pgfpathlineto{\pgfqpoint{3.639050in}{3.509459in}}%
\pgfpathlineto{\pgfqpoint{3.626220in}{3.525899in}}%
\pgfpathlineto{\pgfqpoint{3.613386in}{3.542587in}}%
\pgfpathlineto{\pgfqpoint{3.605938in}{3.522231in}}%
\pgfpathlineto{\pgfqpoint{3.598487in}{3.502156in}}%
\pgfpathlineto{\pgfqpoint{3.591031in}{3.482358in}}%
\pgfpathlineto{\pgfqpoint{3.583570in}{3.462830in}}%
\pgfpathclose%
\pgfusepath{fill}%
\end{pgfscope}%
\begin{pgfscope}%
\pgfpathrectangle{\pgfqpoint{1.254980in}{0.150000in}}{\pgfqpoint{5.490039in}{5.490039in}}%
\pgfusepath{clip}%
\pgfsetbuttcap%
\pgfsetroundjoin%
\definecolor{currentfill}{rgb}{0.126453,0.570633,0.549841}%
\pgfsetfillcolor{currentfill}%
\pgfsetfillopacity{0.700000}%
\pgfsetlinewidth{0.000000pt}%
\definecolor{currentstroke}{rgb}{0.000000,0.000000,0.000000}%
\pgfsetstrokecolor{currentstroke}%
\pgfsetdash{}{0pt}%
\pgfpathmoveto{\pgfqpoint{3.510584in}{3.685292in}}%
\pgfpathlineto{\pgfqpoint{3.523449in}{3.666538in}}%
\pgfpathlineto{\pgfqpoint{3.536310in}{3.648051in}}%
\pgfpathlineto{\pgfqpoint{3.549166in}{3.629827in}}%
\pgfpathlineto{\pgfqpoint{3.562017in}{3.611865in}}%
\pgfpathlineto{\pgfqpoint{3.569463in}{3.632980in}}%
\pgfpathlineto{\pgfqpoint{3.576904in}{3.654395in}}%
\pgfpathlineto{\pgfqpoint{3.584341in}{3.676116in}}%
\pgfpathlineto{\pgfqpoint{3.591774in}{3.698149in}}%
\pgfpathlineto{\pgfqpoint{3.578924in}{3.716596in}}%
\pgfpathlineto{\pgfqpoint{3.566069in}{3.735306in}}%
\pgfpathlineto{\pgfqpoint{3.553210in}{3.754280in}}%
\pgfpathlineto{\pgfqpoint{3.540345in}{3.773520in}}%
\pgfpathlineto{\pgfqpoint{3.532912in}{3.750990in}}%
\pgfpathlineto{\pgfqpoint{3.525474in}{3.728779in}}%
\pgfpathlineto{\pgfqpoint{3.518031in}{3.706881in}}%
\pgfpathlineto{\pgfqpoint{3.510584in}{3.685292in}}%
\pgfpathclose%
\pgfusepath{fill}%
\end{pgfscope}%
\begin{pgfscope}%
\pgfpathrectangle{\pgfqpoint{1.254980in}{0.150000in}}{\pgfqpoint{5.490039in}{5.490039in}}%
\pgfusepath{clip}%
\pgfsetbuttcap%
\pgfsetroundjoin%
\definecolor{currentfill}{rgb}{0.185556,0.418570,0.556753}%
\pgfsetfillcolor{currentfill}%
\pgfsetfillopacity{0.700000}%
\pgfsetlinewidth{0.000000pt}%
\definecolor{currentstroke}{rgb}{0.000000,0.000000,0.000000}%
\pgfsetstrokecolor{currentstroke}%
\pgfsetdash{}{0pt}%
\pgfpathmoveto{\pgfqpoint{3.737454in}{3.283918in}}%
\pgfpathlineto{\pgfqpoint{3.750268in}{3.270529in}}%
\pgfpathlineto{\pgfqpoint{3.763081in}{3.257366in}}%
\pgfpathlineto{\pgfqpoint{3.775894in}{3.244427in}}%
\pgfpathlineto{\pgfqpoint{3.788706in}{3.231710in}}%
\pgfpathlineto{\pgfqpoint{3.796153in}{3.249493in}}%
\pgfpathlineto{\pgfqpoint{3.803596in}{3.267516in}}%
\pgfpathlineto{\pgfqpoint{3.811035in}{3.285787in}}%
\pgfpathlineto{\pgfqpoint{3.818471in}{3.304309in}}%
\pgfpathlineto{\pgfqpoint{3.805663in}{3.317469in}}%
\pgfpathlineto{\pgfqpoint{3.792854in}{3.330853in}}%
\pgfpathlineto{\pgfqpoint{3.780044in}{3.344460in}}%
\pgfpathlineto{\pgfqpoint{3.767234in}{3.358295in}}%
\pgfpathlineto{\pgfqpoint{3.759795in}{3.339317in}}%
\pgfpathlineto{\pgfqpoint{3.752351in}{3.320598in}}%
\pgfpathlineto{\pgfqpoint{3.744905in}{3.302134in}}%
\pgfpathlineto{\pgfqpoint{3.737454in}{3.283918in}}%
\pgfpathclose%
\pgfusepath{fill}%
\end{pgfscope}%
\begin{pgfscope}%
\pgfpathrectangle{\pgfqpoint{1.254980in}{0.150000in}}{\pgfqpoint{5.490039in}{5.490039in}}%
\pgfusepath{clip}%
\pgfsetbuttcap%
\pgfsetroundjoin%
\definecolor{currentfill}{rgb}{0.160665,0.478540,0.558115}%
\pgfsetfillcolor{currentfill}%
\pgfsetfillopacity{0.700000}%
\pgfsetlinewidth{0.000000pt}%
\definecolor{currentstroke}{rgb}{0.000000,0.000000,0.000000}%
\pgfsetstrokecolor{currentstroke}%
\pgfsetdash{}{0pt}%
\pgfpathmoveto{\pgfqpoint{4.538396in}{3.435376in}}%
\pgfpathlineto{\pgfqpoint{4.551268in}{3.426428in}}%
\pgfpathlineto{\pgfqpoint{4.564145in}{3.417657in}}%
\pgfpathlineto{\pgfqpoint{4.577026in}{3.409063in}}%
\pgfpathlineto{\pgfqpoint{4.589913in}{3.400644in}}%
\pgfpathlineto{\pgfqpoint{4.597273in}{3.421915in}}%
\pgfpathlineto{\pgfqpoint{4.604635in}{3.443555in}}%
\pgfpathlineto{\pgfqpoint{4.612001in}{3.465572in}}%
\pgfpathlineto{\pgfqpoint{4.599120in}{3.474486in}}%
\pgfpathlineto{\pgfqpoint{4.586243in}{3.483574in}}%
\pgfpathlineto{\pgfqpoint{4.573372in}{3.492839in}}%
\pgfpathlineto{\pgfqpoint{4.560504in}{3.502282in}}%
\pgfpathlineto{\pgfqpoint{4.553132in}{3.479598in}}%
\pgfpathlineto{\pgfqpoint{4.545763in}{3.457299in}}%
\pgfpathlineto{\pgfqpoint{4.538396in}{3.435376in}}%
\pgfpathclose%
\pgfusepath{fill}%
\end{pgfscope}%
\begin{pgfscope}%
\pgfpathrectangle{\pgfqpoint{1.254980in}{0.150000in}}{\pgfqpoint{5.490039in}{5.490039in}}%
\pgfusepath{clip}%
\pgfsetbuttcap%
\pgfsetroundjoin%
\definecolor{currentfill}{rgb}{0.150148,0.676631,0.506589}%
\pgfsetfillcolor{currentfill}%
\pgfsetfillopacity{0.700000}%
\pgfsetlinewidth{0.000000pt}%
\definecolor{currentstroke}{rgb}{0.000000,0.000000,0.000000}%
\pgfsetstrokecolor{currentstroke}%
\pgfsetdash{}{0pt}%
\pgfpathmoveto{\pgfqpoint{3.518530in}{3.948726in}}%
\pgfpathlineto{\pgfqpoint{3.531416in}{3.927870in}}%
\pgfpathlineto{\pgfqpoint{3.544296in}{3.907292in}}%
\pgfpathlineto{\pgfqpoint{3.557170in}{3.886989in}}%
\pgfpathlineto{\pgfqpoint{3.570039in}{3.866959in}}%
\pgfpathlineto{\pgfqpoint{3.577453in}{3.891179in}}%
\pgfpathlineto{\pgfqpoint{3.584863in}{3.915755in}}%
\pgfpathlineto{\pgfqpoint{3.592269in}{3.940695in}}%
\pgfpathlineto{\pgfqpoint{3.599672in}{3.966004in}}%
\pgfpathlineto{\pgfqpoint{3.586802in}{3.986586in}}%
\pgfpathlineto{\pgfqpoint{3.573926in}{4.007442in}}%
\pgfpathlineto{\pgfqpoint{3.561044in}{4.028574in}}%
\pgfpathlineto{\pgfqpoint{3.548157in}{4.049985in}}%
\pgfpathlineto{\pgfqpoint{3.540756in}{4.024109in}}%
\pgfpathlineto{\pgfqpoint{3.533351in}{3.998612in}}%
\pgfpathlineto{\pgfqpoint{3.525942in}{3.973487in}}%
\pgfpathlineto{\pgfqpoint{3.518530in}{3.948726in}}%
\pgfpathclose%
\pgfusepath{fill}%
\end{pgfscope}%
\begin{pgfscope}%
\pgfpathrectangle{\pgfqpoint{1.254980in}{0.150000in}}{\pgfqpoint{5.490039in}{5.490039in}}%
\pgfusepath{clip}%
\pgfsetbuttcap%
\pgfsetroundjoin%
\definecolor{currentfill}{rgb}{0.196571,0.711827,0.479221}%
\pgfsetfillcolor{currentfill}%
\pgfsetfillopacity{0.700000}%
\pgfsetlinewidth{0.000000pt}%
\definecolor{currentstroke}{rgb}{0.000000,0.000000,0.000000}%
\pgfsetstrokecolor{currentstroke}%
\pgfsetdash{}{0pt}%
\pgfpathmoveto{\pgfqpoint{3.548157in}{4.049985in}}%
\pgfpathlineto{\pgfqpoint{3.561044in}{4.028574in}}%
\pgfpathlineto{\pgfqpoint{3.573926in}{4.007442in}}%
\pgfpathlineto{\pgfqpoint{3.586802in}{3.986586in}}%
\pgfpathlineto{\pgfqpoint{3.599672in}{3.966004in}}%
\pgfpathlineto{\pgfqpoint{3.607071in}{3.991690in}}%
\pgfpathlineto{\pgfqpoint{3.614467in}{4.017760in}}%
\pgfpathlineto{\pgfqpoint{3.621860in}{4.044220in}}%
\pgfpathlineto{\pgfqpoint{3.629249in}{4.071077in}}%
\pgfpathlineto{\pgfqpoint{3.616377in}{4.092244in}}%
\pgfpathlineto{\pgfqpoint{3.603498in}{4.113685in}}%
\pgfpathlineto{\pgfqpoint{3.590614in}{4.135404in}}%
\pgfpathlineto{\pgfqpoint{3.577724in}{4.157404in}}%
\pgfpathlineto{\pgfqpoint{3.570337in}{4.129947in}}%
\pgfpathlineto{\pgfqpoint{3.562947in}{4.102897in}}%
\pgfpathlineto{\pgfqpoint{3.555554in}{4.076245in}}%
\pgfpathlineto{\pgfqpoint{3.548157in}{4.049985in}}%
\pgfpathclose%
\pgfusepath{fill}%
\end{pgfscope}%
\begin{pgfscope}%
\pgfpathrectangle{\pgfqpoint{1.254980in}{0.150000in}}{\pgfqpoint{5.490039in}{5.490039in}}%
\pgfusepath{clip}%
\pgfsetbuttcap%
\pgfsetroundjoin%
\definecolor{currentfill}{rgb}{0.199430,0.387607,0.554642}%
\pgfsetfillcolor{currentfill}%
\pgfsetfillopacity{0.700000}%
\pgfsetlinewidth{0.000000pt}%
\definecolor{currentstroke}{rgb}{0.000000,0.000000,0.000000}%
\pgfsetstrokecolor{currentstroke}%
\pgfsetdash{}{0pt}%
\pgfpathmoveto{\pgfqpoint{3.920945in}{3.206866in}}%
\pgfpathlineto{\pgfqpoint{3.933758in}{3.195643in}}%
\pgfpathlineto{\pgfqpoint{3.946571in}{3.184629in}}%
\pgfpathlineto{\pgfqpoint{3.959386in}{3.173822in}}%
\pgfpathlineto{\pgfqpoint{3.972203in}{3.163220in}}%
\pgfpathlineto{\pgfqpoint{3.979623in}{3.180620in}}%
\pgfpathlineto{\pgfqpoint{3.987041in}{3.198262in}}%
\pgfpathlineto{\pgfqpoint{3.994456in}{3.216150in}}%
\pgfpathlineto{\pgfqpoint{4.001868in}{3.234291in}}%
\pgfpathlineto{\pgfqpoint{3.989056in}{3.245361in}}%
\pgfpathlineto{\pgfqpoint{3.976245in}{3.256637in}}%
\pgfpathlineto{\pgfqpoint{3.963436in}{3.268121in}}%
\pgfpathlineto{\pgfqpoint{3.950627in}{3.279813in}}%
\pgfpathlineto{\pgfqpoint{3.943211in}{3.261192in}}%
\pgfpathlineto{\pgfqpoint{3.935792in}{3.242831in}}%
\pgfpathlineto{\pgfqpoint{3.928370in}{3.224724in}}%
\pgfpathlineto{\pgfqpoint{3.920945in}{3.206866in}}%
\pgfpathclose%
\pgfusepath{fill}%
\end{pgfscope}%
\begin{pgfscope}%
\pgfpathrectangle{\pgfqpoint{1.254980in}{0.150000in}}{\pgfqpoint{5.490039in}{5.490039in}}%
\pgfusepath{clip}%
\pgfsetbuttcap%
\pgfsetroundjoin%
\definecolor{currentfill}{rgb}{0.421908,0.805774,0.351910}%
\pgfsetfillcolor{currentfill}%
\pgfsetfillopacity{0.700000}%
\pgfsetlinewidth{0.000000pt}%
\definecolor{currentstroke}{rgb}{0.000000,0.000000,0.000000}%
\pgfsetstrokecolor{currentstroke}%
\pgfsetdash{}{0pt}%
\pgfpathmoveto{\pgfqpoint{3.769217in}{4.337320in}}%
\pgfpathlineto{\pgfqpoint{3.782077in}{4.315551in}}%
\pgfpathlineto{\pgfqpoint{3.794933in}{4.294047in}}%
\pgfpathlineto{\pgfqpoint{3.807784in}{4.272804in}}%
\pgfpathlineto{\pgfqpoint{3.820630in}{4.251821in}}%
\pgfpathlineto{\pgfqpoint{3.828003in}{4.283130in}}%
\pgfpathlineto{\pgfqpoint{3.835374in}{4.314928in}}%
\pgfpathlineto{\pgfqpoint{3.842746in}{4.347226in}}%
\pgfpathlineto{\pgfqpoint{3.829895in}{4.368740in}}%
\pgfpathlineto{\pgfqpoint{3.817040in}{4.390515in}}%
\pgfpathlineto{\pgfqpoint{3.804180in}{4.412553in}}%
\pgfpathlineto{\pgfqpoint{3.791315in}{4.434857in}}%
\pgfpathlineto{\pgfqpoint{3.783950in}{4.401840in}}%
\pgfpathlineto{\pgfqpoint{3.776584in}{4.369331in}}%
\pgfpathlineto{\pgfqpoint{3.769217in}{4.337320in}}%
\pgfpathclose%
\pgfusepath{fill}%
\end{pgfscope}%
\begin{pgfscope}%
\pgfpathrectangle{\pgfqpoint{1.254980in}{0.150000in}}{\pgfqpoint{5.490039in}{5.490039in}}%
\pgfusepath{clip}%
\pgfsetbuttcap%
\pgfsetroundjoin%
\definecolor{currentfill}{rgb}{0.201239,0.383670,0.554294}%
\pgfsetfillcolor{currentfill}%
\pgfsetfillopacity{0.700000}%
\pgfsetlinewidth{0.000000pt}%
\definecolor{currentstroke}{rgb}{0.000000,0.000000,0.000000}%
\pgfsetstrokecolor{currentstroke}%
\pgfsetdash{}{0pt}%
\pgfpathmoveto{\pgfqpoint{4.053134in}{3.192043in}}%
\pgfpathlineto{\pgfqpoint{4.065956in}{3.181984in}}%
\pgfpathlineto{\pgfqpoint{4.078780in}{3.172123in}}%
\pgfpathlineto{\pgfqpoint{4.091606in}{3.162459in}}%
\pgfpathlineto{\pgfqpoint{4.104436in}{3.152992in}}%
\pgfpathlineto{\pgfqpoint{4.111836in}{3.170419in}}%
\pgfpathlineto{\pgfqpoint{4.119233in}{3.188097in}}%
\pgfpathlineto{\pgfqpoint{4.126629in}{3.206030in}}%
\pgfpathlineto{\pgfqpoint{4.134023in}{3.224224in}}%
\pgfpathlineto{\pgfqpoint{4.121199in}{3.234186in}}%
\pgfpathlineto{\pgfqpoint{4.108377in}{3.244345in}}%
\pgfpathlineto{\pgfqpoint{4.095558in}{3.254702in}}%
\pgfpathlineto{\pgfqpoint{4.082741in}{3.265258in}}%
\pgfpathlineto{\pgfqpoint{4.075343in}{3.246557in}}%
\pgfpathlineto{\pgfqpoint{4.067942in}{3.228125in}}%
\pgfpathlineto{\pgfqpoint{4.060539in}{3.209956in}}%
\pgfpathlineto{\pgfqpoint{4.053134in}{3.192043in}}%
\pgfpathclose%
\pgfusepath{fill}%
\end{pgfscope}%
\begin{pgfscope}%
\pgfpathrectangle{\pgfqpoint{1.254980in}{0.150000in}}{\pgfqpoint{5.490039in}{5.490039in}}%
\pgfusepath{clip}%
\pgfsetbuttcap%
\pgfsetroundjoin%
\definecolor{currentfill}{rgb}{0.147607,0.511733,0.557049}%
\pgfsetfillcolor{currentfill}%
\pgfsetfillopacity{0.700000}%
\pgfsetlinewidth{0.000000pt}%
\definecolor{currentstroke}{rgb}{0.000000,0.000000,0.000000}%
\pgfsetstrokecolor{currentstroke}%
\pgfsetdash{}{0pt}%
\pgfpathmoveto{\pgfqpoint{3.532192in}{3.530297in}}%
\pgfpathlineto{\pgfqpoint{3.545042in}{3.513048in}}%
\pgfpathlineto{\pgfqpoint{3.557888in}{3.496055in}}%
\pgfpathlineto{\pgfqpoint{3.570731in}{3.479317in}}%
\pgfpathlineto{\pgfqpoint{3.583570in}{3.462830in}}%
\pgfpathlineto{\pgfqpoint{3.591031in}{3.482358in}}%
\pgfpathlineto{\pgfqpoint{3.598487in}{3.502156in}}%
\pgfpathlineto{\pgfqpoint{3.605938in}{3.522231in}}%
\pgfpathlineto{\pgfqpoint{3.613386in}{3.542587in}}%
\pgfpathlineto{\pgfqpoint{3.600549in}{3.559525in}}%
\pgfpathlineto{\pgfqpoint{3.587709in}{3.576716in}}%
\pgfpathlineto{\pgfqpoint{3.574865in}{3.594162in}}%
\pgfpathlineto{\pgfqpoint{3.562017in}{3.611865in}}%
\pgfpathlineto{\pgfqpoint{3.554568in}{3.591045in}}%
\pgfpathlineto{\pgfqpoint{3.547114in}{3.570514in}}%
\pgfpathlineto{\pgfqpoint{3.539655in}{3.550267in}}%
\pgfpathlineto{\pgfqpoint{3.532192in}{3.530297in}}%
\pgfpathclose%
\pgfusepath{fill}%
\end{pgfscope}%
\begin{pgfscope}%
\pgfpathrectangle{\pgfqpoint{1.254980in}{0.150000in}}{\pgfqpoint{5.490039in}{5.490039in}}%
\pgfusepath{clip}%
\pgfsetbuttcap%
\pgfsetroundjoin%
\definecolor{currentfill}{rgb}{0.123444,0.636809,0.528763}%
\pgfsetfillcolor{currentfill}%
\pgfsetfillopacity{0.700000}%
\pgfsetlinewidth{0.000000pt}%
\definecolor{currentstroke}{rgb}{0.000000,0.000000,0.000000}%
\pgfsetstrokecolor{currentstroke}%
\pgfsetdash{}{0pt}%
\pgfpathmoveto{\pgfqpoint{3.488837in}{3.853203in}}%
\pgfpathlineto{\pgfqpoint{3.501722in}{3.832869in}}%
\pgfpathlineto{\pgfqpoint{3.514602in}{3.812813in}}%
\pgfpathlineto{\pgfqpoint{3.527476in}{3.793031in}}%
\pgfpathlineto{\pgfqpoint{3.540345in}{3.773520in}}%
\pgfpathlineto{\pgfqpoint{3.547775in}{3.796377in}}%
\pgfpathlineto{\pgfqpoint{3.555200in}{3.819564in}}%
\pgfpathlineto{\pgfqpoint{3.562622in}{3.843090in}}%
\pgfpathlineto{\pgfqpoint{3.570039in}{3.866959in}}%
\pgfpathlineto{\pgfqpoint{3.557170in}{3.886989in}}%
\pgfpathlineto{\pgfqpoint{3.544296in}{3.907292in}}%
\pgfpathlineto{\pgfqpoint{3.531416in}{3.927870in}}%
\pgfpathlineto{\pgfqpoint{3.518530in}{3.948726in}}%
\pgfpathlineto{\pgfqpoint{3.511113in}{3.924323in}}%
\pgfpathlineto{\pgfqpoint{3.503692in}{3.900273in}}%
\pgfpathlineto{\pgfqpoint{3.496267in}{3.876568in}}%
\pgfpathlineto{\pgfqpoint{3.488837in}{3.853203in}}%
\pgfpathclose%
\pgfusepath{fill}%
\end{pgfscope}%
\begin{pgfscope}%
\pgfpathrectangle{\pgfqpoint{1.254980in}{0.150000in}}{\pgfqpoint{5.490039in}{5.490039in}}%
\pgfusepath{clip}%
\pgfsetbuttcap%
\pgfsetroundjoin%
\definecolor{currentfill}{rgb}{0.274149,0.751988,0.436601}%
\pgfsetfillcolor{currentfill}%
\pgfsetfillopacity{0.700000}%
\pgfsetlinewidth{0.000000pt}%
\definecolor{currentstroke}{rgb}{0.000000,0.000000,0.000000}%
\pgfsetstrokecolor{currentstroke}%
\pgfsetdash{}{0pt}%
\pgfpathmoveto{\pgfqpoint{3.577724in}{4.157404in}}%
\pgfpathlineto{\pgfqpoint{3.590614in}{4.135404in}}%
\pgfpathlineto{\pgfqpoint{3.603498in}{4.113685in}}%
\pgfpathlineto{\pgfqpoint{3.616377in}{4.092244in}}%
\pgfpathlineto{\pgfqpoint{3.629249in}{4.071077in}}%
\pgfpathlineto{\pgfqpoint{3.636636in}{4.098339in}}%
\pgfpathlineto{\pgfqpoint{3.644020in}{4.126013in}}%
\pgfpathlineto{\pgfqpoint{3.651401in}{4.154106in}}%
\pgfpathlineto{\pgfqpoint{3.658779in}{4.182626in}}%
\pgfpathlineto{\pgfqpoint{3.645902in}{4.204411in}}%
\pgfpathlineto{\pgfqpoint{3.633020in}{4.226472in}}%
\pgfpathlineto{\pgfqpoint{3.620132in}{4.248813in}}%
\pgfpathlineto{\pgfqpoint{3.607238in}{4.271434in}}%
\pgfpathlineto{\pgfqpoint{3.599864in}{4.242281in}}%
\pgfpathlineto{\pgfqpoint{3.592487in}{4.213563in}}%
\pgfpathlineto{\pgfqpoint{3.585107in}{4.185273in}}%
\pgfpathlineto{\pgfqpoint{3.577724in}{4.157404in}}%
\pgfpathclose%
\pgfusepath{fill}%
\end{pgfscope}%
\begin{pgfscope}%
\pgfpathrectangle{\pgfqpoint{1.254980in}{0.150000in}}{\pgfqpoint{5.490039in}{5.490039in}}%
\pgfusepath{clip}%
\pgfsetbuttcap%
\pgfsetroundjoin%
\definecolor{currentfill}{rgb}{0.195860,0.395433,0.555276}%
\pgfsetfillcolor{currentfill}%
\pgfsetfillopacity{0.700000}%
\pgfsetlinewidth{0.000000pt}%
\definecolor{currentstroke}{rgb}{0.000000,0.000000,0.000000}%
\pgfsetstrokecolor{currentstroke}%
\pgfsetdash{}{0pt}%
\pgfpathmoveto{\pgfqpoint{3.788706in}{3.231710in}}%
\pgfpathlineto{\pgfqpoint{3.801518in}{3.219215in}}%
\pgfpathlineto{\pgfqpoint{3.814331in}{3.206939in}}%
\pgfpathlineto{\pgfqpoint{3.827143in}{3.194881in}}%
\pgfpathlineto{\pgfqpoint{3.839956in}{3.183039in}}%
\pgfpathlineto{\pgfqpoint{3.847398in}{3.200389in}}%
\pgfpathlineto{\pgfqpoint{3.854837in}{3.217973in}}%
\pgfpathlineto{\pgfqpoint{3.862272in}{3.235797in}}%
\pgfpathlineto{\pgfqpoint{3.869705in}{3.253865in}}%
\pgfpathlineto{\pgfqpoint{3.856896in}{3.266149in}}%
\pgfpathlineto{\pgfqpoint{3.844088in}{3.278650in}}%
\pgfpathlineto{\pgfqpoint{3.831280in}{3.291370in}}%
\pgfpathlineto{\pgfqpoint{3.818471in}{3.304309in}}%
\pgfpathlineto{\pgfqpoint{3.811035in}{3.285787in}}%
\pgfpathlineto{\pgfqpoint{3.803596in}{3.267516in}}%
\pgfpathlineto{\pgfqpoint{3.796153in}{3.249493in}}%
\pgfpathlineto{\pgfqpoint{3.788706in}{3.231710in}}%
\pgfpathclose%
\pgfusepath{fill}%
\end{pgfscope}%
\begin{pgfscope}%
\pgfpathrectangle{\pgfqpoint{1.254980in}{0.150000in}}{\pgfqpoint{5.490039in}{5.490039in}}%
\pgfusepath{clip}%
\pgfsetbuttcap%
\pgfsetroundjoin%
\definecolor{currentfill}{rgb}{0.187231,0.414746,0.556547}%
\pgfsetfillcolor{currentfill}%
\pgfsetfillopacity{0.700000}%
\pgfsetlinewidth{0.000000pt}%
\definecolor{currentstroke}{rgb}{0.000000,0.000000,0.000000}%
\pgfsetstrokecolor{currentstroke}%
\pgfsetdash{}{0pt}%
\pgfpathmoveto{\pgfqpoint{4.347128in}{3.262433in}}%
\pgfpathlineto{\pgfqpoint{4.359984in}{3.253860in}}%
\pgfpathlineto{\pgfqpoint{4.372845in}{3.245470in}}%
\pgfpathlineto{\pgfqpoint{4.385711in}{3.237262in}}%
\pgfpathlineto{\pgfqpoint{4.398580in}{3.229235in}}%
\pgfpathlineto{\pgfqpoint{4.405942in}{3.247788in}}%
\pgfpathlineto{\pgfqpoint{4.413303in}{3.266636in}}%
\pgfpathlineto{\pgfqpoint{4.420664in}{3.285788in}}%
\pgfpathlineto{\pgfqpoint{4.428025in}{3.305249in}}%
\pgfpathlineto{\pgfqpoint{4.415162in}{3.313853in}}%
\pgfpathlineto{\pgfqpoint{4.402303in}{3.322638in}}%
\pgfpathlineto{\pgfqpoint{4.389449in}{3.331606in}}%
\pgfpathlineto{\pgfqpoint{4.376598in}{3.340756in}}%
\pgfpathlineto{\pgfqpoint{4.369230in}{3.320707in}}%
\pgfpathlineto{\pgfqpoint{4.361863in}{3.300974in}}%
\pgfpathlineto{\pgfqpoint{4.354495in}{3.281552in}}%
\pgfpathlineto{\pgfqpoint{4.347128in}{3.262433in}}%
\pgfpathclose%
\pgfusepath{fill}%
\end{pgfscope}%
\begin{pgfscope}%
\pgfpathrectangle{\pgfqpoint{1.254980in}{0.150000in}}{\pgfqpoint{5.490039in}{5.490039in}}%
\pgfusepath{clip}%
\pgfsetbuttcap%
\pgfsetroundjoin%
\definecolor{currentfill}{rgb}{0.404001,0.800275,0.362552}%
\pgfsetfillcolor{currentfill}%
\pgfsetfillopacity{0.700000}%
\pgfsetlinewidth{0.000000pt}%
\definecolor{currentstroke}{rgb}{0.000000,0.000000,0.000000}%
\pgfsetstrokecolor{currentstroke}%
\pgfsetdash{}{0pt}%
\pgfpathmoveto{\pgfqpoint{3.688268in}{4.301127in}}%
\pgfpathlineto{\pgfqpoint{3.701143in}{4.278964in}}%
\pgfpathlineto{\pgfqpoint{3.714013in}{4.257074in}}%
\pgfpathlineto{\pgfqpoint{3.726878in}{4.235453in}}%
\pgfpathlineto{\pgfqpoint{3.739738in}{4.214100in}}%
\pgfpathlineto{\pgfqpoint{3.747110in}{4.244199in}}%
\pgfpathlineto{\pgfqpoint{3.754480in}{4.274763in}}%
\pgfpathlineto{\pgfqpoint{3.761849in}{4.305801in}}%
\pgfpathlineto{\pgfqpoint{3.769217in}{4.337320in}}%
\pgfpathlineto{\pgfqpoint{3.756352in}{4.359356in}}%
\pgfpathlineto{\pgfqpoint{3.743481in}{4.381661in}}%
\pgfpathlineto{\pgfqpoint{3.730606in}{4.404237in}}%
\pgfpathlineto{\pgfqpoint{3.717724in}{4.427087in}}%
\pgfpathlineto{\pgfqpoint{3.710363in}{4.394869in}}%
\pgfpathlineto{\pgfqpoint{3.702999in}{4.363142in}}%
\pgfpathlineto{\pgfqpoint{3.695634in}{4.331898in}}%
\pgfpathlineto{\pgfqpoint{3.688268in}{4.301127in}}%
\pgfpathclose%
\pgfusepath{fill}%
\end{pgfscope}%
\begin{pgfscope}%
\pgfpathrectangle{\pgfqpoint{1.254980in}{0.150000in}}{\pgfqpoint{5.490039in}{5.490039in}}%
\pgfusepath{clip}%
\pgfsetbuttcap%
\pgfsetroundjoin%
\definecolor{currentfill}{rgb}{0.179019,0.433756,0.557430}%
\pgfsetfillcolor{currentfill}%
\pgfsetfillopacity{0.700000}%
\pgfsetlinewidth{0.000000pt}%
\definecolor{currentstroke}{rgb}{0.000000,0.000000,0.000000}%
\pgfsetstrokecolor{currentstroke}%
\pgfsetdash{}{0pt}%
\pgfpathmoveto{\pgfqpoint{4.428025in}{3.305249in}}%
\pgfpathlineto{\pgfqpoint{4.440893in}{3.296826in}}%
\pgfpathlineto{\pgfqpoint{4.453765in}{3.288583in}}%
\pgfpathlineto{\pgfqpoint{4.466641in}{3.280518in}}%
\pgfpathlineto{\pgfqpoint{4.479523in}{3.272631in}}%
\pgfpathlineto{\pgfqpoint{4.486878in}{3.291815in}}%
\pgfpathlineto{\pgfqpoint{4.494234in}{3.311316in}}%
\pgfpathlineto{\pgfqpoint{4.501590in}{3.331140in}}%
\pgfpathlineto{\pgfqpoint{4.508948in}{3.351295in}}%
\pgfpathlineto{\pgfqpoint{4.496074in}{3.359787in}}%
\pgfpathlineto{\pgfqpoint{4.483204in}{3.368457in}}%
\pgfpathlineto{\pgfqpoint{4.470339in}{3.377306in}}%
\pgfpathlineto{\pgfqpoint{4.457478in}{3.386335in}}%
\pgfpathlineto{\pgfqpoint{4.450113in}{3.365563in}}%
\pgfpathlineto{\pgfqpoint{4.442750in}{3.345130in}}%
\pgfpathlineto{\pgfqpoint{4.435387in}{3.325028in}}%
\pgfpathlineto{\pgfqpoint{4.428025in}{3.305249in}}%
\pgfpathclose%
\pgfusepath{fill}%
\end{pgfscope}%
\begin{pgfscope}%
\pgfpathrectangle{\pgfqpoint{1.254980in}{0.150000in}}{\pgfqpoint{5.490039in}{5.490039in}}%
\pgfusepath{clip}%
\pgfsetbuttcap%
\pgfsetroundjoin%
\definecolor{currentfill}{rgb}{0.194100,0.399323,0.555565}%
\pgfsetfillcolor{currentfill}%
\pgfsetfillopacity{0.700000}%
\pgfsetlinewidth{0.000000pt}%
\definecolor{currentstroke}{rgb}{0.000000,0.000000,0.000000}%
\pgfsetstrokecolor{currentstroke}%
\pgfsetdash{}{0pt}%
\pgfpathmoveto{\pgfqpoint{4.266240in}{3.222797in}}%
\pgfpathlineto{\pgfqpoint{4.279087in}{3.214033in}}%
\pgfpathlineto{\pgfqpoint{4.291938in}{3.205456in}}%
\pgfpathlineto{\pgfqpoint{4.304793in}{3.197064in}}%
\pgfpathlineto{\pgfqpoint{4.317652in}{3.188857in}}%
\pgfpathlineto{\pgfqpoint{4.325022in}{3.206829in}}%
\pgfpathlineto{\pgfqpoint{4.332391in}{3.225078in}}%
\pgfpathlineto{\pgfqpoint{4.339760in}{3.243610in}}%
\pgfpathlineto{\pgfqpoint{4.347128in}{3.262433in}}%
\pgfpathlineto{\pgfqpoint{4.334275in}{3.271189in}}%
\pgfpathlineto{\pgfqpoint{4.321426in}{3.280131in}}%
\pgfpathlineto{\pgfqpoint{4.308581in}{3.289258in}}%
\pgfpathlineto{\pgfqpoint{4.295739in}{3.298571in}}%
\pgfpathlineto{\pgfqpoint{4.288365in}{3.279188in}}%
\pgfpathlineto{\pgfqpoint{4.280991in}{3.260103in}}%
\pgfpathlineto{\pgfqpoint{4.273616in}{3.241308in}}%
\pgfpathlineto{\pgfqpoint{4.266240in}{3.222797in}}%
\pgfpathclose%
\pgfusepath{fill}%
\end{pgfscope}%
\begin{pgfscope}%
\pgfpathrectangle{\pgfqpoint{1.254980in}{0.150000in}}{\pgfqpoint{5.490039in}{5.490039in}}%
\pgfusepath{clip}%
\pgfsetbuttcap%
\pgfsetroundjoin%
\definecolor{currentfill}{rgb}{0.171176,0.452530,0.557965}%
\pgfsetfillcolor{currentfill}%
\pgfsetfillopacity{0.700000}%
\pgfsetlinewidth{0.000000pt}%
\definecolor{currentstroke}{rgb}{0.000000,0.000000,0.000000}%
\pgfsetstrokecolor{currentstroke}%
\pgfsetdash{}{0pt}%
\pgfpathmoveto{\pgfqpoint{4.508948in}{3.351295in}}%
\pgfpathlineto{\pgfqpoint{4.521827in}{3.342981in}}%
\pgfpathlineto{\pgfqpoint{4.534711in}{3.334844in}}%
\pgfpathlineto{\pgfqpoint{4.547600in}{3.326882in}}%
\pgfpathlineto{\pgfqpoint{4.560494in}{3.319096in}}%
\pgfpathlineto{\pgfqpoint{4.567846in}{3.338968in}}%
\pgfpathlineto{\pgfqpoint{4.575199in}{3.359178in}}%
\pgfpathlineto{\pgfqpoint{4.582555in}{3.379734in}}%
\pgfpathlineto{\pgfqpoint{4.589913in}{3.400644in}}%
\pgfpathlineto{\pgfqpoint{4.577026in}{3.409063in}}%
\pgfpathlineto{\pgfqpoint{4.564145in}{3.417657in}}%
\pgfpathlineto{\pgfqpoint{4.551268in}{3.426428in}}%
\pgfpathlineto{\pgfqpoint{4.538396in}{3.435376in}}%
\pgfpathlineto{\pgfqpoint{4.531031in}{3.413822in}}%
\pgfpathlineto{\pgfqpoint{4.523669in}{3.392629in}}%
\pgfpathlineto{\pgfqpoint{4.516308in}{3.371789in}}%
\pgfpathlineto{\pgfqpoint{4.508948in}{3.351295in}}%
\pgfpathclose%
\pgfusepath{fill}%
\end{pgfscope}%
\begin{pgfscope}%
\pgfpathrectangle{\pgfqpoint{1.254980in}{0.150000in}}{\pgfqpoint{5.490039in}{5.490039in}}%
\pgfusepath{clip}%
\pgfsetbuttcap%
\pgfsetroundjoin%
\definecolor{currentfill}{rgb}{0.119738,0.603785,0.541400}%
\pgfsetfillcolor{currentfill}%
\pgfsetfillopacity{0.700000}%
\pgfsetlinewidth{0.000000pt}%
\definecolor{currentstroke}{rgb}{0.000000,0.000000,0.000000}%
\pgfsetstrokecolor{currentstroke}%
\pgfsetdash{}{0pt}%
\pgfpathmoveto{\pgfqpoint{3.459073in}{3.763016in}}%
\pgfpathlineto{\pgfqpoint{3.471959in}{3.743174in}}%
\pgfpathlineto{\pgfqpoint{3.484839in}{3.723607in}}%
\pgfpathlineto{\pgfqpoint{3.497714in}{3.704314in}}%
\pgfpathlineto{\pgfqpoint{3.510584in}{3.685292in}}%
\pgfpathlineto{\pgfqpoint{3.518031in}{3.706881in}}%
\pgfpathlineto{\pgfqpoint{3.525474in}{3.728779in}}%
\pgfpathlineto{\pgfqpoint{3.532912in}{3.750990in}}%
\pgfpathlineto{\pgfqpoint{3.540345in}{3.773520in}}%
\pgfpathlineto{\pgfqpoint{3.527476in}{3.793031in}}%
\pgfpathlineto{\pgfqpoint{3.514602in}{3.812813in}}%
\pgfpathlineto{\pgfqpoint{3.501722in}{3.832869in}}%
\pgfpathlineto{\pgfqpoint{3.488837in}{3.853203in}}%
\pgfpathlineto{\pgfqpoint{3.481403in}{3.830171in}}%
\pgfpathlineto{\pgfqpoint{3.473964in}{3.807466in}}%
\pgfpathlineto{\pgfqpoint{3.466521in}{3.785083in}}%
\pgfpathlineto{\pgfqpoint{3.459073in}{3.763016in}}%
\pgfpathclose%
\pgfusepath{fill}%
\end{pgfscope}%
\begin{pgfscope}%
\pgfpathrectangle{\pgfqpoint{1.254980in}{0.150000in}}{\pgfqpoint{5.490039in}{5.490039in}}%
\pgfusepath{clip}%
\pgfsetbuttcap%
\pgfsetroundjoin%
\definecolor{currentfill}{rgb}{0.136408,0.541173,0.554483}%
\pgfsetfillcolor{currentfill}%
\pgfsetfillopacity{0.700000}%
\pgfsetlinewidth{0.000000pt}%
\definecolor{currentstroke}{rgb}{0.000000,0.000000,0.000000}%
\pgfsetstrokecolor{currentstroke}%
\pgfsetdash{}{0pt}%
\pgfpathmoveto{\pgfqpoint{3.480751in}{3.601902in}}%
\pgfpathlineto{\pgfqpoint{3.493618in}{3.583605in}}%
\pgfpathlineto{\pgfqpoint{3.506480in}{3.565574in}}%
\pgfpathlineto{\pgfqpoint{3.519338in}{3.547805in}}%
\pgfpathlineto{\pgfqpoint{3.532192in}{3.530297in}}%
\pgfpathlineto{\pgfqpoint{3.539655in}{3.550267in}}%
\pgfpathlineto{\pgfqpoint{3.547114in}{3.570514in}}%
\pgfpathlineto{\pgfqpoint{3.554568in}{3.591045in}}%
\pgfpathlineto{\pgfqpoint{3.562017in}{3.611865in}}%
\pgfpathlineto{\pgfqpoint{3.549166in}{3.629827in}}%
\pgfpathlineto{\pgfqpoint{3.536310in}{3.648051in}}%
\pgfpathlineto{\pgfqpoint{3.523449in}{3.666538in}}%
\pgfpathlineto{\pgfqpoint{3.510584in}{3.685292in}}%
\pgfpathlineto{\pgfqpoint{3.503133in}{3.664005in}}%
\pgfpathlineto{\pgfqpoint{3.495677in}{3.643015in}}%
\pgfpathlineto{\pgfqpoint{3.488216in}{3.622316in}}%
\pgfpathlineto{\pgfqpoint{3.480751in}{3.601902in}}%
\pgfpathclose%
\pgfusepath{fill}%
\end{pgfscope}%
\begin{pgfscope}%
\pgfpathrectangle{\pgfqpoint{1.254980in}{0.150000in}}{\pgfqpoint{5.490039in}{5.490039in}}%
\pgfusepath{clip}%
\pgfsetbuttcap%
\pgfsetroundjoin%
\definecolor{currentfill}{rgb}{0.201239,0.383670,0.554294}%
\pgfsetfillcolor{currentfill}%
\pgfsetfillopacity{0.700000}%
\pgfsetlinewidth{0.000000pt}%
\definecolor{currentstroke}{rgb}{0.000000,0.000000,0.000000}%
\pgfsetstrokecolor{currentstroke}%
\pgfsetdash{}{0pt}%
\pgfpathmoveto{\pgfqpoint{4.185348in}{3.186317in}}%
\pgfpathlineto{\pgfqpoint{4.198187in}{3.177320in}}%
\pgfpathlineto{\pgfqpoint{4.211029in}{3.168514in}}%
\pgfpathlineto{\pgfqpoint{4.223875in}{3.159897in}}%
\pgfpathlineto{\pgfqpoint{4.236724in}{3.151467in}}%
\pgfpathlineto{\pgfqpoint{4.244105in}{3.168905in}}%
\pgfpathlineto{\pgfqpoint{4.251485in}{3.186602in}}%
\pgfpathlineto{\pgfqpoint{4.258863in}{3.204563in}}%
\pgfpathlineto{\pgfqpoint{4.266240in}{3.222797in}}%
\pgfpathlineto{\pgfqpoint{4.253396in}{3.231748in}}%
\pgfpathlineto{\pgfqpoint{4.240556in}{3.240887in}}%
\pgfpathlineto{\pgfqpoint{4.227720in}{3.250215in}}%
\pgfpathlineto{\pgfqpoint{4.214886in}{3.259734in}}%
\pgfpathlineto{\pgfqpoint{4.207504in}{3.240968in}}%
\pgfpathlineto{\pgfqpoint{4.200120in}{3.222481in}}%
\pgfpathlineto{\pgfqpoint{4.192734in}{3.204266in}}%
\pgfpathlineto{\pgfqpoint{4.185348in}{3.186317in}}%
\pgfpathclose%
\pgfusepath{fill}%
\end{pgfscope}%
\begin{pgfscope}%
\pgfpathrectangle{\pgfqpoint{1.254980in}{0.150000in}}{\pgfqpoint{5.490039in}{5.490039in}}%
\pgfusepath{clip}%
\pgfsetbuttcap%
\pgfsetroundjoin%
\definecolor{currentfill}{rgb}{0.206756,0.371758,0.553117}%
\pgfsetfillcolor{currentfill}%
\pgfsetfillopacity{0.700000}%
\pgfsetlinewidth{0.000000pt}%
\definecolor{currentstroke}{rgb}{0.000000,0.000000,0.000000}%
\pgfsetstrokecolor{currentstroke}%
\pgfsetdash{}{0pt}%
\pgfpathmoveto{\pgfqpoint{3.972203in}{3.163220in}}%
\pgfpathlineto{\pgfqpoint{3.985022in}{3.152823in}}%
\pgfpathlineto{\pgfqpoint{3.997842in}{3.142629in}}%
\pgfpathlineto{\pgfqpoint{4.010664in}{3.132637in}}%
\pgfpathlineto{\pgfqpoint{4.023489in}{3.122845in}}%
\pgfpathlineto{\pgfqpoint{4.030904in}{3.139788in}}%
\pgfpathlineto{\pgfqpoint{4.038316in}{3.156965in}}%
\pgfpathlineto{\pgfqpoint{4.045726in}{3.174381in}}%
\pgfpathlineto{\pgfqpoint{4.053134in}{3.192043in}}%
\pgfpathlineto{\pgfqpoint{4.040314in}{3.202303in}}%
\pgfpathlineto{\pgfqpoint{4.027497in}{3.212763in}}%
\pgfpathlineto{\pgfqpoint{4.014682in}{3.223425in}}%
\pgfpathlineto{\pgfqpoint{4.001868in}{3.234291in}}%
\pgfpathlineto{\pgfqpoint{3.994456in}{3.216150in}}%
\pgfpathlineto{\pgfqpoint{3.987041in}{3.198262in}}%
\pgfpathlineto{\pgfqpoint{3.979623in}{3.180620in}}%
\pgfpathlineto{\pgfqpoint{3.972203in}{3.163220in}}%
\pgfpathclose%
\pgfusepath{fill}%
\end{pgfscope}%
\begin{pgfscope}%
\pgfpathrectangle{\pgfqpoint{1.254980in}{0.150000in}}{\pgfqpoint{5.490039in}{5.490039in}}%
\pgfusepath{clip}%
\pgfsetbuttcap%
\pgfsetroundjoin%
\definecolor{currentfill}{rgb}{0.203063,0.379716,0.553925}%
\pgfsetfillcolor{currentfill}%
\pgfsetfillopacity{0.700000}%
\pgfsetlinewidth{0.000000pt}%
\definecolor{currentstroke}{rgb}{0.000000,0.000000,0.000000}%
\pgfsetstrokecolor{currentstroke}%
\pgfsetdash{}{0pt}%
\pgfpathmoveto{\pgfqpoint{3.839956in}{3.183039in}}%
\pgfpathlineto{\pgfqpoint{3.852769in}{3.171412in}}%
\pgfpathlineto{\pgfqpoint{3.865584in}{3.159999in}}%
\pgfpathlineto{\pgfqpoint{3.878398in}{3.148798in}}%
\pgfpathlineto{\pgfqpoint{3.891215in}{3.137807in}}%
\pgfpathlineto{\pgfqpoint{3.898652in}{3.154726in}}%
\pgfpathlineto{\pgfqpoint{3.906086in}{3.171872in}}%
\pgfpathlineto{\pgfqpoint{3.913517in}{3.189250in}}%
\pgfpathlineto{\pgfqpoint{3.920945in}{3.206866in}}%
\pgfpathlineto{\pgfqpoint{3.908134in}{3.218298in}}%
\pgfpathlineto{\pgfqpoint{3.895323in}{3.229941in}}%
\pgfpathlineto{\pgfqpoint{3.882514in}{3.241796in}}%
\pgfpathlineto{\pgfqpoint{3.869705in}{3.253865in}}%
\pgfpathlineto{\pgfqpoint{3.862272in}{3.235797in}}%
\pgfpathlineto{\pgfqpoint{3.854837in}{3.217973in}}%
\pgfpathlineto{\pgfqpoint{3.847398in}{3.200389in}}%
\pgfpathlineto{\pgfqpoint{3.839956in}{3.183039in}}%
\pgfpathclose%
\pgfusepath{fill}%
\end{pgfscope}%
\begin{pgfscope}%
\pgfpathrectangle{\pgfqpoint{1.254980in}{0.150000in}}{\pgfqpoint{5.490039in}{5.490039in}}%
\pgfusepath{clip}%
\pgfsetbuttcap%
\pgfsetroundjoin%
\definecolor{currentfill}{rgb}{0.377779,0.791781,0.377939}%
\pgfsetfillcolor{currentfill}%
\pgfsetfillopacity{0.700000}%
\pgfsetlinewidth{0.000000pt}%
\definecolor{currentstroke}{rgb}{0.000000,0.000000,0.000000}%
\pgfsetstrokecolor{currentstroke}%
\pgfsetdash{}{0pt}%
\pgfpathmoveto{\pgfqpoint{3.607238in}{4.271434in}}%
\pgfpathlineto{\pgfqpoint{3.620132in}{4.248813in}}%
\pgfpathlineto{\pgfqpoint{3.633020in}{4.226472in}}%
\pgfpathlineto{\pgfqpoint{3.645902in}{4.204411in}}%
\pgfpathlineto{\pgfqpoint{3.658779in}{4.182626in}}%
\pgfpathlineto{\pgfqpoint{3.666154in}{4.211580in}}%
\pgfpathlineto{\pgfqpoint{3.673528in}{4.240977in}}%
\pgfpathlineto{\pgfqpoint{3.680899in}{4.270823in}}%
\pgfpathlineto{\pgfqpoint{3.688268in}{4.301127in}}%
\pgfpathlineto{\pgfqpoint{3.675386in}{4.323566in}}%
\pgfpathlineto{\pgfqpoint{3.662499in}{4.346281in}}%
\pgfpathlineto{\pgfqpoint{3.649606in}{4.369278in}}%
\pgfpathlineto{\pgfqpoint{3.636707in}{4.392557in}}%
\pgfpathlineto{\pgfqpoint{3.629344in}{4.361584in}}%
\pgfpathlineto{\pgfqpoint{3.621978in}{4.331078in}}%
\pgfpathlineto{\pgfqpoint{3.614609in}{4.301031in}}%
\pgfpathlineto{\pgfqpoint{3.607238in}{4.271434in}}%
\pgfpathclose%
\pgfusepath{fill}%
\end{pgfscope}%
\begin{pgfscope}%
\pgfpathrectangle{\pgfqpoint{1.254980in}{0.150000in}}{\pgfqpoint{5.490039in}{5.490039in}}%
\pgfusepath{clip}%
\pgfsetbuttcap%
\pgfsetroundjoin%
\definecolor{currentfill}{rgb}{0.179019,0.433756,0.557430}%
\pgfsetfillcolor{currentfill}%
\pgfsetfillopacity{0.700000}%
\pgfsetlinewidth{0.000000pt}%
\definecolor{currentstroke}{rgb}{0.000000,0.000000,0.000000}%
\pgfsetstrokecolor{currentstroke}%
\pgfsetdash{}{0pt}%
\pgfpathmoveto{\pgfqpoint{3.605027in}{3.325538in}}%
\pgfpathlineto{\pgfqpoint{3.617856in}{3.310700in}}%
\pgfpathlineto{\pgfqpoint{3.630683in}{3.296102in}}%
\pgfpathlineto{\pgfqpoint{3.643508in}{3.281741in}}%
\pgfpathlineto{\pgfqpoint{3.656332in}{3.267616in}}%
\pgfpathlineto{\pgfqpoint{3.663802in}{3.285294in}}%
\pgfpathlineto{\pgfqpoint{3.671269in}{3.303209in}}%
\pgfpathlineto{\pgfqpoint{3.678731in}{3.321364in}}%
\pgfpathlineto{\pgfqpoint{3.686189in}{3.339764in}}%
\pgfpathlineto{\pgfqpoint{3.673369in}{3.354307in}}%
\pgfpathlineto{\pgfqpoint{3.660547in}{3.369087in}}%
\pgfpathlineto{\pgfqpoint{3.647724in}{3.384104in}}%
\pgfpathlineto{\pgfqpoint{3.634898in}{3.399362in}}%
\pgfpathlineto{\pgfqpoint{3.627437in}{3.380531in}}%
\pgfpathlineto{\pgfqpoint{3.619971in}{3.361954in}}%
\pgfpathlineto{\pgfqpoint{3.612501in}{3.343624in}}%
\pgfpathlineto{\pgfqpoint{3.605027in}{3.325538in}}%
\pgfpathclose%
\pgfusepath{fill}%
\end{pgfscope}%
\begin{pgfscope}%
\pgfpathrectangle{\pgfqpoint{1.254980in}{0.150000in}}{\pgfqpoint{5.490039in}{5.490039in}}%
\pgfusepath{clip}%
\pgfsetbuttcap%
\pgfsetroundjoin%
\definecolor{currentfill}{rgb}{0.168126,0.459988,0.558082}%
\pgfsetfillcolor{currentfill}%
\pgfsetfillopacity{0.700000}%
\pgfsetlinewidth{0.000000pt}%
\definecolor{currentstroke}{rgb}{0.000000,0.000000,0.000000}%
\pgfsetstrokecolor{currentstroke}%
\pgfsetdash{}{0pt}%
\pgfpathmoveto{\pgfqpoint{3.553685in}{3.387323in}}%
\pgfpathlineto{\pgfqpoint{3.566524in}{3.371508in}}%
\pgfpathlineto{\pgfqpoint{3.579361in}{3.355940in}}%
\pgfpathlineto{\pgfqpoint{3.592195in}{3.340617in}}%
\pgfpathlineto{\pgfqpoint{3.605027in}{3.325538in}}%
\pgfpathlineto{\pgfqpoint{3.612501in}{3.343624in}}%
\pgfpathlineto{\pgfqpoint{3.619971in}{3.361954in}}%
\pgfpathlineto{\pgfqpoint{3.627437in}{3.380531in}}%
\pgfpathlineto{\pgfqpoint{3.634898in}{3.399362in}}%
\pgfpathlineto{\pgfqpoint{3.622070in}{3.414861in}}%
\pgfpathlineto{\pgfqpoint{3.609240in}{3.430604in}}%
\pgfpathlineto{\pgfqpoint{3.596406in}{3.446593in}}%
\pgfpathlineto{\pgfqpoint{3.583570in}{3.462830in}}%
\pgfpathlineto{\pgfqpoint{3.576106in}{3.443568in}}%
\pgfpathlineto{\pgfqpoint{3.568637in}{3.424566in}}%
\pgfpathlineto{\pgfqpoint{3.561163in}{3.405819in}}%
\pgfpathlineto{\pgfqpoint{3.553685in}{3.387323in}}%
\pgfpathclose%
\pgfusepath{fill}%
\end{pgfscope}%
\begin{pgfscope}%
\pgfpathrectangle{\pgfqpoint{1.254980in}{0.150000in}}{\pgfqpoint{5.490039in}{5.490039in}}%
\pgfusepath{clip}%
\pgfsetbuttcap%
\pgfsetroundjoin%
\definecolor{currentfill}{rgb}{0.187231,0.414746,0.556547}%
\pgfsetfillcolor{currentfill}%
\pgfsetfillopacity{0.700000}%
\pgfsetlinewidth{0.000000pt}%
\definecolor{currentstroke}{rgb}{0.000000,0.000000,0.000000}%
\pgfsetstrokecolor{currentstroke}%
\pgfsetdash{}{0pt}%
\pgfpathmoveto{\pgfqpoint{3.656332in}{3.267616in}}%
\pgfpathlineto{\pgfqpoint{3.669154in}{3.253725in}}%
\pgfpathlineto{\pgfqpoint{3.681975in}{3.240066in}}%
\pgfpathlineto{\pgfqpoint{3.694795in}{3.226638in}}%
\pgfpathlineto{\pgfqpoint{3.707614in}{3.213438in}}%
\pgfpathlineto{\pgfqpoint{3.715080in}{3.230710in}}%
\pgfpathlineto{\pgfqpoint{3.722542in}{3.248211in}}%
\pgfpathlineto{\pgfqpoint{3.730000in}{3.265945in}}%
\pgfpathlineto{\pgfqpoint{3.737454in}{3.283918in}}%
\pgfpathlineto{\pgfqpoint{3.724640in}{3.297534in}}%
\pgfpathlineto{\pgfqpoint{3.711824in}{3.311379in}}%
\pgfpathlineto{\pgfqpoint{3.699007in}{3.325455in}}%
\pgfpathlineto{\pgfqpoint{3.686189in}{3.339764in}}%
\pgfpathlineto{\pgfqpoint{3.678731in}{3.321364in}}%
\pgfpathlineto{\pgfqpoint{3.671269in}{3.303209in}}%
\pgfpathlineto{\pgfqpoint{3.663802in}{3.285294in}}%
\pgfpathlineto{\pgfqpoint{3.656332in}{3.267616in}}%
\pgfpathclose%
\pgfusepath{fill}%
\end{pgfscope}%
\begin{pgfscope}%
\pgfpathrectangle{\pgfqpoint{1.254980in}{0.150000in}}{\pgfqpoint{5.490039in}{5.490039in}}%
\pgfusepath{clip}%
\pgfsetbuttcap%
\pgfsetroundjoin%
\definecolor{currentfill}{rgb}{0.165117,0.467423,0.558141}%
\pgfsetfillcolor{currentfill}%
\pgfsetfillopacity{0.700000}%
\pgfsetlinewidth{0.000000pt}%
\definecolor{currentstroke}{rgb}{0.000000,0.000000,0.000000}%
\pgfsetstrokecolor{currentstroke}%
\pgfsetdash{}{0pt}%
\pgfpathmoveto{\pgfqpoint{4.589913in}{3.400644in}}%
\pgfpathlineto{\pgfqpoint{4.602804in}{3.392399in}}%
\pgfpathlineto{\pgfqpoint{4.615700in}{3.384328in}}%
\pgfpathlineto{\pgfqpoint{4.628602in}{3.376430in}}%
\pgfpathlineto{\pgfqpoint{4.641509in}{3.368705in}}%
\pgfpathlineto{\pgfqpoint{4.648861in}{3.389326in}}%
\pgfpathlineto{\pgfqpoint{4.656216in}{3.410308in}}%
\pgfpathlineto{\pgfqpoint{4.663574in}{3.431661in}}%
\pgfpathlineto{\pgfqpoint{4.650673in}{3.439879in}}%
\pgfpathlineto{\pgfqpoint{4.637777in}{3.448270in}}%
\pgfpathlineto{\pgfqpoint{4.624886in}{3.456834in}}%
\pgfpathlineto{\pgfqpoint{4.612001in}{3.465572in}}%
\pgfpathlineto{\pgfqpoint{4.604635in}{3.443555in}}%
\pgfpathlineto{\pgfqpoint{4.597273in}{3.421915in}}%
\pgfpathlineto{\pgfqpoint{4.589913in}{3.400644in}}%
\pgfpathclose%
\pgfusepath{fill}%
\end{pgfscope}%
\begin{pgfscope}%
\pgfpathrectangle{\pgfqpoint{1.254980in}{0.150000in}}{\pgfqpoint{5.490039in}{5.490039in}}%
\pgfusepath{clip}%
\pgfsetbuttcap%
\pgfsetroundjoin%
\definecolor{currentfill}{rgb}{0.191090,0.708366,0.482284}%
\pgfsetfillcolor{currentfill}%
\pgfsetfillopacity{0.700000}%
\pgfsetlinewidth{0.000000pt}%
\definecolor{currentstroke}{rgb}{0.000000,0.000000,0.000000}%
\pgfsetstrokecolor{currentstroke}%
\pgfsetdash{}{0pt}%
\pgfpathmoveto{\pgfqpoint{3.466924in}{4.034981in}}%
\pgfpathlineto{\pgfqpoint{3.479836in}{4.012987in}}%
\pgfpathlineto{\pgfqpoint{3.492740in}{3.991282in}}%
\pgfpathlineto{\pgfqpoint{3.505638in}{3.969862in}}%
\pgfpathlineto{\pgfqpoint{3.518530in}{3.948726in}}%
\pgfpathlineto{\pgfqpoint{3.525942in}{3.973487in}}%
\pgfpathlineto{\pgfqpoint{3.533351in}{3.998612in}}%
\pgfpathlineto{\pgfqpoint{3.540756in}{4.024109in}}%
\pgfpathlineto{\pgfqpoint{3.548157in}{4.049985in}}%
\pgfpathlineto{\pgfqpoint{3.535263in}{4.071677in}}%
\pgfpathlineto{\pgfqpoint{3.522363in}{4.093654in}}%
\pgfpathlineto{\pgfqpoint{3.509457in}{4.115918in}}%
\pgfpathlineto{\pgfqpoint{3.496544in}{4.138471in}}%
\pgfpathlineto{\pgfqpoint{3.489145in}{4.112025in}}%
\pgfpathlineto{\pgfqpoint{3.481743in}{4.085966in}}%
\pgfpathlineto{\pgfqpoint{3.474336in}{4.060287in}}%
\pgfpathlineto{\pgfqpoint{3.466924in}{4.034981in}}%
\pgfpathclose%
\pgfusepath{fill}%
\end{pgfscope}%
\begin{pgfscope}%
\pgfpathrectangle{\pgfqpoint{1.254980in}{0.150000in}}{\pgfqpoint{5.490039in}{5.490039in}}%
\pgfusepath{clip}%
\pgfsetbuttcap%
\pgfsetroundjoin%
\definecolor{currentfill}{rgb}{0.208623,0.367752,0.552675}%
\pgfsetfillcolor{currentfill}%
\pgfsetfillopacity{0.700000}%
\pgfsetlinewidth{0.000000pt}%
\definecolor{currentstroke}{rgb}{0.000000,0.000000,0.000000}%
\pgfsetstrokecolor{currentstroke}%
\pgfsetdash{}{0pt}%
\pgfpathmoveto{\pgfqpoint{4.104436in}{3.152992in}}%
\pgfpathlineto{\pgfqpoint{4.117268in}{3.143719in}}%
\pgfpathlineto{\pgfqpoint{4.130103in}{3.134641in}}%
\pgfpathlineto{\pgfqpoint{4.142941in}{3.125756in}}%
\pgfpathlineto{\pgfqpoint{4.155782in}{3.117063in}}%
\pgfpathlineto{\pgfqpoint{4.163177in}{3.134007in}}%
\pgfpathlineto{\pgfqpoint{4.170569in}{3.151193in}}%
\pgfpathlineto{\pgfqpoint{4.177959in}{3.168628in}}%
\pgfpathlineto{\pgfqpoint{4.185348in}{3.186317in}}%
\pgfpathlineto{\pgfqpoint{4.172512in}{3.195504in}}%
\pgfpathlineto{\pgfqpoint{4.159679in}{3.204884in}}%
\pgfpathlineto{\pgfqpoint{4.146850in}{3.214457in}}%
\pgfpathlineto{\pgfqpoint{4.134023in}{3.224224in}}%
\pgfpathlineto{\pgfqpoint{4.126629in}{3.206030in}}%
\pgfpathlineto{\pgfqpoint{4.119233in}{3.188097in}}%
\pgfpathlineto{\pgfqpoint{4.111836in}{3.170419in}}%
\pgfpathlineto{\pgfqpoint{4.104436in}{3.152992in}}%
\pgfpathclose%
\pgfusepath{fill}%
\end{pgfscope}%
\begin{pgfscope}%
\pgfpathrectangle{\pgfqpoint{1.254980in}{0.150000in}}{\pgfqpoint{5.490039in}{5.490039in}}%
\pgfusepath{clip}%
\pgfsetbuttcap%
\pgfsetroundjoin%
\definecolor{currentfill}{rgb}{0.266941,0.748751,0.440573}%
\pgfsetfillcolor{currentfill}%
\pgfsetfillopacity{0.700000}%
\pgfsetlinewidth{0.000000pt}%
\definecolor{currentstroke}{rgb}{0.000000,0.000000,0.000000}%
\pgfsetstrokecolor{currentstroke}%
\pgfsetdash{}{0pt}%
\pgfpathmoveto{\pgfqpoint{3.496544in}{4.138471in}}%
\pgfpathlineto{\pgfqpoint{3.509457in}{4.115918in}}%
\pgfpathlineto{\pgfqpoint{3.522363in}{4.093654in}}%
\pgfpathlineto{\pgfqpoint{3.535263in}{4.071677in}}%
\pgfpathlineto{\pgfqpoint{3.548157in}{4.049985in}}%
\pgfpathlineto{\pgfqpoint{3.555554in}{4.076245in}}%
\pgfpathlineto{\pgfqpoint{3.562947in}{4.102897in}}%
\pgfpathlineto{\pgfqpoint{3.570337in}{4.129947in}}%
\pgfpathlineto{\pgfqpoint{3.577724in}{4.157404in}}%
\pgfpathlineto{\pgfqpoint{3.564828in}{4.179686in}}%
\pgfpathlineto{\pgfqpoint{3.551925in}{4.202253in}}%
\pgfpathlineto{\pgfqpoint{3.539015in}{4.225109in}}%
\pgfpathlineto{\pgfqpoint{3.526099in}{4.248256in}}%
\pgfpathlineto{\pgfqpoint{3.518716in}{4.220195in}}%
\pgfpathlineto{\pgfqpoint{3.511329in}{4.192549in}}%
\pgfpathlineto{\pgfqpoint{3.503939in}{4.165310in}}%
\pgfpathlineto{\pgfqpoint{3.496544in}{4.138471in}}%
\pgfpathclose%
\pgfusepath{fill}%
\end{pgfscope}%
\begin{pgfscope}%
\pgfpathrectangle{\pgfqpoint{1.254980in}{0.150000in}}{\pgfqpoint{5.490039in}{5.490039in}}%
\pgfusepath{clip}%
\pgfsetbuttcap%
\pgfsetroundjoin%
\definecolor{currentfill}{rgb}{0.146616,0.673050,0.508936}%
\pgfsetfillcolor{currentfill}%
\pgfsetfillopacity{0.700000}%
\pgfsetlinewidth{0.000000pt}%
\definecolor{currentstroke}{rgb}{0.000000,0.000000,0.000000}%
\pgfsetstrokecolor{currentstroke}%
\pgfsetdash{}{0pt}%
\pgfpathmoveto{\pgfqpoint{3.437235in}{3.937357in}}%
\pgfpathlineto{\pgfqpoint{3.450145in}{3.915890in}}%
\pgfpathlineto{\pgfqpoint{3.463048in}{3.894710in}}%
\pgfpathlineto{\pgfqpoint{3.475946in}{3.873815in}}%
\pgfpathlineto{\pgfqpoint{3.488837in}{3.853203in}}%
\pgfpathlineto{\pgfqpoint{3.496267in}{3.876568in}}%
\pgfpathlineto{\pgfqpoint{3.503692in}{3.900273in}}%
\pgfpathlineto{\pgfqpoint{3.511113in}{3.924323in}}%
\pgfpathlineto{\pgfqpoint{3.518530in}{3.948726in}}%
\pgfpathlineto{\pgfqpoint{3.505638in}{3.969862in}}%
\pgfpathlineto{\pgfqpoint{3.492740in}{3.991282in}}%
\pgfpathlineto{\pgfqpoint{3.479836in}{4.012987in}}%
\pgfpathlineto{\pgfqpoint{3.466924in}{4.034981in}}%
\pgfpathlineto{\pgfqpoint{3.459509in}{4.010041in}}%
\pgfpathlineto{\pgfqpoint{3.452089in}{3.985461in}}%
\pgfpathlineto{\pgfqpoint{3.444664in}{3.961235in}}%
\pgfpathlineto{\pgfqpoint{3.437235in}{3.937357in}}%
\pgfpathclose%
\pgfusepath{fill}%
\end{pgfscope}%
\begin{pgfscope}%
\pgfpathrectangle{\pgfqpoint{1.254980in}{0.150000in}}{\pgfqpoint{5.490039in}{5.490039in}}%
\pgfusepath{clip}%
\pgfsetbuttcap%
\pgfsetroundjoin%
\definecolor{currentfill}{rgb}{0.157729,0.485932,0.558013}%
\pgfsetfillcolor{currentfill}%
\pgfsetfillopacity{0.700000}%
\pgfsetlinewidth{0.000000pt}%
\definecolor{currentstroke}{rgb}{0.000000,0.000000,0.000000}%
\pgfsetstrokecolor{currentstroke}%
\pgfsetdash{}{0pt}%
\pgfpathmoveto{\pgfqpoint{3.502294in}{3.453099in}}%
\pgfpathlineto{\pgfqpoint{3.515147in}{3.436274in}}%
\pgfpathlineto{\pgfqpoint{3.527996in}{3.419704in}}%
\pgfpathlineto{\pgfqpoint{3.540842in}{3.403388in}}%
\pgfpathlineto{\pgfqpoint{3.553685in}{3.387323in}}%
\pgfpathlineto{\pgfqpoint{3.561163in}{3.405819in}}%
\pgfpathlineto{\pgfqpoint{3.568637in}{3.424566in}}%
\pgfpathlineto{\pgfqpoint{3.576106in}{3.443568in}}%
\pgfpathlineto{\pgfqpoint{3.583570in}{3.462830in}}%
\pgfpathlineto{\pgfqpoint{3.570731in}{3.479317in}}%
\pgfpathlineto{\pgfqpoint{3.557888in}{3.496055in}}%
\pgfpathlineto{\pgfqpoint{3.545042in}{3.513048in}}%
\pgfpathlineto{\pgfqpoint{3.532192in}{3.530297in}}%
\pgfpathlineto{\pgfqpoint{3.524725in}{3.510601in}}%
\pgfpathlineto{\pgfqpoint{3.517253in}{3.491173in}}%
\pgfpathlineto{\pgfqpoint{3.509776in}{3.472007in}}%
\pgfpathlineto{\pgfqpoint{3.502294in}{3.453099in}}%
\pgfpathclose%
\pgfusepath{fill}%
\end{pgfscope}%
\begin{pgfscope}%
\pgfpathrectangle{\pgfqpoint{1.254980in}{0.150000in}}{\pgfqpoint{5.490039in}{5.490039in}}%
\pgfusepath{clip}%
\pgfsetbuttcap%
\pgfsetroundjoin%
\definecolor{currentfill}{rgb}{0.197636,0.391528,0.554969}%
\pgfsetfillcolor{currentfill}%
\pgfsetfillopacity{0.700000}%
\pgfsetlinewidth{0.000000pt}%
\definecolor{currentstroke}{rgb}{0.000000,0.000000,0.000000}%
\pgfsetstrokecolor{currentstroke}%
\pgfsetdash{}{0pt}%
\pgfpathmoveto{\pgfqpoint{3.707614in}{3.213438in}}%
\pgfpathlineto{\pgfqpoint{3.720432in}{3.200466in}}%
\pgfpathlineto{\pgfqpoint{3.733249in}{3.187719in}}%
\pgfpathlineto{\pgfqpoint{3.746066in}{3.175195in}}%
\pgfpathlineto{\pgfqpoint{3.758883in}{3.162894in}}%
\pgfpathlineto{\pgfqpoint{3.766345in}{3.179761in}}%
\pgfpathlineto{\pgfqpoint{3.773802in}{3.196850in}}%
\pgfpathlineto{\pgfqpoint{3.781256in}{3.214165in}}%
\pgfpathlineto{\pgfqpoint{3.788706in}{3.231710in}}%
\pgfpathlineto{\pgfqpoint{3.775894in}{3.244427in}}%
\pgfpathlineto{\pgfqpoint{3.763081in}{3.257366in}}%
\pgfpathlineto{\pgfqpoint{3.750268in}{3.270529in}}%
\pgfpathlineto{\pgfqpoint{3.737454in}{3.283918in}}%
\pgfpathlineto{\pgfqpoint{3.730000in}{3.265945in}}%
\pgfpathlineto{\pgfqpoint{3.722542in}{3.248211in}}%
\pgfpathlineto{\pgfqpoint{3.715080in}{3.230710in}}%
\pgfpathlineto{\pgfqpoint{3.707614in}{3.213438in}}%
\pgfpathclose%
\pgfusepath{fill}%
\end{pgfscope}%
\begin{pgfscope}%
\pgfpathrectangle{\pgfqpoint{1.254980in}{0.150000in}}{\pgfqpoint{5.490039in}{5.490039in}}%
\pgfusepath{clip}%
\pgfsetbuttcap%
\pgfsetroundjoin%
\definecolor{currentfill}{rgb}{0.126453,0.570633,0.549841}%
\pgfsetfillcolor{currentfill}%
\pgfsetfillopacity{0.700000}%
\pgfsetlinewidth{0.000000pt}%
\definecolor{currentstroke}{rgb}{0.000000,0.000000,0.000000}%
\pgfsetstrokecolor{currentstroke}%
\pgfsetdash{}{0pt}%
\pgfpathmoveto{\pgfqpoint{3.429233in}{3.677794in}}%
\pgfpathlineto{\pgfqpoint{3.442120in}{3.658411in}}%
\pgfpathlineto{\pgfqpoint{3.455003in}{3.639303in}}%
\pgfpathlineto{\pgfqpoint{3.467879in}{3.620467in}}%
\pgfpathlineto{\pgfqpoint{3.480751in}{3.601902in}}%
\pgfpathlineto{\pgfqpoint{3.488216in}{3.622316in}}%
\pgfpathlineto{\pgfqpoint{3.495677in}{3.643015in}}%
\pgfpathlineto{\pgfqpoint{3.503133in}{3.664005in}}%
\pgfpathlineto{\pgfqpoint{3.510584in}{3.685292in}}%
\pgfpathlineto{\pgfqpoint{3.497714in}{3.704314in}}%
\pgfpathlineto{\pgfqpoint{3.484839in}{3.723607in}}%
\pgfpathlineto{\pgfqpoint{3.471959in}{3.743174in}}%
\pgfpathlineto{\pgfqpoint{3.459073in}{3.763016in}}%
\pgfpathlineto{\pgfqpoint{3.451620in}{3.741259in}}%
\pgfpathlineto{\pgfqpoint{3.444163in}{3.719807in}}%
\pgfpathlineto{\pgfqpoint{3.436700in}{3.698654in}}%
\pgfpathlineto{\pgfqpoint{3.429233in}{3.677794in}}%
\pgfpathclose%
\pgfusepath{fill}%
\end{pgfscope}%
\begin{pgfscope}%
\pgfpathrectangle{\pgfqpoint{1.254980in}{0.150000in}}{\pgfqpoint{5.490039in}{5.490039in}}%
\pgfusepath{clip}%
\pgfsetbuttcap%
\pgfsetroundjoin%
\definecolor{currentfill}{rgb}{0.515992,0.831158,0.294279}%
\pgfsetfillcolor{currentfill}%
\pgfsetfillopacity{0.700000}%
\pgfsetlinewidth{0.000000pt}%
\definecolor{currentstroke}{rgb}{0.000000,0.000000,0.000000}%
\pgfsetstrokecolor{currentstroke}%
\pgfsetdash{}{0pt}%
\pgfpathmoveto{\pgfqpoint{3.717724in}{4.427087in}}%
\pgfpathlineto{\pgfqpoint{3.730606in}{4.404237in}}%
\pgfpathlineto{\pgfqpoint{3.743481in}{4.381661in}}%
\pgfpathlineto{\pgfqpoint{3.756352in}{4.359356in}}%
\pgfpathlineto{\pgfqpoint{3.769217in}{4.337320in}}%
\pgfpathlineto{\pgfqpoint{3.776584in}{4.369331in}}%
\pgfpathlineto{\pgfqpoint{3.783950in}{4.401840in}}%
\pgfpathlineto{\pgfqpoint{3.791315in}{4.434857in}}%
\pgfpathlineto{\pgfqpoint{3.778445in}{4.457428in}}%
\pgfpathlineto{\pgfqpoint{3.765569in}{4.480270in}}%
\pgfpathlineto{\pgfqpoint{3.752688in}{4.503384in}}%
\pgfpathlineto{\pgfqpoint{3.739802in}{4.526773in}}%
\pgfpathlineto{\pgfqpoint{3.732444in}{4.493030in}}%
\pgfpathlineto{\pgfqpoint{3.725085in}{4.459805in}}%
\pgfpathlineto{\pgfqpoint{3.717724in}{4.427087in}}%
\pgfpathclose%
\pgfusepath{fill}%
\end{pgfscope}%
\begin{pgfscope}%
\pgfpathrectangle{\pgfqpoint{1.254980in}{0.150000in}}{\pgfqpoint{5.490039in}{5.490039in}}%
\pgfusepath{clip}%
\pgfsetbuttcap%
\pgfsetroundjoin%
\definecolor{currentfill}{rgb}{0.210503,0.363727,0.552206}%
\pgfsetfillcolor{currentfill}%
\pgfsetfillopacity{0.700000}%
\pgfsetlinewidth{0.000000pt}%
\definecolor{currentstroke}{rgb}{0.000000,0.000000,0.000000}%
\pgfsetstrokecolor{currentstroke}%
\pgfsetdash{}{0pt}%
\pgfpathmoveto{\pgfqpoint{3.891215in}{3.137807in}}%
\pgfpathlineto{\pgfqpoint{3.904032in}{3.127026in}}%
\pgfpathlineto{\pgfqpoint{3.916850in}{3.116452in}}%
\pgfpathlineto{\pgfqpoint{3.929670in}{3.106086in}}%
\pgfpathlineto{\pgfqpoint{3.942492in}{3.095924in}}%
\pgfpathlineto{\pgfqpoint{3.949925in}{3.112413in}}%
\pgfpathlineto{\pgfqpoint{3.957354in}{3.129122in}}%
\pgfpathlineto{\pgfqpoint{3.964780in}{3.146056in}}%
\pgfpathlineto{\pgfqpoint{3.972203in}{3.163220in}}%
\pgfpathlineto{\pgfqpoint{3.959386in}{3.173822in}}%
\pgfpathlineto{\pgfqpoint{3.946571in}{3.184629in}}%
\pgfpathlineto{\pgfqpoint{3.933758in}{3.195643in}}%
\pgfpathlineto{\pgfqpoint{3.920945in}{3.206866in}}%
\pgfpathlineto{\pgfqpoint{3.913517in}{3.189250in}}%
\pgfpathlineto{\pgfqpoint{3.906086in}{3.171872in}}%
\pgfpathlineto{\pgfqpoint{3.898652in}{3.154726in}}%
\pgfpathlineto{\pgfqpoint{3.891215in}{3.137807in}}%
\pgfpathclose%
\pgfusepath{fill}%
\end{pgfscope}%
\begin{pgfscope}%
\pgfpathrectangle{\pgfqpoint{1.254980in}{0.150000in}}{\pgfqpoint{5.490039in}{5.490039in}}%
\pgfusepath{clip}%
\pgfsetbuttcap%
\pgfsetroundjoin%
\definecolor{currentfill}{rgb}{0.192357,0.403199,0.555836}%
\pgfsetfillcolor{currentfill}%
\pgfsetfillopacity{0.700000}%
\pgfsetlinewidth{0.000000pt}%
\definecolor{currentstroke}{rgb}{0.000000,0.000000,0.000000}%
\pgfsetstrokecolor{currentstroke}%
\pgfsetdash{}{0pt}%
\pgfpathmoveto{\pgfqpoint{4.398580in}{3.229235in}}%
\pgfpathlineto{\pgfqpoint{4.411455in}{3.221389in}}%
\pgfpathlineto{\pgfqpoint{4.424334in}{3.213722in}}%
\pgfpathlineto{\pgfqpoint{4.437218in}{3.206234in}}%
\pgfpathlineto{\pgfqpoint{4.450106in}{3.198923in}}%
\pgfpathlineto{\pgfqpoint{4.457460in}{3.216910in}}%
\pgfpathlineto{\pgfqpoint{4.464814in}{3.235186in}}%
\pgfpathlineto{\pgfqpoint{4.472168in}{3.253757in}}%
\pgfpathlineto{\pgfqpoint{4.479523in}{3.272631in}}%
\pgfpathlineto{\pgfqpoint{4.466641in}{3.280518in}}%
\pgfpathlineto{\pgfqpoint{4.453765in}{3.288583in}}%
\pgfpathlineto{\pgfqpoint{4.440893in}{3.296826in}}%
\pgfpathlineto{\pgfqpoint{4.428025in}{3.305249in}}%
\pgfpathlineto{\pgfqpoint{4.420664in}{3.285788in}}%
\pgfpathlineto{\pgfqpoint{4.413303in}{3.266636in}}%
\pgfpathlineto{\pgfqpoint{4.405942in}{3.247788in}}%
\pgfpathlineto{\pgfqpoint{4.398580in}{3.229235in}}%
\pgfpathclose%
\pgfusepath{fill}%
\end{pgfscope}%
\begin{pgfscope}%
\pgfpathrectangle{\pgfqpoint{1.254980in}{0.150000in}}{\pgfqpoint{5.490039in}{5.490039in}}%
\pgfusepath{clip}%
\pgfsetbuttcap%
\pgfsetroundjoin%
\definecolor{currentfill}{rgb}{0.214298,0.355619,0.551184}%
\pgfsetfillcolor{currentfill}%
\pgfsetfillopacity{0.700000}%
\pgfsetlinewidth{0.000000pt}%
\definecolor{currentstroke}{rgb}{0.000000,0.000000,0.000000}%
\pgfsetstrokecolor{currentstroke}%
\pgfsetdash{}{0pt}%
\pgfpathmoveto{\pgfqpoint{4.023489in}{3.122845in}}%
\pgfpathlineto{\pgfqpoint{4.036316in}{3.113253in}}%
\pgfpathlineto{\pgfqpoint{4.049145in}{3.103859in}}%
\pgfpathlineto{\pgfqpoint{4.061977in}{3.094662in}}%
\pgfpathlineto{\pgfqpoint{4.074812in}{3.085662in}}%
\pgfpathlineto{\pgfqpoint{4.082222in}{3.102148in}}%
\pgfpathlineto{\pgfqpoint{4.089629in}{3.118861in}}%
\pgfpathlineto{\pgfqpoint{4.097033in}{3.135807in}}%
\pgfpathlineto{\pgfqpoint{4.104436in}{3.152992in}}%
\pgfpathlineto{\pgfqpoint{4.091606in}{3.162459in}}%
\pgfpathlineto{\pgfqpoint{4.078780in}{3.172123in}}%
\pgfpathlineto{\pgfqpoint{4.065956in}{3.181984in}}%
\pgfpathlineto{\pgfqpoint{4.053134in}{3.192043in}}%
\pgfpathlineto{\pgfqpoint{4.045726in}{3.174381in}}%
\pgfpathlineto{\pgfqpoint{4.038316in}{3.156965in}}%
\pgfpathlineto{\pgfqpoint{4.030904in}{3.139788in}}%
\pgfpathlineto{\pgfqpoint{4.023489in}{3.122845in}}%
\pgfpathclose%
\pgfusepath{fill}%
\end{pgfscope}%
\begin{pgfscope}%
\pgfpathrectangle{\pgfqpoint{1.254980in}{0.150000in}}{\pgfqpoint{5.490039in}{5.490039in}}%
\pgfusepath{clip}%
\pgfsetbuttcap%
\pgfsetroundjoin%
\definecolor{currentfill}{rgb}{0.123444,0.636809,0.528763}%
\pgfsetfillcolor{currentfill}%
\pgfsetfillopacity{0.700000}%
\pgfsetlinewidth{0.000000pt}%
\definecolor{currentstroke}{rgb}{0.000000,0.000000,0.000000}%
\pgfsetstrokecolor{currentstroke}%
\pgfsetdash{}{0pt}%
\pgfpathmoveto{\pgfqpoint{3.407469in}{3.845199in}}%
\pgfpathlineto{\pgfqpoint{3.420379in}{3.824226in}}%
\pgfpathlineto{\pgfqpoint{3.433283in}{3.803540in}}%
\pgfpathlineto{\pgfqpoint{3.446181in}{3.783138in}}%
\pgfpathlineto{\pgfqpoint{3.459073in}{3.763016in}}%
\pgfpathlineto{\pgfqpoint{3.466521in}{3.785083in}}%
\pgfpathlineto{\pgfqpoint{3.473964in}{3.807466in}}%
\pgfpathlineto{\pgfqpoint{3.481403in}{3.830171in}}%
\pgfpathlineto{\pgfqpoint{3.488837in}{3.853203in}}%
\pgfpathlineto{\pgfqpoint{3.475946in}{3.873815in}}%
\pgfpathlineto{\pgfqpoint{3.463048in}{3.894710in}}%
\pgfpathlineto{\pgfqpoint{3.450145in}{3.915890in}}%
\pgfpathlineto{\pgfqpoint{3.437235in}{3.937357in}}%
\pgfpathlineto{\pgfqpoint{3.429800in}{3.913820in}}%
\pgfpathlineto{\pgfqpoint{3.422361in}{3.890619in}}%
\pgfpathlineto{\pgfqpoint{3.414918in}{3.867747in}}%
\pgfpathlineto{\pgfqpoint{3.407469in}{3.845199in}}%
\pgfpathclose%
\pgfusepath{fill}%
\end{pgfscope}%
\begin{pgfscope}%
\pgfpathrectangle{\pgfqpoint{1.254980in}{0.150000in}}{\pgfqpoint{5.490039in}{5.490039in}}%
\pgfusepath{clip}%
\pgfsetbuttcap%
\pgfsetroundjoin%
\definecolor{currentfill}{rgb}{0.360741,0.785964,0.387814}%
\pgfsetfillcolor{currentfill}%
\pgfsetfillopacity{0.700000}%
\pgfsetlinewidth{0.000000pt}%
\definecolor{currentstroke}{rgb}{0.000000,0.000000,0.000000}%
\pgfsetstrokecolor{currentstroke}%
\pgfsetdash{}{0pt}%
\pgfpathmoveto{\pgfqpoint{3.526099in}{4.248256in}}%
\pgfpathlineto{\pgfqpoint{3.539015in}{4.225109in}}%
\pgfpathlineto{\pgfqpoint{3.551925in}{4.202253in}}%
\pgfpathlineto{\pgfqpoint{3.564828in}{4.179686in}}%
\pgfpathlineto{\pgfqpoint{3.577724in}{4.157404in}}%
\pgfpathlineto{\pgfqpoint{3.585107in}{4.185273in}}%
\pgfpathlineto{\pgfqpoint{3.592487in}{4.213563in}}%
\pgfpathlineto{\pgfqpoint{3.599864in}{4.242281in}}%
\pgfpathlineto{\pgfqpoint{3.607238in}{4.271434in}}%
\pgfpathlineto{\pgfqpoint{3.594338in}{4.294340in}}%
\pgfpathlineto{\pgfqpoint{3.581431in}{4.317533in}}%
\pgfpathlineto{\pgfqpoint{3.568517in}{4.341015in}}%
\pgfpathlineto{\pgfqpoint{3.555596in}{4.364790in}}%
\pgfpathlineto{\pgfqpoint{3.548227in}{4.334997in}}%
\pgfpathlineto{\pgfqpoint{3.540855in}{4.305649in}}%
\pgfpathlineto{\pgfqpoint{3.533479in}{4.276738in}}%
\pgfpathlineto{\pgfqpoint{3.526099in}{4.248256in}}%
\pgfpathclose%
\pgfusepath{fill}%
\end{pgfscope}%
\begin{pgfscope}%
\pgfpathrectangle{\pgfqpoint{1.254980in}{0.150000in}}{\pgfqpoint{5.490039in}{5.490039in}}%
\pgfusepath{clip}%
\pgfsetbuttcap%
\pgfsetroundjoin%
\definecolor{currentfill}{rgb}{0.199430,0.387607,0.554642}%
\pgfsetfillcolor{currentfill}%
\pgfsetfillopacity{0.700000}%
\pgfsetlinewidth{0.000000pt}%
\definecolor{currentstroke}{rgb}{0.000000,0.000000,0.000000}%
\pgfsetstrokecolor{currentstroke}%
\pgfsetdash{}{0pt}%
\pgfpathmoveto{\pgfqpoint{4.317652in}{3.188857in}}%
\pgfpathlineto{\pgfqpoint{4.330516in}{3.180833in}}%
\pgfpathlineto{\pgfqpoint{4.343383in}{3.172992in}}%
\pgfpathlineto{\pgfqpoint{4.356255in}{3.165333in}}%
\pgfpathlineto{\pgfqpoint{4.369132in}{3.157855in}}%
\pgfpathlineto{\pgfqpoint{4.376495in}{3.175289in}}%
\pgfpathlineto{\pgfqpoint{4.383857in}{3.192992in}}%
\pgfpathlineto{\pgfqpoint{4.391219in}{3.210973in}}%
\pgfpathlineto{\pgfqpoint{4.398580in}{3.229235in}}%
\pgfpathlineto{\pgfqpoint{4.385711in}{3.237262in}}%
\pgfpathlineto{\pgfqpoint{4.372845in}{3.245470in}}%
\pgfpathlineto{\pgfqpoint{4.359984in}{3.253860in}}%
\pgfpathlineto{\pgfqpoint{4.347128in}{3.262433in}}%
\pgfpathlineto{\pgfqpoint{4.339760in}{3.243610in}}%
\pgfpathlineto{\pgfqpoint{4.332391in}{3.225078in}}%
\pgfpathlineto{\pgfqpoint{4.325022in}{3.206829in}}%
\pgfpathlineto{\pgfqpoint{4.317652in}{3.188857in}}%
\pgfpathclose%
\pgfusepath{fill}%
\end{pgfscope}%
\begin{pgfscope}%
\pgfpathrectangle{\pgfqpoint{1.254980in}{0.150000in}}{\pgfqpoint{5.490039in}{5.490039in}}%
\pgfusepath{clip}%
\pgfsetbuttcap%
\pgfsetroundjoin%
\definecolor{currentfill}{rgb}{0.183898,0.422383,0.556944}%
\pgfsetfillcolor{currentfill}%
\pgfsetfillopacity{0.700000}%
\pgfsetlinewidth{0.000000pt}%
\definecolor{currentstroke}{rgb}{0.000000,0.000000,0.000000}%
\pgfsetstrokecolor{currentstroke}%
\pgfsetdash{}{0pt}%
\pgfpathmoveto{\pgfqpoint{4.479523in}{3.272631in}}%
\pgfpathlineto{\pgfqpoint{4.492409in}{3.264922in}}%
\pgfpathlineto{\pgfqpoint{4.505300in}{3.257389in}}%
\pgfpathlineto{\pgfqpoint{4.518197in}{3.250032in}}%
\pgfpathlineto{\pgfqpoint{4.531099in}{3.242849in}}%
\pgfpathlineto{\pgfqpoint{4.538446in}{3.261439in}}%
\pgfpathlineto{\pgfqpoint{4.545794in}{3.280339in}}%
\pgfpathlineto{\pgfqpoint{4.553143in}{3.299556in}}%
\pgfpathlineto{\pgfqpoint{4.560494in}{3.319096in}}%
\pgfpathlineto{\pgfqpoint{4.547600in}{3.326882in}}%
\pgfpathlineto{\pgfqpoint{4.534711in}{3.334844in}}%
\pgfpathlineto{\pgfqpoint{4.521827in}{3.342981in}}%
\pgfpathlineto{\pgfqpoint{4.508948in}{3.351295in}}%
\pgfpathlineto{\pgfqpoint{4.501590in}{3.331140in}}%
\pgfpathlineto{\pgfqpoint{4.494234in}{3.311316in}}%
\pgfpathlineto{\pgfqpoint{4.486878in}{3.291815in}}%
\pgfpathlineto{\pgfqpoint{4.479523in}{3.272631in}}%
\pgfpathclose%
\pgfusepath{fill}%
\end{pgfscope}%
\begin{pgfscope}%
\pgfpathrectangle{\pgfqpoint{1.254980in}{0.150000in}}{\pgfqpoint{5.490039in}{5.490039in}}%
\pgfusepath{clip}%
\pgfsetbuttcap%
\pgfsetroundjoin%
\definecolor{currentfill}{rgb}{0.147607,0.511733,0.557049}%
\pgfsetfillcolor{currentfill}%
\pgfsetfillopacity{0.700000}%
\pgfsetlinewidth{0.000000pt}%
\definecolor{currentstroke}{rgb}{0.000000,0.000000,0.000000}%
\pgfsetstrokecolor{currentstroke}%
\pgfsetdash{}{0pt}%
\pgfpathmoveto{\pgfqpoint{3.450841in}{3.523004in}}%
\pgfpathlineto{\pgfqpoint{3.463711in}{3.505133in}}%
\pgfpathlineto{\pgfqpoint{3.476576in}{3.487527in}}%
\pgfpathlineto{\pgfqpoint{3.489437in}{3.470183in}}%
\pgfpathlineto{\pgfqpoint{3.502294in}{3.453099in}}%
\pgfpathlineto{\pgfqpoint{3.509776in}{3.472007in}}%
\pgfpathlineto{\pgfqpoint{3.517253in}{3.491173in}}%
\pgfpathlineto{\pgfqpoint{3.524725in}{3.510601in}}%
\pgfpathlineto{\pgfqpoint{3.532192in}{3.530297in}}%
\pgfpathlineto{\pgfqpoint{3.519338in}{3.547805in}}%
\pgfpathlineto{\pgfqpoint{3.506480in}{3.565574in}}%
\pgfpathlineto{\pgfqpoint{3.493618in}{3.583605in}}%
\pgfpathlineto{\pgfqpoint{3.480751in}{3.601902in}}%
\pgfpathlineto{\pgfqpoint{3.473281in}{3.581770in}}%
\pgfpathlineto{\pgfqpoint{3.465806in}{3.561913in}}%
\pgfpathlineto{\pgfqpoint{3.458326in}{3.542326in}}%
\pgfpathlineto{\pgfqpoint{3.450841in}{3.523004in}}%
\pgfpathclose%
\pgfusepath{fill}%
\end{pgfscope}%
\begin{pgfscope}%
\pgfpathrectangle{\pgfqpoint{1.254980in}{0.150000in}}{\pgfqpoint{5.490039in}{5.490039in}}%
\pgfusepath{clip}%
\pgfsetbuttcap%
\pgfsetroundjoin%
\definecolor{currentfill}{rgb}{0.206756,0.371758,0.553117}%
\pgfsetfillcolor{currentfill}%
\pgfsetfillopacity{0.700000}%
\pgfsetlinewidth{0.000000pt}%
\definecolor{currentstroke}{rgb}{0.000000,0.000000,0.000000}%
\pgfsetstrokecolor{currentstroke}%
\pgfsetdash{}{0pt}%
\pgfpathmoveto{\pgfqpoint{3.758883in}{3.162894in}}%
\pgfpathlineto{\pgfqpoint{3.771700in}{3.150814in}}%
\pgfpathlineto{\pgfqpoint{3.784517in}{3.138952in}}%
\pgfpathlineto{\pgfqpoint{3.797334in}{3.127308in}}%
\pgfpathlineto{\pgfqpoint{3.810152in}{3.115881in}}%
\pgfpathlineto{\pgfqpoint{3.817608in}{3.132344in}}%
\pgfpathlineto{\pgfqpoint{3.825061in}{3.149022in}}%
\pgfpathlineto{\pgfqpoint{3.832510in}{3.165918in}}%
\pgfpathlineto{\pgfqpoint{3.839956in}{3.183039in}}%
\pgfpathlineto{\pgfqpoint{3.827143in}{3.194881in}}%
\pgfpathlineto{\pgfqpoint{3.814331in}{3.206939in}}%
\pgfpathlineto{\pgfqpoint{3.801518in}{3.219215in}}%
\pgfpathlineto{\pgfqpoint{3.788706in}{3.231710in}}%
\pgfpathlineto{\pgfqpoint{3.781256in}{3.214165in}}%
\pgfpathlineto{\pgfqpoint{3.773802in}{3.196850in}}%
\pgfpathlineto{\pgfqpoint{3.766345in}{3.179761in}}%
\pgfpathlineto{\pgfqpoint{3.758883in}{3.162894in}}%
\pgfpathclose%
\pgfusepath{fill}%
\end{pgfscope}%
\begin{pgfscope}%
\pgfpathrectangle{\pgfqpoint{1.254980in}{0.150000in}}{\pgfqpoint{5.490039in}{5.490039in}}%
\pgfusepath{clip}%
\pgfsetbuttcap%
\pgfsetroundjoin%
\definecolor{currentfill}{rgb}{0.506271,0.828786,0.300362}%
\pgfsetfillcolor{currentfill}%
\pgfsetfillopacity{0.700000}%
\pgfsetlinewidth{0.000000pt}%
\definecolor{currentstroke}{rgb}{0.000000,0.000000,0.000000}%
\pgfsetstrokecolor{currentstroke}%
\pgfsetdash{}{0pt}%
\pgfpathmoveto{\pgfqpoint{3.636707in}{4.392557in}}%
\pgfpathlineto{\pgfqpoint{3.649606in}{4.369278in}}%
\pgfpathlineto{\pgfqpoint{3.662499in}{4.346281in}}%
\pgfpathlineto{\pgfqpoint{3.675386in}{4.323566in}}%
\pgfpathlineto{\pgfqpoint{3.688268in}{4.301127in}}%
\pgfpathlineto{\pgfqpoint{3.695634in}{4.331898in}}%
\pgfpathlineto{\pgfqpoint{3.702999in}{4.363142in}}%
\pgfpathlineto{\pgfqpoint{3.710363in}{4.394869in}}%
\pgfpathlineto{\pgfqpoint{3.717724in}{4.427087in}}%
\pgfpathlineto{\pgfqpoint{3.704837in}{4.450214in}}%
\pgfpathlineto{\pgfqpoint{3.691944in}{4.473620in}}%
\pgfpathlineto{\pgfqpoint{3.679045in}{4.497308in}}%
\pgfpathlineto{\pgfqpoint{3.666139in}{4.521280in}}%
\pgfpathlineto{\pgfqpoint{3.658784in}{4.488358in}}%
\pgfpathlineto{\pgfqpoint{3.651427in}{4.455935in}}%
\pgfpathlineto{\pgfqpoint{3.644068in}{4.424004in}}%
\pgfpathlineto{\pgfqpoint{3.636707in}{4.392557in}}%
\pgfpathclose%
\pgfusepath{fill}%
\end{pgfscope}%
\begin{pgfscope}%
\pgfpathrectangle{\pgfqpoint{1.254980in}{0.150000in}}{\pgfqpoint{5.490039in}{5.490039in}}%
\pgfusepath{clip}%
\pgfsetbuttcap%
\pgfsetroundjoin%
\definecolor{currentfill}{rgb}{0.206756,0.371758,0.553117}%
\pgfsetfillcolor{currentfill}%
\pgfsetfillopacity{0.700000}%
\pgfsetlinewidth{0.000000pt}%
\definecolor{currentstroke}{rgb}{0.000000,0.000000,0.000000}%
\pgfsetstrokecolor{currentstroke}%
\pgfsetdash{}{0pt}%
\pgfpathmoveto{\pgfqpoint{4.236724in}{3.151467in}}%
\pgfpathlineto{\pgfqpoint{4.249578in}{3.143225in}}%
\pgfpathlineto{\pgfqpoint{4.262435in}{3.135169in}}%
\pgfpathlineto{\pgfqpoint{4.275297in}{3.127299in}}%
\pgfpathlineto{\pgfqpoint{4.288162in}{3.119612in}}%
\pgfpathlineto{\pgfqpoint{4.295537in}{3.136539in}}%
\pgfpathlineto{\pgfqpoint{4.302910in}{3.153718in}}%
\pgfpathlineto{\pgfqpoint{4.310282in}{3.171155in}}%
\pgfpathlineto{\pgfqpoint{4.317652in}{3.188857in}}%
\pgfpathlineto{\pgfqpoint{4.304793in}{3.197064in}}%
\pgfpathlineto{\pgfqpoint{4.291938in}{3.205456in}}%
\pgfpathlineto{\pgfqpoint{4.279087in}{3.214033in}}%
\pgfpathlineto{\pgfqpoint{4.266240in}{3.222797in}}%
\pgfpathlineto{\pgfqpoint{4.258863in}{3.204563in}}%
\pgfpathlineto{\pgfqpoint{4.251485in}{3.186602in}}%
\pgfpathlineto{\pgfqpoint{4.244105in}{3.168905in}}%
\pgfpathlineto{\pgfqpoint{4.236724in}{3.151467in}}%
\pgfpathclose%
\pgfusepath{fill}%
\end{pgfscope}%
\begin{pgfscope}%
\pgfpathrectangle{\pgfqpoint{1.254980in}{0.150000in}}{\pgfqpoint{5.490039in}{5.490039in}}%
\pgfusepath{clip}%
\pgfsetbuttcap%
\pgfsetroundjoin%
\definecolor{currentfill}{rgb}{0.175841,0.441290,0.557685}%
\pgfsetfillcolor{currentfill}%
\pgfsetfillopacity{0.700000}%
\pgfsetlinewidth{0.000000pt}%
\definecolor{currentstroke}{rgb}{0.000000,0.000000,0.000000}%
\pgfsetstrokecolor{currentstroke}%
\pgfsetdash{}{0pt}%
\pgfpathmoveto{\pgfqpoint{4.560494in}{3.319096in}}%
\pgfpathlineto{\pgfqpoint{4.573393in}{3.311484in}}%
\pgfpathlineto{\pgfqpoint{4.586297in}{3.304046in}}%
\pgfpathlineto{\pgfqpoint{4.599207in}{3.296780in}}%
\pgfpathlineto{\pgfqpoint{4.612123in}{3.289687in}}%
\pgfpathlineto{\pgfqpoint{4.619466in}{3.308937in}}%
\pgfpathlineto{\pgfqpoint{4.626812in}{3.328518in}}%
\pgfpathlineto{\pgfqpoint{4.634159in}{3.348438in}}%
\pgfpathlineto{\pgfqpoint{4.641509in}{3.368705in}}%
\pgfpathlineto{\pgfqpoint{4.628602in}{3.376430in}}%
\pgfpathlineto{\pgfqpoint{4.615700in}{3.384328in}}%
\pgfpathlineto{\pgfqpoint{4.602804in}{3.392399in}}%
\pgfpathlineto{\pgfqpoint{4.589913in}{3.400644in}}%
\pgfpathlineto{\pgfqpoint{4.582555in}{3.379734in}}%
\pgfpathlineto{\pgfqpoint{4.575199in}{3.359178in}}%
\pgfpathlineto{\pgfqpoint{4.567846in}{3.338968in}}%
\pgfpathlineto{\pgfqpoint{4.560494in}{3.319096in}}%
\pgfpathclose%
\pgfusepath{fill}%
\end{pgfscope}%
\begin{pgfscope}%
\pgfpathrectangle{\pgfqpoint{1.254980in}{0.150000in}}{\pgfqpoint{5.490039in}{5.490039in}}%
\pgfusepath{clip}%
\pgfsetbuttcap%
\pgfsetroundjoin%
\definecolor{currentfill}{rgb}{0.214298,0.355619,0.551184}%
\pgfsetfillcolor{currentfill}%
\pgfsetfillopacity{0.700000}%
\pgfsetlinewidth{0.000000pt}%
\definecolor{currentstroke}{rgb}{0.000000,0.000000,0.000000}%
\pgfsetstrokecolor{currentstroke}%
\pgfsetdash{}{0pt}%
\pgfpathmoveto{\pgfqpoint{4.155782in}{3.117063in}}%
\pgfpathlineto{\pgfqpoint{4.168627in}{3.108560in}}%
\pgfpathlineto{\pgfqpoint{4.181476in}{3.100248in}}%
\pgfpathlineto{\pgfqpoint{4.194328in}{3.092124in}}%
\pgfpathlineto{\pgfqpoint{4.207184in}{3.084188in}}%
\pgfpathlineto{\pgfqpoint{4.214572in}{3.100649in}}%
\pgfpathlineto{\pgfqpoint{4.221958in}{3.117345in}}%
\pgfpathlineto{\pgfqpoint{4.229342in}{3.134283in}}%
\pgfpathlineto{\pgfqpoint{4.236724in}{3.151467in}}%
\pgfpathlineto{\pgfqpoint{4.223875in}{3.159897in}}%
\pgfpathlineto{\pgfqpoint{4.211029in}{3.168514in}}%
\pgfpathlineto{\pgfqpoint{4.198187in}{3.177320in}}%
\pgfpathlineto{\pgfqpoint{4.185348in}{3.186317in}}%
\pgfpathlineto{\pgfqpoint{4.177959in}{3.168628in}}%
\pgfpathlineto{\pgfqpoint{4.170569in}{3.151193in}}%
\pgfpathlineto{\pgfqpoint{4.163177in}{3.134007in}}%
\pgfpathlineto{\pgfqpoint{4.155782in}{3.117063in}}%
\pgfpathclose%
\pgfusepath{fill}%
\end{pgfscope}%
\begin{pgfscope}%
\pgfpathrectangle{\pgfqpoint{1.254980in}{0.150000in}}{\pgfqpoint{5.490039in}{5.490039in}}%
\pgfusepath{clip}%
\pgfsetbuttcap%
\pgfsetroundjoin%
\definecolor{currentfill}{rgb}{0.119738,0.603785,0.541400}%
\pgfsetfillcolor{currentfill}%
\pgfsetfillopacity{0.700000}%
\pgfsetlinewidth{0.000000pt}%
\definecolor{currentstroke}{rgb}{0.000000,0.000000,0.000000}%
\pgfsetstrokecolor{currentstroke}%
\pgfsetdash{}{0pt}%
\pgfpathmoveto{\pgfqpoint{3.377623in}{3.758130in}}%
\pgfpathlineto{\pgfqpoint{3.390535in}{3.737620in}}%
\pgfpathlineto{\pgfqpoint{3.403440in}{3.717396in}}%
\pgfpathlineto{\pgfqpoint{3.416340in}{3.697455in}}%
\pgfpathlineto{\pgfqpoint{3.429233in}{3.677794in}}%
\pgfpathlineto{\pgfqpoint{3.436700in}{3.698654in}}%
\pgfpathlineto{\pgfqpoint{3.444163in}{3.719807in}}%
\pgfpathlineto{\pgfqpoint{3.451620in}{3.741259in}}%
\pgfpathlineto{\pgfqpoint{3.459073in}{3.763016in}}%
\pgfpathlineto{\pgfqpoint{3.446181in}{3.783138in}}%
\pgfpathlineto{\pgfqpoint{3.433283in}{3.803540in}}%
\pgfpathlineto{\pgfqpoint{3.420379in}{3.824226in}}%
\pgfpathlineto{\pgfqpoint{3.407469in}{3.845199in}}%
\pgfpathlineto{\pgfqpoint{3.400015in}{3.822969in}}%
\pgfpathlineto{\pgfqpoint{3.392556in}{3.801051in}}%
\pgfpathlineto{\pgfqpoint{3.385092in}{3.779440in}}%
\pgfpathlineto{\pgfqpoint{3.377623in}{3.758130in}}%
\pgfpathclose%
\pgfusepath{fill}%
\end{pgfscope}%
\begin{pgfscope}%
\pgfpathrectangle{\pgfqpoint{1.254980in}{0.150000in}}{\pgfqpoint{5.490039in}{5.490039in}}%
\pgfusepath{clip}%
\pgfsetbuttcap%
\pgfsetroundjoin%
\definecolor{currentfill}{rgb}{0.135066,0.544853,0.554029}%
\pgfsetfillcolor{currentfill}%
\pgfsetfillopacity{0.700000}%
\pgfsetlinewidth{0.000000pt}%
\definecolor{currentstroke}{rgb}{0.000000,0.000000,0.000000}%
\pgfsetstrokecolor{currentstroke}%
\pgfsetdash{}{0pt}%
\pgfpathmoveto{\pgfqpoint{3.399312in}{3.597185in}}%
\pgfpathlineto{\pgfqpoint{3.412202in}{3.578230in}}%
\pgfpathlineto{\pgfqpoint{3.425087in}{3.559550in}}%
\pgfpathlineto{\pgfqpoint{3.437967in}{3.541142in}}%
\pgfpathlineto{\pgfqpoint{3.450841in}{3.523004in}}%
\pgfpathlineto{\pgfqpoint{3.458326in}{3.542326in}}%
\pgfpathlineto{\pgfqpoint{3.465806in}{3.561913in}}%
\pgfpathlineto{\pgfqpoint{3.473281in}{3.581770in}}%
\pgfpathlineto{\pgfqpoint{3.480751in}{3.601902in}}%
\pgfpathlineto{\pgfqpoint{3.467879in}{3.620467in}}%
\pgfpathlineto{\pgfqpoint{3.455003in}{3.639303in}}%
\pgfpathlineto{\pgfqpoint{3.442120in}{3.658411in}}%
\pgfpathlineto{\pgfqpoint{3.429233in}{3.677794in}}%
\pgfpathlineto{\pgfqpoint{3.421760in}{3.657222in}}%
\pgfpathlineto{\pgfqpoint{3.414283in}{3.636933in}}%
\pgfpathlineto{\pgfqpoint{3.406800in}{3.616923in}}%
\pgfpathlineto{\pgfqpoint{3.399312in}{3.597185in}}%
\pgfpathclose%
\pgfusepath{fill}%
\end{pgfscope}%
\begin{pgfscope}%
\pgfpathrectangle{\pgfqpoint{1.254980in}{0.150000in}}{\pgfqpoint{5.490039in}{5.490039in}}%
\pgfusepath{clip}%
\pgfsetbuttcap%
\pgfsetroundjoin%
\definecolor{currentfill}{rgb}{0.218130,0.347432,0.550038}%
\pgfsetfillcolor{currentfill}%
\pgfsetfillopacity{0.700000}%
\pgfsetlinewidth{0.000000pt}%
\definecolor{currentstroke}{rgb}{0.000000,0.000000,0.000000}%
\pgfsetstrokecolor{currentstroke}%
\pgfsetdash{}{0pt}%
\pgfpathmoveto{\pgfqpoint{3.942492in}{3.095924in}}%
\pgfpathlineto{\pgfqpoint{3.955316in}{3.085967in}}%
\pgfpathlineto{\pgfqpoint{3.968142in}{3.076213in}}%
\pgfpathlineto{\pgfqpoint{3.980969in}{3.066660in}}%
\pgfpathlineto{\pgfqpoint{3.993799in}{3.057308in}}%
\pgfpathlineto{\pgfqpoint{4.001226in}{3.073368in}}%
\pgfpathlineto{\pgfqpoint{4.008650in}{3.089640in}}%
\pgfpathlineto{\pgfqpoint{4.016071in}{3.106131in}}%
\pgfpathlineto{\pgfqpoint{4.023489in}{3.122845in}}%
\pgfpathlineto{\pgfqpoint{4.010664in}{3.132637in}}%
\pgfpathlineto{\pgfqpoint{3.997842in}{3.142629in}}%
\pgfpathlineto{\pgfqpoint{3.985022in}{3.152823in}}%
\pgfpathlineto{\pgfqpoint{3.972203in}{3.163220in}}%
\pgfpathlineto{\pgfqpoint{3.964780in}{3.146056in}}%
\pgfpathlineto{\pgfqpoint{3.957354in}{3.129122in}}%
\pgfpathlineto{\pgfqpoint{3.949925in}{3.112413in}}%
\pgfpathlineto{\pgfqpoint{3.942492in}{3.095924in}}%
\pgfpathclose%
\pgfusepath{fill}%
\end{pgfscope}%
\begin{pgfscope}%
\pgfpathrectangle{\pgfqpoint{1.254980in}{0.150000in}}{\pgfqpoint{5.490039in}{5.490039in}}%
\pgfusepath{clip}%
\pgfsetbuttcap%
\pgfsetroundjoin%
\definecolor{currentfill}{rgb}{0.168126,0.459988,0.558082}%
\pgfsetfillcolor{currentfill}%
\pgfsetfillopacity{0.700000}%
\pgfsetlinewidth{0.000000pt}%
\definecolor{currentstroke}{rgb}{0.000000,0.000000,0.000000}%
\pgfsetstrokecolor{currentstroke}%
\pgfsetdash{}{0pt}%
\pgfpathmoveto{\pgfqpoint{4.641509in}{3.368705in}}%
\pgfpathlineto{\pgfqpoint{4.654422in}{3.361151in}}%
\pgfpathlineto{\pgfqpoint{4.667340in}{3.353769in}}%
\pgfpathlineto{\pgfqpoint{4.680264in}{3.346556in}}%
\pgfpathlineto{\pgfqpoint{4.693193in}{3.339514in}}%
\pgfpathlineto{\pgfqpoint{4.700537in}{3.359485in}}%
\pgfpathlineto{\pgfqpoint{4.707883in}{3.379811in}}%
\pgfpathlineto{\pgfqpoint{4.715232in}{3.400499in}}%
\pgfpathlineto{\pgfqpoint{4.702309in}{3.408034in}}%
\pgfpathlineto{\pgfqpoint{4.689392in}{3.415739in}}%
\pgfpathlineto{\pgfqpoint{4.676480in}{3.423614in}}%
\pgfpathlineto{\pgfqpoint{4.663574in}{3.431661in}}%
\pgfpathlineto{\pgfqpoint{4.656216in}{3.410308in}}%
\pgfpathlineto{\pgfqpoint{4.648861in}{3.389326in}}%
\pgfpathlineto{\pgfqpoint{4.641509in}{3.368705in}}%
\pgfpathclose%
\pgfusepath{fill}%
\end{pgfscope}%
\begin{pgfscope}%
\pgfpathrectangle{\pgfqpoint{1.254980in}{0.150000in}}{\pgfqpoint{5.490039in}{5.490039in}}%
\pgfusepath{clip}%
\pgfsetbuttcap%
\pgfsetroundjoin%
\definecolor{currentfill}{rgb}{0.477504,0.821444,0.318195}%
\pgfsetfillcolor{currentfill}%
\pgfsetfillopacity{0.700000}%
\pgfsetlinewidth{0.000000pt}%
\definecolor{currentstroke}{rgb}{0.000000,0.000000,0.000000}%
\pgfsetstrokecolor{currentstroke}%
\pgfsetdash{}{0pt}%
\pgfpathmoveto{\pgfqpoint{3.555596in}{4.364790in}}%
\pgfpathlineto{\pgfqpoint{3.568517in}{4.341015in}}%
\pgfpathlineto{\pgfqpoint{3.581431in}{4.317533in}}%
\pgfpathlineto{\pgfqpoint{3.594338in}{4.294340in}}%
\pgfpathlineto{\pgfqpoint{3.607238in}{4.271434in}}%
\pgfpathlineto{\pgfqpoint{3.614609in}{4.301031in}}%
\pgfpathlineto{\pgfqpoint{3.621978in}{4.331078in}}%
\pgfpathlineto{\pgfqpoint{3.629344in}{4.361584in}}%
\pgfpathlineto{\pgfqpoint{3.636707in}{4.392557in}}%
\pgfpathlineto{\pgfqpoint{3.623801in}{4.416121in}}%
\pgfpathlineto{\pgfqpoint{3.610889in}{4.439974in}}%
\pgfpathlineto{\pgfqpoint{3.597969in}{4.464118in}}%
\pgfpathlineto{\pgfqpoint{3.585043in}{4.488557in}}%
\pgfpathlineto{\pgfqpoint{3.577685in}{4.456909in}}%
\pgfpathlineto{\pgfqpoint{3.570325in}{4.425738in}}%
\pgfpathlineto{\pgfqpoint{3.562962in}{4.395034in}}%
\pgfpathlineto{\pgfqpoint{3.555596in}{4.364790in}}%
\pgfpathclose%
\pgfusepath{fill}%
\end{pgfscope}%
\begin{pgfscope}%
\pgfpathrectangle{\pgfqpoint{1.254980in}{0.150000in}}{\pgfqpoint{5.490039in}{5.490039in}}%
\pgfusepath{clip}%
\pgfsetbuttcap%
\pgfsetroundjoin%
\definecolor{currentfill}{rgb}{0.214298,0.355619,0.551184}%
\pgfsetfillcolor{currentfill}%
\pgfsetfillopacity{0.700000}%
\pgfsetlinewidth{0.000000pt}%
\definecolor{currentstroke}{rgb}{0.000000,0.000000,0.000000}%
\pgfsetstrokecolor{currentstroke}%
\pgfsetdash{}{0pt}%
\pgfpathmoveto{\pgfqpoint{3.810152in}{3.115881in}}%
\pgfpathlineto{\pgfqpoint{3.822970in}{3.104667in}}%
\pgfpathlineto{\pgfqpoint{3.835789in}{3.093668in}}%
\pgfpathlineto{\pgfqpoint{3.848609in}{3.082880in}}%
\pgfpathlineto{\pgfqpoint{3.861430in}{3.072302in}}%
\pgfpathlineto{\pgfqpoint{3.868882in}{3.088363in}}%
\pgfpathlineto{\pgfqpoint{3.876329in}{3.104630in}}%
\pgfpathlineto{\pgfqpoint{3.883774in}{3.121110in}}%
\pgfpathlineto{\pgfqpoint{3.891215in}{3.137807in}}%
\pgfpathlineto{\pgfqpoint{3.878398in}{3.148798in}}%
\pgfpathlineto{\pgfqpoint{3.865584in}{3.159999in}}%
\pgfpathlineto{\pgfqpoint{3.852769in}{3.171412in}}%
\pgfpathlineto{\pgfqpoint{3.839956in}{3.183039in}}%
\pgfpathlineto{\pgfqpoint{3.832510in}{3.165918in}}%
\pgfpathlineto{\pgfqpoint{3.825061in}{3.149022in}}%
\pgfpathlineto{\pgfqpoint{3.817608in}{3.132344in}}%
\pgfpathlineto{\pgfqpoint{3.810152in}{3.115881in}}%
\pgfpathclose%
\pgfusepath{fill}%
\end{pgfscope}%
\begin{pgfscope}%
\pgfpathrectangle{\pgfqpoint{1.254980in}{0.150000in}}{\pgfqpoint{5.490039in}{5.490039in}}%
\pgfusepath{clip}%
\pgfsetbuttcap%
\pgfsetroundjoin%
\definecolor{currentfill}{rgb}{0.179019,0.433756,0.557430}%
\pgfsetfillcolor{currentfill}%
\pgfsetfillopacity{0.700000}%
\pgfsetlinewidth{0.000000pt}%
\definecolor{currentstroke}{rgb}{0.000000,0.000000,0.000000}%
\pgfsetstrokecolor{currentstroke}%
\pgfsetdash{}{0pt}%
\pgfpathmoveto{\pgfqpoint{3.523726in}{3.315745in}}%
\pgfpathlineto{\pgfqpoint{3.536569in}{3.300322in}}%
\pgfpathlineto{\pgfqpoint{3.549410in}{3.285146in}}%
\pgfpathlineto{\pgfqpoint{3.562248in}{3.270214in}}%
\pgfpathlineto{\pgfqpoint{3.575084in}{3.255526in}}%
\pgfpathlineto{\pgfqpoint{3.582577in}{3.272688in}}%
\pgfpathlineto{\pgfqpoint{3.590064in}{3.290075in}}%
\pgfpathlineto{\pgfqpoint{3.597548in}{3.307690in}}%
\pgfpathlineto{\pgfqpoint{3.605027in}{3.325538in}}%
\pgfpathlineto{\pgfqpoint{3.592195in}{3.340617in}}%
\pgfpathlineto{\pgfqpoint{3.579361in}{3.355940in}}%
\pgfpathlineto{\pgfqpoint{3.566524in}{3.371508in}}%
\pgfpathlineto{\pgfqpoint{3.553685in}{3.387323in}}%
\pgfpathlineto{\pgfqpoint{3.546202in}{3.369072in}}%
\pgfpathlineto{\pgfqpoint{3.538715in}{3.351062in}}%
\pgfpathlineto{\pgfqpoint{3.531222in}{3.333288in}}%
\pgfpathlineto{\pgfqpoint{3.523726in}{3.315745in}}%
\pgfpathclose%
\pgfusepath{fill}%
\end{pgfscope}%
\begin{pgfscope}%
\pgfpathrectangle{\pgfqpoint{1.254980in}{0.150000in}}{\pgfqpoint{5.490039in}{5.490039in}}%
\pgfusepath{clip}%
\pgfsetbuttcap%
\pgfsetroundjoin%
\definecolor{currentfill}{rgb}{0.188923,0.410910,0.556326}%
\pgfsetfillcolor{currentfill}%
\pgfsetfillopacity{0.700000}%
\pgfsetlinewidth{0.000000pt}%
\definecolor{currentstroke}{rgb}{0.000000,0.000000,0.000000}%
\pgfsetstrokecolor{currentstroke}%
\pgfsetdash{}{0pt}%
\pgfpathmoveto{\pgfqpoint{3.575084in}{3.255526in}}%
\pgfpathlineto{\pgfqpoint{3.587918in}{3.241078in}}%
\pgfpathlineto{\pgfqpoint{3.600749in}{3.226870in}}%
\pgfpathlineto{\pgfqpoint{3.613579in}{3.212899in}}%
\pgfpathlineto{\pgfqpoint{3.626407in}{3.199163in}}%
\pgfpathlineto{\pgfqpoint{3.633895in}{3.215946in}}%
\pgfpathlineto{\pgfqpoint{3.641378in}{3.232947in}}%
\pgfpathlineto{\pgfqpoint{3.648857in}{3.250168in}}%
\pgfpathlineto{\pgfqpoint{3.656332in}{3.267616in}}%
\pgfpathlineto{\pgfqpoint{3.643508in}{3.281741in}}%
\pgfpathlineto{\pgfqpoint{3.630683in}{3.296102in}}%
\pgfpathlineto{\pgfqpoint{3.617856in}{3.310700in}}%
\pgfpathlineto{\pgfqpoint{3.605027in}{3.325538in}}%
\pgfpathlineto{\pgfqpoint{3.597548in}{3.307690in}}%
\pgfpathlineto{\pgfqpoint{3.590064in}{3.290075in}}%
\pgfpathlineto{\pgfqpoint{3.582577in}{3.272688in}}%
\pgfpathlineto{\pgfqpoint{3.575084in}{3.255526in}}%
\pgfpathclose%
\pgfusepath{fill}%
\end{pgfscope}%
\begin{pgfscope}%
\pgfpathrectangle{\pgfqpoint{1.254980in}{0.150000in}}{\pgfqpoint{5.490039in}{5.490039in}}%
\pgfusepath{clip}%
\pgfsetbuttcap%
\pgfsetroundjoin%
\definecolor{currentfill}{rgb}{0.259857,0.745492,0.444467}%
\pgfsetfillcolor{currentfill}%
\pgfsetfillopacity{0.700000}%
\pgfsetlinewidth{0.000000pt}%
\definecolor{currentstroke}{rgb}{0.000000,0.000000,0.000000}%
\pgfsetstrokecolor{currentstroke}%
\pgfsetdash{}{0pt}%
\pgfpathmoveto{\pgfqpoint{3.415209in}{4.125899in}}%
\pgfpathlineto{\pgfqpoint{3.428149in}{4.102722in}}%
\pgfpathlineto{\pgfqpoint{3.441081in}{4.079845in}}%
\pgfpathlineto{\pgfqpoint{3.454006in}{4.057266in}}%
\pgfpathlineto{\pgfqpoint{3.466924in}{4.034981in}}%
\pgfpathlineto{\pgfqpoint{3.474336in}{4.060287in}}%
\pgfpathlineto{\pgfqpoint{3.481743in}{4.085966in}}%
\pgfpathlineto{\pgfqpoint{3.489145in}{4.112025in}}%
\pgfpathlineto{\pgfqpoint{3.496544in}{4.138471in}}%
\pgfpathlineto{\pgfqpoint{3.483624in}{4.161317in}}%
\pgfpathlineto{\pgfqpoint{3.470697in}{4.184459in}}%
\pgfpathlineto{\pgfqpoint{3.457762in}{4.207898in}}%
\pgfpathlineto{\pgfqpoint{3.444819in}{4.231640in}}%
\pgfpathlineto{\pgfqpoint{3.437423in}{4.204619in}}%
\pgfpathlineto{\pgfqpoint{3.430023in}{4.177993in}}%
\pgfpathlineto{\pgfqpoint{3.422618in}{4.151755in}}%
\pgfpathlineto{\pgfqpoint{3.415209in}{4.125899in}}%
\pgfpathclose%
\pgfusepath{fill}%
\end{pgfscope}%
\begin{pgfscope}%
\pgfpathrectangle{\pgfqpoint{1.254980in}{0.150000in}}{\pgfqpoint{5.490039in}{5.490039in}}%
\pgfusepath{clip}%
\pgfsetbuttcap%
\pgfsetroundjoin%
\definecolor{currentfill}{rgb}{0.191090,0.708366,0.482284}%
\pgfsetfillcolor{currentfill}%
\pgfsetfillopacity{0.700000}%
\pgfsetlinewidth{0.000000pt}%
\definecolor{currentstroke}{rgb}{0.000000,0.000000,0.000000}%
\pgfsetstrokecolor{currentstroke}%
\pgfsetdash{}{0pt}%
\pgfpathmoveto{\pgfqpoint{3.385523in}{4.026158in}}%
\pgfpathlineto{\pgfqpoint{3.398462in}{4.003512in}}%
\pgfpathlineto{\pgfqpoint{3.411393in}{3.981165in}}%
\pgfpathlineto{\pgfqpoint{3.424317in}{3.959115in}}%
\pgfpathlineto{\pgfqpoint{3.437235in}{3.937357in}}%
\pgfpathlineto{\pgfqpoint{3.444664in}{3.961235in}}%
\pgfpathlineto{\pgfqpoint{3.452089in}{3.985461in}}%
\pgfpathlineto{\pgfqpoint{3.459509in}{4.010041in}}%
\pgfpathlineto{\pgfqpoint{3.466924in}{4.034981in}}%
\pgfpathlineto{\pgfqpoint{3.454006in}{4.057266in}}%
\pgfpathlineto{\pgfqpoint{3.441081in}{4.079845in}}%
\pgfpathlineto{\pgfqpoint{3.428149in}{4.102722in}}%
\pgfpathlineto{\pgfqpoint{3.415209in}{4.125899in}}%
\pgfpathlineto{\pgfqpoint{3.407794in}{4.100417in}}%
\pgfpathlineto{\pgfqpoint{3.400376in}{4.075304in}}%
\pgfpathlineto{\pgfqpoint{3.392952in}{4.050553in}}%
\pgfpathlineto{\pgfqpoint{3.385523in}{4.026158in}}%
\pgfpathclose%
\pgfusepath{fill}%
\end{pgfscope}%
\begin{pgfscope}%
\pgfpathrectangle{\pgfqpoint{1.254980in}{0.150000in}}{\pgfqpoint{5.490039in}{5.490039in}}%
\pgfusepath{clip}%
\pgfsetbuttcap%
\pgfsetroundjoin%
\definecolor{currentfill}{rgb}{0.168126,0.459988,0.558082}%
\pgfsetfillcolor{currentfill}%
\pgfsetfillopacity{0.700000}%
\pgfsetlinewidth{0.000000pt}%
\definecolor{currentstroke}{rgb}{0.000000,0.000000,0.000000}%
\pgfsetstrokecolor{currentstroke}%
\pgfsetdash{}{0pt}%
\pgfpathmoveto{\pgfqpoint{3.472319in}{3.379949in}}%
\pgfpathlineto{\pgfqpoint{3.485176in}{3.363517in}}%
\pgfpathlineto{\pgfqpoint{3.498029in}{3.347341in}}%
\pgfpathlineto{\pgfqpoint{3.510879in}{3.331417in}}%
\pgfpathlineto{\pgfqpoint{3.523726in}{3.315745in}}%
\pgfpathlineto{\pgfqpoint{3.531222in}{3.333288in}}%
\pgfpathlineto{\pgfqpoint{3.538715in}{3.351062in}}%
\pgfpathlineto{\pgfqpoint{3.546202in}{3.369072in}}%
\pgfpathlineto{\pgfqpoint{3.553685in}{3.387323in}}%
\pgfpathlineto{\pgfqpoint{3.540842in}{3.403388in}}%
\pgfpathlineto{\pgfqpoint{3.527996in}{3.419704in}}%
\pgfpathlineto{\pgfqpoint{3.515147in}{3.436274in}}%
\pgfpathlineto{\pgfqpoint{3.502294in}{3.453099in}}%
\pgfpathlineto{\pgfqpoint{3.494808in}{3.434444in}}%
\pgfpathlineto{\pgfqpoint{3.487316in}{3.416037in}}%
\pgfpathlineto{\pgfqpoint{3.479820in}{3.397874in}}%
\pgfpathlineto{\pgfqpoint{3.472319in}{3.379949in}}%
\pgfpathclose%
\pgfusepath{fill}%
\end{pgfscope}%
\begin{pgfscope}%
\pgfpathrectangle{\pgfqpoint{1.254980in}{0.150000in}}{\pgfqpoint{5.490039in}{5.490039in}}%
\pgfusepath{clip}%
\pgfsetbuttcap%
\pgfsetroundjoin%
\definecolor{currentfill}{rgb}{0.199430,0.387607,0.554642}%
\pgfsetfillcolor{currentfill}%
\pgfsetfillopacity{0.700000}%
\pgfsetlinewidth{0.000000pt}%
\definecolor{currentstroke}{rgb}{0.000000,0.000000,0.000000}%
\pgfsetstrokecolor{currentstroke}%
\pgfsetdash{}{0pt}%
\pgfpathmoveto{\pgfqpoint{3.626407in}{3.199163in}}%
\pgfpathlineto{\pgfqpoint{3.639234in}{3.185661in}}%
\pgfpathlineto{\pgfqpoint{3.652060in}{3.172391in}}%
\pgfpathlineto{\pgfqpoint{3.664884in}{3.159351in}}%
\pgfpathlineto{\pgfqpoint{3.677708in}{3.146539in}}%
\pgfpathlineto{\pgfqpoint{3.685191in}{3.162945in}}%
\pgfpathlineto{\pgfqpoint{3.692669in}{3.179560in}}%
\pgfpathlineto{\pgfqpoint{3.700143in}{3.196390in}}%
\pgfpathlineto{\pgfqpoint{3.707614in}{3.213438in}}%
\pgfpathlineto{\pgfqpoint{3.694795in}{3.226638in}}%
\pgfpathlineto{\pgfqpoint{3.681975in}{3.240066in}}%
\pgfpathlineto{\pgfqpoint{3.669154in}{3.253725in}}%
\pgfpathlineto{\pgfqpoint{3.656332in}{3.267616in}}%
\pgfpathlineto{\pgfqpoint{3.648857in}{3.250168in}}%
\pgfpathlineto{\pgfqpoint{3.641378in}{3.232947in}}%
\pgfpathlineto{\pgfqpoint{3.633895in}{3.215946in}}%
\pgfpathlineto{\pgfqpoint{3.626407in}{3.199163in}}%
\pgfpathclose%
\pgfusepath{fill}%
\end{pgfscope}%
\begin{pgfscope}%
\pgfpathrectangle{\pgfqpoint{1.254980in}{0.150000in}}{\pgfqpoint{5.490039in}{5.490039in}}%
\pgfusepath{clip}%
\pgfsetbuttcap%
\pgfsetroundjoin%
\definecolor{currentfill}{rgb}{0.220057,0.343307,0.549413}%
\pgfsetfillcolor{currentfill}%
\pgfsetfillopacity{0.700000}%
\pgfsetlinewidth{0.000000pt}%
\definecolor{currentstroke}{rgb}{0.000000,0.000000,0.000000}%
\pgfsetstrokecolor{currentstroke}%
\pgfsetdash{}{0pt}%
\pgfpathmoveto{\pgfqpoint{4.074812in}{3.085662in}}%
\pgfpathlineto{\pgfqpoint{4.087650in}{3.076856in}}%
\pgfpathlineto{\pgfqpoint{4.100491in}{3.068244in}}%
\pgfpathlineto{\pgfqpoint{4.113335in}{3.059825in}}%
\pgfpathlineto{\pgfqpoint{4.126183in}{3.051598in}}%
\pgfpathlineto{\pgfqpoint{4.133586in}{3.067629in}}%
\pgfpathlineto{\pgfqpoint{4.140987in}{3.083879in}}%
\pgfpathlineto{\pgfqpoint{4.148386in}{3.100355in}}%
\pgfpathlineto{\pgfqpoint{4.155782in}{3.117063in}}%
\pgfpathlineto{\pgfqpoint{4.142941in}{3.125756in}}%
\pgfpathlineto{\pgfqpoint{4.130103in}{3.134641in}}%
\pgfpathlineto{\pgfqpoint{4.117268in}{3.143719in}}%
\pgfpathlineto{\pgfqpoint{4.104436in}{3.152992in}}%
\pgfpathlineto{\pgfqpoint{4.097033in}{3.135807in}}%
\pgfpathlineto{\pgfqpoint{4.089629in}{3.118861in}}%
\pgfpathlineto{\pgfqpoint{4.082222in}{3.102148in}}%
\pgfpathlineto{\pgfqpoint{4.074812in}{3.085662in}}%
\pgfpathclose%
\pgfusepath{fill}%
\end{pgfscope}%
\begin{pgfscope}%
\pgfpathrectangle{\pgfqpoint{1.254980in}{0.150000in}}{\pgfqpoint{5.490039in}{5.490039in}}%
\pgfusepath{clip}%
\pgfsetbuttcap%
\pgfsetroundjoin%
\definecolor{currentfill}{rgb}{0.344074,0.780029,0.397381}%
\pgfsetfillcolor{currentfill}%
\pgfsetfillopacity{0.700000}%
\pgfsetlinewidth{0.000000pt}%
\definecolor{currentstroke}{rgb}{0.000000,0.000000,0.000000}%
\pgfsetstrokecolor{currentstroke}%
\pgfsetdash{}{0pt}%
\pgfpathmoveto{\pgfqpoint{3.444819in}{4.231640in}}%
\pgfpathlineto{\pgfqpoint{3.457762in}{4.207898in}}%
\pgfpathlineto{\pgfqpoint{3.470697in}{4.184459in}}%
\pgfpathlineto{\pgfqpoint{3.483624in}{4.161317in}}%
\pgfpathlineto{\pgfqpoint{3.496544in}{4.138471in}}%
\pgfpathlineto{\pgfqpoint{3.503939in}{4.165310in}}%
\pgfpathlineto{\pgfqpoint{3.511329in}{4.192549in}}%
\pgfpathlineto{\pgfqpoint{3.518716in}{4.220195in}}%
\pgfpathlineto{\pgfqpoint{3.526099in}{4.248256in}}%
\pgfpathlineto{\pgfqpoint{3.513176in}{4.271696in}}%
\pgfpathlineto{\pgfqpoint{3.500245in}{4.295434in}}%
\pgfpathlineto{\pgfqpoint{3.487306in}{4.319471in}}%
\pgfpathlineto{\pgfqpoint{3.474360in}{4.343811in}}%
\pgfpathlineto{\pgfqpoint{3.466981in}{4.315141in}}%
\pgfpathlineto{\pgfqpoint{3.459598in}{4.286894in}}%
\pgfpathlineto{\pgfqpoint{3.452211in}{4.259062in}}%
\pgfpathlineto{\pgfqpoint{3.444819in}{4.231640in}}%
\pgfpathclose%
\pgfusepath{fill}%
\end{pgfscope}%
\begin{pgfscope}%
\pgfpathrectangle{\pgfqpoint{1.254980in}{0.150000in}}{\pgfqpoint{5.490039in}{5.490039in}}%
\pgfusepath{clip}%
\pgfsetbuttcap%
\pgfsetroundjoin%
\definecolor{currentfill}{rgb}{0.146616,0.673050,0.508936}%
\pgfsetfillcolor{currentfill}%
\pgfsetfillopacity{0.700000}%
\pgfsetlinewidth{0.000000pt}%
\definecolor{currentstroke}{rgb}{0.000000,0.000000,0.000000}%
\pgfsetstrokecolor{currentstroke}%
\pgfsetdash{}{0pt}%
\pgfpathmoveto{\pgfqpoint{3.355757in}{3.932012in}}%
\pgfpathlineto{\pgfqpoint{3.368696in}{3.909865in}}%
\pgfpathlineto{\pgfqpoint{3.381627in}{3.888016in}}%
\pgfpathlineto{\pgfqpoint{3.394552in}{3.866461in}}%
\pgfpathlineto{\pgfqpoint{3.407469in}{3.845199in}}%
\pgfpathlineto{\pgfqpoint{3.414918in}{3.867747in}}%
\pgfpathlineto{\pgfqpoint{3.422361in}{3.890619in}}%
\pgfpathlineto{\pgfqpoint{3.429800in}{3.913820in}}%
\pgfpathlineto{\pgfqpoint{3.437235in}{3.937357in}}%
\pgfpathlineto{\pgfqpoint{3.424317in}{3.959115in}}%
\pgfpathlineto{\pgfqpoint{3.411393in}{3.981165in}}%
\pgfpathlineto{\pgfqpoint{3.398462in}{4.003512in}}%
\pgfpathlineto{\pgfqpoint{3.385523in}{4.026158in}}%
\pgfpathlineto{\pgfqpoint{3.378089in}{4.002112in}}%
\pgfpathlineto{\pgfqpoint{3.370650in}{3.978410in}}%
\pgfpathlineto{\pgfqpoint{3.363206in}{3.955045in}}%
\pgfpathlineto{\pgfqpoint{3.355757in}{3.932012in}}%
\pgfpathclose%
\pgfusepath{fill}%
\end{pgfscope}%
\begin{pgfscope}%
\pgfpathrectangle{\pgfqpoint{1.254980in}{0.150000in}}{\pgfqpoint{5.490039in}{5.490039in}}%
\pgfusepath{clip}%
\pgfsetbuttcap%
\pgfsetroundjoin%
\definecolor{currentfill}{rgb}{0.157729,0.485932,0.558013}%
\pgfsetfillcolor{currentfill}%
\pgfsetfillopacity{0.700000}%
\pgfsetlinewidth{0.000000pt}%
\definecolor{currentstroke}{rgb}{0.000000,0.000000,0.000000}%
\pgfsetstrokecolor{currentstroke}%
\pgfsetdash{}{0pt}%
\pgfpathmoveto{\pgfqpoint{3.420851in}{3.448272in}}%
\pgfpathlineto{\pgfqpoint{3.433724in}{3.430797in}}%
\pgfpathlineto{\pgfqpoint{3.446593in}{3.413586in}}%
\pgfpathlineto{\pgfqpoint{3.459458in}{3.396638in}}%
\pgfpathlineto{\pgfqpoint{3.472319in}{3.379949in}}%
\pgfpathlineto{\pgfqpoint{3.479820in}{3.397874in}}%
\pgfpathlineto{\pgfqpoint{3.487316in}{3.416037in}}%
\pgfpathlineto{\pgfqpoint{3.494808in}{3.434444in}}%
\pgfpathlineto{\pgfqpoint{3.502294in}{3.453099in}}%
\pgfpathlineto{\pgfqpoint{3.489437in}{3.470183in}}%
\pgfpathlineto{\pgfqpoint{3.476576in}{3.487527in}}%
\pgfpathlineto{\pgfqpoint{3.463711in}{3.505133in}}%
\pgfpathlineto{\pgfqpoint{3.450841in}{3.523004in}}%
\pgfpathlineto{\pgfqpoint{3.443351in}{3.503943in}}%
\pgfpathlineto{\pgfqpoint{3.435856in}{3.485137in}}%
\pgfpathlineto{\pgfqpoint{3.428356in}{3.466581in}}%
\pgfpathlineto{\pgfqpoint{3.420851in}{3.448272in}}%
\pgfpathclose%
\pgfusepath{fill}%
\end{pgfscope}%
\begin{pgfscope}%
\pgfpathrectangle{\pgfqpoint{1.254980in}{0.150000in}}{\pgfqpoint{5.490039in}{5.490039in}}%
\pgfusepath{clip}%
\pgfsetbuttcap%
\pgfsetroundjoin%
\definecolor{currentfill}{rgb}{0.626579,0.854645,0.223353}%
\pgfsetfillcolor{currentfill}%
\pgfsetfillopacity{0.700000}%
\pgfsetlinewidth{0.000000pt}%
\definecolor{currentstroke}{rgb}{0.000000,0.000000,0.000000}%
\pgfsetstrokecolor{currentstroke}%
\pgfsetdash{}{0pt}%
\pgfpathmoveto{\pgfqpoint{3.666139in}{4.521280in}}%
\pgfpathlineto{\pgfqpoint{3.679045in}{4.497308in}}%
\pgfpathlineto{\pgfqpoint{3.691944in}{4.473620in}}%
\pgfpathlineto{\pgfqpoint{3.704837in}{4.450214in}}%
\pgfpathlineto{\pgfqpoint{3.717724in}{4.427087in}}%
\pgfpathlineto{\pgfqpoint{3.725085in}{4.459805in}}%
\pgfpathlineto{\pgfqpoint{3.732444in}{4.493030in}}%
\pgfpathlineto{\pgfqpoint{3.739802in}{4.526773in}}%
\pgfpathlineto{\pgfqpoint{3.726909in}{4.550441in}}%
\pgfpathlineto{\pgfqpoint{3.714010in}{4.574389in}}%
\pgfpathlineto{\pgfqpoint{3.701105in}{4.598620in}}%
\pgfpathlineto{\pgfqpoint{3.688193in}{4.623136in}}%
\pgfpathlineto{\pgfqpoint{3.680843in}{4.588661in}}%
\pgfpathlineto{\pgfqpoint{3.673492in}{4.554712in}}%
\pgfpathlineto{\pgfqpoint{3.666139in}{4.521280in}}%
\pgfpathclose%
\pgfusepath{fill}%
\end{pgfscope}%
\begin{pgfscope}%
\pgfpathrectangle{\pgfqpoint{1.254980in}{0.150000in}}{\pgfqpoint{5.490039in}{5.490039in}}%
\pgfusepath{clip}%
\pgfsetbuttcap%
\pgfsetroundjoin%
\definecolor{currentfill}{rgb}{0.195860,0.395433,0.555276}%
\pgfsetfillcolor{currentfill}%
\pgfsetfillopacity{0.700000}%
\pgfsetlinewidth{0.000000pt}%
\definecolor{currentstroke}{rgb}{0.000000,0.000000,0.000000}%
\pgfsetstrokecolor{currentstroke}%
\pgfsetdash{}{0pt}%
\pgfpathmoveto{\pgfqpoint{4.450106in}{3.198923in}}%
\pgfpathlineto{\pgfqpoint{4.463000in}{3.191790in}}%
\pgfpathlineto{\pgfqpoint{4.475899in}{3.184833in}}%
\pgfpathlineto{\pgfqpoint{4.488804in}{3.178051in}}%
\pgfpathlineto{\pgfqpoint{4.501714in}{3.171445in}}%
\pgfpathlineto{\pgfqpoint{4.509060in}{3.188866in}}%
\pgfpathlineto{\pgfqpoint{4.516406in}{3.206569in}}%
\pgfpathlineto{\pgfqpoint{4.523752in}{3.224561in}}%
\pgfpathlineto{\pgfqpoint{4.531099in}{3.242849in}}%
\pgfpathlineto{\pgfqpoint{4.518197in}{3.250032in}}%
\pgfpathlineto{\pgfqpoint{4.505300in}{3.257389in}}%
\pgfpathlineto{\pgfqpoint{4.492409in}{3.264922in}}%
\pgfpathlineto{\pgfqpoint{4.479523in}{3.272631in}}%
\pgfpathlineto{\pgfqpoint{4.472168in}{3.253757in}}%
\pgfpathlineto{\pgfqpoint{4.464814in}{3.235186in}}%
\pgfpathlineto{\pgfqpoint{4.457460in}{3.216910in}}%
\pgfpathlineto{\pgfqpoint{4.450106in}{3.198923in}}%
\pgfpathclose%
\pgfusepath{fill}%
\end{pgfscope}%
\begin{pgfscope}%
\pgfpathrectangle{\pgfqpoint{1.254980in}{0.150000in}}{\pgfqpoint{5.490039in}{5.490039in}}%
\pgfusepath{clip}%
\pgfsetbuttcap%
\pgfsetroundjoin%
\definecolor{currentfill}{rgb}{0.208623,0.367752,0.552675}%
\pgfsetfillcolor{currentfill}%
\pgfsetfillopacity{0.700000}%
\pgfsetlinewidth{0.000000pt}%
\definecolor{currentstroke}{rgb}{0.000000,0.000000,0.000000}%
\pgfsetstrokecolor{currentstroke}%
\pgfsetdash{}{0pt}%
\pgfpathmoveto{\pgfqpoint{3.677708in}{3.146539in}}%
\pgfpathlineto{\pgfqpoint{3.690531in}{3.133954in}}%
\pgfpathlineto{\pgfqpoint{3.703353in}{3.121594in}}%
\pgfpathlineto{\pgfqpoint{3.716175in}{3.109458in}}%
\pgfpathlineto{\pgfqpoint{3.728997in}{3.097543in}}%
\pgfpathlineto{\pgfqpoint{3.736475in}{3.113573in}}%
\pgfpathlineto{\pgfqpoint{3.743948in}{3.129804in}}%
\pgfpathlineto{\pgfqpoint{3.751418in}{3.146243in}}%
\pgfpathlineto{\pgfqpoint{3.758883in}{3.162894in}}%
\pgfpathlineto{\pgfqpoint{3.746066in}{3.175195in}}%
\pgfpathlineto{\pgfqpoint{3.733249in}{3.187719in}}%
\pgfpathlineto{\pgfqpoint{3.720432in}{3.200466in}}%
\pgfpathlineto{\pgfqpoint{3.707614in}{3.213438in}}%
\pgfpathlineto{\pgfqpoint{3.700143in}{3.196390in}}%
\pgfpathlineto{\pgfqpoint{3.692669in}{3.179560in}}%
\pgfpathlineto{\pgfqpoint{3.685191in}{3.162945in}}%
\pgfpathlineto{\pgfqpoint{3.677708in}{3.146539in}}%
\pgfpathclose%
\pgfusepath{fill}%
\end{pgfscope}%
\begin{pgfscope}%
\pgfpathrectangle{\pgfqpoint{1.254980in}{0.150000in}}{\pgfqpoint{5.490039in}{5.490039in}}%
\pgfusepath{clip}%
\pgfsetbuttcap%
\pgfsetroundjoin%
\definecolor{currentfill}{rgb}{0.203063,0.379716,0.553925}%
\pgfsetfillcolor{currentfill}%
\pgfsetfillopacity{0.700000}%
\pgfsetlinewidth{0.000000pt}%
\definecolor{currentstroke}{rgb}{0.000000,0.000000,0.000000}%
\pgfsetstrokecolor{currentstroke}%
\pgfsetdash{}{0pt}%
\pgfpathmoveto{\pgfqpoint{4.369132in}{3.157855in}}%
\pgfpathlineto{\pgfqpoint{4.382014in}{3.150557in}}%
\pgfpathlineto{\pgfqpoint{4.394900in}{3.143438in}}%
\pgfpathlineto{\pgfqpoint{4.407791in}{3.136498in}}%
\pgfpathlineto{\pgfqpoint{4.420688in}{3.129736in}}%
\pgfpathlineto{\pgfqpoint{4.428043in}{3.146632in}}%
\pgfpathlineto{\pgfqpoint{4.435398in}{3.163791in}}%
\pgfpathlineto{\pgfqpoint{4.442752in}{3.181219in}}%
\pgfpathlineto{\pgfqpoint{4.450106in}{3.198923in}}%
\pgfpathlineto{\pgfqpoint{4.437218in}{3.206234in}}%
\pgfpathlineto{\pgfqpoint{4.424334in}{3.213722in}}%
\pgfpathlineto{\pgfqpoint{4.411455in}{3.221389in}}%
\pgfpathlineto{\pgfqpoint{4.398580in}{3.229235in}}%
\pgfpathlineto{\pgfqpoint{4.391219in}{3.210973in}}%
\pgfpathlineto{\pgfqpoint{4.383857in}{3.192992in}}%
\pgfpathlineto{\pgfqpoint{4.376495in}{3.175289in}}%
\pgfpathlineto{\pgfqpoint{4.369132in}{3.157855in}}%
\pgfpathclose%
\pgfusepath{fill}%
\end{pgfscope}%
\begin{pgfscope}%
\pgfpathrectangle{\pgfqpoint{1.254980in}{0.150000in}}{\pgfqpoint{5.490039in}{5.490039in}}%
\pgfusepath{clip}%
\pgfsetbuttcap%
\pgfsetroundjoin%
\definecolor{currentfill}{rgb}{0.187231,0.414746,0.556547}%
\pgfsetfillcolor{currentfill}%
\pgfsetfillopacity{0.700000}%
\pgfsetlinewidth{0.000000pt}%
\definecolor{currentstroke}{rgb}{0.000000,0.000000,0.000000}%
\pgfsetstrokecolor{currentstroke}%
\pgfsetdash{}{0pt}%
\pgfpathmoveto{\pgfqpoint{4.531099in}{3.242849in}}%
\pgfpathlineto{\pgfqpoint{4.544006in}{3.235841in}}%
\pgfpathlineto{\pgfqpoint{4.556918in}{3.229006in}}%
\pgfpathlineto{\pgfqpoint{4.569837in}{3.222344in}}%
\pgfpathlineto{\pgfqpoint{4.582761in}{3.215854in}}%
\pgfpathlineto{\pgfqpoint{4.590100in}{3.233852in}}%
\pgfpathlineto{\pgfqpoint{4.597440in}{3.252151in}}%
\pgfpathlineto{\pgfqpoint{4.604780in}{3.270761in}}%
\pgfpathlineto{\pgfqpoint{4.612123in}{3.289687in}}%
\pgfpathlineto{\pgfqpoint{4.599207in}{3.296780in}}%
\pgfpathlineto{\pgfqpoint{4.586297in}{3.304046in}}%
\pgfpathlineto{\pgfqpoint{4.573393in}{3.311484in}}%
\pgfpathlineto{\pgfqpoint{4.560494in}{3.319096in}}%
\pgfpathlineto{\pgfqpoint{4.553143in}{3.299556in}}%
\pgfpathlineto{\pgfqpoint{4.545794in}{3.280339in}}%
\pgfpathlineto{\pgfqpoint{4.538446in}{3.261439in}}%
\pgfpathlineto{\pgfqpoint{4.531099in}{3.242849in}}%
\pgfpathclose%
\pgfusepath{fill}%
\end{pgfscope}%
\begin{pgfscope}%
\pgfpathrectangle{\pgfqpoint{1.254980in}{0.150000in}}{\pgfqpoint{5.490039in}{5.490039in}}%
\pgfusepath{clip}%
\pgfsetbuttcap%
\pgfsetroundjoin%
\definecolor{currentfill}{rgb}{0.125394,0.574318,0.549086}%
\pgfsetfillcolor{currentfill}%
\pgfsetfillopacity{0.700000}%
\pgfsetlinewidth{0.000000pt}%
\definecolor{currentstroke}{rgb}{0.000000,0.000000,0.000000}%
\pgfsetstrokecolor{currentstroke}%
\pgfsetdash{}{0pt}%
\pgfpathmoveto{\pgfqpoint{3.347692in}{3.675800in}}%
\pgfpathlineto{\pgfqpoint{3.360607in}{3.655721in}}%
\pgfpathlineto{\pgfqpoint{3.373515in}{3.635927in}}%
\pgfpathlineto{\pgfqpoint{3.386416in}{3.616416in}}%
\pgfpathlineto{\pgfqpoint{3.399312in}{3.597185in}}%
\pgfpathlineto{\pgfqpoint{3.406800in}{3.616923in}}%
\pgfpathlineto{\pgfqpoint{3.414283in}{3.636933in}}%
\pgfpathlineto{\pgfqpoint{3.421760in}{3.657222in}}%
\pgfpathlineto{\pgfqpoint{3.429233in}{3.677794in}}%
\pgfpathlineto{\pgfqpoint{3.416340in}{3.697455in}}%
\pgfpathlineto{\pgfqpoint{3.403440in}{3.717396in}}%
\pgfpathlineto{\pgfqpoint{3.390535in}{3.737620in}}%
\pgfpathlineto{\pgfqpoint{3.377623in}{3.758130in}}%
\pgfpathlineto{\pgfqpoint{3.370148in}{3.737117in}}%
\pgfpathlineto{\pgfqpoint{3.362668in}{3.716394in}}%
\pgfpathlineto{\pgfqpoint{3.355183in}{3.695956in}}%
\pgfpathlineto{\pgfqpoint{3.347692in}{3.675800in}}%
\pgfpathclose%
\pgfusepath{fill}%
\end{pgfscope}%
\begin{pgfscope}%
\pgfpathrectangle{\pgfqpoint{1.254980in}{0.150000in}}{\pgfqpoint{5.490039in}{5.490039in}}%
\pgfusepath{clip}%
\pgfsetbuttcap%
\pgfsetroundjoin%
\definecolor{currentfill}{rgb}{0.210503,0.363727,0.552206}%
\pgfsetfillcolor{currentfill}%
\pgfsetfillopacity{0.700000}%
\pgfsetlinewidth{0.000000pt}%
\definecolor{currentstroke}{rgb}{0.000000,0.000000,0.000000}%
\pgfsetstrokecolor{currentstroke}%
\pgfsetdash{}{0pt}%
\pgfpathmoveto{\pgfqpoint{4.288162in}{3.119612in}}%
\pgfpathlineto{\pgfqpoint{4.301032in}{3.112110in}}%
\pgfpathlineto{\pgfqpoint{4.313907in}{3.104789in}}%
\pgfpathlineto{\pgfqpoint{4.326786in}{3.097651in}}%
\pgfpathlineto{\pgfqpoint{4.339670in}{3.090693in}}%
\pgfpathlineto{\pgfqpoint{4.347038in}{3.107110in}}%
\pgfpathlineto{\pgfqpoint{4.354404in}{3.123772in}}%
\pgfpathlineto{\pgfqpoint{4.361768in}{3.140685in}}%
\pgfpathlineto{\pgfqpoint{4.369132in}{3.157855in}}%
\pgfpathlineto{\pgfqpoint{4.356255in}{3.165333in}}%
\pgfpathlineto{\pgfqpoint{4.343383in}{3.172992in}}%
\pgfpathlineto{\pgfqpoint{4.330516in}{3.180833in}}%
\pgfpathlineto{\pgfqpoint{4.317652in}{3.188857in}}%
\pgfpathlineto{\pgfqpoint{4.310282in}{3.171155in}}%
\pgfpathlineto{\pgfqpoint{4.302910in}{3.153718in}}%
\pgfpathlineto{\pgfqpoint{4.295537in}{3.136539in}}%
\pgfpathlineto{\pgfqpoint{4.288162in}{3.119612in}}%
\pgfpathclose%
\pgfusepath{fill}%
\end{pgfscope}%
\begin{pgfscope}%
\pgfpathrectangle{\pgfqpoint{1.254980in}{0.150000in}}{\pgfqpoint{5.490039in}{5.490039in}}%
\pgfusepath{clip}%
\pgfsetbuttcap%
\pgfsetroundjoin%
\definecolor{currentfill}{rgb}{0.221989,0.339161,0.548752}%
\pgfsetfillcolor{currentfill}%
\pgfsetfillopacity{0.700000}%
\pgfsetlinewidth{0.000000pt}%
\definecolor{currentstroke}{rgb}{0.000000,0.000000,0.000000}%
\pgfsetstrokecolor{currentstroke}%
\pgfsetdash{}{0pt}%
\pgfpathmoveto{\pgfqpoint{3.861430in}{3.072302in}}%
\pgfpathlineto{\pgfqpoint{3.874253in}{3.061933in}}%
\pgfpathlineto{\pgfqpoint{3.887077in}{3.051773in}}%
\pgfpathlineto{\pgfqpoint{3.899902in}{3.041818in}}%
\pgfpathlineto{\pgfqpoint{3.912730in}{3.032069in}}%
\pgfpathlineto{\pgfqpoint{3.920175in}{3.047728in}}%
\pgfpathlineto{\pgfqpoint{3.927618in}{3.063587in}}%
\pgfpathlineto{\pgfqpoint{3.935057in}{3.079651in}}%
\pgfpathlineto{\pgfqpoint{3.942492in}{3.095924in}}%
\pgfpathlineto{\pgfqpoint{3.929670in}{3.106086in}}%
\pgfpathlineto{\pgfqpoint{3.916850in}{3.116452in}}%
\pgfpathlineto{\pgfqpoint{3.904032in}{3.127026in}}%
\pgfpathlineto{\pgfqpoint{3.891215in}{3.137807in}}%
\pgfpathlineto{\pgfqpoint{3.883774in}{3.121110in}}%
\pgfpathlineto{\pgfqpoint{3.876329in}{3.104630in}}%
\pgfpathlineto{\pgfqpoint{3.868882in}{3.088363in}}%
\pgfpathlineto{\pgfqpoint{3.861430in}{3.072302in}}%
\pgfpathclose%
\pgfusepath{fill}%
\end{pgfscope}%
\begin{pgfscope}%
\pgfpathrectangle{\pgfqpoint{1.254980in}{0.150000in}}{\pgfqpoint{5.490039in}{5.490039in}}%
\pgfusepath{clip}%
\pgfsetbuttcap%
\pgfsetroundjoin%
\definecolor{currentfill}{rgb}{0.458674,0.816363,0.329727}%
\pgfsetfillcolor{currentfill}%
\pgfsetfillopacity{0.700000}%
\pgfsetlinewidth{0.000000pt}%
\definecolor{currentstroke}{rgb}{0.000000,0.000000,0.000000}%
\pgfsetstrokecolor{currentstroke}%
\pgfsetdash{}{0pt}%
\pgfpathmoveto{\pgfqpoint{3.474360in}{4.343811in}}%
\pgfpathlineto{\pgfqpoint{3.487306in}{4.319471in}}%
\pgfpathlineto{\pgfqpoint{3.500245in}{4.295434in}}%
\pgfpathlineto{\pgfqpoint{3.513176in}{4.271696in}}%
\pgfpathlineto{\pgfqpoint{3.526099in}{4.248256in}}%
\pgfpathlineto{\pgfqpoint{3.533479in}{4.276738in}}%
\pgfpathlineto{\pgfqpoint{3.540855in}{4.305649in}}%
\pgfpathlineto{\pgfqpoint{3.548227in}{4.334997in}}%
\pgfpathlineto{\pgfqpoint{3.555596in}{4.364790in}}%
\pgfpathlineto{\pgfqpoint{3.542668in}{4.388860in}}%
\pgfpathlineto{\pgfqpoint{3.529733in}{4.413228in}}%
\pgfpathlineto{\pgfqpoint{3.516789in}{4.437898in}}%
\pgfpathlineto{\pgfqpoint{3.503838in}{4.462872in}}%
\pgfpathlineto{\pgfqpoint{3.496474in}{4.432435in}}%
\pgfpathlineto{\pgfqpoint{3.489106in}{4.402451in}}%
\pgfpathlineto{\pgfqpoint{3.481735in}{4.372912in}}%
\pgfpathlineto{\pgfqpoint{3.474360in}{4.343811in}}%
\pgfpathclose%
\pgfusepath{fill}%
\end{pgfscope}%
\begin{pgfscope}%
\pgfpathrectangle{\pgfqpoint{1.254980in}{0.150000in}}{\pgfqpoint{5.490039in}{5.490039in}}%
\pgfusepath{clip}%
\pgfsetbuttcap%
\pgfsetroundjoin%
\definecolor{currentfill}{rgb}{0.179019,0.433756,0.557430}%
\pgfsetfillcolor{currentfill}%
\pgfsetfillopacity{0.700000}%
\pgfsetlinewidth{0.000000pt}%
\definecolor{currentstroke}{rgb}{0.000000,0.000000,0.000000}%
\pgfsetstrokecolor{currentstroke}%
\pgfsetdash{}{0pt}%
\pgfpathmoveto{\pgfqpoint{4.612123in}{3.289687in}}%
\pgfpathlineto{\pgfqpoint{4.625044in}{3.282765in}}%
\pgfpathlineto{\pgfqpoint{4.637971in}{3.276014in}}%
\pgfpathlineto{\pgfqpoint{4.650904in}{3.269433in}}%
\pgfpathlineto{\pgfqpoint{4.663843in}{3.263022in}}%
\pgfpathlineto{\pgfqpoint{4.671177in}{3.281651in}}%
\pgfpathlineto{\pgfqpoint{4.678514in}{3.300604in}}%
\pgfpathlineto{\pgfqpoint{4.685852in}{3.319889in}}%
\pgfpathlineto{\pgfqpoint{4.693193in}{3.339514in}}%
\pgfpathlineto{\pgfqpoint{4.680264in}{3.346556in}}%
\pgfpathlineto{\pgfqpoint{4.667340in}{3.353769in}}%
\pgfpathlineto{\pgfqpoint{4.654422in}{3.361151in}}%
\pgfpathlineto{\pgfqpoint{4.641509in}{3.368705in}}%
\pgfpathlineto{\pgfqpoint{4.634159in}{3.348438in}}%
\pgfpathlineto{\pgfqpoint{4.626812in}{3.328518in}}%
\pgfpathlineto{\pgfqpoint{4.619466in}{3.308937in}}%
\pgfpathlineto{\pgfqpoint{4.612123in}{3.289687in}}%
\pgfpathclose%
\pgfusepath{fill}%
\end{pgfscope}%
\begin{pgfscope}%
\pgfpathrectangle{\pgfqpoint{1.254980in}{0.150000in}}{\pgfqpoint{5.490039in}{5.490039in}}%
\pgfusepath{clip}%
\pgfsetbuttcap%
\pgfsetroundjoin%
\definecolor{currentfill}{rgb}{0.223925,0.334994,0.548053}%
\pgfsetfillcolor{currentfill}%
\pgfsetfillopacity{0.700000}%
\pgfsetlinewidth{0.000000pt}%
\definecolor{currentstroke}{rgb}{0.000000,0.000000,0.000000}%
\pgfsetstrokecolor{currentstroke}%
\pgfsetdash{}{0pt}%
\pgfpathmoveto{\pgfqpoint{3.993799in}{3.057308in}}%
\pgfpathlineto{\pgfqpoint{4.006632in}{3.048155in}}%
\pgfpathlineto{\pgfqpoint{4.019467in}{3.039201in}}%
\pgfpathlineto{\pgfqpoint{4.032305in}{3.030443in}}%
\pgfpathlineto{\pgfqpoint{4.045146in}{3.021881in}}%
\pgfpathlineto{\pgfqpoint{4.052567in}{3.037512in}}%
\pgfpathlineto{\pgfqpoint{4.059985in}{3.053349in}}%
\pgfpathlineto{\pgfqpoint{4.067400in}{3.069397in}}%
\pgfpathlineto{\pgfqpoint{4.074812in}{3.085662in}}%
\pgfpathlineto{\pgfqpoint{4.061977in}{3.094662in}}%
\pgfpathlineto{\pgfqpoint{4.049145in}{3.103859in}}%
\pgfpathlineto{\pgfqpoint{4.036316in}{3.113253in}}%
\pgfpathlineto{\pgfqpoint{4.023489in}{3.122845in}}%
\pgfpathlineto{\pgfqpoint{4.016071in}{3.106131in}}%
\pgfpathlineto{\pgfqpoint{4.008650in}{3.089640in}}%
\pgfpathlineto{\pgfqpoint{4.001226in}{3.073368in}}%
\pgfpathlineto{\pgfqpoint{3.993799in}{3.057308in}}%
\pgfpathclose%
\pgfusepath{fill}%
\end{pgfscope}%
\begin{pgfscope}%
\pgfpathrectangle{\pgfqpoint{1.254980in}{0.150000in}}{\pgfqpoint{5.490039in}{5.490039in}}%
\pgfusepath{clip}%
\pgfsetbuttcap%
\pgfsetroundjoin%
\definecolor{currentfill}{rgb}{0.124780,0.640461,0.527068}%
\pgfsetfillcolor{currentfill}%
\pgfsetfillopacity{0.700000}%
\pgfsetlinewidth{0.000000pt}%
\definecolor{currentstroke}{rgb}{0.000000,0.000000,0.000000}%
\pgfsetstrokecolor{currentstroke}%
\pgfsetdash{}{0pt}%
\pgfpathmoveto{\pgfqpoint{3.325906in}{3.843084in}}%
\pgfpathlineto{\pgfqpoint{3.338846in}{3.821403in}}%
\pgfpathlineto{\pgfqpoint{3.351779in}{3.800019in}}%
\pgfpathlineto{\pgfqpoint{3.364704in}{3.778929in}}%
\pgfpathlineto{\pgfqpoint{3.377623in}{3.758130in}}%
\pgfpathlineto{\pgfqpoint{3.385092in}{3.779440in}}%
\pgfpathlineto{\pgfqpoint{3.392556in}{3.801051in}}%
\pgfpathlineto{\pgfqpoint{3.400015in}{3.822969in}}%
\pgfpathlineto{\pgfqpoint{3.407469in}{3.845199in}}%
\pgfpathlineto{\pgfqpoint{3.394552in}{3.866461in}}%
\pgfpathlineto{\pgfqpoint{3.381627in}{3.888016in}}%
\pgfpathlineto{\pgfqpoint{3.368696in}{3.909865in}}%
\pgfpathlineto{\pgfqpoint{3.355757in}{3.932012in}}%
\pgfpathlineto{\pgfqpoint{3.348302in}{3.909305in}}%
\pgfpathlineto{\pgfqpoint{3.340842in}{3.886919in}}%
\pgfpathlineto{\pgfqpoint{3.333377in}{3.864847in}}%
\pgfpathlineto{\pgfqpoint{3.325906in}{3.843084in}}%
\pgfpathclose%
\pgfusepath{fill}%
\end{pgfscope}%
\begin{pgfscope}%
\pgfpathrectangle{\pgfqpoint{1.254980in}{0.150000in}}{\pgfqpoint{5.490039in}{5.490039in}}%
\pgfusepath{clip}%
\pgfsetbuttcap%
\pgfsetroundjoin%
\definecolor{currentfill}{rgb}{0.616293,0.852709,0.230052}%
\pgfsetfillcolor{currentfill}%
\pgfsetfillopacity{0.700000}%
\pgfsetlinewidth{0.000000pt}%
\definecolor{currentstroke}{rgb}{0.000000,0.000000,0.000000}%
\pgfsetstrokecolor{currentstroke}%
\pgfsetdash{}{0pt}%
\pgfpathmoveto{\pgfqpoint{3.585043in}{4.488557in}}%
\pgfpathlineto{\pgfqpoint{3.597969in}{4.464118in}}%
\pgfpathlineto{\pgfqpoint{3.610889in}{4.439974in}}%
\pgfpathlineto{\pgfqpoint{3.623801in}{4.416121in}}%
\pgfpathlineto{\pgfqpoint{3.636707in}{4.392557in}}%
\pgfpathlineto{\pgfqpoint{3.644068in}{4.424004in}}%
\pgfpathlineto{\pgfqpoint{3.651427in}{4.455935in}}%
\pgfpathlineto{\pgfqpoint{3.658784in}{4.488358in}}%
\pgfpathlineto{\pgfqpoint{3.666139in}{4.521280in}}%
\pgfpathlineto{\pgfqpoint{3.653226in}{4.545540in}}%
\pgfpathlineto{\pgfqpoint{3.640307in}{4.570090in}}%
\pgfpathlineto{\pgfqpoint{3.627380in}{4.594933in}}%
\pgfpathlineto{\pgfqpoint{3.614446in}{4.620071in}}%
\pgfpathlineto{\pgfqpoint{3.607099in}{4.586437in}}%
\pgfpathlineto{\pgfqpoint{3.599749in}{4.553312in}}%
\pgfpathlineto{\pgfqpoint{3.592397in}{4.520688in}}%
\pgfpathlineto{\pgfqpoint{3.585043in}{4.488557in}}%
\pgfpathclose%
\pgfusepath{fill}%
\end{pgfscope}%
\begin{pgfscope}%
\pgfpathrectangle{\pgfqpoint{1.254980in}{0.150000in}}{\pgfqpoint{5.490039in}{5.490039in}}%
\pgfusepath{clip}%
\pgfsetbuttcap%
\pgfsetroundjoin%
\definecolor{currentfill}{rgb}{0.218130,0.347432,0.550038}%
\pgfsetfillcolor{currentfill}%
\pgfsetfillopacity{0.700000}%
\pgfsetlinewidth{0.000000pt}%
\definecolor{currentstroke}{rgb}{0.000000,0.000000,0.000000}%
\pgfsetstrokecolor{currentstroke}%
\pgfsetdash{}{0pt}%
\pgfpathmoveto{\pgfqpoint{4.207184in}{3.084188in}}%
\pgfpathlineto{\pgfqpoint{4.220044in}{3.076440in}}%
\pgfpathlineto{\pgfqpoint{4.232908in}{3.068877in}}%
\pgfpathlineto{\pgfqpoint{4.245776in}{3.061499in}}%
\pgfpathlineto{\pgfqpoint{4.258649in}{3.054306in}}%
\pgfpathlineto{\pgfqpoint{4.266030in}{3.070284in}}%
\pgfpathlineto{\pgfqpoint{4.273409in}{3.086491in}}%
\pgfpathlineto{\pgfqpoint{4.280787in}{3.102932in}}%
\pgfpathlineto{\pgfqpoint{4.288162in}{3.119612in}}%
\pgfpathlineto{\pgfqpoint{4.275297in}{3.127299in}}%
\pgfpathlineto{\pgfqpoint{4.262435in}{3.135169in}}%
\pgfpathlineto{\pgfqpoint{4.249578in}{3.143225in}}%
\pgfpathlineto{\pgfqpoint{4.236724in}{3.151467in}}%
\pgfpathlineto{\pgfqpoint{4.229342in}{3.134283in}}%
\pgfpathlineto{\pgfqpoint{4.221958in}{3.117345in}}%
\pgfpathlineto{\pgfqpoint{4.214572in}{3.100649in}}%
\pgfpathlineto{\pgfqpoint{4.207184in}{3.084188in}}%
\pgfpathclose%
\pgfusepath{fill}%
\end{pgfscope}%
\begin{pgfscope}%
\pgfpathrectangle{\pgfqpoint{1.254980in}{0.150000in}}{\pgfqpoint{5.490039in}{5.490039in}}%
\pgfusepath{clip}%
\pgfsetbuttcap%
\pgfsetroundjoin%
\definecolor{currentfill}{rgb}{0.216210,0.351535,0.550627}%
\pgfsetfillcolor{currentfill}%
\pgfsetfillopacity{0.700000}%
\pgfsetlinewidth{0.000000pt}%
\definecolor{currentstroke}{rgb}{0.000000,0.000000,0.000000}%
\pgfsetstrokecolor{currentstroke}%
\pgfsetdash{}{0pt}%
\pgfpathmoveto{\pgfqpoint{3.728997in}{3.097543in}}%
\pgfpathlineto{\pgfqpoint{3.741819in}{3.085849in}}%
\pgfpathlineto{\pgfqpoint{3.754641in}{3.074374in}}%
\pgfpathlineto{\pgfqpoint{3.767463in}{3.063116in}}%
\pgfpathlineto{\pgfqpoint{3.780286in}{3.052074in}}%
\pgfpathlineto{\pgfqpoint{3.787759in}{3.067728in}}%
\pgfpathlineto{\pgfqpoint{3.795227in}{3.083577in}}%
\pgfpathlineto{\pgfqpoint{3.802691in}{3.099627in}}%
\pgfpathlineto{\pgfqpoint{3.810152in}{3.115881in}}%
\pgfpathlineto{\pgfqpoint{3.797334in}{3.127308in}}%
\pgfpathlineto{\pgfqpoint{3.784517in}{3.138952in}}%
\pgfpathlineto{\pgfqpoint{3.771700in}{3.150814in}}%
\pgfpathlineto{\pgfqpoint{3.758883in}{3.162894in}}%
\pgfpathlineto{\pgfqpoint{3.751418in}{3.146243in}}%
\pgfpathlineto{\pgfqpoint{3.743948in}{3.129804in}}%
\pgfpathlineto{\pgfqpoint{3.736475in}{3.113573in}}%
\pgfpathlineto{\pgfqpoint{3.728997in}{3.097543in}}%
\pgfpathclose%
\pgfusepath{fill}%
\end{pgfscope}%
\begin{pgfscope}%
\pgfpathrectangle{\pgfqpoint{1.254980in}{0.150000in}}{\pgfqpoint{5.490039in}{5.490039in}}%
\pgfusepath{clip}%
\pgfsetbuttcap%
\pgfsetroundjoin%
\definecolor{currentfill}{rgb}{0.146180,0.515413,0.556823}%
\pgfsetfillcolor{currentfill}%
\pgfsetfillopacity{0.700000}%
\pgfsetlinewidth{0.000000pt}%
\definecolor{currentstroke}{rgb}{0.000000,0.000000,0.000000}%
\pgfsetstrokecolor{currentstroke}%
\pgfsetdash{}{0pt}%
\pgfpathmoveto{\pgfqpoint{3.369307in}{3.520863in}}%
\pgfpathlineto{\pgfqpoint{3.382201in}{3.502307in}}%
\pgfpathlineto{\pgfqpoint{3.395089in}{3.484024in}}%
\pgfpathlineto{\pgfqpoint{3.407972in}{3.466014in}}%
\pgfpathlineto{\pgfqpoint{3.420851in}{3.448272in}}%
\pgfpathlineto{\pgfqpoint{3.428356in}{3.466581in}}%
\pgfpathlineto{\pgfqpoint{3.435856in}{3.485137in}}%
\pgfpathlineto{\pgfqpoint{3.443351in}{3.503943in}}%
\pgfpathlineto{\pgfqpoint{3.450841in}{3.523004in}}%
\pgfpathlineto{\pgfqpoint{3.437967in}{3.541142in}}%
\pgfpathlineto{\pgfqpoint{3.425087in}{3.559550in}}%
\pgfpathlineto{\pgfqpoint{3.412202in}{3.578230in}}%
\pgfpathlineto{\pgfqpoint{3.399312in}{3.597185in}}%
\pgfpathlineto{\pgfqpoint{3.391819in}{3.577714in}}%
\pgfpathlineto{\pgfqpoint{3.384320in}{3.558507in}}%
\pgfpathlineto{\pgfqpoint{3.376817in}{3.539558in}}%
\pgfpathlineto{\pgfqpoint{3.369307in}{3.520863in}}%
\pgfpathclose%
\pgfusepath{fill}%
\end{pgfscope}%
\begin{pgfscope}%
\pgfpathrectangle{\pgfqpoint{1.254980in}{0.150000in}}{\pgfqpoint{5.490039in}{5.490039in}}%
\pgfusepath{clip}%
\pgfsetbuttcap%
\pgfsetroundjoin%
\definecolor{currentfill}{rgb}{0.171176,0.452530,0.557965}%
\pgfsetfillcolor{currentfill}%
\pgfsetfillopacity{0.700000}%
\pgfsetlinewidth{0.000000pt}%
\definecolor{currentstroke}{rgb}{0.000000,0.000000,0.000000}%
\pgfsetstrokecolor{currentstroke}%
\pgfsetdash{}{0pt}%
\pgfpathmoveto{\pgfqpoint{4.693193in}{3.339514in}}%
\pgfpathlineto{\pgfqpoint{4.706129in}{3.332640in}}%
\pgfpathlineto{\pgfqpoint{4.719071in}{3.325935in}}%
\pgfpathlineto{\pgfqpoint{4.732019in}{3.319397in}}%
\pgfpathlineto{\pgfqpoint{4.744973in}{3.313027in}}%
\pgfpathlineto{\pgfqpoint{4.752306in}{3.332349in}}%
\pgfpathlineto{\pgfqpoint{4.759643in}{3.352019in}}%
\pgfpathlineto{\pgfqpoint{4.766983in}{3.372044in}}%
\pgfpathlineto{\pgfqpoint{4.754036in}{3.378906in}}%
\pgfpathlineto{\pgfqpoint{4.741095in}{3.385936in}}%
\pgfpathlineto{\pgfqpoint{4.728160in}{3.393133in}}%
\pgfpathlineto{\pgfqpoint{4.715232in}{3.400499in}}%
\pgfpathlineto{\pgfqpoint{4.707883in}{3.379811in}}%
\pgfpathlineto{\pgfqpoint{4.700537in}{3.359485in}}%
\pgfpathlineto{\pgfqpoint{4.693193in}{3.339514in}}%
\pgfpathclose%
\pgfusepath{fill}%
\end{pgfscope}%
\begin{pgfscope}%
\pgfpathrectangle{\pgfqpoint{1.254980in}{0.150000in}}{\pgfqpoint{5.490039in}{5.490039in}}%
\pgfusepath{clip}%
\pgfsetbuttcap%
\pgfsetroundjoin%
\definecolor{currentfill}{rgb}{0.223925,0.334994,0.548053}%
\pgfsetfillcolor{currentfill}%
\pgfsetfillopacity{0.700000}%
\pgfsetlinewidth{0.000000pt}%
\definecolor{currentstroke}{rgb}{0.000000,0.000000,0.000000}%
\pgfsetstrokecolor{currentstroke}%
\pgfsetdash{}{0pt}%
\pgfpathmoveto{\pgfqpoint{4.126183in}{3.051598in}}%
\pgfpathlineto{\pgfqpoint{4.139034in}{3.043562in}}%
\pgfpathlineto{\pgfqpoint{4.151889in}{3.035715in}}%
\pgfpathlineto{\pgfqpoint{4.164747in}{3.028057in}}%
\pgfpathlineto{\pgfqpoint{4.177610in}{3.020587in}}%
\pgfpathlineto{\pgfqpoint{4.185007in}{3.036162in}}%
\pgfpathlineto{\pgfqpoint{4.192401in}{3.051950in}}%
\pgfpathlineto{\pgfqpoint{4.199794in}{3.067957in}}%
\pgfpathlineto{\pgfqpoint{4.207184in}{3.084188in}}%
\pgfpathlineto{\pgfqpoint{4.194328in}{3.092124in}}%
\pgfpathlineto{\pgfqpoint{4.181476in}{3.100248in}}%
\pgfpathlineto{\pgfqpoint{4.168627in}{3.108560in}}%
\pgfpathlineto{\pgfqpoint{4.155782in}{3.117063in}}%
\pgfpathlineto{\pgfqpoint{4.148386in}{3.100355in}}%
\pgfpathlineto{\pgfqpoint{4.140987in}{3.083879in}}%
\pgfpathlineto{\pgfqpoint{4.133586in}{3.067629in}}%
\pgfpathlineto{\pgfqpoint{4.126183in}{3.051598in}}%
\pgfpathclose%
\pgfusepath{fill}%
\end{pgfscope}%
\begin{pgfscope}%
\pgfpathrectangle{\pgfqpoint{1.254980in}{0.150000in}}{\pgfqpoint{5.490039in}{5.490039in}}%
\pgfusepath{clip}%
\pgfsetbuttcap%
\pgfsetroundjoin%
\definecolor{currentfill}{rgb}{0.188923,0.410910,0.556326}%
\pgfsetfillcolor{currentfill}%
\pgfsetfillopacity{0.700000}%
\pgfsetlinewidth{0.000000pt}%
\definecolor{currentstroke}{rgb}{0.000000,0.000000,0.000000}%
\pgfsetstrokecolor{currentstroke}%
\pgfsetdash{}{0pt}%
\pgfpathmoveto{\pgfqpoint{3.493688in}{3.247796in}}%
\pgfpathlineto{\pgfqpoint{3.506537in}{3.232736in}}%
\pgfpathlineto{\pgfqpoint{3.519383in}{3.217922in}}%
\pgfpathlineto{\pgfqpoint{3.532226in}{3.203353in}}%
\pgfpathlineto{\pgfqpoint{3.545067in}{3.189026in}}%
\pgfpathlineto{\pgfqpoint{3.552578in}{3.205337in}}%
\pgfpathlineto{\pgfqpoint{3.560085in}{3.221855in}}%
\pgfpathlineto{\pgfqpoint{3.567587in}{3.238583in}}%
\pgfpathlineto{\pgfqpoint{3.575084in}{3.255526in}}%
\pgfpathlineto{\pgfqpoint{3.562248in}{3.270214in}}%
\pgfpathlineto{\pgfqpoint{3.549410in}{3.285146in}}%
\pgfpathlineto{\pgfqpoint{3.536569in}{3.300322in}}%
\pgfpathlineto{\pgfqpoint{3.523726in}{3.315745in}}%
\pgfpathlineto{\pgfqpoint{3.516224in}{3.298429in}}%
\pgfpathlineto{\pgfqpoint{3.508717in}{3.281335in}}%
\pgfpathlineto{\pgfqpoint{3.501205in}{3.264459in}}%
\pgfpathlineto{\pgfqpoint{3.493688in}{3.247796in}}%
\pgfpathclose%
\pgfusepath{fill}%
\end{pgfscope}%
\begin{pgfscope}%
\pgfpathrectangle{\pgfqpoint{1.254980in}{0.150000in}}{\pgfqpoint{5.490039in}{5.490039in}}%
\pgfusepath{clip}%
\pgfsetbuttcap%
\pgfsetroundjoin%
\definecolor{currentfill}{rgb}{0.259857,0.745492,0.444467}%
\pgfsetfillcolor{currentfill}%
\pgfsetfillopacity{0.700000}%
\pgfsetlinewidth{0.000000pt}%
\definecolor{currentstroke}{rgb}{0.000000,0.000000,0.000000}%
\pgfsetstrokecolor{currentstroke}%
\pgfsetdash{}{0pt}%
\pgfpathmoveto{\pgfqpoint{3.333686in}{4.119791in}}%
\pgfpathlineto{\pgfqpoint{3.346658in}{4.095919in}}%
\pgfpathlineto{\pgfqpoint{3.359621in}{4.072358in}}%
\pgfpathlineto{\pgfqpoint{3.372576in}{4.049105in}}%
\pgfpathlineto{\pgfqpoint{3.385523in}{4.026158in}}%
\pgfpathlineto{\pgfqpoint{3.392952in}{4.050553in}}%
\pgfpathlineto{\pgfqpoint{3.400376in}{4.075304in}}%
\pgfpathlineto{\pgfqpoint{3.407794in}{4.100417in}}%
\pgfpathlineto{\pgfqpoint{3.415209in}{4.125899in}}%
\pgfpathlineto{\pgfqpoint{3.402261in}{4.149379in}}%
\pgfpathlineto{\pgfqpoint{3.389304in}{4.173165in}}%
\pgfpathlineto{\pgfqpoint{3.376340in}{4.197260in}}%
\pgfpathlineto{\pgfqpoint{3.363366in}{4.221668in}}%
\pgfpathlineto{\pgfqpoint{3.355954in}{4.195640in}}%
\pgfpathlineto{\pgfqpoint{3.348537in}{4.169989in}}%
\pgfpathlineto{\pgfqpoint{3.341114in}{4.144708in}}%
\pgfpathlineto{\pgfqpoint{3.333686in}{4.119791in}}%
\pgfpathclose%
\pgfusepath{fill}%
\end{pgfscope}%
\begin{pgfscope}%
\pgfpathrectangle{\pgfqpoint{1.254980in}{0.150000in}}{\pgfqpoint{5.490039in}{5.490039in}}%
\pgfusepath{clip}%
\pgfsetbuttcap%
\pgfsetroundjoin%
\definecolor{currentfill}{rgb}{0.199430,0.387607,0.554642}%
\pgfsetfillcolor{currentfill}%
\pgfsetfillopacity{0.700000}%
\pgfsetlinewidth{0.000000pt}%
\definecolor{currentstroke}{rgb}{0.000000,0.000000,0.000000}%
\pgfsetstrokecolor{currentstroke}%
\pgfsetdash{}{0pt}%
\pgfpathmoveto{\pgfqpoint{3.545067in}{3.189026in}}%
\pgfpathlineto{\pgfqpoint{3.557906in}{3.174940in}}%
\pgfpathlineto{\pgfqpoint{3.570742in}{3.161093in}}%
\pgfpathlineto{\pgfqpoint{3.583577in}{3.147483in}}%
\pgfpathlineto{\pgfqpoint{3.596411in}{3.134108in}}%
\pgfpathlineto{\pgfqpoint{3.603917in}{3.150069in}}%
\pgfpathlineto{\pgfqpoint{3.611418in}{3.166228in}}%
\pgfpathlineto{\pgfqpoint{3.618915in}{3.182592in}}%
\pgfpathlineto{\pgfqpoint{3.626407in}{3.199163in}}%
\pgfpathlineto{\pgfqpoint{3.613579in}{3.212899in}}%
\pgfpathlineto{\pgfqpoint{3.600749in}{3.226870in}}%
\pgfpathlineto{\pgfqpoint{3.587918in}{3.241078in}}%
\pgfpathlineto{\pgfqpoint{3.575084in}{3.255526in}}%
\pgfpathlineto{\pgfqpoint{3.567587in}{3.238583in}}%
\pgfpathlineto{\pgfqpoint{3.560085in}{3.221855in}}%
\pgfpathlineto{\pgfqpoint{3.552578in}{3.205337in}}%
\pgfpathlineto{\pgfqpoint{3.545067in}{3.189026in}}%
\pgfpathclose%
\pgfusepath{fill}%
\end{pgfscope}%
\begin{pgfscope}%
\pgfpathrectangle{\pgfqpoint{1.254980in}{0.150000in}}{\pgfqpoint{5.490039in}{5.490039in}}%
\pgfusepath{clip}%
\pgfsetbuttcap%
\pgfsetroundjoin%
\definecolor{currentfill}{rgb}{0.177423,0.437527,0.557565}%
\pgfsetfillcolor{currentfill}%
\pgfsetfillopacity{0.700000}%
\pgfsetlinewidth{0.000000pt}%
\definecolor{currentstroke}{rgb}{0.000000,0.000000,0.000000}%
\pgfsetstrokecolor{currentstroke}%
\pgfsetdash{}{0pt}%
\pgfpathmoveto{\pgfqpoint{3.442263in}{3.310543in}}%
\pgfpathlineto{\pgfqpoint{3.455124in}{3.294476in}}%
\pgfpathlineto{\pgfqpoint{3.467982in}{3.278664in}}%
\pgfpathlineto{\pgfqpoint{3.480837in}{3.263105in}}%
\pgfpathlineto{\pgfqpoint{3.493688in}{3.247796in}}%
\pgfpathlineto{\pgfqpoint{3.501205in}{3.264459in}}%
\pgfpathlineto{\pgfqpoint{3.508717in}{3.281335in}}%
\pgfpathlineto{\pgfqpoint{3.516224in}{3.298429in}}%
\pgfpathlineto{\pgfqpoint{3.523726in}{3.315745in}}%
\pgfpathlineto{\pgfqpoint{3.510879in}{3.331417in}}%
\pgfpathlineto{\pgfqpoint{3.498029in}{3.347341in}}%
\pgfpathlineto{\pgfqpoint{3.485176in}{3.363517in}}%
\pgfpathlineto{\pgfqpoint{3.472319in}{3.379949in}}%
\pgfpathlineto{\pgfqpoint{3.464813in}{3.362257in}}%
\pgfpathlineto{\pgfqpoint{3.457301in}{3.344796in}}%
\pgfpathlineto{\pgfqpoint{3.449784in}{3.327559in}}%
\pgfpathlineto{\pgfqpoint{3.442263in}{3.310543in}}%
\pgfpathclose%
\pgfusepath{fill}%
\end{pgfscope}%
\begin{pgfscope}%
\pgfpathrectangle{\pgfqpoint{1.254980in}{0.150000in}}{\pgfqpoint{5.490039in}{5.490039in}}%
\pgfusepath{clip}%
\pgfsetbuttcap%
\pgfsetroundjoin%
\definecolor{currentfill}{rgb}{0.344074,0.780029,0.397381}%
\pgfsetfillcolor{currentfill}%
\pgfsetfillopacity{0.700000}%
\pgfsetlinewidth{0.000000pt}%
\definecolor{currentstroke}{rgb}{0.000000,0.000000,0.000000}%
\pgfsetstrokecolor{currentstroke}%
\pgfsetdash{}{0pt}%
\pgfpathmoveto{\pgfqpoint{3.363366in}{4.221668in}}%
\pgfpathlineto{\pgfqpoint{3.376340in}{4.197260in}}%
\pgfpathlineto{\pgfqpoint{3.389304in}{4.173165in}}%
\pgfpathlineto{\pgfqpoint{3.402261in}{4.149379in}}%
\pgfpathlineto{\pgfqpoint{3.415209in}{4.125899in}}%
\pgfpathlineto{\pgfqpoint{3.422618in}{4.151755in}}%
\pgfpathlineto{\pgfqpoint{3.430023in}{4.177993in}}%
\pgfpathlineto{\pgfqpoint{3.437423in}{4.204619in}}%
\pgfpathlineto{\pgfqpoint{3.444819in}{4.231640in}}%
\pgfpathlineto{\pgfqpoint{3.431868in}{4.255685in}}%
\pgfpathlineto{\pgfqpoint{3.418910in}{4.280039in}}%
\pgfpathlineto{\pgfqpoint{3.405942in}{4.304703in}}%
\pgfpathlineto{\pgfqpoint{3.392966in}{4.329680in}}%
\pgfpathlineto{\pgfqpoint{3.385573in}{4.302079in}}%
\pgfpathlineto{\pgfqpoint{3.378176in}{4.274881in}}%
\pgfpathlineto{\pgfqpoint{3.370774in}{4.248080in}}%
\pgfpathlineto{\pgfqpoint{3.363366in}{4.221668in}}%
\pgfpathclose%
\pgfusepath{fill}%
\end{pgfscope}%
\begin{pgfscope}%
\pgfpathrectangle{\pgfqpoint{1.254980in}{0.150000in}}{\pgfqpoint{5.490039in}{5.490039in}}%
\pgfusepath{clip}%
\pgfsetbuttcap%
\pgfsetroundjoin%
\definecolor{currentfill}{rgb}{0.229739,0.322361,0.545706}%
\pgfsetfillcolor{currentfill}%
\pgfsetfillopacity{0.700000}%
\pgfsetlinewidth{0.000000pt}%
\definecolor{currentstroke}{rgb}{0.000000,0.000000,0.000000}%
\pgfsetstrokecolor{currentstroke}%
\pgfsetdash{}{0pt}%
\pgfpathmoveto{\pgfqpoint{3.912730in}{3.032069in}}%
\pgfpathlineto{\pgfqpoint{3.925559in}{3.022524in}}%
\pgfpathlineto{\pgfqpoint{3.938390in}{3.013182in}}%
\pgfpathlineto{\pgfqpoint{3.951224in}{3.004041in}}%
\pgfpathlineto{\pgfqpoint{3.964060in}{2.995100in}}%
\pgfpathlineto{\pgfqpoint{3.971500in}{3.010358in}}%
\pgfpathlineto{\pgfqpoint{3.978936in}{3.025808in}}%
\pgfpathlineto{\pgfqpoint{3.986370in}{3.041457in}}%
\pgfpathlineto{\pgfqpoint{3.993799in}{3.057308in}}%
\pgfpathlineto{\pgfqpoint{3.980969in}{3.066660in}}%
\pgfpathlineto{\pgfqpoint{3.968142in}{3.076213in}}%
\pgfpathlineto{\pgfqpoint{3.955316in}{3.085967in}}%
\pgfpathlineto{\pgfqpoint{3.942492in}{3.095924in}}%
\pgfpathlineto{\pgfqpoint{3.935057in}{3.079651in}}%
\pgfpathlineto{\pgfqpoint{3.927618in}{3.063587in}}%
\pgfpathlineto{\pgfqpoint{3.920175in}{3.047728in}}%
\pgfpathlineto{\pgfqpoint{3.912730in}{3.032069in}}%
\pgfpathclose%
\pgfusepath{fill}%
\end{pgfscope}%
\begin{pgfscope}%
\pgfpathrectangle{\pgfqpoint{1.254980in}{0.150000in}}{\pgfqpoint{5.490039in}{5.490039in}}%
\pgfusepath{clip}%
\pgfsetbuttcap%
\pgfsetroundjoin%
\definecolor{currentfill}{rgb}{0.585678,0.846661,0.249897}%
\pgfsetfillcolor{currentfill}%
\pgfsetfillopacity{0.700000}%
\pgfsetlinewidth{0.000000pt}%
\definecolor{currentstroke}{rgb}{0.000000,0.000000,0.000000}%
\pgfsetstrokecolor{currentstroke}%
\pgfsetdash{}{0pt}%
\pgfpathmoveto{\pgfqpoint{3.503838in}{4.462872in}}%
\pgfpathlineto{\pgfqpoint{3.516789in}{4.437898in}}%
\pgfpathlineto{\pgfqpoint{3.529733in}{4.413228in}}%
\pgfpathlineto{\pgfqpoint{3.542668in}{4.388860in}}%
\pgfpathlineto{\pgfqpoint{3.555596in}{4.364790in}}%
\pgfpathlineto{\pgfqpoint{3.562962in}{4.395034in}}%
\pgfpathlineto{\pgfqpoint{3.570325in}{4.425738in}}%
\pgfpathlineto{\pgfqpoint{3.577685in}{4.456909in}}%
\pgfpathlineto{\pgfqpoint{3.585043in}{4.488557in}}%
\pgfpathlineto{\pgfqpoint{3.572109in}{4.513292in}}%
\pgfpathlineto{\pgfqpoint{3.559167in}{4.538327in}}%
\pgfpathlineto{\pgfqpoint{3.546217in}{4.563666in}}%
\pgfpathlineto{\pgfqpoint{3.533259in}{4.589310in}}%
\pgfpathlineto{\pgfqpoint{3.525909in}{4.556981in}}%
\pgfpathlineto{\pgfqpoint{3.518555in}{4.525137in}}%
\pgfpathlineto{\pgfqpoint{3.511198in}{4.493770in}}%
\pgfpathlineto{\pgfqpoint{3.503838in}{4.462872in}}%
\pgfpathclose%
\pgfusepath{fill}%
\end{pgfscope}%
\begin{pgfscope}%
\pgfpathrectangle{\pgfqpoint{1.254980in}{0.150000in}}{\pgfqpoint{5.490039in}{5.490039in}}%
\pgfusepath{clip}%
\pgfsetbuttcap%
\pgfsetroundjoin%
\definecolor{currentfill}{rgb}{0.225863,0.330805,0.547314}%
\pgfsetfillcolor{currentfill}%
\pgfsetfillopacity{0.700000}%
\pgfsetlinewidth{0.000000pt}%
\definecolor{currentstroke}{rgb}{0.000000,0.000000,0.000000}%
\pgfsetstrokecolor{currentstroke}%
\pgfsetdash{}{0pt}%
\pgfpathmoveto{\pgfqpoint{3.780286in}{3.052074in}}%
\pgfpathlineto{\pgfqpoint{3.793110in}{3.041247in}}%
\pgfpathlineto{\pgfqpoint{3.805935in}{3.030632in}}%
\pgfpathlineto{\pgfqpoint{3.818760in}{3.020229in}}%
\pgfpathlineto{\pgfqpoint{3.831587in}{3.010037in}}%
\pgfpathlineto{\pgfqpoint{3.839054in}{3.025316in}}%
\pgfpathlineto{\pgfqpoint{3.846516in}{3.040783in}}%
\pgfpathlineto{\pgfqpoint{3.853975in}{3.056444in}}%
\pgfpathlineto{\pgfqpoint{3.861430in}{3.072302in}}%
\pgfpathlineto{\pgfqpoint{3.848609in}{3.082880in}}%
\pgfpathlineto{\pgfqpoint{3.835789in}{3.093668in}}%
\pgfpathlineto{\pgfqpoint{3.822970in}{3.104667in}}%
\pgfpathlineto{\pgfqpoint{3.810152in}{3.115881in}}%
\pgfpathlineto{\pgfqpoint{3.802691in}{3.099627in}}%
\pgfpathlineto{\pgfqpoint{3.795227in}{3.083577in}}%
\pgfpathlineto{\pgfqpoint{3.787759in}{3.067728in}}%
\pgfpathlineto{\pgfqpoint{3.780286in}{3.052074in}}%
\pgfpathclose%
\pgfusepath{fill}%
\end{pgfscope}%
\begin{pgfscope}%
\pgfpathrectangle{\pgfqpoint{1.254980in}{0.150000in}}{\pgfqpoint{5.490039in}{5.490039in}}%
\pgfusepath{clip}%
\pgfsetbuttcap%
\pgfsetroundjoin%
\definecolor{currentfill}{rgb}{0.119423,0.611141,0.538982}%
\pgfsetfillcolor{currentfill}%
\pgfsetfillopacity{0.700000}%
\pgfsetlinewidth{0.000000pt}%
\definecolor{currentstroke}{rgb}{0.000000,0.000000,0.000000}%
\pgfsetstrokecolor{currentstroke}%
\pgfsetdash{}{0pt}%
\pgfpathmoveto{\pgfqpoint{3.295966in}{3.759019in}}%
\pgfpathlineto{\pgfqpoint{3.308909in}{3.737773in}}%
\pgfpathlineto{\pgfqpoint{3.321844in}{3.716823in}}%
\pgfpathlineto{\pgfqpoint{3.334771in}{3.696166in}}%
\pgfpathlineto{\pgfqpoint{3.347692in}{3.675800in}}%
\pgfpathlineto{\pgfqpoint{3.355183in}{3.695956in}}%
\pgfpathlineto{\pgfqpoint{3.362668in}{3.716394in}}%
\pgfpathlineto{\pgfqpoint{3.370148in}{3.737117in}}%
\pgfpathlineto{\pgfqpoint{3.377623in}{3.758130in}}%
\pgfpathlineto{\pgfqpoint{3.364704in}{3.778929in}}%
\pgfpathlineto{\pgfqpoint{3.351779in}{3.800019in}}%
\pgfpathlineto{\pgfqpoint{3.338846in}{3.821403in}}%
\pgfpathlineto{\pgfqpoint{3.325906in}{3.843084in}}%
\pgfpathlineto{\pgfqpoint{3.318430in}{3.821625in}}%
\pgfpathlineto{\pgfqpoint{3.310948in}{3.800465in}}%
\pgfpathlineto{\pgfqpoint{3.303460in}{3.779598in}}%
\pgfpathlineto{\pgfqpoint{3.295966in}{3.759019in}}%
\pgfpathclose%
\pgfusepath{fill}%
\end{pgfscope}%
\begin{pgfscope}%
\pgfpathrectangle{\pgfqpoint{1.254980in}{0.150000in}}{\pgfqpoint{5.490039in}{5.490039in}}%
\pgfusepath{clip}%
\pgfsetbuttcap%
\pgfsetroundjoin%
\definecolor{currentfill}{rgb}{0.133743,0.548535,0.553541}%
\pgfsetfillcolor{currentfill}%
\pgfsetfillopacity{0.700000}%
\pgfsetlinewidth{0.000000pt}%
\definecolor{currentstroke}{rgb}{0.000000,0.000000,0.000000}%
\pgfsetstrokecolor{currentstroke}%
\pgfsetdash{}{0pt}%
\pgfpathmoveto{\pgfqpoint{3.317673in}{3.597878in}}%
\pgfpathlineto{\pgfqpoint{3.330591in}{3.578200in}}%
\pgfpathlineto{\pgfqpoint{3.343502in}{3.558807in}}%
\pgfpathlineto{\pgfqpoint{3.356408in}{3.539695in}}%
\pgfpathlineto{\pgfqpoint{3.369307in}{3.520863in}}%
\pgfpathlineto{\pgfqpoint{3.376817in}{3.539558in}}%
\pgfpathlineto{\pgfqpoint{3.384320in}{3.558507in}}%
\pgfpathlineto{\pgfqpoint{3.391819in}{3.577714in}}%
\pgfpathlineto{\pgfqpoint{3.399312in}{3.597185in}}%
\pgfpathlineto{\pgfqpoint{3.386416in}{3.616416in}}%
\pgfpathlineto{\pgfqpoint{3.373515in}{3.635927in}}%
\pgfpathlineto{\pgfqpoint{3.360607in}{3.655721in}}%
\pgfpathlineto{\pgfqpoint{3.347692in}{3.675800in}}%
\pgfpathlineto{\pgfqpoint{3.340196in}{3.655918in}}%
\pgfpathlineto{\pgfqpoint{3.332694in}{3.636307in}}%
\pgfpathlineto{\pgfqpoint{3.325187in}{3.616962in}}%
\pgfpathlineto{\pgfqpoint{3.317673in}{3.597878in}}%
\pgfpathclose%
\pgfusepath{fill}%
\end{pgfscope}%
\begin{pgfscope}%
\pgfpathrectangle{\pgfqpoint{1.254980in}{0.150000in}}{\pgfqpoint{5.490039in}{5.490039in}}%
\pgfusepath{clip}%
\pgfsetbuttcap%
\pgfsetroundjoin%
\definecolor{currentfill}{rgb}{0.196571,0.711827,0.479221}%
\pgfsetfillcolor{currentfill}%
\pgfsetfillopacity{0.700000}%
\pgfsetlinewidth{0.000000pt}%
\definecolor{currentstroke}{rgb}{0.000000,0.000000,0.000000}%
\pgfsetstrokecolor{currentstroke}%
\pgfsetdash{}{0pt}%
\pgfpathmoveto{\pgfqpoint{3.303921in}{4.023641in}}%
\pgfpathlineto{\pgfqpoint{3.316893in}{4.000272in}}%
\pgfpathlineto{\pgfqpoint{3.329856in}{3.977213in}}%
\pgfpathlineto{\pgfqpoint{3.342810in}{3.954460in}}%
\pgfpathlineto{\pgfqpoint{3.355757in}{3.932012in}}%
\pgfpathlineto{\pgfqpoint{3.363206in}{3.955045in}}%
\pgfpathlineto{\pgfqpoint{3.370650in}{3.978410in}}%
\pgfpathlineto{\pgfqpoint{3.378089in}{4.002112in}}%
\pgfpathlineto{\pgfqpoint{3.385523in}{4.026158in}}%
\pgfpathlineto{\pgfqpoint{3.372576in}{4.049105in}}%
\pgfpathlineto{\pgfqpoint{3.359621in}{4.072358in}}%
\pgfpathlineto{\pgfqpoint{3.346658in}{4.095919in}}%
\pgfpathlineto{\pgfqpoint{3.333686in}{4.119791in}}%
\pgfpathlineto{\pgfqpoint{3.326253in}{4.095232in}}%
\pgfpathlineto{\pgfqpoint{3.318815in}{4.071024in}}%
\pgfpathlineto{\pgfqpoint{3.311371in}{4.047163in}}%
\pgfpathlineto{\pgfqpoint{3.303921in}{4.023641in}}%
\pgfpathclose%
\pgfusepath{fill}%
\end{pgfscope}%
\begin{pgfscope}%
\pgfpathrectangle{\pgfqpoint{1.254980in}{0.150000in}}{\pgfqpoint{5.490039in}{5.490039in}}%
\pgfusepath{clip}%
\pgfsetbuttcap%
\pgfsetroundjoin%
\definecolor{currentfill}{rgb}{0.208623,0.367752,0.552675}%
\pgfsetfillcolor{currentfill}%
\pgfsetfillopacity{0.700000}%
\pgfsetlinewidth{0.000000pt}%
\definecolor{currentstroke}{rgb}{0.000000,0.000000,0.000000}%
\pgfsetstrokecolor{currentstroke}%
\pgfsetdash{}{0pt}%
\pgfpathmoveto{\pgfqpoint{3.596411in}{3.134108in}}%
\pgfpathlineto{\pgfqpoint{3.609243in}{3.120966in}}%
\pgfpathlineto{\pgfqpoint{3.622074in}{3.108056in}}%
\pgfpathlineto{\pgfqpoint{3.634903in}{3.095376in}}%
\pgfpathlineto{\pgfqpoint{3.647732in}{3.082923in}}%
\pgfpathlineto{\pgfqpoint{3.655233in}{3.098535in}}%
\pgfpathlineto{\pgfqpoint{3.662729in}{3.114338in}}%
\pgfpathlineto{\pgfqpoint{3.670221in}{3.130338in}}%
\pgfpathlineto{\pgfqpoint{3.677708in}{3.146539in}}%
\pgfpathlineto{\pgfqpoint{3.664884in}{3.159351in}}%
\pgfpathlineto{\pgfqpoint{3.652060in}{3.172391in}}%
\pgfpathlineto{\pgfqpoint{3.639234in}{3.185661in}}%
\pgfpathlineto{\pgfqpoint{3.626407in}{3.199163in}}%
\pgfpathlineto{\pgfqpoint{3.618915in}{3.182592in}}%
\pgfpathlineto{\pgfqpoint{3.611418in}{3.166228in}}%
\pgfpathlineto{\pgfqpoint{3.603917in}{3.150069in}}%
\pgfpathlineto{\pgfqpoint{3.596411in}{3.134108in}}%
\pgfpathclose%
\pgfusepath{fill}%
\end{pgfscope}%
\begin{pgfscope}%
\pgfpathrectangle{\pgfqpoint{1.254980in}{0.150000in}}{\pgfqpoint{5.490039in}{5.490039in}}%
\pgfusepath{clip}%
\pgfsetbuttcap%
\pgfsetroundjoin%
\definecolor{currentfill}{rgb}{0.166617,0.463708,0.558119}%
\pgfsetfillcolor{currentfill}%
\pgfsetfillopacity{0.700000}%
\pgfsetlinewidth{0.000000pt}%
\definecolor{currentstroke}{rgb}{0.000000,0.000000,0.000000}%
\pgfsetstrokecolor{currentstroke}%
\pgfsetdash{}{0pt}%
\pgfpathmoveto{\pgfqpoint{3.390776in}{3.377404in}}%
\pgfpathlineto{\pgfqpoint{3.403654in}{3.360295in}}%
\pgfpathlineto{\pgfqpoint{3.416528in}{3.343451in}}%
\pgfpathlineto{\pgfqpoint{3.429397in}{3.326867in}}%
\pgfpathlineto{\pgfqpoint{3.442263in}{3.310543in}}%
\pgfpathlineto{\pgfqpoint{3.449784in}{3.327559in}}%
\pgfpathlineto{\pgfqpoint{3.457301in}{3.344796in}}%
\pgfpathlineto{\pgfqpoint{3.464813in}{3.362257in}}%
\pgfpathlineto{\pgfqpoint{3.472319in}{3.379949in}}%
\pgfpathlineto{\pgfqpoint{3.459458in}{3.396638in}}%
\pgfpathlineto{\pgfqpoint{3.446593in}{3.413586in}}%
\pgfpathlineto{\pgfqpoint{3.433724in}{3.430797in}}%
\pgfpathlineto{\pgfqpoint{3.420851in}{3.448272in}}%
\pgfpathlineto{\pgfqpoint{3.413340in}{3.430204in}}%
\pgfpathlineto{\pgfqpoint{3.405824in}{3.412373in}}%
\pgfpathlineto{\pgfqpoint{3.398303in}{3.394775in}}%
\pgfpathlineto{\pgfqpoint{3.390776in}{3.377404in}}%
\pgfpathclose%
\pgfusepath{fill}%
\end{pgfscope}%
\begin{pgfscope}%
\pgfpathrectangle{\pgfqpoint{1.254980in}{0.150000in}}{\pgfqpoint{5.490039in}{5.490039in}}%
\pgfusepath{clip}%
\pgfsetbuttcap%
\pgfsetroundjoin%
\definecolor{currentfill}{rgb}{0.449368,0.813768,0.335384}%
\pgfsetfillcolor{currentfill}%
\pgfsetfillopacity{0.700000}%
\pgfsetlinewidth{0.000000pt}%
\definecolor{currentstroke}{rgb}{0.000000,0.000000,0.000000}%
\pgfsetstrokecolor{currentstroke}%
\pgfsetdash{}{0pt}%
\pgfpathmoveto{\pgfqpoint{3.392966in}{4.329680in}}%
\pgfpathlineto{\pgfqpoint{3.405942in}{4.304703in}}%
\pgfpathlineto{\pgfqpoint{3.418910in}{4.280039in}}%
\pgfpathlineto{\pgfqpoint{3.431868in}{4.255685in}}%
\pgfpathlineto{\pgfqpoint{3.444819in}{4.231640in}}%
\pgfpathlineto{\pgfqpoint{3.452211in}{4.259062in}}%
\pgfpathlineto{\pgfqpoint{3.459598in}{4.286894in}}%
\pgfpathlineto{\pgfqpoint{3.466981in}{4.315141in}}%
\pgfpathlineto{\pgfqpoint{3.474360in}{4.343811in}}%
\pgfpathlineto{\pgfqpoint{3.461406in}{4.368457in}}%
\pgfpathlineto{\pgfqpoint{3.448443in}{4.393412in}}%
\pgfpathlineto{\pgfqpoint{3.435471in}{4.418679in}}%
\pgfpathlineto{\pgfqpoint{3.422491in}{4.444262in}}%
\pgfpathlineto{\pgfqpoint{3.415116in}{4.414976in}}%
\pgfpathlineto{\pgfqpoint{3.407738in}{4.386122in}}%
\pgfpathlineto{\pgfqpoint{3.400354in}{4.357692in}}%
\pgfpathlineto{\pgfqpoint{3.392966in}{4.329680in}}%
\pgfpathclose%
\pgfusepath{fill}%
\end{pgfscope}%
\begin{pgfscope}%
\pgfpathrectangle{\pgfqpoint{1.254980in}{0.150000in}}{\pgfqpoint{5.490039in}{5.490039in}}%
\pgfusepath{clip}%
\pgfsetbuttcap%
\pgfsetroundjoin%
\definecolor{currentfill}{rgb}{0.199430,0.387607,0.554642}%
\pgfsetfillcolor{currentfill}%
\pgfsetfillopacity{0.700000}%
\pgfsetlinewidth{0.000000pt}%
\definecolor{currentstroke}{rgb}{0.000000,0.000000,0.000000}%
\pgfsetstrokecolor{currentstroke}%
\pgfsetdash{}{0pt}%
\pgfpathmoveto{\pgfqpoint{4.501714in}{3.171445in}}%
\pgfpathlineto{\pgfqpoint{4.514630in}{3.165012in}}%
\pgfpathlineto{\pgfqpoint{4.527551in}{3.158753in}}%
\pgfpathlineto{\pgfqpoint{4.540478in}{3.152666in}}%
\pgfpathlineto{\pgfqpoint{4.553411in}{3.146751in}}%
\pgfpathlineto{\pgfqpoint{4.560748in}{3.163608in}}%
\pgfpathlineto{\pgfqpoint{4.568086in}{3.180739in}}%
\pgfpathlineto{\pgfqpoint{4.575423in}{3.198152in}}%
\pgfpathlineto{\pgfqpoint{4.582761in}{3.215854in}}%
\pgfpathlineto{\pgfqpoint{4.569837in}{3.222344in}}%
\pgfpathlineto{\pgfqpoint{4.556918in}{3.229006in}}%
\pgfpathlineto{\pgfqpoint{4.544006in}{3.235841in}}%
\pgfpathlineto{\pgfqpoint{4.531099in}{3.242849in}}%
\pgfpathlineto{\pgfqpoint{4.523752in}{3.224561in}}%
\pgfpathlineto{\pgfqpoint{4.516406in}{3.206569in}}%
\pgfpathlineto{\pgfqpoint{4.509060in}{3.188866in}}%
\pgfpathlineto{\pgfqpoint{4.501714in}{3.171445in}}%
\pgfpathclose%
\pgfusepath{fill}%
\end{pgfscope}%
\begin{pgfscope}%
\pgfpathrectangle{\pgfqpoint{1.254980in}{0.150000in}}{\pgfqpoint{5.490039in}{5.490039in}}%
\pgfusepath{clip}%
\pgfsetbuttcap%
\pgfsetroundjoin%
\definecolor{currentfill}{rgb}{0.206756,0.371758,0.553117}%
\pgfsetfillcolor{currentfill}%
\pgfsetfillopacity{0.700000}%
\pgfsetlinewidth{0.000000pt}%
\definecolor{currentstroke}{rgb}{0.000000,0.000000,0.000000}%
\pgfsetstrokecolor{currentstroke}%
\pgfsetdash{}{0pt}%
\pgfpathmoveto{\pgfqpoint{4.420688in}{3.129736in}}%
\pgfpathlineto{\pgfqpoint{4.433590in}{3.123150in}}%
\pgfpathlineto{\pgfqpoint{4.446497in}{3.116741in}}%
\pgfpathlineto{\pgfqpoint{4.459410in}{3.110507in}}%
\pgfpathlineto{\pgfqpoint{4.472328in}{3.104448in}}%
\pgfpathlineto{\pgfqpoint{4.479675in}{3.120807in}}%
\pgfpathlineto{\pgfqpoint{4.487022in}{3.137421in}}%
\pgfpathlineto{\pgfqpoint{4.494368in}{3.154299in}}%
\pgfpathlineto{\pgfqpoint{4.501714in}{3.171445in}}%
\pgfpathlineto{\pgfqpoint{4.488804in}{3.178051in}}%
\pgfpathlineto{\pgfqpoint{4.475899in}{3.184833in}}%
\pgfpathlineto{\pgfqpoint{4.463000in}{3.191790in}}%
\pgfpathlineto{\pgfqpoint{4.450106in}{3.198923in}}%
\pgfpathlineto{\pgfqpoint{4.442752in}{3.181219in}}%
\pgfpathlineto{\pgfqpoint{4.435398in}{3.163791in}}%
\pgfpathlineto{\pgfqpoint{4.428043in}{3.146632in}}%
\pgfpathlineto{\pgfqpoint{4.420688in}{3.129736in}}%
\pgfpathclose%
\pgfusepath{fill}%
\end{pgfscope}%
\begin{pgfscope}%
\pgfpathrectangle{\pgfqpoint{1.254980in}{0.150000in}}{\pgfqpoint{5.490039in}{5.490039in}}%
\pgfusepath{clip}%
\pgfsetbuttcap%
\pgfsetroundjoin%
\definecolor{currentfill}{rgb}{0.229739,0.322361,0.545706}%
\pgfsetfillcolor{currentfill}%
\pgfsetfillopacity{0.700000}%
\pgfsetlinewidth{0.000000pt}%
\definecolor{currentstroke}{rgb}{0.000000,0.000000,0.000000}%
\pgfsetstrokecolor{currentstroke}%
\pgfsetdash{}{0pt}%
\pgfpathmoveto{\pgfqpoint{4.045146in}{3.021881in}}%
\pgfpathlineto{\pgfqpoint{4.057990in}{3.013514in}}%
\pgfpathlineto{\pgfqpoint{4.070837in}{3.005341in}}%
\pgfpathlineto{\pgfqpoint{4.083688in}{2.997360in}}%
\pgfpathlineto{\pgfqpoint{4.096542in}{2.989571in}}%
\pgfpathlineto{\pgfqpoint{4.103956in}{3.004774in}}%
\pgfpathlineto{\pgfqpoint{4.111368in}{3.020176in}}%
\pgfpathlineto{\pgfqpoint{4.118777in}{3.035783in}}%
\pgfpathlineto{\pgfqpoint{4.126183in}{3.051598in}}%
\pgfpathlineto{\pgfqpoint{4.113335in}{3.059825in}}%
\pgfpathlineto{\pgfqpoint{4.100491in}{3.068244in}}%
\pgfpathlineto{\pgfqpoint{4.087650in}{3.076856in}}%
\pgfpathlineto{\pgfqpoint{4.074812in}{3.085662in}}%
\pgfpathlineto{\pgfqpoint{4.067400in}{3.069397in}}%
\pgfpathlineto{\pgfqpoint{4.059985in}{3.053349in}}%
\pgfpathlineto{\pgfqpoint{4.052567in}{3.037512in}}%
\pgfpathlineto{\pgfqpoint{4.045146in}{3.021881in}}%
\pgfpathclose%
\pgfusepath{fill}%
\end{pgfscope}%
\begin{pgfscope}%
\pgfpathrectangle{\pgfqpoint{1.254980in}{0.150000in}}{\pgfqpoint{5.490039in}{5.490039in}}%
\pgfusepath{clip}%
\pgfsetbuttcap%
\pgfsetroundjoin%
\definecolor{currentfill}{rgb}{0.190631,0.407061,0.556089}%
\pgfsetfillcolor{currentfill}%
\pgfsetfillopacity{0.700000}%
\pgfsetlinewidth{0.000000pt}%
\definecolor{currentstroke}{rgb}{0.000000,0.000000,0.000000}%
\pgfsetstrokecolor{currentstroke}%
\pgfsetdash{}{0pt}%
\pgfpathmoveto{\pgfqpoint{4.582761in}{3.215854in}}%
\pgfpathlineto{\pgfqpoint{4.595691in}{3.209536in}}%
\pgfpathlineto{\pgfqpoint{4.608627in}{3.203388in}}%
\pgfpathlineto{\pgfqpoint{4.621570in}{3.197410in}}%
\pgfpathlineto{\pgfqpoint{4.634518in}{3.191602in}}%
\pgfpathlineto{\pgfqpoint{4.641848in}{3.209006in}}%
\pgfpathlineto{\pgfqpoint{4.649178in}{3.226707in}}%
\pgfpathlineto{\pgfqpoint{4.656510in}{3.244709in}}%
\pgfpathlineto{\pgfqpoint{4.663843in}{3.263022in}}%
\pgfpathlineto{\pgfqpoint{4.650904in}{3.269433in}}%
\pgfpathlineto{\pgfqpoint{4.637971in}{3.276014in}}%
\pgfpathlineto{\pgfqpoint{4.625044in}{3.282765in}}%
\pgfpathlineto{\pgfqpoint{4.612123in}{3.289687in}}%
\pgfpathlineto{\pgfqpoint{4.604780in}{3.270761in}}%
\pgfpathlineto{\pgfqpoint{4.597440in}{3.252151in}}%
\pgfpathlineto{\pgfqpoint{4.590100in}{3.233852in}}%
\pgfpathlineto{\pgfqpoint{4.582761in}{3.215854in}}%
\pgfpathclose%
\pgfusepath{fill}%
\end{pgfscope}%
\begin{pgfscope}%
\pgfpathrectangle{\pgfqpoint{1.254980in}{0.150000in}}{\pgfqpoint{5.490039in}{5.490039in}}%
\pgfusepath{clip}%
\pgfsetbuttcap%
\pgfsetroundjoin%
\definecolor{currentfill}{rgb}{0.214298,0.355619,0.551184}%
\pgfsetfillcolor{currentfill}%
\pgfsetfillopacity{0.700000}%
\pgfsetlinewidth{0.000000pt}%
\definecolor{currentstroke}{rgb}{0.000000,0.000000,0.000000}%
\pgfsetstrokecolor{currentstroke}%
\pgfsetdash{}{0pt}%
\pgfpathmoveto{\pgfqpoint{4.339670in}{3.090693in}}%
\pgfpathlineto{\pgfqpoint{4.352559in}{3.083915in}}%
\pgfpathlineto{\pgfqpoint{4.365453in}{3.077317in}}%
\pgfpathlineto{\pgfqpoint{4.378353in}{3.070897in}}%
\pgfpathlineto{\pgfqpoint{4.391257in}{3.064655in}}%
\pgfpathlineto{\pgfqpoint{4.398617in}{3.080562in}}%
\pgfpathlineto{\pgfqpoint{4.405975in}{3.096707in}}%
\pgfpathlineto{\pgfqpoint{4.413332in}{3.113096in}}%
\pgfpathlineto{\pgfqpoint{4.420688in}{3.129736in}}%
\pgfpathlineto{\pgfqpoint{4.407791in}{3.136498in}}%
\pgfpathlineto{\pgfqpoint{4.394900in}{3.143438in}}%
\pgfpathlineto{\pgfqpoint{4.382014in}{3.150557in}}%
\pgfpathlineto{\pgfqpoint{4.369132in}{3.157855in}}%
\pgfpathlineto{\pgfqpoint{4.361768in}{3.140685in}}%
\pgfpathlineto{\pgfqpoint{4.354404in}{3.123772in}}%
\pgfpathlineto{\pgfqpoint{4.347038in}{3.107110in}}%
\pgfpathlineto{\pgfqpoint{4.339670in}{3.090693in}}%
\pgfpathclose%
\pgfusepath{fill}%
\end{pgfscope}%
\begin{pgfscope}%
\pgfpathrectangle{\pgfqpoint{1.254980in}{0.150000in}}{\pgfqpoint{5.490039in}{5.490039in}}%
\pgfusepath{clip}%
\pgfsetbuttcap%
\pgfsetroundjoin%
\definecolor{currentfill}{rgb}{0.751884,0.874951,0.143228}%
\pgfsetfillcolor{currentfill}%
\pgfsetfillopacity{0.700000}%
\pgfsetlinewidth{0.000000pt}%
\definecolor{currentstroke}{rgb}{0.000000,0.000000,0.000000}%
\pgfsetstrokecolor{currentstroke}%
\pgfsetdash{}{0pt}%
\pgfpathmoveto{\pgfqpoint{3.614446in}{4.620071in}}%
\pgfpathlineto{\pgfqpoint{3.627380in}{4.594933in}}%
\pgfpathlineto{\pgfqpoint{3.640307in}{4.570090in}}%
\pgfpathlineto{\pgfqpoint{3.653226in}{4.545540in}}%
\pgfpathlineto{\pgfqpoint{3.666139in}{4.521280in}}%
\pgfpathlineto{\pgfqpoint{3.673492in}{4.554712in}}%
\pgfpathlineto{\pgfqpoint{3.680843in}{4.588661in}}%
\pgfpathlineto{\pgfqpoint{3.688193in}{4.623136in}}%
\pgfpathlineto{\pgfqpoint{3.675275in}{4.647942in}}%
\pgfpathlineto{\pgfqpoint{3.662349in}{4.673039in}}%
\pgfpathlineto{\pgfqpoint{3.649416in}{4.698430in}}%
\pgfpathlineto{\pgfqpoint{3.636476in}{4.724119in}}%
\pgfpathlineto{\pgfqpoint{3.629135in}{4.688903in}}%
\pgfpathlineto{\pgfqpoint{3.621791in}{4.654224in}}%
\pgfpathlineto{\pgfqpoint{3.614446in}{4.620071in}}%
\pgfpathclose%
\pgfusepath{fill}%
\end{pgfscope}%
\begin{pgfscope}%
\pgfpathrectangle{\pgfqpoint{1.254980in}{0.150000in}}{\pgfqpoint{5.490039in}{5.490039in}}%
\pgfusepath{clip}%
\pgfsetbuttcap%
\pgfsetroundjoin%
\definecolor{currentfill}{rgb}{0.150148,0.676631,0.506589}%
\pgfsetfillcolor{currentfill}%
\pgfsetfillopacity{0.700000}%
\pgfsetlinewidth{0.000000pt}%
\definecolor{currentstroke}{rgb}{0.000000,0.000000,0.000000}%
\pgfsetstrokecolor{currentstroke}%
\pgfsetdash{}{0pt}%
\pgfpathmoveto{\pgfqpoint{3.274067in}{3.932839in}}%
\pgfpathlineto{\pgfqpoint{3.287039in}{3.909939in}}%
\pgfpathlineto{\pgfqpoint{3.300003in}{3.887349in}}%
\pgfpathlineto{\pgfqpoint{3.312958in}{3.865065in}}%
\pgfpathlineto{\pgfqpoint{3.325906in}{3.843084in}}%
\pgfpathlineto{\pgfqpoint{3.333377in}{3.864847in}}%
\pgfpathlineto{\pgfqpoint{3.340842in}{3.886919in}}%
\pgfpathlineto{\pgfqpoint{3.348302in}{3.909305in}}%
\pgfpathlineto{\pgfqpoint{3.355757in}{3.932012in}}%
\pgfpathlineto{\pgfqpoint{3.342810in}{3.954460in}}%
\pgfpathlineto{\pgfqpoint{3.329856in}{3.977213in}}%
\pgfpathlineto{\pgfqpoint{3.316893in}{4.000272in}}%
\pgfpathlineto{\pgfqpoint{3.303921in}{4.023641in}}%
\pgfpathlineto{\pgfqpoint{3.296466in}{4.000454in}}%
\pgfpathlineto{\pgfqpoint{3.289005in}{3.977594in}}%
\pgfpathlineto{\pgfqpoint{3.281539in}{3.955058in}}%
\pgfpathlineto{\pgfqpoint{3.274067in}{3.932839in}}%
\pgfpathclose%
\pgfusepath{fill}%
\end{pgfscope}%
\begin{pgfscope}%
\pgfpathrectangle{\pgfqpoint{1.254980in}{0.150000in}}{\pgfqpoint{5.490039in}{5.490039in}}%
\pgfusepath{clip}%
\pgfsetbuttcap%
\pgfsetroundjoin%
\definecolor{currentfill}{rgb}{0.218130,0.347432,0.550038}%
\pgfsetfillcolor{currentfill}%
\pgfsetfillopacity{0.700000}%
\pgfsetlinewidth{0.000000pt}%
\definecolor{currentstroke}{rgb}{0.000000,0.000000,0.000000}%
\pgfsetstrokecolor{currentstroke}%
\pgfsetdash{}{0pt}%
\pgfpathmoveto{\pgfqpoint{3.647732in}{3.082923in}}%
\pgfpathlineto{\pgfqpoint{3.660561in}{3.070698in}}%
\pgfpathlineto{\pgfqpoint{3.673389in}{3.058697in}}%
\pgfpathlineto{\pgfqpoint{3.686216in}{3.046919in}}%
\pgfpathlineto{\pgfqpoint{3.699044in}{3.035363in}}%
\pgfpathlineto{\pgfqpoint{3.706538in}{3.050626in}}%
\pgfpathlineto{\pgfqpoint{3.714029in}{3.066075in}}%
\pgfpathlineto{\pgfqpoint{3.721515in}{3.081712in}}%
\pgfpathlineto{\pgfqpoint{3.728997in}{3.097543in}}%
\pgfpathlineto{\pgfqpoint{3.716175in}{3.109458in}}%
\pgfpathlineto{\pgfqpoint{3.703353in}{3.121594in}}%
\pgfpathlineto{\pgfqpoint{3.690531in}{3.133954in}}%
\pgfpathlineto{\pgfqpoint{3.677708in}{3.146539in}}%
\pgfpathlineto{\pgfqpoint{3.670221in}{3.130338in}}%
\pgfpathlineto{\pgfqpoint{3.662729in}{3.114338in}}%
\pgfpathlineto{\pgfqpoint{3.655233in}{3.098535in}}%
\pgfpathlineto{\pgfqpoint{3.647732in}{3.082923in}}%
\pgfpathclose%
\pgfusepath{fill}%
\end{pgfscope}%
\begin{pgfscope}%
\pgfpathrectangle{\pgfqpoint{1.254980in}{0.150000in}}{\pgfqpoint{5.490039in}{5.490039in}}%
\pgfusepath{clip}%
\pgfsetbuttcap%
\pgfsetroundjoin%
\definecolor{currentfill}{rgb}{0.180629,0.429975,0.557282}%
\pgfsetfillcolor{currentfill}%
\pgfsetfillopacity{0.700000}%
\pgfsetlinewidth{0.000000pt}%
\definecolor{currentstroke}{rgb}{0.000000,0.000000,0.000000}%
\pgfsetstrokecolor{currentstroke}%
\pgfsetdash{}{0pt}%
\pgfpathmoveto{\pgfqpoint{4.663843in}{3.263022in}}%
\pgfpathlineto{\pgfqpoint{4.676788in}{3.256779in}}%
\pgfpathlineto{\pgfqpoint{4.689740in}{3.250705in}}%
\pgfpathlineto{\pgfqpoint{4.702698in}{3.244798in}}%
\pgfpathlineto{\pgfqpoint{4.715662in}{3.239059in}}%
\pgfpathlineto{\pgfqpoint{4.722987in}{3.257067in}}%
\pgfpathlineto{\pgfqpoint{4.730313in}{3.275393in}}%
\pgfpathlineto{\pgfqpoint{4.737642in}{3.294044in}}%
\pgfpathlineto{\pgfqpoint{4.744973in}{3.313027in}}%
\pgfpathlineto{\pgfqpoint{4.732019in}{3.319397in}}%
\pgfpathlineto{\pgfqpoint{4.719071in}{3.325935in}}%
\pgfpathlineto{\pgfqpoint{4.706129in}{3.332640in}}%
\pgfpathlineto{\pgfqpoint{4.693193in}{3.339514in}}%
\pgfpathlineto{\pgfqpoint{4.685852in}{3.319889in}}%
\pgfpathlineto{\pgfqpoint{4.678514in}{3.300604in}}%
\pgfpathlineto{\pgfqpoint{4.671177in}{3.281651in}}%
\pgfpathlineto{\pgfqpoint{4.663843in}{3.263022in}}%
\pgfpathclose%
\pgfusepath{fill}%
\end{pgfscope}%
\begin{pgfscope}%
\pgfpathrectangle{\pgfqpoint{1.254980in}{0.150000in}}{\pgfqpoint{5.490039in}{5.490039in}}%
\pgfusepath{clip}%
\pgfsetbuttcap%
\pgfsetroundjoin%
\definecolor{currentfill}{rgb}{0.156270,0.489624,0.557936}%
\pgfsetfillcolor{currentfill}%
\pgfsetfillopacity{0.700000}%
\pgfsetlinewidth{0.000000pt}%
\definecolor{currentstroke}{rgb}{0.000000,0.000000,0.000000}%
\pgfsetstrokecolor{currentstroke}%
\pgfsetdash{}{0pt}%
\pgfpathmoveto{\pgfqpoint{3.339214in}{3.448524in}}%
\pgfpathlineto{\pgfqpoint{3.352112in}{3.430336in}}%
\pgfpathlineto{\pgfqpoint{3.365005in}{3.412422in}}%
\pgfpathlineto{\pgfqpoint{3.377893in}{3.394779in}}%
\pgfpathlineto{\pgfqpoint{3.390776in}{3.377404in}}%
\pgfpathlineto{\pgfqpoint{3.398303in}{3.394775in}}%
\pgfpathlineto{\pgfqpoint{3.405824in}{3.412373in}}%
\pgfpathlineto{\pgfqpoint{3.413340in}{3.430204in}}%
\pgfpathlineto{\pgfqpoint{3.420851in}{3.448272in}}%
\pgfpathlineto{\pgfqpoint{3.407972in}{3.466014in}}%
\pgfpathlineto{\pgfqpoint{3.395089in}{3.484024in}}%
\pgfpathlineto{\pgfqpoint{3.382201in}{3.502307in}}%
\pgfpathlineto{\pgfqpoint{3.369307in}{3.520863in}}%
\pgfpathlineto{\pgfqpoint{3.361792in}{3.502416in}}%
\pgfpathlineto{\pgfqpoint{3.354272in}{3.484214in}}%
\pgfpathlineto{\pgfqpoint{3.346746in}{3.466251in}}%
\pgfpathlineto{\pgfqpoint{3.339214in}{3.448524in}}%
\pgfpathclose%
\pgfusepath{fill}%
\end{pgfscope}%
\begin{pgfscope}%
\pgfpathrectangle{\pgfqpoint{1.254980in}{0.150000in}}{\pgfqpoint{5.490039in}{5.490039in}}%
\pgfusepath{clip}%
\pgfsetbuttcap%
\pgfsetroundjoin%
\definecolor{currentfill}{rgb}{0.221989,0.339161,0.548752}%
\pgfsetfillcolor{currentfill}%
\pgfsetfillopacity{0.700000}%
\pgfsetlinewidth{0.000000pt}%
\definecolor{currentstroke}{rgb}{0.000000,0.000000,0.000000}%
\pgfsetstrokecolor{currentstroke}%
\pgfsetdash{}{0pt}%
\pgfpathmoveto{\pgfqpoint{4.258649in}{3.054306in}}%
\pgfpathlineto{\pgfqpoint{4.271526in}{3.047296in}}%
\pgfpathlineto{\pgfqpoint{4.284408in}{3.040469in}}%
\pgfpathlineto{\pgfqpoint{4.297294in}{3.033823in}}%
\pgfpathlineto{\pgfqpoint{4.310186in}{3.027358in}}%
\pgfpathlineto{\pgfqpoint{4.317560in}{3.042853in}}%
\pgfpathlineto{\pgfqpoint{4.324932in}{3.058571in}}%
\pgfpathlineto{\pgfqpoint{4.332302in}{3.074515in}}%
\pgfpathlineto{\pgfqpoint{4.339670in}{3.090693in}}%
\pgfpathlineto{\pgfqpoint{4.326786in}{3.097651in}}%
\pgfpathlineto{\pgfqpoint{4.313907in}{3.104789in}}%
\pgfpathlineto{\pgfqpoint{4.301032in}{3.112110in}}%
\pgfpathlineto{\pgfqpoint{4.288162in}{3.119612in}}%
\pgfpathlineto{\pgfqpoint{4.280787in}{3.102932in}}%
\pgfpathlineto{\pgfqpoint{4.273409in}{3.086491in}}%
\pgfpathlineto{\pgfqpoint{4.266030in}{3.070284in}}%
\pgfpathlineto{\pgfqpoint{4.258649in}{3.054306in}}%
\pgfpathclose%
\pgfusepath{fill}%
\end{pgfscope}%
\begin{pgfscope}%
\pgfpathrectangle{\pgfqpoint{1.254980in}{0.150000in}}{\pgfqpoint{5.490039in}{5.490039in}}%
\pgfusepath{clip}%
\pgfsetbuttcap%
\pgfsetroundjoin%
\definecolor{currentfill}{rgb}{0.231674,0.318106,0.544834}%
\pgfsetfillcolor{currentfill}%
\pgfsetfillopacity{0.700000}%
\pgfsetlinewidth{0.000000pt}%
\definecolor{currentstroke}{rgb}{0.000000,0.000000,0.000000}%
\pgfsetstrokecolor{currentstroke}%
\pgfsetdash{}{0pt}%
\pgfpathmoveto{\pgfqpoint{3.831587in}{3.010037in}}%
\pgfpathlineto{\pgfqpoint{3.844415in}{3.000053in}}%
\pgfpathlineto{\pgfqpoint{3.857245in}{2.990277in}}%
\pgfpathlineto{\pgfqpoint{3.870077in}{2.980707in}}%
\pgfpathlineto{\pgfqpoint{3.882910in}{2.971342in}}%
\pgfpathlineto{\pgfqpoint{3.890370in}{2.986247in}}%
\pgfpathlineto{\pgfqpoint{3.897827in}{3.001333in}}%
\pgfpathlineto{\pgfqpoint{3.905280in}{3.016606in}}%
\pgfpathlineto{\pgfqpoint{3.912730in}{3.032069in}}%
\pgfpathlineto{\pgfqpoint{3.899902in}{3.041818in}}%
\pgfpathlineto{\pgfqpoint{3.887077in}{3.051773in}}%
\pgfpathlineto{\pgfqpoint{3.874253in}{3.061933in}}%
\pgfpathlineto{\pgfqpoint{3.861430in}{3.072302in}}%
\pgfpathlineto{\pgfqpoint{3.853975in}{3.056444in}}%
\pgfpathlineto{\pgfqpoint{3.846516in}{3.040783in}}%
\pgfpathlineto{\pgfqpoint{3.839054in}{3.025316in}}%
\pgfpathlineto{\pgfqpoint{3.831587in}{3.010037in}}%
\pgfpathclose%
\pgfusepath{fill}%
\end{pgfscope}%
\begin{pgfscope}%
\pgfpathrectangle{\pgfqpoint{1.254980in}{0.150000in}}{\pgfqpoint{5.490039in}{5.490039in}}%
\pgfusepath{clip}%
\pgfsetbuttcap%
\pgfsetroundjoin%
\definecolor{currentfill}{rgb}{0.565498,0.842430,0.262877}%
\pgfsetfillcolor{currentfill}%
\pgfsetfillopacity{0.700000}%
\pgfsetlinewidth{0.000000pt}%
\definecolor{currentstroke}{rgb}{0.000000,0.000000,0.000000}%
\pgfsetstrokecolor{currentstroke}%
\pgfsetdash{}{0pt}%
\pgfpathmoveto{\pgfqpoint{3.422491in}{4.444262in}}%
\pgfpathlineto{\pgfqpoint{3.435471in}{4.418679in}}%
\pgfpathlineto{\pgfqpoint{3.448443in}{4.393412in}}%
\pgfpathlineto{\pgfqpoint{3.461406in}{4.368457in}}%
\pgfpathlineto{\pgfqpoint{3.474360in}{4.343811in}}%
\pgfpathlineto{\pgfqpoint{3.481735in}{4.372912in}}%
\pgfpathlineto{\pgfqpoint{3.489106in}{4.402451in}}%
\pgfpathlineto{\pgfqpoint{3.496474in}{4.432435in}}%
\pgfpathlineto{\pgfqpoint{3.503838in}{4.462872in}}%
\pgfpathlineto{\pgfqpoint{3.490878in}{4.488154in}}%
\pgfpathlineto{\pgfqpoint{3.477910in}{4.513746in}}%
\pgfpathlineto{\pgfqpoint{3.464933in}{4.539652in}}%
\pgfpathlineto{\pgfqpoint{3.451947in}{4.565874in}}%
\pgfpathlineto{\pgfqpoint{3.444589in}{4.534786in}}%
\pgfpathlineto{\pgfqpoint{3.437227in}{4.504159in}}%
\pgfpathlineto{\pgfqpoint{3.429861in}{4.473987in}}%
\pgfpathlineto{\pgfqpoint{3.422491in}{4.444262in}}%
\pgfpathclose%
\pgfusepath{fill}%
\end{pgfscope}%
\begin{pgfscope}%
\pgfpathrectangle{\pgfqpoint{1.254980in}{0.150000in}}{\pgfqpoint{5.490039in}{5.490039in}}%
\pgfusepath{clip}%
\pgfsetbuttcap%
\pgfsetroundjoin%
\definecolor{currentfill}{rgb}{0.235526,0.309527,0.542944}%
\pgfsetfillcolor{currentfill}%
\pgfsetfillopacity{0.700000}%
\pgfsetlinewidth{0.000000pt}%
\definecolor{currentstroke}{rgb}{0.000000,0.000000,0.000000}%
\pgfsetstrokecolor{currentstroke}%
\pgfsetdash{}{0pt}%
\pgfpathmoveto{\pgfqpoint{3.964060in}{2.995100in}}%
\pgfpathlineto{\pgfqpoint{3.976899in}{2.986359in}}%
\pgfpathlineto{\pgfqpoint{3.989740in}{2.977815in}}%
\pgfpathlineto{\pgfqpoint{4.002584in}{2.969469in}}%
\pgfpathlineto{\pgfqpoint{4.015431in}{2.961318in}}%
\pgfpathlineto{\pgfqpoint{4.022865in}{2.976174in}}%
\pgfpathlineto{\pgfqpoint{4.030295in}{2.991217in}}%
\pgfpathlineto{\pgfqpoint{4.037722in}{3.006451in}}%
\pgfpathlineto{\pgfqpoint{4.045146in}{3.021881in}}%
\pgfpathlineto{\pgfqpoint{4.032305in}{3.030443in}}%
\pgfpathlineto{\pgfqpoint{4.019467in}{3.039201in}}%
\pgfpathlineto{\pgfqpoint{4.006632in}{3.048155in}}%
\pgfpathlineto{\pgfqpoint{3.993799in}{3.057308in}}%
\pgfpathlineto{\pgfqpoint{3.986370in}{3.041457in}}%
\pgfpathlineto{\pgfqpoint{3.978936in}{3.025808in}}%
\pgfpathlineto{\pgfqpoint{3.971500in}{3.010358in}}%
\pgfpathlineto{\pgfqpoint{3.964060in}{2.995100in}}%
\pgfpathclose%
\pgfusepath{fill}%
\end{pgfscope}%
\begin{pgfscope}%
\pgfpathrectangle{\pgfqpoint{1.254980in}{0.150000in}}{\pgfqpoint{5.490039in}{5.490039in}}%
\pgfusepath{clip}%
\pgfsetbuttcap%
\pgfsetroundjoin%
\definecolor{currentfill}{rgb}{0.123463,0.581687,0.547445}%
\pgfsetfillcolor{currentfill}%
\pgfsetfillopacity{0.700000}%
\pgfsetlinewidth{0.000000pt}%
\definecolor{currentstroke}{rgb}{0.000000,0.000000,0.000000}%
\pgfsetstrokecolor{currentstroke}%
\pgfsetdash{}{0pt}%
\pgfpathmoveto{\pgfqpoint{3.265934in}{3.679487in}}%
\pgfpathlineto{\pgfqpoint{3.278880in}{3.658644in}}%
\pgfpathlineto{\pgfqpoint{3.291818in}{3.638097in}}%
\pgfpathlineto{\pgfqpoint{3.304749in}{3.617842in}}%
\pgfpathlineto{\pgfqpoint{3.317673in}{3.597878in}}%
\pgfpathlineto{\pgfqpoint{3.325187in}{3.616962in}}%
\pgfpathlineto{\pgfqpoint{3.332694in}{3.636307in}}%
\pgfpathlineto{\pgfqpoint{3.340196in}{3.655918in}}%
\pgfpathlineto{\pgfqpoint{3.347692in}{3.675800in}}%
\pgfpathlineto{\pgfqpoint{3.334771in}{3.696166in}}%
\pgfpathlineto{\pgfqpoint{3.321844in}{3.716823in}}%
\pgfpathlineto{\pgfqpoint{3.308909in}{3.737773in}}%
\pgfpathlineto{\pgfqpoint{3.295966in}{3.759019in}}%
\pgfpathlineto{\pgfqpoint{3.288467in}{3.738723in}}%
\pgfpathlineto{\pgfqpoint{3.280962in}{3.718706in}}%
\pgfpathlineto{\pgfqpoint{3.273451in}{3.698962in}}%
\pgfpathlineto{\pgfqpoint{3.265934in}{3.679487in}}%
\pgfpathclose%
\pgfusepath{fill}%
\end{pgfscope}%
\begin{pgfscope}%
\pgfpathrectangle{\pgfqpoint{1.254980in}{0.150000in}}{\pgfqpoint{5.490039in}{5.490039in}}%
\pgfusepath{clip}%
\pgfsetbuttcap%
\pgfsetroundjoin%
\definecolor{currentfill}{rgb}{0.227802,0.326594,0.546532}%
\pgfsetfillcolor{currentfill}%
\pgfsetfillopacity{0.700000}%
\pgfsetlinewidth{0.000000pt}%
\definecolor{currentstroke}{rgb}{0.000000,0.000000,0.000000}%
\pgfsetstrokecolor{currentstroke}%
\pgfsetdash{}{0pt}%
\pgfpathmoveto{\pgfqpoint{4.177610in}{3.020587in}}%
\pgfpathlineto{\pgfqpoint{4.190477in}{3.013304in}}%
\pgfpathlineto{\pgfqpoint{4.203348in}{3.006207in}}%
\pgfpathlineto{\pgfqpoint{4.216223in}{2.999294in}}%
\pgfpathlineto{\pgfqpoint{4.229103in}{2.992566in}}%
\pgfpathlineto{\pgfqpoint{4.236493in}{3.007686in}}%
\pgfpathlineto{\pgfqpoint{4.243880in}{3.023013in}}%
\pgfpathlineto{\pgfqpoint{4.251265in}{3.038551in}}%
\pgfpathlineto{\pgfqpoint{4.258649in}{3.054306in}}%
\pgfpathlineto{\pgfqpoint{4.245776in}{3.061499in}}%
\pgfpathlineto{\pgfqpoint{4.232908in}{3.068877in}}%
\pgfpathlineto{\pgfqpoint{4.220044in}{3.076440in}}%
\pgfpathlineto{\pgfqpoint{4.207184in}{3.084188in}}%
\pgfpathlineto{\pgfqpoint{4.199794in}{3.067957in}}%
\pgfpathlineto{\pgfqpoint{4.192401in}{3.051950in}}%
\pgfpathlineto{\pgfqpoint{4.185007in}{3.036162in}}%
\pgfpathlineto{\pgfqpoint{4.177610in}{3.020587in}}%
\pgfpathclose%
\pgfusepath{fill}%
\end{pgfscope}%
\begin{pgfscope}%
\pgfpathrectangle{\pgfqpoint{1.254980in}{0.150000in}}{\pgfqpoint{5.490039in}{5.490039in}}%
\pgfusepath{clip}%
\pgfsetbuttcap%
\pgfsetroundjoin%
\definecolor{currentfill}{rgb}{0.730889,0.871916,0.156029}%
\pgfsetfillcolor{currentfill}%
\pgfsetfillopacity{0.700000}%
\pgfsetlinewidth{0.000000pt}%
\definecolor{currentstroke}{rgb}{0.000000,0.000000,0.000000}%
\pgfsetstrokecolor{currentstroke}%
\pgfsetdash{}{0pt}%
\pgfpathmoveto{\pgfqpoint{3.533259in}{4.589310in}}%
\pgfpathlineto{\pgfqpoint{3.546217in}{4.563666in}}%
\pgfpathlineto{\pgfqpoint{3.559167in}{4.538327in}}%
\pgfpathlineto{\pgfqpoint{3.572109in}{4.513292in}}%
\pgfpathlineto{\pgfqpoint{3.585043in}{4.488557in}}%
\pgfpathlineto{\pgfqpoint{3.592397in}{4.520688in}}%
\pgfpathlineto{\pgfqpoint{3.599749in}{4.553312in}}%
\pgfpathlineto{\pgfqpoint{3.607099in}{4.586437in}}%
\pgfpathlineto{\pgfqpoint{3.614446in}{4.620071in}}%
\pgfpathlineto{\pgfqpoint{3.601505in}{4.645509in}}%
\pgfpathlineto{\pgfqpoint{3.588555in}{4.671248in}}%
\pgfpathlineto{\pgfqpoint{3.575598in}{4.697291in}}%
\pgfpathlineto{\pgfqpoint{3.562632in}{4.723643in}}%
\pgfpathlineto{\pgfqpoint{3.555293in}{4.689290in}}%
\pgfpathlineto{\pgfqpoint{3.547951in}{4.655456in}}%
\pgfpathlineto{\pgfqpoint{3.540607in}{4.622132in}}%
\pgfpathlineto{\pgfqpoint{3.533259in}{4.589310in}}%
\pgfpathclose%
\pgfusepath{fill}%
\end{pgfscope}%
\begin{pgfscope}%
\pgfpathrectangle{\pgfqpoint{1.254980in}{0.150000in}}{\pgfqpoint{5.490039in}{5.490039in}}%
\pgfusepath{clip}%
\pgfsetbuttcap%
\pgfsetroundjoin%
\definecolor{currentfill}{rgb}{0.174274,0.445044,0.557792}%
\pgfsetfillcolor{currentfill}%
\pgfsetfillopacity{0.700000}%
\pgfsetlinewidth{0.000000pt}%
\definecolor{currentstroke}{rgb}{0.000000,0.000000,0.000000}%
\pgfsetstrokecolor{currentstroke}%
\pgfsetdash{}{0pt}%
\pgfpathmoveto{\pgfqpoint{4.744973in}{3.313027in}}%
\pgfpathlineto{\pgfqpoint{4.757934in}{3.306823in}}%
\pgfpathlineto{\pgfqpoint{4.770901in}{3.300785in}}%
\pgfpathlineto{\pgfqpoint{4.783875in}{3.294912in}}%
\pgfpathlineto{\pgfqpoint{4.796856in}{3.289204in}}%
\pgfpathlineto{\pgfqpoint{4.804179in}{3.307878in}}%
\pgfpathlineto{\pgfqpoint{4.811505in}{3.326893in}}%
\pgfpathlineto{\pgfqpoint{4.818834in}{3.346255in}}%
\pgfpathlineto{\pgfqpoint{4.805861in}{3.352454in}}%
\pgfpathlineto{\pgfqpoint{4.792895in}{3.358819in}}%
\pgfpathlineto{\pgfqpoint{4.779936in}{3.365348in}}%
\pgfpathlineto{\pgfqpoint{4.766983in}{3.372044in}}%
\pgfpathlineto{\pgfqpoint{4.759643in}{3.352019in}}%
\pgfpathlineto{\pgfqpoint{4.752306in}{3.332349in}}%
\pgfpathlineto{\pgfqpoint{4.744973in}{3.313027in}}%
\pgfpathclose%
\pgfusepath{fill}%
\end{pgfscope}%
\begin{pgfscope}%
\pgfpathrectangle{\pgfqpoint{1.254980in}{0.150000in}}{\pgfqpoint{5.490039in}{5.490039in}}%
\pgfusepath{clip}%
\pgfsetbuttcap%
\pgfsetroundjoin%
\definecolor{currentfill}{rgb}{0.227802,0.326594,0.546532}%
\pgfsetfillcolor{currentfill}%
\pgfsetfillopacity{0.700000}%
\pgfsetlinewidth{0.000000pt}%
\definecolor{currentstroke}{rgb}{0.000000,0.000000,0.000000}%
\pgfsetstrokecolor{currentstroke}%
\pgfsetdash{}{0pt}%
\pgfpathmoveto{\pgfqpoint{3.699044in}{3.035363in}}%
\pgfpathlineto{\pgfqpoint{3.711871in}{3.024028in}}%
\pgfpathlineto{\pgfqpoint{3.724699in}{3.012910in}}%
\pgfpathlineto{\pgfqpoint{3.737527in}{3.002011in}}%
\pgfpathlineto{\pgfqpoint{3.750356in}{2.991326in}}%
\pgfpathlineto{\pgfqpoint{3.757845in}{3.006242in}}%
\pgfpathlineto{\pgfqpoint{3.765330in}{3.021336in}}%
\pgfpathlineto{\pgfqpoint{3.772810in}{3.036612in}}%
\pgfpathlineto{\pgfqpoint{3.780286in}{3.052074in}}%
\pgfpathlineto{\pgfqpoint{3.767463in}{3.063116in}}%
\pgfpathlineto{\pgfqpoint{3.754641in}{3.074374in}}%
\pgfpathlineto{\pgfqpoint{3.741819in}{3.085849in}}%
\pgfpathlineto{\pgfqpoint{3.728997in}{3.097543in}}%
\pgfpathlineto{\pgfqpoint{3.721515in}{3.081712in}}%
\pgfpathlineto{\pgfqpoint{3.714029in}{3.066075in}}%
\pgfpathlineto{\pgfqpoint{3.706538in}{3.050626in}}%
\pgfpathlineto{\pgfqpoint{3.699044in}{3.035363in}}%
\pgfpathclose%
\pgfusepath{fill}%
\end{pgfscope}%
\begin{pgfscope}%
\pgfpathrectangle{\pgfqpoint{1.254980in}{0.150000in}}{\pgfqpoint{5.490039in}{5.490039in}}%
\pgfusepath{clip}%
\pgfsetbuttcap%
\pgfsetroundjoin%
\definecolor{currentfill}{rgb}{0.128087,0.647749,0.523491}%
\pgfsetfillcolor{currentfill}%
\pgfsetfillopacity{0.700000}%
\pgfsetlinewidth{0.000000pt}%
\definecolor{currentstroke}{rgb}{0.000000,0.000000,0.000000}%
\pgfsetstrokecolor{currentstroke}%
\pgfsetdash{}{0pt}%
\pgfpathmoveto{\pgfqpoint{3.244118in}{3.847026in}}%
\pgfpathlineto{\pgfqpoint{3.257093in}{3.824565in}}%
\pgfpathlineto{\pgfqpoint{3.270059in}{3.802412in}}%
\pgfpathlineto{\pgfqpoint{3.283016in}{3.780564in}}%
\pgfpathlineto{\pgfqpoint{3.295966in}{3.759019in}}%
\pgfpathlineto{\pgfqpoint{3.303460in}{3.779598in}}%
\pgfpathlineto{\pgfqpoint{3.310948in}{3.800465in}}%
\pgfpathlineto{\pgfqpoint{3.318430in}{3.821625in}}%
\pgfpathlineto{\pgfqpoint{3.325906in}{3.843084in}}%
\pgfpathlineto{\pgfqpoint{3.312958in}{3.865065in}}%
\pgfpathlineto{\pgfqpoint{3.300003in}{3.887349in}}%
\pgfpathlineto{\pgfqpoint{3.287039in}{3.909939in}}%
\pgfpathlineto{\pgfqpoint{3.274067in}{3.932839in}}%
\pgfpathlineto{\pgfqpoint{3.266588in}{3.910931in}}%
\pgfpathlineto{\pgfqpoint{3.259104in}{3.889330in}}%
\pgfpathlineto{\pgfqpoint{3.251614in}{3.868030in}}%
\pgfpathlineto{\pgfqpoint{3.244118in}{3.847026in}}%
\pgfpathclose%
\pgfusepath{fill}%
\end{pgfscope}%
\begin{pgfscope}%
\pgfpathrectangle{\pgfqpoint{1.254980in}{0.150000in}}{\pgfqpoint{5.490039in}{5.490039in}}%
\pgfusepath{clip}%
\pgfsetbuttcap%
\pgfsetroundjoin%
\definecolor{currentfill}{rgb}{0.143343,0.522773,0.556295}%
\pgfsetfillcolor{currentfill}%
\pgfsetfillopacity{0.700000}%
\pgfsetlinewidth{0.000000pt}%
\definecolor{currentstroke}{rgb}{0.000000,0.000000,0.000000}%
\pgfsetstrokecolor{currentstroke}%
\pgfsetdash{}{0pt}%
\pgfpathmoveto{\pgfqpoint{3.287562in}{3.524060in}}%
\pgfpathlineto{\pgfqpoint{3.300484in}{3.504753in}}%
\pgfpathlineto{\pgfqpoint{3.313400in}{3.485729in}}%
\pgfpathlineto{\pgfqpoint{3.326310in}{3.466987in}}%
\pgfpathlineto{\pgfqpoint{3.339214in}{3.448524in}}%
\pgfpathlineto{\pgfqpoint{3.346746in}{3.466251in}}%
\pgfpathlineto{\pgfqpoint{3.354272in}{3.484214in}}%
\pgfpathlineto{\pgfqpoint{3.361792in}{3.502416in}}%
\pgfpathlineto{\pgfqpoint{3.369307in}{3.520863in}}%
\pgfpathlineto{\pgfqpoint{3.356408in}{3.539695in}}%
\pgfpathlineto{\pgfqpoint{3.343502in}{3.558807in}}%
\pgfpathlineto{\pgfqpoint{3.330591in}{3.578200in}}%
\pgfpathlineto{\pgfqpoint{3.317673in}{3.597878in}}%
\pgfpathlineto{\pgfqpoint{3.310154in}{3.579050in}}%
\pgfpathlineto{\pgfqpoint{3.302630in}{3.560474in}}%
\pgfpathlineto{\pgfqpoint{3.295099in}{3.542145in}}%
\pgfpathlineto{\pgfqpoint{3.287562in}{3.524060in}}%
\pgfpathclose%
\pgfusepath{fill}%
\end{pgfscope}%
\begin{pgfscope}%
\pgfpathrectangle{\pgfqpoint{1.254980in}{0.150000in}}{\pgfqpoint{5.490039in}{5.490039in}}%
\pgfusepath{clip}%
\pgfsetbuttcap%
\pgfsetroundjoin%
\definecolor{currentfill}{rgb}{0.197636,0.391528,0.554969}%
\pgfsetfillcolor{currentfill}%
\pgfsetfillopacity{0.700000}%
\pgfsetlinewidth{0.000000pt}%
\definecolor{currentstroke}{rgb}{0.000000,0.000000,0.000000}%
\pgfsetstrokecolor{currentstroke}%
\pgfsetdash{}{0pt}%
\pgfpathmoveto{\pgfqpoint{3.463570in}{3.183198in}}%
\pgfpathlineto{\pgfqpoint{3.476424in}{3.168472in}}%
\pgfpathlineto{\pgfqpoint{3.489276in}{3.153992in}}%
\pgfpathlineto{\pgfqpoint{3.502125in}{3.139756in}}%
\pgfpathlineto{\pgfqpoint{3.514971in}{3.125763in}}%
\pgfpathlineto{\pgfqpoint{3.522503in}{3.141290in}}%
\pgfpathlineto{\pgfqpoint{3.530029in}{3.157007in}}%
\pgfpathlineto{\pgfqpoint{3.537550in}{3.172918in}}%
\pgfpathlineto{\pgfqpoint{3.545067in}{3.189026in}}%
\pgfpathlineto{\pgfqpoint{3.532226in}{3.203353in}}%
\pgfpathlineto{\pgfqpoint{3.519383in}{3.217922in}}%
\pgfpathlineto{\pgfqpoint{3.506537in}{3.232736in}}%
\pgfpathlineto{\pgfqpoint{3.493688in}{3.247796in}}%
\pgfpathlineto{\pgfqpoint{3.486167in}{3.231343in}}%
\pgfpathlineto{\pgfqpoint{3.478640in}{3.215095in}}%
\pgfpathlineto{\pgfqpoint{3.471108in}{3.199048in}}%
\pgfpathlineto{\pgfqpoint{3.463570in}{3.183198in}}%
\pgfpathclose%
\pgfusepath{fill}%
\end{pgfscope}%
\begin{pgfscope}%
\pgfpathrectangle{\pgfqpoint{1.254980in}{0.150000in}}{\pgfqpoint{5.490039in}{5.490039in}}%
\pgfusepath{clip}%
\pgfsetbuttcap%
\pgfsetroundjoin%
\definecolor{currentfill}{rgb}{0.187231,0.414746,0.556547}%
\pgfsetfillcolor{currentfill}%
\pgfsetfillopacity{0.700000}%
\pgfsetlinewidth{0.000000pt}%
\definecolor{currentstroke}{rgb}{0.000000,0.000000,0.000000}%
\pgfsetstrokecolor{currentstroke}%
\pgfsetdash{}{0pt}%
\pgfpathmoveto{\pgfqpoint{3.412122in}{3.244605in}}%
\pgfpathlineto{\pgfqpoint{3.424989in}{3.228873in}}%
\pgfpathlineto{\pgfqpoint{3.437853in}{3.213396in}}%
\pgfpathlineto{\pgfqpoint{3.450713in}{3.198172in}}%
\pgfpathlineto{\pgfqpoint{3.463570in}{3.183198in}}%
\pgfpathlineto{\pgfqpoint{3.471108in}{3.199048in}}%
\pgfpathlineto{\pgfqpoint{3.478640in}{3.215095in}}%
\pgfpathlineto{\pgfqpoint{3.486167in}{3.231343in}}%
\pgfpathlineto{\pgfqpoint{3.493688in}{3.247796in}}%
\pgfpathlineto{\pgfqpoint{3.480837in}{3.263105in}}%
\pgfpathlineto{\pgfqpoint{3.467982in}{3.278664in}}%
\pgfpathlineto{\pgfqpoint{3.455124in}{3.294476in}}%
\pgfpathlineto{\pgfqpoint{3.442263in}{3.310543in}}%
\pgfpathlineto{\pgfqpoint{3.434735in}{3.293744in}}%
\pgfpathlineto{\pgfqpoint{3.427203in}{3.277157in}}%
\pgfpathlineto{\pgfqpoint{3.419665in}{3.260779in}}%
\pgfpathlineto{\pgfqpoint{3.412122in}{3.244605in}}%
\pgfpathclose%
\pgfusepath{fill}%
\end{pgfscope}%
\begin{pgfscope}%
\pgfpathrectangle{\pgfqpoint{1.254980in}{0.150000in}}{\pgfqpoint{5.490039in}{5.490039in}}%
\pgfusepath{clip}%
\pgfsetbuttcap%
\pgfsetroundjoin%
\definecolor{currentfill}{rgb}{0.233603,0.313828,0.543914}%
\pgfsetfillcolor{currentfill}%
\pgfsetfillopacity{0.700000}%
\pgfsetlinewidth{0.000000pt}%
\definecolor{currentstroke}{rgb}{0.000000,0.000000,0.000000}%
\pgfsetstrokecolor{currentstroke}%
\pgfsetdash{}{0pt}%
\pgfpathmoveto{\pgfqpoint{4.096542in}{2.989571in}}%
\pgfpathlineto{\pgfqpoint{4.109400in}{2.981973in}}%
\pgfpathlineto{\pgfqpoint{4.122261in}{2.974565in}}%
\pgfpathlineto{\pgfqpoint{4.135126in}{2.967345in}}%
\pgfpathlineto{\pgfqpoint{4.147996in}{2.960312in}}%
\pgfpathlineto{\pgfqpoint{4.155404in}{2.975088in}}%
\pgfpathlineto{\pgfqpoint{4.162808in}{2.990055in}}%
\pgfpathlineto{\pgfqpoint{4.170210in}{3.005220in}}%
\pgfpathlineto{\pgfqpoint{4.177610in}{3.020587in}}%
\pgfpathlineto{\pgfqpoint{4.164747in}{3.028057in}}%
\pgfpathlineto{\pgfqpoint{4.151889in}{3.035715in}}%
\pgfpathlineto{\pgfqpoint{4.139034in}{3.043562in}}%
\pgfpathlineto{\pgfqpoint{4.126183in}{3.051598in}}%
\pgfpathlineto{\pgfqpoint{4.118777in}{3.035783in}}%
\pgfpathlineto{\pgfqpoint{4.111368in}{3.020176in}}%
\pgfpathlineto{\pgfqpoint{4.103956in}{3.004774in}}%
\pgfpathlineto{\pgfqpoint{4.096542in}{2.989571in}}%
\pgfpathclose%
\pgfusepath{fill}%
\end{pgfscope}%
\begin{pgfscope}%
\pgfpathrectangle{\pgfqpoint{1.254980in}{0.150000in}}{\pgfqpoint{5.490039in}{5.490039in}}%
\pgfusepath{clip}%
\pgfsetbuttcap%
\pgfsetroundjoin%
\definecolor{currentfill}{rgb}{0.208623,0.367752,0.552675}%
\pgfsetfillcolor{currentfill}%
\pgfsetfillopacity{0.700000}%
\pgfsetlinewidth{0.000000pt}%
\definecolor{currentstroke}{rgb}{0.000000,0.000000,0.000000}%
\pgfsetstrokecolor{currentstroke}%
\pgfsetdash{}{0pt}%
\pgfpathmoveto{\pgfqpoint{3.514971in}{3.125763in}}%
\pgfpathlineto{\pgfqpoint{3.527816in}{3.112010in}}%
\pgfpathlineto{\pgfqpoint{3.540658in}{3.098495in}}%
\pgfpathlineto{\pgfqpoint{3.553499in}{3.085218in}}%
\pgfpathlineto{\pgfqpoint{3.566338in}{3.072175in}}%
\pgfpathlineto{\pgfqpoint{3.573864in}{3.087380in}}%
\pgfpathlineto{\pgfqpoint{3.581384in}{3.102768in}}%
\pgfpathlineto{\pgfqpoint{3.588900in}{3.118342in}}%
\pgfpathlineto{\pgfqpoint{3.596411in}{3.134108in}}%
\pgfpathlineto{\pgfqpoint{3.583577in}{3.147483in}}%
\pgfpathlineto{\pgfqpoint{3.570742in}{3.161093in}}%
\pgfpathlineto{\pgfqpoint{3.557906in}{3.174940in}}%
\pgfpathlineto{\pgfqpoint{3.545067in}{3.189026in}}%
\pgfpathlineto{\pgfqpoint{3.537550in}{3.172918in}}%
\pgfpathlineto{\pgfqpoint{3.530029in}{3.157007in}}%
\pgfpathlineto{\pgfqpoint{3.522503in}{3.141290in}}%
\pgfpathlineto{\pgfqpoint{3.514971in}{3.125763in}}%
\pgfpathclose%
\pgfusepath{fill}%
\end{pgfscope}%
\begin{pgfscope}%
\pgfpathrectangle{\pgfqpoint{1.254980in}{0.150000in}}{\pgfqpoint{5.490039in}{5.490039in}}%
\pgfusepath{clip}%
\pgfsetbuttcap%
\pgfsetroundjoin%
\definecolor{currentfill}{rgb}{0.175841,0.441290,0.557685}%
\pgfsetfillcolor{currentfill}%
\pgfsetfillopacity{0.700000}%
\pgfsetlinewidth{0.000000pt}%
\definecolor{currentstroke}{rgb}{0.000000,0.000000,0.000000}%
\pgfsetstrokecolor{currentstroke}%
\pgfsetdash{}{0pt}%
\pgfpathmoveto{\pgfqpoint{3.360613in}{3.310119in}}%
\pgfpathlineto{\pgfqpoint{3.373497in}{3.293348in}}%
\pgfpathlineto{\pgfqpoint{3.386376in}{3.276840in}}%
\pgfpathlineto{\pgfqpoint{3.399251in}{3.260593in}}%
\pgfpathlineto{\pgfqpoint{3.412122in}{3.244605in}}%
\pgfpathlineto{\pgfqpoint{3.419665in}{3.260779in}}%
\pgfpathlineto{\pgfqpoint{3.427203in}{3.277157in}}%
\pgfpathlineto{\pgfqpoint{3.434735in}{3.293744in}}%
\pgfpathlineto{\pgfqpoint{3.442263in}{3.310543in}}%
\pgfpathlineto{\pgfqpoint{3.429397in}{3.326867in}}%
\pgfpathlineto{\pgfqpoint{3.416528in}{3.343451in}}%
\pgfpathlineto{\pgfqpoint{3.403654in}{3.360295in}}%
\pgfpathlineto{\pgfqpoint{3.390776in}{3.377404in}}%
\pgfpathlineto{\pgfqpoint{3.383243in}{3.360257in}}%
\pgfpathlineto{\pgfqpoint{3.375706in}{3.343330in}}%
\pgfpathlineto{\pgfqpoint{3.368162in}{3.326619in}}%
\pgfpathlineto{\pgfqpoint{3.360613in}{3.310119in}}%
\pgfpathclose%
\pgfusepath{fill}%
\end{pgfscope}%
\begin{pgfscope}%
\pgfpathrectangle{\pgfqpoint{1.254980in}{0.150000in}}{\pgfqpoint{5.490039in}{5.490039in}}%
\pgfusepath{clip}%
\pgfsetbuttcap%
\pgfsetroundjoin%
\definecolor{currentfill}{rgb}{0.239346,0.300855,0.540844}%
\pgfsetfillcolor{currentfill}%
\pgfsetfillopacity{0.700000}%
\pgfsetlinewidth{0.000000pt}%
\definecolor{currentstroke}{rgb}{0.000000,0.000000,0.000000}%
\pgfsetstrokecolor{currentstroke}%
\pgfsetdash{}{0pt}%
\pgfpathmoveto{\pgfqpoint{3.882910in}{2.971342in}}%
\pgfpathlineto{\pgfqpoint{3.895745in}{2.962181in}}%
\pgfpathlineto{\pgfqpoint{3.908583in}{2.953223in}}%
\pgfpathlineto{\pgfqpoint{3.921423in}{2.944465in}}%
\pgfpathlineto{\pgfqpoint{3.934265in}{2.935909in}}%
\pgfpathlineto{\pgfqpoint{3.941719in}{2.950440in}}%
\pgfpathlineto{\pgfqpoint{3.949170in}{2.965146in}}%
\pgfpathlineto{\pgfqpoint{3.956617in}{2.980031in}}%
\pgfpathlineto{\pgfqpoint{3.964060in}{2.995100in}}%
\pgfpathlineto{\pgfqpoint{3.951224in}{3.004041in}}%
\pgfpathlineto{\pgfqpoint{3.938390in}{3.013182in}}%
\pgfpathlineto{\pgfqpoint{3.925559in}{3.022524in}}%
\pgfpathlineto{\pgfqpoint{3.912730in}{3.032069in}}%
\pgfpathlineto{\pgfqpoint{3.905280in}{3.016606in}}%
\pgfpathlineto{\pgfqpoint{3.897827in}{3.001333in}}%
\pgfpathlineto{\pgfqpoint{3.890370in}{2.986247in}}%
\pgfpathlineto{\pgfqpoint{3.882910in}{2.971342in}}%
\pgfpathclose%
\pgfusepath{fill}%
\end{pgfscope}%
\begin{pgfscope}%
\pgfpathrectangle{\pgfqpoint{1.254980in}{0.150000in}}{\pgfqpoint{5.490039in}{5.490039in}}%
\pgfusepath{clip}%
\pgfsetbuttcap%
\pgfsetroundjoin%
\definecolor{currentfill}{rgb}{0.709898,0.868751,0.169257}%
\pgfsetfillcolor{currentfill}%
\pgfsetfillopacity{0.700000}%
\pgfsetlinewidth{0.000000pt}%
\definecolor{currentstroke}{rgb}{0.000000,0.000000,0.000000}%
\pgfsetstrokecolor{currentstroke}%
\pgfsetdash{}{0pt}%
\pgfpathmoveto{\pgfqpoint{3.451947in}{4.565874in}}%
\pgfpathlineto{\pgfqpoint{3.464933in}{4.539652in}}%
\pgfpathlineto{\pgfqpoint{3.477910in}{4.513746in}}%
\pgfpathlineto{\pgfqpoint{3.490878in}{4.488154in}}%
\pgfpathlineto{\pgfqpoint{3.503838in}{4.462872in}}%
\pgfpathlineto{\pgfqpoint{3.511198in}{4.493770in}}%
\pgfpathlineto{\pgfqpoint{3.518555in}{4.525137in}}%
\pgfpathlineto{\pgfqpoint{3.525909in}{4.556981in}}%
\pgfpathlineto{\pgfqpoint{3.533259in}{4.589310in}}%
\pgfpathlineto{\pgfqpoint{3.520293in}{4.615263in}}%
\pgfpathlineto{\pgfqpoint{3.507318in}{4.641529in}}%
\pgfpathlineto{\pgfqpoint{3.494334in}{4.668110in}}%
\pgfpathlineto{\pgfqpoint{3.481341in}{4.695010in}}%
\pgfpathlineto{\pgfqpoint{3.473998in}{4.661993in}}%
\pgfpathlineto{\pgfqpoint{3.466651in}{4.629470in}}%
\pgfpathlineto{\pgfqpoint{3.459301in}{4.597433in}}%
\pgfpathlineto{\pgfqpoint{3.451947in}{4.565874in}}%
\pgfpathclose%
\pgfusepath{fill}%
\end{pgfscope}%
\begin{pgfscope}%
\pgfpathrectangle{\pgfqpoint{1.254980in}{0.150000in}}{\pgfqpoint{5.490039in}{5.490039in}}%
\pgfusepath{clip}%
\pgfsetbuttcap%
\pgfsetroundjoin%
\definecolor{currentfill}{rgb}{0.210503,0.363727,0.552206}%
\pgfsetfillcolor{currentfill}%
\pgfsetfillopacity{0.700000}%
\pgfsetlinewidth{0.000000pt}%
\definecolor{currentstroke}{rgb}{0.000000,0.000000,0.000000}%
\pgfsetstrokecolor{currentstroke}%
\pgfsetdash{}{0pt}%
\pgfpathmoveto{\pgfqpoint{4.472328in}{3.104448in}}%
\pgfpathlineto{\pgfqpoint{4.485253in}{3.098563in}}%
\pgfpathlineto{\pgfqpoint{4.498183in}{3.092851in}}%
\pgfpathlineto{\pgfqpoint{4.511119in}{3.087312in}}%
\pgfpathlineto{\pgfqpoint{4.524062in}{3.081944in}}%
\pgfpathlineto{\pgfqpoint{4.531400in}{3.097766in}}%
\pgfpathlineto{\pgfqpoint{4.538737in}{3.113837in}}%
\pgfpathlineto{\pgfqpoint{4.546074in}{3.130163in}}%
\pgfpathlineto{\pgfqpoint{4.553411in}{3.146751in}}%
\pgfpathlineto{\pgfqpoint{4.540478in}{3.152666in}}%
\pgfpathlineto{\pgfqpoint{4.527551in}{3.158753in}}%
\pgfpathlineto{\pgfqpoint{4.514630in}{3.165012in}}%
\pgfpathlineto{\pgfqpoint{4.501714in}{3.171445in}}%
\pgfpathlineto{\pgfqpoint{4.494368in}{3.154299in}}%
\pgfpathlineto{\pgfqpoint{4.487022in}{3.137421in}}%
\pgfpathlineto{\pgfqpoint{4.479675in}{3.120807in}}%
\pgfpathlineto{\pgfqpoint{4.472328in}{3.104448in}}%
\pgfpathclose%
\pgfusepath{fill}%
\end{pgfscope}%
\begin{pgfscope}%
\pgfpathrectangle{\pgfqpoint{1.254980in}{0.150000in}}{\pgfqpoint{5.490039in}{5.490039in}}%
\pgfusepath{clip}%
\pgfsetbuttcap%
\pgfsetroundjoin%
\definecolor{currentfill}{rgb}{0.201239,0.383670,0.554294}%
\pgfsetfillcolor{currentfill}%
\pgfsetfillopacity{0.700000}%
\pgfsetlinewidth{0.000000pt}%
\definecolor{currentstroke}{rgb}{0.000000,0.000000,0.000000}%
\pgfsetstrokecolor{currentstroke}%
\pgfsetdash{}{0pt}%
\pgfpathmoveto{\pgfqpoint{4.553411in}{3.146751in}}%
\pgfpathlineto{\pgfqpoint{4.566351in}{3.141008in}}%
\pgfpathlineto{\pgfqpoint{4.579296in}{3.135435in}}%
\pgfpathlineto{\pgfqpoint{4.592248in}{3.130032in}}%
\pgfpathlineto{\pgfqpoint{4.605207in}{3.124799in}}%
\pgfpathlineto{\pgfqpoint{4.612534in}{3.141091in}}%
\pgfpathlineto{\pgfqpoint{4.619862in}{3.157651in}}%
\pgfpathlineto{\pgfqpoint{4.627190in}{3.174485in}}%
\pgfpathlineto{\pgfqpoint{4.634518in}{3.191602in}}%
\pgfpathlineto{\pgfqpoint{4.621570in}{3.197410in}}%
\pgfpathlineto{\pgfqpoint{4.608627in}{3.203388in}}%
\pgfpathlineto{\pgfqpoint{4.595691in}{3.209536in}}%
\pgfpathlineto{\pgfqpoint{4.582761in}{3.215854in}}%
\pgfpathlineto{\pgfqpoint{4.575423in}{3.198152in}}%
\pgfpathlineto{\pgfqpoint{4.568086in}{3.180739in}}%
\pgfpathlineto{\pgfqpoint{4.560748in}{3.163608in}}%
\pgfpathlineto{\pgfqpoint{4.553411in}{3.146751in}}%
\pgfpathclose%
\pgfusepath{fill}%
\end{pgfscope}%
\begin{pgfscope}%
\pgfpathrectangle{\pgfqpoint{1.254980in}{0.150000in}}{\pgfqpoint{5.490039in}{5.490039in}}%
\pgfusepath{clip}%
\pgfsetbuttcap%
\pgfsetroundjoin%
\definecolor{currentfill}{rgb}{0.235526,0.309527,0.542944}%
\pgfsetfillcolor{currentfill}%
\pgfsetfillopacity{0.700000}%
\pgfsetlinewidth{0.000000pt}%
\definecolor{currentstroke}{rgb}{0.000000,0.000000,0.000000}%
\pgfsetstrokecolor{currentstroke}%
\pgfsetdash{}{0pt}%
\pgfpathmoveto{\pgfqpoint{3.750356in}{2.991326in}}%
\pgfpathlineto{\pgfqpoint{3.763186in}{2.980856in}}%
\pgfpathlineto{\pgfqpoint{3.776016in}{2.970599in}}%
\pgfpathlineto{\pgfqpoint{3.788848in}{2.960553in}}%
\pgfpathlineto{\pgfqpoint{3.801681in}{2.950718in}}%
\pgfpathlineto{\pgfqpoint{3.809164in}{2.965286in}}%
\pgfpathlineto{\pgfqpoint{3.816642in}{2.980026in}}%
\pgfpathlineto{\pgfqpoint{3.824117in}{2.994942in}}%
\pgfpathlineto{\pgfqpoint{3.831587in}{3.010037in}}%
\pgfpathlineto{\pgfqpoint{3.818760in}{3.020229in}}%
\pgfpathlineto{\pgfqpoint{3.805935in}{3.030632in}}%
\pgfpathlineto{\pgfqpoint{3.793110in}{3.041247in}}%
\pgfpathlineto{\pgfqpoint{3.780286in}{3.052074in}}%
\pgfpathlineto{\pgfqpoint{3.772810in}{3.036612in}}%
\pgfpathlineto{\pgfqpoint{3.765330in}{3.021336in}}%
\pgfpathlineto{\pgfqpoint{3.757845in}{3.006242in}}%
\pgfpathlineto{\pgfqpoint{3.750356in}{2.991326in}}%
\pgfpathclose%
\pgfusepath{fill}%
\end{pgfscope}%
\begin{pgfscope}%
\pgfpathrectangle{\pgfqpoint{1.254980in}{0.150000in}}{\pgfqpoint{5.490039in}{5.490039in}}%
\pgfusepath{clip}%
\pgfsetbuttcap%
\pgfsetroundjoin%
\definecolor{currentfill}{rgb}{0.218130,0.347432,0.550038}%
\pgfsetfillcolor{currentfill}%
\pgfsetfillopacity{0.700000}%
\pgfsetlinewidth{0.000000pt}%
\definecolor{currentstroke}{rgb}{0.000000,0.000000,0.000000}%
\pgfsetstrokecolor{currentstroke}%
\pgfsetdash{}{0pt}%
\pgfpathmoveto{\pgfqpoint{4.391257in}{3.064655in}}%
\pgfpathlineto{\pgfqpoint{4.404167in}{3.058589in}}%
\pgfpathlineto{\pgfqpoint{4.417083in}{3.052700in}}%
\pgfpathlineto{\pgfqpoint{4.430004in}{3.046986in}}%
\pgfpathlineto{\pgfqpoint{4.442931in}{3.041447in}}%
\pgfpathlineto{\pgfqpoint{4.450282in}{3.056844in}}%
\pgfpathlineto{\pgfqpoint{4.457632in}{3.072473in}}%
\pgfpathlineto{\pgfqpoint{4.464981in}{3.088338in}}%
\pgfpathlineto{\pgfqpoint{4.472328in}{3.104448in}}%
\pgfpathlineto{\pgfqpoint{4.459410in}{3.110507in}}%
\pgfpathlineto{\pgfqpoint{4.446497in}{3.116741in}}%
\pgfpathlineto{\pgfqpoint{4.433590in}{3.123150in}}%
\pgfpathlineto{\pgfqpoint{4.420688in}{3.129736in}}%
\pgfpathlineto{\pgfqpoint{4.413332in}{3.113096in}}%
\pgfpathlineto{\pgfqpoint{4.405975in}{3.096707in}}%
\pgfpathlineto{\pgfqpoint{4.398617in}{3.080562in}}%
\pgfpathlineto{\pgfqpoint{4.391257in}{3.064655in}}%
\pgfpathclose%
\pgfusepath{fill}%
\end{pgfscope}%
\begin{pgfscope}%
\pgfpathrectangle{\pgfqpoint{1.254980in}{0.150000in}}{\pgfqpoint{5.490039in}{5.490039in}}%
\pgfusepath{clip}%
\pgfsetbuttcap%
\pgfsetroundjoin%
\definecolor{currentfill}{rgb}{0.220057,0.343307,0.549413}%
\pgfsetfillcolor{currentfill}%
\pgfsetfillopacity{0.700000}%
\pgfsetlinewidth{0.000000pt}%
\definecolor{currentstroke}{rgb}{0.000000,0.000000,0.000000}%
\pgfsetstrokecolor{currentstroke}%
\pgfsetdash{}{0pt}%
\pgfpathmoveto{\pgfqpoint{3.566338in}{3.072175in}}%
\pgfpathlineto{\pgfqpoint{3.579176in}{3.059365in}}%
\pgfpathlineto{\pgfqpoint{3.592013in}{3.046786in}}%
\pgfpathlineto{\pgfqpoint{3.604849in}{3.034437in}}%
\pgfpathlineto{\pgfqpoint{3.617684in}{3.022317in}}%
\pgfpathlineto{\pgfqpoint{3.625203in}{3.037201in}}%
\pgfpathlineto{\pgfqpoint{3.632717in}{3.052260in}}%
\pgfpathlineto{\pgfqpoint{3.640227in}{3.067500in}}%
\pgfpathlineto{\pgfqpoint{3.647732in}{3.082923in}}%
\pgfpathlineto{\pgfqpoint{3.634903in}{3.095376in}}%
\pgfpathlineto{\pgfqpoint{3.622074in}{3.108056in}}%
\pgfpathlineto{\pgfqpoint{3.609243in}{3.120966in}}%
\pgfpathlineto{\pgfqpoint{3.596411in}{3.134108in}}%
\pgfpathlineto{\pgfqpoint{3.588900in}{3.118342in}}%
\pgfpathlineto{\pgfqpoint{3.581384in}{3.102768in}}%
\pgfpathlineto{\pgfqpoint{3.573864in}{3.087380in}}%
\pgfpathlineto{\pgfqpoint{3.566338in}{3.072175in}}%
\pgfpathclose%
\pgfusepath{fill}%
\end{pgfscope}%
\begin{pgfscope}%
\pgfpathrectangle{\pgfqpoint{1.254980in}{0.150000in}}{\pgfqpoint{5.490039in}{5.490039in}}%
\pgfusepath{clip}%
\pgfsetbuttcap%
\pgfsetroundjoin%
\definecolor{currentfill}{rgb}{0.192357,0.403199,0.555836}%
\pgfsetfillcolor{currentfill}%
\pgfsetfillopacity{0.700000}%
\pgfsetlinewidth{0.000000pt}%
\definecolor{currentstroke}{rgb}{0.000000,0.000000,0.000000}%
\pgfsetstrokecolor{currentstroke}%
\pgfsetdash{}{0pt}%
\pgfpathmoveto{\pgfqpoint{4.634518in}{3.191602in}}%
\pgfpathlineto{\pgfqpoint{4.647474in}{3.185962in}}%
\pgfpathlineto{\pgfqpoint{4.660435in}{3.180491in}}%
\pgfpathlineto{\pgfqpoint{4.673404in}{3.175187in}}%
\pgfpathlineto{\pgfqpoint{4.686379in}{3.170050in}}%
\pgfpathlineto{\pgfqpoint{4.693698in}{3.186862in}}%
\pgfpathlineto{\pgfqpoint{4.701018in}{3.203963in}}%
\pgfpathlineto{\pgfqpoint{4.708339in}{3.221360in}}%
\pgfpathlineto{\pgfqpoint{4.715662in}{3.239059in}}%
\pgfpathlineto{\pgfqpoint{4.702698in}{3.244798in}}%
\pgfpathlineto{\pgfqpoint{4.689740in}{3.250705in}}%
\pgfpathlineto{\pgfqpoint{4.676788in}{3.256779in}}%
\pgfpathlineto{\pgfqpoint{4.663843in}{3.263022in}}%
\pgfpathlineto{\pgfqpoint{4.656510in}{3.244709in}}%
\pgfpathlineto{\pgfqpoint{4.649178in}{3.226707in}}%
\pgfpathlineto{\pgfqpoint{4.641848in}{3.209006in}}%
\pgfpathlineto{\pgfqpoint{4.634518in}{3.191602in}}%
\pgfpathclose%
\pgfusepath{fill}%
\end{pgfscope}%
\begin{pgfscope}%
\pgfpathrectangle{\pgfqpoint{1.254980in}{0.150000in}}{\pgfqpoint{5.490039in}{5.490039in}}%
\pgfusepath{clip}%
\pgfsetbuttcap%
\pgfsetroundjoin%
\definecolor{currentfill}{rgb}{0.165117,0.467423,0.558141}%
\pgfsetfillcolor{currentfill}%
\pgfsetfillopacity{0.700000}%
\pgfsetlinewidth{0.000000pt}%
\definecolor{currentstroke}{rgb}{0.000000,0.000000,0.000000}%
\pgfsetstrokecolor{currentstroke}%
\pgfsetdash{}{0pt}%
\pgfpathmoveto{\pgfqpoint{3.309030in}{3.379886in}}%
\pgfpathlineto{\pgfqpoint{3.321933in}{3.362037in}}%
\pgfpathlineto{\pgfqpoint{3.334832in}{3.344461in}}%
\pgfpathlineto{\pgfqpoint{3.347725in}{3.327156in}}%
\pgfpathlineto{\pgfqpoint{3.360613in}{3.310119in}}%
\pgfpathlineto{\pgfqpoint{3.368162in}{3.326619in}}%
\pgfpathlineto{\pgfqpoint{3.375706in}{3.343330in}}%
\pgfpathlineto{\pgfqpoint{3.383243in}{3.360257in}}%
\pgfpathlineto{\pgfqpoint{3.390776in}{3.377404in}}%
\pgfpathlineto{\pgfqpoint{3.377893in}{3.394779in}}%
\pgfpathlineto{\pgfqpoint{3.365005in}{3.412422in}}%
\pgfpathlineto{\pgfqpoint{3.352112in}{3.430336in}}%
\pgfpathlineto{\pgfqpoint{3.339214in}{3.448524in}}%
\pgfpathlineto{\pgfqpoint{3.331677in}{3.431028in}}%
\pgfpathlineto{\pgfqpoint{3.324133in}{3.413759in}}%
\pgfpathlineto{\pgfqpoint{3.316584in}{3.396713in}}%
\pgfpathlineto{\pgfqpoint{3.309030in}{3.379886in}}%
\pgfpathclose%
\pgfusepath{fill}%
\end{pgfscope}%
\begin{pgfscope}%
\pgfpathrectangle{\pgfqpoint{1.254980in}{0.150000in}}{\pgfqpoint{5.490039in}{5.490039in}}%
\pgfusepath{clip}%
\pgfsetbuttcap%
\pgfsetroundjoin%
\definecolor{currentfill}{rgb}{0.119699,0.618490,0.536347}%
\pgfsetfillcolor{currentfill}%
\pgfsetfillopacity{0.700000}%
\pgfsetlinewidth{0.000000pt}%
\definecolor{currentstroke}{rgb}{0.000000,0.000000,0.000000}%
\pgfsetstrokecolor{currentstroke}%
\pgfsetdash{}{0pt}%
\pgfpathmoveto{\pgfqpoint{3.214073in}{3.765871in}}%
\pgfpathlineto{\pgfqpoint{3.227050in}{3.743817in}}%
\pgfpathlineto{\pgfqpoint{3.240020in}{3.722070in}}%
\pgfpathlineto{\pgfqpoint{3.252981in}{3.700628in}}%
\pgfpathlineto{\pgfqpoint{3.265934in}{3.679487in}}%
\pgfpathlineto{\pgfqpoint{3.273451in}{3.698962in}}%
\pgfpathlineto{\pgfqpoint{3.280962in}{3.718706in}}%
\pgfpathlineto{\pgfqpoint{3.288467in}{3.738723in}}%
\pgfpathlineto{\pgfqpoint{3.295966in}{3.759019in}}%
\pgfpathlineto{\pgfqpoint{3.283016in}{3.780564in}}%
\pgfpathlineto{\pgfqpoint{3.270059in}{3.802412in}}%
\pgfpathlineto{\pgfqpoint{3.257093in}{3.824565in}}%
\pgfpathlineto{\pgfqpoint{3.244118in}{3.847026in}}%
\pgfpathlineto{\pgfqpoint{3.236616in}{3.826313in}}%
\pgfpathlineto{\pgfqpoint{3.229108in}{3.805886in}}%
\pgfpathlineto{\pgfqpoint{3.221593in}{3.785740in}}%
\pgfpathlineto{\pgfqpoint{3.214073in}{3.765871in}}%
\pgfpathclose%
\pgfusepath{fill}%
\end{pgfscope}%
\begin{pgfscope}%
\pgfpathrectangle{\pgfqpoint{1.254980in}{0.150000in}}{\pgfqpoint{5.490039in}{5.490039in}}%
\pgfusepath{clip}%
\pgfsetbuttcap%
\pgfsetroundjoin%
\definecolor{currentfill}{rgb}{0.225863,0.330805,0.547314}%
\pgfsetfillcolor{currentfill}%
\pgfsetfillopacity{0.700000}%
\pgfsetlinewidth{0.000000pt}%
\definecolor{currentstroke}{rgb}{0.000000,0.000000,0.000000}%
\pgfsetstrokecolor{currentstroke}%
\pgfsetdash{}{0pt}%
\pgfpathmoveto{\pgfqpoint{4.310186in}{3.027358in}}%
\pgfpathlineto{\pgfqpoint{4.323083in}{3.021073in}}%
\pgfpathlineto{\pgfqpoint{4.335985in}{3.014967in}}%
\pgfpathlineto{\pgfqpoint{4.348892in}{3.009039in}}%
\pgfpathlineto{\pgfqpoint{4.361805in}{3.003289in}}%
\pgfpathlineto{\pgfqpoint{4.369170in}{3.018303in}}%
\pgfpathlineto{\pgfqpoint{4.376534in}{3.033531in}}%
\pgfpathlineto{\pgfqpoint{4.383896in}{3.048980in}}%
\pgfpathlineto{\pgfqpoint{4.391257in}{3.064655in}}%
\pgfpathlineto{\pgfqpoint{4.378353in}{3.070897in}}%
\pgfpathlineto{\pgfqpoint{4.365453in}{3.077317in}}%
\pgfpathlineto{\pgfqpoint{4.352559in}{3.083915in}}%
\pgfpathlineto{\pgfqpoint{4.339670in}{3.090693in}}%
\pgfpathlineto{\pgfqpoint{4.332302in}{3.074515in}}%
\pgfpathlineto{\pgfqpoint{4.324932in}{3.058571in}}%
\pgfpathlineto{\pgfqpoint{4.317560in}{3.042853in}}%
\pgfpathlineto{\pgfqpoint{4.310186in}{3.027358in}}%
\pgfpathclose%
\pgfusepath{fill}%
\end{pgfscope}%
\begin{pgfscope}%
\pgfpathrectangle{\pgfqpoint{1.254980in}{0.150000in}}{\pgfqpoint{5.490039in}{5.490039in}}%
\pgfusepath{clip}%
\pgfsetbuttcap%
\pgfsetroundjoin%
\definecolor{currentfill}{rgb}{0.131172,0.555899,0.552459}%
\pgfsetfillcolor{currentfill}%
\pgfsetfillopacity{0.700000}%
\pgfsetlinewidth{0.000000pt}%
\definecolor{currentstroke}{rgb}{0.000000,0.000000,0.000000}%
\pgfsetstrokecolor{currentstroke}%
\pgfsetdash{}{0pt}%
\pgfpathmoveto{\pgfqpoint{3.235805in}{3.604179in}}%
\pgfpathlineto{\pgfqpoint{3.248755in}{3.583710in}}%
\pgfpathlineto{\pgfqpoint{3.261698in}{3.563535in}}%
\pgfpathlineto{\pgfqpoint{3.274634in}{3.543653in}}%
\pgfpathlineto{\pgfqpoint{3.287562in}{3.524060in}}%
\pgfpathlineto{\pgfqpoint{3.295099in}{3.542145in}}%
\pgfpathlineto{\pgfqpoint{3.302630in}{3.560474in}}%
\pgfpathlineto{\pgfqpoint{3.310154in}{3.579050in}}%
\pgfpathlineto{\pgfqpoint{3.317673in}{3.597878in}}%
\pgfpathlineto{\pgfqpoint{3.304749in}{3.617842in}}%
\pgfpathlineto{\pgfqpoint{3.291818in}{3.638097in}}%
\pgfpathlineto{\pgfqpoint{3.278880in}{3.658644in}}%
\pgfpathlineto{\pgfqpoint{3.265934in}{3.679487in}}%
\pgfpathlineto{\pgfqpoint{3.258411in}{3.660275in}}%
\pgfpathlineto{\pgfqpoint{3.250882in}{3.641323in}}%
\pgfpathlineto{\pgfqpoint{3.243347in}{3.622626in}}%
\pgfpathlineto{\pgfqpoint{3.235805in}{3.604179in}}%
\pgfpathclose%
\pgfusepath{fill}%
\end{pgfscope}%
\begin{pgfscope}%
\pgfpathrectangle{\pgfqpoint{1.254980in}{0.150000in}}{\pgfqpoint{5.490039in}{5.490039in}}%
\pgfusepath{clip}%
\pgfsetbuttcap%
\pgfsetroundjoin%
\definecolor{currentfill}{rgb}{0.239346,0.300855,0.540844}%
\pgfsetfillcolor{currentfill}%
\pgfsetfillopacity{0.700000}%
\pgfsetlinewidth{0.000000pt}%
\definecolor{currentstroke}{rgb}{0.000000,0.000000,0.000000}%
\pgfsetstrokecolor{currentstroke}%
\pgfsetdash{}{0pt}%
\pgfpathmoveto{\pgfqpoint{4.015431in}{2.961318in}}%
\pgfpathlineto{\pgfqpoint{4.028282in}{2.953362in}}%
\pgfpathlineto{\pgfqpoint{4.041135in}{2.945599in}}%
\pgfpathlineto{\pgfqpoint{4.053993in}{2.938029in}}%
\pgfpathlineto{\pgfqpoint{4.066853in}{2.930651in}}%
\pgfpathlineto{\pgfqpoint{4.074280in}{2.945107in}}%
\pgfpathlineto{\pgfqpoint{4.081704in}{2.959743in}}%
\pgfpathlineto{\pgfqpoint{4.089124in}{2.974563in}}%
\pgfpathlineto{\pgfqpoint{4.096542in}{2.989571in}}%
\pgfpathlineto{\pgfqpoint{4.083688in}{2.997360in}}%
\pgfpathlineto{\pgfqpoint{4.070837in}{3.005341in}}%
\pgfpathlineto{\pgfqpoint{4.057990in}{3.013514in}}%
\pgfpathlineto{\pgfqpoint{4.045146in}{3.021881in}}%
\pgfpathlineto{\pgfqpoint{4.037722in}{3.006451in}}%
\pgfpathlineto{\pgfqpoint{4.030295in}{2.991217in}}%
\pgfpathlineto{\pgfqpoint{4.022865in}{2.976174in}}%
\pgfpathlineto{\pgfqpoint{4.015431in}{2.961318in}}%
\pgfpathclose%
\pgfusepath{fill}%
\end{pgfscope}%
\begin{pgfscope}%
\pgfpathrectangle{\pgfqpoint{1.254980in}{0.150000in}}{\pgfqpoint{5.490039in}{5.490039in}}%
\pgfusepath{clip}%
\pgfsetbuttcap%
\pgfsetroundjoin%
\definecolor{currentfill}{rgb}{0.183898,0.422383,0.556944}%
\pgfsetfillcolor{currentfill}%
\pgfsetfillopacity{0.700000}%
\pgfsetlinewidth{0.000000pt}%
\definecolor{currentstroke}{rgb}{0.000000,0.000000,0.000000}%
\pgfsetstrokecolor{currentstroke}%
\pgfsetdash{}{0pt}%
\pgfpathmoveto{\pgfqpoint{4.715662in}{3.239059in}}%
\pgfpathlineto{\pgfqpoint{4.728634in}{3.233485in}}%
\pgfpathlineto{\pgfqpoint{4.741612in}{3.228078in}}%
\pgfpathlineto{\pgfqpoint{4.754597in}{3.222836in}}%
\pgfpathlineto{\pgfqpoint{4.767589in}{3.217759in}}%
\pgfpathlineto{\pgfqpoint{4.774903in}{3.235147in}}%
\pgfpathlineto{\pgfqpoint{4.782218in}{3.252846in}}%
\pgfpathlineto{\pgfqpoint{4.789536in}{3.270863in}}%
\pgfpathlineto{\pgfqpoint{4.796856in}{3.289204in}}%
\pgfpathlineto{\pgfqpoint{4.783875in}{3.294912in}}%
\pgfpathlineto{\pgfqpoint{4.770901in}{3.300785in}}%
\pgfpathlineto{\pgfqpoint{4.757934in}{3.306823in}}%
\pgfpathlineto{\pgfqpoint{4.744973in}{3.313027in}}%
\pgfpathlineto{\pgfqpoint{4.737642in}{3.294044in}}%
\pgfpathlineto{\pgfqpoint{4.730313in}{3.275393in}}%
\pgfpathlineto{\pgfqpoint{4.722987in}{3.257067in}}%
\pgfpathlineto{\pgfqpoint{4.715662in}{3.239059in}}%
\pgfpathclose%
\pgfusepath{fill}%
\end{pgfscope}%
\begin{pgfscope}%
\pgfpathrectangle{\pgfqpoint{1.254980in}{0.150000in}}{\pgfqpoint{5.490039in}{5.490039in}}%
\pgfusepath{clip}%
\pgfsetbuttcap%
\pgfsetroundjoin%
\definecolor{currentfill}{rgb}{0.876168,0.891125,0.095250}%
\pgfsetfillcolor{currentfill}%
\pgfsetfillopacity{0.700000}%
\pgfsetlinewidth{0.000000pt}%
\definecolor{currentstroke}{rgb}{0.000000,0.000000,0.000000}%
\pgfsetstrokecolor{currentstroke}%
\pgfsetdash{}{0pt}%
\pgfpathmoveto{\pgfqpoint{3.562632in}{4.723643in}}%
\pgfpathlineto{\pgfqpoint{3.575598in}{4.697291in}}%
\pgfpathlineto{\pgfqpoint{3.588555in}{4.671248in}}%
\pgfpathlineto{\pgfqpoint{3.601505in}{4.645509in}}%
\pgfpathlineto{\pgfqpoint{3.614446in}{4.620071in}}%
\pgfpathlineto{\pgfqpoint{3.621791in}{4.654224in}}%
\pgfpathlineto{\pgfqpoint{3.629135in}{4.688903in}}%
\pgfpathlineto{\pgfqpoint{3.636476in}{4.724119in}}%
\pgfpathlineto{\pgfqpoint{3.623528in}{4.750107in}}%
\pgfpathlineto{\pgfqpoint{3.610572in}{4.776399in}}%
\pgfpathlineto{\pgfqpoint{3.597608in}{4.802997in}}%
\pgfpathlineto{\pgfqpoint{3.584635in}{4.829905in}}%
\pgfpathlineto{\pgfqpoint{3.577303in}{4.793942in}}%
\pgfpathlineto{\pgfqpoint{3.569969in}{4.758524in}}%
\pgfpathlineto{\pgfqpoint{3.562632in}{4.723643in}}%
\pgfpathclose%
\pgfusepath{fill}%
\end{pgfscope}%
\begin{pgfscope}%
\pgfpathrectangle{\pgfqpoint{1.254980in}{0.150000in}}{\pgfqpoint{5.490039in}{5.490039in}}%
\pgfusepath{clip}%
\pgfsetbuttcap%
\pgfsetroundjoin%
\definecolor{currentfill}{rgb}{0.227802,0.326594,0.546532}%
\pgfsetfillcolor{currentfill}%
\pgfsetfillopacity{0.700000}%
\pgfsetlinewidth{0.000000pt}%
\definecolor{currentstroke}{rgb}{0.000000,0.000000,0.000000}%
\pgfsetstrokecolor{currentstroke}%
\pgfsetdash{}{0pt}%
\pgfpathmoveto{\pgfqpoint{3.617684in}{3.022317in}}%
\pgfpathlineto{\pgfqpoint{3.630518in}{3.010422in}}%
\pgfpathlineto{\pgfqpoint{3.643352in}{2.998752in}}%
\pgfpathlineto{\pgfqpoint{3.656186in}{2.987305in}}%
\pgfpathlineto{\pgfqpoint{3.669019in}{2.976079in}}%
\pgfpathlineto{\pgfqpoint{3.676532in}{2.990643in}}%
\pgfpathlineto{\pgfqpoint{3.684041in}{3.005376in}}%
\pgfpathlineto{\pgfqpoint{3.691544in}{3.020281in}}%
\pgfpathlineto{\pgfqpoint{3.699044in}{3.035363in}}%
\pgfpathlineto{\pgfqpoint{3.686216in}{3.046919in}}%
\pgfpathlineto{\pgfqpoint{3.673389in}{3.058697in}}%
\pgfpathlineto{\pgfqpoint{3.660561in}{3.070698in}}%
\pgfpathlineto{\pgfqpoint{3.647732in}{3.082923in}}%
\pgfpathlineto{\pgfqpoint{3.640227in}{3.067500in}}%
\pgfpathlineto{\pgfqpoint{3.632717in}{3.052260in}}%
\pgfpathlineto{\pgfqpoint{3.625203in}{3.037201in}}%
\pgfpathlineto{\pgfqpoint{3.617684in}{3.022317in}}%
\pgfpathclose%
\pgfusepath{fill}%
\end{pgfscope}%
\begin{pgfscope}%
\pgfpathrectangle{\pgfqpoint{1.254980in}{0.150000in}}{\pgfqpoint{5.490039in}{5.490039in}}%
\pgfusepath{clip}%
\pgfsetbuttcap%
\pgfsetroundjoin%
\definecolor{currentfill}{rgb}{0.231674,0.318106,0.544834}%
\pgfsetfillcolor{currentfill}%
\pgfsetfillopacity{0.700000}%
\pgfsetlinewidth{0.000000pt}%
\definecolor{currentstroke}{rgb}{0.000000,0.000000,0.000000}%
\pgfsetstrokecolor{currentstroke}%
\pgfsetdash{}{0pt}%
\pgfpathmoveto{\pgfqpoint{4.229103in}{2.992566in}}%
\pgfpathlineto{\pgfqpoint{4.241987in}{2.986022in}}%
\pgfpathlineto{\pgfqpoint{4.254877in}{2.979659in}}%
\pgfpathlineto{\pgfqpoint{4.267771in}{2.973478in}}%
\pgfpathlineto{\pgfqpoint{4.280671in}{2.967478in}}%
\pgfpathlineto{\pgfqpoint{4.288053in}{2.982144in}}%
\pgfpathlineto{\pgfqpoint{4.295433in}{2.997008in}}%
\pgfpathlineto{\pgfqpoint{4.302810in}{3.012078in}}%
\pgfpathlineto{\pgfqpoint{4.310186in}{3.027358in}}%
\pgfpathlineto{\pgfqpoint{4.297294in}{3.033823in}}%
\pgfpathlineto{\pgfqpoint{4.284408in}{3.040469in}}%
\pgfpathlineto{\pgfqpoint{4.271526in}{3.047296in}}%
\pgfpathlineto{\pgfqpoint{4.258649in}{3.054306in}}%
\pgfpathlineto{\pgfqpoint{4.251265in}{3.038551in}}%
\pgfpathlineto{\pgfqpoint{4.243880in}{3.023013in}}%
\pgfpathlineto{\pgfqpoint{4.236493in}{3.007686in}}%
\pgfpathlineto{\pgfqpoint{4.229103in}{2.992566in}}%
\pgfpathclose%
\pgfusepath{fill}%
\end{pgfscope}%
\begin{pgfscope}%
\pgfpathrectangle{\pgfqpoint{1.254980in}{0.150000in}}{\pgfqpoint{5.490039in}{5.490039in}}%
\pgfusepath{clip}%
\pgfsetbuttcap%
\pgfsetroundjoin%
\definecolor{currentfill}{rgb}{0.175841,0.441290,0.557685}%
\pgfsetfillcolor{currentfill}%
\pgfsetfillopacity{0.700000}%
\pgfsetlinewidth{0.000000pt}%
\definecolor{currentstroke}{rgb}{0.000000,0.000000,0.000000}%
\pgfsetstrokecolor{currentstroke}%
\pgfsetdash{}{0pt}%
\pgfpathmoveto{\pgfqpoint{4.796856in}{3.289204in}}%
\pgfpathlineto{\pgfqpoint{4.809844in}{3.283661in}}%
\pgfpathlineto{\pgfqpoint{4.822839in}{3.278281in}}%
\pgfpathlineto{\pgfqpoint{4.835841in}{3.273064in}}%
\pgfpathlineto{\pgfqpoint{4.848850in}{3.268010in}}%
\pgfpathlineto{\pgfqpoint{4.856162in}{3.286036in}}%
\pgfpathlineto{\pgfqpoint{4.863476in}{3.304395in}}%
\pgfpathlineto{\pgfqpoint{4.870793in}{3.323095in}}%
\pgfpathlineto{\pgfqpoint{4.857793in}{3.328641in}}%
\pgfpathlineto{\pgfqpoint{4.844799in}{3.334349in}}%
\pgfpathlineto{\pgfqpoint{4.831813in}{3.340220in}}%
\pgfpathlineto{\pgfqpoint{4.818834in}{3.346255in}}%
\pgfpathlineto{\pgfqpoint{4.811505in}{3.326893in}}%
\pgfpathlineto{\pgfqpoint{4.804179in}{3.307878in}}%
\pgfpathlineto{\pgfqpoint{4.796856in}{3.289204in}}%
\pgfpathclose%
\pgfusepath{fill}%
\end{pgfscope}%
\begin{pgfscope}%
\pgfpathrectangle{\pgfqpoint{1.254980in}{0.150000in}}{\pgfqpoint{5.490039in}{5.490039in}}%
\pgfusepath{clip}%
\pgfsetbuttcap%
\pgfsetroundjoin%
\definecolor{currentfill}{rgb}{0.153364,0.497000,0.557724}%
\pgfsetfillcolor{currentfill}%
\pgfsetfillopacity{0.700000}%
\pgfsetlinewidth{0.000000pt}%
\definecolor{currentstroke}{rgb}{0.000000,0.000000,0.000000}%
\pgfsetstrokecolor{currentstroke}%
\pgfsetdash{}{0pt}%
\pgfpathmoveto{\pgfqpoint{3.257356in}{3.454061in}}%
\pgfpathlineto{\pgfqpoint{3.270284in}{3.435095in}}%
\pgfpathlineto{\pgfqpoint{3.283205in}{3.416412in}}%
\pgfpathlineto{\pgfqpoint{3.296120in}{3.398010in}}%
\pgfpathlineto{\pgfqpoint{3.309030in}{3.379886in}}%
\pgfpathlineto{\pgfqpoint{3.316584in}{3.396713in}}%
\pgfpathlineto{\pgfqpoint{3.324133in}{3.413759in}}%
\pgfpathlineto{\pgfqpoint{3.331677in}{3.431028in}}%
\pgfpathlineto{\pgfqpoint{3.339214in}{3.448524in}}%
\pgfpathlineto{\pgfqpoint{3.326310in}{3.466987in}}%
\pgfpathlineto{\pgfqpoint{3.313400in}{3.485729in}}%
\pgfpathlineto{\pgfqpoint{3.300484in}{3.504753in}}%
\pgfpathlineto{\pgfqpoint{3.287562in}{3.524060in}}%
\pgfpathlineto{\pgfqpoint{3.280020in}{3.506212in}}%
\pgfpathlineto{\pgfqpoint{3.272471in}{3.488600in}}%
\pgfpathlineto{\pgfqpoint{3.264917in}{3.471217in}}%
\pgfpathlineto{\pgfqpoint{3.257356in}{3.454061in}}%
\pgfpathclose%
\pgfusepath{fill}%
\end{pgfscope}%
\begin{pgfscope}%
\pgfpathrectangle{\pgfqpoint{1.254980in}{0.150000in}}{\pgfqpoint{5.490039in}{5.490039in}}%
\pgfusepath{clip}%
\pgfsetbuttcap%
\pgfsetroundjoin%
\definecolor{currentfill}{rgb}{0.241237,0.296485,0.539709}%
\pgfsetfillcolor{currentfill}%
\pgfsetfillopacity{0.700000}%
\pgfsetlinewidth{0.000000pt}%
\definecolor{currentstroke}{rgb}{0.000000,0.000000,0.000000}%
\pgfsetstrokecolor{currentstroke}%
\pgfsetdash{}{0pt}%
\pgfpathmoveto{\pgfqpoint{3.801681in}{2.950718in}}%
\pgfpathlineto{\pgfqpoint{3.814515in}{2.941091in}}%
\pgfpathlineto{\pgfqpoint{3.827351in}{2.931671in}}%
\pgfpathlineto{\pgfqpoint{3.840189in}{2.922458in}}%
\pgfpathlineto{\pgfqpoint{3.853029in}{2.913450in}}%
\pgfpathlineto{\pgfqpoint{3.860505in}{2.927672in}}%
\pgfpathlineto{\pgfqpoint{3.867977in}{2.942059in}}%
\pgfpathlineto{\pgfqpoint{3.875446in}{2.956614in}}%
\pgfpathlineto{\pgfqpoint{3.882910in}{2.971342in}}%
\pgfpathlineto{\pgfqpoint{3.870077in}{2.980707in}}%
\pgfpathlineto{\pgfqpoint{3.857245in}{2.990277in}}%
\pgfpathlineto{\pgfqpoint{3.844415in}{3.000053in}}%
\pgfpathlineto{\pgfqpoint{3.831587in}{3.010037in}}%
\pgfpathlineto{\pgfqpoint{3.824117in}{2.994942in}}%
\pgfpathlineto{\pgfqpoint{3.816642in}{2.980026in}}%
\pgfpathlineto{\pgfqpoint{3.809164in}{2.965286in}}%
\pgfpathlineto{\pgfqpoint{3.801681in}{2.950718in}}%
\pgfpathclose%
\pgfusepath{fill}%
\end{pgfscope}%
\begin{pgfscope}%
\pgfpathrectangle{\pgfqpoint{1.254980in}{0.150000in}}{\pgfqpoint{5.490039in}{5.490039in}}%
\pgfusepath{clip}%
\pgfsetbuttcap%
\pgfsetroundjoin%
\definecolor{currentfill}{rgb}{0.243113,0.292092,0.538516}%
\pgfsetfillcolor{currentfill}%
\pgfsetfillopacity{0.700000}%
\pgfsetlinewidth{0.000000pt}%
\definecolor{currentstroke}{rgb}{0.000000,0.000000,0.000000}%
\pgfsetstrokecolor{currentstroke}%
\pgfsetdash{}{0pt}%
\pgfpathmoveto{\pgfqpoint{3.934265in}{2.935909in}}%
\pgfpathlineto{\pgfqpoint{3.947110in}{2.927551in}}%
\pgfpathlineto{\pgfqpoint{3.959958in}{2.919391in}}%
\pgfpathlineto{\pgfqpoint{3.972809in}{2.911428in}}%
\pgfpathlineto{\pgfqpoint{3.985662in}{2.903660in}}%
\pgfpathlineto{\pgfqpoint{3.993110in}{2.917818in}}%
\pgfpathlineto{\pgfqpoint{4.000554in}{2.932144in}}%
\pgfpathlineto{\pgfqpoint{4.007994in}{2.946643in}}%
\pgfpathlineto{\pgfqpoint{4.015431in}{2.961318in}}%
\pgfpathlineto{\pgfqpoint{4.002584in}{2.969469in}}%
\pgfpathlineto{\pgfqpoint{3.989740in}{2.977815in}}%
\pgfpathlineto{\pgfqpoint{3.976899in}{2.986359in}}%
\pgfpathlineto{\pgfqpoint{3.964060in}{2.995100in}}%
\pgfpathlineto{\pgfqpoint{3.956617in}{2.980031in}}%
\pgfpathlineto{\pgfqpoint{3.949170in}{2.965146in}}%
\pgfpathlineto{\pgfqpoint{3.941719in}{2.950440in}}%
\pgfpathlineto{\pgfqpoint{3.934265in}{2.935909in}}%
\pgfpathclose%
\pgfusepath{fill}%
\end{pgfscope}%
\begin{pgfscope}%
\pgfpathrectangle{\pgfqpoint{1.254980in}{0.150000in}}{\pgfqpoint{5.490039in}{5.490039in}}%
\pgfusepath{clip}%
\pgfsetbuttcap%
\pgfsetroundjoin%
\definecolor{currentfill}{rgb}{0.237441,0.305202,0.541921}%
\pgfsetfillcolor{currentfill}%
\pgfsetfillopacity{0.700000}%
\pgfsetlinewidth{0.000000pt}%
\definecolor{currentstroke}{rgb}{0.000000,0.000000,0.000000}%
\pgfsetstrokecolor{currentstroke}%
\pgfsetdash{}{0pt}%
\pgfpathmoveto{\pgfqpoint{4.147996in}{2.960312in}}%
\pgfpathlineto{\pgfqpoint{4.160870in}{2.953467in}}%
\pgfpathlineto{\pgfqpoint{4.173748in}{2.946808in}}%
\pgfpathlineto{\pgfqpoint{4.186631in}{2.940333in}}%
\pgfpathlineto{\pgfqpoint{4.199518in}{2.934043in}}%
\pgfpathlineto{\pgfqpoint{4.206918in}{2.948391in}}%
\pgfpathlineto{\pgfqpoint{4.214316in}{2.962924in}}%
\pgfpathlineto{\pgfqpoint{4.221711in}{2.977647in}}%
\pgfpathlineto{\pgfqpoint{4.229103in}{2.992566in}}%
\pgfpathlineto{\pgfqpoint{4.216223in}{2.999294in}}%
\pgfpathlineto{\pgfqpoint{4.203348in}{3.006207in}}%
\pgfpathlineto{\pgfqpoint{4.190477in}{3.013304in}}%
\pgfpathlineto{\pgfqpoint{4.177610in}{3.020587in}}%
\pgfpathlineto{\pgfqpoint{4.170210in}{3.005220in}}%
\pgfpathlineto{\pgfqpoint{4.162808in}{2.990055in}}%
\pgfpathlineto{\pgfqpoint{4.155404in}{2.975088in}}%
\pgfpathlineto{\pgfqpoint{4.147996in}{2.960312in}}%
\pgfpathclose%
\pgfusepath{fill}%
\end{pgfscope}%
\begin{pgfscope}%
\pgfpathrectangle{\pgfqpoint{1.254980in}{0.150000in}}{\pgfqpoint{5.490039in}{5.490039in}}%
\pgfusepath{clip}%
\pgfsetbuttcap%
\pgfsetroundjoin%
\definecolor{currentfill}{rgb}{0.237441,0.305202,0.541921}%
\pgfsetfillcolor{currentfill}%
\pgfsetfillopacity{0.700000}%
\pgfsetlinewidth{0.000000pt}%
\definecolor{currentstroke}{rgb}{0.000000,0.000000,0.000000}%
\pgfsetstrokecolor{currentstroke}%
\pgfsetdash{}{0pt}%
\pgfpathmoveto{\pgfqpoint{3.669019in}{2.976079in}}%
\pgfpathlineto{\pgfqpoint{3.681853in}{2.965074in}}%
\pgfpathlineto{\pgfqpoint{3.694687in}{2.954287in}}%
\pgfpathlineto{\pgfqpoint{3.707522in}{2.943717in}}%
\pgfpathlineto{\pgfqpoint{3.720357in}{2.933363in}}%
\pgfpathlineto{\pgfqpoint{3.727864in}{2.947607in}}%
\pgfpathlineto{\pgfqpoint{3.735365in}{2.962013in}}%
\pgfpathlineto{\pgfqpoint{3.742863in}{2.976585in}}%
\pgfpathlineto{\pgfqpoint{3.750356in}{2.991326in}}%
\pgfpathlineto{\pgfqpoint{3.737527in}{3.002011in}}%
\pgfpathlineto{\pgfqpoint{3.724699in}{3.012910in}}%
\pgfpathlineto{\pgfqpoint{3.711871in}{3.024028in}}%
\pgfpathlineto{\pgfqpoint{3.699044in}{3.035363in}}%
\pgfpathlineto{\pgfqpoint{3.691544in}{3.020281in}}%
\pgfpathlineto{\pgfqpoint{3.684041in}{3.005376in}}%
\pgfpathlineto{\pgfqpoint{3.676532in}{2.990643in}}%
\pgfpathlineto{\pgfqpoint{3.669019in}{2.976079in}}%
\pgfpathclose%
\pgfusepath{fill}%
\end{pgfscope}%
\begin{pgfscope}%
\pgfpathrectangle{\pgfqpoint{1.254980in}{0.150000in}}{\pgfqpoint{5.490039in}{5.490039in}}%
\pgfusepath{clip}%
\pgfsetbuttcap%
\pgfsetroundjoin%
\definecolor{currentfill}{rgb}{0.866013,0.889868,0.095953}%
\pgfsetfillcolor{currentfill}%
\pgfsetfillopacity{0.700000}%
\pgfsetlinewidth{0.000000pt}%
\definecolor{currentstroke}{rgb}{0.000000,0.000000,0.000000}%
\pgfsetstrokecolor{currentstroke}%
\pgfsetdash{}{0pt}%
\pgfpathmoveto{\pgfqpoint{3.481341in}{4.695010in}}%
\pgfpathlineto{\pgfqpoint{3.494334in}{4.668110in}}%
\pgfpathlineto{\pgfqpoint{3.507318in}{4.641529in}}%
\pgfpathlineto{\pgfqpoint{3.520293in}{4.615263in}}%
\pgfpathlineto{\pgfqpoint{3.533259in}{4.589310in}}%
\pgfpathlineto{\pgfqpoint{3.540607in}{4.622132in}}%
\pgfpathlineto{\pgfqpoint{3.547951in}{4.655456in}}%
\pgfpathlineto{\pgfqpoint{3.555293in}{4.689290in}}%
\pgfpathlineto{\pgfqpoint{3.562632in}{4.723643in}}%
\pgfpathlineto{\pgfqpoint{3.549658in}{4.750306in}}%
\pgfpathlineto{\pgfqpoint{3.536674in}{4.777283in}}%
\pgfpathlineto{\pgfqpoint{3.523682in}{4.804577in}}%
\pgfpathlineto{\pgfqpoint{3.510681in}{4.832193in}}%
\pgfpathlineto{\pgfqpoint{3.503350in}{4.797113in}}%
\pgfpathlineto{\pgfqpoint{3.496017in}{4.762562in}}%
\pgfpathlineto{\pgfqpoint{3.488681in}{4.728531in}}%
\pgfpathlineto{\pgfqpoint{3.481341in}{4.695010in}}%
\pgfpathclose%
\pgfusepath{fill}%
\end{pgfscope}%
\begin{pgfscope}%
\pgfpathrectangle{\pgfqpoint{1.254980in}{0.150000in}}{\pgfqpoint{5.490039in}{5.490039in}}%
\pgfusepath{clip}%
\pgfsetbuttcap%
\pgfsetroundjoin%
\definecolor{currentfill}{rgb}{0.121831,0.589055,0.545623}%
\pgfsetfillcolor{currentfill}%
\pgfsetfillopacity{0.700000}%
\pgfsetlinewidth{0.000000pt}%
\definecolor{currentstroke}{rgb}{0.000000,0.000000,0.000000}%
\pgfsetstrokecolor{currentstroke}%
\pgfsetdash{}{0pt}%
\pgfpathmoveto{\pgfqpoint{3.183927in}{3.689065in}}%
\pgfpathlineto{\pgfqpoint{3.196909in}{3.667386in}}%
\pgfpathlineto{\pgfqpoint{3.209882in}{3.646015in}}%
\pgfpathlineto{\pgfqpoint{3.222848in}{3.624947in}}%
\pgfpathlineto{\pgfqpoint{3.235805in}{3.604179in}}%
\pgfpathlineto{\pgfqpoint{3.243347in}{3.622626in}}%
\pgfpathlineto{\pgfqpoint{3.250882in}{3.641323in}}%
\pgfpathlineto{\pgfqpoint{3.258411in}{3.660275in}}%
\pgfpathlineto{\pgfqpoint{3.265934in}{3.679487in}}%
\pgfpathlineto{\pgfqpoint{3.252981in}{3.700628in}}%
\pgfpathlineto{\pgfqpoint{3.240020in}{3.722070in}}%
\pgfpathlineto{\pgfqpoint{3.227050in}{3.743817in}}%
\pgfpathlineto{\pgfqpoint{3.214073in}{3.765871in}}%
\pgfpathlineto{\pgfqpoint{3.206546in}{3.746273in}}%
\pgfpathlineto{\pgfqpoint{3.199013in}{3.726943in}}%
\pgfpathlineto{\pgfqpoint{3.191473in}{3.707875in}}%
\pgfpathlineto{\pgfqpoint{3.183927in}{3.689065in}}%
\pgfpathclose%
\pgfusepath{fill}%
\end{pgfscope}%
\begin{pgfscope}%
\pgfpathrectangle{\pgfqpoint{1.254980in}{0.150000in}}{\pgfqpoint{5.490039in}{5.490039in}}%
\pgfusepath{clip}%
\pgfsetbuttcap%
\pgfsetroundjoin%
\definecolor{currentfill}{rgb}{0.195860,0.395433,0.555276}%
\pgfsetfillcolor{currentfill}%
\pgfsetfillopacity{0.700000}%
\pgfsetlinewidth{0.000000pt}%
\definecolor{currentstroke}{rgb}{0.000000,0.000000,0.000000}%
\pgfsetstrokecolor{currentstroke}%
\pgfsetdash{}{0pt}%
\pgfpathmoveto{\pgfqpoint{3.381894in}{3.181873in}}%
\pgfpathlineto{\pgfqpoint{3.394768in}{3.166449in}}%
\pgfpathlineto{\pgfqpoint{3.407638in}{3.151278in}}%
\pgfpathlineto{\pgfqpoint{3.420504in}{3.136360in}}%
\pgfpathlineto{\pgfqpoint{3.433368in}{3.121692in}}%
\pgfpathlineto{\pgfqpoint{3.440926in}{3.136792in}}%
\pgfpathlineto{\pgfqpoint{3.448480in}{3.152073in}}%
\pgfpathlineto{\pgfqpoint{3.456028in}{3.167541in}}%
\pgfpathlineto{\pgfqpoint{3.463570in}{3.183198in}}%
\pgfpathlineto{\pgfqpoint{3.450713in}{3.198172in}}%
\pgfpathlineto{\pgfqpoint{3.437853in}{3.213396in}}%
\pgfpathlineto{\pgfqpoint{3.424989in}{3.228873in}}%
\pgfpathlineto{\pgfqpoint{3.412122in}{3.244605in}}%
\pgfpathlineto{\pgfqpoint{3.404573in}{3.228631in}}%
\pgfpathlineto{\pgfqpoint{3.397019in}{3.212853in}}%
\pgfpathlineto{\pgfqpoint{3.389459in}{3.197269in}}%
\pgfpathlineto{\pgfqpoint{3.381894in}{3.181873in}}%
\pgfpathclose%
\pgfusepath{fill}%
\end{pgfscope}%
\begin{pgfscope}%
\pgfpathrectangle{\pgfqpoint{1.254980in}{0.150000in}}{\pgfqpoint{5.490039in}{5.490039in}}%
\pgfusepath{clip}%
\pgfsetbuttcap%
\pgfsetroundjoin%
\definecolor{currentfill}{rgb}{0.208623,0.367752,0.552675}%
\pgfsetfillcolor{currentfill}%
\pgfsetfillopacity{0.700000}%
\pgfsetlinewidth{0.000000pt}%
\definecolor{currentstroke}{rgb}{0.000000,0.000000,0.000000}%
\pgfsetstrokecolor{currentstroke}%
\pgfsetdash{}{0pt}%
\pgfpathmoveto{\pgfqpoint{3.433368in}{3.121692in}}%
\pgfpathlineto{\pgfqpoint{3.446228in}{3.107271in}}%
\pgfpathlineto{\pgfqpoint{3.459086in}{3.093097in}}%
\pgfpathlineto{\pgfqpoint{3.471941in}{3.079166in}}%
\pgfpathlineto{\pgfqpoint{3.484794in}{3.065478in}}%
\pgfpathlineto{\pgfqpoint{3.492346in}{3.080283in}}%
\pgfpathlineto{\pgfqpoint{3.499893in}{3.095263in}}%
\pgfpathlineto{\pgfqpoint{3.507435in}{3.110422in}}%
\pgfpathlineto{\pgfqpoint{3.514971in}{3.125763in}}%
\pgfpathlineto{\pgfqpoint{3.502125in}{3.139756in}}%
\pgfpathlineto{\pgfqpoint{3.489276in}{3.153992in}}%
\pgfpathlineto{\pgfqpoint{3.476424in}{3.168472in}}%
\pgfpathlineto{\pgfqpoint{3.463570in}{3.183198in}}%
\pgfpathlineto{\pgfqpoint{3.456028in}{3.167541in}}%
\pgfpathlineto{\pgfqpoint{3.448480in}{3.152073in}}%
\pgfpathlineto{\pgfqpoint{3.440926in}{3.136792in}}%
\pgfpathlineto{\pgfqpoint{3.433368in}{3.121692in}}%
\pgfpathclose%
\pgfusepath{fill}%
\end{pgfscope}%
\begin{pgfscope}%
\pgfpathrectangle{\pgfqpoint{1.254980in}{0.150000in}}{\pgfqpoint{5.490039in}{5.490039in}}%
\pgfusepath{clip}%
\pgfsetbuttcap%
\pgfsetroundjoin%
\definecolor{currentfill}{rgb}{0.185556,0.418570,0.556753}%
\pgfsetfillcolor{currentfill}%
\pgfsetfillopacity{0.700000}%
\pgfsetlinewidth{0.000000pt}%
\definecolor{currentstroke}{rgb}{0.000000,0.000000,0.000000}%
\pgfsetstrokecolor{currentstroke}%
\pgfsetdash{}{0pt}%
\pgfpathmoveto{\pgfqpoint{3.330360in}{3.246157in}}%
\pgfpathlineto{\pgfqpoint{3.343250in}{3.229694in}}%
\pgfpathlineto{\pgfqpoint{3.356135in}{3.213494in}}%
\pgfpathlineto{\pgfqpoint{3.369017in}{3.197554in}}%
\pgfpathlineto{\pgfqpoint{3.381894in}{3.181873in}}%
\pgfpathlineto{\pgfqpoint{3.389459in}{3.197269in}}%
\pgfpathlineto{\pgfqpoint{3.397019in}{3.212853in}}%
\pgfpathlineto{\pgfqpoint{3.404573in}{3.228631in}}%
\pgfpathlineto{\pgfqpoint{3.412122in}{3.244605in}}%
\pgfpathlineto{\pgfqpoint{3.399251in}{3.260593in}}%
\pgfpathlineto{\pgfqpoint{3.386376in}{3.276840in}}%
\pgfpathlineto{\pgfqpoint{3.373497in}{3.293348in}}%
\pgfpathlineto{\pgfqpoint{3.360613in}{3.310119in}}%
\pgfpathlineto{\pgfqpoint{3.353058in}{3.293827in}}%
\pgfpathlineto{\pgfqpoint{3.345498in}{3.277738in}}%
\pgfpathlineto{\pgfqpoint{3.337932in}{3.261849in}}%
\pgfpathlineto{\pgfqpoint{3.330360in}{3.246157in}}%
\pgfpathclose%
\pgfusepath{fill}%
\end{pgfscope}%
\begin{pgfscope}%
\pgfpathrectangle{\pgfqpoint{1.254980in}{0.150000in}}{\pgfqpoint{5.490039in}{5.490039in}}%
\pgfusepath{clip}%
\pgfsetbuttcap%
\pgfsetroundjoin%
\definecolor{currentfill}{rgb}{0.212395,0.359683,0.551710}%
\pgfsetfillcolor{currentfill}%
\pgfsetfillopacity{0.700000}%
\pgfsetlinewidth{0.000000pt}%
\definecolor{currentstroke}{rgb}{0.000000,0.000000,0.000000}%
\pgfsetstrokecolor{currentstroke}%
\pgfsetdash{}{0pt}%
\pgfpathmoveto{\pgfqpoint{4.524062in}{3.081944in}}%
\pgfpathlineto{\pgfqpoint{4.537011in}{3.076748in}}%
\pgfpathlineto{\pgfqpoint{4.549966in}{3.071723in}}%
\pgfpathlineto{\pgfqpoint{4.562928in}{3.066868in}}%
\pgfpathlineto{\pgfqpoint{4.575896in}{3.062182in}}%
\pgfpathlineto{\pgfqpoint{4.583224in}{3.077467in}}%
\pgfpathlineto{\pgfqpoint{4.590552in}{3.092994in}}%
\pgfpathlineto{\pgfqpoint{4.597880in}{3.108769in}}%
\pgfpathlineto{\pgfqpoint{4.605207in}{3.124799in}}%
\pgfpathlineto{\pgfqpoint{4.592248in}{3.130032in}}%
\pgfpathlineto{\pgfqpoint{4.579296in}{3.135435in}}%
\pgfpathlineto{\pgfqpoint{4.566351in}{3.141008in}}%
\pgfpathlineto{\pgfqpoint{4.553411in}{3.146751in}}%
\pgfpathlineto{\pgfqpoint{4.546074in}{3.130163in}}%
\pgfpathlineto{\pgfqpoint{4.538737in}{3.113837in}}%
\pgfpathlineto{\pgfqpoint{4.531400in}{3.097766in}}%
\pgfpathlineto{\pgfqpoint{4.524062in}{3.081944in}}%
\pgfpathclose%
\pgfusepath{fill}%
\end{pgfscope}%
\begin{pgfscope}%
\pgfpathrectangle{\pgfqpoint{1.254980in}{0.150000in}}{\pgfqpoint{5.490039in}{5.490039in}}%
\pgfusepath{clip}%
\pgfsetbuttcap%
\pgfsetroundjoin%
\definecolor{currentfill}{rgb}{0.243113,0.292092,0.538516}%
\pgfsetfillcolor{currentfill}%
\pgfsetfillopacity{0.700000}%
\pgfsetlinewidth{0.000000pt}%
\definecolor{currentstroke}{rgb}{0.000000,0.000000,0.000000}%
\pgfsetstrokecolor{currentstroke}%
\pgfsetdash{}{0pt}%
\pgfpathmoveto{\pgfqpoint{4.066853in}{2.930651in}}%
\pgfpathlineto{\pgfqpoint{4.079718in}{2.923463in}}%
\pgfpathlineto{\pgfqpoint{4.092586in}{2.916465in}}%
\pgfpathlineto{\pgfqpoint{4.105459in}{2.909656in}}%
\pgfpathlineto{\pgfqpoint{4.118336in}{2.903034in}}%
\pgfpathlineto{\pgfqpoint{4.125755in}{2.917090in}}%
\pgfpathlineto{\pgfqpoint{4.133172in}{2.931319in}}%
\pgfpathlineto{\pgfqpoint{4.140586in}{2.945724in}}%
\pgfpathlineto{\pgfqpoint{4.147996in}{2.960312in}}%
\pgfpathlineto{\pgfqpoint{4.135126in}{2.967345in}}%
\pgfpathlineto{\pgfqpoint{4.122261in}{2.974565in}}%
\pgfpathlineto{\pgfqpoint{4.109400in}{2.981973in}}%
\pgfpathlineto{\pgfqpoint{4.096542in}{2.989571in}}%
\pgfpathlineto{\pgfqpoint{4.089124in}{2.974563in}}%
\pgfpathlineto{\pgfqpoint{4.081704in}{2.959743in}}%
\pgfpathlineto{\pgfqpoint{4.074280in}{2.945107in}}%
\pgfpathlineto{\pgfqpoint{4.066853in}{2.930651in}}%
\pgfpathclose%
\pgfusepath{fill}%
\end{pgfscope}%
\begin{pgfscope}%
\pgfpathrectangle{\pgfqpoint{1.254980in}{0.150000in}}{\pgfqpoint{5.490039in}{5.490039in}}%
\pgfusepath{clip}%
\pgfsetbuttcap%
\pgfsetroundjoin%
\definecolor{currentfill}{rgb}{0.218130,0.347432,0.550038}%
\pgfsetfillcolor{currentfill}%
\pgfsetfillopacity{0.700000}%
\pgfsetlinewidth{0.000000pt}%
\definecolor{currentstroke}{rgb}{0.000000,0.000000,0.000000}%
\pgfsetstrokecolor{currentstroke}%
\pgfsetdash{}{0pt}%
\pgfpathmoveto{\pgfqpoint{3.484794in}{3.065478in}}%
\pgfpathlineto{\pgfqpoint{3.497645in}{3.052030in}}%
\pgfpathlineto{\pgfqpoint{3.510494in}{3.038820in}}%
\pgfpathlineto{\pgfqpoint{3.523342in}{3.025846in}}%
\pgfpathlineto{\pgfqpoint{3.536187in}{3.013107in}}%
\pgfpathlineto{\pgfqpoint{3.543733in}{3.027619in}}%
\pgfpathlineto{\pgfqpoint{3.551273in}{3.042298in}}%
\pgfpathlineto{\pgfqpoint{3.558808in}{3.057149in}}%
\pgfpathlineto{\pgfqpoint{3.566338in}{3.072175in}}%
\pgfpathlineto{\pgfqpoint{3.553499in}{3.085218in}}%
\pgfpathlineto{\pgfqpoint{3.540658in}{3.098495in}}%
\pgfpathlineto{\pgfqpoint{3.527816in}{3.112010in}}%
\pgfpathlineto{\pgfqpoint{3.514971in}{3.125763in}}%
\pgfpathlineto{\pgfqpoint{3.507435in}{3.110422in}}%
\pgfpathlineto{\pgfqpoint{3.499893in}{3.095263in}}%
\pgfpathlineto{\pgfqpoint{3.492346in}{3.080283in}}%
\pgfpathlineto{\pgfqpoint{3.484794in}{3.065478in}}%
\pgfpathclose%
\pgfusepath{fill}%
\end{pgfscope}%
\begin{pgfscope}%
\pgfpathrectangle{\pgfqpoint{1.254980in}{0.150000in}}{\pgfqpoint{5.490039in}{5.490039in}}%
\pgfusepath{clip}%
\pgfsetbuttcap%
\pgfsetroundjoin%
\definecolor{currentfill}{rgb}{0.140536,0.530132,0.555659}%
\pgfsetfillcolor{currentfill}%
\pgfsetfillopacity{0.700000}%
\pgfsetlinewidth{0.000000pt}%
\definecolor{currentstroke}{rgb}{0.000000,0.000000,0.000000}%
\pgfsetstrokecolor{currentstroke}%
\pgfsetdash{}{0pt}%
\pgfpathmoveto{\pgfqpoint{3.205578in}{3.532812in}}%
\pgfpathlineto{\pgfqpoint{3.218533in}{3.512686in}}%
\pgfpathlineto{\pgfqpoint{3.231481in}{3.492854in}}%
\pgfpathlineto{\pgfqpoint{3.244422in}{3.473313in}}%
\pgfpathlineto{\pgfqpoint{3.257356in}{3.454061in}}%
\pgfpathlineto{\pgfqpoint{3.264917in}{3.471217in}}%
\pgfpathlineto{\pgfqpoint{3.272471in}{3.488600in}}%
\pgfpathlineto{\pgfqpoint{3.280020in}{3.506212in}}%
\pgfpathlineto{\pgfqpoint{3.287562in}{3.524060in}}%
\pgfpathlineto{\pgfqpoint{3.274634in}{3.543653in}}%
\pgfpathlineto{\pgfqpoint{3.261698in}{3.563535in}}%
\pgfpathlineto{\pgfqpoint{3.248755in}{3.583710in}}%
\pgfpathlineto{\pgfqpoint{3.235805in}{3.604179in}}%
\pgfpathlineto{\pgfqpoint{3.228258in}{3.585979in}}%
\pgfpathlineto{\pgfqpoint{3.220704in}{3.568020in}}%
\pgfpathlineto{\pgfqpoint{3.213144in}{3.550300in}}%
\pgfpathlineto{\pgfqpoint{3.205578in}{3.532812in}}%
\pgfpathclose%
\pgfusepath{fill}%
\end{pgfscope}%
\begin{pgfscope}%
\pgfpathrectangle{\pgfqpoint{1.254980in}{0.150000in}}{\pgfqpoint{5.490039in}{5.490039in}}%
\pgfusepath{clip}%
\pgfsetbuttcap%
\pgfsetroundjoin%
\definecolor{currentfill}{rgb}{0.203063,0.379716,0.553925}%
\pgfsetfillcolor{currentfill}%
\pgfsetfillopacity{0.700000}%
\pgfsetlinewidth{0.000000pt}%
\definecolor{currentstroke}{rgb}{0.000000,0.000000,0.000000}%
\pgfsetstrokecolor{currentstroke}%
\pgfsetdash{}{0pt}%
\pgfpathmoveto{\pgfqpoint{4.605207in}{3.124799in}}%
\pgfpathlineto{\pgfqpoint{4.618172in}{3.119735in}}%
\pgfpathlineto{\pgfqpoint{4.631144in}{3.114838in}}%
\pgfpathlineto{\pgfqpoint{4.644123in}{3.110109in}}%
\pgfpathlineto{\pgfqpoint{4.657109in}{3.105548in}}%
\pgfpathlineto{\pgfqpoint{4.664426in}{3.121275in}}%
\pgfpathlineto{\pgfqpoint{4.671743in}{3.137263in}}%
\pgfpathlineto{\pgfqpoint{4.679060in}{3.153519in}}%
\pgfpathlineto{\pgfqpoint{4.686379in}{3.170050in}}%
\pgfpathlineto{\pgfqpoint{4.673404in}{3.175187in}}%
\pgfpathlineto{\pgfqpoint{4.660435in}{3.180491in}}%
\pgfpathlineto{\pgfqpoint{4.647474in}{3.185962in}}%
\pgfpathlineto{\pgfqpoint{4.634518in}{3.191602in}}%
\pgfpathlineto{\pgfqpoint{4.627190in}{3.174485in}}%
\pgfpathlineto{\pgfqpoint{4.619862in}{3.157651in}}%
\pgfpathlineto{\pgfqpoint{4.612534in}{3.141091in}}%
\pgfpathlineto{\pgfqpoint{4.605207in}{3.124799in}}%
\pgfpathclose%
\pgfusepath{fill}%
\end{pgfscope}%
\begin{pgfscope}%
\pgfpathrectangle{\pgfqpoint{1.254980in}{0.150000in}}{\pgfqpoint{5.490039in}{5.490039in}}%
\pgfusepath{clip}%
\pgfsetbuttcap%
\pgfsetroundjoin%
\definecolor{currentfill}{rgb}{0.220057,0.343307,0.549413}%
\pgfsetfillcolor{currentfill}%
\pgfsetfillopacity{0.700000}%
\pgfsetlinewidth{0.000000pt}%
\definecolor{currentstroke}{rgb}{0.000000,0.000000,0.000000}%
\pgfsetstrokecolor{currentstroke}%
\pgfsetdash{}{0pt}%
\pgfpathmoveto{\pgfqpoint{4.442931in}{3.041447in}}%
\pgfpathlineto{\pgfqpoint{4.455864in}{3.036081in}}%
\pgfpathlineto{\pgfqpoint{4.468804in}{3.030890in}}%
\pgfpathlineto{\pgfqpoint{4.481749in}{3.025870in}}%
\pgfpathlineto{\pgfqpoint{4.494701in}{3.021023in}}%
\pgfpathlineto{\pgfqpoint{4.502043in}{3.035911in}}%
\pgfpathlineto{\pgfqpoint{4.509383in}{3.051023in}}%
\pgfpathlineto{\pgfqpoint{4.516723in}{3.066365in}}%
\pgfpathlineto{\pgfqpoint{4.524062in}{3.081944in}}%
\pgfpathlineto{\pgfqpoint{4.511119in}{3.087312in}}%
\pgfpathlineto{\pgfqpoint{4.498183in}{3.092851in}}%
\pgfpathlineto{\pgfqpoint{4.485253in}{3.098563in}}%
\pgfpathlineto{\pgfqpoint{4.472328in}{3.104448in}}%
\pgfpathlineto{\pgfqpoint{4.464981in}{3.088338in}}%
\pgfpathlineto{\pgfqpoint{4.457632in}{3.072473in}}%
\pgfpathlineto{\pgfqpoint{4.450282in}{3.056844in}}%
\pgfpathlineto{\pgfqpoint{4.442931in}{3.041447in}}%
\pgfpathclose%
\pgfusepath{fill}%
\end{pgfscope}%
\begin{pgfscope}%
\pgfpathrectangle{\pgfqpoint{1.254980in}{0.150000in}}{\pgfqpoint{5.490039in}{5.490039in}}%
\pgfusepath{clip}%
\pgfsetbuttcap%
\pgfsetroundjoin%
\definecolor{currentfill}{rgb}{0.194100,0.399323,0.555565}%
\pgfsetfillcolor{currentfill}%
\pgfsetfillopacity{0.700000}%
\pgfsetlinewidth{0.000000pt}%
\definecolor{currentstroke}{rgb}{0.000000,0.000000,0.000000}%
\pgfsetstrokecolor{currentstroke}%
\pgfsetdash{}{0pt}%
\pgfpathmoveto{\pgfqpoint{4.686379in}{3.170050in}}%
\pgfpathlineto{\pgfqpoint{4.699361in}{3.165080in}}%
\pgfpathlineto{\pgfqpoint{4.712350in}{3.160275in}}%
\pgfpathlineto{\pgfqpoint{4.725346in}{3.155636in}}%
\pgfpathlineto{\pgfqpoint{4.738350in}{3.151161in}}%
\pgfpathlineto{\pgfqpoint{4.745658in}{3.167381in}}%
\pgfpathlineto{\pgfqpoint{4.752967in}{3.183883in}}%
\pgfpathlineto{\pgfqpoint{4.760277in}{3.200673in}}%
\pgfpathlineto{\pgfqpoint{4.767589in}{3.217759in}}%
\pgfpathlineto{\pgfqpoint{4.754597in}{3.222836in}}%
\pgfpathlineto{\pgfqpoint{4.741612in}{3.228078in}}%
\pgfpathlineto{\pgfqpoint{4.728634in}{3.233485in}}%
\pgfpathlineto{\pgfqpoint{4.715662in}{3.239059in}}%
\pgfpathlineto{\pgfqpoint{4.708339in}{3.221360in}}%
\pgfpathlineto{\pgfqpoint{4.701018in}{3.203963in}}%
\pgfpathlineto{\pgfqpoint{4.693698in}{3.186862in}}%
\pgfpathlineto{\pgfqpoint{4.686379in}{3.170050in}}%
\pgfpathclose%
\pgfusepath{fill}%
\end{pgfscope}%
\begin{pgfscope}%
\pgfpathrectangle{\pgfqpoint{1.254980in}{0.150000in}}{\pgfqpoint{5.490039in}{5.490039in}}%
\pgfusepath{clip}%
\pgfsetbuttcap%
\pgfsetroundjoin%
\definecolor{currentfill}{rgb}{0.227802,0.326594,0.546532}%
\pgfsetfillcolor{currentfill}%
\pgfsetfillopacity{0.700000}%
\pgfsetlinewidth{0.000000pt}%
\definecolor{currentstroke}{rgb}{0.000000,0.000000,0.000000}%
\pgfsetstrokecolor{currentstroke}%
\pgfsetdash{}{0pt}%
\pgfpathmoveto{\pgfqpoint{4.361805in}{3.003289in}}%
\pgfpathlineto{\pgfqpoint{4.374723in}{2.997716in}}%
\pgfpathlineto{\pgfqpoint{4.387647in}{2.992319in}}%
\pgfpathlineto{\pgfqpoint{4.400577in}{2.987098in}}%
\pgfpathlineto{\pgfqpoint{4.413513in}{2.982051in}}%
\pgfpathlineto{\pgfqpoint{4.420870in}{2.996583in}}%
\pgfpathlineto{\pgfqpoint{4.428225in}{3.011322in}}%
\pgfpathlineto{\pgfqpoint{4.435579in}{3.026275in}}%
\pgfpathlineto{\pgfqpoint{4.442931in}{3.041447in}}%
\pgfpathlineto{\pgfqpoint{4.430004in}{3.046986in}}%
\pgfpathlineto{\pgfqpoint{4.417083in}{3.052700in}}%
\pgfpathlineto{\pgfqpoint{4.404167in}{3.058589in}}%
\pgfpathlineto{\pgfqpoint{4.391257in}{3.064655in}}%
\pgfpathlineto{\pgfqpoint{4.383896in}{3.048980in}}%
\pgfpathlineto{\pgfqpoint{4.376534in}{3.033531in}}%
\pgfpathlineto{\pgfqpoint{4.369170in}{3.018303in}}%
\pgfpathlineto{\pgfqpoint{4.361805in}{3.003289in}}%
\pgfpathclose%
\pgfusepath{fill}%
\end{pgfscope}%
\begin{pgfscope}%
\pgfpathrectangle{\pgfqpoint{1.254980in}{0.150000in}}{\pgfqpoint{5.490039in}{5.490039in}}%
\pgfusepath{clip}%
\pgfsetbuttcap%
\pgfsetroundjoin%
\definecolor{currentfill}{rgb}{0.172719,0.448791,0.557885}%
\pgfsetfillcolor{currentfill}%
\pgfsetfillopacity{0.700000}%
\pgfsetlinewidth{0.000000pt}%
\definecolor{currentstroke}{rgb}{0.000000,0.000000,0.000000}%
\pgfsetstrokecolor{currentstroke}%
\pgfsetdash{}{0pt}%
\pgfpathmoveto{\pgfqpoint{3.278751in}{3.314688in}}%
\pgfpathlineto{\pgfqpoint{3.291661in}{3.297149in}}%
\pgfpathlineto{\pgfqpoint{3.304566in}{3.279882in}}%
\pgfpathlineto{\pgfqpoint{3.317465in}{3.262886in}}%
\pgfpathlineto{\pgfqpoint{3.330360in}{3.246157in}}%
\pgfpathlineto{\pgfqpoint{3.337932in}{3.261849in}}%
\pgfpathlineto{\pgfqpoint{3.345498in}{3.277738in}}%
\pgfpathlineto{\pgfqpoint{3.353058in}{3.293827in}}%
\pgfpathlineto{\pgfqpoint{3.360613in}{3.310119in}}%
\pgfpathlineto{\pgfqpoint{3.347725in}{3.327156in}}%
\pgfpathlineto{\pgfqpoint{3.334832in}{3.344461in}}%
\pgfpathlineto{\pgfqpoint{3.321933in}{3.362037in}}%
\pgfpathlineto{\pgfqpoint{3.309030in}{3.379886in}}%
\pgfpathlineto{\pgfqpoint{3.301469in}{3.363274in}}%
\pgfpathlineto{\pgfqpoint{3.293902in}{3.346872in}}%
\pgfpathlineto{\pgfqpoint{3.286330in}{3.330678in}}%
\pgfpathlineto{\pgfqpoint{3.278751in}{3.314688in}}%
\pgfpathclose%
\pgfusepath{fill}%
\end{pgfscope}%
\begin{pgfscope}%
\pgfpathrectangle{\pgfqpoint{1.254980in}{0.150000in}}{\pgfqpoint{5.490039in}{5.490039in}}%
\pgfusepath{clip}%
\pgfsetbuttcap%
\pgfsetroundjoin%
\definecolor{currentfill}{rgb}{0.246811,0.283237,0.535941}%
\pgfsetfillcolor{currentfill}%
\pgfsetfillopacity{0.700000}%
\pgfsetlinewidth{0.000000pt}%
\definecolor{currentstroke}{rgb}{0.000000,0.000000,0.000000}%
\pgfsetstrokecolor{currentstroke}%
\pgfsetdash{}{0pt}%
\pgfpathmoveto{\pgfqpoint{3.853029in}{2.913450in}}%
\pgfpathlineto{\pgfqpoint{3.865871in}{2.904645in}}%
\pgfpathlineto{\pgfqpoint{3.878715in}{2.896043in}}%
\pgfpathlineto{\pgfqpoint{3.891561in}{2.887642in}}%
\pgfpathlineto{\pgfqpoint{3.904410in}{2.879441in}}%
\pgfpathlineto{\pgfqpoint{3.911880in}{2.893318in}}%
\pgfpathlineto{\pgfqpoint{3.919345in}{2.907352in}}%
\pgfpathlineto{\pgfqpoint{3.926807in}{2.921547in}}%
\pgfpathlineto{\pgfqpoint{3.934265in}{2.935909in}}%
\pgfpathlineto{\pgfqpoint{3.921423in}{2.944465in}}%
\pgfpathlineto{\pgfqpoint{3.908583in}{2.953223in}}%
\pgfpathlineto{\pgfqpoint{3.895745in}{2.962181in}}%
\pgfpathlineto{\pgfqpoint{3.882910in}{2.971342in}}%
\pgfpathlineto{\pgfqpoint{3.875446in}{2.956614in}}%
\pgfpathlineto{\pgfqpoint{3.867977in}{2.942059in}}%
\pgfpathlineto{\pgfqpoint{3.860505in}{2.927672in}}%
\pgfpathlineto{\pgfqpoint{3.853029in}{2.913450in}}%
\pgfpathclose%
\pgfusepath{fill}%
\end{pgfscope}%
\begin{pgfscope}%
\pgfpathrectangle{\pgfqpoint{1.254980in}{0.150000in}}{\pgfqpoint{5.490039in}{5.490039in}}%
\pgfusepath{clip}%
\pgfsetbuttcap%
\pgfsetroundjoin%
\definecolor{currentfill}{rgb}{0.243113,0.292092,0.538516}%
\pgfsetfillcolor{currentfill}%
\pgfsetfillopacity{0.700000}%
\pgfsetlinewidth{0.000000pt}%
\definecolor{currentstroke}{rgb}{0.000000,0.000000,0.000000}%
\pgfsetstrokecolor{currentstroke}%
\pgfsetdash{}{0pt}%
\pgfpathmoveto{\pgfqpoint{3.720357in}{2.933363in}}%
\pgfpathlineto{\pgfqpoint{3.733193in}{2.923222in}}%
\pgfpathlineto{\pgfqpoint{3.746030in}{2.913295in}}%
\pgfpathlineto{\pgfqpoint{3.758868in}{2.903578in}}%
\pgfpathlineto{\pgfqpoint{3.771708in}{2.894072in}}%
\pgfpathlineto{\pgfqpoint{3.779208in}{2.907997in}}%
\pgfpathlineto{\pgfqpoint{3.786703in}{2.922077in}}%
\pgfpathlineto{\pgfqpoint{3.794194in}{2.936316in}}%
\pgfpathlineto{\pgfqpoint{3.801681in}{2.950718in}}%
\pgfpathlineto{\pgfqpoint{3.788848in}{2.960553in}}%
\pgfpathlineto{\pgfqpoint{3.776016in}{2.970599in}}%
\pgfpathlineto{\pgfqpoint{3.763186in}{2.980856in}}%
\pgfpathlineto{\pgfqpoint{3.750356in}{2.991326in}}%
\pgfpathlineto{\pgfqpoint{3.742863in}{2.976585in}}%
\pgfpathlineto{\pgfqpoint{3.735365in}{2.962013in}}%
\pgfpathlineto{\pgfqpoint{3.727864in}{2.947607in}}%
\pgfpathlineto{\pgfqpoint{3.720357in}{2.933363in}}%
\pgfpathclose%
\pgfusepath{fill}%
\end{pgfscope}%
\begin{pgfscope}%
\pgfpathrectangle{\pgfqpoint{1.254980in}{0.150000in}}{\pgfqpoint{5.490039in}{5.490039in}}%
\pgfusepath{clip}%
\pgfsetbuttcap%
\pgfsetroundjoin%
\definecolor{currentfill}{rgb}{0.227802,0.326594,0.546532}%
\pgfsetfillcolor{currentfill}%
\pgfsetfillopacity{0.700000}%
\pgfsetlinewidth{0.000000pt}%
\definecolor{currentstroke}{rgb}{0.000000,0.000000,0.000000}%
\pgfsetstrokecolor{currentstroke}%
\pgfsetdash{}{0pt}%
\pgfpathmoveto{\pgfqpoint{3.536187in}{3.013107in}}%
\pgfpathlineto{\pgfqpoint{3.549032in}{3.000601in}}%
\pgfpathlineto{\pgfqpoint{3.561875in}{2.988326in}}%
\pgfpathlineto{\pgfqpoint{3.574717in}{2.976281in}}%
\pgfpathlineto{\pgfqpoint{3.587559in}{2.964463in}}%
\pgfpathlineto{\pgfqpoint{3.595097in}{2.978682in}}%
\pgfpathlineto{\pgfqpoint{3.602631in}{2.993061in}}%
\pgfpathlineto{\pgfqpoint{3.610160in}{3.007605in}}%
\pgfpathlineto{\pgfqpoint{3.617684in}{3.022317in}}%
\pgfpathlineto{\pgfqpoint{3.604849in}{3.034437in}}%
\pgfpathlineto{\pgfqpoint{3.592013in}{3.046786in}}%
\pgfpathlineto{\pgfqpoint{3.579176in}{3.059365in}}%
\pgfpathlineto{\pgfqpoint{3.566338in}{3.072175in}}%
\pgfpathlineto{\pgfqpoint{3.558808in}{3.057149in}}%
\pgfpathlineto{\pgfqpoint{3.551273in}{3.042298in}}%
\pgfpathlineto{\pgfqpoint{3.543733in}{3.027619in}}%
\pgfpathlineto{\pgfqpoint{3.536187in}{3.013107in}}%
\pgfpathclose%
\pgfusepath{fill}%
\end{pgfscope}%
\begin{pgfscope}%
\pgfpathrectangle{\pgfqpoint{1.254980in}{0.150000in}}{\pgfqpoint{5.490039in}{5.490039in}}%
\pgfusepath{clip}%
\pgfsetbuttcap%
\pgfsetroundjoin%
\definecolor{currentfill}{rgb}{0.185556,0.418570,0.556753}%
\pgfsetfillcolor{currentfill}%
\pgfsetfillopacity{0.700000}%
\pgfsetlinewidth{0.000000pt}%
\definecolor{currentstroke}{rgb}{0.000000,0.000000,0.000000}%
\pgfsetstrokecolor{currentstroke}%
\pgfsetdash{}{0pt}%
\pgfpathmoveto{\pgfqpoint{4.767589in}{3.217759in}}%
\pgfpathlineto{\pgfqpoint{4.780588in}{3.212846in}}%
\pgfpathlineto{\pgfqpoint{4.793595in}{3.208096in}}%
\pgfpathlineto{\pgfqpoint{4.806609in}{3.203510in}}%
\pgfpathlineto{\pgfqpoint{4.819631in}{3.199086in}}%
\pgfpathlineto{\pgfqpoint{4.826932in}{3.215855in}}%
\pgfpathlineto{\pgfqpoint{4.834236in}{3.232927in}}%
\pgfpathlineto{\pgfqpoint{4.841542in}{3.250309in}}%
\pgfpathlineto{\pgfqpoint{4.848850in}{3.268010in}}%
\pgfpathlineto{\pgfqpoint{4.835841in}{3.273064in}}%
\pgfpathlineto{\pgfqpoint{4.822839in}{3.278281in}}%
\pgfpathlineto{\pgfqpoint{4.809844in}{3.283661in}}%
\pgfpathlineto{\pgfqpoint{4.796856in}{3.289204in}}%
\pgfpathlineto{\pgfqpoint{4.789536in}{3.270863in}}%
\pgfpathlineto{\pgfqpoint{4.782218in}{3.252846in}}%
\pgfpathlineto{\pgfqpoint{4.774903in}{3.235147in}}%
\pgfpathlineto{\pgfqpoint{4.767589in}{3.217759in}}%
\pgfpathclose%
\pgfusepath{fill}%
\end{pgfscope}%
\begin{pgfscope}%
\pgfpathrectangle{\pgfqpoint{1.254980in}{0.150000in}}{\pgfqpoint{5.490039in}{5.490039in}}%
\pgfusepath{clip}%
\pgfsetbuttcap%
\pgfsetroundjoin%
\definecolor{currentfill}{rgb}{0.235526,0.309527,0.542944}%
\pgfsetfillcolor{currentfill}%
\pgfsetfillopacity{0.700000}%
\pgfsetlinewidth{0.000000pt}%
\definecolor{currentstroke}{rgb}{0.000000,0.000000,0.000000}%
\pgfsetstrokecolor{currentstroke}%
\pgfsetdash{}{0pt}%
\pgfpathmoveto{\pgfqpoint{4.280671in}{2.967478in}}%
\pgfpathlineto{\pgfqpoint{4.293575in}{2.961658in}}%
\pgfpathlineto{\pgfqpoint{4.306485in}{2.956018in}}%
\pgfpathlineto{\pgfqpoint{4.319401in}{2.950555in}}%
\pgfpathlineto{\pgfqpoint{4.332322in}{2.945270in}}%
\pgfpathlineto{\pgfqpoint{4.339696in}{2.959481in}}%
\pgfpathlineto{\pgfqpoint{4.347067in}{2.973884in}}%
\pgfpathlineto{\pgfqpoint{4.354437in}{2.988485in}}%
\pgfpathlineto{\pgfqpoint{4.361805in}{3.003289in}}%
\pgfpathlineto{\pgfqpoint{4.348892in}{3.009039in}}%
\pgfpathlineto{\pgfqpoint{4.335985in}{3.014967in}}%
\pgfpathlineto{\pgfqpoint{4.323083in}{3.021073in}}%
\pgfpathlineto{\pgfqpoint{4.310186in}{3.027358in}}%
\pgfpathlineto{\pgfqpoint{4.302810in}{3.012078in}}%
\pgfpathlineto{\pgfqpoint{4.295433in}{2.997008in}}%
\pgfpathlineto{\pgfqpoint{4.288053in}{2.982144in}}%
\pgfpathlineto{\pgfqpoint{4.280671in}{2.967478in}}%
\pgfpathclose%
\pgfusepath{fill}%
\end{pgfscope}%
\begin{pgfscope}%
\pgfpathrectangle{\pgfqpoint{1.254980in}{0.150000in}}{\pgfqpoint{5.490039in}{5.490039in}}%
\pgfusepath{clip}%
\pgfsetbuttcap%
\pgfsetroundjoin%
\definecolor{currentfill}{rgb}{0.248629,0.278775,0.534556}%
\pgfsetfillcolor{currentfill}%
\pgfsetfillopacity{0.700000}%
\pgfsetlinewidth{0.000000pt}%
\definecolor{currentstroke}{rgb}{0.000000,0.000000,0.000000}%
\pgfsetstrokecolor{currentstroke}%
\pgfsetdash{}{0pt}%
\pgfpathmoveto{\pgfqpoint{3.985662in}{2.903660in}}%
\pgfpathlineto{\pgfqpoint{3.998520in}{2.896087in}}%
\pgfpathlineto{\pgfqpoint{4.011380in}{2.888708in}}%
\pgfpathlineto{\pgfqpoint{4.024244in}{2.881521in}}%
\pgfpathlineto{\pgfqpoint{4.037112in}{2.874526in}}%
\pgfpathlineto{\pgfqpoint{4.044553in}{2.888312in}}%
\pgfpathlineto{\pgfqpoint{4.051990in}{2.902258in}}%
\pgfpathlineto{\pgfqpoint{4.059423in}{2.916369in}}%
\pgfpathlineto{\pgfqpoint{4.066853in}{2.930651in}}%
\pgfpathlineto{\pgfqpoint{4.053993in}{2.938029in}}%
\pgfpathlineto{\pgfqpoint{4.041135in}{2.945599in}}%
\pgfpathlineto{\pgfqpoint{4.028282in}{2.953362in}}%
\pgfpathlineto{\pgfqpoint{4.015431in}{2.961318in}}%
\pgfpathlineto{\pgfqpoint{4.007994in}{2.946643in}}%
\pgfpathlineto{\pgfqpoint{4.000554in}{2.932144in}}%
\pgfpathlineto{\pgfqpoint{3.993110in}{2.917818in}}%
\pgfpathlineto{\pgfqpoint{3.985662in}{2.903660in}}%
\pgfpathclose%
\pgfusepath{fill}%
\end{pgfscope}%
\begin{pgfscope}%
\pgfpathrectangle{\pgfqpoint{1.254980in}{0.150000in}}{\pgfqpoint{5.490039in}{5.490039in}}%
\pgfusepath{clip}%
\pgfsetbuttcap%
\pgfsetroundjoin%
\definecolor{currentfill}{rgb}{0.162142,0.474838,0.558140}%
\pgfsetfillcolor{currentfill}%
\pgfsetfillopacity{0.700000}%
\pgfsetlinewidth{0.000000pt}%
\definecolor{currentstroke}{rgb}{0.000000,0.000000,0.000000}%
\pgfsetstrokecolor{currentstroke}%
\pgfsetdash{}{0pt}%
\pgfpathmoveto{\pgfqpoint{3.227052in}{3.387620in}}%
\pgfpathlineto{\pgfqpoint{3.239986in}{3.368966in}}%
\pgfpathlineto{\pgfqpoint{3.252914in}{3.350594in}}%
\pgfpathlineto{\pgfqpoint{3.265835in}{3.332502in}}%
\pgfpathlineto{\pgfqpoint{3.278751in}{3.314688in}}%
\pgfpathlineto{\pgfqpoint{3.286330in}{3.330678in}}%
\pgfpathlineto{\pgfqpoint{3.293902in}{3.346872in}}%
\pgfpathlineto{\pgfqpoint{3.301469in}{3.363274in}}%
\pgfpathlineto{\pgfqpoint{3.309030in}{3.379886in}}%
\pgfpathlineto{\pgfqpoint{3.296120in}{3.398010in}}%
\pgfpathlineto{\pgfqpoint{3.283205in}{3.416412in}}%
\pgfpathlineto{\pgfqpoint{3.270284in}{3.435095in}}%
\pgfpathlineto{\pgfqpoint{3.257356in}{3.454061in}}%
\pgfpathlineto{\pgfqpoint{3.249790in}{3.437127in}}%
\pgfpathlineto{\pgfqpoint{3.242217in}{3.420412in}}%
\pgfpathlineto{\pgfqpoint{3.234638in}{3.403911in}}%
\pgfpathlineto{\pgfqpoint{3.227052in}{3.387620in}}%
\pgfpathclose%
\pgfusepath{fill}%
\end{pgfscope}%
\begin{pgfscope}%
\pgfpathrectangle{\pgfqpoint{1.254980in}{0.150000in}}{\pgfqpoint{5.490039in}{5.490039in}}%
\pgfusepath{clip}%
\pgfsetbuttcap%
\pgfsetroundjoin%
\definecolor{currentfill}{rgb}{0.237441,0.305202,0.541921}%
\pgfsetfillcolor{currentfill}%
\pgfsetfillopacity{0.700000}%
\pgfsetlinewidth{0.000000pt}%
\definecolor{currentstroke}{rgb}{0.000000,0.000000,0.000000}%
\pgfsetstrokecolor{currentstroke}%
\pgfsetdash{}{0pt}%
\pgfpathmoveto{\pgfqpoint{3.587559in}{2.964463in}}%
\pgfpathlineto{\pgfqpoint{3.600400in}{2.952871in}}%
\pgfpathlineto{\pgfqpoint{3.613240in}{2.941504in}}%
\pgfpathlineto{\pgfqpoint{3.626081in}{2.930360in}}%
\pgfpathlineto{\pgfqpoint{3.638921in}{2.919437in}}%
\pgfpathlineto{\pgfqpoint{3.646453in}{2.933363in}}%
\pgfpathlineto{\pgfqpoint{3.653980in}{2.947443in}}%
\pgfpathlineto{\pgfqpoint{3.661502in}{2.961681in}}%
\pgfpathlineto{\pgfqpoint{3.669019in}{2.976079in}}%
\pgfpathlineto{\pgfqpoint{3.656186in}{2.987305in}}%
\pgfpathlineto{\pgfqpoint{3.643352in}{2.998752in}}%
\pgfpathlineto{\pgfqpoint{3.630518in}{3.010422in}}%
\pgfpathlineto{\pgfqpoint{3.617684in}{3.022317in}}%
\pgfpathlineto{\pgfqpoint{3.610160in}{3.007605in}}%
\pgfpathlineto{\pgfqpoint{3.602631in}{2.993061in}}%
\pgfpathlineto{\pgfqpoint{3.595097in}{2.978682in}}%
\pgfpathlineto{\pgfqpoint{3.587559in}{2.964463in}}%
\pgfpathclose%
\pgfusepath{fill}%
\end{pgfscope}%
\begin{pgfscope}%
\pgfpathrectangle{\pgfqpoint{1.254980in}{0.150000in}}{\pgfqpoint{5.490039in}{5.490039in}}%
\pgfusepath{clip}%
\pgfsetbuttcap%
\pgfsetroundjoin%
\definecolor{currentfill}{rgb}{0.177423,0.437527,0.557565}%
\pgfsetfillcolor{currentfill}%
\pgfsetfillopacity{0.700000}%
\pgfsetlinewidth{0.000000pt}%
\definecolor{currentstroke}{rgb}{0.000000,0.000000,0.000000}%
\pgfsetstrokecolor{currentstroke}%
\pgfsetdash{}{0pt}%
\pgfpathmoveto{\pgfqpoint{4.848850in}{3.268010in}}%
\pgfpathlineto{\pgfqpoint{4.861867in}{3.263118in}}%
\pgfpathlineto{\pgfqpoint{4.874892in}{3.258387in}}%
\pgfpathlineto{\pgfqpoint{4.887924in}{3.253818in}}%
\pgfpathlineto{\pgfqpoint{4.900964in}{3.249410in}}%
\pgfpathlineto{\pgfqpoint{4.908262in}{3.266788in}}%
\pgfpathlineto{\pgfqpoint{4.915564in}{3.284492in}}%
\pgfpathlineto{\pgfqpoint{4.922868in}{3.302530in}}%
\pgfpathlineto{\pgfqpoint{4.909838in}{3.307430in}}%
\pgfpathlineto{\pgfqpoint{4.896816in}{3.312491in}}%
\pgfpathlineto{\pgfqpoint{4.883801in}{3.317712in}}%
\pgfpathlineto{\pgfqpoint{4.870793in}{3.323095in}}%
\pgfpathlineto{\pgfqpoint{4.863476in}{3.304395in}}%
\pgfpathlineto{\pgfqpoint{4.856162in}{3.286036in}}%
\pgfpathlineto{\pgfqpoint{4.848850in}{3.268010in}}%
\pgfpathclose%
\pgfusepath{fill}%
\end{pgfscope}%
\begin{pgfscope}%
\pgfpathrectangle{\pgfqpoint{1.254980in}{0.150000in}}{\pgfqpoint{5.490039in}{5.490039in}}%
\pgfusepath{clip}%
\pgfsetbuttcap%
\pgfsetroundjoin%
\definecolor{currentfill}{rgb}{0.993248,0.906157,0.143936}%
\pgfsetfillcolor{currentfill}%
\pgfsetfillopacity{0.700000}%
\pgfsetlinewidth{0.000000pt}%
\definecolor{currentstroke}{rgb}{0.000000,0.000000,0.000000}%
\pgfsetstrokecolor{currentstroke}%
\pgfsetdash{}{0pt}%
\pgfpathmoveto{\pgfqpoint{3.510681in}{4.832193in}}%
\pgfpathlineto{\pgfqpoint{3.523682in}{4.804577in}}%
\pgfpathlineto{\pgfqpoint{3.536674in}{4.777283in}}%
\pgfpathlineto{\pgfqpoint{3.549658in}{4.750306in}}%
\pgfpathlineto{\pgfqpoint{3.562632in}{4.723643in}}%
\pgfpathlineto{\pgfqpoint{3.569969in}{4.758524in}}%
\pgfpathlineto{\pgfqpoint{3.577303in}{4.793942in}}%
\pgfpathlineto{\pgfqpoint{3.584635in}{4.829905in}}%
\pgfpathlineto{\pgfqpoint{3.571653in}{4.857125in}}%
\pgfpathlineto{\pgfqpoint{3.558663in}{4.884661in}}%
\pgfpathlineto{\pgfqpoint{3.545663in}{4.912516in}}%
\pgfpathlineto{\pgfqpoint{3.532654in}{4.940694in}}%
\pgfpathlineto{\pgfqpoint{3.525332in}{4.903974in}}%
\pgfpathlineto{\pgfqpoint{3.518008in}{4.867810in}}%
\pgfpathlineto{\pgfqpoint{3.510681in}{4.832193in}}%
\pgfpathclose%
\pgfusepath{fill}%
\end{pgfscope}%
\begin{pgfscope}%
\pgfpathrectangle{\pgfqpoint{1.254980in}{0.150000in}}{\pgfqpoint{5.490039in}{5.490039in}}%
\pgfusepath{clip}%
\pgfsetbuttcap%
\pgfsetroundjoin%
\definecolor{currentfill}{rgb}{0.241237,0.296485,0.539709}%
\pgfsetfillcolor{currentfill}%
\pgfsetfillopacity{0.700000}%
\pgfsetlinewidth{0.000000pt}%
\definecolor{currentstroke}{rgb}{0.000000,0.000000,0.000000}%
\pgfsetstrokecolor{currentstroke}%
\pgfsetdash{}{0pt}%
\pgfpathmoveto{\pgfqpoint{4.199518in}{2.934043in}}%
\pgfpathlineto{\pgfqpoint{4.212410in}{2.927936in}}%
\pgfpathlineto{\pgfqpoint{4.225307in}{2.922011in}}%
\pgfpathlineto{\pgfqpoint{4.238209in}{2.916268in}}%
\pgfpathlineto{\pgfqpoint{4.251117in}{2.910706in}}%
\pgfpathlineto{\pgfqpoint{4.258509in}{2.924626in}}%
\pgfpathlineto{\pgfqpoint{4.265899in}{2.938725in}}%
\pgfpathlineto{\pgfqpoint{4.273286in}{2.953007in}}%
\pgfpathlineto{\pgfqpoint{4.280671in}{2.967478in}}%
\pgfpathlineto{\pgfqpoint{4.267771in}{2.973478in}}%
\pgfpathlineto{\pgfqpoint{4.254877in}{2.979659in}}%
\pgfpathlineto{\pgfqpoint{4.241987in}{2.986022in}}%
\pgfpathlineto{\pgfqpoint{4.229103in}{2.992566in}}%
\pgfpathlineto{\pgfqpoint{4.221711in}{2.977647in}}%
\pgfpathlineto{\pgfqpoint{4.214316in}{2.962924in}}%
\pgfpathlineto{\pgfqpoint{4.206918in}{2.948391in}}%
\pgfpathlineto{\pgfqpoint{4.199518in}{2.934043in}}%
\pgfpathclose%
\pgfusepath{fill}%
\end{pgfscope}%
\begin{pgfscope}%
\pgfpathrectangle{\pgfqpoint{1.254980in}{0.150000in}}{\pgfqpoint{5.490039in}{5.490039in}}%
\pgfusepath{clip}%
\pgfsetbuttcap%
\pgfsetroundjoin%
\definecolor{currentfill}{rgb}{0.128729,0.563265,0.551229}%
\pgfsetfillcolor{currentfill}%
\pgfsetfillopacity{0.700000}%
\pgfsetlinewidth{0.000000pt}%
\definecolor{currentstroke}{rgb}{0.000000,0.000000,0.000000}%
\pgfsetstrokecolor{currentstroke}%
\pgfsetdash{}{0pt}%
\pgfpathmoveto{\pgfqpoint{3.153678in}{3.616320in}}%
\pgfpathlineto{\pgfqpoint{3.166665in}{3.594987in}}%
\pgfpathlineto{\pgfqpoint{3.179644in}{3.573960in}}%
\pgfpathlineto{\pgfqpoint{3.192615in}{3.553236in}}%
\pgfpathlineto{\pgfqpoint{3.205578in}{3.532812in}}%
\pgfpathlineto{\pgfqpoint{3.213144in}{3.550300in}}%
\pgfpathlineto{\pgfqpoint{3.220704in}{3.568020in}}%
\pgfpathlineto{\pgfqpoint{3.228258in}{3.585979in}}%
\pgfpathlineto{\pgfqpoint{3.235805in}{3.604179in}}%
\pgfpathlineto{\pgfqpoint{3.222848in}{3.624947in}}%
\pgfpathlineto{\pgfqpoint{3.209882in}{3.646015in}}%
\pgfpathlineto{\pgfqpoint{3.196909in}{3.667386in}}%
\pgfpathlineto{\pgfqpoint{3.183927in}{3.689065in}}%
\pgfpathlineto{\pgfqpoint{3.176374in}{3.670508in}}%
\pgfpathlineto{\pgfqpoint{3.168815in}{3.652202in}}%
\pgfpathlineto{\pgfqpoint{3.161250in}{3.634140in}}%
\pgfpathlineto{\pgfqpoint{3.153678in}{3.616320in}}%
\pgfpathclose%
\pgfusepath{fill}%
\end{pgfscope}%
\begin{pgfscope}%
\pgfpathrectangle{\pgfqpoint{1.254980in}{0.150000in}}{\pgfqpoint{5.490039in}{5.490039in}}%
\pgfusepath{clip}%
\pgfsetbuttcap%
\pgfsetroundjoin%
\definecolor{currentfill}{rgb}{0.250425,0.274290,0.533103}%
\pgfsetfillcolor{currentfill}%
\pgfsetfillopacity{0.700000}%
\pgfsetlinewidth{0.000000pt}%
\definecolor{currentstroke}{rgb}{0.000000,0.000000,0.000000}%
\pgfsetstrokecolor{currentstroke}%
\pgfsetdash{}{0pt}%
\pgfpathmoveto{\pgfqpoint{3.771708in}{2.894072in}}%
\pgfpathlineto{\pgfqpoint{3.784549in}{2.884774in}}%
\pgfpathlineto{\pgfqpoint{3.797391in}{2.875684in}}%
\pgfpathlineto{\pgfqpoint{3.810236in}{2.866800in}}%
\pgfpathlineto{\pgfqpoint{3.823082in}{2.858120in}}%
\pgfpathlineto{\pgfqpoint{3.830575in}{2.871727in}}%
\pgfpathlineto{\pgfqpoint{3.838064in}{2.885481in}}%
\pgfpathlineto{\pgfqpoint{3.845548in}{2.899387in}}%
\pgfpathlineto{\pgfqpoint{3.853029in}{2.913450in}}%
\pgfpathlineto{\pgfqpoint{3.840189in}{2.922458in}}%
\pgfpathlineto{\pgfqpoint{3.827351in}{2.931671in}}%
\pgfpathlineto{\pgfqpoint{3.814515in}{2.941091in}}%
\pgfpathlineto{\pgfqpoint{3.801681in}{2.950718in}}%
\pgfpathlineto{\pgfqpoint{3.794194in}{2.936316in}}%
\pgfpathlineto{\pgfqpoint{3.786703in}{2.922077in}}%
\pgfpathlineto{\pgfqpoint{3.779208in}{2.907997in}}%
\pgfpathlineto{\pgfqpoint{3.771708in}{2.894072in}}%
\pgfpathclose%
\pgfusepath{fill}%
\end{pgfscope}%
\begin{pgfscope}%
\pgfpathrectangle{\pgfqpoint{1.254980in}{0.150000in}}{\pgfqpoint{5.490039in}{5.490039in}}%
\pgfusepath{clip}%
\pgfsetbuttcap%
\pgfsetroundjoin%
\definecolor{currentfill}{rgb}{0.246811,0.283237,0.535941}%
\pgfsetfillcolor{currentfill}%
\pgfsetfillopacity{0.700000}%
\pgfsetlinewidth{0.000000pt}%
\definecolor{currentstroke}{rgb}{0.000000,0.000000,0.000000}%
\pgfsetstrokecolor{currentstroke}%
\pgfsetdash{}{0pt}%
\pgfpathmoveto{\pgfqpoint{4.118336in}{2.903034in}}%
\pgfpathlineto{\pgfqpoint{4.131217in}{2.896599in}}%
\pgfpathlineto{\pgfqpoint{4.144102in}{2.890350in}}%
\pgfpathlineto{\pgfqpoint{4.156993in}{2.884286in}}%
\pgfpathlineto{\pgfqpoint{4.169888in}{2.878406in}}%
\pgfpathlineto{\pgfqpoint{4.177300in}{2.892062in}}%
\pgfpathlineto{\pgfqpoint{4.184709in}{2.905883in}}%
\pgfpathlineto{\pgfqpoint{4.192115in}{2.919875in}}%
\pgfpathlineto{\pgfqpoint{4.199518in}{2.934043in}}%
\pgfpathlineto{\pgfqpoint{4.186631in}{2.940333in}}%
\pgfpathlineto{\pgfqpoint{4.173748in}{2.946808in}}%
\pgfpathlineto{\pgfqpoint{4.160870in}{2.953467in}}%
\pgfpathlineto{\pgfqpoint{4.147996in}{2.960312in}}%
\pgfpathlineto{\pgfqpoint{4.140586in}{2.945724in}}%
\pgfpathlineto{\pgfqpoint{4.133172in}{2.931319in}}%
\pgfpathlineto{\pgfqpoint{4.125755in}{2.917090in}}%
\pgfpathlineto{\pgfqpoint{4.118336in}{2.903034in}}%
\pgfpathclose%
\pgfusepath{fill}%
\end{pgfscope}%
\begin{pgfscope}%
\pgfpathrectangle{\pgfqpoint{1.254980in}{0.150000in}}{\pgfqpoint{5.490039in}{5.490039in}}%
\pgfusepath{clip}%
\pgfsetbuttcap%
\pgfsetroundjoin%
\definecolor{currentfill}{rgb}{0.252194,0.269783,0.531579}%
\pgfsetfillcolor{currentfill}%
\pgfsetfillopacity{0.700000}%
\pgfsetlinewidth{0.000000pt}%
\definecolor{currentstroke}{rgb}{0.000000,0.000000,0.000000}%
\pgfsetstrokecolor{currentstroke}%
\pgfsetdash{}{0pt}%
\pgfpathmoveto{\pgfqpoint{3.904410in}{2.879441in}}%
\pgfpathlineto{\pgfqpoint{3.917262in}{2.871440in}}%
\pgfpathlineto{\pgfqpoint{3.930116in}{2.863636in}}%
\pgfpathlineto{\pgfqpoint{3.942974in}{2.856029in}}%
\pgfpathlineto{\pgfqpoint{3.955835in}{2.848617in}}%
\pgfpathlineto{\pgfqpoint{3.963297in}{2.862148in}}%
\pgfpathlineto{\pgfqpoint{3.970756in}{2.875829in}}%
\pgfpathlineto{\pgfqpoint{3.978211in}{2.889665in}}%
\pgfpathlineto{\pgfqpoint{3.985662in}{2.903660in}}%
\pgfpathlineto{\pgfqpoint{3.972809in}{2.911428in}}%
\pgfpathlineto{\pgfqpoint{3.959958in}{2.919391in}}%
\pgfpathlineto{\pgfqpoint{3.947110in}{2.927551in}}%
\pgfpathlineto{\pgfqpoint{3.934265in}{2.935909in}}%
\pgfpathlineto{\pgfqpoint{3.926807in}{2.921547in}}%
\pgfpathlineto{\pgfqpoint{3.919345in}{2.907352in}}%
\pgfpathlineto{\pgfqpoint{3.911880in}{2.893318in}}%
\pgfpathlineto{\pgfqpoint{3.904410in}{2.879441in}}%
\pgfpathclose%
\pgfusepath{fill}%
\end{pgfscope}%
\begin{pgfscope}%
\pgfpathrectangle{\pgfqpoint{1.254980in}{0.150000in}}{\pgfqpoint{5.490039in}{5.490039in}}%
\pgfusepath{clip}%
\pgfsetbuttcap%
\pgfsetroundjoin%
\definecolor{currentfill}{rgb}{0.149039,0.508051,0.557250}%
\pgfsetfillcolor{currentfill}%
\pgfsetfillopacity{0.700000}%
\pgfsetlinewidth{0.000000pt}%
\definecolor{currentstroke}{rgb}{0.000000,0.000000,0.000000}%
\pgfsetstrokecolor{currentstroke}%
\pgfsetdash{}{0pt}%
\pgfpathmoveto{\pgfqpoint{3.175248in}{3.465122in}}%
\pgfpathlineto{\pgfqpoint{3.188210in}{3.445309in}}%
\pgfpathlineto{\pgfqpoint{3.201165in}{3.425789in}}%
\pgfpathlineto{\pgfqpoint{3.214112in}{3.406561in}}%
\pgfpathlineto{\pgfqpoint{3.227052in}{3.387620in}}%
\pgfpathlineto{\pgfqpoint{3.234638in}{3.403911in}}%
\pgfpathlineto{\pgfqpoint{3.242217in}{3.420412in}}%
\pgfpathlineto{\pgfqpoint{3.249790in}{3.437127in}}%
\pgfpathlineto{\pgfqpoint{3.257356in}{3.454061in}}%
\pgfpathlineto{\pgfqpoint{3.244422in}{3.473313in}}%
\pgfpathlineto{\pgfqpoint{3.231481in}{3.492854in}}%
\pgfpathlineto{\pgfqpoint{3.218533in}{3.512686in}}%
\pgfpathlineto{\pgfqpoint{3.205578in}{3.532812in}}%
\pgfpathlineto{\pgfqpoint{3.198005in}{3.515555in}}%
\pgfpathlineto{\pgfqpoint{3.190426in}{3.498523in}}%
\pgfpathlineto{\pgfqpoint{3.182840in}{3.481713in}}%
\pgfpathlineto{\pgfqpoint{3.175248in}{3.465122in}}%
\pgfpathclose%
\pgfusepath{fill}%
\end{pgfscope}%
\begin{pgfscope}%
\pgfpathrectangle{\pgfqpoint{1.254980in}{0.150000in}}{\pgfqpoint{5.490039in}{5.490039in}}%
\pgfusepath{clip}%
\pgfsetbuttcap%
\pgfsetroundjoin%
\definecolor{currentfill}{rgb}{0.244972,0.287675,0.537260}%
\pgfsetfillcolor{currentfill}%
\pgfsetfillopacity{0.700000}%
\pgfsetlinewidth{0.000000pt}%
\definecolor{currentstroke}{rgb}{0.000000,0.000000,0.000000}%
\pgfsetstrokecolor{currentstroke}%
\pgfsetdash{}{0pt}%
\pgfpathmoveto{\pgfqpoint{3.638921in}{2.919437in}}%
\pgfpathlineto{\pgfqpoint{3.651762in}{2.908734in}}%
\pgfpathlineto{\pgfqpoint{3.664602in}{2.898250in}}%
\pgfpathlineto{\pgfqpoint{3.677444in}{2.887982in}}%
\pgfpathlineto{\pgfqpoint{3.690286in}{2.877930in}}%
\pgfpathlineto{\pgfqpoint{3.697811in}{2.891564in}}%
\pgfpathlineto{\pgfqpoint{3.705331in}{2.905345in}}%
\pgfpathlineto{\pgfqpoint{3.712846in}{2.919277in}}%
\pgfpathlineto{\pgfqpoint{3.720357in}{2.933363in}}%
\pgfpathlineto{\pgfqpoint{3.707522in}{2.943717in}}%
\pgfpathlineto{\pgfqpoint{3.694687in}{2.954287in}}%
\pgfpathlineto{\pgfqpoint{3.681853in}{2.965074in}}%
\pgfpathlineto{\pgfqpoint{3.669019in}{2.976079in}}%
\pgfpathlineto{\pgfqpoint{3.661502in}{2.961681in}}%
\pgfpathlineto{\pgfqpoint{3.653980in}{2.947443in}}%
\pgfpathlineto{\pgfqpoint{3.646453in}{2.933363in}}%
\pgfpathlineto{\pgfqpoint{3.638921in}{2.919437in}}%
\pgfpathclose%
\pgfusepath{fill}%
\end{pgfscope}%
\begin{pgfscope}%
\pgfpathrectangle{\pgfqpoint{1.254980in}{0.150000in}}{\pgfqpoint{5.490039in}{5.490039in}}%
\pgfusepath{clip}%
\pgfsetbuttcap%
\pgfsetroundjoin%
\definecolor{currentfill}{rgb}{0.204903,0.375746,0.553533}%
\pgfsetfillcolor{currentfill}%
\pgfsetfillopacity{0.700000}%
\pgfsetlinewidth{0.000000pt}%
\definecolor{currentstroke}{rgb}{0.000000,0.000000,0.000000}%
\pgfsetstrokecolor{currentstroke}%
\pgfsetdash{}{0pt}%
\pgfpathmoveto{\pgfqpoint{3.351576in}{3.122111in}}%
\pgfpathlineto{\pgfqpoint{3.364457in}{3.106965in}}%
\pgfpathlineto{\pgfqpoint{3.377334in}{3.092073in}}%
\pgfpathlineto{\pgfqpoint{3.390208in}{3.077432in}}%
\pgfpathlineto{\pgfqpoint{3.403078in}{3.063041in}}%
\pgfpathlineto{\pgfqpoint{3.410659in}{3.077448in}}%
\pgfpathlineto{\pgfqpoint{3.418234in}{3.092024in}}%
\pgfpathlineto{\pgfqpoint{3.425804in}{3.106770in}}%
\pgfpathlineto{\pgfqpoint{3.433368in}{3.121692in}}%
\pgfpathlineto{\pgfqpoint{3.420504in}{3.136360in}}%
\pgfpathlineto{\pgfqpoint{3.407638in}{3.151278in}}%
\pgfpathlineto{\pgfqpoint{3.394768in}{3.166449in}}%
\pgfpathlineto{\pgfqpoint{3.381894in}{3.181873in}}%
\pgfpathlineto{\pgfqpoint{3.374323in}{3.166663in}}%
\pgfpathlineto{\pgfqpoint{3.366747in}{3.151635in}}%
\pgfpathlineto{\pgfqpoint{3.359164in}{3.136786in}}%
\pgfpathlineto{\pgfqpoint{3.351576in}{3.122111in}}%
\pgfpathclose%
\pgfusepath{fill}%
\end{pgfscope}%
\begin{pgfscope}%
\pgfpathrectangle{\pgfqpoint{1.254980in}{0.150000in}}{\pgfqpoint{5.490039in}{5.490039in}}%
\pgfusepath{clip}%
\pgfsetbuttcap%
\pgfsetroundjoin%
\definecolor{currentfill}{rgb}{0.216210,0.351535,0.550627}%
\pgfsetfillcolor{currentfill}%
\pgfsetfillopacity{0.700000}%
\pgfsetlinewidth{0.000000pt}%
\definecolor{currentstroke}{rgb}{0.000000,0.000000,0.000000}%
\pgfsetstrokecolor{currentstroke}%
\pgfsetdash{}{0pt}%
\pgfpathmoveto{\pgfqpoint{3.403078in}{3.063041in}}%
\pgfpathlineto{\pgfqpoint{3.415946in}{3.048898in}}%
\pgfpathlineto{\pgfqpoint{3.428811in}{3.035001in}}%
\pgfpathlineto{\pgfqpoint{3.441673in}{3.021347in}}%
\pgfpathlineto{\pgfqpoint{3.454533in}{3.007936in}}%
\pgfpathlineto{\pgfqpoint{3.462107in}{3.022076in}}%
\pgfpathlineto{\pgfqpoint{3.469675in}{3.036378in}}%
\pgfpathlineto{\pgfqpoint{3.477237in}{3.050844in}}%
\pgfpathlineto{\pgfqpoint{3.484794in}{3.065478in}}%
\pgfpathlineto{\pgfqpoint{3.471941in}{3.079166in}}%
\pgfpathlineto{\pgfqpoint{3.459086in}{3.093097in}}%
\pgfpathlineto{\pgfqpoint{3.446228in}{3.107271in}}%
\pgfpathlineto{\pgfqpoint{3.433368in}{3.121692in}}%
\pgfpathlineto{\pgfqpoint{3.425804in}{3.106770in}}%
\pgfpathlineto{\pgfqpoint{3.418234in}{3.092024in}}%
\pgfpathlineto{\pgfqpoint{3.410659in}{3.077448in}}%
\pgfpathlineto{\pgfqpoint{3.403078in}{3.063041in}}%
\pgfpathclose%
\pgfusepath{fill}%
\end{pgfscope}%
\begin{pgfscope}%
\pgfpathrectangle{\pgfqpoint{1.254980in}{0.150000in}}{\pgfqpoint{5.490039in}{5.490039in}}%
\pgfusepath{clip}%
\pgfsetbuttcap%
\pgfsetroundjoin%
\definecolor{currentfill}{rgb}{0.194100,0.399323,0.555565}%
\pgfsetfillcolor{currentfill}%
\pgfsetfillopacity{0.700000}%
\pgfsetlinewidth{0.000000pt}%
\definecolor{currentstroke}{rgb}{0.000000,0.000000,0.000000}%
\pgfsetstrokecolor{currentstroke}%
\pgfsetdash{}{0pt}%
\pgfpathmoveto{\pgfqpoint{3.300013in}{3.185279in}}%
\pgfpathlineto{\pgfqpoint{3.312911in}{3.169095in}}%
\pgfpathlineto{\pgfqpoint{3.325803in}{3.153174in}}%
\pgfpathlineto{\pgfqpoint{3.338692in}{3.137514in}}%
\pgfpathlineto{\pgfqpoint{3.351576in}{3.122111in}}%
\pgfpathlineto{\pgfqpoint{3.359164in}{3.136786in}}%
\pgfpathlineto{\pgfqpoint{3.366747in}{3.151635in}}%
\pgfpathlineto{\pgfqpoint{3.374323in}{3.166663in}}%
\pgfpathlineto{\pgfqpoint{3.381894in}{3.181873in}}%
\pgfpathlineto{\pgfqpoint{3.369017in}{3.197554in}}%
\pgfpathlineto{\pgfqpoint{3.356135in}{3.213494in}}%
\pgfpathlineto{\pgfqpoint{3.343250in}{3.229694in}}%
\pgfpathlineto{\pgfqpoint{3.330360in}{3.246157in}}%
\pgfpathlineto{\pgfqpoint{3.322782in}{3.230657in}}%
\pgfpathlineto{\pgfqpoint{3.315199in}{3.215347in}}%
\pgfpathlineto{\pgfqpoint{3.307609in}{3.200222in}}%
\pgfpathlineto{\pgfqpoint{3.300013in}{3.185279in}}%
\pgfpathclose%
\pgfusepath{fill}%
\end{pgfscope}%
\begin{pgfscope}%
\pgfpathrectangle{\pgfqpoint{1.254980in}{0.150000in}}{\pgfqpoint{5.490039in}{5.490039in}}%
\pgfusepath{clip}%
\pgfsetbuttcap%
\pgfsetroundjoin%
\definecolor{currentfill}{rgb}{0.212395,0.359683,0.551710}%
\pgfsetfillcolor{currentfill}%
\pgfsetfillopacity{0.700000}%
\pgfsetlinewidth{0.000000pt}%
\definecolor{currentstroke}{rgb}{0.000000,0.000000,0.000000}%
\pgfsetstrokecolor{currentstroke}%
\pgfsetdash{}{0pt}%
\pgfpathmoveto{\pgfqpoint{4.575896in}{3.062182in}}%
\pgfpathlineto{\pgfqpoint{4.588872in}{3.057665in}}%
\pgfpathlineto{\pgfqpoint{4.601854in}{3.053316in}}%
\pgfpathlineto{\pgfqpoint{4.614843in}{3.049135in}}%
\pgfpathlineto{\pgfqpoint{4.627840in}{3.045121in}}%
\pgfpathlineto{\pgfqpoint{4.635158in}{3.059868in}}%
\pgfpathlineto{\pgfqpoint{4.642475in}{3.074851in}}%
\pgfpathlineto{\pgfqpoint{4.649792in}{3.090075in}}%
\pgfpathlineto{\pgfqpoint{4.657109in}{3.105548in}}%
\pgfpathlineto{\pgfqpoint{4.644123in}{3.110109in}}%
\pgfpathlineto{\pgfqpoint{4.631144in}{3.114838in}}%
\pgfpathlineto{\pgfqpoint{4.618172in}{3.119735in}}%
\pgfpathlineto{\pgfqpoint{4.605207in}{3.124799in}}%
\pgfpathlineto{\pgfqpoint{4.597880in}{3.108769in}}%
\pgfpathlineto{\pgfqpoint{4.590552in}{3.092994in}}%
\pgfpathlineto{\pgfqpoint{4.583224in}{3.077467in}}%
\pgfpathlineto{\pgfqpoint{4.575896in}{3.062182in}}%
\pgfpathclose%
\pgfusepath{fill}%
\end{pgfscope}%
\begin{pgfscope}%
\pgfpathrectangle{\pgfqpoint{1.254980in}{0.150000in}}{\pgfqpoint{5.490039in}{5.490039in}}%
\pgfusepath{clip}%
\pgfsetbuttcap%
\pgfsetroundjoin%
\definecolor{currentfill}{rgb}{0.204903,0.375746,0.553533}%
\pgfsetfillcolor{currentfill}%
\pgfsetfillopacity{0.700000}%
\pgfsetlinewidth{0.000000pt}%
\definecolor{currentstroke}{rgb}{0.000000,0.000000,0.000000}%
\pgfsetstrokecolor{currentstroke}%
\pgfsetdash{}{0pt}%
\pgfpathmoveto{\pgfqpoint{4.657109in}{3.105548in}}%
\pgfpathlineto{\pgfqpoint{4.670102in}{3.101152in}}%
\pgfpathlineto{\pgfqpoint{4.683102in}{3.096923in}}%
\pgfpathlineto{\pgfqpoint{4.696110in}{3.092859in}}%
\pgfpathlineto{\pgfqpoint{4.709125in}{3.088960in}}%
\pgfpathlineto{\pgfqpoint{4.716430in}{3.104122in}}%
\pgfpathlineto{\pgfqpoint{4.723736in}{3.119538in}}%
\pgfpathlineto{\pgfqpoint{4.731043in}{3.135216in}}%
\pgfpathlineto{\pgfqpoint{4.738350in}{3.151161in}}%
\pgfpathlineto{\pgfqpoint{4.725346in}{3.155636in}}%
\pgfpathlineto{\pgfqpoint{4.712350in}{3.160275in}}%
\pgfpathlineto{\pgfqpoint{4.699361in}{3.165080in}}%
\pgfpathlineto{\pgfqpoint{4.686379in}{3.170050in}}%
\pgfpathlineto{\pgfqpoint{4.679060in}{3.153519in}}%
\pgfpathlineto{\pgfqpoint{4.671743in}{3.137263in}}%
\pgfpathlineto{\pgfqpoint{4.664426in}{3.121275in}}%
\pgfpathlineto{\pgfqpoint{4.657109in}{3.105548in}}%
\pgfpathclose%
\pgfusepath{fill}%
\end{pgfscope}%
\begin{pgfscope}%
\pgfpathrectangle{\pgfqpoint{1.254980in}{0.150000in}}{\pgfqpoint{5.490039in}{5.490039in}}%
\pgfusepath{clip}%
\pgfsetbuttcap%
\pgfsetroundjoin%
\definecolor{currentfill}{rgb}{0.221989,0.339161,0.548752}%
\pgfsetfillcolor{currentfill}%
\pgfsetfillopacity{0.700000}%
\pgfsetlinewidth{0.000000pt}%
\definecolor{currentstroke}{rgb}{0.000000,0.000000,0.000000}%
\pgfsetstrokecolor{currentstroke}%
\pgfsetdash{}{0pt}%
\pgfpathmoveto{\pgfqpoint{4.494701in}{3.021023in}}%
\pgfpathlineto{\pgfqpoint{4.507660in}{3.016347in}}%
\pgfpathlineto{\pgfqpoint{4.520625in}{3.011842in}}%
\pgfpathlineto{\pgfqpoint{4.533596in}{3.007507in}}%
\pgfpathlineto{\pgfqpoint{4.546575in}{3.003341in}}%
\pgfpathlineto{\pgfqpoint{4.553907in}{3.017719in}}%
\pgfpathlineto{\pgfqpoint{4.561238in}{3.032314in}}%
\pgfpathlineto{\pgfqpoint{4.568567in}{3.047133in}}%
\pgfpathlineto{\pgfqpoint{4.575896in}{3.062182in}}%
\pgfpathlineto{\pgfqpoint{4.562928in}{3.066868in}}%
\pgfpathlineto{\pgfqpoint{4.549966in}{3.071723in}}%
\pgfpathlineto{\pgfqpoint{4.537011in}{3.076748in}}%
\pgfpathlineto{\pgfqpoint{4.524062in}{3.081944in}}%
\pgfpathlineto{\pgfqpoint{4.516723in}{3.066365in}}%
\pgfpathlineto{\pgfqpoint{4.509383in}{3.051023in}}%
\pgfpathlineto{\pgfqpoint{4.502043in}{3.035911in}}%
\pgfpathlineto{\pgfqpoint{4.494701in}{3.021023in}}%
\pgfpathclose%
\pgfusepath{fill}%
\end{pgfscope}%
\begin{pgfscope}%
\pgfpathrectangle{\pgfqpoint{1.254980in}{0.150000in}}{\pgfqpoint{5.490039in}{5.490039in}}%
\pgfusepath{clip}%
\pgfsetbuttcap%
\pgfsetroundjoin%
\definecolor{currentfill}{rgb}{0.195860,0.395433,0.555276}%
\pgfsetfillcolor{currentfill}%
\pgfsetfillopacity{0.700000}%
\pgfsetlinewidth{0.000000pt}%
\definecolor{currentstroke}{rgb}{0.000000,0.000000,0.000000}%
\pgfsetstrokecolor{currentstroke}%
\pgfsetdash{}{0pt}%
\pgfpathmoveto{\pgfqpoint{4.738350in}{3.151161in}}%
\pgfpathlineto{\pgfqpoint{4.751361in}{3.146851in}}%
\pgfpathlineto{\pgfqpoint{4.764379in}{3.142704in}}%
\pgfpathlineto{\pgfqpoint{4.777406in}{3.138721in}}%
\pgfpathlineto{\pgfqpoint{4.790439in}{3.134900in}}%
\pgfpathlineto{\pgfqpoint{4.797735in}{3.150528in}}%
\pgfpathlineto{\pgfqpoint{4.805032in}{3.166430in}}%
\pgfpathlineto{\pgfqpoint{4.812331in}{3.182614in}}%
\pgfpathlineto{\pgfqpoint{4.819631in}{3.199086in}}%
\pgfpathlineto{\pgfqpoint{4.806609in}{3.203510in}}%
\pgfpathlineto{\pgfqpoint{4.793595in}{3.208096in}}%
\pgfpathlineto{\pgfqpoint{4.780588in}{3.212846in}}%
\pgfpathlineto{\pgfqpoint{4.767589in}{3.217759in}}%
\pgfpathlineto{\pgfqpoint{4.760277in}{3.200673in}}%
\pgfpathlineto{\pgfqpoint{4.752967in}{3.183883in}}%
\pgfpathlineto{\pgfqpoint{4.745658in}{3.167381in}}%
\pgfpathlineto{\pgfqpoint{4.738350in}{3.151161in}}%
\pgfpathclose%
\pgfusepath{fill}%
\end{pgfscope}%
\begin{pgfscope}%
\pgfpathrectangle{\pgfqpoint{1.254980in}{0.150000in}}{\pgfqpoint{5.490039in}{5.490039in}}%
\pgfusepath{clip}%
\pgfsetbuttcap%
\pgfsetroundjoin%
\definecolor{currentfill}{rgb}{0.229739,0.322361,0.545706}%
\pgfsetfillcolor{currentfill}%
\pgfsetfillopacity{0.700000}%
\pgfsetlinewidth{0.000000pt}%
\definecolor{currentstroke}{rgb}{0.000000,0.000000,0.000000}%
\pgfsetstrokecolor{currentstroke}%
\pgfsetdash{}{0pt}%
\pgfpathmoveto{\pgfqpoint{4.413513in}{2.982051in}}%
\pgfpathlineto{\pgfqpoint{4.426455in}{2.977179in}}%
\pgfpathlineto{\pgfqpoint{4.439404in}{2.972479in}}%
\pgfpathlineto{\pgfqpoint{4.452359in}{2.967952in}}%
\pgfpathlineto{\pgfqpoint{4.465320in}{2.963598in}}%
\pgfpathlineto{\pgfqpoint{4.472668in}{2.977647in}}%
\pgfpathlineto{\pgfqpoint{4.480014in}{2.991897in}}%
\pgfpathlineto{\pgfqpoint{4.487358in}{3.006354in}}%
\pgfpathlineto{\pgfqpoint{4.494701in}{3.021023in}}%
\pgfpathlineto{\pgfqpoint{4.481749in}{3.025870in}}%
\pgfpathlineto{\pgfqpoint{4.468804in}{3.030890in}}%
\pgfpathlineto{\pgfqpoint{4.455864in}{3.036081in}}%
\pgfpathlineto{\pgfqpoint{4.442931in}{3.041447in}}%
\pgfpathlineto{\pgfqpoint{4.435579in}{3.026275in}}%
\pgfpathlineto{\pgfqpoint{4.428225in}{3.011322in}}%
\pgfpathlineto{\pgfqpoint{4.420870in}{2.996583in}}%
\pgfpathlineto{\pgfqpoint{4.413513in}{2.982051in}}%
\pgfpathclose%
\pgfusepath{fill}%
\end{pgfscope}%
\begin{pgfscope}%
\pgfpathrectangle{\pgfqpoint{1.254980in}{0.150000in}}{\pgfqpoint{5.490039in}{5.490039in}}%
\pgfusepath{clip}%
\pgfsetbuttcap%
\pgfsetroundjoin%
\definecolor{currentfill}{rgb}{0.227802,0.326594,0.546532}%
\pgfsetfillcolor{currentfill}%
\pgfsetfillopacity{0.700000}%
\pgfsetlinewidth{0.000000pt}%
\definecolor{currentstroke}{rgb}{0.000000,0.000000,0.000000}%
\pgfsetstrokecolor{currentstroke}%
\pgfsetdash{}{0pt}%
\pgfpathmoveto{\pgfqpoint{3.454533in}{3.007936in}}%
\pgfpathlineto{\pgfqpoint{3.467391in}{2.994764in}}%
\pgfpathlineto{\pgfqpoint{3.480247in}{2.981830in}}%
\pgfpathlineto{\pgfqpoint{3.493102in}{2.969133in}}%
\pgfpathlineto{\pgfqpoint{3.505955in}{2.956670in}}%
\pgfpathlineto{\pgfqpoint{3.513521in}{2.970545in}}%
\pgfpathlineto{\pgfqpoint{3.521081in}{2.984574in}}%
\pgfpathlineto{\pgfqpoint{3.528637in}{2.998760in}}%
\pgfpathlineto{\pgfqpoint{3.536187in}{3.013107in}}%
\pgfpathlineto{\pgfqpoint{3.523342in}{3.025846in}}%
\pgfpathlineto{\pgfqpoint{3.510494in}{3.038820in}}%
\pgfpathlineto{\pgfqpoint{3.497645in}{3.052030in}}%
\pgfpathlineto{\pgfqpoint{3.484794in}{3.065478in}}%
\pgfpathlineto{\pgfqpoint{3.477237in}{3.050844in}}%
\pgfpathlineto{\pgfqpoint{3.469675in}{3.036378in}}%
\pgfpathlineto{\pgfqpoint{3.462107in}{3.022076in}}%
\pgfpathlineto{\pgfqpoint{3.454533in}{3.007936in}}%
\pgfpathclose%
\pgfusepath{fill}%
\end{pgfscope}%
\begin{pgfscope}%
\pgfpathrectangle{\pgfqpoint{1.254980in}{0.150000in}}{\pgfqpoint{5.490039in}{5.490039in}}%
\pgfusepath{clip}%
\pgfsetbuttcap%
\pgfsetroundjoin%
\definecolor{currentfill}{rgb}{0.182256,0.426184,0.557120}%
\pgfsetfillcolor{currentfill}%
\pgfsetfillopacity{0.700000}%
\pgfsetlinewidth{0.000000pt}%
\definecolor{currentstroke}{rgb}{0.000000,0.000000,0.000000}%
\pgfsetstrokecolor{currentstroke}%
\pgfsetdash{}{0pt}%
\pgfpathmoveto{\pgfqpoint{3.248376in}{3.252689in}}%
\pgfpathlineto{\pgfqpoint{3.261293in}{3.235431in}}%
\pgfpathlineto{\pgfqpoint{3.274205in}{3.218445in}}%
\pgfpathlineto{\pgfqpoint{3.287112in}{3.201728in}}%
\pgfpathlineto{\pgfqpoint{3.300013in}{3.185279in}}%
\pgfpathlineto{\pgfqpoint{3.307609in}{3.200222in}}%
\pgfpathlineto{\pgfqpoint{3.315199in}{3.215347in}}%
\pgfpathlineto{\pgfqpoint{3.322782in}{3.230657in}}%
\pgfpathlineto{\pgfqpoint{3.330360in}{3.246157in}}%
\pgfpathlineto{\pgfqpoint{3.317465in}{3.262886in}}%
\pgfpathlineto{\pgfqpoint{3.304566in}{3.279882in}}%
\pgfpathlineto{\pgfqpoint{3.291661in}{3.297149in}}%
\pgfpathlineto{\pgfqpoint{3.278751in}{3.314688in}}%
\pgfpathlineto{\pgfqpoint{3.271166in}{3.298897in}}%
\pgfpathlineto{\pgfqpoint{3.263576in}{3.283303in}}%
\pgfpathlineto{\pgfqpoint{3.255979in}{3.267901in}}%
\pgfpathlineto{\pgfqpoint{3.248376in}{3.252689in}}%
\pgfpathclose%
\pgfusepath{fill}%
\end{pgfscope}%
\begin{pgfscope}%
\pgfpathrectangle{\pgfqpoint{1.254980in}{0.150000in}}{\pgfqpoint{5.490039in}{5.490039in}}%
\pgfusepath{clip}%
\pgfsetbuttcap%
\pgfsetroundjoin%
\definecolor{currentfill}{rgb}{0.252194,0.269783,0.531579}%
\pgfsetfillcolor{currentfill}%
\pgfsetfillopacity{0.700000}%
\pgfsetlinewidth{0.000000pt}%
\definecolor{currentstroke}{rgb}{0.000000,0.000000,0.000000}%
\pgfsetstrokecolor{currentstroke}%
\pgfsetdash{}{0pt}%
\pgfpathmoveto{\pgfqpoint{4.037112in}{2.874526in}}%
\pgfpathlineto{\pgfqpoint{4.049984in}{2.867722in}}%
\pgfpathlineto{\pgfqpoint{4.062859in}{2.861107in}}%
\pgfpathlineto{\pgfqpoint{4.075739in}{2.854680in}}%
\pgfpathlineto{\pgfqpoint{4.088623in}{2.848442in}}%
\pgfpathlineto{\pgfqpoint{4.096057in}{2.861854in}}%
\pgfpathlineto{\pgfqpoint{4.103486in}{2.875421in}}%
\pgfpathlineto{\pgfqpoint{4.110913in}{2.889146in}}%
\pgfpathlineto{\pgfqpoint{4.118336in}{2.903034in}}%
\pgfpathlineto{\pgfqpoint{4.105459in}{2.909656in}}%
\pgfpathlineto{\pgfqpoint{4.092586in}{2.916465in}}%
\pgfpathlineto{\pgfqpoint{4.079718in}{2.923463in}}%
\pgfpathlineto{\pgfqpoint{4.066853in}{2.930651in}}%
\pgfpathlineto{\pgfqpoint{4.059423in}{2.916369in}}%
\pgfpathlineto{\pgfqpoint{4.051990in}{2.902258in}}%
\pgfpathlineto{\pgfqpoint{4.044553in}{2.888312in}}%
\pgfpathlineto{\pgfqpoint{4.037112in}{2.874526in}}%
\pgfpathclose%
\pgfusepath{fill}%
\end{pgfscope}%
\begin{pgfscope}%
\pgfpathrectangle{\pgfqpoint{1.254980in}{0.150000in}}{\pgfqpoint{5.490039in}{5.490039in}}%
\pgfusepath{clip}%
\pgfsetbuttcap%
\pgfsetroundjoin%
\definecolor{currentfill}{rgb}{0.185556,0.418570,0.556753}%
\pgfsetfillcolor{currentfill}%
\pgfsetfillopacity{0.700000}%
\pgfsetlinewidth{0.000000pt}%
\definecolor{currentstroke}{rgb}{0.000000,0.000000,0.000000}%
\pgfsetstrokecolor{currentstroke}%
\pgfsetdash{}{0pt}%
\pgfpathmoveto{\pgfqpoint{4.819631in}{3.199086in}}%
\pgfpathlineto{\pgfqpoint{4.832660in}{3.194825in}}%
\pgfpathlineto{\pgfqpoint{4.845697in}{3.190725in}}%
\pgfpathlineto{\pgfqpoint{4.858742in}{3.186787in}}%
\pgfpathlineto{\pgfqpoint{4.871795in}{3.183009in}}%
\pgfpathlineto{\pgfqpoint{4.879084in}{3.199157in}}%
\pgfpathlineto{\pgfqpoint{4.886375in}{3.215602in}}%
\pgfpathlineto{\pgfqpoint{4.893668in}{3.232350in}}%
\pgfpathlineto{\pgfqpoint{4.900964in}{3.249410in}}%
\pgfpathlineto{\pgfqpoint{4.887924in}{3.253818in}}%
\pgfpathlineto{\pgfqpoint{4.874892in}{3.258387in}}%
\pgfpathlineto{\pgfqpoint{4.861867in}{3.263118in}}%
\pgfpathlineto{\pgfqpoint{4.848850in}{3.268010in}}%
\pgfpathlineto{\pgfqpoint{4.841542in}{3.250309in}}%
\pgfpathlineto{\pgfqpoint{4.834236in}{3.232927in}}%
\pgfpathlineto{\pgfqpoint{4.826932in}{3.215855in}}%
\pgfpathlineto{\pgfqpoint{4.819631in}{3.199086in}}%
\pgfpathclose%
\pgfusepath{fill}%
\end{pgfscope}%
\begin{pgfscope}%
\pgfpathrectangle{\pgfqpoint{1.254980in}{0.150000in}}{\pgfqpoint{5.490039in}{5.490039in}}%
\pgfusepath{clip}%
\pgfsetbuttcap%
\pgfsetroundjoin%
\definecolor{currentfill}{rgb}{0.237441,0.305202,0.541921}%
\pgfsetfillcolor{currentfill}%
\pgfsetfillopacity{0.700000}%
\pgfsetlinewidth{0.000000pt}%
\definecolor{currentstroke}{rgb}{0.000000,0.000000,0.000000}%
\pgfsetstrokecolor{currentstroke}%
\pgfsetdash{}{0pt}%
\pgfpathmoveto{\pgfqpoint{4.332322in}{2.945270in}}%
\pgfpathlineto{\pgfqpoint{4.345249in}{2.940162in}}%
\pgfpathlineto{\pgfqpoint{4.358181in}{2.935231in}}%
\pgfpathlineto{\pgfqpoint{4.371120in}{2.930474in}}%
\pgfpathlineto{\pgfqpoint{4.384065in}{2.925893in}}%
\pgfpathlineto{\pgfqpoint{4.391430in}{2.939648in}}%
\pgfpathlineto{\pgfqpoint{4.398793in}{2.953590in}}%
\pgfpathlineto{\pgfqpoint{4.406154in}{2.967722in}}%
\pgfpathlineto{\pgfqpoint{4.413513in}{2.982051in}}%
\pgfpathlineto{\pgfqpoint{4.400577in}{2.987098in}}%
\pgfpathlineto{\pgfqpoint{4.387647in}{2.992319in}}%
\pgfpathlineto{\pgfqpoint{4.374723in}{2.997716in}}%
\pgfpathlineto{\pgfqpoint{4.361805in}{3.003289in}}%
\pgfpathlineto{\pgfqpoint{4.354437in}{2.988485in}}%
\pgfpathlineto{\pgfqpoint{4.347067in}{2.973884in}}%
\pgfpathlineto{\pgfqpoint{4.339696in}{2.959481in}}%
\pgfpathlineto{\pgfqpoint{4.332322in}{2.945270in}}%
\pgfpathclose%
\pgfusepath{fill}%
\end{pgfscope}%
\begin{pgfscope}%
\pgfpathrectangle{\pgfqpoint{1.254980in}{0.150000in}}{\pgfqpoint{5.490039in}{5.490039in}}%
\pgfusepath{clip}%
\pgfsetbuttcap%
\pgfsetroundjoin%
\definecolor{currentfill}{rgb}{0.253935,0.265254,0.529983}%
\pgfsetfillcolor{currentfill}%
\pgfsetfillopacity{0.700000}%
\pgfsetlinewidth{0.000000pt}%
\definecolor{currentstroke}{rgb}{0.000000,0.000000,0.000000}%
\pgfsetstrokecolor{currentstroke}%
\pgfsetdash{}{0pt}%
\pgfpathmoveto{\pgfqpoint{3.823082in}{2.858120in}}%
\pgfpathlineto{\pgfqpoint{3.835931in}{2.849645in}}%
\pgfpathlineto{\pgfqpoint{3.848782in}{2.841371in}}%
\pgfpathlineto{\pgfqpoint{3.861635in}{2.833299in}}%
\pgfpathlineto{\pgfqpoint{3.874491in}{2.825427in}}%
\pgfpathlineto{\pgfqpoint{3.881977in}{2.838715in}}%
\pgfpathlineto{\pgfqpoint{3.889459in}{2.852144in}}%
\pgfpathlineto{\pgfqpoint{3.896936in}{2.865718in}}%
\pgfpathlineto{\pgfqpoint{3.904410in}{2.879441in}}%
\pgfpathlineto{\pgfqpoint{3.891561in}{2.887642in}}%
\pgfpathlineto{\pgfqpoint{3.878715in}{2.896043in}}%
\pgfpathlineto{\pgfqpoint{3.865871in}{2.904645in}}%
\pgfpathlineto{\pgfqpoint{3.853029in}{2.913450in}}%
\pgfpathlineto{\pgfqpoint{3.845548in}{2.899387in}}%
\pgfpathlineto{\pgfqpoint{3.838064in}{2.885481in}}%
\pgfpathlineto{\pgfqpoint{3.830575in}{2.871727in}}%
\pgfpathlineto{\pgfqpoint{3.823082in}{2.858120in}}%
\pgfpathclose%
\pgfusepath{fill}%
\end{pgfscope}%
\begin{pgfscope}%
\pgfpathrectangle{\pgfqpoint{1.254980in}{0.150000in}}{\pgfqpoint{5.490039in}{5.490039in}}%
\pgfusepath{clip}%
\pgfsetbuttcap%
\pgfsetroundjoin%
\definecolor{currentfill}{rgb}{0.237441,0.305202,0.541921}%
\pgfsetfillcolor{currentfill}%
\pgfsetfillopacity{0.700000}%
\pgfsetlinewidth{0.000000pt}%
\definecolor{currentstroke}{rgb}{0.000000,0.000000,0.000000}%
\pgfsetstrokecolor{currentstroke}%
\pgfsetdash{}{0pt}%
\pgfpathmoveto{\pgfqpoint{3.505955in}{2.956670in}}%
\pgfpathlineto{\pgfqpoint{3.518806in}{2.944439in}}%
\pgfpathlineto{\pgfqpoint{3.531657in}{2.932440in}}%
\pgfpathlineto{\pgfqpoint{3.544506in}{2.920670in}}%
\pgfpathlineto{\pgfqpoint{3.557355in}{2.909128in}}%
\pgfpathlineto{\pgfqpoint{3.564913in}{2.922738in}}%
\pgfpathlineto{\pgfqpoint{3.572467in}{2.936495in}}%
\pgfpathlineto{\pgfqpoint{3.580015in}{2.950402in}}%
\pgfpathlineto{\pgfqpoint{3.587559in}{2.964463in}}%
\pgfpathlineto{\pgfqpoint{3.574717in}{2.976281in}}%
\pgfpathlineto{\pgfqpoint{3.561875in}{2.988326in}}%
\pgfpathlineto{\pgfqpoint{3.549032in}{3.000601in}}%
\pgfpathlineto{\pgfqpoint{3.536187in}{3.013107in}}%
\pgfpathlineto{\pgfqpoint{3.528637in}{2.998760in}}%
\pgfpathlineto{\pgfqpoint{3.521081in}{2.984574in}}%
\pgfpathlineto{\pgfqpoint{3.513521in}{2.970545in}}%
\pgfpathlineto{\pgfqpoint{3.505955in}{2.956670in}}%
\pgfpathclose%
\pgfusepath{fill}%
\end{pgfscope}%
\begin{pgfscope}%
\pgfpathrectangle{\pgfqpoint{1.254980in}{0.150000in}}{\pgfqpoint{5.490039in}{5.490039in}}%
\pgfusepath{clip}%
\pgfsetbuttcap%
\pgfsetroundjoin%
\definecolor{currentfill}{rgb}{0.136408,0.541173,0.554483}%
\pgfsetfillcolor{currentfill}%
\pgfsetfillopacity{0.700000}%
\pgfsetlinewidth{0.000000pt}%
\definecolor{currentstroke}{rgb}{0.000000,0.000000,0.000000}%
\pgfsetstrokecolor{currentstroke}%
\pgfsetdash{}{0pt}%
\pgfpathmoveto{\pgfqpoint{3.123322in}{3.547372in}}%
\pgfpathlineto{\pgfqpoint{3.136316in}{3.526354in}}%
\pgfpathlineto{\pgfqpoint{3.149302in}{3.505641in}}%
\pgfpathlineto{\pgfqpoint{3.162279in}{3.485232in}}%
\pgfpathlineto{\pgfqpoint{3.175248in}{3.465122in}}%
\pgfpathlineto{\pgfqpoint{3.182840in}{3.481713in}}%
\pgfpathlineto{\pgfqpoint{3.190426in}{3.498523in}}%
\pgfpathlineto{\pgfqpoint{3.198005in}{3.515555in}}%
\pgfpathlineto{\pgfqpoint{3.205578in}{3.532812in}}%
\pgfpathlineto{\pgfqpoint{3.192615in}{3.553236in}}%
\pgfpathlineto{\pgfqpoint{3.179644in}{3.573960in}}%
\pgfpathlineto{\pgfqpoint{3.166665in}{3.594987in}}%
\pgfpathlineto{\pgfqpoint{3.153678in}{3.616320in}}%
\pgfpathlineto{\pgfqpoint{3.146099in}{3.598737in}}%
\pgfpathlineto{\pgfqpoint{3.138513in}{3.581387in}}%
\pgfpathlineto{\pgfqpoint{3.130921in}{3.564266in}}%
\pgfpathlineto{\pgfqpoint{3.123322in}{3.547372in}}%
\pgfpathclose%
\pgfusepath{fill}%
\end{pgfscope}%
\begin{pgfscope}%
\pgfpathrectangle{\pgfqpoint{1.254980in}{0.150000in}}{\pgfqpoint{5.490039in}{5.490039in}}%
\pgfusepath{clip}%
\pgfsetbuttcap%
\pgfsetroundjoin%
\definecolor{currentfill}{rgb}{0.252194,0.269783,0.531579}%
\pgfsetfillcolor{currentfill}%
\pgfsetfillopacity{0.700000}%
\pgfsetlinewidth{0.000000pt}%
\definecolor{currentstroke}{rgb}{0.000000,0.000000,0.000000}%
\pgfsetstrokecolor{currentstroke}%
\pgfsetdash{}{0pt}%
\pgfpathmoveto{\pgfqpoint{3.690286in}{2.877930in}}%
\pgfpathlineto{\pgfqpoint{3.703129in}{2.868091in}}%
\pgfpathlineto{\pgfqpoint{3.715973in}{2.858466in}}%
\pgfpathlineto{\pgfqpoint{3.728818in}{2.849051in}}%
\pgfpathlineto{\pgfqpoint{3.741664in}{2.839846in}}%
\pgfpathlineto{\pgfqpoint{3.749182in}{2.853189in}}%
\pgfpathlineto{\pgfqpoint{3.756695in}{2.866672in}}%
\pgfpathlineto{\pgfqpoint{3.764204in}{2.880298in}}%
\pgfpathlineto{\pgfqpoint{3.771708in}{2.894072in}}%
\pgfpathlineto{\pgfqpoint{3.758868in}{2.903578in}}%
\pgfpathlineto{\pgfqpoint{3.746030in}{2.913295in}}%
\pgfpathlineto{\pgfqpoint{3.733193in}{2.923222in}}%
\pgfpathlineto{\pgfqpoint{3.720357in}{2.933363in}}%
\pgfpathlineto{\pgfqpoint{3.712846in}{2.919277in}}%
\pgfpathlineto{\pgfqpoint{3.705331in}{2.905345in}}%
\pgfpathlineto{\pgfqpoint{3.697811in}{2.891564in}}%
\pgfpathlineto{\pgfqpoint{3.690286in}{2.877930in}}%
\pgfpathclose%
\pgfusepath{fill}%
\end{pgfscope}%
\begin{pgfscope}%
\pgfpathrectangle{\pgfqpoint{1.254980in}{0.150000in}}{\pgfqpoint{5.490039in}{5.490039in}}%
\pgfusepath{clip}%
\pgfsetbuttcap%
\pgfsetroundjoin%
\definecolor{currentfill}{rgb}{0.169646,0.456262,0.558030}%
\pgfsetfillcolor{currentfill}%
\pgfsetfillopacity{0.700000}%
\pgfsetlinewidth{0.000000pt}%
\definecolor{currentstroke}{rgb}{0.000000,0.000000,0.000000}%
\pgfsetstrokecolor{currentstroke}%
\pgfsetdash{}{0pt}%
\pgfpathmoveto{\pgfqpoint{3.196648in}{3.324496in}}%
\pgfpathlineto{\pgfqpoint{3.209589in}{3.306123in}}%
\pgfpathlineto{\pgfqpoint{3.222524in}{3.288033in}}%
\pgfpathlineto{\pgfqpoint{3.235453in}{3.270222in}}%
\pgfpathlineto{\pgfqpoint{3.248376in}{3.252689in}}%
\pgfpathlineto{\pgfqpoint{3.255979in}{3.267901in}}%
\pgfpathlineto{\pgfqpoint{3.263576in}{3.283303in}}%
\pgfpathlineto{\pgfqpoint{3.271166in}{3.298897in}}%
\pgfpathlineto{\pgfqpoint{3.278751in}{3.314688in}}%
\pgfpathlineto{\pgfqpoint{3.265835in}{3.332502in}}%
\pgfpathlineto{\pgfqpoint{3.252914in}{3.350594in}}%
\pgfpathlineto{\pgfqpoint{3.239986in}{3.368966in}}%
\pgfpathlineto{\pgfqpoint{3.227052in}{3.387620in}}%
\pgfpathlineto{\pgfqpoint{3.219461in}{3.371537in}}%
\pgfpathlineto{\pgfqpoint{3.211863in}{3.355658in}}%
\pgfpathlineto{\pgfqpoint{3.204259in}{3.339979in}}%
\pgfpathlineto{\pgfqpoint{3.196648in}{3.324496in}}%
\pgfpathclose%
\pgfusepath{fill}%
\end{pgfscope}%
\begin{pgfscope}%
\pgfpathrectangle{\pgfqpoint{1.254980in}{0.150000in}}{\pgfqpoint{5.490039in}{5.490039in}}%
\pgfusepath{clip}%
\pgfsetbuttcap%
\pgfsetroundjoin%
\definecolor{currentfill}{rgb}{0.243113,0.292092,0.538516}%
\pgfsetfillcolor{currentfill}%
\pgfsetfillopacity{0.700000}%
\pgfsetlinewidth{0.000000pt}%
\definecolor{currentstroke}{rgb}{0.000000,0.000000,0.000000}%
\pgfsetstrokecolor{currentstroke}%
\pgfsetdash{}{0pt}%
\pgfpathmoveto{\pgfqpoint{4.251117in}{2.910706in}}%
\pgfpathlineto{\pgfqpoint{4.264029in}{2.905323in}}%
\pgfpathlineto{\pgfqpoint{4.276948in}{2.900120in}}%
\pgfpathlineto{\pgfqpoint{4.289871in}{2.895096in}}%
\pgfpathlineto{\pgfqpoint{4.302801in}{2.890249in}}%
\pgfpathlineto{\pgfqpoint{4.310185in}{2.903741in}}%
\pgfpathlineto{\pgfqpoint{4.317567in}{2.917406in}}%
\pgfpathlineto{\pgfqpoint{4.324945in}{2.931247in}}%
\pgfpathlineto{\pgfqpoint{4.332322in}{2.945270in}}%
\pgfpathlineto{\pgfqpoint{4.319401in}{2.950555in}}%
\pgfpathlineto{\pgfqpoint{4.306485in}{2.956018in}}%
\pgfpathlineto{\pgfqpoint{4.293575in}{2.961658in}}%
\pgfpathlineto{\pgfqpoint{4.280671in}{2.967478in}}%
\pgfpathlineto{\pgfqpoint{4.273286in}{2.953007in}}%
\pgfpathlineto{\pgfqpoint{4.265899in}{2.938725in}}%
\pgfpathlineto{\pgfqpoint{4.258509in}{2.924626in}}%
\pgfpathlineto{\pgfqpoint{4.251117in}{2.910706in}}%
\pgfpathclose%
\pgfusepath{fill}%
\end{pgfscope}%
\begin{pgfscope}%
\pgfpathrectangle{\pgfqpoint{1.254980in}{0.150000in}}{\pgfqpoint{5.490039in}{5.490039in}}%
\pgfusepath{clip}%
\pgfsetbuttcap%
\pgfsetroundjoin%
\definecolor{currentfill}{rgb}{0.177423,0.437527,0.557565}%
\pgfsetfillcolor{currentfill}%
\pgfsetfillopacity{0.700000}%
\pgfsetlinewidth{0.000000pt}%
\definecolor{currentstroke}{rgb}{0.000000,0.000000,0.000000}%
\pgfsetstrokecolor{currentstroke}%
\pgfsetdash{}{0pt}%
\pgfpathmoveto{\pgfqpoint{4.900964in}{3.249410in}}%
\pgfpathlineto{\pgfqpoint{4.914012in}{3.245162in}}%
\pgfpathlineto{\pgfqpoint{4.927067in}{3.241073in}}%
\pgfpathlineto{\pgfqpoint{4.940131in}{3.237144in}}%
\pgfpathlineto{\pgfqpoint{4.953204in}{3.233374in}}%
\pgfpathlineto{\pgfqpoint{4.960488in}{3.250104in}}%
\pgfpathlineto{\pgfqpoint{4.967776in}{3.267153in}}%
\pgfpathlineto{\pgfqpoint{4.975067in}{3.284529in}}%
\pgfpathlineto{\pgfqpoint{4.962006in}{3.288791in}}%
\pgfpathlineto{\pgfqpoint{4.948952in}{3.293211in}}%
\pgfpathlineto{\pgfqpoint{4.935906in}{3.297791in}}%
\pgfpathlineto{\pgfqpoint{4.922868in}{3.302530in}}%
\pgfpathlineto{\pgfqpoint{4.915564in}{3.284492in}}%
\pgfpathlineto{\pgfqpoint{4.908262in}{3.266788in}}%
\pgfpathlineto{\pgfqpoint{4.900964in}{3.249410in}}%
\pgfpathclose%
\pgfusepath{fill}%
\end{pgfscope}%
\begin{pgfscope}%
\pgfpathrectangle{\pgfqpoint{1.254980in}{0.150000in}}{\pgfqpoint{5.490039in}{5.490039in}}%
\pgfusepath{clip}%
\pgfsetbuttcap%
\pgfsetroundjoin%
\definecolor{currentfill}{rgb}{0.255645,0.260703,0.528312}%
\pgfsetfillcolor{currentfill}%
\pgfsetfillopacity{0.700000}%
\pgfsetlinewidth{0.000000pt}%
\definecolor{currentstroke}{rgb}{0.000000,0.000000,0.000000}%
\pgfsetstrokecolor{currentstroke}%
\pgfsetdash{}{0pt}%
\pgfpathmoveto{\pgfqpoint{3.955835in}{2.848617in}}%
\pgfpathlineto{\pgfqpoint{3.968699in}{2.841400in}}%
\pgfpathlineto{\pgfqpoint{3.981567in}{2.834377in}}%
\pgfpathlineto{\pgfqpoint{3.994438in}{2.827546in}}%
\pgfpathlineto{\pgfqpoint{4.007313in}{2.820907in}}%
\pgfpathlineto{\pgfqpoint{4.014768in}{2.834092in}}%
\pgfpathlineto{\pgfqpoint{4.022220in}{2.847421in}}%
\pgfpathlineto{\pgfqpoint{4.029668in}{2.860897in}}%
\pgfpathlineto{\pgfqpoint{4.037112in}{2.874526in}}%
\pgfpathlineto{\pgfqpoint{4.024244in}{2.881521in}}%
\pgfpathlineto{\pgfqpoint{4.011380in}{2.888708in}}%
\pgfpathlineto{\pgfqpoint{3.998520in}{2.896087in}}%
\pgfpathlineto{\pgfqpoint{3.985662in}{2.903660in}}%
\pgfpathlineto{\pgfqpoint{3.978211in}{2.889665in}}%
\pgfpathlineto{\pgfqpoint{3.970756in}{2.875829in}}%
\pgfpathlineto{\pgfqpoint{3.963297in}{2.862148in}}%
\pgfpathlineto{\pgfqpoint{3.955835in}{2.848617in}}%
\pgfpathclose%
\pgfusepath{fill}%
\end{pgfscope}%
\begin{pgfscope}%
\pgfpathrectangle{\pgfqpoint{1.254980in}{0.150000in}}{\pgfqpoint{5.490039in}{5.490039in}}%
\pgfusepath{clip}%
\pgfsetbuttcap%
\pgfsetroundjoin%
\definecolor{currentfill}{rgb}{0.244972,0.287675,0.537260}%
\pgfsetfillcolor{currentfill}%
\pgfsetfillopacity{0.700000}%
\pgfsetlinewidth{0.000000pt}%
\definecolor{currentstroke}{rgb}{0.000000,0.000000,0.000000}%
\pgfsetstrokecolor{currentstroke}%
\pgfsetdash{}{0pt}%
\pgfpathmoveto{\pgfqpoint{3.557355in}{2.909128in}}%
\pgfpathlineto{\pgfqpoint{3.570203in}{2.897811in}}%
\pgfpathlineto{\pgfqpoint{3.583051in}{2.886719in}}%
\pgfpathlineto{\pgfqpoint{3.595898in}{2.875850in}}%
\pgfpathlineto{\pgfqpoint{3.608746in}{2.865202in}}%
\pgfpathlineto{\pgfqpoint{3.616297in}{2.878548in}}%
\pgfpathlineto{\pgfqpoint{3.623843in}{2.892033in}}%
\pgfpathlineto{\pgfqpoint{3.631385in}{2.905662in}}%
\pgfpathlineto{\pgfqpoint{3.638921in}{2.919437in}}%
\pgfpathlineto{\pgfqpoint{3.626081in}{2.930360in}}%
\pgfpathlineto{\pgfqpoint{3.613240in}{2.941504in}}%
\pgfpathlineto{\pgfqpoint{3.600400in}{2.952871in}}%
\pgfpathlineto{\pgfqpoint{3.587559in}{2.964463in}}%
\pgfpathlineto{\pgfqpoint{3.580015in}{2.950402in}}%
\pgfpathlineto{\pgfqpoint{3.572467in}{2.936495in}}%
\pgfpathlineto{\pgfqpoint{3.564913in}{2.922738in}}%
\pgfpathlineto{\pgfqpoint{3.557355in}{2.909128in}}%
\pgfpathclose%
\pgfusepath{fill}%
\end{pgfscope}%
\begin{pgfscope}%
\pgfpathrectangle{\pgfqpoint{1.254980in}{0.150000in}}{\pgfqpoint{5.490039in}{5.490039in}}%
\pgfusepath{clip}%
\pgfsetbuttcap%
\pgfsetroundjoin%
\definecolor{currentfill}{rgb}{0.248629,0.278775,0.534556}%
\pgfsetfillcolor{currentfill}%
\pgfsetfillopacity{0.700000}%
\pgfsetlinewidth{0.000000pt}%
\definecolor{currentstroke}{rgb}{0.000000,0.000000,0.000000}%
\pgfsetstrokecolor{currentstroke}%
\pgfsetdash{}{0pt}%
\pgfpathmoveto{\pgfqpoint{4.169888in}{2.878406in}}%
\pgfpathlineto{\pgfqpoint{4.182787in}{2.872709in}}%
\pgfpathlineto{\pgfqpoint{4.195692in}{2.867195in}}%
\pgfpathlineto{\pgfqpoint{4.208602in}{2.861862in}}%
\pgfpathlineto{\pgfqpoint{4.221518in}{2.856711in}}%
\pgfpathlineto{\pgfqpoint{4.228922in}{2.869966in}}%
\pgfpathlineto{\pgfqpoint{4.236323in}{2.883381in}}%
\pgfpathlineto{\pgfqpoint{4.243721in}{2.896959in}}%
\pgfpathlineto{\pgfqpoint{4.251117in}{2.910706in}}%
\pgfpathlineto{\pgfqpoint{4.238209in}{2.916268in}}%
\pgfpathlineto{\pgfqpoint{4.225307in}{2.922011in}}%
\pgfpathlineto{\pgfqpoint{4.212410in}{2.927936in}}%
\pgfpathlineto{\pgfqpoint{4.199518in}{2.934043in}}%
\pgfpathlineto{\pgfqpoint{4.192115in}{2.919875in}}%
\pgfpathlineto{\pgfqpoint{4.184709in}{2.905883in}}%
\pgfpathlineto{\pgfqpoint{4.177300in}{2.892062in}}%
\pgfpathlineto{\pgfqpoint{4.169888in}{2.878406in}}%
\pgfpathclose%
\pgfusepath{fill}%
\end{pgfscope}%
\begin{pgfscope}%
\pgfpathrectangle{\pgfqpoint{1.254980in}{0.150000in}}{\pgfqpoint{5.490039in}{5.490039in}}%
\pgfusepath{clip}%
\pgfsetbuttcap%
\pgfsetroundjoin%
\definecolor{currentfill}{rgb}{0.157729,0.485932,0.558013}%
\pgfsetfillcolor{currentfill}%
\pgfsetfillopacity{0.700000}%
\pgfsetlinewidth{0.000000pt}%
\definecolor{currentstroke}{rgb}{0.000000,0.000000,0.000000}%
\pgfsetstrokecolor{currentstroke}%
\pgfsetdash{}{0pt}%
\pgfpathmoveto{\pgfqpoint{3.144814in}{3.400865in}}%
\pgfpathlineto{\pgfqpoint{3.157784in}{3.381336in}}%
\pgfpathlineto{\pgfqpoint{3.170745in}{3.362099in}}%
\pgfpathlineto{\pgfqpoint{3.183700in}{3.343154in}}%
\pgfpathlineto{\pgfqpoint{3.196648in}{3.324496in}}%
\pgfpathlineto{\pgfqpoint{3.204259in}{3.339979in}}%
\pgfpathlineto{\pgfqpoint{3.211863in}{3.355658in}}%
\pgfpathlineto{\pgfqpoint{3.219461in}{3.371537in}}%
\pgfpathlineto{\pgfqpoint{3.227052in}{3.387620in}}%
\pgfpathlineto{\pgfqpoint{3.214112in}{3.406561in}}%
\pgfpathlineto{\pgfqpoint{3.201165in}{3.425789in}}%
\pgfpathlineto{\pgfqpoint{3.188210in}{3.445309in}}%
\pgfpathlineto{\pgfqpoint{3.175248in}{3.465122in}}%
\pgfpathlineto{\pgfqpoint{3.167650in}{3.448745in}}%
\pgfpathlineto{\pgfqpoint{3.160045in}{3.432579in}}%
\pgfpathlineto{\pgfqpoint{3.152433in}{3.416620in}}%
\pgfpathlineto{\pgfqpoint{3.144814in}{3.400865in}}%
\pgfpathclose%
\pgfusepath{fill}%
\end{pgfscope}%
\begin{pgfscope}%
\pgfpathrectangle{\pgfqpoint{1.254980in}{0.150000in}}{\pgfqpoint{5.490039in}{5.490039in}}%
\pgfusepath{clip}%
\pgfsetbuttcap%
\pgfsetroundjoin%
\definecolor{currentfill}{rgb}{0.257322,0.256130,0.526563}%
\pgfsetfillcolor{currentfill}%
\pgfsetfillopacity{0.700000}%
\pgfsetlinewidth{0.000000pt}%
\definecolor{currentstroke}{rgb}{0.000000,0.000000,0.000000}%
\pgfsetstrokecolor{currentstroke}%
\pgfsetdash{}{0pt}%
\pgfpathmoveto{\pgfqpoint{3.741664in}{2.839846in}}%
\pgfpathlineto{\pgfqpoint{3.754512in}{2.830850in}}%
\pgfpathlineto{\pgfqpoint{3.767362in}{2.822062in}}%
\pgfpathlineto{\pgfqpoint{3.780213in}{2.813479in}}%
\pgfpathlineto{\pgfqpoint{3.793067in}{2.805101in}}%
\pgfpathlineto{\pgfqpoint{3.800577in}{2.818153in}}%
\pgfpathlineto{\pgfqpoint{3.808083in}{2.831337in}}%
\pgfpathlineto{\pgfqpoint{3.815585in}{2.844659in}}%
\pgfpathlineto{\pgfqpoint{3.823082in}{2.858120in}}%
\pgfpathlineto{\pgfqpoint{3.810236in}{2.866800in}}%
\pgfpathlineto{\pgfqpoint{3.797391in}{2.875684in}}%
\pgfpathlineto{\pgfqpoint{3.784549in}{2.884774in}}%
\pgfpathlineto{\pgfqpoint{3.771708in}{2.894072in}}%
\pgfpathlineto{\pgfqpoint{3.764204in}{2.880298in}}%
\pgfpathlineto{\pgfqpoint{3.756695in}{2.866672in}}%
\pgfpathlineto{\pgfqpoint{3.749182in}{2.853189in}}%
\pgfpathlineto{\pgfqpoint{3.741664in}{2.839846in}}%
\pgfpathclose%
\pgfusepath{fill}%
\end{pgfscope}%
\begin{pgfscope}%
\pgfpathrectangle{\pgfqpoint{1.254980in}{0.150000in}}{\pgfqpoint{5.490039in}{5.490039in}}%
\pgfusepath{clip}%
\pgfsetbuttcap%
\pgfsetroundjoin%
\definecolor{currentfill}{rgb}{0.253935,0.265254,0.529983}%
\pgfsetfillcolor{currentfill}%
\pgfsetfillopacity{0.700000}%
\pgfsetlinewidth{0.000000pt}%
\definecolor{currentstroke}{rgb}{0.000000,0.000000,0.000000}%
\pgfsetstrokecolor{currentstroke}%
\pgfsetdash{}{0pt}%
\pgfpathmoveto{\pgfqpoint{4.088623in}{2.848442in}}%
\pgfpathlineto{\pgfqpoint{4.101512in}{2.842390in}}%
\pgfpathlineto{\pgfqpoint{4.114405in}{2.836524in}}%
\pgfpathlineto{\pgfqpoint{4.127303in}{2.830843in}}%
\pgfpathlineto{\pgfqpoint{4.140206in}{2.825346in}}%
\pgfpathlineto{\pgfqpoint{4.147631in}{2.838386in}}%
\pgfpathlineto{\pgfqpoint{4.155053in}{2.851573in}}%
\pgfpathlineto{\pgfqpoint{4.162472in}{2.864911in}}%
\pgfpathlineto{\pgfqpoint{4.169888in}{2.878406in}}%
\pgfpathlineto{\pgfqpoint{4.156993in}{2.884286in}}%
\pgfpathlineto{\pgfqpoint{4.144102in}{2.890350in}}%
\pgfpathlineto{\pgfqpoint{4.131217in}{2.896599in}}%
\pgfpathlineto{\pgfqpoint{4.118336in}{2.903034in}}%
\pgfpathlineto{\pgfqpoint{4.110913in}{2.889146in}}%
\pgfpathlineto{\pgfqpoint{4.103486in}{2.875421in}}%
\pgfpathlineto{\pgfqpoint{4.096057in}{2.861854in}}%
\pgfpathlineto{\pgfqpoint{4.088623in}{2.848442in}}%
\pgfpathclose%
\pgfusepath{fill}%
\end{pgfscope}%
\begin{pgfscope}%
\pgfpathrectangle{\pgfqpoint{1.254980in}{0.150000in}}{\pgfqpoint{5.490039in}{5.490039in}}%
\pgfusepath{clip}%
\pgfsetbuttcap%
\pgfsetroundjoin%
\definecolor{currentfill}{rgb}{0.258965,0.251537,0.524736}%
\pgfsetfillcolor{currentfill}%
\pgfsetfillopacity{0.700000}%
\pgfsetlinewidth{0.000000pt}%
\definecolor{currentstroke}{rgb}{0.000000,0.000000,0.000000}%
\pgfsetstrokecolor{currentstroke}%
\pgfsetdash{}{0pt}%
\pgfpathmoveto{\pgfqpoint{3.874491in}{2.825427in}}%
\pgfpathlineto{\pgfqpoint{3.887350in}{2.817754in}}%
\pgfpathlineto{\pgfqpoint{3.900211in}{2.810279in}}%
\pgfpathlineto{\pgfqpoint{3.913076in}{2.803000in}}%
\pgfpathlineto{\pgfqpoint{3.925944in}{2.795918in}}%
\pgfpathlineto{\pgfqpoint{3.933423in}{2.808887in}}%
\pgfpathlineto{\pgfqpoint{3.940898in}{2.821991in}}%
\pgfpathlineto{\pgfqpoint{3.948368in}{2.835233in}}%
\pgfpathlineto{\pgfqpoint{3.955835in}{2.848617in}}%
\pgfpathlineto{\pgfqpoint{3.942974in}{2.856029in}}%
\pgfpathlineto{\pgfqpoint{3.930116in}{2.863636in}}%
\pgfpathlineto{\pgfqpoint{3.917262in}{2.871440in}}%
\pgfpathlineto{\pgfqpoint{3.904410in}{2.879441in}}%
\pgfpathlineto{\pgfqpoint{3.896936in}{2.865718in}}%
\pgfpathlineto{\pgfqpoint{3.889459in}{2.852144in}}%
\pgfpathlineto{\pgfqpoint{3.881977in}{2.838715in}}%
\pgfpathlineto{\pgfqpoint{3.874491in}{2.825427in}}%
\pgfpathclose%
\pgfusepath{fill}%
\end{pgfscope}%
\begin{pgfscope}%
\pgfpathrectangle{\pgfqpoint{1.254980in}{0.150000in}}{\pgfqpoint{5.490039in}{5.490039in}}%
\pgfusepath{clip}%
\pgfsetbuttcap%
\pgfsetroundjoin%
\definecolor{currentfill}{rgb}{0.214298,0.355619,0.551184}%
\pgfsetfillcolor{currentfill}%
\pgfsetfillopacity{0.700000}%
\pgfsetlinewidth{0.000000pt}%
\definecolor{currentstroke}{rgb}{0.000000,0.000000,0.000000}%
\pgfsetstrokecolor{currentstroke}%
\pgfsetdash{}{0pt}%
\pgfpathmoveto{\pgfqpoint{3.321166in}{3.065100in}}%
\pgfpathlineto{\pgfqpoint{3.334054in}{3.050204in}}%
\pgfpathlineto{\pgfqpoint{3.346939in}{3.035561in}}%
\pgfpathlineto{\pgfqpoint{3.359821in}{3.021170in}}%
\pgfpathlineto{\pgfqpoint{3.372699in}{3.007029in}}%
\pgfpathlineto{\pgfqpoint{3.380303in}{3.020796in}}%
\pgfpathlineto{\pgfqpoint{3.387900in}{3.034718in}}%
\pgfpathlineto{\pgfqpoint{3.395492in}{3.048799in}}%
\pgfpathlineto{\pgfqpoint{3.403078in}{3.063041in}}%
\pgfpathlineto{\pgfqpoint{3.390208in}{3.077432in}}%
\pgfpathlineto{\pgfqpoint{3.377334in}{3.092073in}}%
\pgfpathlineto{\pgfqpoint{3.364457in}{3.106965in}}%
\pgfpathlineto{\pgfqpoint{3.351576in}{3.122111in}}%
\pgfpathlineto{\pgfqpoint{3.343982in}{3.107609in}}%
\pgfpathlineto{\pgfqpoint{3.336383in}{3.093275in}}%
\pgfpathlineto{\pgfqpoint{3.328777in}{3.079106in}}%
\pgfpathlineto{\pgfqpoint{3.321166in}{3.065100in}}%
\pgfpathclose%
\pgfusepath{fill}%
\end{pgfscope}%
\begin{pgfscope}%
\pgfpathrectangle{\pgfqpoint{1.254980in}{0.150000in}}{\pgfqpoint{5.490039in}{5.490039in}}%
\pgfusepath{clip}%
\pgfsetbuttcap%
\pgfsetroundjoin%
\definecolor{currentfill}{rgb}{0.214298,0.355619,0.551184}%
\pgfsetfillcolor{currentfill}%
\pgfsetfillopacity{0.700000}%
\pgfsetlinewidth{0.000000pt}%
\definecolor{currentstroke}{rgb}{0.000000,0.000000,0.000000}%
\pgfsetstrokecolor{currentstroke}%
\pgfsetdash{}{0pt}%
\pgfpathmoveto{\pgfqpoint{4.627840in}{3.045121in}}%
\pgfpathlineto{\pgfqpoint{4.640844in}{3.041273in}}%
\pgfpathlineto{\pgfqpoint{4.653855in}{3.037592in}}%
\pgfpathlineto{\pgfqpoint{4.666874in}{3.034075in}}%
\pgfpathlineto{\pgfqpoint{4.679901in}{3.030724in}}%
\pgfpathlineto{\pgfqpoint{4.687208in}{3.044934in}}%
\pgfpathlineto{\pgfqpoint{4.694513in}{3.059372in}}%
\pgfpathlineto{\pgfqpoint{4.701819in}{3.074045in}}%
\pgfpathlineto{\pgfqpoint{4.709125in}{3.088960in}}%
\pgfpathlineto{\pgfqpoint{4.696110in}{3.092859in}}%
\pgfpathlineto{\pgfqpoint{4.683102in}{3.096923in}}%
\pgfpathlineto{\pgfqpoint{4.670102in}{3.101152in}}%
\pgfpathlineto{\pgfqpoint{4.657109in}{3.105548in}}%
\pgfpathlineto{\pgfqpoint{4.649792in}{3.090075in}}%
\pgfpathlineto{\pgfqpoint{4.642475in}{3.074851in}}%
\pgfpathlineto{\pgfqpoint{4.635158in}{3.059868in}}%
\pgfpathlineto{\pgfqpoint{4.627840in}{3.045121in}}%
\pgfpathclose%
\pgfusepath{fill}%
\end{pgfscope}%
\begin{pgfscope}%
\pgfpathrectangle{\pgfqpoint{1.254980in}{0.150000in}}{\pgfqpoint{5.490039in}{5.490039in}}%
\pgfusepath{clip}%
\pgfsetbuttcap%
\pgfsetroundjoin%
\definecolor{currentfill}{rgb}{0.221989,0.339161,0.548752}%
\pgfsetfillcolor{currentfill}%
\pgfsetfillopacity{0.700000}%
\pgfsetlinewidth{0.000000pt}%
\definecolor{currentstroke}{rgb}{0.000000,0.000000,0.000000}%
\pgfsetstrokecolor{currentstroke}%
\pgfsetdash{}{0pt}%
\pgfpathmoveto{\pgfqpoint{4.546575in}{3.003341in}}%
\pgfpathlineto{\pgfqpoint{4.559561in}{2.999344in}}%
\pgfpathlineto{\pgfqpoint{4.572553in}{2.995515in}}%
\pgfpathlineto{\pgfqpoint{4.585554in}{2.991855in}}%
\pgfpathlineto{\pgfqpoint{4.598561in}{2.988361in}}%
\pgfpathlineto{\pgfqpoint{4.605882in}{3.002229in}}%
\pgfpathlineto{\pgfqpoint{4.613202in}{3.016307in}}%
\pgfpathlineto{\pgfqpoint{4.620522in}{3.030603in}}%
\pgfpathlineto{\pgfqpoint{4.627840in}{3.045121in}}%
\pgfpathlineto{\pgfqpoint{4.614843in}{3.049135in}}%
\pgfpathlineto{\pgfqpoint{4.601854in}{3.053316in}}%
\pgfpathlineto{\pgfqpoint{4.588872in}{3.057665in}}%
\pgfpathlineto{\pgfqpoint{4.575896in}{3.062182in}}%
\pgfpathlineto{\pgfqpoint{4.568567in}{3.047133in}}%
\pgfpathlineto{\pgfqpoint{4.561238in}{3.032314in}}%
\pgfpathlineto{\pgfqpoint{4.553907in}{3.017719in}}%
\pgfpathlineto{\pgfqpoint{4.546575in}{3.003341in}}%
\pgfpathclose%
\pgfusepath{fill}%
\end{pgfscope}%
\begin{pgfscope}%
\pgfpathrectangle{\pgfqpoint{1.254980in}{0.150000in}}{\pgfqpoint{5.490039in}{5.490039in}}%
\pgfusepath{clip}%
\pgfsetbuttcap%
\pgfsetroundjoin%
\definecolor{currentfill}{rgb}{0.203063,0.379716,0.553925}%
\pgfsetfillcolor{currentfill}%
\pgfsetfillopacity{0.700000}%
\pgfsetlinewidth{0.000000pt}%
\definecolor{currentstroke}{rgb}{0.000000,0.000000,0.000000}%
\pgfsetstrokecolor{currentstroke}%
\pgfsetdash{}{0pt}%
\pgfpathmoveto{\pgfqpoint{3.269571in}{3.127267in}}%
\pgfpathlineto{\pgfqpoint{3.282476in}{3.111333in}}%
\pgfpathlineto{\pgfqpoint{3.295377in}{3.095663in}}%
\pgfpathlineto{\pgfqpoint{3.308273in}{3.080253in}}%
\pgfpathlineto{\pgfqpoint{3.321166in}{3.065100in}}%
\pgfpathlineto{\pgfqpoint{3.328777in}{3.079106in}}%
\pgfpathlineto{\pgfqpoint{3.336383in}{3.093275in}}%
\pgfpathlineto{\pgfqpoint{3.343982in}{3.107609in}}%
\pgfpathlineto{\pgfqpoint{3.351576in}{3.122111in}}%
\pgfpathlineto{\pgfqpoint{3.338692in}{3.137514in}}%
\pgfpathlineto{\pgfqpoint{3.325803in}{3.153174in}}%
\pgfpathlineto{\pgfqpoint{3.312911in}{3.169095in}}%
\pgfpathlineto{\pgfqpoint{3.300013in}{3.185279in}}%
\pgfpathlineto{\pgfqpoint{3.292412in}{3.170516in}}%
\pgfpathlineto{\pgfqpoint{3.284804in}{3.155928in}}%
\pgfpathlineto{\pgfqpoint{3.277191in}{3.141512in}}%
\pgfpathlineto{\pgfqpoint{3.269571in}{3.127267in}}%
\pgfpathclose%
\pgfusepath{fill}%
\end{pgfscope}%
\begin{pgfscope}%
\pgfpathrectangle{\pgfqpoint{1.254980in}{0.150000in}}{\pgfqpoint{5.490039in}{5.490039in}}%
\pgfusepath{clip}%
\pgfsetbuttcap%
\pgfsetroundjoin%
\definecolor{currentfill}{rgb}{0.204903,0.375746,0.553533}%
\pgfsetfillcolor{currentfill}%
\pgfsetfillopacity{0.700000}%
\pgfsetlinewidth{0.000000pt}%
\definecolor{currentstroke}{rgb}{0.000000,0.000000,0.000000}%
\pgfsetstrokecolor{currentstroke}%
\pgfsetdash{}{0pt}%
\pgfpathmoveto{\pgfqpoint{4.709125in}{3.088960in}}%
\pgfpathlineto{\pgfqpoint{4.722147in}{3.085225in}}%
\pgfpathlineto{\pgfqpoint{4.735178in}{3.081653in}}%
\pgfpathlineto{\pgfqpoint{4.748216in}{3.078246in}}%
\pgfpathlineto{\pgfqpoint{4.761263in}{3.075001in}}%
\pgfpathlineto{\pgfqpoint{4.768556in}{3.089598in}}%
\pgfpathlineto{\pgfqpoint{4.775850in}{3.104442in}}%
\pgfpathlineto{\pgfqpoint{4.783144in}{3.119541in}}%
\pgfpathlineto{\pgfqpoint{4.790439in}{3.134900in}}%
\pgfpathlineto{\pgfqpoint{4.777406in}{3.138721in}}%
\pgfpathlineto{\pgfqpoint{4.764379in}{3.142704in}}%
\pgfpathlineto{\pgfqpoint{4.751361in}{3.146851in}}%
\pgfpathlineto{\pgfqpoint{4.738350in}{3.151161in}}%
\pgfpathlineto{\pgfqpoint{4.731043in}{3.135216in}}%
\pgfpathlineto{\pgfqpoint{4.723736in}{3.119538in}}%
\pgfpathlineto{\pgfqpoint{4.716430in}{3.104122in}}%
\pgfpathlineto{\pgfqpoint{4.709125in}{3.088960in}}%
\pgfpathclose%
\pgfusepath{fill}%
\end{pgfscope}%
\begin{pgfscope}%
\pgfpathrectangle{\pgfqpoint{1.254980in}{0.150000in}}{\pgfqpoint{5.490039in}{5.490039in}}%
\pgfusepath{clip}%
\pgfsetbuttcap%
\pgfsetroundjoin%
\definecolor{currentfill}{rgb}{0.252194,0.269783,0.531579}%
\pgfsetfillcolor{currentfill}%
\pgfsetfillopacity{0.700000}%
\pgfsetlinewidth{0.000000pt}%
\definecolor{currentstroke}{rgb}{0.000000,0.000000,0.000000}%
\pgfsetstrokecolor{currentstroke}%
\pgfsetdash{}{0pt}%
\pgfpathmoveto{\pgfqpoint{3.608746in}{2.865202in}}%
\pgfpathlineto{\pgfqpoint{3.621593in}{2.854774in}}%
\pgfpathlineto{\pgfqpoint{3.634442in}{2.844564in}}%
\pgfpathlineto{\pgfqpoint{3.647290in}{2.834571in}}%
\pgfpathlineto{\pgfqpoint{3.660139in}{2.824793in}}%
\pgfpathlineto{\pgfqpoint{3.667683in}{2.837874in}}%
\pgfpathlineto{\pgfqpoint{3.675222in}{2.851088in}}%
\pgfpathlineto{\pgfqpoint{3.682756in}{2.864439in}}%
\pgfpathlineto{\pgfqpoint{3.690286in}{2.877930in}}%
\pgfpathlineto{\pgfqpoint{3.677444in}{2.887982in}}%
\pgfpathlineto{\pgfqpoint{3.664602in}{2.898250in}}%
\pgfpathlineto{\pgfqpoint{3.651762in}{2.908734in}}%
\pgfpathlineto{\pgfqpoint{3.638921in}{2.919437in}}%
\pgfpathlineto{\pgfqpoint{3.631385in}{2.905662in}}%
\pgfpathlineto{\pgfqpoint{3.623843in}{2.892033in}}%
\pgfpathlineto{\pgfqpoint{3.616297in}{2.878548in}}%
\pgfpathlineto{\pgfqpoint{3.608746in}{2.865202in}}%
\pgfpathclose%
\pgfusepath{fill}%
\end{pgfscope}%
\begin{pgfscope}%
\pgfpathrectangle{\pgfqpoint{1.254980in}{0.150000in}}{\pgfqpoint{5.490039in}{5.490039in}}%
\pgfusepath{clip}%
\pgfsetbuttcap%
\pgfsetroundjoin%
\definecolor{currentfill}{rgb}{0.225863,0.330805,0.547314}%
\pgfsetfillcolor{currentfill}%
\pgfsetfillopacity{0.700000}%
\pgfsetlinewidth{0.000000pt}%
\definecolor{currentstroke}{rgb}{0.000000,0.000000,0.000000}%
\pgfsetstrokecolor{currentstroke}%
\pgfsetdash{}{0pt}%
\pgfpathmoveto{\pgfqpoint{3.372699in}{3.007029in}}%
\pgfpathlineto{\pgfqpoint{3.385575in}{2.993135in}}%
\pgfpathlineto{\pgfqpoint{3.398447in}{2.979487in}}%
\pgfpathlineto{\pgfqpoint{3.411318in}{2.966082in}}%
\pgfpathlineto{\pgfqpoint{3.424186in}{2.952919in}}%
\pgfpathlineto{\pgfqpoint{3.431781in}{2.966447in}}%
\pgfpathlineto{\pgfqpoint{3.439370in}{2.980124in}}%
\pgfpathlineto{\pgfqpoint{3.446955in}{2.993953in}}%
\pgfpathlineto{\pgfqpoint{3.454533in}{3.007936in}}%
\pgfpathlineto{\pgfqpoint{3.441673in}{3.021347in}}%
\pgfpathlineto{\pgfqpoint{3.428811in}{3.035001in}}%
\pgfpathlineto{\pgfqpoint{3.415946in}{3.048898in}}%
\pgfpathlineto{\pgfqpoint{3.403078in}{3.063041in}}%
\pgfpathlineto{\pgfqpoint{3.395492in}{3.048799in}}%
\pgfpathlineto{\pgfqpoint{3.387900in}{3.034718in}}%
\pgfpathlineto{\pgfqpoint{3.380303in}{3.020796in}}%
\pgfpathlineto{\pgfqpoint{3.372699in}{3.007029in}}%
\pgfpathclose%
\pgfusepath{fill}%
\end{pgfscope}%
\begin{pgfscope}%
\pgfpathrectangle{\pgfqpoint{1.254980in}{0.150000in}}{\pgfqpoint{5.490039in}{5.490039in}}%
\pgfusepath{clip}%
\pgfsetbuttcap%
\pgfsetroundjoin%
\definecolor{currentfill}{rgb}{0.229739,0.322361,0.545706}%
\pgfsetfillcolor{currentfill}%
\pgfsetfillopacity{0.700000}%
\pgfsetlinewidth{0.000000pt}%
\definecolor{currentstroke}{rgb}{0.000000,0.000000,0.000000}%
\pgfsetstrokecolor{currentstroke}%
\pgfsetdash{}{0pt}%
\pgfpathmoveto{\pgfqpoint{4.465320in}{2.963598in}}%
\pgfpathlineto{\pgfqpoint{4.478288in}{2.959415in}}%
\pgfpathlineto{\pgfqpoint{4.491263in}{2.955402in}}%
\pgfpathlineto{\pgfqpoint{4.504245in}{2.951560in}}%
\pgfpathlineto{\pgfqpoint{4.517233in}{2.947887in}}%
\pgfpathlineto{\pgfqpoint{4.524571in}{2.961453in}}%
\pgfpathlineto{\pgfqpoint{4.531907in}{2.975214in}}%
\pgfpathlineto{\pgfqpoint{4.539242in}{2.989174in}}%
\pgfpathlineto{\pgfqpoint{4.546575in}{3.003341in}}%
\pgfpathlineto{\pgfqpoint{4.533596in}{3.007507in}}%
\pgfpathlineto{\pgfqpoint{4.520625in}{3.011842in}}%
\pgfpathlineto{\pgfqpoint{4.507660in}{3.016347in}}%
\pgfpathlineto{\pgfqpoint{4.494701in}{3.021023in}}%
\pgfpathlineto{\pgfqpoint{4.487358in}{3.006354in}}%
\pgfpathlineto{\pgfqpoint{4.480014in}{2.991897in}}%
\pgfpathlineto{\pgfqpoint{4.472668in}{2.977647in}}%
\pgfpathlineto{\pgfqpoint{4.465320in}{2.963598in}}%
\pgfpathclose%
\pgfusepath{fill}%
\end{pgfscope}%
\begin{pgfscope}%
\pgfpathrectangle{\pgfqpoint{1.254980in}{0.150000in}}{\pgfqpoint{5.490039in}{5.490039in}}%
\pgfusepath{clip}%
\pgfsetbuttcap%
\pgfsetroundjoin%
\definecolor{currentfill}{rgb}{0.195860,0.395433,0.555276}%
\pgfsetfillcolor{currentfill}%
\pgfsetfillopacity{0.700000}%
\pgfsetlinewidth{0.000000pt}%
\definecolor{currentstroke}{rgb}{0.000000,0.000000,0.000000}%
\pgfsetstrokecolor{currentstroke}%
\pgfsetdash{}{0pt}%
\pgfpathmoveto{\pgfqpoint{4.790439in}{3.134900in}}%
\pgfpathlineto{\pgfqpoint{4.803481in}{3.131242in}}%
\pgfpathlineto{\pgfqpoint{4.816531in}{3.127745in}}%
\pgfpathlineto{\pgfqpoint{4.829589in}{3.124410in}}%
\pgfpathlineto{\pgfqpoint{4.842656in}{3.121236in}}%
\pgfpathlineto{\pgfqpoint{4.849939in}{3.136270in}}%
\pgfpathlineto{\pgfqpoint{4.857223in}{3.151573in}}%
\pgfpathlineto{\pgfqpoint{4.864508in}{3.167150in}}%
\pgfpathlineto{\pgfqpoint{4.871795in}{3.183009in}}%
\pgfpathlineto{\pgfqpoint{4.858742in}{3.186787in}}%
\pgfpathlineto{\pgfqpoint{4.845697in}{3.190725in}}%
\pgfpathlineto{\pgfqpoint{4.832660in}{3.194825in}}%
\pgfpathlineto{\pgfqpoint{4.819631in}{3.199086in}}%
\pgfpathlineto{\pgfqpoint{4.812331in}{3.182614in}}%
\pgfpathlineto{\pgfqpoint{4.805032in}{3.166430in}}%
\pgfpathlineto{\pgfqpoint{4.797735in}{3.150528in}}%
\pgfpathlineto{\pgfqpoint{4.790439in}{3.134900in}}%
\pgfpathclose%
\pgfusepath{fill}%
\end{pgfscope}%
\begin{pgfscope}%
\pgfpathrectangle{\pgfqpoint{1.254980in}{0.150000in}}{\pgfqpoint{5.490039in}{5.490039in}}%
\pgfusepath{clip}%
\pgfsetbuttcap%
\pgfsetroundjoin%
\definecolor{currentfill}{rgb}{0.190631,0.407061,0.556089}%
\pgfsetfillcolor{currentfill}%
\pgfsetfillopacity{0.700000}%
\pgfsetlinewidth{0.000000pt}%
\definecolor{currentstroke}{rgb}{0.000000,0.000000,0.000000}%
\pgfsetstrokecolor{currentstroke}%
\pgfsetdash{}{0pt}%
\pgfpathmoveto{\pgfqpoint{3.217901in}{3.193671in}}%
\pgfpathlineto{\pgfqpoint{3.230826in}{3.176664in}}%
\pgfpathlineto{\pgfqpoint{3.243746in}{3.159930in}}%
\pgfpathlineto{\pgfqpoint{3.256661in}{3.143464in}}%
\pgfpathlineto{\pgfqpoint{3.269571in}{3.127267in}}%
\pgfpathlineto{\pgfqpoint{3.277191in}{3.141512in}}%
\pgfpathlineto{\pgfqpoint{3.284804in}{3.155928in}}%
\pgfpathlineto{\pgfqpoint{3.292412in}{3.170516in}}%
\pgfpathlineto{\pgfqpoint{3.300013in}{3.185279in}}%
\pgfpathlineto{\pgfqpoint{3.287112in}{3.201728in}}%
\pgfpathlineto{\pgfqpoint{3.274205in}{3.218445in}}%
\pgfpathlineto{\pgfqpoint{3.261293in}{3.235431in}}%
\pgfpathlineto{\pgfqpoint{3.248376in}{3.252689in}}%
\pgfpathlineto{\pgfqpoint{3.240766in}{3.237663in}}%
\pgfpathlineto{\pgfqpoint{3.233151in}{3.222820in}}%
\pgfpathlineto{\pgfqpoint{3.225529in}{3.208157in}}%
\pgfpathlineto{\pgfqpoint{3.217901in}{3.193671in}}%
\pgfpathclose%
\pgfusepath{fill}%
\end{pgfscope}%
\begin{pgfscope}%
\pgfpathrectangle{\pgfqpoint{1.254980in}{0.150000in}}{\pgfqpoint{5.490039in}{5.490039in}}%
\pgfusepath{clip}%
\pgfsetbuttcap%
\pgfsetroundjoin%
\definecolor{currentfill}{rgb}{0.237441,0.305202,0.541921}%
\pgfsetfillcolor{currentfill}%
\pgfsetfillopacity{0.700000}%
\pgfsetlinewidth{0.000000pt}%
\definecolor{currentstroke}{rgb}{0.000000,0.000000,0.000000}%
\pgfsetstrokecolor{currentstroke}%
\pgfsetdash{}{0pt}%
\pgfpathmoveto{\pgfqpoint{4.384065in}{2.925893in}}%
\pgfpathlineto{\pgfqpoint{4.397016in}{2.921485in}}%
\pgfpathlineto{\pgfqpoint{4.409974in}{2.917251in}}%
\pgfpathlineto{\pgfqpoint{4.422938in}{2.913190in}}%
\pgfpathlineto{\pgfqpoint{4.435909in}{2.909301in}}%
\pgfpathlineto{\pgfqpoint{4.443265in}{2.922601in}}%
\pgfpathlineto{\pgfqpoint{4.450619in}{2.936080in}}%
\pgfpathlineto{\pgfqpoint{4.457970in}{2.949744in}}%
\pgfpathlineto{\pgfqpoint{4.465320in}{2.963598in}}%
\pgfpathlineto{\pgfqpoint{4.452359in}{2.967952in}}%
\pgfpathlineto{\pgfqpoint{4.439404in}{2.972479in}}%
\pgfpathlineto{\pgfqpoint{4.426455in}{2.977179in}}%
\pgfpathlineto{\pgfqpoint{4.413513in}{2.982051in}}%
\pgfpathlineto{\pgfqpoint{4.406154in}{2.967722in}}%
\pgfpathlineto{\pgfqpoint{4.398793in}{2.953590in}}%
\pgfpathlineto{\pgfqpoint{4.391430in}{2.939648in}}%
\pgfpathlineto{\pgfqpoint{4.384065in}{2.925893in}}%
\pgfpathclose%
\pgfusepath{fill}%
\end{pgfscope}%
\begin{pgfscope}%
\pgfpathrectangle{\pgfqpoint{1.254980in}{0.150000in}}{\pgfqpoint{5.490039in}{5.490039in}}%
\pgfusepath{clip}%
\pgfsetbuttcap%
\pgfsetroundjoin%
\definecolor{currentfill}{rgb}{0.144759,0.519093,0.556572}%
\pgfsetfillcolor{currentfill}%
\pgfsetfillopacity{0.700000}%
\pgfsetlinewidth{0.000000pt}%
\definecolor{currentstroke}{rgb}{0.000000,0.000000,0.000000}%
\pgfsetstrokecolor{currentstroke}%
\pgfsetdash{}{0pt}%
\pgfpathmoveto{\pgfqpoint{3.092858in}{3.481977in}}%
\pgfpathlineto{\pgfqpoint{3.105860in}{3.461244in}}%
\pgfpathlineto{\pgfqpoint{3.118853in}{3.440817in}}%
\pgfpathlineto{\pgfqpoint{3.131837in}{3.420691in}}%
\pgfpathlineto{\pgfqpoint{3.144814in}{3.400865in}}%
\pgfpathlineto{\pgfqpoint{3.152433in}{3.416620in}}%
\pgfpathlineto{\pgfqpoint{3.160045in}{3.432579in}}%
\pgfpathlineto{\pgfqpoint{3.167650in}{3.448745in}}%
\pgfpathlineto{\pgfqpoint{3.175248in}{3.465122in}}%
\pgfpathlineto{\pgfqpoint{3.162279in}{3.485232in}}%
\pgfpathlineto{\pgfqpoint{3.149302in}{3.505641in}}%
\pgfpathlineto{\pgfqpoint{3.136316in}{3.526354in}}%
\pgfpathlineto{\pgfqpoint{3.123322in}{3.547372in}}%
\pgfpathlineto{\pgfqpoint{3.115717in}{3.530699in}}%
\pgfpathlineto{\pgfqpoint{3.108104in}{3.514244in}}%
\pgfpathlineto{\pgfqpoint{3.100485in}{3.498005in}}%
\pgfpathlineto{\pgfqpoint{3.092858in}{3.481977in}}%
\pgfpathclose%
\pgfusepath{fill}%
\end{pgfscope}%
\begin{pgfscope}%
\pgfpathrectangle{\pgfqpoint{1.254980in}{0.150000in}}{\pgfqpoint{5.490039in}{5.490039in}}%
\pgfusepath{clip}%
\pgfsetbuttcap%
\pgfsetroundjoin%
\definecolor{currentfill}{rgb}{0.235526,0.309527,0.542944}%
\pgfsetfillcolor{currentfill}%
\pgfsetfillopacity{0.700000}%
\pgfsetlinewidth{0.000000pt}%
\definecolor{currentstroke}{rgb}{0.000000,0.000000,0.000000}%
\pgfsetstrokecolor{currentstroke}%
\pgfsetdash{}{0pt}%
\pgfpathmoveto{\pgfqpoint{3.424186in}{2.952919in}}%
\pgfpathlineto{\pgfqpoint{3.437051in}{2.939995in}}%
\pgfpathlineto{\pgfqpoint{3.449915in}{2.927310in}}%
\pgfpathlineto{\pgfqpoint{3.462777in}{2.914861in}}%
\pgfpathlineto{\pgfqpoint{3.475638in}{2.902646in}}%
\pgfpathlineto{\pgfqpoint{3.483225in}{2.915937in}}%
\pgfpathlineto{\pgfqpoint{3.490807in}{2.929369in}}%
\pgfpathlineto{\pgfqpoint{3.498383in}{2.942946in}}%
\pgfpathlineto{\pgfqpoint{3.505955in}{2.956670in}}%
\pgfpathlineto{\pgfqpoint{3.493102in}{2.969133in}}%
\pgfpathlineto{\pgfqpoint{3.480247in}{2.981830in}}%
\pgfpathlineto{\pgfqpoint{3.467391in}{2.994764in}}%
\pgfpathlineto{\pgfqpoint{3.454533in}{3.007936in}}%
\pgfpathlineto{\pgfqpoint{3.446955in}{2.993953in}}%
\pgfpathlineto{\pgfqpoint{3.439370in}{2.980124in}}%
\pgfpathlineto{\pgfqpoint{3.431781in}{2.966447in}}%
\pgfpathlineto{\pgfqpoint{3.424186in}{2.952919in}}%
\pgfpathclose%
\pgfusepath{fill}%
\end{pgfscope}%
\begin{pgfscope}%
\pgfpathrectangle{\pgfqpoint{1.254980in}{0.150000in}}{\pgfqpoint{5.490039in}{5.490039in}}%
\pgfusepath{clip}%
\pgfsetbuttcap%
\pgfsetroundjoin%
\definecolor{currentfill}{rgb}{0.185556,0.418570,0.556753}%
\pgfsetfillcolor{currentfill}%
\pgfsetfillopacity{0.700000}%
\pgfsetlinewidth{0.000000pt}%
\definecolor{currentstroke}{rgb}{0.000000,0.000000,0.000000}%
\pgfsetstrokecolor{currentstroke}%
\pgfsetdash{}{0pt}%
\pgfpathmoveto{\pgfqpoint{4.871795in}{3.183009in}}%
\pgfpathlineto{\pgfqpoint{4.884856in}{3.179392in}}%
\pgfpathlineto{\pgfqpoint{4.897926in}{3.175934in}}%
\pgfpathlineto{\pgfqpoint{4.911004in}{3.172636in}}%
\pgfpathlineto{\pgfqpoint{4.924090in}{3.169497in}}%
\pgfpathlineto{\pgfqpoint{4.931365in}{3.185024in}}%
\pgfpathlineto{\pgfqpoint{4.938642in}{3.200841in}}%
\pgfpathlineto{\pgfqpoint{4.945922in}{3.216955in}}%
\pgfpathlineto{\pgfqpoint{4.953204in}{3.233374in}}%
\pgfpathlineto{\pgfqpoint{4.940131in}{3.237144in}}%
\pgfpathlineto{\pgfqpoint{4.927067in}{3.241073in}}%
\pgfpathlineto{\pgfqpoint{4.914012in}{3.245162in}}%
\pgfpathlineto{\pgfqpoint{4.900964in}{3.249410in}}%
\pgfpathlineto{\pgfqpoint{4.893668in}{3.232350in}}%
\pgfpathlineto{\pgfqpoint{4.886375in}{3.215602in}}%
\pgfpathlineto{\pgfqpoint{4.879084in}{3.199157in}}%
\pgfpathlineto{\pgfqpoint{4.871795in}{3.183009in}}%
\pgfpathclose%
\pgfusepath{fill}%
\end{pgfscope}%
\begin{pgfscope}%
\pgfpathrectangle{\pgfqpoint{1.254980in}{0.150000in}}{\pgfqpoint{5.490039in}{5.490039in}}%
\pgfusepath{clip}%
\pgfsetbuttcap%
\pgfsetroundjoin%
\definecolor{currentfill}{rgb}{0.257322,0.256130,0.526563}%
\pgfsetfillcolor{currentfill}%
\pgfsetfillopacity{0.700000}%
\pgfsetlinewidth{0.000000pt}%
\definecolor{currentstroke}{rgb}{0.000000,0.000000,0.000000}%
\pgfsetstrokecolor{currentstroke}%
\pgfsetdash{}{0pt}%
\pgfpathmoveto{\pgfqpoint{4.007313in}{2.820907in}}%
\pgfpathlineto{\pgfqpoint{4.020192in}{2.814458in}}%
\pgfpathlineto{\pgfqpoint{4.033075in}{2.808199in}}%
\pgfpathlineto{\pgfqpoint{4.045962in}{2.802128in}}%
\pgfpathlineto{\pgfqpoint{4.058854in}{2.796246in}}%
\pgfpathlineto{\pgfqpoint{4.066302in}{2.809085in}}%
\pgfpathlineto{\pgfqpoint{4.073746in}{2.822062in}}%
\pgfpathlineto{\pgfqpoint{4.081187in}{2.835179in}}%
\pgfpathlineto{\pgfqpoint{4.088623in}{2.848442in}}%
\pgfpathlineto{\pgfqpoint{4.075739in}{2.854680in}}%
\pgfpathlineto{\pgfqpoint{4.062859in}{2.861107in}}%
\pgfpathlineto{\pgfqpoint{4.049984in}{2.867722in}}%
\pgfpathlineto{\pgfqpoint{4.037112in}{2.874526in}}%
\pgfpathlineto{\pgfqpoint{4.029668in}{2.860897in}}%
\pgfpathlineto{\pgfqpoint{4.022220in}{2.847421in}}%
\pgfpathlineto{\pgfqpoint{4.014768in}{2.834092in}}%
\pgfpathlineto{\pgfqpoint{4.007313in}{2.820907in}}%
\pgfpathclose%
\pgfusepath{fill}%
\end{pgfscope}%
\begin{pgfscope}%
\pgfpathrectangle{\pgfqpoint{1.254980in}{0.150000in}}{\pgfqpoint{5.490039in}{5.490039in}}%
\pgfusepath{clip}%
\pgfsetbuttcap%
\pgfsetroundjoin%
\definecolor{currentfill}{rgb}{0.244972,0.287675,0.537260}%
\pgfsetfillcolor{currentfill}%
\pgfsetfillopacity{0.700000}%
\pgfsetlinewidth{0.000000pt}%
\definecolor{currentstroke}{rgb}{0.000000,0.000000,0.000000}%
\pgfsetstrokecolor{currentstroke}%
\pgfsetdash{}{0pt}%
\pgfpathmoveto{\pgfqpoint{4.302801in}{2.890249in}}%
\pgfpathlineto{\pgfqpoint{4.315737in}{2.885579in}}%
\pgfpathlineto{\pgfqpoint{4.328678in}{2.881085in}}%
\pgfpathlineto{\pgfqpoint{4.341626in}{2.876766in}}%
\pgfpathlineto{\pgfqpoint{4.354580in}{2.872623in}}%
\pgfpathlineto{\pgfqpoint{4.361955in}{2.885688in}}%
\pgfpathlineto{\pgfqpoint{4.369328in}{2.898918in}}%
\pgfpathlineto{\pgfqpoint{4.376698in}{2.912318in}}%
\pgfpathlineto{\pgfqpoint{4.384065in}{2.925893in}}%
\pgfpathlineto{\pgfqpoint{4.371120in}{2.930474in}}%
\pgfpathlineto{\pgfqpoint{4.358181in}{2.935231in}}%
\pgfpathlineto{\pgfqpoint{4.345249in}{2.940162in}}%
\pgfpathlineto{\pgfqpoint{4.332322in}{2.945270in}}%
\pgfpathlineto{\pgfqpoint{4.324945in}{2.931247in}}%
\pgfpathlineto{\pgfqpoint{4.317567in}{2.917406in}}%
\pgfpathlineto{\pgfqpoint{4.310185in}{2.903741in}}%
\pgfpathlineto{\pgfqpoint{4.302801in}{2.890249in}}%
\pgfpathclose%
\pgfusepath{fill}%
\end{pgfscope}%
\begin{pgfscope}%
\pgfpathrectangle{\pgfqpoint{1.254980in}{0.150000in}}{\pgfqpoint{5.490039in}{5.490039in}}%
\pgfusepath{clip}%
\pgfsetbuttcap%
\pgfsetroundjoin%
\definecolor{currentfill}{rgb}{0.177423,0.437527,0.557565}%
\pgfsetfillcolor{currentfill}%
\pgfsetfillopacity{0.700000}%
\pgfsetlinewidth{0.000000pt}%
\definecolor{currentstroke}{rgb}{0.000000,0.000000,0.000000}%
\pgfsetstrokecolor{currentstroke}%
\pgfsetdash{}{0pt}%
\pgfpathmoveto{\pgfqpoint{3.166140in}{3.264467in}}%
\pgfpathlineto{\pgfqpoint{3.179090in}{3.246348in}}%
\pgfpathlineto{\pgfqpoint{3.192033in}{3.228510in}}%
\pgfpathlineto{\pgfqpoint{3.204970in}{3.210952in}}%
\pgfpathlineto{\pgfqpoint{3.217901in}{3.193671in}}%
\pgfpathlineto{\pgfqpoint{3.225529in}{3.208157in}}%
\pgfpathlineto{\pgfqpoint{3.233151in}{3.222820in}}%
\pgfpathlineto{\pgfqpoint{3.240766in}{3.237663in}}%
\pgfpathlineto{\pgfqpoint{3.248376in}{3.252689in}}%
\pgfpathlineto{\pgfqpoint{3.235453in}{3.270222in}}%
\pgfpathlineto{\pgfqpoint{3.222524in}{3.288033in}}%
\pgfpathlineto{\pgfqpoint{3.209589in}{3.306123in}}%
\pgfpathlineto{\pgfqpoint{3.196648in}{3.324496in}}%
\pgfpathlineto{\pgfqpoint{3.189031in}{3.309207in}}%
\pgfpathlineto{\pgfqpoint{3.181407in}{3.294108in}}%
\pgfpathlineto{\pgfqpoint{3.173777in}{3.279196in}}%
\pgfpathlineto{\pgfqpoint{3.166140in}{3.264467in}}%
\pgfpathclose%
\pgfusepath{fill}%
\end{pgfscope}%
\begin{pgfscope}%
\pgfpathrectangle{\pgfqpoint{1.254980in}{0.150000in}}{\pgfqpoint{5.490039in}{5.490039in}}%
\pgfusepath{clip}%
\pgfsetbuttcap%
\pgfsetroundjoin%
\definecolor{currentfill}{rgb}{0.244972,0.287675,0.537260}%
\pgfsetfillcolor{currentfill}%
\pgfsetfillopacity{0.700000}%
\pgfsetlinewidth{0.000000pt}%
\definecolor{currentstroke}{rgb}{0.000000,0.000000,0.000000}%
\pgfsetstrokecolor{currentstroke}%
\pgfsetdash{}{0pt}%
\pgfpathmoveto{\pgfqpoint{3.475638in}{2.902646in}}%
\pgfpathlineto{\pgfqpoint{3.488497in}{2.890664in}}%
\pgfpathlineto{\pgfqpoint{3.501355in}{2.878912in}}%
\pgfpathlineto{\pgfqpoint{3.514212in}{2.867390in}}%
\pgfpathlineto{\pgfqpoint{3.527069in}{2.856095in}}%
\pgfpathlineto{\pgfqpoint{3.534648in}{2.869149in}}%
\pgfpathlineto{\pgfqpoint{3.542222in}{2.882336in}}%
\pgfpathlineto{\pgfqpoint{3.549791in}{2.895662in}}%
\pgfpathlineto{\pgfqpoint{3.557355in}{2.909128in}}%
\pgfpathlineto{\pgfqpoint{3.544506in}{2.920670in}}%
\pgfpathlineto{\pgfqpoint{3.531657in}{2.932440in}}%
\pgfpathlineto{\pgfqpoint{3.518806in}{2.944439in}}%
\pgfpathlineto{\pgfqpoint{3.505955in}{2.956670in}}%
\pgfpathlineto{\pgfqpoint{3.498383in}{2.942946in}}%
\pgfpathlineto{\pgfqpoint{3.490807in}{2.929369in}}%
\pgfpathlineto{\pgfqpoint{3.483225in}{2.915937in}}%
\pgfpathlineto{\pgfqpoint{3.475638in}{2.902646in}}%
\pgfpathclose%
\pgfusepath{fill}%
\end{pgfscope}%
\begin{pgfscope}%
\pgfpathrectangle{\pgfqpoint{1.254980in}{0.150000in}}{\pgfqpoint{5.490039in}{5.490039in}}%
\pgfusepath{clip}%
\pgfsetbuttcap%
\pgfsetroundjoin%
\definecolor{currentfill}{rgb}{0.260571,0.246922,0.522828}%
\pgfsetfillcolor{currentfill}%
\pgfsetfillopacity{0.700000}%
\pgfsetlinewidth{0.000000pt}%
\definecolor{currentstroke}{rgb}{0.000000,0.000000,0.000000}%
\pgfsetstrokecolor{currentstroke}%
\pgfsetdash{}{0pt}%
\pgfpathmoveto{\pgfqpoint{3.793067in}{2.805101in}}%
\pgfpathlineto{\pgfqpoint{3.805922in}{2.796927in}}%
\pgfpathlineto{\pgfqpoint{3.818781in}{2.788955in}}%
\pgfpathlineto{\pgfqpoint{3.831641in}{2.781184in}}%
\pgfpathlineto{\pgfqpoint{3.844504in}{2.773614in}}%
\pgfpathlineto{\pgfqpoint{3.852008in}{2.786374in}}%
\pgfpathlineto{\pgfqpoint{3.859506in}{2.799261in}}%
\pgfpathlineto{\pgfqpoint{3.867001in}{2.812277in}}%
\pgfpathlineto{\pgfqpoint{3.874491in}{2.825427in}}%
\pgfpathlineto{\pgfqpoint{3.861635in}{2.833299in}}%
\pgfpathlineto{\pgfqpoint{3.848782in}{2.841371in}}%
\pgfpathlineto{\pgfqpoint{3.835931in}{2.849645in}}%
\pgfpathlineto{\pgfqpoint{3.823082in}{2.858120in}}%
\pgfpathlineto{\pgfqpoint{3.815585in}{2.844659in}}%
\pgfpathlineto{\pgfqpoint{3.808083in}{2.831337in}}%
\pgfpathlineto{\pgfqpoint{3.800577in}{2.818153in}}%
\pgfpathlineto{\pgfqpoint{3.793067in}{2.805101in}}%
\pgfpathclose%
\pgfusepath{fill}%
\end{pgfscope}%
\begin{pgfscope}%
\pgfpathrectangle{\pgfqpoint{1.254980in}{0.150000in}}{\pgfqpoint{5.490039in}{5.490039in}}%
\pgfusepath{clip}%
\pgfsetbuttcap%
\pgfsetroundjoin%
\definecolor{currentfill}{rgb}{0.257322,0.256130,0.526563}%
\pgfsetfillcolor{currentfill}%
\pgfsetfillopacity{0.700000}%
\pgfsetlinewidth{0.000000pt}%
\definecolor{currentstroke}{rgb}{0.000000,0.000000,0.000000}%
\pgfsetstrokecolor{currentstroke}%
\pgfsetdash{}{0pt}%
\pgfpathmoveto{\pgfqpoint{3.660139in}{2.824793in}}%
\pgfpathlineto{\pgfqpoint{3.672989in}{2.815229in}}%
\pgfpathlineto{\pgfqpoint{3.685841in}{2.805878in}}%
\pgfpathlineto{\pgfqpoint{3.698693in}{2.796738in}}%
\pgfpathlineto{\pgfqpoint{3.711547in}{2.787808in}}%
\pgfpathlineto{\pgfqpoint{3.719083in}{2.800624in}}%
\pgfpathlineto{\pgfqpoint{3.726615in}{2.813568in}}%
\pgfpathlineto{\pgfqpoint{3.734142in}{2.826640in}}%
\pgfpathlineto{\pgfqpoint{3.741664in}{2.839846in}}%
\pgfpathlineto{\pgfqpoint{3.728818in}{2.849051in}}%
\pgfpathlineto{\pgfqpoint{3.715973in}{2.858466in}}%
\pgfpathlineto{\pgfqpoint{3.703129in}{2.868091in}}%
\pgfpathlineto{\pgfqpoint{3.690286in}{2.877930in}}%
\pgfpathlineto{\pgfqpoint{3.682756in}{2.864439in}}%
\pgfpathlineto{\pgfqpoint{3.675222in}{2.851088in}}%
\pgfpathlineto{\pgfqpoint{3.667683in}{2.837874in}}%
\pgfpathlineto{\pgfqpoint{3.660139in}{2.824793in}}%
\pgfpathclose%
\pgfusepath{fill}%
\end{pgfscope}%
\begin{pgfscope}%
\pgfpathrectangle{\pgfqpoint{1.254980in}{0.150000in}}{\pgfqpoint{5.490039in}{5.490039in}}%
\pgfusepath{clip}%
\pgfsetbuttcap%
\pgfsetroundjoin%
\definecolor{currentfill}{rgb}{0.177423,0.437527,0.557565}%
\pgfsetfillcolor{currentfill}%
\pgfsetfillopacity{0.700000}%
\pgfsetlinewidth{0.000000pt}%
\definecolor{currentstroke}{rgb}{0.000000,0.000000,0.000000}%
\pgfsetstrokecolor{currentstroke}%
\pgfsetdash{}{0pt}%
\pgfpathmoveto{\pgfqpoint{4.953204in}{3.233374in}}%
\pgfpathlineto{\pgfqpoint{4.966284in}{3.229762in}}%
\pgfpathlineto{\pgfqpoint{4.979373in}{3.226308in}}%
\pgfpathlineto{\pgfqpoint{4.992471in}{3.223012in}}%
\pgfpathlineto{\pgfqpoint{5.005578in}{3.219874in}}%
\pgfpathlineto{\pgfqpoint{5.012848in}{3.235955in}}%
\pgfpathlineto{\pgfqpoint{5.020121in}{3.252349in}}%
\pgfpathlineto{\pgfqpoint{5.027397in}{3.269063in}}%
\pgfpathlineto{\pgfqpoint{5.014302in}{3.272693in}}%
\pgfpathlineto{\pgfqpoint{5.001215in}{3.276481in}}%
\pgfpathlineto{\pgfqpoint{4.988137in}{3.280426in}}%
\pgfpathlineto{\pgfqpoint{4.975067in}{3.284529in}}%
\pgfpathlineto{\pgfqpoint{4.967776in}{3.267153in}}%
\pgfpathlineto{\pgfqpoint{4.960488in}{3.250104in}}%
\pgfpathlineto{\pgfqpoint{4.953204in}{3.233374in}}%
\pgfpathclose%
\pgfusepath{fill}%
\end{pgfscope}%
\begin{pgfscope}%
\pgfpathrectangle{\pgfqpoint{1.254980in}{0.150000in}}{\pgfqpoint{5.490039in}{5.490039in}}%
\pgfusepath{clip}%
\pgfsetbuttcap%
\pgfsetroundjoin%
\definecolor{currentfill}{rgb}{0.250425,0.274290,0.533103}%
\pgfsetfillcolor{currentfill}%
\pgfsetfillopacity{0.700000}%
\pgfsetlinewidth{0.000000pt}%
\definecolor{currentstroke}{rgb}{0.000000,0.000000,0.000000}%
\pgfsetstrokecolor{currentstroke}%
\pgfsetdash{}{0pt}%
\pgfpathmoveto{\pgfqpoint{4.221518in}{2.856711in}}%
\pgfpathlineto{\pgfqpoint{4.234439in}{2.851739in}}%
\pgfpathlineto{\pgfqpoint{4.247365in}{2.846946in}}%
\pgfpathlineto{\pgfqpoint{4.260298in}{2.842332in}}%
\pgfpathlineto{\pgfqpoint{4.273236in}{2.837896in}}%
\pgfpathlineto{\pgfqpoint{4.280632in}{2.850751in}}%
\pgfpathlineto{\pgfqpoint{4.288024in}{2.863758in}}%
\pgfpathlineto{\pgfqpoint{4.295414in}{2.876923in}}%
\pgfpathlineto{\pgfqpoint{4.302801in}{2.890249in}}%
\pgfpathlineto{\pgfqpoint{4.289871in}{2.895096in}}%
\pgfpathlineto{\pgfqpoint{4.276948in}{2.900120in}}%
\pgfpathlineto{\pgfqpoint{4.264029in}{2.905323in}}%
\pgfpathlineto{\pgfqpoint{4.251117in}{2.910706in}}%
\pgfpathlineto{\pgfqpoint{4.243721in}{2.896959in}}%
\pgfpathlineto{\pgfqpoint{4.236323in}{2.883381in}}%
\pgfpathlineto{\pgfqpoint{4.228922in}{2.869966in}}%
\pgfpathlineto{\pgfqpoint{4.221518in}{2.856711in}}%
\pgfpathclose%
\pgfusepath{fill}%
\end{pgfscope}%
\begin{pgfscope}%
\pgfpathrectangle{\pgfqpoint{1.254980in}{0.150000in}}{\pgfqpoint{5.490039in}{5.490039in}}%
\pgfusepath{clip}%
\pgfsetbuttcap%
\pgfsetroundjoin%
\definecolor{currentfill}{rgb}{0.165117,0.467423,0.558141}%
\pgfsetfillcolor{currentfill}%
\pgfsetfillopacity{0.700000}%
\pgfsetlinewidth{0.000000pt}%
\definecolor{currentstroke}{rgb}{0.000000,0.000000,0.000000}%
\pgfsetstrokecolor{currentstroke}%
\pgfsetdash{}{0pt}%
\pgfpathmoveto{\pgfqpoint{3.114273in}{3.339822in}}%
\pgfpathlineto{\pgfqpoint{3.127251in}{3.320546in}}%
\pgfpathlineto{\pgfqpoint{3.140221in}{3.301564in}}%
\pgfpathlineto{\pgfqpoint{3.153184in}{3.282872in}}%
\pgfpathlineto{\pgfqpoint{3.166140in}{3.264467in}}%
\pgfpathlineto{\pgfqpoint{3.173777in}{3.279196in}}%
\pgfpathlineto{\pgfqpoint{3.181407in}{3.294108in}}%
\pgfpathlineto{\pgfqpoint{3.189031in}{3.309207in}}%
\pgfpathlineto{\pgfqpoint{3.196648in}{3.324496in}}%
\pgfpathlineto{\pgfqpoint{3.183700in}{3.343154in}}%
\pgfpathlineto{\pgfqpoint{3.170745in}{3.362099in}}%
\pgfpathlineto{\pgfqpoint{3.157784in}{3.381336in}}%
\pgfpathlineto{\pgfqpoint{3.144814in}{3.400865in}}%
\pgfpathlineto{\pgfqpoint{3.137189in}{3.385312in}}%
\pgfpathlineto{\pgfqpoint{3.129557in}{3.369955in}}%
\pgfpathlineto{\pgfqpoint{3.121919in}{3.354793in}}%
\pgfpathlineto{\pgfqpoint{3.114273in}{3.339822in}}%
\pgfpathclose%
\pgfusepath{fill}%
\end{pgfscope}%
\begin{pgfscope}%
\pgfpathrectangle{\pgfqpoint{1.254980in}{0.150000in}}{\pgfqpoint{5.490039in}{5.490039in}}%
\pgfusepath{clip}%
\pgfsetbuttcap%
\pgfsetroundjoin%
\definecolor{currentfill}{rgb}{0.262138,0.242286,0.520837}%
\pgfsetfillcolor{currentfill}%
\pgfsetfillopacity{0.700000}%
\pgfsetlinewidth{0.000000pt}%
\definecolor{currentstroke}{rgb}{0.000000,0.000000,0.000000}%
\pgfsetstrokecolor{currentstroke}%
\pgfsetdash{}{0pt}%
\pgfpathmoveto{\pgfqpoint{3.925944in}{2.795918in}}%
\pgfpathlineto{\pgfqpoint{3.938816in}{2.789029in}}%
\pgfpathlineto{\pgfqpoint{3.951691in}{2.782334in}}%
\pgfpathlineto{\pgfqpoint{3.964569in}{2.775832in}}%
\pgfpathlineto{\pgfqpoint{3.977452in}{2.769522in}}%
\pgfpathlineto{\pgfqpoint{3.984923in}{2.782173in}}%
\pgfpathlineto{\pgfqpoint{3.992390in}{2.794951in}}%
\pgfpathlineto{\pgfqpoint{3.999854in}{2.807861in}}%
\pgfpathlineto{\pgfqpoint{4.007313in}{2.820907in}}%
\pgfpathlineto{\pgfqpoint{3.994438in}{2.827546in}}%
\pgfpathlineto{\pgfqpoint{3.981567in}{2.834377in}}%
\pgfpathlineto{\pgfqpoint{3.968699in}{2.841400in}}%
\pgfpathlineto{\pgfqpoint{3.955835in}{2.848617in}}%
\pgfpathlineto{\pgfqpoint{3.948368in}{2.835233in}}%
\pgfpathlineto{\pgfqpoint{3.940898in}{2.821991in}}%
\pgfpathlineto{\pgfqpoint{3.933423in}{2.808887in}}%
\pgfpathlineto{\pgfqpoint{3.925944in}{2.795918in}}%
\pgfpathclose%
\pgfusepath{fill}%
\end{pgfscope}%
\begin{pgfscope}%
\pgfpathrectangle{\pgfqpoint{1.254980in}{0.150000in}}{\pgfqpoint{5.490039in}{5.490039in}}%
\pgfusepath{clip}%
\pgfsetbuttcap%
\pgfsetroundjoin%
\definecolor{currentfill}{rgb}{0.252194,0.269783,0.531579}%
\pgfsetfillcolor{currentfill}%
\pgfsetfillopacity{0.700000}%
\pgfsetlinewidth{0.000000pt}%
\definecolor{currentstroke}{rgb}{0.000000,0.000000,0.000000}%
\pgfsetstrokecolor{currentstroke}%
\pgfsetdash{}{0pt}%
\pgfpathmoveto{\pgfqpoint{3.527069in}{2.856095in}}%
\pgfpathlineto{\pgfqpoint{3.539925in}{2.845026in}}%
\pgfpathlineto{\pgfqpoint{3.552780in}{2.834182in}}%
\pgfpathlineto{\pgfqpoint{3.565635in}{2.823560in}}%
\pgfpathlineto{\pgfqpoint{3.578491in}{2.813159in}}%
\pgfpathlineto{\pgfqpoint{3.586062in}{2.825976in}}%
\pgfpathlineto{\pgfqpoint{3.593628in}{2.838920in}}%
\pgfpathlineto{\pgfqpoint{3.601190in}{2.851994in}}%
\pgfpathlineto{\pgfqpoint{3.608746in}{2.865202in}}%
\pgfpathlineto{\pgfqpoint{3.595898in}{2.875850in}}%
\pgfpathlineto{\pgfqpoint{3.583051in}{2.886719in}}%
\pgfpathlineto{\pgfqpoint{3.570203in}{2.897811in}}%
\pgfpathlineto{\pgfqpoint{3.557355in}{2.909128in}}%
\pgfpathlineto{\pgfqpoint{3.549791in}{2.895662in}}%
\pgfpathlineto{\pgfqpoint{3.542222in}{2.882336in}}%
\pgfpathlineto{\pgfqpoint{3.534648in}{2.869149in}}%
\pgfpathlineto{\pgfqpoint{3.527069in}{2.856095in}}%
\pgfpathclose%
\pgfusepath{fill}%
\end{pgfscope}%
\begin{pgfscope}%
\pgfpathrectangle{\pgfqpoint{1.254980in}{0.150000in}}{\pgfqpoint{5.490039in}{5.490039in}}%
\pgfusepath{clip}%
\pgfsetbuttcap%
\pgfsetroundjoin%
\definecolor{currentfill}{rgb}{0.255645,0.260703,0.528312}%
\pgfsetfillcolor{currentfill}%
\pgfsetfillopacity{0.700000}%
\pgfsetlinewidth{0.000000pt}%
\definecolor{currentstroke}{rgb}{0.000000,0.000000,0.000000}%
\pgfsetstrokecolor{currentstroke}%
\pgfsetdash{}{0pt}%
\pgfpathmoveto{\pgfqpoint{4.140206in}{2.825346in}}%
\pgfpathlineto{\pgfqpoint{4.153114in}{2.820033in}}%
\pgfpathlineto{\pgfqpoint{4.166026in}{2.814902in}}%
\pgfpathlineto{\pgfqpoint{4.178945in}{2.809953in}}%
\pgfpathlineto{\pgfqpoint{4.191868in}{2.805184in}}%
\pgfpathlineto{\pgfqpoint{4.199286in}{2.817850in}}%
\pgfpathlineto{\pgfqpoint{4.206700in}{2.830657in}}%
\pgfpathlineto{\pgfqpoint{4.214110in}{2.843609in}}%
\pgfpathlineto{\pgfqpoint{4.221518in}{2.856711in}}%
\pgfpathlineto{\pgfqpoint{4.208602in}{2.861862in}}%
\pgfpathlineto{\pgfqpoint{4.195692in}{2.867195in}}%
\pgfpathlineto{\pgfqpoint{4.182787in}{2.872709in}}%
\pgfpathlineto{\pgfqpoint{4.169888in}{2.878406in}}%
\pgfpathlineto{\pgfqpoint{4.162472in}{2.864911in}}%
\pgfpathlineto{\pgfqpoint{4.155053in}{2.851573in}}%
\pgfpathlineto{\pgfqpoint{4.147631in}{2.838386in}}%
\pgfpathlineto{\pgfqpoint{4.140206in}{2.825346in}}%
\pgfpathclose%
\pgfusepath{fill}%
\end{pgfscope}%
\begin{pgfscope}%
\pgfpathrectangle{\pgfqpoint{1.254980in}{0.150000in}}{\pgfqpoint{5.490039in}{5.490039in}}%
\pgfusepath{clip}%
\pgfsetbuttcap%
\pgfsetroundjoin%
\definecolor{currentfill}{rgb}{0.212395,0.359683,0.551710}%
\pgfsetfillcolor{currentfill}%
\pgfsetfillopacity{0.700000}%
\pgfsetlinewidth{0.000000pt}%
\definecolor{currentstroke}{rgb}{0.000000,0.000000,0.000000}%
\pgfsetstrokecolor{currentstroke}%
\pgfsetdash{}{0pt}%
\pgfpathmoveto{\pgfqpoint{4.679901in}{3.030724in}}%
\pgfpathlineto{\pgfqpoint{4.692936in}{3.027537in}}%
\pgfpathlineto{\pgfqpoint{4.705978in}{3.024515in}}%
\pgfpathlineto{\pgfqpoint{4.719029in}{3.021655in}}%
\pgfpathlineto{\pgfqpoint{4.732088in}{3.018959in}}%
\pgfpathlineto{\pgfqpoint{4.739382in}{3.032630in}}%
\pgfpathlineto{\pgfqpoint{4.746676in}{3.046523in}}%
\pgfpathlineto{\pgfqpoint{4.753969in}{3.060645in}}%
\pgfpathlineto{\pgfqpoint{4.761263in}{3.075001in}}%
\pgfpathlineto{\pgfqpoint{4.748216in}{3.078246in}}%
\pgfpathlineto{\pgfqpoint{4.735178in}{3.081653in}}%
\pgfpathlineto{\pgfqpoint{4.722147in}{3.085225in}}%
\pgfpathlineto{\pgfqpoint{4.709125in}{3.088960in}}%
\pgfpathlineto{\pgfqpoint{4.701819in}{3.074045in}}%
\pgfpathlineto{\pgfqpoint{4.694513in}{3.059372in}}%
\pgfpathlineto{\pgfqpoint{4.687208in}{3.044934in}}%
\pgfpathlineto{\pgfqpoint{4.679901in}{3.030724in}}%
\pgfpathclose%
\pgfusepath{fill}%
\end{pgfscope}%
\begin{pgfscope}%
\pgfpathrectangle{\pgfqpoint{1.254980in}{0.150000in}}{\pgfqpoint{5.490039in}{5.490039in}}%
\pgfusepath{clip}%
\pgfsetbuttcap%
\pgfsetroundjoin%
\definecolor{currentfill}{rgb}{0.221989,0.339161,0.548752}%
\pgfsetfillcolor{currentfill}%
\pgfsetfillopacity{0.700000}%
\pgfsetlinewidth{0.000000pt}%
\definecolor{currentstroke}{rgb}{0.000000,0.000000,0.000000}%
\pgfsetstrokecolor{currentstroke}%
\pgfsetdash{}{0pt}%
\pgfpathmoveto{\pgfqpoint{4.598561in}{2.988361in}}%
\pgfpathlineto{\pgfqpoint{4.611576in}{2.985034in}}%
\pgfpathlineto{\pgfqpoint{4.624599in}{2.981873in}}%
\pgfpathlineto{\pgfqpoint{4.637629in}{2.978878in}}%
\pgfpathlineto{\pgfqpoint{4.650668in}{2.976047in}}%
\pgfpathlineto{\pgfqpoint{4.657977in}{2.989404in}}%
\pgfpathlineto{\pgfqpoint{4.665286in}{3.002966in}}%
\pgfpathlineto{\pgfqpoint{4.672594in}{3.016737in}}%
\pgfpathlineto{\pgfqpoint{4.679901in}{3.030724in}}%
\pgfpathlineto{\pgfqpoint{4.666874in}{3.034075in}}%
\pgfpathlineto{\pgfqpoint{4.653855in}{3.037592in}}%
\pgfpathlineto{\pgfqpoint{4.640844in}{3.041273in}}%
\pgfpathlineto{\pgfqpoint{4.627840in}{3.045121in}}%
\pgfpathlineto{\pgfqpoint{4.620522in}{3.030603in}}%
\pgfpathlineto{\pgfqpoint{4.613202in}{3.016307in}}%
\pgfpathlineto{\pgfqpoint{4.605882in}{3.002229in}}%
\pgfpathlineto{\pgfqpoint{4.598561in}{2.988361in}}%
\pgfpathclose%
\pgfusepath{fill}%
\end{pgfscope}%
\begin{pgfscope}%
\pgfpathrectangle{\pgfqpoint{1.254980in}{0.150000in}}{\pgfqpoint{5.490039in}{5.490039in}}%
\pgfusepath{clip}%
\pgfsetbuttcap%
\pgfsetroundjoin%
\definecolor{currentfill}{rgb}{0.262138,0.242286,0.520837}%
\pgfsetfillcolor{currentfill}%
\pgfsetfillopacity{0.700000}%
\pgfsetlinewidth{0.000000pt}%
\definecolor{currentstroke}{rgb}{0.000000,0.000000,0.000000}%
\pgfsetstrokecolor{currentstroke}%
\pgfsetdash{}{0pt}%
\pgfpathmoveto{\pgfqpoint{3.711547in}{2.787808in}}%
\pgfpathlineto{\pgfqpoint{3.724402in}{2.779086in}}%
\pgfpathlineto{\pgfqpoint{3.737259in}{2.770571in}}%
\pgfpathlineto{\pgfqpoint{3.750118in}{2.762263in}}%
\pgfpathlineto{\pgfqpoint{3.762979in}{2.754159in}}%
\pgfpathlineto{\pgfqpoint{3.770508in}{2.766712in}}%
\pgfpathlineto{\pgfqpoint{3.778032in}{2.779384in}}%
\pgfpathlineto{\pgfqpoint{3.785552in}{2.792180in}}%
\pgfpathlineto{\pgfqpoint{3.793067in}{2.805101in}}%
\pgfpathlineto{\pgfqpoint{3.780213in}{2.813479in}}%
\pgfpathlineto{\pgfqpoint{3.767362in}{2.822062in}}%
\pgfpathlineto{\pgfqpoint{3.754512in}{2.830850in}}%
\pgfpathlineto{\pgfqpoint{3.741664in}{2.839846in}}%
\pgfpathlineto{\pgfqpoint{3.734142in}{2.826640in}}%
\pgfpathlineto{\pgfqpoint{3.726615in}{2.813568in}}%
\pgfpathlineto{\pgfqpoint{3.719083in}{2.800624in}}%
\pgfpathlineto{\pgfqpoint{3.711547in}{2.787808in}}%
\pgfpathclose%
\pgfusepath{fill}%
\end{pgfscope}%
\begin{pgfscope}%
\pgfpathrectangle{\pgfqpoint{1.254980in}{0.150000in}}{\pgfqpoint{5.490039in}{5.490039in}}%
\pgfusepath{clip}%
\pgfsetbuttcap%
\pgfsetroundjoin%
\definecolor{currentfill}{rgb}{0.204903,0.375746,0.553533}%
\pgfsetfillcolor{currentfill}%
\pgfsetfillopacity{0.700000}%
\pgfsetlinewidth{0.000000pt}%
\definecolor{currentstroke}{rgb}{0.000000,0.000000,0.000000}%
\pgfsetstrokecolor{currentstroke}%
\pgfsetdash{}{0pt}%
\pgfpathmoveto{\pgfqpoint{4.761263in}{3.075001in}}%
\pgfpathlineto{\pgfqpoint{4.774317in}{3.071918in}}%
\pgfpathlineto{\pgfqpoint{4.787380in}{3.068997in}}%
\pgfpathlineto{\pgfqpoint{4.800451in}{3.066238in}}%
\pgfpathlineto{\pgfqpoint{4.813531in}{3.063640in}}%
\pgfpathlineto{\pgfqpoint{4.820811in}{3.077671in}}%
\pgfpathlineto{\pgfqpoint{4.828092in}{3.091943in}}%
\pgfpathlineto{\pgfqpoint{4.835374in}{3.106462in}}%
\pgfpathlineto{\pgfqpoint{4.842656in}{3.121236in}}%
\pgfpathlineto{\pgfqpoint{4.829589in}{3.124410in}}%
\pgfpathlineto{\pgfqpoint{4.816531in}{3.127745in}}%
\pgfpathlineto{\pgfqpoint{4.803481in}{3.131242in}}%
\pgfpathlineto{\pgfqpoint{4.790439in}{3.134900in}}%
\pgfpathlineto{\pgfqpoint{4.783144in}{3.119541in}}%
\pgfpathlineto{\pgfqpoint{4.775850in}{3.104442in}}%
\pgfpathlineto{\pgfqpoint{4.768556in}{3.089598in}}%
\pgfpathlineto{\pgfqpoint{4.761263in}{3.075001in}}%
\pgfpathclose%
\pgfusepath{fill}%
\end{pgfscope}%
\begin{pgfscope}%
\pgfpathrectangle{\pgfqpoint{1.254980in}{0.150000in}}{\pgfqpoint{5.490039in}{5.490039in}}%
\pgfusepath{clip}%
\pgfsetbuttcap%
\pgfsetroundjoin%
\definecolor{currentfill}{rgb}{0.221989,0.339161,0.548752}%
\pgfsetfillcolor{currentfill}%
\pgfsetfillopacity{0.700000}%
\pgfsetlinewidth{0.000000pt}%
\definecolor{currentstroke}{rgb}{0.000000,0.000000,0.000000}%
\pgfsetstrokecolor{currentstroke}%
\pgfsetdash{}{0pt}%
\pgfpathmoveto{\pgfqpoint{3.290660in}{3.010643in}}%
\pgfpathlineto{\pgfqpoint{3.303558in}{2.995968in}}%
\pgfpathlineto{\pgfqpoint{3.316451in}{2.981547in}}%
\pgfpathlineto{\pgfqpoint{3.329341in}{2.967377in}}%
\pgfpathlineto{\pgfqpoint{3.342228in}{2.953457in}}%
\pgfpathlineto{\pgfqpoint{3.349855in}{2.966631in}}%
\pgfpathlineto{\pgfqpoint{3.357475in}{2.979949in}}%
\pgfpathlineto{\pgfqpoint{3.365090in}{2.993414in}}%
\pgfpathlineto{\pgfqpoint{3.372699in}{3.007029in}}%
\pgfpathlineto{\pgfqpoint{3.359821in}{3.021170in}}%
\pgfpathlineto{\pgfqpoint{3.346939in}{3.035561in}}%
\pgfpathlineto{\pgfqpoint{3.334054in}{3.050204in}}%
\pgfpathlineto{\pgfqpoint{3.321166in}{3.065100in}}%
\pgfpathlineto{\pgfqpoint{3.313548in}{3.051254in}}%
\pgfpathlineto{\pgfqpoint{3.305925in}{3.037564in}}%
\pgfpathlineto{\pgfqpoint{3.298296in}{3.024028in}}%
\pgfpathlineto{\pgfqpoint{3.290660in}{3.010643in}}%
\pgfpathclose%
\pgfusepath{fill}%
\end{pgfscope}%
\begin{pgfscope}%
\pgfpathrectangle{\pgfqpoint{1.254980in}{0.150000in}}{\pgfqpoint{5.490039in}{5.490039in}}%
\pgfusepath{clip}%
\pgfsetbuttcap%
\pgfsetroundjoin%
\definecolor{currentfill}{rgb}{0.229739,0.322361,0.545706}%
\pgfsetfillcolor{currentfill}%
\pgfsetfillopacity{0.700000}%
\pgfsetlinewidth{0.000000pt}%
\definecolor{currentstroke}{rgb}{0.000000,0.000000,0.000000}%
\pgfsetstrokecolor{currentstroke}%
\pgfsetdash{}{0pt}%
\pgfpathmoveto{\pgfqpoint{4.517233in}{2.947887in}}%
\pgfpathlineto{\pgfqpoint{4.530230in}{2.944383in}}%
\pgfpathlineto{\pgfqpoint{4.543233in}{2.941048in}}%
\pgfpathlineto{\pgfqpoint{4.556244in}{2.937880in}}%
\pgfpathlineto{\pgfqpoint{4.569262in}{2.934880in}}%
\pgfpathlineto{\pgfqpoint{4.576589in}{2.947963in}}%
\pgfpathlineto{\pgfqpoint{4.583915in}{2.961234in}}%
\pgfpathlineto{\pgfqpoint{4.591239in}{2.974698in}}%
\pgfpathlineto{\pgfqpoint{4.598561in}{2.988361in}}%
\pgfpathlineto{\pgfqpoint{4.585554in}{2.991855in}}%
\pgfpathlineto{\pgfqpoint{4.572553in}{2.995515in}}%
\pgfpathlineto{\pgfqpoint{4.559561in}{2.999344in}}%
\pgfpathlineto{\pgfqpoint{4.546575in}{3.003341in}}%
\pgfpathlineto{\pgfqpoint{4.539242in}{2.989174in}}%
\pgfpathlineto{\pgfqpoint{4.531907in}{2.975214in}}%
\pgfpathlineto{\pgfqpoint{4.524571in}{2.961453in}}%
\pgfpathlineto{\pgfqpoint{4.517233in}{2.947887in}}%
\pgfpathclose%
\pgfusepath{fill}%
\end{pgfscope}%
\begin{pgfscope}%
\pgfpathrectangle{\pgfqpoint{1.254980in}{0.150000in}}{\pgfqpoint{5.490039in}{5.490039in}}%
\pgfusepath{clip}%
\pgfsetbuttcap%
\pgfsetroundjoin%
\definecolor{currentfill}{rgb}{0.210503,0.363727,0.552206}%
\pgfsetfillcolor{currentfill}%
\pgfsetfillopacity{0.700000}%
\pgfsetlinewidth{0.000000pt}%
\definecolor{currentstroke}{rgb}{0.000000,0.000000,0.000000}%
\pgfsetstrokecolor{currentstroke}%
\pgfsetdash{}{0pt}%
\pgfpathmoveto{\pgfqpoint{3.239031in}{3.071920in}}%
\pgfpathlineto{\pgfqpoint{3.251945in}{3.056210in}}%
\pgfpathlineto{\pgfqpoint{3.264854in}{3.040761in}}%
\pgfpathlineto{\pgfqpoint{3.277759in}{3.025573in}}%
\pgfpathlineto{\pgfqpoint{3.290660in}{3.010643in}}%
\pgfpathlineto{\pgfqpoint{3.298296in}{3.024028in}}%
\pgfpathlineto{\pgfqpoint{3.305925in}{3.037564in}}%
\pgfpathlineto{\pgfqpoint{3.313548in}{3.051254in}}%
\pgfpathlineto{\pgfqpoint{3.321166in}{3.065100in}}%
\pgfpathlineto{\pgfqpoint{3.308273in}{3.080253in}}%
\pgfpathlineto{\pgfqpoint{3.295377in}{3.095663in}}%
\pgfpathlineto{\pgfqpoint{3.282476in}{3.111333in}}%
\pgfpathlineto{\pgfqpoint{3.269571in}{3.127267in}}%
\pgfpathlineto{\pgfqpoint{3.261945in}{3.113187in}}%
\pgfpathlineto{\pgfqpoint{3.254313in}{3.099272in}}%
\pgfpathlineto{\pgfqpoint{3.246675in}{3.085517in}}%
\pgfpathlineto{\pgfqpoint{3.239031in}{3.071920in}}%
\pgfpathclose%
\pgfusepath{fill}%
\end{pgfscope}%
\begin{pgfscope}%
\pgfpathrectangle{\pgfqpoint{1.254980in}{0.150000in}}{\pgfqpoint{5.490039in}{5.490039in}}%
\pgfusepath{clip}%
\pgfsetbuttcap%
\pgfsetroundjoin%
\definecolor{currentfill}{rgb}{0.151918,0.500685,0.557587}%
\pgfsetfillcolor{currentfill}%
\pgfsetfillopacity{0.700000}%
\pgfsetlinewidth{0.000000pt}%
\definecolor{currentstroke}{rgb}{0.000000,0.000000,0.000000}%
\pgfsetstrokecolor{currentstroke}%
\pgfsetdash{}{0pt}%
\pgfpathmoveto{\pgfqpoint{3.062283in}{3.419912in}}%
\pgfpathlineto{\pgfqpoint{3.075293in}{3.399435in}}%
\pgfpathlineto{\pgfqpoint{3.088295in}{3.379263in}}%
\pgfpathlineto{\pgfqpoint{3.101288in}{3.359393in}}%
\pgfpathlineto{\pgfqpoint{3.114273in}{3.339822in}}%
\pgfpathlineto{\pgfqpoint{3.121919in}{3.354793in}}%
\pgfpathlineto{\pgfqpoint{3.129557in}{3.369955in}}%
\pgfpathlineto{\pgfqpoint{3.137189in}{3.385312in}}%
\pgfpathlineto{\pgfqpoint{3.144814in}{3.400865in}}%
\pgfpathlineto{\pgfqpoint{3.131837in}{3.420691in}}%
\pgfpathlineto{\pgfqpoint{3.118853in}{3.440817in}}%
\pgfpathlineto{\pgfqpoint{3.105860in}{3.461244in}}%
\pgfpathlineto{\pgfqpoint{3.092858in}{3.481977in}}%
\pgfpathlineto{\pgfqpoint{3.085225in}{3.466156in}}%
\pgfpathlineto{\pgfqpoint{3.077585in}{3.450541in}}%
\pgfpathlineto{\pgfqpoint{3.069937in}{3.435127in}}%
\pgfpathlineto{\pgfqpoint{3.062283in}{3.419912in}}%
\pgfpathclose%
\pgfusepath{fill}%
\end{pgfscope}%
\begin{pgfscope}%
\pgfpathrectangle{\pgfqpoint{1.254980in}{0.150000in}}{\pgfqpoint{5.490039in}{5.490039in}}%
\pgfusepath{clip}%
\pgfsetbuttcap%
\pgfsetroundjoin%
\definecolor{currentfill}{rgb}{0.260571,0.246922,0.522828}%
\pgfsetfillcolor{currentfill}%
\pgfsetfillopacity{0.700000}%
\pgfsetlinewidth{0.000000pt}%
\definecolor{currentstroke}{rgb}{0.000000,0.000000,0.000000}%
\pgfsetstrokecolor{currentstroke}%
\pgfsetdash{}{0pt}%
\pgfpathmoveto{\pgfqpoint{4.058854in}{2.796246in}}%
\pgfpathlineto{\pgfqpoint{4.071750in}{2.790550in}}%
\pgfpathlineto{\pgfqpoint{4.084651in}{2.785040in}}%
\pgfpathlineto{\pgfqpoint{4.097557in}{2.779715in}}%
\pgfpathlineto{\pgfqpoint{4.110468in}{2.774574in}}%
\pgfpathlineto{\pgfqpoint{4.117908in}{2.787068in}}%
\pgfpathlineto{\pgfqpoint{4.125344in}{2.799692in}}%
\pgfpathlineto{\pgfqpoint{4.132777in}{2.812450in}}%
\pgfpathlineto{\pgfqpoint{4.140206in}{2.825346in}}%
\pgfpathlineto{\pgfqpoint{4.127303in}{2.830843in}}%
\pgfpathlineto{\pgfqpoint{4.114405in}{2.836524in}}%
\pgfpathlineto{\pgfqpoint{4.101512in}{2.842390in}}%
\pgfpathlineto{\pgfqpoint{4.088623in}{2.848442in}}%
\pgfpathlineto{\pgfqpoint{4.081187in}{2.835179in}}%
\pgfpathlineto{\pgfqpoint{4.073746in}{2.822062in}}%
\pgfpathlineto{\pgfqpoint{4.066302in}{2.809085in}}%
\pgfpathlineto{\pgfqpoint{4.058854in}{2.796246in}}%
\pgfpathclose%
\pgfusepath{fill}%
\end{pgfscope}%
\begin{pgfscope}%
\pgfpathrectangle{\pgfqpoint{1.254980in}{0.150000in}}{\pgfqpoint{5.490039in}{5.490039in}}%
\pgfusepath{clip}%
\pgfsetbuttcap%
\pgfsetroundjoin%
\definecolor{currentfill}{rgb}{0.263663,0.237631,0.518762}%
\pgfsetfillcolor{currentfill}%
\pgfsetfillopacity{0.700000}%
\pgfsetlinewidth{0.000000pt}%
\definecolor{currentstroke}{rgb}{0.000000,0.000000,0.000000}%
\pgfsetstrokecolor{currentstroke}%
\pgfsetdash{}{0pt}%
\pgfpathmoveto{\pgfqpoint{3.844504in}{2.773614in}}%
\pgfpathlineto{\pgfqpoint{3.857370in}{2.766242in}}%
\pgfpathlineto{\pgfqpoint{3.870240in}{2.759068in}}%
\pgfpathlineto{\pgfqpoint{3.883112in}{2.752091in}}%
\pgfpathlineto{\pgfqpoint{3.895987in}{2.745310in}}%
\pgfpathlineto{\pgfqpoint{3.903483in}{2.757778in}}%
\pgfpathlineto{\pgfqpoint{3.910974in}{2.770367in}}%
\pgfpathlineto{\pgfqpoint{3.918461in}{2.783079in}}%
\pgfpathlineto{\pgfqpoint{3.925944in}{2.795918in}}%
\pgfpathlineto{\pgfqpoint{3.913076in}{2.803000in}}%
\pgfpathlineto{\pgfqpoint{3.900211in}{2.810279in}}%
\pgfpathlineto{\pgfqpoint{3.887350in}{2.817754in}}%
\pgfpathlineto{\pgfqpoint{3.874491in}{2.825427in}}%
\pgfpathlineto{\pgfqpoint{3.867001in}{2.812277in}}%
\pgfpathlineto{\pgfqpoint{3.859506in}{2.799261in}}%
\pgfpathlineto{\pgfqpoint{3.852008in}{2.786374in}}%
\pgfpathlineto{\pgfqpoint{3.844504in}{2.773614in}}%
\pgfpathclose%
\pgfusepath{fill}%
\end{pgfscope}%
\begin{pgfscope}%
\pgfpathrectangle{\pgfqpoint{1.254980in}{0.150000in}}{\pgfqpoint{5.490039in}{5.490039in}}%
\pgfusepath{clip}%
\pgfsetbuttcap%
\pgfsetroundjoin%
\definecolor{currentfill}{rgb}{0.195860,0.395433,0.555276}%
\pgfsetfillcolor{currentfill}%
\pgfsetfillopacity{0.700000}%
\pgfsetlinewidth{0.000000pt}%
\definecolor{currentstroke}{rgb}{0.000000,0.000000,0.000000}%
\pgfsetstrokecolor{currentstroke}%
\pgfsetdash{}{0pt}%
\pgfpathmoveto{\pgfqpoint{4.842656in}{3.121236in}}%
\pgfpathlineto{\pgfqpoint{4.855730in}{3.118222in}}%
\pgfpathlineto{\pgfqpoint{4.868814in}{3.115367in}}%
\pgfpathlineto{\pgfqpoint{4.881906in}{3.112673in}}%
\pgfpathlineto{\pgfqpoint{4.895006in}{3.110138in}}%
\pgfpathlineto{\pgfqpoint{4.902275in}{3.124579in}}%
\pgfpathlineto{\pgfqpoint{4.909546in}{3.139281in}}%
\pgfpathlineto{\pgfqpoint{4.916817in}{3.154251in}}%
\pgfpathlineto{\pgfqpoint{4.924090in}{3.169497in}}%
\pgfpathlineto{\pgfqpoint{4.911004in}{3.172636in}}%
\pgfpathlineto{\pgfqpoint{4.897926in}{3.175934in}}%
\pgfpathlineto{\pgfqpoint{4.884856in}{3.179392in}}%
\pgfpathlineto{\pgfqpoint{4.871795in}{3.183009in}}%
\pgfpathlineto{\pgfqpoint{4.864508in}{3.167150in}}%
\pgfpathlineto{\pgfqpoint{4.857223in}{3.151573in}}%
\pgfpathlineto{\pgfqpoint{4.849939in}{3.136270in}}%
\pgfpathlineto{\pgfqpoint{4.842656in}{3.121236in}}%
\pgfpathclose%
\pgfusepath{fill}%
\end{pgfscope}%
\begin{pgfscope}%
\pgfpathrectangle{\pgfqpoint{1.254980in}{0.150000in}}{\pgfqpoint{5.490039in}{5.490039in}}%
\pgfusepath{clip}%
\pgfsetbuttcap%
\pgfsetroundjoin%
\definecolor{currentfill}{rgb}{0.233603,0.313828,0.543914}%
\pgfsetfillcolor{currentfill}%
\pgfsetfillopacity{0.700000}%
\pgfsetlinewidth{0.000000pt}%
\definecolor{currentstroke}{rgb}{0.000000,0.000000,0.000000}%
\pgfsetstrokecolor{currentstroke}%
\pgfsetdash{}{0pt}%
\pgfpathmoveto{\pgfqpoint{3.342228in}{2.953457in}}%
\pgfpathlineto{\pgfqpoint{3.355112in}{2.939784in}}%
\pgfpathlineto{\pgfqpoint{3.367994in}{2.926357in}}%
\pgfpathlineto{\pgfqpoint{3.380872in}{2.913173in}}%
\pgfpathlineto{\pgfqpoint{3.393749in}{2.900230in}}%
\pgfpathlineto{\pgfqpoint{3.401366in}{2.913194in}}%
\pgfpathlineto{\pgfqpoint{3.408978in}{2.926295in}}%
\pgfpathlineto{\pgfqpoint{3.416585in}{2.939536in}}%
\pgfpathlineto{\pgfqpoint{3.424186in}{2.952919in}}%
\pgfpathlineto{\pgfqpoint{3.411318in}{2.966082in}}%
\pgfpathlineto{\pgfqpoint{3.398447in}{2.979487in}}%
\pgfpathlineto{\pgfqpoint{3.385575in}{2.993135in}}%
\pgfpathlineto{\pgfqpoint{3.372699in}{3.007029in}}%
\pgfpathlineto{\pgfqpoint{3.365090in}{2.993414in}}%
\pgfpathlineto{\pgfqpoint{3.357475in}{2.979949in}}%
\pgfpathlineto{\pgfqpoint{3.349855in}{2.966631in}}%
\pgfpathlineto{\pgfqpoint{3.342228in}{2.953457in}}%
\pgfpathclose%
\pgfusepath{fill}%
\end{pgfscope}%
\begin{pgfscope}%
\pgfpathrectangle{\pgfqpoint{1.254980in}{0.150000in}}{\pgfqpoint{5.490039in}{5.490039in}}%
\pgfusepath{clip}%
\pgfsetbuttcap%
\pgfsetroundjoin%
\definecolor{currentfill}{rgb}{0.237441,0.305202,0.541921}%
\pgfsetfillcolor{currentfill}%
\pgfsetfillopacity{0.700000}%
\pgfsetlinewidth{0.000000pt}%
\definecolor{currentstroke}{rgb}{0.000000,0.000000,0.000000}%
\pgfsetstrokecolor{currentstroke}%
\pgfsetdash{}{0pt}%
\pgfpathmoveto{\pgfqpoint{4.435909in}{2.909301in}}%
\pgfpathlineto{\pgfqpoint{4.448887in}{2.905583in}}%
\pgfpathlineto{\pgfqpoint{4.461872in}{2.902036in}}%
\pgfpathlineto{\pgfqpoint{4.474864in}{2.898660in}}%
\pgfpathlineto{\pgfqpoint{4.487863in}{2.895453in}}%
\pgfpathlineto{\pgfqpoint{4.495209in}{2.908298in}}%
\pgfpathlineto{\pgfqpoint{4.502552in}{2.921315in}}%
\pgfpathlineto{\pgfqpoint{4.509894in}{2.934509in}}%
\pgfpathlineto{\pgfqpoint{4.517233in}{2.947887in}}%
\pgfpathlineto{\pgfqpoint{4.504245in}{2.951560in}}%
\pgfpathlineto{\pgfqpoint{4.491263in}{2.955402in}}%
\pgfpathlineto{\pgfqpoint{4.478288in}{2.959415in}}%
\pgfpathlineto{\pgfqpoint{4.465320in}{2.963598in}}%
\pgfpathlineto{\pgfqpoint{4.457970in}{2.949744in}}%
\pgfpathlineto{\pgfqpoint{4.450619in}{2.936080in}}%
\pgfpathlineto{\pgfqpoint{4.443265in}{2.922601in}}%
\pgfpathlineto{\pgfqpoint{4.435909in}{2.909301in}}%
\pgfpathclose%
\pgfusepath{fill}%
\end{pgfscope}%
\begin{pgfscope}%
\pgfpathrectangle{\pgfqpoint{1.254980in}{0.150000in}}{\pgfqpoint{5.490039in}{5.490039in}}%
\pgfusepath{clip}%
\pgfsetbuttcap%
\pgfsetroundjoin%
\definecolor{currentfill}{rgb}{0.197636,0.391528,0.554969}%
\pgfsetfillcolor{currentfill}%
\pgfsetfillopacity{0.700000}%
\pgfsetlinewidth{0.000000pt}%
\definecolor{currentstroke}{rgb}{0.000000,0.000000,0.000000}%
\pgfsetstrokecolor{currentstroke}%
\pgfsetdash{}{0pt}%
\pgfpathmoveto{\pgfqpoint{3.187325in}{3.137433in}}%
\pgfpathlineto{\pgfqpoint{3.200259in}{3.120650in}}%
\pgfpathlineto{\pgfqpoint{3.213188in}{3.104138in}}%
\pgfpathlineto{\pgfqpoint{3.226112in}{3.087896in}}%
\pgfpathlineto{\pgfqpoint{3.239031in}{3.071920in}}%
\pgfpathlineto{\pgfqpoint{3.246675in}{3.085517in}}%
\pgfpathlineto{\pgfqpoint{3.254313in}{3.099272in}}%
\pgfpathlineto{\pgfqpoint{3.261945in}{3.113187in}}%
\pgfpathlineto{\pgfqpoint{3.269571in}{3.127267in}}%
\pgfpathlineto{\pgfqpoint{3.256661in}{3.143464in}}%
\pgfpathlineto{\pgfqpoint{3.243746in}{3.159930in}}%
\pgfpathlineto{\pgfqpoint{3.230826in}{3.176664in}}%
\pgfpathlineto{\pgfqpoint{3.217901in}{3.193671in}}%
\pgfpathlineto{\pgfqpoint{3.210267in}{3.179358in}}%
\pgfpathlineto{\pgfqpoint{3.202626in}{3.165216in}}%
\pgfpathlineto{\pgfqpoint{3.194979in}{3.151242in}}%
\pgfpathlineto{\pgfqpoint{3.187325in}{3.137433in}}%
\pgfpathclose%
\pgfusepath{fill}%
\end{pgfscope}%
\begin{pgfscope}%
\pgfpathrectangle{\pgfqpoint{1.254980in}{0.150000in}}{\pgfqpoint{5.490039in}{5.490039in}}%
\pgfusepath{clip}%
\pgfsetbuttcap%
\pgfsetroundjoin%
\definecolor{currentfill}{rgb}{0.258965,0.251537,0.524736}%
\pgfsetfillcolor{currentfill}%
\pgfsetfillopacity{0.700000}%
\pgfsetlinewidth{0.000000pt}%
\definecolor{currentstroke}{rgb}{0.000000,0.000000,0.000000}%
\pgfsetstrokecolor{currentstroke}%
\pgfsetdash{}{0pt}%
\pgfpathmoveto{\pgfqpoint{3.578491in}{2.813159in}}%
\pgfpathlineto{\pgfqpoint{3.591346in}{2.802978in}}%
\pgfpathlineto{\pgfqpoint{3.604202in}{2.793016in}}%
\pgfpathlineto{\pgfqpoint{3.617058in}{2.783270in}}%
\pgfpathlineto{\pgfqpoint{3.629915in}{2.773739in}}%
\pgfpathlineto{\pgfqpoint{3.637479in}{2.786318in}}%
\pgfpathlineto{\pgfqpoint{3.645037in}{2.799018in}}%
\pgfpathlineto{\pgfqpoint{3.652591in}{2.811842in}}%
\pgfpathlineto{\pgfqpoint{3.660139in}{2.824793in}}%
\pgfpathlineto{\pgfqpoint{3.647290in}{2.834571in}}%
\pgfpathlineto{\pgfqpoint{3.634442in}{2.844564in}}%
\pgfpathlineto{\pgfqpoint{3.621593in}{2.854774in}}%
\pgfpathlineto{\pgfqpoint{3.608746in}{2.865202in}}%
\pgfpathlineto{\pgfqpoint{3.601190in}{2.851994in}}%
\pgfpathlineto{\pgfqpoint{3.593628in}{2.838920in}}%
\pgfpathlineto{\pgfqpoint{3.586062in}{2.825976in}}%
\pgfpathlineto{\pgfqpoint{3.578491in}{2.813159in}}%
\pgfpathclose%
\pgfusepath{fill}%
\end{pgfscope}%
\begin{pgfscope}%
\pgfpathrectangle{\pgfqpoint{1.254980in}{0.150000in}}{\pgfqpoint{5.490039in}{5.490039in}}%
\pgfusepath{clip}%
\pgfsetbuttcap%
\pgfsetroundjoin%
\definecolor{currentfill}{rgb}{0.185556,0.418570,0.556753}%
\pgfsetfillcolor{currentfill}%
\pgfsetfillopacity{0.700000}%
\pgfsetlinewidth{0.000000pt}%
\definecolor{currentstroke}{rgb}{0.000000,0.000000,0.000000}%
\pgfsetstrokecolor{currentstroke}%
\pgfsetdash{}{0pt}%
\pgfpathmoveto{\pgfqpoint{4.924090in}{3.169497in}}%
\pgfpathlineto{\pgfqpoint{4.937185in}{3.166516in}}%
\pgfpathlineto{\pgfqpoint{4.950289in}{3.163694in}}%
\pgfpathlineto{\pgfqpoint{4.963402in}{3.161029in}}%
\pgfpathlineto{\pgfqpoint{4.976524in}{3.158522in}}%
\pgfpathlineto{\pgfqpoint{4.983784in}{3.173429in}}%
\pgfpathlineto{\pgfqpoint{4.991046in}{3.188618in}}%
\pgfpathlineto{\pgfqpoint{4.998311in}{3.204097in}}%
\pgfpathlineto{\pgfqpoint{5.005578in}{3.219874in}}%
\pgfpathlineto{\pgfqpoint{4.992471in}{3.223012in}}%
\pgfpathlineto{\pgfqpoint{4.979373in}{3.226308in}}%
\pgfpathlineto{\pgfqpoint{4.966284in}{3.229762in}}%
\pgfpathlineto{\pgfqpoint{4.953204in}{3.233374in}}%
\pgfpathlineto{\pgfqpoint{4.945922in}{3.216955in}}%
\pgfpathlineto{\pgfqpoint{4.938642in}{3.200841in}}%
\pgfpathlineto{\pgfqpoint{4.931365in}{3.185024in}}%
\pgfpathlineto{\pgfqpoint{4.924090in}{3.169497in}}%
\pgfpathclose%
\pgfusepath{fill}%
\end{pgfscope}%
\begin{pgfscope}%
\pgfpathrectangle{\pgfqpoint{1.254980in}{0.150000in}}{\pgfqpoint{5.490039in}{5.490039in}}%
\pgfusepath{clip}%
\pgfsetbuttcap%
\pgfsetroundjoin%
\definecolor{currentfill}{rgb}{0.244972,0.287675,0.537260}%
\pgfsetfillcolor{currentfill}%
\pgfsetfillopacity{0.700000}%
\pgfsetlinewidth{0.000000pt}%
\definecolor{currentstroke}{rgb}{0.000000,0.000000,0.000000}%
\pgfsetstrokecolor{currentstroke}%
\pgfsetdash{}{0pt}%
\pgfpathmoveto{\pgfqpoint{4.354580in}{2.872623in}}%
\pgfpathlineto{\pgfqpoint{4.367540in}{2.868654in}}%
\pgfpathlineto{\pgfqpoint{4.380507in}{2.864858in}}%
\pgfpathlineto{\pgfqpoint{4.393481in}{2.861235in}}%
\pgfpathlineto{\pgfqpoint{4.406461in}{2.857785in}}%
\pgfpathlineto{\pgfqpoint{4.413827in}{2.870421in}}%
\pgfpathlineto{\pgfqpoint{4.421190in}{2.883216in}}%
\pgfpathlineto{\pgfqpoint{4.428551in}{2.896174in}}%
\pgfpathlineto{\pgfqpoint{4.435909in}{2.909301in}}%
\pgfpathlineto{\pgfqpoint{4.422938in}{2.913190in}}%
\pgfpathlineto{\pgfqpoint{4.409974in}{2.917251in}}%
\pgfpathlineto{\pgfqpoint{4.397016in}{2.921485in}}%
\pgfpathlineto{\pgfqpoint{4.384065in}{2.925893in}}%
\pgfpathlineto{\pgfqpoint{4.376698in}{2.912318in}}%
\pgfpathlineto{\pgfqpoint{4.369328in}{2.898918in}}%
\pgfpathlineto{\pgfqpoint{4.361955in}{2.885688in}}%
\pgfpathlineto{\pgfqpoint{4.354580in}{2.872623in}}%
\pgfpathclose%
\pgfusepath{fill}%
\end{pgfscope}%
\begin{pgfscope}%
\pgfpathrectangle{\pgfqpoint{1.254980in}{0.150000in}}{\pgfqpoint{5.490039in}{5.490039in}}%
\pgfusepath{clip}%
\pgfsetbuttcap%
\pgfsetroundjoin%
\definecolor{currentfill}{rgb}{0.243113,0.292092,0.538516}%
\pgfsetfillcolor{currentfill}%
\pgfsetfillopacity{0.700000}%
\pgfsetlinewidth{0.000000pt}%
\definecolor{currentstroke}{rgb}{0.000000,0.000000,0.000000}%
\pgfsetstrokecolor{currentstroke}%
\pgfsetdash{}{0pt}%
\pgfpathmoveto{\pgfqpoint{3.393749in}{2.900230in}}%
\pgfpathlineto{\pgfqpoint{3.406623in}{2.887528in}}%
\pgfpathlineto{\pgfqpoint{3.419495in}{2.875063in}}%
\pgfpathlineto{\pgfqpoint{3.432366in}{2.862834in}}%
\pgfpathlineto{\pgfqpoint{3.445235in}{2.850840in}}%
\pgfpathlineto{\pgfqpoint{3.452844in}{2.863593in}}%
\pgfpathlineto{\pgfqpoint{3.460447in}{2.876477in}}%
\pgfpathlineto{\pgfqpoint{3.468045in}{2.889494in}}%
\pgfpathlineto{\pgfqpoint{3.475638in}{2.902646in}}%
\pgfpathlineto{\pgfqpoint{3.462777in}{2.914861in}}%
\pgfpathlineto{\pgfqpoint{3.449915in}{2.927310in}}%
\pgfpathlineto{\pgfqpoint{3.437051in}{2.939995in}}%
\pgfpathlineto{\pgfqpoint{3.424186in}{2.952919in}}%
\pgfpathlineto{\pgfqpoint{3.416585in}{2.939536in}}%
\pgfpathlineto{\pgfqpoint{3.408978in}{2.926295in}}%
\pgfpathlineto{\pgfqpoint{3.401366in}{2.913194in}}%
\pgfpathlineto{\pgfqpoint{3.393749in}{2.900230in}}%
\pgfpathclose%
\pgfusepath{fill}%
\end{pgfscope}%
\begin{pgfscope}%
\pgfpathrectangle{\pgfqpoint{1.254980in}{0.150000in}}{\pgfqpoint{5.490039in}{5.490039in}}%
\pgfusepath{clip}%
\pgfsetbuttcap%
\pgfsetroundjoin%
\definecolor{currentfill}{rgb}{0.185556,0.418570,0.556753}%
\pgfsetfillcolor{currentfill}%
\pgfsetfillopacity{0.700000}%
\pgfsetlinewidth{0.000000pt}%
\definecolor{currentstroke}{rgb}{0.000000,0.000000,0.000000}%
\pgfsetstrokecolor{currentstroke}%
\pgfsetdash{}{0pt}%
\pgfpathmoveto{\pgfqpoint{3.135528in}{3.207334in}}%
\pgfpathlineto{\pgfqpoint{3.148486in}{3.189439in}}%
\pgfpathlineto{\pgfqpoint{3.161439in}{3.171825in}}%
\pgfpathlineto{\pgfqpoint{3.174385in}{3.154491in}}%
\pgfpathlineto{\pgfqpoint{3.187325in}{3.137433in}}%
\pgfpathlineto{\pgfqpoint{3.194979in}{3.151242in}}%
\pgfpathlineto{\pgfqpoint{3.202626in}{3.165216in}}%
\pgfpathlineto{\pgfqpoint{3.210267in}{3.179358in}}%
\pgfpathlineto{\pgfqpoint{3.217901in}{3.193671in}}%
\pgfpathlineto{\pgfqpoint{3.204970in}{3.210952in}}%
\pgfpathlineto{\pgfqpoint{3.192033in}{3.228510in}}%
\pgfpathlineto{\pgfqpoint{3.179090in}{3.246348in}}%
\pgfpathlineto{\pgfqpoint{3.166140in}{3.264467in}}%
\pgfpathlineto{\pgfqpoint{3.158497in}{3.249920in}}%
\pgfpathlineto{\pgfqpoint{3.150847in}{3.235551in}}%
\pgfpathlineto{\pgfqpoint{3.143191in}{3.221356in}}%
\pgfpathlineto{\pgfqpoint{3.135528in}{3.207334in}}%
\pgfpathclose%
\pgfusepath{fill}%
\end{pgfscope}%
\begin{pgfscope}%
\pgfpathrectangle{\pgfqpoint{1.254980in}{0.150000in}}{\pgfqpoint{5.490039in}{5.490039in}}%
\pgfusepath{clip}%
\pgfsetbuttcap%
\pgfsetroundjoin%
\definecolor{currentfill}{rgb}{0.263663,0.237631,0.518762}%
\pgfsetfillcolor{currentfill}%
\pgfsetfillopacity{0.700000}%
\pgfsetlinewidth{0.000000pt}%
\definecolor{currentstroke}{rgb}{0.000000,0.000000,0.000000}%
\pgfsetstrokecolor{currentstroke}%
\pgfsetdash{}{0pt}%
\pgfpathmoveto{\pgfqpoint{3.977452in}{2.769522in}}%
\pgfpathlineto{\pgfqpoint{3.990339in}{2.763402in}}%
\pgfpathlineto{\pgfqpoint{4.003229in}{2.757471in}}%
\pgfpathlineto{\pgfqpoint{4.016124in}{2.751729in}}%
\pgfpathlineto{\pgfqpoint{4.029024in}{2.746175in}}%
\pgfpathlineto{\pgfqpoint{4.036487in}{2.758508in}}%
\pgfpathlineto{\pgfqpoint{4.043947in}{2.770961in}}%
\pgfpathlineto{\pgfqpoint{4.051402in}{2.783539in}}%
\pgfpathlineto{\pgfqpoint{4.058854in}{2.796246in}}%
\pgfpathlineto{\pgfqpoint{4.045962in}{2.802128in}}%
\pgfpathlineto{\pgfqpoint{4.033075in}{2.808199in}}%
\pgfpathlineto{\pgfqpoint{4.020192in}{2.814458in}}%
\pgfpathlineto{\pgfqpoint{4.007313in}{2.820907in}}%
\pgfpathlineto{\pgfqpoint{3.999854in}{2.807861in}}%
\pgfpathlineto{\pgfqpoint{3.992390in}{2.794951in}}%
\pgfpathlineto{\pgfqpoint{3.984923in}{2.782173in}}%
\pgfpathlineto{\pgfqpoint{3.977452in}{2.769522in}}%
\pgfpathclose%
\pgfusepath{fill}%
\end{pgfscope}%
\begin{pgfscope}%
\pgfpathrectangle{\pgfqpoint{1.254980in}{0.150000in}}{\pgfqpoint{5.490039in}{5.490039in}}%
\pgfusepath{clip}%
\pgfsetbuttcap%
\pgfsetroundjoin%
\definecolor{currentfill}{rgb}{0.250425,0.274290,0.533103}%
\pgfsetfillcolor{currentfill}%
\pgfsetfillopacity{0.700000}%
\pgfsetlinewidth{0.000000pt}%
\definecolor{currentstroke}{rgb}{0.000000,0.000000,0.000000}%
\pgfsetstrokecolor{currentstroke}%
\pgfsetdash{}{0pt}%
\pgfpathmoveto{\pgfqpoint{4.273236in}{2.837896in}}%
\pgfpathlineto{\pgfqpoint{4.286180in}{2.833637in}}%
\pgfpathlineto{\pgfqpoint{4.299130in}{2.829554in}}%
\pgfpathlineto{\pgfqpoint{4.312087in}{2.825646in}}%
\pgfpathlineto{\pgfqpoint{4.325050in}{2.821914in}}%
\pgfpathlineto{\pgfqpoint{4.332437in}{2.834368in}}%
\pgfpathlineto{\pgfqpoint{4.339821in}{2.846968in}}%
\pgfpathlineto{\pgfqpoint{4.347202in}{2.859718in}}%
\pgfpathlineto{\pgfqpoint{4.354580in}{2.872623in}}%
\pgfpathlineto{\pgfqpoint{4.341626in}{2.876766in}}%
\pgfpathlineto{\pgfqpoint{4.328678in}{2.881085in}}%
\pgfpathlineto{\pgfqpoint{4.315737in}{2.885579in}}%
\pgfpathlineto{\pgfqpoint{4.302801in}{2.890249in}}%
\pgfpathlineto{\pgfqpoint{4.295414in}{2.876923in}}%
\pgfpathlineto{\pgfqpoint{4.288024in}{2.863758in}}%
\pgfpathlineto{\pgfqpoint{4.280632in}{2.850751in}}%
\pgfpathlineto{\pgfqpoint{4.273236in}{2.837896in}}%
\pgfpathclose%
\pgfusepath{fill}%
\end{pgfscope}%
\begin{pgfscope}%
\pgfpathrectangle{\pgfqpoint{1.254980in}{0.150000in}}{\pgfqpoint{5.490039in}{5.490039in}}%
\pgfusepath{clip}%
\pgfsetbuttcap%
\pgfsetroundjoin%
\definecolor{currentfill}{rgb}{0.177423,0.437527,0.557565}%
\pgfsetfillcolor{currentfill}%
\pgfsetfillopacity{0.700000}%
\pgfsetlinewidth{0.000000pt}%
\definecolor{currentstroke}{rgb}{0.000000,0.000000,0.000000}%
\pgfsetstrokecolor{currentstroke}%
\pgfsetdash{}{0pt}%
\pgfpathmoveto{\pgfqpoint{5.005578in}{3.219874in}}%
\pgfpathlineto{\pgfqpoint{5.018693in}{3.216892in}}%
\pgfpathlineto{\pgfqpoint{5.031817in}{3.214066in}}%
\pgfpathlineto{\pgfqpoint{5.044951in}{3.211397in}}%
\pgfpathlineto{\pgfqpoint{5.058094in}{3.208884in}}%
\pgfpathlineto{\pgfqpoint{5.065348in}{3.224317in}}%
\pgfpathlineto{\pgfqpoint{5.072605in}{3.240055in}}%
\pgfpathlineto{\pgfqpoint{5.079865in}{3.256106in}}%
\pgfpathlineto{\pgfqpoint{5.066735in}{3.259111in}}%
\pgfpathlineto{\pgfqpoint{5.053613in}{3.262272in}}%
\pgfpathlineto{\pgfqpoint{5.040501in}{3.265590in}}%
\pgfpathlineto{\pgfqpoint{5.027397in}{3.269063in}}%
\pgfpathlineto{\pgfqpoint{5.020121in}{3.252349in}}%
\pgfpathlineto{\pgfqpoint{5.012848in}{3.235955in}}%
\pgfpathlineto{\pgfqpoint{5.005578in}{3.219874in}}%
\pgfpathclose%
\pgfusepath{fill}%
\end{pgfscope}%
\begin{pgfscope}%
\pgfpathrectangle{\pgfqpoint{1.254980in}{0.150000in}}{\pgfqpoint{5.490039in}{5.490039in}}%
\pgfusepath{clip}%
\pgfsetbuttcap%
\pgfsetroundjoin%
\definecolor{currentfill}{rgb}{0.250425,0.274290,0.533103}%
\pgfsetfillcolor{currentfill}%
\pgfsetfillopacity{0.700000}%
\pgfsetlinewidth{0.000000pt}%
\definecolor{currentstroke}{rgb}{0.000000,0.000000,0.000000}%
\pgfsetstrokecolor{currentstroke}%
\pgfsetdash{}{0pt}%
\pgfpathmoveto{\pgfqpoint{3.445235in}{2.850840in}}%
\pgfpathlineto{\pgfqpoint{3.458102in}{2.839077in}}%
\pgfpathlineto{\pgfqpoint{3.470969in}{2.827546in}}%
\pgfpathlineto{\pgfqpoint{3.483834in}{2.816244in}}%
\pgfpathlineto{\pgfqpoint{3.496699in}{2.805169in}}%
\pgfpathlineto{\pgfqpoint{3.504299in}{2.817713in}}%
\pgfpathlineto{\pgfqpoint{3.511894in}{2.830380in}}%
\pgfpathlineto{\pgfqpoint{3.519484in}{2.843174in}}%
\pgfpathlineto{\pgfqpoint{3.527069in}{2.856095in}}%
\pgfpathlineto{\pgfqpoint{3.514212in}{2.867390in}}%
\pgfpathlineto{\pgfqpoint{3.501355in}{2.878912in}}%
\pgfpathlineto{\pgfqpoint{3.488497in}{2.890664in}}%
\pgfpathlineto{\pgfqpoint{3.475638in}{2.902646in}}%
\pgfpathlineto{\pgfqpoint{3.468045in}{2.889494in}}%
\pgfpathlineto{\pgfqpoint{3.460447in}{2.876477in}}%
\pgfpathlineto{\pgfqpoint{3.452844in}{2.863593in}}%
\pgfpathlineto{\pgfqpoint{3.445235in}{2.850840in}}%
\pgfpathclose%
\pgfusepath{fill}%
\end{pgfscope}%
\begin{pgfscope}%
\pgfpathrectangle{\pgfqpoint{1.254980in}{0.150000in}}{\pgfqpoint{5.490039in}{5.490039in}}%
\pgfusepath{clip}%
\pgfsetbuttcap%
\pgfsetroundjoin%
\definecolor{currentfill}{rgb}{0.266580,0.228262,0.514349}%
\pgfsetfillcolor{currentfill}%
\pgfsetfillopacity{0.700000}%
\pgfsetlinewidth{0.000000pt}%
\definecolor{currentstroke}{rgb}{0.000000,0.000000,0.000000}%
\pgfsetstrokecolor{currentstroke}%
\pgfsetdash{}{0pt}%
\pgfpathmoveto{\pgfqpoint{3.762979in}{2.754159in}}%
\pgfpathlineto{\pgfqpoint{3.775842in}{2.746259in}}%
\pgfpathlineto{\pgfqpoint{3.788708in}{2.738561in}}%
\pgfpathlineto{\pgfqpoint{3.801576in}{2.731065in}}%
\pgfpathlineto{\pgfqpoint{3.814447in}{2.723768in}}%
\pgfpathlineto{\pgfqpoint{3.821968in}{2.736057in}}%
\pgfpathlineto{\pgfqpoint{3.829485in}{2.748459in}}%
\pgfpathlineto{\pgfqpoint{3.836997in}{2.760977in}}%
\pgfpathlineto{\pgfqpoint{3.844504in}{2.773614in}}%
\pgfpathlineto{\pgfqpoint{3.831641in}{2.781184in}}%
\pgfpathlineto{\pgfqpoint{3.818781in}{2.788955in}}%
\pgfpathlineto{\pgfqpoint{3.805922in}{2.796927in}}%
\pgfpathlineto{\pgfqpoint{3.793067in}{2.805101in}}%
\pgfpathlineto{\pgfqpoint{3.785552in}{2.792180in}}%
\pgfpathlineto{\pgfqpoint{3.778032in}{2.779384in}}%
\pgfpathlineto{\pgfqpoint{3.770508in}{2.766712in}}%
\pgfpathlineto{\pgfqpoint{3.762979in}{2.754159in}}%
\pgfpathclose%
\pgfusepath{fill}%
\end{pgfscope}%
\begin{pgfscope}%
\pgfpathrectangle{\pgfqpoint{1.254980in}{0.150000in}}{\pgfqpoint{5.490039in}{5.490039in}}%
\pgfusepath{clip}%
\pgfsetbuttcap%
\pgfsetroundjoin%
\definecolor{currentfill}{rgb}{0.171176,0.452530,0.557965}%
\pgfsetfillcolor{currentfill}%
\pgfsetfillopacity{0.700000}%
\pgfsetlinewidth{0.000000pt}%
\definecolor{currentstroke}{rgb}{0.000000,0.000000,0.000000}%
\pgfsetstrokecolor{currentstroke}%
\pgfsetdash{}{0pt}%
\pgfpathmoveto{\pgfqpoint{3.083623in}{3.281790in}}%
\pgfpathlineto{\pgfqpoint{3.096610in}{3.262739in}}%
\pgfpathlineto{\pgfqpoint{3.109590in}{3.243982in}}%
\pgfpathlineto{\pgfqpoint{3.122562in}{3.225515in}}%
\pgfpathlineto{\pgfqpoint{3.135528in}{3.207334in}}%
\pgfpathlineto{\pgfqpoint{3.143191in}{3.221356in}}%
\pgfpathlineto{\pgfqpoint{3.150847in}{3.235551in}}%
\pgfpathlineto{\pgfqpoint{3.158497in}{3.249920in}}%
\pgfpathlineto{\pgfqpoint{3.166140in}{3.264467in}}%
\pgfpathlineto{\pgfqpoint{3.153184in}{3.282872in}}%
\pgfpathlineto{\pgfqpoint{3.140221in}{3.301564in}}%
\pgfpathlineto{\pgfqpoint{3.127251in}{3.320546in}}%
\pgfpathlineto{\pgfqpoint{3.114273in}{3.339822in}}%
\pgfpathlineto{\pgfqpoint{3.106621in}{3.325039in}}%
\pgfpathlineto{\pgfqpoint{3.098962in}{3.310441in}}%
\pgfpathlineto{\pgfqpoint{3.091296in}{3.296025in}}%
\pgfpathlineto{\pgfqpoint{3.083623in}{3.281790in}}%
\pgfpathclose%
\pgfusepath{fill}%
\end{pgfscope}%
\begin{pgfscope}%
\pgfpathrectangle{\pgfqpoint{1.254980in}{0.150000in}}{\pgfqpoint{5.490039in}{5.490039in}}%
\pgfusepath{clip}%
\pgfsetbuttcap%
\pgfsetroundjoin%
\definecolor{currentfill}{rgb}{0.263663,0.237631,0.518762}%
\pgfsetfillcolor{currentfill}%
\pgfsetfillopacity{0.700000}%
\pgfsetlinewidth{0.000000pt}%
\definecolor{currentstroke}{rgb}{0.000000,0.000000,0.000000}%
\pgfsetstrokecolor{currentstroke}%
\pgfsetdash{}{0pt}%
\pgfpathmoveto{\pgfqpoint{3.629915in}{2.773739in}}%
\pgfpathlineto{\pgfqpoint{3.642773in}{2.764422in}}%
\pgfpathlineto{\pgfqpoint{3.655632in}{2.755318in}}%
\pgfpathlineto{\pgfqpoint{3.668492in}{2.746424in}}%
\pgfpathlineto{\pgfqpoint{3.681354in}{2.737741in}}%
\pgfpathlineto{\pgfqpoint{3.688909in}{2.750084in}}%
\pgfpathlineto{\pgfqpoint{3.696460in}{2.762540in}}%
\pgfpathlineto{\pgfqpoint{3.704006in}{2.775114in}}%
\pgfpathlineto{\pgfqpoint{3.711547in}{2.787808in}}%
\pgfpathlineto{\pgfqpoint{3.698693in}{2.796738in}}%
\pgfpathlineto{\pgfqpoint{3.685841in}{2.805878in}}%
\pgfpathlineto{\pgfqpoint{3.672989in}{2.815229in}}%
\pgfpathlineto{\pgfqpoint{3.660139in}{2.824793in}}%
\pgfpathlineto{\pgfqpoint{3.652591in}{2.811842in}}%
\pgfpathlineto{\pgfqpoint{3.645037in}{2.799018in}}%
\pgfpathlineto{\pgfqpoint{3.637479in}{2.786318in}}%
\pgfpathlineto{\pgfqpoint{3.629915in}{2.773739in}}%
\pgfpathclose%
\pgfusepath{fill}%
\end{pgfscope}%
\begin{pgfscope}%
\pgfpathrectangle{\pgfqpoint{1.254980in}{0.150000in}}{\pgfqpoint{5.490039in}{5.490039in}}%
\pgfusepath{clip}%
\pgfsetbuttcap%
\pgfsetroundjoin%
\definecolor{currentfill}{rgb}{0.257322,0.256130,0.526563}%
\pgfsetfillcolor{currentfill}%
\pgfsetfillopacity{0.700000}%
\pgfsetlinewidth{0.000000pt}%
\definecolor{currentstroke}{rgb}{0.000000,0.000000,0.000000}%
\pgfsetstrokecolor{currentstroke}%
\pgfsetdash{}{0pt}%
\pgfpathmoveto{\pgfqpoint{4.191868in}{2.805184in}}%
\pgfpathlineto{\pgfqpoint{4.204798in}{2.800596in}}%
\pgfpathlineto{\pgfqpoint{4.217732in}{2.796187in}}%
\pgfpathlineto{\pgfqpoint{4.230673in}{2.791956in}}%
\pgfpathlineto{\pgfqpoint{4.243620in}{2.787904in}}%
\pgfpathlineto{\pgfqpoint{4.251029in}{2.800197in}}%
\pgfpathlineto{\pgfqpoint{4.258435in}{2.812623in}}%
\pgfpathlineto{\pgfqpoint{4.265837in}{2.825188in}}%
\pgfpathlineto{\pgfqpoint{4.273236in}{2.837896in}}%
\pgfpathlineto{\pgfqpoint{4.260298in}{2.842332in}}%
\pgfpathlineto{\pgfqpoint{4.247365in}{2.846946in}}%
\pgfpathlineto{\pgfqpoint{4.234439in}{2.851739in}}%
\pgfpathlineto{\pgfqpoint{4.221518in}{2.856711in}}%
\pgfpathlineto{\pgfqpoint{4.214110in}{2.843609in}}%
\pgfpathlineto{\pgfqpoint{4.206700in}{2.830657in}}%
\pgfpathlineto{\pgfqpoint{4.199286in}{2.817850in}}%
\pgfpathlineto{\pgfqpoint{4.191868in}{2.805184in}}%
\pgfpathclose%
\pgfusepath{fill}%
\end{pgfscope}%
\begin{pgfscope}%
\pgfpathrectangle{\pgfqpoint{1.254980in}{0.150000in}}{\pgfqpoint{5.490039in}{5.490039in}}%
\pgfusepath{clip}%
\pgfsetbuttcap%
\pgfsetroundjoin%
\definecolor{currentfill}{rgb}{0.266580,0.228262,0.514349}%
\pgfsetfillcolor{currentfill}%
\pgfsetfillopacity{0.700000}%
\pgfsetlinewidth{0.000000pt}%
\definecolor{currentstroke}{rgb}{0.000000,0.000000,0.000000}%
\pgfsetstrokecolor{currentstroke}%
\pgfsetdash{}{0pt}%
\pgfpathmoveto{\pgfqpoint{3.895987in}{2.745310in}}%
\pgfpathlineto{\pgfqpoint{3.908866in}{2.738723in}}%
\pgfpathlineto{\pgfqpoint{3.921749in}{2.732329in}}%
\pgfpathlineto{\pgfqpoint{3.934635in}{2.726128in}}%
\pgfpathlineto{\pgfqpoint{3.947525in}{2.720119in}}%
\pgfpathlineto{\pgfqpoint{3.955013in}{2.732297in}}%
\pgfpathlineto{\pgfqpoint{3.962497in}{2.744588in}}%
\pgfpathlineto{\pgfqpoint{3.969977in}{2.756995in}}%
\pgfpathlineto{\pgfqpoint{3.977452in}{2.769522in}}%
\pgfpathlineto{\pgfqpoint{3.964569in}{2.775832in}}%
\pgfpathlineto{\pgfqpoint{3.951691in}{2.782334in}}%
\pgfpathlineto{\pgfqpoint{3.938816in}{2.789029in}}%
\pgfpathlineto{\pgfqpoint{3.925944in}{2.795918in}}%
\pgfpathlineto{\pgfqpoint{3.918461in}{2.783079in}}%
\pgfpathlineto{\pgfqpoint{3.910974in}{2.770367in}}%
\pgfpathlineto{\pgfqpoint{3.903483in}{2.757778in}}%
\pgfpathlineto{\pgfqpoint{3.895987in}{2.745310in}}%
\pgfpathclose%
\pgfusepath{fill}%
\end{pgfscope}%
\begin{pgfscope}%
\pgfpathrectangle{\pgfqpoint{1.254980in}{0.150000in}}{\pgfqpoint{5.490039in}{5.490039in}}%
\pgfusepath{clip}%
\pgfsetbuttcap%
\pgfsetroundjoin%
\definecolor{currentfill}{rgb}{0.258965,0.251537,0.524736}%
\pgfsetfillcolor{currentfill}%
\pgfsetfillopacity{0.700000}%
\pgfsetlinewidth{0.000000pt}%
\definecolor{currentstroke}{rgb}{0.000000,0.000000,0.000000}%
\pgfsetstrokecolor{currentstroke}%
\pgfsetdash{}{0pt}%
\pgfpathmoveto{\pgfqpoint{3.496699in}{2.805169in}}%
\pgfpathlineto{\pgfqpoint{3.509563in}{2.794320in}}%
\pgfpathlineto{\pgfqpoint{3.522427in}{2.783696in}}%
\pgfpathlineto{\pgfqpoint{3.535290in}{2.773294in}}%
\pgfpathlineto{\pgfqpoint{3.548154in}{2.763113in}}%
\pgfpathlineto{\pgfqpoint{3.555746in}{2.775447in}}%
\pgfpathlineto{\pgfqpoint{3.563332in}{2.787898in}}%
\pgfpathlineto{\pgfqpoint{3.570914in}{2.800468in}}%
\pgfpathlineto{\pgfqpoint{3.578491in}{2.813159in}}%
\pgfpathlineto{\pgfqpoint{3.565635in}{2.823560in}}%
\pgfpathlineto{\pgfqpoint{3.552780in}{2.834182in}}%
\pgfpathlineto{\pgfqpoint{3.539925in}{2.845026in}}%
\pgfpathlineto{\pgfqpoint{3.527069in}{2.856095in}}%
\pgfpathlineto{\pgfqpoint{3.519484in}{2.843174in}}%
\pgfpathlineto{\pgfqpoint{3.511894in}{2.830380in}}%
\pgfpathlineto{\pgfqpoint{3.504299in}{2.817713in}}%
\pgfpathlineto{\pgfqpoint{3.496699in}{2.805169in}}%
\pgfpathclose%
\pgfusepath{fill}%
\end{pgfscope}%
\begin{pgfscope}%
\pgfpathrectangle{\pgfqpoint{1.254980in}{0.150000in}}{\pgfqpoint{5.490039in}{5.490039in}}%
\pgfusepath{clip}%
\pgfsetbuttcap%
\pgfsetroundjoin%
\definecolor{currentfill}{rgb}{0.262138,0.242286,0.520837}%
\pgfsetfillcolor{currentfill}%
\pgfsetfillopacity{0.700000}%
\pgfsetlinewidth{0.000000pt}%
\definecolor{currentstroke}{rgb}{0.000000,0.000000,0.000000}%
\pgfsetstrokecolor{currentstroke}%
\pgfsetdash{}{0pt}%
\pgfpathmoveto{\pgfqpoint{4.110468in}{2.774574in}}%
\pgfpathlineto{\pgfqpoint{4.123383in}{2.769617in}}%
\pgfpathlineto{\pgfqpoint{4.136304in}{2.764842in}}%
\pgfpathlineto{\pgfqpoint{4.149231in}{2.760249in}}%
\pgfpathlineto{\pgfqpoint{4.162163in}{2.755837in}}%
\pgfpathlineto{\pgfqpoint{4.169595in}{2.767984in}}%
\pgfpathlineto{\pgfqpoint{4.177023in}{2.780255in}}%
\pgfpathlineto{\pgfqpoint{4.184447in}{2.792654in}}%
\pgfpathlineto{\pgfqpoint{4.191868in}{2.805184in}}%
\pgfpathlineto{\pgfqpoint{4.178945in}{2.809953in}}%
\pgfpathlineto{\pgfqpoint{4.166026in}{2.814902in}}%
\pgfpathlineto{\pgfqpoint{4.153114in}{2.820033in}}%
\pgfpathlineto{\pgfqpoint{4.140206in}{2.825346in}}%
\pgfpathlineto{\pgfqpoint{4.132777in}{2.812450in}}%
\pgfpathlineto{\pgfqpoint{4.125344in}{2.799692in}}%
\pgfpathlineto{\pgfqpoint{4.117908in}{2.787068in}}%
\pgfpathlineto{\pgfqpoint{4.110468in}{2.774574in}}%
\pgfpathclose%
\pgfusepath{fill}%
\end{pgfscope}%
\begin{pgfscope}%
\pgfpathrectangle{\pgfqpoint{1.254980in}{0.150000in}}{\pgfqpoint{5.490039in}{5.490039in}}%
\pgfusepath{clip}%
\pgfsetbuttcap%
\pgfsetroundjoin%
\definecolor{currentfill}{rgb}{0.159194,0.482237,0.558073}%
\pgfsetfillcolor{currentfill}%
\pgfsetfillopacity{0.700000}%
\pgfsetlinewidth{0.000000pt}%
\definecolor{currentstroke}{rgb}{0.000000,0.000000,0.000000}%
\pgfsetstrokecolor{currentstroke}%
\pgfsetdash{}{0pt}%
\pgfpathmoveto{\pgfqpoint{3.031594in}{3.360976in}}%
\pgfpathlineto{\pgfqpoint{3.044614in}{3.340726in}}%
\pgfpathlineto{\pgfqpoint{3.057625in}{3.320780in}}%
\pgfpathlineto{\pgfqpoint{3.070628in}{3.301135in}}%
\pgfpathlineto{\pgfqpoint{3.083623in}{3.281790in}}%
\pgfpathlineto{\pgfqpoint{3.091296in}{3.296025in}}%
\pgfpathlineto{\pgfqpoint{3.098962in}{3.310441in}}%
\pgfpathlineto{\pgfqpoint{3.106621in}{3.325039in}}%
\pgfpathlineto{\pgfqpoint{3.114273in}{3.339822in}}%
\pgfpathlineto{\pgfqpoint{3.101288in}{3.359393in}}%
\pgfpathlineto{\pgfqpoint{3.088295in}{3.379263in}}%
\pgfpathlineto{\pgfqpoint{3.075293in}{3.399435in}}%
\pgfpathlineto{\pgfqpoint{3.062283in}{3.419912in}}%
\pgfpathlineto{\pgfqpoint{3.054622in}{3.404892in}}%
\pgfpathlineto{\pgfqpoint{3.046953in}{3.390065in}}%
\pgfpathlineto{\pgfqpoint{3.039277in}{3.375427in}}%
\pgfpathlineto{\pgfqpoint{3.031594in}{3.360976in}}%
\pgfpathclose%
\pgfusepath{fill}%
\end{pgfscope}%
\begin{pgfscope}%
\pgfpathrectangle{\pgfqpoint{1.254980in}{0.150000in}}{\pgfqpoint{5.490039in}{5.490039in}}%
\pgfusepath{clip}%
\pgfsetbuttcap%
\pgfsetroundjoin%
\definecolor{currentfill}{rgb}{0.221989,0.339161,0.548752}%
\pgfsetfillcolor{currentfill}%
\pgfsetfillopacity{0.700000}%
\pgfsetlinewidth{0.000000pt}%
\definecolor{currentstroke}{rgb}{0.000000,0.000000,0.000000}%
\pgfsetstrokecolor{currentstroke}%
\pgfsetdash{}{0pt}%
\pgfpathmoveto{\pgfqpoint{4.650668in}{2.976047in}}%
\pgfpathlineto{\pgfqpoint{4.663714in}{2.973382in}}%
\pgfpathlineto{\pgfqpoint{4.676768in}{2.970880in}}%
\pgfpathlineto{\pgfqpoint{4.689831in}{2.968542in}}%
\pgfpathlineto{\pgfqpoint{4.702903in}{2.966367in}}%
\pgfpathlineto{\pgfqpoint{4.710200in}{2.979213in}}%
\pgfpathlineto{\pgfqpoint{4.717497in}{2.992256in}}%
\pgfpathlineto{\pgfqpoint{4.724793in}{3.005503in}}%
\pgfpathlineto{\pgfqpoint{4.732088in}{3.018959in}}%
\pgfpathlineto{\pgfqpoint{4.719029in}{3.021655in}}%
\pgfpathlineto{\pgfqpoint{4.705978in}{3.024515in}}%
\pgfpathlineto{\pgfqpoint{4.692936in}{3.027537in}}%
\pgfpathlineto{\pgfqpoint{4.679901in}{3.030724in}}%
\pgfpathlineto{\pgfqpoint{4.672594in}{3.016737in}}%
\pgfpathlineto{\pgfqpoint{4.665286in}{3.002966in}}%
\pgfpathlineto{\pgfqpoint{4.657977in}{2.989404in}}%
\pgfpathlineto{\pgfqpoint{4.650668in}{2.976047in}}%
\pgfpathclose%
\pgfusepath{fill}%
\end{pgfscope}%
\begin{pgfscope}%
\pgfpathrectangle{\pgfqpoint{1.254980in}{0.150000in}}{\pgfqpoint{5.490039in}{5.490039in}}%
\pgfusepath{clip}%
\pgfsetbuttcap%
\pgfsetroundjoin%
\definecolor{currentfill}{rgb}{0.212395,0.359683,0.551710}%
\pgfsetfillcolor{currentfill}%
\pgfsetfillopacity{0.700000}%
\pgfsetlinewidth{0.000000pt}%
\definecolor{currentstroke}{rgb}{0.000000,0.000000,0.000000}%
\pgfsetstrokecolor{currentstroke}%
\pgfsetdash{}{0pt}%
\pgfpathmoveto{\pgfqpoint{4.732088in}{3.018959in}}%
\pgfpathlineto{\pgfqpoint{4.745155in}{3.016425in}}%
\pgfpathlineto{\pgfqpoint{4.758230in}{3.014053in}}%
\pgfpathlineto{\pgfqpoint{4.771315in}{3.011843in}}%
\pgfpathlineto{\pgfqpoint{4.784408in}{3.009794in}}%
\pgfpathlineto{\pgfqpoint{4.791689in}{3.022926in}}%
\pgfpathlineto{\pgfqpoint{4.798970in}{3.036274in}}%
\pgfpathlineto{\pgfqpoint{4.806250in}{3.049843in}}%
\pgfpathlineto{\pgfqpoint{4.813531in}{3.063640in}}%
\pgfpathlineto{\pgfqpoint{4.800451in}{3.066238in}}%
\pgfpathlineto{\pgfqpoint{4.787380in}{3.068997in}}%
\pgfpathlineto{\pgfqpoint{4.774317in}{3.071918in}}%
\pgfpathlineto{\pgfqpoint{4.761263in}{3.075001in}}%
\pgfpathlineto{\pgfqpoint{4.753969in}{3.060645in}}%
\pgfpathlineto{\pgfqpoint{4.746676in}{3.046523in}}%
\pgfpathlineto{\pgfqpoint{4.739382in}{3.032630in}}%
\pgfpathlineto{\pgfqpoint{4.732088in}{3.018959in}}%
\pgfpathclose%
\pgfusepath{fill}%
\end{pgfscope}%
\begin{pgfscope}%
\pgfpathrectangle{\pgfqpoint{1.254980in}{0.150000in}}{\pgfqpoint{5.490039in}{5.490039in}}%
\pgfusepath{clip}%
\pgfsetbuttcap%
\pgfsetroundjoin%
\definecolor{currentfill}{rgb}{0.229739,0.322361,0.545706}%
\pgfsetfillcolor{currentfill}%
\pgfsetfillopacity{0.700000}%
\pgfsetlinewidth{0.000000pt}%
\definecolor{currentstroke}{rgb}{0.000000,0.000000,0.000000}%
\pgfsetstrokecolor{currentstroke}%
\pgfsetdash{}{0pt}%
\pgfpathmoveto{\pgfqpoint{4.569262in}{2.934880in}}%
\pgfpathlineto{\pgfqpoint{4.582288in}{2.932047in}}%
\pgfpathlineto{\pgfqpoint{4.595322in}{2.929379in}}%
\pgfpathlineto{\pgfqpoint{4.608364in}{2.926878in}}%
\pgfpathlineto{\pgfqpoint{4.621414in}{2.924542in}}%
\pgfpathlineto{\pgfqpoint{4.628730in}{2.937141in}}%
\pgfpathlineto{\pgfqpoint{4.636044in}{2.949921in}}%
\pgfpathlineto{\pgfqpoint{4.643356in}{2.962888in}}%
\pgfpathlineto{\pgfqpoint{4.650668in}{2.976047in}}%
\pgfpathlineto{\pgfqpoint{4.637629in}{2.978878in}}%
\pgfpathlineto{\pgfqpoint{4.624599in}{2.981873in}}%
\pgfpathlineto{\pgfqpoint{4.611576in}{2.985034in}}%
\pgfpathlineto{\pgfqpoint{4.598561in}{2.988361in}}%
\pgfpathlineto{\pgfqpoint{4.591239in}{2.974698in}}%
\pgfpathlineto{\pgfqpoint{4.583915in}{2.961234in}}%
\pgfpathlineto{\pgfqpoint{4.576589in}{2.947963in}}%
\pgfpathlineto{\pgfqpoint{4.569262in}{2.934880in}}%
\pgfpathclose%
\pgfusepath{fill}%
\end{pgfscope}%
\begin{pgfscope}%
\pgfpathrectangle{\pgfqpoint{1.254980in}{0.150000in}}{\pgfqpoint{5.490039in}{5.490039in}}%
\pgfusepath{clip}%
\pgfsetbuttcap%
\pgfsetroundjoin%
\definecolor{currentfill}{rgb}{0.203063,0.379716,0.553925}%
\pgfsetfillcolor{currentfill}%
\pgfsetfillopacity{0.700000}%
\pgfsetlinewidth{0.000000pt}%
\definecolor{currentstroke}{rgb}{0.000000,0.000000,0.000000}%
\pgfsetstrokecolor{currentstroke}%
\pgfsetdash{}{0pt}%
\pgfpathmoveto{\pgfqpoint{4.813531in}{3.063640in}}%
\pgfpathlineto{\pgfqpoint{4.826619in}{3.061202in}}%
\pgfpathlineto{\pgfqpoint{4.839716in}{3.058924in}}%
\pgfpathlineto{\pgfqpoint{4.852822in}{3.056807in}}%
\pgfpathlineto{\pgfqpoint{4.865937in}{3.054848in}}%
\pgfpathlineto{\pgfqpoint{4.873204in}{3.068313in}}%
\pgfpathlineto{\pgfqpoint{4.880471in}{3.082012in}}%
\pgfpathlineto{\pgfqpoint{4.887738in}{3.095951in}}%
\pgfpathlineto{\pgfqpoint{4.895006in}{3.110138in}}%
\pgfpathlineto{\pgfqpoint{4.881906in}{3.112673in}}%
\pgfpathlineto{\pgfqpoint{4.868814in}{3.115367in}}%
\pgfpathlineto{\pgfqpoint{4.855730in}{3.118222in}}%
\pgfpathlineto{\pgfqpoint{4.842656in}{3.121236in}}%
\pgfpathlineto{\pgfqpoint{4.835374in}{3.106462in}}%
\pgfpathlineto{\pgfqpoint{4.828092in}{3.091943in}}%
\pgfpathlineto{\pgfqpoint{4.820811in}{3.077671in}}%
\pgfpathlineto{\pgfqpoint{4.813531in}{3.063640in}}%
\pgfpathclose%
\pgfusepath{fill}%
\end{pgfscope}%
\begin{pgfscope}%
\pgfpathrectangle{\pgfqpoint{1.254980in}{0.150000in}}{\pgfqpoint{5.490039in}{5.490039in}}%
\pgfusepath{clip}%
\pgfsetbuttcap%
\pgfsetroundjoin%
\definecolor{currentfill}{rgb}{0.218130,0.347432,0.550038}%
\pgfsetfillcolor{currentfill}%
\pgfsetfillopacity{0.700000}%
\pgfsetlinewidth{0.000000pt}%
\definecolor{currentstroke}{rgb}{0.000000,0.000000,0.000000}%
\pgfsetstrokecolor{currentstroke}%
\pgfsetdash{}{0pt}%
\pgfpathmoveto{\pgfqpoint{3.208390in}{3.019061in}}%
\pgfpathlineto{\pgfqpoint{3.221314in}{3.003545in}}%
\pgfpathlineto{\pgfqpoint{3.234233in}{2.988290in}}%
\pgfpathlineto{\pgfqpoint{3.247147in}{2.973296in}}%
\pgfpathlineto{\pgfqpoint{3.260058in}{2.958559in}}%
\pgfpathlineto{\pgfqpoint{3.267718in}{2.971366in}}%
\pgfpathlineto{\pgfqpoint{3.275371in}{2.984314in}}%
\pgfpathlineto{\pgfqpoint{3.283019in}{2.997406in}}%
\pgfpathlineto{\pgfqpoint{3.290660in}{3.010643in}}%
\pgfpathlineto{\pgfqpoint{3.277759in}{3.025573in}}%
\pgfpathlineto{\pgfqpoint{3.264854in}{3.040761in}}%
\pgfpathlineto{\pgfqpoint{3.251945in}{3.056210in}}%
\pgfpathlineto{\pgfqpoint{3.239031in}{3.071920in}}%
\pgfpathlineto{\pgfqpoint{3.231380in}{3.058479in}}%
\pgfpathlineto{\pgfqpoint{3.223723in}{3.045190in}}%
\pgfpathlineto{\pgfqpoint{3.216060in}{3.032052in}}%
\pgfpathlineto{\pgfqpoint{3.208390in}{3.019061in}}%
\pgfpathclose%
\pgfusepath{fill}%
\end{pgfscope}%
\begin{pgfscope}%
\pgfpathrectangle{\pgfqpoint{1.254980in}{0.150000in}}{\pgfqpoint{5.490039in}{5.490039in}}%
\pgfusepath{clip}%
\pgfsetbuttcap%
\pgfsetroundjoin%
\definecolor{currentfill}{rgb}{0.237441,0.305202,0.541921}%
\pgfsetfillcolor{currentfill}%
\pgfsetfillopacity{0.700000}%
\pgfsetlinewidth{0.000000pt}%
\definecolor{currentstroke}{rgb}{0.000000,0.000000,0.000000}%
\pgfsetstrokecolor{currentstroke}%
\pgfsetdash{}{0pt}%
\pgfpathmoveto{\pgfqpoint{4.487863in}{2.895453in}}%
\pgfpathlineto{\pgfqpoint{4.500869in}{2.892415in}}%
\pgfpathlineto{\pgfqpoint{4.513883in}{2.889546in}}%
\pgfpathlineto{\pgfqpoint{4.526904in}{2.886845in}}%
\pgfpathlineto{\pgfqpoint{4.539934in}{2.884312in}}%
\pgfpathlineto{\pgfqpoint{4.547269in}{2.896700in}}%
\pgfpathlineto{\pgfqpoint{4.554602in}{2.909254in}}%
\pgfpathlineto{\pgfqpoint{4.561933in}{2.921979in}}%
\pgfpathlineto{\pgfqpoint{4.569262in}{2.934880in}}%
\pgfpathlineto{\pgfqpoint{4.556244in}{2.937880in}}%
\pgfpathlineto{\pgfqpoint{4.543233in}{2.941048in}}%
\pgfpathlineto{\pgfqpoint{4.530230in}{2.944383in}}%
\pgfpathlineto{\pgfqpoint{4.517233in}{2.947887in}}%
\pgfpathlineto{\pgfqpoint{4.509894in}{2.934509in}}%
\pgfpathlineto{\pgfqpoint{4.502552in}{2.921315in}}%
\pgfpathlineto{\pgfqpoint{4.495209in}{2.908298in}}%
\pgfpathlineto{\pgfqpoint{4.487863in}{2.895453in}}%
\pgfpathclose%
\pgfusepath{fill}%
\end{pgfscope}%
\begin{pgfscope}%
\pgfpathrectangle{\pgfqpoint{1.254980in}{0.150000in}}{\pgfqpoint{5.490039in}{5.490039in}}%
\pgfusepath{clip}%
\pgfsetbuttcap%
\pgfsetroundjoin%
\definecolor{currentfill}{rgb}{0.229739,0.322361,0.545706}%
\pgfsetfillcolor{currentfill}%
\pgfsetfillopacity{0.700000}%
\pgfsetlinewidth{0.000000pt}%
\definecolor{currentstroke}{rgb}{0.000000,0.000000,0.000000}%
\pgfsetstrokecolor{currentstroke}%
\pgfsetdash{}{0pt}%
\pgfpathmoveto{\pgfqpoint{3.260058in}{2.958559in}}%
\pgfpathlineto{\pgfqpoint{3.272965in}{2.944078in}}%
\pgfpathlineto{\pgfqpoint{3.285868in}{2.929850in}}%
\pgfpathlineto{\pgfqpoint{3.298767in}{2.915874in}}%
\pgfpathlineto{\pgfqpoint{3.311663in}{2.902147in}}%
\pgfpathlineto{\pgfqpoint{3.319314in}{2.914772in}}%
\pgfpathlineto{\pgfqpoint{3.326958in}{2.927530in}}%
\pgfpathlineto{\pgfqpoint{3.334596in}{2.940424in}}%
\pgfpathlineto{\pgfqpoint{3.342228in}{2.953457in}}%
\pgfpathlineto{\pgfqpoint{3.329341in}{2.967377in}}%
\pgfpathlineto{\pgfqpoint{3.316451in}{2.981547in}}%
\pgfpathlineto{\pgfqpoint{3.303558in}{2.995968in}}%
\pgfpathlineto{\pgfqpoint{3.290660in}{3.010643in}}%
\pgfpathlineto{\pgfqpoint{3.283019in}{2.997406in}}%
\pgfpathlineto{\pgfqpoint{3.275371in}{2.984314in}}%
\pgfpathlineto{\pgfqpoint{3.267718in}{2.971366in}}%
\pgfpathlineto{\pgfqpoint{3.260058in}{2.958559in}}%
\pgfpathclose%
\pgfusepath{fill}%
\end{pgfscope}%
\begin{pgfscope}%
\pgfpathrectangle{\pgfqpoint{1.254980in}{0.150000in}}{\pgfqpoint{5.490039in}{5.490039in}}%
\pgfusepath{clip}%
\pgfsetbuttcap%
\pgfsetroundjoin%
\definecolor{currentfill}{rgb}{0.267968,0.223549,0.512008}%
\pgfsetfillcolor{currentfill}%
\pgfsetfillopacity{0.700000}%
\pgfsetlinewidth{0.000000pt}%
\definecolor{currentstroke}{rgb}{0.000000,0.000000,0.000000}%
\pgfsetstrokecolor{currentstroke}%
\pgfsetdash{}{0pt}%
\pgfpathmoveto{\pgfqpoint{3.681354in}{2.737741in}}%
\pgfpathlineto{\pgfqpoint{3.694217in}{2.729266in}}%
\pgfpathlineto{\pgfqpoint{3.707082in}{2.720998in}}%
\pgfpathlineto{\pgfqpoint{3.719948in}{2.712937in}}%
\pgfpathlineto{\pgfqpoint{3.732817in}{2.705080in}}%
\pgfpathlineto{\pgfqpoint{3.740365in}{2.717186in}}%
\pgfpathlineto{\pgfqpoint{3.747908in}{2.729399in}}%
\pgfpathlineto{\pgfqpoint{3.755446in}{2.741723in}}%
\pgfpathlineto{\pgfqpoint{3.762979in}{2.754159in}}%
\pgfpathlineto{\pgfqpoint{3.750118in}{2.762263in}}%
\pgfpathlineto{\pgfqpoint{3.737259in}{2.770571in}}%
\pgfpathlineto{\pgfqpoint{3.724402in}{2.779086in}}%
\pgfpathlineto{\pgfqpoint{3.711547in}{2.787808in}}%
\pgfpathlineto{\pgfqpoint{3.704006in}{2.775114in}}%
\pgfpathlineto{\pgfqpoint{3.696460in}{2.762540in}}%
\pgfpathlineto{\pgfqpoint{3.688909in}{2.750084in}}%
\pgfpathlineto{\pgfqpoint{3.681354in}{2.737741in}}%
\pgfpathclose%
\pgfusepath{fill}%
\end{pgfscope}%
\begin{pgfscope}%
\pgfpathrectangle{\pgfqpoint{1.254980in}{0.150000in}}{\pgfqpoint{5.490039in}{5.490039in}}%
\pgfusepath{clip}%
\pgfsetbuttcap%
\pgfsetroundjoin%
\definecolor{currentfill}{rgb}{0.194100,0.399323,0.555565}%
\pgfsetfillcolor{currentfill}%
\pgfsetfillopacity{0.700000}%
\pgfsetlinewidth{0.000000pt}%
\definecolor{currentstroke}{rgb}{0.000000,0.000000,0.000000}%
\pgfsetstrokecolor{currentstroke}%
\pgfsetdash{}{0pt}%
\pgfpathmoveto{\pgfqpoint{4.895006in}{3.110138in}}%
\pgfpathlineto{\pgfqpoint{4.908116in}{3.107761in}}%
\pgfpathlineto{\pgfqpoint{4.921234in}{3.105543in}}%
\pgfpathlineto{\pgfqpoint{4.934362in}{3.103483in}}%
\pgfpathlineto{\pgfqpoint{4.947499in}{3.101580in}}%
\pgfpathlineto{\pgfqpoint{4.954753in}{3.115427in}}%
\pgfpathlineto{\pgfqpoint{4.962009in}{3.129529in}}%
\pgfpathlineto{\pgfqpoint{4.969265in}{3.143891in}}%
\pgfpathlineto{\pgfqpoint{4.976524in}{3.158522in}}%
\pgfpathlineto{\pgfqpoint{4.963402in}{3.161029in}}%
\pgfpathlineto{\pgfqpoint{4.950289in}{3.163694in}}%
\pgfpathlineto{\pgfqpoint{4.937185in}{3.166516in}}%
\pgfpathlineto{\pgfqpoint{4.924090in}{3.169497in}}%
\pgfpathlineto{\pgfqpoint{4.916817in}{3.154251in}}%
\pgfpathlineto{\pgfqpoint{4.909546in}{3.139281in}}%
\pgfpathlineto{\pgfqpoint{4.902275in}{3.124579in}}%
\pgfpathlineto{\pgfqpoint{4.895006in}{3.110138in}}%
\pgfpathclose%
\pgfusepath{fill}%
\end{pgfscope}%
\begin{pgfscope}%
\pgfpathrectangle{\pgfqpoint{1.254980in}{0.150000in}}{\pgfqpoint{5.490039in}{5.490039in}}%
\pgfusepath{clip}%
\pgfsetbuttcap%
\pgfsetroundjoin%
\definecolor{currentfill}{rgb}{0.204903,0.375746,0.553533}%
\pgfsetfillcolor{currentfill}%
\pgfsetfillopacity{0.700000}%
\pgfsetlinewidth{0.000000pt}%
\definecolor{currentstroke}{rgb}{0.000000,0.000000,0.000000}%
\pgfsetstrokecolor{currentstroke}%
\pgfsetdash{}{0pt}%
\pgfpathmoveto{\pgfqpoint{3.156645in}{3.083796in}}%
\pgfpathlineto{\pgfqpoint{3.169589in}{3.067207in}}%
\pgfpathlineto{\pgfqpoint{3.182528in}{3.050890in}}%
\pgfpathlineto{\pgfqpoint{3.195461in}{3.034842in}}%
\pgfpathlineto{\pgfqpoint{3.208390in}{3.019061in}}%
\pgfpathlineto{\pgfqpoint{3.216060in}{3.032052in}}%
\pgfpathlineto{\pgfqpoint{3.223723in}{3.045190in}}%
\pgfpathlineto{\pgfqpoint{3.231380in}{3.058479in}}%
\pgfpathlineto{\pgfqpoint{3.239031in}{3.071920in}}%
\pgfpathlineto{\pgfqpoint{3.226112in}{3.087896in}}%
\pgfpathlineto{\pgfqpoint{3.213188in}{3.104138in}}%
\pgfpathlineto{\pgfqpoint{3.200259in}{3.120650in}}%
\pgfpathlineto{\pgfqpoint{3.187325in}{3.137433in}}%
\pgfpathlineto{\pgfqpoint{3.179665in}{3.123786in}}%
\pgfpathlineto{\pgfqpoint{3.171998in}{3.110300in}}%
\pgfpathlineto{\pgfqpoint{3.164325in}{3.096970in}}%
\pgfpathlineto{\pgfqpoint{3.156645in}{3.083796in}}%
\pgfpathclose%
\pgfusepath{fill}%
\end{pgfscope}%
\begin{pgfscope}%
\pgfpathrectangle{\pgfqpoint{1.254980in}{0.150000in}}{\pgfqpoint{5.490039in}{5.490039in}}%
\pgfusepath{clip}%
\pgfsetbuttcap%
\pgfsetroundjoin%
\definecolor{currentfill}{rgb}{0.265145,0.232956,0.516599}%
\pgfsetfillcolor{currentfill}%
\pgfsetfillopacity{0.700000}%
\pgfsetlinewidth{0.000000pt}%
\definecolor{currentstroke}{rgb}{0.000000,0.000000,0.000000}%
\pgfsetstrokecolor{currentstroke}%
\pgfsetdash{}{0pt}%
\pgfpathmoveto{\pgfqpoint{4.029024in}{2.746175in}}%
\pgfpathlineto{\pgfqpoint{4.041928in}{2.740808in}}%
\pgfpathlineto{\pgfqpoint{4.054837in}{2.735627in}}%
\pgfpathlineto{\pgfqpoint{4.067750in}{2.730631in}}%
\pgfpathlineto{\pgfqpoint{4.080669in}{2.725819in}}%
\pgfpathlineto{\pgfqpoint{4.088125in}{2.737833in}}%
\pgfpathlineto{\pgfqpoint{4.095576in}{2.749961in}}%
\pgfpathlineto{\pgfqpoint{4.103024in}{2.762207in}}%
\pgfpathlineto{\pgfqpoint{4.110468in}{2.774574in}}%
\pgfpathlineto{\pgfqpoint{4.097557in}{2.779715in}}%
\pgfpathlineto{\pgfqpoint{4.084651in}{2.785040in}}%
\pgfpathlineto{\pgfqpoint{4.071750in}{2.790550in}}%
\pgfpathlineto{\pgfqpoint{4.058854in}{2.796246in}}%
\pgfpathlineto{\pgfqpoint{4.051402in}{2.783539in}}%
\pgfpathlineto{\pgfqpoint{4.043947in}{2.770961in}}%
\pgfpathlineto{\pgfqpoint{4.036487in}{2.758508in}}%
\pgfpathlineto{\pgfqpoint{4.029024in}{2.746175in}}%
\pgfpathclose%
\pgfusepath{fill}%
\end{pgfscope}%
\begin{pgfscope}%
\pgfpathrectangle{\pgfqpoint{1.254980in}{0.150000in}}{\pgfqpoint{5.490039in}{5.490039in}}%
\pgfusepath{clip}%
\pgfsetbuttcap%
\pgfsetroundjoin%
\definecolor{currentfill}{rgb}{0.239346,0.300855,0.540844}%
\pgfsetfillcolor{currentfill}%
\pgfsetfillopacity{0.700000}%
\pgfsetlinewidth{0.000000pt}%
\definecolor{currentstroke}{rgb}{0.000000,0.000000,0.000000}%
\pgfsetstrokecolor{currentstroke}%
\pgfsetdash{}{0pt}%
\pgfpathmoveto{\pgfqpoint{3.311663in}{2.902147in}}%
\pgfpathlineto{\pgfqpoint{3.324557in}{2.888668in}}%
\pgfpathlineto{\pgfqpoint{3.337447in}{2.875433in}}%
\pgfpathlineto{\pgfqpoint{3.350335in}{2.862442in}}%
\pgfpathlineto{\pgfqpoint{3.363221in}{2.849693in}}%
\pgfpathlineto{\pgfqpoint{3.370861in}{2.862135in}}%
\pgfpathlineto{\pgfqpoint{3.378496in}{2.874703in}}%
\pgfpathlineto{\pgfqpoint{3.386125in}{2.887401in}}%
\pgfpathlineto{\pgfqpoint{3.393749in}{2.900230in}}%
\pgfpathlineto{\pgfqpoint{3.380872in}{2.913173in}}%
\pgfpathlineto{\pgfqpoint{3.367994in}{2.926357in}}%
\pgfpathlineto{\pgfqpoint{3.355112in}{2.939784in}}%
\pgfpathlineto{\pgfqpoint{3.342228in}{2.953457in}}%
\pgfpathlineto{\pgfqpoint{3.334596in}{2.940424in}}%
\pgfpathlineto{\pgfqpoint{3.326958in}{2.927530in}}%
\pgfpathlineto{\pgfqpoint{3.319314in}{2.914772in}}%
\pgfpathlineto{\pgfqpoint{3.311663in}{2.902147in}}%
\pgfpathclose%
\pgfusepath{fill}%
\end{pgfscope}%
\begin{pgfscope}%
\pgfpathrectangle{\pgfqpoint{1.254980in}{0.150000in}}{\pgfqpoint{5.490039in}{5.490039in}}%
\pgfusepath{clip}%
\pgfsetbuttcap%
\pgfsetroundjoin%
\definecolor{currentfill}{rgb}{0.244972,0.287675,0.537260}%
\pgfsetfillcolor{currentfill}%
\pgfsetfillopacity{0.700000}%
\pgfsetlinewidth{0.000000pt}%
\definecolor{currentstroke}{rgb}{0.000000,0.000000,0.000000}%
\pgfsetstrokecolor{currentstroke}%
\pgfsetdash{}{0pt}%
\pgfpathmoveto{\pgfqpoint{4.406461in}{2.857785in}}%
\pgfpathlineto{\pgfqpoint{4.419449in}{2.854506in}}%
\pgfpathlineto{\pgfqpoint{4.432443in}{2.851398in}}%
\pgfpathlineto{\pgfqpoint{4.445445in}{2.848460in}}%
\pgfpathlineto{\pgfqpoint{4.458455in}{2.845692in}}%
\pgfpathlineto{\pgfqpoint{4.465810in}{2.857900in}}%
\pgfpathlineto{\pgfqpoint{4.473164in}{2.870259in}}%
\pgfpathlineto{\pgfqpoint{4.480514in}{2.882775in}}%
\pgfpathlineto{\pgfqpoint{4.487863in}{2.895453in}}%
\pgfpathlineto{\pgfqpoint{4.474864in}{2.898660in}}%
\pgfpathlineto{\pgfqpoint{4.461872in}{2.902036in}}%
\pgfpathlineto{\pgfqpoint{4.448887in}{2.905583in}}%
\pgfpathlineto{\pgfqpoint{4.435909in}{2.909301in}}%
\pgfpathlineto{\pgfqpoint{4.428551in}{2.896174in}}%
\pgfpathlineto{\pgfqpoint{4.421190in}{2.883216in}}%
\pgfpathlineto{\pgfqpoint{4.413827in}{2.870421in}}%
\pgfpathlineto{\pgfqpoint{4.406461in}{2.857785in}}%
\pgfpathclose%
\pgfusepath{fill}%
\end{pgfscope}%
\begin{pgfscope}%
\pgfpathrectangle{\pgfqpoint{1.254980in}{0.150000in}}{\pgfqpoint{5.490039in}{5.490039in}}%
\pgfusepath{clip}%
\pgfsetbuttcap%
\pgfsetroundjoin%
\definecolor{currentfill}{rgb}{0.269308,0.218818,0.509577}%
\pgfsetfillcolor{currentfill}%
\pgfsetfillopacity{0.700000}%
\pgfsetlinewidth{0.000000pt}%
\definecolor{currentstroke}{rgb}{0.000000,0.000000,0.000000}%
\pgfsetstrokecolor{currentstroke}%
\pgfsetdash{}{0pt}%
\pgfpathmoveto{\pgfqpoint{3.814447in}{2.723768in}}%
\pgfpathlineto{\pgfqpoint{3.827321in}{2.716671in}}%
\pgfpathlineto{\pgfqpoint{3.840197in}{2.709771in}}%
\pgfpathlineto{\pgfqpoint{3.853077in}{2.703068in}}%
\pgfpathlineto{\pgfqpoint{3.865961in}{2.696561in}}%
\pgfpathlineto{\pgfqpoint{3.873474in}{2.708585in}}%
\pgfpathlineto{\pgfqpoint{3.880983in}{2.720716in}}%
\pgfpathlineto{\pgfqpoint{3.888487in}{2.732957in}}%
\pgfpathlineto{\pgfqpoint{3.895987in}{2.745310in}}%
\pgfpathlineto{\pgfqpoint{3.883112in}{2.752091in}}%
\pgfpathlineto{\pgfqpoint{3.870240in}{2.759068in}}%
\pgfpathlineto{\pgfqpoint{3.857370in}{2.766242in}}%
\pgfpathlineto{\pgfqpoint{3.844504in}{2.773614in}}%
\pgfpathlineto{\pgfqpoint{3.836997in}{2.760977in}}%
\pgfpathlineto{\pgfqpoint{3.829485in}{2.748459in}}%
\pgfpathlineto{\pgfqpoint{3.821968in}{2.736057in}}%
\pgfpathlineto{\pgfqpoint{3.814447in}{2.723768in}}%
\pgfpathclose%
\pgfusepath{fill}%
\end{pgfscope}%
\begin{pgfscope}%
\pgfpathrectangle{\pgfqpoint{1.254980in}{0.150000in}}{\pgfqpoint{5.490039in}{5.490039in}}%
\pgfusepath{clip}%
\pgfsetbuttcap%
\pgfsetroundjoin%
\definecolor{currentfill}{rgb}{0.185556,0.418570,0.556753}%
\pgfsetfillcolor{currentfill}%
\pgfsetfillopacity{0.700000}%
\pgfsetlinewidth{0.000000pt}%
\definecolor{currentstroke}{rgb}{0.000000,0.000000,0.000000}%
\pgfsetstrokecolor{currentstroke}%
\pgfsetdash{}{0pt}%
\pgfpathmoveto{\pgfqpoint{4.976524in}{3.158522in}}%
\pgfpathlineto{\pgfqpoint{4.989654in}{3.156172in}}%
\pgfpathlineto{\pgfqpoint{5.002795in}{3.153979in}}%
\pgfpathlineto{\pgfqpoint{5.015944in}{3.151942in}}%
\pgfpathlineto{\pgfqpoint{5.029103in}{3.150061in}}%
\pgfpathlineto{\pgfqpoint{5.036347in}{3.164345in}}%
\pgfpathlineto{\pgfqpoint{5.043594in}{3.178906in}}%
\pgfpathlineto{\pgfqpoint{5.050842in}{3.193750in}}%
\pgfpathlineto{\pgfqpoint{5.058094in}{3.208884in}}%
\pgfpathlineto{\pgfqpoint{5.044951in}{3.211397in}}%
\pgfpathlineto{\pgfqpoint{5.031817in}{3.214066in}}%
\pgfpathlineto{\pgfqpoint{5.018693in}{3.216892in}}%
\pgfpathlineto{\pgfqpoint{5.005578in}{3.219874in}}%
\pgfpathlineto{\pgfqpoint{4.998311in}{3.204097in}}%
\pgfpathlineto{\pgfqpoint{4.991046in}{3.188618in}}%
\pgfpathlineto{\pgfqpoint{4.983784in}{3.173429in}}%
\pgfpathlineto{\pgfqpoint{4.976524in}{3.158522in}}%
\pgfpathclose%
\pgfusepath{fill}%
\end{pgfscope}%
\begin{pgfscope}%
\pgfpathrectangle{\pgfqpoint{1.254980in}{0.150000in}}{\pgfqpoint{5.490039in}{5.490039in}}%
\pgfusepath{clip}%
\pgfsetbuttcap%
\pgfsetroundjoin%
\definecolor{currentfill}{rgb}{0.263663,0.237631,0.518762}%
\pgfsetfillcolor{currentfill}%
\pgfsetfillopacity{0.700000}%
\pgfsetlinewidth{0.000000pt}%
\definecolor{currentstroke}{rgb}{0.000000,0.000000,0.000000}%
\pgfsetstrokecolor{currentstroke}%
\pgfsetdash{}{0pt}%
\pgfpathmoveto{\pgfqpoint{3.548154in}{2.763113in}}%
\pgfpathlineto{\pgfqpoint{3.561017in}{2.753151in}}%
\pgfpathlineto{\pgfqpoint{3.573881in}{2.743408in}}%
\pgfpathlineto{\pgfqpoint{3.586746in}{2.733882in}}%
\pgfpathlineto{\pgfqpoint{3.599611in}{2.724571in}}%
\pgfpathlineto{\pgfqpoint{3.607195in}{2.736696in}}%
\pgfpathlineto{\pgfqpoint{3.614773in}{2.748930in}}%
\pgfpathlineto{\pgfqpoint{3.622347in}{2.761277in}}%
\pgfpathlineto{\pgfqpoint{3.629915in}{2.773739in}}%
\pgfpathlineto{\pgfqpoint{3.617058in}{2.783270in}}%
\pgfpathlineto{\pgfqpoint{3.604202in}{2.793016in}}%
\pgfpathlineto{\pgfqpoint{3.591346in}{2.802978in}}%
\pgfpathlineto{\pgfqpoint{3.578491in}{2.813159in}}%
\pgfpathlineto{\pgfqpoint{3.570914in}{2.800468in}}%
\pgfpathlineto{\pgfqpoint{3.563332in}{2.787898in}}%
\pgfpathlineto{\pgfqpoint{3.555746in}{2.775447in}}%
\pgfpathlineto{\pgfqpoint{3.548154in}{2.763113in}}%
\pgfpathclose%
\pgfusepath{fill}%
\end{pgfscope}%
\begin{pgfscope}%
\pgfpathrectangle{\pgfqpoint{1.254980in}{0.150000in}}{\pgfqpoint{5.490039in}{5.490039in}}%
\pgfusepath{clip}%
\pgfsetbuttcap%
\pgfsetroundjoin%
\definecolor{currentfill}{rgb}{0.192357,0.403199,0.555836}%
\pgfsetfillcolor{currentfill}%
\pgfsetfillopacity{0.700000}%
\pgfsetlinewidth{0.000000pt}%
\definecolor{currentstroke}{rgb}{0.000000,0.000000,0.000000}%
\pgfsetstrokecolor{currentstroke}%
\pgfsetdash{}{0pt}%
\pgfpathmoveto{\pgfqpoint{3.104807in}{3.152916in}}%
\pgfpathlineto{\pgfqpoint{3.117776in}{3.135216in}}%
\pgfpathlineto{\pgfqpoint{3.130739in}{3.117798in}}%
\pgfpathlineto{\pgfqpoint{3.143695in}{3.100659in}}%
\pgfpathlineto{\pgfqpoint{3.156645in}{3.083796in}}%
\pgfpathlineto{\pgfqpoint{3.164325in}{3.096970in}}%
\pgfpathlineto{\pgfqpoint{3.171998in}{3.110300in}}%
\pgfpathlineto{\pgfqpoint{3.179665in}{3.123786in}}%
\pgfpathlineto{\pgfqpoint{3.187325in}{3.137433in}}%
\pgfpathlineto{\pgfqpoint{3.174385in}{3.154491in}}%
\pgfpathlineto{\pgfqpoint{3.161439in}{3.171825in}}%
\pgfpathlineto{\pgfqpoint{3.148486in}{3.189439in}}%
\pgfpathlineto{\pgfqpoint{3.135528in}{3.207334in}}%
\pgfpathlineto{\pgfqpoint{3.127858in}{3.193482in}}%
\pgfpathlineto{\pgfqpoint{3.120181in}{3.179797in}}%
\pgfpathlineto{\pgfqpoint{3.112498in}{3.166275in}}%
\pgfpathlineto{\pgfqpoint{3.104807in}{3.152916in}}%
\pgfpathclose%
\pgfusepath{fill}%
\end{pgfscope}%
\begin{pgfscope}%
\pgfpathrectangle{\pgfqpoint{1.254980in}{0.150000in}}{\pgfqpoint{5.490039in}{5.490039in}}%
\pgfusepath{clip}%
\pgfsetbuttcap%
\pgfsetroundjoin%
\definecolor{currentfill}{rgb}{0.252194,0.269783,0.531579}%
\pgfsetfillcolor{currentfill}%
\pgfsetfillopacity{0.700000}%
\pgfsetlinewidth{0.000000pt}%
\definecolor{currentstroke}{rgb}{0.000000,0.000000,0.000000}%
\pgfsetstrokecolor{currentstroke}%
\pgfsetdash{}{0pt}%
\pgfpathmoveto{\pgfqpoint{4.325050in}{2.821914in}}%
\pgfpathlineto{\pgfqpoint{4.338020in}{2.818356in}}%
\pgfpathlineto{\pgfqpoint{4.350996in}{2.814972in}}%
\pgfpathlineto{\pgfqpoint{4.363979in}{2.811760in}}%
\pgfpathlineto{\pgfqpoint{4.376969in}{2.808721in}}%
\pgfpathlineto{\pgfqpoint{4.384347in}{2.820774in}}%
\pgfpathlineto{\pgfqpoint{4.391721in}{2.832966in}}%
\pgfpathlineto{\pgfqpoint{4.399093in}{2.845301in}}%
\pgfpathlineto{\pgfqpoint{4.406461in}{2.857785in}}%
\pgfpathlineto{\pgfqpoint{4.393481in}{2.861235in}}%
\pgfpathlineto{\pgfqpoint{4.380507in}{2.864858in}}%
\pgfpathlineto{\pgfqpoint{4.367540in}{2.868654in}}%
\pgfpathlineto{\pgfqpoint{4.354580in}{2.872623in}}%
\pgfpathlineto{\pgfqpoint{4.347202in}{2.859718in}}%
\pgfpathlineto{\pgfqpoint{4.339821in}{2.846968in}}%
\pgfpathlineto{\pgfqpoint{4.332437in}{2.834368in}}%
\pgfpathlineto{\pgfqpoint{4.325050in}{2.821914in}}%
\pgfpathclose%
\pgfusepath{fill}%
\end{pgfscope}%
\begin{pgfscope}%
\pgfpathrectangle{\pgfqpoint{1.254980in}{0.150000in}}{\pgfqpoint{5.490039in}{5.490039in}}%
\pgfusepath{clip}%
\pgfsetbuttcap%
\pgfsetroundjoin%
\definecolor{currentfill}{rgb}{0.248629,0.278775,0.534556}%
\pgfsetfillcolor{currentfill}%
\pgfsetfillopacity{0.700000}%
\pgfsetlinewidth{0.000000pt}%
\definecolor{currentstroke}{rgb}{0.000000,0.000000,0.000000}%
\pgfsetstrokecolor{currentstroke}%
\pgfsetdash{}{0pt}%
\pgfpathmoveto{\pgfqpoint{3.363221in}{2.849693in}}%
\pgfpathlineto{\pgfqpoint{3.376104in}{2.837183in}}%
\pgfpathlineto{\pgfqpoint{3.388985in}{2.824911in}}%
\pgfpathlineto{\pgfqpoint{3.401865in}{2.812875in}}%
\pgfpathlineto{\pgfqpoint{3.414743in}{2.801073in}}%
\pgfpathlineto{\pgfqpoint{3.422374in}{2.813332in}}%
\pgfpathlineto{\pgfqpoint{3.430000in}{2.825712in}}%
\pgfpathlineto{\pgfqpoint{3.437620in}{2.838213in}}%
\pgfpathlineto{\pgfqpoint{3.445235in}{2.850840in}}%
\pgfpathlineto{\pgfqpoint{3.432366in}{2.862834in}}%
\pgfpathlineto{\pgfqpoint{3.419495in}{2.875063in}}%
\pgfpathlineto{\pgfqpoint{3.406623in}{2.887528in}}%
\pgfpathlineto{\pgfqpoint{3.393749in}{2.900230in}}%
\pgfpathlineto{\pgfqpoint{3.386125in}{2.887401in}}%
\pgfpathlineto{\pgfqpoint{3.378496in}{2.874703in}}%
\pgfpathlineto{\pgfqpoint{3.370861in}{2.862135in}}%
\pgfpathlineto{\pgfqpoint{3.363221in}{2.849693in}}%
\pgfpathclose%
\pgfusepath{fill}%
\end{pgfscope}%
\begin{pgfscope}%
\pgfpathrectangle{\pgfqpoint{1.254980in}{0.150000in}}{\pgfqpoint{5.490039in}{5.490039in}}%
\pgfusepath{clip}%
\pgfsetbuttcap%
\pgfsetroundjoin%
\definecolor{currentfill}{rgb}{0.177423,0.437527,0.557565}%
\pgfsetfillcolor{currentfill}%
\pgfsetfillopacity{0.700000}%
\pgfsetlinewidth{0.000000pt}%
\definecolor{currentstroke}{rgb}{0.000000,0.000000,0.000000}%
\pgfsetstrokecolor{currentstroke}%
\pgfsetdash{}{0pt}%
\pgfpathmoveto{\pgfqpoint{5.058094in}{3.208884in}}%
\pgfpathlineto{\pgfqpoint{5.071246in}{3.206526in}}%
\pgfpathlineto{\pgfqpoint{5.084407in}{3.204324in}}%
\pgfpathlineto{\pgfqpoint{5.097578in}{3.202276in}}%
\pgfpathlineto{\pgfqpoint{5.110759in}{3.200382in}}%
\pgfpathlineto{\pgfqpoint{5.117996in}{3.215165in}}%
\pgfpathlineto{\pgfqpoint{5.125236in}{3.230247in}}%
\pgfpathlineto{\pgfqpoint{5.132480in}{3.245635in}}%
\pgfpathlineto{\pgfqpoint{5.119312in}{3.248021in}}%
\pgfpathlineto{\pgfqpoint{5.106154in}{3.250561in}}%
\pgfpathlineto{\pgfqpoint{5.093005in}{3.253256in}}%
\pgfpathlineto{\pgfqpoint{5.079865in}{3.256106in}}%
\pgfpathlineto{\pgfqpoint{5.072605in}{3.240055in}}%
\pgfpathlineto{\pgfqpoint{5.065348in}{3.224317in}}%
\pgfpathlineto{\pgfqpoint{5.058094in}{3.208884in}}%
\pgfpathclose%
\pgfusepath{fill}%
\end{pgfscope}%
\begin{pgfscope}%
\pgfpathrectangle{\pgfqpoint{1.254980in}{0.150000in}}{\pgfqpoint{5.490039in}{5.490039in}}%
\pgfusepath{clip}%
\pgfsetbuttcap%
\pgfsetroundjoin%
\definecolor{currentfill}{rgb}{0.267968,0.223549,0.512008}%
\pgfsetfillcolor{currentfill}%
\pgfsetfillopacity{0.700000}%
\pgfsetlinewidth{0.000000pt}%
\definecolor{currentstroke}{rgb}{0.000000,0.000000,0.000000}%
\pgfsetstrokecolor{currentstroke}%
\pgfsetdash{}{0pt}%
\pgfpathmoveto{\pgfqpoint{3.947525in}{2.720119in}}%
\pgfpathlineto{\pgfqpoint{3.960420in}{2.714301in}}%
\pgfpathlineto{\pgfqpoint{3.973318in}{2.708672in}}%
\pgfpathlineto{\pgfqpoint{3.986221in}{2.703232in}}%
\pgfpathlineto{\pgfqpoint{3.999128in}{2.697979in}}%
\pgfpathlineto{\pgfqpoint{4.006608in}{2.709866in}}%
\pgfpathlineto{\pgfqpoint{4.014084in}{2.721858in}}%
\pgfpathlineto{\pgfqpoint{4.021556in}{2.733960in}}%
\pgfpathlineto{\pgfqpoint{4.029024in}{2.746175in}}%
\pgfpathlineto{\pgfqpoint{4.016124in}{2.751729in}}%
\pgfpathlineto{\pgfqpoint{4.003229in}{2.757471in}}%
\pgfpathlineto{\pgfqpoint{3.990339in}{2.763402in}}%
\pgfpathlineto{\pgfqpoint{3.977452in}{2.769522in}}%
\pgfpathlineto{\pgfqpoint{3.969977in}{2.756995in}}%
\pgfpathlineto{\pgfqpoint{3.962497in}{2.744588in}}%
\pgfpathlineto{\pgfqpoint{3.955013in}{2.732297in}}%
\pgfpathlineto{\pgfqpoint{3.947525in}{2.720119in}}%
\pgfpathclose%
\pgfusepath{fill}%
\end{pgfscope}%
\begin{pgfscope}%
\pgfpathrectangle{\pgfqpoint{1.254980in}{0.150000in}}{\pgfqpoint{5.490039in}{5.490039in}}%
\pgfusepath{clip}%
\pgfsetbuttcap%
\pgfsetroundjoin%
\definecolor{currentfill}{rgb}{0.257322,0.256130,0.526563}%
\pgfsetfillcolor{currentfill}%
\pgfsetfillopacity{0.700000}%
\pgfsetlinewidth{0.000000pt}%
\definecolor{currentstroke}{rgb}{0.000000,0.000000,0.000000}%
\pgfsetstrokecolor{currentstroke}%
\pgfsetdash{}{0pt}%
\pgfpathmoveto{\pgfqpoint{4.243620in}{2.787904in}}%
\pgfpathlineto{\pgfqpoint{4.256573in}{2.784028in}}%
\pgfpathlineto{\pgfqpoint{4.269532in}{2.780329in}}%
\pgfpathlineto{\pgfqpoint{4.282498in}{2.776806in}}%
\pgfpathlineto{\pgfqpoint{4.295470in}{2.773458in}}%
\pgfpathlineto{\pgfqpoint{4.302870in}{2.785377in}}%
\pgfpathlineto{\pgfqpoint{4.310267in}{2.797423in}}%
\pgfpathlineto{\pgfqpoint{4.317660in}{2.809601in}}%
\pgfpathlineto{\pgfqpoint{4.325050in}{2.821914in}}%
\pgfpathlineto{\pgfqpoint{4.312087in}{2.825646in}}%
\pgfpathlineto{\pgfqpoint{4.299130in}{2.829554in}}%
\pgfpathlineto{\pgfqpoint{4.286180in}{2.833637in}}%
\pgfpathlineto{\pgfqpoint{4.273236in}{2.837896in}}%
\pgfpathlineto{\pgfqpoint{4.265837in}{2.825188in}}%
\pgfpathlineto{\pgfqpoint{4.258435in}{2.812623in}}%
\pgfpathlineto{\pgfqpoint{4.251029in}{2.800197in}}%
\pgfpathlineto{\pgfqpoint{4.243620in}{2.787904in}}%
\pgfpathclose%
\pgfusepath{fill}%
\end{pgfscope}%
\begin{pgfscope}%
\pgfpathrectangle{\pgfqpoint{1.254980in}{0.150000in}}{\pgfqpoint{5.490039in}{5.490039in}}%
\pgfusepath{clip}%
\pgfsetbuttcap%
\pgfsetroundjoin%
\definecolor{currentfill}{rgb}{0.179019,0.433756,0.557430}%
\pgfsetfillcolor{currentfill}%
\pgfsetfillopacity{0.700000}%
\pgfsetlinewidth{0.000000pt}%
\definecolor{currentstroke}{rgb}{0.000000,0.000000,0.000000}%
\pgfsetstrokecolor{currentstroke}%
\pgfsetdash{}{0pt}%
\pgfpathmoveto{\pgfqpoint{3.052861in}{3.226588in}}%
\pgfpathlineto{\pgfqpoint{3.065859in}{3.207734in}}%
\pgfpathlineto{\pgfqpoint{3.078849in}{3.189172in}}%
\pgfpathlineto{\pgfqpoint{3.091832in}{3.170901in}}%
\pgfpathlineto{\pgfqpoint{3.104807in}{3.152916in}}%
\pgfpathlineto{\pgfqpoint{3.112498in}{3.166275in}}%
\pgfpathlineto{\pgfqpoint{3.120181in}{3.179797in}}%
\pgfpathlineto{\pgfqpoint{3.127858in}{3.193482in}}%
\pgfpathlineto{\pgfqpoint{3.135528in}{3.207334in}}%
\pgfpathlineto{\pgfqpoint{3.122562in}{3.225515in}}%
\pgfpathlineto{\pgfqpoint{3.109590in}{3.243982in}}%
\pgfpathlineto{\pgfqpoint{3.096610in}{3.262739in}}%
\pgfpathlineto{\pgfqpoint{3.083623in}{3.281790in}}%
\pgfpathlineto{\pgfqpoint{3.075943in}{3.267730in}}%
\pgfpathlineto{\pgfqpoint{3.068256in}{3.253845in}}%
\pgfpathlineto{\pgfqpoint{3.060562in}{3.240132in}}%
\pgfpathlineto{\pgfqpoint{3.052861in}{3.226588in}}%
\pgfpathclose%
\pgfusepath{fill}%
\end{pgfscope}%
\begin{pgfscope}%
\pgfpathrectangle{\pgfqpoint{1.254980in}{0.150000in}}{\pgfqpoint{5.490039in}{5.490039in}}%
\pgfusepath{clip}%
\pgfsetbuttcap%
\pgfsetroundjoin%
\definecolor{currentfill}{rgb}{0.257322,0.256130,0.526563}%
\pgfsetfillcolor{currentfill}%
\pgfsetfillopacity{0.700000}%
\pgfsetlinewidth{0.000000pt}%
\definecolor{currentstroke}{rgb}{0.000000,0.000000,0.000000}%
\pgfsetstrokecolor{currentstroke}%
\pgfsetdash{}{0pt}%
\pgfpathmoveto{\pgfqpoint{3.414743in}{2.801073in}}%
\pgfpathlineto{\pgfqpoint{3.427619in}{2.789503in}}%
\pgfpathlineto{\pgfqpoint{3.440495in}{2.778165in}}%
\pgfpathlineto{\pgfqpoint{3.453369in}{2.767055in}}%
\pgfpathlineto{\pgfqpoint{3.466242in}{2.756173in}}%
\pgfpathlineto{\pgfqpoint{3.473865in}{2.768250in}}%
\pgfpathlineto{\pgfqpoint{3.481481in}{2.780440in}}%
\pgfpathlineto{\pgfqpoint{3.489093in}{2.792746in}}%
\pgfpathlineto{\pgfqpoint{3.496699in}{2.805169in}}%
\pgfpathlineto{\pgfqpoint{3.483834in}{2.816244in}}%
\pgfpathlineto{\pgfqpoint{3.470969in}{2.827546in}}%
\pgfpathlineto{\pgfqpoint{3.458102in}{2.839077in}}%
\pgfpathlineto{\pgfqpoint{3.445235in}{2.850840in}}%
\pgfpathlineto{\pgfqpoint{3.437620in}{2.838213in}}%
\pgfpathlineto{\pgfqpoint{3.430000in}{2.825712in}}%
\pgfpathlineto{\pgfqpoint{3.422374in}{2.813332in}}%
\pgfpathlineto{\pgfqpoint{3.414743in}{2.801073in}}%
\pgfpathclose%
\pgfusepath{fill}%
\end{pgfscope}%
\begin{pgfscope}%
\pgfpathrectangle{\pgfqpoint{1.254980in}{0.150000in}}{\pgfqpoint{5.490039in}{5.490039in}}%
\pgfusepath{clip}%
\pgfsetbuttcap%
\pgfsetroundjoin%
\definecolor{currentfill}{rgb}{0.270595,0.214069,0.507052}%
\pgfsetfillcolor{currentfill}%
\pgfsetfillopacity{0.700000}%
\pgfsetlinewidth{0.000000pt}%
\definecolor{currentstroke}{rgb}{0.000000,0.000000,0.000000}%
\pgfsetstrokecolor{currentstroke}%
\pgfsetdash{}{0pt}%
\pgfpathmoveto{\pgfqpoint{3.732817in}{2.705080in}}%
\pgfpathlineto{\pgfqpoint{3.745688in}{2.697427in}}%
\pgfpathlineto{\pgfqpoint{3.758562in}{2.689976in}}%
\pgfpathlineto{\pgfqpoint{3.771438in}{2.682726in}}%
\pgfpathlineto{\pgfqpoint{3.784316in}{2.675677in}}%
\pgfpathlineto{\pgfqpoint{3.791856in}{2.687546in}}%
\pgfpathlineto{\pgfqpoint{3.799391in}{2.699516in}}%
\pgfpathlineto{\pgfqpoint{3.806921in}{2.711589in}}%
\pgfpathlineto{\pgfqpoint{3.814447in}{2.723768in}}%
\pgfpathlineto{\pgfqpoint{3.801576in}{2.731065in}}%
\pgfpathlineto{\pgfqpoint{3.788708in}{2.738561in}}%
\pgfpathlineto{\pgfqpoint{3.775842in}{2.746259in}}%
\pgfpathlineto{\pgfqpoint{3.762979in}{2.754159in}}%
\pgfpathlineto{\pgfqpoint{3.755446in}{2.741723in}}%
\pgfpathlineto{\pgfqpoint{3.747908in}{2.729399in}}%
\pgfpathlineto{\pgfqpoint{3.740365in}{2.717186in}}%
\pgfpathlineto{\pgfqpoint{3.732817in}{2.705080in}}%
\pgfpathclose%
\pgfusepath{fill}%
\end{pgfscope}%
\begin{pgfscope}%
\pgfpathrectangle{\pgfqpoint{1.254980in}{0.150000in}}{\pgfqpoint{5.490039in}{5.490039in}}%
\pgfusepath{clip}%
\pgfsetbuttcap%
\pgfsetroundjoin%
\definecolor{currentfill}{rgb}{0.267968,0.223549,0.512008}%
\pgfsetfillcolor{currentfill}%
\pgfsetfillopacity{0.700000}%
\pgfsetlinewidth{0.000000pt}%
\definecolor{currentstroke}{rgb}{0.000000,0.000000,0.000000}%
\pgfsetstrokecolor{currentstroke}%
\pgfsetdash{}{0pt}%
\pgfpathmoveto{\pgfqpoint{3.599611in}{2.724571in}}%
\pgfpathlineto{\pgfqpoint{3.612477in}{2.715474in}}%
\pgfpathlineto{\pgfqpoint{3.625344in}{2.706589in}}%
\pgfpathlineto{\pgfqpoint{3.638212in}{2.697915in}}%
\pgfpathlineto{\pgfqpoint{3.651082in}{2.689451in}}%
\pgfpathlineto{\pgfqpoint{3.658657in}{2.701367in}}%
\pgfpathlineto{\pgfqpoint{3.666228in}{2.713385in}}%
\pgfpathlineto{\pgfqpoint{3.673793in}{2.725509in}}%
\pgfpathlineto{\pgfqpoint{3.681354in}{2.737741in}}%
\pgfpathlineto{\pgfqpoint{3.668492in}{2.746424in}}%
\pgfpathlineto{\pgfqpoint{3.655632in}{2.755318in}}%
\pgfpathlineto{\pgfqpoint{3.642773in}{2.764422in}}%
\pgfpathlineto{\pgfqpoint{3.629915in}{2.773739in}}%
\pgfpathlineto{\pgfqpoint{3.622347in}{2.761277in}}%
\pgfpathlineto{\pgfqpoint{3.614773in}{2.748930in}}%
\pgfpathlineto{\pgfqpoint{3.607195in}{2.736696in}}%
\pgfpathlineto{\pgfqpoint{3.599611in}{2.724571in}}%
\pgfpathclose%
\pgfusepath{fill}%
\end{pgfscope}%
\begin{pgfscope}%
\pgfpathrectangle{\pgfqpoint{1.254980in}{0.150000in}}{\pgfqpoint{5.490039in}{5.490039in}}%
\pgfusepath{clip}%
\pgfsetbuttcap%
\pgfsetroundjoin%
\definecolor{currentfill}{rgb}{0.262138,0.242286,0.520837}%
\pgfsetfillcolor{currentfill}%
\pgfsetfillopacity{0.700000}%
\pgfsetlinewidth{0.000000pt}%
\definecolor{currentstroke}{rgb}{0.000000,0.000000,0.000000}%
\pgfsetstrokecolor{currentstroke}%
\pgfsetdash{}{0pt}%
\pgfpathmoveto{\pgfqpoint{4.162163in}{2.755837in}}%
\pgfpathlineto{\pgfqpoint{4.175100in}{2.751605in}}%
\pgfpathlineto{\pgfqpoint{4.188044in}{2.747553in}}%
\pgfpathlineto{\pgfqpoint{4.200993in}{2.743679in}}%
\pgfpathlineto{\pgfqpoint{4.213949in}{2.739983in}}%
\pgfpathlineto{\pgfqpoint{4.221372in}{2.751784in}}%
\pgfpathlineto{\pgfqpoint{4.228792in}{2.763702in}}%
\pgfpathlineto{\pgfqpoint{4.236208in}{2.775740in}}%
\pgfpathlineto{\pgfqpoint{4.243620in}{2.787904in}}%
\pgfpathlineto{\pgfqpoint{4.230673in}{2.791956in}}%
\pgfpathlineto{\pgfqpoint{4.217732in}{2.796187in}}%
\pgfpathlineto{\pgfqpoint{4.204798in}{2.800596in}}%
\pgfpathlineto{\pgfqpoint{4.191868in}{2.805184in}}%
\pgfpathlineto{\pgfqpoint{4.184447in}{2.792654in}}%
\pgfpathlineto{\pgfqpoint{4.177023in}{2.780255in}}%
\pgfpathlineto{\pgfqpoint{4.169595in}{2.767984in}}%
\pgfpathlineto{\pgfqpoint{4.162163in}{2.755837in}}%
\pgfpathclose%
\pgfusepath{fill}%
\end{pgfscope}%
\begin{pgfscope}%
\pgfpathrectangle{\pgfqpoint{1.254980in}{0.150000in}}{\pgfqpoint{5.490039in}{5.490039in}}%
\pgfusepath{clip}%
\pgfsetbuttcap%
\pgfsetroundjoin%
\definecolor{currentfill}{rgb}{0.165117,0.467423,0.558141}%
\pgfsetfillcolor{currentfill}%
\pgfsetfillopacity{0.700000}%
\pgfsetlinewidth{0.000000pt}%
\definecolor{currentstroke}{rgb}{0.000000,0.000000,0.000000}%
\pgfsetstrokecolor{currentstroke}%
\pgfsetdash{}{0pt}%
\pgfpathmoveto{\pgfqpoint{3.000790in}{3.304987in}}%
\pgfpathlineto{\pgfqpoint{3.013820in}{3.284933in}}%
\pgfpathlineto{\pgfqpoint{3.026842in}{3.265185in}}%
\pgfpathlineto{\pgfqpoint{3.039856in}{3.245737in}}%
\pgfpathlineto{\pgfqpoint{3.052861in}{3.226588in}}%
\pgfpathlineto{\pgfqpoint{3.060562in}{3.240132in}}%
\pgfpathlineto{\pgfqpoint{3.068256in}{3.253845in}}%
\pgfpathlineto{\pgfqpoint{3.075943in}{3.267730in}}%
\pgfpathlineto{\pgfqpoint{3.083623in}{3.281790in}}%
\pgfpathlineto{\pgfqpoint{3.070628in}{3.301135in}}%
\pgfpathlineto{\pgfqpoint{3.057625in}{3.320780in}}%
\pgfpathlineto{\pgfqpoint{3.044614in}{3.340726in}}%
\pgfpathlineto{\pgfqpoint{3.031594in}{3.360976in}}%
\pgfpathlineto{\pgfqpoint{3.023904in}{3.346709in}}%
\pgfpathlineto{\pgfqpoint{3.016207in}{3.332624in}}%
\pgfpathlineto{\pgfqpoint{3.008502in}{3.318717in}}%
\pgfpathlineto{\pgfqpoint{3.000790in}{3.304987in}}%
\pgfpathclose%
\pgfusepath{fill}%
\end{pgfscope}%
\begin{pgfscope}%
\pgfpathrectangle{\pgfqpoint{1.254980in}{0.150000in}}{\pgfqpoint{5.490039in}{5.490039in}}%
\pgfusepath{clip}%
\pgfsetbuttcap%
\pgfsetroundjoin%
\definecolor{currentfill}{rgb}{0.270595,0.214069,0.507052}%
\pgfsetfillcolor{currentfill}%
\pgfsetfillopacity{0.700000}%
\pgfsetlinewidth{0.000000pt}%
\definecolor{currentstroke}{rgb}{0.000000,0.000000,0.000000}%
\pgfsetstrokecolor{currentstroke}%
\pgfsetdash{}{0pt}%
\pgfpathmoveto{\pgfqpoint{3.865961in}{2.696561in}}%
\pgfpathlineto{\pgfqpoint{3.878847in}{2.690248in}}%
\pgfpathlineto{\pgfqpoint{3.891738in}{2.684129in}}%
\pgfpathlineto{\pgfqpoint{3.904632in}{2.678202in}}%
\pgfpathlineto{\pgfqpoint{3.917530in}{2.672468in}}%
\pgfpathlineto{\pgfqpoint{3.925035in}{2.684228in}}%
\pgfpathlineto{\pgfqpoint{3.932536in}{2.696088in}}%
\pgfpathlineto{\pgfqpoint{3.940033in}{2.708051in}}%
\pgfpathlineto{\pgfqpoint{3.947525in}{2.720119in}}%
\pgfpathlineto{\pgfqpoint{3.934635in}{2.726128in}}%
\pgfpathlineto{\pgfqpoint{3.921749in}{2.732329in}}%
\pgfpathlineto{\pgfqpoint{3.908866in}{2.738723in}}%
\pgfpathlineto{\pgfqpoint{3.895987in}{2.745310in}}%
\pgfpathlineto{\pgfqpoint{3.888487in}{2.732957in}}%
\pgfpathlineto{\pgfqpoint{3.880983in}{2.720716in}}%
\pgfpathlineto{\pgfqpoint{3.873474in}{2.708585in}}%
\pgfpathlineto{\pgfqpoint{3.865961in}{2.696561in}}%
\pgfpathclose%
\pgfusepath{fill}%
\end{pgfscope}%
\begin{pgfscope}%
\pgfpathrectangle{\pgfqpoint{1.254980in}{0.150000in}}{\pgfqpoint{5.490039in}{5.490039in}}%
\pgfusepath{clip}%
\pgfsetbuttcap%
\pgfsetroundjoin%
\definecolor{currentfill}{rgb}{0.263663,0.237631,0.518762}%
\pgfsetfillcolor{currentfill}%
\pgfsetfillopacity{0.700000}%
\pgfsetlinewidth{0.000000pt}%
\definecolor{currentstroke}{rgb}{0.000000,0.000000,0.000000}%
\pgfsetstrokecolor{currentstroke}%
\pgfsetdash{}{0pt}%
\pgfpathmoveto{\pgfqpoint{3.466242in}{2.756173in}}%
\pgfpathlineto{\pgfqpoint{3.479115in}{2.745516in}}%
\pgfpathlineto{\pgfqpoint{3.491988in}{2.735084in}}%
\pgfpathlineto{\pgfqpoint{3.504860in}{2.724874in}}%
\pgfpathlineto{\pgfqpoint{3.517732in}{2.714885in}}%
\pgfpathlineto{\pgfqpoint{3.525346in}{2.726781in}}%
\pgfpathlineto{\pgfqpoint{3.532954in}{2.738782in}}%
\pgfpathlineto{\pgfqpoint{3.540556in}{2.750892in}}%
\pgfpathlineto{\pgfqpoint{3.548154in}{2.763113in}}%
\pgfpathlineto{\pgfqpoint{3.535290in}{2.773294in}}%
\pgfpathlineto{\pgfqpoint{3.522427in}{2.783696in}}%
\pgfpathlineto{\pgfqpoint{3.509563in}{2.794320in}}%
\pgfpathlineto{\pgfqpoint{3.496699in}{2.805169in}}%
\pgfpathlineto{\pgfqpoint{3.489093in}{2.792746in}}%
\pgfpathlineto{\pgfqpoint{3.481481in}{2.780440in}}%
\pgfpathlineto{\pgfqpoint{3.473865in}{2.768250in}}%
\pgfpathlineto{\pgfqpoint{3.466242in}{2.756173in}}%
\pgfpathclose%
\pgfusepath{fill}%
\end{pgfscope}%
\begin{pgfscope}%
\pgfpathrectangle{\pgfqpoint{1.254980in}{0.150000in}}{\pgfqpoint{5.490039in}{5.490039in}}%
\pgfusepath{clip}%
\pgfsetbuttcap%
\pgfsetroundjoin%
\definecolor{currentfill}{rgb}{0.266580,0.228262,0.514349}%
\pgfsetfillcolor{currentfill}%
\pgfsetfillopacity{0.700000}%
\pgfsetlinewidth{0.000000pt}%
\definecolor{currentstroke}{rgb}{0.000000,0.000000,0.000000}%
\pgfsetstrokecolor{currentstroke}%
\pgfsetdash{}{0pt}%
\pgfpathmoveto{\pgfqpoint{4.080669in}{2.725819in}}%
\pgfpathlineto{\pgfqpoint{4.093593in}{2.721191in}}%
\pgfpathlineto{\pgfqpoint{4.106522in}{2.716746in}}%
\pgfpathlineto{\pgfqpoint{4.119457in}{2.712482in}}%
\pgfpathlineto{\pgfqpoint{4.132397in}{2.708399in}}%
\pgfpathlineto{\pgfqpoint{4.139844in}{2.720094in}}%
\pgfpathlineto{\pgfqpoint{4.147288in}{2.731895in}}%
\pgfpathlineto{\pgfqpoint{4.154727in}{2.743809in}}%
\pgfpathlineto{\pgfqpoint{4.162163in}{2.755837in}}%
\pgfpathlineto{\pgfqpoint{4.149231in}{2.760249in}}%
\pgfpathlineto{\pgfqpoint{4.136304in}{2.764842in}}%
\pgfpathlineto{\pgfqpoint{4.123383in}{2.769617in}}%
\pgfpathlineto{\pgfqpoint{4.110468in}{2.774574in}}%
\pgfpathlineto{\pgfqpoint{4.103024in}{2.762207in}}%
\pgfpathlineto{\pgfqpoint{4.095576in}{2.749961in}}%
\pgfpathlineto{\pgfqpoint{4.088125in}{2.737833in}}%
\pgfpathlineto{\pgfqpoint{4.080669in}{2.725819in}}%
\pgfpathclose%
\pgfusepath{fill}%
\end{pgfscope}%
\begin{pgfscope}%
\pgfpathrectangle{\pgfqpoint{1.254980in}{0.150000in}}{\pgfqpoint{5.490039in}{5.490039in}}%
\pgfusepath{clip}%
\pgfsetbuttcap%
\pgfsetroundjoin%
\definecolor{currentfill}{rgb}{0.220057,0.343307,0.549413}%
\pgfsetfillcolor{currentfill}%
\pgfsetfillopacity{0.700000}%
\pgfsetlinewidth{0.000000pt}%
\definecolor{currentstroke}{rgb}{0.000000,0.000000,0.000000}%
\pgfsetstrokecolor{currentstroke}%
\pgfsetdash{}{0pt}%
\pgfpathmoveto{\pgfqpoint{4.702903in}{2.966367in}}%
\pgfpathlineto{\pgfqpoint{4.715982in}{2.964355in}}%
\pgfpathlineto{\pgfqpoint{4.729071in}{2.962505in}}%
\pgfpathlineto{\pgfqpoint{4.742168in}{2.960818in}}%
\pgfpathlineto{\pgfqpoint{4.755274in}{2.959291in}}%
\pgfpathlineto{\pgfqpoint{4.762559in}{2.971625in}}%
\pgfpathlineto{\pgfqpoint{4.769843in}{2.984149in}}%
\pgfpathlineto{\pgfqpoint{4.777126in}{2.996870in}}%
\pgfpathlineto{\pgfqpoint{4.784408in}{3.009794in}}%
\pgfpathlineto{\pgfqpoint{4.771315in}{3.011843in}}%
\pgfpathlineto{\pgfqpoint{4.758230in}{3.014053in}}%
\pgfpathlineto{\pgfqpoint{4.745155in}{3.016425in}}%
\pgfpathlineto{\pgfqpoint{4.732088in}{3.018959in}}%
\pgfpathlineto{\pgfqpoint{4.724793in}{3.005503in}}%
\pgfpathlineto{\pgfqpoint{4.717497in}{2.992256in}}%
\pgfpathlineto{\pgfqpoint{4.710200in}{2.979213in}}%
\pgfpathlineto{\pgfqpoint{4.702903in}{2.966367in}}%
\pgfpathclose%
\pgfusepath{fill}%
\end{pgfscope}%
\begin{pgfscope}%
\pgfpathrectangle{\pgfqpoint{1.254980in}{0.150000in}}{\pgfqpoint{5.490039in}{5.490039in}}%
\pgfusepath{clip}%
\pgfsetbuttcap%
\pgfsetroundjoin%
\definecolor{currentfill}{rgb}{0.210503,0.363727,0.552206}%
\pgfsetfillcolor{currentfill}%
\pgfsetfillopacity{0.700000}%
\pgfsetlinewidth{0.000000pt}%
\definecolor{currentstroke}{rgb}{0.000000,0.000000,0.000000}%
\pgfsetstrokecolor{currentstroke}%
\pgfsetdash{}{0pt}%
\pgfpathmoveto{\pgfqpoint{4.784408in}{3.009794in}}%
\pgfpathlineto{\pgfqpoint{4.797509in}{3.007905in}}%
\pgfpathlineto{\pgfqpoint{4.810620in}{3.006178in}}%
\pgfpathlineto{\pgfqpoint{4.823740in}{3.004610in}}%
\pgfpathlineto{\pgfqpoint{4.836869in}{3.003202in}}%
\pgfpathlineto{\pgfqpoint{4.844136in}{3.015794in}}%
\pgfpathlineto{\pgfqpoint{4.851403in}{3.028595in}}%
\pgfpathlineto{\pgfqpoint{4.858670in}{3.041611in}}%
\pgfpathlineto{\pgfqpoint{4.865937in}{3.054848in}}%
\pgfpathlineto{\pgfqpoint{4.852822in}{3.056807in}}%
\pgfpathlineto{\pgfqpoint{4.839716in}{3.058924in}}%
\pgfpathlineto{\pgfqpoint{4.826619in}{3.061202in}}%
\pgfpathlineto{\pgfqpoint{4.813531in}{3.063640in}}%
\pgfpathlineto{\pgfqpoint{4.806250in}{3.049843in}}%
\pgfpathlineto{\pgfqpoint{4.798970in}{3.036274in}}%
\pgfpathlineto{\pgfqpoint{4.791689in}{3.022926in}}%
\pgfpathlineto{\pgfqpoint{4.784408in}{3.009794in}}%
\pgfpathclose%
\pgfusepath{fill}%
\end{pgfscope}%
\begin{pgfscope}%
\pgfpathrectangle{\pgfqpoint{1.254980in}{0.150000in}}{\pgfqpoint{5.490039in}{5.490039in}}%
\pgfusepath{clip}%
\pgfsetbuttcap%
\pgfsetroundjoin%
\definecolor{currentfill}{rgb}{0.227802,0.326594,0.546532}%
\pgfsetfillcolor{currentfill}%
\pgfsetfillopacity{0.700000}%
\pgfsetlinewidth{0.000000pt}%
\definecolor{currentstroke}{rgb}{0.000000,0.000000,0.000000}%
\pgfsetstrokecolor{currentstroke}%
\pgfsetdash{}{0pt}%
\pgfpathmoveto{\pgfqpoint{4.621414in}{2.924542in}}%
\pgfpathlineto{\pgfqpoint{4.634472in}{2.922370in}}%
\pgfpathlineto{\pgfqpoint{4.647538in}{2.920363in}}%
\pgfpathlineto{\pgfqpoint{4.660613in}{2.918520in}}%
\pgfpathlineto{\pgfqpoint{4.673697in}{2.916840in}}%
\pgfpathlineto{\pgfqpoint{4.681001in}{2.928955in}}%
\pgfpathlineto{\pgfqpoint{4.688303in}{2.941244in}}%
\pgfpathlineto{\pgfqpoint{4.695603in}{2.953713in}}%
\pgfpathlineto{\pgfqpoint{4.702903in}{2.966367in}}%
\pgfpathlineto{\pgfqpoint{4.689831in}{2.968542in}}%
\pgfpathlineto{\pgfqpoint{4.676768in}{2.970880in}}%
\pgfpathlineto{\pgfqpoint{4.663714in}{2.973382in}}%
\pgfpathlineto{\pgfqpoint{4.650668in}{2.976047in}}%
\pgfpathlineto{\pgfqpoint{4.643356in}{2.962888in}}%
\pgfpathlineto{\pgfqpoint{4.636044in}{2.949921in}}%
\pgfpathlineto{\pgfqpoint{4.628730in}{2.937141in}}%
\pgfpathlineto{\pgfqpoint{4.621414in}{2.924542in}}%
\pgfpathclose%
\pgfusepath{fill}%
\end{pgfscope}%
\begin{pgfscope}%
\pgfpathrectangle{\pgfqpoint{1.254980in}{0.150000in}}{\pgfqpoint{5.490039in}{5.490039in}}%
\pgfusepath{clip}%
\pgfsetbuttcap%
\pgfsetroundjoin%
\definecolor{currentfill}{rgb}{0.201239,0.383670,0.554294}%
\pgfsetfillcolor{currentfill}%
\pgfsetfillopacity{0.700000}%
\pgfsetlinewidth{0.000000pt}%
\definecolor{currentstroke}{rgb}{0.000000,0.000000,0.000000}%
\pgfsetstrokecolor{currentstroke}%
\pgfsetdash{}{0pt}%
\pgfpathmoveto{\pgfqpoint{4.865937in}{3.054848in}}%
\pgfpathlineto{\pgfqpoint{4.879061in}{3.053049in}}%
\pgfpathlineto{\pgfqpoint{4.892194in}{3.051408in}}%
\pgfpathlineto{\pgfqpoint{4.905336in}{3.049926in}}%
\pgfpathlineto{\pgfqpoint{4.918488in}{3.048601in}}%
\pgfpathlineto{\pgfqpoint{4.925741in}{3.061498in}}%
\pgfpathlineto{\pgfqpoint{4.932993in}{3.074623in}}%
\pgfpathlineto{\pgfqpoint{4.940245in}{3.087981in}}%
\pgfpathlineto{\pgfqpoint{4.947499in}{3.101580in}}%
\pgfpathlineto{\pgfqpoint{4.934362in}{3.103483in}}%
\pgfpathlineto{\pgfqpoint{4.921234in}{3.105543in}}%
\pgfpathlineto{\pgfqpoint{4.908116in}{3.107761in}}%
\pgfpathlineto{\pgfqpoint{4.895006in}{3.110138in}}%
\pgfpathlineto{\pgfqpoint{4.887738in}{3.095951in}}%
\pgfpathlineto{\pgfqpoint{4.880471in}{3.082012in}}%
\pgfpathlineto{\pgfqpoint{4.873204in}{3.068313in}}%
\pgfpathlineto{\pgfqpoint{4.865937in}{3.054848in}}%
\pgfpathclose%
\pgfusepath{fill}%
\end{pgfscope}%
\begin{pgfscope}%
\pgfpathrectangle{\pgfqpoint{1.254980in}{0.150000in}}{\pgfqpoint{5.490039in}{5.490039in}}%
\pgfusepath{clip}%
\pgfsetbuttcap%
\pgfsetroundjoin%
\definecolor{currentfill}{rgb}{0.235526,0.309527,0.542944}%
\pgfsetfillcolor{currentfill}%
\pgfsetfillopacity{0.700000}%
\pgfsetlinewidth{0.000000pt}%
\definecolor{currentstroke}{rgb}{0.000000,0.000000,0.000000}%
\pgfsetstrokecolor{currentstroke}%
\pgfsetdash{}{0pt}%
\pgfpathmoveto{\pgfqpoint{4.539934in}{2.884312in}}%
\pgfpathlineto{\pgfqpoint{4.552971in}{2.881945in}}%
\pgfpathlineto{\pgfqpoint{4.566016in}{2.879745in}}%
\pgfpathlineto{\pgfqpoint{4.579069in}{2.877710in}}%
\pgfpathlineto{\pgfqpoint{4.592130in}{2.875842in}}%
\pgfpathlineto{\pgfqpoint{4.599454in}{2.887773in}}%
\pgfpathlineto{\pgfqpoint{4.606776in}{2.899863in}}%
\pgfpathlineto{\pgfqpoint{4.614096in}{2.912118in}}%
\pgfpathlineto{\pgfqpoint{4.621414in}{2.924542in}}%
\pgfpathlineto{\pgfqpoint{4.608364in}{2.926878in}}%
\pgfpathlineto{\pgfqpoint{4.595322in}{2.929379in}}%
\pgfpathlineto{\pgfqpoint{4.582288in}{2.932047in}}%
\pgfpathlineto{\pgfqpoint{4.569262in}{2.934880in}}%
\pgfpathlineto{\pgfqpoint{4.561933in}{2.921979in}}%
\pgfpathlineto{\pgfqpoint{4.554602in}{2.909254in}}%
\pgfpathlineto{\pgfqpoint{4.547269in}{2.896700in}}%
\pgfpathlineto{\pgfqpoint{4.539934in}{2.884312in}}%
\pgfpathclose%
\pgfusepath{fill}%
\end{pgfscope}%
\begin{pgfscope}%
\pgfpathrectangle{\pgfqpoint{1.254980in}{0.150000in}}{\pgfqpoint{5.490039in}{5.490039in}}%
\pgfusepath{clip}%
\pgfsetbuttcap%
\pgfsetroundjoin%
\definecolor{currentfill}{rgb}{0.223925,0.334994,0.548053}%
\pgfsetfillcolor{currentfill}%
\pgfsetfillopacity{0.700000}%
\pgfsetlinewidth{0.000000pt}%
\definecolor{currentstroke}{rgb}{0.000000,0.000000,0.000000}%
\pgfsetstrokecolor{currentstroke}%
\pgfsetdash{}{0pt}%
\pgfpathmoveto{\pgfqpoint{3.177647in}{2.968529in}}%
\pgfpathlineto{\pgfqpoint{3.190581in}{2.953179in}}%
\pgfpathlineto{\pgfqpoint{3.203510in}{2.938090in}}%
\pgfpathlineto{\pgfqpoint{3.216435in}{2.923262in}}%
\pgfpathlineto{\pgfqpoint{3.229356in}{2.908691in}}%
\pgfpathlineto{\pgfqpoint{3.237041in}{2.920958in}}%
\pgfpathlineto{\pgfqpoint{3.244720in}{2.933357in}}%
\pgfpathlineto{\pgfqpoint{3.252392in}{2.945890in}}%
\pgfpathlineto{\pgfqpoint{3.260058in}{2.958559in}}%
\pgfpathlineto{\pgfqpoint{3.247147in}{2.973296in}}%
\pgfpathlineto{\pgfqpoint{3.234233in}{2.988290in}}%
\pgfpathlineto{\pgfqpoint{3.221314in}{3.003545in}}%
\pgfpathlineto{\pgfqpoint{3.208390in}{3.019061in}}%
\pgfpathlineto{\pgfqpoint{3.200714in}{3.006216in}}%
\pgfpathlineto{\pgfqpoint{3.193031in}{2.993514in}}%
\pgfpathlineto{\pgfqpoint{3.185342in}{2.980952in}}%
\pgfpathlineto{\pgfqpoint{3.177647in}{2.968529in}}%
\pgfpathclose%
\pgfusepath{fill}%
\end{pgfscope}%
\begin{pgfscope}%
\pgfpathrectangle{\pgfqpoint{1.254980in}{0.150000in}}{\pgfqpoint{5.490039in}{5.490039in}}%
\pgfusepath{clip}%
\pgfsetbuttcap%
\pgfsetroundjoin%
\definecolor{currentfill}{rgb}{0.192357,0.403199,0.555836}%
\pgfsetfillcolor{currentfill}%
\pgfsetfillopacity{0.700000}%
\pgfsetlinewidth{0.000000pt}%
\definecolor{currentstroke}{rgb}{0.000000,0.000000,0.000000}%
\pgfsetstrokecolor{currentstroke}%
\pgfsetdash{}{0pt}%
\pgfpathmoveto{\pgfqpoint{4.947499in}{3.101580in}}%
\pgfpathlineto{\pgfqpoint{4.960645in}{3.099835in}}%
\pgfpathlineto{\pgfqpoint{4.973801in}{3.098247in}}%
\pgfpathlineto{\pgfqpoint{4.986966in}{3.096815in}}%
\pgfpathlineto{\pgfqpoint{5.000141in}{3.095540in}}%
\pgfpathlineto{\pgfqpoint{5.007380in}{3.108792in}}%
\pgfpathlineto{\pgfqpoint{5.014619in}{3.122291in}}%
\pgfpathlineto{\pgfqpoint{5.021860in}{3.136045in}}%
\pgfpathlineto{\pgfqpoint{5.029103in}{3.150061in}}%
\pgfpathlineto{\pgfqpoint{5.015944in}{3.151942in}}%
\pgfpathlineto{\pgfqpoint{5.002795in}{3.153979in}}%
\pgfpathlineto{\pgfqpoint{4.989654in}{3.156172in}}%
\pgfpathlineto{\pgfqpoint{4.976524in}{3.158522in}}%
\pgfpathlineto{\pgfqpoint{4.969265in}{3.143891in}}%
\pgfpathlineto{\pgfqpoint{4.962009in}{3.129529in}}%
\pgfpathlineto{\pgfqpoint{4.954753in}{3.115427in}}%
\pgfpathlineto{\pgfqpoint{4.947499in}{3.101580in}}%
\pgfpathclose%
\pgfusepath{fill}%
\end{pgfscope}%
\begin{pgfscope}%
\pgfpathrectangle{\pgfqpoint{1.254980in}{0.150000in}}{\pgfqpoint{5.490039in}{5.490039in}}%
\pgfusepath{clip}%
\pgfsetbuttcap%
\pgfsetroundjoin%
\definecolor{currentfill}{rgb}{0.235526,0.309527,0.542944}%
\pgfsetfillcolor{currentfill}%
\pgfsetfillopacity{0.700000}%
\pgfsetlinewidth{0.000000pt}%
\definecolor{currentstroke}{rgb}{0.000000,0.000000,0.000000}%
\pgfsetstrokecolor{currentstroke}%
\pgfsetdash{}{0pt}%
\pgfpathmoveto{\pgfqpoint{3.229356in}{2.908691in}}%
\pgfpathlineto{\pgfqpoint{3.242273in}{2.894375in}}%
\pgfpathlineto{\pgfqpoint{3.255186in}{2.880313in}}%
\pgfpathlineto{\pgfqpoint{3.268096in}{2.866502in}}%
\pgfpathlineto{\pgfqpoint{3.281003in}{2.852941in}}%
\pgfpathlineto{\pgfqpoint{3.288677in}{2.865053in}}%
\pgfpathlineto{\pgfqpoint{3.296345in}{2.877290in}}%
\pgfpathlineto{\pgfqpoint{3.304007in}{2.889654in}}%
\pgfpathlineto{\pgfqpoint{3.311663in}{2.902147in}}%
\pgfpathlineto{\pgfqpoint{3.298767in}{2.915874in}}%
\pgfpathlineto{\pgfqpoint{3.285868in}{2.929850in}}%
\pgfpathlineto{\pgfqpoint{3.272965in}{2.944078in}}%
\pgfpathlineto{\pgfqpoint{3.260058in}{2.958559in}}%
\pgfpathlineto{\pgfqpoint{3.252392in}{2.945890in}}%
\pgfpathlineto{\pgfqpoint{3.244720in}{2.933357in}}%
\pgfpathlineto{\pgfqpoint{3.237041in}{2.920958in}}%
\pgfpathlineto{\pgfqpoint{3.229356in}{2.908691in}}%
\pgfpathclose%
\pgfusepath{fill}%
\end{pgfscope}%
\begin{pgfscope}%
\pgfpathrectangle{\pgfqpoint{1.254980in}{0.150000in}}{\pgfqpoint{5.490039in}{5.490039in}}%
\pgfusepath{clip}%
\pgfsetbuttcap%
\pgfsetroundjoin%
\definecolor{currentfill}{rgb}{0.243113,0.292092,0.538516}%
\pgfsetfillcolor{currentfill}%
\pgfsetfillopacity{0.700000}%
\pgfsetlinewidth{0.000000pt}%
\definecolor{currentstroke}{rgb}{0.000000,0.000000,0.000000}%
\pgfsetstrokecolor{currentstroke}%
\pgfsetdash{}{0pt}%
\pgfpathmoveto{\pgfqpoint{4.458455in}{2.845692in}}%
\pgfpathlineto{\pgfqpoint{4.471471in}{2.843094in}}%
\pgfpathlineto{\pgfqpoint{4.484495in}{2.840664in}}%
\pgfpathlineto{\pgfqpoint{4.497528in}{2.838403in}}%
\pgfpathlineto{\pgfqpoint{4.510567in}{2.836309in}}%
\pgfpathlineto{\pgfqpoint{4.517913in}{2.848087in}}%
\pgfpathlineto{\pgfqpoint{4.525256in}{2.860010in}}%
\pgfpathlineto{\pgfqpoint{4.532596in}{2.872083in}}%
\pgfpathlineto{\pgfqpoint{4.539934in}{2.884312in}}%
\pgfpathlineto{\pgfqpoint{4.526904in}{2.886845in}}%
\pgfpathlineto{\pgfqpoint{4.513883in}{2.889546in}}%
\pgfpathlineto{\pgfqpoint{4.500869in}{2.892415in}}%
\pgfpathlineto{\pgfqpoint{4.487863in}{2.895453in}}%
\pgfpathlineto{\pgfqpoint{4.480514in}{2.882775in}}%
\pgfpathlineto{\pgfqpoint{4.473164in}{2.870259in}}%
\pgfpathlineto{\pgfqpoint{4.465810in}{2.857900in}}%
\pgfpathlineto{\pgfqpoint{4.458455in}{2.845692in}}%
\pgfpathclose%
\pgfusepath{fill}%
\end{pgfscope}%
\begin{pgfscope}%
\pgfpathrectangle{\pgfqpoint{1.254980in}{0.150000in}}{\pgfqpoint{5.490039in}{5.490039in}}%
\pgfusepath{clip}%
\pgfsetbuttcap%
\pgfsetroundjoin%
\definecolor{currentfill}{rgb}{0.212395,0.359683,0.551710}%
\pgfsetfillcolor{currentfill}%
\pgfsetfillopacity{0.700000}%
\pgfsetlinewidth{0.000000pt}%
\definecolor{currentstroke}{rgb}{0.000000,0.000000,0.000000}%
\pgfsetstrokecolor{currentstroke}%
\pgfsetdash{}{0pt}%
\pgfpathmoveto{\pgfqpoint{3.125859in}{3.032599in}}%
\pgfpathlineto{\pgfqpoint{3.138814in}{3.016177in}}%
\pgfpathlineto{\pgfqpoint{3.151763in}{3.000026in}}%
\pgfpathlineto{\pgfqpoint{3.164708in}{2.984144in}}%
\pgfpathlineto{\pgfqpoint{3.177647in}{2.968529in}}%
\pgfpathlineto{\pgfqpoint{3.185342in}{2.980952in}}%
\pgfpathlineto{\pgfqpoint{3.193031in}{2.993514in}}%
\pgfpathlineto{\pgfqpoint{3.200714in}{3.006216in}}%
\pgfpathlineto{\pgfqpoint{3.208390in}{3.019061in}}%
\pgfpathlineto{\pgfqpoint{3.195461in}{3.034842in}}%
\pgfpathlineto{\pgfqpoint{3.182528in}{3.050890in}}%
\pgfpathlineto{\pgfqpoint{3.169589in}{3.067207in}}%
\pgfpathlineto{\pgfqpoint{3.156645in}{3.083796in}}%
\pgfpathlineto{\pgfqpoint{3.148958in}{3.070774in}}%
\pgfpathlineto{\pgfqpoint{3.141265in}{3.057902in}}%
\pgfpathlineto{\pgfqpoint{3.133565in}{3.045178in}}%
\pgfpathlineto{\pgfqpoint{3.125859in}{3.032599in}}%
\pgfpathclose%
\pgfusepath{fill}%
\end{pgfscope}%
\begin{pgfscope}%
\pgfpathrectangle{\pgfqpoint{1.254980in}{0.150000in}}{\pgfqpoint{5.490039in}{5.490039in}}%
\pgfusepath{clip}%
\pgfsetbuttcap%
\pgfsetroundjoin%
\definecolor{currentfill}{rgb}{0.271828,0.209303,0.504434}%
\pgfsetfillcolor{currentfill}%
\pgfsetfillopacity{0.700000}%
\pgfsetlinewidth{0.000000pt}%
\definecolor{currentstroke}{rgb}{0.000000,0.000000,0.000000}%
\pgfsetstrokecolor{currentstroke}%
\pgfsetdash{}{0pt}%
\pgfpathmoveto{\pgfqpoint{3.651082in}{2.689451in}}%
\pgfpathlineto{\pgfqpoint{3.663953in}{2.681196in}}%
\pgfpathlineto{\pgfqpoint{3.676826in}{2.673148in}}%
\pgfpathlineto{\pgfqpoint{3.689701in}{2.665306in}}%
\pgfpathlineto{\pgfqpoint{3.702578in}{2.657669in}}%
\pgfpathlineto{\pgfqpoint{3.710145in}{2.669375in}}%
\pgfpathlineto{\pgfqpoint{3.717707in}{2.681177in}}%
\pgfpathlineto{\pgfqpoint{3.725265in}{2.693078in}}%
\pgfpathlineto{\pgfqpoint{3.732817in}{2.705080in}}%
\pgfpathlineto{\pgfqpoint{3.719948in}{2.712937in}}%
\pgfpathlineto{\pgfqpoint{3.707082in}{2.720998in}}%
\pgfpathlineto{\pgfqpoint{3.694217in}{2.729266in}}%
\pgfpathlineto{\pgfqpoint{3.681354in}{2.737741in}}%
\pgfpathlineto{\pgfqpoint{3.673793in}{2.725509in}}%
\pgfpathlineto{\pgfqpoint{3.666228in}{2.713385in}}%
\pgfpathlineto{\pgfqpoint{3.658657in}{2.701367in}}%
\pgfpathlineto{\pgfqpoint{3.651082in}{2.689451in}}%
\pgfpathclose%
\pgfusepath{fill}%
\end{pgfscope}%
\begin{pgfscope}%
\pgfpathrectangle{\pgfqpoint{1.254980in}{0.150000in}}{\pgfqpoint{5.490039in}{5.490039in}}%
\pgfusepath{clip}%
\pgfsetbuttcap%
\pgfsetroundjoin%
\definecolor{currentfill}{rgb}{0.183898,0.422383,0.556944}%
\pgfsetfillcolor{currentfill}%
\pgfsetfillopacity{0.700000}%
\pgfsetlinewidth{0.000000pt}%
\definecolor{currentstroke}{rgb}{0.000000,0.000000,0.000000}%
\pgfsetstrokecolor{currentstroke}%
\pgfsetdash{}{0pt}%
\pgfpathmoveto{\pgfqpoint{5.029103in}{3.150061in}}%
\pgfpathlineto{\pgfqpoint{5.042271in}{3.148336in}}%
\pgfpathlineto{\pgfqpoint{5.055450in}{3.146766in}}%
\pgfpathlineto{\pgfqpoint{5.068638in}{3.145351in}}%
\pgfpathlineto{\pgfqpoint{5.081836in}{3.144091in}}%
\pgfpathlineto{\pgfqpoint{5.089063in}{3.157752in}}%
\pgfpathlineto{\pgfqpoint{5.096293in}{3.171683in}}%
\pgfpathlineto{\pgfqpoint{5.103524in}{3.185891in}}%
\pgfpathlineto{\pgfqpoint{5.110759in}{3.200382in}}%
\pgfpathlineto{\pgfqpoint{5.097578in}{3.202276in}}%
\pgfpathlineto{\pgfqpoint{5.084407in}{3.204324in}}%
\pgfpathlineto{\pgfqpoint{5.071246in}{3.206526in}}%
\pgfpathlineto{\pgfqpoint{5.058094in}{3.208884in}}%
\pgfpathlineto{\pgfqpoint{5.050842in}{3.193750in}}%
\pgfpathlineto{\pgfqpoint{5.043594in}{3.178906in}}%
\pgfpathlineto{\pgfqpoint{5.036347in}{3.164345in}}%
\pgfpathlineto{\pgfqpoint{5.029103in}{3.150061in}}%
\pgfpathclose%
\pgfusepath{fill}%
\end{pgfscope}%
\begin{pgfscope}%
\pgfpathrectangle{\pgfqpoint{1.254980in}{0.150000in}}{\pgfqpoint{5.490039in}{5.490039in}}%
\pgfusepath{clip}%
\pgfsetbuttcap%
\pgfsetroundjoin%
\definecolor{currentfill}{rgb}{0.246811,0.283237,0.535941}%
\pgfsetfillcolor{currentfill}%
\pgfsetfillopacity{0.700000}%
\pgfsetlinewidth{0.000000pt}%
\definecolor{currentstroke}{rgb}{0.000000,0.000000,0.000000}%
\pgfsetstrokecolor{currentstroke}%
\pgfsetdash{}{0pt}%
\pgfpathmoveto{\pgfqpoint{3.281003in}{2.852941in}}%
\pgfpathlineto{\pgfqpoint{3.293906in}{2.839627in}}%
\pgfpathlineto{\pgfqpoint{3.306806in}{2.826557in}}%
\pgfpathlineto{\pgfqpoint{3.319704in}{2.813732in}}%
\pgfpathlineto{\pgfqpoint{3.332599in}{2.801147in}}%
\pgfpathlineto{\pgfqpoint{3.340263in}{2.813105in}}%
\pgfpathlineto{\pgfqpoint{3.347922in}{2.825180in}}%
\pgfpathlineto{\pgfqpoint{3.355574in}{2.837376in}}%
\pgfpathlineto{\pgfqpoint{3.363221in}{2.849693in}}%
\pgfpathlineto{\pgfqpoint{3.350335in}{2.862442in}}%
\pgfpathlineto{\pgfqpoint{3.337447in}{2.875433in}}%
\pgfpathlineto{\pgfqpoint{3.324557in}{2.888668in}}%
\pgfpathlineto{\pgfqpoint{3.311663in}{2.902147in}}%
\pgfpathlineto{\pgfqpoint{3.304007in}{2.889654in}}%
\pgfpathlineto{\pgfqpoint{3.296345in}{2.877290in}}%
\pgfpathlineto{\pgfqpoint{3.288677in}{2.865053in}}%
\pgfpathlineto{\pgfqpoint{3.281003in}{2.852941in}}%
\pgfpathclose%
\pgfusepath{fill}%
\end{pgfscope}%
\begin{pgfscope}%
\pgfpathrectangle{\pgfqpoint{1.254980in}{0.150000in}}{\pgfqpoint{5.490039in}{5.490039in}}%
\pgfusepath{clip}%
\pgfsetbuttcap%
\pgfsetroundjoin%
\definecolor{currentfill}{rgb}{0.269308,0.218818,0.509577}%
\pgfsetfillcolor{currentfill}%
\pgfsetfillopacity{0.700000}%
\pgfsetlinewidth{0.000000pt}%
\definecolor{currentstroke}{rgb}{0.000000,0.000000,0.000000}%
\pgfsetstrokecolor{currentstroke}%
\pgfsetdash{}{0pt}%
\pgfpathmoveto{\pgfqpoint{3.999128in}{2.697979in}}%
\pgfpathlineto{\pgfqpoint{4.012040in}{2.692914in}}%
\pgfpathlineto{\pgfqpoint{4.024957in}{2.688035in}}%
\pgfpathlineto{\pgfqpoint{4.037879in}{2.683341in}}%
\pgfpathlineto{\pgfqpoint{4.050806in}{2.678831in}}%
\pgfpathlineto{\pgfqpoint{4.058278in}{2.690426in}}%
\pgfpathlineto{\pgfqpoint{4.065746in}{2.702119in}}%
\pgfpathlineto{\pgfqpoint{4.073209in}{2.713916in}}%
\pgfpathlineto{\pgfqpoint{4.080669in}{2.725819in}}%
\pgfpathlineto{\pgfqpoint{4.067750in}{2.730631in}}%
\pgfpathlineto{\pgfqpoint{4.054837in}{2.735627in}}%
\pgfpathlineto{\pgfqpoint{4.041928in}{2.740808in}}%
\pgfpathlineto{\pgfqpoint{4.029024in}{2.746175in}}%
\pgfpathlineto{\pgfqpoint{4.021556in}{2.733960in}}%
\pgfpathlineto{\pgfqpoint{4.014084in}{2.721858in}}%
\pgfpathlineto{\pgfqpoint{4.006608in}{2.709866in}}%
\pgfpathlineto{\pgfqpoint{3.999128in}{2.697979in}}%
\pgfpathclose%
\pgfusepath{fill}%
\end{pgfscope}%
\begin{pgfscope}%
\pgfpathrectangle{\pgfqpoint{1.254980in}{0.150000in}}{\pgfqpoint{5.490039in}{5.490039in}}%
\pgfusepath{clip}%
\pgfsetbuttcap%
\pgfsetroundjoin%
\definecolor{currentfill}{rgb}{0.250425,0.274290,0.533103}%
\pgfsetfillcolor{currentfill}%
\pgfsetfillopacity{0.700000}%
\pgfsetlinewidth{0.000000pt}%
\definecolor{currentstroke}{rgb}{0.000000,0.000000,0.000000}%
\pgfsetstrokecolor{currentstroke}%
\pgfsetdash{}{0pt}%
\pgfpathmoveto{\pgfqpoint{4.376969in}{2.808721in}}%
\pgfpathlineto{\pgfqpoint{4.389966in}{2.805854in}}%
\pgfpathlineto{\pgfqpoint{4.402971in}{2.803158in}}%
\pgfpathlineto{\pgfqpoint{4.415983in}{2.800633in}}%
\pgfpathlineto{\pgfqpoint{4.429002in}{2.798277in}}%
\pgfpathlineto{\pgfqpoint{4.436370in}{2.809928in}}%
\pgfpathlineto{\pgfqpoint{4.443734in}{2.821711in}}%
\pgfpathlineto{\pgfqpoint{4.451096in}{2.833631in}}%
\pgfpathlineto{\pgfqpoint{4.458455in}{2.845692in}}%
\pgfpathlineto{\pgfqpoint{4.445445in}{2.848460in}}%
\pgfpathlineto{\pgfqpoint{4.432443in}{2.851398in}}%
\pgfpathlineto{\pgfqpoint{4.419449in}{2.854506in}}%
\pgfpathlineto{\pgfqpoint{4.406461in}{2.857785in}}%
\pgfpathlineto{\pgfqpoint{4.399093in}{2.845301in}}%
\pgfpathlineto{\pgfqpoint{4.391721in}{2.832966in}}%
\pgfpathlineto{\pgfqpoint{4.384347in}{2.820774in}}%
\pgfpathlineto{\pgfqpoint{4.376969in}{2.808721in}}%
\pgfpathclose%
\pgfusepath{fill}%
\end{pgfscope}%
\begin{pgfscope}%
\pgfpathrectangle{\pgfqpoint{1.254980in}{0.150000in}}{\pgfqpoint{5.490039in}{5.490039in}}%
\pgfusepath{clip}%
\pgfsetbuttcap%
\pgfsetroundjoin%
\definecolor{currentfill}{rgb}{0.273006,0.204520,0.501721}%
\pgfsetfillcolor{currentfill}%
\pgfsetfillopacity{0.700000}%
\pgfsetlinewidth{0.000000pt}%
\definecolor{currentstroke}{rgb}{0.000000,0.000000,0.000000}%
\pgfsetstrokecolor{currentstroke}%
\pgfsetdash{}{0pt}%
\pgfpathmoveto{\pgfqpoint{3.784316in}{2.675677in}}%
\pgfpathlineto{\pgfqpoint{3.797198in}{2.668826in}}%
\pgfpathlineto{\pgfqpoint{3.810083in}{2.662173in}}%
\pgfpathlineto{\pgfqpoint{3.822970in}{2.655717in}}%
\pgfpathlineto{\pgfqpoint{3.835862in}{2.649457in}}%
\pgfpathlineto{\pgfqpoint{3.843393in}{2.661090in}}%
\pgfpathlineto{\pgfqpoint{3.850920in}{2.672816in}}%
\pgfpathlineto{\pgfqpoint{3.858443in}{2.684638in}}%
\pgfpathlineto{\pgfqpoint{3.865961in}{2.696561in}}%
\pgfpathlineto{\pgfqpoint{3.853077in}{2.703068in}}%
\pgfpathlineto{\pgfqpoint{3.840197in}{2.709771in}}%
\pgfpathlineto{\pgfqpoint{3.827321in}{2.716671in}}%
\pgfpathlineto{\pgfqpoint{3.814447in}{2.723768in}}%
\pgfpathlineto{\pgfqpoint{3.806921in}{2.711589in}}%
\pgfpathlineto{\pgfqpoint{3.799391in}{2.699516in}}%
\pgfpathlineto{\pgfqpoint{3.791856in}{2.687546in}}%
\pgfpathlineto{\pgfqpoint{3.784316in}{2.675677in}}%
\pgfpathclose%
\pgfusepath{fill}%
\end{pgfscope}%
\begin{pgfscope}%
\pgfpathrectangle{\pgfqpoint{1.254980in}{0.150000in}}{\pgfqpoint{5.490039in}{5.490039in}}%
\pgfusepath{clip}%
\pgfsetbuttcap%
\pgfsetroundjoin%
\definecolor{currentfill}{rgb}{0.197636,0.391528,0.554969}%
\pgfsetfillcolor{currentfill}%
\pgfsetfillopacity{0.700000}%
\pgfsetlinewidth{0.000000pt}%
\definecolor{currentstroke}{rgb}{0.000000,0.000000,0.000000}%
\pgfsetstrokecolor{currentstroke}%
\pgfsetdash{}{0pt}%
\pgfpathmoveto{\pgfqpoint{3.073977in}{3.101053in}}%
\pgfpathlineto{\pgfqpoint{3.086957in}{3.083520in}}%
\pgfpathlineto{\pgfqpoint{3.099931in}{3.066268in}}%
\pgfpathlineto{\pgfqpoint{3.112898in}{3.049295in}}%
\pgfpathlineto{\pgfqpoint{3.125859in}{3.032599in}}%
\pgfpathlineto{\pgfqpoint{3.133565in}{3.045178in}}%
\pgfpathlineto{\pgfqpoint{3.141265in}{3.057902in}}%
\pgfpathlineto{\pgfqpoint{3.148958in}{3.070774in}}%
\pgfpathlineto{\pgfqpoint{3.156645in}{3.083796in}}%
\pgfpathlineto{\pgfqpoint{3.143695in}{3.100659in}}%
\pgfpathlineto{\pgfqpoint{3.130739in}{3.117798in}}%
\pgfpathlineto{\pgfqpoint{3.117776in}{3.135216in}}%
\pgfpathlineto{\pgfqpoint{3.104807in}{3.152916in}}%
\pgfpathlineto{\pgfqpoint{3.097110in}{3.139717in}}%
\pgfpathlineto{\pgfqpoint{3.089406in}{3.126675in}}%
\pgfpathlineto{\pgfqpoint{3.081695in}{3.113787in}}%
\pgfpathlineto{\pgfqpoint{3.073977in}{3.101053in}}%
\pgfpathclose%
\pgfusepath{fill}%
\end{pgfscope}%
\begin{pgfscope}%
\pgfpathrectangle{\pgfqpoint{1.254980in}{0.150000in}}{\pgfqpoint{5.490039in}{5.490039in}}%
\pgfusepath{clip}%
\pgfsetbuttcap%
\pgfsetroundjoin%
\definecolor{currentfill}{rgb}{0.267968,0.223549,0.512008}%
\pgfsetfillcolor{currentfill}%
\pgfsetfillopacity{0.700000}%
\pgfsetlinewidth{0.000000pt}%
\definecolor{currentstroke}{rgb}{0.000000,0.000000,0.000000}%
\pgfsetstrokecolor{currentstroke}%
\pgfsetdash{}{0pt}%
\pgfpathmoveto{\pgfqpoint{3.517732in}{2.714885in}}%
\pgfpathlineto{\pgfqpoint{3.530605in}{2.705117in}}%
\pgfpathlineto{\pgfqpoint{3.543477in}{2.695566in}}%
\pgfpathlineto{\pgfqpoint{3.556351in}{2.686232in}}%
\pgfpathlineto{\pgfqpoint{3.569224in}{2.677113in}}%
\pgfpathlineto{\pgfqpoint{3.576829in}{2.688826in}}%
\pgfpathlineto{\pgfqpoint{3.584428in}{2.700638in}}%
\pgfpathlineto{\pgfqpoint{3.592022in}{2.712552in}}%
\pgfpathlineto{\pgfqpoint{3.599611in}{2.724571in}}%
\pgfpathlineto{\pgfqpoint{3.586746in}{2.733882in}}%
\pgfpathlineto{\pgfqpoint{3.573881in}{2.743408in}}%
\pgfpathlineto{\pgfqpoint{3.561017in}{2.753151in}}%
\pgfpathlineto{\pgfqpoint{3.548154in}{2.763113in}}%
\pgfpathlineto{\pgfqpoint{3.540556in}{2.750892in}}%
\pgfpathlineto{\pgfqpoint{3.532954in}{2.738782in}}%
\pgfpathlineto{\pgfqpoint{3.525346in}{2.726781in}}%
\pgfpathlineto{\pgfqpoint{3.517732in}{2.714885in}}%
\pgfpathclose%
\pgfusepath{fill}%
\end{pgfscope}%
\begin{pgfscope}%
\pgfpathrectangle{\pgfqpoint{1.254980in}{0.150000in}}{\pgfqpoint{5.490039in}{5.490039in}}%
\pgfusepath{clip}%
\pgfsetbuttcap%
\pgfsetroundjoin%
\definecolor{currentfill}{rgb}{0.257322,0.256130,0.526563}%
\pgfsetfillcolor{currentfill}%
\pgfsetfillopacity{0.700000}%
\pgfsetlinewidth{0.000000pt}%
\definecolor{currentstroke}{rgb}{0.000000,0.000000,0.000000}%
\pgfsetstrokecolor{currentstroke}%
\pgfsetdash{}{0pt}%
\pgfpathmoveto{\pgfqpoint{4.295470in}{2.773458in}}%
\pgfpathlineto{\pgfqpoint{4.308449in}{2.770284in}}%
\pgfpathlineto{\pgfqpoint{4.321434in}{2.767284in}}%
\pgfpathlineto{\pgfqpoint{4.334427in}{2.764458in}}%
\pgfpathlineto{\pgfqpoint{4.347426in}{2.761803in}}%
\pgfpathlineto{\pgfqpoint{4.354817in}{2.773348in}}%
\pgfpathlineto{\pgfqpoint{4.362204in}{2.785013in}}%
\pgfpathlineto{\pgfqpoint{4.369588in}{2.796802in}}%
\pgfpathlineto{\pgfqpoint{4.376969in}{2.808721in}}%
\pgfpathlineto{\pgfqpoint{4.363979in}{2.811760in}}%
\pgfpathlineto{\pgfqpoint{4.350996in}{2.814972in}}%
\pgfpathlineto{\pgfqpoint{4.338020in}{2.818356in}}%
\pgfpathlineto{\pgfqpoint{4.325050in}{2.821914in}}%
\pgfpathlineto{\pgfqpoint{4.317660in}{2.809601in}}%
\pgfpathlineto{\pgfqpoint{4.310267in}{2.797423in}}%
\pgfpathlineto{\pgfqpoint{4.302870in}{2.785377in}}%
\pgfpathlineto{\pgfqpoint{4.295470in}{2.773458in}}%
\pgfpathclose%
\pgfusepath{fill}%
\end{pgfscope}%
\begin{pgfscope}%
\pgfpathrectangle{\pgfqpoint{1.254980in}{0.150000in}}{\pgfqpoint{5.490039in}{5.490039in}}%
\pgfusepath{clip}%
\pgfsetbuttcap%
\pgfsetroundjoin%
\definecolor{currentfill}{rgb}{0.255645,0.260703,0.528312}%
\pgfsetfillcolor{currentfill}%
\pgfsetfillopacity{0.700000}%
\pgfsetlinewidth{0.000000pt}%
\definecolor{currentstroke}{rgb}{0.000000,0.000000,0.000000}%
\pgfsetstrokecolor{currentstroke}%
\pgfsetdash{}{0pt}%
\pgfpathmoveto{\pgfqpoint{3.332599in}{2.801147in}}%
\pgfpathlineto{\pgfqpoint{3.345492in}{2.788803in}}%
\pgfpathlineto{\pgfqpoint{3.358383in}{2.776696in}}%
\pgfpathlineto{\pgfqpoint{3.371273in}{2.764825in}}%
\pgfpathlineto{\pgfqpoint{3.384160in}{2.753188in}}%
\pgfpathlineto{\pgfqpoint{3.391814in}{2.764991in}}%
\pgfpathlineto{\pgfqpoint{3.399463in}{2.776904in}}%
\pgfpathlineto{\pgfqpoint{3.407106in}{2.788931in}}%
\pgfpathlineto{\pgfqpoint{3.414743in}{2.801073in}}%
\pgfpathlineto{\pgfqpoint{3.401865in}{2.812875in}}%
\pgfpathlineto{\pgfqpoint{3.388985in}{2.824911in}}%
\pgfpathlineto{\pgfqpoint{3.376104in}{2.837183in}}%
\pgfpathlineto{\pgfqpoint{3.363221in}{2.849693in}}%
\pgfpathlineto{\pgfqpoint{3.355574in}{2.837376in}}%
\pgfpathlineto{\pgfqpoint{3.347922in}{2.825180in}}%
\pgfpathlineto{\pgfqpoint{3.340263in}{2.813105in}}%
\pgfpathlineto{\pgfqpoint{3.332599in}{2.801147in}}%
\pgfpathclose%
\pgfusepath{fill}%
\end{pgfscope}%
\begin{pgfscope}%
\pgfpathrectangle{\pgfqpoint{1.254980in}{0.150000in}}{\pgfqpoint{5.490039in}{5.490039in}}%
\pgfusepath{clip}%
\pgfsetbuttcap%
\pgfsetroundjoin%
\definecolor{currentfill}{rgb}{0.175841,0.441290,0.557685}%
\pgfsetfillcolor{currentfill}%
\pgfsetfillopacity{0.700000}%
\pgfsetlinewidth{0.000000pt}%
\definecolor{currentstroke}{rgb}{0.000000,0.000000,0.000000}%
\pgfsetstrokecolor{currentstroke}%
\pgfsetdash{}{0pt}%
\pgfpathmoveto{\pgfqpoint{5.110759in}{3.200382in}}%
\pgfpathlineto{\pgfqpoint{5.123949in}{3.198643in}}%
\pgfpathlineto{\pgfqpoint{5.137150in}{3.197058in}}%
\pgfpathlineto{\pgfqpoint{5.150360in}{3.195626in}}%
\pgfpathlineto{\pgfqpoint{5.163581in}{3.194347in}}%
\pgfpathlineto{\pgfqpoint{5.170800in}{3.208479in}}%
\pgfpathlineto{\pgfqpoint{5.178023in}{3.222904in}}%
\pgfpathlineto{\pgfqpoint{5.185248in}{3.237628in}}%
\pgfpathlineto{\pgfqpoint{5.172041in}{3.239400in}}%
\pgfpathlineto{\pgfqpoint{5.158844in}{3.241324in}}%
\pgfpathlineto{\pgfqpoint{5.145657in}{3.243403in}}%
\pgfpathlineto{\pgfqpoint{5.132480in}{3.245635in}}%
\pgfpathlineto{\pgfqpoint{5.125236in}{3.230247in}}%
\pgfpathlineto{\pgfqpoint{5.117996in}{3.215165in}}%
\pgfpathlineto{\pgfqpoint{5.110759in}{3.200382in}}%
\pgfpathclose%
\pgfusepath{fill}%
\end{pgfscope}%
\begin{pgfscope}%
\pgfpathrectangle{\pgfqpoint{1.254980in}{0.150000in}}{\pgfqpoint{5.490039in}{5.490039in}}%
\pgfusepath{clip}%
\pgfsetbuttcap%
\pgfsetroundjoin%
\definecolor{currentfill}{rgb}{0.185556,0.418570,0.556753}%
\pgfsetfillcolor{currentfill}%
\pgfsetfillopacity{0.700000}%
\pgfsetlinewidth{0.000000pt}%
\definecolor{currentstroke}{rgb}{0.000000,0.000000,0.000000}%
\pgfsetstrokecolor{currentstroke}%
\pgfsetdash{}{0pt}%
\pgfpathmoveto{\pgfqpoint{3.021986in}{3.174055in}}%
\pgfpathlineto{\pgfqpoint{3.034995in}{3.155368in}}%
\pgfpathlineto{\pgfqpoint{3.047996in}{3.136975in}}%
\pgfpathlineto{\pgfqpoint{3.060990in}{3.118870in}}%
\pgfpathlineto{\pgfqpoint{3.073977in}{3.101053in}}%
\pgfpathlineto{\pgfqpoint{3.081695in}{3.113787in}}%
\pgfpathlineto{\pgfqpoint{3.089406in}{3.126675in}}%
\pgfpathlineto{\pgfqpoint{3.097110in}{3.139717in}}%
\pgfpathlineto{\pgfqpoint{3.104807in}{3.152916in}}%
\pgfpathlineto{\pgfqpoint{3.091832in}{3.170901in}}%
\pgfpathlineto{\pgfqpoint{3.078849in}{3.189172in}}%
\pgfpathlineto{\pgfqpoint{3.065859in}{3.207734in}}%
\pgfpathlineto{\pgfqpoint{3.052861in}{3.226588in}}%
\pgfpathlineto{\pgfqpoint{3.045153in}{3.213210in}}%
\pgfpathlineto{\pgfqpoint{3.037438in}{3.199997in}}%
\pgfpathlineto{\pgfqpoint{3.029715in}{3.186946in}}%
\pgfpathlineto{\pgfqpoint{3.021986in}{3.174055in}}%
\pgfpathclose%
\pgfusepath{fill}%
\end{pgfscope}%
\begin{pgfscope}%
\pgfpathrectangle{\pgfqpoint{1.254980in}{0.150000in}}{\pgfqpoint{5.490039in}{5.490039in}}%
\pgfusepath{clip}%
\pgfsetbuttcap%
\pgfsetroundjoin%
\definecolor{currentfill}{rgb}{0.271828,0.209303,0.504434}%
\pgfsetfillcolor{currentfill}%
\pgfsetfillopacity{0.700000}%
\pgfsetlinewidth{0.000000pt}%
\definecolor{currentstroke}{rgb}{0.000000,0.000000,0.000000}%
\pgfsetstrokecolor{currentstroke}%
\pgfsetdash{}{0pt}%
\pgfpathmoveto{\pgfqpoint{3.917530in}{2.672468in}}%
\pgfpathlineto{\pgfqpoint{3.930432in}{2.666923in}}%
\pgfpathlineto{\pgfqpoint{3.943338in}{2.661569in}}%
\pgfpathlineto{\pgfqpoint{3.956249in}{2.656404in}}%
\pgfpathlineto{\pgfqpoint{3.969165in}{2.651426in}}%
\pgfpathlineto{\pgfqpoint{3.976662in}{2.662922in}}%
\pgfpathlineto{\pgfqpoint{3.984155in}{2.674511in}}%
\pgfpathlineto{\pgfqpoint{3.991644in}{2.686196in}}%
\pgfpathlineto{\pgfqpoint{3.999128in}{2.697979in}}%
\pgfpathlineto{\pgfqpoint{3.986221in}{2.703232in}}%
\pgfpathlineto{\pgfqpoint{3.973318in}{2.708672in}}%
\pgfpathlineto{\pgfqpoint{3.960420in}{2.714301in}}%
\pgfpathlineto{\pgfqpoint{3.947525in}{2.720119in}}%
\pgfpathlineto{\pgfqpoint{3.940033in}{2.708051in}}%
\pgfpathlineto{\pgfqpoint{3.932536in}{2.696088in}}%
\pgfpathlineto{\pgfqpoint{3.925035in}{2.684228in}}%
\pgfpathlineto{\pgfqpoint{3.917530in}{2.672468in}}%
\pgfpathclose%
\pgfusepath{fill}%
\end{pgfscope}%
\begin{pgfscope}%
\pgfpathrectangle{\pgfqpoint{1.254980in}{0.150000in}}{\pgfqpoint{5.490039in}{5.490039in}}%
\pgfusepath{clip}%
\pgfsetbuttcap%
\pgfsetroundjoin%
\definecolor{currentfill}{rgb}{0.262138,0.242286,0.520837}%
\pgfsetfillcolor{currentfill}%
\pgfsetfillopacity{0.700000}%
\pgfsetlinewidth{0.000000pt}%
\definecolor{currentstroke}{rgb}{0.000000,0.000000,0.000000}%
\pgfsetstrokecolor{currentstroke}%
\pgfsetdash{}{0pt}%
\pgfpathmoveto{\pgfqpoint{4.213949in}{2.739983in}}%
\pgfpathlineto{\pgfqpoint{4.226910in}{2.736464in}}%
\pgfpathlineto{\pgfqpoint{4.239878in}{2.733122in}}%
\pgfpathlineto{\pgfqpoint{4.252853in}{2.729956in}}%
\pgfpathlineto{\pgfqpoint{4.265834in}{2.726966in}}%
\pgfpathlineto{\pgfqpoint{4.273248in}{2.738420in}}%
\pgfpathlineto{\pgfqpoint{4.280659in}{2.749984in}}%
\pgfpathlineto{\pgfqpoint{4.288066in}{2.761662in}}%
\pgfpathlineto{\pgfqpoint{4.295470in}{2.773458in}}%
\pgfpathlineto{\pgfqpoint{4.282498in}{2.776806in}}%
\pgfpathlineto{\pgfqpoint{4.269532in}{2.780329in}}%
\pgfpathlineto{\pgfqpoint{4.256573in}{2.784028in}}%
\pgfpathlineto{\pgfqpoint{4.243620in}{2.787904in}}%
\pgfpathlineto{\pgfqpoint{4.236208in}{2.775740in}}%
\pgfpathlineto{\pgfqpoint{4.228792in}{2.763702in}}%
\pgfpathlineto{\pgfqpoint{4.221372in}{2.751784in}}%
\pgfpathlineto{\pgfqpoint{4.213949in}{2.739983in}}%
\pgfpathclose%
\pgfusepath{fill}%
\end{pgfscope}%
\begin{pgfscope}%
\pgfpathrectangle{\pgfqpoint{1.254980in}{0.150000in}}{\pgfqpoint{5.490039in}{5.490039in}}%
\pgfusepath{clip}%
\pgfsetbuttcap%
\pgfsetroundjoin%
\definecolor{currentfill}{rgb}{0.262138,0.242286,0.520837}%
\pgfsetfillcolor{currentfill}%
\pgfsetfillopacity{0.700000}%
\pgfsetlinewidth{0.000000pt}%
\definecolor{currentstroke}{rgb}{0.000000,0.000000,0.000000}%
\pgfsetstrokecolor{currentstroke}%
\pgfsetdash{}{0pt}%
\pgfpathmoveto{\pgfqpoint{3.384160in}{2.753188in}}%
\pgfpathlineto{\pgfqpoint{3.397046in}{2.741783in}}%
\pgfpathlineto{\pgfqpoint{3.409931in}{2.730609in}}%
\pgfpathlineto{\pgfqpoint{3.422815in}{2.719665in}}%
\pgfpathlineto{\pgfqpoint{3.435698in}{2.708947in}}%
\pgfpathlineto{\pgfqpoint{3.443342in}{2.720595in}}%
\pgfpathlineto{\pgfqpoint{3.450981in}{2.732348in}}%
\pgfpathlineto{\pgfqpoint{3.458615in}{2.744206in}}%
\pgfpathlineto{\pgfqpoint{3.466242in}{2.756173in}}%
\pgfpathlineto{\pgfqpoint{3.453369in}{2.767055in}}%
\pgfpathlineto{\pgfqpoint{3.440495in}{2.778165in}}%
\pgfpathlineto{\pgfqpoint{3.427619in}{2.789503in}}%
\pgfpathlineto{\pgfqpoint{3.414743in}{2.801073in}}%
\pgfpathlineto{\pgfqpoint{3.407106in}{2.788931in}}%
\pgfpathlineto{\pgfqpoint{3.399463in}{2.776904in}}%
\pgfpathlineto{\pgfqpoint{3.391814in}{2.764991in}}%
\pgfpathlineto{\pgfqpoint{3.384160in}{2.753188in}}%
\pgfpathclose%
\pgfusepath{fill}%
\end{pgfscope}%
\begin{pgfscope}%
\pgfpathrectangle{\pgfqpoint{1.254980in}{0.150000in}}{\pgfqpoint{5.490039in}{5.490039in}}%
\pgfusepath{clip}%
\pgfsetbuttcap%
\pgfsetroundjoin%
\definecolor{currentfill}{rgb}{0.274128,0.199721,0.498911}%
\pgfsetfillcolor{currentfill}%
\pgfsetfillopacity{0.700000}%
\pgfsetlinewidth{0.000000pt}%
\definecolor{currentstroke}{rgb}{0.000000,0.000000,0.000000}%
\pgfsetstrokecolor{currentstroke}%
\pgfsetdash{}{0pt}%
\pgfpathmoveto{\pgfqpoint{3.702578in}{2.657669in}}%
\pgfpathlineto{\pgfqpoint{3.715457in}{2.650235in}}%
\pgfpathlineto{\pgfqpoint{3.728339in}{2.643004in}}%
\pgfpathlineto{\pgfqpoint{3.741223in}{2.635974in}}%
\pgfpathlineto{\pgfqpoint{3.754110in}{2.629144in}}%
\pgfpathlineto{\pgfqpoint{3.761669in}{2.640641in}}%
\pgfpathlineto{\pgfqpoint{3.769223in}{2.652227in}}%
\pgfpathlineto{\pgfqpoint{3.776772in}{2.663905in}}%
\pgfpathlineto{\pgfqpoint{3.784316in}{2.675677in}}%
\pgfpathlineto{\pgfqpoint{3.771438in}{2.682726in}}%
\pgfpathlineto{\pgfqpoint{3.758562in}{2.689976in}}%
\pgfpathlineto{\pgfqpoint{3.745688in}{2.697427in}}%
\pgfpathlineto{\pgfqpoint{3.732817in}{2.705080in}}%
\pgfpathlineto{\pgfqpoint{3.725265in}{2.693078in}}%
\pgfpathlineto{\pgfqpoint{3.717707in}{2.681177in}}%
\pgfpathlineto{\pgfqpoint{3.710145in}{2.669375in}}%
\pgfpathlineto{\pgfqpoint{3.702578in}{2.657669in}}%
\pgfpathclose%
\pgfusepath{fill}%
\end{pgfscope}%
\begin{pgfscope}%
\pgfpathrectangle{\pgfqpoint{1.254980in}{0.150000in}}{\pgfqpoint{5.490039in}{5.490039in}}%
\pgfusepath{clip}%
\pgfsetbuttcap%
\pgfsetroundjoin%
\definecolor{currentfill}{rgb}{0.271828,0.209303,0.504434}%
\pgfsetfillcolor{currentfill}%
\pgfsetfillopacity{0.700000}%
\pgfsetlinewidth{0.000000pt}%
\definecolor{currentstroke}{rgb}{0.000000,0.000000,0.000000}%
\pgfsetstrokecolor{currentstroke}%
\pgfsetdash{}{0pt}%
\pgfpathmoveto{\pgfqpoint{3.569224in}{2.677113in}}%
\pgfpathlineto{\pgfqpoint{3.582099in}{2.668208in}}%
\pgfpathlineto{\pgfqpoint{3.594975in}{2.659515in}}%
\pgfpathlineto{\pgfqpoint{3.607852in}{2.651034in}}%
\pgfpathlineto{\pgfqpoint{3.620730in}{2.642763in}}%
\pgfpathlineto{\pgfqpoint{3.628326in}{2.654294in}}%
\pgfpathlineto{\pgfqpoint{3.635916in}{2.665917in}}%
\pgfpathlineto{\pgfqpoint{3.643502in}{2.677636in}}%
\pgfpathlineto{\pgfqpoint{3.651082in}{2.689451in}}%
\pgfpathlineto{\pgfqpoint{3.638212in}{2.697915in}}%
\pgfpathlineto{\pgfqpoint{3.625344in}{2.706589in}}%
\pgfpathlineto{\pgfqpoint{3.612477in}{2.715474in}}%
\pgfpathlineto{\pgfqpoint{3.599611in}{2.724571in}}%
\pgfpathlineto{\pgfqpoint{3.592022in}{2.712552in}}%
\pgfpathlineto{\pgfqpoint{3.584428in}{2.700638in}}%
\pgfpathlineto{\pgfqpoint{3.576829in}{2.688826in}}%
\pgfpathlineto{\pgfqpoint{3.569224in}{2.677113in}}%
\pgfpathclose%
\pgfusepath{fill}%
\end{pgfscope}%
\begin{pgfscope}%
\pgfpathrectangle{\pgfqpoint{1.254980in}{0.150000in}}{\pgfqpoint{5.490039in}{5.490039in}}%
\pgfusepath{clip}%
\pgfsetbuttcap%
\pgfsetroundjoin%
\definecolor{currentfill}{rgb}{0.266580,0.228262,0.514349}%
\pgfsetfillcolor{currentfill}%
\pgfsetfillopacity{0.700000}%
\pgfsetlinewidth{0.000000pt}%
\definecolor{currentstroke}{rgb}{0.000000,0.000000,0.000000}%
\pgfsetstrokecolor{currentstroke}%
\pgfsetdash{}{0pt}%
\pgfpathmoveto{\pgfqpoint{4.132397in}{2.708399in}}%
\pgfpathlineto{\pgfqpoint{4.145343in}{2.704497in}}%
\pgfpathlineto{\pgfqpoint{4.158295in}{2.700774in}}%
\pgfpathlineto{\pgfqpoint{4.171253in}{2.697230in}}%
\pgfpathlineto{\pgfqpoint{4.184217in}{2.693864in}}%
\pgfpathlineto{\pgfqpoint{4.191656in}{2.705239in}}%
\pgfpathlineto{\pgfqpoint{4.199091in}{2.716714in}}%
\pgfpathlineto{\pgfqpoint{4.206522in}{2.728294in}}%
\pgfpathlineto{\pgfqpoint{4.213949in}{2.739983in}}%
\pgfpathlineto{\pgfqpoint{4.200993in}{2.743679in}}%
\pgfpathlineto{\pgfqpoint{4.188044in}{2.747553in}}%
\pgfpathlineto{\pgfqpoint{4.175100in}{2.751605in}}%
\pgfpathlineto{\pgfqpoint{4.162163in}{2.755837in}}%
\pgfpathlineto{\pgfqpoint{4.154727in}{2.743809in}}%
\pgfpathlineto{\pgfqpoint{4.147288in}{2.731895in}}%
\pgfpathlineto{\pgfqpoint{4.139844in}{2.720094in}}%
\pgfpathlineto{\pgfqpoint{4.132397in}{2.708399in}}%
\pgfpathclose%
\pgfusepath{fill}%
\end{pgfscope}%
\begin{pgfscope}%
\pgfpathrectangle{\pgfqpoint{1.254980in}{0.150000in}}{\pgfqpoint{5.490039in}{5.490039in}}%
\pgfusepath{clip}%
\pgfsetbuttcap%
\pgfsetroundjoin%
\definecolor{currentfill}{rgb}{0.171176,0.452530,0.557965}%
\pgfsetfillcolor{currentfill}%
\pgfsetfillopacity{0.700000}%
\pgfsetlinewidth{0.000000pt}%
\definecolor{currentstroke}{rgb}{0.000000,0.000000,0.000000}%
\pgfsetstrokecolor{currentstroke}%
\pgfsetdash{}{0pt}%
\pgfpathmoveto{\pgfqpoint{2.969867in}{3.251782in}}%
\pgfpathlineto{\pgfqpoint{2.982910in}{3.231897in}}%
\pgfpathlineto{\pgfqpoint{2.995943in}{3.212316in}}%
\pgfpathlineto{\pgfqpoint{3.008969in}{3.193036in}}%
\pgfpathlineto{\pgfqpoint{3.021986in}{3.174055in}}%
\pgfpathlineto{\pgfqpoint{3.029715in}{3.186946in}}%
\pgfpathlineto{\pgfqpoint{3.037438in}{3.199997in}}%
\pgfpathlineto{\pgfqpoint{3.045153in}{3.213210in}}%
\pgfpathlineto{\pgfqpoint{3.052861in}{3.226588in}}%
\pgfpathlineto{\pgfqpoint{3.039856in}{3.245737in}}%
\pgfpathlineto{\pgfqpoint{3.026842in}{3.265185in}}%
\pgfpathlineto{\pgfqpoint{3.013820in}{3.284933in}}%
\pgfpathlineto{\pgfqpoint{3.000790in}{3.304987in}}%
\pgfpathlineto{\pgfqpoint{2.993070in}{3.291430in}}%
\pgfpathlineto{\pgfqpoint{2.985343in}{3.278046in}}%
\pgfpathlineto{\pgfqpoint{2.977609in}{3.264830in}}%
\pgfpathlineto{\pgfqpoint{2.969867in}{3.251782in}}%
\pgfpathclose%
\pgfusepath{fill}%
\end{pgfscope}%
\begin{pgfscope}%
\pgfpathrectangle{\pgfqpoint{1.254980in}{0.150000in}}{\pgfqpoint{5.490039in}{5.490039in}}%
\pgfusepath{clip}%
\pgfsetbuttcap%
\pgfsetroundjoin%
\definecolor{currentfill}{rgb}{0.274128,0.199721,0.498911}%
\pgfsetfillcolor{currentfill}%
\pgfsetfillopacity{0.700000}%
\pgfsetlinewidth{0.000000pt}%
\definecolor{currentstroke}{rgb}{0.000000,0.000000,0.000000}%
\pgfsetstrokecolor{currentstroke}%
\pgfsetdash{}{0pt}%
\pgfpathmoveto{\pgfqpoint{3.835862in}{2.649457in}}%
\pgfpathlineto{\pgfqpoint{3.848756in}{2.643392in}}%
\pgfpathlineto{\pgfqpoint{3.861655in}{2.637520in}}%
\pgfpathlineto{\pgfqpoint{3.874557in}{2.631840in}}%
\pgfpathlineto{\pgfqpoint{3.887463in}{2.626353in}}%
\pgfpathlineto{\pgfqpoint{3.894987in}{2.637748in}}%
\pgfpathlineto{\pgfqpoint{3.902505in}{2.649231in}}%
\pgfpathlineto{\pgfqpoint{3.910020in}{2.660803in}}%
\pgfpathlineto{\pgfqpoint{3.917530in}{2.672468in}}%
\pgfpathlineto{\pgfqpoint{3.904632in}{2.678202in}}%
\pgfpathlineto{\pgfqpoint{3.891738in}{2.684129in}}%
\pgfpathlineto{\pgfqpoint{3.878847in}{2.690248in}}%
\pgfpathlineto{\pgfqpoint{3.865961in}{2.696561in}}%
\pgfpathlineto{\pgfqpoint{3.858443in}{2.684638in}}%
\pgfpathlineto{\pgfqpoint{3.850920in}{2.672816in}}%
\pgfpathlineto{\pgfqpoint{3.843393in}{2.661090in}}%
\pgfpathlineto{\pgfqpoint{3.835862in}{2.649457in}}%
\pgfpathclose%
\pgfusepath{fill}%
\end{pgfscope}%
\begin{pgfscope}%
\pgfpathrectangle{\pgfqpoint{1.254980in}{0.150000in}}{\pgfqpoint{5.490039in}{5.490039in}}%
\pgfusepath{clip}%
\pgfsetbuttcap%
\pgfsetroundjoin%
\definecolor{currentfill}{rgb}{0.218130,0.347432,0.550038}%
\pgfsetfillcolor{currentfill}%
\pgfsetfillopacity{0.700000}%
\pgfsetlinewidth{0.000000pt}%
\definecolor{currentstroke}{rgb}{0.000000,0.000000,0.000000}%
\pgfsetstrokecolor{currentstroke}%
\pgfsetdash{}{0pt}%
\pgfpathmoveto{\pgfqpoint{4.755274in}{2.959291in}}%
\pgfpathlineto{\pgfqpoint{4.768389in}{2.957926in}}%
\pgfpathlineto{\pgfqpoint{4.781513in}{2.956721in}}%
\pgfpathlineto{\pgfqpoint{4.794647in}{2.955676in}}%
\pgfpathlineto{\pgfqpoint{4.807790in}{2.954792in}}%
\pgfpathlineto{\pgfqpoint{4.815061in}{2.966612in}}%
\pgfpathlineto{\pgfqpoint{4.822331in}{2.978617in}}%
\pgfpathlineto{\pgfqpoint{4.829600in}{2.990811in}}%
\pgfpathlineto{\pgfqpoint{4.836869in}{3.003202in}}%
\pgfpathlineto{\pgfqpoint{4.823740in}{3.004610in}}%
\pgfpathlineto{\pgfqpoint{4.810620in}{3.006178in}}%
\pgfpathlineto{\pgfqpoint{4.797509in}{3.007905in}}%
\pgfpathlineto{\pgfqpoint{4.784408in}{3.009794in}}%
\pgfpathlineto{\pgfqpoint{4.777126in}{2.996870in}}%
\pgfpathlineto{\pgfqpoint{4.769843in}{2.984149in}}%
\pgfpathlineto{\pgfqpoint{4.762559in}{2.971625in}}%
\pgfpathlineto{\pgfqpoint{4.755274in}{2.959291in}}%
\pgfpathclose%
\pgfusepath{fill}%
\end{pgfscope}%
\begin{pgfscope}%
\pgfpathrectangle{\pgfqpoint{1.254980in}{0.150000in}}{\pgfqpoint{5.490039in}{5.490039in}}%
\pgfusepath{clip}%
\pgfsetbuttcap%
\pgfsetroundjoin%
\definecolor{currentfill}{rgb}{0.267968,0.223549,0.512008}%
\pgfsetfillcolor{currentfill}%
\pgfsetfillopacity{0.700000}%
\pgfsetlinewidth{0.000000pt}%
\definecolor{currentstroke}{rgb}{0.000000,0.000000,0.000000}%
\pgfsetstrokecolor{currentstroke}%
\pgfsetdash{}{0pt}%
\pgfpathmoveto{\pgfqpoint{3.435698in}{2.708947in}}%
\pgfpathlineto{\pgfqpoint{3.448580in}{2.698456in}}%
\pgfpathlineto{\pgfqpoint{3.461462in}{2.688188in}}%
\pgfpathlineto{\pgfqpoint{3.474344in}{2.678143in}}%
\pgfpathlineto{\pgfqpoint{3.487225in}{2.668320in}}%
\pgfpathlineto{\pgfqpoint{3.494860in}{2.679813in}}%
\pgfpathlineto{\pgfqpoint{3.502490in}{2.691404in}}%
\pgfpathlineto{\pgfqpoint{3.510114in}{2.703094in}}%
\pgfpathlineto{\pgfqpoint{3.517732in}{2.714885in}}%
\pgfpathlineto{\pgfqpoint{3.504860in}{2.724874in}}%
\pgfpathlineto{\pgfqpoint{3.491988in}{2.735084in}}%
\pgfpathlineto{\pgfqpoint{3.479115in}{2.745516in}}%
\pgfpathlineto{\pgfqpoint{3.466242in}{2.756173in}}%
\pgfpathlineto{\pgfqpoint{3.458615in}{2.744206in}}%
\pgfpathlineto{\pgfqpoint{3.450981in}{2.732348in}}%
\pgfpathlineto{\pgfqpoint{3.443342in}{2.720595in}}%
\pgfpathlineto{\pgfqpoint{3.435698in}{2.708947in}}%
\pgfpathclose%
\pgfusepath{fill}%
\end{pgfscope}%
\begin{pgfscope}%
\pgfpathrectangle{\pgfqpoint{1.254980in}{0.150000in}}{\pgfqpoint{5.490039in}{5.490039in}}%
\pgfusepath{clip}%
\pgfsetbuttcap%
\pgfsetroundjoin%
\definecolor{currentfill}{rgb}{0.225863,0.330805,0.547314}%
\pgfsetfillcolor{currentfill}%
\pgfsetfillopacity{0.700000}%
\pgfsetlinewidth{0.000000pt}%
\definecolor{currentstroke}{rgb}{0.000000,0.000000,0.000000}%
\pgfsetstrokecolor{currentstroke}%
\pgfsetdash{}{0pt}%
\pgfpathmoveto{\pgfqpoint{4.673697in}{2.916840in}}%
\pgfpathlineto{\pgfqpoint{4.686789in}{2.915323in}}%
\pgfpathlineto{\pgfqpoint{4.699890in}{2.913969in}}%
\pgfpathlineto{\pgfqpoint{4.713000in}{2.912777in}}%
\pgfpathlineto{\pgfqpoint{4.726119in}{2.911746in}}%
\pgfpathlineto{\pgfqpoint{4.733410in}{2.923375in}}%
\pgfpathlineto{\pgfqpoint{4.740700in}{2.935172in}}%
\pgfpathlineto{\pgfqpoint{4.747988in}{2.947142in}}%
\pgfpathlineto{\pgfqpoint{4.755274in}{2.959291in}}%
\pgfpathlineto{\pgfqpoint{4.742168in}{2.960818in}}%
\pgfpathlineto{\pgfqpoint{4.729071in}{2.962505in}}%
\pgfpathlineto{\pgfqpoint{4.715982in}{2.964355in}}%
\pgfpathlineto{\pgfqpoint{4.702903in}{2.966367in}}%
\pgfpathlineto{\pgfqpoint{4.695603in}{2.953713in}}%
\pgfpathlineto{\pgfqpoint{4.688303in}{2.941244in}}%
\pgfpathlineto{\pgfqpoint{4.681001in}{2.928955in}}%
\pgfpathlineto{\pgfqpoint{4.673697in}{2.916840in}}%
\pgfpathclose%
\pgfusepath{fill}%
\end{pgfscope}%
\begin{pgfscope}%
\pgfpathrectangle{\pgfqpoint{1.254980in}{0.150000in}}{\pgfqpoint{5.490039in}{5.490039in}}%
\pgfusepath{clip}%
\pgfsetbuttcap%
\pgfsetroundjoin%
\definecolor{currentfill}{rgb}{0.208623,0.367752,0.552675}%
\pgfsetfillcolor{currentfill}%
\pgfsetfillopacity{0.700000}%
\pgfsetlinewidth{0.000000pt}%
\definecolor{currentstroke}{rgb}{0.000000,0.000000,0.000000}%
\pgfsetstrokecolor{currentstroke}%
\pgfsetdash{}{0pt}%
\pgfpathmoveto{\pgfqpoint{4.836869in}{3.003202in}}%
\pgfpathlineto{\pgfqpoint{4.850007in}{3.001953in}}%
\pgfpathlineto{\pgfqpoint{4.863155in}{3.000862in}}%
\pgfpathlineto{\pgfqpoint{4.876312in}{2.999931in}}%
\pgfpathlineto{\pgfqpoint{4.889479in}{2.999157in}}%
\pgfpathlineto{\pgfqpoint{4.896732in}{3.011209in}}%
\pgfpathlineto{\pgfqpoint{4.903984in}{3.023463in}}%
\pgfpathlineto{\pgfqpoint{4.911237in}{3.035925in}}%
\pgfpathlineto{\pgfqpoint{4.918488in}{3.048601in}}%
\pgfpathlineto{\pgfqpoint{4.905336in}{3.049926in}}%
\pgfpathlineto{\pgfqpoint{4.892194in}{3.051408in}}%
\pgfpathlineto{\pgfqpoint{4.879061in}{3.053049in}}%
\pgfpathlineto{\pgfqpoint{4.865937in}{3.054848in}}%
\pgfpathlineto{\pgfqpoint{4.858670in}{3.041611in}}%
\pgfpathlineto{\pgfqpoint{4.851403in}{3.028595in}}%
\pgfpathlineto{\pgfqpoint{4.844136in}{3.015794in}}%
\pgfpathlineto{\pgfqpoint{4.836869in}{3.003202in}}%
\pgfpathclose%
\pgfusepath{fill}%
\end{pgfscope}%
\begin{pgfscope}%
\pgfpathrectangle{\pgfqpoint{1.254980in}{0.150000in}}{\pgfqpoint{5.490039in}{5.490039in}}%
\pgfusepath{clip}%
\pgfsetbuttcap%
\pgfsetroundjoin%
\definecolor{currentfill}{rgb}{0.270595,0.214069,0.507052}%
\pgfsetfillcolor{currentfill}%
\pgfsetfillopacity{0.700000}%
\pgfsetlinewidth{0.000000pt}%
\definecolor{currentstroke}{rgb}{0.000000,0.000000,0.000000}%
\pgfsetstrokecolor{currentstroke}%
\pgfsetdash{}{0pt}%
\pgfpathmoveto{\pgfqpoint{4.050806in}{2.678831in}}%
\pgfpathlineto{\pgfqpoint{4.063738in}{2.674505in}}%
\pgfpathlineto{\pgfqpoint{4.076675in}{2.670362in}}%
\pgfpathlineto{\pgfqpoint{4.089618in}{2.666400in}}%
\pgfpathlineto{\pgfqpoint{4.102567in}{2.662620in}}%
\pgfpathlineto{\pgfqpoint{4.110031in}{2.673922in}}%
\pgfpathlineto{\pgfqpoint{4.117490in}{2.685317in}}%
\pgfpathlineto{\pgfqpoint{4.124946in}{2.696808in}}%
\pgfpathlineto{\pgfqpoint{4.132397in}{2.708399in}}%
\pgfpathlineto{\pgfqpoint{4.119457in}{2.712482in}}%
\pgfpathlineto{\pgfqpoint{4.106522in}{2.716746in}}%
\pgfpathlineto{\pgfqpoint{4.093593in}{2.721191in}}%
\pgfpathlineto{\pgfqpoint{4.080669in}{2.725819in}}%
\pgfpathlineto{\pgfqpoint{4.073209in}{2.713916in}}%
\pgfpathlineto{\pgfqpoint{4.065746in}{2.702119in}}%
\pgfpathlineto{\pgfqpoint{4.058278in}{2.690426in}}%
\pgfpathlineto{\pgfqpoint{4.050806in}{2.678831in}}%
\pgfpathclose%
\pgfusepath{fill}%
\end{pgfscope}%
\begin{pgfscope}%
\pgfpathrectangle{\pgfqpoint{1.254980in}{0.150000in}}{\pgfqpoint{5.490039in}{5.490039in}}%
\pgfusepath{clip}%
\pgfsetbuttcap%
\pgfsetroundjoin%
\definecolor{currentfill}{rgb}{0.233603,0.313828,0.543914}%
\pgfsetfillcolor{currentfill}%
\pgfsetfillopacity{0.700000}%
\pgfsetlinewidth{0.000000pt}%
\definecolor{currentstroke}{rgb}{0.000000,0.000000,0.000000}%
\pgfsetstrokecolor{currentstroke}%
\pgfsetdash{}{0pt}%
\pgfpathmoveto{\pgfqpoint{4.592130in}{2.875842in}}%
\pgfpathlineto{\pgfqpoint{4.605200in}{2.874138in}}%
\pgfpathlineto{\pgfqpoint{4.618278in}{2.872599in}}%
\pgfpathlineto{\pgfqpoint{4.631365in}{2.871223in}}%
\pgfpathlineto{\pgfqpoint{4.644461in}{2.870012in}}%
\pgfpathlineto{\pgfqpoint{4.651773in}{2.881485in}}%
\pgfpathlineto{\pgfqpoint{4.659083in}{2.893111in}}%
\pgfpathlineto{\pgfqpoint{4.666391in}{2.904894in}}%
\pgfpathlineto{\pgfqpoint{4.673697in}{2.916840in}}%
\pgfpathlineto{\pgfqpoint{4.660613in}{2.918520in}}%
\pgfpathlineto{\pgfqpoint{4.647538in}{2.920363in}}%
\pgfpathlineto{\pgfqpoint{4.634472in}{2.922370in}}%
\pgfpathlineto{\pgfqpoint{4.621414in}{2.924542in}}%
\pgfpathlineto{\pgfqpoint{4.614096in}{2.912118in}}%
\pgfpathlineto{\pgfqpoint{4.606776in}{2.899863in}}%
\pgfpathlineto{\pgfqpoint{4.599454in}{2.887773in}}%
\pgfpathlineto{\pgfqpoint{4.592130in}{2.875842in}}%
\pgfpathclose%
\pgfusepath{fill}%
\end{pgfscope}%
\begin{pgfscope}%
\pgfpathrectangle{\pgfqpoint{1.254980in}{0.150000in}}{\pgfqpoint{5.490039in}{5.490039in}}%
\pgfusepath{clip}%
\pgfsetbuttcap%
\pgfsetroundjoin%
\definecolor{currentfill}{rgb}{0.199430,0.387607,0.554642}%
\pgfsetfillcolor{currentfill}%
\pgfsetfillopacity{0.700000}%
\pgfsetlinewidth{0.000000pt}%
\definecolor{currentstroke}{rgb}{0.000000,0.000000,0.000000}%
\pgfsetstrokecolor{currentstroke}%
\pgfsetdash{}{0pt}%
\pgfpathmoveto{\pgfqpoint{4.918488in}{3.048601in}}%
\pgfpathlineto{\pgfqpoint{4.931650in}{3.047434in}}%
\pgfpathlineto{\pgfqpoint{4.944821in}{3.046424in}}%
\pgfpathlineto{\pgfqpoint{4.958002in}{3.045571in}}%
\pgfpathlineto{\pgfqpoint{4.971194in}{3.044874in}}%
\pgfpathlineto{\pgfqpoint{4.978430in}{3.057203in}}%
\pgfpathlineto{\pgfqpoint{4.985667in}{3.069752in}}%
\pgfpathlineto{\pgfqpoint{4.992904in}{3.082529in}}%
\pgfpathlineto{\pgfqpoint{5.000141in}{3.095540in}}%
\pgfpathlineto{\pgfqpoint{4.986966in}{3.096815in}}%
\pgfpathlineto{\pgfqpoint{4.973801in}{3.098247in}}%
\pgfpathlineto{\pgfqpoint{4.960645in}{3.099835in}}%
\pgfpathlineto{\pgfqpoint{4.947499in}{3.101580in}}%
\pgfpathlineto{\pgfqpoint{4.940245in}{3.087981in}}%
\pgfpathlineto{\pgfqpoint{4.932993in}{3.074623in}}%
\pgfpathlineto{\pgfqpoint{4.925741in}{3.061498in}}%
\pgfpathlineto{\pgfqpoint{4.918488in}{3.048601in}}%
\pgfpathclose%
\pgfusepath{fill}%
\end{pgfscope}%
\begin{pgfscope}%
\pgfpathrectangle{\pgfqpoint{1.254980in}{0.150000in}}{\pgfqpoint{5.490039in}{5.490039in}}%
\pgfusepath{clip}%
\pgfsetbuttcap%
\pgfsetroundjoin%
\definecolor{currentfill}{rgb}{0.243113,0.292092,0.538516}%
\pgfsetfillcolor{currentfill}%
\pgfsetfillopacity{0.700000}%
\pgfsetlinewidth{0.000000pt}%
\definecolor{currentstroke}{rgb}{0.000000,0.000000,0.000000}%
\pgfsetstrokecolor{currentstroke}%
\pgfsetdash{}{0pt}%
\pgfpathmoveto{\pgfqpoint{4.510567in}{2.836309in}}%
\pgfpathlineto{\pgfqpoint{4.523615in}{2.834382in}}%
\pgfpathlineto{\pgfqpoint{4.536671in}{2.832622in}}%
\pgfpathlineto{\pgfqpoint{4.549736in}{2.831028in}}%
\pgfpathlineto{\pgfqpoint{4.562808in}{2.829600in}}%
\pgfpathlineto{\pgfqpoint{4.570143in}{2.840948in}}%
\pgfpathlineto{\pgfqpoint{4.577475in}{2.852435in}}%
\pgfpathlineto{\pgfqpoint{4.584804in}{2.864064in}}%
\pgfpathlineto{\pgfqpoint{4.592130in}{2.875842in}}%
\pgfpathlineto{\pgfqpoint{4.579069in}{2.877710in}}%
\pgfpathlineto{\pgfqpoint{4.566016in}{2.879745in}}%
\pgfpathlineto{\pgfqpoint{4.552971in}{2.881945in}}%
\pgfpathlineto{\pgfqpoint{4.539934in}{2.884312in}}%
\pgfpathlineto{\pgfqpoint{4.532596in}{2.872083in}}%
\pgfpathlineto{\pgfqpoint{4.525256in}{2.860010in}}%
\pgfpathlineto{\pgfqpoint{4.517913in}{2.848087in}}%
\pgfpathlineto{\pgfqpoint{4.510567in}{2.836309in}}%
\pgfpathclose%
\pgfusepath{fill}%
\end{pgfscope}%
\begin{pgfscope}%
\pgfpathrectangle{\pgfqpoint{1.254980in}{0.150000in}}{\pgfqpoint{5.490039in}{5.490039in}}%
\pgfusepath{clip}%
\pgfsetbuttcap%
\pgfsetroundjoin%
\definecolor{currentfill}{rgb}{0.229739,0.322361,0.545706}%
\pgfsetfillcolor{currentfill}%
\pgfsetfillopacity{0.700000}%
\pgfsetlinewidth{0.000000pt}%
\definecolor{currentstroke}{rgb}{0.000000,0.000000,0.000000}%
\pgfsetstrokecolor{currentstroke}%
\pgfsetdash{}{0pt}%
\pgfpathmoveto{\pgfqpoint{3.146799in}{2.920185in}}%
\pgfpathlineto{\pgfqpoint{3.159744in}{2.904972in}}%
\pgfpathlineto{\pgfqpoint{3.172685in}{2.890021in}}%
\pgfpathlineto{\pgfqpoint{3.185621in}{2.875330in}}%
\pgfpathlineto{\pgfqpoint{3.198553in}{2.860897in}}%
\pgfpathlineto{\pgfqpoint{3.206264in}{2.872658in}}%
\pgfpathlineto{\pgfqpoint{3.213968in}{2.884543in}}%
\pgfpathlineto{\pgfqpoint{3.221665in}{2.896553in}}%
\pgfpathlineto{\pgfqpoint{3.229356in}{2.908691in}}%
\pgfpathlineto{\pgfqpoint{3.216435in}{2.923262in}}%
\pgfpathlineto{\pgfqpoint{3.203510in}{2.938090in}}%
\pgfpathlineto{\pgfqpoint{3.190581in}{2.953179in}}%
\pgfpathlineto{\pgfqpoint{3.177647in}{2.968529in}}%
\pgfpathlineto{\pgfqpoint{3.169945in}{2.956243in}}%
\pgfpathlineto{\pgfqpoint{3.162236in}{2.944092in}}%
\pgfpathlineto{\pgfqpoint{3.154521in}{2.932073in}}%
\pgfpathlineto{\pgfqpoint{3.146799in}{2.920185in}}%
\pgfpathclose%
\pgfusepath{fill}%
\end{pgfscope}%
\begin{pgfscope}%
\pgfpathrectangle{\pgfqpoint{1.254980in}{0.150000in}}{\pgfqpoint{5.490039in}{5.490039in}}%
\pgfusepath{clip}%
\pgfsetbuttcap%
\pgfsetroundjoin%
\definecolor{currentfill}{rgb}{0.190631,0.407061,0.556089}%
\pgfsetfillcolor{currentfill}%
\pgfsetfillopacity{0.700000}%
\pgfsetlinewidth{0.000000pt}%
\definecolor{currentstroke}{rgb}{0.000000,0.000000,0.000000}%
\pgfsetstrokecolor{currentstroke}%
\pgfsetdash{}{0pt}%
\pgfpathmoveto{\pgfqpoint{5.000141in}{3.095540in}}%
\pgfpathlineto{\pgfqpoint{5.013326in}{3.094420in}}%
\pgfpathlineto{\pgfqpoint{5.026521in}{3.093457in}}%
\pgfpathlineto{\pgfqpoint{5.039726in}{3.092648in}}%
\pgfpathlineto{\pgfqpoint{5.052941in}{3.091995in}}%
\pgfpathlineto{\pgfqpoint{5.060163in}{3.104650in}}%
\pgfpathlineto{\pgfqpoint{5.067386in}{3.117547in}}%
\pgfpathlineto{\pgfqpoint{5.074610in}{3.130691in}}%
\pgfpathlineto{\pgfqpoint{5.081836in}{3.144091in}}%
\pgfpathlineto{\pgfqpoint{5.068638in}{3.145351in}}%
\pgfpathlineto{\pgfqpoint{5.055450in}{3.146766in}}%
\pgfpathlineto{\pgfqpoint{5.042271in}{3.148336in}}%
\pgfpathlineto{\pgfqpoint{5.029103in}{3.150061in}}%
\pgfpathlineto{\pgfqpoint{5.021860in}{3.136045in}}%
\pgfpathlineto{\pgfqpoint{5.014619in}{3.122291in}}%
\pgfpathlineto{\pgfqpoint{5.007380in}{3.108792in}}%
\pgfpathlineto{\pgfqpoint{5.000141in}{3.095540in}}%
\pgfpathclose%
\pgfusepath{fill}%
\end{pgfscope}%
\begin{pgfscope}%
\pgfpathrectangle{\pgfqpoint{1.254980in}{0.150000in}}{\pgfqpoint{5.490039in}{5.490039in}}%
\pgfusepath{clip}%
\pgfsetbuttcap%
\pgfsetroundjoin%
\definecolor{currentfill}{rgb}{0.218130,0.347432,0.550038}%
\pgfsetfillcolor{currentfill}%
\pgfsetfillopacity{0.700000}%
\pgfsetlinewidth{0.000000pt}%
\definecolor{currentstroke}{rgb}{0.000000,0.000000,0.000000}%
\pgfsetstrokecolor{currentstroke}%
\pgfsetdash{}{0pt}%
\pgfpathmoveto{\pgfqpoint{3.094964in}{2.983702in}}%
\pgfpathlineto{\pgfqpoint{3.107931in}{2.967418in}}%
\pgfpathlineto{\pgfqpoint{3.120892in}{2.951405in}}%
\pgfpathlineto{\pgfqpoint{3.133848in}{2.935662in}}%
\pgfpathlineto{\pgfqpoint{3.146799in}{2.920185in}}%
\pgfpathlineto{\pgfqpoint{3.154521in}{2.932073in}}%
\pgfpathlineto{\pgfqpoint{3.162236in}{2.944092in}}%
\pgfpathlineto{\pgfqpoint{3.169945in}{2.956243in}}%
\pgfpathlineto{\pgfqpoint{3.177647in}{2.968529in}}%
\pgfpathlineto{\pgfqpoint{3.164708in}{2.984144in}}%
\pgfpathlineto{\pgfqpoint{3.151763in}{3.000026in}}%
\pgfpathlineto{\pgfqpoint{3.138814in}{3.016177in}}%
\pgfpathlineto{\pgfqpoint{3.125859in}{3.032599in}}%
\pgfpathlineto{\pgfqpoint{3.118145in}{3.020164in}}%
\pgfpathlineto{\pgfqpoint{3.110425in}{3.007871in}}%
\pgfpathlineto{\pgfqpoint{3.102698in}{2.995718in}}%
\pgfpathlineto{\pgfqpoint{3.094964in}{2.983702in}}%
\pgfpathclose%
\pgfusepath{fill}%
\end{pgfscope}%
\begin{pgfscope}%
\pgfpathrectangle{\pgfqpoint{1.254980in}{0.150000in}}{\pgfqpoint{5.490039in}{5.490039in}}%
\pgfusepath{clip}%
\pgfsetbuttcap%
\pgfsetroundjoin%
\definecolor{currentfill}{rgb}{0.241237,0.296485,0.539709}%
\pgfsetfillcolor{currentfill}%
\pgfsetfillopacity{0.700000}%
\pgfsetlinewidth{0.000000pt}%
\definecolor{currentstroke}{rgb}{0.000000,0.000000,0.000000}%
\pgfsetstrokecolor{currentstroke}%
\pgfsetdash{}{0pt}%
\pgfpathmoveto{\pgfqpoint{3.198553in}{2.860897in}}%
\pgfpathlineto{\pgfqpoint{3.211481in}{2.846719in}}%
\pgfpathlineto{\pgfqpoint{3.224406in}{2.832795in}}%
\pgfpathlineto{\pgfqpoint{3.237326in}{2.819122in}}%
\pgfpathlineto{\pgfqpoint{3.250243in}{2.805698in}}%
\pgfpathlineto{\pgfqpoint{3.257943in}{2.817331in}}%
\pgfpathlineto{\pgfqpoint{3.265635in}{2.829082in}}%
\pgfpathlineto{\pgfqpoint{3.273322in}{2.840951in}}%
\pgfpathlineto{\pgfqpoint{3.281003in}{2.852941in}}%
\pgfpathlineto{\pgfqpoint{3.268096in}{2.866502in}}%
\pgfpathlineto{\pgfqpoint{3.255186in}{2.880313in}}%
\pgfpathlineto{\pgfqpoint{3.242273in}{2.894375in}}%
\pgfpathlineto{\pgfqpoint{3.229356in}{2.908691in}}%
\pgfpathlineto{\pgfqpoint{3.221665in}{2.896553in}}%
\pgfpathlineto{\pgfqpoint{3.213968in}{2.884543in}}%
\pgfpathlineto{\pgfqpoint{3.206264in}{2.872658in}}%
\pgfpathlineto{\pgfqpoint{3.198553in}{2.860897in}}%
\pgfpathclose%
\pgfusepath{fill}%
\end{pgfscope}%
\begin{pgfscope}%
\pgfpathrectangle{\pgfqpoint{1.254980in}{0.150000in}}{\pgfqpoint{5.490039in}{5.490039in}}%
\pgfusepath{clip}%
\pgfsetbuttcap%
\pgfsetroundjoin%
\definecolor{currentfill}{rgb}{0.248629,0.278775,0.534556}%
\pgfsetfillcolor{currentfill}%
\pgfsetfillopacity{0.700000}%
\pgfsetlinewidth{0.000000pt}%
\definecolor{currentstroke}{rgb}{0.000000,0.000000,0.000000}%
\pgfsetstrokecolor{currentstroke}%
\pgfsetdash{}{0pt}%
\pgfpathmoveto{\pgfqpoint{4.429002in}{2.798277in}}%
\pgfpathlineto{\pgfqpoint{4.442029in}{2.796091in}}%
\pgfpathlineto{\pgfqpoint{4.455063in}{2.794074in}}%
\pgfpathlineto{\pgfqpoint{4.468106in}{2.792226in}}%
\pgfpathlineto{\pgfqpoint{4.481156in}{2.790545in}}%
\pgfpathlineto{\pgfqpoint{4.488514in}{2.801793in}}%
\pgfpathlineto{\pgfqpoint{4.495868in}{2.813167in}}%
\pgfpathlineto{\pgfqpoint{4.503219in}{2.824670in}}%
\pgfpathlineto{\pgfqpoint{4.510567in}{2.836309in}}%
\pgfpathlineto{\pgfqpoint{4.497528in}{2.838403in}}%
\pgfpathlineto{\pgfqpoint{4.484495in}{2.840664in}}%
\pgfpathlineto{\pgfqpoint{4.471471in}{2.843094in}}%
\pgfpathlineto{\pgfqpoint{4.458455in}{2.845692in}}%
\pgfpathlineto{\pgfqpoint{4.451096in}{2.833631in}}%
\pgfpathlineto{\pgfqpoint{4.443734in}{2.821711in}}%
\pgfpathlineto{\pgfqpoint{4.436370in}{2.809928in}}%
\pgfpathlineto{\pgfqpoint{4.429002in}{2.798277in}}%
\pgfpathclose%
\pgfusepath{fill}%
\end{pgfscope}%
\begin{pgfscope}%
\pgfpathrectangle{\pgfqpoint{1.254980in}{0.150000in}}{\pgfqpoint{5.490039in}{5.490039in}}%
\pgfusepath{clip}%
\pgfsetbuttcap%
\pgfsetroundjoin%
\definecolor{currentfill}{rgb}{0.182256,0.426184,0.557120}%
\pgfsetfillcolor{currentfill}%
\pgfsetfillopacity{0.700000}%
\pgfsetlinewidth{0.000000pt}%
\definecolor{currentstroke}{rgb}{0.000000,0.000000,0.000000}%
\pgfsetstrokecolor{currentstroke}%
\pgfsetdash{}{0pt}%
\pgfpathmoveto{\pgfqpoint{5.081836in}{3.144091in}}%
\pgfpathlineto{\pgfqpoint{5.095044in}{3.142985in}}%
\pgfpathlineto{\pgfqpoint{5.108262in}{3.142034in}}%
\pgfpathlineto{\pgfqpoint{5.121490in}{3.141236in}}%
\pgfpathlineto{\pgfqpoint{5.134729in}{3.140592in}}%
\pgfpathlineto{\pgfqpoint{5.141939in}{3.153630in}}%
\pgfpathlineto{\pgfqpoint{5.149151in}{3.166930in}}%
\pgfpathlineto{\pgfqpoint{5.156364in}{3.180500in}}%
\pgfpathlineto{\pgfqpoint{5.163581in}{3.194347in}}%
\pgfpathlineto{\pgfqpoint{5.150360in}{3.195626in}}%
\pgfpathlineto{\pgfqpoint{5.137150in}{3.197058in}}%
\pgfpathlineto{\pgfqpoint{5.123949in}{3.198643in}}%
\pgfpathlineto{\pgfqpoint{5.110759in}{3.200382in}}%
\pgfpathlineto{\pgfqpoint{5.103524in}{3.185891in}}%
\pgfpathlineto{\pgfqpoint{5.096293in}{3.171683in}}%
\pgfpathlineto{\pgfqpoint{5.089063in}{3.157752in}}%
\pgfpathlineto{\pgfqpoint{5.081836in}{3.144091in}}%
\pgfpathclose%
\pgfusepath{fill}%
\end{pgfscope}%
\begin{pgfscope}%
\pgfpathrectangle{\pgfqpoint{1.254980in}{0.150000in}}{\pgfqpoint{5.490039in}{5.490039in}}%
\pgfusepath{clip}%
\pgfsetbuttcap%
\pgfsetroundjoin%
\definecolor{currentfill}{rgb}{0.275191,0.194905,0.496005}%
\pgfsetfillcolor{currentfill}%
\pgfsetfillopacity{0.700000}%
\pgfsetlinewidth{0.000000pt}%
\definecolor{currentstroke}{rgb}{0.000000,0.000000,0.000000}%
\pgfsetstrokecolor{currentstroke}%
\pgfsetdash{}{0pt}%
\pgfpathmoveto{\pgfqpoint{3.620730in}{2.642763in}}%
\pgfpathlineto{\pgfqpoint{3.633610in}{2.634700in}}%
\pgfpathlineto{\pgfqpoint{3.646491in}{2.626844in}}%
\pgfpathlineto{\pgfqpoint{3.659375in}{2.619194in}}%
\pgfpathlineto{\pgfqpoint{3.672260in}{2.611750in}}%
\pgfpathlineto{\pgfqpoint{3.679847in}{2.623099in}}%
\pgfpathlineto{\pgfqpoint{3.687429in}{2.634533in}}%
\pgfpathlineto{\pgfqpoint{3.695006in}{2.646056in}}%
\pgfpathlineto{\pgfqpoint{3.702578in}{2.657669in}}%
\pgfpathlineto{\pgfqpoint{3.689701in}{2.665306in}}%
\pgfpathlineto{\pgfqpoint{3.676826in}{2.673148in}}%
\pgfpathlineto{\pgfqpoint{3.663953in}{2.681196in}}%
\pgfpathlineto{\pgfqpoint{3.651082in}{2.689451in}}%
\pgfpathlineto{\pgfqpoint{3.643502in}{2.677636in}}%
\pgfpathlineto{\pgfqpoint{3.635916in}{2.665917in}}%
\pgfpathlineto{\pgfqpoint{3.628326in}{2.654294in}}%
\pgfpathlineto{\pgfqpoint{3.620730in}{2.642763in}}%
\pgfpathclose%
\pgfusepath{fill}%
\end{pgfscope}%
\begin{pgfscope}%
\pgfpathrectangle{\pgfqpoint{1.254980in}{0.150000in}}{\pgfqpoint{5.490039in}{5.490039in}}%
\pgfusepath{clip}%
\pgfsetbuttcap%
\pgfsetroundjoin%
\definecolor{currentfill}{rgb}{0.204903,0.375746,0.553533}%
\pgfsetfillcolor{currentfill}%
\pgfsetfillopacity{0.700000}%
\pgfsetlinewidth{0.000000pt}%
\definecolor{currentstroke}{rgb}{0.000000,0.000000,0.000000}%
\pgfsetstrokecolor{currentstroke}%
\pgfsetdash{}{0pt}%
\pgfpathmoveto{\pgfqpoint{3.043035in}{3.051602in}}%
\pgfpathlineto{\pgfqpoint{3.056027in}{3.034208in}}%
\pgfpathlineto{\pgfqpoint{3.069012in}{3.017094in}}%
\pgfpathlineto{\pgfqpoint{3.081991in}{3.000260in}}%
\pgfpathlineto{\pgfqpoint{3.094964in}{2.983702in}}%
\pgfpathlineto{\pgfqpoint{3.102698in}{2.995718in}}%
\pgfpathlineto{\pgfqpoint{3.110425in}{3.007871in}}%
\pgfpathlineto{\pgfqpoint{3.118145in}{3.020164in}}%
\pgfpathlineto{\pgfqpoint{3.125859in}{3.032599in}}%
\pgfpathlineto{\pgfqpoint{3.112898in}{3.049295in}}%
\pgfpathlineto{\pgfqpoint{3.099931in}{3.066268in}}%
\pgfpathlineto{\pgfqpoint{3.086957in}{3.083520in}}%
\pgfpathlineto{\pgfqpoint{3.073977in}{3.101053in}}%
\pgfpathlineto{\pgfqpoint{3.066252in}{3.088469in}}%
\pgfpathlineto{\pgfqpoint{3.058520in}{3.076034in}}%
\pgfpathlineto{\pgfqpoint{3.050781in}{3.063746in}}%
\pgfpathlineto{\pgfqpoint{3.043035in}{3.051602in}}%
\pgfpathclose%
\pgfusepath{fill}%
\end{pgfscope}%
\begin{pgfscope}%
\pgfpathrectangle{\pgfqpoint{1.254980in}{0.150000in}}{\pgfqpoint{5.490039in}{5.490039in}}%
\pgfusepath{clip}%
\pgfsetbuttcap%
\pgfsetroundjoin%
\definecolor{currentfill}{rgb}{0.252194,0.269783,0.531579}%
\pgfsetfillcolor{currentfill}%
\pgfsetfillopacity{0.700000}%
\pgfsetlinewidth{0.000000pt}%
\definecolor{currentstroke}{rgb}{0.000000,0.000000,0.000000}%
\pgfsetstrokecolor{currentstroke}%
\pgfsetdash{}{0pt}%
\pgfpathmoveto{\pgfqpoint{3.250243in}{2.805698in}}%
\pgfpathlineto{\pgfqpoint{3.263158in}{2.792521in}}%
\pgfpathlineto{\pgfqpoint{3.276069in}{2.779590in}}%
\pgfpathlineto{\pgfqpoint{3.288977in}{2.766901in}}%
\pgfpathlineto{\pgfqpoint{3.301883in}{2.754455in}}%
\pgfpathlineto{\pgfqpoint{3.309571in}{2.765961in}}%
\pgfpathlineto{\pgfqpoint{3.317253in}{2.777577in}}%
\pgfpathlineto{\pgfqpoint{3.324929in}{2.789306in}}%
\pgfpathlineto{\pgfqpoint{3.332599in}{2.801147in}}%
\pgfpathlineto{\pgfqpoint{3.319704in}{2.813732in}}%
\pgfpathlineto{\pgfqpoint{3.306806in}{2.826557in}}%
\pgfpathlineto{\pgfqpoint{3.293906in}{2.839627in}}%
\pgfpathlineto{\pgfqpoint{3.281003in}{2.852941in}}%
\pgfpathlineto{\pgfqpoint{3.273322in}{2.840951in}}%
\pgfpathlineto{\pgfqpoint{3.265635in}{2.829082in}}%
\pgfpathlineto{\pgfqpoint{3.257943in}{2.817331in}}%
\pgfpathlineto{\pgfqpoint{3.250243in}{2.805698in}}%
\pgfpathclose%
\pgfusepath{fill}%
\end{pgfscope}%
\begin{pgfscope}%
\pgfpathrectangle{\pgfqpoint{1.254980in}{0.150000in}}{\pgfqpoint{5.490039in}{5.490039in}}%
\pgfusepath{clip}%
\pgfsetbuttcap%
\pgfsetroundjoin%
\definecolor{currentfill}{rgb}{0.255645,0.260703,0.528312}%
\pgfsetfillcolor{currentfill}%
\pgfsetfillopacity{0.700000}%
\pgfsetlinewidth{0.000000pt}%
\definecolor{currentstroke}{rgb}{0.000000,0.000000,0.000000}%
\pgfsetstrokecolor{currentstroke}%
\pgfsetdash{}{0pt}%
\pgfpathmoveto{\pgfqpoint{4.347426in}{2.761803in}}%
\pgfpathlineto{\pgfqpoint{4.360433in}{2.759321in}}%
\pgfpathlineto{\pgfqpoint{4.373447in}{2.757010in}}%
\pgfpathlineto{\pgfqpoint{4.386469in}{2.754870in}}%
\pgfpathlineto{\pgfqpoint{4.399498in}{2.752900in}}%
\pgfpathlineto{\pgfqpoint{4.406879in}{2.764069in}}%
\pgfpathlineto{\pgfqpoint{4.414257in}{2.775352in}}%
\pgfpathlineto{\pgfqpoint{4.421631in}{2.786753in}}%
\pgfpathlineto{\pgfqpoint{4.429002in}{2.798277in}}%
\pgfpathlineto{\pgfqpoint{4.415983in}{2.800633in}}%
\pgfpathlineto{\pgfqpoint{4.402971in}{2.803158in}}%
\pgfpathlineto{\pgfqpoint{4.389966in}{2.805854in}}%
\pgfpathlineto{\pgfqpoint{4.376969in}{2.808721in}}%
\pgfpathlineto{\pgfqpoint{4.369588in}{2.796802in}}%
\pgfpathlineto{\pgfqpoint{4.362204in}{2.785013in}}%
\pgfpathlineto{\pgfqpoint{4.354817in}{2.773348in}}%
\pgfpathlineto{\pgfqpoint{4.347426in}{2.761803in}}%
\pgfpathclose%
\pgfusepath{fill}%
\end{pgfscope}%
\begin{pgfscope}%
\pgfpathrectangle{\pgfqpoint{1.254980in}{0.150000in}}{\pgfqpoint{5.490039in}{5.490039in}}%
\pgfusepath{clip}%
\pgfsetbuttcap%
\pgfsetroundjoin%
\definecolor{currentfill}{rgb}{0.273006,0.204520,0.501721}%
\pgfsetfillcolor{currentfill}%
\pgfsetfillopacity{0.700000}%
\pgfsetlinewidth{0.000000pt}%
\definecolor{currentstroke}{rgb}{0.000000,0.000000,0.000000}%
\pgfsetstrokecolor{currentstroke}%
\pgfsetdash{}{0pt}%
\pgfpathmoveto{\pgfqpoint{3.969165in}{2.651426in}}%
\pgfpathlineto{\pgfqpoint{3.982085in}{2.646635in}}%
\pgfpathlineto{\pgfqpoint{3.995010in}{2.642031in}}%
\pgfpathlineto{\pgfqpoint{4.007940in}{2.637612in}}%
\pgfpathlineto{\pgfqpoint{4.020875in}{2.633377in}}%
\pgfpathlineto{\pgfqpoint{4.028364in}{2.644609in}}%
\pgfpathlineto{\pgfqpoint{4.035849in}{2.655926in}}%
\pgfpathlineto{\pgfqpoint{4.043329in}{2.667332in}}%
\pgfpathlineto{\pgfqpoint{4.050806in}{2.678831in}}%
\pgfpathlineto{\pgfqpoint{4.037879in}{2.683341in}}%
\pgfpathlineto{\pgfqpoint{4.024957in}{2.688035in}}%
\pgfpathlineto{\pgfqpoint{4.012040in}{2.692914in}}%
\pgfpathlineto{\pgfqpoint{3.999128in}{2.697979in}}%
\pgfpathlineto{\pgfqpoint{3.991644in}{2.686196in}}%
\pgfpathlineto{\pgfqpoint{3.984155in}{2.674511in}}%
\pgfpathlineto{\pgfqpoint{3.976662in}{2.662922in}}%
\pgfpathlineto{\pgfqpoint{3.969165in}{2.651426in}}%
\pgfpathclose%
\pgfusepath{fill}%
\end{pgfscope}%
\begin{pgfscope}%
\pgfpathrectangle{\pgfqpoint{1.254980in}{0.150000in}}{\pgfqpoint{5.490039in}{5.490039in}}%
\pgfusepath{clip}%
\pgfsetbuttcap%
\pgfsetroundjoin%
\definecolor{currentfill}{rgb}{0.276194,0.190074,0.493001}%
\pgfsetfillcolor{currentfill}%
\pgfsetfillopacity{0.700000}%
\pgfsetlinewidth{0.000000pt}%
\definecolor{currentstroke}{rgb}{0.000000,0.000000,0.000000}%
\pgfsetstrokecolor{currentstroke}%
\pgfsetdash{}{0pt}%
\pgfpathmoveto{\pgfqpoint{3.754110in}{2.629144in}}%
\pgfpathlineto{\pgfqpoint{3.767000in}{2.622514in}}%
\pgfpathlineto{\pgfqpoint{3.779893in}{2.616081in}}%
\pgfpathlineto{\pgfqpoint{3.792789in}{2.609845in}}%
\pgfpathlineto{\pgfqpoint{3.805688in}{2.603804in}}%
\pgfpathlineto{\pgfqpoint{3.813238in}{2.615092in}}%
\pgfpathlineto{\pgfqpoint{3.820784in}{2.626461in}}%
\pgfpathlineto{\pgfqpoint{3.828325in}{2.637915in}}%
\pgfpathlineto{\pgfqpoint{3.835862in}{2.649457in}}%
\pgfpathlineto{\pgfqpoint{3.822970in}{2.655717in}}%
\pgfpathlineto{\pgfqpoint{3.810083in}{2.662173in}}%
\pgfpathlineto{\pgfqpoint{3.797198in}{2.668826in}}%
\pgfpathlineto{\pgfqpoint{3.784316in}{2.675677in}}%
\pgfpathlineto{\pgfqpoint{3.776772in}{2.663905in}}%
\pgfpathlineto{\pgfqpoint{3.769223in}{2.652227in}}%
\pgfpathlineto{\pgfqpoint{3.761669in}{2.640641in}}%
\pgfpathlineto{\pgfqpoint{3.754110in}{2.629144in}}%
\pgfpathclose%
\pgfusepath{fill}%
\end{pgfscope}%
\begin{pgfscope}%
\pgfpathrectangle{\pgfqpoint{1.254980in}{0.150000in}}{\pgfqpoint{5.490039in}{5.490039in}}%
\pgfusepath{clip}%
\pgfsetbuttcap%
\pgfsetroundjoin%
\definecolor{currentfill}{rgb}{0.271828,0.209303,0.504434}%
\pgfsetfillcolor{currentfill}%
\pgfsetfillopacity{0.700000}%
\pgfsetlinewidth{0.000000pt}%
\definecolor{currentstroke}{rgb}{0.000000,0.000000,0.000000}%
\pgfsetstrokecolor{currentstroke}%
\pgfsetdash{}{0pt}%
\pgfpathmoveto{\pgfqpoint{3.487225in}{2.668320in}}%
\pgfpathlineto{\pgfqpoint{3.500107in}{2.658716in}}%
\pgfpathlineto{\pgfqpoint{3.512989in}{2.649330in}}%
\pgfpathlineto{\pgfqpoint{3.525871in}{2.640161in}}%
\pgfpathlineto{\pgfqpoint{3.538754in}{2.631207in}}%
\pgfpathlineto{\pgfqpoint{3.546380in}{2.642546in}}%
\pgfpathlineto{\pgfqpoint{3.554000in}{2.653975in}}%
\pgfpathlineto{\pgfqpoint{3.561615in}{2.665496in}}%
\pgfpathlineto{\pgfqpoint{3.569224in}{2.677113in}}%
\pgfpathlineto{\pgfqpoint{3.556351in}{2.686232in}}%
\pgfpathlineto{\pgfqpoint{3.543477in}{2.695566in}}%
\pgfpathlineto{\pgfqpoint{3.530605in}{2.705117in}}%
\pgfpathlineto{\pgfqpoint{3.517732in}{2.714885in}}%
\pgfpathlineto{\pgfqpoint{3.510114in}{2.703094in}}%
\pgfpathlineto{\pgfqpoint{3.502490in}{2.691404in}}%
\pgfpathlineto{\pgfqpoint{3.494860in}{2.679813in}}%
\pgfpathlineto{\pgfqpoint{3.487225in}{2.668320in}}%
\pgfpathclose%
\pgfusepath{fill}%
\end{pgfscope}%
\begin{pgfscope}%
\pgfpathrectangle{\pgfqpoint{1.254980in}{0.150000in}}{\pgfqpoint{5.490039in}{5.490039in}}%
\pgfusepath{clip}%
\pgfsetbuttcap%
\pgfsetroundjoin%
\definecolor{currentfill}{rgb}{0.262138,0.242286,0.520837}%
\pgfsetfillcolor{currentfill}%
\pgfsetfillopacity{0.700000}%
\pgfsetlinewidth{0.000000pt}%
\definecolor{currentstroke}{rgb}{0.000000,0.000000,0.000000}%
\pgfsetstrokecolor{currentstroke}%
\pgfsetdash{}{0pt}%
\pgfpathmoveto{\pgfqpoint{4.265834in}{2.726966in}}%
\pgfpathlineto{\pgfqpoint{4.278822in}{2.724149in}}%
\pgfpathlineto{\pgfqpoint{4.291817in}{2.721507in}}%
\pgfpathlineto{\pgfqpoint{4.304819in}{2.719038in}}%
\pgfpathlineto{\pgfqpoint{4.317828in}{2.716742in}}%
\pgfpathlineto{\pgfqpoint{4.325233in}{2.727848in}}%
\pgfpathlineto{\pgfqpoint{4.332634in}{2.739058in}}%
\pgfpathlineto{\pgfqpoint{4.340032in}{2.750375in}}%
\pgfpathlineto{\pgfqpoint{4.347426in}{2.761803in}}%
\pgfpathlineto{\pgfqpoint{4.334427in}{2.764458in}}%
\pgfpathlineto{\pgfqpoint{4.321434in}{2.767284in}}%
\pgfpathlineto{\pgfqpoint{4.308449in}{2.770284in}}%
\pgfpathlineto{\pgfqpoint{4.295470in}{2.773458in}}%
\pgfpathlineto{\pgfqpoint{4.288066in}{2.761662in}}%
\pgfpathlineto{\pgfqpoint{4.280659in}{2.749984in}}%
\pgfpathlineto{\pgfqpoint{4.273248in}{2.738420in}}%
\pgfpathlineto{\pgfqpoint{4.265834in}{2.726966in}}%
\pgfpathclose%
\pgfusepath{fill}%
\end{pgfscope}%
\begin{pgfscope}%
\pgfpathrectangle{\pgfqpoint{1.254980in}{0.150000in}}{\pgfqpoint{5.490039in}{5.490039in}}%
\pgfusepath{clip}%
\pgfsetbuttcap%
\pgfsetroundjoin%
\definecolor{currentfill}{rgb}{0.260571,0.246922,0.522828}%
\pgfsetfillcolor{currentfill}%
\pgfsetfillopacity{0.700000}%
\pgfsetlinewidth{0.000000pt}%
\definecolor{currentstroke}{rgb}{0.000000,0.000000,0.000000}%
\pgfsetstrokecolor{currentstroke}%
\pgfsetdash{}{0pt}%
\pgfpathmoveto{\pgfqpoint{3.301883in}{2.754455in}}%
\pgfpathlineto{\pgfqpoint{3.314786in}{2.742247in}}%
\pgfpathlineto{\pgfqpoint{3.327688in}{2.730278in}}%
\pgfpathlineto{\pgfqpoint{3.340587in}{2.718544in}}%
\pgfpathlineto{\pgfqpoint{3.353485in}{2.707045in}}%
\pgfpathlineto{\pgfqpoint{3.361163in}{2.718424in}}%
\pgfpathlineto{\pgfqpoint{3.368834in}{2.729906in}}%
\pgfpathlineto{\pgfqpoint{3.376500in}{2.741494in}}%
\pgfpathlineto{\pgfqpoint{3.384160in}{2.753188in}}%
\pgfpathlineto{\pgfqpoint{3.371273in}{2.764825in}}%
\pgfpathlineto{\pgfqpoint{3.358383in}{2.776696in}}%
\pgfpathlineto{\pgfqpoint{3.345492in}{2.788803in}}%
\pgfpathlineto{\pgfqpoint{3.332599in}{2.801147in}}%
\pgfpathlineto{\pgfqpoint{3.324929in}{2.789306in}}%
\pgfpathlineto{\pgfqpoint{3.317253in}{2.777577in}}%
\pgfpathlineto{\pgfqpoint{3.309571in}{2.765961in}}%
\pgfpathlineto{\pgfqpoint{3.301883in}{2.754455in}}%
\pgfpathclose%
\pgfusepath{fill}%
\end{pgfscope}%
\begin{pgfscope}%
\pgfpathrectangle{\pgfqpoint{1.254980in}{0.150000in}}{\pgfqpoint{5.490039in}{5.490039in}}%
\pgfusepath{clip}%
\pgfsetbuttcap%
\pgfsetroundjoin%
\definecolor{currentfill}{rgb}{0.174274,0.445044,0.557792}%
\pgfsetfillcolor{currentfill}%
\pgfsetfillopacity{0.700000}%
\pgfsetlinewidth{0.000000pt}%
\definecolor{currentstroke}{rgb}{0.000000,0.000000,0.000000}%
\pgfsetstrokecolor{currentstroke}%
\pgfsetdash{}{0pt}%
\pgfpathmoveto{\pgfqpoint{5.163581in}{3.194347in}}%
\pgfpathlineto{\pgfqpoint{5.176812in}{3.193222in}}%
\pgfpathlineto{\pgfqpoint{5.190053in}{3.192249in}}%
\pgfpathlineto{\pgfqpoint{5.203305in}{3.191429in}}%
\pgfpathlineto{\pgfqpoint{5.216568in}{3.190761in}}%
\pgfpathlineto{\pgfqpoint{5.223768in}{3.204242in}}%
\pgfpathlineto{\pgfqpoint{5.230971in}{3.218007in}}%
\pgfpathlineto{\pgfqpoint{5.238178in}{3.232066in}}%
\pgfpathlineto{\pgfqpoint{5.224930in}{3.233228in}}%
\pgfpathlineto{\pgfqpoint{5.211692in}{3.234542in}}%
\pgfpathlineto{\pgfqpoint{5.198465in}{3.236009in}}%
\pgfpathlineto{\pgfqpoint{5.185248in}{3.237628in}}%
\pgfpathlineto{\pgfqpoint{5.178023in}{3.222904in}}%
\pgfpathlineto{\pgfqpoint{5.170800in}{3.208479in}}%
\pgfpathlineto{\pgfqpoint{5.163581in}{3.194347in}}%
\pgfpathclose%
\pgfusepath{fill}%
\end{pgfscope}%
\begin{pgfscope}%
\pgfpathrectangle{\pgfqpoint{1.254980in}{0.150000in}}{\pgfqpoint{5.490039in}{5.490039in}}%
\pgfusepath{clip}%
\pgfsetbuttcap%
\pgfsetroundjoin%
\definecolor{currentfill}{rgb}{0.190631,0.407061,0.556089}%
\pgfsetfillcolor{currentfill}%
\pgfsetfillopacity{0.700000}%
\pgfsetlinewidth{0.000000pt}%
\definecolor{currentstroke}{rgb}{0.000000,0.000000,0.000000}%
\pgfsetstrokecolor{currentstroke}%
\pgfsetdash{}{0pt}%
\pgfpathmoveto{\pgfqpoint{2.990994in}{3.124049in}}%
\pgfpathlineto{\pgfqpoint{3.004015in}{3.105502in}}%
\pgfpathlineto{\pgfqpoint{3.017029in}{3.087247in}}%
\pgfpathlineto{\pgfqpoint{3.030036in}{3.069281in}}%
\pgfpathlineto{\pgfqpoint{3.043035in}{3.051602in}}%
\pgfpathlineto{\pgfqpoint{3.050781in}{3.063746in}}%
\pgfpathlineto{\pgfqpoint{3.058520in}{3.076034in}}%
\pgfpathlineto{\pgfqpoint{3.066252in}{3.088469in}}%
\pgfpathlineto{\pgfqpoint{3.073977in}{3.101053in}}%
\pgfpathlineto{\pgfqpoint{3.060990in}{3.118870in}}%
\pgfpathlineto{\pgfqpoint{3.047996in}{3.136975in}}%
\pgfpathlineto{\pgfqpoint{3.034995in}{3.155368in}}%
\pgfpathlineto{\pgfqpoint{3.021986in}{3.174055in}}%
\pgfpathlineto{\pgfqpoint{3.014249in}{3.161322in}}%
\pgfpathlineto{\pgfqpoint{3.006504in}{3.148744in}}%
\pgfpathlineto{\pgfqpoint{2.998753in}{3.136321in}}%
\pgfpathlineto{\pgfqpoint{2.990994in}{3.124049in}}%
\pgfpathclose%
\pgfusepath{fill}%
\end{pgfscope}%
\begin{pgfscope}%
\pgfpathrectangle{\pgfqpoint{1.254980in}{0.150000in}}{\pgfqpoint{5.490039in}{5.490039in}}%
\pgfusepath{clip}%
\pgfsetbuttcap%
\pgfsetroundjoin%
\definecolor{currentfill}{rgb}{0.275191,0.194905,0.496005}%
\pgfsetfillcolor{currentfill}%
\pgfsetfillopacity{0.700000}%
\pgfsetlinewidth{0.000000pt}%
\definecolor{currentstroke}{rgb}{0.000000,0.000000,0.000000}%
\pgfsetstrokecolor{currentstroke}%
\pgfsetdash{}{0pt}%
\pgfpathmoveto{\pgfqpoint{3.887463in}{2.626353in}}%
\pgfpathlineto{\pgfqpoint{3.900373in}{2.621056in}}%
\pgfpathlineto{\pgfqpoint{3.913288in}{2.615949in}}%
\pgfpathlineto{\pgfqpoint{3.926207in}{2.611031in}}%
\pgfpathlineto{\pgfqpoint{3.939130in}{2.606301in}}%
\pgfpathlineto{\pgfqpoint{3.946646in}{2.617460in}}%
\pgfpathlineto{\pgfqpoint{3.954157in}{2.628698in}}%
\pgfpathlineto{\pgfqpoint{3.961663in}{2.640019in}}%
\pgfpathlineto{\pgfqpoint{3.969165in}{2.651426in}}%
\pgfpathlineto{\pgfqpoint{3.956249in}{2.656404in}}%
\pgfpathlineto{\pgfqpoint{3.943338in}{2.661569in}}%
\pgfpathlineto{\pgfqpoint{3.930432in}{2.666923in}}%
\pgfpathlineto{\pgfqpoint{3.917530in}{2.672468in}}%
\pgfpathlineto{\pgfqpoint{3.910020in}{2.660803in}}%
\pgfpathlineto{\pgfqpoint{3.902505in}{2.649231in}}%
\pgfpathlineto{\pgfqpoint{3.894987in}{2.637748in}}%
\pgfpathlineto{\pgfqpoint{3.887463in}{2.626353in}}%
\pgfpathclose%
\pgfusepath{fill}%
\end{pgfscope}%
\begin{pgfscope}%
\pgfpathrectangle{\pgfqpoint{1.254980in}{0.150000in}}{\pgfqpoint{5.490039in}{5.490039in}}%
\pgfusepath{clip}%
\pgfsetbuttcap%
\pgfsetroundjoin%
\definecolor{currentfill}{rgb}{0.266580,0.228262,0.514349}%
\pgfsetfillcolor{currentfill}%
\pgfsetfillopacity{0.700000}%
\pgfsetlinewidth{0.000000pt}%
\definecolor{currentstroke}{rgb}{0.000000,0.000000,0.000000}%
\pgfsetstrokecolor{currentstroke}%
\pgfsetdash{}{0pt}%
\pgfpathmoveto{\pgfqpoint{4.184217in}{2.693864in}}%
\pgfpathlineto{\pgfqpoint{4.197187in}{2.690675in}}%
\pgfpathlineto{\pgfqpoint{4.210164in}{2.687664in}}%
\pgfpathlineto{\pgfqpoint{4.223148in}{2.684828in}}%
\pgfpathlineto{\pgfqpoint{4.236138in}{2.682168in}}%
\pgfpathlineto{\pgfqpoint{4.243568in}{2.693222in}}%
\pgfpathlineto{\pgfqpoint{4.250994in}{2.704371in}}%
\pgfpathlineto{\pgfqpoint{4.258416in}{2.715617in}}%
\pgfpathlineto{\pgfqpoint{4.265834in}{2.726966in}}%
\pgfpathlineto{\pgfqpoint{4.252853in}{2.729956in}}%
\pgfpathlineto{\pgfqpoint{4.239878in}{2.733122in}}%
\pgfpathlineto{\pgfqpoint{4.226910in}{2.736464in}}%
\pgfpathlineto{\pgfqpoint{4.213949in}{2.739983in}}%
\pgfpathlineto{\pgfqpoint{4.206522in}{2.728294in}}%
\pgfpathlineto{\pgfqpoint{4.199091in}{2.716714in}}%
\pgfpathlineto{\pgfqpoint{4.191656in}{2.705239in}}%
\pgfpathlineto{\pgfqpoint{4.184217in}{2.693864in}}%
\pgfpathclose%
\pgfusepath{fill}%
\end{pgfscope}%
\begin{pgfscope}%
\pgfpathrectangle{\pgfqpoint{1.254980in}{0.150000in}}{\pgfqpoint{5.490039in}{5.490039in}}%
\pgfusepath{clip}%
\pgfsetbuttcap%
\pgfsetroundjoin%
\definecolor{currentfill}{rgb}{0.266580,0.228262,0.514349}%
\pgfsetfillcolor{currentfill}%
\pgfsetfillopacity{0.700000}%
\pgfsetlinewidth{0.000000pt}%
\definecolor{currentstroke}{rgb}{0.000000,0.000000,0.000000}%
\pgfsetstrokecolor{currentstroke}%
\pgfsetdash{}{0pt}%
\pgfpathmoveto{\pgfqpoint{3.353485in}{2.707045in}}%
\pgfpathlineto{\pgfqpoint{3.366382in}{2.695778in}}%
\pgfpathlineto{\pgfqpoint{3.379277in}{2.684741in}}%
\pgfpathlineto{\pgfqpoint{3.392171in}{2.673934in}}%
\pgfpathlineto{\pgfqpoint{3.405064in}{2.663354in}}%
\pgfpathlineto{\pgfqpoint{3.412731in}{2.674606in}}%
\pgfpathlineto{\pgfqpoint{3.420392in}{2.685955in}}%
\pgfpathlineto{\pgfqpoint{3.428048in}{2.697401in}}%
\pgfpathlineto{\pgfqpoint{3.435698in}{2.708947in}}%
\pgfpathlineto{\pgfqpoint{3.422815in}{2.719665in}}%
\pgfpathlineto{\pgfqpoint{3.409931in}{2.730609in}}%
\pgfpathlineto{\pgfqpoint{3.397046in}{2.741783in}}%
\pgfpathlineto{\pgfqpoint{3.384160in}{2.753188in}}%
\pgfpathlineto{\pgfqpoint{3.376500in}{2.741494in}}%
\pgfpathlineto{\pgfqpoint{3.368834in}{2.729906in}}%
\pgfpathlineto{\pgfqpoint{3.361163in}{2.718424in}}%
\pgfpathlineto{\pgfqpoint{3.353485in}{2.707045in}}%
\pgfpathclose%
\pgfusepath{fill}%
\end{pgfscope}%
\begin{pgfscope}%
\pgfpathrectangle{\pgfqpoint{1.254980in}{0.150000in}}{\pgfqpoint{5.490039in}{5.490039in}}%
\pgfusepath{clip}%
\pgfsetbuttcap%
\pgfsetroundjoin%
\definecolor{currentfill}{rgb}{0.177423,0.437527,0.557565}%
\pgfsetfillcolor{currentfill}%
\pgfsetfillopacity{0.700000}%
\pgfsetlinewidth{0.000000pt}%
\definecolor{currentstroke}{rgb}{0.000000,0.000000,0.000000}%
\pgfsetstrokecolor{currentstroke}%
\pgfsetdash{}{0pt}%
\pgfpathmoveto{\pgfqpoint{2.938824in}{3.201220in}}%
\pgfpathlineto{\pgfqpoint{2.951879in}{3.181474in}}%
\pgfpathlineto{\pgfqpoint{2.964926in}{3.162033in}}%
\pgfpathlineto{\pgfqpoint{2.977964in}{3.142892in}}%
\pgfpathlineto{\pgfqpoint{2.990994in}{3.124049in}}%
\pgfpathlineto{\pgfqpoint{2.998753in}{3.136321in}}%
\pgfpathlineto{\pgfqpoint{3.006504in}{3.148744in}}%
\pgfpathlineto{\pgfqpoint{3.014249in}{3.161322in}}%
\pgfpathlineto{\pgfqpoint{3.021986in}{3.174055in}}%
\pgfpathlineto{\pgfqpoint{3.008969in}{3.193036in}}%
\pgfpathlineto{\pgfqpoint{2.995943in}{3.212316in}}%
\pgfpathlineto{\pgfqpoint{2.982910in}{3.231897in}}%
\pgfpathlineto{\pgfqpoint{2.969867in}{3.251782in}}%
\pgfpathlineto{\pgfqpoint{2.962118in}{3.238899in}}%
\pgfpathlineto{\pgfqpoint{2.954361in}{3.226179in}}%
\pgfpathlineto{\pgfqpoint{2.946596in}{3.213620in}}%
\pgfpathlineto{\pgfqpoint{2.938824in}{3.201220in}}%
\pgfpathclose%
\pgfusepath{fill}%
\end{pgfscope}%
\begin{pgfscope}%
\pgfpathrectangle{\pgfqpoint{1.254980in}{0.150000in}}{\pgfqpoint{5.490039in}{5.490039in}}%
\pgfusepath{clip}%
\pgfsetbuttcap%
\pgfsetroundjoin%
\definecolor{currentfill}{rgb}{0.277134,0.185228,0.489898}%
\pgfsetfillcolor{currentfill}%
\pgfsetfillopacity{0.700000}%
\pgfsetlinewidth{0.000000pt}%
\definecolor{currentstroke}{rgb}{0.000000,0.000000,0.000000}%
\pgfsetstrokecolor{currentstroke}%
\pgfsetdash{}{0pt}%
\pgfpathmoveto{\pgfqpoint{3.672260in}{2.611750in}}%
\pgfpathlineto{\pgfqpoint{3.685148in}{2.604509in}}%
\pgfpathlineto{\pgfqpoint{3.698038in}{2.597470in}}%
\pgfpathlineto{\pgfqpoint{3.710931in}{2.590633in}}%
\pgfpathlineto{\pgfqpoint{3.723826in}{2.583995in}}%
\pgfpathlineto{\pgfqpoint{3.731404in}{2.595162in}}%
\pgfpathlineto{\pgfqpoint{3.738978in}{2.606407in}}%
\pgfpathlineto{\pgfqpoint{3.746546in}{2.617734in}}%
\pgfpathlineto{\pgfqpoint{3.754110in}{2.629144in}}%
\pgfpathlineto{\pgfqpoint{3.741223in}{2.635974in}}%
\pgfpathlineto{\pgfqpoint{3.728339in}{2.643004in}}%
\pgfpathlineto{\pgfqpoint{3.715457in}{2.650235in}}%
\pgfpathlineto{\pgfqpoint{3.702578in}{2.657669in}}%
\pgfpathlineto{\pgfqpoint{3.695006in}{2.646056in}}%
\pgfpathlineto{\pgfqpoint{3.687429in}{2.634533in}}%
\pgfpathlineto{\pgfqpoint{3.679847in}{2.623099in}}%
\pgfpathlineto{\pgfqpoint{3.672260in}{2.611750in}}%
\pgfpathclose%
\pgfusepath{fill}%
\end{pgfscope}%
\begin{pgfscope}%
\pgfpathrectangle{\pgfqpoint{1.254980in}{0.150000in}}{\pgfqpoint{5.490039in}{5.490039in}}%
\pgfusepath{clip}%
\pgfsetbuttcap%
\pgfsetroundjoin%
\definecolor{currentfill}{rgb}{0.275191,0.194905,0.496005}%
\pgfsetfillcolor{currentfill}%
\pgfsetfillopacity{0.700000}%
\pgfsetlinewidth{0.000000pt}%
\definecolor{currentstroke}{rgb}{0.000000,0.000000,0.000000}%
\pgfsetstrokecolor{currentstroke}%
\pgfsetdash{}{0pt}%
\pgfpathmoveto{\pgfqpoint{3.538754in}{2.631207in}}%
\pgfpathlineto{\pgfqpoint{3.551638in}{2.622467in}}%
\pgfpathlineto{\pgfqpoint{3.564522in}{2.613939in}}%
\pgfpathlineto{\pgfqpoint{3.577408in}{2.605623in}}%
\pgfpathlineto{\pgfqpoint{3.590296in}{2.597517in}}%
\pgfpathlineto{\pgfqpoint{3.597912in}{2.608701in}}%
\pgfpathlineto{\pgfqpoint{3.605523in}{2.619968in}}%
\pgfpathlineto{\pgfqpoint{3.613129in}{2.631322in}}%
\pgfpathlineto{\pgfqpoint{3.620730in}{2.642763in}}%
\pgfpathlineto{\pgfqpoint{3.607852in}{2.651034in}}%
\pgfpathlineto{\pgfqpoint{3.594975in}{2.659515in}}%
\pgfpathlineto{\pgfqpoint{3.582099in}{2.668208in}}%
\pgfpathlineto{\pgfqpoint{3.569224in}{2.677113in}}%
\pgfpathlineto{\pgfqpoint{3.561615in}{2.665496in}}%
\pgfpathlineto{\pgfqpoint{3.554000in}{2.653975in}}%
\pgfpathlineto{\pgfqpoint{3.546380in}{2.642546in}}%
\pgfpathlineto{\pgfqpoint{3.538754in}{2.631207in}}%
\pgfpathclose%
\pgfusepath{fill}%
\end{pgfscope}%
\begin{pgfscope}%
\pgfpathrectangle{\pgfqpoint{1.254980in}{0.150000in}}{\pgfqpoint{5.490039in}{5.490039in}}%
\pgfusepath{clip}%
\pgfsetbuttcap%
\pgfsetroundjoin%
\definecolor{currentfill}{rgb}{0.270595,0.214069,0.507052}%
\pgfsetfillcolor{currentfill}%
\pgfsetfillopacity{0.700000}%
\pgfsetlinewidth{0.000000pt}%
\definecolor{currentstroke}{rgb}{0.000000,0.000000,0.000000}%
\pgfsetstrokecolor{currentstroke}%
\pgfsetdash{}{0pt}%
\pgfpathmoveto{\pgfqpoint{4.102567in}{2.662620in}}%
\pgfpathlineto{\pgfqpoint{4.115521in}{2.659020in}}%
\pgfpathlineto{\pgfqpoint{4.128482in}{2.655600in}}%
\pgfpathlineto{\pgfqpoint{4.141448in}{2.652359in}}%
\pgfpathlineto{\pgfqpoint{4.154421in}{2.649296in}}%
\pgfpathlineto{\pgfqpoint{4.161876in}{2.660305in}}%
\pgfpathlineto{\pgfqpoint{4.169327in}{2.671401in}}%
\pgfpathlineto{\pgfqpoint{4.176774in}{2.682586in}}%
\pgfpathlineto{\pgfqpoint{4.184217in}{2.693864in}}%
\pgfpathlineto{\pgfqpoint{4.171253in}{2.697230in}}%
\pgfpathlineto{\pgfqpoint{4.158295in}{2.700774in}}%
\pgfpathlineto{\pgfqpoint{4.145343in}{2.704497in}}%
\pgfpathlineto{\pgfqpoint{4.132397in}{2.708399in}}%
\pgfpathlineto{\pgfqpoint{4.124946in}{2.696808in}}%
\pgfpathlineto{\pgfqpoint{4.117490in}{2.685317in}}%
\pgfpathlineto{\pgfqpoint{4.110031in}{2.673922in}}%
\pgfpathlineto{\pgfqpoint{4.102567in}{2.662620in}}%
\pgfpathclose%
\pgfusepath{fill}%
\end{pgfscope}%
\begin{pgfscope}%
\pgfpathrectangle{\pgfqpoint{1.254980in}{0.150000in}}{\pgfqpoint{5.490039in}{5.490039in}}%
\pgfusepath{clip}%
\pgfsetbuttcap%
\pgfsetroundjoin%
\definecolor{currentfill}{rgb}{0.277134,0.185228,0.489898}%
\pgfsetfillcolor{currentfill}%
\pgfsetfillopacity{0.700000}%
\pgfsetlinewidth{0.000000pt}%
\definecolor{currentstroke}{rgb}{0.000000,0.000000,0.000000}%
\pgfsetstrokecolor{currentstroke}%
\pgfsetdash{}{0pt}%
\pgfpathmoveto{\pgfqpoint{3.805688in}{2.603804in}}%
\pgfpathlineto{\pgfqpoint{3.818591in}{2.597959in}}%
\pgfpathlineto{\pgfqpoint{3.831497in}{2.592307in}}%
\pgfpathlineto{\pgfqpoint{3.844408in}{2.586848in}}%
\pgfpathlineto{\pgfqpoint{3.857322in}{2.581581in}}%
\pgfpathlineto{\pgfqpoint{3.864864in}{2.592658in}}%
\pgfpathlineto{\pgfqpoint{3.872402in}{2.603811in}}%
\pgfpathlineto{\pgfqpoint{3.879935in}{2.615041in}}%
\pgfpathlineto{\pgfqpoint{3.887463in}{2.626353in}}%
\pgfpathlineto{\pgfqpoint{3.874557in}{2.631840in}}%
\pgfpathlineto{\pgfqpoint{3.861655in}{2.637520in}}%
\pgfpathlineto{\pgfqpoint{3.848756in}{2.643392in}}%
\pgfpathlineto{\pgfqpoint{3.835862in}{2.649457in}}%
\pgfpathlineto{\pgfqpoint{3.828325in}{2.637915in}}%
\pgfpathlineto{\pgfqpoint{3.820784in}{2.626461in}}%
\pgfpathlineto{\pgfqpoint{3.813238in}{2.615092in}}%
\pgfpathlineto{\pgfqpoint{3.805688in}{2.603804in}}%
\pgfpathclose%
\pgfusepath{fill}%
\end{pgfscope}%
\begin{pgfscope}%
\pgfpathrectangle{\pgfqpoint{1.254980in}{0.150000in}}{\pgfqpoint{5.490039in}{5.490039in}}%
\pgfusepath{clip}%
\pgfsetbuttcap%
\pgfsetroundjoin%
\definecolor{currentfill}{rgb}{0.214298,0.355619,0.551184}%
\pgfsetfillcolor{currentfill}%
\pgfsetfillopacity{0.700000}%
\pgfsetlinewidth{0.000000pt}%
\definecolor{currentstroke}{rgb}{0.000000,0.000000,0.000000}%
\pgfsetstrokecolor{currentstroke}%
\pgfsetdash{}{0pt}%
\pgfpathmoveto{\pgfqpoint{4.807790in}{2.954792in}}%
\pgfpathlineto{\pgfqpoint{4.820942in}{2.954066in}}%
\pgfpathlineto{\pgfqpoint{4.834104in}{2.953500in}}%
\pgfpathlineto{\pgfqpoint{4.847276in}{2.953093in}}%
\pgfpathlineto{\pgfqpoint{4.860458in}{2.952845in}}%
\pgfpathlineto{\pgfqpoint{4.867715in}{2.964151in}}%
\pgfpathlineto{\pgfqpoint{4.874971in}{2.975634in}}%
\pgfpathlineto{\pgfqpoint{4.882225in}{2.987301in}}%
\pgfpathlineto{\pgfqpoint{4.889479in}{2.999157in}}%
\pgfpathlineto{\pgfqpoint{4.876312in}{2.999931in}}%
\pgfpathlineto{\pgfqpoint{4.863155in}{3.000862in}}%
\pgfpathlineto{\pgfqpoint{4.850007in}{3.001953in}}%
\pgfpathlineto{\pgfqpoint{4.836869in}{3.003202in}}%
\pgfpathlineto{\pgfqpoint{4.829600in}{2.990811in}}%
\pgfpathlineto{\pgfqpoint{4.822331in}{2.978617in}}%
\pgfpathlineto{\pgfqpoint{4.815061in}{2.966612in}}%
\pgfpathlineto{\pgfqpoint{4.807790in}{2.954792in}}%
\pgfpathclose%
\pgfusepath{fill}%
\end{pgfscope}%
\begin{pgfscope}%
\pgfpathrectangle{\pgfqpoint{1.254980in}{0.150000in}}{\pgfqpoint{5.490039in}{5.490039in}}%
\pgfusepath{clip}%
\pgfsetbuttcap%
\pgfsetroundjoin%
\definecolor{currentfill}{rgb}{0.223925,0.334994,0.548053}%
\pgfsetfillcolor{currentfill}%
\pgfsetfillopacity{0.700000}%
\pgfsetlinewidth{0.000000pt}%
\definecolor{currentstroke}{rgb}{0.000000,0.000000,0.000000}%
\pgfsetstrokecolor{currentstroke}%
\pgfsetdash{}{0pt}%
\pgfpathmoveto{\pgfqpoint{4.726119in}{2.911746in}}%
\pgfpathlineto{\pgfqpoint{4.739247in}{2.910877in}}%
\pgfpathlineto{\pgfqpoint{4.752385in}{2.910169in}}%
\pgfpathlineto{\pgfqpoint{4.765532in}{2.909621in}}%
\pgfpathlineto{\pgfqpoint{4.778689in}{2.909233in}}%
\pgfpathlineto{\pgfqpoint{4.785967in}{2.920376in}}%
\pgfpathlineto{\pgfqpoint{4.793243in}{2.931679in}}%
\pgfpathlineto{\pgfqpoint{4.800517in}{2.943149in}}%
\pgfpathlineto{\pgfqpoint{4.807790in}{2.954792in}}%
\pgfpathlineto{\pgfqpoint{4.794647in}{2.955676in}}%
\pgfpathlineto{\pgfqpoint{4.781513in}{2.956721in}}%
\pgfpathlineto{\pgfqpoint{4.768389in}{2.957926in}}%
\pgfpathlineto{\pgfqpoint{4.755274in}{2.959291in}}%
\pgfpathlineto{\pgfqpoint{4.747988in}{2.947142in}}%
\pgfpathlineto{\pgfqpoint{4.740700in}{2.935172in}}%
\pgfpathlineto{\pgfqpoint{4.733410in}{2.923375in}}%
\pgfpathlineto{\pgfqpoint{4.726119in}{2.911746in}}%
\pgfpathclose%
\pgfusepath{fill}%
\end{pgfscope}%
\begin{pgfscope}%
\pgfpathrectangle{\pgfqpoint{1.254980in}{0.150000in}}{\pgfqpoint{5.490039in}{5.490039in}}%
\pgfusepath{clip}%
\pgfsetbuttcap%
\pgfsetroundjoin%
\definecolor{currentfill}{rgb}{0.204903,0.375746,0.553533}%
\pgfsetfillcolor{currentfill}%
\pgfsetfillopacity{0.700000}%
\pgfsetlinewidth{0.000000pt}%
\definecolor{currentstroke}{rgb}{0.000000,0.000000,0.000000}%
\pgfsetstrokecolor{currentstroke}%
\pgfsetdash{}{0pt}%
\pgfpathmoveto{\pgfqpoint{4.889479in}{2.999157in}}%
\pgfpathlineto{\pgfqpoint{4.902656in}{2.998542in}}%
\pgfpathlineto{\pgfqpoint{4.915842in}{2.998084in}}%
\pgfpathlineto{\pgfqpoint{4.929039in}{2.997783in}}%
\pgfpathlineto{\pgfqpoint{4.942246in}{2.997639in}}%
\pgfpathlineto{\pgfqpoint{4.949484in}{3.009149in}}%
\pgfpathlineto{\pgfqpoint{4.956721in}{3.020854in}}%
\pgfpathlineto{\pgfqpoint{4.963957in}{3.032760in}}%
\pgfpathlineto{\pgfqpoint{4.971194in}{3.044874in}}%
\pgfpathlineto{\pgfqpoint{4.958002in}{3.045571in}}%
\pgfpathlineto{\pgfqpoint{4.944821in}{3.046424in}}%
\pgfpathlineto{\pgfqpoint{4.931650in}{3.047434in}}%
\pgfpathlineto{\pgfqpoint{4.918488in}{3.048601in}}%
\pgfpathlineto{\pgfqpoint{4.911237in}{3.035925in}}%
\pgfpathlineto{\pgfqpoint{4.903984in}{3.023463in}}%
\pgfpathlineto{\pgfqpoint{4.896732in}{3.011209in}}%
\pgfpathlineto{\pgfqpoint{4.889479in}{2.999157in}}%
\pgfpathclose%
\pgfusepath{fill}%
\end{pgfscope}%
\begin{pgfscope}%
\pgfpathrectangle{\pgfqpoint{1.254980in}{0.150000in}}{\pgfqpoint{5.490039in}{5.490039in}}%
\pgfusepath{clip}%
\pgfsetbuttcap%
\pgfsetroundjoin%
\definecolor{currentfill}{rgb}{0.231674,0.318106,0.544834}%
\pgfsetfillcolor{currentfill}%
\pgfsetfillopacity{0.700000}%
\pgfsetlinewidth{0.000000pt}%
\definecolor{currentstroke}{rgb}{0.000000,0.000000,0.000000}%
\pgfsetstrokecolor{currentstroke}%
\pgfsetdash{}{0pt}%
\pgfpathmoveto{\pgfqpoint{4.644461in}{2.870012in}}%
\pgfpathlineto{\pgfqpoint{4.657565in}{2.868963in}}%
\pgfpathlineto{\pgfqpoint{4.670679in}{2.868078in}}%
\pgfpathlineto{\pgfqpoint{4.683801in}{2.867355in}}%
\pgfpathlineto{\pgfqpoint{4.696933in}{2.866793in}}%
\pgfpathlineto{\pgfqpoint{4.704233in}{2.877808in}}%
\pgfpathlineto{\pgfqpoint{4.711531in}{2.888968in}}%
\pgfpathlineto{\pgfqpoint{4.718826in}{2.900279in}}%
\pgfpathlineto{\pgfqpoint{4.726119in}{2.911746in}}%
\pgfpathlineto{\pgfqpoint{4.713000in}{2.912777in}}%
\pgfpathlineto{\pgfqpoint{4.699890in}{2.913969in}}%
\pgfpathlineto{\pgfqpoint{4.686789in}{2.915323in}}%
\pgfpathlineto{\pgfqpoint{4.673697in}{2.916840in}}%
\pgfpathlineto{\pgfqpoint{4.666391in}{2.904894in}}%
\pgfpathlineto{\pgfqpoint{4.659083in}{2.893111in}}%
\pgfpathlineto{\pgfqpoint{4.651773in}{2.881485in}}%
\pgfpathlineto{\pgfqpoint{4.644461in}{2.870012in}}%
\pgfpathclose%
\pgfusepath{fill}%
\end{pgfscope}%
\begin{pgfscope}%
\pgfpathrectangle{\pgfqpoint{1.254980in}{0.150000in}}{\pgfqpoint{5.490039in}{5.490039in}}%
\pgfusepath{clip}%
\pgfsetbuttcap%
\pgfsetroundjoin%
\definecolor{currentfill}{rgb}{0.271828,0.209303,0.504434}%
\pgfsetfillcolor{currentfill}%
\pgfsetfillopacity{0.700000}%
\pgfsetlinewidth{0.000000pt}%
\definecolor{currentstroke}{rgb}{0.000000,0.000000,0.000000}%
\pgfsetstrokecolor{currentstroke}%
\pgfsetdash{}{0pt}%
\pgfpathmoveto{\pgfqpoint{3.405064in}{2.663354in}}%
\pgfpathlineto{\pgfqpoint{3.417956in}{2.653000in}}%
\pgfpathlineto{\pgfqpoint{3.430848in}{2.642870in}}%
\pgfpathlineto{\pgfqpoint{3.443739in}{2.632963in}}%
\pgfpathlineto{\pgfqpoint{3.456630in}{2.623277in}}%
\pgfpathlineto{\pgfqpoint{3.464287in}{2.634402in}}%
\pgfpathlineto{\pgfqpoint{3.471939in}{2.645616in}}%
\pgfpathlineto{\pgfqpoint{3.479585in}{2.656921in}}%
\pgfpathlineto{\pgfqpoint{3.487225in}{2.668320in}}%
\pgfpathlineto{\pgfqpoint{3.474344in}{2.678143in}}%
\pgfpathlineto{\pgfqpoint{3.461462in}{2.688188in}}%
\pgfpathlineto{\pgfqpoint{3.448580in}{2.698456in}}%
\pgfpathlineto{\pgfqpoint{3.435698in}{2.708947in}}%
\pgfpathlineto{\pgfqpoint{3.428048in}{2.697401in}}%
\pgfpathlineto{\pgfqpoint{3.420392in}{2.685955in}}%
\pgfpathlineto{\pgfqpoint{3.412731in}{2.674606in}}%
\pgfpathlineto{\pgfqpoint{3.405064in}{2.663354in}}%
\pgfpathclose%
\pgfusepath{fill}%
\end{pgfscope}%
\begin{pgfscope}%
\pgfpathrectangle{\pgfqpoint{1.254980in}{0.150000in}}{\pgfqpoint{5.490039in}{5.490039in}}%
\pgfusepath{clip}%
\pgfsetbuttcap%
\pgfsetroundjoin%
\definecolor{currentfill}{rgb}{0.197636,0.391528,0.554969}%
\pgfsetfillcolor{currentfill}%
\pgfsetfillopacity{0.700000}%
\pgfsetlinewidth{0.000000pt}%
\definecolor{currentstroke}{rgb}{0.000000,0.000000,0.000000}%
\pgfsetstrokecolor{currentstroke}%
\pgfsetdash{}{0pt}%
\pgfpathmoveto{\pgfqpoint{4.971194in}{3.044874in}}%
\pgfpathlineto{\pgfqpoint{4.984395in}{3.044334in}}%
\pgfpathlineto{\pgfqpoint{4.997606in}{3.043950in}}%
\pgfpathlineto{\pgfqpoint{5.010828in}{3.043721in}}%
\pgfpathlineto{\pgfqpoint{5.024060in}{3.043648in}}%
\pgfpathlineto{\pgfqpoint{5.031280in}{3.055407in}}%
\pgfpathlineto{\pgfqpoint{5.038500in}{3.067380in}}%
\pgfpathlineto{\pgfqpoint{5.045720in}{3.079573in}}%
\pgfpathlineto{\pgfqpoint{5.052941in}{3.091995in}}%
\pgfpathlineto{\pgfqpoint{5.039726in}{3.092648in}}%
\pgfpathlineto{\pgfqpoint{5.026521in}{3.093457in}}%
\pgfpathlineto{\pgfqpoint{5.013326in}{3.094420in}}%
\pgfpathlineto{\pgfqpoint{5.000141in}{3.095540in}}%
\pgfpathlineto{\pgfqpoint{4.992904in}{3.082529in}}%
\pgfpathlineto{\pgfqpoint{4.985667in}{3.069752in}}%
\pgfpathlineto{\pgfqpoint{4.978430in}{3.057203in}}%
\pgfpathlineto{\pgfqpoint{4.971194in}{3.044874in}}%
\pgfpathclose%
\pgfusepath{fill}%
\end{pgfscope}%
\begin{pgfscope}%
\pgfpathrectangle{\pgfqpoint{1.254980in}{0.150000in}}{\pgfqpoint{5.490039in}{5.490039in}}%
\pgfusepath{clip}%
\pgfsetbuttcap%
\pgfsetroundjoin%
\definecolor{currentfill}{rgb}{0.239346,0.300855,0.540844}%
\pgfsetfillcolor{currentfill}%
\pgfsetfillopacity{0.700000}%
\pgfsetlinewidth{0.000000pt}%
\definecolor{currentstroke}{rgb}{0.000000,0.000000,0.000000}%
\pgfsetstrokecolor{currentstroke}%
\pgfsetdash{}{0pt}%
\pgfpathmoveto{\pgfqpoint{4.562808in}{2.829600in}}%
\pgfpathlineto{\pgfqpoint{4.575890in}{2.828337in}}%
\pgfpathlineto{\pgfqpoint{4.588980in}{2.827239in}}%
\pgfpathlineto{\pgfqpoint{4.602078in}{2.826306in}}%
\pgfpathlineto{\pgfqpoint{4.615186in}{2.825536in}}%
\pgfpathlineto{\pgfqpoint{4.622509in}{2.836452in}}%
\pgfpathlineto{\pgfqpoint{4.629829in}{2.847501in}}%
\pgfpathlineto{\pgfqpoint{4.637146in}{2.858685in}}%
\pgfpathlineto{\pgfqpoint{4.644461in}{2.870012in}}%
\pgfpathlineto{\pgfqpoint{4.631365in}{2.871223in}}%
\pgfpathlineto{\pgfqpoint{4.618278in}{2.872599in}}%
\pgfpathlineto{\pgfqpoint{4.605200in}{2.874138in}}%
\pgfpathlineto{\pgfqpoint{4.592130in}{2.875842in}}%
\pgfpathlineto{\pgfqpoint{4.584804in}{2.864064in}}%
\pgfpathlineto{\pgfqpoint{4.577475in}{2.852435in}}%
\pgfpathlineto{\pgfqpoint{4.570143in}{2.840948in}}%
\pgfpathlineto{\pgfqpoint{4.562808in}{2.829600in}}%
\pgfpathclose%
\pgfusepath{fill}%
\end{pgfscope}%
\begin{pgfscope}%
\pgfpathrectangle{\pgfqpoint{1.254980in}{0.150000in}}{\pgfqpoint{5.490039in}{5.490039in}}%
\pgfusepath{clip}%
\pgfsetbuttcap%
\pgfsetroundjoin%
\definecolor{currentfill}{rgb}{0.273006,0.204520,0.501721}%
\pgfsetfillcolor{currentfill}%
\pgfsetfillopacity{0.700000}%
\pgfsetlinewidth{0.000000pt}%
\definecolor{currentstroke}{rgb}{0.000000,0.000000,0.000000}%
\pgfsetstrokecolor{currentstroke}%
\pgfsetdash{}{0pt}%
\pgfpathmoveto{\pgfqpoint{4.020875in}{2.633377in}}%
\pgfpathlineto{\pgfqpoint{4.033815in}{2.629326in}}%
\pgfpathlineto{\pgfqpoint{4.046761in}{2.625458in}}%
\pgfpathlineto{\pgfqpoint{4.059712in}{2.621772in}}%
\pgfpathlineto{\pgfqpoint{4.072669in}{2.618267in}}%
\pgfpathlineto{\pgfqpoint{4.080150in}{2.629234in}}%
\pgfpathlineto{\pgfqpoint{4.087627in}{2.640279in}}%
\pgfpathlineto{\pgfqpoint{4.095099in}{2.651407in}}%
\pgfpathlineto{\pgfqpoint{4.102567in}{2.662620in}}%
\pgfpathlineto{\pgfqpoint{4.089618in}{2.666400in}}%
\pgfpathlineto{\pgfqpoint{4.076675in}{2.670362in}}%
\pgfpathlineto{\pgfqpoint{4.063738in}{2.674505in}}%
\pgfpathlineto{\pgfqpoint{4.050806in}{2.678831in}}%
\pgfpathlineto{\pgfqpoint{4.043329in}{2.667332in}}%
\pgfpathlineto{\pgfqpoint{4.035849in}{2.655926in}}%
\pgfpathlineto{\pgfqpoint{4.028364in}{2.644609in}}%
\pgfpathlineto{\pgfqpoint{4.020875in}{2.633377in}}%
\pgfpathclose%
\pgfusepath{fill}%
\end{pgfscope}%
\begin{pgfscope}%
\pgfpathrectangle{\pgfqpoint{1.254980in}{0.150000in}}{\pgfqpoint{5.490039in}{5.490039in}}%
\pgfusepath{clip}%
\pgfsetbuttcap%
\pgfsetroundjoin%
\definecolor{currentfill}{rgb}{0.187231,0.414746,0.556547}%
\pgfsetfillcolor{currentfill}%
\pgfsetfillopacity{0.700000}%
\pgfsetlinewidth{0.000000pt}%
\definecolor{currentstroke}{rgb}{0.000000,0.000000,0.000000}%
\pgfsetstrokecolor{currentstroke}%
\pgfsetdash{}{0pt}%
\pgfpathmoveto{\pgfqpoint{5.052941in}{3.091995in}}%
\pgfpathlineto{\pgfqpoint{5.066166in}{3.091496in}}%
\pgfpathlineto{\pgfqpoint{5.079402in}{3.091152in}}%
\pgfpathlineto{\pgfqpoint{5.092649in}{3.090962in}}%
\pgfpathlineto{\pgfqpoint{5.105906in}{3.090926in}}%
\pgfpathlineto{\pgfqpoint{5.113110in}{3.102984in}}%
\pgfpathlineto{\pgfqpoint{5.120315in}{3.115276in}}%
\pgfpathlineto{\pgfqpoint{5.127521in}{3.127810in}}%
\pgfpathlineto{\pgfqpoint{5.134729in}{3.140592in}}%
\pgfpathlineto{\pgfqpoint{5.121490in}{3.141236in}}%
\pgfpathlineto{\pgfqpoint{5.108262in}{3.142034in}}%
\pgfpathlineto{\pgfqpoint{5.095044in}{3.142985in}}%
\pgfpathlineto{\pgfqpoint{5.081836in}{3.144091in}}%
\pgfpathlineto{\pgfqpoint{5.074610in}{3.130691in}}%
\pgfpathlineto{\pgfqpoint{5.067386in}{3.117547in}}%
\pgfpathlineto{\pgfqpoint{5.060163in}{3.104650in}}%
\pgfpathlineto{\pgfqpoint{5.052941in}{3.091995in}}%
\pgfpathclose%
\pgfusepath{fill}%
\end{pgfscope}%
\begin{pgfscope}%
\pgfpathrectangle{\pgfqpoint{1.254980in}{0.150000in}}{\pgfqpoint{5.490039in}{5.490039in}}%
\pgfusepath{clip}%
\pgfsetbuttcap%
\pgfsetroundjoin%
\definecolor{currentfill}{rgb}{0.246811,0.283237,0.535941}%
\pgfsetfillcolor{currentfill}%
\pgfsetfillopacity{0.700000}%
\pgfsetlinewidth{0.000000pt}%
\definecolor{currentstroke}{rgb}{0.000000,0.000000,0.000000}%
\pgfsetstrokecolor{currentstroke}%
\pgfsetdash{}{0pt}%
\pgfpathmoveto{\pgfqpoint{4.481156in}{2.790545in}}%
\pgfpathlineto{\pgfqpoint{4.494215in}{2.789032in}}%
\pgfpathlineto{\pgfqpoint{4.507282in}{2.787685in}}%
\pgfpathlineto{\pgfqpoint{4.520357in}{2.786505in}}%
\pgfpathlineto{\pgfqpoint{4.533441in}{2.785491in}}%
\pgfpathlineto{\pgfqpoint{4.540788in}{2.796336in}}%
\pgfpathlineto{\pgfqpoint{4.548131in}{2.807299in}}%
\pgfpathlineto{\pgfqpoint{4.555471in}{2.818385in}}%
\pgfpathlineto{\pgfqpoint{4.562808in}{2.829600in}}%
\pgfpathlineto{\pgfqpoint{4.549736in}{2.831028in}}%
\pgfpathlineto{\pgfqpoint{4.536671in}{2.832622in}}%
\pgfpathlineto{\pgfqpoint{4.523615in}{2.834382in}}%
\pgfpathlineto{\pgfqpoint{4.510567in}{2.836309in}}%
\pgfpathlineto{\pgfqpoint{4.503219in}{2.824670in}}%
\pgfpathlineto{\pgfqpoint{4.495868in}{2.813167in}}%
\pgfpathlineto{\pgfqpoint{4.488514in}{2.801793in}}%
\pgfpathlineto{\pgfqpoint{4.481156in}{2.790545in}}%
\pgfpathclose%
\pgfusepath{fill}%
\end{pgfscope}%
\begin{pgfscope}%
\pgfpathrectangle{\pgfqpoint{1.254980in}{0.150000in}}{\pgfqpoint{5.490039in}{5.490039in}}%
\pgfusepath{clip}%
\pgfsetbuttcap%
\pgfsetroundjoin%
\definecolor{currentfill}{rgb}{0.235526,0.309527,0.542944}%
\pgfsetfillcolor{currentfill}%
\pgfsetfillopacity{0.700000}%
\pgfsetlinewidth{0.000000pt}%
\definecolor{currentstroke}{rgb}{0.000000,0.000000,0.000000}%
\pgfsetstrokecolor{currentstroke}%
\pgfsetdash{}{0pt}%
\pgfpathmoveto{\pgfqpoint{3.115844in}{2.873905in}}%
\pgfpathlineto{\pgfqpoint{3.128802in}{2.858802in}}%
\pgfpathlineto{\pgfqpoint{3.141755in}{2.843962in}}%
\pgfpathlineto{\pgfqpoint{3.154703in}{2.829381in}}%
\pgfpathlineto{\pgfqpoint{3.167647in}{2.815057in}}%
\pgfpathlineto{\pgfqpoint{3.175384in}{2.826340in}}%
\pgfpathlineto{\pgfqpoint{3.183113in}{2.837740in}}%
\pgfpathlineto{\pgfqpoint{3.190837in}{2.849258in}}%
\pgfpathlineto{\pgfqpoint{3.198553in}{2.860897in}}%
\pgfpathlineto{\pgfqpoint{3.185621in}{2.875330in}}%
\pgfpathlineto{\pgfqpoint{3.172685in}{2.890021in}}%
\pgfpathlineto{\pgfqpoint{3.159744in}{2.904972in}}%
\pgfpathlineto{\pgfqpoint{3.146799in}{2.920185in}}%
\pgfpathlineto{\pgfqpoint{3.139070in}{2.908425in}}%
\pgfpathlineto{\pgfqpoint{3.131335in}{2.896794in}}%
\pgfpathlineto{\pgfqpoint{3.123593in}{2.885287in}}%
\pgfpathlineto{\pgfqpoint{3.115844in}{2.873905in}}%
\pgfpathclose%
\pgfusepath{fill}%
\end{pgfscope}%
\begin{pgfscope}%
\pgfpathrectangle{\pgfqpoint{1.254980in}{0.150000in}}{\pgfqpoint{5.490039in}{5.490039in}}%
\pgfusepath{clip}%
\pgfsetbuttcap%
\pgfsetroundjoin%
\definecolor{currentfill}{rgb}{0.223925,0.334994,0.548053}%
\pgfsetfillcolor{currentfill}%
\pgfsetfillopacity{0.700000}%
\pgfsetlinewidth{0.000000pt}%
\definecolor{currentstroke}{rgb}{0.000000,0.000000,0.000000}%
\pgfsetstrokecolor{currentstroke}%
\pgfsetdash{}{0pt}%
\pgfpathmoveto{\pgfqpoint{3.063959in}{2.936983in}}%
\pgfpathlineto{\pgfqpoint{3.076939in}{2.920809in}}%
\pgfpathlineto{\pgfqpoint{3.089913in}{2.904906in}}%
\pgfpathlineto{\pgfqpoint{3.102881in}{2.889272in}}%
\pgfpathlineto{\pgfqpoint{3.115844in}{2.873905in}}%
\pgfpathlineto{\pgfqpoint{3.123593in}{2.885287in}}%
\pgfpathlineto{\pgfqpoint{3.131335in}{2.896794in}}%
\pgfpathlineto{\pgfqpoint{3.139070in}{2.908425in}}%
\pgfpathlineto{\pgfqpoint{3.146799in}{2.920185in}}%
\pgfpathlineto{\pgfqpoint{3.133848in}{2.935662in}}%
\pgfpathlineto{\pgfqpoint{3.120892in}{2.951405in}}%
\pgfpathlineto{\pgfqpoint{3.107931in}{2.967418in}}%
\pgfpathlineto{\pgfqpoint{3.094964in}{2.983702in}}%
\pgfpathlineto{\pgfqpoint{3.087223in}{2.971822in}}%
\pgfpathlineto{\pgfqpoint{3.079476in}{2.960077in}}%
\pgfpathlineto{\pgfqpoint{3.071721in}{2.948465in}}%
\pgfpathlineto{\pgfqpoint{3.063959in}{2.936983in}}%
\pgfpathclose%
\pgfusepath{fill}%
\end{pgfscope}%
\begin{pgfscope}%
\pgfpathrectangle{\pgfqpoint{1.254980in}{0.150000in}}{\pgfqpoint{5.490039in}{5.490039in}}%
\pgfusepath{clip}%
\pgfsetbuttcap%
\pgfsetroundjoin%
\definecolor{currentfill}{rgb}{0.246811,0.283237,0.535941}%
\pgfsetfillcolor{currentfill}%
\pgfsetfillopacity{0.700000}%
\pgfsetlinewidth{0.000000pt}%
\definecolor{currentstroke}{rgb}{0.000000,0.000000,0.000000}%
\pgfsetstrokecolor{currentstroke}%
\pgfsetdash{}{0pt}%
\pgfpathmoveto{\pgfqpoint{3.167647in}{2.815057in}}%
\pgfpathlineto{\pgfqpoint{3.180587in}{2.800989in}}%
\pgfpathlineto{\pgfqpoint{3.193523in}{2.787175in}}%
\pgfpathlineto{\pgfqpoint{3.206456in}{2.773611in}}%
\pgfpathlineto{\pgfqpoint{3.219384in}{2.760297in}}%
\pgfpathlineto{\pgfqpoint{3.227109in}{2.771480in}}%
\pgfpathlineto{\pgfqpoint{3.234827in}{2.782774in}}%
\pgfpathlineto{\pgfqpoint{3.242538in}{2.794179in}}%
\pgfpathlineto{\pgfqpoint{3.250243in}{2.805698in}}%
\pgfpathlineto{\pgfqpoint{3.237326in}{2.819122in}}%
\pgfpathlineto{\pgfqpoint{3.224406in}{2.832795in}}%
\pgfpathlineto{\pgfqpoint{3.211481in}{2.846719in}}%
\pgfpathlineto{\pgfqpoint{3.198553in}{2.860897in}}%
\pgfpathlineto{\pgfqpoint{3.190837in}{2.849258in}}%
\pgfpathlineto{\pgfqpoint{3.183113in}{2.837740in}}%
\pgfpathlineto{\pgfqpoint{3.175384in}{2.826340in}}%
\pgfpathlineto{\pgfqpoint{3.167647in}{2.815057in}}%
\pgfpathclose%
\pgfusepath{fill}%
\end{pgfscope}%
\begin{pgfscope}%
\pgfpathrectangle{\pgfqpoint{1.254980in}{0.150000in}}{\pgfqpoint{5.490039in}{5.490039in}}%
\pgfusepath{clip}%
\pgfsetbuttcap%
\pgfsetroundjoin%
\definecolor{currentfill}{rgb}{0.253935,0.265254,0.529983}%
\pgfsetfillcolor{currentfill}%
\pgfsetfillopacity{0.700000}%
\pgfsetlinewidth{0.000000pt}%
\definecolor{currentstroke}{rgb}{0.000000,0.000000,0.000000}%
\pgfsetstrokecolor{currentstroke}%
\pgfsetdash{}{0pt}%
\pgfpathmoveto{\pgfqpoint{4.399498in}{2.752900in}}%
\pgfpathlineto{\pgfqpoint{4.412535in}{2.751100in}}%
\pgfpathlineto{\pgfqpoint{4.425580in}{2.749469in}}%
\pgfpathlineto{\pgfqpoint{4.438633in}{2.748006in}}%
\pgfpathlineto{\pgfqpoint{4.451694in}{2.746712in}}%
\pgfpathlineto{\pgfqpoint{4.459065in}{2.757505in}}%
\pgfpathlineto{\pgfqpoint{4.466432in}{2.768406in}}%
\pgfpathlineto{\pgfqpoint{4.473796in}{2.779417in}}%
\pgfpathlineto{\pgfqpoint{4.481156in}{2.790545in}}%
\pgfpathlineto{\pgfqpoint{4.468106in}{2.792226in}}%
\pgfpathlineto{\pgfqpoint{4.455063in}{2.794074in}}%
\pgfpathlineto{\pgfqpoint{4.442029in}{2.796091in}}%
\pgfpathlineto{\pgfqpoint{4.429002in}{2.798277in}}%
\pgfpathlineto{\pgfqpoint{4.421631in}{2.786753in}}%
\pgfpathlineto{\pgfqpoint{4.414257in}{2.775352in}}%
\pgfpathlineto{\pgfqpoint{4.406879in}{2.764069in}}%
\pgfpathlineto{\pgfqpoint{4.399498in}{2.752900in}}%
\pgfpathclose%
\pgfusepath{fill}%
\end{pgfscope}%
\begin{pgfscope}%
\pgfpathrectangle{\pgfqpoint{1.254980in}{0.150000in}}{\pgfqpoint{5.490039in}{5.490039in}}%
\pgfusepath{clip}%
\pgfsetbuttcap%
\pgfsetroundjoin%
\definecolor{currentfill}{rgb}{0.179019,0.433756,0.557430}%
\pgfsetfillcolor{currentfill}%
\pgfsetfillopacity{0.700000}%
\pgfsetlinewidth{0.000000pt}%
\definecolor{currentstroke}{rgb}{0.000000,0.000000,0.000000}%
\pgfsetstrokecolor{currentstroke}%
\pgfsetdash{}{0pt}%
\pgfpathmoveto{\pgfqpoint{5.134729in}{3.140592in}}%
\pgfpathlineto{\pgfqpoint{5.147979in}{3.140101in}}%
\pgfpathlineto{\pgfqpoint{5.161239in}{3.139764in}}%
\pgfpathlineto{\pgfqpoint{5.174510in}{3.139579in}}%
\pgfpathlineto{\pgfqpoint{5.187792in}{3.139547in}}%
\pgfpathlineto{\pgfqpoint{5.194983in}{3.151959in}}%
\pgfpathlineto{\pgfqpoint{5.202175in}{3.164627in}}%
\pgfpathlineto{\pgfqpoint{5.209370in}{3.177559in}}%
\pgfpathlineto{\pgfqpoint{5.216568in}{3.190761in}}%
\pgfpathlineto{\pgfqpoint{5.203305in}{3.191429in}}%
\pgfpathlineto{\pgfqpoint{5.190053in}{3.192249in}}%
\pgfpathlineto{\pgfqpoint{5.176812in}{3.193222in}}%
\pgfpathlineto{\pgfqpoint{5.163581in}{3.194347in}}%
\pgfpathlineto{\pgfqpoint{5.156364in}{3.180500in}}%
\pgfpathlineto{\pgfqpoint{5.149151in}{3.166930in}}%
\pgfpathlineto{\pgfqpoint{5.141939in}{3.153630in}}%
\pgfpathlineto{\pgfqpoint{5.134729in}{3.140592in}}%
\pgfpathclose%
\pgfusepath{fill}%
\end{pgfscope}%
\begin{pgfscope}%
\pgfpathrectangle{\pgfqpoint{1.254980in}{0.150000in}}{\pgfqpoint{5.490039in}{5.490039in}}%
\pgfusepath{clip}%
\pgfsetbuttcap%
\pgfsetroundjoin%
\definecolor{currentfill}{rgb}{0.210503,0.363727,0.552206}%
\pgfsetfillcolor{currentfill}%
\pgfsetfillopacity{0.700000}%
\pgfsetlinewidth{0.000000pt}%
\definecolor{currentstroke}{rgb}{0.000000,0.000000,0.000000}%
\pgfsetstrokecolor{currentstroke}%
\pgfsetdash{}{0pt}%
\pgfpathmoveto{\pgfqpoint{3.011978in}{3.004443in}}%
\pgfpathlineto{\pgfqpoint{3.024983in}{2.987158in}}%
\pgfpathlineto{\pgfqpoint{3.037982in}{2.970155in}}%
\pgfpathlineto{\pgfqpoint{3.050974in}{2.953431in}}%
\pgfpathlineto{\pgfqpoint{3.063959in}{2.936983in}}%
\pgfpathlineto{\pgfqpoint{3.071721in}{2.948465in}}%
\pgfpathlineto{\pgfqpoint{3.079476in}{2.960077in}}%
\pgfpathlineto{\pgfqpoint{3.087223in}{2.971822in}}%
\pgfpathlineto{\pgfqpoint{3.094964in}{2.983702in}}%
\pgfpathlineto{\pgfqpoint{3.081991in}{3.000260in}}%
\pgfpathlineto{\pgfqpoint{3.069012in}{3.017094in}}%
\pgfpathlineto{\pgfqpoint{3.056027in}{3.034208in}}%
\pgfpathlineto{\pgfqpoint{3.043035in}{3.051602in}}%
\pgfpathlineto{\pgfqpoint{3.035281in}{3.039602in}}%
\pgfpathlineto{\pgfqpoint{3.027521in}{3.027743in}}%
\pgfpathlineto{\pgfqpoint{3.019753in}{3.016024in}}%
\pgfpathlineto{\pgfqpoint{3.011978in}{3.004443in}}%
\pgfpathclose%
\pgfusepath{fill}%
\end{pgfscope}%
\begin{pgfscope}%
\pgfpathrectangle{\pgfqpoint{1.254980in}{0.150000in}}{\pgfqpoint{5.490039in}{5.490039in}}%
\pgfusepath{clip}%
\pgfsetbuttcap%
\pgfsetroundjoin%
\definecolor{currentfill}{rgb}{0.278012,0.180367,0.486697}%
\pgfsetfillcolor{currentfill}%
\pgfsetfillopacity{0.700000}%
\pgfsetlinewidth{0.000000pt}%
\definecolor{currentstroke}{rgb}{0.000000,0.000000,0.000000}%
\pgfsetstrokecolor{currentstroke}%
\pgfsetdash{}{0pt}%
\pgfpathmoveto{\pgfqpoint{3.590296in}{2.597517in}}%
\pgfpathlineto{\pgfqpoint{3.603184in}{2.589619in}}%
\pgfpathlineto{\pgfqpoint{3.616075in}{2.581929in}}%
\pgfpathlineto{\pgfqpoint{3.628967in}{2.574444in}}%
\pgfpathlineto{\pgfqpoint{3.641862in}{2.567165in}}%
\pgfpathlineto{\pgfqpoint{3.649469in}{2.578194in}}%
\pgfpathlineto{\pgfqpoint{3.657071in}{2.589300in}}%
\pgfpathlineto{\pgfqpoint{3.664668in}{2.600484in}}%
\pgfpathlineto{\pgfqpoint{3.672260in}{2.611750in}}%
\pgfpathlineto{\pgfqpoint{3.659375in}{2.619194in}}%
\pgfpathlineto{\pgfqpoint{3.646491in}{2.626844in}}%
\pgfpathlineto{\pgfqpoint{3.633610in}{2.634700in}}%
\pgfpathlineto{\pgfqpoint{3.620730in}{2.642763in}}%
\pgfpathlineto{\pgfqpoint{3.613129in}{2.631322in}}%
\pgfpathlineto{\pgfqpoint{3.605523in}{2.619968in}}%
\pgfpathlineto{\pgfqpoint{3.597912in}{2.608701in}}%
\pgfpathlineto{\pgfqpoint{3.590296in}{2.597517in}}%
\pgfpathclose%
\pgfusepath{fill}%
\end{pgfscope}%
\begin{pgfscope}%
\pgfpathrectangle{\pgfqpoint{1.254980in}{0.150000in}}{\pgfqpoint{5.490039in}{5.490039in}}%
\pgfusepath{clip}%
\pgfsetbuttcap%
\pgfsetroundjoin%
\definecolor{currentfill}{rgb}{0.257322,0.256130,0.526563}%
\pgfsetfillcolor{currentfill}%
\pgfsetfillopacity{0.700000}%
\pgfsetlinewidth{0.000000pt}%
\definecolor{currentstroke}{rgb}{0.000000,0.000000,0.000000}%
\pgfsetstrokecolor{currentstroke}%
\pgfsetdash{}{0pt}%
\pgfpathmoveto{\pgfqpoint{3.219384in}{2.760297in}}%
\pgfpathlineto{\pgfqpoint{3.232310in}{2.747230in}}%
\pgfpathlineto{\pgfqpoint{3.245233in}{2.734409in}}%
\pgfpathlineto{\pgfqpoint{3.258152in}{2.721830in}}%
\pgfpathlineto{\pgfqpoint{3.271069in}{2.709493in}}%
\pgfpathlineto{\pgfqpoint{3.278782in}{2.720577in}}%
\pgfpathlineto{\pgfqpoint{3.286488in}{2.731764in}}%
\pgfpathlineto{\pgfqpoint{3.294189in}{2.743056in}}%
\pgfpathlineto{\pgfqpoint{3.301883in}{2.754455in}}%
\pgfpathlineto{\pgfqpoint{3.288977in}{2.766901in}}%
\pgfpathlineto{\pgfqpoint{3.276069in}{2.779590in}}%
\pgfpathlineto{\pgfqpoint{3.263158in}{2.792521in}}%
\pgfpathlineto{\pgfqpoint{3.250243in}{2.805698in}}%
\pgfpathlineto{\pgfqpoint{3.242538in}{2.794179in}}%
\pgfpathlineto{\pgfqpoint{3.234827in}{2.782774in}}%
\pgfpathlineto{\pgfqpoint{3.227109in}{2.771480in}}%
\pgfpathlineto{\pgfqpoint{3.219384in}{2.760297in}}%
\pgfpathclose%
\pgfusepath{fill}%
\end{pgfscope}%
\begin{pgfscope}%
\pgfpathrectangle{\pgfqpoint{1.254980in}{0.150000in}}{\pgfqpoint{5.490039in}{5.490039in}}%
\pgfusepath{clip}%
\pgfsetbuttcap%
\pgfsetroundjoin%
\definecolor{currentfill}{rgb}{0.260571,0.246922,0.522828}%
\pgfsetfillcolor{currentfill}%
\pgfsetfillopacity{0.700000}%
\pgfsetlinewidth{0.000000pt}%
\definecolor{currentstroke}{rgb}{0.000000,0.000000,0.000000}%
\pgfsetstrokecolor{currentstroke}%
\pgfsetdash{}{0pt}%
\pgfpathmoveto{\pgfqpoint{4.317828in}{2.716742in}}%
\pgfpathlineto{\pgfqpoint{4.330844in}{2.714618in}}%
\pgfpathlineto{\pgfqpoint{4.343868in}{2.712665in}}%
\pgfpathlineto{\pgfqpoint{4.356899in}{2.710884in}}%
\pgfpathlineto{\pgfqpoint{4.369938in}{2.709273in}}%
\pgfpathlineto{\pgfqpoint{4.377334in}{2.720031in}}%
\pgfpathlineto{\pgfqpoint{4.384726in}{2.730885in}}%
\pgfpathlineto{\pgfqpoint{4.392114in}{2.741840in}}%
\pgfpathlineto{\pgfqpoint{4.399498in}{2.752900in}}%
\pgfpathlineto{\pgfqpoint{4.386469in}{2.754870in}}%
\pgfpathlineto{\pgfqpoint{4.373447in}{2.757010in}}%
\pgfpathlineto{\pgfqpoint{4.360433in}{2.759321in}}%
\pgfpathlineto{\pgfqpoint{4.347426in}{2.761803in}}%
\pgfpathlineto{\pgfqpoint{4.340032in}{2.750375in}}%
\pgfpathlineto{\pgfqpoint{4.332634in}{2.739058in}}%
\pgfpathlineto{\pgfqpoint{4.325233in}{2.727848in}}%
\pgfpathlineto{\pgfqpoint{4.317828in}{2.716742in}}%
\pgfpathclose%
\pgfusepath{fill}%
\end{pgfscope}%
\begin{pgfscope}%
\pgfpathrectangle{\pgfqpoint{1.254980in}{0.150000in}}{\pgfqpoint{5.490039in}{5.490039in}}%
\pgfusepath{clip}%
\pgfsetbuttcap%
\pgfsetroundjoin%
\definecolor{currentfill}{rgb}{0.276194,0.190074,0.493001}%
\pgfsetfillcolor{currentfill}%
\pgfsetfillopacity{0.700000}%
\pgfsetlinewidth{0.000000pt}%
\definecolor{currentstroke}{rgb}{0.000000,0.000000,0.000000}%
\pgfsetstrokecolor{currentstroke}%
\pgfsetdash{}{0pt}%
\pgfpathmoveto{\pgfqpoint{3.939130in}{2.606301in}}%
\pgfpathlineto{\pgfqpoint{3.952059in}{2.601759in}}%
\pgfpathlineto{\pgfqpoint{3.964992in}{2.597402in}}%
\pgfpathlineto{\pgfqpoint{3.977930in}{2.593231in}}%
\pgfpathlineto{\pgfqpoint{3.990873in}{2.589244in}}%
\pgfpathlineto{\pgfqpoint{3.998380in}{2.600165in}}%
\pgfpathlineto{\pgfqpoint{4.005883in}{2.611158in}}%
\pgfpathlineto{\pgfqpoint{4.013381in}{2.622228in}}%
\pgfpathlineto{\pgfqpoint{4.020875in}{2.633377in}}%
\pgfpathlineto{\pgfqpoint{4.007940in}{2.637612in}}%
\pgfpathlineto{\pgfqpoint{3.995010in}{2.642031in}}%
\pgfpathlineto{\pgfqpoint{3.982085in}{2.646635in}}%
\pgfpathlineto{\pgfqpoint{3.969165in}{2.651426in}}%
\pgfpathlineto{\pgfqpoint{3.961663in}{2.640019in}}%
\pgfpathlineto{\pgfqpoint{3.954157in}{2.628698in}}%
\pgfpathlineto{\pgfqpoint{3.946646in}{2.617460in}}%
\pgfpathlineto{\pgfqpoint{3.939130in}{2.606301in}}%
\pgfpathclose%
\pgfusepath{fill}%
\end{pgfscope}%
\begin{pgfscope}%
\pgfpathrectangle{\pgfqpoint{1.254980in}{0.150000in}}{\pgfqpoint{5.490039in}{5.490039in}}%
\pgfusepath{clip}%
\pgfsetbuttcap%
\pgfsetroundjoin%
\definecolor{currentfill}{rgb}{0.278012,0.180367,0.486697}%
\pgfsetfillcolor{currentfill}%
\pgfsetfillopacity{0.700000}%
\pgfsetlinewidth{0.000000pt}%
\definecolor{currentstroke}{rgb}{0.000000,0.000000,0.000000}%
\pgfsetstrokecolor{currentstroke}%
\pgfsetdash{}{0pt}%
\pgfpathmoveto{\pgfqpoint{3.723826in}{2.583995in}}%
\pgfpathlineto{\pgfqpoint{3.736724in}{2.577557in}}%
\pgfpathlineto{\pgfqpoint{3.749625in}{2.571317in}}%
\pgfpathlineto{\pgfqpoint{3.762530in}{2.565274in}}%
\pgfpathlineto{\pgfqpoint{3.775438in}{2.559427in}}%
\pgfpathlineto{\pgfqpoint{3.783008in}{2.570410in}}%
\pgfpathlineto{\pgfqpoint{3.790573in}{2.581466in}}%
\pgfpathlineto{\pgfqpoint{3.798133in}{2.592597in}}%
\pgfpathlineto{\pgfqpoint{3.805688in}{2.603804in}}%
\pgfpathlineto{\pgfqpoint{3.792789in}{2.609845in}}%
\pgfpathlineto{\pgfqpoint{3.779893in}{2.616081in}}%
\pgfpathlineto{\pgfqpoint{3.767000in}{2.622514in}}%
\pgfpathlineto{\pgfqpoint{3.754110in}{2.629144in}}%
\pgfpathlineto{\pgfqpoint{3.746546in}{2.617734in}}%
\pgfpathlineto{\pgfqpoint{3.738978in}{2.606407in}}%
\pgfpathlineto{\pgfqpoint{3.731404in}{2.595162in}}%
\pgfpathlineto{\pgfqpoint{3.723826in}{2.583995in}}%
\pgfpathclose%
\pgfusepath{fill}%
\end{pgfscope}%
\begin{pgfscope}%
\pgfpathrectangle{\pgfqpoint{1.254980in}{0.150000in}}{\pgfqpoint{5.490039in}{5.490039in}}%
\pgfusepath{clip}%
\pgfsetbuttcap%
\pgfsetroundjoin%
\definecolor{currentfill}{rgb}{0.275191,0.194905,0.496005}%
\pgfsetfillcolor{currentfill}%
\pgfsetfillopacity{0.700000}%
\pgfsetlinewidth{0.000000pt}%
\definecolor{currentstroke}{rgb}{0.000000,0.000000,0.000000}%
\pgfsetstrokecolor{currentstroke}%
\pgfsetdash{}{0pt}%
\pgfpathmoveto{\pgfqpoint{3.456630in}{2.623277in}}%
\pgfpathlineto{\pgfqpoint{3.469522in}{2.613810in}}%
\pgfpathlineto{\pgfqpoint{3.482413in}{2.604562in}}%
\pgfpathlineto{\pgfqpoint{3.495305in}{2.595530in}}%
\pgfpathlineto{\pgfqpoint{3.508198in}{2.586714in}}%
\pgfpathlineto{\pgfqpoint{3.515845in}{2.597712in}}%
\pgfpathlineto{\pgfqpoint{3.523487in}{2.608792in}}%
\pgfpathlineto{\pgfqpoint{3.531123in}{2.619956in}}%
\pgfpathlineto{\pgfqpoint{3.538754in}{2.631207in}}%
\pgfpathlineto{\pgfqpoint{3.525871in}{2.640161in}}%
\pgfpathlineto{\pgfqpoint{3.512989in}{2.649330in}}%
\pgfpathlineto{\pgfqpoint{3.500107in}{2.658716in}}%
\pgfpathlineto{\pgfqpoint{3.487225in}{2.668320in}}%
\pgfpathlineto{\pgfqpoint{3.479585in}{2.656921in}}%
\pgfpathlineto{\pgfqpoint{3.471939in}{2.645616in}}%
\pgfpathlineto{\pgfqpoint{3.464287in}{2.634402in}}%
\pgfpathlineto{\pgfqpoint{3.456630in}{2.623277in}}%
\pgfpathclose%
\pgfusepath{fill}%
\end{pgfscope}%
\begin{pgfscope}%
\pgfpathrectangle{\pgfqpoint{1.254980in}{0.150000in}}{\pgfqpoint{5.490039in}{5.490039in}}%
\pgfusepath{clip}%
\pgfsetbuttcap%
\pgfsetroundjoin%
\definecolor{currentfill}{rgb}{0.195860,0.395433,0.555276}%
\pgfsetfillcolor{currentfill}%
\pgfsetfillopacity{0.700000}%
\pgfsetlinewidth{0.000000pt}%
\definecolor{currentstroke}{rgb}{0.000000,0.000000,0.000000}%
\pgfsetstrokecolor{currentstroke}%
\pgfsetdash{}{0pt}%
\pgfpathmoveto{\pgfqpoint{2.959883in}{3.076449in}}%
\pgfpathlineto{\pgfqpoint{2.972918in}{3.058012in}}%
\pgfpathlineto{\pgfqpoint{2.985946in}{3.039867in}}%
\pgfpathlineto{\pgfqpoint{2.998965in}{3.022012in}}%
\pgfpathlineto{\pgfqpoint{3.011978in}{3.004443in}}%
\pgfpathlineto{\pgfqpoint{3.019753in}{3.016024in}}%
\pgfpathlineto{\pgfqpoint{3.027521in}{3.027743in}}%
\pgfpathlineto{\pgfqpoint{3.035281in}{3.039602in}}%
\pgfpathlineto{\pgfqpoint{3.043035in}{3.051602in}}%
\pgfpathlineto{\pgfqpoint{3.030036in}{3.069281in}}%
\pgfpathlineto{\pgfqpoint{3.017029in}{3.087247in}}%
\pgfpathlineto{\pgfqpoint{3.004015in}{3.105502in}}%
\pgfpathlineto{\pgfqpoint{2.990994in}{3.124049in}}%
\pgfpathlineto{\pgfqpoint{2.983227in}{3.111928in}}%
\pgfpathlineto{\pgfqpoint{2.975453in}{3.099956in}}%
\pgfpathlineto{\pgfqpoint{2.967672in}{3.088130in}}%
\pgfpathlineto{\pgfqpoint{2.959883in}{3.076449in}}%
\pgfpathclose%
\pgfusepath{fill}%
\end{pgfscope}%
\begin{pgfscope}%
\pgfpathrectangle{\pgfqpoint{1.254980in}{0.150000in}}{\pgfqpoint{5.490039in}{5.490039in}}%
\pgfusepath{clip}%
\pgfsetbuttcap%
\pgfsetroundjoin%
\definecolor{currentfill}{rgb}{0.171176,0.452530,0.557965}%
\pgfsetfillcolor{currentfill}%
\pgfsetfillopacity{0.700000}%
\pgfsetlinewidth{0.000000pt}%
\definecolor{currentstroke}{rgb}{0.000000,0.000000,0.000000}%
\pgfsetstrokecolor{currentstroke}%
\pgfsetdash{}{0pt}%
\pgfpathmoveto{\pgfqpoint{5.216568in}{3.190761in}}%
\pgfpathlineto{\pgfqpoint{5.229841in}{3.190245in}}%
\pgfpathlineto{\pgfqpoint{5.243125in}{3.189881in}}%
\pgfpathlineto{\pgfqpoint{5.256420in}{3.189669in}}%
\pgfpathlineto{\pgfqpoint{5.269726in}{3.189608in}}%
\pgfpathlineto{\pgfqpoint{5.276906in}{3.202435in}}%
\pgfpathlineto{\pgfqpoint{5.284089in}{3.215541in}}%
\pgfpathlineto{\pgfqpoint{5.291276in}{3.228933in}}%
\pgfpathlineto{\pgfqpoint{5.277985in}{3.229490in}}%
\pgfpathlineto{\pgfqpoint{5.264705in}{3.230197in}}%
\pgfpathlineto{\pgfqpoint{5.251436in}{3.231056in}}%
\pgfpathlineto{\pgfqpoint{5.238178in}{3.232066in}}%
\pgfpathlineto{\pgfqpoint{5.230971in}{3.218007in}}%
\pgfpathlineto{\pgfqpoint{5.223768in}{3.204242in}}%
\pgfpathlineto{\pgfqpoint{5.216568in}{3.190761in}}%
\pgfpathclose%
\pgfusepath{fill}%
\end{pgfscope}%
\begin{pgfscope}%
\pgfpathrectangle{\pgfqpoint{1.254980in}{0.150000in}}{\pgfqpoint{5.490039in}{5.490039in}}%
\pgfusepath{clip}%
\pgfsetbuttcap%
\pgfsetroundjoin%
\definecolor{currentfill}{rgb}{0.263663,0.237631,0.518762}%
\pgfsetfillcolor{currentfill}%
\pgfsetfillopacity{0.700000}%
\pgfsetlinewidth{0.000000pt}%
\definecolor{currentstroke}{rgb}{0.000000,0.000000,0.000000}%
\pgfsetstrokecolor{currentstroke}%
\pgfsetdash{}{0pt}%
\pgfpathmoveto{\pgfqpoint{3.271069in}{2.709493in}}%
\pgfpathlineto{\pgfqpoint{3.283984in}{2.697396in}}%
\pgfpathlineto{\pgfqpoint{3.296897in}{2.685536in}}%
\pgfpathlineto{\pgfqpoint{3.309807in}{2.673912in}}%
\pgfpathlineto{\pgfqpoint{3.322716in}{2.662523in}}%
\pgfpathlineto{\pgfqpoint{3.330417in}{2.673507in}}%
\pgfpathlineto{\pgfqpoint{3.338113in}{2.684588in}}%
\pgfpathlineto{\pgfqpoint{3.345802in}{2.695766in}}%
\pgfpathlineto{\pgfqpoint{3.353485in}{2.707045in}}%
\pgfpathlineto{\pgfqpoint{3.340587in}{2.718544in}}%
\pgfpathlineto{\pgfqpoint{3.327688in}{2.730278in}}%
\pgfpathlineto{\pgfqpoint{3.314786in}{2.742247in}}%
\pgfpathlineto{\pgfqpoint{3.301883in}{2.754455in}}%
\pgfpathlineto{\pgfqpoint{3.294189in}{2.743056in}}%
\pgfpathlineto{\pgfqpoint{3.286488in}{2.731764in}}%
\pgfpathlineto{\pgfqpoint{3.278782in}{2.720577in}}%
\pgfpathlineto{\pgfqpoint{3.271069in}{2.709493in}}%
\pgfpathclose%
\pgfusepath{fill}%
\end{pgfscope}%
\begin{pgfscope}%
\pgfpathrectangle{\pgfqpoint{1.254980in}{0.150000in}}{\pgfqpoint{5.490039in}{5.490039in}}%
\pgfusepath{clip}%
\pgfsetbuttcap%
\pgfsetroundjoin%
\definecolor{currentfill}{rgb}{0.265145,0.232956,0.516599}%
\pgfsetfillcolor{currentfill}%
\pgfsetfillopacity{0.700000}%
\pgfsetlinewidth{0.000000pt}%
\definecolor{currentstroke}{rgb}{0.000000,0.000000,0.000000}%
\pgfsetstrokecolor{currentstroke}%
\pgfsetdash{}{0pt}%
\pgfpathmoveto{\pgfqpoint{4.236138in}{2.682168in}}%
\pgfpathlineto{\pgfqpoint{4.249135in}{2.679682in}}%
\pgfpathlineto{\pgfqpoint{4.262139in}{2.677370in}}%
\pgfpathlineto{\pgfqpoint{4.275150in}{2.675233in}}%
\pgfpathlineto{\pgfqpoint{4.288168in}{2.673267in}}%
\pgfpathlineto{\pgfqpoint{4.295589in}{2.684001in}}%
\pgfpathlineto{\pgfqpoint{4.303006in}{2.694822in}}%
\pgfpathlineto{\pgfqpoint{4.310419in}{2.705735in}}%
\pgfpathlineto{\pgfqpoint{4.317828in}{2.716742in}}%
\pgfpathlineto{\pgfqpoint{4.304819in}{2.719038in}}%
\pgfpathlineto{\pgfqpoint{4.291817in}{2.721507in}}%
\pgfpathlineto{\pgfqpoint{4.278822in}{2.724149in}}%
\pgfpathlineto{\pgfqpoint{4.265834in}{2.726966in}}%
\pgfpathlineto{\pgfqpoint{4.258416in}{2.715617in}}%
\pgfpathlineto{\pgfqpoint{4.250994in}{2.704371in}}%
\pgfpathlineto{\pgfqpoint{4.243568in}{2.693222in}}%
\pgfpathlineto{\pgfqpoint{4.236138in}{2.682168in}}%
\pgfpathclose%
\pgfusepath{fill}%
\end{pgfscope}%
\begin{pgfscope}%
\pgfpathrectangle{\pgfqpoint{1.254980in}{0.150000in}}{\pgfqpoint{5.490039in}{5.490039in}}%
\pgfusepath{clip}%
\pgfsetbuttcap%
\pgfsetroundjoin%
\definecolor{currentfill}{rgb}{0.278012,0.180367,0.486697}%
\pgfsetfillcolor{currentfill}%
\pgfsetfillopacity{0.700000}%
\pgfsetlinewidth{0.000000pt}%
\definecolor{currentstroke}{rgb}{0.000000,0.000000,0.000000}%
\pgfsetstrokecolor{currentstroke}%
\pgfsetdash{}{0pt}%
\pgfpathmoveto{\pgfqpoint{3.857322in}{2.581581in}}%
\pgfpathlineto{\pgfqpoint{3.870241in}{2.576505in}}%
\pgfpathlineto{\pgfqpoint{3.883164in}{2.571618in}}%
\pgfpathlineto{\pgfqpoint{3.896091in}{2.566921in}}%
\pgfpathlineto{\pgfqpoint{3.909023in}{2.562411in}}%
\pgfpathlineto{\pgfqpoint{3.916557in}{2.573278in}}%
\pgfpathlineto{\pgfqpoint{3.924086in}{2.584214in}}%
\pgfpathlineto{\pgfqpoint{3.931610in}{2.595220in}}%
\pgfpathlineto{\pgfqpoint{3.939130in}{2.606301in}}%
\pgfpathlineto{\pgfqpoint{3.926207in}{2.611031in}}%
\pgfpathlineto{\pgfqpoint{3.913288in}{2.615949in}}%
\pgfpathlineto{\pgfqpoint{3.900373in}{2.621056in}}%
\pgfpathlineto{\pgfqpoint{3.887463in}{2.626353in}}%
\pgfpathlineto{\pgfqpoint{3.879935in}{2.615041in}}%
\pgfpathlineto{\pgfqpoint{3.872402in}{2.603811in}}%
\pgfpathlineto{\pgfqpoint{3.864864in}{2.592658in}}%
\pgfpathlineto{\pgfqpoint{3.857322in}{2.581581in}}%
\pgfpathclose%
\pgfusepath{fill}%
\end{pgfscope}%
\begin{pgfscope}%
\pgfpathrectangle{\pgfqpoint{1.254980in}{0.150000in}}{\pgfqpoint{5.490039in}{5.490039in}}%
\pgfusepath{clip}%
\pgfsetbuttcap%
\pgfsetroundjoin%
\definecolor{currentfill}{rgb}{0.269308,0.218818,0.509577}%
\pgfsetfillcolor{currentfill}%
\pgfsetfillopacity{0.700000}%
\pgfsetlinewidth{0.000000pt}%
\definecolor{currentstroke}{rgb}{0.000000,0.000000,0.000000}%
\pgfsetstrokecolor{currentstroke}%
\pgfsetdash{}{0pt}%
\pgfpathmoveto{\pgfqpoint{4.154421in}{2.649296in}}%
\pgfpathlineto{\pgfqpoint{4.167400in}{2.646410in}}%
\pgfpathlineto{\pgfqpoint{4.180386in}{2.643702in}}%
\pgfpathlineto{\pgfqpoint{4.193378in}{2.641170in}}%
\pgfpathlineto{\pgfqpoint{4.206377in}{2.638813in}}%
\pgfpathlineto{\pgfqpoint{4.213823in}{2.649529in}}%
\pgfpathlineto{\pgfqpoint{4.221266in}{2.660325in}}%
\pgfpathlineto{\pgfqpoint{4.228704in}{2.671203in}}%
\pgfpathlineto{\pgfqpoint{4.236138in}{2.682168in}}%
\pgfpathlineto{\pgfqpoint{4.223148in}{2.684828in}}%
\pgfpathlineto{\pgfqpoint{4.210164in}{2.687664in}}%
\pgfpathlineto{\pgfqpoint{4.197187in}{2.690675in}}%
\pgfpathlineto{\pgfqpoint{4.184217in}{2.693864in}}%
\pgfpathlineto{\pgfqpoint{4.176774in}{2.682586in}}%
\pgfpathlineto{\pgfqpoint{4.169327in}{2.671401in}}%
\pgfpathlineto{\pgfqpoint{4.161876in}{2.660305in}}%
\pgfpathlineto{\pgfqpoint{4.154421in}{2.649296in}}%
\pgfpathclose%
\pgfusepath{fill}%
\end{pgfscope}%
\begin{pgfscope}%
\pgfpathrectangle{\pgfqpoint{1.254980in}{0.150000in}}{\pgfqpoint{5.490039in}{5.490039in}}%
\pgfusepath{clip}%
\pgfsetbuttcap%
\pgfsetroundjoin%
\definecolor{currentfill}{rgb}{0.182256,0.426184,0.557120}%
\pgfsetfillcolor{currentfill}%
\pgfsetfillopacity{0.700000}%
\pgfsetlinewidth{0.000000pt}%
\definecolor{currentstroke}{rgb}{0.000000,0.000000,0.000000}%
\pgfsetstrokecolor{currentstroke}%
\pgfsetdash{}{0pt}%
\pgfpathmoveto{\pgfqpoint{2.907658in}{3.153178in}}%
\pgfpathlineto{\pgfqpoint{2.920727in}{3.133543in}}%
\pgfpathlineto{\pgfqpoint{2.933788in}{3.114212in}}%
\pgfpathlineto{\pgfqpoint{2.946840in}{3.095181in}}%
\pgfpathlineto{\pgfqpoint{2.959883in}{3.076449in}}%
\pgfpathlineto{\pgfqpoint{2.967672in}{3.088130in}}%
\pgfpathlineto{\pgfqpoint{2.975453in}{3.099956in}}%
\pgfpathlineto{\pgfqpoint{2.983227in}{3.111928in}}%
\pgfpathlineto{\pgfqpoint{2.990994in}{3.124049in}}%
\pgfpathlineto{\pgfqpoint{2.977964in}{3.142892in}}%
\pgfpathlineto{\pgfqpoint{2.964926in}{3.162033in}}%
\pgfpathlineto{\pgfqpoint{2.951879in}{3.181474in}}%
\pgfpathlineto{\pgfqpoint{2.938824in}{3.201220in}}%
\pgfpathlineto{\pgfqpoint{2.931044in}{3.188977in}}%
\pgfpathlineto{\pgfqpoint{2.923256in}{3.176891in}}%
\pgfpathlineto{\pgfqpoint{2.915461in}{3.164958in}}%
\pgfpathlineto{\pgfqpoint{2.907658in}{3.153178in}}%
\pgfpathclose%
\pgfusepath{fill}%
\end{pgfscope}%
\begin{pgfscope}%
\pgfpathrectangle{\pgfqpoint{1.254980in}{0.150000in}}{\pgfqpoint{5.490039in}{5.490039in}}%
\pgfusepath{clip}%
\pgfsetbuttcap%
\pgfsetroundjoin%
\definecolor{currentfill}{rgb}{0.269308,0.218818,0.509577}%
\pgfsetfillcolor{currentfill}%
\pgfsetfillopacity{0.700000}%
\pgfsetlinewidth{0.000000pt}%
\definecolor{currentstroke}{rgb}{0.000000,0.000000,0.000000}%
\pgfsetstrokecolor{currentstroke}%
\pgfsetdash{}{0pt}%
\pgfpathmoveto{\pgfqpoint{3.322716in}{2.662523in}}%
\pgfpathlineto{\pgfqpoint{3.335623in}{2.651366in}}%
\pgfpathlineto{\pgfqpoint{3.348529in}{2.640439in}}%
\pgfpathlineto{\pgfqpoint{3.361434in}{2.629742in}}%
\pgfpathlineto{\pgfqpoint{3.374338in}{2.619272in}}%
\pgfpathlineto{\pgfqpoint{3.382028in}{2.630157in}}%
\pgfpathlineto{\pgfqpoint{3.389712in}{2.641131in}}%
\pgfpathlineto{\pgfqpoint{3.397391in}{2.652196in}}%
\pgfpathlineto{\pgfqpoint{3.405064in}{2.663354in}}%
\pgfpathlineto{\pgfqpoint{3.392171in}{2.673934in}}%
\pgfpathlineto{\pgfqpoint{3.379277in}{2.684741in}}%
\pgfpathlineto{\pgfqpoint{3.366382in}{2.695778in}}%
\pgfpathlineto{\pgfqpoint{3.353485in}{2.707045in}}%
\pgfpathlineto{\pgfqpoint{3.345802in}{2.695766in}}%
\pgfpathlineto{\pgfqpoint{3.338113in}{2.684588in}}%
\pgfpathlineto{\pgfqpoint{3.330417in}{2.673507in}}%
\pgfpathlineto{\pgfqpoint{3.322716in}{2.662523in}}%
\pgfpathclose%
\pgfusepath{fill}%
\end{pgfscope}%
\begin{pgfscope}%
\pgfpathrectangle{\pgfqpoint{1.254980in}{0.150000in}}{\pgfqpoint{5.490039in}{5.490039in}}%
\pgfusepath{clip}%
\pgfsetbuttcap%
\pgfsetroundjoin%
\definecolor{currentfill}{rgb}{0.278826,0.175490,0.483397}%
\pgfsetfillcolor{currentfill}%
\pgfsetfillopacity{0.700000}%
\pgfsetlinewidth{0.000000pt}%
\definecolor{currentstroke}{rgb}{0.000000,0.000000,0.000000}%
\pgfsetstrokecolor{currentstroke}%
\pgfsetdash{}{0pt}%
\pgfpathmoveto{\pgfqpoint{3.641862in}{2.567165in}}%
\pgfpathlineto{\pgfqpoint{3.654758in}{2.560089in}}%
\pgfpathlineto{\pgfqpoint{3.667657in}{2.553216in}}%
\pgfpathlineto{\pgfqpoint{3.680558in}{2.546544in}}%
\pgfpathlineto{\pgfqpoint{3.693463in}{2.540072in}}%
\pgfpathlineto{\pgfqpoint{3.701061in}{2.550946in}}%
\pgfpathlineto{\pgfqpoint{3.708654in}{2.561890in}}%
\pgfpathlineto{\pgfqpoint{3.716243in}{2.572905in}}%
\pgfpathlineto{\pgfqpoint{3.723826in}{2.583995in}}%
\pgfpathlineto{\pgfqpoint{3.710931in}{2.590633in}}%
\pgfpathlineto{\pgfqpoint{3.698038in}{2.597470in}}%
\pgfpathlineto{\pgfqpoint{3.685148in}{2.604509in}}%
\pgfpathlineto{\pgfqpoint{3.672260in}{2.611750in}}%
\pgfpathlineto{\pgfqpoint{3.664668in}{2.600484in}}%
\pgfpathlineto{\pgfqpoint{3.657071in}{2.589300in}}%
\pgfpathlineto{\pgfqpoint{3.649469in}{2.578194in}}%
\pgfpathlineto{\pgfqpoint{3.641862in}{2.567165in}}%
\pgfpathclose%
\pgfusepath{fill}%
\end{pgfscope}%
\begin{pgfscope}%
\pgfpathrectangle{\pgfqpoint{1.254980in}{0.150000in}}{\pgfqpoint{5.490039in}{5.490039in}}%
\pgfusepath{clip}%
\pgfsetbuttcap%
\pgfsetroundjoin%
\definecolor{currentfill}{rgb}{0.278012,0.180367,0.486697}%
\pgfsetfillcolor{currentfill}%
\pgfsetfillopacity{0.700000}%
\pgfsetlinewidth{0.000000pt}%
\definecolor{currentstroke}{rgb}{0.000000,0.000000,0.000000}%
\pgfsetstrokecolor{currentstroke}%
\pgfsetdash{}{0pt}%
\pgfpathmoveto{\pgfqpoint{3.508198in}{2.586714in}}%
\pgfpathlineto{\pgfqpoint{3.521091in}{2.578112in}}%
\pgfpathlineto{\pgfqpoint{3.533985in}{2.569722in}}%
\pgfpathlineto{\pgfqpoint{3.546881in}{2.561544in}}%
\pgfpathlineto{\pgfqpoint{3.559778in}{2.553575in}}%
\pgfpathlineto{\pgfqpoint{3.567415in}{2.564445in}}%
\pgfpathlineto{\pgfqpoint{3.575047in}{2.575391in}}%
\pgfpathlineto{\pgfqpoint{3.582674in}{2.586414in}}%
\pgfpathlineto{\pgfqpoint{3.590296in}{2.597517in}}%
\pgfpathlineto{\pgfqpoint{3.577408in}{2.605623in}}%
\pgfpathlineto{\pgfqpoint{3.564522in}{2.613939in}}%
\pgfpathlineto{\pgfqpoint{3.551638in}{2.622467in}}%
\pgfpathlineto{\pgfqpoint{3.538754in}{2.631207in}}%
\pgfpathlineto{\pgfqpoint{3.531123in}{2.619956in}}%
\pgfpathlineto{\pgfqpoint{3.523487in}{2.608792in}}%
\pgfpathlineto{\pgfqpoint{3.515845in}{2.597712in}}%
\pgfpathlineto{\pgfqpoint{3.508198in}{2.586714in}}%
\pgfpathclose%
\pgfusepath{fill}%
\end{pgfscope}%
\begin{pgfscope}%
\pgfpathrectangle{\pgfqpoint{1.254980in}{0.150000in}}{\pgfqpoint{5.490039in}{5.490039in}}%
\pgfusepath{clip}%
\pgfsetbuttcap%
\pgfsetroundjoin%
\definecolor{currentfill}{rgb}{0.273006,0.204520,0.501721}%
\pgfsetfillcolor{currentfill}%
\pgfsetfillopacity{0.700000}%
\pgfsetlinewidth{0.000000pt}%
\definecolor{currentstroke}{rgb}{0.000000,0.000000,0.000000}%
\pgfsetstrokecolor{currentstroke}%
\pgfsetdash{}{0pt}%
\pgfpathmoveto{\pgfqpoint{4.072669in}{2.618267in}}%
\pgfpathlineto{\pgfqpoint{4.085632in}{2.614943in}}%
\pgfpathlineto{\pgfqpoint{4.098601in}{2.611799in}}%
\pgfpathlineto{\pgfqpoint{4.111576in}{2.608834in}}%
\pgfpathlineto{\pgfqpoint{4.124558in}{2.606047in}}%
\pgfpathlineto{\pgfqpoint{4.132030in}{2.616747in}}%
\pgfpathlineto{\pgfqpoint{4.139498in}{2.627520in}}%
\pgfpathlineto{\pgfqpoint{4.146962in}{2.638369in}}%
\pgfpathlineto{\pgfqpoint{4.154421in}{2.649296in}}%
\pgfpathlineto{\pgfqpoint{4.141448in}{2.652359in}}%
\pgfpathlineto{\pgfqpoint{4.128482in}{2.655600in}}%
\pgfpathlineto{\pgfqpoint{4.115521in}{2.659020in}}%
\pgfpathlineto{\pgfqpoint{4.102567in}{2.662620in}}%
\pgfpathlineto{\pgfqpoint{4.095099in}{2.651407in}}%
\pgfpathlineto{\pgfqpoint{4.087627in}{2.640279in}}%
\pgfpathlineto{\pgfqpoint{4.080150in}{2.629234in}}%
\pgfpathlineto{\pgfqpoint{4.072669in}{2.618267in}}%
\pgfpathclose%
\pgfusepath{fill}%
\end{pgfscope}%
\begin{pgfscope}%
\pgfpathrectangle{\pgfqpoint{1.254980in}{0.150000in}}{\pgfqpoint{5.490039in}{5.490039in}}%
\pgfusepath{clip}%
\pgfsetbuttcap%
\pgfsetroundjoin%
\definecolor{currentfill}{rgb}{0.220057,0.343307,0.549413}%
\pgfsetfillcolor{currentfill}%
\pgfsetfillopacity{0.700000}%
\pgfsetlinewidth{0.000000pt}%
\definecolor{currentstroke}{rgb}{0.000000,0.000000,0.000000}%
\pgfsetstrokecolor{currentstroke}%
\pgfsetdash{}{0pt}%
\pgfpathmoveto{\pgfqpoint{4.778689in}{2.909233in}}%
\pgfpathlineto{\pgfqpoint{4.791855in}{2.909005in}}%
\pgfpathlineto{\pgfqpoint{4.805031in}{2.908937in}}%
\pgfpathlineto{\pgfqpoint{4.818218in}{2.909028in}}%
\pgfpathlineto{\pgfqpoint{4.831414in}{2.909277in}}%
\pgfpathlineto{\pgfqpoint{4.838678in}{2.919932in}}%
\pgfpathlineto{\pgfqpoint{4.845939in}{2.930741in}}%
\pgfpathlineto{\pgfqpoint{4.853199in}{2.941710in}}%
\pgfpathlineto{\pgfqpoint{4.860458in}{2.952845in}}%
\pgfpathlineto{\pgfqpoint{4.847276in}{2.953093in}}%
\pgfpathlineto{\pgfqpoint{4.834104in}{2.953500in}}%
\pgfpathlineto{\pgfqpoint{4.820942in}{2.954066in}}%
\pgfpathlineto{\pgfqpoint{4.807790in}{2.954792in}}%
\pgfpathlineto{\pgfqpoint{4.800517in}{2.943149in}}%
\pgfpathlineto{\pgfqpoint{4.793243in}{2.931679in}}%
\pgfpathlineto{\pgfqpoint{4.785967in}{2.920376in}}%
\pgfpathlineto{\pgfqpoint{4.778689in}{2.909233in}}%
\pgfpathclose%
\pgfusepath{fill}%
\end{pgfscope}%
\begin{pgfscope}%
\pgfpathrectangle{\pgfqpoint{1.254980in}{0.150000in}}{\pgfqpoint{5.490039in}{5.490039in}}%
\pgfusepath{clip}%
\pgfsetbuttcap%
\pgfsetroundjoin%
\definecolor{currentfill}{rgb}{0.210503,0.363727,0.552206}%
\pgfsetfillcolor{currentfill}%
\pgfsetfillopacity{0.700000}%
\pgfsetlinewidth{0.000000pt}%
\definecolor{currentstroke}{rgb}{0.000000,0.000000,0.000000}%
\pgfsetstrokecolor{currentstroke}%
\pgfsetdash{}{0pt}%
\pgfpathmoveto{\pgfqpoint{4.860458in}{2.952845in}}%
\pgfpathlineto{\pgfqpoint{4.873650in}{2.952754in}}%
\pgfpathlineto{\pgfqpoint{4.886851in}{2.952821in}}%
\pgfpathlineto{\pgfqpoint{4.900064in}{2.953046in}}%
\pgfpathlineto{\pgfqpoint{4.913286in}{2.953428in}}%
\pgfpathlineto{\pgfqpoint{4.920528in}{2.964219in}}%
\pgfpathlineto{\pgfqpoint{4.927768in}{2.975180in}}%
\pgfpathlineto{\pgfqpoint{4.935008in}{2.986318in}}%
\pgfpathlineto{\pgfqpoint{4.942246in}{2.997639in}}%
\pgfpathlineto{\pgfqpoint{4.929039in}{2.997783in}}%
\pgfpathlineto{\pgfqpoint{4.915842in}{2.998084in}}%
\pgfpathlineto{\pgfqpoint{4.902656in}{2.998542in}}%
\pgfpathlineto{\pgfqpoint{4.889479in}{2.999157in}}%
\pgfpathlineto{\pgfqpoint{4.882225in}{2.987301in}}%
\pgfpathlineto{\pgfqpoint{4.874971in}{2.975634in}}%
\pgfpathlineto{\pgfqpoint{4.867715in}{2.964151in}}%
\pgfpathlineto{\pgfqpoint{4.860458in}{2.952845in}}%
\pgfpathclose%
\pgfusepath{fill}%
\end{pgfscope}%
\begin{pgfscope}%
\pgfpathrectangle{\pgfqpoint{1.254980in}{0.150000in}}{\pgfqpoint{5.490039in}{5.490039in}}%
\pgfusepath{clip}%
\pgfsetbuttcap%
\pgfsetroundjoin%
\definecolor{currentfill}{rgb}{0.227802,0.326594,0.546532}%
\pgfsetfillcolor{currentfill}%
\pgfsetfillopacity{0.700000}%
\pgfsetlinewidth{0.000000pt}%
\definecolor{currentstroke}{rgb}{0.000000,0.000000,0.000000}%
\pgfsetstrokecolor{currentstroke}%
\pgfsetdash{}{0pt}%
\pgfpathmoveto{\pgfqpoint{4.696933in}{2.866793in}}%
\pgfpathlineto{\pgfqpoint{4.710075in}{2.866394in}}%
\pgfpathlineto{\pgfqpoint{4.723225in}{2.866155in}}%
\pgfpathlineto{\pgfqpoint{4.736386in}{2.866078in}}%
\pgfpathlineto{\pgfqpoint{4.749556in}{2.866161in}}%
\pgfpathlineto{\pgfqpoint{4.756843in}{2.876715in}}%
\pgfpathlineto{\pgfqpoint{4.764127in}{2.887408in}}%
\pgfpathlineto{\pgfqpoint{4.771409in}{2.898246in}}%
\pgfpathlineto{\pgfqpoint{4.778689in}{2.909233in}}%
\pgfpathlineto{\pgfqpoint{4.765532in}{2.909621in}}%
\pgfpathlineto{\pgfqpoint{4.752385in}{2.910169in}}%
\pgfpathlineto{\pgfqpoint{4.739247in}{2.910877in}}%
\pgfpathlineto{\pgfqpoint{4.726119in}{2.911746in}}%
\pgfpathlineto{\pgfqpoint{4.718826in}{2.900279in}}%
\pgfpathlineto{\pgfqpoint{4.711531in}{2.888968in}}%
\pgfpathlineto{\pgfqpoint{4.704233in}{2.877808in}}%
\pgfpathlineto{\pgfqpoint{4.696933in}{2.866793in}}%
\pgfpathclose%
\pgfusepath{fill}%
\end{pgfscope}%
\begin{pgfscope}%
\pgfpathrectangle{\pgfqpoint{1.254980in}{0.150000in}}{\pgfqpoint{5.490039in}{5.490039in}}%
\pgfusepath{clip}%
\pgfsetbuttcap%
\pgfsetroundjoin%
\definecolor{currentfill}{rgb}{0.279574,0.170599,0.479997}%
\pgfsetfillcolor{currentfill}%
\pgfsetfillopacity{0.700000}%
\pgfsetlinewidth{0.000000pt}%
\definecolor{currentstroke}{rgb}{0.000000,0.000000,0.000000}%
\pgfsetstrokecolor{currentstroke}%
\pgfsetdash{}{0pt}%
\pgfpathmoveto{\pgfqpoint{3.775438in}{2.559427in}}%
\pgfpathlineto{\pgfqpoint{3.788349in}{2.553774in}}%
\pgfpathlineto{\pgfqpoint{3.801264in}{2.548315in}}%
\pgfpathlineto{\pgfqpoint{3.814183in}{2.543049in}}%
\pgfpathlineto{\pgfqpoint{3.827106in}{2.537975in}}%
\pgfpathlineto{\pgfqpoint{3.834667in}{2.548776in}}%
\pgfpathlineto{\pgfqpoint{3.842224in}{2.559643in}}%
\pgfpathlineto{\pgfqpoint{3.849775in}{2.570577in}}%
\pgfpathlineto{\pgfqpoint{3.857322in}{2.581581in}}%
\pgfpathlineto{\pgfqpoint{3.844408in}{2.586848in}}%
\pgfpathlineto{\pgfqpoint{3.831497in}{2.592307in}}%
\pgfpathlineto{\pgfqpoint{3.818591in}{2.597959in}}%
\pgfpathlineto{\pgfqpoint{3.805688in}{2.603804in}}%
\pgfpathlineto{\pgfqpoint{3.798133in}{2.592597in}}%
\pgfpathlineto{\pgfqpoint{3.790573in}{2.581466in}}%
\pgfpathlineto{\pgfqpoint{3.783008in}{2.570410in}}%
\pgfpathlineto{\pgfqpoint{3.775438in}{2.559427in}}%
\pgfpathclose%
\pgfusepath{fill}%
\end{pgfscope}%
\begin{pgfscope}%
\pgfpathrectangle{\pgfqpoint{1.254980in}{0.150000in}}{\pgfqpoint{5.490039in}{5.490039in}}%
\pgfusepath{clip}%
\pgfsetbuttcap%
\pgfsetroundjoin%
\definecolor{currentfill}{rgb}{0.203063,0.379716,0.553925}%
\pgfsetfillcolor{currentfill}%
\pgfsetfillopacity{0.700000}%
\pgfsetlinewidth{0.000000pt}%
\definecolor{currentstroke}{rgb}{0.000000,0.000000,0.000000}%
\pgfsetstrokecolor{currentstroke}%
\pgfsetdash{}{0pt}%
\pgfpathmoveto{\pgfqpoint{4.942246in}{2.997639in}}%
\pgfpathlineto{\pgfqpoint{4.955463in}{2.997651in}}%
\pgfpathlineto{\pgfqpoint{4.968691in}{2.997820in}}%
\pgfpathlineto{\pgfqpoint{4.981929in}{2.998145in}}%
\pgfpathlineto{\pgfqpoint{4.995178in}{2.998626in}}%
\pgfpathlineto{\pgfqpoint{5.002399in}{3.009592in}}%
\pgfpathlineto{\pgfqpoint{5.009620in}{3.020747in}}%
\pgfpathlineto{\pgfqpoint{5.016840in}{3.032097in}}%
\pgfpathlineto{\pgfqpoint{5.024060in}{3.043648in}}%
\pgfpathlineto{\pgfqpoint{5.010828in}{3.043721in}}%
\pgfpathlineto{\pgfqpoint{4.997606in}{3.043950in}}%
\pgfpathlineto{\pgfqpoint{4.984395in}{3.044334in}}%
\pgfpathlineto{\pgfqpoint{4.971194in}{3.044874in}}%
\pgfpathlineto{\pgfqpoint{4.963957in}{3.032760in}}%
\pgfpathlineto{\pgfqpoint{4.956721in}{3.020854in}}%
\pgfpathlineto{\pgfqpoint{4.949484in}{3.009149in}}%
\pgfpathlineto{\pgfqpoint{4.942246in}{2.997639in}}%
\pgfpathclose%
\pgfusepath{fill}%
\end{pgfscope}%
\begin{pgfscope}%
\pgfpathrectangle{\pgfqpoint{1.254980in}{0.150000in}}{\pgfqpoint{5.490039in}{5.490039in}}%
\pgfusepath{clip}%
\pgfsetbuttcap%
\pgfsetroundjoin%
\definecolor{currentfill}{rgb}{0.237441,0.305202,0.541921}%
\pgfsetfillcolor{currentfill}%
\pgfsetfillopacity{0.700000}%
\pgfsetlinewidth{0.000000pt}%
\definecolor{currentstroke}{rgb}{0.000000,0.000000,0.000000}%
\pgfsetstrokecolor{currentstroke}%
\pgfsetdash{}{0pt}%
\pgfpathmoveto{\pgfqpoint{4.615186in}{2.825536in}}%
\pgfpathlineto{\pgfqpoint{4.628303in}{2.824929in}}%
\pgfpathlineto{\pgfqpoint{4.641428in}{2.824486in}}%
\pgfpathlineto{\pgfqpoint{4.654563in}{2.824205in}}%
\pgfpathlineto{\pgfqpoint{4.667708in}{2.824087in}}%
\pgfpathlineto{\pgfqpoint{4.675018in}{2.834571in}}%
\pgfpathlineto{\pgfqpoint{4.682326in}{2.845180in}}%
\pgfpathlineto{\pgfqpoint{4.689631in}{2.855919in}}%
\pgfpathlineto{\pgfqpoint{4.696933in}{2.866793in}}%
\pgfpathlineto{\pgfqpoint{4.683801in}{2.867355in}}%
\pgfpathlineto{\pgfqpoint{4.670679in}{2.868078in}}%
\pgfpathlineto{\pgfqpoint{4.657565in}{2.868963in}}%
\pgfpathlineto{\pgfqpoint{4.644461in}{2.870012in}}%
\pgfpathlineto{\pgfqpoint{4.637146in}{2.858685in}}%
\pgfpathlineto{\pgfqpoint{4.629829in}{2.847501in}}%
\pgfpathlineto{\pgfqpoint{4.622509in}{2.836452in}}%
\pgfpathlineto{\pgfqpoint{4.615186in}{2.825536in}}%
\pgfpathclose%
\pgfusepath{fill}%
\end{pgfscope}%
\begin{pgfscope}%
\pgfpathrectangle{\pgfqpoint{1.254980in}{0.150000in}}{\pgfqpoint{5.490039in}{5.490039in}}%
\pgfusepath{clip}%
\pgfsetbuttcap%
\pgfsetroundjoin%
\definecolor{currentfill}{rgb}{0.274128,0.199721,0.498911}%
\pgfsetfillcolor{currentfill}%
\pgfsetfillopacity{0.700000}%
\pgfsetlinewidth{0.000000pt}%
\definecolor{currentstroke}{rgb}{0.000000,0.000000,0.000000}%
\pgfsetstrokecolor{currentstroke}%
\pgfsetdash{}{0pt}%
\pgfpathmoveto{\pgfqpoint{3.374338in}{2.619272in}}%
\pgfpathlineto{\pgfqpoint{3.387240in}{2.609028in}}%
\pgfpathlineto{\pgfqpoint{3.400143in}{2.599008in}}%
\pgfpathlineto{\pgfqpoint{3.413045in}{2.589211in}}%
\pgfpathlineto{\pgfqpoint{3.425946in}{2.579635in}}%
\pgfpathlineto{\pgfqpoint{3.433626in}{2.590420in}}%
\pgfpathlineto{\pgfqpoint{3.441300in}{2.601288in}}%
\pgfpathlineto{\pgfqpoint{3.448968in}{2.612239in}}%
\pgfpathlineto{\pgfqpoint{3.456630in}{2.623277in}}%
\pgfpathlineto{\pgfqpoint{3.443739in}{2.632963in}}%
\pgfpathlineto{\pgfqpoint{3.430848in}{2.642870in}}%
\pgfpathlineto{\pgfqpoint{3.417956in}{2.653000in}}%
\pgfpathlineto{\pgfqpoint{3.405064in}{2.663354in}}%
\pgfpathlineto{\pgfqpoint{3.397391in}{2.652196in}}%
\pgfpathlineto{\pgfqpoint{3.389712in}{2.641131in}}%
\pgfpathlineto{\pgfqpoint{3.382028in}{2.630157in}}%
\pgfpathlineto{\pgfqpoint{3.374338in}{2.619272in}}%
\pgfpathclose%
\pgfusepath{fill}%
\end{pgfscope}%
\begin{pgfscope}%
\pgfpathrectangle{\pgfqpoint{1.254980in}{0.150000in}}{\pgfqpoint{5.490039in}{5.490039in}}%
\pgfusepath{clip}%
\pgfsetbuttcap%
\pgfsetroundjoin%
\definecolor{currentfill}{rgb}{0.194100,0.399323,0.555565}%
\pgfsetfillcolor{currentfill}%
\pgfsetfillopacity{0.700000}%
\pgfsetlinewidth{0.000000pt}%
\definecolor{currentstroke}{rgb}{0.000000,0.000000,0.000000}%
\pgfsetstrokecolor{currentstroke}%
\pgfsetdash{}{0pt}%
\pgfpathmoveto{\pgfqpoint{5.024060in}{3.043648in}}%
\pgfpathlineto{\pgfqpoint{5.037303in}{3.043730in}}%
\pgfpathlineto{\pgfqpoint{5.050556in}{3.043966in}}%
\pgfpathlineto{\pgfqpoint{5.063820in}{3.044358in}}%
\pgfpathlineto{\pgfqpoint{5.077095in}{3.044903in}}%
\pgfpathlineto{\pgfqpoint{5.084298in}{3.056091in}}%
\pgfpathlineto{\pgfqpoint{5.091500in}{3.067486in}}%
\pgfpathlineto{\pgfqpoint{5.098703in}{3.079095in}}%
\pgfpathlineto{\pgfqpoint{5.105906in}{3.090926in}}%
\pgfpathlineto{\pgfqpoint{5.092649in}{3.090962in}}%
\pgfpathlineto{\pgfqpoint{5.079402in}{3.091152in}}%
\pgfpathlineto{\pgfqpoint{5.066166in}{3.091496in}}%
\pgfpathlineto{\pgfqpoint{5.052941in}{3.091995in}}%
\pgfpathlineto{\pgfqpoint{5.045720in}{3.079573in}}%
\pgfpathlineto{\pgfqpoint{5.038500in}{3.067380in}}%
\pgfpathlineto{\pgfqpoint{5.031280in}{3.055407in}}%
\pgfpathlineto{\pgfqpoint{5.024060in}{3.043648in}}%
\pgfpathclose%
\pgfusepath{fill}%
\end{pgfscope}%
\begin{pgfscope}%
\pgfpathrectangle{\pgfqpoint{1.254980in}{0.150000in}}{\pgfqpoint{5.490039in}{5.490039in}}%
\pgfusepath{clip}%
\pgfsetbuttcap%
\pgfsetroundjoin%
\definecolor{currentfill}{rgb}{0.244972,0.287675,0.537260}%
\pgfsetfillcolor{currentfill}%
\pgfsetfillopacity{0.700000}%
\pgfsetlinewidth{0.000000pt}%
\definecolor{currentstroke}{rgb}{0.000000,0.000000,0.000000}%
\pgfsetstrokecolor{currentstroke}%
\pgfsetdash{}{0pt}%
\pgfpathmoveto{\pgfqpoint{4.533441in}{2.785491in}}%
\pgfpathlineto{\pgfqpoint{4.546534in}{2.784642in}}%
\pgfpathlineto{\pgfqpoint{4.559635in}{2.783959in}}%
\pgfpathlineto{\pgfqpoint{4.572745in}{2.783440in}}%
\pgfpathlineto{\pgfqpoint{4.585865in}{2.783085in}}%
\pgfpathlineto{\pgfqpoint{4.593200in}{2.793525in}}%
\pgfpathlineto{\pgfqpoint{4.600532in}{2.804077in}}%
\pgfpathlineto{\pgfqpoint{4.607860in}{2.814745in}}%
\pgfpathlineto{\pgfqpoint{4.615186in}{2.825536in}}%
\pgfpathlineto{\pgfqpoint{4.602078in}{2.826306in}}%
\pgfpathlineto{\pgfqpoint{4.588980in}{2.827239in}}%
\pgfpathlineto{\pgfqpoint{4.575890in}{2.828337in}}%
\pgfpathlineto{\pgfqpoint{4.562808in}{2.829600in}}%
\pgfpathlineto{\pgfqpoint{4.555471in}{2.818385in}}%
\pgfpathlineto{\pgfqpoint{4.548131in}{2.807299in}}%
\pgfpathlineto{\pgfqpoint{4.540788in}{2.796336in}}%
\pgfpathlineto{\pgfqpoint{4.533441in}{2.785491in}}%
\pgfpathclose%
\pgfusepath{fill}%
\end{pgfscope}%
\begin{pgfscope}%
\pgfpathrectangle{\pgfqpoint{1.254980in}{0.150000in}}{\pgfqpoint{5.490039in}{5.490039in}}%
\pgfusepath{clip}%
\pgfsetbuttcap%
\pgfsetroundjoin%
\definecolor{currentfill}{rgb}{0.276194,0.190074,0.493001}%
\pgfsetfillcolor{currentfill}%
\pgfsetfillopacity{0.700000}%
\pgfsetlinewidth{0.000000pt}%
\definecolor{currentstroke}{rgb}{0.000000,0.000000,0.000000}%
\pgfsetstrokecolor{currentstroke}%
\pgfsetdash{}{0pt}%
\pgfpathmoveto{\pgfqpoint{3.990873in}{2.589244in}}%
\pgfpathlineto{\pgfqpoint{4.003822in}{2.585442in}}%
\pgfpathlineto{\pgfqpoint{4.016776in}{2.581822in}}%
\pgfpathlineto{\pgfqpoint{4.029736in}{2.578384in}}%
\pgfpathlineto{\pgfqpoint{4.042701in}{2.575128in}}%
\pgfpathlineto{\pgfqpoint{4.050200in}{2.585810in}}%
\pgfpathlineto{\pgfqpoint{4.057694in}{2.596559in}}%
\pgfpathlineto{\pgfqpoint{4.065184in}{2.607377in}}%
\pgfpathlineto{\pgfqpoint{4.072669in}{2.618267in}}%
\pgfpathlineto{\pgfqpoint{4.059712in}{2.621772in}}%
\pgfpathlineto{\pgfqpoint{4.046761in}{2.625458in}}%
\pgfpathlineto{\pgfqpoint{4.033815in}{2.629326in}}%
\pgfpathlineto{\pgfqpoint{4.020875in}{2.633377in}}%
\pgfpathlineto{\pgfqpoint{4.013381in}{2.622228in}}%
\pgfpathlineto{\pgfqpoint{4.005883in}{2.611158in}}%
\pgfpathlineto{\pgfqpoint{3.998380in}{2.600165in}}%
\pgfpathlineto{\pgfqpoint{3.990873in}{2.589244in}}%
\pgfpathclose%
\pgfusepath{fill}%
\end{pgfscope}%
\begin{pgfscope}%
\pgfpathrectangle{\pgfqpoint{1.254980in}{0.150000in}}{\pgfqpoint{5.490039in}{5.490039in}}%
\pgfusepath{clip}%
\pgfsetbuttcap%
\pgfsetroundjoin%
\definecolor{currentfill}{rgb}{0.183898,0.422383,0.556944}%
\pgfsetfillcolor{currentfill}%
\pgfsetfillopacity{0.700000}%
\pgfsetlinewidth{0.000000pt}%
\definecolor{currentstroke}{rgb}{0.000000,0.000000,0.000000}%
\pgfsetstrokecolor{currentstroke}%
\pgfsetdash{}{0pt}%
\pgfpathmoveto{\pgfqpoint{5.105906in}{3.090926in}}%
\pgfpathlineto{\pgfqpoint{5.119174in}{3.091043in}}%
\pgfpathlineto{\pgfqpoint{5.132453in}{3.091314in}}%
\pgfpathlineto{\pgfqpoint{5.145743in}{3.091739in}}%
\pgfpathlineto{\pgfqpoint{5.159044in}{3.092316in}}%
\pgfpathlineto{\pgfqpoint{5.166229in}{3.103775in}}%
\pgfpathlineto{\pgfqpoint{5.173415in}{3.115462in}}%
\pgfpathlineto{\pgfqpoint{5.180603in}{3.127383in}}%
\pgfpathlineto{\pgfqpoint{5.187792in}{3.139547in}}%
\pgfpathlineto{\pgfqpoint{5.174510in}{3.139579in}}%
\pgfpathlineto{\pgfqpoint{5.161239in}{3.139764in}}%
\pgfpathlineto{\pgfqpoint{5.147979in}{3.140101in}}%
\pgfpathlineto{\pgfqpoint{5.134729in}{3.140592in}}%
\pgfpathlineto{\pgfqpoint{5.127521in}{3.127810in}}%
\pgfpathlineto{\pgfqpoint{5.120315in}{3.115276in}}%
\pgfpathlineto{\pgfqpoint{5.113110in}{3.102984in}}%
\pgfpathlineto{\pgfqpoint{5.105906in}{3.090926in}}%
\pgfpathclose%
\pgfusepath{fill}%
\end{pgfscope}%
\begin{pgfscope}%
\pgfpathrectangle{\pgfqpoint{1.254980in}{0.150000in}}{\pgfqpoint{5.490039in}{5.490039in}}%
\pgfusepath{clip}%
\pgfsetbuttcap%
\pgfsetroundjoin%
\definecolor{currentfill}{rgb}{0.241237,0.296485,0.539709}%
\pgfsetfillcolor{currentfill}%
\pgfsetfillopacity{0.700000}%
\pgfsetlinewidth{0.000000pt}%
\definecolor{currentstroke}{rgb}{0.000000,0.000000,0.000000}%
\pgfsetstrokecolor{currentstroke}%
\pgfsetdash{}{0pt}%
\pgfpathmoveto{\pgfqpoint{3.084780in}{2.829589in}}%
\pgfpathlineto{\pgfqpoint{3.097751in}{2.814568in}}%
\pgfpathlineto{\pgfqpoint{3.110717in}{2.799809in}}%
\pgfpathlineto{\pgfqpoint{3.123679in}{2.785310in}}%
\pgfpathlineto{\pgfqpoint{3.136636in}{2.771069in}}%
\pgfpathlineto{\pgfqpoint{3.144399in}{2.781897in}}%
\pgfpathlineto{\pgfqpoint{3.152155in}{2.792837in}}%
\pgfpathlineto{\pgfqpoint{3.159904in}{2.803890in}}%
\pgfpathlineto{\pgfqpoint{3.167647in}{2.815057in}}%
\pgfpathlineto{\pgfqpoint{3.154703in}{2.829381in}}%
\pgfpathlineto{\pgfqpoint{3.141755in}{2.843962in}}%
\pgfpathlineto{\pgfqpoint{3.128802in}{2.858802in}}%
\pgfpathlineto{\pgfqpoint{3.115844in}{2.873905in}}%
\pgfpathlineto{\pgfqpoint{3.108088in}{2.862646in}}%
\pgfpathlineto{\pgfqpoint{3.100326in}{2.851507in}}%
\pgfpathlineto{\pgfqpoint{3.092556in}{2.840489in}}%
\pgfpathlineto{\pgfqpoint{3.084780in}{2.829589in}}%
\pgfpathclose%
\pgfusepath{fill}%
\end{pgfscope}%
\begin{pgfscope}%
\pgfpathrectangle{\pgfqpoint{1.254980in}{0.150000in}}{\pgfqpoint{5.490039in}{5.490039in}}%
\pgfusepath{clip}%
\pgfsetbuttcap%
\pgfsetroundjoin%
\definecolor{currentfill}{rgb}{0.252194,0.269783,0.531579}%
\pgfsetfillcolor{currentfill}%
\pgfsetfillopacity{0.700000}%
\pgfsetlinewidth{0.000000pt}%
\definecolor{currentstroke}{rgb}{0.000000,0.000000,0.000000}%
\pgfsetstrokecolor{currentstroke}%
\pgfsetdash{}{0pt}%
\pgfpathmoveto{\pgfqpoint{4.451694in}{2.746712in}}%
\pgfpathlineto{\pgfqpoint{4.464763in}{2.745585in}}%
\pgfpathlineto{\pgfqpoint{4.477841in}{2.744626in}}%
\pgfpathlineto{\pgfqpoint{4.490927in}{2.743833in}}%
\pgfpathlineto{\pgfqpoint{4.504022in}{2.743206in}}%
\pgfpathlineto{\pgfqpoint{4.511382in}{2.753622in}}%
\pgfpathlineto{\pgfqpoint{4.518738in}{2.764139in}}%
\pgfpathlineto{\pgfqpoint{4.526092in}{2.774760in}}%
\pgfpathlineto{\pgfqpoint{4.533441in}{2.785491in}}%
\pgfpathlineto{\pgfqpoint{4.520357in}{2.786505in}}%
\pgfpathlineto{\pgfqpoint{4.507282in}{2.787685in}}%
\pgfpathlineto{\pgfqpoint{4.494215in}{2.789032in}}%
\pgfpathlineto{\pgfqpoint{4.481156in}{2.790545in}}%
\pgfpathlineto{\pgfqpoint{4.473796in}{2.779417in}}%
\pgfpathlineto{\pgfqpoint{4.466432in}{2.768406in}}%
\pgfpathlineto{\pgfqpoint{4.459065in}{2.757505in}}%
\pgfpathlineto{\pgfqpoint{4.451694in}{2.746712in}}%
\pgfpathclose%
\pgfusepath{fill}%
\end{pgfscope}%
\begin{pgfscope}%
\pgfpathrectangle{\pgfqpoint{1.254980in}{0.150000in}}{\pgfqpoint{5.490039in}{5.490039in}}%
\pgfusepath{clip}%
\pgfsetbuttcap%
\pgfsetroundjoin%
\definecolor{currentfill}{rgb}{0.229739,0.322361,0.545706}%
\pgfsetfillcolor{currentfill}%
\pgfsetfillopacity{0.700000}%
\pgfsetlinewidth{0.000000pt}%
\definecolor{currentstroke}{rgb}{0.000000,0.000000,0.000000}%
\pgfsetstrokecolor{currentstroke}%
\pgfsetdash{}{0pt}%
\pgfpathmoveto{\pgfqpoint{3.032841in}{2.892339in}}%
\pgfpathlineto{\pgfqpoint{3.045835in}{2.876247in}}%
\pgfpathlineto{\pgfqpoint{3.058822in}{2.860426in}}%
\pgfpathlineto{\pgfqpoint{3.071804in}{2.844874in}}%
\pgfpathlineto{\pgfqpoint{3.084780in}{2.829589in}}%
\pgfpathlineto{\pgfqpoint{3.092556in}{2.840489in}}%
\pgfpathlineto{\pgfqpoint{3.100326in}{2.851507in}}%
\pgfpathlineto{\pgfqpoint{3.108088in}{2.862646in}}%
\pgfpathlineto{\pgfqpoint{3.115844in}{2.873905in}}%
\pgfpathlineto{\pgfqpoint{3.102881in}{2.889272in}}%
\pgfpathlineto{\pgfqpoint{3.089913in}{2.904906in}}%
\pgfpathlineto{\pgfqpoint{3.076939in}{2.920809in}}%
\pgfpathlineto{\pgfqpoint{3.063959in}{2.936983in}}%
\pgfpathlineto{\pgfqpoint{3.056191in}{2.925631in}}%
\pgfpathlineto{\pgfqpoint{3.048415in}{2.914407in}}%
\pgfpathlineto{\pgfqpoint{3.040632in}{2.903310in}}%
\pgfpathlineto{\pgfqpoint{3.032841in}{2.892339in}}%
\pgfpathclose%
\pgfusepath{fill}%
\end{pgfscope}%
\begin{pgfscope}%
\pgfpathrectangle{\pgfqpoint{1.254980in}{0.150000in}}{\pgfqpoint{5.490039in}{5.490039in}}%
\pgfusepath{clip}%
\pgfsetbuttcap%
\pgfsetroundjoin%
\definecolor{currentfill}{rgb}{0.252194,0.269783,0.531579}%
\pgfsetfillcolor{currentfill}%
\pgfsetfillopacity{0.700000}%
\pgfsetlinewidth{0.000000pt}%
\definecolor{currentstroke}{rgb}{0.000000,0.000000,0.000000}%
\pgfsetstrokecolor{currentstroke}%
\pgfsetdash{}{0pt}%
\pgfpathmoveto{\pgfqpoint{3.136636in}{2.771069in}}%
\pgfpathlineto{\pgfqpoint{3.149588in}{2.757083in}}%
\pgfpathlineto{\pgfqpoint{3.162537in}{2.743350in}}%
\pgfpathlineto{\pgfqpoint{3.175482in}{2.729869in}}%
\pgfpathlineto{\pgfqpoint{3.188423in}{2.716637in}}%
\pgfpathlineto{\pgfqpoint{3.196173in}{2.727394in}}%
\pgfpathlineto{\pgfqpoint{3.203917in}{2.738255in}}%
\pgfpathlineto{\pgfqpoint{3.211654in}{2.749222in}}%
\pgfpathlineto{\pgfqpoint{3.219384in}{2.760297in}}%
\pgfpathlineto{\pgfqpoint{3.206456in}{2.773611in}}%
\pgfpathlineto{\pgfqpoint{3.193523in}{2.787175in}}%
\pgfpathlineto{\pgfqpoint{3.180587in}{2.800989in}}%
\pgfpathlineto{\pgfqpoint{3.167647in}{2.815057in}}%
\pgfpathlineto{\pgfqpoint{3.159904in}{2.803890in}}%
\pgfpathlineto{\pgfqpoint{3.152155in}{2.792837in}}%
\pgfpathlineto{\pgfqpoint{3.144399in}{2.781897in}}%
\pgfpathlineto{\pgfqpoint{3.136636in}{2.771069in}}%
\pgfpathclose%
\pgfusepath{fill}%
\end{pgfscope}%
\begin{pgfscope}%
\pgfpathrectangle{\pgfqpoint{1.254980in}{0.150000in}}{\pgfqpoint{5.490039in}{5.490039in}}%
\pgfusepath{clip}%
\pgfsetbuttcap%
\pgfsetroundjoin%
\definecolor{currentfill}{rgb}{0.175841,0.441290,0.557685}%
\pgfsetfillcolor{currentfill}%
\pgfsetfillopacity{0.700000}%
\pgfsetlinewidth{0.000000pt}%
\definecolor{currentstroke}{rgb}{0.000000,0.000000,0.000000}%
\pgfsetstrokecolor{currentstroke}%
\pgfsetdash{}{0pt}%
\pgfpathmoveto{\pgfqpoint{5.187792in}{3.139547in}}%
\pgfpathlineto{\pgfqpoint{5.201085in}{3.139667in}}%
\pgfpathlineto{\pgfqpoint{5.214389in}{3.139939in}}%
\pgfpathlineto{\pgfqpoint{5.227704in}{3.140364in}}%
\pgfpathlineto{\pgfqpoint{5.241031in}{3.140940in}}%
\pgfpathlineto{\pgfqpoint{5.248202in}{3.152725in}}%
\pgfpathlineto{\pgfqpoint{5.255374in}{3.164760in}}%
\pgfpathlineto{\pgfqpoint{5.262549in}{3.177052in}}%
\pgfpathlineto{\pgfqpoint{5.269726in}{3.189608in}}%
\pgfpathlineto{\pgfqpoint{5.256420in}{3.189669in}}%
\pgfpathlineto{\pgfqpoint{5.243125in}{3.189881in}}%
\pgfpathlineto{\pgfqpoint{5.229841in}{3.190245in}}%
\pgfpathlineto{\pgfqpoint{5.216568in}{3.190761in}}%
\pgfpathlineto{\pgfqpoint{5.209370in}{3.177559in}}%
\pgfpathlineto{\pgfqpoint{5.202175in}{3.164627in}}%
\pgfpathlineto{\pgfqpoint{5.194983in}{3.151959in}}%
\pgfpathlineto{\pgfqpoint{5.187792in}{3.139547in}}%
\pgfpathclose%
\pgfusepath{fill}%
\end{pgfscope}%
\begin{pgfscope}%
\pgfpathrectangle{\pgfqpoint{1.254980in}{0.150000in}}{\pgfqpoint{5.490039in}{5.490039in}}%
\pgfusepath{clip}%
\pgfsetbuttcap%
\pgfsetroundjoin%
\definecolor{currentfill}{rgb}{0.216210,0.351535,0.550627}%
\pgfsetfillcolor{currentfill}%
\pgfsetfillopacity{0.700000}%
\pgfsetlinewidth{0.000000pt}%
\definecolor{currentstroke}{rgb}{0.000000,0.000000,0.000000}%
\pgfsetstrokecolor{currentstroke}%
\pgfsetdash{}{0pt}%
\pgfpathmoveto{\pgfqpoint{2.980804in}{2.959472in}}%
\pgfpathlineto{\pgfqpoint{2.993823in}{2.942269in}}%
\pgfpathlineto{\pgfqpoint{3.006836in}{2.925348in}}%
\pgfpathlineto{\pgfqpoint{3.019842in}{2.908705in}}%
\pgfpathlineto{\pgfqpoint{3.032841in}{2.892339in}}%
\pgfpathlineto{\pgfqpoint{3.040632in}{2.903310in}}%
\pgfpathlineto{\pgfqpoint{3.048415in}{2.914407in}}%
\pgfpathlineto{\pgfqpoint{3.056191in}{2.925631in}}%
\pgfpathlineto{\pgfqpoint{3.063959in}{2.936983in}}%
\pgfpathlineto{\pgfqpoint{3.050974in}{2.953431in}}%
\pgfpathlineto{\pgfqpoint{3.037982in}{2.970155in}}%
\pgfpathlineto{\pgfqpoint{3.024983in}{2.987158in}}%
\pgfpathlineto{\pgfqpoint{3.011978in}{3.004443in}}%
\pgfpathlineto{\pgfqpoint{3.004195in}{2.992999in}}%
\pgfpathlineto{\pgfqpoint{2.996406in}{2.981690in}}%
\pgfpathlineto{\pgfqpoint{2.988609in}{2.970514in}}%
\pgfpathlineto{\pgfqpoint{2.980804in}{2.959472in}}%
\pgfpathclose%
\pgfusepath{fill}%
\end{pgfscope}%
\begin{pgfscope}%
\pgfpathrectangle{\pgfqpoint{1.254980in}{0.150000in}}{\pgfqpoint{5.490039in}{5.490039in}}%
\pgfusepath{clip}%
\pgfsetbuttcap%
\pgfsetroundjoin%
\definecolor{currentfill}{rgb}{0.258965,0.251537,0.524736}%
\pgfsetfillcolor{currentfill}%
\pgfsetfillopacity{0.700000}%
\pgfsetlinewidth{0.000000pt}%
\definecolor{currentstroke}{rgb}{0.000000,0.000000,0.000000}%
\pgfsetstrokecolor{currentstroke}%
\pgfsetdash{}{0pt}%
\pgfpathmoveto{\pgfqpoint{4.369938in}{2.709273in}}%
\pgfpathlineto{\pgfqpoint{4.382985in}{2.707831in}}%
\pgfpathlineto{\pgfqpoint{4.396040in}{2.706560in}}%
\pgfpathlineto{\pgfqpoint{4.409103in}{2.705457in}}%
\pgfpathlineto{\pgfqpoint{4.422174in}{2.704522in}}%
\pgfpathlineto{\pgfqpoint{4.429560in}{2.714931in}}%
\pgfpathlineto{\pgfqpoint{4.436942in}{2.725429in}}%
\pgfpathlineto{\pgfqpoint{4.444320in}{2.736021in}}%
\pgfpathlineto{\pgfqpoint{4.451694in}{2.746712in}}%
\pgfpathlineto{\pgfqpoint{4.438633in}{2.748006in}}%
\pgfpathlineto{\pgfqpoint{4.425580in}{2.749469in}}%
\pgfpathlineto{\pgfqpoint{4.412535in}{2.751100in}}%
\pgfpathlineto{\pgfqpoint{4.399498in}{2.752900in}}%
\pgfpathlineto{\pgfqpoint{4.392114in}{2.741840in}}%
\pgfpathlineto{\pgfqpoint{4.384726in}{2.730885in}}%
\pgfpathlineto{\pgfqpoint{4.377334in}{2.720031in}}%
\pgfpathlineto{\pgfqpoint{4.369938in}{2.709273in}}%
\pgfpathclose%
\pgfusepath{fill}%
\end{pgfscope}%
\begin{pgfscope}%
\pgfpathrectangle{\pgfqpoint{1.254980in}{0.150000in}}{\pgfqpoint{5.490039in}{5.490039in}}%
\pgfusepath{clip}%
\pgfsetbuttcap%
\pgfsetroundjoin%
\definecolor{currentfill}{rgb}{0.279574,0.170599,0.479997}%
\pgfsetfillcolor{currentfill}%
\pgfsetfillopacity{0.700000}%
\pgfsetlinewidth{0.000000pt}%
\definecolor{currentstroke}{rgb}{0.000000,0.000000,0.000000}%
\pgfsetstrokecolor{currentstroke}%
\pgfsetdash{}{0pt}%
\pgfpathmoveto{\pgfqpoint{3.559778in}{2.553575in}}%
\pgfpathlineto{\pgfqpoint{3.572676in}{2.545815in}}%
\pgfpathlineto{\pgfqpoint{3.585576in}{2.538263in}}%
\pgfpathlineto{\pgfqpoint{3.598477in}{2.530917in}}%
\pgfpathlineto{\pgfqpoint{3.611381in}{2.523775in}}%
\pgfpathlineto{\pgfqpoint{3.619009in}{2.534517in}}%
\pgfpathlineto{\pgfqpoint{3.626632in}{2.545328in}}%
\pgfpathlineto{\pgfqpoint{3.634249in}{2.556210in}}%
\pgfpathlineto{\pgfqpoint{3.641862in}{2.567165in}}%
\pgfpathlineto{\pgfqpoint{3.628967in}{2.574444in}}%
\pgfpathlineto{\pgfqpoint{3.616075in}{2.581929in}}%
\pgfpathlineto{\pgfqpoint{3.603184in}{2.589619in}}%
\pgfpathlineto{\pgfqpoint{3.590296in}{2.597517in}}%
\pgfpathlineto{\pgfqpoint{3.582674in}{2.586414in}}%
\pgfpathlineto{\pgfqpoint{3.575047in}{2.575391in}}%
\pgfpathlineto{\pgfqpoint{3.567415in}{2.564445in}}%
\pgfpathlineto{\pgfqpoint{3.559778in}{2.553575in}}%
\pgfpathclose%
\pgfusepath{fill}%
\end{pgfscope}%
\begin{pgfscope}%
\pgfpathrectangle{\pgfqpoint{1.254980in}{0.150000in}}{\pgfqpoint{5.490039in}{5.490039in}}%
\pgfusepath{clip}%
\pgfsetbuttcap%
\pgfsetroundjoin%
\definecolor{currentfill}{rgb}{0.260571,0.246922,0.522828}%
\pgfsetfillcolor{currentfill}%
\pgfsetfillopacity{0.700000}%
\pgfsetlinewidth{0.000000pt}%
\definecolor{currentstroke}{rgb}{0.000000,0.000000,0.000000}%
\pgfsetstrokecolor{currentstroke}%
\pgfsetdash{}{0pt}%
\pgfpathmoveto{\pgfqpoint{3.188423in}{2.716637in}}%
\pgfpathlineto{\pgfqpoint{3.201361in}{2.703652in}}%
\pgfpathlineto{\pgfqpoint{3.214296in}{2.690912in}}%
\pgfpathlineto{\pgfqpoint{3.227228in}{2.678416in}}%
\pgfpathlineto{\pgfqpoint{3.240157in}{2.666161in}}%
\pgfpathlineto{\pgfqpoint{3.247894in}{2.676847in}}%
\pgfpathlineto{\pgfqpoint{3.255626in}{2.687629in}}%
\pgfpathlineto{\pgfqpoint{3.263351in}{2.698511in}}%
\pgfpathlineto{\pgfqpoint{3.271069in}{2.709493in}}%
\pgfpathlineto{\pgfqpoint{3.258152in}{2.721830in}}%
\pgfpathlineto{\pgfqpoint{3.245233in}{2.734409in}}%
\pgfpathlineto{\pgfqpoint{3.232310in}{2.747230in}}%
\pgfpathlineto{\pgfqpoint{3.219384in}{2.760297in}}%
\pgfpathlineto{\pgfqpoint{3.211654in}{2.749222in}}%
\pgfpathlineto{\pgfqpoint{3.203917in}{2.738255in}}%
\pgfpathlineto{\pgfqpoint{3.196173in}{2.727394in}}%
\pgfpathlineto{\pgfqpoint{3.188423in}{2.716637in}}%
\pgfpathclose%
\pgfusepath{fill}%
\end{pgfscope}%
\begin{pgfscope}%
\pgfpathrectangle{\pgfqpoint{1.254980in}{0.150000in}}{\pgfqpoint{5.490039in}{5.490039in}}%
\pgfusepath{clip}%
\pgfsetbuttcap%
\pgfsetroundjoin%
\definecolor{currentfill}{rgb}{0.278012,0.180367,0.486697}%
\pgfsetfillcolor{currentfill}%
\pgfsetfillopacity{0.700000}%
\pgfsetlinewidth{0.000000pt}%
\definecolor{currentstroke}{rgb}{0.000000,0.000000,0.000000}%
\pgfsetstrokecolor{currentstroke}%
\pgfsetdash{}{0pt}%
\pgfpathmoveto{\pgfqpoint{3.909023in}{2.562411in}}%
\pgfpathlineto{\pgfqpoint{3.921959in}{2.558089in}}%
\pgfpathlineto{\pgfqpoint{3.934901in}{2.553954in}}%
\pgfpathlineto{\pgfqpoint{3.947847in}{2.550003in}}%
\pgfpathlineto{\pgfqpoint{3.960799in}{2.546238in}}%
\pgfpathlineto{\pgfqpoint{3.968325in}{2.556894in}}%
\pgfpathlineto{\pgfqpoint{3.975845in}{2.567612in}}%
\pgfpathlineto{\pgfqpoint{3.983362in}{2.578394in}}%
\pgfpathlineto{\pgfqpoint{3.990873in}{2.589244in}}%
\pgfpathlineto{\pgfqpoint{3.977930in}{2.593231in}}%
\pgfpathlineto{\pgfqpoint{3.964992in}{2.597402in}}%
\pgfpathlineto{\pgfqpoint{3.952059in}{2.601759in}}%
\pgfpathlineto{\pgfqpoint{3.939130in}{2.606301in}}%
\pgfpathlineto{\pgfqpoint{3.931610in}{2.595220in}}%
\pgfpathlineto{\pgfqpoint{3.924086in}{2.584214in}}%
\pgfpathlineto{\pgfqpoint{3.916557in}{2.573278in}}%
\pgfpathlineto{\pgfqpoint{3.909023in}{2.562411in}}%
\pgfpathclose%
\pgfusepath{fill}%
\end{pgfscope}%
\begin{pgfscope}%
\pgfpathrectangle{\pgfqpoint{1.254980in}{0.150000in}}{\pgfqpoint{5.490039in}{5.490039in}}%
\pgfusepath{clip}%
\pgfsetbuttcap%
\pgfsetroundjoin%
\definecolor{currentfill}{rgb}{0.263663,0.237631,0.518762}%
\pgfsetfillcolor{currentfill}%
\pgfsetfillopacity{0.700000}%
\pgfsetlinewidth{0.000000pt}%
\definecolor{currentstroke}{rgb}{0.000000,0.000000,0.000000}%
\pgfsetstrokecolor{currentstroke}%
\pgfsetdash{}{0pt}%
\pgfpathmoveto{\pgfqpoint{4.288168in}{2.673267in}}%
\pgfpathlineto{\pgfqpoint{4.301194in}{2.671475in}}%
\pgfpathlineto{\pgfqpoint{4.314227in}{2.669854in}}%
\pgfpathlineto{\pgfqpoint{4.327268in}{2.668404in}}%
\pgfpathlineto{\pgfqpoint{4.340317in}{2.667125in}}%
\pgfpathlineto{\pgfqpoint{4.347728in}{2.677537in}}%
\pgfpathlineto{\pgfqpoint{4.355136in}{2.688030in}}%
\pgfpathlineto{\pgfqpoint{4.362539in}{2.698607in}}%
\pgfpathlineto{\pgfqpoint{4.369938in}{2.709273in}}%
\pgfpathlineto{\pgfqpoint{4.356899in}{2.710884in}}%
\pgfpathlineto{\pgfqpoint{4.343868in}{2.712665in}}%
\pgfpathlineto{\pgfqpoint{4.330844in}{2.714618in}}%
\pgfpathlineto{\pgfqpoint{4.317828in}{2.716742in}}%
\pgfpathlineto{\pgfqpoint{4.310419in}{2.705735in}}%
\pgfpathlineto{\pgfqpoint{4.303006in}{2.694822in}}%
\pgfpathlineto{\pgfqpoint{4.295589in}{2.684001in}}%
\pgfpathlineto{\pgfqpoint{4.288168in}{2.673267in}}%
\pgfpathclose%
\pgfusepath{fill}%
\end{pgfscope}%
\begin{pgfscope}%
\pgfpathrectangle{\pgfqpoint{1.254980in}{0.150000in}}{\pgfqpoint{5.490039in}{5.490039in}}%
\pgfusepath{clip}%
\pgfsetbuttcap%
\pgfsetroundjoin%
\definecolor{currentfill}{rgb}{0.280255,0.165693,0.476498}%
\pgfsetfillcolor{currentfill}%
\pgfsetfillopacity{0.700000}%
\pgfsetlinewidth{0.000000pt}%
\definecolor{currentstroke}{rgb}{0.000000,0.000000,0.000000}%
\pgfsetstrokecolor{currentstroke}%
\pgfsetdash{}{0pt}%
\pgfpathmoveto{\pgfqpoint{3.693463in}{2.540072in}}%
\pgfpathlineto{\pgfqpoint{3.706370in}{2.533800in}}%
\pgfpathlineto{\pgfqpoint{3.719280in}{2.527725in}}%
\pgfpathlineto{\pgfqpoint{3.732193in}{2.521848in}}%
\pgfpathlineto{\pgfqpoint{3.745109in}{2.516166in}}%
\pgfpathlineto{\pgfqpoint{3.752699in}{2.526884in}}%
\pgfpathlineto{\pgfqpoint{3.760283in}{2.537666in}}%
\pgfpathlineto{\pgfqpoint{3.767863in}{2.548512in}}%
\pgfpathlineto{\pgfqpoint{3.775438in}{2.559427in}}%
\pgfpathlineto{\pgfqpoint{3.762530in}{2.565274in}}%
\pgfpathlineto{\pgfqpoint{3.749625in}{2.571317in}}%
\pgfpathlineto{\pgfqpoint{3.736724in}{2.577557in}}%
\pgfpathlineto{\pgfqpoint{3.723826in}{2.583995in}}%
\pgfpathlineto{\pgfqpoint{3.716243in}{2.572905in}}%
\pgfpathlineto{\pgfqpoint{3.708654in}{2.561890in}}%
\pgfpathlineto{\pgfqpoint{3.701061in}{2.550946in}}%
\pgfpathlineto{\pgfqpoint{3.693463in}{2.540072in}}%
\pgfpathclose%
\pgfusepath{fill}%
\end{pgfscope}%
\begin{pgfscope}%
\pgfpathrectangle{\pgfqpoint{1.254980in}{0.150000in}}{\pgfqpoint{5.490039in}{5.490039in}}%
\pgfusepath{clip}%
\pgfsetbuttcap%
\pgfsetroundjoin%
\definecolor{currentfill}{rgb}{0.201239,0.383670,0.554294}%
\pgfsetfillcolor{currentfill}%
\pgfsetfillopacity{0.700000}%
\pgfsetlinewidth{0.000000pt}%
\definecolor{currentstroke}{rgb}{0.000000,0.000000,0.000000}%
\pgfsetstrokecolor{currentstroke}%
\pgfsetdash{}{0pt}%
\pgfpathmoveto{\pgfqpoint{2.928651in}{3.031151in}}%
\pgfpathlineto{\pgfqpoint{2.941701in}{3.012795in}}%
\pgfpathlineto{\pgfqpoint{2.954743in}{2.994732in}}%
\pgfpathlineto{\pgfqpoint{2.967777in}{2.976959in}}%
\pgfpathlineto{\pgfqpoint{2.980804in}{2.959472in}}%
\pgfpathlineto{\pgfqpoint{2.988609in}{2.970514in}}%
\pgfpathlineto{\pgfqpoint{2.996406in}{2.981690in}}%
\pgfpathlineto{\pgfqpoint{3.004195in}{2.992999in}}%
\pgfpathlineto{\pgfqpoint{3.011978in}{3.004443in}}%
\pgfpathlineto{\pgfqpoint{2.998965in}{3.022012in}}%
\pgfpathlineto{\pgfqpoint{2.985946in}{3.039867in}}%
\pgfpathlineto{\pgfqpoint{2.972918in}{3.058012in}}%
\pgfpathlineto{\pgfqpoint{2.959883in}{3.076449in}}%
\pgfpathlineto{\pgfqpoint{2.952087in}{3.064912in}}%
\pgfpathlineto{\pgfqpoint{2.944282in}{3.053518in}}%
\pgfpathlineto{\pgfqpoint{2.936471in}{3.042264in}}%
\pgfpathlineto{\pgfqpoint{2.928651in}{3.031151in}}%
\pgfpathclose%
\pgfusepath{fill}%
\end{pgfscope}%
\begin{pgfscope}%
\pgfpathrectangle{\pgfqpoint{1.254980in}{0.150000in}}{\pgfqpoint{5.490039in}{5.490039in}}%
\pgfusepath{clip}%
\pgfsetbuttcap%
\pgfsetroundjoin%
\definecolor{currentfill}{rgb}{0.277134,0.185228,0.489898}%
\pgfsetfillcolor{currentfill}%
\pgfsetfillopacity{0.700000}%
\pgfsetlinewidth{0.000000pt}%
\definecolor{currentstroke}{rgb}{0.000000,0.000000,0.000000}%
\pgfsetstrokecolor{currentstroke}%
\pgfsetdash{}{0pt}%
\pgfpathmoveto{\pgfqpoint{3.425946in}{2.579635in}}%
\pgfpathlineto{\pgfqpoint{3.438848in}{2.570279in}}%
\pgfpathlineto{\pgfqpoint{3.451750in}{2.561141in}}%
\pgfpathlineto{\pgfqpoint{3.464652in}{2.552220in}}%
\pgfpathlineto{\pgfqpoint{3.477555in}{2.543514in}}%
\pgfpathlineto{\pgfqpoint{3.485224in}{2.554199in}}%
\pgfpathlineto{\pgfqpoint{3.492887in}{2.564959in}}%
\pgfpathlineto{\pgfqpoint{3.500545in}{2.575797in}}%
\pgfpathlineto{\pgfqpoint{3.508198in}{2.586714in}}%
\pgfpathlineto{\pgfqpoint{3.495305in}{2.595530in}}%
\pgfpathlineto{\pgfqpoint{3.482413in}{2.604562in}}%
\pgfpathlineto{\pgfqpoint{3.469522in}{2.613810in}}%
\pgfpathlineto{\pgfqpoint{3.456630in}{2.623277in}}%
\pgfpathlineto{\pgfqpoint{3.448968in}{2.612239in}}%
\pgfpathlineto{\pgfqpoint{3.441300in}{2.601288in}}%
\pgfpathlineto{\pgfqpoint{3.433626in}{2.590420in}}%
\pgfpathlineto{\pgfqpoint{3.425946in}{2.579635in}}%
\pgfpathclose%
\pgfusepath{fill}%
\end{pgfscope}%
\begin{pgfscope}%
\pgfpathrectangle{\pgfqpoint{1.254980in}{0.150000in}}{\pgfqpoint{5.490039in}{5.490039in}}%
\pgfusepath{clip}%
\pgfsetbuttcap%
\pgfsetroundjoin%
\definecolor{currentfill}{rgb}{0.168126,0.459988,0.558082}%
\pgfsetfillcolor{currentfill}%
\pgfsetfillopacity{0.700000}%
\pgfsetlinewidth{0.000000pt}%
\definecolor{currentstroke}{rgb}{0.000000,0.000000,0.000000}%
\pgfsetstrokecolor{currentstroke}%
\pgfsetdash{}{0pt}%
\pgfpathmoveto{\pgfqpoint{5.269726in}{3.189608in}}%
\pgfpathlineto{\pgfqpoint{5.283044in}{3.189698in}}%
\pgfpathlineto{\pgfqpoint{5.296372in}{3.189938in}}%
\pgfpathlineto{\pgfqpoint{5.309713in}{3.190330in}}%
\pgfpathlineto{\pgfqpoint{5.323065in}{3.190872in}}%
\pgfpathlineto{\pgfqpoint{5.330223in}{3.203045in}}%
\pgfpathlineto{\pgfqpoint{5.337385in}{3.215490in}}%
\pgfpathlineto{\pgfqpoint{5.344550in}{3.228214in}}%
\pgfpathlineto{\pgfqpoint{5.331214in}{3.228169in}}%
\pgfpathlineto{\pgfqpoint{5.317890in}{3.228273in}}%
\pgfpathlineto{\pgfqpoint{5.304577in}{3.228528in}}%
\pgfpathlineto{\pgfqpoint{5.291276in}{3.228933in}}%
\pgfpathlineto{\pgfqpoint{5.284089in}{3.215541in}}%
\pgfpathlineto{\pgfqpoint{5.276906in}{3.202435in}}%
\pgfpathlineto{\pgfqpoint{5.269726in}{3.189608in}}%
\pgfpathclose%
\pgfusepath{fill}%
\end{pgfscope}%
\begin{pgfscope}%
\pgfpathrectangle{\pgfqpoint{1.254980in}{0.150000in}}{\pgfqpoint{5.490039in}{5.490039in}}%
\pgfusepath{clip}%
\pgfsetbuttcap%
\pgfsetroundjoin%
\definecolor{currentfill}{rgb}{0.267968,0.223549,0.512008}%
\pgfsetfillcolor{currentfill}%
\pgfsetfillopacity{0.700000}%
\pgfsetlinewidth{0.000000pt}%
\definecolor{currentstroke}{rgb}{0.000000,0.000000,0.000000}%
\pgfsetstrokecolor{currentstroke}%
\pgfsetdash{}{0pt}%
\pgfpathmoveto{\pgfqpoint{3.240157in}{2.666161in}}%
\pgfpathlineto{\pgfqpoint{3.253084in}{2.654146in}}%
\pgfpathlineto{\pgfqpoint{3.266008in}{2.642369in}}%
\pgfpathlineto{\pgfqpoint{3.278930in}{2.630828in}}%
\pgfpathlineto{\pgfqpoint{3.291851in}{2.619520in}}%
\pgfpathlineto{\pgfqpoint{3.299576in}{2.630133in}}%
\pgfpathlineto{\pgfqpoint{3.307296in}{2.640837in}}%
\pgfpathlineto{\pgfqpoint{3.315009in}{2.651633in}}%
\pgfpathlineto{\pgfqpoint{3.322716in}{2.662523in}}%
\pgfpathlineto{\pgfqpoint{3.309807in}{2.673912in}}%
\pgfpathlineto{\pgfqpoint{3.296897in}{2.685536in}}%
\pgfpathlineto{\pgfqpoint{3.283984in}{2.697396in}}%
\pgfpathlineto{\pgfqpoint{3.271069in}{2.709493in}}%
\pgfpathlineto{\pgfqpoint{3.263351in}{2.698511in}}%
\pgfpathlineto{\pgfqpoint{3.255626in}{2.687629in}}%
\pgfpathlineto{\pgfqpoint{3.247894in}{2.676847in}}%
\pgfpathlineto{\pgfqpoint{3.240157in}{2.666161in}}%
\pgfpathclose%
\pgfusepath{fill}%
\end{pgfscope}%
\begin{pgfscope}%
\pgfpathrectangle{\pgfqpoint{1.254980in}{0.150000in}}{\pgfqpoint{5.490039in}{5.490039in}}%
\pgfusepath{clip}%
\pgfsetbuttcap%
\pgfsetroundjoin%
\definecolor{currentfill}{rgb}{0.269308,0.218818,0.509577}%
\pgfsetfillcolor{currentfill}%
\pgfsetfillopacity{0.700000}%
\pgfsetlinewidth{0.000000pt}%
\definecolor{currentstroke}{rgb}{0.000000,0.000000,0.000000}%
\pgfsetstrokecolor{currentstroke}%
\pgfsetdash{}{0pt}%
\pgfpathmoveto{\pgfqpoint{4.206377in}{2.638813in}}%
\pgfpathlineto{\pgfqpoint{4.219383in}{2.636631in}}%
\pgfpathlineto{\pgfqpoint{4.232396in}{2.634624in}}%
\pgfpathlineto{\pgfqpoint{4.245416in}{2.632790in}}%
\pgfpathlineto{\pgfqpoint{4.258444in}{2.631129in}}%
\pgfpathlineto{\pgfqpoint{4.265881in}{2.641552in}}%
\pgfpathlineto{\pgfqpoint{4.273314in}{2.652046in}}%
\pgfpathlineto{\pgfqpoint{4.280743in}{2.662617in}}%
\pgfpathlineto{\pgfqpoint{4.288168in}{2.673267in}}%
\pgfpathlineto{\pgfqpoint{4.275150in}{2.675233in}}%
\pgfpathlineto{\pgfqpoint{4.262139in}{2.677370in}}%
\pgfpathlineto{\pgfqpoint{4.249135in}{2.679682in}}%
\pgfpathlineto{\pgfqpoint{4.236138in}{2.682168in}}%
\pgfpathlineto{\pgfqpoint{4.228704in}{2.671203in}}%
\pgfpathlineto{\pgfqpoint{4.221266in}{2.660325in}}%
\pgfpathlineto{\pgfqpoint{4.213823in}{2.649529in}}%
\pgfpathlineto{\pgfqpoint{4.206377in}{2.638813in}}%
\pgfpathclose%
\pgfusepath{fill}%
\end{pgfscope}%
\begin{pgfscope}%
\pgfpathrectangle{\pgfqpoint{1.254980in}{0.150000in}}{\pgfqpoint{5.490039in}{5.490039in}}%
\pgfusepath{clip}%
\pgfsetbuttcap%
\pgfsetroundjoin%
\definecolor{currentfill}{rgb}{0.187231,0.414746,0.556547}%
\pgfsetfillcolor{currentfill}%
\pgfsetfillopacity{0.700000}%
\pgfsetlinewidth{0.000000pt}%
\definecolor{currentstroke}{rgb}{0.000000,0.000000,0.000000}%
\pgfsetstrokecolor{currentstroke}%
\pgfsetdash{}{0pt}%
\pgfpathmoveto{\pgfqpoint{2.876365in}{3.107552in}}%
\pgfpathlineto{\pgfqpoint{2.889450in}{3.087999in}}%
\pgfpathlineto{\pgfqpoint{2.902526in}{3.068750in}}%
\pgfpathlineto{\pgfqpoint{2.915593in}{3.049801in}}%
\pgfpathlineto{\pgfqpoint{2.928651in}{3.031151in}}%
\pgfpathlineto{\pgfqpoint{2.936471in}{3.042264in}}%
\pgfpathlineto{\pgfqpoint{2.944282in}{3.053518in}}%
\pgfpathlineto{\pgfqpoint{2.952087in}{3.064912in}}%
\pgfpathlineto{\pgfqpoint{2.959883in}{3.076449in}}%
\pgfpathlineto{\pgfqpoint{2.946840in}{3.095181in}}%
\pgfpathlineto{\pgfqpoint{2.933788in}{3.114212in}}%
\pgfpathlineto{\pgfqpoint{2.920727in}{3.133543in}}%
\pgfpathlineto{\pgfqpoint{2.907658in}{3.153178in}}%
\pgfpathlineto{\pgfqpoint{2.899846in}{3.141548in}}%
\pgfpathlineto{\pgfqpoint{2.892027in}{3.130068in}}%
\pgfpathlineto{\pgfqpoint{2.884200in}{3.118737in}}%
\pgfpathlineto{\pgfqpoint{2.876365in}{3.107552in}}%
\pgfpathclose%
\pgfusepath{fill}%
\end{pgfscope}%
\begin{pgfscope}%
\pgfpathrectangle{\pgfqpoint{1.254980in}{0.150000in}}{\pgfqpoint{5.490039in}{5.490039in}}%
\pgfusepath{clip}%
\pgfsetbuttcap%
\pgfsetroundjoin%
\definecolor{currentfill}{rgb}{0.279574,0.170599,0.479997}%
\pgfsetfillcolor{currentfill}%
\pgfsetfillopacity{0.700000}%
\pgfsetlinewidth{0.000000pt}%
\definecolor{currentstroke}{rgb}{0.000000,0.000000,0.000000}%
\pgfsetstrokecolor{currentstroke}%
\pgfsetdash{}{0pt}%
\pgfpathmoveto{\pgfqpoint{3.827106in}{2.537975in}}%
\pgfpathlineto{\pgfqpoint{3.840033in}{2.533092in}}%
\pgfpathlineto{\pgfqpoint{3.852964in}{2.528399in}}%
\pgfpathlineto{\pgfqpoint{3.865900in}{2.523896in}}%
\pgfpathlineto{\pgfqpoint{3.878840in}{2.519580in}}%
\pgfpathlineto{\pgfqpoint{3.886393in}{2.530197in}}%
\pgfpathlineto{\pgfqpoint{3.893941in}{2.540874in}}%
\pgfpathlineto{\pgfqpoint{3.901484in}{2.551611in}}%
\pgfpathlineto{\pgfqpoint{3.909023in}{2.562411in}}%
\pgfpathlineto{\pgfqpoint{3.896091in}{2.566921in}}%
\pgfpathlineto{\pgfqpoint{3.883164in}{2.571618in}}%
\pgfpathlineto{\pgfqpoint{3.870241in}{2.576505in}}%
\pgfpathlineto{\pgfqpoint{3.857322in}{2.581581in}}%
\pgfpathlineto{\pgfqpoint{3.849775in}{2.570577in}}%
\pgfpathlineto{\pgfqpoint{3.842224in}{2.559643in}}%
\pgfpathlineto{\pgfqpoint{3.834667in}{2.548776in}}%
\pgfpathlineto{\pgfqpoint{3.827106in}{2.537975in}}%
\pgfpathclose%
\pgfusepath{fill}%
\end{pgfscope}%
\begin{pgfscope}%
\pgfpathrectangle{\pgfqpoint{1.254980in}{0.150000in}}{\pgfqpoint{5.490039in}{5.490039in}}%
\pgfusepath{clip}%
\pgfsetbuttcap%
\pgfsetroundjoin%
\definecolor{currentfill}{rgb}{0.273006,0.204520,0.501721}%
\pgfsetfillcolor{currentfill}%
\pgfsetfillopacity{0.700000}%
\pgfsetlinewidth{0.000000pt}%
\definecolor{currentstroke}{rgb}{0.000000,0.000000,0.000000}%
\pgfsetstrokecolor{currentstroke}%
\pgfsetdash{}{0pt}%
\pgfpathmoveto{\pgfqpoint{4.124558in}{2.606047in}}%
\pgfpathlineto{\pgfqpoint{4.137545in}{2.603438in}}%
\pgfpathlineto{\pgfqpoint{4.150540in}{2.601005in}}%
\pgfpathlineto{\pgfqpoint{4.163541in}{2.598750in}}%
\pgfpathlineto{\pgfqpoint{4.176549in}{2.596670in}}%
\pgfpathlineto{\pgfqpoint{4.184012in}{2.607104in}}%
\pgfpathlineto{\pgfqpoint{4.191472in}{2.617604in}}%
\pgfpathlineto{\pgfqpoint{4.198926in}{2.628172in}}%
\pgfpathlineto{\pgfqpoint{4.206377in}{2.638813in}}%
\pgfpathlineto{\pgfqpoint{4.193378in}{2.641170in}}%
\pgfpathlineto{\pgfqpoint{4.180386in}{2.643702in}}%
\pgfpathlineto{\pgfqpoint{4.167400in}{2.646410in}}%
\pgfpathlineto{\pgfqpoint{4.154421in}{2.649296in}}%
\pgfpathlineto{\pgfqpoint{4.146962in}{2.638369in}}%
\pgfpathlineto{\pgfqpoint{4.139498in}{2.627520in}}%
\pgfpathlineto{\pgfqpoint{4.132030in}{2.616747in}}%
\pgfpathlineto{\pgfqpoint{4.124558in}{2.606047in}}%
\pgfpathclose%
\pgfusepath{fill}%
\end{pgfscope}%
\begin{pgfscope}%
\pgfpathrectangle{\pgfqpoint{1.254980in}{0.150000in}}{\pgfqpoint{5.490039in}{5.490039in}}%
\pgfusepath{clip}%
\pgfsetbuttcap%
\pgfsetroundjoin%
\definecolor{currentfill}{rgb}{0.273006,0.204520,0.501721}%
\pgfsetfillcolor{currentfill}%
\pgfsetfillopacity{0.700000}%
\pgfsetlinewidth{0.000000pt}%
\definecolor{currentstroke}{rgb}{0.000000,0.000000,0.000000}%
\pgfsetstrokecolor{currentstroke}%
\pgfsetdash{}{0pt}%
\pgfpathmoveto{\pgfqpoint{3.291851in}{2.619520in}}%
\pgfpathlineto{\pgfqpoint{3.304770in}{2.608446in}}%
\pgfpathlineto{\pgfqpoint{3.317687in}{2.597602in}}%
\pgfpathlineto{\pgfqpoint{3.330603in}{2.586987in}}%
\pgfpathlineto{\pgfqpoint{3.343518in}{2.576600in}}%
\pgfpathlineto{\pgfqpoint{3.351232in}{2.587141in}}%
\pgfpathlineto{\pgfqpoint{3.358940in}{2.597765in}}%
\pgfpathlineto{\pgfqpoint{3.366641in}{2.608476in}}%
\pgfpathlineto{\pgfqpoint{3.374338in}{2.619272in}}%
\pgfpathlineto{\pgfqpoint{3.361434in}{2.629742in}}%
\pgfpathlineto{\pgfqpoint{3.348529in}{2.640439in}}%
\pgfpathlineto{\pgfqpoint{3.335623in}{2.651366in}}%
\pgfpathlineto{\pgfqpoint{3.322716in}{2.662523in}}%
\pgfpathlineto{\pgfqpoint{3.315009in}{2.651633in}}%
\pgfpathlineto{\pgfqpoint{3.307296in}{2.640837in}}%
\pgfpathlineto{\pgfqpoint{3.299576in}{2.630133in}}%
\pgfpathlineto{\pgfqpoint{3.291851in}{2.619520in}}%
\pgfpathclose%
\pgfusepath{fill}%
\end{pgfscope}%
\begin{pgfscope}%
\pgfpathrectangle{\pgfqpoint{1.254980in}{0.150000in}}{\pgfqpoint{5.490039in}{5.490039in}}%
\pgfusepath{clip}%
\pgfsetbuttcap%
\pgfsetroundjoin%
\definecolor{currentfill}{rgb}{0.280868,0.160771,0.472899}%
\pgfsetfillcolor{currentfill}%
\pgfsetfillopacity{0.700000}%
\pgfsetlinewidth{0.000000pt}%
\definecolor{currentstroke}{rgb}{0.000000,0.000000,0.000000}%
\pgfsetstrokecolor{currentstroke}%
\pgfsetdash{}{0pt}%
\pgfpathmoveto{\pgfqpoint{3.611381in}{2.523775in}}%
\pgfpathlineto{\pgfqpoint{3.624286in}{2.516838in}}%
\pgfpathlineto{\pgfqpoint{3.637195in}{2.510102in}}%
\pgfpathlineto{\pgfqpoint{3.650105in}{2.503569in}}%
\pgfpathlineto{\pgfqpoint{3.663018in}{2.497236in}}%
\pgfpathlineto{\pgfqpoint{3.670637in}{2.507850in}}%
\pgfpathlineto{\pgfqpoint{3.678251in}{2.518526in}}%
\pgfpathlineto{\pgfqpoint{3.685859in}{2.529266in}}%
\pgfpathlineto{\pgfqpoint{3.693463in}{2.540072in}}%
\pgfpathlineto{\pgfqpoint{3.680558in}{2.546544in}}%
\pgfpathlineto{\pgfqpoint{3.667657in}{2.553216in}}%
\pgfpathlineto{\pgfqpoint{3.654758in}{2.560089in}}%
\pgfpathlineto{\pgfqpoint{3.641862in}{2.567165in}}%
\pgfpathlineto{\pgfqpoint{3.634249in}{2.556210in}}%
\pgfpathlineto{\pgfqpoint{3.626632in}{2.545328in}}%
\pgfpathlineto{\pgfqpoint{3.619009in}{2.534517in}}%
\pgfpathlineto{\pgfqpoint{3.611381in}{2.523775in}}%
\pgfpathclose%
\pgfusepath{fill}%
\end{pgfscope}%
\begin{pgfscope}%
\pgfpathrectangle{\pgfqpoint{1.254980in}{0.150000in}}{\pgfqpoint{5.490039in}{5.490039in}}%
\pgfusepath{clip}%
\pgfsetbuttcap%
\pgfsetroundjoin%
\definecolor{currentfill}{rgb}{0.279574,0.170599,0.479997}%
\pgfsetfillcolor{currentfill}%
\pgfsetfillopacity{0.700000}%
\pgfsetlinewidth{0.000000pt}%
\definecolor{currentstroke}{rgb}{0.000000,0.000000,0.000000}%
\pgfsetstrokecolor{currentstroke}%
\pgfsetdash{}{0pt}%
\pgfpathmoveto{\pgfqpoint{3.477555in}{2.543514in}}%
\pgfpathlineto{\pgfqpoint{3.490458in}{2.535022in}}%
\pgfpathlineto{\pgfqpoint{3.503362in}{2.526743in}}%
\pgfpathlineto{\pgfqpoint{3.516267in}{2.518675in}}%
\pgfpathlineto{\pgfqpoint{3.529174in}{2.510817in}}%
\pgfpathlineto{\pgfqpoint{3.536833in}{2.521402in}}%
\pgfpathlineto{\pgfqpoint{3.544487in}{2.532055in}}%
\pgfpathlineto{\pgfqpoint{3.552135in}{2.542779in}}%
\pgfpathlineto{\pgfqpoint{3.559778in}{2.553575in}}%
\pgfpathlineto{\pgfqpoint{3.546881in}{2.561544in}}%
\pgfpathlineto{\pgfqpoint{3.533985in}{2.569722in}}%
\pgfpathlineto{\pgfqpoint{3.521091in}{2.578112in}}%
\pgfpathlineto{\pgfqpoint{3.508198in}{2.586714in}}%
\pgfpathlineto{\pgfqpoint{3.500545in}{2.575797in}}%
\pgfpathlineto{\pgfqpoint{3.492887in}{2.564959in}}%
\pgfpathlineto{\pgfqpoint{3.485224in}{2.554199in}}%
\pgfpathlineto{\pgfqpoint{3.477555in}{2.543514in}}%
\pgfpathclose%
\pgfusepath{fill}%
\end{pgfscope}%
\begin{pgfscope}%
\pgfpathrectangle{\pgfqpoint{1.254980in}{0.150000in}}{\pgfqpoint{5.490039in}{5.490039in}}%
\pgfusepath{clip}%
\pgfsetbuttcap%
\pgfsetroundjoin%
\definecolor{currentfill}{rgb}{0.276194,0.190074,0.493001}%
\pgfsetfillcolor{currentfill}%
\pgfsetfillopacity{0.700000}%
\pgfsetlinewidth{0.000000pt}%
\definecolor{currentstroke}{rgb}{0.000000,0.000000,0.000000}%
\pgfsetstrokecolor{currentstroke}%
\pgfsetdash{}{0pt}%
\pgfpathmoveto{\pgfqpoint{4.042701in}{2.575128in}}%
\pgfpathlineto{\pgfqpoint{4.055673in}{2.572053in}}%
\pgfpathlineto{\pgfqpoint{4.068650in}{2.569157in}}%
\pgfpathlineto{\pgfqpoint{4.081634in}{2.566441in}}%
\pgfpathlineto{\pgfqpoint{4.094624in}{2.563903in}}%
\pgfpathlineto{\pgfqpoint{4.102114in}{2.574346in}}%
\pgfpathlineto{\pgfqpoint{4.109600in}{2.584849in}}%
\pgfpathlineto{\pgfqpoint{4.117081in}{2.595415in}}%
\pgfpathlineto{\pgfqpoint{4.124558in}{2.606047in}}%
\pgfpathlineto{\pgfqpoint{4.111576in}{2.608834in}}%
\pgfpathlineto{\pgfqpoint{4.098601in}{2.611799in}}%
\pgfpathlineto{\pgfqpoint{4.085632in}{2.614943in}}%
\pgfpathlineto{\pgfqpoint{4.072669in}{2.618267in}}%
\pgfpathlineto{\pgfqpoint{4.065184in}{2.607377in}}%
\pgfpathlineto{\pgfqpoint{4.057694in}{2.596559in}}%
\pgfpathlineto{\pgfqpoint{4.050200in}{2.585810in}}%
\pgfpathlineto{\pgfqpoint{4.042701in}{2.575128in}}%
\pgfpathclose%
\pgfusepath{fill}%
\end{pgfscope}%
\begin{pgfscope}%
\pgfpathrectangle{\pgfqpoint{1.254980in}{0.150000in}}{\pgfqpoint{5.490039in}{5.490039in}}%
\pgfusepath{clip}%
\pgfsetbuttcap%
\pgfsetroundjoin%
\definecolor{currentfill}{rgb}{0.216210,0.351535,0.550627}%
\pgfsetfillcolor{currentfill}%
\pgfsetfillopacity{0.700000}%
\pgfsetlinewidth{0.000000pt}%
\definecolor{currentstroke}{rgb}{0.000000,0.000000,0.000000}%
\pgfsetstrokecolor{currentstroke}%
\pgfsetdash{}{0pt}%
\pgfpathmoveto{\pgfqpoint{4.831414in}{2.909277in}}%
\pgfpathlineto{\pgfqpoint{4.844620in}{2.909686in}}%
\pgfpathlineto{\pgfqpoint{4.857837in}{2.910252in}}%
\pgfpathlineto{\pgfqpoint{4.871064in}{2.910976in}}%
\pgfpathlineto{\pgfqpoint{4.884302in}{2.911858in}}%
\pgfpathlineto{\pgfqpoint{4.891551in}{2.922023in}}%
\pgfpathlineto{\pgfqpoint{4.898798in}{2.932336in}}%
\pgfpathlineto{\pgfqpoint{4.906043in}{2.942803in}}%
\pgfpathlineto{\pgfqpoint{4.913286in}{2.953428in}}%
\pgfpathlineto{\pgfqpoint{4.900064in}{2.953046in}}%
\pgfpathlineto{\pgfqpoint{4.886851in}{2.952821in}}%
\pgfpathlineto{\pgfqpoint{4.873650in}{2.952754in}}%
\pgfpathlineto{\pgfqpoint{4.860458in}{2.952845in}}%
\pgfpathlineto{\pgfqpoint{4.853199in}{2.941710in}}%
\pgfpathlineto{\pgfqpoint{4.845939in}{2.930741in}}%
\pgfpathlineto{\pgfqpoint{4.838678in}{2.919932in}}%
\pgfpathlineto{\pgfqpoint{4.831414in}{2.909277in}}%
\pgfpathclose%
\pgfusepath{fill}%
\end{pgfscope}%
\begin{pgfscope}%
\pgfpathrectangle{\pgfqpoint{1.254980in}{0.150000in}}{\pgfqpoint{5.490039in}{5.490039in}}%
\pgfusepath{clip}%
\pgfsetbuttcap%
\pgfsetroundjoin%
\definecolor{currentfill}{rgb}{0.223925,0.334994,0.548053}%
\pgfsetfillcolor{currentfill}%
\pgfsetfillopacity{0.700000}%
\pgfsetlinewidth{0.000000pt}%
\definecolor{currentstroke}{rgb}{0.000000,0.000000,0.000000}%
\pgfsetstrokecolor{currentstroke}%
\pgfsetdash{}{0pt}%
\pgfpathmoveto{\pgfqpoint{4.749556in}{2.866161in}}%
\pgfpathlineto{\pgfqpoint{4.762736in}{2.866404in}}%
\pgfpathlineto{\pgfqpoint{4.775926in}{2.866807in}}%
\pgfpathlineto{\pgfqpoint{4.789126in}{2.867369in}}%
\pgfpathlineto{\pgfqpoint{4.802337in}{2.868091in}}%
\pgfpathlineto{\pgfqpoint{4.809610in}{2.878183in}}%
\pgfpathlineto{\pgfqpoint{4.816880in}{2.888409in}}%
\pgfpathlineto{\pgfqpoint{4.824148in}{2.898771in}}%
\pgfpathlineto{\pgfqpoint{4.831414in}{2.909277in}}%
\pgfpathlineto{\pgfqpoint{4.818218in}{2.909028in}}%
\pgfpathlineto{\pgfqpoint{4.805031in}{2.908937in}}%
\pgfpathlineto{\pgfqpoint{4.791855in}{2.909005in}}%
\pgfpathlineto{\pgfqpoint{4.778689in}{2.909233in}}%
\pgfpathlineto{\pgfqpoint{4.771409in}{2.898246in}}%
\pgfpathlineto{\pgfqpoint{4.764127in}{2.887408in}}%
\pgfpathlineto{\pgfqpoint{4.756843in}{2.876715in}}%
\pgfpathlineto{\pgfqpoint{4.749556in}{2.866161in}}%
\pgfpathclose%
\pgfusepath{fill}%
\end{pgfscope}%
\begin{pgfscope}%
\pgfpathrectangle{\pgfqpoint{1.254980in}{0.150000in}}{\pgfqpoint{5.490039in}{5.490039in}}%
\pgfusepath{clip}%
\pgfsetbuttcap%
\pgfsetroundjoin%
\definecolor{currentfill}{rgb}{0.206756,0.371758,0.553117}%
\pgfsetfillcolor{currentfill}%
\pgfsetfillopacity{0.700000}%
\pgfsetlinewidth{0.000000pt}%
\definecolor{currentstroke}{rgb}{0.000000,0.000000,0.000000}%
\pgfsetstrokecolor{currentstroke}%
\pgfsetdash{}{0pt}%
\pgfpathmoveto{\pgfqpoint{4.913286in}{2.953428in}}%
\pgfpathlineto{\pgfqpoint{4.926519in}{2.953967in}}%
\pgfpathlineto{\pgfqpoint{4.939763in}{2.954663in}}%
\pgfpathlineto{\pgfqpoint{4.953017in}{2.955515in}}%
\pgfpathlineto{\pgfqpoint{4.966283in}{2.956523in}}%
\pgfpathlineto{\pgfqpoint{4.973508in}{2.966796in}}%
\pgfpathlineto{\pgfqpoint{4.980733in}{2.977234in}}%
\pgfpathlineto{\pgfqpoint{4.987956in}{2.987842in}}%
\pgfpathlineto{\pgfqpoint{4.995178in}{2.998626in}}%
\pgfpathlineto{\pgfqpoint{4.981929in}{2.998145in}}%
\pgfpathlineto{\pgfqpoint{4.968691in}{2.997820in}}%
\pgfpathlineto{\pgfqpoint{4.955463in}{2.997651in}}%
\pgfpathlineto{\pgfqpoint{4.942246in}{2.997639in}}%
\pgfpathlineto{\pgfqpoint{4.935008in}{2.986318in}}%
\pgfpathlineto{\pgfqpoint{4.927768in}{2.975180in}}%
\pgfpathlineto{\pgfqpoint{4.920528in}{2.964219in}}%
\pgfpathlineto{\pgfqpoint{4.913286in}{2.953428in}}%
\pgfpathclose%
\pgfusepath{fill}%
\end{pgfscope}%
\begin{pgfscope}%
\pgfpathrectangle{\pgfqpoint{1.254980in}{0.150000in}}{\pgfqpoint{5.490039in}{5.490039in}}%
\pgfusepath{clip}%
\pgfsetbuttcap%
\pgfsetroundjoin%
\definecolor{currentfill}{rgb}{0.233603,0.313828,0.543914}%
\pgfsetfillcolor{currentfill}%
\pgfsetfillopacity{0.700000}%
\pgfsetlinewidth{0.000000pt}%
\definecolor{currentstroke}{rgb}{0.000000,0.000000,0.000000}%
\pgfsetstrokecolor{currentstroke}%
\pgfsetdash{}{0pt}%
\pgfpathmoveto{\pgfqpoint{4.667708in}{2.824087in}}%
\pgfpathlineto{\pgfqpoint{4.680862in}{2.824130in}}%
\pgfpathlineto{\pgfqpoint{4.694026in}{2.824335in}}%
\pgfpathlineto{\pgfqpoint{4.707199in}{2.824701in}}%
\pgfpathlineto{\pgfqpoint{4.720383in}{2.825228in}}%
\pgfpathlineto{\pgfqpoint{4.727680in}{2.835279in}}%
\pgfpathlineto{\pgfqpoint{4.734975in}{2.845448in}}%
\pgfpathlineto{\pgfqpoint{4.742267in}{2.855740in}}%
\pgfpathlineto{\pgfqpoint{4.749556in}{2.866161in}}%
\pgfpathlineto{\pgfqpoint{4.736386in}{2.866078in}}%
\pgfpathlineto{\pgfqpoint{4.723225in}{2.866155in}}%
\pgfpathlineto{\pgfqpoint{4.710075in}{2.866394in}}%
\pgfpathlineto{\pgfqpoint{4.696933in}{2.866793in}}%
\pgfpathlineto{\pgfqpoint{4.689631in}{2.855919in}}%
\pgfpathlineto{\pgfqpoint{4.682326in}{2.845180in}}%
\pgfpathlineto{\pgfqpoint{4.675018in}{2.834571in}}%
\pgfpathlineto{\pgfqpoint{4.667708in}{2.824087in}}%
\pgfpathclose%
\pgfusepath{fill}%
\end{pgfscope}%
\begin{pgfscope}%
\pgfpathrectangle{\pgfqpoint{1.254980in}{0.150000in}}{\pgfqpoint{5.490039in}{5.490039in}}%
\pgfusepath{clip}%
\pgfsetbuttcap%
\pgfsetroundjoin%
\definecolor{currentfill}{rgb}{0.280868,0.160771,0.472899}%
\pgfsetfillcolor{currentfill}%
\pgfsetfillopacity{0.700000}%
\pgfsetlinewidth{0.000000pt}%
\definecolor{currentstroke}{rgb}{0.000000,0.000000,0.000000}%
\pgfsetstrokecolor{currentstroke}%
\pgfsetdash{}{0pt}%
\pgfpathmoveto{\pgfqpoint{3.745109in}{2.516166in}}%
\pgfpathlineto{\pgfqpoint{3.758029in}{2.510679in}}%
\pgfpathlineto{\pgfqpoint{3.770953in}{2.505387in}}%
\pgfpathlineto{\pgfqpoint{3.783880in}{2.500287in}}%
\pgfpathlineto{\pgfqpoint{3.796812in}{2.495379in}}%
\pgfpathlineto{\pgfqpoint{3.804393in}{2.505941in}}%
\pgfpathlineto{\pgfqpoint{3.811969in}{2.516560in}}%
\pgfpathlineto{\pgfqpoint{3.819540in}{2.527237in}}%
\pgfpathlineto{\pgfqpoint{3.827106in}{2.537975in}}%
\pgfpathlineto{\pgfqpoint{3.814183in}{2.543049in}}%
\pgfpathlineto{\pgfqpoint{3.801264in}{2.548315in}}%
\pgfpathlineto{\pgfqpoint{3.788349in}{2.553774in}}%
\pgfpathlineto{\pgfqpoint{3.775438in}{2.559427in}}%
\pgfpathlineto{\pgfqpoint{3.767863in}{2.548512in}}%
\pgfpathlineto{\pgfqpoint{3.760283in}{2.537666in}}%
\pgfpathlineto{\pgfqpoint{3.752699in}{2.526884in}}%
\pgfpathlineto{\pgfqpoint{3.745109in}{2.516166in}}%
\pgfpathclose%
\pgfusepath{fill}%
\end{pgfscope}%
\begin{pgfscope}%
\pgfpathrectangle{\pgfqpoint{1.254980in}{0.150000in}}{\pgfqpoint{5.490039in}{5.490039in}}%
\pgfusepath{clip}%
\pgfsetbuttcap%
\pgfsetroundjoin%
\definecolor{currentfill}{rgb}{0.197636,0.391528,0.554969}%
\pgfsetfillcolor{currentfill}%
\pgfsetfillopacity{0.700000}%
\pgfsetlinewidth{0.000000pt}%
\definecolor{currentstroke}{rgb}{0.000000,0.000000,0.000000}%
\pgfsetstrokecolor{currentstroke}%
\pgfsetdash{}{0pt}%
\pgfpathmoveto{\pgfqpoint{4.995178in}{2.998626in}}%
\pgfpathlineto{\pgfqpoint{5.008438in}{2.999262in}}%
\pgfpathlineto{\pgfqpoint{5.021708in}{3.000053in}}%
\pgfpathlineto{\pgfqpoint{5.034990in}{3.000999in}}%
\pgfpathlineto{\pgfqpoint{5.048283in}{3.002100in}}%
\pgfpathlineto{\pgfqpoint{5.055487in}{3.012522in}}%
\pgfpathlineto{\pgfqpoint{5.062690in}{3.023125in}}%
\pgfpathlineto{\pgfqpoint{5.069893in}{3.033917in}}%
\pgfpathlineto{\pgfqpoint{5.077095in}{3.044903in}}%
\pgfpathlineto{\pgfqpoint{5.063820in}{3.044358in}}%
\pgfpathlineto{\pgfqpoint{5.050556in}{3.043966in}}%
\pgfpathlineto{\pgfqpoint{5.037303in}{3.043730in}}%
\pgfpathlineto{\pgfqpoint{5.024060in}{3.043648in}}%
\pgfpathlineto{\pgfqpoint{5.016840in}{3.032097in}}%
\pgfpathlineto{\pgfqpoint{5.009620in}{3.020747in}}%
\pgfpathlineto{\pgfqpoint{5.002399in}{3.009592in}}%
\pgfpathlineto{\pgfqpoint{4.995178in}{2.998626in}}%
\pgfpathclose%
\pgfusepath{fill}%
\end{pgfscope}%
\begin{pgfscope}%
\pgfpathrectangle{\pgfqpoint{1.254980in}{0.150000in}}{\pgfqpoint{5.490039in}{5.490039in}}%
\pgfusepath{clip}%
\pgfsetbuttcap%
\pgfsetroundjoin%
\definecolor{currentfill}{rgb}{0.241237,0.296485,0.539709}%
\pgfsetfillcolor{currentfill}%
\pgfsetfillopacity{0.700000}%
\pgfsetlinewidth{0.000000pt}%
\definecolor{currentstroke}{rgb}{0.000000,0.000000,0.000000}%
\pgfsetstrokecolor{currentstroke}%
\pgfsetdash{}{0pt}%
\pgfpathmoveto{\pgfqpoint{4.585865in}{2.783085in}}%
\pgfpathlineto{\pgfqpoint{4.598993in}{2.782894in}}%
\pgfpathlineto{\pgfqpoint{4.612131in}{2.782866in}}%
\pgfpathlineto{\pgfqpoint{4.625278in}{2.783001in}}%
\pgfpathlineto{\pgfqpoint{4.638435in}{2.783299in}}%
\pgfpathlineto{\pgfqpoint{4.645758in}{2.793333in}}%
\pgfpathlineto{\pgfqpoint{4.653078in}{2.803473in}}%
\pgfpathlineto{\pgfqpoint{4.660394in}{2.813722in}}%
\pgfpathlineto{\pgfqpoint{4.667708in}{2.824087in}}%
\pgfpathlineto{\pgfqpoint{4.654563in}{2.824205in}}%
\pgfpathlineto{\pgfqpoint{4.641428in}{2.824486in}}%
\pgfpathlineto{\pgfqpoint{4.628303in}{2.824929in}}%
\pgfpathlineto{\pgfqpoint{4.615186in}{2.825536in}}%
\pgfpathlineto{\pgfqpoint{4.607860in}{2.814745in}}%
\pgfpathlineto{\pgfqpoint{4.600532in}{2.804077in}}%
\pgfpathlineto{\pgfqpoint{4.593200in}{2.793525in}}%
\pgfpathlineto{\pgfqpoint{4.585865in}{2.783085in}}%
\pgfpathclose%
\pgfusepath{fill}%
\end{pgfscope}%
\begin{pgfscope}%
\pgfpathrectangle{\pgfqpoint{1.254980in}{0.150000in}}{\pgfqpoint{5.490039in}{5.490039in}}%
\pgfusepath{clip}%
\pgfsetbuttcap%
\pgfsetroundjoin%
\definecolor{currentfill}{rgb}{0.277134,0.185228,0.489898}%
\pgfsetfillcolor{currentfill}%
\pgfsetfillopacity{0.700000}%
\pgfsetlinewidth{0.000000pt}%
\definecolor{currentstroke}{rgb}{0.000000,0.000000,0.000000}%
\pgfsetstrokecolor{currentstroke}%
\pgfsetdash{}{0pt}%
\pgfpathmoveto{\pgfqpoint{3.343518in}{2.576600in}}%
\pgfpathlineto{\pgfqpoint{3.356432in}{2.566438in}}%
\pgfpathlineto{\pgfqpoint{3.369345in}{2.556501in}}%
\pgfpathlineto{\pgfqpoint{3.382258in}{2.546787in}}%
\pgfpathlineto{\pgfqpoint{3.395171in}{2.537294in}}%
\pgfpathlineto{\pgfqpoint{3.402873in}{2.547762in}}%
\pgfpathlineto{\pgfqpoint{3.410570in}{2.558308in}}%
\pgfpathlineto{\pgfqpoint{3.418261in}{2.568932in}}%
\pgfpathlineto{\pgfqpoint{3.425946in}{2.579635in}}%
\pgfpathlineto{\pgfqpoint{3.413045in}{2.589211in}}%
\pgfpathlineto{\pgfqpoint{3.400143in}{2.599008in}}%
\pgfpathlineto{\pgfqpoint{3.387240in}{2.609028in}}%
\pgfpathlineto{\pgfqpoint{3.374338in}{2.619272in}}%
\pgfpathlineto{\pgfqpoint{3.366641in}{2.608476in}}%
\pgfpathlineto{\pgfqpoint{3.358940in}{2.597765in}}%
\pgfpathlineto{\pgfqpoint{3.351232in}{2.587141in}}%
\pgfpathlineto{\pgfqpoint{3.343518in}{2.576600in}}%
\pgfpathclose%
\pgfusepath{fill}%
\end{pgfscope}%
\begin{pgfscope}%
\pgfpathrectangle{\pgfqpoint{1.254980in}{0.150000in}}{\pgfqpoint{5.490039in}{5.490039in}}%
\pgfusepath{clip}%
\pgfsetbuttcap%
\pgfsetroundjoin%
\definecolor{currentfill}{rgb}{0.188923,0.410910,0.556326}%
\pgfsetfillcolor{currentfill}%
\pgfsetfillopacity{0.700000}%
\pgfsetlinewidth{0.000000pt}%
\definecolor{currentstroke}{rgb}{0.000000,0.000000,0.000000}%
\pgfsetstrokecolor{currentstroke}%
\pgfsetdash{}{0pt}%
\pgfpathmoveto{\pgfqpoint{5.077095in}{3.044903in}}%
\pgfpathlineto{\pgfqpoint{5.090381in}{3.045603in}}%
\pgfpathlineto{\pgfqpoint{5.103679in}{3.046456in}}%
\pgfpathlineto{\pgfqpoint{5.116987in}{3.047464in}}%
\pgfpathlineto{\pgfqpoint{5.130307in}{3.048624in}}%
\pgfpathlineto{\pgfqpoint{5.137491in}{3.059239in}}%
\pgfpathlineto{\pgfqpoint{5.144675in}{3.070055in}}%
\pgfpathlineto{\pgfqpoint{5.151859in}{3.081078in}}%
\pgfpathlineto{\pgfqpoint{5.159044in}{3.092316in}}%
\pgfpathlineto{\pgfqpoint{5.145743in}{3.091739in}}%
\pgfpathlineto{\pgfqpoint{5.132453in}{3.091314in}}%
\pgfpathlineto{\pgfqpoint{5.119174in}{3.091043in}}%
\pgfpathlineto{\pgfqpoint{5.105906in}{3.090926in}}%
\pgfpathlineto{\pgfqpoint{5.098703in}{3.079095in}}%
\pgfpathlineto{\pgfqpoint{5.091500in}{3.067486in}}%
\pgfpathlineto{\pgfqpoint{5.084298in}{3.056091in}}%
\pgfpathlineto{\pgfqpoint{5.077095in}{3.044903in}}%
\pgfpathclose%
\pgfusepath{fill}%
\end{pgfscope}%
\begin{pgfscope}%
\pgfpathrectangle{\pgfqpoint{1.254980in}{0.150000in}}{\pgfqpoint{5.490039in}{5.490039in}}%
\pgfusepath{clip}%
\pgfsetbuttcap%
\pgfsetroundjoin%
\definecolor{currentfill}{rgb}{0.248629,0.278775,0.534556}%
\pgfsetfillcolor{currentfill}%
\pgfsetfillopacity{0.700000}%
\pgfsetlinewidth{0.000000pt}%
\definecolor{currentstroke}{rgb}{0.000000,0.000000,0.000000}%
\pgfsetstrokecolor{currentstroke}%
\pgfsetdash{}{0pt}%
\pgfpathmoveto{\pgfqpoint{4.504022in}{2.743206in}}%
\pgfpathlineto{\pgfqpoint{4.517125in}{2.742745in}}%
\pgfpathlineto{\pgfqpoint{4.530238in}{2.742449in}}%
\pgfpathlineto{\pgfqpoint{4.543360in}{2.742319in}}%
\pgfpathlineto{\pgfqpoint{4.556490in}{2.742352in}}%
\pgfpathlineto{\pgfqpoint{4.563839in}{2.752391in}}%
\pgfpathlineto{\pgfqpoint{4.571185in}{2.762522in}}%
\pgfpathlineto{\pgfqpoint{4.578526in}{2.772752in}}%
\pgfpathlineto{\pgfqpoint{4.585865in}{2.783085in}}%
\pgfpathlineto{\pgfqpoint{4.572745in}{2.783440in}}%
\pgfpathlineto{\pgfqpoint{4.559635in}{2.783959in}}%
\pgfpathlineto{\pgfqpoint{4.546534in}{2.784642in}}%
\pgfpathlineto{\pgfqpoint{4.533441in}{2.785491in}}%
\pgfpathlineto{\pgfqpoint{4.526092in}{2.774760in}}%
\pgfpathlineto{\pgfqpoint{4.518738in}{2.764139in}}%
\pgfpathlineto{\pgfqpoint{4.511382in}{2.753622in}}%
\pgfpathlineto{\pgfqpoint{4.504022in}{2.743206in}}%
\pgfpathclose%
\pgfusepath{fill}%
\end{pgfscope}%
\begin{pgfscope}%
\pgfpathrectangle{\pgfqpoint{1.254980in}{0.150000in}}{\pgfqpoint{5.490039in}{5.490039in}}%
\pgfusepath{clip}%
\pgfsetbuttcap%
\pgfsetroundjoin%
\definecolor{currentfill}{rgb}{0.278012,0.180367,0.486697}%
\pgfsetfillcolor{currentfill}%
\pgfsetfillopacity{0.700000}%
\pgfsetlinewidth{0.000000pt}%
\definecolor{currentstroke}{rgb}{0.000000,0.000000,0.000000}%
\pgfsetstrokecolor{currentstroke}%
\pgfsetdash{}{0pt}%
\pgfpathmoveto{\pgfqpoint{3.960799in}{2.546238in}}%
\pgfpathlineto{\pgfqpoint{3.973756in}{2.542656in}}%
\pgfpathlineto{\pgfqpoint{3.986719in}{2.539258in}}%
\pgfpathlineto{\pgfqpoint{3.999687in}{2.536042in}}%
\pgfpathlineto{\pgfqpoint{4.012661in}{2.533007in}}%
\pgfpathlineto{\pgfqpoint{4.020178in}{2.543452in}}%
\pgfpathlineto{\pgfqpoint{4.027690in}{2.553952in}}%
\pgfpathlineto{\pgfqpoint{4.035198in}{2.564510in}}%
\pgfpathlineto{\pgfqpoint{4.042701in}{2.575128in}}%
\pgfpathlineto{\pgfqpoint{4.029736in}{2.578384in}}%
\pgfpathlineto{\pgfqpoint{4.016776in}{2.581822in}}%
\pgfpathlineto{\pgfqpoint{4.003822in}{2.585442in}}%
\pgfpathlineto{\pgfqpoint{3.990873in}{2.589244in}}%
\pgfpathlineto{\pgfqpoint{3.983362in}{2.578394in}}%
\pgfpathlineto{\pgfqpoint{3.975845in}{2.567612in}}%
\pgfpathlineto{\pgfqpoint{3.968325in}{2.556894in}}%
\pgfpathlineto{\pgfqpoint{3.960799in}{2.546238in}}%
\pgfpathclose%
\pgfusepath{fill}%
\end{pgfscope}%
\begin{pgfscope}%
\pgfpathrectangle{\pgfqpoint{1.254980in}{0.150000in}}{\pgfqpoint{5.490039in}{5.490039in}}%
\pgfusepath{clip}%
\pgfsetbuttcap%
\pgfsetroundjoin%
\definecolor{currentfill}{rgb}{0.180629,0.429975,0.557282}%
\pgfsetfillcolor{currentfill}%
\pgfsetfillopacity{0.700000}%
\pgfsetlinewidth{0.000000pt}%
\definecolor{currentstroke}{rgb}{0.000000,0.000000,0.000000}%
\pgfsetstrokecolor{currentstroke}%
\pgfsetdash{}{0pt}%
\pgfpathmoveto{\pgfqpoint{5.159044in}{3.092316in}}%
\pgfpathlineto{\pgfqpoint{5.172356in}{3.093046in}}%
\pgfpathlineto{\pgfqpoint{5.185680in}{3.093929in}}%
\pgfpathlineto{\pgfqpoint{5.199015in}{3.094964in}}%
\pgfpathlineto{\pgfqpoint{5.212362in}{3.096151in}}%
\pgfpathlineto{\pgfqpoint{5.219528in}{3.107009in}}%
\pgfpathlineto{\pgfqpoint{5.226694in}{3.118089in}}%
\pgfpathlineto{\pgfqpoint{5.233862in}{3.129396in}}%
\pgfpathlineto{\pgfqpoint{5.241031in}{3.140940in}}%
\pgfpathlineto{\pgfqpoint{5.227704in}{3.140364in}}%
\pgfpathlineto{\pgfqpoint{5.214389in}{3.139939in}}%
\pgfpathlineto{\pgfqpoint{5.201085in}{3.139667in}}%
\pgfpathlineto{\pgfqpoint{5.187792in}{3.139547in}}%
\pgfpathlineto{\pgfqpoint{5.180603in}{3.127383in}}%
\pgfpathlineto{\pgfqpoint{5.173415in}{3.115462in}}%
\pgfpathlineto{\pgfqpoint{5.166229in}{3.103775in}}%
\pgfpathlineto{\pgfqpoint{5.159044in}{3.092316in}}%
\pgfpathclose%
\pgfusepath{fill}%
\end{pgfscope}%
\begin{pgfscope}%
\pgfpathrectangle{\pgfqpoint{1.254980in}{0.150000in}}{\pgfqpoint{5.490039in}{5.490039in}}%
\pgfusepath{clip}%
\pgfsetbuttcap%
\pgfsetroundjoin%
\definecolor{currentfill}{rgb}{0.244972,0.287675,0.537260}%
\pgfsetfillcolor{currentfill}%
\pgfsetfillopacity{0.700000}%
\pgfsetlinewidth{0.000000pt}%
\definecolor{currentstroke}{rgb}{0.000000,0.000000,0.000000}%
\pgfsetstrokecolor{currentstroke}%
\pgfsetdash{}{0pt}%
\pgfpathmoveto{\pgfqpoint{3.053605in}{2.787152in}}%
\pgfpathlineto{\pgfqpoint{3.066590in}{2.772185in}}%
\pgfpathlineto{\pgfqpoint{3.079570in}{2.757480in}}%
\pgfpathlineto{\pgfqpoint{3.092545in}{2.743035in}}%
\pgfpathlineto{\pgfqpoint{3.105516in}{2.728848in}}%
\pgfpathlineto{\pgfqpoint{3.113306in}{2.739241in}}%
\pgfpathlineto{\pgfqpoint{3.121089in}{2.749742in}}%
\pgfpathlineto{\pgfqpoint{3.128866in}{2.760351in}}%
\pgfpathlineto{\pgfqpoint{3.136636in}{2.771069in}}%
\pgfpathlineto{\pgfqpoint{3.123679in}{2.785310in}}%
\pgfpathlineto{\pgfqpoint{3.110717in}{2.799809in}}%
\pgfpathlineto{\pgfqpoint{3.097751in}{2.814568in}}%
\pgfpathlineto{\pgfqpoint{3.084780in}{2.829589in}}%
\pgfpathlineto{\pgfqpoint{3.076997in}{2.818806in}}%
\pgfpathlineto{\pgfqpoint{3.069206in}{2.808140in}}%
\pgfpathlineto{\pgfqpoint{3.061409in}{2.797589in}}%
\pgfpathlineto{\pgfqpoint{3.053605in}{2.787152in}}%
\pgfpathclose%
\pgfusepath{fill}%
\end{pgfscope}%
\begin{pgfscope}%
\pgfpathrectangle{\pgfqpoint{1.254980in}{0.150000in}}{\pgfqpoint{5.490039in}{5.490039in}}%
\pgfusepath{clip}%
\pgfsetbuttcap%
\pgfsetroundjoin%
\definecolor{currentfill}{rgb}{0.233603,0.313828,0.543914}%
\pgfsetfillcolor{currentfill}%
\pgfsetfillopacity{0.700000}%
\pgfsetlinewidth{0.000000pt}%
\definecolor{currentstroke}{rgb}{0.000000,0.000000,0.000000}%
\pgfsetstrokecolor{currentstroke}%
\pgfsetdash{}{0pt}%
\pgfpathmoveto{\pgfqpoint{3.001608in}{2.849687in}}%
\pgfpathlineto{\pgfqpoint{3.014616in}{2.833648in}}%
\pgfpathlineto{\pgfqpoint{3.027618in}{2.817881in}}%
\pgfpathlineto{\pgfqpoint{3.040614in}{2.802383in}}%
\pgfpathlineto{\pgfqpoint{3.053605in}{2.787152in}}%
\pgfpathlineto{\pgfqpoint{3.061409in}{2.797589in}}%
\pgfpathlineto{\pgfqpoint{3.069206in}{2.808140in}}%
\pgfpathlineto{\pgfqpoint{3.076997in}{2.818806in}}%
\pgfpathlineto{\pgfqpoint{3.084780in}{2.829589in}}%
\pgfpathlineto{\pgfqpoint{3.071804in}{2.844874in}}%
\pgfpathlineto{\pgfqpoint{3.058822in}{2.860426in}}%
\pgfpathlineto{\pgfqpoint{3.045835in}{2.876247in}}%
\pgfpathlineto{\pgfqpoint{3.032841in}{2.892339in}}%
\pgfpathlineto{\pgfqpoint{3.025044in}{2.881492in}}%
\pgfpathlineto{\pgfqpoint{3.017239in}{2.870769in}}%
\pgfpathlineto{\pgfqpoint{3.009427in}{2.860167in}}%
\pgfpathlineto{\pgfqpoint{3.001608in}{2.849687in}}%
\pgfpathclose%
\pgfusepath{fill}%
\end{pgfscope}%
\begin{pgfscope}%
\pgfpathrectangle{\pgfqpoint{1.254980in}{0.150000in}}{\pgfqpoint{5.490039in}{5.490039in}}%
\pgfusepath{clip}%
\pgfsetbuttcap%
\pgfsetroundjoin%
\definecolor{currentfill}{rgb}{0.255645,0.260703,0.528312}%
\pgfsetfillcolor{currentfill}%
\pgfsetfillopacity{0.700000}%
\pgfsetlinewidth{0.000000pt}%
\definecolor{currentstroke}{rgb}{0.000000,0.000000,0.000000}%
\pgfsetstrokecolor{currentstroke}%
\pgfsetdash{}{0pt}%
\pgfpathmoveto{\pgfqpoint{4.422174in}{2.704522in}}%
\pgfpathlineto{\pgfqpoint{4.435254in}{2.703755in}}%
\pgfpathlineto{\pgfqpoint{4.448342in}{2.703156in}}%
\pgfpathlineto{\pgfqpoint{4.461439in}{2.702723in}}%
\pgfpathlineto{\pgfqpoint{4.474545in}{2.702457in}}%
\pgfpathlineto{\pgfqpoint{4.481920in}{2.712515in}}%
\pgfpathlineto{\pgfqpoint{4.489291in}{2.722657in}}%
\pgfpathlineto{\pgfqpoint{4.496658in}{2.732886in}}%
\pgfpathlineto{\pgfqpoint{4.504022in}{2.743206in}}%
\pgfpathlineto{\pgfqpoint{4.490927in}{2.743833in}}%
\pgfpathlineto{\pgfqpoint{4.477841in}{2.744626in}}%
\pgfpathlineto{\pgfqpoint{4.464763in}{2.745585in}}%
\pgfpathlineto{\pgfqpoint{4.451694in}{2.746712in}}%
\pgfpathlineto{\pgfqpoint{4.444320in}{2.736021in}}%
\pgfpathlineto{\pgfqpoint{4.436942in}{2.725429in}}%
\pgfpathlineto{\pgfqpoint{4.429560in}{2.714931in}}%
\pgfpathlineto{\pgfqpoint{4.422174in}{2.704522in}}%
\pgfpathclose%
\pgfusepath{fill}%
\end{pgfscope}%
\begin{pgfscope}%
\pgfpathrectangle{\pgfqpoint{1.254980in}{0.150000in}}{\pgfqpoint{5.490039in}{5.490039in}}%
\pgfusepath{clip}%
\pgfsetbuttcap%
\pgfsetroundjoin%
\definecolor{currentfill}{rgb}{0.255645,0.260703,0.528312}%
\pgfsetfillcolor{currentfill}%
\pgfsetfillopacity{0.700000}%
\pgfsetlinewidth{0.000000pt}%
\definecolor{currentstroke}{rgb}{0.000000,0.000000,0.000000}%
\pgfsetstrokecolor{currentstroke}%
\pgfsetdash{}{0pt}%
\pgfpathmoveto{\pgfqpoint{3.105516in}{2.728848in}}%
\pgfpathlineto{\pgfqpoint{3.118482in}{2.714916in}}%
\pgfpathlineto{\pgfqpoint{3.131445in}{2.701238in}}%
\pgfpathlineto{\pgfqpoint{3.144403in}{2.687811in}}%
\pgfpathlineto{\pgfqpoint{3.157357in}{2.674633in}}%
\pgfpathlineto{\pgfqpoint{3.165134in}{2.684982in}}%
\pgfpathlineto{\pgfqpoint{3.172903in}{2.695432in}}%
\pgfpathlineto{\pgfqpoint{3.180667in}{2.705983in}}%
\pgfpathlineto{\pgfqpoint{3.188423in}{2.716637in}}%
\pgfpathlineto{\pgfqpoint{3.175482in}{2.729869in}}%
\pgfpathlineto{\pgfqpoint{3.162537in}{2.743350in}}%
\pgfpathlineto{\pgfqpoint{3.149588in}{2.757083in}}%
\pgfpathlineto{\pgfqpoint{3.136636in}{2.771069in}}%
\pgfpathlineto{\pgfqpoint{3.128866in}{2.760351in}}%
\pgfpathlineto{\pgfqpoint{3.121089in}{2.749742in}}%
\pgfpathlineto{\pgfqpoint{3.113306in}{2.739241in}}%
\pgfpathlineto{\pgfqpoint{3.105516in}{2.728848in}}%
\pgfpathclose%
\pgfusepath{fill}%
\end{pgfscope}%
\begin{pgfscope}%
\pgfpathrectangle{\pgfqpoint{1.254980in}{0.150000in}}{\pgfqpoint{5.490039in}{5.490039in}}%
\pgfusepath{clip}%
\pgfsetbuttcap%
\pgfsetroundjoin%
\definecolor{currentfill}{rgb}{0.172719,0.448791,0.557885}%
\pgfsetfillcolor{currentfill}%
\pgfsetfillopacity{0.700000}%
\pgfsetlinewidth{0.000000pt}%
\definecolor{currentstroke}{rgb}{0.000000,0.000000,0.000000}%
\pgfsetstrokecolor{currentstroke}%
\pgfsetdash{}{0pt}%
\pgfpathmoveto{\pgfqpoint{5.241031in}{3.140940in}}%
\pgfpathlineto{\pgfqpoint{5.254369in}{3.141667in}}%
\pgfpathlineto{\pgfqpoint{5.267719in}{3.142546in}}%
\pgfpathlineto{\pgfqpoint{5.281081in}{3.143576in}}%
\pgfpathlineto{\pgfqpoint{5.294454in}{3.144757in}}%
\pgfpathlineto{\pgfqpoint{5.301604in}{3.155914in}}%
\pgfpathlineto{\pgfqpoint{5.308755in}{3.167314in}}%
\pgfpathlineto{\pgfqpoint{5.315909in}{3.178964in}}%
\pgfpathlineto{\pgfqpoint{5.323065in}{3.190872in}}%
\pgfpathlineto{\pgfqpoint{5.309713in}{3.190330in}}%
\pgfpathlineto{\pgfqpoint{5.296372in}{3.189938in}}%
\pgfpathlineto{\pgfqpoint{5.283044in}{3.189698in}}%
\pgfpathlineto{\pgfqpoint{5.269726in}{3.189608in}}%
\pgfpathlineto{\pgfqpoint{5.262549in}{3.177052in}}%
\pgfpathlineto{\pgfqpoint{5.255374in}{3.164760in}}%
\pgfpathlineto{\pgfqpoint{5.248202in}{3.152725in}}%
\pgfpathlineto{\pgfqpoint{5.241031in}{3.140940in}}%
\pgfpathclose%
\pgfusepath{fill}%
\end{pgfscope}%
\begin{pgfscope}%
\pgfpathrectangle{\pgfqpoint{1.254980in}{0.150000in}}{\pgfqpoint{5.490039in}{5.490039in}}%
\pgfusepath{clip}%
\pgfsetbuttcap%
\pgfsetroundjoin%
\definecolor{currentfill}{rgb}{0.220057,0.343307,0.549413}%
\pgfsetfillcolor{currentfill}%
\pgfsetfillopacity{0.700000}%
\pgfsetlinewidth{0.000000pt}%
\definecolor{currentstroke}{rgb}{0.000000,0.000000,0.000000}%
\pgfsetstrokecolor{currentstroke}%
\pgfsetdash{}{0pt}%
\pgfpathmoveto{\pgfqpoint{2.949510in}{2.916605in}}%
\pgfpathlineto{\pgfqpoint{2.962545in}{2.899456in}}%
\pgfpathlineto{\pgfqpoint{2.975573in}{2.882588in}}%
\pgfpathlineto{\pgfqpoint{2.988594in}{2.865999in}}%
\pgfpathlineto{\pgfqpoint{3.001608in}{2.849687in}}%
\pgfpathlineto{\pgfqpoint{3.009427in}{2.860167in}}%
\pgfpathlineto{\pgfqpoint{3.017239in}{2.870769in}}%
\pgfpathlineto{\pgfqpoint{3.025044in}{2.881492in}}%
\pgfpathlineto{\pgfqpoint{3.032841in}{2.892339in}}%
\pgfpathlineto{\pgfqpoint{3.019842in}{2.908705in}}%
\pgfpathlineto{\pgfqpoint{3.006836in}{2.925348in}}%
\pgfpathlineto{\pgfqpoint{2.993823in}{2.942269in}}%
\pgfpathlineto{\pgfqpoint{2.980804in}{2.959472in}}%
\pgfpathlineto{\pgfqpoint{2.972992in}{2.948561in}}%
\pgfpathlineto{\pgfqpoint{2.965172in}{2.937780in}}%
\pgfpathlineto{\pgfqpoint{2.957345in}{2.927129in}}%
\pgfpathlineto{\pgfqpoint{2.949510in}{2.916605in}}%
\pgfpathclose%
\pgfusepath{fill}%
\end{pgfscope}%
\begin{pgfscope}%
\pgfpathrectangle{\pgfqpoint{1.254980in}{0.150000in}}{\pgfqpoint{5.490039in}{5.490039in}}%
\pgfusepath{clip}%
\pgfsetbuttcap%
\pgfsetroundjoin%
\definecolor{currentfill}{rgb}{0.262138,0.242286,0.520837}%
\pgfsetfillcolor{currentfill}%
\pgfsetfillopacity{0.700000}%
\pgfsetlinewidth{0.000000pt}%
\definecolor{currentstroke}{rgb}{0.000000,0.000000,0.000000}%
\pgfsetstrokecolor{currentstroke}%
\pgfsetdash{}{0pt}%
\pgfpathmoveto{\pgfqpoint{4.340317in}{2.667125in}}%
\pgfpathlineto{\pgfqpoint{4.353374in}{2.666016in}}%
\pgfpathlineto{\pgfqpoint{4.366439in}{2.665077in}}%
\pgfpathlineto{\pgfqpoint{4.379512in}{2.664306in}}%
\pgfpathlineto{\pgfqpoint{4.392593in}{2.663705in}}%
\pgfpathlineto{\pgfqpoint{4.399994in}{2.673794in}}%
\pgfpathlineto{\pgfqpoint{4.407392in}{2.683958in}}%
\pgfpathlineto{\pgfqpoint{4.414785in}{2.694199in}}%
\pgfpathlineto{\pgfqpoint{4.422174in}{2.704522in}}%
\pgfpathlineto{\pgfqpoint{4.409103in}{2.705457in}}%
\pgfpathlineto{\pgfqpoint{4.396040in}{2.706560in}}%
\pgfpathlineto{\pgfqpoint{4.382985in}{2.707831in}}%
\pgfpathlineto{\pgfqpoint{4.369938in}{2.709273in}}%
\pgfpathlineto{\pgfqpoint{4.362539in}{2.698607in}}%
\pgfpathlineto{\pgfqpoint{4.355136in}{2.688030in}}%
\pgfpathlineto{\pgfqpoint{4.347728in}{2.677537in}}%
\pgfpathlineto{\pgfqpoint{4.340317in}{2.667125in}}%
\pgfpathclose%
\pgfusepath{fill}%
\end{pgfscope}%
\begin{pgfscope}%
\pgfpathrectangle{\pgfqpoint{1.254980in}{0.150000in}}{\pgfqpoint{5.490039in}{5.490039in}}%
\pgfusepath{clip}%
\pgfsetbuttcap%
\pgfsetroundjoin%
\definecolor{currentfill}{rgb}{0.280868,0.160771,0.472899}%
\pgfsetfillcolor{currentfill}%
\pgfsetfillopacity{0.700000}%
\pgfsetlinewidth{0.000000pt}%
\definecolor{currentstroke}{rgb}{0.000000,0.000000,0.000000}%
\pgfsetstrokecolor{currentstroke}%
\pgfsetdash{}{0pt}%
\pgfpathmoveto{\pgfqpoint{3.529174in}{2.510817in}}%
\pgfpathlineto{\pgfqpoint{3.542082in}{2.503168in}}%
\pgfpathlineto{\pgfqpoint{3.554992in}{2.495726in}}%
\pgfpathlineto{\pgfqpoint{3.567903in}{2.488491in}}%
\pgfpathlineto{\pgfqpoint{3.580816in}{2.481460in}}%
\pgfpathlineto{\pgfqpoint{3.588465in}{2.491944in}}%
\pgfpathlineto{\pgfqpoint{3.596109in}{2.502490in}}%
\pgfpathlineto{\pgfqpoint{3.603747in}{2.513100in}}%
\pgfpathlineto{\pgfqpoint{3.611381in}{2.523775in}}%
\pgfpathlineto{\pgfqpoint{3.598477in}{2.530917in}}%
\pgfpathlineto{\pgfqpoint{3.585576in}{2.538263in}}%
\pgfpathlineto{\pgfqpoint{3.572676in}{2.545815in}}%
\pgfpathlineto{\pgfqpoint{3.559778in}{2.553575in}}%
\pgfpathlineto{\pgfqpoint{3.552135in}{2.542779in}}%
\pgfpathlineto{\pgfqpoint{3.544487in}{2.532055in}}%
\pgfpathlineto{\pgfqpoint{3.536833in}{2.521402in}}%
\pgfpathlineto{\pgfqpoint{3.529174in}{2.510817in}}%
\pgfpathclose%
\pgfusepath{fill}%
\end{pgfscope}%
\begin{pgfscope}%
\pgfpathrectangle{\pgfqpoint{1.254980in}{0.150000in}}{\pgfqpoint{5.490039in}{5.490039in}}%
\pgfusepath{clip}%
\pgfsetbuttcap%
\pgfsetroundjoin%
\definecolor{currentfill}{rgb}{0.263663,0.237631,0.518762}%
\pgfsetfillcolor{currentfill}%
\pgfsetfillopacity{0.700000}%
\pgfsetlinewidth{0.000000pt}%
\definecolor{currentstroke}{rgb}{0.000000,0.000000,0.000000}%
\pgfsetstrokecolor{currentstroke}%
\pgfsetdash{}{0pt}%
\pgfpathmoveto{\pgfqpoint{3.157357in}{2.674633in}}%
\pgfpathlineto{\pgfqpoint{3.170309in}{2.661703in}}%
\pgfpathlineto{\pgfqpoint{3.183256in}{2.649018in}}%
\pgfpathlineto{\pgfqpoint{3.196201in}{2.636576in}}%
\pgfpathlineto{\pgfqpoint{3.209143in}{2.624376in}}%
\pgfpathlineto{\pgfqpoint{3.216906in}{2.634681in}}%
\pgfpathlineto{\pgfqpoint{3.224663in}{2.645080in}}%
\pgfpathlineto{\pgfqpoint{3.232413in}{2.655573in}}%
\pgfpathlineto{\pgfqpoint{3.240157in}{2.666161in}}%
\pgfpathlineto{\pgfqpoint{3.227228in}{2.678416in}}%
\pgfpathlineto{\pgfqpoint{3.214296in}{2.690912in}}%
\pgfpathlineto{\pgfqpoint{3.201361in}{2.703652in}}%
\pgfpathlineto{\pgfqpoint{3.188423in}{2.716637in}}%
\pgfpathlineto{\pgfqpoint{3.180667in}{2.705983in}}%
\pgfpathlineto{\pgfqpoint{3.172903in}{2.695432in}}%
\pgfpathlineto{\pgfqpoint{3.165134in}{2.684982in}}%
\pgfpathlineto{\pgfqpoint{3.157357in}{2.674633in}}%
\pgfpathclose%
\pgfusepath{fill}%
\end{pgfscope}%
\begin{pgfscope}%
\pgfpathrectangle{\pgfqpoint{1.254980in}{0.150000in}}{\pgfqpoint{5.490039in}{5.490039in}}%
\pgfusepath{clip}%
\pgfsetbuttcap%
\pgfsetroundjoin%
\definecolor{currentfill}{rgb}{0.280255,0.165693,0.476498}%
\pgfsetfillcolor{currentfill}%
\pgfsetfillopacity{0.700000}%
\pgfsetlinewidth{0.000000pt}%
\definecolor{currentstroke}{rgb}{0.000000,0.000000,0.000000}%
\pgfsetstrokecolor{currentstroke}%
\pgfsetdash{}{0pt}%
\pgfpathmoveto{\pgfqpoint{3.878840in}{2.519580in}}%
\pgfpathlineto{\pgfqpoint{3.891785in}{2.515451in}}%
\pgfpathlineto{\pgfqpoint{3.904735in}{2.511510in}}%
\pgfpathlineto{\pgfqpoint{3.917690in}{2.507753in}}%
\pgfpathlineto{\pgfqpoint{3.930650in}{2.504182in}}%
\pgfpathlineto{\pgfqpoint{3.938194in}{2.514616in}}%
\pgfpathlineto{\pgfqpoint{3.945734in}{2.525101in}}%
\pgfpathlineto{\pgfqpoint{3.953269in}{2.535641in}}%
\pgfpathlineto{\pgfqpoint{3.960799in}{2.546238in}}%
\pgfpathlineto{\pgfqpoint{3.947847in}{2.550003in}}%
\pgfpathlineto{\pgfqpoint{3.934901in}{2.553954in}}%
\pgfpathlineto{\pgfqpoint{3.921959in}{2.558089in}}%
\pgfpathlineto{\pgfqpoint{3.909023in}{2.562411in}}%
\pgfpathlineto{\pgfqpoint{3.901484in}{2.551611in}}%
\pgfpathlineto{\pgfqpoint{3.893941in}{2.540874in}}%
\pgfpathlineto{\pgfqpoint{3.886393in}{2.530197in}}%
\pgfpathlineto{\pgfqpoint{3.878840in}{2.519580in}}%
\pgfpathclose%
\pgfusepath{fill}%
\end{pgfscope}%
\begin{pgfscope}%
\pgfpathrectangle{\pgfqpoint{1.254980in}{0.150000in}}{\pgfqpoint{5.490039in}{5.490039in}}%
\pgfusepath{clip}%
\pgfsetbuttcap%
\pgfsetroundjoin%
\definecolor{currentfill}{rgb}{0.206756,0.371758,0.553117}%
\pgfsetfillcolor{currentfill}%
\pgfsetfillopacity{0.700000}%
\pgfsetlinewidth{0.000000pt}%
\definecolor{currentstroke}{rgb}{0.000000,0.000000,0.000000}%
\pgfsetstrokecolor{currentstroke}%
\pgfsetdash{}{0pt}%
\pgfpathmoveto{\pgfqpoint{2.897295in}{2.988071in}}%
\pgfpathlineto{\pgfqpoint{2.910361in}{2.969769in}}%
\pgfpathlineto{\pgfqpoint{2.923418in}{2.951759in}}%
\pgfpathlineto{\pgfqpoint{2.936468in}{2.934039in}}%
\pgfpathlineto{\pgfqpoint{2.949510in}{2.916605in}}%
\pgfpathlineto{\pgfqpoint{2.957345in}{2.927129in}}%
\pgfpathlineto{\pgfqpoint{2.965172in}{2.937780in}}%
\pgfpathlineto{\pgfqpoint{2.972992in}{2.948561in}}%
\pgfpathlineto{\pgfqpoint{2.980804in}{2.959472in}}%
\pgfpathlineto{\pgfqpoint{2.967777in}{2.976959in}}%
\pgfpathlineto{\pgfqpoint{2.954743in}{2.994732in}}%
\pgfpathlineto{\pgfqpoint{2.941701in}{3.012795in}}%
\pgfpathlineto{\pgfqpoint{2.928651in}{3.031151in}}%
\pgfpathlineto{\pgfqpoint{2.920824in}{3.020176in}}%
\pgfpathlineto{\pgfqpoint{2.912989in}{3.009338in}}%
\pgfpathlineto{\pgfqpoint{2.905146in}{2.998637in}}%
\pgfpathlineto{\pgfqpoint{2.897295in}{2.988071in}}%
\pgfpathclose%
\pgfusepath{fill}%
\end{pgfscope}%
\begin{pgfscope}%
\pgfpathrectangle{\pgfqpoint{1.254980in}{0.150000in}}{\pgfqpoint{5.490039in}{5.490039in}}%
\pgfusepath{clip}%
\pgfsetbuttcap%
\pgfsetroundjoin%
\definecolor{currentfill}{rgb}{0.266580,0.228262,0.514349}%
\pgfsetfillcolor{currentfill}%
\pgfsetfillopacity{0.700000}%
\pgfsetlinewidth{0.000000pt}%
\definecolor{currentstroke}{rgb}{0.000000,0.000000,0.000000}%
\pgfsetstrokecolor{currentstroke}%
\pgfsetdash{}{0pt}%
\pgfpathmoveto{\pgfqpoint{4.258444in}{2.631129in}}%
\pgfpathlineto{\pgfqpoint{4.271479in}{2.629641in}}%
\pgfpathlineto{\pgfqpoint{4.284522in}{2.628325in}}%
\pgfpathlineto{\pgfqpoint{4.297572in}{2.627180in}}%
\pgfpathlineto{\pgfqpoint{4.310630in}{2.626206in}}%
\pgfpathlineto{\pgfqpoint{4.318058in}{2.636334in}}%
\pgfpathlineto{\pgfqpoint{4.325482in}{2.646527in}}%
\pgfpathlineto{\pgfqpoint{4.332902in}{2.656789in}}%
\pgfpathlineto{\pgfqpoint{4.340317in}{2.667125in}}%
\pgfpathlineto{\pgfqpoint{4.327268in}{2.668404in}}%
\pgfpathlineto{\pgfqpoint{4.314227in}{2.669854in}}%
\pgfpathlineto{\pgfqpoint{4.301194in}{2.671475in}}%
\pgfpathlineto{\pgfqpoint{4.288168in}{2.673267in}}%
\pgfpathlineto{\pgfqpoint{4.280743in}{2.662617in}}%
\pgfpathlineto{\pgfqpoint{4.273314in}{2.652046in}}%
\pgfpathlineto{\pgfqpoint{4.265881in}{2.641552in}}%
\pgfpathlineto{\pgfqpoint{4.258444in}{2.631129in}}%
\pgfpathclose%
\pgfusepath{fill}%
\end{pgfscope}%
\begin{pgfscope}%
\pgfpathrectangle{\pgfqpoint{1.254980in}{0.150000in}}{\pgfqpoint{5.490039in}{5.490039in}}%
\pgfusepath{clip}%
\pgfsetbuttcap%
\pgfsetroundjoin%
\definecolor{currentfill}{rgb}{0.281412,0.155834,0.469201}%
\pgfsetfillcolor{currentfill}%
\pgfsetfillopacity{0.700000}%
\pgfsetlinewidth{0.000000pt}%
\definecolor{currentstroke}{rgb}{0.000000,0.000000,0.000000}%
\pgfsetstrokecolor{currentstroke}%
\pgfsetdash{}{0pt}%
\pgfpathmoveto{\pgfqpoint{3.663018in}{2.497236in}}%
\pgfpathlineto{\pgfqpoint{3.675934in}{2.491101in}}%
\pgfpathlineto{\pgfqpoint{3.688853in}{2.485165in}}%
\pgfpathlineto{\pgfqpoint{3.701775in}{2.479427in}}%
\pgfpathlineto{\pgfqpoint{3.714701in}{2.473884in}}%
\pgfpathlineto{\pgfqpoint{3.722311in}{2.484369in}}%
\pgfpathlineto{\pgfqpoint{3.729915in}{2.494910in}}%
\pgfpathlineto{\pgfqpoint{3.737515in}{2.505509in}}%
\pgfpathlineto{\pgfqpoint{3.745109in}{2.516166in}}%
\pgfpathlineto{\pgfqpoint{3.732193in}{2.521848in}}%
\pgfpathlineto{\pgfqpoint{3.719280in}{2.527725in}}%
\pgfpathlineto{\pgfqpoint{3.706370in}{2.533800in}}%
\pgfpathlineto{\pgfqpoint{3.693463in}{2.540072in}}%
\pgfpathlineto{\pgfqpoint{3.685859in}{2.529266in}}%
\pgfpathlineto{\pgfqpoint{3.678251in}{2.518526in}}%
\pgfpathlineto{\pgfqpoint{3.670637in}{2.507850in}}%
\pgfpathlineto{\pgfqpoint{3.663018in}{2.497236in}}%
\pgfpathclose%
\pgfusepath{fill}%
\end{pgfscope}%
\begin{pgfscope}%
\pgfpathrectangle{\pgfqpoint{1.254980in}{0.150000in}}{\pgfqpoint{5.490039in}{5.490039in}}%
\pgfusepath{clip}%
\pgfsetbuttcap%
\pgfsetroundjoin%
\definecolor{currentfill}{rgb}{0.279574,0.170599,0.479997}%
\pgfsetfillcolor{currentfill}%
\pgfsetfillopacity{0.700000}%
\pgfsetlinewidth{0.000000pt}%
\definecolor{currentstroke}{rgb}{0.000000,0.000000,0.000000}%
\pgfsetstrokecolor{currentstroke}%
\pgfsetdash{}{0pt}%
\pgfpathmoveto{\pgfqpoint{3.395171in}{2.537294in}}%
\pgfpathlineto{\pgfqpoint{3.408084in}{2.528021in}}%
\pgfpathlineto{\pgfqpoint{3.420996in}{2.518966in}}%
\pgfpathlineto{\pgfqpoint{3.433909in}{2.510127in}}%
\pgfpathlineto{\pgfqpoint{3.446822in}{2.501505in}}%
\pgfpathlineto{\pgfqpoint{3.454514in}{2.511900in}}%
\pgfpathlineto{\pgfqpoint{3.462200in}{2.522366in}}%
\pgfpathlineto{\pgfqpoint{3.469880in}{2.532904in}}%
\pgfpathlineto{\pgfqpoint{3.477555in}{2.543514in}}%
\pgfpathlineto{\pgfqpoint{3.464652in}{2.552220in}}%
\pgfpathlineto{\pgfqpoint{3.451750in}{2.561141in}}%
\pgfpathlineto{\pgfqpoint{3.438848in}{2.570279in}}%
\pgfpathlineto{\pgfqpoint{3.425946in}{2.579635in}}%
\pgfpathlineto{\pgfqpoint{3.418261in}{2.568932in}}%
\pgfpathlineto{\pgfqpoint{3.410570in}{2.558308in}}%
\pgfpathlineto{\pgfqpoint{3.402873in}{2.547762in}}%
\pgfpathlineto{\pgfqpoint{3.395171in}{2.537294in}}%
\pgfpathclose%
\pgfusepath{fill}%
\end{pgfscope}%
\begin{pgfscope}%
\pgfpathrectangle{\pgfqpoint{1.254980in}{0.150000in}}{\pgfqpoint{5.490039in}{5.490039in}}%
\pgfusepath{clip}%
\pgfsetbuttcap%
\pgfsetroundjoin%
\definecolor{currentfill}{rgb}{0.165117,0.467423,0.558141}%
\pgfsetfillcolor{currentfill}%
\pgfsetfillopacity{0.700000}%
\pgfsetlinewidth{0.000000pt}%
\definecolor{currentstroke}{rgb}{0.000000,0.000000,0.000000}%
\pgfsetstrokecolor{currentstroke}%
\pgfsetdash{}{0pt}%
\pgfpathmoveto{\pgfqpoint{5.323065in}{3.190872in}}%
\pgfpathlineto{\pgfqpoint{5.336428in}{3.191565in}}%
\pgfpathlineto{\pgfqpoint{5.349803in}{3.192407in}}%
\pgfpathlineto{\pgfqpoint{5.363191in}{3.193400in}}%
\pgfpathlineto{\pgfqpoint{5.376590in}{3.194542in}}%
\pgfpathlineto{\pgfqpoint{5.383726in}{3.206059in}}%
\pgfpathlineto{\pgfqpoint{5.390866in}{3.217841in}}%
\pgfpathlineto{\pgfqpoint{5.398008in}{3.229896in}}%
\pgfpathlineto{\pgfqpoint{5.384626in}{3.229251in}}%
\pgfpathlineto{\pgfqpoint{5.371255in}{3.228756in}}%
\pgfpathlineto{\pgfqpoint{5.357897in}{3.228410in}}%
\pgfpathlineto{\pgfqpoint{5.344550in}{3.228214in}}%
\pgfpathlineto{\pgfqpoint{5.337385in}{3.215490in}}%
\pgfpathlineto{\pgfqpoint{5.330223in}{3.203045in}}%
\pgfpathlineto{\pgfqpoint{5.323065in}{3.190872in}}%
\pgfpathclose%
\pgfusepath{fill}%
\end{pgfscope}%
\begin{pgfscope}%
\pgfpathrectangle{\pgfqpoint{1.254980in}{0.150000in}}{\pgfqpoint{5.490039in}{5.490039in}}%
\pgfusepath{clip}%
\pgfsetbuttcap%
\pgfsetroundjoin%
\definecolor{currentfill}{rgb}{0.270595,0.214069,0.507052}%
\pgfsetfillcolor{currentfill}%
\pgfsetfillopacity{0.700000}%
\pgfsetlinewidth{0.000000pt}%
\definecolor{currentstroke}{rgb}{0.000000,0.000000,0.000000}%
\pgfsetstrokecolor{currentstroke}%
\pgfsetdash{}{0pt}%
\pgfpathmoveto{\pgfqpoint{3.209143in}{2.624376in}}%
\pgfpathlineto{\pgfqpoint{3.222083in}{2.612415in}}%
\pgfpathlineto{\pgfqpoint{3.235020in}{2.600693in}}%
\pgfpathlineto{\pgfqpoint{3.247954in}{2.589206in}}%
\pgfpathlineto{\pgfqpoint{3.260887in}{2.577954in}}%
\pgfpathlineto{\pgfqpoint{3.268637in}{2.588215in}}%
\pgfpathlineto{\pgfqpoint{3.276381in}{2.598562in}}%
\pgfpathlineto{\pgfqpoint{3.284119in}{2.608997in}}%
\pgfpathlineto{\pgfqpoint{3.291851in}{2.619520in}}%
\pgfpathlineto{\pgfqpoint{3.278930in}{2.630828in}}%
\pgfpathlineto{\pgfqpoint{3.266008in}{2.642369in}}%
\pgfpathlineto{\pgfqpoint{3.253084in}{2.654146in}}%
\pgfpathlineto{\pgfqpoint{3.240157in}{2.666161in}}%
\pgfpathlineto{\pgfqpoint{3.232413in}{2.655573in}}%
\pgfpathlineto{\pgfqpoint{3.224663in}{2.645080in}}%
\pgfpathlineto{\pgfqpoint{3.216906in}{2.634681in}}%
\pgfpathlineto{\pgfqpoint{3.209143in}{2.624376in}}%
\pgfpathclose%
\pgfusepath{fill}%
\end{pgfscope}%
\begin{pgfscope}%
\pgfpathrectangle{\pgfqpoint{1.254980in}{0.150000in}}{\pgfqpoint{5.490039in}{5.490039in}}%
\pgfusepath{clip}%
\pgfsetbuttcap%
\pgfsetroundjoin%
\definecolor{currentfill}{rgb}{0.271828,0.209303,0.504434}%
\pgfsetfillcolor{currentfill}%
\pgfsetfillopacity{0.700000}%
\pgfsetlinewidth{0.000000pt}%
\definecolor{currentstroke}{rgb}{0.000000,0.000000,0.000000}%
\pgfsetstrokecolor{currentstroke}%
\pgfsetdash{}{0pt}%
\pgfpathmoveto{\pgfqpoint{4.176549in}{2.596670in}}%
\pgfpathlineto{\pgfqpoint{4.189564in}{2.594765in}}%
\pgfpathlineto{\pgfqpoint{4.202586in}{2.593035in}}%
\pgfpathlineto{\pgfqpoint{4.215615in}{2.591478in}}%
\pgfpathlineto{\pgfqpoint{4.228651in}{2.590095in}}%
\pgfpathlineto{\pgfqpoint{4.236106in}{2.600262in}}%
\pgfpathlineto{\pgfqpoint{4.243556in}{2.610488in}}%
\pgfpathlineto{\pgfqpoint{4.251002in}{2.620776in}}%
\pgfpathlineto{\pgfqpoint{4.258444in}{2.631129in}}%
\pgfpathlineto{\pgfqpoint{4.245416in}{2.632790in}}%
\pgfpathlineto{\pgfqpoint{4.232396in}{2.634624in}}%
\pgfpathlineto{\pgfqpoint{4.219383in}{2.636631in}}%
\pgfpathlineto{\pgfqpoint{4.206377in}{2.638813in}}%
\pgfpathlineto{\pgfqpoint{4.198926in}{2.628172in}}%
\pgfpathlineto{\pgfqpoint{4.191472in}{2.617604in}}%
\pgfpathlineto{\pgfqpoint{4.184012in}{2.607104in}}%
\pgfpathlineto{\pgfqpoint{4.176549in}{2.596670in}}%
\pgfpathclose%
\pgfusepath{fill}%
\end{pgfscope}%
\begin{pgfscope}%
\pgfpathrectangle{\pgfqpoint{1.254980in}{0.150000in}}{\pgfqpoint{5.490039in}{5.490039in}}%
\pgfusepath{clip}%
\pgfsetbuttcap%
\pgfsetroundjoin%
\definecolor{currentfill}{rgb}{0.190631,0.407061,0.556089}%
\pgfsetfillcolor{currentfill}%
\pgfsetfillopacity{0.700000}%
\pgfsetlinewidth{0.000000pt}%
\definecolor{currentstroke}{rgb}{0.000000,0.000000,0.000000}%
\pgfsetstrokecolor{currentstroke}%
\pgfsetdash{}{0pt}%
\pgfpathmoveto{\pgfqpoint{2.844944in}{3.064259in}}%
\pgfpathlineto{\pgfqpoint{2.858046in}{3.044759in}}%
\pgfpathlineto{\pgfqpoint{2.871138in}{3.025563in}}%
\pgfpathlineto{\pgfqpoint{2.884221in}{3.006668in}}%
\pgfpathlineto{\pgfqpoint{2.897295in}{2.988071in}}%
\pgfpathlineto{\pgfqpoint{2.905146in}{2.998637in}}%
\pgfpathlineto{\pgfqpoint{2.912989in}{3.009338in}}%
\pgfpathlineto{\pgfqpoint{2.920824in}{3.020176in}}%
\pgfpathlineto{\pgfqpoint{2.928651in}{3.031151in}}%
\pgfpathlineto{\pgfqpoint{2.915593in}{3.049801in}}%
\pgfpathlineto{\pgfqpoint{2.902526in}{3.068750in}}%
\pgfpathlineto{\pgfqpoint{2.889450in}{3.087999in}}%
\pgfpathlineto{\pgfqpoint{2.876365in}{3.107552in}}%
\pgfpathlineto{\pgfqpoint{2.868522in}{3.096513in}}%
\pgfpathlineto{\pgfqpoint{2.860671in}{3.085618in}}%
\pgfpathlineto{\pgfqpoint{2.852812in}{3.074867in}}%
\pgfpathlineto{\pgfqpoint{2.844944in}{3.064259in}}%
\pgfpathclose%
\pgfusepath{fill}%
\end{pgfscope}%
\begin{pgfscope}%
\pgfpathrectangle{\pgfqpoint{1.254980in}{0.150000in}}{\pgfqpoint{5.490039in}{5.490039in}}%
\pgfusepath{clip}%
\pgfsetbuttcap%
\pgfsetroundjoin%
\definecolor{currentfill}{rgb}{0.281412,0.155834,0.469201}%
\pgfsetfillcolor{currentfill}%
\pgfsetfillopacity{0.700000}%
\pgfsetlinewidth{0.000000pt}%
\definecolor{currentstroke}{rgb}{0.000000,0.000000,0.000000}%
\pgfsetstrokecolor{currentstroke}%
\pgfsetdash{}{0pt}%
\pgfpathmoveto{\pgfqpoint{3.796812in}{2.495379in}}%
\pgfpathlineto{\pgfqpoint{3.809747in}{2.490662in}}%
\pgfpathlineto{\pgfqpoint{3.822687in}{2.486136in}}%
\pgfpathlineto{\pgfqpoint{3.835631in}{2.481798in}}%
\pgfpathlineto{\pgfqpoint{3.848580in}{2.477649in}}%
\pgfpathlineto{\pgfqpoint{3.856152in}{2.488055in}}%
\pgfpathlineto{\pgfqpoint{3.863720in}{2.498511in}}%
\pgfpathlineto{\pgfqpoint{3.871282in}{2.509018in}}%
\pgfpathlineto{\pgfqpoint{3.878840in}{2.519580in}}%
\pgfpathlineto{\pgfqpoint{3.865900in}{2.523896in}}%
\pgfpathlineto{\pgfqpoint{3.852964in}{2.528399in}}%
\pgfpathlineto{\pgfqpoint{3.840033in}{2.533092in}}%
\pgfpathlineto{\pgfqpoint{3.827106in}{2.537975in}}%
\pgfpathlineto{\pgfqpoint{3.819540in}{2.527237in}}%
\pgfpathlineto{\pgfqpoint{3.811969in}{2.516560in}}%
\pgfpathlineto{\pgfqpoint{3.804393in}{2.505941in}}%
\pgfpathlineto{\pgfqpoint{3.796812in}{2.495379in}}%
\pgfpathclose%
\pgfusepath{fill}%
\end{pgfscope}%
\begin{pgfscope}%
\pgfpathrectangle{\pgfqpoint{1.254980in}{0.150000in}}{\pgfqpoint{5.490039in}{5.490039in}}%
\pgfusepath{clip}%
\pgfsetbuttcap%
\pgfsetroundjoin%
\definecolor{currentfill}{rgb}{0.275191,0.194905,0.496005}%
\pgfsetfillcolor{currentfill}%
\pgfsetfillopacity{0.700000}%
\pgfsetlinewidth{0.000000pt}%
\definecolor{currentstroke}{rgb}{0.000000,0.000000,0.000000}%
\pgfsetstrokecolor{currentstroke}%
\pgfsetdash{}{0pt}%
\pgfpathmoveto{\pgfqpoint{4.094624in}{2.563903in}}%
\pgfpathlineto{\pgfqpoint{4.107620in}{2.561543in}}%
\pgfpathlineto{\pgfqpoint{4.120623in}{2.559361in}}%
\pgfpathlineto{\pgfqpoint{4.133633in}{2.557355in}}%
\pgfpathlineto{\pgfqpoint{4.146650in}{2.555525in}}%
\pgfpathlineto{\pgfqpoint{4.154131in}{2.565728in}}%
\pgfpathlineto{\pgfqpoint{4.161608in}{2.575985in}}%
\pgfpathlineto{\pgfqpoint{4.169081in}{2.586298in}}%
\pgfpathlineto{\pgfqpoint{4.176549in}{2.596670in}}%
\pgfpathlineto{\pgfqpoint{4.163541in}{2.598750in}}%
\pgfpathlineto{\pgfqpoint{4.150540in}{2.601005in}}%
\pgfpathlineto{\pgfqpoint{4.137545in}{2.603438in}}%
\pgfpathlineto{\pgfqpoint{4.124558in}{2.606047in}}%
\pgfpathlineto{\pgfqpoint{4.117081in}{2.595415in}}%
\pgfpathlineto{\pgfqpoint{4.109600in}{2.584849in}}%
\pgfpathlineto{\pgfqpoint{4.102114in}{2.574346in}}%
\pgfpathlineto{\pgfqpoint{4.094624in}{2.563903in}}%
\pgfpathclose%
\pgfusepath{fill}%
\end{pgfscope}%
\begin{pgfscope}%
\pgfpathrectangle{\pgfqpoint{1.254980in}{0.150000in}}{\pgfqpoint{5.490039in}{5.490039in}}%
\pgfusepath{clip}%
\pgfsetbuttcap%
\pgfsetroundjoin%
\definecolor{currentfill}{rgb}{0.275191,0.194905,0.496005}%
\pgfsetfillcolor{currentfill}%
\pgfsetfillopacity{0.700000}%
\pgfsetlinewidth{0.000000pt}%
\definecolor{currentstroke}{rgb}{0.000000,0.000000,0.000000}%
\pgfsetstrokecolor{currentstroke}%
\pgfsetdash{}{0pt}%
\pgfpathmoveto{\pgfqpoint{3.260887in}{2.577954in}}%
\pgfpathlineto{\pgfqpoint{3.273818in}{2.566934in}}%
\pgfpathlineto{\pgfqpoint{3.286748in}{2.556145in}}%
\pgfpathlineto{\pgfqpoint{3.299676in}{2.545586in}}%
\pgfpathlineto{\pgfqpoint{3.312603in}{2.535254in}}%
\pgfpathlineto{\pgfqpoint{3.320341in}{2.545470in}}%
\pgfpathlineto{\pgfqpoint{3.328072in}{2.555765in}}%
\pgfpathlineto{\pgfqpoint{3.335798in}{2.566142in}}%
\pgfpathlineto{\pgfqpoint{3.343518in}{2.576600in}}%
\pgfpathlineto{\pgfqpoint{3.330603in}{2.586987in}}%
\pgfpathlineto{\pgfqpoint{3.317687in}{2.597602in}}%
\pgfpathlineto{\pgfqpoint{3.304770in}{2.608446in}}%
\pgfpathlineto{\pgfqpoint{3.291851in}{2.619520in}}%
\pgfpathlineto{\pgfqpoint{3.284119in}{2.608997in}}%
\pgfpathlineto{\pgfqpoint{3.276381in}{2.598562in}}%
\pgfpathlineto{\pgfqpoint{3.268637in}{2.588215in}}%
\pgfpathlineto{\pgfqpoint{3.260887in}{2.577954in}}%
\pgfpathclose%
\pgfusepath{fill}%
\end{pgfscope}%
\begin{pgfscope}%
\pgfpathrectangle{\pgfqpoint{1.254980in}{0.150000in}}{\pgfqpoint{5.490039in}{5.490039in}}%
\pgfusepath{clip}%
\pgfsetbuttcap%
\pgfsetroundjoin%
\definecolor{currentfill}{rgb}{0.281887,0.150881,0.465405}%
\pgfsetfillcolor{currentfill}%
\pgfsetfillopacity{0.700000}%
\pgfsetlinewidth{0.000000pt}%
\definecolor{currentstroke}{rgb}{0.000000,0.000000,0.000000}%
\pgfsetstrokecolor{currentstroke}%
\pgfsetdash{}{0pt}%
\pgfpathmoveto{\pgfqpoint{3.580816in}{2.481460in}}%
\pgfpathlineto{\pgfqpoint{3.593731in}{2.474634in}}%
\pgfpathlineto{\pgfqpoint{3.606649in}{2.468010in}}%
\pgfpathlineto{\pgfqpoint{3.619569in}{2.461587in}}%
\pgfpathlineto{\pgfqpoint{3.632492in}{2.455365in}}%
\pgfpathlineto{\pgfqpoint{3.640131in}{2.465748in}}%
\pgfpathlineto{\pgfqpoint{3.647765in}{2.476186in}}%
\pgfpathlineto{\pgfqpoint{3.655394in}{2.486682in}}%
\pgfpathlineto{\pgfqpoint{3.663018in}{2.497236in}}%
\pgfpathlineto{\pgfqpoint{3.650105in}{2.503569in}}%
\pgfpathlineto{\pgfqpoint{3.637195in}{2.510102in}}%
\pgfpathlineto{\pgfqpoint{3.624286in}{2.516838in}}%
\pgfpathlineto{\pgfqpoint{3.611381in}{2.523775in}}%
\pgfpathlineto{\pgfqpoint{3.603747in}{2.513100in}}%
\pgfpathlineto{\pgfqpoint{3.596109in}{2.502490in}}%
\pgfpathlineto{\pgfqpoint{3.588465in}{2.491944in}}%
\pgfpathlineto{\pgfqpoint{3.580816in}{2.481460in}}%
\pgfpathclose%
\pgfusepath{fill}%
\end{pgfscope}%
\begin{pgfscope}%
\pgfpathrectangle{\pgfqpoint{1.254980in}{0.150000in}}{\pgfqpoint{5.490039in}{5.490039in}}%
\pgfusepath{clip}%
\pgfsetbuttcap%
\pgfsetroundjoin%
\definecolor{currentfill}{rgb}{0.280868,0.160771,0.472899}%
\pgfsetfillcolor{currentfill}%
\pgfsetfillopacity{0.700000}%
\pgfsetlinewidth{0.000000pt}%
\definecolor{currentstroke}{rgb}{0.000000,0.000000,0.000000}%
\pgfsetstrokecolor{currentstroke}%
\pgfsetdash{}{0pt}%
\pgfpathmoveto{\pgfqpoint{3.446822in}{2.501505in}}%
\pgfpathlineto{\pgfqpoint{3.459736in}{2.493096in}}%
\pgfpathlineto{\pgfqpoint{3.472651in}{2.484900in}}%
\pgfpathlineto{\pgfqpoint{3.485567in}{2.476916in}}%
\pgfpathlineto{\pgfqpoint{3.498484in}{2.469141in}}%
\pgfpathlineto{\pgfqpoint{3.506164in}{2.479464in}}%
\pgfpathlineto{\pgfqpoint{3.513840in}{2.489850in}}%
\pgfpathlineto{\pgfqpoint{3.521510in}{2.500300in}}%
\pgfpathlineto{\pgfqpoint{3.529174in}{2.510817in}}%
\pgfpathlineto{\pgfqpoint{3.516267in}{2.518675in}}%
\pgfpathlineto{\pgfqpoint{3.503362in}{2.526743in}}%
\pgfpathlineto{\pgfqpoint{3.490458in}{2.535022in}}%
\pgfpathlineto{\pgfqpoint{3.477555in}{2.543514in}}%
\pgfpathlineto{\pgfqpoint{3.469880in}{2.532904in}}%
\pgfpathlineto{\pgfqpoint{3.462200in}{2.522366in}}%
\pgfpathlineto{\pgfqpoint{3.454514in}{2.511900in}}%
\pgfpathlineto{\pgfqpoint{3.446822in}{2.501505in}}%
\pgfpathclose%
\pgfusepath{fill}%
\end{pgfscope}%
\begin{pgfscope}%
\pgfpathrectangle{\pgfqpoint{1.254980in}{0.150000in}}{\pgfqpoint{5.490039in}{5.490039in}}%
\pgfusepath{clip}%
\pgfsetbuttcap%
\pgfsetroundjoin%
\definecolor{currentfill}{rgb}{0.278012,0.180367,0.486697}%
\pgfsetfillcolor{currentfill}%
\pgfsetfillopacity{0.700000}%
\pgfsetlinewidth{0.000000pt}%
\definecolor{currentstroke}{rgb}{0.000000,0.000000,0.000000}%
\pgfsetstrokecolor{currentstroke}%
\pgfsetdash{}{0pt}%
\pgfpathmoveto{\pgfqpoint{4.012661in}{2.533007in}}%
\pgfpathlineto{\pgfqpoint{4.025641in}{2.530154in}}%
\pgfpathlineto{\pgfqpoint{4.038627in}{2.527480in}}%
\pgfpathlineto{\pgfqpoint{4.051619in}{2.524986in}}%
\pgfpathlineto{\pgfqpoint{4.064617in}{2.522671in}}%
\pgfpathlineto{\pgfqpoint{4.072126in}{2.532903in}}%
\pgfpathlineto{\pgfqpoint{4.079630in}{2.543184in}}%
\pgfpathlineto{\pgfqpoint{4.087129in}{2.553517in}}%
\pgfpathlineto{\pgfqpoint{4.094624in}{2.563903in}}%
\pgfpathlineto{\pgfqpoint{4.081634in}{2.566441in}}%
\pgfpathlineto{\pgfqpoint{4.068650in}{2.569157in}}%
\pgfpathlineto{\pgfqpoint{4.055673in}{2.572053in}}%
\pgfpathlineto{\pgfqpoint{4.042701in}{2.575128in}}%
\pgfpathlineto{\pgfqpoint{4.035198in}{2.564510in}}%
\pgfpathlineto{\pgfqpoint{4.027690in}{2.553952in}}%
\pgfpathlineto{\pgfqpoint{4.020178in}{2.543452in}}%
\pgfpathlineto{\pgfqpoint{4.012661in}{2.533007in}}%
\pgfpathclose%
\pgfusepath{fill}%
\end{pgfscope}%
\begin{pgfscope}%
\pgfpathrectangle{\pgfqpoint{1.254980in}{0.150000in}}{\pgfqpoint{5.490039in}{5.490039in}}%
\pgfusepath{clip}%
\pgfsetbuttcap%
\pgfsetroundjoin%
\definecolor{currentfill}{rgb}{0.220057,0.343307,0.549413}%
\pgfsetfillcolor{currentfill}%
\pgfsetfillopacity{0.700000}%
\pgfsetlinewidth{0.000000pt}%
\definecolor{currentstroke}{rgb}{0.000000,0.000000,0.000000}%
\pgfsetstrokecolor{currentstroke}%
\pgfsetdash{}{0pt}%
\pgfpathmoveto{\pgfqpoint{4.802337in}{2.868091in}}%
\pgfpathlineto{\pgfqpoint{4.815558in}{2.868971in}}%
\pgfpathlineto{\pgfqpoint{4.828789in}{2.870010in}}%
\pgfpathlineto{\pgfqpoint{4.842031in}{2.871207in}}%
\pgfpathlineto{\pgfqpoint{4.855284in}{2.872562in}}%
\pgfpathlineto{\pgfqpoint{4.862542in}{2.882192in}}%
\pgfpathlineto{\pgfqpoint{4.869798in}{2.891948in}}%
\pgfpathlineto{\pgfqpoint{4.877051in}{2.901835in}}%
\pgfpathlineto{\pgfqpoint{4.884302in}{2.911858in}}%
\pgfpathlineto{\pgfqpoint{4.871064in}{2.910976in}}%
\pgfpathlineto{\pgfqpoint{4.857837in}{2.910252in}}%
\pgfpathlineto{\pgfqpoint{4.844620in}{2.909686in}}%
\pgfpathlineto{\pgfqpoint{4.831414in}{2.909277in}}%
\pgfpathlineto{\pgfqpoint{4.824148in}{2.898771in}}%
\pgfpathlineto{\pgfqpoint{4.816880in}{2.888409in}}%
\pgfpathlineto{\pgfqpoint{4.809610in}{2.878183in}}%
\pgfpathlineto{\pgfqpoint{4.802337in}{2.868091in}}%
\pgfpathclose%
\pgfusepath{fill}%
\end{pgfscope}%
\begin{pgfscope}%
\pgfpathrectangle{\pgfqpoint{1.254980in}{0.150000in}}{\pgfqpoint{5.490039in}{5.490039in}}%
\pgfusepath{clip}%
\pgfsetbuttcap%
\pgfsetroundjoin%
\definecolor{currentfill}{rgb}{0.210503,0.363727,0.552206}%
\pgfsetfillcolor{currentfill}%
\pgfsetfillopacity{0.700000}%
\pgfsetlinewidth{0.000000pt}%
\definecolor{currentstroke}{rgb}{0.000000,0.000000,0.000000}%
\pgfsetstrokecolor{currentstroke}%
\pgfsetdash{}{0pt}%
\pgfpathmoveto{\pgfqpoint{4.884302in}{2.911858in}}%
\pgfpathlineto{\pgfqpoint{4.897551in}{2.912897in}}%
\pgfpathlineto{\pgfqpoint{4.910810in}{2.914093in}}%
\pgfpathlineto{\pgfqpoint{4.924081in}{2.915446in}}%
\pgfpathlineto{\pgfqpoint{4.937362in}{2.916956in}}%
\pgfpathlineto{\pgfqpoint{4.944595in}{2.926630in}}%
\pgfpathlineto{\pgfqpoint{4.951826in}{2.936446in}}%
\pgfpathlineto{\pgfqpoint{4.959055in}{2.946408in}}%
\pgfpathlineto{\pgfqpoint{4.966283in}{2.956523in}}%
\pgfpathlineto{\pgfqpoint{4.953017in}{2.955515in}}%
\pgfpathlineto{\pgfqpoint{4.939763in}{2.954663in}}%
\pgfpathlineto{\pgfqpoint{4.926519in}{2.953967in}}%
\pgfpathlineto{\pgfqpoint{4.913286in}{2.953428in}}%
\pgfpathlineto{\pgfqpoint{4.906043in}{2.942803in}}%
\pgfpathlineto{\pgfqpoint{4.898798in}{2.932336in}}%
\pgfpathlineto{\pgfqpoint{4.891551in}{2.922023in}}%
\pgfpathlineto{\pgfqpoint{4.884302in}{2.911858in}}%
\pgfpathclose%
\pgfusepath{fill}%
\end{pgfscope}%
\begin{pgfscope}%
\pgfpathrectangle{\pgfqpoint{1.254980in}{0.150000in}}{\pgfqpoint{5.490039in}{5.490039in}}%
\pgfusepath{clip}%
\pgfsetbuttcap%
\pgfsetroundjoin%
\definecolor{currentfill}{rgb}{0.227802,0.326594,0.546532}%
\pgfsetfillcolor{currentfill}%
\pgfsetfillopacity{0.700000}%
\pgfsetlinewidth{0.000000pt}%
\definecolor{currentstroke}{rgb}{0.000000,0.000000,0.000000}%
\pgfsetstrokecolor{currentstroke}%
\pgfsetdash{}{0pt}%
\pgfpathmoveto{\pgfqpoint{4.720383in}{2.825228in}}%
\pgfpathlineto{\pgfqpoint{4.733576in}{2.825916in}}%
\pgfpathlineto{\pgfqpoint{4.746780in}{2.826764in}}%
\pgfpathlineto{\pgfqpoint{4.759994in}{2.827771in}}%
\pgfpathlineto{\pgfqpoint{4.773218in}{2.828938in}}%
\pgfpathlineto{\pgfqpoint{4.780502in}{2.838554in}}%
\pgfpathlineto{\pgfqpoint{4.787783in}{2.848281in}}%
\pgfpathlineto{\pgfqpoint{4.795061in}{2.858125in}}%
\pgfpathlineto{\pgfqpoint{4.802337in}{2.868091in}}%
\pgfpathlineto{\pgfqpoint{4.789126in}{2.867369in}}%
\pgfpathlineto{\pgfqpoint{4.775926in}{2.866807in}}%
\pgfpathlineto{\pgfqpoint{4.762736in}{2.866404in}}%
\pgfpathlineto{\pgfqpoint{4.749556in}{2.866161in}}%
\pgfpathlineto{\pgfqpoint{4.742267in}{2.855740in}}%
\pgfpathlineto{\pgfqpoint{4.734975in}{2.845448in}}%
\pgfpathlineto{\pgfqpoint{4.727680in}{2.835279in}}%
\pgfpathlineto{\pgfqpoint{4.720383in}{2.825228in}}%
\pgfpathclose%
\pgfusepath{fill}%
\end{pgfscope}%
\begin{pgfscope}%
\pgfpathrectangle{\pgfqpoint{1.254980in}{0.150000in}}{\pgfqpoint{5.490039in}{5.490039in}}%
\pgfusepath{clip}%
\pgfsetbuttcap%
\pgfsetroundjoin%
\definecolor{currentfill}{rgb}{0.281887,0.150881,0.465405}%
\pgfsetfillcolor{currentfill}%
\pgfsetfillopacity{0.700000}%
\pgfsetlinewidth{0.000000pt}%
\definecolor{currentstroke}{rgb}{0.000000,0.000000,0.000000}%
\pgfsetstrokecolor{currentstroke}%
\pgfsetdash{}{0pt}%
\pgfpathmoveto{\pgfqpoint{3.714701in}{2.473884in}}%
\pgfpathlineto{\pgfqpoint{3.727630in}{2.468536in}}%
\pgfpathlineto{\pgfqpoint{3.740562in}{2.463382in}}%
\pgfpathlineto{\pgfqpoint{3.753499in}{2.458421in}}%
\pgfpathlineto{\pgfqpoint{3.766439in}{2.453652in}}%
\pgfpathlineto{\pgfqpoint{3.774040in}{2.464009in}}%
\pgfpathlineto{\pgfqpoint{3.781635in}{2.474415in}}%
\pgfpathlineto{\pgfqpoint{3.789226in}{2.484871in}}%
\pgfpathlineto{\pgfqpoint{3.796812in}{2.495379in}}%
\pgfpathlineto{\pgfqpoint{3.783880in}{2.500287in}}%
\pgfpathlineto{\pgfqpoint{3.770953in}{2.505387in}}%
\pgfpathlineto{\pgfqpoint{3.758029in}{2.510679in}}%
\pgfpathlineto{\pgfqpoint{3.745109in}{2.516166in}}%
\pgfpathlineto{\pgfqpoint{3.737515in}{2.505509in}}%
\pgfpathlineto{\pgfqpoint{3.729915in}{2.494910in}}%
\pgfpathlineto{\pgfqpoint{3.722311in}{2.484369in}}%
\pgfpathlineto{\pgfqpoint{3.714701in}{2.473884in}}%
\pgfpathclose%
\pgfusepath{fill}%
\end{pgfscope}%
\begin{pgfscope}%
\pgfpathrectangle{\pgfqpoint{1.254980in}{0.150000in}}{\pgfqpoint{5.490039in}{5.490039in}}%
\pgfusepath{clip}%
\pgfsetbuttcap%
\pgfsetroundjoin%
\definecolor{currentfill}{rgb}{0.203063,0.379716,0.553925}%
\pgfsetfillcolor{currentfill}%
\pgfsetfillopacity{0.700000}%
\pgfsetlinewidth{0.000000pt}%
\definecolor{currentstroke}{rgb}{0.000000,0.000000,0.000000}%
\pgfsetstrokecolor{currentstroke}%
\pgfsetdash{}{0pt}%
\pgfpathmoveto{\pgfqpoint{4.966283in}{2.956523in}}%
\pgfpathlineto{\pgfqpoint{4.979559in}{2.957687in}}%
\pgfpathlineto{\pgfqpoint{4.992846in}{2.959007in}}%
\pgfpathlineto{\pgfqpoint{5.006145in}{2.960482in}}%
\pgfpathlineto{\pgfqpoint{5.019455in}{2.962112in}}%
\pgfpathlineto{\pgfqpoint{5.026664in}{2.971867in}}%
\pgfpathlineto{\pgfqpoint{5.033872in}{2.981779in}}%
\pgfpathlineto{\pgfqpoint{5.041078in}{2.991855in}}%
\pgfpathlineto{\pgfqpoint{5.048283in}{3.002100in}}%
\pgfpathlineto{\pgfqpoint{5.034990in}{3.000999in}}%
\pgfpathlineto{\pgfqpoint{5.021708in}{3.000053in}}%
\pgfpathlineto{\pgfqpoint{5.008438in}{2.999262in}}%
\pgfpathlineto{\pgfqpoint{4.995178in}{2.998626in}}%
\pgfpathlineto{\pgfqpoint{4.987956in}{2.987842in}}%
\pgfpathlineto{\pgfqpoint{4.980733in}{2.977234in}}%
\pgfpathlineto{\pgfqpoint{4.973508in}{2.966796in}}%
\pgfpathlineto{\pgfqpoint{4.966283in}{2.956523in}}%
\pgfpathclose%
\pgfusepath{fill}%
\end{pgfscope}%
\begin{pgfscope}%
\pgfpathrectangle{\pgfqpoint{1.254980in}{0.150000in}}{\pgfqpoint{5.490039in}{5.490039in}}%
\pgfusepath{clip}%
\pgfsetbuttcap%
\pgfsetroundjoin%
\definecolor{currentfill}{rgb}{0.237441,0.305202,0.541921}%
\pgfsetfillcolor{currentfill}%
\pgfsetfillopacity{0.700000}%
\pgfsetlinewidth{0.000000pt}%
\definecolor{currentstroke}{rgb}{0.000000,0.000000,0.000000}%
\pgfsetstrokecolor{currentstroke}%
\pgfsetdash{}{0pt}%
\pgfpathmoveto{\pgfqpoint{4.638435in}{2.783299in}}%
\pgfpathlineto{\pgfqpoint{4.651602in}{2.783759in}}%
\pgfpathlineto{\pgfqpoint{4.664778in}{2.784381in}}%
\pgfpathlineto{\pgfqpoint{4.677964in}{2.785165in}}%
\pgfpathlineto{\pgfqpoint{4.691160in}{2.786109in}}%
\pgfpathlineto{\pgfqpoint{4.698471in}{2.795736in}}%
\pgfpathlineto{\pgfqpoint{4.705778in}{2.805462in}}%
\pgfpathlineto{\pgfqpoint{4.713082in}{2.815291in}}%
\pgfpathlineto{\pgfqpoint{4.720383in}{2.825228in}}%
\pgfpathlineto{\pgfqpoint{4.707199in}{2.824701in}}%
\pgfpathlineto{\pgfqpoint{4.694026in}{2.824335in}}%
\pgfpathlineto{\pgfqpoint{4.680862in}{2.824130in}}%
\pgfpathlineto{\pgfqpoint{4.667708in}{2.824087in}}%
\pgfpathlineto{\pgfqpoint{4.660394in}{2.813722in}}%
\pgfpathlineto{\pgfqpoint{4.653078in}{2.803473in}}%
\pgfpathlineto{\pgfqpoint{4.645758in}{2.793333in}}%
\pgfpathlineto{\pgfqpoint{4.638435in}{2.783299in}}%
\pgfpathclose%
\pgfusepath{fill}%
\end{pgfscope}%
\begin{pgfscope}%
\pgfpathrectangle{\pgfqpoint{1.254980in}{0.150000in}}{\pgfqpoint{5.490039in}{5.490039in}}%
\pgfusepath{clip}%
\pgfsetbuttcap%
\pgfsetroundjoin%
\definecolor{currentfill}{rgb}{0.194100,0.399323,0.555565}%
\pgfsetfillcolor{currentfill}%
\pgfsetfillopacity{0.700000}%
\pgfsetlinewidth{0.000000pt}%
\definecolor{currentstroke}{rgb}{0.000000,0.000000,0.000000}%
\pgfsetstrokecolor{currentstroke}%
\pgfsetdash{}{0pt}%
\pgfpathmoveto{\pgfqpoint{5.048283in}{3.002100in}}%
\pgfpathlineto{\pgfqpoint{5.061587in}{3.003356in}}%
\pgfpathlineto{\pgfqpoint{5.074902in}{3.004765in}}%
\pgfpathlineto{\pgfqpoint{5.088229in}{3.006329in}}%
\pgfpathlineto{\pgfqpoint{5.101568in}{3.008047in}}%
\pgfpathlineto{\pgfqpoint{5.108754in}{3.017922in}}%
\pgfpathlineto{\pgfqpoint{5.115939in}{3.027972in}}%
\pgfpathlineto{\pgfqpoint{5.123123in}{3.038204in}}%
\pgfpathlineto{\pgfqpoint{5.130307in}{3.048624in}}%
\pgfpathlineto{\pgfqpoint{5.116987in}{3.047464in}}%
\pgfpathlineto{\pgfqpoint{5.103679in}{3.046456in}}%
\pgfpathlineto{\pgfqpoint{5.090381in}{3.045603in}}%
\pgfpathlineto{\pgfqpoint{5.077095in}{3.044903in}}%
\pgfpathlineto{\pgfqpoint{5.069893in}{3.033917in}}%
\pgfpathlineto{\pgfqpoint{5.062690in}{3.023125in}}%
\pgfpathlineto{\pgfqpoint{5.055487in}{3.012522in}}%
\pgfpathlineto{\pgfqpoint{5.048283in}{3.002100in}}%
\pgfpathclose%
\pgfusepath{fill}%
\end{pgfscope}%
\begin{pgfscope}%
\pgfpathrectangle{\pgfqpoint{1.254980in}{0.150000in}}{\pgfqpoint{5.490039in}{5.490039in}}%
\pgfusepath{clip}%
\pgfsetbuttcap%
\pgfsetroundjoin%
\definecolor{currentfill}{rgb}{0.244972,0.287675,0.537260}%
\pgfsetfillcolor{currentfill}%
\pgfsetfillopacity{0.700000}%
\pgfsetlinewidth{0.000000pt}%
\definecolor{currentstroke}{rgb}{0.000000,0.000000,0.000000}%
\pgfsetstrokecolor{currentstroke}%
\pgfsetdash{}{0pt}%
\pgfpathmoveto{\pgfqpoint{4.556490in}{2.742352in}}%
\pgfpathlineto{\pgfqpoint{4.569630in}{2.742550in}}%
\pgfpathlineto{\pgfqpoint{4.582780in}{2.742911in}}%
\pgfpathlineto{\pgfqpoint{4.595939in}{2.743436in}}%
\pgfpathlineto{\pgfqpoint{4.609108in}{2.744124in}}%
\pgfpathlineto{\pgfqpoint{4.616445in}{2.753783in}}%
\pgfpathlineto{\pgfqpoint{4.623779in}{2.763529in}}%
\pgfpathlineto{\pgfqpoint{4.631109in}{2.773366in}}%
\pgfpathlineto{\pgfqpoint{4.638435in}{2.783299in}}%
\pgfpathlineto{\pgfqpoint{4.625278in}{2.783001in}}%
\pgfpathlineto{\pgfqpoint{4.612131in}{2.782866in}}%
\pgfpathlineto{\pgfqpoint{4.598993in}{2.782894in}}%
\pgfpathlineto{\pgfqpoint{4.585865in}{2.783085in}}%
\pgfpathlineto{\pgfqpoint{4.578526in}{2.772752in}}%
\pgfpathlineto{\pgfqpoint{4.571185in}{2.762522in}}%
\pgfpathlineto{\pgfqpoint{4.563839in}{2.752391in}}%
\pgfpathlineto{\pgfqpoint{4.556490in}{2.742352in}}%
\pgfpathclose%
\pgfusepath{fill}%
\end{pgfscope}%
\begin{pgfscope}%
\pgfpathrectangle{\pgfqpoint{1.254980in}{0.150000in}}{\pgfqpoint{5.490039in}{5.490039in}}%
\pgfusepath{clip}%
\pgfsetbuttcap%
\pgfsetroundjoin%
\definecolor{currentfill}{rgb}{0.278826,0.175490,0.483397}%
\pgfsetfillcolor{currentfill}%
\pgfsetfillopacity{0.700000}%
\pgfsetlinewidth{0.000000pt}%
\definecolor{currentstroke}{rgb}{0.000000,0.000000,0.000000}%
\pgfsetstrokecolor{currentstroke}%
\pgfsetdash{}{0pt}%
\pgfpathmoveto{\pgfqpoint{3.312603in}{2.535254in}}%
\pgfpathlineto{\pgfqpoint{3.325529in}{2.525148in}}%
\pgfpathlineto{\pgfqpoint{3.338454in}{2.515266in}}%
\pgfpathlineto{\pgfqpoint{3.351379in}{2.505607in}}%
\pgfpathlineto{\pgfqpoint{3.364303in}{2.496169in}}%
\pgfpathlineto{\pgfqpoint{3.372029in}{2.506340in}}%
\pgfpathlineto{\pgfqpoint{3.379749in}{2.516584in}}%
\pgfpathlineto{\pgfqpoint{3.387463in}{2.526902in}}%
\pgfpathlineto{\pgfqpoint{3.395171in}{2.537294in}}%
\pgfpathlineto{\pgfqpoint{3.382258in}{2.546787in}}%
\pgfpathlineto{\pgfqpoint{3.369345in}{2.556501in}}%
\pgfpathlineto{\pgfqpoint{3.356432in}{2.566438in}}%
\pgfpathlineto{\pgfqpoint{3.343518in}{2.576600in}}%
\pgfpathlineto{\pgfqpoint{3.335798in}{2.566142in}}%
\pgfpathlineto{\pgfqpoint{3.328072in}{2.555765in}}%
\pgfpathlineto{\pgfqpoint{3.320341in}{2.545470in}}%
\pgfpathlineto{\pgfqpoint{3.312603in}{2.535254in}}%
\pgfpathclose%
\pgfusepath{fill}%
\end{pgfscope}%
\begin{pgfscope}%
\pgfpathrectangle{\pgfqpoint{1.254980in}{0.150000in}}{\pgfqpoint{5.490039in}{5.490039in}}%
\pgfusepath{clip}%
\pgfsetbuttcap%
\pgfsetroundjoin%
\definecolor{currentfill}{rgb}{0.185556,0.418570,0.556753}%
\pgfsetfillcolor{currentfill}%
\pgfsetfillopacity{0.700000}%
\pgfsetlinewidth{0.000000pt}%
\definecolor{currentstroke}{rgb}{0.000000,0.000000,0.000000}%
\pgfsetstrokecolor{currentstroke}%
\pgfsetdash{}{0pt}%
\pgfpathmoveto{\pgfqpoint{5.130307in}{3.048624in}}%
\pgfpathlineto{\pgfqpoint{5.143639in}{3.049938in}}%
\pgfpathlineto{\pgfqpoint{5.156982in}{3.051405in}}%
\pgfpathlineto{\pgfqpoint{5.170337in}{3.053024in}}%
\pgfpathlineto{\pgfqpoint{5.183704in}{3.054796in}}%
\pgfpathlineto{\pgfqpoint{5.190868in}{3.064836in}}%
\pgfpathlineto{\pgfqpoint{5.198033in}{3.075071in}}%
\pgfpathlineto{\pgfqpoint{5.205197in}{3.085507in}}%
\pgfpathlineto{\pgfqpoint{5.212362in}{3.096151in}}%
\pgfpathlineto{\pgfqpoint{5.199015in}{3.094964in}}%
\pgfpathlineto{\pgfqpoint{5.185680in}{3.093929in}}%
\pgfpathlineto{\pgfqpoint{5.172356in}{3.093046in}}%
\pgfpathlineto{\pgfqpoint{5.159044in}{3.092316in}}%
\pgfpathlineto{\pgfqpoint{5.151859in}{3.081078in}}%
\pgfpathlineto{\pgfqpoint{5.144675in}{3.070055in}}%
\pgfpathlineto{\pgfqpoint{5.137491in}{3.059239in}}%
\pgfpathlineto{\pgfqpoint{5.130307in}{3.048624in}}%
\pgfpathclose%
\pgfusepath{fill}%
\end{pgfscope}%
\begin{pgfscope}%
\pgfpathrectangle{\pgfqpoint{1.254980in}{0.150000in}}{\pgfqpoint{5.490039in}{5.490039in}}%
\pgfusepath{clip}%
\pgfsetbuttcap%
\pgfsetroundjoin%
\definecolor{currentfill}{rgb}{0.279574,0.170599,0.479997}%
\pgfsetfillcolor{currentfill}%
\pgfsetfillopacity{0.700000}%
\pgfsetlinewidth{0.000000pt}%
\definecolor{currentstroke}{rgb}{0.000000,0.000000,0.000000}%
\pgfsetstrokecolor{currentstroke}%
\pgfsetdash{}{0pt}%
\pgfpathmoveto{\pgfqpoint{3.930650in}{2.504182in}}%
\pgfpathlineto{\pgfqpoint{3.943615in}{2.500795in}}%
\pgfpathlineto{\pgfqpoint{3.956586in}{2.497591in}}%
\pgfpathlineto{\pgfqpoint{3.969563in}{2.494569in}}%
\pgfpathlineto{\pgfqpoint{3.982546in}{2.491729in}}%
\pgfpathlineto{\pgfqpoint{3.990081in}{2.501979in}}%
\pgfpathlineto{\pgfqpoint{3.997613in}{2.512273in}}%
\pgfpathlineto{\pgfqpoint{4.005139in}{2.522615in}}%
\pgfpathlineto{\pgfqpoint{4.012661in}{2.533007in}}%
\pgfpathlineto{\pgfqpoint{3.999687in}{2.536042in}}%
\pgfpathlineto{\pgfqpoint{3.986719in}{2.539258in}}%
\pgfpathlineto{\pgfqpoint{3.973756in}{2.542656in}}%
\pgfpathlineto{\pgfqpoint{3.960799in}{2.546238in}}%
\pgfpathlineto{\pgfqpoint{3.953269in}{2.535641in}}%
\pgfpathlineto{\pgfqpoint{3.945734in}{2.525101in}}%
\pgfpathlineto{\pgfqpoint{3.938194in}{2.514616in}}%
\pgfpathlineto{\pgfqpoint{3.930650in}{2.504182in}}%
\pgfpathclose%
\pgfusepath{fill}%
\end{pgfscope}%
\begin{pgfscope}%
\pgfpathrectangle{\pgfqpoint{1.254980in}{0.150000in}}{\pgfqpoint{5.490039in}{5.490039in}}%
\pgfusepath{clip}%
\pgfsetbuttcap%
\pgfsetroundjoin%
\definecolor{currentfill}{rgb}{0.252194,0.269783,0.531579}%
\pgfsetfillcolor{currentfill}%
\pgfsetfillopacity{0.700000}%
\pgfsetlinewidth{0.000000pt}%
\definecolor{currentstroke}{rgb}{0.000000,0.000000,0.000000}%
\pgfsetstrokecolor{currentstroke}%
\pgfsetdash{}{0pt}%
\pgfpathmoveto{\pgfqpoint{4.474545in}{2.702457in}}%
\pgfpathlineto{\pgfqpoint{4.487659in}{2.702357in}}%
\pgfpathlineto{\pgfqpoint{4.500782in}{2.702423in}}%
\pgfpathlineto{\pgfqpoint{4.513915in}{2.702653in}}%
\pgfpathlineto{\pgfqpoint{4.527057in}{2.703049in}}%
\pgfpathlineto{\pgfqpoint{4.534421in}{2.712756in}}%
\pgfpathlineto{\pgfqpoint{4.541781in}{2.722539in}}%
\pgfpathlineto{\pgfqpoint{4.549138in}{2.732403in}}%
\pgfpathlineto{\pgfqpoint{4.556490in}{2.742352in}}%
\pgfpathlineto{\pgfqpoint{4.543360in}{2.742319in}}%
\pgfpathlineto{\pgfqpoint{4.530238in}{2.742449in}}%
\pgfpathlineto{\pgfqpoint{4.517125in}{2.742745in}}%
\pgfpathlineto{\pgfqpoint{4.504022in}{2.743206in}}%
\pgfpathlineto{\pgfqpoint{4.496658in}{2.732886in}}%
\pgfpathlineto{\pgfqpoint{4.489291in}{2.722657in}}%
\pgfpathlineto{\pgfqpoint{4.481920in}{2.712515in}}%
\pgfpathlineto{\pgfqpoint{4.474545in}{2.702457in}}%
\pgfpathclose%
\pgfusepath{fill}%
\end{pgfscope}%
\begin{pgfscope}%
\pgfpathrectangle{\pgfqpoint{1.254980in}{0.150000in}}{\pgfqpoint{5.490039in}{5.490039in}}%
\pgfusepath{clip}%
\pgfsetbuttcap%
\pgfsetroundjoin%
\definecolor{currentfill}{rgb}{0.250425,0.274290,0.533103}%
\pgfsetfillcolor{currentfill}%
\pgfsetfillopacity{0.700000}%
\pgfsetlinewidth{0.000000pt}%
\definecolor{currentstroke}{rgb}{0.000000,0.000000,0.000000}%
\pgfsetstrokecolor{currentstroke}%
\pgfsetdash{}{0pt}%
\pgfpathmoveto{\pgfqpoint{3.022315in}{2.746530in}}%
\pgfpathlineto{\pgfqpoint{3.035315in}{2.731589in}}%
\pgfpathlineto{\pgfqpoint{3.048310in}{2.716910in}}%
\pgfpathlineto{\pgfqpoint{3.061301in}{2.702491in}}%
\pgfpathlineto{\pgfqpoint{3.074286in}{2.688330in}}%
\pgfpathlineto{\pgfqpoint{3.082104in}{2.698303in}}%
\pgfpathlineto{\pgfqpoint{3.089915in}{2.708379in}}%
\pgfpathlineto{\pgfqpoint{3.097719in}{2.718561in}}%
\pgfpathlineto{\pgfqpoint{3.105516in}{2.728848in}}%
\pgfpathlineto{\pgfqpoint{3.092545in}{2.743035in}}%
\pgfpathlineto{\pgfqpoint{3.079570in}{2.757480in}}%
\pgfpathlineto{\pgfqpoint{3.066590in}{2.772185in}}%
\pgfpathlineto{\pgfqpoint{3.053605in}{2.787152in}}%
\pgfpathlineto{\pgfqpoint{3.045793in}{2.776828in}}%
\pgfpathlineto{\pgfqpoint{3.037974in}{2.766617in}}%
\pgfpathlineto{\pgfqpoint{3.030148in}{2.756518in}}%
\pgfpathlineto{\pgfqpoint{3.022315in}{2.746530in}}%
\pgfpathclose%
\pgfusepath{fill}%
\end{pgfscope}%
\begin{pgfscope}%
\pgfpathrectangle{\pgfqpoint{1.254980in}{0.150000in}}{\pgfqpoint{5.490039in}{5.490039in}}%
\pgfusepath{clip}%
\pgfsetbuttcap%
\pgfsetroundjoin%
\definecolor{currentfill}{rgb}{0.237441,0.305202,0.541921}%
\pgfsetfillcolor{currentfill}%
\pgfsetfillopacity{0.700000}%
\pgfsetlinewidth{0.000000pt}%
\definecolor{currentstroke}{rgb}{0.000000,0.000000,0.000000}%
\pgfsetstrokecolor{currentstroke}%
\pgfsetdash{}{0pt}%
\pgfpathmoveto{\pgfqpoint{2.970256in}{2.808962in}}%
\pgfpathlineto{\pgfqpoint{2.983280in}{2.792949in}}%
\pgfpathlineto{\pgfqpoint{2.996298in}{2.777207in}}%
\pgfpathlineto{\pgfqpoint{3.009309in}{2.761735in}}%
\pgfpathlineto{\pgfqpoint{3.022315in}{2.746530in}}%
\pgfpathlineto{\pgfqpoint{3.030148in}{2.756518in}}%
\pgfpathlineto{\pgfqpoint{3.037974in}{2.766617in}}%
\pgfpathlineto{\pgfqpoint{3.045793in}{2.776828in}}%
\pgfpathlineto{\pgfqpoint{3.053605in}{2.787152in}}%
\pgfpathlineto{\pgfqpoint{3.040614in}{2.802383in}}%
\pgfpathlineto{\pgfqpoint{3.027618in}{2.817881in}}%
\pgfpathlineto{\pgfqpoint{3.014616in}{2.833648in}}%
\pgfpathlineto{\pgfqpoint{3.001608in}{2.849687in}}%
\pgfpathlineto{\pgfqpoint{2.993781in}{2.839327in}}%
\pgfpathlineto{\pgfqpoint{2.985947in}{2.829087in}}%
\pgfpathlineto{\pgfqpoint{2.978105in}{2.818965in}}%
\pgfpathlineto{\pgfqpoint{2.970256in}{2.808962in}}%
\pgfpathclose%
\pgfusepath{fill}%
\end{pgfscope}%
\begin{pgfscope}%
\pgfpathrectangle{\pgfqpoint{1.254980in}{0.150000in}}{\pgfqpoint{5.490039in}{5.490039in}}%
\pgfusepath{clip}%
\pgfsetbuttcap%
\pgfsetroundjoin%
\definecolor{currentfill}{rgb}{0.177423,0.437527,0.557565}%
\pgfsetfillcolor{currentfill}%
\pgfsetfillopacity{0.700000}%
\pgfsetlinewidth{0.000000pt}%
\definecolor{currentstroke}{rgb}{0.000000,0.000000,0.000000}%
\pgfsetstrokecolor{currentstroke}%
\pgfsetdash{}{0pt}%
\pgfpathmoveto{\pgfqpoint{5.212362in}{3.096151in}}%
\pgfpathlineto{\pgfqpoint{5.225721in}{3.097490in}}%
\pgfpathlineto{\pgfqpoint{5.239092in}{3.098981in}}%
\pgfpathlineto{\pgfqpoint{5.252474in}{3.100623in}}%
\pgfpathlineto{\pgfqpoint{5.265869in}{3.102417in}}%
\pgfpathlineto{\pgfqpoint{5.273014in}{3.112673in}}%
\pgfpathlineto{\pgfqpoint{5.280159in}{3.123144in}}%
\pgfpathlineto{\pgfqpoint{5.287306in}{3.133836in}}%
\pgfpathlineto{\pgfqpoint{5.294454in}{3.144757in}}%
\pgfpathlineto{\pgfqpoint{5.281081in}{3.143576in}}%
\pgfpathlineto{\pgfqpoint{5.267719in}{3.142546in}}%
\pgfpathlineto{\pgfqpoint{5.254369in}{3.141667in}}%
\pgfpathlineto{\pgfqpoint{5.241031in}{3.140940in}}%
\pgfpathlineto{\pgfqpoint{5.233862in}{3.129396in}}%
\pgfpathlineto{\pgfqpoint{5.226694in}{3.118089in}}%
\pgfpathlineto{\pgfqpoint{5.219528in}{3.107009in}}%
\pgfpathlineto{\pgfqpoint{5.212362in}{3.096151in}}%
\pgfpathclose%
\pgfusepath{fill}%
\end{pgfscope}%
\begin{pgfscope}%
\pgfpathrectangle{\pgfqpoint{1.254980in}{0.150000in}}{\pgfqpoint{5.490039in}{5.490039in}}%
\pgfusepath{clip}%
\pgfsetbuttcap%
\pgfsetroundjoin%
\definecolor{currentfill}{rgb}{0.258965,0.251537,0.524736}%
\pgfsetfillcolor{currentfill}%
\pgfsetfillopacity{0.700000}%
\pgfsetlinewidth{0.000000pt}%
\definecolor{currentstroke}{rgb}{0.000000,0.000000,0.000000}%
\pgfsetstrokecolor{currentstroke}%
\pgfsetdash{}{0pt}%
\pgfpathmoveto{\pgfqpoint{4.392593in}{2.663705in}}%
\pgfpathlineto{\pgfqpoint{4.405683in}{2.663271in}}%
\pgfpathlineto{\pgfqpoint{4.418781in}{2.663005in}}%
\pgfpathlineto{\pgfqpoint{4.431888in}{2.662906in}}%
\pgfpathlineto{\pgfqpoint{4.445004in}{2.662974in}}%
\pgfpathlineto{\pgfqpoint{4.452395in}{2.672740in}}%
\pgfpathlineto{\pgfqpoint{4.459783in}{2.682574in}}%
\pgfpathlineto{\pgfqpoint{4.467166in}{2.692478in}}%
\pgfpathlineto{\pgfqpoint{4.474545in}{2.702457in}}%
\pgfpathlineto{\pgfqpoint{4.461439in}{2.702723in}}%
\pgfpathlineto{\pgfqpoint{4.448342in}{2.703156in}}%
\pgfpathlineto{\pgfqpoint{4.435254in}{2.703755in}}%
\pgfpathlineto{\pgfqpoint{4.422174in}{2.704522in}}%
\pgfpathlineto{\pgfqpoint{4.414785in}{2.694199in}}%
\pgfpathlineto{\pgfqpoint{4.407392in}{2.683958in}}%
\pgfpathlineto{\pgfqpoint{4.399994in}{2.673794in}}%
\pgfpathlineto{\pgfqpoint{4.392593in}{2.663705in}}%
\pgfpathclose%
\pgfusepath{fill}%
\end{pgfscope}%
\begin{pgfscope}%
\pgfpathrectangle{\pgfqpoint{1.254980in}{0.150000in}}{\pgfqpoint{5.490039in}{5.490039in}}%
\pgfusepath{clip}%
\pgfsetbuttcap%
\pgfsetroundjoin%
\definecolor{currentfill}{rgb}{0.258965,0.251537,0.524736}%
\pgfsetfillcolor{currentfill}%
\pgfsetfillopacity{0.700000}%
\pgfsetlinewidth{0.000000pt}%
\definecolor{currentstroke}{rgb}{0.000000,0.000000,0.000000}%
\pgfsetstrokecolor{currentstroke}%
\pgfsetdash{}{0pt}%
\pgfpathmoveto{\pgfqpoint{3.074286in}{2.688330in}}%
\pgfpathlineto{\pgfqpoint{3.087267in}{2.674425in}}%
\pgfpathlineto{\pgfqpoint{3.100244in}{2.660773in}}%
\pgfpathlineto{\pgfqpoint{3.113216in}{2.647372in}}%
\pgfpathlineto{\pgfqpoint{3.126185in}{2.634221in}}%
\pgfpathlineto{\pgfqpoint{3.133988in}{2.644178in}}%
\pgfpathlineto{\pgfqpoint{3.141785in}{2.654231in}}%
\pgfpathlineto{\pgfqpoint{3.149574in}{2.664383in}}%
\pgfpathlineto{\pgfqpoint{3.157357in}{2.674633in}}%
\pgfpathlineto{\pgfqpoint{3.144403in}{2.687811in}}%
\pgfpathlineto{\pgfqpoint{3.131445in}{2.701238in}}%
\pgfpathlineto{\pgfqpoint{3.118482in}{2.714916in}}%
\pgfpathlineto{\pgfqpoint{3.105516in}{2.728848in}}%
\pgfpathlineto{\pgfqpoint{3.097719in}{2.718561in}}%
\pgfpathlineto{\pgfqpoint{3.089915in}{2.708379in}}%
\pgfpathlineto{\pgfqpoint{3.082104in}{2.698303in}}%
\pgfpathlineto{\pgfqpoint{3.074286in}{2.688330in}}%
\pgfpathclose%
\pgfusepath{fill}%
\end{pgfscope}%
\begin{pgfscope}%
\pgfpathrectangle{\pgfqpoint{1.254980in}{0.150000in}}{\pgfqpoint{5.490039in}{5.490039in}}%
\pgfusepath{clip}%
\pgfsetbuttcap%
\pgfsetroundjoin%
\definecolor{currentfill}{rgb}{0.223925,0.334994,0.548053}%
\pgfsetfillcolor{currentfill}%
\pgfsetfillopacity{0.700000}%
\pgfsetlinewidth{0.000000pt}%
\definecolor{currentstroke}{rgb}{0.000000,0.000000,0.000000}%
\pgfsetstrokecolor{currentstroke}%
\pgfsetdash{}{0pt}%
\pgfpathmoveto{\pgfqpoint{2.918094in}{2.875778in}}%
\pgfpathlineto{\pgfqpoint{2.931145in}{2.858654in}}%
\pgfpathlineto{\pgfqpoint{2.944189in}{2.841812in}}%
\pgfpathlineto{\pgfqpoint{2.957226in}{2.825249in}}%
\pgfpathlineto{\pgfqpoint{2.970256in}{2.808962in}}%
\pgfpathlineto{\pgfqpoint{2.978105in}{2.818965in}}%
\pgfpathlineto{\pgfqpoint{2.985947in}{2.829087in}}%
\pgfpathlineto{\pgfqpoint{2.993781in}{2.839327in}}%
\pgfpathlineto{\pgfqpoint{3.001608in}{2.849687in}}%
\pgfpathlineto{\pgfqpoint{2.988594in}{2.865999in}}%
\pgfpathlineto{\pgfqpoint{2.975573in}{2.882588in}}%
\pgfpathlineto{\pgfqpoint{2.962545in}{2.899456in}}%
\pgfpathlineto{\pgfqpoint{2.949510in}{2.916605in}}%
\pgfpathlineto{\pgfqpoint{2.941668in}{2.906209in}}%
\pgfpathlineto{\pgfqpoint{2.933818in}{2.895940in}}%
\pgfpathlineto{\pgfqpoint{2.925960in}{2.885797in}}%
\pgfpathlineto{\pgfqpoint{2.918094in}{2.875778in}}%
\pgfpathclose%
\pgfusepath{fill}%
\end{pgfscope}%
\begin{pgfscope}%
\pgfpathrectangle{\pgfqpoint{1.254980in}{0.150000in}}{\pgfqpoint{5.490039in}{5.490039in}}%
\pgfusepath{clip}%
\pgfsetbuttcap%
\pgfsetroundjoin%
\definecolor{currentfill}{rgb}{0.168126,0.459988,0.558082}%
\pgfsetfillcolor{currentfill}%
\pgfsetfillopacity{0.700000}%
\pgfsetlinewidth{0.000000pt}%
\definecolor{currentstroke}{rgb}{0.000000,0.000000,0.000000}%
\pgfsetstrokecolor{currentstroke}%
\pgfsetdash{}{0pt}%
\pgfpathmoveto{\pgfqpoint{5.294454in}{3.144757in}}%
\pgfpathlineto{\pgfqpoint{5.307839in}{3.146089in}}%
\pgfpathlineto{\pgfqpoint{5.321237in}{3.147572in}}%
\pgfpathlineto{\pgfqpoint{5.334647in}{3.149205in}}%
\pgfpathlineto{\pgfqpoint{5.348069in}{3.150988in}}%
\pgfpathlineto{\pgfqpoint{5.355196in}{3.161515in}}%
\pgfpathlineto{\pgfqpoint{5.362325in}{3.172278in}}%
\pgfpathlineto{\pgfqpoint{5.369457in}{3.183285in}}%
\pgfpathlineto{\pgfqpoint{5.376590in}{3.194542in}}%
\pgfpathlineto{\pgfqpoint{5.363191in}{3.193400in}}%
\pgfpathlineto{\pgfqpoint{5.349803in}{3.192407in}}%
\pgfpathlineto{\pgfqpoint{5.336428in}{3.191565in}}%
\pgfpathlineto{\pgfqpoint{5.323065in}{3.190872in}}%
\pgfpathlineto{\pgfqpoint{5.315909in}{3.178964in}}%
\pgfpathlineto{\pgfqpoint{5.308755in}{3.167314in}}%
\pgfpathlineto{\pgfqpoint{5.301604in}{3.155914in}}%
\pgfpathlineto{\pgfqpoint{5.294454in}{3.144757in}}%
\pgfpathclose%
\pgfusepath{fill}%
\end{pgfscope}%
\begin{pgfscope}%
\pgfpathrectangle{\pgfqpoint{1.254980in}{0.150000in}}{\pgfqpoint{5.490039in}{5.490039in}}%
\pgfusepath{clip}%
\pgfsetbuttcap%
\pgfsetroundjoin%
\definecolor{currentfill}{rgb}{0.265145,0.232956,0.516599}%
\pgfsetfillcolor{currentfill}%
\pgfsetfillopacity{0.700000}%
\pgfsetlinewidth{0.000000pt}%
\definecolor{currentstroke}{rgb}{0.000000,0.000000,0.000000}%
\pgfsetstrokecolor{currentstroke}%
\pgfsetdash{}{0pt}%
\pgfpathmoveto{\pgfqpoint{4.310630in}{2.626206in}}%
\pgfpathlineto{\pgfqpoint{4.323697in}{2.625403in}}%
\pgfpathlineto{\pgfqpoint{4.336771in}{2.624769in}}%
\pgfpathlineto{\pgfqpoint{4.349854in}{2.624305in}}%
\pgfpathlineto{\pgfqpoint{4.362945in}{2.624009in}}%
\pgfpathlineto{\pgfqpoint{4.370363in}{2.633841in}}%
\pgfpathlineto{\pgfqpoint{4.377777in}{2.643732in}}%
\pgfpathlineto{\pgfqpoint{4.385187in}{2.653685in}}%
\pgfpathlineto{\pgfqpoint{4.392593in}{2.663705in}}%
\pgfpathlineto{\pgfqpoint{4.379512in}{2.664306in}}%
\pgfpathlineto{\pgfqpoint{4.366439in}{2.665077in}}%
\pgfpathlineto{\pgfqpoint{4.353374in}{2.666016in}}%
\pgfpathlineto{\pgfqpoint{4.340317in}{2.667125in}}%
\pgfpathlineto{\pgfqpoint{4.332902in}{2.656789in}}%
\pgfpathlineto{\pgfqpoint{4.325482in}{2.646527in}}%
\pgfpathlineto{\pgfqpoint{4.318058in}{2.636334in}}%
\pgfpathlineto{\pgfqpoint{4.310630in}{2.626206in}}%
\pgfpathclose%
\pgfusepath{fill}%
\end{pgfscope}%
\begin{pgfscope}%
\pgfpathrectangle{\pgfqpoint{1.254980in}{0.150000in}}{\pgfqpoint{5.490039in}{5.490039in}}%
\pgfusepath{clip}%
\pgfsetbuttcap%
\pgfsetroundjoin%
\definecolor{currentfill}{rgb}{0.281887,0.150881,0.465405}%
\pgfsetfillcolor{currentfill}%
\pgfsetfillopacity{0.700000}%
\pgfsetlinewidth{0.000000pt}%
\definecolor{currentstroke}{rgb}{0.000000,0.000000,0.000000}%
\pgfsetstrokecolor{currentstroke}%
\pgfsetdash{}{0pt}%
\pgfpathmoveto{\pgfqpoint{3.498484in}{2.469141in}}%
\pgfpathlineto{\pgfqpoint{3.511402in}{2.461575in}}%
\pgfpathlineto{\pgfqpoint{3.524322in}{2.454217in}}%
\pgfpathlineto{\pgfqpoint{3.537243in}{2.447065in}}%
\pgfpathlineto{\pgfqpoint{3.550166in}{2.440119in}}%
\pgfpathlineto{\pgfqpoint{3.557837in}{2.450368in}}%
\pgfpathlineto{\pgfqpoint{3.565502in}{2.460674in}}%
\pgfpathlineto{\pgfqpoint{3.573162in}{2.471037in}}%
\pgfpathlineto{\pgfqpoint{3.580816in}{2.481460in}}%
\pgfpathlineto{\pgfqpoint{3.567903in}{2.488491in}}%
\pgfpathlineto{\pgfqpoint{3.554992in}{2.495726in}}%
\pgfpathlineto{\pgfqpoint{3.542082in}{2.503168in}}%
\pgfpathlineto{\pgfqpoint{3.529174in}{2.510817in}}%
\pgfpathlineto{\pgfqpoint{3.521510in}{2.500300in}}%
\pgfpathlineto{\pgfqpoint{3.513840in}{2.489850in}}%
\pgfpathlineto{\pgfqpoint{3.506164in}{2.479464in}}%
\pgfpathlineto{\pgfqpoint{3.498484in}{2.469141in}}%
\pgfpathclose%
\pgfusepath{fill}%
\end{pgfscope}%
\begin{pgfscope}%
\pgfpathrectangle{\pgfqpoint{1.254980in}{0.150000in}}{\pgfqpoint{5.490039in}{5.490039in}}%
\pgfusepath{clip}%
\pgfsetbuttcap%
\pgfsetroundjoin%
\definecolor{currentfill}{rgb}{0.266580,0.228262,0.514349}%
\pgfsetfillcolor{currentfill}%
\pgfsetfillopacity{0.700000}%
\pgfsetlinewidth{0.000000pt}%
\definecolor{currentstroke}{rgb}{0.000000,0.000000,0.000000}%
\pgfsetstrokecolor{currentstroke}%
\pgfsetdash{}{0pt}%
\pgfpathmoveto{\pgfqpoint{3.126185in}{2.634221in}}%
\pgfpathlineto{\pgfqpoint{3.139150in}{2.621318in}}%
\pgfpathlineto{\pgfqpoint{3.152112in}{2.608659in}}%
\pgfpathlineto{\pgfqpoint{3.165071in}{2.596245in}}%
\pgfpathlineto{\pgfqpoint{3.178026in}{2.584072in}}%
\pgfpathlineto{\pgfqpoint{3.185815in}{2.594012in}}%
\pgfpathlineto{\pgfqpoint{3.193598in}{2.604042in}}%
\pgfpathlineto{\pgfqpoint{3.201374in}{2.614163in}}%
\pgfpathlineto{\pgfqpoint{3.209143in}{2.624376in}}%
\pgfpathlineto{\pgfqpoint{3.196201in}{2.636576in}}%
\pgfpathlineto{\pgfqpoint{3.183256in}{2.649018in}}%
\pgfpathlineto{\pgfqpoint{3.170309in}{2.661703in}}%
\pgfpathlineto{\pgfqpoint{3.157357in}{2.674633in}}%
\pgfpathlineto{\pgfqpoint{3.149574in}{2.664383in}}%
\pgfpathlineto{\pgfqpoint{3.141785in}{2.654231in}}%
\pgfpathlineto{\pgfqpoint{3.133988in}{2.644178in}}%
\pgfpathlineto{\pgfqpoint{3.126185in}{2.634221in}}%
\pgfpathclose%
\pgfusepath{fill}%
\end{pgfscope}%
\begin{pgfscope}%
\pgfpathrectangle{\pgfqpoint{1.254980in}{0.150000in}}{\pgfqpoint{5.490039in}{5.490039in}}%
\pgfusepath{clip}%
\pgfsetbuttcap%
\pgfsetroundjoin%
\definecolor{currentfill}{rgb}{0.281412,0.155834,0.469201}%
\pgfsetfillcolor{currentfill}%
\pgfsetfillopacity{0.700000}%
\pgfsetlinewidth{0.000000pt}%
\definecolor{currentstroke}{rgb}{0.000000,0.000000,0.000000}%
\pgfsetstrokecolor{currentstroke}%
\pgfsetdash{}{0pt}%
\pgfpathmoveto{\pgfqpoint{3.848580in}{2.477649in}}%
\pgfpathlineto{\pgfqpoint{3.861534in}{2.473687in}}%
\pgfpathlineto{\pgfqpoint{3.874492in}{2.469912in}}%
\pgfpathlineto{\pgfqpoint{3.887456in}{2.466323in}}%
\pgfpathlineto{\pgfqpoint{3.900425in}{2.462919in}}%
\pgfpathlineto{\pgfqpoint{3.907988in}{2.473168in}}%
\pgfpathlineto{\pgfqpoint{3.915547in}{2.483460in}}%
\pgfpathlineto{\pgfqpoint{3.923101in}{2.493798in}}%
\pgfpathlineto{\pgfqpoint{3.930650in}{2.504182in}}%
\pgfpathlineto{\pgfqpoint{3.917690in}{2.507753in}}%
\pgfpathlineto{\pgfqpoint{3.904735in}{2.511510in}}%
\pgfpathlineto{\pgfqpoint{3.891785in}{2.515451in}}%
\pgfpathlineto{\pgfqpoint{3.878840in}{2.519580in}}%
\pgfpathlineto{\pgfqpoint{3.871282in}{2.509018in}}%
\pgfpathlineto{\pgfqpoint{3.863720in}{2.498511in}}%
\pgfpathlineto{\pgfqpoint{3.856152in}{2.488055in}}%
\pgfpathlineto{\pgfqpoint{3.848580in}{2.477649in}}%
\pgfpathclose%
\pgfusepath{fill}%
\end{pgfscope}%
\begin{pgfscope}%
\pgfpathrectangle{\pgfqpoint{1.254980in}{0.150000in}}{\pgfqpoint{5.490039in}{5.490039in}}%
\pgfusepath{clip}%
\pgfsetbuttcap%
\pgfsetroundjoin%
\definecolor{currentfill}{rgb}{0.210503,0.363727,0.552206}%
\pgfsetfillcolor{currentfill}%
\pgfsetfillopacity{0.700000}%
\pgfsetlinewidth{0.000000pt}%
\definecolor{currentstroke}{rgb}{0.000000,0.000000,0.000000}%
\pgfsetstrokecolor{currentstroke}%
\pgfsetdash{}{0pt}%
\pgfpathmoveto{\pgfqpoint{2.865812in}{2.947144in}}%
\pgfpathlineto{\pgfqpoint{2.878894in}{2.928866in}}%
\pgfpathlineto{\pgfqpoint{2.891969in}{2.910882in}}%
\pgfpathlineto{\pgfqpoint{2.905035in}{2.893187in}}%
\pgfpathlineto{\pgfqpoint{2.918094in}{2.875778in}}%
\pgfpathlineto{\pgfqpoint{2.925960in}{2.885797in}}%
\pgfpathlineto{\pgfqpoint{2.933818in}{2.895940in}}%
\pgfpathlineto{\pgfqpoint{2.941668in}{2.906209in}}%
\pgfpathlineto{\pgfqpoint{2.949510in}{2.916605in}}%
\pgfpathlineto{\pgfqpoint{2.936468in}{2.934039in}}%
\pgfpathlineto{\pgfqpoint{2.923418in}{2.951759in}}%
\pgfpathlineto{\pgfqpoint{2.910361in}{2.969769in}}%
\pgfpathlineto{\pgfqpoint{2.897295in}{2.988071in}}%
\pgfpathlineto{\pgfqpoint{2.889436in}{2.977639in}}%
\pgfpathlineto{\pgfqpoint{2.881569in}{2.967341in}}%
\pgfpathlineto{\pgfqpoint{2.873695in}{2.957176in}}%
\pgfpathlineto{\pgfqpoint{2.865812in}{2.947144in}}%
\pgfpathclose%
\pgfusepath{fill}%
\end{pgfscope}%
\begin{pgfscope}%
\pgfpathrectangle{\pgfqpoint{1.254980in}{0.150000in}}{\pgfqpoint{5.490039in}{5.490039in}}%
\pgfusepath{clip}%
\pgfsetbuttcap%
\pgfsetroundjoin%
\definecolor{currentfill}{rgb}{0.282290,0.145912,0.461510}%
\pgfsetfillcolor{currentfill}%
\pgfsetfillopacity{0.700000}%
\pgfsetlinewidth{0.000000pt}%
\definecolor{currentstroke}{rgb}{0.000000,0.000000,0.000000}%
\pgfsetstrokecolor{currentstroke}%
\pgfsetdash{}{0pt}%
\pgfpathmoveto{\pgfqpoint{3.632492in}{2.455365in}}%
\pgfpathlineto{\pgfqpoint{3.645417in}{2.449342in}}%
\pgfpathlineto{\pgfqpoint{3.658345in}{2.443518in}}%
\pgfpathlineto{\pgfqpoint{3.671277in}{2.437890in}}%
\pgfpathlineto{\pgfqpoint{3.684212in}{2.432459in}}%
\pgfpathlineto{\pgfqpoint{3.691842in}{2.442741in}}%
\pgfpathlineto{\pgfqpoint{3.699466in}{2.453071in}}%
\pgfpathlineto{\pgfqpoint{3.707086in}{2.463451in}}%
\pgfpathlineto{\pgfqpoint{3.714701in}{2.473884in}}%
\pgfpathlineto{\pgfqpoint{3.701775in}{2.479427in}}%
\pgfpathlineto{\pgfqpoint{3.688853in}{2.485165in}}%
\pgfpathlineto{\pgfqpoint{3.675934in}{2.491101in}}%
\pgfpathlineto{\pgfqpoint{3.663018in}{2.497236in}}%
\pgfpathlineto{\pgfqpoint{3.655394in}{2.486682in}}%
\pgfpathlineto{\pgfqpoint{3.647765in}{2.476186in}}%
\pgfpathlineto{\pgfqpoint{3.640131in}{2.465748in}}%
\pgfpathlineto{\pgfqpoint{3.632492in}{2.455365in}}%
\pgfpathclose%
\pgfusepath{fill}%
\end{pgfscope}%
\begin{pgfscope}%
\pgfpathrectangle{\pgfqpoint{1.254980in}{0.150000in}}{\pgfqpoint{5.490039in}{5.490039in}}%
\pgfusepath{clip}%
\pgfsetbuttcap%
\pgfsetroundjoin%
\definecolor{currentfill}{rgb}{0.269308,0.218818,0.509577}%
\pgfsetfillcolor{currentfill}%
\pgfsetfillopacity{0.700000}%
\pgfsetlinewidth{0.000000pt}%
\definecolor{currentstroke}{rgb}{0.000000,0.000000,0.000000}%
\pgfsetstrokecolor{currentstroke}%
\pgfsetdash{}{0pt}%
\pgfpathmoveto{\pgfqpoint{4.228651in}{2.590095in}}%
\pgfpathlineto{\pgfqpoint{4.241696in}{2.588885in}}%
\pgfpathlineto{\pgfqpoint{4.254748in}{2.587846in}}%
\pgfpathlineto{\pgfqpoint{4.267807in}{2.586980in}}%
\pgfpathlineto{\pgfqpoint{4.280875in}{2.586284in}}%
\pgfpathlineto{\pgfqpoint{4.288320in}{2.596183in}}%
\pgfpathlineto{\pgfqpoint{4.295761in}{2.606134in}}%
\pgfpathlineto{\pgfqpoint{4.303198in}{2.616141in}}%
\pgfpathlineto{\pgfqpoint{4.310630in}{2.626206in}}%
\pgfpathlineto{\pgfqpoint{4.297572in}{2.627180in}}%
\pgfpathlineto{\pgfqpoint{4.284522in}{2.628325in}}%
\pgfpathlineto{\pgfqpoint{4.271479in}{2.629641in}}%
\pgfpathlineto{\pgfqpoint{4.258444in}{2.631129in}}%
\pgfpathlineto{\pgfqpoint{4.251002in}{2.620776in}}%
\pgfpathlineto{\pgfqpoint{4.243556in}{2.610488in}}%
\pgfpathlineto{\pgfqpoint{4.236106in}{2.600262in}}%
\pgfpathlineto{\pgfqpoint{4.228651in}{2.590095in}}%
\pgfpathclose%
\pgfusepath{fill}%
\end{pgfscope}%
\begin{pgfscope}%
\pgfpathrectangle{\pgfqpoint{1.254980in}{0.150000in}}{\pgfqpoint{5.490039in}{5.490039in}}%
\pgfusepath{clip}%
\pgfsetbuttcap%
\pgfsetroundjoin%
\definecolor{currentfill}{rgb}{0.280868,0.160771,0.472899}%
\pgfsetfillcolor{currentfill}%
\pgfsetfillopacity{0.700000}%
\pgfsetlinewidth{0.000000pt}%
\definecolor{currentstroke}{rgb}{0.000000,0.000000,0.000000}%
\pgfsetstrokecolor{currentstroke}%
\pgfsetdash{}{0pt}%
\pgfpathmoveto{\pgfqpoint{3.364303in}{2.496169in}}%
\pgfpathlineto{\pgfqpoint{3.377227in}{2.486951in}}%
\pgfpathlineto{\pgfqpoint{3.390151in}{2.477952in}}%
\pgfpathlineto{\pgfqpoint{3.403075in}{2.469170in}}%
\pgfpathlineto{\pgfqpoint{3.416000in}{2.460603in}}%
\pgfpathlineto{\pgfqpoint{3.423714in}{2.470728in}}%
\pgfpathlineto{\pgfqpoint{3.431422in}{2.480920in}}%
\pgfpathlineto{\pgfqpoint{3.439125in}{2.491178in}}%
\pgfpathlineto{\pgfqpoint{3.446822in}{2.501505in}}%
\pgfpathlineto{\pgfqpoint{3.433909in}{2.510127in}}%
\pgfpathlineto{\pgfqpoint{3.420996in}{2.518966in}}%
\pgfpathlineto{\pgfqpoint{3.408084in}{2.528021in}}%
\pgfpathlineto{\pgfqpoint{3.395171in}{2.537294in}}%
\pgfpathlineto{\pgfqpoint{3.387463in}{2.526902in}}%
\pgfpathlineto{\pgfqpoint{3.379749in}{2.516584in}}%
\pgfpathlineto{\pgfqpoint{3.372029in}{2.506340in}}%
\pgfpathlineto{\pgfqpoint{3.364303in}{2.496169in}}%
\pgfpathclose%
\pgfusepath{fill}%
\end{pgfscope}%
\begin{pgfscope}%
\pgfpathrectangle{\pgfqpoint{1.254980in}{0.150000in}}{\pgfqpoint{5.490039in}{5.490039in}}%
\pgfusepath{clip}%
\pgfsetbuttcap%
\pgfsetroundjoin%
\definecolor{currentfill}{rgb}{0.160665,0.478540,0.558115}%
\pgfsetfillcolor{currentfill}%
\pgfsetfillopacity{0.700000}%
\pgfsetlinewidth{0.000000pt}%
\definecolor{currentstroke}{rgb}{0.000000,0.000000,0.000000}%
\pgfsetstrokecolor{currentstroke}%
\pgfsetdash{}{0pt}%
\pgfpathmoveto{\pgfqpoint{5.376590in}{3.194542in}}%
\pgfpathlineto{\pgfqpoint{5.390002in}{3.195835in}}%
\pgfpathlineto{\pgfqpoint{5.403426in}{3.197276in}}%
\pgfpathlineto{\pgfqpoint{5.416862in}{3.198868in}}%
\pgfpathlineto{\pgfqpoint{5.430311in}{3.200608in}}%
\pgfpathlineto{\pgfqpoint{5.437423in}{3.211466in}}%
\pgfpathlineto{\pgfqpoint{5.444539in}{3.222583in}}%
\pgfpathlineto{\pgfqpoint{5.451657in}{3.233967in}}%
\pgfpathlineto{\pgfqpoint{5.438227in}{3.232726in}}%
\pgfpathlineto{\pgfqpoint{5.424808in}{3.231634in}}%
\pgfpathlineto{\pgfqpoint{5.411402in}{3.230690in}}%
\pgfpathlineto{\pgfqpoint{5.398008in}{3.229896in}}%
\pgfpathlineto{\pgfqpoint{5.390866in}{3.217841in}}%
\pgfpathlineto{\pgfqpoint{5.383726in}{3.206059in}}%
\pgfpathlineto{\pgfqpoint{5.376590in}{3.194542in}}%
\pgfpathclose%
\pgfusepath{fill}%
\end{pgfscope}%
\begin{pgfscope}%
\pgfpathrectangle{\pgfqpoint{1.254980in}{0.150000in}}{\pgfqpoint{5.490039in}{5.490039in}}%
\pgfusepath{clip}%
\pgfsetbuttcap%
\pgfsetroundjoin%
\definecolor{currentfill}{rgb}{0.273006,0.204520,0.501721}%
\pgfsetfillcolor{currentfill}%
\pgfsetfillopacity{0.700000}%
\pgfsetlinewidth{0.000000pt}%
\definecolor{currentstroke}{rgb}{0.000000,0.000000,0.000000}%
\pgfsetstrokecolor{currentstroke}%
\pgfsetdash{}{0pt}%
\pgfpathmoveto{\pgfqpoint{3.178026in}{2.584072in}}%
\pgfpathlineto{\pgfqpoint{3.190979in}{2.572138in}}%
\pgfpathlineto{\pgfqpoint{3.203930in}{2.560443in}}%
\pgfpathlineto{\pgfqpoint{3.216878in}{2.548983in}}%
\pgfpathlineto{\pgfqpoint{3.229824in}{2.537759in}}%
\pgfpathlineto{\pgfqpoint{3.237599in}{2.547682in}}%
\pgfpathlineto{\pgfqpoint{3.245368in}{2.557688in}}%
\pgfpathlineto{\pgfqpoint{3.253131in}{2.567779in}}%
\pgfpathlineto{\pgfqpoint{3.260887in}{2.577954in}}%
\pgfpathlineto{\pgfqpoint{3.247954in}{2.589206in}}%
\pgfpathlineto{\pgfqpoint{3.235020in}{2.600693in}}%
\pgfpathlineto{\pgfqpoint{3.222083in}{2.612415in}}%
\pgfpathlineto{\pgfqpoint{3.209143in}{2.624376in}}%
\pgfpathlineto{\pgfqpoint{3.201374in}{2.614163in}}%
\pgfpathlineto{\pgfqpoint{3.193598in}{2.604042in}}%
\pgfpathlineto{\pgfqpoint{3.185815in}{2.594012in}}%
\pgfpathlineto{\pgfqpoint{3.178026in}{2.584072in}}%
\pgfpathclose%
\pgfusepath{fill}%
\end{pgfscope}%
\begin{pgfscope}%
\pgfpathrectangle{\pgfqpoint{1.254980in}{0.150000in}}{\pgfqpoint{5.490039in}{5.490039in}}%
\pgfusepath{clip}%
\pgfsetbuttcap%
\pgfsetroundjoin%
\definecolor{currentfill}{rgb}{0.274128,0.199721,0.498911}%
\pgfsetfillcolor{currentfill}%
\pgfsetfillopacity{0.700000}%
\pgfsetlinewidth{0.000000pt}%
\definecolor{currentstroke}{rgb}{0.000000,0.000000,0.000000}%
\pgfsetstrokecolor{currentstroke}%
\pgfsetdash{}{0pt}%
\pgfpathmoveto{\pgfqpoint{4.146650in}{2.555525in}}%
\pgfpathlineto{\pgfqpoint{4.159673in}{2.553870in}}%
\pgfpathlineto{\pgfqpoint{4.172704in}{2.552390in}}%
\pgfpathlineto{\pgfqpoint{4.185742in}{2.551084in}}%
\pgfpathlineto{\pgfqpoint{4.198788in}{2.549951in}}%
\pgfpathlineto{\pgfqpoint{4.206261in}{2.559915in}}%
\pgfpathlineto{\pgfqpoint{4.213729in}{2.569924in}}%
\pgfpathlineto{\pgfqpoint{4.221192in}{2.579984in}}%
\pgfpathlineto{\pgfqpoint{4.228651in}{2.590095in}}%
\pgfpathlineto{\pgfqpoint{4.215615in}{2.591478in}}%
\pgfpathlineto{\pgfqpoint{4.202586in}{2.593035in}}%
\pgfpathlineto{\pgfqpoint{4.189564in}{2.594765in}}%
\pgfpathlineto{\pgfqpoint{4.176549in}{2.596670in}}%
\pgfpathlineto{\pgfqpoint{4.169081in}{2.586298in}}%
\pgfpathlineto{\pgfqpoint{4.161608in}{2.575985in}}%
\pgfpathlineto{\pgfqpoint{4.154131in}{2.565728in}}%
\pgfpathlineto{\pgfqpoint{4.146650in}{2.555525in}}%
\pgfpathclose%
\pgfusepath{fill}%
\end{pgfscope}%
\begin{pgfscope}%
\pgfpathrectangle{\pgfqpoint{1.254980in}{0.150000in}}{\pgfqpoint{5.490039in}{5.490039in}}%
\pgfusepath{clip}%
\pgfsetbuttcap%
\pgfsetroundjoin%
\definecolor{currentfill}{rgb}{0.195860,0.395433,0.555276}%
\pgfsetfillcolor{currentfill}%
\pgfsetfillopacity{0.700000}%
\pgfsetlinewidth{0.000000pt}%
\definecolor{currentstroke}{rgb}{0.000000,0.000000,0.000000}%
\pgfsetstrokecolor{currentstroke}%
\pgfsetdash{}{0pt}%
\pgfpathmoveto{\pgfqpoint{2.813391in}{3.023233in}}%
\pgfpathlineto{\pgfqpoint{2.826510in}{3.003758in}}%
\pgfpathlineto{\pgfqpoint{2.839620in}{2.984586in}}%
\pgfpathlineto{\pgfqpoint{2.852720in}{2.965716in}}%
\pgfpathlineto{\pgfqpoint{2.865812in}{2.947144in}}%
\pgfpathlineto{\pgfqpoint{2.873695in}{2.957176in}}%
\pgfpathlineto{\pgfqpoint{2.881569in}{2.967341in}}%
\pgfpathlineto{\pgfqpoint{2.889436in}{2.977639in}}%
\pgfpathlineto{\pgfqpoint{2.897295in}{2.988071in}}%
\pgfpathlineto{\pgfqpoint{2.884221in}{3.006668in}}%
\pgfpathlineto{\pgfqpoint{2.871138in}{3.025563in}}%
\pgfpathlineto{\pgfqpoint{2.858046in}{3.044759in}}%
\pgfpathlineto{\pgfqpoint{2.844944in}{3.064259in}}%
\pgfpathlineto{\pgfqpoint{2.837069in}{3.053792in}}%
\pgfpathlineto{\pgfqpoint{2.829184in}{3.043466in}}%
\pgfpathlineto{\pgfqpoint{2.821292in}{3.033280in}}%
\pgfpathlineto{\pgfqpoint{2.813391in}{3.023233in}}%
\pgfpathclose%
\pgfusepath{fill}%
\end{pgfscope}%
\begin{pgfscope}%
\pgfpathrectangle{\pgfqpoint{1.254980in}{0.150000in}}{\pgfqpoint{5.490039in}{5.490039in}}%
\pgfusepath{clip}%
\pgfsetbuttcap%
\pgfsetroundjoin%
\definecolor{currentfill}{rgb}{0.282290,0.145912,0.461510}%
\pgfsetfillcolor{currentfill}%
\pgfsetfillopacity{0.700000}%
\pgfsetlinewidth{0.000000pt}%
\definecolor{currentstroke}{rgb}{0.000000,0.000000,0.000000}%
\pgfsetstrokecolor{currentstroke}%
\pgfsetdash{}{0pt}%
\pgfpathmoveto{\pgfqpoint{3.766439in}{2.453652in}}%
\pgfpathlineto{\pgfqpoint{3.779383in}{2.449075in}}%
\pgfpathlineto{\pgfqpoint{3.792332in}{2.444688in}}%
\pgfpathlineto{\pgfqpoint{3.805285in}{2.440489in}}%
\pgfpathlineto{\pgfqpoint{3.818243in}{2.436480in}}%
\pgfpathlineto{\pgfqpoint{3.825834in}{2.446708in}}%
\pgfpathlineto{\pgfqpoint{3.833421in}{2.456977in}}%
\pgfpathlineto{\pgfqpoint{3.841003in}{2.467290in}}%
\pgfpathlineto{\pgfqpoint{3.848580in}{2.477649in}}%
\pgfpathlineto{\pgfqpoint{3.835631in}{2.481798in}}%
\pgfpathlineto{\pgfqpoint{3.822687in}{2.486136in}}%
\pgfpathlineto{\pgfqpoint{3.809747in}{2.490662in}}%
\pgfpathlineto{\pgfqpoint{3.796812in}{2.495379in}}%
\pgfpathlineto{\pgfqpoint{3.789226in}{2.484871in}}%
\pgfpathlineto{\pgfqpoint{3.781635in}{2.474415in}}%
\pgfpathlineto{\pgfqpoint{3.774040in}{2.464009in}}%
\pgfpathlineto{\pgfqpoint{3.766439in}{2.453652in}}%
\pgfpathclose%
\pgfusepath{fill}%
\end{pgfscope}%
\begin{pgfscope}%
\pgfpathrectangle{\pgfqpoint{1.254980in}{0.150000in}}{\pgfqpoint{5.490039in}{5.490039in}}%
\pgfusepath{clip}%
\pgfsetbuttcap%
\pgfsetroundjoin%
\definecolor{currentfill}{rgb}{0.277134,0.185228,0.489898}%
\pgfsetfillcolor{currentfill}%
\pgfsetfillopacity{0.700000}%
\pgfsetlinewidth{0.000000pt}%
\definecolor{currentstroke}{rgb}{0.000000,0.000000,0.000000}%
\pgfsetstrokecolor{currentstroke}%
\pgfsetdash{}{0pt}%
\pgfpathmoveto{\pgfqpoint{4.064617in}{2.522671in}}%
\pgfpathlineto{\pgfqpoint{4.077622in}{2.520533in}}%
\pgfpathlineto{\pgfqpoint{4.090634in}{2.518573in}}%
\pgfpathlineto{\pgfqpoint{4.103652in}{2.516790in}}%
\pgfpathlineto{\pgfqpoint{4.116677in}{2.515183in}}%
\pgfpathlineto{\pgfqpoint{4.124177in}{2.525203in}}%
\pgfpathlineto{\pgfqpoint{4.131673in}{2.535265in}}%
\pgfpathlineto{\pgfqpoint{4.139163in}{2.545371in}}%
\pgfpathlineto{\pgfqpoint{4.146650in}{2.555525in}}%
\pgfpathlineto{\pgfqpoint{4.133633in}{2.557355in}}%
\pgfpathlineto{\pgfqpoint{4.120623in}{2.559361in}}%
\pgfpathlineto{\pgfqpoint{4.107620in}{2.561543in}}%
\pgfpathlineto{\pgfqpoint{4.094624in}{2.563903in}}%
\pgfpathlineto{\pgfqpoint{4.087129in}{2.553517in}}%
\pgfpathlineto{\pgfqpoint{4.079630in}{2.543184in}}%
\pgfpathlineto{\pgfqpoint{4.072126in}{2.532903in}}%
\pgfpathlineto{\pgfqpoint{4.064617in}{2.522671in}}%
\pgfpathclose%
\pgfusepath{fill}%
\end{pgfscope}%
\begin{pgfscope}%
\pgfpathrectangle{\pgfqpoint{1.254980in}{0.150000in}}{\pgfqpoint{5.490039in}{5.490039in}}%
\pgfusepath{clip}%
\pgfsetbuttcap%
\pgfsetroundjoin%
\definecolor{currentfill}{rgb}{0.277134,0.185228,0.489898}%
\pgfsetfillcolor{currentfill}%
\pgfsetfillopacity{0.700000}%
\pgfsetlinewidth{0.000000pt}%
\definecolor{currentstroke}{rgb}{0.000000,0.000000,0.000000}%
\pgfsetstrokecolor{currentstroke}%
\pgfsetdash{}{0pt}%
\pgfpathmoveto{\pgfqpoint{3.229824in}{2.537759in}}%
\pgfpathlineto{\pgfqpoint{3.242768in}{2.526766in}}%
\pgfpathlineto{\pgfqpoint{3.255710in}{2.516005in}}%
\pgfpathlineto{\pgfqpoint{3.268651in}{2.505473in}}%
\pgfpathlineto{\pgfqpoint{3.281591in}{2.495169in}}%
\pgfpathlineto{\pgfqpoint{3.289353in}{2.505074in}}%
\pgfpathlineto{\pgfqpoint{3.297109in}{2.515057in}}%
\pgfpathlineto{\pgfqpoint{3.304859in}{2.525116in}}%
\pgfpathlineto{\pgfqpoint{3.312603in}{2.535254in}}%
\pgfpathlineto{\pgfqpoint{3.299676in}{2.545586in}}%
\pgfpathlineto{\pgfqpoint{3.286748in}{2.556145in}}%
\pgfpathlineto{\pgfqpoint{3.273818in}{2.566934in}}%
\pgfpathlineto{\pgfqpoint{3.260887in}{2.577954in}}%
\pgfpathlineto{\pgfqpoint{3.253131in}{2.567779in}}%
\pgfpathlineto{\pgfqpoint{3.245368in}{2.557688in}}%
\pgfpathlineto{\pgfqpoint{3.237599in}{2.547682in}}%
\pgfpathlineto{\pgfqpoint{3.229824in}{2.537759in}}%
\pgfpathclose%
\pgfusepath{fill}%
\end{pgfscope}%
\begin{pgfscope}%
\pgfpathrectangle{\pgfqpoint{1.254980in}{0.150000in}}{\pgfqpoint{5.490039in}{5.490039in}}%
\pgfusepath{clip}%
\pgfsetbuttcap%
\pgfsetroundjoin%
\definecolor{currentfill}{rgb}{0.282623,0.140926,0.457517}%
\pgfsetfillcolor{currentfill}%
\pgfsetfillopacity{0.700000}%
\pgfsetlinewidth{0.000000pt}%
\definecolor{currentstroke}{rgb}{0.000000,0.000000,0.000000}%
\pgfsetstrokecolor{currentstroke}%
\pgfsetdash{}{0pt}%
\pgfpathmoveto{\pgfqpoint{3.550166in}{2.440119in}}%
\pgfpathlineto{\pgfqpoint{3.563092in}{2.433376in}}%
\pgfpathlineto{\pgfqpoint{3.576019in}{2.426836in}}%
\pgfpathlineto{\pgfqpoint{3.588949in}{2.420497in}}%
\pgfpathlineto{\pgfqpoint{3.601882in}{2.414359in}}%
\pgfpathlineto{\pgfqpoint{3.609542in}{2.424534in}}%
\pgfpathlineto{\pgfqpoint{3.617197in}{2.434760in}}%
\pgfpathlineto{\pgfqpoint{3.624847in}{2.445036in}}%
\pgfpathlineto{\pgfqpoint{3.632492in}{2.455365in}}%
\pgfpathlineto{\pgfqpoint{3.619569in}{2.461587in}}%
\pgfpathlineto{\pgfqpoint{3.606649in}{2.468010in}}%
\pgfpathlineto{\pgfqpoint{3.593731in}{2.474634in}}%
\pgfpathlineto{\pgfqpoint{3.580816in}{2.481460in}}%
\pgfpathlineto{\pgfqpoint{3.573162in}{2.471037in}}%
\pgfpathlineto{\pgfqpoint{3.565502in}{2.460674in}}%
\pgfpathlineto{\pgfqpoint{3.557837in}{2.450368in}}%
\pgfpathlineto{\pgfqpoint{3.550166in}{2.440119in}}%
\pgfpathclose%
\pgfusepath{fill}%
\end{pgfscope}%
\begin{pgfscope}%
\pgfpathrectangle{\pgfqpoint{1.254980in}{0.150000in}}{\pgfqpoint{5.490039in}{5.490039in}}%
\pgfusepath{clip}%
\pgfsetbuttcap%
\pgfsetroundjoin%
\definecolor{currentfill}{rgb}{0.281887,0.150881,0.465405}%
\pgfsetfillcolor{currentfill}%
\pgfsetfillopacity{0.700000}%
\pgfsetlinewidth{0.000000pt}%
\definecolor{currentstroke}{rgb}{0.000000,0.000000,0.000000}%
\pgfsetstrokecolor{currentstroke}%
\pgfsetdash{}{0pt}%
\pgfpathmoveto{\pgfqpoint{3.416000in}{2.460603in}}%
\pgfpathlineto{\pgfqpoint{3.428925in}{2.452250in}}%
\pgfpathlineto{\pgfqpoint{3.441851in}{2.444110in}}%
\pgfpathlineto{\pgfqpoint{3.454777in}{2.436181in}}%
\pgfpathlineto{\pgfqpoint{3.467705in}{2.428463in}}%
\pgfpathlineto{\pgfqpoint{3.475408in}{2.438543in}}%
\pgfpathlineto{\pgfqpoint{3.483105in}{2.448682in}}%
\pgfpathlineto{\pgfqpoint{3.490797in}{2.458881in}}%
\pgfpathlineto{\pgfqpoint{3.498484in}{2.469141in}}%
\pgfpathlineto{\pgfqpoint{3.485567in}{2.476916in}}%
\pgfpathlineto{\pgfqpoint{3.472651in}{2.484900in}}%
\pgfpathlineto{\pgfqpoint{3.459736in}{2.493096in}}%
\pgfpathlineto{\pgfqpoint{3.446822in}{2.501505in}}%
\pgfpathlineto{\pgfqpoint{3.439125in}{2.491178in}}%
\pgfpathlineto{\pgfqpoint{3.431422in}{2.480920in}}%
\pgfpathlineto{\pgfqpoint{3.423714in}{2.470728in}}%
\pgfpathlineto{\pgfqpoint{3.416000in}{2.460603in}}%
\pgfpathclose%
\pgfusepath{fill}%
\end{pgfscope}%
\begin{pgfscope}%
\pgfpathrectangle{\pgfqpoint{1.254980in}{0.150000in}}{\pgfqpoint{5.490039in}{5.490039in}}%
\pgfusepath{clip}%
\pgfsetbuttcap%
\pgfsetroundjoin%
\definecolor{currentfill}{rgb}{0.279574,0.170599,0.479997}%
\pgfsetfillcolor{currentfill}%
\pgfsetfillopacity{0.700000}%
\pgfsetlinewidth{0.000000pt}%
\definecolor{currentstroke}{rgb}{0.000000,0.000000,0.000000}%
\pgfsetstrokecolor{currentstroke}%
\pgfsetdash{}{0pt}%
\pgfpathmoveto{\pgfqpoint{3.982546in}{2.491729in}}%
\pgfpathlineto{\pgfqpoint{3.995534in}{2.489070in}}%
\pgfpathlineto{\pgfqpoint{4.008528in}{2.486592in}}%
\pgfpathlineto{\pgfqpoint{4.021529in}{2.484292in}}%
\pgfpathlineto{\pgfqpoint{4.034536in}{2.482172in}}%
\pgfpathlineto{\pgfqpoint{4.042063in}{2.492237in}}%
\pgfpathlineto{\pgfqpoint{4.049586in}{2.502340in}}%
\pgfpathlineto{\pgfqpoint{4.057104in}{2.512484in}}%
\pgfpathlineto{\pgfqpoint{4.064617in}{2.522671in}}%
\pgfpathlineto{\pgfqpoint{4.051619in}{2.524986in}}%
\pgfpathlineto{\pgfqpoint{4.038627in}{2.527480in}}%
\pgfpathlineto{\pgfqpoint{4.025641in}{2.530154in}}%
\pgfpathlineto{\pgfqpoint{4.012661in}{2.533007in}}%
\pgfpathlineto{\pgfqpoint{4.005139in}{2.522615in}}%
\pgfpathlineto{\pgfqpoint{3.997613in}{2.512273in}}%
\pgfpathlineto{\pgfqpoint{3.990081in}{2.501979in}}%
\pgfpathlineto{\pgfqpoint{3.982546in}{2.491729in}}%
\pgfpathclose%
\pgfusepath{fill}%
\end{pgfscope}%
\begin{pgfscope}%
\pgfpathrectangle{\pgfqpoint{1.254980in}{0.150000in}}{\pgfqpoint{5.490039in}{5.490039in}}%
\pgfusepath{clip}%
\pgfsetbuttcap%
\pgfsetroundjoin%
\definecolor{currentfill}{rgb}{0.282623,0.140926,0.457517}%
\pgfsetfillcolor{currentfill}%
\pgfsetfillopacity{0.700000}%
\pgfsetlinewidth{0.000000pt}%
\definecolor{currentstroke}{rgb}{0.000000,0.000000,0.000000}%
\pgfsetstrokecolor{currentstroke}%
\pgfsetdash{}{0pt}%
\pgfpathmoveto{\pgfqpoint{3.684212in}{2.432459in}}%
\pgfpathlineto{\pgfqpoint{3.697150in}{2.427223in}}%
\pgfpathlineto{\pgfqpoint{3.710091in}{2.422181in}}%
\pgfpathlineto{\pgfqpoint{3.723037in}{2.417332in}}%
\pgfpathlineto{\pgfqpoint{3.735986in}{2.412675in}}%
\pgfpathlineto{\pgfqpoint{3.743607in}{2.422855in}}%
\pgfpathlineto{\pgfqpoint{3.751223in}{2.433077in}}%
\pgfpathlineto{\pgfqpoint{3.758833in}{2.443342in}}%
\pgfpathlineto{\pgfqpoint{3.766439in}{2.453652in}}%
\pgfpathlineto{\pgfqpoint{3.753499in}{2.458421in}}%
\pgfpathlineto{\pgfqpoint{3.740562in}{2.463382in}}%
\pgfpathlineto{\pgfqpoint{3.727630in}{2.468536in}}%
\pgfpathlineto{\pgfqpoint{3.714701in}{2.473884in}}%
\pgfpathlineto{\pgfqpoint{3.707086in}{2.463451in}}%
\pgfpathlineto{\pgfqpoint{3.699466in}{2.453071in}}%
\pgfpathlineto{\pgfqpoint{3.691842in}{2.442741in}}%
\pgfpathlineto{\pgfqpoint{3.684212in}{2.432459in}}%
\pgfpathclose%
\pgfusepath{fill}%
\end{pgfscope}%
\begin{pgfscope}%
\pgfpathrectangle{\pgfqpoint{1.254980in}{0.150000in}}{\pgfqpoint{5.490039in}{5.490039in}}%
\pgfusepath{clip}%
\pgfsetbuttcap%
\pgfsetroundjoin%
\definecolor{currentfill}{rgb}{0.223925,0.334994,0.548053}%
\pgfsetfillcolor{currentfill}%
\pgfsetfillopacity{0.700000}%
\pgfsetlinewidth{0.000000pt}%
\definecolor{currentstroke}{rgb}{0.000000,0.000000,0.000000}%
\pgfsetstrokecolor{currentstroke}%
\pgfsetdash{}{0pt}%
\pgfpathmoveto{\pgfqpoint{4.773218in}{2.828938in}}%
\pgfpathlineto{\pgfqpoint{4.786453in}{2.830264in}}%
\pgfpathlineto{\pgfqpoint{4.799699in}{2.831750in}}%
\pgfpathlineto{\pgfqpoint{4.812955in}{2.833394in}}%
\pgfpathlineto{\pgfqpoint{4.826222in}{2.835196in}}%
\pgfpathlineto{\pgfqpoint{4.833492in}{2.844375in}}%
\pgfpathlineto{\pgfqpoint{4.840759in}{2.853659in}}%
\pgfpathlineto{\pgfqpoint{4.848023in}{2.863053in}}%
\pgfpathlineto{\pgfqpoint{4.855284in}{2.872562in}}%
\pgfpathlineto{\pgfqpoint{4.842031in}{2.871207in}}%
\pgfpathlineto{\pgfqpoint{4.828789in}{2.870010in}}%
\pgfpathlineto{\pgfqpoint{4.815558in}{2.868971in}}%
\pgfpathlineto{\pgfqpoint{4.802337in}{2.868091in}}%
\pgfpathlineto{\pgfqpoint{4.795061in}{2.858125in}}%
\pgfpathlineto{\pgfqpoint{4.787783in}{2.848281in}}%
\pgfpathlineto{\pgfqpoint{4.780502in}{2.838554in}}%
\pgfpathlineto{\pgfqpoint{4.773218in}{2.828938in}}%
\pgfpathclose%
\pgfusepath{fill}%
\end{pgfscope}%
\begin{pgfscope}%
\pgfpathrectangle{\pgfqpoint{1.254980in}{0.150000in}}{\pgfqpoint{5.490039in}{5.490039in}}%
\pgfusepath{clip}%
\pgfsetbuttcap%
\pgfsetroundjoin%
\definecolor{currentfill}{rgb}{0.214298,0.355619,0.551184}%
\pgfsetfillcolor{currentfill}%
\pgfsetfillopacity{0.700000}%
\pgfsetlinewidth{0.000000pt}%
\definecolor{currentstroke}{rgb}{0.000000,0.000000,0.000000}%
\pgfsetstrokecolor{currentstroke}%
\pgfsetdash{}{0pt}%
\pgfpathmoveto{\pgfqpoint{4.855284in}{2.872562in}}%
\pgfpathlineto{\pgfqpoint{4.868548in}{2.874075in}}%
\pgfpathlineto{\pgfqpoint{4.881823in}{2.875746in}}%
\pgfpathlineto{\pgfqpoint{4.895108in}{2.877573in}}%
\pgfpathlineto{\pgfqpoint{4.908406in}{2.879558in}}%
\pgfpathlineto{\pgfqpoint{4.915649in}{2.888723in}}%
\pgfpathlineto{\pgfqpoint{4.922889in}{2.898008in}}%
\pgfpathlineto{\pgfqpoint{4.930127in}{2.907417in}}%
\pgfpathlineto{\pgfqpoint{4.937362in}{2.916956in}}%
\pgfpathlineto{\pgfqpoint{4.924081in}{2.915446in}}%
\pgfpathlineto{\pgfqpoint{4.910810in}{2.914093in}}%
\pgfpathlineto{\pgfqpoint{4.897551in}{2.912897in}}%
\pgfpathlineto{\pgfqpoint{4.884302in}{2.911858in}}%
\pgfpathlineto{\pgfqpoint{4.877051in}{2.901835in}}%
\pgfpathlineto{\pgfqpoint{4.869798in}{2.891948in}}%
\pgfpathlineto{\pgfqpoint{4.862542in}{2.882192in}}%
\pgfpathlineto{\pgfqpoint{4.855284in}{2.872562in}}%
\pgfpathclose%
\pgfusepath{fill}%
\end{pgfscope}%
\begin{pgfscope}%
\pgfpathrectangle{\pgfqpoint{1.254980in}{0.150000in}}{\pgfqpoint{5.490039in}{5.490039in}}%
\pgfusepath{clip}%
\pgfsetbuttcap%
\pgfsetroundjoin%
\definecolor{currentfill}{rgb}{0.231674,0.318106,0.544834}%
\pgfsetfillcolor{currentfill}%
\pgfsetfillopacity{0.700000}%
\pgfsetlinewidth{0.000000pt}%
\definecolor{currentstroke}{rgb}{0.000000,0.000000,0.000000}%
\pgfsetstrokecolor{currentstroke}%
\pgfsetdash{}{0pt}%
\pgfpathmoveto{\pgfqpoint{4.691160in}{2.786109in}}%
\pgfpathlineto{\pgfqpoint{4.704367in}{2.787215in}}%
\pgfpathlineto{\pgfqpoint{4.717584in}{2.788481in}}%
\pgfpathlineto{\pgfqpoint{4.730811in}{2.789907in}}%
\pgfpathlineto{\pgfqpoint{4.744049in}{2.791493in}}%
\pgfpathlineto{\pgfqpoint{4.751346in}{2.800712in}}%
\pgfpathlineto{\pgfqpoint{4.758640in}{2.810022in}}%
\pgfpathlineto{\pgfqpoint{4.765931in}{2.819429in}}%
\pgfpathlineto{\pgfqpoint{4.773218in}{2.828938in}}%
\pgfpathlineto{\pgfqpoint{4.759994in}{2.827771in}}%
\pgfpathlineto{\pgfqpoint{4.746780in}{2.826764in}}%
\pgfpathlineto{\pgfqpoint{4.733576in}{2.825916in}}%
\pgfpathlineto{\pgfqpoint{4.720383in}{2.825228in}}%
\pgfpathlineto{\pgfqpoint{4.713082in}{2.815291in}}%
\pgfpathlineto{\pgfqpoint{4.705778in}{2.805462in}}%
\pgfpathlineto{\pgfqpoint{4.698471in}{2.795736in}}%
\pgfpathlineto{\pgfqpoint{4.691160in}{2.786109in}}%
\pgfpathclose%
\pgfusepath{fill}%
\end{pgfscope}%
\begin{pgfscope}%
\pgfpathrectangle{\pgfqpoint{1.254980in}{0.150000in}}{\pgfqpoint{5.490039in}{5.490039in}}%
\pgfusepath{clip}%
\pgfsetbuttcap%
\pgfsetroundjoin%
\definecolor{currentfill}{rgb}{0.206756,0.371758,0.553117}%
\pgfsetfillcolor{currentfill}%
\pgfsetfillopacity{0.700000}%
\pgfsetlinewidth{0.000000pt}%
\definecolor{currentstroke}{rgb}{0.000000,0.000000,0.000000}%
\pgfsetstrokecolor{currentstroke}%
\pgfsetdash{}{0pt}%
\pgfpathmoveto{\pgfqpoint{4.937362in}{2.916956in}}%
\pgfpathlineto{\pgfqpoint{4.950655in}{2.918621in}}%
\pgfpathlineto{\pgfqpoint{4.963959in}{2.920443in}}%
\pgfpathlineto{\pgfqpoint{4.977274in}{2.922421in}}%
\pgfpathlineto{\pgfqpoint{4.990601in}{2.924554in}}%
\pgfpathlineto{\pgfqpoint{4.997818in}{2.933736in}}%
\pgfpathlineto{\pgfqpoint{5.005032in}{2.943053in}}%
\pgfpathlineto{\pgfqpoint{5.012245in}{2.952509in}}%
\pgfpathlineto{\pgfqpoint{5.019455in}{2.962112in}}%
\pgfpathlineto{\pgfqpoint{5.006145in}{2.960482in}}%
\pgfpathlineto{\pgfqpoint{4.992846in}{2.959007in}}%
\pgfpathlineto{\pgfqpoint{4.979559in}{2.957687in}}%
\pgfpathlineto{\pgfqpoint{4.966283in}{2.956523in}}%
\pgfpathlineto{\pgfqpoint{4.959055in}{2.946408in}}%
\pgfpathlineto{\pgfqpoint{4.951826in}{2.936446in}}%
\pgfpathlineto{\pgfqpoint{4.944595in}{2.926630in}}%
\pgfpathlineto{\pgfqpoint{4.937362in}{2.916956in}}%
\pgfpathclose%
\pgfusepath{fill}%
\end{pgfscope}%
\begin{pgfscope}%
\pgfpathrectangle{\pgfqpoint{1.254980in}{0.150000in}}{\pgfqpoint{5.490039in}{5.490039in}}%
\pgfusepath{clip}%
\pgfsetbuttcap%
\pgfsetroundjoin%
\definecolor{currentfill}{rgb}{0.241237,0.296485,0.539709}%
\pgfsetfillcolor{currentfill}%
\pgfsetfillopacity{0.700000}%
\pgfsetlinewidth{0.000000pt}%
\definecolor{currentstroke}{rgb}{0.000000,0.000000,0.000000}%
\pgfsetstrokecolor{currentstroke}%
\pgfsetdash{}{0pt}%
\pgfpathmoveto{\pgfqpoint{4.609108in}{2.744124in}}%
\pgfpathlineto{\pgfqpoint{4.622286in}{2.744974in}}%
\pgfpathlineto{\pgfqpoint{4.635475in}{2.745986in}}%
\pgfpathlineto{\pgfqpoint{4.648674in}{2.747160in}}%
\pgfpathlineto{\pgfqpoint{4.661882in}{2.748496in}}%
\pgfpathlineto{\pgfqpoint{4.669208in}{2.757774in}}%
\pgfpathlineto{\pgfqpoint{4.676529in}{2.767133in}}%
\pgfpathlineto{\pgfqpoint{4.683846in}{2.776577in}}%
\pgfpathlineto{\pgfqpoint{4.691160in}{2.786109in}}%
\pgfpathlineto{\pgfqpoint{4.677964in}{2.785165in}}%
\pgfpathlineto{\pgfqpoint{4.664778in}{2.784381in}}%
\pgfpathlineto{\pgfqpoint{4.651602in}{2.783759in}}%
\pgfpathlineto{\pgfqpoint{4.638435in}{2.783299in}}%
\pgfpathlineto{\pgfqpoint{4.631109in}{2.773366in}}%
\pgfpathlineto{\pgfqpoint{4.623779in}{2.763529in}}%
\pgfpathlineto{\pgfqpoint{4.616445in}{2.753783in}}%
\pgfpathlineto{\pgfqpoint{4.609108in}{2.744124in}}%
\pgfpathclose%
\pgfusepath{fill}%
\end{pgfscope}%
\begin{pgfscope}%
\pgfpathrectangle{\pgfqpoint{1.254980in}{0.150000in}}{\pgfqpoint{5.490039in}{5.490039in}}%
\pgfusepath{clip}%
\pgfsetbuttcap%
\pgfsetroundjoin%
\definecolor{currentfill}{rgb}{0.197636,0.391528,0.554969}%
\pgfsetfillcolor{currentfill}%
\pgfsetfillopacity{0.700000}%
\pgfsetlinewidth{0.000000pt}%
\definecolor{currentstroke}{rgb}{0.000000,0.000000,0.000000}%
\pgfsetstrokecolor{currentstroke}%
\pgfsetdash{}{0pt}%
\pgfpathmoveto{\pgfqpoint{5.019455in}{2.962112in}}%
\pgfpathlineto{\pgfqpoint{5.032777in}{2.963897in}}%
\pgfpathlineto{\pgfqpoint{5.046110in}{2.965837in}}%
\pgfpathlineto{\pgfqpoint{5.059455in}{2.967932in}}%
\pgfpathlineto{\pgfqpoint{5.072812in}{2.970180in}}%
\pgfpathlineto{\pgfqpoint{5.080003in}{2.979414in}}%
\pgfpathlineto{\pgfqpoint{5.087193in}{2.988799in}}%
\pgfpathlineto{\pgfqpoint{5.094381in}{2.998342in}}%
\pgfpathlineto{\pgfqpoint{5.101568in}{3.008047in}}%
\pgfpathlineto{\pgfqpoint{5.088229in}{3.006329in}}%
\pgfpathlineto{\pgfqpoint{5.074902in}{3.004765in}}%
\pgfpathlineto{\pgfqpoint{5.061587in}{3.003356in}}%
\pgfpathlineto{\pgfqpoint{5.048283in}{3.002100in}}%
\pgfpathlineto{\pgfqpoint{5.041078in}{2.991855in}}%
\pgfpathlineto{\pgfqpoint{5.033872in}{2.981779in}}%
\pgfpathlineto{\pgfqpoint{5.026664in}{2.971867in}}%
\pgfpathlineto{\pgfqpoint{5.019455in}{2.962112in}}%
\pgfpathclose%
\pgfusepath{fill}%
\end{pgfscope}%
\begin{pgfscope}%
\pgfpathrectangle{\pgfqpoint{1.254980in}{0.150000in}}{\pgfqpoint{5.490039in}{5.490039in}}%
\pgfusepath{clip}%
\pgfsetbuttcap%
\pgfsetroundjoin%
\definecolor{currentfill}{rgb}{0.280255,0.165693,0.476498}%
\pgfsetfillcolor{currentfill}%
\pgfsetfillopacity{0.700000}%
\pgfsetlinewidth{0.000000pt}%
\definecolor{currentstroke}{rgb}{0.000000,0.000000,0.000000}%
\pgfsetstrokecolor{currentstroke}%
\pgfsetdash{}{0pt}%
\pgfpathmoveto{\pgfqpoint{3.281591in}{2.495169in}}%
\pgfpathlineto{\pgfqpoint{3.294529in}{2.485090in}}%
\pgfpathlineto{\pgfqpoint{3.307467in}{2.475236in}}%
\pgfpathlineto{\pgfqpoint{3.320404in}{2.465605in}}%
\pgfpathlineto{\pgfqpoint{3.333340in}{2.456196in}}%
\pgfpathlineto{\pgfqpoint{3.341090in}{2.466084in}}%
\pgfpathlineto{\pgfqpoint{3.348834in}{2.476042in}}%
\pgfpathlineto{\pgfqpoint{3.356571in}{2.486070in}}%
\pgfpathlineto{\pgfqpoint{3.364303in}{2.496169in}}%
\pgfpathlineto{\pgfqpoint{3.351379in}{2.505607in}}%
\pgfpathlineto{\pgfqpoint{3.338454in}{2.515266in}}%
\pgfpathlineto{\pgfqpoint{3.325529in}{2.525148in}}%
\pgfpathlineto{\pgfqpoint{3.312603in}{2.535254in}}%
\pgfpathlineto{\pgfqpoint{3.304859in}{2.525116in}}%
\pgfpathlineto{\pgfqpoint{3.297109in}{2.515057in}}%
\pgfpathlineto{\pgfqpoint{3.289353in}{2.505074in}}%
\pgfpathlineto{\pgfqpoint{3.281591in}{2.495169in}}%
\pgfpathclose%
\pgfusepath{fill}%
\end{pgfscope}%
\begin{pgfscope}%
\pgfpathrectangle{\pgfqpoint{1.254980in}{0.150000in}}{\pgfqpoint{5.490039in}{5.490039in}}%
\pgfusepath{clip}%
\pgfsetbuttcap%
\pgfsetroundjoin%
\definecolor{currentfill}{rgb}{0.248629,0.278775,0.534556}%
\pgfsetfillcolor{currentfill}%
\pgfsetfillopacity{0.700000}%
\pgfsetlinewidth{0.000000pt}%
\definecolor{currentstroke}{rgb}{0.000000,0.000000,0.000000}%
\pgfsetstrokecolor{currentstroke}%
\pgfsetdash{}{0pt}%
\pgfpathmoveto{\pgfqpoint{4.527057in}{2.703049in}}%
\pgfpathlineto{\pgfqpoint{4.540208in}{2.703609in}}%
\pgfpathlineto{\pgfqpoint{4.553369in}{2.704333in}}%
\pgfpathlineto{\pgfqpoint{4.566540in}{2.705220in}}%
\pgfpathlineto{\pgfqpoint{4.579720in}{2.706271in}}%
\pgfpathlineto{\pgfqpoint{4.587073in}{2.715625in}}%
\pgfpathlineto{\pgfqpoint{4.594422in}{2.725049in}}%
\pgfpathlineto{\pgfqpoint{4.601767in}{2.734547in}}%
\pgfpathlineto{\pgfqpoint{4.609108in}{2.744124in}}%
\pgfpathlineto{\pgfqpoint{4.595939in}{2.743436in}}%
\pgfpathlineto{\pgfqpoint{4.582780in}{2.742911in}}%
\pgfpathlineto{\pgfqpoint{4.569630in}{2.742550in}}%
\pgfpathlineto{\pgfqpoint{4.556490in}{2.742352in}}%
\pgfpathlineto{\pgfqpoint{4.549138in}{2.732403in}}%
\pgfpathlineto{\pgfqpoint{4.541781in}{2.722539in}}%
\pgfpathlineto{\pgfqpoint{4.534421in}{2.712756in}}%
\pgfpathlineto{\pgfqpoint{4.527057in}{2.703049in}}%
\pgfpathclose%
\pgfusepath{fill}%
\end{pgfscope}%
\begin{pgfscope}%
\pgfpathrectangle{\pgfqpoint{1.254980in}{0.150000in}}{\pgfqpoint{5.490039in}{5.490039in}}%
\pgfusepath{clip}%
\pgfsetbuttcap%
\pgfsetroundjoin%
\definecolor{currentfill}{rgb}{0.188923,0.410910,0.556326}%
\pgfsetfillcolor{currentfill}%
\pgfsetfillopacity{0.700000}%
\pgfsetlinewidth{0.000000pt}%
\definecolor{currentstroke}{rgb}{0.000000,0.000000,0.000000}%
\pgfsetstrokecolor{currentstroke}%
\pgfsetdash{}{0pt}%
\pgfpathmoveto{\pgfqpoint{5.101568in}{3.008047in}}%
\pgfpathlineto{\pgfqpoint{5.114918in}{3.009918in}}%
\pgfpathlineto{\pgfqpoint{5.128280in}{3.011943in}}%
\pgfpathlineto{\pgfqpoint{5.141655in}{3.014121in}}%
\pgfpathlineto{\pgfqpoint{5.155041in}{3.016453in}}%
\pgfpathlineto{\pgfqpoint{5.162208in}{3.025779in}}%
\pgfpathlineto{\pgfqpoint{5.169374in}{3.035274in}}%
\pgfpathlineto{\pgfqpoint{5.176539in}{3.044944in}}%
\pgfpathlineto{\pgfqpoint{5.183704in}{3.054796in}}%
\pgfpathlineto{\pgfqpoint{5.170337in}{3.053024in}}%
\pgfpathlineto{\pgfqpoint{5.156982in}{3.051405in}}%
\pgfpathlineto{\pgfqpoint{5.143639in}{3.049938in}}%
\pgfpathlineto{\pgfqpoint{5.130307in}{3.048624in}}%
\pgfpathlineto{\pgfqpoint{5.123123in}{3.038204in}}%
\pgfpathlineto{\pgfqpoint{5.115939in}{3.027972in}}%
\pgfpathlineto{\pgfqpoint{5.108754in}{3.017922in}}%
\pgfpathlineto{\pgfqpoint{5.101568in}{3.008047in}}%
\pgfpathclose%
\pgfusepath{fill}%
\end{pgfscope}%
\begin{pgfscope}%
\pgfpathrectangle{\pgfqpoint{1.254980in}{0.150000in}}{\pgfqpoint{5.490039in}{5.490039in}}%
\pgfusepath{clip}%
\pgfsetbuttcap%
\pgfsetroundjoin%
\definecolor{currentfill}{rgb}{0.280868,0.160771,0.472899}%
\pgfsetfillcolor{currentfill}%
\pgfsetfillopacity{0.700000}%
\pgfsetlinewidth{0.000000pt}%
\definecolor{currentstroke}{rgb}{0.000000,0.000000,0.000000}%
\pgfsetstrokecolor{currentstroke}%
\pgfsetdash{}{0pt}%
\pgfpathmoveto{\pgfqpoint{3.900425in}{2.462919in}}%
\pgfpathlineto{\pgfqpoint{3.913399in}{2.459699in}}%
\pgfpathlineto{\pgfqpoint{3.926378in}{2.456662in}}%
\pgfpathlineto{\pgfqpoint{3.939363in}{2.453808in}}%
\pgfpathlineto{\pgfqpoint{3.952354in}{2.451135in}}%
\pgfpathlineto{\pgfqpoint{3.959909in}{2.461228in}}%
\pgfpathlineto{\pgfqpoint{3.967459in}{2.471356in}}%
\pgfpathlineto{\pgfqpoint{3.975005in}{2.481522in}}%
\pgfpathlineto{\pgfqpoint{3.982546in}{2.491729in}}%
\pgfpathlineto{\pgfqpoint{3.969563in}{2.494569in}}%
\pgfpathlineto{\pgfqpoint{3.956586in}{2.497591in}}%
\pgfpathlineto{\pgfqpoint{3.943615in}{2.500795in}}%
\pgfpathlineto{\pgfqpoint{3.930650in}{2.504182in}}%
\pgfpathlineto{\pgfqpoint{3.923101in}{2.493798in}}%
\pgfpathlineto{\pgfqpoint{3.915547in}{2.483460in}}%
\pgfpathlineto{\pgfqpoint{3.907988in}{2.473168in}}%
\pgfpathlineto{\pgfqpoint{3.900425in}{2.462919in}}%
\pgfpathclose%
\pgfusepath{fill}%
\end{pgfscope}%
\begin{pgfscope}%
\pgfpathrectangle{\pgfqpoint{1.254980in}{0.150000in}}{\pgfqpoint{5.490039in}{5.490039in}}%
\pgfusepath{clip}%
\pgfsetbuttcap%
\pgfsetroundjoin%
\definecolor{currentfill}{rgb}{0.180629,0.429975,0.557282}%
\pgfsetfillcolor{currentfill}%
\pgfsetfillopacity{0.700000}%
\pgfsetlinewidth{0.000000pt}%
\definecolor{currentstroke}{rgb}{0.000000,0.000000,0.000000}%
\pgfsetstrokecolor{currentstroke}%
\pgfsetdash{}{0pt}%
\pgfpathmoveto{\pgfqpoint{5.183704in}{3.054796in}}%
\pgfpathlineto{\pgfqpoint{5.197083in}{3.056721in}}%
\pgfpathlineto{\pgfqpoint{5.210474in}{3.058798in}}%
\pgfpathlineto{\pgfqpoint{5.223877in}{3.061027in}}%
\pgfpathlineto{\pgfqpoint{5.237293in}{3.063408in}}%
\pgfpathlineto{\pgfqpoint{5.244437in}{3.072872in}}%
\pgfpathlineto{\pgfqpoint{5.251581in}{3.082523in}}%
\pgfpathlineto{\pgfqpoint{5.258725in}{3.092370in}}%
\pgfpathlineto{\pgfqpoint{5.265869in}{3.102417in}}%
\pgfpathlineto{\pgfqpoint{5.252474in}{3.100623in}}%
\pgfpathlineto{\pgfqpoint{5.239092in}{3.098981in}}%
\pgfpathlineto{\pgfqpoint{5.225721in}{3.097490in}}%
\pgfpathlineto{\pgfqpoint{5.212362in}{3.096151in}}%
\pgfpathlineto{\pgfqpoint{5.205197in}{3.085507in}}%
\pgfpathlineto{\pgfqpoint{5.198033in}{3.075071in}}%
\pgfpathlineto{\pgfqpoint{5.190868in}{3.064836in}}%
\pgfpathlineto{\pgfqpoint{5.183704in}{3.054796in}}%
\pgfpathclose%
\pgfusepath{fill}%
\end{pgfscope}%
\begin{pgfscope}%
\pgfpathrectangle{\pgfqpoint{1.254980in}{0.150000in}}{\pgfqpoint{5.490039in}{5.490039in}}%
\pgfusepath{clip}%
\pgfsetbuttcap%
\pgfsetroundjoin%
\definecolor{currentfill}{rgb}{0.255645,0.260703,0.528312}%
\pgfsetfillcolor{currentfill}%
\pgfsetfillopacity{0.700000}%
\pgfsetlinewidth{0.000000pt}%
\definecolor{currentstroke}{rgb}{0.000000,0.000000,0.000000}%
\pgfsetstrokecolor{currentstroke}%
\pgfsetdash{}{0pt}%
\pgfpathmoveto{\pgfqpoint{4.445004in}{2.662974in}}%
\pgfpathlineto{\pgfqpoint{4.458129in}{2.663208in}}%
\pgfpathlineto{\pgfqpoint{4.471263in}{2.663609in}}%
\pgfpathlineto{\pgfqpoint{4.484407in}{2.664174in}}%
\pgfpathlineto{\pgfqpoint{4.497560in}{2.664905in}}%
\pgfpathlineto{\pgfqpoint{4.504940in}{2.674346in}}%
\pgfpathlineto{\pgfqpoint{4.512317in}{2.683848in}}%
\pgfpathlineto{\pgfqpoint{4.519689in}{2.693414in}}%
\pgfpathlineto{\pgfqpoint{4.527057in}{2.703049in}}%
\pgfpathlineto{\pgfqpoint{4.513915in}{2.702653in}}%
\pgfpathlineto{\pgfqpoint{4.500782in}{2.702423in}}%
\pgfpathlineto{\pgfqpoint{4.487659in}{2.702357in}}%
\pgfpathlineto{\pgfqpoint{4.474545in}{2.702457in}}%
\pgfpathlineto{\pgfqpoint{4.467166in}{2.692478in}}%
\pgfpathlineto{\pgfqpoint{4.459783in}{2.682574in}}%
\pgfpathlineto{\pgfqpoint{4.452395in}{2.672740in}}%
\pgfpathlineto{\pgfqpoint{4.445004in}{2.662974in}}%
\pgfpathclose%
\pgfusepath{fill}%
\end{pgfscope}%
\begin{pgfscope}%
\pgfpathrectangle{\pgfqpoint{1.254980in}{0.150000in}}{\pgfqpoint{5.490039in}{5.490039in}}%
\pgfusepath{clip}%
\pgfsetbuttcap%
\pgfsetroundjoin%
\definecolor{currentfill}{rgb}{0.252194,0.269783,0.531579}%
\pgfsetfillcolor{currentfill}%
\pgfsetfillopacity{0.700000}%
\pgfsetlinewidth{0.000000pt}%
\definecolor{currentstroke}{rgb}{0.000000,0.000000,0.000000}%
\pgfsetstrokecolor{currentstroke}%
\pgfsetdash{}{0pt}%
\pgfpathmoveto{\pgfqpoint{2.990908in}{2.707676in}}%
\pgfpathlineto{\pgfqpoint{3.003925in}{2.692733in}}%
\pgfpathlineto{\pgfqpoint{3.016936in}{2.678053in}}%
\pgfpathlineto{\pgfqpoint{3.029942in}{2.663632in}}%
\pgfpathlineto{\pgfqpoint{3.042943in}{2.649469in}}%
\pgfpathlineto{\pgfqpoint{3.050790in}{2.659031in}}%
\pgfpathlineto{\pgfqpoint{3.058629in}{2.668695in}}%
\pgfpathlineto{\pgfqpoint{3.066461in}{2.678461in}}%
\pgfpathlineto{\pgfqpoint{3.074286in}{2.688330in}}%
\pgfpathlineto{\pgfqpoint{3.061301in}{2.702491in}}%
\pgfpathlineto{\pgfqpoint{3.048310in}{2.716910in}}%
\pgfpathlineto{\pgfqpoint{3.035315in}{2.731589in}}%
\pgfpathlineto{\pgfqpoint{3.022315in}{2.746530in}}%
\pgfpathlineto{\pgfqpoint{3.014474in}{2.736652in}}%
\pgfpathlineto{\pgfqpoint{3.006626in}{2.726884in}}%
\pgfpathlineto{\pgfqpoint{2.998771in}{2.717226in}}%
\pgfpathlineto{\pgfqpoint{2.990908in}{2.707676in}}%
\pgfpathclose%
\pgfusepath{fill}%
\end{pgfscope}%
\begin{pgfscope}%
\pgfpathrectangle{\pgfqpoint{1.254980in}{0.150000in}}{\pgfqpoint{5.490039in}{5.490039in}}%
\pgfusepath{clip}%
\pgfsetbuttcap%
\pgfsetroundjoin%
\definecolor{currentfill}{rgb}{0.241237,0.296485,0.539709}%
\pgfsetfillcolor{currentfill}%
\pgfsetfillopacity{0.700000}%
\pgfsetlinewidth{0.000000pt}%
\definecolor{currentstroke}{rgb}{0.000000,0.000000,0.000000}%
\pgfsetstrokecolor{currentstroke}%
\pgfsetdash{}{0pt}%
\pgfpathmoveto{\pgfqpoint{2.938783in}{2.770117in}}%
\pgfpathlineto{\pgfqpoint{2.951824in}{2.754102in}}%
\pgfpathlineto{\pgfqpoint{2.964858in}{2.738358in}}%
\pgfpathlineto{\pgfqpoint{2.977886in}{2.722884in}}%
\pgfpathlineto{\pgfqpoint{2.990908in}{2.707676in}}%
\pgfpathlineto{\pgfqpoint{2.998771in}{2.717226in}}%
\pgfpathlineto{\pgfqpoint{3.006626in}{2.726884in}}%
\pgfpathlineto{\pgfqpoint{3.014474in}{2.736652in}}%
\pgfpathlineto{\pgfqpoint{3.022315in}{2.746530in}}%
\pgfpathlineto{\pgfqpoint{3.009309in}{2.761735in}}%
\pgfpathlineto{\pgfqpoint{2.996298in}{2.777207in}}%
\pgfpathlineto{\pgfqpoint{2.983280in}{2.792949in}}%
\pgfpathlineto{\pgfqpoint{2.970256in}{2.808962in}}%
\pgfpathlineto{\pgfqpoint{2.962400in}{2.799076in}}%
\pgfpathlineto{\pgfqpoint{2.954535in}{2.789307in}}%
\pgfpathlineto{\pgfqpoint{2.946663in}{2.779654in}}%
\pgfpathlineto{\pgfqpoint{2.938783in}{2.770117in}}%
\pgfpathclose%
\pgfusepath{fill}%
\end{pgfscope}%
\begin{pgfscope}%
\pgfpathrectangle{\pgfqpoint{1.254980in}{0.150000in}}{\pgfqpoint{5.490039in}{5.490039in}}%
\pgfusepath{clip}%
\pgfsetbuttcap%
\pgfsetroundjoin%
\definecolor{currentfill}{rgb}{0.172719,0.448791,0.557885}%
\pgfsetfillcolor{currentfill}%
\pgfsetfillopacity{0.700000}%
\pgfsetlinewidth{0.000000pt}%
\definecolor{currentstroke}{rgb}{0.000000,0.000000,0.000000}%
\pgfsetstrokecolor{currentstroke}%
\pgfsetdash{}{0pt}%
\pgfpathmoveto{\pgfqpoint{5.265869in}{3.102417in}}%
\pgfpathlineto{\pgfqpoint{5.279276in}{3.104363in}}%
\pgfpathlineto{\pgfqpoint{5.292695in}{3.106459in}}%
\pgfpathlineto{\pgfqpoint{5.306127in}{3.108706in}}%
\pgfpathlineto{\pgfqpoint{5.319572in}{3.111105in}}%
\pgfpathlineto{\pgfqpoint{5.326695in}{3.120756in}}%
\pgfpathlineto{\pgfqpoint{5.333818in}{3.130616in}}%
\pgfpathlineto{\pgfqpoint{5.340943in}{3.140691in}}%
\pgfpathlineto{\pgfqpoint{5.348069in}{3.150988in}}%
\pgfpathlineto{\pgfqpoint{5.334647in}{3.149205in}}%
\pgfpathlineto{\pgfqpoint{5.321237in}{3.147572in}}%
\pgfpathlineto{\pgfqpoint{5.307839in}{3.146089in}}%
\pgfpathlineto{\pgfqpoint{5.294454in}{3.144757in}}%
\pgfpathlineto{\pgfqpoint{5.287306in}{3.133836in}}%
\pgfpathlineto{\pgfqpoint{5.280159in}{3.123144in}}%
\pgfpathlineto{\pgfqpoint{5.273014in}{3.112673in}}%
\pgfpathlineto{\pgfqpoint{5.265869in}{3.102417in}}%
\pgfpathclose%
\pgfusepath{fill}%
\end{pgfscope}%
\begin{pgfscope}%
\pgfpathrectangle{\pgfqpoint{1.254980in}{0.150000in}}{\pgfqpoint{5.490039in}{5.490039in}}%
\pgfusepath{clip}%
\pgfsetbuttcap%
\pgfsetroundjoin%
\definecolor{currentfill}{rgb}{0.262138,0.242286,0.520837}%
\pgfsetfillcolor{currentfill}%
\pgfsetfillopacity{0.700000}%
\pgfsetlinewidth{0.000000pt}%
\definecolor{currentstroke}{rgb}{0.000000,0.000000,0.000000}%
\pgfsetstrokecolor{currentstroke}%
\pgfsetdash{}{0pt}%
\pgfpathmoveto{\pgfqpoint{4.362945in}{2.624009in}}%
\pgfpathlineto{\pgfqpoint{4.376045in}{2.623882in}}%
\pgfpathlineto{\pgfqpoint{4.389153in}{2.623923in}}%
\pgfpathlineto{\pgfqpoint{4.402270in}{2.624131in}}%
\pgfpathlineto{\pgfqpoint{4.415397in}{2.624506in}}%
\pgfpathlineto{\pgfqpoint{4.422805in}{2.634041in}}%
\pgfpathlineto{\pgfqpoint{4.430209in}{2.643628in}}%
\pgfpathlineto{\pgfqpoint{4.437609in}{2.653271in}}%
\pgfpathlineto{\pgfqpoint{4.445004in}{2.662974in}}%
\pgfpathlineto{\pgfqpoint{4.431888in}{2.662906in}}%
\pgfpathlineto{\pgfqpoint{4.418781in}{2.663005in}}%
\pgfpathlineto{\pgfqpoint{4.405683in}{2.663271in}}%
\pgfpathlineto{\pgfqpoint{4.392593in}{2.663705in}}%
\pgfpathlineto{\pgfqpoint{4.385187in}{2.653685in}}%
\pgfpathlineto{\pgfqpoint{4.377777in}{2.643732in}}%
\pgfpathlineto{\pgfqpoint{4.370363in}{2.633841in}}%
\pgfpathlineto{\pgfqpoint{4.362945in}{2.624009in}}%
\pgfpathclose%
\pgfusepath{fill}%
\end{pgfscope}%
\begin{pgfscope}%
\pgfpathrectangle{\pgfqpoint{1.254980in}{0.150000in}}{\pgfqpoint{5.490039in}{5.490039in}}%
\pgfusepath{clip}%
\pgfsetbuttcap%
\pgfsetroundjoin%
\definecolor{currentfill}{rgb}{0.262138,0.242286,0.520837}%
\pgfsetfillcolor{currentfill}%
\pgfsetfillopacity{0.700000}%
\pgfsetlinewidth{0.000000pt}%
\definecolor{currentstroke}{rgb}{0.000000,0.000000,0.000000}%
\pgfsetstrokecolor{currentstroke}%
\pgfsetdash{}{0pt}%
\pgfpathmoveto{\pgfqpoint{3.042943in}{2.649469in}}%
\pgfpathlineto{\pgfqpoint{3.055940in}{2.635563in}}%
\pgfpathlineto{\pgfqpoint{3.068932in}{2.621909in}}%
\pgfpathlineto{\pgfqpoint{3.081919in}{2.608508in}}%
\pgfpathlineto{\pgfqpoint{3.094903in}{2.595355in}}%
\pgfpathlineto{\pgfqpoint{3.102734in}{2.604929in}}%
\pgfpathlineto{\pgfqpoint{3.110558in}{2.614597in}}%
\pgfpathlineto{\pgfqpoint{3.118375in}{2.624361in}}%
\pgfpathlineto{\pgfqpoint{3.126185in}{2.634221in}}%
\pgfpathlineto{\pgfqpoint{3.113216in}{2.647372in}}%
\pgfpathlineto{\pgfqpoint{3.100244in}{2.660773in}}%
\pgfpathlineto{\pgfqpoint{3.087267in}{2.674425in}}%
\pgfpathlineto{\pgfqpoint{3.074286in}{2.688330in}}%
\pgfpathlineto{\pgfqpoint{3.066461in}{2.678461in}}%
\pgfpathlineto{\pgfqpoint{3.058629in}{2.668695in}}%
\pgfpathlineto{\pgfqpoint{3.050790in}{2.659031in}}%
\pgfpathlineto{\pgfqpoint{3.042943in}{2.649469in}}%
\pgfpathclose%
\pgfusepath{fill}%
\end{pgfscope}%
\begin{pgfscope}%
\pgfpathrectangle{\pgfqpoint{1.254980in}{0.150000in}}{\pgfqpoint{5.490039in}{5.490039in}}%
\pgfusepath{clip}%
\pgfsetbuttcap%
\pgfsetroundjoin%
\definecolor{currentfill}{rgb}{0.227802,0.326594,0.546532}%
\pgfsetfillcolor{currentfill}%
\pgfsetfillopacity{0.700000}%
\pgfsetlinewidth{0.000000pt}%
\definecolor{currentstroke}{rgb}{0.000000,0.000000,0.000000}%
\pgfsetstrokecolor{currentstroke}%
\pgfsetdash{}{0pt}%
\pgfpathmoveto{\pgfqpoint{2.886552in}{2.836945in}}%
\pgfpathlineto{\pgfqpoint{2.899621in}{2.819818in}}%
\pgfpathlineto{\pgfqpoint{2.912682in}{2.802973in}}%
\pgfpathlineto{\pgfqpoint{2.925736in}{2.786407in}}%
\pgfpathlineto{\pgfqpoint{2.938783in}{2.770117in}}%
\pgfpathlineto{\pgfqpoint{2.946663in}{2.779654in}}%
\pgfpathlineto{\pgfqpoint{2.954535in}{2.789307in}}%
\pgfpathlineto{\pgfqpoint{2.962400in}{2.799076in}}%
\pgfpathlineto{\pgfqpoint{2.970256in}{2.808962in}}%
\pgfpathlineto{\pgfqpoint{2.957226in}{2.825249in}}%
\pgfpathlineto{\pgfqpoint{2.944189in}{2.841812in}}%
\pgfpathlineto{\pgfqpoint{2.931145in}{2.858654in}}%
\pgfpathlineto{\pgfqpoint{2.918094in}{2.875778in}}%
\pgfpathlineto{\pgfqpoint{2.910221in}{2.865885in}}%
\pgfpathlineto{\pgfqpoint{2.902339in}{2.856115in}}%
\pgfpathlineto{\pgfqpoint{2.894449in}{2.846468in}}%
\pgfpathlineto{\pgfqpoint{2.886552in}{2.836945in}}%
\pgfpathclose%
\pgfusepath{fill}%
\end{pgfscope}%
\begin{pgfscope}%
\pgfpathrectangle{\pgfqpoint{1.254980in}{0.150000in}}{\pgfqpoint{5.490039in}{5.490039in}}%
\pgfusepath{clip}%
\pgfsetbuttcap%
\pgfsetroundjoin%
\definecolor{currentfill}{rgb}{0.163625,0.471133,0.558148}%
\pgfsetfillcolor{currentfill}%
\pgfsetfillopacity{0.700000}%
\pgfsetlinewidth{0.000000pt}%
\definecolor{currentstroke}{rgb}{0.000000,0.000000,0.000000}%
\pgfsetstrokecolor{currentstroke}%
\pgfsetdash{}{0pt}%
\pgfpathmoveto{\pgfqpoint{5.348069in}{3.150988in}}%
\pgfpathlineto{\pgfqpoint{5.361503in}{3.152921in}}%
\pgfpathlineto{\pgfqpoint{5.374951in}{3.155005in}}%
\pgfpathlineto{\pgfqpoint{5.388411in}{3.157239in}}%
\pgfpathlineto{\pgfqpoint{5.401883in}{3.159622in}}%
\pgfpathlineto{\pgfqpoint{5.408987in}{3.169516in}}%
\pgfpathlineto{\pgfqpoint{5.416093in}{3.179640in}}%
\pgfpathlineto{\pgfqpoint{5.423201in}{3.190002in}}%
\pgfpathlineto{\pgfqpoint{5.430311in}{3.200608in}}%
\pgfpathlineto{\pgfqpoint{5.416862in}{3.198868in}}%
\pgfpathlineto{\pgfqpoint{5.403426in}{3.197276in}}%
\pgfpathlineto{\pgfqpoint{5.390002in}{3.195835in}}%
\pgfpathlineto{\pgfqpoint{5.376590in}{3.194542in}}%
\pgfpathlineto{\pgfqpoint{5.369457in}{3.183285in}}%
\pgfpathlineto{\pgfqpoint{5.362325in}{3.172278in}}%
\pgfpathlineto{\pgfqpoint{5.355196in}{3.161515in}}%
\pgfpathlineto{\pgfqpoint{5.348069in}{3.150988in}}%
\pgfpathclose%
\pgfusepath{fill}%
\end{pgfscope}%
\begin{pgfscope}%
\pgfpathrectangle{\pgfqpoint{1.254980in}{0.150000in}}{\pgfqpoint{5.490039in}{5.490039in}}%
\pgfusepath{clip}%
\pgfsetbuttcap%
\pgfsetroundjoin%
\definecolor{currentfill}{rgb}{0.267968,0.223549,0.512008}%
\pgfsetfillcolor{currentfill}%
\pgfsetfillopacity{0.700000}%
\pgfsetlinewidth{0.000000pt}%
\definecolor{currentstroke}{rgb}{0.000000,0.000000,0.000000}%
\pgfsetstrokecolor{currentstroke}%
\pgfsetdash{}{0pt}%
\pgfpathmoveto{\pgfqpoint{4.280875in}{2.586284in}}%
\pgfpathlineto{\pgfqpoint{4.293951in}{2.585759in}}%
\pgfpathlineto{\pgfqpoint{4.307034in}{2.585404in}}%
\pgfpathlineto{\pgfqpoint{4.320127in}{2.585219in}}%
\pgfpathlineto{\pgfqpoint{4.333227in}{2.585203in}}%
\pgfpathlineto{\pgfqpoint{4.340664in}{2.594833in}}%
\pgfpathlineto{\pgfqpoint{4.348095in}{2.604509in}}%
\pgfpathlineto{\pgfqpoint{4.355522in}{2.614233in}}%
\pgfpathlineto{\pgfqpoint{4.362945in}{2.624009in}}%
\pgfpathlineto{\pgfqpoint{4.349854in}{2.624305in}}%
\pgfpathlineto{\pgfqpoint{4.336771in}{2.624769in}}%
\pgfpathlineto{\pgfqpoint{4.323697in}{2.625403in}}%
\pgfpathlineto{\pgfqpoint{4.310630in}{2.626206in}}%
\pgfpathlineto{\pgfqpoint{4.303198in}{2.616141in}}%
\pgfpathlineto{\pgfqpoint{4.295761in}{2.606134in}}%
\pgfpathlineto{\pgfqpoint{4.288320in}{2.596183in}}%
\pgfpathlineto{\pgfqpoint{4.280875in}{2.586284in}}%
\pgfpathclose%
\pgfusepath{fill}%
\end{pgfscope}%
\begin{pgfscope}%
\pgfpathrectangle{\pgfqpoint{1.254980in}{0.150000in}}{\pgfqpoint{5.490039in}{5.490039in}}%
\pgfusepath{clip}%
\pgfsetbuttcap%
\pgfsetroundjoin%
\definecolor{currentfill}{rgb}{0.282623,0.140926,0.457517}%
\pgfsetfillcolor{currentfill}%
\pgfsetfillopacity{0.700000}%
\pgfsetlinewidth{0.000000pt}%
\definecolor{currentstroke}{rgb}{0.000000,0.000000,0.000000}%
\pgfsetstrokecolor{currentstroke}%
\pgfsetdash{}{0pt}%
\pgfpathmoveto{\pgfqpoint{3.467705in}{2.428463in}}%
\pgfpathlineto{\pgfqpoint{3.480634in}{2.420953in}}%
\pgfpathlineto{\pgfqpoint{3.493564in}{2.413652in}}%
\pgfpathlineto{\pgfqpoint{3.506496in}{2.406556in}}%
\pgfpathlineto{\pgfqpoint{3.519430in}{2.399666in}}%
\pgfpathlineto{\pgfqpoint{3.527122in}{2.409700in}}%
\pgfpathlineto{\pgfqpoint{3.534809in}{2.419786in}}%
\pgfpathlineto{\pgfqpoint{3.542490in}{2.429925in}}%
\pgfpathlineto{\pgfqpoint{3.550166in}{2.440119in}}%
\pgfpathlineto{\pgfqpoint{3.537243in}{2.447065in}}%
\pgfpathlineto{\pgfqpoint{3.524322in}{2.454217in}}%
\pgfpathlineto{\pgfqpoint{3.511402in}{2.461575in}}%
\pgfpathlineto{\pgfqpoint{3.498484in}{2.469141in}}%
\pgfpathlineto{\pgfqpoint{3.490797in}{2.458881in}}%
\pgfpathlineto{\pgfqpoint{3.483105in}{2.448682in}}%
\pgfpathlineto{\pgfqpoint{3.475408in}{2.438543in}}%
\pgfpathlineto{\pgfqpoint{3.467705in}{2.428463in}}%
\pgfpathclose%
\pgfusepath{fill}%
\end{pgfscope}%
\begin{pgfscope}%
\pgfpathrectangle{\pgfqpoint{1.254980in}{0.150000in}}{\pgfqpoint{5.490039in}{5.490039in}}%
\pgfusepath{clip}%
\pgfsetbuttcap%
\pgfsetroundjoin%
\definecolor{currentfill}{rgb}{0.269308,0.218818,0.509577}%
\pgfsetfillcolor{currentfill}%
\pgfsetfillopacity{0.700000}%
\pgfsetlinewidth{0.000000pt}%
\definecolor{currentstroke}{rgb}{0.000000,0.000000,0.000000}%
\pgfsetstrokecolor{currentstroke}%
\pgfsetdash{}{0pt}%
\pgfpathmoveto{\pgfqpoint{3.094903in}{2.595355in}}%
\pgfpathlineto{\pgfqpoint{3.107883in}{2.582451in}}%
\pgfpathlineto{\pgfqpoint{3.120860in}{2.569792in}}%
\pgfpathlineto{\pgfqpoint{3.133833in}{2.557376in}}%
\pgfpathlineto{\pgfqpoint{3.146803in}{2.545202in}}%
\pgfpathlineto{\pgfqpoint{3.154619in}{2.554787in}}%
\pgfpathlineto{\pgfqpoint{3.162428in}{2.564460in}}%
\pgfpathlineto{\pgfqpoint{3.170231in}{2.574221in}}%
\pgfpathlineto{\pgfqpoint{3.178026in}{2.584072in}}%
\pgfpathlineto{\pgfqpoint{3.165071in}{2.596245in}}%
\pgfpathlineto{\pgfqpoint{3.152112in}{2.608659in}}%
\pgfpathlineto{\pgfqpoint{3.139150in}{2.621318in}}%
\pgfpathlineto{\pgfqpoint{3.126185in}{2.634221in}}%
\pgfpathlineto{\pgfqpoint{3.118375in}{2.624361in}}%
\pgfpathlineto{\pgfqpoint{3.110558in}{2.614597in}}%
\pgfpathlineto{\pgfqpoint{3.102734in}{2.604929in}}%
\pgfpathlineto{\pgfqpoint{3.094903in}{2.595355in}}%
\pgfpathclose%
\pgfusepath{fill}%
\end{pgfscope}%
\begin{pgfscope}%
\pgfpathrectangle{\pgfqpoint{1.254980in}{0.150000in}}{\pgfqpoint{5.490039in}{5.490039in}}%
\pgfusepath{clip}%
\pgfsetbuttcap%
\pgfsetroundjoin%
\definecolor{currentfill}{rgb}{0.282290,0.145912,0.461510}%
\pgfsetfillcolor{currentfill}%
\pgfsetfillopacity{0.700000}%
\pgfsetlinewidth{0.000000pt}%
\definecolor{currentstroke}{rgb}{0.000000,0.000000,0.000000}%
\pgfsetstrokecolor{currentstroke}%
\pgfsetdash{}{0pt}%
\pgfpathmoveto{\pgfqpoint{3.818243in}{2.436480in}}%
\pgfpathlineto{\pgfqpoint{3.831205in}{2.432658in}}%
\pgfpathlineto{\pgfqpoint{3.844172in}{2.429023in}}%
\pgfpathlineto{\pgfqpoint{3.857144in}{2.425573in}}%
\pgfpathlineto{\pgfqpoint{3.870121in}{2.422309in}}%
\pgfpathlineto{\pgfqpoint{3.877705in}{2.432407in}}%
\pgfpathlineto{\pgfqpoint{3.885283in}{2.442540in}}%
\pgfpathlineto{\pgfqpoint{3.892856in}{2.452710in}}%
\pgfpathlineto{\pgfqpoint{3.900425in}{2.462919in}}%
\pgfpathlineto{\pgfqpoint{3.887456in}{2.466323in}}%
\pgfpathlineto{\pgfqpoint{3.874492in}{2.469912in}}%
\pgfpathlineto{\pgfqpoint{3.861534in}{2.473687in}}%
\pgfpathlineto{\pgfqpoint{3.848580in}{2.477649in}}%
\pgfpathlineto{\pgfqpoint{3.841003in}{2.467290in}}%
\pgfpathlineto{\pgfqpoint{3.833421in}{2.456977in}}%
\pgfpathlineto{\pgfqpoint{3.825834in}{2.446708in}}%
\pgfpathlineto{\pgfqpoint{3.818243in}{2.436480in}}%
\pgfpathclose%
\pgfusepath{fill}%
\end{pgfscope}%
\begin{pgfscope}%
\pgfpathrectangle{\pgfqpoint{1.254980in}{0.150000in}}{\pgfqpoint{5.490039in}{5.490039in}}%
\pgfusepath{clip}%
\pgfsetbuttcap%
\pgfsetroundjoin%
\definecolor{currentfill}{rgb}{0.214298,0.355619,0.551184}%
\pgfsetfillcolor{currentfill}%
\pgfsetfillopacity{0.700000}%
\pgfsetlinewidth{0.000000pt}%
\definecolor{currentstroke}{rgb}{0.000000,0.000000,0.000000}%
\pgfsetstrokecolor{currentstroke}%
\pgfsetdash{}{0pt}%
\pgfpathmoveto{\pgfqpoint{2.834198in}{2.908324in}}%
\pgfpathlineto{\pgfqpoint{2.847299in}{2.890043in}}%
\pgfpathlineto{\pgfqpoint{2.860391in}{2.872055in}}%
\pgfpathlineto{\pgfqpoint{2.873476in}{2.854357in}}%
\pgfpathlineto{\pgfqpoint{2.886552in}{2.836945in}}%
\pgfpathlineto{\pgfqpoint{2.894449in}{2.846468in}}%
\pgfpathlineto{\pgfqpoint{2.902339in}{2.856115in}}%
\pgfpathlineto{\pgfqpoint{2.910221in}{2.865885in}}%
\pgfpathlineto{\pgfqpoint{2.918094in}{2.875778in}}%
\pgfpathlineto{\pgfqpoint{2.905035in}{2.893187in}}%
\pgfpathlineto{\pgfqpoint{2.891969in}{2.910882in}}%
\pgfpathlineto{\pgfqpoint{2.878894in}{2.928866in}}%
\pgfpathlineto{\pgfqpoint{2.865812in}{2.947144in}}%
\pgfpathlineto{\pgfqpoint{2.857921in}{2.937242in}}%
\pgfpathlineto{\pgfqpoint{2.850021in}{2.927472in}}%
\pgfpathlineto{\pgfqpoint{2.842114in}{2.917833in}}%
\pgfpathlineto{\pgfqpoint{2.834198in}{2.908324in}}%
\pgfpathclose%
\pgfusepath{fill}%
\end{pgfscope}%
\begin{pgfscope}%
\pgfpathrectangle{\pgfqpoint{1.254980in}{0.150000in}}{\pgfqpoint{5.490039in}{5.490039in}}%
\pgfusepath{clip}%
\pgfsetbuttcap%
\pgfsetroundjoin%
\definecolor{currentfill}{rgb}{0.282884,0.135920,0.453427}%
\pgfsetfillcolor{currentfill}%
\pgfsetfillopacity{0.700000}%
\pgfsetlinewidth{0.000000pt}%
\definecolor{currentstroke}{rgb}{0.000000,0.000000,0.000000}%
\pgfsetstrokecolor{currentstroke}%
\pgfsetdash{}{0pt}%
\pgfpathmoveto{\pgfqpoint{3.601882in}{2.414359in}}%
\pgfpathlineto{\pgfqpoint{3.614817in}{2.408420in}}%
\pgfpathlineto{\pgfqpoint{3.627755in}{2.402680in}}%
\pgfpathlineto{\pgfqpoint{3.640696in}{2.397137in}}%
\pgfpathlineto{\pgfqpoint{3.653640in}{2.391790in}}%
\pgfpathlineto{\pgfqpoint{3.661291in}{2.401891in}}%
\pgfpathlineto{\pgfqpoint{3.668936in}{2.412036in}}%
\pgfpathlineto{\pgfqpoint{3.676576in}{2.422225in}}%
\pgfpathlineto{\pgfqpoint{3.684212in}{2.432459in}}%
\pgfpathlineto{\pgfqpoint{3.671277in}{2.437890in}}%
\pgfpathlineto{\pgfqpoint{3.658345in}{2.443518in}}%
\pgfpathlineto{\pgfqpoint{3.645417in}{2.449342in}}%
\pgfpathlineto{\pgfqpoint{3.632492in}{2.455365in}}%
\pgfpathlineto{\pgfqpoint{3.624847in}{2.445036in}}%
\pgfpathlineto{\pgfqpoint{3.617197in}{2.434760in}}%
\pgfpathlineto{\pgfqpoint{3.609542in}{2.424534in}}%
\pgfpathlineto{\pgfqpoint{3.601882in}{2.414359in}}%
\pgfpathclose%
\pgfusepath{fill}%
\end{pgfscope}%
\begin{pgfscope}%
\pgfpathrectangle{\pgfqpoint{1.254980in}{0.150000in}}{\pgfqpoint{5.490039in}{5.490039in}}%
\pgfusepath{clip}%
\pgfsetbuttcap%
\pgfsetroundjoin%
\definecolor{currentfill}{rgb}{0.271828,0.209303,0.504434}%
\pgfsetfillcolor{currentfill}%
\pgfsetfillopacity{0.700000}%
\pgfsetlinewidth{0.000000pt}%
\definecolor{currentstroke}{rgb}{0.000000,0.000000,0.000000}%
\pgfsetstrokecolor{currentstroke}%
\pgfsetdash{}{0pt}%
\pgfpathmoveto{\pgfqpoint{4.198788in}{2.549951in}}%
\pgfpathlineto{\pgfqpoint{4.211841in}{2.548992in}}%
\pgfpathlineto{\pgfqpoint{4.224902in}{2.548204in}}%
\pgfpathlineto{\pgfqpoint{4.237971in}{2.547589in}}%
\pgfpathlineto{\pgfqpoint{4.251047in}{2.547145in}}%
\pgfpathlineto{\pgfqpoint{4.258511in}{2.556867in}}%
\pgfpathlineto{\pgfqpoint{4.265970in}{2.566629in}}%
\pgfpathlineto{\pgfqpoint{4.273425in}{2.576434in}}%
\pgfpathlineto{\pgfqpoint{4.280875in}{2.586284in}}%
\pgfpathlineto{\pgfqpoint{4.267807in}{2.586980in}}%
\pgfpathlineto{\pgfqpoint{4.254748in}{2.587846in}}%
\pgfpathlineto{\pgfqpoint{4.241696in}{2.588885in}}%
\pgfpathlineto{\pgfqpoint{4.228651in}{2.590095in}}%
\pgfpathlineto{\pgfqpoint{4.221192in}{2.579984in}}%
\pgfpathlineto{\pgfqpoint{4.213729in}{2.569924in}}%
\pgfpathlineto{\pgfqpoint{4.206261in}{2.559915in}}%
\pgfpathlineto{\pgfqpoint{4.198788in}{2.549951in}}%
\pgfpathclose%
\pgfusepath{fill}%
\end{pgfscope}%
\begin{pgfscope}%
\pgfpathrectangle{\pgfqpoint{1.254980in}{0.150000in}}{\pgfqpoint{5.490039in}{5.490039in}}%
\pgfusepath{clip}%
\pgfsetbuttcap%
\pgfsetroundjoin%
\definecolor{currentfill}{rgb}{0.281887,0.150881,0.465405}%
\pgfsetfillcolor{currentfill}%
\pgfsetfillopacity{0.700000}%
\pgfsetlinewidth{0.000000pt}%
\definecolor{currentstroke}{rgb}{0.000000,0.000000,0.000000}%
\pgfsetstrokecolor{currentstroke}%
\pgfsetdash{}{0pt}%
\pgfpathmoveto{\pgfqpoint{3.333340in}{2.456196in}}%
\pgfpathlineto{\pgfqpoint{3.346277in}{2.447006in}}%
\pgfpathlineto{\pgfqpoint{3.359213in}{2.438035in}}%
\pgfpathlineto{\pgfqpoint{3.372149in}{2.429281in}}%
\pgfpathlineto{\pgfqpoint{3.385085in}{2.420743in}}%
\pgfpathlineto{\pgfqpoint{3.392822in}{2.430613in}}%
\pgfpathlineto{\pgfqpoint{3.400554in}{2.440546in}}%
\pgfpathlineto{\pgfqpoint{3.408280in}{2.450542in}}%
\pgfpathlineto{\pgfqpoint{3.416000in}{2.460603in}}%
\pgfpathlineto{\pgfqpoint{3.403075in}{2.469170in}}%
\pgfpathlineto{\pgfqpoint{3.390151in}{2.477952in}}%
\pgfpathlineto{\pgfqpoint{3.377227in}{2.486951in}}%
\pgfpathlineto{\pgfqpoint{3.364303in}{2.496169in}}%
\pgfpathlineto{\pgfqpoint{3.356571in}{2.486070in}}%
\pgfpathlineto{\pgfqpoint{3.348834in}{2.476042in}}%
\pgfpathlineto{\pgfqpoint{3.341090in}{2.466084in}}%
\pgfpathlineto{\pgfqpoint{3.333340in}{2.456196in}}%
\pgfpathclose%
\pgfusepath{fill}%
\end{pgfscope}%
\begin{pgfscope}%
\pgfpathrectangle{\pgfqpoint{1.254980in}{0.150000in}}{\pgfqpoint{5.490039in}{5.490039in}}%
\pgfusepath{clip}%
\pgfsetbuttcap%
\pgfsetroundjoin%
\definecolor{currentfill}{rgb}{0.156270,0.489624,0.557936}%
\pgfsetfillcolor{currentfill}%
\pgfsetfillopacity{0.700000}%
\pgfsetlinewidth{0.000000pt}%
\definecolor{currentstroke}{rgb}{0.000000,0.000000,0.000000}%
\pgfsetstrokecolor{currentstroke}%
\pgfsetdash{}{0pt}%
\pgfpathmoveto{\pgfqpoint{5.430311in}{3.200608in}}%
\pgfpathlineto{\pgfqpoint{5.443772in}{3.202498in}}%
\pgfpathlineto{\pgfqpoint{5.457247in}{3.204536in}}%
\pgfpathlineto{\pgfqpoint{5.470734in}{3.206724in}}%
\pgfpathlineto{\pgfqpoint{5.484234in}{3.209060in}}%
\pgfpathlineto{\pgfqpoint{5.491322in}{3.219258in}}%
\pgfpathlineto{\pgfqpoint{5.498412in}{3.229708in}}%
\pgfpathlineto{\pgfqpoint{5.505506in}{3.240419in}}%
\pgfpathlineto{\pgfqpoint{5.492025in}{3.238584in}}%
\pgfpathlineto{\pgfqpoint{5.478556in}{3.236896in}}%
\pgfpathlineto{\pgfqpoint{5.465100in}{3.235358in}}%
\pgfpathlineto{\pgfqpoint{5.451657in}{3.233967in}}%
\pgfpathlineto{\pgfqpoint{5.444539in}{3.222583in}}%
\pgfpathlineto{\pgfqpoint{5.437423in}{3.211466in}}%
\pgfpathlineto{\pgfqpoint{5.430311in}{3.200608in}}%
\pgfpathclose%
\pgfusepath{fill}%
\end{pgfscope}%
\begin{pgfscope}%
\pgfpathrectangle{\pgfqpoint{1.254980in}{0.150000in}}{\pgfqpoint{5.490039in}{5.490039in}}%
\pgfusepath{clip}%
\pgfsetbuttcap%
\pgfsetroundjoin%
\definecolor{currentfill}{rgb}{0.275191,0.194905,0.496005}%
\pgfsetfillcolor{currentfill}%
\pgfsetfillopacity{0.700000}%
\pgfsetlinewidth{0.000000pt}%
\definecolor{currentstroke}{rgb}{0.000000,0.000000,0.000000}%
\pgfsetstrokecolor{currentstroke}%
\pgfsetdash{}{0pt}%
\pgfpathmoveto{\pgfqpoint{3.146803in}{2.545202in}}%
\pgfpathlineto{\pgfqpoint{3.159771in}{2.533268in}}%
\pgfpathlineto{\pgfqpoint{3.172735in}{2.521573in}}%
\pgfpathlineto{\pgfqpoint{3.185698in}{2.510113in}}%
\pgfpathlineto{\pgfqpoint{3.198658in}{2.498888in}}%
\pgfpathlineto{\pgfqpoint{3.206459in}{2.508483in}}%
\pgfpathlineto{\pgfqpoint{3.214254in}{2.518160in}}%
\pgfpathlineto{\pgfqpoint{3.222042in}{2.527918in}}%
\pgfpathlineto{\pgfqpoint{3.229824in}{2.537759in}}%
\pgfpathlineto{\pgfqpoint{3.216878in}{2.548983in}}%
\pgfpathlineto{\pgfqpoint{3.203930in}{2.560443in}}%
\pgfpathlineto{\pgfqpoint{3.190979in}{2.572138in}}%
\pgfpathlineto{\pgfqpoint{3.178026in}{2.584072in}}%
\pgfpathlineto{\pgfqpoint{3.170231in}{2.574221in}}%
\pgfpathlineto{\pgfqpoint{3.162428in}{2.564460in}}%
\pgfpathlineto{\pgfqpoint{3.154619in}{2.554787in}}%
\pgfpathlineto{\pgfqpoint{3.146803in}{2.545202in}}%
\pgfpathclose%
\pgfusepath{fill}%
\end{pgfscope}%
\begin{pgfscope}%
\pgfpathrectangle{\pgfqpoint{1.254980in}{0.150000in}}{\pgfqpoint{5.490039in}{5.490039in}}%
\pgfusepath{clip}%
\pgfsetbuttcap%
\pgfsetroundjoin%
\definecolor{currentfill}{rgb}{0.276194,0.190074,0.493001}%
\pgfsetfillcolor{currentfill}%
\pgfsetfillopacity{0.700000}%
\pgfsetlinewidth{0.000000pt}%
\definecolor{currentstroke}{rgb}{0.000000,0.000000,0.000000}%
\pgfsetstrokecolor{currentstroke}%
\pgfsetdash{}{0pt}%
\pgfpathmoveto{\pgfqpoint{4.116677in}{2.515183in}}%
\pgfpathlineto{\pgfqpoint{4.129710in}{2.513751in}}%
\pgfpathlineto{\pgfqpoint{4.142749in}{2.512494in}}%
\pgfpathlineto{\pgfqpoint{4.155796in}{2.511412in}}%
\pgfpathlineto{\pgfqpoint{4.168851in}{2.510503in}}%
\pgfpathlineto{\pgfqpoint{4.176342in}{2.520310in}}%
\pgfpathlineto{\pgfqpoint{4.183829in}{2.530152in}}%
\pgfpathlineto{\pgfqpoint{4.191311in}{2.540031in}}%
\pgfpathlineto{\pgfqpoint{4.198788in}{2.549951in}}%
\pgfpathlineto{\pgfqpoint{4.185742in}{2.551084in}}%
\pgfpathlineto{\pgfqpoint{4.172704in}{2.552390in}}%
\pgfpathlineto{\pgfqpoint{4.159673in}{2.553870in}}%
\pgfpathlineto{\pgfqpoint{4.146650in}{2.555525in}}%
\pgfpathlineto{\pgfqpoint{4.139163in}{2.545371in}}%
\pgfpathlineto{\pgfqpoint{4.131673in}{2.535265in}}%
\pgfpathlineto{\pgfqpoint{4.124177in}{2.525203in}}%
\pgfpathlineto{\pgfqpoint{4.116677in}{2.515183in}}%
\pgfpathclose%
\pgfusepath{fill}%
\end{pgfscope}%
\begin{pgfscope}%
\pgfpathrectangle{\pgfqpoint{1.254980in}{0.150000in}}{\pgfqpoint{5.490039in}{5.490039in}}%
\pgfusepath{clip}%
\pgfsetbuttcap%
\pgfsetroundjoin%
\definecolor{currentfill}{rgb}{0.199430,0.387607,0.554642}%
\pgfsetfillcolor{currentfill}%
\pgfsetfillopacity{0.700000}%
\pgfsetlinewidth{0.000000pt}%
\definecolor{currentstroke}{rgb}{0.000000,0.000000,0.000000}%
\pgfsetstrokecolor{currentstroke}%
\pgfsetdash{}{0pt}%
\pgfpathmoveto{\pgfqpoint{2.781702in}{2.984429in}}%
\pgfpathlineto{\pgfqpoint{2.794840in}{2.964950in}}%
\pgfpathlineto{\pgfqpoint{2.807969in}{2.945774in}}%
\pgfpathlineto{\pgfqpoint{2.821088in}{2.926900in}}%
\pgfpathlineto{\pgfqpoint{2.834198in}{2.908324in}}%
\pgfpathlineto{\pgfqpoint{2.842114in}{2.917833in}}%
\pgfpathlineto{\pgfqpoint{2.850021in}{2.927472in}}%
\pgfpathlineto{\pgfqpoint{2.857921in}{2.937242in}}%
\pgfpathlineto{\pgfqpoint{2.865812in}{2.947144in}}%
\pgfpathlineto{\pgfqpoint{2.852720in}{2.965716in}}%
\pgfpathlineto{\pgfqpoint{2.839620in}{2.984586in}}%
\pgfpathlineto{\pgfqpoint{2.826510in}{3.003758in}}%
\pgfpathlineto{\pgfqpoint{2.813391in}{3.023233in}}%
\pgfpathlineto{\pgfqpoint{2.805482in}{3.013325in}}%
\pgfpathlineto{\pgfqpoint{2.797564in}{3.003556in}}%
\pgfpathlineto{\pgfqpoint{2.789637in}{2.993924in}}%
\pgfpathlineto{\pgfqpoint{2.781702in}{2.984429in}}%
\pgfpathclose%
\pgfusepath{fill}%
\end{pgfscope}%
\begin{pgfscope}%
\pgfpathrectangle{\pgfqpoint{1.254980in}{0.150000in}}{\pgfqpoint{5.490039in}{5.490039in}}%
\pgfusepath{clip}%
\pgfsetbuttcap%
\pgfsetroundjoin%
\definecolor{currentfill}{rgb}{0.282884,0.135920,0.453427}%
\pgfsetfillcolor{currentfill}%
\pgfsetfillopacity{0.700000}%
\pgfsetlinewidth{0.000000pt}%
\definecolor{currentstroke}{rgb}{0.000000,0.000000,0.000000}%
\pgfsetstrokecolor{currentstroke}%
\pgfsetdash{}{0pt}%
\pgfpathmoveto{\pgfqpoint{3.735986in}{2.412675in}}%
\pgfpathlineto{\pgfqpoint{3.748939in}{2.408210in}}%
\pgfpathlineto{\pgfqpoint{3.761897in}{2.403934in}}%
\pgfpathlineto{\pgfqpoint{3.774859in}{2.399849in}}%
\pgfpathlineto{\pgfqpoint{3.787825in}{2.395952in}}%
\pgfpathlineto{\pgfqpoint{3.795437in}{2.406029in}}%
\pgfpathlineto{\pgfqpoint{3.803044in}{2.416142in}}%
\pgfpathlineto{\pgfqpoint{3.810646in}{2.426292in}}%
\pgfpathlineto{\pgfqpoint{3.818243in}{2.436480in}}%
\pgfpathlineto{\pgfqpoint{3.805285in}{2.440489in}}%
\pgfpathlineto{\pgfqpoint{3.792332in}{2.444688in}}%
\pgfpathlineto{\pgfqpoint{3.779383in}{2.449075in}}%
\pgfpathlineto{\pgfqpoint{3.766439in}{2.453652in}}%
\pgfpathlineto{\pgfqpoint{3.758833in}{2.443342in}}%
\pgfpathlineto{\pgfqpoint{3.751223in}{2.433077in}}%
\pgfpathlineto{\pgfqpoint{3.743607in}{2.422855in}}%
\pgfpathlineto{\pgfqpoint{3.735986in}{2.412675in}}%
\pgfpathclose%
\pgfusepath{fill}%
\end{pgfscope}%
\begin{pgfscope}%
\pgfpathrectangle{\pgfqpoint{1.254980in}{0.150000in}}{\pgfqpoint{5.490039in}{5.490039in}}%
\pgfusepath{clip}%
\pgfsetbuttcap%
\pgfsetroundjoin%
\definecolor{currentfill}{rgb}{0.278826,0.175490,0.483397}%
\pgfsetfillcolor{currentfill}%
\pgfsetfillopacity{0.700000}%
\pgfsetlinewidth{0.000000pt}%
\definecolor{currentstroke}{rgb}{0.000000,0.000000,0.000000}%
\pgfsetstrokecolor{currentstroke}%
\pgfsetdash{}{0pt}%
\pgfpathmoveto{\pgfqpoint{4.034536in}{2.482172in}}%
\pgfpathlineto{\pgfqpoint{4.047549in}{2.480230in}}%
\pgfpathlineto{\pgfqpoint{4.060570in}{2.478466in}}%
\pgfpathlineto{\pgfqpoint{4.073597in}{2.476878in}}%
\pgfpathlineto{\pgfqpoint{4.086630in}{2.475467in}}%
\pgfpathlineto{\pgfqpoint{4.094149in}{2.485346in}}%
\pgfpathlineto{\pgfqpoint{4.101663in}{2.495257in}}%
\pgfpathlineto{\pgfqpoint{4.109173in}{2.505202in}}%
\pgfpathlineto{\pgfqpoint{4.116677in}{2.515183in}}%
\pgfpathlineto{\pgfqpoint{4.103652in}{2.516790in}}%
\pgfpathlineto{\pgfqpoint{4.090634in}{2.518573in}}%
\pgfpathlineto{\pgfqpoint{4.077622in}{2.520533in}}%
\pgfpathlineto{\pgfqpoint{4.064617in}{2.522671in}}%
\pgfpathlineto{\pgfqpoint{4.057104in}{2.512484in}}%
\pgfpathlineto{\pgfqpoint{4.049586in}{2.502340in}}%
\pgfpathlineto{\pgfqpoint{4.042063in}{2.492237in}}%
\pgfpathlineto{\pgfqpoint{4.034536in}{2.482172in}}%
\pgfpathclose%
\pgfusepath{fill}%
\end{pgfscope}%
\begin{pgfscope}%
\pgfpathrectangle{\pgfqpoint{1.254980in}{0.150000in}}{\pgfqpoint{5.490039in}{5.490039in}}%
\pgfusepath{clip}%
\pgfsetbuttcap%
\pgfsetroundjoin%
\definecolor{currentfill}{rgb}{0.278826,0.175490,0.483397}%
\pgfsetfillcolor{currentfill}%
\pgfsetfillopacity{0.700000}%
\pgfsetlinewidth{0.000000pt}%
\definecolor{currentstroke}{rgb}{0.000000,0.000000,0.000000}%
\pgfsetstrokecolor{currentstroke}%
\pgfsetdash{}{0pt}%
\pgfpathmoveto{\pgfqpoint{3.198658in}{2.498888in}}%
\pgfpathlineto{\pgfqpoint{3.211616in}{2.487895in}}%
\pgfpathlineto{\pgfqpoint{3.224572in}{2.477134in}}%
\pgfpathlineto{\pgfqpoint{3.237526in}{2.466602in}}%
\pgfpathlineto{\pgfqpoint{3.250479in}{2.456298in}}%
\pgfpathlineto{\pgfqpoint{3.258267in}{2.465904in}}%
\pgfpathlineto{\pgfqpoint{3.266048in}{2.475584in}}%
\pgfpathlineto{\pgfqpoint{3.273822in}{2.485338in}}%
\pgfpathlineto{\pgfqpoint{3.281591in}{2.495169in}}%
\pgfpathlineto{\pgfqpoint{3.268651in}{2.505473in}}%
\pgfpathlineto{\pgfqpoint{3.255710in}{2.516005in}}%
\pgfpathlineto{\pgfqpoint{3.242768in}{2.526766in}}%
\pgfpathlineto{\pgfqpoint{3.229824in}{2.537759in}}%
\pgfpathlineto{\pgfqpoint{3.222042in}{2.527918in}}%
\pgfpathlineto{\pgfqpoint{3.214254in}{2.518160in}}%
\pgfpathlineto{\pgfqpoint{3.206459in}{2.508483in}}%
\pgfpathlineto{\pgfqpoint{3.198658in}{2.498888in}}%
\pgfpathclose%
\pgfusepath{fill}%
\end{pgfscope}%
\begin{pgfscope}%
\pgfpathrectangle{\pgfqpoint{1.254980in}{0.150000in}}{\pgfqpoint{5.490039in}{5.490039in}}%
\pgfusepath{clip}%
\pgfsetbuttcap%
\pgfsetroundjoin%
\definecolor{currentfill}{rgb}{0.283072,0.130895,0.449241}%
\pgfsetfillcolor{currentfill}%
\pgfsetfillopacity{0.700000}%
\pgfsetlinewidth{0.000000pt}%
\definecolor{currentstroke}{rgb}{0.000000,0.000000,0.000000}%
\pgfsetstrokecolor{currentstroke}%
\pgfsetdash{}{0pt}%
\pgfpathmoveto{\pgfqpoint{3.519430in}{2.399666in}}%
\pgfpathlineto{\pgfqpoint{3.532366in}{2.392980in}}%
\pgfpathlineto{\pgfqpoint{3.545304in}{2.386496in}}%
\pgfpathlineto{\pgfqpoint{3.558244in}{2.380214in}}%
\pgfpathlineto{\pgfqpoint{3.571187in}{2.374133in}}%
\pgfpathlineto{\pgfqpoint{3.578868in}{2.384120in}}%
\pgfpathlineto{\pgfqpoint{3.586545in}{2.394153in}}%
\pgfpathlineto{\pgfqpoint{3.594216in}{2.404232in}}%
\pgfpathlineto{\pgfqpoint{3.601882in}{2.414359in}}%
\pgfpathlineto{\pgfqpoint{3.588949in}{2.420497in}}%
\pgfpathlineto{\pgfqpoint{3.576019in}{2.426836in}}%
\pgfpathlineto{\pgfqpoint{3.563092in}{2.433376in}}%
\pgfpathlineto{\pgfqpoint{3.550166in}{2.440119in}}%
\pgfpathlineto{\pgfqpoint{3.542490in}{2.429925in}}%
\pgfpathlineto{\pgfqpoint{3.534809in}{2.419786in}}%
\pgfpathlineto{\pgfqpoint{3.527122in}{2.409700in}}%
\pgfpathlineto{\pgfqpoint{3.519430in}{2.399666in}}%
\pgfpathclose%
\pgfusepath{fill}%
\end{pgfscope}%
\begin{pgfscope}%
\pgfpathrectangle{\pgfqpoint{1.254980in}{0.150000in}}{\pgfqpoint{5.490039in}{5.490039in}}%
\pgfusepath{clip}%
\pgfsetbuttcap%
\pgfsetroundjoin%
\definecolor{currentfill}{rgb}{0.282623,0.140926,0.457517}%
\pgfsetfillcolor{currentfill}%
\pgfsetfillopacity{0.700000}%
\pgfsetlinewidth{0.000000pt}%
\definecolor{currentstroke}{rgb}{0.000000,0.000000,0.000000}%
\pgfsetstrokecolor{currentstroke}%
\pgfsetdash{}{0pt}%
\pgfpathmoveto{\pgfqpoint{3.385085in}{2.420743in}}%
\pgfpathlineto{\pgfqpoint{3.398022in}{2.412418in}}%
\pgfpathlineto{\pgfqpoint{3.410959in}{2.404307in}}%
\pgfpathlineto{\pgfqpoint{3.423897in}{2.396407in}}%
\pgfpathlineto{\pgfqpoint{3.436836in}{2.388718in}}%
\pgfpathlineto{\pgfqpoint{3.444562in}{2.398570in}}%
\pgfpathlineto{\pgfqpoint{3.452282in}{2.408477in}}%
\pgfpathlineto{\pgfqpoint{3.459996in}{2.418441in}}%
\pgfpathlineto{\pgfqpoint{3.467705in}{2.428463in}}%
\pgfpathlineto{\pgfqpoint{3.454777in}{2.436181in}}%
\pgfpathlineto{\pgfqpoint{3.441851in}{2.444110in}}%
\pgfpathlineto{\pgfqpoint{3.428925in}{2.452250in}}%
\pgfpathlineto{\pgfqpoint{3.416000in}{2.460603in}}%
\pgfpathlineto{\pgfqpoint{3.408280in}{2.450542in}}%
\pgfpathlineto{\pgfqpoint{3.400554in}{2.440546in}}%
\pgfpathlineto{\pgfqpoint{3.392822in}{2.430613in}}%
\pgfpathlineto{\pgfqpoint{3.385085in}{2.420743in}}%
\pgfpathclose%
\pgfusepath{fill}%
\end{pgfscope}%
\begin{pgfscope}%
\pgfpathrectangle{\pgfqpoint{1.254980in}{0.150000in}}{\pgfqpoint{5.490039in}{5.490039in}}%
\pgfusepath{clip}%
\pgfsetbuttcap%
\pgfsetroundjoin%
\definecolor{currentfill}{rgb}{0.280868,0.160771,0.472899}%
\pgfsetfillcolor{currentfill}%
\pgfsetfillopacity{0.700000}%
\pgfsetlinewidth{0.000000pt}%
\definecolor{currentstroke}{rgb}{0.000000,0.000000,0.000000}%
\pgfsetstrokecolor{currentstroke}%
\pgfsetdash{}{0pt}%
\pgfpathmoveto{\pgfqpoint{3.952354in}{2.451135in}}%
\pgfpathlineto{\pgfqpoint{3.965351in}{2.448644in}}%
\pgfpathlineto{\pgfqpoint{3.978354in}{2.446333in}}%
\pgfpathlineto{\pgfqpoint{3.991363in}{2.444202in}}%
\pgfpathlineto{\pgfqpoint{4.004379in}{2.442250in}}%
\pgfpathlineto{\pgfqpoint{4.011925in}{2.452185in}}%
\pgfpathlineto{\pgfqpoint{4.019467in}{2.462148in}}%
\pgfpathlineto{\pgfqpoint{4.027004in}{2.472144in}}%
\pgfpathlineto{\pgfqpoint{4.034536in}{2.482172in}}%
\pgfpathlineto{\pgfqpoint{4.021529in}{2.484292in}}%
\pgfpathlineto{\pgfqpoint{4.008528in}{2.486592in}}%
\pgfpathlineto{\pgfqpoint{3.995534in}{2.489070in}}%
\pgfpathlineto{\pgfqpoint{3.982546in}{2.491729in}}%
\pgfpathlineto{\pgfqpoint{3.975005in}{2.481522in}}%
\pgfpathlineto{\pgfqpoint{3.967459in}{2.471356in}}%
\pgfpathlineto{\pgfqpoint{3.959909in}{2.461228in}}%
\pgfpathlineto{\pgfqpoint{3.952354in}{2.451135in}}%
\pgfpathclose%
\pgfusepath{fill}%
\end{pgfscope}%
\begin{pgfscope}%
\pgfpathrectangle{\pgfqpoint{1.254980in}{0.150000in}}{\pgfqpoint{5.490039in}{5.490039in}}%
\pgfusepath{clip}%
\pgfsetbuttcap%
\pgfsetroundjoin%
\definecolor{currentfill}{rgb}{0.283072,0.130895,0.449241}%
\pgfsetfillcolor{currentfill}%
\pgfsetfillopacity{0.700000}%
\pgfsetlinewidth{0.000000pt}%
\definecolor{currentstroke}{rgb}{0.000000,0.000000,0.000000}%
\pgfsetstrokecolor{currentstroke}%
\pgfsetdash{}{0pt}%
\pgfpathmoveto{\pgfqpoint{3.653640in}{2.391790in}}%
\pgfpathlineto{\pgfqpoint{3.666588in}{2.386638in}}%
\pgfpathlineto{\pgfqpoint{3.679539in}{2.381681in}}%
\pgfpathlineto{\pgfqpoint{3.692493in}{2.376917in}}%
\pgfpathlineto{\pgfqpoint{3.705452in}{2.372345in}}%
\pgfpathlineto{\pgfqpoint{3.713093in}{2.382372in}}%
\pgfpathlineto{\pgfqpoint{3.720729in}{2.392435in}}%
\pgfpathlineto{\pgfqpoint{3.728360in}{2.402536in}}%
\pgfpathlineto{\pgfqpoint{3.735986in}{2.412675in}}%
\pgfpathlineto{\pgfqpoint{3.723037in}{2.417332in}}%
\pgfpathlineto{\pgfqpoint{3.710091in}{2.422181in}}%
\pgfpathlineto{\pgfqpoint{3.697150in}{2.427223in}}%
\pgfpathlineto{\pgfqpoint{3.684212in}{2.432459in}}%
\pgfpathlineto{\pgfqpoint{3.676576in}{2.422225in}}%
\pgfpathlineto{\pgfqpoint{3.668936in}{2.412036in}}%
\pgfpathlineto{\pgfqpoint{3.661291in}{2.401891in}}%
\pgfpathlineto{\pgfqpoint{3.653640in}{2.391790in}}%
\pgfpathclose%
\pgfusepath{fill}%
\end{pgfscope}%
\begin{pgfscope}%
\pgfpathrectangle{\pgfqpoint{1.254980in}{0.150000in}}{\pgfqpoint{5.490039in}{5.490039in}}%
\pgfusepath{clip}%
\pgfsetbuttcap%
\pgfsetroundjoin%
\definecolor{currentfill}{rgb}{0.225863,0.330805,0.547314}%
\pgfsetfillcolor{currentfill}%
\pgfsetfillopacity{0.700000}%
\pgfsetlinewidth{0.000000pt}%
\definecolor{currentstroke}{rgb}{0.000000,0.000000,0.000000}%
\pgfsetstrokecolor{currentstroke}%
\pgfsetdash{}{0pt}%
\pgfpathmoveto{\pgfqpoint{4.744049in}{2.791493in}}%
\pgfpathlineto{\pgfqpoint{4.757298in}{2.793239in}}%
\pgfpathlineto{\pgfqpoint{4.770557in}{2.795144in}}%
\pgfpathlineto{\pgfqpoint{4.783827in}{2.797208in}}%
\pgfpathlineto{\pgfqpoint{4.797109in}{2.799431in}}%
\pgfpathlineto{\pgfqpoint{4.804393in}{2.808239in}}%
\pgfpathlineto{\pgfqpoint{4.811673in}{2.817133in}}%
\pgfpathlineto{\pgfqpoint{4.818949in}{2.826117in}}%
\pgfpathlineto{\pgfqpoint{4.826222in}{2.835196in}}%
\pgfpathlineto{\pgfqpoint{4.812955in}{2.833394in}}%
\pgfpathlineto{\pgfqpoint{4.799699in}{2.831750in}}%
\pgfpathlineto{\pgfqpoint{4.786453in}{2.830264in}}%
\pgfpathlineto{\pgfqpoint{4.773218in}{2.828938in}}%
\pgfpathlineto{\pgfqpoint{4.765931in}{2.819429in}}%
\pgfpathlineto{\pgfqpoint{4.758640in}{2.810022in}}%
\pgfpathlineto{\pgfqpoint{4.751346in}{2.800712in}}%
\pgfpathlineto{\pgfqpoint{4.744049in}{2.791493in}}%
\pgfpathclose%
\pgfusepath{fill}%
\end{pgfscope}%
\begin{pgfscope}%
\pgfpathrectangle{\pgfqpoint{1.254980in}{0.150000in}}{\pgfqpoint{5.490039in}{5.490039in}}%
\pgfusepath{clip}%
\pgfsetbuttcap%
\pgfsetroundjoin%
\definecolor{currentfill}{rgb}{0.218130,0.347432,0.550038}%
\pgfsetfillcolor{currentfill}%
\pgfsetfillopacity{0.700000}%
\pgfsetlinewidth{0.000000pt}%
\definecolor{currentstroke}{rgb}{0.000000,0.000000,0.000000}%
\pgfsetstrokecolor{currentstroke}%
\pgfsetdash{}{0pt}%
\pgfpathmoveto{\pgfqpoint{4.826222in}{2.835196in}}%
\pgfpathlineto{\pgfqpoint{4.839501in}{2.837156in}}%
\pgfpathlineto{\pgfqpoint{4.852790in}{2.839275in}}%
\pgfpathlineto{\pgfqpoint{4.866091in}{2.841551in}}%
\pgfpathlineto{\pgfqpoint{4.879404in}{2.843984in}}%
\pgfpathlineto{\pgfqpoint{4.886659in}{2.852725in}}%
\pgfpathlineto{\pgfqpoint{4.893911in}{2.861564in}}%
\pgfpathlineto{\pgfqpoint{4.901160in}{2.870507in}}%
\pgfpathlineto{\pgfqpoint{4.908406in}{2.879558in}}%
\pgfpathlineto{\pgfqpoint{4.895108in}{2.877573in}}%
\pgfpathlineto{\pgfqpoint{4.881823in}{2.875746in}}%
\pgfpathlineto{\pgfqpoint{4.868548in}{2.874075in}}%
\pgfpathlineto{\pgfqpoint{4.855284in}{2.872562in}}%
\pgfpathlineto{\pgfqpoint{4.848023in}{2.863053in}}%
\pgfpathlineto{\pgfqpoint{4.840759in}{2.853659in}}%
\pgfpathlineto{\pgfqpoint{4.833492in}{2.844375in}}%
\pgfpathlineto{\pgfqpoint{4.826222in}{2.835196in}}%
\pgfpathclose%
\pgfusepath{fill}%
\end{pgfscope}%
\begin{pgfscope}%
\pgfpathrectangle{\pgfqpoint{1.254980in}{0.150000in}}{\pgfqpoint{5.490039in}{5.490039in}}%
\pgfusepath{clip}%
\pgfsetbuttcap%
\pgfsetroundjoin%
\definecolor{currentfill}{rgb}{0.235526,0.309527,0.542944}%
\pgfsetfillcolor{currentfill}%
\pgfsetfillopacity{0.700000}%
\pgfsetlinewidth{0.000000pt}%
\definecolor{currentstroke}{rgb}{0.000000,0.000000,0.000000}%
\pgfsetstrokecolor{currentstroke}%
\pgfsetdash{}{0pt}%
\pgfpathmoveto{\pgfqpoint{4.661882in}{2.748496in}}%
\pgfpathlineto{\pgfqpoint{4.675102in}{2.749993in}}%
\pgfpathlineto{\pgfqpoint{4.688331in}{2.751651in}}%
\pgfpathlineto{\pgfqpoint{4.701572in}{2.753469in}}%
\pgfpathlineto{\pgfqpoint{4.714823in}{2.755448in}}%
\pgfpathlineto{\pgfqpoint{4.722135in}{2.764344in}}%
\pgfpathlineto{\pgfqpoint{4.729443in}{2.773314in}}%
\pgfpathlineto{\pgfqpoint{4.736748in}{2.782362in}}%
\pgfpathlineto{\pgfqpoint{4.744049in}{2.791493in}}%
\pgfpathlineto{\pgfqpoint{4.730811in}{2.789907in}}%
\pgfpathlineto{\pgfqpoint{4.717584in}{2.788481in}}%
\pgfpathlineto{\pgfqpoint{4.704367in}{2.787215in}}%
\pgfpathlineto{\pgfqpoint{4.691160in}{2.786109in}}%
\pgfpathlineto{\pgfqpoint{4.683846in}{2.776577in}}%
\pgfpathlineto{\pgfqpoint{4.676529in}{2.767133in}}%
\pgfpathlineto{\pgfqpoint{4.669208in}{2.757774in}}%
\pgfpathlineto{\pgfqpoint{4.661882in}{2.748496in}}%
\pgfpathclose%
\pgfusepath{fill}%
\end{pgfscope}%
\begin{pgfscope}%
\pgfpathrectangle{\pgfqpoint{1.254980in}{0.150000in}}{\pgfqpoint{5.490039in}{5.490039in}}%
\pgfusepath{clip}%
\pgfsetbuttcap%
\pgfsetroundjoin%
\definecolor{currentfill}{rgb}{0.208623,0.367752,0.552675}%
\pgfsetfillcolor{currentfill}%
\pgfsetfillopacity{0.700000}%
\pgfsetlinewidth{0.000000pt}%
\definecolor{currentstroke}{rgb}{0.000000,0.000000,0.000000}%
\pgfsetstrokecolor{currentstroke}%
\pgfsetdash{}{0pt}%
\pgfpathmoveto{\pgfqpoint{4.908406in}{2.879558in}}%
\pgfpathlineto{\pgfqpoint{4.921714in}{2.881699in}}%
\pgfpathlineto{\pgfqpoint{4.935034in}{2.883997in}}%
\pgfpathlineto{\pgfqpoint{4.948366in}{2.886451in}}%
\pgfpathlineto{\pgfqpoint{4.961709in}{2.889062in}}%
\pgfpathlineto{\pgfqpoint{4.968936in}{2.897760in}}%
\pgfpathlineto{\pgfqpoint{4.976160in}{2.906572in}}%
\pgfpathlineto{\pgfqpoint{4.983382in}{2.915501in}}%
\pgfpathlineto{\pgfqpoint{4.990601in}{2.924554in}}%
\pgfpathlineto{\pgfqpoint{4.977274in}{2.922421in}}%
\pgfpathlineto{\pgfqpoint{4.963959in}{2.920443in}}%
\pgfpathlineto{\pgfqpoint{4.950655in}{2.918621in}}%
\pgfpathlineto{\pgfqpoint{4.937362in}{2.916956in}}%
\pgfpathlineto{\pgfqpoint{4.930127in}{2.907417in}}%
\pgfpathlineto{\pgfqpoint{4.922889in}{2.898008in}}%
\pgfpathlineto{\pgfqpoint{4.915649in}{2.888723in}}%
\pgfpathlineto{\pgfqpoint{4.908406in}{2.879558in}}%
\pgfpathclose%
\pgfusepath{fill}%
\end{pgfscope}%
\begin{pgfscope}%
\pgfpathrectangle{\pgfqpoint{1.254980in}{0.150000in}}{\pgfqpoint{5.490039in}{5.490039in}}%
\pgfusepath{clip}%
\pgfsetbuttcap%
\pgfsetroundjoin%
\definecolor{currentfill}{rgb}{0.199430,0.387607,0.554642}%
\pgfsetfillcolor{currentfill}%
\pgfsetfillopacity{0.700000}%
\pgfsetlinewidth{0.000000pt}%
\definecolor{currentstroke}{rgb}{0.000000,0.000000,0.000000}%
\pgfsetstrokecolor{currentstroke}%
\pgfsetdash{}{0pt}%
\pgfpathmoveto{\pgfqpoint{4.990601in}{2.924554in}}%
\pgfpathlineto{\pgfqpoint{5.003940in}{2.926843in}}%
\pgfpathlineto{\pgfqpoint{5.017290in}{2.929287in}}%
\pgfpathlineto{\pgfqpoint{5.030653in}{2.931886in}}%
\pgfpathlineto{\pgfqpoint{5.044027in}{2.934640in}}%
\pgfpathlineto{\pgfqpoint{5.051227in}{2.943327in}}%
\pgfpathlineto{\pgfqpoint{5.058424in}{2.952143in}}%
\pgfpathlineto{\pgfqpoint{5.065619in}{2.961092in}}%
\pgfpathlineto{\pgfqpoint{5.072812in}{2.970180in}}%
\pgfpathlineto{\pgfqpoint{5.059455in}{2.967932in}}%
\pgfpathlineto{\pgfqpoint{5.046110in}{2.965837in}}%
\pgfpathlineto{\pgfqpoint{5.032777in}{2.963897in}}%
\pgfpathlineto{\pgfqpoint{5.019455in}{2.962112in}}%
\pgfpathlineto{\pgfqpoint{5.012245in}{2.952509in}}%
\pgfpathlineto{\pgfqpoint{5.005032in}{2.943053in}}%
\pgfpathlineto{\pgfqpoint{4.997818in}{2.933736in}}%
\pgfpathlineto{\pgfqpoint{4.990601in}{2.924554in}}%
\pgfpathclose%
\pgfusepath{fill}%
\end{pgfscope}%
\begin{pgfscope}%
\pgfpathrectangle{\pgfqpoint{1.254980in}{0.150000in}}{\pgfqpoint{5.490039in}{5.490039in}}%
\pgfusepath{clip}%
\pgfsetbuttcap%
\pgfsetroundjoin%
\definecolor{currentfill}{rgb}{0.243113,0.292092,0.538516}%
\pgfsetfillcolor{currentfill}%
\pgfsetfillopacity{0.700000}%
\pgfsetlinewidth{0.000000pt}%
\definecolor{currentstroke}{rgb}{0.000000,0.000000,0.000000}%
\pgfsetstrokecolor{currentstroke}%
\pgfsetdash{}{0pt}%
\pgfpathmoveto{\pgfqpoint{4.579720in}{2.706271in}}%
\pgfpathlineto{\pgfqpoint{4.592911in}{2.707484in}}%
\pgfpathlineto{\pgfqpoint{4.606111in}{2.708860in}}%
\pgfpathlineto{\pgfqpoint{4.619322in}{2.710399in}}%
\pgfpathlineto{\pgfqpoint{4.632543in}{2.712099in}}%
\pgfpathlineto{\pgfqpoint{4.639884in}{2.721099in}}%
\pgfpathlineto{\pgfqpoint{4.647221in}{2.730163in}}%
\pgfpathlineto{\pgfqpoint{4.654554in}{2.739294in}}%
\pgfpathlineto{\pgfqpoint{4.661882in}{2.748496in}}%
\pgfpathlineto{\pgfqpoint{4.648674in}{2.747160in}}%
\pgfpathlineto{\pgfqpoint{4.635475in}{2.745986in}}%
\pgfpathlineto{\pgfqpoint{4.622286in}{2.744974in}}%
\pgfpathlineto{\pgfqpoint{4.609108in}{2.744124in}}%
\pgfpathlineto{\pgfqpoint{4.601767in}{2.734547in}}%
\pgfpathlineto{\pgfqpoint{4.594422in}{2.725049in}}%
\pgfpathlineto{\pgfqpoint{4.587073in}{2.715625in}}%
\pgfpathlineto{\pgfqpoint{4.579720in}{2.706271in}}%
\pgfpathclose%
\pgfusepath{fill}%
\end{pgfscope}%
\begin{pgfscope}%
\pgfpathrectangle{\pgfqpoint{1.254980in}{0.150000in}}{\pgfqpoint{5.490039in}{5.490039in}}%
\pgfusepath{clip}%
\pgfsetbuttcap%
\pgfsetroundjoin%
\definecolor{currentfill}{rgb}{0.280868,0.160771,0.472899}%
\pgfsetfillcolor{currentfill}%
\pgfsetfillopacity{0.700000}%
\pgfsetlinewidth{0.000000pt}%
\definecolor{currentstroke}{rgb}{0.000000,0.000000,0.000000}%
\pgfsetstrokecolor{currentstroke}%
\pgfsetdash{}{0pt}%
\pgfpathmoveto{\pgfqpoint{3.250479in}{2.456298in}}%
\pgfpathlineto{\pgfqpoint{3.263431in}{2.446220in}}%
\pgfpathlineto{\pgfqpoint{3.276382in}{2.436367in}}%
\pgfpathlineto{\pgfqpoint{3.289332in}{2.426736in}}%
\pgfpathlineto{\pgfqpoint{3.302282in}{2.417328in}}%
\pgfpathlineto{\pgfqpoint{3.310055in}{2.426943in}}%
\pgfpathlineto{\pgfqpoint{3.317823in}{2.436626in}}%
\pgfpathlineto{\pgfqpoint{3.325585in}{2.446377in}}%
\pgfpathlineto{\pgfqpoint{3.333340in}{2.456196in}}%
\pgfpathlineto{\pgfqpoint{3.320404in}{2.465605in}}%
\pgfpathlineto{\pgfqpoint{3.307467in}{2.475236in}}%
\pgfpathlineto{\pgfqpoint{3.294529in}{2.485090in}}%
\pgfpathlineto{\pgfqpoint{3.281591in}{2.495169in}}%
\pgfpathlineto{\pgfqpoint{3.273822in}{2.485338in}}%
\pgfpathlineto{\pgfqpoint{3.266048in}{2.475584in}}%
\pgfpathlineto{\pgfqpoint{3.258267in}{2.465904in}}%
\pgfpathlineto{\pgfqpoint{3.250479in}{2.456298in}}%
\pgfpathclose%
\pgfusepath{fill}%
\end{pgfscope}%
\begin{pgfscope}%
\pgfpathrectangle{\pgfqpoint{1.254980in}{0.150000in}}{\pgfqpoint{5.490039in}{5.490039in}}%
\pgfusepath{clip}%
\pgfsetbuttcap%
\pgfsetroundjoin%
\definecolor{currentfill}{rgb}{0.281887,0.150881,0.465405}%
\pgfsetfillcolor{currentfill}%
\pgfsetfillopacity{0.700000}%
\pgfsetlinewidth{0.000000pt}%
\definecolor{currentstroke}{rgb}{0.000000,0.000000,0.000000}%
\pgfsetstrokecolor{currentstroke}%
\pgfsetdash{}{0pt}%
\pgfpathmoveto{\pgfqpoint{3.870121in}{2.422309in}}%
\pgfpathlineto{\pgfqpoint{3.883104in}{2.419229in}}%
\pgfpathlineto{\pgfqpoint{3.896092in}{2.416332in}}%
\pgfpathlineto{\pgfqpoint{3.909086in}{2.413619in}}%
\pgfpathlineto{\pgfqpoint{3.922085in}{2.411087in}}%
\pgfpathlineto{\pgfqpoint{3.929660in}{2.421055in}}%
\pgfpathlineto{\pgfqpoint{3.937229in}{2.431051in}}%
\pgfpathlineto{\pgfqpoint{3.944794in}{2.441077in}}%
\pgfpathlineto{\pgfqpoint{3.952354in}{2.451135in}}%
\pgfpathlineto{\pgfqpoint{3.939363in}{2.453808in}}%
\pgfpathlineto{\pgfqpoint{3.926378in}{2.456662in}}%
\pgfpathlineto{\pgfqpoint{3.913399in}{2.459699in}}%
\pgfpathlineto{\pgfqpoint{3.900425in}{2.462919in}}%
\pgfpathlineto{\pgfqpoint{3.892856in}{2.452710in}}%
\pgfpathlineto{\pgfqpoint{3.885283in}{2.442540in}}%
\pgfpathlineto{\pgfqpoint{3.877705in}{2.432407in}}%
\pgfpathlineto{\pgfqpoint{3.870121in}{2.422309in}}%
\pgfpathclose%
\pgfusepath{fill}%
\end{pgfscope}%
\begin{pgfscope}%
\pgfpathrectangle{\pgfqpoint{1.254980in}{0.150000in}}{\pgfqpoint{5.490039in}{5.490039in}}%
\pgfusepath{clip}%
\pgfsetbuttcap%
\pgfsetroundjoin%
\definecolor{currentfill}{rgb}{0.192357,0.403199,0.555836}%
\pgfsetfillcolor{currentfill}%
\pgfsetfillopacity{0.700000}%
\pgfsetlinewidth{0.000000pt}%
\definecolor{currentstroke}{rgb}{0.000000,0.000000,0.000000}%
\pgfsetstrokecolor{currentstroke}%
\pgfsetdash{}{0pt}%
\pgfpathmoveto{\pgfqpoint{5.072812in}{2.970180in}}%
\pgfpathlineto{\pgfqpoint{5.086181in}{2.972583in}}%
\pgfpathlineto{\pgfqpoint{5.099561in}{2.975140in}}%
\pgfpathlineto{\pgfqpoint{5.112955in}{2.977851in}}%
\pgfpathlineto{\pgfqpoint{5.126360in}{2.980715in}}%
\pgfpathlineto{\pgfqpoint{5.133533in}{2.989427in}}%
\pgfpathlineto{\pgfqpoint{5.140704in}{2.998283in}}%
\pgfpathlineto{\pgfqpoint{5.147873in}{3.007289in}}%
\pgfpathlineto{\pgfqpoint{5.155041in}{3.016453in}}%
\pgfpathlineto{\pgfqpoint{5.141655in}{3.014121in}}%
\pgfpathlineto{\pgfqpoint{5.128280in}{3.011943in}}%
\pgfpathlineto{\pgfqpoint{5.114918in}{3.009918in}}%
\pgfpathlineto{\pgfqpoint{5.101568in}{3.008047in}}%
\pgfpathlineto{\pgfqpoint{5.094381in}{2.998342in}}%
\pgfpathlineto{\pgfqpoint{5.087193in}{2.988799in}}%
\pgfpathlineto{\pgfqpoint{5.080003in}{2.979414in}}%
\pgfpathlineto{\pgfqpoint{5.072812in}{2.970180in}}%
\pgfpathclose%
\pgfusepath{fill}%
\end{pgfscope}%
\begin{pgfscope}%
\pgfpathrectangle{\pgfqpoint{1.254980in}{0.150000in}}{\pgfqpoint{5.490039in}{5.490039in}}%
\pgfusepath{clip}%
\pgfsetbuttcap%
\pgfsetroundjoin%
\definecolor{currentfill}{rgb}{0.250425,0.274290,0.533103}%
\pgfsetfillcolor{currentfill}%
\pgfsetfillopacity{0.700000}%
\pgfsetlinewidth{0.000000pt}%
\definecolor{currentstroke}{rgb}{0.000000,0.000000,0.000000}%
\pgfsetstrokecolor{currentstroke}%
\pgfsetdash{}{0pt}%
\pgfpathmoveto{\pgfqpoint{4.497560in}{2.664905in}}%
\pgfpathlineto{\pgfqpoint{4.510722in}{2.665800in}}%
\pgfpathlineto{\pgfqpoint{4.523894in}{2.666860in}}%
\pgfpathlineto{\pgfqpoint{4.537076in}{2.668083in}}%
\pgfpathlineto{\pgfqpoint{4.550267in}{2.669470in}}%
\pgfpathlineto{\pgfqpoint{4.557637in}{2.678586in}}%
\pgfpathlineto{\pgfqpoint{4.565002in}{2.687755in}}%
\pgfpathlineto{\pgfqpoint{4.572363in}{2.696982in}}%
\pgfpathlineto{\pgfqpoint{4.579720in}{2.706271in}}%
\pgfpathlineto{\pgfqpoint{4.566540in}{2.705220in}}%
\pgfpathlineto{\pgfqpoint{4.553369in}{2.704333in}}%
\pgfpathlineto{\pgfqpoint{4.540208in}{2.703609in}}%
\pgfpathlineto{\pgfqpoint{4.527057in}{2.703049in}}%
\pgfpathlineto{\pgfqpoint{4.519689in}{2.693414in}}%
\pgfpathlineto{\pgfqpoint{4.512317in}{2.683848in}}%
\pgfpathlineto{\pgfqpoint{4.504940in}{2.674346in}}%
\pgfpathlineto{\pgfqpoint{4.497560in}{2.664905in}}%
\pgfpathclose%
\pgfusepath{fill}%
\end{pgfscope}%
\begin{pgfscope}%
\pgfpathrectangle{\pgfqpoint{1.254980in}{0.150000in}}{\pgfqpoint{5.490039in}{5.490039in}}%
\pgfusepath{clip}%
\pgfsetbuttcap%
\pgfsetroundjoin%
\definecolor{currentfill}{rgb}{0.183898,0.422383,0.556944}%
\pgfsetfillcolor{currentfill}%
\pgfsetfillopacity{0.700000}%
\pgfsetlinewidth{0.000000pt}%
\definecolor{currentstroke}{rgb}{0.000000,0.000000,0.000000}%
\pgfsetstrokecolor{currentstroke}%
\pgfsetdash{}{0pt}%
\pgfpathmoveto{\pgfqpoint{5.155041in}{3.016453in}}%
\pgfpathlineto{\pgfqpoint{5.168440in}{3.018937in}}%
\pgfpathlineto{\pgfqpoint{5.181851in}{3.021574in}}%
\pgfpathlineto{\pgfqpoint{5.195274in}{3.024364in}}%
\pgfpathlineto{\pgfqpoint{5.208711in}{3.027306in}}%
\pgfpathlineto{\pgfqpoint{5.215858in}{3.036081in}}%
\pgfpathlineto{\pgfqpoint{5.223003in}{3.045019in}}%
\pgfpathlineto{\pgfqpoint{5.230148in}{3.054126in}}%
\pgfpathlineto{\pgfqpoint{5.237293in}{3.063408in}}%
\pgfpathlineto{\pgfqpoint{5.223877in}{3.061027in}}%
\pgfpathlineto{\pgfqpoint{5.210474in}{3.058798in}}%
\pgfpathlineto{\pgfqpoint{5.197083in}{3.056721in}}%
\pgfpathlineto{\pgfqpoint{5.183704in}{3.054796in}}%
\pgfpathlineto{\pgfqpoint{5.176539in}{3.044944in}}%
\pgfpathlineto{\pgfqpoint{5.169374in}{3.035274in}}%
\pgfpathlineto{\pgfqpoint{5.162208in}{3.025779in}}%
\pgfpathlineto{\pgfqpoint{5.155041in}{3.016453in}}%
\pgfpathclose%
\pgfusepath{fill}%
\end{pgfscope}%
\begin{pgfscope}%
\pgfpathrectangle{\pgfqpoint{1.254980in}{0.150000in}}{\pgfqpoint{5.490039in}{5.490039in}}%
\pgfusepath{clip}%
\pgfsetbuttcap%
\pgfsetroundjoin%
\definecolor{currentfill}{rgb}{0.258965,0.251537,0.524736}%
\pgfsetfillcolor{currentfill}%
\pgfsetfillopacity{0.700000}%
\pgfsetlinewidth{0.000000pt}%
\definecolor{currentstroke}{rgb}{0.000000,0.000000,0.000000}%
\pgfsetstrokecolor{currentstroke}%
\pgfsetdash{}{0pt}%
\pgfpathmoveto{\pgfqpoint{4.415397in}{2.624506in}}%
\pgfpathlineto{\pgfqpoint{4.428532in}{2.625048in}}%
\pgfpathlineto{\pgfqpoint{4.441676in}{2.625755in}}%
\pgfpathlineto{\pgfqpoint{4.454830in}{2.626629in}}%
\pgfpathlineto{\pgfqpoint{4.467993in}{2.627668in}}%
\pgfpathlineto{\pgfqpoint{4.475392in}{2.636905in}}%
\pgfpathlineto{\pgfqpoint{4.482785in}{2.646188in}}%
\pgfpathlineto{\pgfqpoint{4.490175in}{2.655520in}}%
\pgfpathlineto{\pgfqpoint{4.497560in}{2.664905in}}%
\pgfpathlineto{\pgfqpoint{4.484407in}{2.664174in}}%
\pgfpathlineto{\pgfqpoint{4.471263in}{2.663609in}}%
\pgfpathlineto{\pgfqpoint{4.458129in}{2.663208in}}%
\pgfpathlineto{\pgfqpoint{4.445004in}{2.662974in}}%
\pgfpathlineto{\pgfqpoint{4.437609in}{2.653271in}}%
\pgfpathlineto{\pgfqpoint{4.430209in}{2.643628in}}%
\pgfpathlineto{\pgfqpoint{4.422805in}{2.634041in}}%
\pgfpathlineto{\pgfqpoint{4.415397in}{2.624506in}}%
\pgfpathclose%
\pgfusepath{fill}%
\end{pgfscope}%
\begin{pgfscope}%
\pgfpathrectangle{\pgfqpoint{1.254980in}{0.150000in}}{\pgfqpoint{5.490039in}{5.490039in}}%
\pgfusepath{clip}%
\pgfsetbuttcap%
\pgfsetroundjoin%
\definecolor{currentfill}{rgb}{0.255645,0.260703,0.528312}%
\pgfsetfillcolor{currentfill}%
\pgfsetfillopacity{0.700000}%
\pgfsetlinewidth{0.000000pt}%
\definecolor{currentstroke}{rgb}{0.000000,0.000000,0.000000}%
\pgfsetstrokecolor{currentstroke}%
\pgfsetdash{}{0pt}%
\pgfpathmoveto{\pgfqpoint{2.959381in}{2.670564in}}%
\pgfpathlineto{\pgfqpoint{2.972415in}{2.655591in}}%
\pgfpathlineto{\pgfqpoint{2.985443in}{2.640881in}}%
\pgfpathlineto{\pgfqpoint{2.998466in}{2.626430in}}%
\pgfpathlineto{\pgfqpoint{3.011484in}{2.612238in}}%
\pgfpathlineto{\pgfqpoint{3.019360in}{2.621394in}}%
\pgfpathlineto{\pgfqpoint{3.027228in}{2.630651in}}%
\pgfpathlineto{\pgfqpoint{3.035089in}{2.640009in}}%
\pgfpathlineto{\pgfqpoint{3.042943in}{2.649469in}}%
\pgfpathlineto{\pgfqpoint{3.029942in}{2.663632in}}%
\pgfpathlineto{\pgfqpoint{3.016936in}{2.678053in}}%
\pgfpathlineto{\pgfqpoint{3.003925in}{2.692733in}}%
\pgfpathlineto{\pgfqpoint{2.990908in}{2.707676in}}%
\pgfpathlineto{\pgfqpoint{2.983038in}{2.698236in}}%
\pgfpathlineto{\pgfqpoint{2.975160in}{2.688903in}}%
\pgfpathlineto{\pgfqpoint{2.967274in}{2.679680in}}%
\pgfpathlineto{\pgfqpoint{2.959381in}{2.670564in}}%
\pgfpathclose%
\pgfusepath{fill}%
\end{pgfscope}%
\begin{pgfscope}%
\pgfpathrectangle{\pgfqpoint{1.254980in}{0.150000in}}{\pgfqpoint{5.490039in}{5.490039in}}%
\pgfusepath{clip}%
\pgfsetbuttcap%
\pgfsetroundjoin%
\definecolor{currentfill}{rgb}{0.244972,0.287675,0.537260}%
\pgfsetfillcolor{currentfill}%
\pgfsetfillopacity{0.700000}%
\pgfsetlinewidth{0.000000pt}%
\definecolor{currentstroke}{rgb}{0.000000,0.000000,0.000000}%
\pgfsetstrokecolor{currentstroke}%
\pgfsetdash{}{0pt}%
\pgfpathmoveto{\pgfqpoint{2.907185in}{2.733127in}}%
\pgfpathlineto{\pgfqpoint{2.920244in}{2.717081in}}%
\pgfpathlineto{\pgfqpoint{2.933296in}{2.701306in}}%
\pgfpathlineto{\pgfqpoint{2.946341in}{2.685802in}}%
\pgfpathlineto{\pgfqpoint{2.959381in}{2.670564in}}%
\pgfpathlineto{\pgfqpoint{2.967274in}{2.679680in}}%
\pgfpathlineto{\pgfqpoint{2.975160in}{2.688903in}}%
\pgfpathlineto{\pgfqpoint{2.983038in}{2.698236in}}%
\pgfpathlineto{\pgfqpoint{2.990908in}{2.707676in}}%
\pgfpathlineto{\pgfqpoint{2.977886in}{2.722884in}}%
\pgfpathlineto{\pgfqpoint{2.964858in}{2.738358in}}%
\pgfpathlineto{\pgfqpoint{2.951824in}{2.754102in}}%
\pgfpathlineto{\pgfqpoint{2.938783in}{2.770117in}}%
\pgfpathlineto{\pgfqpoint{2.930895in}{2.760697in}}%
\pgfpathlineto{\pgfqpoint{2.923000in}{2.751391in}}%
\pgfpathlineto{\pgfqpoint{2.915096in}{2.742201in}}%
\pgfpathlineto{\pgfqpoint{2.907185in}{2.733127in}}%
\pgfpathclose%
\pgfusepath{fill}%
\end{pgfscope}%
\begin{pgfscope}%
\pgfpathrectangle{\pgfqpoint{1.254980in}{0.150000in}}{\pgfqpoint{5.490039in}{5.490039in}}%
\pgfusepath{clip}%
\pgfsetbuttcap%
\pgfsetroundjoin%
\definecolor{currentfill}{rgb}{0.175841,0.441290,0.557685}%
\pgfsetfillcolor{currentfill}%
\pgfsetfillopacity{0.700000}%
\pgfsetlinewidth{0.000000pt}%
\definecolor{currentstroke}{rgb}{0.000000,0.000000,0.000000}%
\pgfsetstrokecolor{currentstroke}%
\pgfsetdash{}{0pt}%
\pgfpathmoveto{\pgfqpoint{5.237293in}{3.063408in}}%
\pgfpathlineto{\pgfqpoint{5.250721in}{3.065941in}}%
\pgfpathlineto{\pgfqpoint{5.264162in}{3.068625in}}%
\pgfpathlineto{\pgfqpoint{5.277616in}{3.071461in}}%
\pgfpathlineto{\pgfqpoint{5.291082in}{3.074449in}}%
\pgfpathlineto{\pgfqpoint{5.298205in}{3.083334in}}%
\pgfpathlineto{\pgfqpoint{5.305327in}{3.092400in}}%
\pgfpathlineto{\pgfqpoint{5.312449in}{3.101655in}}%
\pgfpathlineto{\pgfqpoint{5.319572in}{3.111105in}}%
\pgfpathlineto{\pgfqpoint{5.306127in}{3.108706in}}%
\pgfpathlineto{\pgfqpoint{5.292695in}{3.106459in}}%
\pgfpathlineto{\pgfqpoint{5.279276in}{3.104363in}}%
\pgfpathlineto{\pgfqpoint{5.265869in}{3.102417in}}%
\pgfpathlineto{\pgfqpoint{5.258725in}{3.092370in}}%
\pgfpathlineto{\pgfqpoint{5.251581in}{3.082523in}}%
\pgfpathlineto{\pgfqpoint{5.244437in}{3.072872in}}%
\pgfpathlineto{\pgfqpoint{5.237293in}{3.063408in}}%
\pgfpathclose%
\pgfusepath{fill}%
\end{pgfscope}%
\begin{pgfscope}%
\pgfpathrectangle{\pgfqpoint{1.254980in}{0.150000in}}{\pgfqpoint{5.490039in}{5.490039in}}%
\pgfusepath{clip}%
\pgfsetbuttcap%
\pgfsetroundjoin%
\definecolor{currentfill}{rgb}{0.265145,0.232956,0.516599}%
\pgfsetfillcolor{currentfill}%
\pgfsetfillopacity{0.700000}%
\pgfsetlinewidth{0.000000pt}%
\definecolor{currentstroke}{rgb}{0.000000,0.000000,0.000000}%
\pgfsetstrokecolor{currentstroke}%
\pgfsetdash{}{0pt}%
\pgfpathmoveto{\pgfqpoint{4.333227in}{2.585203in}}%
\pgfpathlineto{\pgfqpoint{4.346337in}{2.585355in}}%
\pgfpathlineto{\pgfqpoint{4.359455in}{2.585676in}}%
\pgfpathlineto{\pgfqpoint{4.372582in}{2.586164in}}%
\pgfpathlineto{\pgfqpoint{4.385718in}{2.586820in}}%
\pgfpathlineto{\pgfqpoint{4.393144in}{2.596180in}}%
\pgfpathlineto{\pgfqpoint{4.400566in}{2.605579in}}%
\pgfpathlineto{\pgfqpoint{4.407984in}{2.615020in}}%
\pgfpathlineto{\pgfqpoint{4.415397in}{2.624506in}}%
\pgfpathlineto{\pgfqpoint{4.402270in}{2.624131in}}%
\pgfpathlineto{\pgfqpoint{4.389153in}{2.623923in}}%
\pgfpathlineto{\pgfqpoint{4.376045in}{2.623882in}}%
\pgfpathlineto{\pgfqpoint{4.362945in}{2.624009in}}%
\pgfpathlineto{\pgfqpoint{4.355522in}{2.614233in}}%
\pgfpathlineto{\pgfqpoint{4.348095in}{2.604509in}}%
\pgfpathlineto{\pgfqpoint{4.340664in}{2.594833in}}%
\pgfpathlineto{\pgfqpoint{4.333227in}{2.585203in}}%
\pgfpathclose%
\pgfusepath{fill}%
\end{pgfscope}%
\begin{pgfscope}%
\pgfpathrectangle{\pgfqpoint{1.254980in}{0.150000in}}{\pgfqpoint{5.490039in}{5.490039in}}%
\pgfusepath{clip}%
\pgfsetbuttcap%
\pgfsetroundjoin%
\definecolor{currentfill}{rgb}{0.265145,0.232956,0.516599}%
\pgfsetfillcolor{currentfill}%
\pgfsetfillopacity{0.700000}%
\pgfsetlinewidth{0.000000pt}%
\definecolor{currentstroke}{rgb}{0.000000,0.000000,0.000000}%
\pgfsetstrokecolor{currentstroke}%
\pgfsetdash{}{0pt}%
\pgfpathmoveto{\pgfqpoint{3.011484in}{2.612238in}}%
\pgfpathlineto{\pgfqpoint{3.024497in}{2.598302in}}%
\pgfpathlineto{\pgfqpoint{3.037505in}{2.584619in}}%
\pgfpathlineto{\pgfqpoint{3.050509in}{2.571189in}}%
\pgfpathlineto{\pgfqpoint{3.063509in}{2.558008in}}%
\pgfpathlineto{\pgfqpoint{3.071369in}{2.567203in}}%
\pgfpathlineto{\pgfqpoint{3.079221in}{2.576493in}}%
\pgfpathlineto{\pgfqpoint{3.087065in}{2.585877in}}%
\pgfpathlineto{\pgfqpoint{3.094903in}{2.595355in}}%
\pgfpathlineto{\pgfqpoint{3.081919in}{2.608508in}}%
\pgfpathlineto{\pgfqpoint{3.068932in}{2.621909in}}%
\pgfpathlineto{\pgfqpoint{3.055940in}{2.635563in}}%
\pgfpathlineto{\pgfqpoint{3.042943in}{2.649469in}}%
\pgfpathlineto{\pgfqpoint{3.035089in}{2.640009in}}%
\pgfpathlineto{\pgfqpoint{3.027228in}{2.630651in}}%
\pgfpathlineto{\pgfqpoint{3.019360in}{2.621394in}}%
\pgfpathlineto{\pgfqpoint{3.011484in}{2.612238in}}%
\pgfpathclose%
\pgfusepath{fill}%
\end{pgfscope}%
\begin{pgfscope}%
\pgfpathrectangle{\pgfqpoint{1.254980in}{0.150000in}}{\pgfqpoint{5.490039in}{5.490039in}}%
\pgfusepath{clip}%
\pgfsetbuttcap%
\pgfsetroundjoin%
\definecolor{currentfill}{rgb}{0.166617,0.463708,0.558119}%
\pgfsetfillcolor{currentfill}%
\pgfsetfillopacity{0.700000}%
\pgfsetlinewidth{0.000000pt}%
\definecolor{currentstroke}{rgb}{0.000000,0.000000,0.000000}%
\pgfsetstrokecolor{currentstroke}%
\pgfsetdash{}{0pt}%
\pgfpathmoveto{\pgfqpoint{5.319572in}{3.111105in}}%
\pgfpathlineto{\pgfqpoint{5.333029in}{3.113654in}}%
\pgfpathlineto{\pgfqpoint{5.346499in}{3.116353in}}%
\pgfpathlineto{\pgfqpoint{5.359982in}{3.119204in}}%
\pgfpathlineto{\pgfqpoint{5.373478in}{3.122204in}}%
\pgfpathlineto{\pgfqpoint{5.380579in}{3.131249in}}%
\pgfpathlineto{\pgfqpoint{5.387679in}{3.140495in}}%
\pgfpathlineto{\pgfqpoint{5.394781in}{3.149951in}}%
\pgfpathlineto{\pgfqpoint{5.401883in}{3.159622in}}%
\pgfpathlineto{\pgfqpoint{5.388411in}{3.157239in}}%
\pgfpathlineto{\pgfqpoint{5.374951in}{3.155005in}}%
\pgfpathlineto{\pgfqpoint{5.361503in}{3.152921in}}%
\pgfpathlineto{\pgfqpoint{5.348069in}{3.150988in}}%
\pgfpathlineto{\pgfqpoint{5.340943in}{3.140691in}}%
\pgfpathlineto{\pgfqpoint{5.333818in}{3.130616in}}%
\pgfpathlineto{\pgfqpoint{5.326695in}{3.120756in}}%
\pgfpathlineto{\pgfqpoint{5.319572in}{3.111105in}}%
\pgfpathclose%
\pgfusepath{fill}%
\end{pgfscope}%
\begin{pgfscope}%
\pgfpathrectangle{\pgfqpoint{1.254980in}{0.150000in}}{\pgfqpoint{5.490039in}{5.490039in}}%
\pgfusepath{clip}%
\pgfsetbuttcap%
\pgfsetroundjoin%
\definecolor{currentfill}{rgb}{0.231674,0.318106,0.544834}%
\pgfsetfillcolor{currentfill}%
\pgfsetfillopacity{0.700000}%
\pgfsetlinewidth{0.000000pt}%
\definecolor{currentstroke}{rgb}{0.000000,0.000000,0.000000}%
\pgfsetstrokecolor{currentstroke}%
\pgfsetdash{}{0pt}%
\pgfpathmoveto{\pgfqpoint{2.854880in}{2.800079in}}%
\pgfpathlineto{\pgfqpoint{2.867967in}{2.782921in}}%
\pgfpathlineto{\pgfqpoint{2.881047in}{2.766044in}}%
\pgfpathlineto{\pgfqpoint{2.894120in}{2.749447in}}%
\pgfpathlineto{\pgfqpoint{2.907185in}{2.733127in}}%
\pgfpathlineto{\pgfqpoint{2.915096in}{2.742201in}}%
\pgfpathlineto{\pgfqpoint{2.923000in}{2.751391in}}%
\pgfpathlineto{\pgfqpoint{2.930895in}{2.760697in}}%
\pgfpathlineto{\pgfqpoint{2.938783in}{2.770117in}}%
\pgfpathlineto{\pgfqpoint{2.925736in}{2.786407in}}%
\pgfpathlineto{\pgfqpoint{2.912682in}{2.802973in}}%
\pgfpathlineto{\pgfqpoint{2.899621in}{2.819818in}}%
\pgfpathlineto{\pgfqpoint{2.886552in}{2.836945in}}%
\pgfpathlineto{\pgfqpoint{2.878646in}{2.827545in}}%
\pgfpathlineto{\pgfqpoint{2.870732in}{2.818267in}}%
\pgfpathlineto{\pgfqpoint{2.862810in}{2.809112in}}%
\pgfpathlineto{\pgfqpoint{2.854880in}{2.800079in}}%
\pgfpathclose%
\pgfusepath{fill}%
\end{pgfscope}%
\begin{pgfscope}%
\pgfpathrectangle{\pgfqpoint{1.254980in}{0.150000in}}{\pgfqpoint{5.490039in}{5.490039in}}%
\pgfusepath{clip}%
\pgfsetbuttcap%
\pgfsetroundjoin%
\definecolor{currentfill}{rgb}{0.283072,0.130895,0.449241}%
\pgfsetfillcolor{currentfill}%
\pgfsetfillopacity{0.700000}%
\pgfsetlinewidth{0.000000pt}%
\definecolor{currentstroke}{rgb}{0.000000,0.000000,0.000000}%
\pgfsetstrokecolor{currentstroke}%
\pgfsetdash{}{0pt}%
\pgfpathmoveto{\pgfqpoint{3.436836in}{2.388718in}}%
\pgfpathlineto{\pgfqpoint{3.449777in}{2.381237in}}%
\pgfpathlineto{\pgfqpoint{3.462718in}{2.373964in}}%
\pgfpathlineto{\pgfqpoint{3.475661in}{2.366898in}}%
\pgfpathlineto{\pgfqpoint{3.488606in}{2.360037in}}%
\pgfpathlineto{\pgfqpoint{3.496321in}{2.369870in}}%
\pgfpathlineto{\pgfqpoint{3.504029in}{2.379752in}}%
\pgfpathlineto{\pgfqpoint{3.511733in}{2.389684in}}%
\pgfpathlineto{\pgfqpoint{3.519430in}{2.399666in}}%
\pgfpathlineto{\pgfqpoint{3.506496in}{2.406556in}}%
\pgfpathlineto{\pgfqpoint{3.493564in}{2.413652in}}%
\pgfpathlineto{\pgfqpoint{3.480634in}{2.420953in}}%
\pgfpathlineto{\pgfqpoint{3.467705in}{2.428463in}}%
\pgfpathlineto{\pgfqpoint{3.459996in}{2.418441in}}%
\pgfpathlineto{\pgfqpoint{3.452282in}{2.408477in}}%
\pgfpathlineto{\pgfqpoint{3.444562in}{2.398570in}}%
\pgfpathlineto{\pgfqpoint{3.436836in}{2.388718in}}%
\pgfpathclose%
\pgfusepath{fill}%
\end{pgfscope}%
\begin{pgfscope}%
\pgfpathrectangle{\pgfqpoint{1.254980in}{0.150000in}}{\pgfqpoint{5.490039in}{5.490039in}}%
\pgfusepath{clip}%
\pgfsetbuttcap%
\pgfsetroundjoin%
\definecolor{currentfill}{rgb}{0.269308,0.218818,0.509577}%
\pgfsetfillcolor{currentfill}%
\pgfsetfillopacity{0.700000}%
\pgfsetlinewidth{0.000000pt}%
\definecolor{currentstroke}{rgb}{0.000000,0.000000,0.000000}%
\pgfsetstrokecolor{currentstroke}%
\pgfsetdash{}{0pt}%
\pgfpathmoveto{\pgfqpoint{4.251047in}{2.547145in}}%
\pgfpathlineto{\pgfqpoint{4.264132in}{2.546872in}}%
\pgfpathlineto{\pgfqpoint{4.277225in}{2.546769in}}%
\pgfpathlineto{\pgfqpoint{4.290327in}{2.546836in}}%
\pgfpathlineto{\pgfqpoint{4.303437in}{2.547072in}}%
\pgfpathlineto{\pgfqpoint{4.310892in}{2.556552in}}%
\pgfpathlineto{\pgfqpoint{4.318342in}{2.566066in}}%
\pgfpathlineto{\pgfqpoint{4.325787in}{2.575615in}}%
\pgfpathlineto{\pgfqpoint{4.333227in}{2.585203in}}%
\pgfpathlineto{\pgfqpoint{4.320127in}{2.585219in}}%
\pgfpathlineto{\pgfqpoint{4.307034in}{2.585404in}}%
\pgfpathlineto{\pgfqpoint{4.293951in}{2.585759in}}%
\pgfpathlineto{\pgfqpoint{4.280875in}{2.586284in}}%
\pgfpathlineto{\pgfqpoint{4.273425in}{2.576434in}}%
\pgfpathlineto{\pgfqpoint{4.265970in}{2.566629in}}%
\pgfpathlineto{\pgfqpoint{4.258511in}{2.556867in}}%
\pgfpathlineto{\pgfqpoint{4.251047in}{2.547145in}}%
\pgfpathclose%
\pgfusepath{fill}%
\end{pgfscope}%
\begin{pgfscope}%
\pgfpathrectangle{\pgfqpoint{1.254980in}{0.150000in}}{\pgfqpoint{5.490039in}{5.490039in}}%
\pgfusepath{clip}%
\pgfsetbuttcap%
\pgfsetroundjoin%
\definecolor{currentfill}{rgb}{0.159194,0.482237,0.558073}%
\pgfsetfillcolor{currentfill}%
\pgfsetfillopacity{0.700000}%
\pgfsetlinewidth{0.000000pt}%
\definecolor{currentstroke}{rgb}{0.000000,0.000000,0.000000}%
\pgfsetstrokecolor{currentstroke}%
\pgfsetdash{}{0pt}%
\pgfpathmoveto{\pgfqpoint{5.401883in}{3.159622in}}%
\pgfpathlineto{\pgfqpoint{5.415369in}{3.162155in}}%
\pgfpathlineto{\pgfqpoint{5.428868in}{3.164838in}}%
\pgfpathlineto{\pgfqpoint{5.442380in}{3.167670in}}%
\pgfpathlineto{\pgfqpoint{5.455905in}{3.170651in}}%
\pgfpathlineto{\pgfqpoint{5.462984in}{3.179911in}}%
\pgfpathlineto{\pgfqpoint{5.470066in}{3.189394in}}%
\pgfpathlineto{\pgfqpoint{5.477149in}{3.199108in}}%
\pgfpathlineto{\pgfqpoint{5.484234in}{3.209060in}}%
\pgfpathlineto{\pgfqpoint{5.470734in}{3.206724in}}%
\pgfpathlineto{\pgfqpoint{5.457247in}{3.204536in}}%
\pgfpathlineto{\pgfqpoint{5.443772in}{3.202498in}}%
\pgfpathlineto{\pgfqpoint{5.430311in}{3.200608in}}%
\pgfpathlineto{\pgfqpoint{5.423201in}{3.190002in}}%
\pgfpathlineto{\pgfqpoint{5.416093in}{3.179640in}}%
\pgfpathlineto{\pgfqpoint{5.408987in}{3.169516in}}%
\pgfpathlineto{\pgfqpoint{5.401883in}{3.159622in}}%
\pgfpathclose%
\pgfusepath{fill}%
\end{pgfscope}%
\begin{pgfscope}%
\pgfpathrectangle{\pgfqpoint{1.254980in}{0.150000in}}{\pgfqpoint{5.490039in}{5.490039in}}%
\pgfusepath{clip}%
\pgfsetbuttcap%
\pgfsetroundjoin%
\definecolor{currentfill}{rgb}{0.282884,0.135920,0.453427}%
\pgfsetfillcolor{currentfill}%
\pgfsetfillopacity{0.700000}%
\pgfsetlinewidth{0.000000pt}%
\definecolor{currentstroke}{rgb}{0.000000,0.000000,0.000000}%
\pgfsetstrokecolor{currentstroke}%
\pgfsetdash{}{0pt}%
\pgfpathmoveto{\pgfqpoint{3.787825in}{2.395952in}}%
\pgfpathlineto{\pgfqpoint{3.800797in}{2.392242in}}%
\pgfpathlineto{\pgfqpoint{3.813773in}{2.388719in}}%
\pgfpathlineto{\pgfqpoint{3.826753in}{2.385383in}}%
\pgfpathlineto{\pgfqpoint{3.839740in}{2.382231in}}%
\pgfpathlineto{\pgfqpoint{3.847342in}{2.392207in}}%
\pgfpathlineto{\pgfqpoint{3.854940in}{2.402210in}}%
\pgfpathlineto{\pgfqpoint{3.862533in}{2.412244in}}%
\pgfpathlineto{\pgfqpoint{3.870121in}{2.422309in}}%
\pgfpathlineto{\pgfqpoint{3.857144in}{2.425573in}}%
\pgfpathlineto{\pgfqpoint{3.844172in}{2.429023in}}%
\pgfpathlineto{\pgfqpoint{3.831205in}{2.432658in}}%
\pgfpathlineto{\pgfqpoint{3.818243in}{2.436480in}}%
\pgfpathlineto{\pgfqpoint{3.810646in}{2.426292in}}%
\pgfpathlineto{\pgfqpoint{3.803044in}{2.416142in}}%
\pgfpathlineto{\pgfqpoint{3.795437in}{2.406029in}}%
\pgfpathlineto{\pgfqpoint{3.787825in}{2.395952in}}%
\pgfpathclose%
\pgfusepath{fill}%
\end{pgfscope}%
\begin{pgfscope}%
\pgfpathrectangle{\pgfqpoint{1.254980in}{0.150000in}}{\pgfqpoint{5.490039in}{5.490039in}}%
\pgfusepath{clip}%
\pgfsetbuttcap%
\pgfsetroundjoin%
\definecolor{currentfill}{rgb}{0.271828,0.209303,0.504434}%
\pgfsetfillcolor{currentfill}%
\pgfsetfillopacity{0.700000}%
\pgfsetlinewidth{0.000000pt}%
\definecolor{currentstroke}{rgb}{0.000000,0.000000,0.000000}%
\pgfsetstrokecolor{currentstroke}%
\pgfsetdash{}{0pt}%
\pgfpathmoveto{\pgfqpoint{3.063509in}{2.558008in}}%
\pgfpathlineto{\pgfqpoint{3.076505in}{2.545074in}}%
\pgfpathlineto{\pgfqpoint{3.089498in}{2.532386in}}%
\pgfpathlineto{\pgfqpoint{3.102487in}{2.519942in}}%
\pgfpathlineto{\pgfqpoint{3.115472in}{2.507740in}}%
\pgfpathlineto{\pgfqpoint{3.123315in}{2.516975in}}%
\pgfpathlineto{\pgfqpoint{3.131152in}{2.526296in}}%
\pgfpathlineto{\pgfqpoint{3.138981in}{2.535706in}}%
\pgfpathlineto{\pgfqpoint{3.146803in}{2.545202in}}%
\pgfpathlineto{\pgfqpoint{3.133833in}{2.557376in}}%
\pgfpathlineto{\pgfqpoint{3.120860in}{2.569792in}}%
\pgfpathlineto{\pgfqpoint{3.107883in}{2.582451in}}%
\pgfpathlineto{\pgfqpoint{3.094903in}{2.595355in}}%
\pgfpathlineto{\pgfqpoint{3.087065in}{2.585877in}}%
\pgfpathlineto{\pgfqpoint{3.079221in}{2.576493in}}%
\pgfpathlineto{\pgfqpoint{3.071369in}{2.567203in}}%
\pgfpathlineto{\pgfqpoint{3.063509in}{2.558008in}}%
\pgfpathclose%
\pgfusepath{fill}%
\end{pgfscope}%
\begin{pgfscope}%
\pgfpathrectangle{\pgfqpoint{1.254980in}{0.150000in}}{\pgfqpoint{5.490039in}{5.490039in}}%
\pgfusepath{clip}%
\pgfsetbuttcap%
\pgfsetroundjoin%
\definecolor{currentfill}{rgb}{0.283187,0.125848,0.444960}%
\pgfsetfillcolor{currentfill}%
\pgfsetfillopacity{0.700000}%
\pgfsetlinewidth{0.000000pt}%
\definecolor{currentstroke}{rgb}{0.000000,0.000000,0.000000}%
\pgfsetstrokecolor{currentstroke}%
\pgfsetdash{}{0pt}%
\pgfpathmoveto{\pgfqpoint{3.571187in}{2.374133in}}%
\pgfpathlineto{\pgfqpoint{3.584132in}{2.368251in}}%
\pgfpathlineto{\pgfqpoint{3.597080in}{2.362568in}}%
\pgfpathlineto{\pgfqpoint{3.610031in}{2.357082in}}%
\pgfpathlineto{\pgfqpoint{3.622985in}{2.351792in}}%
\pgfpathlineto{\pgfqpoint{3.630657in}{2.361733in}}%
\pgfpathlineto{\pgfqpoint{3.638323in}{2.371712in}}%
\pgfpathlineto{\pgfqpoint{3.645984in}{2.381731in}}%
\pgfpathlineto{\pgfqpoint{3.653640in}{2.391790in}}%
\pgfpathlineto{\pgfqpoint{3.640696in}{2.397137in}}%
\pgfpathlineto{\pgfqpoint{3.627755in}{2.402680in}}%
\pgfpathlineto{\pgfqpoint{3.614817in}{2.408420in}}%
\pgfpathlineto{\pgfqpoint{3.601882in}{2.414359in}}%
\pgfpathlineto{\pgfqpoint{3.594216in}{2.404232in}}%
\pgfpathlineto{\pgfqpoint{3.586545in}{2.394153in}}%
\pgfpathlineto{\pgfqpoint{3.578868in}{2.384120in}}%
\pgfpathlineto{\pgfqpoint{3.571187in}{2.374133in}}%
\pgfpathclose%
\pgfusepath{fill}%
\end{pgfscope}%
\begin{pgfscope}%
\pgfpathrectangle{\pgfqpoint{1.254980in}{0.150000in}}{\pgfqpoint{5.490039in}{5.490039in}}%
\pgfusepath{clip}%
\pgfsetbuttcap%
\pgfsetroundjoin%
\definecolor{currentfill}{rgb}{0.216210,0.351535,0.550627}%
\pgfsetfillcolor{currentfill}%
\pgfsetfillopacity{0.700000}%
\pgfsetlinewidth{0.000000pt}%
\definecolor{currentstroke}{rgb}{0.000000,0.000000,0.000000}%
\pgfsetstrokecolor{currentstroke}%
\pgfsetdash{}{0pt}%
\pgfpathmoveto{\pgfqpoint{2.802449in}{2.871585in}}%
\pgfpathlineto{\pgfqpoint{2.815569in}{2.853272in}}%
\pgfpathlineto{\pgfqpoint{2.828681in}{2.835252in}}%
\pgfpathlineto{\pgfqpoint{2.841785in}{2.817522in}}%
\pgfpathlineto{\pgfqpoint{2.854880in}{2.800079in}}%
\pgfpathlineto{\pgfqpoint{2.862810in}{2.809112in}}%
\pgfpathlineto{\pgfqpoint{2.870732in}{2.818267in}}%
\pgfpathlineto{\pgfqpoint{2.878646in}{2.827545in}}%
\pgfpathlineto{\pgfqpoint{2.886552in}{2.836945in}}%
\pgfpathlineto{\pgfqpoint{2.873476in}{2.854357in}}%
\pgfpathlineto{\pgfqpoint{2.860391in}{2.872055in}}%
\pgfpathlineto{\pgfqpoint{2.847299in}{2.890043in}}%
\pgfpathlineto{\pgfqpoint{2.834198in}{2.908324in}}%
\pgfpathlineto{\pgfqpoint{2.826273in}{2.898945in}}%
\pgfpathlineto{\pgfqpoint{2.818340in}{2.889695in}}%
\pgfpathlineto{\pgfqpoint{2.810399in}{2.880575in}}%
\pgfpathlineto{\pgfqpoint{2.802449in}{2.871585in}}%
\pgfpathclose%
\pgfusepath{fill}%
\end{pgfscope}%
\begin{pgfscope}%
\pgfpathrectangle{\pgfqpoint{1.254980in}{0.150000in}}{\pgfqpoint{5.490039in}{5.490039in}}%
\pgfusepath{clip}%
\pgfsetbuttcap%
\pgfsetroundjoin%
\definecolor{currentfill}{rgb}{0.274128,0.199721,0.498911}%
\pgfsetfillcolor{currentfill}%
\pgfsetfillopacity{0.700000}%
\pgfsetlinewidth{0.000000pt}%
\definecolor{currentstroke}{rgb}{0.000000,0.000000,0.000000}%
\pgfsetstrokecolor{currentstroke}%
\pgfsetdash{}{0pt}%
\pgfpathmoveto{\pgfqpoint{4.168851in}{2.510503in}}%
\pgfpathlineto{\pgfqpoint{4.181913in}{2.509767in}}%
\pgfpathlineto{\pgfqpoint{4.194982in}{2.509204in}}%
\pgfpathlineto{\pgfqpoint{4.208060in}{2.508813in}}%
\pgfpathlineto{\pgfqpoint{4.221146in}{2.508593in}}%
\pgfpathlineto{\pgfqpoint{4.228628in}{2.518186in}}%
\pgfpathlineto{\pgfqpoint{4.236106in}{2.527807in}}%
\pgfpathlineto{\pgfqpoint{4.243579in}{2.537459in}}%
\pgfpathlineto{\pgfqpoint{4.251047in}{2.547145in}}%
\pgfpathlineto{\pgfqpoint{4.237971in}{2.547589in}}%
\pgfpathlineto{\pgfqpoint{4.224902in}{2.548204in}}%
\pgfpathlineto{\pgfqpoint{4.211841in}{2.548992in}}%
\pgfpathlineto{\pgfqpoint{4.198788in}{2.549951in}}%
\pgfpathlineto{\pgfqpoint{4.191311in}{2.540031in}}%
\pgfpathlineto{\pgfqpoint{4.183829in}{2.530152in}}%
\pgfpathlineto{\pgfqpoint{4.176342in}{2.520310in}}%
\pgfpathlineto{\pgfqpoint{4.168851in}{2.510503in}}%
\pgfpathclose%
\pgfusepath{fill}%
\end{pgfscope}%
\begin{pgfscope}%
\pgfpathrectangle{\pgfqpoint{1.254980in}{0.150000in}}{\pgfqpoint{5.490039in}{5.490039in}}%
\pgfusepath{clip}%
\pgfsetbuttcap%
\pgfsetroundjoin%
\definecolor{currentfill}{rgb}{0.282290,0.145912,0.461510}%
\pgfsetfillcolor{currentfill}%
\pgfsetfillopacity{0.700000}%
\pgfsetlinewidth{0.000000pt}%
\definecolor{currentstroke}{rgb}{0.000000,0.000000,0.000000}%
\pgfsetstrokecolor{currentstroke}%
\pgfsetdash{}{0pt}%
\pgfpathmoveto{\pgfqpoint{3.302282in}{2.417328in}}%
\pgfpathlineto{\pgfqpoint{3.315230in}{2.408139in}}%
\pgfpathlineto{\pgfqpoint{3.328179in}{2.399169in}}%
\pgfpathlineto{\pgfqpoint{3.341128in}{2.390415in}}%
\pgfpathlineto{\pgfqpoint{3.354077in}{2.381878in}}%
\pgfpathlineto{\pgfqpoint{3.361838in}{2.391503in}}%
\pgfpathlineto{\pgfqpoint{3.369593in}{2.401188in}}%
\pgfpathlineto{\pgfqpoint{3.377342in}{2.410935in}}%
\pgfpathlineto{\pgfqpoint{3.385085in}{2.420743in}}%
\pgfpathlineto{\pgfqpoint{3.372149in}{2.429281in}}%
\pgfpathlineto{\pgfqpoint{3.359213in}{2.438035in}}%
\pgfpathlineto{\pgfqpoint{3.346277in}{2.447006in}}%
\pgfpathlineto{\pgfqpoint{3.333340in}{2.456196in}}%
\pgfpathlineto{\pgfqpoint{3.325585in}{2.446377in}}%
\pgfpathlineto{\pgfqpoint{3.317823in}{2.436626in}}%
\pgfpathlineto{\pgfqpoint{3.310055in}{2.426943in}}%
\pgfpathlineto{\pgfqpoint{3.302282in}{2.417328in}}%
\pgfpathclose%
\pgfusepath{fill}%
\end{pgfscope}%
\begin{pgfscope}%
\pgfpathrectangle{\pgfqpoint{1.254980in}{0.150000in}}{\pgfqpoint{5.490039in}{5.490039in}}%
\pgfusepath{clip}%
\pgfsetbuttcap%
\pgfsetroundjoin%
\definecolor{currentfill}{rgb}{0.277134,0.185228,0.489898}%
\pgfsetfillcolor{currentfill}%
\pgfsetfillopacity{0.700000}%
\pgfsetlinewidth{0.000000pt}%
\definecolor{currentstroke}{rgb}{0.000000,0.000000,0.000000}%
\pgfsetstrokecolor{currentstroke}%
\pgfsetdash{}{0pt}%
\pgfpathmoveto{\pgfqpoint{4.086630in}{2.475467in}}%
\pgfpathlineto{\pgfqpoint{4.099671in}{2.474231in}}%
\pgfpathlineto{\pgfqpoint{4.112720in}{2.473171in}}%
\pgfpathlineto{\pgfqpoint{4.125775in}{2.472285in}}%
\pgfpathlineto{\pgfqpoint{4.138838in}{2.471573in}}%
\pgfpathlineto{\pgfqpoint{4.146348in}{2.481265in}}%
\pgfpathlineto{\pgfqpoint{4.153854in}{2.490983in}}%
\pgfpathlineto{\pgfqpoint{4.161355in}{2.500728in}}%
\pgfpathlineto{\pgfqpoint{4.168851in}{2.510503in}}%
\pgfpathlineto{\pgfqpoint{4.155796in}{2.511412in}}%
\pgfpathlineto{\pgfqpoint{4.142749in}{2.512494in}}%
\pgfpathlineto{\pgfqpoint{4.129710in}{2.513751in}}%
\pgfpathlineto{\pgfqpoint{4.116677in}{2.515183in}}%
\pgfpathlineto{\pgfqpoint{4.109173in}{2.505202in}}%
\pgfpathlineto{\pgfqpoint{4.101663in}{2.495257in}}%
\pgfpathlineto{\pgfqpoint{4.094149in}{2.485346in}}%
\pgfpathlineto{\pgfqpoint{4.086630in}{2.475467in}}%
\pgfpathclose%
\pgfusepath{fill}%
\end{pgfscope}%
\begin{pgfscope}%
\pgfpathrectangle{\pgfqpoint{1.254980in}{0.150000in}}{\pgfqpoint{5.490039in}{5.490039in}}%
\pgfusepath{clip}%
\pgfsetbuttcap%
\pgfsetroundjoin%
\definecolor{currentfill}{rgb}{0.151918,0.500685,0.557587}%
\pgfsetfillcolor{currentfill}%
\pgfsetfillopacity{0.700000}%
\pgfsetlinewidth{0.000000pt}%
\definecolor{currentstroke}{rgb}{0.000000,0.000000,0.000000}%
\pgfsetstrokecolor{currentstroke}%
\pgfsetdash{}{0pt}%
\pgfpathmoveto{\pgfqpoint{5.484234in}{3.209060in}}%
\pgfpathlineto{\pgfqpoint{5.497747in}{3.211545in}}%
\pgfpathlineto{\pgfqpoint{5.511274in}{3.214179in}}%
\pgfpathlineto{\pgfqpoint{5.524814in}{3.216961in}}%
\pgfpathlineto{\pgfqpoint{5.538367in}{3.219892in}}%
\pgfpathlineto{\pgfqpoint{5.545429in}{3.229427in}}%
\pgfpathlineto{\pgfqpoint{5.552493in}{3.239209in}}%
\pgfpathlineto{\pgfqpoint{5.559561in}{3.249244in}}%
\pgfpathlineto{\pgfqpoint{5.546027in}{3.246816in}}%
\pgfpathlineto{\pgfqpoint{5.532507in}{3.244536in}}%
\pgfpathlineto{\pgfqpoint{5.519000in}{3.242403in}}%
\pgfpathlineto{\pgfqpoint{5.505506in}{3.240419in}}%
\pgfpathlineto{\pgfqpoint{5.498412in}{3.229708in}}%
\pgfpathlineto{\pgfqpoint{5.491322in}{3.219258in}}%
\pgfpathlineto{\pgfqpoint{5.484234in}{3.209060in}}%
\pgfpathclose%
\pgfusepath{fill}%
\end{pgfscope}%
\begin{pgfscope}%
\pgfpathrectangle{\pgfqpoint{1.254980in}{0.150000in}}{\pgfqpoint{5.490039in}{5.490039in}}%
\pgfusepath{clip}%
\pgfsetbuttcap%
\pgfsetroundjoin%
\definecolor{currentfill}{rgb}{0.276194,0.190074,0.493001}%
\pgfsetfillcolor{currentfill}%
\pgfsetfillopacity{0.700000}%
\pgfsetlinewidth{0.000000pt}%
\definecolor{currentstroke}{rgb}{0.000000,0.000000,0.000000}%
\pgfsetstrokecolor{currentstroke}%
\pgfsetdash{}{0pt}%
\pgfpathmoveto{\pgfqpoint{3.115472in}{2.507740in}}%
\pgfpathlineto{\pgfqpoint{3.128455in}{2.495778in}}%
\pgfpathlineto{\pgfqpoint{3.141435in}{2.484054in}}%
\pgfpathlineto{\pgfqpoint{3.154412in}{2.472567in}}%
\pgfpathlineto{\pgfqpoint{3.167387in}{2.461314in}}%
\pgfpathlineto{\pgfqpoint{3.175214in}{2.470587in}}%
\pgfpathlineto{\pgfqpoint{3.183036in}{2.479940in}}%
\pgfpathlineto{\pgfqpoint{3.190850in}{2.489373in}}%
\pgfpathlineto{\pgfqpoint{3.198658in}{2.498888in}}%
\pgfpathlineto{\pgfqpoint{3.185698in}{2.510113in}}%
\pgfpathlineto{\pgfqpoint{3.172735in}{2.521573in}}%
\pgfpathlineto{\pgfqpoint{3.159771in}{2.533268in}}%
\pgfpathlineto{\pgfqpoint{3.146803in}{2.545202in}}%
\pgfpathlineto{\pgfqpoint{3.138981in}{2.535706in}}%
\pgfpathlineto{\pgfqpoint{3.131152in}{2.526296in}}%
\pgfpathlineto{\pgfqpoint{3.123315in}{2.516975in}}%
\pgfpathlineto{\pgfqpoint{3.115472in}{2.507740in}}%
\pgfpathclose%
\pgfusepath{fill}%
\end{pgfscope}%
\begin{pgfscope}%
\pgfpathrectangle{\pgfqpoint{1.254980in}{0.150000in}}{\pgfqpoint{5.490039in}{5.490039in}}%
\pgfusepath{clip}%
\pgfsetbuttcap%
\pgfsetroundjoin%
\definecolor{currentfill}{rgb}{0.201239,0.383670,0.554294}%
\pgfsetfillcolor{currentfill}%
\pgfsetfillopacity{0.700000}%
\pgfsetlinewidth{0.000000pt}%
\definecolor{currentstroke}{rgb}{0.000000,0.000000,0.000000}%
\pgfsetstrokecolor{currentstroke}%
\pgfsetdash{}{0pt}%
\pgfpathmoveto{\pgfqpoint{2.749873in}{2.947820in}}%
\pgfpathlineto{\pgfqpoint{2.763032in}{2.928308in}}%
\pgfpathlineto{\pgfqpoint{2.776180in}{2.909100in}}%
\pgfpathlineto{\pgfqpoint{2.789319in}{2.890193in}}%
\pgfpathlineto{\pgfqpoint{2.802449in}{2.871585in}}%
\pgfpathlineto{\pgfqpoint{2.810399in}{2.880575in}}%
\pgfpathlineto{\pgfqpoint{2.818340in}{2.889695in}}%
\pgfpathlineto{\pgfqpoint{2.826273in}{2.898945in}}%
\pgfpathlineto{\pgfqpoint{2.834198in}{2.908324in}}%
\pgfpathlineto{\pgfqpoint{2.821088in}{2.926900in}}%
\pgfpathlineto{\pgfqpoint{2.807969in}{2.945774in}}%
\pgfpathlineto{\pgfqpoint{2.794840in}{2.964950in}}%
\pgfpathlineto{\pgfqpoint{2.781702in}{2.984429in}}%
\pgfpathlineto{\pgfqpoint{2.773758in}{2.975072in}}%
\pgfpathlineto{\pgfqpoint{2.765806in}{2.965851in}}%
\pgfpathlineto{\pgfqpoint{2.757844in}{2.956768in}}%
\pgfpathlineto{\pgfqpoint{2.749873in}{2.947820in}}%
\pgfpathclose%
\pgfusepath{fill}%
\end{pgfscope}%
\begin{pgfscope}%
\pgfpathrectangle{\pgfqpoint{1.254980in}{0.150000in}}{\pgfqpoint{5.490039in}{5.490039in}}%
\pgfusepath{clip}%
\pgfsetbuttcap%
\pgfsetroundjoin%
\definecolor{currentfill}{rgb}{0.283187,0.125848,0.444960}%
\pgfsetfillcolor{currentfill}%
\pgfsetfillopacity{0.700000}%
\pgfsetlinewidth{0.000000pt}%
\definecolor{currentstroke}{rgb}{0.000000,0.000000,0.000000}%
\pgfsetstrokecolor{currentstroke}%
\pgfsetdash{}{0pt}%
\pgfpathmoveto{\pgfqpoint{3.705452in}{2.372345in}}%
\pgfpathlineto{\pgfqpoint{3.718415in}{2.367964in}}%
\pgfpathlineto{\pgfqpoint{3.731381in}{2.363774in}}%
\pgfpathlineto{\pgfqpoint{3.744353in}{2.359773in}}%
\pgfpathlineto{\pgfqpoint{3.757328in}{2.355962in}}%
\pgfpathlineto{\pgfqpoint{3.764960in}{2.365913in}}%
\pgfpathlineto{\pgfqpoint{3.772587in}{2.375895in}}%
\pgfpathlineto{\pgfqpoint{3.780209in}{2.385907in}}%
\pgfpathlineto{\pgfqpoint{3.787825in}{2.395952in}}%
\pgfpathlineto{\pgfqpoint{3.774859in}{2.399849in}}%
\pgfpathlineto{\pgfqpoint{3.761897in}{2.403934in}}%
\pgfpathlineto{\pgfqpoint{3.748939in}{2.408210in}}%
\pgfpathlineto{\pgfqpoint{3.735986in}{2.412675in}}%
\pgfpathlineto{\pgfqpoint{3.728360in}{2.402536in}}%
\pgfpathlineto{\pgfqpoint{3.720729in}{2.392435in}}%
\pgfpathlineto{\pgfqpoint{3.713093in}{2.382372in}}%
\pgfpathlineto{\pgfqpoint{3.705452in}{2.372345in}}%
\pgfpathclose%
\pgfusepath{fill}%
\end{pgfscope}%
\begin{pgfscope}%
\pgfpathrectangle{\pgfqpoint{1.254980in}{0.150000in}}{\pgfqpoint{5.490039in}{5.490039in}}%
\pgfusepath{clip}%
\pgfsetbuttcap%
\pgfsetroundjoin%
\definecolor{currentfill}{rgb}{0.280255,0.165693,0.476498}%
\pgfsetfillcolor{currentfill}%
\pgfsetfillopacity{0.700000}%
\pgfsetlinewidth{0.000000pt}%
\definecolor{currentstroke}{rgb}{0.000000,0.000000,0.000000}%
\pgfsetstrokecolor{currentstroke}%
\pgfsetdash{}{0pt}%
\pgfpathmoveto{\pgfqpoint{4.004379in}{2.442250in}}%
\pgfpathlineto{\pgfqpoint{4.017401in}{2.440477in}}%
\pgfpathlineto{\pgfqpoint{4.030429in}{2.438881in}}%
\pgfpathlineto{\pgfqpoint{4.043465in}{2.437462in}}%
\pgfpathlineto{\pgfqpoint{4.056507in}{2.436220in}}%
\pgfpathlineto{\pgfqpoint{4.064045in}{2.445996in}}%
\pgfpathlineto{\pgfqpoint{4.071578in}{2.455794in}}%
\pgfpathlineto{\pgfqpoint{4.079107in}{2.465617in}}%
\pgfpathlineto{\pgfqpoint{4.086630in}{2.475467in}}%
\pgfpathlineto{\pgfqpoint{4.073597in}{2.476878in}}%
\pgfpathlineto{\pgfqpoint{4.060570in}{2.478466in}}%
\pgfpathlineto{\pgfqpoint{4.047549in}{2.480230in}}%
\pgfpathlineto{\pgfqpoint{4.034536in}{2.482172in}}%
\pgfpathlineto{\pgfqpoint{4.027004in}{2.472144in}}%
\pgfpathlineto{\pgfqpoint{4.019467in}{2.462148in}}%
\pgfpathlineto{\pgfqpoint{4.011925in}{2.452185in}}%
\pgfpathlineto{\pgfqpoint{4.004379in}{2.442250in}}%
\pgfpathclose%
\pgfusepath{fill}%
\end{pgfscope}%
\begin{pgfscope}%
\pgfpathrectangle{\pgfqpoint{1.254980in}{0.150000in}}{\pgfqpoint{5.490039in}{5.490039in}}%
\pgfusepath{clip}%
\pgfsetbuttcap%
\pgfsetroundjoin%
\definecolor{currentfill}{rgb}{0.279574,0.170599,0.479997}%
\pgfsetfillcolor{currentfill}%
\pgfsetfillopacity{0.700000}%
\pgfsetlinewidth{0.000000pt}%
\definecolor{currentstroke}{rgb}{0.000000,0.000000,0.000000}%
\pgfsetstrokecolor{currentstroke}%
\pgfsetdash{}{0pt}%
\pgfpathmoveto{\pgfqpoint{3.167387in}{2.461314in}}%
\pgfpathlineto{\pgfqpoint{3.180359in}{2.450294in}}%
\pgfpathlineto{\pgfqpoint{3.193330in}{2.439505in}}%
\pgfpathlineto{\pgfqpoint{3.206299in}{2.428946in}}%
\pgfpathlineto{\pgfqpoint{3.219266in}{2.418614in}}%
\pgfpathlineto{\pgfqpoint{3.227079in}{2.427925in}}%
\pgfpathlineto{\pgfqpoint{3.234886in}{2.437309in}}%
\pgfpathlineto{\pgfqpoint{3.242686in}{2.446767in}}%
\pgfpathlineto{\pgfqpoint{3.250479in}{2.456298in}}%
\pgfpathlineto{\pgfqpoint{3.237526in}{2.466602in}}%
\pgfpathlineto{\pgfqpoint{3.224572in}{2.477134in}}%
\pgfpathlineto{\pgfqpoint{3.211616in}{2.487895in}}%
\pgfpathlineto{\pgfqpoint{3.198658in}{2.498888in}}%
\pgfpathlineto{\pgfqpoint{3.190850in}{2.489373in}}%
\pgfpathlineto{\pgfqpoint{3.183036in}{2.479940in}}%
\pgfpathlineto{\pgfqpoint{3.175214in}{2.470587in}}%
\pgfpathlineto{\pgfqpoint{3.167387in}{2.461314in}}%
\pgfpathclose%
\pgfusepath{fill}%
\end{pgfscope}%
\begin{pgfscope}%
\pgfpathrectangle{\pgfqpoint{1.254980in}{0.150000in}}{\pgfqpoint{5.490039in}{5.490039in}}%
\pgfusepath{clip}%
\pgfsetbuttcap%
\pgfsetroundjoin%
\definecolor{currentfill}{rgb}{0.283229,0.120777,0.440584}%
\pgfsetfillcolor{currentfill}%
\pgfsetfillopacity{0.700000}%
\pgfsetlinewidth{0.000000pt}%
\definecolor{currentstroke}{rgb}{0.000000,0.000000,0.000000}%
\pgfsetstrokecolor{currentstroke}%
\pgfsetdash{}{0pt}%
\pgfpathmoveto{\pgfqpoint{3.488606in}{2.360037in}}%
\pgfpathlineto{\pgfqpoint{3.501553in}{2.353380in}}%
\pgfpathlineto{\pgfqpoint{3.514502in}{2.346926in}}%
\pgfpathlineto{\pgfqpoint{3.527452in}{2.340674in}}%
\pgfpathlineto{\pgfqpoint{3.540406in}{2.334622in}}%
\pgfpathlineto{\pgfqpoint{3.548109in}{2.344436in}}%
\pgfpathlineto{\pgfqpoint{3.555807in}{2.354292in}}%
\pgfpathlineto{\pgfqpoint{3.563500in}{2.364190in}}%
\pgfpathlineto{\pgfqpoint{3.571187in}{2.374133in}}%
\pgfpathlineto{\pgfqpoint{3.558244in}{2.380214in}}%
\pgfpathlineto{\pgfqpoint{3.545304in}{2.386496in}}%
\pgfpathlineto{\pgfqpoint{3.532366in}{2.392980in}}%
\pgfpathlineto{\pgfqpoint{3.519430in}{2.399666in}}%
\pgfpathlineto{\pgfqpoint{3.511733in}{2.389684in}}%
\pgfpathlineto{\pgfqpoint{3.504029in}{2.379752in}}%
\pgfpathlineto{\pgfqpoint{3.496321in}{2.369870in}}%
\pgfpathlineto{\pgfqpoint{3.488606in}{2.360037in}}%
\pgfpathclose%
\pgfusepath{fill}%
\end{pgfscope}%
\begin{pgfscope}%
\pgfpathrectangle{\pgfqpoint{1.254980in}{0.150000in}}{\pgfqpoint{5.490039in}{5.490039in}}%
\pgfusepath{clip}%
\pgfsetbuttcap%
\pgfsetroundjoin%
\definecolor{currentfill}{rgb}{0.283072,0.130895,0.449241}%
\pgfsetfillcolor{currentfill}%
\pgfsetfillopacity{0.700000}%
\pgfsetlinewidth{0.000000pt}%
\definecolor{currentstroke}{rgb}{0.000000,0.000000,0.000000}%
\pgfsetstrokecolor{currentstroke}%
\pgfsetdash{}{0pt}%
\pgfpathmoveto{\pgfqpoint{3.354077in}{2.381878in}}%
\pgfpathlineto{\pgfqpoint{3.367026in}{2.373555in}}%
\pgfpathlineto{\pgfqpoint{3.379975in}{2.365445in}}%
\pgfpathlineto{\pgfqpoint{3.392925in}{2.357547in}}%
\pgfpathlineto{\pgfqpoint{3.405877in}{2.349859in}}%
\pgfpathlineto{\pgfqpoint{3.413625in}{2.359492in}}%
\pgfpathlineto{\pgfqpoint{3.421368in}{2.369180in}}%
\pgfpathlineto{\pgfqpoint{3.429105in}{2.378921in}}%
\pgfpathlineto{\pgfqpoint{3.436836in}{2.388718in}}%
\pgfpathlineto{\pgfqpoint{3.423897in}{2.396407in}}%
\pgfpathlineto{\pgfqpoint{3.410959in}{2.404307in}}%
\pgfpathlineto{\pgfqpoint{3.398022in}{2.412418in}}%
\pgfpathlineto{\pgfqpoint{3.385085in}{2.420743in}}%
\pgfpathlineto{\pgfqpoint{3.377342in}{2.410935in}}%
\pgfpathlineto{\pgfqpoint{3.369593in}{2.401188in}}%
\pgfpathlineto{\pgfqpoint{3.361838in}{2.391503in}}%
\pgfpathlineto{\pgfqpoint{3.354077in}{2.381878in}}%
\pgfpathclose%
\pgfusepath{fill}%
\end{pgfscope}%
\begin{pgfscope}%
\pgfpathrectangle{\pgfqpoint{1.254980in}{0.150000in}}{\pgfqpoint{5.490039in}{5.490039in}}%
\pgfusepath{clip}%
\pgfsetbuttcap%
\pgfsetroundjoin%
\definecolor{currentfill}{rgb}{0.281887,0.150881,0.465405}%
\pgfsetfillcolor{currentfill}%
\pgfsetfillopacity{0.700000}%
\pgfsetlinewidth{0.000000pt}%
\definecolor{currentstroke}{rgb}{0.000000,0.000000,0.000000}%
\pgfsetstrokecolor{currentstroke}%
\pgfsetdash{}{0pt}%
\pgfpathmoveto{\pgfqpoint{3.922085in}{2.411087in}}%
\pgfpathlineto{\pgfqpoint{3.935091in}{2.408736in}}%
\pgfpathlineto{\pgfqpoint{3.948102in}{2.406566in}}%
\pgfpathlineto{\pgfqpoint{3.961120in}{2.404576in}}%
\pgfpathlineto{\pgfqpoint{3.974144in}{2.402765in}}%
\pgfpathlineto{\pgfqpoint{3.981710in}{2.412603in}}%
\pgfpathlineto{\pgfqpoint{3.989271in}{2.422461in}}%
\pgfpathlineto{\pgfqpoint{3.996827in}{2.432343in}}%
\pgfpathlineto{\pgfqpoint{4.004379in}{2.442250in}}%
\pgfpathlineto{\pgfqpoint{3.991363in}{2.444202in}}%
\pgfpathlineto{\pgfqpoint{3.978354in}{2.446333in}}%
\pgfpathlineto{\pgfqpoint{3.965351in}{2.448644in}}%
\pgfpathlineto{\pgfqpoint{3.952354in}{2.451135in}}%
\pgfpathlineto{\pgfqpoint{3.944794in}{2.441077in}}%
\pgfpathlineto{\pgfqpoint{3.937229in}{2.431051in}}%
\pgfpathlineto{\pgfqpoint{3.929660in}{2.421055in}}%
\pgfpathlineto{\pgfqpoint{3.922085in}{2.411087in}}%
\pgfpathclose%
\pgfusepath{fill}%
\end{pgfscope}%
\begin{pgfscope}%
\pgfpathrectangle{\pgfqpoint{1.254980in}{0.150000in}}{\pgfqpoint{5.490039in}{5.490039in}}%
\pgfusepath{clip}%
\pgfsetbuttcap%
\pgfsetroundjoin%
\definecolor{currentfill}{rgb}{0.283229,0.120777,0.440584}%
\pgfsetfillcolor{currentfill}%
\pgfsetfillopacity{0.700000}%
\pgfsetlinewidth{0.000000pt}%
\definecolor{currentstroke}{rgb}{0.000000,0.000000,0.000000}%
\pgfsetstrokecolor{currentstroke}%
\pgfsetdash{}{0pt}%
\pgfpathmoveto{\pgfqpoint{3.622985in}{2.351792in}}%
\pgfpathlineto{\pgfqpoint{3.635942in}{2.346698in}}%
\pgfpathlineto{\pgfqpoint{3.648903in}{2.341798in}}%
\pgfpathlineto{\pgfqpoint{3.661868in}{2.337091in}}%
\pgfpathlineto{\pgfqpoint{3.674836in}{2.332577in}}%
\pgfpathlineto{\pgfqpoint{3.682498in}{2.342470in}}%
\pgfpathlineto{\pgfqpoint{3.690154in}{2.352395in}}%
\pgfpathlineto{\pgfqpoint{3.697806in}{2.362353in}}%
\pgfpathlineto{\pgfqpoint{3.705452in}{2.372345in}}%
\pgfpathlineto{\pgfqpoint{3.692493in}{2.376917in}}%
\pgfpathlineto{\pgfqpoint{3.679539in}{2.381681in}}%
\pgfpathlineto{\pgfqpoint{3.666588in}{2.386638in}}%
\pgfpathlineto{\pgfqpoint{3.653640in}{2.391790in}}%
\pgfpathlineto{\pgfqpoint{3.645984in}{2.381731in}}%
\pgfpathlineto{\pgfqpoint{3.638323in}{2.371712in}}%
\pgfpathlineto{\pgfqpoint{3.630657in}{2.361733in}}%
\pgfpathlineto{\pgfqpoint{3.622985in}{2.351792in}}%
\pgfpathclose%
\pgfusepath{fill}%
\end{pgfscope}%
\begin{pgfscope}%
\pgfpathrectangle{\pgfqpoint{1.254980in}{0.150000in}}{\pgfqpoint{5.490039in}{5.490039in}}%
\pgfusepath{clip}%
\pgfsetbuttcap%
\pgfsetroundjoin%
\definecolor{currentfill}{rgb}{0.282623,0.140926,0.457517}%
\pgfsetfillcolor{currentfill}%
\pgfsetfillopacity{0.700000}%
\pgfsetlinewidth{0.000000pt}%
\definecolor{currentstroke}{rgb}{0.000000,0.000000,0.000000}%
\pgfsetstrokecolor{currentstroke}%
\pgfsetdash{}{0pt}%
\pgfpathmoveto{\pgfqpoint{3.839740in}{2.382231in}}%
\pgfpathlineto{\pgfqpoint{3.852731in}{2.379264in}}%
\pgfpathlineto{\pgfqpoint{3.865728in}{2.376481in}}%
\pgfpathlineto{\pgfqpoint{3.878730in}{2.373880in}}%
\pgfpathlineto{\pgfqpoint{3.891738in}{2.371462in}}%
\pgfpathlineto{\pgfqpoint{3.899332in}{2.381334in}}%
\pgfpathlineto{\pgfqpoint{3.906922in}{2.391228in}}%
\pgfpathlineto{\pgfqpoint{3.914506in}{2.401145in}}%
\pgfpathlineto{\pgfqpoint{3.922085in}{2.411087in}}%
\pgfpathlineto{\pgfqpoint{3.909086in}{2.413619in}}%
\pgfpathlineto{\pgfqpoint{3.896092in}{2.416332in}}%
\pgfpathlineto{\pgfqpoint{3.883104in}{2.419229in}}%
\pgfpathlineto{\pgfqpoint{3.870121in}{2.422309in}}%
\pgfpathlineto{\pgfqpoint{3.862533in}{2.412244in}}%
\pgfpathlineto{\pgfqpoint{3.854940in}{2.402210in}}%
\pgfpathlineto{\pgfqpoint{3.847342in}{2.392207in}}%
\pgfpathlineto{\pgfqpoint{3.839740in}{2.382231in}}%
\pgfpathclose%
\pgfusepath{fill}%
\end{pgfscope}%
\begin{pgfscope}%
\pgfpathrectangle{\pgfqpoint{1.254980in}{0.150000in}}{\pgfqpoint{5.490039in}{5.490039in}}%
\pgfusepath{clip}%
\pgfsetbuttcap%
\pgfsetroundjoin%
\definecolor{currentfill}{rgb}{0.229739,0.322361,0.545706}%
\pgfsetfillcolor{currentfill}%
\pgfsetfillopacity{0.700000}%
\pgfsetlinewidth{0.000000pt}%
\definecolor{currentstroke}{rgb}{0.000000,0.000000,0.000000}%
\pgfsetstrokecolor{currentstroke}%
\pgfsetdash{}{0pt}%
\pgfpathmoveto{\pgfqpoint{4.714823in}{2.755448in}}%
\pgfpathlineto{\pgfqpoint{4.728084in}{2.757586in}}%
\pgfpathlineto{\pgfqpoint{4.741357in}{2.759885in}}%
\pgfpathlineto{\pgfqpoint{4.754641in}{2.762343in}}%
\pgfpathlineto{\pgfqpoint{4.767936in}{2.764960in}}%
\pgfpathlineto{\pgfqpoint{4.775235in}{2.773473in}}%
\pgfpathlineto{\pgfqpoint{4.782530in}{2.782052in}}%
\pgfpathlineto{\pgfqpoint{4.789822in}{2.790704in}}%
\pgfpathlineto{\pgfqpoint{4.797109in}{2.799431in}}%
\pgfpathlineto{\pgfqpoint{4.783827in}{2.797208in}}%
\pgfpathlineto{\pgfqpoint{4.770557in}{2.795144in}}%
\pgfpathlineto{\pgfqpoint{4.757298in}{2.793239in}}%
\pgfpathlineto{\pgfqpoint{4.744049in}{2.791493in}}%
\pgfpathlineto{\pgfqpoint{4.736748in}{2.782362in}}%
\pgfpathlineto{\pgfqpoint{4.729443in}{2.773314in}}%
\pgfpathlineto{\pgfqpoint{4.722135in}{2.764344in}}%
\pgfpathlineto{\pgfqpoint{4.714823in}{2.755448in}}%
\pgfpathclose%
\pgfusepath{fill}%
\end{pgfscope}%
\begin{pgfscope}%
\pgfpathrectangle{\pgfqpoint{1.254980in}{0.150000in}}{\pgfqpoint{5.490039in}{5.490039in}}%
\pgfusepath{clip}%
\pgfsetbuttcap%
\pgfsetroundjoin%
\definecolor{currentfill}{rgb}{0.220057,0.343307,0.549413}%
\pgfsetfillcolor{currentfill}%
\pgfsetfillopacity{0.700000}%
\pgfsetlinewidth{0.000000pt}%
\definecolor{currentstroke}{rgb}{0.000000,0.000000,0.000000}%
\pgfsetstrokecolor{currentstroke}%
\pgfsetdash{}{0pt}%
\pgfpathmoveto{\pgfqpoint{4.797109in}{2.799431in}}%
\pgfpathlineto{\pgfqpoint{4.810402in}{2.801813in}}%
\pgfpathlineto{\pgfqpoint{4.823706in}{2.804353in}}%
\pgfpathlineto{\pgfqpoint{4.837021in}{2.807050in}}%
\pgfpathlineto{\pgfqpoint{4.850348in}{2.809906in}}%
\pgfpathlineto{\pgfqpoint{4.857618in}{2.818303in}}%
\pgfpathlineto{\pgfqpoint{4.864883in}{2.826778in}}%
\pgfpathlineto{\pgfqpoint{4.872145in}{2.835337in}}%
\pgfpathlineto{\pgfqpoint{4.879404in}{2.843984in}}%
\pgfpathlineto{\pgfqpoint{4.866091in}{2.841551in}}%
\pgfpathlineto{\pgfqpoint{4.852790in}{2.839275in}}%
\pgfpathlineto{\pgfqpoint{4.839501in}{2.837156in}}%
\pgfpathlineto{\pgfqpoint{4.826222in}{2.835196in}}%
\pgfpathlineto{\pgfqpoint{4.818949in}{2.826117in}}%
\pgfpathlineto{\pgfqpoint{4.811673in}{2.817133in}}%
\pgfpathlineto{\pgfqpoint{4.804393in}{2.808239in}}%
\pgfpathlineto{\pgfqpoint{4.797109in}{2.799431in}}%
\pgfpathclose%
\pgfusepath{fill}%
\end{pgfscope}%
\begin{pgfscope}%
\pgfpathrectangle{\pgfqpoint{1.254980in}{0.150000in}}{\pgfqpoint{5.490039in}{5.490039in}}%
\pgfusepath{clip}%
\pgfsetbuttcap%
\pgfsetroundjoin%
\definecolor{currentfill}{rgb}{0.281887,0.150881,0.465405}%
\pgfsetfillcolor{currentfill}%
\pgfsetfillopacity{0.700000}%
\pgfsetlinewidth{0.000000pt}%
\definecolor{currentstroke}{rgb}{0.000000,0.000000,0.000000}%
\pgfsetstrokecolor{currentstroke}%
\pgfsetdash{}{0pt}%
\pgfpathmoveto{\pgfqpoint{3.219266in}{2.418614in}}%
\pgfpathlineto{\pgfqpoint{3.232232in}{2.408509in}}%
\pgfpathlineto{\pgfqpoint{3.245197in}{2.398629in}}%
\pgfpathlineto{\pgfqpoint{3.258161in}{2.388972in}}%
\pgfpathlineto{\pgfqpoint{3.271124in}{2.379536in}}%
\pgfpathlineto{\pgfqpoint{3.278923in}{2.388884in}}%
\pgfpathlineto{\pgfqpoint{3.286715in}{2.398298in}}%
\pgfpathlineto{\pgfqpoint{3.294501in}{2.407779in}}%
\pgfpathlineto{\pgfqpoint{3.302282in}{2.417328in}}%
\pgfpathlineto{\pgfqpoint{3.289332in}{2.426736in}}%
\pgfpathlineto{\pgfqpoint{3.276382in}{2.436367in}}%
\pgfpathlineto{\pgfqpoint{3.263431in}{2.446220in}}%
\pgfpathlineto{\pgfqpoint{3.250479in}{2.456298in}}%
\pgfpathlineto{\pgfqpoint{3.242686in}{2.446767in}}%
\pgfpathlineto{\pgfqpoint{3.234886in}{2.437309in}}%
\pgfpathlineto{\pgfqpoint{3.227079in}{2.427925in}}%
\pgfpathlineto{\pgfqpoint{3.219266in}{2.418614in}}%
\pgfpathclose%
\pgfusepath{fill}%
\end{pgfscope}%
\begin{pgfscope}%
\pgfpathrectangle{\pgfqpoint{1.254980in}{0.150000in}}{\pgfqpoint{5.490039in}{5.490039in}}%
\pgfusepath{clip}%
\pgfsetbuttcap%
\pgfsetroundjoin%
\definecolor{currentfill}{rgb}{0.237441,0.305202,0.541921}%
\pgfsetfillcolor{currentfill}%
\pgfsetfillopacity{0.700000}%
\pgfsetlinewidth{0.000000pt}%
\definecolor{currentstroke}{rgb}{0.000000,0.000000,0.000000}%
\pgfsetstrokecolor{currentstroke}%
\pgfsetdash{}{0pt}%
\pgfpathmoveto{\pgfqpoint{4.632543in}{2.712099in}}%
\pgfpathlineto{\pgfqpoint{4.645774in}{2.713960in}}%
\pgfpathlineto{\pgfqpoint{4.659016in}{2.715983in}}%
\pgfpathlineto{\pgfqpoint{4.672269in}{2.718167in}}%
\pgfpathlineto{\pgfqpoint{4.685532in}{2.720512in}}%
\pgfpathlineto{\pgfqpoint{4.692861in}{2.729157in}}%
\pgfpathlineto{\pgfqpoint{4.700186in}{2.737859in}}%
\pgfpathlineto{\pgfqpoint{4.707506in}{2.746621in}}%
\pgfpathlineto{\pgfqpoint{4.714823in}{2.755448in}}%
\pgfpathlineto{\pgfqpoint{4.701572in}{2.753469in}}%
\pgfpathlineto{\pgfqpoint{4.688331in}{2.751651in}}%
\pgfpathlineto{\pgfqpoint{4.675102in}{2.749993in}}%
\pgfpathlineto{\pgfqpoint{4.661882in}{2.748496in}}%
\pgfpathlineto{\pgfqpoint{4.654554in}{2.739294in}}%
\pgfpathlineto{\pgfqpoint{4.647221in}{2.730163in}}%
\pgfpathlineto{\pgfqpoint{4.639884in}{2.721099in}}%
\pgfpathlineto{\pgfqpoint{4.632543in}{2.712099in}}%
\pgfpathclose%
\pgfusepath{fill}%
\end{pgfscope}%
\begin{pgfscope}%
\pgfpathrectangle{\pgfqpoint{1.254980in}{0.150000in}}{\pgfqpoint{5.490039in}{5.490039in}}%
\pgfusepath{clip}%
\pgfsetbuttcap%
\pgfsetroundjoin%
\definecolor{currentfill}{rgb}{0.210503,0.363727,0.552206}%
\pgfsetfillcolor{currentfill}%
\pgfsetfillopacity{0.700000}%
\pgfsetlinewidth{0.000000pt}%
\definecolor{currentstroke}{rgb}{0.000000,0.000000,0.000000}%
\pgfsetstrokecolor{currentstroke}%
\pgfsetdash{}{0pt}%
\pgfpathmoveto{\pgfqpoint{4.879404in}{2.843984in}}%
\pgfpathlineto{\pgfqpoint{4.892728in}{2.846575in}}%
\pgfpathlineto{\pgfqpoint{4.906063in}{2.849322in}}%
\pgfpathlineto{\pgfqpoint{4.919411in}{2.852226in}}%
\pgfpathlineto{\pgfqpoint{4.932770in}{2.855287in}}%
\pgfpathlineto{\pgfqpoint{4.940010in}{2.863588in}}%
\pgfpathlineto{\pgfqpoint{4.947246in}{2.871981in}}%
\pgfpathlineto{\pgfqpoint{4.954479in}{2.880470in}}%
\pgfpathlineto{\pgfqpoint{4.961709in}{2.889062in}}%
\pgfpathlineto{\pgfqpoint{4.948366in}{2.886451in}}%
\pgfpathlineto{\pgfqpoint{4.935034in}{2.883997in}}%
\pgfpathlineto{\pgfqpoint{4.921714in}{2.881699in}}%
\pgfpathlineto{\pgfqpoint{4.908406in}{2.879558in}}%
\pgfpathlineto{\pgfqpoint{4.901160in}{2.870507in}}%
\pgfpathlineto{\pgfqpoint{4.893911in}{2.861564in}}%
\pgfpathlineto{\pgfqpoint{4.886659in}{2.852725in}}%
\pgfpathlineto{\pgfqpoint{4.879404in}{2.843984in}}%
\pgfpathclose%
\pgfusepath{fill}%
\end{pgfscope}%
\begin{pgfscope}%
\pgfpathrectangle{\pgfqpoint{1.254980in}{0.150000in}}{\pgfqpoint{5.490039in}{5.490039in}}%
\pgfusepath{clip}%
\pgfsetbuttcap%
\pgfsetroundjoin%
\definecolor{currentfill}{rgb}{0.246811,0.283237,0.535941}%
\pgfsetfillcolor{currentfill}%
\pgfsetfillopacity{0.700000}%
\pgfsetlinewidth{0.000000pt}%
\definecolor{currentstroke}{rgb}{0.000000,0.000000,0.000000}%
\pgfsetstrokecolor{currentstroke}%
\pgfsetdash{}{0pt}%
\pgfpathmoveto{\pgfqpoint{4.550267in}{2.669470in}}%
\pgfpathlineto{\pgfqpoint{4.563469in}{2.671020in}}%
\pgfpathlineto{\pgfqpoint{4.576681in}{2.672734in}}%
\pgfpathlineto{\pgfqpoint{4.589903in}{2.674609in}}%
\pgfpathlineto{\pgfqpoint{4.603136in}{2.676647in}}%
\pgfpathlineto{\pgfqpoint{4.610494in}{2.685436in}}%
\pgfpathlineto{\pgfqpoint{4.617848in}{2.694271in}}%
\pgfpathlineto{\pgfqpoint{4.625197in}{2.703157in}}%
\pgfpathlineto{\pgfqpoint{4.632543in}{2.712099in}}%
\pgfpathlineto{\pgfqpoint{4.619322in}{2.710399in}}%
\pgfpathlineto{\pgfqpoint{4.606111in}{2.708860in}}%
\pgfpathlineto{\pgfqpoint{4.592911in}{2.707484in}}%
\pgfpathlineto{\pgfqpoint{4.579720in}{2.706271in}}%
\pgfpathlineto{\pgfqpoint{4.572363in}{2.696982in}}%
\pgfpathlineto{\pgfqpoint{4.565002in}{2.687755in}}%
\pgfpathlineto{\pgfqpoint{4.557637in}{2.678586in}}%
\pgfpathlineto{\pgfqpoint{4.550267in}{2.669470in}}%
\pgfpathclose%
\pgfusepath{fill}%
\end{pgfscope}%
\begin{pgfscope}%
\pgfpathrectangle{\pgfqpoint{1.254980in}{0.150000in}}{\pgfqpoint{5.490039in}{5.490039in}}%
\pgfusepath{clip}%
\pgfsetbuttcap%
\pgfsetroundjoin%
\definecolor{currentfill}{rgb}{0.203063,0.379716,0.553925}%
\pgfsetfillcolor{currentfill}%
\pgfsetfillopacity{0.700000}%
\pgfsetlinewidth{0.000000pt}%
\definecolor{currentstroke}{rgb}{0.000000,0.000000,0.000000}%
\pgfsetstrokecolor{currentstroke}%
\pgfsetdash{}{0pt}%
\pgfpathmoveto{\pgfqpoint{4.961709in}{2.889062in}}%
\pgfpathlineto{\pgfqpoint{4.975065in}{2.891828in}}%
\pgfpathlineto{\pgfqpoint{4.988432in}{2.894749in}}%
\pgfpathlineto{\pgfqpoint{5.001812in}{2.897827in}}%
\pgfpathlineto{\pgfqpoint{5.015203in}{2.901059in}}%
\pgfpathlineto{\pgfqpoint{5.022414in}{2.909290in}}%
\pgfpathlineto{\pgfqpoint{5.029621in}{2.917627in}}%
\pgfpathlineto{\pgfqpoint{5.036825in}{2.926075in}}%
\pgfpathlineto{\pgfqpoint{5.044027in}{2.934640in}}%
\pgfpathlineto{\pgfqpoint{5.030653in}{2.931886in}}%
\pgfpathlineto{\pgfqpoint{5.017290in}{2.929287in}}%
\pgfpathlineto{\pgfqpoint{5.003940in}{2.926843in}}%
\pgfpathlineto{\pgfqpoint{4.990601in}{2.924554in}}%
\pgfpathlineto{\pgfqpoint{4.983382in}{2.915501in}}%
\pgfpathlineto{\pgfqpoint{4.976160in}{2.906572in}}%
\pgfpathlineto{\pgfqpoint{4.968936in}{2.897760in}}%
\pgfpathlineto{\pgfqpoint{4.961709in}{2.889062in}}%
\pgfpathclose%
\pgfusepath{fill}%
\end{pgfscope}%
\begin{pgfscope}%
\pgfpathrectangle{\pgfqpoint{1.254980in}{0.150000in}}{\pgfqpoint{5.490039in}{5.490039in}}%
\pgfusepath{clip}%
\pgfsetbuttcap%
\pgfsetroundjoin%
\definecolor{currentfill}{rgb}{0.253935,0.265254,0.529983}%
\pgfsetfillcolor{currentfill}%
\pgfsetfillopacity{0.700000}%
\pgfsetlinewidth{0.000000pt}%
\definecolor{currentstroke}{rgb}{0.000000,0.000000,0.000000}%
\pgfsetstrokecolor{currentstroke}%
\pgfsetdash{}{0pt}%
\pgfpathmoveto{\pgfqpoint{4.467993in}{2.627668in}}%
\pgfpathlineto{\pgfqpoint{4.481166in}{2.628872in}}%
\pgfpathlineto{\pgfqpoint{4.494349in}{2.630241in}}%
\pgfpathlineto{\pgfqpoint{4.507542in}{2.631773in}}%
\pgfpathlineto{\pgfqpoint{4.520744in}{2.633470in}}%
\pgfpathlineto{\pgfqpoint{4.528132in}{2.642408in}}%
\pgfpathlineto{\pgfqpoint{4.535515in}{2.651385in}}%
\pgfpathlineto{\pgfqpoint{4.542893in}{2.660404in}}%
\pgfpathlineto{\pgfqpoint{4.550267in}{2.669470in}}%
\pgfpathlineto{\pgfqpoint{4.537076in}{2.668083in}}%
\pgfpathlineto{\pgfqpoint{4.523894in}{2.666860in}}%
\pgfpathlineto{\pgfqpoint{4.510722in}{2.665800in}}%
\pgfpathlineto{\pgfqpoint{4.497560in}{2.664905in}}%
\pgfpathlineto{\pgfqpoint{4.490175in}{2.655520in}}%
\pgfpathlineto{\pgfqpoint{4.482785in}{2.646188in}}%
\pgfpathlineto{\pgfqpoint{4.475392in}{2.636905in}}%
\pgfpathlineto{\pgfqpoint{4.467993in}{2.627668in}}%
\pgfpathclose%
\pgfusepath{fill}%
\end{pgfscope}%
\begin{pgfscope}%
\pgfpathrectangle{\pgfqpoint{1.254980in}{0.150000in}}{\pgfqpoint{5.490039in}{5.490039in}}%
\pgfusepath{clip}%
\pgfsetbuttcap%
\pgfsetroundjoin%
\definecolor{currentfill}{rgb}{0.194100,0.399323,0.555565}%
\pgfsetfillcolor{currentfill}%
\pgfsetfillopacity{0.700000}%
\pgfsetlinewidth{0.000000pt}%
\definecolor{currentstroke}{rgb}{0.000000,0.000000,0.000000}%
\pgfsetstrokecolor{currentstroke}%
\pgfsetdash{}{0pt}%
\pgfpathmoveto{\pgfqpoint{5.044027in}{2.934640in}}%
\pgfpathlineto{\pgfqpoint{5.057414in}{2.937548in}}%
\pgfpathlineto{\pgfqpoint{5.070813in}{2.940611in}}%
\pgfpathlineto{\pgfqpoint{5.084225in}{2.943828in}}%
\pgfpathlineto{\pgfqpoint{5.097649in}{2.947200in}}%
\pgfpathlineto{\pgfqpoint{5.104830in}{2.955391in}}%
\pgfpathlineto{\pgfqpoint{5.112009in}{2.963703in}}%
\pgfpathlineto{\pgfqpoint{5.119186in}{2.972143in}}%
\pgfpathlineto{\pgfqpoint{5.126360in}{2.980715in}}%
\pgfpathlineto{\pgfqpoint{5.112955in}{2.977851in}}%
\pgfpathlineto{\pgfqpoint{5.099561in}{2.975140in}}%
\pgfpathlineto{\pgfqpoint{5.086181in}{2.972583in}}%
\pgfpathlineto{\pgfqpoint{5.072812in}{2.970180in}}%
\pgfpathlineto{\pgfqpoint{5.065619in}{2.961092in}}%
\pgfpathlineto{\pgfqpoint{5.058424in}{2.952143in}}%
\pgfpathlineto{\pgfqpoint{5.051227in}{2.943327in}}%
\pgfpathlineto{\pgfqpoint{5.044027in}{2.934640in}}%
\pgfpathclose%
\pgfusepath{fill}%
\end{pgfscope}%
\begin{pgfscope}%
\pgfpathrectangle{\pgfqpoint{1.254980in}{0.150000in}}{\pgfqpoint{5.490039in}{5.490039in}}%
\pgfusepath{clip}%
\pgfsetbuttcap%
\pgfsetroundjoin%
\definecolor{currentfill}{rgb}{0.260571,0.246922,0.522828}%
\pgfsetfillcolor{currentfill}%
\pgfsetfillopacity{0.700000}%
\pgfsetlinewidth{0.000000pt}%
\definecolor{currentstroke}{rgb}{0.000000,0.000000,0.000000}%
\pgfsetstrokecolor{currentstroke}%
\pgfsetdash{}{0pt}%
\pgfpathmoveto{\pgfqpoint{4.385718in}{2.586820in}}%
\pgfpathlineto{\pgfqpoint{4.398863in}{2.587642in}}%
\pgfpathlineto{\pgfqpoint{4.412018in}{2.588631in}}%
\pgfpathlineto{\pgfqpoint{4.425182in}{2.589785in}}%
\pgfpathlineto{\pgfqpoint{4.438355in}{2.591106in}}%
\pgfpathlineto{\pgfqpoint{4.445772in}{2.600195in}}%
\pgfpathlineto{\pgfqpoint{4.453183in}{2.609317in}}%
\pgfpathlineto{\pgfqpoint{4.460591in}{2.618473in}}%
\pgfpathlineto{\pgfqpoint{4.467993in}{2.627668in}}%
\pgfpathlineto{\pgfqpoint{4.454830in}{2.626629in}}%
\pgfpathlineto{\pgfqpoint{4.441676in}{2.625755in}}%
\pgfpathlineto{\pgfqpoint{4.428532in}{2.625048in}}%
\pgfpathlineto{\pgfqpoint{4.415397in}{2.624506in}}%
\pgfpathlineto{\pgfqpoint{4.407984in}{2.615020in}}%
\pgfpathlineto{\pgfqpoint{4.400566in}{2.605579in}}%
\pgfpathlineto{\pgfqpoint{4.393144in}{2.596180in}}%
\pgfpathlineto{\pgfqpoint{4.385718in}{2.586820in}}%
\pgfpathclose%
\pgfusepath{fill}%
\end{pgfscope}%
\begin{pgfscope}%
\pgfpathrectangle{\pgfqpoint{1.254980in}{0.150000in}}{\pgfqpoint{5.490039in}{5.490039in}}%
\pgfusepath{clip}%
\pgfsetbuttcap%
\pgfsetroundjoin%
\definecolor{currentfill}{rgb}{0.185556,0.418570,0.556753}%
\pgfsetfillcolor{currentfill}%
\pgfsetfillopacity{0.700000}%
\pgfsetlinewidth{0.000000pt}%
\definecolor{currentstroke}{rgb}{0.000000,0.000000,0.000000}%
\pgfsetstrokecolor{currentstroke}%
\pgfsetdash{}{0pt}%
\pgfpathmoveto{\pgfqpoint{5.126360in}{2.980715in}}%
\pgfpathlineto{\pgfqpoint{5.139778in}{2.983733in}}%
\pgfpathlineto{\pgfqpoint{5.153209in}{2.986904in}}%
\pgfpathlineto{\pgfqpoint{5.166652in}{2.990229in}}%
\pgfpathlineto{\pgfqpoint{5.180108in}{2.993706in}}%
\pgfpathlineto{\pgfqpoint{5.187262in}{3.001893in}}%
\pgfpathlineto{\pgfqpoint{5.194413in}{3.010218in}}%
\pgfpathlineto{\pgfqpoint{5.201563in}{3.018687in}}%
\pgfpathlineto{\pgfqpoint{5.208711in}{3.027306in}}%
\pgfpathlineto{\pgfqpoint{5.195274in}{3.024364in}}%
\pgfpathlineto{\pgfqpoint{5.181851in}{3.021574in}}%
\pgfpathlineto{\pgfqpoint{5.168440in}{3.018937in}}%
\pgfpathlineto{\pgfqpoint{5.155041in}{3.016453in}}%
\pgfpathlineto{\pgfqpoint{5.147873in}{3.007289in}}%
\pgfpathlineto{\pgfqpoint{5.140704in}{2.998283in}}%
\pgfpathlineto{\pgfqpoint{5.133533in}{2.989427in}}%
\pgfpathlineto{\pgfqpoint{5.126360in}{2.980715in}}%
\pgfpathclose%
\pgfusepath{fill}%
\end{pgfscope}%
\begin{pgfscope}%
\pgfpathrectangle{\pgfqpoint{1.254980in}{0.150000in}}{\pgfqpoint{5.490039in}{5.490039in}}%
\pgfusepath{clip}%
\pgfsetbuttcap%
\pgfsetroundjoin%
\definecolor{currentfill}{rgb}{0.177423,0.437527,0.557565}%
\pgfsetfillcolor{currentfill}%
\pgfsetfillopacity{0.700000}%
\pgfsetlinewidth{0.000000pt}%
\definecolor{currentstroke}{rgb}{0.000000,0.000000,0.000000}%
\pgfsetstrokecolor{currentstroke}%
\pgfsetdash{}{0pt}%
\pgfpathmoveto{\pgfqpoint{5.208711in}{3.027306in}}%
\pgfpathlineto{\pgfqpoint{5.222160in}{3.030400in}}%
\pgfpathlineto{\pgfqpoint{5.235622in}{3.033647in}}%
\pgfpathlineto{\pgfqpoint{5.249096in}{3.037046in}}%
\pgfpathlineto{\pgfqpoint{5.262584in}{3.040597in}}%
\pgfpathlineto{\pgfqpoint{5.269710in}{3.048819in}}%
\pgfpathlineto{\pgfqpoint{5.276835in}{3.057198in}}%
\pgfpathlineto{\pgfqpoint{5.283959in}{3.065739in}}%
\pgfpathlineto{\pgfqpoint{5.291082in}{3.074449in}}%
\pgfpathlineto{\pgfqpoint{5.277616in}{3.071461in}}%
\pgfpathlineto{\pgfqpoint{5.264162in}{3.068625in}}%
\pgfpathlineto{\pgfqpoint{5.250721in}{3.065941in}}%
\pgfpathlineto{\pgfqpoint{5.237293in}{3.063408in}}%
\pgfpathlineto{\pgfqpoint{5.230148in}{3.054126in}}%
\pgfpathlineto{\pgfqpoint{5.223003in}{3.045019in}}%
\pgfpathlineto{\pgfqpoint{5.215858in}{3.036081in}}%
\pgfpathlineto{\pgfqpoint{5.208711in}{3.027306in}}%
\pgfpathclose%
\pgfusepath{fill}%
\end{pgfscope}%
\begin{pgfscope}%
\pgfpathrectangle{\pgfqpoint{1.254980in}{0.150000in}}{\pgfqpoint{5.490039in}{5.490039in}}%
\pgfusepath{clip}%
\pgfsetbuttcap%
\pgfsetroundjoin%
\definecolor{currentfill}{rgb}{0.257322,0.256130,0.526563}%
\pgfsetfillcolor{currentfill}%
\pgfsetfillopacity{0.700000}%
\pgfsetlinewidth{0.000000pt}%
\definecolor{currentstroke}{rgb}{0.000000,0.000000,0.000000}%
\pgfsetstrokecolor{currentstroke}%
\pgfsetdash{}{0pt}%
\pgfpathmoveto{\pgfqpoint{2.927730in}{2.635185in}}%
\pgfpathlineto{\pgfqpoint{2.940782in}{2.620154in}}%
\pgfpathlineto{\pgfqpoint{2.953828in}{2.605385in}}%
\pgfpathlineto{\pgfqpoint{2.966869in}{2.590877in}}%
\pgfpathlineto{\pgfqpoint{2.979905in}{2.576628in}}%
\pgfpathlineto{\pgfqpoint{2.987811in}{2.585378in}}%
\pgfpathlineto{\pgfqpoint{2.995710in}{2.594230in}}%
\pgfpathlineto{\pgfqpoint{3.003601in}{2.603184in}}%
\pgfpathlineto{\pgfqpoint{3.011484in}{2.612238in}}%
\pgfpathlineto{\pgfqpoint{2.998466in}{2.626430in}}%
\pgfpathlineto{\pgfqpoint{2.985443in}{2.640881in}}%
\pgfpathlineto{\pgfqpoint{2.972415in}{2.655591in}}%
\pgfpathlineto{\pgfqpoint{2.959381in}{2.670564in}}%
\pgfpathlineto{\pgfqpoint{2.951480in}{2.661557in}}%
\pgfpathlineto{\pgfqpoint{2.943571in}{2.652657in}}%
\pgfpathlineto{\pgfqpoint{2.935654in}{2.643867in}}%
\pgfpathlineto{\pgfqpoint{2.927730in}{2.635185in}}%
\pgfpathclose%
\pgfusepath{fill}%
\end{pgfscope}%
\begin{pgfscope}%
\pgfpathrectangle{\pgfqpoint{1.254980in}{0.150000in}}{\pgfqpoint{5.490039in}{5.490039in}}%
\pgfusepath{clip}%
\pgfsetbuttcap%
\pgfsetroundjoin%
\definecolor{currentfill}{rgb}{0.246811,0.283237,0.535941}%
\pgfsetfillcolor{currentfill}%
\pgfsetfillopacity{0.700000}%
\pgfsetlinewidth{0.000000pt}%
\definecolor{currentstroke}{rgb}{0.000000,0.000000,0.000000}%
\pgfsetstrokecolor{currentstroke}%
\pgfsetdash{}{0pt}%
\pgfpathmoveto{\pgfqpoint{2.875458in}{2.697981in}}%
\pgfpathlineto{\pgfqpoint{2.888536in}{2.681877in}}%
\pgfpathlineto{\pgfqpoint{2.901607in}{2.666044in}}%
\pgfpathlineto{\pgfqpoint{2.914671in}{2.650481in}}%
\pgfpathlineto{\pgfqpoint{2.927730in}{2.635185in}}%
\pgfpathlineto{\pgfqpoint{2.935654in}{2.643867in}}%
\pgfpathlineto{\pgfqpoint{2.943571in}{2.652657in}}%
\pgfpathlineto{\pgfqpoint{2.951480in}{2.661557in}}%
\pgfpathlineto{\pgfqpoint{2.959381in}{2.670564in}}%
\pgfpathlineto{\pgfqpoint{2.946341in}{2.685802in}}%
\pgfpathlineto{\pgfqpoint{2.933296in}{2.701306in}}%
\pgfpathlineto{\pgfqpoint{2.920244in}{2.717081in}}%
\pgfpathlineto{\pgfqpoint{2.907185in}{2.733127in}}%
\pgfpathlineto{\pgfqpoint{2.899266in}{2.724167in}}%
\pgfpathlineto{\pgfqpoint{2.891338in}{2.715323in}}%
\pgfpathlineto{\pgfqpoint{2.883402in}{2.706595in}}%
\pgfpathlineto{\pgfqpoint{2.875458in}{2.697981in}}%
\pgfpathclose%
\pgfusepath{fill}%
\end{pgfscope}%
\begin{pgfscope}%
\pgfpathrectangle{\pgfqpoint{1.254980in}{0.150000in}}{\pgfqpoint{5.490039in}{5.490039in}}%
\pgfusepath{clip}%
\pgfsetbuttcap%
\pgfsetroundjoin%
\definecolor{currentfill}{rgb}{0.266580,0.228262,0.514349}%
\pgfsetfillcolor{currentfill}%
\pgfsetfillopacity{0.700000}%
\pgfsetlinewidth{0.000000pt}%
\definecolor{currentstroke}{rgb}{0.000000,0.000000,0.000000}%
\pgfsetstrokecolor{currentstroke}%
\pgfsetdash{}{0pt}%
\pgfpathmoveto{\pgfqpoint{4.303437in}{2.547072in}}%
\pgfpathlineto{\pgfqpoint{4.316556in}{2.547477in}}%
\pgfpathlineto{\pgfqpoint{4.329683in}{2.548050in}}%
\pgfpathlineto{\pgfqpoint{4.342820in}{2.548792in}}%
\pgfpathlineto{\pgfqpoint{4.355966in}{2.549701in}}%
\pgfpathlineto{\pgfqpoint{4.363411in}{2.558938in}}%
\pgfpathlineto{\pgfqpoint{4.370851in}{2.568202in}}%
\pgfpathlineto{\pgfqpoint{4.378287in}{2.577495in}}%
\pgfpathlineto{\pgfqpoint{4.385718in}{2.586820in}}%
\pgfpathlineto{\pgfqpoint{4.372582in}{2.586164in}}%
\pgfpathlineto{\pgfqpoint{4.359455in}{2.585676in}}%
\pgfpathlineto{\pgfqpoint{4.346337in}{2.585355in}}%
\pgfpathlineto{\pgfqpoint{4.333227in}{2.585203in}}%
\pgfpathlineto{\pgfqpoint{4.325787in}{2.575615in}}%
\pgfpathlineto{\pgfqpoint{4.318342in}{2.566066in}}%
\pgfpathlineto{\pgfqpoint{4.310892in}{2.556552in}}%
\pgfpathlineto{\pgfqpoint{4.303437in}{2.547072in}}%
\pgfpathclose%
\pgfusepath{fill}%
\end{pgfscope}%
\begin{pgfscope}%
\pgfpathrectangle{\pgfqpoint{1.254980in}{0.150000in}}{\pgfqpoint{5.490039in}{5.490039in}}%
\pgfusepath{clip}%
\pgfsetbuttcap%
\pgfsetroundjoin%
\definecolor{currentfill}{rgb}{0.169646,0.456262,0.558030}%
\pgfsetfillcolor{currentfill}%
\pgfsetfillopacity{0.700000}%
\pgfsetlinewidth{0.000000pt}%
\definecolor{currentstroke}{rgb}{0.000000,0.000000,0.000000}%
\pgfsetstrokecolor{currentstroke}%
\pgfsetdash{}{0pt}%
\pgfpathmoveto{\pgfqpoint{5.291082in}{3.074449in}}%
\pgfpathlineto{\pgfqpoint{5.304562in}{3.077588in}}%
\pgfpathlineto{\pgfqpoint{5.318054in}{3.080878in}}%
\pgfpathlineto{\pgfqpoint{5.331560in}{3.084319in}}%
\pgfpathlineto{\pgfqpoint{5.345079in}{3.087911in}}%
\pgfpathlineto{\pgfqpoint{5.352180in}{3.096215in}}%
\pgfpathlineto{\pgfqpoint{5.359279in}{3.104694in}}%
\pgfpathlineto{\pgfqpoint{5.366379in}{3.113355in}}%
\pgfpathlineto{\pgfqpoint{5.373478in}{3.122204in}}%
\pgfpathlineto{\pgfqpoint{5.359982in}{3.119204in}}%
\pgfpathlineto{\pgfqpoint{5.346499in}{3.116353in}}%
\pgfpathlineto{\pgfqpoint{5.333029in}{3.113654in}}%
\pgfpathlineto{\pgfqpoint{5.319572in}{3.111105in}}%
\pgfpathlineto{\pgfqpoint{5.312449in}{3.101655in}}%
\pgfpathlineto{\pgfqpoint{5.305327in}{3.092400in}}%
\pgfpathlineto{\pgfqpoint{5.298205in}{3.083334in}}%
\pgfpathlineto{\pgfqpoint{5.291082in}{3.074449in}}%
\pgfpathclose%
\pgfusepath{fill}%
\end{pgfscope}%
\begin{pgfscope}%
\pgfpathrectangle{\pgfqpoint{1.254980in}{0.150000in}}{\pgfqpoint{5.490039in}{5.490039in}}%
\pgfusepath{clip}%
\pgfsetbuttcap%
\pgfsetroundjoin%
\definecolor{currentfill}{rgb}{0.266580,0.228262,0.514349}%
\pgfsetfillcolor{currentfill}%
\pgfsetfillopacity{0.700000}%
\pgfsetlinewidth{0.000000pt}%
\definecolor{currentstroke}{rgb}{0.000000,0.000000,0.000000}%
\pgfsetstrokecolor{currentstroke}%
\pgfsetdash{}{0pt}%
\pgfpathmoveto{\pgfqpoint{2.979905in}{2.576628in}}%
\pgfpathlineto{\pgfqpoint{2.992936in}{2.562634in}}%
\pgfpathlineto{\pgfqpoint{3.005962in}{2.548895in}}%
\pgfpathlineto{\pgfqpoint{3.018983in}{2.535407in}}%
\pgfpathlineto{\pgfqpoint{3.032000in}{2.522169in}}%
\pgfpathlineto{\pgfqpoint{3.039888in}{2.530987in}}%
\pgfpathlineto{\pgfqpoint{3.047769in}{2.539900in}}%
\pgfpathlineto{\pgfqpoint{3.055643in}{2.548906in}}%
\pgfpathlineto{\pgfqpoint{3.063509in}{2.558008in}}%
\pgfpathlineto{\pgfqpoint{3.050509in}{2.571189in}}%
\pgfpathlineto{\pgfqpoint{3.037505in}{2.584619in}}%
\pgfpathlineto{\pgfqpoint{3.024497in}{2.598302in}}%
\pgfpathlineto{\pgfqpoint{3.011484in}{2.612238in}}%
\pgfpathlineto{\pgfqpoint{3.003601in}{2.603184in}}%
\pgfpathlineto{\pgfqpoint{2.995710in}{2.594230in}}%
\pgfpathlineto{\pgfqpoint{2.987811in}{2.585378in}}%
\pgfpathlineto{\pgfqpoint{2.979905in}{2.576628in}}%
\pgfpathclose%
\pgfusepath{fill}%
\end{pgfscope}%
\begin{pgfscope}%
\pgfpathrectangle{\pgfqpoint{1.254980in}{0.150000in}}{\pgfqpoint{5.490039in}{5.490039in}}%
\pgfusepath{clip}%
\pgfsetbuttcap%
\pgfsetroundjoin%
\definecolor{currentfill}{rgb}{0.283229,0.120777,0.440584}%
\pgfsetfillcolor{currentfill}%
\pgfsetfillopacity{0.700000}%
\pgfsetlinewidth{0.000000pt}%
\definecolor{currentstroke}{rgb}{0.000000,0.000000,0.000000}%
\pgfsetstrokecolor{currentstroke}%
\pgfsetdash{}{0pt}%
\pgfpathmoveto{\pgfqpoint{3.405877in}{2.349859in}}%
\pgfpathlineto{\pgfqpoint{3.418829in}{2.342380in}}%
\pgfpathlineto{\pgfqpoint{3.431782in}{2.335109in}}%
\pgfpathlineto{\pgfqpoint{3.444737in}{2.328044in}}%
\pgfpathlineto{\pgfqpoint{3.457693in}{2.321185in}}%
\pgfpathlineto{\pgfqpoint{3.465430in}{2.330827in}}%
\pgfpathlineto{\pgfqpoint{3.473161in}{2.340516in}}%
\pgfpathlineto{\pgfqpoint{3.480886in}{2.350253in}}%
\pgfpathlineto{\pgfqpoint{3.488606in}{2.360037in}}%
\pgfpathlineto{\pgfqpoint{3.475661in}{2.366898in}}%
\pgfpathlineto{\pgfqpoint{3.462718in}{2.373964in}}%
\pgfpathlineto{\pgfqpoint{3.449777in}{2.381237in}}%
\pgfpathlineto{\pgfqpoint{3.436836in}{2.388718in}}%
\pgfpathlineto{\pgfqpoint{3.429105in}{2.378921in}}%
\pgfpathlineto{\pgfqpoint{3.421368in}{2.369180in}}%
\pgfpathlineto{\pgfqpoint{3.413625in}{2.359492in}}%
\pgfpathlineto{\pgfqpoint{3.405877in}{2.349859in}}%
\pgfpathclose%
\pgfusepath{fill}%
\end{pgfscope}%
\begin{pgfscope}%
\pgfpathrectangle{\pgfqpoint{1.254980in}{0.150000in}}{\pgfqpoint{5.490039in}{5.490039in}}%
\pgfusepath{clip}%
\pgfsetbuttcap%
\pgfsetroundjoin%
\definecolor{currentfill}{rgb}{0.233603,0.313828,0.543914}%
\pgfsetfillcolor{currentfill}%
\pgfsetfillopacity{0.700000}%
\pgfsetlinewidth{0.000000pt}%
\definecolor{currentstroke}{rgb}{0.000000,0.000000,0.000000}%
\pgfsetstrokecolor{currentstroke}%
\pgfsetdash{}{0pt}%
\pgfpathmoveto{\pgfqpoint{2.823075in}{2.765171in}}%
\pgfpathlineto{\pgfqpoint{2.836182in}{2.747953in}}%
\pgfpathlineto{\pgfqpoint{2.849282in}{2.731017in}}%
\pgfpathlineto{\pgfqpoint{2.862374in}{2.714361in}}%
\pgfpathlineto{\pgfqpoint{2.875458in}{2.697981in}}%
\pgfpathlineto{\pgfqpoint{2.883402in}{2.706595in}}%
\pgfpathlineto{\pgfqpoint{2.891338in}{2.715323in}}%
\pgfpathlineto{\pgfqpoint{2.899266in}{2.724167in}}%
\pgfpathlineto{\pgfqpoint{2.907185in}{2.733127in}}%
\pgfpathlineto{\pgfqpoint{2.894120in}{2.749447in}}%
\pgfpathlineto{\pgfqpoint{2.881047in}{2.766044in}}%
\pgfpathlineto{\pgfqpoint{2.867967in}{2.782921in}}%
\pgfpathlineto{\pgfqpoint{2.854880in}{2.800079in}}%
\pgfpathlineto{\pgfqpoint{2.846941in}{2.791168in}}%
\pgfpathlineto{\pgfqpoint{2.838994in}{2.782380in}}%
\pgfpathlineto{\pgfqpoint{2.831039in}{2.773714in}}%
\pgfpathlineto{\pgfqpoint{2.823075in}{2.765171in}}%
\pgfpathclose%
\pgfusepath{fill}%
\end{pgfscope}%
\begin{pgfscope}%
\pgfpathrectangle{\pgfqpoint{1.254980in}{0.150000in}}{\pgfqpoint{5.490039in}{5.490039in}}%
\pgfusepath{clip}%
\pgfsetbuttcap%
\pgfsetroundjoin%
\definecolor{currentfill}{rgb}{0.271828,0.209303,0.504434}%
\pgfsetfillcolor{currentfill}%
\pgfsetfillopacity{0.700000}%
\pgfsetlinewidth{0.000000pt}%
\definecolor{currentstroke}{rgb}{0.000000,0.000000,0.000000}%
\pgfsetstrokecolor{currentstroke}%
\pgfsetdash{}{0pt}%
\pgfpathmoveto{\pgfqpoint{4.221146in}{2.508593in}}%
\pgfpathlineto{\pgfqpoint{4.234240in}{2.508545in}}%
\pgfpathlineto{\pgfqpoint{4.247342in}{2.508667in}}%
\pgfpathlineto{\pgfqpoint{4.260452in}{2.508959in}}%
\pgfpathlineto{\pgfqpoint{4.273572in}{2.509420in}}%
\pgfpathlineto{\pgfqpoint{4.281045in}{2.518798in}}%
\pgfpathlineto{\pgfqpoint{4.288514in}{2.528198in}}%
\pgfpathlineto{\pgfqpoint{4.295978in}{2.537621in}}%
\pgfpathlineto{\pgfqpoint{4.303437in}{2.547072in}}%
\pgfpathlineto{\pgfqpoint{4.290327in}{2.546836in}}%
\pgfpathlineto{\pgfqpoint{4.277225in}{2.546769in}}%
\pgfpathlineto{\pgfqpoint{4.264132in}{2.546872in}}%
\pgfpathlineto{\pgfqpoint{4.251047in}{2.547145in}}%
\pgfpathlineto{\pgfqpoint{4.243579in}{2.537459in}}%
\pgfpathlineto{\pgfqpoint{4.236106in}{2.527807in}}%
\pgfpathlineto{\pgfqpoint{4.228628in}{2.518186in}}%
\pgfpathlineto{\pgfqpoint{4.221146in}{2.508593in}}%
\pgfpathclose%
\pgfusepath{fill}%
\end{pgfscope}%
\begin{pgfscope}%
\pgfpathrectangle{\pgfqpoint{1.254980in}{0.150000in}}{\pgfqpoint{5.490039in}{5.490039in}}%
\pgfusepath{clip}%
\pgfsetbuttcap%
\pgfsetroundjoin%
\definecolor{currentfill}{rgb}{0.162142,0.474838,0.558140}%
\pgfsetfillcolor{currentfill}%
\pgfsetfillopacity{0.700000}%
\pgfsetlinewidth{0.000000pt}%
\definecolor{currentstroke}{rgb}{0.000000,0.000000,0.000000}%
\pgfsetstrokecolor{currentstroke}%
\pgfsetdash{}{0pt}%
\pgfpathmoveto{\pgfqpoint{5.373478in}{3.122204in}}%
\pgfpathlineto{\pgfqpoint{5.386988in}{3.125355in}}%
\pgfpathlineto{\pgfqpoint{5.400511in}{3.128656in}}%
\pgfpathlineto{\pgfqpoint{5.414047in}{3.132107in}}%
\pgfpathlineto{\pgfqpoint{5.427597in}{3.135708in}}%
\pgfpathlineto{\pgfqpoint{5.434673in}{3.144144in}}%
\pgfpathlineto{\pgfqpoint{5.441749in}{3.152775in}}%
\pgfpathlineto{\pgfqpoint{5.448827in}{3.161608in}}%
\pgfpathlineto{\pgfqpoint{5.455905in}{3.170651in}}%
\pgfpathlineto{\pgfqpoint{5.442380in}{3.167670in}}%
\pgfpathlineto{\pgfqpoint{5.428868in}{3.164838in}}%
\pgfpathlineto{\pgfqpoint{5.415369in}{3.162155in}}%
\pgfpathlineto{\pgfqpoint{5.401883in}{3.159622in}}%
\pgfpathlineto{\pgfqpoint{5.394781in}{3.149951in}}%
\pgfpathlineto{\pgfqpoint{5.387679in}{3.140495in}}%
\pgfpathlineto{\pgfqpoint{5.380579in}{3.131249in}}%
\pgfpathlineto{\pgfqpoint{5.373478in}{3.122204in}}%
\pgfpathclose%
\pgfusepath{fill}%
\end{pgfscope}%
\begin{pgfscope}%
\pgfpathrectangle{\pgfqpoint{1.254980in}{0.150000in}}{\pgfqpoint{5.490039in}{5.490039in}}%
\pgfusepath{clip}%
\pgfsetbuttcap%
\pgfsetroundjoin%
\definecolor{currentfill}{rgb}{0.283072,0.130895,0.449241}%
\pgfsetfillcolor{currentfill}%
\pgfsetfillopacity{0.700000}%
\pgfsetlinewidth{0.000000pt}%
\definecolor{currentstroke}{rgb}{0.000000,0.000000,0.000000}%
\pgfsetstrokecolor{currentstroke}%
\pgfsetdash{}{0pt}%
\pgfpathmoveto{\pgfqpoint{3.757328in}{2.355962in}}%
\pgfpathlineto{\pgfqpoint{3.770308in}{2.352337in}}%
\pgfpathlineto{\pgfqpoint{3.783293in}{2.348900in}}%
\pgfpathlineto{\pgfqpoint{3.796283in}{2.345649in}}%
\pgfpathlineto{\pgfqpoint{3.809278in}{2.342583in}}%
\pgfpathlineto{\pgfqpoint{3.816901in}{2.352460in}}%
\pgfpathlineto{\pgfqpoint{3.824519in}{2.362359in}}%
\pgfpathlineto{\pgfqpoint{3.832132in}{2.372283in}}%
\pgfpathlineto{\pgfqpoint{3.839740in}{2.382231in}}%
\pgfpathlineto{\pgfqpoint{3.826753in}{2.385383in}}%
\pgfpathlineto{\pgfqpoint{3.813773in}{2.388719in}}%
\pgfpathlineto{\pgfqpoint{3.800797in}{2.392242in}}%
\pgfpathlineto{\pgfqpoint{3.787825in}{2.395952in}}%
\pgfpathlineto{\pgfqpoint{3.780209in}{2.385907in}}%
\pgfpathlineto{\pgfqpoint{3.772587in}{2.375895in}}%
\pgfpathlineto{\pgfqpoint{3.764960in}{2.365913in}}%
\pgfpathlineto{\pgfqpoint{3.757328in}{2.355962in}}%
\pgfpathclose%
\pgfusepath{fill}%
\end{pgfscope}%
\begin{pgfscope}%
\pgfpathrectangle{\pgfqpoint{1.254980in}{0.150000in}}{\pgfqpoint{5.490039in}{5.490039in}}%
\pgfusepath{clip}%
\pgfsetbuttcap%
\pgfsetroundjoin%
\definecolor{currentfill}{rgb}{0.283197,0.115680,0.436115}%
\pgfsetfillcolor{currentfill}%
\pgfsetfillopacity{0.700000}%
\pgfsetlinewidth{0.000000pt}%
\definecolor{currentstroke}{rgb}{0.000000,0.000000,0.000000}%
\pgfsetstrokecolor{currentstroke}%
\pgfsetdash{}{0pt}%
\pgfpathmoveto{\pgfqpoint{3.540406in}{2.334622in}}%
\pgfpathlineto{\pgfqpoint{3.553361in}{2.328770in}}%
\pgfpathlineto{\pgfqpoint{3.566320in}{2.323117in}}%
\pgfpathlineto{\pgfqpoint{3.579281in}{2.317660in}}%
\pgfpathlineto{\pgfqpoint{3.592246in}{2.312401in}}%
\pgfpathlineto{\pgfqpoint{3.599938in}{2.322195in}}%
\pgfpathlineto{\pgfqpoint{3.607626in}{2.332024in}}%
\pgfpathlineto{\pgfqpoint{3.615308in}{2.341890in}}%
\pgfpathlineto{\pgfqpoint{3.622985in}{2.351792in}}%
\pgfpathlineto{\pgfqpoint{3.610031in}{2.357082in}}%
\pgfpathlineto{\pgfqpoint{3.597080in}{2.362568in}}%
\pgfpathlineto{\pgfqpoint{3.584132in}{2.368251in}}%
\pgfpathlineto{\pgfqpoint{3.571187in}{2.374133in}}%
\pgfpathlineto{\pgfqpoint{3.563500in}{2.364190in}}%
\pgfpathlineto{\pgfqpoint{3.555807in}{2.354292in}}%
\pgfpathlineto{\pgfqpoint{3.548109in}{2.344436in}}%
\pgfpathlineto{\pgfqpoint{3.540406in}{2.334622in}}%
\pgfpathclose%
\pgfusepath{fill}%
\end{pgfscope}%
\begin{pgfscope}%
\pgfpathrectangle{\pgfqpoint{1.254980in}{0.150000in}}{\pgfqpoint{5.490039in}{5.490039in}}%
\pgfusepath{clip}%
\pgfsetbuttcap%
\pgfsetroundjoin%
\definecolor{currentfill}{rgb}{0.154815,0.493313,0.557840}%
\pgfsetfillcolor{currentfill}%
\pgfsetfillopacity{0.700000}%
\pgfsetlinewidth{0.000000pt}%
\definecolor{currentstroke}{rgb}{0.000000,0.000000,0.000000}%
\pgfsetstrokecolor{currentstroke}%
\pgfsetdash{}{0pt}%
\pgfpathmoveto{\pgfqpoint{5.455905in}{3.170651in}}%
\pgfpathlineto{\pgfqpoint{5.469444in}{3.173782in}}%
\pgfpathlineto{\pgfqpoint{5.482996in}{3.177062in}}%
\pgfpathlineto{\pgfqpoint{5.496562in}{3.180492in}}%
\pgfpathlineto{\pgfqpoint{5.510142in}{3.184070in}}%
\pgfpathlineto{\pgfqpoint{5.517196in}{3.192692in}}%
\pgfpathlineto{\pgfqpoint{5.524251in}{3.201532in}}%
\pgfpathlineto{\pgfqpoint{5.531308in}{3.210596in}}%
\pgfpathlineto{\pgfqpoint{5.538367in}{3.219892in}}%
\pgfpathlineto{\pgfqpoint{5.524814in}{3.216961in}}%
\pgfpathlineto{\pgfqpoint{5.511274in}{3.214179in}}%
\pgfpathlineto{\pgfqpoint{5.497747in}{3.211545in}}%
\pgfpathlineto{\pgfqpoint{5.484234in}{3.209060in}}%
\pgfpathlineto{\pgfqpoint{5.477149in}{3.199108in}}%
\pgfpathlineto{\pgfqpoint{5.470066in}{3.189394in}}%
\pgfpathlineto{\pgfqpoint{5.462984in}{3.179911in}}%
\pgfpathlineto{\pgfqpoint{5.455905in}{3.170651in}}%
\pgfpathclose%
\pgfusepath{fill}%
\end{pgfscope}%
\begin{pgfscope}%
\pgfpathrectangle{\pgfqpoint{1.254980in}{0.150000in}}{\pgfqpoint{5.490039in}{5.490039in}}%
\pgfusepath{clip}%
\pgfsetbuttcap%
\pgfsetroundjoin%
\definecolor{currentfill}{rgb}{0.273006,0.204520,0.501721}%
\pgfsetfillcolor{currentfill}%
\pgfsetfillopacity{0.700000}%
\pgfsetlinewidth{0.000000pt}%
\definecolor{currentstroke}{rgb}{0.000000,0.000000,0.000000}%
\pgfsetstrokecolor{currentstroke}%
\pgfsetdash{}{0pt}%
\pgfpathmoveto{\pgfqpoint{3.032000in}{2.522169in}}%
\pgfpathlineto{\pgfqpoint{3.045013in}{2.509179in}}%
\pgfpathlineto{\pgfqpoint{3.058022in}{2.496435in}}%
\pgfpathlineto{\pgfqpoint{3.071028in}{2.483935in}}%
\pgfpathlineto{\pgfqpoint{3.084030in}{2.471677in}}%
\pgfpathlineto{\pgfqpoint{3.091901in}{2.480561in}}%
\pgfpathlineto{\pgfqpoint{3.099765in}{2.489534in}}%
\pgfpathlineto{\pgfqpoint{3.107622in}{2.498593in}}%
\pgfpathlineto{\pgfqpoint{3.115472in}{2.507740in}}%
\pgfpathlineto{\pgfqpoint{3.102487in}{2.519942in}}%
\pgfpathlineto{\pgfqpoint{3.089498in}{2.532386in}}%
\pgfpathlineto{\pgfqpoint{3.076505in}{2.545074in}}%
\pgfpathlineto{\pgfqpoint{3.063509in}{2.558008in}}%
\pgfpathlineto{\pgfqpoint{3.055643in}{2.548906in}}%
\pgfpathlineto{\pgfqpoint{3.047769in}{2.539900in}}%
\pgfpathlineto{\pgfqpoint{3.039888in}{2.530987in}}%
\pgfpathlineto{\pgfqpoint{3.032000in}{2.522169in}}%
\pgfpathclose%
\pgfusepath{fill}%
\end{pgfscope}%
\begin{pgfscope}%
\pgfpathrectangle{\pgfqpoint{1.254980in}{0.150000in}}{\pgfqpoint{5.490039in}{5.490039in}}%
\pgfusepath{clip}%
\pgfsetbuttcap%
\pgfsetroundjoin%
\definecolor{currentfill}{rgb}{0.276194,0.190074,0.493001}%
\pgfsetfillcolor{currentfill}%
\pgfsetfillopacity{0.700000}%
\pgfsetlinewidth{0.000000pt}%
\definecolor{currentstroke}{rgb}{0.000000,0.000000,0.000000}%
\pgfsetstrokecolor{currentstroke}%
\pgfsetdash{}{0pt}%
\pgfpathmoveto{\pgfqpoint{4.138838in}{2.471573in}}%
\pgfpathlineto{\pgfqpoint{4.151909in}{2.471034in}}%
\pgfpathlineto{\pgfqpoint{4.164987in}{2.470668in}}%
\pgfpathlineto{\pgfqpoint{4.178074in}{2.470474in}}%
\pgfpathlineto{\pgfqpoint{4.191168in}{2.470452in}}%
\pgfpathlineto{\pgfqpoint{4.198670in}{2.479958in}}%
\pgfpathlineto{\pgfqpoint{4.206167in}{2.489481in}}%
\pgfpathlineto{\pgfqpoint{4.213659in}{2.499026in}}%
\pgfpathlineto{\pgfqpoint{4.221146in}{2.508593in}}%
\pgfpathlineto{\pgfqpoint{4.208060in}{2.508813in}}%
\pgfpathlineto{\pgfqpoint{4.194982in}{2.509204in}}%
\pgfpathlineto{\pgfqpoint{4.181913in}{2.509767in}}%
\pgfpathlineto{\pgfqpoint{4.168851in}{2.510503in}}%
\pgfpathlineto{\pgfqpoint{4.161355in}{2.500728in}}%
\pgfpathlineto{\pgfqpoint{4.153854in}{2.490983in}}%
\pgfpathlineto{\pgfqpoint{4.146348in}{2.481265in}}%
\pgfpathlineto{\pgfqpoint{4.138838in}{2.471573in}}%
\pgfpathclose%
\pgfusepath{fill}%
\end{pgfscope}%
\begin{pgfscope}%
\pgfpathrectangle{\pgfqpoint{1.254980in}{0.150000in}}{\pgfqpoint{5.490039in}{5.490039in}}%
\pgfusepath{clip}%
\pgfsetbuttcap%
\pgfsetroundjoin%
\definecolor{currentfill}{rgb}{0.220057,0.343307,0.549413}%
\pgfsetfillcolor{currentfill}%
\pgfsetfillopacity{0.700000}%
\pgfsetlinewidth{0.000000pt}%
\definecolor{currentstroke}{rgb}{0.000000,0.000000,0.000000}%
\pgfsetstrokecolor{currentstroke}%
\pgfsetdash{}{0pt}%
\pgfpathmoveto{\pgfqpoint{2.770561in}{2.836918in}}%
\pgfpathlineto{\pgfqpoint{2.783703in}{2.818545in}}%
\pgfpathlineto{\pgfqpoint{2.796835in}{2.800464in}}%
\pgfpathlineto{\pgfqpoint{2.809959in}{2.782674in}}%
\pgfpathlineto{\pgfqpoint{2.823075in}{2.765171in}}%
\pgfpathlineto{\pgfqpoint{2.831039in}{2.773714in}}%
\pgfpathlineto{\pgfqpoint{2.838994in}{2.782380in}}%
\pgfpathlineto{\pgfqpoint{2.846941in}{2.791168in}}%
\pgfpathlineto{\pgfqpoint{2.854880in}{2.800079in}}%
\pgfpathlineto{\pgfqpoint{2.841785in}{2.817522in}}%
\pgfpathlineto{\pgfqpoint{2.828681in}{2.835252in}}%
\pgfpathlineto{\pgfqpoint{2.815569in}{2.853272in}}%
\pgfpathlineto{\pgfqpoint{2.802449in}{2.871585in}}%
\pgfpathlineto{\pgfqpoint{2.794490in}{2.862724in}}%
\pgfpathlineto{\pgfqpoint{2.786523in}{2.853992in}}%
\pgfpathlineto{\pgfqpoint{2.778546in}{2.845390in}}%
\pgfpathlineto{\pgfqpoint{2.770561in}{2.836918in}}%
\pgfpathclose%
\pgfusepath{fill}%
\end{pgfscope}%
\begin{pgfscope}%
\pgfpathrectangle{\pgfqpoint{1.254980in}{0.150000in}}{\pgfqpoint{5.490039in}{5.490039in}}%
\pgfusepath{clip}%
\pgfsetbuttcap%
\pgfsetroundjoin%
\definecolor{currentfill}{rgb}{0.282884,0.135920,0.453427}%
\pgfsetfillcolor{currentfill}%
\pgfsetfillopacity{0.700000}%
\pgfsetlinewidth{0.000000pt}%
\definecolor{currentstroke}{rgb}{0.000000,0.000000,0.000000}%
\pgfsetstrokecolor{currentstroke}%
\pgfsetdash{}{0pt}%
\pgfpathmoveto{\pgfqpoint{3.271124in}{2.379536in}}%
\pgfpathlineto{\pgfqpoint{3.284086in}{2.370321in}}%
\pgfpathlineto{\pgfqpoint{3.297049in}{2.361324in}}%
\pgfpathlineto{\pgfqpoint{3.310010in}{2.352544in}}%
\pgfpathlineto{\pgfqpoint{3.322972in}{2.343981in}}%
\pgfpathlineto{\pgfqpoint{3.330757in}{2.353365in}}%
\pgfpathlineto{\pgfqpoint{3.338537in}{2.362809in}}%
\pgfpathlineto{\pgfqpoint{3.346310in}{2.372314in}}%
\pgfpathlineto{\pgfqpoint{3.354077in}{2.381878in}}%
\pgfpathlineto{\pgfqpoint{3.341128in}{2.390415in}}%
\pgfpathlineto{\pgfqpoint{3.328179in}{2.399169in}}%
\pgfpathlineto{\pgfqpoint{3.315230in}{2.408139in}}%
\pgfpathlineto{\pgfqpoint{3.302282in}{2.417328in}}%
\pgfpathlineto{\pgfqpoint{3.294501in}{2.407779in}}%
\pgfpathlineto{\pgfqpoint{3.286715in}{2.398298in}}%
\pgfpathlineto{\pgfqpoint{3.278923in}{2.388884in}}%
\pgfpathlineto{\pgfqpoint{3.271124in}{2.379536in}}%
\pgfpathclose%
\pgfusepath{fill}%
\end{pgfscope}%
\begin{pgfscope}%
\pgfpathrectangle{\pgfqpoint{1.254980in}{0.150000in}}{\pgfqpoint{5.490039in}{5.490039in}}%
\pgfusepath{clip}%
\pgfsetbuttcap%
\pgfsetroundjoin%
\definecolor{currentfill}{rgb}{0.278826,0.175490,0.483397}%
\pgfsetfillcolor{currentfill}%
\pgfsetfillopacity{0.700000}%
\pgfsetlinewidth{0.000000pt}%
\definecolor{currentstroke}{rgb}{0.000000,0.000000,0.000000}%
\pgfsetstrokecolor{currentstroke}%
\pgfsetdash{}{0pt}%
\pgfpathmoveto{\pgfqpoint{4.056507in}{2.436220in}}%
\pgfpathlineto{\pgfqpoint{4.069556in}{2.435153in}}%
\pgfpathlineto{\pgfqpoint{4.082613in}{2.434262in}}%
\pgfpathlineto{\pgfqpoint{4.095677in}{2.433545in}}%
\pgfpathlineto{\pgfqpoint{4.108749in}{2.433002in}}%
\pgfpathlineto{\pgfqpoint{4.116278in}{2.442619in}}%
\pgfpathlineto{\pgfqpoint{4.123803in}{2.452252in}}%
\pgfpathlineto{\pgfqpoint{4.131323in}{2.461902in}}%
\pgfpathlineto{\pgfqpoint{4.138838in}{2.471573in}}%
\pgfpathlineto{\pgfqpoint{4.125775in}{2.472285in}}%
\pgfpathlineto{\pgfqpoint{4.112720in}{2.473171in}}%
\pgfpathlineto{\pgfqpoint{4.099671in}{2.474231in}}%
\pgfpathlineto{\pgfqpoint{4.086630in}{2.475467in}}%
\pgfpathlineto{\pgfqpoint{4.079107in}{2.465617in}}%
\pgfpathlineto{\pgfqpoint{4.071578in}{2.455794in}}%
\pgfpathlineto{\pgfqpoint{4.064045in}{2.445996in}}%
\pgfpathlineto{\pgfqpoint{4.056507in}{2.436220in}}%
\pgfpathclose%
\pgfusepath{fill}%
\end{pgfscope}%
\begin{pgfscope}%
\pgfpathrectangle{\pgfqpoint{1.254980in}{0.150000in}}{\pgfqpoint{5.490039in}{5.490039in}}%
\pgfusepath{clip}%
\pgfsetbuttcap%
\pgfsetroundjoin%
\definecolor{currentfill}{rgb}{0.278012,0.180367,0.486697}%
\pgfsetfillcolor{currentfill}%
\pgfsetfillopacity{0.700000}%
\pgfsetlinewidth{0.000000pt}%
\definecolor{currentstroke}{rgb}{0.000000,0.000000,0.000000}%
\pgfsetstrokecolor{currentstroke}%
\pgfsetdash{}{0pt}%
\pgfpathmoveto{\pgfqpoint{3.084030in}{2.471677in}}%
\pgfpathlineto{\pgfqpoint{3.097029in}{2.459659in}}%
\pgfpathlineto{\pgfqpoint{3.110025in}{2.447879in}}%
\pgfpathlineto{\pgfqpoint{3.123018in}{2.436336in}}%
\pgfpathlineto{\pgfqpoint{3.136008in}{2.425027in}}%
\pgfpathlineto{\pgfqpoint{3.143863in}{2.433978in}}%
\pgfpathlineto{\pgfqpoint{3.151711in}{2.443010in}}%
\pgfpathlineto{\pgfqpoint{3.159552in}{2.452122in}}%
\pgfpathlineto{\pgfqpoint{3.167387in}{2.461314in}}%
\pgfpathlineto{\pgfqpoint{3.154412in}{2.472567in}}%
\pgfpathlineto{\pgfqpoint{3.141435in}{2.484054in}}%
\pgfpathlineto{\pgfqpoint{3.128455in}{2.495778in}}%
\pgfpathlineto{\pgfqpoint{3.115472in}{2.507740in}}%
\pgfpathlineto{\pgfqpoint{3.107622in}{2.498593in}}%
\pgfpathlineto{\pgfqpoint{3.099765in}{2.489534in}}%
\pgfpathlineto{\pgfqpoint{3.091901in}{2.480561in}}%
\pgfpathlineto{\pgfqpoint{3.084030in}{2.471677in}}%
\pgfpathclose%
\pgfusepath{fill}%
\end{pgfscope}%
\begin{pgfscope}%
\pgfpathrectangle{\pgfqpoint{1.254980in}{0.150000in}}{\pgfqpoint{5.490039in}{5.490039in}}%
\pgfusepath{clip}%
\pgfsetbuttcap%
\pgfsetroundjoin%
\definecolor{currentfill}{rgb}{0.147607,0.511733,0.557049}%
\pgfsetfillcolor{currentfill}%
\pgfsetfillopacity{0.700000}%
\pgfsetlinewidth{0.000000pt}%
\definecolor{currentstroke}{rgb}{0.000000,0.000000,0.000000}%
\pgfsetstrokecolor{currentstroke}%
\pgfsetdash{}{0pt}%
\pgfpathmoveto{\pgfqpoint{5.538367in}{3.219892in}}%
\pgfpathlineto{\pgfqpoint{5.551935in}{3.222971in}}%
\pgfpathlineto{\pgfqpoint{5.565516in}{3.226199in}}%
\pgfpathlineto{\pgfqpoint{5.579110in}{3.229574in}}%
\pgfpathlineto{\pgfqpoint{5.592719in}{3.233098in}}%
\pgfpathlineto{\pgfqpoint{5.599754in}{3.241968in}}%
\pgfpathlineto{\pgfqpoint{5.606790in}{3.251078in}}%
\pgfpathlineto{\pgfqpoint{5.613830in}{3.260436in}}%
\pgfpathlineto{\pgfqpoint{5.600242in}{3.257416in}}%
\pgfpathlineto{\pgfqpoint{5.586668in}{3.254544in}}%
\pgfpathlineto{\pgfqpoint{5.573108in}{3.251820in}}%
\pgfpathlineto{\pgfqpoint{5.559561in}{3.249244in}}%
\pgfpathlineto{\pgfqpoint{5.552493in}{3.239209in}}%
\pgfpathlineto{\pgfqpoint{5.545429in}{3.229427in}}%
\pgfpathlineto{\pgfqpoint{5.538367in}{3.219892in}}%
\pgfpathclose%
\pgfusepath{fill}%
\end{pgfscope}%
\begin{pgfscope}%
\pgfpathrectangle{\pgfqpoint{1.254980in}{0.150000in}}{\pgfqpoint{5.490039in}{5.490039in}}%
\pgfusepath{clip}%
\pgfsetbuttcap%
\pgfsetroundjoin%
\definecolor{currentfill}{rgb}{0.283229,0.120777,0.440584}%
\pgfsetfillcolor{currentfill}%
\pgfsetfillopacity{0.700000}%
\pgfsetlinewidth{0.000000pt}%
\definecolor{currentstroke}{rgb}{0.000000,0.000000,0.000000}%
\pgfsetstrokecolor{currentstroke}%
\pgfsetdash{}{0pt}%
\pgfpathmoveto{\pgfqpoint{3.674836in}{2.332577in}}%
\pgfpathlineto{\pgfqpoint{3.687808in}{2.328254in}}%
\pgfpathlineto{\pgfqpoint{3.700784in}{2.324122in}}%
\pgfpathlineto{\pgfqpoint{3.713765in}{2.320179in}}%
\pgfpathlineto{\pgfqpoint{3.726750in}{2.316425in}}%
\pgfpathlineto{\pgfqpoint{3.734402in}{2.326271in}}%
\pgfpathlineto{\pgfqpoint{3.742049in}{2.336141in}}%
\pgfpathlineto{\pgfqpoint{3.749691in}{2.346038in}}%
\pgfpathlineto{\pgfqpoint{3.757328in}{2.355962in}}%
\pgfpathlineto{\pgfqpoint{3.744353in}{2.359773in}}%
\pgfpathlineto{\pgfqpoint{3.731381in}{2.363774in}}%
\pgfpathlineto{\pgfqpoint{3.718415in}{2.367964in}}%
\pgfpathlineto{\pgfqpoint{3.705452in}{2.372345in}}%
\pgfpathlineto{\pgfqpoint{3.697806in}{2.362353in}}%
\pgfpathlineto{\pgfqpoint{3.690154in}{2.352395in}}%
\pgfpathlineto{\pgfqpoint{3.682498in}{2.342470in}}%
\pgfpathlineto{\pgfqpoint{3.674836in}{2.332577in}}%
\pgfpathclose%
\pgfusepath{fill}%
\end{pgfscope}%
\begin{pgfscope}%
\pgfpathrectangle{\pgfqpoint{1.254980in}{0.150000in}}{\pgfqpoint{5.490039in}{5.490039in}}%
\pgfusepath{clip}%
\pgfsetbuttcap%
\pgfsetroundjoin%
\definecolor{currentfill}{rgb}{0.204903,0.375746,0.553533}%
\pgfsetfillcolor{currentfill}%
\pgfsetfillopacity{0.700000}%
\pgfsetlinewidth{0.000000pt}%
\definecolor{currentstroke}{rgb}{0.000000,0.000000,0.000000}%
\pgfsetstrokecolor{currentstroke}%
\pgfsetdash{}{0pt}%
\pgfpathmoveto{\pgfqpoint{2.717900in}{2.913399in}}%
\pgfpathlineto{\pgfqpoint{2.731081in}{2.893825in}}%
\pgfpathlineto{\pgfqpoint{2.744251in}{2.874555in}}%
\pgfpathlineto{\pgfqpoint{2.757411in}{2.855587in}}%
\pgfpathlineto{\pgfqpoint{2.770561in}{2.836918in}}%
\pgfpathlineto{\pgfqpoint{2.778546in}{2.845390in}}%
\pgfpathlineto{\pgfqpoint{2.786523in}{2.853992in}}%
\pgfpathlineto{\pgfqpoint{2.794490in}{2.862724in}}%
\pgfpathlineto{\pgfqpoint{2.802449in}{2.871585in}}%
\pgfpathlineto{\pgfqpoint{2.789319in}{2.890193in}}%
\pgfpathlineto{\pgfqpoint{2.776180in}{2.909100in}}%
\pgfpathlineto{\pgfqpoint{2.763032in}{2.928308in}}%
\pgfpathlineto{\pgfqpoint{2.749873in}{2.947820in}}%
\pgfpathlineto{\pgfqpoint{2.741894in}{2.939010in}}%
\pgfpathlineto{\pgfqpoint{2.733905in}{2.930336in}}%
\pgfpathlineto{\pgfqpoint{2.725907in}{2.921799in}}%
\pgfpathlineto{\pgfqpoint{2.717900in}{2.913399in}}%
\pgfpathclose%
\pgfusepath{fill}%
\end{pgfscope}%
\begin{pgfscope}%
\pgfpathrectangle{\pgfqpoint{1.254980in}{0.150000in}}{\pgfqpoint{5.490039in}{5.490039in}}%
\pgfusepath{clip}%
\pgfsetbuttcap%
\pgfsetroundjoin%
\definecolor{currentfill}{rgb}{0.280868,0.160771,0.472899}%
\pgfsetfillcolor{currentfill}%
\pgfsetfillopacity{0.700000}%
\pgfsetlinewidth{0.000000pt}%
\definecolor{currentstroke}{rgb}{0.000000,0.000000,0.000000}%
\pgfsetstrokecolor{currentstroke}%
\pgfsetdash{}{0pt}%
\pgfpathmoveto{\pgfqpoint{3.974144in}{2.402765in}}%
\pgfpathlineto{\pgfqpoint{3.987174in}{2.401133in}}%
\pgfpathlineto{\pgfqpoint{4.000211in}{2.399678in}}%
\pgfpathlineto{\pgfqpoint{4.013255in}{2.398401in}}%
\pgfpathlineto{\pgfqpoint{4.026306in}{2.397300in}}%
\pgfpathlineto{\pgfqpoint{4.033864in}{2.407006in}}%
\pgfpathlineto{\pgfqpoint{4.041416in}{2.416727in}}%
\pgfpathlineto{\pgfqpoint{4.048964in}{2.426464in}}%
\pgfpathlineto{\pgfqpoint{4.056507in}{2.436220in}}%
\pgfpathlineto{\pgfqpoint{4.043465in}{2.437462in}}%
\pgfpathlineto{\pgfqpoint{4.030429in}{2.438881in}}%
\pgfpathlineto{\pgfqpoint{4.017401in}{2.440477in}}%
\pgfpathlineto{\pgfqpoint{4.004379in}{2.442250in}}%
\pgfpathlineto{\pgfqpoint{3.996827in}{2.432343in}}%
\pgfpathlineto{\pgfqpoint{3.989271in}{2.422461in}}%
\pgfpathlineto{\pgfqpoint{3.981710in}{2.412603in}}%
\pgfpathlineto{\pgfqpoint{3.974144in}{2.402765in}}%
\pgfpathclose%
\pgfusepath{fill}%
\end{pgfscope}%
\begin{pgfscope}%
\pgfpathrectangle{\pgfqpoint{1.254980in}{0.150000in}}{\pgfqpoint{5.490039in}{5.490039in}}%
\pgfusepath{clip}%
\pgfsetbuttcap%
\pgfsetroundjoin%
\definecolor{currentfill}{rgb}{0.283091,0.110553,0.431554}%
\pgfsetfillcolor{currentfill}%
\pgfsetfillopacity{0.700000}%
\pgfsetlinewidth{0.000000pt}%
\definecolor{currentstroke}{rgb}{0.000000,0.000000,0.000000}%
\pgfsetstrokecolor{currentstroke}%
\pgfsetdash{}{0pt}%
\pgfpathmoveto{\pgfqpoint{3.457693in}{2.321185in}}%
\pgfpathlineto{\pgfqpoint{3.470651in}{2.314530in}}%
\pgfpathlineto{\pgfqpoint{3.483611in}{2.308078in}}%
\pgfpathlineto{\pgfqpoint{3.496573in}{2.301828in}}%
\pgfpathlineto{\pgfqpoint{3.509537in}{2.295779in}}%
\pgfpathlineto{\pgfqpoint{3.517263in}{2.305429in}}%
\pgfpathlineto{\pgfqpoint{3.524983in}{2.315119in}}%
\pgfpathlineto{\pgfqpoint{3.532697in}{2.324850in}}%
\pgfpathlineto{\pgfqpoint{3.540406in}{2.334622in}}%
\pgfpathlineto{\pgfqpoint{3.527452in}{2.340674in}}%
\pgfpathlineto{\pgfqpoint{3.514502in}{2.346926in}}%
\pgfpathlineto{\pgfqpoint{3.501553in}{2.353380in}}%
\pgfpathlineto{\pgfqpoint{3.488606in}{2.360037in}}%
\pgfpathlineto{\pgfqpoint{3.480886in}{2.350253in}}%
\pgfpathlineto{\pgfqpoint{3.473161in}{2.340516in}}%
\pgfpathlineto{\pgfqpoint{3.465430in}{2.330827in}}%
\pgfpathlineto{\pgfqpoint{3.457693in}{2.321185in}}%
\pgfpathclose%
\pgfusepath{fill}%
\end{pgfscope}%
\begin{pgfscope}%
\pgfpathrectangle{\pgfqpoint{1.254980in}{0.150000in}}{\pgfqpoint{5.490039in}{5.490039in}}%
\pgfusepath{clip}%
\pgfsetbuttcap%
\pgfsetroundjoin%
\definecolor{currentfill}{rgb}{0.280868,0.160771,0.472899}%
\pgfsetfillcolor{currentfill}%
\pgfsetfillopacity{0.700000}%
\pgfsetlinewidth{0.000000pt}%
\definecolor{currentstroke}{rgb}{0.000000,0.000000,0.000000}%
\pgfsetstrokecolor{currentstroke}%
\pgfsetdash{}{0pt}%
\pgfpathmoveto{\pgfqpoint{3.136008in}{2.425027in}}%
\pgfpathlineto{\pgfqpoint{3.148996in}{2.413952in}}%
\pgfpathlineto{\pgfqpoint{3.161982in}{2.403108in}}%
\pgfpathlineto{\pgfqpoint{3.174967in}{2.392494in}}%
\pgfpathlineto{\pgfqpoint{3.187949in}{2.382108in}}%
\pgfpathlineto{\pgfqpoint{3.195788in}{2.391124in}}%
\pgfpathlineto{\pgfqpoint{3.203621in}{2.400214in}}%
\pgfpathlineto{\pgfqpoint{3.211447in}{2.409377in}}%
\pgfpathlineto{\pgfqpoint{3.219266in}{2.418614in}}%
\pgfpathlineto{\pgfqpoint{3.206299in}{2.428946in}}%
\pgfpathlineto{\pgfqpoint{3.193330in}{2.439505in}}%
\pgfpathlineto{\pgfqpoint{3.180359in}{2.450294in}}%
\pgfpathlineto{\pgfqpoint{3.167387in}{2.461314in}}%
\pgfpathlineto{\pgfqpoint{3.159552in}{2.452122in}}%
\pgfpathlineto{\pgfqpoint{3.151711in}{2.443010in}}%
\pgfpathlineto{\pgfqpoint{3.143863in}{2.433978in}}%
\pgfpathlineto{\pgfqpoint{3.136008in}{2.425027in}}%
\pgfpathclose%
\pgfusepath{fill}%
\end{pgfscope}%
\begin{pgfscope}%
\pgfpathrectangle{\pgfqpoint{1.254980in}{0.150000in}}{\pgfqpoint{5.490039in}{5.490039in}}%
\pgfusepath{clip}%
\pgfsetbuttcap%
\pgfsetroundjoin%
\definecolor{currentfill}{rgb}{0.283229,0.120777,0.440584}%
\pgfsetfillcolor{currentfill}%
\pgfsetfillopacity{0.700000}%
\pgfsetlinewidth{0.000000pt}%
\definecolor{currentstroke}{rgb}{0.000000,0.000000,0.000000}%
\pgfsetstrokecolor{currentstroke}%
\pgfsetdash{}{0pt}%
\pgfpathmoveto{\pgfqpoint{3.322972in}{2.343981in}}%
\pgfpathlineto{\pgfqpoint{3.335934in}{2.335631in}}%
\pgfpathlineto{\pgfqpoint{3.348897in}{2.327495in}}%
\pgfpathlineto{\pgfqpoint{3.361860in}{2.319571in}}%
\pgfpathlineto{\pgfqpoint{3.374823in}{2.311857in}}%
\pgfpathlineto{\pgfqpoint{3.382596in}{2.321278in}}%
\pgfpathlineto{\pgfqpoint{3.390362in}{2.330752in}}%
\pgfpathlineto{\pgfqpoint{3.398122in}{2.340279in}}%
\pgfpathlineto{\pgfqpoint{3.405877in}{2.349859in}}%
\pgfpathlineto{\pgfqpoint{3.392925in}{2.357547in}}%
\pgfpathlineto{\pgfqpoint{3.379975in}{2.365445in}}%
\pgfpathlineto{\pgfqpoint{3.367026in}{2.373555in}}%
\pgfpathlineto{\pgfqpoint{3.354077in}{2.381878in}}%
\pgfpathlineto{\pgfqpoint{3.346310in}{2.372314in}}%
\pgfpathlineto{\pgfqpoint{3.338537in}{2.362809in}}%
\pgfpathlineto{\pgfqpoint{3.330757in}{2.353365in}}%
\pgfpathlineto{\pgfqpoint{3.322972in}{2.343981in}}%
\pgfpathclose%
\pgfusepath{fill}%
\end{pgfscope}%
\begin{pgfscope}%
\pgfpathrectangle{\pgfqpoint{1.254980in}{0.150000in}}{\pgfqpoint{5.490039in}{5.490039in}}%
\pgfusepath{clip}%
\pgfsetbuttcap%
\pgfsetroundjoin%
\definecolor{currentfill}{rgb}{0.282290,0.145912,0.461510}%
\pgfsetfillcolor{currentfill}%
\pgfsetfillopacity{0.700000}%
\pgfsetlinewidth{0.000000pt}%
\definecolor{currentstroke}{rgb}{0.000000,0.000000,0.000000}%
\pgfsetstrokecolor{currentstroke}%
\pgfsetdash{}{0pt}%
\pgfpathmoveto{\pgfqpoint{3.891738in}{2.371462in}}%
\pgfpathlineto{\pgfqpoint{3.904752in}{2.369225in}}%
\pgfpathlineto{\pgfqpoint{3.917772in}{2.367169in}}%
\pgfpathlineto{\pgfqpoint{3.930798in}{2.365292in}}%
\pgfpathlineto{\pgfqpoint{3.943831in}{2.363595in}}%
\pgfpathlineto{\pgfqpoint{3.951416in}{2.373364in}}%
\pgfpathlineto{\pgfqpoint{3.958997in}{2.383148in}}%
\pgfpathlineto{\pgfqpoint{3.966573in}{2.392948in}}%
\pgfpathlineto{\pgfqpoint{3.974144in}{2.402765in}}%
\pgfpathlineto{\pgfqpoint{3.961120in}{2.404576in}}%
\pgfpathlineto{\pgfqpoint{3.948102in}{2.406566in}}%
\pgfpathlineto{\pgfqpoint{3.935091in}{2.408736in}}%
\pgfpathlineto{\pgfqpoint{3.922085in}{2.411087in}}%
\pgfpathlineto{\pgfqpoint{3.914506in}{2.401145in}}%
\pgfpathlineto{\pgfqpoint{3.906922in}{2.391228in}}%
\pgfpathlineto{\pgfqpoint{3.899332in}{2.381334in}}%
\pgfpathlineto{\pgfqpoint{3.891738in}{2.371462in}}%
\pgfpathclose%
\pgfusepath{fill}%
\end{pgfscope}%
\begin{pgfscope}%
\pgfpathrectangle{\pgfqpoint{1.254980in}{0.150000in}}{\pgfqpoint{5.490039in}{5.490039in}}%
\pgfusepath{clip}%
\pgfsetbuttcap%
\pgfsetroundjoin%
\definecolor{currentfill}{rgb}{0.283091,0.110553,0.431554}%
\pgfsetfillcolor{currentfill}%
\pgfsetfillopacity{0.700000}%
\pgfsetlinewidth{0.000000pt}%
\definecolor{currentstroke}{rgb}{0.000000,0.000000,0.000000}%
\pgfsetstrokecolor{currentstroke}%
\pgfsetdash{}{0pt}%
\pgfpathmoveto{\pgfqpoint{3.592246in}{2.312401in}}%
\pgfpathlineto{\pgfqpoint{3.605213in}{2.307337in}}%
\pgfpathlineto{\pgfqpoint{3.618184in}{2.302467in}}%
\pgfpathlineto{\pgfqpoint{3.631158in}{2.297790in}}%
\pgfpathlineto{\pgfqpoint{3.644137in}{2.293307in}}%
\pgfpathlineto{\pgfqpoint{3.651819in}{2.303081in}}%
\pgfpathlineto{\pgfqpoint{3.659497in}{2.312883in}}%
\pgfpathlineto{\pgfqpoint{3.667169in}{2.322715in}}%
\pgfpathlineto{\pgfqpoint{3.674836in}{2.332577in}}%
\pgfpathlineto{\pgfqpoint{3.661868in}{2.337091in}}%
\pgfpathlineto{\pgfqpoint{3.648903in}{2.341798in}}%
\pgfpathlineto{\pgfqpoint{3.635942in}{2.346698in}}%
\pgfpathlineto{\pgfqpoint{3.622985in}{2.351792in}}%
\pgfpathlineto{\pgfqpoint{3.615308in}{2.341890in}}%
\pgfpathlineto{\pgfqpoint{3.607626in}{2.332024in}}%
\pgfpathlineto{\pgfqpoint{3.599938in}{2.322195in}}%
\pgfpathlineto{\pgfqpoint{3.592246in}{2.312401in}}%
\pgfpathclose%
\pgfusepath{fill}%
\end{pgfscope}%
\begin{pgfscope}%
\pgfpathrectangle{\pgfqpoint{1.254980in}{0.150000in}}{\pgfqpoint{5.490039in}{5.490039in}}%
\pgfusepath{clip}%
\pgfsetbuttcap%
\pgfsetroundjoin%
\definecolor{currentfill}{rgb}{0.283072,0.130895,0.449241}%
\pgfsetfillcolor{currentfill}%
\pgfsetfillopacity{0.700000}%
\pgfsetlinewidth{0.000000pt}%
\definecolor{currentstroke}{rgb}{0.000000,0.000000,0.000000}%
\pgfsetstrokecolor{currentstroke}%
\pgfsetdash{}{0pt}%
\pgfpathmoveto{\pgfqpoint{3.809278in}{2.342583in}}%
\pgfpathlineto{\pgfqpoint{3.822278in}{2.339702in}}%
\pgfpathlineto{\pgfqpoint{3.835284in}{2.337005in}}%
\pgfpathlineto{\pgfqpoint{3.848295in}{2.334490in}}%
\pgfpathlineto{\pgfqpoint{3.861312in}{2.332158in}}%
\pgfpathlineto{\pgfqpoint{3.868926in}{2.341959in}}%
\pgfpathlineto{\pgfqpoint{3.876535in}{2.351776in}}%
\pgfpathlineto{\pgfqpoint{3.884139in}{2.361610in}}%
\pgfpathlineto{\pgfqpoint{3.891738in}{2.371462in}}%
\pgfpathlineto{\pgfqpoint{3.878730in}{2.373880in}}%
\pgfpathlineto{\pgfqpoint{3.865728in}{2.376481in}}%
\pgfpathlineto{\pgfqpoint{3.852731in}{2.379264in}}%
\pgfpathlineto{\pgfqpoint{3.839740in}{2.382231in}}%
\pgfpathlineto{\pgfqpoint{3.832132in}{2.372283in}}%
\pgfpathlineto{\pgfqpoint{3.824519in}{2.362359in}}%
\pgfpathlineto{\pgfqpoint{3.816901in}{2.352460in}}%
\pgfpathlineto{\pgfqpoint{3.809278in}{2.342583in}}%
\pgfpathclose%
\pgfusepath{fill}%
\end{pgfscope}%
\begin{pgfscope}%
\pgfpathrectangle{\pgfqpoint{1.254980in}{0.150000in}}{\pgfqpoint{5.490039in}{5.490039in}}%
\pgfusepath{clip}%
\pgfsetbuttcap%
\pgfsetroundjoin%
\definecolor{currentfill}{rgb}{0.282290,0.145912,0.461510}%
\pgfsetfillcolor{currentfill}%
\pgfsetfillopacity{0.700000}%
\pgfsetlinewidth{0.000000pt}%
\definecolor{currentstroke}{rgb}{0.000000,0.000000,0.000000}%
\pgfsetstrokecolor{currentstroke}%
\pgfsetdash{}{0pt}%
\pgfpathmoveto{\pgfqpoint{3.187949in}{2.382108in}}%
\pgfpathlineto{\pgfqpoint{3.200930in}{2.371948in}}%
\pgfpathlineto{\pgfqpoint{3.213910in}{2.362013in}}%
\pgfpathlineto{\pgfqpoint{3.226888in}{2.352301in}}%
\pgfpathlineto{\pgfqpoint{3.239865in}{2.342811in}}%
\pgfpathlineto{\pgfqpoint{3.247690in}{2.351892in}}%
\pgfpathlineto{\pgfqpoint{3.255508in}{2.361040in}}%
\pgfpathlineto{\pgfqpoint{3.263319in}{2.370255in}}%
\pgfpathlineto{\pgfqpoint{3.271124in}{2.379536in}}%
\pgfpathlineto{\pgfqpoint{3.258161in}{2.388972in}}%
\pgfpathlineto{\pgfqpoint{3.245197in}{2.398629in}}%
\pgfpathlineto{\pgfqpoint{3.232232in}{2.408509in}}%
\pgfpathlineto{\pgfqpoint{3.219266in}{2.418614in}}%
\pgfpathlineto{\pgfqpoint{3.211447in}{2.409377in}}%
\pgfpathlineto{\pgfqpoint{3.203621in}{2.400214in}}%
\pgfpathlineto{\pgfqpoint{3.195788in}{2.391124in}}%
\pgfpathlineto{\pgfqpoint{3.187949in}{2.382108in}}%
\pgfpathclose%
\pgfusepath{fill}%
\end{pgfscope}%
\begin{pgfscope}%
\pgfpathrectangle{\pgfqpoint{1.254980in}{0.150000in}}{\pgfqpoint{5.490039in}{5.490039in}}%
\pgfusepath{clip}%
\pgfsetbuttcap%
\pgfsetroundjoin%
\definecolor{currentfill}{rgb}{0.239346,0.300855,0.540844}%
\pgfsetfillcolor{currentfill}%
\pgfsetfillopacity{0.700000}%
\pgfsetlinewidth{0.000000pt}%
\definecolor{currentstroke}{rgb}{0.000000,0.000000,0.000000}%
\pgfsetstrokecolor{currentstroke}%
\pgfsetdash{}{0pt}%
\pgfpathmoveto{\pgfqpoint{4.603136in}{2.676647in}}%
\pgfpathlineto{\pgfqpoint{4.616379in}{2.678847in}}%
\pgfpathlineto{\pgfqpoint{4.629633in}{2.681208in}}%
\pgfpathlineto{\pgfqpoint{4.642898in}{2.683731in}}%
\pgfpathlineto{\pgfqpoint{4.656173in}{2.686415in}}%
\pgfpathlineto{\pgfqpoint{4.663520in}{2.694875in}}%
\pgfpathlineto{\pgfqpoint{4.670862in}{2.703375in}}%
\pgfpathlineto{\pgfqpoint{4.678199in}{2.711919in}}%
\pgfpathlineto{\pgfqpoint{4.685532in}{2.720512in}}%
\pgfpathlineto{\pgfqpoint{4.672269in}{2.718167in}}%
\pgfpathlineto{\pgfqpoint{4.659016in}{2.715983in}}%
\pgfpathlineto{\pgfqpoint{4.645774in}{2.713960in}}%
\pgfpathlineto{\pgfqpoint{4.632543in}{2.712099in}}%
\pgfpathlineto{\pgfqpoint{4.625197in}{2.703157in}}%
\pgfpathlineto{\pgfqpoint{4.617848in}{2.694271in}}%
\pgfpathlineto{\pgfqpoint{4.610494in}{2.685436in}}%
\pgfpathlineto{\pgfqpoint{4.603136in}{2.676647in}}%
\pgfpathclose%
\pgfusepath{fill}%
\end{pgfscope}%
\begin{pgfscope}%
\pgfpathrectangle{\pgfqpoint{1.254980in}{0.150000in}}{\pgfqpoint{5.490039in}{5.490039in}}%
\pgfusepath{clip}%
\pgfsetbuttcap%
\pgfsetroundjoin%
\definecolor{currentfill}{rgb}{0.231674,0.318106,0.544834}%
\pgfsetfillcolor{currentfill}%
\pgfsetfillopacity{0.700000}%
\pgfsetlinewidth{0.000000pt}%
\definecolor{currentstroke}{rgb}{0.000000,0.000000,0.000000}%
\pgfsetstrokecolor{currentstroke}%
\pgfsetdash{}{0pt}%
\pgfpathmoveto{\pgfqpoint{4.685532in}{2.720512in}}%
\pgfpathlineto{\pgfqpoint{4.698807in}{2.723017in}}%
\pgfpathlineto{\pgfqpoint{4.712093in}{2.725682in}}%
\pgfpathlineto{\pgfqpoint{4.725390in}{2.728507in}}%
\pgfpathlineto{\pgfqpoint{4.738698in}{2.731492in}}%
\pgfpathlineto{\pgfqpoint{4.746014in}{2.739780in}}%
\pgfpathlineto{\pgfqpoint{4.753326in}{2.748118in}}%
\pgfpathlineto{\pgfqpoint{4.760633in}{2.756510in}}%
\pgfpathlineto{\pgfqpoint{4.767936in}{2.764960in}}%
\pgfpathlineto{\pgfqpoint{4.754641in}{2.762343in}}%
\pgfpathlineto{\pgfqpoint{4.741357in}{2.759885in}}%
\pgfpathlineto{\pgfqpoint{4.728084in}{2.757586in}}%
\pgfpathlineto{\pgfqpoint{4.714823in}{2.755448in}}%
\pgfpathlineto{\pgfqpoint{4.707506in}{2.746621in}}%
\pgfpathlineto{\pgfqpoint{4.700186in}{2.737859in}}%
\pgfpathlineto{\pgfqpoint{4.692861in}{2.729157in}}%
\pgfpathlineto{\pgfqpoint{4.685532in}{2.720512in}}%
\pgfpathclose%
\pgfusepath{fill}%
\end{pgfscope}%
\begin{pgfscope}%
\pgfpathrectangle{\pgfqpoint{1.254980in}{0.150000in}}{\pgfqpoint{5.490039in}{5.490039in}}%
\pgfusepath{clip}%
\pgfsetbuttcap%
\pgfsetroundjoin%
\definecolor{currentfill}{rgb}{0.221989,0.339161,0.548752}%
\pgfsetfillcolor{currentfill}%
\pgfsetfillopacity{0.700000}%
\pgfsetlinewidth{0.000000pt}%
\definecolor{currentstroke}{rgb}{0.000000,0.000000,0.000000}%
\pgfsetstrokecolor{currentstroke}%
\pgfsetdash{}{0pt}%
\pgfpathmoveto{\pgfqpoint{4.767936in}{2.764960in}}%
\pgfpathlineto{\pgfqpoint{4.781243in}{2.767736in}}%
\pgfpathlineto{\pgfqpoint{4.794561in}{2.770671in}}%
\pgfpathlineto{\pgfqpoint{4.807890in}{2.773764in}}%
\pgfpathlineto{\pgfqpoint{4.821232in}{2.777016in}}%
\pgfpathlineto{\pgfqpoint{4.828517in}{2.785143in}}%
\pgfpathlineto{\pgfqpoint{4.835798in}{2.793331in}}%
\pgfpathlineto{\pgfqpoint{4.843075in}{2.801584in}}%
\pgfpathlineto{\pgfqpoint{4.850348in}{2.809906in}}%
\pgfpathlineto{\pgfqpoint{4.837021in}{2.807050in}}%
\pgfpathlineto{\pgfqpoint{4.823706in}{2.804353in}}%
\pgfpathlineto{\pgfqpoint{4.810402in}{2.801813in}}%
\pgfpathlineto{\pgfqpoint{4.797109in}{2.799431in}}%
\pgfpathlineto{\pgfqpoint{4.789822in}{2.790704in}}%
\pgfpathlineto{\pgfqpoint{4.782530in}{2.782052in}}%
\pgfpathlineto{\pgfqpoint{4.775235in}{2.773473in}}%
\pgfpathlineto{\pgfqpoint{4.767936in}{2.764960in}}%
\pgfpathclose%
\pgfusepath{fill}%
\end{pgfscope}%
\begin{pgfscope}%
\pgfpathrectangle{\pgfqpoint{1.254980in}{0.150000in}}{\pgfqpoint{5.490039in}{5.490039in}}%
\pgfusepath{clip}%
\pgfsetbuttcap%
\pgfsetroundjoin%
\definecolor{currentfill}{rgb}{0.248629,0.278775,0.534556}%
\pgfsetfillcolor{currentfill}%
\pgfsetfillopacity{0.700000}%
\pgfsetlinewidth{0.000000pt}%
\definecolor{currentstroke}{rgb}{0.000000,0.000000,0.000000}%
\pgfsetstrokecolor{currentstroke}%
\pgfsetdash{}{0pt}%
\pgfpathmoveto{\pgfqpoint{4.520744in}{2.633470in}}%
\pgfpathlineto{\pgfqpoint{4.533957in}{2.635330in}}%
\pgfpathlineto{\pgfqpoint{4.547180in}{2.637354in}}%
\pgfpathlineto{\pgfqpoint{4.560413in}{2.639540in}}%
\pgfpathlineto{\pgfqpoint{4.573657in}{2.641889in}}%
\pgfpathlineto{\pgfqpoint{4.581034in}{2.650526in}}%
\pgfpathlineto{\pgfqpoint{4.588406in}{2.659196in}}%
\pgfpathlineto{\pgfqpoint{4.595773in}{2.667902in}}%
\pgfpathlineto{\pgfqpoint{4.603136in}{2.676647in}}%
\pgfpathlineto{\pgfqpoint{4.589903in}{2.674609in}}%
\pgfpathlineto{\pgfqpoint{4.576681in}{2.672734in}}%
\pgfpathlineto{\pgfqpoint{4.563469in}{2.671020in}}%
\pgfpathlineto{\pgfqpoint{4.550267in}{2.669470in}}%
\pgfpathlineto{\pgfqpoint{4.542893in}{2.660404in}}%
\pgfpathlineto{\pgfqpoint{4.535515in}{2.651385in}}%
\pgfpathlineto{\pgfqpoint{4.528132in}{2.642408in}}%
\pgfpathlineto{\pgfqpoint{4.520744in}{2.633470in}}%
\pgfpathclose%
\pgfusepath{fill}%
\end{pgfscope}%
\begin{pgfscope}%
\pgfpathrectangle{\pgfqpoint{1.254980in}{0.150000in}}{\pgfqpoint{5.490039in}{5.490039in}}%
\pgfusepath{clip}%
\pgfsetbuttcap%
\pgfsetroundjoin%
\definecolor{currentfill}{rgb}{0.214298,0.355619,0.551184}%
\pgfsetfillcolor{currentfill}%
\pgfsetfillopacity{0.700000}%
\pgfsetlinewidth{0.000000pt}%
\definecolor{currentstroke}{rgb}{0.000000,0.000000,0.000000}%
\pgfsetstrokecolor{currentstroke}%
\pgfsetdash{}{0pt}%
\pgfpathmoveto{\pgfqpoint{4.850348in}{2.809906in}}%
\pgfpathlineto{\pgfqpoint{4.863687in}{2.812920in}}%
\pgfpathlineto{\pgfqpoint{4.877038in}{2.816090in}}%
\pgfpathlineto{\pgfqpoint{4.890401in}{2.819419in}}%
\pgfpathlineto{\pgfqpoint{4.903776in}{2.822904in}}%
\pgfpathlineto{\pgfqpoint{4.911030in}{2.830886in}}%
\pgfpathlineto{\pgfqpoint{4.918280in}{2.838941in}}%
\pgfpathlineto{\pgfqpoint{4.925527in}{2.847073in}}%
\pgfpathlineto{\pgfqpoint{4.932770in}{2.855287in}}%
\pgfpathlineto{\pgfqpoint{4.919411in}{2.852226in}}%
\pgfpathlineto{\pgfqpoint{4.906063in}{2.849322in}}%
\pgfpathlineto{\pgfqpoint{4.892728in}{2.846575in}}%
\pgfpathlineto{\pgfqpoint{4.879404in}{2.843984in}}%
\pgfpathlineto{\pgfqpoint{4.872145in}{2.835337in}}%
\pgfpathlineto{\pgfqpoint{4.864883in}{2.826778in}}%
\pgfpathlineto{\pgfqpoint{4.857618in}{2.818303in}}%
\pgfpathlineto{\pgfqpoint{4.850348in}{2.809906in}}%
\pgfpathclose%
\pgfusepath{fill}%
\end{pgfscope}%
\begin{pgfscope}%
\pgfpathrectangle{\pgfqpoint{1.254980in}{0.150000in}}{\pgfqpoint{5.490039in}{5.490039in}}%
\pgfusepath{clip}%
\pgfsetbuttcap%
\pgfsetroundjoin%
\definecolor{currentfill}{rgb}{0.255645,0.260703,0.528312}%
\pgfsetfillcolor{currentfill}%
\pgfsetfillopacity{0.700000}%
\pgfsetlinewidth{0.000000pt}%
\definecolor{currentstroke}{rgb}{0.000000,0.000000,0.000000}%
\pgfsetstrokecolor{currentstroke}%
\pgfsetdash{}{0pt}%
\pgfpathmoveto{\pgfqpoint{4.438355in}{2.591106in}}%
\pgfpathlineto{\pgfqpoint{4.451538in}{2.592592in}}%
\pgfpathlineto{\pgfqpoint{4.464731in}{2.594242in}}%
\pgfpathlineto{\pgfqpoint{4.477934in}{2.596057in}}%
\pgfpathlineto{\pgfqpoint{4.491147in}{2.598037in}}%
\pgfpathlineto{\pgfqpoint{4.498553in}{2.606854in}}%
\pgfpathlineto{\pgfqpoint{4.505955in}{2.615696in}}%
\pgfpathlineto{\pgfqpoint{4.513352in}{2.624567in}}%
\pgfpathlineto{\pgfqpoint{4.520744in}{2.633470in}}%
\pgfpathlineto{\pgfqpoint{4.507542in}{2.631773in}}%
\pgfpathlineto{\pgfqpoint{4.494349in}{2.630241in}}%
\pgfpathlineto{\pgfqpoint{4.481166in}{2.628872in}}%
\pgfpathlineto{\pgfqpoint{4.467993in}{2.627668in}}%
\pgfpathlineto{\pgfqpoint{4.460591in}{2.618473in}}%
\pgfpathlineto{\pgfqpoint{4.453183in}{2.609317in}}%
\pgfpathlineto{\pgfqpoint{4.445772in}{2.600195in}}%
\pgfpathlineto{\pgfqpoint{4.438355in}{2.591106in}}%
\pgfpathclose%
\pgfusepath{fill}%
\end{pgfscope}%
\begin{pgfscope}%
\pgfpathrectangle{\pgfqpoint{1.254980in}{0.150000in}}{\pgfqpoint{5.490039in}{5.490039in}}%
\pgfusepath{clip}%
\pgfsetbuttcap%
\pgfsetroundjoin%
\definecolor{currentfill}{rgb}{0.204903,0.375746,0.553533}%
\pgfsetfillcolor{currentfill}%
\pgfsetfillopacity{0.700000}%
\pgfsetlinewidth{0.000000pt}%
\definecolor{currentstroke}{rgb}{0.000000,0.000000,0.000000}%
\pgfsetstrokecolor{currentstroke}%
\pgfsetdash{}{0pt}%
\pgfpathmoveto{\pgfqpoint{4.932770in}{2.855287in}}%
\pgfpathlineto{\pgfqpoint{4.946142in}{2.858504in}}%
\pgfpathlineto{\pgfqpoint{4.959526in}{2.861878in}}%
\pgfpathlineto{\pgfqpoint{4.972922in}{2.865407in}}%
\pgfpathlineto{\pgfqpoint{4.986330in}{2.869092in}}%
\pgfpathlineto{\pgfqpoint{4.993554in}{2.876951in}}%
\pgfpathlineto{\pgfqpoint{5.000773in}{2.884895in}}%
\pgfpathlineto{\pgfqpoint{5.007990in}{2.892930in}}%
\pgfpathlineto{\pgfqpoint{5.015203in}{2.901059in}}%
\pgfpathlineto{\pgfqpoint{5.001812in}{2.897827in}}%
\pgfpathlineto{\pgfqpoint{4.988432in}{2.894749in}}%
\pgfpathlineto{\pgfqpoint{4.975065in}{2.891828in}}%
\pgfpathlineto{\pgfqpoint{4.961709in}{2.889062in}}%
\pgfpathlineto{\pgfqpoint{4.954479in}{2.880470in}}%
\pgfpathlineto{\pgfqpoint{4.947246in}{2.871981in}}%
\pgfpathlineto{\pgfqpoint{4.940010in}{2.863588in}}%
\pgfpathlineto{\pgfqpoint{4.932770in}{2.855287in}}%
\pgfpathclose%
\pgfusepath{fill}%
\end{pgfscope}%
\begin{pgfscope}%
\pgfpathrectangle{\pgfqpoint{1.254980in}{0.150000in}}{\pgfqpoint{5.490039in}{5.490039in}}%
\pgfusepath{clip}%
\pgfsetbuttcap%
\pgfsetroundjoin%
\definecolor{currentfill}{rgb}{0.263663,0.237631,0.518762}%
\pgfsetfillcolor{currentfill}%
\pgfsetfillopacity{0.700000}%
\pgfsetlinewidth{0.000000pt}%
\definecolor{currentstroke}{rgb}{0.000000,0.000000,0.000000}%
\pgfsetstrokecolor{currentstroke}%
\pgfsetdash{}{0pt}%
\pgfpathmoveto{\pgfqpoint{4.355966in}{2.549701in}}%
\pgfpathlineto{\pgfqpoint{4.369120in}{2.550777in}}%
\pgfpathlineto{\pgfqpoint{4.382285in}{2.552019in}}%
\pgfpathlineto{\pgfqpoint{4.395458in}{2.553428in}}%
\pgfpathlineto{\pgfqpoint{4.408641in}{2.555003in}}%
\pgfpathlineto{\pgfqpoint{4.416077in}{2.563997in}}%
\pgfpathlineto{\pgfqpoint{4.423508in}{2.573010in}}%
\pgfpathlineto{\pgfqpoint{4.430934in}{2.582045in}}%
\pgfpathlineto{\pgfqpoint{4.438355in}{2.591106in}}%
\pgfpathlineto{\pgfqpoint{4.425182in}{2.589785in}}%
\pgfpathlineto{\pgfqpoint{4.412018in}{2.588631in}}%
\pgfpathlineto{\pgfqpoint{4.398863in}{2.587642in}}%
\pgfpathlineto{\pgfqpoint{4.385718in}{2.586820in}}%
\pgfpathlineto{\pgfqpoint{4.378287in}{2.577495in}}%
\pgfpathlineto{\pgfqpoint{4.370851in}{2.568202in}}%
\pgfpathlineto{\pgfqpoint{4.363411in}{2.558938in}}%
\pgfpathlineto{\pgfqpoint{4.355966in}{2.549701in}}%
\pgfpathclose%
\pgfusepath{fill}%
\end{pgfscope}%
\begin{pgfscope}%
\pgfpathrectangle{\pgfqpoint{1.254980in}{0.150000in}}{\pgfqpoint{5.490039in}{5.490039in}}%
\pgfusepath{clip}%
\pgfsetbuttcap%
\pgfsetroundjoin%
\definecolor{currentfill}{rgb}{0.195860,0.395433,0.555276}%
\pgfsetfillcolor{currentfill}%
\pgfsetfillopacity{0.700000}%
\pgfsetlinewidth{0.000000pt}%
\definecolor{currentstroke}{rgb}{0.000000,0.000000,0.000000}%
\pgfsetstrokecolor{currentstroke}%
\pgfsetdash{}{0pt}%
\pgfpathmoveto{\pgfqpoint{5.015203in}{2.901059in}}%
\pgfpathlineto{\pgfqpoint{5.028608in}{2.904447in}}%
\pgfpathlineto{\pgfqpoint{5.042024in}{2.907990in}}%
\pgfpathlineto{\pgfqpoint{5.055454in}{2.911688in}}%
\pgfpathlineto{\pgfqpoint{5.068896in}{2.915540in}}%
\pgfpathlineto{\pgfqpoint{5.076089in}{2.923300in}}%
\pgfpathlineto{\pgfqpoint{5.083278in}{2.931160in}}%
\pgfpathlineto{\pgfqpoint{5.090465in}{2.939125in}}%
\pgfpathlineto{\pgfqpoint{5.097649in}{2.947200in}}%
\pgfpathlineto{\pgfqpoint{5.084225in}{2.943828in}}%
\pgfpathlineto{\pgfqpoint{5.070813in}{2.940611in}}%
\pgfpathlineto{\pgfqpoint{5.057414in}{2.937548in}}%
\pgfpathlineto{\pgfqpoint{5.044027in}{2.934640in}}%
\pgfpathlineto{\pgfqpoint{5.036825in}{2.926075in}}%
\pgfpathlineto{\pgfqpoint{5.029621in}{2.917627in}}%
\pgfpathlineto{\pgfqpoint{5.022414in}{2.909290in}}%
\pgfpathlineto{\pgfqpoint{5.015203in}{2.901059in}}%
\pgfpathclose%
\pgfusepath{fill}%
\end{pgfscope}%
\begin{pgfscope}%
\pgfpathrectangle{\pgfqpoint{1.254980in}{0.150000in}}{\pgfqpoint{5.490039in}{5.490039in}}%
\pgfusepath{clip}%
\pgfsetbuttcap%
\pgfsetroundjoin%
\definecolor{currentfill}{rgb}{0.187231,0.414746,0.556547}%
\pgfsetfillcolor{currentfill}%
\pgfsetfillopacity{0.700000}%
\pgfsetlinewidth{0.000000pt}%
\definecolor{currentstroke}{rgb}{0.000000,0.000000,0.000000}%
\pgfsetstrokecolor{currentstroke}%
\pgfsetdash{}{0pt}%
\pgfpathmoveto{\pgfqpoint{5.097649in}{2.947200in}}%
\pgfpathlineto{\pgfqpoint{5.111086in}{2.950725in}}%
\pgfpathlineto{\pgfqpoint{5.124535in}{2.954404in}}%
\pgfpathlineto{\pgfqpoint{5.137998in}{2.958237in}}%
\pgfpathlineto{\pgfqpoint{5.151474in}{2.962224in}}%
\pgfpathlineto{\pgfqpoint{5.158636in}{2.969916in}}%
\pgfpathlineto{\pgfqpoint{5.165796in}{2.977724in}}%
\pgfpathlineto{\pgfqpoint{5.172953in}{2.985652in}}%
\pgfpathlineto{\pgfqpoint{5.180108in}{2.993706in}}%
\pgfpathlineto{\pgfqpoint{5.166652in}{2.990229in}}%
\pgfpathlineto{\pgfqpoint{5.153209in}{2.986904in}}%
\pgfpathlineto{\pgfqpoint{5.139778in}{2.983733in}}%
\pgfpathlineto{\pgfqpoint{5.126360in}{2.980715in}}%
\pgfpathlineto{\pgfqpoint{5.119186in}{2.972143in}}%
\pgfpathlineto{\pgfqpoint{5.112009in}{2.963703in}}%
\pgfpathlineto{\pgfqpoint{5.104830in}{2.955391in}}%
\pgfpathlineto{\pgfqpoint{5.097649in}{2.947200in}}%
\pgfpathclose%
\pgfusepath{fill}%
\end{pgfscope}%
\begin{pgfscope}%
\pgfpathrectangle{\pgfqpoint{1.254980in}{0.150000in}}{\pgfqpoint{5.490039in}{5.490039in}}%
\pgfusepath{clip}%
\pgfsetbuttcap%
\pgfsetroundjoin%
\definecolor{currentfill}{rgb}{0.269308,0.218818,0.509577}%
\pgfsetfillcolor{currentfill}%
\pgfsetfillopacity{0.700000}%
\pgfsetlinewidth{0.000000pt}%
\definecolor{currentstroke}{rgb}{0.000000,0.000000,0.000000}%
\pgfsetstrokecolor{currentstroke}%
\pgfsetdash{}{0pt}%
\pgfpathmoveto{\pgfqpoint{4.273572in}{2.509420in}}%
\pgfpathlineto{\pgfqpoint{4.286699in}{2.510051in}}%
\pgfpathlineto{\pgfqpoint{4.299836in}{2.510850in}}%
\pgfpathlineto{\pgfqpoint{4.312982in}{2.511818in}}%
\pgfpathlineto{\pgfqpoint{4.326137in}{2.512953in}}%
\pgfpathlineto{\pgfqpoint{4.333601in}{2.522115in}}%
\pgfpathlineto{\pgfqpoint{4.341061in}{2.531292in}}%
\pgfpathlineto{\pgfqpoint{4.348516in}{2.540486in}}%
\pgfpathlineto{\pgfqpoint{4.355966in}{2.549701in}}%
\pgfpathlineto{\pgfqpoint{4.342820in}{2.548792in}}%
\pgfpathlineto{\pgfqpoint{4.329683in}{2.548050in}}%
\pgfpathlineto{\pgfqpoint{4.316556in}{2.547477in}}%
\pgfpathlineto{\pgfqpoint{4.303437in}{2.547072in}}%
\pgfpathlineto{\pgfqpoint{4.295978in}{2.537621in}}%
\pgfpathlineto{\pgfqpoint{4.288514in}{2.528198in}}%
\pgfpathlineto{\pgfqpoint{4.281045in}{2.518798in}}%
\pgfpathlineto{\pgfqpoint{4.273572in}{2.509420in}}%
\pgfpathclose%
\pgfusepath{fill}%
\end{pgfscope}%
\begin{pgfscope}%
\pgfpathrectangle{\pgfqpoint{1.254980in}{0.150000in}}{\pgfqpoint{5.490039in}{5.490039in}}%
\pgfusepath{clip}%
\pgfsetbuttcap%
\pgfsetroundjoin%
\definecolor{currentfill}{rgb}{0.258965,0.251537,0.524736}%
\pgfsetfillcolor{currentfill}%
\pgfsetfillopacity{0.700000}%
\pgfsetlinewidth{0.000000pt}%
\definecolor{currentstroke}{rgb}{0.000000,0.000000,0.000000}%
\pgfsetstrokecolor{currentstroke}%
\pgfsetdash{}{0pt}%
\pgfpathmoveto{\pgfqpoint{2.895950in}{2.601548in}}%
\pgfpathlineto{\pgfqpoint{2.909022in}{2.586431in}}%
\pgfpathlineto{\pgfqpoint{2.922088in}{2.571577in}}%
\pgfpathlineto{\pgfqpoint{2.935148in}{2.556983in}}%
\pgfpathlineto{\pgfqpoint{2.948203in}{2.542648in}}%
\pgfpathlineto{\pgfqpoint{2.956140in}{2.550989in}}%
\pgfpathlineto{\pgfqpoint{2.964070in}{2.559433in}}%
\pgfpathlineto{\pgfqpoint{2.971991in}{2.567979in}}%
\pgfpathlineto{\pgfqpoint{2.979905in}{2.576628in}}%
\pgfpathlineto{\pgfqpoint{2.966869in}{2.590877in}}%
\pgfpathlineto{\pgfqpoint{2.953828in}{2.605385in}}%
\pgfpathlineto{\pgfqpoint{2.940782in}{2.620154in}}%
\pgfpathlineto{\pgfqpoint{2.927730in}{2.635185in}}%
\pgfpathlineto{\pgfqpoint{2.919797in}{2.626611in}}%
\pgfpathlineto{\pgfqpoint{2.911856in}{2.618147in}}%
\pgfpathlineto{\pgfqpoint{2.903907in}{2.609793in}}%
\pgfpathlineto{\pgfqpoint{2.895950in}{2.601548in}}%
\pgfpathclose%
\pgfusepath{fill}%
\end{pgfscope}%
\begin{pgfscope}%
\pgfpathrectangle{\pgfqpoint{1.254980in}{0.150000in}}{\pgfqpoint{5.490039in}{5.490039in}}%
\pgfusepath{clip}%
\pgfsetbuttcap%
\pgfsetroundjoin%
\definecolor{currentfill}{rgb}{0.179019,0.433756,0.557430}%
\pgfsetfillcolor{currentfill}%
\pgfsetfillopacity{0.700000}%
\pgfsetlinewidth{0.000000pt}%
\definecolor{currentstroke}{rgb}{0.000000,0.000000,0.000000}%
\pgfsetstrokecolor{currentstroke}%
\pgfsetdash{}{0pt}%
\pgfpathmoveto{\pgfqpoint{5.180108in}{2.993706in}}%
\pgfpathlineto{\pgfqpoint{5.193578in}{2.997336in}}%
\pgfpathlineto{\pgfqpoint{5.207060in}{3.001119in}}%
\pgfpathlineto{\pgfqpoint{5.220556in}{3.005055in}}%
\pgfpathlineto{\pgfqpoint{5.234065in}{3.009144in}}%
\pgfpathlineto{\pgfqpoint{5.241197in}{3.016803in}}%
\pgfpathlineto{\pgfqpoint{5.248328in}{3.024595in}}%
\pgfpathlineto{\pgfqpoint{5.255457in}{3.032524in}}%
\pgfpathlineto{\pgfqpoint{5.262584in}{3.040597in}}%
\pgfpathlineto{\pgfqpoint{5.249096in}{3.037046in}}%
\pgfpathlineto{\pgfqpoint{5.235622in}{3.033647in}}%
\pgfpathlineto{\pgfqpoint{5.222160in}{3.030400in}}%
\pgfpathlineto{\pgfqpoint{5.208711in}{3.027306in}}%
\pgfpathlineto{\pgfqpoint{5.201563in}{3.018687in}}%
\pgfpathlineto{\pgfqpoint{5.194413in}{3.010218in}}%
\pgfpathlineto{\pgfqpoint{5.187262in}{3.001893in}}%
\pgfpathlineto{\pgfqpoint{5.180108in}{2.993706in}}%
\pgfpathclose%
\pgfusepath{fill}%
\end{pgfscope}%
\begin{pgfscope}%
\pgfpathrectangle{\pgfqpoint{1.254980in}{0.150000in}}{\pgfqpoint{5.490039in}{5.490039in}}%
\pgfusepath{clip}%
\pgfsetbuttcap%
\pgfsetroundjoin%
\definecolor{currentfill}{rgb}{0.248629,0.278775,0.534556}%
\pgfsetfillcolor{currentfill}%
\pgfsetfillopacity{0.700000}%
\pgfsetlinewidth{0.000000pt}%
\definecolor{currentstroke}{rgb}{0.000000,0.000000,0.000000}%
\pgfsetstrokecolor{currentstroke}%
\pgfsetdash{}{0pt}%
\pgfpathmoveto{\pgfqpoint{2.843598in}{2.664692in}}%
\pgfpathlineto{\pgfqpoint{2.856697in}{2.648500in}}%
\pgfpathlineto{\pgfqpoint{2.869788in}{2.632580in}}%
\pgfpathlineto{\pgfqpoint{2.882872in}{2.616930in}}%
\pgfpathlineto{\pgfqpoint{2.895950in}{2.601548in}}%
\pgfpathlineto{\pgfqpoint{2.903907in}{2.609793in}}%
\pgfpathlineto{\pgfqpoint{2.911856in}{2.618147in}}%
\pgfpathlineto{\pgfqpoint{2.919797in}{2.626611in}}%
\pgfpathlineto{\pgfqpoint{2.927730in}{2.635185in}}%
\pgfpathlineto{\pgfqpoint{2.914671in}{2.650481in}}%
\pgfpathlineto{\pgfqpoint{2.901607in}{2.666044in}}%
\pgfpathlineto{\pgfqpoint{2.888536in}{2.681877in}}%
\pgfpathlineto{\pgfqpoint{2.875458in}{2.697981in}}%
\pgfpathlineto{\pgfqpoint{2.867506in}{2.689484in}}%
\pgfpathlineto{\pgfqpoint{2.859545in}{2.681103in}}%
\pgfpathlineto{\pgfqpoint{2.851576in}{2.672839in}}%
\pgfpathlineto{\pgfqpoint{2.843598in}{2.664692in}}%
\pgfpathclose%
\pgfusepath{fill}%
\end{pgfscope}%
\begin{pgfscope}%
\pgfpathrectangle{\pgfqpoint{1.254980in}{0.150000in}}{\pgfqpoint{5.490039in}{5.490039in}}%
\pgfusepath{clip}%
\pgfsetbuttcap%
\pgfsetroundjoin%
\definecolor{currentfill}{rgb}{0.283091,0.110553,0.431554}%
\pgfsetfillcolor{currentfill}%
\pgfsetfillopacity{0.700000}%
\pgfsetlinewidth{0.000000pt}%
\definecolor{currentstroke}{rgb}{0.000000,0.000000,0.000000}%
\pgfsetstrokecolor{currentstroke}%
\pgfsetdash{}{0pt}%
\pgfpathmoveto{\pgfqpoint{3.374823in}{2.311857in}}%
\pgfpathlineto{\pgfqpoint{3.387788in}{2.304352in}}%
\pgfpathlineto{\pgfqpoint{3.400754in}{2.297056in}}%
\pgfpathlineto{\pgfqpoint{3.413721in}{2.289966in}}%
\pgfpathlineto{\pgfqpoint{3.426689in}{2.283081in}}%
\pgfpathlineto{\pgfqpoint{3.434449in}{2.292538in}}%
\pgfpathlineto{\pgfqpoint{3.442203in}{2.302041in}}%
\pgfpathlineto{\pgfqpoint{3.449951in}{2.311590in}}%
\pgfpathlineto{\pgfqpoint{3.457693in}{2.321185in}}%
\pgfpathlineto{\pgfqpoint{3.444737in}{2.328044in}}%
\pgfpathlineto{\pgfqpoint{3.431782in}{2.335109in}}%
\pgfpathlineto{\pgfqpoint{3.418829in}{2.342380in}}%
\pgfpathlineto{\pgfqpoint{3.405877in}{2.349859in}}%
\pgfpathlineto{\pgfqpoint{3.398122in}{2.340279in}}%
\pgfpathlineto{\pgfqpoint{3.390362in}{2.330752in}}%
\pgfpathlineto{\pgfqpoint{3.382596in}{2.321278in}}%
\pgfpathlineto{\pgfqpoint{3.374823in}{2.311857in}}%
\pgfpathclose%
\pgfusepath{fill}%
\end{pgfscope}%
\begin{pgfscope}%
\pgfpathrectangle{\pgfqpoint{1.254980in}{0.150000in}}{\pgfqpoint{5.490039in}{5.490039in}}%
\pgfusepath{clip}%
\pgfsetbuttcap%
\pgfsetroundjoin%
\definecolor{currentfill}{rgb}{0.283229,0.120777,0.440584}%
\pgfsetfillcolor{currentfill}%
\pgfsetfillopacity{0.700000}%
\pgfsetlinewidth{0.000000pt}%
\definecolor{currentstroke}{rgb}{0.000000,0.000000,0.000000}%
\pgfsetstrokecolor{currentstroke}%
\pgfsetdash{}{0pt}%
\pgfpathmoveto{\pgfqpoint{3.726750in}{2.316425in}}%
\pgfpathlineto{\pgfqpoint{3.739739in}{2.312860in}}%
\pgfpathlineto{\pgfqpoint{3.752733in}{2.309481in}}%
\pgfpathlineto{\pgfqpoint{3.765732in}{2.306288in}}%
\pgfpathlineto{\pgfqpoint{3.778736in}{2.303281in}}%
\pgfpathlineto{\pgfqpoint{3.786379in}{2.313078in}}%
\pgfpathlineto{\pgfqpoint{3.794017in}{2.322893in}}%
\pgfpathlineto{\pgfqpoint{3.801650in}{2.332728in}}%
\pgfpathlineto{\pgfqpoint{3.809278in}{2.342583in}}%
\pgfpathlineto{\pgfqpoint{3.796283in}{2.345649in}}%
\pgfpathlineto{\pgfqpoint{3.783293in}{2.348900in}}%
\pgfpathlineto{\pgfqpoint{3.770308in}{2.352337in}}%
\pgfpathlineto{\pgfqpoint{3.757328in}{2.355962in}}%
\pgfpathlineto{\pgfqpoint{3.749691in}{2.346038in}}%
\pgfpathlineto{\pgfqpoint{3.742049in}{2.336141in}}%
\pgfpathlineto{\pgfqpoint{3.734402in}{2.326271in}}%
\pgfpathlineto{\pgfqpoint{3.726750in}{2.316425in}}%
\pgfpathclose%
\pgfusepath{fill}%
\end{pgfscope}%
\begin{pgfscope}%
\pgfpathrectangle{\pgfqpoint{1.254980in}{0.150000in}}{\pgfqpoint{5.490039in}{5.490039in}}%
\pgfusepath{clip}%
\pgfsetbuttcap%
\pgfsetroundjoin%
\definecolor{currentfill}{rgb}{0.274128,0.199721,0.498911}%
\pgfsetfillcolor{currentfill}%
\pgfsetfillopacity{0.700000}%
\pgfsetlinewidth{0.000000pt}%
\definecolor{currentstroke}{rgb}{0.000000,0.000000,0.000000}%
\pgfsetstrokecolor{currentstroke}%
\pgfsetdash{}{0pt}%
\pgfpathmoveto{\pgfqpoint{4.191168in}{2.470452in}}%
\pgfpathlineto{\pgfqpoint{4.204271in}{2.470601in}}%
\pgfpathlineto{\pgfqpoint{4.217382in}{2.470921in}}%
\pgfpathlineto{\pgfqpoint{4.230501in}{2.471411in}}%
\pgfpathlineto{\pgfqpoint{4.243629in}{2.472071in}}%
\pgfpathlineto{\pgfqpoint{4.251122in}{2.481389in}}%
\pgfpathlineto{\pgfqpoint{4.258610in}{2.490718in}}%
\pgfpathlineto{\pgfqpoint{4.266093in}{2.500061in}}%
\pgfpathlineto{\pgfqpoint{4.273572in}{2.509420in}}%
\pgfpathlineto{\pgfqpoint{4.260452in}{2.508959in}}%
\pgfpathlineto{\pgfqpoint{4.247342in}{2.508667in}}%
\pgfpathlineto{\pgfqpoint{4.234240in}{2.508545in}}%
\pgfpathlineto{\pgfqpoint{4.221146in}{2.508593in}}%
\pgfpathlineto{\pgfqpoint{4.213659in}{2.499026in}}%
\pgfpathlineto{\pgfqpoint{4.206167in}{2.489481in}}%
\pgfpathlineto{\pgfqpoint{4.198670in}{2.479958in}}%
\pgfpathlineto{\pgfqpoint{4.191168in}{2.470452in}}%
\pgfpathclose%
\pgfusepath{fill}%
\end{pgfscope}%
\begin{pgfscope}%
\pgfpathrectangle{\pgfqpoint{1.254980in}{0.150000in}}{\pgfqpoint{5.490039in}{5.490039in}}%
\pgfusepath{clip}%
\pgfsetbuttcap%
\pgfsetroundjoin%
\definecolor{currentfill}{rgb}{0.171176,0.452530,0.557965}%
\pgfsetfillcolor{currentfill}%
\pgfsetfillopacity{0.700000}%
\pgfsetlinewidth{0.000000pt}%
\definecolor{currentstroke}{rgb}{0.000000,0.000000,0.000000}%
\pgfsetstrokecolor{currentstroke}%
\pgfsetdash{}{0pt}%
\pgfpathmoveto{\pgfqpoint{5.262584in}{3.040597in}}%
\pgfpathlineto{\pgfqpoint{5.276086in}{3.044299in}}%
\pgfpathlineto{\pgfqpoint{5.289600in}{3.048154in}}%
\pgfpathlineto{\pgfqpoint{5.303128in}{3.052160in}}%
\pgfpathlineto{\pgfqpoint{5.316670in}{3.056318in}}%
\pgfpathlineto{\pgfqpoint{5.323774in}{3.063985in}}%
\pgfpathlineto{\pgfqpoint{5.330877in}{3.071802in}}%
\pgfpathlineto{\pgfqpoint{5.337979in}{3.079775in}}%
\pgfpathlineto{\pgfqpoint{5.345079in}{3.087911in}}%
\pgfpathlineto{\pgfqpoint{5.331560in}{3.084319in}}%
\pgfpathlineto{\pgfqpoint{5.318054in}{3.080878in}}%
\pgfpathlineto{\pgfqpoint{5.304562in}{3.077588in}}%
\pgfpathlineto{\pgfqpoint{5.291082in}{3.074449in}}%
\pgfpathlineto{\pgfqpoint{5.283959in}{3.065739in}}%
\pgfpathlineto{\pgfqpoint{5.276835in}{3.057198in}}%
\pgfpathlineto{\pgfqpoint{5.269710in}{3.048819in}}%
\pgfpathlineto{\pgfqpoint{5.262584in}{3.040597in}}%
\pgfpathclose%
\pgfusepath{fill}%
\end{pgfscope}%
\begin{pgfscope}%
\pgfpathrectangle{\pgfqpoint{1.254980in}{0.150000in}}{\pgfqpoint{5.490039in}{5.490039in}}%
\pgfusepath{clip}%
\pgfsetbuttcap%
\pgfsetroundjoin%
\definecolor{currentfill}{rgb}{0.267968,0.223549,0.512008}%
\pgfsetfillcolor{currentfill}%
\pgfsetfillopacity{0.700000}%
\pgfsetlinewidth{0.000000pt}%
\definecolor{currentstroke}{rgb}{0.000000,0.000000,0.000000}%
\pgfsetstrokecolor{currentstroke}%
\pgfsetdash{}{0pt}%
\pgfpathmoveto{\pgfqpoint{2.948203in}{2.542648in}}%
\pgfpathlineto{\pgfqpoint{2.961252in}{2.528569in}}%
\pgfpathlineto{\pgfqpoint{2.974297in}{2.514745in}}%
\pgfpathlineto{\pgfqpoint{2.987337in}{2.501172in}}%
\pgfpathlineto{\pgfqpoint{3.000372in}{2.487850in}}%
\pgfpathlineto{\pgfqpoint{3.008290in}{2.496286in}}%
\pgfpathlineto{\pgfqpoint{3.016201in}{2.504818in}}%
\pgfpathlineto{\pgfqpoint{3.024104in}{2.513446in}}%
\pgfpathlineto{\pgfqpoint{3.032000in}{2.522169in}}%
\pgfpathlineto{\pgfqpoint{3.018983in}{2.535407in}}%
\pgfpathlineto{\pgfqpoint{3.005962in}{2.548895in}}%
\pgfpathlineto{\pgfqpoint{2.992936in}{2.562634in}}%
\pgfpathlineto{\pgfqpoint{2.979905in}{2.576628in}}%
\pgfpathlineto{\pgfqpoint{2.971991in}{2.567979in}}%
\pgfpathlineto{\pgfqpoint{2.964070in}{2.559433in}}%
\pgfpathlineto{\pgfqpoint{2.956140in}{2.550989in}}%
\pgfpathlineto{\pgfqpoint{2.948203in}{2.542648in}}%
\pgfpathclose%
\pgfusepath{fill}%
\end{pgfscope}%
\begin{pgfscope}%
\pgfpathrectangle{\pgfqpoint{1.254980in}{0.150000in}}{\pgfqpoint{5.490039in}{5.490039in}}%
\pgfusepath{clip}%
\pgfsetbuttcap%
\pgfsetroundjoin%
\definecolor{currentfill}{rgb}{0.282910,0.105393,0.426902}%
\pgfsetfillcolor{currentfill}%
\pgfsetfillopacity{0.700000}%
\pgfsetlinewidth{0.000000pt}%
\definecolor{currentstroke}{rgb}{0.000000,0.000000,0.000000}%
\pgfsetstrokecolor{currentstroke}%
\pgfsetdash{}{0pt}%
\pgfpathmoveto{\pgfqpoint{3.509537in}{2.295779in}}%
\pgfpathlineto{\pgfqpoint{3.522504in}{2.289929in}}%
\pgfpathlineto{\pgfqpoint{3.535474in}{2.284278in}}%
\pgfpathlineto{\pgfqpoint{3.548446in}{2.278825in}}%
\pgfpathlineto{\pgfqpoint{3.561421in}{2.273568in}}%
\pgfpathlineto{\pgfqpoint{3.569135in}{2.283226in}}%
\pgfpathlineto{\pgfqpoint{3.576844in}{2.292917in}}%
\pgfpathlineto{\pgfqpoint{3.584547in}{2.302642in}}%
\pgfpathlineto{\pgfqpoint{3.592246in}{2.312401in}}%
\pgfpathlineto{\pgfqpoint{3.579281in}{2.317660in}}%
\pgfpathlineto{\pgfqpoint{3.566320in}{2.323117in}}%
\pgfpathlineto{\pgfqpoint{3.553361in}{2.328770in}}%
\pgfpathlineto{\pgfqpoint{3.540406in}{2.334622in}}%
\pgfpathlineto{\pgfqpoint{3.532697in}{2.324850in}}%
\pgfpathlineto{\pgfqpoint{3.524983in}{2.315119in}}%
\pgfpathlineto{\pgfqpoint{3.517263in}{2.305429in}}%
\pgfpathlineto{\pgfqpoint{3.509537in}{2.295779in}}%
\pgfpathclose%
\pgfusepath{fill}%
\end{pgfscope}%
\begin{pgfscope}%
\pgfpathrectangle{\pgfqpoint{1.254980in}{0.150000in}}{\pgfqpoint{5.490039in}{5.490039in}}%
\pgfusepath{clip}%
\pgfsetbuttcap%
\pgfsetroundjoin%
\definecolor{currentfill}{rgb}{0.235526,0.309527,0.542944}%
\pgfsetfillcolor{currentfill}%
\pgfsetfillopacity{0.700000}%
\pgfsetlinewidth{0.000000pt}%
\definecolor{currentstroke}{rgb}{0.000000,0.000000,0.000000}%
\pgfsetstrokecolor{currentstroke}%
\pgfsetdash{}{0pt}%
\pgfpathmoveto{\pgfqpoint{2.791131in}{2.732233in}}%
\pgfpathlineto{\pgfqpoint{2.804260in}{2.714927in}}%
\pgfpathlineto{\pgfqpoint{2.817380in}{2.697903in}}%
\pgfpathlineto{\pgfqpoint{2.830493in}{2.681159in}}%
\pgfpathlineto{\pgfqpoint{2.843598in}{2.664692in}}%
\pgfpathlineto{\pgfqpoint{2.851576in}{2.672839in}}%
\pgfpathlineto{\pgfqpoint{2.859545in}{2.681103in}}%
\pgfpathlineto{\pgfqpoint{2.867506in}{2.689484in}}%
\pgfpathlineto{\pgfqpoint{2.875458in}{2.697981in}}%
\pgfpathlineto{\pgfqpoint{2.862374in}{2.714361in}}%
\pgfpathlineto{\pgfqpoint{2.849282in}{2.731017in}}%
\pgfpathlineto{\pgfqpoint{2.836182in}{2.747953in}}%
\pgfpathlineto{\pgfqpoint{2.823075in}{2.765171in}}%
\pgfpathlineto{\pgfqpoint{2.815102in}{2.756751in}}%
\pgfpathlineto{\pgfqpoint{2.807120in}{2.748454in}}%
\pgfpathlineto{\pgfqpoint{2.799130in}{2.740282in}}%
\pgfpathlineto{\pgfqpoint{2.791131in}{2.732233in}}%
\pgfpathclose%
\pgfusepath{fill}%
\end{pgfscope}%
\begin{pgfscope}%
\pgfpathrectangle{\pgfqpoint{1.254980in}{0.150000in}}{\pgfqpoint{5.490039in}{5.490039in}}%
\pgfusepath{clip}%
\pgfsetbuttcap%
\pgfsetroundjoin%
\definecolor{currentfill}{rgb}{0.163625,0.471133,0.558148}%
\pgfsetfillcolor{currentfill}%
\pgfsetfillopacity{0.700000}%
\pgfsetlinewidth{0.000000pt}%
\definecolor{currentstroke}{rgb}{0.000000,0.000000,0.000000}%
\pgfsetstrokecolor{currentstroke}%
\pgfsetdash{}{0pt}%
\pgfpathmoveto{\pgfqpoint{5.345079in}{3.087911in}}%
\pgfpathlineto{\pgfqpoint{5.358612in}{3.091654in}}%
\pgfpathlineto{\pgfqpoint{5.372159in}{3.095547in}}%
\pgfpathlineto{\pgfqpoint{5.385719in}{3.099592in}}%
\pgfpathlineto{\pgfqpoint{5.399293in}{3.103787in}}%
\pgfpathlineto{\pgfqpoint{5.406369in}{3.111507in}}%
\pgfpathlineto{\pgfqpoint{5.413445in}{3.119397in}}%
\pgfpathlineto{\pgfqpoint{5.420521in}{3.127461in}}%
\pgfpathlineto{\pgfqpoint{5.427597in}{3.135708in}}%
\pgfpathlineto{\pgfqpoint{5.414047in}{3.132107in}}%
\pgfpathlineto{\pgfqpoint{5.400511in}{3.128656in}}%
\pgfpathlineto{\pgfqpoint{5.386988in}{3.125355in}}%
\pgfpathlineto{\pgfqpoint{5.373478in}{3.122204in}}%
\pgfpathlineto{\pgfqpoint{5.366379in}{3.113355in}}%
\pgfpathlineto{\pgfqpoint{5.359279in}{3.104694in}}%
\pgfpathlineto{\pgfqpoint{5.352180in}{3.096215in}}%
\pgfpathlineto{\pgfqpoint{5.345079in}{3.087911in}}%
\pgfpathclose%
\pgfusepath{fill}%
\end{pgfscope}%
\begin{pgfscope}%
\pgfpathrectangle{\pgfqpoint{1.254980in}{0.150000in}}{\pgfqpoint{5.490039in}{5.490039in}}%
\pgfusepath{clip}%
\pgfsetbuttcap%
\pgfsetroundjoin%
\definecolor{currentfill}{rgb}{0.277134,0.185228,0.489898}%
\pgfsetfillcolor{currentfill}%
\pgfsetfillopacity{0.700000}%
\pgfsetlinewidth{0.000000pt}%
\definecolor{currentstroke}{rgb}{0.000000,0.000000,0.000000}%
\pgfsetstrokecolor{currentstroke}%
\pgfsetdash{}{0pt}%
\pgfpathmoveto{\pgfqpoint{4.108749in}{2.433002in}}%
\pgfpathlineto{\pgfqpoint{4.121828in}{2.432633in}}%
\pgfpathlineto{\pgfqpoint{4.134915in}{2.432437in}}%
\pgfpathlineto{\pgfqpoint{4.148010in}{2.432413in}}%
\pgfpathlineto{\pgfqpoint{4.161113in}{2.432561in}}%
\pgfpathlineto{\pgfqpoint{4.168634in}{2.442018in}}%
\pgfpathlineto{\pgfqpoint{4.176150in}{2.451484in}}%
\pgfpathlineto{\pgfqpoint{4.183662in}{2.460961in}}%
\pgfpathlineto{\pgfqpoint{4.191168in}{2.470452in}}%
\pgfpathlineto{\pgfqpoint{4.178074in}{2.470474in}}%
\pgfpathlineto{\pgfqpoint{4.164987in}{2.470668in}}%
\pgfpathlineto{\pgfqpoint{4.151909in}{2.471034in}}%
\pgfpathlineto{\pgfqpoint{4.138838in}{2.471573in}}%
\pgfpathlineto{\pgfqpoint{4.131323in}{2.461902in}}%
\pgfpathlineto{\pgfqpoint{4.123803in}{2.452252in}}%
\pgfpathlineto{\pgfqpoint{4.116278in}{2.442619in}}%
\pgfpathlineto{\pgfqpoint{4.108749in}{2.433002in}}%
\pgfpathclose%
\pgfusepath{fill}%
\end{pgfscope}%
\begin{pgfscope}%
\pgfpathrectangle{\pgfqpoint{1.254980in}{0.150000in}}{\pgfqpoint{5.490039in}{5.490039in}}%
\pgfusepath{clip}%
\pgfsetbuttcap%
\pgfsetroundjoin%
\definecolor{currentfill}{rgb}{0.156270,0.489624,0.557936}%
\pgfsetfillcolor{currentfill}%
\pgfsetfillopacity{0.700000}%
\pgfsetlinewidth{0.000000pt}%
\definecolor{currentstroke}{rgb}{0.000000,0.000000,0.000000}%
\pgfsetstrokecolor{currentstroke}%
\pgfsetdash{}{0pt}%
\pgfpathmoveto{\pgfqpoint{5.427597in}{3.135708in}}%
\pgfpathlineto{\pgfqpoint{5.441161in}{3.139459in}}%
\pgfpathlineto{\pgfqpoint{5.454738in}{3.143360in}}%
\pgfpathlineto{\pgfqpoint{5.468330in}{3.147411in}}%
\pgfpathlineto{\pgfqpoint{5.481935in}{3.151611in}}%
\pgfpathlineto{\pgfqpoint{5.488986in}{3.159435in}}%
\pgfpathlineto{\pgfqpoint{5.496037in}{3.167448in}}%
\pgfpathlineto{\pgfqpoint{5.503089in}{3.175658in}}%
\pgfpathlineto{\pgfqpoint{5.510142in}{3.184070in}}%
\pgfpathlineto{\pgfqpoint{5.496562in}{3.180492in}}%
\pgfpathlineto{\pgfqpoint{5.482996in}{3.177062in}}%
\pgfpathlineto{\pgfqpoint{5.469444in}{3.173782in}}%
\pgfpathlineto{\pgfqpoint{5.455905in}{3.170651in}}%
\pgfpathlineto{\pgfqpoint{5.448827in}{3.161608in}}%
\pgfpathlineto{\pgfqpoint{5.441749in}{3.152775in}}%
\pgfpathlineto{\pgfqpoint{5.434673in}{3.144144in}}%
\pgfpathlineto{\pgfqpoint{5.427597in}{3.135708in}}%
\pgfpathclose%
\pgfusepath{fill}%
\end{pgfscope}%
\begin{pgfscope}%
\pgfpathrectangle{\pgfqpoint{1.254980in}{0.150000in}}{\pgfqpoint{5.490039in}{5.490039in}}%
\pgfusepath{clip}%
\pgfsetbuttcap%
\pgfsetroundjoin%
\definecolor{currentfill}{rgb}{0.274128,0.199721,0.498911}%
\pgfsetfillcolor{currentfill}%
\pgfsetfillopacity{0.700000}%
\pgfsetlinewidth{0.000000pt}%
\definecolor{currentstroke}{rgb}{0.000000,0.000000,0.000000}%
\pgfsetstrokecolor{currentstroke}%
\pgfsetdash{}{0pt}%
\pgfpathmoveto{\pgfqpoint{3.000372in}{2.487850in}}%
\pgfpathlineto{\pgfqpoint{3.013403in}{2.474775in}}%
\pgfpathlineto{\pgfqpoint{3.026430in}{2.461947in}}%
\pgfpathlineto{\pgfqpoint{3.039453in}{2.449362in}}%
\pgfpathlineto{\pgfqpoint{3.052472in}{2.437020in}}%
\pgfpathlineto{\pgfqpoint{3.060373in}{2.445551in}}%
\pgfpathlineto{\pgfqpoint{3.068266in}{2.454171in}}%
\pgfpathlineto{\pgfqpoint{3.076151in}{2.462880in}}%
\pgfpathlineto{\pgfqpoint{3.084030in}{2.471677in}}%
\pgfpathlineto{\pgfqpoint{3.071028in}{2.483935in}}%
\pgfpathlineto{\pgfqpoint{3.058022in}{2.496435in}}%
\pgfpathlineto{\pgfqpoint{3.045013in}{2.509179in}}%
\pgfpathlineto{\pgfqpoint{3.032000in}{2.522169in}}%
\pgfpathlineto{\pgfqpoint{3.024104in}{2.513446in}}%
\pgfpathlineto{\pgfqpoint{3.016201in}{2.504818in}}%
\pgfpathlineto{\pgfqpoint{3.008290in}{2.496286in}}%
\pgfpathlineto{\pgfqpoint{3.000372in}{2.487850in}}%
\pgfpathclose%
\pgfusepath{fill}%
\end{pgfscope}%
\begin{pgfscope}%
\pgfpathrectangle{\pgfqpoint{1.254980in}{0.150000in}}{\pgfqpoint{5.490039in}{5.490039in}}%
\pgfusepath{clip}%
\pgfsetbuttcap%
\pgfsetroundjoin%
\definecolor{currentfill}{rgb}{0.283072,0.130895,0.449241}%
\pgfsetfillcolor{currentfill}%
\pgfsetfillopacity{0.700000}%
\pgfsetlinewidth{0.000000pt}%
\definecolor{currentstroke}{rgb}{0.000000,0.000000,0.000000}%
\pgfsetstrokecolor{currentstroke}%
\pgfsetdash{}{0pt}%
\pgfpathmoveto{\pgfqpoint{3.239865in}{2.342811in}}%
\pgfpathlineto{\pgfqpoint{3.252842in}{2.333542in}}%
\pgfpathlineto{\pgfqpoint{3.265819in}{2.324491in}}%
\pgfpathlineto{\pgfqpoint{3.278794in}{2.315658in}}%
\pgfpathlineto{\pgfqpoint{3.291770in}{2.307040in}}%
\pgfpathlineto{\pgfqpoint{3.299580in}{2.316185in}}%
\pgfpathlineto{\pgfqpoint{3.307384in}{2.325391in}}%
\pgfpathlineto{\pgfqpoint{3.315181in}{2.334656in}}%
\pgfpathlineto{\pgfqpoint{3.322972in}{2.343981in}}%
\pgfpathlineto{\pgfqpoint{3.310010in}{2.352544in}}%
\pgfpathlineto{\pgfqpoint{3.297049in}{2.361324in}}%
\pgfpathlineto{\pgfqpoint{3.284086in}{2.370321in}}%
\pgfpathlineto{\pgfqpoint{3.271124in}{2.379536in}}%
\pgfpathlineto{\pgfqpoint{3.263319in}{2.370255in}}%
\pgfpathlineto{\pgfqpoint{3.255508in}{2.361040in}}%
\pgfpathlineto{\pgfqpoint{3.247690in}{2.351892in}}%
\pgfpathlineto{\pgfqpoint{3.239865in}{2.342811in}}%
\pgfpathclose%
\pgfusepath{fill}%
\end{pgfscope}%
\begin{pgfscope}%
\pgfpathrectangle{\pgfqpoint{1.254980in}{0.150000in}}{\pgfqpoint{5.490039in}{5.490039in}}%
\pgfusepath{clip}%
\pgfsetbuttcap%
\pgfsetroundjoin%
\definecolor{currentfill}{rgb}{0.221989,0.339161,0.548752}%
\pgfsetfillcolor{currentfill}%
\pgfsetfillopacity{0.700000}%
\pgfsetlinewidth{0.000000pt}%
\definecolor{currentstroke}{rgb}{0.000000,0.000000,0.000000}%
\pgfsetstrokecolor{currentstroke}%
\pgfsetdash{}{0pt}%
\pgfpathmoveto{\pgfqpoint{2.738530in}{2.804335in}}%
\pgfpathlineto{\pgfqpoint{2.751694in}{2.785872in}}%
\pgfpathlineto{\pgfqpoint{2.764848in}{2.767703in}}%
\pgfpathlineto{\pgfqpoint{2.777994in}{2.749824in}}%
\pgfpathlineto{\pgfqpoint{2.791131in}{2.732233in}}%
\pgfpathlineto{\pgfqpoint{2.799130in}{2.740282in}}%
\pgfpathlineto{\pgfqpoint{2.807120in}{2.748454in}}%
\pgfpathlineto{\pgfqpoint{2.815102in}{2.756751in}}%
\pgfpathlineto{\pgfqpoint{2.823075in}{2.765171in}}%
\pgfpathlineto{\pgfqpoint{2.809959in}{2.782674in}}%
\pgfpathlineto{\pgfqpoint{2.796835in}{2.800464in}}%
\pgfpathlineto{\pgfqpoint{2.783703in}{2.818545in}}%
\pgfpathlineto{\pgfqpoint{2.770561in}{2.836918in}}%
\pgfpathlineto{\pgfqpoint{2.762567in}{2.828576in}}%
\pgfpathlineto{\pgfqpoint{2.754564in}{2.820365in}}%
\pgfpathlineto{\pgfqpoint{2.746552in}{2.812284in}}%
\pgfpathlineto{\pgfqpoint{2.738530in}{2.804335in}}%
\pgfpathclose%
\pgfusepath{fill}%
\end{pgfscope}%
\begin{pgfscope}%
\pgfpathrectangle{\pgfqpoint{1.254980in}{0.150000in}}{\pgfqpoint{5.490039in}{5.490039in}}%
\pgfusepath{clip}%
\pgfsetbuttcap%
\pgfsetroundjoin%
\definecolor{currentfill}{rgb}{0.149039,0.508051,0.557250}%
\pgfsetfillcolor{currentfill}%
\pgfsetfillopacity{0.700000}%
\pgfsetlinewidth{0.000000pt}%
\definecolor{currentstroke}{rgb}{0.000000,0.000000,0.000000}%
\pgfsetstrokecolor{currentstroke}%
\pgfsetdash{}{0pt}%
\pgfpathmoveto{\pgfqpoint{5.510142in}{3.184070in}}%
\pgfpathlineto{\pgfqpoint{5.523735in}{3.187797in}}%
\pgfpathlineto{\pgfqpoint{5.537343in}{3.191674in}}%
\pgfpathlineto{\pgfqpoint{5.550965in}{3.195699in}}%
\pgfpathlineto{\pgfqpoint{5.564602in}{3.199873in}}%
\pgfpathlineto{\pgfqpoint{5.571629in}{3.207856in}}%
\pgfpathlineto{\pgfqpoint{5.578657in}{3.216049in}}%
\pgfpathlineto{\pgfqpoint{5.585687in}{3.224461in}}%
\pgfpathlineto{\pgfqpoint{5.592719in}{3.233098in}}%
\pgfpathlineto{\pgfqpoint{5.579110in}{3.229574in}}%
\pgfpathlineto{\pgfqpoint{5.565516in}{3.226199in}}%
\pgfpathlineto{\pgfqpoint{5.551935in}{3.222971in}}%
\pgfpathlineto{\pgfqpoint{5.538367in}{3.219892in}}%
\pgfpathlineto{\pgfqpoint{5.531308in}{3.210596in}}%
\pgfpathlineto{\pgfqpoint{5.524251in}{3.201532in}}%
\pgfpathlineto{\pgfqpoint{5.517196in}{3.192692in}}%
\pgfpathlineto{\pgfqpoint{5.510142in}{3.184070in}}%
\pgfpathclose%
\pgfusepath{fill}%
\end{pgfscope}%
\begin{pgfscope}%
\pgfpathrectangle{\pgfqpoint{1.254980in}{0.150000in}}{\pgfqpoint{5.490039in}{5.490039in}}%
\pgfusepath{clip}%
\pgfsetbuttcap%
\pgfsetroundjoin%
\definecolor{currentfill}{rgb}{0.280255,0.165693,0.476498}%
\pgfsetfillcolor{currentfill}%
\pgfsetfillopacity{0.700000}%
\pgfsetlinewidth{0.000000pt}%
\definecolor{currentstroke}{rgb}{0.000000,0.000000,0.000000}%
\pgfsetstrokecolor{currentstroke}%
\pgfsetdash{}{0pt}%
\pgfpathmoveto{\pgfqpoint{4.026306in}{2.397300in}}%
\pgfpathlineto{\pgfqpoint{4.039364in}{2.396376in}}%
\pgfpathlineto{\pgfqpoint{4.052429in}{2.395626in}}%
\pgfpathlineto{\pgfqpoint{4.065501in}{2.395052in}}%
\pgfpathlineto{\pgfqpoint{4.078581in}{2.394652in}}%
\pgfpathlineto{\pgfqpoint{4.086130in}{2.404226in}}%
\pgfpathlineto{\pgfqpoint{4.093675in}{2.413807in}}%
\pgfpathlineto{\pgfqpoint{4.101214in}{2.423399in}}%
\pgfpathlineto{\pgfqpoint{4.108749in}{2.433002in}}%
\pgfpathlineto{\pgfqpoint{4.095677in}{2.433545in}}%
\pgfpathlineto{\pgfqpoint{4.082613in}{2.434262in}}%
\pgfpathlineto{\pgfqpoint{4.069556in}{2.435153in}}%
\pgfpathlineto{\pgfqpoint{4.056507in}{2.436220in}}%
\pgfpathlineto{\pgfqpoint{4.048964in}{2.426464in}}%
\pgfpathlineto{\pgfqpoint{4.041416in}{2.416727in}}%
\pgfpathlineto{\pgfqpoint{4.033864in}{2.407006in}}%
\pgfpathlineto{\pgfqpoint{4.026306in}{2.397300in}}%
\pgfpathclose%
\pgfusepath{fill}%
\end{pgfscope}%
\begin{pgfscope}%
\pgfpathrectangle{\pgfqpoint{1.254980in}{0.150000in}}{\pgfqpoint{5.490039in}{5.490039in}}%
\pgfusepath{clip}%
\pgfsetbuttcap%
\pgfsetroundjoin%
\definecolor{currentfill}{rgb}{0.283091,0.110553,0.431554}%
\pgfsetfillcolor{currentfill}%
\pgfsetfillopacity{0.700000}%
\pgfsetlinewidth{0.000000pt}%
\definecolor{currentstroke}{rgb}{0.000000,0.000000,0.000000}%
\pgfsetstrokecolor{currentstroke}%
\pgfsetdash{}{0pt}%
\pgfpathmoveto{\pgfqpoint{3.644137in}{2.293307in}}%
\pgfpathlineto{\pgfqpoint{3.657119in}{2.289014in}}%
\pgfpathlineto{\pgfqpoint{3.670104in}{2.284913in}}%
\pgfpathlineto{\pgfqpoint{3.683095in}{2.281001in}}%
\pgfpathlineto{\pgfqpoint{3.696089in}{2.277278in}}%
\pgfpathlineto{\pgfqpoint{3.703762in}{2.287031in}}%
\pgfpathlineto{\pgfqpoint{3.711430in}{2.296806in}}%
\pgfpathlineto{\pgfqpoint{3.719092in}{2.306604in}}%
\pgfpathlineto{\pgfqpoint{3.726750in}{2.316425in}}%
\pgfpathlineto{\pgfqpoint{3.713765in}{2.320179in}}%
\pgfpathlineto{\pgfqpoint{3.700784in}{2.324122in}}%
\pgfpathlineto{\pgfqpoint{3.687808in}{2.328254in}}%
\pgfpathlineto{\pgfqpoint{3.674836in}{2.332577in}}%
\pgfpathlineto{\pgfqpoint{3.667169in}{2.322715in}}%
\pgfpathlineto{\pgfqpoint{3.659497in}{2.312883in}}%
\pgfpathlineto{\pgfqpoint{3.651819in}{2.303081in}}%
\pgfpathlineto{\pgfqpoint{3.644137in}{2.293307in}}%
\pgfpathclose%
\pgfusepath{fill}%
\end{pgfscope}%
\begin{pgfscope}%
\pgfpathrectangle{\pgfqpoint{1.254980in}{0.150000in}}{\pgfqpoint{5.490039in}{5.490039in}}%
\pgfusepath{clip}%
\pgfsetbuttcap%
\pgfsetroundjoin%
\definecolor{currentfill}{rgb}{0.278826,0.175490,0.483397}%
\pgfsetfillcolor{currentfill}%
\pgfsetfillopacity{0.700000}%
\pgfsetlinewidth{0.000000pt}%
\definecolor{currentstroke}{rgb}{0.000000,0.000000,0.000000}%
\pgfsetstrokecolor{currentstroke}%
\pgfsetdash{}{0pt}%
\pgfpathmoveto{\pgfqpoint{3.052472in}{2.437020in}}%
\pgfpathlineto{\pgfqpoint{3.065489in}{2.424919in}}%
\pgfpathlineto{\pgfqpoint{3.078502in}{2.413056in}}%
\pgfpathlineto{\pgfqpoint{3.091512in}{2.401429in}}%
\pgfpathlineto{\pgfqpoint{3.104519in}{2.390038in}}%
\pgfpathlineto{\pgfqpoint{3.112402in}{2.398662in}}%
\pgfpathlineto{\pgfqpoint{3.120278in}{2.407369in}}%
\pgfpathlineto{\pgfqpoint{3.128146in}{2.416158in}}%
\pgfpathlineto{\pgfqpoint{3.136008in}{2.425027in}}%
\pgfpathlineto{\pgfqpoint{3.123018in}{2.436336in}}%
\pgfpathlineto{\pgfqpoint{3.110025in}{2.447879in}}%
\pgfpathlineto{\pgfqpoint{3.097029in}{2.459659in}}%
\pgfpathlineto{\pgfqpoint{3.084030in}{2.471677in}}%
\pgfpathlineto{\pgfqpoint{3.076151in}{2.462880in}}%
\pgfpathlineto{\pgfqpoint{3.068266in}{2.454171in}}%
\pgfpathlineto{\pgfqpoint{3.060373in}{2.445551in}}%
\pgfpathlineto{\pgfqpoint{3.052472in}{2.437020in}}%
\pgfpathclose%
\pgfusepath{fill}%
\end{pgfscope}%
\begin{pgfscope}%
\pgfpathrectangle{\pgfqpoint{1.254980in}{0.150000in}}{\pgfqpoint{5.490039in}{5.490039in}}%
\pgfusepath{clip}%
\pgfsetbuttcap%
\pgfsetroundjoin%
\definecolor{currentfill}{rgb}{0.281887,0.150881,0.465405}%
\pgfsetfillcolor{currentfill}%
\pgfsetfillopacity{0.700000}%
\pgfsetlinewidth{0.000000pt}%
\definecolor{currentstroke}{rgb}{0.000000,0.000000,0.000000}%
\pgfsetstrokecolor{currentstroke}%
\pgfsetdash{}{0pt}%
\pgfpathmoveto{\pgfqpoint{3.943831in}{2.363595in}}%
\pgfpathlineto{\pgfqpoint{3.956869in}{2.362077in}}%
\pgfpathlineto{\pgfqpoint{3.969915in}{2.360737in}}%
\pgfpathlineto{\pgfqpoint{3.982967in}{2.359574in}}%
\pgfpathlineto{\pgfqpoint{3.996026in}{2.358588in}}%
\pgfpathlineto{\pgfqpoint{4.003604in}{2.368252in}}%
\pgfpathlineto{\pgfqpoint{4.011176in}{2.377925in}}%
\pgfpathlineto{\pgfqpoint{4.018743in}{2.387607in}}%
\pgfpathlineto{\pgfqpoint{4.026306in}{2.397300in}}%
\pgfpathlineto{\pgfqpoint{4.013255in}{2.398401in}}%
\pgfpathlineto{\pgfqpoint{4.000211in}{2.399678in}}%
\pgfpathlineto{\pgfqpoint{3.987174in}{2.401133in}}%
\pgfpathlineto{\pgfqpoint{3.974144in}{2.402765in}}%
\pgfpathlineto{\pgfqpoint{3.966573in}{2.392948in}}%
\pgfpathlineto{\pgfqpoint{3.958997in}{2.383148in}}%
\pgfpathlineto{\pgfqpoint{3.951416in}{2.373364in}}%
\pgfpathlineto{\pgfqpoint{3.943831in}{2.363595in}}%
\pgfpathclose%
\pgfusepath{fill}%
\end{pgfscope}%
\begin{pgfscope}%
\pgfpathrectangle{\pgfqpoint{1.254980in}{0.150000in}}{\pgfqpoint{5.490039in}{5.490039in}}%
\pgfusepath{clip}%
\pgfsetbuttcap%
\pgfsetroundjoin%
\definecolor{currentfill}{rgb}{0.141935,0.526453,0.555991}%
\pgfsetfillcolor{currentfill}%
\pgfsetfillopacity{0.700000}%
\pgfsetlinewidth{0.000000pt}%
\definecolor{currentstroke}{rgb}{0.000000,0.000000,0.000000}%
\pgfsetstrokecolor{currentstroke}%
\pgfsetdash{}{0pt}%
\pgfpathmoveto{\pgfqpoint{5.592719in}{3.233098in}}%
\pgfpathlineto{\pgfqpoint{5.606342in}{3.236770in}}%
\pgfpathlineto{\pgfqpoint{5.619979in}{3.240590in}}%
\pgfpathlineto{\pgfqpoint{5.633631in}{3.244559in}}%
\pgfpathlineto{\pgfqpoint{5.647297in}{3.248675in}}%
\pgfpathlineto{\pgfqpoint{5.654303in}{3.256877in}}%
\pgfpathlineto{\pgfqpoint{5.661311in}{3.265313in}}%
\pgfpathlineto{\pgfqpoint{5.668322in}{3.273990in}}%
\pgfpathlineto{\pgfqpoint{5.654678in}{3.270380in}}%
\pgfpathlineto{\pgfqpoint{5.641048in}{3.266918in}}%
\pgfpathlineto{\pgfqpoint{5.627432in}{3.263603in}}%
\pgfpathlineto{\pgfqpoint{5.613830in}{3.260436in}}%
\pgfpathlineto{\pgfqpoint{5.606790in}{3.251078in}}%
\pgfpathlineto{\pgfqpoint{5.599754in}{3.241968in}}%
\pgfpathlineto{\pgfqpoint{5.592719in}{3.233098in}}%
\pgfpathclose%
\pgfusepath{fill}%
\end{pgfscope}%
\begin{pgfscope}%
\pgfpathrectangle{\pgfqpoint{1.254980in}{0.150000in}}{\pgfqpoint{5.490039in}{5.490039in}}%
\pgfusepath{clip}%
\pgfsetbuttcap%
\pgfsetroundjoin%
\definecolor{currentfill}{rgb}{0.206756,0.371758,0.553117}%
\pgfsetfillcolor{currentfill}%
\pgfsetfillopacity{0.700000}%
\pgfsetlinewidth{0.000000pt}%
\definecolor{currentstroke}{rgb}{0.000000,0.000000,0.000000}%
\pgfsetstrokecolor{currentstroke}%
\pgfsetdash{}{0pt}%
\pgfpathmoveto{\pgfqpoint{2.685778in}{2.881176in}}%
\pgfpathlineto{\pgfqpoint{2.698981in}{2.861511in}}%
\pgfpathlineto{\pgfqpoint{2.712174in}{2.842152in}}%
\pgfpathlineto{\pgfqpoint{2.725357in}{2.823094in}}%
\pgfpathlineto{\pgfqpoint{2.738530in}{2.804335in}}%
\pgfpathlineto{\pgfqpoint{2.746552in}{2.812284in}}%
\pgfpathlineto{\pgfqpoint{2.754564in}{2.820365in}}%
\pgfpathlineto{\pgfqpoint{2.762567in}{2.828576in}}%
\pgfpathlineto{\pgfqpoint{2.770561in}{2.836918in}}%
\pgfpathlineto{\pgfqpoint{2.757411in}{2.855587in}}%
\pgfpathlineto{\pgfqpoint{2.744251in}{2.874555in}}%
\pgfpathlineto{\pgfqpoint{2.731081in}{2.893825in}}%
\pgfpathlineto{\pgfqpoint{2.717900in}{2.913399in}}%
\pgfpathlineto{\pgfqpoint{2.709884in}{2.905136in}}%
\pgfpathlineto{\pgfqpoint{2.701858in}{2.897011in}}%
\pgfpathlineto{\pgfqpoint{2.693823in}{2.889024in}}%
\pgfpathlineto{\pgfqpoint{2.685778in}{2.881176in}}%
\pgfpathclose%
\pgfusepath{fill}%
\end{pgfscope}%
\begin{pgfscope}%
\pgfpathrectangle{\pgfqpoint{1.254980in}{0.150000in}}{\pgfqpoint{5.490039in}{5.490039in}}%
\pgfusepath{clip}%
\pgfsetbuttcap%
\pgfsetroundjoin%
\definecolor{currentfill}{rgb}{0.282910,0.105393,0.426902}%
\pgfsetfillcolor{currentfill}%
\pgfsetfillopacity{0.700000}%
\pgfsetlinewidth{0.000000pt}%
\definecolor{currentstroke}{rgb}{0.000000,0.000000,0.000000}%
\pgfsetstrokecolor{currentstroke}%
\pgfsetdash{}{0pt}%
\pgfpathmoveto{\pgfqpoint{3.426689in}{2.283081in}}%
\pgfpathlineto{\pgfqpoint{3.439659in}{2.276401in}}%
\pgfpathlineto{\pgfqpoint{3.452631in}{2.269924in}}%
\pgfpathlineto{\pgfqpoint{3.465605in}{2.263649in}}%
\pgfpathlineto{\pgfqpoint{3.478581in}{2.257574in}}%
\pgfpathlineto{\pgfqpoint{3.486328in}{2.267066in}}%
\pgfpathlineto{\pgfqpoint{3.494070in}{2.276598in}}%
\pgfpathlineto{\pgfqpoint{3.501807in}{2.286168in}}%
\pgfpathlineto{\pgfqpoint{3.509537in}{2.295779in}}%
\pgfpathlineto{\pgfqpoint{3.496573in}{2.301828in}}%
\pgfpathlineto{\pgfqpoint{3.483611in}{2.308078in}}%
\pgfpathlineto{\pgfqpoint{3.470651in}{2.314530in}}%
\pgfpathlineto{\pgfqpoint{3.457693in}{2.321185in}}%
\pgfpathlineto{\pgfqpoint{3.449951in}{2.311590in}}%
\pgfpathlineto{\pgfqpoint{3.442203in}{2.302041in}}%
\pgfpathlineto{\pgfqpoint{3.434449in}{2.292538in}}%
\pgfpathlineto{\pgfqpoint{3.426689in}{2.283081in}}%
\pgfpathclose%
\pgfusepath{fill}%
\end{pgfscope}%
\begin{pgfscope}%
\pgfpathrectangle{\pgfqpoint{1.254980in}{0.150000in}}{\pgfqpoint{5.490039in}{5.490039in}}%
\pgfusepath{clip}%
\pgfsetbuttcap%
\pgfsetroundjoin%
\definecolor{currentfill}{rgb}{0.282884,0.135920,0.453427}%
\pgfsetfillcolor{currentfill}%
\pgfsetfillopacity{0.700000}%
\pgfsetlinewidth{0.000000pt}%
\definecolor{currentstroke}{rgb}{0.000000,0.000000,0.000000}%
\pgfsetstrokecolor{currentstroke}%
\pgfsetdash{}{0pt}%
\pgfpathmoveto{\pgfqpoint{3.861312in}{2.332158in}}%
\pgfpathlineto{\pgfqpoint{3.874334in}{2.330007in}}%
\pgfpathlineto{\pgfqpoint{3.887363in}{2.328038in}}%
\pgfpathlineto{\pgfqpoint{3.900397in}{2.326248in}}%
\pgfpathlineto{\pgfqpoint{3.913438in}{2.324638in}}%
\pgfpathlineto{\pgfqpoint{3.921044in}{2.334362in}}%
\pgfpathlineto{\pgfqpoint{3.928644in}{2.344096in}}%
\pgfpathlineto{\pgfqpoint{3.936240in}{2.353840in}}%
\pgfpathlineto{\pgfqpoint{3.943831in}{2.363595in}}%
\pgfpathlineto{\pgfqpoint{3.930798in}{2.365292in}}%
\pgfpathlineto{\pgfqpoint{3.917772in}{2.367169in}}%
\pgfpathlineto{\pgfqpoint{3.904752in}{2.369225in}}%
\pgfpathlineto{\pgfqpoint{3.891738in}{2.371462in}}%
\pgfpathlineto{\pgfqpoint{3.884139in}{2.361610in}}%
\pgfpathlineto{\pgfqpoint{3.876535in}{2.351776in}}%
\pgfpathlineto{\pgfqpoint{3.868926in}{2.341959in}}%
\pgfpathlineto{\pgfqpoint{3.861312in}{2.332158in}}%
\pgfpathclose%
\pgfusepath{fill}%
\end{pgfscope}%
\begin{pgfscope}%
\pgfpathrectangle{\pgfqpoint{1.254980in}{0.150000in}}{\pgfqpoint{5.490039in}{5.490039in}}%
\pgfusepath{clip}%
\pgfsetbuttcap%
\pgfsetroundjoin%
\definecolor{currentfill}{rgb}{0.283197,0.115680,0.436115}%
\pgfsetfillcolor{currentfill}%
\pgfsetfillopacity{0.700000}%
\pgfsetlinewidth{0.000000pt}%
\definecolor{currentstroke}{rgb}{0.000000,0.000000,0.000000}%
\pgfsetstrokecolor{currentstroke}%
\pgfsetdash{}{0pt}%
\pgfpathmoveto{\pgfqpoint{3.291770in}{2.307040in}}%
\pgfpathlineto{\pgfqpoint{3.304746in}{2.298637in}}%
\pgfpathlineto{\pgfqpoint{3.317722in}{2.290448in}}%
\pgfpathlineto{\pgfqpoint{3.330698in}{2.282470in}}%
\pgfpathlineto{\pgfqpoint{3.343675in}{2.274703in}}%
\pgfpathlineto{\pgfqpoint{3.351471in}{2.283912in}}%
\pgfpathlineto{\pgfqpoint{3.359261in}{2.293174in}}%
\pgfpathlineto{\pgfqpoint{3.367045in}{2.302489in}}%
\pgfpathlineto{\pgfqpoint{3.374823in}{2.311857in}}%
\pgfpathlineto{\pgfqpoint{3.361860in}{2.319571in}}%
\pgfpathlineto{\pgfqpoint{3.348897in}{2.327495in}}%
\pgfpathlineto{\pgfqpoint{3.335934in}{2.335631in}}%
\pgfpathlineto{\pgfqpoint{3.322972in}{2.343981in}}%
\pgfpathlineto{\pgfqpoint{3.315181in}{2.334656in}}%
\pgfpathlineto{\pgfqpoint{3.307384in}{2.325391in}}%
\pgfpathlineto{\pgfqpoint{3.299580in}{2.316185in}}%
\pgfpathlineto{\pgfqpoint{3.291770in}{2.307040in}}%
\pgfpathclose%
\pgfusepath{fill}%
\end{pgfscope}%
\begin{pgfscope}%
\pgfpathrectangle{\pgfqpoint{1.254980in}{0.150000in}}{\pgfqpoint{5.490039in}{5.490039in}}%
\pgfusepath{clip}%
\pgfsetbuttcap%
\pgfsetroundjoin%
\definecolor{currentfill}{rgb}{0.281412,0.155834,0.469201}%
\pgfsetfillcolor{currentfill}%
\pgfsetfillopacity{0.700000}%
\pgfsetlinewidth{0.000000pt}%
\definecolor{currentstroke}{rgb}{0.000000,0.000000,0.000000}%
\pgfsetstrokecolor{currentstroke}%
\pgfsetdash{}{0pt}%
\pgfpathmoveto{\pgfqpoint{3.104519in}{2.390038in}}%
\pgfpathlineto{\pgfqpoint{3.117524in}{2.378879in}}%
\pgfpathlineto{\pgfqpoint{3.130526in}{2.367953in}}%
\pgfpathlineto{\pgfqpoint{3.143526in}{2.357256in}}%
\pgfpathlineto{\pgfqpoint{3.156525in}{2.346787in}}%
\pgfpathlineto{\pgfqpoint{3.164391in}{2.355505in}}%
\pgfpathlineto{\pgfqpoint{3.172250in}{2.364298in}}%
\pgfpathlineto{\pgfqpoint{3.180103in}{2.373166in}}%
\pgfpathlineto{\pgfqpoint{3.187949in}{2.382108in}}%
\pgfpathlineto{\pgfqpoint{3.174967in}{2.392494in}}%
\pgfpathlineto{\pgfqpoint{3.161982in}{2.403108in}}%
\pgfpathlineto{\pgfqpoint{3.148996in}{2.413952in}}%
\pgfpathlineto{\pgfqpoint{3.136008in}{2.425027in}}%
\pgfpathlineto{\pgfqpoint{3.128146in}{2.416158in}}%
\pgfpathlineto{\pgfqpoint{3.120278in}{2.407369in}}%
\pgfpathlineto{\pgfqpoint{3.112402in}{2.398662in}}%
\pgfpathlineto{\pgfqpoint{3.104519in}{2.390038in}}%
\pgfpathclose%
\pgfusepath{fill}%
\end{pgfscope}%
\begin{pgfscope}%
\pgfpathrectangle{\pgfqpoint{1.254980in}{0.150000in}}{\pgfqpoint{5.490039in}{5.490039in}}%
\pgfusepath{clip}%
\pgfsetbuttcap%
\pgfsetroundjoin%
\definecolor{currentfill}{rgb}{0.282910,0.105393,0.426902}%
\pgfsetfillcolor{currentfill}%
\pgfsetfillopacity{0.700000}%
\pgfsetlinewidth{0.000000pt}%
\definecolor{currentstroke}{rgb}{0.000000,0.000000,0.000000}%
\pgfsetstrokecolor{currentstroke}%
\pgfsetdash{}{0pt}%
\pgfpathmoveto{\pgfqpoint{3.561421in}{2.273568in}}%
\pgfpathlineto{\pgfqpoint{3.574399in}{2.268506in}}%
\pgfpathlineto{\pgfqpoint{3.587380in}{2.263640in}}%
\pgfpathlineto{\pgfqpoint{3.600365in}{2.258966in}}%
\pgfpathlineto{\pgfqpoint{3.613353in}{2.254485in}}%
\pgfpathlineto{\pgfqpoint{3.621057in}{2.264151in}}%
\pgfpathlineto{\pgfqpoint{3.628755in}{2.273842in}}%
\pgfpathlineto{\pgfqpoint{3.636449in}{2.283561in}}%
\pgfpathlineto{\pgfqpoint{3.644137in}{2.293307in}}%
\pgfpathlineto{\pgfqpoint{3.631158in}{2.297790in}}%
\pgfpathlineto{\pgfqpoint{3.618184in}{2.302467in}}%
\pgfpathlineto{\pgfqpoint{3.605213in}{2.307337in}}%
\pgfpathlineto{\pgfqpoint{3.592246in}{2.312401in}}%
\pgfpathlineto{\pgfqpoint{3.584547in}{2.302642in}}%
\pgfpathlineto{\pgfqpoint{3.576844in}{2.292917in}}%
\pgfpathlineto{\pgfqpoint{3.569135in}{2.283226in}}%
\pgfpathlineto{\pgfqpoint{3.561421in}{2.273568in}}%
\pgfpathclose%
\pgfusepath{fill}%
\end{pgfscope}%
\begin{pgfscope}%
\pgfpathrectangle{\pgfqpoint{1.254980in}{0.150000in}}{\pgfqpoint{5.490039in}{5.490039in}}%
\pgfusepath{clip}%
\pgfsetbuttcap%
\pgfsetroundjoin%
\definecolor{currentfill}{rgb}{0.283229,0.120777,0.440584}%
\pgfsetfillcolor{currentfill}%
\pgfsetfillopacity{0.700000}%
\pgfsetlinewidth{0.000000pt}%
\definecolor{currentstroke}{rgb}{0.000000,0.000000,0.000000}%
\pgfsetstrokecolor{currentstroke}%
\pgfsetdash{}{0pt}%
\pgfpathmoveto{\pgfqpoint{3.778736in}{2.303281in}}%
\pgfpathlineto{\pgfqpoint{3.791745in}{2.300458in}}%
\pgfpathlineto{\pgfqpoint{3.804760in}{2.297819in}}%
\pgfpathlineto{\pgfqpoint{3.817780in}{2.295364in}}%
\pgfpathlineto{\pgfqpoint{3.830805in}{2.293091in}}%
\pgfpathlineto{\pgfqpoint{3.838439in}{2.302839in}}%
\pgfpathlineto{\pgfqpoint{3.846068in}{2.312599in}}%
\pgfpathlineto{\pgfqpoint{3.853692in}{2.322372in}}%
\pgfpathlineto{\pgfqpoint{3.861312in}{2.332158in}}%
\pgfpathlineto{\pgfqpoint{3.848295in}{2.334490in}}%
\pgfpathlineto{\pgfqpoint{3.835284in}{2.337005in}}%
\pgfpathlineto{\pgfqpoint{3.822278in}{2.339702in}}%
\pgfpathlineto{\pgfqpoint{3.809278in}{2.342583in}}%
\pgfpathlineto{\pgfqpoint{3.801650in}{2.332728in}}%
\pgfpathlineto{\pgfqpoint{3.794017in}{2.322893in}}%
\pgfpathlineto{\pgfqpoint{3.786379in}{2.313078in}}%
\pgfpathlineto{\pgfqpoint{3.778736in}{2.303281in}}%
\pgfpathclose%
\pgfusepath{fill}%
\end{pgfscope}%
\begin{pgfscope}%
\pgfpathrectangle{\pgfqpoint{1.254980in}{0.150000in}}{\pgfqpoint{5.490039in}{5.490039in}}%
\pgfusepath{clip}%
\pgfsetbuttcap%
\pgfsetroundjoin%
\definecolor{currentfill}{rgb}{0.282884,0.135920,0.453427}%
\pgfsetfillcolor{currentfill}%
\pgfsetfillopacity{0.700000}%
\pgfsetlinewidth{0.000000pt}%
\definecolor{currentstroke}{rgb}{0.000000,0.000000,0.000000}%
\pgfsetstrokecolor{currentstroke}%
\pgfsetdash{}{0pt}%
\pgfpathmoveto{\pgfqpoint{3.156525in}{2.346787in}}%
\pgfpathlineto{\pgfqpoint{3.169521in}{2.336545in}}%
\pgfpathlineto{\pgfqpoint{3.182517in}{2.326528in}}%
\pgfpathlineto{\pgfqpoint{3.195511in}{2.316734in}}%
\pgfpathlineto{\pgfqpoint{3.208503in}{2.307162in}}%
\pgfpathlineto{\pgfqpoint{3.216354in}{2.315973in}}%
\pgfpathlineto{\pgfqpoint{3.224198in}{2.324851in}}%
\pgfpathlineto{\pgfqpoint{3.232035in}{2.333797in}}%
\pgfpathlineto{\pgfqpoint{3.239865in}{2.342811in}}%
\pgfpathlineto{\pgfqpoint{3.226888in}{2.352301in}}%
\pgfpathlineto{\pgfqpoint{3.213910in}{2.362013in}}%
\pgfpathlineto{\pgfqpoint{3.200930in}{2.371948in}}%
\pgfpathlineto{\pgfqpoint{3.187949in}{2.382108in}}%
\pgfpathlineto{\pgfqpoint{3.180103in}{2.373166in}}%
\pgfpathlineto{\pgfqpoint{3.172250in}{2.364298in}}%
\pgfpathlineto{\pgfqpoint{3.164391in}{2.355505in}}%
\pgfpathlineto{\pgfqpoint{3.156525in}{2.346787in}}%
\pgfpathclose%
\pgfusepath{fill}%
\end{pgfscope}%
\begin{pgfscope}%
\pgfpathrectangle{\pgfqpoint{1.254980in}{0.150000in}}{\pgfqpoint{5.490039in}{5.490039in}}%
\pgfusepath{clip}%
\pgfsetbuttcap%
\pgfsetroundjoin%
\definecolor{currentfill}{rgb}{0.243113,0.292092,0.538516}%
\pgfsetfillcolor{currentfill}%
\pgfsetfillopacity{0.700000}%
\pgfsetlinewidth{0.000000pt}%
\definecolor{currentstroke}{rgb}{0.000000,0.000000,0.000000}%
\pgfsetstrokecolor{currentstroke}%
\pgfsetdash{}{0pt}%
\pgfpathmoveto{\pgfqpoint{4.573657in}{2.641889in}}%
\pgfpathlineto{\pgfqpoint{4.586912in}{2.644400in}}%
\pgfpathlineto{\pgfqpoint{4.600177in}{2.647073in}}%
\pgfpathlineto{\pgfqpoint{4.613453in}{2.649908in}}%
\pgfpathlineto{\pgfqpoint{4.626741in}{2.652904in}}%
\pgfpathlineto{\pgfqpoint{4.634106in}{2.661240in}}%
\pgfpathlineto{\pgfqpoint{4.641466in}{2.669601in}}%
\pgfpathlineto{\pgfqpoint{4.648822in}{2.677992in}}%
\pgfpathlineto{\pgfqpoint{4.656173in}{2.686415in}}%
\pgfpathlineto{\pgfqpoint{4.642898in}{2.683731in}}%
\pgfpathlineto{\pgfqpoint{4.629633in}{2.681208in}}%
\pgfpathlineto{\pgfqpoint{4.616379in}{2.678847in}}%
\pgfpathlineto{\pgfqpoint{4.603136in}{2.676647in}}%
\pgfpathlineto{\pgfqpoint{4.595773in}{2.667902in}}%
\pgfpathlineto{\pgfqpoint{4.588406in}{2.659196in}}%
\pgfpathlineto{\pgfqpoint{4.581034in}{2.650526in}}%
\pgfpathlineto{\pgfqpoint{4.573657in}{2.641889in}}%
\pgfpathclose%
\pgfusepath{fill}%
\end{pgfscope}%
\begin{pgfscope}%
\pgfpathrectangle{\pgfqpoint{1.254980in}{0.150000in}}{\pgfqpoint{5.490039in}{5.490039in}}%
\pgfusepath{clip}%
\pgfsetbuttcap%
\pgfsetroundjoin%
\definecolor{currentfill}{rgb}{0.250425,0.274290,0.533103}%
\pgfsetfillcolor{currentfill}%
\pgfsetfillopacity{0.700000}%
\pgfsetlinewidth{0.000000pt}%
\definecolor{currentstroke}{rgb}{0.000000,0.000000,0.000000}%
\pgfsetstrokecolor{currentstroke}%
\pgfsetdash{}{0pt}%
\pgfpathmoveto{\pgfqpoint{4.491147in}{2.598037in}}%
\pgfpathlineto{\pgfqpoint{4.504370in}{2.600180in}}%
\pgfpathlineto{\pgfqpoint{4.517604in}{2.602487in}}%
\pgfpathlineto{\pgfqpoint{4.530848in}{2.604957in}}%
\pgfpathlineto{\pgfqpoint{4.544103in}{2.607590in}}%
\pgfpathlineto{\pgfqpoint{4.551499in}{2.616133in}}%
\pgfpathlineto{\pgfqpoint{4.558889in}{2.624696in}}%
\pgfpathlineto{\pgfqpoint{4.566276in}{2.633280in}}%
\pgfpathlineto{\pgfqpoint{4.573657in}{2.641889in}}%
\pgfpathlineto{\pgfqpoint{4.560413in}{2.639540in}}%
\pgfpathlineto{\pgfqpoint{4.547180in}{2.637354in}}%
\pgfpathlineto{\pgfqpoint{4.533957in}{2.635330in}}%
\pgfpathlineto{\pgfqpoint{4.520744in}{2.633470in}}%
\pgfpathlineto{\pgfqpoint{4.513352in}{2.624567in}}%
\pgfpathlineto{\pgfqpoint{4.505955in}{2.615696in}}%
\pgfpathlineto{\pgfqpoint{4.498553in}{2.606854in}}%
\pgfpathlineto{\pgfqpoint{4.491147in}{2.598037in}}%
\pgfpathclose%
\pgfusepath{fill}%
\end{pgfscope}%
\begin{pgfscope}%
\pgfpathrectangle{\pgfqpoint{1.254980in}{0.150000in}}{\pgfqpoint{5.490039in}{5.490039in}}%
\pgfusepath{clip}%
\pgfsetbuttcap%
\pgfsetroundjoin%
\definecolor{currentfill}{rgb}{0.233603,0.313828,0.543914}%
\pgfsetfillcolor{currentfill}%
\pgfsetfillopacity{0.700000}%
\pgfsetlinewidth{0.000000pt}%
\definecolor{currentstroke}{rgb}{0.000000,0.000000,0.000000}%
\pgfsetstrokecolor{currentstroke}%
\pgfsetdash{}{0pt}%
\pgfpathmoveto{\pgfqpoint{4.656173in}{2.686415in}}%
\pgfpathlineto{\pgfqpoint{4.669460in}{2.689260in}}%
\pgfpathlineto{\pgfqpoint{4.682758in}{2.692265in}}%
\pgfpathlineto{\pgfqpoint{4.696068in}{2.695430in}}%
\pgfpathlineto{\pgfqpoint{4.709389in}{2.698756in}}%
\pgfpathlineto{\pgfqpoint{4.716723in}{2.706885in}}%
\pgfpathlineto{\pgfqpoint{4.724052in}{2.715049in}}%
\pgfpathlineto{\pgfqpoint{4.731377in}{2.723250in}}%
\pgfpathlineto{\pgfqpoint{4.738698in}{2.731492in}}%
\pgfpathlineto{\pgfqpoint{4.725390in}{2.728507in}}%
\pgfpathlineto{\pgfqpoint{4.712093in}{2.725682in}}%
\pgfpathlineto{\pgfqpoint{4.698807in}{2.723017in}}%
\pgfpathlineto{\pgfqpoint{4.685532in}{2.720512in}}%
\pgfpathlineto{\pgfqpoint{4.678199in}{2.711919in}}%
\pgfpathlineto{\pgfqpoint{4.670862in}{2.703375in}}%
\pgfpathlineto{\pgfqpoint{4.663520in}{2.694875in}}%
\pgfpathlineto{\pgfqpoint{4.656173in}{2.686415in}}%
\pgfpathclose%
\pgfusepath{fill}%
\end{pgfscope}%
\begin{pgfscope}%
\pgfpathrectangle{\pgfqpoint{1.254980in}{0.150000in}}{\pgfqpoint{5.490039in}{5.490039in}}%
\pgfusepath{clip}%
\pgfsetbuttcap%
\pgfsetroundjoin%
\definecolor{currentfill}{rgb}{0.258965,0.251537,0.524736}%
\pgfsetfillcolor{currentfill}%
\pgfsetfillopacity{0.700000}%
\pgfsetlinewidth{0.000000pt}%
\definecolor{currentstroke}{rgb}{0.000000,0.000000,0.000000}%
\pgfsetstrokecolor{currentstroke}%
\pgfsetdash{}{0pt}%
\pgfpathmoveto{\pgfqpoint{4.408641in}{2.555003in}}%
\pgfpathlineto{\pgfqpoint{4.421834in}{2.556744in}}%
\pgfpathlineto{\pgfqpoint{4.435037in}{2.558650in}}%
\pgfpathlineto{\pgfqpoint{4.448250in}{2.560720in}}%
\pgfpathlineto{\pgfqpoint{4.461473in}{2.562955in}}%
\pgfpathlineto{\pgfqpoint{4.468899in}{2.571704in}}%
\pgfpathlineto{\pgfqpoint{4.476320in}{2.580465in}}%
\pgfpathlineto{\pgfqpoint{4.483736in}{2.589241in}}%
\pgfpathlineto{\pgfqpoint{4.491147in}{2.598037in}}%
\pgfpathlineto{\pgfqpoint{4.477934in}{2.596057in}}%
\pgfpathlineto{\pgfqpoint{4.464731in}{2.594242in}}%
\pgfpathlineto{\pgfqpoint{4.451538in}{2.592592in}}%
\pgfpathlineto{\pgfqpoint{4.438355in}{2.591106in}}%
\pgfpathlineto{\pgfqpoint{4.430934in}{2.582045in}}%
\pgfpathlineto{\pgfqpoint{4.423508in}{2.573010in}}%
\pgfpathlineto{\pgfqpoint{4.416077in}{2.563997in}}%
\pgfpathlineto{\pgfqpoint{4.408641in}{2.555003in}}%
\pgfpathclose%
\pgfusepath{fill}%
\end{pgfscope}%
\begin{pgfscope}%
\pgfpathrectangle{\pgfqpoint{1.254980in}{0.150000in}}{\pgfqpoint{5.490039in}{5.490039in}}%
\pgfusepath{clip}%
\pgfsetbuttcap%
\pgfsetroundjoin%
\definecolor{currentfill}{rgb}{0.223925,0.334994,0.548053}%
\pgfsetfillcolor{currentfill}%
\pgfsetfillopacity{0.700000}%
\pgfsetlinewidth{0.000000pt}%
\definecolor{currentstroke}{rgb}{0.000000,0.000000,0.000000}%
\pgfsetstrokecolor{currentstroke}%
\pgfsetdash{}{0pt}%
\pgfpathmoveto{\pgfqpoint{4.738698in}{2.731492in}}%
\pgfpathlineto{\pgfqpoint{4.752018in}{2.734636in}}%
\pgfpathlineto{\pgfqpoint{4.765349in}{2.737939in}}%
\pgfpathlineto{\pgfqpoint{4.778692in}{2.741401in}}%
\pgfpathlineto{\pgfqpoint{4.792047in}{2.745022in}}%
\pgfpathlineto{\pgfqpoint{4.799350in}{2.752952in}}%
\pgfpathlineto{\pgfqpoint{4.806648in}{2.760924in}}%
\pgfpathlineto{\pgfqpoint{4.813942in}{2.768944in}}%
\pgfpathlineto{\pgfqpoint{4.821232in}{2.777016in}}%
\pgfpathlineto{\pgfqpoint{4.807890in}{2.773764in}}%
\pgfpathlineto{\pgfqpoint{4.794561in}{2.770671in}}%
\pgfpathlineto{\pgfqpoint{4.781243in}{2.767736in}}%
\pgfpathlineto{\pgfqpoint{4.767936in}{2.764960in}}%
\pgfpathlineto{\pgfqpoint{4.760633in}{2.756510in}}%
\pgfpathlineto{\pgfqpoint{4.753326in}{2.748118in}}%
\pgfpathlineto{\pgfqpoint{4.746014in}{2.739780in}}%
\pgfpathlineto{\pgfqpoint{4.738698in}{2.731492in}}%
\pgfpathclose%
\pgfusepath{fill}%
\end{pgfscope}%
\begin{pgfscope}%
\pgfpathrectangle{\pgfqpoint{1.254980in}{0.150000in}}{\pgfqpoint{5.490039in}{5.490039in}}%
\pgfusepath{clip}%
\pgfsetbuttcap%
\pgfsetroundjoin%
\definecolor{currentfill}{rgb}{0.265145,0.232956,0.516599}%
\pgfsetfillcolor{currentfill}%
\pgfsetfillopacity{0.700000}%
\pgfsetlinewidth{0.000000pt}%
\definecolor{currentstroke}{rgb}{0.000000,0.000000,0.000000}%
\pgfsetstrokecolor{currentstroke}%
\pgfsetdash{}{0pt}%
\pgfpathmoveto{\pgfqpoint{4.326137in}{2.512953in}}%
\pgfpathlineto{\pgfqpoint{4.339301in}{2.514256in}}%
\pgfpathlineto{\pgfqpoint{4.352474in}{2.515725in}}%
\pgfpathlineto{\pgfqpoint{4.365657in}{2.517362in}}%
\pgfpathlineto{\pgfqpoint{4.378850in}{2.519164in}}%
\pgfpathlineto{\pgfqpoint{4.386305in}{2.528109in}}%
\pgfpathlineto{\pgfqpoint{4.393756in}{2.537062in}}%
\pgfpathlineto{\pgfqpoint{4.401201in}{2.546026in}}%
\pgfpathlineto{\pgfqpoint{4.408641in}{2.555003in}}%
\pgfpathlineto{\pgfqpoint{4.395458in}{2.553428in}}%
\pgfpathlineto{\pgfqpoint{4.382285in}{2.552019in}}%
\pgfpathlineto{\pgfqpoint{4.369120in}{2.550777in}}%
\pgfpathlineto{\pgfqpoint{4.355966in}{2.549701in}}%
\pgfpathlineto{\pgfqpoint{4.348516in}{2.540486in}}%
\pgfpathlineto{\pgfqpoint{4.341061in}{2.531292in}}%
\pgfpathlineto{\pgfqpoint{4.333601in}{2.522115in}}%
\pgfpathlineto{\pgfqpoint{4.326137in}{2.512953in}}%
\pgfpathclose%
\pgfusepath{fill}%
\end{pgfscope}%
\begin{pgfscope}%
\pgfpathrectangle{\pgfqpoint{1.254980in}{0.150000in}}{\pgfqpoint{5.490039in}{5.490039in}}%
\pgfusepath{clip}%
\pgfsetbuttcap%
\pgfsetroundjoin%
\definecolor{currentfill}{rgb}{0.214298,0.355619,0.551184}%
\pgfsetfillcolor{currentfill}%
\pgfsetfillopacity{0.700000}%
\pgfsetlinewidth{0.000000pt}%
\definecolor{currentstroke}{rgb}{0.000000,0.000000,0.000000}%
\pgfsetstrokecolor{currentstroke}%
\pgfsetdash{}{0pt}%
\pgfpathmoveto{\pgfqpoint{4.821232in}{2.777016in}}%
\pgfpathlineto{\pgfqpoint{4.834585in}{2.780426in}}%
\pgfpathlineto{\pgfqpoint{4.847950in}{2.783993in}}%
\pgfpathlineto{\pgfqpoint{4.861328in}{2.787719in}}%
\pgfpathlineto{\pgfqpoint{4.874717in}{2.791601in}}%
\pgfpathlineto{\pgfqpoint{4.881988in}{2.799342in}}%
\pgfpathlineto{\pgfqpoint{4.889255in}{2.807136in}}%
\pgfpathlineto{\pgfqpoint{4.896517in}{2.814988in}}%
\pgfpathlineto{\pgfqpoint{4.903776in}{2.822904in}}%
\pgfpathlineto{\pgfqpoint{4.890401in}{2.819419in}}%
\pgfpathlineto{\pgfqpoint{4.877038in}{2.816090in}}%
\pgfpathlineto{\pgfqpoint{4.863687in}{2.812920in}}%
\pgfpathlineto{\pgfqpoint{4.850348in}{2.809906in}}%
\pgfpathlineto{\pgfqpoint{4.843075in}{2.801584in}}%
\pgfpathlineto{\pgfqpoint{4.835798in}{2.793331in}}%
\pgfpathlineto{\pgfqpoint{4.828517in}{2.785143in}}%
\pgfpathlineto{\pgfqpoint{4.821232in}{2.777016in}}%
\pgfpathclose%
\pgfusepath{fill}%
\end{pgfscope}%
\begin{pgfscope}%
\pgfpathrectangle{\pgfqpoint{1.254980in}{0.150000in}}{\pgfqpoint{5.490039in}{5.490039in}}%
\pgfusepath{clip}%
\pgfsetbuttcap%
\pgfsetroundjoin%
\definecolor{currentfill}{rgb}{0.206756,0.371758,0.553117}%
\pgfsetfillcolor{currentfill}%
\pgfsetfillopacity{0.700000}%
\pgfsetlinewidth{0.000000pt}%
\definecolor{currentstroke}{rgb}{0.000000,0.000000,0.000000}%
\pgfsetstrokecolor{currentstroke}%
\pgfsetdash{}{0pt}%
\pgfpathmoveto{\pgfqpoint{4.903776in}{2.822904in}}%
\pgfpathlineto{\pgfqpoint{4.917163in}{2.826546in}}%
\pgfpathlineto{\pgfqpoint{4.930562in}{2.830344in}}%
\pgfpathlineto{\pgfqpoint{4.943974in}{2.834299in}}%
\pgfpathlineto{\pgfqpoint{4.957399in}{2.838411in}}%
\pgfpathlineto{\pgfqpoint{4.964637in}{2.845978in}}%
\pgfpathlineto{\pgfqpoint{4.971872in}{2.853611in}}%
\pgfpathlineto{\pgfqpoint{4.979103in}{2.861314in}}%
\pgfpathlineto{\pgfqpoint{4.986330in}{2.869092in}}%
\pgfpathlineto{\pgfqpoint{4.972922in}{2.865407in}}%
\pgfpathlineto{\pgfqpoint{4.959526in}{2.861878in}}%
\pgfpathlineto{\pgfqpoint{4.946142in}{2.858504in}}%
\pgfpathlineto{\pgfqpoint{4.932770in}{2.855287in}}%
\pgfpathlineto{\pgfqpoint{4.925527in}{2.847073in}}%
\pgfpathlineto{\pgfqpoint{4.918280in}{2.838941in}}%
\pgfpathlineto{\pgfqpoint{4.911030in}{2.830886in}}%
\pgfpathlineto{\pgfqpoint{4.903776in}{2.822904in}}%
\pgfpathclose%
\pgfusepath{fill}%
\end{pgfscope}%
\begin{pgfscope}%
\pgfpathrectangle{\pgfqpoint{1.254980in}{0.150000in}}{\pgfqpoint{5.490039in}{5.490039in}}%
\pgfusepath{clip}%
\pgfsetbuttcap%
\pgfsetroundjoin%
\definecolor{currentfill}{rgb}{0.270595,0.214069,0.507052}%
\pgfsetfillcolor{currentfill}%
\pgfsetfillopacity{0.700000}%
\pgfsetlinewidth{0.000000pt}%
\definecolor{currentstroke}{rgb}{0.000000,0.000000,0.000000}%
\pgfsetstrokecolor{currentstroke}%
\pgfsetdash{}{0pt}%
\pgfpathmoveto{\pgfqpoint{4.243629in}{2.472071in}}%
\pgfpathlineto{\pgfqpoint{4.256766in}{2.472900in}}%
\pgfpathlineto{\pgfqpoint{4.269911in}{2.473898in}}%
\pgfpathlineto{\pgfqpoint{4.283066in}{2.475064in}}%
\pgfpathlineto{\pgfqpoint{4.296230in}{2.476399in}}%
\pgfpathlineto{\pgfqpoint{4.303714in}{2.485528in}}%
\pgfpathlineto{\pgfqpoint{4.311193in}{2.494662in}}%
\pgfpathlineto{\pgfqpoint{4.318667in}{2.503803in}}%
\pgfpathlineto{\pgfqpoint{4.326137in}{2.512953in}}%
\pgfpathlineto{\pgfqpoint{4.312982in}{2.511818in}}%
\pgfpathlineto{\pgfqpoint{4.299836in}{2.510850in}}%
\pgfpathlineto{\pgfqpoint{4.286699in}{2.510051in}}%
\pgfpathlineto{\pgfqpoint{4.273572in}{2.509420in}}%
\pgfpathlineto{\pgfqpoint{4.266093in}{2.500061in}}%
\pgfpathlineto{\pgfqpoint{4.258610in}{2.490718in}}%
\pgfpathlineto{\pgfqpoint{4.251122in}{2.481389in}}%
\pgfpathlineto{\pgfqpoint{4.243629in}{2.472071in}}%
\pgfpathclose%
\pgfusepath{fill}%
\end{pgfscope}%
\begin{pgfscope}%
\pgfpathrectangle{\pgfqpoint{1.254980in}{0.150000in}}{\pgfqpoint{5.490039in}{5.490039in}}%
\pgfusepath{clip}%
\pgfsetbuttcap%
\pgfsetroundjoin%
\definecolor{currentfill}{rgb}{0.197636,0.391528,0.554969}%
\pgfsetfillcolor{currentfill}%
\pgfsetfillopacity{0.700000}%
\pgfsetlinewidth{0.000000pt}%
\definecolor{currentstroke}{rgb}{0.000000,0.000000,0.000000}%
\pgfsetstrokecolor{currentstroke}%
\pgfsetdash{}{0pt}%
\pgfpathmoveto{\pgfqpoint{4.986330in}{2.869092in}}%
\pgfpathlineto{\pgfqpoint{4.999751in}{2.872933in}}%
\pgfpathlineto{\pgfqpoint{5.013185in}{2.876930in}}%
\pgfpathlineto{\pgfqpoint{5.026632in}{2.881081in}}%
\pgfpathlineto{\pgfqpoint{5.040091in}{2.885389in}}%
\pgfpathlineto{\pgfqpoint{5.047298in}{2.892803in}}%
\pgfpathlineto{\pgfqpoint{5.054501in}{2.900297in}}%
\pgfpathlineto{\pgfqpoint{5.061700in}{2.907874in}}%
\pgfpathlineto{\pgfqpoint{5.068896in}{2.915540in}}%
\pgfpathlineto{\pgfqpoint{5.055454in}{2.911688in}}%
\pgfpathlineto{\pgfqpoint{5.042024in}{2.907990in}}%
\pgfpathlineto{\pgfqpoint{5.028608in}{2.904447in}}%
\pgfpathlineto{\pgfqpoint{5.015203in}{2.901059in}}%
\pgfpathlineto{\pgfqpoint{5.007990in}{2.892930in}}%
\pgfpathlineto{\pgfqpoint{5.000773in}{2.884895in}}%
\pgfpathlineto{\pgfqpoint{4.993554in}{2.876951in}}%
\pgfpathlineto{\pgfqpoint{4.986330in}{2.869092in}}%
\pgfpathclose%
\pgfusepath{fill}%
\end{pgfscope}%
\begin{pgfscope}%
\pgfpathrectangle{\pgfqpoint{1.254980in}{0.150000in}}{\pgfqpoint{5.490039in}{5.490039in}}%
\pgfusepath{clip}%
\pgfsetbuttcap%
\pgfsetroundjoin%
\definecolor{currentfill}{rgb}{0.283091,0.110553,0.431554}%
\pgfsetfillcolor{currentfill}%
\pgfsetfillopacity{0.700000}%
\pgfsetlinewidth{0.000000pt}%
\definecolor{currentstroke}{rgb}{0.000000,0.000000,0.000000}%
\pgfsetstrokecolor{currentstroke}%
\pgfsetdash{}{0pt}%
\pgfpathmoveto{\pgfqpoint{3.696089in}{2.277278in}}%
\pgfpathlineto{\pgfqpoint{3.709088in}{2.273742in}}%
\pgfpathlineto{\pgfqpoint{3.722091in}{2.270395in}}%
\pgfpathlineto{\pgfqpoint{3.735100in}{2.267233in}}%
\pgfpathlineto{\pgfqpoint{3.748113in}{2.264257in}}%
\pgfpathlineto{\pgfqpoint{3.755776in}{2.273990in}}%
\pgfpathlineto{\pgfqpoint{3.763435in}{2.283737in}}%
\pgfpathlineto{\pgfqpoint{3.771088in}{2.293501in}}%
\pgfpathlineto{\pgfqpoint{3.778736in}{2.303281in}}%
\pgfpathlineto{\pgfqpoint{3.765732in}{2.306288in}}%
\pgfpathlineto{\pgfqpoint{3.752733in}{2.309481in}}%
\pgfpathlineto{\pgfqpoint{3.739739in}{2.312860in}}%
\pgfpathlineto{\pgfqpoint{3.726750in}{2.316425in}}%
\pgfpathlineto{\pgfqpoint{3.719092in}{2.306604in}}%
\pgfpathlineto{\pgfqpoint{3.711430in}{2.296806in}}%
\pgfpathlineto{\pgfqpoint{3.703762in}{2.287031in}}%
\pgfpathlineto{\pgfqpoint{3.696089in}{2.277278in}}%
\pgfpathclose%
\pgfusepath{fill}%
\end{pgfscope}%
\begin{pgfscope}%
\pgfpathrectangle{\pgfqpoint{1.254980in}{0.150000in}}{\pgfqpoint{5.490039in}{5.490039in}}%
\pgfusepath{clip}%
\pgfsetbuttcap%
\pgfsetroundjoin%
\definecolor{currentfill}{rgb}{0.282910,0.105393,0.426902}%
\pgfsetfillcolor{currentfill}%
\pgfsetfillopacity{0.700000}%
\pgfsetlinewidth{0.000000pt}%
\definecolor{currentstroke}{rgb}{0.000000,0.000000,0.000000}%
\pgfsetstrokecolor{currentstroke}%
\pgfsetdash{}{0pt}%
\pgfpathmoveto{\pgfqpoint{3.343675in}{2.274703in}}%
\pgfpathlineto{\pgfqpoint{3.356653in}{2.267145in}}%
\pgfpathlineto{\pgfqpoint{3.369632in}{2.259795in}}%
\pgfpathlineto{\pgfqpoint{3.382611in}{2.252652in}}%
\pgfpathlineto{\pgfqpoint{3.395593in}{2.245715in}}%
\pgfpathlineto{\pgfqpoint{3.403375in}{2.254987in}}%
\pgfpathlineto{\pgfqpoint{3.411152in}{2.264306in}}%
\pgfpathlineto{\pgfqpoint{3.418924in}{2.273670in}}%
\pgfpathlineto{\pgfqpoint{3.426689in}{2.283081in}}%
\pgfpathlineto{\pgfqpoint{3.413721in}{2.289966in}}%
\pgfpathlineto{\pgfqpoint{3.400754in}{2.297056in}}%
\pgfpathlineto{\pgfqpoint{3.387788in}{2.304352in}}%
\pgfpathlineto{\pgfqpoint{3.374823in}{2.311857in}}%
\pgfpathlineto{\pgfqpoint{3.367045in}{2.302489in}}%
\pgfpathlineto{\pgfqpoint{3.359261in}{2.293174in}}%
\pgfpathlineto{\pgfqpoint{3.351471in}{2.283912in}}%
\pgfpathlineto{\pgfqpoint{3.343675in}{2.274703in}}%
\pgfpathclose%
\pgfusepath{fill}%
\end{pgfscope}%
\begin{pgfscope}%
\pgfpathrectangle{\pgfqpoint{1.254980in}{0.150000in}}{\pgfqpoint{5.490039in}{5.490039in}}%
\pgfusepath{clip}%
\pgfsetbuttcap%
\pgfsetroundjoin%
\definecolor{currentfill}{rgb}{0.275191,0.194905,0.496005}%
\pgfsetfillcolor{currentfill}%
\pgfsetfillopacity{0.700000}%
\pgfsetlinewidth{0.000000pt}%
\definecolor{currentstroke}{rgb}{0.000000,0.000000,0.000000}%
\pgfsetstrokecolor{currentstroke}%
\pgfsetdash{}{0pt}%
\pgfpathmoveto{\pgfqpoint{4.161113in}{2.432561in}}%
\pgfpathlineto{\pgfqpoint{4.174224in}{2.432881in}}%
\pgfpathlineto{\pgfqpoint{4.187343in}{2.433372in}}%
\pgfpathlineto{\pgfqpoint{4.200471in}{2.434033in}}%
\pgfpathlineto{\pgfqpoint{4.213608in}{2.434864in}}%
\pgfpathlineto{\pgfqpoint{4.221120in}{2.444160in}}%
\pgfpathlineto{\pgfqpoint{4.228628in}{2.453459in}}%
\pgfpathlineto{\pgfqpoint{4.236131in}{2.462761in}}%
\pgfpathlineto{\pgfqpoint{4.243629in}{2.472071in}}%
\pgfpathlineto{\pgfqpoint{4.230501in}{2.471411in}}%
\pgfpathlineto{\pgfqpoint{4.217382in}{2.470921in}}%
\pgfpathlineto{\pgfqpoint{4.204271in}{2.470601in}}%
\pgfpathlineto{\pgfqpoint{4.191168in}{2.470452in}}%
\pgfpathlineto{\pgfqpoint{4.183662in}{2.460961in}}%
\pgfpathlineto{\pgfqpoint{4.176150in}{2.451484in}}%
\pgfpathlineto{\pgfqpoint{4.168634in}{2.442018in}}%
\pgfpathlineto{\pgfqpoint{4.161113in}{2.432561in}}%
\pgfpathclose%
\pgfusepath{fill}%
\end{pgfscope}%
\begin{pgfscope}%
\pgfpathrectangle{\pgfqpoint{1.254980in}{0.150000in}}{\pgfqpoint{5.490039in}{5.490039in}}%
\pgfusepath{clip}%
\pgfsetbuttcap%
\pgfsetroundjoin%
\definecolor{currentfill}{rgb}{0.188923,0.410910,0.556326}%
\pgfsetfillcolor{currentfill}%
\pgfsetfillopacity{0.700000}%
\pgfsetlinewidth{0.000000pt}%
\definecolor{currentstroke}{rgb}{0.000000,0.000000,0.000000}%
\pgfsetstrokecolor{currentstroke}%
\pgfsetdash{}{0pt}%
\pgfpathmoveto{\pgfqpoint{5.068896in}{2.915540in}}%
\pgfpathlineto{\pgfqpoint{5.082351in}{2.919547in}}%
\pgfpathlineto{\pgfqpoint{5.095819in}{2.923708in}}%
\pgfpathlineto{\pgfqpoint{5.109300in}{2.928024in}}%
\pgfpathlineto{\pgfqpoint{5.122795in}{2.932494in}}%
\pgfpathlineto{\pgfqpoint{5.129969in}{2.939781in}}%
\pgfpathlineto{\pgfqpoint{5.137141in}{2.947162in}}%
\pgfpathlineto{\pgfqpoint{5.144309in}{2.954641in}}%
\pgfpathlineto{\pgfqpoint{5.151474in}{2.962224in}}%
\pgfpathlineto{\pgfqpoint{5.137998in}{2.958237in}}%
\pgfpathlineto{\pgfqpoint{5.124535in}{2.954404in}}%
\pgfpathlineto{\pgfqpoint{5.111086in}{2.950725in}}%
\pgfpathlineto{\pgfqpoint{5.097649in}{2.947200in}}%
\pgfpathlineto{\pgfqpoint{5.090465in}{2.939125in}}%
\pgfpathlineto{\pgfqpoint{5.083278in}{2.931160in}}%
\pgfpathlineto{\pgfqpoint{5.076089in}{2.923300in}}%
\pgfpathlineto{\pgfqpoint{5.068896in}{2.915540in}}%
\pgfpathclose%
\pgfusepath{fill}%
\end{pgfscope}%
\begin{pgfscope}%
\pgfpathrectangle{\pgfqpoint{1.254980in}{0.150000in}}{\pgfqpoint{5.490039in}{5.490039in}}%
\pgfusepath{clip}%
\pgfsetbuttcap%
\pgfsetroundjoin%
\definecolor{currentfill}{rgb}{0.282656,0.100196,0.422160}%
\pgfsetfillcolor{currentfill}%
\pgfsetfillopacity{0.700000}%
\pgfsetlinewidth{0.000000pt}%
\definecolor{currentstroke}{rgb}{0.000000,0.000000,0.000000}%
\pgfsetstrokecolor{currentstroke}%
\pgfsetdash{}{0pt}%
\pgfpathmoveto{\pgfqpoint{3.478581in}{2.257574in}}%
\pgfpathlineto{\pgfqpoint{3.491559in}{2.251700in}}%
\pgfpathlineto{\pgfqpoint{3.504539in}{2.246024in}}%
\pgfpathlineto{\pgfqpoint{3.517523in}{2.240546in}}%
\pgfpathlineto{\pgfqpoint{3.530509in}{2.235264in}}%
\pgfpathlineto{\pgfqpoint{3.538245in}{2.244792in}}%
\pgfpathlineto{\pgfqpoint{3.545976in}{2.254351in}}%
\pgfpathlineto{\pgfqpoint{3.553701in}{2.263943in}}%
\pgfpathlineto{\pgfqpoint{3.561421in}{2.273568in}}%
\pgfpathlineto{\pgfqpoint{3.548446in}{2.278825in}}%
\pgfpathlineto{\pgfqpoint{3.535474in}{2.284278in}}%
\pgfpathlineto{\pgfqpoint{3.522504in}{2.289929in}}%
\pgfpathlineto{\pgfqpoint{3.509537in}{2.295779in}}%
\pgfpathlineto{\pgfqpoint{3.501807in}{2.286168in}}%
\pgfpathlineto{\pgfqpoint{3.494070in}{2.276598in}}%
\pgfpathlineto{\pgfqpoint{3.486328in}{2.267066in}}%
\pgfpathlineto{\pgfqpoint{3.478581in}{2.257574in}}%
\pgfpathclose%
\pgfusepath{fill}%
\end{pgfscope}%
\begin{pgfscope}%
\pgfpathrectangle{\pgfqpoint{1.254980in}{0.150000in}}{\pgfqpoint{5.490039in}{5.490039in}}%
\pgfusepath{clip}%
\pgfsetbuttcap%
\pgfsetroundjoin%
\definecolor{currentfill}{rgb}{0.260571,0.246922,0.522828}%
\pgfsetfillcolor{currentfill}%
\pgfsetfillopacity{0.700000}%
\pgfsetlinewidth{0.000000pt}%
\definecolor{currentstroke}{rgb}{0.000000,0.000000,0.000000}%
\pgfsetstrokecolor{currentstroke}%
\pgfsetdash{}{0pt}%
\pgfpathmoveto{\pgfqpoint{2.864038in}{2.569684in}}%
\pgfpathlineto{\pgfqpoint{2.877131in}{2.554452in}}%
\pgfpathlineto{\pgfqpoint{2.890218in}{2.539484in}}%
\pgfpathlineto{\pgfqpoint{2.903298in}{2.524776in}}%
\pgfpathlineto{\pgfqpoint{2.916373in}{2.510328in}}%
\pgfpathlineto{\pgfqpoint{2.924343in}{2.518250in}}%
\pgfpathlineto{\pgfqpoint{2.932304in}{2.526278in}}%
\pgfpathlineto{\pgfqpoint{2.940258in}{2.534411in}}%
\pgfpathlineto{\pgfqpoint{2.948203in}{2.542648in}}%
\pgfpathlineto{\pgfqpoint{2.935148in}{2.556983in}}%
\pgfpathlineto{\pgfqpoint{2.922088in}{2.571577in}}%
\pgfpathlineto{\pgfqpoint{2.909022in}{2.586431in}}%
\pgfpathlineto{\pgfqpoint{2.895950in}{2.601548in}}%
\pgfpathlineto{\pgfqpoint{2.887985in}{2.593414in}}%
\pgfpathlineto{\pgfqpoint{2.880011in}{2.585392in}}%
\pgfpathlineto{\pgfqpoint{2.872029in}{2.577481in}}%
\pgfpathlineto{\pgfqpoint{2.864038in}{2.569684in}}%
\pgfpathclose%
\pgfusepath{fill}%
\end{pgfscope}%
\begin{pgfscope}%
\pgfpathrectangle{\pgfqpoint{1.254980in}{0.150000in}}{\pgfqpoint{5.490039in}{5.490039in}}%
\pgfusepath{clip}%
\pgfsetbuttcap%
\pgfsetroundjoin%
\definecolor{currentfill}{rgb}{0.250425,0.274290,0.533103}%
\pgfsetfillcolor{currentfill}%
\pgfsetfillopacity{0.700000}%
\pgfsetlinewidth{0.000000pt}%
\definecolor{currentstroke}{rgb}{0.000000,0.000000,0.000000}%
\pgfsetstrokecolor{currentstroke}%
\pgfsetdash{}{0pt}%
\pgfpathmoveto{\pgfqpoint{2.811601in}{2.633288in}}%
\pgfpathlineto{\pgfqpoint{2.824721in}{2.616980in}}%
\pgfpathlineto{\pgfqpoint{2.837833in}{2.600945in}}%
\pgfpathlineto{\pgfqpoint{2.850939in}{2.585181in}}%
\pgfpathlineto{\pgfqpoint{2.864038in}{2.569684in}}%
\pgfpathlineto{\pgfqpoint{2.872029in}{2.577481in}}%
\pgfpathlineto{\pgfqpoint{2.880011in}{2.585392in}}%
\pgfpathlineto{\pgfqpoint{2.887985in}{2.593414in}}%
\pgfpathlineto{\pgfqpoint{2.895950in}{2.601548in}}%
\pgfpathlineto{\pgfqpoint{2.882872in}{2.616930in}}%
\pgfpathlineto{\pgfqpoint{2.869788in}{2.632580in}}%
\pgfpathlineto{\pgfqpoint{2.856697in}{2.648500in}}%
\pgfpathlineto{\pgfqpoint{2.843598in}{2.664692in}}%
\pgfpathlineto{\pgfqpoint{2.835612in}{2.656663in}}%
\pgfpathlineto{\pgfqpoint{2.827617in}{2.648752in}}%
\pgfpathlineto{\pgfqpoint{2.819613in}{2.640960in}}%
\pgfpathlineto{\pgfqpoint{2.811601in}{2.633288in}}%
\pgfpathclose%
\pgfusepath{fill}%
\end{pgfscope}%
\begin{pgfscope}%
\pgfpathrectangle{\pgfqpoint{1.254980in}{0.150000in}}{\pgfqpoint{5.490039in}{5.490039in}}%
\pgfusepath{clip}%
\pgfsetbuttcap%
\pgfsetroundjoin%
\definecolor{currentfill}{rgb}{0.180629,0.429975,0.557282}%
\pgfsetfillcolor{currentfill}%
\pgfsetfillopacity{0.700000}%
\pgfsetlinewidth{0.000000pt}%
\definecolor{currentstroke}{rgb}{0.000000,0.000000,0.000000}%
\pgfsetstrokecolor{currentstroke}%
\pgfsetdash{}{0pt}%
\pgfpathmoveto{\pgfqpoint{5.151474in}{2.962224in}}%
\pgfpathlineto{\pgfqpoint{5.164963in}{2.966364in}}%
\pgfpathlineto{\pgfqpoint{5.178465in}{2.970658in}}%
\pgfpathlineto{\pgfqpoint{5.191981in}{2.975105in}}%
\pgfpathlineto{\pgfqpoint{5.205510in}{2.979705in}}%
\pgfpathlineto{\pgfqpoint{5.212653in}{2.986896in}}%
\pgfpathlineto{\pgfqpoint{5.219792in}{2.994196in}}%
\pgfpathlineto{\pgfqpoint{5.226930in}{3.001610in}}%
\pgfpathlineto{\pgfqpoint{5.234065in}{3.009144in}}%
\pgfpathlineto{\pgfqpoint{5.220556in}{3.005055in}}%
\pgfpathlineto{\pgfqpoint{5.207060in}{3.001119in}}%
\pgfpathlineto{\pgfqpoint{5.193578in}{2.997336in}}%
\pgfpathlineto{\pgfqpoint{5.180108in}{2.993706in}}%
\pgfpathlineto{\pgfqpoint{5.172953in}{2.985652in}}%
\pgfpathlineto{\pgfqpoint{5.165796in}{2.977724in}}%
\pgfpathlineto{\pgfqpoint{5.158636in}{2.969916in}}%
\pgfpathlineto{\pgfqpoint{5.151474in}{2.962224in}}%
\pgfpathclose%
\pgfusepath{fill}%
\end{pgfscope}%
\begin{pgfscope}%
\pgfpathrectangle{\pgfqpoint{1.254980in}{0.150000in}}{\pgfqpoint{5.490039in}{5.490039in}}%
\pgfusepath{clip}%
\pgfsetbuttcap%
\pgfsetroundjoin%
\definecolor{currentfill}{rgb}{0.269308,0.218818,0.509577}%
\pgfsetfillcolor{currentfill}%
\pgfsetfillopacity{0.700000}%
\pgfsetlinewidth{0.000000pt}%
\definecolor{currentstroke}{rgb}{0.000000,0.000000,0.000000}%
\pgfsetstrokecolor{currentstroke}%
\pgfsetdash{}{0pt}%
\pgfpathmoveto{\pgfqpoint{2.916373in}{2.510328in}}%
\pgfpathlineto{\pgfqpoint{2.929442in}{2.496136in}}%
\pgfpathlineto{\pgfqpoint{2.942507in}{2.482198in}}%
\pgfpathlineto{\pgfqpoint{2.955566in}{2.468512in}}%
\pgfpathlineto{\pgfqpoint{2.968621in}{2.455077in}}%
\pgfpathlineto{\pgfqpoint{2.976570in}{2.463123in}}%
\pgfpathlineto{\pgfqpoint{2.984512in}{2.471268in}}%
\pgfpathlineto{\pgfqpoint{2.992446in}{2.479510in}}%
\pgfpathlineto{\pgfqpoint{3.000372in}{2.487850in}}%
\pgfpathlineto{\pgfqpoint{2.987337in}{2.501172in}}%
\pgfpathlineto{\pgfqpoint{2.974297in}{2.514745in}}%
\pgfpathlineto{\pgfqpoint{2.961252in}{2.528569in}}%
\pgfpathlineto{\pgfqpoint{2.948203in}{2.542648in}}%
\pgfpathlineto{\pgfqpoint{2.940258in}{2.534411in}}%
\pgfpathlineto{\pgfqpoint{2.932304in}{2.526278in}}%
\pgfpathlineto{\pgfqpoint{2.924343in}{2.518250in}}%
\pgfpathlineto{\pgfqpoint{2.916373in}{2.510328in}}%
\pgfpathclose%
\pgfusepath{fill}%
\end{pgfscope}%
\begin{pgfscope}%
\pgfpathrectangle{\pgfqpoint{1.254980in}{0.150000in}}{\pgfqpoint{5.490039in}{5.490039in}}%
\pgfusepath{clip}%
\pgfsetbuttcap%
\pgfsetroundjoin%
\definecolor{currentfill}{rgb}{0.278826,0.175490,0.483397}%
\pgfsetfillcolor{currentfill}%
\pgfsetfillopacity{0.700000}%
\pgfsetlinewidth{0.000000pt}%
\definecolor{currentstroke}{rgb}{0.000000,0.000000,0.000000}%
\pgfsetstrokecolor{currentstroke}%
\pgfsetdash{}{0pt}%
\pgfpathmoveto{\pgfqpoint{4.078581in}{2.394652in}}%
\pgfpathlineto{\pgfqpoint{4.091669in}{2.394425in}}%
\pgfpathlineto{\pgfqpoint{4.104764in}{2.394372in}}%
\pgfpathlineto{\pgfqpoint{4.117867in}{2.394491in}}%
\pgfpathlineto{\pgfqpoint{4.130979in}{2.394782in}}%
\pgfpathlineto{\pgfqpoint{4.138520in}{2.404223in}}%
\pgfpathlineto{\pgfqpoint{4.146056in}{2.413666in}}%
\pgfpathlineto{\pgfqpoint{4.153587in}{2.423111in}}%
\pgfpathlineto{\pgfqpoint{4.161113in}{2.432561in}}%
\pgfpathlineto{\pgfqpoint{4.148010in}{2.432413in}}%
\pgfpathlineto{\pgfqpoint{4.134915in}{2.432437in}}%
\pgfpathlineto{\pgfqpoint{4.121828in}{2.432633in}}%
\pgfpathlineto{\pgfqpoint{4.108749in}{2.433002in}}%
\pgfpathlineto{\pgfqpoint{4.101214in}{2.423399in}}%
\pgfpathlineto{\pgfqpoint{4.093675in}{2.413807in}}%
\pgfpathlineto{\pgfqpoint{4.086130in}{2.404226in}}%
\pgfpathlineto{\pgfqpoint{4.078581in}{2.394652in}}%
\pgfpathclose%
\pgfusepath{fill}%
\end{pgfscope}%
\begin{pgfscope}%
\pgfpathrectangle{\pgfqpoint{1.254980in}{0.150000in}}{\pgfqpoint{5.490039in}{5.490039in}}%
\pgfusepath{clip}%
\pgfsetbuttcap%
\pgfsetroundjoin%
\definecolor{currentfill}{rgb}{0.172719,0.448791,0.557885}%
\pgfsetfillcolor{currentfill}%
\pgfsetfillopacity{0.700000}%
\pgfsetlinewidth{0.000000pt}%
\definecolor{currentstroke}{rgb}{0.000000,0.000000,0.000000}%
\pgfsetstrokecolor{currentstroke}%
\pgfsetdash{}{0pt}%
\pgfpathmoveto{\pgfqpoint{5.234065in}{3.009144in}}%
\pgfpathlineto{\pgfqpoint{5.247587in}{3.013385in}}%
\pgfpathlineto{\pgfqpoint{5.261123in}{3.017778in}}%
\pgfpathlineto{\pgfqpoint{5.274673in}{3.022323in}}%
\pgfpathlineto{\pgfqpoint{5.288237in}{3.027021in}}%
\pgfpathlineto{\pgfqpoint{5.295348in}{3.034152in}}%
\pgfpathlineto{\pgfqpoint{5.302457in}{3.041407in}}%
\pgfpathlineto{\pgfqpoint{5.309565in}{3.048794in}}%
\pgfpathlineto{\pgfqpoint{5.316670in}{3.056318in}}%
\pgfpathlineto{\pgfqpoint{5.303128in}{3.052160in}}%
\pgfpathlineto{\pgfqpoint{5.289600in}{3.048154in}}%
\pgfpathlineto{\pgfqpoint{5.276086in}{3.044299in}}%
\pgfpathlineto{\pgfqpoint{5.262584in}{3.040597in}}%
\pgfpathlineto{\pgfqpoint{5.255457in}{3.032524in}}%
\pgfpathlineto{\pgfqpoint{5.248328in}{3.024595in}}%
\pgfpathlineto{\pgfqpoint{5.241197in}{3.016803in}}%
\pgfpathlineto{\pgfqpoint{5.234065in}{3.009144in}}%
\pgfpathclose%
\pgfusepath{fill}%
\end{pgfscope}%
\begin{pgfscope}%
\pgfpathrectangle{\pgfqpoint{1.254980in}{0.150000in}}{\pgfqpoint{5.490039in}{5.490039in}}%
\pgfusepath{clip}%
\pgfsetbuttcap%
\pgfsetroundjoin%
\definecolor{currentfill}{rgb}{0.237441,0.305202,0.541921}%
\pgfsetfillcolor{currentfill}%
\pgfsetfillopacity{0.700000}%
\pgfsetlinewidth{0.000000pt}%
\definecolor{currentstroke}{rgb}{0.000000,0.000000,0.000000}%
\pgfsetstrokecolor{currentstroke}%
\pgfsetdash{}{0pt}%
\pgfpathmoveto{\pgfqpoint{2.759044in}{2.701294in}}%
\pgfpathlineto{\pgfqpoint{2.772195in}{2.683871in}}%
\pgfpathlineto{\pgfqpoint{2.785338in}{2.666731in}}%
\pgfpathlineto{\pgfqpoint{2.798473in}{2.649870in}}%
\pgfpathlineto{\pgfqpoint{2.811601in}{2.633288in}}%
\pgfpathlineto{\pgfqpoint{2.819613in}{2.640960in}}%
\pgfpathlineto{\pgfqpoint{2.827617in}{2.648752in}}%
\pgfpathlineto{\pgfqpoint{2.835612in}{2.656663in}}%
\pgfpathlineto{\pgfqpoint{2.843598in}{2.664692in}}%
\pgfpathlineto{\pgfqpoint{2.830493in}{2.681159in}}%
\pgfpathlineto{\pgfqpoint{2.817380in}{2.697903in}}%
\pgfpathlineto{\pgfqpoint{2.804260in}{2.714927in}}%
\pgfpathlineto{\pgfqpoint{2.791131in}{2.732233in}}%
\pgfpathlineto{\pgfqpoint{2.783123in}{2.724309in}}%
\pgfpathlineto{\pgfqpoint{2.775106in}{2.716511in}}%
\pgfpathlineto{\pgfqpoint{2.767079in}{2.708839in}}%
\pgfpathlineto{\pgfqpoint{2.759044in}{2.701294in}}%
\pgfpathclose%
\pgfusepath{fill}%
\end{pgfscope}%
\begin{pgfscope}%
\pgfpathrectangle{\pgfqpoint{1.254980in}{0.150000in}}{\pgfqpoint{5.490039in}{5.490039in}}%
\pgfusepath{clip}%
\pgfsetbuttcap%
\pgfsetroundjoin%
\definecolor{currentfill}{rgb}{0.165117,0.467423,0.558141}%
\pgfsetfillcolor{currentfill}%
\pgfsetfillopacity{0.700000}%
\pgfsetlinewidth{0.000000pt}%
\definecolor{currentstroke}{rgb}{0.000000,0.000000,0.000000}%
\pgfsetstrokecolor{currentstroke}%
\pgfsetdash{}{0pt}%
\pgfpathmoveto{\pgfqpoint{5.316670in}{3.056318in}}%
\pgfpathlineto{\pgfqpoint{5.330226in}{3.060627in}}%
\pgfpathlineto{\pgfqpoint{5.343795in}{3.065088in}}%
\pgfpathlineto{\pgfqpoint{5.357379in}{3.069700in}}%
\pgfpathlineto{\pgfqpoint{5.370976in}{3.074464in}}%
\pgfpathlineto{\pgfqpoint{5.378057in}{3.081573in}}%
\pgfpathlineto{\pgfqpoint{5.385137in}{3.088826in}}%
\pgfpathlineto{\pgfqpoint{5.392215in}{3.096229in}}%
\pgfpathlineto{\pgfqpoint{5.399293in}{3.103787in}}%
\pgfpathlineto{\pgfqpoint{5.385719in}{3.099592in}}%
\pgfpathlineto{\pgfqpoint{5.372159in}{3.095547in}}%
\pgfpathlineto{\pgfqpoint{5.358612in}{3.091654in}}%
\pgfpathlineto{\pgfqpoint{5.345079in}{3.087911in}}%
\pgfpathlineto{\pgfqpoint{5.337979in}{3.079775in}}%
\pgfpathlineto{\pgfqpoint{5.330877in}{3.071802in}}%
\pgfpathlineto{\pgfqpoint{5.323774in}{3.063985in}}%
\pgfpathlineto{\pgfqpoint{5.316670in}{3.056318in}}%
\pgfpathclose%
\pgfusepath{fill}%
\end{pgfscope}%
\begin{pgfscope}%
\pgfpathrectangle{\pgfqpoint{1.254980in}{0.150000in}}{\pgfqpoint{5.490039in}{5.490039in}}%
\pgfusepath{clip}%
\pgfsetbuttcap%
\pgfsetroundjoin%
\definecolor{currentfill}{rgb}{0.283229,0.120777,0.440584}%
\pgfsetfillcolor{currentfill}%
\pgfsetfillopacity{0.700000}%
\pgfsetlinewidth{0.000000pt}%
\definecolor{currentstroke}{rgb}{0.000000,0.000000,0.000000}%
\pgfsetstrokecolor{currentstroke}%
\pgfsetdash{}{0pt}%
\pgfpathmoveto{\pgfqpoint{3.208503in}{2.307162in}}%
\pgfpathlineto{\pgfqpoint{3.221495in}{2.297811in}}%
\pgfpathlineto{\pgfqpoint{3.234487in}{2.288679in}}%
\pgfpathlineto{\pgfqpoint{3.247477in}{2.279764in}}%
\pgfpathlineto{\pgfqpoint{3.260468in}{2.271065in}}%
\pgfpathlineto{\pgfqpoint{3.268303in}{2.279967in}}%
\pgfpathlineto{\pgfqpoint{3.276132in}{2.288931in}}%
\pgfpathlineto{\pgfqpoint{3.283954in}{2.297955in}}%
\pgfpathlineto{\pgfqpoint{3.291770in}{2.307040in}}%
\pgfpathlineto{\pgfqpoint{3.278794in}{2.315658in}}%
\pgfpathlineto{\pgfqpoint{3.265819in}{2.324491in}}%
\pgfpathlineto{\pgfqpoint{3.252842in}{2.333542in}}%
\pgfpathlineto{\pgfqpoint{3.239865in}{2.342811in}}%
\pgfpathlineto{\pgfqpoint{3.232035in}{2.333797in}}%
\pgfpathlineto{\pgfqpoint{3.224198in}{2.324851in}}%
\pgfpathlineto{\pgfqpoint{3.216354in}{2.315973in}}%
\pgfpathlineto{\pgfqpoint{3.208503in}{2.307162in}}%
\pgfpathclose%
\pgfusepath{fill}%
\end{pgfscope}%
\begin{pgfscope}%
\pgfpathrectangle{\pgfqpoint{1.254980in}{0.150000in}}{\pgfqpoint{5.490039in}{5.490039in}}%
\pgfusepath{clip}%
\pgfsetbuttcap%
\pgfsetroundjoin%
\definecolor{currentfill}{rgb}{0.157729,0.485932,0.558013}%
\pgfsetfillcolor{currentfill}%
\pgfsetfillopacity{0.700000}%
\pgfsetlinewidth{0.000000pt}%
\definecolor{currentstroke}{rgb}{0.000000,0.000000,0.000000}%
\pgfsetstrokecolor{currentstroke}%
\pgfsetdash{}{0pt}%
\pgfpathmoveto{\pgfqpoint{5.399293in}{3.103787in}}%
\pgfpathlineto{\pgfqpoint{5.412881in}{3.108133in}}%
\pgfpathlineto{\pgfqpoint{5.426483in}{3.112629in}}%
\pgfpathlineto{\pgfqpoint{5.440099in}{3.117276in}}%
\pgfpathlineto{\pgfqpoint{5.453730in}{3.122073in}}%
\pgfpathlineto{\pgfqpoint{5.460782in}{3.129207in}}%
\pgfpathlineto{\pgfqpoint{5.467834in}{3.136504in}}%
\pgfpathlineto{\pgfqpoint{5.474884in}{3.143970in}}%
\pgfpathlineto{\pgfqpoint{5.481935in}{3.151611in}}%
\pgfpathlineto{\pgfqpoint{5.468330in}{3.147411in}}%
\pgfpathlineto{\pgfqpoint{5.454738in}{3.143360in}}%
\pgfpathlineto{\pgfqpoint{5.441161in}{3.139459in}}%
\pgfpathlineto{\pgfqpoint{5.427597in}{3.135708in}}%
\pgfpathlineto{\pgfqpoint{5.420521in}{3.127461in}}%
\pgfpathlineto{\pgfqpoint{5.413445in}{3.119397in}}%
\pgfpathlineto{\pgfqpoint{5.406369in}{3.111507in}}%
\pgfpathlineto{\pgfqpoint{5.399293in}{3.103787in}}%
\pgfpathclose%
\pgfusepath{fill}%
\end{pgfscope}%
\begin{pgfscope}%
\pgfpathrectangle{\pgfqpoint{1.254980in}{0.150000in}}{\pgfqpoint{5.490039in}{5.490039in}}%
\pgfusepath{clip}%
\pgfsetbuttcap%
\pgfsetroundjoin%
\definecolor{currentfill}{rgb}{0.275191,0.194905,0.496005}%
\pgfsetfillcolor{currentfill}%
\pgfsetfillopacity{0.700000}%
\pgfsetlinewidth{0.000000pt}%
\definecolor{currentstroke}{rgb}{0.000000,0.000000,0.000000}%
\pgfsetstrokecolor{currentstroke}%
\pgfsetdash{}{0pt}%
\pgfpathmoveto{\pgfqpoint{2.968621in}{2.455077in}}%
\pgfpathlineto{\pgfqpoint{2.981671in}{2.441890in}}%
\pgfpathlineto{\pgfqpoint{2.994717in}{2.428950in}}%
\pgfpathlineto{\pgfqpoint{3.007759in}{2.416254in}}%
\pgfpathlineto{\pgfqpoint{3.020797in}{2.403800in}}%
\pgfpathlineto{\pgfqpoint{3.028727in}{2.411968in}}%
\pgfpathlineto{\pgfqpoint{3.036650in}{2.420228in}}%
\pgfpathlineto{\pgfqpoint{3.044565in}{2.428579in}}%
\pgfpathlineto{\pgfqpoint{3.052472in}{2.437020in}}%
\pgfpathlineto{\pgfqpoint{3.039453in}{2.449362in}}%
\pgfpathlineto{\pgfqpoint{3.026430in}{2.461947in}}%
\pgfpathlineto{\pgfqpoint{3.013403in}{2.474775in}}%
\pgfpathlineto{\pgfqpoint{3.000372in}{2.487850in}}%
\pgfpathlineto{\pgfqpoint{2.992446in}{2.479510in}}%
\pgfpathlineto{\pgfqpoint{2.984512in}{2.471268in}}%
\pgfpathlineto{\pgfqpoint{2.976570in}{2.463123in}}%
\pgfpathlineto{\pgfqpoint{2.968621in}{2.455077in}}%
\pgfpathclose%
\pgfusepath{fill}%
\end{pgfscope}%
\begin{pgfscope}%
\pgfpathrectangle{\pgfqpoint{1.254980in}{0.150000in}}{\pgfqpoint{5.490039in}{5.490039in}}%
\pgfusepath{clip}%
\pgfsetbuttcap%
\pgfsetroundjoin%
\definecolor{currentfill}{rgb}{0.280868,0.160771,0.472899}%
\pgfsetfillcolor{currentfill}%
\pgfsetfillopacity{0.700000}%
\pgfsetlinewidth{0.000000pt}%
\definecolor{currentstroke}{rgb}{0.000000,0.000000,0.000000}%
\pgfsetstrokecolor{currentstroke}%
\pgfsetdash{}{0pt}%
\pgfpathmoveto{\pgfqpoint{3.996026in}{2.358588in}}%
\pgfpathlineto{\pgfqpoint{4.009093in}{2.357778in}}%
\pgfpathlineto{\pgfqpoint{4.022166in}{2.357143in}}%
\pgfpathlineto{\pgfqpoint{4.035247in}{2.356684in}}%
\pgfpathlineto{\pgfqpoint{4.048335in}{2.356398in}}%
\pgfpathlineto{\pgfqpoint{4.055904in}{2.365959in}}%
\pgfpathlineto{\pgfqpoint{4.063468in}{2.375520in}}%
\pgfpathlineto{\pgfqpoint{4.071027in}{2.385083in}}%
\pgfpathlineto{\pgfqpoint{4.078581in}{2.394652in}}%
\pgfpathlineto{\pgfqpoint{4.065501in}{2.395052in}}%
\pgfpathlineto{\pgfqpoint{4.052429in}{2.395626in}}%
\pgfpathlineto{\pgfqpoint{4.039364in}{2.396376in}}%
\pgfpathlineto{\pgfqpoint{4.026306in}{2.397300in}}%
\pgfpathlineto{\pgfqpoint{4.018743in}{2.387607in}}%
\pgfpathlineto{\pgfqpoint{4.011176in}{2.377925in}}%
\pgfpathlineto{\pgfqpoint{4.003604in}{2.368252in}}%
\pgfpathlineto{\pgfqpoint{3.996026in}{2.358588in}}%
\pgfpathclose%
\pgfusepath{fill}%
\end{pgfscope}%
\begin{pgfscope}%
\pgfpathrectangle{\pgfqpoint{1.254980in}{0.150000in}}{\pgfqpoint{5.490039in}{5.490039in}}%
\pgfusepath{clip}%
\pgfsetbuttcap%
\pgfsetroundjoin%
\definecolor{currentfill}{rgb}{0.150476,0.504369,0.557430}%
\pgfsetfillcolor{currentfill}%
\pgfsetfillopacity{0.700000}%
\pgfsetlinewidth{0.000000pt}%
\definecolor{currentstroke}{rgb}{0.000000,0.000000,0.000000}%
\pgfsetstrokecolor{currentstroke}%
\pgfsetdash{}{0pt}%
\pgfpathmoveto{\pgfqpoint{5.481935in}{3.151611in}}%
\pgfpathlineto{\pgfqpoint{5.495555in}{3.155961in}}%
\pgfpathlineto{\pgfqpoint{5.509189in}{3.160461in}}%
\pgfpathlineto{\pgfqpoint{5.522838in}{3.165111in}}%
\pgfpathlineto{\pgfqpoint{5.536501in}{3.169910in}}%
\pgfpathlineto{\pgfqpoint{5.543526in}{3.177119in}}%
\pgfpathlineto{\pgfqpoint{5.550550in}{3.184512in}}%
\pgfpathlineto{\pgfqpoint{5.557576in}{3.192094in}}%
\pgfpathlineto{\pgfqpoint{5.564602in}{3.199873in}}%
\pgfpathlineto{\pgfqpoint{5.550965in}{3.195699in}}%
\pgfpathlineto{\pgfqpoint{5.537343in}{3.191674in}}%
\pgfpathlineto{\pgfqpoint{5.523735in}{3.187797in}}%
\pgfpathlineto{\pgfqpoint{5.510142in}{3.184070in}}%
\pgfpathlineto{\pgfqpoint{5.503089in}{3.175658in}}%
\pgfpathlineto{\pgfqpoint{5.496037in}{3.167448in}}%
\pgfpathlineto{\pgfqpoint{5.488986in}{3.159435in}}%
\pgfpathlineto{\pgfqpoint{5.481935in}{3.151611in}}%
\pgfpathclose%
\pgfusepath{fill}%
\end{pgfscope}%
\begin{pgfscope}%
\pgfpathrectangle{\pgfqpoint{1.254980in}{0.150000in}}{\pgfqpoint{5.490039in}{5.490039in}}%
\pgfusepath{clip}%
\pgfsetbuttcap%
\pgfsetroundjoin%
\definecolor{currentfill}{rgb}{0.221989,0.339161,0.548752}%
\pgfsetfillcolor{currentfill}%
\pgfsetfillopacity{0.700000}%
\pgfsetlinewidth{0.000000pt}%
\definecolor{currentstroke}{rgb}{0.000000,0.000000,0.000000}%
\pgfsetstrokecolor{currentstroke}%
\pgfsetdash{}{0pt}%
\pgfpathmoveto{\pgfqpoint{2.706350in}{2.773866in}}%
\pgfpathlineto{\pgfqpoint{2.719537in}{2.755285in}}%
\pgfpathlineto{\pgfqpoint{2.732715in}{2.736998in}}%
\pgfpathlineto{\pgfqpoint{2.745884in}{2.719002in}}%
\pgfpathlineto{\pgfqpoint{2.759044in}{2.701294in}}%
\pgfpathlineto{\pgfqpoint{2.767079in}{2.708839in}}%
\pgfpathlineto{\pgfqpoint{2.775106in}{2.716511in}}%
\pgfpathlineto{\pgfqpoint{2.783123in}{2.724309in}}%
\pgfpathlineto{\pgfqpoint{2.791131in}{2.732233in}}%
\pgfpathlineto{\pgfqpoint{2.777994in}{2.749824in}}%
\pgfpathlineto{\pgfqpoint{2.764848in}{2.767703in}}%
\pgfpathlineto{\pgfqpoint{2.751694in}{2.785872in}}%
\pgfpathlineto{\pgfqpoint{2.738530in}{2.804335in}}%
\pgfpathlineto{\pgfqpoint{2.730499in}{2.796518in}}%
\pgfpathlineto{\pgfqpoint{2.722459in}{2.788834in}}%
\pgfpathlineto{\pgfqpoint{2.714409in}{2.781283in}}%
\pgfpathlineto{\pgfqpoint{2.706350in}{2.773866in}}%
\pgfpathclose%
\pgfusepath{fill}%
\end{pgfscope}%
\begin{pgfscope}%
\pgfpathrectangle{\pgfqpoint{1.254980in}{0.150000in}}{\pgfqpoint{5.490039in}{5.490039in}}%
\pgfusepath{clip}%
\pgfsetbuttcap%
\pgfsetroundjoin%
\definecolor{currentfill}{rgb}{0.282656,0.100196,0.422160}%
\pgfsetfillcolor{currentfill}%
\pgfsetfillopacity{0.700000}%
\pgfsetlinewidth{0.000000pt}%
\definecolor{currentstroke}{rgb}{0.000000,0.000000,0.000000}%
\pgfsetstrokecolor{currentstroke}%
\pgfsetdash{}{0pt}%
\pgfpathmoveto{\pgfqpoint{3.613353in}{2.254485in}}%
\pgfpathlineto{\pgfqpoint{3.626345in}{2.250196in}}%
\pgfpathlineto{\pgfqpoint{3.639341in}{2.246098in}}%
\pgfpathlineto{\pgfqpoint{3.652341in}{2.242190in}}%
\pgfpathlineto{\pgfqpoint{3.665346in}{2.238470in}}%
\pgfpathlineto{\pgfqpoint{3.673039in}{2.248142in}}%
\pgfpathlineto{\pgfqpoint{3.680728in}{2.257833in}}%
\pgfpathlineto{\pgfqpoint{3.688411in}{2.267545in}}%
\pgfpathlineto{\pgfqpoint{3.696089in}{2.277278in}}%
\pgfpathlineto{\pgfqpoint{3.683095in}{2.281001in}}%
\pgfpathlineto{\pgfqpoint{3.670104in}{2.284913in}}%
\pgfpathlineto{\pgfqpoint{3.657119in}{2.289014in}}%
\pgfpathlineto{\pgfqpoint{3.644137in}{2.293307in}}%
\pgfpathlineto{\pgfqpoint{3.636449in}{2.283561in}}%
\pgfpathlineto{\pgfqpoint{3.628755in}{2.273842in}}%
\pgfpathlineto{\pgfqpoint{3.621057in}{2.264151in}}%
\pgfpathlineto{\pgfqpoint{3.613353in}{2.254485in}}%
\pgfpathclose%
\pgfusepath{fill}%
\end{pgfscope}%
\begin{pgfscope}%
\pgfpathrectangle{\pgfqpoint{1.254980in}{0.150000in}}{\pgfqpoint{5.490039in}{5.490039in}}%
\pgfusepath{clip}%
\pgfsetbuttcap%
\pgfsetroundjoin%
\definecolor{currentfill}{rgb}{0.143343,0.522773,0.556295}%
\pgfsetfillcolor{currentfill}%
\pgfsetfillopacity{0.700000}%
\pgfsetlinewidth{0.000000pt}%
\definecolor{currentstroke}{rgb}{0.000000,0.000000,0.000000}%
\pgfsetstrokecolor{currentstroke}%
\pgfsetdash{}{0pt}%
\pgfpathmoveto{\pgfqpoint{5.564602in}{3.199873in}}%
\pgfpathlineto{\pgfqpoint{5.578252in}{3.204196in}}%
\pgfpathlineto{\pgfqpoint{5.591918in}{3.208668in}}%
\pgfpathlineto{\pgfqpoint{5.605598in}{3.213288in}}%
\pgfpathlineto{\pgfqpoint{5.619292in}{3.218058in}}%
\pgfpathlineto{\pgfqpoint{5.626291in}{3.225398in}}%
\pgfpathlineto{\pgfqpoint{5.633292in}{3.232942in}}%
\pgfpathlineto{\pgfqpoint{5.640293in}{3.240699in}}%
\pgfpathlineto{\pgfqpoint{5.647297in}{3.248675in}}%
\pgfpathlineto{\pgfqpoint{5.633631in}{3.244559in}}%
\pgfpathlineto{\pgfqpoint{5.619979in}{3.240590in}}%
\pgfpathlineto{\pgfqpoint{5.606342in}{3.236770in}}%
\pgfpathlineto{\pgfqpoint{5.592719in}{3.233098in}}%
\pgfpathlineto{\pgfqpoint{5.585687in}{3.224461in}}%
\pgfpathlineto{\pgfqpoint{5.578657in}{3.216049in}}%
\pgfpathlineto{\pgfqpoint{5.571629in}{3.207856in}}%
\pgfpathlineto{\pgfqpoint{5.564602in}{3.199873in}}%
\pgfpathclose%
\pgfusepath{fill}%
\end{pgfscope}%
\begin{pgfscope}%
\pgfpathrectangle{\pgfqpoint{1.254980in}{0.150000in}}{\pgfqpoint{5.490039in}{5.490039in}}%
\pgfusepath{clip}%
\pgfsetbuttcap%
\pgfsetroundjoin%
\definecolor{currentfill}{rgb}{0.282623,0.140926,0.457517}%
\pgfsetfillcolor{currentfill}%
\pgfsetfillopacity{0.700000}%
\pgfsetlinewidth{0.000000pt}%
\definecolor{currentstroke}{rgb}{0.000000,0.000000,0.000000}%
\pgfsetstrokecolor{currentstroke}%
\pgfsetdash{}{0pt}%
\pgfpathmoveto{\pgfqpoint{3.913438in}{2.324638in}}%
\pgfpathlineto{\pgfqpoint{3.926486in}{2.323206in}}%
\pgfpathlineto{\pgfqpoint{3.939540in}{2.321953in}}%
\pgfpathlineto{\pgfqpoint{3.952600in}{2.320877in}}%
\pgfpathlineto{\pgfqpoint{3.965668in}{2.319978in}}%
\pgfpathlineto{\pgfqpoint{3.973265in}{2.329626in}}%
\pgfpathlineto{\pgfqpoint{3.980857in}{2.339276in}}%
\pgfpathlineto{\pgfqpoint{3.988444in}{2.348929in}}%
\pgfpathlineto{\pgfqpoint{3.996026in}{2.358588in}}%
\pgfpathlineto{\pgfqpoint{3.982967in}{2.359574in}}%
\pgfpathlineto{\pgfqpoint{3.969915in}{2.360737in}}%
\pgfpathlineto{\pgfqpoint{3.956869in}{2.362077in}}%
\pgfpathlineto{\pgfqpoint{3.943831in}{2.363595in}}%
\pgfpathlineto{\pgfqpoint{3.936240in}{2.353840in}}%
\pgfpathlineto{\pgfqpoint{3.928644in}{2.344096in}}%
\pgfpathlineto{\pgfqpoint{3.921044in}{2.334362in}}%
\pgfpathlineto{\pgfqpoint{3.913438in}{2.324638in}}%
\pgfpathclose%
\pgfusepath{fill}%
\end{pgfscope}%
\begin{pgfscope}%
\pgfpathrectangle{\pgfqpoint{1.254980in}{0.150000in}}{\pgfqpoint{5.490039in}{5.490039in}}%
\pgfusepath{clip}%
\pgfsetbuttcap%
\pgfsetroundjoin%
\definecolor{currentfill}{rgb}{0.279574,0.170599,0.479997}%
\pgfsetfillcolor{currentfill}%
\pgfsetfillopacity{0.700000}%
\pgfsetlinewidth{0.000000pt}%
\definecolor{currentstroke}{rgb}{0.000000,0.000000,0.000000}%
\pgfsetstrokecolor{currentstroke}%
\pgfsetdash{}{0pt}%
\pgfpathmoveto{\pgfqpoint{3.020797in}{2.403800in}}%
\pgfpathlineto{\pgfqpoint{3.033831in}{2.391587in}}%
\pgfpathlineto{\pgfqpoint{3.046862in}{2.379612in}}%
\pgfpathlineto{\pgfqpoint{3.059890in}{2.367875in}}%
\pgfpathlineto{\pgfqpoint{3.072915in}{2.356372in}}%
\pgfpathlineto{\pgfqpoint{3.080827in}{2.364662in}}%
\pgfpathlineto{\pgfqpoint{3.088731in}{2.373037in}}%
\pgfpathlineto{\pgfqpoint{3.096629in}{2.381496in}}%
\pgfpathlineto{\pgfqpoint{3.104519in}{2.390038in}}%
\pgfpathlineto{\pgfqpoint{3.091512in}{2.401429in}}%
\pgfpathlineto{\pgfqpoint{3.078502in}{2.413056in}}%
\pgfpathlineto{\pgfqpoint{3.065489in}{2.424919in}}%
\pgfpathlineto{\pgfqpoint{3.052472in}{2.437020in}}%
\pgfpathlineto{\pgfqpoint{3.044565in}{2.428579in}}%
\pgfpathlineto{\pgfqpoint{3.036650in}{2.420228in}}%
\pgfpathlineto{\pgfqpoint{3.028727in}{2.411968in}}%
\pgfpathlineto{\pgfqpoint{3.020797in}{2.403800in}}%
\pgfpathclose%
\pgfusepath{fill}%
\end{pgfscope}%
\begin{pgfscope}%
\pgfpathrectangle{\pgfqpoint{1.254980in}{0.150000in}}{\pgfqpoint{5.490039in}{5.490039in}}%
\pgfusepath{clip}%
\pgfsetbuttcap%
\pgfsetroundjoin%
\definecolor{currentfill}{rgb}{0.206756,0.371758,0.553117}%
\pgfsetfillcolor{currentfill}%
\pgfsetfillopacity{0.700000}%
\pgfsetlinewidth{0.000000pt}%
\definecolor{currentstroke}{rgb}{0.000000,0.000000,0.000000}%
\pgfsetstrokecolor{currentstroke}%
\pgfsetdash{}{0pt}%
\pgfpathmoveto{\pgfqpoint{2.653501in}{2.851182in}}%
\pgfpathlineto{\pgfqpoint{2.666728in}{2.831398in}}%
\pgfpathlineto{\pgfqpoint{2.679946in}{2.811919in}}%
\pgfpathlineto{\pgfqpoint{2.693153in}{2.792743in}}%
\pgfpathlineto{\pgfqpoint{2.706350in}{2.773866in}}%
\pgfpathlineto{\pgfqpoint{2.714409in}{2.781283in}}%
\pgfpathlineto{\pgfqpoint{2.722459in}{2.788834in}}%
\pgfpathlineto{\pgfqpoint{2.730499in}{2.796518in}}%
\pgfpathlineto{\pgfqpoint{2.738530in}{2.804335in}}%
\pgfpathlineto{\pgfqpoint{2.725357in}{2.823094in}}%
\pgfpathlineto{\pgfqpoint{2.712174in}{2.842152in}}%
\pgfpathlineto{\pgfqpoint{2.698981in}{2.861511in}}%
\pgfpathlineto{\pgfqpoint{2.685778in}{2.881176in}}%
\pgfpathlineto{\pgfqpoint{2.677723in}{2.873467in}}%
\pgfpathlineto{\pgfqpoint{2.669659in}{2.865897in}}%
\pgfpathlineto{\pgfqpoint{2.661585in}{2.858469in}}%
\pgfpathlineto{\pgfqpoint{2.653501in}{2.851182in}}%
\pgfpathclose%
\pgfusepath{fill}%
\end{pgfscope}%
\begin{pgfscope}%
\pgfpathrectangle{\pgfqpoint{1.254980in}{0.150000in}}{\pgfqpoint{5.490039in}{5.490039in}}%
\pgfusepath{clip}%
\pgfsetbuttcap%
\pgfsetroundjoin%
\definecolor{currentfill}{rgb}{0.282327,0.094955,0.417331}%
\pgfsetfillcolor{currentfill}%
\pgfsetfillopacity{0.700000}%
\pgfsetlinewidth{0.000000pt}%
\definecolor{currentstroke}{rgb}{0.000000,0.000000,0.000000}%
\pgfsetstrokecolor{currentstroke}%
\pgfsetdash{}{0pt}%
\pgfpathmoveto{\pgfqpoint{3.395593in}{2.245715in}}%
\pgfpathlineto{\pgfqpoint{3.408575in}{2.238982in}}%
\pgfpathlineto{\pgfqpoint{3.421559in}{2.232452in}}%
\pgfpathlineto{\pgfqpoint{3.434545in}{2.226125in}}%
\pgfpathlineto{\pgfqpoint{3.447533in}{2.219998in}}%
\pgfpathlineto{\pgfqpoint{3.455304in}{2.229333in}}%
\pgfpathlineto{\pgfqpoint{3.463068in}{2.238708in}}%
\pgfpathlineto{\pgfqpoint{3.470827in}{2.248121in}}%
\pgfpathlineto{\pgfqpoint{3.478581in}{2.257574in}}%
\pgfpathlineto{\pgfqpoint{3.465605in}{2.263649in}}%
\pgfpathlineto{\pgfqpoint{3.452631in}{2.269924in}}%
\pgfpathlineto{\pgfqpoint{3.439659in}{2.276401in}}%
\pgfpathlineto{\pgfqpoint{3.426689in}{2.283081in}}%
\pgfpathlineto{\pgfqpoint{3.418924in}{2.273670in}}%
\pgfpathlineto{\pgfqpoint{3.411152in}{2.264306in}}%
\pgfpathlineto{\pgfqpoint{3.403375in}{2.254987in}}%
\pgfpathlineto{\pgfqpoint{3.395593in}{2.245715in}}%
\pgfpathclose%
\pgfusepath{fill}%
\end{pgfscope}%
\begin{pgfscope}%
\pgfpathrectangle{\pgfqpoint{1.254980in}{0.150000in}}{\pgfqpoint{5.490039in}{5.490039in}}%
\pgfusepath{clip}%
\pgfsetbuttcap%
\pgfsetroundjoin%
\definecolor{currentfill}{rgb}{0.283187,0.125848,0.444960}%
\pgfsetfillcolor{currentfill}%
\pgfsetfillopacity{0.700000}%
\pgfsetlinewidth{0.000000pt}%
\definecolor{currentstroke}{rgb}{0.000000,0.000000,0.000000}%
\pgfsetstrokecolor{currentstroke}%
\pgfsetdash{}{0pt}%
\pgfpathmoveto{\pgfqpoint{3.830805in}{2.293091in}}%
\pgfpathlineto{\pgfqpoint{3.843836in}{2.290999in}}%
\pgfpathlineto{\pgfqpoint{3.856874in}{2.289088in}}%
\pgfpathlineto{\pgfqpoint{3.869917in}{2.287358in}}%
\pgfpathlineto{\pgfqpoint{3.882966in}{2.285807in}}%
\pgfpathlineto{\pgfqpoint{3.890592in}{2.295507in}}%
\pgfpathlineto{\pgfqpoint{3.898212in}{2.305211in}}%
\pgfpathlineto{\pgfqpoint{3.905828in}{2.314921in}}%
\pgfpathlineto{\pgfqpoint{3.913438in}{2.324638in}}%
\pgfpathlineto{\pgfqpoint{3.900397in}{2.326248in}}%
\pgfpathlineto{\pgfqpoint{3.887363in}{2.328038in}}%
\pgfpathlineto{\pgfqpoint{3.874334in}{2.330007in}}%
\pgfpathlineto{\pgfqpoint{3.861312in}{2.332158in}}%
\pgfpathlineto{\pgfqpoint{3.853692in}{2.322372in}}%
\pgfpathlineto{\pgfqpoint{3.846068in}{2.312599in}}%
\pgfpathlineto{\pgfqpoint{3.838439in}{2.302839in}}%
\pgfpathlineto{\pgfqpoint{3.830805in}{2.293091in}}%
\pgfpathclose%
\pgfusepath{fill}%
\end{pgfscope}%
\begin{pgfscope}%
\pgfpathrectangle{\pgfqpoint{1.254980in}{0.150000in}}{\pgfqpoint{5.490039in}{5.490039in}}%
\pgfusepath{clip}%
\pgfsetbuttcap%
\pgfsetroundjoin%
\definecolor{currentfill}{rgb}{0.136408,0.541173,0.554483}%
\pgfsetfillcolor{currentfill}%
\pgfsetfillopacity{0.700000}%
\pgfsetlinewidth{0.000000pt}%
\definecolor{currentstroke}{rgb}{0.000000,0.000000,0.000000}%
\pgfsetstrokecolor{currentstroke}%
\pgfsetdash{}{0pt}%
\pgfpathmoveto{\pgfqpoint{5.647297in}{3.248675in}}%
\pgfpathlineto{\pgfqpoint{5.660978in}{3.252939in}}%
\pgfpathlineto{\pgfqpoint{5.674673in}{3.257351in}}%
\pgfpathlineto{\pgfqpoint{5.688383in}{3.261911in}}%
\pgfpathlineto{\pgfqpoint{5.702109in}{3.266619in}}%
\pgfpathlineto{\pgfqpoint{5.709085in}{3.274151in}}%
\pgfpathlineto{\pgfqpoint{5.716063in}{3.281910in}}%
\pgfpathlineto{\pgfqpoint{5.723044in}{3.289904in}}%
\pgfpathlineto{\pgfqpoint{5.709341in}{3.285704in}}%
\pgfpathlineto{\pgfqpoint{5.695654in}{3.281652in}}%
\pgfpathlineto{\pgfqpoint{5.681981in}{3.277747in}}%
\pgfpathlineto{\pgfqpoint{5.668322in}{3.273990in}}%
\pgfpathlineto{\pgfqpoint{5.661311in}{3.265313in}}%
\pgfpathlineto{\pgfqpoint{5.654303in}{3.256877in}}%
\pgfpathlineto{\pgfqpoint{5.647297in}{3.248675in}}%
\pgfpathclose%
\pgfusepath{fill}%
\end{pgfscope}%
\begin{pgfscope}%
\pgfpathrectangle{\pgfqpoint{1.254980in}{0.150000in}}{\pgfqpoint{5.490039in}{5.490039in}}%
\pgfusepath{clip}%
\pgfsetbuttcap%
\pgfsetroundjoin%
\definecolor{currentfill}{rgb}{0.283091,0.110553,0.431554}%
\pgfsetfillcolor{currentfill}%
\pgfsetfillopacity{0.700000}%
\pgfsetlinewidth{0.000000pt}%
\definecolor{currentstroke}{rgb}{0.000000,0.000000,0.000000}%
\pgfsetstrokecolor{currentstroke}%
\pgfsetdash{}{0pt}%
\pgfpathmoveto{\pgfqpoint{3.260468in}{2.271065in}}%
\pgfpathlineto{\pgfqpoint{3.273458in}{2.262581in}}%
\pgfpathlineto{\pgfqpoint{3.286448in}{2.254311in}}%
\pgfpathlineto{\pgfqpoint{3.299439in}{2.246252in}}%
\pgfpathlineto{\pgfqpoint{3.312430in}{2.238404in}}%
\pgfpathlineto{\pgfqpoint{3.320250in}{2.247398in}}%
\pgfpathlineto{\pgfqpoint{3.328065in}{2.256445in}}%
\pgfpathlineto{\pgfqpoint{3.335873in}{2.265547in}}%
\pgfpathlineto{\pgfqpoint{3.343675in}{2.274703in}}%
\pgfpathlineto{\pgfqpoint{3.330698in}{2.282470in}}%
\pgfpathlineto{\pgfqpoint{3.317722in}{2.290448in}}%
\pgfpathlineto{\pgfqpoint{3.304746in}{2.298637in}}%
\pgfpathlineto{\pgfqpoint{3.291770in}{2.307040in}}%
\pgfpathlineto{\pgfqpoint{3.283954in}{2.297955in}}%
\pgfpathlineto{\pgfqpoint{3.276132in}{2.288931in}}%
\pgfpathlineto{\pgfqpoint{3.268303in}{2.279967in}}%
\pgfpathlineto{\pgfqpoint{3.260468in}{2.271065in}}%
\pgfpathclose%
\pgfusepath{fill}%
\end{pgfscope}%
\begin{pgfscope}%
\pgfpathrectangle{\pgfqpoint{1.254980in}{0.150000in}}{\pgfqpoint{5.490039in}{5.490039in}}%
\pgfusepath{clip}%
\pgfsetbuttcap%
\pgfsetroundjoin%
\definecolor{currentfill}{rgb}{0.281887,0.150881,0.465405}%
\pgfsetfillcolor{currentfill}%
\pgfsetfillopacity{0.700000}%
\pgfsetlinewidth{0.000000pt}%
\definecolor{currentstroke}{rgb}{0.000000,0.000000,0.000000}%
\pgfsetstrokecolor{currentstroke}%
\pgfsetdash{}{0pt}%
\pgfpathmoveto{\pgfqpoint{3.072915in}{2.356372in}}%
\pgfpathlineto{\pgfqpoint{3.085937in}{2.345103in}}%
\pgfpathlineto{\pgfqpoint{3.098957in}{2.334066in}}%
\pgfpathlineto{\pgfqpoint{3.111975in}{2.323259in}}%
\pgfpathlineto{\pgfqpoint{3.124990in}{2.312680in}}%
\pgfpathlineto{\pgfqpoint{3.132884in}{2.321091in}}%
\pgfpathlineto{\pgfqpoint{3.140771in}{2.329579in}}%
\pgfpathlineto{\pgfqpoint{3.148651in}{2.338145in}}%
\pgfpathlineto{\pgfqpoint{3.156525in}{2.346787in}}%
\pgfpathlineto{\pgfqpoint{3.143526in}{2.357256in}}%
\pgfpathlineto{\pgfqpoint{3.130526in}{2.367953in}}%
\pgfpathlineto{\pgfqpoint{3.117524in}{2.378879in}}%
\pgfpathlineto{\pgfqpoint{3.104519in}{2.390038in}}%
\pgfpathlineto{\pgfqpoint{3.096629in}{2.381496in}}%
\pgfpathlineto{\pgfqpoint{3.088731in}{2.373037in}}%
\pgfpathlineto{\pgfqpoint{3.080827in}{2.364662in}}%
\pgfpathlineto{\pgfqpoint{3.072915in}{2.356372in}}%
\pgfpathclose%
\pgfusepath{fill}%
\end{pgfscope}%
\begin{pgfscope}%
\pgfpathrectangle{\pgfqpoint{1.254980in}{0.150000in}}{\pgfqpoint{5.490039in}{5.490039in}}%
\pgfusepath{clip}%
\pgfsetbuttcap%
\pgfsetroundjoin%
\definecolor{currentfill}{rgb}{0.282327,0.094955,0.417331}%
\pgfsetfillcolor{currentfill}%
\pgfsetfillopacity{0.700000}%
\pgfsetlinewidth{0.000000pt}%
\definecolor{currentstroke}{rgb}{0.000000,0.000000,0.000000}%
\pgfsetstrokecolor{currentstroke}%
\pgfsetdash{}{0pt}%
\pgfpathmoveto{\pgfqpoint{3.530509in}{2.235264in}}%
\pgfpathlineto{\pgfqpoint{3.543498in}{2.230179in}}%
\pgfpathlineto{\pgfqpoint{3.556490in}{2.225287in}}%
\pgfpathlineto{\pgfqpoint{3.569486in}{2.220590in}}%
\pgfpathlineto{\pgfqpoint{3.582485in}{2.216085in}}%
\pgfpathlineto{\pgfqpoint{3.590210in}{2.225646in}}%
\pgfpathlineto{\pgfqpoint{3.597930in}{2.235234in}}%
\pgfpathlineto{\pgfqpoint{3.605644in}{2.244847in}}%
\pgfpathlineto{\pgfqpoint{3.613353in}{2.254485in}}%
\pgfpathlineto{\pgfqpoint{3.600365in}{2.258966in}}%
\pgfpathlineto{\pgfqpoint{3.587380in}{2.263640in}}%
\pgfpathlineto{\pgfqpoint{3.574399in}{2.268506in}}%
\pgfpathlineto{\pgfqpoint{3.561421in}{2.273568in}}%
\pgfpathlineto{\pgfqpoint{3.553701in}{2.263943in}}%
\pgfpathlineto{\pgfqpoint{3.545976in}{2.254351in}}%
\pgfpathlineto{\pgfqpoint{3.538245in}{2.244792in}}%
\pgfpathlineto{\pgfqpoint{3.530509in}{2.235264in}}%
\pgfpathclose%
\pgfusepath{fill}%
\end{pgfscope}%
\begin{pgfscope}%
\pgfpathrectangle{\pgfqpoint{1.254980in}{0.150000in}}{\pgfqpoint{5.490039in}{5.490039in}}%
\pgfusepath{clip}%
\pgfsetbuttcap%
\pgfsetroundjoin%
\definecolor{currentfill}{rgb}{0.283197,0.115680,0.436115}%
\pgfsetfillcolor{currentfill}%
\pgfsetfillopacity{0.700000}%
\pgfsetlinewidth{0.000000pt}%
\definecolor{currentstroke}{rgb}{0.000000,0.000000,0.000000}%
\pgfsetstrokecolor{currentstroke}%
\pgfsetdash{}{0pt}%
\pgfpathmoveto{\pgfqpoint{3.748113in}{2.264257in}}%
\pgfpathlineto{\pgfqpoint{3.761131in}{2.261466in}}%
\pgfpathlineto{\pgfqpoint{3.774155in}{2.258859in}}%
\pgfpathlineto{\pgfqpoint{3.787184in}{2.256434in}}%
\pgfpathlineto{\pgfqpoint{3.800218in}{2.254193in}}%
\pgfpathlineto{\pgfqpoint{3.807872in}{2.263904in}}%
\pgfpathlineto{\pgfqpoint{3.815522in}{2.273624in}}%
\pgfpathlineto{\pgfqpoint{3.823166in}{2.283352in}}%
\pgfpathlineto{\pgfqpoint{3.830805in}{2.293091in}}%
\pgfpathlineto{\pgfqpoint{3.817780in}{2.295364in}}%
\pgfpathlineto{\pgfqpoint{3.804760in}{2.297819in}}%
\pgfpathlineto{\pgfqpoint{3.791745in}{2.300458in}}%
\pgfpathlineto{\pgfqpoint{3.778736in}{2.303281in}}%
\pgfpathlineto{\pgfqpoint{3.771088in}{2.293501in}}%
\pgfpathlineto{\pgfqpoint{3.763435in}{2.283737in}}%
\pgfpathlineto{\pgfqpoint{3.755776in}{2.273990in}}%
\pgfpathlineto{\pgfqpoint{3.748113in}{2.264257in}}%
\pgfpathclose%
\pgfusepath{fill}%
\end{pgfscope}%
\begin{pgfscope}%
\pgfpathrectangle{\pgfqpoint{1.254980in}{0.150000in}}{\pgfqpoint{5.490039in}{5.490039in}}%
\pgfusepath{clip}%
\pgfsetbuttcap%
\pgfsetroundjoin%
\definecolor{currentfill}{rgb}{0.283072,0.130895,0.449241}%
\pgfsetfillcolor{currentfill}%
\pgfsetfillopacity{0.700000}%
\pgfsetlinewidth{0.000000pt}%
\definecolor{currentstroke}{rgb}{0.000000,0.000000,0.000000}%
\pgfsetstrokecolor{currentstroke}%
\pgfsetdash{}{0pt}%
\pgfpathmoveto{\pgfqpoint{3.124990in}{2.312680in}}%
\pgfpathlineto{\pgfqpoint{3.138003in}{2.302328in}}%
\pgfpathlineto{\pgfqpoint{3.151015in}{2.292201in}}%
\pgfpathlineto{\pgfqpoint{3.164026in}{2.282297in}}%
\pgfpathlineto{\pgfqpoint{3.177035in}{2.272616in}}%
\pgfpathlineto{\pgfqpoint{3.184912in}{2.281147in}}%
\pgfpathlineto{\pgfqpoint{3.192783in}{2.289749in}}%
\pgfpathlineto{\pgfqpoint{3.200646in}{2.298421in}}%
\pgfpathlineto{\pgfqpoint{3.208503in}{2.307162in}}%
\pgfpathlineto{\pgfqpoint{3.195511in}{2.316734in}}%
\pgfpathlineto{\pgfqpoint{3.182517in}{2.326528in}}%
\pgfpathlineto{\pgfqpoint{3.169521in}{2.336545in}}%
\pgfpathlineto{\pgfqpoint{3.156525in}{2.346787in}}%
\pgfpathlineto{\pgfqpoint{3.148651in}{2.338145in}}%
\pgfpathlineto{\pgfqpoint{3.140771in}{2.329579in}}%
\pgfpathlineto{\pgfqpoint{3.132884in}{2.321091in}}%
\pgfpathlineto{\pgfqpoint{3.124990in}{2.312680in}}%
\pgfpathclose%
\pgfusepath{fill}%
\end{pgfscope}%
\begin{pgfscope}%
\pgfpathrectangle{\pgfqpoint{1.254980in}{0.150000in}}{\pgfqpoint{5.490039in}{5.490039in}}%
\pgfusepath{clip}%
\pgfsetbuttcap%
\pgfsetroundjoin%
\definecolor{currentfill}{rgb}{0.252194,0.269783,0.531579}%
\pgfsetfillcolor{currentfill}%
\pgfsetfillopacity{0.700000}%
\pgfsetlinewidth{0.000000pt}%
\definecolor{currentstroke}{rgb}{0.000000,0.000000,0.000000}%
\pgfsetstrokecolor{currentstroke}%
\pgfsetdash{}{0pt}%
\pgfpathmoveto{\pgfqpoint{4.461473in}{2.562955in}}%
\pgfpathlineto{\pgfqpoint{4.474707in}{2.565355in}}%
\pgfpathlineto{\pgfqpoint{4.487950in}{2.567918in}}%
\pgfpathlineto{\pgfqpoint{4.501205in}{2.570644in}}%
\pgfpathlineto{\pgfqpoint{4.514470in}{2.573534in}}%
\pgfpathlineto{\pgfqpoint{4.521885in}{2.582036in}}%
\pgfpathlineto{\pgfqpoint{4.529296in}{2.590544in}}%
\pgfpathlineto{\pgfqpoint{4.536702in}{2.599061in}}%
\pgfpathlineto{\pgfqpoint{4.544103in}{2.607590in}}%
\pgfpathlineto{\pgfqpoint{4.530848in}{2.604957in}}%
\pgfpathlineto{\pgfqpoint{4.517604in}{2.602487in}}%
\pgfpathlineto{\pgfqpoint{4.504370in}{2.600180in}}%
\pgfpathlineto{\pgfqpoint{4.491147in}{2.598037in}}%
\pgfpathlineto{\pgfqpoint{4.483736in}{2.589241in}}%
\pgfpathlineto{\pgfqpoint{4.476320in}{2.580465in}}%
\pgfpathlineto{\pgfqpoint{4.468899in}{2.571704in}}%
\pgfpathlineto{\pgfqpoint{4.461473in}{2.562955in}}%
\pgfpathclose%
\pgfusepath{fill}%
\end{pgfscope}%
\begin{pgfscope}%
\pgfpathrectangle{\pgfqpoint{1.254980in}{0.150000in}}{\pgfqpoint{5.490039in}{5.490039in}}%
\pgfusepath{clip}%
\pgfsetbuttcap%
\pgfsetroundjoin%
\definecolor{currentfill}{rgb}{0.260571,0.246922,0.522828}%
\pgfsetfillcolor{currentfill}%
\pgfsetfillopacity{0.700000}%
\pgfsetlinewidth{0.000000pt}%
\definecolor{currentstroke}{rgb}{0.000000,0.000000,0.000000}%
\pgfsetstrokecolor{currentstroke}%
\pgfsetdash{}{0pt}%
\pgfpathmoveto{\pgfqpoint{4.378850in}{2.519164in}}%
\pgfpathlineto{\pgfqpoint{4.392052in}{2.521132in}}%
\pgfpathlineto{\pgfqpoint{4.405265in}{2.523266in}}%
\pgfpathlineto{\pgfqpoint{4.418487in}{2.525565in}}%
\pgfpathlineto{\pgfqpoint{4.431720in}{2.528028in}}%
\pgfpathlineto{\pgfqpoint{4.439166in}{2.536756in}}%
\pgfpathlineto{\pgfqpoint{4.446607in}{2.545484in}}%
\pgfpathlineto{\pgfqpoint{4.454042in}{2.554216in}}%
\pgfpathlineto{\pgfqpoint{4.461473in}{2.562955in}}%
\pgfpathlineto{\pgfqpoint{4.448250in}{2.560720in}}%
\pgfpathlineto{\pgfqpoint{4.435037in}{2.558650in}}%
\pgfpathlineto{\pgfqpoint{4.421834in}{2.556744in}}%
\pgfpathlineto{\pgfqpoint{4.408641in}{2.555003in}}%
\pgfpathlineto{\pgfqpoint{4.401201in}{2.546026in}}%
\pgfpathlineto{\pgfqpoint{4.393756in}{2.537062in}}%
\pgfpathlineto{\pgfqpoint{4.386305in}{2.528109in}}%
\pgfpathlineto{\pgfqpoint{4.378850in}{2.519164in}}%
\pgfpathclose%
\pgfusepath{fill}%
\end{pgfscope}%
\begin{pgfscope}%
\pgfpathrectangle{\pgfqpoint{1.254980in}{0.150000in}}{\pgfqpoint{5.490039in}{5.490039in}}%
\pgfusepath{clip}%
\pgfsetbuttcap%
\pgfsetroundjoin%
\definecolor{currentfill}{rgb}{0.244972,0.287675,0.537260}%
\pgfsetfillcolor{currentfill}%
\pgfsetfillopacity{0.700000}%
\pgfsetlinewidth{0.000000pt}%
\definecolor{currentstroke}{rgb}{0.000000,0.000000,0.000000}%
\pgfsetstrokecolor{currentstroke}%
\pgfsetdash{}{0pt}%
\pgfpathmoveto{\pgfqpoint{4.544103in}{2.607590in}}%
\pgfpathlineto{\pgfqpoint{4.557368in}{2.610385in}}%
\pgfpathlineto{\pgfqpoint{4.570644in}{2.613343in}}%
\pgfpathlineto{\pgfqpoint{4.583932in}{2.616463in}}%
\pgfpathlineto{\pgfqpoint{4.597230in}{2.619744in}}%
\pgfpathlineto{\pgfqpoint{4.604615in}{2.628014in}}%
\pgfpathlineto{\pgfqpoint{4.611995in}{2.636294in}}%
\pgfpathlineto{\pgfqpoint{4.619370in}{2.644590in}}%
\pgfpathlineto{\pgfqpoint{4.626741in}{2.652904in}}%
\pgfpathlineto{\pgfqpoint{4.613453in}{2.649908in}}%
\pgfpathlineto{\pgfqpoint{4.600177in}{2.647073in}}%
\pgfpathlineto{\pgfqpoint{4.586912in}{2.644400in}}%
\pgfpathlineto{\pgfqpoint{4.573657in}{2.641889in}}%
\pgfpathlineto{\pgfqpoint{4.566276in}{2.633280in}}%
\pgfpathlineto{\pgfqpoint{4.558889in}{2.624696in}}%
\pgfpathlineto{\pgfqpoint{4.551499in}{2.616133in}}%
\pgfpathlineto{\pgfqpoint{4.544103in}{2.607590in}}%
\pgfpathclose%
\pgfusepath{fill}%
\end{pgfscope}%
\begin{pgfscope}%
\pgfpathrectangle{\pgfqpoint{1.254980in}{0.150000in}}{\pgfqpoint{5.490039in}{5.490039in}}%
\pgfusepath{clip}%
\pgfsetbuttcap%
\pgfsetroundjoin%
\definecolor{currentfill}{rgb}{0.266580,0.228262,0.514349}%
\pgfsetfillcolor{currentfill}%
\pgfsetfillopacity{0.700000}%
\pgfsetlinewidth{0.000000pt}%
\definecolor{currentstroke}{rgb}{0.000000,0.000000,0.000000}%
\pgfsetstrokecolor{currentstroke}%
\pgfsetdash{}{0pt}%
\pgfpathmoveto{\pgfqpoint{4.296230in}{2.476399in}}%
\pgfpathlineto{\pgfqpoint{4.309403in}{2.477901in}}%
\pgfpathlineto{\pgfqpoint{4.322585in}{2.479571in}}%
\pgfpathlineto{\pgfqpoint{4.335777in}{2.481407in}}%
\pgfpathlineto{\pgfqpoint{4.348979in}{2.483409in}}%
\pgfpathlineto{\pgfqpoint{4.356454in}{2.492349in}}%
\pgfpathlineto{\pgfqpoint{4.363924in}{2.501287in}}%
\pgfpathlineto{\pgfqpoint{4.371390in}{2.510224in}}%
\pgfpathlineto{\pgfqpoint{4.378850in}{2.519164in}}%
\pgfpathlineto{\pgfqpoint{4.365657in}{2.517362in}}%
\pgfpathlineto{\pgfqpoint{4.352474in}{2.515725in}}%
\pgfpathlineto{\pgfqpoint{4.339301in}{2.514256in}}%
\pgfpathlineto{\pgfqpoint{4.326137in}{2.512953in}}%
\pgfpathlineto{\pgfqpoint{4.318667in}{2.503803in}}%
\pgfpathlineto{\pgfqpoint{4.311193in}{2.494662in}}%
\pgfpathlineto{\pgfqpoint{4.303714in}{2.485528in}}%
\pgfpathlineto{\pgfqpoint{4.296230in}{2.476399in}}%
\pgfpathclose%
\pgfusepath{fill}%
\end{pgfscope}%
\begin{pgfscope}%
\pgfpathrectangle{\pgfqpoint{1.254980in}{0.150000in}}{\pgfqpoint{5.490039in}{5.490039in}}%
\pgfusepath{clip}%
\pgfsetbuttcap%
\pgfsetroundjoin%
\definecolor{currentfill}{rgb}{0.235526,0.309527,0.542944}%
\pgfsetfillcolor{currentfill}%
\pgfsetfillopacity{0.700000}%
\pgfsetlinewidth{0.000000pt}%
\definecolor{currentstroke}{rgb}{0.000000,0.000000,0.000000}%
\pgfsetstrokecolor{currentstroke}%
\pgfsetdash{}{0pt}%
\pgfpathmoveto{\pgfqpoint{4.626741in}{2.652904in}}%
\pgfpathlineto{\pgfqpoint{4.640039in}{2.656061in}}%
\pgfpathlineto{\pgfqpoint{4.653349in}{2.659380in}}%
\pgfpathlineto{\pgfqpoint{4.666670in}{2.662859in}}%
\pgfpathlineto{\pgfqpoint{4.680003in}{2.666498in}}%
\pgfpathlineto{\pgfqpoint{4.687357in}{2.674531in}}%
\pgfpathlineto{\pgfqpoint{4.694706in}{2.682582in}}%
\pgfpathlineto{\pgfqpoint{4.702049in}{2.690656in}}%
\pgfpathlineto{\pgfqpoint{4.709389in}{2.698756in}}%
\pgfpathlineto{\pgfqpoint{4.696068in}{2.695430in}}%
\pgfpathlineto{\pgfqpoint{4.682758in}{2.692265in}}%
\pgfpathlineto{\pgfqpoint{4.669460in}{2.689260in}}%
\pgfpathlineto{\pgfqpoint{4.656173in}{2.686415in}}%
\pgfpathlineto{\pgfqpoint{4.648822in}{2.677992in}}%
\pgfpathlineto{\pgfqpoint{4.641466in}{2.669601in}}%
\pgfpathlineto{\pgfqpoint{4.634106in}{2.661240in}}%
\pgfpathlineto{\pgfqpoint{4.626741in}{2.652904in}}%
\pgfpathclose%
\pgfusepath{fill}%
\end{pgfscope}%
\begin{pgfscope}%
\pgfpathrectangle{\pgfqpoint{1.254980in}{0.150000in}}{\pgfqpoint{5.490039in}{5.490039in}}%
\pgfusepath{clip}%
\pgfsetbuttcap%
\pgfsetroundjoin%
\definecolor{currentfill}{rgb}{0.271828,0.209303,0.504434}%
\pgfsetfillcolor{currentfill}%
\pgfsetfillopacity{0.700000}%
\pgfsetlinewidth{0.000000pt}%
\definecolor{currentstroke}{rgb}{0.000000,0.000000,0.000000}%
\pgfsetstrokecolor{currentstroke}%
\pgfsetdash{}{0pt}%
\pgfpathmoveto{\pgfqpoint{4.213608in}{2.434864in}}%
\pgfpathlineto{\pgfqpoint{4.226753in}{2.435864in}}%
\pgfpathlineto{\pgfqpoint{4.239907in}{2.437034in}}%
\pgfpathlineto{\pgfqpoint{4.253070in}{2.438372in}}%
\pgfpathlineto{\pgfqpoint{4.266243in}{2.439879in}}%
\pgfpathlineto{\pgfqpoint{4.273747in}{2.449014in}}%
\pgfpathlineto{\pgfqpoint{4.281246in}{2.458144in}}%
\pgfpathlineto{\pgfqpoint{4.288740in}{2.467272in}}%
\pgfpathlineto{\pgfqpoint{4.296230in}{2.476399in}}%
\pgfpathlineto{\pgfqpoint{4.283066in}{2.475064in}}%
\pgfpathlineto{\pgfqpoint{4.269911in}{2.473898in}}%
\pgfpathlineto{\pgfqpoint{4.256766in}{2.472900in}}%
\pgfpathlineto{\pgfqpoint{4.243629in}{2.472071in}}%
\pgfpathlineto{\pgfqpoint{4.236131in}{2.462761in}}%
\pgfpathlineto{\pgfqpoint{4.228628in}{2.453459in}}%
\pgfpathlineto{\pgfqpoint{4.221120in}{2.444160in}}%
\pgfpathlineto{\pgfqpoint{4.213608in}{2.434864in}}%
\pgfpathclose%
\pgfusepath{fill}%
\end{pgfscope}%
\begin{pgfscope}%
\pgfpathrectangle{\pgfqpoint{1.254980in}{0.150000in}}{\pgfqpoint{5.490039in}{5.490039in}}%
\pgfusepath{clip}%
\pgfsetbuttcap%
\pgfsetroundjoin%
\definecolor{currentfill}{rgb}{0.225863,0.330805,0.547314}%
\pgfsetfillcolor{currentfill}%
\pgfsetfillopacity{0.700000}%
\pgfsetlinewidth{0.000000pt}%
\definecolor{currentstroke}{rgb}{0.000000,0.000000,0.000000}%
\pgfsetstrokecolor{currentstroke}%
\pgfsetdash{}{0pt}%
\pgfpathmoveto{\pgfqpoint{4.709389in}{2.698756in}}%
\pgfpathlineto{\pgfqpoint{4.722721in}{2.702241in}}%
\pgfpathlineto{\pgfqpoint{4.736065in}{2.705886in}}%
\pgfpathlineto{\pgfqpoint{4.749421in}{2.709690in}}%
\pgfpathlineto{\pgfqpoint{4.762789in}{2.713654in}}%
\pgfpathlineto{\pgfqpoint{4.770111in}{2.721452in}}%
\pgfpathlineto{\pgfqpoint{4.777428in}{2.729276in}}%
\pgfpathlineto{\pgfqpoint{4.784740in}{2.737132in}}%
\pgfpathlineto{\pgfqpoint{4.792047in}{2.745022in}}%
\pgfpathlineto{\pgfqpoint{4.778692in}{2.741401in}}%
\pgfpathlineto{\pgfqpoint{4.765349in}{2.737939in}}%
\pgfpathlineto{\pgfqpoint{4.752018in}{2.734636in}}%
\pgfpathlineto{\pgfqpoint{4.738698in}{2.731492in}}%
\pgfpathlineto{\pgfqpoint{4.731377in}{2.723250in}}%
\pgfpathlineto{\pgfqpoint{4.724052in}{2.715049in}}%
\pgfpathlineto{\pgfqpoint{4.716723in}{2.706885in}}%
\pgfpathlineto{\pgfqpoint{4.709389in}{2.698756in}}%
\pgfpathclose%
\pgfusepath{fill}%
\end{pgfscope}%
\begin{pgfscope}%
\pgfpathrectangle{\pgfqpoint{1.254980in}{0.150000in}}{\pgfqpoint{5.490039in}{5.490039in}}%
\pgfusepath{clip}%
\pgfsetbuttcap%
\pgfsetroundjoin%
\definecolor{currentfill}{rgb}{0.282656,0.100196,0.422160}%
\pgfsetfillcolor{currentfill}%
\pgfsetfillopacity{0.700000}%
\pgfsetlinewidth{0.000000pt}%
\definecolor{currentstroke}{rgb}{0.000000,0.000000,0.000000}%
\pgfsetstrokecolor{currentstroke}%
\pgfsetdash{}{0pt}%
\pgfpathmoveto{\pgfqpoint{3.665346in}{2.238470in}}%
\pgfpathlineto{\pgfqpoint{3.678354in}{2.234939in}}%
\pgfpathlineto{\pgfqpoint{3.691367in}{2.231594in}}%
\pgfpathlineto{\pgfqpoint{3.704385in}{2.228437in}}%
\pgfpathlineto{\pgfqpoint{3.717408in}{2.225465in}}%
\pgfpathlineto{\pgfqpoint{3.725092in}{2.235143in}}%
\pgfpathlineto{\pgfqpoint{3.732771in}{2.244834in}}%
\pgfpathlineto{\pgfqpoint{3.740444in}{2.254539in}}%
\pgfpathlineto{\pgfqpoint{3.748113in}{2.264257in}}%
\pgfpathlineto{\pgfqpoint{3.735100in}{2.267233in}}%
\pgfpathlineto{\pgfqpoint{3.722091in}{2.270395in}}%
\pgfpathlineto{\pgfqpoint{3.709088in}{2.273742in}}%
\pgfpathlineto{\pgfqpoint{3.696089in}{2.277278in}}%
\pgfpathlineto{\pgfqpoint{3.688411in}{2.267545in}}%
\pgfpathlineto{\pgfqpoint{3.680728in}{2.257833in}}%
\pgfpathlineto{\pgfqpoint{3.673039in}{2.248142in}}%
\pgfpathlineto{\pgfqpoint{3.665346in}{2.238470in}}%
\pgfpathclose%
\pgfusepath{fill}%
\end{pgfscope}%
\begin{pgfscope}%
\pgfpathrectangle{\pgfqpoint{1.254980in}{0.150000in}}{\pgfqpoint{5.490039in}{5.490039in}}%
\pgfusepath{clip}%
\pgfsetbuttcap%
\pgfsetroundjoin%
\definecolor{currentfill}{rgb}{0.216210,0.351535,0.550627}%
\pgfsetfillcolor{currentfill}%
\pgfsetfillopacity{0.700000}%
\pgfsetlinewidth{0.000000pt}%
\definecolor{currentstroke}{rgb}{0.000000,0.000000,0.000000}%
\pgfsetstrokecolor{currentstroke}%
\pgfsetdash{}{0pt}%
\pgfpathmoveto{\pgfqpoint{4.792047in}{2.745022in}}%
\pgfpathlineto{\pgfqpoint{4.805414in}{2.748802in}}%
\pgfpathlineto{\pgfqpoint{4.818794in}{2.752740in}}%
\pgfpathlineto{\pgfqpoint{4.832185in}{2.756836in}}%
\pgfpathlineto{\pgfqpoint{4.845589in}{2.761090in}}%
\pgfpathlineto{\pgfqpoint{4.852878in}{2.768659in}}%
\pgfpathlineto{\pgfqpoint{4.860162in}{2.776264in}}%
\pgfpathlineto{\pgfqpoint{4.867442in}{2.783910in}}%
\pgfpathlineto{\pgfqpoint{4.874717in}{2.791601in}}%
\pgfpathlineto{\pgfqpoint{4.861328in}{2.787719in}}%
\pgfpathlineto{\pgfqpoint{4.847950in}{2.783993in}}%
\pgfpathlineto{\pgfqpoint{4.834585in}{2.780426in}}%
\pgfpathlineto{\pgfqpoint{4.821232in}{2.777016in}}%
\pgfpathlineto{\pgfqpoint{4.813942in}{2.768944in}}%
\pgfpathlineto{\pgfqpoint{4.806648in}{2.760924in}}%
\pgfpathlineto{\pgfqpoint{4.799350in}{2.752952in}}%
\pgfpathlineto{\pgfqpoint{4.792047in}{2.745022in}}%
\pgfpathclose%
\pgfusepath{fill}%
\end{pgfscope}%
\begin{pgfscope}%
\pgfpathrectangle{\pgfqpoint{1.254980in}{0.150000in}}{\pgfqpoint{5.490039in}{5.490039in}}%
\pgfusepath{clip}%
\pgfsetbuttcap%
\pgfsetroundjoin%
\definecolor{currentfill}{rgb}{0.282656,0.100196,0.422160}%
\pgfsetfillcolor{currentfill}%
\pgfsetfillopacity{0.700000}%
\pgfsetlinewidth{0.000000pt}%
\definecolor{currentstroke}{rgb}{0.000000,0.000000,0.000000}%
\pgfsetstrokecolor{currentstroke}%
\pgfsetdash{}{0pt}%
\pgfpathmoveto{\pgfqpoint{3.312430in}{2.238404in}}%
\pgfpathlineto{\pgfqpoint{3.325421in}{2.230766in}}%
\pgfpathlineto{\pgfqpoint{3.338414in}{2.223336in}}%
\pgfpathlineto{\pgfqpoint{3.351407in}{2.216112in}}%
\pgfpathlineto{\pgfqpoint{3.364401in}{2.209095in}}%
\pgfpathlineto{\pgfqpoint{3.372208in}{2.218179in}}%
\pgfpathlineto{\pgfqpoint{3.380009in}{2.227310in}}%
\pgfpathlineto{\pgfqpoint{3.387804in}{2.236489in}}%
\pgfpathlineto{\pgfqpoint{3.395593in}{2.245715in}}%
\pgfpathlineto{\pgfqpoint{3.382611in}{2.252652in}}%
\pgfpathlineto{\pgfqpoint{3.369632in}{2.259795in}}%
\pgfpathlineto{\pgfqpoint{3.356653in}{2.267145in}}%
\pgfpathlineto{\pgfqpoint{3.343675in}{2.274703in}}%
\pgfpathlineto{\pgfqpoint{3.335873in}{2.265547in}}%
\pgfpathlineto{\pgfqpoint{3.328065in}{2.256445in}}%
\pgfpathlineto{\pgfqpoint{3.320250in}{2.247398in}}%
\pgfpathlineto{\pgfqpoint{3.312430in}{2.238404in}}%
\pgfpathclose%
\pgfusepath{fill}%
\end{pgfscope}%
\begin{pgfscope}%
\pgfpathrectangle{\pgfqpoint{1.254980in}{0.150000in}}{\pgfqpoint{5.490039in}{5.490039in}}%
\pgfusepath{clip}%
\pgfsetbuttcap%
\pgfsetroundjoin%
\definecolor{currentfill}{rgb}{0.277134,0.185228,0.489898}%
\pgfsetfillcolor{currentfill}%
\pgfsetfillopacity{0.700000}%
\pgfsetlinewidth{0.000000pt}%
\definecolor{currentstroke}{rgb}{0.000000,0.000000,0.000000}%
\pgfsetstrokecolor{currentstroke}%
\pgfsetdash{}{0pt}%
\pgfpathmoveto{\pgfqpoint{4.130979in}{2.394782in}}%
\pgfpathlineto{\pgfqpoint{4.144098in}{2.395245in}}%
\pgfpathlineto{\pgfqpoint{4.157226in}{2.395879in}}%
\pgfpathlineto{\pgfqpoint{4.170362in}{2.396684in}}%
\pgfpathlineto{\pgfqpoint{4.183507in}{2.397659in}}%
\pgfpathlineto{\pgfqpoint{4.191040in}{2.406967in}}%
\pgfpathlineto{\pgfqpoint{4.198567in}{2.416269in}}%
\pgfpathlineto{\pgfqpoint{4.206090in}{2.425567in}}%
\pgfpathlineto{\pgfqpoint{4.213608in}{2.434864in}}%
\pgfpathlineto{\pgfqpoint{4.200471in}{2.434033in}}%
\pgfpathlineto{\pgfqpoint{4.187343in}{2.433372in}}%
\pgfpathlineto{\pgfqpoint{4.174224in}{2.432881in}}%
\pgfpathlineto{\pgfqpoint{4.161113in}{2.432561in}}%
\pgfpathlineto{\pgfqpoint{4.153587in}{2.423111in}}%
\pgfpathlineto{\pgfqpoint{4.146056in}{2.413666in}}%
\pgfpathlineto{\pgfqpoint{4.138520in}{2.404223in}}%
\pgfpathlineto{\pgfqpoint{4.130979in}{2.394782in}}%
\pgfpathclose%
\pgfusepath{fill}%
\end{pgfscope}%
\begin{pgfscope}%
\pgfpathrectangle{\pgfqpoint{1.254980in}{0.150000in}}{\pgfqpoint{5.490039in}{5.490039in}}%
\pgfusepath{clip}%
\pgfsetbuttcap%
\pgfsetroundjoin%
\definecolor{currentfill}{rgb}{0.281924,0.089666,0.412415}%
\pgfsetfillcolor{currentfill}%
\pgfsetfillopacity{0.700000}%
\pgfsetlinewidth{0.000000pt}%
\definecolor{currentstroke}{rgb}{0.000000,0.000000,0.000000}%
\pgfsetstrokecolor{currentstroke}%
\pgfsetdash{}{0pt}%
\pgfpathmoveto{\pgfqpoint{3.447533in}{2.219998in}}%
\pgfpathlineto{\pgfqpoint{3.460524in}{2.214071in}}%
\pgfpathlineto{\pgfqpoint{3.473516in}{2.208344in}}%
\pgfpathlineto{\pgfqpoint{3.486511in}{2.202813in}}%
\pgfpathlineto{\pgfqpoint{3.499509in}{2.197480in}}%
\pgfpathlineto{\pgfqpoint{3.507267in}{2.206877in}}%
\pgfpathlineto{\pgfqpoint{3.515020in}{2.216307in}}%
\pgfpathlineto{\pgfqpoint{3.522767in}{2.225770in}}%
\pgfpathlineto{\pgfqpoint{3.530509in}{2.235264in}}%
\pgfpathlineto{\pgfqpoint{3.517523in}{2.240546in}}%
\pgfpathlineto{\pgfqpoint{3.504539in}{2.246024in}}%
\pgfpathlineto{\pgfqpoint{3.491559in}{2.251700in}}%
\pgfpathlineto{\pgfqpoint{3.478581in}{2.257574in}}%
\pgfpathlineto{\pgfqpoint{3.470827in}{2.248121in}}%
\pgfpathlineto{\pgfqpoint{3.463068in}{2.238708in}}%
\pgfpathlineto{\pgfqpoint{3.455304in}{2.229333in}}%
\pgfpathlineto{\pgfqpoint{3.447533in}{2.219998in}}%
\pgfpathclose%
\pgfusepath{fill}%
\end{pgfscope}%
\begin{pgfscope}%
\pgfpathrectangle{\pgfqpoint{1.254980in}{0.150000in}}{\pgfqpoint{5.490039in}{5.490039in}}%
\pgfusepath{clip}%
\pgfsetbuttcap%
\pgfsetroundjoin%
\definecolor{currentfill}{rgb}{0.206756,0.371758,0.553117}%
\pgfsetfillcolor{currentfill}%
\pgfsetfillopacity{0.700000}%
\pgfsetlinewidth{0.000000pt}%
\definecolor{currentstroke}{rgb}{0.000000,0.000000,0.000000}%
\pgfsetstrokecolor{currentstroke}%
\pgfsetdash{}{0pt}%
\pgfpathmoveto{\pgfqpoint{4.874717in}{2.791601in}}%
\pgfpathlineto{\pgfqpoint{4.888119in}{2.795642in}}%
\pgfpathlineto{\pgfqpoint{4.901534in}{2.799839in}}%
\pgfpathlineto{\pgfqpoint{4.914961in}{2.804193in}}%
\pgfpathlineto{\pgfqpoint{4.928401in}{2.808704in}}%
\pgfpathlineto{\pgfqpoint{4.935657in}{2.816056in}}%
\pgfpathlineto{\pgfqpoint{4.942909in}{2.823454in}}%
\pgfpathlineto{\pgfqpoint{4.950156in}{2.830904in}}%
\pgfpathlineto{\pgfqpoint{4.957399in}{2.838411in}}%
\pgfpathlineto{\pgfqpoint{4.943974in}{2.834299in}}%
\pgfpathlineto{\pgfqpoint{4.930562in}{2.830344in}}%
\pgfpathlineto{\pgfqpoint{4.917163in}{2.826546in}}%
\pgfpathlineto{\pgfqpoint{4.903776in}{2.822904in}}%
\pgfpathlineto{\pgfqpoint{4.896517in}{2.814988in}}%
\pgfpathlineto{\pgfqpoint{4.889255in}{2.807136in}}%
\pgfpathlineto{\pgfqpoint{4.881988in}{2.799342in}}%
\pgfpathlineto{\pgfqpoint{4.874717in}{2.791601in}}%
\pgfpathclose%
\pgfusepath{fill}%
\end{pgfscope}%
\begin{pgfscope}%
\pgfpathrectangle{\pgfqpoint{1.254980in}{0.150000in}}{\pgfqpoint{5.490039in}{5.490039in}}%
\pgfusepath{clip}%
\pgfsetbuttcap%
\pgfsetroundjoin%
\definecolor{currentfill}{rgb}{0.197636,0.391528,0.554969}%
\pgfsetfillcolor{currentfill}%
\pgfsetfillopacity{0.700000}%
\pgfsetlinewidth{0.000000pt}%
\definecolor{currentstroke}{rgb}{0.000000,0.000000,0.000000}%
\pgfsetstrokecolor{currentstroke}%
\pgfsetdash{}{0pt}%
\pgfpathmoveto{\pgfqpoint{4.957399in}{2.838411in}}%
\pgfpathlineto{\pgfqpoint{4.970836in}{2.842678in}}%
\pgfpathlineto{\pgfqpoint{4.984286in}{2.847102in}}%
\pgfpathlineto{\pgfqpoint{4.997749in}{2.851682in}}%
\pgfpathlineto{\pgfqpoint{5.011226in}{2.856417in}}%
\pgfpathlineto{\pgfqpoint{5.018448in}{2.863567in}}%
\pgfpathlineto{\pgfqpoint{5.025667in}{2.870775in}}%
\pgfpathlineto{\pgfqpoint{5.032881in}{2.878047in}}%
\pgfpathlineto{\pgfqpoint{5.040091in}{2.885389in}}%
\pgfpathlineto{\pgfqpoint{5.026632in}{2.881081in}}%
\pgfpathlineto{\pgfqpoint{5.013185in}{2.876930in}}%
\pgfpathlineto{\pgfqpoint{4.999751in}{2.872933in}}%
\pgfpathlineto{\pgfqpoint{4.986330in}{2.869092in}}%
\pgfpathlineto{\pgfqpoint{4.979103in}{2.861314in}}%
\pgfpathlineto{\pgfqpoint{4.971872in}{2.853611in}}%
\pgfpathlineto{\pgfqpoint{4.964637in}{2.845978in}}%
\pgfpathlineto{\pgfqpoint{4.957399in}{2.838411in}}%
\pgfpathclose%
\pgfusepath{fill}%
\end{pgfscope}%
\begin{pgfscope}%
\pgfpathrectangle{\pgfqpoint{1.254980in}{0.150000in}}{\pgfqpoint{5.490039in}{5.490039in}}%
\pgfusepath{clip}%
\pgfsetbuttcap%
\pgfsetroundjoin%
\definecolor{currentfill}{rgb}{0.279574,0.170599,0.479997}%
\pgfsetfillcolor{currentfill}%
\pgfsetfillopacity{0.700000}%
\pgfsetlinewidth{0.000000pt}%
\definecolor{currentstroke}{rgb}{0.000000,0.000000,0.000000}%
\pgfsetstrokecolor{currentstroke}%
\pgfsetdash{}{0pt}%
\pgfpathmoveto{\pgfqpoint{4.048335in}{2.356398in}}%
\pgfpathlineto{\pgfqpoint{4.061431in}{2.356287in}}%
\pgfpathlineto{\pgfqpoint{4.074534in}{2.356349in}}%
\pgfpathlineto{\pgfqpoint{4.087646in}{2.356584in}}%
\pgfpathlineto{\pgfqpoint{4.100765in}{2.356991in}}%
\pgfpathlineto{\pgfqpoint{4.108326in}{2.366446in}}%
\pgfpathlineto{\pgfqpoint{4.115882in}{2.375895in}}%
\pgfpathlineto{\pgfqpoint{4.123433in}{2.385340in}}%
\pgfpathlineto{\pgfqpoint{4.130979in}{2.394782in}}%
\pgfpathlineto{\pgfqpoint{4.117867in}{2.394491in}}%
\pgfpathlineto{\pgfqpoint{4.104764in}{2.394372in}}%
\pgfpathlineto{\pgfqpoint{4.091669in}{2.394425in}}%
\pgfpathlineto{\pgfqpoint{4.078581in}{2.394652in}}%
\pgfpathlineto{\pgfqpoint{4.071027in}{2.385083in}}%
\pgfpathlineto{\pgfqpoint{4.063468in}{2.375520in}}%
\pgfpathlineto{\pgfqpoint{4.055904in}{2.365959in}}%
\pgfpathlineto{\pgfqpoint{4.048335in}{2.356398in}}%
\pgfpathclose%
\pgfusepath{fill}%
\end{pgfscope}%
\begin{pgfscope}%
\pgfpathrectangle{\pgfqpoint{1.254980in}{0.150000in}}{\pgfqpoint{5.490039in}{5.490039in}}%
\pgfusepath{clip}%
\pgfsetbuttcap%
\pgfsetroundjoin%
\definecolor{currentfill}{rgb}{0.188923,0.410910,0.556326}%
\pgfsetfillcolor{currentfill}%
\pgfsetfillopacity{0.700000}%
\pgfsetlinewidth{0.000000pt}%
\definecolor{currentstroke}{rgb}{0.000000,0.000000,0.000000}%
\pgfsetstrokecolor{currentstroke}%
\pgfsetdash{}{0pt}%
\pgfpathmoveto{\pgfqpoint{5.040091in}{2.885389in}}%
\pgfpathlineto{\pgfqpoint{5.053564in}{2.889851in}}%
\pgfpathlineto{\pgfqpoint{5.067050in}{2.894468in}}%
\pgfpathlineto{\pgfqpoint{5.080549in}{2.899240in}}%
\pgfpathlineto{\pgfqpoint{5.094062in}{2.904167in}}%
\pgfpathlineto{\pgfqpoint{5.101251in}{2.911135in}}%
\pgfpathlineto{\pgfqpoint{5.108436in}{2.918176in}}%
\pgfpathlineto{\pgfqpoint{5.115617in}{2.925293in}}%
\pgfpathlineto{\pgfqpoint{5.122795in}{2.932494in}}%
\pgfpathlineto{\pgfqpoint{5.109300in}{2.928024in}}%
\pgfpathlineto{\pgfqpoint{5.095819in}{2.923708in}}%
\pgfpathlineto{\pgfqpoint{5.082351in}{2.919547in}}%
\pgfpathlineto{\pgfqpoint{5.068896in}{2.915540in}}%
\pgfpathlineto{\pgfqpoint{5.061700in}{2.907874in}}%
\pgfpathlineto{\pgfqpoint{5.054501in}{2.900297in}}%
\pgfpathlineto{\pgfqpoint{5.047298in}{2.892803in}}%
\pgfpathlineto{\pgfqpoint{5.040091in}{2.885389in}}%
\pgfpathclose%
\pgfusepath{fill}%
\end{pgfscope}%
\begin{pgfscope}%
\pgfpathrectangle{\pgfqpoint{1.254980in}{0.150000in}}{\pgfqpoint{5.490039in}{5.490039in}}%
\pgfusepath{clip}%
\pgfsetbuttcap%
\pgfsetroundjoin%
\definecolor{currentfill}{rgb}{0.262138,0.242286,0.520837}%
\pgfsetfillcolor{currentfill}%
\pgfsetfillopacity{0.700000}%
\pgfsetlinewidth{0.000000pt}%
\definecolor{currentstroke}{rgb}{0.000000,0.000000,0.000000}%
\pgfsetstrokecolor{currentstroke}%
\pgfsetdash{}{0pt}%
\pgfpathmoveto{\pgfqpoint{2.831988in}{2.539639in}}%
\pgfpathlineto{\pgfqpoint{2.845103in}{2.524265in}}%
\pgfpathlineto{\pgfqpoint{2.858212in}{2.509154in}}%
\pgfpathlineto{\pgfqpoint{2.871314in}{2.494305in}}%
\pgfpathlineto{\pgfqpoint{2.884411in}{2.479714in}}%
\pgfpathlineto{\pgfqpoint{2.892414in}{2.487204in}}%
\pgfpathlineto{\pgfqpoint{2.900409in}{2.494804in}}%
\pgfpathlineto{\pgfqpoint{2.908395in}{2.502512in}}%
\pgfpathlineto{\pgfqpoint{2.916373in}{2.510328in}}%
\pgfpathlineto{\pgfqpoint{2.903298in}{2.524776in}}%
\pgfpathlineto{\pgfqpoint{2.890218in}{2.539484in}}%
\pgfpathlineto{\pgfqpoint{2.877131in}{2.554452in}}%
\pgfpathlineto{\pgfqpoint{2.864038in}{2.569684in}}%
\pgfpathlineto{\pgfqpoint{2.856039in}{2.562000in}}%
\pgfpathlineto{\pgfqpoint{2.848031in}{2.554430in}}%
\pgfpathlineto{\pgfqpoint{2.840014in}{2.546976in}}%
\pgfpathlineto{\pgfqpoint{2.831988in}{2.539639in}}%
\pgfpathclose%
\pgfusepath{fill}%
\end{pgfscope}%
\begin{pgfscope}%
\pgfpathrectangle{\pgfqpoint{1.254980in}{0.150000in}}{\pgfqpoint{5.490039in}{5.490039in}}%
\pgfusepath{clip}%
\pgfsetbuttcap%
\pgfsetroundjoin%
\definecolor{currentfill}{rgb}{0.252194,0.269783,0.531579}%
\pgfsetfillcolor{currentfill}%
\pgfsetfillopacity{0.700000}%
\pgfsetlinewidth{0.000000pt}%
\definecolor{currentstroke}{rgb}{0.000000,0.000000,0.000000}%
\pgfsetstrokecolor{currentstroke}%
\pgfsetdash{}{0pt}%
\pgfpathmoveto{\pgfqpoint{2.779460in}{2.603817in}}%
\pgfpathlineto{\pgfqpoint{2.792603in}{2.587366in}}%
\pgfpathlineto{\pgfqpoint{2.805739in}{2.571187in}}%
\pgfpathlineto{\pgfqpoint{2.818867in}{2.555279in}}%
\pgfpathlineto{\pgfqpoint{2.831988in}{2.539639in}}%
\pgfpathlineto{\pgfqpoint{2.840014in}{2.546976in}}%
\pgfpathlineto{\pgfqpoint{2.848031in}{2.554430in}}%
\pgfpathlineto{\pgfqpoint{2.856039in}{2.562000in}}%
\pgfpathlineto{\pgfqpoint{2.864038in}{2.569684in}}%
\pgfpathlineto{\pgfqpoint{2.850939in}{2.585181in}}%
\pgfpathlineto{\pgfqpoint{2.837833in}{2.600945in}}%
\pgfpathlineto{\pgfqpoint{2.824721in}{2.616980in}}%
\pgfpathlineto{\pgfqpoint{2.811601in}{2.633288in}}%
\pgfpathlineto{\pgfqpoint{2.803579in}{2.625737in}}%
\pgfpathlineto{\pgfqpoint{2.795549in}{2.618307in}}%
\pgfpathlineto{\pgfqpoint{2.787509in}{2.611000in}}%
\pgfpathlineto{\pgfqpoint{2.779460in}{2.603817in}}%
\pgfpathclose%
\pgfusepath{fill}%
\end{pgfscope}%
\begin{pgfscope}%
\pgfpathrectangle{\pgfqpoint{1.254980in}{0.150000in}}{\pgfqpoint{5.490039in}{5.490039in}}%
\pgfusepath{clip}%
\pgfsetbuttcap%
\pgfsetroundjoin%
\definecolor{currentfill}{rgb}{0.180629,0.429975,0.557282}%
\pgfsetfillcolor{currentfill}%
\pgfsetfillopacity{0.700000}%
\pgfsetlinewidth{0.000000pt}%
\definecolor{currentstroke}{rgb}{0.000000,0.000000,0.000000}%
\pgfsetstrokecolor{currentstroke}%
\pgfsetdash{}{0pt}%
\pgfpathmoveto{\pgfqpoint{5.122795in}{2.932494in}}%
\pgfpathlineto{\pgfqpoint{5.136303in}{2.937117in}}%
\pgfpathlineto{\pgfqpoint{5.149824in}{2.941895in}}%
\pgfpathlineto{\pgfqpoint{5.163360in}{2.946827in}}%
\pgfpathlineto{\pgfqpoint{5.176909in}{2.951913in}}%
\pgfpathlineto{\pgfqpoint{5.184064in}{2.958726in}}%
\pgfpathlineto{\pgfqpoint{5.191216in}{2.965625in}}%
\pgfpathlineto{\pgfqpoint{5.198364in}{2.972616in}}%
\pgfpathlineto{\pgfqpoint{5.205510in}{2.979705in}}%
\pgfpathlineto{\pgfqpoint{5.191981in}{2.975105in}}%
\pgfpathlineto{\pgfqpoint{5.178465in}{2.970658in}}%
\pgfpathlineto{\pgfqpoint{5.164963in}{2.966364in}}%
\pgfpathlineto{\pgfqpoint{5.151474in}{2.962224in}}%
\pgfpathlineto{\pgfqpoint{5.144309in}{2.954641in}}%
\pgfpathlineto{\pgfqpoint{5.137141in}{2.947162in}}%
\pgfpathlineto{\pgfqpoint{5.129969in}{2.939781in}}%
\pgfpathlineto{\pgfqpoint{5.122795in}{2.932494in}}%
\pgfpathclose%
\pgfusepath{fill}%
\end{pgfscope}%
\begin{pgfscope}%
\pgfpathrectangle{\pgfqpoint{1.254980in}{0.150000in}}{\pgfqpoint{5.490039in}{5.490039in}}%
\pgfusepath{clip}%
\pgfsetbuttcap%
\pgfsetroundjoin%
\definecolor{currentfill}{rgb}{0.270595,0.214069,0.507052}%
\pgfsetfillcolor{currentfill}%
\pgfsetfillopacity{0.700000}%
\pgfsetlinewidth{0.000000pt}%
\definecolor{currentstroke}{rgb}{0.000000,0.000000,0.000000}%
\pgfsetstrokecolor{currentstroke}%
\pgfsetdash{}{0pt}%
\pgfpathmoveto{\pgfqpoint{2.884411in}{2.479714in}}%
\pgfpathlineto{\pgfqpoint{2.897501in}{2.465380in}}%
\pgfpathlineto{\pgfqpoint{2.910587in}{2.451301in}}%
\pgfpathlineto{\pgfqpoint{2.923667in}{2.437475in}}%
\pgfpathlineto{\pgfqpoint{2.936742in}{2.423899in}}%
\pgfpathlineto{\pgfqpoint{2.944724in}{2.431541in}}%
\pgfpathlineto{\pgfqpoint{2.952698in}{2.439285in}}%
\pgfpathlineto{\pgfqpoint{2.960663in}{2.447131in}}%
\pgfpathlineto{\pgfqpoint{2.968621in}{2.455077in}}%
\pgfpathlineto{\pgfqpoint{2.955566in}{2.468512in}}%
\pgfpathlineto{\pgfqpoint{2.942507in}{2.482198in}}%
\pgfpathlineto{\pgfqpoint{2.929442in}{2.496136in}}%
\pgfpathlineto{\pgfqpoint{2.916373in}{2.510328in}}%
\pgfpathlineto{\pgfqpoint{2.908395in}{2.502512in}}%
\pgfpathlineto{\pgfqpoint{2.900409in}{2.494804in}}%
\pgfpathlineto{\pgfqpoint{2.892414in}{2.487204in}}%
\pgfpathlineto{\pgfqpoint{2.884411in}{2.479714in}}%
\pgfpathclose%
\pgfusepath{fill}%
\end{pgfscope}%
\begin{pgfscope}%
\pgfpathrectangle{\pgfqpoint{1.254980in}{0.150000in}}{\pgfqpoint{5.490039in}{5.490039in}}%
\pgfusepath{clip}%
\pgfsetbuttcap%
\pgfsetroundjoin%
\definecolor{currentfill}{rgb}{0.281887,0.150881,0.465405}%
\pgfsetfillcolor{currentfill}%
\pgfsetfillopacity{0.700000}%
\pgfsetlinewidth{0.000000pt}%
\definecolor{currentstroke}{rgb}{0.000000,0.000000,0.000000}%
\pgfsetstrokecolor{currentstroke}%
\pgfsetdash{}{0pt}%
\pgfpathmoveto{\pgfqpoint{3.965668in}{2.319978in}}%
\pgfpathlineto{\pgfqpoint{3.978742in}{2.319255in}}%
\pgfpathlineto{\pgfqpoint{3.991824in}{2.318708in}}%
\pgfpathlineto{\pgfqpoint{4.004913in}{2.318337in}}%
\pgfpathlineto{\pgfqpoint{4.018009in}{2.318139in}}%
\pgfpathlineto{\pgfqpoint{4.025598in}{2.327710in}}%
\pgfpathlineto{\pgfqpoint{4.033182in}{2.337275in}}%
\pgfpathlineto{\pgfqpoint{4.040761in}{2.346838in}}%
\pgfpathlineto{\pgfqpoint{4.048335in}{2.356398in}}%
\pgfpathlineto{\pgfqpoint{4.035247in}{2.356684in}}%
\pgfpathlineto{\pgfqpoint{4.022166in}{2.357143in}}%
\pgfpathlineto{\pgfqpoint{4.009093in}{2.357778in}}%
\pgfpathlineto{\pgfqpoint{3.996026in}{2.358588in}}%
\pgfpathlineto{\pgfqpoint{3.988444in}{2.348929in}}%
\pgfpathlineto{\pgfqpoint{3.980857in}{2.339276in}}%
\pgfpathlineto{\pgfqpoint{3.973265in}{2.329626in}}%
\pgfpathlineto{\pgfqpoint{3.965668in}{2.319978in}}%
\pgfpathclose%
\pgfusepath{fill}%
\end{pgfscope}%
\begin{pgfscope}%
\pgfpathrectangle{\pgfqpoint{1.254980in}{0.150000in}}{\pgfqpoint{5.490039in}{5.490039in}}%
\pgfusepath{clip}%
\pgfsetbuttcap%
\pgfsetroundjoin%
\definecolor{currentfill}{rgb}{0.172719,0.448791,0.557885}%
\pgfsetfillcolor{currentfill}%
\pgfsetfillopacity{0.700000}%
\pgfsetlinewidth{0.000000pt}%
\definecolor{currentstroke}{rgb}{0.000000,0.000000,0.000000}%
\pgfsetstrokecolor{currentstroke}%
\pgfsetdash{}{0pt}%
\pgfpathmoveto{\pgfqpoint{5.205510in}{2.979705in}}%
\pgfpathlineto{\pgfqpoint{5.219053in}{2.984458in}}%
\pgfpathlineto{\pgfqpoint{5.232610in}{2.989364in}}%
\pgfpathlineto{\pgfqpoint{5.246181in}{2.994423in}}%
\pgfpathlineto{\pgfqpoint{5.259766in}{2.999635in}}%
\pgfpathlineto{\pgfqpoint{5.266888in}{3.006323in}}%
\pgfpathlineto{\pgfqpoint{5.274007in}{3.013113in}}%
\pgfpathlineto{\pgfqpoint{5.281123in}{3.020010in}}%
\pgfpathlineto{\pgfqpoint{5.288237in}{3.027021in}}%
\pgfpathlineto{\pgfqpoint{5.274673in}{3.022323in}}%
\pgfpathlineto{\pgfqpoint{5.261123in}{3.017778in}}%
\pgfpathlineto{\pgfqpoint{5.247587in}{3.013385in}}%
\pgfpathlineto{\pgfqpoint{5.234065in}{3.009144in}}%
\pgfpathlineto{\pgfqpoint{5.226930in}{3.001610in}}%
\pgfpathlineto{\pgfqpoint{5.219792in}{2.994196in}}%
\pgfpathlineto{\pgfqpoint{5.212653in}{2.986896in}}%
\pgfpathlineto{\pgfqpoint{5.205510in}{2.979705in}}%
\pgfpathclose%
\pgfusepath{fill}%
\end{pgfscope}%
\begin{pgfscope}%
\pgfpathrectangle{\pgfqpoint{1.254980in}{0.150000in}}{\pgfqpoint{5.490039in}{5.490039in}}%
\pgfusepath{clip}%
\pgfsetbuttcap%
\pgfsetroundjoin%
\definecolor{currentfill}{rgb}{0.283197,0.115680,0.436115}%
\pgfsetfillcolor{currentfill}%
\pgfsetfillopacity{0.700000}%
\pgfsetlinewidth{0.000000pt}%
\definecolor{currentstroke}{rgb}{0.000000,0.000000,0.000000}%
\pgfsetstrokecolor{currentstroke}%
\pgfsetdash{}{0pt}%
\pgfpathmoveto{\pgfqpoint{3.177035in}{2.272616in}}%
\pgfpathlineto{\pgfqpoint{3.190043in}{2.263156in}}%
\pgfpathlineto{\pgfqpoint{3.203050in}{2.253914in}}%
\pgfpathlineto{\pgfqpoint{3.216056in}{2.244890in}}%
\pgfpathlineto{\pgfqpoint{3.229062in}{2.236083in}}%
\pgfpathlineto{\pgfqpoint{3.236923in}{2.244733in}}%
\pgfpathlineto{\pgfqpoint{3.244778in}{2.253448in}}%
\pgfpathlineto{\pgfqpoint{3.252626in}{2.262225in}}%
\pgfpathlineto{\pgfqpoint{3.260468in}{2.271065in}}%
\pgfpathlineto{\pgfqpoint{3.247477in}{2.279764in}}%
\pgfpathlineto{\pgfqpoint{3.234487in}{2.288679in}}%
\pgfpathlineto{\pgfqpoint{3.221495in}{2.297811in}}%
\pgfpathlineto{\pgfqpoint{3.208503in}{2.307162in}}%
\pgfpathlineto{\pgfqpoint{3.200646in}{2.298421in}}%
\pgfpathlineto{\pgfqpoint{3.192783in}{2.289749in}}%
\pgfpathlineto{\pgfqpoint{3.184912in}{2.281147in}}%
\pgfpathlineto{\pgfqpoint{3.177035in}{2.272616in}}%
\pgfpathclose%
\pgfusepath{fill}%
\end{pgfscope}%
\begin{pgfscope}%
\pgfpathrectangle{\pgfqpoint{1.254980in}{0.150000in}}{\pgfqpoint{5.490039in}{5.490039in}}%
\pgfusepath{clip}%
\pgfsetbuttcap%
\pgfsetroundjoin%
\definecolor{currentfill}{rgb}{0.239346,0.300855,0.540844}%
\pgfsetfillcolor{currentfill}%
\pgfsetfillopacity{0.700000}%
\pgfsetlinewidth{0.000000pt}%
\definecolor{currentstroke}{rgb}{0.000000,0.000000,0.000000}%
\pgfsetstrokecolor{currentstroke}%
\pgfsetdash{}{0pt}%
\pgfpathmoveto{\pgfqpoint{2.726808in}{2.672402in}}%
\pgfpathlineto{\pgfqpoint{2.739983in}{2.654834in}}%
\pgfpathlineto{\pgfqpoint{2.753150in}{2.637549in}}%
\pgfpathlineto{\pgfqpoint{2.766309in}{2.620544in}}%
\pgfpathlineto{\pgfqpoint{2.779460in}{2.603817in}}%
\pgfpathlineto{\pgfqpoint{2.787509in}{2.611000in}}%
\pgfpathlineto{\pgfqpoint{2.795549in}{2.618307in}}%
\pgfpathlineto{\pgfqpoint{2.803579in}{2.625737in}}%
\pgfpathlineto{\pgfqpoint{2.811601in}{2.633288in}}%
\pgfpathlineto{\pgfqpoint{2.798473in}{2.649870in}}%
\pgfpathlineto{\pgfqpoint{2.785338in}{2.666731in}}%
\pgfpathlineto{\pgfqpoint{2.772195in}{2.683871in}}%
\pgfpathlineto{\pgfqpoint{2.759044in}{2.701294in}}%
\pgfpathlineto{\pgfqpoint{2.750999in}{2.693877in}}%
\pgfpathlineto{\pgfqpoint{2.742945in}{2.686588in}}%
\pgfpathlineto{\pgfqpoint{2.734881in}{2.679430in}}%
\pgfpathlineto{\pgfqpoint{2.726808in}{2.672402in}}%
\pgfpathclose%
\pgfusepath{fill}%
\end{pgfscope}%
\begin{pgfscope}%
\pgfpathrectangle{\pgfqpoint{1.254980in}{0.150000in}}{\pgfqpoint{5.490039in}{5.490039in}}%
\pgfusepath{clip}%
\pgfsetbuttcap%
\pgfsetroundjoin%
\definecolor{currentfill}{rgb}{0.165117,0.467423,0.558141}%
\pgfsetfillcolor{currentfill}%
\pgfsetfillopacity{0.700000}%
\pgfsetlinewidth{0.000000pt}%
\definecolor{currentstroke}{rgb}{0.000000,0.000000,0.000000}%
\pgfsetstrokecolor{currentstroke}%
\pgfsetdash{}{0pt}%
\pgfpathmoveto{\pgfqpoint{5.288237in}{3.027021in}}%
\pgfpathlineto{\pgfqpoint{5.301814in}{3.031871in}}%
\pgfpathlineto{\pgfqpoint{5.315406in}{3.036873in}}%
\pgfpathlineto{\pgfqpoint{5.329012in}{3.042028in}}%
\pgfpathlineto{\pgfqpoint{5.342633in}{3.047334in}}%
\pgfpathlineto{\pgfqpoint{5.349722in}{3.053932in}}%
\pgfpathlineto{\pgfqpoint{5.356809in}{3.060649in}}%
\pgfpathlineto{\pgfqpoint{5.363893in}{3.067491in}}%
\pgfpathlineto{\pgfqpoint{5.370976in}{3.074464in}}%
\pgfpathlineto{\pgfqpoint{5.357379in}{3.069700in}}%
\pgfpathlineto{\pgfqpoint{5.343795in}{3.065088in}}%
\pgfpathlineto{\pgfqpoint{5.330226in}{3.060627in}}%
\pgfpathlineto{\pgfqpoint{5.316670in}{3.056318in}}%
\pgfpathlineto{\pgfqpoint{5.309565in}{3.048794in}}%
\pgfpathlineto{\pgfqpoint{5.302457in}{3.041407in}}%
\pgfpathlineto{\pgfqpoint{5.295348in}{3.034152in}}%
\pgfpathlineto{\pgfqpoint{5.288237in}{3.027021in}}%
\pgfpathclose%
\pgfusepath{fill}%
\end{pgfscope}%
\begin{pgfscope}%
\pgfpathrectangle{\pgfqpoint{1.254980in}{0.150000in}}{\pgfqpoint{5.490039in}{5.490039in}}%
\pgfusepath{clip}%
\pgfsetbuttcap%
\pgfsetroundjoin%
\definecolor{currentfill}{rgb}{0.282327,0.094955,0.417331}%
\pgfsetfillcolor{currentfill}%
\pgfsetfillopacity{0.700000}%
\pgfsetlinewidth{0.000000pt}%
\definecolor{currentstroke}{rgb}{0.000000,0.000000,0.000000}%
\pgfsetstrokecolor{currentstroke}%
\pgfsetdash{}{0pt}%
\pgfpathmoveto{\pgfqpoint{3.582485in}{2.216085in}}%
\pgfpathlineto{\pgfqpoint{3.595487in}{2.211772in}}%
\pgfpathlineto{\pgfqpoint{3.608494in}{2.207649in}}%
\pgfpathlineto{\pgfqpoint{3.621504in}{2.203717in}}%
\pgfpathlineto{\pgfqpoint{3.634519in}{2.199973in}}%
\pgfpathlineto{\pgfqpoint{3.642233in}{2.209569in}}%
\pgfpathlineto{\pgfqpoint{3.649943in}{2.219184in}}%
\pgfpathlineto{\pgfqpoint{3.657647in}{2.228817in}}%
\pgfpathlineto{\pgfqpoint{3.665346in}{2.238470in}}%
\pgfpathlineto{\pgfqpoint{3.652341in}{2.242190in}}%
\pgfpathlineto{\pgfqpoint{3.639341in}{2.246098in}}%
\pgfpathlineto{\pgfqpoint{3.626345in}{2.250196in}}%
\pgfpathlineto{\pgfqpoint{3.613353in}{2.254485in}}%
\pgfpathlineto{\pgfqpoint{3.605644in}{2.244847in}}%
\pgfpathlineto{\pgfqpoint{3.597930in}{2.235234in}}%
\pgfpathlineto{\pgfqpoint{3.590210in}{2.225646in}}%
\pgfpathlineto{\pgfqpoint{3.582485in}{2.216085in}}%
\pgfpathclose%
\pgfusepath{fill}%
\end{pgfscope}%
\begin{pgfscope}%
\pgfpathrectangle{\pgfqpoint{1.254980in}{0.150000in}}{\pgfqpoint{5.490039in}{5.490039in}}%
\pgfusepath{clip}%
\pgfsetbuttcap%
\pgfsetroundjoin%
\definecolor{currentfill}{rgb}{0.276194,0.190074,0.493001}%
\pgfsetfillcolor{currentfill}%
\pgfsetfillopacity{0.700000}%
\pgfsetlinewidth{0.000000pt}%
\definecolor{currentstroke}{rgb}{0.000000,0.000000,0.000000}%
\pgfsetstrokecolor{currentstroke}%
\pgfsetdash{}{0pt}%
\pgfpathmoveto{\pgfqpoint{2.936742in}{2.423899in}}%
\pgfpathlineto{\pgfqpoint{2.949812in}{2.410572in}}%
\pgfpathlineto{\pgfqpoint{2.962878in}{2.397491in}}%
\pgfpathlineto{\pgfqpoint{2.975940in}{2.384655in}}%
\pgfpathlineto{\pgfqpoint{2.988998in}{2.372061in}}%
\pgfpathlineto{\pgfqpoint{2.996959in}{2.379854in}}%
\pgfpathlineto{\pgfqpoint{3.004913in}{2.387742in}}%
\pgfpathlineto{\pgfqpoint{3.012859in}{2.395724in}}%
\pgfpathlineto{\pgfqpoint{3.020797in}{2.403800in}}%
\pgfpathlineto{\pgfqpoint{3.007759in}{2.416254in}}%
\pgfpathlineto{\pgfqpoint{2.994717in}{2.428950in}}%
\pgfpathlineto{\pgfqpoint{2.981671in}{2.441890in}}%
\pgfpathlineto{\pgfqpoint{2.968621in}{2.455077in}}%
\pgfpathlineto{\pgfqpoint{2.960663in}{2.447131in}}%
\pgfpathlineto{\pgfqpoint{2.952698in}{2.439285in}}%
\pgfpathlineto{\pgfqpoint{2.944724in}{2.431541in}}%
\pgfpathlineto{\pgfqpoint{2.936742in}{2.423899in}}%
\pgfpathclose%
\pgfusepath{fill}%
\end{pgfscope}%
\begin{pgfscope}%
\pgfpathrectangle{\pgfqpoint{1.254980in}{0.150000in}}{\pgfqpoint{5.490039in}{5.490039in}}%
\pgfusepath{clip}%
\pgfsetbuttcap%
\pgfsetroundjoin%
\definecolor{currentfill}{rgb}{0.157729,0.485932,0.558013}%
\pgfsetfillcolor{currentfill}%
\pgfsetfillopacity{0.700000}%
\pgfsetlinewidth{0.000000pt}%
\definecolor{currentstroke}{rgb}{0.000000,0.000000,0.000000}%
\pgfsetstrokecolor{currentstroke}%
\pgfsetdash{}{0pt}%
\pgfpathmoveto{\pgfqpoint{5.370976in}{3.074464in}}%
\pgfpathlineto{\pgfqpoint{5.384588in}{3.079379in}}%
\pgfpathlineto{\pgfqpoint{5.398214in}{3.084444in}}%
\pgfpathlineto{\pgfqpoint{5.411855in}{3.089662in}}%
\pgfpathlineto{\pgfqpoint{5.425510in}{3.095030in}}%
\pgfpathlineto{\pgfqpoint{5.432568in}{3.101579in}}%
\pgfpathlineto{\pgfqpoint{5.439623in}{3.108265in}}%
\pgfpathlineto{\pgfqpoint{5.446677in}{3.115094in}}%
\pgfpathlineto{\pgfqpoint{5.453730in}{3.122073in}}%
\pgfpathlineto{\pgfqpoint{5.440099in}{3.117276in}}%
\pgfpathlineto{\pgfqpoint{5.426483in}{3.112629in}}%
\pgfpathlineto{\pgfqpoint{5.412881in}{3.108133in}}%
\pgfpathlineto{\pgfqpoint{5.399293in}{3.103787in}}%
\pgfpathlineto{\pgfqpoint{5.392215in}{3.096229in}}%
\pgfpathlineto{\pgfqpoint{5.385137in}{3.088826in}}%
\pgfpathlineto{\pgfqpoint{5.378057in}{3.081573in}}%
\pgfpathlineto{\pgfqpoint{5.370976in}{3.074464in}}%
\pgfpathclose%
\pgfusepath{fill}%
\end{pgfscope}%
\begin{pgfscope}%
\pgfpathrectangle{\pgfqpoint{1.254980in}{0.150000in}}{\pgfqpoint{5.490039in}{5.490039in}}%
\pgfusepath{clip}%
\pgfsetbuttcap%
\pgfsetroundjoin%
\definecolor{currentfill}{rgb}{0.282884,0.135920,0.453427}%
\pgfsetfillcolor{currentfill}%
\pgfsetfillopacity{0.700000}%
\pgfsetlinewidth{0.000000pt}%
\definecolor{currentstroke}{rgb}{0.000000,0.000000,0.000000}%
\pgfsetstrokecolor{currentstroke}%
\pgfsetdash{}{0pt}%
\pgfpathmoveto{\pgfqpoint{3.882966in}{2.285807in}}%
\pgfpathlineto{\pgfqpoint{3.896022in}{2.284435in}}%
\pgfpathlineto{\pgfqpoint{3.909085in}{2.283242in}}%
\pgfpathlineto{\pgfqpoint{3.922154in}{2.282226in}}%
\pgfpathlineto{\pgfqpoint{3.935229in}{2.281387in}}%
\pgfpathlineto{\pgfqpoint{3.942847in}{2.291037in}}%
\pgfpathlineto{\pgfqpoint{3.950459in}{2.300685in}}%
\pgfpathlineto{\pgfqpoint{3.958066in}{2.310331in}}%
\pgfpathlineto{\pgfqpoint{3.965668in}{2.319978in}}%
\pgfpathlineto{\pgfqpoint{3.952600in}{2.320877in}}%
\pgfpathlineto{\pgfqpoint{3.939540in}{2.321953in}}%
\pgfpathlineto{\pgfqpoint{3.926486in}{2.323206in}}%
\pgfpathlineto{\pgfqpoint{3.913438in}{2.324638in}}%
\pgfpathlineto{\pgfqpoint{3.905828in}{2.314921in}}%
\pgfpathlineto{\pgfqpoint{3.898212in}{2.305211in}}%
\pgfpathlineto{\pgfqpoint{3.890592in}{2.295507in}}%
\pgfpathlineto{\pgfqpoint{3.882966in}{2.285807in}}%
\pgfpathclose%
\pgfusepath{fill}%
\end{pgfscope}%
\begin{pgfscope}%
\pgfpathrectangle{\pgfqpoint{1.254980in}{0.150000in}}{\pgfqpoint{5.490039in}{5.490039in}}%
\pgfusepath{clip}%
\pgfsetbuttcap%
\pgfsetroundjoin%
\definecolor{currentfill}{rgb}{0.150476,0.504369,0.557430}%
\pgfsetfillcolor{currentfill}%
\pgfsetfillopacity{0.700000}%
\pgfsetlinewidth{0.000000pt}%
\definecolor{currentstroke}{rgb}{0.000000,0.000000,0.000000}%
\pgfsetstrokecolor{currentstroke}%
\pgfsetdash{}{0pt}%
\pgfpathmoveto{\pgfqpoint{5.453730in}{3.122073in}}%
\pgfpathlineto{\pgfqpoint{5.467375in}{3.127020in}}%
\pgfpathlineto{\pgfqpoint{5.481035in}{3.132118in}}%
\pgfpathlineto{\pgfqpoint{5.494710in}{3.137366in}}%
\pgfpathlineto{\pgfqpoint{5.508400in}{3.142765in}}%
\pgfpathlineto{\pgfqpoint{5.515426in}{3.149311in}}%
\pgfpathlineto{\pgfqpoint{5.522452in}{3.156012in}}%
\pgfpathlineto{\pgfqpoint{5.529476in}{3.162876in}}%
\pgfpathlineto{\pgfqpoint{5.536501in}{3.169910in}}%
\pgfpathlineto{\pgfqpoint{5.522838in}{3.165111in}}%
\pgfpathlineto{\pgfqpoint{5.509189in}{3.160461in}}%
\pgfpathlineto{\pgfqpoint{5.495555in}{3.155961in}}%
\pgfpathlineto{\pgfqpoint{5.481935in}{3.151611in}}%
\pgfpathlineto{\pgfqpoint{5.474884in}{3.143970in}}%
\pgfpathlineto{\pgfqpoint{5.467834in}{3.136504in}}%
\pgfpathlineto{\pgfqpoint{5.460782in}{3.129207in}}%
\pgfpathlineto{\pgfqpoint{5.453730in}{3.122073in}}%
\pgfpathclose%
\pgfusepath{fill}%
\end{pgfscope}%
\begin{pgfscope}%
\pgfpathrectangle{\pgfqpoint{1.254980in}{0.150000in}}{\pgfqpoint{5.490039in}{5.490039in}}%
\pgfusepath{clip}%
\pgfsetbuttcap%
\pgfsetroundjoin%
\definecolor{currentfill}{rgb}{0.223925,0.334994,0.548053}%
\pgfsetfillcolor{currentfill}%
\pgfsetfillopacity{0.700000}%
\pgfsetlinewidth{0.000000pt}%
\definecolor{currentstroke}{rgb}{0.000000,0.000000,0.000000}%
\pgfsetstrokecolor{currentstroke}%
\pgfsetdash{}{0pt}%
\pgfpathmoveto{\pgfqpoint{2.674015in}{2.745559in}}%
\pgfpathlineto{\pgfqpoint{2.687227in}{2.726832in}}%
\pgfpathlineto{\pgfqpoint{2.700430in}{2.708399in}}%
\pgfpathlineto{\pgfqpoint{2.713623in}{2.690256in}}%
\pgfpathlineto{\pgfqpoint{2.726808in}{2.672402in}}%
\pgfpathlineto{\pgfqpoint{2.734881in}{2.679430in}}%
\pgfpathlineto{\pgfqpoint{2.742945in}{2.686588in}}%
\pgfpathlineto{\pgfqpoint{2.750999in}{2.693877in}}%
\pgfpathlineto{\pgfqpoint{2.759044in}{2.701294in}}%
\pgfpathlineto{\pgfqpoint{2.745884in}{2.719002in}}%
\pgfpathlineto{\pgfqpoint{2.732715in}{2.736998in}}%
\pgfpathlineto{\pgfqpoint{2.719537in}{2.755285in}}%
\pgfpathlineto{\pgfqpoint{2.706350in}{2.773866in}}%
\pgfpathlineto{\pgfqpoint{2.698281in}{2.766584in}}%
\pgfpathlineto{\pgfqpoint{2.690202in}{2.759438in}}%
\pgfpathlineto{\pgfqpoint{2.682113in}{2.752430in}}%
\pgfpathlineto{\pgfqpoint{2.674015in}{2.745559in}}%
\pgfpathclose%
\pgfusepath{fill}%
\end{pgfscope}%
\begin{pgfscope}%
\pgfpathrectangle{\pgfqpoint{1.254980in}{0.150000in}}{\pgfqpoint{5.490039in}{5.490039in}}%
\pgfusepath{clip}%
\pgfsetbuttcap%
\pgfsetroundjoin%
\definecolor{currentfill}{rgb}{0.143343,0.522773,0.556295}%
\pgfsetfillcolor{currentfill}%
\pgfsetfillopacity{0.700000}%
\pgfsetlinewidth{0.000000pt}%
\definecolor{currentstroke}{rgb}{0.000000,0.000000,0.000000}%
\pgfsetstrokecolor{currentstroke}%
\pgfsetdash{}{0pt}%
\pgfpathmoveto{\pgfqpoint{5.536501in}{3.169910in}}%
\pgfpathlineto{\pgfqpoint{5.550179in}{3.174858in}}%
\pgfpathlineto{\pgfqpoint{5.563872in}{3.179956in}}%
\pgfpathlineto{\pgfqpoint{5.577580in}{3.185204in}}%
\pgfpathlineto{\pgfqpoint{5.591303in}{3.190601in}}%
\pgfpathlineto{\pgfqpoint{5.598300in}{3.197194in}}%
\pgfpathlineto{\pgfqpoint{5.605297in}{3.203962in}}%
\pgfpathlineto{\pgfqpoint{5.612294in}{3.210915in}}%
\pgfpathlineto{\pgfqpoint{5.619292in}{3.218058in}}%
\pgfpathlineto{\pgfqpoint{5.605598in}{3.213288in}}%
\pgfpathlineto{\pgfqpoint{5.591918in}{3.208668in}}%
\pgfpathlineto{\pgfqpoint{5.578252in}{3.204196in}}%
\pgfpathlineto{\pgfqpoint{5.564602in}{3.199873in}}%
\pgfpathlineto{\pgfqpoint{5.557576in}{3.192094in}}%
\pgfpathlineto{\pgfqpoint{5.550550in}{3.184512in}}%
\pgfpathlineto{\pgfqpoint{5.543526in}{3.177119in}}%
\pgfpathlineto{\pgfqpoint{5.536501in}{3.169910in}}%
\pgfpathclose%
\pgfusepath{fill}%
\end{pgfscope}%
\begin{pgfscope}%
\pgfpathrectangle{\pgfqpoint{1.254980in}{0.150000in}}{\pgfqpoint{5.490039in}{5.490039in}}%
\pgfusepath{clip}%
\pgfsetbuttcap%
\pgfsetroundjoin%
\definecolor{currentfill}{rgb}{0.280255,0.165693,0.476498}%
\pgfsetfillcolor{currentfill}%
\pgfsetfillopacity{0.700000}%
\pgfsetlinewidth{0.000000pt}%
\definecolor{currentstroke}{rgb}{0.000000,0.000000,0.000000}%
\pgfsetstrokecolor{currentstroke}%
\pgfsetdash{}{0pt}%
\pgfpathmoveto{\pgfqpoint{2.988998in}{2.372061in}}%
\pgfpathlineto{\pgfqpoint{3.002052in}{2.359709in}}%
\pgfpathlineto{\pgfqpoint{3.015102in}{2.347595in}}%
\pgfpathlineto{\pgfqpoint{3.028149in}{2.335718in}}%
\pgfpathlineto{\pgfqpoint{3.041193in}{2.324077in}}%
\pgfpathlineto{\pgfqpoint{3.049135in}{2.332019in}}%
\pgfpathlineto{\pgfqpoint{3.057069in}{2.340050in}}%
\pgfpathlineto{\pgfqpoint{3.064996in}{2.348168in}}%
\pgfpathlineto{\pgfqpoint{3.072915in}{2.356372in}}%
\pgfpathlineto{\pgfqpoint{3.059890in}{2.367875in}}%
\pgfpathlineto{\pgfqpoint{3.046862in}{2.379612in}}%
\pgfpathlineto{\pgfqpoint{3.033831in}{2.391587in}}%
\pgfpathlineto{\pgfqpoint{3.020797in}{2.403800in}}%
\pgfpathlineto{\pgfqpoint{3.012859in}{2.395724in}}%
\pgfpathlineto{\pgfqpoint{3.004913in}{2.387742in}}%
\pgfpathlineto{\pgfqpoint{2.996959in}{2.379854in}}%
\pgfpathlineto{\pgfqpoint{2.988998in}{2.372061in}}%
\pgfpathclose%
\pgfusepath{fill}%
\end{pgfscope}%
\begin{pgfscope}%
\pgfpathrectangle{\pgfqpoint{1.254980in}{0.150000in}}{\pgfqpoint{5.490039in}{5.490039in}}%
\pgfusepath{clip}%
\pgfsetbuttcap%
\pgfsetroundjoin%
\definecolor{currentfill}{rgb}{0.136408,0.541173,0.554483}%
\pgfsetfillcolor{currentfill}%
\pgfsetfillopacity{0.700000}%
\pgfsetlinewidth{0.000000pt}%
\definecolor{currentstroke}{rgb}{0.000000,0.000000,0.000000}%
\pgfsetstrokecolor{currentstroke}%
\pgfsetdash{}{0pt}%
\pgfpathmoveto{\pgfqpoint{5.619292in}{3.218058in}}%
\pgfpathlineto{\pgfqpoint{5.633002in}{3.222976in}}%
\pgfpathlineto{\pgfqpoint{5.646727in}{3.228042in}}%
\pgfpathlineto{\pgfqpoint{5.660467in}{3.233258in}}%
\pgfpathlineto{\pgfqpoint{5.674222in}{3.238622in}}%
\pgfpathlineto{\pgfqpoint{5.681192in}{3.245317in}}%
\pgfpathlineto{\pgfqpoint{5.688163in}{3.252210in}}%
\pgfpathlineto{\pgfqpoint{5.695135in}{3.259308in}}%
\pgfpathlineto{\pgfqpoint{5.702109in}{3.266619in}}%
\pgfpathlineto{\pgfqpoint{5.688383in}{3.261911in}}%
\pgfpathlineto{\pgfqpoint{5.674673in}{3.257351in}}%
\pgfpathlineto{\pgfqpoint{5.660978in}{3.252939in}}%
\pgfpathlineto{\pgfqpoint{5.647297in}{3.248675in}}%
\pgfpathlineto{\pgfqpoint{5.640293in}{3.240699in}}%
\pgfpathlineto{\pgfqpoint{5.633292in}{3.232942in}}%
\pgfpathlineto{\pgfqpoint{5.626291in}{3.225398in}}%
\pgfpathlineto{\pgfqpoint{5.619292in}{3.218058in}}%
\pgfpathclose%
\pgfusepath{fill}%
\end{pgfscope}%
\begin{pgfscope}%
\pgfpathrectangle{\pgfqpoint{1.254980in}{0.150000in}}{\pgfqpoint{5.490039in}{5.490039in}}%
\pgfusepath{clip}%
\pgfsetbuttcap%
\pgfsetroundjoin%
\definecolor{currentfill}{rgb}{0.283229,0.120777,0.440584}%
\pgfsetfillcolor{currentfill}%
\pgfsetfillopacity{0.700000}%
\pgfsetlinewidth{0.000000pt}%
\definecolor{currentstroke}{rgb}{0.000000,0.000000,0.000000}%
\pgfsetstrokecolor{currentstroke}%
\pgfsetdash{}{0pt}%
\pgfpathmoveto{\pgfqpoint{3.800218in}{2.254193in}}%
\pgfpathlineto{\pgfqpoint{3.813258in}{2.252133in}}%
\pgfpathlineto{\pgfqpoint{3.826304in}{2.250254in}}%
\pgfpathlineto{\pgfqpoint{3.839356in}{2.248556in}}%
\pgfpathlineto{\pgfqpoint{3.852414in}{2.247037in}}%
\pgfpathlineto{\pgfqpoint{3.860060in}{2.256727in}}%
\pgfpathlineto{\pgfqpoint{3.867700in}{2.266418in}}%
\pgfpathlineto{\pgfqpoint{3.875336in}{2.276111in}}%
\pgfpathlineto{\pgfqpoint{3.882966in}{2.285807in}}%
\pgfpathlineto{\pgfqpoint{3.869917in}{2.287358in}}%
\pgfpathlineto{\pgfqpoint{3.856874in}{2.289088in}}%
\pgfpathlineto{\pgfqpoint{3.843836in}{2.290999in}}%
\pgfpathlineto{\pgfqpoint{3.830805in}{2.293091in}}%
\pgfpathlineto{\pgfqpoint{3.823166in}{2.283352in}}%
\pgfpathlineto{\pgfqpoint{3.815522in}{2.273624in}}%
\pgfpathlineto{\pgfqpoint{3.807872in}{2.263904in}}%
\pgfpathlineto{\pgfqpoint{3.800218in}{2.254193in}}%
\pgfpathclose%
\pgfusepath{fill}%
\end{pgfscope}%
\begin{pgfscope}%
\pgfpathrectangle{\pgfqpoint{1.254980in}{0.150000in}}{\pgfqpoint{5.490039in}{5.490039in}}%
\pgfusepath{clip}%
\pgfsetbuttcap%
\pgfsetroundjoin%
\definecolor{currentfill}{rgb}{0.281924,0.089666,0.412415}%
\pgfsetfillcolor{currentfill}%
\pgfsetfillopacity{0.700000}%
\pgfsetlinewidth{0.000000pt}%
\definecolor{currentstroke}{rgb}{0.000000,0.000000,0.000000}%
\pgfsetstrokecolor{currentstroke}%
\pgfsetdash{}{0pt}%
\pgfpathmoveto{\pgfqpoint{3.364401in}{2.209095in}}%
\pgfpathlineto{\pgfqpoint{3.377397in}{2.202282in}}%
\pgfpathlineto{\pgfqpoint{3.390394in}{2.195672in}}%
\pgfpathlineto{\pgfqpoint{3.403393in}{2.189265in}}%
\pgfpathlineto{\pgfqpoint{3.416394in}{2.183058in}}%
\pgfpathlineto{\pgfqpoint{3.424188in}{2.192232in}}%
\pgfpathlineto{\pgfqpoint{3.431975in}{2.201447in}}%
\pgfpathlineto{\pgfqpoint{3.439757in}{2.210703in}}%
\pgfpathlineto{\pgfqpoint{3.447533in}{2.219998in}}%
\pgfpathlineto{\pgfqpoint{3.434545in}{2.226125in}}%
\pgfpathlineto{\pgfqpoint{3.421559in}{2.232452in}}%
\pgfpathlineto{\pgfqpoint{3.408575in}{2.238982in}}%
\pgfpathlineto{\pgfqpoint{3.395593in}{2.245715in}}%
\pgfpathlineto{\pgfqpoint{3.387804in}{2.236489in}}%
\pgfpathlineto{\pgfqpoint{3.380009in}{2.227310in}}%
\pgfpathlineto{\pgfqpoint{3.372208in}{2.218179in}}%
\pgfpathlineto{\pgfqpoint{3.364401in}{2.209095in}}%
\pgfpathclose%
\pgfusepath{fill}%
\end{pgfscope}%
\begin{pgfscope}%
\pgfpathrectangle{\pgfqpoint{1.254980in}{0.150000in}}{\pgfqpoint{5.490039in}{5.490039in}}%
\pgfusepath{clip}%
\pgfsetbuttcap%
\pgfsetroundjoin%
\definecolor{currentfill}{rgb}{0.282910,0.105393,0.426902}%
\pgfsetfillcolor{currentfill}%
\pgfsetfillopacity{0.700000}%
\pgfsetlinewidth{0.000000pt}%
\definecolor{currentstroke}{rgb}{0.000000,0.000000,0.000000}%
\pgfsetstrokecolor{currentstroke}%
\pgfsetdash{}{0pt}%
\pgfpathmoveto{\pgfqpoint{3.229062in}{2.236083in}}%
\pgfpathlineto{\pgfqpoint{3.242068in}{2.227490in}}%
\pgfpathlineto{\pgfqpoint{3.255073in}{2.219111in}}%
\pgfpathlineto{\pgfqpoint{3.268078in}{2.210944in}}%
\pgfpathlineto{\pgfqpoint{3.281084in}{2.202988in}}%
\pgfpathlineto{\pgfqpoint{3.288930in}{2.211757in}}%
\pgfpathlineto{\pgfqpoint{3.296770in}{2.220583in}}%
\pgfpathlineto{\pgfqpoint{3.304603in}{2.229466in}}%
\pgfpathlineto{\pgfqpoint{3.312430in}{2.238404in}}%
\pgfpathlineto{\pgfqpoint{3.299439in}{2.246252in}}%
\pgfpathlineto{\pgfqpoint{3.286448in}{2.254311in}}%
\pgfpathlineto{\pgfqpoint{3.273458in}{2.262581in}}%
\pgfpathlineto{\pgfqpoint{3.260468in}{2.271065in}}%
\pgfpathlineto{\pgfqpoint{3.252626in}{2.262225in}}%
\pgfpathlineto{\pgfqpoint{3.244778in}{2.253448in}}%
\pgfpathlineto{\pgfqpoint{3.236923in}{2.244733in}}%
\pgfpathlineto{\pgfqpoint{3.229062in}{2.236083in}}%
\pgfpathclose%
\pgfusepath{fill}%
\end{pgfscope}%
\begin{pgfscope}%
\pgfpathrectangle{\pgfqpoint{1.254980in}{0.150000in}}{\pgfqpoint{5.490039in}{5.490039in}}%
\pgfusepath{clip}%
\pgfsetbuttcap%
\pgfsetroundjoin%
\definecolor{currentfill}{rgb}{0.208623,0.367752,0.552675}%
\pgfsetfillcolor{currentfill}%
\pgfsetfillopacity{0.700000}%
\pgfsetlinewidth{0.000000pt}%
\definecolor{currentstroke}{rgb}{0.000000,0.000000,0.000000}%
\pgfsetstrokecolor{currentstroke}%
\pgfsetdash{}{0pt}%
\pgfpathmoveto{\pgfqpoint{2.621062in}{2.823467in}}%
\pgfpathlineto{\pgfqpoint{2.634316in}{2.803534in}}%
\pgfpathlineto{\pgfqpoint{2.647559in}{2.783908in}}%
\pgfpathlineto{\pgfqpoint{2.660792in}{2.764584in}}%
\pgfpathlineto{\pgfqpoint{2.674015in}{2.745559in}}%
\pgfpathlineto{\pgfqpoint{2.682113in}{2.752430in}}%
\pgfpathlineto{\pgfqpoint{2.690202in}{2.759438in}}%
\pgfpathlineto{\pgfqpoint{2.698281in}{2.766584in}}%
\pgfpathlineto{\pgfqpoint{2.706350in}{2.773866in}}%
\pgfpathlineto{\pgfqpoint{2.693153in}{2.792743in}}%
\pgfpathlineto{\pgfqpoint{2.679946in}{2.811919in}}%
\pgfpathlineto{\pgfqpoint{2.666728in}{2.831398in}}%
\pgfpathlineto{\pgfqpoint{2.653501in}{2.851182in}}%
\pgfpathlineto{\pgfqpoint{2.645406in}{2.844037in}}%
\pgfpathlineto{\pgfqpoint{2.637302in}{2.837035in}}%
\pgfpathlineto{\pgfqpoint{2.629187in}{2.830178in}}%
\pgfpathlineto{\pgfqpoint{2.621062in}{2.823467in}}%
\pgfpathclose%
\pgfusepath{fill}%
\end{pgfscope}%
\begin{pgfscope}%
\pgfpathrectangle{\pgfqpoint{1.254980in}{0.150000in}}{\pgfqpoint{5.490039in}{5.490039in}}%
\pgfusepath{clip}%
\pgfsetbuttcap%
\pgfsetroundjoin%
\definecolor{currentfill}{rgb}{0.281446,0.084320,0.407414}%
\pgfsetfillcolor{currentfill}%
\pgfsetfillopacity{0.700000}%
\pgfsetlinewidth{0.000000pt}%
\definecolor{currentstroke}{rgb}{0.000000,0.000000,0.000000}%
\pgfsetstrokecolor{currentstroke}%
\pgfsetdash{}{0pt}%
\pgfpathmoveto{\pgfqpoint{3.499509in}{2.197480in}}%
\pgfpathlineto{\pgfqpoint{3.512509in}{2.192342in}}%
\pgfpathlineto{\pgfqpoint{3.525513in}{2.187399in}}%
\pgfpathlineto{\pgfqpoint{3.538520in}{2.182650in}}%
\pgfpathlineto{\pgfqpoint{3.551530in}{2.178093in}}%
\pgfpathlineto{\pgfqpoint{3.559277in}{2.187553in}}%
\pgfpathlineto{\pgfqpoint{3.567018in}{2.197038in}}%
\pgfpathlineto{\pgfqpoint{3.574754in}{2.206549in}}%
\pgfpathlineto{\pgfqpoint{3.582485in}{2.216085in}}%
\pgfpathlineto{\pgfqpoint{3.569486in}{2.220590in}}%
\pgfpathlineto{\pgfqpoint{3.556490in}{2.225287in}}%
\pgfpathlineto{\pgfqpoint{3.543498in}{2.230179in}}%
\pgfpathlineto{\pgfqpoint{3.530509in}{2.235264in}}%
\pgfpathlineto{\pgfqpoint{3.522767in}{2.225770in}}%
\pgfpathlineto{\pgfqpoint{3.515020in}{2.216307in}}%
\pgfpathlineto{\pgfqpoint{3.507267in}{2.206877in}}%
\pgfpathlineto{\pgfqpoint{3.499509in}{2.197480in}}%
\pgfpathclose%
\pgfusepath{fill}%
\end{pgfscope}%
\begin{pgfscope}%
\pgfpathrectangle{\pgfqpoint{1.254980in}{0.150000in}}{\pgfqpoint{5.490039in}{5.490039in}}%
\pgfusepath{clip}%
\pgfsetbuttcap%
\pgfsetroundjoin%
\definecolor{currentfill}{rgb}{0.282290,0.145912,0.461510}%
\pgfsetfillcolor{currentfill}%
\pgfsetfillopacity{0.700000}%
\pgfsetlinewidth{0.000000pt}%
\definecolor{currentstroke}{rgb}{0.000000,0.000000,0.000000}%
\pgfsetstrokecolor{currentstroke}%
\pgfsetdash{}{0pt}%
\pgfpathmoveto{\pgfqpoint{3.041193in}{2.324077in}}%
\pgfpathlineto{\pgfqpoint{3.054234in}{2.312670in}}%
\pgfpathlineto{\pgfqpoint{3.067272in}{2.301494in}}%
\pgfpathlineto{\pgfqpoint{3.080308in}{2.290549in}}%
\pgfpathlineto{\pgfqpoint{3.093341in}{2.279832in}}%
\pgfpathlineto{\pgfqpoint{3.101264in}{2.287923in}}%
\pgfpathlineto{\pgfqpoint{3.109180in}{2.296095in}}%
\pgfpathlineto{\pgfqpoint{3.117089in}{2.304347in}}%
\pgfpathlineto{\pgfqpoint{3.124990in}{2.312680in}}%
\pgfpathlineto{\pgfqpoint{3.111975in}{2.323259in}}%
\pgfpathlineto{\pgfqpoint{3.098957in}{2.334066in}}%
\pgfpathlineto{\pgfqpoint{3.085937in}{2.345103in}}%
\pgfpathlineto{\pgfqpoint{3.072915in}{2.356372in}}%
\pgfpathlineto{\pgfqpoint{3.064996in}{2.348168in}}%
\pgfpathlineto{\pgfqpoint{3.057069in}{2.340050in}}%
\pgfpathlineto{\pgfqpoint{3.049135in}{2.332019in}}%
\pgfpathlineto{\pgfqpoint{3.041193in}{2.324077in}}%
\pgfpathclose%
\pgfusepath{fill}%
\end{pgfscope}%
\begin{pgfscope}%
\pgfpathrectangle{\pgfqpoint{1.254980in}{0.150000in}}{\pgfqpoint{5.490039in}{5.490039in}}%
\pgfusepath{clip}%
\pgfsetbuttcap%
\pgfsetroundjoin%
\definecolor{currentfill}{rgb}{0.129933,0.559582,0.551864}%
\pgfsetfillcolor{currentfill}%
\pgfsetfillopacity{0.700000}%
\pgfsetlinewidth{0.000000pt}%
\definecolor{currentstroke}{rgb}{0.000000,0.000000,0.000000}%
\pgfsetstrokecolor{currentstroke}%
\pgfsetdash{}{0pt}%
\pgfpathmoveto{\pgfqpoint{5.702109in}{3.266619in}}%
\pgfpathlineto{\pgfqpoint{5.715849in}{3.271475in}}%
\pgfpathlineto{\pgfqpoint{5.729605in}{3.276480in}}%
\pgfpathlineto{\pgfqpoint{5.743376in}{3.281632in}}%
\pgfpathlineto{\pgfqpoint{5.757162in}{3.286932in}}%
\pgfpathlineto{\pgfqpoint{5.764107in}{3.293789in}}%
\pgfpathlineto{\pgfqpoint{5.771054in}{3.300868in}}%
\pgfpathlineto{\pgfqpoint{5.778004in}{3.308176in}}%
\pgfpathlineto{\pgfqpoint{5.764241in}{3.303387in}}%
\pgfpathlineto{\pgfqpoint{5.750494in}{3.298745in}}%
\pgfpathlineto{\pgfqpoint{5.736761in}{3.294251in}}%
\pgfpathlineto{\pgfqpoint{5.723044in}{3.289904in}}%
\pgfpathlineto{\pgfqpoint{5.716063in}{3.281910in}}%
\pgfpathlineto{\pgfqpoint{5.709085in}{3.274151in}}%
\pgfpathlineto{\pgfqpoint{5.702109in}{3.266619in}}%
\pgfpathclose%
\pgfusepath{fill}%
\end{pgfscope}%
\begin{pgfscope}%
\pgfpathrectangle{\pgfqpoint{1.254980in}{0.150000in}}{\pgfqpoint{5.490039in}{5.490039in}}%
\pgfusepath{clip}%
\pgfsetbuttcap%
\pgfsetroundjoin%
\definecolor{currentfill}{rgb}{0.282910,0.105393,0.426902}%
\pgfsetfillcolor{currentfill}%
\pgfsetfillopacity{0.700000}%
\pgfsetlinewidth{0.000000pt}%
\definecolor{currentstroke}{rgb}{0.000000,0.000000,0.000000}%
\pgfsetstrokecolor{currentstroke}%
\pgfsetdash{}{0pt}%
\pgfpathmoveto{\pgfqpoint{3.717408in}{2.225465in}}%
\pgfpathlineto{\pgfqpoint{3.730436in}{2.222677in}}%
\pgfpathlineto{\pgfqpoint{3.743469in}{2.220074in}}%
\pgfpathlineto{\pgfqpoint{3.756507in}{2.217654in}}%
\pgfpathlineto{\pgfqpoint{3.769550in}{2.215417in}}%
\pgfpathlineto{\pgfqpoint{3.777225in}{2.225102in}}%
\pgfpathlineto{\pgfqpoint{3.784894in}{2.234792in}}%
\pgfpathlineto{\pgfqpoint{3.792559in}{2.244489in}}%
\pgfpathlineto{\pgfqpoint{3.800218in}{2.254193in}}%
\pgfpathlineto{\pgfqpoint{3.787184in}{2.256434in}}%
\pgfpathlineto{\pgfqpoint{3.774155in}{2.258859in}}%
\pgfpathlineto{\pgfqpoint{3.761131in}{2.261466in}}%
\pgfpathlineto{\pgfqpoint{3.748113in}{2.264257in}}%
\pgfpathlineto{\pgfqpoint{3.740444in}{2.254539in}}%
\pgfpathlineto{\pgfqpoint{3.732771in}{2.244834in}}%
\pgfpathlineto{\pgfqpoint{3.725092in}{2.235143in}}%
\pgfpathlineto{\pgfqpoint{3.717408in}{2.225465in}}%
\pgfpathclose%
\pgfusepath{fill}%
\end{pgfscope}%
\begin{pgfscope}%
\pgfpathrectangle{\pgfqpoint{1.254980in}{0.150000in}}{\pgfqpoint{5.490039in}{5.490039in}}%
\pgfusepath{clip}%
\pgfsetbuttcap%
\pgfsetroundjoin%
\definecolor{currentfill}{rgb}{0.283187,0.125848,0.444960}%
\pgfsetfillcolor{currentfill}%
\pgfsetfillopacity{0.700000}%
\pgfsetlinewidth{0.000000pt}%
\definecolor{currentstroke}{rgb}{0.000000,0.000000,0.000000}%
\pgfsetstrokecolor{currentstroke}%
\pgfsetdash{}{0pt}%
\pgfpathmoveto{\pgfqpoint{3.093341in}{2.279832in}}%
\pgfpathlineto{\pgfqpoint{3.106372in}{2.269342in}}%
\pgfpathlineto{\pgfqpoint{3.119402in}{2.259078in}}%
\pgfpathlineto{\pgfqpoint{3.132429in}{2.249037in}}%
\pgfpathlineto{\pgfqpoint{3.145456in}{2.239219in}}%
\pgfpathlineto{\pgfqpoint{3.153361in}{2.247457in}}%
\pgfpathlineto{\pgfqpoint{3.161259in}{2.255770in}}%
\pgfpathlineto{\pgfqpoint{3.169150in}{2.264157in}}%
\pgfpathlineto{\pgfqpoint{3.177035in}{2.272616in}}%
\pgfpathlineto{\pgfqpoint{3.164026in}{2.282297in}}%
\pgfpathlineto{\pgfqpoint{3.151015in}{2.292201in}}%
\pgfpathlineto{\pgfqpoint{3.138003in}{2.302328in}}%
\pgfpathlineto{\pgfqpoint{3.124990in}{2.312680in}}%
\pgfpathlineto{\pgfqpoint{3.117089in}{2.304347in}}%
\pgfpathlineto{\pgfqpoint{3.109180in}{2.296095in}}%
\pgfpathlineto{\pgfqpoint{3.101264in}{2.287923in}}%
\pgfpathlineto{\pgfqpoint{3.093341in}{2.279832in}}%
\pgfpathclose%
\pgfusepath{fill}%
\end{pgfscope}%
\begin{pgfscope}%
\pgfpathrectangle{\pgfqpoint{1.254980in}{0.150000in}}{\pgfqpoint{5.490039in}{5.490039in}}%
\pgfusepath{clip}%
\pgfsetbuttcap%
\pgfsetroundjoin%
\definecolor{currentfill}{rgb}{0.262138,0.242286,0.520837}%
\pgfsetfillcolor{currentfill}%
\pgfsetfillopacity{0.700000}%
\pgfsetlinewidth{0.000000pt}%
\definecolor{currentstroke}{rgb}{0.000000,0.000000,0.000000}%
\pgfsetstrokecolor{currentstroke}%
\pgfsetdash{}{0pt}%
\pgfpathmoveto{\pgfqpoint{4.348979in}{2.483409in}}%
\pgfpathlineto{\pgfqpoint{4.362190in}{2.485578in}}%
\pgfpathlineto{\pgfqpoint{4.375412in}{2.487913in}}%
\pgfpathlineto{\pgfqpoint{4.388644in}{2.490413in}}%
\pgfpathlineto{\pgfqpoint{4.401885in}{2.493077in}}%
\pgfpathlineto{\pgfqpoint{4.409352in}{2.501827in}}%
\pgfpathlineto{\pgfqpoint{4.416813in}{2.510566in}}%
\pgfpathlineto{\pgfqpoint{4.424269in}{2.519300in}}%
\pgfpathlineto{\pgfqpoint{4.431720in}{2.528028in}}%
\pgfpathlineto{\pgfqpoint{4.418487in}{2.525565in}}%
\pgfpathlineto{\pgfqpoint{4.405265in}{2.523266in}}%
\pgfpathlineto{\pgfqpoint{4.392052in}{2.521132in}}%
\pgfpathlineto{\pgfqpoint{4.378850in}{2.519164in}}%
\pgfpathlineto{\pgfqpoint{4.371390in}{2.510224in}}%
\pgfpathlineto{\pgfqpoint{4.363924in}{2.501287in}}%
\pgfpathlineto{\pgfqpoint{4.356454in}{2.492349in}}%
\pgfpathlineto{\pgfqpoint{4.348979in}{2.483409in}}%
\pgfpathclose%
\pgfusepath{fill}%
\end{pgfscope}%
\begin{pgfscope}%
\pgfpathrectangle{\pgfqpoint{1.254980in}{0.150000in}}{\pgfqpoint{5.490039in}{5.490039in}}%
\pgfusepath{clip}%
\pgfsetbuttcap%
\pgfsetroundjoin%
\definecolor{currentfill}{rgb}{0.267968,0.223549,0.512008}%
\pgfsetfillcolor{currentfill}%
\pgfsetfillopacity{0.700000}%
\pgfsetlinewidth{0.000000pt}%
\definecolor{currentstroke}{rgb}{0.000000,0.000000,0.000000}%
\pgfsetstrokecolor{currentstroke}%
\pgfsetdash{}{0pt}%
\pgfpathmoveto{\pgfqpoint{4.266243in}{2.439879in}}%
\pgfpathlineto{\pgfqpoint{4.279425in}{2.441553in}}%
\pgfpathlineto{\pgfqpoint{4.292616in}{2.443395in}}%
\pgfpathlineto{\pgfqpoint{4.305816in}{2.445404in}}%
\pgfpathlineto{\pgfqpoint{4.319027in}{2.447580in}}%
\pgfpathlineto{\pgfqpoint{4.326522in}{2.456552in}}%
\pgfpathlineto{\pgfqpoint{4.334013in}{2.465513in}}%
\pgfpathlineto{\pgfqpoint{4.341498in}{2.474465in}}%
\pgfpathlineto{\pgfqpoint{4.348979in}{2.483409in}}%
\pgfpathlineto{\pgfqpoint{4.335777in}{2.481407in}}%
\pgfpathlineto{\pgfqpoint{4.322585in}{2.479571in}}%
\pgfpathlineto{\pgfqpoint{4.309403in}{2.477901in}}%
\pgfpathlineto{\pgfqpoint{4.296230in}{2.476399in}}%
\pgfpathlineto{\pgfqpoint{4.288740in}{2.467272in}}%
\pgfpathlineto{\pgfqpoint{4.281246in}{2.458144in}}%
\pgfpathlineto{\pgfqpoint{4.273747in}{2.449014in}}%
\pgfpathlineto{\pgfqpoint{4.266243in}{2.439879in}}%
\pgfpathclose%
\pgfusepath{fill}%
\end{pgfscope}%
\begin{pgfscope}%
\pgfpathrectangle{\pgfqpoint{1.254980in}{0.150000in}}{\pgfqpoint{5.490039in}{5.490039in}}%
\pgfusepath{clip}%
\pgfsetbuttcap%
\pgfsetroundjoin%
\definecolor{currentfill}{rgb}{0.253935,0.265254,0.529983}%
\pgfsetfillcolor{currentfill}%
\pgfsetfillopacity{0.700000}%
\pgfsetlinewidth{0.000000pt}%
\definecolor{currentstroke}{rgb}{0.000000,0.000000,0.000000}%
\pgfsetstrokecolor{currentstroke}%
\pgfsetdash{}{0pt}%
\pgfpathmoveto{\pgfqpoint{4.431720in}{2.528028in}}%
\pgfpathlineto{\pgfqpoint{4.444963in}{2.530657in}}%
\pgfpathlineto{\pgfqpoint{4.458216in}{2.533449in}}%
\pgfpathlineto{\pgfqpoint{4.471480in}{2.536405in}}%
\pgfpathlineto{\pgfqpoint{4.484755in}{2.539524in}}%
\pgfpathlineto{\pgfqpoint{4.492192in}{2.548033in}}%
\pgfpathlineto{\pgfqpoint{4.499623in}{2.556535in}}%
\pgfpathlineto{\pgfqpoint{4.507049in}{2.565035in}}%
\pgfpathlineto{\pgfqpoint{4.514470in}{2.573534in}}%
\pgfpathlineto{\pgfqpoint{4.501205in}{2.570644in}}%
\pgfpathlineto{\pgfqpoint{4.487950in}{2.567918in}}%
\pgfpathlineto{\pgfqpoint{4.474707in}{2.565355in}}%
\pgfpathlineto{\pgfqpoint{4.461473in}{2.562955in}}%
\pgfpathlineto{\pgfqpoint{4.454042in}{2.554216in}}%
\pgfpathlineto{\pgfqpoint{4.446607in}{2.545484in}}%
\pgfpathlineto{\pgfqpoint{4.439166in}{2.536756in}}%
\pgfpathlineto{\pgfqpoint{4.431720in}{2.528028in}}%
\pgfpathclose%
\pgfusepath{fill}%
\end{pgfscope}%
\begin{pgfscope}%
\pgfpathrectangle{\pgfqpoint{1.254980in}{0.150000in}}{\pgfqpoint{5.490039in}{5.490039in}}%
\pgfusepath{clip}%
\pgfsetbuttcap%
\pgfsetroundjoin%
\definecolor{currentfill}{rgb}{0.282327,0.094955,0.417331}%
\pgfsetfillcolor{currentfill}%
\pgfsetfillopacity{0.700000}%
\pgfsetlinewidth{0.000000pt}%
\definecolor{currentstroke}{rgb}{0.000000,0.000000,0.000000}%
\pgfsetstrokecolor{currentstroke}%
\pgfsetdash{}{0pt}%
\pgfpathmoveto{\pgfqpoint{3.634519in}{2.199973in}}%
\pgfpathlineto{\pgfqpoint{3.647537in}{2.196418in}}%
\pgfpathlineto{\pgfqpoint{3.660560in}{2.193051in}}%
\pgfpathlineto{\pgfqpoint{3.673588in}{2.189869in}}%
\pgfpathlineto{\pgfqpoint{3.686621in}{2.186874in}}%
\pgfpathlineto{\pgfqpoint{3.694325in}{2.196503in}}%
\pgfpathlineto{\pgfqpoint{3.702025in}{2.206145in}}%
\pgfpathlineto{\pgfqpoint{3.709719in}{2.215799in}}%
\pgfpathlineto{\pgfqpoint{3.717408in}{2.225465in}}%
\pgfpathlineto{\pgfqpoint{3.704385in}{2.228437in}}%
\pgfpathlineto{\pgfqpoint{3.691367in}{2.231594in}}%
\pgfpathlineto{\pgfqpoint{3.678354in}{2.234939in}}%
\pgfpathlineto{\pgfqpoint{3.665346in}{2.238470in}}%
\pgfpathlineto{\pgfqpoint{3.657647in}{2.228817in}}%
\pgfpathlineto{\pgfqpoint{3.649943in}{2.219184in}}%
\pgfpathlineto{\pgfqpoint{3.642233in}{2.209569in}}%
\pgfpathlineto{\pgfqpoint{3.634519in}{2.199973in}}%
\pgfpathclose%
\pgfusepath{fill}%
\end{pgfscope}%
\begin{pgfscope}%
\pgfpathrectangle{\pgfqpoint{1.254980in}{0.150000in}}{\pgfqpoint{5.490039in}{5.490039in}}%
\pgfusepath{clip}%
\pgfsetbuttcap%
\pgfsetroundjoin%
\definecolor{currentfill}{rgb}{0.274128,0.199721,0.498911}%
\pgfsetfillcolor{currentfill}%
\pgfsetfillopacity{0.700000}%
\pgfsetlinewidth{0.000000pt}%
\definecolor{currentstroke}{rgb}{0.000000,0.000000,0.000000}%
\pgfsetstrokecolor{currentstroke}%
\pgfsetdash{}{0pt}%
\pgfpathmoveto{\pgfqpoint{4.183507in}{2.397659in}}%
\pgfpathlineto{\pgfqpoint{4.196661in}{2.398803in}}%
\pgfpathlineto{\pgfqpoint{4.209823in}{2.400117in}}%
\pgfpathlineto{\pgfqpoint{4.222995in}{2.401600in}}%
\pgfpathlineto{\pgfqpoint{4.236175in}{2.403251in}}%
\pgfpathlineto{\pgfqpoint{4.243700in}{2.412425in}}%
\pgfpathlineto{\pgfqpoint{4.251219in}{2.421587in}}%
\pgfpathlineto{\pgfqpoint{4.258734in}{2.430737in}}%
\pgfpathlineto{\pgfqpoint{4.266243in}{2.439879in}}%
\pgfpathlineto{\pgfqpoint{4.253070in}{2.438372in}}%
\pgfpathlineto{\pgfqpoint{4.239907in}{2.437034in}}%
\pgfpathlineto{\pgfqpoint{4.226753in}{2.435864in}}%
\pgfpathlineto{\pgfqpoint{4.213608in}{2.434864in}}%
\pgfpathlineto{\pgfqpoint{4.206090in}{2.425567in}}%
\pgfpathlineto{\pgfqpoint{4.198567in}{2.416269in}}%
\pgfpathlineto{\pgfqpoint{4.191040in}{2.406967in}}%
\pgfpathlineto{\pgfqpoint{4.183507in}{2.397659in}}%
\pgfpathclose%
\pgfusepath{fill}%
\end{pgfscope}%
\begin{pgfscope}%
\pgfpathrectangle{\pgfqpoint{1.254980in}{0.150000in}}{\pgfqpoint{5.490039in}{5.490039in}}%
\pgfusepath{clip}%
\pgfsetbuttcap%
\pgfsetroundjoin%
\definecolor{currentfill}{rgb}{0.244972,0.287675,0.537260}%
\pgfsetfillcolor{currentfill}%
\pgfsetfillopacity{0.700000}%
\pgfsetlinewidth{0.000000pt}%
\definecolor{currentstroke}{rgb}{0.000000,0.000000,0.000000}%
\pgfsetstrokecolor{currentstroke}%
\pgfsetdash{}{0pt}%
\pgfpathmoveto{\pgfqpoint{4.514470in}{2.573534in}}%
\pgfpathlineto{\pgfqpoint{4.527745in}{2.576587in}}%
\pgfpathlineto{\pgfqpoint{4.541032in}{2.579802in}}%
\pgfpathlineto{\pgfqpoint{4.554330in}{2.583180in}}%
\pgfpathlineto{\pgfqpoint{4.567639in}{2.586720in}}%
\pgfpathlineto{\pgfqpoint{4.575044in}{2.594975in}}%
\pgfpathlineto{\pgfqpoint{4.582445in}{2.603228in}}%
\pgfpathlineto{\pgfqpoint{4.589840in}{2.611484in}}%
\pgfpathlineto{\pgfqpoint{4.597230in}{2.619744in}}%
\pgfpathlineto{\pgfqpoint{4.583932in}{2.616463in}}%
\pgfpathlineto{\pgfqpoint{4.570644in}{2.613343in}}%
\pgfpathlineto{\pgfqpoint{4.557368in}{2.610385in}}%
\pgfpathlineto{\pgfqpoint{4.544103in}{2.607590in}}%
\pgfpathlineto{\pgfqpoint{4.536702in}{2.599061in}}%
\pgfpathlineto{\pgfqpoint{4.529296in}{2.590544in}}%
\pgfpathlineto{\pgfqpoint{4.521885in}{2.582036in}}%
\pgfpathlineto{\pgfqpoint{4.514470in}{2.573534in}}%
\pgfpathclose%
\pgfusepath{fill}%
\end{pgfscope}%
\begin{pgfscope}%
\pgfpathrectangle{\pgfqpoint{1.254980in}{0.150000in}}{\pgfqpoint{5.490039in}{5.490039in}}%
\pgfusepath{clip}%
\pgfsetbuttcap%
\pgfsetroundjoin%
\definecolor{currentfill}{rgb}{0.237441,0.305202,0.541921}%
\pgfsetfillcolor{currentfill}%
\pgfsetfillopacity{0.700000}%
\pgfsetlinewidth{0.000000pt}%
\definecolor{currentstroke}{rgb}{0.000000,0.000000,0.000000}%
\pgfsetstrokecolor{currentstroke}%
\pgfsetdash{}{0pt}%
\pgfpathmoveto{\pgfqpoint{4.597230in}{2.619744in}}%
\pgfpathlineto{\pgfqpoint{4.610540in}{2.623188in}}%
\pgfpathlineto{\pgfqpoint{4.623861in}{2.626792in}}%
\pgfpathlineto{\pgfqpoint{4.637194in}{2.630558in}}%
\pgfpathlineto{\pgfqpoint{4.650538in}{2.634484in}}%
\pgfpathlineto{\pgfqpoint{4.657912in}{2.642477in}}%
\pgfpathlineto{\pgfqpoint{4.665281in}{2.650475in}}%
\pgfpathlineto{\pgfqpoint{4.672645in}{2.658481in}}%
\pgfpathlineto{\pgfqpoint{4.680003in}{2.666498in}}%
\pgfpathlineto{\pgfqpoint{4.666670in}{2.662859in}}%
\pgfpathlineto{\pgfqpoint{4.653349in}{2.659380in}}%
\pgfpathlineto{\pgfqpoint{4.640039in}{2.656061in}}%
\pgfpathlineto{\pgfqpoint{4.626741in}{2.652904in}}%
\pgfpathlineto{\pgfqpoint{4.619370in}{2.644590in}}%
\pgfpathlineto{\pgfqpoint{4.611995in}{2.636294in}}%
\pgfpathlineto{\pgfqpoint{4.604615in}{2.628014in}}%
\pgfpathlineto{\pgfqpoint{4.597230in}{2.619744in}}%
\pgfpathclose%
\pgfusepath{fill}%
\end{pgfscope}%
\begin{pgfscope}%
\pgfpathrectangle{\pgfqpoint{1.254980in}{0.150000in}}{\pgfqpoint{5.490039in}{5.490039in}}%
\pgfusepath{clip}%
\pgfsetbuttcap%
\pgfsetroundjoin%
\definecolor{currentfill}{rgb}{0.278012,0.180367,0.486697}%
\pgfsetfillcolor{currentfill}%
\pgfsetfillopacity{0.700000}%
\pgfsetlinewidth{0.000000pt}%
\definecolor{currentstroke}{rgb}{0.000000,0.000000,0.000000}%
\pgfsetstrokecolor{currentstroke}%
\pgfsetdash{}{0pt}%
\pgfpathmoveto{\pgfqpoint{4.100765in}{2.356991in}}%
\pgfpathlineto{\pgfqpoint{4.113893in}{2.357570in}}%
\pgfpathlineto{\pgfqpoint{4.127029in}{2.358320in}}%
\pgfpathlineto{\pgfqpoint{4.140173in}{2.359241in}}%
\pgfpathlineto{\pgfqpoint{4.153326in}{2.360333in}}%
\pgfpathlineto{\pgfqpoint{4.160879in}{2.369682in}}%
\pgfpathlineto{\pgfqpoint{4.168427in}{2.379018in}}%
\pgfpathlineto{\pgfqpoint{4.175969in}{2.388343in}}%
\pgfpathlineto{\pgfqpoint{4.183507in}{2.397659in}}%
\pgfpathlineto{\pgfqpoint{4.170362in}{2.396684in}}%
\pgfpathlineto{\pgfqpoint{4.157226in}{2.395879in}}%
\pgfpathlineto{\pgfqpoint{4.144098in}{2.395245in}}%
\pgfpathlineto{\pgfqpoint{4.130979in}{2.394782in}}%
\pgfpathlineto{\pgfqpoint{4.123433in}{2.385340in}}%
\pgfpathlineto{\pgfqpoint{4.115882in}{2.375895in}}%
\pgfpathlineto{\pgfqpoint{4.108326in}{2.366446in}}%
\pgfpathlineto{\pgfqpoint{4.100765in}{2.356991in}}%
\pgfpathclose%
\pgfusepath{fill}%
\end{pgfscope}%
\begin{pgfscope}%
\pgfpathrectangle{\pgfqpoint{1.254980in}{0.150000in}}{\pgfqpoint{5.490039in}{5.490039in}}%
\pgfusepath{clip}%
\pgfsetbuttcap%
\pgfsetroundjoin%
\definecolor{currentfill}{rgb}{0.281924,0.089666,0.412415}%
\pgfsetfillcolor{currentfill}%
\pgfsetfillopacity{0.700000}%
\pgfsetlinewidth{0.000000pt}%
\definecolor{currentstroke}{rgb}{0.000000,0.000000,0.000000}%
\pgfsetstrokecolor{currentstroke}%
\pgfsetdash{}{0pt}%
\pgfpathmoveto{\pgfqpoint{3.281084in}{2.202988in}}%
\pgfpathlineto{\pgfqpoint{3.294090in}{2.195241in}}%
\pgfpathlineto{\pgfqpoint{3.307097in}{2.187703in}}%
\pgfpathlineto{\pgfqpoint{3.320105in}{2.180372in}}%
\pgfpathlineto{\pgfqpoint{3.333113in}{2.173247in}}%
\pgfpathlineto{\pgfqpoint{3.340945in}{2.182134in}}%
\pgfpathlineto{\pgfqpoint{3.348770in}{2.191072in}}%
\pgfpathlineto{\pgfqpoint{3.356589in}{2.200059in}}%
\pgfpathlineto{\pgfqpoint{3.364401in}{2.209095in}}%
\pgfpathlineto{\pgfqpoint{3.351407in}{2.216112in}}%
\pgfpathlineto{\pgfqpoint{3.338414in}{2.223336in}}%
\pgfpathlineto{\pgfqpoint{3.325421in}{2.230766in}}%
\pgfpathlineto{\pgfqpoint{3.312430in}{2.238404in}}%
\pgfpathlineto{\pgfqpoint{3.304603in}{2.229466in}}%
\pgfpathlineto{\pgfqpoint{3.296770in}{2.220583in}}%
\pgfpathlineto{\pgfqpoint{3.288930in}{2.211757in}}%
\pgfpathlineto{\pgfqpoint{3.281084in}{2.202988in}}%
\pgfpathclose%
\pgfusepath{fill}%
\end{pgfscope}%
\begin{pgfscope}%
\pgfpathrectangle{\pgfqpoint{1.254980in}{0.150000in}}{\pgfqpoint{5.490039in}{5.490039in}}%
\pgfusepath{clip}%
\pgfsetbuttcap%
\pgfsetroundjoin%
\definecolor{currentfill}{rgb}{0.281446,0.084320,0.407414}%
\pgfsetfillcolor{currentfill}%
\pgfsetfillopacity{0.700000}%
\pgfsetlinewidth{0.000000pt}%
\definecolor{currentstroke}{rgb}{0.000000,0.000000,0.000000}%
\pgfsetstrokecolor{currentstroke}%
\pgfsetdash{}{0pt}%
\pgfpathmoveto{\pgfqpoint{3.416394in}{2.183058in}}%
\pgfpathlineto{\pgfqpoint{3.429397in}{2.177052in}}%
\pgfpathlineto{\pgfqpoint{3.442402in}{2.171244in}}%
\pgfpathlineto{\pgfqpoint{3.455409in}{2.165635in}}%
\pgfpathlineto{\pgfqpoint{3.468419in}{2.160222in}}%
\pgfpathlineto{\pgfqpoint{3.476200in}{2.169486in}}%
\pgfpathlineto{\pgfqpoint{3.483975in}{2.178784in}}%
\pgfpathlineto{\pgfqpoint{3.491745in}{2.188115in}}%
\pgfpathlineto{\pgfqpoint{3.499509in}{2.197480in}}%
\pgfpathlineto{\pgfqpoint{3.486511in}{2.202813in}}%
\pgfpathlineto{\pgfqpoint{3.473516in}{2.208344in}}%
\pgfpathlineto{\pgfqpoint{3.460524in}{2.214071in}}%
\pgfpathlineto{\pgfqpoint{3.447533in}{2.219998in}}%
\pgfpathlineto{\pgfqpoint{3.439757in}{2.210703in}}%
\pgfpathlineto{\pgfqpoint{3.431975in}{2.201447in}}%
\pgfpathlineto{\pgfqpoint{3.424188in}{2.192232in}}%
\pgfpathlineto{\pgfqpoint{3.416394in}{2.183058in}}%
\pgfpathclose%
\pgfusepath{fill}%
\end{pgfscope}%
\begin{pgfscope}%
\pgfpathrectangle{\pgfqpoint{1.254980in}{0.150000in}}{\pgfqpoint{5.490039in}{5.490039in}}%
\pgfusepath{clip}%
\pgfsetbuttcap%
\pgfsetroundjoin%
\definecolor{currentfill}{rgb}{0.227802,0.326594,0.546532}%
\pgfsetfillcolor{currentfill}%
\pgfsetfillopacity{0.700000}%
\pgfsetlinewidth{0.000000pt}%
\definecolor{currentstroke}{rgb}{0.000000,0.000000,0.000000}%
\pgfsetstrokecolor{currentstroke}%
\pgfsetdash{}{0pt}%
\pgfpathmoveto{\pgfqpoint{4.680003in}{2.666498in}}%
\pgfpathlineto{\pgfqpoint{4.693348in}{2.670298in}}%
\pgfpathlineto{\pgfqpoint{4.706704in}{2.674258in}}%
\pgfpathlineto{\pgfqpoint{4.720073in}{2.678377in}}%
\pgfpathlineto{\pgfqpoint{4.733453in}{2.682656in}}%
\pgfpathlineto{\pgfqpoint{4.740795in}{2.690384in}}%
\pgfpathlineto{\pgfqpoint{4.748131in}{2.698124in}}%
\pgfpathlineto{\pgfqpoint{4.755463in}{2.705879in}}%
\pgfpathlineto{\pgfqpoint{4.762789in}{2.713654in}}%
\pgfpathlineto{\pgfqpoint{4.749421in}{2.709690in}}%
\pgfpathlineto{\pgfqpoint{4.736065in}{2.705886in}}%
\pgfpathlineto{\pgfqpoint{4.722721in}{2.702241in}}%
\pgfpathlineto{\pgfqpoint{4.709389in}{2.698756in}}%
\pgfpathlineto{\pgfqpoint{4.702049in}{2.690656in}}%
\pgfpathlineto{\pgfqpoint{4.694706in}{2.682582in}}%
\pgfpathlineto{\pgfqpoint{4.687357in}{2.674531in}}%
\pgfpathlineto{\pgfqpoint{4.680003in}{2.666498in}}%
\pgfpathclose%
\pgfusepath{fill}%
\end{pgfscope}%
\begin{pgfscope}%
\pgfpathrectangle{\pgfqpoint{1.254980in}{0.150000in}}{\pgfqpoint{5.490039in}{5.490039in}}%
\pgfusepath{clip}%
\pgfsetbuttcap%
\pgfsetroundjoin%
\definecolor{currentfill}{rgb}{0.280868,0.160771,0.472899}%
\pgfsetfillcolor{currentfill}%
\pgfsetfillopacity{0.700000}%
\pgfsetlinewidth{0.000000pt}%
\definecolor{currentstroke}{rgb}{0.000000,0.000000,0.000000}%
\pgfsetstrokecolor{currentstroke}%
\pgfsetdash{}{0pt}%
\pgfpathmoveto{\pgfqpoint{4.018009in}{2.318139in}}%
\pgfpathlineto{\pgfqpoint{4.031113in}{2.318116in}}%
\pgfpathlineto{\pgfqpoint{4.044225in}{2.318266in}}%
\pgfpathlineto{\pgfqpoint{4.057344in}{2.318589in}}%
\pgfpathlineto{\pgfqpoint{4.070472in}{2.319085in}}%
\pgfpathlineto{\pgfqpoint{4.078052in}{2.328577in}}%
\pgfpathlineto{\pgfqpoint{4.085628in}{2.338058in}}%
\pgfpathlineto{\pgfqpoint{4.093199in}{2.347529in}}%
\pgfpathlineto{\pgfqpoint{4.100765in}{2.356991in}}%
\pgfpathlineto{\pgfqpoint{4.087646in}{2.356584in}}%
\pgfpathlineto{\pgfqpoint{4.074534in}{2.356349in}}%
\pgfpathlineto{\pgfqpoint{4.061431in}{2.356287in}}%
\pgfpathlineto{\pgfqpoint{4.048335in}{2.356398in}}%
\pgfpathlineto{\pgfqpoint{4.040761in}{2.346838in}}%
\pgfpathlineto{\pgfqpoint{4.033182in}{2.337275in}}%
\pgfpathlineto{\pgfqpoint{4.025598in}{2.327710in}}%
\pgfpathlineto{\pgfqpoint{4.018009in}{2.318139in}}%
\pgfpathclose%
\pgfusepath{fill}%
\end{pgfscope}%
\begin{pgfscope}%
\pgfpathrectangle{\pgfqpoint{1.254980in}{0.150000in}}{\pgfqpoint{5.490039in}{5.490039in}}%
\pgfusepath{clip}%
\pgfsetbuttcap%
\pgfsetroundjoin%
\definecolor{currentfill}{rgb}{0.218130,0.347432,0.550038}%
\pgfsetfillcolor{currentfill}%
\pgfsetfillopacity{0.700000}%
\pgfsetlinewidth{0.000000pt}%
\definecolor{currentstroke}{rgb}{0.000000,0.000000,0.000000}%
\pgfsetstrokecolor{currentstroke}%
\pgfsetdash{}{0pt}%
\pgfpathmoveto{\pgfqpoint{4.762789in}{2.713654in}}%
\pgfpathlineto{\pgfqpoint{4.776170in}{2.717776in}}%
\pgfpathlineto{\pgfqpoint{4.789562in}{2.722057in}}%
\pgfpathlineto{\pgfqpoint{4.802967in}{2.726497in}}%
\pgfpathlineto{\pgfqpoint{4.816385in}{2.731095in}}%
\pgfpathlineto{\pgfqpoint{4.823693in}{2.738560in}}%
\pgfpathlineto{\pgfqpoint{4.830997in}{2.746044in}}%
\pgfpathlineto{\pgfqpoint{4.838295in}{2.753553in}}%
\pgfpathlineto{\pgfqpoint{4.845589in}{2.761090in}}%
\pgfpathlineto{\pgfqpoint{4.832185in}{2.756836in}}%
\pgfpathlineto{\pgfqpoint{4.818794in}{2.752740in}}%
\pgfpathlineto{\pgfqpoint{4.805414in}{2.748802in}}%
\pgfpathlineto{\pgfqpoint{4.792047in}{2.745022in}}%
\pgfpathlineto{\pgfqpoint{4.784740in}{2.737132in}}%
\pgfpathlineto{\pgfqpoint{4.777428in}{2.729276in}}%
\pgfpathlineto{\pgfqpoint{4.770111in}{2.721452in}}%
\pgfpathlineto{\pgfqpoint{4.762789in}{2.713654in}}%
\pgfpathclose%
\pgfusepath{fill}%
\end{pgfscope}%
\begin{pgfscope}%
\pgfpathrectangle{\pgfqpoint{1.254980in}{0.150000in}}{\pgfqpoint{5.490039in}{5.490039in}}%
\pgfusepath{clip}%
\pgfsetbuttcap%
\pgfsetroundjoin%
\definecolor{currentfill}{rgb}{0.208623,0.367752,0.552675}%
\pgfsetfillcolor{currentfill}%
\pgfsetfillopacity{0.700000}%
\pgfsetlinewidth{0.000000pt}%
\definecolor{currentstroke}{rgb}{0.000000,0.000000,0.000000}%
\pgfsetstrokecolor{currentstroke}%
\pgfsetdash{}{0pt}%
\pgfpathmoveto{\pgfqpoint{4.845589in}{2.761090in}}%
\pgfpathlineto{\pgfqpoint{4.859005in}{2.765501in}}%
\pgfpathlineto{\pgfqpoint{4.872434in}{2.770071in}}%
\pgfpathlineto{\pgfqpoint{4.885876in}{2.774798in}}%
\pgfpathlineto{\pgfqpoint{4.899331in}{2.779682in}}%
\pgfpathlineto{\pgfqpoint{4.906606in}{2.786889in}}%
\pgfpathlineto{\pgfqpoint{4.913876in}{2.794125in}}%
\pgfpathlineto{\pgfqpoint{4.921141in}{2.801396in}}%
\pgfpathlineto{\pgfqpoint{4.928401in}{2.808704in}}%
\pgfpathlineto{\pgfqpoint{4.914961in}{2.804193in}}%
\pgfpathlineto{\pgfqpoint{4.901534in}{2.799839in}}%
\pgfpathlineto{\pgfqpoint{4.888119in}{2.795642in}}%
\pgfpathlineto{\pgfqpoint{4.874717in}{2.791601in}}%
\pgfpathlineto{\pgfqpoint{4.867442in}{2.783910in}}%
\pgfpathlineto{\pgfqpoint{4.860162in}{2.776264in}}%
\pgfpathlineto{\pgfqpoint{4.852878in}{2.768659in}}%
\pgfpathlineto{\pgfqpoint{4.845589in}{2.761090in}}%
\pgfpathclose%
\pgfusepath{fill}%
\end{pgfscope}%
\begin{pgfscope}%
\pgfpathrectangle{\pgfqpoint{1.254980in}{0.150000in}}{\pgfqpoint{5.490039in}{5.490039in}}%
\pgfusepath{clip}%
\pgfsetbuttcap%
\pgfsetroundjoin%
\definecolor{currentfill}{rgb}{0.282623,0.140926,0.457517}%
\pgfsetfillcolor{currentfill}%
\pgfsetfillopacity{0.700000}%
\pgfsetlinewidth{0.000000pt}%
\definecolor{currentstroke}{rgb}{0.000000,0.000000,0.000000}%
\pgfsetstrokecolor{currentstroke}%
\pgfsetdash{}{0pt}%
\pgfpathmoveto{\pgfqpoint{3.935229in}{2.281387in}}%
\pgfpathlineto{\pgfqpoint{3.948312in}{2.280724in}}%
\pgfpathlineto{\pgfqpoint{3.961402in}{2.280237in}}%
\pgfpathlineto{\pgfqpoint{3.974499in}{2.279926in}}%
\pgfpathlineto{\pgfqpoint{3.987604in}{2.279789in}}%
\pgfpathlineto{\pgfqpoint{3.995212in}{2.289389in}}%
\pgfpathlineto{\pgfqpoint{4.002816in}{2.298980in}}%
\pgfpathlineto{\pgfqpoint{4.010415in}{2.308563in}}%
\pgfpathlineto{\pgfqpoint{4.018009in}{2.318139in}}%
\pgfpathlineto{\pgfqpoint{4.004913in}{2.318337in}}%
\pgfpathlineto{\pgfqpoint{3.991824in}{2.318708in}}%
\pgfpathlineto{\pgfqpoint{3.978742in}{2.319255in}}%
\pgfpathlineto{\pgfqpoint{3.965668in}{2.319978in}}%
\pgfpathlineto{\pgfqpoint{3.958066in}{2.310331in}}%
\pgfpathlineto{\pgfqpoint{3.950459in}{2.300685in}}%
\pgfpathlineto{\pgfqpoint{3.942847in}{2.291037in}}%
\pgfpathlineto{\pgfqpoint{3.935229in}{2.281387in}}%
\pgfpathclose%
\pgfusepath{fill}%
\end{pgfscope}%
\begin{pgfscope}%
\pgfpathrectangle{\pgfqpoint{1.254980in}{0.150000in}}{\pgfqpoint{5.490039in}{5.490039in}}%
\pgfusepath{clip}%
\pgfsetbuttcap%
\pgfsetroundjoin%
\definecolor{currentfill}{rgb}{0.199430,0.387607,0.554642}%
\pgfsetfillcolor{currentfill}%
\pgfsetfillopacity{0.700000}%
\pgfsetlinewidth{0.000000pt}%
\definecolor{currentstroke}{rgb}{0.000000,0.000000,0.000000}%
\pgfsetstrokecolor{currentstroke}%
\pgfsetdash{}{0pt}%
\pgfpathmoveto{\pgfqpoint{4.928401in}{2.808704in}}%
\pgfpathlineto{\pgfqpoint{4.941854in}{2.813372in}}%
\pgfpathlineto{\pgfqpoint{4.955320in}{2.818197in}}%
\pgfpathlineto{\pgfqpoint{4.968799in}{2.823178in}}%
\pgfpathlineto{\pgfqpoint{4.982292in}{2.828315in}}%
\pgfpathlineto{\pgfqpoint{4.989532in}{2.835275in}}%
\pgfpathlineto{\pgfqpoint{4.996768in}{2.842276in}}%
\pgfpathlineto{\pgfqpoint{5.003999in}{2.849322in}}%
\pgfpathlineto{\pgfqpoint{5.011226in}{2.856417in}}%
\pgfpathlineto{\pgfqpoint{4.997749in}{2.851682in}}%
\pgfpathlineto{\pgfqpoint{4.984286in}{2.847102in}}%
\pgfpathlineto{\pgfqpoint{4.970836in}{2.842678in}}%
\pgfpathlineto{\pgfqpoint{4.957399in}{2.838411in}}%
\pgfpathlineto{\pgfqpoint{4.950156in}{2.830904in}}%
\pgfpathlineto{\pgfqpoint{4.942909in}{2.823454in}}%
\pgfpathlineto{\pgfqpoint{4.935657in}{2.816056in}}%
\pgfpathlineto{\pgfqpoint{4.928401in}{2.808704in}}%
\pgfpathclose%
\pgfusepath{fill}%
\end{pgfscope}%
\begin{pgfscope}%
\pgfpathrectangle{\pgfqpoint{1.254980in}{0.150000in}}{\pgfqpoint{5.490039in}{5.490039in}}%
\pgfusepath{clip}%
\pgfsetbuttcap%
\pgfsetroundjoin%
\definecolor{currentfill}{rgb}{0.190631,0.407061,0.556089}%
\pgfsetfillcolor{currentfill}%
\pgfsetfillopacity{0.700000}%
\pgfsetlinewidth{0.000000pt}%
\definecolor{currentstroke}{rgb}{0.000000,0.000000,0.000000}%
\pgfsetstrokecolor{currentstroke}%
\pgfsetdash{}{0pt}%
\pgfpathmoveto{\pgfqpoint{5.011226in}{2.856417in}}%
\pgfpathlineto{\pgfqpoint{5.024715in}{2.861308in}}%
\pgfpathlineto{\pgfqpoint{5.038219in}{2.866355in}}%
\pgfpathlineto{\pgfqpoint{5.051735in}{2.871557in}}%
\pgfpathlineto{\pgfqpoint{5.065265in}{2.876914in}}%
\pgfpathlineto{\pgfqpoint{5.072471in}{2.883644in}}%
\pgfpathlineto{\pgfqpoint{5.079672in}{2.890426in}}%
\pgfpathlineto{\pgfqpoint{5.086869in}{2.897265in}}%
\pgfpathlineto{\pgfqpoint{5.094062in}{2.904167in}}%
\pgfpathlineto{\pgfqpoint{5.080549in}{2.899240in}}%
\pgfpathlineto{\pgfqpoint{5.067050in}{2.894468in}}%
\pgfpathlineto{\pgfqpoint{5.053564in}{2.889851in}}%
\pgfpathlineto{\pgfqpoint{5.040091in}{2.885389in}}%
\pgfpathlineto{\pgfqpoint{5.032881in}{2.878047in}}%
\pgfpathlineto{\pgfqpoint{5.025667in}{2.870775in}}%
\pgfpathlineto{\pgfqpoint{5.018448in}{2.863567in}}%
\pgfpathlineto{\pgfqpoint{5.011226in}{2.856417in}}%
\pgfpathclose%
\pgfusepath{fill}%
\end{pgfscope}%
\begin{pgfscope}%
\pgfpathrectangle{\pgfqpoint{1.254980in}{0.150000in}}{\pgfqpoint{5.490039in}{5.490039in}}%
\pgfusepath{clip}%
\pgfsetbuttcap%
\pgfsetroundjoin%
\definecolor{currentfill}{rgb}{0.262138,0.242286,0.520837}%
\pgfsetfillcolor{currentfill}%
\pgfsetfillopacity{0.700000}%
\pgfsetlinewidth{0.000000pt}%
\definecolor{currentstroke}{rgb}{0.000000,0.000000,0.000000}%
\pgfsetstrokecolor{currentstroke}%
\pgfsetdash{}{0pt}%
\pgfpathmoveto{\pgfqpoint{2.799795in}{2.511481in}}%
\pgfpathlineto{\pgfqpoint{2.812934in}{2.495936in}}%
\pgfpathlineto{\pgfqpoint{2.826065in}{2.480654in}}%
\pgfpathlineto{\pgfqpoint{2.839191in}{2.465634in}}%
\pgfpathlineto{\pgfqpoint{2.852310in}{2.450874in}}%
\pgfpathlineto{\pgfqpoint{2.860348in}{2.457913in}}%
\pgfpathlineto{\pgfqpoint{2.868378in}{2.465067in}}%
\pgfpathlineto{\pgfqpoint{2.876399in}{2.472335in}}%
\pgfpathlineto{\pgfqpoint{2.884411in}{2.479714in}}%
\pgfpathlineto{\pgfqpoint{2.871314in}{2.494305in}}%
\pgfpathlineto{\pgfqpoint{2.858212in}{2.509154in}}%
\pgfpathlineto{\pgfqpoint{2.845103in}{2.524265in}}%
\pgfpathlineto{\pgfqpoint{2.831988in}{2.539639in}}%
\pgfpathlineto{\pgfqpoint{2.823954in}{2.532420in}}%
\pgfpathlineto{\pgfqpoint{2.815910in}{2.525319in}}%
\pgfpathlineto{\pgfqpoint{2.807857in}{2.518339in}}%
\pgfpathlineto{\pgfqpoint{2.799795in}{2.511481in}}%
\pgfpathclose%
\pgfusepath{fill}%
\end{pgfscope}%
\begin{pgfscope}%
\pgfpathrectangle{\pgfqpoint{1.254980in}{0.150000in}}{\pgfqpoint{5.490039in}{5.490039in}}%
\pgfusepath{clip}%
\pgfsetbuttcap%
\pgfsetroundjoin%
\definecolor{currentfill}{rgb}{0.283091,0.110553,0.431554}%
\pgfsetfillcolor{currentfill}%
\pgfsetfillopacity{0.700000}%
\pgfsetlinewidth{0.000000pt}%
\definecolor{currentstroke}{rgb}{0.000000,0.000000,0.000000}%
\pgfsetstrokecolor{currentstroke}%
\pgfsetdash{}{0pt}%
\pgfpathmoveto{\pgfqpoint{3.145456in}{2.239219in}}%
\pgfpathlineto{\pgfqpoint{3.158481in}{2.229621in}}%
\pgfpathlineto{\pgfqpoint{3.171505in}{2.220243in}}%
\pgfpathlineto{\pgfqpoint{3.184528in}{2.211082in}}%
\pgfpathlineto{\pgfqpoint{3.197550in}{2.202138in}}%
\pgfpathlineto{\pgfqpoint{3.205438in}{2.210524in}}%
\pgfpathlineto{\pgfqpoint{3.213319in}{2.218977in}}%
\pgfpathlineto{\pgfqpoint{3.221194in}{2.227497in}}%
\pgfpathlineto{\pgfqpoint{3.229062in}{2.236083in}}%
\pgfpathlineto{\pgfqpoint{3.216056in}{2.244890in}}%
\pgfpathlineto{\pgfqpoint{3.203050in}{2.253914in}}%
\pgfpathlineto{\pgfqpoint{3.190043in}{2.263156in}}%
\pgfpathlineto{\pgfqpoint{3.177035in}{2.272616in}}%
\pgfpathlineto{\pgfqpoint{3.169150in}{2.264157in}}%
\pgfpathlineto{\pgfqpoint{3.161259in}{2.255770in}}%
\pgfpathlineto{\pgfqpoint{3.153361in}{2.247457in}}%
\pgfpathlineto{\pgfqpoint{3.145456in}{2.239219in}}%
\pgfpathclose%
\pgfusepath{fill}%
\end{pgfscope}%
\begin{pgfscope}%
\pgfpathrectangle{\pgfqpoint{1.254980in}{0.150000in}}{\pgfqpoint{5.490039in}{5.490039in}}%
\pgfusepath{clip}%
\pgfsetbuttcap%
\pgfsetroundjoin%
\definecolor{currentfill}{rgb}{0.281446,0.084320,0.407414}%
\pgfsetfillcolor{currentfill}%
\pgfsetfillopacity{0.700000}%
\pgfsetlinewidth{0.000000pt}%
\definecolor{currentstroke}{rgb}{0.000000,0.000000,0.000000}%
\pgfsetstrokecolor{currentstroke}%
\pgfsetdash{}{0pt}%
\pgfpathmoveto{\pgfqpoint{3.551530in}{2.178093in}}%
\pgfpathlineto{\pgfqpoint{3.564543in}{2.173729in}}%
\pgfpathlineto{\pgfqpoint{3.577561in}{2.169555in}}%
\pgfpathlineto{\pgfqpoint{3.590582in}{2.165571in}}%
\pgfpathlineto{\pgfqpoint{3.603607in}{2.161777in}}%
\pgfpathlineto{\pgfqpoint{3.611343in}{2.171298in}}%
\pgfpathlineto{\pgfqpoint{3.619073in}{2.180838in}}%
\pgfpathlineto{\pgfqpoint{3.626799in}{2.190396in}}%
\pgfpathlineto{\pgfqpoint{3.634519in}{2.199973in}}%
\pgfpathlineto{\pgfqpoint{3.621504in}{2.203717in}}%
\pgfpathlineto{\pgfqpoint{3.608494in}{2.207649in}}%
\pgfpathlineto{\pgfqpoint{3.595487in}{2.211772in}}%
\pgfpathlineto{\pgfqpoint{3.582485in}{2.216085in}}%
\pgfpathlineto{\pgfqpoint{3.574754in}{2.206549in}}%
\pgfpathlineto{\pgfqpoint{3.567018in}{2.197038in}}%
\pgfpathlineto{\pgfqpoint{3.559277in}{2.187553in}}%
\pgfpathlineto{\pgfqpoint{3.551530in}{2.178093in}}%
\pgfpathclose%
\pgfusepath{fill}%
\end{pgfscope}%
\begin{pgfscope}%
\pgfpathrectangle{\pgfqpoint{1.254980in}{0.150000in}}{\pgfqpoint{5.490039in}{5.490039in}}%
\pgfusepath{clip}%
\pgfsetbuttcap%
\pgfsetroundjoin%
\definecolor{currentfill}{rgb}{0.252194,0.269783,0.531579}%
\pgfsetfillcolor{currentfill}%
\pgfsetfillopacity{0.700000}%
\pgfsetlinewidth{0.000000pt}%
\definecolor{currentstroke}{rgb}{0.000000,0.000000,0.000000}%
\pgfsetstrokecolor{currentstroke}%
\pgfsetdash{}{0pt}%
\pgfpathmoveto{\pgfqpoint{2.747169in}{2.576347in}}%
\pgfpathlineto{\pgfqpoint{2.760337in}{2.559723in}}%
\pgfpathlineto{\pgfqpoint{2.773497in}{2.543372in}}%
\pgfpathlineto{\pgfqpoint{2.786650in}{2.527293in}}%
\pgfpathlineto{\pgfqpoint{2.799795in}{2.511481in}}%
\pgfpathlineto{\pgfqpoint{2.807857in}{2.518339in}}%
\pgfpathlineto{\pgfqpoint{2.815910in}{2.525319in}}%
\pgfpathlineto{\pgfqpoint{2.823954in}{2.532420in}}%
\pgfpathlineto{\pgfqpoint{2.831988in}{2.539639in}}%
\pgfpathlineto{\pgfqpoint{2.818867in}{2.555279in}}%
\pgfpathlineto{\pgfqpoint{2.805739in}{2.571187in}}%
\pgfpathlineto{\pgfqpoint{2.792603in}{2.587366in}}%
\pgfpathlineto{\pgfqpoint{2.779460in}{2.603817in}}%
\pgfpathlineto{\pgfqpoint{2.771401in}{2.596759in}}%
\pgfpathlineto{\pgfqpoint{2.763334in}{2.589827in}}%
\pgfpathlineto{\pgfqpoint{2.755256in}{2.583023in}}%
\pgfpathlineto{\pgfqpoint{2.747169in}{2.576347in}}%
\pgfpathclose%
\pgfusepath{fill}%
\end{pgfscope}%
\begin{pgfscope}%
\pgfpathrectangle{\pgfqpoint{1.254980in}{0.150000in}}{\pgfqpoint{5.490039in}{5.490039in}}%
\pgfusepath{clip}%
\pgfsetbuttcap%
\pgfsetroundjoin%
\definecolor{currentfill}{rgb}{0.270595,0.214069,0.507052}%
\pgfsetfillcolor{currentfill}%
\pgfsetfillopacity{0.700000}%
\pgfsetlinewidth{0.000000pt}%
\definecolor{currentstroke}{rgb}{0.000000,0.000000,0.000000}%
\pgfsetstrokecolor{currentstroke}%
\pgfsetdash{}{0pt}%
\pgfpathmoveto{\pgfqpoint{2.852310in}{2.450874in}}%
\pgfpathlineto{\pgfqpoint{2.865423in}{2.436370in}}%
\pgfpathlineto{\pgfqpoint{2.878531in}{2.422121in}}%
\pgfpathlineto{\pgfqpoint{2.891633in}{2.408126in}}%
\pgfpathlineto{\pgfqpoint{2.904730in}{2.394381in}}%
\pgfpathlineto{\pgfqpoint{2.912746in}{2.401601in}}%
\pgfpathlineto{\pgfqpoint{2.920753in}{2.408928in}}%
\pgfpathlineto{\pgfqpoint{2.928752in}{2.416361in}}%
\pgfpathlineto{\pgfqpoint{2.936742in}{2.423899in}}%
\pgfpathlineto{\pgfqpoint{2.923667in}{2.437475in}}%
\pgfpathlineto{\pgfqpoint{2.910587in}{2.451301in}}%
\pgfpathlineto{\pgfqpoint{2.897501in}{2.465380in}}%
\pgfpathlineto{\pgfqpoint{2.884411in}{2.479714in}}%
\pgfpathlineto{\pgfqpoint{2.876399in}{2.472335in}}%
\pgfpathlineto{\pgfqpoint{2.868378in}{2.465067in}}%
\pgfpathlineto{\pgfqpoint{2.860348in}{2.457913in}}%
\pgfpathlineto{\pgfqpoint{2.852310in}{2.450874in}}%
\pgfpathclose%
\pgfusepath{fill}%
\end{pgfscope}%
\begin{pgfscope}%
\pgfpathrectangle{\pgfqpoint{1.254980in}{0.150000in}}{\pgfqpoint{5.490039in}{5.490039in}}%
\pgfusepath{clip}%
\pgfsetbuttcap%
\pgfsetroundjoin%
\definecolor{currentfill}{rgb}{0.182256,0.426184,0.557120}%
\pgfsetfillcolor{currentfill}%
\pgfsetfillopacity{0.700000}%
\pgfsetlinewidth{0.000000pt}%
\definecolor{currentstroke}{rgb}{0.000000,0.000000,0.000000}%
\pgfsetstrokecolor{currentstroke}%
\pgfsetdash{}{0pt}%
\pgfpathmoveto{\pgfqpoint{5.094062in}{2.904167in}}%
\pgfpathlineto{\pgfqpoint{5.107588in}{2.909248in}}%
\pgfpathlineto{\pgfqpoint{5.121128in}{2.914484in}}%
\pgfpathlineto{\pgfqpoint{5.134682in}{2.919875in}}%
\pgfpathlineto{\pgfqpoint{5.148250in}{2.925420in}}%
\pgfpathlineto{\pgfqpoint{5.155421in}{2.931940in}}%
\pgfpathlineto{\pgfqpoint{5.162587in}{2.938525in}}%
\pgfpathlineto{\pgfqpoint{5.169750in}{2.945181in}}%
\pgfpathlineto{\pgfqpoint{5.176909in}{2.951913in}}%
\pgfpathlineto{\pgfqpoint{5.163360in}{2.946827in}}%
\pgfpathlineto{\pgfqpoint{5.149824in}{2.941895in}}%
\pgfpathlineto{\pgfqpoint{5.136303in}{2.937117in}}%
\pgfpathlineto{\pgfqpoint{5.122795in}{2.932494in}}%
\pgfpathlineto{\pgfqpoint{5.115617in}{2.925293in}}%
\pgfpathlineto{\pgfqpoint{5.108436in}{2.918176in}}%
\pgfpathlineto{\pgfqpoint{5.101251in}{2.911135in}}%
\pgfpathlineto{\pgfqpoint{5.094062in}{2.904167in}}%
\pgfpathclose%
\pgfusepath{fill}%
\end{pgfscope}%
\begin{pgfscope}%
\pgfpathrectangle{\pgfqpoint{1.254980in}{0.150000in}}{\pgfqpoint{5.490039in}{5.490039in}}%
\pgfusepath{clip}%
\pgfsetbuttcap%
\pgfsetroundjoin%
\definecolor{currentfill}{rgb}{0.283187,0.125848,0.444960}%
\pgfsetfillcolor{currentfill}%
\pgfsetfillopacity{0.700000}%
\pgfsetlinewidth{0.000000pt}%
\definecolor{currentstroke}{rgb}{0.000000,0.000000,0.000000}%
\pgfsetstrokecolor{currentstroke}%
\pgfsetdash{}{0pt}%
\pgfpathmoveto{\pgfqpoint{3.852414in}{2.247037in}}%
\pgfpathlineto{\pgfqpoint{3.865479in}{2.245698in}}%
\pgfpathlineto{\pgfqpoint{3.878550in}{2.244536in}}%
\pgfpathlineto{\pgfqpoint{3.891627in}{2.243553in}}%
\pgfpathlineto{\pgfqpoint{3.904711in}{2.242746in}}%
\pgfpathlineto{\pgfqpoint{3.912348in}{2.252414in}}%
\pgfpathlineto{\pgfqpoint{3.919980in}{2.262076in}}%
\pgfpathlineto{\pgfqpoint{3.927607in}{2.271733in}}%
\pgfpathlineto{\pgfqpoint{3.935229in}{2.281387in}}%
\pgfpathlineto{\pgfqpoint{3.922154in}{2.282226in}}%
\pgfpathlineto{\pgfqpoint{3.909085in}{2.283242in}}%
\pgfpathlineto{\pgfqpoint{3.896022in}{2.284435in}}%
\pgfpathlineto{\pgfqpoint{3.882966in}{2.285807in}}%
\pgfpathlineto{\pgfqpoint{3.875336in}{2.276111in}}%
\pgfpathlineto{\pgfqpoint{3.867700in}{2.266418in}}%
\pgfpathlineto{\pgfqpoint{3.860060in}{2.256727in}}%
\pgfpathlineto{\pgfqpoint{3.852414in}{2.247037in}}%
\pgfpathclose%
\pgfusepath{fill}%
\end{pgfscope}%
\begin{pgfscope}%
\pgfpathrectangle{\pgfqpoint{1.254980in}{0.150000in}}{\pgfqpoint{5.490039in}{5.490039in}}%
\pgfusepath{clip}%
\pgfsetbuttcap%
\pgfsetroundjoin%
\definecolor{currentfill}{rgb}{0.174274,0.445044,0.557792}%
\pgfsetfillcolor{currentfill}%
\pgfsetfillopacity{0.700000}%
\pgfsetlinewidth{0.000000pt}%
\definecolor{currentstroke}{rgb}{0.000000,0.000000,0.000000}%
\pgfsetstrokecolor{currentstroke}%
\pgfsetdash{}{0pt}%
\pgfpathmoveto{\pgfqpoint{5.176909in}{2.951913in}}%
\pgfpathlineto{\pgfqpoint{5.190472in}{2.957152in}}%
\pgfpathlineto{\pgfqpoint{5.204049in}{2.962545in}}%
\pgfpathlineto{\pgfqpoint{5.217640in}{2.968092in}}%
\pgfpathlineto{\pgfqpoint{5.231245in}{2.973792in}}%
\pgfpathlineto{\pgfqpoint{5.238381in}{2.980127in}}%
\pgfpathlineto{\pgfqpoint{5.245512in}{2.986543in}}%
\pgfpathlineto{\pgfqpoint{5.252641in}{2.993043in}}%
\pgfpathlineto{\pgfqpoint{5.259766in}{2.999635in}}%
\pgfpathlineto{\pgfqpoint{5.246181in}{2.994423in}}%
\pgfpathlineto{\pgfqpoint{5.232610in}{2.989364in}}%
\pgfpathlineto{\pgfqpoint{5.219053in}{2.984458in}}%
\pgfpathlineto{\pgfqpoint{5.205510in}{2.979705in}}%
\pgfpathlineto{\pgfqpoint{5.198364in}{2.972616in}}%
\pgfpathlineto{\pgfqpoint{5.191216in}{2.965625in}}%
\pgfpathlineto{\pgfqpoint{5.184064in}{2.958726in}}%
\pgfpathlineto{\pgfqpoint{5.176909in}{2.951913in}}%
\pgfpathclose%
\pgfusepath{fill}%
\end{pgfscope}%
\begin{pgfscope}%
\pgfpathrectangle{\pgfqpoint{1.254980in}{0.150000in}}{\pgfqpoint{5.490039in}{5.490039in}}%
\pgfusepath{clip}%
\pgfsetbuttcap%
\pgfsetroundjoin%
\definecolor{currentfill}{rgb}{0.239346,0.300855,0.540844}%
\pgfsetfillcolor{currentfill}%
\pgfsetfillopacity{0.700000}%
\pgfsetlinewidth{0.000000pt}%
\definecolor{currentstroke}{rgb}{0.000000,0.000000,0.000000}%
\pgfsetstrokecolor{currentstroke}%
\pgfsetdash{}{0pt}%
\pgfpathmoveto{\pgfqpoint{2.694416in}{2.645626in}}%
\pgfpathlineto{\pgfqpoint{2.707617in}{2.627884in}}%
\pgfpathlineto{\pgfqpoint{2.720810in}{2.610425in}}%
\pgfpathlineto{\pgfqpoint{2.733994in}{2.593247in}}%
\pgfpathlineto{\pgfqpoint{2.747169in}{2.576347in}}%
\pgfpathlineto{\pgfqpoint{2.755256in}{2.583023in}}%
\pgfpathlineto{\pgfqpoint{2.763334in}{2.589827in}}%
\pgfpathlineto{\pgfqpoint{2.771401in}{2.596759in}}%
\pgfpathlineto{\pgfqpoint{2.779460in}{2.603817in}}%
\pgfpathlineto{\pgfqpoint{2.766309in}{2.620544in}}%
\pgfpathlineto{\pgfqpoint{2.753150in}{2.637549in}}%
\pgfpathlineto{\pgfqpoint{2.739983in}{2.654834in}}%
\pgfpathlineto{\pgfqpoint{2.726808in}{2.672402in}}%
\pgfpathlineto{\pgfqpoint{2.718725in}{2.665507in}}%
\pgfpathlineto{\pgfqpoint{2.710632in}{2.658745in}}%
\pgfpathlineto{\pgfqpoint{2.702529in}{2.652118in}}%
\pgfpathlineto{\pgfqpoint{2.694416in}{2.645626in}}%
\pgfpathclose%
\pgfusepath{fill}%
\end{pgfscope}%
\begin{pgfscope}%
\pgfpathrectangle{\pgfqpoint{1.254980in}{0.150000in}}{\pgfqpoint{5.490039in}{5.490039in}}%
\pgfusepath{clip}%
\pgfsetbuttcap%
\pgfsetroundjoin%
\definecolor{currentfill}{rgb}{0.277134,0.185228,0.489898}%
\pgfsetfillcolor{currentfill}%
\pgfsetfillopacity{0.700000}%
\pgfsetlinewidth{0.000000pt}%
\definecolor{currentstroke}{rgb}{0.000000,0.000000,0.000000}%
\pgfsetstrokecolor{currentstroke}%
\pgfsetdash{}{0pt}%
\pgfpathmoveto{\pgfqpoint{2.904730in}{2.394381in}}%
\pgfpathlineto{\pgfqpoint{2.917822in}{2.380885in}}%
\pgfpathlineto{\pgfqpoint{2.930910in}{2.367636in}}%
\pgfpathlineto{\pgfqpoint{2.943993in}{2.354632in}}%
\pgfpathlineto{\pgfqpoint{2.957071in}{2.341871in}}%
\pgfpathlineto{\pgfqpoint{2.965065in}{2.349269in}}%
\pgfpathlineto{\pgfqpoint{2.973051in}{2.356768in}}%
\pgfpathlineto{\pgfqpoint{2.981028in}{2.364366in}}%
\pgfpathlineto{\pgfqpoint{2.988998in}{2.372061in}}%
\pgfpathlineto{\pgfqpoint{2.975940in}{2.384655in}}%
\pgfpathlineto{\pgfqpoint{2.962878in}{2.397491in}}%
\pgfpathlineto{\pgfqpoint{2.949812in}{2.410572in}}%
\pgfpathlineto{\pgfqpoint{2.936742in}{2.423899in}}%
\pgfpathlineto{\pgfqpoint{2.928752in}{2.416361in}}%
\pgfpathlineto{\pgfqpoint{2.920753in}{2.408928in}}%
\pgfpathlineto{\pgfqpoint{2.912746in}{2.401601in}}%
\pgfpathlineto{\pgfqpoint{2.904730in}{2.394381in}}%
\pgfpathclose%
\pgfusepath{fill}%
\end{pgfscope}%
\begin{pgfscope}%
\pgfpathrectangle{\pgfqpoint{1.254980in}{0.150000in}}{\pgfqpoint{5.490039in}{5.490039in}}%
\pgfusepath{clip}%
\pgfsetbuttcap%
\pgfsetroundjoin%
\definecolor{currentfill}{rgb}{0.166617,0.463708,0.558119}%
\pgfsetfillcolor{currentfill}%
\pgfsetfillopacity{0.700000}%
\pgfsetlinewidth{0.000000pt}%
\definecolor{currentstroke}{rgb}{0.000000,0.000000,0.000000}%
\pgfsetstrokecolor{currentstroke}%
\pgfsetdash{}{0pt}%
\pgfpathmoveto{\pgfqpoint{5.259766in}{2.999635in}}%
\pgfpathlineto{\pgfqpoint{5.273365in}{3.005000in}}%
\pgfpathlineto{\pgfqpoint{5.286978in}{3.010517in}}%
\pgfpathlineto{\pgfqpoint{5.300606in}{3.016188in}}%
\pgfpathlineto{\pgfqpoint{5.314249in}{3.022011in}}%
\pgfpathlineto{\pgfqpoint{5.321350in}{3.028192in}}%
\pgfpathlineto{\pgfqpoint{5.328447in}{3.034470in}}%
\pgfpathlineto{\pgfqpoint{5.335541in}{3.040848in}}%
\pgfpathlineto{\pgfqpoint{5.342633in}{3.047334in}}%
\pgfpathlineto{\pgfqpoint{5.329012in}{3.042028in}}%
\pgfpathlineto{\pgfqpoint{5.315406in}{3.036873in}}%
\pgfpathlineto{\pgfqpoint{5.301814in}{3.031871in}}%
\pgfpathlineto{\pgfqpoint{5.288237in}{3.027021in}}%
\pgfpathlineto{\pgfqpoint{5.281123in}{3.020010in}}%
\pgfpathlineto{\pgfqpoint{5.274007in}{3.013113in}}%
\pgfpathlineto{\pgfqpoint{5.266888in}{3.006323in}}%
\pgfpathlineto{\pgfqpoint{5.259766in}{2.999635in}}%
\pgfpathclose%
\pgfusepath{fill}%
\end{pgfscope}%
\begin{pgfscope}%
\pgfpathrectangle{\pgfqpoint{1.254980in}{0.150000in}}{\pgfqpoint{5.490039in}{5.490039in}}%
\pgfusepath{clip}%
\pgfsetbuttcap%
\pgfsetroundjoin%
\definecolor{currentfill}{rgb}{0.159194,0.482237,0.558073}%
\pgfsetfillcolor{currentfill}%
\pgfsetfillopacity{0.700000}%
\pgfsetlinewidth{0.000000pt}%
\definecolor{currentstroke}{rgb}{0.000000,0.000000,0.000000}%
\pgfsetstrokecolor{currentstroke}%
\pgfsetdash{}{0pt}%
\pgfpathmoveto{\pgfqpoint{5.342633in}{3.047334in}}%
\pgfpathlineto{\pgfqpoint{5.356268in}{3.052792in}}%
\pgfpathlineto{\pgfqpoint{5.369917in}{3.058402in}}%
\pgfpathlineto{\pgfqpoint{5.383582in}{3.064164in}}%
\pgfpathlineto{\pgfqpoint{5.397261in}{3.070078in}}%
\pgfpathlineto{\pgfqpoint{5.404327in}{3.076141in}}%
\pgfpathlineto{\pgfqpoint{5.411390in}{3.082317in}}%
\pgfpathlineto{\pgfqpoint{5.418451in}{3.088611in}}%
\pgfpathlineto{\pgfqpoint{5.425510in}{3.095030in}}%
\pgfpathlineto{\pgfqpoint{5.411855in}{3.089662in}}%
\pgfpathlineto{\pgfqpoint{5.398214in}{3.084444in}}%
\pgfpathlineto{\pgfqpoint{5.384588in}{3.079379in}}%
\pgfpathlineto{\pgfqpoint{5.370976in}{3.074464in}}%
\pgfpathlineto{\pgfqpoint{5.363893in}{3.067491in}}%
\pgfpathlineto{\pgfqpoint{5.356809in}{3.060649in}}%
\pgfpathlineto{\pgfqpoint{5.349722in}{3.053932in}}%
\pgfpathlineto{\pgfqpoint{5.342633in}{3.047334in}}%
\pgfpathclose%
\pgfusepath{fill}%
\end{pgfscope}%
\begin{pgfscope}%
\pgfpathrectangle{\pgfqpoint{1.254980in}{0.150000in}}{\pgfqpoint{5.490039in}{5.490039in}}%
\pgfusepath{clip}%
\pgfsetbuttcap%
\pgfsetroundjoin%
\definecolor{currentfill}{rgb}{0.283091,0.110553,0.431554}%
\pgfsetfillcolor{currentfill}%
\pgfsetfillopacity{0.700000}%
\pgfsetlinewidth{0.000000pt}%
\definecolor{currentstroke}{rgb}{0.000000,0.000000,0.000000}%
\pgfsetstrokecolor{currentstroke}%
\pgfsetdash{}{0pt}%
\pgfpathmoveto{\pgfqpoint{3.769550in}{2.215417in}}%
\pgfpathlineto{\pgfqpoint{3.782599in}{2.213362in}}%
\pgfpathlineto{\pgfqpoint{3.795654in}{2.211487in}}%
\pgfpathlineto{\pgfqpoint{3.808715in}{2.209794in}}%
\pgfpathlineto{\pgfqpoint{3.821782in}{2.208280in}}%
\pgfpathlineto{\pgfqpoint{3.829448in}{2.217970in}}%
\pgfpathlineto{\pgfqpoint{3.837108in}{2.227659in}}%
\pgfpathlineto{\pgfqpoint{3.844764in}{2.237348in}}%
\pgfpathlineto{\pgfqpoint{3.852414in}{2.247037in}}%
\pgfpathlineto{\pgfqpoint{3.839356in}{2.248556in}}%
\pgfpathlineto{\pgfqpoint{3.826304in}{2.250254in}}%
\pgfpathlineto{\pgfqpoint{3.813258in}{2.252133in}}%
\pgfpathlineto{\pgfqpoint{3.800218in}{2.254193in}}%
\pgfpathlineto{\pgfqpoint{3.792559in}{2.244489in}}%
\pgfpathlineto{\pgfqpoint{3.784894in}{2.234792in}}%
\pgfpathlineto{\pgfqpoint{3.777225in}{2.225102in}}%
\pgfpathlineto{\pgfqpoint{3.769550in}{2.215417in}}%
\pgfpathclose%
\pgfusepath{fill}%
\end{pgfscope}%
\begin{pgfscope}%
\pgfpathrectangle{\pgfqpoint{1.254980in}{0.150000in}}{\pgfqpoint{5.490039in}{5.490039in}}%
\pgfusepath{clip}%
\pgfsetbuttcap%
\pgfsetroundjoin%
\definecolor{currentfill}{rgb}{0.150476,0.504369,0.557430}%
\pgfsetfillcolor{currentfill}%
\pgfsetfillopacity{0.700000}%
\pgfsetlinewidth{0.000000pt}%
\definecolor{currentstroke}{rgb}{0.000000,0.000000,0.000000}%
\pgfsetstrokecolor{currentstroke}%
\pgfsetdash{}{0pt}%
\pgfpathmoveto{\pgfqpoint{5.425510in}{3.095030in}}%
\pgfpathlineto{\pgfqpoint{5.439181in}{3.100549in}}%
\pgfpathlineto{\pgfqpoint{5.452866in}{3.106220in}}%
\pgfpathlineto{\pgfqpoint{5.466566in}{3.112041in}}%
\pgfpathlineto{\pgfqpoint{5.480281in}{3.118014in}}%
\pgfpathlineto{\pgfqpoint{5.487313in}{3.124000in}}%
\pgfpathlineto{\pgfqpoint{5.494344in}{3.130116in}}%
\pgfpathlineto{\pgfqpoint{5.501372in}{3.136369in}}%
\pgfpathlineto{\pgfqpoint{5.508400in}{3.142765in}}%
\pgfpathlineto{\pgfqpoint{5.494710in}{3.137366in}}%
\pgfpathlineto{\pgfqpoint{5.481035in}{3.132118in}}%
\pgfpathlineto{\pgfqpoint{5.467375in}{3.127020in}}%
\pgfpathlineto{\pgfqpoint{5.453730in}{3.122073in}}%
\pgfpathlineto{\pgfqpoint{5.446677in}{3.115094in}}%
\pgfpathlineto{\pgfqpoint{5.439623in}{3.108265in}}%
\pgfpathlineto{\pgfqpoint{5.432568in}{3.101579in}}%
\pgfpathlineto{\pgfqpoint{5.425510in}{3.095030in}}%
\pgfpathclose%
\pgfusepath{fill}%
\end{pgfscope}%
\begin{pgfscope}%
\pgfpathrectangle{\pgfqpoint{1.254980in}{0.150000in}}{\pgfqpoint{5.490039in}{5.490039in}}%
\pgfusepath{clip}%
\pgfsetbuttcap%
\pgfsetroundjoin%
\definecolor{currentfill}{rgb}{0.281446,0.084320,0.407414}%
\pgfsetfillcolor{currentfill}%
\pgfsetfillopacity{0.700000}%
\pgfsetlinewidth{0.000000pt}%
\definecolor{currentstroke}{rgb}{0.000000,0.000000,0.000000}%
\pgfsetstrokecolor{currentstroke}%
\pgfsetdash{}{0pt}%
\pgfpathmoveto{\pgfqpoint{3.333113in}{2.173247in}}%
\pgfpathlineto{\pgfqpoint{3.346123in}{2.166326in}}%
\pgfpathlineto{\pgfqpoint{3.359134in}{2.159609in}}%
\pgfpathlineto{\pgfqpoint{3.372147in}{2.153094in}}%
\pgfpathlineto{\pgfqpoint{3.385161in}{2.146780in}}%
\pgfpathlineto{\pgfqpoint{3.392978in}{2.155786in}}%
\pgfpathlineto{\pgfqpoint{3.400789in}{2.164834in}}%
\pgfpathlineto{\pgfqpoint{3.408595in}{2.173925in}}%
\pgfpathlineto{\pgfqpoint{3.416394in}{2.183058in}}%
\pgfpathlineto{\pgfqpoint{3.403393in}{2.189265in}}%
\pgfpathlineto{\pgfqpoint{3.390394in}{2.195672in}}%
\pgfpathlineto{\pgfqpoint{3.377397in}{2.202282in}}%
\pgfpathlineto{\pgfqpoint{3.364401in}{2.209095in}}%
\pgfpathlineto{\pgfqpoint{3.356589in}{2.200059in}}%
\pgfpathlineto{\pgfqpoint{3.348770in}{2.191072in}}%
\pgfpathlineto{\pgfqpoint{3.340945in}{2.182134in}}%
\pgfpathlineto{\pgfqpoint{3.333113in}{2.173247in}}%
\pgfpathclose%
\pgfusepath{fill}%
\end{pgfscope}%
\begin{pgfscope}%
\pgfpathrectangle{\pgfqpoint{1.254980in}{0.150000in}}{\pgfqpoint{5.490039in}{5.490039in}}%
\pgfusepath{clip}%
\pgfsetbuttcap%
\pgfsetroundjoin%
\definecolor{currentfill}{rgb}{0.223925,0.334994,0.548053}%
\pgfsetfillcolor{currentfill}%
\pgfsetfillopacity{0.700000}%
\pgfsetlinewidth{0.000000pt}%
\definecolor{currentstroke}{rgb}{0.000000,0.000000,0.000000}%
\pgfsetstrokecolor{currentstroke}%
\pgfsetdash{}{0pt}%
\pgfpathmoveto{\pgfqpoint{2.641517in}{2.719484in}}%
\pgfpathlineto{\pgfqpoint{2.654757in}{2.700581in}}%
\pgfpathlineto{\pgfqpoint{2.667986in}{2.681972in}}%
\pgfpathlineto{\pgfqpoint{2.681206in}{2.663655in}}%
\pgfpathlineto{\pgfqpoint{2.694416in}{2.645626in}}%
\pgfpathlineto{\pgfqpoint{2.702529in}{2.652118in}}%
\pgfpathlineto{\pgfqpoint{2.710632in}{2.658745in}}%
\pgfpathlineto{\pgfqpoint{2.718725in}{2.665507in}}%
\pgfpathlineto{\pgfqpoint{2.726808in}{2.672402in}}%
\pgfpathlineto{\pgfqpoint{2.713623in}{2.690256in}}%
\pgfpathlineto{\pgfqpoint{2.700430in}{2.708399in}}%
\pgfpathlineto{\pgfqpoint{2.687227in}{2.726832in}}%
\pgfpathlineto{\pgfqpoint{2.674015in}{2.745559in}}%
\pgfpathlineto{\pgfqpoint{2.665906in}{2.738828in}}%
\pgfpathlineto{\pgfqpoint{2.657787in}{2.732238in}}%
\pgfpathlineto{\pgfqpoint{2.649657in}{2.725790in}}%
\pgfpathlineto{\pgfqpoint{2.641517in}{2.719484in}}%
\pgfpathclose%
\pgfusepath{fill}%
\end{pgfscope}%
\begin{pgfscope}%
\pgfpathrectangle{\pgfqpoint{1.254980in}{0.150000in}}{\pgfqpoint{5.490039in}{5.490039in}}%
\pgfusepath{clip}%
\pgfsetbuttcap%
\pgfsetroundjoin%
\definecolor{currentfill}{rgb}{0.280255,0.165693,0.476498}%
\pgfsetfillcolor{currentfill}%
\pgfsetfillopacity{0.700000}%
\pgfsetlinewidth{0.000000pt}%
\definecolor{currentstroke}{rgb}{0.000000,0.000000,0.000000}%
\pgfsetstrokecolor{currentstroke}%
\pgfsetdash{}{0pt}%
\pgfpathmoveto{\pgfqpoint{2.957071in}{2.341871in}}%
\pgfpathlineto{\pgfqpoint{2.970146in}{2.329350in}}%
\pgfpathlineto{\pgfqpoint{2.983217in}{2.317069in}}%
\pgfpathlineto{\pgfqpoint{2.996284in}{2.305026in}}%
\pgfpathlineto{\pgfqpoint{3.009348in}{2.293218in}}%
\pgfpathlineto{\pgfqpoint{3.017321in}{2.300794in}}%
\pgfpathlineto{\pgfqpoint{3.025286in}{2.308463in}}%
\pgfpathlineto{\pgfqpoint{3.033243in}{2.316225in}}%
\pgfpathlineto{\pgfqpoint{3.041193in}{2.324077in}}%
\pgfpathlineto{\pgfqpoint{3.028149in}{2.335718in}}%
\pgfpathlineto{\pgfqpoint{3.015102in}{2.347595in}}%
\pgfpathlineto{\pgfqpoint{3.002052in}{2.359709in}}%
\pgfpathlineto{\pgfqpoint{2.988998in}{2.372061in}}%
\pgfpathlineto{\pgfqpoint{2.981028in}{2.364366in}}%
\pgfpathlineto{\pgfqpoint{2.973051in}{2.356768in}}%
\pgfpathlineto{\pgfqpoint{2.965065in}{2.349269in}}%
\pgfpathlineto{\pgfqpoint{2.957071in}{2.341871in}}%
\pgfpathclose%
\pgfusepath{fill}%
\end{pgfscope}%
\begin{pgfscope}%
\pgfpathrectangle{\pgfqpoint{1.254980in}{0.150000in}}{\pgfqpoint{5.490039in}{5.490039in}}%
\pgfusepath{clip}%
\pgfsetbuttcap%
\pgfsetroundjoin%
\definecolor{currentfill}{rgb}{0.143343,0.522773,0.556295}%
\pgfsetfillcolor{currentfill}%
\pgfsetfillopacity{0.700000}%
\pgfsetlinewidth{0.000000pt}%
\definecolor{currentstroke}{rgb}{0.000000,0.000000,0.000000}%
\pgfsetstrokecolor{currentstroke}%
\pgfsetdash{}{0pt}%
\pgfpathmoveto{\pgfqpoint{5.508400in}{3.142765in}}%
\pgfpathlineto{\pgfqpoint{5.522104in}{3.148314in}}%
\pgfpathlineto{\pgfqpoint{5.535824in}{3.154013in}}%
\pgfpathlineto{\pgfqpoint{5.549559in}{3.159862in}}%
\pgfpathlineto{\pgfqpoint{5.563309in}{3.165862in}}%
\pgfpathlineto{\pgfqpoint{5.570309in}{3.171816in}}%
\pgfpathlineto{\pgfqpoint{5.577308in}{3.177919in}}%
\pgfpathlineto{\pgfqpoint{5.584305in}{3.184179in}}%
\pgfpathlineto{\pgfqpoint{5.591303in}{3.190601in}}%
\pgfpathlineto{\pgfqpoint{5.577580in}{3.185204in}}%
\pgfpathlineto{\pgfqpoint{5.563872in}{3.179956in}}%
\pgfpathlineto{\pgfqpoint{5.550179in}{3.174858in}}%
\pgfpathlineto{\pgfqpoint{5.536501in}{3.169910in}}%
\pgfpathlineto{\pgfqpoint{5.529476in}{3.162876in}}%
\pgfpathlineto{\pgfqpoint{5.522452in}{3.156012in}}%
\pgfpathlineto{\pgfqpoint{5.515426in}{3.149311in}}%
\pgfpathlineto{\pgfqpoint{5.508400in}{3.142765in}}%
\pgfpathclose%
\pgfusepath{fill}%
\end{pgfscope}%
\begin{pgfscope}%
\pgfpathrectangle{\pgfqpoint{1.254980in}{0.150000in}}{\pgfqpoint{5.490039in}{5.490039in}}%
\pgfusepath{clip}%
\pgfsetbuttcap%
\pgfsetroundjoin%
\definecolor{currentfill}{rgb}{0.136408,0.541173,0.554483}%
\pgfsetfillcolor{currentfill}%
\pgfsetfillopacity{0.700000}%
\pgfsetlinewidth{0.000000pt}%
\definecolor{currentstroke}{rgb}{0.000000,0.000000,0.000000}%
\pgfsetstrokecolor{currentstroke}%
\pgfsetdash{}{0pt}%
\pgfpathmoveto{\pgfqpoint{5.591303in}{3.190601in}}%
\pgfpathlineto{\pgfqpoint{5.605041in}{3.196148in}}%
\pgfpathlineto{\pgfqpoint{5.618794in}{3.201844in}}%
\pgfpathlineto{\pgfqpoint{5.632563in}{3.207690in}}%
\pgfpathlineto{\pgfqpoint{5.646348in}{3.213685in}}%
\pgfpathlineto{\pgfqpoint{5.653316in}{3.219657in}}%
\pgfpathlineto{\pgfqpoint{5.660285in}{3.225799in}}%
\pgfpathlineto{\pgfqpoint{5.667253in}{3.232119in}}%
\pgfpathlineto{\pgfqpoint{5.674222in}{3.238622in}}%
\pgfpathlineto{\pgfqpoint{5.660467in}{3.233258in}}%
\pgfpathlineto{\pgfqpoint{5.646727in}{3.228042in}}%
\pgfpathlineto{\pgfqpoint{5.633002in}{3.222976in}}%
\pgfpathlineto{\pgfqpoint{5.619292in}{3.218058in}}%
\pgfpathlineto{\pgfqpoint{5.612294in}{3.210915in}}%
\pgfpathlineto{\pgfqpoint{5.605297in}{3.203962in}}%
\pgfpathlineto{\pgfqpoint{5.598300in}{3.197194in}}%
\pgfpathlineto{\pgfqpoint{5.591303in}{3.190601in}}%
\pgfpathclose%
\pgfusepath{fill}%
\end{pgfscope}%
\begin{pgfscope}%
\pgfpathrectangle{\pgfqpoint{1.254980in}{0.150000in}}{\pgfqpoint{5.490039in}{5.490039in}}%
\pgfusepath{clip}%
\pgfsetbuttcap%
\pgfsetroundjoin%
\definecolor{currentfill}{rgb}{0.280894,0.078907,0.402329}%
\pgfsetfillcolor{currentfill}%
\pgfsetfillopacity{0.700000}%
\pgfsetlinewidth{0.000000pt}%
\definecolor{currentstroke}{rgb}{0.000000,0.000000,0.000000}%
\pgfsetstrokecolor{currentstroke}%
\pgfsetdash{}{0pt}%
\pgfpathmoveto{\pgfqpoint{3.468419in}{2.160222in}}%
\pgfpathlineto{\pgfqpoint{3.481432in}{2.155005in}}%
\pgfpathlineto{\pgfqpoint{3.494447in}{2.149983in}}%
\pgfpathlineto{\pgfqpoint{3.507466in}{2.145154in}}%
\pgfpathlineto{\pgfqpoint{3.520487in}{2.140519in}}%
\pgfpathlineto{\pgfqpoint{3.528256in}{2.149872in}}%
\pgfpathlineto{\pgfqpoint{3.536020in}{2.159253in}}%
\pgfpathlineto{\pgfqpoint{3.543778in}{2.168660in}}%
\pgfpathlineto{\pgfqpoint{3.551530in}{2.178093in}}%
\pgfpathlineto{\pgfqpoint{3.538520in}{2.182650in}}%
\pgfpathlineto{\pgfqpoint{3.525513in}{2.187399in}}%
\pgfpathlineto{\pgfqpoint{3.512509in}{2.192342in}}%
\pgfpathlineto{\pgfqpoint{3.499509in}{2.197480in}}%
\pgfpathlineto{\pgfqpoint{3.491745in}{2.188115in}}%
\pgfpathlineto{\pgfqpoint{3.483975in}{2.178784in}}%
\pgfpathlineto{\pgfqpoint{3.476200in}{2.169486in}}%
\pgfpathlineto{\pgfqpoint{3.468419in}{2.160222in}}%
\pgfpathclose%
\pgfusepath{fill}%
\end{pgfscope}%
\begin{pgfscope}%
\pgfpathrectangle{\pgfqpoint{1.254980in}{0.150000in}}{\pgfqpoint{5.490039in}{5.490039in}}%
\pgfusepath{clip}%
\pgfsetbuttcap%
\pgfsetroundjoin%
\definecolor{currentfill}{rgb}{0.282656,0.100196,0.422160}%
\pgfsetfillcolor{currentfill}%
\pgfsetfillopacity{0.700000}%
\pgfsetlinewidth{0.000000pt}%
\definecolor{currentstroke}{rgb}{0.000000,0.000000,0.000000}%
\pgfsetstrokecolor{currentstroke}%
\pgfsetdash{}{0pt}%
\pgfpathmoveto{\pgfqpoint{3.197550in}{2.202138in}}%
\pgfpathlineto{\pgfqpoint{3.210572in}{2.193409in}}%
\pgfpathlineto{\pgfqpoint{3.223593in}{2.184894in}}%
\pgfpathlineto{\pgfqpoint{3.236614in}{2.176591in}}%
\pgfpathlineto{\pgfqpoint{3.249636in}{2.168499in}}%
\pgfpathlineto{\pgfqpoint{3.257508in}{2.177031in}}%
\pgfpathlineto{\pgfqpoint{3.265373in}{2.185624in}}%
\pgfpathlineto{\pgfqpoint{3.273232in}{2.194277in}}%
\pgfpathlineto{\pgfqpoint{3.281084in}{2.202988in}}%
\pgfpathlineto{\pgfqpoint{3.268078in}{2.210944in}}%
\pgfpathlineto{\pgfqpoint{3.255073in}{2.219111in}}%
\pgfpathlineto{\pgfqpoint{3.242068in}{2.227490in}}%
\pgfpathlineto{\pgfqpoint{3.229062in}{2.236083in}}%
\pgfpathlineto{\pgfqpoint{3.221194in}{2.227497in}}%
\pgfpathlineto{\pgfqpoint{3.213319in}{2.218977in}}%
\pgfpathlineto{\pgfqpoint{3.205438in}{2.210524in}}%
\pgfpathlineto{\pgfqpoint{3.197550in}{2.202138in}}%
\pgfpathclose%
\pgfusepath{fill}%
\end{pgfscope}%
\begin{pgfscope}%
\pgfpathrectangle{\pgfqpoint{1.254980in}{0.150000in}}{\pgfqpoint{5.490039in}{5.490039in}}%
\pgfusepath{clip}%
\pgfsetbuttcap%
\pgfsetroundjoin%
\definecolor{currentfill}{rgb}{0.129933,0.559582,0.551864}%
\pgfsetfillcolor{currentfill}%
\pgfsetfillopacity{0.700000}%
\pgfsetlinewidth{0.000000pt}%
\definecolor{currentstroke}{rgb}{0.000000,0.000000,0.000000}%
\pgfsetstrokecolor{currentstroke}%
\pgfsetdash{}{0pt}%
\pgfpathmoveto{\pgfqpoint{5.674222in}{3.238622in}}%
\pgfpathlineto{\pgfqpoint{5.687993in}{3.244135in}}%
\pgfpathlineto{\pgfqpoint{5.701779in}{3.249797in}}%
\pgfpathlineto{\pgfqpoint{5.715581in}{3.255607in}}%
\pgfpathlineto{\pgfqpoint{5.729399in}{3.261567in}}%
\pgfpathlineto{\pgfqpoint{5.736338in}{3.267613in}}%
\pgfpathlineto{\pgfqpoint{5.743278in}{3.273851in}}%
\pgfpathlineto{\pgfqpoint{5.750220in}{3.280288in}}%
\pgfpathlineto{\pgfqpoint{5.757162in}{3.286932in}}%
\pgfpathlineto{\pgfqpoint{5.743376in}{3.281632in}}%
\pgfpathlineto{\pgfqpoint{5.729605in}{3.276480in}}%
\pgfpathlineto{\pgfqpoint{5.715849in}{3.271475in}}%
\pgfpathlineto{\pgfqpoint{5.702109in}{3.266619in}}%
\pgfpathlineto{\pgfqpoint{5.695135in}{3.259308in}}%
\pgfpathlineto{\pgfqpoint{5.688163in}{3.252210in}}%
\pgfpathlineto{\pgfqpoint{5.681192in}{3.245317in}}%
\pgfpathlineto{\pgfqpoint{5.674222in}{3.238622in}}%
\pgfpathclose%
\pgfusepath{fill}%
\end{pgfscope}%
\begin{pgfscope}%
\pgfpathrectangle{\pgfqpoint{1.254980in}{0.150000in}}{\pgfqpoint{5.490039in}{5.490039in}}%
\pgfusepath{clip}%
\pgfsetbuttcap%
\pgfsetroundjoin%
\definecolor{currentfill}{rgb}{0.282327,0.094955,0.417331}%
\pgfsetfillcolor{currentfill}%
\pgfsetfillopacity{0.700000}%
\pgfsetlinewidth{0.000000pt}%
\definecolor{currentstroke}{rgb}{0.000000,0.000000,0.000000}%
\pgfsetstrokecolor{currentstroke}%
\pgfsetdash{}{0pt}%
\pgfpathmoveto{\pgfqpoint{3.686621in}{2.186874in}}%
\pgfpathlineto{\pgfqpoint{3.699658in}{2.184063in}}%
\pgfpathlineto{\pgfqpoint{3.712700in}{2.181437in}}%
\pgfpathlineto{\pgfqpoint{3.725748in}{2.178994in}}%
\pgfpathlineto{\pgfqpoint{3.738801in}{2.176733in}}%
\pgfpathlineto{\pgfqpoint{3.746496in}{2.186397in}}%
\pgfpathlineto{\pgfqpoint{3.754186in}{2.196065in}}%
\pgfpathlineto{\pgfqpoint{3.761871in}{2.205738in}}%
\pgfpathlineto{\pgfqpoint{3.769550in}{2.215417in}}%
\pgfpathlineto{\pgfqpoint{3.756507in}{2.217654in}}%
\pgfpathlineto{\pgfqpoint{3.743469in}{2.220074in}}%
\pgfpathlineto{\pgfqpoint{3.730436in}{2.222677in}}%
\pgfpathlineto{\pgfqpoint{3.717408in}{2.225465in}}%
\pgfpathlineto{\pgfqpoint{3.709719in}{2.215799in}}%
\pgfpathlineto{\pgfqpoint{3.702025in}{2.206145in}}%
\pgfpathlineto{\pgfqpoint{3.694325in}{2.196503in}}%
\pgfpathlineto{\pgfqpoint{3.686621in}{2.186874in}}%
\pgfpathclose%
\pgfusepath{fill}%
\end{pgfscope}%
\begin{pgfscope}%
\pgfpathrectangle{\pgfqpoint{1.254980in}{0.150000in}}{\pgfqpoint{5.490039in}{5.490039in}}%
\pgfusepath{clip}%
\pgfsetbuttcap%
\pgfsetroundjoin%
\definecolor{currentfill}{rgb}{0.282623,0.140926,0.457517}%
\pgfsetfillcolor{currentfill}%
\pgfsetfillopacity{0.700000}%
\pgfsetlinewidth{0.000000pt}%
\definecolor{currentstroke}{rgb}{0.000000,0.000000,0.000000}%
\pgfsetstrokecolor{currentstroke}%
\pgfsetdash{}{0pt}%
\pgfpathmoveto{\pgfqpoint{3.009348in}{2.293218in}}%
\pgfpathlineto{\pgfqpoint{3.022408in}{2.281644in}}%
\pgfpathlineto{\pgfqpoint{3.035466in}{2.270302in}}%
\pgfpathlineto{\pgfqpoint{3.048521in}{2.259191in}}%
\pgfpathlineto{\pgfqpoint{3.061574in}{2.248308in}}%
\pgfpathlineto{\pgfqpoint{3.069527in}{2.256061in}}%
\pgfpathlineto{\pgfqpoint{3.077472in}{2.263900in}}%
\pgfpathlineto{\pgfqpoint{3.085410in}{2.271824in}}%
\pgfpathlineto{\pgfqpoint{3.093341in}{2.279832in}}%
\pgfpathlineto{\pgfqpoint{3.080308in}{2.290549in}}%
\pgfpathlineto{\pgfqpoint{3.067272in}{2.301494in}}%
\pgfpathlineto{\pgfqpoint{3.054234in}{2.312670in}}%
\pgfpathlineto{\pgfqpoint{3.041193in}{2.324077in}}%
\pgfpathlineto{\pgfqpoint{3.033243in}{2.316225in}}%
\pgfpathlineto{\pgfqpoint{3.025286in}{2.308463in}}%
\pgfpathlineto{\pgfqpoint{3.017321in}{2.300794in}}%
\pgfpathlineto{\pgfqpoint{3.009348in}{2.293218in}}%
\pgfpathclose%
\pgfusepath{fill}%
\end{pgfscope}%
\begin{pgfscope}%
\pgfpathrectangle{\pgfqpoint{1.254980in}{0.150000in}}{\pgfqpoint{5.490039in}{5.490039in}}%
\pgfusepath{clip}%
\pgfsetbuttcap%
\pgfsetroundjoin%
\definecolor{currentfill}{rgb}{0.208623,0.367752,0.552675}%
\pgfsetfillcolor{currentfill}%
\pgfsetfillopacity{0.700000}%
\pgfsetlinewidth{0.000000pt}%
\definecolor{currentstroke}{rgb}{0.000000,0.000000,0.000000}%
\pgfsetstrokecolor{currentstroke}%
\pgfsetdash{}{0pt}%
\pgfpathmoveto{\pgfqpoint{2.588455in}{2.798099in}}%
\pgfpathlineto{\pgfqpoint{2.601737in}{2.777989in}}%
\pgfpathlineto{\pgfqpoint{2.615008in}{2.758186in}}%
\pgfpathlineto{\pgfqpoint{2.628268in}{2.738685in}}%
\pgfpathlineto{\pgfqpoint{2.641517in}{2.719484in}}%
\pgfpathlineto{\pgfqpoint{2.649657in}{2.725790in}}%
\pgfpathlineto{\pgfqpoint{2.657787in}{2.732238in}}%
\pgfpathlineto{\pgfqpoint{2.665906in}{2.738828in}}%
\pgfpathlineto{\pgfqpoint{2.674015in}{2.745559in}}%
\pgfpathlineto{\pgfqpoint{2.660792in}{2.764584in}}%
\pgfpathlineto{\pgfqpoint{2.647559in}{2.783908in}}%
\pgfpathlineto{\pgfqpoint{2.634316in}{2.803534in}}%
\pgfpathlineto{\pgfqpoint{2.621062in}{2.823467in}}%
\pgfpathlineto{\pgfqpoint{2.612926in}{2.816902in}}%
\pgfpathlineto{\pgfqpoint{2.604780in}{2.810484in}}%
\pgfpathlineto{\pgfqpoint{2.596623in}{2.804217in}}%
\pgfpathlineto{\pgfqpoint{2.588455in}{2.798099in}}%
\pgfpathclose%
\pgfusepath{fill}%
\end{pgfscope}%
\begin{pgfscope}%
\pgfpathrectangle{\pgfqpoint{1.254980in}{0.150000in}}{\pgfqpoint{5.490039in}{5.490039in}}%
\pgfusepath{clip}%
\pgfsetbuttcap%
\pgfsetroundjoin%
\definecolor{currentfill}{rgb}{0.124395,0.578002,0.548287}%
\pgfsetfillcolor{currentfill}%
\pgfsetfillopacity{0.700000}%
\pgfsetlinewidth{0.000000pt}%
\definecolor{currentstroke}{rgb}{0.000000,0.000000,0.000000}%
\pgfsetstrokecolor{currentstroke}%
\pgfsetdash{}{0pt}%
\pgfpathmoveto{\pgfqpoint{5.757162in}{3.286932in}}%
\pgfpathlineto{\pgfqpoint{5.770965in}{3.292379in}}%
\pgfpathlineto{\pgfqpoint{5.784783in}{3.297975in}}%
\pgfpathlineto{\pgfqpoint{5.798616in}{3.303720in}}%
\pgfpathlineto{\pgfqpoint{5.812466in}{3.309612in}}%
\pgfpathlineto{\pgfqpoint{5.819379in}{3.315792in}}%
\pgfpathlineto{\pgfqpoint{5.826293in}{3.322188in}}%
\pgfpathlineto{\pgfqpoint{5.833210in}{3.328806in}}%
\pgfpathlineto{\pgfqpoint{5.819385in}{3.323428in}}%
\pgfpathlineto{\pgfqpoint{5.805575in}{3.318196in}}%
\pgfpathlineto{\pgfqpoint{5.791782in}{3.313112in}}%
\pgfpathlineto{\pgfqpoint{5.778004in}{3.308176in}}%
\pgfpathlineto{\pgfqpoint{5.771054in}{3.300868in}}%
\pgfpathlineto{\pgfqpoint{5.764107in}{3.293789in}}%
\pgfpathlineto{\pgfqpoint{5.757162in}{3.286932in}}%
\pgfpathclose%
\pgfusepath{fill}%
\end{pgfscope}%
\begin{pgfscope}%
\pgfpathrectangle{\pgfqpoint{1.254980in}{0.150000in}}{\pgfqpoint{5.490039in}{5.490039in}}%
\pgfusepath{clip}%
\pgfsetbuttcap%
\pgfsetroundjoin%
\definecolor{currentfill}{rgb}{0.281446,0.084320,0.407414}%
\pgfsetfillcolor{currentfill}%
\pgfsetfillopacity{0.700000}%
\pgfsetlinewidth{0.000000pt}%
\definecolor{currentstroke}{rgb}{0.000000,0.000000,0.000000}%
\pgfsetstrokecolor{currentstroke}%
\pgfsetdash{}{0pt}%
\pgfpathmoveto{\pgfqpoint{3.603607in}{2.161777in}}%
\pgfpathlineto{\pgfqpoint{3.616636in}{2.158171in}}%
\pgfpathlineto{\pgfqpoint{3.629669in}{2.154752in}}%
\pgfpathlineto{\pgfqpoint{3.642707in}{2.151520in}}%
\pgfpathlineto{\pgfqpoint{3.655750in}{2.148473in}}%
\pgfpathlineto{\pgfqpoint{3.663475in}{2.158055in}}%
\pgfpathlineto{\pgfqpoint{3.671196in}{2.167650in}}%
\pgfpathlineto{\pgfqpoint{3.678911in}{2.177256in}}%
\pgfpathlineto{\pgfqpoint{3.686621in}{2.186874in}}%
\pgfpathlineto{\pgfqpoint{3.673588in}{2.189869in}}%
\pgfpathlineto{\pgfqpoint{3.660560in}{2.193051in}}%
\pgfpathlineto{\pgfqpoint{3.647537in}{2.196418in}}%
\pgfpathlineto{\pgfqpoint{3.634519in}{2.199973in}}%
\pgfpathlineto{\pgfqpoint{3.626799in}{2.190396in}}%
\pgfpathlineto{\pgfqpoint{3.619073in}{2.180838in}}%
\pgfpathlineto{\pgfqpoint{3.611343in}{2.171298in}}%
\pgfpathlineto{\pgfqpoint{3.603607in}{2.161777in}}%
\pgfpathclose%
\pgfusepath{fill}%
\end{pgfscope}%
\begin{pgfscope}%
\pgfpathrectangle{\pgfqpoint{1.254980in}{0.150000in}}{\pgfqpoint{5.490039in}{5.490039in}}%
\pgfusepath{clip}%
\pgfsetbuttcap%
\pgfsetroundjoin%
\definecolor{currentfill}{rgb}{0.270595,0.214069,0.507052}%
\pgfsetfillcolor{currentfill}%
\pgfsetfillopacity{0.700000}%
\pgfsetlinewidth{0.000000pt}%
\definecolor{currentstroke}{rgb}{0.000000,0.000000,0.000000}%
\pgfsetstrokecolor{currentstroke}%
\pgfsetdash{}{0pt}%
\pgfpathmoveto{\pgfqpoint{4.236175in}{2.403251in}}%
\pgfpathlineto{\pgfqpoint{4.249365in}{2.405071in}}%
\pgfpathlineto{\pgfqpoint{4.262565in}{2.407058in}}%
\pgfpathlineto{\pgfqpoint{4.275774in}{2.409212in}}%
\pgfpathlineto{\pgfqpoint{4.288992in}{2.411533in}}%
\pgfpathlineto{\pgfqpoint{4.296509in}{2.420572in}}%
\pgfpathlineto{\pgfqpoint{4.304020in}{2.429592in}}%
\pgfpathlineto{\pgfqpoint{4.311526in}{2.438593in}}%
\pgfpathlineto{\pgfqpoint{4.319027in}{2.447580in}}%
\pgfpathlineto{\pgfqpoint{4.305816in}{2.445404in}}%
\pgfpathlineto{\pgfqpoint{4.292616in}{2.443395in}}%
\pgfpathlineto{\pgfqpoint{4.279425in}{2.441553in}}%
\pgfpathlineto{\pgfqpoint{4.266243in}{2.439879in}}%
\pgfpathlineto{\pgfqpoint{4.258734in}{2.430737in}}%
\pgfpathlineto{\pgfqpoint{4.251219in}{2.421587in}}%
\pgfpathlineto{\pgfqpoint{4.243700in}{2.412425in}}%
\pgfpathlineto{\pgfqpoint{4.236175in}{2.403251in}}%
\pgfpathclose%
\pgfusepath{fill}%
\end{pgfscope}%
\begin{pgfscope}%
\pgfpathrectangle{\pgfqpoint{1.254980in}{0.150000in}}{\pgfqpoint{5.490039in}{5.490039in}}%
\pgfusepath{clip}%
\pgfsetbuttcap%
\pgfsetroundjoin%
\definecolor{currentfill}{rgb}{0.263663,0.237631,0.518762}%
\pgfsetfillcolor{currentfill}%
\pgfsetfillopacity{0.700000}%
\pgfsetlinewidth{0.000000pt}%
\definecolor{currentstroke}{rgb}{0.000000,0.000000,0.000000}%
\pgfsetstrokecolor{currentstroke}%
\pgfsetdash{}{0pt}%
\pgfpathmoveto{\pgfqpoint{4.319027in}{2.447580in}}%
\pgfpathlineto{\pgfqpoint{4.332247in}{2.449922in}}%
\pgfpathlineto{\pgfqpoint{4.345477in}{2.452430in}}%
\pgfpathlineto{\pgfqpoint{4.358717in}{2.455103in}}%
\pgfpathlineto{\pgfqpoint{4.371968in}{2.457942in}}%
\pgfpathlineto{\pgfqpoint{4.379455in}{2.466751in}}%
\pgfpathlineto{\pgfqpoint{4.386937in}{2.475542in}}%
\pgfpathlineto{\pgfqpoint{4.394414in}{2.484317in}}%
\pgfpathlineto{\pgfqpoint{4.401885in}{2.493077in}}%
\pgfpathlineto{\pgfqpoint{4.388644in}{2.490413in}}%
\pgfpathlineto{\pgfqpoint{4.375412in}{2.487913in}}%
\pgfpathlineto{\pgfqpoint{4.362190in}{2.485578in}}%
\pgfpathlineto{\pgfqpoint{4.348979in}{2.483409in}}%
\pgfpathlineto{\pgfqpoint{4.341498in}{2.474465in}}%
\pgfpathlineto{\pgfqpoint{4.334013in}{2.465513in}}%
\pgfpathlineto{\pgfqpoint{4.326522in}{2.456552in}}%
\pgfpathlineto{\pgfqpoint{4.319027in}{2.447580in}}%
\pgfpathclose%
\pgfusepath{fill}%
\end{pgfscope}%
\begin{pgfscope}%
\pgfpathrectangle{\pgfqpoint{1.254980in}{0.150000in}}{\pgfqpoint{5.490039in}{5.490039in}}%
\pgfusepath{clip}%
\pgfsetbuttcap%
\pgfsetroundjoin%
\definecolor{currentfill}{rgb}{0.275191,0.194905,0.496005}%
\pgfsetfillcolor{currentfill}%
\pgfsetfillopacity{0.700000}%
\pgfsetlinewidth{0.000000pt}%
\definecolor{currentstroke}{rgb}{0.000000,0.000000,0.000000}%
\pgfsetstrokecolor{currentstroke}%
\pgfsetdash{}{0pt}%
\pgfpathmoveto{\pgfqpoint{4.153326in}{2.360333in}}%
\pgfpathlineto{\pgfqpoint{4.166488in}{2.361594in}}%
\pgfpathlineto{\pgfqpoint{4.179658in}{2.363025in}}%
\pgfpathlineto{\pgfqpoint{4.192838in}{2.364625in}}%
\pgfpathlineto{\pgfqpoint{4.206026in}{2.366393in}}%
\pgfpathlineto{\pgfqpoint{4.213571in}{2.375635in}}%
\pgfpathlineto{\pgfqpoint{4.221111in}{2.384858in}}%
\pgfpathlineto{\pgfqpoint{4.228646in}{2.394063in}}%
\pgfpathlineto{\pgfqpoint{4.236175in}{2.403251in}}%
\pgfpathlineto{\pgfqpoint{4.222995in}{2.401600in}}%
\pgfpathlineto{\pgfqpoint{4.209823in}{2.400117in}}%
\pgfpathlineto{\pgfqpoint{4.196661in}{2.398803in}}%
\pgfpathlineto{\pgfqpoint{4.183507in}{2.397659in}}%
\pgfpathlineto{\pgfqpoint{4.175969in}{2.388343in}}%
\pgfpathlineto{\pgfqpoint{4.168427in}{2.379018in}}%
\pgfpathlineto{\pgfqpoint{4.160879in}{2.369682in}}%
\pgfpathlineto{\pgfqpoint{4.153326in}{2.360333in}}%
\pgfpathclose%
\pgfusepath{fill}%
\end{pgfscope}%
\begin{pgfscope}%
\pgfpathrectangle{\pgfqpoint{1.254980in}{0.150000in}}{\pgfqpoint{5.490039in}{5.490039in}}%
\pgfusepath{clip}%
\pgfsetbuttcap%
\pgfsetroundjoin%
\definecolor{currentfill}{rgb}{0.255645,0.260703,0.528312}%
\pgfsetfillcolor{currentfill}%
\pgfsetfillopacity{0.700000}%
\pgfsetlinewidth{0.000000pt}%
\definecolor{currentstroke}{rgb}{0.000000,0.000000,0.000000}%
\pgfsetstrokecolor{currentstroke}%
\pgfsetdash{}{0pt}%
\pgfpathmoveto{\pgfqpoint{4.401885in}{2.493077in}}%
\pgfpathlineto{\pgfqpoint{4.415138in}{2.495907in}}%
\pgfpathlineto{\pgfqpoint{4.428400in}{2.498901in}}%
\pgfpathlineto{\pgfqpoint{4.441674in}{2.502060in}}%
\pgfpathlineto{\pgfqpoint{4.454958in}{2.505382in}}%
\pgfpathlineto{\pgfqpoint{4.462415in}{2.513939in}}%
\pgfpathlineto{\pgfqpoint{4.469867in}{2.522480in}}%
\pgfpathlineto{\pgfqpoint{4.477314in}{2.531008in}}%
\pgfpathlineto{\pgfqpoint{4.484755in}{2.539524in}}%
\pgfpathlineto{\pgfqpoint{4.471480in}{2.536405in}}%
\pgfpathlineto{\pgfqpoint{4.458216in}{2.533449in}}%
\pgfpathlineto{\pgfqpoint{4.444963in}{2.530657in}}%
\pgfpathlineto{\pgfqpoint{4.431720in}{2.528028in}}%
\pgfpathlineto{\pgfqpoint{4.424269in}{2.519300in}}%
\pgfpathlineto{\pgfqpoint{4.416813in}{2.510566in}}%
\pgfpathlineto{\pgfqpoint{4.409352in}{2.501827in}}%
\pgfpathlineto{\pgfqpoint{4.401885in}{2.493077in}}%
\pgfpathclose%
\pgfusepath{fill}%
\end{pgfscope}%
\begin{pgfscope}%
\pgfpathrectangle{\pgfqpoint{1.254980in}{0.150000in}}{\pgfqpoint{5.490039in}{5.490039in}}%
\pgfusepath{clip}%
\pgfsetbuttcap%
\pgfsetroundjoin%
\definecolor{currentfill}{rgb}{0.278826,0.175490,0.483397}%
\pgfsetfillcolor{currentfill}%
\pgfsetfillopacity{0.700000}%
\pgfsetlinewidth{0.000000pt}%
\definecolor{currentstroke}{rgb}{0.000000,0.000000,0.000000}%
\pgfsetstrokecolor{currentstroke}%
\pgfsetdash{}{0pt}%
\pgfpathmoveto{\pgfqpoint{4.070472in}{2.319085in}}%
\pgfpathlineto{\pgfqpoint{4.083607in}{2.319752in}}%
\pgfpathlineto{\pgfqpoint{4.096751in}{2.320591in}}%
\pgfpathlineto{\pgfqpoint{4.109903in}{2.321601in}}%
\pgfpathlineto{\pgfqpoint{4.123064in}{2.322782in}}%
\pgfpathlineto{\pgfqpoint{4.130637in}{2.332196in}}%
\pgfpathlineto{\pgfqpoint{4.138205in}{2.341591in}}%
\pgfpathlineto{\pgfqpoint{4.145768in}{2.350970in}}%
\pgfpathlineto{\pgfqpoint{4.153326in}{2.360333in}}%
\pgfpathlineto{\pgfqpoint{4.140173in}{2.359241in}}%
\pgfpathlineto{\pgfqpoint{4.127029in}{2.358320in}}%
\pgfpathlineto{\pgfqpoint{4.113893in}{2.357570in}}%
\pgfpathlineto{\pgfqpoint{4.100765in}{2.356991in}}%
\pgfpathlineto{\pgfqpoint{4.093199in}{2.347529in}}%
\pgfpathlineto{\pgfqpoint{4.085628in}{2.338058in}}%
\pgfpathlineto{\pgfqpoint{4.078052in}{2.328577in}}%
\pgfpathlineto{\pgfqpoint{4.070472in}{2.319085in}}%
\pgfpathclose%
\pgfusepath{fill}%
\end{pgfscope}%
\begin{pgfscope}%
\pgfpathrectangle{\pgfqpoint{1.254980in}{0.150000in}}{\pgfqpoint{5.490039in}{5.490039in}}%
\pgfusepath{clip}%
\pgfsetbuttcap%
\pgfsetroundjoin%
\definecolor{currentfill}{rgb}{0.283187,0.125848,0.444960}%
\pgfsetfillcolor{currentfill}%
\pgfsetfillopacity{0.700000}%
\pgfsetlinewidth{0.000000pt}%
\definecolor{currentstroke}{rgb}{0.000000,0.000000,0.000000}%
\pgfsetstrokecolor{currentstroke}%
\pgfsetdash{}{0pt}%
\pgfpathmoveto{\pgfqpoint{3.061574in}{2.248308in}}%
\pgfpathlineto{\pgfqpoint{3.074624in}{2.237653in}}%
\pgfpathlineto{\pgfqpoint{3.087672in}{2.227223in}}%
\pgfpathlineto{\pgfqpoint{3.100718in}{2.217017in}}%
\pgfpathlineto{\pgfqpoint{3.113763in}{2.207034in}}%
\pgfpathlineto{\pgfqpoint{3.121697in}{2.214962in}}%
\pgfpathlineto{\pgfqpoint{3.129624in}{2.222970in}}%
\pgfpathlineto{\pgfqpoint{3.137543in}{2.231056in}}%
\pgfpathlineto{\pgfqpoint{3.145456in}{2.239219in}}%
\pgfpathlineto{\pgfqpoint{3.132429in}{2.249037in}}%
\pgfpathlineto{\pgfqpoint{3.119402in}{2.259078in}}%
\pgfpathlineto{\pgfqpoint{3.106372in}{2.269342in}}%
\pgfpathlineto{\pgfqpoint{3.093341in}{2.279832in}}%
\pgfpathlineto{\pgfqpoint{3.085410in}{2.271824in}}%
\pgfpathlineto{\pgfqpoint{3.077472in}{2.263900in}}%
\pgfpathlineto{\pgfqpoint{3.069527in}{2.256061in}}%
\pgfpathlineto{\pgfqpoint{3.061574in}{2.248308in}}%
\pgfpathclose%
\pgfusepath{fill}%
\end{pgfscope}%
\begin{pgfscope}%
\pgfpathrectangle{\pgfqpoint{1.254980in}{0.150000in}}{\pgfqpoint{5.490039in}{5.490039in}}%
\pgfusepath{clip}%
\pgfsetbuttcap%
\pgfsetroundjoin%
\definecolor{currentfill}{rgb}{0.246811,0.283237,0.535941}%
\pgfsetfillcolor{currentfill}%
\pgfsetfillopacity{0.700000}%
\pgfsetlinewidth{0.000000pt}%
\definecolor{currentstroke}{rgb}{0.000000,0.000000,0.000000}%
\pgfsetstrokecolor{currentstroke}%
\pgfsetdash{}{0pt}%
\pgfpathmoveto{\pgfqpoint{4.484755in}{2.539524in}}%
\pgfpathlineto{\pgfqpoint{4.498041in}{2.542807in}}%
\pgfpathlineto{\pgfqpoint{4.511337in}{2.546253in}}%
\pgfpathlineto{\pgfqpoint{4.524645in}{2.549862in}}%
\pgfpathlineto{\pgfqpoint{4.537964in}{2.553633in}}%
\pgfpathlineto{\pgfqpoint{4.545391in}{2.561921in}}%
\pgfpathlineto{\pgfqpoint{4.552812in}{2.570196in}}%
\pgfpathlineto{\pgfqpoint{4.560228in}{2.578461in}}%
\pgfpathlineto{\pgfqpoint{4.567639in}{2.586720in}}%
\pgfpathlineto{\pgfqpoint{4.554330in}{2.583180in}}%
\pgfpathlineto{\pgfqpoint{4.541032in}{2.579802in}}%
\pgfpathlineto{\pgfqpoint{4.527745in}{2.576587in}}%
\pgfpathlineto{\pgfqpoint{4.514470in}{2.573534in}}%
\pgfpathlineto{\pgfqpoint{4.507049in}{2.565035in}}%
\pgfpathlineto{\pgfqpoint{4.499623in}{2.556535in}}%
\pgfpathlineto{\pgfqpoint{4.492192in}{2.548033in}}%
\pgfpathlineto{\pgfqpoint{4.484755in}{2.539524in}}%
\pgfpathclose%
\pgfusepath{fill}%
\end{pgfscope}%
\begin{pgfscope}%
\pgfpathrectangle{\pgfqpoint{1.254980in}{0.150000in}}{\pgfqpoint{5.490039in}{5.490039in}}%
\pgfusepath{clip}%
\pgfsetbuttcap%
\pgfsetroundjoin%
\definecolor{currentfill}{rgb}{0.280267,0.073417,0.397163}%
\pgfsetfillcolor{currentfill}%
\pgfsetfillopacity{0.700000}%
\pgfsetlinewidth{0.000000pt}%
\definecolor{currentstroke}{rgb}{0.000000,0.000000,0.000000}%
\pgfsetstrokecolor{currentstroke}%
\pgfsetdash{}{0pt}%
\pgfpathmoveto{\pgfqpoint{3.385161in}{2.146780in}}%
\pgfpathlineto{\pgfqpoint{3.398177in}{2.140667in}}%
\pgfpathlineto{\pgfqpoint{3.411195in}{2.134752in}}%
\pgfpathlineto{\pgfqpoint{3.424215in}{2.129036in}}%
\pgfpathlineto{\pgfqpoint{3.437238in}{2.123516in}}%
\pgfpathlineto{\pgfqpoint{3.445042in}{2.132639in}}%
\pgfpathlineto{\pgfqpoint{3.452840in}{2.141798in}}%
\pgfpathlineto{\pgfqpoint{3.460633in}{2.150992in}}%
\pgfpathlineto{\pgfqpoint{3.468419in}{2.160222in}}%
\pgfpathlineto{\pgfqpoint{3.455409in}{2.165635in}}%
\pgfpathlineto{\pgfqpoint{3.442402in}{2.171244in}}%
\pgfpathlineto{\pgfqpoint{3.429397in}{2.177052in}}%
\pgfpathlineto{\pgfqpoint{3.416394in}{2.183058in}}%
\pgfpathlineto{\pgfqpoint{3.408595in}{2.173925in}}%
\pgfpathlineto{\pgfqpoint{3.400789in}{2.164834in}}%
\pgfpathlineto{\pgfqpoint{3.392978in}{2.155786in}}%
\pgfpathlineto{\pgfqpoint{3.385161in}{2.146780in}}%
\pgfpathclose%
\pgfusepath{fill}%
\end{pgfscope}%
\begin{pgfscope}%
\pgfpathrectangle{\pgfqpoint{1.254980in}{0.150000in}}{\pgfqpoint{5.490039in}{5.490039in}}%
\pgfusepath{clip}%
\pgfsetbuttcap%
\pgfsetroundjoin%
\definecolor{currentfill}{rgb}{0.281412,0.155834,0.469201}%
\pgfsetfillcolor{currentfill}%
\pgfsetfillopacity{0.700000}%
\pgfsetlinewidth{0.000000pt}%
\definecolor{currentstroke}{rgb}{0.000000,0.000000,0.000000}%
\pgfsetstrokecolor{currentstroke}%
\pgfsetdash{}{0pt}%
\pgfpathmoveto{\pgfqpoint{3.987604in}{2.279789in}}%
\pgfpathlineto{\pgfqpoint{4.000716in}{2.279826in}}%
\pgfpathlineto{\pgfqpoint{4.013835in}{2.280037in}}%
\pgfpathlineto{\pgfqpoint{4.026963in}{2.280421in}}%
\pgfpathlineto{\pgfqpoint{4.040098in}{2.280978in}}%
\pgfpathlineto{\pgfqpoint{4.047699in}{2.290527in}}%
\pgfpathlineto{\pgfqpoint{4.055295in}{2.300061in}}%
\pgfpathlineto{\pgfqpoint{4.062886in}{2.309580in}}%
\pgfpathlineto{\pgfqpoint{4.070472in}{2.319085in}}%
\pgfpathlineto{\pgfqpoint{4.057344in}{2.318589in}}%
\pgfpathlineto{\pgfqpoint{4.044225in}{2.318266in}}%
\pgfpathlineto{\pgfqpoint{4.031113in}{2.318116in}}%
\pgfpathlineto{\pgfqpoint{4.018009in}{2.318139in}}%
\pgfpathlineto{\pgfqpoint{4.010415in}{2.308563in}}%
\pgfpathlineto{\pgfqpoint{4.002816in}{2.298980in}}%
\pgfpathlineto{\pgfqpoint{3.995212in}{2.289389in}}%
\pgfpathlineto{\pgfqpoint{3.987604in}{2.279789in}}%
\pgfpathclose%
\pgfusepath{fill}%
\end{pgfscope}%
\begin{pgfscope}%
\pgfpathrectangle{\pgfqpoint{1.254980in}{0.150000in}}{\pgfqpoint{5.490039in}{5.490039in}}%
\pgfusepath{clip}%
\pgfsetbuttcap%
\pgfsetroundjoin%
\definecolor{currentfill}{rgb}{0.281446,0.084320,0.407414}%
\pgfsetfillcolor{currentfill}%
\pgfsetfillopacity{0.700000}%
\pgfsetlinewidth{0.000000pt}%
\definecolor{currentstroke}{rgb}{0.000000,0.000000,0.000000}%
\pgfsetstrokecolor{currentstroke}%
\pgfsetdash{}{0pt}%
\pgfpathmoveto{\pgfqpoint{3.249636in}{2.168499in}}%
\pgfpathlineto{\pgfqpoint{3.262657in}{2.160617in}}%
\pgfpathlineto{\pgfqpoint{3.275680in}{2.152943in}}%
\pgfpathlineto{\pgfqpoint{3.288702in}{2.145477in}}%
\pgfpathlineto{\pgfqpoint{3.301726in}{2.138216in}}%
\pgfpathlineto{\pgfqpoint{3.309582in}{2.146894in}}%
\pgfpathlineto{\pgfqpoint{3.317432in}{2.155626in}}%
\pgfpathlineto{\pgfqpoint{3.325276in}{2.164410in}}%
\pgfpathlineto{\pgfqpoint{3.333113in}{2.173247in}}%
\pgfpathlineto{\pgfqpoint{3.320105in}{2.180372in}}%
\pgfpathlineto{\pgfqpoint{3.307097in}{2.187703in}}%
\pgfpathlineto{\pgfqpoint{3.294090in}{2.195241in}}%
\pgfpathlineto{\pgfqpoint{3.281084in}{2.202988in}}%
\pgfpathlineto{\pgfqpoint{3.273232in}{2.194277in}}%
\pgfpathlineto{\pgfqpoint{3.265373in}{2.185624in}}%
\pgfpathlineto{\pgfqpoint{3.257508in}{2.177031in}}%
\pgfpathlineto{\pgfqpoint{3.249636in}{2.168499in}}%
\pgfpathclose%
\pgfusepath{fill}%
\end{pgfscope}%
\begin{pgfscope}%
\pgfpathrectangle{\pgfqpoint{1.254980in}{0.150000in}}{\pgfqpoint{5.490039in}{5.490039in}}%
\pgfusepath{clip}%
\pgfsetbuttcap%
\pgfsetroundjoin%
\definecolor{currentfill}{rgb}{0.237441,0.305202,0.541921}%
\pgfsetfillcolor{currentfill}%
\pgfsetfillopacity{0.700000}%
\pgfsetlinewidth{0.000000pt}%
\definecolor{currentstroke}{rgb}{0.000000,0.000000,0.000000}%
\pgfsetstrokecolor{currentstroke}%
\pgfsetdash{}{0pt}%
\pgfpathmoveto{\pgfqpoint{4.567639in}{2.586720in}}%
\pgfpathlineto{\pgfqpoint{4.580959in}{2.590422in}}%
\pgfpathlineto{\pgfqpoint{4.594291in}{2.594285in}}%
\pgfpathlineto{\pgfqpoint{4.607635in}{2.598310in}}%
\pgfpathlineto{\pgfqpoint{4.620990in}{2.602496in}}%
\pgfpathlineto{\pgfqpoint{4.628385in}{2.610502in}}%
\pgfpathlineto{\pgfqpoint{4.635775in}{2.618500in}}%
\pgfpathlineto{\pgfqpoint{4.643159in}{2.626493in}}%
\pgfpathlineto{\pgfqpoint{4.650538in}{2.634484in}}%
\pgfpathlineto{\pgfqpoint{4.637194in}{2.630558in}}%
\pgfpathlineto{\pgfqpoint{4.623861in}{2.626792in}}%
\pgfpathlineto{\pgfqpoint{4.610540in}{2.623188in}}%
\pgfpathlineto{\pgfqpoint{4.597230in}{2.619744in}}%
\pgfpathlineto{\pgfqpoint{4.589840in}{2.611484in}}%
\pgfpathlineto{\pgfqpoint{4.582445in}{2.603228in}}%
\pgfpathlineto{\pgfqpoint{4.575044in}{2.594975in}}%
\pgfpathlineto{\pgfqpoint{4.567639in}{2.586720in}}%
\pgfpathclose%
\pgfusepath{fill}%
\end{pgfscope}%
\begin{pgfscope}%
\pgfpathrectangle{\pgfqpoint{1.254980in}{0.150000in}}{\pgfqpoint{5.490039in}{5.490039in}}%
\pgfusepath{clip}%
\pgfsetbuttcap%
\pgfsetroundjoin%
\definecolor{currentfill}{rgb}{0.227802,0.326594,0.546532}%
\pgfsetfillcolor{currentfill}%
\pgfsetfillopacity{0.700000}%
\pgfsetlinewidth{0.000000pt}%
\definecolor{currentstroke}{rgb}{0.000000,0.000000,0.000000}%
\pgfsetstrokecolor{currentstroke}%
\pgfsetdash{}{0pt}%
\pgfpathmoveto{\pgfqpoint{4.650538in}{2.634484in}}%
\pgfpathlineto{\pgfqpoint{4.663894in}{2.638571in}}%
\pgfpathlineto{\pgfqpoint{4.677262in}{2.642818in}}%
\pgfpathlineto{\pgfqpoint{4.690642in}{2.647226in}}%
\pgfpathlineto{\pgfqpoint{4.704035in}{2.651793in}}%
\pgfpathlineto{\pgfqpoint{4.711397in}{2.659509in}}%
\pgfpathlineto{\pgfqpoint{4.718755in}{2.667222in}}%
\pgfpathlineto{\pgfqpoint{4.726107in}{2.674936in}}%
\pgfpathlineto{\pgfqpoint{4.733453in}{2.682656in}}%
\pgfpathlineto{\pgfqpoint{4.720073in}{2.678377in}}%
\pgfpathlineto{\pgfqpoint{4.706704in}{2.674258in}}%
\pgfpathlineto{\pgfqpoint{4.693348in}{2.670298in}}%
\pgfpathlineto{\pgfqpoint{4.680003in}{2.666498in}}%
\pgfpathlineto{\pgfqpoint{4.672645in}{2.658481in}}%
\pgfpathlineto{\pgfqpoint{4.665281in}{2.650475in}}%
\pgfpathlineto{\pgfqpoint{4.657912in}{2.642477in}}%
\pgfpathlineto{\pgfqpoint{4.650538in}{2.634484in}}%
\pgfpathclose%
\pgfusepath{fill}%
\end{pgfscope}%
\begin{pgfscope}%
\pgfpathrectangle{\pgfqpoint{1.254980in}{0.150000in}}{\pgfqpoint{5.490039in}{5.490039in}}%
\pgfusepath{clip}%
\pgfsetbuttcap%
\pgfsetroundjoin%
\definecolor{currentfill}{rgb}{0.282884,0.135920,0.453427}%
\pgfsetfillcolor{currentfill}%
\pgfsetfillopacity{0.700000}%
\pgfsetlinewidth{0.000000pt}%
\definecolor{currentstroke}{rgb}{0.000000,0.000000,0.000000}%
\pgfsetstrokecolor{currentstroke}%
\pgfsetdash{}{0pt}%
\pgfpathmoveto{\pgfqpoint{3.904711in}{2.242746in}}%
\pgfpathlineto{\pgfqpoint{3.917802in}{2.242116in}}%
\pgfpathlineto{\pgfqpoint{3.930900in}{2.241662in}}%
\pgfpathlineto{\pgfqpoint{3.944006in}{2.241384in}}%
\pgfpathlineto{\pgfqpoint{3.957118in}{2.241280in}}%
\pgfpathlineto{\pgfqpoint{3.964747in}{2.250925in}}%
\pgfpathlineto{\pgfqpoint{3.972371in}{2.260558in}}%
\pgfpathlineto{\pgfqpoint{3.979990in}{2.270179in}}%
\pgfpathlineto{\pgfqpoint{3.987604in}{2.279789in}}%
\pgfpathlineto{\pgfqpoint{3.974499in}{2.279926in}}%
\pgfpathlineto{\pgfqpoint{3.961402in}{2.280237in}}%
\pgfpathlineto{\pgfqpoint{3.948312in}{2.280724in}}%
\pgfpathlineto{\pgfqpoint{3.935229in}{2.281387in}}%
\pgfpathlineto{\pgfqpoint{3.927607in}{2.271733in}}%
\pgfpathlineto{\pgfqpoint{3.919980in}{2.262076in}}%
\pgfpathlineto{\pgfqpoint{3.912348in}{2.252414in}}%
\pgfpathlineto{\pgfqpoint{3.904711in}{2.242746in}}%
\pgfpathclose%
\pgfusepath{fill}%
\end{pgfscope}%
\begin{pgfscope}%
\pgfpathrectangle{\pgfqpoint{1.254980in}{0.150000in}}{\pgfqpoint{5.490039in}{5.490039in}}%
\pgfusepath{clip}%
\pgfsetbuttcap%
\pgfsetroundjoin%
\definecolor{currentfill}{rgb}{0.218130,0.347432,0.550038}%
\pgfsetfillcolor{currentfill}%
\pgfsetfillopacity{0.700000}%
\pgfsetlinewidth{0.000000pt}%
\definecolor{currentstroke}{rgb}{0.000000,0.000000,0.000000}%
\pgfsetstrokecolor{currentstroke}%
\pgfsetdash{}{0pt}%
\pgfpathmoveto{\pgfqpoint{4.733453in}{2.682656in}}%
\pgfpathlineto{\pgfqpoint{4.746846in}{2.687094in}}%
\pgfpathlineto{\pgfqpoint{4.760251in}{2.691692in}}%
\pgfpathlineto{\pgfqpoint{4.773669in}{2.696449in}}%
\pgfpathlineto{\pgfqpoint{4.787099in}{2.701364in}}%
\pgfpathlineto{\pgfqpoint{4.794428in}{2.708786in}}%
\pgfpathlineto{\pgfqpoint{4.801752in}{2.716212in}}%
\pgfpathlineto{\pgfqpoint{4.809071in}{2.723647in}}%
\pgfpathlineto{\pgfqpoint{4.816385in}{2.731095in}}%
\pgfpathlineto{\pgfqpoint{4.802967in}{2.726497in}}%
\pgfpathlineto{\pgfqpoint{4.789562in}{2.722057in}}%
\pgfpathlineto{\pgfqpoint{4.776170in}{2.717776in}}%
\pgfpathlineto{\pgfqpoint{4.762789in}{2.713654in}}%
\pgfpathlineto{\pgfqpoint{4.755463in}{2.705879in}}%
\pgfpathlineto{\pgfqpoint{4.748131in}{2.698124in}}%
\pgfpathlineto{\pgfqpoint{4.740795in}{2.690384in}}%
\pgfpathlineto{\pgfqpoint{4.733453in}{2.682656in}}%
\pgfpathclose%
\pgfusepath{fill}%
\end{pgfscope}%
\begin{pgfscope}%
\pgfpathrectangle{\pgfqpoint{1.254980in}{0.150000in}}{\pgfqpoint{5.490039in}{5.490039in}}%
\pgfusepath{clip}%
\pgfsetbuttcap%
\pgfsetroundjoin%
\definecolor{currentfill}{rgb}{0.208623,0.367752,0.552675}%
\pgfsetfillcolor{currentfill}%
\pgfsetfillopacity{0.700000}%
\pgfsetlinewidth{0.000000pt}%
\definecolor{currentstroke}{rgb}{0.000000,0.000000,0.000000}%
\pgfsetstrokecolor{currentstroke}%
\pgfsetdash{}{0pt}%
\pgfpathmoveto{\pgfqpoint{4.816385in}{2.731095in}}%
\pgfpathlineto{\pgfqpoint{4.829815in}{2.735852in}}%
\pgfpathlineto{\pgfqpoint{4.843257in}{2.740766in}}%
\pgfpathlineto{\pgfqpoint{4.856713in}{2.745839in}}%
\pgfpathlineto{\pgfqpoint{4.870182in}{2.751070in}}%
\pgfpathlineto{\pgfqpoint{4.877477in}{2.758198in}}%
\pgfpathlineto{\pgfqpoint{4.884767in}{2.765341in}}%
\pgfpathlineto{\pgfqpoint{4.892051in}{2.772501in}}%
\pgfpathlineto{\pgfqpoint{4.899331in}{2.779682in}}%
\pgfpathlineto{\pgfqpoint{4.885876in}{2.774798in}}%
\pgfpathlineto{\pgfqpoint{4.872434in}{2.770071in}}%
\pgfpathlineto{\pgfqpoint{4.859005in}{2.765501in}}%
\pgfpathlineto{\pgfqpoint{4.845589in}{2.761090in}}%
\pgfpathlineto{\pgfqpoint{4.838295in}{2.753553in}}%
\pgfpathlineto{\pgfqpoint{4.830997in}{2.746044in}}%
\pgfpathlineto{\pgfqpoint{4.823693in}{2.738560in}}%
\pgfpathlineto{\pgfqpoint{4.816385in}{2.731095in}}%
\pgfpathclose%
\pgfusepath{fill}%
\end{pgfscope}%
\begin{pgfscope}%
\pgfpathrectangle{\pgfqpoint{1.254980in}{0.150000in}}{\pgfqpoint{5.490039in}{5.490039in}}%
\pgfusepath{clip}%
\pgfsetbuttcap%
\pgfsetroundjoin%
\definecolor{currentfill}{rgb}{0.280894,0.078907,0.402329}%
\pgfsetfillcolor{currentfill}%
\pgfsetfillopacity{0.700000}%
\pgfsetlinewidth{0.000000pt}%
\definecolor{currentstroke}{rgb}{0.000000,0.000000,0.000000}%
\pgfsetstrokecolor{currentstroke}%
\pgfsetdash{}{0pt}%
\pgfpathmoveto{\pgfqpoint{3.520487in}{2.140519in}}%
\pgfpathlineto{\pgfqpoint{3.533512in}{2.136075in}}%
\pgfpathlineto{\pgfqpoint{3.546541in}{2.131823in}}%
\pgfpathlineto{\pgfqpoint{3.559573in}{2.127760in}}%
\pgfpathlineto{\pgfqpoint{3.572609in}{2.123887in}}%
\pgfpathlineto{\pgfqpoint{3.580367in}{2.133330in}}%
\pgfpathlineto{\pgfqpoint{3.588119in}{2.142792in}}%
\pgfpathlineto{\pgfqpoint{3.595866in}{2.152275in}}%
\pgfpathlineto{\pgfqpoint{3.603607in}{2.161777in}}%
\pgfpathlineto{\pgfqpoint{3.590582in}{2.165571in}}%
\pgfpathlineto{\pgfqpoint{3.577561in}{2.169555in}}%
\pgfpathlineto{\pgfqpoint{3.564543in}{2.173729in}}%
\pgfpathlineto{\pgfqpoint{3.551530in}{2.178093in}}%
\pgfpathlineto{\pgfqpoint{3.543778in}{2.168660in}}%
\pgfpathlineto{\pgfqpoint{3.536020in}{2.159253in}}%
\pgfpathlineto{\pgfqpoint{3.528256in}{2.149872in}}%
\pgfpathlineto{\pgfqpoint{3.520487in}{2.140519in}}%
\pgfpathclose%
\pgfusepath{fill}%
\end{pgfscope}%
\begin{pgfscope}%
\pgfpathrectangle{\pgfqpoint{1.254980in}{0.150000in}}{\pgfqpoint{5.490039in}{5.490039in}}%
\pgfusepath{clip}%
\pgfsetbuttcap%
\pgfsetroundjoin%
\definecolor{currentfill}{rgb}{0.283197,0.115680,0.436115}%
\pgfsetfillcolor{currentfill}%
\pgfsetfillopacity{0.700000}%
\pgfsetlinewidth{0.000000pt}%
\definecolor{currentstroke}{rgb}{0.000000,0.000000,0.000000}%
\pgfsetstrokecolor{currentstroke}%
\pgfsetdash{}{0pt}%
\pgfpathmoveto{\pgfqpoint{3.821782in}{2.208280in}}%
\pgfpathlineto{\pgfqpoint{3.834855in}{2.206945in}}%
\pgfpathlineto{\pgfqpoint{3.847935in}{2.205788in}}%
\pgfpathlineto{\pgfqpoint{3.861021in}{2.204810in}}%
\pgfpathlineto{\pgfqpoint{3.874113in}{2.204008in}}%
\pgfpathlineto{\pgfqpoint{3.881770in}{2.213704in}}%
\pgfpathlineto{\pgfqpoint{3.889422in}{2.223392in}}%
\pgfpathlineto{\pgfqpoint{3.897069in}{2.233072in}}%
\pgfpathlineto{\pgfqpoint{3.904711in}{2.242746in}}%
\pgfpathlineto{\pgfqpoint{3.891627in}{2.243553in}}%
\pgfpathlineto{\pgfqpoint{3.878550in}{2.244536in}}%
\pgfpathlineto{\pgfqpoint{3.865479in}{2.245698in}}%
\pgfpathlineto{\pgfqpoint{3.852414in}{2.247037in}}%
\pgfpathlineto{\pgfqpoint{3.844764in}{2.237348in}}%
\pgfpathlineto{\pgfqpoint{3.837108in}{2.227659in}}%
\pgfpathlineto{\pgfqpoint{3.829448in}{2.217970in}}%
\pgfpathlineto{\pgfqpoint{3.821782in}{2.208280in}}%
\pgfpathclose%
\pgfusepath{fill}%
\end{pgfscope}%
\begin{pgfscope}%
\pgfpathrectangle{\pgfqpoint{1.254980in}{0.150000in}}{\pgfqpoint{5.490039in}{5.490039in}}%
\pgfusepath{clip}%
\pgfsetbuttcap%
\pgfsetroundjoin%
\definecolor{currentfill}{rgb}{0.199430,0.387607,0.554642}%
\pgfsetfillcolor{currentfill}%
\pgfsetfillopacity{0.700000}%
\pgfsetlinewidth{0.000000pt}%
\definecolor{currentstroke}{rgb}{0.000000,0.000000,0.000000}%
\pgfsetstrokecolor{currentstroke}%
\pgfsetdash{}{0pt}%
\pgfpathmoveto{\pgfqpoint{4.899331in}{2.779682in}}%
\pgfpathlineto{\pgfqpoint{4.912799in}{2.784723in}}%
\pgfpathlineto{\pgfqpoint{4.926280in}{2.789922in}}%
\pgfpathlineto{\pgfqpoint{4.939774in}{2.795277in}}%
\pgfpathlineto{\pgfqpoint{4.953282in}{2.800790in}}%
\pgfpathlineto{\pgfqpoint{4.960542in}{2.807632in}}%
\pgfpathlineto{\pgfqpoint{4.967797in}{2.814498in}}%
\pgfpathlineto{\pgfqpoint{4.975047in}{2.821390in}}%
\pgfpathlineto{\pgfqpoint{4.982292in}{2.828315in}}%
\pgfpathlineto{\pgfqpoint{4.968799in}{2.823178in}}%
\pgfpathlineto{\pgfqpoint{4.955320in}{2.818197in}}%
\pgfpathlineto{\pgfqpoint{4.941854in}{2.813372in}}%
\pgfpathlineto{\pgfqpoint{4.928401in}{2.808704in}}%
\pgfpathlineto{\pgfqpoint{4.921141in}{2.801396in}}%
\pgfpathlineto{\pgfqpoint{4.913876in}{2.794125in}}%
\pgfpathlineto{\pgfqpoint{4.906606in}{2.786889in}}%
\pgfpathlineto{\pgfqpoint{4.899331in}{2.779682in}}%
\pgfpathclose%
\pgfusepath{fill}%
\end{pgfscope}%
\begin{pgfscope}%
\pgfpathrectangle{\pgfqpoint{1.254980in}{0.150000in}}{\pgfqpoint{5.490039in}{5.490039in}}%
\pgfusepath{clip}%
\pgfsetbuttcap%
\pgfsetroundjoin%
\definecolor{currentfill}{rgb}{0.282910,0.105393,0.426902}%
\pgfsetfillcolor{currentfill}%
\pgfsetfillopacity{0.700000}%
\pgfsetlinewidth{0.000000pt}%
\definecolor{currentstroke}{rgb}{0.000000,0.000000,0.000000}%
\pgfsetstrokecolor{currentstroke}%
\pgfsetdash{}{0pt}%
\pgfpathmoveto{\pgfqpoint{3.113763in}{2.207034in}}%
\pgfpathlineto{\pgfqpoint{3.126806in}{2.197271in}}%
\pgfpathlineto{\pgfqpoint{3.139847in}{2.187728in}}%
\pgfpathlineto{\pgfqpoint{3.152888in}{2.178404in}}%
\pgfpathlineto{\pgfqpoint{3.165928in}{2.169295in}}%
\pgfpathlineto{\pgfqpoint{3.173844in}{2.177399in}}%
\pgfpathlineto{\pgfqpoint{3.181753in}{2.185574in}}%
\pgfpathlineto{\pgfqpoint{3.189655in}{2.193821in}}%
\pgfpathlineto{\pgfqpoint{3.197550in}{2.202138in}}%
\pgfpathlineto{\pgfqpoint{3.184528in}{2.211082in}}%
\pgfpathlineto{\pgfqpoint{3.171505in}{2.220243in}}%
\pgfpathlineto{\pgfqpoint{3.158481in}{2.229621in}}%
\pgfpathlineto{\pgfqpoint{3.145456in}{2.239219in}}%
\pgfpathlineto{\pgfqpoint{3.137543in}{2.231056in}}%
\pgfpathlineto{\pgfqpoint{3.129624in}{2.222970in}}%
\pgfpathlineto{\pgfqpoint{3.121697in}{2.214962in}}%
\pgfpathlineto{\pgfqpoint{3.113763in}{2.207034in}}%
\pgfpathclose%
\pgfusepath{fill}%
\end{pgfscope}%
\begin{pgfscope}%
\pgfpathrectangle{\pgfqpoint{1.254980in}{0.150000in}}{\pgfqpoint{5.490039in}{5.490039in}}%
\pgfusepath{clip}%
\pgfsetbuttcap%
\pgfsetroundjoin%
\definecolor{currentfill}{rgb}{0.190631,0.407061,0.556089}%
\pgfsetfillcolor{currentfill}%
\pgfsetfillopacity{0.700000}%
\pgfsetlinewidth{0.000000pt}%
\definecolor{currentstroke}{rgb}{0.000000,0.000000,0.000000}%
\pgfsetstrokecolor{currentstroke}%
\pgfsetdash{}{0pt}%
\pgfpathmoveto{\pgfqpoint{4.982292in}{2.828315in}}%
\pgfpathlineto{\pgfqpoint{4.995798in}{2.833608in}}%
\pgfpathlineto{\pgfqpoint{5.009317in}{2.839058in}}%
\pgfpathlineto{\pgfqpoint{5.022850in}{2.844663in}}%
\pgfpathlineto{\pgfqpoint{5.036397in}{2.850425in}}%
\pgfpathlineto{\pgfqpoint{5.043621in}{2.856992in}}%
\pgfpathlineto{\pgfqpoint{5.050841in}{2.863593in}}%
\pgfpathlineto{\pgfqpoint{5.058055in}{2.870232in}}%
\pgfpathlineto{\pgfqpoint{5.065265in}{2.876914in}}%
\pgfpathlineto{\pgfqpoint{5.051735in}{2.871557in}}%
\pgfpathlineto{\pgfqpoint{5.038219in}{2.866355in}}%
\pgfpathlineto{\pgfqpoint{5.024715in}{2.861308in}}%
\pgfpathlineto{\pgfqpoint{5.011226in}{2.856417in}}%
\pgfpathlineto{\pgfqpoint{5.003999in}{2.849322in}}%
\pgfpathlineto{\pgfqpoint{4.996768in}{2.842276in}}%
\pgfpathlineto{\pgfqpoint{4.989532in}{2.835275in}}%
\pgfpathlineto{\pgfqpoint{4.982292in}{2.828315in}}%
\pgfpathclose%
\pgfusepath{fill}%
\end{pgfscope}%
\begin{pgfscope}%
\pgfpathrectangle{\pgfqpoint{1.254980in}{0.150000in}}{\pgfqpoint{5.490039in}{5.490039in}}%
\pgfusepath{clip}%
\pgfsetbuttcap%
\pgfsetroundjoin%
\definecolor{currentfill}{rgb}{0.262138,0.242286,0.520837}%
\pgfsetfillcolor{currentfill}%
\pgfsetfillopacity{0.700000}%
\pgfsetlinewidth{0.000000pt}%
\definecolor{currentstroke}{rgb}{0.000000,0.000000,0.000000}%
\pgfsetstrokecolor{currentstroke}%
\pgfsetdash{}{0pt}%
\pgfpathmoveto{\pgfqpoint{2.767452in}{2.485295in}}%
\pgfpathlineto{\pgfqpoint{2.780615in}{2.469550in}}%
\pgfpathlineto{\pgfqpoint{2.793772in}{2.454069in}}%
\pgfpathlineto{\pgfqpoint{2.806922in}{2.438851in}}%
\pgfpathlineto{\pgfqpoint{2.820065in}{2.423891in}}%
\pgfpathlineto{\pgfqpoint{2.828140in}{2.430457in}}%
\pgfpathlineto{\pgfqpoint{2.836206in}{2.437144in}}%
\pgfpathlineto{\pgfqpoint{2.844263in}{2.443950in}}%
\pgfpathlineto{\pgfqpoint{2.852310in}{2.450874in}}%
\pgfpathlineto{\pgfqpoint{2.839191in}{2.465634in}}%
\pgfpathlineto{\pgfqpoint{2.826065in}{2.480654in}}%
\pgfpathlineto{\pgfqpoint{2.812934in}{2.495936in}}%
\pgfpathlineto{\pgfqpoint{2.799795in}{2.511481in}}%
\pgfpathlineto{\pgfqpoint{2.791724in}{2.504746in}}%
\pgfpathlineto{\pgfqpoint{2.783643in}{2.498136in}}%
\pgfpathlineto{\pgfqpoint{2.775552in}{2.491652in}}%
\pgfpathlineto{\pgfqpoint{2.767452in}{2.485295in}}%
\pgfpathclose%
\pgfusepath{fill}%
\end{pgfscope}%
\begin{pgfscope}%
\pgfpathrectangle{\pgfqpoint{1.254980in}{0.150000in}}{\pgfqpoint{5.490039in}{5.490039in}}%
\pgfusepath{clip}%
\pgfsetbuttcap%
\pgfsetroundjoin%
\definecolor{currentfill}{rgb}{0.270595,0.214069,0.507052}%
\pgfsetfillcolor{currentfill}%
\pgfsetfillopacity{0.700000}%
\pgfsetlinewidth{0.000000pt}%
\definecolor{currentstroke}{rgb}{0.000000,0.000000,0.000000}%
\pgfsetstrokecolor{currentstroke}%
\pgfsetdash{}{0pt}%
\pgfpathmoveto{\pgfqpoint{2.820065in}{2.423891in}}%
\pgfpathlineto{\pgfqpoint{2.833203in}{2.409190in}}%
\pgfpathlineto{\pgfqpoint{2.846334in}{2.394743in}}%
\pgfpathlineto{\pgfqpoint{2.859460in}{2.380550in}}%
\pgfpathlineto{\pgfqpoint{2.872580in}{2.366608in}}%
\pgfpathlineto{\pgfqpoint{2.880631in}{2.373382in}}%
\pgfpathlineto{\pgfqpoint{2.888673in}{2.380270in}}%
\pgfpathlineto{\pgfqpoint{2.896706in}{2.387271in}}%
\pgfpathlineto{\pgfqpoint{2.904730in}{2.394381in}}%
\pgfpathlineto{\pgfqpoint{2.891633in}{2.408126in}}%
\pgfpathlineto{\pgfqpoint{2.878531in}{2.422121in}}%
\pgfpathlineto{\pgfqpoint{2.865423in}{2.436370in}}%
\pgfpathlineto{\pgfqpoint{2.852310in}{2.450874in}}%
\pgfpathlineto{\pgfqpoint{2.844263in}{2.443950in}}%
\pgfpathlineto{\pgfqpoint{2.836206in}{2.437144in}}%
\pgfpathlineto{\pgfqpoint{2.828140in}{2.430457in}}%
\pgfpathlineto{\pgfqpoint{2.820065in}{2.423891in}}%
\pgfpathclose%
\pgfusepath{fill}%
\end{pgfscope}%
\begin{pgfscope}%
\pgfpathrectangle{\pgfqpoint{1.254980in}{0.150000in}}{\pgfqpoint{5.490039in}{5.490039in}}%
\pgfusepath{clip}%
\pgfsetbuttcap%
\pgfsetroundjoin%
\definecolor{currentfill}{rgb}{0.252194,0.269783,0.531579}%
\pgfsetfillcolor{currentfill}%
\pgfsetfillopacity{0.700000}%
\pgfsetlinewidth{0.000000pt}%
\definecolor{currentstroke}{rgb}{0.000000,0.000000,0.000000}%
\pgfsetstrokecolor{currentstroke}%
\pgfsetdash{}{0pt}%
\pgfpathmoveto{\pgfqpoint{2.714723in}{2.550965in}}%
\pgfpathlineto{\pgfqpoint{2.727917in}{2.534139in}}%
\pgfpathlineto{\pgfqpoint{2.741103in}{2.517588in}}%
\pgfpathlineto{\pgfqpoint{2.754281in}{2.501307in}}%
\pgfpathlineto{\pgfqpoint{2.767452in}{2.485295in}}%
\pgfpathlineto{\pgfqpoint{2.775552in}{2.491652in}}%
\pgfpathlineto{\pgfqpoint{2.783643in}{2.498136in}}%
\pgfpathlineto{\pgfqpoint{2.791724in}{2.504746in}}%
\pgfpathlineto{\pgfqpoint{2.799795in}{2.511481in}}%
\pgfpathlineto{\pgfqpoint{2.786650in}{2.527293in}}%
\pgfpathlineto{\pgfqpoint{2.773497in}{2.543372in}}%
\pgfpathlineto{\pgfqpoint{2.760337in}{2.559723in}}%
\pgfpathlineto{\pgfqpoint{2.747169in}{2.576347in}}%
\pgfpathlineto{\pgfqpoint{2.739073in}{2.569802in}}%
\pgfpathlineto{\pgfqpoint{2.730966in}{2.563389in}}%
\pgfpathlineto{\pgfqpoint{2.722850in}{2.557109in}}%
\pgfpathlineto{\pgfqpoint{2.714723in}{2.550965in}}%
\pgfpathclose%
\pgfusepath{fill}%
\end{pgfscope}%
\begin{pgfscope}%
\pgfpathrectangle{\pgfqpoint{1.254980in}{0.150000in}}{\pgfqpoint{5.490039in}{5.490039in}}%
\pgfusepath{clip}%
\pgfsetbuttcap%
\pgfsetroundjoin%
\definecolor{currentfill}{rgb}{0.182256,0.426184,0.557120}%
\pgfsetfillcolor{currentfill}%
\pgfsetfillopacity{0.700000}%
\pgfsetlinewidth{0.000000pt}%
\definecolor{currentstroke}{rgb}{0.000000,0.000000,0.000000}%
\pgfsetstrokecolor{currentstroke}%
\pgfsetdash{}{0pt}%
\pgfpathmoveto{\pgfqpoint{5.065265in}{2.876914in}}%
\pgfpathlineto{\pgfqpoint{5.078809in}{2.882427in}}%
\pgfpathlineto{\pgfqpoint{5.092367in}{2.888095in}}%
\pgfpathlineto{\pgfqpoint{5.105939in}{2.893918in}}%
\pgfpathlineto{\pgfqpoint{5.119526in}{2.899896in}}%
\pgfpathlineto{\pgfqpoint{5.126713in}{2.906203in}}%
\pgfpathlineto{\pgfqpoint{5.133897in}{2.912556in}}%
\pgfpathlineto{\pgfqpoint{5.141076in}{2.918960in}}%
\pgfpathlineto{\pgfqpoint{5.148250in}{2.925420in}}%
\pgfpathlineto{\pgfqpoint{5.134682in}{2.919875in}}%
\pgfpathlineto{\pgfqpoint{5.121128in}{2.914484in}}%
\pgfpathlineto{\pgfqpoint{5.107588in}{2.909248in}}%
\pgfpathlineto{\pgfqpoint{5.094062in}{2.904167in}}%
\pgfpathlineto{\pgfqpoint{5.086869in}{2.897265in}}%
\pgfpathlineto{\pgfqpoint{5.079672in}{2.890426in}}%
\pgfpathlineto{\pgfqpoint{5.072471in}{2.883644in}}%
\pgfpathlineto{\pgfqpoint{5.065265in}{2.876914in}}%
\pgfpathclose%
\pgfusepath{fill}%
\end{pgfscope}%
\begin{pgfscope}%
\pgfpathrectangle{\pgfqpoint{1.254980in}{0.150000in}}{\pgfqpoint{5.490039in}{5.490039in}}%
\pgfusepath{clip}%
\pgfsetbuttcap%
\pgfsetroundjoin%
\definecolor{currentfill}{rgb}{0.282656,0.100196,0.422160}%
\pgfsetfillcolor{currentfill}%
\pgfsetfillopacity{0.700000}%
\pgfsetlinewidth{0.000000pt}%
\definecolor{currentstroke}{rgb}{0.000000,0.000000,0.000000}%
\pgfsetstrokecolor{currentstroke}%
\pgfsetdash{}{0pt}%
\pgfpathmoveto{\pgfqpoint{3.738801in}{2.176733in}}%
\pgfpathlineto{\pgfqpoint{3.751859in}{2.174655in}}%
\pgfpathlineto{\pgfqpoint{3.764923in}{2.172758in}}%
\pgfpathlineto{\pgfqpoint{3.777993in}{2.171041in}}%
\pgfpathlineto{\pgfqpoint{3.791069in}{2.169504in}}%
\pgfpathlineto{\pgfqpoint{3.798755in}{2.179201in}}%
\pgfpathlineto{\pgfqpoint{3.806436in}{2.188895in}}%
\pgfpathlineto{\pgfqpoint{3.814111in}{2.198588in}}%
\pgfpathlineto{\pgfqpoint{3.821782in}{2.208280in}}%
\pgfpathlineto{\pgfqpoint{3.808715in}{2.209794in}}%
\pgfpathlineto{\pgfqpoint{3.795654in}{2.211487in}}%
\pgfpathlineto{\pgfqpoint{3.782599in}{2.213362in}}%
\pgfpathlineto{\pgfqpoint{3.769550in}{2.215417in}}%
\pgfpathlineto{\pgfqpoint{3.761871in}{2.205738in}}%
\pgfpathlineto{\pgfqpoint{3.754186in}{2.196065in}}%
\pgfpathlineto{\pgfqpoint{3.746496in}{2.186397in}}%
\pgfpathlineto{\pgfqpoint{3.738801in}{2.176733in}}%
\pgfpathclose%
\pgfusepath{fill}%
\end{pgfscope}%
\begin{pgfscope}%
\pgfpathrectangle{\pgfqpoint{1.254980in}{0.150000in}}{\pgfqpoint{5.490039in}{5.490039in}}%
\pgfusepath{clip}%
\pgfsetbuttcap%
\pgfsetroundjoin%
\definecolor{currentfill}{rgb}{0.174274,0.445044,0.557792}%
\pgfsetfillcolor{currentfill}%
\pgfsetfillopacity{0.700000}%
\pgfsetlinewidth{0.000000pt}%
\definecolor{currentstroke}{rgb}{0.000000,0.000000,0.000000}%
\pgfsetstrokecolor{currentstroke}%
\pgfsetdash{}{0pt}%
\pgfpathmoveto{\pgfqpoint{5.148250in}{2.925420in}}%
\pgfpathlineto{\pgfqpoint{5.161832in}{2.931119in}}%
\pgfpathlineto{\pgfqpoint{5.175429in}{2.936973in}}%
\pgfpathlineto{\pgfqpoint{5.189040in}{2.942981in}}%
\pgfpathlineto{\pgfqpoint{5.202665in}{2.949143in}}%
\pgfpathlineto{\pgfqpoint{5.209816in}{2.955211in}}%
\pgfpathlineto{\pgfqpoint{5.216963in}{2.961339in}}%
\pgfpathlineto{\pgfqpoint{5.224106in}{2.967531in}}%
\pgfpathlineto{\pgfqpoint{5.231245in}{2.973792in}}%
\pgfpathlineto{\pgfqpoint{5.217640in}{2.968092in}}%
\pgfpathlineto{\pgfqpoint{5.204049in}{2.962545in}}%
\pgfpathlineto{\pgfqpoint{5.190472in}{2.957152in}}%
\pgfpathlineto{\pgfqpoint{5.176909in}{2.951913in}}%
\pgfpathlineto{\pgfqpoint{5.169750in}{2.945181in}}%
\pgfpathlineto{\pgfqpoint{5.162587in}{2.938525in}}%
\pgfpathlineto{\pgfqpoint{5.155421in}{2.931940in}}%
\pgfpathlineto{\pgfqpoint{5.148250in}{2.925420in}}%
\pgfpathclose%
\pgfusepath{fill}%
\end{pgfscope}%
\begin{pgfscope}%
\pgfpathrectangle{\pgfqpoint{1.254980in}{0.150000in}}{\pgfqpoint{5.490039in}{5.490039in}}%
\pgfusepath{clip}%
\pgfsetbuttcap%
\pgfsetroundjoin%
\definecolor{currentfill}{rgb}{0.277134,0.185228,0.489898}%
\pgfsetfillcolor{currentfill}%
\pgfsetfillopacity{0.700000}%
\pgfsetlinewidth{0.000000pt}%
\definecolor{currentstroke}{rgb}{0.000000,0.000000,0.000000}%
\pgfsetstrokecolor{currentstroke}%
\pgfsetdash{}{0pt}%
\pgfpathmoveto{\pgfqpoint{2.872580in}{2.366608in}}%
\pgfpathlineto{\pgfqpoint{2.885695in}{2.352915in}}%
\pgfpathlineto{\pgfqpoint{2.898805in}{2.339469in}}%
\pgfpathlineto{\pgfqpoint{2.911911in}{2.326269in}}%
\pgfpathlineto{\pgfqpoint{2.925012in}{2.313312in}}%
\pgfpathlineto{\pgfqpoint{2.933039in}{2.320293in}}%
\pgfpathlineto{\pgfqpoint{2.941058in}{2.327381in}}%
\pgfpathlineto{\pgfqpoint{2.949069in}{2.334574in}}%
\pgfpathlineto{\pgfqpoint{2.957071in}{2.341871in}}%
\pgfpathlineto{\pgfqpoint{2.943993in}{2.354632in}}%
\pgfpathlineto{\pgfqpoint{2.930910in}{2.367636in}}%
\pgfpathlineto{\pgfqpoint{2.917822in}{2.380885in}}%
\pgfpathlineto{\pgfqpoint{2.904730in}{2.394381in}}%
\pgfpathlineto{\pgfqpoint{2.896706in}{2.387271in}}%
\pgfpathlineto{\pgfqpoint{2.888673in}{2.380270in}}%
\pgfpathlineto{\pgfqpoint{2.880631in}{2.373382in}}%
\pgfpathlineto{\pgfqpoint{2.872580in}{2.366608in}}%
\pgfpathclose%
\pgfusepath{fill}%
\end{pgfscope}%
\begin{pgfscope}%
\pgfpathrectangle{\pgfqpoint{1.254980in}{0.150000in}}{\pgfqpoint{5.490039in}{5.490039in}}%
\pgfusepath{clip}%
\pgfsetbuttcap%
\pgfsetroundjoin%
\definecolor{currentfill}{rgb}{0.239346,0.300855,0.540844}%
\pgfsetfillcolor{currentfill}%
\pgfsetfillopacity{0.700000}%
\pgfsetlinewidth{0.000000pt}%
\definecolor{currentstroke}{rgb}{0.000000,0.000000,0.000000}%
\pgfsetstrokecolor{currentstroke}%
\pgfsetdash{}{0pt}%
\pgfpathmoveto{\pgfqpoint{2.661862in}{2.621053in}}%
\pgfpathlineto{\pgfqpoint{2.675091in}{2.603108in}}%
\pgfpathlineto{\pgfqpoint{2.688310in}{2.585446in}}%
\pgfpathlineto{\pgfqpoint{2.701521in}{2.568066in}}%
\pgfpathlineto{\pgfqpoint{2.714723in}{2.550965in}}%
\pgfpathlineto{\pgfqpoint{2.722850in}{2.557109in}}%
\pgfpathlineto{\pgfqpoint{2.730966in}{2.563389in}}%
\pgfpathlineto{\pgfqpoint{2.739073in}{2.569802in}}%
\pgfpathlineto{\pgfqpoint{2.747169in}{2.576347in}}%
\pgfpathlineto{\pgfqpoint{2.733994in}{2.593247in}}%
\pgfpathlineto{\pgfqpoint{2.720810in}{2.610425in}}%
\pgfpathlineto{\pgfqpoint{2.707617in}{2.627884in}}%
\pgfpathlineto{\pgfqpoint{2.694416in}{2.645626in}}%
\pgfpathlineto{\pgfqpoint{2.686293in}{2.639273in}}%
\pgfpathlineto{\pgfqpoint{2.678160in}{2.633058in}}%
\pgfpathlineto{\pgfqpoint{2.670016in}{2.626985in}}%
\pgfpathlineto{\pgfqpoint{2.661862in}{2.621053in}}%
\pgfpathclose%
\pgfusepath{fill}%
\end{pgfscope}%
\begin{pgfscope}%
\pgfpathrectangle{\pgfqpoint{1.254980in}{0.150000in}}{\pgfqpoint{5.490039in}{5.490039in}}%
\pgfusepath{clip}%
\pgfsetbuttcap%
\pgfsetroundjoin%
\definecolor{currentfill}{rgb}{0.165117,0.467423,0.558141}%
\pgfsetfillcolor{currentfill}%
\pgfsetfillopacity{0.700000}%
\pgfsetlinewidth{0.000000pt}%
\definecolor{currentstroke}{rgb}{0.000000,0.000000,0.000000}%
\pgfsetstrokecolor{currentstroke}%
\pgfsetdash{}{0pt}%
\pgfpathmoveto{\pgfqpoint{5.231245in}{2.973792in}}%
\pgfpathlineto{\pgfqpoint{5.244865in}{2.979645in}}%
\pgfpathlineto{\pgfqpoint{5.258500in}{2.985652in}}%
\pgfpathlineto{\pgfqpoint{5.272149in}{2.991813in}}%
\pgfpathlineto{\pgfqpoint{5.285813in}{2.998127in}}%
\pgfpathlineto{\pgfqpoint{5.292928in}{3.003982in}}%
\pgfpathlineto{\pgfqpoint{5.300038in}{3.009911in}}%
\pgfpathlineto{\pgfqpoint{5.307145in}{3.015919in}}%
\pgfpathlineto{\pgfqpoint{5.314249in}{3.022011in}}%
\pgfpathlineto{\pgfqpoint{5.300606in}{3.016188in}}%
\pgfpathlineto{\pgfqpoint{5.286978in}{3.010517in}}%
\pgfpathlineto{\pgfqpoint{5.273365in}{3.005000in}}%
\pgfpathlineto{\pgfqpoint{5.259766in}{2.999635in}}%
\pgfpathlineto{\pgfqpoint{5.252641in}{2.993043in}}%
\pgfpathlineto{\pgfqpoint{5.245512in}{2.986543in}}%
\pgfpathlineto{\pgfqpoint{5.238381in}{2.980127in}}%
\pgfpathlineto{\pgfqpoint{5.231245in}{2.973792in}}%
\pgfpathclose%
\pgfusepath{fill}%
\end{pgfscope}%
\begin{pgfscope}%
\pgfpathrectangle{\pgfqpoint{1.254980in}{0.150000in}}{\pgfqpoint{5.490039in}{5.490039in}}%
\pgfusepath{clip}%
\pgfsetbuttcap%
\pgfsetroundjoin%
\definecolor{currentfill}{rgb}{0.280894,0.078907,0.402329}%
\pgfsetfillcolor{currentfill}%
\pgfsetfillopacity{0.700000}%
\pgfsetlinewidth{0.000000pt}%
\definecolor{currentstroke}{rgb}{0.000000,0.000000,0.000000}%
\pgfsetstrokecolor{currentstroke}%
\pgfsetdash{}{0pt}%
\pgfpathmoveto{\pgfqpoint{3.301726in}{2.138216in}}%
\pgfpathlineto{\pgfqpoint{3.314750in}{2.131160in}}%
\pgfpathlineto{\pgfqpoint{3.327776in}{2.124308in}}%
\pgfpathlineto{\pgfqpoint{3.340803in}{2.117658in}}%
\pgfpathlineto{\pgfqpoint{3.353831in}{2.111210in}}%
\pgfpathlineto{\pgfqpoint{3.361673in}{2.120033in}}%
\pgfpathlineto{\pgfqpoint{3.369508in}{2.128903in}}%
\pgfpathlineto{\pgfqpoint{3.377338in}{2.137819in}}%
\pgfpathlineto{\pgfqpoint{3.385161in}{2.146780in}}%
\pgfpathlineto{\pgfqpoint{3.372147in}{2.153094in}}%
\pgfpathlineto{\pgfqpoint{3.359134in}{2.159609in}}%
\pgfpathlineto{\pgfqpoint{3.346123in}{2.166326in}}%
\pgfpathlineto{\pgfqpoint{3.333113in}{2.173247in}}%
\pgfpathlineto{\pgfqpoint{3.325276in}{2.164410in}}%
\pgfpathlineto{\pgfqpoint{3.317432in}{2.155626in}}%
\pgfpathlineto{\pgfqpoint{3.309582in}{2.146894in}}%
\pgfpathlineto{\pgfqpoint{3.301726in}{2.138216in}}%
\pgfpathclose%
\pgfusepath{fill}%
\end{pgfscope}%
\begin{pgfscope}%
\pgfpathrectangle{\pgfqpoint{1.254980in}{0.150000in}}{\pgfqpoint{5.490039in}{5.490039in}}%
\pgfusepath{clip}%
\pgfsetbuttcap%
\pgfsetroundjoin%
\definecolor{currentfill}{rgb}{0.157729,0.485932,0.558013}%
\pgfsetfillcolor{currentfill}%
\pgfsetfillopacity{0.700000}%
\pgfsetlinewidth{0.000000pt}%
\definecolor{currentstroke}{rgb}{0.000000,0.000000,0.000000}%
\pgfsetstrokecolor{currentstroke}%
\pgfsetdash{}{0pt}%
\pgfpathmoveto{\pgfqpoint{5.314249in}{3.022011in}}%
\pgfpathlineto{\pgfqpoint{5.327907in}{3.027986in}}%
\pgfpathlineto{\pgfqpoint{5.341579in}{3.034115in}}%
\pgfpathlineto{\pgfqpoint{5.355266in}{3.040395in}}%
\pgfpathlineto{\pgfqpoint{5.368968in}{3.046829in}}%
\pgfpathlineto{\pgfqpoint{5.376046in}{3.052502in}}%
\pgfpathlineto{\pgfqpoint{5.383121in}{3.058264in}}%
\pgfpathlineto{\pgfqpoint{5.390192in}{3.064120in}}%
\pgfpathlineto{\pgfqpoint{5.397261in}{3.070078in}}%
\pgfpathlineto{\pgfqpoint{5.383582in}{3.064164in}}%
\pgfpathlineto{\pgfqpoint{5.369917in}{3.058402in}}%
\pgfpathlineto{\pgfqpoint{5.356268in}{3.052792in}}%
\pgfpathlineto{\pgfqpoint{5.342633in}{3.047334in}}%
\pgfpathlineto{\pgfqpoint{5.335541in}{3.040848in}}%
\pgfpathlineto{\pgfqpoint{5.328447in}{3.034470in}}%
\pgfpathlineto{\pgfqpoint{5.321350in}{3.028192in}}%
\pgfpathlineto{\pgfqpoint{5.314249in}{3.022011in}}%
\pgfpathclose%
\pgfusepath{fill}%
\end{pgfscope}%
\begin{pgfscope}%
\pgfpathrectangle{\pgfqpoint{1.254980in}{0.150000in}}{\pgfqpoint{5.490039in}{5.490039in}}%
\pgfusepath{clip}%
\pgfsetbuttcap%
\pgfsetroundjoin%
\definecolor{currentfill}{rgb}{0.280267,0.073417,0.397163}%
\pgfsetfillcolor{currentfill}%
\pgfsetfillopacity{0.700000}%
\pgfsetlinewidth{0.000000pt}%
\definecolor{currentstroke}{rgb}{0.000000,0.000000,0.000000}%
\pgfsetstrokecolor{currentstroke}%
\pgfsetdash{}{0pt}%
\pgfpathmoveto{\pgfqpoint{3.437238in}{2.123516in}}%
\pgfpathlineto{\pgfqpoint{3.450263in}{2.118193in}}%
\pgfpathlineto{\pgfqpoint{3.463291in}{2.113064in}}%
\pgfpathlineto{\pgfqpoint{3.476322in}{2.108129in}}%
\pgfpathlineto{\pgfqpoint{3.489356in}{2.103387in}}%
\pgfpathlineto{\pgfqpoint{3.497147in}{2.112626in}}%
\pgfpathlineto{\pgfqpoint{3.504933in}{2.121895in}}%
\pgfpathlineto{\pgfqpoint{3.512713in}{2.131193in}}%
\pgfpathlineto{\pgfqpoint{3.520487in}{2.140519in}}%
\pgfpathlineto{\pgfqpoint{3.507466in}{2.145154in}}%
\pgfpathlineto{\pgfqpoint{3.494447in}{2.149983in}}%
\pgfpathlineto{\pgfqpoint{3.481432in}{2.155005in}}%
\pgfpathlineto{\pgfqpoint{3.468419in}{2.160222in}}%
\pgfpathlineto{\pgfqpoint{3.460633in}{2.150992in}}%
\pgfpathlineto{\pgfqpoint{3.452840in}{2.141798in}}%
\pgfpathlineto{\pgfqpoint{3.445042in}{2.132639in}}%
\pgfpathlineto{\pgfqpoint{3.437238in}{2.123516in}}%
\pgfpathclose%
\pgfusepath{fill}%
\end{pgfscope}%
\begin{pgfscope}%
\pgfpathrectangle{\pgfqpoint{1.254980in}{0.150000in}}{\pgfqpoint{5.490039in}{5.490039in}}%
\pgfusepath{clip}%
\pgfsetbuttcap%
\pgfsetroundjoin%
\definecolor{currentfill}{rgb}{0.280868,0.160771,0.472899}%
\pgfsetfillcolor{currentfill}%
\pgfsetfillopacity{0.700000}%
\pgfsetlinewidth{0.000000pt}%
\definecolor{currentstroke}{rgb}{0.000000,0.000000,0.000000}%
\pgfsetstrokecolor{currentstroke}%
\pgfsetdash{}{0pt}%
\pgfpathmoveto{\pgfqpoint{2.925012in}{2.313312in}}%
\pgfpathlineto{\pgfqpoint{2.938108in}{2.300596in}}%
\pgfpathlineto{\pgfqpoint{2.951201in}{2.288120in}}%
\pgfpathlineto{\pgfqpoint{2.964289in}{2.275881in}}%
\pgfpathlineto{\pgfqpoint{2.977375in}{2.263878in}}%
\pgfpathlineto{\pgfqpoint{2.985380in}{2.271065in}}%
\pgfpathlineto{\pgfqpoint{2.993378in}{2.278352in}}%
\pgfpathlineto{\pgfqpoint{3.001367in}{2.285737in}}%
\pgfpathlineto{\pgfqpoint{3.009348in}{2.293218in}}%
\pgfpathlineto{\pgfqpoint{2.996284in}{2.305026in}}%
\pgfpathlineto{\pgfqpoint{2.983217in}{2.317069in}}%
\pgfpathlineto{\pgfqpoint{2.970146in}{2.329350in}}%
\pgfpathlineto{\pgfqpoint{2.957071in}{2.341871in}}%
\pgfpathlineto{\pgfqpoint{2.949069in}{2.334574in}}%
\pgfpathlineto{\pgfqpoint{2.941058in}{2.327381in}}%
\pgfpathlineto{\pgfqpoint{2.933039in}{2.320293in}}%
\pgfpathlineto{\pgfqpoint{2.925012in}{2.313312in}}%
\pgfpathclose%
\pgfusepath{fill}%
\end{pgfscope}%
\begin{pgfscope}%
\pgfpathrectangle{\pgfqpoint{1.254980in}{0.150000in}}{\pgfqpoint{5.490039in}{5.490039in}}%
\pgfusepath{clip}%
\pgfsetbuttcap%
\pgfsetroundjoin%
\definecolor{currentfill}{rgb}{0.150476,0.504369,0.557430}%
\pgfsetfillcolor{currentfill}%
\pgfsetfillopacity{0.700000}%
\pgfsetlinewidth{0.000000pt}%
\definecolor{currentstroke}{rgb}{0.000000,0.000000,0.000000}%
\pgfsetstrokecolor{currentstroke}%
\pgfsetdash{}{0pt}%
\pgfpathmoveto{\pgfqpoint{5.397261in}{3.070078in}}%
\pgfpathlineto{\pgfqpoint{5.410955in}{3.076143in}}%
\pgfpathlineto{\pgfqpoint{5.424665in}{3.082361in}}%
\pgfpathlineto{\pgfqpoint{5.438390in}{3.088730in}}%
\pgfpathlineto{\pgfqpoint{5.452130in}{3.095251in}}%
\pgfpathlineto{\pgfqpoint{5.459171in}{3.100777in}}%
\pgfpathlineto{\pgfqpoint{5.466210in}{3.106409in}}%
\pgfpathlineto{\pgfqpoint{5.473247in}{3.112152in}}%
\pgfpathlineto{\pgfqpoint{5.480281in}{3.118014in}}%
\pgfpathlineto{\pgfqpoint{5.466566in}{3.112041in}}%
\pgfpathlineto{\pgfqpoint{5.452866in}{3.106220in}}%
\pgfpathlineto{\pgfqpoint{5.439181in}{3.100549in}}%
\pgfpathlineto{\pgfqpoint{5.425510in}{3.095030in}}%
\pgfpathlineto{\pgfqpoint{5.418451in}{3.088611in}}%
\pgfpathlineto{\pgfqpoint{5.411390in}{3.082317in}}%
\pgfpathlineto{\pgfqpoint{5.404327in}{3.076141in}}%
\pgfpathlineto{\pgfqpoint{5.397261in}{3.070078in}}%
\pgfpathclose%
\pgfusepath{fill}%
\end{pgfscope}%
\begin{pgfscope}%
\pgfpathrectangle{\pgfqpoint{1.254980in}{0.150000in}}{\pgfqpoint{5.490039in}{5.490039in}}%
\pgfusepath{clip}%
\pgfsetbuttcap%
\pgfsetroundjoin%
\definecolor{currentfill}{rgb}{0.281924,0.089666,0.412415}%
\pgfsetfillcolor{currentfill}%
\pgfsetfillopacity{0.700000}%
\pgfsetlinewidth{0.000000pt}%
\definecolor{currentstroke}{rgb}{0.000000,0.000000,0.000000}%
\pgfsetstrokecolor{currentstroke}%
\pgfsetdash{}{0pt}%
\pgfpathmoveto{\pgfqpoint{3.655750in}{2.148473in}}%
\pgfpathlineto{\pgfqpoint{3.668797in}{2.145612in}}%
\pgfpathlineto{\pgfqpoint{3.681849in}{2.142935in}}%
\pgfpathlineto{\pgfqpoint{3.694907in}{2.140441in}}%
\pgfpathlineto{\pgfqpoint{3.707969in}{2.138130in}}%
\pgfpathlineto{\pgfqpoint{3.715685in}{2.147773in}}%
\pgfpathlineto{\pgfqpoint{3.723395in}{2.157422in}}%
\pgfpathlineto{\pgfqpoint{3.731101in}{2.167075in}}%
\pgfpathlineto{\pgfqpoint{3.738801in}{2.176733in}}%
\pgfpathlineto{\pgfqpoint{3.725748in}{2.178994in}}%
\pgfpathlineto{\pgfqpoint{3.712700in}{2.181437in}}%
\pgfpathlineto{\pgfqpoint{3.699658in}{2.184063in}}%
\pgfpathlineto{\pgfqpoint{3.686621in}{2.186874in}}%
\pgfpathlineto{\pgfqpoint{3.678911in}{2.177256in}}%
\pgfpathlineto{\pgfqpoint{3.671196in}{2.167650in}}%
\pgfpathlineto{\pgfqpoint{3.663475in}{2.158055in}}%
\pgfpathlineto{\pgfqpoint{3.655750in}{2.148473in}}%
\pgfpathclose%
\pgfusepath{fill}%
\end{pgfscope}%
\begin{pgfscope}%
\pgfpathrectangle{\pgfqpoint{1.254980in}{0.150000in}}{\pgfqpoint{5.490039in}{5.490039in}}%
\pgfusepath{clip}%
\pgfsetbuttcap%
\pgfsetroundjoin%
\definecolor{currentfill}{rgb}{0.223925,0.334994,0.548053}%
\pgfsetfillcolor{currentfill}%
\pgfsetfillopacity{0.700000}%
\pgfsetlinewidth{0.000000pt}%
\definecolor{currentstroke}{rgb}{0.000000,0.000000,0.000000}%
\pgfsetstrokecolor{currentstroke}%
\pgfsetdash{}{0pt}%
\pgfpathmoveto{\pgfqpoint{2.608851in}{2.695728in}}%
\pgfpathlineto{\pgfqpoint{2.622119in}{2.676620in}}%
\pgfpathlineto{\pgfqpoint{2.635376in}{2.657807in}}%
\pgfpathlineto{\pgfqpoint{2.648624in}{2.639285in}}%
\pgfpathlineto{\pgfqpoint{2.661862in}{2.621053in}}%
\pgfpathlineto{\pgfqpoint{2.670016in}{2.626985in}}%
\pgfpathlineto{\pgfqpoint{2.678160in}{2.633058in}}%
\pgfpathlineto{\pgfqpoint{2.686293in}{2.639273in}}%
\pgfpathlineto{\pgfqpoint{2.694416in}{2.645626in}}%
\pgfpathlineto{\pgfqpoint{2.681206in}{2.663655in}}%
\pgfpathlineto{\pgfqpoint{2.667986in}{2.681972in}}%
\pgfpathlineto{\pgfqpoint{2.654757in}{2.700581in}}%
\pgfpathlineto{\pgfqpoint{2.641517in}{2.719484in}}%
\pgfpathlineto{\pgfqpoint{2.633367in}{2.713324in}}%
\pgfpathlineto{\pgfqpoint{2.625206in}{2.707310in}}%
\pgfpathlineto{\pgfqpoint{2.617034in}{2.701444in}}%
\pgfpathlineto{\pgfqpoint{2.608851in}{2.695728in}}%
\pgfpathclose%
\pgfusepath{fill}%
\end{pgfscope}%
\begin{pgfscope}%
\pgfpathrectangle{\pgfqpoint{1.254980in}{0.150000in}}{\pgfqpoint{5.490039in}{5.490039in}}%
\pgfusepath{clip}%
\pgfsetbuttcap%
\pgfsetroundjoin%
\definecolor{currentfill}{rgb}{0.282327,0.094955,0.417331}%
\pgfsetfillcolor{currentfill}%
\pgfsetfillopacity{0.700000}%
\pgfsetlinewidth{0.000000pt}%
\definecolor{currentstroke}{rgb}{0.000000,0.000000,0.000000}%
\pgfsetstrokecolor{currentstroke}%
\pgfsetdash{}{0pt}%
\pgfpathmoveto{\pgfqpoint{3.165928in}{2.169295in}}%
\pgfpathlineto{\pgfqpoint{3.178967in}{2.160402in}}%
\pgfpathlineto{\pgfqpoint{3.192005in}{2.151723in}}%
\pgfpathlineto{\pgfqpoint{3.205043in}{2.143257in}}%
\pgfpathlineto{\pgfqpoint{3.218081in}{2.135001in}}%
\pgfpathlineto{\pgfqpoint{3.225980in}{2.143279in}}%
\pgfpathlineto{\pgfqpoint{3.233872in}{2.151622in}}%
\pgfpathlineto{\pgfqpoint{3.241757in}{2.160029in}}%
\pgfpathlineto{\pgfqpoint{3.249636in}{2.168499in}}%
\pgfpathlineto{\pgfqpoint{3.236614in}{2.176591in}}%
\pgfpathlineto{\pgfqpoint{3.223593in}{2.184894in}}%
\pgfpathlineto{\pgfqpoint{3.210572in}{2.193409in}}%
\pgfpathlineto{\pgfqpoint{3.197550in}{2.202138in}}%
\pgfpathlineto{\pgfqpoint{3.189655in}{2.193821in}}%
\pgfpathlineto{\pgfqpoint{3.181753in}{2.185574in}}%
\pgfpathlineto{\pgfqpoint{3.173844in}{2.177399in}}%
\pgfpathlineto{\pgfqpoint{3.165928in}{2.169295in}}%
\pgfpathclose%
\pgfusepath{fill}%
\end{pgfscope}%
\begin{pgfscope}%
\pgfpathrectangle{\pgfqpoint{1.254980in}{0.150000in}}{\pgfqpoint{5.490039in}{5.490039in}}%
\pgfusepath{clip}%
\pgfsetbuttcap%
\pgfsetroundjoin%
\definecolor{currentfill}{rgb}{0.143343,0.522773,0.556295}%
\pgfsetfillcolor{currentfill}%
\pgfsetfillopacity{0.700000}%
\pgfsetlinewidth{0.000000pt}%
\definecolor{currentstroke}{rgb}{0.000000,0.000000,0.000000}%
\pgfsetstrokecolor{currentstroke}%
\pgfsetdash{}{0pt}%
\pgfpathmoveto{\pgfqpoint{5.480281in}{3.118014in}}%
\pgfpathlineto{\pgfqpoint{5.494012in}{3.124138in}}%
\pgfpathlineto{\pgfqpoint{5.507757in}{3.130412in}}%
\pgfpathlineto{\pgfqpoint{5.521519in}{3.136838in}}%
\pgfpathlineto{\pgfqpoint{5.535296in}{3.143415in}}%
\pgfpathlineto{\pgfqpoint{5.542302in}{3.148835in}}%
\pgfpathlineto{\pgfqpoint{5.549306in}{3.154378in}}%
\pgfpathlineto{\pgfqpoint{5.556309in}{3.160052in}}%
\pgfpathlineto{\pgfqpoint{5.563309in}{3.165862in}}%
\pgfpathlineto{\pgfqpoint{5.549559in}{3.159862in}}%
\pgfpathlineto{\pgfqpoint{5.535824in}{3.154013in}}%
\pgfpathlineto{\pgfqpoint{5.522104in}{3.148314in}}%
\pgfpathlineto{\pgfqpoint{5.508400in}{3.142765in}}%
\pgfpathlineto{\pgfqpoint{5.501372in}{3.136369in}}%
\pgfpathlineto{\pgfqpoint{5.494344in}{3.130116in}}%
\pgfpathlineto{\pgfqpoint{5.487313in}{3.124000in}}%
\pgfpathlineto{\pgfqpoint{5.480281in}{3.118014in}}%
\pgfpathclose%
\pgfusepath{fill}%
\end{pgfscope}%
\begin{pgfscope}%
\pgfpathrectangle{\pgfqpoint{1.254980in}{0.150000in}}{\pgfqpoint{5.490039in}{5.490039in}}%
\pgfusepath{clip}%
\pgfsetbuttcap%
\pgfsetroundjoin%
\definecolor{currentfill}{rgb}{0.136408,0.541173,0.554483}%
\pgfsetfillcolor{currentfill}%
\pgfsetfillopacity{0.700000}%
\pgfsetlinewidth{0.000000pt}%
\definecolor{currentstroke}{rgb}{0.000000,0.000000,0.000000}%
\pgfsetstrokecolor{currentstroke}%
\pgfsetdash{}{0pt}%
\pgfpathmoveto{\pgfqpoint{5.563309in}{3.165862in}}%
\pgfpathlineto{\pgfqpoint{5.577075in}{3.172012in}}%
\pgfpathlineto{\pgfqpoint{5.590857in}{3.178313in}}%
\pgfpathlineto{\pgfqpoint{5.604654in}{3.184763in}}%
\pgfpathlineto{\pgfqpoint{5.618468in}{3.191364in}}%
\pgfpathlineto{\pgfqpoint{5.625439in}{3.196723in}}%
\pgfpathlineto{\pgfqpoint{5.632410in}{3.202225in}}%
\pgfpathlineto{\pgfqpoint{5.639379in}{3.207877in}}%
\pgfpathlineto{\pgfqpoint{5.646348in}{3.213685in}}%
\pgfpathlineto{\pgfqpoint{5.632563in}{3.207690in}}%
\pgfpathlineto{\pgfqpoint{5.618794in}{3.201844in}}%
\pgfpathlineto{\pgfqpoint{5.605041in}{3.196148in}}%
\pgfpathlineto{\pgfqpoint{5.591303in}{3.190601in}}%
\pgfpathlineto{\pgfqpoint{5.584305in}{3.184179in}}%
\pgfpathlineto{\pgfqpoint{5.577308in}{3.177919in}}%
\pgfpathlineto{\pgfqpoint{5.570309in}{3.171816in}}%
\pgfpathlineto{\pgfqpoint{5.563309in}{3.165862in}}%
\pgfpathclose%
\pgfusepath{fill}%
\end{pgfscope}%
\begin{pgfscope}%
\pgfpathrectangle{\pgfqpoint{1.254980in}{0.150000in}}{\pgfqpoint{5.490039in}{5.490039in}}%
\pgfusepath{clip}%
\pgfsetbuttcap%
\pgfsetroundjoin%
\definecolor{currentfill}{rgb}{0.129933,0.559582,0.551864}%
\pgfsetfillcolor{currentfill}%
\pgfsetfillopacity{0.700000}%
\pgfsetlinewidth{0.000000pt}%
\definecolor{currentstroke}{rgb}{0.000000,0.000000,0.000000}%
\pgfsetstrokecolor{currentstroke}%
\pgfsetdash{}{0pt}%
\pgfpathmoveto{\pgfqpoint{5.646348in}{3.213685in}}%
\pgfpathlineto{\pgfqpoint{5.660148in}{3.219830in}}%
\pgfpathlineto{\pgfqpoint{5.673965in}{3.226124in}}%
\pgfpathlineto{\pgfqpoint{5.687797in}{3.232568in}}%
\pgfpathlineto{\pgfqpoint{5.701645in}{3.239162in}}%
\pgfpathlineto{\pgfqpoint{5.708584in}{3.244510in}}%
\pgfpathlineto{\pgfqpoint{5.715522in}{3.250022in}}%
\pgfpathlineto{\pgfqpoint{5.722460in}{3.255706in}}%
\pgfpathlineto{\pgfqpoint{5.729399in}{3.261567in}}%
\pgfpathlineto{\pgfqpoint{5.715581in}{3.255607in}}%
\pgfpathlineto{\pgfqpoint{5.701779in}{3.249797in}}%
\pgfpathlineto{\pgfqpoint{5.687993in}{3.244135in}}%
\pgfpathlineto{\pgfqpoint{5.674222in}{3.238622in}}%
\pgfpathlineto{\pgfqpoint{5.667253in}{3.232119in}}%
\pgfpathlineto{\pgfqpoint{5.660285in}{3.225799in}}%
\pgfpathlineto{\pgfqpoint{5.653316in}{3.219657in}}%
\pgfpathlineto{\pgfqpoint{5.646348in}{3.213685in}}%
\pgfpathclose%
\pgfusepath{fill}%
\end{pgfscope}%
\begin{pgfscope}%
\pgfpathrectangle{\pgfqpoint{1.254980in}{0.150000in}}{\pgfqpoint{5.490039in}{5.490039in}}%
\pgfusepath{clip}%
\pgfsetbuttcap%
\pgfsetroundjoin%
\definecolor{currentfill}{rgb}{0.282623,0.140926,0.457517}%
\pgfsetfillcolor{currentfill}%
\pgfsetfillopacity{0.700000}%
\pgfsetlinewidth{0.000000pt}%
\definecolor{currentstroke}{rgb}{0.000000,0.000000,0.000000}%
\pgfsetstrokecolor{currentstroke}%
\pgfsetdash{}{0pt}%
\pgfpathmoveto{\pgfqpoint{2.977375in}{2.263878in}}%
\pgfpathlineto{\pgfqpoint{2.990456in}{2.252110in}}%
\pgfpathlineto{\pgfqpoint{3.003535in}{2.240573in}}%
\pgfpathlineto{\pgfqpoint{3.016610in}{2.229268in}}%
\pgfpathlineto{\pgfqpoint{3.029683in}{2.218191in}}%
\pgfpathlineto{\pgfqpoint{3.037668in}{2.225583in}}%
\pgfpathlineto{\pgfqpoint{3.045644in}{2.233068in}}%
\pgfpathlineto{\pgfqpoint{3.053613in}{2.240643in}}%
\pgfpathlineto{\pgfqpoint{3.061574in}{2.248308in}}%
\pgfpathlineto{\pgfqpoint{3.048521in}{2.259191in}}%
\pgfpathlineto{\pgfqpoint{3.035466in}{2.270302in}}%
\pgfpathlineto{\pgfqpoint{3.022408in}{2.281644in}}%
\pgfpathlineto{\pgfqpoint{3.009348in}{2.293218in}}%
\pgfpathlineto{\pgfqpoint{3.001367in}{2.285737in}}%
\pgfpathlineto{\pgfqpoint{2.993378in}{2.278352in}}%
\pgfpathlineto{\pgfqpoint{2.985380in}{2.271065in}}%
\pgfpathlineto{\pgfqpoint{2.977375in}{2.263878in}}%
\pgfpathclose%
\pgfusepath{fill}%
\end{pgfscope}%
\begin{pgfscope}%
\pgfpathrectangle{\pgfqpoint{1.254980in}{0.150000in}}{\pgfqpoint{5.490039in}{5.490039in}}%
\pgfusepath{clip}%
\pgfsetbuttcap%
\pgfsetroundjoin%
\definecolor{currentfill}{rgb}{0.124395,0.578002,0.548287}%
\pgfsetfillcolor{currentfill}%
\pgfsetfillopacity{0.700000}%
\pgfsetlinewidth{0.000000pt}%
\definecolor{currentstroke}{rgb}{0.000000,0.000000,0.000000}%
\pgfsetstrokecolor{currentstroke}%
\pgfsetdash{}{0pt}%
\pgfpathmoveto{\pgfqpoint{5.729399in}{3.261567in}}%
\pgfpathlineto{\pgfqpoint{5.743233in}{3.267675in}}%
\pgfpathlineto{\pgfqpoint{5.757083in}{3.273932in}}%
\pgfpathlineto{\pgfqpoint{5.770949in}{3.280338in}}%
\pgfpathlineto{\pgfqpoint{5.784831in}{3.286893in}}%
\pgfpathlineto{\pgfqpoint{5.791738in}{3.292286in}}%
\pgfpathlineto{\pgfqpoint{5.798646in}{3.297866in}}%
\pgfpathlineto{\pgfqpoint{5.805556in}{3.303639in}}%
\pgfpathlineto{\pgfqpoint{5.812466in}{3.309612in}}%
\pgfpathlineto{\pgfqpoint{5.798616in}{3.303720in}}%
\pgfpathlineto{\pgfqpoint{5.784783in}{3.297975in}}%
\pgfpathlineto{\pgfqpoint{5.770965in}{3.292379in}}%
\pgfpathlineto{\pgfqpoint{5.757162in}{3.286932in}}%
\pgfpathlineto{\pgfqpoint{5.750220in}{3.280288in}}%
\pgfpathlineto{\pgfqpoint{5.743278in}{3.273851in}}%
\pgfpathlineto{\pgfqpoint{5.736338in}{3.267613in}}%
\pgfpathlineto{\pgfqpoint{5.729399in}{3.261567in}}%
\pgfpathclose%
\pgfusepath{fill}%
\end{pgfscope}%
\begin{pgfscope}%
\pgfpathrectangle{\pgfqpoint{1.254980in}{0.150000in}}{\pgfqpoint{5.490039in}{5.490039in}}%
\pgfusepath{clip}%
\pgfsetbuttcap%
\pgfsetroundjoin%
\definecolor{currentfill}{rgb}{0.206756,0.371758,0.553117}%
\pgfsetfillcolor{currentfill}%
\pgfsetfillopacity{0.700000}%
\pgfsetlinewidth{0.000000pt}%
\definecolor{currentstroke}{rgb}{0.000000,0.000000,0.000000}%
\pgfsetstrokecolor{currentstroke}%
\pgfsetdash{}{0pt}%
\pgfpathmoveto{\pgfqpoint{2.555671in}{2.775168in}}%
\pgfpathlineto{\pgfqpoint{2.568983in}{2.754851in}}%
\pgfpathlineto{\pgfqpoint{2.582283in}{2.734841in}}%
\pgfpathlineto{\pgfqpoint{2.595573in}{2.715134in}}%
\pgfpathlineto{\pgfqpoint{2.608851in}{2.695728in}}%
\pgfpathlineto{\pgfqpoint{2.617034in}{2.701444in}}%
\pgfpathlineto{\pgfqpoint{2.625206in}{2.707310in}}%
\pgfpathlineto{\pgfqpoint{2.633367in}{2.713324in}}%
\pgfpathlineto{\pgfqpoint{2.641517in}{2.719484in}}%
\pgfpathlineto{\pgfqpoint{2.628268in}{2.738685in}}%
\pgfpathlineto{\pgfqpoint{2.615008in}{2.758186in}}%
\pgfpathlineto{\pgfqpoint{2.601737in}{2.777989in}}%
\pgfpathlineto{\pgfqpoint{2.588455in}{2.798099in}}%
\pgfpathlineto{\pgfqpoint{2.580276in}{2.792134in}}%
\pgfpathlineto{\pgfqpoint{2.572085in}{2.786323in}}%
\pgfpathlineto{\pgfqpoint{2.563884in}{2.780667in}}%
\pgfpathlineto{\pgfqpoint{2.555671in}{2.775168in}}%
\pgfpathclose%
\pgfusepath{fill}%
\end{pgfscope}%
\begin{pgfscope}%
\pgfpathrectangle{\pgfqpoint{1.254980in}{0.150000in}}{\pgfqpoint{5.490039in}{5.490039in}}%
\pgfusepath{clip}%
\pgfsetbuttcap%
\pgfsetroundjoin%
\definecolor{currentfill}{rgb}{0.280894,0.078907,0.402329}%
\pgfsetfillcolor{currentfill}%
\pgfsetfillopacity{0.700000}%
\pgfsetlinewidth{0.000000pt}%
\definecolor{currentstroke}{rgb}{0.000000,0.000000,0.000000}%
\pgfsetstrokecolor{currentstroke}%
\pgfsetdash{}{0pt}%
\pgfpathmoveto{\pgfqpoint{3.572609in}{2.123887in}}%
\pgfpathlineto{\pgfqpoint{3.585649in}{2.120202in}}%
\pgfpathlineto{\pgfqpoint{3.598694in}{2.116705in}}%
\pgfpathlineto{\pgfqpoint{3.611742in}{2.113395in}}%
\pgfpathlineto{\pgfqpoint{3.624795in}{2.110270in}}%
\pgfpathlineto{\pgfqpoint{3.632542in}{2.119801in}}%
\pgfpathlineto{\pgfqpoint{3.640283in}{2.129346in}}%
\pgfpathlineto{\pgfqpoint{3.648019in}{2.138903in}}%
\pgfpathlineto{\pgfqpoint{3.655750in}{2.148473in}}%
\pgfpathlineto{\pgfqpoint{3.642707in}{2.151520in}}%
\pgfpathlineto{\pgfqpoint{3.629669in}{2.154752in}}%
\pgfpathlineto{\pgfqpoint{3.616636in}{2.158171in}}%
\pgfpathlineto{\pgfqpoint{3.603607in}{2.161777in}}%
\pgfpathlineto{\pgfqpoint{3.595866in}{2.152275in}}%
\pgfpathlineto{\pgfqpoint{3.588119in}{2.142792in}}%
\pgfpathlineto{\pgfqpoint{3.580367in}{2.133330in}}%
\pgfpathlineto{\pgfqpoint{3.572609in}{2.123887in}}%
\pgfpathclose%
\pgfusepath{fill}%
\end{pgfscope}%
\begin{pgfscope}%
\pgfpathrectangle{\pgfqpoint{1.254980in}{0.150000in}}{\pgfqpoint{5.490039in}{5.490039in}}%
\pgfusepath{clip}%
\pgfsetbuttcap%
\pgfsetroundjoin%
\definecolor{currentfill}{rgb}{0.271828,0.209303,0.504434}%
\pgfsetfillcolor{currentfill}%
\pgfsetfillopacity{0.700000}%
\pgfsetlinewidth{0.000000pt}%
\definecolor{currentstroke}{rgb}{0.000000,0.000000,0.000000}%
\pgfsetstrokecolor{currentstroke}%
\pgfsetdash{}{0pt}%
\pgfpathmoveto{\pgfqpoint{4.206026in}{2.366393in}}%
\pgfpathlineto{\pgfqpoint{4.219224in}{2.368330in}}%
\pgfpathlineto{\pgfqpoint{4.232432in}{2.370435in}}%
\pgfpathlineto{\pgfqpoint{4.245649in}{2.372707in}}%
\pgfpathlineto{\pgfqpoint{4.258876in}{2.375146in}}%
\pgfpathlineto{\pgfqpoint{4.266413in}{2.384281in}}%
\pgfpathlineto{\pgfqpoint{4.273944in}{2.393389in}}%
\pgfpathlineto{\pgfqpoint{4.281471in}{2.402473in}}%
\pgfpathlineto{\pgfqpoint{4.288992in}{2.411533in}}%
\pgfpathlineto{\pgfqpoint{4.275774in}{2.409212in}}%
\pgfpathlineto{\pgfqpoint{4.262565in}{2.407058in}}%
\pgfpathlineto{\pgfqpoint{4.249365in}{2.405071in}}%
\pgfpathlineto{\pgfqpoint{4.236175in}{2.403251in}}%
\pgfpathlineto{\pgfqpoint{4.228646in}{2.394063in}}%
\pgfpathlineto{\pgfqpoint{4.221111in}{2.384858in}}%
\pgfpathlineto{\pgfqpoint{4.213571in}{2.375635in}}%
\pgfpathlineto{\pgfqpoint{4.206026in}{2.366393in}}%
\pgfpathclose%
\pgfusepath{fill}%
\end{pgfscope}%
\begin{pgfscope}%
\pgfpathrectangle{\pgfqpoint{1.254980in}{0.150000in}}{\pgfqpoint{5.490039in}{5.490039in}}%
\pgfusepath{clip}%
\pgfsetbuttcap%
\pgfsetroundjoin%
\definecolor{currentfill}{rgb}{0.277134,0.185228,0.489898}%
\pgfsetfillcolor{currentfill}%
\pgfsetfillopacity{0.700000}%
\pgfsetlinewidth{0.000000pt}%
\definecolor{currentstroke}{rgb}{0.000000,0.000000,0.000000}%
\pgfsetstrokecolor{currentstroke}%
\pgfsetdash{}{0pt}%
\pgfpathmoveto{\pgfqpoint{4.123064in}{2.322782in}}%
\pgfpathlineto{\pgfqpoint{4.136234in}{2.324132in}}%
\pgfpathlineto{\pgfqpoint{4.149412in}{2.325653in}}%
\pgfpathlineto{\pgfqpoint{4.162600in}{2.327342in}}%
\pgfpathlineto{\pgfqpoint{4.175796in}{2.329200in}}%
\pgfpathlineto{\pgfqpoint{4.183361in}{2.338535in}}%
\pgfpathlineto{\pgfqpoint{4.190921in}{2.347845in}}%
\pgfpathlineto{\pgfqpoint{4.198477in}{2.357130in}}%
\pgfpathlineto{\pgfqpoint{4.206026in}{2.366393in}}%
\pgfpathlineto{\pgfqpoint{4.192838in}{2.364625in}}%
\pgfpathlineto{\pgfqpoint{4.179658in}{2.363025in}}%
\pgfpathlineto{\pgfqpoint{4.166488in}{2.361594in}}%
\pgfpathlineto{\pgfqpoint{4.153326in}{2.360333in}}%
\pgfpathlineto{\pgfqpoint{4.145768in}{2.350970in}}%
\pgfpathlineto{\pgfqpoint{4.138205in}{2.341591in}}%
\pgfpathlineto{\pgfqpoint{4.130637in}{2.332196in}}%
\pgfpathlineto{\pgfqpoint{4.123064in}{2.322782in}}%
\pgfpathclose%
\pgfusepath{fill}%
\end{pgfscope}%
\begin{pgfscope}%
\pgfpathrectangle{\pgfqpoint{1.254980in}{0.150000in}}{\pgfqpoint{5.490039in}{5.490039in}}%
\pgfusepath{clip}%
\pgfsetbuttcap%
\pgfsetroundjoin%
\definecolor{currentfill}{rgb}{0.265145,0.232956,0.516599}%
\pgfsetfillcolor{currentfill}%
\pgfsetfillopacity{0.700000}%
\pgfsetlinewidth{0.000000pt}%
\definecolor{currentstroke}{rgb}{0.000000,0.000000,0.000000}%
\pgfsetstrokecolor{currentstroke}%
\pgfsetdash{}{0pt}%
\pgfpathmoveto{\pgfqpoint{4.288992in}{2.411533in}}%
\pgfpathlineto{\pgfqpoint{4.302221in}{2.414021in}}%
\pgfpathlineto{\pgfqpoint{4.315460in}{2.416675in}}%
\pgfpathlineto{\pgfqpoint{4.328708in}{2.419495in}}%
\pgfpathlineto{\pgfqpoint{4.341967in}{2.422480in}}%
\pgfpathlineto{\pgfqpoint{4.349475in}{2.431383in}}%
\pgfpathlineto{\pgfqpoint{4.356978in}{2.440260in}}%
\pgfpathlineto{\pgfqpoint{4.364476in}{2.449112in}}%
\pgfpathlineto{\pgfqpoint{4.371968in}{2.457942in}}%
\pgfpathlineto{\pgfqpoint{4.358717in}{2.455103in}}%
\pgfpathlineto{\pgfqpoint{4.345477in}{2.452430in}}%
\pgfpathlineto{\pgfqpoint{4.332247in}{2.449922in}}%
\pgfpathlineto{\pgfqpoint{4.319027in}{2.447580in}}%
\pgfpathlineto{\pgfqpoint{4.311526in}{2.438593in}}%
\pgfpathlineto{\pgfqpoint{4.304020in}{2.429592in}}%
\pgfpathlineto{\pgfqpoint{4.296509in}{2.420572in}}%
\pgfpathlineto{\pgfqpoint{4.288992in}{2.411533in}}%
\pgfpathclose%
\pgfusepath{fill}%
\end{pgfscope}%
\begin{pgfscope}%
\pgfpathrectangle{\pgfqpoint{1.254980in}{0.150000in}}{\pgfqpoint{5.490039in}{5.490039in}}%
\pgfusepath{clip}%
\pgfsetbuttcap%
\pgfsetroundjoin%
\definecolor{currentfill}{rgb}{0.280255,0.165693,0.476498}%
\pgfsetfillcolor{currentfill}%
\pgfsetfillopacity{0.700000}%
\pgfsetlinewidth{0.000000pt}%
\definecolor{currentstroke}{rgb}{0.000000,0.000000,0.000000}%
\pgfsetstrokecolor{currentstroke}%
\pgfsetdash{}{0pt}%
\pgfpathmoveto{\pgfqpoint{4.040098in}{2.280978in}}%
\pgfpathlineto{\pgfqpoint{4.053241in}{2.281706in}}%
\pgfpathlineto{\pgfqpoint{4.066393in}{2.282606in}}%
\pgfpathlineto{\pgfqpoint{4.079553in}{2.283678in}}%
\pgfpathlineto{\pgfqpoint{4.092722in}{2.284920in}}%
\pgfpathlineto{\pgfqpoint{4.100315in}{2.294419in}}%
\pgfpathlineto{\pgfqpoint{4.107903in}{2.303894in}}%
\pgfpathlineto{\pgfqpoint{4.115486in}{2.313348in}}%
\pgfpathlineto{\pgfqpoint{4.123064in}{2.322782in}}%
\pgfpathlineto{\pgfqpoint{4.109903in}{2.321601in}}%
\pgfpathlineto{\pgfqpoint{4.096751in}{2.320591in}}%
\pgfpathlineto{\pgfqpoint{4.083607in}{2.319752in}}%
\pgfpathlineto{\pgfqpoint{4.070472in}{2.319085in}}%
\pgfpathlineto{\pgfqpoint{4.062886in}{2.309580in}}%
\pgfpathlineto{\pgfqpoint{4.055295in}{2.300061in}}%
\pgfpathlineto{\pgfqpoint{4.047699in}{2.290527in}}%
\pgfpathlineto{\pgfqpoint{4.040098in}{2.280978in}}%
\pgfpathclose%
\pgfusepath{fill}%
\end{pgfscope}%
\begin{pgfscope}%
\pgfpathrectangle{\pgfqpoint{1.254980in}{0.150000in}}{\pgfqpoint{5.490039in}{5.490039in}}%
\pgfusepath{clip}%
\pgfsetbuttcap%
\pgfsetroundjoin%
\definecolor{currentfill}{rgb}{0.257322,0.256130,0.526563}%
\pgfsetfillcolor{currentfill}%
\pgfsetfillopacity{0.700000}%
\pgfsetlinewidth{0.000000pt}%
\definecolor{currentstroke}{rgb}{0.000000,0.000000,0.000000}%
\pgfsetstrokecolor{currentstroke}%
\pgfsetdash{}{0pt}%
\pgfpathmoveto{\pgfqpoint{4.371968in}{2.457942in}}%
\pgfpathlineto{\pgfqpoint{4.385229in}{2.460946in}}%
\pgfpathlineto{\pgfqpoint{4.398500in}{2.464114in}}%
\pgfpathlineto{\pgfqpoint{4.411783in}{2.467447in}}%
\pgfpathlineto{\pgfqpoint{4.425076in}{2.470945in}}%
\pgfpathlineto{\pgfqpoint{4.432554in}{2.479589in}}%
\pgfpathlineto{\pgfqpoint{4.440027in}{2.488209in}}%
\pgfpathlineto{\pgfqpoint{4.447495in}{2.496806in}}%
\pgfpathlineto{\pgfqpoint{4.454958in}{2.505382in}}%
\pgfpathlineto{\pgfqpoint{4.441674in}{2.502060in}}%
\pgfpathlineto{\pgfqpoint{4.428400in}{2.498901in}}%
\pgfpathlineto{\pgfqpoint{4.415138in}{2.495907in}}%
\pgfpathlineto{\pgfqpoint{4.401885in}{2.493077in}}%
\pgfpathlineto{\pgfqpoint{4.394414in}{2.484317in}}%
\pgfpathlineto{\pgfqpoint{4.386937in}{2.475542in}}%
\pgfpathlineto{\pgfqpoint{4.379455in}{2.466751in}}%
\pgfpathlineto{\pgfqpoint{4.371968in}{2.457942in}}%
\pgfpathclose%
\pgfusepath{fill}%
\end{pgfscope}%
\begin{pgfscope}%
\pgfpathrectangle{\pgfqpoint{1.254980in}{0.150000in}}{\pgfqpoint{5.490039in}{5.490039in}}%
\pgfusepath{clip}%
\pgfsetbuttcap%
\pgfsetroundjoin%
\definecolor{currentfill}{rgb}{0.120565,0.596422,0.543611}%
\pgfsetfillcolor{currentfill}%
\pgfsetfillopacity{0.700000}%
\pgfsetlinewidth{0.000000pt}%
\definecolor{currentstroke}{rgb}{0.000000,0.000000,0.000000}%
\pgfsetstrokecolor{currentstroke}%
\pgfsetdash{}{0pt}%
\pgfpathmoveto{\pgfqpoint{5.812466in}{3.309612in}}%
\pgfpathlineto{\pgfqpoint{5.826332in}{3.315652in}}%
\pgfpathlineto{\pgfqpoint{5.840214in}{3.321840in}}%
\pgfpathlineto{\pgfqpoint{5.854113in}{3.328177in}}%
\pgfpathlineto{\pgfqpoint{5.868028in}{3.334662in}}%
\pgfpathlineto{\pgfqpoint{5.874907in}{3.340162in}}%
\pgfpathlineto{\pgfqpoint{5.881787in}{3.345871in}}%
\pgfpathlineto{\pgfqpoint{5.888670in}{3.351797in}}%
\pgfpathlineto{\pgfqpoint{5.874780in}{3.345828in}}%
\pgfpathlineto{\pgfqpoint{5.860907in}{3.340006in}}%
\pgfpathlineto{\pgfqpoint{5.847051in}{3.334333in}}%
\pgfpathlineto{\pgfqpoint{5.833210in}{3.328806in}}%
\pgfpathlineto{\pgfqpoint{5.826293in}{3.322188in}}%
\pgfpathlineto{\pgfqpoint{5.819379in}{3.315792in}}%
\pgfpathlineto{\pgfqpoint{5.812466in}{3.309612in}}%
\pgfpathclose%
\pgfusepath{fill}%
\end{pgfscope}%
\begin{pgfscope}%
\pgfpathrectangle{\pgfqpoint{1.254980in}{0.150000in}}{\pgfqpoint{5.490039in}{5.490039in}}%
\pgfusepath{clip}%
\pgfsetbuttcap%
\pgfsetroundjoin%
\definecolor{currentfill}{rgb}{0.282290,0.145912,0.461510}%
\pgfsetfillcolor{currentfill}%
\pgfsetfillopacity{0.700000}%
\pgfsetlinewidth{0.000000pt}%
\definecolor{currentstroke}{rgb}{0.000000,0.000000,0.000000}%
\pgfsetstrokecolor{currentstroke}%
\pgfsetdash{}{0pt}%
\pgfpathmoveto{\pgfqpoint{3.957118in}{2.241280in}}%
\pgfpathlineto{\pgfqpoint{3.970238in}{2.241350in}}%
\pgfpathlineto{\pgfqpoint{3.983366in}{2.241594in}}%
\pgfpathlineto{\pgfqpoint{3.996501in}{2.242012in}}%
\pgfpathlineto{\pgfqpoint{4.009644in}{2.242602in}}%
\pgfpathlineto{\pgfqpoint{4.017265in}{2.252224in}}%
\pgfpathlineto{\pgfqpoint{4.024881in}{2.261827in}}%
\pgfpathlineto{\pgfqpoint{4.032492in}{2.271411in}}%
\pgfpathlineto{\pgfqpoint{4.040098in}{2.280978in}}%
\pgfpathlineto{\pgfqpoint{4.026963in}{2.280421in}}%
\pgfpathlineto{\pgfqpoint{4.013835in}{2.280037in}}%
\pgfpathlineto{\pgfqpoint{4.000716in}{2.279826in}}%
\pgfpathlineto{\pgfqpoint{3.987604in}{2.279789in}}%
\pgfpathlineto{\pgfqpoint{3.979990in}{2.270179in}}%
\pgfpathlineto{\pgfqpoint{3.972371in}{2.260558in}}%
\pgfpathlineto{\pgfqpoint{3.964747in}{2.250925in}}%
\pgfpathlineto{\pgfqpoint{3.957118in}{2.241280in}}%
\pgfpathclose%
\pgfusepath{fill}%
\end{pgfscope}%
\begin{pgfscope}%
\pgfpathrectangle{\pgfqpoint{1.254980in}{0.150000in}}{\pgfqpoint{5.490039in}{5.490039in}}%
\pgfusepath{clip}%
\pgfsetbuttcap%
\pgfsetroundjoin%
\definecolor{currentfill}{rgb}{0.279566,0.067836,0.391917}%
\pgfsetfillcolor{currentfill}%
\pgfsetfillopacity{0.700000}%
\pgfsetlinewidth{0.000000pt}%
\definecolor{currentstroke}{rgb}{0.000000,0.000000,0.000000}%
\pgfsetstrokecolor{currentstroke}%
\pgfsetdash{}{0pt}%
\pgfpathmoveto{\pgfqpoint{3.353831in}{2.111210in}}%
\pgfpathlineto{\pgfqpoint{3.366861in}{2.104962in}}%
\pgfpathlineto{\pgfqpoint{3.379893in}{2.098913in}}%
\pgfpathlineto{\pgfqpoint{3.392927in}{2.093062in}}%
\pgfpathlineto{\pgfqpoint{3.405963in}{2.087408in}}%
\pgfpathlineto{\pgfqpoint{3.413791in}{2.096376in}}%
\pgfpathlineto{\pgfqpoint{3.421612in}{2.105384in}}%
\pgfpathlineto{\pgfqpoint{3.429428in}{2.114431in}}%
\pgfpathlineto{\pgfqpoint{3.437238in}{2.123516in}}%
\pgfpathlineto{\pgfqpoint{3.424215in}{2.129036in}}%
\pgfpathlineto{\pgfqpoint{3.411195in}{2.134752in}}%
\pgfpathlineto{\pgfqpoint{3.398177in}{2.140667in}}%
\pgfpathlineto{\pgfqpoint{3.385161in}{2.146780in}}%
\pgfpathlineto{\pgfqpoint{3.377338in}{2.137819in}}%
\pgfpathlineto{\pgfqpoint{3.369508in}{2.128903in}}%
\pgfpathlineto{\pgfqpoint{3.361673in}{2.120033in}}%
\pgfpathlineto{\pgfqpoint{3.353831in}{2.111210in}}%
\pgfpathclose%
\pgfusepath{fill}%
\end{pgfscope}%
\begin{pgfscope}%
\pgfpathrectangle{\pgfqpoint{1.254980in}{0.150000in}}{\pgfqpoint{5.490039in}{5.490039in}}%
\pgfusepath{clip}%
\pgfsetbuttcap%
\pgfsetroundjoin%
\definecolor{currentfill}{rgb}{0.248629,0.278775,0.534556}%
\pgfsetfillcolor{currentfill}%
\pgfsetfillopacity{0.700000}%
\pgfsetlinewidth{0.000000pt}%
\definecolor{currentstroke}{rgb}{0.000000,0.000000,0.000000}%
\pgfsetstrokecolor{currentstroke}%
\pgfsetdash{}{0pt}%
\pgfpathmoveto{\pgfqpoint{4.454958in}{2.505382in}}%
\pgfpathlineto{\pgfqpoint{4.468253in}{2.508867in}}%
\pgfpathlineto{\pgfqpoint{4.481559in}{2.512516in}}%
\pgfpathlineto{\pgfqpoint{4.494876in}{2.516328in}}%
\pgfpathlineto{\pgfqpoint{4.508204in}{2.520303in}}%
\pgfpathlineto{\pgfqpoint{4.515652in}{2.528667in}}%
\pgfpathlineto{\pgfqpoint{4.523095in}{2.537009in}}%
\pgfpathlineto{\pgfqpoint{4.530532in}{2.545330in}}%
\pgfpathlineto{\pgfqpoint{4.537964in}{2.553633in}}%
\pgfpathlineto{\pgfqpoint{4.524645in}{2.549862in}}%
\pgfpathlineto{\pgfqpoint{4.511337in}{2.546253in}}%
\pgfpathlineto{\pgfqpoint{4.498041in}{2.542807in}}%
\pgfpathlineto{\pgfqpoint{4.484755in}{2.539524in}}%
\pgfpathlineto{\pgfqpoint{4.477314in}{2.531008in}}%
\pgfpathlineto{\pgfqpoint{4.469867in}{2.522480in}}%
\pgfpathlineto{\pgfqpoint{4.462415in}{2.513939in}}%
\pgfpathlineto{\pgfqpoint{4.454958in}{2.505382in}}%
\pgfpathclose%
\pgfusepath{fill}%
\end{pgfscope}%
\begin{pgfscope}%
\pgfpathrectangle{\pgfqpoint{1.254980in}{0.150000in}}{\pgfqpoint{5.490039in}{5.490039in}}%
\pgfusepath{clip}%
\pgfsetbuttcap%
\pgfsetroundjoin%
\definecolor{currentfill}{rgb}{0.283229,0.120777,0.440584}%
\pgfsetfillcolor{currentfill}%
\pgfsetfillopacity{0.700000}%
\pgfsetlinewidth{0.000000pt}%
\definecolor{currentstroke}{rgb}{0.000000,0.000000,0.000000}%
\pgfsetstrokecolor{currentstroke}%
\pgfsetdash{}{0pt}%
\pgfpathmoveto{\pgfqpoint{3.029683in}{2.218191in}}%
\pgfpathlineto{\pgfqpoint{3.042753in}{2.207343in}}%
\pgfpathlineto{\pgfqpoint{3.055821in}{2.196719in}}%
\pgfpathlineto{\pgfqpoint{3.068887in}{2.186320in}}%
\pgfpathlineto{\pgfqpoint{3.081951in}{2.176144in}}%
\pgfpathlineto{\pgfqpoint{3.089915in}{2.183740in}}%
\pgfpathlineto{\pgfqpoint{3.097872in}{2.191421in}}%
\pgfpathlineto{\pgfqpoint{3.105821in}{2.199186in}}%
\pgfpathlineto{\pgfqpoint{3.113763in}{2.207034in}}%
\pgfpathlineto{\pgfqpoint{3.100718in}{2.217017in}}%
\pgfpathlineto{\pgfqpoint{3.087672in}{2.227223in}}%
\pgfpathlineto{\pgfqpoint{3.074624in}{2.237653in}}%
\pgfpathlineto{\pgfqpoint{3.061574in}{2.248308in}}%
\pgfpathlineto{\pgfqpoint{3.053613in}{2.240643in}}%
\pgfpathlineto{\pgfqpoint{3.045644in}{2.233068in}}%
\pgfpathlineto{\pgfqpoint{3.037668in}{2.225583in}}%
\pgfpathlineto{\pgfqpoint{3.029683in}{2.218191in}}%
\pgfpathclose%
\pgfusepath{fill}%
\end{pgfscope}%
\begin{pgfscope}%
\pgfpathrectangle{\pgfqpoint{1.254980in}{0.150000in}}{\pgfqpoint{5.490039in}{5.490039in}}%
\pgfusepath{clip}%
\pgfsetbuttcap%
\pgfsetroundjoin%
\definecolor{currentfill}{rgb}{0.280894,0.078907,0.402329}%
\pgfsetfillcolor{currentfill}%
\pgfsetfillopacity{0.700000}%
\pgfsetlinewidth{0.000000pt}%
\definecolor{currentstroke}{rgb}{0.000000,0.000000,0.000000}%
\pgfsetstrokecolor{currentstroke}%
\pgfsetdash{}{0pt}%
\pgfpathmoveto{\pgfqpoint{3.218081in}{2.135001in}}%
\pgfpathlineto{\pgfqpoint{3.231119in}{2.126956in}}%
\pgfpathlineto{\pgfqpoint{3.244157in}{2.119119in}}%
\pgfpathlineto{\pgfqpoint{3.257196in}{2.111489in}}%
\pgfpathlineto{\pgfqpoint{3.270235in}{2.104065in}}%
\pgfpathlineto{\pgfqpoint{3.278118in}{2.112516in}}%
\pgfpathlineto{\pgfqpoint{3.285994in}{2.121026in}}%
\pgfpathlineto{\pgfqpoint{3.293863in}{2.129593in}}%
\pgfpathlineto{\pgfqpoint{3.301726in}{2.138216in}}%
\pgfpathlineto{\pgfqpoint{3.288702in}{2.145477in}}%
\pgfpathlineto{\pgfqpoint{3.275680in}{2.152943in}}%
\pgfpathlineto{\pgfqpoint{3.262657in}{2.160617in}}%
\pgfpathlineto{\pgfqpoint{3.249636in}{2.168499in}}%
\pgfpathlineto{\pgfqpoint{3.241757in}{2.160029in}}%
\pgfpathlineto{\pgfqpoint{3.233872in}{2.151622in}}%
\pgfpathlineto{\pgfqpoint{3.225980in}{2.143279in}}%
\pgfpathlineto{\pgfqpoint{3.218081in}{2.135001in}}%
\pgfpathclose%
\pgfusepath{fill}%
\end{pgfscope}%
\begin{pgfscope}%
\pgfpathrectangle{\pgfqpoint{1.254980in}{0.150000in}}{\pgfqpoint{5.490039in}{5.490039in}}%
\pgfusepath{clip}%
\pgfsetbuttcap%
\pgfsetroundjoin%
\definecolor{currentfill}{rgb}{0.239346,0.300855,0.540844}%
\pgfsetfillcolor{currentfill}%
\pgfsetfillopacity{0.700000}%
\pgfsetlinewidth{0.000000pt}%
\definecolor{currentstroke}{rgb}{0.000000,0.000000,0.000000}%
\pgfsetstrokecolor{currentstroke}%
\pgfsetdash{}{0pt}%
\pgfpathmoveto{\pgfqpoint{4.537964in}{2.553633in}}%
\pgfpathlineto{\pgfqpoint{4.551295in}{2.557566in}}%
\pgfpathlineto{\pgfqpoint{4.564637in}{2.561661in}}%
\pgfpathlineto{\pgfqpoint{4.577990in}{2.565918in}}%
\pgfpathlineto{\pgfqpoint{4.591356in}{2.570337in}}%
\pgfpathlineto{\pgfqpoint{4.598772in}{2.578403in}}%
\pgfpathlineto{\pgfqpoint{4.606184in}{2.586450in}}%
\pgfpathlineto{\pgfqpoint{4.613589in}{2.594480in}}%
\pgfpathlineto{\pgfqpoint{4.620990in}{2.602496in}}%
\pgfpathlineto{\pgfqpoint{4.607635in}{2.598310in}}%
\pgfpathlineto{\pgfqpoint{4.594291in}{2.594285in}}%
\pgfpathlineto{\pgfqpoint{4.580959in}{2.590422in}}%
\pgfpathlineto{\pgfqpoint{4.567639in}{2.586720in}}%
\pgfpathlineto{\pgfqpoint{4.560228in}{2.578461in}}%
\pgfpathlineto{\pgfqpoint{4.552812in}{2.570196in}}%
\pgfpathlineto{\pgfqpoint{4.545391in}{2.561921in}}%
\pgfpathlineto{\pgfqpoint{4.537964in}{2.553633in}}%
\pgfpathclose%
\pgfusepath{fill}%
\end{pgfscope}%
\begin{pgfscope}%
\pgfpathrectangle{\pgfqpoint{1.254980in}{0.150000in}}{\pgfqpoint{5.490039in}{5.490039in}}%
\pgfusepath{clip}%
\pgfsetbuttcap%
\pgfsetroundjoin%
\definecolor{currentfill}{rgb}{0.283187,0.125848,0.444960}%
\pgfsetfillcolor{currentfill}%
\pgfsetfillopacity{0.700000}%
\pgfsetlinewidth{0.000000pt}%
\definecolor{currentstroke}{rgb}{0.000000,0.000000,0.000000}%
\pgfsetstrokecolor{currentstroke}%
\pgfsetdash{}{0pt}%
\pgfpathmoveto{\pgfqpoint{3.874113in}{2.204008in}}%
\pgfpathlineto{\pgfqpoint{3.887212in}{2.203384in}}%
\pgfpathlineto{\pgfqpoint{3.900319in}{2.202935in}}%
\pgfpathlineto{\pgfqpoint{3.913432in}{2.202662in}}%
\pgfpathlineto{\pgfqpoint{3.926553in}{2.202563in}}%
\pgfpathlineto{\pgfqpoint{3.934202in}{2.212264in}}%
\pgfpathlineto{\pgfqpoint{3.941845in}{2.221950in}}%
\pgfpathlineto{\pgfqpoint{3.949484in}{2.231622in}}%
\pgfpathlineto{\pgfqpoint{3.957118in}{2.241280in}}%
\pgfpathlineto{\pgfqpoint{3.944006in}{2.241384in}}%
\pgfpathlineto{\pgfqpoint{3.930900in}{2.241662in}}%
\pgfpathlineto{\pgfqpoint{3.917802in}{2.242116in}}%
\pgfpathlineto{\pgfqpoint{3.904711in}{2.242746in}}%
\pgfpathlineto{\pgfqpoint{3.897069in}{2.233072in}}%
\pgfpathlineto{\pgfqpoint{3.889422in}{2.223392in}}%
\pgfpathlineto{\pgfqpoint{3.881770in}{2.213704in}}%
\pgfpathlineto{\pgfqpoint{3.874113in}{2.204008in}}%
\pgfpathclose%
\pgfusepath{fill}%
\end{pgfscope}%
\begin{pgfscope}%
\pgfpathrectangle{\pgfqpoint{1.254980in}{0.150000in}}{\pgfqpoint{5.490039in}{5.490039in}}%
\pgfusepath{clip}%
\pgfsetbuttcap%
\pgfsetroundjoin%
\definecolor{currentfill}{rgb}{0.229739,0.322361,0.545706}%
\pgfsetfillcolor{currentfill}%
\pgfsetfillopacity{0.700000}%
\pgfsetlinewidth{0.000000pt}%
\definecolor{currentstroke}{rgb}{0.000000,0.000000,0.000000}%
\pgfsetstrokecolor{currentstroke}%
\pgfsetdash{}{0pt}%
\pgfpathmoveto{\pgfqpoint{4.620990in}{2.602496in}}%
\pgfpathlineto{\pgfqpoint{4.634357in}{2.606843in}}%
\pgfpathlineto{\pgfqpoint{4.647736in}{2.611351in}}%
\pgfpathlineto{\pgfqpoint{4.661127in}{2.616020in}}%
\pgfpathlineto{\pgfqpoint{4.674531in}{2.620848in}}%
\pgfpathlineto{\pgfqpoint{4.681915in}{2.628604in}}%
\pgfpathlineto{\pgfqpoint{4.689294in}{2.636344in}}%
\pgfpathlineto{\pgfqpoint{4.696667in}{2.644073in}}%
\pgfpathlineto{\pgfqpoint{4.704035in}{2.651793in}}%
\pgfpathlineto{\pgfqpoint{4.690642in}{2.647226in}}%
\pgfpathlineto{\pgfqpoint{4.677262in}{2.642818in}}%
\pgfpathlineto{\pgfqpoint{4.663894in}{2.638571in}}%
\pgfpathlineto{\pgfqpoint{4.650538in}{2.634484in}}%
\pgfpathlineto{\pgfqpoint{4.643159in}{2.626493in}}%
\pgfpathlineto{\pgfqpoint{4.635775in}{2.618500in}}%
\pgfpathlineto{\pgfqpoint{4.628385in}{2.610502in}}%
\pgfpathlineto{\pgfqpoint{4.620990in}{2.602496in}}%
\pgfpathclose%
\pgfusepath{fill}%
\end{pgfscope}%
\begin{pgfscope}%
\pgfpathrectangle{\pgfqpoint{1.254980in}{0.150000in}}{\pgfqpoint{5.490039in}{5.490039in}}%
\pgfusepath{clip}%
\pgfsetbuttcap%
\pgfsetroundjoin%
\definecolor{currentfill}{rgb}{0.279566,0.067836,0.391917}%
\pgfsetfillcolor{currentfill}%
\pgfsetfillopacity{0.700000}%
\pgfsetlinewidth{0.000000pt}%
\definecolor{currentstroke}{rgb}{0.000000,0.000000,0.000000}%
\pgfsetstrokecolor{currentstroke}%
\pgfsetdash{}{0pt}%
\pgfpathmoveto{\pgfqpoint{3.489356in}{2.103387in}}%
\pgfpathlineto{\pgfqpoint{3.502393in}{2.098837in}}%
\pgfpathlineto{\pgfqpoint{3.515433in}{2.094478in}}%
\pgfpathlineto{\pgfqpoint{3.528477in}{2.090310in}}%
\pgfpathlineto{\pgfqpoint{3.541525in}{2.086330in}}%
\pgfpathlineto{\pgfqpoint{3.549304in}{2.095686in}}%
\pgfpathlineto{\pgfqpoint{3.557078in}{2.105065in}}%
\pgfpathlineto{\pgfqpoint{3.564846in}{2.114465in}}%
\pgfpathlineto{\pgfqpoint{3.572609in}{2.123887in}}%
\pgfpathlineto{\pgfqpoint{3.559573in}{2.127760in}}%
\pgfpathlineto{\pgfqpoint{3.546541in}{2.131823in}}%
\pgfpathlineto{\pgfqpoint{3.533512in}{2.136075in}}%
\pgfpathlineto{\pgfqpoint{3.520487in}{2.140519in}}%
\pgfpathlineto{\pgfqpoint{3.512713in}{2.131193in}}%
\pgfpathlineto{\pgfqpoint{3.504933in}{2.121895in}}%
\pgfpathlineto{\pgfqpoint{3.497147in}{2.112626in}}%
\pgfpathlineto{\pgfqpoint{3.489356in}{2.103387in}}%
\pgfpathclose%
\pgfusepath{fill}%
\end{pgfscope}%
\begin{pgfscope}%
\pgfpathrectangle{\pgfqpoint{1.254980in}{0.150000in}}{\pgfqpoint{5.490039in}{5.490039in}}%
\pgfusepath{clip}%
\pgfsetbuttcap%
\pgfsetroundjoin%
\definecolor{currentfill}{rgb}{0.283091,0.110553,0.431554}%
\pgfsetfillcolor{currentfill}%
\pgfsetfillopacity{0.700000}%
\pgfsetlinewidth{0.000000pt}%
\definecolor{currentstroke}{rgb}{0.000000,0.000000,0.000000}%
\pgfsetstrokecolor{currentstroke}%
\pgfsetdash{}{0pt}%
\pgfpathmoveto{\pgfqpoint{3.791069in}{2.169504in}}%
\pgfpathlineto{\pgfqpoint{3.804151in}{2.168147in}}%
\pgfpathlineto{\pgfqpoint{3.817239in}{2.166968in}}%
\pgfpathlineto{\pgfqpoint{3.830333in}{2.165967in}}%
\pgfpathlineto{\pgfqpoint{3.843435in}{2.165143in}}%
\pgfpathlineto{\pgfqpoint{3.851112in}{2.174872in}}%
\pgfpathlineto{\pgfqpoint{3.858784in}{2.184593in}}%
\pgfpathlineto{\pgfqpoint{3.866451in}{2.194305in}}%
\pgfpathlineto{\pgfqpoint{3.874113in}{2.204008in}}%
\pgfpathlineto{\pgfqpoint{3.861021in}{2.204810in}}%
\pgfpathlineto{\pgfqpoint{3.847935in}{2.205788in}}%
\pgfpathlineto{\pgfqpoint{3.834855in}{2.206945in}}%
\pgfpathlineto{\pgfqpoint{3.821782in}{2.208280in}}%
\pgfpathlineto{\pgfqpoint{3.814111in}{2.198588in}}%
\pgfpathlineto{\pgfqpoint{3.806436in}{2.188895in}}%
\pgfpathlineto{\pgfqpoint{3.798755in}{2.179201in}}%
\pgfpathlineto{\pgfqpoint{3.791069in}{2.169504in}}%
\pgfpathclose%
\pgfusepath{fill}%
\end{pgfscope}%
\begin{pgfscope}%
\pgfpathrectangle{\pgfqpoint{1.254980in}{0.150000in}}{\pgfqpoint{5.490039in}{5.490039in}}%
\pgfusepath{clip}%
\pgfsetbuttcap%
\pgfsetroundjoin%
\definecolor{currentfill}{rgb}{0.220057,0.343307,0.549413}%
\pgfsetfillcolor{currentfill}%
\pgfsetfillopacity{0.700000}%
\pgfsetlinewidth{0.000000pt}%
\definecolor{currentstroke}{rgb}{0.000000,0.000000,0.000000}%
\pgfsetstrokecolor{currentstroke}%
\pgfsetdash{}{0pt}%
\pgfpathmoveto{\pgfqpoint{4.704035in}{2.651793in}}%
\pgfpathlineto{\pgfqpoint{4.717440in}{2.656521in}}%
\pgfpathlineto{\pgfqpoint{4.730857in}{2.661408in}}%
\pgfpathlineto{\pgfqpoint{4.744286in}{2.666454in}}%
\pgfpathlineto{\pgfqpoint{4.757729in}{2.671660in}}%
\pgfpathlineto{\pgfqpoint{4.765080in}{2.679096in}}%
\pgfpathlineto{\pgfqpoint{4.772425in}{2.686523in}}%
\pgfpathlineto{\pgfqpoint{4.779765in}{2.693945in}}%
\pgfpathlineto{\pgfqpoint{4.787099in}{2.701364in}}%
\pgfpathlineto{\pgfqpoint{4.773669in}{2.696449in}}%
\pgfpathlineto{\pgfqpoint{4.760251in}{2.691692in}}%
\pgfpathlineto{\pgfqpoint{4.746846in}{2.687094in}}%
\pgfpathlineto{\pgfqpoint{4.733453in}{2.682656in}}%
\pgfpathlineto{\pgfqpoint{4.726107in}{2.674936in}}%
\pgfpathlineto{\pgfqpoint{4.718755in}{2.667222in}}%
\pgfpathlineto{\pgfqpoint{4.711397in}{2.659509in}}%
\pgfpathlineto{\pgfqpoint{4.704035in}{2.651793in}}%
\pgfpathclose%
\pgfusepath{fill}%
\end{pgfscope}%
\begin{pgfscope}%
\pgfpathrectangle{\pgfqpoint{1.254980in}{0.150000in}}{\pgfqpoint{5.490039in}{5.490039in}}%
\pgfusepath{clip}%
\pgfsetbuttcap%
\pgfsetroundjoin%
\definecolor{currentfill}{rgb}{0.208623,0.367752,0.552675}%
\pgfsetfillcolor{currentfill}%
\pgfsetfillopacity{0.700000}%
\pgfsetlinewidth{0.000000pt}%
\definecolor{currentstroke}{rgb}{0.000000,0.000000,0.000000}%
\pgfsetstrokecolor{currentstroke}%
\pgfsetdash{}{0pt}%
\pgfpathmoveto{\pgfqpoint{4.787099in}{2.701364in}}%
\pgfpathlineto{\pgfqpoint{4.800542in}{2.706439in}}%
\pgfpathlineto{\pgfqpoint{4.813998in}{2.711671in}}%
\pgfpathlineto{\pgfqpoint{4.827467in}{2.717063in}}%
\pgfpathlineto{\pgfqpoint{4.840949in}{2.722612in}}%
\pgfpathlineto{\pgfqpoint{4.848266in}{2.729725in}}%
\pgfpathlineto{\pgfqpoint{4.855577in}{2.736837in}}%
\pgfpathlineto{\pgfqpoint{4.862882in}{2.743950in}}%
\pgfpathlineto{\pgfqpoint{4.870182in}{2.751070in}}%
\pgfpathlineto{\pgfqpoint{4.856713in}{2.745839in}}%
\pgfpathlineto{\pgfqpoint{4.843257in}{2.740766in}}%
\pgfpathlineto{\pgfqpoint{4.829815in}{2.735852in}}%
\pgfpathlineto{\pgfqpoint{4.816385in}{2.731095in}}%
\pgfpathlineto{\pgfqpoint{4.809071in}{2.723647in}}%
\pgfpathlineto{\pgfqpoint{4.801752in}{2.716212in}}%
\pgfpathlineto{\pgfqpoint{4.794428in}{2.708786in}}%
\pgfpathlineto{\pgfqpoint{4.787099in}{2.701364in}}%
\pgfpathclose%
\pgfusepath{fill}%
\end{pgfscope}%
\begin{pgfscope}%
\pgfpathrectangle{\pgfqpoint{1.254980in}{0.150000in}}{\pgfqpoint{5.490039in}{5.490039in}}%
\pgfusepath{clip}%
\pgfsetbuttcap%
\pgfsetroundjoin%
\definecolor{currentfill}{rgb}{0.282910,0.105393,0.426902}%
\pgfsetfillcolor{currentfill}%
\pgfsetfillopacity{0.700000}%
\pgfsetlinewidth{0.000000pt}%
\definecolor{currentstroke}{rgb}{0.000000,0.000000,0.000000}%
\pgfsetstrokecolor{currentstroke}%
\pgfsetdash{}{0pt}%
\pgfpathmoveto{\pgfqpoint{3.081951in}{2.176144in}}%
\pgfpathlineto{\pgfqpoint{3.095013in}{2.166189in}}%
\pgfpathlineto{\pgfqpoint{3.108074in}{2.156454in}}%
\pgfpathlineto{\pgfqpoint{3.121133in}{2.146937in}}%
\pgfpathlineto{\pgfqpoint{3.134191in}{2.137637in}}%
\pgfpathlineto{\pgfqpoint{3.142136in}{2.145435in}}%
\pgfpathlineto{\pgfqpoint{3.150074in}{2.153312in}}%
\pgfpathlineto{\pgfqpoint{3.158005in}{2.161266in}}%
\pgfpathlineto{\pgfqpoint{3.165928in}{2.169295in}}%
\pgfpathlineto{\pgfqpoint{3.152888in}{2.178404in}}%
\pgfpathlineto{\pgfqpoint{3.139847in}{2.187728in}}%
\pgfpathlineto{\pgfqpoint{3.126806in}{2.197271in}}%
\pgfpathlineto{\pgfqpoint{3.113763in}{2.207034in}}%
\pgfpathlineto{\pgfqpoint{3.105821in}{2.199186in}}%
\pgfpathlineto{\pgfqpoint{3.097872in}{2.191421in}}%
\pgfpathlineto{\pgfqpoint{3.089915in}{2.183740in}}%
\pgfpathlineto{\pgfqpoint{3.081951in}{2.176144in}}%
\pgfpathclose%
\pgfusepath{fill}%
\end{pgfscope}%
\begin{pgfscope}%
\pgfpathrectangle{\pgfqpoint{1.254980in}{0.150000in}}{\pgfqpoint{5.490039in}{5.490039in}}%
\pgfusepath{clip}%
\pgfsetbuttcap%
\pgfsetroundjoin%
\definecolor{currentfill}{rgb}{0.199430,0.387607,0.554642}%
\pgfsetfillcolor{currentfill}%
\pgfsetfillopacity{0.700000}%
\pgfsetlinewidth{0.000000pt}%
\definecolor{currentstroke}{rgb}{0.000000,0.000000,0.000000}%
\pgfsetstrokecolor{currentstroke}%
\pgfsetdash{}{0pt}%
\pgfpathmoveto{\pgfqpoint{4.870182in}{2.751070in}}%
\pgfpathlineto{\pgfqpoint{4.883664in}{2.756458in}}%
\pgfpathlineto{\pgfqpoint{4.897160in}{2.762003in}}%
\pgfpathlineto{\pgfqpoint{4.910668in}{2.767707in}}%
\pgfpathlineto{\pgfqpoint{4.924191in}{2.773567in}}%
\pgfpathlineto{\pgfqpoint{4.931471in}{2.780359in}}%
\pgfpathlineto{\pgfqpoint{4.938747in}{2.787157in}}%
\pgfpathlineto{\pgfqpoint{4.946017in}{2.793966in}}%
\pgfpathlineto{\pgfqpoint{4.953282in}{2.800790in}}%
\pgfpathlineto{\pgfqpoint{4.939774in}{2.795277in}}%
\pgfpathlineto{\pgfqpoint{4.926280in}{2.789922in}}%
\pgfpathlineto{\pgfqpoint{4.912799in}{2.784723in}}%
\pgfpathlineto{\pgfqpoint{4.899331in}{2.779682in}}%
\pgfpathlineto{\pgfqpoint{4.892051in}{2.772501in}}%
\pgfpathlineto{\pgfqpoint{4.884767in}{2.765341in}}%
\pgfpathlineto{\pgfqpoint{4.877477in}{2.758198in}}%
\pgfpathlineto{\pgfqpoint{4.870182in}{2.751070in}}%
\pgfpathclose%
\pgfusepath{fill}%
\end{pgfscope}%
\begin{pgfscope}%
\pgfpathrectangle{\pgfqpoint{1.254980in}{0.150000in}}{\pgfqpoint{5.490039in}{5.490039in}}%
\pgfusepath{clip}%
\pgfsetbuttcap%
\pgfsetroundjoin%
\definecolor{currentfill}{rgb}{0.282327,0.094955,0.417331}%
\pgfsetfillcolor{currentfill}%
\pgfsetfillopacity{0.700000}%
\pgfsetlinewidth{0.000000pt}%
\definecolor{currentstroke}{rgb}{0.000000,0.000000,0.000000}%
\pgfsetstrokecolor{currentstroke}%
\pgfsetdash{}{0pt}%
\pgfpathmoveto{\pgfqpoint{3.707969in}{2.138130in}}%
\pgfpathlineto{\pgfqpoint{3.721037in}{2.136001in}}%
\pgfpathlineto{\pgfqpoint{3.734111in}{2.134054in}}%
\pgfpathlineto{\pgfqpoint{3.747190in}{2.132287in}}%
\pgfpathlineto{\pgfqpoint{3.760275in}{2.130700in}}%
\pgfpathlineto{\pgfqpoint{3.767981in}{2.140404in}}%
\pgfpathlineto{\pgfqpoint{3.775682in}{2.150106in}}%
\pgfpathlineto{\pgfqpoint{3.783378in}{2.159806in}}%
\pgfpathlineto{\pgfqpoint{3.791069in}{2.169504in}}%
\pgfpathlineto{\pgfqpoint{3.777993in}{2.171041in}}%
\pgfpathlineto{\pgfqpoint{3.764923in}{2.172758in}}%
\pgfpathlineto{\pgfqpoint{3.751859in}{2.174655in}}%
\pgfpathlineto{\pgfqpoint{3.738801in}{2.176733in}}%
\pgfpathlineto{\pgfqpoint{3.731101in}{2.167075in}}%
\pgfpathlineto{\pgfqpoint{3.723395in}{2.157422in}}%
\pgfpathlineto{\pgfqpoint{3.715685in}{2.147773in}}%
\pgfpathlineto{\pgfqpoint{3.707969in}{2.138130in}}%
\pgfpathclose%
\pgfusepath{fill}%
\end{pgfscope}%
\begin{pgfscope}%
\pgfpathrectangle{\pgfqpoint{1.254980in}{0.150000in}}{\pgfqpoint{5.490039in}{5.490039in}}%
\pgfusepath{clip}%
\pgfsetbuttcap%
\pgfsetroundjoin%
\definecolor{currentfill}{rgb}{0.190631,0.407061,0.556089}%
\pgfsetfillcolor{currentfill}%
\pgfsetfillopacity{0.700000}%
\pgfsetlinewidth{0.000000pt}%
\definecolor{currentstroke}{rgb}{0.000000,0.000000,0.000000}%
\pgfsetstrokecolor{currentstroke}%
\pgfsetdash{}{0pt}%
\pgfpathmoveto{\pgfqpoint{4.953282in}{2.800790in}}%
\pgfpathlineto{\pgfqpoint{4.966803in}{2.806459in}}%
\pgfpathlineto{\pgfqpoint{4.980339in}{2.812284in}}%
\pgfpathlineto{\pgfqpoint{4.993887in}{2.818267in}}%
\pgfpathlineto{\pgfqpoint{5.007450in}{2.824406in}}%
\pgfpathlineto{\pgfqpoint{5.014695in}{2.830882in}}%
\pgfpathlineto{\pgfqpoint{5.021934in}{2.837374in}}%
\pgfpathlineto{\pgfqpoint{5.029168in}{2.843887in}}%
\pgfpathlineto{\pgfqpoint{5.036397in}{2.850425in}}%
\pgfpathlineto{\pgfqpoint{5.022850in}{2.844663in}}%
\pgfpathlineto{\pgfqpoint{5.009317in}{2.839058in}}%
\pgfpathlineto{\pgfqpoint{4.995798in}{2.833608in}}%
\pgfpathlineto{\pgfqpoint{4.982292in}{2.828315in}}%
\pgfpathlineto{\pgfqpoint{4.975047in}{2.821390in}}%
\pgfpathlineto{\pgfqpoint{4.967797in}{2.814498in}}%
\pgfpathlineto{\pgfqpoint{4.960542in}{2.807632in}}%
\pgfpathlineto{\pgfqpoint{4.953282in}{2.800790in}}%
\pgfpathclose%
\pgfusepath{fill}%
\end{pgfscope}%
\begin{pgfscope}%
\pgfpathrectangle{\pgfqpoint{1.254980in}{0.150000in}}{\pgfqpoint{5.490039in}{5.490039in}}%
\pgfusepath{clip}%
\pgfsetbuttcap%
\pgfsetroundjoin%
\definecolor{currentfill}{rgb}{0.262138,0.242286,0.520837}%
\pgfsetfillcolor{currentfill}%
\pgfsetfillopacity{0.700000}%
\pgfsetlinewidth{0.000000pt}%
\definecolor{currentstroke}{rgb}{0.000000,0.000000,0.000000}%
\pgfsetstrokecolor{currentstroke}%
\pgfsetdash{}{0pt}%
\pgfpathmoveto{\pgfqpoint{2.734951in}{2.461187in}}%
\pgfpathlineto{\pgfqpoint{2.748141in}{2.445214in}}%
\pgfpathlineto{\pgfqpoint{2.761324in}{2.429505in}}%
\pgfpathlineto{\pgfqpoint{2.774500in}{2.414058in}}%
\pgfpathlineto{\pgfqpoint{2.787669in}{2.398872in}}%
\pgfpathlineto{\pgfqpoint{2.795783in}{2.404936in}}%
\pgfpathlineto{\pgfqpoint{2.803887in}{2.411129in}}%
\pgfpathlineto{\pgfqpoint{2.811981in}{2.417448in}}%
\pgfpathlineto{\pgfqpoint{2.820065in}{2.423891in}}%
\pgfpathlineto{\pgfqpoint{2.806922in}{2.438851in}}%
\pgfpathlineto{\pgfqpoint{2.793772in}{2.454069in}}%
\pgfpathlineto{\pgfqpoint{2.780615in}{2.469550in}}%
\pgfpathlineto{\pgfqpoint{2.767452in}{2.485295in}}%
\pgfpathlineto{\pgfqpoint{2.759341in}{2.479069in}}%
\pgfpathlineto{\pgfqpoint{2.751221in}{2.472974in}}%
\pgfpathlineto{\pgfqpoint{2.743091in}{2.467013in}}%
\pgfpathlineto{\pgfqpoint{2.734951in}{2.461187in}}%
\pgfpathclose%
\pgfusepath{fill}%
\end{pgfscope}%
\begin{pgfscope}%
\pgfpathrectangle{\pgfqpoint{1.254980in}{0.150000in}}{\pgfqpoint{5.490039in}{5.490039in}}%
\pgfusepath{clip}%
\pgfsetbuttcap%
\pgfsetroundjoin%
\definecolor{currentfill}{rgb}{0.270595,0.214069,0.507052}%
\pgfsetfillcolor{currentfill}%
\pgfsetfillopacity{0.700000}%
\pgfsetlinewidth{0.000000pt}%
\definecolor{currentstroke}{rgb}{0.000000,0.000000,0.000000}%
\pgfsetstrokecolor{currentstroke}%
\pgfsetdash{}{0pt}%
\pgfpathmoveto{\pgfqpoint{2.787669in}{2.398872in}}%
\pgfpathlineto{\pgfqpoint{2.800832in}{2.383943in}}%
\pgfpathlineto{\pgfqpoint{2.813989in}{2.369270in}}%
\pgfpathlineto{\pgfqpoint{2.827140in}{2.354851in}}%
\pgfpathlineto{\pgfqpoint{2.840285in}{2.340683in}}%
\pgfpathlineto{\pgfqpoint{2.848373in}{2.346985in}}%
\pgfpathlineto{\pgfqpoint{2.856451in}{2.353407in}}%
\pgfpathlineto{\pgfqpoint{2.864520in}{2.359949in}}%
\pgfpathlineto{\pgfqpoint{2.872580in}{2.366608in}}%
\pgfpathlineto{\pgfqpoint{2.859460in}{2.380550in}}%
\pgfpathlineto{\pgfqpoint{2.846334in}{2.394743in}}%
\pgfpathlineto{\pgfqpoint{2.833203in}{2.409190in}}%
\pgfpathlineto{\pgfqpoint{2.820065in}{2.423891in}}%
\pgfpathlineto{\pgfqpoint{2.811981in}{2.417448in}}%
\pgfpathlineto{\pgfqpoint{2.803887in}{2.411129in}}%
\pgfpathlineto{\pgfqpoint{2.795783in}{2.404936in}}%
\pgfpathlineto{\pgfqpoint{2.787669in}{2.398872in}}%
\pgfpathclose%
\pgfusepath{fill}%
\end{pgfscope}%
\begin{pgfscope}%
\pgfpathrectangle{\pgfqpoint{1.254980in}{0.150000in}}{\pgfqpoint{5.490039in}{5.490039in}}%
\pgfusepath{clip}%
\pgfsetbuttcap%
\pgfsetroundjoin%
\definecolor{currentfill}{rgb}{0.182256,0.426184,0.557120}%
\pgfsetfillcolor{currentfill}%
\pgfsetfillopacity{0.700000}%
\pgfsetlinewidth{0.000000pt}%
\definecolor{currentstroke}{rgb}{0.000000,0.000000,0.000000}%
\pgfsetstrokecolor{currentstroke}%
\pgfsetdash{}{0pt}%
\pgfpathmoveto{\pgfqpoint{5.036397in}{2.850425in}}%
\pgfpathlineto{\pgfqpoint{5.049958in}{2.856342in}}%
\pgfpathlineto{\pgfqpoint{5.063533in}{2.862415in}}%
\pgfpathlineto{\pgfqpoint{5.077122in}{2.868644in}}%
\pgfpathlineto{\pgfqpoint{5.090726in}{2.875028in}}%
\pgfpathlineto{\pgfqpoint{5.097933in}{2.881200in}}%
\pgfpathlineto{\pgfqpoint{5.105136in}{2.887399in}}%
\pgfpathlineto{\pgfqpoint{5.112333in}{2.893629in}}%
\pgfpathlineto{\pgfqpoint{5.119526in}{2.899896in}}%
\pgfpathlineto{\pgfqpoint{5.105939in}{2.893918in}}%
\pgfpathlineto{\pgfqpoint{5.092367in}{2.888095in}}%
\pgfpathlineto{\pgfqpoint{5.078809in}{2.882427in}}%
\pgfpathlineto{\pgfqpoint{5.065265in}{2.876914in}}%
\pgfpathlineto{\pgfqpoint{5.058055in}{2.870232in}}%
\pgfpathlineto{\pgfqpoint{5.050841in}{2.863593in}}%
\pgfpathlineto{\pgfqpoint{5.043621in}{2.856992in}}%
\pgfpathlineto{\pgfqpoint{5.036397in}{2.850425in}}%
\pgfpathclose%
\pgfusepath{fill}%
\end{pgfscope}%
\begin{pgfscope}%
\pgfpathrectangle{\pgfqpoint{1.254980in}{0.150000in}}{\pgfqpoint{5.490039in}{5.490039in}}%
\pgfusepath{clip}%
\pgfsetbuttcap%
\pgfsetroundjoin%
\definecolor{currentfill}{rgb}{0.280267,0.073417,0.397163}%
\pgfsetfillcolor{currentfill}%
\pgfsetfillopacity{0.700000}%
\pgfsetlinewidth{0.000000pt}%
\definecolor{currentstroke}{rgb}{0.000000,0.000000,0.000000}%
\pgfsetstrokecolor{currentstroke}%
\pgfsetdash{}{0pt}%
\pgfpathmoveto{\pgfqpoint{3.270235in}{2.104065in}}%
\pgfpathlineto{\pgfqpoint{3.283275in}{2.096847in}}%
\pgfpathlineto{\pgfqpoint{3.296316in}{2.089832in}}%
\pgfpathlineto{\pgfqpoint{3.309358in}{2.083020in}}%
\pgfpathlineto{\pgfqpoint{3.322402in}{2.076409in}}%
\pgfpathlineto{\pgfqpoint{3.330269in}{2.085033in}}%
\pgfpathlineto{\pgfqpoint{3.338129in}{2.093708in}}%
\pgfpathlineto{\pgfqpoint{3.345983in}{2.102435in}}%
\pgfpathlineto{\pgfqpoint{3.353831in}{2.111210in}}%
\pgfpathlineto{\pgfqpoint{3.340803in}{2.117658in}}%
\pgfpathlineto{\pgfqpoint{3.327776in}{2.124308in}}%
\pgfpathlineto{\pgfqpoint{3.314750in}{2.131160in}}%
\pgfpathlineto{\pgfqpoint{3.301726in}{2.138216in}}%
\pgfpathlineto{\pgfqpoint{3.293863in}{2.129593in}}%
\pgfpathlineto{\pgfqpoint{3.285994in}{2.121026in}}%
\pgfpathlineto{\pgfqpoint{3.278118in}{2.112516in}}%
\pgfpathlineto{\pgfqpoint{3.270235in}{2.104065in}}%
\pgfpathclose%
\pgfusepath{fill}%
\end{pgfscope}%
\begin{pgfscope}%
\pgfpathrectangle{\pgfqpoint{1.254980in}{0.150000in}}{\pgfqpoint{5.490039in}{5.490039in}}%
\pgfusepath{clip}%
\pgfsetbuttcap%
\pgfsetroundjoin%
\definecolor{currentfill}{rgb}{0.250425,0.274290,0.533103}%
\pgfsetfillcolor{currentfill}%
\pgfsetfillopacity{0.700000}%
\pgfsetlinewidth{0.000000pt}%
\definecolor{currentstroke}{rgb}{0.000000,0.000000,0.000000}%
\pgfsetstrokecolor{currentstroke}%
\pgfsetdash{}{0pt}%
\pgfpathmoveto{\pgfqpoint{2.682112in}{2.527775in}}%
\pgfpathlineto{\pgfqpoint{2.695334in}{2.510719in}}%
\pgfpathlineto{\pgfqpoint{2.708548in}{2.493938in}}%
\pgfpathlineto{\pgfqpoint{2.721753in}{2.477428in}}%
\pgfpathlineto{\pgfqpoint{2.734951in}{2.461187in}}%
\pgfpathlineto{\pgfqpoint{2.743091in}{2.467013in}}%
\pgfpathlineto{\pgfqpoint{2.751221in}{2.472974in}}%
\pgfpathlineto{\pgfqpoint{2.759341in}{2.479069in}}%
\pgfpathlineto{\pgfqpoint{2.767452in}{2.485295in}}%
\pgfpathlineto{\pgfqpoint{2.754281in}{2.501307in}}%
\pgfpathlineto{\pgfqpoint{2.741103in}{2.517588in}}%
\pgfpathlineto{\pgfqpoint{2.727917in}{2.534139in}}%
\pgfpathlineto{\pgfqpoint{2.714723in}{2.550965in}}%
\pgfpathlineto{\pgfqpoint{2.706586in}{2.544957in}}%
\pgfpathlineto{\pgfqpoint{2.698439in}{2.539088in}}%
\pgfpathlineto{\pgfqpoint{2.690281in}{2.533360in}}%
\pgfpathlineto{\pgfqpoint{2.682112in}{2.527775in}}%
\pgfpathclose%
\pgfusepath{fill}%
\end{pgfscope}%
\begin{pgfscope}%
\pgfpathrectangle{\pgfqpoint{1.254980in}{0.150000in}}{\pgfqpoint{5.490039in}{5.490039in}}%
\pgfusepath{clip}%
\pgfsetbuttcap%
\pgfsetroundjoin%
\definecolor{currentfill}{rgb}{0.278791,0.062145,0.386592}%
\pgfsetfillcolor{currentfill}%
\pgfsetfillopacity{0.700000}%
\pgfsetlinewidth{0.000000pt}%
\definecolor{currentstroke}{rgb}{0.000000,0.000000,0.000000}%
\pgfsetstrokecolor{currentstroke}%
\pgfsetdash{}{0pt}%
\pgfpathmoveto{\pgfqpoint{3.405963in}{2.087408in}}%
\pgfpathlineto{\pgfqpoint{3.419002in}{2.081950in}}%
\pgfpathlineto{\pgfqpoint{3.432043in}{2.076687in}}%
\pgfpathlineto{\pgfqpoint{3.445086in}{2.071618in}}%
\pgfpathlineto{\pgfqpoint{3.458133in}{2.066742in}}%
\pgfpathlineto{\pgfqpoint{3.465947in}{2.075854in}}%
\pgfpathlineto{\pgfqpoint{3.473756in}{2.085000in}}%
\pgfpathlineto{\pgfqpoint{3.481559in}{2.094178in}}%
\pgfpathlineto{\pgfqpoint{3.489356in}{2.103387in}}%
\pgfpathlineto{\pgfqpoint{3.476322in}{2.108129in}}%
\pgfpathlineto{\pgfqpoint{3.463291in}{2.113064in}}%
\pgfpathlineto{\pgfqpoint{3.450263in}{2.118193in}}%
\pgfpathlineto{\pgfqpoint{3.437238in}{2.123516in}}%
\pgfpathlineto{\pgfqpoint{3.429428in}{2.114431in}}%
\pgfpathlineto{\pgfqpoint{3.421612in}{2.105384in}}%
\pgfpathlineto{\pgfqpoint{3.413791in}{2.096376in}}%
\pgfpathlineto{\pgfqpoint{3.405963in}{2.087408in}}%
\pgfpathclose%
\pgfusepath{fill}%
\end{pgfscope}%
\begin{pgfscope}%
\pgfpathrectangle{\pgfqpoint{1.254980in}{0.150000in}}{\pgfqpoint{5.490039in}{5.490039in}}%
\pgfusepath{clip}%
\pgfsetbuttcap%
\pgfsetroundjoin%
\definecolor{currentfill}{rgb}{0.277134,0.185228,0.489898}%
\pgfsetfillcolor{currentfill}%
\pgfsetfillopacity{0.700000}%
\pgfsetlinewidth{0.000000pt}%
\definecolor{currentstroke}{rgb}{0.000000,0.000000,0.000000}%
\pgfsetstrokecolor{currentstroke}%
\pgfsetdash{}{0pt}%
\pgfpathmoveto{\pgfqpoint{2.840285in}{2.340683in}}%
\pgfpathlineto{\pgfqpoint{2.853424in}{2.326765in}}%
\pgfpathlineto{\pgfqpoint{2.866559in}{2.313095in}}%
\pgfpathlineto{\pgfqpoint{2.879688in}{2.299669in}}%
\pgfpathlineto{\pgfqpoint{2.892813in}{2.286488in}}%
\pgfpathlineto{\pgfqpoint{2.900876in}{2.293025in}}%
\pgfpathlineto{\pgfqpoint{2.908930in}{2.299676in}}%
\pgfpathlineto{\pgfqpoint{2.916975in}{2.306439in}}%
\pgfpathlineto{\pgfqpoint{2.925012in}{2.313312in}}%
\pgfpathlineto{\pgfqpoint{2.911911in}{2.326269in}}%
\pgfpathlineto{\pgfqpoint{2.898805in}{2.339469in}}%
\pgfpathlineto{\pgfqpoint{2.885695in}{2.352915in}}%
\pgfpathlineto{\pgfqpoint{2.872580in}{2.366608in}}%
\pgfpathlineto{\pgfqpoint{2.864520in}{2.359949in}}%
\pgfpathlineto{\pgfqpoint{2.856451in}{2.353407in}}%
\pgfpathlineto{\pgfqpoint{2.848373in}{2.346985in}}%
\pgfpathlineto{\pgfqpoint{2.840285in}{2.340683in}}%
\pgfpathclose%
\pgfusepath{fill}%
\end{pgfscope}%
\begin{pgfscope}%
\pgfpathrectangle{\pgfqpoint{1.254980in}{0.150000in}}{\pgfqpoint{5.490039in}{5.490039in}}%
\pgfusepath{clip}%
\pgfsetbuttcap%
\pgfsetroundjoin%
\definecolor{currentfill}{rgb}{0.172719,0.448791,0.557885}%
\pgfsetfillcolor{currentfill}%
\pgfsetfillopacity{0.700000}%
\pgfsetlinewidth{0.000000pt}%
\definecolor{currentstroke}{rgb}{0.000000,0.000000,0.000000}%
\pgfsetstrokecolor{currentstroke}%
\pgfsetdash{}{0pt}%
\pgfpathmoveto{\pgfqpoint{5.119526in}{2.899896in}}%
\pgfpathlineto{\pgfqpoint{5.133126in}{2.906029in}}%
\pgfpathlineto{\pgfqpoint{5.146741in}{2.912316in}}%
\pgfpathlineto{\pgfqpoint{5.160370in}{2.918759in}}%
\pgfpathlineto{\pgfqpoint{5.174015in}{2.925357in}}%
\pgfpathlineto{\pgfqpoint{5.181184in}{2.931240in}}%
\pgfpathlineto{\pgfqpoint{5.188349in}{2.937162in}}%
\pgfpathlineto{\pgfqpoint{5.195509in}{2.943128in}}%
\pgfpathlineto{\pgfqpoint{5.202665in}{2.949143in}}%
\pgfpathlineto{\pgfqpoint{5.189040in}{2.942981in}}%
\pgfpathlineto{\pgfqpoint{5.175429in}{2.936973in}}%
\pgfpathlineto{\pgfqpoint{5.161832in}{2.931119in}}%
\pgfpathlineto{\pgfqpoint{5.148250in}{2.925420in}}%
\pgfpathlineto{\pgfqpoint{5.141076in}{2.918960in}}%
\pgfpathlineto{\pgfqpoint{5.133897in}{2.912556in}}%
\pgfpathlineto{\pgfqpoint{5.126713in}{2.906203in}}%
\pgfpathlineto{\pgfqpoint{5.119526in}{2.899896in}}%
\pgfpathclose%
\pgfusepath{fill}%
\end{pgfscope}%
\begin{pgfscope}%
\pgfpathrectangle{\pgfqpoint{1.254980in}{0.150000in}}{\pgfqpoint{5.490039in}{5.490039in}}%
\pgfusepath{clip}%
\pgfsetbuttcap%
\pgfsetroundjoin%
\definecolor{currentfill}{rgb}{0.280894,0.078907,0.402329}%
\pgfsetfillcolor{currentfill}%
\pgfsetfillopacity{0.700000}%
\pgfsetlinewidth{0.000000pt}%
\definecolor{currentstroke}{rgb}{0.000000,0.000000,0.000000}%
\pgfsetstrokecolor{currentstroke}%
\pgfsetdash{}{0pt}%
\pgfpathmoveto{\pgfqpoint{3.624795in}{2.110270in}}%
\pgfpathlineto{\pgfqpoint{3.637853in}{2.107330in}}%
\pgfpathlineto{\pgfqpoint{3.650915in}{2.104575in}}%
\pgfpathlineto{\pgfqpoint{3.663982in}{2.102003in}}%
\pgfpathlineto{\pgfqpoint{3.677055in}{2.099614in}}%
\pgfpathlineto{\pgfqpoint{3.684791in}{2.109234in}}%
\pgfpathlineto{\pgfqpoint{3.692522in}{2.118860in}}%
\pgfpathlineto{\pgfqpoint{3.700248in}{2.128492in}}%
\pgfpathlineto{\pgfqpoint{3.707969in}{2.138130in}}%
\pgfpathlineto{\pgfqpoint{3.694907in}{2.140441in}}%
\pgfpathlineto{\pgfqpoint{3.681849in}{2.142935in}}%
\pgfpathlineto{\pgfqpoint{3.668797in}{2.145612in}}%
\pgfpathlineto{\pgfqpoint{3.655750in}{2.148473in}}%
\pgfpathlineto{\pgfqpoint{3.648019in}{2.138903in}}%
\pgfpathlineto{\pgfqpoint{3.640283in}{2.129346in}}%
\pgfpathlineto{\pgfqpoint{3.632542in}{2.119801in}}%
\pgfpathlineto{\pgfqpoint{3.624795in}{2.110270in}}%
\pgfpathclose%
\pgfusepath{fill}%
\end{pgfscope}%
\begin{pgfscope}%
\pgfpathrectangle{\pgfqpoint{1.254980in}{0.150000in}}{\pgfqpoint{5.490039in}{5.490039in}}%
\pgfusepath{clip}%
\pgfsetbuttcap%
\pgfsetroundjoin%
\definecolor{currentfill}{rgb}{0.237441,0.305202,0.541921}%
\pgfsetfillcolor{currentfill}%
\pgfsetfillopacity{0.700000}%
\pgfsetlinewidth{0.000000pt}%
\definecolor{currentstroke}{rgb}{0.000000,0.000000,0.000000}%
\pgfsetstrokecolor{currentstroke}%
\pgfsetdash{}{0pt}%
\pgfpathmoveto{\pgfqpoint{2.629137in}{2.598790in}}%
\pgfpathlineto{\pgfqpoint{2.642395in}{2.580612in}}%
\pgfpathlineto{\pgfqpoint{2.655643in}{2.562719in}}%
\pgfpathlineto{\pgfqpoint{2.668882in}{2.545107in}}%
\pgfpathlineto{\pgfqpoint{2.682112in}{2.527775in}}%
\pgfpathlineto{\pgfqpoint{2.690281in}{2.533360in}}%
\pgfpathlineto{\pgfqpoint{2.698439in}{2.539088in}}%
\pgfpathlineto{\pgfqpoint{2.706586in}{2.544957in}}%
\pgfpathlineto{\pgfqpoint{2.714723in}{2.550965in}}%
\pgfpathlineto{\pgfqpoint{2.701521in}{2.568066in}}%
\pgfpathlineto{\pgfqpoint{2.688310in}{2.585446in}}%
\pgfpathlineto{\pgfqpoint{2.675091in}{2.603108in}}%
\pgfpathlineto{\pgfqpoint{2.661862in}{2.621053in}}%
\pgfpathlineto{\pgfqpoint{2.653697in}{2.615266in}}%
\pgfpathlineto{\pgfqpoint{2.645521in}{2.609625in}}%
\pgfpathlineto{\pgfqpoint{2.637335in}{2.604132in}}%
\pgfpathlineto{\pgfqpoint{2.629137in}{2.598790in}}%
\pgfpathclose%
\pgfusepath{fill}%
\end{pgfscope}%
\begin{pgfscope}%
\pgfpathrectangle{\pgfqpoint{1.254980in}{0.150000in}}{\pgfqpoint{5.490039in}{5.490039in}}%
\pgfusepath{clip}%
\pgfsetbuttcap%
\pgfsetroundjoin%
\definecolor{currentfill}{rgb}{0.165117,0.467423,0.558141}%
\pgfsetfillcolor{currentfill}%
\pgfsetfillopacity{0.700000}%
\pgfsetlinewidth{0.000000pt}%
\definecolor{currentstroke}{rgb}{0.000000,0.000000,0.000000}%
\pgfsetstrokecolor{currentstroke}%
\pgfsetdash{}{0pt}%
\pgfpathmoveto{\pgfqpoint{5.202665in}{2.949143in}}%
\pgfpathlineto{\pgfqpoint{5.216305in}{2.955459in}}%
\pgfpathlineto{\pgfqpoint{5.229959in}{2.961929in}}%
\pgfpathlineto{\pgfqpoint{5.243629in}{2.968553in}}%
\pgfpathlineto{\pgfqpoint{5.257314in}{2.975332in}}%
\pgfpathlineto{\pgfqpoint{5.264445in}{2.980947in}}%
\pgfpathlineto{\pgfqpoint{5.271572in}{2.986614in}}%
\pgfpathlineto{\pgfqpoint{5.278694in}{2.992339in}}%
\pgfpathlineto{\pgfqpoint{5.285813in}{2.998127in}}%
\pgfpathlineto{\pgfqpoint{5.272149in}{2.991813in}}%
\pgfpathlineto{\pgfqpoint{5.258500in}{2.985652in}}%
\pgfpathlineto{\pgfqpoint{5.244865in}{2.979645in}}%
\pgfpathlineto{\pgfqpoint{5.231245in}{2.973792in}}%
\pgfpathlineto{\pgfqpoint{5.224106in}{2.967531in}}%
\pgfpathlineto{\pgfqpoint{5.216963in}{2.961339in}}%
\pgfpathlineto{\pgfqpoint{5.209816in}{2.955211in}}%
\pgfpathlineto{\pgfqpoint{5.202665in}{2.949143in}}%
\pgfpathclose%
\pgfusepath{fill}%
\end{pgfscope}%
\begin{pgfscope}%
\pgfpathrectangle{\pgfqpoint{1.254980in}{0.150000in}}{\pgfqpoint{5.490039in}{5.490039in}}%
\pgfusepath{clip}%
\pgfsetbuttcap%
\pgfsetroundjoin%
\definecolor{currentfill}{rgb}{0.281924,0.089666,0.412415}%
\pgfsetfillcolor{currentfill}%
\pgfsetfillopacity{0.700000}%
\pgfsetlinewidth{0.000000pt}%
\definecolor{currentstroke}{rgb}{0.000000,0.000000,0.000000}%
\pgfsetstrokecolor{currentstroke}%
\pgfsetdash{}{0pt}%
\pgfpathmoveto{\pgfqpoint{3.134191in}{2.137637in}}%
\pgfpathlineto{\pgfqpoint{3.147249in}{2.128552in}}%
\pgfpathlineto{\pgfqpoint{3.160305in}{2.119681in}}%
\pgfpathlineto{\pgfqpoint{3.173361in}{2.111023in}}%
\pgfpathlineto{\pgfqpoint{3.186416in}{2.102576in}}%
\pgfpathlineto{\pgfqpoint{3.194343in}{2.110577in}}%
\pgfpathlineto{\pgfqpoint{3.202263in}{2.118649in}}%
\pgfpathlineto{\pgfqpoint{3.210176in}{2.126791in}}%
\pgfpathlineto{\pgfqpoint{3.218081in}{2.135001in}}%
\pgfpathlineto{\pgfqpoint{3.205043in}{2.143257in}}%
\pgfpathlineto{\pgfqpoint{3.192005in}{2.151723in}}%
\pgfpathlineto{\pgfqpoint{3.178967in}{2.160402in}}%
\pgfpathlineto{\pgfqpoint{3.165928in}{2.169295in}}%
\pgfpathlineto{\pgfqpoint{3.158005in}{2.161266in}}%
\pgfpathlineto{\pgfqpoint{3.150074in}{2.153312in}}%
\pgfpathlineto{\pgfqpoint{3.142136in}{2.145435in}}%
\pgfpathlineto{\pgfqpoint{3.134191in}{2.137637in}}%
\pgfpathclose%
\pgfusepath{fill}%
\end{pgfscope}%
\begin{pgfscope}%
\pgfpathrectangle{\pgfqpoint{1.254980in}{0.150000in}}{\pgfqpoint{5.490039in}{5.490039in}}%
\pgfusepath{clip}%
\pgfsetbuttcap%
\pgfsetroundjoin%
\definecolor{currentfill}{rgb}{0.280868,0.160771,0.472899}%
\pgfsetfillcolor{currentfill}%
\pgfsetfillopacity{0.700000}%
\pgfsetlinewidth{0.000000pt}%
\definecolor{currentstroke}{rgb}{0.000000,0.000000,0.000000}%
\pgfsetstrokecolor{currentstroke}%
\pgfsetdash{}{0pt}%
\pgfpathmoveto{\pgfqpoint{2.892813in}{2.286488in}}%
\pgfpathlineto{\pgfqpoint{2.905932in}{2.273548in}}%
\pgfpathlineto{\pgfqpoint{2.919048in}{2.260848in}}%
\pgfpathlineto{\pgfqpoint{2.932160in}{2.248386in}}%
\pgfpathlineto{\pgfqpoint{2.945267in}{2.236160in}}%
\pgfpathlineto{\pgfqpoint{2.953307in}{2.242931in}}%
\pgfpathlineto{\pgfqpoint{2.961338in}{2.249809in}}%
\pgfpathlineto{\pgfqpoint{2.969361in}{2.256792in}}%
\pgfpathlineto{\pgfqpoint{2.977375in}{2.263878in}}%
\pgfpathlineto{\pgfqpoint{2.964289in}{2.275881in}}%
\pgfpathlineto{\pgfqpoint{2.951201in}{2.288120in}}%
\pgfpathlineto{\pgfqpoint{2.938108in}{2.300596in}}%
\pgfpathlineto{\pgfqpoint{2.925012in}{2.313312in}}%
\pgfpathlineto{\pgfqpoint{2.916975in}{2.306439in}}%
\pgfpathlineto{\pgfqpoint{2.908930in}{2.299676in}}%
\pgfpathlineto{\pgfqpoint{2.900876in}{2.293025in}}%
\pgfpathlineto{\pgfqpoint{2.892813in}{2.286488in}}%
\pgfpathclose%
\pgfusepath{fill}%
\end{pgfscope}%
\begin{pgfscope}%
\pgfpathrectangle{\pgfqpoint{1.254980in}{0.150000in}}{\pgfqpoint{5.490039in}{5.490039in}}%
\pgfusepath{clip}%
\pgfsetbuttcap%
\pgfsetroundjoin%
\definecolor{currentfill}{rgb}{0.157729,0.485932,0.558013}%
\pgfsetfillcolor{currentfill}%
\pgfsetfillopacity{0.700000}%
\pgfsetlinewidth{0.000000pt}%
\definecolor{currentstroke}{rgb}{0.000000,0.000000,0.000000}%
\pgfsetstrokecolor{currentstroke}%
\pgfsetdash{}{0pt}%
\pgfpathmoveto{\pgfqpoint{5.285813in}{2.998127in}}%
\pgfpathlineto{\pgfqpoint{5.299492in}{3.004594in}}%
\pgfpathlineto{\pgfqpoint{5.313186in}{3.011214in}}%
\pgfpathlineto{\pgfqpoint{5.326895in}{3.017988in}}%
\pgfpathlineto{\pgfqpoint{5.340620in}{3.024915in}}%
\pgfpathlineto{\pgfqpoint{5.347713in}{3.030288in}}%
\pgfpathlineto{\pgfqpoint{5.354802in}{3.035727in}}%
\pgfpathlineto{\pgfqpoint{5.361887in}{3.041239in}}%
\pgfpathlineto{\pgfqpoint{5.368968in}{3.046829in}}%
\pgfpathlineto{\pgfqpoint{5.355266in}{3.040395in}}%
\pgfpathlineto{\pgfqpoint{5.341579in}{3.034115in}}%
\pgfpathlineto{\pgfqpoint{5.327907in}{3.027986in}}%
\pgfpathlineto{\pgfqpoint{5.314249in}{3.022011in}}%
\pgfpathlineto{\pgfqpoint{5.307145in}{3.015919in}}%
\pgfpathlineto{\pgfqpoint{5.300038in}{3.009911in}}%
\pgfpathlineto{\pgfqpoint{5.292928in}{3.003982in}}%
\pgfpathlineto{\pgfqpoint{5.285813in}{2.998127in}}%
\pgfpathclose%
\pgfusepath{fill}%
\end{pgfscope}%
\begin{pgfscope}%
\pgfpathrectangle{\pgfqpoint{1.254980in}{0.150000in}}{\pgfqpoint{5.490039in}{5.490039in}}%
\pgfusepath{clip}%
\pgfsetbuttcap%
\pgfsetroundjoin%
\definecolor{currentfill}{rgb}{0.150476,0.504369,0.557430}%
\pgfsetfillcolor{currentfill}%
\pgfsetfillopacity{0.700000}%
\pgfsetlinewidth{0.000000pt}%
\definecolor{currentstroke}{rgb}{0.000000,0.000000,0.000000}%
\pgfsetstrokecolor{currentstroke}%
\pgfsetdash{}{0pt}%
\pgfpathmoveto{\pgfqpoint{5.368968in}{3.046829in}}%
\pgfpathlineto{\pgfqpoint{5.382686in}{3.053415in}}%
\pgfpathlineto{\pgfqpoint{5.396419in}{3.060153in}}%
\pgfpathlineto{\pgfqpoint{5.410168in}{3.067045in}}%
\pgfpathlineto{\pgfqpoint{5.423932in}{3.074088in}}%
\pgfpathlineto{\pgfqpoint{5.430986in}{3.079249in}}%
\pgfpathlineto{\pgfqpoint{5.438037in}{3.084493in}}%
\pgfpathlineto{\pgfqpoint{5.445085in}{3.089825in}}%
\pgfpathlineto{\pgfqpoint{5.452130in}{3.095251in}}%
\pgfpathlineto{\pgfqpoint{5.438390in}{3.088730in}}%
\pgfpathlineto{\pgfqpoint{5.424665in}{3.082361in}}%
\pgfpathlineto{\pgfqpoint{5.410955in}{3.076143in}}%
\pgfpathlineto{\pgfqpoint{5.397261in}{3.070078in}}%
\pgfpathlineto{\pgfqpoint{5.390192in}{3.064120in}}%
\pgfpathlineto{\pgfqpoint{5.383121in}{3.058264in}}%
\pgfpathlineto{\pgfqpoint{5.376046in}{3.052502in}}%
\pgfpathlineto{\pgfqpoint{5.368968in}{3.046829in}}%
\pgfpathclose%
\pgfusepath{fill}%
\end{pgfscope}%
\begin{pgfscope}%
\pgfpathrectangle{\pgfqpoint{1.254980in}{0.150000in}}{\pgfqpoint{5.490039in}{5.490039in}}%
\pgfusepath{clip}%
\pgfsetbuttcap%
\pgfsetroundjoin%
\definecolor{currentfill}{rgb}{0.221989,0.339161,0.548752}%
\pgfsetfillcolor{currentfill}%
\pgfsetfillopacity{0.700000}%
\pgfsetlinewidth{0.000000pt}%
\definecolor{currentstroke}{rgb}{0.000000,0.000000,0.000000}%
\pgfsetstrokecolor{currentstroke}%
\pgfsetdash{}{0pt}%
\pgfpathmoveto{\pgfqpoint{2.576007in}{2.674399in}}%
\pgfpathlineto{\pgfqpoint{2.589305in}{2.655056in}}%
\pgfpathlineto{\pgfqpoint{2.602593in}{2.636009in}}%
\pgfpathlineto{\pgfqpoint{2.615870in}{2.617254in}}%
\pgfpathlineto{\pgfqpoint{2.629137in}{2.598790in}}%
\pgfpathlineto{\pgfqpoint{2.637335in}{2.604132in}}%
\pgfpathlineto{\pgfqpoint{2.645521in}{2.609625in}}%
\pgfpathlineto{\pgfqpoint{2.653697in}{2.615266in}}%
\pgfpathlineto{\pgfqpoint{2.661862in}{2.621053in}}%
\pgfpathlineto{\pgfqpoint{2.648624in}{2.639285in}}%
\pgfpathlineto{\pgfqpoint{2.635376in}{2.657807in}}%
\pgfpathlineto{\pgfqpoint{2.622119in}{2.676620in}}%
\pgfpathlineto{\pgfqpoint{2.608851in}{2.695728in}}%
\pgfpathlineto{\pgfqpoint{2.600657in}{2.690164in}}%
\pgfpathlineto{\pgfqpoint{2.592452in}{2.684753in}}%
\pgfpathlineto{\pgfqpoint{2.584235in}{2.679497in}}%
\pgfpathlineto{\pgfqpoint{2.576007in}{2.674399in}}%
\pgfpathclose%
\pgfusepath{fill}%
\end{pgfscope}%
\begin{pgfscope}%
\pgfpathrectangle{\pgfqpoint{1.254980in}{0.150000in}}{\pgfqpoint{5.490039in}{5.490039in}}%
\pgfusepath{clip}%
\pgfsetbuttcap%
\pgfsetroundjoin%
\definecolor{currentfill}{rgb}{0.143343,0.522773,0.556295}%
\pgfsetfillcolor{currentfill}%
\pgfsetfillopacity{0.700000}%
\pgfsetlinewidth{0.000000pt}%
\definecolor{currentstroke}{rgb}{0.000000,0.000000,0.000000}%
\pgfsetstrokecolor{currentstroke}%
\pgfsetdash{}{0pt}%
\pgfpathmoveto{\pgfqpoint{5.452130in}{3.095251in}}%
\pgfpathlineto{\pgfqpoint{5.465885in}{3.101924in}}%
\pgfpathlineto{\pgfqpoint{5.479657in}{3.108749in}}%
\pgfpathlineto{\pgfqpoint{5.493444in}{3.115725in}}%
\pgfpathlineto{\pgfqpoint{5.507247in}{3.122854in}}%
\pgfpathlineto{\pgfqpoint{5.514263in}{3.127839in}}%
\pgfpathlineto{\pgfqpoint{5.521277in}{3.132923in}}%
\pgfpathlineto{\pgfqpoint{5.528288in}{3.138113in}}%
\pgfpathlineto{\pgfqpoint{5.535296in}{3.143415in}}%
\pgfpathlineto{\pgfqpoint{5.521519in}{3.136838in}}%
\pgfpathlineto{\pgfqpoint{5.507757in}{3.130412in}}%
\pgfpathlineto{\pgfqpoint{5.494012in}{3.124138in}}%
\pgfpathlineto{\pgfqpoint{5.480281in}{3.118014in}}%
\pgfpathlineto{\pgfqpoint{5.473247in}{3.112152in}}%
\pgfpathlineto{\pgfqpoint{5.466210in}{3.106409in}}%
\pgfpathlineto{\pgfqpoint{5.459171in}{3.100777in}}%
\pgfpathlineto{\pgfqpoint{5.452130in}{3.095251in}}%
\pgfpathclose%
\pgfusepath{fill}%
\end{pgfscope}%
\begin{pgfscope}%
\pgfpathrectangle{\pgfqpoint{1.254980in}{0.150000in}}{\pgfqpoint{5.490039in}{5.490039in}}%
\pgfusepath{clip}%
\pgfsetbuttcap%
\pgfsetroundjoin%
\definecolor{currentfill}{rgb}{0.279566,0.067836,0.391917}%
\pgfsetfillcolor{currentfill}%
\pgfsetfillopacity{0.700000}%
\pgfsetlinewidth{0.000000pt}%
\definecolor{currentstroke}{rgb}{0.000000,0.000000,0.000000}%
\pgfsetstrokecolor{currentstroke}%
\pgfsetdash{}{0pt}%
\pgfpathmoveto{\pgfqpoint{3.541525in}{2.086330in}}%
\pgfpathlineto{\pgfqpoint{3.554576in}{2.082540in}}%
\pgfpathlineto{\pgfqpoint{3.567631in}{2.078936in}}%
\pgfpathlineto{\pgfqpoint{3.580691in}{2.075520in}}%
\pgfpathlineto{\pgfqpoint{3.593755in}{2.072289in}}%
\pgfpathlineto{\pgfqpoint{3.601523in}{2.081762in}}%
\pgfpathlineto{\pgfqpoint{3.609286in}{2.091250in}}%
\pgfpathlineto{\pgfqpoint{3.617043in}{2.100753in}}%
\pgfpathlineto{\pgfqpoint{3.624795in}{2.110270in}}%
\pgfpathlineto{\pgfqpoint{3.611742in}{2.113395in}}%
\pgfpathlineto{\pgfqpoint{3.598694in}{2.116705in}}%
\pgfpathlineto{\pgfqpoint{3.585649in}{2.120202in}}%
\pgfpathlineto{\pgfqpoint{3.572609in}{2.123887in}}%
\pgfpathlineto{\pgfqpoint{3.564846in}{2.114465in}}%
\pgfpathlineto{\pgfqpoint{3.557078in}{2.105065in}}%
\pgfpathlineto{\pgfqpoint{3.549304in}{2.095686in}}%
\pgfpathlineto{\pgfqpoint{3.541525in}{2.086330in}}%
\pgfpathclose%
\pgfusepath{fill}%
\end{pgfscope}%
\begin{pgfscope}%
\pgfpathrectangle{\pgfqpoint{1.254980in}{0.150000in}}{\pgfqpoint{5.490039in}{5.490039in}}%
\pgfusepath{clip}%
\pgfsetbuttcap%
\pgfsetroundjoin%
\definecolor{currentfill}{rgb}{0.136408,0.541173,0.554483}%
\pgfsetfillcolor{currentfill}%
\pgfsetfillopacity{0.700000}%
\pgfsetlinewidth{0.000000pt}%
\definecolor{currentstroke}{rgb}{0.000000,0.000000,0.000000}%
\pgfsetstrokecolor{currentstroke}%
\pgfsetdash{}{0pt}%
\pgfpathmoveto{\pgfqpoint{5.535296in}{3.143415in}}%
\pgfpathlineto{\pgfqpoint{5.549089in}{3.150143in}}%
\pgfpathlineto{\pgfqpoint{5.562898in}{3.157022in}}%
\pgfpathlineto{\pgfqpoint{5.576723in}{3.164052in}}%
\pgfpathlineto{\pgfqpoint{5.590564in}{3.171234in}}%
\pgfpathlineto{\pgfqpoint{5.597543in}{3.176084in}}%
\pgfpathlineto{\pgfqpoint{5.604520in}{3.181051in}}%
\pgfpathlineto{\pgfqpoint{5.611495in}{3.186142in}}%
\pgfpathlineto{\pgfqpoint{5.618468in}{3.191364in}}%
\pgfpathlineto{\pgfqpoint{5.604654in}{3.184763in}}%
\pgfpathlineto{\pgfqpoint{5.590857in}{3.178313in}}%
\pgfpathlineto{\pgfqpoint{5.577075in}{3.172012in}}%
\pgfpathlineto{\pgfqpoint{5.563309in}{3.165862in}}%
\pgfpathlineto{\pgfqpoint{5.556309in}{3.160052in}}%
\pgfpathlineto{\pgfqpoint{5.549306in}{3.154378in}}%
\pgfpathlineto{\pgfqpoint{5.542302in}{3.148835in}}%
\pgfpathlineto{\pgfqpoint{5.535296in}{3.143415in}}%
\pgfpathclose%
\pgfusepath{fill}%
\end{pgfscope}%
\begin{pgfscope}%
\pgfpathrectangle{\pgfqpoint{1.254980in}{0.150000in}}{\pgfqpoint{5.490039in}{5.490039in}}%
\pgfusepath{clip}%
\pgfsetbuttcap%
\pgfsetroundjoin%
\definecolor{currentfill}{rgb}{0.278012,0.180367,0.486697}%
\pgfsetfillcolor{currentfill}%
\pgfsetfillopacity{0.700000}%
\pgfsetlinewidth{0.000000pt}%
\definecolor{currentstroke}{rgb}{0.000000,0.000000,0.000000}%
\pgfsetstrokecolor{currentstroke}%
\pgfsetdash{}{0pt}%
\pgfpathmoveto{\pgfqpoint{4.092722in}{2.284920in}}%
\pgfpathlineto{\pgfqpoint{4.105899in}{2.286332in}}%
\pgfpathlineto{\pgfqpoint{4.119085in}{2.287914in}}%
\pgfpathlineto{\pgfqpoint{4.132280in}{2.289666in}}%
\pgfpathlineto{\pgfqpoint{4.145484in}{2.291586in}}%
\pgfpathlineto{\pgfqpoint{4.153070in}{2.301033in}}%
\pgfpathlineto{\pgfqpoint{4.160650in}{2.310451in}}%
\pgfpathlineto{\pgfqpoint{4.168226in}{2.319839in}}%
\pgfpathlineto{\pgfqpoint{4.175796in}{2.329200in}}%
\pgfpathlineto{\pgfqpoint{4.162600in}{2.327342in}}%
\pgfpathlineto{\pgfqpoint{4.149412in}{2.325653in}}%
\pgfpathlineto{\pgfqpoint{4.136234in}{2.324132in}}%
\pgfpathlineto{\pgfqpoint{4.123064in}{2.322782in}}%
\pgfpathlineto{\pgfqpoint{4.115486in}{2.313348in}}%
\pgfpathlineto{\pgfqpoint{4.107903in}{2.303894in}}%
\pgfpathlineto{\pgfqpoint{4.100315in}{2.294419in}}%
\pgfpathlineto{\pgfqpoint{4.092722in}{2.284920in}}%
\pgfpathclose%
\pgfusepath{fill}%
\end{pgfscope}%
\begin{pgfscope}%
\pgfpathrectangle{\pgfqpoint{1.254980in}{0.150000in}}{\pgfqpoint{5.490039in}{5.490039in}}%
\pgfusepath{clip}%
\pgfsetbuttcap%
\pgfsetroundjoin%
\definecolor{currentfill}{rgb}{0.282623,0.140926,0.457517}%
\pgfsetfillcolor{currentfill}%
\pgfsetfillopacity{0.700000}%
\pgfsetlinewidth{0.000000pt}%
\definecolor{currentstroke}{rgb}{0.000000,0.000000,0.000000}%
\pgfsetstrokecolor{currentstroke}%
\pgfsetdash{}{0pt}%
\pgfpathmoveto{\pgfqpoint{2.945267in}{2.236160in}}%
\pgfpathlineto{\pgfqpoint{2.958371in}{2.224169in}}%
\pgfpathlineto{\pgfqpoint{2.971472in}{2.212410in}}%
\pgfpathlineto{\pgfqpoint{2.984570in}{2.200883in}}%
\pgfpathlineto{\pgfqpoint{2.997664in}{2.189584in}}%
\pgfpathlineto{\pgfqpoint{3.005681in}{2.196588in}}%
\pgfpathlineto{\pgfqpoint{3.013690in}{2.203692in}}%
\pgfpathlineto{\pgfqpoint{3.021691in}{2.210894in}}%
\pgfpathlineto{\pgfqpoint{3.029683in}{2.218191in}}%
\pgfpathlineto{\pgfqpoint{3.016610in}{2.229268in}}%
\pgfpathlineto{\pgfqpoint{3.003535in}{2.240573in}}%
\pgfpathlineto{\pgfqpoint{2.990456in}{2.252110in}}%
\pgfpathlineto{\pgfqpoint{2.977375in}{2.263878in}}%
\pgfpathlineto{\pgfqpoint{2.969361in}{2.256792in}}%
\pgfpathlineto{\pgfqpoint{2.961338in}{2.249809in}}%
\pgfpathlineto{\pgfqpoint{2.953307in}{2.242931in}}%
\pgfpathlineto{\pgfqpoint{2.945267in}{2.236160in}}%
\pgfpathclose%
\pgfusepath{fill}%
\end{pgfscope}%
\begin{pgfscope}%
\pgfpathrectangle{\pgfqpoint{1.254980in}{0.150000in}}{\pgfqpoint{5.490039in}{5.490039in}}%
\pgfusepath{clip}%
\pgfsetbuttcap%
\pgfsetroundjoin%
\definecolor{currentfill}{rgb}{0.273006,0.204520,0.501721}%
\pgfsetfillcolor{currentfill}%
\pgfsetfillopacity{0.700000}%
\pgfsetlinewidth{0.000000pt}%
\definecolor{currentstroke}{rgb}{0.000000,0.000000,0.000000}%
\pgfsetstrokecolor{currentstroke}%
\pgfsetdash{}{0pt}%
\pgfpathmoveto{\pgfqpoint{4.175796in}{2.329200in}}%
\pgfpathlineto{\pgfqpoint{4.189002in}{2.331227in}}%
\pgfpathlineto{\pgfqpoint{4.202217in}{2.333422in}}%
\pgfpathlineto{\pgfqpoint{4.215442in}{2.335784in}}%
\pgfpathlineto{\pgfqpoint{4.228676in}{2.338314in}}%
\pgfpathlineto{\pgfqpoint{4.236234in}{2.347569in}}%
\pgfpathlineto{\pgfqpoint{4.243786in}{2.356791in}}%
\pgfpathlineto{\pgfqpoint{4.251333in}{2.365983in}}%
\pgfpathlineto{\pgfqpoint{4.258876in}{2.375146in}}%
\pgfpathlineto{\pgfqpoint{4.245649in}{2.372707in}}%
\pgfpathlineto{\pgfqpoint{4.232432in}{2.370435in}}%
\pgfpathlineto{\pgfqpoint{4.219224in}{2.368330in}}%
\pgfpathlineto{\pgfqpoint{4.206026in}{2.366393in}}%
\pgfpathlineto{\pgfqpoint{4.198477in}{2.357130in}}%
\pgfpathlineto{\pgfqpoint{4.190921in}{2.347845in}}%
\pgfpathlineto{\pgfqpoint{4.183361in}{2.338535in}}%
\pgfpathlineto{\pgfqpoint{4.175796in}{2.329200in}}%
\pgfpathclose%
\pgfusepath{fill}%
\end{pgfscope}%
\begin{pgfscope}%
\pgfpathrectangle{\pgfqpoint{1.254980in}{0.150000in}}{\pgfqpoint{5.490039in}{5.490039in}}%
\pgfusepath{clip}%
\pgfsetbuttcap%
\pgfsetroundjoin%
\definecolor{currentfill}{rgb}{0.280868,0.160771,0.472899}%
\pgfsetfillcolor{currentfill}%
\pgfsetfillopacity{0.700000}%
\pgfsetlinewidth{0.000000pt}%
\definecolor{currentstroke}{rgb}{0.000000,0.000000,0.000000}%
\pgfsetstrokecolor{currentstroke}%
\pgfsetdash{}{0pt}%
\pgfpathmoveto{\pgfqpoint{4.009644in}{2.242602in}}%
\pgfpathlineto{\pgfqpoint{4.022795in}{2.243364in}}%
\pgfpathlineto{\pgfqpoint{4.035955in}{2.244298in}}%
\pgfpathlineto{\pgfqpoint{4.049122in}{2.245403in}}%
\pgfpathlineto{\pgfqpoint{4.062299in}{2.246679in}}%
\pgfpathlineto{\pgfqpoint{4.069912in}{2.256278in}}%
\pgfpathlineto{\pgfqpoint{4.077520in}{2.265850in}}%
\pgfpathlineto{\pgfqpoint{4.085123in}{2.275397in}}%
\pgfpathlineto{\pgfqpoint{4.092722in}{2.284920in}}%
\pgfpathlineto{\pgfqpoint{4.079553in}{2.283678in}}%
\pgfpathlineto{\pgfqpoint{4.066393in}{2.282606in}}%
\pgfpathlineto{\pgfqpoint{4.053241in}{2.281706in}}%
\pgfpathlineto{\pgfqpoint{4.040098in}{2.280978in}}%
\pgfpathlineto{\pgfqpoint{4.032492in}{2.271411in}}%
\pgfpathlineto{\pgfqpoint{4.024881in}{2.261827in}}%
\pgfpathlineto{\pgfqpoint{4.017265in}{2.252224in}}%
\pgfpathlineto{\pgfqpoint{4.009644in}{2.242602in}}%
\pgfpathclose%
\pgfusepath{fill}%
\end{pgfscope}%
\begin{pgfscope}%
\pgfpathrectangle{\pgfqpoint{1.254980in}{0.150000in}}{\pgfqpoint{5.490039in}{5.490039in}}%
\pgfusepath{clip}%
\pgfsetbuttcap%
\pgfsetroundjoin%
\definecolor{currentfill}{rgb}{0.129933,0.559582,0.551864}%
\pgfsetfillcolor{currentfill}%
\pgfsetfillopacity{0.700000}%
\pgfsetlinewidth{0.000000pt}%
\definecolor{currentstroke}{rgb}{0.000000,0.000000,0.000000}%
\pgfsetstrokecolor{currentstroke}%
\pgfsetdash{}{0pt}%
\pgfpathmoveto{\pgfqpoint{5.618468in}{3.191364in}}%
\pgfpathlineto{\pgfqpoint{5.632297in}{3.198116in}}%
\pgfpathlineto{\pgfqpoint{5.646143in}{3.205017in}}%
\pgfpathlineto{\pgfqpoint{5.660004in}{3.212070in}}%
\pgfpathlineto{\pgfqpoint{5.673883in}{3.219273in}}%
\pgfpathlineto{\pgfqpoint{5.680825in}{3.224033in}}%
\pgfpathlineto{\pgfqpoint{5.687766in}{3.228930in}}%
\pgfpathlineto{\pgfqpoint{5.694706in}{3.233971in}}%
\pgfpathlineto{\pgfqpoint{5.701645in}{3.239162in}}%
\pgfpathlineto{\pgfqpoint{5.687797in}{3.232568in}}%
\pgfpathlineto{\pgfqpoint{5.673965in}{3.226124in}}%
\pgfpathlineto{\pgfqpoint{5.660148in}{3.219830in}}%
\pgfpathlineto{\pgfqpoint{5.646348in}{3.213685in}}%
\pgfpathlineto{\pgfqpoint{5.639379in}{3.207877in}}%
\pgfpathlineto{\pgfqpoint{5.632410in}{3.202225in}}%
\pgfpathlineto{\pgfqpoint{5.625439in}{3.196723in}}%
\pgfpathlineto{\pgfqpoint{5.618468in}{3.191364in}}%
\pgfpathclose%
\pgfusepath{fill}%
\end{pgfscope}%
\begin{pgfscope}%
\pgfpathrectangle{\pgfqpoint{1.254980in}{0.150000in}}{\pgfqpoint{5.490039in}{5.490039in}}%
\pgfusepath{clip}%
\pgfsetbuttcap%
\pgfsetroundjoin%
\definecolor{currentfill}{rgb}{0.266580,0.228262,0.514349}%
\pgfsetfillcolor{currentfill}%
\pgfsetfillopacity{0.700000}%
\pgfsetlinewidth{0.000000pt}%
\definecolor{currentstroke}{rgb}{0.000000,0.000000,0.000000}%
\pgfsetstrokecolor{currentstroke}%
\pgfsetdash{}{0pt}%
\pgfpathmoveto{\pgfqpoint{4.258876in}{2.375146in}}%
\pgfpathlineto{\pgfqpoint{4.272112in}{2.377752in}}%
\pgfpathlineto{\pgfqpoint{4.285359in}{2.380524in}}%
\pgfpathlineto{\pgfqpoint{4.298615in}{2.383463in}}%
\pgfpathlineto{\pgfqpoint{4.311882in}{2.386567in}}%
\pgfpathlineto{\pgfqpoint{4.319411in}{2.395594in}}%
\pgfpathlineto{\pgfqpoint{4.326935in}{2.404587in}}%
\pgfpathlineto{\pgfqpoint{4.334454in}{2.413548in}}%
\pgfpathlineto{\pgfqpoint{4.341967in}{2.422480in}}%
\pgfpathlineto{\pgfqpoint{4.328708in}{2.419495in}}%
\pgfpathlineto{\pgfqpoint{4.315460in}{2.416675in}}%
\pgfpathlineto{\pgfqpoint{4.302221in}{2.414021in}}%
\pgfpathlineto{\pgfqpoint{4.288992in}{2.411533in}}%
\pgfpathlineto{\pgfqpoint{4.281471in}{2.402473in}}%
\pgfpathlineto{\pgfqpoint{4.273944in}{2.393389in}}%
\pgfpathlineto{\pgfqpoint{4.266413in}{2.384281in}}%
\pgfpathlineto{\pgfqpoint{4.258876in}{2.375146in}}%
\pgfpathclose%
\pgfusepath{fill}%
\end{pgfscope}%
\begin{pgfscope}%
\pgfpathrectangle{\pgfqpoint{1.254980in}{0.150000in}}{\pgfqpoint{5.490039in}{5.490039in}}%
\pgfusepath{clip}%
\pgfsetbuttcap%
\pgfsetroundjoin%
\definecolor{currentfill}{rgb}{0.282884,0.135920,0.453427}%
\pgfsetfillcolor{currentfill}%
\pgfsetfillopacity{0.700000}%
\pgfsetlinewidth{0.000000pt}%
\definecolor{currentstroke}{rgb}{0.000000,0.000000,0.000000}%
\pgfsetstrokecolor{currentstroke}%
\pgfsetdash{}{0pt}%
\pgfpathmoveto{\pgfqpoint{3.926553in}{2.202563in}}%
\pgfpathlineto{\pgfqpoint{3.939681in}{2.202639in}}%
\pgfpathlineto{\pgfqpoint{3.952816in}{2.202889in}}%
\pgfpathlineto{\pgfqpoint{3.965959in}{2.203312in}}%
\pgfpathlineto{\pgfqpoint{3.979110in}{2.203908in}}%
\pgfpathlineto{\pgfqpoint{3.986751in}{2.213613in}}%
\pgfpathlineto{\pgfqpoint{3.994387in}{2.223297in}}%
\pgfpathlineto{\pgfqpoint{4.002018in}{2.232960in}}%
\pgfpathlineto{\pgfqpoint{4.009644in}{2.242602in}}%
\pgfpathlineto{\pgfqpoint{3.996501in}{2.242012in}}%
\pgfpathlineto{\pgfqpoint{3.983366in}{2.241594in}}%
\pgfpathlineto{\pgfqpoint{3.970238in}{2.241350in}}%
\pgfpathlineto{\pgfqpoint{3.957118in}{2.241280in}}%
\pgfpathlineto{\pgfqpoint{3.949484in}{2.231622in}}%
\pgfpathlineto{\pgfqpoint{3.941845in}{2.221950in}}%
\pgfpathlineto{\pgfqpoint{3.934202in}{2.212264in}}%
\pgfpathlineto{\pgfqpoint{3.926553in}{2.202563in}}%
\pgfpathclose%
\pgfusepath{fill}%
\end{pgfscope}%
\begin{pgfscope}%
\pgfpathrectangle{\pgfqpoint{1.254980in}{0.150000in}}{\pgfqpoint{5.490039in}{5.490039in}}%
\pgfusepath{clip}%
\pgfsetbuttcap%
\pgfsetroundjoin%
\definecolor{currentfill}{rgb}{0.124395,0.578002,0.548287}%
\pgfsetfillcolor{currentfill}%
\pgfsetfillopacity{0.700000}%
\pgfsetlinewidth{0.000000pt}%
\definecolor{currentstroke}{rgb}{0.000000,0.000000,0.000000}%
\pgfsetstrokecolor{currentstroke}%
\pgfsetdash{}{0pt}%
\pgfpathmoveto{\pgfqpoint{5.701645in}{3.239162in}}%
\pgfpathlineto{\pgfqpoint{5.715510in}{3.245905in}}%
\pgfpathlineto{\pgfqpoint{5.729391in}{3.252798in}}%
\pgfpathlineto{\pgfqpoint{5.743289in}{3.259841in}}%
\pgfpathlineto{\pgfqpoint{5.757203in}{3.267034in}}%
\pgfpathlineto{\pgfqpoint{5.764110in}{3.271755in}}%
\pgfpathlineto{\pgfqpoint{5.771017in}{3.276634in}}%
\pgfpathlineto{\pgfqpoint{5.777924in}{3.281677in}}%
\pgfpathlineto{\pgfqpoint{5.784831in}{3.286893in}}%
\pgfpathlineto{\pgfqpoint{5.770949in}{3.280338in}}%
\pgfpathlineto{\pgfqpoint{5.757083in}{3.273932in}}%
\pgfpathlineto{\pgfqpoint{5.743233in}{3.267675in}}%
\pgfpathlineto{\pgfqpoint{5.729399in}{3.261567in}}%
\pgfpathlineto{\pgfqpoint{5.722460in}{3.255706in}}%
\pgfpathlineto{\pgfqpoint{5.715522in}{3.250022in}}%
\pgfpathlineto{\pgfqpoint{5.708584in}{3.244510in}}%
\pgfpathlineto{\pgfqpoint{5.701645in}{3.239162in}}%
\pgfpathclose%
\pgfusepath{fill}%
\end{pgfscope}%
\begin{pgfscope}%
\pgfpathrectangle{\pgfqpoint{1.254980in}{0.150000in}}{\pgfqpoint{5.490039in}{5.490039in}}%
\pgfusepath{clip}%
\pgfsetbuttcap%
\pgfsetroundjoin%
\definecolor{currentfill}{rgb}{0.258965,0.251537,0.524736}%
\pgfsetfillcolor{currentfill}%
\pgfsetfillopacity{0.700000}%
\pgfsetlinewidth{0.000000pt}%
\definecolor{currentstroke}{rgb}{0.000000,0.000000,0.000000}%
\pgfsetstrokecolor{currentstroke}%
\pgfsetdash{}{0pt}%
\pgfpathmoveto{\pgfqpoint{4.341967in}{2.422480in}}%
\pgfpathlineto{\pgfqpoint{4.355236in}{2.425630in}}%
\pgfpathlineto{\pgfqpoint{4.368516in}{2.428946in}}%
\pgfpathlineto{\pgfqpoint{4.381807in}{2.432426in}}%
\pgfpathlineto{\pgfqpoint{4.395108in}{2.436071in}}%
\pgfpathlineto{\pgfqpoint{4.402608in}{2.444837in}}%
\pgfpathlineto{\pgfqpoint{4.410102in}{2.453570in}}%
\pgfpathlineto{\pgfqpoint{4.417592in}{2.462272in}}%
\pgfpathlineto{\pgfqpoint{4.425076in}{2.470945in}}%
\pgfpathlineto{\pgfqpoint{4.411783in}{2.467447in}}%
\pgfpathlineto{\pgfqpoint{4.398500in}{2.464114in}}%
\pgfpathlineto{\pgfqpoint{4.385229in}{2.460946in}}%
\pgfpathlineto{\pgfqpoint{4.371968in}{2.457942in}}%
\pgfpathlineto{\pgfqpoint{4.364476in}{2.449112in}}%
\pgfpathlineto{\pgfqpoint{4.356978in}{2.440260in}}%
\pgfpathlineto{\pgfqpoint{4.349475in}{2.431383in}}%
\pgfpathlineto{\pgfqpoint{4.341967in}{2.422480in}}%
\pgfpathclose%
\pgfusepath{fill}%
\end{pgfscope}%
\begin{pgfscope}%
\pgfpathrectangle{\pgfqpoint{1.254980in}{0.150000in}}{\pgfqpoint{5.490039in}{5.490039in}}%
\pgfusepath{clip}%
\pgfsetbuttcap%
\pgfsetroundjoin%
\definecolor{currentfill}{rgb}{0.120565,0.596422,0.543611}%
\pgfsetfillcolor{currentfill}%
\pgfsetfillopacity{0.700000}%
\pgfsetlinewidth{0.000000pt}%
\definecolor{currentstroke}{rgb}{0.000000,0.000000,0.000000}%
\pgfsetstrokecolor{currentstroke}%
\pgfsetdash{}{0pt}%
\pgfpathmoveto{\pgfqpoint{5.784831in}{3.286893in}}%
\pgfpathlineto{\pgfqpoint{5.798730in}{3.293596in}}%
\pgfpathlineto{\pgfqpoint{5.812645in}{3.300449in}}%
\pgfpathlineto{\pgfqpoint{5.826578in}{3.307451in}}%
\pgfpathlineto{\pgfqpoint{5.840527in}{3.314603in}}%
\pgfpathlineto{\pgfqpoint{5.847401in}{3.319341in}}%
\pgfpathlineto{\pgfqpoint{5.854276in}{3.324259in}}%
\pgfpathlineto{\pgfqpoint{5.861151in}{3.329363in}}%
\pgfpathlineto{\pgfqpoint{5.868028in}{3.334662in}}%
\pgfpathlineto{\pgfqpoint{5.854113in}{3.328177in}}%
\pgfpathlineto{\pgfqpoint{5.840214in}{3.321840in}}%
\pgfpathlineto{\pgfqpoint{5.826332in}{3.315652in}}%
\pgfpathlineto{\pgfqpoint{5.812466in}{3.309612in}}%
\pgfpathlineto{\pgfqpoint{5.805556in}{3.303639in}}%
\pgfpathlineto{\pgfqpoint{5.798646in}{3.297866in}}%
\pgfpathlineto{\pgfqpoint{5.791738in}{3.292286in}}%
\pgfpathlineto{\pgfqpoint{5.784831in}{3.286893in}}%
\pgfpathclose%
\pgfusepath{fill}%
\end{pgfscope}%
\begin{pgfscope}%
\pgfpathrectangle{\pgfqpoint{1.254980in}{0.150000in}}{\pgfqpoint{5.490039in}{5.490039in}}%
\pgfusepath{clip}%
\pgfsetbuttcap%
\pgfsetroundjoin%
\definecolor{currentfill}{rgb}{0.278791,0.062145,0.386592}%
\pgfsetfillcolor{currentfill}%
\pgfsetfillopacity{0.700000}%
\pgfsetlinewidth{0.000000pt}%
\definecolor{currentstroke}{rgb}{0.000000,0.000000,0.000000}%
\pgfsetstrokecolor{currentstroke}%
\pgfsetdash{}{0pt}%
\pgfpathmoveto{\pgfqpoint{3.322402in}{2.076409in}}%
\pgfpathlineto{\pgfqpoint{3.335447in}{2.069999in}}%
\pgfpathlineto{\pgfqpoint{3.348493in}{2.063787in}}%
\pgfpathlineto{\pgfqpoint{3.361542in}{2.057774in}}%
\pgfpathlineto{\pgfqpoint{3.374592in}{2.051958in}}%
\pgfpathlineto{\pgfqpoint{3.382444in}{2.060755in}}%
\pgfpathlineto{\pgfqpoint{3.390290in}{2.069596in}}%
\pgfpathlineto{\pgfqpoint{3.398129in}{2.078481in}}%
\pgfpathlineto{\pgfqpoint{3.405963in}{2.087408in}}%
\pgfpathlineto{\pgfqpoint{3.392927in}{2.093062in}}%
\pgfpathlineto{\pgfqpoint{3.379893in}{2.098913in}}%
\pgfpathlineto{\pgfqpoint{3.366861in}{2.104962in}}%
\pgfpathlineto{\pgfqpoint{3.353831in}{2.111210in}}%
\pgfpathlineto{\pgfqpoint{3.345983in}{2.102435in}}%
\pgfpathlineto{\pgfqpoint{3.338129in}{2.093708in}}%
\pgfpathlineto{\pgfqpoint{3.330269in}{2.085033in}}%
\pgfpathlineto{\pgfqpoint{3.322402in}{2.076409in}}%
\pgfpathclose%
\pgfusepath{fill}%
\end{pgfscope}%
\begin{pgfscope}%
\pgfpathrectangle{\pgfqpoint{1.254980in}{0.150000in}}{\pgfqpoint{5.490039in}{5.490039in}}%
\pgfusepath{clip}%
\pgfsetbuttcap%
\pgfsetroundjoin%
\definecolor{currentfill}{rgb}{0.283229,0.120777,0.440584}%
\pgfsetfillcolor{currentfill}%
\pgfsetfillopacity{0.700000}%
\pgfsetlinewidth{0.000000pt}%
\definecolor{currentstroke}{rgb}{0.000000,0.000000,0.000000}%
\pgfsetstrokecolor{currentstroke}%
\pgfsetdash{}{0pt}%
\pgfpathmoveto{\pgfqpoint{3.843435in}{2.165143in}}%
\pgfpathlineto{\pgfqpoint{3.856542in}{2.164496in}}%
\pgfpathlineto{\pgfqpoint{3.869657in}{2.164025in}}%
\pgfpathlineto{\pgfqpoint{3.882779in}{2.163730in}}%
\pgfpathlineto{\pgfqpoint{3.895907in}{2.163609in}}%
\pgfpathlineto{\pgfqpoint{3.903576in}{2.173371in}}%
\pgfpathlineto{\pgfqpoint{3.911240in}{2.183117in}}%
\pgfpathlineto{\pgfqpoint{3.918899in}{2.192848in}}%
\pgfpathlineto{\pgfqpoint{3.926553in}{2.202563in}}%
\pgfpathlineto{\pgfqpoint{3.913432in}{2.202662in}}%
\pgfpathlineto{\pgfqpoint{3.900319in}{2.202935in}}%
\pgfpathlineto{\pgfqpoint{3.887212in}{2.203384in}}%
\pgfpathlineto{\pgfqpoint{3.874113in}{2.204008in}}%
\pgfpathlineto{\pgfqpoint{3.866451in}{2.194305in}}%
\pgfpathlineto{\pgfqpoint{3.858784in}{2.184593in}}%
\pgfpathlineto{\pgfqpoint{3.851112in}{2.174872in}}%
\pgfpathlineto{\pgfqpoint{3.843435in}{2.165143in}}%
\pgfpathclose%
\pgfusepath{fill}%
\end{pgfscope}%
\begin{pgfscope}%
\pgfpathrectangle{\pgfqpoint{1.254980in}{0.150000in}}{\pgfqpoint{5.490039in}{5.490039in}}%
\pgfusepath{clip}%
\pgfsetbuttcap%
\pgfsetroundjoin%
\definecolor{currentfill}{rgb}{0.250425,0.274290,0.533103}%
\pgfsetfillcolor{currentfill}%
\pgfsetfillopacity{0.700000}%
\pgfsetlinewidth{0.000000pt}%
\definecolor{currentstroke}{rgb}{0.000000,0.000000,0.000000}%
\pgfsetstrokecolor{currentstroke}%
\pgfsetdash{}{0pt}%
\pgfpathmoveto{\pgfqpoint{4.425076in}{2.470945in}}%
\pgfpathlineto{\pgfqpoint{4.438379in}{2.474605in}}%
\pgfpathlineto{\pgfqpoint{4.451694in}{2.478430in}}%
\pgfpathlineto{\pgfqpoint{4.465020in}{2.482418in}}%
\pgfpathlineto{\pgfqpoint{4.478358in}{2.486568in}}%
\pgfpathlineto{\pgfqpoint{4.485828in}{2.495048in}}%
\pgfpathlineto{\pgfqpoint{4.493292in}{2.503495in}}%
\pgfpathlineto{\pgfqpoint{4.500751in}{2.511913in}}%
\pgfpathlineto{\pgfqpoint{4.508204in}{2.520303in}}%
\pgfpathlineto{\pgfqpoint{4.494876in}{2.516328in}}%
\pgfpathlineto{\pgfqpoint{4.481559in}{2.512516in}}%
\pgfpathlineto{\pgfqpoint{4.468253in}{2.508867in}}%
\pgfpathlineto{\pgfqpoint{4.454958in}{2.505382in}}%
\pgfpathlineto{\pgfqpoint{4.447495in}{2.496806in}}%
\pgfpathlineto{\pgfqpoint{4.440027in}{2.488209in}}%
\pgfpathlineto{\pgfqpoint{4.432554in}{2.479589in}}%
\pgfpathlineto{\pgfqpoint{4.425076in}{2.470945in}}%
\pgfpathclose%
\pgfusepath{fill}%
\end{pgfscope}%
\begin{pgfscope}%
\pgfpathrectangle{\pgfqpoint{1.254980in}{0.150000in}}{\pgfqpoint{5.490039in}{5.490039in}}%
\pgfusepath{clip}%
\pgfsetbuttcap%
\pgfsetroundjoin%
\definecolor{currentfill}{rgb}{0.206756,0.371758,0.553117}%
\pgfsetfillcolor{currentfill}%
\pgfsetfillopacity{0.700000}%
\pgfsetlinewidth{0.000000pt}%
\definecolor{currentstroke}{rgb}{0.000000,0.000000,0.000000}%
\pgfsetstrokecolor{currentstroke}%
\pgfsetdash{}{0pt}%
\pgfpathmoveto{\pgfqpoint{2.522703in}{2.754782in}}%
\pgfpathlineto{\pgfqpoint{2.536046in}{2.734228in}}%
\pgfpathlineto{\pgfqpoint{2.549378in}{2.713982in}}%
\pgfpathlineto{\pgfqpoint{2.562698in}{2.694040in}}%
\pgfpathlineto{\pgfqpoint{2.576007in}{2.674399in}}%
\pgfpathlineto{\pgfqpoint{2.584235in}{2.679497in}}%
\pgfpathlineto{\pgfqpoint{2.592452in}{2.684753in}}%
\pgfpathlineto{\pgfqpoint{2.600657in}{2.690164in}}%
\pgfpathlineto{\pgfqpoint{2.608851in}{2.695728in}}%
\pgfpathlineto{\pgfqpoint{2.595573in}{2.715134in}}%
\pgfpathlineto{\pgfqpoint{2.582283in}{2.734841in}}%
\pgfpathlineto{\pgfqpoint{2.568983in}{2.754851in}}%
\pgfpathlineto{\pgfqpoint{2.555671in}{2.775168in}}%
\pgfpathlineto{\pgfqpoint{2.547447in}{2.769829in}}%
\pgfpathlineto{\pgfqpoint{2.539211in}{2.764650in}}%
\pgfpathlineto{\pgfqpoint{2.530963in}{2.759633in}}%
\pgfpathlineto{\pgfqpoint{2.522703in}{2.754782in}}%
\pgfpathclose%
\pgfusepath{fill}%
\end{pgfscope}%
\begin{pgfscope}%
\pgfpathrectangle{\pgfqpoint{1.254980in}{0.150000in}}{\pgfqpoint{5.490039in}{5.490039in}}%
\pgfusepath{clip}%
\pgfsetbuttcap%
\pgfsetroundjoin%
\definecolor{currentfill}{rgb}{0.280894,0.078907,0.402329}%
\pgfsetfillcolor{currentfill}%
\pgfsetfillopacity{0.700000}%
\pgfsetlinewidth{0.000000pt}%
\definecolor{currentstroke}{rgb}{0.000000,0.000000,0.000000}%
\pgfsetstrokecolor{currentstroke}%
\pgfsetdash{}{0pt}%
\pgfpathmoveto{\pgfqpoint{3.186416in}{2.102576in}}%
\pgfpathlineto{\pgfqpoint{3.199472in}{2.094340in}}%
\pgfpathlineto{\pgfqpoint{3.212527in}{2.086312in}}%
\pgfpathlineto{\pgfqpoint{3.225583in}{2.078491in}}%
\pgfpathlineto{\pgfqpoint{3.238639in}{2.070877in}}%
\pgfpathlineto{\pgfqpoint{3.246548in}{2.079079in}}%
\pgfpathlineto{\pgfqpoint{3.254451in}{2.087345in}}%
\pgfpathlineto{\pgfqpoint{3.262346in}{2.095675in}}%
\pgfpathlineto{\pgfqpoint{3.270235in}{2.104065in}}%
\pgfpathlineto{\pgfqpoint{3.257196in}{2.111489in}}%
\pgfpathlineto{\pgfqpoint{3.244157in}{2.119119in}}%
\pgfpathlineto{\pgfqpoint{3.231119in}{2.126956in}}%
\pgfpathlineto{\pgfqpoint{3.218081in}{2.135001in}}%
\pgfpathlineto{\pgfqpoint{3.210176in}{2.126791in}}%
\pgfpathlineto{\pgfqpoint{3.202263in}{2.118649in}}%
\pgfpathlineto{\pgfqpoint{3.194343in}{2.110577in}}%
\pgfpathlineto{\pgfqpoint{3.186416in}{2.102576in}}%
\pgfpathclose%
\pgfusepath{fill}%
\end{pgfscope}%
\begin{pgfscope}%
\pgfpathrectangle{\pgfqpoint{1.254980in}{0.150000in}}{\pgfqpoint{5.490039in}{5.490039in}}%
\pgfusepath{clip}%
\pgfsetbuttcap%
\pgfsetroundjoin%
\definecolor{currentfill}{rgb}{0.241237,0.296485,0.539709}%
\pgfsetfillcolor{currentfill}%
\pgfsetfillopacity{0.700000}%
\pgfsetlinewidth{0.000000pt}%
\definecolor{currentstroke}{rgb}{0.000000,0.000000,0.000000}%
\pgfsetstrokecolor{currentstroke}%
\pgfsetdash{}{0pt}%
\pgfpathmoveto{\pgfqpoint{4.508204in}{2.520303in}}%
\pgfpathlineto{\pgfqpoint{4.521544in}{2.524440in}}%
\pgfpathlineto{\pgfqpoint{4.534896in}{2.528739in}}%
\pgfpathlineto{\pgfqpoint{4.548259in}{2.533201in}}%
\pgfpathlineto{\pgfqpoint{4.561634in}{2.537824in}}%
\pgfpathlineto{\pgfqpoint{4.569073in}{2.545995in}}%
\pgfpathlineto{\pgfqpoint{4.576506in}{2.554135in}}%
\pgfpathlineto{\pgfqpoint{4.583934in}{2.562248in}}%
\pgfpathlineto{\pgfqpoint{4.591356in}{2.570337in}}%
\pgfpathlineto{\pgfqpoint{4.577990in}{2.565918in}}%
\pgfpathlineto{\pgfqpoint{4.564637in}{2.561661in}}%
\pgfpathlineto{\pgfqpoint{4.551295in}{2.557566in}}%
\pgfpathlineto{\pgfqpoint{4.537964in}{2.553633in}}%
\pgfpathlineto{\pgfqpoint{4.530532in}{2.545330in}}%
\pgfpathlineto{\pgfqpoint{4.523095in}{2.537009in}}%
\pgfpathlineto{\pgfqpoint{4.515652in}{2.528667in}}%
\pgfpathlineto{\pgfqpoint{4.508204in}{2.520303in}}%
\pgfpathclose%
\pgfusepath{fill}%
\end{pgfscope}%
\begin{pgfscope}%
\pgfpathrectangle{\pgfqpoint{1.254980in}{0.150000in}}{\pgfqpoint{5.490039in}{5.490039in}}%
\pgfusepath{clip}%
\pgfsetbuttcap%
\pgfsetroundjoin%
\definecolor{currentfill}{rgb}{0.283229,0.120777,0.440584}%
\pgfsetfillcolor{currentfill}%
\pgfsetfillopacity{0.700000}%
\pgfsetlinewidth{0.000000pt}%
\definecolor{currentstroke}{rgb}{0.000000,0.000000,0.000000}%
\pgfsetstrokecolor{currentstroke}%
\pgfsetdash{}{0pt}%
\pgfpathmoveto{\pgfqpoint{2.997664in}{2.189584in}}%
\pgfpathlineto{\pgfqpoint{3.010756in}{2.178514in}}%
\pgfpathlineto{\pgfqpoint{3.023845in}{2.167669in}}%
\pgfpathlineto{\pgfqpoint{3.036931in}{2.157049in}}%
\pgfpathlineto{\pgfqpoint{3.050016in}{2.146652in}}%
\pgfpathlineto{\pgfqpoint{3.058011in}{2.153888in}}%
\pgfpathlineto{\pgfqpoint{3.065999in}{2.161217in}}%
\pgfpathlineto{\pgfqpoint{3.073979in}{2.168636in}}%
\pgfpathlineto{\pgfqpoint{3.081951in}{2.176144in}}%
\pgfpathlineto{\pgfqpoint{3.068887in}{2.186320in}}%
\pgfpathlineto{\pgfqpoint{3.055821in}{2.196719in}}%
\pgfpathlineto{\pgfqpoint{3.042753in}{2.207343in}}%
\pgfpathlineto{\pgfqpoint{3.029683in}{2.218191in}}%
\pgfpathlineto{\pgfqpoint{3.021691in}{2.210894in}}%
\pgfpathlineto{\pgfqpoint{3.013690in}{2.203692in}}%
\pgfpathlineto{\pgfqpoint{3.005681in}{2.196588in}}%
\pgfpathlineto{\pgfqpoint{2.997664in}{2.189584in}}%
\pgfpathclose%
\pgfusepath{fill}%
\end{pgfscope}%
\begin{pgfscope}%
\pgfpathrectangle{\pgfqpoint{1.254980in}{0.150000in}}{\pgfqpoint{5.490039in}{5.490039in}}%
\pgfusepath{clip}%
\pgfsetbuttcap%
\pgfsetroundjoin%
\definecolor{currentfill}{rgb}{0.282656,0.100196,0.422160}%
\pgfsetfillcolor{currentfill}%
\pgfsetfillopacity{0.700000}%
\pgfsetlinewidth{0.000000pt}%
\definecolor{currentstroke}{rgb}{0.000000,0.000000,0.000000}%
\pgfsetstrokecolor{currentstroke}%
\pgfsetdash{}{0pt}%
\pgfpathmoveto{\pgfqpoint{3.760275in}{2.130700in}}%
\pgfpathlineto{\pgfqpoint{3.773365in}{2.129292in}}%
\pgfpathlineto{\pgfqpoint{3.786462in}{2.128063in}}%
\pgfpathlineto{\pgfqpoint{3.799566in}{2.127012in}}%
\pgfpathlineto{\pgfqpoint{3.812676in}{2.126138in}}%
\pgfpathlineto{\pgfqpoint{3.820373in}{2.135903in}}%
\pgfpathlineto{\pgfqpoint{3.828065in}{2.145658in}}%
\pgfpathlineto{\pgfqpoint{3.835752in}{2.155405in}}%
\pgfpathlineto{\pgfqpoint{3.843435in}{2.165143in}}%
\pgfpathlineto{\pgfqpoint{3.830333in}{2.165967in}}%
\pgfpathlineto{\pgfqpoint{3.817239in}{2.166968in}}%
\pgfpathlineto{\pgfqpoint{3.804151in}{2.168147in}}%
\pgfpathlineto{\pgfqpoint{3.791069in}{2.169504in}}%
\pgfpathlineto{\pgfqpoint{3.783378in}{2.159806in}}%
\pgfpathlineto{\pgfqpoint{3.775682in}{2.150106in}}%
\pgfpathlineto{\pgfqpoint{3.767981in}{2.140404in}}%
\pgfpathlineto{\pgfqpoint{3.760275in}{2.130700in}}%
\pgfpathclose%
\pgfusepath{fill}%
\end{pgfscope}%
\begin{pgfscope}%
\pgfpathrectangle{\pgfqpoint{1.254980in}{0.150000in}}{\pgfqpoint{5.490039in}{5.490039in}}%
\pgfusepath{clip}%
\pgfsetbuttcap%
\pgfsetroundjoin%
\definecolor{currentfill}{rgb}{0.278791,0.062145,0.386592}%
\pgfsetfillcolor{currentfill}%
\pgfsetfillopacity{0.700000}%
\pgfsetlinewidth{0.000000pt}%
\definecolor{currentstroke}{rgb}{0.000000,0.000000,0.000000}%
\pgfsetstrokecolor{currentstroke}%
\pgfsetdash{}{0pt}%
\pgfpathmoveto{\pgfqpoint{3.458133in}{2.066742in}}%
\pgfpathlineto{\pgfqpoint{3.471182in}{2.062058in}}%
\pgfpathlineto{\pgfqpoint{3.484235in}{2.057565in}}%
\pgfpathlineto{\pgfqpoint{3.497291in}{2.053263in}}%
\pgfpathlineto{\pgfqpoint{3.510351in}{2.049150in}}%
\pgfpathlineto{\pgfqpoint{3.518153in}{2.058407in}}%
\pgfpathlineto{\pgfqpoint{3.525949in}{2.067690in}}%
\pgfpathlineto{\pgfqpoint{3.533740in}{2.076998in}}%
\pgfpathlineto{\pgfqpoint{3.541525in}{2.086330in}}%
\pgfpathlineto{\pgfqpoint{3.528477in}{2.090310in}}%
\pgfpathlineto{\pgfqpoint{3.515433in}{2.094478in}}%
\pgfpathlineto{\pgfqpoint{3.502393in}{2.098837in}}%
\pgfpathlineto{\pgfqpoint{3.489356in}{2.103387in}}%
\pgfpathlineto{\pgfqpoint{3.481559in}{2.094178in}}%
\pgfpathlineto{\pgfqpoint{3.473756in}{2.085000in}}%
\pgfpathlineto{\pgfqpoint{3.465947in}{2.075854in}}%
\pgfpathlineto{\pgfqpoint{3.458133in}{2.066742in}}%
\pgfpathclose%
\pgfusepath{fill}%
\end{pgfscope}%
\begin{pgfscope}%
\pgfpathrectangle{\pgfqpoint{1.254980in}{0.150000in}}{\pgfqpoint{5.490039in}{5.490039in}}%
\pgfusepath{clip}%
\pgfsetbuttcap%
\pgfsetroundjoin%
\definecolor{currentfill}{rgb}{0.231674,0.318106,0.544834}%
\pgfsetfillcolor{currentfill}%
\pgfsetfillopacity{0.700000}%
\pgfsetlinewidth{0.000000pt}%
\definecolor{currentstroke}{rgb}{0.000000,0.000000,0.000000}%
\pgfsetstrokecolor{currentstroke}%
\pgfsetdash{}{0pt}%
\pgfpathmoveto{\pgfqpoint{4.591356in}{2.570337in}}%
\pgfpathlineto{\pgfqpoint{4.604733in}{2.574916in}}%
\pgfpathlineto{\pgfqpoint{4.618123in}{2.579657in}}%
\pgfpathlineto{\pgfqpoint{4.631524in}{2.584559in}}%
\pgfpathlineto{\pgfqpoint{4.644938in}{2.589622in}}%
\pgfpathlineto{\pgfqpoint{4.652345in}{2.597465in}}%
\pgfpathlineto{\pgfqpoint{4.659746in}{2.605282in}}%
\pgfpathlineto{\pgfqpoint{4.667141in}{2.613076in}}%
\pgfpathlineto{\pgfqpoint{4.674531in}{2.620848in}}%
\pgfpathlineto{\pgfqpoint{4.661127in}{2.616020in}}%
\pgfpathlineto{\pgfqpoint{4.647736in}{2.611351in}}%
\pgfpathlineto{\pgfqpoint{4.634357in}{2.606843in}}%
\pgfpathlineto{\pgfqpoint{4.620990in}{2.602496in}}%
\pgfpathlineto{\pgfqpoint{4.613589in}{2.594480in}}%
\pgfpathlineto{\pgfqpoint{4.606184in}{2.586450in}}%
\pgfpathlineto{\pgfqpoint{4.598772in}{2.578403in}}%
\pgfpathlineto{\pgfqpoint{4.591356in}{2.570337in}}%
\pgfpathclose%
\pgfusepath{fill}%
\end{pgfscope}%
\begin{pgfscope}%
\pgfpathrectangle{\pgfqpoint{1.254980in}{0.150000in}}{\pgfqpoint{5.490039in}{5.490039in}}%
\pgfusepath{clip}%
\pgfsetbuttcap%
\pgfsetroundjoin%
\definecolor{currentfill}{rgb}{0.119483,0.614817,0.537692}%
\pgfsetfillcolor{currentfill}%
\pgfsetfillopacity{0.700000}%
\pgfsetlinewidth{0.000000pt}%
\definecolor{currentstroke}{rgb}{0.000000,0.000000,0.000000}%
\pgfsetstrokecolor{currentstroke}%
\pgfsetdash{}{0pt}%
\pgfpathmoveto{\pgfqpoint{5.868028in}{3.334662in}}%
\pgfpathlineto{\pgfqpoint{5.881960in}{3.341295in}}%
\pgfpathlineto{\pgfqpoint{5.895909in}{3.348077in}}%
\pgfpathlineto{\pgfqpoint{5.909874in}{3.355007in}}%
\pgfpathlineto{\pgfqpoint{5.923857in}{3.362086in}}%
\pgfpathlineto{\pgfqpoint{5.930700in}{3.366902in}}%
\pgfpathlineto{\pgfqpoint{5.937545in}{3.371920in}}%
\pgfpathlineto{\pgfqpoint{5.944392in}{3.377149in}}%
\pgfpathlineto{\pgfqpoint{5.930436in}{3.370590in}}%
\pgfpathlineto{\pgfqpoint{5.916498in}{3.364178in}}%
\pgfpathlineto{\pgfqpoint{5.902575in}{3.357913in}}%
\pgfpathlineto{\pgfqpoint{5.888670in}{3.351797in}}%
\pgfpathlineto{\pgfqpoint{5.881787in}{3.345871in}}%
\pgfpathlineto{\pgfqpoint{5.874907in}{3.340162in}}%
\pgfpathlineto{\pgfqpoint{5.868028in}{3.334662in}}%
\pgfpathclose%
\pgfusepath{fill}%
\end{pgfscope}%
\begin{pgfscope}%
\pgfpathrectangle{\pgfqpoint{1.254980in}{0.150000in}}{\pgfqpoint{5.490039in}{5.490039in}}%
\pgfusepath{clip}%
\pgfsetbuttcap%
\pgfsetroundjoin%
\definecolor{currentfill}{rgb}{0.220057,0.343307,0.549413}%
\pgfsetfillcolor{currentfill}%
\pgfsetfillopacity{0.700000}%
\pgfsetlinewidth{0.000000pt}%
\definecolor{currentstroke}{rgb}{0.000000,0.000000,0.000000}%
\pgfsetstrokecolor{currentstroke}%
\pgfsetdash{}{0pt}%
\pgfpathmoveto{\pgfqpoint{4.674531in}{2.620848in}}%
\pgfpathlineto{\pgfqpoint{4.687947in}{2.625837in}}%
\pgfpathlineto{\pgfqpoint{4.701375in}{2.630986in}}%
\pgfpathlineto{\pgfqpoint{4.714816in}{2.636295in}}%
\pgfpathlineto{\pgfqpoint{4.728270in}{2.641764in}}%
\pgfpathlineto{\pgfqpoint{4.735643in}{2.649267in}}%
\pgfpathlineto{\pgfqpoint{4.743011in}{2.656749in}}%
\pgfpathlineto{\pgfqpoint{4.750373in}{2.664212in}}%
\pgfpathlineto{\pgfqpoint{4.757729in}{2.671660in}}%
\pgfpathlineto{\pgfqpoint{4.744286in}{2.666454in}}%
\pgfpathlineto{\pgfqpoint{4.730857in}{2.661408in}}%
\pgfpathlineto{\pgfqpoint{4.717440in}{2.656521in}}%
\pgfpathlineto{\pgfqpoint{4.704035in}{2.651793in}}%
\pgfpathlineto{\pgfqpoint{4.696667in}{2.644073in}}%
\pgfpathlineto{\pgfqpoint{4.689294in}{2.636344in}}%
\pgfpathlineto{\pgfqpoint{4.681915in}{2.628604in}}%
\pgfpathlineto{\pgfqpoint{4.674531in}{2.620848in}}%
\pgfpathclose%
\pgfusepath{fill}%
\end{pgfscope}%
\begin{pgfscope}%
\pgfpathrectangle{\pgfqpoint{1.254980in}{0.150000in}}{\pgfqpoint{5.490039in}{5.490039in}}%
\pgfusepath{clip}%
\pgfsetbuttcap%
\pgfsetroundjoin%
\definecolor{currentfill}{rgb}{0.281446,0.084320,0.407414}%
\pgfsetfillcolor{currentfill}%
\pgfsetfillopacity{0.700000}%
\pgfsetlinewidth{0.000000pt}%
\definecolor{currentstroke}{rgb}{0.000000,0.000000,0.000000}%
\pgfsetstrokecolor{currentstroke}%
\pgfsetdash{}{0pt}%
\pgfpathmoveto{\pgfqpoint{3.677055in}{2.099614in}}%
\pgfpathlineto{\pgfqpoint{3.690132in}{2.097408in}}%
\pgfpathlineto{\pgfqpoint{3.703216in}{2.095382in}}%
\pgfpathlineto{\pgfqpoint{3.716304in}{2.093537in}}%
\pgfpathlineto{\pgfqpoint{3.729398in}{2.091873in}}%
\pgfpathlineto{\pgfqpoint{3.737125in}{2.101581in}}%
\pgfpathlineto{\pgfqpoint{3.744847in}{2.111288in}}%
\pgfpathlineto{\pgfqpoint{3.752563in}{2.120995in}}%
\pgfpathlineto{\pgfqpoint{3.760275in}{2.130700in}}%
\pgfpathlineto{\pgfqpoint{3.747190in}{2.132287in}}%
\pgfpathlineto{\pgfqpoint{3.734111in}{2.134054in}}%
\pgfpathlineto{\pgfqpoint{3.721037in}{2.136001in}}%
\pgfpathlineto{\pgfqpoint{3.707969in}{2.138130in}}%
\pgfpathlineto{\pgfqpoint{3.700248in}{2.128492in}}%
\pgfpathlineto{\pgfqpoint{3.692522in}{2.118860in}}%
\pgfpathlineto{\pgfqpoint{3.684791in}{2.109234in}}%
\pgfpathlineto{\pgfqpoint{3.677055in}{2.099614in}}%
\pgfpathclose%
\pgfusepath{fill}%
\end{pgfscope}%
\begin{pgfscope}%
\pgfpathrectangle{\pgfqpoint{1.254980in}{0.150000in}}{\pgfqpoint{5.490039in}{5.490039in}}%
\pgfusepath{clip}%
\pgfsetbuttcap%
\pgfsetroundjoin%
\definecolor{currentfill}{rgb}{0.210503,0.363727,0.552206}%
\pgfsetfillcolor{currentfill}%
\pgfsetfillopacity{0.700000}%
\pgfsetlinewidth{0.000000pt}%
\definecolor{currentstroke}{rgb}{0.000000,0.000000,0.000000}%
\pgfsetstrokecolor{currentstroke}%
\pgfsetdash{}{0pt}%
\pgfpathmoveto{\pgfqpoint{4.757729in}{2.671660in}}%
\pgfpathlineto{\pgfqpoint{4.771184in}{2.677025in}}%
\pgfpathlineto{\pgfqpoint{4.784653in}{2.682549in}}%
\pgfpathlineto{\pgfqpoint{4.798134in}{2.688232in}}%
\pgfpathlineto{\pgfqpoint{4.811629in}{2.694073in}}%
\pgfpathlineto{\pgfqpoint{4.818967in}{2.701228in}}%
\pgfpathlineto{\pgfqpoint{4.826300in}{2.708367in}}%
\pgfpathlineto{\pgfqpoint{4.833628in}{2.715494in}}%
\pgfpathlineto{\pgfqpoint{4.840949in}{2.722612in}}%
\pgfpathlineto{\pgfqpoint{4.827467in}{2.717063in}}%
\pgfpathlineto{\pgfqpoint{4.813998in}{2.711671in}}%
\pgfpathlineto{\pgfqpoint{4.800542in}{2.706439in}}%
\pgfpathlineto{\pgfqpoint{4.787099in}{2.701364in}}%
\pgfpathlineto{\pgfqpoint{4.779765in}{2.693945in}}%
\pgfpathlineto{\pgfqpoint{4.772425in}{2.686523in}}%
\pgfpathlineto{\pgfqpoint{4.765080in}{2.679096in}}%
\pgfpathlineto{\pgfqpoint{4.757729in}{2.671660in}}%
\pgfpathclose%
\pgfusepath{fill}%
\end{pgfscope}%
\begin{pgfscope}%
\pgfpathrectangle{\pgfqpoint{1.254980in}{0.150000in}}{\pgfqpoint{5.490039in}{5.490039in}}%
\pgfusepath{clip}%
\pgfsetbuttcap%
\pgfsetroundjoin%
\definecolor{currentfill}{rgb}{0.282656,0.100196,0.422160}%
\pgfsetfillcolor{currentfill}%
\pgfsetfillopacity{0.700000}%
\pgfsetlinewidth{0.000000pt}%
\definecolor{currentstroke}{rgb}{0.000000,0.000000,0.000000}%
\pgfsetstrokecolor{currentstroke}%
\pgfsetdash{}{0pt}%
\pgfpathmoveto{\pgfqpoint{3.050016in}{2.146652in}}%
\pgfpathlineto{\pgfqpoint{3.063098in}{2.136477in}}%
\pgfpathlineto{\pgfqpoint{3.076179in}{2.126521in}}%
\pgfpathlineto{\pgfqpoint{3.089258in}{2.116784in}}%
\pgfpathlineto{\pgfqpoint{3.102336in}{2.107264in}}%
\pgfpathlineto{\pgfqpoint{3.110311in}{2.114730in}}%
\pgfpathlineto{\pgfqpoint{3.118279in}{2.122283in}}%
\pgfpathlineto{\pgfqpoint{3.126239in}{2.129919in}}%
\pgfpathlineto{\pgfqpoint{3.134191in}{2.137637in}}%
\pgfpathlineto{\pgfqpoint{3.121133in}{2.146937in}}%
\pgfpathlineto{\pgfqpoint{3.108074in}{2.156454in}}%
\pgfpathlineto{\pgfqpoint{3.095013in}{2.166189in}}%
\pgfpathlineto{\pgfqpoint{3.081951in}{2.176144in}}%
\pgfpathlineto{\pgfqpoint{3.073979in}{2.168636in}}%
\pgfpathlineto{\pgfqpoint{3.065999in}{2.161217in}}%
\pgfpathlineto{\pgfqpoint{3.058011in}{2.153888in}}%
\pgfpathlineto{\pgfqpoint{3.050016in}{2.146652in}}%
\pgfpathclose%
\pgfusepath{fill}%
\end{pgfscope}%
\begin{pgfscope}%
\pgfpathrectangle{\pgfqpoint{1.254980in}{0.150000in}}{\pgfqpoint{5.490039in}{5.490039in}}%
\pgfusepath{clip}%
\pgfsetbuttcap%
\pgfsetroundjoin%
\definecolor{currentfill}{rgb}{0.199430,0.387607,0.554642}%
\pgfsetfillcolor{currentfill}%
\pgfsetfillopacity{0.700000}%
\pgfsetlinewidth{0.000000pt}%
\definecolor{currentstroke}{rgb}{0.000000,0.000000,0.000000}%
\pgfsetstrokecolor{currentstroke}%
\pgfsetdash{}{0pt}%
\pgfpathmoveto{\pgfqpoint{4.840949in}{2.722612in}}%
\pgfpathlineto{\pgfqpoint{4.854445in}{2.728320in}}%
\pgfpathlineto{\pgfqpoint{4.867954in}{2.734186in}}%
\pgfpathlineto{\pgfqpoint{4.881476in}{2.740210in}}%
\pgfpathlineto{\pgfqpoint{4.895012in}{2.746392in}}%
\pgfpathlineto{\pgfqpoint{4.902315in}{2.753195in}}%
\pgfpathlineto{\pgfqpoint{4.909613in}{2.759989in}}%
\pgfpathlineto{\pgfqpoint{4.916904in}{2.766779in}}%
\pgfpathlineto{\pgfqpoint{4.924191in}{2.773567in}}%
\pgfpathlineto{\pgfqpoint{4.910668in}{2.767707in}}%
\pgfpathlineto{\pgfqpoint{4.897160in}{2.762003in}}%
\pgfpathlineto{\pgfqpoint{4.883664in}{2.756458in}}%
\pgfpathlineto{\pgfqpoint{4.870182in}{2.751070in}}%
\pgfpathlineto{\pgfqpoint{4.862882in}{2.743950in}}%
\pgfpathlineto{\pgfqpoint{4.855577in}{2.736837in}}%
\pgfpathlineto{\pgfqpoint{4.848266in}{2.729725in}}%
\pgfpathlineto{\pgfqpoint{4.840949in}{2.722612in}}%
\pgfpathclose%
\pgfusepath{fill}%
\end{pgfscope}%
\begin{pgfscope}%
\pgfpathrectangle{\pgfqpoint{1.254980in}{0.150000in}}{\pgfqpoint{5.490039in}{5.490039in}}%
\pgfusepath{clip}%
\pgfsetbuttcap%
\pgfsetroundjoin%
\definecolor{currentfill}{rgb}{0.279566,0.067836,0.391917}%
\pgfsetfillcolor{currentfill}%
\pgfsetfillopacity{0.700000}%
\pgfsetlinewidth{0.000000pt}%
\definecolor{currentstroke}{rgb}{0.000000,0.000000,0.000000}%
\pgfsetstrokecolor{currentstroke}%
\pgfsetdash{}{0pt}%
\pgfpathmoveto{\pgfqpoint{3.238639in}{2.070877in}}%
\pgfpathlineto{\pgfqpoint{3.251695in}{2.063468in}}%
\pgfpathlineto{\pgfqpoint{3.264753in}{2.056263in}}%
\pgfpathlineto{\pgfqpoint{3.277811in}{2.049261in}}%
\pgfpathlineto{\pgfqpoint{3.290870in}{2.042460in}}%
\pgfpathlineto{\pgfqpoint{3.298763in}{2.050862in}}%
\pgfpathlineto{\pgfqpoint{3.306649in}{2.059322in}}%
\pgfpathlineto{\pgfqpoint{3.314529in}{2.067838in}}%
\pgfpathlineto{\pgfqpoint{3.322402in}{2.076409in}}%
\pgfpathlineto{\pgfqpoint{3.309358in}{2.083020in}}%
\pgfpathlineto{\pgfqpoint{3.296316in}{2.089832in}}%
\pgfpathlineto{\pgfqpoint{3.283275in}{2.096847in}}%
\pgfpathlineto{\pgfqpoint{3.270235in}{2.104065in}}%
\pgfpathlineto{\pgfqpoint{3.262346in}{2.095675in}}%
\pgfpathlineto{\pgfqpoint{3.254451in}{2.087345in}}%
\pgfpathlineto{\pgfqpoint{3.246548in}{2.079079in}}%
\pgfpathlineto{\pgfqpoint{3.238639in}{2.070877in}}%
\pgfpathclose%
\pgfusepath{fill}%
\end{pgfscope}%
\begin{pgfscope}%
\pgfpathrectangle{\pgfqpoint{1.254980in}{0.150000in}}{\pgfqpoint{5.490039in}{5.490039in}}%
\pgfusepath{clip}%
\pgfsetbuttcap%
\pgfsetroundjoin%
\definecolor{currentfill}{rgb}{0.280267,0.073417,0.397163}%
\pgfsetfillcolor{currentfill}%
\pgfsetfillopacity{0.700000}%
\pgfsetlinewidth{0.000000pt}%
\definecolor{currentstroke}{rgb}{0.000000,0.000000,0.000000}%
\pgfsetstrokecolor{currentstroke}%
\pgfsetdash{}{0pt}%
\pgfpathmoveto{\pgfqpoint{3.593755in}{2.072289in}}%
\pgfpathlineto{\pgfqpoint{3.606823in}{2.069244in}}%
\pgfpathlineto{\pgfqpoint{3.619896in}{2.066383in}}%
\pgfpathlineto{\pgfqpoint{3.632974in}{2.063706in}}%
\pgfpathlineto{\pgfqpoint{3.646057in}{2.061211in}}%
\pgfpathlineto{\pgfqpoint{3.653814in}{2.070800in}}%
\pgfpathlineto{\pgfqpoint{3.661566in}{2.080397in}}%
\pgfpathlineto{\pgfqpoint{3.669313in}{2.090002in}}%
\pgfpathlineto{\pgfqpoint{3.677055in}{2.099614in}}%
\pgfpathlineto{\pgfqpoint{3.663982in}{2.102003in}}%
\pgfpathlineto{\pgfqpoint{3.650915in}{2.104575in}}%
\pgfpathlineto{\pgfqpoint{3.637853in}{2.107330in}}%
\pgfpathlineto{\pgfqpoint{3.624795in}{2.110270in}}%
\pgfpathlineto{\pgfqpoint{3.617043in}{2.100753in}}%
\pgfpathlineto{\pgfqpoint{3.609286in}{2.091250in}}%
\pgfpathlineto{\pgfqpoint{3.601523in}{2.081762in}}%
\pgfpathlineto{\pgfqpoint{3.593755in}{2.072289in}}%
\pgfpathclose%
\pgfusepath{fill}%
\end{pgfscope}%
\begin{pgfscope}%
\pgfpathrectangle{\pgfqpoint{1.254980in}{0.150000in}}{\pgfqpoint{5.490039in}{5.490039in}}%
\pgfusepath{clip}%
\pgfsetbuttcap%
\pgfsetroundjoin%
\definecolor{currentfill}{rgb}{0.190631,0.407061,0.556089}%
\pgfsetfillcolor{currentfill}%
\pgfsetfillopacity{0.700000}%
\pgfsetlinewidth{0.000000pt}%
\definecolor{currentstroke}{rgb}{0.000000,0.000000,0.000000}%
\pgfsetstrokecolor{currentstroke}%
\pgfsetdash{}{0pt}%
\pgfpathmoveto{\pgfqpoint{4.924191in}{2.773567in}}%
\pgfpathlineto{\pgfqpoint{4.937727in}{2.779585in}}%
\pgfpathlineto{\pgfqpoint{4.951277in}{2.785760in}}%
\pgfpathlineto{\pgfqpoint{4.964841in}{2.792092in}}%
\pgfpathlineto{\pgfqpoint{4.978419in}{2.798582in}}%
\pgfpathlineto{\pgfqpoint{4.985685in}{2.805034in}}%
\pgfpathlineto{\pgfqpoint{4.992945in}{2.811486in}}%
\pgfpathlineto{\pgfqpoint{5.000201in}{2.817942in}}%
\pgfpathlineto{\pgfqpoint{5.007450in}{2.824406in}}%
\pgfpathlineto{\pgfqpoint{4.993887in}{2.818267in}}%
\pgfpathlineto{\pgfqpoint{4.980339in}{2.812284in}}%
\pgfpathlineto{\pgfqpoint{4.966803in}{2.806459in}}%
\pgfpathlineto{\pgfqpoint{4.953282in}{2.800790in}}%
\pgfpathlineto{\pgfqpoint{4.946017in}{2.793966in}}%
\pgfpathlineto{\pgfqpoint{4.938747in}{2.787157in}}%
\pgfpathlineto{\pgfqpoint{4.931471in}{2.780359in}}%
\pgfpathlineto{\pgfqpoint{4.924191in}{2.773567in}}%
\pgfpathclose%
\pgfusepath{fill}%
\end{pgfscope}%
\begin{pgfscope}%
\pgfpathrectangle{\pgfqpoint{1.254980in}{0.150000in}}{\pgfqpoint{5.490039in}{5.490039in}}%
\pgfusepath{clip}%
\pgfsetbuttcap%
\pgfsetroundjoin%
\definecolor{currentfill}{rgb}{0.277941,0.056324,0.381191}%
\pgfsetfillcolor{currentfill}%
\pgfsetfillopacity{0.700000}%
\pgfsetlinewidth{0.000000pt}%
\definecolor{currentstroke}{rgb}{0.000000,0.000000,0.000000}%
\pgfsetstrokecolor{currentstroke}%
\pgfsetdash{}{0pt}%
\pgfpathmoveto{\pgfqpoint{3.374592in}{2.051958in}}%
\pgfpathlineto{\pgfqpoint{3.387644in}{2.046339in}}%
\pgfpathlineto{\pgfqpoint{3.400699in}{2.040914in}}%
\pgfpathlineto{\pgfqpoint{3.413757in}{2.035683in}}%
\pgfpathlineto{\pgfqpoint{3.426817in}{2.030646in}}%
\pgfpathlineto{\pgfqpoint{3.434655in}{2.039614in}}%
\pgfpathlineto{\pgfqpoint{3.442487in}{2.048621in}}%
\pgfpathlineto{\pgfqpoint{3.450313in}{2.057664in}}%
\pgfpathlineto{\pgfqpoint{3.458133in}{2.066742in}}%
\pgfpathlineto{\pgfqpoint{3.445086in}{2.071618in}}%
\pgfpathlineto{\pgfqpoint{3.432043in}{2.076687in}}%
\pgfpathlineto{\pgfqpoint{3.419002in}{2.081950in}}%
\pgfpathlineto{\pgfqpoint{3.405963in}{2.087408in}}%
\pgfpathlineto{\pgfqpoint{3.398129in}{2.078481in}}%
\pgfpathlineto{\pgfqpoint{3.390290in}{2.069596in}}%
\pgfpathlineto{\pgfqpoint{3.382444in}{2.060755in}}%
\pgfpathlineto{\pgfqpoint{3.374592in}{2.051958in}}%
\pgfpathclose%
\pgfusepath{fill}%
\end{pgfscope}%
\begin{pgfscope}%
\pgfpathrectangle{\pgfqpoint{1.254980in}{0.150000in}}{\pgfqpoint{5.490039in}{5.490039in}}%
\pgfusepath{clip}%
\pgfsetbuttcap%
\pgfsetroundjoin%
\definecolor{currentfill}{rgb}{0.180629,0.429975,0.557282}%
\pgfsetfillcolor{currentfill}%
\pgfsetfillopacity{0.700000}%
\pgfsetlinewidth{0.000000pt}%
\definecolor{currentstroke}{rgb}{0.000000,0.000000,0.000000}%
\pgfsetstrokecolor{currentstroke}%
\pgfsetdash{}{0pt}%
\pgfpathmoveto{\pgfqpoint{5.007450in}{2.824406in}}%
\pgfpathlineto{\pgfqpoint{5.021027in}{2.830701in}}%
\pgfpathlineto{\pgfqpoint{5.034619in}{2.837152in}}%
\pgfpathlineto{\pgfqpoint{5.048224in}{2.843760in}}%
\pgfpathlineto{\pgfqpoint{5.061844in}{2.850524in}}%
\pgfpathlineto{\pgfqpoint{5.069073in}{2.856632in}}%
\pgfpathlineto{\pgfqpoint{5.076296in}{2.862749in}}%
\pgfpathlineto{\pgfqpoint{5.083513in}{2.868879in}}%
\pgfpathlineto{\pgfqpoint{5.090726in}{2.875028in}}%
\pgfpathlineto{\pgfqpoint{5.077122in}{2.868644in}}%
\pgfpathlineto{\pgfqpoint{5.063533in}{2.862415in}}%
\pgfpathlineto{\pgfqpoint{5.049958in}{2.856342in}}%
\pgfpathlineto{\pgfqpoint{5.036397in}{2.850425in}}%
\pgfpathlineto{\pgfqpoint{5.029168in}{2.843887in}}%
\pgfpathlineto{\pgfqpoint{5.021934in}{2.837374in}}%
\pgfpathlineto{\pgfqpoint{5.014695in}{2.830882in}}%
\pgfpathlineto{\pgfqpoint{5.007450in}{2.824406in}}%
\pgfpathclose%
\pgfusepath{fill}%
\end{pgfscope}%
\begin{pgfscope}%
\pgfpathrectangle{\pgfqpoint{1.254980in}{0.150000in}}{\pgfqpoint{5.490039in}{5.490039in}}%
\pgfusepath{clip}%
\pgfsetbuttcap%
\pgfsetroundjoin%
\definecolor{currentfill}{rgb}{0.262138,0.242286,0.520837}%
\pgfsetfillcolor{currentfill}%
\pgfsetfillopacity{0.700000}%
\pgfsetlinewidth{0.000000pt}%
\definecolor{currentstroke}{rgb}{0.000000,0.000000,0.000000}%
\pgfsetstrokecolor{currentstroke}%
\pgfsetdash{}{0pt}%
\pgfpathmoveto{\pgfqpoint{2.702285in}{2.439281in}}%
\pgfpathlineto{\pgfqpoint{2.715503in}{2.423050in}}%
\pgfpathlineto{\pgfqpoint{2.728714in}{2.407084in}}%
\pgfpathlineto{\pgfqpoint{2.741918in}{2.391381in}}%
\pgfpathlineto{\pgfqpoint{2.755115in}{2.375938in}}%
\pgfpathlineto{\pgfqpoint{2.763269in}{2.381469in}}%
\pgfpathlineto{\pgfqpoint{2.771413in}{2.387136in}}%
\pgfpathlineto{\pgfqpoint{2.779546in}{2.392938in}}%
\pgfpathlineto{\pgfqpoint{2.787669in}{2.398872in}}%
\pgfpathlineto{\pgfqpoint{2.774500in}{2.414058in}}%
\pgfpathlineto{\pgfqpoint{2.761324in}{2.429505in}}%
\pgfpathlineto{\pgfqpoint{2.748141in}{2.445214in}}%
\pgfpathlineto{\pgfqpoint{2.734951in}{2.461187in}}%
\pgfpathlineto{\pgfqpoint{2.726800in}{2.455499in}}%
\pgfpathlineto{\pgfqpoint{2.718639in}{2.449950in}}%
\pgfpathlineto{\pgfqpoint{2.710467in}{2.444544in}}%
\pgfpathlineto{\pgfqpoint{2.702285in}{2.439281in}}%
\pgfpathclose%
\pgfusepath{fill}%
\end{pgfscope}%
\begin{pgfscope}%
\pgfpathrectangle{\pgfqpoint{1.254980in}{0.150000in}}{\pgfqpoint{5.490039in}{5.490039in}}%
\pgfusepath{clip}%
\pgfsetbuttcap%
\pgfsetroundjoin%
\definecolor{currentfill}{rgb}{0.270595,0.214069,0.507052}%
\pgfsetfillcolor{currentfill}%
\pgfsetfillopacity{0.700000}%
\pgfsetlinewidth{0.000000pt}%
\definecolor{currentstroke}{rgb}{0.000000,0.000000,0.000000}%
\pgfsetstrokecolor{currentstroke}%
\pgfsetdash{}{0pt}%
\pgfpathmoveto{\pgfqpoint{2.755115in}{2.375938in}}%
\pgfpathlineto{\pgfqpoint{2.768305in}{2.360754in}}%
\pgfpathlineto{\pgfqpoint{2.781489in}{2.345826in}}%
\pgfpathlineto{\pgfqpoint{2.794666in}{2.331152in}}%
\pgfpathlineto{\pgfqpoint{2.807837in}{2.316730in}}%
\pgfpathlineto{\pgfqpoint{2.815964in}{2.322526in}}%
\pgfpathlineto{\pgfqpoint{2.824081in}{2.328452in}}%
\pgfpathlineto{\pgfqpoint{2.832187in}{2.334505in}}%
\pgfpathlineto{\pgfqpoint{2.840285in}{2.340683in}}%
\pgfpathlineto{\pgfqpoint{2.827140in}{2.354851in}}%
\pgfpathlineto{\pgfqpoint{2.813989in}{2.369270in}}%
\pgfpathlineto{\pgfqpoint{2.800832in}{2.383943in}}%
\pgfpathlineto{\pgfqpoint{2.787669in}{2.398872in}}%
\pgfpathlineto{\pgfqpoint{2.779546in}{2.392938in}}%
\pgfpathlineto{\pgfqpoint{2.771413in}{2.387136in}}%
\pgfpathlineto{\pgfqpoint{2.763269in}{2.381469in}}%
\pgfpathlineto{\pgfqpoint{2.755115in}{2.375938in}}%
\pgfpathclose%
\pgfusepath{fill}%
\end{pgfscope}%
\begin{pgfscope}%
\pgfpathrectangle{\pgfqpoint{1.254980in}{0.150000in}}{\pgfqpoint{5.490039in}{5.490039in}}%
\pgfusepath{clip}%
\pgfsetbuttcap%
\pgfsetroundjoin%
\definecolor{currentfill}{rgb}{0.250425,0.274290,0.533103}%
\pgfsetfillcolor{currentfill}%
\pgfsetfillopacity{0.700000}%
\pgfsetlinewidth{0.000000pt}%
\definecolor{currentstroke}{rgb}{0.000000,0.000000,0.000000}%
\pgfsetstrokecolor{currentstroke}%
\pgfsetdash{}{0pt}%
\pgfpathmoveto{\pgfqpoint{2.649330in}{2.506903in}}%
\pgfpathlineto{\pgfqpoint{2.662581in}{2.489588in}}%
\pgfpathlineto{\pgfqpoint{2.675824in}{2.472547in}}%
\pgfpathlineto{\pgfqpoint{2.689058in}{2.455779in}}%
\pgfpathlineto{\pgfqpoint{2.702285in}{2.439281in}}%
\pgfpathlineto{\pgfqpoint{2.710467in}{2.444544in}}%
\pgfpathlineto{\pgfqpoint{2.718639in}{2.449950in}}%
\pgfpathlineto{\pgfqpoint{2.726800in}{2.455499in}}%
\pgfpathlineto{\pgfqpoint{2.734951in}{2.461187in}}%
\pgfpathlineto{\pgfqpoint{2.721753in}{2.477428in}}%
\pgfpathlineto{\pgfqpoint{2.708548in}{2.493938in}}%
\pgfpathlineto{\pgfqpoint{2.695334in}{2.510719in}}%
\pgfpathlineto{\pgfqpoint{2.682112in}{2.527775in}}%
\pgfpathlineto{\pgfqpoint{2.673933in}{2.522334in}}%
\pgfpathlineto{\pgfqpoint{2.665743in}{2.517041in}}%
\pgfpathlineto{\pgfqpoint{2.657542in}{2.511896in}}%
\pgfpathlineto{\pgfqpoint{2.649330in}{2.506903in}}%
\pgfpathclose%
\pgfusepath{fill}%
\end{pgfscope}%
\begin{pgfscope}%
\pgfpathrectangle{\pgfqpoint{1.254980in}{0.150000in}}{\pgfqpoint{5.490039in}{5.490039in}}%
\pgfusepath{clip}%
\pgfsetbuttcap%
\pgfsetroundjoin%
\definecolor{currentfill}{rgb}{0.277134,0.185228,0.489898}%
\pgfsetfillcolor{currentfill}%
\pgfsetfillopacity{0.700000}%
\pgfsetlinewidth{0.000000pt}%
\definecolor{currentstroke}{rgb}{0.000000,0.000000,0.000000}%
\pgfsetstrokecolor{currentstroke}%
\pgfsetdash{}{0pt}%
\pgfpathmoveto{\pgfqpoint{2.807837in}{2.316730in}}%
\pgfpathlineto{\pgfqpoint{2.821003in}{2.302558in}}%
\pgfpathlineto{\pgfqpoint{2.834163in}{2.288634in}}%
\pgfpathlineto{\pgfqpoint{2.847317in}{2.274956in}}%
\pgfpathlineto{\pgfqpoint{2.860467in}{2.261521in}}%
\pgfpathlineto{\pgfqpoint{2.868568in}{2.267581in}}%
\pgfpathlineto{\pgfqpoint{2.876659in}{2.273764in}}%
\pgfpathlineto{\pgfqpoint{2.884740in}{2.280067in}}%
\pgfpathlineto{\pgfqpoint{2.892813in}{2.286488in}}%
\pgfpathlineto{\pgfqpoint{2.879688in}{2.299669in}}%
\pgfpathlineto{\pgfqpoint{2.866559in}{2.313095in}}%
\pgfpathlineto{\pgfqpoint{2.853424in}{2.326765in}}%
\pgfpathlineto{\pgfqpoint{2.840285in}{2.340683in}}%
\pgfpathlineto{\pgfqpoint{2.832187in}{2.334505in}}%
\pgfpathlineto{\pgfqpoint{2.824081in}{2.328452in}}%
\pgfpathlineto{\pgfqpoint{2.815964in}{2.322526in}}%
\pgfpathlineto{\pgfqpoint{2.807837in}{2.316730in}}%
\pgfpathclose%
\pgfusepath{fill}%
\end{pgfscope}%
\begin{pgfscope}%
\pgfpathrectangle{\pgfqpoint{1.254980in}{0.150000in}}{\pgfqpoint{5.490039in}{5.490039in}}%
\pgfusepath{clip}%
\pgfsetbuttcap%
\pgfsetroundjoin%
\definecolor{currentfill}{rgb}{0.172719,0.448791,0.557885}%
\pgfsetfillcolor{currentfill}%
\pgfsetfillopacity{0.700000}%
\pgfsetlinewidth{0.000000pt}%
\definecolor{currentstroke}{rgb}{0.000000,0.000000,0.000000}%
\pgfsetstrokecolor{currentstroke}%
\pgfsetdash{}{0pt}%
\pgfpathmoveto{\pgfqpoint{5.090726in}{2.875028in}}%
\pgfpathlineto{\pgfqpoint{5.104344in}{2.881568in}}%
\pgfpathlineto{\pgfqpoint{5.117977in}{2.888264in}}%
\pgfpathlineto{\pgfqpoint{5.131624in}{2.895115in}}%
\pgfpathlineto{\pgfqpoint{5.145286in}{2.902121in}}%
\pgfpathlineto{\pgfqpoint{5.152476in}{2.907895in}}%
\pgfpathlineto{\pgfqpoint{5.159661in}{2.913689in}}%
\pgfpathlineto{\pgfqpoint{5.166840in}{2.919508in}}%
\pgfpathlineto{\pgfqpoint{5.174015in}{2.925357in}}%
\pgfpathlineto{\pgfqpoint{5.160370in}{2.918759in}}%
\pgfpathlineto{\pgfqpoint{5.146741in}{2.912316in}}%
\pgfpathlineto{\pgfqpoint{5.133126in}{2.906029in}}%
\pgfpathlineto{\pgfqpoint{5.119526in}{2.899896in}}%
\pgfpathlineto{\pgfqpoint{5.112333in}{2.893629in}}%
\pgfpathlineto{\pgfqpoint{5.105136in}{2.887399in}}%
\pgfpathlineto{\pgfqpoint{5.097933in}{2.881200in}}%
\pgfpathlineto{\pgfqpoint{5.090726in}{2.875028in}}%
\pgfpathclose%
\pgfusepath{fill}%
\end{pgfscope}%
\begin{pgfscope}%
\pgfpathrectangle{\pgfqpoint{1.254980in}{0.150000in}}{\pgfqpoint{5.490039in}{5.490039in}}%
\pgfusepath{clip}%
\pgfsetbuttcap%
\pgfsetroundjoin%
\definecolor{currentfill}{rgb}{0.281446,0.084320,0.407414}%
\pgfsetfillcolor{currentfill}%
\pgfsetfillopacity{0.700000}%
\pgfsetlinewidth{0.000000pt}%
\definecolor{currentstroke}{rgb}{0.000000,0.000000,0.000000}%
\pgfsetstrokecolor{currentstroke}%
\pgfsetdash{}{0pt}%
\pgfpathmoveto{\pgfqpoint{3.102336in}{2.107264in}}%
\pgfpathlineto{\pgfqpoint{3.115412in}{2.097959in}}%
\pgfpathlineto{\pgfqpoint{3.128488in}{2.088869in}}%
\pgfpathlineto{\pgfqpoint{3.141563in}{2.079992in}}%
\pgfpathlineto{\pgfqpoint{3.154637in}{2.071326in}}%
\pgfpathlineto{\pgfqpoint{3.162593in}{2.079022in}}%
\pgfpathlineto{\pgfqpoint{3.170541in}{2.086797in}}%
\pgfpathlineto{\pgfqpoint{3.178483in}{2.094649in}}%
\pgfpathlineto{\pgfqpoint{3.186416in}{2.102576in}}%
\pgfpathlineto{\pgfqpoint{3.173361in}{2.111023in}}%
\pgfpathlineto{\pgfqpoint{3.160305in}{2.119681in}}%
\pgfpathlineto{\pgfqpoint{3.147249in}{2.128552in}}%
\pgfpathlineto{\pgfqpoint{3.134191in}{2.137637in}}%
\pgfpathlineto{\pgfqpoint{3.126239in}{2.129919in}}%
\pgfpathlineto{\pgfqpoint{3.118279in}{2.122283in}}%
\pgfpathlineto{\pgfqpoint{3.110311in}{2.114730in}}%
\pgfpathlineto{\pgfqpoint{3.102336in}{2.107264in}}%
\pgfpathclose%
\pgfusepath{fill}%
\end{pgfscope}%
\begin{pgfscope}%
\pgfpathrectangle{\pgfqpoint{1.254980in}{0.150000in}}{\pgfqpoint{5.490039in}{5.490039in}}%
\pgfusepath{clip}%
\pgfsetbuttcap%
\pgfsetroundjoin%
\definecolor{currentfill}{rgb}{0.165117,0.467423,0.558141}%
\pgfsetfillcolor{currentfill}%
\pgfsetfillopacity{0.700000}%
\pgfsetlinewidth{0.000000pt}%
\definecolor{currentstroke}{rgb}{0.000000,0.000000,0.000000}%
\pgfsetstrokecolor{currentstroke}%
\pgfsetdash{}{0pt}%
\pgfpathmoveto{\pgfqpoint{5.174015in}{2.925357in}}%
\pgfpathlineto{\pgfqpoint{5.187674in}{2.932109in}}%
\pgfpathlineto{\pgfqpoint{5.201348in}{2.939016in}}%
\pgfpathlineto{\pgfqpoint{5.215037in}{2.946078in}}%
\pgfpathlineto{\pgfqpoint{5.228741in}{2.953295in}}%
\pgfpathlineto{\pgfqpoint{5.235892in}{2.958750in}}%
\pgfpathlineto{\pgfqpoint{5.243037in}{2.964239in}}%
\pgfpathlineto{\pgfqpoint{5.250178in}{2.969764in}}%
\pgfpathlineto{\pgfqpoint{5.257314in}{2.975332in}}%
\pgfpathlineto{\pgfqpoint{5.243629in}{2.968553in}}%
\pgfpathlineto{\pgfqpoint{5.229959in}{2.961929in}}%
\pgfpathlineto{\pgfqpoint{5.216305in}{2.955459in}}%
\pgfpathlineto{\pgfqpoint{5.202665in}{2.949143in}}%
\pgfpathlineto{\pgfqpoint{5.195509in}{2.943128in}}%
\pgfpathlineto{\pgfqpoint{5.188349in}{2.937162in}}%
\pgfpathlineto{\pgfqpoint{5.181184in}{2.931240in}}%
\pgfpathlineto{\pgfqpoint{5.174015in}{2.925357in}}%
\pgfpathclose%
\pgfusepath{fill}%
\end{pgfscope}%
\begin{pgfscope}%
\pgfpathrectangle{\pgfqpoint{1.254980in}{0.150000in}}{\pgfqpoint{5.490039in}{5.490039in}}%
\pgfusepath{clip}%
\pgfsetbuttcap%
\pgfsetroundjoin%
\definecolor{currentfill}{rgb}{0.278791,0.062145,0.386592}%
\pgfsetfillcolor{currentfill}%
\pgfsetfillopacity{0.700000}%
\pgfsetlinewidth{0.000000pt}%
\definecolor{currentstroke}{rgb}{0.000000,0.000000,0.000000}%
\pgfsetstrokecolor{currentstroke}%
\pgfsetdash{}{0pt}%
\pgfpathmoveto{\pgfqpoint{3.510351in}{2.049150in}}%
\pgfpathlineto{\pgfqpoint{3.523414in}{2.045226in}}%
\pgfpathlineto{\pgfqpoint{3.536481in}{2.041489in}}%
\pgfpathlineto{\pgfqpoint{3.549553in}{2.037939in}}%
\pgfpathlineto{\pgfqpoint{3.562628in}{2.034575in}}%
\pgfpathlineto{\pgfqpoint{3.570418in}{2.043976in}}%
\pgfpathlineto{\pgfqpoint{3.578202in}{2.053396in}}%
\pgfpathlineto{\pgfqpoint{3.585981in}{2.062834in}}%
\pgfpathlineto{\pgfqpoint{3.593755in}{2.072289in}}%
\pgfpathlineto{\pgfqpoint{3.580691in}{2.075520in}}%
\pgfpathlineto{\pgfqpoint{3.567631in}{2.078936in}}%
\pgfpathlineto{\pgfqpoint{3.554576in}{2.082540in}}%
\pgfpathlineto{\pgfqpoint{3.541525in}{2.086330in}}%
\pgfpathlineto{\pgfqpoint{3.533740in}{2.076998in}}%
\pgfpathlineto{\pgfqpoint{3.525949in}{2.067690in}}%
\pgfpathlineto{\pgfqpoint{3.518153in}{2.058407in}}%
\pgfpathlineto{\pgfqpoint{3.510351in}{2.049150in}}%
\pgfpathclose%
\pgfusepath{fill}%
\end{pgfscope}%
\begin{pgfscope}%
\pgfpathrectangle{\pgfqpoint{1.254980in}{0.150000in}}{\pgfqpoint{5.490039in}{5.490039in}}%
\pgfusepath{clip}%
\pgfsetbuttcap%
\pgfsetroundjoin%
\definecolor{currentfill}{rgb}{0.235526,0.309527,0.542944}%
\pgfsetfillcolor{currentfill}%
\pgfsetfillopacity{0.700000}%
\pgfsetlinewidth{0.000000pt}%
\definecolor{currentstroke}{rgb}{0.000000,0.000000,0.000000}%
\pgfsetstrokecolor{currentstroke}%
\pgfsetdash{}{0pt}%
\pgfpathmoveto{\pgfqpoint{2.596233in}{2.578961in}}%
\pgfpathlineto{\pgfqpoint{2.609522in}{2.560521in}}%
\pgfpathlineto{\pgfqpoint{2.622800in}{2.542367in}}%
\pgfpathlineto{\pgfqpoint{2.636070in}{2.524495in}}%
\pgfpathlineto{\pgfqpoint{2.649330in}{2.506903in}}%
\pgfpathlineto{\pgfqpoint{2.657542in}{2.511896in}}%
\pgfpathlineto{\pgfqpoint{2.665743in}{2.517041in}}%
\pgfpathlineto{\pgfqpoint{2.673933in}{2.522334in}}%
\pgfpathlineto{\pgfqpoint{2.682112in}{2.527775in}}%
\pgfpathlineto{\pgfqpoint{2.668882in}{2.545107in}}%
\pgfpathlineto{\pgfqpoint{2.655643in}{2.562719in}}%
\pgfpathlineto{\pgfqpoint{2.642395in}{2.580612in}}%
\pgfpathlineto{\pgfqpoint{2.629137in}{2.598790in}}%
\pgfpathlineto{\pgfqpoint{2.620928in}{2.593599in}}%
\pgfpathlineto{\pgfqpoint{2.612708in}{2.588562in}}%
\pgfpathlineto{\pgfqpoint{2.604477in}{2.583682in}}%
\pgfpathlineto{\pgfqpoint{2.596233in}{2.578961in}}%
\pgfpathclose%
\pgfusepath{fill}%
\end{pgfscope}%
\begin{pgfscope}%
\pgfpathrectangle{\pgfqpoint{1.254980in}{0.150000in}}{\pgfqpoint{5.490039in}{5.490039in}}%
\pgfusepath{clip}%
\pgfsetbuttcap%
\pgfsetroundjoin%
\definecolor{currentfill}{rgb}{0.280868,0.160771,0.472899}%
\pgfsetfillcolor{currentfill}%
\pgfsetfillopacity{0.700000}%
\pgfsetlinewidth{0.000000pt}%
\definecolor{currentstroke}{rgb}{0.000000,0.000000,0.000000}%
\pgfsetstrokecolor{currentstroke}%
\pgfsetdash{}{0pt}%
\pgfpathmoveto{\pgfqpoint{2.860467in}{2.261521in}}%
\pgfpathlineto{\pgfqpoint{2.873612in}{2.248329in}}%
\pgfpathlineto{\pgfqpoint{2.886752in}{2.235377in}}%
\pgfpathlineto{\pgfqpoint{2.899888in}{2.222663in}}%
\pgfpathlineto{\pgfqpoint{2.913020in}{2.210186in}}%
\pgfpathlineto{\pgfqpoint{2.921095in}{2.216509in}}%
\pgfpathlineto{\pgfqpoint{2.929162in}{2.222947in}}%
\pgfpathlineto{\pgfqpoint{2.937219in}{2.229498in}}%
\pgfpathlineto{\pgfqpoint{2.945267in}{2.236160in}}%
\pgfpathlineto{\pgfqpoint{2.932160in}{2.248386in}}%
\pgfpathlineto{\pgfqpoint{2.919048in}{2.260848in}}%
\pgfpathlineto{\pgfqpoint{2.905932in}{2.273548in}}%
\pgfpathlineto{\pgfqpoint{2.892813in}{2.286488in}}%
\pgfpathlineto{\pgfqpoint{2.884740in}{2.280067in}}%
\pgfpathlineto{\pgfqpoint{2.876659in}{2.273764in}}%
\pgfpathlineto{\pgfqpoint{2.868568in}{2.267581in}}%
\pgfpathlineto{\pgfqpoint{2.860467in}{2.261521in}}%
\pgfpathclose%
\pgfusepath{fill}%
\end{pgfscope}%
\begin{pgfscope}%
\pgfpathrectangle{\pgfqpoint{1.254980in}{0.150000in}}{\pgfqpoint{5.490039in}{5.490039in}}%
\pgfusepath{clip}%
\pgfsetbuttcap%
\pgfsetroundjoin%
\definecolor{currentfill}{rgb}{0.278826,0.175490,0.483397}%
\pgfsetfillcolor{currentfill}%
\pgfsetfillopacity{0.700000}%
\pgfsetlinewidth{0.000000pt}%
\definecolor{currentstroke}{rgb}{0.000000,0.000000,0.000000}%
\pgfsetstrokecolor{currentstroke}%
\pgfsetdash{}{0pt}%
\pgfpathmoveto{\pgfqpoint{4.062299in}{2.246679in}}%
\pgfpathlineto{\pgfqpoint{4.075483in}{2.248125in}}%
\pgfpathlineto{\pgfqpoint{4.088677in}{2.249742in}}%
\pgfpathlineto{\pgfqpoint{4.101879in}{2.251527in}}%
\pgfpathlineto{\pgfqpoint{4.115090in}{2.253482in}}%
\pgfpathlineto{\pgfqpoint{4.122696in}{2.263057in}}%
\pgfpathlineto{\pgfqpoint{4.130297in}{2.272599in}}%
\pgfpathlineto{\pgfqpoint{4.137893in}{2.282108in}}%
\pgfpathlineto{\pgfqpoint{4.145484in}{2.291586in}}%
\pgfpathlineto{\pgfqpoint{4.132280in}{2.289666in}}%
\pgfpathlineto{\pgfqpoint{4.119085in}{2.287914in}}%
\pgfpathlineto{\pgfqpoint{4.105899in}{2.286332in}}%
\pgfpathlineto{\pgfqpoint{4.092722in}{2.284920in}}%
\pgfpathlineto{\pgfqpoint{4.085123in}{2.275397in}}%
\pgfpathlineto{\pgfqpoint{4.077520in}{2.265850in}}%
\pgfpathlineto{\pgfqpoint{4.069912in}{2.256278in}}%
\pgfpathlineto{\pgfqpoint{4.062299in}{2.246679in}}%
\pgfpathclose%
\pgfusepath{fill}%
\end{pgfscope}%
\begin{pgfscope}%
\pgfpathrectangle{\pgfqpoint{1.254980in}{0.150000in}}{\pgfqpoint{5.490039in}{5.490039in}}%
\pgfusepath{clip}%
\pgfsetbuttcap%
\pgfsetroundjoin%
\definecolor{currentfill}{rgb}{0.281887,0.150881,0.465405}%
\pgfsetfillcolor{currentfill}%
\pgfsetfillopacity{0.700000}%
\pgfsetlinewidth{0.000000pt}%
\definecolor{currentstroke}{rgb}{0.000000,0.000000,0.000000}%
\pgfsetstrokecolor{currentstroke}%
\pgfsetdash{}{0pt}%
\pgfpathmoveto{\pgfqpoint{3.979110in}{2.203908in}}%
\pgfpathlineto{\pgfqpoint{3.992269in}{2.204677in}}%
\pgfpathlineto{\pgfqpoint{4.005436in}{2.205617in}}%
\pgfpathlineto{\pgfqpoint{4.018611in}{2.206728in}}%
\pgfpathlineto{\pgfqpoint{4.031795in}{2.208010in}}%
\pgfpathlineto{\pgfqpoint{4.039428in}{2.217720in}}%
\pgfpathlineto{\pgfqpoint{4.047057in}{2.227401in}}%
\pgfpathlineto{\pgfqpoint{4.054680in}{2.237054in}}%
\pgfpathlineto{\pgfqpoint{4.062299in}{2.246679in}}%
\pgfpathlineto{\pgfqpoint{4.049122in}{2.245403in}}%
\pgfpathlineto{\pgfqpoint{4.035955in}{2.244298in}}%
\pgfpathlineto{\pgfqpoint{4.022795in}{2.243364in}}%
\pgfpathlineto{\pgfqpoint{4.009644in}{2.242602in}}%
\pgfpathlineto{\pgfqpoint{4.002018in}{2.232960in}}%
\pgfpathlineto{\pgfqpoint{3.994387in}{2.223297in}}%
\pgfpathlineto{\pgfqpoint{3.986751in}{2.213613in}}%
\pgfpathlineto{\pgfqpoint{3.979110in}{2.203908in}}%
\pgfpathclose%
\pgfusepath{fill}%
\end{pgfscope}%
\begin{pgfscope}%
\pgfpathrectangle{\pgfqpoint{1.254980in}{0.150000in}}{\pgfqpoint{5.490039in}{5.490039in}}%
\pgfusepath{clip}%
\pgfsetbuttcap%
\pgfsetroundjoin%
\definecolor{currentfill}{rgb}{0.275191,0.194905,0.496005}%
\pgfsetfillcolor{currentfill}%
\pgfsetfillopacity{0.700000}%
\pgfsetlinewidth{0.000000pt}%
\definecolor{currentstroke}{rgb}{0.000000,0.000000,0.000000}%
\pgfsetstrokecolor{currentstroke}%
\pgfsetdash{}{0pt}%
\pgfpathmoveto{\pgfqpoint{4.145484in}{2.291586in}}%
\pgfpathlineto{\pgfqpoint{4.158697in}{2.293675in}}%
\pgfpathlineto{\pgfqpoint{4.171920in}{2.295932in}}%
\pgfpathlineto{\pgfqpoint{4.185152in}{2.298357in}}%
\pgfpathlineto{\pgfqpoint{4.198393in}{2.300949in}}%
\pgfpathlineto{\pgfqpoint{4.205972in}{2.310344in}}%
\pgfpathlineto{\pgfqpoint{4.213545in}{2.319703in}}%
\pgfpathlineto{\pgfqpoint{4.221113in}{2.329026in}}%
\pgfpathlineto{\pgfqpoint{4.228676in}{2.338314in}}%
\pgfpathlineto{\pgfqpoint{4.215442in}{2.335784in}}%
\pgfpathlineto{\pgfqpoint{4.202217in}{2.333422in}}%
\pgfpathlineto{\pgfqpoint{4.189002in}{2.331227in}}%
\pgfpathlineto{\pgfqpoint{4.175796in}{2.329200in}}%
\pgfpathlineto{\pgfqpoint{4.168226in}{2.319839in}}%
\pgfpathlineto{\pgfqpoint{4.160650in}{2.310451in}}%
\pgfpathlineto{\pgfqpoint{4.153070in}{2.301033in}}%
\pgfpathlineto{\pgfqpoint{4.145484in}{2.291586in}}%
\pgfpathclose%
\pgfusepath{fill}%
\end{pgfscope}%
\begin{pgfscope}%
\pgfpathrectangle{\pgfqpoint{1.254980in}{0.150000in}}{\pgfqpoint{5.490039in}{5.490039in}}%
\pgfusepath{clip}%
\pgfsetbuttcap%
\pgfsetroundjoin%
\definecolor{currentfill}{rgb}{0.156270,0.489624,0.557936}%
\pgfsetfillcolor{currentfill}%
\pgfsetfillopacity{0.700000}%
\pgfsetlinewidth{0.000000pt}%
\definecolor{currentstroke}{rgb}{0.000000,0.000000,0.000000}%
\pgfsetstrokecolor{currentstroke}%
\pgfsetdash{}{0pt}%
\pgfpathmoveto{\pgfqpoint{5.257314in}{2.975332in}}%
\pgfpathlineto{\pgfqpoint{5.271013in}{2.982264in}}%
\pgfpathlineto{\pgfqpoint{5.284728in}{2.989351in}}%
\pgfpathlineto{\pgfqpoint{5.298459in}{2.996591in}}%
\pgfpathlineto{\pgfqpoint{5.312205in}{3.003986in}}%
\pgfpathlineto{\pgfqpoint{5.319316in}{3.009144in}}%
\pgfpathlineto{\pgfqpoint{5.326422in}{3.014349in}}%
\pgfpathlineto{\pgfqpoint{5.333523in}{3.019604in}}%
\pgfpathlineto{\pgfqpoint{5.340620in}{3.024915in}}%
\pgfpathlineto{\pgfqpoint{5.326895in}{3.017988in}}%
\pgfpathlineto{\pgfqpoint{5.313186in}{3.011214in}}%
\pgfpathlineto{\pgfqpoint{5.299492in}{3.004594in}}%
\pgfpathlineto{\pgfqpoint{5.285813in}{2.998127in}}%
\pgfpathlineto{\pgfqpoint{5.278694in}{2.992339in}}%
\pgfpathlineto{\pgfqpoint{5.271572in}{2.986614in}}%
\pgfpathlineto{\pgfqpoint{5.264445in}{2.980947in}}%
\pgfpathlineto{\pgfqpoint{5.257314in}{2.975332in}}%
\pgfpathclose%
\pgfusepath{fill}%
\end{pgfscope}%
\begin{pgfscope}%
\pgfpathrectangle{\pgfqpoint{1.254980in}{0.150000in}}{\pgfqpoint{5.490039in}{5.490039in}}%
\pgfusepath{clip}%
\pgfsetbuttcap%
\pgfsetroundjoin%
\definecolor{currentfill}{rgb}{0.283072,0.130895,0.449241}%
\pgfsetfillcolor{currentfill}%
\pgfsetfillopacity{0.700000}%
\pgfsetlinewidth{0.000000pt}%
\definecolor{currentstroke}{rgb}{0.000000,0.000000,0.000000}%
\pgfsetstrokecolor{currentstroke}%
\pgfsetdash{}{0pt}%
\pgfpathmoveto{\pgfqpoint{3.895907in}{2.163609in}}%
\pgfpathlineto{\pgfqpoint{3.909043in}{2.163663in}}%
\pgfpathlineto{\pgfqpoint{3.922187in}{2.163891in}}%
\pgfpathlineto{\pgfqpoint{3.935338in}{2.164293in}}%
\pgfpathlineto{\pgfqpoint{3.948496in}{2.164867in}}%
\pgfpathlineto{\pgfqpoint{3.956157in}{2.174661in}}%
\pgfpathlineto{\pgfqpoint{3.963813in}{2.184432in}}%
\pgfpathlineto{\pgfqpoint{3.971464in}{2.194181in}}%
\pgfpathlineto{\pgfqpoint{3.979110in}{2.203908in}}%
\pgfpathlineto{\pgfqpoint{3.965959in}{2.203312in}}%
\pgfpathlineto{\pgfqpoint{3.952816in}{2.202889in}}%
\pgfpathlineto{\pgfqpoint{3.939681in}{2.202639in}}%
\pgfpathlineto{\pgfqpoint{3.926553in}{2.202563in}}%
\pgfpathlineto{\pgfqpoint{3.918899in}{2.192848in}}%
\pgfpathlineto{\pgfqpoint{3.911240in}{2.183117in}}%
\pgfpathlineto{\pgfqpoint{3.903576in}{2.173371in}}%
\pgfpathlineto{\pgfqpoint{3.895907in}{2.163609in}}%
\pgfpathclose%
\pgfusepath{fill}%
\end{pgfscope}%
\begin{pgfscope}%
\pgfpathrectangle{\pgfqpoint{1.254980in}{0.150000in}}{\pgfqpoint{5.490039in}{5.490039in}}%
\pgfusepath{clip}%
\pgfsetbuttcap%
\pgfsetroundjoin%
\definecolor{currentfill}{rgb}{0.269308,0.218818,0.509577}%
\pgfsetfillcolor{currentfill}%
\pgfsetfillopacity{0.700000}%
\pgfsetlinewidth{0.000000pt}%
\definecolor{currentstroke}{rgb}{0.000000,0.000000,0.000000}%
\pgfsetstrokecolor{currentstroke}%
\pgfsetdash{}{0pt}%
\pgfpathmoveto{\pgfqpoint{4.228676in}{2.338314in}}%
\pgfpathlineto{\pgfqpoint{4.241920in}{2.341010in}}%
\pgfpathlineto{\pgfqpoint{4.255174in}{2.343874in}}%
\pgfpathlineto{\pgfqpoint{4.268438in}{2.346903in}}%
\pgfpathlineto{\pgfqpoint{4.281713in}{2.350098in}}%
\pgfpathlineto{\pgfqpoint{4.289263in}{2.359273in}}%
\pgfpathlineto{\pgfqpoint{4.296808in}{2.368408in}}%
\pgfpathlineto{\pgfqpoint{4.304348in}{2.377506in}}%
\pgfpathlineto{\pgfqpoint{4.311882in}{2.386567in}}%
\pgfpathlineto{\pgfqpoint{4.298615in}{2.383463in}}%
\pgfpathlineto{\pgfqpoint{4.285359in}{2.380524in}}%
\pgfpathlineto{\pgfqpoint{4.272112in}{2.377752in}}%
\pgfpathlineto{\pgfqpoint{4.258876in}{2.375146in}}%
\pgfpathlineto{\pgfqpoint{4.251333in}{2.365983in}}%
\pgfpathlineto{\pgfqpoint{4.243786in}{2.356791in}}%
\pgfpathlineto{\pgfqpoint{4.236234in}{2.347569in}}%
\pgfpathlineto{\pgfqpoint{4.228676in}{2.338314in}}%
\pgfpathclose%
\pgfusepath{fill}%
\end{pgfscope}%
\begin{pgfscope}%
\pgfpathrectangle{\pgfqpoint{1.254980in}{0.150000in}}{\pgfqpoint{5.490039in}{5.490039in}}%
\pgfusepath{clip}%
\pgfsetbuttcap%
\pgfsetroundjoin%
\definecolor{currentfill}{rgb}{0.149039,0.508051,0.557250}%
\pgfsetfillcolor{currentfill}%
\pgfsetfillopacity{0.700000}%
\pgfsetlinewidth{0.000000pt}%
\definecolor{currentstroke}{rgb}{0.000000,0.000000,0.000000}%
\pgfsetstrokecolor{currentstroke}%
\pgfsetdash{}{0pt}%
\pgfpathmoveto{\pgfqpoint{5.340620in}{3.024915in}}%
\pgfpathlineto{\pgfqpoint{5.354360in}{3.031995in}}%
\pgfpathlineto{\pgfqpoint{5.368116in}{3.039229in}}%
\pgfpathlineto{\pgfqpoint{5.381888in}{3.046616in}}%
\pgfpathlineto{\pgfqpoint{5.395675in}{3.054157in}}%
\pgfpathlineto{\pgfqpoint{5.402746in}{3.059044in}}%
\pgfpathlineto{\pgfqpoint{5.409812in}{3.063991in}}%
\pgfpathlineto{\pgfqpoint{5.416874in}{3.069004in}}%
\pgfpathlineto{\pgfqpoint{5.423932in}{3.074088in}}%
\pgfpathlineto{\pgfqpoint{5.410168in}{3.067045in}}%
\pgfpathlineto{\pgfqpoint{5.396419in}{3.060153in}}%
\pgfpathlineto{\pgfqpoint{5.382686in}{3.053415in}}%
\pgfpathlineto{\pgfqpoint{5.368968in}{3.046829in}}%
\pgfpathlineto{\pgfqpoint{5.361887in}{3.041239in}}%
\pgfpathlineto{\pgfqpoint{5.354802in}{3.035727in}}%
\pgfpathlineto{\pgfqpoint{5.347713in}{3.030288in}}%
\pgfpathlineto{\pgfqpoint{5.340620in}{3.024915in}}%
\pgfpathclose%
\pgfusepath{fill}%
\end{pgfscope}%
\begin{pgfscope}%
\pgfpathrectangle{\pgfqpoint{1.254980in}{0.150000in}}{\pgfqpoint{5.490039in}{5.490039in}}%
\pgfusepath{clip}%
\pgfsetbuttcap%
\pgfsetroundjoin%
\definecolor{currentfill}{rgb}{0.262138,0.242286,0.520837}%
\pgfsetfillcolor{currentfill}%
\pgfsetfillopacity{0.700000}%
\pgfsetlinewidth{0.000000pt}%
\definecolor{currentstroke}{rgb}{0.000000,0.000000,0.000000}%
\pgfsetstrokecolor{currentstroke}%
\pgfsetdash{}{0pt}%
\pgfpathmoveto{\pgfqpoint{4.311882in}{2.386567in}}%
\pgfpathlineto{\pgfqpoint{4.325159in}{2.389837in}}%
\pgfpathlineto{\pgfqpoint{4.338447in}{2.393272in}}%
\pgfpathlineto{\pgfqpoint{4.351745in}{2.396871in}}%
\pgfpathlineto{\pgfqpoint{4.365054in}{2.400636in}}%
\pgfpathlineto{\pgfqpoint{4.372576in}{2.409553in}}%
\pgfpathlineto{\pgfqpoint{4.380092in}{2.418430in}}%
\pgfpathlineto{\pgfqpoint{4.387603in}{2.427269in}}%
\pgfpathlineto{\pgfqpoint{4.395108in}{2.436071in}}%
\pgfpathlineto{\pgfqpoint{4.381807in}{2.432426in}}%
\pgfpathlineto{\pgfqpoint{4.368516in}{2.428946in}}%
\pgfpathlineto{\pgfqpoint{4.355236in}{2.425630in}}%
\pgfpathlineto{\pgfqpoint{4.341967in}{2.422480in}}%
\pgfpathlineto{\pgfqpoint{4.334454in}{2.413548in}}%
\pgfpathlineto{\pgfqpoint{4.326935in}{2.404587in}}%
\pgfpathlineto{\pgfqpoint{4.319411in}{2.395594in}}%
\pgfpathlineto{\pgfqpoint{4.311882in}{2.386567in}}%
\pgfpathclose%
\pgfusepath{fill}%
\end{pgfscope}%
\begin{pgfscope}%
\pgfpathrectangle{\pgfqpoint{1.254980in}{0.150000in}}{\pgfqpoint{5.490039in}{5.490039in}}%
\pgfusepath{clip}%
\pgfsetbuttcap%
\pgfsetroundjoin%
\definecolor{currentfill}{rgb}{0.283091,0.110553,0.431554}%
\pgfsetfillcolor{currentfill}%
\pgfsetfillopacity{0.700000}%
\pgfsetlinewidth{0.000000pt}%
\definecolor{currentstroke}{rgb}{0.000000,0.000000,0.000000}%
\pgfsetstrokecolor{currentstroke}%
\pgfsetdash{}{0pt}%
\pgfpathmoveto{\pgfqpoint{3.812676in}{2.126138in}}%
\pgfpathlineto{\pgfqpoint{3.825792in}{2.125441in}}%
\pgfpathlineto{\pgfqpoint{3.838915in}{2.124921in}}%
\pgfpathlineto{\pgfqpoint{3.852045in}{2.124576in}}%
\pgfpathlineto{\pgfqpoint{3.865182in}{2.124405in}}%
\pgfpathlineto{\pgfqpoint{3.872871in}{2.134230in}}%
\pgfpathlineto{\pgfqpoint{3.880555in}{2.144039in}}%
\pgfpathlineto{\pgfqpoint{3.888233in}{2.153832in}}%
\pgfpathlineto{\pgfqpoint{3.895907in}{2.163609in}}%
\pgfpathlineto{\pgfqpoint{3.882779in}{2.163730in}}%
\pgfpathlineto{\pgfqpoint{3.869657in}{2.164025in}}%
\pgfpathlineto{\pgfqpoint{3.856542in}{2.164496in}}%
\pgfpathlineto{\pgfqpoint{3.843435in}{2.165143in}}%
\pgfpathlineto{\pgfqpoint{3.835752in}{2.155405in}}%
\pgfpathlineto{\pgfqpoint{3.828065in}{2.145658in}}%
\pgfpathlineto{\pgfqpoint{3.820373in}{2.135903in}}%
\pgfpathlineto{\pgfqpoint{3.812676in}{2.126138in}}%
\pgfpathclose%
\pgfusepath{fill}%
\end{pgfscope}%
\begin{pgfscope}%
\pgfpathrectangle{\pgfqpoint{1.254980in}{0.150000in}}{\pgfqpoint{5.490039in}{5.490039in}}%
\pgfusepath{clip}%
\pgfsetbuttcap%
\pgfsetroundjoin%
\definecolor{currentfill}{rgb}{0.282623,0.140926,0.457517}%
\pgfsetfillcolor{currentfill}%
\pgfsetfillopacity{0.700000}%
\pgfsetlinewidth{0.000000pt}%
\definecolor{currentstroke}{rgb}{0.000000,0.000000,0.000000}%
\pgfsetstrokecolor{currentstroke}%
\pgfsetdash{}{0pt}%
\pgfpathmoveto{\pgfqpoint{2.913020in}{2.210186in}}%
\pgfpathlineto{\pgfqpoint{2.926148in}{2.197944in}}%
\pgfpathlineto{\pgfqpoint{2.939272in}{2.185934in}}%
\pgfpathlineto{\pgfqpoint{2.952392in}{2.174156in}}%
\pgfpathlineto{\pgfqpoint{2.965510in}{2.162608in}}%
\pgfpathlineto{\pgfqpoint{2.973561in}{2.169192in}}%
\pgfpathlineto{\pgfqpoint{2.981604in}{2.175884in}}%
\pgfpathlineto{\pgfqpoint{2.989638in}{2.182682in}}%
\pgfpathlineto{\pgfqpoint{2.997664in}{2.189584in}}%
\pgfpathlineto{\pgfqpoint{2.984570in}{2.200883in}}%
\pgfpathlineto{\pgfqpoint{2.971472in}{2.212410in}}%
\pgfpathlineto{\pgfqpoint{2.958371in}{2.224169in}}%
\pgfpathlineto{\pgfqpoint{2.945267in}{2.236160in}}%
\pgfpathlineto{\pgfqpoint{2.937219in}{2.229498in}}%
\pgfpathlineto{\pgfqpoint{2.929162in}{2.222947in}}%
\pgfpathlineto{\pgfqpoint{2.921095in}{2.216509in}}%
\pgfpathlineto{\pgfqpoint{2.913020in}{2.210186in}}%
\pgfpathclose%
\pgfusepath{fill}%
\end{pgfscope}%
\begin{pgfscope}%
\pgfpathrectangle{\pgfqpoint{1.254980in}{0.150000in}}{\pgfqpoint{5.490039in}{5.490039in}}%
\pgfusepath{clip}%
\pgfsetbuttcap%
\pgfsetroundjoin%
\definecolor{currentfill}{rgb}{0.141935,0.526453,0.555991}%
\pgfsetfillcolor{currentfill}%
\pgfsetfillopacity{0.700000}%
\pgfsetlinewidth{0.000000pt}%
\definecolor{currentstroke}{rgb}{0.000000,0.000000,0.000000}%
\pgfsetstrokecolor{currentstroke}%
\pgfsetdash{}{0pt}%
\pgfpathmoveto{\pgfqpoint{5.423932in}{3.074088in}}%
\pgfpathlineto{\pgfqpoint{5.437712in}{3.081285in}}%
\pgfpathlineto{\pgfqpoint{5.451508in}{3.088634in}}%
\pgfpathlineto{\pgfqpoint{5.465320in}{3.096135in}}%
\pgfpathlineto{\pgfqpoint{5.479148in}{3.103790in}}%
\pgfpathlineto{\pgfqpoint{5.486178in}{3.108436in}}%
\pgfpathlineto{\pgfqpoint{5.493205in}{3.113158in}}%
\pgfpathlineto{\pgfqpoint{5.500228in}{3.117962in}}%
\pgfpathlineto{\pgfqpoint{5.507247in}{3.122854in}}%
\pgfpathlineto{\pgfqpoint{5.493444in}{3.115725in}}%
\pgfpathlineto{\pgfqpoint{5.479657in}{3.108749in}}%
\pgfpathlineto{\pgfqpoint{5.465885in}{3.101924in}}%
\pgfpathlineto{\pgfqpoint{5.452130in}{3.095251in}}%
\pgfpathlineto{\pgfqpoint{5.445085in}{3.089825in}}%
\pgfpathlineto{\pgfqpoint{5.438037in}{3.084493in}}%
\pgfpathlineto{\pgfqpoint{5.430986in}{3.079249in}}%
\pgfpathlineto{\pgfqpoint{5.423932in}{3.074088in}}%
\pgfpathclose%
\pgfusepath{fill}%
\end{pgfscope}%
\begin{pgfscope}%
\pgfpathrectangle{\pgfqpoint{1.254980in}{0.150000in}}{\pgfqpoint{5.490039in}{5.490039in}}%
\pgfusepath{clip}%
\pgfsetbuttcap%
\pgfsetroundjoin%
\definecolor{currentfill}{rgb}{0.220057,0.343307,0.549413}%
\pgfsetfillcolor{currentfill}%
\pgfsetfillopacity{0.700000}%
\pgfsetlinewidth{0.000000pt}%
\definecolor{currentstroke}{rgb}{0.000000,0.000000,0.000000}%
\pgfsetstrokecolor{currentstroke}%
\pgfsetdash{}{0pt}%
\pgfpathmoveto{\pgfqpoint{2.542976in}{2.655622in}}%
\pgfpathlineto{\pgfqpoint{2.556307in}{2.636016in}}%
\pgfpathlineto{\pgfqpoint{2.569626in}{2.616705in}}%
\pgfpathlineto{\pgfqpoint{2.582935in}{2.597688in}}%
\pgfpathlineto{\pgfqpoint{2.596233in}{2.578961in}}%
\pgfpathlineto{\pgfqpoint{2.604477in}{2.583682in}}%
\pgfpathlineto{\pgfqpoint{2.612708in}{2.588562in}}%
\pgfpathlineto{\pgfqpoint{2.620928in}{2.593599in}}%
\pgfpathlineto{\pgfqpoint{2.629137in}{2.598790in}}%
\pgfpathlineto{\pgfqpoint{2.615870in}{2.617254in}}%
\pgfpathlineto{\pgfqpoint{2.602593in}{2.636009in}}%
\pgfpathlineto{\pgfqpoint{2.589305in}{2.655056in}}%
\pgfpathlineto{\pgfqpoint{2.576007in}{2.674399in}}%
\pgfpathlineto{\pgfqpoint{2.567767in}{2.669460in}}%
\pgfpathlineto{\pgfqpoint{2.559516in}{2.664683in}}%
\pgfpathlineto{\pgfqpoint{2.551252in}{2.660069in}}%
\pgfpathlineto{\pgfqpoint{2.542976in}{2.655622in}}%
\pgfpathclose%
\pgfusepath{fill}%
\end{pgfscope}%
\begin{pgfscope}%
\pgfpathrectangle{\pgfqpoint{1.254980in}{0.150000in}}{\pgfqpoint{5.490039in}{5.490039in}}%
\pgfusepath{clip}%
\pgfsetbuttcap%
\pgfsetroundjoin%
\definecolor{currentfill}{rgb}{0.277941,0.056324,0.381191}%
\pgfsetfillcolor{currentfill}%
\pgfsetfillopacity{0.700000}%
\pgfsetlinewidth{0.000000pt}%
\definecolor{currentstroke}{rgb}{0.000000,0.000000,0.000000}%
\pgfsetstrokecolor{currentstroke}%
\pgfsetdash{}{0pt}%
\pgfpathmoveto{\pgfqpoint{3.290870in}{2.042460in}}%
\pgfpathlineto{\pgfqpoint{3.303930in}{2.035859in}}%
\pgfpathlineto{\pgfqpoint{3.316992in}{2.029458in}}%
\pgfpathlineto{\pgfqpoint{3.330056in}{2.023256in}}%
\pgfpathlineto{\pgfqpoint{3.343121in}{2.017250in}}%
\pgfpathlineto{\pgfqpoint{3.350998in}{2.025853in}}%
\pgfpathlineto{\pgfqpoint{3.358869in}{2.034506in}}%
\pgfpathlineto{\pgfqpoint{3.366734in}{2.043208in}}%
\pgfpathlineto{\pgfqpoint{3.374592in}{2.051958in}}%
\pgfpathlineto{\pgfqpoint{3.361542in}{2.057774in}}%
\pgfpathlineto{\pgfqpoint{3.348493in}{2.063787in}}%
\pgfpathlineto{\pgfqpoint{3.335447in}{2.069999in}}%
\pgfpathlineto{\pgfqpoint{3.322402in}{2.076409in}}%
\pgfpathlineto{\pgfqpoint{3.314529in}{2.067838in}}%
\pgfpathlineto{\pgfqpoint{3.306649in}{2.059322in}}%
\pgfpathlineto{\pgfqpoint{3.298763in}{2.050862in}}%
\pgfpathlineto{\pgfqpoint{3.290870in}{2.042460in}}%
\pgfpathclose%
\pgfusepath{fill}%
\end{pgfscope}%
\begin{pgfscope}%
\pgfpathrectangle{\pgfqpoint{1.254980in}{0.150000in}}{\pgfqpoint{5.490039in}{5.490039in}}%
\pgfusepath{clip}%
\pgfsetbuttcap%
\pgfsetroundjoin%
\definecolor{currentfill}{rgb}{0.252194,0.269783,0.531579}%
\pgfsetfillcolor{currentfill}%
\pgfsetfillopacity{0.700000}%
\pgfsetlinewidth{0.000000pt}%
\definecolor{currentstroke}{rgb}{0.000000,0.000000,0.000000}%
\pgfsetstrokecolor{currentstroke}%
\pgfsetdash{}{0pt}%
\pgfpathmoveto{\pgfqpoint{4.395108in}{2.436071in}}%
\pgfpathlineto{\pgfqpoint{4.408420in}{2.439879in}}%
\pgfpathlineto{\pgfqpoint{4.421743in}{2.443852in}}%
\pgfpathlineto{\pgfqpoint{4.435078in}{2.447987in}}%
\pgfpathlineto{\pgfqpoint{4.448424in}{2.452287in}}%
\pgfpathlineto{\pgfqpoint{4.455916in}{2.460916in}}%
\pgfpathlineto{\pgfqpoint{4.463402in}{2.469504in}}%
\pgfpathlineto{\pgfqpoint{4.470883in}{2.478054in}}%
\pgfpathlineto{\pgfqpoint{4.478358in}{2.486568in}}%
\pgfpathlineto{\pgfqpoint{4.465020in}{2.482418in}}%
\pgfpathlineto{\pgfqpoint{4.451694in}{2.478430in}}%
\pgfpathlineto{\pgfqpoint{4.438379in}{2.474605in}}%
\pgfpathlineto{\pgfqpoint{4.425076in}{2.470945in}}%
\pgfpathlineto{\pgfqpoint{4.417592in}{2.462272in}}%
\pgfpathlineto{\pgfqpoint{4.410102in}{2.453570in}}%
\pgfpathlineto{\pgfqpoint{4.402608in}{2.444837in}}%
\pgfpathlineto{\pgfqpoint{4.395108in}{2.436071in}}%
\pgfpathclose%
\pgfusepath{fill}%
\end{pgfscope}%
\begin{pgfscope}%
\pgfpathrectangle{\pgfqpoint{1.254980in}{0.150000in}}{\pgfqpoint{5.490039in}{5.490039in}}%
\pgfusepath{clip}%
\pgfsetbuttcap%
\pgfsetroundjoin%
\definecolor{currentfill}{rgb}{0.135066,0.544853,0.554029}%
\pgfsetfillcolor{currentfill}%
\pgfsetfillopacity{0.700000}%
\pgfsetlinewidth{0.000000pt}%
\definecolor{currentstroke}{rgb}{0.000000,0.000000,0.000000}%
\pgfsetstrokecolor{currentstroke}%
\pgfsetdash{}{0pt}%
\pgfpathmoveto{\pgfqpoint{5.507247in}{3.122854in}}%
\pgfpathlineto{\pgfqpoint{5.521066in}{3.130134in}}%
\pgfpathlineto{\pgfqpoint{5.534902in}{3.137566in}}%
\pgfpathlineto{\pgfqpoint{5.548753in}{3.145151in}}%
\pgfpathlineto{\pgfqpoint{5.562622in}{3.152887in}}%
\pgfpathlineto{\pgfqpoint{5.569612in}{3.157328in}}%
\pgfpathlineto{\pgfqpoint{5.576599in}{3.161862in}}%
\pgfpathlineto{\pgfqpoint{5.583583in}{3.166496in}}%
\pgfpathlineto{\pgfqpoint{5.590564in}{3.171234in}}%
\pgfpathlineto{\pgfqpoint{5.576723in}{3.164052in}}%
\pgfpathlineto{\pgfqpoint{5.562898in}{3.157022in}}%
\pgfpathlineto{\pgfqpoint{5.549089in}{3.150143in}}%
\pgfpathlineto{\pgfqpoint{5.535296in}{3.143415in}}%
\pgfpathlineto{\pgfqpoint{5.528288in}{3.138113in}}%
\pgfpathlineto{\pgfqpoint{5.521277in}{3.132923in}}%
\pgfpathlineto{\pgfqpoint{5.514263in}{3.127839in}}%
\pgfpathlineto{\pgfqpoint{5.507247in}{3.122854in}}%
\pgfpathclose%
\pgfusepath{fill}%
\end{pgfscope}%
\begin{pgfscope}%
\pgfpathrectangle{\pgfqpoint{1.254980in}{0.150000in}}{\pgfqpoint{5.490039in}{5.490039in}}%
\pgfusepath{clip}%
\pgfsetbuttcap%
\pgfsetroundjoin%
\definecolor{currentfill}{rgb}{0.282327,0.094955,0.417331}%
\pgfsetfillcolor{currentfill}%
\pgfsetfillopacity{0.700000}%
\pgfsetlinewidth{0.000000pt}%
\definecolor{currentstroke}{rgb}{0.000000,0.000000,0.000000}%
\pgfsetstrokecolor{currentstroke}%
\pgfsetdash{}{0pt}%
\pgfpathmoveto{\pgfqpoint{3.729398in}{2.091873in}}%
\pgfpathlineto{\pgfqpoint{3.742499in}{2.090387in}}%
\pgfpathlineto{\pgfqpoint{3.755605in}{2.089081in}}%
\pgfpathlineto{\pgfqpoint{3.768717in}{2.087952in}}%
\pgfpathlineto{\pgfqpoint{3.781836in}{2.087001in}}%
\pgfpathlineto{\pgfqpoint{3.789553in}{2.096796in}}%
\pgfpathlineto{\pgfqpoint{3.797266in}{2.106585in}}%
\pgfpathlineto{\pgfqpoint{3.804973in}{2.116366in}}%
\pgfpathlineto{\pgfqpoint{3.812676in}{2.126138in}}%
\pgfpathlineto{\pgfqpoint{3.799566in}{2.127012in}}%
\pgfpathlineto{\pgfqpoint{3.786462in}{2.128063in}}%
\pgfpathlineto{\pgfqpoint{3.773365in}{2.129292in}}%
\pgfpathlineto{\pgfqpoint{3.760275in}{2.130700in}}%
\pgfpathlineto{\pgfqpoint{3.752563in}{2.120995in}}%
\pgfpathlineto{\pgfqpoint{3.744847in}{2.111288in}}%
\pgfpathlineto{\pgfqpoint{3.737125in}{2.101581in}}%
\pgfpathlineto{\pgfqpoint{3.729398in}{2.091873in}}%
\pgfpathclose%
\pgfusepath{fill}%
\end{pgfscope}%
\begin{pgfscope}%
\pgfpathrectangle{\pgfqpoint{1.254980in}{0.150000in}}{\pgfqpoint{5.490039in}{5.490039in}}%
\pgfusepath{clip}%
\pgfsetbuttcap%
\pgfsetroundjoin%
\definecolor{currentfill}{rgb}{0.128729,0.563265,0.551229}%
\pgfsetfillcolor{currentfill}%
\pgfsetfillopacity{0.700000}%
\pgfsetlinewidth{0.000000pt}%
\definecolor{currentstroke}{rgb}{0.000000,0.000000,0.000000}%
\pgfsetstrokecolor{currentstroke}%
\pgfsetdash{}{0pt}%
\pgfpathmoveto{\pgfqpoint{5.590564in}{3.171234in}}%
\pgfpathlineto{\pgfqpoint{5.604422in}{3.178567in}}%
\pgfpathlineto{\pgfqpoint{5.618296in}{3.186051in}}%
\pgfpathlineto{\pgfqpoint{5.632186in}{3.193686in}}%
\pgfpathlineto{\pgfqpoint{5.646093in}{3.201473in}}%
\pgfpathlineto{\pgfqpoint{5.653044in}{3.205749in}}%
\pgfpathlineto{\pgfqpoint{5.659992in}{3.210137in}}%
\pgfpathlineto{\pgfqpoint{5.666938in}{3.214643in}}%
\pgfpathlineto{\pgfqpoint{5.673883in}{3.219273in}}%
\pgfpathlineto{\pgfqpoint{5.660004in}{3.212070in}}%
\pgfpathlineto{\pgfqpoint{5.646143in}{3.205017in}}%
\pgfpathlineto{\pgfqpoint{5.632297in}{3.198116in}}%
\pgfpathlineto{\pgfqpoint{5.618468in}{3.191364in}}%
\pgfpathlineto{\pgfqpoint{5.611495in}{3.186142in}}%
\pgfpathlineto{\pgfqpoint{5.604520in}{3.181051in}}%
\pgfpathlineto{\pgfqpoint{5.597543in}{3.176084in}}%
\pgfpathlineto{\pgfqpoint{5.590564in}{3.171234in}}%
\pgfpathclose%
\pgfusepath{fill}%
\end{pgfscope}%
\begin{pgfscope}%
\pgfpathrectangle{\pgfqpoint{1.254980in}{0.150000in}}{\pgfqpoint{5.490039in}{5.490039in}}%
\pgfusepath{clip}%
\pgfsetbuttcap%
\pgfsetroundjoin%
\definecolor{currentfill}{rgb}{0.280267,0.073417,0.397163}%
\pgfsetfillcolor{currentfill}%
\pgfsetfillopacity{0.700000}%
\pgfsetlinewidth{0.000000pt}%
\definecolor{currentstroke}{rgb}{0.000000,0.000000,0.000000}%
\pgfsetstrokecolor{currentstroke}%
\pgfsetdash{}{0pt}%
\pgfpathmoveto{\pgfqpoint{3.154637in}{2.071326in}}%
\pgfpathlineto{\pgfqpoint{3.167711in}{2.062870in}}%
\pgfpathlineto{\pgfqpoint{3.180785in}{2.054623in}}%
\pgfpathlineto{\pgfqpoint{3.193858in}{2.046584in}}%
\pgfpathlineto{\pgfqpoint{3.206932in}{2.038752in}}%
\pgfpathlineto{\pgfqpoint{3.214869in}{2.046677in}}%
\pgfpathlineto{\pgfqpoint{3.222799in}{2.054675in}}%
\pgfpathlineto{\pgfqpoint{3.230723in}{2.062742in}}%
\pgfpathlineto{\pgfqpoint{3.238639in}{2.070877in}}%
\pgfpathlineto{\pgfqpoint{3.225583in}{2.078491in}}%
\pgfpathlineto{\pgfqpoint{3.212527in}{2.086312in}}%
\pgfpathlineto{\pgfqpoint{3.199472in}{2.094340in}}%
\pgfpathlineto{\pgfqpoint{3.186416in}{2.102576in}}%
\pgfpathlineto{\pgfqpoint{3.178483in}{2.094649in}}%
\pgfpathlineto{\pgfqpoint{3.170541in}{2.086797in}}%
\pgfpathlineto{\pgfqpoint{3.162593in}{2.079022in}}%
\pgfpathlineto{\pgfqpoint{3.154637in}{2.071326in}}%
\pgfpathclose%
\pgfusepath{fill}%
\end{pgfscope}%
\begin{pgfscope}%
\pgfpathrectangle{\pgfqpoint{1.254980in}{0.150000in}}{\pgfqpoint{5.490039in}{5.490039in}}%
\pgfusepath{clip}%
\pgfsetbuttcap%
\pgfsetroundjoin%
\definecolor{currentfill}{rgb}{0.243113,0.292092,0.538516}%
\pgfsetfillcolor{currentfill}%
\pgfsetfillopacity{0.700000}%
\pgfsetlinewidth{0.000000pt}%
\definecolor{currentstroke}{rgb}{0.000000,0.000000,0.000000}%
\pgfsetstrokecolor{currentstroke}%
\pgfsetdash{}{0pt}%
\pgfpathmoveto{\pgfqpoint{4.478358in}{2.486568in}}%
\pgfpathlineto{\pgfqpoint{4.491707in}{2.490882in}}%
\pgfpathlineto{\pgfqpoint{4.505067in}{2.495358in}}%
\pgfpathlineto{\pgfqpoint{4.518440in}{2.499997in}}%
\pgfpathlineto{\pgfqpoint{4.531824in}{2.504797in}}%
\pgfpathlineto{\pgfqpoint{4.539285in}{2.513111in}}%
\pgfpathlineto{\pgfqpoint{4.546740in}{2.521385in}}%
\pgfpathlineto{\pgfqpoint{4.554190in}{2.529622in}}%
\pgfpathlineto{\pgfqpoint{4.561634in}{2.537824in}}%
\pgfpathlineto{\pgfqpoint{4.548259in}{2.533201in}}%
\pgfpathlineto{\pgfqpoint{4.534896in}{2.528739in}}%
\pgfpathlineto{\pgfqpoint{4.521544in}{2.524440in}}%
\pgfpathlineto{\pgfqpoint{4.508204in}{2.520303in}}%
\pgfpathlineto{\pgfqpoint{4.500751in}{2.511913in}}%
\pgfpathlineto{\pgfqpoint{4.493292in}{2.503495in}}%
\pgfpathlineto{\pgfqpoint{4.485828in}{2.495048in}}%
\pgfpathlineto{\pgfqpoint{4.478358in}{2.486568in}}%
\pgfpathclose%
\pgfusepath{fill}%
\end{pgfscope}%
\begin{pgfscope}%
\pgfpathrectangle{\pgfqpoint{1.254980in}{0.150000in}}{\pgfqpoint{5.490039in}{5.490039in}}%
\pgfusepath{clip}%
\pgfsetbuttcap%
\pgfsetroundjoin%
\definecolor{currentfill}{rgb}{0.123463,0.581687,0.547445}%
\pgfsetfillcolor{currentfill}%
\pgfsetfillopacity{0.700000}%
\pgfsetlinewidth{0.000000pt}%
\definecolor{currentstroke}{rgb}{0.000000,0.000000,0.000000}%
\pgfsetstrokecolor{currentstroke}%
\pgfsetdash{}{0pt}%
\pgfpathmoveto{\pgfqpoint{5.673883in}{3.219273in}}%
\pgfpathlineto{\pgfqpoint{5.687777in}{3.226626in}}%
\pgfpathlineto{\pgfqpoint{5.701689in}{3.234130in}}%
\pgfpathlineto{\pgfqpoint{5.715617in}{3.241785in}}%
\pgfpathlineto{\pgfqpoint{5.729563in}{3.249590in}}%
\pgfpathlineto{\pgfqpoint{5.736475in}{3.253748in}}%
\pgfpathlineto{\pgfqpoint{5.743385in}{3.258037in}}%
\pgfpathlineto{\pgfqpoint{5.750295in}{3.262464in}}%
\pgfpathlineto{\pgfqpoint{5.757203in}{3.267034in}}%
\pgfpathlineto{\pgfqpoint{5.743289in}{3.259841in}}%
\pgfpathlineto{\pgfqpoint{5.729391in}{3.252798in}}%
\pgfpathlineto{\pgfqpoint{5.715510in}{3.245905in}}%
\pgfpathlineto{\pgfqpoint{5.701645in}{3.239162in}}%
\pgfpathlineto{\pgfqpoint{5.694706in}{3.233971in}}%
\pgfpathlineto{\pgfqpoint{5.687766in}{3.228930in}}%
\pgfpathlineto{\pgfqpoint{5.680825in}{3.224033in}}%
\pgfpathlineto{\pgfqpoint{5.673883in}{3.219273in}}%
\pgfpathclose%
\pgfusepath{fill}%
\end{pgfscope}%
\begin{pgfscope}%
\pgfpathrectangle{\pgfqpoint{1.254980in}{0.150000in}}{\pgfqpoint{5.490039in}{5.490039in}}%
\pgfusepath{clip}%
\pgfsetbuttcap%
\pgfsetroundjoin%
\definecolor{currentfill}{rgb}{0.277941,0.056324,0.381191}%
\pgfsetfillcolor{currentfill}%
\pgfsetfillopacity{0.700000}%
\pgfsetlinewidth{0.000000pt}%
\definecolor{currentstroke}{rgb}{0.000000,0.000000,0.000000}%
\pgfsetstrokecolor{currentstroke}%
\pgfsetdash{}{0pt}%
\pgfpathmoveto{\pgfqpoint{3.426817in}{2.030646in}}%
\pgfpathlineto{\pgfqpoint{3.439879in}{2.025801in}}%
\pgfpathlineto{\pgfqpoint{3.452945in}{2.021147in}}%
\pgfpathlineto{\pgfqpoint{3.466014in}{2.016683in}}%
\pgfpathlineto{\pgfqpoint{3.479087in}{2.012409in}}%
\pgfpathlineto{\pgfqpoint{3.486911in}{2.021549in}}%
\pgfpathlineto{\pgfqpoint{3.494730in}{2.030720in}}%
\pgfpathlineto{\pgfqpoint{3.502543in}{2.039921in}}%
\pgfpathlineto{\pgfqpoint{3.510351in}{2.049150in}}%
\pgfpathlineto{\pgfqpoint{3.497291in}{2.053263in}}%
\pgfpathlineto{\pgfqpoint{3.484235in}{2.057565in}}%
\pgfpathlineto{\pgfqpoint{3.471182in}{2.062058in}}%
\pgfpathlineto{\pgfqpoint{3.458133in}{2.066742in}}%
\pgfpathlineto{\pgfqpoint{3.450313in}{2.057664in}}%
\pgfpathlineto{\pgfqpoint{3.442487in}{2.048621in}}%
\pgfpathlineto{\pgfqpoint{3.434655in}{2.039614in}}%
\pgfpathlineto{\pgfqpoint{3.426817in}{2.030646in}}%
\pgfpathclose%
\pgfusepath{fill}%
\end{pgfscope}%
\begin{pgfscope}%
\pgfpathrectangle{\pgfqpoint{1.254980in}{0.150000in}}{\pgfqpoint{5.490039in}{5.490039in}}%
\pgfusepath{clip}%
\pgfsetbuttcap%
\pgfsetroundjoin%
\definecolor{currentfill}{rgb}{0.120092,0.600104,0.542530}%
\pgfsetfillcolor{currentfill}%
\pgfsetfillopacity{0.700000}%
\pgfsetlinewidth{0.000000pt}%
\definecolor{currentstroke}{rgb}{0.000000,0.000000,0.000000}%
\pgfsetstrokecolor{currentstroke}%
\pgfsetdash{}{0pt}%
\pgfpathmoveto{\pgfqpoint{5.757203in}{3.267034in}}%
\pgfpathlineto{\pgfqpoint{5.771134in}{3.274377in}}%
\pgfpathlineto{\pgfqpoint{5.785082in}{3.281869in}}%
\pgfpathlineto{\pgfqpoint{5.799047in}{3.289512in}}%
\pgfpathlineto{\pgfqpoint{5.813029in}{3.297305in}}%
\pgfpathlineto{\pgfqpoint{5.819904in}{3.301395in}}%
\pgfpathlineto{\pgfqpoint{5.826779in}{3.305637in}}%
\pgfpathlineto{\pgfqpoint{5.833653in}{3.310037in}}%
\pgfpathlineto{\pgfqpoint{5.840527in}{3.314603in}}%
\pgfpathlineto{\pgfqpoint{5.826578in}{3.307451in}}%
\pgfpathlineto{\pgfqpoint{5.812645in}{3.300449in}}%
\pgfpathlineto{\pgfqpoint{5.798730in}{3.293596in}}%
\pgfpathlineto{\pgfqpoint{5.784831in}{3.286893in}}%
\pgfpathlineto{\pgfqpoint{5.777924in}{3.281677in}}%
\pgfpathlineto{\pgfqpoint{5.771017in}{3.276634in}}%
\pgfpathlineto{\pgfqpoint{5.764110in}{3.271755in}}%
\pgfpathlineto{\pgfqpoint{5.757203in}{3.267034in}}%
\pgfpathclose%
\pgfusepath{fill}%
\end{pgfscope}%
\begin{pgfscope}%
\pgfpathrectangle{\pgfqpoint{1.254980in}{0.150000in}}{\pgfqpoint{5.490039in}{5.490039in}}%
\pgfusepath{clip}%
\pgfsetbuttcap%
\pgfsetroundjoin%
\definecolor{currentfill}{rgb}{0.283229,0.120777,0.440584}%
\pgfsetfillcolor{currentfill}%
\pgfsetfillopacity{0.700000}%
\pgfsetlinewidth{0.000000pt}%
\definecolor{currentstroke}{rgb}{0.000000,0.000000,0.000000}%
\pgfsetstrokecolor{currentstroke}%
\pgfsetdash{}{0pt}%
\pgfpathmoveto{\pgfqpoint{2.965510in}{2.162608in}}%
\pgfpathlineto{\pgfqpoint{2.978624in}{2.151287in}}%
\pgfpathlineto{\pgfqpoint{2.991735in}{2.140193in}}%
\pgfpathlineto{\pgfqpoint{3.004844in}{2.129324in}}%
\pgfpathlineto{\pgfqpoint{3.017951in}{2.118678in}}%
\pgfpathlineto{\pgfqpoint{3.025979in}{2.125522in}}%
\pgfpathlineto{\pgfqpoint{3.034000in}{2.132467in}}%
\pgfpathlineto{\pgfqpoint{3.042012in}{2.139511in}}%
\pgfpathlineto{\pgfqpoint{3.050016in}{2.146652in}}%
\pgfpathlineto{\pgfqpoint{3.036931in}{2.157049in}}%
\pgfpathlineto{\pgfqpoint{3.023845in}{2.167669in}}%
\pgfpathlineto{\pgfqpoint{3.010756in}{2.178514in}}%
\pgfpathlineto{\pgfqpoint{2.997664in}{2.189584in}}%
\pgfpathlineto{\pgfqpoint{2.989638in}{2.182682in}}%
\pgfpathlineto{\pgfqpoint{2.981604in}{2.175884in}}%
\pgfpathlineto{\pgfqpoint{2.973561in}{2.169192in}}%
\pgfpathlineto{\pgfqpoint{2.965510in}{2.162608in}}%
\pgfpathclose%
\pgfusepath{fill}%
\end{pgfscope}%
\begin{pgfscope}%
\pgfpathrectangle{\pgfqpoint{1.254980in}{0.150000in}}{\pgfqpoint{5.490039in}{5.490039in}}%
\pgfusepath{clip}%
\pgfsetbuttcap%
\pgfsetroundjoin%
\definecolor{currentfill}{rgb}{0.231674,0.318106,0.544834}%
\pgfsetfillcolor{currentfill}%
\pgfsetfillopacity{0.700000}%
\pgfsetlinewidth{0.000000pt}%
\definecolor{currentstroke}{rgb}{0.000000,0.000000,0.000000}%
\pgfsetstrokecolor{currentstroke}%
\pgfsetdash{}{0pt}%
\pgfpathmoveto{\pgfqpoint{4.561634in}{2.537824in}}%
\pgfpathlineto{\pgfqpoint{4.575021in}{2.542609in}}%
\pgfpathlineto{\pgfqpoint{4.588421in}{2.547556in}}%
\pgfpathlineto{\pgfqpoint{4.601832in}{2.552664in}}%
\pgfpathlineto{\pgfqpoint{4.615256in}{2.557933in}}%
\pgfpathlineto{\pgfqpoint{4.622685in}{2.565908in}}%
\pgfpathlineto{\pgfqpoint{4.630108in}{2.573846in}}%
\pgfpathlineto{\pgfqpoint{4.637526in}{2.581750in}}%
\pgfpathlineto{\pgfqpoint{4.644938in}{2.589622in}}%
\pgfpathlineto{\pgfqpoint{4.631524in}{2.584559in}}%
\pgfpathlineto{\pgfqpoint{4.618123in}{2.579657in}}%
\pgfpathlineto{\pgfqpoint{4.604733in}{2.574916in}}%
\pgfpathlineto{\pgfqpoint{4.591356in}{2.570337in}}%
\pgfpathlineto{\pgfqpoint{4.583934in}{2.562248in}}%
\pgfpathlineto{\pgfqpoint{4.576506in}{2.554135in}}%
\pgfpathlineto{\pgfqpoint{4.569073in}{2.545995in}}%
\pgfpathlineto{\pgfqpoint{4.561634in}{2.537824in}}%
\pgfpathclose%
\pgfusepath{fill}%
\end{pgfscope}%
\begin{pgfscope}%
\pgfpathrectangle{\pgfqpoint{1.254980in}{0.150000in}}{\pgfqpoint{5.490039in}{5.490039in}}%
\pgfusepath{clip}%
\pgfsetbuttcap%
\pgfsetroundjoin%
\definecolor{currentfill}{rgb}{0.119699,0.618490,0.536347}%
\pgfsetfillcolor{currentfill}%
\pgfsetfillopacity{0.700000}%
\pgfsetlinewidth{0.000000pt}%
\definecolor{currentstroke}{rgb}{0.000000,0.000000,0.000000}%
\pgfsetstrokecolor{currentstroke}%
\pgfsetdash{}{0pt}%
\pgfpathmoveto{\pgfqpoint{5.840527in}{3.314603in}}%
\pgfpathlineto{\pgfqpoint{5.854493in}{3.321903in}}%
\pgfpathlineto{\pgfqpoint{5.868476in}{3.329353in}}%
\pgfpathlineto{\pgfqpoint{5.882477in}{3.336953in}}%
\pgfpathlineto{\pgfqpoint{5.896494in}{3.344701in}}%
\pgfpathlineto{\pgfqpoint{5.903334in}{3.348780in}}%
\pgfpathlineto{\pgfqpoint{5.910174in}{3.353032in}}%
\pgfpathlineto{\pgfqpoint{5.917015in}{3.357465in}}%
\pgfpathlineto{\pgfqpoint{5.923857in}{3.362086in}}%
\pgfpathlineto{\pgfqpoint{5.909874in}{3.355007in}}%
\pgfpathlineto{\pgfqpoint{5.895909in}{3.348077in}}%
\pgfpathlineto{\pgfqpoint{5.881960in}{3.341295in}}%
\pgfpathlineto{\pgfqpoint{5.868028in}{3.334662in}}%
\pgfpathlineto{\pgfqpoint{5.861151in}{3.329363in}}%
\pgfpathlineto{\pgfqpoint{5.854276in}{3.324259in}}%
\pgfpathlineto{\pgfqpoint{5.847401in}{3.319341in}}%
\pgfpathlineto{\pgfqpoint{5.840527in}{3.314603in}}%
\pgfpathclose%
\pgfusepath{fill}%
\end{pgfscope}%
\begin{pgfscope}%
\pgfpathrectangle{\pgfqpoint{1.254980in}{0.150000in}}{\pgfqpoint{5.490039in}{5.490039in}}%
\pgfusepath{clip}%
\pgfsetbuttcap%
\pgfsetroundjoin%
\definecolor{currentfill}{rgb}{0.204903,0.375746,0.553533}%
\pgfsetfillcolor{currentfill}%
\pgfsetfillopacity{0.700000}%
\pgfsetlinewidth{0.000000pt}%
\definecolor{currentstroke}{rgb}{0.000000,0.000000,0.000000}%
\pgfsetstrokecolor{currentstroke}%
\pgfsetdash{}{0pt}%
\pgfpathmoveto{\pgfqpoint{2.489540in}{2.737068in}}%
\pgfpathlineto{\pgfqpoint{2.502917in}{2.716247in}}%
\pgfpathlineto{\pgfqpoint{2.516282in}{2.695735in}}%
\pgfpathlineto{\pgfqpoint{2.529635in}{2.675528in}}%
\pgfpathlineto{\pgfqpoint{2.542976in}{2.655622in}}%
\pgfpathlineto{\pgfqpoint{2.551252in}{2.660069in}}%
\pgfpathlineto{\pgfqpoint{2.559516in}{2.664683in}}%
\pgfpathlineto{\pgfqpoint{2.567767in}{2.669460in}}%
\pgfpathlineto{\pgfqpoint{2.576007in}{2.674399in}}%
\pgfpathlineto{\pgfqpoint{2.562698in}{2.694040in}}%
\pgfpathlineto{\pgfqpoint{2.549378in}{2.713982in}}%
\pgfpathlineto{\pgfqpoint{2.536046in}{2.734228in}}%
\pgfpathlineto{\pgfqpoint{2.522703in}{2.754782in}}%
\pgfpathlineto{\pgfqpoint{2.514431in}{2.750097in}}%
\pgfpathlineto{\pgfqpoint{2.506146in}{2.745582in}}%
\pgfpathlineto{\pgfqpoint{2.497849in}{2.741238in}}%
\pgfpathlineto{\pgfqpoint{2.489540in}{2.737068in}}%
\pgfpathclose%
\pgfusepath{fill}%
\end{pgfscope}%
\begin{pgfscope}%
\pgfpathrectangle{\pgfqpoint{1.254980in}{0.150000in}}{\pgfqpoint{5.490039in}{5.490039in}}%
\pgfusepath{clip}%
\pgfsetbuttcap%
\pgfsetroundjoin%
\definecolor{currentfill}{rgb}{0.280894,0.078907,0.402329}%
\pgfsetfillcolor{currentfill}%
\pgfsetfillopacity{0.700000}%
\pgfsetlinewidth{0.000000pt}%
\definecolor{currentstroke}{rgb}{0.000000,0.000000,0.000000}%
\pgfsetstrokecolor{currentstroke}%
\pgfsetdash{}{0pt}%
\pgfpathmoveto{\pgfqpoint{3.646057in}{2.061211in}}%
\pgfpathlineto{\pgfqpoint{3.659145in}{2.058899in}}%
\pgfpathlineto{\pgfqpoint{3.672238in}{2.056768in}}%
\pgfpathlineto{\pgfqpoint{3.685336in}{2.054818in}}%
\pgfpathlineto{\pgfqpoint{3.698440in}{2.053048in}}%
\pgfpathlineto{\pgfqpoint{3.706187in}{2.062752in}}%
\pgfpathlineto{\pgfqpoint{3.713930in}{2.072458in}}%
\pgfpathlineto{\pgfqpoint{3.721667in}{2.082165in}}%
\pgfpathlineto{\pgfqpoint{3.729398in}{2.091873in}}%
\pgfpathlineto{\pgfqpoint{3.716304in}{2.093537in}}%
\pgfpathlineto{\pgfqpoint{3.703216in}{2.095382in}}%
\pgfpathlineto{\pgfqpoint{3.690132in}{2.097408in}}%
\pgfpathlineto{\pgfqpoint{3.677055in}{2.099614in}}%
\pgfpathlineto{\pgfqpoint{3.669313in}{2.090002in}}%
\pgfpathlineto{\pgfqpoint{3.661566in}{2.080397in}}%
\pgfpathlineto{\pgfqpoint{3.653814in}{2.070800in}}%
\pgfpathlineto{\pgfqpoint{3.646057in}{2.061211in}}%
\pgfpathclose%
\pgfusepath{fill}%
\end{pgfscope}%
\begin{pgfscope}%
\pgfpathrectangle{\pgfqpoint{1.254980in}{0.150000in}}{\pgfqpoint{5.490039in}{5.490039in}}%
\pgfusepath{clip}%
\pgfsetbuttcap%
\pgfsetroundjoin%
\definecolor{currentfill}{rgb}{0.221989,0.339161,0.548752}%
\pgfsetfillcolor{currentfill}%
\pgfsetfillopacity{0.700000}%
\pgfsetlinewidth{0.000000pt}%
\definecolor{currentstroke}{rgb}{0.000000,0.000000,0.000000}%
\pgfsetstrokecolor{currentstroke}%
\pgfsetdash{}{0pt}%
\pgfpathmoveto{\pgfqpoint{4.644938in}{2.589622in}}%
\pgfpathlineto{\pgfqpoint{4.658365in}{2.594845in}}%
\pgfpathlineto{\pgfqpoint{4.671804in}{2.600229in}}%
\pgfpathlineto{\pgfqpoint{4.685256in}{2.605773in}}%
\pgfpathlineto{\pgfqpoint{4.698720in}{2.611477in}}%
\pgfpathlineto{\pgfqpoint{4.706116in}{2.619095in}}%
\pgfpathlineto{\pgfqpoint{4.713507in}{2.626681in}}%
\pgfpathlineto{\pgfqpoint{4.720891in}{2.634236in}}%
\pgfpathlineto{\pgfqpoint{4.728270in}{2.641764in}}%
\pgfpathlineto{\pgfqpoint{4.714816in}{2.636295in}}%
\pgfpathlineto{\pgfqpoint{4.701375in}{2.630986in}}%
\pgfpathlineto{\pgfqpoint{4.687947in}{2.625837in}}%
\pgfpathlineto{\pgfqpoint{4.674531in}{2.620848in}}%
\pgfpathlineto{\pgfqpoint{4.667141in}{2.613076in}}%
\pgfpathlineto{\pgfqpoint{4.659746in}{2.605282in}}%
\pgfpathlineto{\pgfqpoint{4.652345in}{2.597465in}}%
\pgfpathlineto{\pgfqpoint{4.644938in}{2.589622in}}%
\pgfpathclose%
\pgfusepath{fill}%
\end{pgfscope}%
\begin{pgfscope}%
\pgfpathrectangle{\pgfqpoint{1.254980in}{0.150000in}}{\pgfqpoint{5.490039in}{5.490039in}}%
\pgfusepath{clip}%
\pgfsetbuttcap%
\pgfsetroundjoin%
\definecolor{currentfill}{rgb}{0.123444,0.636809,0.528763}%
\pgfsetfillcolor{currentfill}%
\pgfsetfillopacity{0.700000}%
\pgfsetlinewidth{0.000000pt}%
\definecolor{currentstroke}{rgb}{0.000000,0.000000,0.000000}%
\pgfsetstrokecolor{currentstroke}%
\pgfsetdash{}{0pt}%
\pgfpathmoveto{\pgfqpoint{5.923857in}{3.362086in}}%
\pgfpathlineto{\pgfqpoint{5.937857in}{3.369313in}}%
\pgfpathlineto{\pgfqpoint{5.951874in}{3.376689in}}%
\pgfpathlineto{\pgfqpoint{5.965908in}{3.384214in}}%
\pgfpathlineto{\pgfqpoint{5.979960in}{3.391887in}}%
\pgfpathlineto{\pgfqpoint{5.986767in}{3.396015in}}%
\pgfpathlineto{\pgfqpoint{5.993575in}{3.400340in}}%
\pgfpathlineto{\pgfqpoint{6.000385in}{3.404869in}}%
\pgfpathlineto{\pgfqpoint{5.986361in}{3.397717in}}%
\pgfpathlineto{\pgfqpoint{5.972354in}{3.390713in}}%
\pgfpathlineto{\pgfqpoint{5.958365in}{3.383857in}}%
\pgfpathlineto{\pgfqpoint{5.944392in}{3.377149in}}%
\pgfpathlineto{\pgfqpoint{5.937545in}{3.371920in}}%
\pgfpathlineto{\pgfqpoint{5.930700in}{3.366902in}}%
\pgfpathlineto{\pgfqpoint{5.923857in}{3.362086in}}%
\pgfpathclose%
\pgfusepath{fill}%
\end{pgfscope}%
\begin{pgfscope}%
\pgfpathrectangle{\pgfqpoint{1.254980in}{0.150000in}}{\pgfqpoint{5.490039in}{5.490039in}}%
\pgfusepath{clip}%
\pgfsetbuttcap%
\pgfsetroundjoin%
\definecolor{currentfill}{rgb}{0.210503,0.363727,0.552206}%
\pgfsetfillcolor{currentfill}%
\pgfsetfillopacity{0.700000}%
\pgfsetlinewidth{0.000000pt}%
\definecolor{currentstroke}{rgb}{0.000000,0.000000,0.000000}%
\pgfsetstrokecolor{currentstroke}%
\pgfsetdash{}{0pt}%
\pgfpathmoveto{\pgfqpoint{4.728270in}{2.641764in}}%
\pgfpathlineto{\pgfqpoint{4.741737in}{2.647392in}}%
\pgfpathlineto{\pgfqpoint{4.755217in}{2.653180in}}%
\pgfpathlineto{\pgfqpoint{4.768710in}{2.659127in}}%
\pgfpathlineto{\pgfqpoint{4.782217in}{2.665233in}}%
\pgfpathlineto{\pgfqpoint{4.789578in}{2.672483in}}%
\pgfpathlineto{\pgfqpoint{4.796934in}{2.679704in}}%
\pgfpathlineto{\pgfqpoint{4.804284in}{2.686900in}}%
\pgfpathlineto{\pgfqpoint{4.811629in}{2.694073in}}%
\pgfpathlineto{\pgfqpoint{4.798134in}{2.688232in}}%
\pgfpathlineto{\pgfqpoint{4.784653in}{2.682549in}}%
\pgfpathlineto{\pgfqpoint{4.771184in}{2.677025in}}%
\pgfpathlineto{\pgfqpoint{4.757729in}{2.671660in}}%
\pgfpathlineto{\pgfqpoint{4.750373in}{2.664212in}}%
\pgfpathlineto{\pgfqpoint{4.743011in}{2.656749in}}%
\pgfpathlineto{\pgfqpoint{4.735643in}{2.649267in}}%
\pgfpathlineto{\pgfqpoint{4.728270in}{2.641764in}}%
\pgfpathclose%
\pgfusepath{fill}%
\end{pgfscope}%
\begin{pgfscope}%
\pgfpathrectangle{\pgfqpoint{1.254980in}{0.150000in}}{\pgfqpoint{5.490039in}{5.490039in}}%
\pgfusepath{clip}%
\pgfsetbuttcap%
\pgfsetroundjoin%
\definecolor{currentfill}{rgb}{0.282656,0.100196,0.422160}%
\pgfsetfillcolor{currentfill}%
\pgfsetfillopacity{0.700000}%
\pgfsetlinewidth{0.000000pt}%
\definecolor{currentstroke}{rgb}{0.000000,0.000000,0.000000}%
\pgfsetstrokecolor{currentstroke}%
\pgfsetdash{}{0pt}%
\pgfpathmoveto{\pgfqpoint{3.017951in}{2.118678in}}%
\pgfpathlineto{\pgfqpoint{3.031055in}{2.108254in}}%
\pgfpathlineto{\pgfqpoint{3.044157in}{2.098050in}}%
\pgfpathlineto{\pgfqpoint{3.057257in}{2.088064in}}%
\pgfpathlineto{\pgfqpoint{3.070356in}{2.078296in}}%
\pgfpathlineto{\pgfqpoint{3.078363in}{2.085399in}}%
\pgfpathlineto{\pgfqpoint{3.086362in}{2.092596in}}%
\pgfpathlineto{\pgfqpoint{3.094353in}{2.099885in}}%
\pgfpathlineto{\pgfqpoint{3.102336in}{2.107264in}}%
\pgfpathlineto{\pgfqpoint{3.089258in}{2.116784in}}%
\pgfpathlineto{\pgfqpoint{3.076179in}{2.126521in}}%
\pgfpathlineto{\pgfqpoint{3.063098in}{2.136477in}}%
\pgfpathlineto{\pgfqpoint{3.050016in}{2.146652in}}%
\pgfpathlineto{\pgfqpoint{3.042012in}{2.139511in}}%
\pgfpathlineto{\pgfqpoint{3.034000in}{2.132467in}}%
\pgfpathlineto{\pgfqpoint{3.025979in}{2.125522in}}%
\pgfpathlineto{\pgfqpoint{3.017951in}{2.118678in}}%
\pgfpathclose%
\pgfusepath{fill}%
\end{pgfscope}%
\begin{pgfscope}%
\pgfpathrectangle{\pgfqpoint{1.254980in}{0.150000in}}{\pgfqpoint{5.490039in}{5.490039in}}%
\pgfusepath{clip}%
\pgfsetbuttcap%
\pgfsetroundjoin%
\definecolor{currentfill}{rgb}{0.279566,0.067836,0.391917}%
\pgfsetfillcolor{currentfill}%
\pgfsetfillopacity{0.700000}%
\pgfsetlinewidth{0.000000pt}%
\definecolor{currentstroke}{rgb}{0.000000,0.000000,0.000000}%
\pgfsetstrokecolor{currentstroke}%
\pgfsetdash{}{0pt}%
\pgfpathmoveto{\pgfqpoint{3.562628in}{2.034575in}}%
\pgfpathlineto{\pgfqpoint{3.575707in}{2.031397in}}%
\pgfpathlineto{\pgfqpoint{3.588792in}{2.028403in}}%
\pgfpathlineto{\pgfqpoint{3.601880in}{2.025592in}}%
\pgfpathlineto{\pgfqpoint{3.614974in}{2.022964in}}%
\pgfpathlineto{\pgfqpoint{3.622752in}{2.032508in}}%
\pgfpathlineto{\pgfqpoint{3.630526in}{2.042065in}}%
\pgfpathlineto{\pgfqpoint{3.638294in}{2.051633in}}%
\pgfpathlineto{\pgfqpoint{3.646057in}{2.061211in}}%
\pgfpathlineto{\pgfqpoint{3.632974in}{2.063706in}}%
\pgfpathlineto{\pgfqpoint{3.619896in}{2.066383in}}%
\pgfpathlineto{\pgfqpoint{3.606823in}{2.069244in}}%
\pgfpathlineto{\pgfqpoint{3.593755in}{2.072289in}}%
\pgfpathlineto{\pgfqpoint{3.585981in}{2.062834in}}%
\pgfpathlineto{\pgfqpoint{3.578202in}{2.053396in}}%
\pgfpathlineto{\pgfqpoint{3.570418in}{2.043976in}}%
\pgfpathlineto{\pgfqpoint{3.562628in}{2.034575in}}%
\pgfpathclose%
\pgfusepath{fill}%
\end{pgfscope}%
\begin{pgfscope}%
\pgfpathrectangle{\pgfqpoint{1.254980in}{0.150000in}}{\pgfqpoint{5.490039in}{5.490039in}}%
\pgfusepath{clip}%
\pgfsetbuttcap%
\pgfsetroundjoin%
\definecolor{currentfill}{rgb}{0.199430,0.387607,0.554642}%
\pgfsetfillcolor{currentfill}%
\pgfsetfillopacity{0.700000}%
\pgfsetlinewidth{0.000000pt}%
\definecolor{currentstroke}{rgb}{0.000000,0.000000,0.000000}%
\pgfsetstrokecolor{currentstroke}%
\pgfsetdash{}{0pt}%
\pgfpathmoveto{\pgfqpoint{4.811629in}{2.694073in}}%
\pgfpathlineto{\pgfqpoint{4.825137in}{2.700073in}}%
\pgfpathlineto{\pgfqpoint{4.838659in}{2.706232in}}%
\pgfpathlineto{\pgfqpoint{4.852194in}{2.712550in}}%
\pgfpathlineto{\pgfqpoint{4.865743in}{2.719025in}}%
\pgfpathlineto{\pgfqpoint{4.873069in}{2.725897in}}%
\pgfpathlineto{\pgfqpoint{4.880389in}{2.732746in}}%
\pgfpathlineto{\pgfqpoint{4.887704in}{2.739577in}}%
\pgfpathlineto{\pgfqpoint{4.895012in}{2.746392in}}%
\pgfpathlineto{\pgfqpoint{4.881476in}{2.740210in}}%
\pgfpathlineto{\pgfqpoint{4.867954in}{2.734186in}}%
\pgfpathlineto{\pgfqpoint{4.854445in}{2.728320in}}%
\pgfpathlineto{\pgfqpoint{4.840949in}{2.722612in}}%
\pgfpathlineto{\pgfqpoint{4.833628in}{2.715494in}}%
\pgfpathlineto{\pgfqpoint{4.826300in}{2.708367in}}%
\pgfpathlineto{\pgfqpoint{4.818967in}{2.701228in}}%
\pgfpathlineto{\pgfqpoint{4.811629in}{2.694073in}}%
\pgfpathclose%
\pgfusepath{fill}%
\end{pgfscope}%
\begin{pgfscope}%
\pgfpathrectangle{\pgfqpoint{1.254980in}{0.150000in}}{\pgfqpoint{5.490039in}{5.490039in}}%
\pgfusepath{clip}%
\pgfsetbuttcap%
\pgfsetroundjoin%
\definecolor{currentfill}{rgb}{0.277018,0.050344,0.375715}%
\pgfsetfillcolor{currentfill}%
\pgfsetfillopacity{0.700000}%
\pgfsetlinewidth{0.000000pt}%
\definecolor{currentstroke}{rgb}{0.000000,0.000000,0.000000}%
\pgfsetstrokecolor{currentstroke}%
\pgfsetdash{}{0pt}%
\pgfpathmoveto{\pgfqpoint{3.343121in}{2.017250in}}%
\pgfpathlineto{\pgfqpoint{3.356189in}{2.011441in}}%
\pgfpathlineto{\pgfqpoint{3.369258in}{2.005827in}}%
\pgfpathlineto{\pgfqpoint{3.382330in}{2.000407in}}%
\pgfpathlineto{\pgfqpoint{3.395404in}{1.995180in}}%
\pgfpathlineto{\pgfqpoint{3.403266in}{2.003983in}}%
\pgfpathlineto{\pgfqpoint{3.411123in}{2.012829in}}%
\pgfpathlineto{\pgfqpoint{3.418973in}{2.021717in}}%
\pgfpathlineto{\pgfqpoint{3.426817in}{2.030646in}}%
\pgfpathlineto{\pgfqpoint{3.413757in}{2.035683in}}%
\pgfpathlineto{\pgfqpoint{3.400699in}{2.040914in}}%
\pgfpathlineto{\pgfqpoint{3.387644in}{2.046339in}}%
\pgfpathlineto{\pgfqpoint{3.374592in}{2.051958in}}%
\pgfpathlineto{\pgfqpoint{3.366734in}{2.043208in}}%
\pgfpathlineto{\pgfqpoint{3.358869in}{2.034506in}}%
\pgfpathlineto{\pgfqpoint{3.350998in}{2.025853in}}%
\pgfpathlineto{\pgfqpoint{3.343121in}{2.017250in}}%
\pgfpathclose%
\pgfusepath{fill}%
\end{pgfscope}%
\begin{pgfscope}%
\pgfpathrectangle{\pgfqpoint{1.254980in}{0.150000in}}{\pgfqpoint{5.490039in}{5.490039in}}%
\pgfusepath{clip}%
\pgfsetbuttcap%
\pgfsetroundjoin%
\definecolor{currentfill}{rgb}{0.278791,0.062145,0.386592}%
\pgfsetfillcolor{currentfill}%
\pgfsetfillopacity{0.700000}%
\pgfsetlinewidth{0.000000pt}%
\definecolor{currentstroke}{rgb}{0.000000,0.000000,0.000000}%
\pgfsetstrokecolor{currentstroke}%
\pgfsetdash{}{0pt}%
\pgfpathmoveto{\pgfqpoint{3.206932in}{2.038752in}}%
\pgfpathlineto{\pgfqpoint{3.220006in}{2.031124in}}%
\pgfpathlineto{\pgfqpoint{3.233080in}{2.023701in}}%
\pgfpathlineto{\pgfqpoint{3.246156in}{2.016481in}}%
\pgfpathlineto{\pgfqpoint{3.259232in}{2.009462in}}%
\pgfpathlineto{\pgfqpoint{3.267151in}{2.017616in}}%
\pgfpathlineto{\pgfqpoint{3.275064in}{2.025835in}}%
\pgfpathlineto{\pgfqpoint{3.282971in}{2.034117in}}%
\pgfpathlineto{\pgfqpoint{3.290870in}{2.042460in}}%
\pgfpathlineto{\pgfqpoint{3.277811in}{2.049261in}}%
\pgfpathlineto{\pgfqpoint{3.264753in}{2.056263in}}%
\pgfpathlineto{\pgfqpoint{3.251695in}{2.063468in}}%
\pgfpathlineto{\pgfqpoint{3.238639in}{2.070877in}}%
\pgfpathlineto{\pgfqpoint{3.230723in}{2.062742in}}%
\pgfpathlineto{\pgfqpoint{3.222799in}{2.054675in}}%
\pgfpathlineto{\pgfqpoint{3.214869in}{2.046677in}}%
\pgfpathlineto{\pgfqpoint{3.206932in}{2.038752in}}%
\pgfpathclose%
\pgfusepath{fill}%
\end{pgfscope}%
\begin{pgfscope}%
\pgfpathrectangle{\pgfqpoint{1.254980in}{0.150000in}}{\pgfqpoint{5.490039in}{5.490039in}}%
\pgfusepath{clip}%
\pgfsetbuttcap%
\pgfsetroundjoin%
\definecolor{currentfill}{rgb}{0.190631,0.407061,0.556089}%
\pgfsetfillcolor{currentfill}%
\pgfsetfillopacity{0.700000}%
\pgfsetlinewidth{0.000000pt}%
\definecolor{currentstroke}{rgb}{0.000000,0.000000,0.000000}%
\pgfsetstrokecolor{currentstroke}%
\pgfsetdash{}{0pt}%
\pgfpathmoveto{\pgfqpoint{4.895012in}{2.746392in}}%
\pgfpathlineto{\pgfqpoint{4.908562in}{2.752731in}}%
\pgfpathlineto{\pgfqpoint{4.922126in}{2.759228in}}%
\pgfpathlineto{\pgfqpoint{4.935704in}{2.765883in}}%
\pgfpathlineto{\pgfqpoint{4.949297in}{2.772696in}}%
\pgfpathlineto{\pgfqpoint{4.956586in}{2.779186in}}%
\pgfpathlineto{\pgfqpoint{4.963869in}{2.785662in}}%
\pgfpathlineto{\pgfqpoint{4.971147in}{2.792125in}}%
\pgfpathlineto{\pgfqpoint{4.978419in}{2.798582in}}%
\pgfpathlineto{\pgfqpoint{4.964841in}{2.792092in}}%
\pgfpathlineto{\pgfqpoint{4.951277in}{2.785760in}}%
\pgfpathlineto{\pgfqpoint{4.937727in}{2.779585in}}%
\pgfpathlineto{\pgfqpoint{4.924191in}{2.773567in}}%
\pgfpathlineto{\pgfqpoint{4.916904in}{2.766779in}}%
\pgfpathlineto{\pgfqpoint{4.909613in}{2.759989in}}%
\pgfpathlineto{\pgfqpoint{4.902315in}{2.753195in}}%
\pgfpathlineto{\pgfqpoint{4.895012in}{2.746392in}}%
\pgfpathclose%
\pgfusepath{fill}%
\end{pgfscope}%
\begin{pgfscope}%
\pgfpathrectangle{\pgfqpoint{1.254980in}{0.150000in}}{\pgfqpoint{5.490039in}{5.490039in}}%
\pgfusepath{clip}%
\pgfsetbuttcap%
\pgfsetroundjoin%
\definecolor{currentfill}{rgb}{0.180629,0.429975,0.557282}%
\pgfsetfillcolor{currentfill}%
\pgfsetfillopacity{0.700000}%
\pgfsetlinewidth{0.000000pt}%
\definecolor{currentstroke}{rgb}{0.000000,0.000000,0.000000}%
\pgfsetstrokecolor{currentstroke}%
\pgfsetdash{}{0pt}%
\pgfpathmoveto{\pgfqpoint{4.978419in}{2.798582in}}%
\pgfpathlineto{\pgfqpoint{4.992011in}{2.805228in}}%
\pgfpathlineto{\pgfqpoint{5.005617in}{2.812031in}}%
\pgfpathlineto{\pgfqpoint{5.019238in}{2.818991in}}%
\pgfpathlineto{\pgfqpoint{5.032874in}{2.826107in}}%
\pgfpathlineto{\pgfqpoint{5.040125in}{2.832218in}}%
\pgfpathlineto{\pgfqpoint{5.047371in}{2.838321in}}%
\pgfpathlineto{\pgfqpoint{5.054610in}{2.844422in}}%
\pgfpathlineto{\pgfqpoint{5.061844in}{2.850524in}}%
\pgfpathlineto{\pgfqpoint{5.048224in}{2.843760in}}%
\pgfpathlineto{\pgfqpoint{5.034619in}{2.837152in}}%
\pgfpathlineto{\pgfqpoint{5.021027in}{2.830701in}}%
\pgfpathlineto{\pgfqpoint{5.007450in}{2.824406in}}%
\pgfpathlineto{\pgfqpoint{5.000201in}{2.817942in}}%
\pgfpathlineto{\pgfqpoint{4.992945in}{2.811486in}}%
\pgfpathlineto{\pgfqpoint{4.985685in}{2.805034in}}%
\pgfpathlineto{\pgfqpoint{4.978419in}{2.798582in}}%
\pgfpathclose%
\pgfusepath{fill}%
\end{pgfscope}%
\begin{pgfscope}%
\pgfpathrectangle{\pgfqpoint{1.254980in}{0.150000in}}{\pgfqpoint{5.490039in}{5.490039in}}%
\pgfusepath{clip}%
\pgfsetbuttcap%
\pgfsetroundjoin%
\definecolor{currentfill}{rgb}{0.277941,0.056324,0.381191}%
\pgfsetfillcolor{currentfill}%
\pgfsetfillopacity{0.700000}%
\pgfsetlinewidth{0.000000pt}%
\definecolor{currentstroke}{rgb}{0.000000,0.000000,0.000000}%
\pgfsetstrokecolor{currentstroke}%
\pgfsetdash{}{0pt}%
\pgfpathmoveto{\pgfqpoint{3.479087in}{2.012409in}}%
\pgfpathlineto{\pgfqpoint{3.492162in}{2.008324in}}%
\pgfpathlineto{\pgfqpoint{3.505242in}{2.004426in}}%
\pgfpathlineto{\pgfqpoint{3.518325in}{2.000715in}}%
\pgfpathlineto{\pgfqpoint{3.531413in}{1.997190in}}%
\pgfpathlineto{\pgfqpoint{3.539225in}{2.006502in}}%
\pgfpathlineto{\pgfqpoint{3.547031in}{2.015837in}}%
\pgfpathlineto{\pgfqpoint{3.554832in}{2.025195in}}%
\pgfpathlineto{\pgfqpoint{3.562628in}{2.034575in}}%
\pgfpathlineto{\pgfqpoint{3.549553in}{2.037939in}}%
\pgfpathlineto{\pgfqpoint{3.536481in}{2.041489in}}%
\pgfpathlineto{\pgfqpoint{3.523414in}{2.045226in}}%
\pgfpathlineto{\pgfqpoint{3.510351in}{2.049150in}}%
\pgfpathlineto{\pgfqpoint{3.502543in}{2.039921in}}%
\pgfpathlineto{\pgfqpoint{3.494730in}{2.030720in}}%
\pgfpathlineto{\pgfqpoint{3.486911in}{2.021549in}}%
\pgfpathlineto{\pgfqpoint{3.479087in}{2.012409in}}%
\pgfpathclose%
\pgfusepath{fill}%
\end{pgfscope}%
\begin{pgfscope}%
\pgfpathrectangle{\pgfqpoint{1.254980in}{0.150000in}}{\pgfqpoint{5.490039in}{5.490039in}}%
\pgfusepath{clip}%
\pgfsetbuttcap%
\pgfsetroundjoin%
\definecolor{currentfill}{rgb}{0.280255,0.165693,0.476498}%
\pgfsetfillcolor{currentfill}%
\pgfsetfillopacity{0.700000}%
\pgfsetlinewidth{0.000000pt}%
\definecolor{currentstroke}{rgb}{0.000000,0.000000,0.000000}%
\pgfsetstrokecolor{currentstroke}%
\pgfsetdash{}{0pt}%
\pgfpathmoveto{\pgfqpoint{4.031795in}{2.208010in}}%
\pgfpathlineto{\pgfqpoint{4.044987in}{2.209463in}}%
\pgfpathlineto{\pgfqpoint{4.058188in}{2.211086in}}%
\pgfpathlineto{\pgfqpoint{4.071397in}{2.212878in}}%
\pgfpathlineto{\pgfqpoint{4.084616in}{2.214840in}}%
\pgfpathlineto{\pgfqpoint{4.092242in}{2.224553in}}%
\pgfpathlineto{\pgfqpoint{4.099863in}{2.234231in}}%
\pgfpathlineto{\pgfqpoint{4.107479in}{2.243874in}}%
\pgfpathlineto{\pgfqpoint{4.115090in}{2.253482in}}%
\pgfpathlineto{\pgfqpoint{4.101879in}{2.251527in}}%
\pgfpathlineto{\pgfqpoint{4.088677in}{2.249742in}}%
\pgfpathlineto{\pgfqpoint{4.075483in}{2.248125in}}%
\pgfpathlineto{\pgfqpoint{4.062299in}{2.246679in}}%
\pgfpathlineto{\pgfqpoint{4.054680in}{2.237054in}}%
\pgfpathlineto{\pgfqpoint{4.047057in}{2.227401in}}%
\pgfpathlineto{\pgfqpoint{4.039428in}{2.217720in}}%
\pgfpathlineto{\pgfqpoint{4.031795in}{2.208010in}}%
\pgfpathclose%
\pgfusepath{fill}%
\end{pgfscope}%
\begin{pgfscope}%
\pgfpathrectangle{\pgfqpoint{1.254980in}{0.150000in}}{\pgfqpoint{5.490039in}{5.490039in}}%
\pgfusepath{clip}%
\pgfsetbuttcap%
\pgfsetroundjoin%
\definecolor{currentfill}{rgb}{0.282623,0.140926,0.457517}%
\pgfsetfillcolor{currentfill}%
\pgfsetfillopacity{0.700000}%
\pgfsetlinewidth{0.000000pt}%
\definecolor{currentstroke}{rgb}{0.000000,0.000000,0.000000}%
\pgfsetstrokecolor{currentstroke}%
\pgfsetdash{}{0pt}%
\pgfpathmoveto{\pgfqpoint{3.948496in}{2.164867in}}%
\pgfpathlineto{\pgfqpoint{3.961663in}{2.165613in}}%
\pgfpathlineto{\pgfqpoint{3.974837in}{2.166532in}}%
\pgfpathlineto{\pgfqpoint{3.988020in}{2.167622in}}%
\pgfpathlineto{\pgfqpoint{4.001211in}{2.168882in}}%
\pgfpathlineto{\pgfqpoint{4.008864in}{2.178708in}}%
\pgfpathlineto{\pgfqpoint{4.016513in}{2.188505in}}%
\pgfpathlineto{\pgfqpoint{4.024156in}{2.198272in}}%
\pgfpathlineto{\pgfqpoint{4.031795in}{2.208010in}}%
\pgfpathlineto{\pgfqpoint{4.018611in}{2.206728in}}%
\pgfpathlineto{\pgfqpoint{4.005436in}{2.205617in}}%
\pgfpathlineto{\pgfqpoint{3.992269in}{2.204677in}}%
\pgfpathlineto{\pgfqpoint{3.979110in}{2.203908in}}%
\pgfpathlineto{\pgfqpoint{3.971464in}{2.194181in}}%
\pgfpathlineto{\pgfqpoint{3.963813in}{2.184432in}}%
\pgfpathlineto{\pgfqpoint{3.956157in}{2.174661in}}%
\pgfpathlineto{\pgfqpoint{3.948496in}{2.164867in}}%
\pgfpathclose%
\pgfusepath{fill}%
\end{pgfscope}%
\begin{pgfscope}%
\pgfpathrectangle{\pgfqpoint{1.254980in}{0.150000in}}{\pgfqpoint{5.490039in}{5.490039in}}%
\pgfusepath{clip}%
\pgfsetbuttcap%
\pgfsetroundjoin%
\definecolor{currentfill}{rgb}{0.281446,0.084320,0.407414}%
\pgfsetfillcolor{currentfill}%
\pgfsetfillopacity{0.700000}%
\pgfsetlinewidth{0.000000pt}%
\definecolor{currentstroke}{rgb}{0.000000,0.000000,0.000000}%
\pgfsetstrokecolor{currentstroke}%
\pgfsetdash{}{0pt}%
\pgfpathmoveto{\pgfqpoint{3.070356in}{2.078296in}}%
\pgfpathlineto{\pgfqpoint{3.083453in}{2.068744in}}%
\pgfpathlineto{\pgfqpoint{3.096549in}{2.059406in}}%
\pgfpathlineto{\pgfqpoint{3.109644in}{2.050281in}}%
\pgfpathlineto{\pgfqpoint{3.122738in}{2.041368in}}%
\pgfpathlineto{\pgfqpoint{3.130724in}{2.048729in}}%
\pgfpathlineto{\pgfqpoint{3.138703in}{2.056177in}}%
\pgfpathlineto{\pgfqpoint{3.146674in}{2.063710in}}%
\pgfpathlineto{\pgfqpoint{3.154637in}{2.071326in}}%
\pgfpathlineto{\pgfqpoint{3.141563in}{2.079992in}}%
\pgfpathlineto{\pgfqpoint{3.128488in}{2.088869in}}%
\pgfpathlineto{\pgfqpoint{3.115412in}{2.097959in}}%
\pgfpathlineto{\pgfqpoint{3.102336in}{2.107264in}}%
\pgfpathlineto{\pgfqpoint{3.094353in}{2.099885in}}%
\pgfpathlineto{\pgfqpoint{3.086362in}{2.092596in}}%
\pgfpathlineto{\pgfqpoint{3.078363in}{2.085399in}}%
\pgfpathlineto{\pgfqpoint{3.070356in}{2.078296in}}%
\pgfpathclose%
\pgfusepath{fill}%
\end{pgfscope}%
\begin{pgfscope}%
\pgfpathrectangle{\pgfqpoint{1.254980in}{0.150000in}}{\pgfqpoint{5.490039in}{5.490039in}}%
\pgfusepath{clip}%
\pgfsetbuttcap%
\pgfsetroundjoin%
\definecolor{currentfill}{rgb}{0.276194,0.190074,0.493001}%
\pgfsetfillcolor{currentfill}%
\pgfsetfillopacity{0.700000}%
\pgfsetlinewidth{0.000000pt}%
\definecolor{currentstroke}{rgb}{0.000000,0.000000,0.000000}%
\pgfsetstrokecolor{currentstroke}%
\pgfsetdash{}{0pt}%
\pgfpathmoveto{\pgfqpoint{4.115090in}{2.253482in}}%
\pgfpathlineto{\pgfqpoint{4.128311in}{2.255606in}}%
\pgfpathlineto{\pgfqpoint{4.141540in}{2.257898in}}%
\pgfpathlineto{\pgfqpoint{4.154780in}{2.260357in}}%
\pgfpathlineto{\pgfqpoint{4.168028in}{2.262985in}}%
\pgfpathlineto{\pgfqpoint{4.175627in}{2.272535in}}%
\pgfpathlineto{\pgfqpoint{4.183221in}{2.282046in}}%
\pgfpathlineto{\pgfqpoint{4.190810in}{2.291517in}}%
\pgfpathlineto{\pgfqpoint{4.198393in}{2.300949in}}%
\pgfpathlineto{\pgfqpoint{4.185152in}{2.298357in}}%
\pgfpathlineto{\pgfqpoint{4.171920in}{2.295932in}}%
\pgfpathlineto{\pgfqpoint{4.158697in}{2.293675in}}%
\pgfpathlineto{\pgfqpoint{4.145484in}{2.291586in}}%
\pgfpathlineto{\pgfqpoint{4.137893in}{2.282108in}}%
\pgfpathlineto{\pgfqpoint{4.130297in}{2.272599in}}%
\pgfpathlineto{\pgfqpoint{4.122696in}{2.263057in}}%
\pgfpathlineto{\pgfqpoint{4.115090in}{2.253482in}}%
\pgfpathclose%
\pgfusepath{fill}%
\end{pgfscope}%
\begin{pgfscope}%
\pgfpathrectangle{\pgfqpoint{1.254980in}{0.150000in}}{\pgfqpoint{5.490039in}{5.490039in}}%
\pgfusepath{clip}%
\pgfsetbuttcap%
\pgfsetroundjoin%
\definecolor{currentfill}{rgb}{0.269308,0.218818,0.509577}%
\pgfsetfillcolor{currentfill}%
\pgfsetfillopacity{0.700000}%
\pgfsetlinewidth{0.000000pt}%
\definecolor{currentstroke}{rgb}{0.000000,0.000000,0.000000}%
\pgfsetstrokecolor{currentstroke}%
\pgfsetdash{}{0pt}%
\pgfpathmoveto{\pgfqpoint{2.722393in}{2.355234in}}%
\pgfpathlineto{\pgfqpoint{2.735612in}{2.339766in}}%
\pgfpathlineto{\pgfqpoint{2.748824in}{2.324553in}}%
\pgfpathlineto{\pgfqpoint{2.762030in}{2.309596in}}%
\pgfpathlineto{\pgfqpoint{2.775229in}{2.294891in}}%
\pgfpathlineto{\pgfqpoint{2.783397in}{2.300144in}}%
\pgfpathlineto{\pgfqpoint{2.791554in}{2.305537in}}%
\pgfpathlineto{\pgfqpoint{2.799701in}{2.311066in}}%
\pgfpathlineto{\pgfqpoint{2.807837in}{2.316730in}}%
\pgfpathlineto{\pgfqpoint{2.794666in}{2.331152in}}%
\pgfpathlineto{\pgfqpoint{2.781489in}{2.345826in}}%
\pgfpathlineto{\pgfqpoint{2.768305in}{2.360754in}}%
\pgfpathlineto{\pgfqpoint{2.755115in}{2.375938in}}%
\pgfpathlineto{\pgfqpoint{2.746951in}{2.370547in}}%
\pgfpathlineto{\pgfqpoint{2.738776in}{2.365298in}}%
\pgfpathlineto{\pgfqpoint{2.730590in}{2.360193in}}%
\pgfpathlineto{\pgfqpoint{2.722393in}{2.355234in}}%
\pgfpathclose%
\pgfusepath{fill}%
\end{pgfscope}%
\begin{pgfscope}%
\pgfpathrectangle{\pgfqpoint{1.254980in}{0.150000in}}{\pgfqpoint{5.490039in}{5.490039in}}%
\pgfusepath{clip}%
\pgfsetbuttcap%
\pgfsetroundjoin%
\definecolor{currentfill}{rgb}{0.283229,0.120777,0.440584}%
\pgfsetfillcolor{currentfill}%
\pgfsetfillopacity{0.700000}%
\pgfsetlinewidth{0.000000pt}%
\definecolor{currentstroke}{rgb}{0.000000,0.000000,0.000000}%
\pgfsetstrokecolor{currentstroke}%
\pgfsetdash{}{0pt}%
\pgfpathmoveto{\pgfqpoint{3.865182in}{2.124405in}}%
\pgfpathlineto{\pgfqpoint{3.878326in}{2.124410in}}%
\pgfpathlineto{\pgfqpoint{3.891477in}{2.124588in}}%
\pgfpathlineto{\pgfqpoint{3.904636in}{2.124940in}}%
\pgfpathlineto{\pgfqpoint{3.917803in}{2.125465in}}%
\pgfpathlineto{\pgfqpoint{3.925483in}{2.135349in}}%
\pgfpathlineto{\pgfqpoint{3.933159in}{2.145211in}}%
\pgfpathlineto{\pgfqpoint{3.940830in}{2.155050in}}%
\pgfpathlineto{\pgfqpoint{3.948496in}{2.164867in}}%
\pgfpathlineto{\pgfqpoint{3.935338in}{2.164293in}}%
\pgfpathlineto{\pgfqpoint{3.922187in}{2.163891in}}%
\pgfpathlineto{\pgfqpoint{3.909043in}{2.163663in}}%
\pgfpathlineto{\pgfqpoint{3.895907in}{2.163609in}}%
\pgfpathlineto{\pgfqpoint{3.888233in}{2.153832in}}%
\pgfpathlineto{\pgfqpoint{3.880555in}{2.144039in}}%
\pgfpathlineto{\pgfqpoint{3.872871in}{2.134230in}}%
\pgfpathlineto{\pgfqpoint{3.865182in}{2.124405in}}%
\pgfpathclose%
\pgfusepath{fill}%
\end{pgfscope}%
\begin{pgfscope}%
\pgfpathrectangle{\pgfqpoint{1.254980in}{0.150000in}}{\pgfqpoint{5.490039in}{5.490039in}}%
\pgfusepath{clip}%
\pgfsetbuttcap%
\pgfsetroundjoin%
\definecolor{currentfill}{rgb}{0.260571,0.246922,0.522828}%
\pgfsetfillcolor{currentfill}%
\pgfsetfillopacity{0.700000}%
\pgfsetlinewidth{0.000000pt}%
\definecolor{currentstroke}{rgb}{0.000000,0.000000,0.000000}%
\pgfsetstrokecolor{currentstroke}%
\pgfsetdash{}{0pt}%
\pgfpathmoveto{\pgfqpoint{2.669444in}{2.419720in}}%
\pgfpathlineto{\pgfqpoint{2.682693in}{2.403203in}}%
\pgfpathlineto{\pgfqpoint{2.695934in}{2.386951in}}%
\pgfpathlineto{\pgfqpoint{2.709167in}{2.370962in}}%
\pgfpathlineto{\pgfqpoint{2.722393in}{2.355234in}}%
\pgfpathlineto{\pgfqpoint{2.730590in}{2.360193in}}%
\pgfpathlineto{\pgfqpoint{2.738776in}{2.365298in}}%
\pgfpathlineto{\pgfqpoint{2.746951in}{2.370547in}}%
\pgfpathlineto{\pgfqpoint{2.755115in}{2.375938in}}%
\pgfpathlineto{\pgfqpoint{2.741918in}{2.391381in}}%
\pgfpathlineto{\pgfqpoint{2.728714in}{2.407084in}}%
\pgfpathlineto{\pgfqpoint{2.715503in}{2.423050in}}%
\pgfpathlineto{\pgfqpoint{2.702285in}{2.439281in}}%
\pgfpathlineto{\pgfqpoint{2.694091in}{2.434165in}}%
\pgfpathlineto{\pgfqpoint{2.685887in}{2.429198in}}%
\pgfpathlineto{\pgfqpoint{2.677671in}{2.424382in}}%
\pgfpathlineto{\pgfqpoint{2.669444in}{2.419720in}}%
\pgfpathclose%
\pgfusepath{fill}%
\end{pgfscope}%
\begin{pgfscope}%
\pgfpathrectangle{\pgfqpoint{1.254980in}{0.150000in}}{\pgfqpoint{5.490039in}{5.490039in}}%
\pgfusepath{clip}%
\pgfsetbuttcap%
\pgfsetroundjoin%
\definecolor{currentfill}{rgb}{0.172719,0.448791,0.557885}%
\pgfsetfillcolor{currentfill}%
\pgfsetfillopacity{0.700000}%
\pgfsetlinewidth{0.000000pt}%
\definecolor{currentstroke}{rgb}{0.000000,0.000000,0.000000}%
\pgfsetstrokecolor{currentstroke}%
\pgfsetdash{}{0pt}%
\pgfpathmoveto{\pgfqpoint{5.061844in}{2.850524in}}%
\pgfpathlineto{\pgfqpoint{5.075479in}{2.857444in}}%
\pgfpathlineto{\pgfqpoint{5.089128in}{2.864521in}}%
\pgfpathlineto{\pgfqpoint{5.102793in}{2.871753in}}%
\pgfpathlineto{\pgfqpoint{5.116472in}{2.879142in}}%
\pgfpathlineto{\pgfqpoint{5.123684in}{2.884878in}}%
\pgfpathlineto{\pgfqpoint{5.130890in}{2.890617in}}%
\pgfpathlineto{\pgfqpoint{5.138091in}{2.896364in}}%
\pgfpathlineto{\pgfqpoint{5.145286in}{2.902121in}}%
\pgfpathlineto{\pgfqpoint{5.131624in}{2.895115in}}%
\pgfpathlineto{\pgfqpoint{5.117977in}{2.888264in}}%
\pgfpathlineto{\pgfqpoint{5.104344in}{2.881568in}}%
\pgfpathlineto{\pgfqpoint{5.090726in}{2.875028in}}%
\pgfpathlineto{\pgfqpoint{5.083513in}{2.868879in}}%
\pgfpathlineto{\pgfqpoint{5.076296in}{2.862749in}}%
\pgfpathlineto{\pgfqpoint{5.069073in}{2.856632in}}%
\pgfpathlineto{\pgfqpoint{5.061844in}{2.850524in}}%
\pgfpathclose%
\pgfusepath{fill}%
\end{pgfscope}%
\begin{pgfscope}%
\pgfpathrectangle{\pgfqpoint{1.254980in}{0.150000in}}{\pgfqpoint{5.490039in}{5.490039in}}%
\pgfusepath{clip}%
\pgfsetbuttcap%
\pgfsetroundjoin%
\definecolor{currentfill}{rgb}{0.270595,0.214069,0.507052}%
\pgfsetfillcolor{currentfill}%
\pgfsetfillopacity{0.700000}%
\pgfsetlinewidth{0.000000pt}%
\definecolor{currentstroke}{rgb}{0.000000,0.000000,0.000000}%
\pgfsetstrokecolor{currentstroke}%
\pgfsetdash{}{0pt}%
\pgfpathmoveto{\pgfqpoint{4.198393in}{2.300949in}}%
\pgfpathlineto{\pgfqpoint{4.211645in}{2.303709in}}%
\pgfpathlineto{\pgfqpoint{4.224906in}{2.306635in}}%
\pgfpathlineto{\pgfqpoint{4.238177in}{2.309728in}}%
\pgfpathlineto{\pgfqpoint{4.251459in}{2.312987in}}%
\pgfpathlineto{\pgfqpoint{4.259030in}{2.322329in}}%
\pgfpathlineto{\pgfqpoint{4.266596in}{2.331628in}}%
\pgfpathlineto{\pgfqpoint{4.274157in}{2.340884in}}%
\pgfpathlineto{\pgfqpoint{4.281713in}{2.350098in}}%
\pgfpathlineto{\pgfqpoint{4.268438in}{2.346903in}}%
\pgfpathlineto{\pgfqpoint{4.255174in}{2.343874in}}%
\pgfpathlineto{\pgfqpoint{4.241920in}{2.341010in}}%
\pgfpathlineto{\pgfqpoint{4.228676in}{2.338314in}}%
\pgfpathlineto{\pgfqpoint{4.221113in}{2.329026in}}%
\pgfpathlineto{\pgfqpoint{4.213545in}{2.319703in}}%
\pgfpathlineto{\pgfqpoint{4.205972in}{2.310344in}}%
\pgfpathlineto{\pgfqpoint{4.198393in}{2.300949in}}%
\pgfpathclose%
\pgfusepath{fill}%
\end{pgfscope}%
\begin{pgfscope}%
\pgfpathrectangle{\pgfqpoint{1.254980in}{0.150000in}}{\pgfqpoint{5.490039in}{5.490039in}}%
\pgfusepath{clip}%
\pgfsetbuttcap%
\pgfsetroundjoin%
\definecolor{currentfill}{rgb}{0.276194,0.190074,0.493001}%
\pgfsetfillcolor{currentfill}%
\pgfsetfillopacity{0.700000}%
\pgfsetlinewidth{0.000000pt}%
\definecolor{currentstroke}{rgb}{0.000000,0.000000,0.000000}%
\pgfsetstrokecolor{currentstroke}%
\pgfsetdash{}{0pt}%
\pgfpathmoveto{\pgfqpoint{2.775229in}{2.294891in}}%
\pgfpathlineto{\pgfqpoint{2.788422in}{2.280436in}}%
\pgfpathlineto{\pgfqpoint{2.801609in}{2.266229in}}%
\pgfpathlineto{\pgfqpoint{2.814791in}{2.252269in}}%
\pgfpathlineto{\pgfqpoint{2.827968in}{2.238553in}}%
\pgfpathlineto{\pgfqpoint{2.836107in}{2.244100in}}%
\pgfpathlineto{\pgfqpoint{2.844237in}{2.249778in}}%
\pgfpathlineto{\pgfqpoint{2.852357in}{2.255586in}}%
\pgfpathlineto{\pgfqpoint{2.860467in}{2.261521in}}%
\pgfpathlineto{\pgfqpoint{2.847317in}{2.274956in}}%
\pgfpathlineto{\pgfqpoint{2.834163in}{2.288634in}}%
\pgfpathlineto{\pgfqpoint{2.821003in}{2.302558in}}%
\pgfpathlineto{\pgfqpoint{2.807837in}{2.316730in}}%
\pgfpathlineto{\pgfqpoint{2.799701in}{2.311066in}}%
\pgfpathlineto{\pgfqpoint{2.791554in}{2.305537in}}%
\pgfpathlineto{\pgfqpoint{2.783397in}{2.300144in}}%
\pgfpathlineto{\pgfqpoint{2.775229in}{2.294891in}}%
\pgfpathclose%
\pgfusepath{fill}%
\end{pgfscope}%
\begin{pgfscope}%
\pgfpathrectangle{\pgfqpoint{1.254980in}{0.150000in}}{\pgfqpoint{5.490039in}{5.490039in}}%
\pgfusepath{clip}%
\pgfsetbuttcap%
\pgfsetroundjoin%
\definecolor{currentfill}{rgb}{0.248629,0.278775,0.534556}%
\pgfsetfillcolor{currentfill}%
\pgfsetfillopacity{0.700000}%
\pgfsetlinewidth{0.000000pt}%
\definecolor{currentstroke}{rgb}{0.000000,0.000000,0.000000}%
\pgfsetstrokecolor{currentstroke}%
\pgfsetdash{}{0pt}%
\pgfpathmoveto{\pgfqpoint{2.616366in}{2.488494in}}%
\pgfpathlineto{\pgfqpoint{2.629648in}{2.470890in}}%
\pgfpathlineto{\pgfqpoint{2.642922in}{2.453562in}}%
\pgfpathlineto{\pgfqpoint{2.656187in}{2.436506in}}%
\pgfpathlineto{\pgfqpoint{2.669444in}{2.419720in}}%
\pgfpathlineto{\pgfqpoint{2.677671in}{2.424382in}}%
\pgfpathlineto{\pgfqpoint{2.685887in}{2.429198in}}%
\pgfpathlineto{\pgfqpoint{2.694091in}{2.434165in}}%
\pgfpathlineto{\pgfqpoint{2.702285in}{2.439281in}}%
\pgfpathlineto{\pgfqpoint{2.689058in}{2.455779in}}%
\pgfpathlineto{\pgfqpoint{2.675824in}{2.472547in}}%
\pgfpathlineto{\pgfqpoint{2.662581in}{2.489588in}}%
\pgfpathlineto{\pgfqpoint{2.649330in}{2.506903in}}%
\pgfpathlineto{\pgfqpoint{2.641106in}{2.502064in}}%
\pgfpathlineto{\pgfqpoint{2.632871in}{2.497381in}}%
\pgfpathlineto{\pgfqpoint{2.624624in}{2.492857in}}%
\pgfpathlineto{\pgfqpoint{2.616366in}{2.488494in}}%
\pgfpathclose%
\pgfusepath{fill}%
\end{pgfscope}%
\begin{pgfscope}%
\pgfpathrectangle{\pgfqpoint{1.254980in}{0.150000in}}{\pgfqpoint{5.490039in}{5.490039in}}%
\pgfusepath{clip}%
\pgfsetbuttcap%
\pgfsetroundjoin%
\definecolor{currentfill}{rgb}{0.282910,0.105393,0.426902}%
\pgfsetfillcolor{currentfill}%
\pgfsetfillopacity{0.700000}%
\pgfsetlinewidth{0.000000pt}%
\definecolor{currentstroke}{rgb}{0.000000,0.000000,0.000000}%
\pgfsetstrokecolor{currentstroke}%
\pgfsetdash{}{0pt}%
\pgfpathmoveto{\pgfqpoint{3.781836in}{2.087001in}}%
\pgfpathlineto{\pgfqpoint{3.794961in}{2.086226in}}%
\pgfpathlineto{\pgfqpoint{3.808092in}{2.085628in}}%
\pgfpathlineto{\pgfqpoint{3.821231in}{2.085206in}}%
\pgfpathlineto{\pgfqpoint{3.834376in}{2.084958in}}%
\pgfpathlineto{\pgfqpoint{3.842085in}{2.094842in}}%
\pgfpathlineto{\pgfqpoint{3.849789in}{2.104711in}}%
\pgfpathlineto{\pgfqpoint{3.857488in}{2.114566in}}%
\pgfpathlineto{\pgfqpoint{3.865182in}{2.124405in}}%
\pgfpathlineto{\pgfqpoint{3.852045in}{2.124576in}}%
\pgfpathlineto{\pgfqpoint{3.838915in}{2.124921in}}%
\pgfpathlineto{\pgfqpoint{3.825792in}{2.125441in}}%
\pgfpathlineto{\pgfqpoint{3.812676in}{2.126138in}}%
\pgfpathlineto{\pgfqpoint{3.804973in}{2.116366in}}%
\pgfpathlineto{\pgfqpoint{3.797266in}{2.106585in}}%
\pgfpathlineto{\pgfqpoint{3.789553in}{2.096796in}}%
\pgfpathlineto{\pgfqpoint{3.781836in}{2.087001in}}%
\pgfpathclose%
\pgfusepath{fill}%
\end{pgfscope}%
\begin{pgfscope}%
\pgfpathrectangle{\pgfqpoint{1.254980in}{0.150000in}}{\pgfqpoint{5.490039in}{5.490039in}}%
\pgfusepath{clip}%
\pgfsetbuttcap%
\pgfsetroundjoin%
\definecolor{currentfill}{rgb}{0.263663,0.237631,0.518762}%
\pgfsetfillcolor{currentfill}%
\pgfsetfillopacity{0.700000}%
\pgfsetlinewidth{0.000000pt}%
\definecolor{currentstroke}{rgb}{0.000000,0.000000,0.000000}%
\pgfsetstrokecolor{currentstroke}%
\pgfsetdash{}{0pt}%
\pgfpathmoveto{\pgfqpoint{4.281713in}{2.350098in}}%
\pgfpathlineto{\pgfqpoint{4.294997in}{2.353459in}}%
\pgfpathlineto{\pgfqpoint{4.308292in}{2.356986in}}%
\pgfpathlineto{\pgfqpoint{4.321598in}{2.360677in}}%
\pgfpathlineto{\pgfqpoint{4.334915in}{2.364533in}}%
\pgfpathlineto{\pgfqpoint{4.342458in}{2.373627in}}%
\pgfpathlineto{\pgfqpoint{4.349995in}{2.382674in}}%
\pgfpathlineto{\pgfqpoint{4.357528in}{2.391676in}}%
\pgfpathlineto{\pgfqpoint{4.365054in}{2.400636in}}%
\pgfpathlineto{\pgfqpoint{4.351745in}{2.396871in}}%
\pgfpathlineto{\pgfqpoint{4.338447in}{2.393272in}}%
\pgfpathlineto{\pgfqpoint{4.325159in}{2.389837in}}%
\pgfpathlineto{\pgfqpoint{4.311882in}{2.386567in}}%
\pgfpathlineto{\pgfqpoint{4.304348in}{2.377506in}}%
\pgfpathlineto{\pgfqpoint{4.296808in}{2.368408in}}%
\pgfpathlineto{\pgfqpoint{4.289263in}{2.359273in}}%
\pgfpathlineto{\pgfqpoint{4.281713in}{2.350098in}}%
\pgfpathclose%
\pgfusepath{fill}%
\end{pgfscope}%
\begin{pgfscope}%
\pgfpathrectangle{\pgfqpoint{1.254980in}{0.150000in}}{\pgfqpoint{5.490039in}{5.490039in}}%
\pgfusepath{clip}%
\pgfsetbuttcap%
\pgfsetroundjoin%
\definecolor{currentfill}{rgb}{0.163625,0.471133,0.558148}%
\pgfsetfillcolor{currentfill}%
\pgfsetfillopacity{0.700000}%
\pgfsetlinewidth{0.000000pt}%
\definecolor{currentstroke}{rgb}{0.000000,0.000000,0.000000}%
\pgfsetstrokecolor{currentstroke}%
\pgfsetdash{}{0pt}%
\pgfpathmoveto{\pgfqpoint{5.145286in}{2.902121in}}%
\pgfpathlineto{\pgfqpoint{5.158964in}{2.909283in}}%
\pgfpathlineto{\pgfqpoint{5.172656in}{2.916601in}}%
\pgfpathlineto{\pgfqpoint{5.186364in}{2.924073in}}%
\pgfpathlineto{\pgfqpoint{5.200087in}{2.931702in}}%
\pgfpathlineto{\pgfqpoint{5.207258in}{2.937075in}}%
\pgfpathlineto{\pgfqpoint{5.214425in}{2.942462in}}%
\pgfpathlineto{\pgfqpoint{5.221585in}{2.947867in}}%
\pgfpathlineto{\pgfqpoint{5.228741in}{2.953295in}}%
\pgfpathlineto{\pgfqpoint{5.215037in}{2.946078in}}%
\pgfpathlineto{\pgfqpoint{5.201348in}{2.939016in}}%
\pgfpathlineto{\pgfqpoint{5.187674in}{2.932109in}}%
\pgfpathlineto{\pgfqpoint{5.174015in}{2.925357in}}%
\pgfpathlineto{\pgfqpoint{5.166840in}{2.919508in}}%
\pgfpathlineto{\pgfqpoint{5.159661in}{2.913689in}}%
\pgfpathlineto{\pgfqpoint{5.152476in}{2.907895in}}%
\pgfpathlineto{\pgfqpoint{5.145286in}{2.902121in}}%
\pgfpathclose%
\pgfusepath{fill}%
\end{pgfscope}%
\begin{pgfscope}%
\pgfpathrectangle{\pgfqpoint{1.254980in}{0.150000in}}{\pgfqpoint{5.490039in}{5.490039in}}%
\pgfusepath{clip}%
\pgfsetbuttcap%
\pgfsetroundjoin%
\definecolor{currentfill}{rgb}{0.280255,0.165693,0.476498}%
\pgfsetfillcolor{currentfill}%
\pgfsetfillopacity{0.700000}%
\pgfsetlinewidth{0.000000pt}%
\definecolor{currentstroke}{rgb}{0.000000,0.000000,0.000000}%
\pgfsetstrokecolor{currentstroke}%
\pgfsetdash{}{0pt}%
\pgfpathmoveto{\pgfqpoint{2.827968in}{2.238553in}}%
\pgfpathlineto{\pgfqpoint{2.841139in}{2.225080in}}%
\pgfpathlineto{\pgfqpoint{2.854305in}{2.211847in}}%
\pgfpathlineto{\pgfqpoint{2.867467in}{2.198853in}}%
\pgfpathlineto{\pgfqpoint{2.880624in}{2.186096in}}%
\pgfpathlineto{\pgfqpoint{2.888738in}{2.191934in}}%
\pgfpathlineto{\pgfqpoint{2.896841in}{2.197896in}}%
\pgfpathlineto{\pgfqpoint{2.904935in}{2.203981in}}%
\pgfpathlineto{\pgfqpoint{2.913020in}{2.210186in}}%
\pgfpathlineto{\pgfqpoint{2.899888in}{2.222663in}}%
\pgfpathlineto{\pgfqpoint{2.886752in}{2.235377in}}%
\pgfpathlineto{\pgfqpoint{2.873612in}{2.248329in}}%
\pgfpathlineto{\pgfqpoint{2.860467in}{2.261521in}}%
\pgfpathlineto{\pgfqpoint{2.852357in}{2.255586in}}%
\pgfpathlineto{\pgfqpoint{2.844237in}{2.249778in}}%
\pgfpathlineto{\pgfqpoint{2.836107in}{2.244100in}}%
\pgfpathlineto{\pgfqpoint{2.827968in}{2.238553in}}%
\pgfpathclose%
\pgfusepath{fill}%
\end{pgfscope}%
\begin{pgfscope}%
\pgfpathrectangle{\pgfqpoint{1.254980in}{0.150000in}}{\pgfqpoint{5.490039in}{5.490039in}}%
\pgfusepath{clip}%
\pgfsetbuttcap%
\pgfsetroundjoin%
\definecolor{currentfill}{rgb}{0.253935,0.265254,0.529983}%
\pgfsetfillcolor{currentfill}%
\pgfsetfillopacity{0.700000}%
\pgfsetlinewidth{0.000000pt}%
\definecolor{currentstroke}{rgb}{0.000000,0.000000,0.000000}%
\pgfsetstrokecolor{currentstroke}%
\pgfsetdash{}{0pt}%
\pgfpathmoveto{\pgfqpoint{4.365054in}{2.400636in}}%
\pgfpathlineto{\pgfqpoint{4.378375in}{2.404564in}}%
\pgfpathlineto{\pgfqpoint{4.391706in}{2.408657in}}%
\pgfpathlineto{\pgfqpoint{4.405048in}{2.412913in}}%
\pgfpathlineto{\pgfqpoint{4.418402in}{2.417333in}}%
\pgfpathlineto{\pgfqpoint{4.425916in}{2.426140in}}%
\pgfpathlineto{\pgfqpoint{4.433424in}{2.434901in}}%
\pgfpathlineto{\pgfqpoint{4.440927in}{2.443616in}}%
\pgfpathlineto{\pgfqpoint{4.448424in}{2.452287in}}%
\pgfpathlineto{\pgfqpoint{4.435078in}{2.447987in}}%
\pgfpathlineto{\pgfqpoint{4.421743in}{2.443852in}}%
\pgfpathlineto{\pgfqpoint{4.408420in}{2.439879in}}%
\pgfpathlineto{\pgfqpoint{4.395108in}{2.436071in}}%
\pgfpathlineto{\pgfqpoint{4.387603in}{2.427269in}}%
\pgfpathlineto{\pgfqpoint{4.380092in}{2.418430in}}%
\pgfpathlineto{\pgfqpoint{4.372576in}{2.409553in}}%
\pgfpathlineto{\pgfqpoint{4.365054in}{2.400636in}}%
\pgfpathclose%
\pgfusepath{fill}%
\end{pgfscope}%
\begin{pgfscope}%
\pgfpathrectangle{\pgfqpoint{1.254980in}{0.150000in}}{\pgfqpoint{5.490039in}{5.490039in}}%
\pgfusepath{clip}%
\pgfsetbuttcap%
\pgfsetroundjoin%
\definecolor{currentfill}{rgb}{0.156270,0.489624,0.557936}%
\pgfsetfillcolor{currentfill}%
\pgfsetfillopacity{0.700000}%
\pgfsetlinewidth{0.000000pt}%
\definecolor{currentstroke}{rgb}{0.000000,0.000000,0.000000}%
\pgfsetstrokecolor{currentstroke}%
\pgfsetdash{}{0pt}%
\pgfpathmoveto{\pgfqpoint{5.228741in}{2.953295in}}%
\pgfpathlineto{\pgfqpoint{5.242461in}{2.960666in}}%
\pgfpathlineto{\pgfqpoint{5.256196in}{2.968192in}}%
\pgfpathlineto{\pgfqpoint{5.269947in}{2.975873in}}%
\pgfpathlineto{\pgfqpoint{5.283713in}{2.983708in}}%
\pgfpathlineto{\pgfqpoint{5.290844in}{2.988734in}}%
\pgfpathlineto{\pgfqpoint{5.297969in}{2.993786in}}%
\pgfpathlineto{\pgfqpoint{5.305090in}{2.998868in}}%
\pgfpathlineto{\pgfqpoint{5.312205in}{3.003986in}}%
\pgfpathlineto{\pgfqpoint{5.298459in}{2.996591in}}%
\pgfpathlineto{\pgfqpoint{5.284728in}{2.989351in}}%
\pgfpathlineto{\pgfqpoint{5.271013in}{2.982264in}}%
\pgfpathlineto{\pgfqpoint{5.257314in}{2.975332in}}%
\pgfpathlineto{\pgfqpoint{5.250178in}{2.969764in}}%
\pgfpathlineto{\pgfqpoint{5.243037in}{2.964239in}}%
\pgfpathlineto{\pgfqpoint{5.235892in}{2.958750in}}%
\pgfpathlineto{\pgfqpoint{5.228741in}{2.953295in}}%
\pgfpathclose%
\pgfusepath{fill}%
\end{pgfscope}%
\begin{pgfscope}%
\pgfpathrectangle{\pgfqpoint{1.254980in}{0.150000in}}{\pgfqpoint{5.490039in}{5.490039in}}%
\pgfusepath{clip}%
\pgfsetbuttcap%
\pgfsetroundjoin%
\definecolor{currentfill}{rgb}{0.277018,0.050344,0.375715}%
\pgfsetfillcolor{currentfill}%
\pgfsetfillopacity{0.700000}%
\pgfsetlinewidth{0.000000pt}%
\definecolor{currentstroke}{rgb}{0.000000,0.000000,0.000000}%
\pgfsetstrokecolor{currentstroke}%
\pgfsetdash{}{0pt}%
\pgfpathmoveto{\pgfqpoint{3.259232in}{2.009462in}}%
\pgfpathlineto{\pgfqpoint{3.272309in}{2.002644in}}%
\pgfpathlineto{\pgfqpoint{3.285387in}{1.996025in}}%
\pgfpathlineto{\pgfqpoint{3.298467in}{1.989605in}}%
\pgfpathlineto{\pgfqpoint{3.311548in}{1.983383in}}%
\pgfpathlineto{\pgfqpoint{3.319451in}{1.991764in}}%
\pgfpathlineto{\pgfqpoint{3.327348in}{2.000204in}}%
\pgfpathlineto{\pgfqpoint{3.335238in}{2.008700in}}%
\pgfpathlineto{\pgfqpoint{3.343121in}{2.017250in}}%
\pgfpathlineto{\pgfqpoint{3.330056in}{2.023256in}}%
\pgfpathlineto{\pgfqpoint{3.316992in}{2.029458in}}%
\pgfpathlineto{\pgfqpoint{3.303930in}{2.035859in}}%
\pgfpathlineto{\pgfqpoint{3.290870in}{2.042460in}}%
\pgfpathlineto{\pgfqpoint{3.282971in}{2.034117in}}%
\pgfpathlineto{\pgfqpoint{3.275064in}{2.025835in}}%
\pgfpathlineto{\pgfqpoint{3.267151in}{2.017616in}}%
\pgfpathlineto{\pgfqpoint{3.259232in}{2.009462in}}%
\pgfpathclose%
\pgfusepath{fill}%
\end{pgfscope}%
\begin{pgfscope}%
\pgfpathrectangle{\pgfqpoint{1.254980in}{0.150000in}}{\pgfqpoint{5.490039in}{5.490039in}}%
\pgfusepath{clip}%
\pgfsetbuttcap%
\pgfsetroundjoin%
\definecolor{currentfill}{rgb}{0.233603,0.313828,0.543914}%
\pgfsetfillcolor{currentfill}%
\pgfsetfillopacity{0.700000}%
\pgfsetlinewidth{0.000000pt}%
\definecolor{currentstroke}{rgb}{0.000000,0.000000,0.000000}%
\pgfsetstrokecolor{currentstroke}%
\pgfsetdash{}{0pt}%
\pgfpathmoveto{\pgfqpoint{2.563140in}{2.561713in}}%
\pgfpathlineto{\pgfqpoint{2.576461in}{2.542982in}}%
\pgfpathlineto{\pgfqpoint{2.589772in}{2.524538in}}%
\pgfpathlineto{\pgfqpoint{2.603074in}{2.506376in}}%
\pgfpathlineto{\pgfqpoint{2.616366in}{2.488494in}}%
\pgfpathlineto{\pgfqpoint{2.624624in}{2.492857in}}%
\pgfpathlineto{\pgfqpoint{2.632871in}{2.497381in}}%
\pgfpathlineto{\pgfqpoint{2.641106in}{2.502064in}}%
\pgfpathlineto{\pgfqpoint{2.649330in}{2.506903in}}%
\pgfpathlineto{\pgfqpoint{2.636070in}{2.524495in}}%
\pgfpathlineto{\pgfqpoint{2.622800in}{2.542367in}}%
\pgfpathlineto{\pgfqpoint{2.609522in}{2.560521in}}%
\pgfpathlineto{\pgfqpoint{2.596233in}{2.578961in}}%
\pgfpathlineto{\pgfqpoint{2.587978in}{2.574401in}}%
\pgfpathlineto{\pgfqpoint{2.579711in}{2.570004in}}%
\pgfpathlineto{\pgfqpoint{2.571431in}{2.565774in}}%
\pgfpathlineto{\pgfqpoint{2.563140in}{2.561713in}}%
\pgfpathclose%
\pgfusepath{fill}%
\end{pgfscope}%
\begin{pgfscope}%
\pgfpathrectangle{\pgfqpoint{1.254980in}{0.150000in}}{\pgfqpoint{5.490039in}{5.490039in}}%
\pgfusepath{clip}%
\pgfsetbuttcap%
\pgfsetroundjoin%
\definecolor{currentfill}{rgb}{0.281446,0.084320,0.407414}%
\pgfsetfillcolor{currentfill}%
\pgfsetfillopacity{0.700000}%
\pgfsetlinewidth{0.000000pt}%
\definecolor{currentstroke}{rgb}{0.000000,0.000000,0.000000}%
\pgfsetstrokecolor{currentstroke}%
\pgfsetdash{}{0pt}%
\pgfpathmoveto{\pgfqpoint{3.698440in}{2.053048in}}%
\pgfpathlineto{\pgfqpoint{3.711550in}{2.051457in}}%
\pgfpathlineto{\pgfqpoint{3.724665in}{2.050045in}}%
\pgfpathlineto{\pgfqpoint{3.737787in}{2.048811in}}%
\pgfpathlineto{\pgfqpoint{3.750915in}{2.047755in}}%
\pgfpathlineto{\pgfqpoint{3.758653in}{2.057574in}}%
\pgfpathlineto{\pgfqpoint{3.766385in}{2.067389in}}%
\pgfpathlineto{\pgfqpoint{3.774113in}{2.077198in}}%
\pgfpathlineto{\pgfqpoint{3.781836in}{2.087001in}}%
\pgfpathlineto{\pgfqpoint{3.768717in}{2.087952in}}%
\pgfpathlineto{\pgfqpoint{3.755605in}{2.089081in}}%
\pgfpathlineto{\pgfqpoint{3.742499in}{2.090387in}}%
\pgfpathlineto{\pgfqpoint{3.729398in}{2.091873in}}%
\pgfpathlineto{\pgfqpoint{3.721667in}{2.082165in}}%
\pgfpathlineto{\pgfqpoint{3.713930in}{2.072458in}}%
\pgfpathlineto{\pgfqpoint{3.706187in}{2.062752in}}%
\pgfpathlineto{\pgfqpoint{3.698440in}{2.053048in}}%
\pgfpathclose%
\pgfusepath{fill}%
\end{pgfscope}%
\begin{pgfscope}%
\pgfpathrectangle{\pgfqpoint{1.254980in}{0.150000in}}{\pgfqpoint{5.490039in}{5.490039in}}%
\pgfusepath{clip}%
\pgfsetbuttcap%
\pgfsetroundjoin%
\definecolor{currentfill}{rgb}{0.147607,0.511733,0.557049}%
\pgfsetfillcolor{currentfill}%
\pgfsetfillopacity{0.700000}%
\pgfsetlinewidth{0.000000pt}%
\definecolor{currentstroke}{rgb}{0.000000,0.000000,0.000000}%
\pgfsetstrokecolor{currentstroke}%
\pgfsetdash{}{0pt}%
\pgfpathmoveto{\pgfqpoint{5.312205in}{3.003986in}}%
\pgfpathlineto{\pgfqpoint{5.325967in}{3.011534in}}%
\pgfpathlineto{\pgfqpoint{5.339745in}{3.019237in}}%
\pgfpathlineto{\pgfqpoint{5.353538in}{3.027094in}}%
\pgfpathlineto{\pgfqpoint{5.367348in}{3.035105in}}%
\pgfpathlineto{\pgfqpoint{5.374437in}{3.039804in}}%
\pgfpathlineto{\pgfqpoint{5.381521in}{3.044542in}}%
\pgfpathlineto{\pgfqpoint{5.388600in}{3.049325in}}%
\pgfpathlineto{\pgfqpoint{5.395675in}{3.054157in}}%
\pgfpathlineto{\pgfqpoint{5.381888in}{3.046616in}}%
\pgfpathlineto{\pgfqpoint{5.368116in}{3.039229in}}%
\pgfpathlineto{\pgfqpoint{5.354360in}{3.031995in}}%
\pgfpathlineto{\pgfqpoint{5.340620in}{3.024915in}}%
\pgfpathlineto{\pgfqpoint{5.333523in}{3.019604in}}%
\pgfpathlineto{\pgfqpoint{5.326422in}{3.014349in}}%
\pgfpathlineto{\pgfqpoint{5.319316in}{3.009144in}}%
\pgfpathlineto{\pgfqpoint{5.312205in}{3.003986in}}%
\pgfpathclose%
\pgfusepath{fill}%
\end{pgfscope}%
\begin{pgfscope}%
\pgfpathrectangle{\pgfqpoint{1.254980in}{0.150000in}}{\pgfqpoint{5.490039in}{5.490039in}}%
\pgfusepath{clip}%
\pgfsetbuttcap%
\pgfsetroundjoin%
\definecolor{currentfill}{rgb}{0.282623,0.140926,0.457517}%
\pgfsetfillcolor{currentfill}%
\pgfsetfillopacity{0.700000}%
\pgfsetlinewidth{0.000000pt}%
\definecolor{currentstroke}{rgb}{0.000000,0.000000,0.000000}%
\pgfsetstrokecolor{currentstroke}%
\pgfsetdash{}{0pt}%
\pgfpathmoveto{\pgfqpoint{2.880624in}{2.186096in}}%
\pgfpathlineto{\pgfqpoint{2.893777in}{2.173574in}}%
\pgfpathlineto{\pgfqpoint{2.906927in}{2.161286in}}%
\pgfpathlineto{\pgfqpoint{2.920072in}{2.149229in}}%
\pgfpathlineto{\pgfqpoint{2.933214in}{2.137402in}}%
\pgfpathlineto{\pgfqpoint{2.941301in}{2.143529in}}%
\pgfpathlineto{\pgfqpoint{2.949380in}{2.149774in}}%
\pgfpathlineto{\pgfqpoint{2.957449in}{2.156135in}}%
\pgfpathlineto{\pgfqpoint{2.965510in}{2.162608in}}%
\pgfpathlineto{\pgfqpoint{2.952392in}{2.174156in}}%
\pgfpathlineto{\pgfqpoint{2.939272in}{2.185934in}}%
\pgfpathlineto{\pgfqpoint{2.926148in}{2.197944in}}%
\pgfpathlineto{\pgfqpoint{2.913020in}{2.210186in}}%
\pgfpathlineto{\pgfqpoint{2.904935in}{2.203981in}}%
\pgfpathlineto{\pgfqpoint{2.896841in}{2.197896in}}%
\pgfpathlineto{\pgfqpoint{2.888738in}{2.191934in}}%
\pgfpathlineto{\pgfqpoint{2.880624in}{2.186096in}}%
\pgfpathclose%
\pgfusepath{fill}%
\end{pgfscope}%
\begin{pgfscope}%
\pgfpathrectangle{\pgfqpoint{1.254980in}{0.150000in}}{\pgfqpoint{5.490039in}{5.490039in}}%
\pgfusepath{clip}%
\pgfsetbuttcap%
\pgfsetroundjoin%
\definecolor{currentfill}{rgb}{0.277018,0.050344,0.375715}%
\pgfsetfillcolor{currentfill}%
\pgfsetfillopacity{0.700000}%
\pgfsetlinewidth{0.000000pt}%
\definecolor{currentstroke}{rgb}{0.000000,0.000000,0.000000}%
\pgfsetstrokecolor{currentstroke}%
\pgfsetdash{}{0pt}%
\pgfpathmoveto{\pgfqpoint{3.395404in}{1.995180in}}%
\pgfpathlineto{\pgfqpoint{3.408481in}{1.990146in}}%
\pgfpathlineto{\pgfqpoint{3.421561in}{1.985303in}}%
\pgfpathlineto{\pgfqpoint{3.434643in}{1.980651in}}%
\pgfpathlineto{\pgfqpoint{3.447729in}{1.976188in}}%
\pgfpathlineto{\pgfqpoint{3.455577in}{1.985189in}}%
\pgfpathlineto{\pgfqpoint{3.463420in}{1.994228in}}%
\pgfpathlineto{\pgfqpoint{3.471256in}{2.003301in}}%
\pgfpathlineto{\pgfqpoint{3.479087in}{2.012409in}}%
\pgfpathlineto{\pgfqpoint{3.466014in}{2.016683in}}%
\pgfpathlineto{\pgfqpoint{3.452945in}{2.021147in}}%
\pgfpathlineto{\pgfqpoint{3.439879in}{2.025801in}}%
\pgfpathlineto{\pgfqpoint{3.426817in}{2.030646in}}%
\pgfpathlineto{\pgfqpoint{3.418973in}{2.021717in}}%
\pgfpathlineto{\pgfqpoint{3.411123in}{2.012829in}}%
\pgfpathlineto{\pgfqpoint{3.403266in}{2.003983in}}%
\pgfpathlineto{\pgfqpoint{3.395404in}{1.995180in}}%
\pgfpathclose%
\pgfusepath{fill}%
\end{pgfscope}%
\begin{pgfscope}%
\pgfpathrectangle{\pgfqpoint{1.254980in}{0.150000in}}{\pgfqpoint{5.490039in}{5.490039in}}%
\pgfusepath{clip}%
\pgfsetbuttcap%
\pgfsetroundjoin%
\definecolor{currentfill}{rgb}{0.244972,0.287675,0.537260}%
\pgfsetfillcolor{currentfill}%
\pgfsetfillopacity{0.700000}%
\pgfsetlinewidth{0.000000pt}%
\definecolor{currentstroke}{rgb}{0.000000,0.000000,0.000000}%
\pgfsetstrokecolor{currentstroke}%
\pgfsetdash{}{0pt}%
\pgfpathmoveto{\pgfqpoint{4.448424in}{2.452287in}}%
\pgfpathlineto{\pgfqpoint{4.461781in}{2.456749in}}%
\pgfpathlineto{\pgfqpoint{4.475150in}{2.461374in}}%
\pgfpathlineto{\pgfqpoint{4.488531in}{2.466162in}}%
\pgfpathlineto{\pgfqpoint{4.501923in}{2.471112in}}%
\pgfpathlineto{\pgfqpoint{4.509407in}{2.479602in}}%
\pgfpathlineto{\pgfqpoint{4.516885in}{2.488045in}}%
\pgfpathlineto{\pgfqpoint{4.524357in}{2.496443in}}%
\pgfpathlineto{\pgfqpoint{4.531824in}{2.504797in}}%
\pgfpathlineto{\pgfqpoint{4.518440in}{2.499997in}}%
\pgfpathlineto{\pgfqpoint{4.505067in}{2.495358in}}%
\pgfpathlineto{\pgfqpoint{4.491707in}{2.490882in}}%
\pgfpathlineto{\pgfqpoint{4.478358in}{2.486568in}}%
\pgfpathlineto{\pgfqpoint{4.470883in}{2.478054in}}%
\pgfpathlineto{\pgfqpoint{4.463402in}{2.469504in}}%
\pgfpathlineto{\pgfqpoint{4.455916in}{2.460916in}}%
\pgfpathlineto{\pgfqpoint{4.448424in}{2.452287in}}%
\pgfpathclose%
\pgfusepath{fill}%
\end{pgfscope}%
\begin{pgfscope}%
\pgfpathrectangle{\pgfqpoint{1.254980in}{0.150000in}}{\pgfqpoint{5.490039in}{5.490039in}}%
\pgfusepath{clip}%
\pgfsetbuttcap%
\pgfsetroundjoin%
\definecolor{currentfill}{rgb}{0.279566,0.067836,0.391917}%
\pgfsetfillcolor{currentfill}%
\pgfsetfillopacity{0.700000}%
\pgfsetlinewidth{0.000000pt}%
\definecolor{currentstroke}{rgb}{0.000000,0.000000,0.000000}%
\pgfsetstrokecolor{currentstroke}%
\pgfsetdash{}{0pt}%
\pgfpathmoveto{\pgfqpoint{3.122738in}{2.041368in}}%
\pgfpathlineto{\pgfqpoint{3.135831in}{2.032666in}}%
\pgfpathlineto{\pgfqpoint{3.148924in}{2.024173in}}%
\pgfpathlineto{\pgfqpoint{3.162017in}{2.015887in}}%
\pgfpathlineto{\pgfqpoint{3.175110in}{2.007808in}}%
\pgfpathlineto{\pgfqpoint{3.183076in}{2.015426in}}%
\pgfpathlineto{\pgfqpoint{3.191035in}{2.023124in}}%
\pgfpathlineto{\pgfqpoint{3.198987in}{2.030900in}}%
\pgfpathlineto{\pgfqpoint{3.206932in}{2.038752in}}%
\pgfpathlineto{\pgfqpoint{3.193858in}{2.046584in}}%
\pgfpathlineto{\pgfqpoint{3.180785in}{2.054623in}}%
\pgfpathlineto{\pgfqpoint{3.167711in}{2.062870in}}%
\pgfpathlineto{\pgfqpoint{3.154637in}{2.071326in}}%
\pgfpathlineto{\pgfqpoint{3.146674in}{2.063710in}}%
\pgfpathlineto{\pgfqpoint{3.138703in}{2.056177in}}%
\pgfpathlineto{\pgfqpoint{3.130724in}{2.048729in}}%
\pgfpathlineto{\pgfqpoint{3.122738in}{2.041368in}}%
\pgfpathclose%
\pgfusepath{fill}%
\end{pgfscope}%
\begin{pgfscope}%
\pgfpathrectangle{\pgfqpoint{1.254980in}{0.150000in}}{\pgfqpoint{5.490039in}{5.490039in}}%
\pgfusepath{clip}%
\pgfsetbuttcap%
\pgfsetroundjoin%
\definecolor{currentfill}{rgb}{0.140536,0.530132,0.555659}%
\pgfsetfillcolor{currentfill}%
\pgfsetfillopacity{0.700000}%
\pgfsetlinewidth{0.000000pt}%
\definecolor{currentstroke}{rgb}{0.000000,0.000000,0.000000}%
\pgfsetstrokecolor{currentstroke}%
\pgfsetdash{}{0pt}%
\pgfpathmoveto{\pgfqpoint{5.395675in}{3.054157in}}%
\pgfpathlineto{\pgfqpoint{5.409479in}{3.061850in}}%
\pgfpathlineto{\pgfqpoint{5.423298in}{3.069698in}}%
\pgfpathlineto{\pgfqpoint{5.437134in}{3.077698in}}%
\pgfpathlineto{\pgfqpoint{5.450986in}{3.085852in}}%
\pgfpathlineto{\pgfqpoint{5.458033in}{3.090251in}}%
\pgfpathlineto{\pgfqpoint{5.465076in}{3.094703in}}%
\pgfpathlineto{\pgfqpoint{5.472114in}{3.099214in}}%
\pgfpathlineto{\pgfqpoint{5.479148in}{3.103790in}}%
\pgfpathlineto{\pgfqpoint{5.465320in}{3.096135in}}%
\pgfpathlineto{\pgfqpoint{5.451508in}{3.088634in}}%
\pgfpathlineto{\pgfqpoint{5.437712in}{3.081285in}}%
\pgfpathlineto{\pgfqpoint{5.423932in}{3.074088in}}%
\pgfpathlineto{\pgfqpoint{5.416874in}{3.069004in}}%
\pgfpathlineto{\pgfqpoint{5.409812in}{3.063991in}}%
\pgfpathlineto{\pgfqpoint{5.402746in}{3.059044in}}%
\pgfpathlineto{\pgfqpoint{5.395675in}{3.054157in}}%
\pgfpathclose%
\pgfusepath{fill}%
\end{pgfscope}%
\begin{pgfscope}%
\pgfpathrectangle{\pgfqpoint{1.254980in}{0.150000in}}{\pgfqpoint{5.490039in}{5.490039in}}%
\pgfusepath{clip}%
\pgfsetbuttcap%
\pgfsetroundjoin%
\definecolor{currentfill}{rgb}{0.279566,0.067836,0.391917}%
\pgfsetfillcolor{currentfill}%
\pgfsetfillopacity{0.700000}%
\pgfsetlinewidth{0.000000pt}%
\definecolor{currentstroke}{rgb}{0.000000,0.000000,0.000000}%
\pgfsetstrokecolor{currentstroke}%
\pgfsetdash{}{0pt}%
\pgfpathmoveto{\pgfqpoint{3.614974in}{2.022964in}}%
\pgfpathlineto{\pgfqpoint{3.628072in}{2.020519in}}%
\pgfpathlineto{\pgfqpoint{3.641176in}{2.018255in}}%
\pgfpathlineto{\pgfqpoint{3.654284in}{2.016172in}}%
\pgfpathlineto{\pgfqpoint{3.667398in}{2.014269in}}%
\pgfpathlineto{\pgfqpoint{3.675167in}{2.023956in}}%
\pgfpathlineto{\pgfqpoint{3.682930in}{2.033649in}}%
\pgfpathlineto{\pgfqpoint{3.690688in}{2.043347in}}%
\pgfpathlineto{\pgfqpoint{3.698440in}{2.053048in}}%
\pgfpathlineto{\pgfqpoint{3.685336in}{2.054818in}}%
\pgfpathlineto{\pgfqpoint{3.672238in}{2.056768in}}%
\pgfpathlineto{\pgfqpoint{3.659145in}{2.058899in}}%
\pgfpathlineto{\pgfqpoint{3.646057in}{2.061211in}}%
\pgfpathlineto{\pgfqpoint{3.638294in}{2.051633in}}%
\pgfpathlineto{\pgfqpoint{3.630526in}{2.042065in}}%
\pgfpathlineto{\pgfqpoint{3.622752in}{2.032508in}}%
\pgfpathlineto{\pgfqpoint{3.614974in}{2.022964in}}%
\pgfpathclose%
\pgfusepath{fill}%
\end{pgfscope}%
\begin{pgfscope}%
\pgfpathrectangle{\pgfqpoint{1.254980in}{0.150000in}}{\pgfqpoint{5.490039in}{5.490039in}}%
\pgfusepath{clip}%
\pgfsetbuttcap%
\pgfsetroundjoin%
\definecolor{currentfill}{rgb}{0.218130,0.347432,0.550038}%
\pgfsetfillcolor{currentfill}%
\pgfsetfillopacity{0.700000}%
\pgfsetlinewidth{0.000000pt}%
\definecolor{currentstroke}{rgb}{0.000000,0.000000,0.000000}%
\pgfsetstrokecolor{currentstroke}%
\pgfsetdash{}{0pt}%
\pgfpathmoveto{\pgfqpoint{2.509748in}{2.639546in}}%
\pgfpathlineto{\pgfqpoint{2.523113in}{2.619645in}}%
\pgfpathlineto{\pgfqpoint{2.536466in}{2.600041in}}%
\pgfpathlineto{\pgfqpoint{2.549808in}{2.580732in}}%
\pgfpathlineto{\pgfqpoint{2.563140in}{2.561713in}}%
\pgfpathlineto{\pgfqpoint{2.571431in}{2.565774in}}%
\pgfpathlineto{\pgfqpoint{2.579711in}{2.570004in}}%
\pgfpathlineto{\pgfqpoint{2.587978in}{2.574401in}}%
\pgfpathlineto{\pgfqpoint{2.596233in}{2.578961in}}%
\pgfpathlineto{\pgfqpoint{2.582935in}{2.597688in}}%
\pgfpathlineto{\pgfqpoint{2.569626in}{2.616705in}}%
\pgfpathlineto{\pgfqpoint{2.556307in}{2.636016in}}%
\pgfpathlineto{\pgfqpoint{2.542976in}{2.655622in}}%
\pgfpathlineto{\pgfqpoint{2.534688in}{2.651344in}}%
\pgfpathlineto{\pgfqpoint{2.526388in}{2.647236in}}%
\pgfpathlineto{\pgfqpoint{2.518074in}{2.643303in}}%
\pgfpathlineto{\pgfqpoint{2.509748in}{2.639546in}}%
\pgfpathclose%
\pgfusepath{fill}%
\end{pgfscope}%
\begin{pgfscope}%
\pgfpathrectangle{\pgfqpoint{1.254980in}{0.150000in}}{\pgfqpoint{5.490039in}{5.490039in}}%
\pgfusepath{clip}%
\pgfsetbuttcap%
\pgfsetroundjoin%
\definecolor{currentfill}{rgb}{0.233603,0.313828,0.543914}%
\pgfsetfillcolor{currentfill}%
\pgfsetfillopacity{0.700000}%
\pgfsetlinewidth{0.000000pt}%
\definecolor{currentstroke}{rgb}{0.000000,0.000000,0.000000}%
\pgfsetstrokecolor{currentstroke}%
\pgfsetdash{}{0pt}%
\pgfpathmoveto{\pgfqpoint{4.531824in}{2.504797in}}%
\pgfpathlineto{\pgfqpoint{4.545220in}{2.509760in}}%
\pgfpathlineto{\pgfqpoint{4.558628in}{2.514885in}}%
\pgfpathlineto{\pgfqpoint{4.572049in}{2.520171in}}%
\pgfpathlineto{\pgfqpoint{4.585482in}{2.525618in}}%
\pgfpathlineto{\pgfqpoint{4.592934in}{2.533763in}}%
\pgfpathlineto{\pgfqpoint{4.600380in}{2.541863in}}%
\pgfpathlineto{\pgfqpoint{4.607821in}{2.549919in}}%
\pgfpathlineto{\pgfqpoint{4.615256in}{2.557933in}}%
\pgfpathlineto{\pgfqpoint{4.601832in}{2.552664in}}%
\pgfpathlineto{\pgfqpoint{4.588421in}{2.547556in}}%
\pgfpathlineto{\pgfqpoint{4.575021in}{2.542609in}}%
\pgfpathlineto{\pgfqpoint{4.561634in}{2.537824in}}%
\pgfpathlineto{\pgfqpoint{4.554190in}{2.529622in}}%
\pgfpathlineto{\pgfqpoint{4.546740in}{2.521385in}}%
\pgfpathlineto{\pgfqpoint{4.539285in}{2.513111in}}%
\pgfpathlineto{\pgfqpoint{4.531824in}{2.504797in}}%
\pgfpathclose%
\pgfusepath{fill}%
\end{pgfscope}%
\begin{pgfscope}%
\pgfpathrectangle{\pgfqpoint{1.254980in}{0.150000in}}{\pgfqpoint{5.490039in}{5.490039in}}%
\pgfusepath{clip}%
\pgfsetbuttcap%
\pgfsetroundjoin%
\definecolor{currentfill}{rgb}{0.133743,0.548535,0.553541}%
\pgfsetfillcolor{currentfill}%
\pgfsetfillopacity{0.700000}%
\pgfsetlinewidth{0.000000pt}%
\definecolor{currentstroke}{rgb}{0.000000,0.000000,0.000000}%
\pgfsetstrokecolor{currentstroke}%
\pgfsetdash{}{0pt}%
\pgfpathmoveto{\pgfqpoint{5.479148in}{3.103790in}}%
\pgfpathlineto{\pgfqpoint{5.492993in}{3.111597in}}%
\pgfpathlineto{\pgfqpoint{5.506854in}{3.119556in}}%
\pgfpathlineto{\pgfqpoint{5.520731in}{3.127669in}}%
\pgfpathlineto{\pgfqpoint{5.534625in}{3.135935in}}%
\pgfpathlineto{\pgfqpoint{5.541630in}{3.140063in}}%
\pgfpathlineto{\pgfqpoint{5.548631in}{3.144260in}}%
\pgfpathlineto{\pgfqpoint{5.555628in}{3.148533in}}%
\pgfpathlineto{\pgfqpoint{5.562622in}{3.152887in}}%
\pgfpathlineto{\pgfqpoint{5.548753in}{3.145151in}}%
\pgfpathlineto{\pgfqpoint{5.534902in}{3.137566in}}%
\pgfpathlineto{\pgfqpoint{5.521066in}{3.130134in}}%
\pgfpathlineto{\pgfqpoint{5.507247in}{3.122854in}}%
\pgfpathlineto{\pgfqpoint{5.500228in}{3.117962in}}%
\pgfpathlineto{\pgfqpoint{5.493205in}{3.113158in}}%
\pgfpathlineto{\pgfqpoint{5.486178in}{3.108436in}}%
\pgfpathlineto{\pgfqpoint{5.479148in}{3.103790in}}%
\pgfpathclose%
\pgfusepath{fill}%
\end{pgfscope}%
\begin{pgfscope}%
\pgfpathrectangle{\pgfqpoint{1.254980in}{0.150000in}}{\pgfqpoint{5.490039in}{5.490039in}}%
\pgfusepath{clip}%
\pgfsetbuttcap%
\pgfsetroundjoin%
\definecolor{currentfill}{rgb}{0.127568,0.566949,0.550556}%
\pgfsetfillcolor{currentfill}%
\pgfsetfillopacity{0.700000}%
\pgfsetlinewidth{0.000000pt}%
\definecolor{currentstroke}{rgb}{0.000000,0.000000,0.000000}%
\pgfsetstrokecolor{currentstroke}%
\pgfsetdash{}{0pt}%
\pgfpathmoveto{\pgfqpoint{5.562622in}{3.152887in}}%
\pgfpathlineto{\pgfqpoint{5.576506in}{3.160776in}}%
\pgfpathlineto{\pgfqpoint{5.590408in}{3.168816in}}%
\pgfpathlineto{\pgfqpoint{5.604326in}{3.177009in}}%
\pgfpathlineto{\pgfqpoint{5.618262in}{3.185354in}}%
\pgfpathlineto{\pgfqpoint{5.625224in}{3.189248in}}%
\pgfpathlineto{\pgfqpoint{5.632184in}{3.193228in}}%
\pgfpathlineto{\pgfqpoint{5.639140in}{3.197301in}}%
\pgfpathlineto{\pgfqpoint{5.646093in}{3.201473in}}%
\pgfpathlineto{\pgfqpoint{5.632186in}{3.193686in}}%
\pgfpathlineto{\pgfqpoint{5.618296in}{3.186051in}}%
\pgfpathlineto{\pgfqpoint{5.604422in}{3.178567in}}%
\pgfpathlineto{\pgfqpoint{5.590564in}{3.171234in}}%
\pgfpathlineto{\pgfqpoint{5.583583in}{3.166496in}}%
\pgfpathlineto{\pgfqpoint{5.576599in}{3.161862in}}%
\pgfpathlineto{\pgfqpoint{5.569612in}{3.157328in}}%
\pgfpathlineto{\pgfqpoint{5.562622in}{3.152887in}}%
\pgfpathclose%
\pgfusepath{fill}%
\end{pgfscope}%
\begin{pgfscope}%
\pgfpathrectangle{\pgfqpoint{1.254980in}{0.150000in}}{\pgfqpoint{5.490039in}{5.490039in}}%
\pgfusepath{clip}%
\pgfsetbuttcap%
\pgfsetroundjoin%
\definecolor{currentfill}{rgb}{0.283229,0.120777,0.440584}%
\pgfsetfillcolor{currentfill}%
\pgfsetfillopacity{0.700000}%
\pgfsetlinewidth{0.000000pt}%
\definecolor{currentstroke}{rgb}{0.000000,0.000000,0.000000}%
\pgfsetstrokecolor{currentstroke}%
\pgfsetdash{}{0pt}%
\pgfpathmoveto{\pgfqpoint{2.933214in}{2.137402in}}%
\pgfpathlineto{\pgfqpoint{2.946352in}{2.125803in}}%
\pgfpathlineto{\pgfqpoint{2.959487in}{2.114431in}}%
\pgfpathlineto{\pgfqpoint{2.972620in}{2.103285in}}%
\pgfpathlineto{\pgfqpoint{2.985749in}{2.092361in}}%
\pgfpathlineto{\pgfqpoint{2.993813in}{2.098777in}}%
\pgfpathlineto{\pgfqpoint{3.001867in}{2.105303in}}%
\pgfpathlineto{\pgfqpoint{3.009913in}{2.111938in}}%
\pgfpathlineto{\pgfqpoint{3.017951in}{2.118678in}}%
\pgfpathlineto{\pgfqpoint{3.004844in}{2.129324in}}%
\pgfpathlineto{\pgfqpoint{2.991735in}{2.140193in}}%
\pgfpathlineto{\pgfqpoint{2.978624in}{2.151287in}}%
\pgfpathlineto{\pgfqpoint{2.965510in}{2.162608in}}%
\pgfpathlineto{\pgfqpoint{2.957449in}{2.156135in}}%
\pgfpathlineto{\pgfqpoint{2.949380in}{2.149774in}}%
\pgfpathlineto{\pgfqpoint{2.941301in}{2.143529in}}%
\pgfpathlineto{\pgfqpoint{2.933214in}{2.137402in}}%
\pgfpathclose%
\pgfusepath{fill}%
\end{pgfscope}%
\begin{pgfscope}%
\pgfpathrectangle{\pgfqpoint{1.254980in}{0.150000in}}{\pgfqpoint{5.490039in}{5.490039in}}%
\pgfusepath{clip}%
\pgfsetbuttcap%
\pgfsetroundjoin%
\definecolor{currentfill}{rgb}{0.122606,0.585371,0.546557}%
\pgfsetfillcolor{currentfill}%
\pgfsetfillopacity{0.700000}%
\pgfsetlinewidth{0.000000pt}%
\definecolor{currentstroke}{rgb}{0.000000,0.000000,0.000000}%
\pgfsetstrokecolor{currentstroke}%
\pgfsetdash{}{0pt}%
\pgfpathmoveto{\pgfqpoint{5.646093in}{3.201473in}}%
\pgfpathlineto{\pgfqpoint{5.660017in}{3.209411in}}%
\pgfpathlineto{\pgfqpoint{5.673959in}{3.217501in}}%
\pgfpathlineto{\pgfqpoint{5.687917in}{3.225742in}}%
\pgfpathlineto{\pgfqpoint{5.701892in}{3.234135in}}%
\pgfpathlineto{\pgfqpoint{5.708814in}{3.237835in}}%
\pgfpathlineto{\pgfqpoint{5.715732in}{3.241640in}}%
\pgfpathlineto{\pgfqpoint{5.722649in}{3.245556in}}%
\pgfpathlineto{\pgfqpoint{5.729563in}{3.249590in}}%
\pgfpathlineto{\pgfqpoint{5.715617in}{3.241785in}}%
\pgfpathlineto{\pgfqpoint{5.701689in}{3.234130in}}%
\pgfpathlineto{\pgfqpoint{5.687777in}{3.226626in}}%
\pgfpathlineto{\pgfqpoint{5.673883in}{3.219273in}}%
\pgfpathlineto{\pgfqpoint{5.666938in}{3.214643in}}%
\pgfpathlineto{\pgfqpoint{5.659992in}{3.210137in}}%
\pgfpathlineto{\pgfqpoint{5.653044in}{3.205749in}}%
\pgfpathlineto{\pgfqpoint{5.646093in}{3.201473in}}%
\pgfpathclose%
\pgfusepath{fill}%
\end{pgfscope}%
\begin{pgfscope}%
\pgfpathrectangle{\pgfqpoint{1.254980in}{0.150000in}}{\pgfqpoint{5.490039in}{5.490039in}}%
\pgfusepath{clip}%
\pgfsetbuttcap%
\pgfsetroundjoin%
\definecolor{currentfill}{rgb}{0.221989,0.339161,0.548752}%
\pgfsetfillcolor{currentfill}%
\pgfsetfillopacity{0.700000}%
\pgfsetlinewidth{0.000000pt}%
\definecolor{currentstroke}{rgb}{0.000000,0.000000,0.000000}%
\pgfsetstrokecolor{currentstroke}%
\pgfsetdash{}{0pt}%
\pgfpathmoveto{\pgfqpoint{4.615256in}{2.557933in}}%
\pgfpathlineto{\pgfqpoint{4.628692in}{2.563362in}}%
\pgfpathlineto{\pgfqpoint{4.642141in}{2.568953in}}%
\pgfpathlineto{\pgfqpoint{4.655603in}{2.574704in}}%
\pgfpathlineto{\pgfqpoint{4.669078in}{2.580616in}}%
\pgfpathlineto{\pgfqpoint{4.676497in}{2.588394in}}%
\pgfpathlineto{\pgfqpoint{4.683911in}{2.596129in}}%
\pgfpathlineto{\pgfqpoint{4.691318in}{2.603822in}}%
\pgfpathlineto{\pgfqpoint{4.698720in}{2.611477in}}%
\pgfpathlineto{\pgfqpoint{4.685256in}{2.605773in}}%
\pgfpathlineto{\pgfqpoint{4.671804in}{2.600229in}}%
\pgfpathlineto{\pgfqpoint{4.658365in}{2.594845in}}%
\pgfpathlineto{\pgfqpoint{4.644938in}{2.589622in}}%
\pgfpathlineto{\pgfqpoint{4.637526in}{2.581750in}}%
\pgfpathlineto{\pgfqpoint{4.630108in}{2.573846in}}%
\pgfpathlineto{\pgfqpoint{4.622685in}{2.565908in}}%
\pgfpathlineto{\pgfqpoint{4.615256in}{2.557933in}}%
\pgfpathclose%
\pgfusepath{fill}%
\end{pgfscope}%
\begin{pgfscope}%
\pgfpathrectangle{\pgfqpoint{1.254980in}{0.150000in}}{\pgfqpoint{5.490039in}{5.490039in}}%
\pgfusepath{clip}%
\pgfsetbuttcap%
\pgfsetroundjoin%
\definecolor{currentfill}{rgb}{0.119738,0.603785,0.541400}%
\pgfsetfillcolor{currentfill}%
\pgfsetfillopacity{0.700000}%
\pgfsetlinewidth{0.000000pt}%
\definecolor{currentstroke}{rgb}{0.000000,0.000000,0.000000}%
\pgfsetstrokecolor{currentstroke}%
\pgfsetdash{}{0pt}%
\pgfpathmoveto{\pgfqpoint{5.729563in}{3.249590in}}%
\pgfpathlineto{\pgfqpoint{5.743525in}{3.257547in}}%
\pgfpathlineto{\pgfqpoint{5.757505in}{3.265654in}}%
\pgfpathlineto{\pgfqpoint{5.771502in}{3.273912in}}%
\pgfpathlineto{\pgfqpoint{5.785516in}{3.282322in}}%
\pgfpathlineto{\pgfqpoint{5.792397in}{3.285874in}}%
\pgfpathlineto{\pgfqpoint{5.799276in}{3.289551in}}%
\pgfpathlineto{\pgfqpoint{5.806153in}{3.293359in}}%
\pgfpathlineto{\pgfqpoint{5.813029in}{3.297305in}}%
\pgfpathlineto{\pgfqpoint{5.799047in}{3.289512in}}%
\pgfpathlineto{\pgfqpoint{5.785082in}{3.281869in}}%
\pgfpathlineto{\pgfqpoint{5.771134in}{3.274377in}}%
\pgfpathlineto{\pgfqpoint{5.757203in}{3.267034in}}%
\pgfpathlineto{\pgfqpoint{5.750295in}{3.262464in}}%
\pgfpathlineto{\pgfqpoint{5.743385in}{3.258037in}}%
\pgfpathlineto{\pgfqpoint{5.736475in}{3.253748in}}%
\pgfpathlineto{\pgfqpoint{5.729563in}{3.249590in}}%
\pgfpathclose%
\pgfusepath{fill}%
\end{pgfscope}%
\begin{pgfscope}%
\pgfpathrectangle{\pgfqpoint{1.254980in}{0.150000in}}{\pgfqpoint{5.490039in}{5.490039in}}%
\pgfusepath{clip}%
\pgfsetbuttcap%
\pgfsetroundjoin%
\definecolor{currentfill}{rgb}{0.120081,0.622161,0.534946}%
\pgfsetfillcolor{currentfill}%
\pgfsetfillopacity{0.700000}%
\pgfsetlinewidth{0.000000pt}%
\definecolor{currentstroke}{rgb}{0.000000,0.000000,0.000000}%
\pgfsetstrokecolor{currentstroke}%
\pgfsetdash{}{0pt}%
\pgfpathmoveto{\pgfqpoint{5.813029in}{3.297305in}}%
\pgfpathlineto{\pgfqpoint{5.827029in}{3.305248in}}%
\pgfpathlineto{\pgfqpoint{5.841046in}{3.313341in}}%
\pgfpathlineto{\pgfqpoint{5.855081in}{3.321585in}}%
\pgfpathlineto{\pgfqpoint{5.869133in}{3.329979in}}%
\pgfpathlineto{\pgfqpoint{5.875974in}{3.333435in}}%
\pgfpathlineto{\pgfqpoint{5.882815in}{3.337036in}}%
\pgfpathlineto{\pgfqpoint{5.889655in}{3.340789in}}%
\pgfpathlineto{\pgfqpoint{5.896494in}{3.344701in}}%
\pgfpathlineto{\pgfqpoint{5.882477in}{3.336953in}}%
\pgfpathlineto{\pgfqpoint{5.868476in}{3.329353in}}%
\pgfpathlineto{\pgfqpoint{5.854493in}{3.321903in}}%
\pgfpathlineto{\pgfqpoint{5.840527in}{3.314603in}}%
\pgfpathlineto{\pgfqpoint{5.833653in}{3.310037in}}%
\pgfpathlineto{\pgfqpoint{5.826779in}{3.305637in}}%
\pgfpathlineto{\pgfqpoint{5.819904in}{3.301395in}}%
\pgfpathlineto{\pgfqpoint{5.813029in}{3.297305in}}%
\pgfpathclose%
\pgfusepath{fill}%
\end{pgfscope}%
\begin{pgfscope}%
\pgfpathrectangle{\pgfqpoint{1.254980in}{0.150000in}}{\pgfqpoint{5.490039in}{5.490039in}}%
\pgfusepath{clip}%
\pgfsetbuttcap%
\pgfsetroundjoin%
\definecolor{currentfill}{rgb}{0.277941,0.056324,0.381191}%
\pgfsetfillcolor{currentfill}%
\pgfsetfillopacity{0.700000}%
\pgfsetlinewidth{0.000000pt}%
\definecolor{currentstroke}{rgb}{0.000000,0.000000,0.000000}%
\pgfsetstrokecolor{currentstroke}%
\pgfsetdash{}{0pt}%
\pgfpathmoveto{\pgfqpoint{3.531413in}{1.997190in}}%
\pgfpathlineto{\pgfqpoint{3.544504in}{1.993851in}}%
\pgfpathlineto{\pgfqpoint{3.557600in}{1.990696in}}%
\pgfpathlineto{\pgfqpoint{3.570700in}{1.987724in}}%
\pgfpathlineto{\pgfqpoint{3.583804in}{1.984936in}}%
\pgfpathlineto{\pgfqpoint{3.591605in}{1.994418in}}%
\pgfpathlineto{\pgfqpoint{3.599400in}{2.003918in}}%
\pgfpathlineto{\pgfqpoint{3.607190in}{2.013434in}}%
\pgfpathlineto{\pgfqpoint{3.614974in}{2.022964in}}%
\pgfpathlineto{\pgfqpoint{3.601880in}{2.025592in}}%
\pgfpathlineto{\pgfqpoint{3.588792in}{2.028403in}}%
\pgfpathlineto{\pgfqpoint{3.575707in}{2.031397in}}%
\pgfpathlineto{\pgfqpoint{3.562628in}{2.034575in}}%
\pgfpathlineto{\pgfqpoint{3.554832in}{2.025195in}}%
\pgfpathlineto{\pgfqpoint{3.547031in}{2.015837in}}%
\pgfpathlineto{\pgfqpoint{3.539225in}{2.006502in}}%
\pgfpathlineto{\pgfqpoint{3.531413in}{1.997190in}}%
\pgfpathclose%
\pgfusepath{fill}%
\end{pgfscope}%
\begin{pgfscope}%
\pgfpathrectangle{\pgfqpoint{1.254980in}{0.150000in}}{\pgfqpoint{5.490039in}{5.490039in}}%
\pgfusepath{clip}%
\pgfsetbuttcap%
\pgfsetroundjoin%
\definecolor{currentfill}{rgb}{0.124780,0.640461,0.527068}%
\pgfsetfillcolor{currentfill}%
\pgfsetfillopacity{0.700000}%
\pgfsetlinewidth{0.000000pt}%
\definecolor{currentstroke}{rgb}{0.000000,0.000000,0.000000}%
\pgfsetstrokecolor{currentstroke}%
\pgfsetdash{}{0pt}%
\pgfpathmoveto{\pgfqpoint{5.896494in}{3.344701in}}%
\pgfpathlineto{\pgfqpoint{5.910530in}{3.352600in}}%
\pgfpathlineto{\pgfqpoint{5.924583in}{3.360648in}}%
\pgfpathlineto{\pgfqpoint{5.938654in}{3.368846in}}%
\pgfpathlineto{\pgfqpoint{5.952743in}{3.377194in}}%
\pgfpathlineto{\pgfqpoint{5.959546in}{3.380609in}}%
\pgfpathlineto{\pgfqpoint{5.966350in}{3.384192in}}%
\pgfpathlineto{\pgfqpoint{5.973155in}{3.387949in}}%
\pgfpathlineto{\pgfqpoint{5.979960in}{3.391887in}}%
\pgfpathlineto{\pgfqpoint{5.965908in}{3.384214in}}%
\pgfpathlineto{\pgfqpoint{5.951874in}{3.376689in}}%
\pgfpathlineto{\pgfqpoint{5.937857in}{3.369313in}}%
\pgfpathlineto{\pgfqpoint{5.923857in}{3.362086in}}%
\pgfpathlineto{\pgfqpoint{5.917015in}{3.357465in}}%
\pgfpathlineto{\pgfqpoint{5.910174in}{3.353032in}}%
\pgfpathlineto{\pgfqpoint{5.903334in}{3.348780in}}%
\pgfpathlineto{\pgfqpoint{5.896494in}{3.344701in}}%
\pgfpathclose%
\pgfusepath{fill}%
\end{pgfscope}%
\begin{pgfscope}%
\pgfpathrectangle{\pgfqpoint{1.254980in}{0.150000in}}{\pgfqpoint{5.490039in}{5.490039in}}%
\pgfusepath{clip}%
\pgfsetbuttcap%
\pgfsetroundjoin%
\definecolor{currentfill}{rgb}{0.201239,0.383670,0.554294}%
\pgfsetfillcolor{currentfill}%
\pgfsetfillopacity{0.700000}%
\pgfsetlinewidth{0.000000pt}%
\definecolor{currentstroke}{rgb}{0.000000,0.000000,0.000000}%
\pgfsetstrokecolor{currentstroke}%
\pgfsetdash{}{0pt}%
\pgfpathmoveto{\pgfqpoint{2.456172in}{2.722173in}}%
\pgfpathlineto{\pgfqpoint{2.469584in}{2.701056in}}%
\pgfpathlineto{\pgfqpoint{2.482984in}{2.680248in}}%
\pgfpathlineto{\pgfqpoint{2.496372in}{2.659746in}}%
\pgfpathlineto{\pgfqpoint{2.509748in}{2.639546in}}%
\pgfpathlineto{\pgfqpoint{2.518074in}{2.643303in}}%
\pgfpathlineto{\pgfqpoint{2.526388in}{2.647236in}}%
\pgfpathlineto{\pgfqpoint{2.534688in}{2.651344in}}%
\pgfpathlineto{\pgfqpoint{2.542976in}{2.655622in}}%
\pgfpathlineto{\pgfqpoint{2.529635in}{2.675528in}}%
\pgfpathlineto{\pgfqpoint{2.516282in}{2.695735in}}%
\pgfpathlineto{\pgfqpoint{2.502917in}{2.716247in}}%
\pgfpathlineto{\pgfqpoint{2.489540in}{2.737068in}}%
\pgfpathlineto{\pgfqpoint{2.481217in}{2.733073in}}%
\pgfpathlineto{\pgfqpoint{2.472882in}{2.729258in}}%
\pgfpathlineto{\pgfqpoint{2.464534in}{2.725624in}}%
\pgfpathlineto{\pgfqpoint{2.456172in}{2.722173in}}%
\pgfpathclose%
\pgfusepath{fill}%
\end{pgfscope}%
\begin{pgfscope}%
\pgfpathrectangle{\pgfqpoint{1.254980in}{0.150000in}}{\pgfqpoint{5.490039in}{5.490039in}}%
\pgfusepath{clip}%
\pgfsetbuttcap%
\pgfsetroundjoin%
\definecolor{currentfill}{rgb}{0.210503,0.363727,0.552206}%
\pgfsetfillcolor{currentfill}%
\pgfsetfillopacity{0.700000}%
\pgfsetlinewidth{0.000000pt}%
\definecolor{currentstroke}{rgb}{0.000000,0.000000,0.000000}%
\pgfsetstrokecolor{currentstroke}%
\pgfsetdash{}{0pt}%
\pgfpathmoveto{\pgfqpoint{4.698720in}{2.611477in}}%
\pgfpathlineto{\pgfqpoint{4.712198in}{2.617341in}}%
\pgfpathlineto{\pgfqpoint{4.725689in}{2.623364in}}%
\pgfpathlineto{\pgfqpoint{4.739193in}{2.629548in}}%
\pgfpathlineto{\pgfqpoint{4.752710in}{2.635891in}}%
\pgfpathlineto{\pgfqpoint{4.760096in}{2.643284in}}%
\pgfpathlineto{\pgfqpoint{4.767475in}{2.650636in}}%
\pgfpathlineto{\pgfqpoint{4.774849in}{2.657952in}}%
\pgfpathlineto{\pgfqpoint{4.782217in}{2.665233in}}%
\pgfpathlineto{\pgfqpoint{4.768710in}{2.659127in}}%
\pgfpathlineto{\pgfqpoint{4.755217in}{2.653180in}}%
\pgfpathlineto{\pgfqpoint{4.741737in}{2.647392in}}%
\pgfpathlineto{\pgfqpoint{4.728270in}{2.641764in}}%
\pgfpathlineto{\pgfqpoint{4.720891in}{2.634236in}}%
\pgfpathlineto{\pgfqpoint{4.713507in}{2.626681in}}%
\pgfpathlineto{\pgfqpoint{4.706116in}{2.619095in}}%
\pgfpathlineto{\pgfqpoint{4.698720in}{2.611477in}}%
\pgfpathclose%
\pgfusepath{fill}%
\end{pgfscope}%
\begin{pgfscope}%
\pgfpathrectangle{\pgfqpoint{1.254980in}{0.150000in}}{\pgfqpoint{5.490039in}{5.490039in}}%
\pgfusepath{clip}%
\pgfsetbuttcap%
\pgfsetroundjoin%
\definecolor{currentfill}{rgb}{0.276022,0.044167,0.370164}%
\pgfsetfillcolor{currentfill}%
\pgfsetfillopacity{0.700000}%
\pgfsetlinewidth{0.000000pt}%
\definecolor{currentstroke}{rgb}{0.000000,0.000000,0.000000}%
\pgfsetstrokecolor{currentstroke}%
\pgfsetdash{}{0pt}%
\pgfpathmoveto{\pgfqpoint{3.311548in}{1.983383in}}%
\pgfpathlineto{\pgfqpoint{3.324631in}{1.977356in}}%
\pgfpathlineto{\pgfqpoint{3.337716in}{1.971525in}}%
\pgfpathlineto{\pgfqpoint{3.350803in}{1.965888in}}%
\pgfpathlineto{\pgfqpoint{3.363893in}{1.960445in}}%
\pgfpathlineto{\pgfqpoint{3.371780in}{1.969054in}}%
\pgfpathlineto{\pgfqpoint{3.379661in}{1.977714in}}%
\pgfpathlineto{\pgfqpoint{3.387536in}{1.986424in}}%
\pgfpathlineto{\pgfqpoint{3.395404in}{1.995180in}}%
\pgfpathlineto{\pgfqpoint{3.382330in}{2.000407in}}%
\pgfpathlineto{\pgfqpoint{3.369258in}{2.005827in}}%
\pgfpathlineto{\pgfqpoint{3.356189in}{2.011441in}}%
\pgfpathlineto{\pgfqpoint{3.343121in}{2.017250in}}%
\pgfpathlineto{\pgfqpoint{3.335238in}{2.008700in}}%
\pgfpathlineto{\pgfqpoint{3.327348in}{2.000204in}}%
\pgfpathlineto{\pgfqpoint{3.319451in}{1.991764in}}%
\pgfpathlineto{\pgfqpoint{3.311548in}{1.983383in}}%
\pgfpathclose%
\pgfusepath{fill}%
\end{pgfscope}%
\begin{pgfscope}%
\pgfpathrectangle{\pgfqpoint{1.254980in}{0.150000in}}{\pgfqpoint{5.490039in}{5.490039in}}%
\pgfusepath{clip}%
\pgfsetbuttcap%
\pgfsetroundjoin%
\definecolor{currentfill}{rgb}{0.282656,0.100196,0.422160}%
\pgfsetfillcolor{currentfill}%
\pgfsetfillopacity{0.700000}%
\pgfsetlinewidth{0.000000pt}%
\definecolor{currentstroke}{rgb}{0.000000,0.000000,0.000000}%
\pgfsetstrokecolor{currentstroke}%
\pgfsetdash{}{0pt}%
\pgfpathmoveto{\pgfqpoint{2.985749in}{2.092361in}}%
\pgfpathlineto{\pgfqpoint{2.998877in}{2.081660in}}%
\pgfpathlineto{\pgfqpoint{3.012001in}{2.071179in}}%
\pgfpathlineto{\pgfqpoint{3.025124in}{2.060917in}}%
\pgfpathlineto{\pgfqpoint{3.038245in}{2.050873in}}%
\pgfpathlineto{\pgfqpoint{3.046285in}{2.057576in}}%
\pgfpathlineto{\pgfqpoint{3.054317in}{2.064382in}}%
\pgfpathlineto{\pgfqpoint{3.062341in}{2.071290in}}%
\pgfpathlineto{\pgfqpoint{3.070356in}{2.078296in}}%
\pgfpathlineto{\pgfqpoint{3.057257in}{2.088064in}}%
\pgfpathlineto{\pgfqpoint{3.044157in}{2.098050in}}%
\pgfpathlineto{\pgfqpoint{3.031055in}{2.108254in}}%
\pgfpathlineto{\pgfqpoint{3.017951in}{2.118678in}}%
\pgfpathlineto{\pgfqpoint{3.009913in}{2.111938in}}%
\pgfpathlineto{\pgfqpoint{3.001867in}{2.105303in}}%
\pgfpathlineto{\pgfqpoint{2.993813in}{2.098777in}}%
\pgfpathlineto{\pgfqpoint{2.985749in}{2.092361in}}%
\pgfpathclose%
\pgfusepath{fill}%
\end{pgfscope}%
\begin{pgfscope}%
\pgfpathrectangle{\pgfqpoint{1.254980in}{0.150000in}}{\pgfqpoint{5.490039in}{5.490039in}}%
\pgfusepath{clip}%
\pgfsetbuttcap%
\pgfsetroundjoin%
\definecolor{currentfill}{rgb}{0.277941,0.056324,0.381191}%
\pgfsetfillcolor{currentfill}%
\pgfsetfillopacity{0.700000}%
\pgfsetlinewidth{0.000000pt}%
\definecolor{currentstroke}{rgb}{0.000000,0.000000,0.000000}%
\pgfsetstrokecolor{currentstroke}%
\pgfsetdash{}{0pt}%
\pgfpathmoveto{\pgfqpoint{3.175110in}{2.007808in}}%
\pgfpathlineto{\pgfqpoint{3.188202in}{1.999935in}}%
\pgfpathlineto{\pgfqpoint{3.201295in}{1.992265in}}%
\pgfpathlineto{\pgfqpoint{3.214389in}{1.984799in}}%
\pgfpathlineto{\pgfqpoint{3.227483in}{1.977535in}}%
\pgfpathlineto{\pgfqpoint{3.235430in}{1.985409in}}%
\pgfpathlineto{\pgfqpoint{3.243371in}{1.993357in}}%
\pgfpathlineto{\pgfqpoint{3.251305in}{2.001375in}}%
\pgfpathlineto{\pgfqpoint{3.259232in}{2.009462in}}%
\pgfpathlineto{\pgfqpoint{3.246156in}{2.016481in}}%
\pgfpathlineto{\pgfqpoint{3.233080in}{2.023701in}}%
\pgfpathlineto{\pgfqpoint{3.220006in}{2.031124in}}%
\pgfpathlineto{\pgfqpoint{3.206932in}{2.038752in}}%
\pgfpathlineto{\pgfqpoint{3.198987in}{2.030900in}}%
\pgfpathlineto{\pgfqpoint{3.191035in}{2.023124in}}%
\pgfpathlineto{\pgfqpoint{3.183076in}{2.015426in}}%
\pgfpathlineto{\pgfqpoint{3.175110in}{2.007808in}}%
\pgfpathclose%
\pgfusepath{fill}%
\end{pgfscope}%
\begin{pgfscope}%
\pgfpathrectangle{\pgfqpoint{1.254980in}{0.150000in}}{\pgfqpoint{5.490039in}{5.490039in}}%
\pgfusepath{clip}%
\pgfsetbuttcap%
\pgfsetroundjoin%
\definecolor{currentfill}{rgb}{0.201239,0.383670,0.554294}%
\pgfsetfillcolor{currentfill}%
\pgfsetfillopacity{0.700000}%
\pgfsetlinewidth{0.000000pt}%
\definecolor{currentstroke}{rgb}{0.000000,0.000000,0.000000}%
\pgfsetstrokecolor{currentstroke}%
\pgfsetdash{}{0pt}%
\pgfpathmoveto{\pgfqpoint{4.782217in}{2.665233in}}%
\pgfpathlineto{\pgfqpoint{4.795737in}{2.671498in}}%
\pgfpathlineto{\pgfqpoint{4.809270in}{2.677922in}}%
\pgfpathlineto{\pgfqpoint{4.822818in}{2.684506in}}%
\pgfpathlineto{\pgfqpoint{4.836379in}{2.691247in}}%
\pgfpathlineto{\pgfqpoint{4.843729in}{2.698242in}}%
\pgfpathlineto{\pgfqpoint{4.851073in}{2.705201in}}%
\pgfpathlineto{\pgfqpoint{4.858411in}{2.712127in}}%
\pgfpathlineto{\pgfqpoint{4.865743in}{2.719025in}}%
\pgfpathlineto{\pgfqpoint{4.852194in}{2.712550in}}%
\pgfpathlineto{\pgfqpoint{4.838659in}{2.706232in}}%
\pgfpathlineto{\pgfqpoint{4.825137in}{2.700073in}}%
\pgfpathlineto{\pgfqpoint{4.811629in}{2.694073in}}%
\pgfpathlineto{\pgfqpoint{4.804284in}{2.686900in}}%
\pgfpathlineto{\pgfqpoint{4.796934in}{2.679704in}}%
\pgfpathlineto{\pgfqpoint{4.789578in}{2.672483in}}%
\pgfpathlineto{\pgfqpoint{4.782217in}{2.665233in}}%
\pgfpathclose%
\pgfusepath{fill}%
\end{pgfscope}%
\begin{pgfscope}%
\pgfpathrectangle{\pgfqpoint{1.254980in}{0.150000in}}{\pgfqpoint{5.490039in}{5.490039in}}%
\pgfusepath{clip}%
\pgfsetbuttcap%
\pgfsetroundjoin%
\definecolor{currentfill}{rgb}{0.134692,0.658636,0.517649}%
\pgfsetfillcolor{currentfill}%
\pgfsetfillopacity{0.700000}%
\pgfsetlinewidth{0.000000pt}%
\definecolor{currentstroke}{rgb}{0.000000,0.000000,0.000000}%
\pgfsetstrokecolor{currentstroke}%
\pgfsetdash{}{0pt}%
\pgfpathmoveto{\pgfqpoint{5.979960in}{3.391887in}}%
\pgfpathlineto{\pgfqpoint{5.994030in}{3.399710in}}%
\pgfpathlineto{\pgfqpoint{6.008118in}{3.407682in}}%
\pgfpathlineto{\pgfqpoint{6.022223in}{3.415803in}}%
\pgfpathlineto{\pgfqpoint{6.036347in}{3.424073in}}%
\pgfpathlineto{\pgfqpoint{6.043116in}{3.427509in}}%
\pgfpathlineto{\pgfqpoint{6.049886in}{3.431136in}}%
\pgfpathlineto{\pgfqpoint{6.056657in}{3.434960in}}%
\pgfpathlineto{\pgfqpoint{6.042563in}{3.427215in}}%
\pgfpathlineto{\pgfqpoint{6.028486in}{3.419618in}}%
\pgfpathlineto{\pgfqpoint{6.014427in}{3.412169in}}%
\pgfpathlineto{\pgfqpoint{6.000385in}{3.404869in}}%
\pgfpathlineto{\pgfqpoint{5.993575in}{3.400340in}}%
\pgfpathlineto{\pgfqpoint{5.986767in}{3.396015in}}%
\pgfpathlineto{\pgfqpoint{5.979960in}{3.391887in}}%
\pgfpathclose%
\pgfusepath{fill}%
\end{pgfscope}%
\begin{pgfscope}%
\pgfpathrectangle{\pgfqpoint{1.254980in}{0.150000in}}{\pgfqpoint{5.490039in}{5.490039in}}%
\pgfusepath{clip}%
\pgfsetbuttcap%
\pgfsetroundjoin%
\definecolor{currentfill}{rgb}{0.282884,0.135920,0.453427}%
\pgfsetfillcolor{currentfill}%
\pgfsetfillopacity{0.700000}%
\pgfsetlinewidth{0.000000pt}%
\definecolor{currentstroke}{rgb}{0.000000,0.000000,0.000000}%
\pgfsetstrokecolor{currentstroke}%
\pgfsetdash{}{0pt}%
\pgfpathmoveto{\pgfqpoint{3.917803in}{2.125465in}}%
\pgfpathlineto{\pgfqpoint{3.930977in}{2.126162in}}%
\pgfpathlineto{\pgfqpoint{3.944159in}{2.127031in}}%
\pgfpathlineto{\pgfqpoint{3.957349in}{2.128072in}}%
\pgfpathlineto{\pgfqpoint{3.970547in}{2.129283in}}%
\pgfpathlineto{\pgfqpoint{3.978221in}{2.139227in}}%
\pgfpathlineto{\pgfqpoint{3.985889in}{2.149142in}}%
\pgfpathlineto{\pgfqpoint{3.993552in}{2.159027in}}%
\pgfpathlineto{\pgfqpoint{4.001211in}{2.168882in}}%
\pgfpathlineto{\pgfqpoint{3.988020in}{2.167622in}}%
\pgfpathlineto{\pgfqpoint{3.974837in}{2.166532in}}%
\pgfpathlineto{\pgfqpoint{3.961663in}{2.165613in}}%
\pgfpathlineto{\pgfqpoint{3.948496in}{2.164867in}}%
\pgfpathlineto{\pgfqpoint{3.940830in}{2.155050in}}%
\pgfpathlineto{\pgfqpoint{3.933159in}{2.145211in}}%
\pgfpathlineto{\pgfqpoint{3.925483in}{2.135349in}}%
\pgfpathlineto{\pgfqpoint{3.917803in}{2.125465in}}%
\pgfpathclose%
\pgfusepath{fill}%
\end{pgfscope}%
\begin{pgfscope}%
\pgfpathrectangle{\pgfqpoint{1.254980in}{0.150000in}}{\pgfqpoint{5.490039in}{5.490039in}}%
\pgfusepath{clip}%
\pgfsetbuttcap%
\pgfsetroundjoin%
\definecolor{currentfill}{rgb}{0.281412,0.155834,0.469201}%
\pgfsetfillcolor{currentfill}%
\pgfsetfillopacity{0.700000}%
\pgfsetlinewidth{0.000000pt}%
\definecolor{currentstroke}{rgb}{0.000000,0.000000,0.000000}%
\pgfsetstrokecolor{currentstroke}%
\pgfsetdash{}{0pt}%
\pgfpathmoveto{\pgfqpoint{4.001211in}{2.168882in}}%
\pgfpathlineto{\pgfqpoint{4.014410in}{2.170314in}}%
\pgfpathlineto{\pgfqpoint{4.027618in}{2.171915in}}%
\pgfpathlineto{\pgfqpoint{4.040835in}{2.173687in}}%
\pgfpathlineto{\pgfqpoint{4.054060in}{2.175627in}}%
\pgfpathlineto{\pgfqpoint{4.061707in}{2.185485in}}%
\pgfpathlineto{\pgfqpoint{4.069348in}{2.195306in}}%
\pgfpathlineto{\pgfqpoint{4.076984in}{2.205091in}}%
\pgfpathlineto{\pgfqpoint{4.084616in}{2.214840in}}%
\pgfpathlineto{\pgfqpoint{4.071397in}{2.212878in}}%
\pgfpathlineto{\pgfqpoint{4.058188in}{2.211086in}}%
\pgfpathlineto{\pgfqpoint{4.044987in}{2.209463in}}%
\pgfpathlineto{\pgfqpoint{4.031795in}{2.208010in}}%
\pgfpathlineto{\pgfqpoint{4.024156in}{2.198272in}}%
\pgfpathlineto{\pgfqpoint{4.016513in}{2.188505in}}%
\pgfpathlineto{\pgfqpoint{4.008864in}{2.178708in}}%
\pgfpathlineto{\pgfqpoint{4.001211in}{2.168882in}}%
\pgfpathclose%
\pgfusepath{fill}%
\end{pgfscope}%
\begin{pgfscope}%
\pgfpathrectangle{\pgfqpoint{1.254980in}{0.150000in}}{\pgfqpoint{5.490039in}{5.490039in}}%
\pgfusepath{clip}%
\pgfsetbuttcap%
\pgfsetroundjoin%
\definecolor{currentfill}{rgb}{0.190631,0.407061,0.556089}%
\pgfsetfillcolor{currentfill}%
\pgfsetfillopacity{0.700000}%
\pgfsetlinewidth{0.000000pt}%
\definecolor{currentstroke}{rgb}{0.000000,0.000000,0.000000}%
\pgfsetstrokecolor{currentstroke}%
\pgfsetdash{}{0pt}%
\pgfpathmoveto{\pgfqpoint{4.865743in}{2.719025in}}%
\pgfpathlineto{\pgfqpoint{4.879306in}{2.725659in}}%
\pgfpathlineto{\pgfqpoint{4.892883in}{2.732451in}}%
\pgfpathlineto{\pgfqpoint{4.906474in}{2.739401in}}%
\pgfpathlineto{\pgfqpoint{4.920080in}{2.746509in}}%
\pgfpathlineto{\pgfqpoint{4.927393in}{2.753096in}}%
\pgfpathlineto{\pgfqpoint{4.934700in}{2.759654in}}%
\pgfpathlineto{\pgfqpoint{4.942001in}{2.766186in}}%
\pgfpathlineto{\pgfqpoint{4.949297in}{2.772696in}}%
\pgfpathlineto{\pgfqpoint{4.935704in}{2.765883in}}%
\pgfpathlineto{\pgfqpoint{4.922126in}{2.759228in}}%
\pgfpathlineto{\pgfqpoint{4.908562in}{2.752731in}}%
\pgfpathlineto{\pgfqpoint{4.895012in}{2.746392in}}%
\pgfpathlineto{\pgfqpoint{4.887704in}{2.739577in}}%
\pgfpathlineto{\pgfqpoint{4.880389in}{2.732746in}}%
\pgfpathlineto{\pgfqpoint{4.873069in}{2.725897in}}%
\pgfpathlineto{\pgfqpoint{4.865743in}{2.719025in}}%
\pgfpathclose%
\pgfusepath{fill}%
\end{pgfscope}%
\begin{pgfscope}%
\pgfpathrectangle{\pgfqpoint{1.254980in}{0.150000in}}{\pgfqpoint{5.490039in}{5.490039in}}%
\pgfusepath{clip}%
\pgfsetbuttcap%
\pgfsetroundjoin%
\definecolor{currentfill}{rgb}{0.277018,0.050344,0.375715}%
\pgfsetfillcolor{currentfill}%
\pgfsetfillopacity{0.700000}%
\pgfsetlinewidth{0.000000pt}%
\definecolor{currentstroke}{rgb}{0.000000,0.000000,0.000000}%
\pgfsetstrokecolor{currentstroke}%
\pgfsetdash{}{0pt}%
\pgfpathmoveto{\pgfqpoint{3.447729in}{1.976188in}}%
\pgfpathlineto{\pgfqpoint{3.460818in}{1.971913in}}%
\pgfpathlineto{\pgfqpoint{3.473911in}{1.967827in}}%
\pgfpathlineto{\pgfqpoint{3.487007in}{1.963927in}}%
\pgfpathlineto{\pgfqpoint{3.500107in}{1.960214in}}%
\pgfpathlineto{\pgfqpoint{3.507942in}{1.969415in}}%
\pgfpathlineto{\pgfqpoint{3.515771in}{1.978645in}}%
\pgfpathlineto{\pgfqpoint{3.523595in}{1.987904in}}%
\pgfpathlineto{\pgfqpoint{3.531413in}{1.997190in}}%
\pgfpathlineto{\pgfqpoint{3.518325in}{2.000715in}}%
\pgfpathlineto{\pgfqpoint{3.505242in}{2.004426in}}%
\pgfpathlineto{\pgfqpoint{3.492162in}{2.008324in}}%
\pgfpathlineto{\pgfqpoint{3.479087in}{2.012409in}}%
\pgfpathlineto{\pgfqpoint{3.471256in}{2.003301in}}%
\pgfpathlineto{\pgfqpoint{3.463420in}{1.994228in}}%
\pgfpathlineto{\pgfqpoint{3.455577in}{1.985189in}}%
\pgfpathlineto{\pgfqpoint{3.447729in}{1.976188in}}%
\pgfpathclose%
\pgfusepath{fill}%
\end{pgfscope}%
\begin{pgfscope}%
\pgfpathrectangle{\pgfqpoint{1.254980in}{0.150000in}}{\pgfqpoint{5.490039in}{5.490039in}}%
\pgfusepath{clip}%
\pgfsetbuttcap%
\pgfsetroundjoin%
\definecolor{currentfill}{rgb}{0.278012,0.180367,0.486697}%
\pgfsetfillcolor{currentfill}%
\pgfsetfillopacity{0.700000}%
\pgfsetlinewidth{0.000000pt}%
\definecolor{currentstroke}{rgb}{0.000000,0.000000,0.000000}%
\pgfsetstrokecolor{currentstroke}%
\pgfsetdash{}{0pt}%
\pgfpathmoveto{\pgfqpoint{4.084616in}{2.214840in}}%
\pgfpathlineto{\pgfqpoint{4.097843in}{2.216970in}}%
\pgfpathlineto{\pgfqpoint{4.111080in}{2.219269in}}%
\pgfpathlineto{\pgfqpoint{4.124326in}{2.221735in}}%
\pgfpathlineto{\pgfqpoint{4.137581in}{2.224370in}}%
\pgfpathlineto{\pgfqpoint{4.145201in}{2.234087in}}%
\pgfpathlineto{\pgfqpoint{4.152815in}{2.243761in}}%
\pgfpathlineto{\pgfqpoint{4.160424in}{2.253394in}}%
\pgfpathlineto{\pgfqpoint{4.168028in}{2.262985in}}%
\pgfpathlineto{\pgfqpoint{4.154780in}{2.260357in}}%
\pgfpathlineto{\pgfqpoint{4.141540in}{2.257898in}}%
\pgfpathlineto{\pgfqpoint{4.128311in}{2.255606in}}%
\pgfpathlineto{\pgfqpoint{4.115090in}{2.253482in}}%
\pgfpathlineto{\pgfqpoint{4.107479in}{2.243874in}}%
\pgfpathlineto{\pgfqpoint{4.099863in}{2.234231in}}%
\pgfpathlineto{\pgfqpoint{4.092242in}{2.224553in}}%
\pgfpathlineto{\pgfqpoint{4.084616in}{2.214840in}}%
\pgfpathclose%
\pgfusepath{fill}%
\end{pgfscope}%
\begin{pgfscope}%
\pgfpathrectangle{\pgfqpoint{1.254980in}{0.150000in}}{\pgfqpoint{5.490039in}{5.490039in}}%
\pgfusepath{clip}%
\pgfsetbuttcap%
\pgfsetroundjoin%
\definecolor{currentfill}{rgb}{0.283197,0.115680,0.436115}%
\pgfsetfillcolor{currentfill}%
\pgfsetfillopacity{0.700000}%
\pgfsetlinewidth{0.000000pt}%
\definecolor{currentstroke}{rgb}{0.000000,0.000000,0.000000}%
\pgfsetstrokecolor{currentstroke}%
\pgfsetdash{}{0pt}%
\pgfpathmoveto{\pgfqpoint{3.834376in}{2.084958in}}%
\pgfpathlineto{\pgfqpoint{3.847529in}{2.084885in}}%
\pgfpathlineto{\pgfqpoint{3.860688in}{2.084986in}}%
\pgfpathlineto{\pgfqpoint{3.873855in}{2.085261in}}%
\pgfpathlineto{\pgfqpoint{3.887029in}{2.085709in}}%
\pgfpathlineto{\pgfqpoint{3.894730in}{2.095680in}}%
\pgfpathlineto{\pgfqpoint{3.902426in}{2.105630in}}%
\pgfpathlineto{\pgfqpoint{3.910117in}{2.115558in}}%
\pgfpathlineto{\pgfqpoint{3.917803in}{2.125465in}}%
\pgfpathlineto{\pgfqpoint{3.904636in}{2.124940in}}%
\pgfpathlineto{\pgfqpoint{3.891477in}{2.124588in}}%
\pgfpathlineto{\pgfqpoint{3.878326in}{2.124410in}}%
\pgfpathlineto{\pgfqpoint{3.865182in}{2.124405in}}%
\pgfpathlineto{\pgfqpoint{3.857488in}{2.114566in}}%
\pgfpathlineto{\pgfqpoint{3.849789in}{2.104711in}}%
\pgfpathlineto{\pgfqpoint{3.842085in}{2.094842in}}%
\pgfpathlineto{\pgfqpoint{3.834376in}{2.084958in}}%
\pgfpathclose%
\pgfusepath{fill}%
\end{pgfscope}%
\begin{pgfscope}%
\pgfpathrectangle{\pgfqpoint{1.254980in}{0.150000in}}{\pgfqpoint{5.490039in}{5.490039in}}%
\pgfusepath{clip}%
\pgfsetbuttcap%
\pgfsetroundjoin%
\definecolor{currentfill}{rgb}{0.271828,0.209303,0.504434}%
\pgfsetfillcolor{currentfill}%
\pgfsetfillopacity{0.700000}%
\pgfsetlinewidth{0.000000pt}%
\definecolor{currentstroke}{rgb}{0.000000,0.000000,0.000000}%
\pgfsetstrokecolor{currentstroke}%
\pgfsetdash{}{0pt}%
\pgfpathmoveto{\pgfqpoint{4.168028in}{2.262985in}}%
\pgfpathlineto{\pgfqpoint{4.181287in}{2.265779in}}%
\pgfpathlineto{\pgfqpoint{4.194555in}{2.268741in}}%
\pgfpathlineto{\pgfqpoint{4.207833in}{2.271869in}}%
\pgfpathlineto{\pgfqpoint{4.221122in}{2.275163in}}%
\pgfpathlineto{\pgfqpoint{4.228714in}{2.284689in}}%
\pgfpathlineto{\pgfqpoint{4.236301in}{2.294167in}}%
\pgfpathlineto{\pgfqpoint{4.243883in}{2.303600in}}%
\pgfpathlineto{\pgfqpoint{4.251459in}{2.312987in}}%
\pgfpathlineto{\pgfqpoint{4.238177in}{2.309728in}}%
\pgfpathlineto{\pgfqpoint{4.224906in}{2.306635in}}%
\pgfpathlineto{\pgfqpoint{4.211645in}{2.303709in}}%
\pgfpathlineto{\pgfqpoint{4.198393in}{2.300949in}}%
\pgfpathlineto{\pgfqpoint{4.190810in}{2.291517in}}%
\pgfpathlineto{\pgfqpoint{4.183221in}{2.282046in}}%
\pgfpathlineto{\pgfqpoint{4.175627in}{2.272535in}}%
\pgfpathlineto{\pgfqpoint{4.168028in}{2.262985in}}%
\pgfpathclose%
\pgfusepath{fill}%
\end{pgfscope}%
\begin{pgfscope}%
\pgfpathrectangle{\pgfqpoint{1.254980in}{0.150000in}}{\pgfqpoint{5.490039in}{5.490039in}}%
\pgfusepath{clip}%
\pgfsetbuttcap%
\pgfsetroundjoin%
\definecolor{currentfill}{rgb}{0.282327,0.094955,0.417331}%
\pgfsetfillcolor{currentfill}%
\pgfsetfillopacity{0.700000}%
\pgfsetlinewidth{0.000000pt}%
\definecolor{currentstroke}{rgb}{0.000000,0.000000,0.000000}%
\pgfsetstrokecolor{currentstroke}%
\pgfsetdash{}{0pt}%
\pgfpathmoveto{\pgfqpoint{3.750915in}{2.047755in}}%
\pgfpathlineto{\pgfqpoint{3.764049in}{2.046875in}}%
\pgfpathlineto{\pgfqpoint{3.777189in}{2.046172in}}%
\pgfpathlineto{\pgfqpoint{3.790336in}{2.045645in}}%
\pgfpathlineto{\pgfqpoint{3.803490in}{2.045292in}}%
\pgfpathlineto{\pgfqpoint{3.811219in}{2.055227in}}%
\pgfpathlineto{\pgfqpoint{3.818943in}{2.065150in}}%
\pgfpathlineto{\pgfqpoint{3.826662in}{2.075061in}}%
\pgfpathlineto{\pgfqpoint{3.834376in}{2.084958in}}%
\pgfpathlineto{\pgfqpoint{3.821231in}{2.085206in}}%
\pgfpathlineto{\pgfqpoint{3.808092in}{2.085628in}}%
\pgfpathlineto{\pgfqpoint{3.794961in}{2.086226in}}%
\pgfpathlineto{\pgfqpoint{3.781836in}{2.087001in}}%
\pgfpathlineto{\pgfqpoint{3.774113in}{2.077198in}}%
\pgfpathlineto{\pgfqpoint{3.766385in}{2.067389in}}%
\pgfpathlineto{\pgfqpoint{3.758653in}{2.057574in}}%
\pgfpathlineto{\pgfqpoint{3.750915in}{2.047755in}}%
\pgfpathclose%
\pgfusepath{fill}%
\end{pgfscope}%
\begin{pgfscope}%
\pgfpathrectangle{\pgfqpoint{1.254980in}{0.150000in}}{\pgfqpoint{5.490039in}{5.490039in}}%
\pgfusepath{clip}%
\pgfsetbuttcap%
\pgfsetroundjoin%
\definecolor{currentfill}{rgb}{0.180629,0.429975,0.557282}%
\pgfsetfillcolor{currentfill}%
\pgfsetfillopacity{0.700000}%
\pgfsetlinewidth{0.000000pt}%
\definecolor{currentstroke}{rgb}{0.000000,0.000000,0.000000}%
\pgfsetstrokecolor{currentstroke}%
\pgfsetdash{}{0pt}%
\pgfpathmoveto{\pgfqpoint{4.949297in}{2.772696in}}%
\pgfpathlineto{\pgfqpoint{4.962903in}{2.779666in}}%
\pgfpathlineto{\pgfqpoint{4.976524in}{2.786793in}}%
\pgfpathlineto{\pgfqpoint{4.990160in}{2.794078in}}%
\pgfpathlineto{\pgfqpoint{5.003810in}{2.801520in}}%
\pgfpathlineto{\pgfqpoint{5.011085in}{2.807696in}}%
\pgfpathlineto{\pgfqpoint{5.018354in}{2.813850in}}%
\pgfpathlineto{\pgfqpoint{5.025617in}{2.819986in}}%
\pgfpathlineto{\pgfqpoint{5.032874in}{2.826107in}}%
\pgfpathlineto{\pgfqpoint{5.019238in}{2.818991in}}%
\pgfpathlineto{\pgfqpoint{5.005617in}{2.812031in}}%
\pgfpathlineto{\pgfqpoint{4.992011in}{2.805228in}}%
\pgfpathlineto{\pgfqpoint{4.978419in}{2.798582in}}%
\pgfpathlineto{\pgfqpoint{4.971147in}{2.792125in}}%
\pgfpathlineto{\pgfqpoint{4.963869in}{2.785662in}}%
\pgfpathlineto{\pgfqpoint{4.956586in}{2.779186in}}%
\pgfpathlineto{\pgfqpoint{4.949297in}{2.772696in}}%
\pgfpathclose%
\pgfusepath{fill}%
\end{pgfscope}%
\begin{pgfscope}%
\pgfpathrectangle{\pgfqpoint{1.254980in}{0.150000in}}{\pgfqpoint{5.490039in}{5.490039in}}%
\pgfusepath{clip}%
\pgfsetbuttcap%
\pgfsetroundjoin%
\definecolor{currentfill}{rgb}{0.281446,0.084320,0.407414}%
\pgfsetfillcolor{currentfill}%
\pgfsetfillopacity{0.700000}%
\pgfsetlinewidth{0.000000pt}%
\definecolor{currentstroke}{rgb}{0.000000,0.000000,0.000000}%
\pgfsetstrokecolor{currentstroke}%
\pgfsetdash{}{0pt}%
\pgfpathmoveto{\pgfqpoint{3.038245in}{2.050873in}}%
\pgfpathlineto{\pgfqpoint{3.051364in}{2.041045in}}%
\pgfpathlineto{\pgfqpoint{3.064482in}{2.031431in}}%
\pgfpathlineto{\pgfqpoint{3.077598in}{2.022031in}}%
\pgfpathlineto{\pgfqpoint{3.090713in}{2.012843in}}%
\pgfpathlineto{\pgfqpoint{3.098732in}{2.019832in}}%
\pgfpathlineto{\pgfqpoint{3.106742in}{2.026917in}}%
\pgfpathlineto{\pgfqpoint{3.114744in}{2.034097in}}%
\pgfpathlineto{\pgfqpoint{3.122738in}{2.041368in}}%
\pgfpathlineto{\pgfqpoint{3.109644in}{2.050281in}}%
\pgfpathlineto{\pgfqpoint{3.096549in}{2.059406in}}%
\pgfpathlineto{\pgfqpoint{3.083453in}{2.068744in}}%
\pgfpathlineto{\pgfqpoint{3.070356in}{2.078296in}}%
\pgfpathlineto{\pgfqpoint{3.062341in}{2.071290in}}%
\pgfpathlineto{\pgfqpoint{3.054317in}{2.064382in}}%
\pgfpathlineto{\pgfqpoint{3.046285in}{2.057576in}}%
\pgfpathlineto{\pgfqpoint{3.038245in}{2.050873in}}%
\pgfpathclose%
\pgfusepath{fill}%
\end{pgfscope}%
\begin{pgfscope}%
\pgfpathrectangle{\pgfqpoint{1.254980in}{0.150000in}}{\pgfqpoint{5.490039in}{5.490039in}}%
\pgfusepath{clip}%
\pgfsetbuttcap%
\pgfsetroundjoin%
\definecolor{currentfill}{rgb}{0.265145,0.232956,0.516599}%
\pgfsetfillcolor{currentfill}%
\pgfsetfillopacity{0.700000}%
\pgfsetlinewidth{0.000000pt}%
\definecolor{currentstroke}{rgb}{0.000000,0.000000,0.000000}%
\pgfsetstrokecolor{currentstroke}%
\pgfsetdash{}{0pt}%
\pgfpathmoveto{\pgfqpoint{4.251459in}{2.312987in}}%
\pgfpathlineto{\pgfqpoint{4.264751in}{2.316411in}}%
\pgfpathlineto{\pgfqpoint{4.278053in}{2.320001in}}%
\pgfpathlineto{\pgfqpoint{4.291366in}{2.323756in}}%
\pgfpathlineto{\pgfqpoint{4.304690in}{2.327677in}}%
\pgfpathlineto{\pgfqpoint{4.312254in}{2.336966in}}%
\pgfpathlineto{\pgfqpoint{4.319813in}{2.346204in}}%
\pgfpathlineto{\pgfqpoint{4.327367in}{2.355393in}}%
\pgfpathlineto{\pgfqpoint{4.334915in}{2.364533in}}%
\pgfpathlineto{\pgfqpoint{4.321598in}{2.360677in}}%
\pgfpathlineto{\pgfqpoint{4.308292in}{2.356986in}}%
\pgfpathlineto{\pgfqpoint{4.294997in}{2.353459in}}%
\pgfpathlineto{\pgfqpoint{4.281713in}{2.350098in}}%
\pgfpathlineto{\pgfqpoint{4.274157in}{2.340884in}}%
\pgfpathlineto{\pgfqpoint{4.266596in}{2.331628in}}%
\pgfpathlineto{\pgfqpoint{4.259030in}{2.322329in}}%
\pgfpathlineto{\pgfqpoint{4.251459in}{2.312987in}}%
\pgfpathclose%
\pgfusepath{fill}%
\end{pgfscope}%
\begin{pgfscope}%
\pgfpathrectangle{\pgfqpoint{1.254980in}{0.150000in}}{\pgfqpoint{5.490039in}{5.490039in}}%
\pgfusepath{clip}%
\pgfsetbuttcap%
\pgfsetroundjoin%
\definecolor{currentfill}{rgb}{0.280894,0.078907,0.402329}%
\pgfsetfillcolor{currentfill}%
\pgfsetfillopacity{0.700000}%
\pgfsetlinewidth{0.000000pt}%
\definecolor{currentstroke}{rgb}{0.000000,0.000000,0.000000}%
\pgfsetstrokecolor{currentstroke}%
\pgfsetdash{}{0pt}%
\pgfpathmoveto{\pgfqpoint{3.667398in}{2.014269in}}%
\pgfpathlineto{\pgfqpoint{3.680518in}{2.012545in}}%
\pgfpathlineto{\pgfqpoint{3.693644in}{2.011000in}}%
\pgfpathlineto{\pgfqpoint{3.706775in}{2.009634in}}%
\pgfpathlineto{\pgfqpoint{3.719912in}{2.008444in}}%
\pgfpathlineto{\pgfqpoint{3.727670in}{2.018275in}}%
\pgfpathlineto{\pgfqpoint{3.735424in}{2.028104in}}%
\pgfpathlineto{\pgfqpoint{3.743172in}{2.037931in}}%
\pgfpathlineto{\pgfqpoint{3.750915in}{2.047755in}}%
\pgfpathlineto{\pgfqpoint{3.737787in}{2.048811in}}%
\pgfpathlineto{\pgfqpoint{3.724665in}{2.050045in}}%
\pgfpathlineto{\pgfqpoint{3.711550in}{2.051457in}}%
\pgfpathlineto{\pgfqpoint{3.698440in}{2.053048in}}%
\pgfpathlineto{\pgfqpoint{3.690688in}{2.043347in}}%
\pgfpathlineto{\pgfqpoint{3.682930in}{2.033649in}}%
\pgfpathlineto{\pgfqpoint{3.675167in}{2.023956in}}%
\pgfpathlineto{\pgfqpoint{3.667398in}{2.014269in}}%
\pgfpathclose%
\pgfusepath{fill}%
\end{pgfscope}%
\begin{pgfscope}%
\pgfpathrectangle{\pgfqpoint{1.254980in}{0.150000in}}{\pgfqpoint{5.490039in}{5.490039in}}%
\pgfusepath{clip}%
\pgfsetbuttcap%
\pgfsetroundjoin%
\definecolor{currentfill}{rgb}{0.171176,0.452530,0.557965}%
\pgfsetfillcolor{currentfill}%
\pgfsetfillopacity{0.700000}%
\pgfsetlinewidth{0.000000pt}%
\definecolor{currentstroke}{rgb}{0.000000,0.000000,0.000000}%
\pgfsetstrokecolor{currentstroke}%
\pgfsetdash{}{0pt}%
\pgfpathmoveto{\pgfqpoint{5.032874in}{2.826107in}}%
\pgfpathlineto{\pgfqpoint{5.046525in}{2.833381in}}%
\pgfpathlineto{\pgfqpoint{5.060190in}{2.840811in}}%
\pgfpathlineto{\pgfqpoint{5.073871in}{2.848398in}}%
\pgfpathlineto{\pgfqpoint{5.087566in}{2.856141in}}%
\pgfpathlineto{\pgfqpoint{5.094802in}{2.861908in}}%
\pgfpathlineto{\pgfqpoint{5.102031in}{2.867661in}}%
\pgfpathlineto{\pgfqpoint{5.109255in}{2.873404in}}%
\pgfpathlineto{\pgfqpoint{5.116472in}{2.879142in}}%
\pgfpathlineto{\pgfqpoint{5.102793in}{2.871753in}}%
\pgfpathlineto{\pgfqpoint{5.089128in}{2.864521in}}%
\pgfpathlineto{\pgfqpoint{5.075479in}{2.857444in}}%
\pgfpathlineto{\pgfqpoint{5.061844in}{2.850524in}}%
\pgfpathlineto{\pgfqpoint{5.054610in}{2.844422in}}%
\pgfpathlineto{\pgfqpoint{5.047371in}{2.838321in}}%
\pgfpathlineto{\pgfqpoint{5.040125in}{2.832218in}}%
\pgfpathlineto{\pgfqpoint{5.032874in}{2.826107in}}%
\pgfpathclose%
\pgfusepath{fill}%
\end{pgfscope}%
\begin{pgfscope}%
\pgfpathrectangle{\pgfqpoint{1.254980in}{0.150000in}}{\pgfqpoint{5.490039in}{5.490039in}}%
\pgfusepath{clip}%
\pgfsetbuttcap%
\pgfsetroundjoin%
\definecolor{currentfill}{rgb}{0.255645,0.260703,0.528312}%
\pgfsetfillcolor{currentfill}%
\pgfsetfillopacity{0.700000}%
\pgfsetlinewidth{0.000000pt}%
\definecolor{currentstroke}{rgb}{0.000000,0.000000,0.000000}%
\pgfsetstrokecolor{currentstroke}%
\pgfsetdash{}{0pt}%
\pgfpathmoveto{\pgfqpoint{4.334915in}{2.364533in}}%
\pgfpathlineto{\pgfqpoint{4.348242in}{2.368554in}}%
\pgfpathlineto{\pgfqpoint{4.361581in}{2.372739in}}%
\pgfpathlineto{\pgfqpoint{4.374931in}{2.377088in}}%
\pgfpathlineto{\pgfqpoint{4.388292in}{2.381600in}}%
\pgfpathlineto{\pgfqpoint{4.395828in}{2.390612in}}%
\pgfpathlineto{\pgfqpoint{4.403358in}{2.399570in}}%
\pgfpathlineto{\pgfqpoint{4.410883in}{2.408476in}}%
\pgfpathlineto{\pgfqpoint{4.418402in}{2.417333in}}%
\pgfpathlineto{\pgfqpoint{4.405048in}{2.412913in}}%
\pgfpathlineto{\pgfqpoint{4.391706in}{2.408657in}}%
\pgfpathlineto{\pgfqpoint{4.378375in}{2.404564in}}%
\pgfpathlineto{\pgfqpoint{4.365054in}{2.400636in}}%
\pgfpathlineto{\pgfqpoint{4.357528in}{2.391676in}}%
\pgfpathlineto{\pgfqpoint{4.349995in}{2.382674in}}%
\pgfpathlineto{\pgfqpoint{4.342458in}{2.373627in}}%
\pgfpathlineto{\pgfqpoint{4.334915in}{2.364533in}}%
\pgfpathclose%
\pgfusepath{fill}%
\end{pgfscope}%
\begin{pgfscope}%
\pgfpathrectangle{\pgfqpoint{1.254980in}{0.150000in}}{\pgfqpoint{5.490039in}{5.490039in}}%
\pgfusepath{clip}%
\pgfsetbuttcap%
\pgfsetroundjoin%
\definecolor{currentfill}{rgb}{0.267968,0.223549,0.512008}%
\pgfsetfillcolor{currentfill}%
\pgfsetfillopacity{0.700000}%
\pgfsetlinewidth{0.000000pt}%
\definecolor{currentstroke}{rgb}{0.000000,0.000000,0.000000}%
\pgfsetstrokecolor{currentstroke}%
\pgfsetdash{}{0pt}%
\pgfpathmoveto{\pgfqpoint{2.689494in}{2.336922in}}%
\pgfpathlineto{\pgfqpoint{2.702744in}{2.321140in}}%
\pgfpathlineto{\pgfqpoint{2.715987in}{2.305614in}}%
\pgfpathlineto{\pgfqpoint{2.729222in}{2.290344in}}%
\pgfpathlineto{\pgfqpoint{2.742451in}{2.275327in}}%
\pgfpathlineto{\pgfqpoint{2.750662in}{2.279995in}}%
\pgfpathlineto{\pgfqpoint{2.758862in}{2.284813in}}%
\pgfpathlineto{\pgfqpoint{2.767051in}{2.289780in}}%
\pgfpathlineto{\pgfqpoint{2.775229in}{2.294891in}}%
\pgfpathlineto{\pgfqpoint{2.762030in}{2.309596in}}%
\pgfpathlineto{\pgfqpoint{2.748824in}{2.324553in}}%
\pgfpathlineto{\pgfqpoint{2.735612in}{2.339766in}}%
\pgfpathlineto{\pgfqpoint{2.722393in}{2.355234in}}%
\pgfpathlineto{\pgfqpoint{2.714185in}{2.350425in}}%
\pgfpathlineto{\pgfqpoint{2.705967in}{2.345768in}}%
\pgfpathlineto{\pgfqpoint{2.697736in}{2.341266in}}%
\pgfpathlineto{\pgfqpoint{2.689494in}{2.336922in}}%
\pgfpathclose%
\pgfusepath{fill}%
\end{pgfscope}%
\begin{pgfscope}%
\pgfpathrectangle{\pgfqpoint{1.254980in}{0.150000in}}{\pgfqpoint{5.490039in}{5.490039in}}%
\pgfusepath{clip}%
\pgfsetbuttcap%
\pgfsetroundjoin%
\definecolor{currentfill}{rgb}{0.258965,0.251537,0.524736}%
\pgfsetfillcolor{currentfill}%
\pgfsetfillopacity{0.700000}%
\pgfsetlinewidth{0.000000pt}%
\definecolor{currentstroke}{rgb}{0.000000,0.000000,0.000000}%
\pgfsetstrokecolor{currentstroke}%
\pgfsetdash{}{0pt}%
\pgfpathmoveto{\pgfqpoint{2.636420in}{2.402669in}}%
\pgfpathlineto{\pgfqpoint{2.649700in}{2.385835in}}%
\pgfpathlineto{\pgfqpoint{2.662973in}{2.369268in}}%
\pgfpathlineto{\pgfqpoint{2.676237in}{2.352964in}}%
\pgfpathlineto{\pgfqpoint{2.689494in}{2.336922in}}%
\pgfpathlineto{\pgfqpoint{2.697736in}{2.341266in}}%
\pgfpathlineto{\pgfqpoint{2.705967in}{2.345768in}}%
\pgfpathlineto{\pgfqpoint{2.714185in}{2.350425in}}%
\pgfpathlineto{\pgfqpoint{2.722393in}{2.355234in}}%
\pgfpathlineto{\pgfqpoint{2.709167in}{2.370962in}}%
\pgfpathlineto{\pgfqpoint{2.695934in}{2.386951in}}%
\pgfpathlineto{\pgfqpoint{2.682693in}{2.403203in}}%
\pgfpathlineto{\pgfqpoint{2.669444in}{2.419720in}}%
\pgfpathlineto{\pgfqpoint{2.661206in}{2.415215in}}%
\pgfpathlineto{\pgfqpoint{2.652956in}{2.410870in}}%
\pgfpathlineto{\pgfqpoint{2.644694in}{2.406687in}}%
\pgfpathlineto{\pgfqpoint{2.636420in}{2.402669in}}%
\pgfpathclose%
\pgfusepath{fill}%
\end{pgfscope}%
\begin{pgfscope}%
\pgfpathrectangle{\pgfqpoint{1.254980in}{0.150000in}}{\pgfqpoint{5.490039in}{5.490039in}}%
\pgfusepath{clip}%
\pgfsetbuttcap%
\pgfsetroundjoin%
\definecolor{currentfill}{rgb}{0.275191,0.194905,0.496005}%
\pgfsetfillcolor{currentfill}%
\pgfsetfillopacity{0.700000}%
\pgfsetlinewidth{0.000000pt}%
\definecolor{currentstroke}{rgb}{0.000000,0.000000,0.000000}%
\pgfsetstrokecolor{currentstroke}%
\pgfsetdash{}{0pt}%
\pgfpathmoveto{\pgfqpoint{2.742451in}{2.275327in}}%
\pgfpathlineto{\pgfqpoint{2.755674in}{2.260560in}}%
\pgfpathlineto{\pgfqpoint{2.768890in}{2.246042in}}%
\pgfpathlineto{\pgfqpoint{2.782101in}{2.231771in}}%
\pgfpathlineto{\pgfqpoint{2.795306in}{2.217745in}}%
\pgfpathlineto{\pgfqpoint{2.803487in}{2.222735in}}%
\pgfpathlineto{\pgfqpoint{2.811657in}{2.227869in}}%
\pgfpathlineto{\pgfqpoint{2.819818in}{2.233142in}}%
\pgfpathlineto{\pgfqpoint{2.827968in}{2.238553in}}%
\pgfpathlineto{\pgfqpoint{2.814791in}{2.252269in}}%
\pgfpathlineto{\pgfqpoint{2.801609in}{2.266229in}}%
\pgfpathlineto{\pgfqpoint{2.788422in}{2.280436in}}%
\pgfpathlineto{\pgfqpoint{2.775229in}{2.294891in}}%
\pgfpathlineto{\pgfqpoint{2.767051in}{2.289780in}}%
\pgfpathlineto{\pgfqpoint{2.758862in}{2.284813in}}%
\pgfpathlineto{\pgfqpoint{2.750662in}{2.279995in}}%
\pgfpathlineto{\pgfqpoint{2.742451in}{2.275327in}}%
\pgfpathclose%
\pgfusepath{fill}%
\end{pgfscope}%
\begin{pgfscope}%
\pgfpathrectangle{\pgfqpoint{1.254980in}{0.150000in}}{\pgfqpoint{5.490039in}{5.490039in}}%
\pgfusepath{clip}%
\pgfsetbuttcap%
\pgfsetroundjoin%
\definecolor{currentfill}{rgb}{0.277018,0.050344,0.375715}%
\pgfsetfillcolor{currentfill}%
\pgfsetfillopacity{0.700000}%
\pgfsetlinewidth{0.000000pt}%
\definecolor{currentstroke}{rgb}{0.000000,0.000000,0.000000}%
\pgfsetstrokecolor{currentstroke}%
\pgfsetdash{}{0pt}%
\pgfpathmoveto{\pgfqpoint{3.227483in}{1.977535in}}%
\pgfpathlineto{\pgfqpoint{3.240577in}{1.970471in}}%
\pgfpathlineto{\pgfqpoint{3.253673in}{1.963607in}}%
\pgfpathlineto{\pgfqpoint{3.266770in}{1.956942in}}%
\pgfpathlineto{\pgfqpoint{3.279868in}{1.950474in}}%
\pgfpathlineto{\pgfqpoint{3.287798in}{1.958604in}}%
\pgfpathlineto{\pgfqpoint{3.295722in}{1.966800in}}%
\pgfpathlineto{\pgfqpoint{3.303638in}{1.975061in}}%
\pgfpathlineto{\pgfqpoint{3.311548in}{1.983383in}}%
\pgfpathlineto{\pgfqpoint{3.298467in}{1.989605in}}%
\pgfpathlineto{\pgfqpoint{3.285387in}{1.996025in}}%
\pgfpathlineto{\pgfqpoint{3.272309in}{2.002644in}}%
\pgfpathlineto{\pgfqpoint{3.259232in}{2.009462in}}%
\pgfpathlineto{\pgfqpoint{3.251305in}{2.001375in}}%
\pgfpathlineto{\pgfqpoint{3.243371in}{1.993357in}}%
\pgfpathlineto{\pgfqpoint{3.235430in}{1.985409in}}%
\pgfpathlineto{\pgfqpoint{3.227483in}{1.977535in}}%
\pgfpathclose%
\pgfusepath{fill}%
\end{pgfscope}%
\begin{pgfscope}%
\pgfpathrectangle{\pgfqpoint{1.254980in}{0.150000in}}{\pgfqpoint{5.490039in}{5.490039in}}%
\pgfusepath{clip}%
\pgfsetbuttcap%
\pgfsetroundjoin%
\definecolor{currentfill}{rgb}{0.163625,0.471133,0.558148}%
\pgfsetfillcolor{currentfill}%
\pgfsetfillopacity{0.700000}%
\pgfsetlinewidth{0.000000pt}%
\definecolor{currentstroke}{rgb}{0.000000,0.000000,0.000000}%
\pgfsetstrokecolor{currentstroke}%
\pgfsetdash{}{0pt}%
\pgfpathmoveto{\pgfqpoint{5.116472in}{2.879142in}}%
\pgfpathlineto{\pgfqpoint{5.130167in}{2.886686in}}%
\pgfpathlineto{\pgfqpoint{5.143877in}{2.894387in}}%
\pgfpathlineto{\pgfqpoint{5.157602in}{2.902244in}}%
\pgfpathlineto{\pgfqpoint{5.171343in}{2.910257in}}%
\pgfpathlineto{\pgfqpoint{5.178538in}{2.915619in}}%
\pgfpathlineto{\pgfqpoint{5.185726in}{2.920978in}}%
\pgfpathlineto{\pgfqpoint{5.192909in}{2.926338in}}%
\pgfpathlineto{\pgfqpoint{5.200087in}{2.931702in}}%
\pgfpathlineto{\pgfqpoint{5.186364in}{2.924073in}}%
\pgfpathlineto{\pgfqpoint{5.172656in}{2.916601in}}%
\pgfpathlineto{\pgfqpoint{5.158964in}{2.909283in}}%
\pgfpathlineto{\pgfqpoint{5.145286in}{2.902121in}}%
\pgfpathlineto{\pgfqpoint{5.138091in}{2.896364in}}%
\pgfpathlineto{\pgfqpoint{5.130890in}{2.890617in}}%
\pgfpathlineto{\pgfqpoint{5.123684in}{2.884878in}}%
\pgfpathlineto{\pgfqpoint{5.116472in}{2.879142in}}%
\pgfpathclose%
\pgfusepath{fill}%
\end{pgfscope}%
\begin{pgfscope}%
\pgfpathrectangle{\pgfqpoint{1.254980in}{0.150000in}}{\pgfqpoint{5.490039in}{5.490039in}}%
\pgfusepath{clip}%
\pgfsetbuttcap%
\pgfsetroundjoin%
\definecolor{currentfill}{rgb}{0.276022,0.044167,0.370164}%
\pgfsetfillcolor{currentfill}%
\pgfsetfillopacity{0.700000}%
\pgfsetlinewidth{0.000000pt}%
\definecolor{currentstroke}{rgb}{0.000000,0.000000,0.000000}%
\pgfsetstrokecolor{currentstroke}%
\pgfsetdash{}{0pt}%
\pgfpathmoveto{\pgfqpoint{3.363893in}{1.960445in}}%
\pgfpathlineto{\pgfqpoint{3.376984in}{1.955194in}}%
\pgfpathlineto{\pgfqpoint{3.390079in}{1.950134in}}%
\pgfpathlineto{\pgfqpoint{3.403176in}{1.945265in}}%
\pgfpathlineto{\pgfqpoint{3.416276in}{1.940585in}}%
\pgfpathlineto{\pgfqpoint{3.424148in}{1.949422in}}%
\pgfpathlineto{\pgfqpoint{3.432015in}{1.958302in}}%
\pgfpathlineto{\pgfqpoint{3.439875in}{1.967225in}}%
\pgfpathlineto{\pgfqpoint{3.447729in}{1.976188in}}%
\pgfpathlineto{\pgfqpoint{3.434643in}{1.980651in}}%
\pgfpathlineto{\pgfqpoint{3.421561in}{1.985303in}}%
\pgfpathlineto{\pgfqpoint{3.408481in}{1.990146in}}%
\pgfpathlineto{\pgfqpoint{3.395404in}{1.995180in}}%
\pgfpathlineto{\pgfqpoint{3.387536in}{1.986424in}}%
\pgfpathlineto{\pgfqpoint{3.379661in}{1.977714in}}%
\pgfpathlineto{\pgfqpoint{3.371780in}{1.969054in}}%
\pgfpathlineto{\pgfqpoint{3.363893in}{1.960445in}}%
\pgfpathclose%
\pgfusepath{fill}%
\end{pgfscope}%
\begin{pgfscope}%
\pgfpathrectangle{\pgfqpoint{1.254980in}{0.150000in}}{\pgfqpoint{5.490039in}{5.490039in}}%
\pgfusepath{clip}%
\pgfsetbuttcap%
\pgfsetroundjoin%
\definecolor{currentfill}{rgb}{0.246811,0.283237,0.535941}%
\pgfsetfillcolor{currentfill}%
\pgfsetfillopacity{0.700000}%
\pgfsetlinewidth{0.000000pt}%
\definecolor{currentstroke}{rgb}{0.000000,0.000000,0.000000}%
\pgfsetstrokecolor{currentstroke}%
\pgfsetdash{}{0pt}%
\pgfpathmoveto{\pgfqpoint{4.418402in}{2.417333in}}%
\pgfpathlineto{\pgfqpoint{4.431767in}{2.421916in}}%
\pgfpathlineto{\pgfqpoint{4.445144in}{2.426662in}}%
\pgfpathlineto{\pgfqpoint{4.458532in}{2.431571in}}%
\pgfpathlineto{\pgfqpoint{4.471933in}{2.436643in}}%
\pgfpathlineto{\pgfqpoint{4.479439in}{2.445340in}}%
\pgfpathlineto{\pgfqpoint{4.486939in}{2.453983in}}%
\pgfpathlineto{\pgfqpoint{4.494434in}{2.462573in}}%
\pgfpathlineto{\pgfqpoint{4.501923in}{2.471112in}}%
\pgfpathlineto{\pgfqpoint{4.488531in}{2.466162in}}%
\pgfpathlineto{\pgfqpoint{4.475150in}{2.461374in}}%
\pgfpathlineto{\pgfqpoint{4.461781in}{2.456749in}}%
\pgfpathlineto{\pgfqpoint{4.448424in}{2.452287in}}%
\pgfpathlineto{\pgfqpoint{4.440927in}{2.443616in}}%
\pgfpathlineto{\pgfqpoint{4.433424in}{2.434901in}}%
\pgfpathlineto{\pgfqpoint{4.425916in}{2.426140in}}%
\pgfpathlineto{\pgfqpoint{4.418402in}{2.417333in}}%
\pgfpathclose%
\pgfusepath{fill}%
\end{pgfscope}%
\begin{pgfscope}%
\pgfpathrectangle{\pgfqpoint{1.254980in}{0.150000in}}{\pgfqpoint{5.490039in}{5.490039in}}%
\pgfusepath{clip}%
\pgfsetbuttcap%
\pgfsetroundjoin%
\definecolor{currentfill}{rgb}{0.246811,0.283237,0.535941}%
\pgfsetfillcolor{currentfill}%
\pgfsetfillopacity{0.700000}%
\pgfsetlinewidth{0.000000pt}%
\definecolor{currentstroke}{rgb}{0.000000,0.000000,0.000000}%
\pgfsetstrokecolor{currentstroke}%
\pgfsetdash{}{0pt}%
\pgfpathmoveto{\pgfqpoint{2.583209in}{2.472713in}}%
\pgfpathlineto{\pgfqpoint{2.596526in}{2.454790in}}%
\pgfpathlineto{\pgfqpoint{2.609832in}{2.437144in}}%
\pgfpathlineto{\pgfqpoint{2.623130in}{2.419771in}}%
\pgfpathlineto{\pgfqpoint{2.636420in}{2.402669in}}%
\pgfpathlineto{\pgfqpoint{2.644694in}{2.406687in}}%
\pgfpathlineto{\pgfqpoint{2.652956in}{2.410870in}}%
\pgfpathlineto{\pgfqpoint{2.661206in}{2.415215in}}%
\pgfpathlineto{\pgfqpoint{2.669444in}{2.419720in}}%
\pgfpathlineto{\pgfqpoint{2.656187in}{2.436506in}}%
\pgfpathlineto{\pgfqpoint{2.642922in}{2.453562in}}%
\pgfpathlineto{\pgfqpoint{2.629648in}{2.470890in}}%
\pgfpathlineto{\pgfqpoint{2.616366in}{2.488494in}}%
\pgfpathlineto{\pgfqpoint{2.608095in}{2.484295in}}%
\pgfpathlineto{\pgfqpoint{2.599812in}{2.480264in}}%
\pgfpathlineto{\pgfqpoint{2.591517in}{2.476402in}}%
\pgfpathlineto{\pgfqpoint{2.583209in}{2.472713in}}%
\pgfpathclose%
\pgfusepath{fill}%
\end{pgfscope}%
\begin{pgfscope}%
\pgfpathrectangle{\pgfqpoint{1.254980in}{0.150000in}}{\pgfqpoint{5.490039in}{5.490039in}}%
\pgfusepath{clip}%
\pgfsetbuttcap%
\pgfsetroundjoin%
\definecolor{currentfill}{rgb}{0.280255,0.165693,0.476498}%
\pgfsetfillcolor{currentfill}%
\pgfsetfillopacity{0.700000}%
\pgfsetlinewidth{0.000000pt}%
\definecolor{currentstroke}{rgb}{0.000000,0.000000,0.000000}%
\pgfsetstrokecolor{currentstroke}%
\pgfsetdash{}{0pt}%
\pgfpathmoveto{\pgfqpoint{2.795306in}{2.217745in}}%
\pgfpathlineto{\pgfqpoint{2.808505in}{2.203962in}}%
\pgfpathlineto{\pgfqpoint{2.821699in}{2.190420in}}%
\pgfpathlineto{\pgfqpoint{2.834888in}{2.177117in}}%
\pgfpathlineto{\pgfqpoint{2.848073in}{2.164051in}}%
\pgfpathlineto{\pgfqpoint{2.856226in}{2.169361in}}%
\pgfpathlineto{\pgfqpoint{2.864369in}{2.174807in}}%
\pgfpathlineto{\pgfqpoint{2.872501in}{2.180387in}}%
\pgfpathlineto{\pgfqpoint{2.880624in}{2.186096in}}%
\pgfpathlineto{\pgfqpoint{2.867467in}{2.198853in}}%
\pgfpathlineto{\pgfqpoint{2.854305in}{2.211847in}}%
\pgfpathlineto{\pgfqpoint{2.841139in}{2.225080in}}%
\pgfpathlineto{\pgfqpoint{2.827968in}{2.238553in}}%
\pgfpathlineto{\pgfqpoint{2.819818in}{2.233142in}}%
\pgfpathlineto{\pgfqpoint{2.811657in}{2.227869in}}%
\pgfpathlineto{\pgfqpoint{2.803487in}{2.222735in}}%
\pgfpathlineto{\pgfqpoint{2.795306in}{2.217745in}}%
\pgfpathclose%
\pgfusepath{fill}%
\end{pgfscope}%
\begin{pgfscope}%
\pgfpathrectangle{\pgfqpoint{1.254980in}{0.150000in}}{\pgfqpoint{5.490039in}{5.490039in}}%
\pgfusepath{clip}%
\pgfsetbuttcap%
\pgfsetroundjoin%
\definecolor{currentfill}{rgb}{0.278791,0.062145,0.386592}%
\pgfsetfillcolor{currentfill}%
\pgfsetfillopacity{0.700000}%
\pgfsetlinewidth{0.000000pt}%
\definecolor{currentstroke}{rgb}{0.000000,0.000000,0.000000}%
\pgfsetstrokecolor{currentstroke}%
\pgfsetdash{}{0pt}%
\pgfpathmoveto{\pgfqpoint{3.583804in}{1.984936in}}%
\pgfpathlineto{\pgfqpoint{3.596914in}{1.982330in}}%
\pgfpathlineto{\pgfqpoint{3.610028in}{1.979905in}}%
\pgfpathlineto{\pgfqpoint{3.623148in}{1.977661in}}%
\pgfpathlineto{\pgfqpoint{3.636273in}{1.975598in}}%
\pgfpathlineto{\pgfqpoint{3.644062in}{1.985251in}}%
\pgfpathlineto{\pgfqpoint{3.651846in}{1.994915in}}%
\pgfpathlineto{\pgfqpoint{3.659625in}{2.004588in}}%
\pgfpathlineto{\pgfqpoint{3.667398in}{2.014269in}}%
\pgfpathlineto{\pgfqpoint{3.654284in}{2.016172in}}%
\pgfpathlineto{\pgfqpoint{3.641176in}{2.018255in}}%
\pgfpathlineto{\pgfqpoint{3.628072in}{2.020519in}}%
\pgfpathlineto{\pgfqpoint{3.614974in}{2.022964in}}%
\pgfpathlineto{\pgfqpoint{3.607190in}{2.013434in}}%
\pgfpathlineto{\pgfqpoint{3.599400in}{2.003918in}}%
\pgfpathlineto{\pgfqpoint{3.591605in}{1.994418in}}%
\pgfpathlineto{\pgfqpoint{3.583804in}{1.984936in}}%
\pgfpathclose%
\pgfusepath{fill}%
\end{pgfscope}%
\begin{pgfscope}%
\pgfpathrectangle{\pgfqpoint{1.254980in}{0.150000in}}{\pgfqpoint{5.490039in}{5.490039in}}%
\pgfusepath{clip}%
\pgfsetbuttcap%
\pgfsetroundjoin%
\definecolor{currentfill}{rgb}{0.154815,0.493313,0.557840}%
\pgfsetfillcolor{currentfill}%
\pgfsetfillopacity{0.700000}%
\pgfsetlinewidth{0.000000pt}%
\definecolor{currentstroke}{rgb}{0.000000,0.000000,0.000000}%
\pgfsetstrokecolor{currentstroke}%
\pgfsetdash{}{0pt}%
\pgfpathmoveto{\pgfqpoint{5.200087in}{2.931702in}}%
\pgfpathlineto{\pgfqpoint{5.213825in}{2.939485in}}%
\pgfpathlineto{\pgfqpoint{5.227580in}{2.947424in}}%
\pgfpathlineto{\pgfqpoint{5.241350in}{2.955519in}}%
\pgfpathlineto{\pgfqpoint{5.255136in}{2.963768in}}%
\pgfpathlineto{\pgfqpoint{5.262289in}{2.968738in}}%
\pgfpathlineto{\pgfqpoint{5.269436in}{2.973715in}}%
\pgfpathlineto{\pgfqpoint{5.276577in}{2.978704in}}%
\pgfpathlineto{\pgfqpoint{5.283713in}{2.983708in}}%
\pgfpathlineto{\pgfqpoint{5.269947in}{2.975873in}}%
\pgfpathlineto{\pgfqpoint{5.256196in}{2.968192in}}%
\pgfpathlineto{\pgfqpoint{5.242461in}{2.960666in}}%
\pgfpathlineto{\pgfqpoint{5.228741in}{2.953295in}}%
\pgfpathlineto{\pgfqpoint{5.221585in}{2.947867in}}%
\pgfpathlineto{\pgfqpoint{5.214425in}{2.942462in}}%
\pgfpathlineto{\pgfqpoint{5.207258in}{2.937075in}}%
\pgfpathlineto{\pgfqpoint{5.200087in}{2.931702in}}%
\pgfpathclose%
\pgfusepath{fill}%
\end{pgfscope}%
\begin{pgfscope}%
\pgfpathrectangle{\pgfqpoint{1.254980in}{0.150000in}}{\pgfqpoint{5.490039in}{5.490039in}}%
\pgfusepath{clip}%
\pgfsetbuttcap%
\pgfsetroundjoin%
\definecolor{currentfill}{rgb}{0.279566,0.067836,0.391917}%
\pgfsetfillcolor{currentfill}%
\pgfsetfillopacity{0.700000}%
\pgfsetlinewidth{0.000000pt}%
\definecolor{currentstroke}{rgb}{0.000000,0.000000,0.000000}%
\pgfsetstrokecolor{currentstroke}%
\pgfsetdash{}{0pt}%
\pgfpathmoveto{\pgfqpoint{3.090713in}{2.012843in}}%
\pgfpathlineto{\pgfqpoint{3.103828in}{2.003866in}}%
\pgfpathlineto{\pgfqpoint{3.116941in}{1.995098in}}%
\pgfpathlineto{\pgfqpoint{3.130054in}{1.986538in}}%
\pgfpathlineto{\pgfqpoint{3.143167in}{1.978184in}}%
\pgfpathlineto{\pgfqpoint{3.151164in}{1.985458in}}%
\pgfpathlineto{\pgfqpoint{3.159154in}{1.992822in}}%
\pgfpathlineto{\pgfqpoint{3.167135in}{2.000273in}}%
\pgfpathlineto{\pgfqpoint{3.175110in}{2.007808in}}%
\pgfpathlineto{\pgfqpoint{3.162017in}{2.015887in}}%
\pgfpathlineto{\pgfqpoint{3.148924in}{2.024173in}}%
\pgfpathlineto{\pgfqpoint{3.135831in}{2.032666in}}%
\pgfpathlineto{\pgfqpoint{3.122738in}{2.041368in}}%
\pgfpathlineto{\pgfqpoint{3.114744in}{2.034097in}}%
\pgfpathlineto{\pgfqpoint{3.106742in}{2.026917in}}%
\pgfpathlineto{\pgfqpoint{3.098732in}{2.019832in}}%
\pgfpathlineto{\pgfqpoint{3.090713in}{2.012843in}}%
\pgfpathclose%
\pgfusepath{fill}%
\end{pgfscope}%
\begin{pgfscope}%
\pgfpathrectangle{\pgfqpoint{1.254980in}{0.150000in}}{\pgfqpoint{5.490039in}{5.490039in}}%
\pgfusepath{clip}%
\pgfsetbuttcap%
\pgfsetroundjoin%
\definecolor{currentfill}{rgb}{0.235526,0.309527,0.542944}%
\pgfsetfillcolor{currentfill}%
\pgfsetfillopacity{0.700000}%
\pgfsetlinewidth{0.000000pt}%
\definecolor{currentstroke}{rgb}{0.000000,0.000000,0.000000}%
\pgfsetstrokecolor{currentstroke}%
\pgfsetdash{}{0pt}%
\pgfpathmoveto{\pgfqpoint{4.501923in}{2.471112in}}%
\pgfpathlineto{\pgfqpoint{4.515328in}{2.476225in}}%
\pgfpathlineto{\pgfqpoint{4.528745in}{2.481499in}}%
\pgfpathlineto{\pgfqpoint{4.542174in}{2.486935in}}%
\pgfpathlineto{\pgfqpoint{4.555615in}{2.492533in}}%
\pgfpathlineto{\pgfqpoint{4.563090in}{2.500884in}}%
\pgfpathlineto{\pgfqpoint{4.570560in}{2.509180in}}%
\pgfpathlineto{\pgfqpoint{4.578024in}{2.517424in}}%
\pgfpathlineto{\pgfqpoint{4.585482in}{2.525618in}}%
\pgfpathlineto{\pgfqpoint{4.572049in}{2.520171in}}%
\pgfpathlineto{\pgfqpoint{4.558628in}{2.514885in}}%
\pgfpathlineto{\pgfqpoint{4.545220in}{2.509760in}}%
\pgfpathlineto{\pgfqpoint{4.531824in}{2.504797in}}%
\pgfpathlineto{\pgfqpoint{4.524357in}{2.496443in}}%
\pgfpathlineto{\pgfqpoint{4.516885in}{2.488045in}}%
\pgfpathlineto{\pgfqpoint{4.509407in}{2.479602in}}%
\pgfpathlineto{\pgfqpoint{4.501923in}{2.471112in}}%
\pgfpathclose%
\pgfusepath{fill}%
\end{pgfscope}%
\begin{pgfscope}%
\pgfpathrectangle{\pgfqpoint{1.254980in}{0.150000in}}{\pgfqpoint{5.490039in}{5.490039in}}%
\pgfusepath{clip}%
\pgfsetbuttcap%
\pgfsetroundjoin%
\definecolor{currentfill}{rgb}{0.282290,0.145912,0.461510}%
\pgfsetfillcolor{currentfill}%
\pgfsetfillopacity{0.700000}%
\pgfsetlinewidth{0.000000pt}%
\definecolor{currentstroke}{rgb}{0.000000,0.000000,0.000000}%
\pgfsetstrokecolor{currentstroke}%
\pgfsetdash{}{0pt}%
\pgfpathmoveto{\pgfqpoint{2.848073in}{2.164051in}}%
\pgfpathlineto{\pgfqpoint{2.861253in}{2.151221in}}%
\pgfpathlineto{\pgfqpoint{2.874429in}{2.138624in}}%
\pgfpathlineto{\pgfqpoint{2.887600in}{2.126260in}}%
\pgfpathlineto{\pgfqpoint{2.900768in}{2.114126in}}%
\pgfpathlineto{\pgfqpoint{2.908894in}{2.119755in}}%
\pgfpathlineto{\pgfqpoint{2.917010in}{2.125512in}}%
\pgfpathlineto{\pgfqpoint{2.925117in}{2.131395in}}%
\pgfpathlineto{\pgfqpoint{2.933214in}{2.137402in}}%
\pgfpathlineto{\pgfqpoint{2.920072in}{2.149229in}}%
\pgfpathlineto{\pgfqpoint{2.906927in}{2.161286in}}%
\pgfpathlineto{\pgfqpoint{2.893777in}{2.173574in}}%
\pgfpathlineto{\pgfqpoint{2.880624in}{2.186096in}}%
\pgfpathlineto{\pgfqpoint{2.872501in}{2.180387in}}%
\pgfpathlineto{\pgfqpoint{2.864369in}{2.174807in}}%
\pgfpathlineto{\pgfqpoint{2.856226in}{2.169361in}}%
\pgfpathlineto{\pgfqpoint{2.848073in}{2.164051in}}%
\pgfpathclose%
\pgfusepath{fill}%
\end{pgfscope}%
\begin{pgfscope}%
\pgfpathrectangle{\pgfqpoint{1.254980in}{0.150000in}}{\pgfqpoint{5.490039in}{5.490039in}}%
\pgfusepath{clip}%
\pgfsetbuttcap%
\pgfsetroundjoin%
\definecolor{currentfill}{rgb}{0.231674,0.318106,0.544834}%
\pgfsetfillcolor{currentfill}%
\pgfsetfillopacity{0.700000}%
\pgfsetlinewidth{0.000000pt}%
\definecolor{currentstroke}{rgb}{0.000000,0.000000,0.000000}%
\pgfsetstrokecolor{currentstroke}%
\pgfsetdash{}{0pt}%
\pgfpathmoveto{\pgfqpoint{2.529846in}{2.547212in}}%
\pgfpathlineto{\pgfqpoint{2.543202in}{2.528160in}}%
\pgfpathlineto{\pgfqpoint{2.556548in}{2.509395in}}%
\pgfpathlineto{\pgfqpoint{2.569884in}{2.490913in}}%
\pgfpathlineto{\pgfqpoint{2.583209in}{2.472713in}}%
\pgfpathlineto{\pgfqpoint{2.591517in}{2.476402in}}%
\pgfpathlineto{\pgfqpoint{2.599812in}{2.480264in}}%
\pgfpathlineto{\pgfqpoint{2.608095in}{2.484295in}}%
\pgfpathlineto{\pgfqpoint{2.616366in}{2.488494in}}%
\pgfpathlineto{\pgfqpoint{2.603074in}{2.506376in}}%
\pgfpathlineto{\pgfqpoint{2.589772in}{2.524538in}}%
\pgfpathlineto{\pgfqpoint{2.576461in}{2.542982in}}%
\pgfpathlineto{\pgfqpoint{2.563140in}{2.561713in}}%
\pgfpathlineto{\pgfqpoint{2.554836in}{2.557823in}}%
\pgfpathlineto{\pgfqpoint{2.546519in}{2.554108in}}%
\pgfpathlineto{\pgfqpoint{2.538189in}{2.550570in}}%
\pgfpathlineto{\pgfqpoint{2.529846in}{2.547212in}}%
\pgfpathclose%
\pgfusepath{fill}%
\end{pgfscope}%
\begin{pgfscope}%
\pgfpathrectangle{\pgfqpoint{1.254980in}{0.150000in}}{\pgfqpoint{5.490039in}{5.490039in}}%
\pgfusepath{clip}%
\pgfsetbuttcap%
\pgfsetroundjoin%
\definecolor{currentfill}{rgb}{0.147607,0.511733,0.557049}%
\pgfsetfillcolor{currentfill}%
\pgfsetfillopacity{0.700000}%
\pgfsetlinewidth{0.000000pt}%
\definecolor{currentstroke}{rgb}{0.000000,0.000000,0.000000}%
\pgfsetstrokecolor{currentstroke}%
\pgfsetdash{}{0pt}%
\pgfpathmoveto{\pgfqpoint{5.283713in}{2.983708in}}%
\pgfpathlineto{\pgfqpoint{5.297496in}{2.991699in}}%
\pgfpathlineto{\pgfqpoint{5.311294in}{2.999844in}}%
\pgfpathlineto{\pgfqpoint{5.325109in}{3.008144in}}%
\pgfpathlineto{\pgfqpoint{5.338940in}{3.016599in}}%
\pgfpathlineto{\pgfqpoint{5.346050in}{3.021192in}}%
\pgfpathlineto{\pgfqpoint{5.353155in}{3.025804in}}%
\pgfpathlineto{\pgfqpoint{5.360254in}{3.030440in}}%
\pgfpathlineto{\pgfqpoint{5.367348in}{3.035105in}}%
\pgfpathlineto{\pgfqpoint{5.353538in}{3.027094in}}%
\pgfpathlineto{\pgfqpoint{5.339745in}{3.019237in}}%
\pgfpathlineto{\pgfqpoint{5.325967in}{3.011534in}}%
\pgfpathlineto{\pgfqpoint{5.312205in}{3.003986in}}%
\pgfpathlineto{\pgfqpoint{5.305090in}{2.998868in}}%
\pgfpathlineto{\pgfqpoint{5.297969in}{2.993786in}}%
\pgfpathlineto{\pgfqpoint{5.290844in}{2.988734in}}%
\pgfpathlineto{\pgfqpoint{5.283713in}{2.983708in}}%
\pgfpathclose%
\pgfusepath{fill}%
\end{pgfscope}%
\begin{pgfscope}%
\pgfpathrectangle{\pgfqpoint{1.254980in}{0.150000in}}{\pgfqpoint{5.490039in}{5.490039in}}%
\pgfusepath{clip}%
\pgfsetbuttcap%
\pgfsetroundjoin%
\definecolor{currentfill}{rgb}{0.139147,0.533812,0.555298}%
\pgfsetfillcolor{currentfill}%
\pgfsetfillopacity{0.700000}%
\pgfsetlinewidth{0.000000pt}%
\definecolor{currentstroke}{rgb}{0.000000,0.000000,0.000000}%
\pgfsetstrokecolor{currentstroke}%
\pgfsetdash{}{0pt}%
\pgfpathmoveto{\pgfqpoint{5.367348in}{3.035105in}}%
\pgfpathlineto{\pgfqpoint{5.381174in}{3.043270in}}%
\pgfpathlineto{\pgfqpoint{5.395016in}{3.051589in}}%
\pgfpathlineto{\pgfqpoint{5.408875in}{3.060062in}}%
\pgfpathlineto{\pgfqpoint{5.422750in}{3.068690in}}%
\pgfpathlineto{\pgfqpoint{5.429817in}{3.072927in}}%
\pgfpathlineto{\pgfqpoint{5.436878in}{3.077196in}}%
\pgfpathlineto{\pgfqpoint{5.443935in}{3.081503in}}%
\pgfpathlineto{\pgfqpoint{5.450986in}{3.085852in}}%
\pgfpathlineto{\pgfqpoint{5.437134in}{3.077698in}}%
\pgfpathlineto{\pgfqpoint{5.423298in}{3.069698in}}%
\pgfpathlineto{\pgfqpoint{5.409479in}{3.061850in}}%
\pgfpathlineto{\pgfqpoint{5.395675in}{3.054157in}}%
\pgfpathlineto{\pgfqpoint{5.388600in}{3.049325in}}%
\pgfpathlineto{\pgfqpoint{5.381521in}{3.044542in}}%
\pgfpathlineto{\pgfqpoint{5.374437in}{3.039804in}}%
\pgfpathlineto{\pgfqpoint{5.367348in}{3.035105in}}%
\pgfpathclose%
\pgfusepath{fill}%
\end{pgfscope}%
\begin{pgfscope}%
\pgfpathrectangle{\pgfqpoint{1.254980in}{0.150000in}}{\pgfqpoint{5.490039in}{5.490039in}}%
\pgfusepath{clip}%
\pgfsetbuttcap%
\pgfsetroundjoin%
\definecolor{currentfill}{rgb}{0.223925,0.334994,0.548053}%
\pgfsetfillcolor{currentfill}%
\pgfsetfillopacity{0.700000}%
\pgfsetlinewidth{0.000000pt}%
\definecolor{currentstroke}{rgb}{0.000000,0.000000,0.000000}%
\pgfsetstrokecolor{currentstroke}%
\pgfsetdash{}{0pt}%
\pgfpathmoveto{\pgfqpoint{4.585482in}{2.525618in}}%
\pgfpathlineto{\pgfqpoint{4.598927in}{2.531227in}}%
\pgfpathlineto{\pgfqpoint{4.612385in}{2.536996in}}%
\pgfpathlineto{\pgfqpoint{4.625856in}{2.542927in}}%
\pgfpathlineto{\pgfqpoint{4.639340in}{2.549018in}}%
\pgfpathlineto{\pgfqpoint{4.646783in}{2.556995in}}%
\pgfpathlineto{\pgfqpoint{4.654221in}{2.564919in}}%
\pgfpathlineto{\pgfqpoint{4.661652in}{2.572792in}}%
\pgfpathlineto{\pgfqpoint{4.669078in}{2.580616in}}%
\pgfpathlineto{\pgfqpoint{4.655603in}{2.574704in}}%
\pgfpathlineto{\pgfqpoint{4.642141in}{2.568953in}}%
\pgfpathlineto{\pgfqpoint{4.628692in}{2.563362in}}%
\pgfpathlineto{\pgfqpoint{4.615256in}{2.557933in}}%
\pgfpathlineto{\pgfqpoint{4.607821in}{2.549919in}}%
\pgfpathlineto{\pgfqpoint{4.600380in}{2.541863in}}%
\pgfpathlineto{\pgfqpoint{4.592934in}{2.533763in}}%
\pgfpathlineto{\pgfqpoint{4.585482in}{2.525618in}}%
\pgfpathclose%
\pgfusepath{fill}%
\end{pgfscope}%
\begin{pgfscope}%
\pgfpathrectangle{\pgfqpoint{1.254980in}{0.150000in}}{\pgfqpoint{5.490039in}{5.490039in}}%
\pgfusepath{clip}%
\pgfsetbuttcap%
\pgfsetroundjoin%
\definecolor{currentfill}{rgb}{0.277018,0.050344,0.375715}%
\pgfsetfillcolor{currentfill}%
\pgfsetfillopacity{0.700000}%
\pgfsetlinewidth{0.000000pt}%
\definecolor{currentstroke}{rgb}{0.000000,0.000000,0.000000}%
\pgfsetstrokecolor{currentstroke}%
\pgfsetdash{}{0pt}%
\pgfpathmoveto{\pgfqpoint{3.500107in}{1.960214in}}%
\pgfpathlineto{\pgfqpoint{3.513211in}{1.956686in}}%
\pgfpathlineto{\pgfqpoint{3.526318in}{1.953342in}}%
\pgfpathlineto{\pgfqpoint{3.539431in}{1.950183in}}%
\pgfpathlineto{\pgfqpoint{3.552547in}{1.947206in}}%
\pgfpathlineto{\pgfqpoint{3.560370in}{1.956605in}}%
\pgfpathlineto{\pgfqpoint{3.568187in}{1.966028in}}%
\pgfpathlineto{\pgfqpoint{3.575999in}{1.975472in}}%
\pgfpathlineto{\pgfqpoint{3.583804in}{1.984936in}}%
\pgfpathlineto{\pgfqpoint{3.570700in}{1.987724in}}%
\pgfpathlineto{\pgfqpoint{3.557600in}{1.990696in}}%
\pgfpathlineto{\pgfqpoint{3.544504in}{1.993851in}}%
\pgfpathlineto{\pgfqpoint{3.531413in}{1.997190in}}%
\pgfpathlineto{\pgfqpoint{3.523595in}{1.987904in}}%
\pgfpathlineto{\pgfqpoint{3.515771in}{1.978645in}}%
\pgfpathlineto{\pgfqpoint{3.507942in}{1.969415in}}%
\pgfpathlineto{\pgfqpoint{3.500107in}{1.960214in}}%
\pgfpathclose%
\pgfusepath{fill}%
\end{pgfscope}%
\begin{pgfscope}%
\pgfpathrectangle{\pgfqpoint{1.254980in}{0.150000in}}{\pgfqpoint{5.490039in}{5.490039in}}%
\pgfusepath{clip}%
\pgfsetbuttcap%
\pgfsetroundjoin%
\definecolor{currentfill}{rgb}{0.283229,0.120777,0.440584}%
\pgfsetfillcolor{currentfill}%
\pgfsetfillopacity{0.700000}%
\pgfsetlinewidth{0.000000pt}%
\definecolor{currentstroke}{rgb}{0.000000,0.000000,0.000000}%
\pgfsetstrokecolor{currentstroke}%
\pgfsetdash{}{0pt}%
\pgfpathmoveto{\pgfqpoint{2.900768in}{2.114126in}}%
\pgfpathlineto{\pgfqpoint{2.913932in}{2.102221in}}%
\pgfpathlineto{\pgfqpoint{2.927093in}{2.090542in}}%
\pgfpathlineto{\pgfqpoint{2.940250in}{2.079090in}}%
\pgfpathlineto{\pgfqpoint{2.953405in}{2.067861in}}%
\pgfpathlineto{\pgfqpoint{2.961505in}{2.073806in}}%
\pgfpathlineto{\pgfqpoint{2.969595in}{2.079873in}}%
\pgfpathlineto{\pgfqpoint{2.977677in}{2.086059in}}%
\pgfpathlineto{\pgfqpoint{2.985749in}{2.092361in}}%
\pgfpathlineto{\pgfqpoint{2.972620in}{2.103285in}}%
\pgfpathlineto{\pgfqpoint{2.959487in}{2.114431in}}%
\pgfpathlineto{\pgfqpoint{2.946352in}{2.125803in}}%
\pgfpathlineto{\pgfqpoint{2.933214in}{2.137402in}}%
\pgfpathlineto{\pgfqpoint{2.925117in}{2.131395in}}%
\pgfpathlineto{\pgfqpoint{2.917010in}{2.125512in}}%
\pgfpathlineto{\pgfqpoint{2.908894in}{2.119755in}}%
\pgfpathlineto{\pgfqpoint{2.900768in}{2.114126in}}%
\pgfpathclose%
\pgfusepath{fill}%
\end{pgfscope}%
\begin{pgfscope}%
\pgfpathrectangle{\pgfqpoint{1.254980in}{0.150000in}}{\pgfqpoint{5.490039in}{5.490039in}}%
\pgfusepath{clip}%
\pgfsetbuttcap%
\pgfsetroundjoin%
\definecolor{currentfill}{rgb}{0.132444,0.552216,0.553018}%
\pgfsetfillcolor{currentfill}%
\pgfsetfillopacity{0.700000}%
\pgfsetlinewidth{0.000000pt}%
\definecolor{currentstroke}{rgb}{0.000000,0.000000,0.000000}%
\pgfsetstrokecolor{currentstroke}%
\pgfsetdash{}{0pt}%
\pgfpathmoveto{\pgfqpoint{5.450986in}{3.085852in}}%
\pgfpathlineto{\pgfqpoint{5.464855in}{3.094160in}}%
\pgfpathlineto{\pgfqpoint{5.478741in}{3.102621in}}%
\pgfpathlineto{\pgfqpoint{5.492643in}{3.111236in}}%
\pgfpathlineto{\pgfqpoint{5.506563in}{3.120005in}}%
\pgfpathlineto{\pgfqpoint{5.513585in}{3.123911in}}%
\pgfpathlineto{\pgfqpoint{5.520603in}{3.127865in}}%
\pgfpathlineto{\pgfqpoint{5.527616in}{3.131870in}}%
\pgfpathlineto{\pgfqpoint{5.534625in}{3.135935in}}%
\pgfpathlineto{\pgfqpoint{5.520731in}{3.127669in}}%
\pgfpathlineto{\pgfqpoint{5.506854in}{3.119556in}}%
\pgfpathlineto{\pgfqpoint{5.492993in}{3.111597in}}%
\pgfpathlineto{\pgfqpoint{5.479148in}{3.103790in}}%
\pgfpathlineto{\pgfqpoint{5.472114in}{3.099214in}}%
\pgfpathlineto{\pgfqpoint{5.465076in}{3.094703in}}%
\pgfpathlineto{\pgfqpoint{5.458033in}{3.090251in}}%
\pgfpathlineto{\pgfqpoint{5.450986in}{3.085852in}}%
\pgfpathclose%
\pgfusepath{fill}%
\end{pgfscope}%
\begin{pgfscope}%
\pgfpathrectangle{\pgfqpoint{1.254980in}{0.150000in}}{\pgfqpoint{5.490039in}{5.490039in}}%
\pgfusepath{clip}%
\pgfsetbuttcap%
\pgfsetroundjoin%
\definecolor{currentfill}{rgb}{0.216210,0.351535,0.550627}%
\pgfsetfillcolor{currentfill}%
\pgfsetfillopacity{0.700000}%
\pgfsetlinewidth{0.000000pt}%
\definecolor{currentstroke}{rgb}{0.000000,0.000000,0.000000}%
\pgfsetstrokecolor{currentstroke}%
\pgfsetdash{}{0pt}%
\pgfpathmoveto{\pgfqpoint{2.476312in}{2.626336in}}%
\pgfpathlineto{\pgfqpoint{2.489713in}{2.606112in}}%
\pgfpathlineto{\pgfqpoint{2.503102in}{2.586185in}}%
\pgfpathlineto{\pgfqpoint{2.516480in}{2.566552in}}%
\pgfpathlineto{\pgfqpoint{2.529846in}{2.547212in}}%
\pgfpathlineto{\pgfqpoint{2.538189in}{2.550570in}}%
\pgfpathlineto{\pgfqpoint{2.546519in}{2.554108in}}%
\pgfpathlineto{\pgfqpoint{2.554836in}{2.557823in}}%
\pgfpathlineto{\pgfqpoint{2.563140in}{2.561713in}}%
\pgfpathlineto{\pgfqpoint{2.549808in}{2.580732in}}%
\pgfpathlineto{\pgfqpoint{2.536466in}{2.600041in}}%
\pgfpathlineto{\pgfqpoint{2.523113in}{2.619645in}}%
\pgfpathlineto{\pgfqpoint{2.509748in}{2.639546in}}%
\pgfpathlineto{\pgfqpoint{2.501409in}{2.635967in}}%
\pgfpathlineto{\pgfqpoint{2.493057in}{2.632571in}}%
\pgfpathlineto{\pgfqpoint{2.484691in}{2.629360in}}%
\pgfpathlineto{\pgfqpoint{2.476312in}{2.626336in}}%
\pgfpathclose%
\pgfusepath{fill}%
\end{pgfscope}%
\begin{pgfscope}%
\pgfpathrectangle{\pgfqpoint{1.254980in}{0.150000in}}{\pgfqpoint{5.490039in}{5.490039in}}%
\pgfusepath{clip}%
\pgfsetbuttcap%
\pgfsetroundjoin%
\definecolor{currentfill}{rgb}{0.126453,0.570633,0.549841}%
\pgfsetfillcolor{currentfill}%
\pgfsetfillopacity{0.700000}%
\pgfsetlinewidth{0.000000pt}%
\definecolor{currentstroke}{rgb}{0.000000,0.000000,0.000000}%
\pgfsetstrokecolor{currentstroke}%
\pgfsetdash{}{0pt}%
\pgfpathmoveto{\pgfqpoint{5.534625in}{3.135935in}}%
\pgfpathlineto{\pgfqpoint{5.548537in}{3.144353in}}%
\pgfpathlineto{\pgfqpoint{5.562465in}{3.152925in}}%
\pgfpathlineto{\pgfqpoint{5.576410in}{3.161649in}}%
\pgfpathlineto{\pgfqpoint{5.590372in}{3.170527in}}%
\pgfpathlineto{\pgfqpoint{5.597351in}{3.174133in}}%
\pgfpathlineto{\pgfqpoint{5.604325in}{3.177803in}}%
\pgfpathlineto{\pgfqpoint{5.611295in}{3.181541in}}%
\pgfpathlineto{\pgfqpoint{5.618262in}{3.185354in}}%
\pgfpathlineto{\pgfqpoint{5.604326in}{3.177009in}}%
\pgfpathlineto{\pgfqpoint{5.590408in}{3.168816in}}%
\pgfpathlineto{\pgfqpoint{5.576506in}{3.160776in}}%
\pgfpathlineto{\pgfqpoint{5.562622in}{3.152887in}}%
\pgfpathlineto{\pgfqpoint{5.555628in}{3.148533in}}%
\pgfpathlineto{\pgfqpoint{5.548631in}{3.144260in}}%
\pgfpathlineto{\pgfqpoint{5.541630in}{3.140063in}}%
\pgfpathlineto{\pgfqpoint{5.534625in}{3.135935in}}%
\pgfpathclose%
\pgfusepath{fill}%
\end{pgfscope}%
\begin{pgfscope}%
\pgfpathrectangle{\pgfqpoint{1.254980in}{0.150000in}}{\pgfqpoint{5.490039in}{5.490039in}}%
\pgfusepath{clip}%
\pgfsetbuttcap%
\pgfsetroundjoin%
\definecolor{currentfill}{rgb}{0.212395,0.359683,0.551710}%
\pgfsetfillcolor{currentfill}%
\pgfsetfillopacity{0.700000}%
\pgfsetlinewidth{0.000000pt}%
\definecolor{currentstroke}{rgb}{0.000000,0.000000,0.000000}%
\pgfsetstrokecolor{currentstroke}%
\pgfsetdash{}{0pt}%
\pgfpathmoveto{\pgfqpoint{4.669078in}{2.580616in}}%
\pgfpathlineto{\pgfqpoint{4.682565in}{2.586688in}}%
\pgfpathlineto{\pgfqpoint{4.696066in}{2.592920in}}%
\pgfpathlineto{\pgfqpoint{4.709580in}{2.599312in}}%
\pgfpathlineto{\pgfqpoint{4.723108in}{2.605865in}}%
\pgfpathlineto{\pgfqpoint{4.730518in}{2.613445in}}%
\pgfpathlineto{\pgfqpoint{4.737921in}{2.620974in}}%
\pgfpathlineto{\pgfqpoint{4.745319in}{2.628455in}}%
\pgfpathlineto{\pgfqpoint{4.752710in}{2.635891in}}%
\pgfpathlineto{\pgfqpoint{4.739193in}{2.629548in}}%
\pgfpathlineto{\pgfqpoint{4.725689in}{2.623364in}}%
\pgfpathlineto{\pgfqpoint{4.712198in}{2.617341in}}%
\pgfpathlineto{\pgfqpoint{4.698720in}{2.611477in}}%
\pgfpathlineto{\pgfqpoint{4.691318in}{2.603822in}}%
\pgfpathlineto{\pgfqpoint{4.683911in}{2.596129in}}%
\pgfpathlineto{\pgfqpoint{4.676497in}{2.588394in}}%
\pgfpathlineto{\pgfqpoint{4.669078in}{2.580616in}}%
\pgfpathclose%
\pgfusepath{fill}%
\end{pgfscope}%
\begin{pgfscope}%
\pgfpathrectangle{\pgfqpoint{1.254980in}{0.150000in}}{\pgfqpoint{5.490039in}{5.490039in}}%
\pgfusepath{clip}%
\pgfsetbuttcap%
\pgfsetroundjoin%
\definecolor{currentfill}{rgb}{0.274952,0.037752,0.364543}%
\pgfsetfillcolor{currentfill}%
\pgfsetfillopacity{0.700000}%
\pgfsetlinewidth{0.000000pt}%
\definecolor{currentstroke}{rgb}{0.000000,0.000000,0.000000}%
\pgfsetstrokecolor{currentstroke}%
\pgfsetdash{}{0pt}%
\pgfpathmoveto{\pgfqpoint{3.279868in}{1.950474in}}%
\pgfpathlineto{\pgfqpoint{3.292968in}{1.944203in}}%
\pgfpathlineto{\pgfqpoint{3.306069in}{1.938127in}}%
\pgfpathlineto{\pgfqpoint{3.319173in}{1.932245in}}%
\pgfpathlineto{\pgfqpoint{3.332278in}{1.926557in}}%
\pgfpathlineto{\pgfqpoint{3.340192in}{1.934942in}}%
\pgfpathlineto{\pgfqpoint{3.348098in}{1.943387in}}%
\pgfpathlineto{\pgfqpoint{3.355999in}{1.951888in}}%
\pgfpathlineto{\pgfqpoint{3.363893in}{1.960445in}}%
\pgfpathlineto{\pgfqpoint{3.350803in}{1.965888in}}%
\pgfpathlineto{\pgfqpoint{3.337716in}{1.971525in}}%
\pgfpathlineto{\pgfqpoint{3.324631in}{1.977356in}}%
\pgfpathlineto{\pgfqpoint{3.311548in}{1.983383in}}%
\pgfpathlineto{\pgfqpoint{3.303638in}{1.975061in}}%
\pgfpathlineto{\pgfqpoint{3.295722in}{1.966800in}}%
\pgfpathlineto{\pgfqpoint{3.287798in}{1.958604in}}%
\pgfpathlineto{\pgfqpoint{3.279868in}{1.950474in}}%
\pgfpathclose%
\pgfusepath{fill}%
\end{pgfscope}%
\begin{pgfscope}%
\pgfpathrectangle{\pgfqpoint{1.254980in}{0.150000in}}{\pgfqpoint{5.490039in}{5.490039in}}%
\pgfusepath{clip}%
\pgfsetbuttcap%
\pgfsetroundjoin%
\definecolor{currentfill}{rgb}{0.121148,0.592739,0.544641}%
\pgfsetfillcolor{currentfill}%
\pgfsetfillopacity{0.700000}%
\pgfsetlinewidth{0.000000pt}%
\definecolor{currentstroke}{rgb}{0.000000,0.000000,0.000000}%
\pgfsetstrokecolor{currentstroke}%
\pgfsetdash{}{0pt}%
\pgfpathmoveto{\pgfqpoint{5.618262in}{3.185354in}}%
\pgfpathlineto{\pgfqpoint{5.632214in}{3.193852in}}%
\pgfpathlineto{\pgfqpoint{5.646184in}{3.202502in}}%
\pgfpathlineto{\pgfqpoint{5.660171in}{3.211304in}}%
\pgfpathlineto{\pgfqpoint{5.674176in}{3.220259in}}%
\pgfpathlineto{\pgfqpoint{5.681110in}{3.223602in}}%
\pgfpathlineto{\pgfqpoint{5.688041in}{3.227025in}}%
\pgfpathlineto{\pgfqpoint{5.694968in}{3.230534in}}%
\pgfpathlineto{\pgfqpoint{5.701892in}{3.234135in}}%
\pgfpathlineto{\pgfqpoint{5.687917in}{3.225742in}}%
\pgfpathlineto{\pgfqpoint{5.673959in}{3.217501in}}%
\pgfpathlineto{\pgfqpoint{5.660017in}{3.209411in}}%
\pgfpathlineto{\pgfqpoint{5.646093in}{3.201473in}}%
\pgfpathlineto{\pgfqpoint{5.639140in}{3.197301in}}%
\pgfpathlineto{\pgfqpoint{5.632184in}{3.193228in}}%
\pgfpathlineto{\pgfqpoint{5.625224in}{3.189248in}}%
\pgfpathlineto{\pgfqpoint{5.618262in}{3.185354in}}%
\pgfpathclose%
\pgfusepath{fill}%
\end{pgfscope}%
\begin{pgfscope}%
\pgfpathrectangle{\pgfqpoint{1.254980in}{0.150000in}}{\pgfqpoint{5.490039in}{5.490039in}}%
\pgfusepath{clip}%
\pgfsetbuttcap%
\pgfsetroundjoin%
\definecolor{currentfill}{rgb}{0.119423,0.611141,0.538982}%
\pgfsetfillcolor{currentfill}%
\pgfsetfillopacity{0.700000}%
\pgfsetlinewidth{0.000000pt}%
\definecolor{currentstroke}{rgb}{0.000000,0.000000,0.000000}%
\pgfsetstrokecolor{currentstroke}%
\pgfsetdash{}{0pt}%
\pgfpathmoveto{\pgfqpoint{5.701892in}{3.234135in}}%
\pgfpathlineto{\pgfqpoint{5.715885in}{3.242680in}}%
\pgfpathlineto{\pgfqpoint{5.729896in}{3.251377in}}%
\pgfpathlineto{\pgfqpoint{5.743924in}{3.260225in}}%
\pgfpathlineto{\pgfqpoint{5.757970in}{3.269226in}}%
\pgfpathlineto{\pgfqpoint{5.764860in}{3.272346in}}%
\pgfpathlineto{\pgfqpoint{5.771748in}{3.275564in}}%
\pgfpathlineto{\pgfqpoint{5.778634in}{3.278887in}}%
\pgfpathlineto{\pgfqpoint{5.785516in}{3.282322in}}%
\pgfpathlineto{\pgfqpoint{5.771502in}{3.273912in}}%
\pgfpathlineto{\pgfqpoint{5.757505in}{3.265654in}}%
\pgfpathlineto{\pgfqpoint{5.743525in}{3.257547in}}%
\pgfpathlineto{\pgfqpoint{5.729563in}{3.249590in}}%
\pgfpathlineto{\pgfqpoint{5.722649in}{3.245556in}}%
\pgfpathlineto{\pgfqpoint{5.715732in}{3.241640in}}%
\pgfpathlineto{\pgfqpoint{5.708814in}{3.237835in}}%
\pgfpathlineto{\pgfqpoint{5.701892in}{3.234135in}}%
\pgfpathclose%
\pgfusepath{fill}%
\end{pgfscope}%
\begin{pgfscope}%
\pgfpathrectangle{\pgfqpoint{1.254980in}{0.150000in}}{\pgfqpoint{5.490039in}{5.490039in}}%
\pgfusepath{clip}%
\pgfsetbuttcap%
\pgfsetroundjoin%
\definecolor{currentfill}{rgb}{0.277941,0.056324,0.381191}%
\pgfsetfillcolor{currentfill}%
\pgfsetfillopacity{0.700000}%
\pgfsetlinewidth{0.000000pt}%
\definecolor{currentstroke}{rgb}{0.000000,0.000000,0.000000}%
\pgfsetstrokecolor{currentstroke}%
\pgfsetdash{}{0pt}%
\pgfpathmoveto{\pgfqpoint{3.143167in}{1.978184in}}%
\pgfpathlineto{\pgfqpoint{3.156279in}{1.970037in}}%
\pgfpathlineto{\pgfqpoint{3.169392in}{1.962094in}}%
\pgfpathlineto{\pgfqpoint{3.182504in}{1.954354in}}%
\pgfpathlineto{\pgfqpoint{3.195617in}{1.946816in}}%
\pgfpathlineto{\pgfqpoint{3.203595in}{1.954374in}}%
\pgfpathlineto{\pgfqpoint{3.211565in}{1.962015in}}%
\pgfpathlineto{\pgfqpoint{3.219527in}{1.969736in}}%
\pgfpathlineto{\pgfqpoint{3.227483in}{1.977535in}}%
\pgfpathlineto{\pgfqpoint{3.214389in}{1.984799in}}%
\pgfpathlineto{\pgfqpoint{3.201295in}{1.992265in}}%
\pgfpathlineto{\pgfqpoint{3.188202in}{1.999935in}}%
\pgfpathlineto{\pgfqpoint{3.175110in}{2.007808in}}%
\pgfpathlineto{\pgfqpoint{3.167135in}{2.000273in}}%
\pgfpathlineto{\pgfqpoint{3.159154in}{1.992822in}}%
\pgfpathlineto{\pgfqpoint{3.151164in}{1.985458in}}%
\pgfpathlineto{\pgfqpoint{3.143167in}{1.978184in}}%
\pgfpathclose%
\pgfusepath{fill}%
\end{pgfscope}%
\begin{pgfscope}%
\pgfpathrectangle{\pgfqpoint{1.254980in}{0.150000in}}{\pgfqpoint{5.490039in}{5.490039in}}%
\pgfusepath{clip}%
\pgfsetbuttcap%
\pgfsetroundjoin%
\definecolor{currentfill}{rgb}{0.283187,0.125848,0.444960}%
\pgfsetfillcolor{currentfill}%
\pgfsetfillopacity{0.700000}%
\pgfsetlinewidth{0.000000pt}%
\definecolor{currentstroke}{rgb}{0.000000,0.000000,0.000000}%
\pgfsetstrokecolor{currentstroke}%
\pgfsetdash{}{0pt}%
\pgfpathmoveto{\pgfqpoint{3.887029in}{2.085709in}}%
\pgfpathlineto{\pgfqpoint{3.900211in}{2.086329in}}%
\pgfpathlineto{\pgfqpoint{3.913401in}{2.087121in}}%
\pgfpathlineto{\pgfqpoint{3.926598in}{2.088084in}}%
\pgfpathlineto{\pgfqpoint{3.939804in}{2.089219in}}%
\pgfpathlineto{\pgfqpoint{3.947497in}{2.099278in}}%
\pgfpathlineto{\pgfqpoint{3.955185in}{2.109308in}}%
\pgfpathlineto{\pgfqpoint{3.962869in}{2.119310in}}%
\pgfpathlineto{\pgfqpoint{3.970547in}{2.129283in}}%
\pgfpathlineto{\pgfqpoint{3.957349in}{2.128072in}}%
\pgfpathlineto{\pgfqpoint{3.944159in}{2.127031in}}%
\pgfpathlineto{\pgfqpoint{3.930977in}{2.126162in}}%
\pgfpathlineto{\pgfqpoint{3.917803in}{2.125465in}}%
\pgfpathlineto{\pgfqpoint{3.910117in}{2.115558in}}%
\pgfpathlineto{\pgfqpoint{3.902426in}{2.105630in}}%
\pgfpathlineto{\pgfqpoint{3.894730in}{2.095680in}}%
\pgfpathlineto{\pgfqpoint{3.887029in}{2.085709in}}%
\pgfpathclose%
\pgfusepath{fill}%
\end{pgfscope}%
\begin{pgfscope}%
\pgfpathrectangle{\pgfqpoint{1.254980in}{0.150000in}}{\pgfqpoint{5.490039in}{5.490039in}}%
\pgfusepath{clip}%
\pgfsetbuttcap%
\pgfsetroundjoin%
\definecolor{currentfill}{rgb}{0.281887,0.150881,0.465405}%
\pgfsetfillcolor{currentfill}%
\pgfsetfillopacity{0.700000}%
\pgfsetlinewidth{0.000000pt}%
\definecolor{currentstroke}{rgb}{0.000000,0.000000,0.000000}%
\pgfsetstrokecolor{currentstroke}%
\pgfsetdash{}{0pt}%
\pgfpathmoveto{\pgfqpoint{3.970547in}{2.129283in}}%
\pgfpathlineto{\pgfqpoint{3.983754in}{2.130666in}}%
\pgfpathlineto{\pgfqpoint{3.996969in}{2.132218in}}%
\pgfpathlineto{\pgfqpoint{4.010192in}{2.133940in}}%
\pgfpathlineto{\pgfqpoint{4.023425in}{2.135832in}}%
\pgfpathlineto{\pgfqpoint{4.031091in}{2.145835in}}%
\pgfpathlineto{\pgfqpoint{4.038752in}{2.155802in}}%
\pgfpathlineto{\pgfqpoint{4.046409in}{2.165733in}}%
\pgfpathlineto{\pgfqpoint{4.054060in}{2.175627in}}%
\pgfpathlineto{\pgfqpoint{4.040835in}{2.173687in}}%
\pgfpathlineto{\pgfqpoint{4.027618in}{2.171915in}}%
\pgfpathlineto{\pgfqpoint{4.014410in}{2.170314in}}%
\pgfpathlineto{\pgfqpoint{4.001211in}{2.168882in}}%
\pgfpathlineto{\pgfqpoint{3.993552in}{2.159027in}}%
\pgfpathlineto{\pgfqpoint{3.985889in}{2.149142in}}%
\pgfpathlineto{\pgfqpoint{3.978221in}{2.139227in}}%
\pgfpathlineto{\pgfqpoint{3.970547in}{2.129283in}}%
\pgfpathclose%
\pgfusepath{fill}%
\end{pgfscope}%
\begin{pgfscope}%
\pgfpathrectangle{\pgfqpoint{1.254980in}{0.150000in}}{\pgfqpoint{5.490039in}{5.490039in}}%
\pgfusepath{clip}%
\pgfsetbuttcap%
\pgfsetroundjoin%
\definecolor{currentfill}{rgb}{0.282910,0.105393,0.426902}%
\pgfsetfillcolor{currentfill}%
\pgfsetfillopacity{0.700000}%
\pgfsetlinewidth{0.000000pt}%
\definecolor{currentstroke}{rgb}{0.000000,0.000000,0.000000}%
\pgfsetstrokecolor{currentstroke}%
\pgfsetdash{}{0pt}%
\pgfpathmoveto{\pgfqpoint{3.803490in}{2.045292in}}%
\pgfpathlineto{\pgfqpoint{3.816651in}{2.045114in}}%
\pgfpathlineto{\pgfqpoint{3.829819in}{2.045110in}}%
\pgfpathlineto{\pgfqpoint{3.842994in}{2.045280in}}%
\pgfpathlineto{\pgfqpoint{3.856176in}{2.045623in}}%
\pgfpathlineto{\pgfqpoint{3.863897in}{2.055673in}}%
\pgfpathlineto{\pgfqpoint{3.871613in}{2.065705in}}%
\pgfpathlineto{\pgfqpoint{3.879323in}{2.075717in}}%
\pgfpathlineto{\pgfqpoint{3.887029in}{2.085709in}}%
\pgfpathlineto{\pgfqpoint{3.873855in}{2.085261in}}%
\pgfpathlineto{\pgfqpoint{3.860688in}{2.084986in}}%
\pgfpathlineto{\pgfqpoint{3.847529in}{2.084885in}}%
\pgfpathlineto{\pgfqpoint{3.834376in}{2.084958in}}%
\pgfpathlineto{\pgfqpoint{3.826662in}{2.075061in}}%
\pgfpathlineto{\pgfqpoint{3.818943in}{2.065150in}}%
\pgfpathlineto{\pgfqpoint{3.811219in}{2.055227in}}%
\pgfpathlineto{\pgfqpoint{3.803490in}{2.045292in}}%
\pgfpathclose%
\pgfusepath{fill}%
\end{pgfscope}%
\begin{pgfscope}%
\pgfpathrectangle{\pgfqpoint{1.254980in}{0.150000in}}{\pgfqpoint{5.490039in}{5.490039in}}%
\pgfusepath{clip}%
\pgfsetbuttcap%
\pgfsetroundjoin%
\definecolor{currentfill}{rgb}{0.121380,0.629492,0.531973}%
\pgfsetfillcolor{currentfill}%
\pgfsetfillopacity{0.700000}%
\pgfsetlinewidth{0.000000pt}%
\definecolor{currentstroke}{rgb}{0.000000,0.000000,0.000000}%
\pgfsetstrokecolor{currentstroke}%
\pgfsetdash{}{0pt}%
\pgfpathmoveto{\pgfqpoint{5.785516in}{3.282322in}}%
\pgfpathlineto{\pgfqpoint{5.799549in}{3.290882in}}%
\pgfpathlineto{\pgfqpoint{5.813598in}{3.299594in}}%
\pgfpathlineto{\pgfqpoint{5.827666in}{3.308458in}}%
\pgfpathlineto{\pgfqpoint{5.841752in}{3.317473in}}%
\pgfpathlineto{\pgfqpoint{5.848600in}{3.320415in}}%
\pgfpathlineto{\pgfqpoint{5.855446in}{3.323476in}}%
\pgfpathlineto{\pgfqpoint{5.862290in}{3.326662in}}%
\pgfpathlineto{\pgfqpoint{5.869133in}{3.329979in}}%
\pgfpathlineto{\pgfqpoint{5.855081in}{3.321585in}}%
\pgfpathlineto{\pgfqpoint{5.841046in}{3.313341in}}%
\pgfpathlineto{\pgfqpoint{5.827029in}{3.305248in}}%
\pgfpathlineto{\pgfqpoint{5.813029in}{3.297305in}}%
\pgfpathlineto{\pgfqpoint{5.806153in}{3.293359in}}%
\pgfpathlineto{\pgfqpoint{5.799276in}{3.289551in}}%
\pgfpathlineto{\pgfqpoint{5.792397in}{3.285874in}}%
\pgfpathlineto{\pgfqpoint{5.785516in}{3.282322in}}%
\pgfpathclose%
\pgfusepath{fill}%
\end{pgfscope}%
\begin{pgfscope}%
\pgfpathrectangle{\pgfqpoint{1.254980in}{0.150000in}}{\pgfqpoint{5.490039in}{5.490039in}}%
\pgfusepath{clip}%
\pgfsetbuttcap%
\pgfsetroundjoin%
\definecolor{currentfill}{rgb}{0.201239,0.383670,0.554294}%
\pgfsetfillcolor{currentfill}%
\pgfsetfillopacity{0.700000}%
\pgfsetlinewidth{0.000000pt}%
\definecolor{currentstroke}{rgb}{0.000000,0.000000,0.000000}%
\pgfsetstrokecolor{currentstroke}%
\pgfsetdash{}{0pt}%
\pgfpathmoveto{\pgfqpoint{4.752710in}{2.635891in}}%
\pgfpathlineto{\pgfqpoint{4.766242in}{2.642393in}}%
\pgfpathlineto{\pgfqpoint{4.779786in}{2.649055in}}%
\pgfpathlineto{\pgfqpoint{4.793345in}{2.655877in}}%
\pgfpathlineto{\pgfqpoint{4.806917in}{2.662857in}}%
\pgfpathlineto{\pgfqpoint{4.814292in}{2.670023in}}%
\pgfpathlineto{\pgfqpoint{4.821660in}{2.677141in}}%
\pgfpathlineto{\pgfqpoint{4.829023in}{2.684215in}}%
\pgfpathlineto{\pgfqpoint{4.836379in}{2.691247in}}%
\pgfpathlineto{\pgfqpoint{4.822818in}{2.684506in}}%
\pgfpathlineto{\pgfqpoint{4.809270in}{2.677922in}}%
\pgfpathlineto{\pgfqpoint{4.795737in}{2.671498in}}%
\pgfpathlineto{\pgfqpoint{4.782217in}{2.665233in}}%
\pgfpathlineto{\pgfqpoint{4.774849in}{2.657952in}}%
\pgfpathlineto{\pgfqpoint{4.767475in}{2.650636in}}%
\pgfpathlineto{\pgfqpoint{4.760096in}{2.643284in}}%
\pgfpathlineto{\pgfqpoint{4.752710in}{2.635891in}}%
\pgfpathclose%
\pgfusepath{fill}%
\end{pgfscope}%
\begin{pgfscope}%
\pgfpathrectangle{\pgfqpoint{1.254980in}{0.150000in}}{\pgfqpoint{5.490039in}{5.490039in}}%
\pgfusepath{clip}%
\pgfsetbuttcap%
\pgfsetroundjoin%
\definecolor{currentfill}{rgb}{0.278826,0.175490,0.483397}%
\pgfsetfillcolor{currentfill}%
\pgfsetfillopacity{0.700000}%
\pgfsetlinewidth{0.000000pt}%
\definecolor{currentstroke}{rgb}{0.000000,0.000000,0.000000}%
\pgfsetstrokecolor{currentstroke}%
\pgfsetdash{}{0pt}%
\pgfpathmoveto{\pgfqpoint{4.054060in}{2.175627in}}%
\pgfpathlineto{\pgfqpoint{4.067295in}{2.177736in}}%
\pgfpathlineto{\pgfqpoint{4.080538in}{2.180014in}}%
\pgfpathlineto{\pgfqpoint{4.093791in}{2.182460in}}%
\pgfpathlineto{\pgfqpoint{4.107053in}{2.185074in}}%
\pgfpathlineto{\pgfqpoint{4.114693in}{2.194963in}}%
\pgfpathlineto{\pgfqpoint{4.122327in}{2.204808in}}%
\pgfpathlineto{\pgfqpoint{4.129957in}{2.214611in}}%
\pgfpathlineto{\pgfqpoint{4.137581in}{2.224370in}}%
\pgfpathlineto{\pgfqpoint{4.124326in}{2.221735in}}%
\pgfpathlineto{\pgfqpoint{4.111080in}{2.219269in}}%
\pgfpathlineto{\pgfqpoint{4.097843in}{2.216970in}}%
\pgfpathlineto{\pgfqpoint{4.084616in}{2.214840in}}%
\pgfpathlineto{\pgfqpoint{4.076984in}{2.205091in}}%
\pgfpathlineto{\pgfqpoint{4.069348in}{2.195306in}}%
\pgfpathlineto{\pgfqpoint{4.061707in}{2.185485in}}%
\pgfpathlineto{\pgfqpoint{4.054060in}{2.175627in}}%
\pgfpathclose%
\pgfusepath{fill}%
\end{pgfscope}%
\begin{pgfscope}%
\pgfpathrectangle{\pgfqpoint{1.254980in}{0.150000in}}{\pgfqpoint{5.490039in}{5.490039in}}%
\pgfusepath{clip}%
\pgfsetbuttcap%
\pgfsetroundjoin%
\definecolor{currentfill}{rgb}{0.282656,0.100196,0.422160}%
\pgfsetfillcolor{currentfill}%
\pgfsetfillopacity{0.700000}%
\pgfsetlinewidth{0.000000pt}%
\definecolor{currentstroke}{rgb}{0.000000,0.000000,0.000000}%
\pgfsetstrokecolor{currentstroke}%
\pgfsetdash{}{0pt}%
\pgfpathmoveto{\pgfqpoint{2.953405in}{2.067861in}}%
\pgfpathlineto{\pgfqpoint{2.966556in}{2.056854in}}%
\pgfpathlineto{\pgfqpoint{2.979705in}{2.046068in}}%
\pgfpathlineto{\pgfqpoint{2.992852in}{2.035501in}}%
\pgfpathlineto{\pgfqpoint{3.005997in}{2.025153in}}%
\pgfpathlineto{\pgfqpoint{3.014072in}{2.031414in}}%
\pgfpathlineto{\pgfqpoint{3.022138in}{2.037789in}}%
\pgfpathlineto{\pgfqpoint{3.030196in}{2.044277in}}%
\pgfpathlineto{\pgfqpoint{3.038245in}{2.050873in}}%
\pgfpathlineto{\pgfqpoint{3.025124in}{2.060917in}}%
\pgfpathlineto{\pgfqpoint{3.012001in}{2.071179in}}%
\pgfpathlineto{\pgfqpoint{2.998877in}{2.081660in}}%
\pgfpathlineto{\pgfqpoint{2.985749in}{2.092361in}}%
\pgfpathlineto{\pgfqpoint{2.977677in}{2.086059in}}%
\pgfpathlineto{\pgfqpoint{2.969595in}{2.079873in}}%
\pgfpathlineto{\pgfqpoint{2.961505in}{2.073806in}}%
\pgfpathlineto{\pgfqpoint{2.953405in}{2.067861in}}%
\pgfpathclose%
\pgfusepath{fill}%
\end{pgfscope}%
\begin{pgfscope}%
\pgfpathrectangle{\pgfqpoint{1.254980in}{0.150000in}}{\pgfqpoint{5.490039in}{5.490039in}}%
\pgfusepath{clip}%
\pgfsetbuttcap%
\pgfsetroundjoin%
\definecolor{currentfill}{rgb}{0.274952,0.037752,0.364543}%
\pgfsetfillcolor{currentfill}%
\pgfsetfillopacity{0.700000}%
\pgfsetlinewidth{0.000000pt}%
\definecolor{currentstroke}{rgb}{0.000000,0.000000,0.000000}%
\pgfsetstrokecolor{currentstroke}%
\pgfsetdash{}{0pt}%
\pgfpathmoveto{\pgfqpoint{3.416276in}{1.940585in}}%
\pgfpathlineto{\pgfqpoint{3.429379in}{1.936095in}}%
\pgfpathlineto{\pgfqpoint{3.442485in}{1.931792in}}%
\pgfpathlineto{\pgfqpoint{3.455595in}{1.927676in}}%
\pgfpathlineto{\pgfqpoint{3.468708in}{1.923746in}}%
\pgfpathlineto{\pgfqpoint{3.476566in}{1.932809in}}%
\pgfpathlineto{\pgfqpoint{3.484419in}{1.941910in}}%
\pgfpathlineto{\pgfqpoint{3.492266in}{1.951045in}}%
\pgfpathlineto{\pgfqpoint{3.500107in}{1.960214in}}%
\pgfpathlineto{\pgfqpoint{3.487007in}{1.963927in}}%
\pgfpathlineto{\pgfqpoint{3.473911in}{1.967827in}}%
\pgfpathlineto{\pgfqpoint{3.460818in}{1.971913in}}%
\pgfpathlineto{\pgfqpoint{3.447729in}{1.976188in}}%
\pgfpathlineto{\pgfqpoint{3.439875in}{1.967225in}}%
\pgfpathlineto{\pgfqpoint{3.432015in}{1.958302in}}%
\pgfpathlineto{\pgfqpoint{3.424148in}{1.949422in}}%
\pgfpathlineto{\pgfqpoint{3.416276in}{1.940585in}}%
\pgfpathclose%
\pgfusepath{fill}%
\end{pgfscope}%
\begin{pgfscope}%
\pgfpathrectangle{\pgfqpoint{1.254980in}{0.150000in}}{\pgfqpoint{5.490039in}{5.490039in}}%
\pgfusepath{clip}%
\pgfsetbuttcap%
\pgfsetroundjoin%
\definecolor{currentfill}{rgb}{0.128087,0.647749,0.523491}%
\pgfsetfillcolor{currentfill}%
\pgfsetfillopacity{0.700000}%
\pgfsetlinewidth{0.000000pt}%
\definecolor{currentstroke}{rgb}{0.000000,0.000000,0.000000}%
\pgfsetstrokecolor{currentstroke}%
\pgfsetdash{}{0pt}%
\pgfpathmoveto{\pgfqpoint{5.869133in}{3.329979in}}%
\pgfpathlineto{\pgfqpoint{5.883203in}{3.338524in}}%
\pgfpathlineto{\pgfqpoint{5.897291in}{3.347220in}}%
\pgfpathlineto{\pgfqpoint{5.911398in}{3.356067in}}%
\pgfpathlineto{\pgfqpoint{5.925522in}{3.365064in}}%
\pgfpathlineto{\pgfqpoint{5.932329in}{3.367881in}}%
\pgfpathlineto{\pgfqpoint{5.939134in}{3.370837in}}%
\pgfpathlineto{\pgfqpoint{5.945939in}{3.373939in}}%
\pgfpathlineto{\pgfqpoint{5.952743in}{3.377194in}}%
\pgfpathlineto{\pgfqpoint{5.938654in}{3.368846in}}%
\pgfpathlineto{\pgfqpoint{5.924583in}{3.360648in}}%
\pgfpathlineto{\pgfqpoint{5.910530in}{3.352600in}}%
\pgfpathlineto{\pgfqpoint{5.896494in}{3.344701in}}%
\pgfpathlineto{\pgfqpoint{5.889655in}{3.340789in}}%
\pgfpathlineto{\pgfqpoint{5.882815in}{3.337036in}}%
\pgfpathlineto{\pgfqpoint{5.875974in}{3.333435in}}%
\pgfpathlineto{\pgfqpoint{5.869133in}{3.329979in}}%
\pgfpathclose%
\pgfusepath{fill}%
\end{pgfscope}%
\begin{pgfscope}%
\pgfpathrectangle{\pgfqpoint{1.254980in}{0.150000in}}{\pgfqpoint{5.490039in}{5.490039in}}%
\pgfusepath{clip}%
\pgfsetbuttcap%
\pgfsetroundjoin%
\definecolor{currentfill}{rgb}{0.140210,0.665859,0.513427}%
\pgfsetfillcolor{currentfill}%
\pgfsetfillopacity{0.700000}%
\pgfsetlinewidth{0.000000pt}%
\definecolor{currentstroke}{rgb}{0.000000,0.000000,0.000000}%
\pgfsetstrokecolor{currentstroke}%
\pgfsetdash{}{0pt}%
\pgfpathmoveto{\pgfqpoint{5.952743in}{3.377194in}}%
\pgfpathlineto{\pgfqpoint{5.966849in}{3.385692in}}%
\pgfpathlineto{\pgfqpoint{5.980974in}{3.394340in}}%
\pgfpathlineto{\pgfqpoint{5.995118in}{3.403139in}}%
\pgfpathlineto{\pgfqpoint{6.009280in}{3.412088in}}%
\pgfpathlineto{\pgfqpoint{6.016046in}{3.414835in}}%
\pgfpathlineto{\pgfqpoint{6.022813in}{3.417744in}}%
\pgfpathlineto{\pgfqpoint{6.029580in}{3.420821in}}%
\pgfpathlineto{\pgfqpoint{6.036347in}{3.424073in}}%
\pgfpathlineto{\pgfqpoint{6.022223in}{3.415803in}}%
\pgfpathlineto{\pgfqpoint{6.008118in}{3.407682in}}%
\pgfpathlineto{\pgfqpoint{5.994030in}{3.399710in}}%
\pgfpathlineto{\pgfqpoint{5.979960in}{3.391887in}}%
\pgfpathlineto{\pgfqpoint{5.973155in}{3.387949in}}%
\pgfpathlineto{\pgfqpoint{5.966350in}{3.384192in}}%
\pgfpathlineto{\pgfqpoint{5.959546in}{3.380609in}}%
\pgfpathlineto{\pgfqpoint{5.952743in}{3.377194in}}%
\pgfpathclose%
\pgfusepath{fill}%
\end{pgfscope}%
\begin{pgfscope}%
\pgfpathrectangle{\pgfqpoint{1.254980in}{0.150000in}}{\pgfqpoint{5.490039in}{5.490039in}}%
\pgfusepath{clip}%
\pgfsetbuttcap%
\pgfsetroundjoin%
\definecolor{currentfill}{rgb}{0.199430,0.387607,0.554642}%
\pgfsetfillcolor{currentfill}%
\pgfsetfillopacity{0.700000}%
\pgfsetlinewidth{0.000000pt}%
\definecolor{currentstroke}{rgb}{0.000000,0.000000,0.000000}%
\pgfsetstrokecolor{currentstroke}%
\pgfsetdash{}{0pt}%
\pgfpathmoveto{\pgfqpoint{2.422586in}{2.710267in}}%
\pgfpathlineto{\pgfqpoint{2.436037in}{2.688823in}}%
\pgfpathlineto{\pgfqpoint{2.449474in}{2.667689in}}%
\pgfpathlineto{\pgfqpoint{2.462899in}{2.646861in}}%
\pgfpathlineto{\pgfqpoint{2.476312in}{2.626336in}}%
\pgfpathlineto{\pgfqpoint{2.484691in}{2.629360in}}%
\pgfpathlineto{\pgfqpoint{2.493057in}{2.632571in}}%
\pgfpathlineto{\pgfqpoint{2.501409in}{2.635967in}}%
\pgfpathlineto{\pgfqpoint{2.509748in}{2.639546in}}%
\pgfpathlineto{\pgfqpoint{2.496372in}{2.659746in}}%
\pgfpathlineto{\pgfqpoint{2.482984in}{2.680248in}}%
\pgfpathlineto{\pgfqpoint{2.469584in}{2.701056in}}%
\pgfpathlineto{\pgfqpoint{2.456172in}{2.722173in}}%
\pgfpathlineto{\pgfqpoint{2.447796in}{2.718910in}}%
\pgfpathlineto{\pgfqpoint{2.439407in}{2.715835in}}%
\pgfpathlineto{\pgfqpoint{2.431004in}{2.712954in}}%
\pgfpathlineto{\pgfqpoint{2.422586in}{2.710267in}}%
\pgfpathclose%
\pgfusepath{fill}%
\end{pgfscope}%
\begin{pgfscope}%
\pgfpathrectangle{\pgfqpoint{1.254980in}{0.150000in}}{\pgfqpoint{5.490039in}{5.490039in}}%
\pgfusepath{clip}%
\pgfsetbuttcap%
\pgfsetroundjoin%
\definecolor{currentfill}{rgb}{0.274128,0.199721,0.498911}%
\pgfsetfillcolor{currentfill}%
\pgfsetfillopacity{0.700000}%
\pgfsetlinewidth{0.000000pt}%
\definecolor{currentstroke}{rgb}{0.000000,0.000000,0.000000}%
\pgfsetstrokecolor{currentstroke}%
\pgfsetdash{}{0pt}%
\pgfpathmoveto{\pgfqpoint{4.137581in}{2.224370in}}%
\pgfpathlineto{\pgfqpoint{4.150847in}{2.227172in}}%
\pgfpathlineto{\pgfqpoint{4.164122in}{2.230140in}}%
\pgfpathlineto{\pgfqpoint{4.177407in}{2.233276in}}%
\pgfpathlineto{\pgfqpoint{4.190701in}{2.236578in}}%
\pgfpathlineto{\pgfqpoint{4.198314in}{2.246297in}}%
\pgfpathlineto{\pgfqpoint{4.205922in}{2.255968in}}%
\pgfpathlineto{\pgfqpoint{4.213524in}{2.265590in}}%
\pgfpathlineto{\pgfqpoint{4.221122in}{2.275163in}}%
\pgfpathlineto{\pgfqpoint{4.207833in}{2.271869in}}%
\pgfpathlineto{\pgfqpoint{4.194555in}{2.268741in}}%
\pgfpathlineto{\pgfqpoint{4.181287in}{2.265779in}}%
\pgfpathlineto{\pgfqpoint{4.168028in}{2.262985in}}%
\pgfpathlineto{\pgfqpoint{4.160424in}{2.253394in}}%
\pgfpathlineto{\pgfqpoint{4.152815in}{2.243761in}}%
\pgfpathlineto{\pgfqpoint{4.145201in}{2.234087in}}%
\pgfpathlineto{\pgfqpoint{4.137581in}{2.224370in}}%
\pgfpathclose%
\pgfusepath{fill}%
\end{pgfscope}%
\begin{pgfscope}%
\pgfpathrectangle{\pgfqpoint{1.254980in}{0.150000in}}{\pgfqpoint{5.490039in}{5.490039in}}%
\pgfusepath{clip}%
\pgfsetbuttcap%
\pgfsetroundjoin%
\definecolor{currentfill}{rgb}{0.281446,0.084320,0.407414}%
\pgfsetfillcolor{currentfill}%
\pgfsetfillopacity{0.700000}%
\pgfsetlinewidth{0.000000pt}%
\definecolor{currentstroke}{rgb}{0.000000,0.000000,0.000000}%
\pgfsetstrokecolor{currentstroke}%
\pgfsetdash{}{0pt}%
\pgfpathmoveto{\pgfqpoint{3.719912in}{2.008444in}}%
\pgfpathlineto{\pgfqpoint{3.733055in}{2.007432in}}%
\pgfpathlineto{\pgfqpoint{3.746205in}{2.006596in}}%
\pgfpathlineto{\pgfqpoint{3.759361in}{2.005936in}}%
\pgfpathlineto{\pgfqpoint{3.772524in}{2.005451in}}%
\pgfpathlineto{\pgfqpoint{3.780273in}{2.015424in}}%
\pgfpathlineto{\pgfqpoint{3.788017in}{2.025390in}}%
\pgfpathlineto{\pgfqpoint{3.795756in}{2.035346in}}%
\pgfpathlineto{\pgfqpoint{3.803490in}{2.045292in}}%
\pgfpathlineto{\pgfqpoint{3.790336in}{2.045645in}}%
\pgfpathlineto{\pgfqpoint{3.777189in}{2.046172in}}%
\pgfpathlineto{\pgfqpoint{3.764049in}{2.046875in}}%
\pgfpathlineto{\pgfqpoint{3.750915in}{2.047755in}}%
\pgfpathlineto{\pgfqpoint{3.743172in}{2.037931in}}%
\pgfpathlineto{\pgfqpoint{3.735424in}{2.028104in}}%
\pgfpathlineto{\pgfqpoint{3.727670in}{2.018275in}}%
\pgfpathlineto{\pgfqpoint{3.719912in}{2.008444in}}%
\pgfpathclose%
\pgfusepath{fill}%
\end{pgfscope}%
\begin{pgfscope}%
\pgfpathrectangle{\pgfqpoint{1.254980in}{0.150000in}}{\pgfqpoint{5.490039in}{5.490039in}}%
\pgfusepath{clip}%
\pgfsetbuttcap%
\pgfsetroundjoin%
\definecolor{currentfill}{rgb}{0.190631,0.407061,0.556089}%
\pgfsetfillcolor{currentfill}%
\pgfsetfillopacity{0.700000}%
\pgfsetlinewidth{0.000000pt}%
\definecolor{currentstroke}{rgb}{0.000000,0.000000,0.000000}%
\pgfsetstrokecolor{currentstroke}%
\pgfsetdash{}{0pt}%
\pgfpathmoveto{\pgfqpoint{4.836379in}{2.691247in}}%
\pgfpathlineto{\pgfqpoint{4.849954in}{2.698148in}}%
\pgfpathlineto{\pgfqpoint{4.863543in}{2.705207in}}%
\pgfpathlineto{\pgfqpoint{4.877147in}{2.712425in}}%
\pgfpathlineto{\pgfqpoint{4.890765in}{2.719802in}}%
\pgfpathlineto{\pgfqpoint{4.898103in}{2.726539in}}%
\pgfpathlineto{\pgfqpoint{4.905435in}{2.733234in}}%
\pgfpathlineto{\pgfqpoint{4.912761in}{2.739889in}}%
\pgfpathlineto{\pgfqpoint{4.920080in}{2.746509in}}%
\pgfpathlineto{\pgfqpoint{4.906474in}{2.739401in}}%
\pgfpathlineto{\pgfqpoint{4.892883in}{2.732451in}}%
\pgfpathlineto{\pgfqpoint{4.879306in}{2.725659in}}%
\pgfpathlineto{\pgfqpoint{4.865743in}{2.719025in}}%
\pgfpathlineto{\pgfqpoint{4.858411in}{2.712127in}}%
\pgfpathlineto{\pgfqpoint{4.851073in}{2.705201in}}%
\pgfpathlineto{\pgfqpoint{4.843729in}{2.698242in}}%
\pgfpathlineto{\pgfqpoint{4.836379in}{2.691247in}}%
\pgfpathclose%
\pgfusepath{fill}%
\end{pgfscope}%
\begin{pgfscope}%
\pgfpathrectangle{\pgfqpoint{1.254980in}{0.150000in}}{\pgfqpoint{5.490039in}{5.490039in}}%
\pgfusepath{clip}%
\pgfsetbuttcap%
\pgfsetroundjoin%
\definecolor{currentfill}{rgb}{0.266580,0.228262,0.514349}%
\pgfsetfillcolor{currentfill}%
\pgfsetfillopacity{0.700000}%
\pgfsetlinewidth{0.000000pt}%
\definecolor{currentstroke}{rgb}{0.000000,0.000000,0.000000}%
\pgfsetstrokecolor{currentstroke}%
\pgfsetdash{}{0pt}%
\pgfpathmoveto{\pgfqpoint{4.221122in}{2.275163in}}%
\pgfpathlineto{\pgfqpoint{4.234420in}{2.278623in}}%
\pgfpathlineto{\pgfqpoint{4.247729in}{2.282249in}}%
\pgfpathlineto{\pgfqpoint{4.261049in}{2.286040in}}%
\pgfpathlineto{\pgfqpoint{4.274379in}{2.289996in}}%
\pgfpathlineto{\pgfqpoint{4.281965in}{2.299497in}}%
\pgfpathlineto{\pgfqpoint{4.289545in}{2.308943in}}%
\pgfpathlineto{\pgfqpoint{4.297120in}{2.318336in}}%
\pgfpathlineto{\pgfqpoint{4.304690in}{2.327677in}}%
\pgfpathlineto{\pgfqpoint{4.291366in}{2.323756in}}%
\pgfpathlineto{\pgfqpoint{4.278053in}{2.320001in}}%
\pgfpathlineto{\pgfqpoint{4.264751in}{2.316411in}}%
\pgfpathlineto{\pgfqpoint{4.251459in}{2.312987in}}%
\pgfpathlineto{\pgfqpoint{4.243883in}{2.303600in}}%
\pgfpathlineto{\pgfqpoint{4.236301in}{2.294167in}}%
\pgfpathlineto{\pgfqpoint{4.228714in}{2.284689in}}%
\pgfpathlineto{\pgfqpoint{4.221122in}{2.275163in}}%
\pgfpathclose%
\pgfusepath{fill}%
\end{pgfscope}%
\begin{pgfscope}%
\pgfpathrectangle{\pgfqpoint{1.254980in}{0.150000in}}{\pgfqpoint{5.490039in}{5.490039in}}%
\pgfusepath{clip}%
\pgfsetbuttcap%
\pgfsetroundjoin%
\definecolor{currentfill}{rgb}{0.279566,0.067836,0.391917}%
\pgfsetfillcolor{currentfill}%
\pgfsetfillopacity{0.700000}%
\pgfsetlinewidth{0.000000pt}%
\definecolor{currentstroke}{rgb}{0.000000,0.000000,0.000000}%
\pgfsetstrokecolor{currentstroke}%
\pgfsetdash{}{0pt}%
\pgfpathmoveto{\pgfqpoint{3.636273in}{1.975598in}}%
\pgfpathlineto{\pgfqpoint{3.649403in}{1.973714in}}%
\pgfpathlineto{\pgfqpoint{3.662538in}{1.972008in}}%
\pgfpathlineto{\pgfqpoint{3.675679in}{1.970481in}}%
\pgfpathlineto{\pgfqpoint{3.688826in}{1.969131in}}%
\pgfpathlineto{\pgfqpoint{3.696606in}{1.978956in}}%
\pgfpathlineto{\pgfqpoint{3.704380in}{1.988784in}}%
\pgfpathlineto{\pgfqpoint{3.712148in}{1.998614in}}%
\pgfpathlineto{\pgfqpoint{3.719912in}{2.008444in}}%
\pgfpathlineto{\pgfqpoint{3.706775in}{2.009634in}}%
\pgfpathlineto{\pgfqpoint{3.693644in}{2.011000in}}%
\pgfpathlineto{\pgfqpoint{3.680518in}{2.012545in}}%
\pgfpathlineto{\pgfqpoint{3.667398in}{2.014269in}}%
\pgfpathlineto{\pgfqpoint{3.659625in}{2.004588in}}%
\pgfpathlineto{\pgfqpoint{3.651846in}{1.994915in}}%
\pgfpathlineto{\pgfqpoint{3.644062in}{1.985251in}}%
\pgfpathlineto{\pgfqpoint{3.636273in}{1.975598in}}%
\pgfpathclose%
\pgfusepath{fill}%
\end{pgfscope}%
\begin{pgfscope}%
\pgfpathrectangle{\pgfqpoint{1.254980in}{0.150000in}}{\pgfqpoint{5.490039in}{5.490039in}}%
\pgfusepath{clip}%
\pgfsetbuttcap%
\pgfsetroundjoin%
\definecolor{currentfill}{rgb}{0.157851,0.683765,0.501686}%
\pgfsetfillcolor{currentfill}%
\pgfsetfillopacity{0.700000}%
\pgfsetlinewidth{0.000000pt}%
\definecolor{currentstroke}{rgb}{0.000000,0.000000,0.000000}%
\pgfsetstrokecolor{currentstroke}%
\pgfsetdash{}{0pt}%
\pgfpathmoveto{\pgfqpoint{6.036347in}{3.424073in}}%
\pgfpathlineto{\pgfqpoint{6.050489in}{3.432493in}}%
\pgfpathlineto{\pgfqpoint{6.064650in}{3.441063in}}%
\pgfpathlineto{\pgfqpoint{6.078829in}{3.449782in}}%
\pgfpathlineto{\pgfqpoint{6.093026in}{3.458651in}}%
\pgfpathlineto{\pgfqpoint{6.099755in}{3.461390in}}%
\pgfpathlineto{\pgfqpoint{6.106485in}{3.464314in}}%
\pgfpathlineto{\pgfqpoint{6.113217in}{3.467430in}}%
\pgfpathlineto{\pgfqpoint{6.099050in}{3.459089in}}%
\pgfpathlineto{\pgfqpoint{6.084901in}{3.450897in}}%
\pgfpathlineto{\pgfqpoint{6.070770in}{3.442854in}}%
\pgfpathlineto{\pgfqpoint{6.056657in}{3.434960in}}%
\pgfpathlineto{\pgfqpoint{6.049886in}{3.431136in}}%
\pgfpathlineto{\pgfqpoint{6.043116in}{3.427509in}}%
\pgfpathlineto{\pgfqpoint{6.036347in}{3.424073in}}%
\pgfpathclose%
\pgfusepath{fill}%
\end{pgfscope}%
\begin{pgfscope}%
\pgfpathrectangle{\pgfqpoint{1.254980in}{0.150000in}}{\pgfqpoint{5.490039in}{5.490039in}}%
\pgfusepath{clip}%
\pgfsetbuttcap%
\pgfsetroundjoin%
\definecolor{currentfill}{rgb}{0.258965,0.251537,0.524736}%
\pgfsetfillcolor{currentfill}%
\pgfsetfillopacity{0.700000}%
\pgfsetlinewidth{0.000000pt}%
\definecolor{currentstroke}{rgb}{0.000000,0.000000,0.000000}%
\pgfsetstrokecolor{currentstroke}%
\pgfsetdash{}{0pt}%
\pgfpathmoveto{\pgfqpoint{4.304690in}{2.327677in}}%
\pgfpathlineto{\pgfqpoint{4.318024in}{2.331761in}}%
\pgfpathlineto{\pgfqpoint{4.331370in}{2.336011in}}%
\pgfpathlineto{\pgfqpoint{4.344726in}{2.340424in}}%
\pgfpathlineto{\pgfqpoint{4.358094in}{2.345001in}}%
\pgfpathlineto{\pgfqpoint{4.365652in}{2.354236in}}%
\pgfpathlineto{\pgfqpoint{4.373204in}{2.363414in}}%
\pgfpathlineto{\pgfqpoint{4.380751in}{2.372535in}}%
\pgfpathlineto{\pgfqpoint{4.388292in}{2.381600in}}%
\pgfpathlineto{\pgfqpoint{4.374931in}{2.377088in}}%
\pgfpathlineto{\pgfqpoint{4.361581in}{2.372739in}}%
\pgfpathlineto{\pgfqpoint{4.348242in}{2.368554in}}%
\pgfpathlineto{\pgfqpoint{4.334915in}{2.364533in}}%
\pgfpathlineto{\pgfqpoint{4.327367in}{2.355393in}}%
\pgfpathlineto{\pgfqpoint{4.319813in}{2.346204in}}%
\pgfpathlineto{\pgfqpoint{4.312254in}{2.336966in}}%
\pgfpathlineto{\pgfqpoint{4.304690in}{2.327677in}}%
\pgfpathclose%
\pgfusepath{fill}%
\end{pgfscope}%
\begin{pgfscope}%
\pgfpathrectangle{\pgfqpoint{1.254980in}{0.150000in}}{\pgfqpoint{5.490039in}{5.490039in}}%
\pgfusepath{clip}%
\pgfsetbuttcap%
\pgfsetroundjoin%
\definecolor{currentfill}{rgb}{0.180629,0.429975,0.557282}%
\pgfsetfillcolor{currentfill}%
\pgfsetfillopacity{0.700000}%
\pgfsetlinewidth{0.000000pt}%
\definecolor{currentstroke}{rgb}{0.000000,0.000000,0.000000}%
\pgfsetstrokecolor{currentstroke}%
\pgfsetdash{}{0pt}%
\pgfpathmoveto{\pgfqpoint{4.920080in}{2.746509in}}%
\pgfpathlineto{\pgfqpoint{4.933700in}{2.753775in}}%
\pgfpathlineto{\pgfqpoint{4.947335in}{2.761200in}}%
\pgfpathlineto{\pgfqpoint{4.960984in}{2.768782in}}%
\pgfpathlineto{\pgfqpoint{4.974648in}{2.776521in}}%
\pgfpathlineto{\pgfqpoint{4.981948in}{2.782822in}}%
\pgfpathlineto{\pgfqpoint{4.989242in}{2.789086in}}%
\pgfpathlineto{\pgfqpoint{4.996529in}{2.795317in}}%
\pgfpathlineto{\pgfqpoint{5.003810in}{2.801520in}}%
\pgfpathlineto{\pgfqpoint{4.990160in}{2.794078in}}%
\pgfpathlineto{\pgfqpoint{4.976524in}{2.786793in}}%
\pgfpathlineto{\pgfqpoint{4.962903in}{2.779666in}}%
\pgfpathlineto{\pgfqpoint{4.949297in}{2.772696in}}%
\pgfpathlineto{\pgfqpoint{4.942001in}{2.766186in}}%
\pgfpathlineto{\pgfqpoint{4.934700in}{2.759654in}}%
\pgfpathlineto{\pgfqpoint{4.927393in}{2.753096in}}%
\pgfpathlineto{\pgfqpoint{4.920080in}{2.746509in}}%
\pgfpathclose%
\pgfusepath{fill}%
\end{pgfscope}%
\begin{pgfscope}%
\pgfpathrectangle{\pgfqpoint{1.254980in}{0.150000in}}{\pgfqpoint{5.490039in}{5.490039in}}%
\pgfusepath{clip}%
\pgfsetbuttcap%
\pgfsetroundjoin%
\definecolor{currentfill}{rgb}{0.281446,0.084320,0.407414}%
\pgfsetfillcolor{currentfill}%
\pgfsetfillopacity{0.700000}%
\pgfsetlinewidth{0.000000pt}%
\definecolor{currentstroke}{rgb}{0.000000,0.000000,0.000000}%
\pgfsetstrokecolor{currentstroke}%
\pgfsetdash{}{0pt}%
\pgfpathmoveto{\pgfqpoint{3.005997in}{2.025153in}}%
\pgfpathlineto{\pgfqpoint{3.019139in}{2.015020in}}%
\pgfpathlineto{\pgfqpoint{3.032280in}{2.005103in}}%
\pgfpathlineto{\pgfqpoint{3.045419in}{1.995399in}}%
\pgfpathlineto{\pgfqpoint{3.058557in}{1.985908in}}%
\pgfpathlineto{\pgfqpoint{3.066609in}{1.992483in}}%
\pgfpathlineto{\pgfqpoint{3.074652in}{1.999166in}}%
\pgfpathlineto{\pgfqpoint{3.082687in}{2.005954in}}%
\pgfpathlineto{\pgfqpoint{3.090713in}{2.012843in}}%
\pgfpathlineto{\pgfqpoint{3.077598in}{2.022031in}}%
\pgfpathlineto{\pgfqpoint{3.064482in}{2.031431in}}%
\pgfpathlineto{\pgfqpoint{3.051364in}{2.041045in}}%
\pgfpathlineto{\pgfqpoint{3.038245in}{2.050873in}}%
\pgfpathlineto{\pgfqpoint{3.030196in}{2.044277in}}%
\pgfpathlineto{\pgfqpoint{3.022138in}{2.037789in}}%
\pgfpathlineto{\pgfqpoint{3.014072in}{2.031414in}}%
\pgfpathlineto{\pgfqpoint{3.005997in}{2.025153in}}%
\pgfpathclose%
\pgfusepath{fill}%
\end{pgfscope}%
\begin{pgfscope}%
\pgfpathrectangle{\pgfqpoint{1.254980in}{0.150000in}}{\pgfqpoint{5.490039in}{5.490039in}}%
\pgfusepath{clip}%
\pgfsetbuttcap%
\pgfsetroundjoin%
\definecolor{currentfill}{rgb}{0.276022,0.044167,0.370164}%
\pgfsetfillcolor{currentfill}%
\pgfsetfillopacity{0.700000}%
\pgfsetlinewidth{0.000000pt}%
\definecolor{currentstroke}{rgb}{0.000000,0.000000,0.000000}%
\pgfsetstrokecolor{currentstroke}%
\pgfsetdash{}{0pt}%
\pgfpathmoveto{\pgfqpoint{3.195617in}{1.946816in}}%
\pgfpathlineto{\pgfqpoint{3.208731in}{1.939479in}}%
\pgfpathlineto{\pgfqpoint{3.221845in}{1.932342in}}%
\pgfpathlineto{\pgfqpoint{3.234961in}{1.925403in}}%
\pgfpathlineto{\pgfqpoint{3.248077in}{1.918662in}}%
\pgfpathlineto{\pgfqpoint{3.256035in}{1.926504in}}%
\pgfpathlineto{\pgfqpoint{3.263987in}{1.934422in}}%
\pgfpathlineto{\pgfqpoint{3.271931in}{1.942413in}}%
\pgfpathlineto{\pgfqpoint{3.279868in}{1.950474in}}%
\pgfpathlineto{\pgfqpoint{3.266770in}{1.956942in}}%
\pgfpathlineto{\pgfqpoint{3.253673in}{1.963607in}}%
\pgfpathlineto{\pgfqpoint{3.240577in}{1.970471in}}%
\pgfpathlineto{\pgfqpoint{3.227483in}{1.977535in}}%
\pgfpathlineto{\pgfqpoint{3.219527in}{1.969736in}}%
\pgfpathlineto{\pgfqpoint{3.211565in}{1.962015in}}%
\pgfpathlineto{\pgfqpoint{3.203595in}{1.954374in}}%
\pgfpathlineto{\pgfqpoint{3.195617in}{1.946816in}}%
\pgfpathclose%
\pgfusepath{fill}%
\end{pgfscope}%
\begin{pgfscope}%
\pgfpathrectangle{\pgfqpoint{1.254980in}{0.150000in}}{\pgfqpoint{5.490039in}{5.490039in}}%
\pgfusepath{clip}%
\pgfsetbuttcap%
\pgfsetroundjoin%
\definecolor{currentfill}{rgb}{0.277941,0.056324,0.381191}%
\pgfsetfillcolor{currentfill}%
\pgfsetfillopacity{0.700000}%
\pgfsetlinewidth{0.000000pt}%
\definecolor{currentstroke}{rgb}{0.000000,0.000000,0.000000}%
\pgfsetstrokecolor{currentstroke}%
\pgfsetdash{}{0pt}%
\pgfpathmoveto{\pgfqpoint{3.552547in}{1.947206in}}%
\pgfpathlineto{\pgfqpoint{3.565668in}{1.944411in}}%
\pgfpathlineto{\pgfqpoint{3.578794in}{1.941799in}}%
\pgfpathlineto{\pgfqpoint{3.591925in}{1.939366in}}%
\pgfpathlineto{\pgfqpoint{3.605061in}{1.937114in}}%
\pgfpathlineto{\pgfqpoint{3.612872in}{1.946713in}}%
\pgfpathlineto{\pgfqpoint{3.620677in}{1.956327in}}%
\pgfpathlineto{\pgfqpoint{3.628478in}{1.965956in}}%
\pgfpathlineto{\pgfqpoint{3.636273in}{1.975598in}}%
\pgfpathlineto{\pgfqpoint{3.623148in}{1.977661in}}%
\pgfpathlineto{\pgfqpoint{3.610028in}{1.979905in}}%
\pgfpathlineto{\pgfqpoint{3.596914in}{1.982330in}}%
\pgfpathlineto{\pgfqpoint{3.583804in}{1.984936in}}%
\pgfpathlineto{\pgfqpoint{3.575999in}{1.975472in}}%
\pgfpathlineto{\pgfqpoint{3.568187in}{1.966028in}}%
\pgfpathlineto{\pgfqpoint{3.560370in}{1.956605in}}%
\pgfpathlineto{\pgfqpoint{3.552547in}{1.947206in}}%
\pgfpathclose%
\pgfusepath{fill}%
\end{pgfscope}%
\begin{pgfscope}%
\pgfpathrectangle{\pgfqpoint{1.254980in}{0.150000in}}{\pgfqpoint{5.490039in}{5.490039in}}%
\pgfusepath{clip}%
\pgfsetbuttcap%
\pgfsetroundjoin%
\definecolor{currentfill}{rgb}{0.248629,0.278775,0.534556}%
\pgfsetfillcolor{currentfill}%
\pgfsetfillopacity{0.700000}%
\pgfsetlinewidth{0.000000pt}%
\definecolor{currentstroke}{rgb}{0.000000,0.000000,0.000000}%
\pgfsetstrokecolor{currentstroke}%
\pgfsetdash{}{0pt}%
\pgfpathmoveto{\pgfqpoint{4.388292in}{2.381600in}}%
\pgfpathlineto{\pgfqpoint{4.401664in}{2.386276in}}%
\pgfpathlineto{\pgfqpoint{4.415048in}{2.391116in}}%
\pgfpathlineto{\pgfqpoint{4.428444in}{2.396118in}}%
\pgfpathlineto{\pgfqpoint{4.441852in}{2.401284in}}%
\pgfpathlineto{\pgfqpoint{4.449381in}{2.410212in}}%
\pgfpathlineto{\pgfqpoint{4.456904in}{2.419081in}}%
\pgfpathlineto{\pgfqpoint{4.464421in}{2.427891in}}%
\pgfpathlineto{\pgfqpoint{4.471933in}{2.436643in}}%
\pgfpathlineto{\pgfqpoint{4.458532in}{2.431571in}}%
\pgfpathlineto{\pgfqpoint{4.445144in}{2.426662in}}%
\pgfpathlineto{\pgfqpoint{4.431767in}{2.421916in}}%
\pgfpathlineto{\pgfqpoint{4.418402in}{2.417333in}}%
\pgfpathlineto{\pgfqpoint{4.410883in}{2.408476in}}%
\pgfpathlineto{\pgfqpoint{4.403358in}{2.399570in}}%
\pgfpathlineto{\pgfqpoint{4.395828in}{2.390612in}}%
\pgfpathlineto{\pgfqpoint{4.388292in}{2.381600in}}%
\pgfpathclose%
\pgfusepath{fill}%
\end{pgfscope}%
\begin{pgfscope}%
\pgfpathrectangle{\pgfqpoint{1.254980in}{0.150000in}}{\pgfqpoint{5.490039in}{5.490039in}}%
\pgfusepath{clip}%
\pgfsetbuttcap%
\pgfsetroundjoin%
\definecolor{currentfill}{rgb}{0.274952,0.037752,0.364543}%
\pgfsetfillcolor{currentfill}%
\pgfsetfillopacity{0.700000}%
\pgfsetlinewidth{0.000000pt}%
\definecolor{currentstroke}{rgb}{0.000000,0.000000,0.000000}%
\pgfsetstrokecolor{currentstroke}%
\pgfsetdash{}{0pt}%
\pgfpathmoveto{\pgfqpoint{3.332278in}{1.926557in}}%
\pgfpathlineto{\pgfqpoint{3.345386in}{1.921062in}}%
\pgfpathlineto{\pgfqpoint{3.358495in}{1.915758in}}%
\pgfpathlineto{\pgfqpoint{3.371608in}{1.910644in}}%
\pgfpathlineto{\pgfqpoint{3.384723in}{1.905720in}}%
\pgfpathlineto{\pgfqpoint{3.392620in}{1.914360in}}%
\pgfpathlineto{\pgfqpoint{3.400512in}{1.923053in}}%
\pgfpathlineto{\pgfqpoint{3.408397in}{1.931795in}}%
\pgfpathlineto{\pgfqpoint{3.416276in}{1.940585in}}%
\pgfpathlineto{\pgfqpoint{3.403176in}{1.945265in}}%
\pgfpathlineto{\pgfqpoint{3.390079in}{1.950134in}}%
\pgfpathlineto{\pgfqpoint{3.376984in}{1.955194in}}%
\pgfpathlineto{\pgfqpoint{3.363893in}{1.960445in}}%
\pgfpathlineto{\pgfqpoint{3.355999in}{1.951888in}}%
\pgfpathlineto{\pgfqpoint{3.348098in}{1.943387in}}%
\pgfpathlineto{\pgfqpoint{3.340192in}{1.934942in}}%
\pgfpathlineto{\pgfqpoint{3.332278in}{1.926557in}}%
\pgfpathclose%
\pgfusepath{fill}%
\end{pgfscope}%
\begin{pgfscope}%
\pgfpathrectangle{\pgfqpoint{1.254980in}{0.150000in}}{\pgfqpoint{5.490039in}{5.490039in}}%
\pgfusepath{clip}%
\pgfsetbuttcap%
\pgfsetroundjoin%
\definecolor{currentfill}{rgb}{0.171176,0.452530,0.557965}%
\pgfsetfillcolor{currentfill}%
\pgfsetfillopacity{0.700000}%
\pgfsetlinewidth{0.000000pt}%
\definecolor{currentstroke}{rgb}{0.000000,0.000000,0.000000}%
\pgfsetstrokecolor{currentstroke}%
\pgfsetdash{}{0pt}%
\pgfpathmoveto{\pgfqpoint{5.003810in}{2.801520in}}%
\pgfpathlineto{\pgfqpoint{5.017476in}{2.809119in}}%
\pgfpathlineto{\pgfqpoint{5.031156in}{2.816876in}}%
\pgfpathlineto{\pgfqpoint{5.044852in}{2.824789in}}%
\pgfpathlineto{\pgfqpoint{5.058563in}{2.832861in}}%
\pgfpathlineto{\pgfqpoint{5.065823in}{2.838720in}}%
\pgfpathlineto{\pgfqpoint{5.073077in}{2.844551in}}%
\pgfpathlineto{\pgfqpoint{5.080325in}{2.850357in}}%
\pgfpathlineto{\pgfqpoint{5.087566in}{2.856141in}}%
\pgfpathlineto{\pgfqpoint{5.073871in}{2.848398in}}%
\pgfpathlineto{\pgfqpoint{5.060190in}{2.840811in}}%
\pgfpathlineto{\pgfqpoint{5.046525in}{2.833381in}}%
\pgfpathlineto{\pgfqpoint{5.032874in}{2.826107in}}%
\pgfpathlineto{\pgfqpoint{5.025617in}{2.819986in}}%
\pgfpathlineto{\pgfqpoint{5.018354in}{2.813850in}}%
\pgfpathlineto{\pgfqpoint{5.011085in}{2.807696in}}%
\pgfpathlineto{\pgfqpoint{5.003810in}{2.801520in}}%
\pgfpathclose%
\pgfusepath{fill}%
\end{pgfscope}%
\begin{pgfscope}%
\pgfpathrectangle{\pgfqpoint{1.254980in}{0.150000in}}{\pgfqpoint{5.490039in}{5.490039in}}%
\pgfusepath{clip}%
\pgfsetbuttcap%
\pgfsetroundjoin%
\definecolor{currentfill}{rgb}{0.266580,0.228262,0.514349}%
\pgfsetfillcolor{currentfill}%
\pgfsetfillopacity{0.700000}%
\pgfsetlinewidth{0.000000pt}%
\definecolor{currentstroke}{rgb}{0.000000,0.000000,0.000000}%
\pgfsetstrokecolor{currentstroke}%
\pgfsetdash{}{0pt}%
\pgfpathmoveto{\pgfqpoint{2.656409in}{2.321185in}}%
\pgfpathlineto{\pgfqpoint{2.669691in}{2.305059in}}%
\pgfpathlineto{\pgfqpoint{2.682966in}{2.289191in}}%
\pgfpathlineto{\pgfqpoint{2.696233in}{2.273579in}}%
\pgfpathlineto{\pgfqpoint{2.709494in}{2.258220in}}%
\pgfpathlineto{\pgfqpoint{2.717750in}{2.262255in}}%
\pgfpathlineto{\pgfqpoint{2.725996in}{2.266454in}}%
\pgfpathlineto{\pgfqpoint{2.734229in}{2.270812in}}%
\pgfpathlineto{\pgfqpoint{2.742451in}{2.275327in}}%
\pgfpathlineto{\pgfqpoint{2.729222in}{2.290344in}}%
\pgfpathlineto{\pgfqpoint{2.715987in}{2.305614in}}%
\pgfpathlineto{\pgfqpoint{2.702744in}{2.321140in}}%
\pgfpathlineto{\pgfqpoint{2.689494in}{2.336922in}}%
\pgfpathlineto{\pgfqpoint{2.681241in}{2.332739in}}%
\pgfpathlineto{\pgfqpoint{2.672976in}{2.328720in}}%
\pgfpathlineto{\pgfqpoint{2.664698in}{2.324867in}}%
\pgfpathlineto{\pgfqpoint{2.656409in}{2.321185in}}%
\pgfpathclose%
\pgfusepath{fill}%
\end{pgfscope}%
\begin{pgfscope}%
\pgfpathrectangle{\pgfqpoint{1.254980in}{0.150000in}}{\pgfqpoint{5.490039in}{5.490039in}}%
\pgfusepath{clip}%
\pgfsetbuttcap%
\pgfsetroundjoin%
\definecolor{currentfill}{rgb}{0.162142,0.474838,0.558140}%
\pgfsetfillcolor{currentfill}%
\pgfsetfillopacity{0.700000}%
\pgfsetlinewidth{0.000000pt}%
\definecolor{currentstroke}{rgb}{0.000000,0.000000,0.000000}%
\pgfsetstrokecolor{currentstroke}%
\pgfsetdash{}{0pt}%
\pgfpathmoveto{\pgfqpoint{5.087566in}{2.856141in}}%
\pgfpathlineto{\pgfqpoint{5.101277in}{2.864041in}}%
\pgfpathlineto{\pgfqpoint{5.115004in}{2.872098in}}%
\pgfpathlineto{\pgfqpoint{5.128746in}{2.880312in}}%
\pgfpathlineto{\pgfqpoint{5.142503in}{2.888682in}}%
\pgfpathlineto{\pgfqpoint{5.149723in}{2.894102in}}%
\pgfpathlineto{\pgfqpoint{5.156936in}{2.899502in}}%
\pgfpathlineto{\pgfqpoint{5.164142in}{2.904886in}}%
\pgfpathlineto{\pgfqpoint{5.171343in}{2.910257in}}%
\pgfpathlineto{\pgfqpoint{5.157602in}{2.902244in}}%
\pgfpathlineto{\pgfqpoint{5.143877in}{2.894387in}}%
\pgfpathlineto{\pgfqpoint{5.130167in}{2.886686in}}%
\pgfpathlineto{\pgfqpoint{5.116472in}{2.879142in}}%
\pgfpathlineto{\pgfqpoint{5.109255in}{2.873404in}}%
\pgfpathlineto{\pgfqpoint{5.102031in}{2.867661in}}%
\pgfpathlineto{\pgfqpoint{5.094802in}{2.861908in}}%
\pgfpathlineto{\pgfqpoint{5.087566in}{2.856141in}}%
\pgfpathclose%
\pgfusepath{fill}%
\end{pgfscope}%
\begin{pgfscope}%
\pgfpathrectangle{\pgfqpoint{1.254980in}{0.150000in}}{\pgfqpoint{5.490039in}{5.490039in}}%
\pgfusepath{clip}%
\pgfsetbuttcap%
\pgfsetroundjoin%
\definecolor{currentfill}{rgb}{0.274128,0.199721,0.498911}%
\pgfsetfillcolor{currentfill}%
\pgfsetfillopacity{0.700000}%
\pgfsetlinewidth{0.000000pt}%
\definecolor{currentstroke}{rgb}{0.000000,0.000000,0.000000}%
\pgfsetstrokecolor{currentstroke}%
\pgfsetdash{}{0pt}%
\pgfpathmoveto{\pgfqpoint{2.709494in}{2.258220in}}%
\pgfpathlineto{\pgfqpoint{2.722747in}{2.243112in}}%
\pgfpathlineto{\pgfqpoint{2.735995in}{2.228254in}}%
\pgfpathlineto{\pgfqpoint{2.749236in}{2.213643in}}%
\pgfpathlineto{\pgfqpoint{2.762471in}{2.199277in}}%
\pgfpathlineto{\pgfqpoint{2.770696in}{2.203664in}}%
\pgfpathlineto{\pgfqpoint{2.778910in}{2.208207in}}%
\pgfpathlineto{\pgfqpoint{2.787113in}{2.212901in}}%
\pgfpathlineto{\pgfqpoint{2.795306in}{2.217745in}}%
\pgfpathlineto{\pgfqpoint{2.782101in}{2.231771in}}%
\pgfpathlineto{\pgfqpoint{2.768890in}{2.246042in}}%
\pgfpathlineto{\pgfqpoint{2.755674in}{2.260560in}}%
\pgfpathlineto{\pgfqpoint{2.742451in}{2.275327in}}%
\pgfpathlineto{\pgfqpoint{2.734229in}{2.270812in}}%
\pgfpathlineto{\pgfqpoint{2.725996in}{2.266454in}}%
\pgfpathlineto{\pgfqpoint{2.717750in}{2.262255in}}%
\pgfpathlineto{\pgfqpoint{2.709494in}{2.258220in}}%
\pgfpathclose%
\pgfusepath{fill}%
\end{pgfscope}%
\begin{pgfscope}%
\pgfpathrectangle{\pgfqpoint{1.254980in}{0.150000in}}{\pgfqpoint{5.490039in}{5.490039in}}%
\pgfusepath{clip}%
\pgfsetbuttcap%
\pgfsetroundjoin%
\definecolor{currentfill}{rgb}{0.237441,0.305202,0.541921}%
\pgfsetfillcolor{currentfill}%
\pgfsetfillopacity{0.700000}%
\pgfsetlinewidth{0.000000pt}%
\definecolor{currentstroke}{rgb}{0.000000,0.000000,0.000000}%
\pgfsetstrokecolor{currentstroke}%
\pgfsetdash{}{0pt}%
\pgfpathmoveto{\pgfqpoint{4.471933in}{2.436643in}}%
\pgfpathlineto{\pgfqpoint{4.485345in}{2.441877in}}%
\pgfpathlineto{\pgfqpoint{4.498770in}{2.447274in}}%
\pgfpathlineto{\pgfqpoint{4.512206in}{2.452833in}}%
\pgfpathlineto{\pgfqpoint{4.525656in}{2.458553in}}%
\pgfpathlineto{\pgfqpoint{4.533154in}{2.467139in}}%
\pgfpathlineto{\pgfqpoint{4.540647in}{2.475663in}}%
\pgfpathlineto{\pgfqpoint{4.548134in}{2.484127in}}%
\pgfpathlineto{\pgfqpoint{4.555615in}{2.492533in}}%
\pgfpathlineto{\pgfqpoint{4.542174in}{2.486935in}}%
\pgfpathlineto{\pgfqpoint{4.528745in}{2.481499in}}%
\pgfpathlineto{\pgfqpoint{4.515328in}{2.476225in}}%
\pgfpathlineto{\pgfqpoint{4.501923in}{2.471112in}}%
\pgfpathlineto{\pgfqpoint{4.494434in}{2.462573in}}%
\pgfpathlineto{\pgfqpoint{4.486939in}{2.453983in}}%
\pgfpathlineto{\pgfqpoint{4.479439in}{2.445340in}}%
\pgfpathlineto{\pgfqpoint{4.471933in}{2.436643in}}%
\pgfpathclose%
\pgfusepath{fill}%
\end{pgfscope}%
\begin{pgfscope}%
\pgfpathrectangle{\pgfqpoint{1.254980in}{0.150000in}}{\pgfqpoint{5.490039in}{5.490039in}}%
\pgfusepath{clip}%
\pgfsetbuttcap%
\pgfsetroundjoin%
\definecolor{currentfill}{rgb}{0.255645,0.260703,0.528312}%
\pgfsetfillcolor{currentfill}%
\pgfsetfillopacity{0.700000}%
\pgfsetlinewidth{0.000000pt}%
\definecolor{currentstroke}{rgb}{0.000000,0.000000,0.000000}%
\pgfsetstrokecolor{currentstroke}%
\pgfsetdash{}{0pt}%
\pgfpathmoveto{\pgfqpoint{2.603199in}{2.388310in}}%
\pgfpathlineto{\pgfqpoint{2.616514in}{2.371131in}}%
\pgfpathlineto{\pgfqpoint{2.629821in}{2.354219in}}%
\pgfpathlineto{\pgfqpoint{2.643119in}{2.337571in}}%
\pgfpathlineto{\pgfqpoint{2.656409in}{2.321185in}}%
\pgfpathlineto{\pgfqpoint{2.664698in}{2.324867in}}%
\pgfpathlineto{\pgfqpoint{2.672976in}{2.328720in}}%
\pgfpathlineto{\pgfqpoint{2.681241in}{2.332739in}}%
\pgfpathlineto{\pgfqpoint{2.689494in}{2.336922in}}%
\pgfpathlineto{\pgfqpoint{2.676237in}{2.352964in}}%
\pgfpathlineto{\pgfqpoint{2.662973in}{2.369268in}}%
\pgfpathlineto{\pgfqpoint{2.649700in}{2.385835in}}%
\pgfpathlineto{\pgfqpoint{2.636420in}{2.402669in}}%
\pgfpathlineto{\pgfqpoint{2.628133in}{2.398819in}}%
\pgfpathlineto{\pgfqpoint{2.619835in}{2.395141in}}%
\pgfpathlineto{\pgfqpoint{2.611523in}{2.391637in}}%
\pgfpathlineto{\pgfqpoint{2.603199in}{2.388310in}}%
\pgfpathclose%
\pgfusepath{fill}%
\end{pgfscope}%
\begin{pgfscope}%
\pgfpathrectangle{\pgfqpoint{1.254980in}{0.150000in}}{\pgfqpoint{5.490039in}{5.490039in}}%
\pgfusepath{clip}%
\pgfsetbuttcap%
\pgfsetroundjoin%
\definecolor{currentfill}{rgb}{0.278826,0.175490,0.483397}%
\pgfsetfillcolor{currentfill}%
\pgfsetfillopacity{0.700000}%
\pgfsetlinewidth{0.000000pt}%
\definecolor{currentstroke}{rgb}{0.000000,0.000000,0.000000}%
\pgfsetstrokecolor{currentstroke}%
\pgfsetdash{}{0pt}%
\pgfpathmoveto{\pgfqpoint{2.762471in}{2.199277in}}%
\pgfpathlineto{\pgfqpoint{2.775700in}{2.185155in}}%
\pgfpathlineto{\pgfqpoint{2.788924in}{2.171274in}}%
\pgfpathlineto{\pgfqpoint{2.802143in}{2.157633in}}%
\pgfpathlineto{\pgfqpoint{2.815356in}{2.144230in}}%
\pgfpathlineto{\pgfqpoint{2.823551in}{2.148966in}}%
\pgfpathlineto{\pgfqpoint{2.831736in}{2.153851in}}%
\pgfpathlineto{\pgfqpoint{2.839910in}{2.158880in}}%
\pgfpathlineto{\pgfqpoint{2.848073in}{2.164051in}}%
\pgfpathlineto{\pgfqpoint{2.834888in}{2.177117in}}%
\pgfpathlineto{\pgfqpoint{2.821699in}{2.190420in}}%
\pgfpathlineto{\pgfqpoint{2.808505in}{2.203962in}}%
\pgfpathlineto{\pgfqpoint{2.795306in}{2.217745in}}%
\pgfpathlineto{\pgfqpoint{2.787113in}{2.212901in}}%
\pgfpathlineto{\pgfqpoint{2.778910in}{2.208207in}}%
\pgfpathlineto{\pgfqpoint{2.770696in}{2.203664in}}%
\pgfpathlineto{\pgfqpoint{2.762471in}{2.199277in}}%
\pgfpathclose%
\pgfusepath{fill}%
\end{pgfscope}%
\begin{pgfscope}%
\pgfpathrectangle{\pgfqpoint{1.254980in}{0.150000in}}{\pgfqpoint{5.490039in}{5.490039in}}%
\pgfusepath{clip}%
\pgfsetbuttcap%
\pgfsetroundjoin%
\definecolor{currentfill}{rgb}{0.279566,0.067836,0.391917}%
\pgfsetfillcolor{currentfill}%
\pgfsetfillopacity{0.700000}%
\pgfsetlinewidth{0.000000pt}%
\definecolor{currentstroke}{rgb}{0.000000,0.000000,0.000000}%
\pgfsetstrokecolor{currentstroke}%
\pgfsetdash{}{0pt}%
\pgfpathmoveto{\pgfqpoint{3.058557in}{1.985908in}}%
\pgfpathlineto{\pgfqpoint{3.071693in}{1.976627in}}%
\pgfpathlineto{\pgfqpoint{3.084829in}{1.967556in}}%
\pgfpathlineto{\pgfqpoint{3.097963in}{1.958693in}}%
\pgfpathlineto{\pgfqpoint{3.111097in}{1.950037in}}%
\pgfpathlineto{\pgfqpoint{3.119127in}{1.956926in}}%
\pgfpathlineto{\pgfqpoint{3.127148in}{1.963916in}}%
\pgfpathlineto{\pgfqpoint{3.135162in}{1.971003in}}%
\pgfpathlineto{\pgfqpoint{3.143167in}{1.978184in}}%
\pgfpathlineto{\pgfqpoint{3.130054in}{1.986538in}}%
\pgfpathlineto{\pgfqpoint{3.116941in}{1.995098in}}%
\pgfpathlineto{\pgfqpoint{3.103828in}{2.003866in}}%
\pgfpathlineto{\pgfqpoint{3.090713in}{2.012843in}}%
\pgfpathlineto{\pgfqpoint{3.082687in}{2.005954in}}%
\pgfpathlineto{\pgfqpoint{3.074652in}{1.999166in}}%
\pgfpathlineto{\pgfqpoint{3.066609in}{1.992483in}}%
\pgfpathlineto{\pgfqpoint{3.058557in}{1.985908in}}%
\pgfpathclose%
\pgfusepath{fill}%
\end{pgfscope}%
\begin{pgfscope}%
\pgfpathrectangle{\pgfqpoint{1.254980in}{0.150000in}}{\pgfqpoint{5.490039in}{5.490039in}}%
\pgfusepath{clip}%
\pgfsetbuttcap%
\pgfsetroundjoin%
\definecolor{currentfill}{rgb}{0.276022,0.044167,0.370164}%
\pgfsetfillcolor{currentfill}%
\pgfsetfillopacity{0.700000}%
\pgfsetlinewidth{0.000000pt}%
\definecolor{currentstroke}{rgb}{0.000000,0.000000,0.000000}%
\pgfsetstrokecolor{currentstroke}%
\pgfsetdash{}{0pt}%
\pgfpathmoveto{\pgfqpoint{3.468708in}{1.923746in}}%
\pgfpathlineto{\pgfqpoint{3.481825in}{1.920002in}}%
\pgfpathlineto{\pgfqpoint{3.494946in}{1.916442in}}%
\pgfpathlineto{\pgfqpoint{3.508071in}{1.913066in}}%
\pgfpathlineto{\pgfqpoint{3.521200in}{1.909874in}}%
\pgfpathlineto{\pgfqpoint{3.529045in}{1.919163in}}%
\pgfpathlineto{\pgfqpoint{3.536885in}{1.928483in}}%
\pgfpathlineto{\pgfqpoint{3.544719in}{1.937831in}}%
\pgfpathlineto{\pgfqpoint{3.552547in}{1.947206in}}%
\pgfpathlineto{\pgfqpoint{3.539431in}{1.950183in}}%
\pgfpathlineto{\pgfqpoint{3.526318in}{1.953342in}}%
\pgfpathlineto{\pgfqpoint{3.513211in}{1.956686in}}%
\pgfpathlineto{\pgfqpoint{3.500107in}{1.960214in}}%
\pgfpathlineto{\pgfqpoint{3.492266in}{1.951045in}}%
\pgfpathlineto{\pgfqpoint{3.484419in}{1.941910in}}%
\pgfpathlineto{\pgfqpoint{3.476566in}{1.932809in}}%
\pgfpathlineto{\pgfqpoint{3.468708in}{1.923746in}}%
\pgfpathclose%
\pgfusepath{fill}%
\end{pgfscope}%
\begin{pgfscope}%
\pgfpathrectangle{\pgfqpoint{1.254980in}{0.150000in}}{\pgfqpoint{5.490039in}{5.490039in}}%
\pgfusepath{clip}%
\pgfsetbuttcap%
\pgfsetroundjoin%
\definecolor{currentfill}{rgb}{0.153364,0.497000,0.557724}%
\pgfsetfillcolor{currentfill}%
\pgfsetfillopacity{0.700000}%
\pgfsetlinewidth{0.000000pt}%
\definecolor{currentstroke}{rgb}{0.000000,0.000000,0.000000}%
\pgfsetstrokecolor{currentstroke}%
\pgfsetdash{}{0pt}%
\pgfpathmoveto{\pgfqpoint{5.171343in}{2.910257in}}%
\pgfpathlineto{\pgfqpoint{5.185100in}{2.918425in}}%
\pgfpathlineto{\pgfqpoint{5.198872in}{2.926750in}}%
\pgfpathlineto{\pgfqpoint{5.212660in}{2.935231in}}%
\pgfpathlineto{\pgfqpoint{5.226465in}{2.943869in}}%
\pgfpathlineto{\pgfqpoint{5.233642in}{2.948855in}}%
\pgfpathlineto{\pgfqpoint{5.240813in}{2.953831in}}%
\pgfpathlineto{\pgfqpoint{5.247977in}{2.958801in}}%
\pgfpathlineto{\pgfqpoint{5.255136in}{2.963768in}}%
\pgfpathlineto{\pgfqpoint{5.241350in}{2.955519in}}%
\pgfpathlineto{\pgfqpoint{5.227580in}{2.947424in}}%
\pgfpathlineto{\pgfqpoint{5.213825in}{2.939485in}}%
\pgfpathlineto{\pgfqpoint{5.200087in}{2.931702in}}%
\pgfpathlineto{\pgfqpoint{5.192909in}{2.926338in}}%
\pgfpathlineto{\pgfqpoint{5.185726in}{2.920978in}}%
\pgfpathlineto{\pgfqpoint{5.178538in}{2.915619in}}%
\pgfpathlineto{\pgfqpoint{5.171343in}{2.910257in}}%
\pgfpathclose%
\pgfusepath{fill}%
\end{pgfscope}%
\begin{pgfscope}%
\pgfpathrectangle{\pgfqpoint{1.254980in}{0.150000in}}{\pgfqpoint{5.490039in}{5.490039in}}%
\pgfusepath{clip}%
\pgfsetbuttcap%
\pgfsetroundjoin%
\definecolor{currentfill}{rgb}{0.243113,0.292092,0.538516}%
\pgfsetfillcolor{currentfill}%
\pgfsetfillopacity{0.700000}%
\pgfsetlinewidth{0.000000pt}%
\definecolor{currentstroke}{rgb}{0.000000,0.000000,0.000000}%
\pgfsetstrokecolor{currentstroke}%
\pgfsetdash{}{0pt}%
\pgfpathmoveto{\pgfqpoint{2.549850in}{2.459744in}}%
\pgfpathlineto{\pgfqpoint{2.563201in}{2.441473in}}%
\pgfpathlineto{\pgfqpoint{2.576543in}{2.423479in}}%
\pgfpathlineto{\pgfqpoint{2.589876in}{2.405759in}}%
\pgfpathlineto{\pgfqpoint{2.603199in}{2.388310in}}%
\pgfpathlineto{\pgfqpoint{2.611523in}{2.391637in}}%
\pgfpathlineto{\pgfqpoint{2.619835in}{2.395141in}}%
\pgfpathlineto{\pgfqpoint{2.628133in}{2.398819in}}%
\pgfpathlineto{\pgfqpoint{2.636420in}{2.402669in}}%
\pgfpathlineto{\pgfqpoint{2.623130in}{2.419771in}}%
\pgfpathlineto{\pgfqpoint{2.609832in}{2.437144in}}%
\pgfpathlineto{\pgfqpoint{2.596526in}{2.454790in}}%
\pgfpathlineto{\pgfqpoint{2.583209in}{2.472713in}}%
\pgfpathlineto{\pgfqpoint{2.574889in}{2.469199in}}%
\pgfpathlineto{\pgfqpoint{2.566556in}{2.465864in}}%
\pgfpathlineto{\pgfqpoint{2.558209in}{2.462711in}}%
\pgfpathlineto{\pgfqpoint{2.549850in}{2.459744in}}%
\pgfpathclose%
\pgfusepath{fill}%
\end{pgfscope}%
\begin{pgfscope}%
\pgfpathrectangle{\pgfqpoint{1.254980in}{0.150000in}}{\pgfqpoint{5.490039in}{5.490039in}}%
\pgfusepath{clip}%
\pgfsetbuttcap%
\pgfsetroundjoin%
\definecolor{currentfill}{rgb}{0.225863,0.330805,0.547314}%
\pgfsetfillcolor{currentfill}%
\pgfsetfillopacity{0.700000}%
\pgfsetlinewidth{0.000000pt}%
\definecolor{currentstroke}{rgb}{0.000000,0.000000,0.000000}%
\pgfsetstrokecolor{currentstroke}%
\pgfsetdash{}{0pt}%
\pgfpathmoveto{\pgfqpoint{4.555615in}{2.492533in}}%
\pgfpathlineto{\pgfqpoint{4.569069in}{2.498293in}}%
\pgfpathlineto{\pgfqpoint{4.582536in}{2.504214in}}%
\pgfpathlineto{\pgfqpoint{4.596015in}{2.510296in}}%
\pgfpathlineto{\pgfqpoint{4.609508in}{2.516539in}}%
\pgfpathlineto{\pgfqpoint{4.616975in}{2.524749in}}%
\pgfpathlineto{\pgfqpoint{4.624436in}{2.532897in}}%
\pgfpathlineto{\pgfqpoint{4.631891in}{2.540987in}}%
\pgfpathlineto{\pgfqpoint{4.639340in}{2.549018in}}%
\pgfpathlineto{\pgfqpoint{4.625856in}{2.542927in}}%
\pgfpathlineto{\pgfqpoint{4.612385in}{2.536996in}}%
\pgfpathlineto{\pgfqpoint{4.598927in}{2.531227in}}%
\pgfpathlineto{\pgfqpoint{4.585482in}{2.525618in}}%
\pgfpathlineto{\pgfqpoint{4.578024in}{2.517424in}}%
\pgfpathlineto{\pgfqpoint{4.570560in}{2.509180in}}%
\pgfpathlineto{\pgfqpoint{4.563090in}{2.500884in}}%
\pgfpathlineto{\pgfqpoint{4.555615in}{2.492533in}}%
\pgfpathclose%
\pgfusepath{fill}%
\end{pgfscope}%
\begin{pgfscope}%
\pgfpathrectangle{\pgfqpoint{1.254980in}{0.150000in}}{\pgfqpoint{5.490039in}{5.490039in}}%
\pgfusepath{clip}%
\pgfsetbuttcap%
\pgfsetroundjoin%
\definecolor{currentfill}{rgb}{0.282290,0.145912,0.461510}%
\pgfsetfillcolor{currentfill}%
\pgfsetfillopacity{0.700000}%
\pgfsetlinewidth{0.000000pt}%
\definecolor{currentstroke}{rgb}{0.000000,0.000000,0.000000}%
\pgfsetstrokecolor{currentstroke}%
\pgfsetdash{}{0pt}%
\pgfpathmoveto{\pgfqpoint{2.815356in}{2.144230in}}%
\pgfpathlineto{\pgfqpoint{2.828565in}{2.131063in}}%
\pgfpathlineto{\pgfqpoint{2.841769in}{2.118130in}}%
\pgfpathlineto{\pgfqpoint{2.854968in}{2.105429in}}%
\pgfpathlineto{\pgfqpoint{2.868164in}{2.092959in}}%
\pgfpathlineto{\pgfqpoint{2.876330in}{2.098043in}}%
\pgfpathlineto{\pgfqpoint{2.884486in}{2.103267in}}%
\pgfpathlineto{\pgfqpoint{2.892632in}{2.108629in}}%
\pgfpathlineto{\pgfqpoint{2.900768in}{2.114126in}}%
\pgfpathlineto{\pgfqpoint{2.887600in}{2.126260in}}%
\pgfpathlineto{\pgfqpoint{2.874429in}{2.138624in}}%
\pgfpathlineto{\pgfqpoint{2.861253in}{2.151221in}}%
\pgfpathlineto{\pgfqpoint{2.848073in}{2.164051in}}%
\pgfpathlineto{\pgfqpoint{2.839910in}{2.158880in}}%
\pgfpathlineto{\pgfqpoint{2.831736in}{2.153851in}}%
\pgfpathlineto{\pgfqpoint{2.823551in}{2.148966in}}%
\pgfpathlineto{\pgfqpoint{2.815356in}{2.144230in}}%
\pgfpathclose%
\pgfusepath{fill}%
\end{pgfscope}%
\begin{pgfscope}%
\pgfpathrectangle{\pgfqpoint{1.254980in}{0.150000in}}{\pgfqpoint{5.490039in}{5.490039in}}%
\pgfusepath{clip}%
\pgfsetbuttcap%
\pgfsetroundjoin%
\definecolor{currentfill}{rgb}{0.146180,0.515413,0.556823}%
\pgfsetfillcolor{currentfill}%
\pgfsetfillopacity{0.700000}%
\pgfsetlinewidth{0.000000pt}%
\definecolor{currentstroke}{rgb}{0.000000,0.000000,0.000000}%
\pgfsetstrokecolor{currentstroke}%
\pgfsetdash{}{0pt}%
\pgfpathmoveto{\pgfqpoint{5.255136in}{2.963768in}}%
\pgfpathlineto{\pgfqpoint{5.268938in}{2.972174in}}%
\pgfpathlineto{\pgfqpoint{5.282756in}{2.980735in}}%
\pgfpathlineto{\pgfqpoint{5.296591in}{2.989451in}}%
\pgfpathlineto{\pgfqpoint{5.310442in}{2.998323in}}%
\pgfpathlineto{\pgfqpoint{5.317576in}{3.002887in}}%
\pgfpathlineto{\pgfqpoint{5.324703in}{3.007451in}}%
\pgfpathlineto{\pgfqpoint{5.331824in}{3.012020in}}%
\pgfpathlineto{\pgfqpoint{5.338940in}{3.016599in}}%
\pgfpathlineto{\pgfqpoint{5.325109in}{3.008144in}}%
\pgfpathlineto{\pgfqpoint{5.311294in}{2.999844in}}%
\pgfpathlineto{\pgfqpoint{5.297496in}{2.991699in}}%
\pgfpathlineto{\pgfqpoint{5.283713in}{2.983708in}}%
\pgfpathlineto{\pgfqpoint{5.276577in}{2.978704in}}%
\pgfpathlineto{\pgfqpoint{5.269436in}{2.973715in}}%
\pgfpathlineto{\pgfqpoint{5.262289in}{2.968738in}}%
\pgfpathlineto{\pgfqpoint{5.255136in}{2.963768in}}%
\pgfpathclose%
\pgfusepath{fill}%
\end{pgfscope}%
\begin{pgfscope}%
\pgfpathrectangle{\pgfqpoint{1.254980in}{0.150000in}}{\pgfqpoint{5.490039in}{5.490039in}}%
\pgfusepath{clip}%
\pgfsetbuttcap%
\pgfsetroundjoin%
\definecolor{currentfill}{rgb}{0.227802,0.326594,0.546532}%
\pgfsetfillcolor{currentfill}%
\pgfsetfillopacity{0.700000}%
\pgfsetlinewidth{0.000000pt}%
\definecolor{currentstroke}{rgb}{0.000000,0.000000,0.000000}%
\pgfsetstrokecolor{currentstroke}%
\pgfsetdash{}{0pt}%
\pgfpathmoveto{\pgfqpoint{2.496341in}{2.535643in}}%
\pgfpathlineto{\pgfqpoint{2.509734in}{2.516241in}}%
\pgfpathlineto{\pgfqpoint{2.523116in}{2.497125in}}%
\pgfpathlineto{\pgfqpoint{2.536488in}{2.478293in}}%
\pgfpathlineto{\pgfqpoint{2.549850in}{2.459744in}}%
\pgfpathlineto{\pgfqpoint{2.558209in}{2.462711in}}%
\pgfpathlineto{\pgfqpoint{2.566556in}{2.465864in}}%
\pgfpathlineto{\pgfqpoint{2.574889in}{2.469199in}}%
\pgfpathlineto{\pgfqpoint{2.583209in}{2.472713in}}%
\pgfpathlineto{\pgfqpoint{2.569884in}{2.490913in}}%
\pgfpathlineto{\pgfqpoint{2.556548in}{2.509395in}}%
\pgfpathlineto{\pgfqpoint{2.543202in}{2.528160in}}%
\pgfpathlineto{\pgfqpoint{2.529846in}{2.547212in}}%
\pgfpathlineto{\pgfqpoint{2.521490in}{2.544037in}}%
\pgfpathlineto{\pgfqpoint{2.513121in}{2.541049in}}%
\pgfpathlineto{\pgfqpoint{2.504738in}{2.538250in}}%
\pgfpathlineto{\pgfqpoint{2.496341in}{2.535643in}}%
\pgfpathclose%
\pgfusepath{fill}%
\end{pgfscope}%
\begin{pgfscope}%
\pgfpathrectangle{\pgfqpoint{1.254980in}{0.150000in}}{\pgfqpoint{5.490039in}{5.490039in}}%
\pgfusepath{clip}%
\pgfsetbuttcap%
\pgfsetroundjoin%
\definecolor{currentfill}{rgb}{0.283229,0.120777,0.440584}%
\pgfsetfillcolor{currentfill}%
\pgfsetfillopacity{0.700000}%
\pgfsetlinewidth{0.000000pt}%
\definecolor{currentstroke}{rgb}{0.000000,0.000000,0.000000}%
\pgfsetstrokecolor{currentstroke}%
\pgfsetdash{}{0pt}%
\pgfpathmoveto{\pgfqpoint{3.856176in}{2.045623in}}%
\pgfpathlineto{\pgfqpoint{3.869366in}{2.046138in}}%
\pgfpathlineto{\pgfqpoint{3.882563in}{2.046825in}}%
\pgfpathlineto{\pgfqpoint{3.895768in}{2.047684in}}%
\pgfpathlineto{\pgfqpoint{3.908982in}{2.048714in}}%
\pgfpathlineto{\pgfqpoint{3.916695in}{2.058879in}}%
\pgfpathlineto{\pgfqpoint{3.924403in}{2.069019in}}%
\pgfpathlineto{\pgfqpoint{3.932106in}{2.079133in}}%
\pgfpathlineto{\pgfqpoint{3.939804in}{2.089219in}}%
\pgfpathlineto{\pgfqpoint{3.926598in}{2.088084in}}%
\pgfpathlineto{\pgfqpoint{3.913401in}{2.087121in}}%
\pgfpathlineto{\pgfqpoint{3.900211in}{2.086329in}}%
\pgfpathlineto{\pgfqpoint{3.887029in}{2.085709in}}%
\pgfpathlineto{\pgfqpoint{3.879323in}{2.075717in}}%
\pgfpathlineto{\pgfqpoint{3.871613in}{2.065705in}}%
\pgfpathlineto{\pgfqpoint{3.863897in}{2.055673in}}%
\pgfpathlineto{\pgfqpoint{3.856176in}{2.045623in}}%
\pgfpathclose%
\pgfusepath{fill}%
\end{pgfscope}%
\begin{pgfscope}%
\pgfpathrectangle{\pgfqpoint{1.254980in}{0.150000in}}{\pgfqpoint{5.490039in}{5.490039in}}%
\pgfusepath{clip}%
\pgfsetbuttcap%
\pgfsetroundjoin%
\definecolor{currentfill}{rgb}{0.282623,0.140926,0.457517}%
\pgfsetfillcolor{currentfill}%
\pgfsetfillopacity{0.700000}%
\pgfsetlinewidth{0.000000pt}%
\definecolor{currentstroke}{rgb}{0.000000,0.000000,0.000000}%
\pgfsetstrokecolor{currentstroke}%
\pgfsetdash{}{0pt}%
\pgfpathmoveto{\pgfqpoint{3.939804in}{2.089219in}}%
\pgfpathlineto{\pgfqpoint{3.953018in}{2.090524in}}%
\pgfpathlineto{\pgfqpoint{3.966240in}{2.092000in}}%
\pgfpathlineto{\pgfqpoint{3.979471in}{2.093645in}}%
\pgfpathlineto{\pgfqpoint{3.992710in}{2.095459in}}%
\pgfpathlineto{\pgfqpoint{4.000396in}{2.105606in}}%
\pgfpathlineto{\pgfqpoint{4.008077in}{2.115717in}}%
\pgfpathlineto{\pgfqpoint{4.015753in}{2.125792in}}%
\pgfpathlineto{\pgfqpoint{4.023425in}{2.135832in}}%
\pgfpathlineto{\pgfqpoint{4.010192in}{2.133940in}}%
\pgfpathlineto{\pgfqpoint{3.996969in}{2.132218in}}%
\pgfpathlineto{\pgfqpoint{3.983754in}{2.130666in}}%
\pgfpathlineto{\pgfqpoint{3.970547in}{2.129283in}}%
\pgfpathlineto{\pgfqpoint{3.962869in}{2.119310in}}%
\pgfpathlineto{\pgfqpoint{3.955185in}{2.109308in}}%
\pgfpathlineto{\pgfqpoint{3.947497in}{2.099278in}}%
\pgfpathlineto{\pgfqpoint{3.939804in}{2.089219in}}%
\pgfpathclose%
\pgfusepath{fill}%
\end{pgfscope}%
\begin{pgfscope}%
\pgfpathrectangle{\pgfqpoint{1.254980in}{0.150000in}}{\pgfqpoint{5.490039in}{5.490039in}}%
\pgfusepath{clip}%
\pgfsetbuttcap%
\pgfsetroundjoin%
\definecolor{currentfill}{rgb}{0.212395,0.359683,0.551710}%
\pgfsetfillcolor{currentfill}%
\pgfsetfillopacity{0.700000}%
\pgfsetlinewidth{0.000000pt}%
\definecolor{currentstroke}{rgb}{0.000000,0.000000,0.000000}%
\pgfsetstrokecolor{currentstroke}%
\pgfsetdash{}{0pt}%
\pgfpathmoveto{\pgfqpoint{4.639340in}{2.549018in}}%
\pgfpathlineto{\pgfqpoint{4.652837in}{2.555271in}}%
\pgfpathlineto{\pgfqpoint{4.666348in}{2.561683in}}%
\pgfpathlineto{\pgfqpoint{4.679871in}{2.568256in}}%
\pgfpathlineto{\pgfqpoint{4.693408in}{2.574990in}}%
\pgfpathlineto{\pgfqpoint{4.700842in}{2.582796in}}%
\pgfpathlineto{\pgfqpoint{4.708270in}{2.590543in}}%
\pgfpathlineto{\pgfqpoint{4.715692in}{2.598231in}}%
\pgfpathlineto{\pgfqpoint{4.723108in}{2.605865in}}%
\pgfpathlineto{\pgfqpoint{4.709580in}{2.599312in}}%
\pgfpathlineto{\pgfqpoint{4.696066in}{2.592920in}}%
\pgfpathlineto{\pgfqpoint{4.682565in}{2.586688in}}%
\pgfpathlineto{\pgfqpoint{4.669078in}{2.580616in}}%
\pgfpathlineto{\pgfqpoint{4.661652in}{2.572792in}}%
\pgfpathlineto{\pgfqpoint{4.654221in}{2.564919in}}%
\pgfpathlineto{\pgfqpoint{4.646783in}{2.556995in}}%
\pgfpathlineto{\pgfqpoint{4.639340in}{2.549018in}}%
\pgfpathclose%
\pgfusepath{fill}%
\end{pgfscope}%
\begin{pgfscope}%
\pgfpathrectangle{\pgfqpoint{1.254980in}{0.150000in}}{\pgfqpoint{5.490039in}{5.490039in}}%
\pgfusepath{clip}%
\pgfsetbuttcap%
\pgfsetroundjoin%
\definecolor{currentfill}{rgb}{0.274952,0.037752,0.364543}%
\pgfsetfillcolor{currentfill}%
\pgfsetfillopacity{0.700000}%
\pgfsetlinewidth{0.000000pt}%
\definecolor{currentstroke}{rgb}{0.000000,0.000000,0.000000}%
\pgfsetstrokecolor{currentstroke}%
\pgfsetdash{}{0pt}%
\pgfpathmoveto{\pgfqpoint{3.248077in}{1.918662in}}%
\pgfpathlineto{\pgfqpoint{3.261194in}{1.912118in}}%
\pgfpathlineto{\pgfqpoint{3.274313in}{1.905769in}}%
\pgfpathlineto{\pgfqpoint{3.287434in}{1.899615in}}%
\pgfpathlineto{\pgfqpoint{3.300556in}{1.893655in}}%
\pgfpathlineto{\pgfqpoint{3.308497in}{1.901780in}}%
\pgfpathlineto{\pgfqpoint{3.316431in}{1.909974in}}%
\pgfpathlineto{\pgfqpoint{3.324358in}{1.918233in}}%
\pgfpathlineto{\pgfqpoint{3.332278in}{1.926557in}}%
\pgfpathlineto{\pgfqpoint{3.319173in}{1.932245in}}%
\pgfpathlineto{\pgfqpoint{3.306069in}{1.938127in}}%
\pgfpathlineto{\pgfqpoint{3.292968in}{1.944203in}}%
\pgfpathlineto{\pgfqpoint{3.279868in}{1.950474in}}%
\pgfpathlineto{\pgfqpoint{3.271931in}{1.942413in}}%
\pgfpathlineto{\pgfqpoint{3.263987in}{1.934422in}}%
\pgfpathlineto{\pgfqpoint{3.256035in}{1.926504in}}%
\pgfpathlineto{\pgfqpoint{3.248077in}{1.918662in}}%
\pgfpathclose%
\pgfusepath{fill}%
\end{pgfscope}%
\begin{pgfscope}%
\pgfpathrectangle{\pgfqpoint{1.254980in}{0.150000in}}{\pgfqpoint{5.490039in}{5.490039in}}%
\pgfusepath{clip}%
\pgfsetbuttcap%
\pgfsetroundjoin%
\definecolor{currentfill}{rgb}{0.137770,0.537492,0.554906}%
\pgfsetfillcolor{currentfill}%
\pgfsetfillopacity{0.700000}%
\pgfsetlinewidth{0.000000pt}%
\definecolor{currentstroke}{rgb}{0.000000,0.000000,0.000000}%
\pgfsetstrokecolor{currentstroke}%
\pgfsetdash{}{0pt}%
\pgfpathmoveto{\pgfqpoint{5.338940in}{3.016599in}}%
\pgfpathlineto{\pgfqpoint{5.352787in}{3.025208in}}%
\pgfpathlineto{\pgfqpoint{5.366651in}{3.033973in}}%
\pgfpathlineto{\pgfqpoint{5.380532in}{3.042893in}}%
\pgfpathlineto{\pgfqpoint{5.394430in}{3.051968in}}%
\pgfpathlineto{\pgfqpoint{5.401518in}{3.056125in}}%
\pgfpathlineto{\pgfqpoint{5.408601in}{3.060294in}}%
\pgfpathlineto{\pgfqpoint{5.415679in}{3.064481in}}%
\pgfpathlineto{\pgfqpoint{5.422750in}{3.068690in}}%
\pgfpathlineto{\pgfqpoint{5.408875in}{3.060062in}}%
\pgfpathlineto{\pgfqpoint{5.395016in}{3.051589in}}%
\pgfpathlineto{\pgfqpoint{5.381174in}{3.043270in}}%
\pgfpathlineto{\pgfqpoint{5.367348in}{3.035105in}}%
\pgfpathlineto{\pgfqpoint{5.360254in}{3.030440in}}%
\pgfpathlineto{\pgfqpoint{5.353155in}{3.025804in}}%
\pgfpathlineto{\pgfqpoint{5.346050in}{3.021192in}}%
\pgfpathlineto{\pgfqpoint{5.338940in}{3.016599in}}%
\pgfpathclose%
\pgfusepath{fill}%
\end{pgfscope}%
\begin{pgfscope}%
\pgfpathrectangle{\pgfqpoint{1.254980in}{0.150000in}}{\pgfqpoint{5.490039in}{5.490039in}}%
\pgfusepath{clip}%
\pgfsetbuttcap%
\pgfsetroundjoin%
\definecolor{currentfill}{rgb}{0.282327,0.094955,0.417331}%
\pgfsetfillcolor{currentfill}%
\pgfsetfillopacity{0.700000}%
\pgfsetlinewidth{0.000000pt}%
\definecolor{currentstroke}{rgb}{0.000000,0.000000,0.000000}%
\pgfsetstrokecolor{currentstroke}%
\pgfsetdash{}{0pt}%
\pgfpathmoveto{\pgfqpoint{3.772524in}{2.005451in}}%
\pgfpathlineto{\pgfqpoint{3.785693in}{2.005140in}}%
\pgfpathlineto{\pgfqpoint{3.798869in}{2.005003in}}%
\pgfpathlineto{\pgfqpoint{3.812053in}{2.005040in}}%
\pgfpathlineto{\pgfqpoint{3.825243in}{2.005250in}}%
\pgfpathlineto{\pgfqpoint{3.832984in}{2.015367in}}%
\pgfpathlineto{\pgfqpoint{3.840720in}{2.025469in}}%
\pgfpathlineto{\pgfqpoint{3.848450in}{2.035554in}}%
\pgfpathlineto{\pgfqpoint{3.856176in}{2.045623in}}%
\pgfpathlineto{\pgfqpoint{3.842994in}{2.045280in}}%
\pgfpathlineto{\pgfqpoint{3.829819in}{2.045110in}}%
\pgfpathlineto{\pgfqpoint{3.816651in}{2.045114in}}%
\pgfpathlineto{\pgfqpoint{3.803490in}{2.045292in}}%
\pgfpathlineto{\pgfqpoint{3.795756in}{2.035346in}}%
\pgfpathlineto{\pgfqpoint{3.788017in}{2.025390in}}%
\pgfpathlineto{\pgfqpoint{3.780273in}{2.015424in}}%
\pgfpathlineto{\pgfqpoint{3.772524in}{2.005451in}}%
\pgfpathclose%
\pgfusepath{fill}%
\end{pgfscope}%
\begin{pgfscope}%
\pgfpathrectangle{\pgfqpoint{1.254980in}{0.150000in}}{\pgfqpoint{5.490039in}{5.490039in}}%
\pgfusepath{clip}%
\pgfsetbuttcap%
\pgfsetroundjoin%
\definecolor{currentfill}{rgb}{0.280255,0.165693,0.476498}%
\pgfsetfillcolor{currentfill}%
\pgfsetfillopacity{0.700000}%
\pgfsetlinewidth{0.000000pt}%
\definecolor{currentstroke}{rgb}{0.000000,0.000000,0.000000}%
\pgfsetstrokecolor{currentstroke}%
\pgfsetdash{}{0pt}%
\pgfpathmoveto{\pgfqpoint{4.023425in}{2.135832in}}%
\pgfpathlineto{\pgfqpoint{4.036666in}{2.137892in}}%
\pgfpathlineto{\pgfqpoint{4.049916in}{2.140121in}}%
\pgfpathlineto{\pgfqpoint{4.063176in}{2.142518in}}%
\pgfpathlineto{\pgfqpoint{4.076444in}{2.145083in}}%
\pgfpathlineto{\pgfqpoint{4.084104in}{2.155146in}}%
\pgfpathlineto{\pgfqpoint{4.091759in}{2.165165in}}%
\pgfpathlineto{\pgfqpoint{4.099409in}{2.175141in}}%
\pgfpathlineto{\pgfqpoint{4.107053in}{2.185074in}}%
\pgfpathlineto{\pgfqpoint{4.093791in}{2.182460in}}%
\pgfpathlineto{\pgfqpoint{4.080538in}{2.180014in}}%
\pgfpathlineto{\pgfqpoint{4.067295in}{2.177736in}}%
\pgfpathlineto{\pgfqpoint{4.054060in}{2.175627in}}%
\pgfpathlineto{\pgfqpoint{4.046409in}{2.165733in}}%
\pgfpathlineto{\pgfqpoint{4.038752in}{2.155802in}}%
\pgfpathlineto{\pgfqpoint{4.031091in}{2.145835in}}%
\pgfpathlineto{\pgfqpoint{4.023425in}{2.135832in}}%
\pgfpathclose%
\pgfusepath{fill}%
\end{pgfscope}%
\begin{pgfscope}%
\pgfpathrectangle{\pgfqpoint{1.254980in}{0.150000in}}{\pgfqpoint{5.490039in}{5.490039in}}%
\pgfusepath{clip}%
\pgfsetbuttcap%
\pgfsetroundjoin%
\definecolor{currentfill}{rgb}{0.283187,0.125848,0.444960}%
\pgfsetfillcolor{currentfill}%
\pgfsetfillopacity{0.700000}%
\pgfsetlinewidth{0.000000pt}%
\definecolor{currentstroke}{rgb}{0.000000,0.000000,0.000000}%
\pgfsetstrokecolor{currentstroke}%
\pgfsetdash{}{0pt}%
\pgfpathmoveto{\pgfqpoint{2.868164in}{2.092959in}}%
\pgfpathlineto{\pgfqpoint{2.881355in}{2.080718in}}%
\pgfpathlineto{\pgfqpoint{2.894543in}{2.068705in}}%
\pgfpathlineto{\pgfqpoint{2.907727in}{2.056917in}}%
\pgfpathlineto{\pgfqpoint{2.920908in}{2.045354in}}%
\pgfpathlineto{\pgfqpoint{2.929047in}{2.050783in}}%
\pgfpathlineto{\pgfqpoint{2.937176in}{2.056346in}}%
\pgfpathlineto{\pgfqpoint{2.945295in}{2.062040in}}%
\pgfpathlineto{\pgfqpoint{2.953405in}{2.067861in}}%
\pgfpathlineto{\pgfqpoint{2.940250in}{2.079090in}}%
\pgfpathlineto{\pgfqpoint{2.927093in}{2.090542in}}%
\pgfpathlineto{\pgfqpoint{2.913932in}{2.102221in}}%
\pgfpathlineto{\pgfqpoint{2.900768in}{2.114126in}}%
\pgfpathlineto{\pgfqpoint{2.892632in}{2.108629in}}%
\pgfpathlineto{\pgfqpoint{2.884486in}{2.103267in}}%
\pgfpathlineto{\pgfqpoint{2.876330in}{2.098043in}}%
\pgfpathlineto{\pgfqpoint{2.868164in}{2.092959in}}%
\pgfpathclose%
\pgfusepath{fill}%
\end{pgfscope}%
\begin{pgfscope}%
\pgfpathrectangle{\pgfqpoint{1.254980in}{0.150000in}}{\pgfqpoint{5.490039in}{5.490039in}}%
\pgfusepath{clip}%
\pgfsetbuttcap%
\pgfsetroundjoin%
\definecolor{currentfill}{rgb}{0.129933,0.559582,0.551864}%
\pgfsetfillcolor{currentfill}%
\pgfsetfillopacity{0.700000}%
\pgfsetlinewidth{0.000000pt}%
\definecolor{currentstroke}{rgb}{0.000000,0.000000,0.000000}%
\pgfsetstrokecolor{currentstroke}%
\pgfsetdash{}{0pt}%
\pgfpathmoveto{\pgfqpoint{5.422750in}{3.068690in}}%
\pgfpathlineto{\pgfqpoint{5.436643in}{3.077472in}}%
\pgfpathlineto{\pgfqpoint{5.450552in}{3.086409in}}%
\pgfpathlineto{\pgfqpoint{5.464478in}{3.095500in}}%
\pgfpathlineto{\pgfqpoint{5.478421in}{3.104745in}}%
\pgfpathlineto{\pgfqpoint{5.485465in}{3.108516in}}%
\pgfpathlineto{\pgfqpoint{5.492503in}{3.112313in}}%
\pgfpathlineto{\pgfqpoint{5.499535in}{3.116141in}}%
\pgfpathlineto{\pgfqpoint{5.506563in}{3.120005in}}%
\pgfpathlineto{\pgfqpoint{5.492643in}{3.111236in}}%
\pgfpathlineto{\pgfqpoint{5.478741in}{3.102621in}}%
\pgfpathlineto{\pgfqpoint{5.464855in}{3.094160in}}%
\pgfpathlineto{\pgfqpoint{5.450986in}{3.085852in}}%
\pgfpathlineto{\pgfqpoint{5.443935in}{3.081503in}}%
\pgfpathlineto{\pgfqpoint{5.436878in}{3.077196in}}%
\pgfpathlineto{\pgfqpoint{5.429817in}{3.072927in}}%
\pgfpathlineto{\pgfqpoint{5.422750in}{3.068690in}}%
\pgfpathclose%
\pgfusepath{fill}%
\end{pgfscope}%
\begin{pgfscope}%
\pgfpathrectangle{\pgfqpoint{1.254980in}{0.150000in}}{\pgfqpoint{5.490039in}{5.490039in}}%
\pgfusepath{clip}%
\pgfsetbuttcap%
\pgfsetroundjoin%
\definecolor{currentfill}{rgb}{0.276194,0.190074,0.493001}%
\pgfsetfillcolor{currentfill}%
\pgfsetfillopacity{0.700000}%
\pgfsetlinewidth{0.000000pt}%
\definecolor{currentstroke}{rgb}{0.000000,0.000000,0.000000}%
\pgfsetstrokecolor{currentstroke}%
\pgfsetdash{}{0pt}%
\pgfpathmoveto{\pgfqpoint{4.107053in}{2.185074in}}%
\pgfpathlineto{\pgfqpoint{4.120325in}{2.187855in}}%
\pgfpathlineto{\pgfqpoint{4.133606in}{2.190803in}}%
\pgfpathlineto{\pgfqpoint{4.146898in}{2.193918in}}%
\pgfpathlineto{\pgfqpoint{4.160199in}{2.197199in}}%
\pgfpathlineto{\pgfqpoint{4.167832in}{2.207119in}}%
\pgfpathlineto{\pgfqpoint{4.175460in}{2.216989in}}%
\pgfpathlineto{\pgfqpoint{4.183084in}{2.226808in}}%
\pgfpathlineto{\pgfqpoint{4.190701in}{2.236578in}}%
\pgfpathlineto{\pgfqpoint{4.177407in}{2.233276in}}%
\pgfpathlineto{\pgfqpoint{4.164122in}{2.230140in}}%
\pgfpathlineto{\pgfqpoint{4.150847in}{2.227172in}}%
\pgfpathlineto{\pgfqpoint{4.137581in}{2.224370in}}%
\pgfpathlineto{\pgfqpoint{4.129957in}{2.214611in}}%
\pgfpathlineto{\pgfqpoint{4.122327in}{2.204808in}}%
\pgfpathlineto{\pgfqpoint{4.114693in}{2.194963in}}%
\pgfpathlineto{\pgfqpoint{4.107053in}{2.185074in}}%
\pgfpathclose%
\pgfusepath{fill}%
\end{pgfscope}%
\begin{pgfscope}%
\pgfpathrectangle{\pgfqpoint{1.254980in}{0.150000in}}{\pgfqpoint{5.490039in}{5.490039in}}%
\pgfusepath{clip}%
\pgfsetbuttcap%
\pgfsetroundjoin%
\definecolor{currentfill}{rgb}{0.280894,0.078907,0.402329}%
\pgfsetfillcolor{currentfill}%
\pgfsetfillopacity{0.700000}%
\pgfsetlinewidth{0.000000pt}%
\definecolor{currentstroke}{rgb}{0.000000,0.000000,0.000000}%
\pgfsetstrokecolor{currentstroke}%
\pgfsetdash{}{0pt}%
\pgfpathmoveto{\pgfqpoint{3.688826in}{1.969131in}}%
\pgfpathlineto{\pgfqpoint{3.701979in}{1.967958in}}%
\pgfpathlineto{\pgfqpoint{3.715139in}{1.966962in}}%
\pgfpathlineto{\pgfqpoint{3.728304in}{1.966141in}}%
\pgfpathlineto{\pgfqpoint{3.741476in}{1.965495in}}%
\pgfpathlineto{\pgfqpoint{3.749245in}{1.975491in}}%
\pgfpathlineto{\pgfqpoint{3.757010in}{1.985483in}}%
\pgfpathlineto{\pgfqpoint{3.764769in}{1.995470in}}%
\pgfpathlineto{\pgfqpoint{3.772524in}{2.005451in}}%
\pgfpathlineto{\pgfqpoint{3.759361in}{2.005936in}}%
\pgfpathlineto{\pgfqpoint{3.746205in}{2.006596in}}%
\pgfpathlineto{\pgfqpoint{3.733055in}{2.007432in}}%
\pgfpathlineto{\pgfqpoint{3.719912in}{2.008444in}}%
\pgfpathlineto{\pgfqpoint{3.712148in}{1.998614in}}%
\pgfpathlineto{\pgfqpoint{3.704380in}{1.988784in}}%
\pgfpathlineto{\pgfqpoint{3.696606in}{1.978956in}}%
\pgfpathlineto{\pgfqpoint{3.688826in}{1.969131in}}%
\pgfpathclose%
\pgfusepath{fill}%
\end{pgfscope}%
\begin{pgfscope}%
\pgfpathrectangle{\pgfqpoint{1.254980in}{0.150000in}}{\pgfqpoint{5.490039in}{5.490039in}}%
\pgfusepath{clip}%
\pgfsetbuttcap%
\pgfsetroundjoin%
\definecolor{currentfill}{rgb}{0.273809,0.031497,0.358853}%
\pgfsetfillcolor{currentfill}%
\pgfsetfillopacity{0.700000}%
\pgfsetlinewidth{0.000000pt}%
\definecolor{currentstroke}{rgb}{0.000000,0.000000,0.000000}%
\pgfsetstrokecolor{currentstroke}%
\pgfsetdash{}{0pt}%
\pgfpathmoveto{\pgfqpoint{3.384723in}{1.905720in}}%
\pgfpathlineto{\pgfqpoint{3.397841in}{1.900985in}}%
\pgfpathlineto{\pgfqpoint{3.410962in}{1.896438in}}%
\pgfpathlineto{\pgfqpoint{3.424086in}{1.892078in}}%
\pgfpathlineto{\pgfqpoint{3.437213in}{1.887905in}}%
\pgfpathlineto{\pgfqpoint{3.445096in}{1.896799in}}%
\pgfpathlineto{\pgfqpoint{3.452973in}{1.905739in}}%
\pgfpathlineto{\pgfqpoint{3.460843in}{1.914722in}}%
\pgfpathlineto{\pgfqpoint{3.468708in}{1.923746in}}%
\pgfpathlineto{\pgfqpoint{3.455595in}{1.927676in}}%
\pgfpathlineto{\pgfqpoint{3.442485in}{1.931792in}}%
\pgfpathlineto{\pgfqpoint{3.429379in}{1.936095in}}%
\pgfpathlineto{\pgfqpoint{3.416276in}{1.940585in}}%
\pgfpathlineto{\pgfqpoint{3.408397in}{1.931795in}}%
\pgfpathlineto{\pgfqpoint{3.400512in}{1.923053in}}%
\pgfpathlineto{\pgfqpoint{3.392620in}{1.914360in}}%
\pgfpathlineto{\pgfqpoint{3.384723in}{1.905720in}}%
\pgfpathclose%
\pgfusepath{fill}%
\end{pgfscope}%
\begin{pgfscope}%
\pgfpathrectangle{\pgfqpoint{1.254980in}{0.150000in}}{\pgfqpoint{5.490039in}{5.490039in}}%
\pgfusepath{clip}%
\pgfsetbuttcap%
\pgfsetroundjoin%
\definecolor{currentfill}{rgb}{0.277941,0.056324,0.381191}%
\pgfsetfillcolor{currentfill}%
\pgfsetfillopacity{0.700000}%
\pgfsetlinewidth{0.000000pt}%
\definecolor{currentstroke}{rgb}{0.000000,0.000000,0.000000}%
\pgfsetstrokecolor{currentstroke}%
\pgfsetdash{}{0pt}%
\pgfpathmoveto{\pgfqpoint{3.111097in}{1.950037in}}%
\pgfpathlineto{\pgfqpoint{3.124231in}{1.941588in}}%
\pgfpathlineto{\pgfqpoint{3.137364in}{1.933342in}}%
\pgfpathlineto{\pgfqpoint{3.150497in}{1.925301in}}%
\pgfpathlineto{\pgfqpoint{3.163631in}{1.917461in}}%
\pgfpathlineto{\pgfqpoint{3.171639in}{1.924662in}}%
\pgfpathlineto{\pgfqpoint{3.179640in}{1.931957in}}%
\pgfpathlineto{\pgfqpoint{3.187632in}{1.939343in}}%
\pgfpathlineto{\pgfqpoint{3.195617in}{1.946816in}}%
\pgfpathlineto{\pgfqpoint{3.182504in}{1.954354in}}%
\pgfpathlineto{\pgfqpoint{3.169392in}{1.962094in}}%
\pgfpathlineto{\pgfqpoint{3.156279in}{1.970037in}}%
\pgfpathlineto{\pgfqpoint{3.143167in}{1.978184in}}%
\pgfpathlineto{\pgfqpoint{3.135162in}{1.971003in}}%
\pgfpathlineto{\pgfqpoint{3.127148in}{1.963916in}}%
\pgfpathlineto{\pgfqpoint{3.119127in}{1.956926in}}%
\pgfpathlineto{\pgfqpoint{3.111097in}{1.950037in}}%
\pgfpathclose%
\pgfusepath{fill}%
\end{pgfscope}%
\begin{pgfscope}%
\pgfpathrectangle{\pgfqpoint{1.254980in}{0.150000in}}{\pgfqpoint{5.490039in}{5.490039in}}%
\pgfusepath{clip}%
\pgfsetbuttcap%
\pgfsetroundjoin%
\definecolor{currentfill}{rgb}{0.201239,0.383670,0.554294}%
\pgfsetfillcolor{currentfill}%
\pgfsetfillopacity{0.700000}%
\pgfsetlinewidth{0.000000pt}%
\definecolor{currentstroke}{rgb}{0.000000,0.000000,0.000000}%
\pgfsetstrokecolor{currentstroke}%
\pgfsetdash{}{0pt}%
\pgfpathmoveto{\pgfqpoint{4.723108in}{2.605865in}}%
\pgfpathlineto{\pgfqpoint{4.736649in}{2.612577in}}%
\pgfpathlineto{\pgfqpoint{4.750204in}{2.619449in}}%
\pgfpathlineto{\pgfqpoint{4.763773in}{2.626480in}}%
\pgfpathlineto{\pgfqpoint{4.777356in}{2.633671in}}%
\pgfpathlineto{\pgfqpoint{4.784756in}{2.641052in}}%
\pgfpathlineto{\pgfqpoint{4.792149in}{2.648375in}}%
\pgfpathlineto{\pgfqpoint{4.799536in}{2.655642in}}%
\pgfpathlineto{\pgfqpoint{4.806917in}{2.662857in}}%
\pgfpathlineto{\pgfqpoint{4.793345in}{2.655877in}}%
\pgfpathlineto{\pgfqpoint{4.779786in}{2.649055in}}%
\pgfpathlineto{\pgfqpoint{4.766242in}{2.642393in}}%
\pgfpathlineto{\pgfqpoint{4.752710in}{2.635891in}}%
\pgfpathlineto{\pgfqpoint{4.745319in}{2.628455in}}%
\pgfpathlineto{\pgfqpoint{4.737921in}{2.620974in}}%
\pgfpathlineto{\pgfqpoint{4.730518in}{2.613445in}}%
\pgfpathlineto{\pgfqpoint{4.723108in}{2.605865in}}%
\pgfpathclose%
\pgfusepath{fill}%
\end{pgfscope}%
\begin{pgfscope}%
\pgfpathrectangle{\pgfqpoint{1.254980in}{0.150000in}}{\pgfqpoint{5.490039in}{5.490039in}}%
\pgfusepath{clip}%
\pgfsetbuttcap%
\pgfsetroundjoin%
\definecolor{currentfill}{rgb}{0.124395,0.578002,0.548287}%
\pgfsetfillcolor{currentfill}%
\pgfsetfillopacity{0.700000}%
\pgfsetlinewidth{0.000000pt}%
\definecolor{currentstroke}{rgb}{0.000000,0.000000,0.000000}%
\pgfsetstrokecolor{currentstroke}%
\pgfsetdash{}{0pt}%
\pgfpathmoveto{\pgfqpoint{5.506563in}{3.120005in}}%
\pgfpathlineto{\pgfqpoint{5.520499in}{3.128928in}}%
\pgfpathlineto{\pgfqpoint{5.534453in}{3.138004in}}%
\pgfpathlineto{\pgfqpoint{5.548424in}{3.147234in}}%
\pgfpathlineto{\pgfqpoint{5.562412in}{3.156619in}}%
\pgfpathlineto{\pgfqpoint{5.569410in}{3.160029in}}%
\pgfpathlineto{\pgfqpoint{5.576402in}{3.163481in}}%
\pgfpathlineto{\pgfqpoint{5.583389in}{3.166978in}}%
\pgfpathlineto{\pgfqpoint{5.590372in}{3.170527in}}%
\pgfpathlineto{\pgfqpoint{5.576410in}{3.161649in}}%
\pgfpathlineto{\pgfqpoint{5.562465in}{3.152925in}}%
\pgfpathlineto{\pgfqpoint{5.548537in}{3.144353in}}%
\pgfpathlineto{\pgfqpoint{5.534625in}{3.135935in}}%
\pgfpathlineto{\pgfqpoint{5.527616in}{3.131870in}}%
\pgfpathlineto{\pgfqpoint{5.520603in}{3.127865in}}%
\pgfpathlineto{\pgfqpoint{5.513585in}{3.123911in}}%
\pgfpathlineto{\pgfqpoint{5.506563in}{3.120005in}}%
\pgfpathclose%
\pgfusepath{fill}%
\end{pgfscope}%
\begin{pgfscope}%
\pgfpathrectangle{\pgfqpoint{1.254980in}{0.150000in}}{\pgfqpoint{5.490039in}{5.490039in}}%
\pgfusepath{clip}%
\pgfsetbuttcap%
\pgfsetroundjoin%
\definecolor{currentfill}{rgb}{0.212395,0.359683,0.551710}%
\pgfsetfillcolor{currentfill}%
\pgfsetfillopacity{0.700000}%
\pgfsetlinewidth{0.000000pt}%
\definecolor{currentstroke}{rgb}{0.000000,0.000000,0.000000}%
\pgfsetstrokecolor{currentstroke}%
\pgfsetdash{}{0pt}%
\pgfpathmoveto{\pgfqpoint{2.442655in}{2.616181in}}%
\pgfpathlineto{\pgfqpoint{2.456094in}{2.595602in}}%
\pgfpathlineto{\pgfqpoint{2.469521in}{2.575321in}}%
\pgfpathlineto{\pgfqpoint{2.482937in}{2.555336in}}%
\pgfpathlineto{\pgfqpoint{2.496341in}{2.535643in}}%
\pgfpathlineto{\pgfqpoint{2.504738in}{2.538250in}}%
\pgfpathlineto{\pgfqpoint{2.513121in}{2.541049in}}%
\pgfpathlineto{\pgfqpoint{2.521490in}{2.544037in}}%
\pgfpathlineto{\pgfqpoint{2.529846in}{2.547212in}}%
\pgfpathlineto{\pgfqpoint{2.516480in}{2.566552in}}%
\pgfpathlineto{\pgfqpoint{2.503102in}{2.586185in}}%
\pgfpathlineto{\pgfqpoint{2.489713in}{2.606112in}}%
\pgfpathlineto{\pgfqpoint{2.476312in}{2.626336in}}%
\pgfpathlineto{\pgfqpoint{2.467919in}{2.623503in}}%
\pgfpathlineto{\pgfqpoint{2.459512in}{2.620864in}}%
\pgfpathlineto{\pgfqpoint{2.451090in}{2.618422in}}%
\pgfpathlineto{\pgfqpoint{2.442655in}{2.616181in}}%
\pgfpathclose%
\pgfusepath{fill}%
\end{pgfscope}%
\begin{pgfscope}%
\pgfpathrectangle{\pgfqpoint{1.254980in}{0.150000in}}{\pgfqpoint{5.490039in}{5.490039in}}%
\pgfusepath{clip}%
\pgfsetbuttcap%
\pgfsetroundjoin%
\definecolor{currentfill}{rgb}{0.269308,0.218818,0.509577}%
\pgfsetfillcolor{currentfill}%
\pgfsetfillopacity{0.700000}%
\pgfsetlinewidth{0.000000pt}%
\definecolor{currentstroke}{rgb}{0.000000,0.000000,0.000000}%
\pgfsetstrokecolor{currentstroke}%
\pgfsetdash{}{0pt}%
\pgfpathmoveto{\pgfqpoint{4.190701in}{2.236578in}}%
\pgfpathlineto{\pgfqpoint{4.204007in}{2.240045in}}%
\pgfpathlineto{\pgfqpoint{4.217322in}{2.243679in}}%
\pgfpathlineto{\pgfqpoint{4.230648in}{2.247478in}}%
\pgfpathlineto{\pgfqpoint{4.243985in}{2.251442in}}%
\pgfpathlineto{\pgfqpoint{4.251591in}{2.261165in}}%
\pgfpathlineto{\pgfqpoint{4.259192in}{2.270831in}}%
\pgfpathlineto{\pgfqpoint{4.266788in}{2.280441in}}%
\pgfpathlineto{\pgfqpoint{4.274379in}{2.289996in}}%
\pgfpathlineto{\pgfqpoint{4.261049in}{2.286040in}}%
\pgfpathlineto{\pgfqpoint{4.247729in}{2.282249in}}%
\pgfpathlineto{\pgfqpoint{4.234420in}{2.278623in}}%
\pgfpathlineto{\pgfqpoint{4.221122in}{2.275163in}}%
\pgfpathlineto{\pgfqpoint{4.213524in}{2.265590in}}%
\pgfpathlineto{\pgfqpoint{4.205922in}{2.255968in}}%
\pgfpathlineto{\pgfqpoint{4.198314in}{2.246297in}}%
\pgfpathlineto{\pgfqpoint{4.190701in}{2.236578in}}%
\pgfpathclose%
\pgfusepath{fill}%
\end{pgfscope}%
\begin{pgfscope}%
\pgfpathrectangle{\pgfqpoint{1.254980in}{0.150000in}}{\pgfqpoint{5.490039in}{5.490039in}}%
\pgfusepath{clip}%
\pgfsetbuttcap%
\pgfsetroundjoin%
\definecolor{currentfill}{rgb}{0.120565,0.596422,0.543611}%
\pgfsetfillcolor{currentfill}%
\pgfsetfillopacity{0.700000}%
\pgfsetlinewidth{0.000000pt}%
\definecolor{currentstroke}{rgb}{0.000000,0.000000,0.000000}%
\pgfsetstrokecolor{currentstroke}%
\pgfsetdash{}{0pt}%
\pgfpathmoveto{\pgfqpoint{5.590372in}{3.170527in}}%
\pgfpathlineto{\pgfqpoint{5.604352in}{3.179558in}}%
\pgfpathlineto{\pgfqpoint{5.618350in}{3.188743in}}%
\pgfpathlineto{\pgfqpoint{5.632365in}{3.198080in}}%
\pgfpathlineto{\pgfqpoint{5.646398in}{3.207572in}}%
\pgfpathlineto{\pgfqpoint{5.653349in}{3.210653in}}%
\pgfpathlineto{\pgfqpoint{5.660295in}{3.213791in}}%
\pgfpathlineto{\pgfqpoint{5.667237in}{3.216991in}}%
\pgfpathlineto{\pgfqpoint{5.674176in}{3.220259in}}%
\pgfpathlineto{\pgfqpoint{5.660171in}{3.211304in}}%
\pgfpathlineto{\pgfqpoint{5.646184in}{3.202502in}}%
\pgfpathlineto{\pgfqpoint{5.632214in}{3.193852in}}%
\pgfpathlineto{\pgfqpoint{5.618262in}{3.185354in}}%
\pgfpathlineto{\pgfqpoint{5.611295in}{3.181541in}}%
\pgfpathlineto{\pgfqpoint{5.604325in}{3.177803in}}%
\pgfpathlineto{\pgfqpoint{5.597351in}{3.174133in}}%
\pgfpathlineto{\pgfqpoint{5.590372in}{3.170527in}}%
\pgfpathclose%
\pgfusepath{fill}%
\end{pgfscope}%
\begin{pgfscope}%
\pgfpathrectangle{\pgfqpoint{1.254980in}{0.150000in}}{\pgfqpoint{5.490039in}{5.490039in}}%
\pgfusepath{clip}%
\pgfsetbuttcap%
\pgfsetroundjoin%
\definecolor{currentfill}{rgb}{0.282910,0.105393,0.426902}%
\pgfsetfillcolor{currentfill}%
\pgfsetfillopacity{0.700000}%
\pgfsetlinewidth{0.000000pt}%
\definecolor{currentstroke}{rgb}{0.000000,0.000000,0.000000}%
\pgfsetstrokecolor{currentstroke}%
\pgfsetdash{}{0pt}%
\pgfpathmoveto{\pgfqpoint{2.920908in}{2.045354in}}%
\pgfpathlineto{\pgfqpoint{2.934086in}{2.034013in}}%
\pgfpathlineto{\pgfqpoint{2.947261in}{2.022894in}}%
\pgfpathlineto{\pgfqpoint{2.960433in}{2.011994in}}%
\pgfpathlineto{\pgfqpoint{2.973603in}{2.001312in}}%
\pgfpathlineto{\pgfqpoint{2.981715in}{2.007085in}}%
\pgfpathlineto{\pgfqpoint{2.989818in}{2.012985in}}%
\pgfpathlineto{\pgfqpoint{2.997912in}{2.019009in}}%
\pgfpathlineto{\pgfqpoint{3.005997in}{2.025153in}}%
\pgfpathlineto{\pgfqpoint{2.992852in}{2.035501in}}%
\pgfpathlineto{\pgfqpoint{2.979705in}{2.046068in}}%
\pgfpathlineto{\pgfqpoint{2.966556in}{2.056854in}}%
\pgfpathlineto{\pgfqpoint{2.953405in}{2.067861in}}%
\pgfpathlineto{\pgfqpoint{2.945295in}{2.062040in}}%
\pgfpathlineto{\pgfqpoint{2.937176in}{2.056346in}}%
\pgfpathlineto{\pgfqpoint{2.929047in}{2.050783in}}%
\pgfpathlineto{\pgfqpoint{2.920908in}{2.045354in}}%
\pgfpathclose%
\pgfusepath{fill}%
\end{pgfscope}%
\begin{pgfscope}%
\pgfpathrectangle{\pgfqpoint{1.254980in}{0.150000in}}{\pgfqpoint{5.490039in}{5.490039in}}%
\pgfusepath{clip}%
\pgfsetbuttcap%
\pgfsetroundjoin%
\definecolor{currentfill}{rgb}{0.278791,0.062145,0.386592}%
\pgfsetfillcolor{currentfill}%
\pgfsetfillopacity{0.700000}%
\pgfsetlinewidth{0.000000pt}%
\definecolor{currentstroke}{rgb}{0.000000,0.000000,0.000000}%
\pgfsetstrokecolor{currentstroke}%
\pgfsetdash{}{0pt}%
\pgfpathmoveto{\pgfqpoint{3.605061in}{1.937114in}}%
\pgfpathlineto{\pgfqpoint{3.618202in}{1.935042in}}%
\pgfpathlineto{\pgfqpoint{3.631348in}{1.933148in}}%
\pgfpathlineto{\pgfqpoint{3.644500in}{1.931433in}}%
\pgfpathlineto{\pgfqpoint{3.657657in}{1.929895in}}%
\pgfpathlineto{\pgfqpoint{3.665457in}{1.939692in}}%
\pgfpathlineto{\pgfqpoint{3.673252in}{1.949498in}}%
\pgfpathlineto{\pgfqpoint{3.681042in}{1.959311in}}%
\pgfpathlineto{\pgfqpoint{3.688826in}{1.969131in}}%
\pgfpathlineto{\pgfqpoint{3.675679in}{1.970481in}}%
\pgfpathlineto{\pgfqpoint{3.662538in}{1.972008in}}%
\pgfpathlineto{\pgfqpoint{3.649403in}{1.973714in}}%
\pgfpathlineto{\pgfqpoint{3.636273in}{1.975598in}}%
\pgfpathlineto{\pgfqpoint{3.628478in}{1.965956in}}%
\pgfpathlineto{\pgfqpoint{3.620677in}{1.956327in}}%
\pgfpathlineto{\pgfqpoint{3.612872in}{1.946713in}}%
\pgfpathlineto{\pgfqpoint{3.605061in}{1.937114in}}%
\pgfpathclose%
\pgfusepath{fill}%
\end{pgfscope}%
\begin{pgfscope}%
\pgfpathrectangle{\pgfqpoint{1.254980in}{0.150000in}}{\pgfqpoint{5.490039in}{5.490039in}}%
\pgfusepath{clip}%
\pgfsetbuttcap%
\pgfsetroundjoin%
\definecolor{currentfill}{rgb}{0.119483,0.614817,0.537692}%
\pgfsetfillcolor{currentfill}%
\pgfsetfillopacity{0.700000}%
\pgfsetlinewidth{0.000000pt}%
\definecolor{currentstroke}{rgb}{0.000000,0.000000,0.000000}%
\pgfsetstrokecolor{currentstroke}%
\pgfsetdash{}{0pt}%
\pgfpathmoveto{\pgfqpoint{5.674176in}{3.220259in}}%
\pgfpathlineto{\pgfqpoint{5.688198in}{3.229367in}}%
\pgfpathlineto{\pgfqpoint{5.702238in}{3.238628in}}%
\pgfpathlineto{\pgfqpoint{5.716296in}{3.248042in}}%
\pgfpathlineto{\pgfqpoint{5.730373in}{3.257608in}}%
\pgfpathlineto{\pgfqpoint{5.737277in}{3.260396in}}%
\pgfpathlineto{\pgfqpoint{5.744178in}{3.263258in}}%
\pgfpathlineto{\pgfqpoint{5.751076in}{3.266199in}}%
\pgfpathlineto{\pgfqpoint{5.757970in}{3.269226in}}%
\pgfpathlineto{\pgfqpoint{5.743924in}{3.260225in}}%
\pgfpathlineto{\pgfqpoint{5.729896in}{3.251377in}}%
\pgfpathlineto{\pgfqpoint{5.715885in}{3.242680in}}%
\pgfpathlineto{\pgfqpoint{5.701892in}{3.234135in}}%
\pgfpathlineto{\pgfqpoint{5.694968in}{3.230534in}}%
\pgfpathlineto{\pgfqpoint{5.688041in}{3.227025in}}%
\pgfpathlineto{\pgfqpoint{5.681110in}{3.223602in}}%
\pgfpathlineto{\pgfqpoint{5.674176in}{3.220259in}}%
\pgfpathclose%
\pgfusepath{fill}%
\end{pgfscope}%
\begin{pgfscope}%
\pgfpathrectangle{\pgfqpoint{1.254980in}{0.150000in}}{\pgfqpoint{5.490039in}{5.490039in}}%
\pgfusepath{clip}%
\pgfsetbuttcap%
\pgfsetroundjoin%
\definecolor{currentfill}{rgb}{0.190631,0.407061,0.556089}%
\pgfsetfillcolor{currentfill}%
\pgfsetfillopacity{0.700000}%
\pgfsetlinewidth{0.000000pt}%
\definecolor{currentstroke}{rgb}{0.000000,0.000000,0.000000}%
\pgfsetstrokecolor{currentstroke}%
\pgfsetdash{}{0pt}%
\pgfpathmoveto{\pgfqpoint{4.806917in}{2.662857in}}%
\pgfpathlineto{\pgfqpoint{4.820504in}{2.669997in}}%
\pgfpathlineto{\pgfqpoint{4.834105in}{2.677296in}}%
\pgfpathlineto{\pgfqpoint{4.847720in}{2.684753in}}%
\pgfpathlineto{\pgfqpoint{4.861349in}{2.692370in}}%
\pgfpathlineto{\pgfqpoint{4.868713in}{2.699306in}}%
\pgfpathlineto{\pgfqpoint{4.876070in}{2.706188in}}%
\pgfpathlineto{\pgfqpoint{4.883421in}{2.713019in}}%
\pgfpathlineto{\pgfqpoint{4.890765in}{2.719802in}}%
\pgfpathlineto{\pgfqpoint{4.877147in}{2.712425in}}%
\pgfpathlineto{\pgfqpoint{4.863543in}{2.705207in}}%
\pgfpathlineto{\pgfqpoint{4.849954in}{2.698148in}}%
\pgfpathlineto{\pgfqpoint{4.836379in}{2.691247in}}%
\pgfpathlineto{\pgfqpoint{4.829023in}{2.684215in}}%
\pgfpathlineto{\pgfqpoint{4.821660in}{2.677141in}}%
\pgfpathlineto{\pgfqpoint{4.814292in}{2.670023in}}%
\pgfpathlineto{\pgfqpoint{4.806917in}{2.662857in}}%
\pgfpathclose%
\pgfusepath{fill}%
\end{pgfscope}%
\begin{pgfscope}%
\pgfpathrectangle{\pgfqpoint{1.254980in}{0.150000in}}{\pgfqpoint{5.490039in}{5.490039in}}%
\pgfusepath{clip}%
\pgfsetbuttcap%
\pgfsetroundjoin%
\definecolor{currentfill}{rgb}{0.260571,0.246922,0.522828}%
\pgfsetfillcolor{currentfill}%
\pgfsetfillopacity{0.700000}%
\pgfsetlinewidth{0.000000pt}%
\definecolor{currentstroke}{rgb}{0.000000,0.000000,0.000000}%
\pgfsetstrokecolor{currentstroke}%
\pgfsetdash{}{0pt}%
\pgfpathmoveto{\pgfqpoint{4.274379in}{2.289996in}}%
\pgfpathlineto{\pgfqpoint{4.287720in}{2.294117in}}%
\pgfpathlineto{\pgfqpoint{4.301072in}{2.298403in}}%
\pgfpathlineto{\pgfqpoint{4.314436in}{2.302853in}}%
\pgfpathlineto{\pgfqpoint{4.327810in}{2.307466in}}%
\pgfpathlineto{\pgfqpoint{4.335389in}{2.316941in}}%
\pgfpathlineto{\pgfqpoint{4.342963in}{2.326354in}}%
\pgfpathlineto{\pgfqpoint{4.350531in}{2.335708in}}%
\pgfpathlineto{\pgfqpoint{4.358094in}{2.345001in}}%
\pgfpathlineto{\pgfqpoint{4.344726in}{2.340424in}}%
\pgfpathlineto{\pgfqpoint{4.331370in}{2.336011in}}%
\pgfpathlineto{\pgfqpoint{4.318024in}{2.331761in}}%
\pgfpathlineto{\pgfqpoint{4.304690in}{2.327677in}}%
\pgfpathlineto{\pgfqpoint{4.297120in}{2.318336in}}%
\pgfpathlineto{\pgfqpoint{4.289545in}{2.308943in}}%
\pgfpathlineto{\pgfqpoint{4.281965in}{2.299497in}}%
\pgfpathlineto{\pgfqpoint{4.274379in}{2.289996in}}%
\pgfpathclose%
\pgfusepath{fill}%
\end{pgfscope}%
\begin{pgfscope}%
\pgfpathrectangle{\pgfqpoint{1.254980in}{0.150000in}}{\pgfqpoint{5.490039in}{5.490039in}}%
\pgfusepath{clip}%
\pgfsetbuttcap%
\pgfsetroundjoin%
\definecolor{currentfill}{rgb}{0.122312,0.633153,0.530398}%
\pgfsetfillcolor{currentfill}%
\pgfsetfillopacity{0.700000}%
\pgfsetlinewidth{0.000000pt}%
\definecolor{currentstroke}{rgb}{0.000000,0.000000,0.000000}%
\pgfsetstrokecolor{currentstroke}%
\pgfsetdash{}{0pt}%
\pgfpathmoveto{\pgfqpoint{5.757970in}{3.269226in}}%
\pgfpathlineto{\pgfqpoint{5.772034in}{3.278379in}}%
\pgfpathlineto{\pgfqpoint{5.786115in}{3.287685in}}%
\pgfpathlineto{\pgfqpoint{5.800216in}{3.297142in}}%
\pgfpathlineto{\pgfqpoint{5.814334in}{3.306753in}}%
\pgfpathlineto{\pgfqpoint{5.821193in}{3.309288in}}%
\pgfpathlineto{\pgfqpoint{5.828049in}{3.311915in}}%
\pgfpathlineto{\pgfqpoint{5.834902in}{3.314641in}}%
\pgfpathlineto{\pgfqpoint{5.841752in}{3.317473in}}%
\pgfpathlineto{\pgfqpoint{5.827666in}{3.308458in}}%
\pgfpathlineto{\pgfqpoint{5.813598in}{3.299594in}}%
\pgfpathlineto{\pgfqpoint{5.799549in}{3.290882in}}%
\pgfpathlineto{\pgfqpoint{5.785516in}{3.282322in}}%
\pgfpathlineto{\pgfqpoint{5.778634in}{3.278887in}}%
\pgfpathlineto{\pgfqpoint{5.771748in}{3.275564in}}%
\pgfpathlineto{\pgfqpoint{5.764860in}{3.272346in}}%
\pgfpathlineto{\pgfqpoint{5.757970in}{3.269226in}}%
\pgfpathclose%
\pgfusepath{fill}%
\end{pgfscope}%
\begin{pgfscope}%
\pgfpathrectangle{\pgfqpoint{1.254980in}{0.150000in}}{\pgfqpoint{5.490039in}{5.490039in}}%
\pgfusepath{clip}%
\pgfsetbuttcap%
\pgfsetroundjoin%
\definecolor{currentfill}{rgb}{0.130067,0.651384,0.521608}%
\pgfsetfillcolor{currentfill}%
\pgfsetfillopacity{0.700000}%
\pgfsetlinewidth{0.000000pt}%
\definecolor{currentstroke}{rgb}{0.000000,0.000000,0.000000}%
\pgfsetstrokecolor{currentstroke}%
\pgfsetdash{}{0pt}%
\pgfpathmoveto{\pgfqpoint{5.841752in}{3.317473in}}%
\pgfpathlineto{\pgfqpoint{5.855856in}{3.326639in}}%
\pgfpathlineto{\pgfqpoint{5.869979in}{3.335957in}}%
\pgfpathlineto{\pgfqpoint{5.884120in}{3.345428in}}%
\pgfpathlineto{\pgfqpoint{5.898279in}{3.355050in}}%
\pgfpathlineto{\pgfqpoint{5.905093in}{3.357379in}}%
\pgfpathlineto{\pgfqpoint{5.911905in}{3.359820in}}%
\pgfpathlineto{\pgfqpoint{5.918714in}{3.362379in}}%
\pgfpathlineto{\pgfqpoint{5.925522in}{3.365064in}}%
\pgfpathlineto{\pgfqpoint{5.911398in}{3.356067in}}%
\pgfpathlineto{\pgfqpoint{5.897291in}{3.347220in}}%
\pgfpathlineto{\pgfqpoint{5.883203in}{3.338524in}}%
\pgfpathlineto{\pgfqpoint{5.869133in}{3.329979in}}%
\pgfpathlineto{\pgfqpoint{5.862290in}{3.326662in}}%
\pgfpathlineto{\pgfqpoint{5.855446in}{3.323476in}}%
\pgfpathlineto{\pgfqpoint{5.848600in}{3.320415in}}%
\pgfpathlineto{\pgfqpoint{5.841752in}{3.317473in}}%
\pgfpathclose%
\pgfusepath{fill}%
\end{pgfscope}%
\begin{pgfscope}%
\pgfpathrectangle{\pgfqpoint{1.254980in}{0.150000in}}{\pgfqpoint{5.490039in}{5.490039in}}%
\pgfusepath{clip}%
\pgfsetbuttcap%
\pgfsetroundjoin%
\definecolor{currentfill}{rgb}{0.143303,0.669459,0.511215}%
\pgfsetfillcolor{currentfill}%
\pgfsetfillopacity{0.700000}%
\pgfsetlinewidth{0.000000pt}%
\definecolor{currentstroke}{rgb}{0.000000,0.000000,0.000000}%
\pgfsetstrokecolor{currentstroke}%
\pgfsetdash{}{0pt}%
\pgfpathmoveto{\pgfqpoint{5.925522in}{3.365064in}}%
\pgfpathlineto{\pgfqpoint{5.939665in}{3.374213in}}%
\pgfpathlineto{\pgfqpoint{5.953827in}{3.383513in}}%
\pgfpathlineto{\pgfqpoint{5.968007in}{3.392964in}}%
\pgfpathlineto{\pgfqpoint{5.982206in}{3.402567in}}%
\pgfpathlineto{\pgfqpoint{5.988976in}{3.404741in}}%
\pgfpathlineto{\pgfqpoint{5.995745in}{3.407047in}}%
\pgfpathlineto{\pgfqpoint{6.002513in}{3.409494in}}%
\pgfpathlineto{\pgfqpoint{6.009280in}{3.412088in}}%
\pgfpathlineto{\pgfqpoint{5.995118in}{3.403139in}}%
\pgfpathlineto{\pgfqpoint{5.980974in}{3.394340in}}%
\pgfpathlineto{\pgfqpoint{5.966849in}{3.385692in}}%
\pgfpathlineto{\pgfqpoint{5.952743in}{3.377194in}}%
\pgfpathlineto{\pgfqpoint{5.945939in}{3.373939in}}%
\pgfpathlineto{\pgfqpoint{5.939134in}{3.370837in}}%
\pgfpathlineto{\pgfqpoint{5.932329in}{3.367881in}}%
\pgfpathlineto{\pgfqpoint{5.925522in}{3.365064in}}%
\pgfpathclose%
\pgfusepath{fill}%
\end{pgfscope}%
\begin{pgfscope}%
\pgfpathrectangle{\pgfqpoint{1.254980in}{0.150000in}}{\pgfqpoint{5.490039in}{5.490039in}}%
\pgfusepath{clip}%
\pgfsetbuttcap%
\pgfsetroundjoin%
\definecolor{currentfill}{rgb}{0.276022,0.044167,0.370164}%
\pgfsetfillcolor{currentfill}%
\pgfsetfillopacity{0.700000}%
\pgfsetlinewidth{0.000000pt}%
\definecolor{currentstroke}{rgb}{0.000000,0.000000,0.000000}%
\pgfsetstrokecolor{currentstroke}%
\pgfsetdash{}{0pt}%
\pgfpathmoveto{\pgfqpoint{3.521200in}{1.909874in}}%
\pgfpathlineto{\pgfqpoint{3.534333in}{1.906863in}}%
\pgfpathlineto{\pgfqpoint{3.547471in}{1.904034in}}%
\pgfpathlineto{\pgfqpoint{3.560614in}{1.901386in}}%
\pgfpathlineto{\pgfqpoint{3.573761in}{1.898918in}}%
\pgfpathlineto{\pgfqpoint{3.581594in}{1.908434in}}%
\pgfpathlineto{\pgfqpoint{3.589422in}{1.917973in}}%
\pgfpathlineto{\pgfqpoint{3.597244in}{1.927534in}}%
\pgfpathlineto{\pgfqpoint{3.605061in}{1.937114in}}%
\pgfpathlineto{\pgfqpoint{3.591925in}{1.939366in}}%
\pgfpathlineto{\pgfqpoint{3.578794in}{1.941799in}}%
\pgfpathlineto{\pgfqpoint{3.565668in}{1.944411in}}%
\pgfpathlineto{\pgfqpoint{3.552547in}{1.947206in}}%
\pgfpathlineto{\pgfqpoint{3.544719in}{1.937831in}}%
\pgfpathlineto{\pgfqpoint{3.536885in}{1.928483in}}%
\pgfpathlineto{\pgfqpoint{3.529045in}{1.919163in}}%
\pgfpathlineto{\pgfqpoint{3.521200in}{1.909874in}}%
\pgfpathclose%
\pgfusepath{fill}%
\end{pgfscope}%
\begin{pgfscope}%
\pgfpathrectangle{\pgfqpoint{1.254980in}{0.150000in}}{\pgfqpoint{5.490039in}{5.490039in}}%
\pgfusepath{clip}%
\pgfsetbuttcap%
\pgfsetroundjoin%
\definecolor{currentfill}{rgb}{0.162016,0.687316,0.499129}%
\pgfsetfillcolor{currentfill}%
\pgfsetfillopacity{0.700000}%
\pgfsetlinewidth{0.000000pt}%
\definecolor{currentstroke}{rgb}{0.000000,0.000000,0.000000}%
\pgfsetstrokecolor{currentstroke}%
\pgfsetdash{}{0pt}%
\pgfpathmoveto{\pgfqpoint{6.009280in}{3.412088in}}%
\pgfpathlineto{\pgfqpoint{6.023460in}{3.421187in}}%
\pgfpathlineto{\pgfqpoint{6.037659in}{3.430437in}}%
\pgfpathlineto{\pgfqpoint{6.051877in}{3.439838in}}%
\pgfpathlineto{\pgfqpoint{6.066115in}{3.449390in}}%
\pgfpathlineto{\pgfqpoint{6.072843in}{3.451465in}}%
\pgfpathlineto{\pgfqpoint{6.079570in}{3.453696in}}%
\pgfpathlineto{\pgfqpoint{6.086298in}{3.456089in}}%
\pgfpathlineto{\pgfqpoint{6.093026in}{3.458651in}}%
\pgfpathlineto{\pgfqpoint{6.078829in}{3.449782in}}%
\pgfpathlineto{\pgfqpoint{6.064650in}{3.441063in}}%
\pgfpathlineto{\pgfqpoint{6.050489in}{3.432493in}}%
\pgfpathlineto{\pgfqpoint{6.036347in}{3.424073in}}%
\pgfpathlineto{\pgfqpoint{6.029580in}{3.420821in}}%
\pgfpathlineto{\pgfqpoint{6.022813in}{3.417744in}}%
\pgfpathlineto{\pgfqpoint{6.016046in}{3.414835in}}%
\pgfpathlineto{\pgfqpoint{6.009280in}{3.412088in}}%
\pgfpathclose%
\pgfusepath{fill}%
\end{pgfscope}%
\begin{pgfscope}%
\pgfpathrectangle{\pgfqpoint{1.254980in}{0.150000in}}{\pgfqpoint{5.490039in}{5.490039in}}%
\pgfusepath{clip}%
\pgfsetbuttcap%
\pgfsetroundjoin%
\definecolor{currentfill}{rgb}{0.194100,0.399323,0.555565}%
\pgfsetfillcolor{currentfill}%
\pgfsetfillopacity{0.700000}%
\pgfsetlinewidth{0.000000pt}%
\definecolor{currentstroke}{rgb}{0.000000,0.000000,0.000000}%
\pgfsetstrokecolor{currentstroke}%
\pgfsetdash{}{0pt}%
\pgfpathmoveto{\pgfqpoint{2.388771in}{2.701538in}}%
\pgfpathlineto{\pgfqpoint{2.402262in}{2.679736in}}%
\pgfpathlineto{\pgfqpoint{2.415739in}{2.658245in}}%
\pgfpathlineto{\pgfqpoint{2.429203in}{2.637061in}}%
\pgfpathlineto{\pgfqpoint{2.442655in}{2.616181in}}%
\pgfpathlineto{\pgfqpoint{2.451090in}{2.618422in}}%
\pgfpathlineto{\pgfqpoint{2.459512in}{2.620864in}}%
\pgfpathlineto{\pgfqpoint{2.467919in}{2.623503in}}%
\pgfpathlineto{\pgfqpoint{2.476312in}{2.626336in}}%
\pgfpathlineto{\pgfqpoint{2.462899in}{2.646861in}}%
\pgfpathlineto{\pgfqpoint{2.449474in}{2.667689in}}%
\pgfpathlineto{\pgfqpoint{2.436037in}{2.688823in}}%
\pgfpathlineto{\pgfqpoint{2.422586in}{2.710267in}}%
\pgfpathlineto{\pgfqpoint{2.414155in}{2.707779in}}%
\pgfpathlineto{\pgfqpoint{2.405708in}{2.705493in}}%
\pgfpathlineto{\pgfqpoint{2.397247in}{2.703411in}}%
\pgfpathlineto{\pgfqpoint{2.388771in}{2.701538in}}%
\pgfpathclose%
\pgfusepath{fill}%
\end{pgfscope}%
\begin{pgfscope}%
\pgfpathrectangle{\pgfqpoint{1.254980in}{0.150000in}}{\pgfqpoint{5.490039in}{5.490039in}}%
\pgfusepath{clip}%
\pgfsetbuttcap%
\pgfsetroundjoin%
\definecolor{currentfill}{rgb}{0.180653,0.701402,0.488189}%
\pgfsetfillcolor{currentfill}%
\pgfsetfillopacity{0.700000}%
\pgfsetlinewidth{0.000000pt}%
\definecolor{currentstroke}{rgb}{0.000000,0.000000,0.000000}%
\pgfsetstrokecolor{currentstroke}%
\pgfsetdash{}{0pt}%
\pgfpathmoveto{\pgfqpoint{6.093026in}{3.458651in}}%
\pgfpathlineto{\pgfqpoint{6.107243in}{3.467670in}}%
\pgfpathlineto{\pgfqpoint{6.121478in}{3.476839in}}%
\pgfpathlineto{\pgfqpoint{6.135732in}{3.486158in}}%
\pgfpathlineto{\pgfqpoint{6.142430in}{3.488373in}}%
\pgfpathlineto{\pgfqpoint{6.149130in}{3.490767in}}%
\pgfpathlineto{\pgfqpoint{6.155831in}{3.493348in}}%
\pgfpathlineto{\pgfqpoint{6.141607in}{3.484559in}}%
\pgfpathlineto{\pgfqpoint{6.127403in}{3.475920in}}%
\pgfpathlineto{\pgfqpoint{6.113217in}{3.467430in}}%
\pgfpathlineto{\pgfqpoint{6.106485in}{3.464314in}}%
\pgfpathlineto{\pgfqpoint{6.099755in}{3.461390in}}%
\pgfpathlineto{\pgfqpoint{6.093026in}{3.458651in}}%
\pgfpathclose%
\pgfusepath{fill}%
\end{pgfscope}%
\begin{pgfscope}%
\pgfpathrectangle{\pgfqpoint{1.254980in}{0.150000in}}{\pgfqpoint{5.490039in}{5.490039in}}%
\pgfusepath{clip}%
\pgfsetbuttcap%
\pgfsetroundjoin%
\definecolor{currentfill}{rgb}{0.250425,0.274290,0.533103}%
\pgfsetfillcolor{currentfill}%
\pgfsetfillopacity{0.700000}%
\pgfsetlinewidth{0.000000pt}%
\definecolor{currentstroke}{rgb}{0.000000,0.000000,0.000000}%
\pgfsetstrokecolor{currentstroke}%
\pgfsetdash{}{0pt}%
\pgfpathmoveto{\pgfqpoint{4.358094in}{2.345001in}}%
\pgfpathlineto{\pgfqpoint{4.371474in}{2.349742in}}%
\pgfpathlineto{\pgfqpoint{4.384865in}{2.354647in}}%
\pgfpathlineto{\pgfqpoint{4.398267in}{2.359714in}}%
\pgfpathlineto{\pgfqpoint{4.411682in}{2.364945in}}%
\pgfpathlineto{\pgfqpoint{4.419232in}{2.374126in}}%
\pgfpathlineto{\pgfqpoint{4.426778in}{2.383242in}}%
\pgfpathlineto{\pgfqpoint{4.434318in}{2.392294in}}%
\pgfpathlineto{\pgfqpoint{4.441852in}{2.401284in}}%
\pgfpathlineto{\pgfqpoint{4.428444in}{2.396118in}}%
\pgfpathlineto{\pgfqpoint{4.415048in}{2.391116in}}%
\pgfpathlineto{\pgfqpoint{4.401664in}{2.386276in}}%
\pgfpathlineto{\pgfqpoint{4.388292in}{2.381600in}}%
\pgfpathlineto{\pgfqpoint{4.380751in}{2.372535in}}%
\pgfpathlineto{\pgfqpoint{4.373204in}{2.363414in}}%
\pgfpathlineto{\pgfqpoint{4.365652in}{2.354236in}}%
\pgfpathlineto{\pgfqpoint{4.358094in}{2.345001in}}%
\pgfpathclose%
\pgfusepath{fill}%
\end{pgfscope}%
\begin{pgfscope}%
\pgfpathrectangle{\pgfqpoint{1.254980in}{0.150000in}}{\pgfqpoint{5.490039in}{5.490039in}}%
\pgfusepath{clip}%
\pgfsetbuttcap%
\pgfsetroundjoin%
\definecolor{currentfill}{rgb}{0.180629,0.429975,0.557282}%
\pgfsetfillcolor{currentfill}%
\pgfsetfillopacity{0.700000}%
\pgfsetlinewidth{0.000000pt}%
\definecolor{currentstroke}{rgb}{0.000000,0.000000,0.000000}%
\pgfsetstrokecolor{currentstroke}%
\pgfsetdash{}{0pt}%
\pgfpathmoveto{\pgfqpoint{4.890765in}{2.719802in}}%
\pgfpathlineto{\pgfqpoint{4.904398in}{2.727336in}}%
\pgfpathlineto{\pgfqpoint{4.918045in}{2.735030in}}%
\pgfpathlineto{\pgfqpoint{4.931707in}{2.742881in}}%
\pgfpathlineto{\pgfqpoint{4.945384in}{2.750891in}}%
\pgfpathlineto{\pgfqpoint{4.952710in}{2.757370in}}%
\pgfpathlineto{\pgfqpoint{4.960029in}{2.763799in}}%
\pgfpathlineto{\pgfqpoint{4.967342in}{2.770181in}}%
\pgfpathlineto{\pgfqpoint{4.974648in}{2.776521in}}%
\pgfpathlineto{\pgfqpoint{4.960984in}{2.768782in}}%
\pgfpathlineto{\pgfqpoint{4.947335in}{2.761200in}}%
\pgfpathlineto{\pgfqpoint{4.933700in}{2.753775in}}%
\pgfpathlineto{\pgfqpoint{4.920080in}{2.746509in}}%
\pgfpathlineto{\pgfqpoint{4.912761in}{2.739889in}}%
\pgfpathlineto{\pgfqpoint{4.905435in}{2.733234in}}%
\pgfpathlineto{\pgfqpoint{4.898103in}{2.726539in}}%
\pgfpathlineto{\pgfqpoint{4.890765in}{2.719802in}}%
\pgfpathclose%
\pgfusepath{fill}%
\end{pgfscope}%
\begin{pgfscope}%
\pgfpathrectangle{\pgfqpoint{1.254980in}{0.150000in}}{\pgfqpoint{5.490039in}{5.490039in}}%
\pgfusepath{clip}%
\pgfsetbuttcap%
\pgfsetroundjoin%
\definecolor{currentfill}{rgb}{0.281446,0.084320,0.407414}%
\pgfsetfillcolor{currentfill}%
\pgfsetfillopacity{0.700000}%
\pgfsetlinewidth{0.000000pt}%
\definecolor{currentstroke}{rgb}{0.000000,0.000000,0.000000}%
\pgfsetstrokecolor{currentstroke}%
\pgfsetdash{}{0pt}%
\pgfpathmoveto{\pgfqpoint{2.973603in}{2.001312in}}%
\pgfpathlineto{\pgfqpoint{2.986770in}{1.990847in}}%
\pgfpathlineto{\pgfqpoint{2.999935in}{1.980597in}}%
\pgfpathlineto{\pgfqpoint{3.013099in}{1.970561in}}%
\pgfpathlineto{\pgfqpoint{3.026260in}{1.960738in}}%
\pgfpathlineto{\pgfqpoint{3.034348in}{1.966854in}}%
\pgfpathlineto{\pgfqpoint{3.042426in}{1.973090in}}%
\pgfpathlineto{\pgfqpoint{3.050496in}{1.979442in}}%
\pgfpathlineto{\pgfqpoint{3.058557in}{1.985908in}}%
\pgfpathlineto{\pgfqpoint{3.045419in}{1.995399in}}%
\pgfpathlineto{\pgfqpoint{3.032280in}{2.005103in}}%
\pgfpathlineto{\pgfqpoint{3.019139in}{2.015020in}}%
\pgfpathlineto{\pgfqpoint{3.005997in}{2.025153in}}%
\pgfpathlineto{\pgfqpoint{2.997912in}{2.019009in}}%
\pgfpathlineto{\pgfqpoint{2.989818in}{2.012985in}}%
\pgfpathlineto{\pgfqpoint{2.981715in}{2.007085in}}%
\pgfpathlineto{\pgfqpoint{2.973603in}{2.001312in}}%
\pgfpathclose%
\pgfusepath{fill}%
\end{pgfscope}%
\begin{pgfscope}%
\pgfpathrectangle{\pgfqpoint{1.254980in}{0.150000in}}{\pgfqpoint{5.490039in}{5.490039in}}%
\pgfusepath{clip}%
\pgfsetbuttcap%
\pgfsetroundjoin%
\definecolor{currentfill}{rgb}{0.273809,0.031497,0.358853}%
\pgfsetfillcolor{currentfill}%
\pgfsetfillopacity{0.700000}%
\pgfsetlinewidth{0.000000pt}%
\definecolor{currentstroke}{rgb}{0.000000,0.000000,0.000000}%
\pgfsetstrokecolor{currentstroke}%
\pgfsetdash{}{0pt}%
\pgfpathmoveto{\pgfqpoint{3.300556in}{1.893655in}}%
\pgfpathlineto{\pgfqpoint{3.313681in}{1.887887in}}%
\pgfpathlineto{\pgfqpoint{3.326807in}{1.882310in}}%
\pgfpathlineto{\pgfqpoint{3.339936in}{1.876925in}}%
\pgfpathlineto{\pgfqpoint{3.353067in}{1.871729in}}%
\pgfpathlineto{\pgfqpoint{3.360991in}{1.880137in}}%
\pgfpathlineto{\pgfqpoint{3.368908in}{1.888606in}}%
\pgfpathlineto{\pgfqpoint{3.376819in}{1.897135in}}%
\pgfpathlineto{\pgfqpoint{3.384723in}{1.905720in}}%
\pgfpathlineto{\pgfqpoint{3.371608in}{1.910644in}}%
\pgfpathlineto{\pgfqpoint{3.358495in}{1.915758in}}%
\pgfpathlineto{\pgfqpoint{3.345386in}{1.921062in}}%
\pgfpathlineto{\pgfqpoint{3.332278in}{1.926557in}}%
\pgfpathlineto{\pgfqpoint{3.324358in}{1.918233in}}%
\pgfpathlineto{\pgfqpoint{3.316431in}{1.909974in}}%
\pgfpathlineto{\pgfqpoint{3.308497in}{1.901780in}}%
\pgfpathlineto{\pgfqpoint{3.300556in}{1.893655in}}%
\pgfpathclose%
\pgfusepath{fill}%
\end{pgfscope}%
\begin{pgfscope}%
\pgfpathrectangle{\pgfqpoint{1.254980in}{0.150000in}}{\pgfqpoint{5.490039in}{5.490039in}}%
\pgfusepath{clip}%
\pgfsetbuttcap%
\pgfsetroundjoin%
\definecolor{currentfill}{rgb}{0.276022,0.044167,0.370164}%
\pgfsetfillcolor{currentfill}%
\pgfsetfillopacity{0.700000}%
\pgfsetlinewidth{0.000000pt}%
\definecolor{currentstroke}{rgb}{0.000000,0.000000,0.000000}%
\pgfsetstrokecolor{currentstroke}%
\pgfsetdash{}{0pt}%
\pgfpathmoveto{\pgfqpoint{3.163631in}{1.917461in}}%
\pgfpathlineto{\pgfqpoint{3.176764in}{1.909823in}}%
\pgfpathlineto{\pgfqpoint{3.189898in}{1.902384in}}%
\pgfpathlineto{\pgfqpoint{3.203033in}{1.895144in}}%
\pgfpathlineto{\pgfqpoint{3.216169in}{1.888103in}}%
\pgfpathlineto{\pgfqpoint{3.224157in}{1.895616in}}%
\pgfpathlineto{\pgfqpoint{3.232138in}{1.903215in}}%
\pgfpathlineto{\pgfqpoint{3.240111in}{1.910898in}}%
\pgfpathlineto{\pgfqpoint{3.248077in}{1.918662in}}%
\pgfpathlineto{\pgfqpoint{3.234961in}{1.925403in}}%
\pgfpathlineto{\pgfqpoint{3.221845in}{1.932342in}}%
\pgfpathlineto{\pgfqpoint{3.208731in}{1.939479in}}%
\pgfpathlineto{\pgfqpoint{3.195617in}{1.946816in}}%
\pgfpathlineto{\pgfqpoint{3.187632in}{1.939343in}}%
\pgfpathlineto{\pgfqpoint{3.179640in}{1.931957in}}%
\pgfpathlineto{\pgfqpoint{3.171639in}{1.924662in}}%
\pgfpathlineto{\pgfqpoint{3.163631in}{1.917461in}}%
\pgfpathclose%
\pgfusepath{fill}%
\end{pgfscope}%
\begin{pgfscope}%
\pgfpathrectangle{\pgfqpoint{1.254980in}{0.150000in}}{\pgfqpoint{5.490039in}{5.490039in}}%
\pgfusepath{clip}%
\pgfsetbuttcap%
\pgfsetroundjoin%
\definecolor{currentfill}{rgb}{0.171176,0.452530,0.557965}%
\pgfsetfillcolor{currentfill}%
\pgfsetfillopacity{0.700000}%
\pgfsetlinewidth{0.000000pt}%
\definecolor{currentstroke}{rgb}{0.000000,0.000000,0.000000}%
\pgfsetstrokecolor{currentstroke}%
\pgfsetdash{}{0pt}%
\pgfpathmoveto{\pgfqpoint{4.974648in}{2.776521in}}%
\pgfpathlineto{\pgfqpoint{4.988328in}{2.784419in}}%
\pgfpathlineto{\pgfqpoint{5.002022in}{2.792475in}}%
\pgfpathlineto{\pgfqpoint{5.015732in}{2.800688in}}%
\pgfpathlineto{\pgfqpoint{5.029457in}{2.809059in}}%
\pgfpathlineto{\pgfqpoint{5.036743in}{2.815071in}}%
\pgfpathlineto{\pgfqpoint{5.044023in}{2.821039in}}%
\pgfpathlineto{\pgfqpoint{5.051296in}{2.826968in}}%
\pgfpathlineto{\pgfqpoint{5.058563in}{2.832861in}}%
\pgfpathlineto{\pgfqpoint{5.044852in}{2.824789in}}%
\pgfpathlineto{\pgfqpoint{5.031156in}{2.816876in}}%
\pgfpathlineto{\pgfqpoint{5.017476in}{2.809119in}}%
\pgfpathlineto{\pgfqpoint{5.003810in}{2.801520in}}%
\pgfpathlineto{\pgfqpoint{4.996529in}{2.795317in}}%
\pgfpathlineto{\pgfqpoint{4.989242in}{2.789086in}}%
\pgfpathlineto{\pgfqpoint{4.981948in}{2.782822in}}%
\pgfpathlineto{\pgfqpoint{4.974648in}{2.776521in}}%
\pgfpathclose%
\pgfusepath{fill}%
\end{pgfscope}%
\begin{pgfscope}%
\pgfpathrectangle{\pgfqpoint{1.254980in}{0.150000in}}{\pgfqpoint{5.490039in}{5.490039in}}%
\pgfusepath{clip}%
\pgfsetbuttcap%
\pgfsetroundjoin%
\definecolor{currentfill}{rgb}{0.239346,0.300855,0.540844}%
\pgfsetfillcolor{currentfill}%
\pgfsetfillopacity{0.700000}%
\pgfsetlinewidth{0.000000pt}%
\definecolor{currentstroke}{rgb}{0.000000,0.000000,0.000000}%
\pgfsetstrokecolor{currentstroke}%
\pgfsetdash{}{0pt}%
\pgfpathmoveto{\pgfqpoint{4.441852in}{2.401284in}}%
\pgfpathlineto{\pgfqpoint{4.455272in}{2.406612in}}%
\pgfpathlineto{\pgfqpoint{4.468703in}{2.412102in}}%
\pgfpathlineto{\pgfqpoint{4.482147in}{2.417755in}}%
\pgfpathlineto{\pgfqpoint{4.495604in}{2.423570in}}%
\pgfpathlineto{\pgfqpoint{4.503126in}{2.432415in}}%
\pgfpathlineto{\pgfqpoint{4.510641in}{2.441193in}}%
\pgfpathlineto{\pgfqpoint{4.518151in}{2.449905in}}%
\pgfpathlineto{\pgfqpoint{4.525656in}{2.458553in}}%
\pgfpathlineto{\pgfqpoint{4.512206in}{2.452833in}}%
\pgfpathlineto{\pgfqpoint{4.498770in}{2.447274in}}%
\pgfpathlineto{\pgfqpoint{4.485345in}{2.441877in}}%
\pgfpathlineto{\pgfqpoint{4.471933in}{2.436643in}}%
\pgfpathlineto{\pgfqpoint{4.464421in}{2.427891in}}%
\pgfpathlineto{\pgfqpoint{4.456904in}{2.419081in}}%
\pgfpathlineto{\pgfqpoint{4.449381in}{2.410212in}}%
\pgfpathlineto{\pgfqpoint{4.441852in}{2.401284in}}%
\pgfpathclose%
\pgfusepath{fill}%
\end{pgfscope}%
\begin{pgfscope}%
\pgfpathrectangle{\pgfqpoint{1.254980in}{0.150000in}}{\pgfqpoint{5.490039in}{5.490039in}}%
\pgfusepath{clip}%
\pgfsetbuttcap%
\pgfsetroundjoin%
\definecolor{currentfill}{rgb}{0.273809,0.031497,0.358853}%
\pgfsetfillcolor{currentfill}%
\pgfsetfillopacity{0.700000}%
\pgfsetlinewidth{0.000000pt}%
\definecolor{currentstroke}{rgb}{0.000000,0.000000,0.000000}%
\pgfsetstrokecolor{currentstroke}%
\pgfsetdash{}{0pt}%
\pgfpathmoveto{\pgfqpoint{3.437213in}{1.887905in}}%
\pgfpathlineto{\pgfqpoint{3.450344in}{1.883916in}}%
\pgfpathlineto{\pgfqpoint{3.463479in}{1.880113in}}%
\pgfpathlineto{\pgfqpoint{3.476617in}{1.876493in}}%
\pgfpathlineto{\pgfqpoint{3.489759in}{1.873056in}}%
\pgfpathlineto{\pgfqpoint{3.497628in}{1.882205in}}%
\pgfpathlineto{\pgfqpoint{3.505491in}{1.891393in}}%
\pgfpathlineto{\pgfqpoint{3.513348in}{1.900616in}}%
\pgfpathlineto{\pgfqpoint{3.521200in}{1.909874in}}%
\pgfpathlineto{\pgfqpoint{3.508071in}{1.913066in}}%
\pgfpathlineto{\pgfqpoint{3.494946in}{1.916442in}}%
\pgfpathlineto{\pgfqpoint{3.481825in}{1.920002in}}%
\pgfpathlineto{\pgfqpoint{3.468708in}{1.923746in}}%
\pgfpathlineto{\pgfqpoint{3.460843in}{1.914722in}}%
\pgfpathlineto{\pgfqpoint{3.452973in}{1.905739in}}%
\pgfpathlineto{\pgfqpoint{3.445096in}{1.896799in}}%
\pgfpathlineto{\pgfqpoint{3.437213in}{1.887905in}}%
\pgfpathclose%
\pgfusepath{fill}%
\end{pgfscope}%
\begin{pgfscope}%
\pgfpathrectangle{\pgfqpoint{1.254980in}{0.150000in}}{\pgfqpoint{5.490039in}{5.490039in}}%
\pgfusepath{clip}%
\pgfsetbuttcap%
\pgfsetroundjoin%
\definecolor{currentfill}{rgb}{0.160665,0.478540,0.558115}%
\pgfsetfillcolor{currentfill}%
\pgfsetfillopacity{0.700000}%
\pgfsetlinewidth{0.000000pt}%
\definecolor{currentstroke}{rgb}{0.000000,0.000000,0.000000}%
\pgfsetstrokecolor{currentstroke}%
\pgfsetdash{}{0pt}%
\pgfpathmoveto{\pgfqpoint{5.058563in}{2.832861in}}%
\pgfpathlineto{\pgfqpoint{5.072289in}{2.841089in}}%
\pgfpathlineto{\pgfqpoint{5.086031in}{2.849475in}}%
\pgfpathlineto{\pgfqpoint{5.099788in}{2.858018in}}%
\pgfpathlineto{\pgfqpoint{5.113562in}{2.866718in}}%
\pgfpathlineto{\pgfqpoint{5.120807in}{2.872259in}}%
\pgfpathlineto{\pgfqpoint{5.128046in}{2.877764in}}%
\pgfpathlineto{\pgfqpoint{5.135278in}{2.883237in}}%
\pgfpathlineto{\pgfqpoint{5.142503in}{2.888682in}}%
\pgfpathlineto{\pgfqpoint{5.128746in}{2.880312in}}%
\pgfpathlineto{\pgfqpoint{5.115004in}{2.872098in}}%
\pgfpathlineto{\pgfqpoint{5.101277in}{2.864041in}}%
\pgfpathlineto{\pgfqpoint{5.087566in}{2.856141in}}%
\pgfpathlineto{\pgfqpoint{5.080325in}{2.850357in}}%
\pgfpathlineto{\pgfqpoint{5.073077in}{2.844551in}}%
\pgfpathlineto{\pgfqpoint{5.065823in}{2.838720in}}%
\pgfpathlineto{\pgfqpoint{5.058563in}{2.832861in}}%
\pgfpathclose%
\pgfusepath{fill}%
\end{pgfscope}%
\begin{pgfscope}%
\pgfpathrectangle{\pgfqpoint{1.254980in}{0.150000in}}{\pgfqpoint{5.490039in}{5.490039in}}%
\pgfusepath{clip}%
\pgfsetbuttcap%
\pgfsetroundjoin%
\definecolor{currentfill}{rgb}{0.225863,0.330805,0.547314}%
\pgfsetfillcolor{currentfill}%
\pgfsetfillopacity{0.700000}%
\pgfsetlinewidth{0.000000pt}%
\definecolor{currentstroke}{rgb}{0.000000,0.000000,0.000000}%
\pgfsetstrokecolor{currentstroke}%
\pgfsetdash{}{0pt}%
\pgfpathmoveto{\pgfqpoint{4.525656in}{2.458553in}}%
\pgfpathlineto{\pgfqpoint{4.539118in}{2.464436in}}%
\pgfpathlineto{\pgfqpoint{4.552592in}{2.470480in}}%
\pgfpathlineto{\pgfqpoint{4.566080in}{2.476685in}}%
\pgfpathlineto{\pgfqpoint{4.579580in}{2.483052in}}%
\pgfpathlineto{\pgfqpoint{4.587071in}{2.491525in}}%
\pgfpathlineto{\pgfqpoint{4.594556in}{2.499929in}}%
\pgfpathlineto{\pgfqpoint{4.602035in}{2.508267in}}%
\pgfpathlineto{\pgfqpoint{4.609508in}{2.516539in}}%
\pgfpathlineto{\pgfqpoint{4.596015in}{2.510296in}}%
\pgfpathlineto{\pgfqpoint{4.582536in}{2.504214in}}%
\pgfpathlineto{\pgfqpoint{4.569069in}{2.498293in}}%
\pgfpathlineto{\pgfqpoint{4.555615in}{2.492533in}}%
\pgfpathlineto{\pgfqpoint{4.548134in}{2.484127in}}%
\pgfpathlineto{\pgfqpoint{4.540647in}{2.475663in}}%
\pgfpathlineto{\pgfqpoint{4.533154in}{2.467139in}}%
\pgfpathlineto{\pgfqpoint{4.525656in}{2.458553in}}%
\pgfpathclose%
\pgfusepath{fill}%
\end{pgfscope}%
\begin{pgfscope}%
\pgfpathrectangle{\pgfqpoint{1.254980in}{0.150000in}}{\pgfqpoint{5.490039in}{5.490039in}}%
\pgfusepath{clip}%
\pgfsetbuttcap%
\pgfsetroundjoin%
\definecolor{currentfill}{rgb}{0.279566,0.067836,0.391917}%
\pgfsetfillcolor{currentfill}%
\pgfsetfillopacity{0.700000}%
\pgfsetlinewidth{0.000000pt}%
\definecolor{currentstroke}{rgb}{0.000000,0.000000,0.000000}%
\pgfsetstrokecolor{currentstroke}%
\pgfsetdash{}{0pt}%
\pgfpathmoveto{\pgfqpoint{3.026260in}{1.960738in}}%
\pgfpathlineto{\pgfqpoint{3.039421in}{1.951126in}}%
\pgfpathlineto{\pgfqpoint{3.052579in}{1.941724in}}%
\pgfpathlineto{\pgfqpoint{3.065737in}{1.932530in}}%
\pgfpathlineto{\pgfqpoint{3.078894in}{1.923543in}}%
\pgfpathlineto{\pgfqpoint{3.086958in}{1.930002in}}%
\pgfpathlineto{\pgfqpoint{3.095013in}{1.936572in}}%
\pgfpathlineto{\pgfqpoint{3.103059in}{1.943252in}}%
\pgfpathlineto{\pgfqpoint{3.111097in}{1.950037in}}%
\pgfpathlineto{\pgfqpoint{3.097963in}{1.958693in}}%
\pgfpathlineto{\pgfqpoint{3.084829in}{1.967556in}}%
\pgfpathlineto{\pgfqpoint{3.071693in}{1.976627in}}%
\pgfpathlineto{\pgfqpoint{3.058557in}{1.985908in}}%
\pgfpathlineto{\pgfqpoint{3.050496in}{1.979442in}}%
\pgfpathlineto{\pgfqpoint{3.042426in}{1.973090in}}%
\pgfpathlineto{\pgfqpoint{3.034348in}{1.966854in}}%
\pgfpathlineto{\pgfqpoint{3.026260in}{1.960738in}}%
\pgfpathclose%
\pgfusepath{fill}%
\end{pgfscope}%
\begin{pgfscope}%
\pgfpathrectangle{\pgfqpoint{1.254980in}{0.150000in}}{\pgfqpoint{5.490039in}{5.490039in}}%
\pgfusepath{clip}%
\pgfsetbuttcap%
\pgfsetroundjoin%
\definecolor{currentfill}{rgb}{0.283091,0.110553,0.431554}%
\pgfsetfillcolor{currentfill}%
\pgfsetfillopacity{0.700000}%
\pgfsetlinewidth{0.000000pt}%
\definecolor{currentstroke}{rgb}{0.000000,0.000000,0.000000}%
\pgfsetstrokecolor{currentstroke}%
\pgfsetdash{}{0pt}%
\pgfpathmoveto{\pgfqpoint{3.825243in}{2.005250in}}%
\pgfpathlineto{\pgfqpoint{3.838441in}{2.005633in}}%
\pgfpathlineto{\pgfqpoint{3.851646in}{2.006187in}}%
\pgfpathlineto{\pgfqpoint{3.864859in}{2.006913in}}%
\pgfpathlineto{\pgfqpoint{3.878080in}{2.007810in}}%
\pgfpathlineto{\pgfqpoint{3.885813in}{2.018070in}}%
\pgfpathlineto{\pgfqpoint{3.893541in}{2.028308in}}%
\pgfpathlineto{\pgfqpoint{3.901264in}{2.038523in}}%
\pgfpathlineto{\pgfqpoint{3.908982in}{2.048714in}}%
\pgfpathlineto{\pgfqpoint{3.895768in}{2.047684in}}%
\pgfpathlineto{\pgfqpoint{3.882563in}{2.046825in}}%
\pgfpathlineto{\pgfqpoint{3.869366in}{2.046138in}}%
\pgfpathlineto{\pgfqpoint{3.856176in}{2.045623in}}%
\pgfpathlineto{\pgfqpoint{3.848450in}{2.035554in}}%
\pgfpathlineto{\pgfqpoint{3.840720in}{2.025469in}}%
\pgfpathlineto{\pgfqpoint{3.832984in}{2.015367in}}%
\pgfpathlineto{\pgfqpoint{3.825243in}{2.005250in}}%
\pgfpathclose%
\pgfusepath{fill}%
\end{pgfscope}%
\begin{pgfscope}%
\pgfpathrectangle{\pgfqpoint{1.254980in}{0.150000in}}{\pgfqpoint{5.490039in}{5.490039in}}%
\pgfusepath{clip}%
\pgfsetbuttcap%
\pgfsetroundjoin%
\definecolor{currentfill}{rgb}{0.283072,0.130895,0.449241}%
\pgfsetfillcolor{currentfill}%
\pgfsetfillopacity{0.700000}%
\pgfsetlinewidth{0.000000pt}%
\definecolor{currentstroke}{rgb}{0.000000,0.000000,0.000000}%
\pgfsetstrokecolor{currentstroke}%
\pgfsetdash{}{0pt}%
\pgfpathmoveto{\pgfqpoint{3.908982in}{2.048714in}}%
\pgfpathlineto{\pgfqpoint{3.922203in}{2.049914in}}%
\pgfpathlineto{\pgfqpoint{3.935432in}{2.051285in}}%
\pgfpathlineto{\pgfqpoint{3.948670in}{2.052825in}}%
\pgfpathlineto{\pgfqpoint{3.961916in}{2.054535in}}%
\pgfpathlineto{\pgfqpoint{3.969622in}{2.064816in}}%
\pgfpathlineto{\pgfqpoint{3.977323in}{2.075064in}}%
\pgfpathlineto{\pgfqpoint{3.985019in}{2.085279in}}%
\pgfpathlineto{\pgfqpoint{3.992710in}{2.095459in}}%
\pgfpathlineto{\pgfqpoint{3.979471in}{2.093645in}}%
\pgfpathlineto{\pgfqpoint{3.966240in}{2.092000in}}%
\pgfpathlineto{\pgfqpoint{3.953018in}{2.090524in}}%
\pgfpathlineto{\pgfqpoint{3.939804in}{2.089219in}}%
\pgfpathlineto{\pgfqpoint{3.932106in}{2.079133in}}%
\pgfpathlineto{\pgfqpoint{3.924403in}{2.069019in}}%
\pgfpathlineto{\pgfqpoint{3.916695in}{2.058879in}}%
\pgfpathlineto{\pgfqpoint{3.908982in}{2.048714in}}%
\pgfpathclose%
\pgfusepath{fill}%
\end{pgfscope}%
\begin{pgfscope}%
\pgfpathrectangle{\pgfqpoint{1.254980in}{0.150000in}}{\pgfqpoint{5.490039in}{5.490039in}}%
\pgfusepath{clip}%
\pgfsetbuttcap%
\pgfsetroundjoin%
\definecolor{currentfill}{rgb}{0.271828,0.209303,0.504434}%
\pgfsetfillcolor{currentfill}%
\pgfsetfillopacity{0.700000}%
\pgfsetlinewidth{0.000000pt}%
\definecolor{currentstroke}{rgb}{0.000000,0.000000,0.000000}%
\pgfsetstrokecolor{currentstroke}%
\pgfsetdash{}{0pt}%
\pgfpathmoveto{\pgfqpoint{2.676345in}{2.243771in}}%
\pgfpathlineto{\pgfqpoint{2.689632in}{2.228293in}}%
\pgfpathlineto{\pgfqpoint{2.702913in}{2.213065in}}%
\pgfpathlineto{\pgfqpoint{2.716186in}{2.198085in}}%
\pgfpathlineto{\pgfqpoint{2.729454in}{2.183350in}}%
\pgfpathlineto{\pgfqpoint{2.737726in}{2.187082in}}%
\pgfpathlineto{\pgfqpoint{2.745986in}{2.190983in}}%
\pgfpathlineto{\pgfqpoint{2.754234in}{2.195049in}}%
\pgfpathlineto{\pgfqpoint{2.762471in}{2.199277in}}%
\pgfpathlineto{\pgfqpoint{2.749236in}{2.213643in}}%
\pgfpathlineto{\pgfqpoint{2.735995in}{2.228254in}}%
\pgfpathlineto{\pgfqpoint{2.722747in}{2.243112in}}%
\pgfpathlineto{\pgfqpoint{2.709494in}{2.258220in}}%
\pgfpathlineto{\pgfqpoint{2.701225in}{2.254350in}}%
\pgfpathlineto{\pgfqpoint{2.692944in}{2.250650in}}%
\pgfpathlineto{\pgfqpoint{2.684651in}{2.247122in}}%
\pgfpathlineto{\pgfqpoint{2.676345in}{2.243771in}}%
\pgfpathclose%
\pgfusepath{fill}%
\end{pgfscope}%
\begin{pgfscope}%
\pgfpathrectangle{\pgfqpoint{1.254980in}{0.150000in}}{\pgfqpoint{5.490039in}{5.490039in}}%
\pgfusepath{clip}%
\pgfsetbuttcap%
\pgfsetroundjoin%
\definecolor{currentfill}{rgb}{0.263663,0.237631,0.518762}%
\pgfsetfillcolor{currentfill}%
\pgfsetfillopacity{0.700000}%
\pgfsetlinewidth{0.000000pt}%
\definecolor{currentstroke}{rgb}{0.000000,0.000000,0.000000}%
\pgfsetstrokecolor{currentstroke}%
\pgfsetdash{}{0pt}%
\pgfpathmoveto{\pgfqpoint{2.623124in}{2.308225in}}%
\pgfpathlineto{\pgfqpoint{2.636440in}{2.291726in}}%
\pgfpathlineto{\pgfqpoint{2.649750in}{2.275486in}}%
\pgfpathlineto{\pgfqpoint{2.663051in}{2.259501in}}%
\pgfpathlineto{\pgfqpoint{2.676345in}{2.243771in}}%
\pgfpathlineto{\pgfqpoint{2.684651in}{2.247122in}}%
\pgfpathlineto{\pgfqpoint{2.692944in}{2.250650in}}%
\pgfpathlineto{\pgfqpoint{2.701225in}{2.254350in}}%
\pgfpathlineto{\pgfqpoint{2.709494in}{2.258220in}}%
\pgfpathlineto{\pgfqpoint{2.696233in}{2.273579in}}%
\pgfpathlineto{\pgfqpoint{2.682966in}{2.289191in}}%
\pgfpathlineto{\pgfqpoint{2.669691in}{2.305059in}}%
\pgfpathlineto{\pgfqpoint{2.656409in}{2.321185in}}%
\pgfpathlineto{\pgfqpoint{2.648106in}{2.317676in}}%
\pgfpathlineto{\pgfqpoint{2.639792in}{2.314344in}}%
\pgfpathlineto{\pgfqpoint{2.631464in}{2.311193in}}%
\pgfpathlineto{\pgfqpoint{2.623124in}{2.308225in}}%
\pgfpathclose%
\pgfusepath{fill}%
\end{pgfscope}%
\begin{pgfscope}%
\pgfpathrectangle{\pgfqpoint{1.254980in}{0.150000in}}{\pgfqpoint{5.490039in}{5.490039in}}%
\pgfusepath{clip}%
\pgfsetbuttcap%
\pgfsetroundjoin%
\definecolor{currentfill}{rgb}{0.281924,0.089666,0.412415}%
\pgfsetfillcolor{currentfill}%
\pgfsetfillopacity{0.700000}%
\pgfsetlinewidth{0.000000pt}%
\definecolor{currentstroke}{rgb}{0.000000,0.000000,0.000000}%
\pgfsetstrokecolor{currentstroke}%
\pgfsetdash{}{0pt}%
\pgfpathmoveto{\pgfqpoint{3.741476in}{1.965495in}}%
\pgfpathlineto{\pgfqpoint{3.754654in}{1.965024in}}%
\pgfpathlineto{\pgfqpoint{3.767839in}{1.964727in}}%
\pgfpathlineto{\pgfqpoint{3.781031in}{1.964603in}}%
\pgfpathlineto{\pgfqpoint{3.794230in}{1.964653in}}%
\pgfpathlineto{\pgfqpoint{3.801991in}{1.974819in}}%
\pgfpathlineto{\pgfqpoint{3.809747in}{1.984975in}}%
\pgfpathlineto{\pgfqpoint{3.817498in}{1.995119in}}%
\pgfpathlineto{\pgfqpoint{3.825243in}{2.005250in}}%
\pgfpathlineto{\pgfqpoint{3.812053in}{2.005040in}}%
\pgfpathlineto{\pgfqpoint{3.798869in}{2.005003in}}%
\pgfpathlineto{\pgfqpoint{3.785693in}{2.005140in}}%
\pgfpathlineto{\pgfqpoint{3.772524in}{2.005451in}}%
\pgfpathlineto{\pgfqpoint{3.764769in}{1.995470in}}%
\pgfpathlineto{\pgfqpoint{3.757010in}{1.985483in}}%
\pgfpathlineto{\pgfqpoint{3.749245in}{1.975491in}}%
\pgfpathlineto{\pgfqpoint{3.741476in}{1.965495in}}%
\pgfpathclose%
\pgfusepath{fill}%
\end{pgfscope}%
\begin{pgfscope}%
\pgfpathrectangle{\pgfqpoint{1.254980in}{0.150000in}}{\pgfqpoint{5.490039in}{5.490039in}}%
\pgfusepath{clip}%
\pgfsetbuttcap%
\pgfsetroundjoin%
\definecolor{currentfill}{rgb}{0.281412,0.155834,0.469201}%
\pgfsetfillcolor{currentfill}%
\pgfsetfillopacity{0.700000}%
\pgfsetlinewidth{0.000000pt}%
\definecolor{currentstroke}{rgb}{0.000000,0.000000,0.000000}%
\pgfsetstrokecolor{currentstroke}%
\pgfsetdash{}{0pt}%
\pgfpathmoveto{\pgfqpoint{3.992710in}{2.095459in}}%
\pgfpathlineto{\pgfqpoint{4.005958in}{2.097443in}}%
\pgfpathlineto{\pgfqpoint{4.019215in}{2.099595in}}%
\pgfpathlineto{\pgfqpoint{4.032481in}{2.101915in}}%
\pgfpathlineto{\pgfqpoint{4.045756in}{2.104403in}}%
\pgfpathlineto{\pgfqpoint{4.053435in}{2.114637in}}%
\pgfpathlineto{\pgfqpoint{4.061110in}{2.124828in}}%
\pgfpathlineto{\pgfqpoint{4.068780in}{2.134977in}}%
\pgfpathlineto{\pgfqpoint{4.076444in}{2.145083in}}%
\pgfpathlineto{\pgfqpoint{4.063176in}{2.142518in}}%
\pgfpathlineto{\pgfqpoint{4.049916in}{2.140121in}}%
\pgfpathlineto{\pgfqpoint{4.036666in}{2.137892in}}%
\pgfpathlineto{\pgfqpoint{4.023425in}{2.135832in}}%
\pgfpathlineto{\pgfqpoint{4.015753in}{2.125792in}}%
\pgfpathlineto{\pgfqpoint{4.008077in}{2.115717in}}%
\pgfpathlineto{\pgfqpoint{4.000396in}{2.105606in}}%
\pgfpathlineto{\pgfqpoint{3.992710in}{2.095459in}}%
\pgfpathclose%
\pgfusepath{fill}%
\end{pgfscope}%
\begin{pgfscope}%
\pgfpathrectangle{\pgfqpoint{1.254980in}{0.150000in}}{\pgfqpoint{5.490039in}{5.490039in}}%
\pgfusepath{clip}%
\pgfsetbuttcap%
\pgfsetroundjoin%
\definecolor{currentfill}{rgb}{0.151918,0.500685,0.557587}%
\pgfsetfillcolor{currentfill}%
\pgfsetfillopacity{0.700000}%
\pgfsetlinewidth{0.000000pt}%
\definecolor{currentstroke}{rgb}{0.000000,0.000000,0.000000}%
\pgfsetstrokecolor{currentstroke}%
\pgfsetdash{}{0pt}%
\pgfpathmoveto{\pgfqpoint{5.142503in}{2.888682in}}%
\pgfpathlineto{\pgfqpoint{5.156277in}{2.897209in}}%
\pgfpathlineto{\pgfqpoint{5.170066in}{2.905893in}}%
\pgfpathlineto{\pgfqpoint{5.183872in}{2.914733in}}%
\pgfpathlineto{\pgfqpoint{5.197694in}{2.923730in}}%
\pgfpathlineto{\pgfqpoint{5.204896in}{2.928802in}}%
\pgfpathlineto{\pgfqpoint{5.212092in}{2.933846in}}%
\pgfpathlineto{\pgfqpoint{5.219282in}{2.938867in}}%
\pgfpathlineto{\pgfqpoint{5.226465in}{2.943869in}}%
\pgfpathlineto{\pgfqpoint{5.212660in}{2.935231in}}%
\pgfpathlineto{\pgfqpoint{5.198872in}{2.926750in}}%
\pgfpathlineto{\pgfqpoint{5.185100in}{2.918425in}}%
\pgfpathlineto{\pgfqpoint{5.171343in}{2.910257in}}%
\pgfpathlineto{\pgfqpoint{5.164142in}{2.904886in}}%
\pgfpathlineto{\pgfqpoint{5.156936in}{2.899502in}}%
\pgfpathlineto{\pgfqpoint{5.149723in}{2.894102in}}%
\pgfpathlineto{\pgfqpoint{5.142503in}{2.888682in}}%
\pgfpathclose%
\pgfusepath{fill}%
\end{pgfscope}%
\begin{pgfscope}%
\pgfpathrectangle{\pgfqpoint{1.254980in}{0.150000in}}{\pgfqpoint{5.490039in}{5.490039in}}%
\pgfusepath{clip}%
\pgfsetbuttcap%
\pgfsetroundjoin%
\definecolor{currentfill}{rgb}{0.278012,0.180367,0.486697}%
\pgfsetfillcolor{currentfill}%
\pgfsetfillopacity{0.700000}%
\pgfsetlinewidth{0.000000pt}%
\definecolor{currentstroke}{rgb}{0.000000,0.000000,0.000000}%
\pgfsetstrokecolor{currentstroke}%
\pgfsetdash{}{0pt}%
\pgfpathmoveto{\pgfqpoint{2.729454in}{2.183350in}}%
\pgfpathlineto{\pgfqpoint{2.742715in}{2.168860in}}%
\pgfpathlineto{\pgfqpoint{2.755970in}{2.154611in}}%
\pgfpathlineto{\pgfqpoint{2.769220in}{2.140603in}}%
\pgfpathlineto{\pgfqpoint{2.782464in}{2.126833in}}%
\pgfpathlineto{\pgfqpoint{2.790704in}{2.130943in}}%
\pgfpathlineto{\pgfqpoint{2.798933in}{2.135215in}}%
\pgfpathlineto{\pgfqpoint{2.807150in}{2.139645in}}%
\pgfpathlineto{\pgfqpoint{2.815356in}{2.144230in}}%
\pgfpathlineto{\pgfqpoint{2.802143in}{2.157633in}}%
\pgfpathlineto{\pgfqpoint{2.788924in}{2.171274in}}%
\pgfpathlineto{\pgfqpoint{2.775700in}{2.185155in}}%
\pgfpathlineto{\pgfqpoint{2.762471in}{2.199277in}}%
\pgfpathlineto{\pgfqpoint{2.754234in}{2.195049in}}%
\pgfpathlineto{\pgfqpoint{2.745986in}{2.190983in}}%
\pgfpathlineto{\pgfqpoint{2.737726in}{2.187082in}}%
\pgfpathlineto{\pgfqpoint{2.729454in}{2.183350in}}%
\pgfpathclose%
\pgfusepath{fill}%
\end{pgfscope}%
\begin{pgfscope}%
\pgfpathrectangle{\pgfqpoint{1.254980in}{0.150000in}}{\pgfqpoint{5.490039in}{5.490039in}}%
\pgfusepath{clip}%
\pgfsetbuttcap%
\pgfsetroundjoin%
\definecolor{currentfill}{rgb}{0.252194,0.269783,0.531579}%
\pgfsetfillcolor{currentfill}%
\pgfsetfillopacity{0.700000}%
\pgfsetlinewidth{0.000000pt}%
\definecolor{currentstroke}{rgb}{0.000000,0.000000,0.000000}%
\pgfsetstrokecolor{currentstroke}%
\pgfsetdash{}{0pt}%
\pgfpathmoveto{\pgfqpoint{2.569772in}{2.376849in}}%
\pgfpathlineto{\pgfqpoint{2.583123in}{2.359294in}}%
\pgfpathlineto{\pgfqpoint{2.596465in}{2.342007in}}%
\pgfpathlineto{\pgfqpoint{2.609798in}{2.324984in}}%
\pgfpathlineto{\pgfqpoint{2.623124in}{2.308225in}}%
\pgfpathlineto{\pgfqpoint{2.631464in}{2.311193in}}%
\pgfpathlineto{\pgfqpoint{2.639792in}{2.314344in}}%
\pgfpathlineto{\pgfqpoint{2.648106in}{2.317676in}}%
\pgfpathlineto{\pgfqpoint{2.656409in}{2.321185in}}%
\pgfpathlineto{\pgfqpoint{2.643119in}{2.337571in}}%
\pgfpathlineto{\pgfqpoint{2.629821in}{2.354219in}}%
\pgfpathlineto{\pgfqpoint{2.616514in}{2.371131in}}%
\pgfpathlineto{\pgfqpoint{2.603199in}{2.388310in}}%
\pgfpathlineto{\pgfqpoint{2.594863in}{2.385165in}}%
\pgfpathlineto{\pgfqpoint{2.586513in}{2.382204in}}%
\pgfpathlineto{\pgfqpoint{2.578149in}{2.379431in}}%
\pgfpathlineto{\pgfqpoint{2.569772in}{2.376849in}}%
\pgfpathclose%
\pgfusepath{fill}%
\end{pgfscope}%
\begin{pgfscope}%
\pgfpathrectangle{\pgfqpoint{1.254980in}{0.150000in}}{\pgfqpoint{5.490039in}{5.490039in}}%
\pgfusepath{clip}%
\pgfsetbuttcap%
\pgfsetroundjoin%
\definecolor{currentfill}{rgb}{0.214298,0.355619,0.551184}%
\pgfsetfillcolor{currentfill}%
\pgfsetfillopacity{0.700000}%
\pgfsetlinewidth{0.000000pt}%
\definecolor{currentstroke}{rgb}{0.000000,0.000000,0.000000}%
\pgfsetstrokecolor{currentstroke}%
\pgfsetdash{}{0pt}%
\pgfpathmoveto{\pgfqpoint{4.609508in}{2.516539in}}%
\pgfpathlineto{\pgfqpoint{4.623013in}{2.522944in}}%
\pgfpathlineto{\pgfqpoint{4.636532in}{2.529509in}}%
\pgfpathlineto{\pgfqpoint{4.650065in}{2.536234in}}%
\pgfpathlineto{\pgfqpoint{4.663610in}{2.543121in}}%
\pgfpathlineto{\pgfqpoint{4.671069in}{2.551189in}}%
\pgfpathlineto{\pgfqpoint{4.678522in}{2.559188in}}%
\pgfpathlineto{\pgfqpoint{4.685968in}{2.567121in}}%
\pgfpathlineto{\pgfqpoint{4.693408in}{2.574990in}}%
\pgfpathlineto{\pgfqpoint{4.679871in}{2.568256in}}%
\pgfpathlineto{\pgfqpoint{4.666348in}{2.561683in}}%
\pgfpathlineto{\pgfqpoint{4.652837in}{2.555271in}}%
\pgfpathlineto{\pgfqpoint{4.639340in}{2.549018in}}%
\pgfpathlineto{\pgfqpoint{4.631891in}{2.540987in}}%
\pgfpathlineto{\pgfqpoint{4.624436in}{2.532897in}}%
\pgfpathlineto{\pgfqpoint{4.616975in}{2.524749in}}%
\pgfpathlineto{\pgfqpoint{4.609508in}{2.516539in}}%
\pgfpathclose%
\pgfusepath{fill}%
\end{pgfscope}%
\begin{pgfscope}%
\pgfpathrectangle{\pgfqpoint{1.254980in}{0.150000in}}{\pgfqpoint{5.490039in}{5.490039in}}%
\pgfusepath{clip}%
\pgfsetbuttcap%
\pgfsetroundjoin%
\definecolor{currentfill}{rgb}{0.277134,0.185228,0.489898}%
\pgfsetfillcolor{currentfill}%
\pgfsetfillopacity{0.700000}%
\pgfsetlinewidth{0.000000pt}%
\definecolor{currentstroke}{rgb}{0.000000,0.000000,0.000000}%
\pgfsetstrokecolor{currentstroke}%
\pgfsetdash{}{0pt}%
\pgfpathmoveto{\pgfqpoint{4.076444in}{2.145083in}}%
\pgfpathlineto{\pgfqpoint{4.089723in}{2.147815in}}%
\pgfpathlineto{\pgfqpoint{4.103010in}{2.150715in}}%
\pgfpathlineto{\pgfqpoint{4.116308in}{2.153781in}}%
\pgfpathlineto{\pgfqpoint{4.129615in}{2.157013in}}%
\pgfpathlineto{\pgfqpoint{4.137269in}{2.167135in}}%
\pgfpathlineto{\pgfqpoint{4.144917in}{2.177207in}}%
\pgfpathlineto{\pgfqpoint{4.152561in}{2.187228in}}%
\pgfpathlineto{\pgfqpoint{4.160199in}{2.197199in}}%
\pgfpathlineto{\pgfqpoint{4.146898in}{2.193918in}}%
\pgfpathlineto{\pgfqpoint{4.133606in}{2.190803in}}%
\pgfpathlineto{\pgfqpoint{4.120325in}{2.187855in}}%
\pgfpathlineto{\pgfqpoint{4.107053in}{2.185074in}}%
\pgfpathlineto{\pgfqpoint{4.099409in}{2.175141in}}%
\pgfpathlineto{\pgfqpoint{4.091759in}{2.165165in}}%
\pgfpathlineto{\pgfqpoint{4.084104in}{2.155146in}}%
\pgfpathlineto{\pgfqpoint{4.076444in}{2.145083in}}%
\pgfpathclose%
\pgfusepath{fill}%
\end{pgfscope}%
\begin{pgfscope}%
\pgfpathrectangle{\pgfqpoint{1.254980in}{0.150000in}}{\pgfqpoint{5.490039in}{5.490039in}}%
\pgfusepath{clip}%
\pgfsetbuttcap%
\pgfsetroundjoin%
\definecolor{currentfill}{rgb}{0.273809,0.031497,0.358853}%
\pgfsetfillcolor{currentfill}%
\pgfsetfillopacity{0.700000}%
\pgfsetlinewidth{0.000000pt}%
\definecolor{currentstroke}{rgb}{0.000000,0.000000,0.000000}%
\pgfsetstrokecolor{currentstroke}%
\pgfsetdash{}{0pt}%
\pgfpathmoveto{\pgfqpoint{3.216169in}{1.888103in}}%
\pgfpathlineto{\pgfqpoint{3.229305in}{1.881258in}}%
\pgfpathlineto{\pgfqpoint{3.242443in}{1.874608in}}%
\pgfpathlineto{\pgfqpoint{3.255582in}{1.868153in}}%
\pgfpathlineto{\pgfqpoint{3.268722in}{1.861892in}}%
\pgfpathlineto{\pgfqpoint{3.276692in}{1.869717in}}%
\pgfpathlineto{\pgfqpoint{3.284654in}{1.877620in}}%
\pgfpathlineto{\pgfqpoint{3.292608in}{1.885601in}}%
\pgfpathlineto{\pgfqpoint{3.300556in}{1.893655in}}%
\pgfpathlineto{\pgfqpoint{3.287434in}{1.899615in}}%
\pgfpathlineto{\pgfqpoint{3.274313in}{1.905769in}}%
\pgfpathlineto{\pgfqpoint{3.261194in}{1.912118in}}%
\pgfpathlineto{\pgfqpoint{3.248077in}{1.918662in}}%
\pgfpathlineto{\pgfqpoint{3.240111in}{1.910898in}}%
\pgfpathlineto{\pgfqpoint{3.232138in}{1.903215in}}%
\pgfpathlineto{\pgfqpoint{3.224157in}{1.895616in}}%
\pgfpathlineto{\pgfqpoint{3.216169in}{1.888103in}}%
\pgfpathclose%
\pgfusepath{fill}%
\end{pgfscope}%
\begin{pgfscope}%
\pgfpathrectangle{\pgfqpoint{1.254980in}{0.150000in}}{\pgfqpoint{5.490039in}{5.490039in}}%
\pgfusepath{clip}%
\pgfsetbuttcap%
\pgfsetroundjoin%
\definecolor{currentfill}{rgb}{0.279566,0.067836,0.391917}%
\pgfsetfillcolor{currentfill}%
\pgfsetfillopacity{0.700000}%
\pgfsetlinewidth{0.000000pt}%
\definecolor{currentstroke}{rgb}{0.000000,0.000000,0.000000}%
\pgfsetstrokecolor{currentstroke}%
\pgfsetdash{}{0pt}%
\pgfpathmoveto{\pgfqpoint{3.657657in}{1.929895in}}%
\pgfpathlineto{\pgfqpoint{3.670820in}{1.928534in}}%
\pgfpathlineto{\pgfqpoint{3.683989in}{1.927349in}}%
\pgfpathlineto{\pgfqpoint{3.697164in}{1.926340in}}%
\pgfpathlineto{\pgfqpoint{3.710346in}{1.925506in}}%
\pgfpathlineto{\pgfqpoint{3.718136in}{1.935501in}}%
\pgfpathlineto{\pgfqpoint{3.725921in}{1.945499in}}%
\pgfpathlineto{\pgfqpoint{3.733701in}{1.955497in}}%
\pgfpathlineto{\pgfqpoint{3.741476in}{1.965495in}}%
\pgfpathlineto{\pgfqpoint{3.728304in}{1.966141in}}%
\pgfpathlineto{\pgfqpoint{3.715139in}{1.966962in}}%
\pgfpathlineto{\pgfqpoint{3.701979in}{1.967958in}}%
\pgfpathlineto{\pgfqpoint{3.688826in}{1.969131in}}%
\pgfpathlineto{\pgfqpoint{3.681042in}{1.959311in}}%
\pgfpathlineto{\pgfqpoint{3.673252in}{1.949498in}}%
\pgfpathlineto{\pgfqpoint{3.665457in}{1.939692in}}%
\pgfpathlineto{\pgfqpoint{3.657657in}{1.929895in}}%
\pgfpathclose%
\pgfusepath{fill}%
\end{pgfscope}%
\begin{pgfscope}%
\pgfpathrectangle{\pgfqpoint{1.254980in}{0.150000in}}{\pgfqpoint{5.490039in}{5.490039in}}%
\pgfusepath{clip}%
\pgfsetbuttcap%
\pgfsetroundjoin%
\definecolor{currentfill}{rgb}{0.281412,0.155834,0.469201}%
\pgfsetfillcolor{currentfill}%
\pgfsetfillopacity{0.700000}%
\pgfsetlinewidth{0.000000pt}%
\definecolor{currentstroke}{rgb}{0.000000,0.000000,0.000000}%
\pgfsetstrokecolor{currentstroke}%
\pgfsetdash{}{0pt}%
\pgfpathmoveto{\pgfqpoint{2.782464in}{2.126833in}}%
\pgfpathlineto{\pgfqpoint{2.795703in}{2.113299in}}%
\pgfpathlineto{\pgfqpoint{2.808937in}{2.100000in}}%
\pgfpathlineto{\pgfqpoint{2.822167in}{2.086934in}}%
\pgfpathlineto{\pgfqpoint{2.835392in}{2.074099in}}%
\pgfpathlineto{\pgfqpoint{2.843601in}{2.078586in}}%
\pgfpathlineto{\pgfqpoint{2.851799in}{2.083228in}}%
\pgfpathlineto{\pgfqpoint{2.859987in}{2.088020in}}%
\pgfpathlineto{\pgfqpoint{2.868164in}{2.092959in}}%
\pgfpathlineto{\pgfqpoint{2.854968in}{2.105429in}}%
\pgfpathlineto{\pgfqpoint{2.841769in}{2.118130in}}%
\pgfpathlineto{\pgfqpoint{2.828565in}{2.131063in}}%
\pgfpathlineto{\pgfqpoint{2.815356in}{2.144230in}}%
\pgfpathlineto{\pgfqpoint{2.807150in}{2.139645in}}%
\pgfpathlineto{\pgfqpoint{2.798933in}{2.135215in}}%
\pgfpathlineto{\pgfqpoint{2.790704in}{2.130943in}}%
\pgfpathlineto{\pgfqpoint{2.782464in}{2.126833in}}%
\pgfpathclose%
\pgfusepath{fill}%
\end{pgfscope}%
\begin{pgfscope}%
\pgfpathrectangle{\pgfqpoint{1.254980in}{0.150000in}}{\pgfqpoint{5.490039in}{5.490039in}}%
\pgfusepath{clip}%
\pgfsetbuttcap%
\pgfsetroundjoin%
\definecolor{currentfill}{rgb}{0.144759,0.519093,0.556572}%
\pgfsetfillcolor{currentfill}%
\pgfsetfillopacity{0.700000}%
\pgfsetlinewidth{0.000000pt}%
\definecolor{currentstroke}{rgb}{0.000000,0.000000,0.000000}%
\pgfsetstrokecolor{currentstroke}%
\pgfsetdash{}{0pt}%
\pgfpathmoveto{\pgfqpoint{5.226465in}{2.943869in}}%
\pgfpathlineto{\pgfqpoint{5.240286in}{2.952662in}}%
\pgfpathlineto{\pgfqpoint{5.254123in}{2.961611in}}%
\pgfpathlineto{\pgfqpoint{5.267977in}{2.970717in}}%
\pgfpathlineto{\pgfqpoint{5.281847in}{2.979979in}}%
\pgfpathlineto{\pgfqpoint{5.289005in}{2.984587in}}%
\pgfpathlineto{\pgfqpoint{5.296157in}{2.989178in}}%
\pgfpathlineto{\pgfqpoint{5.303303in}{2.993755in}}%
\pgfpathlineto{\pgfqpoint{5.310442in}{2.998323in}}%
\pgfpathlineto{\pgfqpoint{5.296591in}{2.989451in}}%
\pgfpathlineto{\pgfqpoint{5.282756in}{2.980735in}}%
\pgfpathlineto{\pgfqpoint{5.268938in}{2.972174in}}%
\pgfpathlineto{\pgfqpoint{5.255136in}{2.963768in}}%
\pgfpathlineto{\pgfqpoint{5.247977in}{2.958801in}}%
\pgfpathlineto{\pgfqpoint{5.240813in}{2.953831in}}%
\pgfpathlineto{\pgfqpoint{5.233642in}{2.948855in}}%
\pgfpathlineto{\pgfqpoint{5.226465in}{2.943869in}}%
\pgfpathclose%
\pgfusepath{fill}%
\end{pgfscope}%
\begin{pgfscope}%
\pgfpathrectangle{\pgfqpoint{1.254980in}{0.150000in}}{\pgfqpoint{5.490039in}{5.490039in}}%
\pgfusepath{clip}%
\pgfsetbuttcap%
\pgfsetroundjoin%
\definecolor{currentfill}{rgb}{0.239346,0.300855,0.540844}%
\pgfsetfillcolor{currentfill}%
\pgfsetfillopacity{0.700000}%
\pgfsetlinewidth{0.000000pt}%
\definecolor{currentstroke}{rgb}{0.000000,0.000000,0.000000}%
\pgfsetstrokecolor{currentstroke}%
\pgfsetdash{}{0pt}%
\pgfpathmoveto{\pgfqpoint{2.516274in}{2.449792in}}%
\pgfpathlineto{\pgfqpoint{2.529663in}{2.431143in}}%
\pgfpathlineto{\pgfqpoint{2.543042in}{2.412771in}}%
\pgfpathlineto{\pgfqpoint{2.556412in}{2.394674in}}%
\pgfpathlineto{\pgfqpoint{2.569772in}{2.376849in}}%
\pgfpathlineto{\pgfqpoint{2.578149in}{2.379431in}}%
\pgfpathlineto{\pgfqpoint{2.586513in}{2.382204in}}%
\pgfpathlineto{\pgfqpoint{2.594863in}{2.385165in}}%
\pgfpathlineto{\pgfqpoint{2.603199in}{2.388310in}}%
\pgfpathlineto{\pgfqpoint{2.589876in}{2.405759in}}%
\pgfpathlineto{\pgfqpoint{2.576543in}{2.423479in}}%
\pgfpathlineto{\pgfqpoint{2.563201in}{2.441473in}}%
\pgfpathlineto{\pgfqpoint{2.549850in}{2.459744in}}%
\pgfpathlineto{\pgfqpoint{2.541476in}{2.456964in}}%
\pgfpathlineto{\pgfqpoint{2.533089in}{2.454377in}}%
\pgfpathlineto{\pgfqpoint{2.524689in}{2.451985in}}%
\pgfpathlineto{\pgfqpoint{2.516274in}{2.449792in}}%
\pgfpathclose%
\pgfusepath{fill}%
\end{pgfscope}%
\begin{pgfscope}%
\pgfpathrectangle{\pgfqpoint{1.254980in}{0.150000in}}{\pgfqpoint{5.490039in}{5.490039in}}%
\pgfusepath{clip}%
\pgfsetbuttcap%
\pgfsetroundjoin%
\definecolor{currentfill}{rgb}{0.272594,0.025563,0.353093}%
\pgfsetfillcolor{currentfill}%
\pgfsetfillopacity{0.700000}%
\pgfsetlinewidth{0.000000pt}%
\definecolor{currentstroke}{rgb}{0.000000,0.000000,0.000000}%
\pgfsetstrokecolor{currentstroke}%
\pgfsetdash{}{0pt}%
\pgfpathmoveto{\pgfqpoint{3.353067in}{1.871729in}}%
\pgfpathlineto{\pgfqpoint{3.366200in}{1.866722in}}%
\pgfpathlineto{\pgfqpoint{3.379337in}{1.861903in}}%
\pgfpathlineto{\pgfqpoint{3.392476in}{1.857271in}}%
\pgfpathlineto{\pgfqpoint{3.405619in}{1.852825in}}%
\pgfpathlineto{\pgfqpoint{3.413527in}{1.861516in}}%
\pgfpathlineto{\pgfqpoint{3.421429in}{1.870261in}}%
\pgfpathlineto{\pgfqpoint{3.429324in}{1.879058in}}%
\pgfpathlineto{\pgfqpoint{3.437213in}{1.887905in}}%
\pgfpathlineto{\pgfqpoint{3.424086in}{1.892078in}}%
\pgfpathlineto{\pgfqpoint{3.410962in}{1.896438in}}%
\pgfpathlineto{\pgfqpoint{3.397841in}{1.900985in}}%
\pgfpathlineto{\pgfqpoint{3.384723in}{1.905720in}}%
\pgfpathlineto{\pgfqpoint{3.376819in}{1.897135in}}%
\pgfpathlineto{\pgfqpoint{3.368908in}{1.888606in}}%
\pgfpathlineto{\pgfqpoint{3.360991in}{1.880137in}}%
\pgfpathlineto{\pgfqpoint{3.353067in}{1.871729in}}%
\pgfpathclose%
\pgfusepath{fill}%
\end{pgfscope}%
\begin{pgfscope}%
\pgfpathrectangle{\pgfqpoint{1.254980in}{0.150000in}}{\pgfqpoint{5.490039in}{5.490039in}}%
\pgfusepath{clip}%
\pgfsetbuttcap%
\pgfsetroundjoin%
\definecolor{currentfill}{rgb}{0.270595,0.214069,0.507052}%
\pgfsetfillcolor{currentfill}%
\pgfsetfillopacity{0.700000}%
\pgfsetlinewidth{0.000000pt}%
\definecolor{currentstroke}{rgb}{0.000000,0.000000,0.000000}%
\pgfsetstrokecolor{currentstroke}%
\pgfsetdash{}{0pt}%
\pgfpathmoveto{\pgfqpoint{4.160199in}{2.197199in}}%
\pgfpathlineto{\pgfqpoint{4.173510in}{2.200646in}}%
\pgfpathlineto{\pgfqpoint{4.186832in}{2.204259in}}%
\pgfpathlineto{\pgfqpoint{4.200164in}{2.208038in}}%
\pgfpathlineto{\pgfqpoint{4.213506in}{2.211982in}}%
\pgfpathlineto{\pgfqpoint{4.221134in}{2.221933in}}%
\pgfpathlineto{\pgfqpoint{4.228756in}{2.231826in}}%
\pgfpathlineto{\pgfqpoint{4.236373in}{2.241663in}}%
\pgfpathlineto{\pgfqpoint{4.243985in}{2.251442in}}%
\pgfpathlineto{\pgfqpoint{4.230648in}{2.247478in}}%
\pgfpathlineto{\pgfqpoint{4.217322in}{2.243679in}}%
\pgfpathlineto{\pgfqpoint{4.204007in}{2.240045in}}%
\pgfpathlineto{\pgfqpoint{4.190701in}{2.236578in}}%
\pgfpathlineto{\pgfqpoint{4.183084in}{2.226808in}}%
\pgfpathlineto{\pgfqpoint{4.175460in}{2.216989in}}%
\pgfpathlineto{\pgfqpoint{4.167832in}{2.207119in}}%
\pgfpathlineto{\pgfqpoint{4.160199in}{2.197199in}}%
\pgfpathclose%
\pgfusepath{fill}%
\end{pgfscope}%
\begin{pgfscope}%
\pgfpathrectangle{\pgfqpoint{1.254980in}{0.150000in}}{\pgfqpoint{5.490039in}{5.490039in}}%
\pgfusepath{clip}%
\pgfsetbuttcap%
\pgfsetroundjoin%
\definecolor{currentfill}{rgb}{0.277018,0.050344,0.375715}%
\pgfsetfillcolor{currentfill}%
\pgfsetfillopacity{0.700000}%
\pgfsetlinewidth{0.000000pt}%
\definecolor{currentstroke}{rgb}{0.000000,0.000000,0.000000}%
\pgfsetstrokecolor{currentstroke}%
\pgfsetdash{}{0pt}%
\pgfpathmoveto{\pgfqpoint{3.573761in}{1.898918in}}%
\pgfpathlineto{\pgfqpoint{3.586914in}{1.896630in}}%
\pgfpathlineto{\pgfqpoint{3.600071in}{1.894520in}}%
\pgfpathlineto{\pgfqpoint{3.613234in}{1.892588in}}%
\pgfpathlineto{\pgfqpoint{3.626402in}{1.890834in}}%
\pgfpathlineto{\pgfqpoint{3.634224in}{1.900577in}}%
\pgfpathlineto{\pgfqpoint{3.642040in}{1.910336in}}%
\pgfpathlineto{\pgfqpoint{3.649851in}{1.920109in}}%
\pgfpathlineto{\pgfqpoint{3.657657in}{1.929895in}}%
\pgfpathlineto{\pgfqpoint{3.644500in}{1.931433in}}%
\pgfpathlineto{\pgfqpoint{3.631348in}{1.933148in}}%
\pgfpathlineto{\pgfqpoint{3.618202in}{1.935042in}}%
\pgfpathlineto{\pgfqpoint{3.605061in}{1.937114in}}%
\pgfpathlineto{\pgfqpoint{3.597244in}{1.927534in}}%
\pgfpathlineto{\pgfqpoint{3.589422in}{1.917973in}}%
\pgfpathlineto{\pgfqpoint{3.581594in}{1.908434in}}%
\pgfpathlineto{\pgfqpoint{3.573761in}{1.898918in}}%
\pgfpathclose%
\pgfusepath{fill}%
\end{pgfscope}%
\begin{pgfscope}%
\pgfpathrectangle{\pgfqpoint{1.254980in}{0.150000in}}{\pgfqpoint{5.490039in}{5.490039in}}%
\pgfusepath{clip}%
\pgfsetbuttcap%
\pgfsetroundjoin%
\definecolor{currentfill}{rgb}{0.203063,0.379716,0.553925}%
\pgfsetfillcolor{currentfill}%
\pgfsetfillopacity{0.700000}%
\pgfsetlinewidth{0.000000pt}%
\definecolor{currentstroke}{rgb}{0.000000,0.000000,0.000000}%
\pgfsetstrokecolor{currentstroke}%
\pgfsetdash{}{0pt}%
\pgfpathmoveto{\pgfqpoint{4.693408in}{2.574990in}}%
\pgfpathlineto{\pgfqpoint{4.706959in}{2.581883in}}%
\pgfpathlineto{\pgfqpoint{4.720524in}{2.588937in}}%
\pgfpathlineto{\pgfqpoint{4.734102in}{2.596151in}}%
\pgfpathlineto{\pgfqpoint{4.747695in}{2.603525in}}%
\pgfpathlineto{\pgfqpoint{4.755119in}{2.611160in}}%
\pgfpathlineto{\pgfqpoint{4.762538in}{2.618728in}}%
\pgfpathlineto{\pgfqpoint{4.769950in}{2.626231in}}%
\pgfpathlineto{\pgfqpoint{4.777356in}{2.633671in}}%
\pgfpathlineto{\pgfqpoint{4.763773in}{2.626480in}}%
\pgfpathlineto{\pgfqpoint{4.750204in}{2.619449in}}%
\pgfpathlineto{\pgfqpoint{4.736649in}{2.612577in}}%
\pgfpathlineto{\pgfqpoint{4.723108in}{2.605865in}}%
\pgfpathlineto{\pgfqpoint{4.715692in}{2.598231in}}%
\pgfpathlineto{\pgfqpoint{4.708270in}{2.590543in}}%
\pgfpathlineto{\pgfqpoint{4.700842in}{2.582796in}}%
\pgfpathlineto{\pgfqpoint{4.693408in}{2.574990in}}%
\pgfpathclose%
\pgfusepath{fill}%
\end{pgfscope}%
\begin{pgfscope}%
\pgfpathrectangle{\pgfqpoint{1.254980in}{0.150000in}}{\pgfqpoint{5.490039in}{5.490039in}}%
\pgfusepath{clip}%
\pgfsetbuttcap%
\pgfsetroundjoin%
\definecolor{currentfill}{rgb}{0.136408,0.541173,0.554483}%
\pgfsetfillcolor{currentfill}%
\pgfsetfillopacity{0.700000}%
\pgfsetlinewidth{0.000000pt}%
\definecolor{currentstroke}{rgb}{0.000000,0.000000,0.000000}%
\pgfsetstrokecolor{currentstroke}%
\pgfsetdash{}{0pt}%
\pgfpathmoveto{\pgfqpoint{5.310442in}{2.998323in}}%
\pgfpathlineto{\pgfqpoint{5.324310in}{3.007351in}}%
\pgfpathlineto{\pgfqpoint{5.338195in}{3.016534in}}%
\pgfpathlineto{\pgfqpoint{5.352096in}{3.025873in}}%
\pgfpathlineto{\pgfqpoint{5.366015in}{3.035368in}}%
\pgfpathlineto{\pgfqpoint{5.373128in}{3.039523in}}%
\pgfpathlineto{\pgfqpoint{5.380234in}{3.043671in}}%
\pgfpathlineto{\pgfqpoint{5.387335in}{3.047818in}}%
\pgfpathlineto{\pgfqpoint{5.394430in}{3.051968in}}%
\pgfpathlineto{\pgfqpoint{5.380532in}{3.042893in}}%
\pgfpathlineto{\pgfqpoint{5.366651in}{3.033973in}}%
\pgfpathlineto{\pgfqpoint{5.352787in}{3.025208in}}%
\pgfpathlineto{\pgfqpoint{5.338940in}{3.016599in}}%
\pgfpathlineto{\pgfqpoint{5.331824in}{3.012020in}}%
\pgfpathlineto{\pgfqpoint{5.324703in}{3.007451in}}%
\pgfpathlineto{\pgfqpoint{5.317576in}{3.002887in}}%
\pgfpathlineto{\pgfqpoint{5.310442in}{2.998323in}}%
\pgfpathclose%
\pgfusepath{fill}%
\end{pgfscope}%
\begin{pgfscope}%
\pgfpathrectangle{\pgfqpoint{1.254980in}{0.150000in}}{\pgfqpoint{5.490039in}{5.490039in}}%
\pgfusepath{clip}%
\pgfsetbuttcap%
\pgfsetroundjoin%
\definecolor{currentfill}{rgb}{0.277941,0.056324,0.381191}%
\pgfsetfillcolor{currentfill}%
\pgfsetfillopacity{0.700000}%
\pgfsetlinewidth{0.000000pt}%
\definecolor{currentstroke}{rgb}{0.000000,0.000000,0.000000}%
\pgfsetstrokecolor{currentstroke}%
\pgfsetdash{}{0pt}%
\pgfpathmoveto{\pgfqpoint{3.078894in}{1.923543in}}%
\pgfpathlineto{\pgfqpoint{3.092050in}{1.914763in}}%
\pgfpathlineto{\pgfqpoint{3.105206in}{1.906188in}}%
\pgfpathlineto{\pgfqpoint{3.118361in}{1.897816in}}%
\pgfpathlineto{\pgfqpoint{3.131516in}{1.889646in}}%
\pgfpathlineto{\pgfqpoint{3.139557in}{1.896445in}}%
\pgfpathlineto{\pgfqpoint{3.147590in}{1.903349in}}%
\pgfpathlineto{\pgfqpoint{3.155614in}{1.910356in}}%
\pgfpathlineto{\pgfqpoint{3.163631in}{1.917461in}}%
\pgfpathlineto{\pgfqpoint{3.150497in}{1.925301in}}%
\pgfpathlineto{\pgfqpoint{3.137364in}{1.933342in}}%
\pgfpathlineto{\pgfqpoint{3.124231in}{1.941588in}}%
\pgfpathlineto{\pgfqpoint{3.111097in}{1.950037in}}%
\pgfpathlineto{\pgfqpoint{3.103059in}{1.943252in}}%
\pgfpathlineto{\pgfqpoint{3.095013in}{1.936572in}}%
\pgfpathlineto{\pgfqpoint{3.086958in}{1.930002in}}%
\pgfpathlineto{\pgfqpoint{3.078894in}{1.923543in}}%
\pgfpathclose%
\pgfusepath{fill}%
\end{pgfscope}%
\begin{pgfscope}%
\pgfpathrectangle{\pgfqpoint{1.254980in}{0.150000in}}{\pgfqpoint{5.490039in}{5.490039in}}%
\pgfusepath{clip}%
\pgfsetbuttcap%
\pgfsetroundjoin%
\definecolor{currentfill}{rgb}{0.283072,0.130895,0.449241}%
\pgfsetfillcolor{currentfill}%
\pgfsetfillopacity{0.700000}%
\pgfsetlinewidth{0.000000pt}%
\definecolor{currentstroke}{rgb}{0.000000,0.000000,0.000000}%
\pgfsetstrokecolor{currentstroke}%
\pgfsetdash{}{0pt}%
\pgfpathmoveto{\pgfqpoint{2.835392in}{2.074099in}}%
\pgfpathlineto{\pgfqpoint{2.848612in}{2.061494in}}%
\pgfpathlineto{\pgfqpoint{2.861829in}{2.049117in}}%
\pgfpathlineto{\pgfqpoint{2.875041in}{2.036966in}}%
\pgfpathlineto{\pgfqpoint{2.888250in}{2.025039in}}%
\pgfpathlineto{\pgfqpoint{2.896430in}{2.029901in}}%
\pgfpathlineto{\pgfqpoint{2.904600in}{2.034909in}}%
\pgfpathlineto{\pgfqpoint{2.912759in}{2.040062in}}%
\pgfpathlineto{\pgfqpoint{2.920908in}{2.045354in}}%
\pgfpathlineto{\pgfqpoint{2.907727in}{2.056917in}}%
\pgfpathlineto{\pgfqpoint{2.894543in}{2.068705in}}%
\pgfpathlineto{\pgfqpoint{2.881355in}{2.080718in}}%
\pgfpathlineto{\pgfqpoint{2.868164in}{2.092959in}}%
\pgfpathlineto{\pgfqpoint{2.859987in}{2.088020in}}%
\pgfpathlineto{\pgfqpoint{2.851799in}{2.083228in}}%
\pgfpathlineto{\pgfqpoint{2.843601in}{2.078586in}}%
\pgfpathlineto{\pgfqpoint{2.835392in}{2.074099in}}%
\pgfpathclose%
\pgfusepath{fill}%
\end{pgfscope}%
\begin{pgfscope}%
\pgfpathrectangle{\pgfqpoint{1.254980in}{0.150000in}}{\pgfqpoint{5.490039in}{5.490039in}}%
\pgfusepath{clip}%
\pgfsetbuttcap%
\pgfsetroundjoin%
\definecolor{currentfill}{rgb}{0.223925,0.334994,0.548053}%
\pgfsetfillcolor{currentfill}%
\pgfsetfillopacity{0.700000}%
\pgfsetlinewidth{0.000000pt}%
\definecolor{currentstroke}{rgb}{0.000000,0.000000,0.000000}%
\pgfsetstrokecolor{currentstroke}%
\pgfsetdash{}{0pt}%
\pgfpathmoveto{\pgfqpoint{2.462610in}{2.527215in}}%
\pgfpathlineto{\pgfqpoint{2.476043in}{2.507430in}}%
\pgfpathlineto{\pgfqpoint{2.489464in}{2.487933in}}%
\pgfpathlineto{\pgfqpoint{2.502874in}{2.468722in}}%
\pgfpathlineto{\pgfqpoint{2.516274in}{2.449792in}}%
\pgfpathlineto{\pgfqpoint{2.524689in}{2.451985in}}%
\pgfpathlineto{\pgfqpoint{2.533089in}{2.454377in}}%
\pgfpathlineto{\pgfqpoint{2.541476in}{2.456964in}}%
\pgfpathlineto{\pgfqpoint{2.549850in}{2.459744in}}%
\pgfpathlineto{\pgfqpoint{2.536488in}{2.478293in}}%
\pgfpathlineto{\pgfqpoint{2.523116in}{2.497125in}}%
\pgfpathlineto{\pgfqpoint{2.509734in}{2.516241in}}%
\pgfpathlineto{\pgfqpoint{2.496341in}{2.535643in}}%
\pgfpathlineto{\pgfqpoint{2.487930in}{2.533233in}}%
\pgfpathlineto{\pgfqpoint{2.479505in}{2.531023in}}%
\pgfpathlineto{\pgfqpoint{2.471065in}{2.529015in}}%
\pgfpathlineto{\pgfqpoint{2.462610in}{2.527215in}}%
\pgfpathclose%
\pgfusepath{fill}%
\end{pgfscope}%
\begin{pgfscope}%
\pgfpathrectangle{\pgfqpoint{1.254980in}{0.150000in}}{\pgfqpoint{5.490039in}{5.490039in}}%
\pgfusepath{clip}%
\pgfsetbuttcap%
\pgfsetroundjoin%
\definecolor{currentfill}{rgb}{0.262138,0.242286,0.520837}%
\pgfsetfillcolor{currentfill}%
\pgfsetfillopacity{0.700000}%
\pgfsetlinewidth{0.000000pt}%
\definecolor{currentstroke}{rgb}{0.000000,0.000000,0.000000}%
\pgfsetstrokecolor{currentstroke}%
\pgfsetdash{}{0pt}%
\pgfpathmoveto{\pgfqpoint{4.243985in}{2.251442in}}%
\pgfpathlineto{\pgfqpoint{4.257332in}{2.255571in}}%
\pgfpathlineto{\pgfqpoint{4.270690in}{2.259865in}}%
\pgfpathlineto{\pgfqpoint{4.284059in}{2.264323in}}%
\pgfpathlineto{\pgfqpoint{4.297439in}{2.268945in}}%
\pgfpathlineto{\pgfqpoint{4.305040in}{2.278670in}}%
\pgfpathlineto{\pgfqpoint{4.312635in}{2.288331in}}%
\pgfpathlineto{\pgfqpoint{4.320225in}{2.297930in}}%
\pgfpathlineto{\pgfqpoint{4.327810in}{2.307466in}}%
\pgfpathlineto{\pgfqpoint{4.314436in}{2.302853in}}%
\pgfpathlineto{\pgfqpoint{4.301072in}{2.298403in}}%
\pgfpathlineto{\pgfqpoint{4.287720in}{2.294117in}}%
\pgfpathlineto{\pgfqpoint{4.274379in}{2.289996in}}%
\pgfpathlineto{\pgfqpoint{4.266788in}{2.280441in}}%
\pgfpathlineto{\pgfqpoint{4.259192in}{2.270831in}}%
\pgfpathlineto{\pgfqpoint{4.251591in}{2.261165in}}%
\pgfpathlineto{\pgfqpoint{4.243985in}{2.251442in}}%
\pgfpathclose%
\pgfusepath{fill}%
\end{pgfscope}%
\begin{pgfscope}%
\pgfpathrectangle{\pgfqpoint{1.254980in}{0.150000in}}{\pgfqpoint{5.490039in}{5.490039in}}%
\pgfusepath{clip}%
\pgfsetbuttcap%
\pgfsetroundjoin%
\definecolor{currentfill}{rgb}{0.128729,0.563265,0.551229}%
\pgfsetfillcolor{currentfill}%
\pgfsetfillopacity{0.700000}%
\pgfsetlinewidth{0.000000pt}%
\definecolor{currentstroke}{rgb}{0.000000,0.000000,0.000000}%
\pgfsetstrokecolor{currentstroke}%
\pgfsetdash{}{0pt}%
\pgfpathmoveto{\pgfqpoint{5.394430in}{3.051968in}}%
\pgfpathlineto{\pgfqpoint{5.408344in}{3.061197in}}%
\pgfpathlineto{\pgfqpoint{5.422276in}{3.070583in}}%
\pgfpathlineto{\pgfqpoint{5.436225in}{3.080123in}}%
\pgfpathlineto{\pgfqpoint{5.450191in}{3.089819in}}%
\pgfpathlineto{\pgfqpoint{5.457258in}{3.093537in}}%
\pgfpathlineto{\pgfqpoint{5.464318in}{3.097261in}}%
\pgfpathlineto{\pgfqpoint{5.471373in}{3.100995in}}%
\pgfpathlineto{\pgfqpoint{5.478421in}{3.104745in}}%
\pgfpathlineto{\pgfqpoint{5.464478in}{3.095500in}}%
\pgfpathlineto{\pgfqpoint{5.450552in}{3.086409in}}%
\pgfpathlineto{\pgfqpoint{5.436643in}{3.077472in}}%
\pgfpathlineto{\pgfqpoint{5.422750in}{3.068690in}}%
\pgfpathlineto{\pgfqpoint{5.415679in}{3.064481in}}%
\pgfpathlineto{\pgfqpoint{5.408601in}{3.060294in}}%
\pgfpathlineto{\pgfqpoint{5.401518in}{3.056125in}}%
\pgfpathlineto{\pgfqpoint{5.394430in}{3.051968in}}%
\pgfpathclose%
\pgfusepath{fill}%
\end{pgfscope}%
\begin{pgfscope}%
\pgfpathrectangle{\pgfqpoint{1.254980in}{0.150000in}}{\pgfqpoint{5.490039in}{5.490039in}}%
\pgfusepath{clip}%
\pgfsetbuttcap%
\pgfsetroundjoin%
\definecolor{currentfill}{rgb}{0.190631,0.407061,0.556089}%
\pgfsetfillcolor{currentfill}%
\pgfsetfillopacity{0.700000}%
\pgfsetlinewidth{0.000000pt}%
\definecolor{currentstroke}{rgb}{0.000000,0.000000,0.000000}%
\pgfsetstrokecolor{currentstroke}%
\pgfsetdash{}{0pt}%
\pgfpathmoveto{\pgfqpoint{4.777356in}{2.633671in}}%
\pgfpathlineto{\pgfqpoint{4.790953in}{2.641022in}}%
\pgfpathlineto{\pgfqpoint{4.804565in}{2.648532in}}%
\pgfpathlineto{\pgfqpoint{4.818190in}{2.656202in}}%
\pgfpathlineto{\pgfqpoint{4.831831in}{2.664031in}}%
\pgfpathlineto{\pgfqpoint{4.839220in}{2.671210in}}%
\pgfpathlineto{\pgfqpoint{4.846603in}{2.678325in}}%
\pgfpathlineto{\pgfqpoint{4.853979in}{2.685377in}}%
\pgfpathlineto{\pgfqpoint{4.861349in}{2.692370in}}%
\pgfpathlineto{\pgfqpoint{4.847720in}{2.684753in}}%
\pgfpathlineto{\pgfqpoint{4.834105in}{2.677296in}}%
\pgfpathlineto{\pgfqpoint{4.820504in}{2.669997in}}%
\pgfpathlineto{\pgfqpoint{4.806917in}{2.662857in}}%
\pgfpathlineto{\pgfqpoint{4.799536in}{2.655642in}}%
\pgfpathlineto{\pgfqpoint{4.792149in}{2.648375in}}%
\pgfpathlineto{\pgfqpoint{4.784756in}{2.641052in}}%
\pgfpathlineto{\pgfqpoint{4.777356in}{2.633671in}}%
\pgfpathclose%
\pgfusepath{fill}%
\end{pgfscope}%
\begin{pgfscope}%
\pgfpathrectangle{\pgfqpoint{1.254980in}{0.150000in}}{\pgfqpoint{5.490039in}{5.490039in}}%
\pgfusepath{clip}%
\pgfsetbuttcap%
\pgfsetroundjoin%
\definecolor{currentfill}{rgb}{0.283091,0.110553,0.431554}%
\pgfsetfillcolor{currentfill}%
\pgfsetfillopacity{0.700000}%
\pgfsetlinewidth{0.000000pt}%
\definecolor{currentstroke}{rgb}{0.000000,0.000000,0.000000}%
\pgfsetstrokecolor{currentstroke}%
\pgfsetdash{}{0pt}%
\pgfpathmoveto{\pgfqpoint{2.888250in}{2.025039in}}%
\pgfpathlineto{\pgfqpoint{2.901456in}{2.013335in}}%
\pgfpathlineto{\pgfqpoint{2.914658in}{2.001854in}}%
\pgfpathlineto{\pgfqpoint{2.927858in}{1.990592in}}%
\pgfpathlineto{\pgfqpoint{2.941054in}{1.979548in}}%
\pgfpathlineto{\pgfqpoint{2.949206in}{1.984783in}}%
\pgfpathlineto{\pgfqpoint{2.957348in}{1.990157in}}%
\pgfpathlineto{\pgfqpoint{2.965480in}{1.995668in}}%
\pgfpathlineto{\pgfqpoint{2.973603in}{2.001312in}}%
\pgfpathlineto{\pgfqpoint{2.960433in}{2.011994in}}%
\pgfpathlineto{\pgfqpoint{2.947261in}{2.022894in}}%
\pgfpathlineto{\pgfqpoint{2.934086in}{2.034013in}}%
\pgfpathlineto{\pgfqpoint{2.920908in}{2.045354in}}%
\pgfpathlineto{\pgfqpoint{2.912759in}{2.040062in}}%
\pgfpathlineto{\pgfqpoint{2.904600in}{2.034909in}}%
\pgfpathlineto{\pgfqpoint{2.896430in}{2.029901in}}%
\pgfpathlineto{\pgfqpoint{2.888250in}{2.025039in}}%
\pgfpathclose%
\pgfusepath{fill}%
\end{pgfscope}%
\begin{pgfscope}%
\pgfpathrectangle{\pgfqpoint{1.254980in}{0.150000in}}{\pgfqpoint{5.490039in}{5.490039in}}%
\pgfusepath{clip}%
\pgfsetbuttcap%
\pgfsetroundjoin%
\definecolor{currentfill}{rgb}{0.274952,0.037752,0.364543}%
\pgfsetfillcolor{currentfill}%
\pgfsetfillopacity{0.700000}%
\pgfsetlinewidth{0.000000pt}%
\definecolor{currentstroke}{rgb}{0.000000,0.000000,0.000000}%
\pgfsetstrokecolor{currentstroke}%
\pgfsetdash{}{0pt}%
\pgfpathmoveto{\pgfqpoint{3.489759in}{1.873056in}}%
\pgfpathlineto{\pgfqpoint{3.502906in}{1.869802in}}%
\pgfpathlineto{\pgfqpoint{3.516057in}{1.866729in}}%
\pgfpathlineto{\pgfqpoint{3.529212in}{1.863837in}}%
\pgfpathlineto{\pgfqpoint{3.542372in}{1.861126in}}%
\pgfpathlineto{\pgfqpoint{3.550228in}{1.870529in}}%
\pgfpathlineto{\pgfqpoint{3.558078in}{1.879963in}}%
\pgfpathlineto{\pgfqpoint{3.565922in}{1.889427in}}%
\pgfpathlineto{\pgfqpoint{3.573761in}{1.898918in}}%
\pgfpathlineto{\pgfqpoint{3.560614in}{1.901386in}}%
\pgfpathlineto{\pgfqpoint{3.547471in}{1.904034in}}%
\pgfpathlineto{\pgfqpoint{3.534333in}{1.906863in}}%
\pgfpathlineto{\pgfqpoint{3.521200in}{1.909874in}}%
\pgfpathlineto{\pgfqpoint{3.513348in}{1.900616in}}%
\pgfpathlineto{\pgfqpoint{3.505491in}{1.891393in}}%
\pgfpathlineto{\pgfqpoint{3.497628in}{1.882205in}}%
\pgfpathlineto{\pgfqpoint{3.489759in}{1.873056in}}%
\pgfpathclose%
\pgfusepath{fill}%
\end{pgfscope}%
\begin{pgfscope}%
\pgfpathrectangle{\pgfqpoint{1.254980in}{0.150000in}}{\pgfqpoint{5.490039in}{5.490039in}}%
\pgfusepath{clip}%
\pgfsetbuttcap%
\pgfsetroundjoin%
\definecolor{currentfill}{rgb}{0.122606,0.585371,0.546557}%
\pgfsetfillcolor{currentfill}%
\pgfsetfillopacity{0.700000}%
\pgfsetlinewidth{0.000000pt}%
\definecolor{currentstroke}{rgb}{0.000000,0.000000,0.000000}%
\pgfsetstrokecolor{currentstroke}%
\pgfsetdash{}{0pt}%
\pgfpathmoveto{\pgfqpoint{5.478421in}{3.104745in}}%
\pgfpathlineto{\pgfqpoint{5.492382in}{3.114145in}}%
\pgfpathlineto{\pgfqpoint{5.506360in}{3.123700in}}%
\pgfpathlineto{\pgfqpoint{5.520356in}{3.133410in}}%
\pgfpathlineto{\pgfqpoint{5.534370in}{3.143275in}}%
\pgfpathlineto{\pgfqpoint{5.541389in}{3.146577in}}%
\pgfpathlineto{\pgfqpoint{5.548402in}{3.149898in}}%
\pgfpathlineto{\pgfqpoint{5.555410in}{3.153243in}}%
\pgfpathlineto{\pgfqpoint{5.562412in}{3.156619in}}%
\pgfpathlineto{\pgfqpoint{5.548424in}{3.147234in}}%
\pgfpathlineto{\pgfqpoint{5.534453in}{3.138004in}}%
\pgfpathlineto{\pgfqpoint{5.520499in}{3.128928in}}%
\pgfpathlineto{\pgfqpoint{5.506563in}{3.120005in}}%
\pgfpathlineto{\pgfqpoint{5.499535in}{3.116141in}}%
\pgfpathlineto{\pgfqpoint{5.492503in}{3.112313in}}%
\pgfpathlineto{\pgfqpoint{5.485465in}{3.108516in}}%
\pgfpathlineto{\pgfqpoint{5.478421in}{3.104745in}}%
\pgfpathclose%
\pgfusepath{fill}%
\end{pgfscope}%
\begin{pgfscope}%
\pgfpathrectangle{\pgfqpoint{1.254980in}{0.150000in}}{\pgfqpoint{5.490039in}{5.490039in}}%
\pgfusepath{clip}%
\pgfsetbuttcap%
\pgfsetroundjoin%
\definecolor{currentfill}{rgb}{0.252194,0.269783,0.531579}%
\pgfsetfillcolor{currentfill}%
\pgfsetfillopacity{0.700000}%
\pgfsetlinewidth{0.000000pt}%
\definecolor{currentstroke}{rgb}{0.000000,0.000000,0.000000}%
\pgfsetstrokecolor{currentstroke}%
\pgfsetdash{}{0pt}%
\pgfpathmoveto{\pgfqpoint{4.327810in}{2.307466in}}%
\pgfpathlineto{\pgfqpoint{4.341196in}{2.312244in}}%
\pgfpathlineto{\pgfqpoint{4.354593in}{2.317185in}}%
\pgfpathlineto{\pgfqpoint{4.368002in}{2.322290in}}%
\pgfpathlineto{\pgfqpoint{4.381422in}{2.327558in}}%
\pgfpathlineto{\pgfqpoint{4.388995in}{2.337006in}}%
\pgfpathlineto{\pgfqpoint{4.396563in}{2.346386in}}%
\pgfpathlineto{\pgfqpoint{4.404125in}{2.355699in}}%
\pgfpathlineto{\pgfqpoint{4.411682in}{2.364945in}}%
\pgfpathlineto{\pgfqpoint{4.398267in}{2.359714in}}%
\pgfpathlineto{\pgfqpoint{4.384865in}{2.354647in}}%
\pgfpathlineto{\pgfqpoint{4.371474in}{2.349742in}}%
\pgfpathlineto{\pgfqpoint{4.358094in}{2.345001in}}%
\pgfpathlineto{\pgfqpoint{4.350531in}{2.335708in}}%
\pgfpathlineto{\pgfqpoint{4.342963in}{2.326354in}}%
\pgfpathlineto{\pgfqpoint{4.335389in}{2.316941in}}%
\pgfpathlineto{\pgfqpoint{4.327810in}{2.307466in}}%
\pgfpathclose%
\pgfusepath{fill}%
\end{pgfscope}%
\begin{pgfscope}%
\pgfpathrectangle{\pgfqpoint{1.254980in}{0.150000in}}{\pgfqpoint{5.490039in}{5.490039in}}%
\pgfusepath{clip}%
\pgfsetbuttcap%
\pgfsetroundjoin%
\definecolor{currentfill}{rgb}{0.119738,0.603785,0.541400}%
\pgfsetfillcolor{currentfill}%
\pgfsetfillopacity{0.700000}%
\pgfsetlinewidth{0.000000pt}%
\definecolor{currentstroke}{rgb}{0.000000,0.000000,0.000000}%
\pgfsetstrokecolor{currentstroke}%
\pgfsetdash{}{0pt}%
\pgfpathmoveto{\pgfqpoint{5.562412in}{3.156619in}}%
\pgfpathlineto{\pgfqpoint{5.576419in}{3.166157in}}%
\pgfpathlineto{\pgfqpoint{5.590442in}{3.175850in}}%
\pgfpathlineto{\pgfqpoint{5.604484in}{3.185698in}}%
\pgfpathlineto{\pgfqpoint{5.618544in}{3.195699in}}%
\pgfpathlineto{\pgfqpoint{5.625515in}{3.198611in}}%
\pgfpathlineto{\pgfqpoint{5.632481in}{3.201556in}}%
\pgfpathlineto{\pgfqpoint{5.639442in}{3.204542in}}%
\pgfpathlineto{\pgfqpoint{5.646398in}{3.207572in}}%
\pgfpathlineto{\pgfqpoint{5.632365in}{3.198080in}}%
\pgfpathlineto{\pgfqpoint{5.618350in}{3.188743in}}%
\pgfpathlineto{\pgfqpoint{5.604352in}{3.179558in}}%
\pgfpathlineto{\pgfqpoint{5.590372in}{3.170527in}}%
\pgfpathlineto{\pgfqpoint{5.583389in}{3.166978in}}%
\pgfpathlineto{\pgfqpoint{5.576402in}{3.163481in}}%
\pgfpathlineto{\pgfqpoint{5.569410in}{3.160029in}}%
\pgfpathlineto{\pgfqpoint{5.562412in}{3.156619in}}%
\pgfpathclose%
\pgfusepath{fill}%
\end{pgfscope}%
\begin{pgfscope}%
\pgfpathrectangle{\pgfqpoint{1.254980in}{0.150000in}}{\pgfqpoint{5.490039in}{5.490039in}}%
\pgfusepath{clip}%
\pgfsetbuttcap%
\pgfsetroundjoin%
\definecolor{currentfill}{rgb}{0.206756,0.371758,0.553117}%
\pgfsetfillcolor{currentfill}%
\pgfsetfillopacity{0.700000}%
\pgfsetlinewidth{0.000000pt}%
\definecolor{currentstroke}{rgb}{0.000000,0.000000,0.000000}%
\pgfsetstrokecolor{currentstroke}%
\pgfsetdash{}{0pt}%
\pgfpathmoveto{\pgfqpoint{2.408763in}{2.609288in}}%
\pgfpathlineto{\pgfqpoint{2.422243in}{2.588324in}}%
\pgfpathlineto{\pgfqpoint{2.435711in}{2.567659in}}%
\pgfpathlineto{\pgfqpoint{2.449166in}{2.547290in}}%
\pgfpathlineto{\pgfqpoint{2.462610in}{2.527215in}}%
\pgfpathlineto{\pgfqpoint{2.471065in}{2.529015in}}%
\pgfpathlineto{\pgfqpoint{2.479505in}{2.531023in}}%
\pgfpathlineto{\pgfqpoint{2.487930in}{2.533233in}}%
\pgfpathlineto{\pgfqpoint{2.496341in}{2.535643in}}%
\pgfpathlineto{\pgfqpoint{2.482937in}{2.555336in}}%
\pgfpathlineto{\pgfqpoint{2.469521in}{2.575321in}}%
\pgfpathlineto{\pgfqpoint{2.456094in}{2.595602in}}%
\pgfpathlineto{\pgfqpoint{2.442655in}{2.616181in}}%
\pgfpathlineto{\pgfqpoint{2.434204in}{2.614143in}}%
\pgfpathlineto{\pgfqpoint{2.425739in}{2.612313in}}%
\pgfpathlineto{\pgfqpoint{2.417259in}{2.610693in}}%
\pgfpathlineto{\pgfqpoint{2.408763in}{2.609288in}}%
\pgfpathclose%
\pgfusepath{fill}%
\end{pgfscope}%
\begin{pgfscope}%
\pgfpathrectangle{\pgfqpoint{1.254980in}{0.150000in}}{\pgfqpoint{5.490039in}{5.490039in}}%
\pgfusepath{clip}%
\pgfsetbuttcap%
\pgfsetroundjoin%
\definecolor{currentfill}{rgb}{0.272594,0.025563,0.353093}%
\pgfsetfillcolor{currentfill}%
\pgfsetfillopacity{0.700000}%
\pgfsetlinewidth{0.000000pt}%
\definecolor{currentstroke}{rgb}{0.000000,0.000000,0.000000}%
\pgfsetstrokecolor{currentstroke}%
\pgfsetdash{}{0pt}%
\pgfpathmoveto{\pgfqpoint{3.268722in}{1.861892in}}%
\pgfpathlineto{\pgfqpoint{3.281864in}{1.855824in}}%
\pgfpathlineto{\pgfqpoint{3.295008in}{1.849948in}}%
\pgfpathlineto{\pgfqpoint{3.308154in}{1.844262in}}%
\pgfpathlineto{\pgfqpoint{3.321303in}{1.838766in}}%
\pgfpathlineto{\pgfqpoint{3.329254in}{1.846901in}}%
\pgfpathlineto{\pgfqpoint{3.337198in}{1.855108in}}%
\pgfpathlineto{\pgfqpoint{3.345136in}{1.863385in}}%
\pgfpathlineto{\pgfqpoint{3.353067in}{1.871729in}}%
\pgfpathlineto{\pgfqpoint{3.339936in}{1.876925in}}%
\pgfpathlineto{\pgfqpoint{3.326807in}{1.882310in}}%
\pgfpathlineto{\pgfqpoint{3.313681in}{1.887887in}}%
\pgfpathlineto{\pgfqpoint{3.300556in}{1.893655in}}%
\pgfpathlineto{\pgfqpoint{3.292608in}{1.885601in}}%
\pgfpathlineto{\pgfqpoint{3.284654in}{1.877620in}}%
\pgfpathlineto{\pgfqpoint{3.276692in}{1.869717in}}%
\pgfpathlineto{\pgfqpoint{3.268722in}{1.861892in}}%
\pgfpathclose%
\pgfusepath{fill}%
\end{pgfscope}%
\begin{pgfscope}%
\pgfpathrectangle{\pgfqpoint{1.254980in}{0.150000in}}{\pgfqpoint{5.490039in}{5.490039in}}%
\pgfusepath{clip}%
\pgfsetbuttcap%
\pgfsetroundjoin%
\definecolor{currentfill}{rgb}{0.180629,0.429975,0.557282}%
\pgfsetfillcolor{currentfill}%
\pgfsetfillopacity{0.700000}%
\pgfsetlinewidth{0.000000pt}%
\definecolor{currentstroke}{rgb}{0.000000,0.000000,0.000000}%
\pgfsetstrokecolor{currentstroke}%
\pgfsetdash{}{0pt}%
\pgfpathmoveto{\pgfqpoint{4.861349in}{2.692370in}}%
\pgfpathlineto{\pgfqpoint{4.874994in}{2.700146in}}%
\pgfpathlineto{\pgfqpoint{4.888653in}{2.708080in}}%
\pgfpathlineto{\pgfqpoint{4.902327in}{2.716174in}}%
\pgfpathlineto{\pgfqpoint{4.916016in}{2.724426in}}%
\pgfpathlineto{\pgfqpoint{4.923368in}{2.731131in}}%
\pgfpathlineto{\pgfqpoint{4.930713in}{2.737775in}}%
\pgfpathlineto{\pgfqpoint{4.938052in}{2.744361in}}%
\pgfpathlineto{\pgfqpoint{4.945384in}{2.750891in}}%
\pgfpathlineto{\pgfqpoint{4.931707in}{2.742881in}}%
\pgfpathlineto{\pgfqpoint{4.918045in}{2.735030in}}%
\pgfpathlineto{\pgfqpoint{4.904398in}{2.727336in}}%
\pgfpathlineto{\pgfqpoint{4.890765in}{2.719802in}}%
\pgfpathlineto{\pgfqpoint{4.883421in}{2.713019in}}%
\pgfpathlineto{\pgfqpoint{4.876070in}{2.706188in}}%
\pgfpathlineto{\pgfqpoint{4.868713in}{2.699306in}}%
\pgfpathlineto{\pgfqpoint{4.861349in}{2.692370in}}%
\pgfpathclose%
\pgfusepath{fill}%
\end{pgfscope}%
\begin{pgfscope}%
\pgfpathrectangle{\pgfqpoint{1.254980in}{0.150000in}}{\pgfqpoint{5.490039in}{5.490039in}}%
\pgfusepath{clip}%
\pgfsetbuttcap%
\pgfsetroundjoin%
\definecolor{currentfill}{rgb}{0.120081,0.622161,0.534946}%
\pgfsetfillcolor{currentfill}%
\pgfsetfillopacity{0.700000}%
\pgfsetlinewidth{0.000000pt}%
\definecolor{currentstroke}{rgb}{0.000000,0.000000,0.000000}%
\pgfsetstrokecolor{currentstroke}%
\pgfsetdash{}{0pt}%
\pgfpathmoveto{\pgfqpoint{5.646398in}{3.207572in}}%
\pgfpathlineto{\pgfqpoint{5.660448in}{3.217217in}}%
\pgfpathlineto{\pgfqpoint{5.674517in}{3.227016in}}%
\pgfpathlineto{\pgfqpoint{5.688604in}{3.236969in}}%
\pgfpathlineto{\pgfqpoint{5.702710in}{3.247076in}}%
\pgfpathlineto{\pgfqpoint{5.709632in}{3.249628in}}%
\pgfpathlineto{\pgfqpoint{5.716550in}{3.252230in}}%
\pgfpathlineto{\pgfqpoint{5.723464in}{3.254888in}}%
\pgfpathlineto{\pgfqpoint{5.730373in}{3.257608in}}%
\pgfpathlineto{\pgfqpoint{5.716296in}{3.248042in}}%
\pgfpathlineto{\pgfqpoint{5.702238in}{3.238628in}}%
\pgfpathlineto{\pgfqpoint{5.688198in}{3.229367in}}%
\pgfpathlineto{\pgfqpoint{5.674176in}{3.220259in}}%
\pgfpathlineto{\pgfqpoint{5.667237in}{3.216991in}}%
\pgfpathlineto{\pgfqpoint{5.660295in}{3.213791in}}%
\pgfpathlineto{\pgfqpoint{5.653349in}{3.210653in}}%
\pgfpathlineto{\pgfqpoint{5.646398in}{3.207572in}}%
\pgfpathclose%
\pgfusepath{fill}%
\end{pgfscope}%
\begin{pgfscope}%
\pgfpathrectangle{\pgfqpoint{1.254980in}{0.150000in}}{\pgfqpoint{5.490039in}{5.490039in}}%
\pgfusepath{clip}%
\pgfsetbuttcap%
\pgfsetroundjoin%
\definecolor{currentfill}{rgb}{0.241237,0.296485,0.539709}%
\pgfsetfillcolor{currentfill}%
\pgfsetfillopacity{0.700000}%
\pgfsetlinewidth{0.000000pt}%
\definecolor{currentstroke}{rgb}{0.000000,0.000000,0.000000}%
\pgfsetstrokecolor{currentstroke}%
\pgfsetdash{}{0pt}%
\pgfpathmoveto{\pgfqpoint{4.411682in}{2.364945in}}%
\pgfpathlineto{\pgfqpoint{4.425108in}{2.370339in}}%
\pgfpathlineto{\pgfqpoint{4.438546in}{2.375895in}}%
\pgfpathlineto{\pgfqpoint{4.451997in}{2.381614in}}%
\pgfpathlineto{\pgfqpoint{4.465460in}{2.387495in}}%
\pgfpathlineto{\pgfqpoint{4.473005in}{2.396620in}}%
\pgfpathlineto{\pgfqpoint{4.480544in}{2.405674in}}%
\pgfpathlineto{\pgfqpoint{4.488077in}{2.414657in}}%
\pgfpathlineto{\pgfqpoint{4.495604in}{2.423570in}}%
\pgfpathlineto{\pgfqpoint{4.482147in}{2.417755in}}%
\pgfpathlineto{\pgfqpoint{4.468703in}{2.412102in}}%
\pgfpathlineto{\pgfqpoint{4.455272in}{2.406612in}}%
\pgfpathlineto{\pgfqpoint{4.441852in}{2.401284in}}%
\pgfpathlineto{\pgfqpoint{4.434318in}{2.392294in}}%
\pgfpathlineto{\pgfqpoint{4.426778in}{2.383242in}}%
\pgfpathlineto{\pgfqpoint{4.419232in}{2.374126in}}%
\pgfpathlineto{\pgfqpoint{4.411682in}{2.364945in}}%
\pgfpathclose%
\pgfusepath{fill}%
\end{pgfscope}%
\begin{pgfscope}%
\pgfpathrectangle{\pgfqpoint{1.254980in}{0.150000in}}{\pgfqpoint{5.490039in}{5.490039in}}%
\pgfusepath{clip}%
\pgfsetbuttcap%
\pgfsetroundjoin%
\definecolor{currentfill}{rgb}{0.191090,0.708366,0.482284}%
\pgfsetfillcolor{currentfill}%
\pgfsetfillopacity{0.700000}%
\pgfsetlinewidth{0.000000pt}%
\definecolor{currentstroke}{rgb}{0.000000,0.000000,0.000000}%
\pgfsetstrokecolor{currentstroke}%
\pgfsetdash{}{0pt}%
\pgfpathmoveto{\pgfqpoint{6.066115in}{3.449390in}}%
\pgfpathlineto{\pgfqpoint{6.080371in}{3.459093in}}%
\pgfpathlineto{\pgfqpoint{6.094646in}{3.468947in}}%
\pgfpathlineto{\pgfqpoint{6.108941in}{3.478953in}}%
\pgfpathlineto{\pgfqpoint{6.115639in}{3.480521in}}%
\pgfpathlineto{\pgfqpoint{6.122337in}{3.482240in}}%
\pgfpathlineto{\pgfqpoint{6.129034in}{3.484117in}}%
\pgfpathlineto{\pgfqpoint{6.135732in}{3.486158in}}%
\pgfpathlineto{\pgfqpoint{6.121478in}{3.476839in}}%
\pgfpathlineto{\pgfqpoint{6.107243in}{3.467670in}}%
\pgfpathlineto{\pgfqpoint{6.093026in}{3.458651in}}%
\pgfpathlineto{\pgfqpoint{6.086298in}{3.456089in}}%
\pgfpathlineto{\pgfqpoint{6.079570in}{3.453696in}}%
\pgfpathlineto{\pgfqpoint{6.072843in}{3.451465in}}%
\pgfpathlineto{\pgfqpoint{6.066115in}{3.449390in}}%
\pgfpathclose%
\pgfusepath{fill}%
\end{pgfscope}%
\begin{pgfscope}%
\pgfpathrectangle{\pgfqpoint{1.254980in}{0.150000in}}{\pgfqpoint{5.490039in}{5.490039in}}%
\pgfusepath{clip}%
\pgfsetbuttcap%
\pgfsetroundjoin%
\definecolor{currentfill}{rgb}{0.276022,0.044167,0.370164}%
\pgfsetfillcolor{currentfill}%
\pgfsetfillopacity{0.700000}%
\pgfsetlinewidth{0.000000pt}%
\definecolor{currentstroke}{rgb}{0.000000,0.000000,0.000000}%
\pgfsetstrokecolor{currentstroke}%
\pgfsetdash{}{0pt}%
\pgfpathmoveto{\pgfqpoint{3.131516in}{1.889646in}}%
\pgfpathlineto{\pgfqpoint{3.144671in}{1.881678in}}%
\pgfpathlineto{\pgfqpoint{3.157826in}{1.873910in}}%
\pgfpathlineto{\pgfqpoint{3.170981in}{1.866341in}}%
\pgfpathlineto{\pgfqpoint{3.184137in}{1.858970in}}%
\pgfpathlineto{\pgfqpoint{3.192157in}{1.866109in}}%
\pgfpathlineto{\pgfqpoint{3.200169in}{1.873346in}}%
\pgfpathlineto{\pgfqpoint{3.208173in}{1.880678in}}%
\pgfpathlineto{\pgfqpoint{3.216169in}{1.888103in}}%
\pgfpathlineto{\pgfqpoint{3.203033in}{1.895144in}}%
\pgfpathlineto{\pgfqpoint{3.189898in}{1.902384in}}%
\pgfpathlineto{\pgfqpoint{3.176764in}{1.909823in}}%
\pgfpathlineto{\pgfqpoint{3.163631in}{1.917461in}}%
\pgfpathlineto{\pgfqpoint{3.155614in}{1.910356in}}%
\pgfpathlineto{\pgfqpoint{3.147590in}{1.903349in}}%
\pgfpathlineto{\pgfqpoint{3.139557in}{1.896445in}}%
\pgfpathlineto{\pgfqpoint{3.131516in}{1.889646in}}%
\pgfpathclose%
\pgfusepath{fill}%
\end{pgfscope}%
\begin{pgfscope}%
\pgfpathrectangle{\pgfqpoint{1.254980in}{0.150000in}}{\pgfqpoint{5.490039in}{5.490039in}}%
\pgfusepath{clip}%
\pgfsetbuttcap%
\pgfsetroundjoin%
\definecolor{currentfill}{rgb}{0.124780,0.640461,0.527068}%
\pgfsetfillcolor{currentfill}%
\pgfsetfillopacity{0.700000}%
\pgfsetlinewidth{0.000000pt}%
\definecolor{currentstroke}{rgb}{0.000000,0.000000,0.000000}%
\pgfsetstrokecolor{currentstroke}%
\pgfsetdash{}{0pt}%
\pgfpathmoveto{\pgfqpoint{5.730373in}{3.257608in}}%
\pgfpathlineto{\pgfqpoint{5.744467in}{3.267328in}}%
\pgfpathlineto{\pgfqpoint{5.758580in}{3.277201in}}%
\pgfpathlineto{\pgfqpoint{5.772711in}{3.287228in}}%
\pgfpathlineto{\pgfqpoint{5.786861in}{3.297408in}}%
\pgfpathlineto{\pgfqpoint{5.793735in}{3.299637in}}%
\pgfpathlineto{\pgfqpoint{5.800605in}{3.301933in}}%
\pgfpathlineto{\pgfqpoint{5.807472in}{3.304303in}}%
\pgfpathlineto{\pgfqpoint{5.814334in}{3.306753in}}%
\pgfpathlineto{\pgfqpoint{5.800216in}{3.297142in}}%
\pgfpathlineto{\pgfqpoint{5.786115in}{3.287685in}}%
\pgfpathlineto{\pgfqpoint{5.772034in}{3.278379in}}%
\pgfpathlineto{\pgfqpoint{5.757970in}{3.269226in}}%
\pgfpathlineto{\pgfqpoint{5.751076in}{3.266199in}}%
\pgfpathlineto{\pgfqpoint{5.744178in}{3.263258in}}%
\pgfpathlineto{\pgfqpoint{5.737277in}{3.260396in}}%
\pgfpathlineto{\pgfqpoint{5.730373in}{3.257608in}}%
\pgfpathclose%
\pgfusepath{fill}%
\end{pgfscope}%
\begin{pgfscope}%
\pgfpathrectangle{\pgfqpoint{1.254980in}{0.150000in}}{\pgfqpoint{5.490039in}{5.490039in}}%
\pgfusepath{clip}%
\pgfsetbuttcap%
\pgfsetroundjoin%
\definecolor{currentfill}{rgb}{0.281924,0.089666,0.412415}%
\pgfsetfillcolor{currentfill}%
\pgfsetfillopacity{0.700000}%
\pgfsetlinewidth{0.000000pt}%
\definecolor{currentstroke}{rgb}{0.000000,0.000000,0.000000}%
\pgfsetstrokecolor{currentstroke}%
\pgfsetdash{}{0pt}%
\pgfpathmoveto{\pgfqpoint{2.941054in}{1.979548in}}%
\pgfpathlineto{\pgfqpoint{2.954248in}{1.968722in}}%
\pgfpathlineto{\pgfqpoint{2.967440in}{1.958111in}}%
\pgfpathlineto{\pgfqpoint{2.980629in}{1.947714in}}%
\pgfpathlineto{\pgfqpoint{2.993816in}{1.937531in}}%
\pgfpathlineto{\pgfqpoint{3.001942in}{1.943137in}}%
\pgfpathlineto{\pgfqpoint{3.010057in}{1.948876in}}%
\pgfpathlineto{\pgfqpoint{3.018164in}{1.954744in}}%
\pgfpathlineto{\pgfqpoint{3.026260in}{1.960738in}}%
\pgfpathlineto{\pgfqpoint{3.013099in}{1.970561in}}%
\pgfpathlineto{\pgfqpoint{2.999935in}{1.980597in}}%
\pgfpathlineto{\pgfqpoint{2.986770in}{1.990847in}}%
\pgfpathlineto{\pgfqpoint{2.973603in}{2.001312in}}%
\pgfpathlineto{\pgfqpoint{2.965480in}{1.995668in}}%
\pgfpathlineto{\pgfqpoint{2.957348in}{1.990157in}}%
\pgfpathlineto{\pgfqpoint{2.949206in}{1.984783in}}%
\pgfpathlineto{\pgfqpoint{2.941054in}{1.979548in}}%
\pgfpathclose%
\pgfusepath{fill}%
\end{pgfscope}%
\begin{pgfscope}%
\pgfpathrectangle{\pgfqpoint{1.254980in}{0.150000in}}{\pgfqpoint{5.490039in}{5.490039in}}%
\pgfusepath{clip}%
\pgfsetbuttcap%
\pgfsetroundjoin%
\definecolor{currentfill}{rgb}{0.137339,0.662252,0.515571}%
\pgfsetfillcolor{currentfill}%
\pgfsetfillopacity{0.700000}%
\pgfsetlinewidth{0.000000pt}%
\definecolor{currentstroke}{rgb}{0.000000,0.000000,0.000000}%
\pgfsetstrokecolor{currentstroke}%
\pgfsetdash{}{0pt}%
\pgfpathmoveto{\pgfqpoint{5.814334in}{3.306753in}}%
\pgfpathlineto{\pgfqpoint{5.828471in}{3.316515in}}%
\pgfpathlineto{\pgfqpoint{5.842627in}{3.326431in}}%
\pgfpathlineto{\pgfqpoint{5.856801in}{3.336500in}}%
\pgfpathlineto{\pgfqpoint{5.870994in}{3.346722in}}%
\pgfpathlineto{\pgfqpoint{5.877820in}{3.348669in}}%
\pgfpathlineto{\pgfqpoint{5.884643in}{3.350701in}}%
\pgfpathlineto{\pgfqpoint{5.891462in}{3.352826in}}%
\pgfpathlineto{\pgfqpoint{5.898279in}{3.355050in}}%
\pgfpathlineto{\pgfqpoint{5.884120in}{3.345428in}}%
\pgfpathlineto{\pgfqpoint{5.869979in}{3.335957in}}%
\pgfpathlineto{\pgfqpoint{5.855856in}{3.326639in}}%
\pgfpathlineto{\pgfqpoint{5.841752in}{3.317473in}}%
\pgfpathlineto{\pgfqpoint{5.834902in}{3.314641in}}%
\pgfpathlineto{\pgfqpoint{5.828049in}{3.311915in}}%
\pgfpathlineto{\pgfqpoint{5.821193in}{3.309288in}}%
\pgfpathlineto{\pgfqpoint{5.814334in}{3.306753in}}%
\pgfpathclose%
\pgfusepath{fill}%
\end{pgfscope}%
\begin{pgfscope}%
\pgfpathrectangle{\pgfqpoint{1.254980in}{0.150000in}}{\pgfqpoint{5.490039in}{5.490039in}}%
\pgfusepath{clip}%
\pgfsetbuttcap%
\pgfsetroundjoin%
\definecolor{currentfill}{rgb}{0.272594,0.025563,0.353093}%
\pgfsetfillcolor{currentfill}%
\pgfsetfillopacity{0.700000}%
\pgfsetlinewidth{0.000000pt}%
\definecolor{currentstroke}{rgb}{0.000000,0.000000,0.000000}%
\pgfsetstrokecolor{currentstroke}%
\pgfsetdash{}{0pt}%
\pgfpathmoveto{\pgfqpoint{3.405619in}{1.852825in}}%
\pgfpathlineto{\pgfqpoint{3.418764in}{1.848565in}}%
\pgfpathlineto{\pgfqpoint{3.431914in}{1.844490in}}%
\pgfpathlineto{\pgfqpoint{3.445066in}{1.840599in}}%
\pgfpathlineto{\pgfqpoint{3.458223in}{1.836890in}}%
\pgfpathlineto{\pgfqpoint{3.466116in}{1.845863in}}%
\pgfpathlineto{\pgfqpoint{3.474003in}{1.854883in}}%
\pgfpathlineto{\pgfqpoint{3.481884in}{1.863948in}}%
\pgfpathlineto{\pgfqpoint{3.489759in}{1.873056in}}%
\pgfpathlineto{\pgfqpoint{3.476617in}{1.876493in}}%
\pgfpathlineto{\pgfqpoint{3.463479in}{1.880113in}}%
\pgfpathlineto{\pgfqpoint{3.450344in}{1.883916in}}%
\pgfpathlineto{\pgfqpoint{3.437213in}{1.887905in}}%
\pgfpathlineto{\pgfqpoint{3.429324in}{1.879058in}}%
\pgfpathlineto{\pgfqpoint{3.421429in}{1.870261in}}%
\pgfpathlineto{\pgfqpoint{3.413527in}{1.861516in}}%
\pgfpathlineto{\pgfqpoint{3.405619in}{1.852825in}}%
\pgfpathclose%
\pgfusepath{fill}%
\end{pgfscope}%
\begin{pgfscope}%
\pgfpathrectangle{\pgfqpoint{1.254980in}{0.150000in}}{\pgfqpoint{5.490039in}{5.490039in}}%
\pgfusepath{clip}%
\pgfsetbuttcap%
\pgfsetroundjoin%
\definecolor{currentfill}{rgb}{0.153894,0.680203,0.504172}%
\pgfsetfillcolor{currentfill}%
\pgfsetfillopacity{0.700000}%
\pgfsetlinewidth{0.000000pt}%
\definecolor{currentstroke}{rgb}{0.000000,0.000000,0.000000}%
\pgfsetstrokecolor{currentstroke}%
\pgfsetdash{}{0pt}%
\pgfpathmoveto{\pgfqpoint{5.898279in}{3.355050in}}%
\pgfpathlineto{\pgfqpoint{5.912457in}{3.364824in}}%
\pgfpathlineto{\pgfqpoint{5.926654in}{3.374751in}}%
\pgfpathlineto{\pgfqpoint{5.940870in}{3.384830in}}%
\pgfpathlineto{\pgfqpoint{5.955106in}{3.395062in}}%
\pgfpathlineto{\pgfqpoint{5.961884in}{3.396773in}}%
\pgfpathlineto{\pgfqpoint{5.968660in}{3.398590in}}%
\pgfpathlineto{\pgfqpoint{5.975434in}{3.400519in}}%
\pgfpathlineto{\pgfqpoint{5.982206in}{3.402567in}}%
\pgfpathlineto{\pgfqpoint{5.968007in}{3.392964in}}%
\pgfpathlineto{\pgfqpoint{5.953827in}{3.383513in}}%
\pgfpathlineto{\pgfqpoint{5.939665in}{3.374213in}}%
\pgfpathlineto{\pgfqpoint{5.925522in}{3.365064in}}%
\pgfpathlineto{\pgfqpoint{5.918714in}{3.362379in}}%
\pgfpathlineto{\pgfqpoint{5.911905in}{3.359820in}}%
\pgfpathlineto{\pgfqpoint{5.905093in}{3.357379in}}%
\pgfpathlineto{\pgfqpoint{5.898279in}{3.355050in}}%
\pgfpathclose%
\pgfusepath{fill}%
\end{pgfscope}%
\begin{pgfscope}%
\pgfpathrectangle{\pgfqpoint{1.254980in}{0.150000in}}{\pgfqpoint{5.490039in}{5.490039in}}%
\pgfusepath{clip}%
\pgfsetbuttcap%
\pgfsetroundjoin%
\definecolor{currentfill}{rgb}{0.169646,0.456262,0.558030}%
\pgfsetfillcolor{currentfill}%
\pgfsetfillopacity{0.700000}%
\pgfsetlinewidth{0.000000pt}%
\definecolor{currentstroke}{rgb}{0.000000,0.000000,0.000000}%
\pgfsetstrokecolor{currentstroke}%
\pgfsetdash{}{0pt}%
\pgfpathmoveto{\pgfqpoint{4.945384in}{2.750891in}}%
\pgfpathlineto{\pgfqpoint{4.959077in}{2.759060in}}%
\pgfpathlineto{\pgfqpoint{4.972784in}{2.767387in}}%
\pgfpathlineto{\pgfqpoint{4.986507in}{2.775872in}}%
\pgfpathlineto{\pgfqpoint{5.000245in}{2.784515in}}%
\pgfpathlineto{\pgfqpoint{5.007558in}{2.790732in}}%
\pgfpathlineto{\pgfqpoint{5.014864in}{2.796893in}}%
\pgfpathlineto{\pgfqpoint{5.022164in}{2.803001in}}%
\pgfpathlineto{\pgfqpoint{5.029457in}{2.809059in}}%
\pgfpathlineto{\pgfqpoint{5.015732in}{2.800688in}}%
\pgfpathlineto{\pgfqpoint{5.002022in}{2.792475in}}%
\pgfpathlineto{\pgfqpoint{4.988328in}{2.784419in}}%
\pgfpathlineto{\pgfqpoint{4.974648in}{2.776521in}}%
\pgfpathlineto{\pgfqpoint{4.967342in}{2.770181in}}%
\pgfpathlineto{\pgfqpoint{4.960029in}{2.763799in}}%
\pgfpathlineto{\pgfqpoint{4.952710in}{2.757370in}}%
\pgfpathlineto{\pgfqpoint{4.945384in}{2.750891in}}%
\pgfpathclose%
\pgfusepath{fill}%
\end{pgfscope}%
\begin{pgfscope}%
\pgfpathrectangle{\pgfqpoint{1.254980in}{0.150000in}}{\pgfqpoint{5.490039in}{5.490039in}}%
\pgfusepath{clip}%
\pgfsetbuttcap%
\pgfsetroundjoin%
\definecolor{currentfill}{rgb}{0.170948,0.694384,0.493803}%
\pgfsetfillcolor{currentfill}%
\pgfsetfillopacity{0.700000}%
\pgfsetlinewidth{0.000000pt}%
\definecolor{currentstroke}{rgb}{0.000000,0.000000,0.000000}%
\pgfsetstrokecolor{currentstroke}%
\pgfsetdash{}{0pt}%
\pgfpathmoveto{\pgfqpoint{5.982206in}{3.402567in}}%
\pgfpathlineto{\pgfqpoint{5.996424in}{3.412321in}}%
\pgfpathlineto{\pgfqpoint{6.010661in}{3.422227in}}%
\pgfpathlineto{\pgfqpoint{6.024917in}{3.432285in}}%
\pgfpathlineto{\pgfqpoint{6.039193in}{3.442496in}}%
\pgfpathlineto{\pgfqpoint{6.045926in}{3.444022in}}%
\pgfpathlineto{\pgfqpoint{6.052656in}{3.445676in}}%
\pgfpathlineto{\pgfqpoint{6.059386in}{3.447462in}}%
\pgfpathlineto{\pgfqpoint{6.066115in}{3.449390in}}%
\pgfpathlineto{\pgfqpoint{6.051877in}{3.439838in}}%
\pgfpathlineto{\pgfqpoint{6.037659in}{3.430437in}}%
\pgfpathlineto{\pgfqpoint{6.023460in}{3.421187in}}%
\pgfpathlineto{\pgfqpoint{6.009280in}{3.412088in}}%
\pgfpathlineto{\pgfqpoint{6.002513in}{3.409494in}}%
\pgfpathlineto{\pgfqpoint{5.995745in}{3.407047in}}%
\pgfpathlineto{\pgfqpoint{5.988976in}{3.404741in}}%
\pgfpathlineto{\pgfqpoint{5.982206in}{3.402567in}}%
\pgfpathclose%
\pgfusepath{fill}%
\end{pgfscope}%
\begin{pgfscope}%
\pgfpathrectangle{\pgfqpoint{1.254980in}{0.150000in}}{\pgfqpoint{5.490039in}{5.490039in}}%
\pgfusepath{clip}%
\pgfsetbuttcap%
\pgfsetroundjoin%
\definecolor{currentfill}{rgb}{0.282656,0.100196,0.422160}%
\pgfsetfillcolor{currentfill}%
\pgfsetfillopacity{0.700000}%
\pgfsetlinewidth{0.000000pt}%
\definecolor{currentstroke}{rgb}{0.000000,0.000000,0.000000}%
\pgfsetstrokecolor{currentstroke}%
\pgfsetdash{}{0pt}%
\pgfpathmoveto{\pgfqpoint{3.794230in}{1.964653in}}%
\pgfpathlineto{\pgfqpoint{3.807437in}{1.964875in}}%
\pgfpathlineto{\pgfqpoint{3.820650in}{1.965269in}}%
\pgfpathlineto{\pgfqpoint{3.833871in}{1.965834in}}%
\pgfpathlineto{\pgfqpoint{3.847099in}{1.966571in}}%
\pgfpathlineto{\pgfqpoint{3.854852in}{1.976908in}}%
\pgfpathlineto{\pgfqpoint{3.862600in}{1.987228in}}%
\pgfpathlineto{\pgfqpoint{3.870342in}{1.997529in}}%
\pgfpathlineto{\pgfqpoint{3.878080in}{2.007810in}}%
\pgfpathlineto{\pgfqpoint{3.864859in}{2.006913in}}%
\pgfpathlineto{\pgfqpoint{3.851646in}{2.006187in}}%
\pgfpathlineto{\pgfqpoint{3.838441in}{2.005633in}}%
\pgfpathlineto{\pgfqpoint{3.825243in}{2.005250in}}%
\pgfpathlineto{\pgfqpoint{3.817498in}{1.995119in}}%
\pgfpathlineto{\pgfqpoint{3.809747in}{1.984975in}}%
\pgfpathlineto{\pgfqpoint{3.801991in}{1.974819in}}%
\pgfpathlineto{\pgfqpoint{3.794230in}{1.964653in}}%
\pgfpathclose%
\pgfusepath{fill}%
\end{pgfscope}%
\begin{pgfscope}%
\pgfpathrectangle{\pgfqpoint{1.254980in}{0.150000in}}{\pgfqpoint{5.490039in}{5.490039in}}%
\pgfusepath{clip}%
\pgfsetbuttcap%
\pgfsetroundjoin%
\definecolor{currentfill}{rgb}{0.283187,0.125848,0.444960}%
\pgfsetfillcolor{currentfill}%
\pgfsetfillopacity{0.700000}%
\pgfsetlinewidth{0.000000pt}%
\definecolor{currentstroke}{rgb}{0.000000,0.000000,0.000000}%
\pgfsetstrokecolor{currentstroke}%
\pgfsetdash{}{0pt}%
\pgfpathmoveto{\pgfqpoint{3.878080in}{2.007810in}}%
\pgfpathlineto{\pgfqpoint{3.891309in}{2.008878in}}%
\pgfpathlineto{\pgfqpoint{3.904545in}{2.010116in}}%
\pgfpathlineto{\pgfqpoint{3.917790in}{2.011524in}}%
\pgfpathlineto{\pgfqpoint{3.931043in}{2.013101in}}%
\pgfpathlineto{\pgfqpoint{3.938769in}{2.023504in}}%
\pgfpathlineto{\pgfqpoint{3.946489in}{2.033878in}}%
\pgfpathlineto{\pgfqpoint{3.954205in}{2.044222in}}%
\pgfpathlineto{\pgfqpoint{3.961916in}{2.054535in}}%
\pgfpathlineto{\pgfqpoint{3.948670in}{2.052825in}}%
\pgfpathlineto{\pgfqpoint{3.935432in}{2.051285in}}%
\pgfpathlineto{\pgfqpoint{3.922203in}{2.049914in}}%
\pgfpathlineto{\pgfqpoint{3.908982in}{2.048714in}}%
\pgfpathlineto{\pgfqpoint{3.901264in}{2.038523in}}%
\pgfpathlineto{\pgfqpoint{3.893541in}{2.028308in}}%
\pgfpathlineto{\pgfqpoint{3.885813in}{2.018070in}}%
\pgfpathlineto{\pgfqpoint{3.878080in}{2.007810in}}%
\pgfpathclose%
\pgfusepath{fill}%
\end{pgfscope}%
\begin{pgfscope}%
\pgfpathrectangle{\pgfqpoint{1.254980in}{0.150000in}}{\pgfqpoint{5.490039in}{5.490039in}}%
\pgfusepath{clip}%
\pgfsetbuttcap%
\pgfsetroundjoin%
\definecolor{currentfill}{rgb}{0.227802,0.326594,0.546532}%
\pgfsetfillcolor{currentfill}%
\pgfsetfillopacity{0.700000}%
\pgfsetlinewidth{0.000000pt}%
\definecolor{currentstroke}{rgb}{0.000000,0.000000,0.000000}%
\pgfsetstrokecolor{currentstroke}%
\pgfsetdash{}{0pt}%
\pgfpathmoveto{\pgfqpoint{4.495604in}{2.423570in}}%
\pgfpathlineto{\pgfqpoint{4.509073in}{2.429547in}}%
\pgfpathlineto{\pgfqpoint{4.522555in}{2.435686in}}%
\pgfpathlineto{\pgfqpoint{4.536049in}{2.441987in}}%
\pgfpathlineto{\pgfqpoint{4.549557in}{2.448449in}}%
\pgfpathlineto{\pgfqpoint{4.557071in}{2.457209in}}%
\pgfpathlineto{\pgfqpoint{4.564580in}{2.465896in}}%
\pgfpathlineto{\pgfqpoint{4.572083in}{2.474510in}}%
\pgfpathlineto{\pgfqpoint{4.579580in}{2.483052in}}%
\pgfpathlineto{\pgfqpoint{4.566080in}{2.476685in}}%
\pgfpathlineto{\pgfqpoint{4.552592in}{2.470480in}}%
\pgfpathlineto{\pgfqpoint{4.539118in}{2.464436in}}%
\pgfpathlineto{\pgfqpoint{4.525656in}{2.458553in}}%
\pgfpathlineto{\pgfqpoint{4.518151in}{2.449905in}}%
\pgfpathlineto{\pgfqpoint{4.510641in}{2.441193in}}%
\pgfpathlineto{\pgfqpoint{4.503126in}{2.432415in}}%
\pgfpathlineto{\pgfqpoint{4.495604in}{2.423570in}}%
\pgfpathclose%
\pgfusepath{fill}%
\end{pgfscope}%
\begin{pgfscope}%
\pgfpathrectangle{\pgfqpoint{1.254980in}{0.150000in}}{\pgfqpoint{5.490039in}{5.490039in}}%
\pgfusepath{clip}%
\pgfsetbuttcap%
\pgfsetroundjoin%
\definecolor{currentfill}{rgb}{0.190631,0.407061,0.556089}%
\pgfsetfillcolor{currentfill}%
\pgfsetfillopacity{0.700000}%
\pgfsetlinewidth{0.000000pt}%
\definecolor{currentstroke}{rgb}{0.000000,0.000000,0.000000}%
\pgfsetstrokecolor{currentstroke}%
\pgfsetdash{}{0pt}%
\pgfpathmoveto{\pgfqpoint{2.354712in}{2.696196in}}%
\pgfpathlineto{\pgfqpoint{2.368245in}{2.674005in}}%
\pgfpathlineto{\pgfqpoint{2.381764in}{2.652125in}}%
\pgfpathlineto{\pgfqpoint{2.395270in}{2.630554in}}%
\pgfpathlineto{\pgfqpoint{2.408763in}{2.609288in}}%
\pgfpathlineto{\pgfqpoint{2.417259in}{2.610693in}}%
\pgfpathlineto{\pgfqpoint{2.425739in}{2.612313in}}%
\pgfpathlineto{\pgfqpoint{2.434204in}{2.614143in}}%
\pgfpathlineto{\pgfqpoint{2.442655in}{2.616181in}}%
\pgfpathlineto{\pgfqpoint{2.429203in}{2.637061in}}%
\pgfpathlineto{\pgfqpoint{2.415739in}{2.658245in}}%
\pgfpathlineto{\pgfqpoint{2.402262in}{2.679736in}}%
\pgfpathlineto{\pgfqpoint{2.388771in}{2.701538in}}%
\pgfpathlineto{\pgfqpoint{2.380280in}{2.699876in}}%
\pgfpathlineto{\pgfqpoint{2.371773in}{2.698430in}}%
\pgfpathlineto{\pgfqpoint{2.363250in}{2.697202in}}%
\pgfpathlineto{\pgfqpoint{2.354712in}{2.696196in}}%
\pgfpathclose%
\pgfusepath{fill}%
\end{pgfscope}%
\begin{pgfscope}%
\pgfpathrectangle{\pgfqpoint{1.254980in}{0.150000in}}{\pgfqpoint{5.490039in}{5.490039in}}%
\pgfusepath{clip}%
\pgfsetbuttcap%
\pgfsetroundjoin%
\definecolor{currentfill}{rgb}{0.281887,0.150881,0.465405}%
\pgfsetfillcolor{currentfill}%
\pgfsetfillopacity{0.700000}%
\pgfsetlinewidth{0.000000pt}%
\definecolor{currentstroke}{rgb}{0.000000,0.000000,0.000000}%
\pgfsetstrokecolor{currentstroke}%
\pgfsetdash{}{0pt}%
\pgfpathmoveto{\pgfqpoint{3.961916in}{2.054535in}}%
\pgfpathlineto{\pgfqpoint{3.975171in}{2.056414in}}%
\pgfpathlineto{\pgfqpoint{3.988434in}{2.058461in}}%
\pgfpathlineto{\pgfqpoint{4.001706in}{2.060676in}}%
\pgfpathlineto{\pgfqpoint{4.014988in}{2.063060in}}%
\pgfpathlineto{\pgfqpoint{4.022687in}{2.073456in}}%
\pgfpathlineto{\pgfqpoint{4.030382in}{2.083812in}}%
\pgfpathlineto{\pgfqpoint{4.038071in}{2.094128in}}%
\pgfpathlineto{\pgfqpoint{4.045756in}{2.104403in}}%
\pgfpathlineto{\pgfqpoint{4.032481in}{2.101915in}}%
\pgfpathlineto{\pgfqpoint{4.019215in}{2.099595in}}%
\pgfpathlineto{\pgfqpoint{4.005958in}{2.097443in}}%
\pgfpathlineto{\pgfqpoint{3.992710in}{2.095459in}}%
\pgfpathlineto{\pgfqpoint{3.985019in}{2.085279in}}%
\pgfpathlineto{\pgfqpoint{3.977323in}{2.075064in}}%
\pgfpathlineto{\pgfqpoint{3.969622in}{2.064816in}}%
\pgfpathlineto{\pgfqpoint{3.961916in}{2.054535in}}%
\pgfpathclose%
\pgfusepath{fill}%
\end{pgfscope}%
\begin{pgfscope}%
\pgfpathrectangle{\pgfqpoint{1.254980in}{0.150000in}}{\pgfqpoint{5.490039in}{5.490039in}}%
\pgfusepath{clip}%
\pgfsetbuttcap%
\pgfsetroundjoin%
\definecolor{currentfill}{rgb}{0.280894,0.078907,0.402329}%
\pgfsetfillcolor{currentfill}%
\pgfsetfillopacity{0.700000}%
\pgfsetlinewidth{0.000000pt}%
\definecolor{currentstroke}{rgb}{0.000000,0.000000,0.000000}%
\pgfsetstrokecolor{currentstroke}%
\pgfsetdash{}{0pt}%
\pgfpathmoveto{\pgfqpoint{3.710346in}{1.925506in}}%
\pgfpathlineto{\pgfqpoint{3.723534in}{1.924846in}}%
\pgfpathlineto{\pgfqpoint{3.736728in}{1.924361in}}%
\pgfpathlineto{\pgfqpoint{3.749929in}{1.924049in}}%
\pgfpathlineto{\pgfqpoint{3.763137in}{1.923910in}}%
\pgfpathlineto{\pgfqpoint{3.770918in}{1.934105in}}%
\pgfpathlineto{\pgfqpoint{3.778694in}{1.944294in}}%
\pgfpathlineto{\pgfqpoint{3.786465in}{1.954477in}}%
\pgfpathlineto{\pgfqpoint{3.794230in}{1.964653in}}%
\pgfpathlineto{\pgfqpoint{3.781031in}{1.964603in}}%
\pgfpathlineto{\pgfqpoint{3.767839in}{1.964727in}}%
\pgfpathlineto{\pgfqpoint{3.754654in}{1.965024in}}%
\pgfpathlineto{\pgfqpoint{3.741476in}{1.965495in}}%
\pgfpathlineto{\pgfqpoint{3.733701in}{1.955497in}}%
\pgfpathlineto{\pgfqpoint{3.725921in}{1.945499in}}%
\pgfpathlineto{\pgfqpoint{3.718136in}{1.935501in}}%
\pgfpathlineto{\pgfqpoint{3.710346in}{1.925506in}}%
\pgfpathclose%
\pgfusepath{fill}%
\end{pgfscope}%
\begin{pgfscope}%
\pgfpathrectangle{\pgfqpoint{1.254980in}{0.150000in}}{\pgfqpoint{5.490039in}{5.490039in}}%
\pgfusepath{clip}%
\pgfsetbuttcap%
\pgfsetroundjoin%
\definecolor{currentfill}{rgb}{0.278826,0.175490,0.483397}%
\pgfsetfillcolor{currentfill}%
\pgfsetfillopacity{0.700000}%
\pgfsetlinewidth{0.000000pt}%
\definecolor{currentstroke}{rgb}{0.000000,0.000000,0.000000}%
\pgfsetstrokecolor{currentstroke}%
\pgfsetdash{}{0pt}%
\pgfpathmoveto{\pgfqpoint{4.045756in}{2.104403in}}%
\pgfpathlineto{\pgfqpoint{4.059040in}{2.107059in}}%
\pgfpathlineto{\pgfqpoint{4.072334in}{2.109882in}}%
\pgfpathlineto{\pgfqpoint{4.085638in}{2.112871in}}%
\pgfpathlineto{\pgfqpoint{4.098951in}{2.116027in}}%
\pgfpathlineto{\pgfqpoint{4.106624in}{2.126348in}}%
\pgfpathlineto{\pgfqpoint{4.114293in}{2.136619in}}%
\pgfpathlineto{\pgfqpoint{4.121957in}{2.146841in}}%
\pgfpathlineto{\pgfqpoint{4.129615in}{2.157013in}}%
\pgfpathlineto{\pgfqpoint{4.116308in}{2.153781in}}%
\pgfpathlineto{\pgfqpoint{4.103010in}{2.150715in}}%
\pgfpathlineto{\pgfqpoint{4.089723in}{2.147815in}}%
\pgfpathlineto{\pgfqpoint{4.076444in}{2.145083in}}%
\pgfpathlineto{\pgfqpoint{4.068780in}{2.134977in}}%
\pgfpathlineto{\pgfqpoint{4.061110in}{2.124828in}}%
\pgfpathlineto{\pgfqpoint{4.053435in}{2.114637in}}%
\pgfpathlineto{\pgfqpoint{4.045756in}{2.104403in}}%
\pgfpathclose%
\pgfusepath{fill}%
\end{pgfscope}%
\begin{pgfscope}%
\pgfpathrectangle{\pgfqpoint{1.254980in}{0.150000in}}{\pgfqpoint{5.490039in}{5.490039in}}%
\pgfusepath{clip}%
\pgfsetbuttcap%
\pgfsetroundjoin%
\definecolor{currentfill}{rgb}{0.160665,0.478540,0.558115}%
\pgfsetfillcolor{currentfill}%
\pgfsetfillopacity{0.700000}%
\pgfsetlinewidth{0.000000pt}%
\definecolor{currentstroke}{rgb}{0.000000,0.000000,0.000000}%
\pgfsetstrokecolor{currentstroke}%
\pgfsetdash{}{0pt}%
\pgfpathmoveto{\pgfqpoint{5.029457in}{2.809059in}}%
\pgfpathlineto{\pgfqpoint{5.043198in}{2.817589in}}%
\pgfpathlineto{\pgfqpoint{5.056954in}{2.826275in}}%
\pgfpathlineto{\pgfqpoint{5.070726in}{2.835120in}}%
\pgfpathlineto{\pgfqpoint{5.084514in}{2.844123in}}%
\pgfpathlineto{\pgfqpoint{5.091786in}{2.849843in}}%
\pgfpathlineto{\pgfqpoint{5.099052in}{2.855514in}}%
\pgfpathlineto{\pgfqpoint{5.106310in}{2.861137in}}%
\pgfpathlineto{\pgfqpoint{5.113562in}{2.866718in}}%
\pgfpathlineto{\pgfqpoint{5.099788in}{2.858018in}}%
\pgfpathlineto{\pgfqpoint{5.086031in}{2.849475in}}%
\pgfpathlineto{\pgfqpoint{5.072289in}{2.841089in}}%
\pgfpathlineto{\pgfqpoint{5.058563in}{2.832861in}}%
\pgfpathlineto{\pgfqpoint{5.051296in}{2.826968in}}%
\pgfpathlineto{\pgfqpoint{5.044023in}{2.821039in}}%
\pgfpathlineto{\pgfqpoint{5.036743in}{2.815071in}}%
\pgfpathlineto{\pgfqpoint{5.029457in}{2.809059in}}%
\pgfpathclose%
\pgfusepath{fill}%
\end{pgfscope}%
\begin{pgfscope}%
\pgfpathrectangle{\pgfqpoint{1.254980in}{0.150000in}}{\pgfqpoint{5.490039in}{5.490039in}}%
\pgfusepath{clip}%
\pgfsetbuttcap%
\pgfsetroundjoin%
\definecolor{currentfill}{rgb}{0.278791,0.062145,0.386592}%
\pgfsetfillcolor{currentfill}%
\pgfsetfillopacity{0.700000}%
\pgfsetlinewidth{0.000000pt}%
\definecolor{currentstroke}{rgb}{0.000000,0.000000,0.000000}%
\pgfsetstrokecolor{currentstroke}%
\pgfsetdash{}{0pt}%
\pgfpathmoveto{\pgfqpoint{3.626402in}{1.890834in}}%
\pgfpathlineto{\pgfqpoint{3.639576in}{1.889257in}}%
\pgfpathlineto{\pgfqpoint{3.652756in}{1.887856in}}%
\pgfpathlineto{\pgfqpoint{3.665941in}{1.886631in}}%
\pgfpathlineto{\pgfqpoint{3.679132in}{1.885581in}}%
\pgfpathlineto{\pgfqpoint{3.686944in}{1.895550in}}%
\pgfpathlineto{\pgfqpoint{3.694750in}{1.905529in}}%
\pgfpathlineto{\pgfqpoint{3.702550in}{1.915514in}}%
\pgfpathlineto{\pgfqpoint{3.710346in}{1.925506in}}%
\pgfpathlineto{\pgfqpoint{3.697164in}{1.926340in}}%
\pgfpathlineto{\pgfqpoint{3.683989in}{1.927349in}}%
\pgfpathlineto{\pgfqpoint{3.670820in}{1.928534in}}%
\pgfpathlineto{\pgfqpoint{3.657657in}{1.929895in}}%
\pgfpathlineto{\pgfqpoint{3.649851in}{1.920109in}}%
\pgfpathlineto{\pgfqpoint{3.642040in}{1.910336in}}%
\pgfpathlineto{\pgfqpoint{3.634224in}{1.900577in}}%
\pgfpathlineto{\pgfqpoint{3.626402in}{1.890834in}}%
\pgfpathclose%
\pgfusepath{fill}%
\end{pgfscope}%
\begin{pgfscope}%
\pgfpathrectangle{\pgfqpoint{1.254980in}{0.150000in}}{\pgfqpoint{5.490039in}{5.490039in}}%
\pgfusepath{clip}%
\pgfsetbuttcap%
\pgfsetroundjoin%
\definecolor{currentfill}{rgb}{0.280267,0.073417,0.397163}%
\pgfsetfillcolor{currentfill}%
\pgfsetfillopacity{0.700000}%
\pgfsetlinewidth{0.000000pt}%
\definecolor{currentstroke}{rgb}{0.000000,0.000000,0.000000}%
\pgfsetstrokecolor{currentstroke}%
\pgfsetdash{}{0pt}%
\pgfpathmoveto{\pgfqpoint{2.993816in}{1.937531in}}%
\pgfpathlineto{\pgfqpoint{3.007002in}{1.927558in}}%
\pgfpathlineto{\pgfqpoint{3.020186in}{1.917796in}}%
\pgfpathlineto{\pgfqpoint{3.033368in}{1.908243in}}%
\pgfpathlineto{\pgfqpoint{3.046549in}{1.898898in}}%
\pgfpathlineto{\pgfqpoint{3.054649in}{1.904874in}}%
\pgfpathlineto{\pgfqpoint{3.062740in}{1.910976in}}%
\pgfpathlineto{\pgfqpoint{3.070821in}{1.917201in}}%
\pgfpathlineto{\pgfqpoint{3.078894in}{1.923543in}}%
\pgfpathlineto{\pgfqpoint{3.065737in}{1.932530in}}%
\pgfpathlineto{\pgfqpoint{3.052579in}{1.941724in}}%
\pgfpathlineto{\pgfqpoint{3.039421in}{1.951126in}}%
\pgfpathlineto{\pgfqpoint{3.026260in}{1.960738in}}%
\pgfpathlineto{\pgfqpoint{3.018164in}{1.954744in}}%
\pgfpathlineto{\pgfqpoint{3.010057in}{1.948876in}}%
\pgfpathlineto{\pgfqpoint{3.001942in}{1.943137in}}%
\pgfpathlineto{\pgfqpoint{2.993816in}{1.937531in}}%
\pgfpathclose%
\pgfusepath{fill}%
\end{pgfscope}%
\begin{pgfscope}%
\pgfpathrectangle{\pgfqpoint{1.254980in}{0.150000in}}{\pgfqpoint{5.490039in}{5.490039in}}%
\pgfusepath{clip}%
\pgfsetbuttcap%
\pgfsetroundjoin%
\definecolor{currentfill}{rgb}{0.216210,0.351535,0.550627}%
\pgfsetfillcolor{currentfill}%
\pgfsetfillopacity{0.700000}%
\pgfsetlinewidth{0.000000pt}%
\definecolor{currentstroke}{rgb}{0.000000,0.000000,0.000000}%
\pgfsetstrokecolor{currentstroke}%
\pgfsetdash{}{0pt}%
\pgfpathmoveto{\pgfqpoint{4.579580in}{2.483052in}}%
\pgfpathlineto{\pgfqpoint{4.593093in}{2.489580in}}%
\pgfpathlineto{\pgfqpoint{4.606620in}{2.496269in}}%
\pgfpathlineto{\pgfqpoint{4.620160in}{2.503120in}}%
\pgfpathlineto{\pgfqpoint{4.633714in}{2.510131in}}%
\pgfpathlineto{\pgfqpoint{4.641197in}{2.518490in}}%
\pgfpathlineto{\pgfqpoint{4.648674in}{2.526773in}}%
\pgfpathlineto{\pgfqpoint{4.656145in}{2.534983in}}%
\pgfpathlineto{\pgfqpoint{4.663610in}{2.543121in}}%
\pgfpathlineto{\pgfqpoint{4.650065in}{2.536234in}}%
\pgfpathlineto{\pgfqpoint{4.636532in}{2.529509in}}%
\pgfpathlineto{\pgfqpoint{4.623013in}{2.522944in}}%
\pgfpathlineto{\pgfqpoint{4.609508in}{2.516539in}}%
\pgfpathlineto{\pgfqpoint{4.602035in}{2.508267in}}%
\pgfpathlineto{\pgfqpoint{4.594556in}{2.499929in}}%
\pgfpathlineto{\pgfqpoint{4.587071in}{2.491525in}}%
\pgfpathlineto{\pgfqpoint{4.579580in}{2.483052in}}%
\pgfpathclose%
\pgfusepath{fill}%
\end{pgfscope}%
\begin{pgfscope}%
\pgfpathrectangle{\pgfqpoint{1.254980in}{0.150000in}}{\pgfqpoint{5.490039in}{5.490039in}}%
\pgfusepath{clip}%
\pgfsetbuttcap%
\pgfsetroundjoin%
\definecolor{currentfill}{rgb}{0.273006,0.204520,0.501721}%
\pgfsetfillcolor{currentfill}%
\pgfsetfillopacity{0.700000}%
\pgfsetlinewidth{0.000000pt}%
\definecolor{currentstroke}{rgb}{0.000000,0.000000,0.000000}%
\pgfsetstrokecolor{currentstroke}%
\pgfsetdash{}{0pt}%
\pgfpathmoveto{\pgfqpoint{4.129615in}{2.157013in}}%
\pgfpathlineto{\pgfqpoint{4.142932in}{2.160412in}}%
\pgfpathlineto{\pgfqpoint{4.156260in}{2.163977in}}%
\pgfpathlineto{\pgfqpoint{4.169598in}{2.167707in}}%
\pgfpathlineto{\pgfqpoint{4.182946in}{2.171603in}}%
\pgfpathlineto{\pgfqpoint{4.190594in}{2.181784in}}%
\pgfpathlineto{\pgfqpoint{4.198236in}{2.191907in}}%
\pgfpathlineto{\pgfqpoint{4.205874in}{2.201973in}}%
\pgfpathlineto{\pgfqpoint{4.213506in}{2.211982in}}%
\pgfpathlineto{\pgfqpoint{4.200164in}{2.208038in}}%
\pgfpathlineto{\pgfqpoint{4.186832in}{2.204259in}}%
\pgfpathlineto{\pgfqpoint{4.173510in}{2.200646in}}%
\pgfpathlineto{\pgfqpoint{4.160199in}{2.197199in}}%
\pgfpathlineto{\pgfqpoint{4.152561in}{2.187228in}}%
\pgfpathlineto{\pgfqpoint{4.144917in}{2.177207in}}%
\pgfpathlineto{\pgfqpoint{4.137269in}{2.167135in}}%
\pgfpathlineto{\pgfqpoint{4.129615in}{2.157013in}}%
\pgfpathclose%
\pgfusepath{fill}%
\end{pgfscope}%
\begin{pgfscope}%
\pgfpathrectangle{\pgfqpoint{1.254980in}{0.150000in}}{\pgfqpoint{5.490039in}{5.490039in}}%
\pgfusepath{clip}%
\pgfsetbuttcap%
\pgfsetroundjoin%
\definecolor{currentfill}{rgb}{0.273809,0.031497,0.358853}%
\pgfsetfillcolor{currentfill}%
\pgfsetfillopacity{0.700000}%
\pgfsetlinewidth{0.000000pt}%
\definecolor{currentstroke}{rgb}{0.000000,0.000000,0.000000}%
\pgfsetstrokecolor{currentstroke}%
\pgfsetdash{}{0pt}%
\pgfpathmoveto{\pgfqpoint{3.184137in}{1.858970in}}%
\pgfpathlineto{\pgfqpoint{3.197294in}{1.851796in}}%
\pgfpathlineto{\pgfqpoint{3.210452in}{1.844818in}}%
\pgfpathlineto{\pgfqpoint{3.223610in}{1.838035in}}%
\pgfpathlineto{\pgfqpoint{3.236770in}{1.831445in}}%
\pgfpathlineto{\pgfqpoint{3.244770in}{1.838924in}}%
\pgfpathlineto{\pgfqpoint{3.252761in}{1.846493in}}%
\pgfpathlineto{\pgfqpoint{3.260746in}{1.854150in}}%
\pgfpathlineto{\pgfqpoint{3.268722in}{1.861892in}}%
\pgfpathlineto{\pgfqpoint{3.255582in}{1.868153in}}%
\pgfpathlineto{\pgfqpoint{3.242443in}{1.874608in}}%
\pgfpathlineto{\pgfqpoint{3.229305in}{1.881258in}}%
\pgfpathlineto{\pgfqpoint{3.216169in}{1.888103in}}%
\pgfpathlineto{\pgfqpoint{3.208173in}{1.880678in}}%
\pgfpathlineto{\pgfqpoint{3.200169in}{1.873346in}}%
\pgfpathlineto{\pgfqpoint{3.192157in}{1.866109in}}%
\pgfpathlineto{\pgfqpoint{3.184137in}{1.858970in}}%
\pgfpathclose%
\pgfusepath{fill}%
\end{pgfscope}%
\begin{pgfscope}%
\pgfpathrectangle{\pgfqpoint{1.254980in}{0.150000in}}{\pgfqpoint{5.490039in}{5.490039in}}%
\pgfusepath{clip}%
\pgfsetbuttcap%
\pgfsetroundjoin%
\definecolor{currentfill}{rgb}{0.151918,0.500685,0.557587}%
\pgfsetfillcolor{currentfill}%
\pgfsetfillopacity{0.700000}%
\pgfsetlinewidth{0.000000pt}%
\definecolor{currentstroke}{rgb}{0.000000,0.000000,0.000000}%
\pgfsetstrokecolor{currentstroke}%
\pgfsetdash{}{0pt}%
\pgfpathmoveto{\pgfqpoint{5.113562in}{2.866718in}}%
\pgfpathlineto{\pgfqpoint{5.127351in}{2.875576in}}%
\pgfpathlineto{\pgfqpoint{5.141157in}{2.884590in}}%
\pgfpathlineto{\pgfqpoint{5.154979in}{2.893763in}}%
\pgfpathlineto{\pgfqpoint{5.168817in}{2.903093in}}%
\pgfpathlineto{\pgfqpoint{5.176046in}{2.908312in}}%
\pgfpathlineto{\pgfqpoint{5.183269in}{2.913489in}}%
\pgfpathlineto{\pgfqpoint{5.190485in}{2.918627in}}%
\pgfpathlineto{\pgfqpoint{5.197694in}{2.923730in}}%
\pgfpathlineto{\pgfqpoint{5.183872in}{2.914733in}}%
\pgfpathlineto{\pgfqpoint{5.170066in}{2.905893in}}%
\pgfpathlineto{\pgfqpoint{5.156277in}{2.897209in}}%
\pgfpathlineto{\pgfqpoint{5.142503in}{2.888682in}}%
\pgfpathlineto{\pgfqpoint{5.135278in}{2.883237in}}%
\pgfpathlineto{\pgfqpoint{5.128046in}{2.877764in}}%
\pgfpathlineto{\pgfqpoint{5.120807in}{2.872259in}}%
\pgfpathlineto{\pgfqpoint{5.113562in}{2.866718in}}%
\pgfpathclose%
\pgfusepath{fill}%
\end{pgfscope}%
\begin{pgfscope}%
\pgfpathrectangle{\pgfqpoint{1.254980in}{0.150000in}}{\pgfqpoint{5.490039in}{5.490039in}}%
\pgfusepath{clip}%
\pgfsetbuttcap%
\pgfsetroundjoin%
\definecolor{currentfill}{rgb}{0.272594,0.025563,0.353093}%
\pgfsetfillcolor{currentfill}%
\pgfsetfillopacity{0.700000}%
\pgfsetlinewidth{0.000000pt}%
\definecolor{currentstroke}{rgb}{0.000000,0.000000,0.000000}%
\pgfsetstrokecolor{currentstroke}%
\pgfsetdash{}{0pt}%
\pgfpathmoveto{\pgfqpoint{3.321303in}{1.838766in}}%
\pgfpathlineto{\pgfqpoint{3.334453in}{1.833459in}}%
\pgfpathlineto{\pgfqpoint{3.347606in}{1.828340in}}%
\pgfpathlineto{\pgfqpoint{3.360762in}{1.823409in}}%
\pgfpathlineto{\pgfqpoint{3.373921in}{1.818664in}}%
\pgfpathlineto{\pgfqpoint{3.381855in}{1.827109in}}%
\pgfpathlineto{\pgfqpoint{3.389783in}{1.835619in}}%
\pgfpathlineto{\pgfqpoint{3.397704in}{1.844193in}}%
\pgfpathlineto{\pgfqpoint{3.405619in}{1.852825in}}%
\pgfpathlineto{\pgfqpoint{3.392476in}{1.857271in}}%
\pgfpathlineto{\pgfqpoint{3.379337in}{1.861903in}}%
\pgfpathlineto{\pgfqpoint{3.366200in}{1.866722in}}%
\pgfpathlineto{\pgfqpoint{3.353067in}{1.871729in}}%
\pgfpathlineto{\pgfqpoint{3.345136in}{1.863385in}}%
\pgfpathlineto{\pgfqpoint{3.337198in}{1.855108in}}%
\pgfpathlineto{\pgfqpoint{3.329254in}{1.846901in}}%
\pgfpathlineto{\pgfqpoint{3.321303in}{1.838766in}}%
\pgfpathclose%
\pgfusepath{fill}%
\end{pgfscope}%
\begin{pgfscope}%
\pgfpathrectangle{\pgfqpoint{1.254980in}{0.150000in}}{\pgfqpoint{5.490039in}{5.490039in}}%
\pgfusepath{clip}%
\pgfsetbuttcap%
\pgfsetroundjoin%
\definecolor{currentfill}{rgb}{0.276022,0.044167,0.370164}%
\pgfsetfillcolor{currentfill}%
\pgfsetfillopacity{0.700000}%
\pgfsetlinewidth{0.000000pt}%
\definecolor{currentstroke}{rgb}{0.000000,0.000000,0.000000}%
\pgfsetstrokecolor{currentstroke}%
\pgfsetdash{}{0pt}%
\pgfpathmoveto{\pgfqpoint{3.542372in}{1.861126in}}%
\pgfpathlineto{\pgfqpoint{3.555536in}{1.858593in}}%
\pgfpathlineto{\pgfqpoint{3.568706in}{1.856240in}}%
\pgfpathlineto{\pgfqpoint{3.581880in}{1.854065in}}%
\pgfpathlineto{\pgfqpoint{3.595060in}{1.852067in}}%
\pgfpathlineto{\pgfqpoint{3.602904in}{1.861724in}}%
\pgfpathlineto{\pgfqpoint{3.610742in}{1.871406in}}%
\pgfpathlineto{\pgfqpoint{3.618575in}{1.881110in}}%
\pgfpathlineto{\pgfqpoint{3.626402in}{1.890834in}}%
\pgfpathlineto{\pgfqpoint{3.613234in}{1.892588in}}%
\pgfpathlineto{\pgfqpoint{3.600071in}{1.894520in}}%
\pgfpathlineto{\pgfqpoint{3.586914in}{1.896630in}}%
\pgfpathlineto{\pgfqpoint{3.573761in}{1.898918in}}%
\pgfpathlineto{\pgfqpoint{3.565922in}{1.889427in}}%
\pgfpathlineto{\pgfqpoint{3.558078in}{1.879963in}}%
\pgfpathlineto{\pgfqpoint{3.550228in}{1.870529in}}%
\pgfpathlineto{\pgfqpoint{3.542372in}{1.861126in}}%
\pgfpathclose%
\pgfusepath{fill}%
\end{pgfscope}%
\begin{pgfscope}%
\pgfpathrectangle{\pgfqpoint{1.254980in}{0.150000in}}{\pgfqpoint{5.490039in}{5.490039in}}%
\pgfusepath{clip}%
\pgfsetbuttcap%
\pgfsetroundjoin%
\definecolor{currentfill}{rgb}{0.269308,0.218818,0.509577}%
\pgfsetfillcolor{currentfill}%
\pgfsetfillopacity{0.700000}%
\pgfsetlinewidth{0.000000pt}%
\definecolor{currentstroke}{rgb}{0.000000,0.000000,0.000000}%
\pgfsetstrokecolor{currentstroke}%
\pgfsetdash{}{0pt}%
\pgfpathmoveto{\pgfqpoint{2.642994in}{2.232204in}}%
\pgfpathlineto{\pgfqpoint{2.656316in}{2.216325in}}%
\pgfpathlineto{\pgfqpoint{2.669631in}{2.200697in}}%
\pgfpathlineto{\pgfqpoint{2.682940in}{2.185318in}}%
\pgfpathlineto{\pgfqpoint{2.696241in}{2.170185in}}%
\pgfpathlineto{\pgfqpoint{2.704563in}{2.173204in}}%
\pgfpathlineto{\pgfqpoint{2.712873in}{2.176408in}}%
\pgfpathlineto{\pgfqpoint{2.721169in}{2.179791in}}%
\pgfpathlineto{\pgfqpoint{2.729454in}{2.183350in}}%
\pgfpathlineto{\pgfqpoint{2.716186in}{2.198085in}}%
\pgfpathlineto{\pgfqpoint{2.702913in}{2.213065in}}%
\pgfpathlineto{\pgfqpoint{2.689632in}{2.228293in}}%
\pgfpathlineto{\pgfqpoint{2.676345in}{2.243771in}}%
\pgfpathlineto{\pgfqpoint{2.668027in}{2.240600in}}%
\pgfpathlineto{\pgfqpoint{2.659696in}{2.237613in}}%
\pgfpathlineto{\pgfqpoint{2.651351in}{2.234813in}}%
\pgfpathlineto{\pgfqpoint{2.642994in}{2.232204in}}%
\pgfpathclose%
\pgfusepath{fill}%
\end{pgfscope}%
\begin{pgfscope}%
\pgfpathrectangle{\pgfqpoint{1.254980in}{0.150000in}}{\pgfqpoint{5.490039in}{5.490039in}}%
\pgfusepath{clip}%
\pgfsetbuttcap%
\pgfsetroundjoin%
\definecolor{currentfill}{rgb}{0.265145,0.232956,0.516599}%
\pgfsetfillcolor{currentfill}%
\pgfsetfillopacity{0.700000}%
\pgfsetlinewidth{0.000000pt}%
\definecolor{currentstroke}{rgb}{0.000000,0.000000,0.000000}%
\pgfsetstrokecolor{currentstroke}%
\pgfsetdash{}{0pt}%
\pgfpathmoveto{\pgfqpoint{4.213506in}{2.211982in}}%
\pgfpathlineto{\pgfqpoint{4.226859in}{2.216091in}}%
\pgfpathlineto{\pgfqpoint{4.240223in}{2.220364in}}%
\pgfpathlineto{\pgfqpoint{4.253598in}{2.224802in}}%
\pgfpathlineto{\pgfqpoint{4.266984in}{2.229405in}}%
\pgfpathlineto{\pgfqpoint{4.274606in}{2.239386in}}%
\pgfpathlineto{\pgfqpoint{4.282222in}{2.249303in}}%
\pgfpathlineto{\pgfqpoint{4.289834in}{2.259156in}}%
\pgfpathlineto{\pgfqpoint{4.297439in}{2.268945in}}%
\pgfpathlineto{\pgfqpoint{4.284059in}{2.264323in}}%
\pgfpathlineto{\pgfqpoint{4.270690in}{2.259865in}}%
\pgfpathlineto{\pgfqpoint{4.257332in}{2.255571in}}%
\pgfpathlineto{\pgfqpoint{4.243985in}{2.251442in}}%
\pgfpathlineto{\pgfqpoint{4.236373in}{2.241663in}}%
\pgfpathlineto{\pgfqpoint{4.228756in}{2.231826in}}%
\pgfpathlineto{\pgfqpoint{4.221134in}{2.221933in}}%
\pgfpathlineto{\pgfqpoint{4.213506in}{2.211982in}}%
\pgfpathclose%
\pgfusepath{fill}%
\end{pgfscope}%
\begin{pgfscope}%
\pgfpathrectangle{\pgfqpoint{1.254980in}{0.150000in}}{\pgfqpoint{5.490039in}{5.490039in}}%
\pgfusepath{clip}%
\pgfsetbuttcap%
\pgfsetroundjoin%
\definecolor{currentfill}{rgb}{0.260571,0.246922,0.522828}%
\pgfsetfillcolor{currentfill}%
\pgfsetfillopacity{0.700000}%
\pgfsetlinewidth{0.000000pt}%
\definecolor{currentstroke}{rgb}{0.000000,0.000000,0.000000}%
\pgfsetstrokecolor{currentstroke}%
\pgfsetdash{}{0pt}%
\pgfpathmoveto{\pgfqpoint{2.589627in}{2.298266in}}%
\pgfpathlineto{\pgfqpoint{2.602981in}{2.281364in}}%
\pgfpathlineto{\pgfqpoint{2.616326in}{2.264721in}}%
\pgfpathlineto{\pgfqpoint{2.629664in}{2.248335in}}%
\pgfpathlineto{\pgfqpoint{2.642994in}{2.232204in}}%
\pgfpathlineto{\pgfqpoint{2.651351in}{2.234813in}}%
\pgfpathlineto{\pgfqpoint{2.659696in}{2.237613in}}%
\pgfpathlineto{\pgfqpoint{2.668027in}{2.240600in}}%
\pgfpathlineto{\pgfqpoint{2.676345in}{2.243771in}}%
\pgfpathlineto{\pgfqpoint{2.663051in}{2.259501in}}%
\pgfpathlineto{\pgfqpoint{2.649750in}{2.275486in}}%
\pgfpathlineto{\pgfqpoint{2.636440in}{2.291726in}}%
\pgfpathlineto{\pgfqpoint{2.623124in}{2.308225in}}%
\pgfpathlineto{\pgfqpoint{2.614770in}{2.305445in}}%
\pgfpathlineto{\pgfqpoint{2.606402in}{2.302855in}}%
\pgfpathlineto{\pgfqpoint{2.598022in}{2.300461in}}%
\pgfpathlineto{\pgfqpoint{2.589627in}{2.298266in}}%
\pgfpathclose%
\pgfusepath{fill}%
\end{pgfscope}%
\begin{pgfscope}%
\pgfpathrectangle{\pgfqpoint{1.254980in}{0.150000in}}{\pgfqpoint{5.490039in}{5.490039in}}%
\pgfusepath{clip}%
\pgfsetbuttcap%
\pgfsetroundjoin%
\definecolor{currentfill}{rgb}{0.276194,0.190074,0.493001}%
\pgfsetfillcolor{currentfill}%
\pgfsetfillopacity{0.700000}%
\pgfsetlinewidth{0.000000pt}%
\definecolor{currentstroke}{rgb}{0.000000,0.000000,0.000000}%
\pgfsetstrokecolor{currentstroke}%
\pgfsetdash{}{0pt}%
\pgfpathmoveto{\pgfqpoint{2.696241in}{2.170185in}}%
\pgfpathlineto{\pgfqpoint{2.709536in}{2.155296in}}%
\pgfpathlineto{\pgfqpoint{2.722825in}{2.140650in}}%
\pgfpathlineto{\pgfqpoint{2.736108in}{2.126245in}}%
\pgfpathlineto{\pgfqpoint{2.749385in}{2.112079in}}%
\pgfpathlineto{\pgfqpoint{2.757673in}{2.115507in}}%
\pgfpathlineto{\pgfqpoint{2.765949in}{2.119111in}}%
\pgfpathlineto{\pgfqpoint{2.774212in}{2.122887in}}%
\pgfpathlineto{\pgfqpoint{2.782464in}{2.126833in}}%
\pgfpathlineto{\pgfqpoint{2.769220in}{2.140603in}}%
\pgfpathlineto{\pgfqpoint{2.755970in}{2.154611in}}%
\pgfpathlineto{\pgfqpoint{2.742715in}{2.168860in}}%
\pgfpathlineto{\pgfqpoint{2.729454in}{2.183350in}}%
\pgfpathlineto{\pgfqpoint{2.721169in}{2.179791in}}%
\pgfpathlineto{\pgfqpoint{2.712873in}{2.176408in}}%
\pgfpathlineto{\pgfqpoint{2.704563in}{2.173204in}}%
\pgfpathlineto{\pgfqpoint{2.696241in}{2.170185in}}%
\pgfpathclose%
\pgfusepath{fill}%
\end{pgfscope}%
\begin{pgfscope}%
\pgfpathrectangle{\pgfqpoint{1.254980in}{0.150000in}}{\pgfqpoint{5.490039in}{5.490039in}}%
\pgfusepath{clip}%
\pgfsetbuttcap%
\pgfsetroundjoin%
\definecolor{currentfill}{rgb}{0.203063,0.379716,0.553925}%
\pgfsetfillcolor{currentfill}%
\pgfsetfillopacity{0.700000}%
\pgfsetlinewidth{0.000000pt}%
\definecolor{currentstroke}{rgb}{0.000000,0.000000,0.000000}%
\pgfsetstrokecolor{currentstroke}%
\pgfsetdash{}{0pt}%
\pgfpathmoveto{\pgfqpoint{4.663610in}{2.543121in}}%
\pgfpathlineto{\pgfqpoint{4.677170in}{2.550168in}}%
\pgfpathlineto{\pgfqpoint{4.690743in}{2.557375in}}%
\pgfpathlineto{\pgfqpoint{4.704330in}{2.564743in}}%
\pgfpathlineto{\pgfqpoint{4.717931in}{2.572271in}}%
\pgfpathlineto{\pgfqpoint{4.725382in}{2.580196in}}%
\pgfpathlineto{\pgfqpoint{4.732826in}{2.588045in}}%
\pgfpathlineto{\pgfqpoint{4.740263in}{2.595820in}}%
\pgfpathlineto{\pgfqpoint{4.747695in}{2.603525in}}%
\pgfpathlineto{\pgfqpoint{4.734102in}{2.596151in}}%
\pgfpathlineto{\pgfqpoint{4.720524in}{2.588937in}}%
\pgfpathlineto{\pgfqpoint{4.706959in}{2.581883in}}%
\pgfpathlineto{\pgfqpoint{4.693408in}{2.574990in}}%
\pgfpathlineto{\pgfqpoint{4.685968in}{2.567121in}}%
\pgfpathlineto{\pgfqpoint{4.678522in}{2.559188in}}%
\pgfpathlineto{\pgfqpoint{4.671069in}{2.551189in}}%
\pgfpathlineto{\pgfqpoint{4.663610in}{2.543121in}}%
\pgfpathclose%
\pgfusepath{fill}%
\end{pgfscope}%
\begin{pgfscope}%
\pgfpathrectangle{\pgfqpoint{1.254980in}{0.150000in}}{\pgfqpoint{5.490039in}{5.490039in}}%
\pgfusepath{clip}%
\pgfsetbuttcap%
\pgfsetroundjoin%
\definecolor{currentfill}{rgb}{0.143343,0.522773,0.556295}%
\pgfsetfillcolor{currentfill}%
\pgfsetfillopacity{0.700000}%
\pgfsetlinewidth{0.000000pt}%
\definecolor{currentstroke}{rgb}{0.000000,0.000000,0.000000}%
\pgfsetstrokecolor{currentstroke}%
\pgfsetdash{}{0pt}%
\pgfpathmoveto{\pgfqpoint{5.197694in}{2.923730in}}%
\pgfpathlineto{\pgfqpoint{5.211532in}{2.932884in}}%
\pgfpathlineto{\pgfqpoint{5.225387in}{2.942195in}}%
\pgfpathlineto{\pgfqpoint{5.239259in}{2.951663in}}%
\pgfpathlineto{\pgfqpoint{5.253147in}{2.961288in}}%
\pgfpathlineto{\pgfqpoint{5.260332in}{2.966008in}}%
\pgfpathlineto{\pgfqpoint{5.267510in}{2.970694in}}%
\pgfpathlineto{\pgfqpoint{5.274682in}{2.975350in}}%
\pgfpathlineto{\pgfqpoint{5.281847in}{2.979979in}}%
\pgfpathlineto{\pgfqpoint{5.267977in}{2.970717in}}%
\pgfpathlineto{\pgfqpoint{5.254123in}{2.961611in}}%
\pgfpathlineto{\pgfqpoint{5.240286in}{2.952662in}}%
\pgfpathlineto{\pgfqpoint{5.226465in}{2.943869in}}%
\pgfpathlineto{\pgfqpoint{5.219282in}{2.938867in}}%
\pgfpathlineto{\pgfqpoint{5.212092in}{2.933846in}}%
\pgfpathlineto{\pgfqpoint{5.204896in}{2.928802in}}%
\pgfpathlineto{\pgfqpoint{5.197694in}{2.923730in}}%
\pgfpathclose%
\pgfusepath{fill}%
\end{pgfscope}%
\begin{pgfscope}%
\pgfpathrectangle{\pgfqpoint{1.254980in}{0.150000in}}{\pgfqpoint{5.490039in}{5.490039in}}%
\pgfusepath{clip}%
\pgfsetbuttcap%
\pgfsetroundjoin%
\definecolor{currentfill}{rgb}{0.248629,0.278775,0.534556}%
\pgfsetfillcolor{currentfill}%
\pgfsetfillopacity{0.700000}%
\pgfsetlinewidth{0.000000pt}%
\definecolor{currentstroke}{rgb}{0.000000,0.000000,0.000000}%
\pgfsetstrokecolor{currentstroke}%
\pgfsetdash{}{0pt}%
\pgfpathmoveto{\pgfqpoint{2.536124in}{2.368509in}}%
\pgfpathlineto{\pgfqpoint{2.549513in}{2.350548in}}%
\pgfpathlineto{\pgfqpoint{2.562893in}{2.332856in}}%
\pgfpathlineto{\pgfqpoint{2.576264in}{2.315429in}}%
\pgfpathlineto{\pgfqpoint{2.589627in}{2.298266in}}%
\pgfpathlineto{\pgfqpoint{2.598022in}{2.300461in}}%
\pgfpathlineto{\pgfqpoint{2.606402in}{2.302855in}}%
\pgfpathlineto{\pgfqpoint{2.614770in}{2.305445in}}%
\pgfpathlineto{\pgfqpoint{2.623124in}{2.308225in}}%
\pgfpathlineto{\pgfqpoint{2.609798in}{2.324984in}}%
\pgfpathlineto{\pgfqpoint{2.596465in}{2.342007in}}%
\pgfpathlineto{\pgfqpoint{2.583123in}{2.359294in}}%
\pgfpathlineto{\pgfqpoint{2.569772in}{2.376849in}}%
\pgfpathlineto{\pgfqpoint{2.561381in}{2.374462in}}%
\pgfpathlineto{\pgfqpoint{2.552976in}{2.372274in}}%
\pgfpathlineto{\pgfqpoint{2.544557in}{2.370289in}}%
\pgfpathlineto{\pgfqpoint{2.536124in}{2.368509in}}%
\pgfpathclose%
\pgfusepath{fill}%
\end{pgfscope}%
\begin{pgfscope}%
\pgfpathrectangle{\pgfqpoint{1.254980in}{0.150000in}}{\pgfqpoint{5.490039in}{5.490039in}}%
\pgfusepath{clip}%
\pgfsetbuttcap%
\pgfsetroundjoin%
\definecolor{currentfill}{rgb}{0.280868,0.160771,0.472899}%
\pgfsetfillcolor{currentfill}%
\pgfsetfillopacity{0.700000}%
\pgfsetlinewidth{0.000000pt}%
\definecolor{currentstroke}{rgb}{0.000000,0.000000,0.000000}%
\pgfsetstrokecolor{currentstroke}%
\pgfsetdash{}{0pt}%
\pgfpathmoveto{\pgfqpoint{2.749385in}{2.112079in}}%
\pgfpathlineto{\pgfqpoint{2.762657in}{2.098149in}}%
\pgfpathlineto{\pgfqpoint{2.775923in}{2.084455in}}%
\pgfpathlineto{\pgfqpoint{2.789184in}{2.070995in}}%
\pgfpathlineto{\pgfqpoint{2.802440in}{2.057766in}}%
\pgfpathlineto{\pgfqpoint{2.810696in}{2.061600in}}%
\pgfpathlineto{\pgfqpoint{2.818939in}{2.065602in}}%
\pgfpathlineto{\pgfqpoint{2.827171in}{2.069770in}}%
\pgfpathlineto{\pgfqpoint{2.835392in}{2.074099in}}%
\pgfpathlineto{\pgfqpoint{2.822167in}{2.086934in}}%
\pgfpathlineto{\pgfqpoint{2.808937in}{2.100000in}}%
\pgfpathlineto{\pgfqpoint{2.795703in}{2.113299in}}%
\pgfpathlineto{\pgfqpoint{2.782464in}{2.126833in}}%
\pgfpathlineto{\pgfqpoint{2.774212in}{2.122887in}}%
\pgfpathlineto{\pgfqpoint{2.765949in}{2.119111in}}%
\pgfpathlineto{\pgfqpoint{2.757673in}{2.115507in}}%
\pgfpathlineto{\pgfqpoint{2.749385in}{2.112079in}}%
\pgfpathclose%
\pgfusepath{fill}%
\end{pgfscope}%
\begin{pgfscope}%
\pgfpathrectangle{\pgfqpoint{1.254980in}{0.150000in}}{\pgfqpoint{5.490039in}{5.490039in}}%
\pgfusepath{clip}%
\pgfsetbuttcap%
\pgfsetroundjoin%
\definecolor{currentfill}{rgb}{0.277941,0.056324,0.381191}%
\pgfsetfillcolor{currentfill}%
\pgfsetfillopacity{0.700000}%
\pgfsetlinewidth{0.000000pt}%
\definecolor{currentstroke}{rgb}{0.000000,0.000000,0.000000}%
\pgfsetstrokecolor{currentstroke}%
\pgfsetdash{}{0pt}%
\pgfpathmoveto{\pgfqpoint{3.046549in}{1.898898in}}%
\pgfpathlineto{\pgfqpoint{3.059729in}{1.889758in}}%
\pgfpathlineto{\pgfqpoint{3.072909in}{1.880824in}}%
\pgfpathlineto{\pgfqpoint{3.086087in}{1.872094in}}%
\pgfpathlineto{\pgfqpoint{3.099265in}{1.863566in}}%
\pgfpathlineto{\pgfqpoint{3.107341in}{1.869912in}}%
\pgfpathlineto{\pgfqpoint{3.115408in}{1.876376in}}%
\pgfpathlineto{\pgfqpoint{3.123466in}{1.882956in}}%
\pgfpathlineto{\pgfqpoint{3.131516in}{1.889646in}}%
\pgfpathlineto{\pgfqpoint{3.118361in}{1.897816in}}%
\pgfpathlineto{\pgfqpoint{3.105206in}{1.906188in}}%
\pgfpathlineto{\pgfqpoint{3.092050in}{1.914763in}}%
\pgfpathlineto{\pgfqpoint{3.078894in}{1.923543in}}%
\pgfpathlineto{\pgfqpoint{3.070821in}{1.917201in}}%
\pgfpathlineto{\pgfqpoint{3.062740in}{1.910976in}}%
\pgfpathlineto{\pgfqpoint{3.054649in}{1.904874in}}%
\pgfpathlineto{\pgfqpoint{3.046549in}{1.898898in}}%
\pgfpathclose%
\pgfusepath{fill}%
\end{pgfscope}%
\begin{pgfscope}%
\pgfpathrectangle{\pgfqpoint{1.254980in}{0.150000in}}{\pgfqpoint{5.490039in}{5.490039in}}%
\pgfusepath{clip}%
\pgfsetbuttcap%
\pgfsetroundjoin%
\definecolor{currentfill}{rgb}{0.255645,0.260703,0.528312}%
\pgfsetfillcolor{currentfill}%
\pgfsetfillopacity{0.700000}%
\pgfsetlinewidth{0.000000pt}%
\definecolor{currentstroke}{rgb}{0.000000,0.000000,0.000000}%
\pgfsetstrokecolor{currentstroke}%
\pgfsetdash{}{0pt}%
\pgfpathmoveto{\pgfqpoint{4.297439in}{2.268945in}}%
\pgfpathlineto{\pgfqpoint{4.310831in}{2.273731in}}%
\pgfpathlineto{\pgfqpoint{4.324234in}{2.278681in}}%
\pgfpathlineto{\pgfqpoint{4.337649in}{2.283794in}}%
\pgfpathlineto{\pgfqpoint{4.351075in}{2.289071in}}%
\pgfpathlineto{\pgfqpoint{4.358670in}{2.298798in}}%
\pgfpathlineto{\pgfqpoint{4.366260in}{2.308454in}}%
\pgfpathlineto{\pgfqpoint{4.373844in}{2.318041in}}%
\pgfpathlineto{\pgfqpoint{4.381422in}{2.327558in}}%
\pgfpathlineto{\pgfqpoint{4.368002in}{2.322290in}}%
\pgfpathlineto{\pgfqpoint{4.354593in}{2.317185in}}%
\pgfpathlineto{\pgfqpoint{4.341196in}{2.312244in}}%
\pgfpathlineto{\pgfqpoint{4.327810in}{2.307466in}}%
\pgfpathlineto{\pgfqpoint{4.320225in}{2.297930in}}%
\pgfpathlineto{\pgfqpoint{4.312635in}{2.288331in}}%
\pgfpathlineto{\pgfqpoint{4.305040in}{2.278670in}}%
\pgfpathlineto{\pgfqpoint{4.297439in}{2.268945in}}%
\pgfpathclose%
\pgfusepath{fill}%
\end{pgfscope}%
\begin{pgfscope}%
\pgfpathrectangle{\pgfqpoint{1.254980in}{0.150000in}}{\pgfqpoint{5.490039in}{5.490039in}}%
\pgfusepath{clip}%
\pgfsetbuttcap%
\pgfsetroundjoin%
\definecolor{currentfill}{rgb}{0.273809,0.031497,0.358853}%
\pgfsetfillcolor{currentfill}%
\pgfsetfillopacity{0.700000}%
\pgfsetlinewidth{0.000000pt}%
\definecolor{currentstroke}{rgb}{0.000000,0.000000,0.000000}%
\pgfsetstrokecolor{currentstroke}%
\pgfsetdash{}{0pt}%
\pgfpathmoveto{\pgfqpoint{3.458223in}{1.836890in}}%
\pgfpathlineto{\pgfqpoint{3.471383in}{1.833364in}}%
\pgfpathlineto{\pgfqpoint{3.484548in}{1.830020in}}%
\pgfpathlineto{\pgfqpoint{3.497716in}{1.826857in}}%
\pgfpathlineto{\pgfqpoint{3.510889in}{1.823873in}}%
\pgfpathlineto{\pgfqpoint{3.518769in}{1.833128in}}%
\pgfpathlineto{\pgfqpoint{3.526642in}{1.842423in}}%
\pgfpathlineto{\pgfqpoint{3.534510in}{1.851756in}}%
\pgfpathlineto{\pgfqpoint{3.542372in}{1.861126in}}%
\pgfpathlineto{\pgfqpoint{3.529212in}{1.863837in}}%
\pgfpathlineto{\pgfqpoint{3.516057in}{1.866729in}}%
\pgfpathlineto{\pgfqpoint{3.502906in}{1.869802in}}%
\pgfpathlineto{\pgfqpoint{3.489759in}{1.873056in}}%
\pgfpathlineto{\pgfqpoint{3.481884in}{1.863948in}}%
\pgfpathlineto{\pgfqpoint{3.474003in}{1.854883in}}%
\pgfpathlineto{\pgfqpoint{3.466116in}{1.845863in}}%
\pgfpathlineto{\pgfqpoint{3.458223in}{1.836890in}}%
\pgfpathclose%
\pgfusepath{fill}%
\end{pgfscope}%
\begin{pgfscope}%
\pgfpathrectangle{\pgfqpoint{1.254980in}{0.150000in}}{\pgfqpoint{5.490039in}{5.490039in}}%
\pgfusepath{clip}%
\pgfsetbuttcap%
\pgfsetroundjoin%
\definecolor{currentfill}{rgb}{0.135066,0.544853,0.554029}%
\pgfsetfillcolor{currentfill}%
\pgfsetfillopacity{0.700000}%
\pgfsetlinewidth{0.000000pt}%
\definecolor{currentstroke}{rgb}{0.000000,0.000000,0.000000}%
\pgfsetstrokecolor{currentstroke}%
\pgfsetdash{}{0pt}%
\pgfpathmoveto{\pgfqpoint{5.281847in}{2.979979in}}%
\pgfpathlineto{\pgfqpoint{5.295734in}{2.989398in}}%
\pgfpathlineto{\pgfqpoint{5.309638in}{2.998973in}}%
\pgfpathlineto{\pgfqpoint{5.323559in}{3.008704in}}%
\pgfpathlineto{\pgfqpoint{5.337498in}{3.018592in}}%
\pgfpathlineto{\pgfqpoint{5.344637in}{3.022818in}}%
\pgfpathlineto{\pgfqpoint{5.351769in}{3.027020in}}%
\pgfpathlineto{\pgfqpoint{5.358895in}{3.031202in}}%
\pgfpathlineto{\pgfqpoint{5.366015in}{3.035368in}}%
\pgfpathlineto{\pgfqpoint{5.352096in}{3.025873in}}%
\pgfpathlineto{\pgfqpoint{5.338195in}{3.016534in}}%
\pgfpathlineto{\pgfqpoint{5.324310in}{3.007351in}}%
\pgfpathlineto{\pgfqpoint{5.310442in}{2.998323in}}%
\pgfpathlineto{\pgfqpoint{5.303303in}{2.993755in}}%
\pgfpathlineto{\pgfqpoint{5.296157in}{2.989178in}}%
\pgfpathlineto{\pgfqpoint{5.289005in}{2.984587in}}%
\pgfpathlineto{\pgfqpoint{5.281847in}{2.979979in}}%
\pgfpathclose%
\pgfusepath{fill}%
\end{pgfscope}%
\begin{pgfscope}%
\pgfpathrectangle{\pgfqpoint{1.254980in}{0.150000in}}{\pgfqpoint{5.490039in}{5.490039in}}%
\pgfusepath{clip}%
\pgfsetbuttcap%
\pgfsetroundjoin%
\definecolor{currentfill}{rgb}{0.233603,0.313828,0.543914}%
\pgfsetfillcolor{currentfill}%
\pgfsetfillopacity{0.700000}%
\pgfsetlinewidth{0.000000pt}%
\definecolor{currentstroke}{rgb}{0.000000,0.000000,0.000000}%
\pgfsetstrokecolor{currentstroke}%
\pgfsetdash{}{0pt}%
\pgfpathmoveto{\pgfqpoint{2.482468in}{2.443085in}}%
\pgfpathlineto{\pgfqpoint{2.495897in}{2.424026in}}%
\pgfpathlineto{\pgfqpoint{2.509316in}{2.405246in}}%
\pgfpathlineto{\pgfqpoint{2.522725in}{2.386741in}}%
\pgfpathlineto{\pgfqpoint{2.536124in}{2.368509in}}%
\pgfpathlineto{\pgfqpoint{2.544557in}{2.370289in}}%
\pgfpathlineto{\pgfqpoint{2.552976in}{2.372274in}}%
\pgfpathlineto{\pgfqpoint{2.561381in}{2.374462in}}%
\pgfpathlineto{\pgfqpoint{2.569772in}{2.376849in}}%
\pgfpathlineto{\pgfqpoint{2.556412in}{2.394674in}}%
\pgfpathlineto{\pgfqpoint{2.543042in}{2.412771in}}%
\pgfpathlineto{\pgfqpoint{2.529663in}{2.431143in}}%
\pgfpathlineto{\pgfqpoint{2.516274in}{2.449792in}}%
\pgfpathlineto{\pgfqpoint{2.507844in}{2.447802in}}%
\pgfpathlineto{\pgfqpoint{2.499400in}{2.446018in}}%
\pgfpathlineto{\pgfqpoint{2.490942in}{2.444445in}}%
\pgfpathlineto{\pgfqpoint{2.482468in}{2.443085in}}%
\pgfpathclose%
\pgfusepath{fill}%
\end{pgfscope}%
\begin{pgfscope}%
\pgfpathrectangle{\pgfqpoint{1.254980in}{0.150000in}}{\pgfqpoint{5.490039in}{5.490039in}}%
\pgfusepath{clip}%
\pgfsetbuttcap%
\pgfsetroundjoin%
\definecolor{currentfill}{rgb}{0.282884,0.135920,0.453427}%
\pgfsetfillcolor{currentfill}%
\pgfsetfillopacity{0.700000}%
\pgfsetlinewidth{0.000000pt}%
\definecolor{currentstroke}{rgb}{0.000000,0.000000,0.000000}%
\pgfsetstrokecolor{currentstroke}%
\pgfsetdash{}{0pt}%
\pgfpathmoveto{\pgfqpoint{2.802440in}{2.057766in}}%
\pgfpathlineto{\pgfqpoint{2.815692in}{2.044767in}}%
\pgfpathlineto{\pgfqpoint{2.828939in}{2.031996in}}%
\pgfpathlineto{\pgfqpoint{2.842182in}{2.019452in}}%
\pgfpathlineto{\pgfqpoint{2.855421in}{2.007133in}}%
\pgfpathlineto{\pgfqpoint{2.863645in}{2.011371in}}%
\pgfpathlineto{\pgfqpoint{2.871858in}{2.015771in}}%
\pgfpathlineto{\pgfqpoint{2.880060in}{2.020328in}}%
\pgfpathlineto{\pgfqpoint{2.888250in}{2.025039in}}%
\pgfpathlineto{\pgfqpoint{2.875041in}{2.036966in}}%
\pgfpathlineto{\pgfqpoint{2.861829in}{2.049117in}}%
\pgfpathlineto{\pgfqpoint{2.848612in}{2.061494in}}%
\pgfpathlineto{\pgfqpoint{2.835392in}{2.074099in}}%
\pgfpathlineto{\pgfqpoint{2.827171in}{2.069770in}}%
\pgfpathlineto{\pgfqpoint{2.818939in}{2.065602in}}%
\pgfpathlineto{\pgfqpoint{2.810696in}{2.061600in}}%
\pgfpathlineto{\pgfqpoint{2.802440in}{2.057766in}}%
\pgfpathclose%
\pgfusepath{fill}%
\end{pgfscope}%
\begin{pgfscope}%
\pgfpathrectangle{\pgfqpoint{1.254980in}{0.150000in}}{\pgfqpoint{5.490039in}{5.490039in}}%
\pgfusepath{clip}%
\pgfsetbuttcap%
\pgfsetroundjoin%
\definecolor{currentfill}{rgb}{0.190631,0.407061,0.556089}%
\pgfsetfillcolor{currentfill}%
\pgfsetfillopacity{0.700000}%
\pgfsetlinewidth{0.000000pt}%
\definecolor{currentstroke}{rgb}{0.000000,0.000000,0.000000}%
\pgfsetstrokecolor{currentstroke}%
\pgfsetdash{}{0pt}%
\pgfpathmoveto{\pgfqpoint{4.747695in}{2.603525in}}%
\pgfpathlineto{\pgfqpoint{4.761301in}{2.611058in}}%
\pgfpathlineto{\pgfqpoint{4.774922in}{2.618752in}}%
\pgfpathlineto{\pgfqpoint{4.788558in}{2.626605in}}%
\pgfpathlineto{\pgfqpoint{4.802208in}{2.634618in}}%
\pgfpathlineto{\pgfqpoint{4.809623in}{2.642081in}}%
\pgfpathlineto{\pgfqpoint{4.817032in}{2.649469in}}%
\pgfpathlineto{\pgfqpoint{4.824435in}{2.656785in}}%
\pgfpathlineto{\pgfqpoint{4.831831in}{2.664031in}}%
\pgfpathlineto{\pgfqpoint{4.818190in}{2.656202in}}%
\pgfpathlineto{\pgfqpoint{4.804565in}{2.648532in}}%
\pgfpathlineto{\pgfqpoint{4.790953in}{2.641022in}}%
\pgfpathlineto{\pgfqpoint{4.777356in}{2.633671in}}%
\pgfpathlineto{\pgfqpoint{4.769950in}{2.626231in}}%
\pgfpathlineto{\pgfqpoint{4.762538in}{2.618728in}}%
\pgfpathlineto{\pgfqpoint{4.755119in}{2.611160in}}%
\pgfpathlineto{\pgfqpoint{4.747695in}{2.603525in}}%
\pgfpathclose%
\pgfusepath{fill}%
\end{pgfscope}%
\begin{pgfscope}%
\pgfpathrectangle{\pgfqpoint{1.254980in}{0.150000in}}{\pgfqpoint{5.490039in}{5.490039in}}%
\pgfusepath{clip}%
\pgfsetbuttcap%
\pgfsetroundjoin%
\definecolor{currentfill}{rgb}{0.127568,0.566949,0.550556}%
\pgfsetfillcolor{currentfill}%
\pgfsetfillopacity{0.700000}%
\pgfsetlinewidth{0.000000pt}%
\definecolor{currentstroke}{rgb}{0.000000,0.000000,0.000000}%
\pgfsetstrokecolor{currentstroke}%
\pgfsetdash{}{0pt}%
\pgfpathmoveto{\pgfqpoint{5.366015in}{3.035368in}}%
\pgfpathlineto{\pgfqpoint{5.379950in}{3.045019in}}%
\pgfpathlineto{\pgfqpoint{5.393903in}{3.054826in}}%
\pgfpathlineto{\pgfqpoint{5.407874in}{3.064789in}}%
\pgfpathlineto{\pgfqpoint{5.421862in}{3.074908in}}%
\pgfpathlineto{\pgfqpoint{5.428954in}{3.078651in}}%
\pgfpathlineto{\pgfqpoint{5.436039in}{3.082381in}}%
\pgfpathlineto{\pgfqpoint{5.443118in}{3.086102in}}%
\pgfpathlineto{\pgfqpoint{5.450191in}{3.089819in}}%
\pgfpathlineto{\pgfqpoint{5.436225in}{3.080123in}}%
\pgfpathlineto{\pgfqpoint{5.422276in}{3.070583in}}%
\pgfpathlineto{\pgfqpoint{5.408344in}{3.061197in}}%
\pgfpathlineto{\pgfqpoint{5.394430in}{3.051968in}}%
\pgfpathlineto{\pgfqpoint{5.387335in}{3.047818in}}%
\pgfpathlineto{\pgfqpoint{5.380234in}{3.043671in}}%
\pgfpathlineto{\pgfqpoint{5.373128in}{3.039523in}}%
\pgfpathlineto{\pgfqpoint{5.366015in}{3.035368in}}%
\pgfpathclose%
\pgfusepath{fill}%
\end{pgfscope}%
\begin{pgfscope}%
\pgfpathrectangle{\pgfqpoint{1.254980in}{0.150000in}}{\pgfqpoint{5.490039in}{5.490039in}}%
\pgfusepath{clip}%
\pgfsetbuttcap%
\pgfsetroundjoin%
\definecolor{currentfill}{rgb}{0.243113,0.292092,0.538516}%
\pgfsetfillcolor{currentfill}%
\pgfsetfillopacity{0.700000}%
\pgfsetlinewidth{0.000000pt}%
\definecolor{currentstroke}{rgb}{0.000000,0.000000,0.000000}%
\pgfsetstrokecolor{currentstroke}%
\pgfsetdash{}{0pt}%
\pgfpathmoveto{\pgfqpoint{4.381422in}{2.327558in}}%
\pgfpathlineto{\pgfqpoint{4.394855in}{2.332989in}}%
\pgfpathlineto{\pgfqpoint{4.408299in}{2.338583in}}%
\pgfpathlineto{\pgfqpoint{4.421756in}{2.344339in}}%
\pgfpathlineto{\pgfqpoint{4.435225in}{2.350258in}}%
\pgfpathlineto{\pgfqpoint{4.442792in}{2.359679in}}%
\pgfpathlineto{\pgfqpoint{4.450354in}{2.369025in}}%
\pgfpathlineto{\pgfqpoint{4.457910in}{2.378297in}}%
\pgfpathlineto{\pgfqpoint{4.465460in}{2.387495in}}%
\pgfpathlineto{\pgfqpoint{4.451997in}{2.381614in}}%
\pgfpathlineto{\pgfqpoint{4.438546in}{2.375895in}}%
\pgfpathlineto{\pgfqpoint{4.425108in}{2.370339in}}%
\pgfpathlineto{\pgfqpoint{4.411682in}{2.364945in}}%
\pgfpathlineto{\pgfqpoint{4.404125in}{2.355699in}}%
\pgfpathlineto{\pgfqpoint{4.396563in}{2.346386in}}%
\pgfpathlineto{\pgfqpoint{4.388995in}{2.337006in}}%
\pgfpathlineto{\pgfqpoint{4.381422in}{2.327558in}}%
\pgfpathclose%
\pgfusepath{fill}%
\end{pgfscope}%
\begin{pgfscope}%
\pgfpathrectangle{\pgfqpoint{1.254980in}{0.150000in}}{\pgfqpoint{5.490039in}{5.490039in}}%
\pgfusepath{clip}%
\pgfsetbuttcap%
\pgfsetroundjoin%
\definecolor{currentfill}{rgb}{0.272594,0.025563,0.353093}%
\pgfsetfillcolor{currentfill}%
\pgfsetfillopacity{0.700000}%
\pgfsetlinewidth{0.000000pt}%
\definecolor{currentstroke}{rgb}{0.000000,0.000000,0.000000}%
\pgfsetstrokecolor{currentstroke}%
\pgfsetdash{}{0pt}%
\pgfpathmoveto{\pgfqpoint{3.236770in}{1.831445in}}%
\pgfpathlineto{\pgfqpoint{3.249932in}{1.825049in}}%
\pgfpathlineto{\pgfqpoint{3.263094in}{1.818844in}}%
\pgfpathlineto{\pgfqpoint{3.276259in}{1.812830in}}%
\pgfpathlineto{\pgfqpoint{3.289425in}{1.807006in}}%
\pgfpathlineto{\pgfqpoint{3.297406in}{1.814823in}}%
\pgfpathlineto{\pgfqpoint{3.305379in}{1.822724in}}%
\pgfpathlineto{\pgfqpoint{3.313344in}{1.830706in}}%
\pgfpathlineto{\pgfqpoint{3.321303in}{1.838766in}}%
\pgfpathlineto{\pgfqpoint{3.308154in}{1.844262in}}%
\pgfpathlineto{\pgfqpoint{3.295008in}{1.849948in}}%
\pgfpathlineto{\pgfqpoint{3.281864in}{1.855824in}}%
\pgfpathlineto{\pgfqpoint{3.268722in}{1.861892in}}%
\pgfpathlineto{\pgfqpoint{3.260746in}{1.854150in}}%
\pgfpathlineto{\pgfqpoint{3.252761in}{1.846493in}}%
\pgfpathlineto{\pgfqpoint{3.244770in}{1.838924in}}%
\pgfpathlineto{\pgfqpoint{3.236770in}{1.831445in}}%
\pgfpathclose%
\pgfusepath{fill}%
\end{pgfscope}%
\begin{pgfscope}%
\pgfpathrectangle{\pgfqpoint{1.254980in}{0.150000in}}{\pgfqpoint{5.490039in}{5.490039in}}%
\pgfusepath{clip}%
\pgfsetbuttcap%
\pgfsetroundjoin%
\definecolor{currentfill}{rgb}{0.283197,0.115680,0.436115}%
\pgfsetfillcolor{currentfill}%
\pgfsetfillopacity{0.700000}%
\pgfsetlinewidth{0.000000pt}%
\definecolor{currentstroke}{rgb}{0.000000,0.000000,0.000000}%
\pgfsetstrokecolor{currentstroke}%
\pgfsetdash{}{0pt}%
\pgfpathmoveto{\pgfqpoint{2.855421in}{2.007133in}}%
\pgfpathlineto{\pgfqpoint{2.868657in}{1.995038in}}%
\pgfpathlineto{\pgfqpoint{2.881888in}{1.983165in}}%
\pgfpathlineto{\pgfqpoint{2.895117in}{1.971512in}}%
\pgfpathlineto{\pgfqpoint{2.908342in}{1.960077in}}%
\pgfpathlineto{\pgfqpoint{2.916536in}{1.964718in}}%
\pgfpathlineto{\pgfqpoint{2.924719in}{1.969512in}}%
\pgfpathlineto{\pgfqpoint{2.932892in}{1.974457in}}%
\pgfpathlineto{\pgfqpoint{2.941054in}{1.979548in}}%
\pgfpathlineto{\pgfqpoint{2.927858in}{1.990592in}}%
\pgfpathlineto{\pgfqpoint{2.914658in}{2.001854in}}%
\pgfpathlineto{\pgfqpoint{2.901456in}{2.013335in}}%
\pgfpathlineto{\pgfqpoint{2.888250in}{2.025039in}}%
\pgfpathlineto{\pgfqpoint{2.880060in}{2.020328in}}%
\pgfpathlineto{\pgfqpoint{2.871858in}{2.015771in}}%
\pgfpathlineto{\pgfqpoint{2.863645in}{2.011371in}}%
\pgfpathlineto{\pgfqpoint{2.855421in}{2.007133in}}%
\pgfpathclose%
\pgfusepath{fill}%
\end{pgfscope}%
\begin{pgfscope}%
\pgfpathrectangle{\pgfqpoint{1.254980in}{0.150000in}}{\pgfqpoint{5.490039in}{5.490039in}}%
\pgfusepath{clip}%
\pgfsetbuttcap%
\pgfsetroundjoin%
\definecolor{currentfill}{rgb}{0.218130,0.347432,0.550038}%
\pgfsetfillcolor{currentfill}%
\pgfsetfillopacity{0.700000}%
\pgfsetlinewidth{0.000000pt}%
\definecolor{currentstroke}{rgb}{0.000000,0.000000,0.000000}%
\pgfsetstrokecolor{currentstroke}%
\pgfsetdash{}{0pt}%
\pgfpathmoveto{\pgfqpoint{2.428640in}{2.522153in}}%
\pgfpathlineto{\pgfqpoint{2.442114in}{2.501956in}}%
\pgfpathlineto{\pgfqpoint{2.455576in}{2.482047in}}%
\pgfpathlineto{\pgfqpoint{2.469028in}{2.462424in}}%
\pgfpathlineto{\pgfqpoint{2.482468in}{2.443085in}}%
\pgfpathlineto{\pgfqpoint{2.490942in}{2.444445in}}%
\pgfpathlineto{\pgfqpoint{2.499400in}{2.446018in}}%
\pgfpathlineto{\pgfqpoint{2.507844in}{2.447802in}}%
\pgfpathlineto{\pgfqpoint{2.516274in}{2.449792in}}%
\pgfpathlineto{\pgfqpoint{2.502874in}{2.468722in}}%
\pgfpathlineto{\pgfqpoint{2.489464in}{2.487933in}}%
\pgfpathlineto{\pgfqpoint{2.476043in}{2.507430in}}%
\pgfpathlineto{\pgfqpoint{2.462610in}{2.527215in}}%
\pgfpathlineto{\pgfqpoint{2.454141in}{2.525624in}}%
\pgfpathlineto{\pgfqpoint{2.445656in}{2.524248in}}%
\pgfpathlineto{\pgfqpoint{2.437156in}{2.523090in}}%
\pgfpathlineto{\pgfqpoint{2.428640in}{2.522153in}}%
\pgfpathclose%
\pgfusepath{fill}%
\end{pgfscope}%
\begin{pgfscope}%
\pgfpathrectangle{\pgfqpoint{1.254980in}{0.150000in}}{\pgfqpoint{5.490039in}{5.490039in}}%
\pgfusepath{clip}%
\pgfsetbuttcap%
\pgfsetroundjoin%
\definecolor{currentfill}{rgb}{0.180629,0.429975,0.557282}%
\pgfsetfillcolor{currentfill}%
\pgfsetfillopacity{0.700000}%
\pgfsetlinewidth{0.000000pt}%
\definecolor{currentstroke}{rgb}{0.000000,0.000000,0.000000}%
\pgfsetstrokecolor{currentstroke}%
\pgfsetdash{}{0pt}%
\pgfpathmoveto{\pgfqpoint{4.831831in}{2.664031in}}%
\pgfpathlineto{\pgfqpoint{4.845486in}{2.672019in}}%
\pgfpathlineto{\pgfqpoint{4.859155in}{2.680167in}}%
\pgfpathlineto{\pgfqpoint{4.872840in}{2.688474in}}%
\pgfpathlineto{\pgfqpoint{4.886540in}{2.696940in}}%
\pgfpathlineto{\pgfqpoint{4.893919in}{2.703917in}}%
\pgfpathlineto{\pgfqpoint{4.901291in}{2.710822in}}%
\pgfpathlineto{\pgfqpoint{4.908657in}{2.717657in}}%
\pgfpathlineto{\pgfqpoint{4.916016in}{2.724426in}}%
\pgfpathlineto{\pgfqpoint{4.902327in}{2.716174in}}%
\pgfpathlineto{\pgfqpoint{4.888653in}{2.708080in}}%
\pgfpathlineto{\pgfqpoint{4.874994in}{2.700146in}}%
\pgfpathlineto{\pgfqpoint{4.861349in}{2.692370in}}%
\pgfpathlineto{\pgfqpoint{4.853979in}{2.685377in}}%
\pgfpathlineto{\pgfqpoint{4.846603in}{2.678325in}}%
\pgfpathlineto{\pgfqpoint{4.839220in}{2.671210in}}%
\pgfpathlineto{\pgfqpoint{4.831831in}{2.664031in}}%
\pgfpathclose%
\pgfusepath{fill}%
\end{pgfscope}%
\begin{pgfscope}%
\pgfpathrectangle{\pgfqpoint{1.254980in}{0.150000in}}{\pgfqpoint{5.490039in}{5.490039in}}%
\pgfusepath{clip}%
\pgfsetbuttcap%
\pgfsetroundjoin%
\definecolor{currentfill}{rgb}{0.121831,0.589055,0.545623}%
\pgfsetfillcolor{currentfill}%
\pgfsetfillopacity{0.700000}%
\pgfsetlinewidth{0.000000pt}%
\definecolor{currentstroke}{rgb}{0.000000,0.000000,0.000000}%
\pgfsetstrokecolor{currentstroke}%
\pgfsetdash{}{0pt}%
\pgfpathmoveto{\pgfqpoint{5.450191in}{3.089819in}}%
\pgfpathlineto{\pgfqpoint{5.464175in}{3.099670in}}%
\pgfpathlineto{\pgfqpoint{5.478176in}{3.109677in}}%
\pgfpathlineto{\pgfqpoint{5.492196in}{3.119840in}}%
\pgfpathlineto{\pgfqpoint{5.506233in}{3.130158in}}%
\pgfpathlineto{\pgfqpoint{5.513276in}{3.133434in}}%
\pgfpathlineto{\pgfqpoint{5.520314in}{3.136708in}}%
\pgfpathlineto{\pgfqpoint{5.527345in}{3.139987in}}%
\pgfpathlineto{\pgfqpoint{5.534370in}{3.143275in}}%
\pgfpathlineto{\pgfqpoint{5.520356in}{3.133410in}}%
\pgfpathlineto{\pgfqpoint{5.506360in}{3.123700in}}%
\pgfpathlineto{\pgfqpoint{5.492382in}{3.114145in}}%
\pgfpathlineto{\pgfqpoint{5.478421in}{3.104745in}}%
\pgfpathlineto{\pgfqpoint{5.471373in}{3.100995in}}%
\pgfpathlineto{\pgfqpoint{5.464318in}{3.097261in}}%
\pgfpathlineto{\pgfqpoint{5.457258in}{3.093537in}}%
\pgfpathlineto{\pgfqpoint{5.450191in}{3.089819in}}%
\pgfpathclose%
\pgfusepath{fill}%
\end{pgfscope}%
\begin{pgfscope}%
\pgfpathrectangle{\pgfqpoint{1.254980in}{0.150000in}}{\pgfqpoint{5.490039in}{5.490039in}}%
\pgfusepath{clip}%
\pgfsetbuttcap%
\pgfsetroundjoin%
\definecolor{currentfill}{rgb}{0.283197,0.115680,0.436115}%
\pgfsetfillcolor{currentfill}%
\pgfsetfillopacity{0.700000}%
\pgfsetlinewidth{0.000000pt}%
\definecolor{currentstroke}{rgb}{0.000000,0.000000,0.000000}%
\pgfsetstrokecolor{currentstroke}%
\pgfsetdash{}{0pt}%
\pgfpathmoveto{\pgfqpoint{3.847099in}{1.966571in}}%
\pgfpathlineto{\pgfqpoint{3.860336in}{1.967478in}}%
\pgfpathlineto{\pgfqpoint{3.873580in}{1.968555in}}%
\pgfpathlineto{\pgfqpoint{3.886832in}{1.969802in}}%
\pgfpathlineto{\pgfqpoint{3.900092in}{1.971218in}}%
\pgfpathlineto{\pgfqpoint{3.907837in}{1.981727in}}%
\pgfpathlineto{\pgfqpoint{3.915577in}{1.992211in}}%
\pgfpathlineto{\pgfqpoint{3.923313in}{2.002669in}}%
\pgfpathlineto{\pgfqpoint{3.931043in}{2.013101in}}%
\pgfpathlineto{\pgfqpoint{3.917790in}{2.011524in}}%
\pgfpathlineto{\pgfqpoint{3.904545in}{2.010116in}}%
\pgfpathlineto{\pgfqpoint{3.891309in}{2.008878in}}%
\pgfpathlineto{\pgfqpoint{3.878080in}{2.007810in}}%
\pgfpathlineto{\pgfqpoint{3.870342in}{1.997529in}}%
\pgfpathlineto{\pgfqpoint{3.862600in}{1.987228in}}%
\pgfpathlineto{\pgfqpoint{3.854852in}{1.976908in}}%
\pgfpathlineto{\pgfqpoint{3.847099in}{1.966571in}}%
\pgfpathclose%
\pgfusepath{fill}%
\end{pgfscope}%
\begin{pgfscope}%
\pgfpathrectangle{\pgfqpoint{1.254980in}{0.150000in}}{\pgfqpoint{5.490039in}{5.490039in}}%
\pgfusepath{clip}%
\pgfsetbuttcap%
\pgfsetroundjoin%
\definecolor{currentfill}{rgb}{0.281924,0.089666,0.412415}%
\pgfsetfillcolor{currentfill}%
\pgfsetfillopacity{0.700000}%
\pgfsetlinewidth{0.000000pt}%
\definecolor{currentstroke}{rgb}{0.000000,0.000000,0.000000}%
\pgfsetstrokecolor{currentstroke}%
\pgfsetdash{}{0pt}%
\pgfpathmoveto{\pgfqpoint{3.763137in}{1.923910in}}%
\pgfpathlineto{\pgfqpoint{3.776352in}{1.923944in}}%
\pgfpathlineto{\pgfqpoint{3.789573in}{1.924149in}}%
\pgfpathlineto{\pgfqpoint{3.802803in}{1.924526in}}%
\pgfpathlineto{\pgfqpoint{3.816039in}{1.925075in}}%
\pgfpathlineto{\pgfqpoint{3.823812in}{1.935468in}}%
\pgfpathlineto{\pgfqpoint{3.831579in}{1.945849in}}%
\pgfpathlineto{\pgfqpoint{3.839342in}{1.956217in}}%
\pgfpathlineto{\pgfqpoint{3.847099in}{1.966571in}}%
\pgfpathlineto{\pgfqpoint{3.833871in}{1.965834in}}%
\pgfpathlineto{\pgfqpoint{3.820650in}{1.965269in}}%
\pgfpathlineto{\pgfqpoint{3.807437in}{1.964875in}}%
\pgfpathlineto{\pgfqpoint{3.794230in}{1.964653in}}%
\pgfpathlineto{\pgfqpoint{3.786465in}{1.954477in}}%
\pgfpathlineto{\pgfqpoint{3.778694in}{1.944294in}}%
\pgfpathlineto{\pgfqpoint{3.770918in}{1.934105in}}%
\pgfpathlineto{\pgfqpoint{3.763137in}{1.923910in}}%
\pgfpathclose%
\pgfusepath{fill}%
\end{pgfscope}%
\begin{pgfscope}%
\pgfpathrectangle{\pgfqpoint{1.254980in}{0.150000in}}{\pgfqpoint{5.490039in}{5.490039in}}%
\pgfusepath{clip}%
\pgfsetbuttcap%
\pgfsetroundjoin%
\definecolor{currentfill}{rgb}{0.271305,0.019942,0.347269}%
\pgfsetfillcolor{currentfill}%
\pgfsetfillopacity{0.700000}%
\pgfsetlinewidth{0.000000pt}%
\definecolor{currentstroke}{rgb}{0.000000,0.000000,0.000000}%
\pgfsetstrokecolor{currentstroke}%
\pgfsetdash{}{0pt}%
\pgfpathmoveto{\pgfqpoint{3.373921in}{1.818664in}}%
\pgfpathlineto{\pgfqpoint{3.387082in}{1.814104in}}%
\pgfpathlineto{\pgfqpoint{3.400247in}{1.809729in}}%
\pgfpathlineto{\pgfqpoint{3.413415in}{1.805538in}}%
\pgfpathlineto{\pgfqpoint{3.426586in}{1.801530in}}%
\pgfpathlineto{\pgfqpoint{3.434505in}{1.810286in}}%
\pgfpathlineto{\pgfqpoint{3.442417in}{1.819099in}}%
\pgfpathlineto{\pgfqpoint{3.450323in}{1.827968in}}%
\pgfpathlineto{\pgfqpoint{3.458223in}{1.836890in}}%
\pgfpathlineto{\pgfqpoint{3.445066in}{1.840599in}}%
\pgfpathlineto{\pgfqpoint{3.431914in}{1.844490in}}%
\pgfpathlineto{\pgfqpoint{3.418764in}{1.848565in}}%
\pgfpathlineto{\pgfqpoint{3.405619in}{1.852825in}}%
\pgfpathlineto{\pgfqpoint{3.397704in}{1.844193in}}%
\pgfpathlineto{\pgfqpoint{3.389783in}{1.835619in}}%
\pgfpathlineto{\pgfqpoint{3.381855in}{1.827109in}}%
\pgfpathlineto{\pgfqpoint{3.373921in}{1.818664in}}%
\pgfpathclose%
\pgfusepath{fill}%
\end{pgfscope}%
\begin{pgfscope}%
\pgfpathrectangle{\pgfqpoint{1.254980in}{0.150000in}}{\pgfqpoint{5.490039in}{5.490039in}}%
\pgfusepath{clip}%
\pgfsetbuttcap%
\pgfsetroundjoin%
\definecolor{currentfill}{rgb}{0.276022,0.044167,0.370164}%
\pgfsetfillcolor{currentfill}%
\pgfsetfillopacity{0.700000}%
\pgfsetlinewidth{0.000000pt}%
\definecolor{currentstroke}{rgb}{0.000000,0.000000,0.000000}%
\pgfsetstrokecolor{currentstroke}%
\pgfsetdash{}{0pt}%
\pgfpathmoveto{\pgfqpoint{3.099265in}{1.863566in}}%
\pgfpathlineto{\pgfqpoint{3.112443in}{1.855240in}}%
\pgfpathlineto{\pgfqpoint{3.125621in}{1.847114in}}%
\pgfpathlineto{\pgfqpoint{3.138798in}{1.839188in}}%
\pgfpathlineto{\pgfqpoint{3.151976in}{1.831460in}}%
\pgfpathlineto{\pgfqpoint{3.160029in}{1.838174in}}%
\pgfpathlineto{\pgfqpoint{3.168074in}{1.844999in}}%
\pgfpathlineto{\pgfqpoint{3.176110in}{1.851933in}}%
\pgfpathlineto{\pgfqpoint{3.184137in}{1.858970in}}%
\pgfpathlineto{\pgfqpoint{3.170981in}{1.866341in}}%
\pgfpathlineto{\pgfqpoint{3.157826in}{1.873910in}}%
\pgfpathlineto{\pgfqpoint{3.144671in}{1.881678in}}%
\pgfpathlineto{\pgfqpoint{3.131516in}{1.889646in}}%
\pgfpathlineto{\pgfqpoint{3.123466in}{1.882956in}}%
\pgfpathlineto{\pgfqpoint{3.115408in}{1.876376in}}%
\pgfpathlineto{\pgfqpoint{3.107341in}{1.869912in}}%
\pgfpathlineto{\pgfqpoint{3.099265in}{1.863566in}}%
\pgfpathclose%
\pgfusepath{fill}%
\end{pgfscope}%
\begin{pgfscope}%
\pgfpathrectangle{\pgfqpoint{1.254980in}{0.150000in}}{\pgfqpoint{5.490039in}{5.490039in}}%
\pgfusepath{clip}%
\pgfsetbuttcap%
\pgfsetroundjoin%
\definecolor{currentfill}{rgb}{0.282623,0.140926,0.457517}%
\pgfsetfillcolor{currentfill}%
\pgfsetfillopacity{0.700000}%
\pgfsetlinewidth{0.000000pt}%
\definecolor{currentstroke}{rgb}{0.000000,0.000000,0.000000}%
\pgfsetstrokecolor{currentstroke}%
\pgfsetdash{}{0pt}%
\pgfpathmoveto{\pgfqpoint{3.931043in}{2.013101in}}%
\pgfpathlineto{\pgfqpoint{3.944305in}{2.014847in}}%
\pgfpathlineto{\pgfqpoint{3.957575in}{2.016761in}}%
\pgfpathlineto{\pgfqpoint{3.970854in}{2.018844in}}%
\pgfpathlineto{\pgfqpoint{3.984142in}{2.021094in}}%
\pgfpathlineto{\pgfqpoint{3.991861in}{2.031641in}}%
\pgfpathlineto{\pgfqpoint{3.999575in}{2.042151in}}%
\pgfpathlineto{\pgfqpoint{4.007284in}{2.052624in}}%
\pgfpathlineto{\pgfqpoint{4.014988in}{2.063060in}}%
\pgfpathlineto{\pgfqpoint{4.001706in}{2.060676in}}%
\pgfpathlineto{\pgfqpoint{3.988434in}{2.058461in}}%
\pgfpathlineto{\pgfqpoint{3.975171in}{2.056414in}}%
\pgfpathlineto{\pgfqpoint{3.961916in}{2.054535in}}%
\pgfpathlineto{\pgfqpoint{3.954205in}{2.044222in}}%
\pgfpathlineto{\pgfqpoint{3.946489in}{2.033878in}}%
\pgfpathlineto{\pgfqpoint{3.938769in}{2.023504in}}%
\pgfpathlineto{\pgfqpoint{3.931043in}{2.013101in}}%
\pgfpathclose%
\pgfusepath{fill}%
\end{pgfscope}%
\begin{pgfscope}%
\pgfpathrectangle{\pgfqpoint{1.254980in}{0.150000in}}{\pgfqpoint{5.490039in}{5.490039in}}%
\pgfusepath{clip}%
\pgfsetbuttcap%
\pgfsetroundjoin%
\definecolor{currentfill}{rgb}{0.279566,0.067836,0.391917}%
\pgfsetfillcolor{currentfill}%
\pgfsetfillopacity{0.700000}%
\pgfsetlinewidth{0.000000pt}%
\definecolor{currentstroke}{rgb}{0.000000,0.000000,0.000000}%
\pgfsetstrokecolor{currentstroke}%
\pgfsetdash{}{0pt}%
\pgfpathmoveto{\pgfqpoint{3.679132in}{1.885581in}}%
\pgfpathlineto{\pgfqpoint{3.692330in}{1.884706in}}%
\pgfpathlineto{\pgfqpoint{3.705534in}{1.884005in}}%
\pgfpathlineto{\pgfqpoint{3.718745in}{1.883477in}}%
\pgfpathlineto{\pgfqpoint{3.731962in}{1.883121in}}%
\pgfpathlineto{\pgfqpoint{3.739763in}{1.893317in}}%
\pgfpathlineto{\pgfqpoint{3.747559in}{1.903515in}}%
\pgfpathlineto{\pgfqpoint{3.755351in}{1.913713in}}%
\pgfpathlineto{\pgfqpoint{3.763137in}{1.923910in}}%
\pgfpathlineto{\pgfqpoint{3.749929in}{1.924049in}}%
\pgfpathlineto{\pgfqpoint{3.736728in}{1.924361in}}%
\pgfpathlineto{\pgfqpoint{3.723534in}{1.924846in}}%
\pgfpathlineto{\pgfqpoint{3.710346in}{1.925506in}}%
\pgfpathlineto{\pgfqpoint{3.702550in}{1.915514in}}%
\pgfpathlineto{\pgfqpoint{3.694750in}{1.905529in}}%
\pgfpathlineto{\pgfqpoint{3.686944in}{1.895550in}}%
\pgfpathlineto{\pgfqpoint{3.679132in}{1.885581in}}%
\pgfpathclose%
\pgfusepath{fill}%
\end{pgfscope}%
\begin{pgfscope}%
\pgfpathrectangle{\pgfqpoint{1.254980in}{0.150000in}}{\pgfqpoint{5.490039in}{5.490039in}}%
\pgfusepath{clip}%
\pgfsetbuttcap%
\pgfsetroundjoin%
\definecolor{currentfill}{rgb}{0.119423,0.611141,0.538982}%
\pgfsetfillcolor{currentfill}%
\pgfsetfillopacity{0.700000}%
\pgfsetlinewidth{0.000000pt}%
\definecolor{currentstroke}{rgb}{0.000000,0.000000,0.000000}%
\pgfsetstrokecolor{currentstroke}%
\pgfsetdash{}{0pt}%
\pgfpathmoveto{\pgfqpoint{5.534370in}{3.143275in}}%
\pgfpathlineto{\pgfqpoint{5.548401in}{3.153295in}}%
\pgfpathlineto{\pgfqpoint{5.562450in}{3.163470in}}%
\pgfpathlineto{\pgfqpoint{5.576518in}{3.173800in}}%
\pgfpathlineto{\pgfqpoint{5.590604in}{3.184286in}}%
\pgfpathlineto{\pgfqpoint{5.597598in}{3.187115in}}%
\pgfpathlineto{\pgfqpoint{5.604586in}{3.189957in}}%
\pgfpathlineto{\pgfqpoint{5.611568in}{3.192816in}}%
\pgfpathlineto{\pgfqpoint{5.618544in}{3.195699in}}%
\pgfpathlineto{\pgfqpoint{5.604484in}{3.185698in}}%
\pgfpathlineto{\pgfqpoint{5.590442in}{3.175850in}}%
\pgfpathlineto{\pgfqpoint{5.576419in}{3.166157in}}%
\pgfpathlineto{\pgfqpoint{5.562412in}{3.156619in}}%
\pgfpathlineto{\pgfqpoint{5.555410in}{3.153243in}}%
\pgfpathlineto{\pgfqpoint{5.548402in}{3.149898in}}%
\pgfpathlineto{\pgfqpoint{5.541389in}{3.146577in}}%
\pgfpathlineto{\pgfqpoint{5.534370in}{3.143275in}}%
\pgfpathclose%
\pgfusepath{fill}%
\end{pgfscope}%
\begin{pgfscope}%
\pgfpathrectangle{\pgfqpoint{1.254980in}{0.150000in}}{\pgfqpoint{5.490039in}{5.490039in}}%
\pgfusepath{clip}%
\pgfsetbuttcap%
\pgfsetroundjoin%
\definecolor{currentfill}{rgb}{0.229739,0.322361,0.545706}%
\pgfsetfillcolor{currentfill}%
\pgfsetfillopacity{0.700000}%
\pgfsetlinewidth{0.000000pt}%
\definecolor{currentstroke}{rgb}{0.000000,0.000000,0.000000}%
\pgfsetstrokecolor{currentstroke}%
\pgfsetdash{}{0pt}%
\pgfpathmoveto{\pgfqpoint{4.465460in}{2.387495in}}%
\pgfpathlineto{\pgfqpoint{4.478936in}{2.393538in}}%
\pgfpathlineto{\pgfqpoint{4.492424in}{2.399744in}}%
\pgfpathlineto{\pgfqpoint{4.505925in}{2.406111in}}%
\pgfpathlineto{\pgfqpoint{4.519439in}{2.412640in}}%
\pgfpathlineto{\pgfqpoint{4.526977in}{2.421709in}}%
\pgfpathlineto{\pgfqpoint{4.534510in}{2.430700in}}%
\pgfpathlineto{\pgfqpoint{4.542036in}{2.439613in}}%
\pgfpathlineto{\pgfqpoint{4.549557in}{2.448449in}}%
\pgfpathlineto{\pgfqpoint{4.536049in}{2.441987in}}%
\pgfpathlineto{\pgfqpoint{4.522555in}{2.435686in}}%
\pgfpathlineto{\pgfqpoint{4.509073in}{2.429547in}}%
\pgfpathlineto{\pgfqpoint{4.495604in}{2.423570in}}%
\pgfpathlineto{\pgfqpoint{4.488077in}{2.414657in}}%
\pgfpathlineto{\pgfqpoint{4.480544in}{2.405674in}}%
\pgfpathlineto{\pgfqpoint{4.473005in}{2.396620in}}%
\pgfpathlineto{\pgfqpoint{4.465460in}{2.387495in}}%
\pgfpathclose%
\pgfusepath{fill}%
\end{pgfscope}%
\begin{pgfscope}%
\pgfpathrectangle{\pgfqpoint{1.254980in}{0.150000in}}{\pgfqpoint{5.490039in}{5.490039in}}%
\pgfusepath{clip}%
\pgfsetbuttcap%
\pgfsetroundjoin%
\definecolor{currentfill}{rgb}{0.280255,0.165693,0.476498}%
\pgfsetfillcolor{currentfill}%
\pgfsetfillopacity{0.700000}%
\pgfsetlinewidth{0.000000pt}%
\definecolor{currentstroke}{rgb}{0.000000,0.000000,0.000000}%
\pgfsetstrokecolor{currentstroke}%
\pgfsetdash{}{0pt}%
\pgfpathmoveto{\pgfqpoint{4.014988in}{2.063060in}}%
\pgfpathlineto{\pgfqpoint{4.028279in}{2.065610in}}%
\pgfpathlineto{\pgfqpoint{4.041579in}{2.068328in}}%
\pgfpathlineto{\pgfqpoint{4.054888in}{2.071213in}}%
\pgfpathlineto{\pgfqpoint{4.068207in}{2.074264in}}%
\pgfpathlineto{\pgfqpoint{4.075901in}{2.084776in}}%
\pgfpathlineto{\pgfqpoint{4.083589in}{2.095241in}}%
\pgfpathlineto{\pgfqpoint{4.091272in}{2.105658in}}%
\pgfpathlineto{\pgfqpoint{4.098951in}{2.116027in}}%
\pgfpathlineto{\pgfqpoint{4.085638in}{2.112871in}}%
\pgfpathlineto{\pgfqpoint{4.072334in}{2.109882in}}%
\pgfpathlineto{\pgfqpoint{4.059040in}{2.107059in}}%
\pgfpathlineto{\pgfqpoint{4.045756in}{2.104403in}}%
\pgfpathlineto{\pgfqpoint{4.038071in}{2.094128in}}%
\pgfpathlineto{\pgfqpoint{4.030382in}{2.083812in}}%
\pgfpathlineto{\pgfqpoint{4.022687in}{2.073456in}}%
\pgfpathlineto{\pgfqpoint{4.014988in}{2.063060in}}%
\pgfpathclose%
\pgfusepath{fill}%
\end{pgfscope}%
\begin{pgfscope}%
\pgfpathrectangle{\pgfqpoint{1.254980in}{0.150000in}}{\pgfqpoint{5.490039in}{5.490039in}}%
\pgfusepath{clip}%
\pgfsetbuttcap%
\pgfsetroundjoin%
\definecolor{currentfill}{rgb}{0.282327,0.094955,0.417331}%
\pgfsetfillcolor{currentfill}%
\pgfsetfillopacity{0.700000}%
\pgfsetlinewidth{0.000000pt}%
\definecolor{currentstroke}{rgb}{0.000000,0.000000,0.000000}%
\pgfsetstrokecolor{currentstroke}%
\pgfsetdash{}{0pt}%
\pgfpathmoveto{\pgfqpoint{2.908342in}{1.960077in}}%
\pgfpathlineto{\pgfqpoint{2.921564in}{1.948861in}}%
\pgfpathlineto{\pgfqpoint{2.934784in}{1.937860in}}%
\pgfpathlineto{\pgfqpoint{2.948000in}{1.927074in}}%
\pgfpathlineto{\pgfqpoint{2.961215in}{1.916501in}}%
\pgfpathlineto{\pgfqpoint{2.969381in}{1.921542in}}%
\pgfpathlineto{\pgfqpoint{2.977536in}{1.926729in}}%
\pgfpathlineto{\pgfqpoint{2.985681in}{1.932060in}}%
\pgfpathlineto{\pgfqpoint{2.993816in}{1.937531in}}%
\pgfpathlineto{\pgfqpoint{2.980629in}{1.947714in}}%
\pgfpathlineto{\pgfqpoint{2.967440in}{1.958111in}}%
\pgfpathlineto{\pgfqpoint{2.954248in}{1.968722in}}%
\pgfpathlineto{\pgfqpoint{2.941054in}{1.979548in}}%
\pgfpathlineto{\pgfqpoint{2.932892in}{1.974457in}}%
\pgfpathlineto{\pgfqpoint{2.924719in}{1.969512in}}%
\pgfpathlineto{\pgfqpoint{2.916536in}{1.964718in}}%
\pgfpathlineto{\pgfqpoint{2.908342in}{1.960077in}}%
\pgfpathclose%
\pgfusepath{fill}%
\end{pgfscope}%
\begin{pgfscope}%
\pgfpathrectangle{\pgfqpoint{1.254980in}{0.150000in}}{\pgfqpoint{5.490039in}{5.490039in}}%
\pgfusepath{clip}%
\pgfsetbuttcap%
\pgfsetroundjoin%
\definecolor{currentfill}{rgb}{0.169646,0.456262,0.558030}%
\pgfsetfillcolor{currentfill}%
\pgfsetfillopacity{0.700000}%
\pgfsetlinewidth{0.000000pt}%
\definecolor{currentstroke}{rgb}{0.000000,0.000000,0.000000}%
\pgfsetstrokecolor{currentstroke}%
\pgfsetdash{}{0pt}%
\pgfpathmoveto{\pgfqpoint{4.916016in}{2.724426in}}%
\pgfpathlineto{\pgfqpoint{4.929720in}{2.732837in}}%
\pgfpathlineto{\pgfqpoint{4.943439in}{2.741407in}}%
\pgfpathlineto{\pgfqpoint{4.957174in}{2.750136in}}%
\pgfpathlineto{\pgfqpoint{4.970925in}{2.759023in}}%
\pgfpathlineto{\pgfqpoint{4.978265in}{2.765496in}}%
\pgfpathlineto{\pgfqpoint{4.985599in}{2.771900in}}%
\pgfpathlineto{\pgfqpoint{4.992925in}{2.778239in}}%
\pgfpathlineto{\pgfqpoint{5.000245in}{2.784515in}}%
\pgfpathlineto{\pgfqpoint{4.986507in}{2.775872in}}%
\pgfpathlineto{\pgfqpoint{4.972784in}{2.767387in}}%
\pgfpathlineto{\pgfqpoint{4.959077in}{2.759060in}}%
\pgfpathlineto{\pgfqpoint{4.945384in}{2.750891in}}%
\pgfpathlineto{\pgfqpoint{4.938052in}{2.744361in}}%
\pgfpathlineto{\pgfqpoint{4.930713in}{2.737775in}}%
\pgfpathlineto{\pgfqpoint{4.923368in}{2.731131in}}%
\pgfpathlineto{\pgfqpoint{4.916016in}{2.724426in}}%
\pgfpathclose%
\pgfusepath{fill}%
\end{pgfscope}%
\begin{pgfscope}%
\pgfpathrectangle{\pgfqpoint{1.254980in}{0.150000in}}{\pgfqpoint{5.490039in}{5.490039in}}%
\pgfusepath{clip}%
\pgfsetbuttcap%
\pgfsetroundjoin%
\definecolor{currentfill}{rgb}{0.121380,0.629492,0.531973}%
\pgfsetfillcolor{currentfill}%
\pgfsetfillopacity{0.700000}%
\pgfsetlinewidth{0.000000pt}%
\definecolor{currentstroke}{rgb}{0.000000,0.000000,0.000000}%
\pgfsetstrokecolor{currentstroke}%
\pgfsetdash{}{0pt}%
\pgfpathmoveto{\pgfqpoint{5.618544in}{3.195699in}}%
\pgfpathlineto{\pgfqpoint{5.632622in}{3.205856in}}%
\pgfpathlineto{\pgfqpoint{5.646719in}{3.216167in}}%
\pgfpathlineto{\pgfqpoint{5.660834in}{3.226633in}}%
\pgfpathlineto{\pgfqpoint{5.674967in}{3.237254in}}%
\pgfpathlineto{\pgfqpoint{5.681911in}{3.239662in}}%
\pgfpathlineto{\pgfqpoint{5.688849in}{3.242099in}}%
\pgfpathlineto{\pgfqpoint{5.695782in}{3.244568in}}%
\pgfpathlineto{\pgfqpoint{5.702710in}{3.247076in}}%
\pgfpathlineto{\pgfqpoint{5.688604in}{3.236969in}}%
\pgfpathlineto{\pgfqpoint{5.674517in}{3.227016in}}%
\pgfpathlineto{\pgfqpoint{5.660448in}{3.217217in}}%
\pgfpathlineto{\pgfqpoint{5.646398in}{3.207572in}}%
\pgfpathlineto{\pgfqpoint{5.639442in}{3.204542in}}%
\pgfpathlineto{\pgfqpoint{5.632481in}{3.201556in}}%
\pgfpathlineto{\pgfqpoint{5.625515in}{3.198611in}}%
\pgfpathlineto{\pgfqpoint{5.618544in}{3.195699in}}%
\pgfpathclose%
\pgfusepath{fill}%
\end{pgfscope}%
\begin{pgfscope}%
\pgfpathrectangle{\pgfqpoint{1.254980in}{0.150000in}}{\pgfqpoint{5.490039in}{5.490039in}}%
\pgfusepath{clip}%
\pgfsetbuttcap%
\pgfsetroundjoin%
\definecolor{currentfill}{rgb}{0.277018,0.050344,0.375715}%
\pgfsetfillcolor{currentfill}%
\pgfsetfillopacity{0.700000}%
\pgfsetlinewidth{0.000000pt}%
\definecolor{currentstroke}{rgb}{0.000000,0.000000,0.000000}%
\pgfsetstrokecolor{currentstroke}%
\pgfsetdash{}{0pt}%
\pgfpathmoveto{\pgfqpoint{3.595060in}{1.852067in}}%
\pgfpathlineto{\pgfqpoint{3.608245in}{1.850246in}}%
\pgfpathlineto{\pgfqpoint{3.621436in}{1.848602in}}%
\pgfpathlineto{\pgfqpoint{3.634632in}{1.847133in}}%
\pgfpathlineto{\pgfqpoint{3.647834in}{1.845839in}}%
\pgfpathlineto{\pgfqpoint{3.655667in}{1.855750in}}%
\pgfpathlineto{\pgfqpoint{3.663494in}{1.865679in}}%
\pgfpathlineto{\pgfqpoint{3.671316in}{1.875624in}}%
\pgfpathlineto{\pgfqpoint{3.679132in}{1.885581in}}%
\pgfpathlineto{\pgfqpoint{3.665941in}{1.886631in}}%
\pgfpathlineto{\pgfqpoint{3.652756in}{1.887856in}}%
\pgfpathlineto{\pgfqpoint{3.639576in}{1.889257in}}%
\pgfpathlineto{\pgfqpoint{3.626402in}{1.890834in}}%
\pgfpathlineto{\pgfqpoint{3.618575in}{1.881110in}}%
\pgfpathlineto{\pgfqpoint{3.610742in}{1.871406in}}%
\pgfpathlineto{\pgfqpoint{3.602904in}{1.861724in}}%
\pgfpathlineto{\pgfqpoint{3.595060in}{1.852067in}}%
\pgfpathclose%
\pgfusepath{fill}%
\end{pgfscope}%
\begin{pgfscope}%
\pgfpathrectangle{\pgfqpoint{1.254980in}{0.150000in}}{\pgfqpoint{5.490039in}{5.490039in}}%
\pgfusepath{clip}%
\pgfsetbuttcap%
\pgfsetroundjoin%
\definecolor{currentfill}{rgb}{0.201239,0.383670,0.554294}%
\pgfsetfillcolor{currentfill}%
\pgfsetfillopacity{0.700000}%
\pgfsetlinewidth{0.000000pt}%
\definecolor{currentstroke}{rgb}{0.000000,0.000000,0.000000}%
\pgfsetstrokecolor{currentstroke}%
\pgfsetdash{}{0pt}%
\pgfpathmoveto{\pgfqpoint{2.374622in}{2.605887in}}%
\pgfpathlineto{\pgfqpoint{2.388145in}{2.584506in}}%
\pgfpathlineto{\pgfqpoint{2.401656in}{2.563426in}}%
\pgfpathlineto{\pgfqpoint{2.415154in}{2.542642in}}%
\pgfpathlineto{\pgfqpoint{2.428640in}{2.522153in}}%
\pgfpathlineto{\pgfqpoint{2.437156in}{2.523090in}}%
\pgfpathlineto{\pgfqpoint{2.445656in}{2.524248in}}%
\pgfpathlineto{\pgfqpoint{2.454141in}{2.525624in}}%
\pgfpathlineto{\pgfqpoint{2.462610in}{2.527215in}}%
\pgfpathlineto{\pgfqpoint{2.449166in}{2.547290in}}%
\pgfpathlineto{\pgfqpoint{2.435711in}{2.567659in}}%
\pgfpathlineto{\pgfqpoint{2.422243in}{2.588324in}}%
\pgfpathlineto{\pgfqpoint{2.408763in}{2.609288in}}%
\pgfpathlineto{\pgfqpoint{2.400252in}{2.608101in}}%
\pgfpathlineto{\pgfqpoint{2.391725in}{2.607136in}}%
\pgfpathlineto{\pgfqpoint{2.383181in}{2.606397in}}%
\pgfpathlineto{\pgfqpoint{2.374622in}{2.605887in}}%
\pgfpathclose%
\pgfusepath{fill}%
\end{pgfscope}%
\begin{pgfscope}%
\pgfpathrectangle{\pgfqpoint{1.254980in}{0.150000in}}{\pgfqpoint{5.490039in}{5.490039in}}%
\pgfusepath{clip}%
\pgfsetbuttcap%
\pgfsetroundjoin%
\definecolor{currentfill}{rgb}{0.130067,0.651384,0.521608}%
\pgfsetfillcolor{currentfill}%
\pgfsetfillopacity{0.700000}%
\pgfsetlinewidth{0.000000pt}%
\definecolor{currentstroke}{rgb}{0.000000,0.000000,0.000000}%
\pgfsetstrokecolor{currentstroke}%
\pgfsetdash{}{0pt}%
\pgfpathmoveto{\pgfqpoint{5.702710in}{3.247076in}}%
\pgfpathlineto{\pgfqpoint{5.716834in}{3.257337in}}%
\pgfpathlineto{\pgfqpoint{5.730976in}{3.267752in}}%
\pgfpathlineto{\pgfqpoint{5.745138in}{3.278322in}}%
\pgfpathlineto{\pgfqpoint{5.759318in}{3.289046in}}%
\pgfpathlineto{\pgfqpoint{5.766211in}{3.291065in}}%
\pgfpathlineto{\pgfqpoint{5.773099in}{3.293128in}}%
\pgfpathlineto{\pgfqpoint{5.779983in}{3.295241in}}%
\pgfpathlineto{\pgfqpoint{5.786861in}{3.297408in}}%
\pgfpathlineto{\pgfqpoint{5.772711in}{3.287228in}}%
\pgfpathlineto{\pgfqpoint{5.758580in}{3.277201in}}%
\pgfpathlineto{\pgfqpoint{5.744467in}{3.267328in}}%
\pgfpathlineto{\pgfqpoint{5.730373in}{3.257608in}}%
\pgfpathlineto{\pgfqpoint{5.723464in}{3.254888in}}%
\pgfpathlineto{\pgfqpoint{5.716550in}{3.252230in}}%
\pgfpathlineto{\pgfqpoint{5.709632in}{3.249628in}}%
\pgfpathlineto{\pgfqpoint{5.702710in}{3.247076in}}%
\pgfpathclose%
\pgfusepath{fill}%
\end{pgfscope}%
\begin{pgfscope}%
\pgfpathrectangle{\pgfqpoint{1.254980in}{0.150000in}}{\pgfqpoint{5.490039in}{5.490039in}}%
\pgfusepath{clip}%
\pgfsetbuttcap%
\pgfsetroundjoin%
\definecolor{currentfill}{rgb}{0.275191,0.194905,0.496005}%
\pgfsetfillcolor{currentfill}%
\pgfsetfillopacity{0.700000}%
\pgfsetlinewidth{0.000000pt}%
\definecolor{currentstroke}{rgb}{0.000000,0.000000,0.000000}%
\pgfsetstrokecolor{currentstroke}%
\pgfsetdash{}{0pt}%
\pgfpathmoveto{\pgfqpoint{4.098951in}{2.116027in}}%
\pgfpathlineto{\pgfqpoint{4.112274in}{2.119349in}}%
\pgfpathlineto{\pgfqpoint{4.125607in}{2.122837in}}%
\pgfpathlineto{\pgfqpoint{4.138951in}{2.126491in}}%
\pgfpathlineto{\pgfqpoint{4.152304in}{2.130310in}}%
\pgfpathlineto{\pgfqpoint{4.159972in}{2.140718in}}%
\pgfpathlineto{\pgfqpoint{4.167635in}{2.151069in}}%
\pgfpathlineto{\pgfqpoint{4.175293in}{2.161365in}}%
\pgfpathlineto{\pgfqpoint{4.182946in}{2.171603in}}%
\pgfpathlineto{\pgfqpoint{4.169598in}{2.167707in}}%
\pgfpathlineto{\pgfqpoint{4.156260in}{2.163977in}}%
\pgfpathlineto{\pgfqpoint{4.142932in}{2.160412in}}%
\pgfpathlineto{\pgfqpoint{4.129615in}{2.157013in}}%
\pgfpathlineto{\pgfqpoint{4.121957in}{2.146841in}}%
\pgfpathlineto{\pgfqpoint{4.114293in}{2.136619in}}%
\pgfpathlineto{\pgfqpoint{4.106624in}{2.126348in}}%
\pgfpathlineto{\pgfqpoint{4.098951in}{2.116027in}}%
\pgfpathclose%
\pgfusepath{fill}%
\end{pgfscope}%
\begin{pgfscope}%
\pgfpathrectangle{\pgfqpoint{1.254980in}{0.150000in}}{\pgfqpoint{5.490039in}{5.490039in}}%
\pgfusepath{clip}%
\pgfsetbuttcap%
\pgfsetroundjoin%
\definecolor{currentfill}{rgb}{0.208030,0.718701,0.472873}%
\pgfsetfillcolor{currentfill}%
\pgfsetfillopacity{0.700000}%
\pgfsetlinewidth{0.000000pt}%
\definecolor{currentstroke}{rgb}{0.000000,0.000000,0.000000}%
\pgfsetstrokecolor{currentstroke}%
\pgfsetdash{}{0pt}%
\pgfpathmoveto{\pgfqpoint{6.039193in}{3.442496in}}%
\pgfpathlineto{\pgfqpoint{6.053488in}{3.452858in}}%
\pgfpathlineto{\pgfqpoint{6.067803in}{3.463374in}}%
\pgfpathlineto{\pgfqpoint{6.082138in}{3.474041in}}%
\pgfpathlineto{\pgfqpoint{6.088841in}{3.475079in}}%
\pgfpathlineto{\pgfqpoint{6.095543in}{3.476239in}}%
\pgfpathlineto{\pgfqpoint{6.102243in}{3.477528in}}%
\pgfpathlineto{\pgfqpoint{6.108941in}{3.478953in}}%
\pgfpathlineto{\pgfqpoint{6.094646in}{3.468947in}}%
\pgfpathlineto{\pgfqpoint{6.080371in}{3.459093in}}%
\pgfpathlineto{\pgfqpoint{6.066115in}{3.449390in}}%
\pgfpathlineto{\pgfqpoint{6.059386in}{3.447462in}}%
\pgfpathlineto{\pgfqpoint{6.052656in}{3.445676in}}%
\pgfpathlineto{\pgfqpoint{6.045926in}{3.444022in}}%
\pgfpathlineto{\pgfqpoint{6.039193in}{3.442496in}}%
\pgfpathclose%
\pgfusepath{fill}%
\end{pgfscope}%
\begin{pgfscope}%
\pgfpathrectangle{\pgfqpoint{1.254980in}{0.150000in}}{\pgfqpoint{5.490039in}{5.490039in}}%
\pgfusepath{clip}%
\pgfsetbuttcap%
\pgfsetroundjoin%
\definecolor{currentfill}{rgb}{0.218130,0.347432,0.550038}%
\pgfsetfillcolor{currentfill}%
\pgfsetfillopacity{0.700000}%
\pgfsetlinewidth{0.000000pt}%
\definecolor{currentstroke}{rgb}{0.000000,0.000000,0.000000}%
\pgfsetstrokecolor{currentstroke}%
\pgfsetdash{}{0pt}%
\pgfpathmoveto{\pgfqpoint{4.549557in}{2.448449in}}%
\pgfpathlineto{\pgfqpoint{4.563077in}{2.455072in}}%
\pgfpathlineto{\pgfqpoint{4.576611in}{2.461857in}}%
\pgfpathlineto{\pgfqpoint{4.590158in}{2.468803in}}%
\pgfpathlineto{\pgfqpoint{4.603719in}{2.475911in}}%
\pgfpathlineto{\pgfqpoint{4.611227in}{2.484586in}}%
\pgfpathlineto{\pgfqpoint{4.618729in}{2.493181in}}%
\pgfpathlineto{\pgfqpoint{4.626224in}{2.501695in}}%
\pgfpathlineto{\pgfqpoint{4.633714in}{2.510131in}}%
\pgfpathlineto{\pgfqpoint{4.620160in}{2.503120in}}%
\pgfpathlineto{\pgfqpoint{4.606620in}{2.496269in}}%
\pgfpathlineto{\pgfqpoint{4.593093in}{2.489580in}}%
\pgfpathlineto{\pgfqpoint{4.579580in}{2.483052in}}%
\pgfpathlineto{\pgfqpoint{4.572083in}{2.474510in}}%
\pgfpathlineto{\pgfqpoint{4.564580in}{2.465896in}}%
\pgfpathlineto{\pgfqpoint{4.557071in}{2.457209in}}%
\pgfpathlineto{\pgfqpoint{4.549557in}{2.448449in}}%
\pgfpathclose%
\pgfusepath{fill}%
\end{pgfscope}%
\begin{pgfscope}%
\pgfpathrectangle{\pgfqpoint{1.254980in}{0.150000in}}{\pgfqpoint{5.490039in}{5.490039in}}%
\pgfusepath{clip}%
\pgfsetbuttcap%
\pgfsetroundjoin%
\definecolor{currentfill}{rgb}{0.143303,0.669459,0.511215}%
\pgfsetfillcolor{currentfill}%
\pgfsetfillopacity{0.700000}%
\pgfsetlinewidth{0.000000pt}%
\definecolor{currentstroke}{rgb}{0.000000,0.000000,0.000000}%
\pgfsetstrokecolor{currentstroke}%
\pgfsetdash{}{0pt}%
\pgfpathmoveto{\pgfqpoint{5.786861in}{3.297408in}}%
\pgfpathlineto{\pgfqpoint{5.801030in}{3.307742in}}%
\pgfpathlineto{\pgfqpoint{5.815218in}{3.318230in}}%
\pgfpathlineto{\pgfqpoint{5.829424in}{3.328872in}}%
\pgfpathlineto{\pgfqpoint{5.843650in}{3.339668in}}%
\pgfpathlineto{\pgfqpoint{5.850493in}{3.341334in}}%
\pgfpathlineto{\pgfqpoint{5.857331in}{3.343061in}}%
\pgfpathlineto{\pgfqpoint{5.864164in}{3.344855in}}%
\pgfpathlineto{\pgfqpoint{5.870994in}{3.346722in}}%
\pgfpathlineto{\pgfqpoint{5.856801in}{3.336500in}}%
\pgfpathlineto{\pgfqpoint{5.842627in}{3.326431in}}%
\pgfpathlineto{\pgfqpoint{5.828471in}{3.316515in}}%
\pgfpathlineto{\pgfqpoint{5.814334in}{3.306753in}}%
\pgfpathlineto{\pgfqpoint{5.807472in}{3.304303in}}%
\pgfpathlineto{\pgfqpoint{5.800605in}{3.301933in}}%
\pgfpathlineto{\pgfqpoint{5.793735in}{3.299637in}}%
\pgfpathlineto{\pgfqpoint{5.786861in}{3.297408in}}%
\pgfpathclose%
\pgfusepath{fill}%
\end{pgfscope}%
\begin{pgfscope}%
\pgfpathrectangle{\pgfqpoint{1.254980in}{0.150000in}}{\pgfqpoint{5.490039in}{5.490039in}}%
\pgfusepath{clip}%
\pgfsetbuttcap%
\pgfsetroundjoin%
\definecolor{currentfill}{rgb}{0.160665,0.478540,0.558115}%
\pgfsetfillcolor{currentfill}%
\pgfsetfillopacity{0.700000}%
\pgfsetlinewidth{0.000000pt}%
\definecolor{currentstroke}{rgb}{0.000000,0.000000,0.000000}%
\pgfsetstrokecolor{currentstroke}%
\pgfsetdash{}{0pt}%
\pgfpathmoveto{\pgfqpoint{5.000245in}{2.784515in}}%
\pgfpathlineto{\pgfqpoint{5.013999in}{2.793317in}}%
\pgfpathlineto{\pgfqpoint{5.027769in}{2.802277in}}%
\pgfpathlineto{\pgfqpoint{5.041555in}{2.811395in}}%
\pgfpathlineto{\pgfqpoint{5.055356in}{2.820672in}}%
\pgfpathlineto{\pgfqpoint{5.062656in}{2.826627in}}%
\pgfpathlineto{\pgfqpoint{5.069949in}{2.832518in}}%
\pgfpathlineto{\pgfqpoint{5.077235in}{2.838349in}}%
\pgfpathlineto{\pgfqpoint{5.084514in}{2.844123in}}%
\pgfpathlineto{\pgfqpoint{5.070726in}{2.835120in}}%
\pgfpathlineto{\pgfqpoint{5.056954in}{2.826275in}}%
\pgfpathlineto{\pgfqpoint{5.043198in}{2.817589in}}%
\pgfpathlineto{\pgfqpoint{5.029457in}{2.809059in}}%
\pgfpathlineto{\pgfqpoint{5.022164in}{2.803001in}}%
\pgfpathlineto{\pgfqpoint{5.014864in}{2.796893in}}%
\pgfpathlineto{\pgfqpoint{5.007558in}{2.790732in}}%
\pgfpathlineto{\pgfqpoint{5.000245in}{2.784515in}}%
\pgfpathclose%
\pgfusepath{fill}%
\end{pgfscope}%
\begin{pgfscope}%
\pgfpathrectangle{\pgfqpoint{1.254980in}{0.150000in}}{\pgfqpoint{5.490039in}{5.490039in}}%
\pgfusepath{clip}%
\pgfsetbuttcap%
\pgfsetroundjoin%
\definecolor{currentfill}{rgb}{0.162016,0.687316,0.499129}%
\pgfsetfillcolor{currentfill}%
\pgfsetfillopacity{0.700000}%
\pgfsetlinewidth{0.000000pt}%
\definecolor{currentstroke}{rgb}{0.000000,0.000000,0.000000}%
\pgfsetstrokecolor{currentstroke}%
\pgfsetdash{}{0pt}%
\pgfpathmoveto{\pgfqpoint{5.870994in}{3.346722in}}%
\pgfpathlineto{\pgfqpoint{5.885206in}{3.357097in}}%
\pgfpathlineto{\pgfqpoint{5.899438in}{3.367625in}}%
\pgfpathlineto{\pgfqpoint{5.913689in}{3.378307in}}%
\pgfpathlineto{\pgfqpoint{5.927959in}{3.389143in}}%
\pgfpathlineto{\pgfqpoint{5.934751in}{3.390497in}}%
\pgfpathlineto{\pgfqpoint{5.941539in}{3.391931in}}%
\pgfpathlineto{\pgfqpoint{5.948324in}{3.393450in}}%
\pgfpathlineto{\pgfqpoint{5.955106in}{3.395062in}}%
\pgfpathlineto{\pgfqpoint{5.940870in}{3.384830in}}%
\pgfpathlineto{\pgfqpoint{5.926654in}{3.374751in}}%
\pgfpathlineto{\pgfqpoint{5.912457in}{3.364824in}}%
\pgfpathlineto{\pgfqpoint{5.898279in}{3.355050in}}%
\pgfpathlineto{\pgfqpoint{5.891462in}{3.352826in}}%
\pgfpathlineto{\pgfqpoint{5.884643in}{3.350701in}}%
\pgfpathlineto{\pgfqpoint{5.877820in}{3.348669in}}%
\pgfpathlineto{\pgfqpoint{5.870994in}{3.346722in}}%
\pgfpathclose%
\pgfusepath{fill}%
\end{pgfscope}%
\begin{pgfscope}%
\pgfpathrectangle{\pgfqpoint{1.254980in}{0.150000in}}{\pgfqpoint{5.490039in}{5.490039in}}%
\pgfusepath{clip}%
\pgfsetbuttcap%
\pgfsetroundjoin%
\definecolor{currentfill}{rgb}{0.273809,0.031497,0.358853}%
\pgfsetfillcolor{currentfill}%
\pgfsetfillopacity{0.700000}%
\pgfsetlinewidth{0.000000pt}%
\definecolor{currentstroke}{rgb}{0.000000,0.000000,0.000000}%
\pgfsetstrokecolor{currentstroke}%
\pgfsetdash{}{0pt}%
\pgfpathmoveto{\pgfqpoint{3.510889in}{1.823873in}}%
\pgfpathlineto{\pgfqpoint{3.524067in}{1.821070in}}%
\pgfpathlineto{\pgfqpoint{3.537249in}{1.818445in}}%
\pgfpathlineto{\pgfqpoint{3.550436in}{1.815998in}}%
\pgfpathlineto{\pgfqpoint{3.563628in}{1.813729in}}%
\pgfpathlineto{\pgfqpoint{3.571494in}{1.823265in}}%
\pgfpathlineto{\pgfqpoint{3.579355in}{1.832835in}}%
\pgfpathlineto{\pgfqpoint{3.587210in}{1.842437in}}%
\pgfpathlineto{\pgfqpoint{3.595060in}{1.852067in}}%
\pgfpathlineto{\pgfqpoint{3.581880in}{1.854065in}}%
\pgfpathlineto{\pgfqpoint{3.568706in}{1.856240in}}%
\pgfpathlineto{\pgfqpoint{3.555536in}{1.858593in}}%
\pgfpathlineto{\pgfqpoint{3.542372in}{1.861126in}}%
\pgfpathlineto{\pgfqpoint{3.534510in}{1.851756in}}%
\pgfpathlineto{\pgfqpoint{3.526642in}{1.842423in}}%
\pgfpathlineto{\pgfqpoint{3.518769in}{1.833128in}}%
\pgfpathlineto{\pgfqpoint{3.510889in}{1.823873in}}%
\pgfpathclose%
\pgfusepath{fill}%
\end{pgfscope}%
\begin{pgfscope}%
\pgfpathrectangle{\pgfqpoint{1.254980in}{0.150000in}}{\pgfqpoint{5.490039in}{5.490039in}}%
\pgfusepath{clip}%
\pgfsetbuttcap%
\pgfsetroundjoin%
\definecolor{currentfill}{rgb}{0.185783,0.704891,0.485273}%
\pgfsetfillcolor{currentfill}%
\pgfsetfillopacity{0.700000}%
\pgfsetlinewidth{0.000000pt}%
\definecolor{currentstroke}{rgb}{0.000000,0.000000,0.000000}%
\pgfsetstrokecolor{currentstroke}%
\pgfsetdash{}{0pt}%
\pgfpathmoveto{\pgfqpoint{5.955106in}{3.395062in}}%
\pgfpathlineto{\pgfqpoint{5.969360in}{3.405447in}}%
\pgfpathlineto{\pgfqpoint{5.983634in}{3.415984in}}%
\pgfpathlineto{\pgfqpoint{5.997927in}{3.426675in}}%
\pgfpathlineto{\pgfqpoint{6.012241in}{3.437519in}}%
\pgfpathlineto{\pgfqpoint{6.018983in}{3.438607in}}%
\pgfpathlineto{\pgfqpoint{6.025722in}{3.439795in}}%
\pgfpathlineto{\pgfqpoint{6.032459in}{3.441089in}}%
\pgfpathlineto{\pgfqpoint{6.039193in}{3.442496in}}%
\pgfpathlineto{\pgfqpoint{6.024917in}{3.432285in}}%
\pgfpathlineto{\pgfqpoint{6.010661in}{3.422227in}}%
\pgfpathlineto{\pgfqpoint{5.996424in}{3.412321in}}%
\pgfpathlineto{\pgfqpoint{5.982206in}{3.402567in}}%
\pgfpathlineto{\pgfqpoint{5.975434in}{3.400519in}}%
\pgfpathlineto{\pgfqpoint{5.968660in}{3.398590in}}%
\pgfpathlineto{\pgfqpoint{5.961884in}{3.396773in}}%
\pgfpathlineto{\pgfqpoint{5.955106in}{3.395062in}}%
\pgfpathclose%
\pgfusepath{fill}%
\end{pgfscope}%
\begin{pgfscope}%
\pgfpathrectangle{\pgfqpoint{1.254980in}{0.150000in}}{\pgfqpoint{5.490039in}{5.490039in}}%
\pgfusepath{clip}%
\pgfsetbuttcap%
\pgfsetroundjoin%
\definecolor{currentfill}{rgb}{0.267968,0.223549,0.512008}%
\pgfsetfillcolor{currentfill}%
\pgfsetfillopacity{0.700000}%
\pgfsetlinewidth{0.000000pt}%
\definecolor{currentstroke}{rgb}{0.000000,0.000000,0.000000}%
\pgfsetstrokecolor{currentstroke}%
\pgfsetdash{}{0pt}%
\pgfpathmoveto{\pgfqpoint{4.182946in}{2.171603in}}%
\pgfpathlineto{\pgfqpoint{4.196305in}{2.175663in}}%
\pgfpathlineto{\pgfqpoint{4.209674in}{2.179889in}}%
\pgfpathlineto{\pgfqpoint{4.223054in}{2.184278in}}%
\pgfpathlineto{\pgfqpoint{4.236446in}{2.188832in}}%
\pgfpathlineto{\pgfqpoint{4.244088in}{2.199072in}}%
\pgfpathlineto{\pgfqpoint{4.251725in}{2.209248in}}%
\pgfpathlineto{\pgfqpoint{4.259357in}{2.219359in}}%
\pgfpathlineto{\pgfqpoint{4.266984in}{2.229405in}}%
\pgfpathlineto{\pgfqpoint{4.253598in}{2.224802in}}%
\pgfpathlineto{\pgfqpoint{4.240223in}{2.220364in}}%
\pgfpathlineto{\pgfqpoint{4.226859in}{2.216091in}}%
\pgfpathlineto{\pgfqpoint{4.213506in}{2.211982in}}%
\pgfpathlineto{\pgfqpoint{4.205874in}{2.201973in}}%
\pgfpathlineto{\pgfqpoint{4.198236in}{2.191907in}}%
\pgfpathlineto{\pgfqpoint{4.190594in}{2.181784in}}%
\pgfpathlineto{\pgfqpoint{4.182946in}{2.171603in}}%
\pgfpathclose%
\pgfusepath{fill}%
\end{pgfscope}%
\begin{pgfscope}%
\pgfpathrectangle{\pgfqpoint{1.254980in}{0.150000in}}{\pgfqpoint{5.490039in}{5.490039in}}%
\pgfusepath{clip}%
\pgfsetbuttcap%
\pgfsetroundjoin%
\definecolor{currentfill}{rgb}{0.271305,0.019942,0.347269}%
\pgfsetfillcolor{currentfill}%
\pgfsetfillopacity{0.700000}%
\pgfsetlinewidth{0.000000pt}%
\definecolor{currentstroke}{rgb}{0.000000,0.000000,0.000000}%
\pgfsetstrokecolor{currentstroke}%
\pgfsetdash{}{0pt}%
\pgfpathmoveto{\pgfqpoint{3.289425in}{1.807006in}}%
\pgfpathlineto{\pgfqpoint{3.302594in}{1.801372in}}%
\pgfpathlineto{\pgfqpoint{3.315765in}{1.795925in}}%
\pgfpathlineto{\pgfqpoint{3.328938in}{1.790666in}}%
\pgfpathlineto{\pgfqpoint{3.342114in}{1.785593in}}%
\pgfpathlineto{\pgfqpoint{3.350076in}{1.793748in}}%
\pgfpathlineto{\pgfqpoint{3.358031in}{1.801980in}}%
\pgfpathlineto{\pgfqpoint{3.365979in}{1.810286in}}%
\pgfpathlineto{\pgfqpoint{3.373921in}{1.818664in}}%
\pgfpathlineto{\pgfqpoint{3.360762in}{1.823409in}}%
\pgfpathlineto{\pgfqpoint{3.347606in}{1.828340in}}%
\pgfpathlineto{\pgfqpoint{3.334453in}{1.833459in}}%
\pgfpathlineto{\pgfqpoint{3.321303in}{1.838766in}}%
\pgfpathlineto{\pgfqpoint{3.313344in}{1.830706in}}%
\pgfpathlineto{\pgfqpoint{3.305379in}{1.822724in}}%
\pgfpathlineto{\pgfqpoint{3.297406in}{1.814823in}}%
\pgfpathlineto{\pgfqpoint{3.289425in}{1.807006in}}%
\pgfpathclose%
\pgfusepath{fill}%
\end{pgfscope}%
\begin{pgfscope}%
\pgfpathrectangle{\pgfqpoint{1.254980in}{0.150000in}}{\pgfqpoint{5.490039in}{5.490039in}}%
\pgfusepath{clip}%
\pgfsetbuttcap%
\pgfsetroundjoin%
\definecolor{currentfill}{rgb}{0.280894,0.078907,0.402329}%
\pgfsetfillcolor{currentfill}%
\pgfsetfillopacity{0.700000}%
\pgfsetlinewidth{0.000000pt}%
\definecolor{currentstroke}{rgb}{0.000000,0.000000,0.000000}%
\pgfsetstrokecolor{currentstroke}%
\pgfsetdash{}{0pt}%
\pgfpathmoveto{\pgfqpoint{2.961215in}{1.916501in}}%
\pgfpathlineto{\pgfqpoint{2.974428in}{1.906140in}}%
\pgfpathlineto{\pgfqpoint{2.987638in}{1.895990in}}%
\pgfpathlineto{\pgfqpoint{3.000847in}{1.886048in}}%
\pgfpathlineto{\pgfqpoint{3.014054in}{1.876315in}}%
\pgfpathlineto{\pgfqpoint{3.022192in}{1.881755in}}%
\pgfpathlineto{\pgfqpoint{3.030321in}{1.887334in}}%
\pgfpathlineto{\pgfqpoint{3.038440in}{1.893050in}}%
\pgfpathlineto{\pgfqpoint{3.046549in}{1.898898in}}%
\pgfpathlineto{\pgfqpoint{3.033368in}{1.908243in}}%
\pgfpathlineto{\pgfqpoint{3.020186in}{1.917796in}}%
\pgfpathlineto{\pgfqpoint{3.007002in}{1.927558in}}%
\pgfpathlineto{\pgfqpoint{2.993816in}{1.937531in}}%
\pgfpathlineto{\pgfqpoint{2.985681in}{1.932060in}}%
\pgfpathlineto{\pgfqpoint{2.977536in}{1.926729in}}%
\pgfpathlineto{\pgfqpoint{2.969381in}{1.921542in}}%
\pgfpathlineto{\pgfqpoint{2.961215in}{1.916501in}}%
\pgfpathclose%
\pgfusepath{fill}%
\end{pgfscope}%
\begin{pgfscope}%
\pgfpathrectangle{\pgfqpoint{1.254980in}{0.150000in}}{\pgfqpoint{5.490039in}{5.490039in}}%
\pgfusepath{clip}%
\pgfsetbuttcap%
\pgfsetroundjoin%
\definecolor{currentfill}{rgb}{0.273809,0.031497,0.358853}%
\pgfsetfillcolor{currentfill}%
\pgfsetfillopacity{0.700000}%
\pgfsetlinewidth{0.000000pt}%
\definecolor{currentstroke}{rgb}{0.000000,0.000000,0.000000}%
\pgfsetstrokecolor{currentstroke}%
\pgfsetdash{}{0pt}%
\pgfpathmoveto{\pgfqpoint{3.151976in}{1.831460in}}%
\pgfpathlineto{\pgfqpoint{3.165155in}{1.823929in}}%
\pgfpathlineto{\pgfqpoint{3.178334in}{1.816593in}}%
\pgfpathlineto{\pgfqpoint{3.191513in}{1.809453in}}%
\pgfpathlineto{\pgfqpoint{3.204694in}{1.802507in}}%
\pgfpathlineto{\pgfqpoint{3.212725in}{1.809589in}}%
\pgfpathlineto{\pgfqpoint{3.220748in}{1.816775in}}%
\pgfpathlineto{\pgfqpoint{3.228763in}{1.824061in}}%
\pgfpathlineto{\pgfqpoint{3.236770in}{1.831445in}}%
\pgfpathlineto{\pgfqpoint{3.223610in}{1.838035in}}%
\pgfpathlineto{\pgfqpoint{3.210452in}{1.844818in}}%
\pgfpathlineto{\pgfqpoint{3.197294in}{1.851796in}}%
\pgfpathlineto{\pgfqpoint{3.184137in}{1.858970in}}%
\pgfpathlineto{\pgfqpoint{3.176110in}{1.851933in}}%
\pgfpathlineto{\pgfqpoint{3.168074in}{1.844999in}}%
\pgfpathlineto{\pgfqpoint{3.160029in}{1.838174in}}%
\pgfpathlineto{\pgfqpoint{3.151976in}{1.831460in}}%
\pgfpathclose%
\pgfusepath{fill}%
\end{pgfscope}%
\begin{pgfscope}%
\pgfpathrectangle{\pgfqpoint{1.254980in}{0.150000in}}{\pgfqpoint{5.490039in}{5.490039in}}%
\pgfusepath{clip}%
\pgfsetbuttcap%
\pgfsetroundjoin%
\definecolor{currentfill}{rgb}{0.185556,0.418570,0.556753}%
\pgfsetfillcolor{currentfill}%
\pgfsetfillopacity{0.700000}%
\pgfsetlinewidth{0.000000pt}%
\definecolor{currentstroke}{rgb}{0.000000,0.000000,0.000000}%
\pgfsetstrokecolor{currentstroke}%
\pgfsetdash{}{0pt}%
\pgfpathmoveto{\pgfqpoint{2.320392in}{2.694472in}}%
\pgfpathlineto{\pgfqpoint{2.333971in}{2.671860in}}%
\pgfpathlineto{\pgfqpoint{2.347535in}{2.649561in}}%
\pgfpathlineto{\pgfqpoint{2.361085in}{2.627571in}}%
\pgfpathlineto{\pgfqpoint{2.374622in}{2.605887in}}%
\pgfpathlineto{\pgfqpoint{2.383181in}{2.606397in}}%
\pgfpathlineto{\pgfqpoint{2.391725in}{2.607136in}}%
\pgfpathlineto{\pgfqpoint{2.400252in}{2.608101in}}%
\pgfpathlineto{\pgfqpoint{2.408763in}{2.609288in}}%
\pgfpathlineto{\pgfqpoint{2.395270in}{2.630554in}}%
\pgfpathlineto{\pgfqpoint{2.381764in}{2.652125in}}%
\pgfpathlineto{\pgfqpoint{2.368245in}{2.674005in}}%
\pgfpathlineto{\pgfqpoint{2.354712in}{2.696196in}}%
\pgfpathlineto{\pgfqpoint{2.346157in}{2.695416in}}%
\pgfpathlineto{\pgfqpoint{2.337585in}{2.694867in}}%
\pgfpathlineto{\pgfqpoint{2.328997in}{2.694550in}}%
\pgfpathlineto{\pgfqpoint{2.320392in}{2.694472in}}%
\pgfpathclose%
\pgfusepath{fill}%
\end{pgfscope}%
\begin{pgfscope}%
\pgfpathrectangle{\pgfqpoint{1.254980in}{0.150000in}}{\pgfqpoint{5.490039in}{5.490039in}}%
\pgfusepath{clip}%
\pgfsetbuttcap%
\pgfsetroundjoin%
\definecolor{currentfill}{rgb}{0.204903,0.375746,0.553533}%
\pgfsetfillcolor{currentfill}%
\pgfsetfillopacity{0.700000}%
\pgfsetlinewidth{0.000000pt}%
\definecolor{currentstroke}{rgb}{0.000000,0.000000,0.000000}%
\pgfsetstrokecolor{currentstroke}%
\pgfsetdash{}{0pt}%
\pgfpathmoveto{\pgfqpoint{4.633714in}{2.510131in}}%
\pgfpathlineto{\pgfqpoint{4.647281in}{2.517303in}}%
\pgfpathlineto{\pgfqpoint{4.660862in}{2.524635in}}%
\pgfpathlineto{\pgfqpoint{4.674457in}{2.532129in}}%
\pgfpathlineto{\pgfqpoint{4.688066in}{2.539782in}}%
\pgfpathlineto{\pgfqpoint{4.695542in}{2.548027in}}%
\pgfpathlineto{\pgfqpoint{4.703011in}{2.556188in}}%
\pgfpathlineto{\pgfqpoint{4.710474in}{2.564269in}}%
\pgfpathlineto{\pgfqpoint{4.717931in}{2.572271in}}%
\pgfpathlineto{\pgfqpoint{4.704330in}{2.564743in}}%
\pgfpathlineto{\pgfqpoint{4.690743in}{2.557375in}}%
\pgfpathlineto{\pgfqpoint{4.677170in}{2.550168in}}%
\pgfpathlineto{\pgfqpoint{4.663610in}{2.543121in}}%
\pgfpathlineto{\pgfqpoint{4.656145in}{2.534983in}}%
\pgfpathlineto{\pgfqpoint{4.648674in}{2.526773in}}%
\pgfpathlineto{\pgfqpoint{4.641197in}{2.518490in}}%
\pgfpathlineto{\pgfqpoint{4.633714in}{2.510131in}}%
\pgfpathclose%
\pgfusepath{fill}%
\end{pgfscope}%
\begin{pgfscope}%
\pgfpathrectangle{\pgfqpoint{1.254980in}{0.150000in}}{\pgfqpoint{5.490039in}{5.490039in}}%
\pgfusepath{clip}%
\pgfsetbuttcap%
\pgfsetroundjoin%
\definecolor{currentfill}{rgb}{0.150476,0.504369,0.557430}%
\pgfsetfillcolor{currentfill}%
\pgfsetfillopacity{0.700000}%
\pgfsetlinewidth{0.000000pt}%
\definecolor{currentstroke}{rgb}{0.000000,0.000000,0.000000}%
\pgfsetstrokecolor{currentstroke}%
\pgfsetdash{}{0pt}%
\pgfpathmoveto{\pgfqpoint{5.084514in}{2.844123in}}%
\pgfpathlineto{\pgfqpoint{5.098318in}{2.853283in}}%
\pgfpathlineto{\pgfqpoint{5.112139in}{2.862602in}}%
\pgfpathlineto{\pgfqpoint{5.125976in}{2.872078in}}%
\pgfpathlineto{\pgfqpoint{5.139829in}{2.881712in}}%
\pgfpathlineto{\pgfqpoint{5.147087in}{2.887140in}}%
\pgfpathlineto{\pgfqpoint{5.154337in}{2.892510in}}%
\pgfpathlineto{\pgfqpoint{5.161581in}{2.897826in}}%
\pgfpathlineto{\pgfqpoint{5.168817in}{2.903093in}}%
\pgfpathlineto{\pgfqpoint{5.154979in}{2.893763in}}%
\pgfpathlineto{\pgfqpoint{5.141157in}{2.884590in}}%
\pgfpathlineto{\pgfqpoint{5.127351in}{2.875576in}}%
\pgfpathlineto{\pgfqpoint{5.113562in}{2.866718in}}%
\pgfpathlineto{\pgfqpoint{5.106310in}{2.861137in}}%
\pgfpathlineto{\pgfqpoint{5.099052in}{2.855514in}}%
\pgfpathlineto{\pgfqpoint{5.091786in}{2.849843in}}%
\pgfpathlineto{\pgfqpoint{5.084514in}{2.844123in}}%
\pgfpathclose%
\pgfusepath{fill}%
\end{pgfscope}%
\begin{pgfscope}%
\pgfpathrectangle{\pgfqpoint{1.254980in}{0.150000in}}{\pgfqpoint{5.490039in}{5.490039in}}%
\pgfusepath{clip}%
\pgfsetbuttcap%
\pgfsetroundjoin%
\definecolor{currentfill}{rgb}{0.257322,0.256130,0.526563}%
\pgfsetfillcolor{currentfill}%
\pgfsetfillopacity{0.700000}%
\pgfsetlinewidth{0.000000pt}%
\definecolor{currentstroke}{rgb}{0.000000,0.000000,0.000000}%
\pgfsetstrokecolor{currentstroke}%
\pgfsetdash{}{0pt}%
\pgfpathmoveto{\pgfqpoint{4.266984in}{2.229405in}}%
\pgfpathlineto{\pgfqpoint{4.280381in}{2.234171in}}%
\pgfpathlineto{\pgfqpoint{4.293790in}{2.239101in}}%
\pgfpathlineto{\pgfqpoint{4.307210in}{2.244195in}}%
\pgfpathlineto{\pgfqpoint{4.320642in}{2.249452in}}%
\pgfpathlineto{\pgfqpoint{4.328258in}{2.259464in}}%
\pgfpathlineto{\pgfqpoint{4.335869in}{2.269404in}}%
\pgfpathlineto{\pgfqpoint{4.343475in}{2.279273in}}%
\pgfpathlineto{\pgfqpoint{4.351075in}{2.289071in}}%
\pgfpathlineto{\pgfqpoint{4.337649in}{2.283794in}}%
\pgfpathlineto{\pgfqpoint{4.324234in}{2.278681in}}%
\pgfpathlineto{\pgfqpoint{4.310831in}{2.273731in}}%
\pgfpathlineto{\pgfqpoint{4.297439in}{2.268945in}}%
\pgfpathlineto{\pgfqpoint{4.289834in}{2.259156in}}%
\pgfpathlineto{\pgfqpoint{4.282222in}{2.249303in}}%
\pgfpathlineto{\pgfqpoint{4.274606in}{2.239386in}}%
\pgfpathlineto{\pgfqpoint{4.266984in}{2.229405in}}%
\pgfpathclose%
\pgfusepath{fill}%
\end{pgfscope}%
\begin{pgfscope}%
\pgfpathrectangle{\pgfqpoint{1.254980in}{0.150000in}}{\pgfqpoint{5.490039in}{5.490039in}}%
\pgfusepath{clip}%
\pgfsetbuttcap%
\pgfsetroundjoin%
\definecolor{currentfill}{rgb}{0.272594,0.025563,0.353093}%
\pgfsetfillcolor{currentfill}%
\pgfsetfillopacity{0.700000}%
\pgfsetlinewidth{0.000000pt}%
\definecolor{currentstroke}{rgb}{0.000000,0.000000,0.000000}%
\pgfsetstrokecolor{currentstroke}%
\pgfsetdash{}{0pt}%
\pgfpathmoveto{\pgfqpoint{3.426586in}{1.801530in}}%
\pgfpathlineto{\pgfqpoint{3.439762in}{1.797705in}}%
\pgfpathlineto{\pgfqpoint{3.452940in}{1.794061in}}%
\pgfpathlineto{\pgfqpoint{3.466123in}{1.790598in}}%
\pgfpathlineto{\pgfqpoint{3.479310in}{1.787316in}}%
\pgfpathlineto{\pgfqpoint{3.487214in}{1.796381in}}%
\pgfpathlineto{\pgfqpoint{3.495112in}{1.805497in}}%
\pgfpathlineto{\pgfqpoint{3.503004in}{1.814662in}}%
\pgfpathlineto{\pgfqpoint{3.510889in}{1.823873in}}%
\pgfpathlineto{\pgfqpoint{3.497716in}{1.826857in}}%
\pgfpathlineto{\pgfqpoint{3.484548in}{1.830020in}}%
\pgfpathlineto{\pgfqpoint{3.471383in}{1.833364in}}%
\pgfpathlineto{\pgfqpoint{3.458223in}{1.836890in}}%
\pgfpathlineto{\pgfqpoint{3.450323in}{1.827968in}}%
\pgfpathlineto{\pgfqpoint{3.442417in}{1.819099in}}%
\pgfpathlineto{\pgfqpoint{3.434505in}{1.810286in}}%
\pgfpathlineto{\pgfqpoint{3.426586in}{1.801530in}}%
\pgfpathclose%
\pgfusepath{fill}%
\end{pgfscope}%
\begin{pgfscope}%
\pgfpathrectangle{\pgfqpoint{1.254980in}{0.150000in}}{\pgfqpoint{5.490039in}{5.490039in}}%
\pgfusepath{clip}%
\pgfsetbuttcap%
\pgfsetroundjoin%
\definecolor{currentfill}{rgb}{0.141935,0.526453,0.555991}%
\pgfsetfillcolor{currentfill}%
\pgfsetfillopacity{0.700000}%
\pgfsetlinewidth{0.000000pt}%
\definecolor{currentstroke}{rgb}{0.000000,0.000000,0.000000}%
\pgfsetstrokecolor{currentstroke}%
\pgfsetdash{}{0pt}%
\pgfpathmoveto{\pgfqpoint{5.168817in}{2.903093in}}%
\pgfpathlineto{\pgfqpoint{5.182672in}{2.912580in}}%
\pgfpathlineto{\pgfqpoint{5.196543in}{2.922224in}}%
\pgfpathlineto{\pgfqpoint{5.210431in}{2.932027in}}%
\pgfpathlineto{\pgfqpoint{5.224336in}{2.941987in}}%
\pgfpathlineto{\pgfqpoint{5.231550in}{2.946883in}}%
\pgfpathlineto{\pgfqpoint{5.238756in}{2.951729in}}%
\pgfpathlineto{\pgfqpoint{5.245955in}{2.956530in}}%
\pgfpathlineto{\pgfqpoint{5.253147in}{2.961288in}}%
\pgfpathlineto{\pgfqpoint{5.239259in}{2.951663in}}%
\pgfpathlineto{\pgfqpoint{5.225387in}{2.942195in}}%
\pgfpathlineto{\pgfqpoint{5.211532in}{2.932884in}}%
\pgfpathlineto{\pgfqpoint{5.197694in}{2.923730in}}%
\pgfpathlineto{\pgfqpoint{5.190485in}{2.918627in}}%
\pgfpathlineto{\pgfqpoint{5.183269in}{2.913489in}}%
\pgfpathlineto{\pgfqpoint{5.176046in}{2.908312in}}%
\pgfpathlineto{\pgfqpoint{5.168817in}{2.903093in}}%
\pgfpathclose%
\pgfusepath{fill}%
\end{pgfscope}%
\begin{pgfscope}%
\pgfpathrectangle{\pgfqpoint{1.254980in}{0.150000in}}{\pgfqpoint{5.490039in}{5.490039in}}%
\pgfusepath{clip}%
\pgfsetbuttcap%
\pgfsetroundjoin%
\definecolor{currentfill}{rgb}{0.278791,0.062145,0.386592}%
\pgfsetfillcolor{currentfill}%
\pgfsetfillopacity{0.700000}%
\pgfsetlinewidth{0.000000pt}%
\definecolor{currentstroke}{rgb}{0.000000,0.000000,0.000000}%
\pgfsetstrokecolor{currentstroke}%
\pgfsetdash{}{0pt}%
\pgfpathmoveto{\pgfqpoint{3.014054in}{1.876315in}}%
\pgfpathlineto{\pgfqpoint{3.027260in}{1.866788in}}%
\pgfpathlineto{\pgfqpoint{3.040464in}{1.857467in}}%
\pgfpathlineto{\pgfqpoint{3.053668in}{1.848349in}}%
\pgfpathlineto{\pgfqpoint{3.066871in}{1.839435in}}%
\pgfpathlineto{\pgfqpoint{3.074983in}{1.845273in}}%
\pgfpathlineto{\pgfqpoint{3.083087in}{1.851243in}}%
\pgfpathlineto{\pgfqpoint{3.091181in}{1.857342in}}%
\pgfpathlineto{\pgfqpoint{3.099265in}{1.863566in}}%
\pgfpathlineto{\pgfqpoint{3.086087in}{1.872094in}}%
\pgfpathlineto{\pgfqpoint{3.072909in}{1.880824in}}%
\pgfpathlineto{\pgfqpoint{3.059729in}{1.889758in}}%
\pgfpathlineto{\pgfqpoint{3.046549in}{1.898898in}}%
\pgfpathlineto{\pgfqpoint{3.038440in}{1.893050in}}%
\pgfpathlineto{\pgfqpoint{3.030321in}{1.887334in}}%
\pgfpathlineto{\pgfqpoint{3.022192in}{1.881755in}}%
\pgfpathlineto{\pgfqpoint{3.014054in}{1.876315in}}%
\pgfpathclose%
\pgfusepath{fill}%
\end{pgfscope}%
\begin{pgfscope}%
\pgfpathrectangle{\pgfqpoint{1.254980in}{0.150000in}}{\pgfqpoint{5.490039in}{5.490039in}}%
\pgfusepath{clip}%
\pgfsetbuttcap%
\pgfsetroundjoin%
\definecolor{currentfill}{rgb}{0.192357,0.403199,0.555836}%
\pgfsetfillcolor{currentfill}%
\pgfsetfillopacity{0.700000}%
\pgfsetlinewidth{0.000000pt}%
\definecolor{currentstroke}{rgb}{0.000000,0.000000,0.000000}%
\pgfsetstrokecolor{currentstroke}%
\pgfsetdash{}{0pt}%
\pgfpathmoveto{\pgfqpoint{4.717931in}{2.572271in}}%
\pgfpathlineto{\pgfqpoint{4.731547in}{2.579959in}}%
\pgfpathlineto{\pgfqpoint{4.745176in}{2.587808in}}%
\pgfpathlineto{\pgfqpoint{4.758821in}{2.595816in}}%
\pgfpathlineto{\pgfqpoint{4.772479in}{2.603985in}}%
\pgfpathlineto{\pgfqpoint{4.779921in}{2.611765in}}%
\pgfpathlineto{\pgfqpoint{4.787357in}{2.619463in}}%
\pgfpathlineto{\pgfqpoint{4.794785in}{2.627080in}}%
\pgfpathlineto{\pgfqpoint{4.802208in}{2.634618in}}%
\pgfpathlineto{\pgfqpoint{4.788558in}{2.626605in}}%
\pgfpathlineto{\pgfqpoint{4.774922in}{2.618752in}}%
\pgfpathlineto{\pgfqpoint{4.761301in}{2.611058in}}%
\pgfpathlineto{\pgfqpoint{4.747695in}{2.603525in}}%
\pgfpathlineto{\pgfqpoint{4.740263in}{2.595820in}}%
\pgfpathlineto{\pgfqpoint{4.732826in}{2.588045in}}%
\pgfpathlineto{\pgfqpoint{4.725382in}{2.580196in}}%
\pgfpathlineto{\pgfqpoint{4.717931in}{2.572271in}}%
\pgfpathclose%
\pgfusepath{fill}%
\end{pgfscope}%
\begin{pgfscope}%
\pgfpathrectangle{\pgfqpoint{1.254980in}{0.150000in}}{\pgfqpoint{5.490039in}{5.490039in}}%
\pgfusepath{clip}%
\pgfsetbuttcap%
\pgfsetroundjoin%
\definecolor{currentfill}{rgb}{0.266580,0.228262,0.514349}%
\pgfsetfillcolor{currentfill}%
\pgfsetfillopacity{0.700000}%
\pgfsetlinewidth{0.000000pt}%
\definecolor{currentstroke}{rgb}{0.000000,0.000000,0.000000}%
\pgfsetstrokecolor{currentstroke}%
\pgfsetdash{}{0pt}%
\pgfpathmoveto{\pgfqpoint{2.609425in}{2.223759in}}%
\pgfpathlineto{\pgfqpoint{2.622786in}{2.207450in}}%
\pgfpathlineto{\pgfqpoint{2.636138in}{2.191392in}}%
\pgfpathlineto{\pgfqpoint{2.649483in}{2.175584in}}%
\pgfpathlineto{\pgfqpoint{2.662821in}{2.160022in}}%
\pgfpathlineto{\pgfqpoint{2.671196in}{2.162267in}}%
\pgfpathlineto{\pgfqpoint{2.679558in}{2.164712in}}%
\pgfpathlineto{\pgfqpoint{2.687906in}{2.167353in}}%
\pgfpathlineto{\pgfqpoint{2.696241in}{2.170185in}}%
\pgfpathlineto{\pgfqpoint{2.682940in}{2.185318in}}%
\pgfpathlineto{\pgfqpoint{2.669631in}{2.200697in}}%
\pgfpathlineto{\pgfqpoint{2.656316in}{2.216325in}}%
\pgfpathlineto{\pgfqpoint{2.642994in}{2.232204in}}%
\pgfpathlineto{\pgfqpoint{2.634623in}{2.229790in}}%
\pgfpathlineto{\pgfqpoint{2.626238in}{2.227575in}}%
\pgfpathlineto{\pgfqpoint{2.617839in}{2.225564in}}%
\pgfpathlineto{\pgfqpoint{2.609425in}{2.223759in}}%
\pgfpathclose%
\pgfusepath{fill}%
\end{pgfscope}%
\begin{pgfscope}%
\pgfpathrectangle{\pgfqpoint{1.254980in}{0.150000in}}{\pgfqpoint{5.490039in}{5.490039in}}%
\pgfusepath{clip}%
\pgfsetbuttcap%
\pgfsetroundjoin%
\definecolor{currentfill}{rgb}{0.274128,0.199721,0.498911}%
\pgfsetfillcolor{currentfill}%
\pgfsetfillopacity{0.700000}%
\pgfsetlinewidth{0.000000pt}%
\definecolor{currentstroke}{rgb}{0.000000,0.000000,0.000000}%
\pgfsetstrokecolor{currentstroke}%
\pgfsetdash{}{0pt}%
\pgfpathmoveto{\pgfqpoint{2.662821in}{2.160022in}}%
\pgfpathlineto{\pgfqpoint{2.676152in}{2.144706in}}%
\pgfpathlineto{\pgfqpoint{2.689477in}{2.129632in}}%
\pgfpathlineto{\pgfqpoint{2.702795in}{2.114801in}}%
\pgfpathlineto{\pgfqpoint{2.716107in}{2.100208in}}%
\pgfpathlineto{\pgfqpoint{2.724446in}{2.102892in}}%
\pgfpathlineto{\pgfqpoint{2.732772in}{2.105767in}}%
\pgfpathlineto{\pgfqpoint{2.741085in}{2.108831in}}%
\pgfpathlineto{\pgfqpoint{2.749385in}{2.112079in}}%
\pgfpathlineto{\pgfqpoint{2.736108in}{2.126245in}}%
\pgfpathlineto{\pgfqpoint{2.722825in}{2.140650in}}%
\pgfpathlineto{\pgfqpoint{2.709536in}{2.155296in}}%
\pgfpathlineto{\pgfqpoint{2.696241in}{2.170185in}}%
\pgfpathlineto{\pgfqpoint{2.687906in}{2.167353in}}%
\pgfpathlineto{\pgfqpoint{2.679558in}{2.164712in}}%
\pgfpathlineto{\pgfqpoint{2.671196in}{2.162267in}}%
\pgfpathlineto{\pgfqpoint{2.662821in}{2.160022in}}%
\pgfpathclose%
\pgfusepath{fill}%
\end{pgfscope}%
\begin{pgfscope}%
\pgfpathrectangle{\pgfqpoint{1.254980in}{0.150000in}}{\pgfqpoint{5.490039in}{5.490039in}}%
\pgfusepath{clip}%
\pgfsetbuttcap%
\pgfsetroundjoin%
\definecolor{currentfill}{rgb}{0.246811,0.283237,0.535941}%
\pgfsetfillcolor{currentfill}%
\pgfsetfillopacity{0.700000}%
\pgfsetlinewidth{0.000000pt}%
\definecolor{currentstroke}{rgb}{0.000000,0.000000,0.000000}%
\pgfsetstrokecolor{currentstroke}%
\pgfsetdash{}{0pt}%
\pgfpathmoveto{\pgfqpoint{4.351075in}{2.289071in}}%
\pgfpathlineto{\pgfqpoint{4.364513in}{2.294511in}}%
\pgfpathlineto{\pgfqpoint{4.377964in}{2.300113in}}%
\pgfpathlineto{\pgfqpoint{4.391426in}{2.305879in}}%
\pgfpathlineto{\pgfqpoint{4.404901in}{2.311807in}}%
\pgfpathlineto{\pgfqpoint{4.412490in}{2.321536in}}%
\pgfpathlineto{\pgfqpoint{4.420074in}{2.331187in}}%
\pgfpathlineto{\pgfqpoint{4.427652in}{2.340760in}}%
\pgfpathlineto{\pgfqpoint{4.435225in}{2.350258in}}%
\pgfpathlineto{\pgfqpoint{4.421756in}{2.344339in}}%
\pgfpathlineto{\pgfqpoint{4.408299in}{2.338583in}}%
\pgfpathlineto{\pgfqpoint{4.394855in}{2.332989in}}%
\pgfpathlineto{\pgfqpoint{4.381422in}{2.327558in}}%
\pgfpathlineto{\pgfqpoint{4.373844in}{2.318041in}}%
\pgfpathlineto{\pgfqpoint{4.366260in}{2.308454in}}%
\pgfpathlineto{\pgfqpoint{4.358670in}{2.298798in}}%
\pgfpathlineto{\pgfqpoint{4.351075in}{2.289071in}}%
\pgfpathclose%
\pgfusepath{fill}%
\end{pgfscope}%
\begin{pgfscope}%
\pgfpathrectangle{\pgfqpoint{1.254980in}{0.150000in}}{\pgfqpoint{5.490039in}{5.490039in}}%
\pgfusepath{clip}%
\pgfsetbuttcap%
\pgfsetroundjoin%
\definecolor{currentfill}{rgb}{0.255645,0.260703,0.528312}%
\pgfsetfillcolor{currentfill}%
\pgfsetfillopacity{0.700000}%
\pgfsetlinewidth{0.000000pt}%
\definecolor{currentstroke}{rgb}{0.000000,0.000000,0.000000}%
\pgfsetstrokecolor{currentstroke}%
\pgfsetdash{}{0pt}%
\pgfpathmoveto{\pgfqpoint{2.555904in}{2.291551in}}%
\pgfpathlineto{\pgfqpoint{2.569297in}{2.274216in}}%
\pgfpathlineto{\pgfqpoint{2.582681in}{2.257140in}}%
\pgfpathlineto{\pgfqpoint{2.596058in}{2.240322in}}%
\pgfpathlineto{\pgfqpoint{2.609425in}{2.223759in}}%
\pgfpathlineto{\pgfqpoint{2.617839in}{2.225564in}}%
\pgfpathlineto{\pgfqpoint{2.626238in}{2.227575in}}%
\pgfpathlineto{\pgfqpoint{2.634623in}{2.229790in}}%
\pgfpathlineto{\pgfqpoint{2.642994in}{2.232204in}}%
\pgfpathlineto{\pgfqpoint{2.629664in}{2.248335in}}%
\pgfpathlineto{\pgfqpoint{2.616326in}{2.264721in}}%
\pgfpathlineto{\pgfqpoint{2.602981in}{2.281364in}}%
\pgfpathlineto{\pgfqpoint{2.589627in}{2.298266in}}%
\pgfpathlineto{\pgfqpoint{2.581218in}{2.296273in}}%
\pgfpathlineto{\pgfqpoint{2.572795in}{2.294487in}}%
\pgfpathlineto{\pgfqpoint{2.564357in}{2.292911in}}%
\pgfpathlineto{\pgfqpoint{2.555904in}{2.291551in}}%
\pgfpathclose%
\pgfusepath{fill}%
\end{pgfscope}%
\begin{pgfscope}%
\pgfpathrectangle{\pgfqpoint{1.254980in}{0.150000in}}{\pgfqpoint{5.490039in}{5.490039in}}%
\pgfusepath{clip}%
\pgfsetbuttcap%
\pgfsetroundjoin%
\definecolor{currentfill}{rgb}{0.279574,0.170599,0.479997}%
\pgfsetfillcolor{currentfill}%
\pgfsetfillopacity{0.700000}%
\pgfsetlinewidth{0.000000pt}%
\definecolor{currentstroke}{rgb}{0.000000,0.000000,0.000000}%
\pgfsetstrokecolor{currentstroke}%
\pgfsetdash{}{0pt}%
\pgfpathmoveto{\pgfqpoint{2.716107in}{2.100208in}}%
\pgfpathlineto{\pgfqpoint{2.729413in}{2.085853in}}%
\pgfpathlineto{\pgfqpoint{2.742714in}{2.071734in}}%
\pgfpathlineto{\pgfqpoint{2.756009in}{2.057849in}}%
\pgfpathlineto{\pgfqpoint{2.769298in}{2.044197in}}%
\pgfpathlineto{\pgfqpoint{2.777602in}{2.047316in}}%
\pgfpathlineto{\pgfqpoint{2.785894in}{2.050620in}}%
\pgfpathlineto{\pgfqpoint{2.794173in}{2.054105in}}%
\pgfpathlineto{\pgfqpoint{2.802440in}{2.057766in}}%
\pgfpathlineto{\pgfqpoint{2.789184in}{2.070995in}}%
\pgfpathlineto{\pgfqpoint{2.775923in}{2.084455in}}%
\pgfpathlineto{\pgfqpoint{2.762657in}{2.098149in}}%
\pgfpathlineto{\pgfqpoint{2.749385in}{2.112079in}}%
\pgfpathlineto{\pgfqpoint{2.741085in}{2.108831in}}%
\pgfpathlineto{\pgfqpoint{2.732772in}{2.105767in}}%
\pgfpathlineto{\pgfqpoint{2.724446in}{2.102892in}}%
\pgfpathlineto{\pgfqpoint{2.716107in}{2.100208in}}%
\pgfpathclose%
\pgfusepath{fill}%
\end{pgfscope}%
\begin{pgfscope}%
\pgfpathrectangle{\pgfqpoint{1.254980in}{0.150000in}}{\pgfqpoint{5.490039in}{5.490039in}}%
\pgfusepath{clip}%
\pgfsetbuttcap%
\pgfsetroundjoin%
\definecolor{currentfill}{rgb}{0.282910,0.105393,0.426902}%
\pgfsetfillcolor{currentfill}%
\pgfsetfillopacity{0.700000}%
\pgfsetlinewidth{0.000000pt}%
\definecolor{currentstroke}{rgb}{0.000000,0.000000,0.000000}%
\pgfsetstrokecolor{currentstroke}%
\pgfsetdash{}{0pt}%
\pgfpathmoveto{\pgfqpoint{3.816039in}{1.925075in}}%
\pgfpathlineto{\pgfqpoint{3.829283in}{1.925793in}}%
\pgfpathlineto{\pgfqpoint{3.842535in}{1.926682in}}%
\pgfpathlineto{\pgfqpoint{3.855795in}{1.927740in}}%
\pgfpathlineto{\pgfqpoint{3.869062in}{1.928968in}}%
\pgfpathlineto{\pgfqpoint{3.876827in}{1.939560in}}%
\pgfpathlineto{\pgfqpoint{3.884587in}{1.950133in}}%
\pgfpathlineto{\pgfqpoint{3.892342in}{1.960687in}}%
\pgfpathlineto{\pgfqpoint{3.900092in}{1.971218in}}%
\pgfpathlineto{\pgfqpoint{3.886832in}{1.969802in}}%
\pgfpathlineto{\pgfqpoint{3.873580in}{1.968555in}}%
\pgfpathlineto{\pgfqpoint{3.860336in}{1.967478in}}%
\pgfpathlineto{\pgfqpoint{3.847099in}{1.966571in}}%
\pgfpathlineto{\pgfqpoint{3.839342in}{1.956217in}}%
\pgfpathlineto{\pgfqpoint{3.831579in}{1.945849in}}%
\pgfpathlineto{\pgfqpoint{3.823812in}{1.935468in}}%
\pgfpathlineto{\pgfqpoint{3.816039in}{1.925075in}}%
\pgfpathclose%
\pgfusepath{fill}%
\end{pgfscope}%
\begin{pgfscope}%
\pgfpathrectangle{\pgfqpoint{1.254980in}{0.150000in}}{\pgfqpoint{5.490039in}{5.490039in}}%
\pgfusepath{clip}%
\pgfsetbuttcap%
\pgfsetroundjoin%
\definecolor{currentfill}{rgb}{0.280894,0.078907,0.402329}%
\pgfsetfillcolor{currentfill}%
\pgfsetfillopacity{0.700000}%
\pgfsetlinewidth{0.000000pt}%
\definecolor{currentstroke}{rgb}{0.000000,0.000000,0.000000}%
\pgfsetstrokecolor{currentstroke}%
\pgfsetdash{}{0pt}%
\pgfpathmoveto{\pgfqpoint{3.731962in}{1.883121in}}%
\pgfpathlineto{\pgfqpoint{3.745185in}{1.882939in}}%
\pgfpathlineto{\pgfqpoint{3.758416in}{1.882928in}}%
\pgfpathlineto{\pgfqpoint{3.771654in}{1.883089in}}%
\pgfpathlineto{\pgfqpoint{3.784899in}{1.883420in}}%
\pgfpathlineto{\pgfqpoint{3.792692in}{1.893843in}}%
\pgfpathlineto{\pgfqpoint{3.800479in}{1.904260in}}%
\pgfpathlineto{\pgfqpoint{3.808262in}{1.914672in}}%
\pgfpathlineto{\pgfqpoint{3.816039in}{1.925075in}}%
\pgfpathlineto{\pgfqpoint{3.802803in}{1.924526in}}%
\pgfpathlineto{\pgfqpoint{3.789573in}{1.924149in}}%
\pgfpathlineto{\pgfqpoint{3.776352in}{1.923944in}}%
\pgfpathlineto{\pgfqpoint{3.763137in}{1.923910in}}%
\pgfpathlineto{\pgfqpoint{3.755351in}{1.913713in}}%
\pgfpathlineto{\pgfqpoint{3.747559in}{1.903515in}}%
\pgfpathlineto{\pgfqpoint{3.739763in}{1.893317in}}%
\pgfpathlineto{\pgfqpoint{3.731962in}{1.883121in}}%
\pgfpathclose%
\pgfusepath{fill}%
\end{pgfscope}%
\begin{pgfscope}%
\pgfpathrectangle{\pgfqpoint{1.254980in}{0.150000in}}{\pgfqpoint{5.490039in}{5.490039in}}%
\pgfusepath{clip}%
\pgfsetbuttcap%
\pgfsetroundjoin%
\definecolor{currentfill}{rgb}{0.132444,0.552216,0.553018}%
\pgfsetfillcolor{currentfill}%
\pgfsetfillopacity{0.700000}%
\pgfsetlinewidth{0.000000pt}%
\definecolor{currentstroke}{rgb}{0.000000,0.000000,0.000000}%
\pgfsetstrokecolor{currentstroke}%
\pgfsetdash{}{0pt}%
\pgfpathmoveto{\pgfqpoint{5.253147in}{2.961288in}}%
\pgfpathlineto{\pgfqpoint{5.267052in}{2.971070in}}%
\pgfpathlineto{\pgfqpoint{5.280975in}{2.981009in}}%
\pgfpathlineto{\pgfqpoint{5.294914in}{2.991105in}}%
\pgfpathlineto{\pgfqpoint{5.308871in}{3.001359in}}%
\pgfpathlineto{\pgfqpoint{5.316038in}{3.005725in}}%
\pgfpathlineto{\pgfqpoint{5.323199in}{3.010050in}}%
\pgfpathlineto{\pgfqpoint{5.330352in}{3.014337in}}%
\pgfpathlineto{\pgfqpoint{5.337498in}{3.018592in}}%
\pgfpathlineto{\pgfqpoint{5.323559in}{3.008704in}}%
\pgfpathlineto{\pgfqpoint{5.309638in}{2.998973in}}%
\pgfpathlineto{\pgfqpoint{5.295734in}{2.989398in}}%
\pgfpathlineto{\pgfqpoint{5.281847in}{2.979979in}}%
\pgfpathlineto{\pgfqpoint{5.274682in}{2.975350in}}%
\pgfpathlineto{\pgfqpoint{5.267510in}{2.970694in}}%
\pgfpathlineto{\pgfqpoint{5.260332in}{2.966008in}}%
\pgfpathlineto{\pgfqpoint{5.253147in}{2.961288in}}%
\pgfpathclose%
\pgfusepath{fill}%
\end{pgfscope}%
\begin{pgfscope}%
\pgfpathrectangle{\pgfqpoint{1.254980in}{0.150000in}}{\pgfqpoint{5.490039in}{5.490039in}}%
\pgfusepath{clip}%
\pgfsetbuttcap%
\pgfsetroundjoin%
\definecolor{currentfill}{rgb}{0.283072,0.130895,0.449241}%
\pgfsetfillcolor{currentfill}%
\pgfsetfillopacity{0.700000}%
\pgfsetlinewidth{0.000000pt}%
\definecolor{currentstroke}{rgb}{0.000000,0.000000,0.000000}%
\pgfsetstrokecolor{currentstroke}%
\pgfsetdash{}{0pt}%
\pgfpathmoveto{\pgfqpoint{3.900092in}{1.971218in}}%
\pgfpathlineto{\pgfqpoint{3.913361in}{1.972803in}}%
\pgfpathlineto{\pgfqpoint{3.926638in}{1.974557in}}%
\pgfpathlineto{\pgfqpoint{3.939923in}{1.976479in}}%
\pgfpathlineto{\pgfqpoint{3.953218in}{1.978568in}}%
\pgfpathlineto{\pgfqpoint{3.960956in}{1.989248in}}%
\pgfpathlineto{\pgfqpoint{3.968689in}{1.999897in}}%
\pgfpathlineto{\pgfqpoint{3.976418in}{2.010512in}}%
\pgfpathlineto{\pgfqpoint{3.984142in}{2.021094in}}%
\pgfpathlineto{\pgfqpoint{3.970854in}{2.018844in}}%
\pgfpathlineto{\pgfqpoint{3.957575in}{2.016761in}}%
\pgfpathlineto{\pgfqpoint{3.944305in}{2.014847in}}%
\pgfpathlineto{\pgfqpoint{3.931043in}{2.013101in}}%
\pgfpathlineto{\pgfqpoint{3.923313in}{2.002669in}}%
\pgfpathlineto{\pgfqpoint{3.915577in}{1.992211in}}%
\pgfpathlineto{\pgfqpoint{3.907837in}{1.981727in}}%
\pgfpathlineto{\pgfqpoint{3.900092in}{1.971218in}}%
\pgfpathclose%
\pgfusepath{fill}%
\end{pgfscope}%
\begin{pgfscope}%
\pgfpathrectangle{\pgfqpoint{1.254980in}{0.150000in}}{\pgfqpoint{5.490039in}{5.490039in}}%
\pgfusepath{clip}%
\pgfsetbuttcap%
\pgfsetroundjoin%
\definecolor{currentfill}{rgb}{0.271305,0.019942,0.347269}%
\pgfsetfillcolor{currentfill}%
\pgfsetfillopacity{0.700000}%
\pgfsetlinewidth{0.000000pt}%
\definecolor{currentstroke}{rgb}{0.000000,0.000000,0.000000}%
\pgfsetstrokecolor{currentstroke}%
\pgfsetdash{}{0pt}%
\pgfpathmoveto{\pgfqpoint{3.204694in}{1.802507in}}%
\pgfpathlineto{\pgfqpoint{3.217876in}{1.795754in}}%
\pgfpathlineto{\pgfqpoint{3.231059in}{1.789193in}}%
\pgfpathlineto{\pgfqpoint{3.244243in}{1.782823in}}%
\pgfpathlineto{\pgfqpoint{3.257429in}{1.776643in}}%
\pgfpathlineto{\pgfqpoint{3.265440in}{1.784092in}}%
\pgfpathlineto{\pgfqpoint{3.273443in}{1.791637in}}%
\pgfpathlineto{\pgfqpoint{3.281438in}{1.799277in}}%
\pgfpathlineto{\pgfqpoint{3.289425in}{1.807006in}}%
\pgfpathlineto{\pgfqpoint{3.276259in}{1.812830in}}%
\pgfpathlineto{\pgfqpoint{3.263094in}{1.818844in}}%
\pgfpathlineto{\pgfqpoint{3.249932in}{1.825049in}}%
\pgfpathlineto{\pgfqpoint{3.236770in}{1.831445in}}%
\pgfpathlineto{\pgfqpoint{3.228763in}{1.824061in}}%
\pgfpathlineto{\pgfqpoint{3.220748in}{1.816775in}}%
\pgfpathlineto{\pgfqpoint{3.212725in}{1.809589in}}%
\pgfpathlineto{\pgfqpoint{3.204694in}{1.802507in}}%
\pgfpathclose%
\pgfusepath{fill}%
\end{pgfscope}%
\begin{pgfscope}%
\pgfpathrectangle{\pgfqpoint{1.254980in}{0.150000in}}{\pgfqpoint{5.490039in}{5.490039in}}%
\pgfusepath{clip}%
\pgfsetbuttcap%
\pgfsetroundjoin%
\definecolor{currentfill}{rgb}{0.243113,0.292092,0.538516}%
\pgfsetfillcolor{currentfill}%
\pgfsetfillopacity{0.700000}%
\pgfsetlinewidth{0.000000pt}%
\definecolor{currentstroke}{rgb}{0.000000,0.000000,0.000000}%
\pgfsetstrokecolor{currentstroke}%
\pgfsetdash{}{0pt}%
\pgfpathmoveto{\pgfqpoint{2.502240in}{2.363538in}}%
\pgfpathlineto{\pgfqpoint{2.515670in}{2.345140in}}%
\pgfpathlineto{\pgfqpoint{2.529091in}{2.327011in}}%
\pgfpathlineto{\pgfqpoint{2.542502in}{2.309149in}}%
\pgfpathlineto{\pgfqpoint{2.555904in}{2.291551in}}%
\pgfpathlineto{\pgfqpoint{2.564357in}{2.292911in}}%
\pgfpathlineto{\pgfqpoint{2.572795in}{2.294487in}}%
\pgfpathlineto{\pgfqpoint{2.581218in}{2.296273in}}%
\pgfpathlineto{\pgfqpoint{2.589627in}{2.298266in}}%
\pgfpathlineto{\pgfqpoint{2.576264in}{2.315429in}}%
\pgfpathlineto{\pgfqpoint{2.562893in}{2.332856in}}%
\pgfpathlineto{\pgfqpoint{2.549513in}{2.350548in}}%
\pgfpathlineto{\pgfqpoint{2.536124in}{2.368509in}}%
\pgfpathlineto{\pgfqpoint{2.527676in}{2.366941in}}%
\pgfpathlineto{\pgfqpoint{2.519212in}{2.365586in}}%
\pgfpathlineto{\pgfqpoint{2.510734in}{2.364451in}}%
\pgfpathlineto{\pgfqpoint{2.502240in}{2.363538in}}%
\pgfpathclose%
\pgfusepath{fill}%
\end{pgfscope}%
\begin{pgfscope}%
\pgfpathrectangle{\pgfqpoint{1.254980in}{0.150000in}}{\pgfqpoint{5.490039in}{5.490039in}}%
\pgfusepath{clip}%
\pgfsetbuttcap%
\pgfsetroundjoin%
\definecolor{currentfill}{rgb}{0.278791,0.062145,0.386592}%
\pgfsetfillcolor{currentfill}%
\pgfsetfillopacity{0.700000}%
\pgfsetlinewidth{0.000000pt}%
\definecolor{currentstroke}{rgb}{0.000000,0.000000,0.000000}%
\pgfsetstrokecolor{currentstroke}%
\pgfsetdash{}{0pt}%
\pgfpathmoveto{\pgfqpoint{3.647834in}{1.845839in}}%
\pgfpathlineto{\pgfqpoint{3.661042in}{1.844720in}}%
\pgfpathlineto{\pgfqpoint{3.674256in}{1.843774in}}%
\pgfpathlineto{\pgfqpoint{3.687477in}{1.843002in}}%
\pgfpathlineto{\pgfqpoint{3.700704in}{1.842403in}}%
\pgfpathlineto{\pgfqpoint{3.708526in}{1.852569in}}%
\pgfpathlineto{\pgfqpoint{3.716343in}{1.862745in}}%
\pgfpathlineto{\pgfqpoint{3.724155in}{1.872930in}}%
\pgfpathlineto{\pgfqpoint{3.731962in}{1.883121in}}%
\pgfpathlineto{\pgfqpoint{3.718745in}{1.883477in}}%
\pgfpathlineto{\pgfqpoint{3.705534in}{1.884005in}}%
\pgfpathlineto{\pgfqpoint{3.692330in}{1.884706in}}%
\pgfpathlineto{\pgfqpoint{3.679132in}{1.885581in}}%
\pgfpathlineto{\pgfqpoint{3.671316in}{1.875624in}}%
\pgfpathlineto{\pgfqpoint{3.663494in}{1.865679in}}%
\pgfpathlineto{\pgfqpoint{3.655667in}{1.855750in}}%
\pgfpathlineto{\pgfqpoint{3.647834in}{1.845839in}}%
\pgfpathclose%
\pgfusepath{fill}%
\end{pgfscope}%
\begin{pgfscope}%
\pgfpathrectangle{\pgfqpoint{1.254980in}{0.150000in}}{\pgfqpoint{5.490039in}{5.490039in}}%
\pgfusepath{clip}%
\pgfsetbuttcap%
\pgfsetroundjoin%
\definecolor{currentfill}{rgb}{0.282290,0.145912,0.461510}%
\pgfsetfillcolor{currentfill}%
\pgfsetfillopacity{0.700000}%
\pgfsetlinewidth{0.000000pt}%
\definecolor{currentstroke}{rgb}{0.000000,0.000000,0.000000}%
\pgfsetstrokecolor{currentstroke}%
\pgfsetdash{}{0pt}%
\pgfpathmoveto{\pgfqpoint{2.769298in}{2.044197in}}%
\pgfpathlineto{\pgfqpoint{2.782583in}{2.030775in}}%
\pgfpathlineto{\pgfqpoint{2.795863in}{2.017581in}}%
\pgfpathlineto{\pgfqpoint{2.809138in}{2.004615in}}%
\pgfpathlineto{\pgfqpoint{2.822409in}{1.991875in}}%
\pgfpathlineto{\pgfqpoint{2.830680in}{1.995427in}}%
\pgfpathlineto{\pgfqpoint{2.838939in}{1.999157in}}%
\pgfpathlineto{\pgfqpoint{2.847186in}{2.003061in}}%
\pgfpathlineto{\pgfqpoint{2.855421in}{2.007133in}}%
\pgfpathlineto{\pgfqpoint{2.842182in}{2.019452in}}%
\pgfpathlineto{\pgfqpoint{2.828939in}{2.031996in}}%
\pgfpathlineto{\pgfqpoint{2.815692in}{2.044767in}}%
\pgfpathlineto{\pgfqpoint{2.802440in}{2.057766in}}%
\pgfpathlineto{\pgfqpoint{2.794173in}{2.054105in}}%
\pgfpathlineto{\pgfqpoint{2.785894in}{2.050620in}}%
\pgfpathlineto{\pgfqpoint{2.777602in}{2.047316in}}%
\pgfpathlineto{\pgfqpoint{2.769298in}{2.044197in}}%
\pgfpathclose%
\pgfusepath{fill}%
\end{pgfscope}%
\begin{pgfscope}%
\pgfpathrectangle{\pgfqpoint{1.254980in}{0.150000in}}{\pgfqpoint{5.490039in}{5.490039in}}%
\pgfusepath{clip}%
\pgfsetbuttcap%
\pgfsetroundjoin%
\definecolor{currentfill}{rgb}{0.271305,0.019942,0.347269}%
\pgfsetfillcolor{currentfill}%
\pgfsetfillopacity{0.700000}%
\pgfsetlinewidth{0.000000pt}%
\definecolor{currentstroke}{rgb}{0.000000,0.000000,0.000000}%
\pgfsetstrokecolor{currentstroke}%
\pgfsetdash{}{0pt}%
\pgfpathmoveto{\pgfqpoint{3.342114in}{1.785593in}}%
\pgfpathlineto{\pgfqpoint{3.355292in}{1.780706in}}%
\pgfpathlineto{\pgfqpoint{3.368474in}{1.776003in}}%
\pgfpathlineto{\pgfqpoint{3.381658in}{1.771485in}}%
\pgfpathlineto{\pgfqpoint{3.394846in}{1.767150in}}%
\pgfpathlineto{\pgfqpoint{3.402791in}{1.775643in}}%
\pgfpathlineto{\pgfqpoint{3.410729in}{1.784206in}}%
\pgfpathlineto{\pgfqpoint{3.418661in}{1.792836in}}%
\pgfpathlineto{\pgfqpoint{3.426586in}{1.801530in}}%
\pgfpathlineto{\pgfqpoint{3.413415in}{1.805538in}}%
\pgfpathlineto{\pgfqpoint{3.400247in}{1.809729in}}%
\pgfpathlineto{\pgfqpoint{3.387082in}{1.814104in}}%
\pgfpathlineto{\pgfqpoint{3.373921in}{1.818664in}}%
\pgfpathlineto{\pgfqpoint{3.365979in}{1.810286in}}%
\pgfpathlineto{\pgfqpoint{3.358031in}{1.801980in}}%
\pgfpathlineto{\pgfqpoint{3.350076in}{1.793748in}}%
\pgfpathlineto{\pgfqpoint{3.342114in}{1.785593in}}%
\pgfpathclose%
\pgfusepath{fill}%
\end{pgfscope}%
\begin{pgfscope}%
\pgfpathrectangle{\pgfqpoint{1.254980in}{0.150000in}}{\pgfqpoint{5.490039in}{5.490039in}}%
\pgfusepath{clip}%
\pgfsetbuttcap%
\pgfsetroundjoin%
\definecolor{currentfill}{rgb}{0.180629,0.429975,0.557282}%
\pgfsetfillcolor{currentfill}%
\pgfsetfillopacity{0.700000}%
\pgfsetlinewidth{0.000000pt}%
\definecolor{currentstroke}{rgb}{0.000000,0.000000,0.000000}%
\pgfsetstrokecolor{currentstroke}%
\pgfsetdash{}{0pt}%
\pgfpathmoveto{\pgfqpoint{4.802208in}{2.634618in}}%
\pgfpathlineto{\pgfqpoint{4.815872in}{2.642791in}}%
\pgfpathlineto{\pgfqpoint{4.829552in}{2.651124in}}%
\pgfpathlineto{\pgfqpoint{4.843247in}{2.659616in}}%
\pgfpathlineto{\pgfqpoint{4.856956in}{2.668268in}}%
\pgfpathlineto{\pgfqpoint{4.864362in}{2.675556in}}%
\pgfpathlineto{\pgfqpoint{4.871762in}{2.682762in}}%
\pgfpathlineto{\pgfqpoint{4.879154in}{2.689890in}}%
\pgfpathlineto{\pgfqpoint{4.886540in}{2.696940in}}%
\pgfpathlineto{\pgfqpoint{4.872840in}{2.688474in}}%
\pgfpathlineto{\pgfqpoint{4.859155in}{2.680167in}}%
\pgfpathlineto{\pgfqpoint{4.845486in}{2.672019in}}%
\pgfpathlineto{\pgfqpoint{4.831831in}{2.664031in}}%
\pgfpathlineto{\pgfqpoint{4.824435in}{2.656785in}}%
\pgfpathlineto{\pgfqpoint{4.817032in}{2.649469in}}%
\pgfpathlineto{\pgfqpoint{4.809623in}{2.642081in}}%
\pgfpathlineto{\pgfqpoint{4.802208in}{2.634618in}}%
\pgfpathclose%
\pgfusepath{fill}%
\end{pgfscope}%
\begin{pgfscope}%
\pgfpathrectangle{\pgfqpoint{1.254980in}{0.150000in}}{\pgfqpoint{5.490039in}{5.490039in}}%
\pgfusepath{clip}%
\pgfsetbuttcap%
\pgfsetroundjoin%
\definecolor{currentfill}{rgb}{0.281412,0.155834,0.469201}%
\pgfsetfillcolor{currentfill}%
\pgfsetfillopacity{0.700000}%
\pgfsetlinewidth{0.000000pt}%
\definecolor{currentstroke}{rgb}{0.000000,0.000000,0.000000}%
\pgfsetstrokecolor{currentstroke}%
\pgfsetdash{}{0pt}%
\pgfpathmoveto{\pgfqpoint{3.984142in}{2.021094in}}%
\pgfpathlineto{\pgfqpoint{3.997439in}{2.023512in}}%
\pgfpathlineto{\pgfqpoint{4.010745in}{2.026097in}}%
\pgfpathlineto{\pgfqpoint{4.024060in}{2.028849in}}%
\pgfpathlineto{\pgfqpoint{4.037385in}{2.031767in}}%
\pgfpathlineto{\pgfqpoint{4.045098in}{2.042457in}}%
\pgfpathlineto{\pgfqpoint{4.052806in}{2.053104in}}%
\pgfpathlineto{\pgfqpoint{4.060509in}{2.063707in}}%
\pgfpathlineto{\pgfqpoint{4.068207in}{2.074264in}}%
\pgfpathlineto{\pgfqpoint{4.054888in}{2.071213in}}%
\pgfpathlineto{\pgfqpoint{4.041579in}{2.068328in}}%
\pgfpathlineto{\pgfqpoint{4.028279in}{2.065610in}}%
\pgfpathlineto{\pgfqpoint{4.014988in}{2.063060in}}%
\pgfpathlineto{\pgfqpoint{4.007284in}{2.052624in}}%
\pgfpathlineto{\pgfqpoint{3.999575in}{2.042151in}}%
\pgfpathlineto{\pgfqpoint{3.991861in}{2.031641in}}%
\pgfpathlineto{\pgfqpoint{3.984142in}{2.021094in}}%
\pgfpathclose%
\pgfusepath{fill}%
\end{pgfscope}%
\begin{pgfscope}%
\pgfpathrectangle{\pgfqpoint{1.254980in}{0.150000in}}{\pgfqpoint{5.490039in}{5.490039in}}%
\pgfusepath{clip}%
\pgfsetbuttcap%
\pgfsetroundjoin%
\definecolor{currentfill}{rgb}{0.233603,0.313828,0.543914}%
\pgfsetfillcolor{currentfill}%
\pgfsetfillopacity{0.700000}%
\pgfsetlinewidth{0.000000pt}%
\definecolor{currentstroke}{rgb}{0.000000,0.000000,0.000000}%
\pgfsetstrokecolor{currentstroke}%
\pgfsetdash{}{0pt}%
\pgfpathmoveto{\pgfqpoint{4.435225in}{2.350258in}}%
\pgfpathlineto{\pgfqpoint{4.448707in}{2.356339in}}%
\pgfpathlineto{\pgfqpoint{4.462201in}{2.362582in}}%
\pgfpathlineto{\pgfqpoint{4.475708in}{2.368988in}}%
\pgfpathlineto{\pgfqpoint{4.489228in}{2.375555in}}%
\pgfpathlineto{\pgfqpoint{4.496789in}{2.384949in}}%
\pgfpathlineto{\pgfqpoint{4.504345in}{2.394261in}}%
\pgfpathlineto{\pgfqpoint{4.511895in}{2.403491in}}%
\pgfpathlineto{\pgfqpoint{4.519439in}{2.412640in}}%
\pgfpathlineto{\pgfqpoint{4.505925in}{2.406111in}}%
\pgfpathlineto{\pgfqpoint{4.492424in}{2.399744in}}%
\pgfpathlineto{\pgfqpoint{4.478936in}{2.393538in}}%
\pgfpathlineto{\pgfqpoint{4.465460in}{2.387495in}}%
\pgfpathlineto{\pgfqpoint{4.457910in}{2.378297in}}%
\pgfpathlineto{\pgfqpoint{4.450354in}{2.369025in}}%
\pgfpathlineto{\pgfqpoint{4.442792in}{2.359679in}}%
\pgfpathlineto{\pgfqpoint{4.435225in}{2.350258in}}%
\pgfpathclose%
\pgfusepath{fill}%
\end{pgfscope}%
\begin{pgfscope}%
\pgfpathrectangle{\pgfqpoint{1.254980in}{0.150000in}}{\pgfqpoint{5.490039in}{5.490039in}}%
\pgfusepath{clip}%
\pgfsetbuttcap%
\pgfsetroundjoin%
\definecolor{currentfill}{rgb}{0.125394,0.574318,0.549086}%
\pgfsetfillcolor{currentfill}%
\pgfsetfillopacity{0.700000}%
\pgfsetlinewidth{0.000000pt}%
\definecolor{currentstroke}{rgb}{0.000000,0.000000,0.000000}%
\pgfsetstrokecolor{currentstroke}%
\pgfsetdash{}{0pt}%
\pgfpathmoveto{\pgfqpoint{5.337498in}{3.018592in}}%
\pgfpathlineto{\pgfqpoint{5.351453in}{3.028637in}}%
\pgfpathlineto{\pgfqpoint{5.365427in}{3.038839in}}%
\pgfpathlineto{\pgfqpoint{5.379417in}{3.049197in}}%
\pgfpathlineto{\pgfqpoint{5.393426in}{3.059712in}}%
\pgfpathlineto{\pgfqpoint{5.400545in}{3.063554in}}%
\pgfpathlineto{\pgfqpoint{5.407658in}{3.067364in}}%
\pgfpathlineto{\pgfqpoint{5.414763in}{3.071147in}}%
\pgfpathlineto{\pgfqpoint{5.421862in}{3.074908in}}%
\pgfpathlineto{\pgfqpoint{5.407874in}{3.064789in}}%
\pgfpathlineto{\pgfqpoint{5.393903in}{3.054826in}}%
\pgfpathlineto{\pgfqpoint{5.379950in}{3.045019in}}%
\pgfpathlineto{\pgfqpoint{5.366015in}{3.035368in}}%
\pgfpathlineto{\pgfqpoint{5.358895in}{3.031202in}}%
\pgfpathlineto{\pgfqpoint{5.351769in}{3.027020in}}%
\pgfpathlineto{\pgfqpoint{5.344637in}{3.022818in}}%
\pgfpathlineto{\pgfqpoint{5.337498in}{3.018592in}}%
\pgfpathclose%
\pgfusepath{fill}%
\end{pgfscope}%
\begin{pgfscope}%
\pgfpathrectangle{\pgfqpoint{1.254980in}{0.150000in}}{\pgfqpoint{5.490039in}{5.490039in}}%
\pgfusepath{clip}%
\pgfsetbuttcap%
\pgfsetroundjoin%
\definecolor{currentfill}{rgb}{0.227802,0.326594,0.546532}%
\pgfsetfillcolor{currentfill}%
\pgfsetfillopacity{0.700000}%
\pgfsetlinewidth{0.000000pt}%
\definecolor{currentstroke}{rgb}{0.000000,0.000000,0.000000}%
\pgfsetstrokecolor{currentstroke}%
\pgfsetdash{}{0pt}%
\pgfpathmoveto{\pgfqpoint{2.448416in}{2.439869in}}%
\pgfpathlineto{\pgfqpoint{2.461888in}{2.420370in}}%
\pgfpathlineto{\pgfqpoint{2.475349in}{2.401150in}}%
\pgfpathlineto{\pgfqpoint{2.488800in}{2.382207in}}%
\pgfpathlineto{\pgfqpoint{2.502240in}{2.363538in}}%
\pgfpathlineto{\pgfqpoint{2.510734in}{2.364451in}}%
\pgfpathlineto{\pgfqpoint{2.519212in}{2.365586in}}%
\pgfpathlineto{\pgfqpoint{2.527676in}{2.366941in}}%
\pgfpathlineto{\pgfqpoint{2.536124in}{2.368509in}}%
\pgfpathlineto{\pgfqpoint{2.522725in}{2.386741in}}%
\pgfpathlineto{\pgfqpoint{2.509316in}{2.405246in}}%
\pgfpathlineto{\pgfqpoint{2.495897in}{2.424026in}}%
\pgfpathlineto{\pgfqpoint{2.482468in}{2.443085in}}%
\pgfpathlineto{\pgfqpoint{2.473978in}{2.441944in}}%
\pgfpathlineto{\pgfqpoint{2.465474in}{2.441024in}}%
\pgfpathlineto{\pgfqpoint{2.456953in}{2.440332in}}%
\pgfpathlineto{\pgfqpoint{2.448416in}{2.439869in}}%
\pgfpathclose%
\pgfusepath{fill}%
\end{pgfscope}%
\begin{pgfscope}%
\pgfpathrectangle{\pgfqpoint{1.254980in}{0.150000in}}{\pgfqpoint{5.490039in}{5.490039in}}%
\pgfusepath{clip}%
\pgfsetbuttcap%
\pgfsetroundjoin%
\definecolor{currentfill}{rgb}{0.276022,0.044167,0.370164}%
\pgfsetfillcolor{currentfill}%
\pgfsetfillopacity{0.700000}%
\pgfsetlinewidth{0.000000pt}%
\definecolor{currentstroke}{rgb}{0.000000,0.000000,0.000000}%
\pgfsetstrokecolor{currentstroke}%
\pgfsetdash{}{0pt}%
\pgfpathmoveto{\pgfqpoint{3.563628in}{1.813729in}}%
\pgfpathlineto{\pgfqpoint{3.576825in}{1.811636in}}%
\pgfpathlineto{\pgfqpoint{3.590027in}{1.809720in}}%
\pgfpathlineto{\pgfqpoint{3.603235in}{1.807979in}}%
\pgfpathlineto{\pgfqpoint{3.616448in}{1.806414in}}%
\pgfpathlineto{\pgfqpoint{3.624303in}{1.816232in}}%
\pgfpathlineto{\pgfqpoint{3.632152in}{1.826077in}}%
\pgfpathlineto{\pgfqpoint{3.639996in}{1.835947in}}%
\pgfpathlineto{\pgfqpoint{3.647834in}{1.845839in}}%
\pgfpathlineto{\pgfqpoint{3.634632in}{1.847133in}}%
\pgfpathlineto{\pgfqpoint{3.621436in}{1.848602in}}%
\pgfpathlineto{\pgfqpoint{3.608245in}{1.850246in}}%
\pgfpathlineto{\pgfqpoint{3.595060in}{1.852067in}}%
\pgfpathlineto{\pgfqpoint{3.587210in}{1.842437in}}%
\pgfpathlineto{\pgfqpoint{3.579355in}{1.832835in}}%
\pgfpathlineto{\pgfqpoint{3.571494in}{1.823265in}}%
\pgfpathlineto{\pgfqpoint{3.563628in}{1.813729in}}%
\pgfpathclose%
\pgfusepath{fill}%
\end{pgfscope}%
\begin{pgfscope}%
\pgfpathrectangle{\pgfqpoint{1.254980in}{0.150000in}}{\pgfqpoint{5.490039in}{5.490039in}}%
\pgfusepath{clip}%
\pgfsetbuttcap%
\pgfsetroundjoin%
\definecolor{currentfill}{rgb}{0.276022,0.044167,0.370164}%
\pgfsetfillcolor{currentfill}%
\pgfsetfillopacity{0.700000}%
\pgfsetlinewidth{0.000000pt}%
\definecolor{currentstroke}{rgb}{0.000000,0.000000,0.000000}%
\pgfsetstrokecolor{currentstroke}%
\pgfsetdash{}{0pt}%
\pgfpathmoveto{\pgfqpoint{3.066871in}{1.839435in}}%
\pgfpathlineto{\pgfqpoint{3.080073in}{1.830722in}}%
\pgfpathlineto{\pgfqpoint{3.093275in}{1.822210in}}%
\pgfpathlineto{\pgfqpoint{3.106476in}{1.813898in}}%
\pgfpathlineto{\pgfqpoint{3.119678in}{1.805784in}}%
\pgfpathlineto{\pgfqpoint{3.127766in}{1.812018in}}%
\pgfpathlineto{\pgfqpoint{3.135845in}{1.818378in}}%
\pgfpathlineto{\pgfqpoint{3.143915in}{1.824860in}}%
\pgfpathlineto{\pgfqpoint{3.151976in}{1.831460in}}%
\pgfpathlineto{\pgfqpoint{3.138798in}{1.839188in}}%
\pgfpathlineto{\pgfqpoint{3.125621in}{1.847114in}}%
\pgfpathlineto{\pgfqpoint{3.112443in}{1.855240in}}%
\pgfpathlineto{\pgfqpoint{3.099265in}{1.863566in}}%
\pgfpathlineto{\pgfqpoint{3.091181in}{1.857342in}}%
\pgfpathlineto{\pgfqpoint{3.083087in}{1.851243in}}%
\pgfpathlineto{\pgfqpoint{3.074983in}{1.845273in}}%
\pgfpathlineto{\pgfqpoint{3.066871in}{1.839435in}}%
\pgfpathclose%
\pgfusepath{fill}%
\end{pgfscope}%
\begin{pgfscope}%
\pgfpathrectangle{\pgfqpoint{1.254980in}{0.150000in}}{\pgfqpoint{5.490039in}{5.490039in}}%
\pgfusepath{clip}%
\pgfsetbuttcap%
\pgfsetroundjoin%
\definecolor{currentfill}{rgb}{0.283229,0.120777,0.440584}%
\pgfsetfillcolor{currentfill}%
\pgfsetfillopacity{0.700000}%
\pgfsetlinewidth{0.000000pt}%
\definecolor{currentstroke}{rgb}{0.000000,0.000000,0.000000}%
\pgfsetstrokecolor{currentstroke}%
\pgfsetdash{}{0pt}%
\pgfpathmoveto{\pgfqpoint{2.822409in}{1.991875in}}%
\pgfpathlineto{\pgfqpoint{2.835676in}{1.979358in}}%
\pgfpathlineto{\pgfqpoint{2.848939in}{1.967064in}}%
\pgfpathlineto{\pgfqpoint{2.862199in}{1.954991in}}%
\pgfpathlineto{\pgfqpoint{2.875454in}{1.943137in}}%
\pgfpathlineto{\pgfqpoint{2.883693in}{1.947121in}}%
\pgfpathlineto{\pgfqpoint{2.891921in}{1.951275in}}%
\pgfpathlineto{\pgfqpoint{2.900137in}{1.955595in}}%
\pgfpathlineto{\pgfqpoint{2.908342in}{1.960077in}}%
\pgfpathlineto{\pgfqpoint{2.895117in}{1.971512in}}%
\pgfpathlineto{\pgfqpoint{2.881888in}{1.983165in}}%
\pgfpathlineto{\pgfqpoint{2.868657in}{1.995038in}}%
\pgfpathlineto{\pgfqpoint{2.855421in}{2.007133in}}%
\pgfpathlineto{\pgfqpoint{2.847186in}{2.003061in}}%
\pgfpathlineto{\pgfqpoint{2.838939in}{1.999157in}}%
\pgfpathlineto{\pgfqpoint{2.830680in}{1.995427in}}%
\pgfpathlineto{\pgfqpoint{2.822409in}{1.991875in}}%
\pgfpathclose%
\pgfusepath{fill}%
\end{pgfscope}%
\begin{pgfscope}%
\pgfpathrectangle{\pgfqpoint{1.254980in}{0.150000in}}{\pgfqpoint{5.490039in}{5.490039in}}%
\pgfusepath{clip}%
\pgfsetbuttcap%
\pgfsetroundjoin%
\definecolor{currentfill}{rgb}{0.277134,0.185228,0.489898}%
\pgfsetfillcolor{currentfill}%
\pgfsetfillopacity{0.700000}%
\pgfsetlinewidth{0.000000pt}%
\definecolor{currentstroke}{rgb}{0.000000,0.000000,0.000000}%
\pgfsetstrokecolor{currentstroke}%
\pgfsetdash{}{0pt}%
\pgfpathmoveto{\pgfqpoint{4.068207in}{2.074264in}}%
\pgfpathlineto{\pgfqpoint{4.081536in}{2.077481in}}%
\pgfpathlineto{\pgfqpoint{4.094875in}{2.080865in}}%
\pgfpathlineto{\pgfqpoint{4.108224in}{2.084413in}}%
\pgfpathlineto{\pgfqpoint{4.121583in}{2.088127in}}%
\pgfpathlineto{\pgfqpoint{4.129270in}{2.098754in}}%
\pgfpathlineto{\pgfqpoint{4.136953in}{2.109328in}}%
\pgfpathlineto{\pgfqpoint{4.144631in}{2.119846in}}%
\pgfpathlineto{\pgfqpoint{4.152304in}{2.130310in}}%
\pgfpathlineto{\pgfqpoint{4.138951in}{2.126491in}}%
\pgfpathlineto{\pgfqpoint{4.125607in}{2.122837in}}%
\pgfpathlineto{\pgfqpoint{4.112274in}{2.119349in}}%
\pgfpathlineto{\pgfqpoint{4.098951in}{2.116027in}}%
\pgfpathlineto{\pgfqpoint{4.091272in}{2.105658in}}%
\pgfpathlineto{\pgfqpoint{4.083589in}{2.095241in}}%
\pgfpathlineto{\pgfqpoint{4.075901in}{2.084776in}}%
\pgfpathlineto{\pgfqpoint{4.068207in}{2.074264in}}%
\pgfpathclose%
\pgfusepath{fill}%
\end{pgfscope}%
\begin{pgfscope}%
\pgfpathrectangle{\pgfqpoint{1.254980in}{0.150000in}}{\pgfqpoint{5.490039in}{5.490039in}}%
\pgfusepath{clip}%
\pgfsetbuttcap%
\pgfsetroundjoin%
\definecolor{currentfill}{rgb}{0.120565,0.596422,0.543611}%
\pgfsetfillcolor{currentfill}%
\pgfsetfillopacity{0.700000}%
\pgfsetlinewidth{0.000000pt}%
\definecolor{currentstroke}{rgb}{0.000000,0.000000,0.000000}%
\pgfsetstrokecolor{currentstroke}%
\pgfsetdash{}{0pt}%
\pgfpathmoveto{\pgfqpoint{5.421862in}{3.074908in}}%
\pgfpathlineto{\pgfqpoint{5.435868in}{3.085184in}}%
\pgfpathlineto{\pgfqpoint{5.449891in}{3.095615in}}%
\pgfpathlineto{\pgfqpoint{5.463933in}{3.106204in}}%
\pgfpathlineto{\pgfqpoint{5.477993in}{3.116949in}}%
\pgfpathlineto{\pgfqpoint{5.485063in}{3.120277in}}%
\pgfpathlineto{\pgfqpoint{5.492126in}{3.123584in}}%
\pgfpathlineto{\pgfqpoint{5.499183in}{3.126877in}}%
\pgfpathlineto{\pgfqpoint{5.506233in}{3.130158in}}%
\pgfpathlineto{\pgfqpoint{5.492196in}{3.119840in}}%
\pgfpathlineto{\pgfqpoint{5.478176in}{3.109677in}}%
\pgfpathlineto{\pgfqpoint{5.464175in}{3.099670in}}%
\pgfpathlineto{\pgfqpoint{5.450191in}{3.089819in}}%
\pgfpathlineto{\pgfqpoint{5.443118in}{3.086102in}}%
\pgfpathlineto{\pgfqpoint{5.436039in}{3.082381in}}%
\pgfpathlineto{\pgfqpoint{5.428954in}{3.078651in}}%
\pgfpathlineto{\pgfqpoint{5.421862in}{3.074908in}}%
\pgfpathclose%
\pgfusepath{fill}%
\end{pgfscope}%
\begin{pgfscope}%
\pgfpathrectangle{\pgfqpoint{1.254980in}{0.150000in}}{\pgfqpoint{5.490039in}{5.490039in}}%
\pgfusepath{clip}%
\pgfsetbuttcap%
\pgfsetroundjoin%
\definecolor{currentfill}{rgb}{0.169646,0.456262,0.558030}%
\pgfsetfillcolor{currentfill}%
\pgfsetfillopacity{0.700000}%
\pgfsetlinewidth{0.000000pt}%
\definecolor{currentstroke}{rgb}{0.000000,0.000000,0.000000}%
\pgfsetstrokecolor{currentstroke}%
\pgfsetdash{}{0pt}%
\pgfpathmoveto{\pgfqpoint{4.886540in}{2.696940in}}%
\pgfpathlineto{\pgfqpoint{4.900255in}{2.705566in}}%
\pgfpathlineto{\pgfqpoint{4.913986in}{2.714351in}}%
\pgfpathlineto{\pgfqpoint{4.927732in}{2.723295in}}%
\pgfpathlineto{\pgfqpoint{4.941493in}{2.732398in}}%
\pgfpathlineto{\pgfqpoint{4.948862in}{2.739170in}}%
\pgfpathlineto{\pgfqpoint{4.956223in}{2.745863in}}%
\pgfpathlineto{\pgfqpoint{4.963577in}{2.752480in}}%
\pgfpathlineto{\pgfqpoint{4.970925in}{2.759023in}}%
\pgfpathlineto{\pgfqpoint{4.957174in}{2.750136in}}%
\pgfpathlineto{\pgfqpoint{4.943439in}{2.741407in}}%
\pgfpathlineto{\pgfqpoint{4.929720in}{2.732837in}}%
\pgfpathlineto{\pgfqpoint{4.916016in}{2.724426in}}%
\pgfpathlineto{\pgfqpoint{4.908657in}{2.717657in}}%
\pgfpathlineto{\pgfqpoint{4.901291in}{2.710822in}}%
\pgfpathlineto{\pgfqpoint{4.893919in}{2.703917in}}%
\pgfpathlineto{\pgfqpoint{4.886540in}{2.696940in}}%
\pgfpathclose%
\pgfusepath{fill}%
\end{pgfscope}%
\begin{pgfscope}%
\pgfpathrectangle{\pgfqpoint{1.254980in}{0.150000in}}{\pgfqpoint{5.490039in}{5.490039in}}%
\pgfusepath{clip}%
\pgfsetbuttcap%
\pgfsetroundjoin%
\definecolor{currentfill}{rgb}{0.220057,0.343307,0.549413}%
\pgfsetfillcolor{currentfill}%
\pgfsetfillopacity{0.700000}%
\pgfsetlinewidth{0.000000pt}%
\definecolor{currentstroke}{rgb}{0.000000,0.000000,0.000000}%
\pgfsetstrokecolor{currentstroke}%
\pgfsetdash{}{0pt}%
\pgfpathmoveto{\pgfqpoint{4.519439in}{2.412640in}}%
\pgfpathlineto{\pgfqpoint{4.532966in}{2.419331in}}%
\pgfpathlineto{\pgfqpoint{4.546506in}{2.426183in}}%
\pgfpathlineto{\pgfqpoint{4.560060in}{2.433196in}}%
\pgfpathlineto{\pgfqpoint{4.573627in}{2.440371in}}%
\pgfpathlineto{\pgfqpoint{4.581159in}{2.449384in}}%
\pgfpathlineto{\pgfqpoint{4.588685in}{2.458311in}}%
\pgfpathlineto{\pgfqpoint{4.596205in}{2.467153in}}%
\pgfpathlineto{\pgfqpoint{4.603719in}{2.475911in}}%
\pgfpathlineto{\pgfqpoint{4.590158in}{2.468803in}}%
\pgfpathlineto{\pgfqpoint{4.576611in}{2.461857in}}%
\pgfpathlineto{\pgfqpoint{4.563077in}{2.455072in}}%
\pgfpathlineto{\pgfqpoint{4.549557in}{2.448449in}}%
\pgfpathlineto{\pgfqpoint{4.542036in}{2.439613in}}%
\pgfpathlineto{\pgfqpoint{4.534510in}{2.430700in}}%
\pgfpathlineto{\pgfqpoint{4.526977in}{2.421709in}}%
\pgfpathlineto{\pgfqpoint{4.519439in}{2.412640in}}%
\pgfpathclose%
\pgfusepath{fill}%
\end{pgfscope}%
\begin{pgfscope}%
\pgfpathrectangle{\pgfqpoint{1.254980in}{0.150000in}}{\pgfqpoint{5.490039in}{5.490039in}}%
\pgfusepath{clip}%
\pgfsetbuttcap%
\pgfsetroundjoin%
\definecolor{currentfill}{rgb}{0.270595,0.214069,0.507052}%
\pgfsetfillcolor{currentfill}%
\pgfsetfillopacity{0.700000}%
\pgfsetlinewidth{0.000000pt}%
\definecolor{currentstroke}{rgb}{0.000000,0.000000,0.000000}%
\pgfsetstrokecolor{currentstroke}%
\pgfsetdash{}{0pt}%
\pgfpathmoveto{\pgfqpoint{4.152304in}{2.130310in}}%
\pgfpathlineto{\pgfqpoint{4.165668in}{2.134294in}}%
\pgfpathlineto{\pgfqpoint{4.179043in}{2.138442in}}%
\pgfpathlineto{\pgfqpoint{4.192428in}{2.142756in}}%
\pgfpathlineto{\pgfqpoint{4.205825in}{2.147233in}}%
\pgfpathlineto{\pgfqpoint{4.213487in}{2.157728in}}%
\pgfpathlineto{\pgfqpoint{4.221145in}{2.168160in}}%
\pgfpathlineto{\pgfqpoint{4.228798in}{2.178528in}}%
\pgfpathlineto{\pgfqpoint{4.236446in}{2.188832in}}%
\pgfpathlineto{\pgfqpoint{4.223054in}{2.184278in}}%
\pgfpathlineto{\pgfqpoint{4.209674in}{2.179889in}}%
\pgfpathlineto{\pgfqpoint{4.196305in}{2.175663in}}%
\pgfpathlineto{\pgfqpoint{4.182946in}{2.171603in}}%
\pgfpathlineto{\pgfqpoint{4.175293in}{2.161365in}}%
\pgfpathlineto{\pgfqpoint{4.167635in}{2.151069in}}%
\pgfpathlineto{\pgfqpoint{4.159972in}{2.140718in}}%
\pgfpathlineto{\pgfqpoint{4.152304in}{2.130310in}}%
\pgfpathclose%
\pgfusepath{fill}%
\end{pgfscope}%
\begin{pgfscope}%
\pgfpathrectangle{\pgfqpoint{1.254980in}{0.150000in}}{\pgfqpoint{5.490039in}{5.490039in}}%
\pgfusepath{clip}%
\pgfsetbuttcap%
\pgfsetroundjoin%
\definecolor{currentfill}{rgb}{0.212395,0.359683,0.551710}%
\pgfsetfillcolor{currentfill}%
\pgfsetfillopacity{0.700000}%
\pgfsetlinewidth{0.000000pt}%
\definecolor{currentstroke}{rgb}{0.000000,0.000000,0.000000}%
\pgfsetstrokecolor{currentstroke}%
\pgfsetdash{}{0pt}%
\pgfpathmoveto{\pgfqpoint{2.394414in}{2.520709in}}%
\pgfpathlineto{\pgfqpoint{2.407933in}{2.500067in}}%
\pgfpathlineto{\pgfqpoint{2.421439in}{2.479715in}}%
\pgfpathlineto{\pgfqpoint{2.434933in}{2.459650in}}%
\pgfpathlineto{\pgfqpoint{2.448416in}{2.439869in}}%
\pgfpathlineto{\pgfqpoint{2.456953in}{2.440332in}}%
\pgfpathlineto{\pgfqpoint{2.465474in}{2.441024in}}%
\pgfpathlineto{\pgfqpoint{2.473978in}{2.441944in}}%
\pgfpathlineto{\pgfqpoint{2.482468in}{2.443085in}}%
\pgfpathlineto{\pgfqpoint{2.469028in}{2.462424in}}%
\pgfpathlineto{\pgfqpoint{2.455576in}{2.482047in}}%
\pgfpathlineto{\pgfqpoint{2.442114in}{2.501956in}}%
\pgfpathlineto{\pgfqpoint{2.428640in}{2.522153in}}%
\pgfpathlineto{\pgfqpoint{2.420108in}{2.521443in}}%
\pgfpathlineto{\pgfqpoint{2.411560in}{2.520962in}}%
\pgfpathlineto{\pgfqpoint{2.402996in}{2.520716in}}%
\pgfpathlineto{\pgfqpoint{2.394414in}{2.520709in}}%
\pgfpathclose%
\pgfusepath{fill}%
\end{pgfscope}%
\begin{pgfscope}%
\pgfpathrectangle{\pgfqpoint{1.254980in}{0.150000in}}{\pgfqpoint{5.490039in}{5.490039in}}%
\pgfusepath{clip}%
\pgfsetbuttcap%
\pgfsetroundjoin%
\definecolor{currentfill}{rgb}{0.272594,0.025563,0.353093}%
\pgfsetfillcolor{currentfill}%
\pgfsetfillopacity{0.700000}%
\pgfsetlinewidth{0.000000pt}%
\definecolor{currentstroke}{rgb}{0.000000,0.000000,0.000000}%
\pgfsetstrokecolor{currentstroke}%
\pgfsetdash{}{0pt}%
\pgfpathmoveto{\pgfqpoint{3.479310in}{1.787316in}}%
\pgfpathlineto{\pgfqpoint{3.492502in}{1.784213in}}%
\pgfpathlineto{\pgfqpoint{3.505697in}{1.781288in}}%
\pgfpathlineto{\pgfqpoint{3.518898in}{1.778542in}}%
\pgfpathlineto{\pgfqpoint{3.532103in}{1.775973in}}%
\pgfpathlineto{\pgfqpoint{3.539993in}{1.785348in}}%
\pgfpathlineto{\pgfqpoint{3.547877in}{1.794768in}}%
\pgfpathlineto{\pgfqpoint{3.555755in}{1.804229in}}%
\pgfpathlineto{\pgfqpoint{3.563628in}{1.813729in}}%
\pgfpathlineto{\pgfqpoint{3.550436in}{1.815998in}}%
\pgfpathlineto{\pgfqpoint{3.537249in}{1.818445in}}%
\pgfpathlineto{\pgfqpoint{3.524067in}{1.821070in}}%
\pgfpathlineto{\pgfqpoint{3.510889in}{1.823873in}}%
\pgfpathlineto{\pgfqpoint{3.503004in}{1.814662in}}%
\pgfpathlineto{\pgfqpoint{3.495112in}{1.805497in}}%
\pgfpathlineto{\pgfqpoint{3.487214in}{1.796381in}}%
\pgfpathlineto{\pgfqpoint{3.479310in}{1.787316in}}%
\pgfpathclose%
\pgfusepath{fill}%
\end{pgfscope}%
\begin{pgfscope}%
\pgfpathrectangle{\pgfqpoint{1.254980in}{0.150000in}}{\pgfqpoint{5.490039in}{5.490039in}}%
\pgfusepath{clip}%
\pgfsetbuttcap%
\pgfsetroundjoin%
\definecolor{currentfill}{rgb}{0.119699,0.618490,0.536347}%
\pgfsetfillcolor{currentfill}%
\pgfsetfillopacity{0.700000}%
\pgfsetlinewidth{0.000000pt}%
\definecolor{currentstroke}{rgb}{0.000000,0.000000,0.000000}%
\pgfsetstrokecolor{currentstroke}%
\pgfsetdash{}{0pt}%
\pgfpathmoveto{\pgfqpoint{5.506233in}{3.130158in}}%
\pgfpathlineto{\pgfqpoint{5.520288in}{3.140632in}}%
\pgfpathlineto{\pgfqpoint{5.534362in}{3.151263in}}%
\pgfpathlineto{\pgfqpoint{5.548454in}{3.162049in}}%
\pgfpathlineto{\pgfqpoint{5.562564in}{3.172992in}}%
\pgfpathlineto{\pgfqpoint{5.569584in}{3.175822in}}%
\pgfpathlineto{\pgfqpoint{5.576597in}{3.178644in}}%
\pgfpathlineto{\pgfqpoint{5.583603in}{3.181464in}}%
\pgfpathlineto{\pgfqpoint{5.590604in}{3.184286in}}%
\pgfpathlineto{\pgfqpoint{5.576518in}{3.173800in}}%
\pgfpathlineto{\pgfqpoint{5.562450in}{3.163470in}}%
\pgfpathlineto{\pgfqpoint{5.548401in}{3.153295in}}%
\pgfpathlineto{\pgfqpoint{5.534370in}{3.143275in}}%
\pgfpathlineto{\pgfqpoint{5.527345in}{3.139987in}}%
\pgfpathlineto{\pgfqpoint{5.520314in}{3.136708in}}%
\pgfpathlineto{\pgfqpoint{5.513276in}{3.133434in}}%
\pgfpathlineto{\pgfqpoint{5.506233in}{3.130158in}}%
\pgfpathclose%
\pgfusepath{fill}%
\end{pgfscope}%
\begin{pgfscope}%
\pgfpathrectangle{\pgfqpoint{1.254980in}{0.150000in}}{\pgfqpoint{5.490039in}{5.490039in}}%
\pgfusepath{clip}%
\pgfsetbuttcap%
\pgfsetroundjoin%
\definecolor{currentfill}{rgb}{0.282656,0.100196,0.422160}%
\pgfsetfillcolor{currentfill}%
\pgfsetfillopacity{0.700000}%
\pgfsetlinewidth{0.000000pt}%
\definecolor{currentstroke}{rgb}{0.000000,0.000000,0.000000}%
\pgfsetstrokecolor{currentstroke}%
\pgfsetdash{}{0pt}%
\pgfpathmoveto{\pgfqpoint{2.875454in}{1.943137in}}%
\pgfpathlineto{\pgfqpoint{2.888707in}{1.931501in}}%
\pgfpathlineto{\pgfqpoint{2.901956in}{1.920081in}}%
\pgfpathlineto{\pgfqpoint{2.915203in}{1.908876in}}%
\pgfpathlineto{\pgfqpoint{2.928446in}{1.897885in}}%
\pgfpathlineto{\pgfqpoint{2.936655in}{1.902299in}}%
\pgfpathlineto{\pgfqpoint{2.944852in}{1.906876in}}%
\pgfpathlineto{\pgfqpoint{2.953039in}{1.911611in}}%
\pgfpathlineto{\pgfqpoint{2.961215in}{1.916501in}}%
\pgfpathlineto{\pgfqpoint{2.948000in}{1.927074in}}%
\pgfpathlineto{\pgfqpoint{2.934784in}{1.937860in}}%
\pgfpathlineto{\pgfqpoint{2.921564in}{1.948861in}}%
\pgfpathlineto{\pgfqpoint{2.908342in}{1.960077in}}%
\pgfpathlineto{\pgfqpoint{2.900137in}{1.955595in}}%
\pgfpathlineto{\pgfqpoint{2.891921in}{1.951275in}}%
\pgfpathlineto{\pgfqpoint{2.883693in}{1.947121in}}%
\pgfpathlineto{\pgfqpoint{2.875454in}{1.943137in}}%
\pgfpathclose%
\pgfusepath{fill}%
\end{pgfscope}%
\begin{pgfscope}%
\pgfpathrectangle{\pgfqpoint{1.254980in}{0.150000in}}{\pgfqpoint{5.490039in}{5.490039in}}%
\pgfusepath{clip}%
\pgfsetbuttcap%
\pgfsetroundjoin%
\definecolor{currentfill}{rgb}{0.123444,0.636809,0.528763}%
\pgfsetfillcolor{currentfill}%
\pgfsetfillopacity{0.700000}%
\pgfsetlinewidth{0.000000pt}%
\definecolor{currentstroke}{rgb}{0.000000,0.000000,0.000000}%
\pgfsetstrokecolor{currentstroke}%
\pgfsetdash{}{0pt}%
\pgfpathmoveto{\pgfqpoint{5.590604in}{3.184286in}}%
\pgfpathlineto{\pgfqpoint{5.604708in}{3.194927in}}%
\pgfpathlineto{\pgfqpoint{5.618831in}{3.205724in}}%
\pgfpathlineto{\pgfqpoint{5.632972in}{3.216676in}}%
\pgfpathlineto{\pgfqpoint{5.647133in}{3.227785in}}%
\pgfpathlineto{\pgfqpoint{5.654101in}{3.230138in}}%
\pgfpathlineto{\pgfqpoint{5.661062in}{3.232497in}}%
\pgfpathlineto{\pgfqpoint{5.668018in}{3.234867in}}%
\pgfpathlineto{\pgfqpoint{5.674967in}{3.237254in}}%
\pgfpathlineto{\pgfqpoint{5.660834in}{3.226633in}}%
\pgfpathlineto{\pgfqpoint{5.646719in}{3.216167in}}%
\pgfpathlineto{\pgfqpoint{5.632622in}{3.205856in}}%
\pgfpathlineto{\pgfqpoint{5.618544in}{3.195699in}}%
\pgfpathlineto{\pgfqpoint{5.611568in}{3.192816in}}%
\pgfpathlineto{\pgfqpoint{5.604586in}{3.189957in}}%
\pgfpathlineto{\pgfqpoint{5.597598in}{3.187115in}}%
\pgfpathlineto{\pgfqpoint{5.590604in}{3.184286in}}%
\pgfpathclose%
\pgfusepath{fill}%
\end{pgfscope}%
\begin{pgfscope}%
\pgfpathrectangle{\pgfqpoint{1.254980in}{0.150000in}}{\pgfqpoint{5.490039in}{5.490039in}}%
\pgfusepath{clip}%
\pgfsetbuttcap%
\pgfsetroundjoin%
\definecolor{currentfill}{rgb}{0.269944,0.014625,0.341379}%
\pgfsetfillcolor{currentfill}%
\pgfsetfillopacity{0.700000}%
\pgfsetlinewidth{0.000000pt}%
\definecolor{currentstroke}{rgb}{0.000000,0.000000,0.000000}%
\pgfsetstrokecolor{currentstroke}%
\pgfsetdash{}{0pt}%
\pgfpathmoveto{\pgfqpoint{3.257429in}{1.776643in}}%
\pgfpathlineto{\pgfqpoint{3.270617in}{1.770652in}}%
\pgfpathlineto{\pgfqpoint{3.283807in}{1.764850in}}%
\pgfpathlineto{\pgfqpoint{3.296998in}{1.759235in}}%
\pgfpathlineto{\pgfqpoint{3.310193in}{1.753806in}}%
\pgfpathlineto{\pgfqpoint{3.318184in}{1.761621in}}%
\pgfpathlineto{\pgfqpoint{3.326168in}{1.769526in}}%
\pgfpathlineto{\pgfqpoint{3.334144in}{1.777518in}}%
\pgfpathlineto{\pgfqpoint{3.342114in}{1.785593in}}%
\pgfpathlineto{\pgfqpoint{3.328938in}{1.790666in}}%
\pgfpathlineto{\pgfqpoint{3.315765in}{1.795925in}}%
\pgfpathlineto{\pgfqpoint{3.302594in}{1.801372in}}%
\pgfpathlineto{\pgfqpoint{3.289425in}{1.807006in}}%
\pgfpathlineto{\pgfqpoint{3.281438in}{1.799277in}}%
\pgfpathlineto{\pgfqpoint{3.273443in}{1.791637in}}%
\pgfpathlineto{\pgfqpoint{3.265440in}{1.784092in}}%
\pgfpathlineto{\pgfqpoint{3.257429in}{1.776643in}}%
\pgfpathclose%
\pgfusepath{fill}%
\end{pgfscope}%
\begin{pgfscope}%
\pgfpathrectangle{\pgfqpoint{1.254980in}{0.150000in}}{\pgfqpoint{5.490039in}{5.490039in}}%
\pgfusepath{clip}%
\pgfsetbuttcap%
\pgfsetroundjoin%
\definecolor{currentfill}{rgb}{0.159194,0.482237,0.558073}%
\pgfsetfillcolor{currentfill}%
\pgfsetfillopacity{0.700000}%
\pgfsetlinewidth{0.000000pt}%
\definecolor{currentstroke}{rgb}{0.000000,0.000000,0.000000}%
\pgfsetstrokecolor{currentstroke}%
\pgfsetdash{}{0pt}%
\pgfpathmoveto{\pgfqpoint{4.970925in}{2.759023in}}%
\pgfpathlineto{\pgfqpoint{4.984691in}{2.768070in}}%
\pgfpathlineto{\pgfqpoint{4.998473in}{2.777275in}}%
\pgfpathlineto{\pgfqpoint{5.012271in}{2.786639in}}%
\pgfpathlineto{\pgfqpoint{5.026085in}{2.796162in}}%
\pgfpathlineto{\pgfqpoint{5.033414in}{2.802399in}}%
\pgfpathlineto{\pgfqpoint{5.040735in}{2.808562in}}%
\pgfpathlineto{\pgfqpoint{5.048049in}{2.814652in}}%
\pgfpathlineto{\pgfqpoint{5.055356in}{2.820672in}}%
\pgfpathlineto{\pgfqpoint{5.041555in}{2.811395in}}%
\pgfpathlineto{\pgfqpoint{5.027769in}{2.802277in}}%
\pgfpathlineto{\pgfqpoint{5.013999in}{2.793317in}}%
\pgfpathlineto{\pgfqpoint{5.000245in}{2.784515in}}%
\pgfpathlineto{\pgfqpoint{4.992925in}{2.778239in}}%
\pgfpathlineto{\pgfqpoint{4.985599in}{2.771900in}}%
\pgfpathlineto{\pgfqpoint{4.978265in}{2.765496in}}%
\pgfpathlineto{\pgfqpoint{4.970925in}{2.759023in}}%
\pgfpathclose%
\pgfusepath{fill}%
\end{pgfscope}%
\begin{pgfscope}%
\pgfpathrectangle{\pgfqpoint{1.254980in}{0.150000in}}{\pgfqpoint{5.490039in}{5.490039in}}%
\pgfusepath{clip}%
\pgfsetbuttcap%
\pgfsetroundjoin%
\definecolor{currentfill}{rgb}{0.260571,0.246922,0.522828}%
\pgfsetfillcolor{currentfill}%
\pgfsetfillopacity{0.700000}%
\pgfsetlinewidth{0.000000pt}%
\definecolor{currentstroke}{rgb}{0.000000,0.000000,0.000000}%
\pgfsetstrokecolor{currentstroke}%
\pgfsetdash{}{0pt}%
\pgfpathmoveto{\pgfqpoint{4.236446in}{2.188832in}}%
\pgfpathlineto{\pgfqpoint{4.249848in}{2.193551in}}%
\pgfpathlineto{\pgfqpoint{4.263261in}{2.198432in}}%
\pgfpathlineto{\pgfqpoint{4.276687in}{2.203478in}}%
\pgfpathlineto{\pgfqpoint{4.290123in}{2.208687in}}%
\pgfpathlineto{\pgfqpoint{4.297761in}{2.218986in}}%
\pgfpathlineto{\pgfqpoint{4.305393in}{2.229213in}}%
\pgfpathlineto{\pgfqpoint{4.313020in}{2.239368in}}%
\pgfpathlineto{\pgfqpoint{4.320642in}{2.249452in}}%
\pgfpathlineto{\pgfqpoint{4.307210in}{2.244195in}}%
\pgfpathlineto{\pgfqpoint{4.293790in}{2.239101in}}%
\pgfpathlineto{\pgfqpoint{4.280381in}{2.234171in}}%
\pgfpathlineto{\pgfqpoint{4.266984in}{2.229405in}}%
\pgfpathlineto{\pgfqpoint{4.259357in}{2.219359in}}%
\pgfpathlineto{\pgfqpoint{4.251725in}{2.209248in}}%
\pgfpathlineto{\pgfqpoint{4.244088in}{2.199072in}}%
\pgfpathlineto{\pgfqpoint{4.236446in}{2.188832in}}%
\pgfpathclose%
\pgfusepath{fill}%
\end{pgfscope}%
\begin{pgfscope}%
\pgfpathrectangle{\pgfqpoint{1.254980in}{0.150000in}}{\pgfqpoint{5.490039in}{5.490039in}}%
\pgfusepath{clip}%
\pgfsetbuttcap%
\pgfsetroundjoin%
\definecolor{currentfill}{rgb}{0.206756,0.371758,0.553117}%
\pgfsetfillcolor{currentfill}%
\pgfsetfillopacity{0.700000}%
\pgfsetlinewidth{0.000000pt}%
\definecolor{currentstroke}{rgb}{0.000000,0.000000,0.000000}%
\pgfsetstrokecolor{currentstroke}%
\pgfsetdash{}{0pt}%
\pgfpathmoveto{\pgfqpoint{4.603719in}{2.475911in}}%
\pgfpathlineto{\pgfqpoint{4.617293in}{2.483179in}}%
\pgfpathlineto{\pgfqpoint{4.630882in}{2.490609in}}%
\pgfpathlineto{\pgfqpoint{4.644484in}{2.498199in}}%
\pgfpathlineto{\pgfqpoint{4.658100in}{2.505950in}}%
\pgfpathlineto{\pgfqpoint{4.665601in}{2.514539in}}%
\pgfpathlineto{\pgfqpoint{4.673096in}{2.523040in}}%
\pgfpathlineto{\pgfqpoint{4.680584in}{2.531454in}}%
\pgfpathlineto{\pgfqpoint{4.688066in}{2.539782in}}%
\pgfpathlineto{\pgfqpoint{4.674457in}{2.532129in}}%
\pgfpathlineto{\pgfqpoint{4.660862in}{2.524635in}}%
\pgfpathlineto{\pgfqpoint{4.647281in}{2.517303in}}%
\pgfpathlineto{\pgfqpoint{4.633714in}{2.510131in}}%
\pgfpathlineto{\pgfqpoint{4.626224in}{2.501695in}}%
\pgfpathlineto{\pgfqpoint{4.618729in}{2.493181in}}%
\pgfpathlineto{\pgfqpoint{4.611227in}{2.484586in}}%
\pgfpathlineto{\pgfqpoint{4.603719in}{2.475911in}}%
\pgfpathclose%
\pgfusepath{fill}%
\end{pgfscope}%
\begin{pgfscope}%
\pgfpathrectangle{\pgfqpoint{1.254980in}{0.150000in}}{\pgfqpoint{5.490039in}{5.490039in}}%
\pgfusepath{clip}%
\pgfsetbuttcap%
\pgfsetroundjoin%
\definecolor{currentfill}{rgb}{0.273809,0.031497,0.358853}%
\pgfsetfillcolor{currentfill}%
\pgfsetfillopacity{0.700000}%
\pgfsetlinewidth{0.000000pt}%
\definecolor{currentstroke}{rgb}{0.000000,0.000000,0.000000}%
\pgfsetstrokecolor{currentstroke}%
\pgfsetdash{}{0pt}%
\pgfpathmoveto{\pgfqpoint{3.119678in}{1.805784in}}%
\pgfpathlineto{\pgfqpoint{3.132879in}{1.797867in}}%
\pgfpathlineto{\pgfqpoint{3.146081in}{1.790147in}}%
\pgfpathlineto{\pgfqpoint{3.159283in}{1.782622in}}%
\pgfpathlineto{\pgfqpoint{3.172486in}{1.775290in}}%
\pgfpathlineto{\pgfqpoint{3.180551in}{1.781921in}}%
\pgfpathlineto{\pgfqpoint{3.188607in}{1.788670in}}%
\pgfpathlineto{\pgfqpoint{3.196655in}{1.795533in}}%
\pgfpathlineto{\pgfqpoint{3.204694in}{1.802507in}}%
\pgfpathlineto{\pgfqpoint{3.191513in}{1.809453in}}%
\pgfpathlineto{\pgfqpoint{3.178334in}{1.816593in}}%
\pgfpathlineto{\pgfqpoint{3.165155in}{1.823929in}}%
\pgfpathlineto{\pgfqpoint{3.151976in}{1.831460in}}%
\pgfpathlineto{\pgfqpoint{3.143915in}{1.824860in}}%
\pgfpathlineto{\pgfqpoint{3.135845in}{1.818378in}}%
\pgfpathlineto{\pgfqpoint{3.127766in}{1.812018in}}%
\pgfpathlineto{\pgfqpoint{3.119678in}{1.805784in}}%
\pgfpathclose%
\pgfusepath{fill}%
\end{pgfscope}%
\begin{pgfscope}%
\pgfpathrectangle{\pgfqpoint{1.254980in}{0.150000in}}{\pgfqpoint{5.490039in}{5.490039in}}%
\pgfusepath{clip}%
\pgfsetbuttcap%
\pgfsetroundjoin%
\definecolor{currentfill}{rgb}{0.134692,0.658636,0.517649}%
\pgfsetfillcolor{currentfill}%
\pgfsetfillopacity{0.700000}%
\pgfsetlinewidth{0.000000pt}%
\definecolor{currentstroke}{rgb}{0.000000,0.000000,0.000000}%
\pgfsetstrokecolor{currentstroke}%
\pgfsetdash{}{0pt}%
\pgfpathmoveto{\pgfqpoint{5.674967in}{3.237254in}}%
\pgfpathlineto{\pgfqpoint{5.689120in}{3.248030in}}%
\pgfpathlineto{\pgfqpoint{5.703291in}{3.258961in}}%
\pgfpathlineto{\pgfqpoint{5.717481in}{3.270048in}}%
\pgfpathlineto{\pgfqpoint{5.731691in}{3.281290in}}%
\pgfpathlineto{\pgfqpoint{5.738606in}{3.283192in}}%
\pgfpathlineto{\pgfqpoint{5.745516in}{3.285115in}}%
\pgfpathlineto{\pgfqpoint{5.752420in}{3.287065in}}%
\pgfpathlineto{\pgfqpoint{5.759318in}{3.289046in}}%
\pgfpathlineto{\pgfqpoint{5.745138in}{3.278322in}}%
\pgfpathlineto{\pgfqpoint{5.730976in}{3.267752in}}%
\pgfpathlineto{\pgfqpoint{5.716834in}{3.257337in}}%
\pgfpathlineto{\pgfqpoint{5.702710in}{3.247076in}}%
\pgfpathlineto{\pgfqpoint{5.695782in}{3.244568in}}%
\pgfpathlineto{\pgfqpoint{5.688849in}{3.242099in}}%
\pgfpathlineto{\pgfqpoint{5.681911in}{3.239662in}}%
\pgfpathlineto{\pgfqpoint{5.674967in}{3.237254in}}%
\pgfpathclose%
\pgfusepath{fill}%
\end{pgfscope}%
\begin{pgfscope}%
\pgfpathrectangle{\pgfqpoint{1.254980in}{0.150000in}}{\pgfqpoint{5.490039in}{5.490039in}}%
\pgfusepath{clip}%
\pgfsetbuttcap%
\pgfsetroundjoin%
\definecolor{currentfill}{rgb}{0.195860,0.395433,0.555276}%
\pgfsetfillcolor{currentfill}%
\pgfsetfillopacity{0.700000}%
\pgfsetlinewidth{0.000000pt}%
\definecolor{currentstroke}{rgb}{0.000000,0.000000,0.000000}%
\pgfsetstrokecolor{currentstroke}%
\pgfsetdash{}{0pt}%
\pgfpathmoveto{\pgfqpoint{2.340214in}{2.606230in}}%
\pgfpathlineto{\pgfqpoint{2.353784in}{2.584400in}}%
\pgfpathlineto{\pgfqpoint{2.367340in}{2.562873in}}%
\pgfpathlineto{\pgfqpoint{2.380884in}{2.541643in}}%
\pgfpathlineto{\pgfqpoint{2.394414in}{2.520709in}}%
\pgfpathlineto{\pgfqpoint{2.402996in}{2.520716in}}%
\pgfpathlineto{\pgfqpoint{2.411560in}{2.520962in}}%
\pgfpathlineto{\pgfqpoint{2.420108in}{2.521443in}}%
\pgfpathlineto{\pgfqpoint{2.428640in}{2.522153in}}%
\pgfpathlineto{\pgfqpoint{2.415154in}{2.542642in}}%
\pgfpathlineto{\pgfqpoint{2.401656in}{2.563426in}}%
\pgfpathlineto{\pgfqpoint{2.388145in}{2.584506in}}%
\pgfpathlineto{\pgfqpoint{2.374622in}{2.605887in}}%
\pgfpathlineto{\pgfqpoint{2.366045in}{2.605611in}}%
\pgfpathlineto{\pgfqpoint{2.357452in}{2.605574in}}%
\pgfpathlineto{\pgfqpoint{2.348842in}{2.605778in}}%
\pgfpathlineto{\pgfqpoint{2.340214in}{2.606230in}}%
\pgfpathclose%
\pgfusepath{fill}%
\end{pgfscope}%
\begin{pgfscope}%
\pgfpathrectangle{\pgfqpoint{1.254980in}{0.150000in}}{\pgfqpoint{5.490039in}{5.490039in}}%
\pgfusepath{clip}%
\pgfsetbuttcap%
\pgfsetroundjoin%
\definecolor{currentfill}{rgb}{0.281446,0.084320,0.407414}%
\pgfsetfillcolor{currentfill}%
\pgfsetfillopacity{0.700000}%
\pgfsetlinewidth{0.000000pt}%
\definecolor{currentstroke}{rgb}{0.000000,0.000000,0.000000}%
\pgfsetstrokecolor{currentstroke}%
\pgfsetdash{}{0pt}%
\pgfpathmoveto{\pgfqpoint{2.928446in}{1.897885in}}%
\pgfpathlineto{\pgfqpoint{2.941688in}{1.887107in}}%
\pgfpathlineto{\pgfqpoint{2.954927in}{1.876539in}}%
\pgfpathlineto{\pgfqpoint{2.968163in}{1.866180in}}%
\pgfpathlineto{\pgfqpoint{2.981398in}{1.856030in}}%
\pgfpathlineto{\pgfqpoint{2.989578in}{1.860872in}}%
\pgfpathlineto{\pgfqpoint{2.997747in}{1.865870in}}%
\pgfpathlineto{\pgfqpoint{3.005906in}{1.871019in}}%
\pgfpathlineto{\pgfqpoint{3.014054in}{1.876315in}}%
\pgfpathlineto{\pgfqpoint{3.000847in}{1.886048in}}%
\pgfpathlineto{\pgfqpoint{2.987638in}{1.895990in}}%
\pgfpathlineto{\pgfqpoint{2.974428in}{1.906140in}}%
\pgfpathlineto{\pgfqpoint{2.961215in}{1.916501in}}%
\pgfpathlineto{\pgfqpoint{2.953039in}{1.911611in}}%
\pgfpathlineto{\pgfqpoint{2.944852in}{1.906876in}}%
\pgfpathlineto{\pgfqpoint{2.936655in}{1.902299in}}%
\pgfpathlineto{\pgfqpoint{2.928446in}{1.897885in}}%
\pgfpathclose%
\pgfusepath{fill}%
\end{pgfscope}%
\begin{pgfscope}%
\pgfpathrectangle{\pgfqpoint{1.254980in}{0.150000in}}{\pgfqpoint{5.490039in}{5.490039in}}%
\pgfusepath{clip}%
\pgfsetbuttcap%
\pgfsetroundjoin%
\definecolor{currentfill}{rgb}{0.226397,0.728888,0.462789}%
\pgfsetfillcolor{currentfill}%
\pgfsetfillopacity{0.700000}%
\pgfsetlinewidth{0.000000pt}%
\definecolor{currentstroke}{rgb}{0.000000,0.000000,0.000000}%
\pgfsetstrokecolor{currentstroke}%
\pgfsetdash{}{0pt}%
\pgfpathmoveto{\pgfqpoint{6.012241in}{3.437519in}}%
\pgfpathlineto{\pgfqpoint{6.026574in}{3.448516in}}%
\pgfpathlineto{\pgfqpoint{6.040927in}{3.459667in}}%
\pgfpathlineto{\pgfqpoint{6.055300in}{3.470972in}}%
\pgfpathlineto{\pgfqpoint{6.062014in}{3.471591in}}%
\pgfpathlineto{\pgfqpoint{6.068725in}{3.472304in}}%
\pgfpathlineto{\pgfqpoint{6.075432in}{3.473118in}}%
\pgfpathlineto{\pgfqpoint{6.082138in}{3.474041in}}%
\pgfpathlineto{\pgfqpoint{6.067803in}{3.463374in}}%
\pgfpathlineto{\pgfqpoint{6.053488in}{3.452858in}}%
\pgfpathlineto{\pgfqpoint{6.039193in}{3.442496in}}%
\pgfpathlineto{\pgfqpoint{6.032459in}{3.441089in}}%
\pgfpathlineto{\pgfqpoint{6.025722in}{3.439795in}}%
\pgfpathlineto{\pgfqpoint{6.018983in}{3.438607in}}%
\pgfpathlineto{\pgfqpoint{6.012241in}{3.437519in}}%
\pgfpathclose%
\pgfusepath{fill}%
\end{pgfscope}%
\begin{pgfscope}%
\pgfpathrectangle{\pgfqpoint{1.254980in}{0.150000in}}{\pgfqpoint{5.490039in}{5.490039in}}%
\pgfusepath{clip}%
\pgfsetbuttcap%
\pgfsetroundjoin%
\definecolor{currentfill}{rgb}{0.271305,0.019942,0.347269}%
\pgfsetfillcolor{currentfill}%
\pgfsetfillopacity{0.700000}%
\pgfsetlinewidth{0.000000pt}%
\definecolor{currentstroke}{rgb}{0.000000,0.000000,0.000000}%
\pgfsetstrokecolor{currentstroke}%
\pgfsetdash{}{0pt}%
\pgfpathmoveto{\pgfqpoint{3.394846in}{1.767150in}}%
\pgfpathlineto{\pgfqpoint{3.408037in}{1.762997in}}%
\pgfpathlineto{\pgfqpoint{3.421231in}{1.759026in}}%
\pgfpathlineto{\pgfqpoint{3.434429in}{1.755236in}}%
\pgfpathlineto{\pgfqpoint{3.447631in}{1.751626in}}%
\pgfpathlineto{\pgfqpoint{3.455561in}{1.760457in}}%
\pgfpathlineto{\pgfqpoint{3.463484in}{1.769351in}}%
\pgfpathlineto{\pgfqpoint{3.471400in}{1.778305in}}%
\pgfpathlineto{\pgfqpoint{3.479310in}{1.787316in}}%
\pgfpathlineto{\pgfqpoint{3.466123in}{1.790598in}}%
\pgfpathlineto{\pgfqpoint{3.452940in}{1.794061in}}%
\pgfpathlineto{\pgfqpoint{3.439762in}{1.797705in}}%
\pgfpathlineto{\pgfqpoint{3.426586in}{1.801530in}}%
\pgfpathlineto{\pgfqpoint{3.418661in}{1.792836in}}%
\pgfpathlineto{\pgfqpoint{3.410729in}{1.784206in}}%
\pgfpathlineto{\pgfqpoint{3.402791in}{1.775643in}}%
\pgfpathlineto{\pgfqpoint{3.394846in}{1.767150in}}%
\pgfpathclose%
\pgfusepath{fill}%
\end{pgfscope}%
\begin{pgfscope}%
\pgfpathrectangle{\pgfqpoint{1.254980in}{0.150000in}}{\pgfqpoint{5.490039in}{5.490039in}}%
\pgfusepath{clip}%
\pgfsetbuttcap%
\pgfsetroundjoin%
\definecolor{currentfill}{rgb}{0.150148,0.676631,0.506589}%
\pgfsetfillcolor{currentfill}%
\pgfsetfillopacity{0.700000}%
\pgfsetlinewidth{0.000000pt}%
\definecolor{currentstroke}{rgb}{0.000000,0.000000,0.000000}%
\pgfsetstrokecolor{currentstroke}%
\pgfsetdash{}{0pt}%
\pgfpathmoveto{\pgfqpoint{5.759318in}{3.289046in}}%
\pgfpathlineto{\pgfqpoint{5.773518in}{3.299925in}}%
\pgfpathlineto{\pgfqpoint{5.787736in}{3.310959in}}%
\pgfpathlineto{\pgfqpoint{5.801974in}{3.322148in}}%
\pgfpathlineto{\pgfqpoint{5.816232in}{3.333493in}}%
\pgfpathlineto{\pgfqpoint{5.823094in}{3.334975in}}%
\pgfpathlineto{\pgfqpoint{5.829951in}{3.336494in}}%
\pgfpathlineto{\pgfqpoint{5.836803in}{3.338057in}}%
\pgfpathlineto{\pgfqpoint{5.843650in}{3.339668in}}%
\pgfpathlineto{\pgfqpoint{5.829424in}{3.328872in}}%
\pgfpathlineto{\pgfqpoint{5.815218in}{3.318230in}}%
\pgfpathlineto{\pgfqpoint{5.801030in}{3.307742in}}%
\pgfpathlineto{\pgfqpoint{5.786861in}{3.297408in}}%
\pgfpathlineto{\pgfqpoint{5.779983in}{3.295241in}}%
\pgfpathlineto{\pgfqpoint{5.773099in}{3.293128in}}%
\pgfpathlineto{\pgfqpoint{5.766211in}{3.291065in}}%
\pgfpathlineto{\pgfqpoint{5.759318in}{3.289046in}}%
\pgfpathclose%
\pgfusepath{fill}%
\end{pgfscope}%
\begin{pgfscope}%
\pgfpathrectangle{\pgfqpoint{1.254980in}{0.150000in}}{\pgfqpoint{5.490039in}{5.490039in}}%
\pgfusepath{clip}%
\pgfsetbuttcap%
\pgfsetroundjoin%
\definecolor{currentfill}{rgb}{0.149039,0.508051,0.557250}%
\pgfsetfillcolor{currentfill}%
\pgfsetfillopacity{0.700000}%
\pgfsetlinewidth{0.000000pt}%
\definecolor{currentstroke}{rgb}{0.000000,0.000000,0.000000}%
\pgfsetstrokecolor{currentstroke}%
\pgfsetdash{}{0pt}%
\pgfpathmoveto{\pgfqpoint{5.055356in}{2.820672in}}%
\pgfpathlineto{\pgfqpoint{5.069174in}{2.830108in}}%
\pgfpathlineto{\pgfqpoint{5.083009in}{2.839702in}}%
\pgfpathlineto{\pgfqpoint{5.096859in}{2.849454in}}%
\pgfpathlineto{\pgfqpoint{5.110727in}{2.859365in}}%
\pgfpathlineto{\pgfqpoint{5.118013in}{2.865054in}}%
\pgfpathlineto{\pgfqpoint{5.125292in}{2.870673in}}%
\pgfpathlineto{\pgfqpoint{5.132564in}{2.876225in}}%
\pgfpathlineto{\pgfqpoint{5.139829in}{2.881712in}}%
\pgfpathlineto{\pgfqpoint{5.125976in}{2.872078in}}%
\pgfpathlineto{\pgfqpoint{5.112139in}{2.862602in}}%
\pgfpathlineto{\pgfqpoint{5.098318in}{2.853283in}}%
\pgfpathlineto{\pgfqpoint{5.084514in}{2.844123in}}%
\pgfpathlineto{\pgfqpoint{5.077235in}{2.838349in}}%
\pgfpathlineto{\pgfqpoint{5.069949in}{2.832518in}}%
\pgfpathlineto{\pgfqpoint{5.062656in}{2.826627in}}%
\pgfpathlineto{\pgfqpoint{5.055356in}{2.820672in}}%
\pgfpathclose%
\pgfusepath{fill}%
\end{pgfscope}%
\begin{pgfscope}%
\pgfpathrectangle{\pgfqpoint{1.254980in}{0.150000in}}{\pgfqpoint{5.490039in}{5.490039in}}%
\pgfusepath{clip}%
\pgfsetbuttcap%
\pgfsetroundjoin%
\definecolor{currentfill}{rgb}{0.175707,0.697900,0.491033}%
\pgfsetfillcolor{currentfill}%
\pgfsetfillopacity{0.700000}%
\pgfsetlinewidth{0.000000pt}%
\definecolor{currentstroke}{rgb}{0.000000,0.000000,0.000000}%
\pgfsetstrokecolor{currentstroke}%
\pgfsetdash{}{0pt}%
\pgfpathmoveto{\pgfqpoint{5.843650in}{3.339668in}}%
\pgfpathlineto{\pgfqpoint{5.857896in}{3.350618in}}%
\pgfpathlineto{\pgfqpoint{5.872160in}{3.361723in}}%
\pgfpathlineto{\pgfqpoint{5.886445in}{3.372982in}}%
\pgfpathlineto{\pgfqpoint{5.900749in}{3.384397in}}%
\pgfpathlineto{\pgfqpoint{5.907558in}{3.385495in}}%
\pgfpathlineto{\pgfqpoint{5.914363in}{3.386649in}}%
\pgfpathlineto{\pgfqpoint{5.921163in}{3.387862in}}%
\pgfpathlineto{\pgfqpoint{5.927959in}{3.389143in}}%
\pgfpathlineto{\pgfqpoint{5.913689in}{3.378307in}}%
\pgfpathlineto{\pgfqpoint{5.899438in}{3.367625in}}%
\pgfpathlineto{\pgfqpoint{5.885206in}{3.357097in}}%
\pgfpathlineto{\pgfqpoint{5.870994in}{3.346722in}}%
\pgfpathlineto{\pgfqpoint{5.864164in}{3.344855in}}%
\pgfpathlineto{\pgfqpoint{5.857331in}{3.343061in}}%
\pgfpathlineto{\pgfqpoint{5.850493in}{3.341334in}}%
\pgfpathlineto{\pgfqpoint{5.843650in}{3.339668in}}%
\pgfpathclose%
\pgfusepath{fill}%
\end{pgfscope}%
\begin{pgfscope}%
\pgfpathrectangle{\pgfqpoint{1.254980in}{0.150000in}}{\pgfqpoint{5.490039in}{5.490039in}}%
\pgfusepath{clip}%
\pgfsetbuttcap%
\pgfsetroundjoin%
\definecolor{currentfill}{rgb}{0.248629,0.278775,0.534556}%
\pgfsetfillcolor{currentfill}%
\pgfsetfillopacity{0.700000}%
\pgfsetlinewidth{0.000000pt}%
\definecolor{currentstroke}{rgb}{0.000000,0.000000,0.000000}%
\pgfsetstrokecolor{currentstroke}%
\pgfsetdash{}{0pt}%
\pgfpathmoveto{\pgfqpoint{4.320642in}{2.249452in}}%
\pgfpathlineto{\pgfqpoint{4.334085in}{2.254872in}}%
\pgfpathlineto{\pgfqpoint{4.347540in}{2.260455in}}%
\pgfpathlineto{\pgfqpoint{4.361008in}{2.266201in}}%
\pgfpathlineto{\pgfqpoint{4.374487in}{2.272110in}}%
\pgfpathlineto{\pgfqpoint{4.382099in}{2.282152in}}%
\pgfpathlineto{\pgfqpoint{4.389705in}{2.292116in}}%
\pgfpathlineto{\pgfqpoint{4.397305in}{2.302001in}}%
\pgfpathlineto{\pgfqpoint{4.404901in}{2.311807in}}%
\pgfpathlineto{\pgfqpoint{4.391426in}{2.305879in}}%
\pgfpathlineto{\pgfqpoint{4.377964in}{2.300113in}}%
\pgfpathlineto{\pgfqpoint{4.364513in}{2.294511in}}%
\pgfpathlineto{\pgfqpoint{4.351075in}{2.289071in}}%
\pgfpathlineto{\pgfqpoint{4.343475in}{2.279273in}}%
\pgfpathlineto{\pgfqpoint{4.335869in}{2.269404in}}%
\pgfpathlineto{\pgfqpoint{4.328258in}{2.259464in}}%
\pgfpathlineto{\pgfqpoint{4.320642in}{2.249452in}}%
\pgfpathclose%
\pgfusepath{fill}%
\end{pgfscope}%
\begin{pgfscope}%
\pgfpathrectangle{\pgfqpoint{1.254980in}{0.150000in}}{\pgfqpoint{5.490039in}{5.490039in}}%
\pgfusepath{clip}%
\pgfsetbuttcap%
\pgfsetroundjoin%
\definecolor{currentfill}{rgb}{0.202219,0.715272,0.476084}%
\pgfsetfillcolor{currentfill}%
\pgfsetfillopacity{0.700000}%
\pgfsetlinewidth{0.000000pt}%
\definecolor{currentstroke}{rgb}{0.000000,0.000000,0.000000}%
\pgfsetstrokecolor{currentstroke}%
\pgfsetdash{}{0pt}%
\pgfpathmoveto{\pgfqpoint{5.927959in}{3.389143in}}%
\pgfpathlineto{\pgfqpoint{5.942249in}{3.400133in}}%
\pgfpathlineto{\pgfqpoint{5.956559in}{3.411277in}}%
\pgfpathlineto{\pgfqpoint{5.970888in}{3.422575in}}%
\pgfpathlineto{\pgfqpoint{5.985238in}{3.434027in}}%
\pgfpathlineto{\pgfqpoint{5.991994in}{3.434784in}}%
\pgfpathlineto{\pgfqpoint{5.998747in}{3.435614in}}%
\pgfpathlineto{\pgfqpoint{6.005495in}{3.436523in}}%
\pgfpathlineto{\pgfqpoint{6.012241in}{3.437519in}}%
\pgfpathlineto{\pgfqpoint{5.997927in}{3.426675in}}%
\pgfpathlineto{\pgfqpoint{5.983634in}{3.415984in}}%
\pgfpathlineto{\pgfqpoint{5.969360in}{3.405447in}}%
\pgfpathlineto{\pgfqpoint{5.955106in}{3.395062in}}%
\pgfpathlineto{\pgfqpoint{5.948324in}{3.393450in}}%
\pgfpathlineto{\pgfqpoint{5.941539in}{3.391931in}}%
\pgfpathlineto{\pgfqpoint{5.934751in}{3.390497in}}%
\pgfpathlineto{\pgfqpoint{5.927959in}{3.389143in}}%
\pgfpathclose%
\pgfusepath{fill}%
\end{pgfscope}%
\begin{pgfscope}%
\pgfpathrectangle{\pgfqpoint{1.254980in}{0.150000in}}{\pgfqpoint{5.490039in}{5.490039in}}%
\pgfusepath{clip}%
\pgfsetbuttcap%
\pgfsetroundjoin%
\definecolor{currentfill}{rgb}{0.282327,0.094955,0.417331}%
\pgfsetfillcolor{currentfill}%
\pgfsetfillopacity{0.700000}%
\pgfsetlinewidth{0.000000pt}%
\definecolor{currentstroke}{rgb}{0.000000,0.000000,0.000000}%
\pgfsetstrokecolor{currentstroke}%
\pgfsetdash{}{0pt}%
\pgfpathmoveto{\pgfqpoint{3.784899in}{1.883420in}}%
\pgfpathlineto{\pgfqpoint{3.798151in}{1.883922in}}%
\pgfpathlineto{\pgfqpoint{3.811411in}{1.884595in}}%
\pgfpathlineto{\pgfqpoint{3.824679in}{1.885436in}}%
\pgfpathlineto{\pgfqpoint{3.837954in}{1.886447in}}%
\pgfpathlineto{\pgfqpoint{3.845739in}{1.897097in}}%
\pgfpathlineto{\pgfqpoint{3.853518in}{1.907734in}}%
\pgfpathlineto{\pgfqpoint{3.861293in}{1.918359in}}%
\pgfpathlineto{\pgfqpoint{3.869062in}{1.928968in}}%
\pgfpathlineto{\pgfqpoint{3.855795in}{1.927740in}}%
\pgfpathlineto{\pgfqpoint{3.842535in}{1.926682in}}%
\pgfpathlineto{\pgfqpoint{3.829283in}{1.925793in}}%
\pgfpathlineto{\pgfqpoint{3.816039in}{1.925075in}}%
\pgfpathlineto{\pgfqpoint{3.808262in}{1.914672in}}%
\pgfpathlineto{\pgfqpoint{3.800479in}{1.904260in}}%
\pgfpathlineto{\pgfqpoint{3.792692in}{1.893843in}}%
\pgfpathlineto{\pgfqpoint{3.784899in}{1.883420in}}%
\pgfpathclose%
\pgfusepath{fill}%
\end{pgfscope}%
\begin{pgfscope}%
\pgfpathrectangle{\pgfqpoint{1.254980in}{0.150000in}}{\pgfqpoint{5.490039in}{5.490039in}}%
\pgfusepath{clip}%
\pgfsetbuttcap%
\pgfsetroundjoin%
\definecolor{currentfill}{rgb}{0.192357,0.403199,0.555836}%
\pgfsetfillcolor{currentfill}%
\pgfsetfillopacity{0.700000}%
\pgfsetlinewidth{0.000000pt}%
\definecolor{currentstroke}{rgb}{0.000000,0.000000,0.000000}%
\pgfsetstrokecolor{currentstroke}%
\pgfsetdash{}{0pt}%
\pgfpathmoveto{\pgfqpoint{4.688066in}{2.539782in}}%
\pgfpathlineto{\pgfqpoint{4.701689in}{2.547597in}}%
\pgfpathlineto{\pgfqpoint{4.715327in}{2.555572in}}%
\pgfpathlineto{\pgfqpoint{4.728979in}{2.563707in}}%
\pgfpathlineto{\pgfqpoint{4.742646in}{2.572003in}}%
\pgfpathlineto{\pgfqpoint{4.750114in}{2.580131in}}%
\pgfpathlineto{\pgfqpoint{4.757576in}{2.588170in}}%
\pgfpathlineto{\pgfqpoint{4.765031in}{2.596121in}}%
\pgfpathlineto{\pgfqpoint{4.772479in}{2.603985in}}%
\pgfpathlineto{\pgfqpoint{4.758821in}{2.595816in}}%
\pgfpathlineto{\pgfqpoint{4.745176in}{2.587808in}}%
\pgfpathlineto{\pgfqpoint{4.731547in}{2.579959in}}%
\pgfpathlineto{\pgfqpoint{4.717931in}{2.572271in}}%
\pgfpathlineto{\pgfqpoint{4.710474in}{2.564269in}}%
\pgfpathlineto{\pgfqpoint{4.703011in}{2.556188in}}%
\pgfpathlineto{\pgfqpoint{4.695542in}{2.548027in}}%
\pgfpathlineto{\pgfqpoint{4.688066in}{2.539782in}}%
\pgfpathclose%
\pgfusepath{fill}%
\end{pgfscope}%
\begin{pgfscope}%
\pgfpathrectangle{\pgfqpoint{1.254980in}{0.150000in}}{\pgfqpoint{5.490039in}{5.490039in}}%
\pgfusepath{clip}%
\pgfsetbuttcap%
\pgfsetroundjoin%
\definecolor{currentfill}{rgb}{0.280267,0.073417,0.397163}%
\pgfsetfillcolor{currentfill}%
\pgfsetfillopacity{0.700000}%
\pgfsetlinewidth{0.000000pt}%
\definecolor{currentstroke}{rgb}{0.000000,0.000000,0.000000}%
\pgfsetstrokecolor{currentstroke}%
\pgfsetdash{}{0pt}%
\pgfpathmoveto{\pgfqpoint{3.700704in}{1.842403in}}%
\pgfpathlineto{\pgfqpoint{3.713937in}{1.841976in}}%
\pgfpathlineto{\pgfqpoint{3.727177in}{1.841721in}}%
\pgfpathlineto{\pgfqpoint{3.740424in}{1.841638in}}%
\pgfpathlineto{\pgfqpoint{3.753678in}{1.841725in}}%
\pgfpathlineto{\pgfqpoint{3.761491in}{1.852146in}}%
\pgfpathlineto{\pgfqpoint{3.769299in}{1.862570in}}%
\pgfpathlineto{\pgfqpoint{3.777101in}{1.872995in}}%
\pgfpathlineto{\pgfqpoint{3.784899in}{1.883420in}}%
\pgfpathlineto{\pgfqpoint{3.771654in}{1.883089in}}%
\pgfpathlineto{\pgfqpoint{3.758416in}{1.882928in}}%
\pgfpathlineto{\pgfqpoint{3.745185in}{1.882939in}}%
\pgfpathlineto{\pgfqpoint{3.731962in}{1.883121in}}%
\pgfpathlineto{\pgfqpoint{3.724155in}{1.872930in}}%
\pgfpathlineto{\pgfqpoint{3.716343in}{1.862745in}}%
\pgfpathlineto{\pgfqpoint{3.708526in}{1.852569in}}%
\pgfpathlineto{\pgfqpoint{3.700704in}{1.842403in}}%
\pgfpathclose%
\pgfusepath{fill}%
\end{pgfscope}%
\begin{pgfscope}%
\pgfpathrectangle{\pgfqpoint{1.254980in}{0.150000in}}{\pgfqpoint{5.490039in}{5.490039in}}%
\pgfusepath{clip}%
\pgfsetbuttcap%
\pgfsetroundjoin%
\definecolor{currentfill}{rgb}{0.283229,0.120777,0.440584}%
\pgfsetfillcolor{currentfill}%
\pgfsetfillopacity{0.700000}%
\pgfsetlinewidth{0.000000pt}%
\definecolor{currentstroke}{rgb}{0.000000,0.000000,0.000000}%
\pgfsetstrokecolor{currentstroke}%
\pgfsetdash{}{0pt}%
\pgfpathmoveto{\pgfqpoint{3.869062in}{1.928968in}}%
\pgfpathlineto{\pgfqpoint{3.882338in}{1.930364in}}%
\pgfpathlineto{\pgfqpoint{3.895622in}{1.931929in}}%
\pgfpathlineto{\pgfqpoint{3.908915in}{1.933662in}}%
\pgfpathlineto{\pgfqpoint{3.922216in}{1.935562in}}%
\pgfpathlineto{\pgfqpoint{3.929973in}{1.946354in}}%
\pgfpathlineto{\pgfqpoint{3.937726in}{1.957120in}}%
\pgfpathlineto{\pgfqpoint{3.945474in}{1.967858in}}%
\pgfpathlineto{\pgfqpoint{3.953218in}{1.978568in}}%
\pgfpathlineto{\pgfqpoint{3.939923in}{1.976479in}}%
\pgfpathlineto{\pgfqpoint{3.926638in}{1.974557in}}%
\pgfpathlineto{\pgfqpoint{3.913361in}{1.972803in}}%
\pgfpathlineto{\pgfqpoint{3.900092in}{1.971218in}}%
\pgfpathlineto{\pgfqpoint{3.892342in}{1.960687in}}%
\pgfpathlineto{\pgfqpoint{3.884587in}{1.950133in}}%
\pgfpathlineto{\pgfqpoint{3.876827in}{1.939560in}}%
\pgfpathlineto{\pgfqpoint{3.869062in}{1.928968in}}%
\pgfpathclose%
\pgfusepath{fill}%
\end{pgfscope}%
\begin{pgfscope}%
\pgfpathrectangle{\pgfqpoint{1.254980in}{0.150000in}}{\pgfqpoint{5.490039in}{5.490039in}}%
\pgfusepath{clip}%
\pgfsetbuttcap%
\pgfsetroundjoin%
\definecolor{currentfill}{rgb}{0.277018,0.050344,0.375715}%
\pgfsetfillcolor{currentfill}%
\pgfsetfillopacity{0.700000}%
\pgfsetlinewidth{0.000000pt}%
\definecolor{currentstroke}{rgb}{0.000000,0.000000,0.000000}%
\pgfsetstrokecolor{currentstroke}%
\pgfsetdash{}{0pt}%
\pgfpathmoveto{\pgfqpoint{3.616448in}{1.806414in}}%
\pgfpathlineto{\pgfqpoint{3.629668in}{1.805023in}}%
\pgfpathlineto{\pgfqpoint{3.642893in}{1.803806in}}%
\pgfpathlineto{\pgfqpoint{3.656124in}{1.802762in}}%
\pgfpathlineto{\pgfqpoint{3.669361in}{1.801891in}}%
\pgfpathlineto{\pgfqpoint{3.677205in}{1.811992in}}%
\pgfpathlineto{\pgfqpoint{3.685043in}{1.822112in}}%
\pgfpathlineto{\pgfqpoint{3.692876in}{1.832250in}}%
\pgfpathlineto{\pgfqpoint{3.700704in}{1.842403in}}%
\pgfpathlineto{\pgfqpoint{3.687477in}{1.843002in}}%
\pgfpathlineto{\pgfqpoint{3.674256in}{1.843774in}}%
\pgfpathlineto{\pgfqpoint{3.661042in}{1.844720in}}%
\pgfpathlineto{\pgfqpoint{3.647834in}{1.845839in}}%
\pgfpathlineto{\pgfqpoint{3.639996in}{1.835947in}}%
\pgfpathlineto{\pgfqpoint{3.632152in}{1.826077in}}%
\pgfpathlineto{\pgfqpoint{3.624303in}{1.816232in}}%
\pgfpathlineto{\pgfqpoint{3.616448in}{1.806414in}}%
\pgfpathclose%
\pgfusepath{fill}%
\end{pgfscope}%
\begin{pgfscope}%
\pgfpathrectangle{\pgfqpoint{1.254980in}{0.150000in}}{\pgfqpoint{5.490039in}{5.490039in}}%
\pgfusepath{clip}%
\pgfsetbuttcap%
\pgfsetroundjoin%
\definecolor{currentfill}{rgb}{0.282290,0.145912,0.461510}%
\pgfsetfillcolor{currentfill}%
\pgfsetfillopacity{0.700000}%
\pgfsetlinewidth{0.000000pt}%
\definecolor{currentstroke}{rgb}{0.000000,0.000000,0.000000}%
\pgfsetstrokecolor{currentstroke}%
\pgfsetdash{}{0pt}%
\pgfpathmoveto{\pgfqpoint{3.953218in}{1.978568in}}%
\pgfpathlineto{\pgfqpoint{3.966521in}{1.980825in}}%
\pgfpathlineto{\pgfqpoint{3.979833in}{1.983249in}}%
\pgfpathlineto{\pgfqpoint{3.993154in}{1.985840in}}%
\pgfpathlineto{\pgfqpoint{4.006485in}{1.988597in}}%
\pgfpathlineto{\pgfqpoint{4.014217in}{1.999449in}}%
\pgfpathlineto{\pgfqpoint{4.021945in}{2.010262in}}%
\pgfpathlineto{\pgfqpoint{4.029667in}{2.021035in}}%
\pgfpathlineto{\pgfqpoint{4.037385in}{2.031767in}}%
\pgfpathlineto{\pgfqpoint{4.024060in}{2.028849in}}%
\pgfpathlineto{\pgfqpoint{4.010745in}{2.026097in}}%
\pgfpathlineto{\pgfqpoint{3.997439in}{2.023512in}}%
\pgfpathlineto{\pgfqpoint{3.984142in}{2.021094in}}%
\pgfpathlineto{\pgfqpoint{3.976418in}{2.010512in}}%
\pgfpathlineto{\pgfqpoint{3.968689in}{1.999897in}}%
\pgfpathlineto{\pgfqpoint{3.960956in}{1.989248in}}%
\pgfpathlineto{\pgfqpoint{3.953218in}{1.978568in}}%
\pgfpathclose%
\pgfusepath{fill}%
\end{pgfscope}%
\begin{pgfscope}%
\pgfpathrectangle{\pgfqpoint{1.254980in}{0.150000in}}{\pgfqpoint{5.490039in}{5.490039in}}%
\pgfusepath{clip}%
\pgfsetbuttcap%
\pgfsetroundjoin%
\definecolor{currentfill}{rgb}{0.140536,0.530132,0.555659}%
\pgfsetfillcolor{currentfill}%
\pgfsetfillopacity{0.700000}%
\pgfsetlinewidth{0.000000pt}%
\definecolor{currentstroke}{rgb}{0.000000,0.000000,0.000000}%
\pgfsetstrokecolor{currentstroke}%
\pgfsetdash{}{0pt}%
\pgfpathmoveto{\pgfqpoint{5.139829in}{2.881712in}}%
\pgfpathlineto{\pgfqpoint{5.153699in}{2.891505in}}%
\pgfpathlineto{\pgfqpoint{5.167586in}{2.901455in}}%
\pgfpathlineto{\pgfqpoint{5.181490in}{2.911564in}}%
\pgfpathlineto{\pgfqpoint{5.195411in}{2.921831in}}%
\pgfpathlineto{\pgfqpoint{5.202653in}{2.926963in}}%
\pgfpathlineto{\pgfqpoint{5.209888in}{2.932030in}}%
\pgfpathlineto{\pgfqpoint{5.217116in}{2.937037in}}%
\pgfpathlineto{\pgfqpoint{5.224336in}{2.941987in}}%
\pgfpathlineto{\pgfqpoint{5.210431in}{2.932027in}}%
\pgfpathlineto{\pgfqpoint{5.196543in}{2.922224in}}%
\pgfpathlineto{\pgfqpoint{5.182672in}{2.912580in}}%
\pgfpathlineto{\pgfqpoint{5.168817in}{2.903093in}}%
\pgfpathlineto{\pgfqpoint{5.161581in}{2.897826in}}%
\pgfpathlineto{\pgfqpoint{5.154337in}{2.892510in}}%
\pgfpathlineto{\pgfqpoint{5.147087in}{2.887140in}}%
\pgfpathlineto{\pgfqpoint{5.139829in}{2.881712in}}%
\pgfpathclose%
\pgfusepath{fill}%
\end{pgfscope}%
\begin{pgfscope}%
\pgfpathrectangle{\pgfqpoint{1.254980in}{0.150000in}}{\pgfqpoint{5.490039in}{5.490039in}}%
\pgfusepath{clip}%
\pgfsetbuttcap%
\pgfsetroundjoin%
\definecolor{currentfill}{rgb}{0.278791,0.062145,0.386592}%
\pgfsetfillcolor{currentfill}%
\pgfsetfillopacity{0.700000}%
\pgfsetlinewidth{0.000000pt}%
\definecolor{currentstroke}{rgb}{0.000000,0.000000,0.000000}%
\pgfsetstrokecolor{currentstroke}%
\pgfsetdash{}{0pt}%
\pgfpathmoveto{\pgfqpoint{2.981398in}{1.856030in}}%
\pgfpathlineto{\pgfqpoint{2.994632in}{1.846087in}}%
\pgfpathlineto{\pgfqpoint{3.007863in}{1.836349in}}%
\pgfpathlineto{\pgfqpoint{3.021094in}{1.826816in}}%
\pgfpathlineto{\pgfqpoint{3.034323in}{1.817486in}}%
\pgfpathlineto{\pgfqpoint{3.042475in}{1.822755in}}%
\pgfpathlineto{\pgfqpoint{3.050617in}{1.828172in}}%
\pgfpathlineto{\pgfqpoint{3.058749in}{1.833734in}}%
\pgfpathlineto{\pgfqpoint{3.066871in}{1.839435in}}%
\pgfpathlineto{\pgfqpoint{3.053668in}{1.848349in}}%
\pgfpathlineto{\pgfqpoint{3.040464in}{1.857467in}}%
\pgfpathlineto{\pgfqpoint{3.027260in}{1.866788in}}%
\pgfpathlineto{\pgfqpoint{3.014054in}{1.876315in}}%
\pgfpathlineto{\pgfqpoint{3.005906in}{1.871019in}}%
\pgfpathlineto{\pgfqpoint{2.997747in}{1.865870in}}%
\pgfpathlineto{\pgfqpoint{2.989578in}{1.860872in}}%
\pgfpathlineto{\pgfqpoint{2.981398in}{1.856030in}}%
\pgfpathclose%
\pgfusepath{fill}%
\end{pgfscope}%
\begin{pgfscope}%
\pgfpathrectangle{\pgfqpoint{1.254980in}{0.150000in}}{\pgfqpoint{5.490039in}{5.490039in}}%
\pgfusepath{clip}%
\pgfsetbuttcap%
\pgfsetroundjoin%
\definecolor{currentfill}{rgb}{0.179019,0.433756,0.557430}%
\pgfsetfillcolor{currentfill}%
\pgfsetfillopacity{0.700000}%
\pgfsetlinewidth{0.000000pt}%
\definecolor{currentstroke}{rgb}{0.000000,0.000000,0.000000}%
\pgfsetstrokecolor{currentstroke}%
\pgfsetdash{}{0pt}%
\pgfpathmoveto{\pgfqpoint{2.285796in}{2.696619in}}%
\pgfpathlineto{\pgfqpoint{2.299422in}{2.673555in}}%
\pgfpathlineto{\pgfqpoint{2.313034in}{2.650804in}}%
\pgfpathlineto{\pgfqpoint{2.326631in}{2.628363in}}%
\pgfpathlineto{\pgfqpoint{2.340214in}{2.606230in}}%
\pgfpathlineto{\pgfqpoint{2.348842in}{2.605778in}}%
\pgfpathlineto{\pgfqpoint{2.357452in}{2.605574in}}%
\pgfpathlineto{\pgfqpoint{2.366045in}{2.605611in}}%
\pgfpathlineto{\pgfqpoint{2.374622in}{2.605887in}}%
\pgfpathlineto{\pgfqpoint{2.361085in}{2.627571in}}%
\pgfpathlineto{\pgfqpoint{2.347535in}{2.649561in}}%
\pgfpathlineto{\pgfqpoint{2.333971in}{2.671860in}}%
\pgfpathlineto{\pgfqpoint{2.320392in}{2.694472in}}%
\pgfpathlineto{\pgfqpoint{2.311770in}{2.694636in}}%
\pgfpathlineto{\pgfqpoint{2.303130in}{2.695045in}}%
\pgfpathlineto{\pgfqpoint{2.294472in}{2.695705in}}%
\pgfpathlineto{\pgfqpoint{2.285796in}{2.696619in}}%
\pgfpathclose%
\pgfusepath{fill}%
\end{pgfscope}%
\begin{pgfscope}%
\pgfpathrectangle{\pgfqpoint{1.254980in}{0.150000in}}{\pgfqpoint{5.490039in}{5.490039in}}%
\pgfusepath{clip}%
\pgfsetbuttcap%
\pgfsetroundjoin%
\definecolor{currentfill}{rgb}{0.235526,0.309527,0.542944}%
\pgfsetfillcolor{currentfill}%
\pgfsetfillopacity{0.700000}%
\pgfsetlinewidth{0.000000pt}%
\definecolor{currentstroke}{rgb}{0.000000,0.000000,0.000000}%
\pgfsetstrokecolor{currentstroke}%
\pgfsetdash{}{0pt}%
\pgfpathmoveto{\pgfqpoint{4.404901in}{2.311807in}}%
\pgfpathlineto{\pgfqpoint{4.418388in}{2.317898in}}%
\pgfpathlineto{\pgfqpoint{4.431887in}{2.324150in}}%
\pgfpathlineto{\pgfqpoint{4.445399in}{2.330565in}}%
\pgfpathlineto{\pgfqpoint{4.458924in}{2.337142in}}%
\pgfpathlineto{\pgfqpoint{4.466509in}{2.346872in}}%
\pgfpathlineto{\pgfqpoint{4.474088in}{2.356517in}}%
\pgfpathlineto{\pgfqpoint{4.481661in}{2.366078in}}%
\pgfpathlineto{\pgfqpoint{4.489228in}{2.375555in}}%
\pgfpathlineto{\pgfqpoint{4.475708in}{2.368988in}}%
\pgfpathlineto{\pgfqpoint{4.462201in}{2.362582in}}%
\pgfpathlineto{\pgfqpoint{4.448707in}{2.356339in}}%
\pgfpathlineto{\pgfqpoint{4.435225in}{2.350258in}}%
\pgfpathlineto{\pgfqpoint{4.427652in}{2.340760in}}%
\pgfpathlineto{\pgfqpoint{4.420074in}{2.331187in}}%
\pgfpathlineto{\pgfqpoint{4.412490in}{2.321536in}}%
\pgfpathlineto{\pgfqpoint{4.404901in}{2.311807in}}%
\pgfpathclose%
\pgfusepath{fill}%
\end{pgfscope}%
\begin{pgfscope}%
\pgfpathrectangle{\pgfqpoint{1.254980in}{0.150000in}}{\pgfqpoint{5.490039in}{5.490039in}}%
\pgfusepath{clip}%
\pgfsetbuttcap%
\pgfsetroundjoin%
\definecolor{currentfill}{rgb}{0.271305,0.019942,0.347269}%
\pgfsetfillcolor{currentfill}%
\pgfsetfillopacity{0.700000}%
\pgfsetlinewidth{0.000000pt}%
\definecolor{currentstroke}{rgb}{0.000000,0.000000,0.000000}%
\pgfsetstrokecolor{currentstroke}%
\pgfsetdash{}{0pt}%
\pgfpathmoveto{\pgfqpoint{3.172486in}{1.775290in}}%
\pgfpathlineto{\pgfqpoint{3.185689in}{1.768153in}}%
\pgfpathlineto{\pgfqpoint{3.198894in}{1.761207in}}%
\pgfpathlineto{\pgfqpoint{3.212100in}{1.754452in}}%
\pgfpathlineto{\pgfqpoint{3.225306in}{1.747888in}}%
\pgfpathlineto{\pgfqpoint{3.233349in}{1.754914in}}%
\pgfpathlineto{\pgfqpoint{3.241384in}{1.762050in}}%
\pgfpathlineto{\pgfqpoint{3.249411in}{1.769295in}}%
\pgfpathlineto{\pgfqpoint{3.257429in}{1.776643in}}%
\pgfpathlineto{\pgfqpoint{3.244243in}{1.782823in}}%
\pgfpathlineto{\pgfqpoint{3.231059in}{1.789193in}}%
\pgfpathlineto{\pgfqpoint{3.217876in}{1.795754in}}%
\pgfpathlineto{\pgfqpoint{3.204694in}{1.802507in}}%
\pgfpathlineto{\pgfqpoint{3.196655in}{1.795533in}}%
\pgfpathlineto{\pgfqpoint{3.188607in}{1.788670in}}%
\pgfpathlineto{\pgfqpoint{3.180551in}{1.781921in}}%
\pgfpathlineto{\pgfqpoint{3.172486in}{1.775290in}}%
\pgfpathclose%
\pgfusepath{fill}%
\end{pgfscope}%
\begin{pgfscope}%
\pgfpathrectangle{\pgfqpoint{1.254980in}{0.150000in}}{\pgfqpoint{5.490039in}{5.490039in}}%
\pgfusepath{clip}%
\pgfsetbuttcap%
\pgfsetroundjoin%
\definecolor{currentfill}{rgb}{0.269944,0.014625,0.341379}%
\pgfsetfillcolor{currentfill}%
\pgfsetfillopacity{0.700000}%
\pgfsetlinewidth{0.000000pt}%
\definecolor{currentstroke}{rgb}{0.000000,0.000000,0.000000}%
\pgfsetstrokecolor{currentstroke}%
\pgfsetdash{}{0pt}%
\pgfpathmoveto{\pgfqpoint{3.310193in}{1.753806in}}%
\pgfpathlineto{\pgfqpoint{3.323389in}{1.748563in}}%
\pgfpathlineto{\pgfqpoint{3.336588in}{1.743506in}}%
\pgfpathlineto{\pgfqpoint{3.349790in}{1.738632in}}%
\pgfpathlineto{\pgfqpoint{3.362995in}{1.733941in}}%
\pgfpathlineto{\pgfqpoint{3.370968in}{1.742122in}}%
\pgfpathlineto{\pgfqpoint{3.378934in}{1.750386in}}%
\pgfpathlineto{\pgfqpoint{3.386893in}{1.758730in}}%
\pgfpathlineto{\pgfqpoint{3.394846in}{1.767150in}}%
\pgfpathlineto{\pgfqpoint{3.381658in}{1.771485in}}%
\pgfpathlineto{\pgfqpoint{3.368474in}{1.776003in}}%
\pgfpathlineto{\pgfqpoint{3.355292in}{1.780706in}}%
\pgfpathlineto{\pgfqpoint{3.342114in}{1.785593in}}%
\pgfpathlineto{\pgfqpoint{3.334144in}{1.777518in}}%
\pgfpathlineto{\pgfqpoint{3.326168in}{1.769526in}}%
\pgfpathlineto{\pgfqpoint{3.318184in}{1.761621in}}%
\pgfpathlineto{\pgfqpoint{3.310193in}{1.753806in}}%
\pgfpathclose%
\pgfusepath{fill}%
\end{pgfscope}%
\begin{pgfscope}%
\pgfpathrectangle{\pgfqpoint{1.254980in}{0.150000in}}{\pgfqpoint{5.490039in}{5.490039in}}%
\pgfusepath{clip}%
\pgfsetbuttcap%
\pgfsetroundjoin%
\definecolor{currentfill}{rgb}{0.278826,0.175490,0.483397}%
\pgfsetfillcolor{currentfill}%
\pgfsetfillopacity{0.700000}%
\pgfsetlinewidth{0.000000pt}%
\definecolor{currentstroke}{rgb}{0.000000,0.000000,0.000000}%
\pgfsetstrokecolor{currentstroke}%
\pgfsetdash{}{0pt}%
\pgfpathmoveto{\pgfqpoint{4.037385in}{2.031767in}}%
\pgfpathlineto{\pgfqpoint{4.050720in}{2.034851in}}%
\pgfpathlineto{\pgfqpoint{4.064064in}{2.038101in}}%
\pgfpathlineto{\pgfqpoint{4.077418in}{2.041517in}}%
\pgfpathlineto{\pgfqpoint{4.090782in}{2.045098in}}%
\pgfpathlineto{\pgfqpoint{4.098490in}{2.055931in}}%
\pgfpathlineto{\pgfqpoint{4.106192in}{2.066715in}}%
\pgfpathlineto{\pgfqpoint{4.113890in}{2.077447in}}%
\pgfpathlineto{\pgfqpoint{4.121583in}{2.088127in}}%
\pgfpathlineto{\pgfqpoint{4.108224in}{2.084413in}}%
\pgfpathlineto{\pgfqpoint{4.094875in}{2.080865in}}%
\pgfpathlineto{\pgfqpoint{4.081536in}{2.077481in}}%
\pgfpathlineto{\pgfqpoint{4.068207in}{2.074264in}}%
\pgfpathlineto{\pgfqpoint{4.060509in}{2.063707in}}%
\pgfpathlineto{\pgfqpoint{4.052806in}{2.053104in}}%
\pgfpathlineto{\pgfqpoint{4.045098in}{2.042457in}}%
\pgfpathlineto{\pgfqpoint{4.037385in}{2.031767in}}%
\pgfpathclose%
\pgfusepath{fill}%
\end{pgfscope}%
\begin{pgfscope}%
\pgfpathrectangle{\pgfqpoint{1.254980in}{0.150000in}}{\pgfqpoint{5.490039in}{5.490039in}}%
\pgfusepath{clip}%
\pgfsetbuttcap%
\pgfsetroundjoin%
\definecolor{currentfill}{rgb}{0.273809,0.031497,0.358853}%
\pgfsetfillcolor{currentfill}%
\pgfsetfillopacity{0.700000}%
\pgfsetlinewidth{0.000000pt}%
\definecolor{currentstroke}{rgb}{0.000000,0.000000,0.000000}%
\pgfsetstrokecolor{currentstroke}%
\pgfsetdash{}{0pt}%
\pgfpathmoveto{\pgfqpoint{3.532103in}{1.775973in}}%
\pgfpathlineto{\pgfqpoint{3.545313in}{1.773581in}}%
\pgfpathlineto{\pgfqpoint{3.558528in}{1.771366in}}%
\pgfpathlineto{\pgfqpoint{3.571748in}{1.769326in}}%
\pgfpathlineto{\pgfqpoint{3.584973in}{1.767461in}}%
\pgfpathlineto{\pgfqpoint{3.592851in}{1.777146in}}%
\pgfpathlineto{\pgfqpoint{3.600722in}{1.786868in}}%
\pgfpathlineto{\pgfqpoint{3.608588in}{1.796625in}}%
\pgfpathlineto{\pgfqpoint{3.616448in}{1.806414in}}%
\pgfpathlineto{\pgfqpoint{3.603235in}{1.807979in}}%
\pgfpathlineto{\pgfqpoint{3.590027in}{1.809720in}}%
\pgfpathlineto{\pgfqpoint{3.576825in}{1.811636in}}%
\pgfpathlineto{\pgfqpoint{3.563628in}{1.813729in}}%
\pgfpathlineto{\pgfqpoint{3.555755in}{1.804229in}}%
\pgfpathlineto{\pgfqpoint{3.547877in}{1.794768in}}%
\pgfpathlineto{\pgfqpoint{3.539993in}{1.785348in}}%
\pgfpathlineto{\pgfqpoint{3.532103in}{1.775973in}}%
\pgfpathclose%
\pgfusepath{fill}%
\end{pgfscope}%
\begin{pgfscope}%
\pgfpathrectangle{\pgfqpoint{1.254980in}{0.150000in}}{\pgfqpoint{5.490039in}{5.490039in}}%
\pgfusepath{clip}%
\pgfsetbuttcap%
\pgfsetroundjoin%
\definecolor{currentfill}{rgb}{0.180629,0.429975,0.557282}%
\pgfsetfillcolor{currentfill}%
\pgfsetfillopacity{0.700000}%
\pgfsetlinewidth{0.000000pt}%
\definecolor{currentstroke}{rgb}{0.000000,0.000000,0.000000}%
\pgfsetstrokecolor{currentstroke}%
\pgfsetdash{}{0pt}%
\pgfpathmoveto{\pgfqpoint{4.772479in}{2.603985in}}%
\pgfpathlineto{\pgfqpoint{4.786153in}{2.612314in}}%
\pgfpathlineto{\pgfqpoint{4.799841in}{2.620803in}}%
\pgfpathlineto{\pgfqpoint{4.813545in}{2.629452in}}%
\pgfpathlineto{\pgfqpoint{4.827264in}{2.638260in}}%
\pgfpathlineto{\pgfqpoint{4.834697in}{2.645895in}}%
\pgfpathlineto{\pgfqpoint{4.842124in}{2.653439in}}%
\pgfpathlineto{\pgfqpoint{4.849543in}{2.660896in}}%
\pgfpathlineto{\pgfqpoint{4.856956in}{2.668268in}}%
\pgfpathlineto{\pgfqpoint{4.843247in}{2.659616in}}%
\pgfpathlineto{\pgfqpoint{4.829552in}{2.651124in}}%
\pgfpathlineto{\pgfqpoint{4.815872in}{2.642791in}}%
\pgfpathlineto{\pgfqpoint{4.802208in}{2.634618in}}%
\pgfpathlineto{\pgfqpoint{4.794785in}{2.627080in}}%
\pgfpathlineto{\pgfqpoint{4.787357in}{2.619463in}}%
\pgfpathlineto{\pgfqpoint{4.779921in}{2.611765in}}%
\pgfpathlineto{\pgfqpoint{4.772479in}{2.603985in}}%
\pgfpathclose%
\pgfusepath{fill}%
\end{pgfscope}%
\begin{pgfscope}%
\pgfpathrectangle{\pgfqpoint{1.254980in}{0.150000in}}{\pgfqpoint{5.490039in}{5.490039in}}%
\pgfusepath{clip}%
\pgfsetbuttcap%
\pgfsetroundjoin%
\definecolor{currentfill}{rgb}{0.131172,0.555899,0.552459}%
\pgfsetfillcolor{currentfill}%
\pgfsetfillopacity{0.700000}%
\pgfsetlinewidth{0.000000pt}%
\definecolor{currentstroke}{rgb}{0.000000,0.000000,0.000000}%
\pgfsetstrokecolor{currentstroke}%
\pgfsetdash{}{0pt}%
\pgfpathmoveto{\pgfqpoint{5.224336in}{2.941987in}}%
\pgfpathlineto{\pgfqpoint{5.238259in}{2.952104in}}%
\pgfpathlineto{\pgfqpoint{5.252198in}{2.962380in}}%
\pgfpathlineto{\pgfqpoint{5.266155in}{2.972814in}}%
\pgfpathlineto{\pgfqpoint{5.280130in}{2.983405in}}%
\pgfpathlineto{\pgfqpoint{5.287326in}{2.987975in}}%
\pgfpathlineto{\pgfqpoint{5.294515in}{2.992488in}}%
\pgfpathlineto{\pgfqpoint{5.301697in}{2.996948in}}%
\pgfpathlineto{\pgfqpoint{5.308871in}{3.001359in}}%
\pgfpathlineto{\pgfqpoint{5.294914in}{2.991105in}}%
\pgfpathlineto{\pgfqpoint{5.280975in}{2.981009in}}%
\pgfpathlineto{\pgfqpoint{5.267052in}{2.971070in}}%
\pgfpathlineto{\pgfqpoint{5.253147in}{2.961288in}}%
\pgfpathlineto{\pgfqpoint{5.245955in}{2.956530in}}%
\pgfpathlineto{\pgfqpoint{5.238756in}{2.951729in}}%
\pgfpathlineto{\pgfqpoint{5.231550in}{2.946883in}}%
\pgfpathlineto{\pgfqpoint{5.224336in}{2.941987in}}%
\pgfpathclose%
\pgfusepath{fill}%
\end{pgfscope}%
\begin{pgfscope}%
\pgfpathrectangle{\pgfqpoint{1.254980in}{0.150000in}}{\pgfqpoint{5.490039in}{5.490039in}}%
\pgfusepath{clip}%
\pgfsetbuttcap%
\pgfsetroundjoin%
\definecolor{currentfill}{rgb}{0.271828,0.209303,0.504434}%
\pgfsetfillcolor{currentfill}%
\pgfsetfillopacity{0.700000}%
\pgfsetlinewidth{0.000000pt}%
\definecolor{currentstroke}{rgb}{0.000000,0.000000,0.000000}%
\pgfsetstrokecolor{currentstroke}%
\pgfsetdash{}{0pt}%
\pgfpathmoveto{\pgfqpoint{2.629179in}{2.153125in}}%
\pgfpathlineto{\pgfqpoint{2.642549in}{2.137350in}}%
\pgfpathlineto{\pgfqpoint{2.655911in}{2.121819in}}%
\pgfpathlineto{\pgfqpoint{2.669267in}{2.106531in}}%
\pgfpathlineto{\pgfqpoint{2.682616in}{2.091482in}}%
\pgfpathlineto{\pgfqpoint{2.691010in}{2.093354in}}%
\pgfpathlineto{\pgfqpoint{2.699389in}{2.095435in}}%
\pgfpathlineto{\pgfqpoint{2.707755in}{2.097721in}}%
\pgfpathlineto{\pgfqpoint{2.716107in}{2.100208in}}%
\pgfpathlineto{\pgfqpoint{2.702795in}{2.114801in}}%
\pgfpathlineto{\pgfqpoint{2.689477in}{2.129632in}}%
\pgfpathlineto{\pgfqpoint{2.676152in}{2.144706in}}%
\pgfpathlineto{\pgfqpoint{2.662821in}{2.160022in}}%
\pgfpathlineto{\pgfqpoint{2.654432in}{2.157981in}}%
\pgfpathlineto{\pgfqpoint{2.646029in}{2.156148in}}%
\pgfpathlineto{\pgfqpoint{2.637611in}{2.154528in}}%
\pgfpathlineto{\pgfqpoint{2.629179in}{2.153125in}}%
\pgfpathclose%
\pgfusepath{fill}%
\end{pgfscope}%
\begin{pgfscope}%
\pgfpathrectangle{\pgfqpoint{1.254980in}{0.150000in}}{\pgfqpoint{5.490039in}{5.490039in}}%
\pgfusepath{clip}%
\pgfsetbuttcap%
\pgfsetroundjoin%
\definecolor{currentfill}{rgb}{0.262138,0.242286,0.520837}%
\pgfsetfillcolor{currentfill}%
\pgfsetfillopacity{0.700000}%
\pgfsetlinewidth{0.000000pt}%
\definecolor{currentstroke}{rgb}{0.000000,0.000000,0.000000}%
\pgfsetstrokecolor{currentstroke}%
\pgfsetdash{}{0pt}%
\pgfpathmoveto{\pgfqpoint{2.575626in}{2.218703in}}%
\pgfpathlineto{\pgfqpoint{2.589026in}{2.201932in}}%
\pgfpathlineto{\pgfqpoint{2.602418in}{2.185414in}}%
\pgfpathlineto{\pgfqpoint{2.615802in}{2.169145in}}%
\pgfpathlineto{\pgfqpoint{2.629179in}{2.153125in}}%
\pgfpathlineto{\pgfqpoint{2.637611in}{2.154528in}}%
\pgfpathlineto{\pgfqpoint{2.646029in}{2.156148in}}%
\pgfpathlineto{\pgfqpoint{2.654432in}{2.157981in}}%
\pgfpathlineto{\pgfqpoint{2.662821in}{2.160022in}}%
\pgfpathlineto{\pgfqpoint{2.649483in}{2.175584in}}%
\pgfpathlineto{\pgfqpoint{2.636138in}{2.191392in}}%
\pgfpathlineto{\pgfqpoint{2.622786in}{2.207450in}}%
\pgfpathlineto{\pgfqpoint{2.609425in}{2.223759in}}%
\pgfpathlineto{\pgfqpoint{2.600998in}{2.222167in}}%
\pgfpathlineto{\pgfqpoint{2.592555in}{2.220790in}}%
\pgfpathlineto{\pgfqpoint{2.584098in}{2.219634in}}%
\pgfpathlineto{\pgfqpoint{2.575626in}{2.218703in}}%
\pgfpathclose%
\pgfusepath{fill}%
\end{pgfscope}%
\begin{pgfscope}%
\pgfpathrectangle{\pgfqpoint{1.254980in}{0.150000in}}{\pgfqpoint{5.490039in}{5.490039in}}%
\pgfusepath{clip}%
\pgfsetbuttcap%
\pgfsetroundjoin%
\definecolor{currentfill}{rgb}{0.278012,0.180367,0.486697}%
\pgfsetfillcolor{currentfill}%
\pgfsetfillopacity{0.700000}%
\pgfsetlinewidth{0.000000pt}%
\definecolor{currentstroke}{rgb}{0.000000,0.000000,0.000000}%
\pgfsetstrokecolor{currentstroke}%
\pgfsetdash{}{0pt}%
\pgfpathmoveto{\pgfqpoint{2.682616in}{2.091482in}}%
\pgfpathlineto{\pgfqpoint{2.695959in}{2.076672in}}%
\pgfpathlineto{\pgfqpoint{2.709296in}{2.062098in}}%
\pgfpathlineto{\pgfqpoint{2.722627in}{2.047759in}}%
\pgfpathlineto{\pgfqpoint{2.735952in}{2.033652in}}%
\pgfpathlineto{\pgfqpoint{2.744309in}{2.035990in}}%
\pgfpathlineto{\pgfqpoint{2.752652in}{2.038529in}}%
\pgfpathlineto{\pgfqpoint{2.760982in}{2.041266in}}%
\pgfpathlineto{\pgfqpoint{2.769298in}{2.044197in}}%
\pgfpathlineto{\pgfqpoint{2.756009in}{2.057849in}}%
\pgfpathlineto{\pgfqpoint{2.742714in}{2.071734in}}%
\pgfpathlineto{\pgfqpoint{2.729413in}{2.085853in}}%
\pgfpathlineto{\pgfqpoint{2.716107in}{2.100208in}}%
\pgfpathlineto{\pgfqpoint{2.707755in}{2.097721in}}%
\pgfpathlineto{\pgfqpoint{2.699389in}{2.095435in}}%
\pgfpathlineto{\pgfqpoint{2.691010in}{2.093354in}}%
\pgfpathlineto{\pgfqpoint{2.682616in}{2.091482in}}%
\pgfpathclose%
\pgfusepath{fill}%
\end{pgfscope}%
\begin{pgfscope}%
\pgfpathrectangle{\pgfqpoint{1.254980in}{0.150000in}}{\pgfqpoint{5.490039in}{5.490039in}}%
\pgfusepath{clip}%
\pgfsetbuttcap%
\pgfsetroundjoin%
\definecolor{currentfill}{rgb}{0.271828,0.209303,0.504434}%
\pgfsetfillcolor{currentfill}%
\pgfsetfillopacity{0.700000}%
\pgfsetlinewidth{0.000000pt}%
\definecolor{currentstroke}{rgb}{0.000000,0.000000,0.000000}%
\pgfsetstrokecolor{currentstroke}%
\pgfsetdash{}{0pt}%
\pgfpathmoveto{\pgfqpoint{4.121583in}{2.088127in}}%
\pgfpathlineto{\pgfqpoint{4.134952in}{2.092006in}}%
\pgfpathlineto{\pgfqpoint{4.148331in}{2.096050in}}%
\pgfpathlineto{\pgfqpoint{4.161722in}{2.100258in}}%
\pgfpathlineto{\pgfqpoint{4.175123in}{2.104631in}}%
\pgfpathlineto{\pgfqpoint{4.182806in}{2.115373in}}%
\pgfpathlineto{\pgfqpoint{4.190484in}{2.126055in}}%
\pgfpathlineto{\pgfqpoint{4.198157in}{2.136675in}}%
\pgfpathlineto{\pgfqpoint{4.205825in}{2.147233in}}%
\pgfpathlineto{\pgfqpoint{4.192428in}{2.142756in}}%
\pgfpathlineto{\pgfqpoint{4.179043in}{2.138442in}}%
\pgfpathlineto{\pgfqpoint{4.165668in}{2.134294in}}%
\pgfpathlineto{\pgfqpoint{4.152304in}{2.130310in}}%
\pgfpathlineto{\pgfqpoint{4.144631in}{2.119846in}}%
\pgfpathlineto{\pgfqpoint{4.136953in}{2.109328in}}%
\pgfpathlineto{\pgfqpoint{4.129270in}{2.098754in}}%
\pgfpathlineto{\pgfqpoint{4.121583in}{2.088127in}}%
\pgfpathclose%
\pgfusepath{fill}%
\end{pgfscope}%
\begin{pgfscope}%
\pgfpathrectangle{\pgfqpoint{1.254980in}{0.150000in}}{\pgfqpoint{5.490039in}{5.490039in}}%
\pgfusepath{clip}%
\pgfsetbuttcap%
\pgfsetroundjoin%
\definecolor{currentfill}{rgb}{0.250425,0.274290,0.533103}%
\pgfsetfillcolor{currentfill}%
\pgfsetfillopacity{0.700000}%
\pgfsetlinewidth{0.000000pt}%
\definecolor{currentstroke}{rgb}{0.000000,0.000000,0.000000}%
\pgfsetstrokecolor{currentstroke}%
\pgfsetdash{}{0pt}%
\pgfpathmoveto{\pgfqpoint{2.521940in}{2.288347in}}%
\pgfpathlineto{\pgfqpoint{2.535375in}{2.270547in}}%
\pgfpathlineto{\pgfqpoint{2.548800in}{2.253008in}}%
\pgfpathlineto{\pgfqpoint{2.562217in}{2.235727in}}%
\pgfpathlineto{\pgfqpoint{2.575626in}{2.218703in}}%
\pgfpathlineto{\pgfqpoint{2.584098in}{2.219634in}}%
\pgfpathlineto{\pgfqpoint{2.592555in}{2.220790in}}%
\pgfpathlineto{\pgfqpoint{2.600998in}{2.222167in}}%
\pgfpathlineto{\pgfqpoint{2.609425in}{2.223759in}}%
\pgfpathlineto{\pgfqpoint{2.596058in}{2.240322in}}%
\pgfpathlineto{\pgfqpoint{2.582681in}{2.257140in}}%
\pgfpathlineto{\pgfqpoint{2.569297in}{2.274216in}}%
\pgfpathlineto{\pgfqpoint{2.555904in}{2.291551in}}%
\pgfpathlineto{\pgfqpoint{2.547436in}{2.290410in}}%
\pgfpathlineto{\pgfqpoint{2.538953in}{2.289493in}}%
\pgfpathlineto{\pgfqpoint{2.530454in}{2.288804in}}%
\pgfpathlineto{\pgfqpoint{2.521940in}{2.288347in}}%
\pgfpathclose%
\pgfusepath{fill}%
\end{pgfscope}%
\begin{pgfscope}%
\pgfpathrectangle{\pgfqpoint{1.254980in}{0.150000in}}{\pgfqpoint{5.490039in}{5.490039in}}%
\pgfusepath{clip}%
\pgfsetbuttcap%
\pgfsetroundjoin%
\definecolor{currentfill}{rgb}{0.221989,0.339161,0.548752}%
\pgfsetfillcolor{currentfill}%
\pgfsetfillopacity{0.700000}%
\pgfsetlinewidth{0.000000pt}%
\definecolor{currentstroke}{rgb}{0.000000,0.000000,0.000000}%
\pgfsetstrokecolor{currentstroke}%
\pgfsetdash{}{0pt}%
\pgfpathmoveto{\pgfqpoint{4.489228in}{2.375555in}}%
\pgfpathlineto{\pgfqpoint{4.502761in}{2.382284in}}%
\pgfpathlineto{\pgfqpoint{4.516307in}{2.389175in}}%
\pgfpathlineto{\pgfqpoint{4.529866in}{2.396227in}}%
\pgfpathlineto{\pgfqpoint{4.543439in}{2.403441in}}%
\pgfpathlineto{\pgfqpoint{4.550995in}{2.412807in}}%
\pgfpathlineto{\pgfqpoint{4.558545in}{2.422084in}}%
\pgfpathlineto{\pgfqpoint{4.566089in}{2.431272in}}%
\pgfpathlineto{\pgfqpoint{4.573627in}{2.440371in}}%
\pgfpathlineto{\pgfqpoint{4.560060in}{2.433196in}}%
\pgfpathlineto{\pgfqpoint{4.546506in}{2.426183in}}%
\pgfpathlineto{\pgfqpoint{4.532966in}{2.419331in}}%
\pgfpathlineto{\pgfqpoint{4.519439in}{2.412640in}}%
\pgfpathlineto{\pgfqpoint{4.511895in}{2.403491in}}%
\pgfpathlineto{\pgfqpoint{4.504345in}{2.394261in}}%
\pgfpathlineto{\pgfqpoint{4.496789in}{2.384949in}}%
\pgfpathlineto{\pgfqpoint{4.489228in}{2.375555in}}%
\pgfpathclose%
\pgfusepath{fill}%
\end{pgfscope}%
\begin{pgfscope}%
\pgfpathrectangle{\pgfqpoint{1.254980in}{0.150000in}}{\pgfqpoint{5.490039in}{5.490039in}}%
\pgfusepath{clip}%
\pgfsetbuttcap%
\pgfsetroundjoin%
\definecolor{currentfill}{rgb}{0.281412,0.155834,0.469201}%
\pgfsetfillcolor{currentfill}%
\pgfsetfillopacity{0.700000}%
\pgfsetlinewidth{0.000000pt}%
\definecolor{currentstroke}{rgb}{0.000000,0.000000,0.000000}%
\pgfsetstrokecolor{currentstroke}%
\pgfsetdash{}{0pt}%
\pgfpathmoveto{\pgfqpoint{2.735952in}{2.033652in}}%
\pgfpathlineto{\pgfqpoint{2.749272in}{2.019777in}}%
\pgfpathlineto{\pgfqpoint{2.762587in}{2.006131in}}%
\pgfpathlineto{\pgfqpoint{2.775897in}{1.992713in}}%
\pgfpathlineto{\pgfqpoint{2.789202in}{1.979521in}}%
\pgfpathlineto{\pgfqpoint{2.797523in}{1.982322in}}%
\pgfpathlineto{\pgfqpoint{2.805831in}{1.985318in}}%
\pgfpathlineto{\pgfqpoint{2.814126in}{1.988503in}}%
\pgfpathlineto{\pgfqpoint{2.822409in}{1.991875in}}%
\pgfpathlineto{\pgfqpoint{2.809138in}{2.004615in}}%
\pgfpathlineto{\pgfqpoint{2.795863in}{2.017581in}}%
\pgfpathlineto{\pgfqpoint{2.782583in}{2.030775in}}%
\pgfpathlineto{\pgfqpoint{2.769298in}{2.044197in}}%
\pgfpathlineto{\pgfqpoint{2.760982in}{2.041266in}}%
\pgfpathlineto{\pgfqpoint{2.752652in}{2.038529in}}%
\pgfpathlineto{\pgfqpoint{2.744309in}{2.035990in}}%
\pgfpathlineto{\pgfqpoint{2.735952in}{2.033652in}}%
\pgfpathclose%
\pgfusepath{fill}%
\end{pgfscope}%
\begin{pgfscope}%
\pgfpathrectangle{\pgfqpoint{1.254980in}{0.150000in}}{\pgfqpoint{5.490039in}{5.490039in}}%
\pgfusepath{clip}%
\pgfsetbuttcap%
\pgfsetroundjoin%
\definecolor{currentfill}{rgb}{0.277018,0.050344,0.375715}%
\pgfsetfillcolor{currentfill}%
\pgfsetfillopacity{0.700000}%
\pgfsetlinewidth{0.000000pt}%
\definecolor{currentstroke}{rgb}{0.000000,0.000000,0.000000}%
\pgfsetstrokecolor{currentstroke}%
\pgfsetdash{}{0pt}%
\pgfpathmoveto{\pgfqpoint{3.034323in}{1.817486in}}%
\pgfpathlineto{\pgfqpoint{3.047551in}{1.808359in}}%
\pgfpathlineto{\pgfqpoint{3.060779in}{1.799432in}}%
\pgfpathlineto{\pgfqpoint{3.074006in}{1.790705in}}%
\pgfpathlineto{\pgfqpoint{3.087232in}{1.782176in}}%
\pgfpathlineto{\pgfqpoint{3.095358in}{1.787871in}}%
\pgfpathlineto{\pgfqpoint{3.103474in}{1.793706in}}%
\pgfpathlineto{\pgfqpoint{3.111580in}{1.799679in}}%
\pgfpathlineto{\pgfqpoint{3.119678in}{1.805784in}}%
\pgfpathlineto{\pgfqpoint{3.106476in}{1.813898in}}%
\pgfpathlineto{\pgfqpoint{3.093275in}{1.822210in}}%
\pgfpathlineto{\pgfqpoint{3.080073in}{1.830722in}}%
\pgfpathlineto{\pgfqpoint{3.066871in}{1.839435in}}%
\pgfpathlineto{\pgfqpoint{3.058749in}{1.833734in}}%
\pgfpathlineto{\pgfqpoint{3.050617in}{1.828172in}}%
\pgfpathlineto{\pgfqpoint{3.042475in}{1.822755in}}%
\pgfpathlineto{\pgfqpoint{3.034323in}{1.817486in}}%
\pgfpathclose%
\pgfusepath{fill}%
\end{pgfscope}%
\begin{pgfscope}%
\pgfpathrectangle{\pgfqpoint{1.254980in}{0.150000in}}{\pgfqpoint{5.490039in}{5.490039in}}%
\pgfusepath{clip}%
\pgfsetbuttcap%
\pgfsetroundjoin%
\definecolor{currentfill}{rgb}{0.124395,0.578002,0.548287}%
\pgfsetfillcolor{currentfill}%
\pgfsetfillopacity{0.700000}%
\pgfsetlinewidth{0.000000pt}%
\definecolor{currentstroke}{rgb}{0.000000,0.000000,0.000000}%
\pgfsetstrokecolor{currentstroke}%
\pgfsetdash{}{0pt}%
\pgfpathmoveto{\pgfqpoint{5.308871in}{3.001359in}}%
\pgfpathlineto{\pgfqpoint{5.322846in}{3.011770in}}%
\pgfpathlineto{\pgfqpoint{5.336838in}{3.022339in}}%
\pgfpathlineto{\pgfqpoint{5.350848in}{3.033065in}}%
\pgfpathlineto{\pgfqpoint{5.364876in}{3.043949in}}%
\pgfpathlineto{\pgfqpoint{5.372024in}{3.047958in}}%
\pgfpathlineto{\pgfqpoint{5.379166in}{3.051918in}}%
\pgfpathlineto{\pgfqpoint{5.386299in}{3.055835in}}%
\pgfpathlineto{\pgfqpoint{5.393426in}{3.059712in}}%
\pgfpathlineto{\pgfqpoint{5.379417in}{3.049197in}}%
\pgfpathlineto{\pgfqpoint{5.365427in}{3.038839in}}%
\pgfpathlineto{\pgfqpoint{5.351453in}{3.028637in}}%
\pgfpathlineto{\pgfqpoint{5.337498in}{3.018592in}}%
\pgfpathlineto{\pgfqpoint{5.330352in}{3.014337in}}%
\pgfpathlineto{\pgfqpoint{5.323199in}{3.010050in}}%
\pgfpathlineto{\pgfqpoint{5.316038in}{3.005725in}}%
\pgfpathlineto{\pgfqpoint{5.308871in}{3.001359in}}%
\pgfpathclose%
\pgfusepath{fill}%
\end{pgfscope}%
\begin{pgfscope}%
\pgfpathrectangle{\pgfqpoint{1.254980in}{0.150000in}}{\pgfqpoint{5.490039in}{5.490039in}}%
\pgfusepath{clip}%
\pgfsetbuttcap%
\pgfsetroundjoin%
\definecolor{currentfill}{rgb}{0.271305,0.019942,0.347269}%
\pgfsetfillcolor{currentfill}%
\pgfsetfillopacity{0.700000}%
\pgfsetlinewidth{0.000000pt}%
\definecolor{currentstroke}{rgb}{0.000000,0.000000,0.000000}%
\pgfsetstrokecolor{currentstroke}%
\pgfsetdash{}{0pt}%
\pgfpathmoveto{\pgfqpoint{3.447631in}{1.751626in}}%
\pgfpathlineto{\pgfqpoint{3.460837in}{1.748196in}}%
\pgfpathlineto{\pgfqpoint{3.474048in}{1.744944in}}%
\pgfpathlineto{\pgfqpoint{3.487262in}{1.741870in}}%
\pgfpathlineto{\pgfqpoint{3.500481in}{1.738974in}}%
\pgfpathlineto{\pgfqpoint{3.508396in}{1.748143in}}%
\pgfpathlineto{\pgfqpoint{3.516304in}{1.757368in}}%
\pgfpathlineto{\pgfqpoint{3.524207in}{1.766646in}}%
\pgfpathlineto{\pgfqpoint{3.532103in}{1.775973in}}%
\pgfpathlineto{\pgfqpoint{3.518898in}{1.778542in}}%
\pgfpathlineto{\pgfqpoint{3.505697in}{1.781288in}}%
\pgfpathlineto{\pgfqpoint{3.492502in}{1.784213in}}%
\pgfpathlineto{\pgfqpoint{3.479310in}{1.787316in}}%
\pgfpathlineto{\pgfqpoint{3.471400in}{1.778305in}}%
\pgfpathlineto{\pgfqpoint{3.463484in}{1.769351in}}%
\pgfpathlineto{\pgfqpoint{3.455561in}{1.760457in}}%
\pgfpathlineto{\pgfqpoint{3.447631in}{1.751626in}}%
\pgfpathclose%
\pgfusepath{fill}%
\end{pgfscope}%
\begin{pgfscope}%
\pgfpathrectangle{\pgfqpoint{1.254980in}{0.150000in}}{\pgfqpoint{5.490039in}{5.490039in}}%
\pgfusepath{clip}%
\pgfsetbuttcap%
\pgfsetroundjoin%
\definecolor{currentfill}{rgb}{0.169646,0.456262,0.558030}%
\pgfsetfillcolor{currentfill}%
\pgfsetfillopacity{0.700000}%
\pgfsetlinewidth{0.000000pt}%
\definecolor{currentstroke}{rgb}{0.000000,0.000000,0.000000}%
\pgfsetstrokecolor{currentstroke}%
\pgfsetdash{}{0pt}%
\pgfpathmoveto{\pgfqpoint{4.856956in}{2.668268in}}%
\pgfpathlineto{\pgfqpoint{4.870681in}{2.677079in}}%
\pgfpathlineto{\pgfqpoint{4.884422in}{2.686050in}}%
\pgfpathlineto{\pgfqpoint{4.898178in}{2.695181in}}%
\pgfpathlineto{\pgfqpoint{4.911950in}{2.704472in}}%
\pgfpathlineto{\pgfqpoint{4.919346in}{2.711584in}}%
\pgfpathlineto{\pgfqpoint{4.926736in}{2.718607in}}%
\pgfpathlineto{\pgfqpoint{4.934118in}{2.725544in}}%
\pgfpathlineto{\pgfqpoint{4.941493in}{2.732398in}}%
\pgfpathlineto{\pgfqpoint{4.927732in}{2.723295in}}%
\pgfpathlineto{\pgfqpoint{4.913986in}{2.714351in}}%
\pgfpathlineto{\pgfqpoint{4.900255in}{2.705566in}}%
\pgfpathlineto{\pgfqpoint{4.886540in}{2.696940in}}%
\pgfpathlineto{\pgfqpoint{4.879154in}{2.689890in}}%
\pgfpathlineto{\pgfqpoint{4.871762in}{2.682762in}}%
\pgfpathlineto{\pgfqpoint{4.864362in}{2.675556in}}%
\pgfpathlineto{\pgfqpoint{4.856956in}{2.668268in}}%
\pgfpathclose%
\pgfusepath{fill}%
\end{pgfscope}%
\begin{pgfscope}%
\pgfpathrectangle{\pgfqpoint{1.254980in}{0.150000in}}{\pgfqpoint{5.490039in}{5.490039in}}%
\pgfusepath{clip}%
\pgfsetbuttcap%
\pgfsetroundjoin%
\definecolor{currentfill}{rgb}{0.237441,0.305202,0.541921}%
\pgfsetfillcolor{currentfill}%
\pgfsetfillopacity{0.700000}%
\pgfsetlinewidth{0.000000pt}%
\definecolor{currentstroke}{rgb}{0.000000,0.000000,0.000000}%
\pgfsetstrokecolor{currentstroke}%
\pgfsetdash{}{0pt}%
\pgfpathmoveto{\pgfqpoint{2.468105in}{2.362201in}}%
\pgfpathlineto{\pgfqpoint{2.481579in}{2.343335in}}%
\pgfpathlineto{\pgfqpoint{2.495042in}{2.324739in}}%
\pgfpathlineto{\pgfqpoint{2.508496in}{2.306411in}}%
\pgfpathlineto{\pgfqpoint{2.521940in}{2.288347in}}%
\pgfpathlineto{\pgfqpoint{2.530454in}{2.288804in}}%
\pgfpathlineto{\pgfqpoint{2.538953in}{2.289493in}}%
\pgfpathlineto{\pgfqpoint{2.547436in}{2.290410in}}%
\pgfpathlineto{\pgfqpoint{2.555904in}{2.291551in}}%
\pgfpathlineto{\pgfqpoint{2.542502in}{2.309149in}}%
\pgfpathlineto{\pgfqpoint{2.529091in}{2.327011in}}%
\pgfpathlineto{\pgfqpoint{2.515670in}{2.345140in}}%
\pgfpathlineto{\pgfqpoint{2.502240in}{2.363538in}}%
\pgfpathlineto{\pgfqpoint{2.493731in}{2.362852in}}%
\pgfpathlineto{\pgfqpoint{2.485205in}{2.362397in}}%
\pgfpathlineto{\pgfqpoint{2.476663in}{2.362179in}}%
\pgfpathlineto{\pgfqpoint{2.468105in}{2.362201in}}%
\pgfpathclose%
\pgfusepath{fill}%
\end{pgfscope}%
\begin{pgfscope}%
\pgfpathrectangle{\pgfqpoint{1.254980in}{0.150000in}}{\pgfqpoint{5.490039in}{5.490039in}}%
\pgfusepath{clip}%
\pgfsetbuttcap%
\pgfsetroundjoin%
\definecolor{currentfill}{rgb}{0.283072,0.130895,0.449241}%
\pgfsetfillcolor{currentfill}%
\pgfsetfillopacity{0.700000}%
\pgfsetlinewidth{0.000000pt}%
\definecolor{currentstroke}{rgb}{0.000000,0.000000,0.000000}%
\pgfsetstrokecolor{currentstroke}%
\pgfsetdash{}{0pt}%
\pgfpathmoveto{\pgfqpoint{2.789202in}{1.979521in}}%
\pgfpathlineto{\pgfqpoint{2.802503in}{1.966554in}}%
\pgfpathlineto{\pgfqpoint{2.815799in}{1.953809in}}%
\pgfpathlineto{\pgfqpoint{2.829091in}{1.941286in}}%
\pgfpathlineto{\pgfqpoint{2.842380in}{1.928983in}}%
\pgfpathlineto{\pgfqpoint{2.850667in}{1.932245in}}%
\pgfpathlineto{\pgfqpoint{2.858941in}{1.935694in}}%
\pgfpathlineto{\pgfqpoint{2.867204in}{1.939326in}}%
\pgfpathlineto{\pgfqpoint{2.875454in}{1.943137in}}%
\pgfpathlineto{\pgfqpoint{2.862199in}{1.954991in}}%
\pgfpathlineto{\pgfqpoint{2.848939in}{1.967064in}}%
\pgfpathlineto{\pgfqpoint{2.835676in}{1.979358in}}%
\pgfpathlineto{\pgfqpoint{2.822409in}{1.991875in}}%
\pgfpathlineto{\pgfqpoint{2.814126in}{1.988503in}}%
\pgfpathlineto{\pgfqpoint{2.805831in}{1.985318in}}%
\pgfpathlineto{\pgfqpoint{2.797523in}{1.982322in}}%
\pgfpathlineto{\pgfqpoint{2.789202in}{1.979521in}}%
\pgfpathclose%
\pgfusepath{fill}%
\end{pgfscope}%
\begin{pgfscope}%
\pgfpathrectangle{\pgfqpoint{1.254980in}{0.150000in}}{\pgfqpoint{5.490039in}{5.490039in}}%
\pgfusepath{clip}%
\pgfsetbuttcap%
\pgfsetroundjoin%
\definecolor{currentfill}{rgb}{0.263663,0.237631,0.518762}%
\pgfsetfillcolor{currentfill}%
\pgfsetfillopacity{0.700000}%
\pgfsetlinewidth{0.000000pt}%
\definecolor{currentstroke}{rgb}{0.000000,0.000000,0.000000}%
\pgfsetstrokecolor{currentstroke}%
\pgfsetdash{}{0pt}%
\pgfpathmoveto{\pgfqpoint{4.205825in}{2.147233in}}%
\pgfpathlineto{\pgfqpoint{4.219232in}{2.151875in}}%
\pgfpathlineto{\pgfqpoint{4.232650in}{2.156680in}}%
\pgfpathlineto{\pgfqpoint{4.246080in}{2.161649in}}%
\pgfpathlineto{\pgfqpoint{4.259521in}{2.166781in}}%
\pgfpathlineto{\pgfqpoint{4.267179in}{2.177364in}}%
\pgfpathlineto{\pgfqpoint{4.274832in}{2.187876in}}%
\pgfpathlineto{\pgfqpoint{4.282480in}{2.198317in}}%
\pgfpathlineto{\pgfqpoint{4.290123in}{2.208687in}}%
\pgfpathlineto{\pgfqpoint{4.276687in}{2.203478in}}%
\pgfpathlineto{\pgfqpoint{4.263261in}{2.198432in}}%
\pgfpathlineto{\pgfqpoint{4.249848in}{2.193551in}}%
\pgfpathlineto{\pgfqpoint{4.236446in}{2.188832in}}%
\pgfpathlineto{\pgfqpoint{4.228798in}{2.178528in}}%
\pgfpathlineto{\pgfqpoint{4.221145in}{2.168160in}}%
\pgfpathlineto{\pgfqpoint{4.213487in}{2.157728in}}%
\pgfpathlineto{\pgfqpoint{4.205825in}{2.147233in}}%
\pgfpathclose%
\pgfusepath{fill}%
\end{pgfscope}%
\begin{pgfscope}%
\pgfpathrectangle{\pgfqpoint{1.254980in}{0.150000in}}{\pgfqpoint{5.490039in}{5.490039in}}%
\pgfusepath{clip}%
\pgfsetbuttcap%
\pgfsetroundjoin%
\definecolor{currentfill}{rgb}{0.120092,0.600104,0.542530}%
\pgfsetfillcolor{currentfill}%
\pgfsetfillopacity{0.700000}%
\pgfsetlinewidth{0.000000pt}%
\definecolor{currentstroke}{rgb}{0.000000,0.000000,0.000000}%
\pgfsetstrokecolor{currentstroke}%
\pgfsetdash{}{0pt}%
\pgfpathmoveto{\pgfqpoint{5.393426in}{3.059712in}}%
\pgfpathlineto{\pgfqpoint{5.407453in}{3.070385in}}%
\pgfpathlineto{\pgfqpoint{5.421497in}{3.081214in}}%
\pgfpathlineto{\pgfqpoint{5.435560in}{3.092201in}}%
\pgfpathlineto{\pgfqpoint{5.449641in}{3.103346in}}%
\pgfpathlineto{\pgfqpoint{5.456740in}{3.106799in}}%
\pgfpathlineto{\pgfqpoint{5.463831in}{3.110215in}}%
\pgfpathlineto{\pgfqpoint{5.470916in}{3.113596in}}%
\pgfpathlineto{\pgfqpoint{5.477993in}{3.116949in}}%
\pgfpathlineto{\pgfqpoint{5.463933in}{3.106204in}}%
\pgfpathlineto{\pgfqpoint{5.449891in}{3.095615in}}%
\pgfpathlineto{\pgfqpoint{5.435868in}{3.085184in}}%
\pgfpathlineto{\pgfqpoint{5.421862in}{3.074908in}}%
\pgfpathlineto{\pgfqpoint{5.414763in}{3.071147in}}%
\pgfpathlineto{\pgfqpoint{5.407658in}{3.067364in}}%
\pgfpathlineto{\pgfqpoint{5.400545in}{3.063554in}}%
\pgfpathlineto{\pgfqpoint{5.393426in}{3.059712in}}%
\pgfpathclose%
\pgfusepath{fill}%
\end{pgfscope}%
\begin{pgfscope}%
\pgfpathrectangle{\pgfqpoint{1.254980in}{0.150000in}}{\pgfqpoint{5.490039in}{5.490039in}}%
\pgfusepath{clip}%
\pgfsetbuttcap%
\pgfsetroundjoin%
\definecolor{currentfill}{rgb}{0.269944,0.014625,0.341379}%
\pgfsetfillcolor{currentfill}%
\pgfsetfillopacity{0.700000}%
\pgfsetlinewidth{0.000000pt}%
\definecolor{currentstroke}{rgb}{0.000000,0.000000,0.000000}%
\pgfsetstrokecolor{currentstroke}%
\pgfsetdash{}{0pt}%
\pgfpathmoveto{\pgfqpoint{3.225306in}{1.747888in}}%
\pgfpathlineto{\pgfqpoint{3.238515in}{1.741513in}}%
\pgfpathlineto{\pgfqpoint{3.251725in}{1.735327in}}%
\pgfpathlineto{\pgfqpoint{3.264937in}{1.729328in}}%
\pgfpathlineto{\pgfqpoint{3.278151in}{1.723515in}}%
\pgfpathlineto{\pgfqpoint{3.286173in}{1.730936in}}%
\pgfpathlineto{\pgfqpoint{3.294187in}{1.738460in}}%
\pgfpathlineto{\pgfqpoint{3.302194in}{1.746085in}}%
\pgfpathlineto{\pgfqpoint{3.310193in}{1.753806in}}%
\pgfpathlineto{\pgfqpoint{3.296998in}{1.759235in}}%
\pgfpathlineto{\pgfqpoint{3.283807in}{1.764850in}}%
\pgfpathlineto{\pgfqpoint{3.270617in}{1.770652in}}%
\pgfpathlineto{\pgfqpoint{3.257429in}{1.776643in}}%
\pgfpathlineto{\pgfqpoint{3.249411in}{1.769295in}}%
\pgfpathlineto{\pgfqpoint{3.241384in}{1.762050in}}%
\pgfpathlineto{\pgfqpoint{3.233349in}{1.754914in}}%
\pgfpathlineto{\pgfqpoint{3.225306in}{1.747888in}}%
\pgfpathclose%
\pgfusepath{fill}%
\end{pgfscope}%
\begin{pgfscope}%
\pgfpathrectangle{\pgfqpoint{1.254980in}{0.150000in}}{\pgfqpoint{5.490039in}{5.490039in}}%
\pgfusepath{clip}%
\pgfsetbuttcap%
\pgfsetroundjoin%
\definecolor{currentfill}{rgb}{0.208623,0.367752,0.552675}%
\pgfsetfillcolor{currentfill}%
\pgfsetfillopacity{0.700000}%
\pgfsetlinewidth{0.000000pt}%
\definecolor{currentstroke}{rgb}{0.000000,0.000000,0.000000}%
\pgfsetstrokecolor{currentstroke}%
\pgfsetdash{}{0pt}%
\pgfpathmoveto{\pgfqpoint{4.573627in}{2.440371in}}%
\pgfpathlineto{\pgfqpoint{4.587208in}{2.447708in}}%
\pgfpathlineto{\pgfqpoint{4.600802in}{2.455205in}}%
\pgfpathlineto{\pgfqpoint{4.614411in}{2.462863in}}%
\pgfpathlineto{\pgfqpoint{4.628033in}{2.470683in}}%
\pgfpathlineto{\pgfqpoint{4.635559in}{2.479638in}}%
\pgfpathlineto{\pgfqpoint{4.643079in}{2.488501in}}%
\pgfpathlineto{\pgfqpoint{4.650592in}{2.497271in}}%
\pgfpathlineto{\pgfqpoint{4.658100in}{2.505950in}}%
\pgfpathlineto{\pgfqpoint{4.644484in}{2.498199in}}%
\pgfpathlineto{\pgfqpoint{4.630882in}{2.490609in}}%
\pgfpathlineto{\pgfqpoint{4.617293in}{2.483179in}}%
\pgfpathlineto{\pgfqpoint{4.603719in}{2.475911in}}%
\pgfpathlineto{\pgfqpoint{4.596205in}{2.467153in}}%
\pgfpathlineto{\pgfqpoint{4.588685in}{2.458311in}}%
\pgfpathlineto{\pgfqpoint{4.581159in}{2.449384in}}%
\pgfpathlineto{\pgfqpoint{4.573627in}{2.440371in}}%
\pgfpathclose%
\pgfusepath{fill}%
\end{pgfscope}%
\begin{pgfscope}%
\pgfpathrectangle{\pgfqpoint{1.254980in}{0.150000in}}{\pgfqpoint{5.490039in}{5.490039in}}%
\pgfusepath{clip}%
\pgfsetbuttcap%
\pgfsetroundjoin%
\definecolor{currentfill}{rgb}{0.221989,0.339161,0.548752}%
\pgfsetfillcolor{currentfill}%
\pgfsetfillopacity{0.700000}%
\pgfsetlinewidth{0.000000pt}%
\definecolor{currentstroke}{rgb}{0.000000,0.000000,0.000000}%
\pgfsetstrokecolor{currentstroke}%
\pgfsetdash{}{0pt}%
\pgfpathmoveto{\pgfqpoint{2.414103in}{2.440414in}}%
\pgfpathlineto{\pgfqpoint{2.427620in}{2.420444in}}%
\pgfpathlineto{\pgfqpoint{2.441126in}{2.400753in}}%
\pgfpathlineto{\pgfqpoint{2.454621in}{2.381339in}}%
\pgfpathlineto{\pgfqpoint{2.468105in}{2.362201in}}%
\pgfpathlineto{\pgfqpoint{2.476663in}{2.362179in}}%
\pgfpathlineto{\pgfqpoint{2.485205in}{2.362397in}}%
\pgfpathlineto{\pgfqpoint{2.493731in}{2.362852in}}%
\pgfpathlineto{\pgfqpoint{2.502240in}{2.363538in}}%
\pgfpathlineto{\pgfqpoint{2.488800in}{2.382207in}}%
\pgfpathlineto{\pgfqpoint{2.475349in}{2.401150in}}%
\pgfpathlineto{\pgfqpoint{2.461888in}{2.420370in}}%
\pgfpathlineto{\pgfqpoint{2.448416in}{2.439869in}}%
\pgfpathlineto{\pgfqpoint{2.439863in}{2.439642in}}%
\pgfpathlineto{\pgfqpoint{2.431294in}{2.439654in}}%
\pgfpathlineto{\pgfqpoint{2.422707in}{2.439910in}}%
\pgfpathlineto{\pgfqpoint{2.414103in}{2.440414in}}%
\pgfpathclose%
\pgfusepath{fill}%
\end{pgfscope}%
\begin{pgfscope}%
\pgfpathrectangle{\pgfqpoint{1.254980in}{0.150000in}}{\pgfqpoint{5.490039in}{5.490039in}}%
\pgfusepath{clip}%
\pgfsetbuttcap%
\pgfsetroundjoin%
\definecolor{currentfill}{rgb}{0.159194,0.482237,0.558073}%
\pgfsetfillcolor{currentfill}%
\pgfsetfillopacity{0.700000}%
\pgfsetlinewidth{0.000000pt}%
\definecolor{currentstroke}{rgb}{0.000000,0.000000,0.000000}%
\pgfsetstrokecolor{currentstroke}%
\pgfsetdash{}{0pt}%
\pgfpathmoveto{\pgfqpoint{4.941493in}{2.732398in}}%
\pgfpathlineto{\pgfqpoint{4.955271in}{2.741661in}}%
\pgfpathlineto{\pgfqpoint{4.969064in}{2.751082in}}%
\pgfpathlineto{\pgfqpoint{4.982873in}{2.760664in}}%
\pgfpathlineto{\pgfqpoint{4.996699in}{2.770404in}}%
\pgfpathlineto{\pgfqpoint{5.004057in}{2.776970in}}%
\pgfpathlineto{\pgfqpoint{5.011407in}{2.783450in}}%
\pgfpathlineto{\pgfqpoint{5.018750in}{2.789846in}}%
\pgfpathlineto{\pgfqpoint{5.026085in}{2.796162in}}%
\pgfpathlineto{\pgfqpoint{5.012271in}{2.786639in}}%
\pgfpathlineto{\pgfqpoint{4.998473in}{2.777275in}}%
\pgfpathlineto{\pgfqpoint{4.984691in}{2.768070in}}%
\pgfpathlineto{\pgfqpoint{4.970925in}{2.759023in}}%
\pgfpathlineto{\pgfqpoint{4.963577in}{2.752480in}}%
\pgfpathlineto{\pgfqpoint{4.956223in}{2.745863in}}%
\pgfpathlineto{\pgfqpoint{4.948862in}{2.739170in}}%
\pgfpathlineto{\pgfqpoint{4.941493in}{2.732398in}}%
\pgfpathclose%
\pgfusepath{fill}%
\end{pgfscope}%
\begin{pgfscope}%
\pgfpathrectangle{\pgfqpoint{1.254980in}{0.150000in}}{\pgfqpoint{5.490039in}{5.490039in}}%
\pgfusepath{clip}%
\pgfsetbuttcap%
\pgfsetroundjoin%
\definecolor{currentfill}{rgb}{0.283091,0.110553,0.431554}%
\pgfsetfillcolor{currentfill}%
\pgfsetfillopacity{0.700000}%
\pgfsetlinewidth{0.000000pt}%
\definecolor{currentstroke}{rgb}{0.000000,0.000000,0.000000}%
\pgfsetstrokecolor{currentstroke}%
\pgfsetdash{}{0pt}%
\pgfpathmoveto{\pgfqpoint{2.842380in}{1.928983in}}%
\pgfpathlineto{\pgfqpoint{2.855664in}{1.916898in}}%
\pgfpathlineto{\pgfqpoint{2.868946in}{1.905030in}}%
\pgfpathlineto{\pgfqpoint{2.882224in}{1.893377in}}%
\pgfpathlineto{\pgfqpoint{2.895499in}{1.881939in}}%
\pgfpathlineto{\pgfqpoint{2.903753in}{1.885660in}}%
\pgfpathlineto{\pgfqpoint{2.911996in}{1.889562in}}%
\pgfpathlineto{\pgfqpoint{2.920227in}{1.893638in}}%
\pgfpathlineto{\pgfqpoint{2.928446in}{1.897885in}}%
\pgfpathlineto{\pgfqpoint{2.915203in}{1.908876in}}%
\pgfpathlineto{\pgfqpoint{2.901956in}{1.920081in}}%
\pgfpathlineto{\pgfqpoint{2.888707in}{1.931501in}}%
\pgfpathlineto{\pgfqpoint{2.875454in}{1.943137in}}%
\pgfpathlineto{\pgfqpoint{2.867204in}{1.939326in}}%
\pgfpathlineto{\pgfqpoint{2.858941in}{1.935694in}}%
\pgfpathlineto{\pgfqpoint{2.850667in}{1.932245in}}%
\pgfpathlineto{\pgfqpoint{2.842380in}{1.928983in}}%
\pgfpathclose%
\pgfusepath{fill}%
\end{pgfscope}%
\begin{pgfscope}%
\pgfpathrectangle{\pgfqpoint{1.254980in}{0.150000in}}{\pgfqpoint{5.490039in}{5.490039in}}%
\pgfusepath{clip}%
\pgfsetbuttcap%
\pgfsetroundjoin%
\definecolor{currentfill}{rgb}{0.120638,0.625828,0.533488}%
\pgfsetfillcolor{currentfill}%
\pgfsetfillopacity{0.700000}%
\pgfsetlinewidth{0.000000pt}%
\definecolor{currentstroke}{rgb}{0.000000,0.000000,0.000000}%
\pgfsetstrokecolor{currentstroke}%
\pgfsetdash{}{0pt}%
\pgfpathmoveto{\pgfqpoint{5.477993in}{3.116949in}}%
\pgfpathlineto{\pgfqpoint{5.492071in}{3.127851in}}%
\pgfpathlineto{\pgfqpoint{5.506168in}{3.138910in}}%
\pgfpathlineto{\pgfqpoint{5.520283in}{3.150125in}}%
\pgfpathlineto{\pgfqpoint{5.534417in}{3.161498in}}%
\pgfpathlineto{\pgfqpoint{5.541464in}{3.164407in}}%
\pgfpathlineto{\pgfqpoint{5.548505in}{3.167290in}}%
\pgfpathlineto{\pgfqpoint{5.555538in}{3.170150in}}%
\pgfpathlineto{\pgfqpoint{5.562564in}{3.172992in}}%
\pgfpathlineto{\pgfqpoint{5.548454in}{3.162049in}}%
\pgfpathlineto{\pgfqpoint{5.534362in}{3.151263in}}%
\pgfpathlineto{\pgfqpoint{5.520288in}{3.140632in}}%
\pgfpathlineto{\pgfqpoint{5.506233in}{3.130158in}}%
\pgfpathlineto{\pgfqpoint{5.499183in}{3.126877in}}%
\pgfpathlineto{\pgfqpoint{5.492126in}{3.123584in}}%
\pgfpathlineto{\pgfqpoint{5.485063in}{3.120277in}}%
\pgfpathlineto{\pgfqpoint{5.477993in}{3.116949in}}%
\pgfpathclose%
\pgfusepath{fill}%
\end{pgfscope}%
\begin{pgfscope}%
\pgfpathrectangle{\pgfqpoint{1.254980in}{0.150000in}}{\pgfqpoint{5.490039in}{5.490039in}}%
\pgfusepath{clip}%
\pgfsetbuttcap%
\pgfsetroundjoin%
\definecolor{currentfill}{rgb}{0.269944,0.014625,0.341379}%
\pgfsetfillcolor{currentfill}%
\pgfsetfillopacity{0.700000}%
\pgfsetlinewidth{0.000000pt}%
\definecolor{currentstroke}{rgb}{0.000000,0.000000,0.000000}%
\pgfsetstrokecolor{currentstroke}%
\pgfsetdash{}{0pt}%
\pgfpathmoveto{\pgfqpoint{3.362995in}{1.733941in}}%
\pgfpathlineto{\pgfqpoint{3.376203in}{1.729433in}}%
\pgfpathlineto{\pgfqpoint{3.389414in}{1.725107in}}%
\pgfpathlineto{\pgfqpoint{3.402629in}{1.720961in}}%
\pgfpathlineto{\pgfqpoint{3.415847in}{1.716996in}}%
\pgfpathlineto{\pgfqpoint{3.423803in}{1.725543in}}%
\pgfpathlineto{\pgfqpoint{3.431753in}{1.734166in}}%
\pgfpathlineto{\pgfqpoint{3.439695in}{1.742861in}}%
\pgfpathlineto{\pgfqpoint{3.447631in}{1.751626in}}%
\pgfpathlineto{\pgfqpoint{3.434429in}{1.755236in}}%
\pgfpathlineto{\pgfqpoint{3.421231in}{1.759026in}}%
\pgfpathlineto{\pgfqpoint{3.408037in}{1.762997in}}%
\pgfpathlineto{\pgfqpoint{3.394846in}{1.767150in}}%
\pgfpathlineto{\pgfqpoint{3.386893in}{1.758730in}}%
\pgfpathlineto{\pgfqpoint{3.378934in}{1.750386in}}%
\pgfpathlineto{\pgfqpoint{3.370968in}{1.742122in}}%
\pgfpathlineto{\pgfqpoint{3.362995in}{1.733941in}}%
\pgfpathclose%
\pgfusepath{fill}%
\end{pgfscope}%
\begin{pgfscope}%
\pgfpathrectangle{\pgfqpoint{1.254980in}{0.150000in}}{\pgfqpoint{5.490039in}{5.490039in}}%
\pgfusepath{clip}%
\pgfsetbuttcap%
\pgfsetroundjoin%
\definecolor{currentfill}{rgb}{0.252194,0.269783,0.531579}%
\pgfsetfillcolor{currentfill}%
\pgfsetfillopacity{0.700000}%
\pgfsetlinewidth{0.000000pt}%
\definecolor{currentstroke}{rgb}{0.000000,0.000000,0.000000}%
\pgfsetstrokecolor{currentstroke}%
\pgfsetdash{}{0pt}%
\pgfpathmoveto{\pgfqpoint{4.290123in}{2.208687in}}%
\pgfpathlineto{\pgfqpoint{4.303571in}{2.214059in}}%
\pgfpathlineto{\pgfqpoint{4.317031in}{2.219594in}}%
\pgfpathlineto{\pgfqpoint{4.330503in}{2.225293in}}%
\pgfpathlineto{\pgfqpoint{4.343987in}{2.231153in}}%
\pgfpathlineto{\pgfqpoint{4.351620in}{2.241511in}}%
\pgfpathlineto{\pgfqpoint{4.359248in}{2.251789in}}%
\pgfpathlineto{\pgfqpoint{4.366870in}{2.261989in}}%
\pgfpathlineto{\pgfqpoint{4.374487in}{2.272110in}}%
\pgfpathlineto{\pgfqpoint{4.361008in}{2.266201in}}%
\pgfpathlineto{\pgfqpoint{4.347540in}{2.260455in}}%
\pgfpathlineto{\pgfqpoint{4.334085in}{2.254872in}}%
\pgfpathlineto{\pgfqpoint{4.320642in}{2.249452in}}%
\pgfpathlineto{\pgfqpoint{4.313020in}{2.239368in}}%
\pgfpathlineto{\pgfqpoint{4.305393in}{2.229213in}}%
\pgfpathlineto{\pgfqpoint{4.297761in}{2.218986in}}%
\pgfpathlineto{\pgfqpoint{4.290123in}{2.208687in}}%
\pgfpathclose%
\pgfusepath{fill}%
\end{pgfscope}%
\begin{pgfscope}%
\pgfpathrectangle{\pgfqpoint{1.254980in}{0.150000in}}{\pgfqpoint{5.490039in}{5.490039in}}%
\pgfusepath{clip}%
\pgfsetbuttcap%
\pgfsetroundjoin%
\definecolor{currentfill}{rgb}{0.281446,0.084320,0.407414}%
\pgfsetfillcolor{currentfill}%
\pgfsetfillopacity{0.700000}%
\pgfsetlinewidth{0.000000pt}%
\definecolor{currentstroke}{rgb}{0.000000,0.000000,0.000000}%
\pgfsetstrokecolor{currentstroke}%
\pgfsetdash{}{0pt}%
\pgfpathmoveto{\pgfqpoint{3.753678in}{1.841725in}}%
\pgfpathlineto{\pgfqpoint{3.766939in}{1.841983in}}%
\pgfpathlineto{\pgfqpoint{3.780207in}{1.842411in}}%
\pgfpathlineto{\pgfqpoint{3.793483in}{1.843008in}}%
\pgfpathlineto{\pgfqpoint{3.806767in}{1.843774in}}%
\pgfpathlineto{\pgfqpoint{3.814571in}{1.854450in}}%
\pgfpathlineto{\pgfqpoint{3.822370in}{1.865122in}}%
\pgfpathlineto{\pgfqpoint{3.830165in}{1.875788in}}%
\pgfpathlineto{\pgfqpoint{3.837954in}{1.886447in}}%
\pgfpathlineto{\pgfqpoint{3.824679in}{1.885436in}}%
\pgfpathlineto{\pgfqpoint{3.811411in}{1.884595in}}%
\pgfpathlineto{\pgfqpoint{3.798151in}{1.883922in}}%
\pgfpathlineto{\pgfqpoint{3.784899in}{1.883420in}}%
\pgfpathlineto{\pgfqpoint{3.777101in}{1.872995in}}%
\pgfpathlineto{\pgfqpoint{3.769299in}{1.862570in}}%
\pgfpathlineto{\pgfqpoint{3.761491in}{1.852146in}}%
\pgfpathlineto{\pgfqpoint{3.753678in}{1.841725in}}%
\pgfpathclose%
\pgfusepath{fill}%
\end{pgfscope}%
\begin{pgfscope}%
\pgfpathrectangle{\pgfqpoint{1.254980in}{0.150000in}}{\pgfqpoint{5.490039in}{5.490039in}}%
\pgfusepath{clip}%
\pgfsetbuttcap%
\pgfsetroundjoin%
\definecolor{currentfill}{rgb}{0.273809,0.031497,0.358853}%
\pgfsetfillcolor{currentfill}%
\pgfsetfillopacity{0.700000}%
\pgfsetlinewidth{0.000000pt}%
\definecolor{currentstroke}{rgb}{0.000000,0.000000,0.000000}%
\pgfsetstrokecolor{currentstroke}%
\pgfsetdash{}{0pt}%
\pgfpathmoveto{\pgfqpoint{3.087232in}{1.782176in}}%
\pgfpathlineto{\pgfqpoint{3.100458in}{1.773846in}}%
\pgfpathlineto{\pgfqpoint{3.113684in}{1.765711in}}%
\pgfpathlineto{\pgfqpoint{3.126911in}{1.757772in}}%
\pgfpathlineto{\pgfqpoint{3.140137in}{1.750028in}}%
\pgfpathlineto{\pgfqpoint{3.148238in}{1.756147in}}%
\pgfpathlineto{\pgfqpoint{3.156330in}{1.762400in}}%
\pgfpathlineto{\pgfqpoint{3.164412in}{1.768782in}}%
\pgfpathlineto{\pgfqpoint{3.172486in}{1.775290in}}%
\pgfpathlineto{\pgfqpoint{3.159283in}{1.782622in}}%
\pgfpathlineto{\pgfqpoint{3.146081in}{1.790147in}}%
\pgfpathlineto{\pgfqpoint{3.132879in}{1.797867in}}%
\pgfpathlineto{\pgfqpoint{3.119678in}{1.805784in}}%
\pgfpathlineto{\pgfqpoint{3.111580in}{1.799679in}}%
\pgfpathlineto{\pgfqpoint{3.103474in}{1.793706in}}%
\pgfpathlineto{\pgfqpoint{3.095358in}{1.787871in}}%
\pgfpathlineto{\pgfqpoint{3.087232in}{1.782176in}}%
\pgfpathclose%
\pgfusepath{fill}%
\end{pgfscope}%
\begin{pgfscope}%
\pgfpathrectangle{\pgfqpoint{1.254980in}{0.150000in}}{\pgfqpoint{5.490039in}{5.490039in}}%
\pgfusepath{clip}%
\pgfsetbuttcap%
\pgfsetroundjoin%
\definecolor{currentfill}{rgb}{0.283091,0.110553,0.431554}%
\pgfsetfillcolor{currentfill}%
\pgfsetfillopacity{0.700000}%
\pgfsetlinewidth{0.000000pt}%
\definecolor{currentstroke}{rgb}{0.000000,0.000000,0.000000}%
\pgfsetstrokecolor{currentstroke}%
\pgfsetdash{}{0pt}%
\pgfpathmoveto{\pgfqpoint{3.837954in}{1.886447in}}%
\pgfpathlineto{\pgfqpoint{3.851237in}{1.887627in}}%
\pgfpathlineto{\pgfqpoint{3.864529in}{1.888975in}}%
\pgfpathlineto{\pgfqpoint{3.877828in}{1.890491in}}%
\pgfpathlineto{\pgfqpoint{3.891136in}{1.892175in}}%
\pgfpathlineto{\pgfqpoint{3.898913in}{1.903051in}}%
\pgfpathlineto{\pgfqpoint{3.906686in}{1.913909in}}%
\pgfpathlineto{\pgfqpoint{3.914453in}{1.924747in}}%
\pgfpathlineto{\pgfqpoint{3.922216in}{1.935562in}}%
\pgfpathlineto{\pgfqpoint{3.908915in}{1.933662in}}%
\pgfpathlineto{\pgfqpoint{3.895622in}{1.931929in}}%
\pgfpathlineto{\pgfqpoint{3.882338in}{1.930364in}}%
\pgfpathlineto{\pgfqpoint{3.869062in}{1.928968in}}%
\pgfpathlineto{\pgfqpoint{3.861293in}{1.918359in}}%
\pgfpathlineto{\pgfqpoint{3.853518in}{1.907734in}}%
\pgfpathlineto{\pgfqpoint{3.845739in}{1.897097in}}%
\pgfpathlineto{\pgfqpoint{3.837954in}{1.886447in}}%
\pgfpathclose%
\pgfusepath{fill}%
\end{pgfscope}%
\begin{pgfscope}%
\pgfpathrectangle{\pgfqpoint{1.254980in}{0.150000in}}{\pgfqpoint{5.490039in}{5.490039in}}%
\pgfusepath{clip}%
\pgfsetbuttcap%
\pgfsetroundjoin%
\definecolor{currentfill}{rgb}{0.278791,0.062145,0.386592}%
\pgfsetfillcolor{currentfill}%
\pgfsetfillopacity{0.700000}%
\pgfsetlinewidth{0.000000pt}%
\definecolor{currentstroke}{rgb}{0.000000,0.000000,0.000000}%
\pgfsetstrokecolor{currentstroke}%
\pgfsetdash{}{0pt}%
\pgfpathmoveto{\pgfqpoint{3.669361in}{1.801891in}}%
\pgfpathlineto{\pgfqpoint{3.682604in}{1.801192in}}%
\pgfpathlineto{\pgfqpoint{3.695854in}{1.800665in}}%
\pgfpathlineto{\pgfqpoint{3.709111in}{1.800310in}}%
\pgfpathlineto{\pgfqpoint{3.722374in}{1.800125in}}%
\pgfpathlineto{\pgfqpoint{3.730208in}{1.810508in}}%
\pgfpathlineto{\pgfqpoint{3.738037in}{1.820904in}}%
\pgfpathlineto{\pgfqpoint{3.745860in}{1.831311in}}%
\pgfpathlineto{\pgfqpoint{3.753678in}{1.841725in}}%
\pgfpathlineto{\pgfqpoint{3.740424in}{1.841638in}}%
\pgfpathlineto{\pgfqpoint{3.727177in}{1.841721in}}%
\pgfpathlineto{\pgfqpoint{3.713937in}{1.841976in}}%
\pgfpathlineto{\pgfqpoint{3.700704in}{1.842403in}}%
\pgfpathlineto{\pgfqpoint{3.692876in}{1.832250in}}%
\pgfpathlineto{\pgfqpoint{3.685043in}{1.822112in}}%
\pgfpathlineto{\pgfqpoint{3.677205in}{1.811992in}}%
\pgfpathlineto{\pgfqpoint{3.669361in}{1.801891in}}%
\pgfpathclose%
\pgfusepath{fill}%
\end{pgfscope}%
\begin{pgfscope}%
\pgfpathrectangle{\pgfqpoint{1.254980in}{0.150000in}}{\pgfqpoint{5.490039in}{5.490039in}}%
\pgfusepath{clip}%
\pgfsetbuttcap%
\pgfsetroundjoin%
\definecolor{currentfill}{rgb}{0.128087,0.647749,0.523491}%
\pgfsetfillcolor{currentfill}%
\pgfsetfillopacity{0.700000}%
\pgfsetlinewidth{0.000000pt}%
\definecolor{currentstroke}{rgb}{0.000000,0.000000,0.000000}%
\pgfsetstrokecolor{currentstroke}%
\pgfsetdash{}{0pt}%
\pgfpathmoveto{\pgfqpoint{5.562564in}{3.172992in}}%
\pgfpathlineto{\pgfqpoint{5.576694in}{3.184091in}}%
\pgfpathlineto{\pgfqpoint{5.590841in}{3.195347in}}%
\pgfpathlineto{\pgfqpoint{5.605008in}{3.206760in}}%
\pgfpathlineto{\pgfqpoint{5.619194in}{3.218329in}}%
\pgfpathlineto{\pgfqpoint{5.626189in}{3.220710in}}%
\pgfpathlineto{\pgfqpoint{5.633177in}{3.223076in}}%
\pgfpathlineto{\pgfqpoint{5.640158in}{3.225433in}}%
\pgfpathlineto{\pgfqpoint{5.647133in}{3.227785in}}%
\pgfpathlineto{\pgfqpoint{5.632972in}{3.216676in}}%
\pgfpathlineto{\pgfqpoint{5.618831in}{3.205724in}}%
\pgfpathlineto{\pgfqpoint{5.604708in}{3.194927in}}%
\pgfpathlineto{\pgfqpoint{5.590604in}{3.184286in}}%
\pgfpathlineto{\pgfqpoint{5.583603in}{3.181464in}}%
\pgfpathlineto{\pgfqpoint{5.576597in}{3.178644in}}%
\pgfpathlineto{\pgfqpoint{5.569584in}{3.175822in}}%
\pgfpathlineto{\pgfqpoint{5.562564in}{3.172992in}}%
\pgfpathclose%
\pgfusepath{fill}%
\end{pgfscope}%
\begin{pgfscope}%
\pgfpathrectangle{\pgfqpoint{1.254980in}{0.150000in}}{\pgfqpoint{5.490039in}{5.490039in}}%
\pgfusepath{clip}%
\pgfsetbuttcap%
\pgfsetroundjoin%
\definecolor{currentfill}{rgb}{0.204903,0.375746,0.553533}%
\pgfsetfillcolor{currentfill}%
\pgfsetfillopacity{0.700000}%
\pgfsetlinewidth{0.000000pt}%
\definecolor{currentstroke}{rgb}{0.000000,0.000000,0.000000}%
\pgfsetstrokecolor{currentstroke}%
\pgfsetdash{}{0pt}%
\pgfpathmoveto{\pgfqpoint{2.359916in}{2.523152in}}%
\pgfpathlineto{\pgfqpoint{2.373481in}{2.502034in}}%
\pgfpathlineto{\pgfqpoint{2.387034in}{2.481207in}}%
\pgfpathlineto{\pgfqpoint{2.400574in}{2.460668in}}%
\pgfpathlineto{\pgfqpoint{2.414103in}{2.440414in}}%
\pgfpathlineto{\pgfqpoint{2.422707in}{2.439910in}}%
\pgfpathlineto{\pgfqpoint{2.431294in}{2.439654in}}%
\pgfpathlineto{\pgfqpoint{2.439863in}{2.439642in}}%
\pgfpathlineto{\pgfqpoint{2.448416in}{2.439869in}}%
\pgfpathlineto{\pgfqpoint{2.434933in}{2.459650in}}%
\pgfpathlineto{\pgfqpoint{2.421439in}{2.479715in}}%
\pgfpathlineto{\pgfqpoint{2.407933in}{2.500067in}}%
\pgfpathlineto{\pgfqpoint{2.394414in}{2.520709in}}%
\pgfpathlineto{\pgfqpoint{2.385816in}{2.520944in}}%
\pgfpathlineto{\pgfqpoint{2.377200in}{2.521426in}}%
\pgfpathlineto{\pgfqpoint{2.368567in}{2.522161in}}%
\pgfpathlineto{\pgfqpoint{2.359916in}{2.523152in}}%
\pgfpathclose%
\pgfusepath{fill}%
\end{pgfscope}%
\begin{pgfscope}%
\pgfpathrectangle{\pgfqpoint{1.254980in}{0.150000in}}{\pgfqpoint{5.490039in}{5.490039in}}%
\pgfusepath{clip}%
\pgfsetbuttcap%
\pgfsetroundjoin%
\definecolor{currentfill}{rgb}{0.282884,0.135920,0.453427}%
\pgfsetfillcolor{currentfill}%
\pgfsetfillopacity{0.700000}%
\pgfsetlinewidth{0.000000pt}%
\definecolor{currentstroke}{rgb}{0.000000,0.000000,0.000000}%
\pgfsetstrokecolor{currentstroke}%
\pgfsetdash{}{0pt}%
\pgfpathmoveto{\pgfqpoint{3.922216in}{1.935562in}}%
\pgfpathlineto{\pgfqpoint{3.935525in}{1.937630in}}%
\pgfpathlineto{\pgfqpoint{3.948844in}{1.939865in}}%
\pgfpathlineto{\pgfqpoint{3.962171in}{1.942267in}}%
\pgfpathlineto{\pgfqpoint{3.975508in}{1.944834in}}%
\pgfpathlineto{\pgfqpoint{3.983260in}{1.955826in}}%
\pgfpathlineto{\pgfqpoint{3.991006in}{1.966784in}}%
\pgfpathlineto{\pgfqpoint{3.998748in}{1.977708in}}%
\pgfpathlineto{\pgfqpoint{4.006485in}{1.988597in}}%
\pgfpathlineto{\pgfqpoint{3.993154in}{1.985840in}}%
\pgfpathlineto{\pgfqpoint{3.979833in}{1.983249in}}%
\pgfpathlineto{\pgfqpoint{3.966521in}{1.980825in}}%
\pgfpathlineto{\pgfqpoint{3.953218in}{1.978568in}}%
\pgfpathlineto{\pgfqpoint{3.945474in}{1.967858in}}%
\pgfpathlineto{\pgfqpoint{3.937726in}{1.957120in}}%
\pgfpathlineto{\pgfqpoint{3.929973in}{1.946354in}}%
\pgfpathlineto{\pgfqpoint{3.922216in}{1.935562in}}%
\pgfpathclose%
\pgfusepath{fill}%
\end{pgfscope}%
\begin{pgfscope}%
\pgfpathrectangle{\pgfqpoint{1.254980in}{0.150000in}}{\pgfqpoint{5.490039in}{5.490039in}}%
\pgfusepath{clip}%
\pgfsetbuttcap%
\pgfsetroundjoin%
\definecolor{currentfill}{rgb}{0.194100,0.399323,0.555565}%
\pgfsetfillcolor{currentfill}%
\pgfsetfillopacity{0.700000}%
\pgfsetlinewidth{0.000000pt}%
\definecolor{currentstroke}{rgb}{0.000000,0.000000,0.000000}%
\pgfsetstrokecolor{currentstroke}%
\pgfsetdash{}{0pt}%
\pgfpathmoveto{\pgfqpoint{4.658100in}{2.505950in}}%
\pgfpathlineto{\pgfqpoint{4.671730in}{2.513862in}}%
\pgfpathlineto{\pgfqpoint{4.685375in}{2.521934in}}%
\pgfpathlineto{\pgfqpoint{4.699034in}{2.530168in}}%
\pgfpathlineto{\pgfqpoint{4.712708in}{2.538562in}}%
\pgfpathlineto{\pgfqpoint{4.720202in}{2.547064in}}%
\pgfpathlineto{\pgfqpoint{4.727690in}{2.555471in}}%
\pgfpathlineto{\pgfqpoint{4.735171in}{2.563783in}}%
\pgfpathlineto{\pgfqpoint{4.742646in}{2.572003in}}%
\pgfpathlineto{\pgfqpoint{4.728979in}{2.563707in}}%
\pgfpathlineto{\pgfqpoint{4.715327in}{2.555572in}}%
\pgfpathlineto{\pgfqpoint{4.701689in}{2.547597in}}%
\pgfpathlineto{\pgfqpoint{4.688066in}{2.539782in}}%
\pgfpathlineto{\pgfqpoint{4.680584in}{2.531454in}}%
\pgfpathlineto{\pgfqpoint{4.673096in}{2.523040in}}%
\pgfpathlineto{\pgfqpoint{4.665601in}{2.514539in}}%
\pgfpathlineto{\pgfqpoint{4.658100in}{2.505950in}}%
\pgfpathclose%
\pgfusepath{fill}%
\end{pgfscope}%
\begin{pgfscope}%
\pgfpathrectangle{\pgfqpoint{1.254980in}{0.150000in}}{\pgfqpoint{5.490039in}{5.490039in}}%
\pgfusepath{clip}%
\pgfsetbuttcap%
\pgfsetroundjoin%
\definecolor{currentfill}{rgb}{0.274952,0.037752,0.364543}%
\pgfsetfillcolor{currentfill}%
\pgfsetfillopacity{0.700000}%
\pgfsetlinewidth{0.000000pt}%
\definecolor{currentstroke}{rgb}{0.000000,0.000000,0.000000}%
\pgfsetstrokecolor{currentstroke}%
\pgfsetdash{}{0pt}%
\pgfpathmoveto{\pgfqpoint{3.584973in}{1.767461in}}%
\pgfpathlineto{\pgfqpoint{3.598204in}{1.765770in}}%
\pgfpathlineto{\pgfqpoint{3.611441in}{1.764253in}}%
\pgfpathlineto{\pgfqpoint{3.624683in}{1.762910in}}%
\pgfpathlineto{\pgfqpoint{3.637931in}{1.761739in}}%
\pgfpathlineto{\pgfqpoint{3.645797in}{1.771734in}}%
\pgfpathlineto{\pgfqpoint{3.653657in}{1.781760in}}%
\pgfpathlineto{\pgfqpoint{3.661512in}{1.791813in}}%
\pgfpathlineto{\pgfqpoint{3.669361in}{1.801891in}}%
\pgfpathlineto{\pgfqpoint{3.656124in}{1.802762in}}%
\pgfpathlineto{\pgfqpoint{3.642893in}{1.803806in}}%
\pgfpathlineto{\pgfqpoint{3.629668in}{1.805023in}}%
\pgfpathlineto{\pgfqpoint{3.616448in}{1.806414in}}%
\pgfpathlineto{\pgfqpoint{3.608588in}{1.796625in}}%
\pgfpathlineto{\pgfqpoint{3.600722in}{1.786868in}}%
\pgfpathlineto{\pgfqpoint{3.592851in}{1.777146in}}%
\pgfpathlineto{\pgfqpoint{3.584973in}{1.767461in}}%
\pgfpathclose%
\pgfusepath{fill}%
\end{pgfscope}%
\begin{pgfscope}%
\pgfpathrectangle{\pgfqpoint{1.254980in}{0.150000in}}{\pgfqpoint{5.490039in}{5.490039in}}%
\pgfusepath{clip}%
\pgfsetbuttcap%
\pgfsetroundjoin%
\definecolor{currentfill}{rgb}{0.149039,0.508051,0.557250}%
\pgfsetfillcolor{currentfill}%
\pgfsetfillopacity{0.700000}%
\pgfsetlinewidth{0.000000pt}%
\definecolor{currentstroke}{rgb}{0.000000,0.000000,0.000000}%
\pgfsetstrokecolor{currentstroke}%
\pgfsetdash{}{0pt}%
\pgfpathmoveto{\pgfqpoint{5.026085in}{2.796162in}}%
\pgfpathlineto{\pgfqpoint{5.039916in}{2.805844in}}%
\pgfpathlineto{\pgfqpoint{5.053763in}{2.815685in}}%
\pgfpathlineto{\pgfqpoint{5.067626in}{2.825685in}}%
\pgfpathlineto{\pgfqpoint{5.081507in}{2.835844in}}%
\pgfpathlineto{\pgfqpoint{5.088823in}{2.841845in}}%
\pgfpathlineto{\pgfqpoint{5.096132in}{2.847763in}}%
\pgfpathlineto{\pgfqpoint{5.103433in}{2.853602in}}%
\pgfpathlineto{\pgfqpoint{5.110727in}{2.859365in}}%
\pgfpathlineto{\pgfqpoint{5.096859in}{2.849454in}}%
\pgfpathlineto{\pgfqpoint{5.083009in}{2.839702in}}%
\pgfpathlineto{\pgfqpoint{5.069174in}{2.830108in}}%
\pgfpathlineto{\pgfqpoint{5.055356in}{2.820672in}}%
\pgfpathlineto{\pgfqpoint{5.048049in}{2.814652in}}%
\pgfpathlineto{\pgfqpoint{5.040735in}{2.808562in}}%
\pgfpathlineto{\pgfqpoint{5.033414in}{2.802399in}}%
\pgfpathlineto{\pgfqpoint{5.026085in}{2.796162in}}%
\pgfpathclose%
\pgfusepath{fill}%
\end{pgfscope}%
\begin{pgfscope}%
\pgfpathrectangle{\pgfqpoint{1.254980in}{0.150000in}}{\pgfqpoint{5.490039in}{5.490039in}}%
\pgfusepath{clip}%
\pgfsetbuttcap%
\pgfsetroundjoin%
\definecolor{currentfill}{rgb}{0.281924,0.089666,0.412415}%
\pgfsetfillcolor{currentfill}%
\pgfsetfillopacity{0.700000}%
\pgfsetlinewidth{0.000000pt}%
\definecolor{currentstroke}{rgb}{0.000000,0.000000,0.000000}%
\pgfsetstrokecolor{currentstroke}%
\pgfsetdash{}{0pt}%
\pgfpathmoveto{\pgfqpoint{2.895499in}{1.881939in}}%
\pgfpathlineto{\pgfqpoint{2.908771in}{1.870713in}}%
\pgfpathlineto{\pgfqpoint{2.922040in}{1.859698in}}%
\pgfpathlineto{\pgfqpoint{2.935307in}{1.848894in}}%
\pgfpathlineto{\pgfqpoint{2.948572in}{1.838298in}}%
\pgfpathlineto{\pgfqpoint{2.956795in}{1.842477in}}%
\pgfpathlineto{\pgfqpoint{2.965007in}{1.846828in}}%
\pgfpathlineto{\pgfqpoint{2.973208in}{1.851347in}}%
\pgfpathlineto{\pgfqpoint{2.981398in}{1.856030in}}%
\pgfpathlineto{\pgfqpoint{2.968163in}{1.866180in}}%
\pgfpathlineto{\pgfqpoint{2.954927in}{1.876539in}}%
\pgfpathlineto{\pgfqpoint{2.941688in}{1.887107in}}%
\pgfpathlineto{\pgfqpoint{2.928446in}{1.897885in}}%
\pgfpathlineto{\pgfqpoint{2.920227in}{1.893638in}}%
\pgfpathlineto{\pgfqpoint{2.911996in}{1.889562in}}%
\pgfpathlineto{\pgfqpoint{2.903753in}{1.885660in}}%
\pgfpathlineto{\pgfqpoint{2.895499in}{1.881939in}}%
\pgfpathclose%
\pgfusepath{fill}%
\end{pgfscope}%
\begin{pgfscope}%
\pgfpathrectangle{\pgfqpoint{1.254980in}{0.150000in}}{\pgfqpoint{5.490039in}{5.490039in}}%
\pgfusepath{clip}%
\pgfsetbuttcap%
\pgfsetroundjoin%
\definecolor{currentfill}{rgb}{0.140210,0.665859,0.513427}%
\pgfsetfillcolor{currentfill}%
\pgfsetfillopacity{0.700000}%
\pgfsetlinewidth{0.000000pt}%
\definecolor{currentstroke}{rgb}{0.000000,0.000000,0.000000}%
\pgfsetstrokecolor{currentstroke}%
\pgfsetdash{}{0pt}%
\pgfpathmoveto{\pgfqpoint{5.647133in}{3.227785in}}%
\pgfpathlineto{\pgfqpoint{5.661312in}{3.239050in}}%
\pgfpathlineto{\pgfqpoint{5.675511in}{3.250471in}}%
\pgfpathlineto{\pgfqpoint{5.689729in}{3.262048in}}%
\pgfpathlineto{\pgfqpoint{5.703966in}{3.273782in}}%
\pgfpathlineto{\pgfqpoint{5.710907in}{3.275655in}}%
\pgfpathlineto{\pgfqpoint{5.717841in}{3.277527in}}%
\pgfpathlineto{\pgfqpoint{5.724769in}{3.279404in}}%
\pgfpathlineto{\pgfqpoint{5.731691in}{3.281290in}}%
\pgfpathlineto{\pgfqpoint{5.717481in}{3.270048in}}%
\pgfpathlineto{\pgfqpoint{5.703291in}{3.258961in}}%
\pgfpathlineto{\pgfqpoint{5.689120in}{3.248030in}}%
\pgfpathlineto{\pgfqpoint{5.674967in}{3.237254in}}%
\pgfpathlineto{\pgfqpoint{5.668018in}{3.234867in}}%
\pgfpathlineto{\pgfqpoint{5.661062in}{3.232497in}}%
\pgfpathlineto{\pgfqpoint{5.654101in}{3.230138in}}%
\pgfpathlineto{\pgfqpoint{5.647133in}{3.227785in}}%
\pgfpathclose%
\pgfusepath{fill}%
\end{pgfscope}%
\begin{pgfscope}%
\pgfpathrectangle{\pgfqpoint{1.254980in}{0.150000in}}{\pgfqpoint{5.490039in}{5.490039in}}%
\pgfusepath{clip}%
\pgfsetbuttcap%
\pgfsetroundjoin%
\definecolor{currentfill}{rgb}{0.280255,0.165693,0.476498}%
\pgfsetfillcolor{currentfill}%
\pgfsetfillopacity{0.700000}%
\pgfsetlinewidth{0.000000pt}%
\definecolor{currentstroke}{rgb}{0.000000,0.000000,0.000000}%
\pgfsetstrokecolor{currentstroke}%
\pgfsetdash{}{0pt}%
\pgfpathmoveto{\pgfqpoint{4.006485in}{1.988597in}}%
\pgfpathlineto{\pgfqpoint{4.019825in}{1.991520in}}%
\pgfpathlineto{\pgfqpoint{4.033175in}{1.994609in}}%
\pgfpathlineto{\pgfqpoint{4.046534in}{1.997863in}}%
\pgfpathlineto{\pgfqpoint{4.059904in}{2.001283in}}%
\pgfpathlineto{\pgfqpoint{4.067631in}{2.012306in}}%
\pgfpathlineto{\pgfqpoint{4.075353in}{2.023284in}}%
\pgfpathlineto{\pgfqpoint{4.083070in}{2.034215in}}%
\pgfpathlineto{\pgfqpoint{4.090782in}{2.045098in}}%
\pgfpathlineto{\pgfqpoint{4.077418in}{2.041517in}}%
\pgfpathlineto{\pgfqpoint{4.064064in}{2.038101in}}%
\pgfpathlineto{\pgfqpoint{4.050720in}{2.034851in}}%
\pgfpathlineto{\pgfqpoint{4.037385in}{2.031767in}}%
\pgfpathlineto{\pgfqpoint{4.029667in}{2.021035in}}%
\pgfpathlineto{\pgfqpoint{4.021945in}{2.010262in}}%
\pgfpathlineto{\pgfqpoint{4.014217in}{1.999449in}}%
\pgfpathlineto{\pgfqpoint{4.006485in}{1.988597in}}%
\pgfpathclose%
\pgfusepath{fill}%
\end{pgfscope}%
\begin{pgfscope}%
\pgfpathrectangle{\pgfqpoint{1.254980in}{0.150000in}}{\pgfqpoint{5.490039in}{5.490039in}}%
\pgfusepath{clip}%
\pgfsetbuttcap%
\pgfsetroundjoin%
\definecolor{currentfill}{rgb}{0.239346,0.300855,0.540844}%
\pgfsetfillcolor{currentfill}%
\pgfsetfillopacity{0.700000}%
\pgfsetlinewidth{0.000000pt}%
\definecolor{currentstroke}{rgb}{0.000000,0.000000,0.000000}%
\pgfsetstrokecolor{currentstroke}%
\pgfsetdash{}{0pt}%
\pgfpathmoveto{\pgfqpoint{4.374487in}{2.272110in}}%
\pgfpathlineto{\pgfqpoint{4.387979in}{2.278181in}}%
\pgfpathlineto{\pgfqpoint{4.401484in}{2.284415in}}%
\pgfpathlineto{\pgfqpoint{4.415001in}{2.290811in}}%
\pgfpathlineto{\pgfqpoint{4.428530in}{2.297369in}}%
\pgfpathlineto{\pgfqpoint{4.436137in}{2.307441in}}%
\pgfpathlineto{\pgfqpoint{4.443739in}{2.317427in}}%
\pgfpathlineto{\pgfqpoint{4.451334in}{2.327327in}}%
\pgfpathlineto{\pgfqpoint{4.458924in}{2.337142in}}%
\pgfpathlineto{\pgfqpoint{4.445399in}{2.330565in}}%
\pgfpathlineto{\pgfqpoint{4.431887in}{2.324150in}}%
\pgfpathlineto{\pgfqpoint{4.418388in}{2.317898in}}%
\pgfpathlineto{\pgfqpoint{4.404901in}{2.311807in}}%
\pgfpathlineto{\pgfqpoint{4.397305in}{2.302001in}}%
\pgfpathlineto{\pgfqpoint{4.389705in}{2.292116in}}%
\pgfpathlineto{\pgfqpoint{4.382099in}{2.282152in}}%
\pgfpathlineto{\pgfqpoint{4.374487in}{2.272110in}}%
\pgfpathclose%
\pgfusepath{fill}%
\end{pgfscope}%
\begin{pgfscope}%
\pgfpathrectangle{\pgfqpoint{1.254980in}{0.150000in}}{\pgfqpoint{5.490039in}{5.490039in}}%
\pgfusepath{clip}%
\pgfsetbuttcap%
\pgfsetroundjoin%
\definecolor{currentfill}{rgb}{0.162016,0.687316,0.499129}%
\pgfsetfillcolor{currentfill}%
\pgfsetfillopacity{0.700000}%
\pgfsetlinewidth{0.000000pt}%
\definecolor{currentstroke}{rgb}{0.000000,0.000000,0.000000}%
\pgfsetstrokecolor{currentstroke}%
\pgfsetdash{}{0pt}%
\pgfpathmoveto{\pgfqpoint{5.731691in}{3.281290in}}%
\pgfpathlineto{\pgfqpoint{5.745919in}{3.292689in}}%
\pgfpathlineto{\pgfqpoint{5.760168in}{3.304243in}}%
\pgfpathlineto{\pgfqpoint{5.774436in}{3.315953in}}%
\pgfpathlineto{\pgfqpoint{5.788723in}{3.327820in}}%
\pgfpathlineto{\pgfqpoint{5.795609in}{3.329211in}}%
\pgfpathlineto{\pgfqpoint{5.802489in}{3.330616in}}%
\pgfpathlineto{\pgfqpoint{5.809363in}{3.332042in}}%
\pgfpathlineto{\pgfqpoint{5.816232in}{3.333493in}}%
\pgfpathlineto{\pgfqpoint{5.801974in}{3.322148in}}%
\pgfpathlineto{\pgfqpoint{5.787736in}{3.310959in}}%
\pgfpathlineto{\pgfqpoint{5.773518in}{3.299925in}}%
\pgfpathlineto{\pgfqpoint{5.759318in}{3.289046in}}%
\pgfpathlineto{\pgfqpoint{5.752420in}{3.287065in}}%
\pgfpathlineto{\pgfqpoint{5.745516in}{3.285115in}}%
\pgfpathlineto{\pgfqpoint{5.738606in}{3.283192in}}%
\pgfpathlineto{\pgfqpoint{5.731691in}{3.281290in}}%
\pgfpathclose%
\pgfusepath{fill}%
\end{pgfscope}%
\begin{pgfscope}%
\pgfpathrectangle{\pgfqpoint{1.254980in}{0.150000in}}{\pgfqpoint{5.490039in}{5.490039in}}%
\pgfusepath{clip}%
\pgfsetbuttcap%
\pgfsetroundjoin%
\definecolor{currentfill}{rgb}{0.246070,0.738910,0.452024}%
\pgfsetfillcolor{currentfill}%
\pgfsetfillopacity{0.700000}%
\pgfsetlinewidth{0.000000pt}%
\definecolor{currentstroke}{rgb}{0.000000,0.000000,0.000000}%
\pgfsetstrokecolor{currentstroke}%
\pgfsetdash{}{0pt}%
\pgfpathmoveto{\pgfqpoint{5.985238in}{3.434027in}}%
\pgfpathlineto{\pgfqpoint{5.999608in}{3.445635in}}%
\pgfpathlineto{\pgfqpoint{6.013998in}{3.457397in}}%
\pgfpathlineto{\pgfqpoint{6.028408in}{3.469314in}}%
\pgfpathlineto{\pgfqpoint{6.035137in}{3.469619in}}%
\pgfpathlineto{\pgfqpoint{6.041862in}{3.469993in}}%
\pgfpathlineto{\pgfqpoint{6.048583in}{3.470442in}}%
\pgfpathlineto{\pgfqpoint{6.055300in}{3.470972in}}%
\pgfpathlineto{\pgfqpoint{6.040927in}{3.459667in}}%
\pgfpathlineto{\pgfqpoint{6.026574in}{3.448516in}}%
\pgfpathlineto{\pgfqpoint{6.012241in}{3.437519in}}%
\pgfpathlineto{\pgfqpoint{6.005495in}{3.436523in}}%
\pgfpathlineto{\pgfqpoint{5.998747in}{3.435614in}}%
\pgfpathlineto{\pgfqpoint{5.991994in}{3.434784in}}%
\pgfpathlineto{\pgfqpoint{5.985238in}{3.434027in}}%
\pgfpathclose%
\pgfusepath{fill}%
\end{pgfscope}%
\begin{pgfscope}%
\pgfpathrectangle{\pgfqpoint{1.254980in}{0.150000in}}{\pgfqpoint{5.490039in}{5.490039in}}%
\pgfusepath{clip}%
\pgfsetbuttcap%
\pgfsetroundjoin%
\definecolor{currentfill}{rgb}{0.272594,0.025563,0.353093}%
\pgfsetfillcolor{currentfill}%
\pgfsetfillopacity{0.700000}%
\pgfsetlinewidth{0.000000pt}%
\definecolor{currentstroke}{rgb}{0.000000,0.000000,0.000000}%
\pgfsetstrokecolor{currentstroke}%
\pgfsetdash{}{0pt}%
\pgfpathmoveto{\pgfqpoint{3.500481in}{1.738974in}}%
\pgfpathlineto{\pgfqpoint{3.513705in}{1.736255in}}%
\pgfpathlineto{\pgfqpoint{3.526933in}{1.733712in}}%
\pgfpathlineto{\pgfqpoint{3.540166in}{1.731345in}}%
\pgfpathlineto{\pgfqpoint{3.553405in}{1.729152in}}%
\pgfpathlineto{\pgfqpoint{3.561306in}{1.738659in}}%
\pgfpathlineto{\pgfqpoint{3.569201in}{1.748214in}}%
\pgfpathlineto{\pgfqpoint{3.577090in}{1.757816in}}%
\pgfpathlineto{\pgfqpoint{3.584973in}{1.767461in}}%
\pgfpathlineto{\pgfqpoint{3.571748in}{1.769326in}}%
\pgfpathlineto{\pgfqpoint{3.558528in}{1.771366in}}%
\pgfpathlineto{\pgfqpoint{3.545313in}{1.773581in}}%
\pgfpathlineto{\pgfqpoint{3.532103in}{1.775973in}}%
\pgfpathlineto{\pgfqpoint{3.524207in}{1.766646in}}%
\pgfpathlineto{\pgfqpoint{3.516304in}{1.757368in}}%
\pgfpathlineto{\pgfqpoint{3.508396in}{1.748143in}}%
\pgfpathlineto{\pgfqpoint{3.500481in}{1.738974in}}%
\pgfpathclose%
\pgfusepath{fill}%
\end{pgfscope}%
\begin{pgfscope}%
\pgfpathrectangle{\pgfqpoint{1.254980in}{0.150000in}}{\pgfqpoint{5.490039in}{5.490039in}}%
\pgfusepath{clip}%
\pgfsetbuttcap%
\pgfsetroundjoin%
\definecolor{currentfill}{rgb}{0.139147,0.533812,0.555298}%
\pgfsetfillcolor{currentfill}%
\pgfsetfillopacity{0.700000}%
\pgfsetlinewidth{0.000000pt}%
\definecolor{currentstroke}{rgb}{0.000000,0.000000,0.000000}%
\pgfsetstrokecolor{currentstroke}%
\pgfsetdash{}{0pt}%
\pgfpathmoveto{\pgfqpoint{5.110727in}{2.859365in}}%
\pgfpathlineto{\pgfqpoint{5.124611in}{2.869435in}}%
\pgfpathlineto{\pgfqpoint{5.138512in}{2.879663in}}%
\pgfpathlineto{\pgfqpoint{5.152430in}{2.890050in}}%
\pgfpathlineto{\pgfqpoint{5.166365in}{2.900596in}}%
\pgfpathlineto{\pgfqpoint{5.173638in}{2.906018in}}%
\pgfpathlineto{\pgfqpoint{5.180903in}{2.911362in}}%
\pgfpathlineto{\pgfqpoint{5.188161in}{2.916632in}}%
\pgfpathlineto{\pgfqpoint{5.195411in}{2.921831in}}%
\pgfpathlineto{\pgfqpoint{5.181490in}{2.911564in}}%
\pgfpathlineto{\pgfqpoint{5.167586in}{2.901455in}}%
\pgfpathlineto{\pgfqpoint{5.153699in}{2.891505in}}%
\pgfpathlineto{\pgfqpoint{5.139829in}{2.881712in}}%
\pgfpathlineto{\pgfqpoint{5.132564in}{2.876225in}}%
\pgfpathlineto{\pgfqpoint{5.125292in}{2.870673in}}%
\pgfpathlineto{\pgfqpoint{5.118013in}{2.865054in}}%
\pgfpathlineto{\pgfqpoint{5.110727in}{2.859365in}}%
\pgfpathclose%
\pgfusepath{fill}%
\end{pgfscope}%
\begin{pgfscope}%
\pgfpathrectangle{\pgfqpoint{1.254980in}{0.150000in}}{\pgfqpoint{5.490039in}{5.490039in}}%
\pgfusepath{clip}%
\pgfsetbuttcap%
\pgfsetroundjoin%
\definecolor{currentfill}{rgb}{0.268510,0.009605,0.335427}%
\pgfsetfillcolor{currentfill}%
\pgfsetfillopacity{0.700000}%
\pgfsetlinewidth{0.000000pt}%
\definecolor{currentstroke}{rgb}{0.000000,0.000000,0.000000}%
\pgfsetstrokecolor{currentstroke}%
\pgfsetdash{}{0pt}%
\pgfpathmoveto{\pgfqpoint{3.278151in}{1.723515in}}%
\pgfpathlineto{\pgfqpoint{3.291367in}{1.717889in}}%
\pgfpathlineto{\pgfqpoint{3.304585in}{1.712447in}}%
\pgfpathlineto{\pgfqpoint{3.317806in}{1.707190in}}%
\pgfpathlineto{\pgfqpoint{3.331029in}{1.702116in}}%
\pgfpathlineto{\pgfqpoint{3.339032in}{1.709930in}}%
\pgfpathlineto{\pgfqpoint{3.347027in}{1.717842in}}%
\pgfpathlineto{\pgfqpoint{3.355015in}{1.725846in}}%
\pgfpathlineto{\pgfqpoint{3.362995in}{1.733941in}}%
\pgfpathlineto{\pgfqpoint{3.349790in}{1.738632in}}%
\pgfpathlineto{\pgfqpoint{3.336588in}{1.743506in}}%
\pgfpathlineto{\pgfqpoint{3.323389in}{1.748563in}}%
\pgfpathlineto{\pgfqpoint{3.310193in}{1.753806in}}%
\pgfpathlineto{\pgfqpoint{3.302194in}{1.746085in}}%
\pgfpathlineto{\pgfqpoint{3.294187in}{1.738460in}}%
\pgfpathlineto{\pgfqpoint{3.286173in}{1.730936in}}%
\pgfpathlineto{\pgfqpoint{3.278151in}{1.723515in}}%
\pgfpathclose%
\pgfusepath{fill}%
\end{pgfscope}%
\begin{pgfscope}%
\pgfpathrectangle{\pgfqpoint{1.254980in}{0.150000in}}{\pgfqpoint{5.490039in}{5.490039in}}%
\pgfusepath{clip}%
\pgfsetbuttcap%
\pgfsetroundjoin%
\definecolor{currentfill}{rgb}{0.185783,0.704891,0.485273}%
\pgfsetfillcolor{currentfill}%
\pgfsetfillopacity{0.700000}%
\pgfsetlinewidth{0.000000pt}%
\definecolor{currentstroke}{rgb}{0.000000,0.000000,0.000000}%
\pgfsetstrokecolor{currentstroke}%
\pgfsetdash{}{0pt}%
\pgfpathmoveto{\pgfqpoint{5.816232in}{3.333493in}}%
\pgfpathlineto{\pgfqpoint{5.830509in}{3.344993in}}%
\pgfpathlineto{\pgfqpoint{5.844805in}{3.356648in}}%
\pgfpathlineto{\pgfqpoint{5.859122in}{3.368459in}}%
\pgfpathlineto{\pgfqpoint{5.873459in}{3.380426in}}%
\pgfpathlineto{\pgfqpoint{5.880290in}{3.381367in}}%
\pgfpathlineto{\pgfqpoint{5.887115in}{3.382338in}}%
\pgfpathlineto{\pgfqpoint{5.893935in}{3.383346in}}%
\pgfpathlineto{\pgfqpoint{5.900749in}{3.384397in}}%
\pgfpathlineto{\pgfqpoint{5.886445in}{3.372982in}}%
\pgfpathlineto{\pgfqpoint{5.872160in}{3.361723in}}%
\pgfpathlineto{\pgfqpoint{5.857896in}{3.350618in}}%
\pgfpathlineto{\pgfqpoint{5.843650in}{3.339668in}}%
\pgfpathlineto{\pgfqpoint{5.836803in}{3.338057in}}%
\pgfpathlineto{\pgfqpoint{5.829951in}{3.336494in}}%
\pgfpathlineto{\pgfqpoint{5.823094in}{3.334975in}}%
\pgfpathlineto{\pgfqpoint{5.816232in}{3.333493in}}%
\pgfpathclose%
\pgfusepath{fill}%
\end{pgfscope}%
\begin{pgfscope}%
\pgfpathrectangle{\pgfqpoint{1.254980in}{0.150000in}}{\pgfqpoint{5.490039in}{5.490039in}}%
\pgfusepath{clip}%
\pgfsetbuttcap%
\pgfsetroundjoin%
\definecolor{currentfill}{rgb}{0.188923,0.410910,0.556326}%
\pgfsetfillcolor{currentfill}%
\pgfsetfillopacity{0.700000}%
\pgfsetlinewidth{0.000000pt}%
\definecolor{currentstroke}{rgb}{0.000000,0.000000,0.000000}%
\pgfsetstrokecolor{currentstroke}%
\pgfsetdash{}{0pt}%
\pgfpathmoveto{\pgfqpoint{2.305523in}{2.610589in}}%
\pgfpathlineto{\pgfqpoint{2.319142in}{2.588279in}}%
\pgfpathlineto{\pgfqpoint{2.332746in}{2.566271in}}%
\pgfpathlineto{\pgfqpoint{2.346338in}{2.544563in}}%
\pgfpathlineto{\pgfqpoint{2.359916in}{2.523152in}}%
\pgfpathlineto{\pgfqpoint{2.368567in}{2.522161in}}%
\pgfpathlineto{\pgfqpoint{2.377200in}{2.521426in}}%
\pgfpathlineto{\pgfqpoint{2.385816in}{2.520944in}}%
\pgfpathlineto{\pgfqpoint{2.394414in}{2.520709in}}%
\pgfpathlineto{\pgfqpoint{2.380884in}{2.541643in}}%
\pgfpathlineto{\pgfqpoint{2.367340in}{2.562873in}}%
\pgfpathlineto{\pgfqpoint{2.353784in}{2.584400in}}%
\pgfpathlineto{\pgfqpoint{2.340214in}{2.606230in}}%
\pgfpathlineto{\pgfqpoint{2.331569in}{2.606932in}}%
\pgfpathlineto{\pgfqpoint{2.322905in}{2.607889in}}%
\pgfpathlineto{\pgfqpoint{2.314224in}{2.609107in}}%
\pgfpathlineto{\pgfqpoint{2.305523in}{2.610589in}}%
\pgfpathclose%
\pgfusepath{fill}%
\end{pgfscope}%
\begin{pgfscope}%
\pgfpathrectangle{\pgfqpoint{1.254980in}{0.150000in}}{\pgfqpoint{5.490039in}{5.490039in}}%
\pgfusepath{clip}%
\pgfsetbuttcap%
\pgfsetroundjoin%
\definecolor{currentfill}{rgb}{0.274128,0.199721,0.498911}%
\pgfsetfillcolor{currentfill}%
\pgfsetfillopacity{0.700000}%
\pgfsetlinewidth{0.000000pt}%
\definecolor{currentstroke}{rgb}{0.000000,0.000000,0.000000}%
\pgfsetstrokecolor{currentstroke}%
\pgfsetdash{}{0pt}%
\pgfpathmoveto{\pgfqpoint{4.090782in}{2.045098in}}%
\pgfpathlineto{\pgfqpoint{4.104156in}{2.048844in}}%
\pgfpathlineto{\pgfqpoint{4.117541in}{2.052754in}}%
\pgfpathlineto{\pgfqpoint{4.130936in}{2.056829in}}%
\pgfpathlineto{\pgfqpoint{4.144342in}{2.061068in}}%
\pgfpathlineto{\pgfqpoint{4.152044in}{2.072046in}}%
\pgfpathlineto{\pgfqpoint{4.159742in}{2.082966in}}%
\pgfpathlineto{\pgfqpoint{4.167435in}{2.093828in}}%
\pgfpathlineto{\pgfqpoint{4.175123in}{2.104631in}}%
\pgfpathlineto{\pgfqpoint{4.161722in}{2.100258in}}%
\pgfpathlineto{\pgfqpoint{4.148331in}{2.096050in}}%
\pgfpathlineto{\pgfqpoint{4.134952in}{2.092006in}}%
\pgfpathlineto{\pgfqpoint{4.121583in}{2.088127in}}%
\pgfpathlineto{\pgfqpoint{4.113890in}{2.077447in}}%
\pgfpathlineto{\pgfqpoint{4.106192in}{2.066715in}}%
\pgfpathlineto{\pgfqpoint{4.098490in}{2.055931in}}%
\pgfpathlineto{\pgfqpoint{4.090782in}{2.045098in}}%
\pgfpathclose%
\pgfusepath{fill}%
\end{pgfscope}%
\begin{pgfscope}%
\pgfpathrectangle{\pgfqpoint{1.254980in}{0.150000in}}{\pgfqpoint{5.490039in}{5.490039in}}%
\pgfusepath{clip}%
\pgfsetbuttcap%
\pgfsetroundjoin%
\definecolor{currentfill}{rgb}{0.182256,0.426184,0.557120}%
\pgfsetfillcolor{currentfill}%
\pgfsetfillopacity{0.700000}%
\pgfsetlinewidth{0.000000pt}%
\definecolor{currentstroke}{rgb}{0.000000,0.000000,0.000000}%
\pgfsetstrokecolor{currentstroke}%
\pgfsetdash{}{0pt}%
\pgfpathmoveto{\pgfqpoint{4.742646in}{2.572003in}}%
\pgfpathlineto{\pgfqpoint{4.756327in}{2.580459in}}%
\pgfpathlineto{\pgfqpoint{4.770024in}{2.589075in}}%
\pgfpathlineto{\pgfqpoint{4.783735in}{2.597852in}}%
\pgfpathlineto{\pgfqpoint{4.797462in}{2.606789in}}%
\pgfpathlineto{\pgfqpoint{4.804923in}{2.614801in}}%
\pgfpathlineto{\pgfqpoint{4.812376in}{2.622715in}}%
\pgfpathlineto{\pgfqpoint{4.819823in}{2.630534in}}%
\pgfpathlineto{\pgfqpoint{4.827264in}{2.638260in}}%
\pgfpathlineto{\pgfqpoint{4.813545in}{2.629452in}}%
\pgfpathlineto{\pgfqpoint{4.799841in}{2.620803in}}%
\pgfpathlineto{\pgfqpoint{4.786153in}{2.612314in}}%
\pgfpathlineto{\pgfqpoint{4.772479in}{2.603985in}}%
\pgfpathlineto{\pgfqpoint{4.765031in}{2.596121in}}%
\pgfpathlineto{\pgfqpoint{4.757576in}{2.588170in}}%
\pgfpathlineto{\pgfqpoint{4.750114in}{2.580131in}}%
\pgfpathlineto{\pgfqpoint{4.742646in}{2.572003in}}%
\pgfpathclose%
\pgfusepath{fill}%
\end{pgfscope}%
\begin{pgfscope}%
\pgfpathrectangle{\pgfqpoint{1.254980in}{0.150000in}}{\pgfqpoint{5.490039in}{5.490039in}}%
\pgfusepath{clip}%
\pgfsetbuttcap%
\pgfsetroundjoin%
\definecolor{currentfill}{rgb}{0.272594,0.025563,0.353093}%
\pgfsetfillcolor{currentfill}%
\pgfsetfillopacity{0.700000}%
\pgfsetlinewidth{0.000000pt}%
\definecolor{currentstroke}{rgb}{0.000000,0.000000,0.000000}%
\pgfsetstrokecolor{currentstroke}%
\pgfsetdash{}{0pt}%
\pgfpathmoveto{\pgfqpoint{3.140137in}{1.750028in}}%
\pgfpathlineto{\pgfqpoint{3.153364in}{1.742477in}}%
\pgfpathlineto{\pgfqpoint{3.166592in}{1.735118in}}%
\pgfpathlineto{\pgfqpoint{3.179820in}{1.727950in}}%
\pgfpathlineto{\pgfqpoint{3.193050in}{1.720973in}}%
\pgfpathlineto{\pgfqpoint{3.201127in}{1.727516in}}%
\pgfpathlineto{\pgfqpoint{3.209195in}{1.734185in}}%
\pgfpathlineto{\pgfqpoint{3.217255in}{1.740977in}}%
\pgfpathlineto{\pgfqpoint{3.225306in}{1.747888in}}%
\pgfpathlineto{\pgfqpoint{3.212100in}{1.754452in}}%
\pgfpathlineto{\pgfqpoint{3.198894in}{1.761207in}}%
\pgfpathlineto{\pgfqpoint{3.185689in}{1.768153in}}%
\pgfpathlineto{\pgfqpoint{3.172486in}{1.775290in}}%
\pgfpathlineto{\pgfqpoint{3.164412in}{1.768782in}}%
\pgfpathlineto{\pgfqpoint{3.156330in}{1.762400in}}%
\pgfpathlineto{\pgfqpoint{3.148238in}{1.756147in}}%
\pgfpathlineto{\pgfqpoint{3.140137in}{1.750028in}}%
\pgfpathclose%
\pgfusepath{fill}%
\end{pgfscope}%
\begin{pgfscope}%
\pgfpathrectangle{\pgfqpoint{1.254980in}{0.150000in}}{\pgfqpoint{5.490039in}{5.490039in}}%
\pgfusepath{clip}%
\pgfsetbuttcap%
\pgfsetroundjoin%
\definecolor{currentfill}{rgb}{0.280267,0.073417,0.397163}%
\pgfsetfillcolor{currentfill}%
\pgfsetfillopacity{0.700000}%
\pgfsetlinewidth{0.000000pt}%
\definecolor{currentstroke}{rgb}{0.000000,0.000000,0.000000}%
\pgfsetstrokecolor{currentstroke}%
\pgfsetdash{}{0pt}%
\pgfpathmoveto{\pgfqpoint{2.948572in}{1.838298in}}%
\pgfpathlineto{\pgfqpoint{2.961834in}{1.827909in}}%
\pgfpathlineto{\pgfqpoint{2.975095in}{1.817726in}}%
\pgfpathlineto{\pgfqpoint{2.988354in}{1.807748in}}%
\pgfpathlineto{\pgfqpoint{3.001611in}{1.797974in}}%
\pgfpathlineto{\pgfqpoint{3.009805in}{1.802609in}}%
\pgfpathlineto{\pgfqpoint{3.017988in}{1.807409in}}%
\pgfpathlineto{\pgfqpoint{3.026161in}{1.812370in}}%
\pgfpathlineto{\pgfqpoint{3.034323in}{1.817486in}}%
\pgfpathlineto{\pgfqpoint{3.021094in}{1.826816in}}%
\pgfpathlineto{\pgfqpoint{3.007863in}{1.836349in}}%
\pgfpathlineto{\pgfqpoint{2.994632in}{1.846087in}}%
\pgfpathlineto{\pgfqpoint{2.981398in}{1.856030in}}%
\pgfpathlineto{\pgfqpoint{2.973208in}{1.851347in}}%
\pgfpathlineto{\pgfqpoint{2.965007in}{1.846828in}}%
\pgfpathlineto{\pgfqpoint{2.956795in}{1.842477in}}%
\pgfpathlineto{\pgfqpoint{2.948572in}{1.838298in}}%
\pgfpathclose%
\pgfusepath{fill}%
\end{pgfscope}%
\begin{pgfscope}%
\pgfpathrectangle{\pgfqpoint{1.254980in}{0.150000in}}{\pgfqpoint{5.490039in}{5.490039in}}%
\pgfusepath{clip}%
\pgfsetbuttcap%
\pgfsetroundjoin%
\definecolor{currentfill}{rgb}{0.220124,0.725509,0.466226}%
\pgfsetfillcolor{currentfill}%
\pgfsetfillopacity{0.700000}%
\pgfsetlinewidth{0.000000pt}%
\definecolor{currentstroke}{rgb}{0.000000,0.000000,0.000000}%
\pgfsetstrokecolor{currentstroke}%
\pgfsetdash{}{0pt}%
\pgfpathmoveto{\pgfqpoint{5.900749in}{3.384397in}}%
\pgfpathlineto{\pgfqpoint{5.915073in}{3.395966in}}%
\pgfpathlineto{\pgfqpoint{5.929417in}{3.407691in}}%
\pgfpathlineto{\pgfqpoint{5.943782in}{3.419571in}}%
\pgfpathlineto{\pgfqpoint{5.958166in}{3.431606in}}%
\pgfpathlineto{\pgfqpoint{5.964942in}{3.432133in}}%
\pgfpathlineto{\pgfqpoint{5.971712in}{3.432708in}}%
\pgfpathlineto{\pgfqpoint{5.978477in}{3.433338in}}%
\pgfpathlineto{\pgfqpoint{5.985238in}{3.434027in}}%
\pgfpathlineto{\pgfqpoint{5.970888in}{3.422575in}}%
\pgfpathlineto{\pgfqpoint{5.956559in}{3.411277in}}%
\pgfpathlineto{\pgfqpoint{5.942249in}{3.400133in}}%
\pgfpathlineto{\pgfqpoint{5.927959in}{3.389143in}}%
\pgfpathlineto{\pgfqpoint{5.921163in}{3.387862in}}%
\pgfpathlineto{\pgfqpoint{5.914363in}{3.386649in}}%
\pgfpathlineto{\pgfqpoint{5.907558in}{3.385495in}}%
\pgfpathlineto{\pgfqpoint{5.900749in}{3.384397in}}%
\pgfpathclose%
\pgfusepath{fill}%
\end{pgfscope}%
\begin{pgfscope}%
\pgfpathrectangle{\pgfqpoint{1.254980in}{0.150000in}}{\pgfqpoint{5.490039in}{5.490039in}}%
\pgfusepath{clip}%
\pgfsetbuttcap%
\pgfsetroundjoin%
\definecolor{currentfill}{rgb}{0.223925,0.334994,0.548053}%
\pgfsetfillcolor{currentfill}%
\pgfsetfillopacity{0.700000}%
\pgfsetlinewidth{0.000000pt}%
\definecolor{currentstroke}{rgb}{0.000000,0.000000,0.000000}%
\pgfsetstrokecolor{currentstroke}%
\pgfsetdash{}{0pt}%
\pgfpathmoveto{\pgfqpoint{4.458924in}{2.337142in}}%
\pgfpathlineto{\pgfqpoint{4.472462in}{2.343881in}}%
\pgfpathlineto{\pgfqpoint{4.486013in}{2.350782in}}%
\pgfpathlineto{\pgfqpoint{4.499578in}{2.357844in}}%
\pgfpathlineto{\pgfqpoint{4.513155in}{2.365069in}}%
\pgfpathlineto{\pgfqpoint{4.520735in}{2.374799in}}%
\pgfpathlineto{\pgfqpoint{4.528309in}{2.384438in}}%
\pgfpathlineto{\pgfqpoint{4.535877in}{2.393985in}}%
\pgfpathlineto{\pgfqpoint{4.543439in}{2.403441in}}%
\pgfpathlineto{\pgfqpoint{4.529866in}{2.396227in}}%
\pgfpathlineto{\pgfqpoint{4.516307in}{2.389175in}}%
\pgfpathlineto{\pgfqpoint{4.502761in}{2.382284in}}%
\pgfpathlineto{\pgfqpoint{4.489228in}{2.375555in}}%
\pgfpathlineto{\pgfqpoint{4.481661in}{2.366078in}}%
\pgfpathlineto{\pgfqpoint{4.474088in}{2.356517in}}%
\pgfpathlineto{\pgfqpoint{4.466509in}{2.346872in}}%
\pgfpathlineto{\pgfqpoint{4.458924in}{2.337142in}}%
\pgfpathclose%
\pgfusepath{fill}%
\end{pgfscope}%
\begin{pgfscope}%
\pgfpathrectangle{\pgfqpoint{1.254980in}{0.150000in}}{\pgfqpoint{5.490039in}{5.490039in}}%
\pgfusepath{clip}%
\pgfsetbuttcap%
\pgfsetroundjoin%
\definecolor{currentfill}{rgb}{0.269944,0.014625,0.341379}%
\pgfsetfillcolor{currentfill}%
\pgfsetfillopacity{0.700000}%
\pgfsetlinewidth{0.000000pt}%
\definecolor{currentstroke}{rgb}{0.000000,0.000000,0.000000}%
\pgfsetstrokecolor{currentstroke}%
\pgfsetdash{}{0pt}%
\pgfpathmoveto{\pgfqpoint{3.415847in}{1.716996in}}%
\pgfpathlineto{\pgfqpoint{3.429069in}{1.713211in}}%
\pgfpathlineto{\pgfqpoint{3.442295in}{1.709604in}}%
\pgfpathlineto{\pgfqpoint{3.455525in}{1.706175in}}%
\pgfpathlineto{\pgfqpoint{3.468759in}{1.702924in}}%
\pgfpathlineto{\pgfqpoint{3.476699in}{1.711836in}}%
\pgfpathlineto{\pgfqpoint{3.484633in}{1.720818in}}%
\pgfpathlineto{\pgfqpoint{3.492560in}{1.729865in}}%
\pgfpathlineto{\pgfqpoint{3.500481in}{1.738974in}}%
\pgfpathlineto{\pgfqpoint{3.487262in}{1.741870in}}%
\pgfpathlineto{\pgfqpoint{3.474048in}{1.744944in}}%
\pgfpathlineto{\pgfqpoint{3.460837in}{1.748196in}}%
\pgfpathlineto{\pgfqpoint{3.447631in}{1.751626in}}%
\pgfpathlineto{\pgfqpoint{3.439695in}{1.742861in}}%
\pgfpathlineto{\pgfqpoint{3.431753in}{1.734166in}}%
\pgfpathlineto{\pgfqpoint{3.423803in}{1.725543in}}%
\pgfpathlineto{\pgfqpoint{3.415847in}{1.716996in}}%
\pgfpathclose%
\pgfusepath{fill}%
\end{pgfscope}%
\begin{pgfscope}%
\pgfpathrectangle{\pgfqpoint{1.254980in}{0.150000in}}{\pgfqpoint{5.490039in}{5.490039in}}%
\pgfusepath{clip}%
\pgfsetbuttcap%
\pgfsetroundjoin%
\definecolor{currentfill}{rgb}{0.129933,0.559582,0.551864}%
\pgfsetfillcolor{currentfill}%
\pgfsetfillopacity{0.700000}%
\pgfsetlinewidth{0.000000pt}%
\definecolor{currentstroke}{rgb}{0.000000,0.000000,0.000000}%
\pgfsetstrokecolor{currentstroke}%
\pgfsetdash{}{0pt}%
\pgfpathmoveto{\pgfqpoint{5.195411in}{2.921831in}}%
\pgfpathlineto{\pgfqpoint{5.209349in}{2.932257in}}%
\pgfpathlineto{\pgfqpoint{5.223304in}{2.942841in}}%
\pgfpathlineto{\pgfqpoint{5.237277in}{2.953583in}}%
\pgfpathlineto{\pgfqpoint{5.251268in}{2.964485in}}%
\pgfpathlineto{\pgfqpoint{5.258495in}{2.969318in}}%
\pgfpathlineto{\pgfqpoint{5.265714in}{2.974080in}}%
\pgfpathlineto{\pgfqpoint{5.272926in}{2.978775in}}%
\pgfpathlineto{\pgfqpoint{5.280130in}{2.983405in}}%
\pgfpathlineto{\pgfqpoint{5.266155in}{2.972814in}}%
\pgfpathlineto{\pgfqpoint{5.252198in}{2.962380in}}%
\pgfpathlineto{\pgfqpoint{5.238259in}{2.952104in}}%
\pgfpathlineto{\pgfqpoint{5.224336in}{2.941987in}}%
\pgfpathlineto{\pgfqpoint{5.217116in}{2.937037in}}%
\pgfpathlineto{\pgfqpoint{5.209888in}{2.932030in}}%
\pgfpathlineto{\pgfqpoint{5.202653in}{2.926963in}}%
\pgfpathlineto{\pgfqpoint{5.195411in}{2.921831in}}%
\pgfpathclose%
\pgfusepath{fill}%
\end{pgfscope}%
\begin{pgfscope}%
\pgfpathrectangle{\pgfqpoint{1.254980in}{0.150000in}}{\pgfqpoint{5.490039in}{5.490039in}}%
\pgfusepath{clip}%
\pgfsetbuttcap%
\pgfsetroundjoin%
\definecolor{currentfill}{rgb}{0.266580,0.228262,0.514349}%
\pgfsetfillcolor{currentfill}%
\pgfsetfillopacity{0.700000}%
\pgfsetlinewidth{0.000000pt}%
\definecolor{currentstroke}{rgb}{0.000000,0.000000,0.000000}%
\pgfsetstrokecolor{currentstroke}%
\pgfsetdash{}{0pt}%
\pgfpathmoveto{\pgfqpoint{4.175123in}{2.104631in}}%
\pgfpathlineto{\pgfqpoint{4.188535in}{2.109167in}}%
\pgfpathlineto{\pgfqpoint{4.201958in}{2.113868in}}%
\pgfpathlineto{\pgfqpoint{4.215392in}{2.118732in}}%
\pgfpathlineto{\pgfqpoint{4.228837in}{2.123759in}}%
\pgfpathlineto{\pgfqpoint{4.236516in}{2.134617in}}%
\pgfpathlineto{\pgfqpoint{4.244189in}{2.145407in}}%
\pgfpathlineto{\pgfqpoint{4.251858in}{2.156129in}}%
\pgfpathlineto{\pgfqpoint{4.259521in}{2.166781in}}%
\pgfpathlineto{\pgfqpoint{4.246080in}{2.161649in}}%
\pgfpathlineto{\pgfqpoint{4.232650in}{2.156680in}}%
\pgfpathlineto{\pgfqpoint{4.219232in}{2.151875in}}%
\pgfpathlineto{\pgfqpoint{4.205825in}{2.147233in}}%
\pgfpathlineto{\pgfqpoint{4.198157in}{2.136675in}}%
\pgfpathlineto{\pgfqpoint{4.190484in}{2.126055in}}%
\pgfpathlineto{\pgfqpoint{4.182806in}{2.115373in}}%
\pgfpathlineto{\pgfqpoint{4.175123in}{2.104631in}}%
\pgfpathclose%
\pgfusepath{fill}%
\end{pgfscope}%
\begin{pgfscope}%
\pgfpathrectangle{\pgfqpoint{1.254980in}{0.150000in}}{\pgfqpoint{5.490039in}{5.490039in}}%
\pgfusepath{clip}%
\pgfsetbuttcap%
\pgfsetroundjoin%
\definecolor{currentfill}{rgb}{0.169646,0.456262,0.558030}%
\pgfsetfillcolor{currentfill}%
\pgfsetfillopacity{0.700000}%
\pgfsetlinewidth{0.000000pt}%
\definecolor{currentstroke}{rgb}{0.000000,0.000000,0.000000}%
\pgfsetstrokecolor{currentstroke}%
\pgfsetdash{}{0pt}%
\pgfpathmoveto{\pgfqpoint{4.827264in}{2.638260in}}%
\pgfpathlineto{\pgfqpoint{4.840998in}{2.647229in}}%
\pgfpathlineto{\pgfqpoint{4.854747in}{2.656358in}}%
\pgfpathlineto{\pgfqpoint{4.868512in}{2.665647in}}%
\pgfpathlineto{\pgfqpoint{4.882293in}{2.675096in}}%
\pgfpathlineto{\pgfqpoint{4.889718in}{2.682583in}}%
\pgfpathlineto{\pgfqpoint{4.897135in}{2.689974in}}%
\pgfpathlineto{\pgfqpoint{4.904546in}{2.697269in}}%
\pgfpathlineto{\pgfqpoint{4.911950in}{2.704472in}}%
\pgfpathlineto{\pgfqpoint{4.898178in}{2.695181in}}%
\pgfpathlineto{\pgfqpoint{4.884422in}{2.686050in}}%
\pgfpathlineto{\pgfqpoint{4.870681in}{2.677079in}}%
\pgfpathlineto{\pgfqpoint{4.856956in}{2.668268in}}%
\pgfpathlineto{\pgfqpoint{4.849543in}{2.660896in}}%
\pgfpathlineto{\pgfqpoint{4.842124in}{2.653439in}}%
\pgfpathlineto{\pgfqpoint{4.834697in}{2.645895in}}%
\pgfpathlineto{\pgfqpoint{4.827264in}{2.638260in}}%
\pgfpathclose%
\pgfusepath{fill}%
\end{pgfscope}%
\begin{pgfscope}%
\pgfpathrectangle{\pgfqpoint{1.254980in}{0.150000in}}{\pgfqpoint{5.490039in}{5.490039in}}%
\pgfusepath{clip}%
\pgfsetbuttcap%
\pgfsetroundjoin%
\definecolor{currentfill}{rgb}{0.172719,0.448791,0.557885}%
\pgfsetfillcolor{currentfill}%
\pgfsetfillopacity{0.700000}%
\pgfsetlinewidth{0.000000pt}%
\definecolor{currentstroke}{rgb}{0.000000,0.000000,0.000000}%
\pgfsetstrokecolor{currentstroke}%
\pgfsetdash{}{0pt}%
\pgfpathmoveto{\pgfqpoint{2.250905in}{2.702913in}}%
\pgfpathlineto{\pgfqpoint{2.264582in}{2.679363in}}%
\pgfpathlineto{\pgfqpoint{2.278244in}{2.656128in}}%
\pgfpathlineto{\pgfqpoint{2.291891in}{2.633204in}}%
\pgfpathlineto{\pgfqpoint{2.305523in}{2.610589in}}%
\pgfpathlineto{\pgfqpoint{2.314224in}{2.609107in}}%
\pgfpathlineto{\pgfqpoint{2.322905in}{2.607889in}}%
\pgfpathlineto{\pgfqpoint{2.331569in}{2.606932in}}%
\pgfpathlineto{\pgfqpoint{2.340214in}{2.606230in}}%
\pgfpathlineto{\pgfqpoint{2.326631in}{2.628363in}}%
\pgfpathlineto{\pgfqpoint{2.313034in}{2.650804in}}%
\pgfpathlineto{\pgfqpoint{2.299422in}{2.673555in}}%
\pgfpathlineto{\pgfqpoint{2.285796in}{2.696619in}}%
\pgfpathlineto{\pgfqpoint{2.277102in}{2.697793in}}%
\pgfpathlineto{\pgfqpoint{2.268389in}{2.699230in}}%
\pgfpathlineto{\pgfqpoint{2.259656in}{2.700935in}}%
\pgfpathlineto{\pgfqpoint{2.250905in}{2.702913in}}%
\pgfpathclose%
\pgfusepath{fill}%
\end{pgfscope}%
\begin{pgfscope}%
\pgfpathrectangle{\pgfqpoint{1.254980in}{0.150000in}}{\pgfqpoint{5.490039in}{5.490039in}}%
\pgfusepath{clip}%
\pgfsetbuttcap%
\pgfsetroundjoin%
\definecolor{currentfill}{rgb}{0.277941,0.056324,0.381191}%
\pgfsetfillcolor{currentfill}%
\pgfsetfillopacity{0.700000}%
\pgfsetlinewidth{0.000000pt}%
\definecolor{currentstroke}{rgb}{0.000000,0.000000,0.000000}%
\pgfsetstrokecolor{currentstroke}%
\pgfsetdash{}{0pt}%
\pgfpathmoveto{\pgfqpoint{3.001611in}{1.797974in}}%
\pgfpathlineto{\pgfqpoint{3.014868in}{1.788402in}}%
\pgfpathlineto{\pgfqpoint{3.028123in}{1.779032in}}%
\pgfpathlineto{\pgfqpoint{3.041377in}{1.769861in}}%
\pgfpathlineto{\pgfqpoint{3.054630in}{1.760890in}}%
\pgfpathlineto{\pgfqpoint{3.062796in}{1.765979in}}%
\pgfpathlineto{\pgfqpoint{3.070951in}{1.771226in}}%
\pgfpathlineto{\pgfqpoint{3.079097in}{1.776627in}}%
\pgfpathlineto{\pgfqpoint{3.087232in}{1.782176in}}%
\pgfpathlineto{\pgfqpoint{3.074006in}{1.790705in}}%
\pgfpathlineto{\pgfqpoint{3.060779in}{1.799432in}}%
\pgfpathlineto{\pgfqpoint{3.047551in}{1.808359in}}%
\pgfpathlineto{\pgfqpoint{3.034323in}{1.817486in}}%
\pgfpathlineto{\pgfqpoint{3.026161in}{1.812370in}}%
\pgfpathlineto{\pgfqpoint{3.017988in}{1.807409in}}%
\pgfpathlineto{\pgfqpoint{3.009805in}{1.802609in}}%
\pgfpathlineto{\pgfqpoint{3.001611in}{1.797974in}}%
\pgfpathclose%
\pgfusepath{fill}%
\end{pgfscope}%
\begin{pgfscope}%
\pgfpathrectangle{\pgfqpoint{1.254980in}{0.150000in}}{\pgfqpoint{5.490039in}{5.490039in}}%
\pgfusepath{clip}%
\pgfsetbuttcap%
\pgfsetroundjoin%
\definecolor{currentfill}{rgb}{0.210503,0.363727,0.552206}%
\pgfsetfillcolor{currentfill}%
\pgfsetfillopacity{0.700000}%
\pgfsetlinewidth{0.000000pt}%
\definecolor{currentstroke}{rgb}{0.000000,0.000000,0.000000}%
\pgfsetstrokecolor{currentstroke}%
\pgfsetdash{}{0pt}%
\pgfpathmoveto{\pgfqpoint{4.543439in}{2.403441in}}%
\pgfpathlineto{\pgfqpoint{4.557025in}{2.410817in}}%
\pgfpathlineto{\pgfqpoint{4.570625in}{2.418353in}}%
\pgfpathlineto{\pgfqpoint{4.584239in}{2.426051in}}%
\pgfpathlineto{\pgfqpoint{4.597867in}{2.433910in}}%
\pgfpathlineto{\pgfqpoint{4.605417in}{2.443248in}}%
\pgfpathlineto{\pgfqpoint{4.612962in}{2.452489in}}%
\pgfpathlineto{\pgfqpoint{4.620501in}{2.461633in}}%
\pgfpathlineto{\pgfqpoint{4.628033in}{2.470683in}}%
\pgfpathlineto{\pgfqpoint{4.614411in}{2.462863in}}%
\pgfpathlineto{\pgfqpoint{4.600802in}{2.455205in}}%
\pgfpathlineto{\pgfqpoint{4.587208in}{2.447708in}}%
\pgfpathlineto{\pgfqpoint{4.573627in}{2.440371in}}%
\pgfpathlineto{\pgfqpoint{4.566089in}{2.431272in}}%
\pgfpathlineto{\pgfqpoint{4.558545in}{2.422084in}}%
\pgfpathlineto{\pgfqpoint{4.550995in}{2.412807in}}%
\pgfpathlineto{\pgfqpoint{4.543439in}{2.403441in}}%
\pgfpathclose%
\pgfusepath{fill}%
\end{pgfscope}%
\begin{pgfscope}%
\pgfpathrectangle{\pgfqpoint{1.254980in}{0.150000in}}{\pgfqpoint{5.490039in}{5.490039in}}%
\pgfusepath{clip}%
\pgfsetbuttcap%
\pgfsetroundjoin%
\definecolor{currentfill}{rgb}{0.122606,0.585371,0.546557}%
\pgfsetfillcolor{currentfill}%
\pgfsetfillopacity{0.700000}%
\pgfsetlinewidth{0.000000pt}%
\definecolor{currentstroke}{rgb}{0.000000,0.000000,0.000000}%
\pgfsetstrokecolor{currentstroke}%
\pgfsetdash{}{0pt}%
\pgfpathmoveto{\pgfqpoint{5.280130in}{2.983405in}}%
\pgfpathlineto{\pgfqpoint{5.294122in}{2.994155in}}%
\pgfpathlineto{\pgfqpoint{5.308131in}{3.005062in}}%
\pgfpathlineto{\pgfqpoint{5.322159in}{3.016128in}}%
\pgfpathlineto{\pgfqpoint{5.336205in}{3.027353in}}%
\pgfpathlineto{\pgfqpoint{5.343384in}{3.031594in}}%
\pgfpathlineto{\pgfqpoint{5.350556in}{3.035771in}}%
\pgfpathlineto{\pgfqpoint{5.357720in}{3.039888in}}%
\pgfpathlineto{\pgfqpoint{5.364876in}{3.043949in}}%
\pgfpathlineto{\pgfqpoint{5.350848in}{3.033065in}}%
\pgfpathlineto{\pgfqpoint{5.336838in}{3.022339in}}%
\pgfpathlineto{\pgfqpoint{5.322846in}{3.011770in}}%
\pgfpathlineto{\pgfqpoint{5.308871in}{3.001359in}}%
\pgfpathlineto{\pgfqpoint{5.301697in}{2.996948in}}%
\pgfpathlineto{\pgfqpoint{5.294515in}{2.992488in}}%
\pgfpathlineto{\pgfqpoint{5.287326in}{2.987975in}}%
\pgfpathlineto{\pgfqpoint{5.280130in}{2.983405in}}%
\pgfpathclose%
\pgfusepath{fill}%
\end{pgfscope}%
\begin{pgfscope}%
\pgfpathrectangle{\pgfqpoint{1.254980in}{0.150000in}}{\pgfqpoint{5.490039in}{5.490039in}}%
\pgfusepath{clip}%
\pgfsetbuttcap%
\pgfsetroundjoin%
\definecolor{currentfill}{rgb}{0.267968,0.223549,0.512008}%
\pgfsetfillcolor{currentfill}%
\pgfsetfillopacity{0.700000}%
\pgfsetlinewidth{0.000000pt}%
\definecolor{currentstroke}{rgb}{0.000000,0.000000,0.000000}%
\pgfsetstrokecolor{currentstroke}%
\pgfsetdash{}{0pt}%
\pgfpathmoveto{\pgfqpoint{2.595299in}{2.149777in}}%
\pgfpathlineto{\pgfqpoint{2.608710in}{2.133513in}}%
\pgfpathlineto{\pgfqpoint{2.622113in}{2.117494in}}%
\pgfpathlineto{\pgfqpoint{2.635508in}{2.101718in}}%
\pgfpathlineto{\pgfqpoint{2.648897in}{2.086182in}}%
\pgfpathlineto{\pgfqpoint{2.657349in}{2.087170in}}%
\pgfpathlineto{\pgfqpoint{2.665786in}{2.088385in}}%
\pgfpathlineto{\pgfqpoint{2.674208in}{2.089824in}}%
\pgfpathlineto{\pgfqpoint{2.682616in}{2.091482in}}%
\pgfpathlineto{\pgfqpoint{2.669267in}{2.106531in}}%
\pgfpathlineto{\pgfqpoint{2.655911in}{2.121819in}}%
\pgfpathlineto{\pgfqpoint{2.642549in}{2.137350in}}%
\pgfpathlineto{\pgfqpoint{2.629179in}{2.153125in}}%
\pgfpathlineto{\pgfqpoint{2.620732in}{2.151943in}}%
\pgfpathlineto{\pgfqpoint{2.612270in}{2.150988in}}%
\pgfpathlineto{\pgfqpoint{2.603792in}{2.150265in}}%
\pgfpathlineto{\pgfqpoint{2.595299in}{2.149777in}}%
\pgfpathclose%
\pgfusepath{fill}%
\end{pgfscope}%
\begin{pgfscope}%
\pgfpathrectangle{\pgfqpoint{1.254980in}{0.150000in}}{\pgfqpoint{5.490039in}{5.490039in}}%
\pgfusepath{clip}%
\pgfsetbuttcap%
\pgfsetroundjoin%
\definecolor{currentfill}{rgb}{0.275191,0.194905,0.496005}%
\pgfsetfillcolor{currentfill}%
\pgfsetfillopacity{0.700000}%
\pgfsetlinewidth{0.000000pt}%
\definecolor{currentstroke}{rgb}{0.000000,0.000000,0.000000}%
\pgfsetstrokecolor{currentstroke}%
\pgfsetdash{}{0pt}%
\pgfpathmoveto{\pgfqpoint{2.648897in}{2.086182in}}%
\pgfpathlineto{\pgfqpoint{2.662279in}{2.070886in}}%
\pgfpathlineto{\pgfqpoint{2.675655in}{2.055827in}}%
\pgfpathlineto{\pgfqpoint{2.689024in}{2.041003in}}%
\pgfpathlineto{\pgfqpoint{2.702388in}{2.026412in}}%
\pgfpathlineto{\pgfqpoint{2.710800in}{2.027896in}}%
\pgfpathlineto{\pgfqpoint{2.719198in}{2.029601in}}%
\pgfpathlineto{\pgfqpoint{2.727582in}{2.031521in}}%
\pgfpathlineto{\pgfqpoint{2.735952in}{2.033652in}}%
\pgfpathlineto{\pgfqpoint{2.722627in}{2.047759in}}%
\pgfpathlineto{\pgfqpoint{2.709296in}{2.062098in}}%
\pgfpathlineto{\pgfqpoint{2.695959in}{2.076672in}}%
\pgfpathlineto{\pgfqpoint{2.682616in}{2.091482in}}%
\pgfpathlineto{\pgfqpoint{2.674208in}{2.089824in}}%
\pgfpathlineto{\pgfqpoint{2.665786in}{2.088385in}}%
\pgfpathlineto{\pgfqpoint{2.657349in}{2.087170in}}%
\pgfpathlineto{\pgfqpoint{2.648897in}{2.086182in}}%
\pgfpathclose%
\pgfusepath{fill}%
\end{pgfscope}%
\begin{pgfscope}%
\pgfpathrectangle{\pgfqpoint{1.254980in}{0.150000in}}{\pgfqpoint{5.490039in}{5.490039in}}%
\pgfusepath{clip}%
\pgfsetbuttcap%
\pgfsetroundjoin%
\definecolor{currentfill}{rgb}{0.255645,0.260703,0.528312}%
\pgfsetfillcolor{currentfill}%
\pgfsetfillopacity{0.700000}%
\pgfsetlinewidth{0.000000pt}%
\definecolor{currentstroke}{rgb}{0.000000,0.000000,0.000000}%
\pgfsetstrokecolor{currentstroke}%
\pgfsetdash{}{0pt}%
\pgfpathmoveto{\pgfqpoint{4.259521in}{2.166781in}}%
\pgfpathlineto{\pgfqpoint{4.272974in}{2.172077in}}%
\pgfpathlineto{\pgfqpoint{4.286438in}{2.177536in}}%
\pgfpathlineto{\pgfqpoint{4.299914in}{2.183157in}}%
\pgfpathlineto{\pgfqpoint{4.313402in}{2.188941in}}%
\pgfpathlineto{\pgfqpoint{4.321057in}{2.199611in}}%
\pgfpathlineto{\pgfqpoint{4.328705in}{2.210203in}}%
\pgfpathlineto{\pgfqpoint{4.336349in}{2.220717in}}%
\pgfpathlineto{\pgfqpoint{4.343987in}{2.231153in}}%
\pgfpathlineto{\pgfqpoint{4.330503in}{2.225293in}}%
\pgfpathlineto{\pgfqpoint{4.317031in}{2.219594in}}%
\pgfpathlineto{\pgfqpoint{4.303571in}{2.214059in}}%
\pgfpathlineto{\pgfqpoint{4.290123in}{2.208687in}}%
\pgfpathlineto{\pgfqpoint{4.282480in}{2.198317in}}%
\pgfpathlineto{\pgfqpoint{4.274832in}{2.187876in}}%
\pgfpathlineto{\pgfqpoint{4.267179in}{2.177364in}}%
\pgfpathlineto{\pgfqpoint{4.259521in}{2.166781in}}%
\pgfpathclose%
\pgfusepath{fill}%
\end{pgfscope}%
\begin{pgfscope}%
\pgfpathrectangle{\pgfqpoint{1.254980in}{0.150000in}}{\pgfqpoint{5.490039in}{5.490039in}}%
\pgfusepath{clip}%
\pgfsetbuttcap%
\pgfsetroundjoin%
\definecolor{currentfill}{rgb}{0.257322,0.256130,0.526563}%
\pgfsetfillcolor{currentfill}%
\pgfsetfillopacity{0.700000}%
\pgfsetlinewidth{0.000000pt}%
\definecolor{currentstroke}{rgb}{0.000000,0.000000,0.000000}%
\pgfsetstrokecolor{currentstroke}%
\pgfsetdash{}{0pt}%
\pgfpathmoveto{\pgfqpoint{2.541578in}{2.217319in}}%
\pgfpathlineto{\pgfqpoint{2.555021in}{2.200056in}}%
\pgfpathlineto{\pgfqpoint{2.568455in}{2.183046in}}%
\pgfpathlineto{\pgfqpoint{2.581881in}{2.166287in}}%
\pgfpathlineto{\pgfqpoint{2.595299in}{2.149777in}}%
\pgfpathlineto{\pgfqpoint{2.603792in}{2.150265in}}%
\pgfpathlineto{\pgfqpoint{2.612270in}{2.150988in}}%
\pgfpathlineto{\pgfqpoint{2.620732in}{2.151943in}}%
\pgfpathlineto{\pgfqpoint{2.629179in}{2.153125in}}%
\pgfpathlineto{\pgfqpoint{2.615802in}{2.169145in}}%
\pgfpathlineto{\pgfqpoint{2.602418in}{2.185414in}}%
\pgfpathlineto{\pgfqpoint{2.589026in}{2.201932in}}%
\pgfpathlineto{\pgfqpoint{2.575626in}{2.218703in}}%
\pgfpathlineto{\pgfqpoint{2.567138in}{2.218001in}}%
\pgfpathlineto{\pgfqpoint{2.558634in}{2.217533in}}%
\pgfpathlineto{\pgfqpoint{2.550114in}{2.217304in}}%
\pgfpathlineto{\pgfqpoint{2.541578in}{2.217319in}}%
\pgfpathclose%
\pgfusepath{fill}%
\end{pgfscope}%
\begin{pgfscope}%
\pgfpathrectangle{\pgfqpoint{1.254980in}{0.150000in}}{\pgfqpoint{5.490039in}{5.490039in}}%
\pgfusepath{clip}%
\pgfsetbuttcap%
\pgfsetroundjoin%
\definecolor{currentfill}{rgb}{0.280267,0.073417,0.397163}%
\pgfsetfillcolor{currentfill}%
\pgfsetfillopacity{0.700000}%
\pgfsetlinewidth{0.000000pt}%
\definecolor{currentstroke}{rgb}{0.000000,0.000000,0.000000}%
\pgfsetstrokecolor{currentstroke}%
\pgfsetdash{}{0pt}%
\pgfpathmoveto{\pgfqpoint{3.722374in}{1.800125in}}%
\pgfpathlineto{\pgfqpoint{3.735645in}{1.800110in}}%
\pgfpathlineto{\pgfqpoint{3.748922in}{1.800266in}}%
\pgfpathlineto{\pgfqpoint{3.762207in}{1.800591in}}%
\pgfpathlineto{\pgfqpoint{3.775499in}{1.801084in}}%
\pgfpathlineto{\pgfqpoint{3.783323in}{1.811750in}}%
\pgfpathlineto{\pgfqpoint{3.791143in}{1.822422in}}%
\pgfpathlineto{\pgfqpoint{3.798957in}{1.833098in}}%
\pgfpathlineto{\pgfqpoint{3.806767in}{1.843774in}}%
\pgfpathlineto{\pgfqpoint{3.793483in}{1.843008in}}%
\pgfpathlineto{\pgfqpoint{3.780207in}{1.842411in}}%
\pgfpathlineto{\pgfqpoint{3.766939in}{1.841983in}}%
\pgfpathlineto{\pgfqpoint{3.753678in}{1.841725in}}%
\pgfpathlineto{\pgfqpoint{3.745860in}{1.831311in}}%
\pgfpathlineto{\pgfqpoint{3.738037in}{1.820904in}}%
\pgfpathlineto{\pgfqpoint{3.730208in}{1.810508in}}%
\pgfpathlineto{\pgfqpoint{3.722374in}{1.800125in}}%
\pgfpathclose%
\pgfusepath{fill}%
\end{pgfscope}%
\begin{pgfscope}%
\pgfpathrectangle{\pgfqpoint{1.254980in}{0.150000in}}{\pgfqpoint{5.490039in}{5.490039in}}%
\pgfusepath{clip}%
\pgfsetbuttcap%
\pgfsetroundjoin%
\definecolor{currentfill}{rgb}{0.282656,0.100196,0.422160}%
\pgfsetfillcolor{currentfill}%
\pgfsetfillopacity{0.700000}%
\pgfsetlinewidth{0.000000pt}%
\definecolor{currentstroke}{rgb}{0.000000,0.000000,0.000000}%
\pgfsetstrokecolor{currentstroke}%
\pgfsetdash{}{0pt}%
\pgfpathmoveto{\pgfqpoint{3.806767in}{1.843774in}}%
\pgfpathlineto{\pgfqpoint{3.820058in}{1.844709in}}%
\pgfpathlineto{\pgfqpoint{3.833357in}{1.845813in}}%
\pgfpathlineto{\pgfqpoint{3.846664in}{1.847084in}}%
\pgfpathlineto{\pgfqpoint{3.859979in}{1.848522in}}%
\pgfpathlineto{\pgfqpoint{3.867775in}{1.859453in}}%
\pgfpathlineto{\pgfqpoint{3.875567in}{1.870374in}}%
\pgfpathlineto{\pgfqpoint{3.883354in}{1.881281in}}%
\pgfpathlineto{\pgfqpoint{3.891136in}{1.892175in}}%
\pgfpathlineto{\pgfqpoint{3.877828in}{1.890491in}}%
\pgfpathlineto{\pgfqpoint{3.864529in}{1.888975in}}%
\pgfpathlineto{\pgfqpoint{3.851237in}{1.887627in}}%
\pgfpathlineto{\pgfqpoint{3.837954in}{1.886447in}}%
\pgfpathlineto{\pgfqpoint{3.830165in}{1.875788in}}%
\pgfpathlineto{\pgfqpoint{3.822370in}{1.865122in}}%
\pgfpathlineto{\pgfqpoint{3.814571in}{1.854450in}}%
\pgfpathlineto{\pgfqpoint{3.806767in}{1.843774in}}%
\pgfpathclose%
\pgfusepath{fill}%
\end{pgfscope}%
\begin{pgfscope}%
\pgfpathrectangle{\pgfqpoint{1.254980in}{0.150000in}}{\pgfqpoint{5.490039in}{5.490039in}}%
\pgfusepath{clip}%
\pgfsetbuttcap%
\pgfsetroundjoin%
\definecolor{currentfill}{rgb}{0.269944,0.014625,0.341379}%
\pgfsetfillcolor{currentfill}%
\pgfsetfillopacity{0.700000}%
\pgfsetlinewidth{0.000000pt}%
\definecolor{currentstroke}{rgb}{0.000000,0.000000,0.000000}%
\pgfsetstrokecolor{currentstroke}%
\pgfsetdash{}{0pt}%
\pgfpathmoveto{\pgfqpoint{3.193050in}{1.720973in}}%
\pgfpathlineto{\pgfqpoint{3.206280in}{1.714186in}}%
\pgfpathlineto{\pgfqpoint{3.219512in}{1.707587in}}%
\pgfpathlineto{\pgfqpoint{3.232746in}{1.701176in}}%
\pgfpathlineto{\pgfqpoint{3.245981in}{1.694951in}}%
\pgfpathlineto{\pgfqpoint{3.254036in}{1.701917in}}%
\pgfpathlineto{\pgfqpoint{3.262082in}{1.709002in}}%
\pgfpathlineto{\pgfqpoint{3.270121in}{1.716203in}}%
\pgfpathlineto{\pgfqpoint{3.278151in}{1.723515in}}%
\pgfpathlineto{\pgfqpoint{3.264937in}{1.729328in}}%
\pgfpathlineto{\pgfqpoint{3.251725in}{1.735327in}}%
\pgfpathlineto{\pgfqpoint{3.238515in}{1.741513in}}%
\pgfpathlineto{\pgfqpoint{3.225306in}{1.747888in}}%
\pgfpathlineto{\pgfqpoint{3.217255in}{1.740977in}}%
\pgfpathlineto{\pgfqpoint{3.209195in}{1.734185in}}%
\pgfpathlineto{\pgfqpoint{3.201127in}{1.727516in}}%
\pgfpathlineto{\pgfqpoint{3.193050in}{1.720973in}}%
\pgfpathclose%
\pgfusepath{fill}%
\end{pgfscope}%
\begin{pgfscope}%
\pgfpathrectangle{\pgfqpoint{1.254980in}{0.150000in}}{\pgfqpoint{5.490039in}{5.490039in}}%
\pgfusepath{clip}%
\pgfsetbuttcap%
\pgfsetroundjoin%
\definecolor{currentfill}{rgb}{0.279574,0.170599,0.479997}%
\pgfsetfillcolor{currentfill}%
\pgfsetfillopacity{0.700000}%
\pgfsetlinewidth{0.000000pt}%
\definecolor{currentstroke}{rgb}{0.000000,0.000000,0.000000}%
\pgfsetstrokecolor{currentstroke}%
\pgfsetdash{}{0pt}%
\pgfpathmoveto{\pgfqpoint{2.702388in}{2.026412in}}%
\pgfpathlineto{\pgfqpoint{2.715745in}{2.012054in}}%
\pgfpathlineto{\pgfqpoint{2.729097in}{1.997925in}}%
\pgfpathlineto{\pgfqpoint{2.742444in}{1.984025in}}%
\pgfpathlineto{\pgfqpoint{2.755786in}{1.970352in}}%
\pgfpathlineto{\pgfqpoint{2.764160in}{1.972329in}}%
\pgfpathlineto{\pgfqpoint{2.772521in}{1.974520in}}%
\pgfpathlineto{\pgfqpoint{2.780868in}{1.976919in}}%
\pgfpathlineto{\pgfqpoint{2.789202in}{1.979521in}}%
\pgfpathlineto{\pgfqpoint{2.775897in}{1.992713in}}%
\pgfpathlineto{\pgfqpoint{2.762587in}{2.006131in}}%
\pgfpathlineto{\pgfqpoint{2.749272in}{2.019777in}}%
\pgfpathlineto{\pgfqpoint{2.735952in}{2.033652in}}%
\pgfpathlineto{\pgfqpoint{2.727582in}{2.031521in}}%
\pgfpathlineto{\pgfqpoint{2.719198in}{2.029601in}}%
\pgfpathlineto{\pgfqpoint{2.710800in}{2.027896in}}%
\pgfpathlineto{\pgfqpoint{2.702388in}{2.026412in}}%
\pgfpathclose%
\pgfusepath{fill}%
\end{pgfscope}%
\begin{pgfscope}%
\pgfpathrectangle{\pgfqpoint{1.254980in}{0.150000in}}{\pgfqpoint{5.490039in}{5.490039in}}%
\pgfusepath{clip}%
\pgfsetbuttcap%
\pgfsetroundjoin%
\definecolor{currentfill}{rgb}{0.277018,0.050344,0.375715}%
\pgfsetfillcolor{currentfill}%
\pgfsetfillopacity{0.700000}%
\pgfsetlinewidth{0.000000pt}%
\definecolor{currentstroke}{rgb}{0.000000,0.000000,0.000000}%
\pgfsetstrokecolor{currentstroke}%
\pgfsetdash{}{0pt}%
\pgfpathmoveto{\pgfqpoint{3.637931in}{1.761739in}}%
\pgfpathlineto{\pgfqpoint{3.651186in}{1.760741in}}%
\pgfpathlineto{\pgfqpoint{3.664446in}{1.759914in}}%
\pgfpathlineto{\pgfqpoint{3.677713in}{1.759258in}}%
\pgfpathlineto{\pgfqpoint{3.690987in}{1.758773in}}%
\pgfpathlineto{\pgfqpoint{3.698842in}{1.769079in}}%
\pgfpathlineto{\pgfqpoint{3.706691in}{1.779408in}}%
\pgfpathlineto{\pgfqpoint{3.714535in}{1.789757in}}%
\pgfpathlineto{\pgfqpoint{3.722374in}{1.800125in}}%
\pgfpathlineto{\pgfqpoint{3.709111in}{1.800310in}}%
\pgfpathlineto{\pgfqpoint{3.695854in}{1.800665in}}%
\pgfpathlineto{\pgfqpoint{3.682604in}{1.801192in}}%
\pgfpathlineto{\pgfqpoint{3.669361in}{1.801891in}}%
\pgfpathlineto{\pgfqpoint{3.661512in}{1.791813in}}%
\pgfpathlineto{\pgfqpoint{3.653657in}{1.781760in}}%
\pgfpathlineto{\pgfqpoint{3.645797in}{1.771734in}}%
\pgfpathlineto{\pgfqpoint{3.637931in}{1.761739in}}%
\pgfpathclose%
\pgfusepath{fill}%
\end{pgfscope}%
\begin{pgfscope}%
\pgfpathrectangle{\pgfqpoint{1.254980in}{0.150000in}}{\pgfqpoint{5.490039in}{5.490039in}}%
\pgfusepath{clip}%
\pgfsetbuttcap%
\pgfsetroundjoin%
\definecolor{currentfill}{rgb}{0.283187,0.125848,0.444960}%
\pgfsetfillcolor{currentfill}%
\pgfsetfillopacity{0.700000}%
\pgfsetlinewidth{0.000000pt}%
\definecolor{currentstroke}{rgb}{0.000000,0.000000,0.000000}%
\pgfsetstrokecolor{currentstroke}%
\pgfsetdash{}{0pt}%
\pgfpathmoveto{\pgfqpoint{3.891136in}{1.892175in}}%
\pgfpathlineto{\pgfqpoint{3.904453in}{1.894026in}}%
\pgfpathlineto{\pgfqpoint{3.917778in}{1.896043in}}%
\pgfpathlineto{\pgfqpoint{3.931111in}{1.898227in}}%
\pgfpathlineto{\pgfqpoint{3.944454in}{1.900578in}}%
\pgfpathlineto{\pgfqpoint{3.952225in}{1.911682in}}%
\pgfpathlineto{\pgfqpoint{3.959991in}{1.922761in}}%
\pgfpathlineto{\pgfqpoint{3.967752in}{1.933812in}}%
\pgfpathlineto{\pgfqpoint{3.975508in}{1.944834in}}%
\pgfpathlineto{\pgfqpoint{3.962171in}{1.942267in}}%
\pgfpathlineto{\pgfqpoint{3.948844in}{1.939865in}}%
\pgfpathlineto{\pgfqpoint{3.935525in}{1.937630in}}%
\pgfpathlineto{\pgfqpoint{3.922216in}{1.935562in}}%
\pgfpathlineto{\pgfqpoint{3.914453in}{1.924747in}}%
\pgfpathlineto{\pgfqpoint{3.906686in}{1.913909in}}%
\pgfpathlineto{\pgfqpoint{3.898913in}{1.903051in}}%
\pgfpathlineto{\pgfqpoint{3.891136in}{1.892175in}}%
\pgfpathclose%
\pgfusepath{fill}%
\end{pgfscope}%
\begin{pgfscope}%
\pgfpathrectangle{\pgfqpoint{1.254980in}{0.150000in}}{\pgfqpoint{5.490039in}{5.490039in}}%
\pgfusepath{clip}%
\pgfsetbuttcap%
\pgfsetroundjoin%
\definecolor{currentfill}{rgb}{0.268510,0.009605,0.335427}%
\pgfsetfillcolor{currentfill}%
\pgfsetfillopacity{0.700000}%
\pgfsetlinewidth{0.000000pt}%
\definecolor{currentstroke}{rgb}{0.000000,0.000000,0.000000}%
\pgfsetstrokecolor{currentstroke}%
\pgfsetdash{}{0pt}%
\pgfpathmoveto{\pgfqpoint{3.331029in}{1.702116in}}%
\pgfpathlineto{\pgfqpoint{3.344255in}{1.697224in}}%
\pgfpathlineto{\pgfqpoint{3.357485in}{1.692515in}}%
\pgfpathlineto{\pgfqpoint{3.370717in}{1.687986in}}%
\pgfpathlineto{\pgfqpoint{3.383952in}{1.683637in}}%
\pgfpathlineto{\pgfqpoint{3.391937in}{1.691846in}}%
\pgfpathlineto{\pgfqpoint{3.399914in}{1.700144in}}%
\pgfpathlineto{\pgfqpoint{3.407884in}{1.708529in}}%
\pgfpathlineto{\pgfqpoint{3.415847in}{1.716996in}}%
\pgfpathlineto{\pgfqpoint{3.402629in}{1.720961in}}%
\pgfpathlineto{\pgfqpoint{3.389414in}{1.725107in}}%
\pgfpathlineto{\pgfqpoint{3.376203in}{1.729433in}}%
\pgfpathlineto{\pgfqpoint{3.362995in}{1.733941in}}%
\pgfpathlineto{\pgfqpoint{3.355015in}{1.725846in}}%
\pgfpathlineto{\pgfqpoint{3.347027in}{1.717842in}}%
\pgfpathlineto{\pgfqpoint{3.339032in}{1.709930in}}%
\pgfpathlineto{\pgfqpoint{3.331029in}{1.702116in}}%
\pgfpathclose%
\pgfusepath{fill}%
\end{pgfscope}%
\begin{pgfscope}%
\pgfpathrectangle{\pgfqpoint{1.254980in}{0.150000in}}{\pgfqpoint{5.490039in}{5.490039in}}%
\pgfusepath{clip}%
\pgfsetbuttcap%
\pgfsetroundjoin%
\definecolor{currentfill}{rgb}{0.244972,0.287675,0.537260}%
\pgfsetfillcolor{currentfill}%
\pgfsetfillopacity{0.700000}%
\pgfsetlinewidth{0.000000pt}%
\definecolor{currentstroke}{rgb}{0.000000,0.000000,0.000000}%
\pgfsetstrokecolor{currentstroke}%
\pgfsetdash{}{0pt}%
\pgfpathmoveto{\pgfqpoint{2.487717in}{2.288942in}}%
\pgfpathlineto{\pgfqpoint{2.501196in}{2.270646in}}%
\pgfpathlineto{\pgfqpoint{2.514666in}{2.252612in}}%
\pgfpathlineto{\pgfqpoint{2.528126in}{2.234837in}}%
\pgfpathlineto{\pgfqpoint{2.541578in}{2.217319in}}%
\pgfpathlineto{\pgfqpoint{2.550114in}{2.217304in}}%
\pgfpathlineto{\pgfqpoint{2.558634in}{2.217533in}}%
\pgfpathlineto{\pgfqpoint{2.567138in}{2.218001in}}%
\pgfpathlineto{\pgfqpoint{2.575626in}{2.218703in}}%
\pgfpathlineto{\pgfqpoint{2.562217in}{2.235727in}}%
\pgfpathlineto{\pgfqpoint{2.548800in}{2.253008in}}%
\pgfpathlineto{\pgfqpoint{2.535375in}{2.270547in}}%
\pgfpathlineto{\pgfqpoint{2.521940in}{2.288347in}}%
\pgfpathlineto{\pgfqpoint{2.513409in}{2.288128in}}%
\pgfpathlineto{\pgfqpoint{2.504862in}{2.288151in}}%
\pgfpathlineto{\pgfqpoint{2.496298in}{2.288421in}}%
\pgfpathlineto{\pgfqpoint{2.487717in}{2.288942in}}%
\pgfpathclose%
\pgfusepath{fill}%
\end{pgfscope}%
\begin{pgfscope}%
\pgfpathrectangle{\pgfqpoint{1.254980in}{0.150000in}}{\pgfqpoint{5.490039in}{5.490039in}}%
\pgfusepath{clip}%
\pgfsetbuttcap%
\pgfsetroundjoin%
\definecolor{currentfill}{rgb}{0.159194,0.482237,0.558073}%
\pgfsetfillcolor{currentfill}%
\pgfsetfillopacity{0.700000}%
\pgfsetlinewidth{0.000000pt}%
\definecolor{currentstroke}{rgb}{0.000000,0.000000,0.000000}%
\pgfsetstrokecolor{currentstroke}%
\pgfsetdash{}{0pt}%
\pgfpathmoveto{\pgfqpoint{4.911950in}{2.704472in}}%
\pgfpathlineto{\pgfqpoint{4.925737in}{2.713922in}}%
\pgfpathlineto{\pgfqpoint{4.939541in}{2.723532in}}%
\pgfpathlineto{\pgfqpoint{4.953360in}{2.733302in}}%
\pgfpathlineto{\pgfqpoint{4.967196in}{2.743231in}}%
\pgfpathlineto{\pgfqpoint{4.974583in}{2.750166in}}%
\pgfpathlineto{\pgfqpoint{4.981962in}{2.757004in}}%
\pgfpathlineto{\pgfqpoint{4.989334in}{2.763750in}}%
\pgfpathlineto{\pgfqpoint{4.996699in}{2.770404in}}%
\pgfpathlineto{\pgfqpoint{4.982873in}{2.760664in}}%
\pgfpathlineto{\pgfqpoint{4.969064in}{2.751082in}}%
\pgfpathlineto{\pgfqpoint{4.955271in}{2.741661in}}%
\pgfpathlineto{\pgfqpoint{4.941493in}{2.732398in}}%
\pgfpathlineto{\pgfqpoint{4.934118in}{2.725544in}}%
\pgfpathlineto{\pgfqpoint{4.926736in}{2.718607in}}%
\pgfpathlineto{\pgfqpoint{4.919346in}{2.711584in}}%
\pgfpathlineto{\pgfqpoint{4.911950in}{2.704472in}}%
\pgfpathclose%
\pgfusepath{fill}%
\end{pgfscope}%
\begin{pgfscope}%
\pgfpathrectangle{\pgfqpoint{1.254980in}{0.150000in}}{\pgfqpoint{5.490039in}{5.490039in}}%
\pgfusepath{clip}%
\pgfsetbuttcap%
\pgfsetroundjoin%
\definecolor{currentfill}{rgb}{0.119512,0.607464,0.540218}%
\pgfsetfillcolor{currentfill}%
\pgfsetfillopacity{0.700000}%
\pgfsetlinewidth{0.000000pt}%
\definecolor{currentstroke}{rgb}{0.000000,0.000000,0.000000}%
\pgfsetstrokecolor{currentstroke}%
\pgfsetdash{}{0pt}%
\pgfpathmoveto{\pgfqpoint{5.364876in}{3.043949in}}%
\pgfpathlineto{\pgfqpoint{5.378922in}{3.054991in}}%
\pgfpathlineto{\pgfqpoint{5.392986in}{3.066190in}}%
\pgfpathlineto{\pgfqpoint{5.407068in}{3.077548in}}%
\pgfpathlineto{\pgfqpoint{5.421170in}{3.089064in}}%
\pgfpathlineto{\pgfqpoint{5.428299in}{3.092713in}}%
\pgfpathlineto{\pgfqpoint{5.435421in}{3.096307in}}%
\pgfpathlineto{\pgfqpoint{5.442535in}{3.099850in}}%
\pgfpathlineto{\pgfqpoint{5.449641in}{3.103346in}}%
\pgfpathlineto{\pgfqpoint{5.435560in}{3.092201in}}%
\pgfpathlineto{\pgfqpoint{5.421497in}{3.081214in}}%
\pgfpathlineto{\pgfqpoint{5.407453in}{3.070385in}}%
\pgfpathlineto{\pgfqpoint{5.393426in}{3.059712in}}%
\pgfpathlineto{\pgfqpoint{5.386299in}{3.055835in}}%
\pgfpathlineto{\pgfqpoint{5.379166in}{3.051918in}}%
\pgfpathlineto{\pgfqpoint{5.372024in}{3.047958in}}%
\pgfpathlineto{\pgfqpoint{5.364876in}{3.043949in}}%
\pgfpathclose%
\pgfusepath{fill}%
\end{pgfscope}%
\begin{pgfscope}%
\pgfpathrectangle{\pgfqpoint{1.254980in}{0.150000in}}{\pgfqpoint{5.490039in}{5.490039in}}%
\pgfusepath{clip}%
\pgfsetbuttcap%
\pgfsetroundjoin%
\definecolor{currentfill}{rgb}{0.282290,0.145912,0.461510}%
\pgfsetfillcolor{currentfill}%
\pgfsetfillopacity{0.700000}%
\pgfsetlinewidth{0.000000pt}%
\definecolor{currentstroke}{rgb}{0.000000,0.000000,0.000000}%
\pgfsetstrokecolor{currentstroke}%
\pgfsetdash{}{0pt}%
\pgfpathmoveto{\pgfqpoint{2.755786in}{1.970352in}}%
\pgfpathlineto{\pgfqpoint{2.769122in}{1.956903in}}%
\pgfpathlineto{\pgfqpoint{2.782454in}{1.943679in}}%
\pgfpathlineto{\pgfqpoint{2.795782in}{1.930676in}}%
\pgfpathlineto{\pgfqpoint{2.809105in}{1.917893in}}%
\pgfpathlineto{\pgfqpoint{2.817443in}{1.920363in}}%
\pgfpathlineto{\pgfqpoint{2.825768in}{1.923037in}}%
\pgfpathlineto{\pgfqpoint{2.834080in}{1.925912in}}%
\pgfpathlineto{\pgfqpoint{2.842380in}{1.928983in}}%
\pgfpathlineto{\pgfqpoint{2.829091in}{1.941286in}}%
\pgfpathlineto{\pgfqpoint{2.815799in}{1.953809in}}%
\pgfpathlineto{\pgfqpoint{2.802503in}{1.966554in}}%
\pgfpathlineto{\pgfqpoint{2.789202in}{1.979521in}}%
\pgfpathlineto{\pgfqpoint{2.780868in}{1.976919in}}%
\pgfpathlineto{\pgfqpoint{2.772521in}{1.974520in}}%
\pgfpathlineto{\pgfqpoint{2.764160in}{1.972329in}}%
\pgfpathlineto{\pgfqpoint{2.755786in}{1.970352in}}%
\pgfpathclose%
\pgfusepath{fill}%
\end{pgfscope}%
\begin{pgfscope}%
\pgfpathrectangle{\pgfqpoint{1.254980in}{0.150000in}}{\pgfqpoint{5.490039in}{5.490039in}}%
\pgfusepath{clip}%
\pgfsetbuttcap%
\pgfsetroundjoin%
\definecolor{currentfill}{rgb}{0.273809,0.031497,0.358853}%
\pgfsetfillcolor{currentfill}%
\pgfsetfillopacity{0.700000}%
\pgfsetlinewidth{0.000000pt}%
\definecolor{currentstroke}{rgb}{0.000000,0.000000,0.000000}%
\pgfsetstrokecolor{currentstroke}%
\pgfsetdash{}{0pt}%
\pgfpathmoveto{\pgfqpoint{3.553405in}{1.729152in}}%
\pgfpathlineto{\pgfqpoint{3.566648in}{1.727134in}}%
\pgfpathlineto{\pgfqpoint{3.579897in}{1.725290in}}%
\pgfpathlineto{\pgfqpoint{3.593152in}{1.723619in}}%
\pgfpathlineto{\pgfqpoint{3.606412in}{1.722121in}}%
\pgfpathlineto{\pgfqpoint{3.614300in}{1.731965in}}%
\pgfpathlineto{\pgfqpoint{3.622183in}{1.741852in}}%
\pgfpathlineto{\pgfqpoint{3.630060in}{1.751777in}}%
\pgfpathlineto{\pgfqpoint{3.637931in}{1.761739in}}%
\pgfpathlineto{\pgfqpoint{3.624683in}{1.762910in}}%
\pgfpathlineto{\pgfqpoint{3.611441in}{1.764253in}}%
\pgfpathlineto{\pgfqpoint{3.598204in}{1.765770in}}%
\pgfpathlineto{\pgfqpoint{3.584973in}{1.767461in}}%
\pgfpathlineto{\pgfqpoint{3.577090in}{1.757816in}}%
\pgfpathlineto{\pgfqpoint{3.569201in}{1.748214in}}%
\pgfpathlineto{\pgfqpoint{3.561306in}{1.738659in}}%
\pgfpathlineto{\pgfqpoint{3.553405in}{1.729152in}}%
\pgfpathclose%
\pgfusepath{fill}%
\end{pgfscope}%
\begin{pgfscope}%
\pgfpathrectangle{\pgfqpoint{1.254980in}{0.150000in}}{\pgfqpoint{5.490039in}{5.490039in}}%
\pgfusepath{clip}%
\pgfsetbuttcap%
\pgfsetroundjoin%
\definecolor{currentfill}{rgb}{0.281412,0.155834,0.469201}%
\pgfsetfillcolor{currentfill}%
\pgfsetfillopacity{0.700000}%
\pgfsetlinewidth{0.000000pt}%
\definecolor{currentstroke}{rgb}{0.000000,0.000000,0.000000}%
\pgfsetstrokecolor{currentstroke}%
\pgfsetdash{}{0pt}%
\pgfpathmoveto{\pgfqpoint{3.975508in}{1.944834in}}%
\pgfpathlineto{\pgfqpoint{3.988854in}{1.947568in}}%
\pgfpathlineto{\pgfqpoint{4.002209in}{1.950468in}}%
\pgfpathlineto{\pgfqpoint{4.015574in}{1.953532in}}%
\pgfpathlineto{\pgfqpoint{4.028949in}{1.956762in}}%
\pgfpathlineto{\pgfqpoint{4.036695in}{1.967953in}}%
\pgfpathlineto{\pgfqpoint{4.044436in}{1.979105in}}%
\pgfpathlineto{\pgfqpoint{4.052172in}{1.990215in}}%
\pgfpathlineto{\pgfqpoint{4.059904in}{2.001283in}}%
\pgfpathlineto{\pgfqpoint{4.046534in}{1.997863in}}%
\pgfpathlineto{\pgfqpoint{4.033175in}{1.994609in}}%
\pgfpathlineto{\pgfqpoint{4.019825in}{1.991520in}}%
\pgfpathlineto{\pgfqpoint{4.006485in}{1.988597in}}%
\pgfpathlineto{\pgfqpoint{3.998748in}{1.977708in}}%
\pgfpathlineto{\pgfqpoint{3.991006in}{1.966784in}}%
\pgfpathlineto{\pgfqpoint{3.983260in}{1.955826in}}%
\pgfpathlineto{\pgfqpoint{3.975508in}{1.944834in}}%
\pgfpathclose%
\pgfusepath{fill}%
\end{pgfscope}%
\begin{pgfscope}%
\pgfpathrectangle{\pgfqpoint{1.254980in}{0.150000in}}{\pgfqpoint{5.490039in}{5.490039in}}%
\pgfusepath{clip}%
\pgfsetbuttcap%
\pgfsetroundjoin%
\definecolor{currentfill}{rgb}{0.195860,0.395433,0.555276}%
\pgfsetfillcolor{currentfill}%
\pgfsetfillopacity{0.700000}%
\pgfsetlinewidth{0.000000pt}%
\definecolor{currentstroke}{rgb}{0.000000,0.000000,0.000000}%
\pgfsetstrokecolor{currentstroke}%
\pgfsetdash{}{0pt}%
\pgfpathmoveto{\pgfqpoint{4.628033in}{2.470683in}}%
\pgfpathlineto{\pgfqpoint{4.641669in}{2.478664in}}%
\pgfpathlineto{\pgfqpoint{4.655320in}{2.486805in}}%
\pgfpathlineto{\pgfqpoint{4.668986in}{2.495107in}}%
\pgfpathlineto{\pgfqpoint{4.682666in}{2.503570in}}%
\pgfpathlineto{\pgfqpoint{4.690186in}{2.512468in}}%
\pgfpathlineto{\pgfqpoint{4.697700in}{2.521265in}}%
\pgfpathlineto{\pgfqpoint{4.705207in}{2.529962in}}%
\pgfpathlineto{\pgfqpoint{4.712708in}{2.538562in}}%
\pgfpathlineto{\pgfqpoint{4.699034in}{2.530168in}}%
\pgfpathlineto{\pgfqpoint{4.685375in}{2.521934in}}%
\pgfpathlineto{\pgfqpoint{4.671730in}{2.513862in}}%
\pgfpathlineto{\pgfqpoint{4.658100in}{2.505950in}}%
\pgfpathlineto{\pgfqpoint{4.650592in}{2.497271in}}%
\pgfpathlineto{\pgfqpoint{4.643079in}{2.488501in}}%
\pgfpathlineto{\pgfqpoint{4.635559in}{2.479638in}}%
\pgfpathlineto{\pgfqpoint{4.628033in}{2.470683in}}%
\pgfpathclose%
\pgfusepath{fill}%
\end{pgfscope}%
\begin{pgfscope}%
\pgfpathrectangle{\pgfqpoint{1.254980in}{0.150000in}}{\pgfqpoint{5.490039in}{5.490039in}}%
\pgfusepath{clip}%
\pgfsetbuttcap%
\pgfsetroundjoin%
\definecolor{currentfill}{rgb}{0.229739,0.322361,0.545706}%
\pgfsetfillcolor{currentfill}%
\pgfsetfillopacity{0.700000}%
\pgfsetlinewidth{0.000000pt}%
\definecolor{currentstroke}{rgb}{0.000000,0.000000,0.000000}%
\pgfsetstrokecolor{currentstroke}%
\pgfsetdash{}{0pt}%
\pgfpathmoveto{\pgfqpoint{2.433700in}{2.364789in}}%
\pgfpathlineto{\pgfqpoint{2.447220in}{2.345423in}}%
\pgfpathlineto{\pgfqpoint{2.460729in}{2.326328in}}%
\pgfpathlineto{\pgfqpoint{2.474228in}{2.307502in}}%
\pgfpathlineto{\pgfqpoint{2.487717in}{2.288942in}}%
\pgfpathlineto{\pgfqpoint{2.496298in}{2.288421in}}%
\pgfpathlineto{\pgfqpoint{2.504862in}{2.288151in}}%
\pgfpathlineto{\pgfqpoint{2.513409in}{2.288128in}}%
\pgfpathlineto{\pgfqpoint{2.521940in}{2.288347in}}%
\pgfpathlineto{\pgfqpoint{2.508496in}{2.306411in}}%
\pgfpathlineto{\pgfqpoint{2.495042in}{2.324739in}}%
\pgfpathlineto{\pgfqpoint{2.481579in}{2.343335in}}%
\pgfpathlineto{\pgfqpoint{2.468105in}{2.362201in}}%
\pgfpathlineto{\pgfqpoint{2.459530in}{2.362468in}}%
\pgfpathlineto{\pgfqpoint{2.450937in}{2.362985in}}%
\pgfpathlineto{\pgfqpoint{2.442328in}{2.363757in}}%
\pgfpathlineto{\pgfqpoint{2.433700in}{2.364789in}}%
\pgfpathclose%
\pgfusepath{fill}%
\end{pgfscope}%
\begin{pgfscope}%
\pgfpathrectangle{\pgfqpoint{1.254980in}{0.150000in}}{\pgfqpoint{5.490039in}{5.490039in}}%
\pgfusepath{clip}%
\pgfsetbuttcap%
\pgfsetroundjoin%
\definecolor{currentfill}{rgb}{0.241237,0.296485,0.539709}%
\pgfsetfillcolor{currentfill}%
\pgfsetfillopacity{0.700000}%
\pgfsetlinewidth{0.000000pt}%
\definecolor{currentstroke}{rgb}{0.000000,0.000000,0.000000}%
\pgfsetstrokecolor{currentstroke}%
\pgfsetdash{}{0pt}%
\pgfpathmoveto{\pgfqpoint{4.343987in}{2.231153in}}%
\pgfpathlineto{\pgfqpoint{4.357484in}{2.237177in}}%
\pgfpathlineto{\pgfqpoint{4.370992in}{2.243363in}}%
\pgfpathlineto{\pgfqpoint{4.384514in}{2.249711in}}%
\pgfpathlineto{\pgfqpoint{4.398047in}{2.256221in}}%
\pgfpathlineto{\pgfqpoint{4.405676in}{2.266637in}}%
\pgfpathlineto{\pgfqpoint{4.413300in}{2.276967in}}%
\pgfpathlineto{\pgfqpoint{4.420918in}{2.287211in}}%
\pgfpathlineto{\pgfqpoint{4.428530in}{2.297369in}}%
\pgfpathlineto{\pgfqpoint{4.415001in}{2.290811in}}%
\pgfpathlineto{\pgfqpoint{4.401484in}{2.284415in}}%
\pgfpathlineto{\pgfqpoint{4.387979in}{2.278181in}}%
\pgfpathlineto{\pgfqpoint{4.374487in}{2.272110in}}%
\pgfpathlineto{\pgfqpoint{4.366870in}{2.261989in}}%
\pgfpathlineto{\pgfqpoint{4.359248in}{2.251789in}}%
\pgfpathlineto{\pgfqpoint{4.351620in}{2.241511in}}%
\pgfpathlineto{\pgfqpoint{4.343987in}{2.231153in}}%
\pgfpathclose%
\pgfusepath{fill}%
\end{pgfscope}%
\begin{pgfscope}%
\pgfpathrectangle{\pgfqpoint{1.254980in}{0.150000in}}{\pgfqpoint{5.490039in}{5.490039in}}%
\pgfusepath{clip}%
\pgfsetbuttcap%
\pgfsetroundjoin%
\definecolor{currentfill}{rgb}{0.274952,0.037752,0.364543}%
\pgfsetfillcolor{currentfill}%
\pgfsetfillopacity{0.700000}%
\pgfsetlinewidth{0.000000pt}%
\definecolor{currentstroke}{rgb}{0.000000,0.000000,0.000000}%
\pgfsetstrokecolor{currentstroke}%
\pgfsetdash{}{0pt}%
\pgfpathmoveto{\pgfqpoint{3.054630in}{1.760890in}}%
\pgfpathlineto{\pgfqpoint{3.067883in}{1.752116in}}%
\pgfpathlineto{\pgfqpoint{3.081135in}{1.743539in}}%
\pgfpathlineto{\pgfqpoint{3.094387in}{1.735158in}}%
\pgfpathlineto{\pgfqpoint{3.107639in}{1.726971in}}%
\pgfpathlineto{\pgfqpoint{3.115778in}{1.732514in}}%
\pgfpathlineto{\pgfqpoint{3.123908in}{1.738207in}}%
\pgfpathlineto{\pgfqpoint{3.132027in}{1.744046in}}%
\pgfpathlineto{\pgfqpoint{3.140137in}{1.750028in}}%
\pgfpathlineto{\pgfqpoint{3.126911in}{1.757772in}}%
\pgfpathlineto{\pgfqpoint{3.113684in}{1.765711in}}%
\pgfpathlineto{\pgfqpoint{3.100458in}{1.773846in}}%
\pgfpathlineto{\pgfqpoint{3.087232in}{1.782176in}}%
\pgfpathlineto{\pgfqpoint{3.079097in}{1.776627in}}%
\pgfpathlineto{\pgfqpoint{3.070951in}{1.771226in}}%
\pgfpathlineto{\pgfqpoint{3.062796in}{1.765979in}}%
\pgfpathlineto{\pgfqpoint{3.054630in}{1.760890in}}%
\pgfpathclose%
\pgfusepath{fill}%
\end{pgfscope}%
\begin{pgfscope}%
\pgfpathrectangle{\pgfqpoint{1.254980in}{0.150000in}}{\pgfqpoint{5.490039in}{5.490039in}}%
\pgfusepath{clip}%
\pgfsetbuttcap%
\pgfsetroundjoin%
\definecolor{currentfill}{rgb}{0.121380,0.629492,0.531973}%
\pgfsetfillcolor{currentfill}%
\pgfsetfillopacity{0.700000}%
\pgfsetlinewidth{0.000000pt}%
\definecolor{currentstroke}{rgb}{0.000000,0.000000,0.000000}%
\pgfsetstrokecolor{currentstroke}%
\pgfsetdash{}{0pt}%
\pgfpathmoveto{\pgfqpoint{5.449641in}{3.103346in}}%
\pgfpathlineto{\pgfqpoint{5.463741in}{3.114648in}}%
\pgfpathlineto{\pgfqpoint{5.477859in}{3.126107in}}%
\pgfpathlineto{\pgfqpoint{5.491996in}{3.137725in}}%
\pgfpathlineto{\pgfqpoint{5.506152in}{3.149501in}}%
\pgfpathlineto{\pgfqpoint{5.513229in}{3.152563in}}%
\pgfpathlineto{\pgfqpoint{5.520299in}{3.155581in}}%
\pgfpathlineto{\pgfqpoint{5.527362in}{3.158558in}}%
\pgfpathlineto{\pgfqpoint{5.534417in}{3.161498in}}%
\pgfpathlineto{\pgfqpoint{5.520283in}{3.150125in}}%
\pgfpathlineto{\pgfqpoint{5.506168in}{3.138910in}}%
\pgfpathlineto{\pgfqpoint{5.492071in}{3.127851in}}%
\pgfpathlineto{\pgfqpoint{5.477993in}{3.116949in}}%
\pgfpathlineto{\pgfqpoint{5.470916in}{3.113596in}}%
\pgfpathlineto{\pgfqpoint{5.463831in}{3.110215in}}%
\pgfpathlineto{\pgfqpoint{5.456740in}{3.106799in}}%
\pgfpathlineto{\pgfqpoint{5.449641in}{3.103346in}}%
\pgfpathclose%
\pgfusepath{fill}%
\end{pgfscope}%
\begin{pgfscope}%
\pgfpathrectangle{\pgfqpoint{1.254980in}{0.150000in}}{\pgfqpoint{5.490039in}{5.490039in}}%
\pgfusepath{clip}%
\pgfsetbuttcap%
\pgfsetroundjoin%
\definecolor{currentfill}{rgb}{0.283229,0.120777,0.440584}%
\pgfsetfillcolor{currentfill}%
\pgfsetfillopacity{0.700000}%
\pgfsetlinewidth{0.000000pt}%
\definecolor{currentstroke}{rgb}{0.000000,0.000000,0.000000}%
\pgfsetstrokecolor{currentstroke}%
\pgfsetdash{}{0pt}%
\pgfpathmoveto{\pgfqpoint{2.809105in}{1.917893in}}%
\pgfpathlineto{\pgfqpoint{2.822424in}{1.905330in}}%
\pgfpathlineto{\pgfqpoint{2.835739in}{1.892984in}}%
\pgfpathlineto{\pgfqpoint{2.849051in}{1.880854in}}%
\pgfpathlineto{\pgfqpoint{2.862359in}{1.868938in}}%
\pgfpathlineto{\pgfqpoint{2.870663in}{1.871896in}}%
\pgfpathlineto{\pgfqpoint{2.878954in}{1.875052in}}%
\pgfpathlineto{\pgfqpoint{2.887232in}{1.878401in}}%
\pgfpathlineto{\pgfqpoint{2.895499in}{1.881939in}}%
\pgfpathlineto{\pgfqpoint{2.882224in}{1.893377in}}%
\pgfpathlineto{\pgfqpoint{2.868946in}{1.905030in}}%
\pgfpathlineto{\pgfqpoint{2.855664in}{1.916898in}}%
\pgfpathlineto{\pgfqpoint{2.842380in}{1.928983in}}%
\pgfpathlineto{\pgfqpoint{2.834080in}{1.925912in}}%
\pgfpathlineto{\pgfqpoint{2.825768in}{1.923037in}}%
\pgfpathlineto{\pgfqpoint{2.817443in}{1.920363in}}%
\pgfpathlineto{\pgfqpoint{2.809105in}{1.917893in}}%
\pgfpathclose%
\pgfusepath{fill}%
\end{pgfscope}%
\begin{pgfscope}%
\pgfpathrectangle{\pgfqpoint{1.254980in}{0.150000in}}{\pgfqpoint{5.490039in}{5.490039in}}%
\pgfusepath{clip}%
\pgfsetbuttcap%
\pgfsetroundjoin%
\definecolor{currentfill}{rgb}{0.149039,0.508051,0.557250}%
\pgfsetfillcolor{currentfill}%
\pgfsetfillopacity{0.700000}%
\pgfsetlinewidth{0.000000pt}%
\definecolor{currentstroke}{rgb}{0.000000,0.000000,0.000000}%
\pgfsetstrokecolor{currentstroke}%
\pgfsetdash{}{0pt}%
\pgfpathmoveto{\pgfqpoint{4.996699in}{2.770404in}}%
\pgfpathlineto{\pgfqpoint{5.010541in}{2.780304in}}%
\pgfpathlineto{\pgfqpoint{5.024400in}{2.790364in}}%
\pgfpathlineto{\pgfqpoint{5.038275in}{2.800583in}}%
\pgfpathlineto{\pgfqpoint{5.052167in}{2.810962in}}%
\pgfpathlineto{\pgfqpoint{5.059513in}{2.817319in}}%
\pgfpathlineto{\pgfqpoint{5.066852in}{2.823584in}}%
\pgfpathlineto{\pgfqpoint{5.074183in}{2.829758in}}%
\pgfpathlineto{\pgfqpoint{5.081507in}{2.835844in}}%
\pgfpathlineto{\pgfqpoint{5.067626in}{2.825685in}}%
\pgfpathlineto{\pgfqpoint{5.053763in}{2.815685in}}%
\pgfpathlineto{\pgfqpoint{5.039916in}{2.805844in}}%
\pgfpathlineto{\pgfqpoint{5.026085in}{2.796162in}}%
\pgfpathlineto{\pgfqpoint{5.018750in}{2.789846in}}%
\pgfpathlineto{\pgfqpoint{5.011407in}{2.783450in}}%
\pgfpathlineto{\pgfqpoint{5.004057in}{2.776970in}}%
\pgfpathlineto{\pgfqpoint{4.996699in}{2.770404in}}%
\pgfpathclose%
\pgfusepath{fill}%
\end{pgfscope}%
\begin{pgfscope}%
\pgfpathrectangle{\pgfqpoint{1.254980in}{0.150000in}}{\pgfqpoint{5.490039in}{5.490039in}}%
\pgfusepath{clip}%
\pgfsetbuttcap%
\pgfsetroundjoin%
\definecolor{currentfill}{rgb}{0.277134,0.185228,0.489898}%
\pgfsetfillcolor{currentfill}%
\pgfsetfillopacity{0.700000}%
\pgfsetlinewidth{0.000000pt}%
\definecolor{currentstroke}{rgb}{0.000000,0.000000,0.000000}%
\pgfsetstrokecolor{currentstroke}%
\pgfsetdash{}{0pt}%
\pgfpathmoveto{\pgfqpoint{4.059904in}{2.001283in}}%
\pgfpathlineto{\pgfqpoint{4.073283in}{2.004867in}}%
\pgfpathlineto{\pgfqpoint{4.086673in}{2.008616in}}%
\pgfpathlineto{\pgfqpoint{4.100072in}{2.012529in}}%
\pgfpathlineto{\pgfqpoint{4.113483in}{2.016607in}}%
\pgfpathlineto{\pgfqpoint{4.121205in}{2.027802in}}%
\pgfpathlineto{\pgfqpoint{4.128922in}{2.038945in}}%
\pgfpathlineto{\pgfqpoint{4.136634in}{2.050034in}}%
\pgfpathlineto{\pgfqpoint{4.144342in}{2.061068in}}%
\pgfpathlineto{\pgfqpoint{4.130936in}{2.056829in}}%
\pgfpathlineto{\pgfqpoint{4.117541in}{2.052754in}}%
\pgfpathlineto{\pgfqpoint{4.104156in}{2.048844in}}%
\pgfpathlineto{\pgfqpoint{4.090782in}{2.045098in}}%
\pgfpathlineto{\pgfqpoint{4.083070in}{2.034215in}}%
\pgfpathlineto{\pgfqpoint{4.075353in}{2.023284in}}%
\pgfpathlineto{\pgfqpoint{4.067631in}{2.012306in}}%
\pgfpathlineto{\pgfqpoint{4.059904in}{2.001283in}}%
\pgfpathclose%
\pgfusepath{fill}%
\end{pgfscope}%
\begin{pgfscope}%
\pgfpathrectangle{\pgfqpoint{1.254980in}{0.150000in}}{\pgfqpoint{5.490039in}{5.490039in}}%
\pgfusepath{clip}%
\pgfsetbuttcap%
\pgfsetroundjoin%
\definecolor{currentfill}{rgb}{0.271305,0.019942,0.347269}%
\pgfsetfillcolor{currentfill}%
\pgfsetfillopacity{0.700000}%
\pgfsetlinewidth{0.000000pt}%
\definecolor{currentstroke}{rgb}{0.000000,0.000000,0.000000}%
\pgfsetstrokecolor{currentstroke}%
\pgfsetdash{}{0pt}%
\pgfpathmoveto{\pgfqpoint{3.468759in}{1.702924in}}%
\pgfpathlineto{\pgfqpoint{3.481997in}{1.699849in}}%
\pgfpathlineto{\pgfqpoint{3.495240in}{1.696951in}}%
\pgfpathlineto{\pgfqpoint{3.508487in}{1.694229in}}%
\pgfpathlineto{\pgfqpoint{3.521739in}{1.691681in}}%
\pgfpathlineto{\pgfqpoint{3.529665in}{1.700959in}}%
\pgfpathlineto{\pgfqpoint{3.537584in}{1.710299in}}%
\pgfpathlineto{\pgfqpoint{3.545498in}{1.719698in}}%
\pgfpathlineto{\pgfqpoint{3.553405in}{1.729152in}}%
\pgfpathlineto{\pgfqpoint{3.540166in}{1.731345in}}%
\pgfpathlineto{\pgfqpoint{3.526933in}{1.733712in}}%
\pgfpathlineto{\pgfqpoint{3.513705in}{1.736255in}}%
\pgfpathlineto{\pgfqpoint{3.500481in}{1.738974in}}%
\pgfpathlineto{\pgfqpoint{3.492560in}{1.729865in}}%
\pgfpathlineto{\pgfqpoint{3.484633in}{1.720818in}}%
\pgfpathlineto{\pgfqpoint{3.476699in}{1.711836in}}%
\pgfpathlineto{\pgfqpoint{3.468759in}{1.702924in}}%
\pgfpathclose%
\pgfusepath{fill}%
\end{pgfscope}%
\begin{pgfscope}%
\pgfpathrectangle{\pgfqpoint{1.254980in}{0.150000in}}{\pgfqpoint{5.490039in}{5.490039in}}%
\pgfusepath{clip}%
\pgfsetbuttcap%
\pgfsetroundjoin%
\definecolor{currentfill}{rgb}{0.214298,0.355619,0.551184}%
\pgfsetfillcolor{currentfill}%
\pgfsetfillopacity{0.700000}%
\pgfsetlinewidth{0.000000pt}%
\definecolor{currentstroke}{rgb}{0.000000,0.000000,0.000000}%
\pgfsetstrokecolor{currentstroke}%
\pgfsetdash{}{0pt}%
\pgfpathmoveto{\pgfqpoint{2.379509in}{2.445012in}}%
\pgfpathlineto{\pgfqpoint{2.393074in}{2.424537in}}%
\pgfpathlineto{\pgfqpoint{2.406628in}{2.404343in}}%
\pgfpathlineto{\pgfqpoint{2.420170in}{2.384428in}}%
\pgfpathlineto{\pgfqpoint{2.433700in}{2.364789in}}%
\pgfpathlineto{\pgfqpoint{2.442328in}{2.363757in}}%
\pgfpathlineto{\pgfqpoint{2.450937in}{2.362985in}}%
\pgfpathlineto{\pgfqpoint{2.459530in}{2.362468in}}%
\pgfpathlineto{\pgfqpoint{2.468105in}{2.362201in}}%
\pgfpathlineto{\pgfqpoint{2.454621in}{2.381339in}}%
\pgfpathlineto{\pgfqpoint{2.441126in}{2.400753in}}%
\pgfpathlineto{\pgfqpoint{2.427620in}{2.420444in}}%
\pgfpathlineto{\pgfqpoint{2.414103in}{2.440414in}}%
\pgfpathlineto{\pgfqpoint{2.405482in}{2.441172in}}%
\pgfpathlineto{\pgfqpoint{2.396842in}{2.442188in}}%
\pgfpathlineto{\pgfqpoint{2.388185in}{2.443466in}}%
\pgfpathlineto{\pgfqpoint{2.379509in}{2.445012in}}%
\pgfpathclose%
\pgfusepath{fill}%
\end{pgfscope}%
\begin{pgfscope}%
\pgfpathrectangle{\pgfqpoint{1.254980in}{0.150000in}}{\pgfqpoint{5.490039in}{5.490039in}}%
\pgfusepath{clip}%
\pgfsetbuttcap%
\pgfsetroundjoin%
\definecolor{currentfill}{rgb}{0.132268,0.655014,0.519661}%
\pgfsetfillcolor{currentfill}%
\pgfsetfillopacity{0.700000}%
\pgfsetlinewidth{0.000000pt}%
\definecolor{currentstroke}{rgb}{0.000000,0.000000,0.000000}%
\pgfsetstrokecolor{currentstroke}%
\pgfsetdash{}{0pt}%
\pgfpathmoveto{\pgfqpoint{5.534417in}{3.161498in}}%
\pgfpathlineto{\pgfqpoint{5.548569in}{3.173029in}}%
\pgfpathlineto{\pgfqpoint{5.562741in}{3.184716in}}%
\pgfpathlineto{\pgfqpoint{5.576932in}{3.196562in}}%
\pgfpathlineto{\pgfqpoint{5.591142in}{3.208565in}}%
\pgfpathlineto{\pgfqpoint{5.598166in}{3.211052in}}%
\pgfpathlineto{\pgfqpoint{5.605183in}{3.213505in}}%
\pgfpathlineto{\pgfqpoint{5.612192in}{3.215929in}}%
\pgfpathlineto{\pgfqpoint{5.619194in}{3.218329in}}%
\pgfpathlineto{\pgfqpoint{5.605008in}{3.206760in}}%
\pgfpathlineto{\pgfqpoint{5.590841in}{3.195347in}}%
\pgfpathlineto{\pgfqpoint{5.576694in}{3.184091in}}%
\pgfpathlineto{\pgfqpoint{5.562564in}{3.172992in}}%
\pgfpathlineto{\pgfqpoint{5.555538in}{3.170150in}}%
\pgfpathlineto{\pgfqpoint{5.548505in}{3.167290in}}%
\pgfpathlineto{\pgfqpoint{5.541464in}{3.164407in}}%
\pgfpathlineto{\pgfqpoint{5.534417in}{3.161498in}}%
\pgfpathclose%
\pgfusepath{fill}%
\end{pgfscope}%
\begin{pgfscope}%
\pgfpathrectangle{\pgfqpoint{1.254980in}{0.150000in}}{\pgfqpoint{5.490039in}{5.490039in}}%
\pgfusepath{clip}%
\pgfsetbuttcap%
\pgfsetroundjoin%
\definecolor{currentfill}{rgb}{0.282656,0.100196,0.422160}%
\pgfsetfillcolor{currentfill}%
\pgfsetfillopacity{0.700000}%
\pgfsetlinewidth{0.000000pt}%
\definecolor{currentstroke}{rgb}{0.000000,0.000000,0.000000}%
\pgfsetstrokecolor{currentstroke}%
\pgfsetdash{}{0pt}%
\pgfpathmoveto{\pgfqpoint{2.862359in}{1.868938in}}%
\pgfpathlineto{\pgfqpoint{2.875664in}{1.857236in}}%
\pgfpathlineto{\pgfqpoint{2.888966in}{1.845745in}}%
\pgfpathlineto{\pgfqpoint{2.902265in}{1.834464in}}%
\pgfpathlineto{\pgfqpoint{2.915561in}{1.823393in}}%
\pgfpathlineto{\pgfqpoint{2.923832in}{1.826838in}}%
\pgfpathlineto{\pgfqpoint{2.932090in}{1.830474in}}%
\pgfpathlineto{\pgfqpoint{2.940337in}{1.834295in}}%
\pgfpathlineto{\pgfqpoint{2.948572in}{1.838298in}}%
\pgfpathlineto{\pgfqpoint{2.935307in}{1.848894in}}%
\pgfpathlineto{\pgfqpoint{2.922040in}{1.859698in}}%
\pgfpathlineto{\pgfqpoint{2.908771in}{1.870713in}}%
\pgfpathlineto{\pgfqpoint{2.895499in}{1.881939in}}%
\pgfpathlineto{\pgfqpoint{2.887232in}{1.878401in}}%
\pgfpathlineto{\pgfqpoint{2.878954in}{1.875052in}}%
\pgfpathlineto{\pgfqpoint{2.870663in}{1.871896in}}%
\pgfpathlineto{\pgfqpoint{2.862359in}{1.868938in}}%
\pgfpathclose%
\pgfusepath{fill}%
\end{pgfscope}%
\begin{pgfscope}%
\pgfpathrectangle{\pgfqpoint{1.254980in}{0.150000in}}{\pgfqpoint{5.490039in}{5.490039in}}%
\pgfusepath{clip}%
\pgfsetbuttcap%
\pgfsetroundjoin%
\definecolor{currentfill}{rgb}{0.183898,0.422383,0.556944}%
\pgfsetfillcolor{currentfill}%
\pgfsetfillopacity{0.700000}%
\pgfsetlinewidth{0.000000pt}%
\definecolor{currentstroke}{rgb}{0.000000,0.000000,0.000000}%
\pgfsetstrokecolor{currentstroke}%
\pgfsetdash{}{0pt}%
\pgfpathmoveto{\pgfqpoint{4.712708in}{2.538562in}}%
\pgfpathlineto{\pgfqpoint{4.726396in}{2.547116in}}%
\pgfpathlineto{\pgfqpoint{4.740100in}{2.555831in}}%
\pgfpathlineto{\pgfqpoint{4.753818in}{2.564707in}}%
\pgfpathlineto{\pgfqpoint{4.767552in}{2.573743in}}%
\pgfpathlineto{\pgfqpoint{4.775039in}{2.582158in}}%
\pgfpathlineto{\pgfqpoint{4.782520in}{2.590469in}}%
\pgfpathlineto{\pgfqpoint{4.789995in}{2.598679in}}%
\pgfpathlineto{\pgfqpoint{4.797462in}{2.606789in}}%
\pgfpathlineto{\pgfqpoint{4.783735in}{2.597852in}}%
\pgfpathlineto{\pgfqpoint{4.770024in}{2.589075in}}%
\pgfpathlineto{\pgfqpoint{4.756327in}{2.580459in}}%
\pgfpathlineto{\pgfqpoint{4.742646in}{2.572003in}}%
\pgfpathlineto{\pgfqpoint{4.735171in}{2.563783in}}%
\pgfpathlineto{\pgfqpoint{4.727690in}{2.555471in}}%
\pgfpathlineto{\pgfqpoint{4.720202in}{2.547064in}}%
\pgfpathlineto{\pgfqpoint{4.712708in}{2.538562in}}%
\pgfpathclose%
\pgfusepath{fill}%
\end{pgfscope}%
\begin{pgfscope}%
\pgfpathrectangle{\pgfqpoint{1.254980in}{0.150000in}}{\pgfqpoint{5.490039in}{5.490039in}}%
\pgfusepath{clip}%
\pgfsetbuttcap%
\pgfsetroundjoin%
\definecolor{currentfill}{rgb}{0.227802,0.326594,0.546532}%
\pgfsetfillcolor{currentfill}%
\pgfsetfillopacity{0.700000}%
\pgfsetlinewidth{0.000000pt}%
\definecolor{currentstroke}{rgb}{0.000000,0.000000,0.000000}%
\pgfsetstrokecolor{currentstroke}%
\pgfsetdash{}{0pt}%
\pgfpathmoveto{\pgfqpoint{4.428530in}{2.297369in}}%
\pgfpathlineto{\pgfqpoint{4.442073in}{2.304089in}}%
\pgfpathlineto{\pgfqpoint{4.455628in}{2.310971in}}%
\pgfpathlineto{\pgfqpoint{4.469197in}{2.318014in}}%
\pgfpathlineto{\pgfqpoint{4.482779in}{2.325220in}}%
\pgfpathlineto{\pgfqpoint{4.490382in}{2.335321in}}%
\pgfpathlineto{\pgfqpoint{4.497979in}{2.345330in}}%
\pgfpathlineto{\pgfqpoint{4.505570in}{2.355246in}}%
\pgfpathlineto{\pgfqpoint{4.513155in}{2.365069in}}%
\pgfpathlineto{\pgfqpoint{4.499578in}{2.357844in}}%
\pgfpathlineto{\pgfqpoint{4.486013in}{2.350782in}}%
\pgfpathlineto{\pgfqpoint{4.472462in}{2.343881in}}%
\pgfpathlineto{\pgfqpoint{4.458924in}{2.337142in}}%
\pgfpathlineto{\pgfqpoint{4.451334in}{2.327327in}}%
\pgfpathlineto{\pgfqpoint{4.443739in}{2.317427in}}%
\pgfpathlineto{\pgfqpoint{4.436137in}{2.307441in}}%
\pgfpathlineto{\pgfqpoint{4.428530in}{2.297369in}}%
\pgfpathclose%
\pgfusepath{fill}%
\end{pgfscope}%
\begin{pgfscope}%
\pgfpathrectangle{\pgfqpoint{1.254980in}{0.150000in}}{\pgfqpoint{5.490039in}{5.490039in}}%
\pgfusepath{clip}%
\pgfsetbuttcap%
\pgfsetroundjoin%
\definecolor{currentfill}{rgb}{0.268510,0.009605,0.335427}%
\pgfsetfillcolor{currentfill}%
\pgfsetfillopacity{0.700000}%
\pgfsetlinewidth{0.000000pt}%
\definecolor{currentstroke}{rgb}{0.000000,0.000000,0.000000}%
\pgfsetstrokecolor{currentstroke}%
\pgfsetdash{}{0pt}%
\pgfpathmoveto{\pgfqpoint{3.245981in}{1.694951in}}%
\pgfpathlineto{\pgfqpoint{3.259218in}{1.688913in}}%
\pgfpathlineto{\pgfqpoint{3.272457in}{1.683059in}}%
\pgfpathlineto{\pgfqpoint{3.285698in}{1.677390in}}%
\pgfpathlineto{\pgfqpoint{3.298941in}{1.671904in}}%
\pgfpathlineto{\pgfqpoint{3.306975in}{1.679293in}}%
\pgfpathlineto{\pgfqpoint{3.315001in}{1.686793in}}%
\pgfpathlineto{\pgfqpoint{3.323019in}{1.694402in}}%
\pgfpathlineto{\pgfqpoint{3.331029in}{1.702116in}}%
\pgfpathlineto{\pgfqpoint{3.317806in}{1.707190in}}%
\pgfpathlineto{\pgfqpoint{3.304585in}{1.712447in}}%
\pgfpathlineto{\pgfqpoint{3.291367in}{1.717889in}}%
\pgfpathlineto{\pgfqpoint{3.278151in}{1.723515in}}%
\pgfpathlineto{\pgfqpoint{3.270121in}{1.716203in}}%
\pgfpathlineto{\pgfqpoint{3.262082in}{1.709002in}}%
\pgfpathlineto{\pgfqpoint{3.254036in}{1.701917in}}%
\pgfpathlineto{\pgfqpoint{3.245981in}{1.694951in}}%
\pgfpathclose%
\pgfusepath{fill}%
\end{pgfscope}%
\begin{pgfscope}%
\pgfpathrectangle{\pgfqpoint{1.254980in}{0.150000in}}{\pgfqpoint{5.490039in}{5.490039in}}%
\pgfusepath{clip}%
\pgfsetbuttcap%
\pgfsetroundjoin%
\definecolor{currentfill}{rgb}{0.269308,0.218818,0.509577}%
\pgfsetfillcolor{currentfill}%
\pgfsetfillopacity{0.700000}%
\pgfsetlinewidth{0.000000pt}%
\definecolor{currentstroke}{rgb}{0.000000,0.000000,0.000000}%
\pgfsetstrokecolor{currentstroke}%
\pgfsetdash{}{0pt}%
\pgfpathmoveto{\pgfqpoint{4.144342in}{2.061068in}}%
\pgfpathlineto{\pgfqpoint{4.157758in}{2.065471in}}%
\pgfpathlineto{\pgfqpoint{4.171185in}{2.070038in}}%
\pgfpathlineto{\pgfqpoint{4.184624in}{2.074768in}}%
\pgfpathlineto{\pgfqpoint{4.198073in}{2.079662in}}%
\pgfpathlineto{\pgfqpoint{4.205772in}{2.090784in}}%
\pgfpathlineto{\pgfqpoint{4.213465in}{2.101841in}}%
\pgfpathlineto{\pgfqpoint{4.221154in}{2.112833in}}%
\pgfpathlineto{\pgfqpoint{4.228837in}{2.123759in}}%
\pgfpathlineto{\pgfqpoint{4.215392in}{2.118732in}}%
\pgfpathlineto{\pgfqpoint{4.201958in}{2.113868in}}%
\pgfpathlineto{\pgfqpoint{4.188535in}{2.109167in}}%
\pgfpathlineto{\pgfqpoint{4.175123in}{2.104631in}}%
\pgfpathlineto{\pgfqpoint{4.167435in}{2.093828in}}%
\pgfpathlineto{\pgfqpoint{4.159742in}{2.082966in}}%
\pgfpathlineto{\pgfqpoint{4.152044in}{2.072046in}}%
\pgfpathlineto{\pgfqpoint{4.144342in}{2.061068in}}%
\pgfpathclose%
\pgfusepath{fill}%
\end{pgfscope}%
\begin{pgfscope}%
\pgfpathrectangle{\pgfqpoint{1.254980in}{0.150000in}}{\pgfqpoint{5.490039in}{5.490039in}}%
\pgfusepath{clip}%
\pgfsetbuttcap%
\pgfsetroundjoin%
\definecolor{currentfill}{rgb}{0.150148,0.676631,0.506589}%
\pgfsetfillcolor{currentfill}%
\pgfsetfillopacity{0.700000}%
\pgfsetlinewidth{0.000000pt}%
\definecolor{currentstroke}{rgb}{0.000000,0.000000,0.000000}%
\pgfsetstrokecolor{currentstroke}%
\pgfsetdash{}{0pt}%
\pgfpathmoveto{\pgfqpoint{5.619194in}{3.218329in}}%
\pgfpathlineto{\pgfqpoint{5.633400in}{3.230056in}}%
\pgfpathlineto{\pgfqpoint{5.647624in}{3.241940in}}%
\pgfpathlineto{\pgfqpoint{5.661868in}{3.253981in}}%
\pgfpathlineto{\pgfqpoint{5.676132in}{3.266180in}}%
\pgfpathlineto{\pgfqpoint{5.683101in}{3.268107in}}%
\pgfpathlineto{\pgfqpoint{5.690063in}{3.270013in}}%
\pgfpathlineto{\pgfqpoint{5.697018in}{3.271903in}}%
\pgfpathlineto{\pgfqpoint{5.703966in}{3.273782in}}%
\pgfpathlineto{\pgfqpoint{5.689729in}{3.262048in}}%
\pgfpathlineto{\pgfqpoint{5.675511in}{3.250471in}}%
\pgfpathlineto{\pgfqpoint{5.661312in}{3.239050in}}%
\pgfpathlineto{\pgfqpoint{5.647133in}{3.227785in}}%
\pgfpathlineto{\pgfqpoint{5.640158in}{3.225433in}}%
\pgfpathlineto{\pgfqpoint{5.633177in}{3.223076in}}%
\pgfpathlineto{\pgfqpoint{5.626189in}{3.220710in}}%
\pgfpathlineto{\pgfqpoint{5.619194in}{3.218329in}}%
\pgfpathclose%
\pgfusepath{fill}%
\end{pgfscope}%
\begin{pgfscope}%
\pgfpathrectangle{\pgfqpoint{1.254980in}{0.150000in}}{\pgfqpoint{5.490039in}{5.490039in}}%
\pgfusepath{clip}%
\pgfsetbuttcap%
\pgfsetroundjoin%
\definecolor{currentfill}{rgb}{0.137770,0.537492,0.554906}%
\pgfsetfillcolor{currentfill}%
\pgfsetfillopacity{0.700000}%
\pgfsetlinewidth{0.000000pt}%
\definecolor{currentstroke}{rgb}{0.000000,0.000000,0.000000}%
\pgfsetstrokecolor{currentstroke}%
\pgfsetdash{}{0pt}%
\pgfpathmoveto{\pgfqpoint{5.081507in}{2.835844in}}%
\pgfpathlineto{\pgfqpoint{5.095404in}{2.846162in}}%
\pgfpathlineto{\pgfqpoint{5.109318in}{2.856640in}}%
\pgfpathlineto{\pgfqpoint{5.123249in}{2.867277in}}%
\pgfpathlineto{\pgfqpoint{5.137197in}{2.878073in}}%
\pgfpathlineto{\pgfqpoint{5.144501in}{2.883835in}}%
\pgfpathlineto{\pgfqpoint{5.151797in}{2.889508in}}%
\pgfpathlineto{\pgfqpoint{5.159085in}{2.895094in}}%
\pgfpathlineto{\pgfqpoint{5.166365in}{2.900596in}}%
\pgfpathlineto{\pgfqpoint{5.152430in}{2.890050in}}%
\pgfpathlineto{\pgfqpoint{5.138512in}{2.879663in}}%
\pgfpathlineto{\pgfqpoint{5.124611in}{2.869435in}}%
\pgfpathlineto{\pgfqpoint{5.110727in}{2.859365in}}%
\pgfpathlineto{\pgfqpoint{5.103433in}{2.853602in}}%
\pgfpathlineto{\pgfqpoint{5.096132in}{2.847763in}}%
\pgfpathlineto{\pgfqpoint{5.088823in}{2.841845in}}%
\pgfpathlineto{\pgfqpoint{5.081507in}{2.835844in}}%
\pgfpathclose%
\pgfusepath{fill}%
\end{pgfscope}%
\begin{pgfscope}%
\pgfpathrectangle{\pgfqpoint{1.254980in}{0.150000in}}{\pgfqpoint{5.490039in}{5.490039in}}%
\pgfusepath{clip}%
\pgfsetbuttcap%
\pgfsetroundjoin%
\definecolor{currentfill}{rgb}{0.272594,0.025563,0.353093}%
\pgfsetfillcolor{currentfill}%
\pgfsetfillopacity{0.700000}%
\pgfsetlinewidth{0.000000pt}%
\definecolor{currentstroke}{rgb}{0.000000,0.000000,0.000000}%
\pgfsetstrokecolor{currentstroke}%
\pgfsetdash{}{0pt}%
\pgfpathmoveto{\pgfqpoint{3.107639in}{1.726971in}}%
\pgfpathlineto{\pgfqpoint{3.120891in}{1.718978in}}%
\pgfpathlineto{\pgfqpoint{3.134144in}{1.711177in}}%
\pgfpathlineto{\pgfqpoint{3.147397in}{1.703568in}}%
\pgfpathlineto{\pgfqpoint{3.160650in}{1.696150in}}%
\pgfpathlineto{\pgfqpoint{3.168764in}{1.702145in}}%
\pgfpathlineto{\pgfqpoint{3.176869in}{1.708284in}}%
\pgfpathlineto{\pgfqpoint{3.184964in}{1.714561in}}%
\pgfpathlineto{\pgfqpoint{3.193050in}{1.720973in}}%
\pgfpathlineto{\pgfqpoint{3.179820in}{1.727950in}}%
\pgfpathlineto{\pgfqpoint{3.166592in}{1.735118in}}%
\pgfpathlineto{\pgfqpoint{3.153364in}{1.742477in}}%
\pgfpathlineto{\pgfqpoint{3.140137in}{1.750028in}}%
\pgfpathlineto{\pgfqpoint{3.132027in}{1.744046in}}%
\pgfpathlineto{\pgfqpoint{3.123908in}{1.738207in}}%
\pgfpathlineto{\pgfqpoint{3.115778in}{1.732514in}}%
\pgfpathlineto{\pgfqpoint{3.107639in}{1.726971in}}%
\pgfpathclose%
\pgfusepath{fill}%
\end{pgfscope}%
\begin{pgfscope}%
\pgfpathrectangle{\pgfqpoint{1.254980in}{0.150000in}}{\pgfqpoint{5.490039in}{5.490039in}}%
\pgfusepath{clip}%
\pgfsetbuttcap%
\pgfsetroundjoin%
\definecolor{currentfill}{rgb}{0.197636,0.391528,0.554969}%
\pgfsetfillcolor{currentfill}%
\pgfsetfillopacity{0.700000}%
\pgfsetlinewidth{0.000000pt}%
\definecolor{currentstroke}{rgb}{0.000000,0.000000,0.000000}%
\pgfsetstrokecolor{currentstroke}%
\pgfsetdash{}{0pt}%
\pgfpathmoveto{\pgfqpoint{2.325125in}{2.529777in}}%
\pgfpathlineto{\pgfqpoint{2.338741in}{2.508151in}}%
\pgfpathlineto{\pgfqpoint{2.352343in}{2.486816in}}%
\pgfpathlineto{\pgfqpoint{2.365932in}{2.465771in}}%
\pgfpathlineto{\pgfqpoint{2.379509in}{2.445012in}}%
\pgfpathlineto{\pgfqpoint{2.388185in}{2.443466in}}%
\pgfpathlineto{\pgfqpoint{2.396842in}{2.442188in}}%
\pgfpathlineto{\pgfqpoint{2.405482in}{2.441172in}}%
\pgfpathlineto{\pgfqpoint{2.414103in}{2.440414in}}%
\pgfpathlineto{\pgfqpoint{2.400574in}{2.460668in}}%
\pgfpathlineto{\pgfqpoint{2.387034in}{2.481207in}}%
\pgfpathlineto{\pgfqpoint{2.373481in}{2.502034in}}%
\pgfpathlineto{\pgfqpoint{2.359916in}{2.523152in}}%
\pgfpathlineto{\pgfqpoint{2.351246in}{2.524404in}}%
\pgfpathlineto{\pgfqpoint{2.342558in}{2.525923in}}%
\pgfpathlineto{\pgfqpoint{2.333851in}{2.527712in}}%
\pgfpathlineto{\pgfqpoint{2.325125in}{2.529777in}}%
\pgfpathclose%
\pgfusepath{fill}%
\end{pgfscope}%
\begin{pgfscope}%
\pgfpathrectangle{\pgfqpoint{1.254980in}{0.150000in}}{\pgfqpoint{5.490039in}{5.490039in}}%
\pgfusepath{clip}%
\pgfsetbuttcap%
\pgfsetroundjoin%
\definecolor{currentfill}{rgb}{0.268510,0.009605,0.335427}%
\pgfsetfillcolor{currentfill}%
\pgfsetfillopacity{0.700000}%
\pgfsetlinewidth{0.000000pt}%
\definecolor{currentstroke}{rgb}{0.000000,0.000000,0.000000}%
\pgfsetstrokecolor{currentstroke}%
\pgfsetdash{}{0pt}%
\pgfpathmoveto{\pgfqpoint{3.383952in}{1.683637in}}%
\pgfpathlineto{\pgfqpoint{3.397191in}{1.679469in}}%
\pgfpathlineto{\pgfqpoint{3.410434in}{1.675479in}}%
\pgfpathlineto{\pgfqpoint{3.423680in}{1.671667in}}%
\pgfpathlineto{\pgfqpoint{3.436930in}{1.668032in}}%
\pgfpathlineto{\pgfqpoint{3.444897in}{1.676634in}}%
\pgfpathlineto{\pgfqpoint{3.452858in}{1.685319in}}%
\pgfpathlineto{\pgfqpoint{3.460812in}{1.694084in}}%
\pgfpathlineto{\pgfqpoint{3.468759in}{1.702924in}}%
\pgfpathlineto{\pgfqpoint{3.455525in}{1.706175in}}%
\pgfpathlineto{\pgfqpoint{3.442295in}{1.709604in}}%
\pgfpathlineto{\pgfqpoint{3.429069in}{1.713211in}}%
\pgfpathlineto{\pgfqpoint{3.415847in}{1.716996in}}%
\pgfpathlineto{\pgfqpoint{3.407884in}{1.708529in}}%
\pgfpathlineto{\pgfqpoint{3.399914in}{1.700144in}}%
\pgfpathlineto{\pgfqpoint{3.391937in}{1.691846in}}%
\pgfpathlineto{\pgfqpoint{3.383952in}{1.683637in}}%
\pgfpathclose%
\pgfusepath{fill}%
\end{pgfscope}%
\begin{pgfscope}%
\pgfpathrectangle{\pgfqpoint{1.254980in}{0.150000in}}{\pgfqpoint{5.490039in}{5.490039in}}%
\pgfusepath{clip}%
\pgfsetbuttcap%
\pgfsetroundjoin%
\definecolor{currentfill}{rgb}{0.175707,0.697900,0.491033}%
\pgfsetfillcolor{currentfill}%
\pgfsetfillopacity{0.700000}%
\pgfsetlinewidth{0.000000pt}%
\definecolor{currentstroke}{rgb}{0.000000,0.000000,0.000000}%
\pgfsetstrokecolor{currentstroke}%
\pgfsetdash{}{0pt}%
\pgfpathmoveto{\pgfqpoint{5.703966in}{3.273782in}}%
\pgfpathlineto{\pgfqpoint{5.718223in}{3.285673in}}%
\pgfpathlineto{\pgfqpoint{5.732499in}{3.297721in}}%
\pgfpathlineto{\pgfqpoint{5.746796in}{3.309925in}}%
\pgfpathlineto{\pgfqpoint{5.761112in}{3.322288in}}%
\pgfpathlineto{\pgfqpoint{5.768025in}{3.323676in}}%
\pgfpathlineto{\pgfqpoint{5.774931in}{3.325058in}}%
\pgfpathlineto{\pgfqpoint{5.781831in}{3.326437in}}%
\pgfpathlineto{\pgfqpoint{5.788723in}{3.327820in}}%
\pgfpathlineto{\pgfqpoint{5.774436in}{3.315953in}}%
\pgfpathlineto{\pgfqpoint{5.760168in}{3.304243in}}%
\pgfpathlineto{\pgfqpoint{5.745919in}{3.292689in}}%
\pgfpathlineto{\pgfqpoint{5.731691in}{3.281290in}}%
\pgfpathlineto{\pgfqpoint{5.724769in}{3.279404in}}%
\pgfpathlineto{\pgfqpoint{5.717841in}{3.277527in}}%
\pgfpathlineto{\pgfqpoint{5.710907in}{3.275655in}}%
\pgfpathlineto{\pgfqpoint{5.703966in}{3.273782in}}%
\pgfpathclose%
\pgfusepath{fill}%
\end{pgfscope}%
\begin{pgfscope}%
\pgfpathrectangle{\pgfqpoint{1.254980in}{0.150000in}}{\pgfqpoint{5.490039in}{5.490039in}}%
\pgfusepath{clip}%
\pgfsetbuttcap%
\pgfsetroundjoin%
\definecolor{currentfill}{rgb}{0.280894,0.078907,0.402329}%
\pgfsetfillcolor{currentfill}%
\pgfsetfillopacity{0.700000}%
\pgfsetlinewidth{0.000000pt}%
\definecolor{currentstroke}{rgb}{0.000000,0.000000,0.000000}%
\pgfsetstrokecolor{currentstroke}%
\pgfsetdash{}{0pt}%
\pgfpathmoveto{\pgfqpoint{2.915561in}{1.823393in}}%
\pgfpathlineto{\pgfqpoint{2.928855in}{1.812530in}}%
\pgfpathlineto{\pgfqpoint{2.942147in}{1.801872in}}%
\pgfpathlineto{\pgfqpoint{2.955437in}{1.791421in}}%
\pgfpathlineto{\pgfqpoint{2.968725in}{1.781173in}}%
\pgfpathlineto{\pgfqpoint{2.976963in}{1.785103in}}%
\pgfpathlineto{\pgfqpoint{2.985191in}{1.789217in}}%
\pgfpathlineto{\pgfqpoint{2.993407in}{1.793509in}}%
\pgfpathlineto{\pgfqpoint{3.001611in}{1.797974in}}%
\pgfpathlineto{\pgfqpoint{2.988354in}{1.807748in}}%
\pgfpathlineto{\pgfqpoint{2.975095in}{1.817726in}}%
\pgfpathlineto{\pgfqpoint{2.961834in}{1.827909in}}%
\pgfpathlineto{\pgfqpoint{2.948572in}{1.838298in}}%
\pgfpathlineto{\pgfqpoint{2.940337in}{1.834295in}}%
\pgfpathlineto{\pgfqpoint{2.932090in}{1.830474in}}%
\pgfpathlineto{\pgfqpoint{2.923832in}{1.826838in}}%
\pgfpathlineto{\pgfqpoint{2.915561in}{1.823393in}}%
\pgfpathclose%
\pgfusepath{fill}%
\end{pgfscope}%
\begin{pgfscope}%
\pgfpathrectangle{\pgfqpoint{1.254980in}{0.150000in}}{\pgfqpoint{5.490039in}{5.490039in}}%
\pgfusepath{clip}%
\pgfsetbuttcap%
\pgfsetroundjoin%
\definecolor{currentfill}{rgb}{0.266941,0.748751,0.440573}%
\pgfsetfillcolor{currentfill}%
\pgfsetfillopacity{0.700000}%
\pgfsetlinewidth{0.000000pt}%
\definecolor{currentstroke}{rgb}{0.000000,0.000000,0.000000}%
\pgfsetstrokecolor{currentstroke}%
\pgfsetdash{}{0pt}%
\pgfpathmoveto{\pgfqpoint{5.958166in}{3.431606in}}%
\pgfpathlineto{\pgfqpoint{5.972572in}{3.443798in}}%
\pgfpathlineto{\pgfqpoint{5.986998in}{3.456145in}}%
\pgfpathlineto{\pgfqpoint{6.001444in}{3.468649in}}%
\pgfpathlineto{\pgfqpoint{6.008193in}{3.468744in}}%
\pgfpathlineto{\pgfqpoint{6.014936in}{3.468883in}}%
\pgfpathlineto{\pgfqpoint{6.021675in}{3.469070in}}%
\pgfpathlineto{\pgfqpoint{6.028408in}{3.469314in}}%
\pgfpathlineto{\pgfqpoint{6.013998in}{3.457397in}}%
\pgfpathlineto{\pgfqpoint{5.999608in}{3.445635in}}%
\pgfpathlineto{\pgfqpoint{5.985238in}{3.434027in}}%
\pgfpathlineto{\pgfqpoint{5.978477in}{3.433338in}}%
\pgfpathlineto{\pgfqpoint{5.971712in}{3.432708in}}%
\pgfpathlineto{\pgfqpoint{5.964942in}{3.432133in}}%
\pgfpathlineto{\pgfqpoint{5.958166in}{3.431606in}}%
\pgfpathclose%
\pgfusepath{fill}%
\end{pgfscope}%
\begin{pgfscope}%
\pgfpathrectangle{\pgfqpoint{1.254980in}{0.150000in}}{\pgfqpoint{5.490039in}{5.490039in}}%
\pgfusepath{clip}%
\pgfsetbuttcap%
\pgfsetroundjoin%
\definecolor{currentfill}{rgb}{0.171176,0.452530,0.557965}%
\pgfsetfillcolor{currentfill}%
\pgfsetfillopacity{0.700000}%
\pgfsetlinewidth{0.000000pt}%
\definecolor{currentstroke}{rgb}{0.000000,0.000000,0.000000}%
\pgfsetstrokecolor{currentstroke}%
\pgfsetdash{}{0pt}%
\pgfpathmoveto{\pgfqpoint{4.797462in}{2.606789in}}%
\pgfpathlineto{\pgfqpoint{4.811204in}{2.615886in}}%
\pgfpathlineto{\pgfqpoint{4.824961in}{2.625144in}}%
\pgfpathlineto{\pgfqpoint{4.838734in}{2.634562in}}%
\pgfpathlineto{\pgfqpoint{4.852523in}{2.644141in}}%
\pgfpathlineto{\pgfqpoint{4.859976in}{2.652034in}}%
\pgfpathlineto{\pgfqpoint{4.867422in}{2.659823in}}%
\pgfpathlineto{\pgfqpoint{4.874861in}{2.667510in}}%
\pgfpathlineto{\pgfqpoint{4.882293in}{2.675096in}}%
\pgfpathlineto{\pgfqpoint{4.868512in}{2.665647in}}%
\pgfpathlineto{\pgfqpoint{4.854747in}{2.656358in}}%
\pgfpathlineto{\pgfqpoint{4.840998in}{2.647229in}}%
\pgfpathlineto{\pgfqpoint{4.827264in}{2.638260in}}%
\pgfpathlineto{\pgfqpoint{4.819823in}{2.630534in}}%
\pgfpathlineto{\pgfqpoint{4.812376in}{2.622715in}}%
\pgfpathlineto{\pgfqpoint{4.804923in}{2.614801in}}%
\pgfpathlineto{\pgfqpoint{4.797462in}{2.606789in}}%
\pgfpathclose%
\pgfusepath{fill}%
\end{pgfscope}%
\begin{pgfscope}%
\pgfpathrectangle{\pgfqpoint{1.254980in}{0.150000in}}{\pgfqpoint{5.490039in}{5.490039in}}%
\pgfusepath{clip}%
\pgfsetbuttcap%
\pgfsetroundjoin%
\definecolor{currentfill}{rgb}{0.258965,0.251537,0.524736}%
\pgfsetfillcolor{currentfill}%
\pgfsetfillopacity{0.700000}%
\pgfsetlinewidth{0.000000pt}%
\definecolor{currentstroke}{rgb}{0.000000,0.000000,0.000000}%
\pgfsetstrokecolor{currentstroke}%
\pgfsetdash{}{0pt}%
\pgfpathmoveto{\pgfqpoint{4.228837in}{2.123759in}}%
\pgfpathlineto{\pgfqpoint{4.242294in}{2.128949in}}%
\pgfpathlineto{\pgfqpoint{4.255762in}{2.134303in}}%
\pgfpathlineto{\pgfqpoint{4.269242in}{2.139819in}}%
\pgfpathlineto{\pgfqpoint{4.282734in}{2.145498in}}%
\pgfpathlineto{\pgfqpoint{4.290409in}{2.156472in}}%
\pgfpathlineto{\pgfqpoint{4.298079in}{2.167372in}}%
\pgfpathlineto{\pgfqpoint{4.305743in}{2.178195in}}%
\pgfpathlineto{\pgfqpoint{4.313402in}{2.188941in}}%
\pgfpathlineto{\pgfqpoint{4.299914in}{2.183157in}}%
\pgfpathlineto{\pgfqpoint{4.286438in}{2.177536in}}%
\pgfpathlineto{\pgfqpoint{4.272974in}{2.172077in}}%
\pgfpathlineto{\pgfqpoint{4.259521in}{2.166781in}}%
\pgfpathlineto{\pgfqpoint{4.251858in}{2.156129in}}%
\pgfpathlineto{\pgfqpoint{4.244189in}{2.145407in}}%
\pgfpathlineto{\pgfqpoint{4.236516in}{2.134617in}}%
\pgfpathlineto{\pgfqpoint{4.228837in}{2.123759in}}%
\pgfpathclose%
\pgfusepath{fill}%
\end{pgfscope}%
\begin{pgfscope}%
\pgfpathrectangle{\pgfqpoint{1.254980in}{0.150000in}}{\pgfqpoint{5.490039in}{5.490039in}}%
\pgfusepath{clip}%
\pgfsetbuttcap%
\pgfsetroundjoin%
\definecolor{currentfill}{rgb}{0.202219,0.715272,0.476084}%
\pgfsetfillcolor{currentfill}%
\pgfsetfillopacity{0.700000}%
\pgfsetlinewidth{0.000000pt}%
\definecolor{currentstroke}{rgb}{0.000000,0.000000,0.000000}%
\pgfsetstrokecolor{currentstroke}%
\pgfsetdash{}{0pt}%
\pgfpathmoveto{\pgfqpoint{5.788723in}{3.327820in}}%
\pgfpathlineto{\pgfqpoint{5.803031in}{3.339843in}}%
\pgfpathlineto{\pgfqpoint{5.817358in}{3.352022in}}%
\pgfpathlineto{\pgfqpoint{5.831706in}{3.364359in}}%
\pgfpathlineto{\pgfqpoint{5.846074in}{3.376852in}}%
\pgfpathlineto{\pgfqpoint{5.852930in}{3.377729in}}%
\pgfpathlineto{\pgfqpoint{5.859779in}{3.378613in}}%
\pgfpathlineto{\pgfqpoint{5.866622in}{3.379510in}}%
\pgfpathlineto{\pgfqpoint{5.873459in}{3.380426in}}%
\pgfpathlineto{\pgfqpoint{5.859122in}{3.368459in}}%
\pgfpathlineto{\pgfqpoint{5.844805in}{3.356648in}}%
\pgfpathlineto{\pgfqpoint{5.830509in}{3.344993in}}%
\pgfpathlineto{\pgfqpoint{5.816232in}{3.333493in}}%
\pgfpathlineto{\pgfqpoint{5.809363in}{3.332042in}}%
\pgfpathlineto{\pgfqpoint{5.802489in}{3.330616in}}%
\pgfpathlineto{\pgfqpoint{5.795609in}{3.329211in}}%
\pgfpathlineto{\pgfqpoint{5.788723in}{3.327820in}}%
\pgfpathclose%
\pgfusepath{fill}%
\end{pgfscope}%
\begin{pgfscope}%
\pgfpathrectangle{\pgfqpoint{1.254980in}{0.150000in}}{\pgfqpoint{5.490039in}{5.490039in}}%
\pgfusepath{clip}%
\pgfsetbuttcap%
\pgfsetroundjoin%
\definecolor{currentfill}{rgb}{0.212395,0.359683,0.551710}%
\pgfsetfillcolor{currentfill}%
\pgfsetfillopacity{0.700000}%
\pgfsetlinewidth{0.000000pt}%
\definecolor{currentstroke}{rgb}{0.000000,0.000000,0.000000}%
\pgfsetstrokecolor{currentstroke}%
\pgfsetdash{}{0pt}%
\pgfpathmoveto{\pgfqpoint{4.513155in}{2.365069in}}%
\pgfpathlineto{\pgfqpoint{4.526747in}{2.372454in}}%
\pgfpathlineto{\pgfqpoint{4.540352in}{2.380001in}}%
\pgfpathlineto{\pgfqpoint{4.553970in}{2.387710in}}%
\pgfpathlineto{\pgfqpoint{4.567603in}{2.395579in}}%
\pgfpathlineto{\pgfqpoint{4.575178in}{2.405310in}}%
\pgfpathlineto{\pgfqpoint{4.582747in}{2.414942in}}%
\pgfpathlineto{\pgfqpoint{4.590310in}{2.424475in}}%
\pgfpathlineto{\pgfqpoint{4.597867in}{2.433910in}}%
\pgfpathlineto{\pgfqpoint{4.584239in}{2.426051in}}%
\pgfpathlineto{\pgfqpoint{4.570625in}{2.418353in}}%
\pgfpathlineto{\pgfqpoint{4.557025in}{2.410817in}}%
\pgfpathlineto{\pgfqpoint{4.543439in}{2.403441in}}%
\pgfpathlineto{\pgfqpoint{4.535877in}{2.393985in}}%
\pgfpathlineto{\pgfqpoint{4.528309in}{2.384438in}}%
\pgfpathlineto{\pgfqpoint{4.520735in}{2.374799in}}%
\pgfpathlineto{\pgfqpoint{4.513155in}{2.365069in}}%
\pgfpathclose%
\pgfusepath{fill}%
\end{pgfscope}%
\begin{pgfscope}%
\pgfpathrectangle{\pgfqpoint{1.254980in}{0.150000in}}{\pgfqpoint{5.490039in}{5.490039in}}%
\pgfusepath{clip}%
\pgfsetbuttcap%
\pgfsetroundjoin%
\definecolor{currentfill}{rgb}{0.281924,0.089666,0.412415}%
\pgfsetfillcolor{currentfill}%
\pgfsetfillopacity{0.700000}%
\pgfsetlinewidth{0.000000pt}%
\definecolor{currentstroke}{rgb}{0.000000,0.000000,0.000000}%
\pgfsetstrokecolor{currentstroke}%
\pgfsetdash{}{0pt}%
\pgfpathmoveto{\pgfqpoint{3.775499in}{1.801084in}}%
\pgfpathlineto{\pgfqpoint{3.788798in}{1.801747in}}%
\pgfpathlineto{\pgfqpoint{3.802105in}{1.802577in}}%
\pgfpathlineto{\pgfqpoint{3.815420in}{1.803575in}}%
\pgfpathlineto{\pgfqpoint{3.828743in}{1.804741in}}%
\pgfpathlineto{\pgfqpoint{3.836559in}{1.815690in}}%
\pgfpathlineto{\pgfqpoint{3.844371in}{1.826639in}}%
\pgfpathlineto{\pgfqpoint{3.852177in}{1.837583in}}%
\pgfpathlineto{\pgfqpoint{3.859979in}{1.848522in}}%
\pgfpathlineto{\pgfqpoint{3.846664in}{1.847084in}}%
\pgfpathlineto{\pgfqpoint{3.833357in}{1.845813in}}%
\pgfpathlineto{\pgfqpoint{3.820058in}{1.844709in}}%
\pgfpathlineto{\pgfqpoint{3.806767in}{1.843774in}}%
\pgfpathlineto{\pgfqpoint{3.798957in}{1.833098in}}%
\pgfpathlineto{\pgfqpoint{3.791143in}{1.822422in}}%
\pgfpathlineto{\pgfqpoint{3.783323in}{1.811750in}}%
\pgfpathlineto{\pgfqpoint{3.775499in}{1.801084in}}%
\pgfpathclose%
\pgfusepath{fill}%
\end{pgfscope}%
\begin{pgfscope}%
\pgfpathrectangle{\pgfqpoint{1.254980in}{0.150000in}}{\pgfqpoint{5.490039in}{5.490039in}}%
\pgfusepath{clip}%
\pgfsetbuttcap%
\pgfsetroundjoin%
\definecolor{currentfill}{rgb}{0.278791,0.062145,0.386592}%
\pgfsetfillcolor{currentfill}%
\pgfsetfillopacity{0.700000}%
\pgfsetlinewidth{0.000000pt}%
\definecolor{currentstroke}{rgb}{0.000000,0.000000,0.000000}%
\pgfsetstrokecolor{currentstroke}%
\pgfsetdash{}{0pt}%
\pgfpathmoveto{\pgfqpoint{3.690987in}{1.758773in}}%
\pgfpathlineto{\pgfqpoint{3.704267in}{1.758459in}}%
\pgfpathlineto{\pgfqpoint{3.717554in}{1.758314in}}%
\pgfpathlineto{\pgfqpoint{3.730848in}{1.758338in}}%
\pgfpathlineto{\pgfqpoint{3.744149in}{1.758532in}}%
\pgfpathlineto{\pgfqpoint{3.751994in}{1.769148in}}%
\pgfpathlineto{\pgfqpoint{3.759834in}{1.779781in}}%
\pgfpathlineto{\pgfqpoint{3.767669in}{1.790427in}}%
\pgfpathlineto{\pgfqpoint{3.775499in}{1.801084in}}%
\pgfpathlineto{\pgfqpoint{3.762207in}{1.800591in}}%
\pgfpathlineto{\pgfqpoint{3.748922in}{1.800266in}}%
\pgfpathlineto{\pgfqpoint{3.735645in}{1.800110in}}%
\pgfpathlineto{\pgfqpoint{3.722374in}{1.800125in}}%
\pgfpathlineto{\pgfqpoint{3.714535in}{1.789757in}}%
\pgfpathlineto{\pgfqpoint{3.706691in}{1.779408in}}%
\pgfpathlineto{\pgfqpoint{3.698842in}{1.769079in}}%
\pgfpathlineto{\pgfqpoint{3.690987in}{1.758773in}}%
\pgfpathclose%
\pgfusepath{fill}%
\end{pgfscope}%
\begin{pgfscope}%
\pgfpathrectangle{\pgfqpoint{1.254980in}{0.150000in}}{\pgfqpoint{5.490039in}{5.490039in}}%
\pgfusepath{clip}%
\pgfsetbuttcap%
\pgfsetroundjoin%
\definecolor{currentfill}{rgb}{0.128729,0.563265,0.551229}%
\pgfsetfillcolor{currentfill}%
\pgfsetfillopacity{0.700000}%
\pgfsetlinewidth{0.000000pt}%
\definecolor{currentstroke}{rgb}{0.000000,0.000000,0.000000}%
\pgfsetstrokecolor{currentstroke}%
\pgfsetdash{}{0pt}%
\pgfpathmoveto{\pgfqpoint{5.166365in}{2.900596in}}%
\pgfpathlineto{\pgfqpoint{5.180318in}{2.911301in}}%
\pgfpathlineto{\pgfqpoint{5.194288in}{2.922165in}}%
\pgfpathlineto{\pgfqpoint{5.208276in}{2.933188in}}%
\pgfpathlineto{\pgfqpoint{5.222281in}{2.944371in}}%
\pgfpathlineto{\pgfqpoint{5.229539in}{2.949523in}}%
\pgfpathlineto{\pgfqpoint{5.236790in}{2.954590in}}%
\pgfpathlineto{\pgfqpoint{5.244033in}{2.959577in}}%
\pgfpathlineto{\pgfqpoint{5.251268in}{2.964485in}}%
\pgfpathlineto{\pgfqpoint{5.237277in}{2.953583in}}%
\pgfpathlineto{\pgfqpoint{5.223304in}{2.942841in}}%
\pgfpathlineto{\pgfqpoint{5.209349in}{2.932257in}}%
\pgfpathlineto{\pgfqpoint{5.195411in}{2.921831in}}%
\pgfpathlineto{\pgfqpoint{5.188161in}{2.916632in}}%
\pgfpathlineto{\pgfqpoint{5.180903in}{2.911362in}}%
\pgfpathlineto{\pgfqpoint{5.173638in}{2.906018in}}%
\pgfpathlineto{\pgfqpoint{5.166365in}{2.900596in}}%
\pgfpathclose%
\pgfusepath{fill}%
\end{pgfscope}%
\begin{pgfscope}%
\pgfpathrectangle{\pgfqpoint{1.254980in}{0.150000in}}{\pgfqpoint{5.490039in}{5.490039in}}%
\pgfusepath{clip}%
\pgfsetbuttcap%
\pgfsetroundjoin%
\definecolor{currentfill}{rgb}{0.283197,0.115680,0.436115}%
\pgfsetfillcolor{currentfill}%
\pgfsetfillopacity{0.700000}%
\pgfsetlinewidth{0.000000pt}%
\definecolor{currentstroke}{rgb}{0.000000,0.000000,0.000000}%
\pgfsetstrokecolor{currentstroke}%
\pgfsetdash{}{0pt}%
\pgfpathmoveto{\pgfqpoint{3.859979in}{1.848522in}}%
\pgfpathlineto{\pgfqpoint{3.873302in}{1.850128in}}%
\pgfpathlineto{\pgfqpoint{3.886634in}{1.851900in}}%
\pgfpathlineto{\pgfqpoint{3.899974in}{1.853839in}}%
\pgfpathlineto{\pgfqpoint{3.913323in}{1.855944in}}%
\pgfpathlineto{\pgfqpoint{3.921113in}{1.867131in}}%
\pgfpathlineto{\pgfqpoint{3.928898in}{1.878300in}}%
\pgfpathlineto{\pgfqpoint{3.936679in}{1.889450in}}%
\pgfpathlineto{\pgfqpoint{3.944454in}{1.900578in}}%
\pgfpathlineto{\pgfqpoint{3.931111in}{1.898227in}}%
\pgfpathlineto{\pgfqpoint{3.917778in}{1.896043in}}%
\pgfpathlineto{\pgfqpoint{3.904453in}{1.894026in}}%
\pgfpathlineto{\pgfqpoint{3.891136in}{1.892175in}}%
\pgfpathlineto{\pgfqpoint{3.883354in}{1.881281in}}%
\pgfpathlineto{\pgfqpoint{3.875567in}{1.870374in}}%
\pgfpathlineto{\pgfqpoint{3.867775in}{1.859453in}}%
\pgfpathlineto{\pgfqpoint{3.859979in}{1.848522in}}%
\pgfpathclose%
\pgfusepath{fill}%
\end{pgfscope}%
\begin{pgfscope}%
\pgfpathrectangle{\pgfqpoint{1.254980in}{0.150000in}}{\pgfqpoint{5.490039in}{5.490039in}}%
\pgfusepath{clip}%
\pgfsetbuttcap%
\pgfsetroundjoin%
\definecolor{currentfill}{rgb}{0.239374,0.735588,0.455688}%
\pgfsetfillcolor{currentfill}%
\pgfsetfillopacity{0.700000}%
\pgfsetlinewidth{0.000000pt}%
\definecolor{currentstroke}{rgb}{0.000000,0.000000,0.000000}%
\pgfsetstrokecolor{currentstroke}%
\pgfsetdash{}{0pt}%
\pgfpathmoveto{\pgfqpoint{5.873459in}{3.380426in}}%
\pgfpathlineto{\pgfqpoint{5.887816in}{3.392549in}}%
\pgfpathlineto{\pgfqpoint{5.902193in}{3.404829in}}%
\pgfpathlineto{\pgfqpoint{5.916591in}{3.417265in}}%
\pgfpathlineto{\pgfqpoint{5.931010in}{3.429858in}}%
\pgfpathlineto{\pgfqpoint{5.937808in}{3.430253in}}%
\pgfpathlineto{\pgfqpoint{5.944600in}{3.430672in}}%
\pgfpathlineto{\pgfqpoint{5.951386in}{3.431121in}}%
\pgfpathlineto{\pgfqpoint{5.958166in}{3.431606in}}%
\pgfpathlineto{\pgfqpoint{5.943782in}{3.419571in}}%
\pgfpathlineto{\pgfqpoint{5.929417in}{3.407691in}}%
\pgfpathlineto{\pgfqpoint{5.915073in}{3.395966in}}%
\pgfpathlineto{\pgfqpoint{5.900749in}{3.384397in}}%
\pgfpathlineto{\pgfqpoint{5.893935in}{3.383346in}}%
\pgfpathlineto{\pgfqpoint{5.887115in}{3.382338in}}%
\pgfpathlineto{\pgfqpoint{5.880290in}{3.381367in}}%
\pgfpathlineto{\pgfqpoint{5.873459in}{3.380426in}}%
\pgfpathclose%
\pgfusepath{fill}%
\end{pgfscope}%
\begin{pgfscope}%
\pgfpathrectangle{\pgfqpoint{1.254980in}{0.150000in}}{\pgfqpoint{5.490039in}{5.490039in}}%
\pgfusepath{clip}%
\pgfsetbuttcap%
\pgfsetroundjoin%
\definecolor{currentfill}{rgb}{0.276022,0.044167,0.370164}%
\pgfsetfillcolor{currentfill}%
\pgfsetfillopacity{0.700000}%
\pgfsetlinewidth{0.000000pt}%
\definecolor{currentstroke}{rgb}{0.000000,0.000000,0.000000}%
\pgfsetstrokecolor{currentstroke}%
\pgfsetdash{}{0pt}%
\pgfpathmoveto{\pgfqpoint{3.606412in}{1.722121in}}%
\pgfpathlineto{\pgfqpoint{3.619678in}{1.720795in}}%
\pgfpathlineto{\pgfqpoint{3.632950in}{1.719640in}}%
\pgfpathlineto{\pgfqpoint{3.646228in}{1.718657in}}%
\pgfpathlineto{\pgfqpoint{3.659512in}{1.717844in}}%
\pgfpathlineto{\pgfqpoint{3.667389in}{1.728026in}}%
\pgfpathlineto{\pgfqpoint{3.675260in}{1.738244in}}%
\pgfpathlineto{\pgfqpoint{3.683126in}{1.748494in}}%
\pgfpathlineto{\pgfqpoint{3.690987in}{1.758773in}}%
\pgfpathlineto{\pgfqpoint{3.677713in}{1.759258in}}%
\pgfpathlineto{\pgfqpoint{3.664446in}{1.759914in}}%
\pgfpathlineto{\pgfqpoint{3.651186in}{1.760741in}}%
\pgfpathlineto{\pgfqpoint{3.637931in}{1.761739in}}%
\pgfpathlineto{\pgfqpoint{3.630060in}{1.751777in}}%
\pgfpathlineto{\pgfqpoint{3.622183in}{1.741852in}}%
\pgfpathlineto{\pgfqpoint{3.614300in}{1.731965in}}%
\pgfpathlineto{\pgfqpoint{3.606412in}{1.722121in}}%
\pgfpathclose%
\pgfusepath{fill}%
\end{pgfscope}%
\begin{pgfscope}%
\pgfpathrectangle{\pgfqpoint{1.254980in}{0.150000in}}{\pgfqpoint{5.490039in}{5.490039in}}%
\pgfusepath{clip}%
\pgfsetbuttcap%
\pgfsetroundjoin%
\definecolor{currentfill}{rgb}{0.180629,0.429975,0.557282}%
\pgfsetfillcolor{currentfill}%
\pgfsetfillopacity{0.700000}%
\pgfsetlinewidth{0.000000pt}%
\definecolor{currentstroke}{rgb}{0.000000,0.000000,0.000000}%
\pgfsetstrokecolor{currentstroke}%
\pgfsetdash{}{0pt}%
\pgfpathmoveto{\pgfqpoint{2.270528in}{2.619261in}}%
\pgfpathlineto{\pgfqpoint{2.284199in}{2.596437in}}%
\pgfpathlineto{\pgfqpoint{2.297855in}{2.573918in}}%
\pgfpathlineto{\pgfqpoint{2.311497in}{2.551699in}}%
\pgfpathlineto{\pgfqpoint{2.325125in}{2.529777in}}%
\pgfpathlineto{\pgfqpoint{2.333851in}{2.527712in}}%
\pgfpathlineto{\pgfqpoint{2.342558in}{2.525923in}}%
\pgfpathlineto{\pgfqpoint{2.351246in}{2.524404in}}%
\pgfpathlineto{\pgfqpoint{2.359916in}{2.523152in}}%
\pgfpathlineto{\pgfqpoint{2.346338in}{2.544563in}}%
\pgfpathlineto{\pgfqpoint{2.332746in}{2.566271in}}%
\pgfpathlineto{\pgfqpoint{2.319142in}{2.588279in}}%
\pgfpathlineto{\pgfqpoint{2.305523in}{2.610589in}}%
\pgfpathlineto{\pgfqpoint{2.296804in}{2.612340in}}%
\pgfpathlineto{\pgfqpoint{2.288065in}{2.614366in}}%
\pgfpathlineto{\pgfqpoint{2.279307in}{2.616671in}}%
\pgfpathlineto{\pgfqpoint{2.270528in}{2.619261in}}%
\pgfpathclose%
\pgfusepath{fill}%
\end{pgfscope}%
\begin{pgfscope}%
\pgfpathrectangle{\pgfqpoint{1.254980in}{0.150000in}}{\pgfqpoint{5.490039in}{5.490039in}}%
\pgfusepath{clip}%
\pgfsetbuttcap%
\pgfsetroundjoin%
\definecolor{currentfill}{rgb}{0.282290,0.145912,0.461510}%
\pgfsetfillcolor{currentfill}%
\pgfsetfillopacity{0.700000}%
\pgfsetlinewidth{0.000000pt}%
\definecolor{currentstroke}{rgb}{0.000000,0.000000,0.000000}%
\pgfsetstrokecolor{currentstroke}%
\pgfsetdash{}{0pt}%
\pgfpathmoveto{\pgfqpoint{3.944454in}{1.900578in}}%
\pgfpathlineto{\pgfqpoint{3.957806in}{1.903094in}}%
\pgfpathlineto{\pgfqpoint{3.971167in}{1.905776in}}%
\pgfpathlineto{\pgfqpoint{3.984537in}{1.908623in}}%
\pgfpathlineto{\pgfqpoint{3.997917in}{1.911634in}}%
\pgfpathlineto{\pgfqpoint{4.005682in}{1.922967in}}%
\pgfpathlineto{\pgfqpoint{4.013442in}{1.934267in}}%
\pgfpathlineto{\pgfqpoint{4.021198in}{1.945533in}}%
\pgfpathlineto{\pgfqpoint{4.028949in}{1.956762in}}%
\pgfpathlineto{\pgfqpoint{4.015574in}{1.953532in}}%
\pgfpathlineto{\pgfqpoint{4.002209in}{1.950468in}}%
\pgfpathlineto{\pgfqpoint{3.988854in}{1.947568in}}%
\pgfpathlineto{\pgfqpoint{3.975508in}{1.944834in}}%
\pgfpathlineto{\pgfqpoint{3.967752in}{1.933812in}}%
\pgfpathlineto{\pgfqpoint{3.959991in}{1.922761in}}%
\pgfpathlineto{\pgfqpoint{3.952225in}{1.911682in}}%
\pgfpathlineto{\pgfqpoint{3.944454in}{1.900578in}}%
\pgfpathclose%
\pgfusepath{fill}%
\end{pgfscope}%
\begin{pgfscope}%
\pgfpathrectangle{\pgfqpoint{1.254980in}{0.150000in}}{\pgfqpoint{5.490039in}{5.490039in}}%
\pgfusepath{clip}%
\pgfsetbuttcap%
\pgfsetroundjoin%
\definecolor{currentfill}{rgb}{0.278791,0.062145,0.386592}%
\pgfsetfillcolor{currentfill}%
\pgfsetfillopacity{0.700000}%
\pgfsetlinewidth{0.000000pt}%
\definecolor{currentstroke}{rgb}{0.000000,0.000000,0.000000}%
\pgfsetstrokecolor{currentstroke}%
\pgfsetdash{}{0pt}%
\pgfpathmoveto{\pgfqpoint{2.968725in}{1.781173in}}%
\pgfpathlineto{\pgfqpoint{2.982011in}{1.771128in}}%
\pgfpathlineto{\pgfqpoint{2.995295in}{1.761284in}}%
\pgfpathlineto{\pgfqpoint{3.008578in}{1.751641in}}%
\pgfpathlineto{\pgfqpoint{3.021861in}{1.742197in}}%
\pgfpathlineto{\pgfqpoint{3.030069in}{1.746611in}}%
\pgfpathlineto{\pgfqpoint{3.038267in}{1.751201in}}%
\pgfpathlineto{\pgfqpoint{3.046454in}{1.755962in}}%
\pgfpathlineto{\pgfqpoint{3.054630in}{1.760890in}}%
\pgfpathlineto{\pgfqpoint{3.041377in}{1.769861in}}%
\pgfpathlineto{\pgfqpoint{3.028123in}{1.779032in}}%
\pgfpathlineto{\pgfqpoint{3.014868in}{1.788402in}}%
\pgfpathlineto{\pgfqpoint{3.001611in}{1.797974in}}%
\pgfpathlineto{\pgfqpoint{2.993407in}{1.793509in}}%
\pgfpathlineto{\pgfqpoint{2.985191in}{1.789217in}}%
\pgfpathlineto{\pgfqpoint{2.976963in}{1.785103in}}%
\pgfpathlineto{\pgfqpoint{2.968725in}{1.781173in}}%
\pgfpathclose%
\pgfusepath{fill}%
\end{pgfscope}%
\begin{pgfscope}%
\pgfpathrectangle{\pgfqpoint{1.254980in}{0.150000in}}{\pgfqpoint{5.490039in}{5.490039in}}%
\pgfusepath{clip}%
\pgfsetbuttcap%
\pgfsetroundjoin%
\definecolor{currentfill}{rgb}{0.268510,0.009605,0.335427}%
\pgfsetfillcolor{currentfill}%
\pgfsetfillopacity{0.700000}%
\pgfsetlinewidth{0.000000pt}%
\definecolor{currentstroke}{rgb}{0.000000,0.000000,0.000000}%
\pgfsetstrokecolor{currentstroke}%
\pgfsetdash{}{0pt}%
\pgfpathmoveto{\pgfqpoint{3.298941in}{1.671904in}}%
\pgfpathlineto{\pgfqpoint{3.312187in}{1.666601in}}%
\pgfpathlineto{\pgfqpoint{3.325435in}{1.661480in}}%
\pgfpathlineto{\pgfqpoint{3.338686in}{1.656540in}}%
\pgfpathlineto{\pgfqpoint{3.351940in}{1.651780in}}%
\pgfpathlineto{\pgfqpoint{3.359955in}{1.659590in}}%
\pgfpathlineto{\pgfqpoint{3.367961in}{1.667506in}}%
\pgfpathlineto{\pgfqpoint{3.375961in}{1.675523in}}%
\pgfpathlineto{\pgfqpoint{3.383952in}{1.683637in}}%
\pgfpathlineto{\pgfqpoint{3.370717in}{1.687986in}}%
\pgfpathlineto{\pgfqpoint{3.357485in}{1.692515in}}%
\pgfpathlineto{\pgfqpoint{3.344255in}{1.697224in}}%
\pgfpathlineto{\pgfqpoint{3.331029in}{1.702116in}}%
\pgfpathlineto{\pgfqpoint{3.323019in}{1.694402in}}%
\pgfpathlineto{\pgfqpoint{3.315001in}{1.686793in}}%
\pgfpathlineto{\pgfqpoint{3.306975in}{1.679293in}}%
\pgfpathlineto{\pgfqpoint{3.298941in}{1.671904in}}%
\pgfpathclose%
\pgfusepath{fill}%
\end{pgfscope}%
\begin{pgfscope}%
\pgfpathrectangle{\pgfqpoint{1.254980in}{0.150000in}}{\pgfqpoint{5.490039in}{5.490039in}}%
\pgfusepath{clip}%
\pgfsetbuttcap%
\pgfsetroundjoin%
\definecolor{currentfill}{rgb}{0.272594,0.025563,0.353093}%
\pgfsetfillcolor{currentfill}%
\pgfsetfillopacity{0.700000}%
\pgfsetlinewidth{0.000000pt}%
\definecolor{currentstroke}{rgb}{0.000000,0.000000,0.000000}%
\pgfsetstrokecolor{currentstroke}%
\pgfsetdash{}{0pt}%
\pgfpathmoveto{\pgfqpoint{3.521739in}{1.691681in}}%
\pgfpathlineto{\pgfqpoint{3.534997in}{1.689307in}}%
\pgfpathlineto{\pgfqpoint{3.548259in}{1.687108in}}%
\pgfpathlineto{\pgfqpoint{3.561526in}{1.685081in}}%
\pgfpathlineto{\pgfqpoint{3.574799in}{1.683227in}}%
\pgfpathlineto{\pgfqpoint{3.582711in}{1.692872in}}%
\pgfpathlineto{\pgfqpoint{3.590617in}{1.702571in}}%
\pgfpathlineto{\pgfqpoint{3.598518in}{1.712322in}}%
\pgfpathlineto{\pgfqpoint{3.606412in}{1.722121in}}%
\pgfpathlineto{\pgfqpoint{3.593152in}{1.723619in}}%
\pgfpathlineto{\pgfqpoint{3.579897in}{1.725290in}}%
\pgfpathlineto{\pgfqpoint{3.566648in}{1.727134in}}%
\pgfpathlineto{\pgfqpoint{3.553405in}{1.729152in}}%
\pgfpathlineto{\pgfqpoint{3.545498in}{1.719698in}}%
\pgfpathlineto{\pgfqpoint{3.537584in}{1.710299in}}%
\pgfpathlineto{\pgfqpoint{3.529665in}{1.700959in}}%
\pgfpathlineto{\pgfqpoint{3.521739in}{1.691681in}}%
\pgfpathclose%
\pgfusepath{fill}%
\end{pgfscope}%
\begin{pgfscope}%
\pgfpathrectangle{\pgfqpoint{1.254980in}{0.150000in}}{\pgfqpoint{5.490039in}{5.490039in}}%
\pgfusepath{clip}%
\pgfsetbuttcap%
\pgfsetroundjoin%
\definecolor{currentfill}{rgb}{0.121831,0.589055,0.545623}%
\pgfsetfillcolor{currentfill}%
\pgfsetfillopacity{0.700000}%
\pgfsetlinewidth{0.000000pt}%
\definecolor{currentstroke}{rgb}{0.000000,0.000000,0.000000}%
\pgfsetstrokecolor{currentstroke}%
\pgfsetdash{}{0pt}%
\pgfpathmoveto{\pgfqpoint{5.251268in}{2.964485in}}%
\pgfpathlineto{\pgfqpoint{5.265276in}{2.975545in}}%
\pgfpathlineto{\pgfqpoint{5.279302in}{2.986763in}}%
\pgfpathlineto{\pgfqpoint{5.293346in}{2.998141in}}%
\pgfpathlineto{\pgfqpoint{5.307409in}{3.009678in}}%
\pgfpathlineto{\pgfqpoint{5.314620in}{3.014211in}}%
\pgfpathlineto{\pgfqpoint{5.321823in}{3.018665in}}%
\pgfpathlineto{\pgfqpoint{5.329018in}{3.023045in}}%
\pgfpathlineto{\pgfqpoint{5.336205in}{3.027353in}}%
\pgfpathlineto{\pgfqpoint{5.322159in}{3.016128in}}%
\pgfpathlineto{\pgfqpoint{5.308131in}{3.005062in}}%
\pgfpathlineto{\pgfqpoint{5.294122in}{2.994155in}}%
\pgfpathlineto{\pgfqpoint{5.280130in}{2.983405in}}%
\pgfpathlineto{\pgfqpoint{5.272926in}{2.978775in}}%
\pgfpathlineto{\pgfqpoint{5.265714in}{2.974080in}}%
\pgfpathlineto{\pgfqpoint{5.258495in}{2.969318in}}%
\pgfpathlineto{\pgfqpoint{5.251268in}{2.964485in}}%
\pgfpathclose%
\pgfusepath{fill}%
\end{pgfscope}%
\begin{pgfscope}%
\pgfpathrectangle{\pgfqpoint{1.254980in}{0.150000in}}{\pgfqpoint{5.490039in}{5.490039in}}%
\pgfusepath{clip}%
\pgfsetbuttcap%
\pgfsetroundjoin%
\definecolor{currentfill}{rgb}{0.271305,0.019942,0.347269}%
\pgfsetfillcolor{currentfill}%
\pgfsetfillopacity{0.700000}%
\pgfsetlinewidth{0.000000pt}%
\definecolor{currentstroke}{rgb}{0.000000,0.000000,0.000000}%
\pgfsetstrokecolor{currentstroke}%
\pgfsetdash{}{0pt}%
\pgfpathmoveto{\pgfqpoint{3.160650in}{1.696150in}}%
\pgfpathlineto{\pgfqpoint{3.173905in}{1.688922in}}%
\pgfpathlineto{\pgfqpoint{3.187160in}{1.681882in}}%
\pgfpathlineto{\pgfqpoint{3.200417in}{1.675030in}}%
\pgfpathlineto{\pgfqpoint{3.213674in}{1.668365in}}%
\pgfpathlineto{\pgfqpoint{3.221764in}{1.674812in}}%
\pgfpathlineto{\pgfqpoint{3.229845in}{1.681395in}}%
\pgfpathlineto{\pgfqpoint{3.237917in}{1.688109in}}%
\pgfpathlineto{\pgfqpoint{3.245981in}{1.694951in}}%
\pgfpathlineto{\pgfqpoint{3.232746in}{1.701176in}}%
\pgfpathlineto{\pgfqpoint{3.219512in}{1.707587in}}%
\pgfpathlineto{\pgfqpoint{3.206280in}{1.714186in}}%
\pgfpathlineto{\pgfqpoint{3.193050in}{1.720973in}}%
\pgfpathlineto{\pgfqpoint{3.184964in}{1.714561in}}%
\pgfpathlineto{\pgfqpoint{3.176869in}{1.708284in}}%
\pgfpathlineto{\pgfqpoint{3.168764in}{1.702145in}}%
\pgfpathlineto{\pgfqpoint{3.160650in}{1.696150in}}%
\pgfpathclose%
\pgfusepath{fill}%
\end{pgfscope}%
\begin{pgfscope}%
\pgfpathrectangle{\pgfqpoint{1.254980in}{0.150000in}}{\pgfqpoint{5.490039in}{5.490039in}}%
\pgfusepath{clip}%
\pgfsetbuttcap%
\pgfsetroundjoin%
\definecolor{currentfill}{rgb}{0.244972,0.287675,0.537260}%
\pgfsetfillcolor{currentfill}%
\pgfsetfillopacity{0.700000}%
\pgfsetlinewidth{0.000000pt}%
\definecolor{currentstroke}{rgb}{0.000000,0.000000,0.000000}%
\pgfsetstrokecolor{currentstroke}%
\pgfsetdash{}{0pt}%
\pgfpathmoveto{\pgfqpoint{4.313402in}{2.188941in}}%
\pgfpathlineto{\pgfqpoint{4.326903in}{2.194888in}}%
\pgfpathlineto{\pgfqpoint{4.340415in}{2.200997in}}%
\pgfpathlineto{\pgfqpoint{4.353940in}{2.207269in}}%
\pgfpathlineto{\pgfqpoint{4.367478in}{2.213702in}}%
\pgfpathlineto{\pgfqpoint{4.375128in}{2.224460in}}%
\pgfpathlineto{\pgfqpoint{4.382773in}{2.235132in}}%
\pgfpathlineto{\pgfqpoint{4.390413in}{2.245719in}}%
\pgfpathlineto{\pgfqpoint{4.398047in}{2.256221in}}%
\pgfpathlineto{\pgfqpoint{4.384514in}{2.249711in}}%
\pgfpathlineto{\pgfqpoint{4.370992in}{2.243363in}}%
\pgfpathlineto{\pgfqpoint{4.357484in}{2.237177in}}%
\pgfpathlineto{\pgfqpoint{4.343987in}{2.231153in}}%
\pgfpathlineto{\pgfqpoint{4.336349in}{2.220717in}}%
\pgfpathlineto{\pgfqpoint{4.328705in}{2.210203in}}%
\pgfpathlineto{\pgfqpoint{4.321057in}{2.199611in}}%
\pgfpathlineto{\pgfqpoint{4.313402in}{2.188941in}}%
\pgfpathclose%
\pgfusepath{fill}%
\end{pgfscope}%
\begin{pgfscope}%
\pgfpathrectangle{\pgfqpoint{1.254980in}{0.150000in}}{\pgfqpoint{5.490039in}{5.490039in}}%
\pgfusepath{clip}%
\pgfsetbuttcap%
\pgfsetroundjoin%
\definecolor{currentfill}{rgb}{0.159194,0.482237,0.558073}%
\pgfsetfillcolor{currentfill}%
\pgfsetfillopacity{0.700000}%
\pgfsetlinewidth{0.000000pt}%
\definecolor{currentstroke}{rgb}{0.000000,0.000000,0.000000}%
\pgfsetstrokecolor{currentstroke}%
\pgfsetdash{}{0pt}%
\pgfpathmoveto{\pgfqpoint{4.882293in}{2.675096in}}%
\pgfpathlineto{\pgfqpoint{4.896089in}{2.684705in}}%
\pgfpathlineto{\pgfqpoint{4.909902in}{2.694474in}}%
\pgfpathlineto{\pgfqpoint{4.923731in}{2.704403in}}%
\pgfpathlineto{\pgfqpoint{4.937576in}{2.714493in}}%
\pgfpathlineto{\pgfqpoint{4.944992in}{2.721832in}}%
\pgfpathlineto{\pgfqpoint{4.952401in}{2.729066in}}%
\pgfpathlineto{\pgfqpoint{4.959802in}{2.736199in}}%
\pgfpathlineto{\pgfqpoint{4.967196in}{2.743231in}}%
\pgfpathlineto{\pgfqpoint{4.953360in}{2.733302in}}%
\pgfpathlineto{\pgfqpoint{4.939541in}{2.723532in}}%
\pgfpathlineto{\pgfqpoint{4.925737in}{2.713922in}}%
\pgfpathlineto{\pgfqpoint{4.911950in}{2.704472in}}%
\pgfpathlineto{\pgfqpoint{4.904546in}{2.697269in}}%
\pgfpathlineto{\pgfqpoint{4.897135in}{2.689974in}}%
\pgfpathlineto{\pgfqpoint{4.889718in}{2.682583in}}%
\pgfpathlineto{\pgfqpoint{4.882293in}{2.675096in}}%
\pgfpathclose%
\pgfusepath{fill}%
\end{pgfscope}%
\begin{pgfscope}%
\pgfpathrectangle{\pgfqpoint{1.254980in}{0.150000in}}{\pgfqpoint{5.490039in}{5.490039in}}%
\pgfusepath{clip}%
\pgfsetbuttcap%
\pgfsetroundjoin%
\definecolor{currentfill}{rgb}{0.197636,0.391528,0.554969}%
\pgfsetfillcolor{currentfill}%
\pgfsetfillopacity{0.700000}%
\pgfsetlinewidth{0.000000pt}%
\definecolor{currentstroke}{rgb}{0.000000,0.000000,0.000000}%
\pgfsetstrokecolor{currentstroke}%
\pgfsetdash{}{0pt}%
\pgfpathmoveto{\pgfqpoint{4.597867in}{2.433910in}}%
\pgfpathlineto{\pgfqpoint{4.611509in}{2.441931in}}%
\pgfpathlineto{\pgfqpoint{4.625165in}{2.450112in}}%
\pgfpathlineto{\pgfqpoint{4.638836in}{2.458454in}}%
\pgfpathlineto{\pgfqpoint{4.652521in}{2.466958in}}%
\pgfpathlineto{\pgfqpoint{4.660067in}{2.476266in}}%
\pgfpathlineto{\pgfqpoint{4.667606in}{2.485471in}}%
\pgfpathlineto{\pgfqpoint{4.675139in}{2.494572in}}%
\pgfpathlineto{\pgfqpoint{4.682666in}{2.503570in}}%
\pgfpathlineto{\pgfqpoint{4.668986in}{2.495107in}}%
\pgfpathlineto{\pgfqpoint{4.655320in}{2.486805in}}%
\pgfpathlineto{\pgfqpoint{4.641669in}{2.478664in}}%
\pgfpathlineto{\pgfqpoint{4.628033in}{2.470683in}}%
\pgfpathlineto{\pgfqpoint{4.620501in}{2.461633in}}%
\pgfpathlineto{\pgfqpoint{4.612962in}{2.452489in}}%
\pgfpathlineto{\pgfqpoint{4.605417in}{2.443248in}}%
\pgfpathlineto{\pgfqpoint{4.597867in}{2.433910in}}%
\pgfpathclose%
\pgfusepath{fill}%
\end{pgfscope}%
\begin{pgfscope}%
\pgfpathrectangle{\pgfqpoint{1.254980in}{0.150000in}}{\pgfqpoint{5.490039in}{5.490039in}}%
\pgfusepath{clip}%
\pgfsetbuttcap%
\pgfsetroundjoin%
\definecolor{currentfill}{rgb}{0.278826,0.175490,0.483397}%
\pgfsetfillcolor{currentfill}%
\pgfsetfillopacity{0.700000}%
\pgfsetlinewidth{0.000000pt}%
\definecolor{currentstroke}{rgb}{0.000000,0.000000,0.000000}%
\pgfsetstrokecolor{currentstroke}%
\pgfsetdash{}{0pt}%
\pgfpathmoveto{\pgfqpoint{4.028949in}{1.956762in}}%
\pgfpathlineto{\pgfqpoint{4.042333in}{1.960157in}}%
\pgfpathlineto{\pgfqpoint{4.055727in}{1.963716in}}%
\pgfpathlineto{\pgfqpoint{4.069132in}{1.967439in}}%
\pgfpathlineto{\pgfqpoint{4.082547in}{1.971326in}}%
\pgfpathlineto{\pgfqpoint{4.090288in}{1.982718in}}%
\pgfpathlineto{\pgfqpoint{4.098024in}{1.994063in}}%
\pgfpathlineto{\pgfqpoint{4.105756in}{2.005360in}}%
\pgfpathlineto{\pgfqpoint{4.113483in}{2.016607in}}%
\pgfpathlineto{\pgfqpoint{4.100072in}{2.012529in}}%
\pgfpathlineto{\pgfqpoint{4.086673in}{2.008616in}}%
\pgfpathlineto{\pgfqpoint{4.073283in}{2.004867in}}%
\pgfpathlineto{\pgfqpoint{4.059904in}{2.001283in}}%
\pgfpathlineto{\pgfqpoint{4.052172in}{1.990215in}}%
\pgfpathlineto{\pgfqpoint{4.044436in}{1.979105in}}%
\pgfpathlineto{\pgfqpoint{4.036695in}{1.967953in}}%
\pgfpathlineto{\pgfqpoint{4.028949in}{1.956762in}}%
\pgfpathclose%
\pgfusepath{fill}%
\end{pgfscope}%
\begin{pgfscope}%
\pgfpathrectangle{\pgfqpoint{1.254980in}{0.150000in}}{\pgfqpoint{5.490039in}{5.490039in}}%
\pgfusepath{clip}%
\pgfsetbuttcap%
\pgfsetroundjoin%
\definecolor{currentfill}{rgb}{0.119483,0.614817,0.537692}%
\pgfsetfillcolor{currentfill}%
\pgfsetfillopacity{0.700000}%
\pgfsetlinewidth{0.000000pt}%
\definecolor{currentstroke}{rgb}{0.000000,0.000000,0.000000}%
\pgfsetstrokecolor{currentstroke}%
\pgfsetdash{}{0pt}%
\pgfpathmoveto{\pgfqpoint{5.336205in}{3.027353in}}%
\pgfpathlineto{\pgfqpoint{5.350269in}{3.038736in}}%
\pgfpathlineto{\pgfqpoint{5.364352in}{3.050278in}}%
\pgfpathlineto{\pgfqpoint{5.378453in}{3.061979in}}%
\pgfpathlineto{\pgfqpoint{5.392572in}{3.073838in}}%
\pgfpathlineto{\pgfqpoint{5.399734in}{3.077747in}}%
\pgfpathlineto{\pgfqpoint{5.406887in}{3.081585in}}%
\pgfpathlineto{\pgfqpoint{5.414032in}{3.085356in}}%
\pgfpathlineto{\pgfqpoint{5.421170in}{3.089064in}}%
\pgfpathlineto{\pgfqpoint{5.407068in}{3.077548in}}%
\pgfpathlineto{\pgfqpoint{5.392986in}{3.066190in}}%
\pgfpathlineto{\pgfqpoint{5.378922in}{3.054991in}}%
\pgfpathlineto{\pgfqpoint{5.364876in}{3.043949in}}%
\pgfpathlineto{\pgfqpoint{5.357720in}{3.039888in}}%
\pgfpathlineto{\pgfqpoint{5.350556in}{3.035771in}}%
\pgfpathlineto{\pgfqpoint{5.343384in}{3.031594in}}%
\pgfpathlineto{\pgfqpoint{5.336205in}{3.027353in}}%
\pgfpathclose%
\pgfusepath{fill}%
\end{pgfscope}%
\begin{pgfscope}%
\pgfpathrectangle{\pgfqpoint{1.254980in}{0.150000in}}{\pgfqpoint{5.490039in}{5.490039in}}%
\pgfusepath{clip}%
\pgfsetbuttcap%
\pgfsetroundjoin%
\definecolor{currentfill}{rgb}{0.271828,0.209303,0.504434}%
\pgfsetfillcolor{currentfill}%
\pgfsetfillopacity{0.700000}%
\pgfsetlinewidth{0.000000pt}%
\definecolor{currentstroke}{rgb}{0.000000,0.000000,0.000000}%
\pgfsetstrokecolor{currentstroke}%
\pgfsetdash{}{0pt}%
\pgfpathmoveto{\pgfqpoint{2.614934in}{2.084613in}}%
\pgfpathlineto{\pgfqpoint{2.628358in}{2.068800in}}%
\pgfpathlineto{\pgfqpoint{2.641775in}{2.053225in}}%
\pgfpathlineto{\pgfqpoint{2.655185in}{2.037885in}}%
\pgfpathlineto{\pgfqpoint{2.668589in}{2.022780in}}%
\pgfpathlineto{\pgfqpoint{2.677062in}{2.023333in}}%
\pgfpathlineto{\pgfqpoint{2.685519in}{2.024126in}}%
\pgfpathlineto{\pgfqpoint{2.693961in}{2.025154in}}%
\pgfpathlineto{\pgfqpoint{2.702388in}{2.026412in}}%
\pgfpathlineto{\pgfqpoint{2.689024in}{2.041003in}}%
\pgfpathlineto{\pgfqpoint{2.675655in}{2.055827in}}%
\pgfpathlineto{\pgfqpoint{2.662279in}{2.070886in}}%
\pgfpathlineto{\pgfqpoint{2.648897in}{2.086182in}}%
\pgfpathlineto{\pgfqpoint{2.640430in}{2.085428in}}%
\pgfpathlineto{\pgfqpoint{2.631947in}{2.084912in}}%
\pgfpathlineto{\pgfqpoint{2.623449in}{2.084638in}}%
\pgfpathlineto{\pgfqpoint{2.614934in}{2.084613in}}%
\pgfpathclose%
\pgfusepath{fill}%
\end{pgfscope}%
\begin{pgfscope}%
\pgfpathrectangle{\pgfqpoint{1.254980in}{0.150000in}}{\pgfqpoint{5.490039in}{5.490039in}}%
\pgfusepath{clip}%
\pgfsetbuttcap%
\pgfsetroundjoin%
\definecolor{currentfill}{rgb}{0.269944,0.014625,0.341379}%
\pgfsetfillcolor{currentfill}%
\pgfsetfillopacity{0.700000}%
\pgfsetlinewidth{0.000000pt}%
\definecolor{currentstroke}{rgb}{0.000000,0.000000,0.000000}%
\pgfsetstrokecolor{currentstroke}%
\pgfsetdash{}{0pt}%
\pgfpathmoveto{\pgfqpoint{3.436930in}{1.668032in}}%
\pgfpathlineto{\pgfqpoint{3.450184in}{1.664575in}}%
\pgfpathlineto{\pgfqpoint{3.463443in}{1.661293in}}%
\pgfpathlineto{\pgfqpoint{3.476705in}{1.658187in}}%
\pgfpathlineto{\pgfqpoint{3.489972in}{1.655256in}}%
\pgfpathlineto{\pgfqpoint{3.497924in}{1.664252in}}%
\pgfpathlineto{\pgfqpoint{3.505869in}{1.673324in}}%
\pgfpathlineto{\pgfqpoint{3.513807in}{1.682468in}}%
\pgfpathlineto{\pgfqpoint{3.521739in}{1.691681in}}%
\pgfpathlineto{\pgfqpoint{3.508487in}{1.694229in}}%
\pgfpathlineto{\pgfqpoint{3.495240in}{1.696951in}}%
\pgfpathlineto{\pgfqpoint{3.481997in}{1.699849in}}%
\pgfpathlineto{\pgfqpoint{3.468759in}{1.702924in}}%
\pgfpathlineto{\pgfqpoint{3.460812in}{1.694084in}}%
\pgfpathlineto{\pgfqpoint{3.452858in}{1.685319in}}%
\pgfpathlineto{\pgfqpoint{3.444897in}{1.676634in}}%
\pgfpathlineto{\pgfqpoint{3.436930in}{1.668032in}}%
\pgfpathclose%
\pgfusepath{fill}%
\end{pgfscope}%
\begin{pgfscope}%
\pgfpathrectangle{\pgfqpoint{1.254980in}{0.150000in}}{\pgfqpoint{5.490039in}{5.490039in}}%
\pgfusepath{clip}%
\pgfsetbuttcap%
\pgfsetroundjoin%
\definecolor{currentfill}{rgb}{0.262138,0.242286,0.520837}%
\pgfsetfillcolor{currentfill}%
\pgfsetfillopacity{0.700000}%
\pgfsetlinewidth{0.000000pt}%
\definecolor{currentstroke}{rgb}{0.000000,0.000000,0.000000}%
\pgfsetstrokecolor{currentstroke}%
\pgfsetdash{}{0pt}%
\pgfpathmoveto{\pgfqpoint{2.561164in}{2.150284in}}%
\pgfpathlineto{\pgfqpoint{2.574618in}{2.133500in}}%
\pgfpathlineto{\pgfqpoint{2.588064in}{2.116962in}}%
\pgfpathlineto{\pgfqpoint{2.601503in}{2.100667in}}%
\pgfpathlineto{\pgfqpoint{2.614934in}{2.084613in}}%
\pgfpathlineto{\pgfqpoint{2.623449in}{2.084638in}}%
\pgfpathlineto{\pgfqpoint{2.631947in}{2.084912in}}%
\pgfpathlineto{\pgfqpoint{2.640430in}{2.085428in}}%
\pgfpathlineto{\pgfqpoint{2.648897in}{2.086182in}}%
\pgfpathlineto{\pgfqpoint{2.635508in}{2.101718in}}%
\pgfpathlineto{\pgfqpoint{2.622113in}{2.117494in}}%
\pgfpathlineto{\pgfqpoint{2.608710in}{2.133513in}}%
\pgfpathlineto{\pgfqpoint{2.595299in}{2.149777in}}%
\pgfpathlineto{\pgfqpoint{2.586790in}{2.149530in}}%
\pgfpathlineto{\pgfqpoint{2.578265in}{2.149528in}}%
\pgfpathlineto{\pgfqpoint{2.569723in}{2.149778in}}%
\pgfpathlineto{\pgfqpoint{2.561164in}{2.150284in}}%
\pgfpathclose%
\pgfusepath{fill}%
\end{pgfscope}%
\begin{pgfscope}%
\pgfpathrectangle{\pgfqpoint{1.254980in}{0.150000in}}{\pgfqpoint{5.490039in}{5.490039in}}%
\pgfusepath{clip}%
\pgfsetbuttcap%
\pgfsetroundjoin%
\definecolor{currentfill}{rgb}{0.277134,0.185228,0.489898}%
\pgfsetfillcolor{currentfill}%
\pgfsetfillopacity{0.700000}%
\pgfsetlinewidth{0.000000pt}%
\definecolor{currentstroke}{rgb}{0.000000,0.000000,0.000000}%
\pgfsetstrokecolor{currentstroke}%
\pgfsetdash{}{0pt}%
\pgfpathmoveto{\pgfqpoint{2.668589in}{2.022780in}}%
\pgfpathlineto{\pgfqpoint{2.681987in}{2.007907in}}%
\pgfpathlineto{\pgfqpoint{2.695378in}{1.993266in}}%
\pgfpathlineto{\pgfqpoint{2.708764in}{1.978853in}}%
\pgfpathlineto{\pgfqpoint{2.722145in}{1.964667in}}%
\pgfpathlineto{\pgfqpoint{2.730577in}{1.965745in}}%
\pgfpathlineto{\pgfqpoint{2.738994in}{1.967054in}}%
\pgfpathlineto{\pgfqpoint{2.747397in}{1.968592in}}%
\pgfpathlineto{\pgfqpoint{2.755786in}{1.970352in}}%
\pgfpathlineto{\pgfqpoint{2.742444in}{1.984025in}}%
\pgfpathlineto{\pgfqpoint{2.729097in}{1.997925in}}%
\pgfpathlineto{\pgfqpoint{2.715745in}{2.012054in}}%
\pgfpathlineto{\pgfqpoint{2.702388in}{2.026412in}}%
\pgfpathlineto{\pgfqpoint{2.693961in}{2.025154in}}%
\pgfpathlineto{\pgfqpoint{2.685519in}{2.024126in}}%
\pgfpathlineto{\pgfqpoint{2.677062in}{2.023333in}}%
\pgfpathlineto{\pgfqpoint{2.668589in}{2.022780in}}%
\pgfpathclose%
\pgfusepath{fill}%
\end{pgfscope}%
\begin{pgfscope}%
\pgfpathrectangle{\pgfqpoint{1.254980in}{0.150000in}}{\pgfqpoint{5.490039in}{5.490039in}}%
\pgfusepath{clip}%
\pgfsetbuttcap%
\pgfsetroundjoin%
\definecolor{currentfill}{rgb}{0.165117,0.467423,0.558141}%
\pgfsetfillcolor{currentfill}%
\pgfsetfillopacity{0.700000}%
\pgfsetlinewidth{0.000000pt}%
\definecolor{currentstroke}{rgb}{0.000000,0.000000,0.000000}%
\pgfsetstrokecolor{currentstroke}%
\pgfsetdash{}{0pt}%
\pgfpathmoveto{\pgfqpoint{2.215698in}{2.713653in}}%
\pgfpathlineto{\pgfqpoint{2.229428in}{2.689584in}}%
\pgfpathlineto{\pgfqpoint{2.243144in}{2.665831in}}%
\pgfpathlineto{\pgfqpoint{2.256843in}{2.642391in}}%
\pgfpathlineto{\pgfqpoint{2.270528in}{2.619261in}}%
\pgfpathlineto{\pgfqpoint{2.279307in}{2.616671in}}%
\pgfpathlineto{\pgfqpoint{2.288065in}{2.614366in}}%
\pgfpathlineto{\pgfqpoint{2.296804in}{2.612340in}}%
\pgfpathlineto{\pgfqpoint{2.305523in}{2.610589in}}%
\pgfpathlineto{\pgfqpoint{2.291891in}{2.633204in}}%
\pgfpathlineto{\pgfqpoint{2.278244in}{2.656128in}}%
\pgfpathlineto{\pgfqpoint{2.264582in}{2.679363in}}%
\pgfpathlineto{\pgfqpoint{2.250905in}{2.702913in}}%
\pgfpathlineto{\pgfqpoint{2.242133in}{2.705169in}}%
\pgfpathlineto{\pgfqpoint{2.233342in}{2.707708in}}%
\pgfpathlineto{\pgfqpoint{2.224530in}{2.710534in}}%
\pgfpathlineto{\pgfqpoint{2.215698in}{2.713653in}}%
\pgfpathclose%
\pgfusepath{fill}%
\end{pgfscope}%
\begin{pgfscope}%
\pgfpathrectangle{\pgfqpoint{1.254980in}{0.150000in}}{\pgfqpoint{5.490039in}{5.490039in}}%
\pgfusepath{clip}%
\pgfsetbuttcap%
\pgfsetroundjoin%
\definecolor{currentfill}{rgb}{0.276022,0.044167,0.370164}%
\pgfsetfillcolor{currentfill}%
\pgfsetfillopacity{0.700000}%
\pgfsetlinewidth{0.000000pt}%
\definecolor{currentstroke}{rgb}{0.000000,0.000000,0.000000}%
\pgfsetstrokecolor{currentstroke}%
\pgfsetdash{}{0pt}%
\pgfpathmoveto{\pgfqpoint{3.021861in}{1.742197in}}%
\pgfpathlineto{\pgfqpoint{3.035142in}{1.732951in}}%
\pgfpathlineto{\pgfqpoint{3.048422in}{1.723903in}}%
\pgfpathlineto{\pgfqpoint{3.061702in}{1.715050in}}%
\pgfpathlineto{\pgfqpoint{3.074981in}{1.706392in}}%
\pgfpathlineto{\pgfqpoint{3.083161in}{1.711289in}}%
\pgfpathlineto{\pgfqpoint{3.091331in}{1.716354in}}%
\pgfpathlineto{\pgfqpoint{3.099490in}{1.721583in}}%
\pgfpathlineto{\pgfqpoint{3.107639in}{1.726971in}}%
\pgfpathlineto{\pgfqpoint{3.094387in}{1.735158in}}%
\pgfpathlineto{\pgfqpoint{3.081135in}{1.743539in}}%
\pgfpathlineto{\pgfqpoint{3.067883in}{1.752116in}}%
\pgfpathlineto{\pgfqpoint{3.054630in}{1.760890in}}%
\pgfpathlineto{\pgfqpoint{3.046454in}{1.755962in}}%
\pgfpathlineto{\pgfqpoint{3.038267in}{1.751201in}}%
\pgfpathlineto{\pgfqpoint{3.030069in}{1.746611in}}%
\pgfpathlineto{\pgfqpoint{3.021861in}{1.742197in}}%
\pgfpathclose%
\pgfusepath{fill}%
\end{pgfscope}%
\begin{pgfscope}%
\pgfpathrectangle{\pgfqpoint{1.254980in}{0.150000in}}{\pgfqpoint{5.490039in}{5.490039in}}%
\pgfusepath{clip}%
\pgfsetbuttcap%
\pgfsetroundjoin%
\definecolor{currentfill}{rgb}{0.250425,0.274290,0.533103}%
\pgfsetfillcolor{currentfill}%
\pgfsetfillopacity{0.700000}%
\pgfsetlinewidth{0.000000pt}%
\definecolor{currentstroke}{rgb}{0.000000,0.000000,0.000000}%
\pgfsetstrokecolor{currentstroke}%
\pgfsetdash{}{0pt}%
\pgfpathmoveto{\pgfqpoint{2.507264in}{2.219917in}}%
\pgfpathlineto{\pgfqpoint{2.520752in}{2.202130in}}%
\pgfpathlineto{\pgfqpoint{2.534231in}{2.184597in}}%
\pgfpathlineto{\pgfqpoint{2.547702in}{2.167316in}}%
\pgfpathlineto{\pgfqpoint{2.561164in}{2.150284in}}%
\pgfpathlineto{\pgfqpoint{2.569723in}{2.149778in}}%
\pgfpathlineto{\pgfqpoint{2.578265in}{2.149528in}}%
\pgfpathlineto{\pgfqpoint{2.586790in}{2.149530in}}%
\pgfpathlineto{\pgfqpoint{2.595299in}{2.149777in}}%
\pgfpathlineto{\pgfqpoint{2.581881in}{2.166287in}}%
\pgfpathlineto{\pgfqpoint{2.568455in}{2.183046in}}%
\pgfpathlineto{\pgfqpoint{2.555021in}{2.200056in}}%
\pgfpathlineto{\pgfqpoint{2.541578in}{2.217319in}}%
\pgfpathlineto{\pgfqpoint{2.533025in}{2.217583in}}%
\pgfpathlineto{\pgfqpoint{2.524455in}{2.218100in}}%
\pgfpathlineto{\pgfqpoint{2.515869in}{2.218876in}}%
\pgfpathlineto{\pgfqpoint{2.507264in}{2.219917in}}%
\pgfpathclose%
\pgfusepath{fill}%
\end{pgfscope}%
\begin{pgfscope}%
\pgfpathrectangle{\pgfqpoint{1.254980in}{0.150000in}}{\pgfqpoint{5.490039in}{5.490039in}}%
\pgfusepath{clip}%
\pgfsetbuttcap%
\pgfsetroundjoin%
\definecolor{currentfill}{rgb}{0.231674,0.318106,0.544834}%
\pgfsetfillcolor{currentfill}%
\pgfsetfillopacity{0.700000}%
\pgfsetlinewidth{0.000000pt}%
\definecolor{currentstroke}{rgb}{0.000000,0.000000,0.000000}%
\pgfsetstrokecolor{currentstroke}%
\pgfsetdash{}{0pt}%
\pgfpathmoveto{\pgfqpoint{4.398047in}{2.256221in}}%
\pgfpathlineto{\pgfqpoint{4.411594in}{2.262893in}}%
\pgfpathlineto{\pgfqpoint{4.425154in}{2.269727in}}%
\pgfpathlineto{\pgfqpoint{4.438726in}{2.276723in}}%
\pgfpathlineto{\pgfqpoint{4.452312in}{2.283881in}}%
\pgfpathlineto{\pgfqpoint{4.459937in}{2.294355in}}%
\pgfpathlineto{\pgfqpoint{4.467557in}{2.304737in}}%
\pgfpathlineto{\pgfqpoint{4.475171in}{2.315025in}}%
\pgfpathlineto{\pgfqpoint{4.482779in}{2.325220in}}%
\pgfpathlineto{\pgfqpoint{4.469197in}{2.318014in}}%
\pgfpathlineto{\pgfqpoint{4.455628in}{2.310971in}}%
\pgfpathlineto{\pgfqpoint{4.442073in}{2.304089in}}%
\pgfpathlineto{\pgfqpoint{4.428530in}{2.297369in}}%
\pgfpathlineto{\pgfqpoint{4.420918in}{2.287211in}}%
\pgfpathlineto{\pgfqpoint{4.413300in}{2.276967in}}%
\pgfpathlineto{\pgfqpoint{4.405676in}{2.266637in}}%
\pgfpathlineto{\pgfqpoint{4.398047in}{2.256221in}}%
\pgfpathclose%
\pgfusepath{fill}%
\end{pgfscope}%
\begin{pgfscope}%
\pgfpathrectangle{\pgfqpoint{1.254980in}{0.150000in}}{\pgfqpoint{5.490039in}{5.490039in}}%
\pgfusepath{clip}%
\pgfsetbuttcap%
\pgfsetroundjoin%
\definecolor{currentfill}{rgb}{0.149039,0.508051,0.557250}%
\pgfsetfillcolor{currentfill}%
\pgfsetfillopacity{0.700000}%
\pgfsetlinewidth{0.000000pt}%
\definecolor{currentstroke}{rgb}{0.000000,0.000000,0.000000}%
\pgfsetstrokecolor{currentstroke}%
\pgfsetdash{}{0pt}%
\pgfpathmoveto{\pgfqpoint{4.967196in}{2.743231in}}%
\pgfpathlineto{\pgfqpoint{4.981049in}{2.753321in}}%
\pgfpathlineto{\pgfqpoint{4.994917in}{2.763570in}}%
\pgfpathlineto{\pgfqpoint{5.008803in}{2.773979in}}%
\pgfpathlineto{\pgfqpoint{5.022705in}{2.784548in}}%
\pgfpathlineto{\pgfqpoint{5.030082in}{2.791304in}}%
\pgfpathlineto{\pgfqpoint{5.037451in}{2.797956in}}%
\pgfpathlineto{\pgfqpoint{5.044813in}{2.804508in}}%
\pgfpathlineto{\pgfqpoint{5.052167in}{2.810962in}}%
\pgfpathlineto{\pgfqpoint{5.038275in}{2.800583in}}%
\pgfpathlineto{\pgfqpoint{5.024400in}{2.790364in}}%
\pgfpathlineto{\pgfqpoint{5.010541in}{2.780304in}}%
\pgfpathlineto{\pgfqpoint{4.996699in}{2.770404in}}%
\pgfpathlineto{\pgfqpoint{4.989334in}{2.763750in}}%
\pgfpathlineto{\pgfqpoint{4.981962in}{2.757004in}}%
\pgfpathlineto{\pgfqpoint{4.974583in}{2.750166in}}%
\pgfpathlineto{\pgfqpoint{4.967196in}{2.743231in}}%
\pgfpathclose%
\pgfusepath{fill}%
\end{pgfscope}%
\begin{pgfscope}%
\pgfpathrectangle{\pgfqpoint{1.254980in}{0.150000in}}{\pgfqpoint{5.490039in}{5.490039in}}%
\pgfusepath{clip}%
\pgfsetbuttcap%
\pgfsetroundjoin%
\definecolor{currentfill}{rgb}{0.271828,0.209303,0.504434}%
\pgfsetfillcolor{currentfill}%
\pgfsetfillopacity{0.700000}%
\pgfsetlinewidth{0.000000pt}%
\definecolor{currentstroke}{rgb}{0.000000,0.000000,0.000000}%
\pgfsetstrokecolor{currentstroke}%
\pgfsetdash{}{0pt}%
\pgfpathmoveto{\pgfqpoint{4.113483in}{2.016607in}}%
\pgfpathlineto{\pgfqpoint{4.126903in}{2.020848in}}%
\pgfpathlineto{\pgfqpoint{4.140335in}{2.025253in}}%
\pgfpathlineto{\pgfqpoint{4.153777in}{2.029821in}}%
\pgfpathlineto{\pgfqpoint{4.167230in}{2.034553in}}%
\pgfpathlineto{\pgfqpoint{4.174948in}{2.045921in}}%
\pgfpathlineto{\pgfqpoint{4.182661in}{2.057229in}}%
\pgfpathlineto{\pgfqpoint{4.190370in}{2.068477in}}%
\pgfpathlineto{\pgfqpoint{4.198073in}{2.079662in}}%
\pgfpathlineto{\pgfqpoint{4.184624in}{2.074768in}}%
\pgfpathlineto{\pgfqpoint{4.171185in}{2.070038in}}%
\pgfpathlineto{\pgfqpoint{4.157758in}{2.065471in}}%
\pgfpathlineto{\pgfqpoint{4.144342in}{2.061068in}}%
\pgfpathlineto{\pgfqpoint{4.136634in}{2.050034in}}%
\pgfpathlineto{\pgfqpoint{4.128922in}{2.038945in}}%
\pgfpathlineto{\pgfqpoint{4.121205in}{2.027802in}}%
\pgfpathlineto{\pgfqpoint{4.113483in}{2.016607in}}%
\pgfpathclose%
\pgfusepath{fill}%
\end{pgfscope}%
\begin{pgfscope}%
\pgfpathrectangle{\pgfqpoint{1.254980in}{0.150000in}}{\pgfqpoint{5.490039in}{5.490039in}}%
\pgfusepath{clip}%
\pgfsetbuttcap%
\pgfsetroundjoin%
\definecolor{currentfill}{rgb}{0.281412,0.155834,0.469201}%
\pgfsetfillcolor{currentfill}%
\pgfsetfillopacity{0.700000}%
\pgfsetlinewidth{0.000000pt}%
\definecolor{currentstroke}{rgb}{0.000000,0.000000,0.000000}%
\pgfsetstrokecolor{currentstroke}%
\pgfsetdash{}{0pt}%
\pgfpathmoveto{\pgfqpoint{2.722145in}{1.964667in}}%
\pgfpathlineto{\pgfqpoint{2.735520in}{1.950708in}}%
\pgfpathlineto{\pgfqpoint{2.748890in}{1.936973in}}%
\pgfpathlineto{\pgfqpoint{2.762255in}{1.923460in}}%
\pgfpathlineto{\pgfqpoint{2.775615in}{1.910168in}}%
\pgfpathlineto{\pgfqpoint{2.784008in}{1.911767in}}%
\pgfpathlineto{\pgfqpoint{2.792388in}{1.913591in}}%
\pgfpathlineto{\pgfqpoint{2.800753in}{1.915635in}}%
\pgfpathlineto{\pgfqpoint{2.809105in}{1.917893in}}%
\pgfpathlineto{\pgfqpoint{2.795782in}{1.930676in}}%
\pgfpathlineto{\pgfqpoint{2.782454in}{1.943679in}}%
\pgfpathlineto{\pgfqpoint{2.769122in}{1.956903in}}%
\pgfpathlineto{\pgfqpoint{2.755786in}{1.970352in}}%
\pgfpathlineto{\pgfqpoint{2.747397in}{1.968592in}}%
\pgfpathlineto{\pgfqpoint{2.738994in}{1.967054in}}%
\pgfpathlineto{\pgfqpoint{2.730577in}{1.965745in}}%
\pgfpathlineto{\pgfqpoint{2.722145in}{1.964667in}}%
\pgfpathclose%
\pgfusepath{fill}%
\end{pgfscope}%
\begin{pgfscope}%
\pgfpathrectangle{\pgfqpoint{1.254980in}{0.150000in}}{\pgfqpoint{5.490039in}{5.490039in}}%
\pgfusepath{clip}%
\pgfsetbuttcap%
\pgfsetroundjoin%
\definecolor{currentfill}{rgb}{0.183898,0.422383,0.556944}%
\pgfsetfillcolor{currentfill}%
\pgfsetfillopacity{0.700000}%
\pgfsetlinewidth{0.000000pt}%
\definecolor{currentstroke}{rgb}{0.000000,0.000000,0.000000}%
\pgfsetstrokecolor{currentstroke}%
\pgfsetdash{}{0pt}%
\pgfpathmoveto{\pgfqpoint{4.682666in}{2.503570in}}%
\pgfpathlineto{\pgfqpoint{4.696360in}{2.512194in}}%
\pgfpathlineto{\pgfqpoint{4.710070in}{2.520979in}}%
\pgfpathlineto{\pgfqpoint{4.723794in}{2.529925in}}%
\pgfpathlineto{\pgfqpoint{4.737534in}{2.539031in}}%
\pgfpathlineto{\pgfqpoint{4.745048in}{2.547870in}}%
\pgfpathlineto{\pgfqpoint{4.752556in}{2.556601in}}%
\pgfpathlineto{\pgfqpoint{4.760057in}{2.565225in}}%
\pgfpathlineto{\pgfqpoint{4.767552in}{2.573743in}}%
\pgfpathlineto{\pgfqpoint{4.753818in}{2.564707in}}%
\pgfpathlineto{\pgfqpoint{4.740100in}{2.555831in}}%
\pgfpathlineto{\pgfqpoint{4.726396in}{2.547116in}}%
\pgfpathlineto{\pgfqpoint{4.712708in}{2.538562in}}%
\pgfpathlineto{\pgfqpoint{4.705207in}{2.529962in}}%
\pgfpathlineto{\pgfqpoint{4.697700in}{2.521265in}}%
\pgfpathlineto{\pgfqpoint{4.690186in}{2.512468in}}%
\pgfpathlineto{\pgfqpoint{4.682666in}{2.503570in}}%
\pgfpathclose%
\pgfusepath{fill}%
\end{pgfscope}%
\begin{pgfscope}%
\pgfpathrectangle{\pgfqpoint{1.254980in}{0.150000in}}{\pgfqpoint{5.490039in}{5.490039in}}%
\pgfusepath{clip}%
\pgfsetbuttcap%
\pgfsetroundjoin%
\definecolor{currentfill}{rgb}{0.237441,0.305202,0.541921}%
\pgfsetfillcolor{currentfill}%
\pgfsetfillopacity{0.700000}%
\pgfsetlinewidth{0.000000pt}%
\definecolor{currentstroke}{rgb}{0.000000,0.000000,0.000000}%
\pgfsetstrokecolor{currentstroke}%
\pgfsetdash{}{0pt}%
\pgfpathmoveto{\pgfqpoint{2.453218in}{2.293645in}}%
\pgfpathlineto{\pgfqpoint{2.466744in}{2.274821in}}%
\pgfpathlineto{\pgfqpoint{2.480260in}{2.256260in}}%
\pgfpathlineto{\pgfqpoint{2.493767in}{2.237959in}}%
\pgfpathlineto{\pgfqpoint{2.507264in}{2.219917in}}%
\pgfpathlineto{\pgfqpoint{2.515869in}{2.218876in}}%
\pgfpathlineto{\pgfqpoint{2.524455in}{2.218100in}}%
\pgfpathlineto{\pgfqpoint{2.533025in}{2.217583in}}%
\pgfpathlineto{\pgfqpoint{2.541578in}{2.217319in}}%
\pgfpathlineto{\pgfqpoint{2.528126in}{2.234837in}}%
\pgfpathlineto{\pgfqpoint{2.514666in}{2.252612in}}%
\pgfpathlineto{\pgfqpoint{2.501196in}{2.270646in}}%
\pgfpathlineto{\pgfqpoint{2.487717in}{2.288942in}}%
\pgfpathlineto{\pgfqpoint{2.479119in}{2.289720in}}%
\pgfpathlineto{\pgfqpoint{2.470503in}{2.290760in}}%
\pgfpathlineto{\pgfqpoint{2.461870in}{2.292066in}}%
\pgfpathlineto{\pgfqpoint{2.453218in}{2.293645in}}%
\pgfpathclose%
\pgfusepath{fill}%
\end{pgfscope}%
\begin{pgfscope}%
\pgfpathrectangle{\pgfqpoint{1.254980in}{0.150000in}}{\pgfqpoint{5.490039in}{5.490039in}}%
\pgfusepath{clip}%
\pgfsetbuttcap%
\pgfsetroundjoin%
\definecolor{currentfill}{rgb}{0.123444,0.636809,0.528763}%
\pgfsetfillcolor{currentfill}%
\pgfsetfillopacity{0.700000}%
\pgfsetlinewidth{0.000000pt}%
\definecolor{currentstroke}{rgb}{0.000000,0.000000,0.000000}%
\pgfsetstrokecolor{currentstroke}%
\pgfsetdash{}{0pt}%
\pgfpathmoveto{\pgfqpoint{5.421170in}{3.089064in}}%
\pgfpathlineto{\pgfqpoint{5.435289in}{3.100739in}}%
\pgfpathlineto{\pgfqpoint{5.449428in}{3.112572in}}%
\pgfpathlineto{\pgfqpoint{5.463585in}{3.124563in}}%
\pgfpathlineto{\pgfqpoint{5.477762in}{3.136714in}}%
\pgfpathlineto{\pgfqpoint{5.484871in}{3.139999in}}%
\pgfpathlineto{\pgfqpoint{5.491973in}{3.143223in}}%
\pgfpathlineto{\pgfqpoint{5.499066in}{3.146389in}}%
\pgfpathlineto{\pgfqpoint{5.506152in}{3.149501in}}%
\pgfpathlineto{\pgfqpoint{5.491996in}{3.137725in}}%
\pgfpathlineto{\pgfqpoint{5.477859in}{3.126107in}}%
\pgfpathlineto{\pgfqpoint{5.463741in}{3.114648in}}%
\pgfpathlineto{\pgfqpoint{5.449641in}{3.103346in}}%
\pgfpathlineto{\pgfqpoint{5.442535in}{3.099850in}}%
\pgfpathlineto{\pgfqpoint{5.435421in}{3.096307in}}%
\pgfpathlineto{\pgfqpoint{5.428299in}{3.092713in}}%
\pgfpathlineto{\pgfqpoint{5.421170in}{3.089064in}}%
\pgfpathclose%
\pgfusepath{fill}%
\end{pgfscope}%
\begin{pgfscope}%
\pgfpathrectangle{\pgfqpoint{1.254980in}{0.150000in}}{\pgfqpoint{5.490039in}{5.490039in}}%
\pgfusepath{clip}%
\pgfsetbuttcap%
\pgfsetroundjoin%
\definecolor{currentfill}{rgb}{0.268510,0.009605,0.335427}%
\pgfsetfillcolor{currentfill}%
\pgfsetfillopacity{0.700000}%
\pgfsetlinewidth{0.000000pt}%
\definecolor{currentstroke}{rgb}{0.000000,0.000000,0.000000}%
\pgfsetstrokecolor{currentstroke}%
\pgfsetdash{}{0pt}%
\pgfpathmoveto{\pgfqpoint{3.213674in}{1.668365in}}%
\pgfpathlineto{\pgfqpoint{3.226934in}{1.661886in}}%
\pgfpathlineto{\pgfqpoint{3.240195in}{1.655592in}}%
\pgfpathlineto{\pgfqpoint{3.253458in}{1.649483in}}%
\pgfpathlineto{\pgfqpoint{3.266722in}{1.643557in}}%
\pgfpathlineto{\pgfqpoint{3.274790in}{1.650455in}}%
\pgfpathlineto{\pgfqpoint{3.282848in}{1.657481in}}%
\pgfpathlineto{\pgfqpoint{3.290899in}{1.664633in}}%
\pgfpathlineto{\pgfqpoint{3.298941in}{1.671904in}}%
\pgfpathlineto{\pgfqpoint{3.285698in}{1.677390in}}%
\pgfpathlineto{\pgfqpoint{3.272457in}{1.683059in}}%
\pgfpathlineto{\pgfqpoint{3.259218in}{1.688913in}}%
\pgfpathlineto{\pgfqpoint{3.245981in}{1.694951in}}%
\pgfpathlineto{\pgfqpoint{3.237917in}{1.688109in}}%
\pgfpathlineto{\pgfqpoint{3.229845in}{1.681395in}}%
\pgfpathlineto{\pgfqpoint{3.221764in}{1.674812in}}%
\pgfpathlineto{\pgfqpoint{3.213674in}{1.668365in}}%
\pgfpathclose%
\pgfusepath{fill}%
\end{pgfscope}%
\begin{pgfscope}%
\pgfpathrectangle{\pgfqpoint{1.254980in}{0.150000in}}{\pgfqpoint{5.490039in}{5.490039in}}%
\pgfusepath{clip}%
\pgfsetbuttcap%
\pgfsetroundjoin%
\definecolor{currentfill}{rgb}{0.283072,0.130895,0.449241}%
\pgfsetfillcolor{currentfill}%
\pgfsetfillopacity{0.700000}%
\pgfsetlinewidth{0.000000pt}%
\definecolor{currentstroke}{rgb}{0.000000,0.000000,0.000000}%
\pgfsetstrokecolor{currentstroke}%
\pgfsetdash{}{0pt}%
\pgfpathmoveto{\pgfqpoint{2.775615in}{1.910168in}}%
\pgfpathlineto{\pgfqpoint{2.788971in}{1.897096in}}%
\pgfpathlineto{\pgfqpoint{2.802323in}{1.884241in}}%
\pgfpathlineto{\pgfqpoint{2.815670in}{1.871603in}}%
\pgfpathlineto{\pgfqpoint{2.829014in}{1.859181in}}%
\pgfpathlineto{\pgfqpoint{2.837370in}{1.861299in}}%
\pgfpathlineto{\pgfqpoint{2.845713in}{1.863635in}}%
\pgfpathlineto{\pgfqpoint{2.854043in}{1.866183in}}%
\pgfpathlineto{\pgfqpoint{2.862359in}{1.868938in}}%
\pgfpathlineto{\pgfqpoint{2.849051in}{1.880854in}}%
\pgfpathlineto{\pgfqpoint{2.835739in}{1.892984in}}%
\pgfpathlineto{\pgfqpoint{2.822424in}{1.905330in}}%
\pgfpathlineto{\pgfqpoint{2.809105in}{1.917893in}}%
\pgfpathlineto{\pgfqpoint{2.800753in}{1.915635in}}%
\pgfpathlineto{\pgfqpoint{2.792388in}{1.913591in}}%
\pgfpathlineto{\pgfqpoint{2.784008in}{1.911767in}}%
\pgfpathlineto{\pgfqpoint{2.775615in}{1.910168in}}%
\pgfpathclose%
\pgfusepath{fill}%
\end{pgfscope}%
\begin{pgfscope}%
\pgfpathrectangle{\pgfqpoint{1.254980in}{0.150000in}}{\pgfqpoint{5.490039in}{5.490039in}}%
\pgfusepath{clip}%
\pgfsetbuttcap%
\pgfsetroundjoin%
\definecolor{currentfill}{rgb}{0.262138,0.242286,0.520837}%
\pgfsetfillcolor{currentfill}%
\pgfsetfillopacity{0.700000}%
\pgfsetlinewidth{0.000000pt}%
\definecolor{currentstroke}{rgb}{0.000000,0.000000,0.000000}%
\pgfsetstrokecolor{currentstroke}%
\pgfsetdash{}{0pt}%
\pgfpathmoveto{\pgfqpoint{4.198073in}{2.079662in}}%
\pgfpathlineto{\pgfqpoint{4.211534in}{2.084719in}}%
\pgfpathlineto{\pgfqpoint{4.225006in}{2.089939in}}%
\pgfpathlineto{\pgfqpoint{4.238490in}{2.095321in}}%
\pgfpathlineto{\pgfqpoint{4.251985in}{2.100866in}}%
\pgfpathlineto{\pgfqpoint{4.259680in}{2.112133in}}%
\pgfpathlineto{\pgfqpoint{4.267370in}{2.123328in}}%
\pgfpathlineto{\pgfqpoint{4.275055in}{2.134450in}}%
\pgfpathlineto{\pgfqpoint{4.282734in}{2.145498in}}%
\pgfpathlineto{\pgfqpoint{4.269242in}{2.139819in}}%
\pgfpathlineto{\pgfqpoint{4.255762in}{2.134303in}}%
\pgfpathlineto{\pgfqpoint{4.242294in}{2.128949in}}%
\pgfpathlineto{\pgfqpoint{4.228837in}{2.123759in}}%
\pgfpathlineto{\pgfqpoint{4.221154in}{2.112833in}}%
\pgfpathlineto{\pgfqpoint{4.213465in}{2.101841in}}%
\pgfpathlineto{\pgfqpoint{4.205772in}{2.090784in}}%
\pgfpathlineto{\pgfqpoint{4.198073in}{2.079662in}}%
\pgfpathclose%
\pgfusepath{fill}%
\end{pgfscope}%
\begin{pgfscope}%
\pgfpathrectangle{\pgfqpoint{1.254980in}{0.150000in}}{\pgfqpoint{5.490039in}{5.490039in}}%
\pgfusepath{clip}%
\pgfsetbuttcap%
\pgfsetroundjoin%
\definecolor{currentfill}{rgb}{0.268510,0.009605,0.335427}%
\pgfsetfillcolor{currentfill}%
\pgfsetfillopacity{0.700000}%
\pgfsetlinewidth{0.000000pt}%
\definecolor{currentstroke}{rgb}{0.000000,0.000000,0.000000}%
\pgfsetstrokecolor{currentstroke}%
\pgfsetdash{}{0pt}%
\pgfpathmoveto{\pgfqpoint{3.351940in}{1.651780in}}%
\pgfpathlineto{\pgfqpoint{3.365198in}{1.647200in}}%
\pgfpathlineto{\pgfqpoint{3.378458in}{1.642798in}}%
\pgfpathlineto{\pgfqpoint{3.391722in}{1.638575in}}%
\pgfpathlineto{\pgfqpoint{3.404990in}{1.634529in}}%
\pgfpathlineto{\pgfqpoint{3.412986in}{1.642762in}}%
\pgfpathlineto{\pgfqpoint{3.420974in}{1.651092in}}%
\pgfpathlineto{\pgfqpoint{3.428956in}{1.659517in}}%
\pgfpathlineto{\pgfqpoint{3.436930in}{1.668032in}}%
\pgfpathlineto{\pgfqpoint{3.423680in}{1.671667in}}%
\pgfpathlineto{\pgfqpoint{3.410434in}{1.675479in}}%
\pgfpathlineto{\pgfqpoint{3.397191in}{1.679469in}}%
\pgfpathlineto{\pgfqpoint{3.383952in}{1.683637in}}%
\pgfpathlineto{\pgfqpoint{3.375961in}{1.675523in}}%
\pgfpathlineto{\pgfqpoint{3.367961in}{1.667506in}}%
\pgfpathlineto{\pgfqpoint{3.359955in}{1.659590in}}%
\pgfpathlineto{\pgfqpoint{3.351940in}{1.651780in}}%
\pgfpathclose%
\pgfusepath{fill}%
\end{pgfscope}%
\begin{pgfscope}%
\pgfpathrectangle{\pgfqpoint{1.254980in}{0.150000in}}{\pgfqpoint{5.490039in}{5.490039in}}%
\pgfusepath{clip}%
\pgfsetbuttcap%
\pgfsetroundjoin%
\definecolor{currentfill}{rgb}{0.221989,0.339161,0.548752}%
\pgfsetfillcolor{currentfill}%
\pgfsetfillopacity{0.700000}%
\pgfsetlinewidth{0.000000pt}%
\definecolor{currentstroke}{rgb}{0.000000,0.000000,0.000000}%
\pgfsetstrokecolor{currentstroke}%
\pgfsetdash{}{0pt}%
\pgfpathmoveto{\pgfqpoint{2.399008in}{2.371614in}}%
\pgfpathlineto{\pgfqpoint{2.412576in}{2.351716in}}%
\pgfpathlineto{\pgfqpoint{2.426134in}{2.332090in}}%
\pgfpathlineto{\pgfqpoint{2.439681in}{2.312734in}}%
\pgfpathlineto{\pgfqpoint{2.453218in}{2.293645in}}%
\pgfpathlineto{\pgfqpoint{2.461870in}{2.292066in}}%
\pgfpathlineto{\pgfqpoint{2.470503in}{2.290760in}}%
\pgfpathlineto{\pgfqpoint{2.479119in}{2.289720in}}%
\pgfpathlineto{\pgfqpoint{2.487717in}{2.288942in}}%
\pgfpathlineto{\pgfqpoint{2.474228in}{2.307502in}}%
\pgfpathlineto{\pgfqpoint{2.460729in}{2.326328in}}%
\pgfpathlineto{\pgfqpoint{2.447220in}{2.345423in}}%
\pgfpathlineto{\pgfqpoint{2.433700in}{2.364789in}}%
\pgfpathlineto{\pgfqpoint{2.425055in}{2.366085in}}%
\pgfpathlineto{\pgfqpoint{2.416391in}{2.367651in}}%
\pgfpathlineto{\pgfqpoint{2.407709in}{2.369492in}}%
\pgfpathlineto{\pgfqpoint{2.399008in}{2.371614in}}%
\pgfpathclose%
\pgfusepath{fill}%
\end{pgfscope}%
\begin{pgfscope}%
\pgfpathrectangle{\pgfqpoint{1.254980in}{0.150000in}}{\pgfqpoint{5.490039in}{5.490039in}}%
\pgfusepath{clip}%
\pgfsetbuttcap%
\pgfsetroundjoin%
\definecolor{currentfill}{rgb}{0.137770,0.537492,0.554906}%
\pgfsetfillcolor{currentfill}%
\pgfsetfillopacity{0.700000}%
\pgfsetlinewidth{0.000000pt}%
\definecolor{currentstroke}{rgb}{0.000000,0.000000,0.000000}%
\pgfsetstrokecolor{currentstroke}%
\pgfsetdash{}{0pt}%
\pgfpathmoveto{\pgfqpoint{5.052167in}{2.810962in}}%
\pgfpathlineto{\pgfqpoint{5.066075in}{2.821500in}}%
\pgfpathlineto{\pgfqpoint{5.080001in}{2.832198in}}%
\pgfpathlineto{\pgfqpoint{5.093944in}{2.843056in}}%
\pgfpathlineto{\pgfqpoint{5.107905in}{2.854074in}}%
\pgfpathlineto{\pgfqpoint{5.115240in}{2.860222in}}%
\pgfpathlineto{\pgfqpoint{5.122567in}{2.866269in}}%
\pgfpathlineto{\pgfqpoint{5.129886in}{2.872219in}}%
\pgfpathlineto{\pgfqpoint{5.137197in}{2.878073in}}%
\pgfpathlineto{\pgfqpoint{5.123249in}{2.867277in}}%
\pgfpathlineto{\pgfqpoint{5.109318in}{2.856640in}}%
\pgfpathlineto{\pgfqpoint{5.095404in}{2.846162in}}%
\pgfpathlineto{\pgfqpoint{5.081507in}{2.835844in}}%
\pgfpathlineto{\pgfqpoint{5.074183in}{2.829758in}}%
\pgfpathlineto{\pgfqpoint{5.066852in}{2.823584in}}%
\pgfpathlineto{\pgfqpoint{5.059513in}{2.817319in}}%
\pgfpathlineto{\pgfqpoint{5.052167in}{2.810962in}}%
\pgfpathclose%
\pgfusepath{fill}%
\end{pgfscope}%
\begin{pgfscope}%
\pgfpathrectangle{\pgfqpoint{1.254980in}{0.150000in}}{\pgfqpoint{5.490039in}{5.490039in}}%
\pgfusepath{clip}%
\pgfsetbuttcap%
\pgfsetroundjoin%
\definecolor{currentfill}{rgb}{0.280894,0.078907,0.402329}%
\pgfsetfillcolor{currentfill}%
\pgfsetfillopacity{0.700000}%
\pgfsetlinewidth{0.000000pt}%
\definecolor{currentstroke}{rgb}{0.000000,0.000000,0.000000}%
\pgfsetstrokecolor{currentstroke}%
\pgfsetdash{}{0pt}%
\pgfpathmoveto{\pgfqpoint{3.744149in}{1.758532in}}%
\pgfpathlineto{\pgfqpoint{3.757457in}{1.758893in}}%
\pgfpathlineto{\pgfqpoint{3.770773in}{1.759423in}}%
\pgfpathlineto{\pgfqpoint{3.784096in}{1.760120in}}%
\pgfpathlineto{\pgfqpoint{3.797427in}{1.760985in}}%
\pgfpathlineto{\pgfqpoint{3.805263in}{1.771912in}}%
\pgfpathlineto{\pgfqpoint{3.813095in}{1.782849in}}%
\pgfpathlineto{\pgfqpoint{3.820921in}{1.793793in}}%
\pgfpathlineto{\pgfqpoint{3.828743in}{1.804741in}}%
\pgfpathlineto{\pgfqpoint{3.815420in}{1.803575in}}%
\pgfpathlineto{\pgfqpoint{3.802105in}{1.802577in}}%
\pgfpathlineto{\pgfqpoint{3.788798in}{1.801747in}}%
\pgfpathlineto{\pgfqpoint{3.775499in}{1.801084in}}%
\pgfpathlineto{\pgfqpoint{3.767669in}{1.790427in}}%
\pgfpathlineto{\pgfqpoint{3.759834in}{1.779781in}}%
\pgfpathlineto{\pgfqpoint{3.751994in}{1.769148in}}%
\pgfpathlineto{\pgfqpoint{3.744149in}{1.758532in}}%
\pgfpathclose%
\pgfusepath{fill}%
\end{pgfscope}%
\begin{pgfscope}%
\pgfpathrectangle{\pgfqpoint{1.254980in}{0.150000in}}{\pgfqpoint{5.490039in}{5.490039in}}%
\pgfusepath{clip}%
\pgfsetbuttcap%
\pgfsetroundjoin%
\definecolor{currentfill}{rgb}{0.137339,0.662252,0.515571}%
\pgfsetfillcolor{currentfill}%
\pgfsetfillopacity{0.700000}%
\pgfsetlinewidth{0.000000pt}%
\definecolor{currentstroke}{rgb}{0.000000,0.000000,0.000000}%
\pgfsetstrokecolor{currentstroke}%
\pgfsetdash{}{0pt}%
\pgfpathmoveto{\pgfqpoint{5.506152in}{3.149501in}}%
\pgfpathlineto{\pgfqpoint{5.520327in}{3.161434in}}%
\pgfpathlineto{\pgfqpoint{5.534521in}{3.173527in}}%
\pgfpathlineto{\pgfqpoint{5.548734in}{3.185777in}}%
\pgfpathlineto{\pgfqpoint{5.562967in}{3.198187in}}%
\pgfpathlineto{\pgfqpoint{5.570023in}{3.200855in}}%
\pgfpathlineto{\pgfqpoint{5.577070in}{3.203471in}}%
\pgfpathlineto{\pgfqpoint{5.584110in}{3.206039in}}%
\pgfpathlineto{\pgfqpoint{5.591142in}{3.208565in}}%
\pgfpathlineto{\pgfqpoint{5.576932in}{3.196562in}}%
\pgfpathlineto{\pgfqpoint{5.562741in}{3.184716in}}%
\pgfpathlineto{\pgfqpoint{5.548569in}{3.173029in}}%
\pgfpathlineto{\pgfqpoint{5.534417in}{3.161498in}}%
\pgfpathlineto{\pgfqpoint{5.527362in}{3.158558in}}%
\pgfpathlineto{\pgfqpoint{5.520299in}{3.155581in}}%
\pgfpathlineto{\pgfqpoint{5.513229in}{3.152563in}}%
\pgfpathlineto{\pgfqpoint{5.506152in}{3.149501in}}%
\pgfpathclose%
\pgfusepath{fill}%
\end{pgfscope}%
\begin{pgfscope}%
\pgfpathrectangle{\pgfqpoint{1.254980in}{0.150000in}}{\pgfqpoint{5.490039in}{5.490039in}}%
\pgfusepath{clip}%
\pgfsetbuttcap%
\pgfsetroundjoin%
\definecolor{currentfill}{rgb}{0.277941,0.056324,0.381191}%
\pgfsetfillcolor{currentfill}%
\pgfsetfillopacity{0.700000}%
\pgfsetlinewidth{0.000000pt}%
\definecolor{currentstroke}{rgb}{0.000000,0.000000,0.000000}%
\pgfsetstrokecolor{currentstroke}%
\pgfsetdash{}{0pt}%
\pgfpathmoveto{\pgfqpoint{3.659512in}{1.717844in}}%
\pgfpathlineto{\pgfqpoint{3.672803in}{1.717201in}}%
\pgfpathlineto{\pgfqpoint{3.686100in}{1.716728in}}%
\pgfpathlineto{\pgfqpoint{3.699404in}{1.716424in}}%
\pgfpathlineto{\pgfqpoint{3.712715in}{1.716289in}}%
\pgfpathlineto{\pgfqpoint{3.720581in}{1.726810in}}%
\pgfpathlineto{\pgfqpoint{3.728442in}{1.737360in}}%
\pgfpathlineto{\pgfqpoint{3.736298in}{1.747935in}}%
\pgfpathlineto{\pgfqpoint{3.744149in}{1.758532in}}%
\pgfpathlineto{\pgfqpoint{3.730848in}{1.758338in}}%
\pgfpathlineto{\pgfqpoint{3.717554in}{1.758314in}}%
\pgfpathlineto{\pgfqpoint{3.704267in}{1.758459in}}%
\pgfpathlineto{\pgfqpoint{3.690987in}{1.758773in}}%
\pgfpathlineto{\pgfqpoint{3.683126in}{1.748494in}}%
\pgfpathlineto{\pgfqpoint{3.675260in}{1.738244in}}%
\pgfpathlineto{\pgfqpoint{3.667389in}{1.728026in}}%
\pgfpathlineto{\pgfqpoint{3.659512in}{1.717844in}}%
\pgfpathclose%
\pgfusepath{fill}%
\end{pgfscope}%
\begin{pgfscope}%
\pgfpathrectangle{\pgfqpoint{1.254980in}{0.150000in}}{\pgfqpoint{5.490039in}{5.490039in}}%
\pgfusepath{clip}%
\pgfsetbuttcap%
\pgfsetroundjoin%
\definecolor{currentfill}{rgb}{0.216210,0.351535,0.550627}%
\pgfsetfillcolor{currentfill}%
\pgfsetfillopacity{0.700000}%
\pgfsetlinewidth{0.000000pt}%
\definecolor{currentstroke}{rgb}{0.000000,0.000000,0.000000}%
\pgfsetstrokecolor{currentstroke}%
\pgfsetdash{}{0pt}%
\pgfpathmoveto{\pgfqpoint{4.482779in}{2.325220in}}%
\pgfpathlineto{\pgfqpoint{4.496375in}{2.332587in}}%
\pgfpathlineto{\pgfqpoint{4.509984in}{2.340115in}}%
\pgfpathlineto{\pgfqpoint{4.523607in}{2.347805in}}%
\pgfpathlineto{\pgfqpoint{4.537243in}{2.355656in}}%
\pgfpathlineto{\pgfqpoint{4.544842in}{2.365787in}}%
\pgfpathlineto{\pgfqpoint{4.552435in}{2.375818in}}%
\pgfpathlineto{\pgfqpoint{4.560022in}{2.385749in}}%
\pgfpathlineto{\pgfqpoint{4.567603in}{2.395579in}}%
\pgfpathlineto{\pgfqpoint{4.553970in}{2.387710in}}%
\pgfpathlineto{\pgfqpoint{4.540352in}{2.380001in}}%
\pgfpathlineto{\pgfqpoint{4.526747in}{2.372454in}}%
\pgfpathlineto{\pgfqpoint{4.513155in}{2.365069in}}%
\pgfpathlineto{\pgfqpoint{4.505570in}{2.355246in}}%
\pgfpathlineto{\pgfqpoint{4.497979in}{2.345330in}}%
\pgfpathlineto{\pgfqpoint{4.490382in}{2.335321in}}%
\pgfpathlineto{\pgfqpoint{4.482779in}{2.325220in}}%
\pgfpathclose%
\pgfusepath{fill}%
\end{pgfscope}%
\begin{pgfscope}%
\pgfpathrectangle{\pgfqpoint{1.254980in}{0.150000in}}{\pgfqpoint{5.490039in}{5.490039in}}%
\pgfusepath{clip}%
\pgfsetbuttcap%
\pgfsetroundjoin%
\definecolor{currentfill}{rgb}{0.282910,0.105393,0.426902}%
\pgfsetfillcolor{currentfill}%
\pgfsetfillopacity{0.700000}%
\pgfsetlinewidth{0.000000pt}%
\definecolor{currentstroke}{rgb}{0.000000,0.000000,0.000000}%
\pgfsetstrokecolor{currentstroke}%
\pgfsetdash{}{0pt}%
\pgfpathmoveto{\pgfqpoint{3.828743in}{1.804741in}}%
\pgfpathlineto{\pgfqpoint{3.842073in}{1.806073in}}%
\pgfpathlineto{\pgfqpoint{3.855412in}{1.807573in}}%
\pgfpathlineto{\pgfqpoint{3.868760in}{1.809238in}}%
\pgfpathlineto{\pgfqpoint{3.882116in}{1.811069in}}%
\pgfpathlineto{\pgfqpoint{3.889925in}{1.822302in}}%
\pgfpathlineto{\pgfqpoint{3.897729in}{1.833528in}}%
\pgfpathlineto{\pgfqpoint{3.905529in}{1.844742in}}%
\pgfpathlineto{\pgfqpoint{3.913323in}{1.855944in}}%
\pgfpathlineto{\pgfqpoint{3.899974in}{1.853839in}}%
\pgfpathlineto{\pgfqpoint{3.886634in}{1.851900in}}%
\pgfpathlineto{\pgfqpoint{3.873302in}{1.850128in}}%
\pgfpathlineto{\pgfqpoint{3.859979in}{1.848522in}}%
\pgfpathlineto{\pgfqpoint{3.852177in}{1.837583in}}%
\pgfpathlineto{\pgfqpoint{3.844371in}{1.826639in}}%
\pgfpathlineto{\pgfqpoint{3.836559in}{1.815690in}}%
\pgfpathlineto{\pgfqpoint{3.828743in}{1.804741in}}%
\pgfpathclose%
\pgfusepath{fill}%
\end{pgfscope}%
\begin{pgfscope}%
\pgfpathrectangle{\pgfqpoint{1.254980in}{0.150000in}}{\pgfqpoint{5.490039in}{5.490039in}}%
\pgfusepath{clip}%
\pgfsetbuttcap%
\pgfsetroundjoin%
\definecolor{currentfill}{rgb}{0.283091,0.110553,0.431554}%
\pgfsetfillcolor{currentfill}%
\pgfsetfillopacity{0.700000}%
\pgfsetlinewidth{0.000000pt}%
\definecolor{currentstroke}{rgb}{0.000000,0.000000,0.000000}%
\pgfsetstrokecolor{currentstroke}%
\pgfsetdash{}{0pt}%
\pgfpathmoveto{\pgfqpoint{2.829014in}{1.859181in}}%
\pgfpathlineto{\pgfqpoint{2.842354in}{1.846972in}}%
\pgfpathlineto{\pgfqpoint{2.855691in}{1.834975in}}%
\pgfpathlineto{\pgfqpoint{2.869024in}{1.823189in}}%
\pgfpathlineto{\pgfqpoint{2.882354in}{1.811613in}}%
\pgfpathlineto{\pgfqpoint{2.890675in}{1.814248in}}%
\pgfpathlineto{\pgfqpoint{2.898983in}{1.817093in}}%
\pgfpathlineto{\pgfqpoint{2.907279in}{1.820143in}}%
\pgfpathlineto{\pgfqpoint{2.915561in}{1.823393in}}%
\pgfpathlineto{\pgfqpoint{2.902265in}{1.834464in}}%
\pgfpathlineto{\pgfqpoint{2.888966in}{1.845745in}}%
\pgfpathlineto{\pgfqpoint{2.875664in}{1.857236in}}%
\pgfpathlineto{\pgfqpoint{2.862359in}{1.868938in}}%
\pgfpathlineto{\pgfqpoint{2.854043in}{1.866183in}}%
\pgfpathlineto{\pgfqpoint{2.845713in}{1.863635in}}%
\pgfpathlineto{\pgfqpoint{2.837370in}{1.861299in}}%
\pgfpathlineto{\pgfqpoint{2.829014in}{1.859181in}}%
\pgfpathclose%
\pgfusepath{fill}%
\end{pgfscope}%
\begin{pgfscope}%
\pgfpathrectangle{\pgfqpoint{1.254980in}{0.150000in}}{\pgfqpoint{5.490039in}{5.490039in}}%
\pgfusepath{clip}%
\pgfsetbuttcap%
\pgfsetroundjoin%
\definecolor{currentfill}{rgb}{0.273809,0.031497,0.358853}%
\pgfsetfillcolor{currentfill}%
\pgfsetfillopacity{0.700000}%
\pgfsetlinewidth{0.000000pt}%
\definecolor{currentstroke}{rgb}{0.000000,0.000000,0.000000}%
\pgfsetstrokecolor{currentstroke}%
\pgfsetdash{}{0pt}%
\pgfpathmoveto{\pgfqpoint{3.074981in}{1.706392in}}%
\pgfpathlineto{\pgfqpoint{3.088260in}{1.697928in}}%
\pgfpathlineto{\pgfqpoint{3.101539in}{1.689657in}}%
\pgfpathlineto{\pgfqpoint{3.114819in}{1.681578in}}%
\pgfpathlineto{\pgfqpoint{3.128098in}{1.673690in}}%
\pgfpathlineto{\pgfqpoint{3.136251in}{1.679068in}}%
\pgfpathlineto{\pgfqpoint{3.144394in}{1.684607in}}%
\pgfpathlineto{\pgfqpoint{3.152527in}{1.690303in}}%
\pgfpathlineto{\pgfqpoint{3.160650in}{1.696150in}}%
\pgfpathlineto{\pgfqpoint{3.147397in}{1.703568in}}%
\pgfpathlineto{\pgfqpoint{3.134144in}{1.711177in}}%
\pgfpathlineto{\pgfqpoint{3.120891in}{1.718978in}}%
\pgfpathlineto{\pgfqpoint{3.107639in}{1.726971in}}%
\pgfpathlineto{\pgfqpoint{3.099490in}{1.721583in}}%
\pgfpathlineto{\pgfqpoint{3.091331in}{1.716354in}}%
\pgfpathlineto{\pgfqpoint{3.083161in}{1.711289in}}%
\pgfpathlineto{\pgfqpoint{3.074981in}{1.706392in}}%
\pgfpathclose%
\pgfusepath{fill}%
\end{pgfscope}%
\begin{pgfscope}%
\pgfpathrectangle{\pgfqpoint{1.254980in}{0.150000in}}{\pgfqpoint{5.490039in}{5.490039in}}%
\pgfusepath{clip}%
\pgfsetbuttcap%
\pgfsetroundjoin%
\definecolor{currentfill}{rgb}{0.171176,0.452530,0.557965}%
\pgfsetfillcolor{currentfill}%
\pgfsetfillopacity{0.700000}%
\pgfsetlinewidth{0.000000pt}%
\definecolor{currentstroke}{rgb}{0.000000,0.000000,0.000000}%
\pgfsetstrokecolor{currentstroke}%
\pgfsetdash{}{0pt}%
\pgfpathmoveto{\pgfqpoint{4.767552in}{2.573743in}}%
\pgfpathlineto{\pgfqpoint{4.781301in}{2.582940in}}%
\pgfpathlineto{\pgfqpoint{4.795065in}{2.592298in}}%
\pgfpathlineto{\pgfqpoint{4.808845in}{2.601816in}}%
\pgfpathlineto{\pgfqpoint{4.822641in}{2.611494in}}%
\pgfpathlineto{\pgfqpoint{4.830122in}{2.619820in}}%
\pgfpathlineto{\pgfqpoint{4.837596in}{2.628035in}}%
\pgfpathlineto{\pgfqpoint{4.845063in}{2.636141in}}%
\pgfpathlineto{\pgfqpoint{4.852523in}{2.644141in}}%
\pgfpathlineto{\pgfqpoint{4.838734in}{2.634562in}}%
\pgfpathlineto{\pgfqpoint{4.824961in}{2.625144in}}%
\pgfpathlineto{\pgfqpoint{4.811204in}{2.615886in}}%
\pgfpathlineto{\pgfqpoint{4.797462in}{2.606789in}}%
\pgfpathlineto{\pgfqpoint{4.789995in}{2.598679in}}%
\pgfpathlineto{\pgfqpoint{4.782520in}{2.590469in}}%
\pgfpathlineto{\pgfqpoint{4.775039in}{2.582158in}}%
\pgfpathlineto{\pgfqpoint{4.767552in}{2.573743in}}%
\pgfpathclose%
\pgfusepath{fill}%
\end{pgfscope}%
\begin{pgfscope}%
\pgfpathrectangle{\pgfqpoint{1.254980in}{0.150000in}}{\pgfqpoint{5.490039in}{5.490039in}}%
\pgfusepath{clip}%
\pgfsetbuttcap%
\pgfsetroundjoin%
\definecolor{currentfill}{rgb}{0.273809,0.031497,0.358853}%
\pgfsetfillcolor{currentfill}%
\pgfsetfillopacity{0.700000}%
\pgfsetlinewidth{0.000000pt}%
\definecolor{currentstroke}{rgb}{0.000000,0.000000,0.000000}%
\pgfsetstrokecolor{currentstroke}%
\pgfsetdash{}{0pt}%
\pgfpathmoveto{\pgfqpoint{3.574799in}{1.683227in}}%
\pgfpathlineto{\pgfqpoint{3.588078in}{1.681545in}}%
\pgfpathlineto{\pgfqpoint{3.601362in}{1.680035in}}%
\pgfpathlineto{\pgfqpoint{3.614652in}{1.678696in}}%
\pgfpathlineto{\pgfqpoint{3.627948in}{1.677527in}}%
\pgfpathlineto{\pgfqpoint{3.635848in}{1.687538in}}%
\pgfpathlineto{\pgfqpoint{3.643741in}{1.697596in}}%
\pgfpathlineto{\pgfqpoint{3.651630in}{1.707699in}}%
\pgfpathlineto{\pgfqpoint{3.659512in}{1.717844in}}%
\pgfpathlineto{\pgfqpoint{3.646228in}{1.718657in}}%
\pgfpathlineto{\pgfqpoint{3.632950in}{1.719640in}}%
\pgfpathlineto{\pgfqpoint{3.619678in}{1.720795in}}%
\pgfpathlineto{\pgfqpoint{3.606412in}{1.722121in}}%
\pgfpathlineto{\pgfqpoint{3.598518in}{1.712322in}}%
\pgfpathlineto{\pgfqpoint{3.590617in}{1.702571in}}%
\pgfpathlineto{\pgfqpoint{3.582711in}{1.692872in}}%
\pgfpathlineto{\pgfqpoint{3.574799in}{1.683227in}}%
\pgfpathclose%
\pgfusepath{fill}%
\end{pgfscope}%
\begin{pgfscope}%
\pgfpathrectangle{\pgfqpoint{1.254980in}{0.150000in}}{\pgfqpoint{5.490039in}{5.490039in}}%
\pgfusepath{clip}%
\pgfsetbuttcap%
\pgfsetroundjoin%
\definecolor{currentfill}{rgb}{0.282884,0.135920,0.453427}%
\pgfsetfillcolor{currentfill}%
\pgfsetfillopacity{0.700000}%
\pgfsetlinewidth{0.000000pt}%
\definecolor{currentstroke}{rgb}{0.000000,0.000000,0.000000}%
\pgfsetstrokecolor{currentstroke}%
\pgfsetdash{}{0pt}%
\pgfpathmoveto{\pgfqpoint{3.913323in}{1.855944in}}%
\pgfpathlineto{\pgfqpoint{3.926681in}{1.858215in}}%
\pgfpathlineto{\pgfqpoint{3.940048in}{1.860650in}}%
\pgfpathlineto{\pgfqpoint{3.953425in}{1.863252in}}%
\pgfpathlineto{\pgfqpoint{3.966810in}{1.866017in}}%
\pgfpathlineto{\pgfqpoint{3.974594in}{1.877460in}}%
\pgfpathlineto{\pgfqpoint{3.982373in}{1.888879in}}%
\pgfpathlineto{\pgfqpoint{3.990148in}{1.900271in}}%
\pgfpathlineto{\pgfqpoint{3.997917in}{1.911634in}}%
\pgfpathlineto{\pgfqpoint{3.984537in}{1.908623in}}%
\pgfpathlineto{\pgfqpoint{3.971167in}{1.905776in}}%
\pgfpathlineto{\pgfqpoint{3.957806in}{1.903094in}}%
\pgfpathlineto{\pgfqpoint{3.944454in}{1.900578in}}%
\pgfpathlineto{\pgfqpoint{3.936679in}{1.889450in}}%
\pgfpathlineto{\pgfqpoint{3.928898in}{1.878300in}}%
\pgfpathlineto{\pgfqpoint{3.921113in}{1.867131in}}%
\pgfpathlineto{\pgfqpoint{3.913323in}{1.855944in}}%
\pgfpathclose%
\pgfusepath{fill}%
\end{pgfscope}%
\begin{pgfscope}%
\pgfpathrectangle{\pgfqpoint{1.254980in}{0.150000in}}{\pgfqpoint{5.490039in}{5.490039in}}%
\pgfusepath{clip}%
\pgfsetbuttcap%
\pgfsetroundjoin%
\definecolor{currentfill}{rgb}{0.157851,0.683765,0.501686}%
\pgfsetfillcolor{currentfill}%
\pgfsetfillopacity{0.700000}%
\pgfsetlinewidth{0.000000pt}%
\definecolor{currentstroke}{rgb}{0.000000,0.000000,0.000000}%
\pgfsetstrokecolor{currentstroke}%
\pgfsetdash{}{0pt}%
\pgfpathmoveto{\pgfqpoint{5.591142in}{3.208565in}}%
\pgfpathlineto{\pgfqpoint{5.605372in}{3.220726in}}%
\pgfpathlineto{\pgfqpoint{5.619621in}{3.233045in}}%
\pgfpathlineto{\pgfqpoint{5.633890in}{3.245523in}}%
\pgfpathlineto{\pgfqpoint{5.648179in}{3.258159in}}%
\pgfpathlineto{\pgfqpoint{5.655179in}{3.260221in}}%
\pgfpathlineto{\pgfqpoint{5.662171in}{3.262241in}}%
\pgfpathlineto{\pgfqpoint{5.669155in}{3.264226in}}%
\pgfpathlineto{\pgfqpoint{5.676132in}{3.266180in}}%
\pgfpathlineto{\pgfqpoint{5.661868in}{3.253981in}}%
\pgfpathlineto{\pgfqpoint{5.647624in}{3.241940in}}%
\pgfpathlineto{\pgfqpoint{5.633400in}{3.230056in}}%
\pgfpathlineto{\pgfqpoint{5.619194in}{3.218329in}}%
\pgfpathlineto{\pgfqpoint{5.612192in}{3.215929in}}%
\pgfpathlineto{\pgfqpoint{5.605183in}{3.213505in}}%
\pgfpathlineto{\pgfqpoint{5.598166in}{3.211052in}}%
\pgfpathlineto{\pgfqpoint{5.591142in}{3.208565in}}%
\pgfpathclose%
\pgfusepath{fill}%
\end{pgfscope}%
\begin{pgfscope}%
\pgfpathrectangle{\pgfqpoint{1.254980in}{0.150000in}}{\pgfqpoint{5.490039in}{5.490039in}}%
\pgfusepath{clip}%
\pgfsetbuttcap%
\pgfsetroundjoin%
\definecolor{currentfill}{rgb}{0.204903,0.375746,0.553533}%
\pgfsetfillcolor{currentfill}%
\pgfsetfillopacity{0.700000}%
\pgfsetlinewidth{0.000000pt}%
\definecolor{currentstroke}{rgb}{0.000000,0.000000,0.000000}%
\pgfsetstrokecolor{currentstroke}%
\pgfsetdash{}{0pt}%
\pgfpathmoveto{\pgfqpoint{2.344616in}{2.453977in}}%
\pgfpathlineto{\pgfqpoint{2.358232in}{2.432965in}}%
\pgfpathlineto{\pgfqpoint{2.371836in}{2.412236in}}%
\pgfpathlineto{\pgfqpoint{2.385427in}{2.391786in}}%
\pgfpathlineto{\pgfqpoint{2.399008in}{2.371614in}}%
\pgfpathlineto{\pgfqpoint{2.407709in}{2.369492in}}%
\pgfpathlineto{\pgfqpoint{2.416391in}{2.367651in}}%
\pgfpathlineto{\pgfqpoint{2.425055in}{2.366085in}}%
\pgfpathlineto{\pgfqpoint{2.433700in}{2.364789in}}%
\pgfpathlineto{\pgfqpoint{2.420170in}{2.384428in}}%
\pgfpathlineto{\pgfqpoint{2.406628in}{2.404343in}}%
\pgfpathlineto{\pgfqpoint{2.393074in}{2.424537in}}%
\pgfpathlineto{\pgfqpoint{2.379509in}{2.445012in}}%
\pgfpathlineto{\pgfqpoint{2.370815in}{2.446831in}}%
\pgfpathlineto{\pgfqpoint{2.362101in}{2.448928in}}%
\pgfpathlineto{\pgfqpoint{2.353368in}{2.451308in}}%
\pgfpathlineto{\pgfqpoint{2.344616in}{2.453977in}}%
\pgfpathclose%
\pgfusepath{fill}%
\end{pgfscope}%
\begin{pgfscope}%
\pgfpathrectangle{\pgfqpoint{1.254980in}{0.150000in}}{\pgfqpoint{5.490039in}{5.490039in}}%
\pgfusepath{clip}%
\pgfsetbuttcap%
\pgfsetroundjoin%
\definecolor{currentfill}{rgb}{0.248629,0.278775,0.534556}%
\pgfsetfillcolor{currentfill}%
\pgfsetfillopacity{0.700000}%
\pgfsetlinewidth{0.000000pt}%
\definecolor{currentstroke}{rgb}{0.000000,0.000000,0.000000}%
\pgfsetstrokecolor{currentstroke}%
\pgfsetdash{}{0pt}%
\pgfpathmoveto{\pgfqpoint{4.282734in}{2.145498in}}%
\pgfpathlineto{\pgfqpoint{4.296238in}{2.151340in}}%
\pgfpathlineto{\pgfqpoint{4.309754in}{2.157344in}}%
\pgfpathlineto{\pgfqpoint{4.323283in}{2.163510in}}%
\pgfpathlineto{\pgfqpoint{4.336823in}{2.169838in}}%
\pgfpathlineto{\pgfqpoint{4.344495in}{2.180928in}}%
\pgfpathlineto{\pgfqpoint{4.352161in}{2.191936in}}%
\pgfpathlineto{\pgfqpoint{4.359822in}{2.202861in}}%
\pgfpathlineto{\pgfqpoint{4.367478in}{2.213702in}}%
\pgfpathlineto{\pgfqpoint{4.353940in}{2.207269in}}%
\pgfpathlineto{\pgfqpoint{4.340415in}{2.200997in}}%
\pgfpathlineto{\pgfqpoint{4.326903in}{2.194888in}}%
\pgfpathlineto{\pgfqpoint{4.313402in}{2.188941in}}%
\pgfpathlineto{\pgfqpoint{4.305743in}{2.178195in}}%
\pgfpathlineto{\pgfqpoint{4.298079in}{2.167372in}}%
\pgfpathlineto{\pgfqpoint{4.290409in}{2.156472in}}%
\pgfpathlineto{\pgfqpoint{4.282734in}{2.145498in}}%
\pgfpathclose%
\pgfusepath{fill}%
\end{pgfscope}%
\begin{pgfscope}%
\pgfpathrectangle{\pgfqpoint{1.254980in}{0.150000in}}{\pgfqpoint{5.490039in}{5.490039in}}%
\pgfusepath{clip}%
\pgfsetbuttcap%
\pgfsetroundjoin%
\definecolor{currentfill}{rgb}{0.128729,0.563265,0.551229}%
\pgfsetfillcolor{currentfill}%
\pgfsetfillopacity{0.700000}%
\pgfsetlinewidth{0.000000pt}%
\definecolor{currentstroke}{rgb}{0.000000,0.000000,0.000000}%
\pgfsetstrokecolor{currentstroke}%
\pgfsetdash{}{0pt}%
\pgfpathmoveto{\pgfqpoint{5.137197in}{2.878073in}}%
\pgfpathlineto{\pgfqpoint{5.151163in}{2.889029in}}%
\pgfpathlineto{\pgfqpoint{5.165147in}{2.900145in}}%
\pgfpathlineto{\pgfqpoint{5.179148in}{2.911420in}}%
\pgfpathlineto{\pgfqpoint{5.193167in}{2.922855in}}%
\pgfpathlineto{\pgfqpoint{5.200458in}{2.928376in}}%
\pgfpathlineto{\pgfqpoint{5.207740in}{2.933801in}}%
\pgfpathlineto{\pgfqpoint{5.215015in}{2.939131in}}%
\pgfpathlineto{\pgfqpoint{5.222281in}{2.944371in}}%
\pgfpathlineto{\pgfqpoint{5.208276in}{2.933188in}}%
\pgfpathlineto{\pgfqpoint{5.194288in}{2.922165in}}%
\pgfpathlineto{\pgfqpoint{5.180318in}{2.911301in}}%
\pgfpathlineto{\pgfqpoint{5.166365in}{2.900596in}}%
\pgfpathlineto{\pgfqpoint{5.159085in}{2.895094in}}%
\pgfpathlineto{\pgfqpoint{5.151797in}{2.889508in}}%
\pgfpathlineto{\pgfqpoint{5.144501in}{2.883835in}}%
\pgfpathlineto{\pgfqpoint{5.137197in}{2.878073in}}%
\pgfpathclose%
\pgfusepath{fill}%
\end{pgfscope}%
\begin{pgfscope}%
\pgfpathrectangle{\pgfqpoint{1.254980in}{0.150000in}}{\pgfqpoint{5.490039in}{5.490039in}}%
\pgfusepath{clip}%
\pgfsetbuttcap%
\pgfsetroundjoin%
\definecolor{currentfill}{rgb}{0.281924,0.089666,0.412415}%
\pgfsetfillcolor{currentfill}%
\pgfsetfillopacity{0.700000}%
\pgfsetlinewidth{0.000000pt}%
\definecolor{currentstroke}{rgb}{0.000000,0.000000,0.000000}%
\pgfsetstrokecolor{currentstroke}%
\pgfsetdash{}{0pt}%
\pgfpathmoveto{\pgfqpoint{2.882354in}{1.811613in}}%
\pgfpathlineto{\pgfqpoint{2.895682in}{1.800244in}}%
\pgfpathlineto{\pgfqpoint{2.909007in}{1.789083in}}%
\pgfpathlineto{\pgfqpoint{2.922329in}{1.778128in}}%
\pgfpathlineto{\pgfqpoint{2.935649in}{1.767377in}}%
\pgfpathlineto{\pgfqpoint{2.943937in}{1.770527in}}%
\pgfpathlineto{\pgfqpoint{2.952211in}{1.773880in}}%
\pgfpathlineto{\pgfqpoint{2.960474in}{1.777430in}}%
\pgfpathlineto{\pgfqpoint{2.968725in}{1.781173in}}%
\pgfpathlineto{\pgfqpoint{2.955437in}{1.791421in}}%
\pgfpathlineto{\pgfqpoint{2.942147in}{1.801872in}}%
\pgfpathlineto{\pgfqpoint{2.928855in}{1.812530in}}%
\pgfpathlineto{\pgfqpoint{2.915561in}{1.823393in}}%
\pgfpathlineto{\pgfqpoint{2.907279in}{1.820143in}}%
\pgfpathlineto{\pgfqpoint{2.898983in}{1.817093in}}%
\pgfpathlineto{\pgfqpoint{2.890675in}{1.814248in}}%
\pgfpathlineto{\pgfqpoint{2.882354in}{1.811613in}}%
\pgfpathclose%
\pgfusepath{fill}%
\end{pgfscope}%
\begin{pgfscope}%
\pgfpathrectangle{\pgfqpoint{1.254980in}{0.150000in}}{\pgfqpoint{5.490039in}{5.490039in}}%
\pgfusepath{clip}%
\pgfsetbuttcap%
\pgfsetroundjoin%
\definecolor{currentfill}{rgb}{0.269944,0.014625,0.341379}%
\pgfsetfillcolor{currentfill}%
\pgfsetfillopacity{0.700000}%
\pgfsetlinewidth{0.000000pt}%
\definecolor{currentstroke}{rgb}{0.000000,0.000000,0.000000}%
\pgfsetstrokecolor{currentstroke}%
\pgfsetdash{}{0pt}%
\pgfpathmoveto{\pgfqpoint{3.489972in}{1.655256in}}%
\pgfpathlineto{\pgfqpoint{3.503244in}{1.652499in}}%
\pgfpathlineto{\pgfqpoint{3.516521in}{1.649916in}}%
\pgfpathlineto{\pgfqpoint{3.529803in}{1.647506in}}%
\pgfpathlineto{\pgfqpoint{3.543089in}{1.645269in}}%
\pgfpathlineto{\pgfqpoint{3.551026in}{1.654659in}}%
\pgfpathlineto{\pgfqpoint{3.558957in}{1.664117in}}%
\pgfpathlineto{\pgfqpoint{3.566881in}{1.673641in}}%
\pgfpathlineto{\pgfqpoint{3.574799in}{1.683227in}}%
\pgfpathlineto{\pgfqpoint{3.561526in}{1.685081in}}%
\pgfpathlineto{\pgfqpoint{3.548259in}{1.687108in}}%
\pgfpathlineto{\pgfqpoint{3.534997in}{1.689307in}}%
\pgfpathlineto{\pgfqpoint{3.521739in}{1.691681in}}%
\pgfpathlineto{\pgfqpoint{3.513807in}{1.682468in}}%
\pgfpathlineto{\pgfqpoint{3.505869in}{1.673324in}}%
\pgfpathlineto{\pgfqpoint{3.497924in}{1.664252in}}%
\pgfpathlineto{\pgfqpoint{3.489972in}{1.655256in}}%
\pgfpathclose%
\pgfusepath{fill}%
\end{pgfscope}%
\begin{pgfscope}%
\pgfpathrectangle{\pgfqpoint{1.254980in}{0.150000in}}{\pgfqpoint{5.490039in}{5.490039in}}%
\pgfusepath{clip}%
\pgfsetbuttcap%
\pgfsetroundjoin%
\definecolor{currentfill}{rgb}{0.280255,0.165693,0.476498}%
\pgfsetfillcolor{currentfill}%
\pgfsetfillopacity{0.700000}%
\pgfsetlinewidth{0.000000pt}%
\definecolor{currentstroke}{rgb}{0.000000,0.000000,0.000000}%
\pgfsetstrokecolor{currentstroke}%
\pgfsetdash{}{0pt}%
\pgfpathmoveto{\pgfqpoint{3.997917in}{1.911634in}}%
\pgfpathlineto{\pgfqpoint{4.011307in}{1.914811in}}%
\pgfpathlineto{\pgfqpoint{4.024706in}{1.918152in}}%
\pgfpathlineto{\pgfqpoint{4.038115in}{1.921657in}}%
\pgfpathlineto{\pgfqpoint{4.051535in}{1.925326in}}%
\pgfpathlineto{\pgfqpoint{4.059295in}{1.936887in}}%
\pgfpathlineto{\pgfqpoint{4.067050in}{1.948409in}}%
\pgfpathlineto{\pgfqpoint{4.074801in}{1.959889in}}%
\pgfpathlineto{\pgfqpoint{4.082547in}{1.971326in}}%
\pgfpathlineto{\pgfqpoint{4.069132in}{1.967439in}}%
\pgfpathlineto{\pgfqpoint{4.055727in}{1.963716in}}%
\pgfpathlineto{\pgfqpoint{4.042333in}{1.960157in}}%
\pgfpathlineto{\pgfqpoint{4.028949in}{1.956762in}}%
\pgfpathlineto{\pgfqpoint{4.021198in}{1.945533in}}%
\pgfpathlineto{\pgfqpoint{4.013442in}{1.934267in}}%
\pgfpathlineto{\pgfqpoint{4.005682in}{1.922967in}}%
\pgfpathlineto{\pgfqpoint{3.997917in}{1.911634in}}%
\pgfpathclose%
\pgfusepath{fill}%
\end{pgfscope}%
\begin{pgfscope}%
\pgfpathrectangle{\pgfqpoint{1.254980in}{0.150000in}}{\pgfqpoint{5.490039in}{5.490039in}}%
\pgfusepath{clip}%
\pgfsetbuttcap%
\pgfsetroundjoin%
\definecolor{currentfill}{rgb}{0.185783,0.704891,0.485273}%
\pgfsetfillcolor{currentfill}%
\pgfsetfillopacity{0.700000}%
\pgfsetlinewidth{0.000000pt}%
\definecolor{currentstroke}{rgb}{0.000000,0.000000,0.000000}%
\pgfsetstrokecolor{currentstroke}%
\pgfsetdash{}{0pt}%
\pgfpathmoveto{\pgfqpoint{5.676132in}{3.266180in}}%
\pgfpathlineto{\pgfqpoint{5.690416in}{3.278536in}}%
\pgfpathlineto{\pgfqpoint{5.704719in}{3.291051in}}%
\pgfpathlineto{\pgfqpoint{5.719043in}{3.303723in}}%
\pgfpathlineto{\pgfqpoint{5.733387in}{3.316554in}}%
\pgfpathlineto{\pgfqpoint{5.740329in}{3.318024in}}%
\pgfpathlineto{\pgfqpoint{5.747264in}{3.319467in}}%
\pgfpathlineto{\pgfqpoint{5.754192in}{3.320886in}}%
\pgfpathlineto{\pgfqpoint{5.761112in}{3.322288in}}%
\pgfpathlineto{\pgfqpoint{5.746796in}{3.309925in}}%
\pgfpathlineto{\pgfqpoint{5.732499in}{3.297721in}}%
\pgfpathlineto{\pgfqpoint{5.718223in}{3.285673in}}%
\pgfpathlineto{\pgfqpoint{5.703966in}{3.273782in}}%
\pgfpathlineto{\pgfqpoint{5.697018in}{3.271903in}}%
\pgfpathlineto{\pgfqpoint{5.690063in}{3.270013in}}%
\pgfpathlineto{\pgfqpoint{5.683101in}{3.268107in}}%
\pgfpathlineto{\pgfqpoint{5.676132in}{3.266180in}}%
\pgfpathclose%
\pgfusepath{fill}%
\end{pgfscope}%
\begin{pgfscope}%
\pgfpathrectangle{\pgfqpoint{1.254980in}{0.150000in}}{\pgfqpoint{5.490039in}{5.490039in}}%
\pgfusepath{clip}%
\pgfsetbuttcap%
\pgfsetroundjoin%
\definecolor{currentfill}{rgb}{0.201239,0.383670,0.554294}%
\pgfsetfillcolor{currentfill}%
\pgfsetfillopacity{0.700000}%
\pgfsetlinewidth{0.000000pt}%
\definecolor{currentstroke}{rgb}{0.000000,0.000000,0.000000}%
\pgfsetstrokecolor{currentstroke}%
\pgfsetdash{}{0pt}%
\pgfpathmoveto{\pgfqpoint{4.567603in}{2.395579in}}%
\pgfpathlineto{\pgfqpoint{4.581250in}{2.403610in}}%
\pgfpathlineto{\pgfqpoint{4.594910in}{2.411802in}}%
\pgfpathlineto{\pgfqpoint{4.608586in}{2.420156in}}%
\pgfpathlineto{\pgfqpoint{4.622275in}{2.428670in}}%
\pgfpathlineto{\pgfqpoint{4.629846in}{2.438401in}}%
\pgfpathlineto{\pgfqpoint{4.637411in}{2.448026in}}%
\pgfpathlineto{\pgfqpoint{4.644969in}{2.457544in}}%
\pgfpathlineto{\pgfqpoint{4.652521in}{2.466958in}}%
\pgfpathlineto{\pgfqpoint{4.638836in}{2.458454in}}%
\pgfpathlineto{\pgfqpoint{4.625165in}{2.450112in}}%
\pgfpathlineto{\pgfqpoint{4.611509in}{2.441931in}}%
\pgfpathlineto{\pgfqpoint{4.597867in}{2.433910in}}%
\pgfpathlineto{\pgfqpoint{4.590310in}{2.424475in}}%
\pgfpathlineto{\pgfqpoint{4.582747in}{2.414942in}}%
\pgfpathlineto{\pgfqpoint{4.575178in}{2.405310in}}%
\pgfpathlineto{\pgfqpoint{4.567603in}{2.395579in}}%
\pgfpathclose%
\pgfusepath{fill}%
\end{pgfscope}%
\begin{pgfscope}%
\pgfpathrectangle{\pgfqpoint{1.254980in}{0.150000in}}{\pgfqpoint{5.490039in}{5.490039in}}%
\pgfusepath{clip}%
\pgfsetbuttcap%
\pgfsetroundjoin%
\definecolor{currentfill}{rgb}{0.268510,0.009605,0.335427}%
\pgfsetfillcolor{currentfill}%
\pgfsetfillopacity{0.700000}%
\pgfsetlinewidth{0.000000pt}%
\definecolor{currentstroke}{rgb}{0.000000,0.000000,0.000000}%
\pgfsetstrokecolor{currentstroke}%
\pgfsetdash{}{0pt}%
\pgfpathmoveto{\pgfqpoint{3.266722in}{1.643557in}}%
\pgfpathlineto{\pgfqpoint{3.279989in}{1.637814in}}%
\pgfpathlineto{\pgfqpoint{3.293258in}{1.632253in}}%
\pgfpathlineto{\pgfqpoint{3.306530in}{1.626873in}}%
\pgfpathlineto{\pgfqpoint{3.319804in}{1.621673in}}%
\pgfpathlineto{\pgfqpoint{3.327850in}{1.629022in}}%
\pgfpathlineto{\pgfqpoint{3.335888in}{1.636491in}}%
\pgfpathlineto{\pgfqpoint{3.343918in}{1.644079in}}%
\pgfpathlineto{\pgfqpoint{3.351940in}{1.651780in}}%
\pgfpathlineto{\pgfqpoint{3.338686in}{1.656540in}}%
\pgfpathlineto{\pgfqpoint{3.325435in}{1.661480in}}%
\pgfpathlineto{\pgfqpoint{3.312187in}{1.666601in}}%
\pgfpathlineto{\pgfqpoint{3.298941in}{1.671904in}}%
\pgfpathlineto{\pgfqpoint{3.290899in}{1.664633in}}%
\pgfpathlineto{\pgfqpoint{3.282848in}{1.657481in}}%
\pgfpathlineto{\pgfqpoint{3.274790in}{1.650455in}}%
\pgfpathlineto{\pgfqpoint{3.266722in}{1.643557in}}%
\pgfpathclose%
\pgfusepath{fill}%
\end{pgfscope}%
\begin{pgfscope}%
\pgfpathrectangle{\pgfqpoint{1.254980in}{0.150000in}}{\pgfqpoint{5.490039in}{5.490039in}}%
\pgfusepath{clip}%
\pgfsetbuttcap%
\pgfsetroundjoin%
\definecolor{currentfill}{rgb}{0.160665,0.478540,0.558115}%
\pgfsetfillcolor{currentfill}%
\pgfsetfillopacity{0.700000}%
\pgfsetlinewidth{0.000000pt}%
\definecolor{currentstroke}{rgb}{0.000000,0.000000,0.000000}%
\pgfsetstrokecolor{currentstroke}%
\pgfsetdash{}{0pt}%
\pgfpathmoveto{\pgfqpoint{4.852523in}{2.644141in}}%
\pgfpathlineto{\pgfqpoint{4.866328in}{2.653879in}}%
\pgfpathlineto{\pgfqpoint{4.880148in}{2.663779in}}%
\pgfpathlineto{\pgfqpoint{4.893985in}{2.673838in}}%
\pgfpathlineto{\pgfqpoint{4.907838in}{2.684059in}}%
\pgfpathlineto{\pgfqpoint{4.915284in}{2.691833in}}%
\pgfpathlineto{\pgfqpoint{4.922722in}{2.699495in}}%
\pgfpathlineto{\pgfqpoint{4.930153in}{2.707048in}}%
\pgfpathlineto{\pgfqpoint{4.937576in}{2.714493in}}%
\pgfpathlineto{\pgfqpoint{4.923731in}{2.704403in}}%
\pgfpathlineto{\pgfqpoint{4.909902in}{2.694474in}}%
\pgfpathlineto{\pgfqpoint{4.896089in}{2.684705in}}%
\pgfpathlineto{\pgfqpoint{4.882293in}{2.675096in}}%
\pgfpathlineto{\pgfqpoint{4.874861in}{2.667510in}}%
\pgfpathlineto{\pgfqpoint{4.867422in}{2.659823in}}%
\pgfpathlineto{\pgfqpoint{4.859976in}{2.652034in}}%
\pgfpathlineto{\pgfqpoint{4.852523in}{2.644141in}}%
\pgfpathclose%
\pgfusepath{fill}%
\end{pgfscope}%
\begin{pgfscope}%
\pgfpathrectangle{\pgfqpoint{1.254980in}{0.150000in}}{\pgfqpoint{5.490039in}{5.490039in}}%
\pgfusepath{clip}%
\pgfsetbuttcap%
\pgfsetroundjoin%
\definecolor{currentfill}{rgb}{0.220124,0.725509,0.466226}%
\pgfsetfillcolor{currentfill}%
\pgfsetfillopacity{0.700000}%
\pgfsetlinewidth{0.000000pt}%
\definecolor{currentstroke}{rgb}{0.000000,0.000000,0.000000}%
\pgfsetstrokecolor{currentstroke}%
\pgfsetdash{}{0pt}%
\pgfpathmoveto{\pgfqpoint{5.761112in}{3.322288in}}%
\pgfpathlineto{\pgfqpoint{5.775449in}{3.334807in}}%
\pgfpathlineto{\pgfqpoint{5.789806in}{3.347485in}}%
\pgfpathlineto{\pgfqpoint{5.804183in}{3.360320in}}%
\pgfpathlineto{\pgfqpoint{5.818581in}{3.373314in}}%
\pgfpathlineto{\pgfqpoint{5.825465in}{3.374214in}}%
\pgfpathlineto{\pgfqpoint{5.832342in}{3.375101in}}%
\pgfpathlineto{\pgfqpoint{5.839212in}{3.375978in}}%
\pgfpathlineto{\pgfqpoint{5.846074in}{3.376852in}}%
\pgfpathlineto{\pgfqpoint{5.831706in}{3.364359in}}%
\pgfpathlineto{\pgfqpoint{5.817358in}{3.352022in}}%
\pgfpathlineto{\pgfqpoint{5.803031in}{3.339843in}}%
\pgfpathlineto{\pgfqpoint{5.788723in}{3.327820in}}%
\pgfpathlineto{\pgfqpoint{5.781831in}{3.326437in}}%
\pgfpathlineto{\pgfqpoint{5.774931in}{3.325058in}}%
\pgfpathlineto{\pgfqpoint{5.768025in}{3.323676in}}%
\pgfpathlineto{\pgfqpoint{5.761112in}{3.322288in}}%
\pgfpathclose%
\pgfusepath{fill}%
\end{pgfscope}%
\begin{pgfscope}%
\pgfpathrectangle{\pgfqpoint{1.254980in}{0.150000in}}{\pgfqpoint{5.490039in}{5.490039in}}%
\pgfusepath{clip}%
\pgfsetbuttcap%
\pgfsetroundjoin%
\definecolor{currentfill}{rgb}{0.288921,0.758394,0.428426}%
\pgfsetfillcolor{currentfill}%
\pgfsetfillopacity{0.700000}%
\pgfsetlinewidth{0.000000pt}%
\definecolor{currentstroke}{rgb}{0.000000,0.000000,0.000000}%
\pgfsetstrokecolor{currentstroke}%
\pgfsetdash{}{0pt}%
\pgfpathmoveto{\pgfqpoint{5.931010in}{3.429858in}}%
\pgfpathlineto{\pgfqpoint{5.945449in}{3.442608in}}%
\pgfpathlineto{\pgfqpoint{5.959910in}{3.455515in}}%
\pgfpathlineto{\pgfqpoint{5.974391in}{3.468580in}}%
\pgfpathlineto{\pgfqpoint{5.981164in}{3.468563in}}%
\pgfpathlineto{\pgfqpoint{5.987930in}{3.468564in}}%
\pgfpathlineto{\pgfqpoint{5.994690in}{3.468591in}}%
\pgfpathlineto{\pgfqpoint{6.001444in}{3.468649in}}%
\pgfpathlineto{\pgfqpoint{5.986998in}{3.456145in}}%
\pgfpathlineto{\pgfqpoint{5.972572in}{3.443798in}}%
\pgfpathlineto{\pgfqpoint{5.958166in}{3.431606in}}%
\pgfpathlineto{\pgfqpoint{5.951386in}{3.431121in}}%
\pgfpathlineto{\pgfqpoint{5.944600in}{3.430672in}}%
\pgfpathlineto{\pgfqpoint{5.937808in}{3.430253in}}%
\pgfpathlineto{\pgfqpoint{5.931010in}{3.429858in}}%
\pgfpathclose%
\pgfusepath{fill}%
\end{pgfscope}%
\begin{pgfscope}%
\pgfpathrectangle{\pgfqpoint{1.254980in}{0.150000in}}{\pgfqpoint{5.490039in}{5.490039in}}%
\pgfusepath{clip}%
\pgfsetbuttcap%
\pgfsetroundjoin%
\definecolor{currentfill}{rgb}{0.188923,0.410910,0.556326}%
\pgfsetfillcolor{currentfill}%
\pgfsetfillopacity{0.700000}%
\pgfsetlinewidth{0.000000pt}%
\definecolor{currentstroke}{rgb}{0.000000,0.000000,0.000000}%
\pgfsetstrokecolor{currentstroke}%
\pgfsetdash{}{0pt}%
\pgfpathmoveto{\pgfqpoint{2.290023in}{2.540902in}}%
\pgfpathlineto{\pgfqpoint{2.303691in}{2.518733in}}%
\pgfpathlineto{\pgfqpoint{2.317346in}{2.496858in}}%
\pgfpathlineto{\pgfqpoint{2.330987in}{2.475274in}}%
\pgfpathlineto{\pgfqpoint{2.344616in}{2.453977in}}%
\pgfpathlineto{\pgfqpoint{2.353368in}{2.451308in}}%
\pgfpathlineto{\pgfqpoint{2.362101in}{2.448928in}}%
\pgfpathlineto{\pgfqpoint{2.370815in}{2.446831in}}%
\pgfpathlineto{\pgfqpoint{2.379509in}{2.445012in}}%
\pgfpathlineto{\pgfqpoint{2.365932in}{2.465771in}}%
\pgfpathlineto{\pgfqpoint{2.352343in}{2.486816in}}%
\pgfpathlineto{\pgfqpoint{2.338741in}{2.508151in}}%
\pgfpathlineto{\pgfqpoint{2.325125in}{2.529777in}}%
\pgfpathlineto{\pgfqpoint{2.316380in}{2.532124in}}%
\pgfpathlineto{\pgfqpoint{2.307614in}{2.534756in}}%
\pgfpathlineto{\pgfqpoint{2.298829in}{2.537680in}}%
\pgfpathlineto{\pgfqpoint{2.290023in}{2.540902in}}%
\pgfpathclose%
\pgfusepath{fill}%
\end{pgfscope}%
\begin{pgfscope}%
\pgfpathrectangle{\pgfqpoint{1.254980in}{0.150000in}}{\pgfqpoint{5.490039in}{5.490039in}}%
\pgfusepath{clip}%
\pgfsetbuttcap%
\pgfsetroundjoin%
\definecolor{currentfill}{rgb}{0.271305,0.019942,0.347269}%
\pgfsetfillcolor{currentfill}%
\pgfsetfillopacity{0.700000}%
\pgfsetlinewidth{0.000000pt}%
\definecolor{currentstroke}{rgb}{0.000000,0.000000,0.000000}%
\pgfsetstrokecolor{currentstroke}%
\pgfsetdash{}{0pt}%
\pgfpathmoveto{\pgfqpoint{3.128098in}{1.673690in}}%
\pgfpathlineto{\pgfqpoint{3.141378in}{1.665992in}}%
\pgfpathlineto{\pgfqpoint{3.154659in}{1.658482in}}%
\pgfpathlineto{\pgfqpoint{3.167940in}{1.651161in}}%
\pgfpathlineto{\pgfqpoint{3.181222in}{1.644027in}}%
\pgfpathlineto{\pgfqpoint{3.189350in}{1.649885in}}%
\pgfpathlineto{\pgfqpoint{3.197467in}{1.655897in}}%
\pgfpathlineto{\pgfqpoint{3.205575in}{1.662059in}}%
\pgfpathlineto{\pgfqpoint{3.213674in}{1.668365in}}%
\pgfpathlineto{\pgfqpoint{3.200417in}{1.675030in}}%
\pgfpathlineto{\pgfqpoint{3.187160in}{1.681882in}}%
\pgfpathlineto{\pgfqpoint{3.173905in}{1.688922in}}%
\pgfpathlineto{\pgfqpoint{3.160650in}{1.696150in}}%
\pgfpathlineto{\pgfqpoint{3.152527in}{1.690303in}}%
\pgfpathlineto{\pgfqpoint{3.144394in}{1.684607in}}%
\pgfpathlineto{\pgfqpoint{3.136251in}{1.679068in}}%
\pgfpathlineto{\pgfqpoint{3.128098in}{1.673690in}}%
\pgfpathclose%
\pgfusepath{fill}%
\end{pgfscope}%
\begin{pgfscope}%
\pgfpathrectangle{\pgfqpoint{1.254980in}{0.150000in}}{\pgfqpoint{5.490039in}{5.490039in}}%
\pgfusepath{clip}%
\pgfsetbuttcap%
\pgfsetroundjoin%
\definecolor{currentfill}{rgb}{0.274128,0.199721,0.498911}%
\pgfsetfillcolor{currentfill}%
\pgfsetfillopacity{0.700000}%
\pgfsetlinewidth{0.000000pt}%
\definecolor{currentstroke}{rgb}{0.000000,0.000000,0.000000}%
\pgfsetstrokecolor{currentstroke}%
\pgfsetdash{}{0pt}%
\pgfpathmoveto{\pgfqpoint{4.082547in}{1.971326in}}%
\pgfpathlineto{\pgfqpoint{4.095972in}{1.975377in}}%
\pgfpathlineto{\pgfqpoint{4.109407in}{1.979592in}}%
\pgfpathlineto{\pgfqpoint{4.122853in}{1.983970in}}%
\pgfpathlineto{\pgfqpoint{4.136311in}{1.988511in}}%
\pgfpathlineto{\pgfqpoint{4.144048in}{2.000104in}}%
\pgfpathlineto{\pgfqpoint{4.151780in}{2.011643in}}%
\pgfpathlineto{\pgfqpoint{4.159508in}{2.023126in}}%
\pgfpathlineto{\pgfqpoint{4.167230in}{2.034553in}}%
\pgfpathlineto{\pgfqpoint{4.153777in}{2.029821in}}%
\pgfpathlineto{\pgfqpoint{4.140335in}{2.025253in}}%
\pgfpathlineto{\pgfqpoint{4.126903in}{2.020848in}}%
\pgfpathlineto{\pgfqpoint{4.113483in}{2.016607in}}%
\pgfpathlineto{\pgfqpoint{4.105756in}{2.005360in}}%
\pgfpathlineto{\pgfqpoint{4.098024in}{1.994063in}}%
\pgfpathlineto{\pgfqpoint{4.090288in}{1.982718in}}%
\pgfpathlineto{\pgfqpoint{4.082547in}{1.971326in}}%
\pgfpathclose%
\pgfusepath{fill}%
\end{pgfscope}%
\begin{pgfscope}%
\pgfpathrectangle{\pgfqpoint{1.254980in}{0.150000in}}{\pgfqpoint{5.490039in}{5.490039in}}%
\pgfusepath{clip}%
\pgfsetbuttcap%
\pgfsetroundjoin%
\definecolor{currentfill}{rgb}{0.121148,0.592739,0.544641}%
\pgfsetfillcolor{currentfill}%
\pgfsetfillopacity{0.700000}%
\pgfsetlinewidth{0.000000pt}%
\definecolor{currentstroke}{rgb}{0.000000,0.000000,0.000000}%
\pgfsetstrokecolor{currentstroke}%
\pgfsetdash{}{0pt}%
\pgfpathmoveto{\pgfqpoint{5.222281in}{2.944371in}}%
\pgfpathlineto{\pgfqpoint{5.236304in}{2.955713in}}%
\pgfpathlineto{\pgfqpoint{5.250346in}{2.967214in}}%
\pgfpathlineto{\pgfqpoint{5.264405in}{2.978875in}}%
\pgfpathlineto{\pgfqpoint{5.278483in}{2.990696in}}%
\pgfpathlineto{\pgfqpoint{5.285727in}{2.995576in}}%
\pgfpathlineto{\pgfqpoint{5.292963in}{3.000364in}}%
\pgfpathlineto{\pgfqpoint{5.300190in}{3.005064in}}%
\pgfpathlineto{\pgfqpoint{5.307409in}{3.009678in}}%
\pgfpathlineto{\pgfqpoint{5.293346in}{2.998141in}}%
\pgfpathlineto{\pgfqpoint{5.279302in}{2.986763in}}%
\pgfpathlineto{\pgfqpoint{5.265276in}{2.975545in}}%
\pgfpathlineto{\pgfqpoint{5.251268in}{2.964485in}}%
\pgfpathlineto{\pgfqpoint{5.244033in}{2.959577in}}%
\pgfpathlineto{\pgfqpoint{5.236790in}{2.954590in}}%
\pgfpathlineto{\pgfqpoint{5.229539in}{2.949523in}}%
\pgfpathlineto{\pgfqpoint{5.222281in}{2.944371in}}%
\pgfpathclose%
\pgfusepath{fill}%
\end{pgfscope}%
\begin{pgfscope}%
\pgfpathrectangle{\pgfqpoint{1.254980in}{0.150000in}}{\pgfqpoint{5.490039in}{5.490039in}}%
\pgfusepath{clip}%
\pgfsetbuttcap%
\pgfsetroundjoin%
\definecolor{currentfill}{rgb}{0.259857,0.745492,0.444467}%
\pgfsetfillcolor{currentfill}%
\pgfsetfillopacity{0.700000}%
\pgfsetlinewidth{0.000000pt}%
\definecolor{currentstroke}{rgb}{0.000000,0.000000,0.000000}%
\pgfsetstrokecolor{currentstroke}%
\pgfsetdash{}{0pt}%
\pgfpathmoveto{\pgfqpoint{5.846074in}{3.376852in}}%
\pgfpathlineto{\pgfqpoint{5.860463in}{3.389503in}}%
\pgfpathlineto{\pgfqpoint{5.874872in}{3.402312in}}%
\pgfpathlineto{\pgfqpoint{5.889302in}{3.415278in}}%
\pgfpathlineto{\pgfqpoint{5.903753in}{3.428402in}}%
\pgfpathlineto{\pgfqpoint{5.910578in}{3.428759in}}%
\pgfpathlineto{\pgfqpoint{5.917395in}{3.429117in}}%
\pgfpathlineto{\pgfqpoint{5.924206in}{3.429481in}}%
\pgfpathlineto{\pgfqpoint{5.931010in}{3.429858in}}%
\pgfpathlineto{\pgfqpoint{5.916591in}{3.417265in}}%
\pgfpathlineto{\pgfqpoint{5.902193in}{3.404829in}}%
\pgfpathlineto{\pgfqpoint{5.887816in}{3.392549in}}%
\pgfpathlineto{\pgfqpoint{5.873459in}{3.380426in}}%
\pgfpathlineto{\pgfqpoint{5.866622in}{3.379510in}}%
\pgfpathlineto{\pgfqpoint{5.859779in}{3.378613in}}%
\pgfpathlineto{\pgfqpoint{5.852930in}{3.377729in}}%
\pgfpathlineto{\pgfqpoint{5.846074in}{3.376852in}}%
\pgfpathclose%
\pgfusepath{fill}%
\end{pgfscope}%
\begin{pgfscope}%
\pgfpathrectangle{\pgfqpoint{1.254980in}{0.150000in}}{\pgfqpoint{5.490039in}{5.490039in}}%
\pgfusepath{clip}%
\pgfsetbuttcap%
\pgfsetroundjoin%
\definecolor{currentfill}{rgb}{0.233603,0.313828,0.543914}%
\pgfsetfillcolor{currentfill}%
\pgfsetfillopacity{0.700000}%
\pgfsetlinewidth{0.000000pt}%
\definecolor{currentstroke}{rgb}{0.000000,0.000000,0.000000}%
\pgfsetstrokecolor{currentstroke}%
\pgfsetdash{}{0pt}%
\pgfpathmoveto{\pgfqpoint{4.367478in}{2.213702in}}%
\pgfpathlineto{\pgfqpoint{4.381028in}{2.220298in}}%
\pgfpathlineto{\pgfqpoint{4.394591in}{2.227056in}}%
\pgfpathlineto{\pgfqpoint{4.408167in}{2.233975in}}%
\pgfpathlineto{\pgfqpoint{4.421756in}{2.241056in}}%
\pgfpathlineto{\pgfqpoint{4.429403in}{2.251900in}}%
\pgfpathlineto{\pgfqpoint{4.437045in}{2.262653in}}%
\pgfpathlineto{\pgfqpoint{4.444681in}{2.273313in}}%
\pgfpathlineto{\pgfqpoint{4.452312in}{2.283881in}}%
\pgfpathlineto{\pgfqpoint{4.438726in}{2.276723in}}%
\pgfpathlineto{\pgfqpoint{4.425154in}{2.269727in}}%
\pgfpathlineto{\pgfqpoint{4.411594in}{2.262893in}}%
\pgfpathlineto{\pgfqpoint{4.398047in}{2.256221in}}%
\pgfpathlineto{\pgfqpoint{4.390413in}{2.245719in}}%
\pgfpathlineto{\pgfqpoint{4.382773in}{2.235132in}}%
\pgfpathlineto{\pgfqpoint{4.375128in}{2.224460in}}%
\pgfpathlineto{\pgfqpoint{4.367478in}{2.213702in}}%
\pgfpathclose%
\pgfusepath{fill}%
\end{pgfscope}%
\begin{pgfscope}%
\pgfpathrectangle{\pgfqpoint{1.254980in}{0.150000in}}{\pgfqpoint{5.490039in}{5.490039in}}%
\pgfusepath{clip}%
\pgfsetbuttcap%
\pgfsetroundjoin%
\definecolor{currentfill}{rgb}{0.280267,0.073417,0.397163}%
\pgfsetfillcolor{currentfill}%
\pgfsetfillopacity{0.700000}%
\pgfsetlinewidth{0.000000pt}%
\definecolor{currentstroke}{rgb}{0.000000,0.000000,0.000000}%
\pgfsetstrokecolor{currentstroke}%
\pgfsetdash{}{0pt}%
\pgfpathmoveto{\pgfqpoint{2.935649in}{1.767377in}}%
\pgfpathlineto{\pgfqpoint{2.948968in}{1.756829in}}%
\pgfpathlineto{\pgfqpoint{2.962284in}{1.746483in}}%
\pgfpathlineto{\pgfqpoint{2.975598in}{1.736337in}}%
\pgfpathlineto{\pgfqpoint{2.988911in}{1.726392in}}%
\pgfpathlineto{\pgfqpoint{2.997166in}{1.730056in}}%
\pgfpathlineto{\pgfqpoint{3.005409in}{1.733914in}}%
\pgfpathlineto{\pgfqpoint{3.013641in}{1.737963in}}%
\pgfpathlineto{\pgfqpoint{3.021861in}{1.742197in}}%
\pgfpathlineto{\pgfqpoint{3.008578in}{1.751641in}}%
\pgfpathlineto{\pgfqpoint{2.995295in}{1.761284in}}%
\pgfpathlineto{\pgfqpoint{2.982011in}{1.771128in}}%
\pgfpathlineto{\pgfqpoint{2.968725in}{1.781173in}}%
\pgfpathlineto{\pgfqpoint{2.960474in}{1.777430in}}%
\pgfpathlineto{\pgfqpoint{2.952211in}{1.773880in}}%
\pgfpathlineto{\pgfqpoint{2.943937in}{1.770527in}}%
\pgfpathlineto{\pgfqpoint{2.935649in}{1.767377in}}%
\pgfpathclose%
\pgfusepath{fill}%
\end{pgfscope}%
\begin{pgfscope}%
\pgfpathrectangle{\pgfqpoint{1.254980in}{0.150000in}}{\pgfqpoint{5.490039in}{5.490039in}}%
\pgfusepath{clip}%
\pgfsetbuttcap%
\pgfsetroundjoin%
\definecolor{currentfill}{rgb}{0.268510,0.009605,0.335427}%
\pgfsetfillcolor{currentfill}%
\pgfsetfillopacity{0.700000}%
\pgfsetlinewidth{0.000000pt}%
\definecolor{currentstroke}{rgb}{0.000000,0.000000,0.000000}%
\pgfsetstrokecolor{currentstroke}%
\pgfsetdash{}{0pt}%
\pgfpathmoveto{\pgfqpoint{3.404990in}{1.634529in}}%
\pgfpathlineto{\pgfqpoint{3.418261in}{1.630661in}}%
\pgfpathlineto{\pgfqpoint{3.431536in}{1.626968in}}%
\pgfpathlineto{\pgfqpoint{3.444815in}{1.623451in}}%
\pgfpathlineto{\pgfqpoint{3.458099in}{1.620108in}}%
\pgfpathlineto{\pgfqpoint{3.466077in}{1.628762in}}%
\pgfpathlineto{\pgfqpoint{3.474049in}{1.637507in}}%
\pgfpathlineto{\pgfqpoint{3.482014in}{1.646340in}}%
\pgfpathlineto{\pgfqpoint{3.489972in}{1.655256in}}%
\pgfpathlineto{\pgfqpoint{3.476705in}{1.658187in}}%
\pgfpathlineto{\pgfqpoint{3.463443in}{1.661293in}}%
\pgfpathlineto{\pgfqpoint{3.450184in}{1.664575in}}%
\pgfpathlineto{\pgfqpoint{3.436930in}{1.668032in}}%
\pgfpathlineto{\pgfqpoint{3.428956in}{1.659517in}}%
\pgfpathlineto{\pgfqpoint{3.420974in}{1.651092in}}%
\pgfpathlineto{\pgfqpoint{3.412986in}{1.642762in}}%
\pgfpathlineto{\pgfqpoint{3.404990in}{1.634529in}}%
\pgfpathclose%
\pgfusepath{fill}%
\end{pgfscope}%
\begin{pgfscope}%
\pgfpathrectangle{\pgfqpoint{1.254980in}{0.150000in}}{\pgfqpoint{5.490039in}{5.490039in}}%
\pgfusepath{clip}%
\pgfsetbuttcap%
\pgfsetroundjoin%
\definecolor{currentfill}{rgb}{0.187231,0.414746,0.556547}%
\pgfsetfillcolor{currentfill}%
\pgfsetfillopacity{0.700000}%
\pgfsetlinewidth{0.000000pt}%
\definecolor{currentstroke}{rgb}{0.000000,0.000000,0.000000}%
\pgfsetstrokecolor{currentstroke}%
\pgfsetdash{}{0pt}%
\pgfpathmoveto{\pgfqpoint{4.652521in}{2.466958in}}%
\pgfpathlineto{\pgfqpoint{4.666221in}{2.475622in}}%
\pgfpathlineto{\pgfqpoint{4.679936in}{2.484448in}}%
\pgfpathlineto{\pgfqpoint{4.693665in}{2.493434in}}%
\pgfpathlineto{\pgfqpoint{4.707410in}{2.502581in}}%
\pgfpathlineto{\pgfqpoint{4.714951in}{2.511860in}}%
\pgfpathlineto{\pgfqpoint{4.722485in}{2.521027in}}%
\pgfpathlineto{\pgfqpoint{4.730013in}{2.530084in}}%
\pgfpathlineto{\pgfqpoint{4.737534in}{2.539031in}}%
\pgfpathlineto{\pgfqpoint{4.723794in}{2.529925in}}%
\pgfpathlineto{\pgfqpoint{4.710070in}{2.520979in}}%
\pgfpathlineto{\pgfqpoint{4.696360in}{2.512194in}}%
\pgfpathlineto{\pgfqpoint{4.682666in}{2.503570in}}%
\pgfpathlineto{\pgfqpoint{4.675139in}{2.494572in}}%
\pgfpathlineto{\pgfqpoint{4.667606in}{2.485471in}}%
\pgfpathlineto{\pgfqpoint{4.660067in}{2.476266in}}%
\pgfpathlineto{\pgfqpoint{4.652521in}{2.466958in}}%
\pgfpathclose%
\pgfusepath{fill}%
\end{pgfscope}%
\begin{pgfscope}%
\pgfpathrectangle{\pgfqpoint{1.254980in}{0.150000in}}{\pgfqpoint{5.490039in}{5.490039in}}%
\pgfusepath{clip}%
\pgfsetbuttcap%
\pgfsetroundjoin%
\definecolor{currentfill}{rgb}{0.149039,0.508051,0.557250}%
\pgfsetfillcolor{currentfill}%
\pgfsetfillopacity{0.700000}%
\pgfsetlinewidth{0.000000pt}%
\definecolor{currentstroke}{rgb}{0.000000,0.000000,0.000000}%
\pgfsetstrokecolor{currentstroke}%
\pgfsetdash{}{0pt}%
\pgfpathmoveto{\pgfqpoint{4.937576in}{2.714493in}}%
\pgfpathlineto{\pgfqpoint{4.951438in}{2.724743in}}%
\pgfpathlineto{\pgfqpoint{4.965316in}{2.735153in}}%
\pgfpathlineto{\pgfqpoint{4.979210in}{2.745723in}}%
\pgfpathlineto{\pgfqpoint{4.993122in}{2.756454in}}%
\pgfpathlineto{\pgfqpoint{5.000529in}{2.763643in}}%
\pgfpathlineto{\pgfqpoint{5.007929in}{2.770720in}}%
\pgfpathlineto{\pgfqpoint{5.015321in}{2.777688in}}%
\pgfpathlineto{\pgfqpoint{5.022705in}{2.784548in}}%
\pgfpathlineto{\pgfqpoint{5.008803in}{2.773979in}}%
\pgfpathlineto{\pgfqpoint{4.994917in}{2.763570in}}%
\pgfpathlineto{\pgfqpoint{4.981049in}{2.753321in}}%
\pgfpathlineto{\pgfqpoint{4.967196in}{2.743231in}}%
\pgfpathlineto{\pgfqpoint{4.959802in}{2.736199in}}%
\pgfpathlineto{\pgfqpoint{4.952401in}{2.729066in}}%
\pgfpathlineto{\pgfqpoint{4.944992in}{2.721832in}}%
\pgfpathlineto{\pgfqpoint{4.937576in}{2.714493in}}%
\pgfpathclose%
\pgfusepath{fill}%
\end{pgfscope}%
\begin{pgfscope}%
\pgfpathrectangle{\pgfqpoint{1.254980in}{0.150000in}}{\pgfqpoint{5.490039in}{5.490039in}}%
\pgfusepath{clip}%
\pgfsetbuttcap%
\pgfsetroundjoin%
\definecolor{currentfill}{rgb}{0.265145,0.232956,0.516599}%
\pgfsetfillcolor{currentfill}%
\pgfsetfillopacity{0.700000}%
\pgfsetlinewidth{0.000000pt}%
\definecolor{currentstroke}{rgb}{0.000000,0.000000,0.000000}%
\pgfsetstrokecolor{currentstroke}%
\pgfsetdash{}{0pt}%
\pgfpathmoveto{\pgfqpoint{4.167230in}{2.034553in}}%
\pgfpathlineto{\pgfqpoint{4.180695in}{2.039447in}}%
\pgfpathlineto{\pgfqpoint{4.194171in}{2.044505in}}%
\pgfpathlineto{\pgfqpoint{4.207658in}{2.049725in}}%
\pgfpathlineto{\pgfqpoint{4.221156in}{2.055107in}}%
\pgfpathlineto{\pgfqpoint{4.228871in}{2.066649in}}%
\pgfpathlineto{\pgfqpoint{4.236580in}{2.078123in}}%
\pgfpathlineto{\pgfqpoint{4.244285in}{2.089530in}}%
\pgfpathlineto{\pgfqpoint{4.251985in}{2.100866in}}%
\pgfpathlineto{\pgfqpoint{4.238490in}{2.095321in}}%
\pgfpathlineto{\pgfqpoint{4.225006in}{2.089939in}}%
\pgfpathlineto{\pgfqpoint{4.211534in}{2.084719in}}%
\pgfpathlineto{\pgfqpoint{4.198073in}{2.079662in}}%
\pgfpathlineto{\pgfqpoint{4.190370in}{2.068477in}}%
\pgfpathlineto{\pgfqpoint{4.182661in}{2.057229in}}%
\pgfpathlineto{\pgfqpoint{4.174948in}{2.045921in}}%
\pgfpathlineto{\pgfqpoint{4.167230in}{2.034553in}}%
\pgfpathclose%
\pgfusepath{fill}%
\end{pgfscope}%
\begin{pgfscope}%
\pgfpathrectangle{\pgfqpoint{1.254980in}{0.150000in}}{\pgfqpoint{5.490039in}{5.490039in}}%
\pgfusepath{clip}%
\pgfsetbuttcap%
\pgfsetroundjoin%
\definecolor{currentfill}{rgb}{0.119699,0.618490,0.536347}%
\pgfsetfillcolor{currentfill}%
\pgfsetfillopacity{0.700000}%
\pgfsetlinewidth{0.000000pt}%
\definecolor{currentstroke}{rgb}{0.000000,0.000000,0.000000}%
\pgfsetstrokecolor{currentstroke}%
\pgfsetdash{}{0pt}%
\pgfpathmoveto{\pgfqpoint{5.307409in}{3.009678in}}%
\pgfpathlineto{\pgfqpoint{5.321490in}{3.021375in}}%
\pgfpathlineto{\pgfqpoint{5.335589in}{3.033230in}}%
\pgfpathlineto{\pgfqpoint{5.349707in}{3.045245in}}%
\pgfpathlineto{\pgfqpoint{5.363844in}{3.057420in}}%
\pgfpathlineto{\pgfqpoint{5.371039in}{3.061649in}}%
\pgfpathlineto{\pgfqpoint{5.378225in}{3.065793in}}%
\pgfpathlineto{\pgfqpoint{5.385403in}{3.069855in}}%
\pgfpathlineto{\pgfqpoint{5.392572in}{3.073838in}}%
\pgfpathlineto{\pgfqpoint{5.378453in}{3.061979in}}%
\pgfpathlineto{\pgfqpoint{5.364352in}{3.050278in}}%
\pgfpathlineto{\pgfqpoint{5.350269in}{3.038736in}}%
\pgfpathlineto{\pgfqpoint{5.336205in}{3.027353in}}%
\pgfpathlineto{\pgfqpoint{5.329018in}{3.023045in}}%
\pgfpathlineto{\pgfqpoint{5.321823in}{3.018665in}}%
\pgfpathlineto{\pgfqpoint{5.314620in}{3.014211in}}%
\pgfpathlineto{\pgfqpoint{5.307409in}{3.009678in}}%
\pgfpathclose%
\pgfusepath{fill}%
\end{pgfscope}%
\begin{pgfscope}%
\pgfpathrectangle{\pgfqpoint{1.254980in}{0.150000in}}{\pgfqpoint{5.490039in}{5.490039in}}%
\pgfusepath{clip}%
\pgfsetbuttcap%
\pgfsetroundjoin%
\definecolor{currentfill}{rgb}{0.172719,0.448791,0.557885}%
\pgfsetfillcolor{currentfill}%
\pgfsetfillopacity{0.700000}%
\pgfsetlinewidth{0.000000pt}%
\definecolor{currentstroke}{rgb}{0.000000,0.000000,0.000000}%
\pgfsetstrokecolor{currentstroke}%
\pgfsetdash{}{0pt}%
\pgfpathmoveto{\pgfqpoint{2.235209in}{2.632566in}}%
\pgfpathlineto{\pgfqpoint{2.248934in}{2.609195in}}%
\pgfpathlineto{\pgfqpoint{2.262645in}{2.586129in}}%
\pgfpathlineto{\pgfqpoint{2.276341in}{2.563366in}}%
\pgfpathlineto{\pgfqpoint{2.290023in}{2.540902in}}%
\pgfpathlineto{\pgfqpoint{2.298829in}{2.537680in}}%
\pgfpathlineto{\pgfqpoint{2.307614in}{2.534756in}}%
\pgfpathlineto{\pgfqpoint{2.316380in}{2.532124in}}%
\pgfpathlineto{\pgfqpoint{2.325125in}{2.529777in}}%
\pgfpathlineto{\pgfqpoint{2.311497in}{2.551699in}}%
\pgfpathlineto{\pgfqpoint{2.297855in}{2.573918in}}%
\pgfpathlineto{\pgfqpoint{2.284199in}{2.596437in}}%
\pgfpathlineto{\pgfqpoint{2.270528in}{2.619261in}}%
\pgfpathlineto{\pgfqpoint{2.261730in}{2.622140in}}%
\pgfpathlineto{\pgfqpoint{2.252911in}{2.625313in}}%
\pgfpathlineto{\pgfqpoint{2.244070in}{2.628787in}}%
\pgfpathlineto{\pgfqpoint{2.235209in}{2.632566in}}%
\pgfpathclose%
\pgfusepath{fill}%
\end{pgfscope}%
\begin{pgfscope}%
\pgfpathrectangle{\pgfqpoint{1.254980in}{0.150000in}}{\pgfqpoint{5.490039in}{5.490039in}}%
\pgfusepath{clip}%
\pgfsetbuttcap%
\pgfsetroundjoin%
\definecolor{currentfill}{rgb}{0.279566,0.067836,0.391917}%
\pgfsetfillcolor{currentfill}%
\pgfsetfillopacity{0.700000}%
\pgfsetlinewidth{0.000000pt}%
\definecolor{currentstroke}{rgb}{0.000000,0.000000,0.000000}%
\pgfsetstrokecolor{currentstroke}%
\pgfsetdash{}{0pt}%
\pgfpathmoveto{\pgfqpoint{3.712715in}{1.716289in}}%
\pgfpathlineto{\pgfqpoint{3.726033in}{1.716322in}}%
\pgfpathlineto{\pgfqpoint{3.739358in}{1.716523in}}%
\pgfpathlineto{\pgfqpoint{3.752690in}{1.716892in}}%
\pgfpathlineto{\pgfqpoint{3.766030in}{1.717427in}}%
\pgfpathlineto{\pgfqpoint{3.773887in}{1.728288in}}%
\pgfpathlineto{\pgfqpoint{3.781738in}{1.739170in}}%
\pgfpathlineto{\pgfqpoint{3.789585in}{1.750070in}}%
\pgfpathlineto{\pgfqpoint{3.797427in}{1.760985in}}%
\pgfpathlineto{\pgfqpoint{3.784096in}{1.760120in}}%
\pgfpathlineto{\pgfqpoint{3.770773in}{1.759423in}}%
\pgfpathlineto{\pgfqpoint{3.757457in}{1.758893in}}%
\pgfpathlineto{\pgfqpoint{3.744149in}{1.758532in}}%
\pgfpathlineto{\pgfqpoint{3.736298in}{1.747935in}}%
\pgfpathlineto{\pgfqpoint{3.728442in}{1.737360in}}%
\pgfpathlineto{\pgfqpoint{3.720581in}{1.726810in}}%
\pgfpathlineto{\pgfqpoint{3.712715in}{1.716289in}}%
\pgfpathclose%
\pgfusepath{fill}%
\end{pgfscope}%
\begin{pgfscope}%
\pgfpathrectangle{\pgfqpoint{1.254980in}{0.150000in}}{\pgfqpoint{5.490039in}{5.490039in}}%
\pgfusepath{clip}%
\pgfsetbuttcap%
\pgfsetroundjoin%
\definecolor{currentfill}{rgb}{0.277941,0.056324,0.381191}%
\pgfsetfillcolor{currentfill}%
\pgfsetfillopacity{0.700000}%
\pgfsetlinewidth{0.000000pt}%
\definecolor{currentstroke}{rgb}{0.000000,0.000000,0.000000}%
\pgfsetstrokecolor{currentstroke}%
\pgfsetdash{}{0pt}%
\pgfpathmoveto{\pgfqpoint{2.988911in}{1.726392in}}%
\pgfpathlineto{\pgfqpoint{3.002223in}{1.716645in}}%
\pgfpathlineto{\pgfqpoint{3.015533in}{1.707096in}}%
\pgfpathlineto{\pgfqpoint{3.028843in}{1.697742in}}%
\pgfpathlineto{\pgfqpoint{3.042152in}{1.688584in}}%
\pgfpathlineto{\pgfqpoint{3.050376in}{1.692760in}}%
\pgfpathlineto{\pgfqpoint{3.058589in}{1.697123in}}%
\pgfpathlineto{\pgfqpoint{3.066790in}{1.701669in}}%
\pgfpathlineto{\pgfqpoint{3.074981in}{1.706392in}}%
\pgfpathlineto{\pgfqpoint{3.061702in}{1.715050in}}%
\pgfpathlineto{\pgfqpoint{3.048422in}{1.723903in}}%
\pgfpathlineto{\pgfqpoint{3.035142in}{1.732951in}}%
\pgfpathlineto{\pgfqpoint{3.021861in}{1.742197in}}%
\pgfpathlineto{\pgfqpoint{3.013641in}{1.737963in}}%
\pgfpathlineto{\pgfqpoint{3.005409in}{1.733914in}}%
\pgfpathlineto{\pgfqpoint{2.997166in}{1.730056in}}%
\pgfpathlineto{\pgfqpoint{2.988911in}{1.726392in}}%
\pgfpathclose%
\pgfusepath{fill}%
\end{pgfscope}%
\begin{pgfscope}%
\pgfpathrectangle{\pgfqpoint{1.254980in}{0.150000in}}{\pgfqpoint{5.490039in}{5.490039in}}%
\pgfusepath{clip}%
\pgfsetbuttcap%
\pgfsetroundjoin%
\definecolor{currentfill}{rgb}{0.282327,0.094955,0.417331}%
\pgfsetfillcolor{currentfill}%
\pgfsetfillopacity{0.700000}%
\pgfsetlinewidth{0.000000pt}%
\definecolor{currentstroke}{rgb}{0.000000,0.000000,0.000000}%
\pgfsetstrokecolor{currentstroke}%
\pgfsetdash{}{0pt}%
\pgfpathmoveto{\pgfqpoint{3.797427in}{1.760985in}}%
\pgfpathlineto{\pgfqpoint{3.810765in}{1.762016in}}%
\pgfpathlineto{\pgfqpoint{3.824112in}{1.763214in}}%
\pgfpathlineto{\pgfqpoint{3.837467in}{1.764578in}}%
\pgfpathlineto{\pgfqpoint{3.850830in}{1.766108in}}%
\pgfpathlineto{\pgfqpoint{3.858659in}{1.777347in}}%
\pgfpathlineto{\pgfqpoint{3.866483in}{1.788589in}}%
\pgfpathlineto{\pgfqpoint{3.874302in}{1.799830in}}%
\pgfpathlineto{\pgfqpoint{3.882116in}{1.811069in}}%
\pgfpathlineto{\pgfqpoint{3.868760in}{1.809238in}}%
\pgfpathlineto{\pgfqpoint{3.855412in}{1.807573in}}%
\pgfpathlineto{\pgfqpoint{3.842073in}{1.806073in}}%
\pgfpathlineto{\pgfqpoint{3.828743in}{1.804741in}}%
\pgfpathlineto{\pgfqpoint{3.820921in}{1.793793in}}%
\pgfpathlineto{\pgfqpoint{3.813095in}{1.782849in}}%
\pgfpathlineto{\pgfqpoint{3.805263in}{1.771912in}}%
\pgfpathlineto{\pgfqpoint{3.797427in}{1.760985in}}%
\pgfpathclose%
\pgfusepath{fill}%
\end{pgfscope}%
\begin{pgfscope}%
\pgfpathrectangle{\pgfqpoint{1.254980in}{0.150000in}}{\pgfqpoint{5.490039in}{5.490039in}}%
\pgfusepath{clip}%
\pgfsetbuttcap%
\pgfsetroundjoin%
\definecolor{currentfill}{rgb}{0.218130,0.347432,0.550038}%
\pgfsetfillcolor{currentfill}%
\pgfsetfillopacity{0.700000}%
\pgfsetlinewidth{0.000000pt}%
\definecolor{currentstroke}{rgb}{0.000000,0.000000,0.000000}%
\pgfsetstrokecolor{currentstroke}%
\pgfsetdash{}{0pt}%
\pgfpathmoveto{\pgfqpoint{4.452312in}{2.283881in}}%
\pgfpathlineto{\pgfqpoint{4.465911in}{2.291200in}}%
\pgfpathlineto{\pgfqpoint{4.479524in}{2.298681in}}%
\pgfpathlineto{\pgfqpoint{4.493150in}{2.306323in}}%
\pgfpathlineto{\pgfqpoint{4.506790in}{2.314127in}}%
\pgfpathlineto{\pgfqpoint{4.514412in}{2.324660in}}%
\pgfpathlineto{\pgfqpoint{4.522028in}{2.335092in}}%
\pgfpathlineto{\pgfqpoint{4.529639in}{2.345425in}}%
\pgfpathlineto{\pgfqpoint{4.537243in}{2.355656in}}%
\pgfpathlineto{\pgfqpoint{4.523607in}{2.347805in}}%
\pgfpathlineto{\pgfqpoint{4.509984in}{2.340115in}}%
\pgfpathlineto{\pgfqpoint{4.496375in}{2.332587in}}%
\pgfpathlineto{\pgfqpoint{4.482779in}{2.325220in}}%
\pgfpathlineto{\pgfqpoint{4.475171in}{2.315025in}}%
\pgfpathlineto{\pgfqpoint{4.467557in}{2.304737in}}%
\pgfpathlineto{\pgfqpoint{4.459937in}{2.294355in}}%
\pgfpathlineto{\pgfqpoint{4.452312in}{2.283881in}}%
\pgfpathclose%
\pgfusepath{fill}%
\end{pgfscope}%
\begin{pgfscope}%
\pgfpathrectangle{\pgfqpoint{1.254980in}{0.150000in}}{\pgfqpoint{5.490039in}{5.490039in}}%
\pgfusepath{clip}%
\pgfsetbuttcap%
\pgfsetroundjoin%
\definecolor{currentfill}{rgb}{0.276022,0.044167,0.370164}%
\pgfsetfillcolor{currentfill}%
\pgfsetfillopacity{0.700000}%
\pgfsetlinewidth{0.000000pt}%
\definecolor{currentstroke}{rgb}{0.000000,0.000000,0.000000}%
\pgfsetstrokecolor{currentstroke}%
\pgfsetdash{}{0pt}%
\pgfpathmoveto{\pgfqpoint{3.627948in}{1.677527in}}%
\pgfpathlineto{\pgfqpoint{3.641250in}{1.676529in}}%
\pgfpathlineto{\pgfqpoint{3.654558in}{1.675699in}}%
\pgfpathlineto{\pgfqpoint{3.667873in}{1.675039in}}%
\pgfpathlineto{\pgfqpoint{3.681195in}{1.674548in}}%
\pgfpathlineto{\pgfqpoint{3.689083in}{1.684925in}}%
\pgfpathlineto{\pgfqpoint{3.696966in}{1.695343in}}%
\pgfpathlineto{\pgfqpoint{3.704843in}{1.705799in}}%
\pgfpathlineto{\pgfqpoint{3.712715in}{1.716289in}}%
\pgfpathlineto{\pgfqpoint{3.699404in}{1.716424in}}%
\pgfpathlineto{\pgfqpoint{3.686100in}{1.716728in}}%
\pgfpathlineto{\pgfqpoint{3.672803in}{1.717201in}}%
\pgfpathlineto{\pgfqpoint{3.659512in}{1.717844in}}%
\pgfpathlineto{\pgfqpoint{3.651630in}{1.707699in}}%
\pgfpathlineto{\pgfqpoint{3.643741in}{1.697596in}}%
\pgfpathlineto{\pgfqpoint{3.635848in}{1.687538in}}%
\pgfpathlineto{\pgfqpoint{3.627948in}{1.677527in}}%
\pgfpathclose%
\pgfusepath{fill}%
\end{pgfscope}%
\begin{pgfscope}%
\pgfpathrectangle{\pgfqpoint{1.254980in}{0.150000in}}{\pgfqpoint{5.490039in}{5.490039in}}%
\pgfusepath{clip}%
\pgfsetbuttcap%
\pgfsetroundjoin%
\definecolor{currentfill}{rgb}{0.269944,0.014625,0.341379}%
\pgfsetfillcolor{currentfill}%
\pgfsetfillopacity{0.700000}%
\pgfsetlinewidth{0.000000pt}%
\definecolor{currentstroke}{rgb}{0.000000,0.000000,0.000000}%
\pgfsetstrokecolor{currentstroke}%
\pgfsetdash{}{0pt}%
\pgfpathmoveto{\pgfqpoint{3.181222in}{1.644027in}}%
\pgfpathlineto{\pgfqpoint{3.194506in}{1.637079in}}%
\pgfpathlineto{\pgfqpoint{3.207790in}{1.630316in}}%
\pgfpathlineto{\pgfqpoint{3.221077in}{1.623738in}}%
\pgfpathlineto{\pgfqpoint{3.234364in}{1.617343in}}%
\pgfpathlineto{\pgfqpoint{3.242467in}{1.623681in}}%
\pgfpathlineto{\pgfqpoint{3.250561in}{1.630166in}}%
\pgfpathlineto{\pgfqpoint{3.258646in}{1.636793in}}%
\pgfpathlineto{\pgfqpoint{3.266722in}{1.643557in}}%
\pgfpathlineto{\pgfqpoint{3.253458in}{1.649483in}}%
\pgfpathlineto{\pgfqpoint{3.240195in}{1.655592in}}%
\pgfpathlineto{\pgfqpoint{3.226934in}{1.661886in}}%
\pgfpathlineto{\pgfqpoint{3.213674in}{1.668365in}}%
\pgfpathlineto{\pgfqpoint{3.205575in}{1.662059in}}%
\pgfpathlineto{\pgfqpoint{3.197467in}{1.655897in}}%
\pgfpathlineto{\pgfqpoint{3.189350in}{1.649885in}}%
\pgfpathlineto{\pgfqpoint{3.181222in}{1.644027in}}%
\pgfpathclose%
\pgfusepath{fill}%
\end{pgfscope}%
\begin{pgfscope}%
\pgfpathrectangle{\pgfqpoint{1.254980in}{0.150000in}}{\pgfqpoint{5.490039in}{5.490039in}}%
\pgfusepath{clip}%
\pgfsetbuttcap%
\pgfsetroundjoin%
\definecolor{currentfill}{rgb}{0.267004,0.004874,0.329415}%
\pgfsetfillcolor{currentfill}%
\pgfsetfillopacity{0.700000}%
\pgfsetlinewidth{0.000000pt}%
\definecolor{currentstroke}{rgb}{0.000000,0.000000,0.000000}%
\pgfsetstrokecolor{currentstroke}%
\pgfsetdash{}{0pt}%
\pgfpathmoveto{\pgfqpoint{3.319804in}{1.621673in}}%
\pgfpathlineto{\pgfqpoint{3.333081in}{1.616653in}}%
\pgfpathlineto{\pgfqpoint{3.346361in}{1.611813in}}%
\pgfpathlineto{\pgfqpoint{3.359644in}{1.607150in}}%
\pgfpathlineto{\pgfqpoint{3.372931in}{1.602665in}}%
\pgfpathlineto{\pgfqpoint{3.380957in}{1.610463in}}%
\pgfpathlineto{\pgfqpoint{3.388975in}{1.618376in}}%
\pgfpathlineto{\pgfqpoint{3.396986in}{1.626400in}}%
\pgfpathlineto{\pgfqpoint{3.404990in}{1.634529in}}%
\pgfpathlineto{\pgfqpoint{3.391722in}{1.638575in}}%
\pgfpathlineto{\pgfqpoint{3.378458in}{1.642798in}}%
\pgfpathlineto{\pgfqpoint{3.365198in}{1.647200in}}%
\pgfpathlineto{\pgfqpoint{3.351940in}{1.651780in}}%
\pgfpathlineto{\pgfqpoint{3.343918in}{1.644079in}}%
\pgfpathlineto{\pgfqpoint{3.335888in}{1.636491in}}%
\pgfpathlineto{\pgfqpoint{3.327850in}{1.629022in}}%
\pgfpathlineto{\pgfqpoint{3.319804in}{1.621673in}}%
\pgfpathclose%
\pgfusepath{fill}%
\end{pgfscope}%
\begin{pgfscope}%
\pgfpathrectangle{\pgfqpoint{1.254980in}{0.150000in}}{\pgfqpoint{5.490039in}{5.490039in}}%
\pgfusepath{clip}%
\pgfsetbuttcap%
\pgfsetroundjoin%
\definecolor{currentfill}{rgb}{0.283229,0.120777,0.440584}%
\pgfsetfillcolor{currentfill}%
\pgfsetfillopacity{0.700000}%
\pgfsetlinewidth{0.000000pt}%
\definecolor{currentstroke}{rgb}{0.000000,0.000000,0.000000}%
\pgfsetstrokecolor{currentstroke}%
\pgfsetdash{}{0pt}%
\pgfpathmoveto{\pgfqpoint{3.882116in}{1.811069in}}%
\pgfpathlineto{\pgfqpoint{3.895480in}{1.813066in}}%
\pgfpathlineto{\pgfqpoint{3.908853in}{1.815228in}}%
\pgfpathlineto{\pgfqpoint{3.922236in}{1.817555in}}%
\pgfpathlineto{\pgfqpoint{3.935627in}{1.820047in}}%
\pgfpathlineto{\pgfqpoint{3.943430in}{1.831564in}}%
\pgfpathlineto{\pgfqpoint{3.951228in}{1.843067in}}%
\pgfpathlineto{\pgfqpoint{3.959021in}{1.854552in}}%
\pgfpathlineto{\pgfqpoint{3.966810in}{1.866017in}}%
\pgfpathlineto{\pgfqpoint{3.953425in}{1.863252in}}%
\pgfpathlineto{\pgfqpoint{3.940048in}{1.860650in}}%
\pgfpathlineto{\pgfqpoint{3.926681in}{1.858215in}}%
\pgfpathlineto{\pgfqpoint{3.913323in}{1.855944in}}%
\pgfpathlineto{\pgfqpoint{3.905529in}{1.844742in}}%
\pgfpathlineto{\pgfqpoint{3.897729in}{1.833528in}}%
\pgfpathlineto{\pgfqpoint{3.889925in}{1.822302in}}%
\pgfpathlineto{\pgfqpoint{3.882116in}{1.811069in}}%
\pgfpathclose%
\pgfusepath{fill}%
\end{pgfscope}%
\begin{pgfscope}%
\pgfpathrectangle{\pgfqpoint{1.254980in}{0.150000in}}{\pgfqpoint{5.490039in}{5.490039in}}%
\pgfusepath{clip}%
\pgfsetbuttcap%
\pgfsetroundjoin%
\definecolor{currentfill}{rgb}{0.126326,0.644107,0.525311}%
\pgfsetfillcolor{currentfill}%
\pgfsetfillopacity{0.700000}%
\pgfsetlinewidth{0.000000pt}%
\definecolor{currentstroke}{rgb}{0.000000,0.000000,0.000000}%
\pgfsetstrokecolor{currentstroke}%
\pgfsetdash{}{0pt}%
\pgfpathmoveto{\pgfqpoint{5.392572in}{3.073838in}}%
\pgfpathlineto{\pgfqpoint{5.406711in}{3.085857in}}%
\pgfpathlineto{\pgfqpoint{5.420868in}{3.098035in}}%
\pgfpathlineto{\pgfqpoint{5.435045in}{3.110373in}}%
\pgfpathlineto{\pgfqpoint{5.449241in}{3.122870in}}%
\pgfpathlineto{\pgfqpoint{5.456383in}{3.126444in}}%
\pgfpathlineto{\pgfqpoint{5.463518in}{3.129940in}}%
\pgfpathlineto{\pgfqpoint{5.470644in}{3.133362in}}%
\pgfpathlineto{\pgfqpoint{5.477762in}{3.136714in}}%
\pgfpathlineto{\pgfqpoint{5.463585in}{3.124563in}}%
\pgfpathlineto{\pgfqpoint{5.449428in}{3.112572in}}%
\pgfpathlineto{\pgfqpoint{5.435289in}{3.100739in}}%
\pgfpathlineto{\pgfqpoint{5.421170in}{3.089064in}}%
\pgfpathlineto{\pgfqpoint{5.414032in}{3.085356in}}%
\pgfpathlineto{\pgfqpoint{5.406887in}{3.081585in}}%
\pgfpathlineto{\pgfqpoint{5.399734in}{3.077747in}}%
\pgfpathlineto{\pgfqpoint{5.392572in}{3.073838in}}%
\pgfpathclose%
\pgfusepath{fill}%
\end{pgfscope}%
\begin{pgfscope}%
\pgfpathrectangle{\pgfqpoint{1.254980in}{0.150000in}}{\pgfqpoint{5.490039in}{5.490039in}}%
\pgfusepath{clip}%
\pgfsetbuttcap%
\pgfsetroundjoin%
\definecolor{currentfill}{rgb}{0.266580,0.228262,0.514349}%
\pgfsetfillcolor{currentfill}%
\pgfsetfillopacity{0.700000}%
\pgfsetlinewidth{0.000000pt}%
\definecolor{currentstroke}{rgb}{0.000000,0.000000,0.000000}%
\pgfsetstrokecolor{currentstroke}%
\pgfsetdash{}{0pt}%
\pgfpathmoveto{\pgfqpoint{2.580708in}{2.087102in}}%
\pgfpathlineto{\pgfqpoint{2.594176in}{2.070740in}}%
\pgfpathlineto{\pgfqpoint{2.607637in}{2.054617in}}%
\pgfpathlineto{\pgfqpoint{2.621091in}{2.038731in}}%
\pgfpathlineto{\pgfqpoint{2.634538in}{2.023079in}}%
\pgfpathlineto{\pgfqpoint{2.643076in}{2.022617in}}%
\pgfpathlineto{\pgfqpoint{2.651596in}{2.022417in}}%
\pgfpathlineto{\pgfqpoint{2.660101in}{2.022473in}}%
\pgfpathlineto{\pgfqpoint{2.668589in}{2.022780in}}%
\pgfpathlineto{\pgfqpoint{2.655185in}{2.037885in}}%
\pgfpathlineto{\pgfqpoint{2.641775in}{2.053225in}}%
\pgfpathlineto{\pgfqpoint{2.628358in}{2.068800in}}%
\pgfpathlineto{\pgfqpoint{2.614934in}{2.084613in}}%
\pgfpathlineto{\pgfqpoint{2.606403in}{2.084842in}}%
\pgfpathlineto{\pgfqpoint{2.597855in}{2.085329in}}%
\pgfpathlineto{\pgfqpoint{2.589290in}{2.086081in}}%
\pgfpathlineto{\pgfqpoint{2.580708in}{2.087102in}}%
\pgfpathclose%
\pgfusepath{fill}%
\end{pgfscope}%
\begin{pgfscope}%
\pgfpathrectangle{\pgfqpoint{1.254980in}{0.150000in}}{\pgfqpoint{5.490039in}{5.490039in}}%
\pgfusepath{clip}%
\pgfsetbuttcap%
\pgfsetroundjoin%
\definecolor{currentfill}{rgb}{0.272594,0.025563,0.353093}%
\pgfsetfillcolor{currentfill}%
\pgfsetfillopacity{0.700000}%
\pgfsetlinewidth{0.000000pt}%
\definecolor{currentstroke}{rgb}{0.000000,0.000000,0.000000}%
\pgfsetstrokecolor{currentstroke}%
\pgfsetdash{}{0pt}%
\pgfpathmoveto{\pgfqpoint{3.543089in}{1.645269in}}%
\pgfpathlineto{\pgfqpoint{3.556381in}{1.643204in}}%
\pgfpathlineto{\pgfqpoint{3.569679in}{1.641310in}}%
\pgfpathlineto{\pgfqpoint{3.582982in}{1.639587in}}%
\pgfpathlineto{\pgfqpoint{3.596290in}{1.638034in}}%
\pgfpathlineto{\pgfqpoint{3.604214in}{1.647818in}}%
\pgfpathlineto{\pgfqpoint{3.612131in}{1.657664in}}%
\pgfpathlineto{\pgfqpoint{3.620042in}{1.667568in}}%
\pgfpathlineto{\pgfqpoint{3.627948in}{1.677527in}}%
\pgfpathlineto{\pgfqpoint{3.614652in}{1.678696in}}%
\pgfpathlineto{\pgfqpoint{3.601362in}{1.680035in}}%
\pgfpathlineto{\pgfqpoint{3.588078in}{1.681545in}}%
\pgfpathlineto{\pgfqpoint{3.574799in}{1.683227in}}%
\pgfpathlineto{\pgfqpoint{3.566881in}{1.673641in}}%
\pgfpathlineto{\pgfqpoint{3.558957in}{1.664117in}}%
\pgfpathlineto{\pgfqpoint{3.551026in}{1.654659in}}%
\pgfpathlineto{\pgfqpoint{3.543089in}{1.645269in}}%
\pgfpathclose%
\pgfusepath{fill}%
\end{pgfscope}%
\begin{pgfscope}%
\pgfpathrectangle{\pgfqpoint{1.254980in}{0.150000in}}{\pgfqpoint{5.490039in}{5.490039in}}%
\pgfusepath{clip}%
\pgfsetbuttcap%
\pgfsetroundjoin%
\definecolor{currentfill}{rgb}{0.274128,0.199721,0.498911}%
\pgfsetfillcolor{currentfill}%
\pgfsetfillopacity{0.700000}%
\pgfsetlinewidth{0.000000pt}%
\definecolor{currentstroke}{rgb}{0.000000,0.000000,0.000000}%
\pgfsetstrokecolor{currentstroke}%
\pgfsetdash{}{0pt}%
\pgfpathmoveto{\pgfqpoint{2.634538in}{2.023079in}}%
\pgfpathlineto{\pgfqpoint{2.647979in}{2.007662in}}%
\pgfpathlineto{\pgfqpoint{2.661413in}{1.992475in}}%
\pgfpathlineto{\pgfqpoint{2.674840in}{1.977519in}}%
\pgfpathlineto{\pgfqpoint{2.688262in}{1.962791in}}%
\pgfpathlineto{\pgfqpoint{2.696756in}{1.962885in}}%
\pgfpathlineto{\pgfqpoint{2.705235in}{1.963232in}}%
\pgfpathlineto{\pgfqpoint{2.713697in}{1.963828in}}%
\pgfpathlineto{\pgfqpoint{2.722145in}{1.964667in}}%
\pgfpathlineto{\pgfqpoint{2.708764in}{1.978853in}}%
\pgfpathlineto{\pgfqpoint{2.695378in}{1.993266in}}%
\pgfpathlineto{\pgfqpoint{2.681987in}{2.007907in}}%
\pgfpathlineto{\pgfqpoint{2.668589in}{2.022780in}}%
\pgfpathlineto{\pgfqpoint{2.660101in}{2.022473in}}%
\pgfpathlineto{\pgfqpoint{2.651596in}{2.022417in}}%
\pgfpathlineto{\pgfqpoint{2.643076in}{2.022617in}}%
\pgfpathlineto{\pgfqpoint{2.634538in}{2.023079in}}%
\pgfpathclose%
\pgfusepath{fill}%
\end{pgfscope}%
\begin{pgfscope}%
\pgfpathrectangle{\pgfqpoint{1.254980in}{0.150000in}}{\pgfqpoint{5.490039in}{5.490039in}}%
\pgfusepath{clip}%
\pgfsetbuttcap%
\pgfsetroundjoin%
\definecolor{currentfill}{rgb}{0.172719,0.448791,0.557885}%
\pgfsetfillcolor{currentfill}%
\pgfsetfillopacity{0.700000}%
\pgfsetlinewidth{0.000000pt}%
\definecolor{currentstroke}{rgb}{0.000000,0.000000,0.000000}%
\pgfsetstrokecolor{currentstroke}%
\pgfsetdash{}{0pt}%
\pgfpathmoveto{\pgfqpoint{4.737534in}{2.539031in}}%
\pgfpathlineto{\pgfqpoint{4.751289in}{2.548299in}}%
\pgfpathlineto{\pgfqpoint{4.765059in}{2.557727in}}%
\pgfpathlineto{\pgfqpoint{4.778845in}{2.567316in}}%
\pgfpathlineto{\pgfqpoint{4.792647in}{2.577065in}}%
\pgfpathlineto{\pgfqpoint{4.800156in}{2.585844in}}%
\pgfpathlineto{\pgfqpoint{4.807658in}{2.594508in}}%
\pgfpathlineto{\pgfqpoint{4.815153in}{2.603058in}}%
\pgfpathlineto{\pgfqpoint{4.822641in}{2.611494in}}%
\pgfpathlineto{\pgfqpoint{4.808845in}{2.601816in}}%
\pgfpathlineto{\pgfqpoint{4.795065in}{2.592298in}}%
\pgfpathlineto{\pgfqpoint{4.781301in}{2.582940in}}%
\pgfpathlineto{\pgfqpoint{4.767552in}{2.573743in}}%
\pgfpathlineto{\pgfqpoint{4.760057in}{2.565225in}}%
\pgfpathlineto{\pgfqpoint{4.752556in}{2.556601in}}%
\pgfpathlineto{\pgfqpoint{4.745048in}{2.547870in}}%
\pgfpathlineto{\pgfqpoint{4.737534in}{2.539031in}}%
\pgfpathclose%
\pgfusepath{fill}%
\end{pgfscope}%
\begin{pgfscope}%
\pgfpathrectangle{\pgfqpoint{1.254980in}{0.150000in}}{\pgfqpoint{5.490039in}{5.490039in}}%
\pgfusepath{clip}%
\pgfsetbuttcap%
\pgfsetroundjoin%
\definecolor{currentfill}{rgb}{0.137770,0.537492,0.554906}%
\pgfsetfillcolor{currentfill}%
\pgfsetfillopacity{0.700000}%
\pgfsetlinewidth{0.000000pt}%
\definecolor{currentstroke}{rgb}{0.000000,0.000000,0.000000}%
\pgfsetstrokecolor{currentstroke}%
\pgfsetdash{}{0pt}%
\pgfpathmoveto{\pgfqpoint{5.022705in}{2.784548in}}%
\pgfpathlineto{\pgfqpoint{5.036625in}{2.795278in}}%
\pgfpathlineto{\pgfqpoint{5.050561in}{2.806167in}}%
\pgfpathlineto{\pgfqpoint{5.064515in}{2.817217in}}%
\pgfpathlineto{\pgfqpoint{5.078486in}{2.828428in}}%
\pgfpathlineto{\pgfqpoint{5.085853in}{2.835002in}}%
\pgfpathlineto{\pgfqpoint{5.093211in}{2.841467in}}%
\pgfpathlineto{\pgfqpoint{5.100562in}{2.847823in}}%
\pgfpathlineto{\pgfqpoint{5.107905in}{2.854074in}}%
\pgfpathlineto{\pgfqpoint{5.093944in}{2.843056in}}%
\pgfpathlineto{\pgfqpoint{5.080001in}{2.832198in}}%
\pgfpathlineto{\pgfqpoint{5.066075in}{2.821500in}}%
\pgfpathlineto{\pgfqpoint{5.052167in}{2.810962in}}%
\pgfpathlineto{\pgfqpoint{5.044813in}{2.804508in}}%
\pgfpathlineto{\pgfqpoint{5.037451in}{2.797956in}}%
\pgfpathlineto{\pgfqpoint{5.030082in}{2.791304in}}%
\pgfpathlineto{\pgfqpoint{5.022705in}{2.784548in}}%
\pgfpathclose%
\pgfusepath{fill}%
\end{pgfscope}%
\begin{pgfscope}%
\pgfpathrectangle{\pgfqpoint{1.254980in}{0.150000in}}{\pgfqpoint{5.490039in}{5.490039in}}%
\pgfusepath{clip}%
\pgfsetbuttcap%
\pgfsetroundjoin%
\definecolor{currentfill}{rgb}{0.255645,0.260703,0.528312}%
\pgfsetfillcolor{currentfill}%
\pgfsetfillopacity{0.700000}%
\pgfsetlinewidth{0.000000pt}%
\definecolor{currentstroke}{rgb}{0.000000,0.000000,0.000000}%
\pgfsetstrokecolor{currentstroke}%
\pgfsetdash{}{0pt}%
\pgfpathmoveto{\pgfqpoint{2.526755in}{2.154976in}}%
\pgfpathlineto{\pgfqpoint{2.540256in}{2.137639in}}%
\pgfpathlineto{\pgfqpoint{2.553748in}{2.120550in}}%
\pgfpathlineto{\pgfqpoint{2.567232in}{2.103704in}}%
\pgfpathlineto{\pgfqpoint{2.580708in}{2.087102in}}%
\pgfpathlineto{\pgfqpoint{2.589290in}{2.086081in}}%
\pgfpathlineto{\pgfqpoint{2.597855in}{2.085329in}}%
\pgfpathlineto{\pgfqpoint{2.606403in}{2.084842in}}%
\pgfpathlineto{\pgfqpoint{2.614934in}{2.084613in}}%
\pgfpathlineto{\pgfqpoint{2.601503in}{2.100667in}}%
\pgfpathlineto{\pgfqpoint{2.588064in}{2.116962in}}%
\pgfpathlineto{\pgfqpoint{2.574618in}{2.133500in}}%
\pgfpathlineto{\pgfqpoint{2.561164in}{2.150284in}}%
\pgfpathlineto{\pgfqpoint{2.552588in}{2.151051in}}%
\pgfpathlineto{\pgfqpoint{2.543995in}{2.152085in}}%
\pgfpathlineto{\pgfqpoint{2.535384in}{2.153392in}}%
\pgfpathlineto{\pgfqpoint{2.526755in}{2.154976in}}%
\pgfpathclose%
\pgfusepath{fill}%
\end{pgfscope}%
\begin{pgfscope}%
\pgfpathrectangle{\pgfqpoint{1.254980in}{0.150000in}}{\pgfqpoint{5.490039in}{5.490039in}}%
\pgfusepath{clip}%
\pgfsetbuttcap%
\pgfsetroundjoin%
\definecolor{currentfill}{rgb}{0.253935,0.265254,0.529983}%
\pgfsetfillcolor{currentfill}%
\pgfsetfillopacity{0.700000}%
\pgfsetlinewidth{0.000000pt}%
\definecolor{currentstroke}{rgb}{0.000000,0.000000,0.000000}%
\pgfsetstrokecolor{currentstroke}%
\pgfsetdash{}{0pt}%
\pgfpathmoveto{\pgfqpoint{4.251985in}{2.100866in}}%
\pgfpathlineto{\pgfqpoint{4.265492in}{2.106574in}}%
\pgfpathlineto{\pgfqpoint{4.279011in}{2.112444in}}%
\pgfpathlineto{\pgfqpoint{4.292543in}{2.118476in}}%
\pgfpathlineto{\pgfqpoint{4.306086in}{2.124670in}}%
\pgfpathlineto{\pgfqpoint{4.313778in}{2.136081in}}%
\pgfpathlineto{\pgfqpoint{4.321465in}{2.147413in}}%
\pgfpathlineto{\pgfqpoint{4.329147in}{2.158666in}}%
\pgfpathlineto{\pgfqpoint{4.336823in}{2.169838in}}%
\pgfpathlineto{\pgfqpoint{4.323283in}{2.163510in}}%
\pgfpathlineto{\pgfqpoint{4.309754in}{2.157344in}}%
\pgfpathlineto{\pgfqpoint{4.296238in}{2.151340in}}%
\pgfpathlineto{\pgfqpoint{4.282734in}{2.145498in}}%
\pgfpathlineto{\pgfqpoint{4.275055in}{2.134450in}}%
\pgfpathlineto{\pgfqpoint{4.267370in}{2.123328in}}%
\pgfpathlineto{\pgfqpoint{4.259680in}{2.112133in}}%
\pgfpathlineto{\pgfqpoint{4.251985in}{2.100866in}}%
\pgfpathclose%
\pgfusepath{fill}%
\end{pgfscope}%
\begin{pgfscope}%
\pgfpathrectangle{\pgfqpoint{1.254980in}{0.150000in}}{\pgfqpoint{5.490039in}{5.490039in}}%
\pgfusepath{clip}%
\pgfsetbuttcap%
\pgfsetroundjoin%
\definecolor{currentfill}{rgb}{0.279574,0.170599,0.479997}%
\pgfsetfillcolor{currentfill}%
\pgfsetfillopacity{0.700000}%
\pgfsetlinewidth{0.000000pt}%
\definecolor{currentstroke}{rgb}{0.000000,0.000000,0.000000}%
\pgfsetstrokecolor{currentstroke}%
\pgfsetdash{}{0pt}%
\pgfpathmoveto{\pgfqpoint{2.688262in}{1.962791in}}%
\pgfpathlineto{\pgfqpoint{2.701678in}{1.948289in}}%
\pgfpathlineto{\pgfqpoint{2.715089in}{1.934012in}}%
\pgfpathlineto{\pgfqpoint{2.728494in}{1.919959in}}%
\pgfpathlineto{\pgfqpoint{2.741894in}{1.906127in}}%
\pgfpathlineto{\pgfqpoint{2.750347in}{1.906773in}}%
\pgfpathlineto{\pgfqpoint{2.758784in}{1.907666in}}%
\pgfpathlineto{\pgfqpoint{2.767207in}{1.908799in}}%
\pgfpathlineto{\pgfqpoint{2.775615in}{1.910168in}}%
\pgfpathlineto{\pgfqpoint{2.762255in}{1.923460in}}%
\pgfpathlineto{\pgfqpoint{2.748890in}{1.936973in}}%
\pgfpathlineto{\pgfqpoint{2.735520in}{1.950708in}}%
\pgfpathlineto{\pgfqpoint{2.722145in}{1.964667in}}%
\pgfpathlineto{\pgfqpoint{2.713697in}{1.963828in}}%
\pgfpathlineto{\pgfqpoint{2.705235in}{1.963232in}}%
\pgfpathlineto{\pgfqpoint{2.696756in}{1.962885in}}%
\pgfpathlineto{\pgfqpoint{2.688262in}{1.962791in}}%
\pgfpathclose%
\pgfusepath{fill}%
\end{pgfscope}%
\begin{pgfscope}%
\pgfpathrectangle{\pgfqpoint{1.254980in}{0.150000in}}{\pgfqpoint{5.490039in}{5.490039in}}%
\pgfusepath{clip}%
\pgfsetbuttcap%
\pgfsetroundjoin%
\definecolor{currentfill}{rgb}{0.281412,0.155834,0.469201}%
\pgfsetfillcolor{currentfill}%
\pgfsetfillopacity{0.700000}%
\pgfsetlinewidth{0.000000pt}%
\definecolor{currentstroke}{rgb}{0.000000,0.000000,0.000000}%
\pgfsetstrokecolor{currentstroke}%
\pgfsetdash{}{0pt}%
\pgfpathmoveto{\pgfqpoint{3.966810in}{1.866017in}}%
\pgfpathlineto{\pgfqpoint{3.980205in}{1.868948in}}%
\pgfpathlineto{\pgfqpoint{3.993609in}{1.872042in}}%
\pgfpathlineto{\pgfqpoint{4.007023in}{1.875301in}}%
\pgfpathlineto{\pgfqpoint{4.020447in}{1.878723in}}%
\pgfpathlineto{\pgfqpoint{4.028226in}{1.890423in}}%
\pgfpathlineto{\pgfqpoint{4.036000in}{1.902092in}}%
\pgfpathlineto{\pgfqpoint{4.043770in}{1.913727in}}%
\pgfpathlineto{\pgfqpoint{4.051535in}{1.925326in}}%
\pgfpathlineto{\pgfqpoint{4.038115in}{1.921657in}}%
\pgfpathlineto{\pgfqpoint{4.024706in}{1.918152in}}%
\pgfpathlineto{\pgfqpoint{4.011307in}{1.914811in}}%
\pgfpathlineto{\pgfqpoint{3.997917in}{1.911634in}}%
\pgfpathlineto{\pgfqpoint{3.990148in}{1.900271in}}%
\pgfpathlineto{\pgfqpoint{3.982373in}{1.888879in}}%
\pgfpathlineto{\pgfqpoint{3.974594in}{1.877460in}}%
\pgfpathlineto{\pgfqpoint{3.966810in}{1.866017in}}%
\pgfpathclose%
\pgfusepath{fill}%
\end{pgfscope}%
\begin{pgfscope}%
\pgfpathrectangle{\pgfqpoint{1.254980in}{0.150000in}}{\pgfqpoint{5.490039in}{5.490039in}}%
\pgfusepath{clip}%
\pgfsetbuttcap%
\pgfsetroundjoin%
\definecolor{currentfill}{rgb}{0.243113,0.292092,0.538516}%
\pgfsetfillcolor{currentfill}%
\pgfsetfillopacity{0.700000}%
\pgfsetlinewidth{0.000000pt}%
\definecolor{currentstroke}{rgb}{0.000000,0.000000,0.000000}%
\pgfsetstrokecolor{currentstroke}%
\pgfsetdash{}{0pt}%
\pgfpathmoveto{\pgfqpoint{2.472665in}{2.226826in}}%
\pgfpathlineto{\pgfqpoint{2.486202in}{2.208484in}}%
\pgfpathlineto{\pgfqpoint{2.499729in}{2.190396in}}%
\pgfpathlineto{\pgfqpoint{2.513246in}{2.172560in}}%
\pgfpathlineto{\pgfqpoint{2.526755in}{2.154976in}}%
\pgfpathlineto{\pgfqpoint{2.535384in}{2.153392in}}%
\pgfpathlineto{\pgfqpoint{2.543995in}{2.152085in}}%
\pgfpathlineto{\pgfqpoint{2.552588in}{2.151051in}}%
\pgfpathlineto{\pgfqpoint{2.561164in}{2.150284in}}%
\pgfpathlineto{\pgfqpoint{2.547702in}{2.167316in}}%
\pgfpathlineto{\pgfqpoint{2.534231in}{2.184597in}}%
\pgfpathlineto{\pgfqpoint{2.520752in}{2.202130in}}%
\pgfpathlineto{\pgfqpoint{2.507264in}{2.219917in}}%
\pgfpathlineto{\pgfqpoint{2.498642in}{2.221226in}}%
\pgfpathlineto{\pgfqpoint{2.490002in}{2.222811in}}%
\pgfpathlineto{\pgfqpoint{2.481343in}{2.224676in}}%
\pgfpathlineto{\pgfqpoint{2.472665in}{2.226826in}}%
\pgfpathclose%
\pgfusepath{fill}%
\end{pgfscope}%
\begin{pgfscope}%
\pgfpathrectangle{\pgfqpoint{1.254980in}{0.150000in}}{\pgfqpoint{5.490039in}{5.490039in}}%
\pgfusepath{clip}%
\pgfsetbuttcap%
\pgfsetroundjoin%
\definecolor{currentfill}{rgb}{0.282290,0.145912,0.461510}%
\pgfsetfillcolor{currentfill}%
\pgfsetfillopacity{0.700000}%
\pgfsetlinewidth{0.000000pt}%
\definecolor{currentstroke}{rgb}{0.000000,0.000000,0.000000}%
\pgfsetstrokecolor{currentstroke}%
\pgfsetdash{}{0pt}%
\pgfpathmoveto{\pgfqpoint{2.741894in}{1.906127in}}%
\pgfpathlineto{\pgfqpoint{2.755289in}{1.892515in}}%
\pgfpathlineto{\pgfqpoint{2.768680in}{1.879122in}}%
\pgfpathlineto{\pgfqpoint{2.782066in}{1.865946in}}%
\pgfpathlineto{\pgfqpoint{2.795447in}{1.852986in}}%
\pgfpathlineto{\pgfqpoint{2.803861in}{1.854182in}}%
\pgfpathlineto{\pgfqpoint{2.812259in}{1.855617in}}%
\pgfpathlineto{\pgfqpoint{2.820644in}{1.857285in}}%
\pgfpathlineto{\pgfqpoint{2.829014in}{1.859181in}}%
\pgfpathlineto{\pgfqpoint{2.815670in}{1.871603in}}%
\pgfpathlineto{\pgfqpoint{2.802323in}{1.884241in}}%
\pgfpathlineto{\pgfqpoint{2.788971in}{1.897096in}}%
\pgfpathlineto{\pgfqpoint{2.775615in}{1.910168in}}%
\pgfpathlineto{\pgfqpoint{2.767207in}{1.908799in}}%
\pgfpathlineto{\pgfqpoint{2.758784in}{1.907666in}}%
\pgfpathlineto{\pgfqpoint{2.750347in}{1.906773in}}%
\pgfpathlineto{\pgfqpoint{2.741894in}{1.906127in}}%
\pgfpathclose%
\pgfusepath{fill}%
\end{pgfscope}%
\begin{pgfscope}%
\pgfpathrectangle{\pgfqpoint{1.254980in}{0.150000in}}{\pgfqpoint{5.490039in}{5.490039in}}%
\pgfusepath{clip}%
\pgfsetbuttcap%
\pgfsetroundjoin%
\definecolor{currentfill}{rgb}{0.274952,0.037752,0.364543}%
\pgfsetfillcolor{currentfill}%
\pgfsetfillopacity{0.700000}%
\pgfsetlinewidth{0.000000pt}%
\definecolor{currentstroke}{rgb}{0.000000,0.000000,0.000000}%
\pgfsetstrokecolor{currentstroke}%
\pgfsetdash{}{0pt}%
\pgfpathmoveto{\pgfqpoint{3.042152in}{1.688584in}}%
\pgfpathlineto{\pgfqpoint{3.055460in}{1.679620in}}%
\pgfpathlineto{\pgfqpoint{3.068767in}{1.670850in}}%
\pgfpathlineto{\pgfqpoint{3.082075in}{1.662271in}}%
\pgfpathlineto{\pgfqpoint{3.095382in}{1.653884in}}%
\pgfpathlineto{\pgfqpoint{3.103577in}{1.658570in}}%
\pgfpathlineto{\pgfqpoint{3.111761in}{1.663436in}}%
\pgfpathlineto{\pgfqpoint{3.119935in}{1.668478in}}%
\pgfpathlineto{\pgfqpoint{3.128098in}{1.673690in}}%
\pgfpathlineto{\pgfqpoint{3.114819in}{1.681578in}}%
\pgfpathlineto{\pgfqpoint{3.101539in}{1.689657in}}%
\pgfpathlineto{\pgfqpoint{3.088260in}{1.697928in}}%
\pgfpathlineto{\pgfqpoint{3.074981in}{1.706392in}}%
\pgfpathlineto{\pgfqpoint{3.066790in}{1.701669in}}%
\pgfpathlineto{\pgfqpoint{3.058589in}{1.697123in}}%
\pgfpathlineto{\pgfqpoint{3.050376in}{1.692760in}}%
\pgfpathlineto{\pgfqpoint{3.042152in}{1.688584in}}%
\pgfpathclose%
\pgfusepath{fill}%
\end{pgfscope}%
\begin{pgfscope}%
\pgfpathrectangle{\pgfqpoint{1.254980in}{0.150000in}}{\pgfqpoint{5.490039in}{5.490039in}}%
\pgfusepath{clip}%
\pgfsetbuttcap%
\pgfsetroundjoin%
\definecolor{currentfill}{rgb}{0.157729,0.485932,0.558013}%
\pgfsetfillcolor{currentfill}%
\pgfsetfillopacity{0.700000}%
\pgfsetlinewidth{0.000000pt}%
\definecolor{currentstroke}{rgb}{0.000000,0.000000,0.000000}%
\pgfsetstrokecolor{currentstroke}%
\pgfsetdash{}{0pt}%
\pgfpathmoveto{\pgfqpoint{2.180153in}{2.729161in}}%
\pgfpathlineto{\pgfqpoint{2.193941in}{2.704539in}}%
\pgfpathlineto{\pgfqpoint{2.207712in}{2.680234in}}%
\pgfpathlineto{\pgfqpoint{2.221468in}{2.656244in}}%
\pgfpathlineto{\pgfqpoint{2.235209in}{2.632566in}}%
\pgfpathlineto{\pgfqpoint{2.244070in}{2.628787in}}%
\pgfpathlineto{\pgfqpoint{2.252911in}{2.625313in}}%
\pgfpathlineto{\pgfqpoint{2.261730in}{2.622140in}}%
\pgfpathlineto{\pgfqpoint{2.270528in}{2.619261in}}%
\pgfpathlineto{\pgfqpoint{2.256843in}{2.642391in}}%
\pgfpathlineto{\pgfqpoint{2.243144in}{2.665831in}}%
\pgfpathlineto{\pgfqpoint{2.229428in}{2.689584in}}%
\pgfpathlineto{\pgfqpoint{2.215698in}{2.713653in}}%
\pgfpathlineto{\pgfqpoint{2.206844in}{2.717070in}}%
\pgfpathlineto{\pgfqpoint{2.197969in}{2.720790in}}%
\pgfpathlineto{\pgfqpoint{2.189072in}{2.724819in}}%
\pgfpathlineto{\pgfqpoint{2.180153in}{2.729161in}}%
\pgfpathclose%
\pgfusepath{fill}%
\end{pgfscope}%
\begin{pgfscope}%
\pgfpathrectangle{\pgfqpoint{1.254980in}{0.150000in}}{\pgfqpoint{5.490039in}{5.490039in}}%
\pgfusepath{clip}%
\pgfsetbuttcap%
\pgfsetroundjoin%
\definecolor{currentfill}{rgb}{0.143303,0.669459,0.511215}%
\pgfsetfillcolor{currentfill}%
\pgfsetfillopacity{0.700000}%
\pgfsetlinewidth{0.000000pt}%
\definecolor{currentstroke}{rgb}{0.000000,0.000000,0.000000}%
\pgfsetstrokecolor{currentstroke}%
\pgfsetdash{}{0pt}%
\pgfpathmoveto{\pgfqpoint{5.477762in}{3.136714in}}%
\pgfpathlineto{\pgfqpoint{5.491957in}{3.149023in}}%
\pgfpathlineto{\pgfqpoint{5.506173in}{3.161492in}}%
\pgfpathlineto{\pgfqpoint{5.520407in}{3.174120in}}%
\pgfpathlineto{\pgfqpoint{5.534662in}{3.186907in}}%
\pgfpathlineto{\pgfqpoint{5.541751in}{3.189827in}}%
\pgfpathlineto{\pgfqpoint{5.548831in}{3.192677in}}%
\pgfpathlineto{\pgfqpoint{5.555903in}{3.195462in}}%
\pgfpathlineto{\pgfqpoint{5.562967in}{3.198187in}}%
\pgfpathlineto{\pgfqpoint{5.548734in}{3.185777in}}%
\pgfpathlineto{\pgfqpoint{5.534521in}{3.173527in}}%
\pgfpathlineto{\pgfqpoint{5.520327in}{3.161434in}}%
\pgfpathlineto{\pgfqpoint{5.506152in}{3.149501in}}%
\pgfpathlineto{\pgfqpoint{5.499066in}{3.146389in}}%
\pgfpathlineto{\pgfqpoint{5.491973in}{3.143223in}}%
\pgfpathlineto{\pgfqpoint{5.484871in}{3.139999in}}%
\pgfpathlineto{\pgfqpoint{5.477762in}{3.136714in}}%
\pgfpathclose%
\pgfusepath{fill}%
\end{pgfscope}%
\begin{pgfscope}%
\pgfpathrectangle{\pgfqpoint{1.254980in}{0.150000in}}{\pgfqpoint{5.490039in}{5.490039in}}%
\pgfusepath{clip}%
\pgfsetbuttcap%
\pgfsetroundjoin%
\definecolor{currentfill}{rgb}{0.268510,0.009605,0.335427}%
\pgfsetfillcolor{currentfill}%
\pgfsetfillopacity{0.700000}%
\pgfsetlinewidth{0.000000pt}%
\definecolor{currentstroke}{rgb}{0.000000,0.000000,0.000000}%
\pgfsetstrokecolor{currentstroke}%
\pgfsetdash{}{0pt}%
\pgfpathmoveto{\pgfqpoint{3.458099in}{1.620108in}}%
\pgfpathlineto{\pgfqpoint{3.471386in}{1.616940in}}%
\pgfpathlineto{\pgfqpoint{3.484678in}{1.613946in}}%
\pgfpathlineto{\pgfqpoint{3.497975in}{1.611124in}}%
\pgfpathlineto{\pgfqpoint{3.511277in}{1.608475in}}%
\pgfpathlineto{\pgfqpoint{3.519240in}{1.617551in}}%
\pgfpathlineto{\pgfqpoint{3.527196in}{1.626712in}}%
\pgfpathlineto{\pgfqpoint{3.535146in}{1.635952in}}%
\pgfpathlineto{\pgfqpoint{3.543089in}{1.645269in}}%
\pgfpathlineto{\pgfqpoint{3.529803in}{1.647506in}}%
\pgfpathlineto{\pgfqpoint{3.516521in}{1.649916in}}%
\pgfpathlineto{\pgfqpoint{3.503244in}{1.652499in}}%
\pgfpathlineto{\pgfqpoint{3.489972in}{1.655256in}}%
\pgfpathlineto{\pgfqpoint{3.482014in}{1.646340in}}%
\pgfpathlineto{\pgfqpoint{3.474049in}{1.637507in}}%
\pgfpathlineto{\pgfqpoint{3.466077in}{1.628762in}}%
\pgfpathlineto{\pgfqpoint{3.458099in}{1.620108in}}%
\pgfpathclose%
\pgfusepath{fill}%
\end{pgfscope}%
\begin{pgfscope}%
\pgfpathrectangle{\pgfqpoint{1.254980in}{0.150000in}}{\pgfqpoint{5.490039in}{5.490039in}}%
\pgfusepath{clip}%
\pgfsetbuttcap%
\pgfsetroundjoin%
\definecolor{currentfill}{rgb}{0.203063,0.379716,0.553925}%
\pgfsetfillcolor{currentfill}%
\pgfsetfillopacity{0.700000}%
\pgfsetlinewidth{0.000000pt}%
\definecolor{currentstroke}{rgb}{0.000000,0.000000,0.000000}%
\pgfsetstrokecolor{currentstroke}%
\pgfsetdash{}{0pt}%
\pgfpathmoveto{\pgfqpoint{4.537243in}{2.355656in}}%
\pgfpathlineto{\pgfqpoint{4.550894in}{2.363669in}}%
\pgfpathlineto{\pgfqpoint{4.564559in}{2.371843in}}%
\pgfpathlineto{\pgfqpoint{4.578238in}{2.380178in}}%
\pgfpathlineto{\pgfqpoint{4.591931in}{2.388674in}}%
\pgfpathlineto{\pgfqpoint{4.599526in}{2.398834in}}%
\pgfpathlineto{\pgfqpoint{4.607115in}{2.408887in}}%
\pgfpathlineto{\pgfqpoint{4.614699in}{2.418832in}}%
\pgfpathlineto{\pgfqpoint{4.622275in}{2.428670in}}%
\pgfpathlineto{\pgfqpoint{4.608586in}{2.420156in}}%
\pgfpathlineto{\pgfqpoint{4.594910in}{2.411802in}}%
\pgfpathlineto{\pgfqpoint{4.581250in}{2.403610in}}%
\pgfpathlineto{\pgfqpoint{4.567603in}{2.395579in}}%
\pgfpathlineto{\pgfqpoint{4.560022in}{2.385749in}}%
\pgfpathlineto{\pgfqpoint{4.552435in}{2.375818in}}%
\pgfpathlineto{\pgfqpoint{4.544842in}{2.365787in}}%
\pgfpathlineto{\pgfqpoint{4.537243in}{2.355656in}}%
\pgfpathclose%
\pgfusepath{fill}%
\end{pgfscope}%
\begin{pgfscope}%
\pgfpathrectangle{\pgfqpoint{1.254980in}{0.150000in}}{\pgfqpoint{5.490039in}{5.490039in}}%
\pgfusepath{clip}%
\pgfsetbuttcap%
\pgfsetroundjoin%
\definecolor{currentfill}{rgb}{0.227802,0.326594,0.546532}%
\pgfsetfillcolor{currentfill}%
\pgfsetfillopacity{0.700000}%
\pgfsetlinewidth{0.000000pt}%
\definecolor{currentstroke}{rgb}{0.000000,0.000000,0.000000}%
\pgfsetstrokecolor{currentstroke}%
\pgfsetdash{}{0pt}%
\pgfpathmoveto{\pgfqpoint{2.418421in}{2.302790in}}%
\pgfpathlineto{\pgfqpoint{2.431998in}{2.283406in}}%
\pgfpathlineto{\pgfqpoint{2.445564in}{2.264285in}}%
\pgfpathlineto{\pgfqpoint{2.459119in}{2.245426in}}%
\pgfpathlineto{\pgfqpoint{2.472665in}{2.226826in}}%
\pgfpathlineto{\pgfqpoint{2.481343in}{2.224676in}}%
\pgfpathlineto{\pgfqpoint{2.490002in}{2.222811in}}%
\pgfpathlineto{\pgfqpoint{2.498642in}{2.221226in}}%
\pgfpathlineto{\pgfqpoint{2.507264in}{2.219917in}}%
\pgfpathlineto{\pgfqpoint{2.493767in}{2.237959in}}%
\pgfpathlineto{\pgfqpoint{2.480260in}{2.256260in}}%
\pgfpathlineto{\pgfqpoint{2.466744in}{2.274821in}}%
\pgfpathlineto{\pgfqpoint{2.453218in}{2.293645in}}%
\pgfpathlineto{\pgfqpoint{2.444547in}{2.295501in}}%
\pgfpathlineto{\pgfqpoint{2.435858in}{2.297641in}}%
\pgfpathlineto{\pgfqpoint{2.427149in}{2.300068in}}%
\pgfpathlineto{\pgfqpoint{2.418421in}{2.302790in}}%
\pgfpathclose%
\pgfusepath{fill}%
\end{pgfscope}%
\begin{pgfscope}%
\pgfpathrectangle{\pgfqpoint{1.254980in}{0.150000in}}{\pgfqpoint{5.490039in}{5.490039in}}%
\pgfusepath{clip}%
\pgfsetbuttcap%
\pgfsetroundjoin%
\definecolor{currentfill}{rgb}{0.277134,0.185228,0.489898}%
\pgfsetfillcolor{currentfill}%
\pgfsetfillopacity{0.700000}%
\pgfsetlinewidth{0.000000pt}%
\definecolor{currentstroke}{rgb}{0.000000,0.000000,0.000000}%
\pgfsetstrokecolor{currentstroke}%
\pgfsetdash{}{0pt}%
\pgfpathmoveto{\pgfqpoint{4.051535in}{1.925326in}}%
\pgfpathlineto{\pgfqpoint{4.064964in}{1.929158in}}%
\pgfpathlineto{\pgfqpoint{4.078404in}{1.933154in}}%
\pgfpathlineto{\pgfqpoint{4.091854in}{1.937313in}}%
\pgfpathlineto{\pgfqpoint{4.105315in}{1.941635in}}%
\pgfpathlineto{\pgfqpoint{4.113071in}{1.953426in}}%
\pgfpathlineto{\pgfqpoint{4.120822in}{1.965170in}}%
\pgfpathlineto{\pgfqpoint{4.128569in}{1.976866in}}%
\pgfpathlineto{\pgfqpoint{4.136311in}{1.988511in}}%
\pgfpathlineto{\pgfqpoint{4.122853in}{1.983970in}}%
\pgfpathlineto{\pgfqpoint{4.109407in}{1.979592in}}%
\pgfpathlineto{\pgfqpoint{4.095972in}{1.975377in}}%
\pgfpathlineto{\pgfqpoint{4.082547in}{1.971326in}}%
\pgfpathlineto{\pgfqpoint{4.074801in}{1.959889in}}%
\pgfpathlineto{\pgfqpoint{4.067050in}{1.948409in}}%
\pgfpathlineto{\pgfqpoint{4.059295in}{1.936887in}}%
\pgfpathlineto{\pgfqpoint{4.051535in}{1.925326in}}%
\pgfpathclose%
\pgfusepath{fill}%
\end{pgfscope}%
\begin{pgfscope}%
\pgfpathrectangle{\pgfqpoint{1.254980in}{0.150000in}}{\pgfqpoint{5.490039in}{5.490039in}}%
\pgfusepath{clip}%
\pgfsetbuttcap%
\pgfsetroundjoin%
\definecolor{currentfill}{rgb}{0.283187,0.125848,0.444960}%
\pgfsetfillcolor{currentfill}%
\pgfsetfillopacity{0.700000}%
\pgfsetlinewidth{0.000000pt}%
\definecolor{currentstroke}{rgb}{0.000000,0.000000,0.000000}%
\pgfsetstrokecolor{currentstroke}%
\pgfsetdash{}{0pt}%
\pgfpathmoveto{\pgfqpoint{2.795447in}{1.852986in}}%
\pgfpathlineto{\pgfqpoint{2.808825in}{1.840240in}}%
\pgfpathlineto{\pgfqpoint{2.822199in}{1.827707in}}%
\pgfpathlineto{\pgfqpoint{2.835569in}{1.815385in}}%
\pgfpathlineto{\pgfqpoint{2.848936in}{1.803274in}}%
\pgfpathlineto{\pgfqpoint{2.857311in}{1.805017in}}%
\pgfpathlineto{\pgfqpoint{2.865673in}{1.806992in}}%
\pgfpathlineto{\pgfqpoint{2.874020in}{1.809192in}}%
\pgfpathlineto{\pgfqpoint{2.882354in}{1.811613in}}%
\pgfpathlineto{\pgfqpoint{2.869024in}{1.823189in}}%
\pgfpathlineto{\pgfqpoint{2.855691in}{1.834975in}}%
\pgfpathlineto{\pgfqpoint{2.842354in}{1.846972in}}%
\pgfpathlineto{\pgfqpoint{2.829014in}{1.859181in}}%
\pgfpathlineto{\pgfqpoint{2.820644in}{1.857285in}}%
\pgfpathlineto{\pgfqpoint{2.812259in}{1.855617in}}%
\pgfpathlineto{\pgfqpoint{2.803861in}{1.854182in}}%
\pgfpathlineto{\pgfqpoint{2.795447in}{1.852986in}}%
\pgfpathclose%
\pgfusepath{fill}%
\end{pgfscope}%
\begin{pgfscope}%
\pgfpathrectangle{\pgfqpoint{1.254980in}{0.150000in}}{\pgfqpoint{5.490039in}{5.490039in}}%
\pgfusepath{clip}%
\pgfsetbuttcap%
\pgfsetroundjoin%
\definecolor{currentfill}{rgb}{0.127568,0.566949,0.550556}%
\pgfsetfillcolor{currentfill}%
\pgfsetfillopacity{0.700000}%
\pgfsetlinewidth{0.000000pt}%
\definecolor{currentstroke}{rgb}{0.000000,0.000000,0.000000}%
\pgfsetstrokecolor{currentstroke}%
\pgfsetdash{}{0pt}%
\pgfpathmoveto{\pgfqpoint{5.107905in}{2.854074in}}%
\pgfpathlineto{\pgfqpoint{5.121883in}{2.865252in}}%
\pgfpathlineto{\pgfqpoint{5.135878in}{2.876590in}}%
\pgfpathlineto{\pgfqpoint{5.149892in}{2.888088in}}%
\pgfpathlineto{\pgfqpoint{5.163923in}{2.899747in}}%
\pgfpathlineto{\pgfqpoint{5.171246in}{2.905683in}}%
\pgfpathlineto{\pgfqpoint{5.178561in}{2.911511in}}%
\pgfpathlineto{\pgfqpoint{5.185868in}{2.917234in}}%
\pgfpathlineto{\pgfqpoint{5.193167in}{2.922855in}}%
\pgfpathlineto{\pgfqpoint{5.179148in}{2.911420in}}%
\pgfpathlineto{\pgfqpoint{5.165147in}{2.900145in}}%
\pgfpathlineto{\pgfqpoint{5.151163in}{2.889029in}}%
\pgfpathlineto{\pgfqpoint{5.137197in}{2.878073in}}%
\pgfpathlineto{\pgfqpoint{5.129886in}{2.872219in}}%
\pgfpathlineto{\pgfqpoint{5.122567in}{2.866269in}}%
\pgfpathlineto{\pgfqpoint{5.115240in}{2.860222in}}%
\pgfpathlineto{\pgfqpoint{5.107905in}{2.854074in}}%
\pgfpathclose%
\pgfusepath{fill}%
\end{pgfscope}%
\begin{pgfscope}%
\pgfpathrectangle{\pgfqpoint{1.254980in}{0.150000in}}{\pgfqpoint{5.490039in}{5.490039in}}%
\pgfusepath{clip}%
\pgfsetbuttcap%
\pgfsetroundjoin%
\definecolor{currentfill}{rgb}{0.239346,0.300855,0.540844}%
\pgfsetfillcolor{currentfill}%
\pgfsetfillopacity{0.700000}%
\pgfsetlinewidth{0.000000pt}%
\definecolor{currentstroke}{rgb}{0.000000,0.000000,0.000000}%
\pgfsetstrokecolor{currentstroke}%
\pgfsetdash{}{0pt}%
\pgfpathmoveto{\pgfqpoint{4.336823in}{2.169838in}}%
\pgfpathlineto{\pgfqpoint{4.350377in}{2.176328in}}%
\pgfpathlineto{\pgfqpoint{4.363943in}{2.182980in}}%
\pgfpathlineto{\pgfqpoint{4.377521in}{2.189794in}}%
\pgfpathlineto{\pgfqpoint{4.391113in}{2.196769in}}%
\pgfpathlineto{\pgfqpoint{4.398782in}{2.207976in}}%
\pgfpathlineto{\pgfqpoint{4.406445in}{2.219093in}}%
\pgfpathlineto{\pgfqpoint{4.414103in}{2.230120in}}%
\pgfpathlineto{\pgfqpoint{4.421756in}{2.241056in}}%
\pgfpathlineto{\pgfqpoint{4.408167in}{2.233975in}}%
\pgfpathlineto{\pgfqpoint{4.394591in}{2.227056in}}%
\pgfpathlineto{\pgfqpoint{4.381028in}{2.220298in}}%
\pgfpathlineto{\pgfqpoint{4.367478in}{2.213702in}}%
\pgfpathlineto{\pgfqpoint{4.359822in}{2.202861in}}%
\pgfpathlineto{\pgfqpoint{4.352161in}{2.191936in}}%
\pgfpathlineto{\pgfqpoint{4.344495in}{2.180928in}}%
\pgfpathlineto{\pgfqpoint{4.336823in}{2.169838in}}%
\pgfpathclose%
\pgfusepath{fill}%
\end{pgfscope}%
\begin{pgfscope}%
\pgfpathrectangle{\pgfqpoint{1.254980in}{0.150000in}}{\pgfqpoint{5.490039in}{5.490039in}}%
\pgfusepath{clip}%
\pgfsetbuttcap%
\pgfsetroundjoin%
\definecolor{currentfill}{rgb}{0.160665,0.478540,0.558115}%
\pgfsetfillcolor{currentfill}%
\pgfsetfillopacity{0.700000}%
\pgfsetlinewidth{0.000000pt}%
\definecolor{currentstroke}{rgb}{0.000000,0.000000,0.000000}%
\pgfsetstrokecolor{currentstroke}%
\pgfsetdash{}{0pt}%
\pgfpathmoveto{\pgfqpoint{4.822641in}{2.611494in}}%
\pgfpathlineto{\pgfqpoint{4.836452in}{2.621334in}}%
\pgfpathlineto{\pgfqpoint{4.850280in}{2.631334in}}%
\pgfpathlineto{\pgfqpoint{4.864124in}{2.641495in}}%
\pgfpathlineto{\pgfqpoint{4.877984in}{2.651817in}}%
\pgfpathlineto{\pgfqpoint{4.885459in}{2.660052in}}%
\pgfpathlineto{\pgfqpoint{4.892926in}{2.668170in}}%
\pgfpathlineto{\pgfqpoint{4.900386in}{2.676172in}}%
\pgfpathlineto{\pgfqpoint{4.907838in}{2.684059in}}%
\pgfpathlineto{\pgfqpoint{4.893985in}{2.673838in}}%
\pgfpathlineto{\pgfqpoint{4.880148in}{2.663779in}}%
\pgfpathlineto{\pgfqpoint{4.866328in}{2.653879in}}%
\pgfpathlineto{\pgfqpoint{4.852523in}{2.644141in}}%
\pgfpathlineto{\pgfqpoint{4.845063in}{2.636141in}}%
\pgfpathlineto{\pgfqpoint{4.837596in}{2.628035in}}%
\pgfpathlineto{\pgfqpoint{4.830122in}{2.619820in}}%
\pgfpathlineto{\pgfqpoint{4.822641in}{2.611494in}}%
\pgfpathclose%
\pgfusepath{fill}%
\end{pgfscope}%
\begin{pgfscope}%
\pgfpathrectangle{\pgfqpoint{1.254980in}{0.150000in}}{\pgfqpoint{5.490039in}{5.490039in}}%
\pgfusepath{clip}%
\pgfsetbuttcap%
\pgfsetroundjoin%
\definecolor{currentfill}{rgb}{0.166383,0.690856,0.496502}%
\pgfsetfillcolor{currentfill}%
\pgfsetfillopacity{0.700000}%
\pgfsetlinewidth{0.000000pt}%
\definecolor{currentstroke}{rgb}{0.000000,0.000000,0.000000}%
\pgfsetstrokecolor{currentstroke}%
\pgfsetdash{}{0pt}%
\pgfpathmoveto{\pgfqpoint{5.562967in}{3.198187in}}%
\pgfpathlineto{\pgfqpoint{5.577220in}{3.210755in}}%
\pgfpathlineto{\pgfqpoint{5.591492in}{3.223482in}}%
\pgfpathlineto{\pgfqpoint{5.605785in}{3.236369in}}%
\pgfpathlineto{\pgfqpoint{5.620097in}{3.249415in}}%
\pgfpathlineto{\pgfqpoint{5.627130in}{3.251685in}}%
\pgfpathlineto{\pgfqpoint{5.634154in}{3.253896in}}%
\pgfpathlineto{\pgfqpoint{5.641171in}{3.256053in}}%
\pgfpathlineto{\pgfqpoint{5.648179in}{3.258159in}}%
\pgfpathlineto{\pgfqpoint{5.633890in}{3.245523in}}%
\pgfpathlineto{\pgfqpoint{5.619621in}{3.233045in}}%
\pgfpathlineto{\pgfqpoint{5.605372in}{3.220726in}}%
\pgfpathlineto{\pgfqpoint{5.591142in}{3.208565in}}%
\pgfpathlineto{\pgfqpoint{5.584110in}{3.206039in}}%
\pgfpathlineto{\pgfqpoint{5.577070in}{3.203471in}}%
\pgfpathlineto{\pgfqpoint{5.570023in}{3.200855in}}%
\pgfpathlineto{\pgfqpoint{5.562967in}{3.198187in}}%
\pgfpathclose%
\pgfusepath{fill}%
\end{pgfscope}%
\begin{pgfscope}%
\pgfpathrectangle{\pgfqpoint{1.254980in}{0.150000in}}{\pgfqpoint{5.490039in}{5.490039in}}%
\pgfusepath{clip}%
\pgfsetbuttcap%
\pgfsetroundjoin%
\definecolor{currentfill}{rgb}{0.268510,0.009605,0.335427}%
\pgfsetfillcolor{currentfill}%
\pgfsetfillopacity{0.700000}%
\pgfsetlinewidth{0.000000pt}%
\definecolor{currentstroke}{rgb}{0.000000,0.000000,0.000000}%
\pgfsetstrokecolor{currentstroke}%
\pgfsetdash{}{0pt}%
\pgfpathmoveto{\pgfqpoint{3.234364in}{1.617343in}}%
\pgfpathlineto{\pgfqpoint{3.247654in}{1.611132in}}%
\pgfpathlineto{\pgfqpoint{3.260946in}{1.605102in}}%
\pgfpathlineto{\pgfqpoint{3.274239in}{1.599254in}}%
\pgfpathlineto{\pgfqpoint{3.287535in}{1.593587in}}%
\pgfpathlineto{\pgfqpoint{3.295615in}{1.600404in}}%
\pgfpathlineto{\pgfqpoint{3.303687in}{1.607360in}}%
\pgfpathlineto{\pgfqpoint{3.311750in}{1.614451in}}%
\pgfpathlineto{\pgfqpoint{3.319804in}{1.621673in}}%
\pgfpathlineto{\pgfqpoint{3.306530in}{1.626873in}}%
\pgfpathlineto{\pgfqpoint{3.293258in}{1.632253in}}%
\pgfpathlineto{\pgfqpoint{3.279989in}{1.637814in}}%
\pgfpathlineto{\pgfqpoint{3.266722in}{1.643557in}}%
\pgfpathlineto{\pgfqpoint{3.258646in}{1.636793in}}%
\pgfpathlineto{\pgfqpoint{3.250561in}{1.630166in}}%
\pgfpathlineto{\pgfqpoint{3.242467in}{1.623681in}}%
\pgfpathlineto{\pgfqpoint{3.234364in}{1.617343in}}%
\pgfpathclose%
\pgfusepath{fill}%
\end{pgfscope}%
\begin{pgfscope}%
\pgfpathrectangle{\pgfqpoint{1.254980in}{0.150000in}}{\pgfqpoint{5.490039in}{5.490039in}}%
\pgfusepath{clip}%
\pgfsetbuttcap%
\pgfsetroundjoin%
\definecolor{currentfill}{rgb}{0.212395,0.359683,0.551710}%
\pgfsetfillcolor{currentfill}%
\pgfsetfillopacity{0.700000}%
\pgfsetlinewidth{0.000000pt}%
\definecolor{currentstroke}{rgb}{0.000000,0.000000,0.000000}%
\pgfsetstrokecolor{currentstroke}%
\pgfsetdash{}{0pt}%
\pgfpathmoveto{\pgfqpoint{2.364006in}{2.383012in}}%
\pgfpathlineto{\pgfqpoint{2.377627in}{2.362549in}}%
\pgfpathlineto{\pgfqpoint{2.391236in}{2.342359in}}%
\pgfpathlineto{\pgfqpoint{2.404834in}{2.322440in}}%
\pgfpathlineto{\pgfqpoint{2.418421in}{2.302790in}}%
\pgfpathlineto{\pgfqpoint{2.427149in}{2.300068in}}%
\pgfpathlineto{\pgfqpoint{2.435858in}{2.297641in}}%
\pgfpathlineto{\pgfqpoint{2.444547in}{2.295501in}}%
\pgfpathlineto{\pgfqpoint{2.453218in}{2.293645in}}%
\pgfpathlineto{\pgfqpoint{2.439681in}{2.312734in}}%
\pgfpathlineto{\pgfqpoint{2.426134in}{2.332090in}}%
\pgfpathlineto{\pgfqpoint{2.412576in}{2.351716in}}%
\pgfpathlineto{\pgfqpoint{2.399008in}{2.371614in}}%
\pgfpathlineto{\pgfqpoint{2.390287in}{2.374021in}}%
\pgfpathlineto{\pgfqpoint{2.381547in}{2.376719in}}%
\pgfpathlineto{\pgfqpoint{2.372786in}{2.379714in}}%
\pgfpathlineto{\pgfqpoint{2.364006in}{2.383012in}}%
\pgfpathclose%
\pgfusepath{fill}%
\end{pgfscope}%
\begin{pgfscope}%
\pgfpathrectangle{\pgfqpoint{1.254980in}{0.150000in}}{\pgfqpoint{5.490039in}{5.490039in}}%
\pgfusepath{clip}%
\pgfsetbuttcap%
\pgfsetroundjoin%
\definecolor{currentfill}{rgb}{0.282910,0.105393,0.426902}%
\pgfsetfillcolor{currentfill}%
\pgfsetfillopacity{0.700000}%
\pgfsetlinewidth{0.000000pt}%
\definecolor{currentstroke}{rgb}{0.000000,0.000000,0.000000}%
\pgfsetstrokecolor{currentstroke}%
\pgfsetdash{}{0pt}%
\pgfpathmoveto{\pgfqpoint{2.848936in}{1.803274in}}%
\pgfpathlineto{\pgfqpoint{2.862299in}{1.791371in}}%
\pgfpathlineto{\pgfqpoint{2.875660in}{1.779675in}}%
\pgfpathlineto{\pgfqpoint{2.889017in}{1.768186in}}%
\pgfpathlineto{\pgfqpoint{2.902372in}{1.756902in}}%
\pgfpathlineto{\pgfqpoint{2.910711in}{1.759191in}}%
\pgfpathlineto{\pgfqpoint{2.919037in}{1.761703in}}%
\pgfpathlineto{\pgfqpoint{2.927350in}{1.764434in}}%
\pgfpathlineto{\pgfqpoint{2.935649in}{1.767377in}}%
\pgfpathlineto{\pgfqpoint{2.922329in}{1.778128in}}%
\pgfpathlineto{\pgfqpoint{2.909007in}{1.789083in}}%
\pgfpathlineto{\pgfqpoint{2.895682in}{1.800244in}}%
\pgfpathlineto{\pgfqpoint{2.882354in}{1.811613in}}%
\pgfpathlineto{\pgfqpoint{2.874020in}{1.809192in}}%
\pgfpathlineto{\pgfqpoint{2.865673in}{1.806992in}}%
\pgfpathlineto{\pgfqpoint{2.857311in}{1.805017in}}%
\pgfpathlineto{\pgfqpoint{2.848936in}{1.803274in}}%
\pgfpathclose%
\pgfusepath{fill}%
\end{pgfscope}%
\begin{pgfscope}%
\pgfpathrectangle{\pgfqpoint{1.254980in}{0.150000in}}{\pgfqpoint{5.490039in}{5.490039in}}%
\pgfusepath{clip}%
\pgfsetbuttcap%
\pgfsetroundjoin%
\definecolor{currentfill}{rgb}{0.267004,0.004874,0.329415}%
\pgfsetfillcolor{currentfill}%
\pgfsetfillopacity{0.700000}%
\pgfsetlinewidth{0.000000pt}%
\definecolor{currentstroke}{rgb}{0.000000,0.000000,0.000000}%
\pgfsetstrokecolor{currentstroke}%
\pgfsetdash{}{0pt}%
\pgfpathmoveto{\pgfqpoint{3.372931in}{1.602665in}}%
\pgfpathlineto{\pgfqpoint{3.386220in}{1.598356in}}%
\pgfpathlineto{\pgfqpoint{3.399514in}{1.594224in}}%
\pgfpathlineto{\pgfqpoint{3.412811in}{1.590267in}}%
\pgfpathlineto{\pgfqpoint{3.426111in}{1.586485in}}%
\pgfpathlineto{\pgfqpoint{3.434119in}{1.594734in}}%
\pgfpathlineto{\pgfqpoint{3.442120in}{1.603090in}}%
\pgfpathlineto{\pgfqpoint{3.450113in}{1.611549in}}%
\pgfpathlineto{\pgfqpoint{3.458099in}{1.620108in}}%
\pgfpathlineto{\pgfqpoint{3.444815in}{1.623451in}}%
\pgfpathlineto{\pgfqpoint{3.431536in}{1.626968in}}%
\pgfpathlineto{\pgfqpoint{3.418261in}{1.630661in}}%
\pgfpathlineto{\pgfqpoint{3.404990in}{1.634529in}}%
\pgfpathlineto{\pgfqpoint{3.396986in}{1.626400in}}%
\pgfpathlineto{\pgfqpoint{3.388975in}{1.618376in}}%
\pgfpathlineto{\pgfqpoint{3.380957in}{1.610463in}}%
\pgfpathlineto{\pgfqpoint{3.372931in}{1.602665in}}%
\pgfpathclose%
\pgfusepath{fill}%
\end{pgfscope}%
\begin{pgfscope}%
\pgfpathrectangle{\pgfqpoint{1.254980in}{0.150000in}}{\pgfqpoint{5.490039in}{5.490039in}}%
\pgfusepath{clip}%
\pgfsetbuttcap%
\pgfsetroundjoin%
\definecolor{currentfill}{rgb}{0.269308,0.218818,0.509577}%
\pgfsetfillcolor{currentfill}%
\pgfsetfillopacity{0.700000}%
\pgfsetlinewidth{0.000000pt}%
\definecolor{currentstroke}{rgb}{0.000000,0.000000,0.000000}%
\pgfsetstrokecolor{currentstroke}%
\pgfsetdash{}{0pt}%
\pgfpathmoveto{\pgfqpoint{4.136311in}{1.988511in}}%
\pgfpathlineto{\pgfqpoint{4.149779in}{1.993215in}}%
\pgfpathlineto{\pgfqpoint{4.163258in}{1.998081in}}%
\pgfpathlineto{\pgfqpoint{4.176748in}{2.003110in}}%
\pgfpathlineto{\pgfqpoint{4.190250in}{2.008302in}}%
\pgfpathlineto{\pgfqpoint{4.197984in}{2.020096in}}%
\pgfpathlineto{\pgfqpoint{4.205713in}{2.031830in}}%
\pgfpathlineto{\pgfqpoint{4.213437in}{2.043501in}}%
\pgfpathlineto{\pgfqpoint{4.221156in}{2.055107in}}%
\pgfpathlineto{\pgfqpoint{4.207658in}{2.049725in}}%
\pgfpathlineto{\pgfqpoint{4.194171in}{2.044505in}}%
\pgfpathlineto{\pgfqpoint{4.180695in}{2.039447in}}%
\pgfpathlineto{\pgfqpoint{4.167230in}{2.034553in}}%
\pgfpathlineto{\pgfqpoint{4.159508in}{2.023126in}}%
\pgfpathlineto{\pgfqpoint{4.151780in}{2.011643in}}%
\pgfpathlineto{\pgfqpoint{4.144048in}{2.000104in}}%
\pgfpathlineto{\pgfqpoint{4.136311in}{1.988511in}}%
\pgfpathclose%
\pgfusepath{fill}%
\end{pgfscope}%
\begin{pgfscope}%
\pgfpathrectangle{\pgfqpoint{1.254980in}{0.150000in}}{\pgfqpoint{5.490039in}{5.490039in}}%
\pgfusepath{clip}%
\pgfsetbuttcap%
\pgfsetroundjoin%
\definecolor{currentfill}{rgb}{0.202219,0.715272,0.476084}%
\pgfsetfillcolor{currentfill}%
\pgfsetfillopacity{0.700000}%
\pgfsetlinewidth{0.000000pt}%
\definecolor{currentstroke}{rgb}{0.000000,0.000000,0.000000}%
\pgfsetstrokecolor{currentstroke}%
\pgfsetdash{}{0pt}%
\pgfpathmoveto{\pgfqpoint{5.648179in}{3.258159in}}%
\pgfpathlineto{\pgfqpoint{5.662488in}{3.270954in}}%
\pgfpathlineto{\pgfqpoint{5.676817in}{3.283908in}}%
\pgfpathlineto{\pgfqpoint{5.691166in}{3.297021in}}%
\pgfpathlineto{\pgfqpoint{5.705536in}{3.310293in}}%
\pgfpathlineto{\pgfqpoint{5.712511in}{3.311925in}}%
\pgfpathlineto{\pgfqpoint{5.719478in}{3.313509in}}%
\pgfpathlineto{\pgfqpoint{5.726436in}{3.315051in}}%
\pgfpathlineto{\pgfqpoint{5.733387in}{3.316554in}}%
\pgfpathlineto{\pgfqpoint{5.719043in}{3.303723in}}%
\pgfpathlineto{\pgfqpoint{5.704719in}{3.291051in}}%
\pgfpathlineto{\pgfqpoint{5.690416in}{3.278536in}}%
\pgfpathlineto{\pgfqpoint{5.676132in}{3.266180in}}%
\pgfpathlineto{\pgfqpoint{5.669155in}{3.264226in}}%
\pgfpathlineto{\pgfqpoint{5.662171in}{3.262241in}}%
\pgfpathlineto{\pgfqpoint{5.655179in}{3.260221in}}%
\pgfpathlineto{\pgfqpoint{5.648179in}{3.258159in}}%
\pgfpathclose%
\pgfusepath{fill}%
\end{pgfscope}%
\begin{pgfscope}%
\pgfpathrectangle{\pgfqpoint{1.254980in}{0.150000in}}{\pgfqpoint{5.490039in}{5.490039in}}%
\pgfusepath{clip}%
\pgfsetbuttcap%
\pgfsetroundjoin%
\definecolor{currentfill}{rgb}{0.188923,0.410910,0.556326}%
\pgfsetfillcolor{currentfill}%
\pgfsetfillopacity{0.700000}%
\pgfsetlinewidth{0.000000pt}%
\definecolor{currentstroke}{rgb}{0.000000,0.000000,0.000000}%
\pgfsetstrokecolor{currentstroke}%
\pgfsetdash{}{0pt}%
\pgfpathmoveto{\pgfqpoint{4.622275in}{2.428670in}}%
\pgfpathlineto{\pgfqpoint{4.635980in}{2.437346in}}%
\pgfpathlineto{\pgfqpoint{4.649699in}{2.446183in}}%
\pgfpathlineto{\pgfqpoint{4.663433in}{2.455180in}}%
\pgfpathlineto{\pgfqpoint{4.677182in}{2.464339in}}%
\pgfpathlineto{\pgfqpoint{4.684749in}{2.474070in}}%
\pgfpathlineto{\pgfqpoint{4.692309in}{2.483687in}}%
\pgfpathlineto{\pgfqpoint{4.699863in}{2.493190in}}%
\pgfpathlineto{\pgfqpoint{4.707410in}{2.502581in}}%
\pgfpathlineto{\pgfqpoint{4.693665in}{2.493434in}}%
\pgfpathlineto{\pgfqpoint{4.679936in}{2.484448in}}%
\pgfpathlineto{\pgfqpoint{4.666221in}{2.475622in}}%
\pgfpathlineto{\pgfqpoint{4.652521in}{2.466958in}}%
\pgfpathlineto{\pgfqpoint{4.644969in}{2.457544in}}%
\pgfpathlineto{\pgfqpoint{4.637411in}{2.448026in}}%
\pgfpathlineto{\pgfqpoint{4.629846in}{2.438401in}}%
\pgfpathlineto{\pgfqpoint{4.622275in}{2.428670in}}%
\pgfpathclose%
\pgfusepath{fill}%
\end{pgfscope}%
\begin{pgfscope}%
\pgfpathrectangle{\pgfqpoint{1.254980in}{0.150000in}}{\pgfqpoint{5.490039in}{5.490039in}}%
\pgfusepath{clip}%
\pgfsetbuttcap%
\pgfsetroundjoin%
\definecolor{currentfill}{rgb}{0.272594,0.025563,0.353093}%
\pgfsetfillcolor{currentfill}%
\pgfsetfillopacity{0.700000}%
\pgfsetlinewidth{0.000000pt}%
\definecolor{currentstroke}{rgb}{0.000000,0.000000,0.000000}%
\pgfsetstrokecolor{currentstroke}%
\pgfsetdash{}{0pt}%
\pgfpathmoveto{\pgfqpoint{3.095382in}{1.653884in}}%
\pgfpathlineto{\pgfqpoint{3.108689in}{1.645687in}}%
\pgfpathlineto{\pgfqpoint{3.121997in}{1.637679in}}%
\pgfpathlineto{\pgfqpoint{3.135305in}{1.629859in}}%
\pgfpathlineto{\pgfqpoint{3.148614in}{1.622227in}}%
\pgfpathlineto{\pgfqpoint{3.156781in}{1.627422in}}%
\pgfpathlineto{\pgfqpoint{3.164938in}{1.632790in}}%
\pgfpathlineto{\pgfqpoint{3.173085in}{1.638327in}}%
\pgfpathlineto{\pgfqpoint{3.181222in}{1.644027in}}%
\pgfpathlineto{\pgfqpoint{3.167940in}{1.651161in}}%
\pgfpathlineto{\pgfqpoint{3.154659in}{1.658482in}}%
\pgfpathlineto{\pgfqpoint{3.141378in}{1.665992in}}%
\pgfpathlineto{\pgfqpoint{3.128098in}{1.673690in}}%
\pgfpathlineto{\pgfqpoint{3.119935in}{1.668478in}}%
\pgfpathlineto{\pgfqpoint{3.111761in}{1.663436in}}%
\pgfpathlineto{\pgfqpoint{3.103577in}{1.658570in}}%
\pgfpathlineto{\pgfqpoint{3.095382in}{1.653884in}}%
\pgfpathclose%
\pgfusepath{fill}%
\end{pgfscope}%
\begin{pgfscope}%
\pgfpathrectangle{\pgfqpoint{1.254980in}{0.150000in}}{\pgfqpoint{5.490039in}{5.490039in}}%
\pgfusepath{clip}%
\pgfsetbuttcap%
\pgfsetroundjoin%
\definecolor{currentfill}{rgb}{0.120565,0.596422,0.543611}%
\pgfsetfillcolor{currentfill}%
\pgfsetfillopacity{0.700000}%
\pgfsetlinewidth{0.000000pt}%
\definecolor{currentstroke}{rgb}{0.000000,0.000000,0.000000}%
\pgfsetstrokecolor{currentstroke}%
\pgfsetdash{}{0pt}%
\pgfpathmoveto{\pgfqpoint{5.193167in}{2.922855in}}%
\pgfpathlineto{\pgfqpoint{5.207204in}{2.934450in}}%
\pgfpathlineto{\pgfqpoint{5.221259in}{2.946205in}}%
\pgfpathlineto{\pgfqpoint{5.235332in}{2.958121in}}%
\pgfpathlineto{\pgfqpoint{5.249424in}{2.970197in}}%
\pgfpathlineto{\pgfqpoint{5.256702in}{2.975475in}}%
\pgfpathlineto{\pgfqpoint{5.263970in}{2.980649in}}%
\pgfpathlineto{\pgfqpoint{5.271231in}{2.985721in}}%
\pgfpathlineto{\pgfqpoint{5.278483in}{2.990696in}}%
\pgfpathlineto{\pgfqpoint{5.264405in}{2.978875in}}%
\pgfpathlineto{\pgfqpoint{5.250346in}{2.967214in}}%
\pgfpathlineto{\pgfqpoint{5.236304in}{2.955713in}}%
\pgfpathlineto{\pgfqpoint{5.222281in}{2.944371in}}%
\pgfpathlineto{\pgfqpoint{5.215015in}{2.939131in}}%
\pgfpathlineto{\pgfqpoint{5.207740in}{2.933801in}}%
\pgfpathlineto{\pgfqpoint{5.200458in}{2.928376in}}%
\pgfpathlineto{\pgfqpoint{5.193167in}{2.922855in}}%
\pgfpathclose%
\pgfusepath{fill}%
\end{pgfscope}%
\begin{pgfscope}%
\pgfpathrectangle{\pgfqpoint{1.254980in}{0.150000in}}{\pgfqpoint{5.490039in}{5.490039in}}%
\pgfusepath{clip}%
\pgfsetbuttcap%
\pgfsetroundjoin%
\definecolor{currentfill}{rgb}{0.277941,0.056324,0.381191}%
\pgfsetfillcolor{currentfill}%
\pgfsetfillopacity{0.700000}%
\pgfsetlinewidth{0.000000pt}%
\definecolor{currentstroke}{rgb}{0.000000,0.000000,0.000000}%
\pgfsetstrokecolor{currentstroke}%
\pgfsetdash{}{0pt}%
\pgfpathmoveto{\pgfqpoint{3.681195in}{1.674548in}}%
\pgfpathlineto{\pgfqpoint{3.694523in}{1.674224in}}%
\pgfpathlineto{\pgfqpoint{3.707858in}{1.674069in}}%
\pgfpathlineto{\pgfqpoint{3.721200in}{1.674081in}}%
\pgfpathlineto{\pgfqpoint{3.734549in}{1.674259in}}%
\pgfpathlineto{\pgfqpoint{3.742427in}{1.685004in}}%
\pgfpathlineto{\pgfqpoint{3.750300in}{1.695782in}}%
\pgfpathlineto{\pgfqpoint{3.758167in}{1.706591in}}%
\pgfpathlineto{\pgfqpoint{3.766030in}{1.717427in}}%
\pgfpathlineto{\pgfqpoint{3.752690in}{1.716892in}}%
\pgfpathlineto{\pgfqpoint{3.739358in}{1.716523in}}%
\pgfpathlineto{\pgfqpoint{3.726033in}{1.716322in}}%
\pgfpathlineto{\pgfqpoint{3.712715in}{1.716289in}}%
\pgfpathlineto{\pgfqpoint{3.704843in}{1.705799in}}%
\pgfpathlineto{\pgfqpoint{3.696966in}{1.695343in}}%
\pgfpathlineto{\pgfqpoint{3.689083in}{1.684925in}}%
\pgfpathlineto{\pgfqpoint{3.681195in}{1.674548in}}%
\pgfpathclose%
\pgfusepath{fill}%
\end{pgfscope}%
\begin{pgfscope}%
\pgfpathrectangle{\pgfqpoint{1.254980in}{0.150000in}}{\pgfqpoint{5.490039in}{5.490039in}}%
\pgfusepath{clip}%
\pgfsetbuttcap%
\pgfsetroundjoin%
\definecolor{currentfill}{rgb}{0.281446,0.084320,0.407414}%
\pgfsetfillcolor{currentfill}%
\pgfsetfillopacity{0.700000}%
\pgfsetlinewidth{0.000000pt}%
\definecolor{currentstroke}{rgb}{0.000000,0.000000,0.000000}%
\pgfsetstrokecolor{currentstroke}%
\pgfsetdash{}{0pt}%
\pgfpathmoveto{\pgfqpoint{3.766030in}{1.717427in}}%
\pgfpathlineto{\pgfqpoint{3.779377in}{1.718129in}}%
\pgfpathlineto{\pgfqpoint{3.792732in}{1.718998in}}%
\pgfpathlineto{\pgfqpoint{3.806095in}{1.720032in}}%
\pgfpathlineto{\pgfqpoint{3.819465in}{1.721232in}}%
\pgfpathlineto{\pgfqpoint{3.827314in}{1.732433in}}%
\pgfpathlineto{\pgfqpoint{3.835158in}{1.743648in}}%
\pgfpathlineto{\pgfqpoint{3.842996in}{1.754874in}}%
\pgfpathlineto{\pgfqpoint{3.850830in}{1.766108in}}%
\pgfpathlineto{\pgfqpoint{3.837467in}{1.764578in}}%
\pgfpathlineto{\pgfqpoint{3.824112in}{1.763214in}}%
\pgfpathlineto{\pgfqpoint{3.810765in}{1.762016in}}%
\pgfpathlineto{\pgfqpoint{3.797427in}{1.760985in}}%
\pgfpathlineto{\pgfqpoint{3.789585in}{1.750070in}}%
\pgfpathlineto{\pgfqpoint{3.781738in}{1.739170in}}%
\pgfpathlineto{\pgfqpoint{3.773887in}{1.728288in}}%
\pgfpathlineto{\pgfqpoint{3.766030in}{1.717427in}}%
\pgfpathclose%
\pgfusepath{fill}%
\end{pgfscope}%
\begin{pgfscope}%
\pgfpathrectangle{\pgfqpoint{1.254980in}{0.150000in}}{\pgfqpoint{5.490039in}{5.490039in}}%
\pgfusepath{clip}%
\pgfsetbuttcap%
\pgfsetroundjoin%
\definecolor{currentfill}{rgb}{0.195860,0.395433,0.555276}%
\pgfsetfillcolor{currentfill}%
\pgfsetfillopacity{0.700000}%
\pgfsetlinewidth{0.000000pt}%
\definecolor{currentstroke}{rgb}{0.000000,0.000000,0.000000}%
\pgfsetstrokecolor{currentstroke}%
\pgfsetdash{}{0pt}%
\pgfpathmoveto{\pgfqpoint{2.309400in}{2.467648in}}%
\pgfpathlineto{\pgfqpoint{2.323070in}{2.446066in}}%
\pgfpathlineto{\pgfqpoint{2.336728in}{2.424768in}}%
\pgfpathlineto{\pgfqpoint{2.350373in}{2.403751in}}%
\pgfpathlineto{\pgfqpoint{2.364006in}{2.383012in}}%
\pgfpathlineto{\pgfqpoint{2.372786in}{2.379714in}}%
\pgfpathlineto{\pgfqpoint{2.381547in}{2.376719in}}%
\pgfpathlineto{\pgfqpoint{2.390287in}{2.374021in}}%
\pgfpathlineto{\pgfqpoint{2.399008in}{2.371614in}}%
\pgfpathlineto{\pgfqpoint{2.385427in}{2.391786in}}%
\pgfpathlineto{\pgfqpoint{2.371836in}{2.412236in}}%
\pgfpathlineto{\pgfqpoint{2.358232in}{2.432965in}}%
\pgfpathlineto{\pgfqpoint{2.344616in}{2.453977in}}%
\pgfpathlineto{\pgfqpoint{2.335843in}{2.456940in}}%
\pgfpathlineto{\pgfqpoint{2.327050in}{2.460202in}}%
\pgfpathlineto{\pgfqpoint{2.318235in}{2.463770in}}%
\pgfpathlineto{\pgfqpoint{2.309400in}{2.467648in}}%
\pgfpathclose%
\pgfusepath{fill}%
\end{pgfscope}%
\begin{pgfscope}%
\pgfpathrectangle{\pgfqpoint{1.254980in}{0.150000in}}{\pgfqpoint{5.490039in}{5.490039in}}%
\pgfusepath{clip}%
\pgfsetbuttcap%
\pgfsetroundjoin%
\definecolor{currentfill}{rgb}{0.221989,0.339161,0.548752}%
\pgfsetfillcolor{currentfill}%
\pgfsetfillopacity{0.700000}%
\pgfsetlinewidth{0.000000pt}%
\definecolor{currentstroke}{rgb}{0.000000,0.000000,0.000000}%
\pgfsetstrokecolor{currentstroke}%
\pgfsetdash{}{0pt}%
\pgfpathmoveto{\pgfqpoint{4.421756in}{2.241056in}}%
\pgfpathlineto{\pgfqpoint{4.435358in}{2.248299in}}%
\pgfpathlineto{\pgfqpoint{4.448974in}{2.255703in}}%
\pgfpathlineto{\pgfqpoint{4.462603in}{2.263268in}}%
\pgfpathlineto{\pgfqpoint{4.476246in}{2.270995in}}%
\pgfpathlineto{\pgfqpoint{4.483890in}{2.281927in}}%
\pgfpathlineto{\pgfqpoint{4.491529in}{2.292760in}}%
\pgfpathlineto{\pgfqpoint{4.499162in}{2.303493in}}%
\pgfpathlineto{\pgfqpoint{4.506790in}{2.314127in}}%
\pgfpathlineto{\pgfqpoint{4.493150in}{2.306323in}}%
\pgfpathlineto{\pgfqpoint{4.479524in}{2.298681in}}%
\pgfpathlineto{\pgfqpoint{4.465911in}{2.291200in}}%
\pgfpathlineto{\pgfqpoint{4.452312in}{2.283881in}}%
\pgfpathlineto{\pgfqpoint{4.444681in}{2.273313in}}%
\pgfpathlineto{\pgfqpoint{4.437045in}{2.262653in}}%
\pgfpathlineto{\pgfqpoint{4.429403in}{2.251900in}}%
\pgfpathlineto{\pgfqpoint{4.421756in}{2.241056in}}%
\pgfpathclose%
\pgfusepath{fill}%
\end{pgfscope}%
\begin{pgfscope}%
\pgfpathrectangle{\pgfqpoint{1.254980in}{0.150000in}}{\pgfqpoint{5.490039in}{5.490039in}}%
\pgfusepath{clip}%
\pgfsetbuttcap%
\pgfsetroundjoin%
\definecolor{currentfill}{rgb}{0.149039,0.508051,0.557250}%
\pgfsetfillcolor{currentfill}%
\pgfsetfillopacity{0.700000}%
\pgfsetlinewidth{0.000000pt}%
\definecolor{currentstroke}{rgb}{0.000000,0.000000,0.000000}%
\pgfsetstrokecolor{currentstroke}%
\pgfsetdash{}{0pt}%
\pgfpathmoveto{\pgfqpoint{4.907838in}{2.684059in}}%
\pgfpathlineto{\pgfqpoint{4.921708in}{2.694439in}}%
\pgfpathlineto{\pgfqpoint{4.935594in}{2.704981in}}%
\pgfpathlineto{\pgfqpoint{4.949497in}{2.715683in}}%
\pgfpathlineto{\pgfqpoint{4.963417in}{2.726546in}}%
\pgfpathlineto{\pgfqpoint{4.970855in}{2.734199in}}%
\pgfpathlineto{\pgfqpoint{4.978285in}{2.741734in}}%
\pgfpathlineto{\pgfqpoint{4.985707in}{2.749152in}}%
\pgfpathlineto{\pgfqpoint{4.993122in}{2.756454in}}%
\pgfpathlineto{\pgfqpoint{4.979210in}{2.745723in}}%
\pgfpathlineto{\pgfqpoint{4.965316in}{2.735153in}}%
\pgfpathlineto{\pgfqpoint{4.951438in}{2.724743in}}%
\pgfpathlineto{\pgfqpoint{4.937576in}{2.714493in}}%
\pgfpathlineto{\pgfqpoint{4.930153in}{2.707048in}}%
\pgfpathlineto{\pgfqpoint{4.922722in}{2.699495in}}%
\pgfpathlineto{\pgfqpoint{4.915284in}{2.691833in}}%
\pgfpathlineto{\pgfqpoint{4.907838in}{2.684059in}}%
\pgfpathclose%
\pgfusepath{fill}%
\end{pgfscope}%
\begin{pgfscope}%
\pgfpathrectangle{\pgfqpoint{1.254980in}{0.150000in}}{\pgfqpoint{5.490039in}{5.490039in}}%
\pgfusepath{clip}%
\pgfsetbuttcap%
\pgfsetroundjoin%
\definecolor{currentfill}{rgb}{0.239374,0.735588,0.455688}%
\pgfsetfillcolor{currentfill}%
\pgfsetfillopacity{0.700000}%
\pgfsetlinewidth{0.000000pt}%
\definecolor{currentstroke}{rgb}{0.000000,0.000000,0.000000}%
\pgfsetstrokecolor{currentstroke}%
\pgfsetdash{}{0pt}%
\pgfpathmoveto{\pgfqpoint{5.733387in}{3.316554in}}%
\pgfpathlineto{\pgfqpoint{5.747751in}{3.329544in}}%
\pgfpathlineto{\pgfqpoint{5.762136in}{3.342692in}}%
\pgfpathlineto{\pgfqpoint{5.776541in}{3.355999in}}%
\pgfpathlineto{\pgfqpoint{5.790968in}{3.369465in}}%
\pgfpathlineto{\pgfqpoint{5.797883in}{3.370475in}}%
\pgfpathlineto{\pgfqpoint{5.804790in}{3.371449in}}%
\pgfpathlineto{\pgfqpoint{5.811690in}{3.372394in}}%
\pgfpathlineto{\pgfqpoint{5.818581in}{3.373314in}}%
\pgfpathlineto{\pgfqpoint{5.804183in}{3.360320in}}%
\pgfpathlineto{\pgfqpoint{5.789806in}{3.347485in}}%
\pgfpathlineto{\pgfqpoint{5.775449in}{3.334807in}}%
\pgfpathlineto{\pgfqpoint{5.761112in}{3.322288in}}%
\pgfpathlineto{\pgfqpoint{5.754192in}{3.320886in}}%
\pgfpathlineto{\pgfqpoint{5.747264in}{3.319467in}}%
\pgfpathlineto{\pgfqpoint{5.740329in}{3.318024in}}%
\pgfpathlineto{\pgfqpoint{5.733387in}{3.316554in}}%
\pgfpathclose%
\pgfusepath{fill}%
\end{pgfscope}%
\begin{pgfscope}%
\pgfpathrectangle{\pgfqpoint{1.254980in}{0.150000in}}{\pgfqpoint{5.490039in}{5.490039in}}%
\pgfusepath{clip}%
\pgfsetbuttcap%
\pgfsetroundjoin%
\definecolor{currentfill}{rgb}{0.273809,0.031497,0.358853}%
\pgfsetfillcolor{currentfill}%
\pgfsetfillopacity{0.700000}%
\pgfsetlinewidth{0.000000pt}%
\definecolor{currentstroke}{rgb}{0.000000,0.000000,0.000000}%
\pgfsetstrokecolor{currentstroke}%
\pgfsetdash{}{0pt}%
\pgfpathmoveto{\pgfqpoint{3.596290in}{1.638034in}}%
\pgfpathlineto{\pgfqpoint{3.609605in}{1.636652in}}%
\pgfpathlineto{\pgfqpoint{3.622925in}{1.635438in}}%
\pgfpathlineto{\pgfqpoint{3.636252in}{1.634394in}}%
\pgfpathlineto{\pgfqpoint{3.649585in}{1.633518in}}%
\pgfpathlineto{\pgfqpoint{3.657496in}{1.643697in}}%
\pgfpathlineto{\pgfqpoint{3.665401in}{1.653930in}}%
\pgfpathlineto{\pgfqpoint{3.673301in}{1.664215in}}%
\pgfpathlineto{\pgfqpoint{3.681195in}{1.674548in}}%
\pgfpathlineto{\pgfqpoint{3.667873in}{1.675039in}}%
\pgfpathlineto{\pgfqpoint{3.654558in}{1.675699in}}%
\pgfpathlineto{\pgfqpoint{3.641250in}{1.676529in}}%
\pgfpathlineto{\pgfqpoint{3.627948in}{1.677527in}}%
\pgfpathlineto{\pgfqpoint{3.620042in}{1.667568in}}%
\pgfpathlineto{\pgfqpoint{3.612131in}{1.657664in}}%
\pgfpathlineto{\pgfqpoint{3.604214in}{1.647818in}}%
\pgfpathlineto{\pgfqpoint{3.596290in}{1.638034in}}%
\pgfpathclose%
\pgfusepath{fill}%
\end{pgfscope}%
\begin{pgfscope}%
\pgfpathrectangle{\pgfqpoint{1.254980in}{0.150000in}}{\pgfqpoint{5.490039in}{5.490039in}}%
\pgfusepath{clip}%
\pgfsetbuttcap%
\pgfsetroundjoin%
\definecolor{currentfill}{rgb}{0.283091,0.110553,0.431554}%
\pgfsetfillcolor{currentfill}%
\pgfsetfillopacity{0.700000}%
\pgfsetlinewidth{0.000000pt}%
\definecolor{currentstroke}{rgb}{0.000000,0.000000,0.000000}%
\pgfsetstrokecolor{currentstroke}%
\pgfsetdash{}{0pt}%
\pgfpathmoveto{\pgfqpoint{3.850830in}{1.766108in}}%
\pgfpathlineto{\pgfqpoint{3.864201in}{1.767803in}}%
\pgfpathlineto{\pgfqpoint{3.877581in}{1.769663in}}%
\pgfpathlineto{\pgfqpoint{3.890970in}{1.771688in}}%
\pgfpathlineto{\pgfqpoint{3.904368in}{1.773877in}}%
\pgfpathlineto{\pgfqpoint{3.912190in}{1.785429in}}%
\pgfpathlineto{\pgfqpoint{3.920007in}{1.796976in}}%
\pgfpathlineto{\pgfqpoint{3.927819in}{1.808516in}}%
\pgfpathlineto{\pgfqpoint{3.935627in}{1.820047in}}%
\pgfpathlineto{\pgfqpoint{3.922236in}{1.817555in}}%
\pgfpathlineto{\pgfqpoint{3.908853in}{1.815228in}}%
\pgfpathlineto{\pgfqpoint{3.895480in}{1.813066in}}%
\pgfpathlineto{\pgfqpoint{3.882116in}{1.811069in}}%
\pgfpathlineto{\pgfqpoint{3.874302in}{1.799830in}}%
\pgfpathlineto{\pgfqpoint{3.866483in}{1.788589in}}%
\pgfpathlineto{\pgfqpoint{3.858659in}{1.777347in}}%
\pgfpathlineto{\pgfqpoint{3.850830in}{1.766108in}}%
\pgfpathclose%
\pgfusepath{fill}%
\end{pgfscope}%
\begin{pgfscope}%
\pgfpathrectangle{\pgfqpoint{1.254980in}{0.150000in}}{\pgfqpoint{5.490039in}{5.490039in}}%
\pgfusepath{clip}%
\pgfsetbuttcap%
\pgfsetroundjoin%
\definecolor{currentfill}{rgb}{0.281446,0.084320,0.407414}%
\pgfsetfillcolor{currentfill}%
\pgfsetfillopacity{0.700000}%
\pgfsetlinewidth{0.000000pt}%
\definecolor{currentstroke}{rgb}{0.000000,0.000000,0.000000}%
\pgfsetstrokecolor{currentstroke}%
\pgfsetdash{}{0pt}%
\pgfpathmoveto{\pgfqpoint{2.902372in}{1.756902in}}%
\pgfpathlineto{\pgfqpoint{2.915724in}{1.745821in}}%
\pgfpathlineto{\pgfqpoint{2.929074in}{1.734943in}}%
\pgfpathlineto{\pgfqpoint{2.942422in}{1.724266in}}%
\pgfpathlineto{\pgfqpoint{2.955769in}{1.713789in}}%
\pgfpathlineto{\pgfqpoint{2.964073in}{1.716622in}}%
\pgfpathlineto{\pgfqpoint{2.972365in}{1.719670in}}%
\pgfpathlineto{\pgfqpoint{2.980644in}{1.722928in}}%
\pgfpathlineto{\pgfqpoint{2.988911in}{1.726392in}}%
\pgfpathlineto{\pgfqpoint{2.975598in}{1.736337in}}%
\pgfpathlineto{\pgfqpoint{2.962284in}{1.746483in}}%
\pgfpathlineto{\pgfqpoint{2.948968in}{1.756829in}}%
\pgfpathlineto{\pgfqpoint{2.935649in}{1.767377in}}%
\pgfpathlineto{\pgfqpoint{2.927350in}{1.764434in}}%
\pgfpathlineto{\pgfqpoint{2.919037in}{1.761703in}}%
\pgfpathlineto{\pgfqpoint{2.910711in}{1.759191in}}%
\pgfpathlineto{\pgfqpoint{2.902372in}{1.756902in}}%
\pgfpathclose%
\pgfusepath{fill}%
\end{pgfscope}%
\begin{pgfscope}%
\pgfpathrectangle{\pgfqpoint{1.254980in}{0.150000in}}{\pgfqpoint{5.490039in}{5.490039in}}%
\pgfusepath{clip}%
\pgfsetbuttcap%
\pgfsetroundjoin%
\definecolor{currentfill}{rgb}{0.319809,0.770914,0.411152}%
\pgfsetfillcolor{currentfill}%
\pgfsetfillopacity{0.700000}%
\pgfsetlinewidth{0.000000pt}%
\definecolor{currentstroke}{rgb}{0.000000,0.000000,0.000000}%
\pgfsetstrokecolor{currentstroke}%
\pgfsetdash{}{0pt}%
\pgfpathmoveto{\pgfqpoint{5.903753in}{3.428402in}}%
\pgfpathlineto{\pgfqpoint{5.918225in}{3.441684in}}%
\pgfpathlineto{\pgfqpoint{5.932719in}{3.455124in}}%
\pgfpathlineto{\pgfqpoint{5.947233in}{3.468724in}}%
\pgfpathlineto{\pgfqpoint{5.954033in}{3.468688in}}%
\pgfpathlineto{\pgfqpoint{5.960826in}{3.468648in}}%
\pgfpathlineto{\pgfqpoint{5.967612in}{3.468610in}}%
\pgfpathlineto{\pgfqpoint{5.974391in}{3.468580in}}%
\pgfpathlineto{\pgfqpoint{5.959910in}{3.455515in}}%
\pgfpathlineto{\pgfqpoint{5.945449in}{3.442608in}}%
\pgfpathlineto{\pgfqpoint{5.931010in}{3.429858in}}%
\pgfpathlineto{\pgfqpoint{5.924206in}{3.429481in}}%
\pgfpathlineto{\pgfqpoint{5.917395in}{3.429117in}}%
\pgfpathlineto{\pgfqpoint{5.910578in}{3.428759in}}%
\pgfpathlineto{\pgfqpoint{5.903753in}{3.428402in}}%
\pgfpathclose%
\pgfusepath{fill}%
\end{pgfscope}%
\begin{pgfscope}%
\pgfpathrectangle{\pgfqpoint{1.254980in}{0.150000in}}{\pgfqpoint{5.490039in}{5.490039in}}%
\pgfusepath{clip}%
\pgfsetbuttcap%
\pgfsetroundjoin%
\definecolor{currentfill}{rgb}{0.257322,0.256130,0.526563}%
\pgfsetfillcolor{currentfill}%
\pgfsetfillopacity{0.700000}%
\pgfsetlinewidth{0.000000pt}%
\definecolor{currentstroke}{rgb}{0.000000,0.000000,0.000000}%
\pgfsetstrokecolor{currentstroke}%
\pgfsetdash{}{0pt}%
\pgfpathmoveto{\pgfqpoint{4.221156in}{2.055107in}}%
\pgfpathlineto{\pgfqpoint{4.234667in}{2.060652in}}%
\pgfpathlineto{\pgfqpoint{4.248189in}{2.066359in}}%
\pgfpathlineto{\pgfqpoint{4.261723in}{2.072229in}}%
\pgfpathlineto{\pgfqpoint{4.275269in}{2.078259in}}%
\pgfpathlineto{\pgfqpoint{4.282981in}{2.089974in}}%
\pgfpathlineto{\pgfqpoint{4.290688in}{2.101615in}}%
\pgfpathlineto{\pgfqpoint{4.298390in}{2.113181in}}%
\pgfpathlineto{\pgfqpoint{4.306086in}{2.124670in}}%
\pgfpathlineto{\pgfqpoint{4.292543in}{2.118476in}}%
\pgfpathlineto{\pgfqpoint{4.279011in}{2.112444in}}%
\pgfpathlineto{\pgfqpoint{4.265492in}{2.106574in}}%
\pgfpathlineto{\pgfqpoint{4.251985in}{2.100866in}}%
\pgfpathlineto{\pgfqpoint{4.244285in}{2.089530in}}%
\pgfpathlineto{\pgfqpoint{4.236580in}{2.078123in}}%
\pgfpathlineto{\pgfqpoint{4.228871in}{2.066649in}}%
\pgfpathlineto{\pgfqpoint{4.221156in}{2.055107in}}%
\pgfpathclose%
\pgfusepath{fill}%
\end{pgfscope}%
\begin{pgfscope}%
\pgfpathrectangle{\pgfqpoint{1.254980in}{0.150000in}}{\pgfqpoint{5.490039in}{5.490039in}}%
\pgfusepath{clip}%
\pgfsetbuttcap%
\pgfsetroundjoin%
\definecolor{currentfill}{rgb}{0.281477,0.755203,0.432552}%
\pgfsetfillcolor{currentfill}%
\pgfsetfillopacity{0.700000}%
\pgfsetlinewidth{0.000000pt}%
\definecolor{currentstroke}{rgb}{0.000000,0.000000,0.000000}%
\pgfsetstrokecolor{currentstroke}%
\pgfsetdash{}{0pt}%
\pgfpathmoveto{\pgfqpoint{5.818581in}{3.373314in}}%
\pgfpathlineto{\pgfqpoint{5.833000in}{3.386466in}}%
\pgfpathlineto{\pgfqpoint{5.847440in}{3.399776in}}%
\pgfpathlineto{\pgfqpoint{5.861900in}{3.413245in}}%
\pgfpathlineto{\pgfqpoint{5.876382in}{3.426874in}}%
\pgfpathlineto{\pgfqpoint{5.883236in}{3.427282in}}%
\pgfpathlineto{\pgfqpoint{5.890083in}{3.427669in}}%
\pgfpathlineto{\pgfqpoint{5.896922in}{3.428040in}}%
\pgfpathlineto{\pgfqpoint{5.903753in}{3.428402in}}%
\pgfpathlineto{\pgfqpoint{5.889302in}{3.415278in}}%
\pgfpathlineto{\pgfqpoint{5.874872in}{3.402312in}}%
\pgfpathlineto{\pgfqpoint{5.860463in}{3.389503in}}%
\pgfpathlineto{\pgfqpoint{5.846074in}{3.376852in}}%
\pgfpathlineto{\pgfqpoint{5.839212in}{3.375978in}}%
\pgfpathlineto{\pgfqpoint{5.832342in}{3.375101in}}%
\pgfpathlineto{\pgfqpoint{5.825465in}{3.374214in}}%
\pgfpathlineto{\pgfqpoint{5.818581in}{3.373314in}}%
\pgfpathclose%
\pgfusepath{fill}%
\end{pgfscope}%
\begin{pgfscope}%
\pgfpathrectangle{\pgfqpoint{1.254980in}{0.150000in}}{\pgfqpoint{5.490039in}{5.490039in}}%
\pgfusepath{clip}%
\pgfsetbuttcap%
\pgfsetroundjoin%
\definecolor{currentfill}{rgb}{0.282623,0.140926,0.457517}%
\pgfsetfillcolor{currentfill}%
\pgfsetfillopacity{0.700000}%
\pgfsetlinewidth{0.000000pt}%
\definecolor{currentstroke}{rgb}{0.000000,0.000000,0.000000}%
\pgfsetstrokecolor{currentstroke}%
\pgfsetdash{}{0pt}%
\pgfpathmoveto{\pgfqpoint{3.935627in}{1.820047in}}%
\pgfpathlineto{\pgfqpoint{3.949027in}{1.822702in}}%
\pgfpathlineto{\pgfqpoint{3.962437in}{1.825522in}}%
\pgfpathlineto{\pgfqpoint{3.975856in}{1.828506in}}%
\pgfpathlineto{\pgfqpoint{3.989285in}{1.831653in}}%
\pgfpathlineto{\pgfqpoint{3.997083in}{1.843456in}}%
\pgfpathlineto{\pgfqpoint{4.004876in}{1.855237in}}%
\pgfpathlineto{\pgfqpoint{4.012664in}{1.866993in}}%
\pgfpathlineto{\pgfqpoint{4.020447in}{1.878723in}}%
\pgfpathlineto{\pgfqpoint{4.007023in}{1.875301in}}%
\pgfpathlineto{\pgfqpoint{3.993609in}{1.872042in}}%
\pgfpathlineto{\pgfqpoint{3.980205in}{1.868948in}}%
\pgfpathlineto{\pgfqpoint{3.966810in}{1.866017in}}%
\pgfpathlineto{\pgfqpoint{3.959021in}{1.854552in}}%
\pgfpathlineto{\pgfqpoint{3.951228in}{1.843067in}}%
\pgfpathlineto{\pgfqpoint{3.943430in}{1.831564in}}%
\pgfpathlineto{\pgfqpoint{3.935627in}{1.820047in}}%
\pgfpathclose%
\pgfusepath{fill}%
\end{pgfscope}%
\begin{pgfscope}%
\pgfpathrectangle{\pgfqpoint{1.254980in}{0.150000in}}{\pgfqpoint{5.490039in}{5.490039in}}%
\pgfusepath{clip}%
\pgfsetbuttcap%
\pgfsetroundjoin%
\definecolor{currentfill}{rgb}{0.120081,0.622161,0.534946}%
\pgfsetfillcolor{currentfill}%
\pgfsetfillopacity{0.700000}%
\pgfsetlinewidth{0.000000pt}%
\definecolor{currentstroke}{rgb}{0.000000,0.000000,0.000000}%
\pgfsetstrokecolor{currentstroke}%
\pgfsetdash{}{0pt}%
\pgfpathmoveto{\pgfqpoint{5.278483in}{2.990696in}}%
\pgfpathlineto{\pgfqpoint{5.292579in}{3.002677in}}%
\pgfpathlineto{\pgfqpoint{5.306694in}{3.014818in}}%
\pgfpathlineto{\pgfqpoint{5.320828in}{3.027119in}}%
\pgfpathlineto{\pgfqpoint{5.334981in}{3.039580in}}%
\pgfpathlineto{\pgfqpoint{5.342210in}{3.044185in}}%
\pgfpathlineto{\pgfqpoint{5.349430in}{3.048692in}}%
\pgfpathlineto{\pgfqpoint{5.356641in}{3.053102in}}%
\pgfpathlineto{\pgfqpoint{5.363844in}{3.057420in}}%
\pgfpathlineto{\pgfqpoint{5.349707in}{3.045245in}}%
\pgfpathlineto{\pgfqpoint{5.335589in}{3.033230in}}%
\pgfpathlineto{\pgfqpoint{5.321490in}{3.021375in}}%
\pgfpathlineto{\pgfqpoint{5.307409in}{3.009678in}}%
\pgfpathlineto{\pgfqpoint{5.300190in}{3.005064in}}%
\pgfpathlineto{\pgfqpoint{5.292963in}{3.000364in}}%
\pgfpathlineto{\pgfqpoint{5.285727in}{2.995576in}}%
\pgfpathlineto{\pgfqpoint{5.278483in}{2.990696in}}%
\pgfpathclose%
\pgfusepath{fill}%
\end{pgfscope}%
\begin{pgfscope}%
\pgfpathrectangle{\pgfqpoint{1.254980in}{0.150000in}}{\pgfqpoint{5.490039in}{5.490039in}}%
\pgfusepath{clip}%
\pgfsetbuttcap%
\pgfsetroundjoin%
\definecolor{currentfill}{rgb}{0.269944,0.014625,0.341379}%
\pgfsetfillcolor{currentfill}%
\pgfsetfillopacity{0.700000}%
\pgfsetlinewidth{0.000000pt}%
\definecolor{currentstroke}{rgb}{0.000000,0.000000,0.000000}%
\pgfsetstrokecolor{currentstroke}%
\pgfsetdash{}{0pt}%
\pgfpathmoveto{\pgfqpoint{3.511277in}{1.608475in}}%
\pgfpathlineto{\pgfqpoint{3.524584in}{1.605998in}}%
\pgfpathlineto{\pgfqpoint{3.537895in}{1.603693in}}%
\pgfpathlineto{\pgfqpoint{3.551212in}{1.601558in}}%
\pgfpathlineto{\pgfqpoint{3.564534in}{1.599594in}}%
\pgfpathlineto{\pgfqpoint{3.572483in}{1.609092in}}%
\pgfpathlineto{\pgfqpoint{3.580425in}{1.618667in}}%
\pgfpathlineto{\pgfqpoint{3.588361in}{1.628316in}}%
\pgfpathlineto{\pgfqpoint{3.596290in}{1.638034in}}%
\pgfpathlineto{\pgfqpoint{3.582982in}{1.639587in}}%
\pgfpathlineto{\pgfqpoint{3.569679in}{1.641310in}}%
\pgfpathlineto{\pgfqpoint{3.556381in}{1.643204in}}%
\pgfpathlineto{\pgfqpoint{3.543089in}{1.645269in}}%
\pgfpathlineto{\pgfqpoint{3.535146in}{1.635952in}}%
\pgfpathlineto{\pgfqpoint{3.527196in}{1.626712in}}%
\pgfpathlineto{\pgfqpoint{3.519240in}{1.617551in}}%
\pgfpathlineto{\pgfqpoint{3.511277in}{1.608475in}}%
\pgfpathclose%
\pgfusepath{fill}%
\end{pgfscope}%
\begin{pgfscope}%
\pgfpathrectangle{\pgfqpoint{1.254980in}{0.150000in}}{\pgfqpoint{5.490039in}{5.490039in}}%
\pgfusepath{clip}%
\pgfsetbuttcap%
\pgfsetroundjoin%
\definecolor{currentfill}{rgb}{0.174274,0.445044,0.557792}%
\pgfsetfillcolor{currentfill}%
\pgfsetfillopacity{0.700000}%
\pgfsetlinewidth{0.000000pt}%
\definecolor{currentstroke}{rgb}{0.000000,0.000000,0.000000}%
\pgfsetstrokecolor{currentstroke}%
\pgfsetdash{}{0pt}%
\pgfpathmoveto{\pgfqpoint{4.707410in}{2.502581in}}%
\pgfpathlineto{\pgfqpoint{4.721170in}{2.511890in}}%
\pgfpathlineto{\pgfqpoint{4.734946in}{2.521359in}}%
\pgfpathlineto{\pgfqpoint{4.748737in}{2.530989in}}%
\pgfpathlineto{\pgfqpoint{4.762543in}{2.540781in}}%
\pgfpathlineto{\pgfqpoint{4.770080in}{2.550029in}}%
\pgfpathlineto{\pgfqpoint{4.777609in}{2.559159in}}%
\pgfpathlineto{\pgfqpoint{4.785131in}{2.568171in}}%
\pgfpathlineto{\pgfqpoint{4.792647in}{2.577065in}}%
\pgfpathlineto{\pgfqpoint{4.778845in}{2.567316in}}%
\pgfpathlineto{\pgfqpoint{4.765059in}{2.557727in}}%
\pgfpathlineto{\pgfqpoint{4.751289in}{2.548299in}}%
\pgfpathlineto{\pgfqpoint{4.737534in}{2.539031in}}%
\pgfpathlineto{\pgfqpoint{4.730013in}{2.530084in}}%
\pgfpathlineto{\pgfqpoint{4.722485in}{2.521027in}}%
\pgfpathlineto{\pgfqpoint{4.714951in}{2.511860in}}%
\pgfpathlineto{\pgfqpoint{4.707410in}{2.502581in}}%
\pgfpathclose%
\pgfusepath{fill}%
\end{pgfscope}%
\begin{pgfscope}%
\pgfpathrectangle{\pgfqpoint{1.254980in}{0.150000in}}{\pgfqpoint{5.490039in}{5.490039in}}%
\pgfusepath{clip}%
\pgfsetbuttcap%
\pgfsetroundjoin%
\definecolor{currentfill}{rgb}{0.267004,0.004874,0.329415}%
\pgfsetfillcolor{currentfill}%
\pgfsetfillopacity{0.700000}%
\pgfsetlinewidth{0.000000pt}%
\definecolor{currentstroke}{rgb}{0.000000,0.000000,0.000000}%
\pgfsetstrokecolor{currentstroke}%
\pgfsetdash{}{0pt}%
\pgfpathmoveto{\pgfqpoint{3.287535in}{1.593587in}}%
\pgfpathlineto{\pgfqpoint{3.300834in}{1.588099in}}%
\pgfpathlineto{\pgfqpoint{3.314135in}{1.582790in}}%
\pgfpathlineto{\pgfqpoint{3.327438in}{1.577659in}}%
\pgfpathlineto{\pgfqpoint{3.340745in}{1.572706in}}%
\pgfpathlineto{\pgfqpoint{3.348804in}{1.580002in}}%
\pgfpathlineto{\pgfqpoint{3.356854in}{1.587430in}}%
\pgfpathlineto{\pgfqpoint{3.364896in}{1.594985in}}%
\pgfpathlineto{\pgfqpoint{3.372931in}{1.602665in}}%
\pgfpathlineto{\pgfqpoint{3.359644in}{1.607150in}}%
\pgfpathlineto{\pgfqpoint{3.346361in}{1.611813in}}%
\pgfpathlineto{\pgfqpoint{3.333081in}{1.616653in}}%
\pgfpathlineto{\pgfqpoint{3.319804in}{1.621673in}}%
\pgfpathlineto{\pgfqpoint{3.311750in}{1.614451in}}%
\pgfpathlineto{\pgfqpoint{3.303687in}{1.607360in}}%
\pgfpathlineto{\pgfqpoint{3.295615in}{1.600404in}}%
\pgfpathlineto{\pgfqpoint{3.287535in}{1.593587in}}%
\pgfpathclose%
\pgfusepath{fill}%
\end{pgfscope}%
\begin{pgfscope}%
\pgfpathrectangle{\pgfqpoint{1.254980in}{0.150000in}}{\pgfqpoint{5.490039in}{5.490039in}}%
\pgfusepath{clip}%
\pgfsetbuttcap%
\pgfsetroundjoin%
\definecolor{currentfill}{rgb}{0.180629,0.429975,0.557282}%
\pgfsetfillcolor{currentfill}%
\pgfsetfillopacity{0.700000}%
\pgfsetlinewidth{0.000000pt}%
\definecolor{currentstroke}{rgb}{0.000000,0.000000,0.000000}%
\pgfsetstrokecolor{currentstroke}%
\pgfsetdash{}{0pt}%
\pgfpathmoveto{\pgfqpoint{2.254585in}{2.556866in}}%
\pgfpathlineto{\pgfqpoint{2.268310in}{2.534123in}}%
\pgfpathlineto{\pgfqpoint{2.282020in}{2.511674in}}%
\pgfpathlineto{\pgfqpoint{2.295717in}{2.489516in}}%
\pgfpathlineto{\pgfqpoint{2.309400in}{2.467648in}}%
\pgfpathlineto{\pgfqpoint{2.318235in}{2.463770in}}%
\pgfpathlineto{\pgfqpoint{2.327050in}{2.460202in}}%
\pgfpathlineto{\pgfqpoint{2.335843in}{2.456940in}}%
\pgfpathlineto{\pgfqpoint{2.344616in}{2.453977in}}%
\pgfpathlineto{\pgfqpoint{2.330987in}{2.475274in}}%
\pgfpathlineto{\pgfqpoint{2.317346in}{2.496858in}}%
\pgfpathlineto{\pgfqpoint{2.303691in}{2.518733in}}%
\pgfpathlineto{\pgfqpoint{2.290023in}{2.540902in}}%
\pgfpathlineto{\pgfqpoint{2.281196in}{2.544425in}}%
\pgfpathlineto{\pgfqpoint{2.272347in}{2.548257in}}%
\pgfpathlineto{\pgfqpoint{2.263477in}{2.552402in}}%
\pgfpathlineto{\pgfqpoint{2.254585in}{2.556866in}}%
\pgfpathclose%
\pgfusepath{fill}%
\end{pgfscope}%
\begin{pgfscope}%
\pgfpathrectangle{\pgfqpoint{1.254980in}{0.150000in}}{\pgfqpoint{5.490039in}{5.490039in}}%
\pgfusepath{clip}%
\pgfsetbuttcap%
\pgfsetroundjoin%
\definecolor{currentfill}{rgb}{0.269944,0.014625,0.341379}%
\pgfsetfillcolor{currentfill}%
\pgfsetfillopacity{0.700000}%
\pgfsetlinewidth{0.000000pt}%
\definecolor{currentstroke}{rgb}{0.000000,0.000000,0.000000}%
\pgfsetstrokecolor{currentstroke}%
\pgfsetdash{}{0pt}%
\pgfpathmoveto{\pgfqpoint{3.148614in}{1.622227in}}%
\pgfpathlineto{\pgfqpoint{3.161923in}{1.614781in}}%
\pgfpathlineto{\pgfqpoint{3.175234in}{1.607521in}}%
\pgfpathlineto{\pgfqpoint{3.188545in}{1.600445in}}%
\pgfpathlineto{\pgfqpoint{3.201858in}{1.593553in}}%
\pgfpathlineto{\pgfqpoint{3.209999in}{1.599257in}}%
\pgfpathlineto{\pgfqpoint{3.218130in}{1.605127in}}%
\pgfpathlineto{\pgfqpoint{3.226252in}{1.611157in}}%
\pgfpathlineto{\pgfqpoint{3.234364in}{1.617343in}}%
\pgfpathlineto{\pgfqpoint{3.221077in}{1.623738in}}%
\pgfpathlineto{\pgfqpoint{3.207790in}{1.630316in}}%
\pgfpathlineto{\pgfqpoint{3.194506in}{1.637079in}}%
\pgfpathlineto{\pgfqpoint{3.181222in}{1.644027in}}%
\pgfpathlineto{\pgfqpoint{3.173085in}{1.638327in}}%
\pgfpathlineto{\pgfqpoint{3.164938in}{1.632790in}}%
\pgfpathlineto{\pgfqpoint{3.156781in}{1.627422in}}%
\pgfpathlineto{\pgfqpoint{3.148614in}{1.622227in}}%
\pgfpathclose%
\pgfusepath{fill}%
\end{pgfscope}%
\begin{pgfscope}%
\pgfpathrectangle{\pgfqpoint{1.254980in}{0.150000in}}{\pgfqpoint{5.490039in}{5.490039in}}%
\pgfusepath{clip}%
\pgfsetbuttcap%
\pgfsetroundjoin%
\definecolor{currentfill}{rgb}{0.279566,0.067836,0.391917}%
\pgfsetfillcolor{currentfill}%
\pgfsetfillopacity{0.700000}%
\pgfsetlinewidth{0.000000pt}%
\definecolor{currentstroke}{rgb}{0.000000,0.000000,0.000000}%
\pgfsetstrokecolor{currentstroke}%
\pgfsetdash{}{0pt}%
\pgfpathmoveto{\pgfqpoint{2.955769in}{1.713789in}}%
\pgfpathlineto{\pgfqpoint{2.969113in}{1.703511in}}%
\pgfpathlineto{\pgfqpoint{2.982456in}{1.693431in}}%
\pgfpathlineto{\pgfqpoint{2.995797in}{1.683548in}}%
\pgfpathlineto{\pgfqpoint{3.009137in}{1.673860in}}%
\pgfpathlineto{\pgfqpoint{3.017409in}{1.677234in}}%
\pgfpathlineto{\pgfqpoint{3.025668in}{1.680816in}}%
\pgfpathlineto{\pgfqpoint{3.033916in}{1.684601in}}%
\pgfpathlineto{\pgfqpoint{3.042152in}{1.688584in}}%
\pgfpathlineto{\pgfqpoint{3.028843in}{1.697742in}}%
\pgfpathlineto{\pgfqpoint{3.015533in}{1.707096in}}%
\pgfpathlineto{\pgfqpoint{3.002223in}{1.716645in}}%
\pgfpathlineto{\pgfqpoint{2.988911in}{1.726392in}}%
\pgfpathlineto{\pgfqpoint{2.980644in}{1.722928in}}%
\pgfpathlineto{\pgfqpoint{2.972365in}{1.719670in}}%
\pgfpathlineto{\pgfqpoint{2.964073in}{1.716622in}}%
\pgfpathlineto{\pgfqpoint{2.955769in}{1.713789in}}%
\pgfpathclose%
\pgfusepath{fill}%
\end{pgfscope}%
\begin{pgfscope}%
\pgfpathrectangle{\pgfqpoint{1.254980in}{0.150000in}}{\pgfqpoint{5.490039in}{5.490039in}}%
\pgfusepath{clip}%
\pgfsetbuttcap%
\pgfsetroundjoin%
\definecolor{currentfill}{rgb}{0.137770,0.537492,0.554906}%
\pgfsetfillcolor{currentfill}%
\pgfsetfillopacity{0.700000}%
\pgfsetlinewidth{0.000000pt}%
\definecolor{currentstroke}{rgb}{0.000000,0.000000,0.000000}%
\pgfsetstrokecolor{currentstroke}%
\pgfsetdash{}{0pt}%
\pgfpathmoveto{\pgfqpoint{4.993122in}{2.756454in}}%
\pgfpathlineto{\pgfqpoint{5.007051in}{2.767345in}}%
\pgfpathlineto{\pgfqpoint{5.020997in}{2.778397in}}%
\pgfpathlineto{\pgfqpoint{5.034960in}{2.789610in}}%
\pgfpathlineto{\pgfqpoint{5.048940in}{2.800983in}}%
\pgfpathlineto{\pgfqpoint{5.056339in}{2.808021in}}%
\pgfpathlineto{\pgfqpoint{5.063729in}{2.814939in}}%
\pgfpathlineto{\pgfqpoint{5.071112in}{2.821741in}}%
\pgfpathlineto{\pgfqpoint{5.078486in}{2.828428in}}%
\pgfpathlineto{\pgfqpoint{5.064515in}{2.817217in}}%
\pgfpathlineto{\pgfqpoint{5.050561in}{2.806167in}}%
\pgfpathlineto{\pgfqpoint{5.036625in}{2.795278in}}%
\pgfpathlineto{\pgfqpoint{5.022705in}{2.784548in}}%
\pgfpathlineto{\pgfqpoint{5.015321in}{2.777688in}}%
\pgfpathlineto{\pgfqpoint{5.007929in}{2.770720in}}%
\pgfpathlineto{\pgfqpoint{5.000529in}{2.763643in}}%
\pgfpathlineto{\pgfqpoint{4.993122in}{2.756454in}}%
\pgfpathclose%
\pgfusepath{fill}%
\end{pgfscope}%
\begin{pgfscope}%
\pgfpathrectangle{\pgfqpoint{1.254980in}{0.150000in}}{\pgfqpoint{5.490039in}{5.490039in}}%
\pgfusepath{clip}%
\pgfsetbuttcap%
\pgfsetroundjoin%
\definecolor{currentfill}{rgb}{0.206756,0.371758,0.553117}%
\pgfsetfillcolor{currentfill}%
\pgfsetfillopacity{0.700000}%
\pgfsetlinewidth{0.000000pt}%
\definecolor{currentstroke}{rgb}{0.000000,0.000000,0.000000}%
\pgfsetstrokecolor{currentstroke}%
\pgfsetdash{}{0pt}%
\pgfpathmoveto{\pgfqpoint{4.506790in}{2.314127in}}%
\pgfpathlineto{\pgfqpoint{4.520444in}{2.322092in}}%
\pgfpathlineto{\pgfqpoint{4.534112in}{2.330218in}}%
\pgfpathlineto{\pgfqpoint{4.547794in}{2.338506in}}%
\pgfpathlineto{\pgfqpoint{4.561491in}{2.346955in}}%
\pgfpathlineto{\pgfqpoint{4.569110in}{2.357546in}}%
\pgfpathlineto{\pgfqpoint{4.576723in}{2.368030in}}%
\pgfpathlineto{\pgfqpoint{4.584330in}{2.378406in}}%
\pgfpathlineto{\pgfqpoint{4.591931in}{2.388674in}}%
\pgfpathlineto{\pgfqpoint{4.578238in}{2.380178in}}%
\pgfpathlineto{\pgfqpoint{4.564559in}{2.371843in}}%
\pgfpathlineto{\pgfqpoint{4.550894in}{2.363669in}}%
\pgfpathlineto{\pgfqpoint{4.537243in}{2.355656in}}%
\pgfpathlineto{\pgfqpoint{4.529639in}{2.345425in}}%
\pgfpathlineto{\pgfqpoint{4.522028in}{2.335092in}}%
\pgfpathlineto{\pgfqpoint{4.514412in}{2.324660in}}%
\pgfpathlineto{\pgfqpoint{4.506790in}{2.314127in}}%
\pgfpathclose%
\pgfusepath{fill}%
\end{pgfscope}%
\begin{pgfscope}%
\pgfpathrectangle{\pgfqpoint{1.254980in}{0.150000in}}{\pgfqpoint{5.490039in}{5.490039in}}%
\pgfusepath{clip}%
\pgfsetbuttcap%
\pgfsetroundjoin%
\definecolor{currentfill}{rgb}{0.278826,0.175490,0.483397}%
\pgfsetfillcolor{currentfill}%
\pgfsetfillopacity{0.700000}%
\pgfsetlinewidth{0.000000pt}%
\definecolor{currentstroke}{rgb}{0.000000,0.000000,0.000000}%
\pgfsetstrokecolor{currentstroke}%
\pgfsetdash{}{0pt}%
\pgfpathmoveto{\pgfqpoint{4.020447in}{1.878723in}}%
\pgfpathlineto{\pgfqpoint{4.033881in}{1.882308in}}%
\pgfpathlineto{\pgfqpoint{4.047325in}{1.886057in}}%
\pgfpathlineto{\pgfqpoint{4.060780in}{1.889969in}}%
\pgfpathlineto{\pgfqpoint{4.074244in}{1.894043in}}%
\pgfpathlineto{\pgfqpoint{4.082019in}{1.906001in}}%
\pgfpathlineto{\pgfqpoint{4.089789in}{1.917921in}}%
\pgfpathlineto{\pgfqpoint{4.097554in}{1.929799in}}%
\pgfpathlineto{\pgfqpoint{4.105315in}{1.941635in}}%
\pgfpathlineto{\pgfqpoint{4.091854in}{1.937313in}}%
\pgfpathlineto{\pgfqpoint{4.078404in}{1.933154in}}%
\pgfpathlineto{\pgfqpoint{4.064964in}{1.929158in}}%
\pgfpathlineto{\pgfqpoint{4.051535in}{1.925326in}}%
\pgfpathlineto{\pgfqpoint{4.043770in}{1.913727in}}%
\pgfpathlineto{\pgfqpoint{4.036000in}{1.902092in}}%
\pgfpathlineto{\pgfqpoint{4.028226in}{1.890423in}}%
\pgfpathlineto{\pgfqpoint{4.020447in}{1.878723in}}%
\pgfpathclose%
\pgfusepath{fill}%
\end{pgfscope}%
\begin{pgfscope}%
\pgfpathrectangle{\pgfqpoint{1.254980in}{0.150000in}}{\pgfqpoint{5.490039in}{5.490039in}}%
\pgfusepath{clip}%
\pgfsetbuttcap%
\pgfsetroundjoin%
\definecolor{currentfill}{rgb}{0.268510,0.009605,0.335427}%
\pgfsetfillcolor{currentfill}%
\pgfsetfillopacity{0.700000}%
\pgfsetlinewidth{0.000000pt}%
\definecolor{currentstroke}{rgb}{0.000000,0.000000,0.000000}%
\pgfsetstrokecolor{currentstroke}%
\pgfsetdash{}{0pt}%
\pgfpathmoveto{\pgfqpoint{3.426111in}{1.586485in}}%
\pgfpathlineto{\pgfqpoint{3.439416in}{1.582878in}}%
\pgfpathlineto{\pgfqpoint{3.452725in}{1.579444in}}%
\pgfpathlineto{\pgfqpoint{3.466039in}{1.576183in}}%
\pgfpathlineto{\pgfqpoint{3.479356in}{1.573095in}}%
\pgfpathlineto{\pgfqpoint{3.487347in}{1.581793in}}%
\pgfpathlineto{\pgfqpoint{3.495331in}{1.590592in}}%
\pgfpathlineto{\pgfqpoint{3.503307in}{1.599488in}}%
\pgfpathlineto{\pgfqpoint{3.511277in}{1.608475in}}%
\pgfpathlineto{\pgfqpoint{3.497975in}{1.611124in}}%
\pgfpathlineto{\pgfqpoint{3.484678in}{1.613946in}}%
\pgfpathlineto{\pgfqpoint{3.471386in}{1.616940in}}%
\pgfpathlineto{\pgfqpoint{3.458099in}{1.620108in}}%
\pgfpathlineto{\pgfqpoint{3.450113in}{1.611549in}}%
\pgfpathlineto{\pgfqpoint{3.442120in}{1.603090in}}%
\pgfpathlineto{\pgfqpoint{3.434119in}{1.594734in}}%
\pgfpathlineto{\pgfqpoint{3.426111in}{1.586485in}}%
\pgfpathclose%
\pgfusepath{fill}%
\end{pgfscope}%
\begin{pgfscope}%
\pgfpathrectangle{\pgfqpoint{1.254980in}{0.150000in}}{\pgfqpoint{5.490039in}{5.490039in}}%
\pgfusepath{clip}%
\pgfsetbuttcap%
\pgfsetroundjoin%
\definecolor{currentfill}{rgb}{0.128087,0.647749,0.523491}%
\pgfsetfillcolor{currentfill}%
\pgfsetfillopacity{0.700000}%
\pgfsetlinewidth{0.000000pt}%
\definecolor{currentstroke}{rgb}{0.000000,0.000000,0.000000}%
\pgfsetstrokecolor{currentstroke}%
\pgfsetdash{}{0pt}%
\pgfpathmoveto{\pgfqpoint{5.363844in}{3.057420in}}%
\pgfpathlineto{\pgfqpoint{5.378000in}{3.069755in}}%
\pgfpathlineto{\pgfqpoint{5.392175in}{3.082250in}}%
\pgfpathlineto{\pgfqpoint{5.406369in}{3.094905in}}%
\pgfpathlineto{\pgfqpoint{5.420583in}{3.107720in}}%
\pgfpathlineto{\pgfqpoint{5.427760in}{3.111643in}}%
\pgfpathlineto{\pgfqpoint{5.434929in}{3.115473in}}%
\pgfpathlineto{\pgfqpoint{5.442089in}{3.119214in}}%
\pgfpathlineto{\pgfqpoint{5.449241in}{3.122870in}}%
\pgfpathlineto{\pgfqpoint{5.435045in}{3.110373in}}%
\pgfpathlineto{\pgfqpoint{5.420868in}{3.098035in}}%
\pgfpathlineto{\pgfqpoint{5.406711in}{3.085857in}}%
\pgfpathlineto{\pgfqpoint{5.392572in}{3.073838in}}%
\pgfpathlineto{\pgfqpoint{5.385403in}{3.069855in}}%
\pgfpathlineto{\pgfqpoint{5.378225in}{3.065793in}}%
\pgfpathlineto{\pgfqpoint{5.371039in}{3.061649in}}%
\pgfpathlineto{\pgfqpoint{5.363844in}{3.057420in}}%
\pgfpathclose%
\pgfusepath{fill}%
\end{pgfscope}%
\begin{pgfscope}%
\pgfpathrectangle{\pgfqpoint{1.254980in}{0.150000in}}{\pgfqpoint{5.490039in}{5.490039in}}%
\pgfusepath{clip}%
\pgfsetbuttcap%
\pgfsetroundjoin%
\definecolor{currentfill}{rgb}{0.243113,0.292092,0.538516}%
\pgfsetfillcolor{currentfill}%
\pgfsetfillopacity{0.700000}%
\pgfsetlinewidth{0.000000pt}%
\definecolor{currentstroke}{rgb}{0.000000,0.000000,0.000000}%
\pgfsetstrokecolor{currentstroke}%
\pgfsetdash{}{0pt}%
\pgfpathmoveto{\pgfqpoint{4.306086in}{2.124670in}}%
\pgfpathlineto{\pgfqpoint{4.319643in}{2.131026in}}%
\pgfpathlineto{\pgfqpoint{4.333211in}{2.137543in}}%
\pgfpathlineto{\pgfqpoint{4.346792in}{2.144222in}}%
\pgfpathlineto{\pgfqpoint{4.360386in}{2.151063in}}%
\pgfpathlineto{\pgfqpoint{4.368076in}{2.162620in}}%
\pgfpathlineto{\pgfqpoint{4.375760in}{2.174090in}}%
\pgfpathlineto{\pgfqpoint{4.383439in}{2.185474in}}%
\pgfpathlineto{\pgfqpoint{4.391113in}{2.196769in}}%
\pgfpathlineto{\pgfqpoint{4.377521in}{2.189794in}}%
\pgfpathlineto{\pgfqpoint{4.363943in}{2.182980in}}%
\pgfpathlineto{\pgfqpoint{4.350377in}{2.176328in}}%
\pgfpathlineto{\pgfqpoint{4.336823in}{2.169838in}}%
\pgfpathlineto{\pgfqpoint{4.329147in}{2.158666in}}%
\pgfpathlineto{\pgfqpoint{4.321465in}{2.147413in}}%
\pgfpathlineto{\pgfqpoint{4.313778in}{2.136081in}}%
\pgfpathlineto{\pgfqpoint{4.306086in}{2.124670in}}%
\pgfpathclose%
\pgfusepath{fill}%
\end{pgfscope}%
\begin{pgfscope}%
\pgfpathrectangle{\pgfqpoint{1.254980in}{0.150000in}}{\pgfqpoint{5.490039in}{5.490039in}}%
\pgfusepath{clip}%
\pgfsetbuttcap%
\pgfsetroundjoin%
\definecolor{currentfill}{rgb}{0.162142,0.474838,0.558140}%
\pgfsetfillcolor{currentfill}%
\pgfsetfillopacity{0.700000}%
\pgfsetlinewidth{0.000000pt}%
\definecolor{currentstroke}{rgb}{0.000000,0.000000,0.000000}%
\pgfsetstrokecolor{currentstroke}%
\pgfsetdash{}{0pt}%
\pgfpathmoveto{\pgfqpoint{4.792647in}{2.577065in}}%
\pgfpathlineto{\pgfqpoint{4.806465in}{2.586976in}}%
\pgfpathlineto{\pgfqpoint{4.820298in}{2.597048in}}%
\pgfpathlineto{\pgfqpoint{4.834148in}{2.607280in}}%
\pgfpathlineto{\pgfqpoint{4.848014in}{2.617674in}}%
\pgfpathlineto{\pgfqpoint{4.855517in}{2.626392in}}%
\pgfpathlineto{\pgfqpoint{4.863013in}{2.634988in}}%
\pgfpathlineto{\pgfqpoint{4.870502in}{2.643462in}}%
\pgfpathlineto{\pgfqpoint{4.877984in}{2.651817in}}%
\pgfpathlineto{\pgfqpoint{4.864124in}{2.641495in}}%
\pgfpathlineto{\pgfqpoint{4.850280in}{2.631334in}}%
\pgfpathlineto{\pgfqpoint{4.836452in}{2.621334in}}%
\pgfpathlineto{\pgfqpoint{4.822641in}{2.611494in}}%
\pgfpathlineto{\pgfqpoint{4.815153in}{2.603058in}}%
\pgfpathlineto{\pgfqpoint{4.807658in}{2.594508in}}%
\pgfpathlineto{\pgfqpoint{4.800156in}{2.585844in}}%
\pgfpathlineto{\pgfqpoint{4.792647in}{2.577065in}}%
\pgfpathclose%
\pgfusepath{fill}%
\end{pgfscope}%
\begin{pgfscope}%
\pgfpathrectangle{\pgfqpoint{1.254980in}{0.150000in}}{\pgfqpoint{5.490039in}{5.490039in}}%
\pgfusepath{clip}%
\pgfsetbuttcap%
\pgfsetroundjoin%
\definecolor{currentfill}{rgb}{0.271828,0.209303,0.504434}%
\pgfsetfillcolor{currentfill}%
\pgfsetfillopacity{0.700000}%
\pgfsetlinewidth{0.000000pt}%
\definecolor{currentstroke}{rgb}{0.000000,0.000000,0.000000}%
\pgfsetstrokecolor{currentstroke}%
\pgfsetdash{}{0pt}%
\pgfpathmoveto{\pgfqpoint{4.105315in}{1.941635in}}%
\pgfpathlineto{\pgfqpoint{4.118787in}{1.946120in}}%
\pgfpathlineto{\pgfqpoint{4.132269in}{1.950767in}}%
\pgfpathlineto{\pgfqpoint{4.145763in}{1.955577in}}%
\pgfpathlineto{\pgfqpoint{4.159267in}{1.960548in}}%
\pgfpathlineto{\pgfqpoint{4.167020in}{1.972569in}}%
\pgfpathlineto{\pgfqpoint{4.174768in}{1.984536in}}%
\pgfpathlineto{\pgfqpoint{4.182511in}{1.996448in}}%
\pgfpathlineto{\pgfqpoint{4.190250in}{2.008302in}}%
\pgfpathlineto{\pgfqpoint{4.176748in}{2.003110in}}%
\pgfpathlineto{\pgfqpoint{4.163258in}{1.998081in}}%
\pgfpathlineto{\pgfqpoint{4.149779in}{1.993215in}}%
\pgfpathlineto{\pgfqpoint{4.136311in}{1.988511in}}%
\pgfpathlineto{\pgfqpoint{4.128569in}{1.976866in}}%
\pgfpathlineto{\pgfqpoint{4.120822in}{1.965170in}}%
\pgfpathlineto{\pgfqpoint{4.113071in}{1.953426in}}%
\pgfpathlineto{\pgfqpoint{4.105315in}{1.941635in}}%
\pgfpathclose%
\pgfusepath{fill}%
\end{pgfscope}%
\begin{pgfscope}%
\pgfpathrectangle{\pgfqpoint{1.254980in}{0.150000in}}{\pgfqpoint{5.490039in}{5.490039in}}%
\pgfusepath{clip}%
\pgfsetbuttcap%
\pgfsetroundjoin%
\definecolor{currentfill}{rgb}{0.165117,0.467423,0.558141}%
\pgfsetfillcolor{currentfill}%
\pgfsetfillopacity{0.700000}%
\pgfsetlinewidth{0.000000pt}%
\definecolor{currentstroke}{rgb}{0.000000,0.000000,0.000000}%
\pgfsetstrokecolor{currentstroke}%
\pgfsetdash{}{0pt}%
\pgfpathmoveto{\pgfqpoint{2.199541in}{2.650847in}}%
\pgfpathlineto{\pgfqpoint{2.213325in}{2.626895in}}%
\pgfpathlineto{\pgfqpoint{2.227093in}{2.603250in}}%
\pgfpathlineto{\pgfqpoint{2.240847in}{2.579908in}}%
\pgfpathlineto{\pgfqpoint{2.254585in}{2.556866in}}%
\pgfpathlineto{\pgfqpoint{2.263477in}{2.552402in}}%
\pgfpathlineto{\pgfqpoint{2.272347in}{2.548257in}}%
\pgfpathlineto{\pgfqpoint{2.281196in}{2.544425in}}%
\pgfpathlineto{\pgfqpoint{2.290023in}{2.540902in}}%
\pgfpathlineto{\pgfqpoint{2.276341in}{2.563366in}}%
\pgfpathlineto{\pgfqpoint{2.262645in}{2.586129in}}%
\pgfpathlineto{\pgfqpoint{2.248934in}{2.609195in}}%
\pgfpathlineto{\pgfqpoint{2.235209in}{2.632566in}}%
\pgfpathlineto{\pgfqpoint{2.226326in}{2.636655in}}%
\pgfpathlineto{\pgfqpoint{2.217420in}{2.641062in}}%
\pgfpathlineto{\pgfqpoint{2.208492in}{2.645791in}}%
\pgfpathlineto{\pgfqpoint{2.199541in}{2.650847in}}%
\pgfpathclose%
\pgfusepath{fill}%
\end{pgfscope}%
\begin{pgfscope}%
\pgfpathrectangle{\pgfqpoint{1.254980in}{0.150000in}}{\pgfqpoint{5.490039in}{5.490039in}}%
\pgfusepath{clip}%
\pgfsetbuttcap%
\pgfsetroundjoin%
\definecolor{currentfill}{rgb}{0.277018,0.050344,0.375715}%
\pgfsetfillcolor{currentfill}%
\pgfsetfillopacity{0.700000}%
\pgfsetlinewidth{0.000000pt}%
\definecolor{currentstroke}{rgb}{0.000000,0.000000,0.000000}%
\pgfsetstrokecolor{currentstroke}%
\pgfsetdash{}{0pt}%
\pgfpathmoveto{\pgfqpoint{3.009137in}{1.673860in}}%
\pgfpathlineto{\pgfqpoint{3.022476in}{1.664367in}}%
\pgfpathlineto{\pgfqpoint{3.035815in}{1.655067in}}%
\pgfpathlineto{\pgfqpoint{3.049152in}{1.645960in}}%
\pgfpathlineto{\pgfqpoint{3.062490in}{1.637044in}}%
\pgfpathlineto{\pgfqpoint{3.070730in}{1.640958in}}%
\pgfpathlineto{\pgfqpoint{3.078959in}{1.645073in}}%
\pgfpathlineto{\pgfqpoint{3.087176in}{1.649383in}}%
\pgfpathlineto{\pgfqpoint{3.095382in}{1.653884in}}%
\pgfpathlineto{\pgfqpoint{3.082075in}{1.662271in}}%
\pgfpathlineto{\pgfqpoint{3.068767in}{1.670850in}}%
\pgfpathlineto{\pgfqpoint{3.055460in}{1.679620in}}%
\pgfpathlineto{\pgfqpoint{3.042152in}{1.688584in}}%
\pgfpathlineto{\pgfqpoint{3.033916in}{1.684601in}}%
\pgfpathlineto{\pgfqpoint{3.025668in}{1.680816in}}%
\pgfpathlineto{\pgfqpoint{3.017409in}{1.677234in}}%
\pgfpathlineto{\pgfqpoint{3.009137in}{1.673860in}}%
\pgfpathclose%
\pgfusepath{fill}%
\end{pgfscope}%
\begin{pgfscope}%
\pgfpathrectangle{\pgfqpoint{1.254980in}{0.150000in}}{\pgfqpoint{5.490039in}{5.490039in}}%
\pgfusepath{clip}%
\pgfsetbuttcap%
\pgfsetroundjoin%
\definecolor{currentfill}{rgb}{0.127568,0.566949,0.550556}%
\pgfsetfillcolor{currentfill}%
\pgfsetfillopacity{0.700000}%
\pgfsetlinewidth{0.000000pt}%
\definecolor{currentstroke}{rgb}{0.000000,0.000000,0.000000}%
\pgfsetstrokecolor{currentstroke}%
\pgfsetdash{}{0pt}%
\pgfpathmoveto{\pgfqpoint{5.078486in}{2.828428in}}%
\pgfpathlineto{\pgfqpoint{5.092475in}{2.839799in}}%
\pgfpathlineto{\pgfqpoint{5.106481in}{2.851330in}}%
\pgfpathlineto{\pgfqpoint{5.120505in}{2.863022in}}%
\pgfpathlineto{\pgfqpoint{5.134548in}{2.874875in}}%
\pgfpathlineto{\pgfqpoint{5.141904in}{2.881267in}}%
\pgfpathlineto{\pgfqpoint{5.149252in}{2.887542in}}%
\pgfpathlineto{\pgfqpoint{5.156591in}{2.893701in}}%
\pgfpathlineto{\pgfqpoint{5.163923in}{2.899747in}}%
\pgfpathlineto{\pgfqpoint{5.149892in}{2.888088in}}%
\pgfpathlineto{\pgfqpoint{5.135878in}{2.876590in}}%
\pgfpathlineto{\pgfqpoint{5.121883in}{2.865252in}}%
\pgfpathlineto{\pgfqpoint{5.107905in}{2.854074in}}%
\pgfpathlineto{\pgfqpoint{5.100562in}{2.847823in}}%
\pgfpathlineto{\pgfqpoint{5.093211in}{2.841467in}}%
\pgfpathlineto{\pgfqpoint{5.085853in}{2.835002in}}%
\pgfpathlineto{\pgfqpoint{5.078486in}{2.828428in}}%
\pgfpathclose%
\pgfusepath{fill}%
\end{pgfscope}%
\begin{pgfscope}%
\pgfpathrectangle{\pgfqpoint{1.254980in}{0.150000in}}{\pgfqpoint{5.490039in}{5.490039in}}%
\pgfusepath{clip}%
\pgfsetbuttcap%
\pgfsetroundjoin%
\definecolor{currentfill}{rgb}{0.150148,0.676631,0.506589}%
\pgfsetfillcolor{currentfill}%
\pgfsetfillopacity{0.700000}%
\pgfsetlinewidth{0.000000pt}%
\definecolor{currentstroke}{rgb}{0.000000,0.000000,0.000000}%
\pgfsetstrokecolor{currentstroke}%
\pgfsetdash{}{0pt}%
\pgfpathmoveto{\pgfqpoint{5.449241in}{3.122870in}}%
\pgfpathlineto{\pgfqpoint{5.463456in}{3.135527in}}%
\pgfpathlineto{\pgfqpoint{5.477690in}{3.148344in}}%
\pgfpathlineto{\pgfqpoint{5.491945in}{3.161321in}}%
\pgfpathlineto{\pgfqpoint{5.506219in}{3.174458in}}%
\pgfpathlineto{\pgfqpoint{5.513343in}{3.177694in}}%
\pgfpathlineto{\pgfqpoint{5.520458in}{3.180845in}}%
\pgfpathlineto{\pgfqpoint{5.527564in}{3.183915in}}%
\pgfpathlineto{\pgfqpoint{5.534662in}{3.186907in}}%
\pgfpathlineto{\pgfqpoint{5.520407in}{3.174120in}}%
\pgfpathlineto{\pgfqpoint{5.506173in}{3.161492in}}%
\pgfpathlineto{\pgfqpoint{5.491957in}{3.149023in}}%
\pgfpathlineto{\pgfqpoint{5.477762in}{3.136714in}}%
\pgfpathlineto{\pgfqpoint{5.470644in}{3.133362in}}%
\pgfpathlineto{\pgfqpoint{5.463518in}{3.129940in}}%
\pgfpathlineto{\pgfqpoint{5.456383in}{3.126444in}}%
\pgfpathlineto{\pgfqpoint{5.449241in}{3.122870in}}%
\pgfpathclose%
\pgfusepath{fill}%
\end{pgfscope}%
\begin{pgfscope}%
\pgfpathrectangle{\pgfqpoint{1.254980in}{0.150000in}}{\pgfqpoint{5.490039in}{5.490039in}}%
\pgfusepath{clip}%
\pgfsetbuttcap%
\pgfsetroundjoin%
\definecolor{currentfill}{rgb}{0.269308,0.218818,0.509577}%
\pgfsetfillcolor{currentfill}%
\pgfsetfillopacity{0.700000}%
\pgfsetlinewidth{0.000000pt}%
\definecolor{currentstroke}{rgb}{0.000000,0.000000,0.000000}%
\pgfsetstrokecolor{currentstroke}%
\pgfsetdash{}{0pt}%
\pgfpathmoveto{\pgfqpoint{2.600217in}{2.027658in}}%
\pgfpathlineto{\pgfqpoint{2.613703in}{2.011664in}}%
\pgfpathlineto{\pgfqpoint{2.627182in}{1.995901in}}%
\pgfpathlineto{\pgfqpoint{2.640654in}{1.980370in}}%
\pgfpathlineto{\pgfqpoint{2.654120in}{1.965067in}}%
\pgfpathlineto{\pgfqpoint{2.662681in}{1.964089in}}%
\pgfpathlineto{\pgfqpoint{2.671225in}{1.963387in}}%
\pgfpathlineto{\pgfqpoint{2.679752in}{1.962957in}}%
\pgfpathlineto{\pgfqpoint{2.688262in}{1.962791in}}%
\pgfpathlineto{\pgfqpoint{2.674840in}{1.977519in}}%
\pgfpathlineto{\pgfqpoint{2.661413in}{1.992475in}}%
\pgfpathlineto{\pgfqpoint{2.647979in}{2.007662in}}%
\pgfpathlineto{\pgfqpoint{2.634538in}{2.023079in}}%
\pgfpathlineto{\pgfqpoint{2.625984in}{2.023809in}}%
\pgfpathlineto{\pgfqpoint{2.617413in}{2.024811in}}%
\pgfpathlineto{\pgfqpoint{2.608824in}{2.026092in}}%
\pgfpathlineto{\pgfqpoint{2.600217in}{2.027658in}}%
\pgfpathclose%
\pgfusepath{fill}%
\end{pgfscope}%
\begin{pgfscope}%
\pgfpathrectangle{\pgfqpoint{1.254980in}{0.150000in}}{\pgfqpoint{5.490039in}{5.490039in}}%
\pgfusepath{clip}%
\pgfsetbuttcap%
\pgfsetroundjoin%
\definecolor{currentfill}{rgb}{0.258965,0.251537,0.524736}%
\pgfsetfillcolor{currentfill}%
\pgfsetfillopacity{0.700000}%
\pgfsetlinewidth{0.000000pt}%
\definecolor{currentstroke}{rgb}{0.000000,0.000000,0.000000}%
\pgfsetstrokecolor{currentstroke}%
\pgfsetdash{}{0pt}%
\pgfpathmoveto{\pgfqpoint{2.546199in}{2.093997in}}%
\pgfpathlineto{\pgfqpoint{2.559715in}{2.077055in}}%
\pgfpathlineto{\pgfqpoint{2.573223in}{2.060352in}}%
\pgfpathlineto{\pgfqpoint{2.586724in}{2.043887in}}%
\pgfpathlineto{\pgfqpoint{2.600217in}{2.027658in}}%
\pgfpathlineto{\pgfqpoint{2.608824in}{2.026092in}}%
\pgfpathlineto{\pgfqpoint{2.617413in}{2.024811in}}%
\pgfpathlineto{\pgfqpoint{2.625984in}{2.023809in}}%
\pgfpathlineto{\pgfqpoint{2.634538in}{2.023079in}}%
\pgfpathlineto{\pgfqpoint{2.621091in}{2.038731in}}%
\pgfpathlineto{\pgfqpoint{2.607637in}{2.054617in}}%
\pgfpathlineto{\pgfqpoint{2.594176in}{2.070740in}}%
\pgfpathlineto{\pgfqpoint{2.580708in}{2.087102in}}%
\pgfpathlineto{\pgfqpoint{2.572108in}{2.088398in}}%
\pgfpathlineto{\pgfqpoint{2.563490in}{2.089976in}}%
\pgfpathlineto{\pgfqpoint{2.554854in}{2.091840in}}%
\pgfpathlineto{\pgfqpoint{2.546199in}{2.093997in}}%
\pgfpathclose%
\pgfusepath{fill}%
\end{pgfscope}%
\begin{pgfscope}%
\pgfpathrectangle{\pgfqpoint{1.254980in}{0.150000in}}{\pgfqpoint{5.490039in}{5.490039in}}%
\pgfusepath{clip}%
\pgfsetbuttcap%
\pgfsetroundjoin%
\definecolor{currentfill}{rgb}{0.276194,0.190074,0.493001}%
\pgfsetfillcolor{currentfill}%
\pgfsetfillopacity{0.700000}%
\pgfsetlinewidth{0.000000pt}%
\definecolor{currentstroke}{rgb}{0.000000,0.000000,0.000000}%
\pgfsetstrokecolor{currentstroke}%
\pgfsetdash{}{0pt}%
\pgfpathmoveto{\pgfqpoint{2.654120in}{1.965067in}}%
\pgfpathlineto{\pgfqpoint{2.667580in}{1.949992in}}%
\pgfpathlineto{\pgfqpoint{2.681034in}{1.935142in}}%
\pgfpathlineto{\pgfqpoint{2.694482in}{1.920517in}}%
\pgfpathlineto{\pgfqpoint{2.707924in}{1.906114in}}%
\pgfpathlineto{\pgfqpoint{2.716441in}{1.905720in}}%
\pgfpathlineto{\pgfqpoint{2.724941in}{1.905594in}}%
\pgfpathlineto{\pgfqpoint{2.733426in}{1.905732in}}%
\pgfpathlineto{\pgfqpoint{2.741894in}{1.906127in}}%
\pgfpathlineto{\pgfqpoint{2.728494in}{1.919959in}}%
\pgfpathlineto{\pgfqpoint{2.715089in}{1.934012in}}%
\pgfpathlineto{\pgfqpoint{2.701678in}{1.948289in}}%
\pgfpathlineto{\pgfqpoint{2.688262in}{1.962791in}}%
\pgfpathlineto{\pgfqpoint{2.679752in}{1.962957in}}%
\pgfpathlineto{\pgfqpoint{2.671225in}{1.963387in}}%
\pgfpathlineto{\pgfqpoint{2.662681in}{1.964089in}}%
\pgfpathlineto{\pgfqpoint{2.654120in}{1.965067in}}%
\pgfpathclose%
\pgfusepath{fill}%
\end{pgfscope}%
\begin{pgfscope}%
\pgfpathrectangle{\pgfqpoint{1.254980in}{0.150000in}}{\pgfqpoint{5.490039in}{5.490039in}}%
\pgfusepath{clip}%
\pgfsetbuttcap%
\pgfsetroundjoin%
\definecolor{currentfill}{rgb}{0.190631,0.407061,0.556089}%
\pgfsetfillcolor{currentfill}%
\pgfsetfillopacity{0.700000}%
\pgfsetlinewidth{0.000000pt}%
\definecolor{currentstroke}{rgb}{0.000000,0.000000,0.000000}%
\pgfsetstrokecolor{currentstroke}%
\pgfsetdash{}{0pt}%
\pgfpathmoveto{\pgfqpoint{4.591931in}{2.388674in}}%
\pgfpathlineto{\pgfqpoint{4.605639in}{2.397332in}}%
\pgfpathlineto{\pgfqpoint{4.619362in}{2.406150in}}%
\pgfpathlineto{\pgfqpoint{4.633100in}{2.415130in}}%
\pgfpathlineto{\pgfqpoint{4.646852in}{2.424271in}}%
\pgfpathlineto{\pgfqpoint{4.654444in}{2.434461in}}%
\pgfpathlineto{\pgfqpoint{4.662030in}{2.444535in}}%
\pgfpathlineto{\pgfqpoint{4.669609in}{2.454494in}}%
\pgfpathlineto{\pgfqpoint{4.677182in}{2.464339in}}%
\pgfpathlineto{\pgfqpoint{4.663433in}{2.455180in}}%
\pgfpathlineto{\pgfqpoint{4.649699in}{2.446183in}}%
\pgfpathlineto{\pgfqpoint{4.635980in}{2.437346in}}%
\pgfpathlineto{\pgfqpoint{4.622275in}{2.428670in}}%
\pgfpathlineto{\pgfqpoint{4.614699in}{2.418832in}}%
\pgfpathlineto{\pgfqpoint{4.607115in}{2.408887in}}%
\pgfpathlineto{\pgfqpoint{4.599526in}{2.398834in}}%
\pgfpathlineto{\pgfqpoint{4.591931in}{2.388674in}}%
\pgfpathclose%
\pgfusepath{fill}%
\end{pgfscope}%
\begin{pgfscope}%
\pgfpathrectangle{\pgfqpoint{1.254980in}{0.150000in}}{\pgfqpoint{5.490039in}{5.490039in}}%
\pgfusepath{clip}%
\pgfsetbuttcap%
\pgfsetroundjoin%
\definecolor{currentfill}{rgb}{0.268510,0.009605,0.335427}%
\pgfsetfillcolor{currentfill}%
\pgfsetfillopacity{0.700000}%
\pgfsetlinewidth{0.000000pt}%
\definecolor{currentstroke}{rgb}{0.000000,0.000000,0.000000}%
\pgfsetstrokecolor{currentstroke}%
\pgfsetdash{}{0pt}%
\pgfpathmoveto{\pgfqpoint{3.201858in}{1.593553in}}%
\pgfpathlineto{\pgfqpoint{3.215172in}{1.586845in}}%
\pgfpathlineto{\pgfqpoint{3.228488in}{1.580318in}}%
\pgfpathlineto{\pgfqpoint{3.241805in}{1.573973in}}%
\pgfpathlineto{\pgfqpoint{3.255124in}{1.567809in}}%
\pgfpathlineto{\pgfqpoint{3.263241in}{1.574020in}}%
\pgfpathlineto{\pgfqpoint{3.271348in}{1.580390in}}%
\pgfpathlineto{\pgfqpoint{3.279446in}{1.586914in}}%
\pgfpathlineto{\pgfqpoint{3.287535in}{1.593587in}}%
\pgfpathlineto{\pgfqpoint{3.274239in}{1.599254in}}%
\pgfpathlineto{\pgfqpoint{3.260946in}{1.605102in}}%
\pgfpathlineto{\pgfqpoint{3.247654in}{1.611132in}}%
\pgfpathlineto{\pgfqpoint{3.234364in}{1.617343in}}%
\pgfpathlineto{\pgfqpoint{3.226252in}{1.611157in}}%
\pgfpathlineto{\pgfqpoint{3.218130in}{1.605127in}}%
\pgfpathlineto{\pgfqpoint{3.209999in}{1.599257in}}%
\pgfpathlineto{\pgfqpoint{3.201858in}{1.593553in}}%
\pgfpathclose%
\pgfusepath{fill}%
\end{pgfscope}%
\begin{pgfscope}%
\pgfpathrectangle{\pgfqpoint{1.254980in}{0.150000in}}{\pgfqpoint{5.490039in}{5.490039in}}%
\pgfusepath{clip}%
\pgfsetbuttcap%
\pgfsetroundjoin%
\definecolor{currentfill}{rgb}{0.280267,0.073417,0.397163}%
\pgfsetfillcolor{currentfill}%
\pgfsetfillopacity{0.700000}%
\pgfsetlinewidth{0.000000pt}%
\definecolor{currentstroke}{rgb}{0.000000,0.000000,0.000000}%
\pgfsetstrokecolor{currentstroke}%
\pgfsetdash{}{0pt}%
\pgfpathmoveto{\pgfqpoint{3.734549in}{1.674259in}}%
\pgfpathlineto{\pgfqpoint{3.747905in}{1.674604in}}%
\pgfpathlineto{\pgfqpoint{3.761269in}{1.675116in}}%
\pgfpathlineto{\pgfqpoint{3.774641in}{1.675793in}}%
\pgfpathlineto{\pgfqpoint{3.788020in}{1.676635in}}%
\pgfpathlineto{\pgfqpoint{3.795889in}{1.687747in}}%
\pgfpathlineto{\pgfqpoint{3.803753in}{1.698886in}}%
\pgfpathlineto{\pgfqpoint{3.811612in}{1.710049in}}%
\pgfpathlineto{\pgfqpoint{3.819465in}{1.721232in}}%
\pgfpathlineto{\pgfqpoint{3.806095in}{1.720032in}}%
\pgfpathlineto{\pgfqpoint{3.792732in}{1.718998in}}%
\pgfpathlineto{\pgfqpoint{3.779377in}{1.718129in}}%
\pgfpathlineto{\pgfqpoint{3.766030in}{1.717427in}}%
\pgfpathlineto{\pgfqpoint{3.758167in}{1.706591in}}%
\pgfpathlineto{\pgfqpoint{3.750300in}{1.695782in}}%
\pgfpathlineto{\pgfqpoint{3.742427in}{1.685004in}}%
\pgfpathlineto{\pgfqpoint{3.734549in}{1.674259in}}%
\pgfpathclose%
\pgfusepath{fill}%
\end{pgfscope}%
\begin{pgfscope}%
\pgfpathrectangle{\pgfqpoint{1.254980in}{0.150000in}}{\pgfqpoint{5.490039in}{5.490039in}}%
\pgfusepath{clip}%
\pgfsetbuttcap%
\pgfsetroundjoin%
\definecolor{currentfill}{rgb}{0.248629,0.278775,0.534556}%
\pgfsetfillcolor{currentfill}%
\pgfsetfillopacity{0.700000}%
\pgfsetlinewidth{0.000000pt}%
\definecolor{currentstroke}{rgb}{0.000000,0.000000,0.000000}%
\pgfsetstrokecolor{currentstroke}%
\pgfsetdash{}{0pt}%
\pgfpathmoveto{\pgfqpoint{2.492052in}{2.164203in}}%
\pgfpathlineto{\pgfqpoint{2.505602in}{2.146282in}}%
\pgfpathlineto{\pgfqpoint{2.519143in}{2.128609in}}%
\pgfpathlineto{\pgfqpoint{2.532675in}{2.111182in}}%
\pgfpathlineto{\pgfqpoint{2.546199in}{2.093997in}}%
\pgfpathlineto{\pgfqpoint{2.554854in}{2.091840in}}%
\pgfpathlineto{\pgfqpoint{2.563490in}{2.089976in}}%
\pgfpathlineto{\pgfqpoint{2.572108in}{2.088398in}}%
\pgfpathlineto{\pgfqpoint{2.580708in}{2.087102in}}%
\pgfpathlineto{\pgfqpoint{2.567232in}{2.103704in}}%
\pgfpathlineto{\pgfqpoint{2.553748in}{2.120550in}}%
\pgfpathlineto{\pgfqpoint{2.540256in}{2.137639in}}%
\pgfpathlineto{\pgfqpoint{2.526755in}{2.154976in}}%
\pgfpathlineto{\pgfqpoint{2.518108in}{2.156843in}}%
\pgfpathlineto{\pgfqpoint{2.509442in}{2.158999in}}%
\pgfpathlineto{\pgfqpoint{2.500757in}{2.161451in}}%
\pgfpathlineto{\pgfqpoint{2.492052in}{2.164203in}}%
\pgfpathclose%
\pgfusepath{fill}%
\end{pgfscope}%
\begin{pgfscope}%
\pgfpathrectangle{\pgfqpoint{1.254980in}{0.150000in}}{\pgfqpoint{5.490039in}{5.490039in}}%
\pgfusepath{clip}%
\pgfsetbuttcap%
\pgfsetroundjoin%
\definecolor{currentfill}{rgb}{0.267004,0.004874,0.329415}%
\pgfsetfillcolor{currentfill}%
\pgfsetfillopacity{0.700000}%
\pgfsetlinewidth{0.000000pt}%
\definecolor{currentstroke}{rgb}{0.000000,0.000000,0.000000}%
\pgfsetstrokecolor{currentstroke}%
\pgfsetdash{}{0pt}%
\pgfpathmoveto{\pgfqpoint{3.340745in}{1.572706in}}%
\pgfpathlineto{\pgfqpoint{3.354055in}{1.567930in}}%
\pgfpathlineto{\pgfqpoint{3.367368in}{1.563330in}}%
\pgfpathlineto{\pgfqpoint{3.380684in}{1.558906in}}%
\pgfpathlineto{\pgfqpoint{3.394004in}{1.554656in}}%
\pgfpathlineto{\pgfqpoint{3.402042in}{1.562430in}}%
\pgfpathlineto{\pgfqpoint{3.410073in}{1.570329in}}%
\pgfpathlineto{\pgfqpoint{3.418096in}{1.578349in}}%
\pgfpathlineto{\pgfqpoint{3.426111in}{1.586485in}}%
\pgfpathlineto{\pgfqpoint{3.412811in}{1.590267in}}%
\pgfpathlineto{\pgfqpoint{3.399514in}{1.594224in}}%
\pgfpathlineto{\pgfqpoint{3.386220in}{1.598356in}}%
\pgfpathlineto{\pgfqpoint{3.372931in}{1.602665in}}%
\pgfpathlineto{\pgfqpoint{3.364896in}{1.594985in}}%
\pgfpathlineto{\pgfqpoint{3.356854in}{1.587430in}}%
\pgfpathlineto{\pgfqpoint{3.348804in}{1.580002in}}%
\pgfpathlineto{\pgfqpoint{3.340745in}{1.572706in}}%
\pgfpathclose%
\pgfusepath{fill}%
\end{pgfscope}%
\begin{pgfscope}%
\pgfpathrectangle{\pgfqpoint{1.254980in}{0.150000in}}{\pgfqpoint{5.490039in}{5.490039in}}%
\pgfusepath{clip}%
\pgfsetbuttcap%
\pgfsetroundjoin%
\definecolor{currentfill}{rgb}{0.276022,0.044167,0.370164}%
\pgfsetfillcolor{currentfill}%
\pgfsetfillopacity{0.700000}%
\pgfsetlinewidth{0.000000pt}%
\definecolor{currentstroke}{rgb}{0.000000,0.000000,0.000000}%
\pgfsetstrokecolor{currentstroke}%
\pgfsetdash{}{0pt}%
\pgfpathmoveto{\pgfqpoint{3.649585in}{1.633518in}}%
\pgfpathlineto{\pgfqpoint{3.662924in}{1.632811in}}%
\pgfpathlineto{\pgfqpoint{3.676270in}{1.632271in}}%
\pgfpathlineto{\pgfqpoint{3.689622in}{1.631898in}}%
\pgfpathlineto{\pgfqpoint{3.702982in}{1.631691in}}%
\pgfpathlineto{\pgfqpoint{3.710882in}{1.642265in}}%
\pgfpathlineto{\pgfqpoint{3.718776in}{1.652886in}}%
\pgfpathlineto{\pgfqpoint{3.726665in}{1.663552in}}%
\pgfpathlineto{\pgfqpoint{3.734549in}{1.674259in}}%
\pgfpathlineto{\pgfqpoint{3.721200in}{1.674081in}}%
\pgfpathlineto{\pgfqpoint{3.707858in}{1.674069in}}%
\pgfpathlineto{\pgfqpoint{3.694523in}{1.674224in}}%
\pgfpathlineto{\pgfqpoint{3.681195in}{1.674548in}}%
\pgfpathlineto{\pgfqpoint{3.673301in}{1.664215in}}%
\pgfpathlineto{\pgfqpoint{3.665401in}{1.653930in}}%
\pgfpathlineto{\pgfqpoint{3.657496in}{1.643697in}}%
\pgfpathlineto{\pgfqpoint{3.649585in}{1.633518in}}%
\pgfpathclose%
\pgfusepath{fill}%
\end{pgfscope}%
\begin{pgfscope}%
\pgfpathrectangle{\pgfqpoint{1.254980in}{0.150000in}}{\pgfqpoint{5.490039in}{5.490039in}}%
\pgfusepath{clip}%
\pgfsetbuttcap%
\pgfsetroundjoin%
\definecolor{currentfill}{rgb}{0.227802,0.326594,0.546532}%
\pgfsetfillcolor{currentfill}%
\pgfsetfillopacity{0.700000}%
\pgfsetlinewidth{0.000000pt}%
\definecolor{currentstroke}{rgb}{0.000000,0.000000,0.000000}%
\pgfsetstrokecolor{currentstroke}%
\pgfsetdash{}{0pt}%
\pgfpathmoveto{\pgfqpoint{4.391113in}{2.196769in}}%
\pgfpathlineto{\pgfqpoint{4.404718in}{2.203906in}}%
\pgfpathlineto{\pgfqpoint{4.418336in}{2.211205in}}%
\pgfpathlineto{\pgfqpoint{4.431968in}{2.218665in}}%
\pgfpathlineto{\pgfqpoint{4.445613in}{2.226286in}}%
\pgfpathlineto{\pgfqpoint{4.453279in}{2.237609in}}%
\pgfpathlineto{\pgfqpoint{4.460940in}{2.248836in}}%
\pgfpathlineto{\pgfqpoint{4.468596in}{2.259965in}}%
\pgfpathlineto{\pgfqpoint{4.476246in}{2.270995in}}%
\pgfpathlineto{\pgfqpoint{4.462603in}{2.263268in}}%
\pgfpathlineto{\pgfqpoint{4.448974in}{2.255703in}}%
\pgfpathlineto{\pgfqpoint{4.435358in}{2.248299in}}%
\pgfpathlineto{\pgfqpoint{4.421756in}{2.241056in}}%
\pgfpathlineto{\pgfqpoint{4.414103in}{2.230120in}}%
\pgfpathlineto{\pgfqpoint{4.406445in}{2.219093in}}%
\pgfpathlineto{\pgfqpoint{4.398782in}{2.207976in}}%
\pgfpathlineto{\pgfqpoint{4.391113in}{2.196769in}}%
\pgfpathclose%
\pgfusepath{fill}%
\end{pgfscope}%
\begin{pgfscope}%
\pgfpathrectangle{\pgfqpoint{1.254980in}{0.150000in}}{\pgfqpoint{5.490039in}{5.490039in}}%
\pgfusepath{clip}%
\pgfsetbuttcap%
\pgfsetroundjoin%
\definecolor{currentfill}{rgb}{0.280255,0.165693,0.476498}%
\pgfsetfillcolor{currentfill}%
\pgfsetfillopacity{0.700000}%
\pgfsetlinewidth{0.000000pt}%
\definecolor{currentstroke}{rgb}{0.000000,0.000000,0.000000}%
\pgfsetstrokecolor{currentstroke}%
\pgfsetdash{}{0pt}%
\pgfpathmoveto{\pgfqpoint{2.707924in}{1.906114in}}%
\pgfpathlineto{\pgfqpoint{2.721361in}{1.891931in}}%
\pgfpathlineto{\pgfqpoint{2.734793in}{1.877969in}}%
\pgfpathlineto{\pgfqpoint{2.748220in}{1.864224in}}%
\pgfpathlineto{\pgfqpoint{2.761643in}{1.850695in}}%
\pgfpathlineto{\pgfqpoint{2.770117in}{1.850882in}}%
\pgfpathlineto{\pgfqpoint{2.778576in}{1.851330in}}%
\pgfpathlineto{\pgfqpoint{2.787019in}{1.852033in}}%
\pgfpathlineto{\pgfqpoint{2.795447in}{1.852986in}}%
\pgfpathlineto{\pgfqpoint{2.782066in}{1.865946in}}%
\pgfpathlineto{\pgfqpoint{2.768680in}{1.879122in}}%
\pgfpathlineto{\pgfqpoint{2.755289in}{1.892515in}}%
\pgfpathlineto{\pgfqpoint{2.741894in}{1.906127in}}%
\pgfpathlineto{\pgfqpoint{2.733426in}{1.905732in}}%
\pgfpathlineto{\pgfqpoint{2.724941in}{1.905594in}}%
\pgfpathlineto{\pgfqpoint{2.716441in}{1.905720in}}%
\pgfpathlineto{\pgfqpoint{2.707924in}{1.906114in}}%
\pgfpathclose%
\pgfusepath{fill}%
\end{pgfscope}%
\begin{pgfscope}%
\pgfpathrectangle{\pgfqpoint{1.254980in}{0.150000in}}{\pgfqpoint{5.490039in}{5.490039in}}%
\pgfusepath{clip}%
\pgfsetbuttcap%
\pgfsetroundjoin%
\definecolor{currentfill}{rgb}{0.282656,0.100196,0.422160}%
\pgfsetfillcolor{currentfill}%
\pgfsetfillopacity{0.700000}%
\pgfsetlinewidth{0.000000pt}%
\definecolor{currentstroke}{rgb}{0.000000,0.000000,0.000000}%
\pgfsetstrokecolor{currentstroke}%
\pgfsetdash{}{0pt}%
\pgfpathmoveto{\pgfqpoint{3.819465in}{1.721232in}}%
\pgfpathlineto{\pgfqpoint{3.832844in}{1.722598in}}%
\pgfpathlineto{\pgfqpoint{3.846231in}{1.724128in}}%
\pgfpathlineto{\pgfqpoint{3.859627in}{1.725822in}}%
\pgfpathlineto{\pgfqpoint{3.873031in}{1.727681in}}%
\pgfpathlineto{\pgfqpoint{3.880873in}{1.739222in}}%
\pgfpathlineto{\pgfqpoint{3.888709in}{1.750771in}}%
\pgfpathlineto{\pgfqpoint{3.896541in}{1.762323in}}%
\pgfpathlineto{\pgfqpoint{3.904368in}{1.773877in}}%
\pgfpathlineto{\pgfqpoint{3.890970in}{1.771688in}}%
\pgfpathlineto{\pgfqpoint{3.877581in}{1.769663in}}%
\pgfpathlineto{\pgfqpoint{3.864201in}{1.767803in}}%
\pgfpathlineto{\pgfqpoint{3.850830in}{1.766108in}}%
\pgfpathlineto{\pgfqpoint{3.842996in}{1.754874in}}%
\pgfpathlineto{\pgfqpoint{3.835158in}{1.743648in}}%
\pgfpathlineto{\pgfqpoint{3.827314in}{1.732433in}}%
\pgfpathlineto{\pgfqpoint{3.819465in}{1.721232in}}%
\pgfpathclose%
\pgfusepath{fill}%
\end{pgfscope}%
\begin{pgfscope}%
\pgfpathrectangle{\pgfqpoint{1.254980in}{0.150000in}}{\pgfqpoint{5.490039in}{5.490039in}}%
\pgfusepath{clip}%
\pgfsetbuttcap%
\pgfsetroundjoin%
\definecolor{currentfill}{rgb}{0.233603,0.313828,0.543914}%
\pgfsetfillcolor{currentfill}%
\pgfsetfillopacity{0.700000}%
\pgfsetlinewidth{0.000000pt}%
\definecolor{currentstroke}{rgb}{0.000000,0.000000,0.000000}%
\pgfsetstrokecolor{currentstroke}%
\pgfsetdash{}{0pt}%
\pgfpathmoveto{\pgfqpoint{2.437760in}{2.238403in}}%
\pgfpathlineto{\pgfqpoint{2.451348in}{2.219471in}}%
\pgfpathlineto{\pgfqpoint{2.464925in}{2.200796in}}%
\pgfpathlineto{\pgfqpoint{2.478493in}{2.182374in}}%
\pgfpathlineto{\pgfqpoint{2.492052in}{2.164203in}}%
\pgfpathlineto{\pgfqpoint{2.500757in}{2.161451in}}%
\pgfpathlineto{\pgfqpoint{2.509442in}{2.158999in}}%
\pgfpathlineto{\pgfqpoint{2.518108in}{2.156843in}}%
\pgfpathlineto{\pgfqpoint{2.526755in}{2.154976in}}%
\pgfpathlineto{\pgfqpoint{2.513246in}{2.172560in}}%
\pgfpathlineto{\pgfqpoint{2.499729in}{2.190396in}}%
\pgfpathlineto{\pgfqpoint{2.486202in}{2.208484in}}%
\pgfpathlineto{\pgfqpoint{2.472665in}{2.226826in}}%
\pgfpathlineto{\pgfqpoint{2.463969in}{2.229269in}}%
\pgfpathlineto{\pgfqpoint{2.455252in}{2.232008in}}%
\pgfpathlineto{\pgfqpoint{2.446516in}{2.235051in}}%
\pgfpathlineto{\pgfqpoint{2.437760in}{2.238403in}}%
\pgfpathclose%
\pgfusepath{fill}%
\end{pgfscope}%
\begin{pgfscope}%
\pgfpathrectangle{\pgfqpoint{1.254980in}{0.150000in}}{\pgfqpoint{5.490039in}{5.490039in}}%
\pgfusepath{clip}%
\pgfsetbuttcap%
\pgfsetroundjoin%
\definecolor{currentfill}{rgb}{0.272594,0.025563,0.353093}%
\pgfsetfillcolor{currentfill}%
\pgfsetfillopacity{0.700000}%
\pgfsetlinewidth{0.000000pt}%
\definecolor{currentstroke}{rgb}{0.000000,0.000000,0.000000}%
\pgfsetstrokecolor{currentstroke}%
\pgfsetdash{}{0pt}%
\pgfpathmoveto{\pgfqpoint{3.564534in}{1.599594in}}%
\pgfpathlineto{\pgfqpoint{3.577862in}{1.597799in}}%
\pgfpathlineto{\pgfqpoint{3.591196in}{1.596174in}}%
\pgfpathlineto{\pgfqpoint{3.604535in}{1.594718in}}%
\pgfpathlineto{\pgfqpoint{3.617881in}{1.593430in}}%
\pgfpathlineto{\pgfqpoint{3.625816in}{1.603351in}}%
\pgfpathlineto{\pgfqpoint{3.633745in}{1.613341in}}%
\pgfpathlineto{\pgfqpoint{3.641668in}{1.623399in}}%
\pgfpathlineto{\pgfqpoint{3.649585in}{1.633518in}}%
\pgfpathlineto{\pgfqpoint{3.636252in}{1.634394in}}%
\pgfpathlineto{\pgfqpoint{3.622925in}{1.635438in}}%
\pgfpathlineto{\pgfqpoint{3.609605in}{1.636652in}}%
\pgfpathlineto{\pgfqpoint{3.596290in}{1.638034in}}%
\pgfpathlineto{\pgfqpoint{3.588361in}{1.628316in}}%
\pgfpathlineto{\pgfqpoint{3.580425in}{1.618667in}}%
\pgfpathlineto{\pgfqpoint{3.572483in}{1.609092in}}%
\pgfpathlineto{\pgfqpoint{3.564534in}{1.599594in}}%
\pgfpathclose%
\pgfusepath{fill}%
\end{pgfscope}%
\begin{pgfscope}%
\pgfpathrectangle{\pgfqpoint{1.254980in}{0.150000in}}{\pgfqpoint{5.490039in}{5.490039in}}%
\pgfusepath{clip}%
\pgfsetbuttcap%
\pgfsetroundjoin%
\definecolor{currentfill}{rgb}{0.180653,0.701402,0.488189}%
\pgfsetfillcolor{currentfill}%
\pgfsetfillopacity{0.700000}%
\pgfsetlinewidth{0.000000pt}%
\definecolor{currentstroke}{rgb}{0.000000,0.000000,0.000000}%
\pgfsetstrokecolor{currentstroke}%
\pgfsetdash{}{0pt}%
\pgfpathmoveto{\pgfqpoint{5.534662in}{3.186907in}}%
\pgfpathlineto{\pgfqpoint{5.548936in}{3.199855in}}%
\pgfpathlineto{\pgfqpoint{5.563230in}{3.212962in}}%
\pgfpathlineto{\pgfqpoint{5.577544in}{3.226229in}}%
\pgfpathlineto{\pgfqpoint{5.591879in}{3.239657in}}%
\pgfpathlineto{\pgfqpoint{5.598946in}{3.242206in}}%
\pgfpathlineto{\pgfqpoint{5.606005in}{3.244680in}}%
\pgfpathlineto{\pgfqpoint{5.613055in}{3.247081in}}%
\pgfpathlineto{\pgfqpoint{5.620097in}{3.249415in}}%
\pgfpathlineto{\pgfqpoint{5.605785in}{3.236369in}}%
\pgfpathlineto{\pgfqpoint{5.591492in}{3.223482in}}%
\pgfpathlineto{\pgfqpoint{5.577220in}{3.210755in}}%
\pgfpathlineto{\pgfqpoint{5.562967in}{3.198187in}}%
\pgfpathlineto{\pgfqpoint{5.555903in}{3.195462in}}%
\pgfpathlineto{\pgfqpoint{5.548831in}{3.192677in}}%
\pgfpathlineto{\pgfqpoint{5.541751in}{3.189827in}}%
\pgfpathlineto{\pgfqpoint{5.534662in}{3.186907in}}%
\pgfpathclose%
\pgfusepath{fill}%
\end{pgfscope}%
\begin{pgfscope}%
\pgfpathrectangle{\pgfqpoint{1.254980in}{0.150000in}}{\pgfqpoint{5.490039in}{5.490039in}}%
\pgfusepath{clip}%
\pgfsetbuttcap%
\pgfsetroundjoin%
\definecolor{currentfill}{rgb}{0.260571,0.246922,0.522828}%
\pgfsetfillcolor{currentfill}%
\pgfsetfillopacity{0.700000}%
\pgfsetlinewidth{0.000000pt}%
\definecolor{currentstroke}{rgb}{0.000000,0.000000,0.000000}%
\pgfsetstrokecolor{currentstroke}%
\pgfsetdash{}{0pt}%
\pgfpathmoveto{\pgfqpoint{4.190250in}{2.008302in}}%
\pgfpathlineto{\pgfqpoint{4.203763in}{2.013655in}}%
\pgfpathlineto{\pgfqpoint{4.217288in}{2.019171in}}%
\pgfpathlineto{\pgfqpoint{4.230825in}{2.024848in}}%
\pgfpathlineto{\pgfqpoint{4.244373in}{2.030688in}}%
\pgfpathlineto{\pgfqpoint{4.252105in}{2.042684in}}%
\pgfpathlineto{\pgfqpoint{4.259831in}{2.054613in}}%
\pgfpathlineto{\pgfqpoint{4.267552in}{2.066472in}}%
\pgfpathlineto{\pgfqpoint{4.275269in}{2.078259in}}%
\pgfpathlineto{\pgfqpoint{4.261723in}{2.072229in}}%
\pgfpathlineto{\pgfqpoint{4.248189in}{2.066359in}}%
\pgfpathlineto{\pgfqpoint{4.234667in}{2.060652in}}%
\pgfpathlineto{\pgfqpoint{4.221156in}{2.055107in}}%
\pgfpathlineto{\pgfqpoint{4.213437in}{2.043501in}}%
\pgfpathlineto{\pgfqpoint{4.205713in}{2.031830in}}%
\pgfpathlineto{\pgfqpoint{4.197984in}{2.020096in}}%
\pgfpathlineto{\pgfqpoint{4.190250in}{2.008302in}}%
\pgfpathclose%
\pgfusepath{fill}%
\end{pgfscope}%
\begin{pgfscope}%
\pgfpathrectangle{\pgfqpoint{1.254980in}{0.150000in}}{\pgfqpoint{5.490039in}{5.490039in}}%
\pgfusepath{clip}%
\pgfsetbuttcap%
\pgfsetroundjoin%
\definecolor{currentfill}{rgb}{0.282623,0.140926,0.457517}%
\pgfsetfillcolor{currentfill}%
\pgfsetfillopacity{0.700000}%
\pgfsetlinewidth{0.000000pt}%
\definecolor{currentstroke}{rgb}{0.000000,0.000000,0.000000}%
\pgfsetstrokecolor{currentstroke}%
\pgfsetdash{}{0pt}%
\pgfpathmoveto{\pgfqpoint{2.761643in}{1.850695in}}%
\pgfpathlineto{\pgfqpoint{2.775061in}{1.837381in}}%
\pgfpathlineto{\pgfqpoint{2.788474in}{1.824281in}}%
\pgfpathlineto{\pgfqpoint{2.801884in}{1.811393in}}%
\pgfpathlineto{\pgfqpoint{2.815289in}{1.798716in}}%
\pgfpathlineto{\pgfqpoint{2.823723in}{1.799481in}}%
\pgfpathlineto{\pgfqpoint{2.832142in}{1.800500in}}%
\pgfpathlineto{\pgfqpoint{2.840546in}{1.801766in}}%
\pgfpathlineto{\pgfqpoint{2.848936in}{1.803274in}}%
\pgfpathlineto{\pgfqpoint{2.835569in}{1.815385in}}%
\pgfpathlineto{\pgfqpoint{2.822199in}{1.827707in}}%
\pgfpathlineto{\pgfqpoint{2.808825in}{1.840240in}}%
\pgfpathlineto{\pgfqpoint{2.795447in}{1.852986in}}%
\pgfpathlineto{\pgfqpoint{2.787019in}{1.852033in}}%
\pgfpathlineto{\pgfqpoint{2.778576in}{1.851330in}}%
\pgfpathlineto{\pgfqpoint{2.770117in}{1.850882in}}%
\pgfpathlineto{\pgfqpoint{2.761643in}{1.850695in}}%
\pgfpathclose%
\pgfusepath{fill}%
\end{pgfscope}%
\begin{pgfscope}%
\pgfpathrectangle{\pgfqpoint{1.254980in}{0.150000in}}{\pgfqpoint{5.490039in}{5.490039in}}%
\pgfusepath{clip}%
\pgfsetbuttcap%
\pgfsetroundjoin%
\definecolor{currentfill}{rgb}{0.283072,0.130895,0.449241}%
\pgfsetfillcolor{currentfill}%
\pgfsetfillopacity{0.700000}%
\pgfsetlinewidth{0.000000pt}%
\definecolor{currentstroke}{rgb}{0.000000,0.000000,0.000000}%
\pgfsetstrokecolor{currentstroke}%
\pgfsetdash{}{0pt}%
\pgfpathmoveto{\pgfqpoint{3.904368in}{1.773877in}}%
\pgfpathlineto{\pgfqpoint{3.917774in}{1.776230in}}%
\pgfpathlineto{\pgfqpoint{3.931190in}{1.778747in}}%
\pgfpathlineto{\pgfqpoint{3.944614in}{1.781428in}}%
\pgfpathlineto{\pgfqpoint{3.958049in}{1.784272in}}%
\pgfpathlineto{\pgfqpoint{3.965865in}{1.796138in}}%
\pgfpathlineto{\pgfqpoint{3.973676in}{1.807991in}}%
\pgfpathlineto{\pgfqpoint{3.981483in}{1.819831in}}%
\pgfpathlineto{\pgfqpoint{3.989285in}{1.831653in}}%
\pgfpathlineto{\pgfqpoint{3.975856in}{1.828506in}}%
\pgfpathlineto{\pgfqpoint{3.962437in}{1.825522in}}%
\pgfpathlineto{\pgfqpoint{3.949027in}{1.822702in}}%
\pgfpathlineto{\pgfqpoint{3.935627in}{1.820047in}}%
\pgfpathlineto{\pgfqpoint{3.927819in}{1.808516in}}%
\pgfpathlineto{\pgfqpoint{3.920007in}{1.796976in}}%
\pgfpathlineto{\pgfqpoint{3.912190in}{1.785429in}}%
\pgfpathlineto{\pgfqpoint{3.904368in}{1.773877in}}%
\pgfpathclose%
\pgfusepath{fill}%
\end{pgfscope}%
\begin{pgfscope}%
\pgfpathrectangle{\pgfqpoint{1.254980in}{0.150000in}}{\pgfqpoint{5.490039in}{5.490039in}}%
\pgfusepath{clip}%
\pgfsetbuttcap%
\pgfsetroundjoin%
\definecolor{currentfill}{rgb}{0.149039,0.508051,0.557250}%
\pgfsetfillcolor{currentfill}%
\pgfsetfillopacity{0.700000}%
\pgfsetlinewidth{0.000000pt}%
\definecolor{currentstroke}{rgb}{0.000000,0.000000,0.000000}%
\pgfsetstrokecolor{currentstroke}%
\pgfsetdash{}{0pt}%
\pgfpathmoveto{\pgfqpoint{4.877984in}{2.651817in}}%
\pgfpathlineto{\pgfqpoint{4.891861in}{2.662299in}}%
\pgfpathlineto{\pgfqpoint{4.905754in}{2.672943in}}%
\pgfpathlineto{\pgfqpoint{4.919664in}{2.683747in}}%
\pgfpathlineto{\pgfqpoint{4.933590in}{2.694713in}}%
\pgfpathlineto{\pgfqpoint{4.941058in}{2.702857in}}%
\pgfpathlineto{\pgfqpoint{4.948519in}{2.710876in}}%
\pgfpathlineto{\pgfqpoint{4.955972in}{2.718772in}}%
\pgfpathlineto{\pgfqpoint{4.963417in}{2.726546in}}%
\pgfpathlineto{\pgfqpoint{4.949497in}{2.715683in}}%
\pgfpathlineto{\pgfqpoint{4.935594in}{2.704981in}}%
\pgfpathlineto{\pgfqpoint{4.921708in}{2.694439in}}%
\pgfpathlineto{\pgfqpoint{4.907838in}{2.684059in}}%
\pgfpathlineto{\pgfqpoint{4.900386in}{2.676172in}}%
\pgfpathlineto{\pgfqpoint{4.892926in}{2.668170in}}%
\pgfpathlineto{\pgfqpoint{4.885459in}{2.660052in}}%
\pgfpathlineto{\pgfqpoint{4.877984in}{2.651817in}}%
\pgfpathclose%
\pgfusepath{fill}%
\end{pgfscope}%
\begin{pgfscope}%
\pgfpathrectangle{\pgfqpoint{1.254980in}{0.150000in}}{\pgfqpoint{5.490039in}{5.490039in}}%
\pgfusepath{clip}%
\pgfsetbuttcap%
\pgfsetroundjoin%
\definecolor{currentfill}{rgb}{0.120565,0.596422,0.543611}%
\pgfsetfillcolor{currentfill}%
\pgfsetfillopacity{0.700000}%
\pgfsetlinewidth{0.000000pt}%
\definecolor{currentstroke}{rgb}{0.000000,0.000000,0.000000}%
\pgfsetstrokecolor{currentstroke}%
\pgfsetdash{}{0pt}%
\pgfpathmoveto{\pgfqpoint{5.163923in}{2.899747in}}%
\pgfpathlineto{\pgfqpoint{5.177972in}{2.911566in}}%
\pgfpathlineto{\pgfqpoint{5.192040in}{2.923546in}}%
\pgfpathlineto{\pgfqpoint{5.206126in}{2.935687in}}%
\pgfpathlineto{\pgfqpoint{5.220230in}{2.947988in}}%
\pgfpathlineto{\pgfqpoint{5.227541in}{2.953710in}}%
\pgfpathlineto{\pgfqpoint{5.234844in}{2.959317in}}%
\pgfpathlineto{\pgfqpoint{5.242138in}{2.964812in}}%
\pgfpathlineto{\pgfqpoint{5.249424in}{2.970197in}}%
\pgfpathlineto{\pgfqpoint{5.235332in}{2.958121in}}%
\pgfpathlineto{\pgfqpoint{5.221259in}{2.946205in}}%
\pgfpathlineto{\pgfqpoint{5.207204in}{2.934450in}}%
\pgfpathlineto{\pgfqpoint{5.193167in}{2.922855in}}%
\pgfpathlineto{\pgfqpoint{5.185868in}{2.917234in}}%
\pgfpathlineto{\pgfqpoint{5.178561in}{2.911511in}}%
\pgfpathlineto{\pgfqpoint{5.171246in}{2.905683in}}%
\pgfpathlineto{\pgfqpoint{5.163923in}{2.899747in}}%
\pgfpathclose%
\pgfusepath{fill}%
\end{pgfscope}%
\begin{pgfscope}%
\pgfpathrectangle{\pgfqpoint{1.254980in}{0.150000in}}{\pgfqpoint{5.490039in}{5.490039in}}%
\pgfusepath{clip}%
\pgfsetbuttcap%
\pgfsetroundjoin%
\definecolor{currentfill}{rgb}{0.273809,0.031497,0.358853}%
\pgfsetfillcolor{currentfill}%
\pgfsetfillopacity{0.700000}%
\pgfsetlinewidth{0.000000pt}%
\definecolor{currentstroke}{rgb}{0.000000,0.000000,0.000000}%
\pgfsetstrokecolor{currentstroke}%
\pgfsetdash{}{0pt}%
\pgfpathmoveto{\pgfqpoint{3.062490in}{1.637044in}}%
\pgfpathlineto{\pgfqpoint{3.075827in}{1.628319in}}%
\pgfpathlineto{\pgfqpoint{3.089163in}{1.619783in}}%
\pgfpathlineto{\pgfqpoint{3.102500in}{1.611436in}}%
\pgfpathlineto{\pgfqpoint{3.115837in}{1.603276in}}%
\pgfpathlineto{\pgfqpoint{3.124048in}{1.607729in}}%
\pgfpathlineto{\pgfqpoint{3.132247in}{1.612375in}}%
\pgfpathlineto{\pgfqpoint{3.140436in}{1.617209in}}%
\pgfpathlineto{\pgfqpoint{3.148614in}{1.622227in}}%
\pgfpathlineto{\pgfqpoint{3.135305in}{1.629859in}}%
\pgfpathlineto{\pgfqpoint{3.121997in}{1.637679in}}%
\pgfpathlineto{\pgfqpoint{3.108689in}{1.645687in}}%
\pgfpathlineto{\pgfqpoint{3.095382in}{1.653884in}}%
\pgfpathlineto{\pgfqpoint{3.087176in}{1.649383in}}%
\pgfpathlineto{\pgfqpoint{3.078959in}{1.645073in}}%
\pgfpathlineto{\pgfqpoint{3.070730in}{1.640958in}}%
\pgfpathlineto{\pgfqpoint{3.062490in}{1.637044in}}%
\pgfpathclose%
\pgfusepath{fill}%
\end{pgfscope}%
\begin{pgfscope}%
\pgfpathrectangle{\pgfqpoint{1.254980in}{0.150000in}}{\pgfqpoint{5.490039in}{5.490039in}}%
\pgfusepath{clip}%
\pgfsetbuttcap%
\pgfsetroundjoin%
\definecolor{currentfill}{rgb}{0.218130,0.347432,0.550038}%
\pgfsetfillcolor{currentfill}%
\pgfsetfillopacity{0.700000}%
\pgfsetlinewidth{0.000000pt}%
\definecolor{currentstroke}{rgb}{0.000000,0.000000,0.000000}%
\pgfsetstrokecolor{currentstroke}%
\pgfsetdash{}{0pt}%
\pgfpathmoveto{\pgfqpoint{2.383306in}{2.316734in}}%
\pgfpathlineto{\pgfqpoint{2.396935in}{2.296756in}}%
\pgfpathlineto{\pgfqpoint{2.410554in}{2.277043in}}%
\pgfpathlineto{\pgfqpoint{2.424162in}{2.257593in}}%
\pgfpathlineto{\pgfqpoint{2.437760in}{2.238403in}}%
\pgfpathlineto{\pgfqpoint{2.446516in}{2.235051in}}%
\pgfpathlineto{\pgfqpoint{2.455252in}{2.232008in}}%
\pgfpathlineto{\pgfqpoint{2.463969in}{2.229269in}}%
\pgfpathlineto{\pgfqpoint{2.472665in}{2.226826in}}%
\pgfpathlineto{\pgfqpoint{2.459119in}{2.245426in}}%
\pgfpathlineto{\pgfqpoint{2.445564in}{2.264285in}}%
\pgfpathlineto{\pgfqpoint{2.431998in}{2.283406in}}%
\pgfpathlineto{\pgfqpoint{2.418421in}{2.302790in}}%
\pgfpathlineto{\pgfqpoint{2.409673in}{2.305812in}}%
\pgfpathlineto{\pgfqpoint{2.400905in}{2.309139in}}%
\pgfpathlineto{\pgfqpoint{2.392116in}{2.312778in}}%
\pgfpathlineto{\pgfqpoint{2.383306in}{2.316734in}}%
\pgfpathclose%
\pgfusepath{fill}%
\end{pgfscope}%
\begin{pgfscope}%
\pgfpathrectangle{\pgfqpoint{1.254980in}{0.150000in}}{\pgfqpoint{5.490039in}{5.490039in}}%
\pgfusepath{clip}%
\pgfsetbuttcap%
\pgfsetroundjoin%
\definecolor{currentfill}{rgb}{0.149039,0.508051,0.557250}%
\pgfsetfillcolor{currentfill}%
\pgfsetfillopacity{0.700000}%
\pgfsetlinewidth{0.000000pt}%
\definecolor{currentstroke}{rgb}{0.000000,0.000000,0.000000}%
\pgfsetstrokecolor{currentstroke}%
\pgfsetdash{}{0pt}%
\pgfpathmoveto{\pgfqpoint{2.144246in}{2.749785in}}%
\pgfpathlineto{\pgfqpoint{2.158095in}{2.724574in}}%
\pgfpathlineto{\pgfqpoint{2.171927in}{2.699684in}}%
\pgfpathlineto{\pgfqpoint{2.185742in}{2.675109in}}%
\pgfpathlineto{\pgfqpoint{2.199541in}{2.650847in}}%
\pgfpathlineto{\pgfqpoint{2.208492in}{2.645791in}}%
\pgfpathlineto{\pgfqpoint{2.217420in}{2.641062in}}%
\pgfpathlineto{\pgfqpoint{2.226326in}{2.636655in}}%
\pgfpathlineto{\pgfqpoint{2.235209in}{2.632566in}}%
\pgfpathlineto{\pgfqpoint{2.221468in}{2.656244in}}%
\pgfpathlineto{\pgfqpoint{2.207712in}{2.680234in}}%
\pgfpathlineto{\pgfqpoint{2.193941in}{2.704539in}}%
\pgfpathlineto{\pgfqpoint{2.180153in}{2.729161in}}%
\pgfpathlineto{\pgfqpoint{2.171211in}{2.733823in}}%
\pgfpathlineto{\pgfqpoint{2.162246in}{2.738811in}}%
\pgfpathlineto{\pgfqpoint{2.153258in}{2.744129in}}%
\pgfpathlineto{\pgfqpoint{2.144246in}{2.749785in}}%
\pgfpathclose%
\pgfusepath{fill}%
\end{pgfscope}%
\begin{pgfscope}%
\pgfpathrectangle{\pgfqpoint{1.254980in}{0.150000in}}{\pgfqpoint{5.490039in}{5.490039in}}%
\pgfusepath{clip}%
\pgfsetbuttcap%
\pgfsetroundjoin%
\definecolor{currentfill}{rgb}{0.268510,0.009605,0.335427}%
\pgfsetfillcolor{currentfill}%
\pgfsetfillopacity{0.700000}%
\pgfsetlinewidth{0.000000pt}%
\definecolor{currentstroke}{rgb}{0.000000,0.000000,0.000000}%
\pgfsetstrokecolor{currentstroke}%
\pgfsetdash{}{0pt}%
\pgfpathmoveto{\pgfqpoint{3.479356in}{1.573095in}}%
\pgfpathlineto{\pgfqpoint{3.492679in}{1.570178in}}%
\pgfpathlineto{\pgfqpoint{3.506006in}{1.567433in}}%
\pgfpathlineto{\pgfqpoint{3.519338in}{1.564859in}}%
\pgfpathlineto{\pgfqpoint{3.532676in}{1.562454in}}%
\pgfpathlineto{\pgfqpoint{3.540650in}{1.571603in}}%
\pgfpathlineto{\pgfqpoint{3.548618in}{1.580845in}}%
\pgfpathlineto{\pgfqpoint{3.556580in}{1.590177in}}%
\pgfpathlineto{\pgfqpoint{3.564534in}{1.599594in}}%
\pgfpathlineto{\pgfqpoint{3.551212in}{1.601558in}}%
\pgfpathlineto{\pgfqpoint{3.537895in}{1.603693in}}%
\pgfpathlineto{\pgfqpoint{3.524584in}{1.605998in}}%
\pgfpathlineto{\pgfqpoint{3.511277in}{1.608475in}}%
\pgfpathlineto{\pgfqpoint{3.503307in}{1.599488in}}%
\pgfpathlineto{\pgfqpoint{3.495331in}{1.590592in}}%
\pgfpathlineto{\pgfqpoint{3.487347in}{1.581793in}}%
\pgfpathlineto{\pgfqpoint{3.479356in}{1.573095in}}%
\pgfpathclose%
\pgfusepath{fill}%
\end{pgfscope}%
\begin{pgfscope}%
\pgfpathrectangle{\pgfqpoint{1.254980in}{0.150000in}}{\pgfqpoint{5.490039in}{5.490039in}}%
\pgfusepath{clip}%
\pgfsetbuttcap%
\pgfsetroundjoin%
\definecolor{currentfill}{rgb}{0.283197,0.115680,0.436115}%
\pgfsetfillcolor{currentfill}%
\pgfsetfillopacity{0.700000}%
\pgfsetlinewidth{0.000000pt}%
\definecolor{currentstroke}{rgb}{0.000000,0.000000,0.000000}%
\pgfsetstrokecolor{currentstroke}%
\pgfsetdash{}{0pt}%
\pgfpathmoveto{\pgfqpoint{2.815289in}{1.798716in}}%
\pgfpathlineto{\pgfqpoint{2.828691in}{1.786248in}}%
\pgfpathlineto{\pgfqpoint{2.842090in}{1.773988in}}%
\pgfpathlineto{\pgfqpoint{2.855485in}{1.761934in}}%
\pgfpathlineto{\pgfqpoint{2.868877in}{1.750087in}}%
\pgfpathlineto{\pgfqpoint{2.877272in}{1.751428in}}%
\pgfpathlineto{\pgfqpoint{2.885653in}{1.753015in}}%
\pgfpathlineto{\pgfqpoint{2.894019in}{1.754841in}}%
\pgfpathlineto{\pgfqpoint{2.902372in}{1.756902in}}%
\pgfpathlineto{\pgfqpoint{2.889017in}{1.768186in}}%
\pgfpathlineto{\pgfqpoint{2.875660in}{1.779675in}}%
\pgfpathlineto{\pgfqpoint{2.862299in}{1.791371in}}%
\pgfpathlineto{\pgfqpoint{2.848936in}{1.803274in}}%
\pgfpathlineto{\pgfqpoint{2.840546in}{1.801766in}}%
\pgfpathlineto{\pgfqpoint{2.832142in}{1.800500in}}%
\pgfpathlineto{\pgfqpoint{2.823723in}{1.799481in}}%
\pgfpathlineto{\pgfqpoint{2.815289in}{1.798716in}}%
\pgfpathclose%
\pgfusepath{fill}%
\end{pgfscope}%
\begin{pgfscope}%
\pgfpathrectangle{\pgfqpoint{1.254980in}{0.150000in}}{\pgfqpoint{5.490039in}{5.490039in}}%
\pgfusepath{clip}%
\pgfsetbuttcap%
\pgfsetroundjoin%
\definecolor{currentfill}{rgb}{0.214000,0.722114,0.469588}%
\pgfsetfillcolor{currentfill}%
\pgfsetfillopacity{0.700000}%
\pgfsetlinewidth{0.000000pt}%
\definecolor{currentstroke}{rgb}{0.000000,0.000000,0.000000}%
\pgfsetstrokecolor{currentstroke}%
\pgfsetdash{}{0pt}%
\pgfpathmoveto{\pgfqpoint{5.620097in}{3.249415in}}%
\pgfpathlineto{\pgfqpoint{5.634430in}{3.262620in}}%
\pgfpathlineto{\pgfqpoint{5.648783in}{3.275986in}}%
\pgfpathlineto{\pgfqpoint{5.663156in}{3.289511in}}%
\pgfpathlineto{\pgfqpoint{5.677550in}{3.303197in}}%
\pgfpathlineto{\pgfqpoint{5.684560in}{3.305065in}}%
\pgfpathlineto{\pgfqpoint{5.691561in}{3.306868in}}%
\pgfpathlineto{\pgfqpoint{5.698552in}{3.308609in}}%
\pgfpathlineto{\pgfqpoint{5.705536in}{3.310293in}}%
\pgfpathlineto{\pgfqpoint{5.691166in}{3.297021in}}%
\pgfpathlineto{\pgfqpoint{5.676817in}{3.283908in}}%
\pgfpathlineto{\pgfqpoint{5.662488in}{3.270954in}}%
\pgfpathlineto{\pgfqpoint{5.648179in}{3.258159in}}%
\pgfpathlineto{\pgfqpoint{5.641171in}{3.256053in}}%
\pgfpathlineto{\pgfqpoint{5.634154in}{3.253896in}}%
\pgfpathlineto{\pgfqpoint{5.627130in}{3.251685in}}%
\pgfpathlineto{\pgfqpoint{5.620097in}{3.249415in}}%
\pgfpathclose%
\pgfusepath{fill}%
\end{pgfscope}%
\begin{pgfscope}%
\pgfpathrectangle{\pgfqpoint{1.254980in}{0.150000in}}{\pgfqpoint{5.490039in}{5.490039in}}%
\pgfusepath{clip}%
\pgfsetbuttcap%
\pgfsetroundjoin%
\definecolor{currentfill}{rgb}{0.177423,0.437527,0.557565}%
\pgfsetfillcolor{currentfill}%
\pgfsetfillopacity{0.700000}%
\pgfsetlinewidth{0.000000pt}%
\definecolor{currentstroke}{rgb}{0.000000,0.000000,0.000000}%
\pgfsetstrokecolor{currentstroke}%
\pgfsetdash{}{0pt}%
\pgfpathmoveto{\pgfqpoint{4.677182in}{2.464339in}}%
\pgfpathlineto{\pgfqpoint{4.690946in}{2.473659in}}%
\pgfpathlineto{\pgfqpoint{4.704726in}{2.483140in}}%
\pgfpathlineto{\pgfqpoint{4.718521in}{2.492783in}}%
\pgfpathlineto{\pgfqpoint{4.732332in}{2.502586in}}%
\pgfpathlineto{\pgfqpoint{4.739895in}{2.512316in}}%
\pgfpathlineto{\pgfqpoint{4.747451in}{2.521925in}}%
\pgfpathlineto{\pgfqpoint{4.755001in}{2.531413in}}%
\pgfpathlineto{\pgfqpoint{4.762543in}{2.540781in}}%
\pgfpathlineto{\pgfqpoint{4.748737in}{2.530989in}}%
\pgfpathlineto{\pgfqpoint{4.734946in}{2.521359in}}%
\pgfpathlineto{\pgfqpoint{4.721170in}{2.511890in}}%
\pgfpathlineto{\pgfqpoint{4.707410in}{2.502581in}}%
\pgfpathlineto{\pgfqpoint{4.699863in}{2.493190in}}%
\pgfpathlineto{\pgfqpoint{4.692309in}{2.483687in}}%
\pgfpathlineto{\pgfqpoint{4.684749in}{2.474070in}}%
\pgfpathlineto{\pgfqpoint{4.677182in}{2.464339in}}%
\pgfpathclose%
\pgfusepath{fill}%
\end{pgfscope}%
\begin{pgfscope}%
\pgfpathrectangle{\pgfqpoint{1.254980in}{0.150000in}}{\pgfqpoint{5.490039in}{5.490039in}}%
\pgfusepath{clip}%
\pgfsetbuttcap%
\pgfsetroundjoin%
\definecolor{currentfill}{rgb}{0.280868,0.160771,0.472899}%
\pgfsetfillcolor{currentfill}%
\pgfsetfillopacity{0.700000}%
\pgfsetlinewidth{0.000000pt}%
\definecolor{currentstroke}{rgb}{0.000000,0.000000,0.000000}%
\pgfsetstrokecolor{currentstroke}%
\pgfsetdash{}{0pt}%
\pgfpathmoveto{\pgfqpoint{3.989285in}{1.831653in}}%
\pgfpathlineto{\pgfqpoint{4.002724in}{1.834964in}}%
\pgfpathlineto{\pgfqpoint{4.016173in}{1.838437in}}%
\pgfpathlineto{\pgfqpoint{4.029631in}{1.842073in}}%
\pgfpathlineto{\pgfqpoint{4.043100in}{1.845872in}}%
\pgfpathlineto{\pgfqpoint{4.050893in}{1.857961in}}%
\pgfpathlineto{\pgfqpoint{4.058681in}{1.870021in}}%
\pgfpathlineto{\pgfqpoint{4.066465in}{1.882049in}}%
\pgfpathlineto{\pgfqpoint{4.074244in}{1.894043in}}%
\pgfpathlineto{\pgfqpoint{4.060780in}{1.889969in}}%
\pgfpathlineto{\pgfqpoint{4.047325in}{1.886057in}}%
\pgfpathlineto{\pgfqpoint{4.033881in}{1.882308in}}%
\pgfpathlineto{\pgfqpoint{4.020447in}{1.878723in}}%
\pgfpathlineto{\pgfqpoint{4.012664in}{1.866993in}}%
\pgfpathlineto{\pgfqpoint{4.004876in}{1.855237in}}%
\pgfpathlineto{\pgfqpoint{3.997083in}{1.843456in}}%
\pgfpathlineto{\pgfqpoint{3.989285in}{1.831653in}}%
\pgfpathclose%
\pgfusepath{fill}%
\end{pgfscope}%
\begin{pgfscope}%
\pgfpathrectangle{\pgfqpoint{1.254980in}{0.150000in}}{\pgfqpoint{5.490039in}{5.490039in}}%
\pgfusepath{clip}%
\pgfsetbuttcap%
\pgfsetroundjoin%
\definecolor{currentfill}{rgb}{0.210503,0.363727,0.552206}%
\pgfsetfillcolor{currentfill}%
\pgfsetfillopacity{0.700000}%
\pgfsetlinewidth{0.000000pt}%
\definecolor{currentstroke}{rgb}{0.000000,0.000000,0.000000}%
\pgfsetstrokecolor{currentstroke}%
\pgfsetdash{}{0pt}%
\pgfpathmoveto{\pgfqpoint{4.476246in}{2.270995in}}%
\pgfpathlineto{\pgfqpoint{4.489902in}{2.278884in}}%
\pgfpathlineto{\pgfqpoint{4.503573in}{2.286933in}}%
\pgfpathlineto{\pgfqpoint{4.517258in}{2.295144in}}%
\pgfpathlineto{\pgfqpoint{4.530956in}{2.303516in}}%
\pgfpathlineto{\pgfqpoint{4.538599in}{2.314536in}}%
\pgfpathlineto{\pgfqpoint{4.546235in}{2.325449in}}%
\pgfpathlineto{\pgfqpoint{4.553866in}{2.336256in}}%
\pgfpathlineto{\pgfqpoint{4.561491in}{2.346955in}}%
\pgfpathlineto{\pgfqpoint{4.547794in}{2.338506in}}%
\pgfpathlineto{\pgfqpoint{4.534112in}{2.330218in}}%
\pgfpathlineto{\pgfqpoint{4.520444in}{2.322092in}}%
\pgfpathlineto{\pgfqpoint{4.506790in}{2.314127in}}%
\pgfpathlineto{\pgfqpoint{4.499162in}{2.303493in}}%
\pgfpathlineto{\pgfqpoint{4.491529in}{2.292760in}}%
\pgfpathlineto{\pgfqpoint{4.483890in}{2.281927in}}%
\pgfpathlineto{\pgfqpoint{4.476246in}{2.270995in}}%
\pgfpathclose%
\pgfusepath{fill}%
\end{pgfscope}%
\begin{pgfscope}%
\pgfpathrectangle{\pgfqpoint{1.254980in}{0.150000in}}{\pgfqpoint{5.490039in}{5.490039in}}%
\pgfusepath{clip}%
\pgfsetbuttcap%
\pgfsetroundjoin%
\definecolor{currentfill}{rgb}{0.246811,0.283237,0.535941}%
\pgfsetfillcolor{currentfill}%
\pgfsetfillopacity{0.700000}%
\pgfsetlinewidth{0.000000pt}%
\definecolor{currentstroke}{rgb}{0.000000,0.000000,0.000000}%
\pgfsetstrokecolor{currentstroke}%
\pgfsetdash{}{0pt}%
\pgfpathmoveto{\pgfqpoint{4.275269in}{2.078259in}}%
\pgfpathlineto{\pgfqpoint{4.288828in}{2.084452in}}%
\pgfpathlineto{\pgfqpoint{4.302398in}{2.090807in}}%
\pgfpathlineto{\pgfqpoint{4.315982in}{2.097322in}}%
\pgfpathlineto{\pgfqpoint{4.329578in}{2.104000in}}%
\pgfpathlineto{\pgfqpoint{4.337288in}{2.115889in}}%
\pgfpathlineto{\pgfqpoint{4.344992in}{2.127697in}}%
\pgfpathlineto{\pgfqpoint{4.352692in}{2.139422in}}%
\pgfpathlineto{\pgfqpoint{4.360386in}{2.151063in}}%
\pgfpathlineto{\pgfqpoint{4.346792in}{2.144222in}}%
\pgfpathlineto{\pgfqpoint{4.333211in}{2.137543in}}%
\pgfpathlineto{\pgfqpoint{4.319643in}{2.131026in}}%
\pgfpathlineto{\pgfqpoint{4.306086in}{2.124670in}}%
\pgfpathlineto{\pgfqpoint{4.298390in}{2.113181in}}%
\pgfpathlineto{\pgfqpoint{4.290688in}{2.101615in}}%
\pgfpathlineto{\pgfqpoint{4.282981in}{2.089974in}}%
\pgfpathlineto{\pgfqpoint{4.275269in}{2.078259in}}%
\pgfpathclose%
\pgfusepath{fill}%
\end{pgfscope}%
\begin{pgfscope}%
\pgfpathrectangle{\pgfqpoint{1.254980in}{0.150000in}}{\pgfqpoint{5.490039in}{5.490039in}}%
\pgfusepath{clip}%
\pgfsetbuttcap%
\pgfsetroundjoin%
\definecolor{currentfill}{rgb}{0.267004,0.004874,0.329415}%
\pgfsetfillcolor{currentfill}%
\pgfsetfillopacity{0.700000}%
\pgfsetlinewidth{0.000000pt}%
\definecolor{currentstroke}{rgb}{0.000000,0.000000,0.000000}%
\pgfsetstrokecolor{currentstroke}%
\pgfsetdash{}{0pt}%
\pgfpathmoveto{\pgfqpoint{3.255124in}{1.567809in}}%
\pgfpathlineto{\pgfqpoint{3.268446in}{1.561824in}}%
\pgfpathlineto{\pgfqpoint{3.281769in}{1.556019in}}%
\pgfpathlineto{\pgfqpoint{3.295096in}{1.550392in}}%
\pgfpathlineto{\pgfqpoint{3.308424in}{1.544942in}}%
\pgfpathlineto{\pgfqpoint{3.316518in}{1.551661in}}%
\pgfpathlineto{\pgfqpoint{3.324602in}{1.558531in}}%
\pgfpathlineto{\pgfqpoint{3.332678in}{1.565548in}}%
\pgfpathlineto{\pgfqpoint{3.340745in}{1.572706in}}%
\pgfpathlineto{\pgfqpoint{3.327438in}{1.577659in}}%
\pgfpathlineto{\pgfqpoint{3.314135in}{1.582790in}}%
\pgfpathlineto{\pgfqpoint{3.300834in}{1.588099in}}%
\pgfpathlineto{\pgfqpoint{3.287535in}{1.593587in}}%
\pgfpathlineto{\pgfqpoint{3.279446in}{1.586914in}}%
\pgfpathlineto{\pgfqpoint{3.271348in}{1.580390in}}%
\pgfpathlineto{\pgfqpoint{3.263241in}{1.574020in}}%
\pgfpathlineto{\pgfqpoint{3.255124in}{1.567809in}}%
\pgfpathclose%
\pgfusepath{fill}%
\end{pgfscope}%
\begin{pgfscope}%
\pgfpathrectangle{\pgfqpoint{1.254980in}{0.150000in}}{\pgfqpoint{5.490039in}{5.490039in}}%
\pgfusepath{clip}%
\pgfsetbuttcap%
\pgfsetroundjoin%
\definecolor{currentfill}{rgb}{0.203063,0.379716,0.553925}%
\pgfsetfillcolor{currentfill}%
\pgfsetfillopacity{0.700000}%
\pgfsetlinewidth{0.000000pt}%
\definecolor{currentstroke}{rgb}{0.000000,0.000000,0.000000}%
\pgfsetstrokecolor{currentstroke}%
\pgfsetdash{}{0pt}%
\pgfpathmoveto{\pgfqpoint{2.328672in}{2.399343in}}%
\pgfpathlineto{\pgfqpoint{2.342348in}{2.378281in}}%
\pgfpathlineto{\pgfqpoint{2.356012in}{2.357494in}}%
\pgfpathlineto{\pgfqpoint{2.369665in}{2.336979in}}%
\pgfpathlineto{\pgfqpoint{2.383306in}{2.316734in}}%
\pgfpathlineto{\pgfqpoint{2.392116in}{2.312778in}}%
\pgfpathlineto{\pgfqpoint{2.400905in}{2.309139in}}%
\pgfpathlineto{\pgfqpoint{2.409673in}{2.305812in}}%
\pgfpathlineto{\pgfqpoint{2.418421in}{2.302790in}}%
\pgfpathlineto{\pgfqpoint{2.404834in}{2.322440in}}%
\pgfpathlineto{\pgfqpoint{2.391236in}{2.342359in}}%
\pgfpathlineto{\pgfqpoint{2.377627in}{2.362549in}}%
\pgfpathlineto{\pgfqpoint{2.364006in}{2.383012in}}%
\pgfpathlineto{\pgfqpoint{2.355204in}{2.386618in}}%
\pgfpathlineto{\pgfqpoint{2.346382in}{2.390537in}}%
\pgfpathlineto{\pgfqpoint{2.337538in}{2.394777in}}%
\pgfpathlineto{\pgfqpoint{2.328672in}{2.399343in}}%
\pgfpathclose%
\pgfusepath{fill}%
\end{pgfscope}%
\begin{pgfscope}%
\pgfpathrectangle{\pgfqpoint{1.254980in}{0.150000in}}{\pgfqpoint{5.490039in}{5.490039in}}%
\pgfusepath{clip}%
\pgfsetbuttcap%
\pgfsetroundjoin%
\definecolor{currentfill}{rgb}{0.259857,0.745492,0.444467}%
\pgfsetfillcolor{currentfill}%
\pgfsetfillopacity{0.700000}%
\pgfsetlinewidth{0.000000pt}%
\definecolor{currentstroke}{rgb}{0.000000,0.000000,0.000000}%
\pgfsetstrokecolor{currentstroke}%
\pgfsetdash{}{0pt}%
\pgfpathmoveto{\pgfqpoint{5.705536in}{3.310293in}}%
\pgfpathlineto{\pgfqpoint{5.719926in}{3.323725in}}%
\pgfpathlineto{\pgfqpoint{5.734338in}{3.337317in}}%
\pgfpathlineto{\pgfqpoint{5.748770in}{3.351068in}}%
\pgfpathlineto{\pgfqpoint{5.763223in}{3.364980in}}%
\pgfpathlineto{\pgfqpoint{5.770172in}{3.366178in}}%
\pgfpathlineto{\pgfqpoint{5.777112in}{3.367322in}}%
\pgfpathlineto{\pgfqpoint{5.784044in}{3.368416in}}%
\pgfpathlineto{\pgfqpoint{5.790968in}{3.369465in}}%
\pgfpathlineto{\pgfqpoint{5.776541in}{3.355999in}}%
\pgfpathlineto{\pgfqpoint{5.762136in}{3.342692in}}%
\pgfpathlineto{\pgfqpoint{5.747751in}{3.329544in}}%
\pgfpathlineto{\pgfqpoint{5.733387in}{3.316554in}}%
\pgfpathlineto{\pgfqpoint{5.726436in}{3.315051in}}%
\pgfpathlineto{\pgfqpoint{5.719478in}{3.313509in}}%
\pgfpathlineto{\pgfqpoint{5.712511in}{3.311925in}}%
\pgfpathlineto{\pgfqpoint{5.705536in}{3.310293in}}%
\pgfpathclose%
\pgfusepath{fill}%
\end{pgfscope}%
\begin{pgfscope}%
\pgfpathrectangle{\pgfqpoint{1.254980in}{0.150000in}}{\pgfqpoint{5.490039in}{5.490039in}}%
\pgfusepath{clip}%
\pgfsetbuttcap%
\pgfsetroundjoin%
\definecolor{currentfill}{rgb}{0.120638,0.625828,0.533488}%
\pgfsetfillcolor{currentfill}%
\pgfsetfillopacity{0.700000}%
\pgfsetlinewidth{0.000000pt}%
\definecolor{currentstroke}{rgb}{0.000000,0.000000,0.000000}%
\pgfsetstrokecolor{currentstroke}%
\pgfsetdash{}{0pt}%
\pgfpathmoveto{\pgfqpoint{5.249424in}{2.970197in}}%
\pgfpathlineto{\pgfqpoint{5.263535in}{2.982433in}}%
\pgfpathlineto{\pgfqpoint{5.277664in}{2.994830in}}%
\pgfpathlineto{\pgfqpoint{5.291812in}{3.007388in}}%
\pgfpathlineto{\pgfqpoint{5.305979in}{3.020106in}}%
\pgfpathlineto{\pgfqpoint{5.313242in}{3.025139in}}%
\pgfpathlineto{\pgfqpoint{5.320497in}{3.030060in}}%
\pgfpathlineto{\pgfqpoint{5.327743in}{3.034873in}}%
\pgfpathlineto{\pgfqpoint{5.334981in}{3.039580in}}%
\pgfpathlineto{\pgfqpoint{5.320828in}{3.027119in}}%
\pgfpathlineto{\pgfqpoint{5.306694in}{3.014818in}}%
\pgfpathlineto{\pgfqpoint{5.292579in}{3.002677in}}%
\pgfpathlineto{\pgfqpoint{5.278483in}{2.990696in}}%
\pgfpathlineto{\pgfqpoint{5.271231in}{2.985721in}}%
\pgfpathlineto{\pgfqpoint{5.263970in}{2.980649in}}%
\pgfpathlineto{\pgfqpoint{5.256702in}{2.975475in}}%
\pgfpathlineto{\pgfqpoint{5.249424in}{2.970197in}}%
\pgfpathclose%
\pgfusepath{fill}%
\end{pgfscope}%
\begin{pgfscope}%
\pgfpathrectangle{\pgfqpoint{1.254980in}{0.150000in}}{\pgfqpoint{5.490039in}{5.490039in}}%
\pgfusepath{clip}%
\pgfsetbuttcap%
\pgfsetroundjoin%
\definecolor{currentfill}{rgb}{0.282327,0.094955,0.417331}%
\pgfsetfillcolor{currentfill}%
\pgfsetfillopacity{0.700000}%
\pgfsetlinewidth{0.000000pt}%
\definecolor{currentstroke}{rgb}{0.000000,0.000000,0.000000}%
\pgfsetstrokecolor{currentstroke}%
\pgfsetdash{}{0pt}%
\pgfpathmoveto{\pgfqpoint{2.868877in}{1.750087in}}%
\pgfpathlineto{\pgfqpoint{2.882266in}{1.738443in}}%
\pgfpathlineto{\pgfqpoint{2.895652in}{1.727002in}}%
\pgfpathlineto{\pgfqpoint{2.909036in}{1.715763in}}%
\pgfpathlineto{\pgfqpoint{2.922418in}{1.704725in}}%
\pgfpathlineto{\pgfqpoint{2.930776in}{1.706640in}}%
\pgfpathlineto{\pgfqpoint{2.939120in}{1.708793in}}%
\pgfpathlineto{\pgfqpoint{2.947451in}{1.711178in}}%
\pgfpathlineto{\pgfqpoint{2.955769in}{1.713789in}}%
\pgfpathlineto{\pgfqpoint{2.942422in}{1.724266in}}%
\pgfpathlineto{\pgfqpoint{2.929074in}{1.734943in}}%
\pgfpathlineto{\pgfqpoint{2.915724in}{1.745821in}}%
\pgfpathlineto{\pgfqpoint{2.902372in}{1.756902in}}%
\pgfpathlineto{\pgfqpoint{2.894019in}{1.754841in}}%
\pgfpathlineto{\pgfqpoint{2.885653in}{1.753015in}}%
\pgfpathlineto{\pgfqpoint{2.877272in}{1.751428in}}%
\pgfpathlineto{\pgfqpoint{2.868877in}{1.750087in}}%
\pgfpathclose%
\pgfusepath{fill}%
\end{pgfscope}%
\begin{pgfscope}%
\pgfpathrectangle{\pgfqpoint{1.254980in}{0.150000in}}{\pgfqpoint{5.490039in}{5.490039in}}%
\pgfusepath{clip}%
\pgfsetbuttcap%
\pgfsetroundjoin%
\definecolor{currentfill}{rgb}{0.137770,0.537492,0.554906}%
\pgfsetfillcolor{currentfill}%
\pgfsetfillopacity{0.700000}%
\pgfsetlinewidth{0.000000pt}%
\definecolor{currentstroke}{rgb}{0.000000,0.000000,0.000000}%
\pgfsetstrokecolor{currentstroke}%
\pgfsetdash{}{0pt}%
\pgfpathmoveto{\pgfqpoint{4.963417in}{2.726546in}}%
\pgfpathlineto{\pgfqpoint{4.977354in}{2.737570in}}%
\pgfpathlineto{\pgfqpoint{4.991308in}{2.748755in}}%
\pgfpathlineto{\pgfqpoint{5.005279in}{2.760101in}}%
\pgfpathlineto{\pgfqpoint{5.019268in}{2.771608in}}%
\pgfpathlineto{\pgfqpoint{5.026698in}{2.779139in}}%
\pgfpathlineto{\pgfqpoint{5.034120in}{2.786544in}}%
\pgfpathlineto{\pgfqpoint{5.041534in}{2.793825in}}%
\pgfpathlineto{\pgfqpoint{5.048940in}{2.800983in}}%
\pgfpathlineto{\pgfqpoint{5.034960in}{2.789610in}}%
\pgfpathlineto{\pgfqpoint{5.020997in}{2.778397in}}%
\pgfpathlineto{\pgfqpoint{5.007051in}{2.767345in}}%
\pgfpathlineto{\pgfqpoint{4.993122in}{2.756454in}}%
\pgfpathlineto{\pgfqpoint{4.985707in}{2.749152in}}%
\pgfpathlineto{\pgfqpoint{4.978285in}{2.741734in}}%
\pgfpathlineto{\pgfqpoint{4.970855in}{2.734199in}}%
\pgfpathlineto{\pgfqpoint{4.963417in}{2.726546in}}%
\pgfpathclose%
\pgfusepath{fill}%
\end{pgfscope}%
\begin{pgfscope}%
\pgfpathrectangle{\pgfqpoint{1.254980in}{0.150000in}}{\pgfqpoint{5.490039in}{5.490039in}}%
\pgfusepath{clip}%
\pgfsetbuttcap%
\pgfsetroundjoin%
\definecolor{currentfill}{rgb}{0.275191,0.194905,0.496005}%
\pgfsetfillcolor{currentfill}%
\pgfsetfillopacity{0.700000}%
\pgfsetlinewidth{0.000000pt}%
\definecolor{currentstroke}{rgb}{0.000000,0.000000,0.000000}%
\pgfsetstrokecolor{currentstroke}%
\pgfsetdash{}{0pt}%
\pgfpathmoveto{\pgfqpoint{4.074244in}{1.894043in}}%
\pgfpathlineto{\pgfqpoint{4.087720in}{1.898280in}}%
\pgfpathlineto{\pgfqpoint{4.101206in}{1.902680in}}%
\pgfpathlineto{\pgfqpoint{4.114703in}{1.907241in}}%
\pgfpathlineto{\pgfqpoint{4.128210in}{1.911965in}}%
\pgfpathlineto{\pgfqpoint{4.135982in}{1.924181in}}%
\pgfpathlineto{\pgfqpoint{4.143748in}{1.936352in}}%
\pgfpathlineto{\pgfqpoint{4.151510in}{1.948475in}}%
\pgfpathlineto{\pgfqpoint{4.159267in}{1.960548in}}%
\pgfpathlineto{\pgfqpoint{4.145763in}{1.955577in}}%
\pgfpathlineto{\pgfqpoint{4.132269in}{1.950767in}}%
\pgfpathlineto{\pgfqpoint{4.118787in}{1.946120in}}%
\pgfpathlineto{\pgfqpoint{4.105315in}{1.941635in}}%
\pgfpathlineto{\pgfqpoint{4.097554in}{1.929799in}}%
\pgfpathlineto{\pgfqpoint{4.089789in}{1.917921in}}%
\pgfpathlineto{\pgfqpoint{4.082019in}{1.906001in}}%
\pgfpathlineto{\pgfqpoint{4.074244in}{1.894043in}}%
\pgfpathclose%
\pgfusepath{fill}%
\end{pgfscope}%
\begin{pgfscope}%
\pgfpathrectangle{\pgfqpoint{1.254980in}{0.150000in}}{\pgfqpoint{5.490039in}{5.490039in}}%
\pgfusepath{clip}%
\pgfsetbuttcap%
\pgfsetroundjoin%
\definecolor{currentfill}{rgb}{0.267004,0.004874,0.329415}%
\pgfsetfillcolor{currentfill}%
\pgfsetfillopacity{0.700000}%
\pgfsetlinewidth{0.000000pt}%
\definecolor{currentstroke}{rgb}{0.000000,0.000000,0.000000}%
\pgfsetstrokecolor{currentstroke}%
\pgfsetdash{}{0pt}%
\pgfpathmoveto{\pgfqpoint{3.394004in}{1.554656in}}%
\pgfpathlineto{\pgfqpoint{3.407327in}{1.550581in}}%
\pgfpathlineto{\pgfqpoint{3.420654in}{1.546679in}}%
\pgfpathlineto{\pgfqpoint{3.433986in}{1.542951in}}%
\pgfpathlineto{\pgfqpoint{3.447321in}{1.539395in}}%
\pgfpathlineto{\pgfqpoint{3.455341in}{1.547647in}}%
\pgfpathlineto{\pgfqpoint{3.463354in}{1.556017in}}%
\pgfpathlineto{\pgfqpoint{3.471359in}{1.564501in}}%
\pgfpathlineto{\pgfqpoint{3.479356in}{1.573095in}}%
\pgfpathlineto{\pgfqpoint{3.466039in}{1.576183in}}%
\pgfpathlineto{\pgfqpoint{3.452725in}{1.579444in}}%
\pgfpathlineto{\pgfqpoint{3.439416in}{1.582878in}}%
\pgfpathlineto{\pgfqpoint{3.426111in}{1.586485in}}%
\pgfpathlineto{\pgfqpoint{3.418096in}{1.578349in}}%
\pgfpathlineto{\pgfqpoint{3.410073in}{1.570329in}}%
\pgfpathlineto{\pgfqpoint{3.402042in}{1.562430in}}%
\pgfpathlineto{\pgfqpoint{3.394004in}{1.554656in}}%
\pgfpathclose%
\pgfusepath{fill}%
\end{pgfscope}%
\begin{pgfscope}%
\pgfpathrectangle{\pgfqpoint{1.254980in}{0.150000in}}{\pgfqpoint{5.490039in}{5.490039in}}%
\pgfusepath{clip}%
\pgfsetbuttcap%
\pgfsetroundjoin%
\definecolor{currentfill}{rgb}{0.344074,0.780029,0.397381}%
\pgfsetfillcolor{currentfill}%
\pgfsetfillopacity{0.700000}%
\pgfsetlinewidth{0.000000pt}%
\definecolor{currentstroke}{rgb}{0.000000,0.000000,0.000000}%
\pgfsetstrokecolor{currentstroke}%
\pgfsetdash{}{0pt}%
\pgfpathmoveto{\pgfqpoint{5.876382in}{3.426874in}}%
\pgfpathlineto{\pgfqpoint{5.890886in}{3.440661in}}%
\pgfpathlineto{\pgfqpoint{5.905410in}{3.454609in}}%
\pgfpathlineto{\pgfqpoint{5.919957in}{3.468715in}}%
\pgfpathlineto{\pgfqpoint{5.926788in}{3.468751in}}%
\pgfpathlineto{\pgfqpoint{5.933611in}{3.468761in}}%
\pgfpathlineto{\pgfqpoint{5.940426in}{3.468750in}}%
\pgfpathlineto{\pgfqpoint{5.947233in}{3.468724in}}%
\pgfpathlineto{\pgfqpoint{5.932719in}{3.455124in}}%
\pgfpathlineto{\pgfqpoint{5.918225in}{3.441684in}}%
\pgfpathlineto{\pgfqpoint{5.903753in}{3.428402in}}%
\pgfpathlineto{\pgfqpoint{5.896922in}{3.428040in}}%
\pgfpathlineto{\pgfqpoint{5.890083in}{3.427669in}}%
\pgfpathlineto{\pgfqpoint{5.883236in}{3.427282in}}%
\pgfpathlineto{\pgfqpoint{5.876382in}{3.426874in}}%
\pgfpathclose%
\pgfusepath{fill}%
\end{pgfscope}%
\begin{pgfscope}%
\pgfpathrectangle{\pgfqpoint{1.254980in}{0.150000in}}{\pgfqpoint{5.490039in}{5.490039in}}%
\pgfusepath{clip}%
\pgfsetbuttcap%
\pgfsetroundjoin%
\definecolor{currentfill}{rgb}{0.271305,0.019942,0.347269}%
\pgfsetfillcolor{currentfill}%
\pgfsetfillopacity{0.700000}%
\pgfsetlinewidth{0.000000pt}%
\definecolor{currentstroke}{rgb}{0.000000,0.000000,0.000000}%
\pgfsetstrokecolor{currentstroke}%
\pgfsetdash{}{0pt}%
\pgfpathmoveto{\pgfqpoint{3.115837in}{1.603276in}}%
\pgfpathlineto{\pgfqpoint{3.129175in}{1.595304in}}%
\pgfpathlineto{\pgfqpoint{3.142513in}{1.587517in}}%
\pgfpathlineto{\pgfqpoint{3.155851in}{1.579914in}}%
\pgfpathlineto{\pgfqpoint{3.169191in}{1.572496in}}%
\pgfpathlineto{\pgfqpoint{3.177373in}{1.577487in}}%
\pgfpathlineto{\pgfqpoint{3.185545in}{1.582663in}}%
\pgfpathlineto{\pgfqpoint{3.193706in}{1.588020in}}%
\pgfpathlineto{\pgfqpoint{3.201858in}{1.593553in}}%
\pgfpathlineto{\pgfqpoint{3.188545in}{1.600445in}}%
\pgfpathlineto{\pgfqpoint{3.175234in}{1.607521in}}%
\pgfpathlineto{\pgfqpoint{3.161923in}{1.614781in}}%
\pgfpathlineto{\pgfqpoint{3.148614in}{1.622227in}}%
\pgfpathlineto{\pgfqpoint{3.140436in}{1.617209in}}%
\pgfpathlineto{\pgfqpoint{3.132247in}{1.612375in}}%
\pgfpathlineto{\pgfqpoint{3.124048in}{1.607729in}}%
\pgfpathlineto{\pgfqpoint{3.115837in}{1.603276in}}%
\pgfpathclose%
\pgfusepath{fill}%
\end{pgfscope}%
\begin{pgfscope}%
\pgfpathrectangle{\pgfqpoint{1.254980in}{0.150000in}}{\pgfqpoint{5.490039in}{5.490039in}}%
\pgfusepath{clip}%
\pgfsetbuttcap%
\pgfsetroundjoin%
\definecolor{currentfill}{rgb}{0.304148,0.764704,0.419943}%
\pgfsetfillcolor{currentfill}%
\pgfsetfillopacity{0.700000}%
\pgfsetlinewidth{0.000000pt}%
\definecolor{currentstroke}{rgb}{0.000000,0.000000,0.000000}%
\pgfsetstrokecolor{currentstroke}%
\pgfsetdash{}{0pt}%
\pgfpathmoveto{\pgfqpoint{5.790968in}{3.369465in}}%
\pgfpathlineto{\pgfqpoint{5.805415in}{3.383091in}}%
\pgfpathlineto{\pgfqpoint{5.819884in}{3.396877in}}%
\pgfpathlineto{\pgfqpoint{5.834374in}{3.410822in}}%
\pgfpathlineto{\pgfqpoint{5.848885in}{3.424928in}}%
\pgfpathlineto{\pgfqpoint{5.855772in}{3.425472in}}%
\pgfpathlineto{\pgfqpoint{5.862650in}{3.425974in}}%
\pgfpathlineto{\pgfqpoint{5.869520in}{3.426440in}}%
\pgfpathlineto{\pgfqpoint{5.876382in}{3.426874in}}%
\pgfpathlineto{\pgfqpoint{5.861900in}{3.413245in}}%
\pgfpathlineto{\pgfqpoint{5.847440in}{3.399776in}}%
\pgfpathlineto{\pgfqpoint{5.833000in}{3.386466in}}%
\pgfpathlineto{\pgfqpoint{5.818581in}{3.373314in}}%
\pgfpathlineto{\pgfqpoint{5.811690in}{3.372394in}}%
\pgfpathlineto{\pgfqpoint{5.804790in}{3.371449in}}%
\pgfpathlineto{\pgfqpoint{5.797883in}{3.370475in}}%
\pgfpathlineto{\pgfqpoint{5.790968in}{3.369465in}}%
\pgfpathclose%
\pgfusepath{fill}%
\end{pgfscope}%
\begin{pgfscope}%
\pgfpathrectangle{\pgfqpoint{1.254980in}{0.150000in}}{\pgfqpoint{5.490039in}{5.490039in}}%
\pgfusepath{clip}%
\pgfsetbuttcap%
\pgfsetroundjoin%
\definecolor{currentfill}{rgb}{0.163625,0.471133,0.558148}%
\pgfsetfillcolor{currentfill}%
\pgfsetfillopacity{0.700000}%
\pgfsetlinewidth{0.000000pt}%
\definecolor{currentstroke}{rgb}{0.000000,0.000000,0.000000}%
\pgfsetstrokecolor{currentstroke}%
\pgfsetdash{}{0pt}%
\pgfpathmoveto{\pgfqpoint{4.762543in}{2.540781in}}%
\pgfpathlineto{\pgfqpoint{4.776366in}{2.550733in}}%
\pgfpathlineto{\pgfqpoint{4.790205in}{2.560847in}}%
\pgfpathlineto{\pgfqpoint{4.804059in}{2.571121in}}%
\pgfpathlineto{\pgfqpoint{4.817930in}{2.581557in}}%
\pgfpathlineto{\pgfqpoint{4.825462in}{2.590775in}}%
\pgfpathlineto{\pgfqpoint{4.832986in}{2.599866in}}%
\pgfpathlineto{\pgfqpoint{4.840504in}{2.608832in}}%
\pgfpathlineto{\pgfqpoint{4.848014in}{2.617674in}}%
\pgfpathlineto{\pgfqpoint{4.834148in}{2.607280in}}%
\pgfpathlineto{\pgfqpoint{4.820298in}{2.597048in}}%
\pgfpathlineto{\pgfqpoint{4.806465in}{2.586976in}}%
\pgfpathlineto{\pgfqpoint{4.792647in}{2.577065in}}%
\pgfpathlineto{\pgfqpoint{4.785131in}{2.568171in}}%
\pgfpathlineto{\pgfqpoint{4.777609in}{2.559159in}}%
\pgfpathlineto{\pgfqpoint{4.770080in}{2.550029in}}%
\pgfpathlineto{\pgfqpoint{4.762543in}{2.540781in}}%
\pgfpathclose%
\pgfusepath{fill}%
\end{pgfscope}%
\begin{pgfscope}%
\pgfpathrectangle{\pgfqpoint{1.254980in}{0.150000in}}{\pgfqpoint{5.490039in}{5.490039in}}%
\pgfusepath{clip}%
\pgfsetbuttcap%
\pgfsetroundjoin%
\definecolor{currentfill}{rgb}{0.185556,0.418570,0.556753}%
\pgfsetfillcolor{currentfill}%
\pgfsetfillopacity{0.700000}%
\pgfsetlinewidth{0.000000pt}%
\definecolor{currentstroke}{rgb}{0.000000,0.000000,0.000000}%
\pgfsetstrokecolor{currentstroke}%
\pgfsetdash{}{0pt}%
\pgfpathmoveto{\pgfqpoint{2.273839in}{2.486388in}}%
\pgfpathlineto{\pgfqpoint{2.287567in}{2.464202in}}%
\pgfpathlineto{\pgfqpoint{2.301281in}{2.442301in}}%
\pgfpathlineto{\pgfqpoint{2.314983in}{2.420682in}}%
\pgfpathlineto{\pgfqpoint{2.328672in}{2.399343in}}%
\pgfpathlineto{\pgfqpoint{2.337538in}{2.394777in}}%
\pgfpathlineto{\pgfqpoint{2.346382in}{2.390537in}}%
\pgfpathlineto{\pgfqpoint{2.355204in}{2.386618in}}%
\pgfpathlineto{\pgfqpoint{2.364006in}{2.383012in}}%
\pgfpathlineto{\pgfqpoint{2.350373in}{2.403751in}}%
\pgfpathlineto{\pgfqpoint{2.336728in}{2.424768in}}%
\pgfpathlineto{\pgfqpoint{2.323070in}{2.446066in}}%
\pgfpathlineto{\pgfqpoint{2.309400in}{2.467648in}}%
\pgfpathlineto{\pgfqpoint{2.300543in}{2.471843in}}%
\pgfpathlineto{\pgfqpoint{2.291664in}{2.476361in}}%
\pgfpathlineto{\pgfqpoint{2.282763in}{2.481207in}}%
\pgfpathlineto{\pgfqpoint{2.273839in}{2.486388in}}%
\pgfpathclose%
\pgfusepath{fill}%
\end{pgfscope}%
\begin{pgfscope}%
\pgfpathrectangle{\pgfqpoint{1.254980in}{0.150000in}}{\pgfqpoint{5.490039in}{5.490039in}}%
\pgfusepath{clip}%
\pgfsetbuttcap%
\pgfsetroundjoin%
\definecolor{currentfill}{rgb}{0.280894,0.078907,0.402329}%
\pgfsetfillcolor{currentfill}%
\pgfsetfillopacity{0.700000}%
\pgfsetlinewidth{0.000000pt}%
\definecolor{currentstroke}{rgb}{0.000000,0.000000,0.000000}%
\pgfsetstrokecolor{currentstroke}%
\pgfsetdash{}{0pt}%
\pgfpathmoveto{\pgfqpoint{2.922418in}{1.704725in}}%
\pgfpathlineto{\pgfqpoint{2.935797in}{1.693886in}}%
\pgfpathlineto{\pgfqpoint{2.949174in}{1.683246in}}%
\pgfpathlineto{\pgfqpoint{2.962550in}{1.672802in}}%
\pgfpathlineto{\pgfqpoint{2.975924in}{1.662555in}}%
\pgfpathlineto{\pgfqpoint{2.984247in}{1.665042in}}%
\pgfpathlineto{\pgfqpoint{2.992556in}{1.667758in}}%
\pgfpathlineto{\pgfqpoint{3.000853in}{1.670700in}}%
\pgfpathlineto{\pgfqpoint{3.009137in}{1.673860in}}%
\pgfpathlineto{\pgfqpoint{2.995797in}{1.683548in}}%
\pgfpathlineto{\pgfqpoint{2.982456in}{1.693431in}}%
\pgfpathlineto{\pgfqpoint{2.969113in}{1.703511in}}%
\pgfpathlineto{\pgfqpoint{2.955769in}{1.713789in}}%
\pgfpathlineto{\pgfqpoint{2.947451in}{1.711178in}}%
\pgfpathlineto{\pgfqpoint{2.939120in}{1.708793in}}%
\pgfpathlineto{\pgfqpoint{2.930776in}{1.706640in}}%
\pgfpathlineto{\pgfqpoint{2.922418in}{1.704725in}}%
\pgfpathclose%
\pgfusepath{fill}%
\end{pgfscope}%
\begin{pgfscope}%
\pgfpathrectangle{\pgfqpoint{1.254980in}{0.150000in}}{\pgfqpoint{5.490039in}{5.490039in}}%
\pgfusepath{clip}%
\pgfsetbuttcap%
\pgfsetroundjoin%
\definecolor{currentfill}{rgb}{0.194100,0.399323,0.555565}%
\pgfsetfillcolor{currentfill}%
\pgfsetfillopacity{0.700000}%
\pgfsetlinewidth{0.000000pt}%
\definecolor{currentstroke}{rgb}{0.000000,0.000000,0.000000}%
\pgfsetstrokecolor{currentstroke}%
\pgfsetdash{}{0pt}%
\pgfpathmoveto{\pgfqpoint{4.561491in}{2.346955in}}%
\pgfpathlineto{\pgfqpoint{4.575202in}{2.355565in}}%
\pgfpathlineto{\pgfqpoint{4.588927in}{2.364336in}}%
\pgfpathlineto{\pgfqpoint{4.602668in}{2.373269in}}%
\pgfpathlineto{\pgfqpoint{4.616423in}{2.382362in}}%
\pgfpathlineto{\pgfqpoint{4.624039in}{2.393012in}}%
\pgfpathlineto{\pgfqpoint{4.631650in}{2.403547in}}%
\pgfpathlineto{\pgfqpoint{4.639254in}{2.413967in}}%
\pgfpathlineto{\pgfqpoint{4.646852in}{2.424271in}}%
\pgfpathlineto{\pgfqpoint{4.633100in}{2.415130in}}%
\pgfpathlineto{\pgfqpoint{4.619362in}{2.406150in}}%
\pgfpathlineto{\pgfqpoint{4.605639in}{2.397332in}}%
\pgfpathlineto{\pgfqpoint{4.591931in}{2.388674in}}%
\pgfpathlineto{\pgfqpoint{4.584330in}{2.378406in}}%
\pgfpathlineto{\pgfqpoint{4.576723in}{2.368030in}}%
\pgfpathlineto{\pgfqpoint{4.569110in}{2.357546in}}%
\pgfpathlineto{\pgfqpoint{4.561491in}{2.346955in}}%
\pgfpathclose%
\pgfusepath{fill}%
\end{pgfscope}%
\begin{pgfscope}%
\pgfpathrectangle{\pgfqpoint{1.254980in}{0.150000in}}{\pgfqpoint{5.490039in}{5.490039in}}%
\pgfusepath{clip}%
\pgfsetbuttcap%
\pgfsetroundjoin%
\definecolor{currentfill}{rgb}{0.132268,0.655014,0.519661}%
\pgfsetfillcolor{currentfill}%
\pgfsetfillopacity{0.700000}%
\pgfsetlinewidth{0.000000pt}%
\definecolor{currentstroke}{rgb}{0.000000,0.000000,0.000000}%
\pgfsetstrokecolor{currentstroke}%
\pgfsetdash{}{0pt}%
\pgfpathmoveto{\pgfqpoint{5.334981in}{3.039580in}}%
\pgfpathlineto{\pgfqpoint{5.349153in}{3.052202in}}%
\pgfpathlineto{\pgfqpoint{5.363343in}{3.064984in}}%
\pgfpathlineto{\pgfqpoint{5.377554in}{3.077928in}}%
\pgfpathlineto{\pgfqpoint{5.391783in}{3.091032in}}%
\pgfpathlineto{\pgfqpoint{5.398997in}{3.095360in}}%
\pgfpathlineto{\pgfqpoint{5.406201in}{3.099582in}}%
\pgfpathlineto{\pgfqpoint{5.413396in}{3.103701in}}%
\pgfpathlineto{\pgfqpoint{5.420583in}{3.107720in}}%
\pgfpathlineto{\pgfqpoint{5.406369in}{3.094905in}}%
\pgfpathlineto{\pgfqpoint{5.392175in}{3.082250in}}%
\pgfpathlineto{\pgfqpoint{5.378000in}{3.069755in}}%
\pgfpathlineto{\pgfqpoint{5.363844in}{3.057420in}}%
\pgfpathlineto{\pgfqpoint{5.356641in}{3.053102in}}%
\pgfpathlineto{\pgfqpoint{5.349430in}{3.048692in}}%
\pgfpathlineto{\pgfqpoint{5.342210in}{3.044185in}}%
\pgfpathlineto{\pgfqpoint{5.334981in}{3.039580in}}%
\pgfpathclose%
\pgfusepath{fill}%
\end{pgfscope}%
\begin{pgfscope}%
\pgfpathrectangle{\pgfqpoint{1.254980in}{0.150000in}}{\pgfqpoint{5.490039in}{5.490039in}}%
\pgfusepath{clip}%
\pgfsetbuttcap%
\pgfsetroundjoin%
\definecolor{currentfill}{rgb}{0.278791,0.062145,0.386592}%
\pgfsetfillcolor{currentfill}%
\pgfsetfillopacity{0.700000}%
\pgfsetlinewidth{0.000000pt}%
\definecolor{currentstroke}{rgb}{0.000000,0.000000,0.000000}%
\pgfsetstrokecolor{currentstroke}%
\pgfsetdash{}{0pt}%
\pgfpathmoveto{\pgfqpoint{3.702982in}{1.631691in}}%
\pgfpathlineto{\pgfqpoint{3.716348in}{1.631651in}}%
\pgfpathlineto{\pgfqpoint{3.729722in}{1.631777in}}%
\pgfpathlineto{\pgfqpoint{3.743103in}{1.632069in}}%
\pgfpathlineto{\pgfqpoint{3.756491in}{1.632526in}}%
\pgfpathlineto{\pgfqpoint{3.764381in}{1.643495in}}%
\pgfpathlineto{\pgfqpoint{3.772266in}{1.654506in}}%
\pgfpathlineto{\pgfqpoint{3.780146in}{1.665553in}}%
\pgfpathlineto{\pgfqpoint{3.788020in}{1.676635in}}%
\pgfpathlineto{\pgfqpoint{3.774641in}{1.675793in}}%
\pgfpathlineto{\pgfqpoint{3.761269in}{1.675116in}}%
\pgfpathlineto{\pgfqpoint{3.747905in}{1.674604in}}%
\pgfpathlineto{\pgfqpoint{3.734549in}{1.674259in}}%
\pgfpathlineto{\pgfqpoint{3.726665in}{1.663552in}}%
\pgfpathlineto{\pgfqpoint{3.718776in}{1.652886in}}%
\pgfpathlineto{\pgfqpoint{3.710882in}{1.642265in}}%
\pgfpathlineto{\pgfqpoint{3.702982in}{1.631691in}}%
\pgfpathclose%
\pgfusepath{fill}%
\end{pgfscope}%
\begin{pgfscope}%
\pgfpathrectangle{\pgfqpoint{1.254980in}{0.150000in}}{\pgfqpoint{5.490039in}{5.490039in}}%
\pgfusepath{clip}%
\pgfsetbuttcap%
\pgfsetroundjoin%
\definecolor{currentfill}{rgb}{0.231674,0.318106,0.544834}%
\pgfsetfillcolor{currentfill}%
\pgfsetfillopacity{0.700000}%
\pgfsetlinewidth{0.000000pt}%
\definecolor{currentstroke}{rgb}{0.000000,0.000000,0.000000}%
\pgfsetstrokecolor{currentstroke}%
\pgfsetdash{}{0pt}%
\pgfpathmoveto{\pgfqpoint{4.360386in}{2.151063in}}%
\pgfpathlineto{\pgfqpoint{4.373993in}{2.158066in}}%
\pgfpathlineto{\pgfqpoint{4.387614in}{2.165229in}}%
\pgfpathlineto{\pgfqpoint{4.401247in}{2.172554in}}%
\pgfpathlineto{\pgfqpoint{4.414894in}{2.180040in}}%
\pgfpathlineto{\pgfqpoint{4.422582in}{2.191743in}}%
\pgfpathlineto{\pgfqpoint{4.430264in}{2.203352in}}%
\pgfpathlineto{\pgfqpoint{4.437941in}{2.214866in}}%
\pgfpathlineto{\pgfqpoint{4.445613in}{2.226286in}}%
\pgfpathlineto{\pgfqpoint{4.431968in}{2.218665in}}%
\pgfpathlineto{\pgfqpoint{4.418336in}{2.211205in}}%
\pgfpathlineto{\pgfqpoint{4.404718in}{2.203906in}}%
\pgfpathlineto{\pgfqpoint{4.391113in}{2.196769in}}%
\pgfpathlineto{\pgfqpoint{4.383439in}{2.185474in}}%
\pgfpathlineto{\pgfqpoint{4.375760in}{2.174090in}}%
\pgfpathlineto{\pgfqpoint{4.368076in}{2.162620in}}%
\pgfpathlineto{\pgfqpoint{4.360386in}{2.151063in}}%
\pgfpathclose%
\pgfusepath{fill}%
\end{pgfscope}%
\begin{pgfscope}%
\pgfpathrectangle{\pgfqpoint{1.254980in}{0.150000in}}{\pgfqpoint{5.490039in}{5.490039in}}%
\pgfusepath{clip}%
\pgfsetbuttcap%
\pgfsetroundjoin%
\definecolor{currentfill}{rgb}{0.281446,0.084320,0.407414}%
\pgfsetfillcolor{currentfill}%
\pgfsetfillopacity{0.700000}%
\pgfsetlinewidth{0.000000pt}%
\definecolor{currentstroke}{rgb}{0.000000,0.000000,0.000000}%
\pgfsetstrokecolor{currentstroke}%
\pgfsetdash{}{0pt}%
\pgfpathmoveto{\pgfqpoint{3.788020in}{1.676635in}}%
\pgfpathlineto{\pgfqpoint{3.801407in}{1.677642in}}%
\pgfpathlineto{\pgfqpoint{3.814802in}{1.678814in}}%
\pgfpathlineto{\pgfqpoint{3.828205in}{1.680151in}}%
\pgfpathlineto{\pgfqpoint{3.841617in}{1.681651in}}%
\pgfpathlineto{\pgfqpoint{3.849478in}{1.693132in}}%
\pgfpathlineto{\pgfqpoint{3.857334in}{1.704633in}}%
\pgfpathlineto{\pgfqpoint{3.865185in}{1.716150in}}%
\pgfpathlineto{\pgfqpoint{3.873031in}{1.727681in}}%
\pgfpathlineto{\pgfqpoint{3.859627in}{1.725822in}}%
\pgfpathlineto{\pgfqpoint{3.846231in}{1.724128in}}%
\pgfpathlineto{\pgfqpoint{3.832844in}{1.722598in}}%
\pgfpathlineto{\pgfqpoint{3.819465in}{1.721232in}}%
\pgfpathlineto{\pgfqpoint{3.811612in}{1.710049in}}%
\pgfpathlineto{\pgfqpoint{3.803753in}{1.698886in}}%
\pgfpathlineto{\pgfqpoint{3.795889in}{1.687747in}}%
\pgfpathlineto{\pgfqpoint{3.788020in}{1.676635in}}%
\pgfpathclose%
\pgfusepath{fill}%
\end{pgfscope}%
\begin{pgfscope}%
\pgfpathrectangle{\pgfqpoint{1.254980in}{0.150000in}}{\pgfqpoint{5.490039in}{5.490039in}}%
\pgfusepath{clip}%
\pgfsetbuttcap%
\pgfsetroundjoin%
\definecolor{currentfill}{rgb}{0.274952,0.037752,0.364543}%
\pgfsetfillcolor{currentfill}%
\pgfsetfillopacity{0.700000}%
\pgfsetlinewidth{0.000000pt}%
\definecolor{currentstroke}{rgb}{0.000000,0.000000,0.000000}%
\pgfsetstrokecolor{currentstroke}%
\pgfsetdash{}{0pt}%
\pgfpathmoveto{\pgfqpoint{3.617881in}{1.593430in}}%
\pgfpathlineto{\pgfqpoint{3.631232in}{1.592310in}}%
\pgfpathlineto{\pgfqpoint{3.644590in}{1.591357in}}%
\pgfpathlineto{\pgfqpoint{3.657954in}{1.590571in}}%
\pgfpathlineto{\pgfqpoint{3.671325in}{1.589952in}}%
\pgfpathlineto{\pgfqpoint{3.679248in}{1.600296in}}%
\pgfpathlineto{\pgfqpoint{3.687165in}{1.610703in}}%
\pgfpathlineto{\pgfqpoint{3.695076in}{1.621170in}}%
\pgfpathlineto{\pgfqpoint{3.702982in}{1.631691in}}%
\pgfpathlineto{\pgfqpoint{3.689622in}{1.631898in}}%
\pgfpathlineto{\pgfqpoint{3.676270in}{1.632271in}}%
\pgfpathlineto{\pgfqpoint{3.662924in}{1.632811in}}%
\pgfpathlineto{\pgfqpoint{3.649585in}{1.633518in}}%
\pgfpathlineto{\pgfqpoint{3.641668in}{1.623399in}}%
\pgfpathlineto{\pgfqpoint{3.633745in}{1.613341in}}%
\pgfpathlineto{\pgfqpoint{3.625816in}{1.603351in}}%
\pgfpathlineto{\pgfqpoint{3.617881in}{1.593430in}}%
\pgfpathclose%
\pgfusepath{fill}%
\end{pgfscope}%
\begin{pgfscope}%
\pgfpathrectangle{\pgfqpoint{1.254980in}{0.150000in}}{\pgfqpoint{5.490039in}{5.490039in}}%
\pgfusepath{clip}%
\pgfsetbuttcap%
\pgfsetroundjoin%
\definecolor{currentfill}{rgb}{0.265145,0.232956,0.516599}%
\pgfsetfillcolor{currentfill}%
\pgfsetfillopacity{0.700000}%
\pgfsetlinewidth{0.000000pt}%
\definecolor{currentstroke}{rgb}{0.000000,0.000000,0.000000}%
\pgfsetstrokecolor{currentstroke}%
\pgfsetdash{}{0pt}%
\pgfpathmoveto{\pgfqpoint{4.159267in}{1.960548in}}%
\pgfpathlineto{\pgfqpoint{4.172784in}{1.965682in}}%
\pgfpathlineto{\pgfqpoint{4.186311in}{1.970977in}}%
\pgfpathlineto{\pgfqpoint{4.199850in}{1.976435in}}%
\pgfpathlineto{\pgfqpoint{4.213401in}{1.982053in}}%
\pgfpathlineto{\pgfqpoint{4.221151in}{1.994305in}}%
\pgfpathlineto{\pgfqpoint{4.228897in}{2.006496in}}%
\pgfpathlineto{\pgfqpoint{4.236637in}{2.018624in}}%
\pgfpathlineto{\pgfqpoint{4.244373in}{2.030688in}}%
\pgfpathlineto{\pgfqpoint{4.230825in}{2.024848in}}%
\pgfpathlineto{\pgfqpoint{4.217288in}{2.019171in}}%
\pgfpathlineto{\pgfqpoint{4.203763in}{2.013655in}}%
\pgfpathlineto{\pgfqpoint{4.190250in}{2.008302in}}%
\pgfpathlineto{\pgfqpoint{4.182511in}{1.996448in}}%
\pgfpathlineto{\pgfqpoint{4.174768in}{1.984536in}}%
\pgfpathlineto{\pgfqpoint{4.167020in}{1.972569in}}%
\pgfpathlineto{\pgfqpoint{4.159267in}{1.960548in}}%
\pgfpathclose%
\pgfusepath{fill}%
\end{pgfscope}%
\begin{pgfscope}%
\pgfpathrectangle{\pgfqpoint{1.254980in}{0.150000in}}{\pgfqpoint{5.490039in}{5.490039in}}%
\pgfusepath{clip}%
\pgfsetbuttcap%
\pgfsetroundjoin%
\definecolor{currentfill}{rgb}{0.126453,0.570633,0.549841}%
\pgfsetfillcolor{currentfill}%
\pgfsetfillopacity{0.700000}%
\pgfsetlinewidth{0.000000pt}%
\definecolor{currentstroke}{rgb}{0.000000,0.000000,0.000000}%
\pgfsetstrokecolor{currentstroke}%
\pgfsetdash{}{0pt}%
\pgfpathmoveto{\pgfqpoint{5.048940in}{2.800983in}}%
\pgfpathlineto{\pgfqpoint{5.062939in}{2.812518in}}%
\pgfpathlineto{\pgfqpoint{5.076955in}{2.824213in}}%
\pgfpathlineto{\pgfqpoint{5.090988in}{2.836070in}}%
\pgfpathlineto{\pgfqpoint{5.105040in}{2.848088in}}%
\pgfpathlineto{\pgfqpoint{5.112429in}{2.854972in}}%
\pgfpathlineto{\pgfqpoint{5.119810in}{2.861730in}}%
\pgfpathlineto{\pgfqpoint{5.127183in}{2.868364in}}%
\pgfpathlineto{\pgfqpoint{5.134548in}{2.874875in}}%
\pgfpathlineto{\pgfqpoint{5.120505in}{2.863022in}}%
\pgfpathlineto{\pgfqpoint{5.106481in}{2.851330in}}%
\pgfpathlineto{\pgfqpoint{5.092475in}{2.839799in}}%
\pgfpathlineto{\pgfqpoint{5.078486in}{2.828428in}}%
\pgfpathlineto{\pgfqpoint{5.071112in}{2.821741in}}%
\pgfpathlineto{\pgfqpoint{5.063729in}{2.814939in}}%
\pgfpathlineto{\pgfqpoint{5.056339in}{2.808021in}}%
\pgfpathlineto{\pgfqpoint{5.048940in}{2.800983in}}%
\pgfpathclose%
\pgfusepath{fill}%
\end{pgfscope}%
\begin{pgfscope}%
\pgfpathrectangle{\pgfqpoint{1.254980in}{0.150000in}}{\pgfqpoint{5.490039in}{5.490039in}}%
\pgfusepath{clip}%
\pgfsetbuttcap%
\pgfsetroundjoin%
\definecolor{currentfill}{rgb}{0.283197,0.115680,0.436115}%
\pgfsetfillcolor{currentfill}%
\pgfsetfillopacity{0.700000}%
\pgfsetlinewidth{0.000000pt}%
\definecolor{currentstroke}{rgb}{0.000000,0.000000,0.000000}%
\pgfsetstrokecolor{currentstroke}%
\pgfsetdash{}{0pt}%
\pgfpathmoveto{\pgfqpoint{3.873031in}{1.727681in}}%
\pgfpathlineto{\pgfqpoint{3.886444in}{1.729704in}}%
\pgfpathlineto{\pgfqpoint{3.899866in}{1.731890in}}%
\pgfpathlineto{\pgfqpoint{3.913297in}{1.734240in}}%
\pgfpathlineto{\pgfqpoint{3.926737in}{1.736753in}}%
\pgfpathlineto{\pgfqpoint{3.934572in}{1.748635in}}%
\pgfpathlineto{\pgfqpoint{3.942402in}{1.760518in}}%
\pgfpathlineto{\pgfqpoint{3.950228in}{1.772398in}}%
\pgfpathlineto{\pgfqpoint{3.958049in}{1.784272in}}%
\pgfpathlineto{\pgfqpoint{3.944614in}{1.781428in}}%
\pgfpathlineto{\pgfqpoint{3.931190in}{1.778747in}}%
\pgfpathlineto{\pgfqpoint{3.917774in}{1.776230in}}%
\pgfpathlineto{\pgfqpoint{3.904368in}{1.773877in}}%
\pgfpathlineto{\pgfqpoint{3.896541in}{1.762323in}}%
\pgfpathlineto{\pgfqpoint{3.888709in}{1.750771in}}%
\pgfpathlineto{\pgfqpoint{3.880873in}{1.739222in}}%
\pgfpathlineto{\pgfqpoint{3.873031in}{1.727681in}}%
\pgfpathclose%
\pgfusepath{fill}%
\end{pgfscope}%
\begin{pgfscope}%
\pgfpathrectangle{\pgfqpoint{1.254980in}{0.150000in}}{\pgfqpoint{5.490039in}{5.490039in}}%
\pgfusepath{clip}%
\pgfsetbuttcap%
\pgfsetroundjoin%
\definecolor{currentfill}{rgb}{0.269944,0.014625,0.341379}%
\pgfsetfillcolor{currentfill}%
\pgfsetfillopacity{0.700000}%
\pgfsetlinewidth{0.000000pt}%
\definecolor{currentstroke}{rgb}{0.000000,0.000000,0.000000}%
\pgfsetstrokecolor{currentstroke}%
\pgfsetdash{}{0pt}%
\pgfpathmoveto{\pgfqpoint{3.532676in}{1.562454in}}%
\pgfpathlineto{\pgfqpoint{3.546018in}{1.560220in}}%
\pgfpathlineto{\pgfqpoint{3.559366in}{1.558155in}}%
\pgfpathlineto{\pgfqpoint{3.572719in}{1.556258in}}%
\pgfpathlineto{\pgfqpoint{3.586078in}{1.554530in}}%
\pgfpathlineto{\pgfqpoint{3.594038in}{1.564129in}}%
\pgfpathlineto{\pgfqpoint{3.601992in}{1.573815in}}%
\pgfpathlineto{\pgfqpoint{3.609939in}{1.583583in}}%
\pgfpathlineto{\pgfqpoint{3.617881in}{1.593430in}}%
\pgfpathlineto{\pgfqpoint{3.604535in}{1.594718in}}%
\pgfpathlineto{\pgfqpoint{3.591196in}{1.596174in}}%
\pgfpathlineto{\pgfqpoint{3.577862in}{1.597799in}}%
\pgfpathlineto{\pgfqpoint{3.564534in}{1.599594in}}%
\pgfpathlineto{\pgfqpoint{3.556580in}{1.590177in}}%
\pgfpathlineto{\pgfqpoint{3.548618in}{1.580845in}}%
\pgfpathlineto{\pgfqpoint{3.540650in}{1.571603in}}%
\pgfpathlineto{\pgfqpoint{3.532676in}{1.562454in}}%
\pgfpathclose%
\pgfusepath{fill}%
\end{pgfscope}%
\begin{pgfscope}%
\pgfpathrectangle{\pgfqpoint{1.254980in}{0.150000in}}{\pgfqpoint{5.490039in}{5.490039in}}%
\pgfusepath{clip}%
\pgfsetbuttcap%
\pgfsetroundjoin%
\definecolor{currentfill}{rgb}{0.267004,0.004874,0.329415}%
\pgfsetfillcolor{currentfill}%
\pgfsetfillopacity{0.700000}%
\pgfsetlinewidth{0.000000pt}%
\definecolor{currentstroke}{rgb}{0.000000,0.000000,0.000000}%
\pgfsetstrokecolor{currentstroke}%
\pgfsetdash{}{0pt}%
\pgfpathmoveto{\pgfqpoint{3.308424in}{1.544942in}}%
\pgfpathlineto{\pgfqpoint{3.321755in}{1.539670in}}%
\pgfpathlineto{\pgfqpoint{3.335090in}{1.534574in}}%
\pgfpathlineto{\pgfqpoint{3.348427in}{1.529653in}}%
\pgfpathlineto{\pgfqpoint{3.361767in}{1.524908in}}%
\pgfpathlineto{\pgfqpoint{3.369839in}{1.532133in}}%
\pgfpathlineto{\pgfqpoint{3.377902in}{1.539503in}}%
\pgfpathlineto{\pgfqpoint{3.385957in}{1.547012in}}%
\pgfpathlineto{\pgfqpoint{3.394004in}{1.554656in}}%
\pgfpathlineto{\pgfqpoint{3.380684in}{1.558906in}}%
\pgfpathlineto{\pgfqpoint{3.367368in}{1.563330in}}%
\pgfpathlineto{\pgfqpoint{3.354055in}{1.567930in}}%
\pgfpathlineto{\pgfqpoint{3.340745in}{1.572706in}}%
\pgfpathlineto{\pgfqpoint{3.332678in}{1.565548in}}%
\pgfpathlineto{\pgfqpoint{3.324602in}{1.558531in}}%
\pgfpathlineto{\pgfqpoint{3.316518in}{1.551661in}}%
\pgfpathlineto{\pgfqpoint{3.308424in}{1.544942in}}%
\pgfpathclose%
\pgfusepath{fill}%
\end{pgfscope}%
\begin{pgfscope}%
\pgfpathrectangle{\pgfqpoint{1.254980in}{0.150000in}}{\pgfqpoint{5.490039in}{5.490039in}}%
\pgfusepath{clip}%
\pgfsetbuttcap%
\pgfsetroundjoin%
\definecolor{currentfill}{rgb}{0.269944,0.014625,0.341379}%
\pgfsetfillcolor{currentfill}%
\pgfsetfillopacity{0.700000}%
\pgfsetlinewidth{0.000000pt}%
\definecolor{currentstroke}{rgb}{0.000000,0.000000,0.000000}%
\pgfsetstrokecolor{currentstroke}%
\pgfsetdash{}{0pt}%
\pgfpathmoveto{\pgfqpoint{3.169191in}{1.572496in}}%
\pgfpathlineto{\pgfqpoint{3.182531in}{1.565262in}}%
\pgfpathlineto{\pgfqpoint{3.195873in}{1.558209in}}%
\pgfpathlineto{\pgfqpoint{3.209216in}{1.551339in}}%
\pgfpathlineto{\pgfqpoint{3.222561in}{1.544649in}}%
\pgfpathlineto{\pgfqpoint{3.230717in}{1.550175in}}%
\pgfpathlineto{\pgfqpoint{3.238862in}{1.555881in}}%
\pgfpathlineto{\pgfqpoint{3.246998in}{1.561761in}}%
\pgfpathlineto{\pgfqpoint{3.255124in}{1.567809in}}%
\pgfpathlineto{\pgfqpoint{3.241805in}{1.573973in}}%
\pgfpathlineto{\pgfqpoint{3.228488in}{1.580318in}}%
\pgfpathlineto{\pgfqpoint{3.215172in}{1.586845in}}%
\pgfpathlineto{\pgfqpoint{3.201858in}{1.593553in}}%
\pgfpathlineto{\pgfqpoint{3.193706in}{1.588020in}}%
\pgfpathlineto{\pgfqpoint{3.185545in}{1.582663in}}%
\pgfpathlineto{\pgfqpoint{3.177373in}{1.577487in}}%
\pgfpathlineto{\pgfqpoint{3.169191in}{1.572496in}}%
\pgfpathclose%
\pgfusepath{fill}%
\end{pgfscope}%
\begin{pgfscope}%
\pgfpathrectangle{\pgfqpoint{1.254980in}{0.150000in}}{\pgfqpoint{5.490039in}{5.490039in}}%
\pgfusepath{clip}%
\pgfsetbuttcap%
\pgfsetroundjoin%
\definecolor{currentfill}{rgb}{0.278791,0.062145,0.386592}%
\pgfsetfillcolor{currentfill}%
\pgfsetfillopacity{0.700000}%
\pgfsetlinewidth{0.000000pt}%
\definecolor{currentstroke}{rgb}{0.000000,0.000000,0.000000}%
\pgfsetstrokecolor{currentstroke}%
\pgfsetdash{}{0pt}%
\pgfpathmoveto{\pgfqpoint{2.975924in}{1.662555in}}%
\pgfpathlineto{\pgfqpoint{2.989297in}{1.652502in}}%
\pgfpathlineto{\pgfqpoint{3.002668in}{1.642644in}}%
\pgfpathlineto{\pgfqpoint{3.016038in}{1.632978in}}%
\pgfpathlineto{\pgfqpoint{3.029408in}{1.623504in}}%
\pgfpathlineto{\pgfqpoint{3.037697in}{1.626561in}}%
\pgfpathlineto{\pgfqpoint{3.045973in}{1.629840in}}%
\pgfpathlineto{\pgfqpoint{3.054238in}{1.633336in}}%
\pgfpathlineto{\pgfqpoint{3.062490in}{1.637044in}}%
\pgfpathlineto{\pgfqpoint{3.049152in}{1.645960in}}%
\pgfpathlineto{\pgfqpoint{3.035815in}{1.655067in}}%
\pgfpathlineto{\pgfqpoint{3.022476in}{1.664367in}}%
\pgfpathlineto{\pgfqpoint{3.009137in}{1.673860in}}%
\pgfpathlineto{\pgfqpoint{3.000853in}{1.670700in}}%
\pgfpathlineto{\pgfqpoint{2.992556in}{1.667758in}}%
\pgfpathlineto{\pgfqpoint{2.984247in}{1.665042in}}%
\pgfpathlineto{\pgfqpoint{2.975924in}{1.662555in}}%
\pgfpathclose%
\pgfusepath{fill}%
\end{pgfscope}%
\begin{pgfscope}%
\pgfpathrectangle{\pgfqpoint{1.254980in}{0.150000in}}{\pgfqpoint{5.490039in}{5.490039in}}%
\pgfusepath{clip}%
\pgfsetbuttcap%
\pgfsetroundjoin%
\definecolor{currentfill}{rgb}{0.153894,0.680203,0.504172}%
\pgfsetfillcolor{currentfill}%
\pgfsetfillopacity{0.700000}%
\pgfsetlinewidth{0.000000pt}%
\definecolor{currentstroke}{rgb}{0.000000,0.000000,0.000000}%
\pgfsetstrokecolor{currentstroke}%
\pgfsetdash{}{0pt}%
\pgfpathmoveto{\pgfqpoint{5.420583in}{3.107720in}}%
\pgfpathlineto{\pgfqpoint{5.434816in}{3.120696in}}%
\pgfpathlineto{\pgfqpoint{5.449068in}{3.133832in}}%
\pgfpathlineto{\pgfqpoint{5.463341in}{3.147130in}}%
\pgfpathlineto{\pgfqpoint{5.477633in}{3.160588in}}%
\pgfpathlineto{\pgfqpoint{5.484793in}{3.164202in}}%
\pgfpathlineto{\pgfqpoint{5.491944in}{3.167716in}}%
\pgfpathlineto{\pgfqpoint{5.499086in}{3.171133in}}%
\pgfpathlineto{\pgfqpoint{5.506219in}{3.174458in}}%
\pgfpathlineto{\pgfqpoint{5.491945in}{3.161321in}}%
\pgfpathlineto{\pgfqpoint{5.477690in}{3.148344in}}%
\pgfpathlineto{\pgfqpoint{5.463456in}{3.135527in}}%
\pgfpathlineto{\pgfqpoint{5.449241in}{3.122870in}}%
\pgfpathlineto{\pgfqpoint{5.442089in}{3.119214in}}%
\pgfpathlineto{\pgfqpoint{5.434929in}{3.115473in}}%
\pgfpathlineto{\pgfqpoint{5.427760in}{3.111643in}}%
\pgfpathlineto{\pgfqpoint{5.420583in}{3.107720in}}%
\pgfpathclose%
\pgfusepath{fill}%
\end{pgfscope}%
\begin{pgfscope}%
\pgfpathrectangle{\pgfqpoint{1.254980in}{0.150000in}}{\pgfqpoint{5.490039in}{5.490039in}}%
\pgfusepath{clip}%
\pgfsetbuttcap%
\pgfsetroundjoin%
\definecolor{currentfill}{rgb}{0.171176,0.452530,0.557965}%
\pgfsetfillcolor{currentfill}%
\pgfsetfillopacity{0.700000}%
\pgfsetlinewidth{0.000000pt}%
\definecolor{currentstroke}{rgb}{0.000000,0.000000,0.000000}%
\pgfsetstrokecolor{currentstroke}%
\pgfsetdash{}{0pt}%
\pgfpathmoveto{\pgfqpoint{2.218789in}{2.578038in}}%
\pgfpathlineto{\pgfqpoint{2.232573in}{2.554684in}}%
\pgfpathlineto{\pgfqpoint{2.246343in}{2.531626in}}%
\pgfpathlineto{\pgfqpoint{2.260098in}{2.508862in}}%
\pgfpathlineto{\pgfqpoint{2.273839in}{2.486388in}}%
\pgfpathlineto{\pgfqpoint{2.282763in}{2.481207in}}%
\pgfpathlineto{\pgfqpoint{2.291664in}{2.476361in}}%
\pgfpathlineto{\pgfqpoint{2.300543in}{2.471843in}}%
\pgfpathlineto{\pgfqpoint{2.309400in}{2.467648in}}%
\pgfpathlineto{\pgfqpoint{2.295717in}{2.489516in}}%
\pgfpathlineto{\pgfqpoint{2.282020in}{2.511674in}}%
\pgfpathlineto{\pgfqpoint{2.268310in}{2.534123in}}%
\pgfpathlineto{\pgfqpoint{2.254585in}{2.556866in}}%
\pgfpathlineto{\pgfqpoint{2.245671in}{2.561656in}}%
\pgfpathlineto{\pgfqpoint{2.236734in}{2.566778in}}%
\pgfpathlineto{\pgfqpoint{2.227773in}{2.572236in}}%
\pgfpathlineto{\pgfqpoint{2.218789in}{2.578038in}}%
\pgfpathclose%
\pgfusepath{fill}%
\end{pgfscope}%
\begin{pgfscope}%
\pgfpathrectangle{\pgfqpoint{1.254980in}{0.150000in}}{\pgfqpoint{5.490039in}{5.490039in}}%
\pgfusepath{clip}%
\pgfsetbuttcap%
\pgfsetroundjoin%
\definecolor{currentfill}{rgb}{0.150476,0.504369,0.557430}%
\pgfsetfillcolor{currentfill}%
\pgfsetfillopacity{0.700000}%
\pgfsetlinewidth{0.000000pt}%
\definecolor{currentstroke}{rgb}{0.000000,0.000000,0.000000}%
\pgfsetstrokecolor{currentstroke}%
\pgfsetdash{}{0pt}%
\pgfpathmoveto{\pgfqpoint{4.848014in}{2.617674in}}%
\pgfpathlineto{\pgfqpoint{4.861896in}{2.628229in}}%
\pgfpathlineto{\pgfqpoint{4.875796in}{2.638945in}}%
\pgfpathlineto{\pgfqpoint{4.889711in}{2.649822in}}%
\pgfpathlineto{\pgfqpoint{4.903644in}{2.660860in}}%
\pgfpathlineto{\pgfqpoint{4.911142in}{2.669517in}}%
\pgfpathlineto{\pgfqpoint{4.918632in}{2.678044in}}%
\pgfpathlineto{\pgfqpoint{4.926115in}{2.686442in}}%
\pgfpathlineto{\pgfqpoint{4.933590in}{2.694713in}}%
\pgfpathlineto{\pgfqpoint{4.919664in}{2.683747in}}%
\pgfpathlineto{\pgfqpoint{4.905754in}{2.672943in}}%
\pgfpathlineto{\pgfqpoint{4.891861in}{2.662299in}}%
\pgfpathlineto{\pgfqpoint{4.877984in}{2.651817in}}%
\pgfpathlineto{\pgfqpoint{4.870502in}{2.643462in}}%
\pgfpathlineto{\pgfqpoint{4.863013in}{2.634988in}}%
\pgfpathlineto{\pgfqpoint{4.855517in}{2.626392in}}%
\pgfpathlineto{\pgfqpoint{4.848014in}{2.617674in}}%
\pgfpathclose%
\pgfusepath{fill}%
\end{pgfscope}%
\begin{pgfscope}%
\pgfpathrectangle{\pgfqpoint{1.254980in}{0.150000in}}{\pgfqpoint{5.490039in}{5.490039in}}%
\pgfusepath{clip}%
\pgfsetbuttcap%
\pgfsetroundjoin%
\definecolor{currentfill}{rgb}{0.281887,0.150881,0.465405}%
\pgfsetfillcolor{currentfill}%
\pgfsetfillopacity{0.700000}%
\pgfsetlinewidth{0.000000pt}%
\definecolor{currentstroke}{rgb}{0.000000,0.000000,0.000000}%
\pgfsetstrokecolor{currentstroke}%
\pgfsetdash{}{0pt}%
\pgfpathmoveto{\pgfqpoint{3.958049in}{1.784272in}}%
\pgfpathlineto{\pgfqpoint{3.971492in}{1.787279in}}%
\pgfpathlineto{\pgfqpoint{3.984946in}{1.790449in}}%
\pgfpathlineto{\pgfqpoint{3.998409in}{1.793782in}}%
\pgfpathlineto{\pgfqpoint{4.011882in}{1.797276in}}%
\pgfpathlineto{\pgfqpoint{4.019693in}{1.809456in}}%
\pgfpathlineto{\pgfqpoint{4.027500in}{1.821617in}}%
\pgfpathlineto{\pgfqpoint{4.035302in}{1.833757in}}%
\pgfpathlineto{\pgfqpoint{4.043100in}{1.845872in}}%
\pgfpathlineto{\pgfqpoint{4.029631in}{1.842073in}}%
\pgfpathlineto{\pgfqpoint{4.016173in}{1.838437in}}%
\pgfpathlineto{\pgfqpoint{4.002724in}{1.834964in}}%
\pgfpathlineto{\pgfqpoint{3.989285in}{1.831653in}}%
\pgfpathlineto{\pgfqpoint{3.981483in}{1.819831in}}%
\pgfpathlineto{\pgfqpoint{3.973676in}{1.807991in}}%
\pgfpathlineto{\pgfqpoint{3.965865in}{1.796138in}}%
\pgfpathlineto{\pgfqpoint{3.958049in}{1.784272in}}%
\pgfpathclose%
\pgfusepath{fill}%
\end{pgfscope}%
\begin{pgfscope}%
\pgfpathrectangle{\pgfqpoint{1.254980in}{0.150000in}}{\pgfqpoint{5.490039in}{5.490039in}}%
\pgfusepath{clip}%
\pgfsetbuttcap%
\pgfsetroundjoin%
\definecolor{currentfill}{rgb}{0.179019,0.433756,0.557430}%
\pgfsetfillcolor{currentfill}%
\pgfsetfillopacity{0.700000}%
\pgfsetlinewidth{0.000000pt}%
\definecolor{currentstroke}{rgb}{0.000000,0.000000,0.000000}%
\pgfsetstrokecolor{currentstroke}%
\pgfsetdash{}{0pt}%
\pgfpathmoveto{\pgfqpoint{4.646852in}{2.424271in}}%
\pgfpathlineto{\pgfqpoint{4.660620in}{2.433573in}}%
\pgfpathlineto{\pgfqpoint{4.674403in}{2.443037in}}%
\pgfpathlineto{\pgfqpoint{4.688201in}{2.452662in}}%
\pgfpathlineto{\pgfqpoint{4.702015in}{2.462447in}}%
\pgfpathlineto{\pgfqpoint{4.709604in}{2.472665in}}%
\pgfpathlineto{\pgfqpoint{4.717187in}{2.482761in}}%
\pgfpathlineto{\pgfqpoint{4.724763in}{2.492735in}}%
\pgfpathlineto{\pgfqpoint{4.732332in}{2.502586in}}%
\pgfpathlineto{\pgfqpoint{4.718521in}{2.492783in}}%
\pgfpathlineto{\pgfqpoint{4.704726in}{2.483140in}}%
\pgfpathlineto{\pgfqpoint{4.690946in}{2.473659in}}%
\pgfpathlineto{\pgfqpoint{4.677182in}{2.464339in}}%
\pgfpathlineto{\pgfqpoint{4.669609in}{2.454494in}}%
\pgfpathlineto{\pgfqpoint{4.662030in}{2.444535in}}%
\pgfpathlineto{\pgfqpoint{4.654444in}{2.434461in}}%
\pgfpathlineto{\pgfqpoint{4.646852in}{2.424271in}}%
\pgfpathclose%
\pgfusepath{fill}%
\end{pgfscope}%
\begin{pgfscope}%
\pgfpathrectangle{\pgfqpoint{1.254980in}{0.150000in}}{\pgfqpoint{5.490039in}{5.490039in}}%
\pgfusepath{clip}%
\pgfsetbuttcap%
\pgfsetroundjoin%
\definecolor{currentfill}{rgb}{0.214298,0.355619,0.551184}%
\pgfsetfillcolor{currentfill}%
\pgfsetfillopacity{0.700000}%
\pgfsetlinewidth{0.000000pt}%
\definecolor{currentstroke}{rgb}{0.000000,0.000000,0.000000}%
\pgfsetstrokecolor{currentstroke}%
\pgfsetdash{}{0pt}%
\pgfpathmoveto{\pgfqpoint{4.445613in}{2.226286in}}%
\pgfpathlineto{\pgfqpoint{4.459272in}{2.234068in}}%
\pgfpathlineto{\pgfqpoint{4.472944in}{2.242012in}}%
\pgfpathlineto{\pgfqpoint{4.486631in}{2.250117in}}%
\pgfpathlineto{\pgfqpoint{4.500331in}{2.258383in}}%
\pgfpathlineto{\pgfqpoint{4.507996in}{2.269823in}}%
\pgfpathlineto{\pgfqpoint{4.515655in}{2.281159in}}%
\pgfpathlineto{\pgfqpoint{4.523308in}{2.292391in}}%
\pgfpathlineto{\pgfqpoint{4.530956in}{2.303516in}}%
\pgfpathlineto{\pgfqpoint{4.517258in}{2.295144in}}%
\pgfpathlineto{\pgfqpoint{4.503573in}{2.286933in}}%
\pgfpathlineto{\pgfqpoint{4.489902in}{2.278884in}}%
\pgfpathlineto{\pgfqpoint{4.476246in}{2.270995in}}%
\pgfpathlineto{\pgfqpoint{4.468596in}{2.259965in}}%
\pgfpathlineto{\pgfqpoint{4.460940in}{2.248836in}}%
\pgfpathlineto{\pgfqpoint{4.453279in}{2.237609in}}%
\pgfpathlineto{\pgfqpoint{4.445613in}{2.226286in}}%
\pgfpathclose%
\pgfusepath{fill}%
\end{pgfscope}%
\begin{pgfscope}%
\pgfpathrectangle{\pgfqpoint{1.254980in}{0.150000in}}{\pgfqpoint{5.490039in}{5.490039in}}%
\pgfusepath{clip}%
\pgfsetbuttcap%
\pgfsetroundjoin%
\definecolor{currentfill}{rgb}{0.252194,0.269783,0.531579}%
\pgfsetfillcolor{currentfill}%
\pgfsetfillopacity{0.700000}%
\pgfsetlinewidth{0.000000pt}%
\definecolor{currentstroke}{rgb}{0.000000,0.000000,0.000000}%
\pgfsetstrokecolor{currentstroke}%
\pgfsetdash{}{0pt}%
\pgfpathmoveto{\pgfqpoint{4.244373in}{2.030688in}}%
\pgfpathlineto{\pgfqpoint{4.257934in}{2.036688in}}%
\pgfpathlineto{\pgfqpoint{4.271507in}{2.042851in}}%
\pgfpathlineto{\pgfqpoint{4.285092in}{2.049174in}}%
\pgfpathlineto{\pgfqpoint{4.298690in}{2.055659in}}%
\pgfpathlineto{\pgfqpoint{4.306419in}{2.067859in}}%
\pgfpathlineto{\pgfqpoint{4.314144in}{2.079983in}}%
\pgfpathlineto{\pgfqpoint{4.321863in}{2.092031in}}%
\pgfpathlineto{\pgfqpoint{4.329578in}{2.104000in}}%
\pgfpathlineto{\pgfqpoint{4.315982in}{2.097322in}}%
\pgfpathlineto{\pgfqpoint{4.302398in}{2.090807in}}%
\pgfpathlineto{\pgfqpoint{4.288828in}{2.084452in}}%
\pgfpathlineto{\pgfqpoint{4.275269in}{2.078259in}}%
\pgfpathlineto{\pgfqpoint{4.267552in}{2.066472in}}%
\pgfpathlineto{\pgfqpoint{4.259831in}{2.054613in}}%
\pgfpathlineto{\pgfqpoint{4.252105in}{2.042684in}}%
\pgfpathlineto{\pgfqpoint{4.244373in}{2.030688in}}%
\pgfpathclose%
\pgfusepath{fill}%
\end{pgfscope}%
\begin{pgfscope}%
\pgfpathrectangle{\pgfqpoint{1.254980in}{0.150000in}}{\pgfqpoint{5.490039in}{5.490039in}}%
\pgfusepath{clip}%
\pgfsetbuttcap%
\pgfsetroundjoin%
\definecolor{currentfill}{rgb}{0.267004,0.004874,0.329415}%
\pgfsetfillcolor{currentfill}%
\pgfsetfillopacity{0.700000}%
\pgfsetlinewidth{0.000000pt}%
\definecolor{currentstroke}{rgb}{0.000000,0.000000,0.000000}%
\pgfsetstrokecolor{currentstroke}%
\pgfsetdash{}{0pt}%
\pgfpathmoveto{\pgfqpoint{3.447321in}{1.539395in}}%
\pgfpathlineto{\pgfqpoint{3.460661in}{1.536010in}}%
\pgfpathlineto{\pgfqpoint{3.474005in}{1.532797in}}%
\pgfpathlineto{\pgfqpoint{3.487354in}{1.529755in}}%
\pgfpathlineto{\pgfqpoint{3.500707in}{1.526883in}}%
\pgfpathlineto{\pgfqpoint{3.508710in}{1.535613in}}%
\pgfpathlineto{\pgfqpoint{3.516705in}{1.544455in}}%
\pgfpathlineto{\pgfqpoint{3.524694in}{1.553404in}}%
\pgfpathlineto{\pgfqpoint{3.532676in}{1.562454in}}%
\pgfpathlineto{\pgfqpoint{3.519338in}{1.564859in}}%
\pgfpathlineto{\pgfqpoint{3.506006in}{1.567433in}}%
\pgfpathlineto{\pgfqpoint{3.492679in}{1.570178in}}%
\pgfpathlineto{\pgfqpoint{3.479356in}{1.573095in}}%
\pgfpathlineto{\pgfqpoint{3.471359in}{1.564501in}}%
\pgfpathlineto{\pgfqpoint{3.463354in}{1.556017in}}%
\pgfpathlineto{\pgfqpoint{3.455341in}{1.547647in}}%
\pgfpathlineto{\pgfqpoint{3.447321in}{1.539395in}}%
\pgfpathclose%
\pgfusepath{fill}%
\end{pgfscope}%
\begin{pgfscope}%
\pgfpathrectangle{\pgfqpoint{1.254980in}{0.150000in}}{\pgfqpoint{5.490039in}{5.490039in}}%
\pgfusepath{clip}%
\pgfsetbuttcap%
\pgfsetroundjoin%
\definecolor{currentfill}{rgb}{0.120092,0.600104,0.542530}%
\pgfsetfillcolor{currentfill}%
\pgfsetfillopacity{0.700000}%
\pgfsetlinewidth{0.000000pt}%
\definecolor{currentstroke}{rgb}{0.000000,0.000000,0.000000}%
\pgfsetstrokecolor{currentstroke}%
\pgfsetdash{}{0pt}%
\pgfpathmoveto{\pgfqpoint{5.134548in}{2.874875in}}%
\pgfpathlineto{\pgfqpoint{5.148608in}{2.886890in}}%
\pgfpathlineto{\pgfqpoint{5.162686in}{2.899065in}}%
\pgfpathlineto{\pgfqpoint{5.176783in}{2.911401in}}%
\pgfpathlineto{\pgfqpoint{5.190899in}{2.923899in}}%
\pgfpathlineto{\pgfqpoint{5.198245in}{2.930106in}}%
\pgfpathlineto{\pgfqpoint{5.205582in}{2.936189in}}%
\pgfpathlineto{\pgfqpoint{5.212910in}{2.942149in}}%
\pgfpathlineto{\pgfqpoint{5.220230in}{2.947988in}}%
\pgfpathlineto{\pgfqpoint{5.206126in}{2.935687in}}%
\pgfpathlineto{\pgfqpoint{5.192040in}{2.923546in}}%
\pgfpathlineto{\pgfqpoint{5.177972in}{2.911566in}}%
\pgfpathlineto{\pgfqpoint{5.163923in}{2.899747in}}%
\pgfpathlineto{\pgfqpoint{5.156591in}{2.893701in}}%
\pgfpathlineto{\pgfqpoint{5.149252in}{2.887542in}}%
\pgfpathlineto{\pgfqpoint{5.141904in}{2.881267in}}%
\pgfpathlineto{\pgfqpoint{5.134548in}{2.874875in}}%
\pgfpathclose%
\pgfusepath{fill}%
\end{pgfscope}%
\begin{pgfscope}%
\pgfpathrectangle{\pgfqpoint{1.254980in}{0.150000in}}{\pgfqpoint{5.490039in}{5.490039in}}%
\pgfusepath{clip}%
\pgfsetbuttcap%
\pgfsetroundjoin%
\definecolor{currentfill}{rgb}{0.191090,0.708366,0.482284}%
\pgfsetfillcolor{currentfill}%
\pgfsetfillopacity{0.700000}%
\pgfsetlinewidth{0.000000pt}%
\definecolor{currentstroke}{rgb}{0.000000,0.000000,0.000000}%
\pgfsetstrokecolor{currentstroke}%
\pgfsetdash{}{0pt}%
\pgfpathmoveto{\pgfqpoint{5.506219in}{3.174458in}}%
\pgfpathlineto{\pgfqpoint{5.520513in}{3.187756in}}%
\pgfpathlineto{\pgfqpoint{5.534827in}{3.201215in}}%
\pgfpathlineto{\pgfqpoint{5.549162in}{3.214835in}}%
\pgfpathlineto{\pgfqpoint{5.563517in}{3.228615in}}%
\pgfpathlineto{\pgfqpoint{5.570621in}{3.231510in}}%
\pgfpathlineto{\pgfqpoint{5.577716in}{3.234313in}}%
\pgfpathlineto{\pgfqpoint{5.584802in}{3.237027in}}%
\pgfpathlineto{\pgfqpoint{5.591879in}{3.239657in}}%
\pgfpathlineto{\pgfqpoint{5.577544in}{3.226229in}}%
\pgfpathlineto{\pgfqpoint{5.563230in}{3.212962in}}%
\pgfpathlineto{\pgfqpoint{5.548936in}{3.199855in}}%
\pgfpathlineto{\pgfqpoint{5.534662in}{3.186907in}}%
\pgfpathlineto{\pgfqpoint{5.527564in}{3.183915in}}%
\pgfpathlineto{\pgfqpoint{5.520458in}{3.180845in}}%
\pgfpathlineto{\pgfqpoint{5.513343in}{3.177694in}}%
\pgfpathlineto{\pgfqpoint{5.506219in}{3.174458in}}%
\pgfpathclose%
\pgfusepath{fill}%
\end{pgfscope}%
\begin{pgfscope}%
\pgfpathrectangle{\pgfqpoint{1.254980in}{0.150000in}}{\pgfqpoint{5.490039in}{5.490039in}}%
\pgfusepath{clip}%
\pgfsetbuttcap%
\pgfsetroundjoin%
\definecolor{currentfill}{rgb}{0.277134,0.185228,0.489898}%
\pgfsetfillcolor{currentfill}%
\pgfsetfillopacity{0.700000}%
\pgfsetlinewidth{0.000000pt}%
\definecolor{currentstroke}{rgb}{0.000000,0.000000,0.000000}%
\pgfsetstrokecolor{currentstroke}%
\pgfsetdash{}{0pt}%
\pgfpathmoveto{\pgfqpoint{4.043100in}{1.845872in}}%
\pgfpathlineto{\pgfqpoint{4.056579in}{1.849833in}}%
\pgfpathlineto{\pgfqpoint{4.070069in}{1.853956in}}%
\pgfpathlineto{\pgfqpoint{4.083569in}{1.858241in}}%
\pgfpathlineto{\pgfqpoint{4.097080in}{1.862688in}}%
\pgfpathlineto{\pgfqpoint{4.104869in}{1.875064in}}%
\pgfpathlineto{\pgfqpoint{4.112654in}{1.887404in}}%
\pgfpathlineto{\pgfqpoint{4.120435in}{1.899705in}}%
\pgfpathlineto{\pgfqpoint{4.128210in}{1.911965in}}%
\pgfpathlineto{\pgfqpoint{4.114703in}{1.907241in}}%
\pgfpathlineto{\pgfqpoint{4.101206in}{1.902680in}}%
\pgfpathlineto{\pgfqpoint{4.087720in}{1.898280in}}%
\pgfpathlineto{\pgfqpoint{4.074244in}{1.894043in}}%
\pgfpathlineto{\pgfqpoint{4.066465in}{1.882049in}}%
\pgfpathlineto{\pgfqpoint{4.058681in}{1.870021in}}%
\pgfpathlineto{\pgfqpoint{4.050893in}{1.857961in}}%
\pgfpathlineto{\pgfqpoint{4.043100in}{1.845872in}}%
\pgfpathclose%
\pgfusepath{fill}%
\end{pgfscope}%
\begin{pgfscope}%
\pgfpathrectangle{\pgfqpoint{1.254980in}{0.150000in}}{\pgfqpoint{5.490039in}{5.490039in}}%
\pgfusepath{clip}%
\pgfsetbuttcap%
\pgfsetroundjoin%
\definecolor{currentfill}{rgb}{0.262138,0.242286,0.520837}%
\pgfsetfillcolor{currentfill}%
\pgfsetfillopacity{0.700000}%
\pgfsetlinewidth{0.000000pt}%
\definecolor{currentstroke}{rgb}{0.000000,0.000000,0.000000}%
\pgfsetstrokecolor{currentstroke}%
\pgfsetdash{}{0pt}%
\pgfpathmoveto{\pgfqpoint{2.565604in}{2.036888in}}%
\pgfpathlineto{\pgfqpoint{2.579138in}{2.020284in}}%
\pgfpathlineto{\pgfqpoint{2.592665in}{2.003913in}}%
\pgfpathlineto{\pgfqpoint{2.606185in}{1.987774in}}%
\pgfpathlineto{\pgfqpoint{2.619699in}{1.971865in}}%
\pgfpathlineto{\pgfqpoint{2.628331in}{1.969721in}}%
\pgfpathlineto{\pgfqpoint{2.636946in}{1.967877in}}%
\pgfpathlineto{\pgfqpoint{2.645542in}{1.966328in}}%
\pgfpathlineto{\pgfqpoint{2.654120in}{1.965067in}}%
\pgfpathlineto{\pgfqpoint{2.640654in}{1.980370in}}%
\pgfpathlineto{\pgfqpoint{2.627182in}{1.995901in}}%
\pgfpathlineto{\pgfqpoint{2.613703in}{2.011664in}}%
\pgfpathlineto{\pgfqpoint{2.600217in}{2.027658in}}%
\pgfpathlineto{\pgfqpoint{2.591592in}{2.029515in}}%
\pgfpathlineto{\pgfqpoint{2.582948in}{2.031667in}}%
\pgfpathlineto{\pgfqpoint{2.574285in}{2.034123in}}%
\pgfpathlineto{\pgfqpoint{2.565604in}{2.036888in}}%
\pgfpathclose%
\pgfusepath{fill}%
\end{pgfscope}%
\begin{pgfscope}%
\pgfpathrectangle{\pgfqpoint{1.254980in}{0.150000in}}{\pgfqpoint{5.490039in}{5.490039in}}%
\pgfusepath{clip}%
\pgfsetbuttcap%
\pgfsetroundjoin%
\definecolor{currentfill}{rgb}{0.270595,0.214069,0.507052}%
\pgfsetfillcolor{currentfill}%
\pgfsetfillopacity{0.700000}%
\pgfsetlinewidth{0.000000pt}%
\definecolor{currentstroke}{rgb}{0.000000,0.000000,0.000000}%
\pgfsetstrokecolor{currentstroke}%
\pgfsetdash{}{0pt}%
\pgfpathmoveto{\pgfqpoint{2.619699in}{1.971865in}}%
\pgfpathlineto{\pgfqpoint{2.633205in}{1.956184in}}%
\pgfpathlineto{\pgfqpoint{2.646705in}{1.940730in}}%
\pgfpathlineto{\pgfqpoint{2.660199in}{1.925501in}}%
\pgfpathlineto{\pgfqpoint{2.673686in}{1.910495in}}%
\pgfpathlineto{\pgfqpoint{2.682272in}{1.908967in}}%
\pgfpathlineto{\pgfqpoint{2.690840in}{1.907731in}}%
\pgfpathlineto{\pgfqpoint{2.699391in}{1.906782in}}%
\pgfpathlineto{\pgfqpoint{2.707924in}{1.906114in}}%
\pgfpathlineto{\pgfqpoint{2.694482in}{1.920517in}}%
\pgfpathlineto{\pgfqpoint{2.681034in}{1.935142in}}%
\pgfpathlineto{\pgfqpoint{2.667580in}{1.949992in}}%
\pgfpathlineto{\pgfqpoint{2.654120in}{1.965067in}}%
\pgfpathlineto{\pgfqpoint{2.645542in}{1.966328in}}%
\pgfpathlineto{\pgfqpoint{2.636946in}{1.967877in}}%
\pgfpathlineto{\pgfqpoint{2.628331in}{1.969721in}}%
\pgfpathlineto{\pgfqpoint{2.619699in}{1.971865in}}%
\pgfpathclose%
\pgfusepath{fill}%
\end{pgfscope}%
\begin{pgfscope}%
\pgfpathrectangle{\pgfqpoint{1.254980in}{0.150000in}}{\pgfqpoint{5.490039in}{5.490039in}}%
\pgfusepath{clip}%
\pgfsetbuttcap%
\pgfsetroundjoin%
\definecolor{currentfill}{rgb}{0.276022,0.044167,0.370164}%
\pgfsetfillcolor{currentfill}%
\pgfsetfillopacity{0.700000}%
\pgfsetlinewidth{0.000000pt}%
\definecolor{currentstroke}{rgb}{0.000000,0.000000,0.000000}%
\pgfsetstrokecolor{currentstroke}%
\pgfsetdash{}{0pt}%
\pgfpathmoveto{\pgfqpoint{3.029408in}{1.623504in}}%
\pgfpathlineto{\pgfqpoint{3.042776in}{1.614221in}}%
\pgfpathlineto{\pgfqpoint{3.056144in}{1.605128in}}%
\pgfpathlineto{\pgfqpoint{3.069512in}{1.596224in}}%
\pgfpathlineto{\pgfqpoint{3.082880in}{1.587508in}}%
\pgfpathlineto{\pgfqpoint{3.091137in}{1.591133in}}%
\pgfpathlineto{\pgfqpoint{3.099382in}{1.594973in}}%
\pgfpathlineto{\pgfqpoint{3.107615in}{1.599023in}}%
\pgfpathlineto{\pgfqpoint{3.115837in}{1.603276in}}%
\pgfpathlineto{\pgfqpoint{3.102500in}{1.611436in}}%
\pgfpathlineto{\pgfqpoint{3.089163in}{1.619783in}}%
\pgfpathlineto{\pgfqpoint{3.075827in}{1.628319in}}%
\pgfpathlineto{\pgfqpoint{3.062490in}{1.637044in}}%
\pgfpathlineto{\pgfqpoint{3.054238in}{1.633336in}}%
\pgfpathlineto{\pgfqpoint{3.045973in}{1.629840in}}%
\pgfpathlineto{\pgfqpoint{3.037697in}{1.626561in}}%
\pgfpathlineto{\pgfqpoint{3.029408in}{1.623504in}}%
\pgfpathclose%
\pgfusepath{fill}%
\end{pgfscope}%
\begin{pgfscope}%
\pgfpathrectangle{\pgfqpoint{1.254980in}{0.150000in}}{\pgfqpoint{5.490039in}{5.490039in}}%
\pgfusepath{clip}%
\pgfsetbuttcap%
\pgfsetroundjoin%
\definecolor{currentfill}{rgb}{0.250425,0.274290,0.533103}%
\pgfsetfillcolor{currentfill}%
\pgfsetfillopacity{0.700000}%
\pgfsetlinewidth{0.000000pt}%
\definecolor{currentstroke}{rgb}{0.000000,0.000000,0.000000}%
\pgfsetstrokecolor{currentstroke}%
\pgfsetdash{}{0pt}%
\pgfpathmoveto{\pgfqpoint{2.511387in}{2.105674in}}%
\pgfpathlineto{\pgfqpoint{2.524953in}{2.088118in}}%
\pgfpathlineto{\pgfqpoint{2.538511in}{2.070803in}}%
\pgfpathlineto{\pgfqpoint{2.552061in}{2.053727in}}%
\pgfpathlineto{\pgfqpoint{2.565604in}{2.036888in}}%
\pgfpathlineto{\pgfqpoint{2.574285in}{2.034123in}}%
\pgfpathlineto{\pgfqpoint{2.582948in}{2.031667in}}%
\pgfpathlineto{\pgfqpoint{2.591592in}{2.029515in}}%
\pgfpathlineto{\pgfqpoint{2.600217in}{2.027658in}}%
\pgfpathlineto{\pgfqpoint{2.586724in}{2.043887in}}%
\pgfpathlineto{\pgfqpoint{2.573223in}{2.060352in}}%
\pgfpathlineto{\pgfqpoint{2.559715in}{2.077055in}}%
\pgfpathlineto{\pgfqpoint{2.546199in}{2.093997in}}%
\pgfpathlineto{\pgfqpoint{2.537525in}{2.096453in}}%
\pgfpathlineto{\pgfqpoint{2.528832in}{2.099214in}}%
\pgfpathlineto{\pgfqpoint{2.520120in}{2.102285in}}%
\pgfpathlineto{\pgfqpoint{2.511387in}{2.105674in}}%
\pgfpathclose%
\pgfusepath{fill}%
\end{pgfscope}%
\begin{pgfscope}%
\pgfpathrectangle{\pgfqpoint{1.254980in}{0.150000in}}{\pgfqpoint{5.490039in}{5.490039in}}%
\pgfusepath{clip}%
\pgfsetbuttcap%
\pgfsetroundjoin%
\definecolor{currentfill}{rgb}{0.277134,0.185228,0.489898}%
\pgfsetfillcolor{currentfill}%
\pgfsetfillopacity{0.700000}%
\pgfsetlinewidth{0.000000pt}%
\definecolor{currentstroke}{rgb}{0.000000,0.000000,0.000000}%
\pgfsetstrokecolor{currentstroke}%
\pgfsetdash{}{0pt}%
\pgfpathmoveto{\pgfqpoint{2.673686in}{1.910495in}}%
\pgfpathlineto{\pgfqpoint{2.687168in}{1.895710in}}%
\pgfpathlineto{\pgfqpoint{2.700645in}{1.881146in}}%
\pgfpathlineto{\pgfqpoint{2.714116in}{1.866801in}}%
\pgfpathlineto{\pgfqpoint{2.727581in}{1.852672in}}%
\pgfpathlineto{\pgfqpoint{2.736122in}{1.851757in}}%
\pgfpathlineto{\pgfqpoint{2.744645in}{1.851126in}}%
\pgfpathlineto{\pgfqpoint{2.753152in}{1.850774in}}%
\pgfpathlineto{\pgfqpoint{2.761643in}{1.850695in}}%
\pgfpathlineto{\pgfqpoint{2.748220in}{1.864224in}}%
\pgfpathlineto{\pgfqpoint{2.734793in}{1.877969in}}%
\pgfpathlineto{\pgfqpoint{2.721361in}{1.891931in}}%
\pgfpathlineto{\pgfqpoint{2.707924in}{1.906114in}}%
\pgfpathlineto{\pgfqpoint{2.699391in}{1.906782in}}%
\pgfpathlineto{\pgfqpoint{2.690840in}{1.907731in}}%
\pgfpathlineto{\pgfqpoint{2.682272in}{1.908967in}}%
\pgfpathlineto{\pgfqpoint{2.673686in}{1.910495in}}%
\pgfpathclose%
\pgfusepath{fill}%
\end{pgfscope}%
\begin{pgfscope}%
\pgfpathrectangle{\pgfqpoint{1.254980in}{0.150000in}}{\pgfqpoint{5.490039in}{5.490039in}}%
\pgfusepath{clip}%
\pgfsetbuttcap%
\pgfsetroundjoin%
\definecolor{currentfill}{rgb}{0.154815,0.493313,0.557840}%
\pgfsetfillcolor{currentfill}%
\pgfsetfillopacity{0.700000}%
\pgfsetlinewidth{0.000000pt}%
\definecolor{currentstroke}{rgb}{0.000000,0.000000,0.000000}%
\pgfsetstrokecolor{currentstroke}%
\pgfsetdash{}{0pt}%
\pgfpathmoveto{\pgfqpoint{2.163501in}{2.674476in}}%
\pgfpathlineto{\pgfqpoint{2.177346in}{2.649907in}}%
\pgfpathlineto{\pgfqpoint{2.191176in}{2.625647in}}%
\pgfpathlineto{\pgfqpoint{2.204990in}{2.601691in}}%
\pgfpathlineto{\pgfqpoint{2.218789in}{2.578038in}}%
\pgfpathlineto{\pgfqpoint{2.227773in}{2.572236in}}%
\pgfpathlineto{\pgfqpoint{2.236734in}{2.566778in}}%
\pgfpathlineto{\pgfqpoint{2.245671in}{2.561656in}}%
\pgfpathlineto{\pgfqpoint{2.254585in}{2.556866in}}%
\pgfpathlineto{\pgfqpoint{2.240847in}{2.579908in}}%
\pgfpathlineto{\pgfqpoint{2.227093in}{2.603250in}}%
\pgfpathlineto{\pgfqpoint{2.213325in}{2.626895in}}%
\pgfpathlineto{\pgfqpoint{2.199541in}{2.650847in}}%
\pgfpathlineto{\pgfqpoint{2.190567in}{2.656238in}}%
\pgfpathlineto{\pgfqpoint{2.181569in}{2.661969in}}%
\pgfpathlineto{\pgfqpoint{2.172547in}{2.668046in}}%
\pgfpathlineto{\pgfqpoint{2.163501in}{2.674476in}}%
\pgfpathclose%
\pgfusepath{fill}%
\end{pgfscope}%
\begin{pgfscope}%
\pgfpathrectangle{\pgfqpoint{1.254980in}{0.150000in}}{\pgfqpoint{5.490039in}{5.490039in}}%
\pgfusepath{clip}%
\pgfsetbuttcap%
\pgfsetroundjoin%
\definecolor{currentfill}{rgb}{0.237441,0.305202,0.541921}%
\pgfsetfillcolor{currentfill}%
\pgfsetfillopacity{0.700000}%
\pgfsetlinewidth{0.000000pt}%
\definecolor{currentstroke}{rgb}{0.000000,0.000000,0.000000}%
\pgfsetstrokecolor{currentstroke}%
\pgfsetdash{}{0pt}%
\pgfpathmoveto{\pgfqpoint{2.457033in}{2.178344in}}%
\pgfpathlineto{\pgfqpoint{2.470635in}{2.159805in}}%
\pgfpathlineto{\pgfqpoint{2.484228in}{2.141515in}}%
\pgfpathlineto{\pgfqpoint{2.497812in}{2.123472in}}%
\pgfpathlineto{\pgfqpoint{2.511387in}{2.105674in}}%
\pgfpathlineto{\pgfqpoint{2.520120in}{2.102285in}}%
\pgfpathlineto{\pgfqpoint{2.528832in}{2.099214in}}%
\pgfpathlineto{\pgfqpoint{2.537525in}{2.096453in}}%
\pgfpathlineto{\pgfqpoint{2.546199in}{2.093997in}}%
\pgfpathlineto{\pgfqpoint{2.532675in}{2.111182in}}%
\pgfpathlineto{\pgfqpoint{2.519143in}{2.128609in}}%
\pgfpathlineto{\pgfqpoint{2.505602in}{2.146282in}}%
\pgfpathlineto{\pgfqpoint{2.492052in}{2.164203in}}%
\pgfpathlineto{\pgfqpoint{2.483328in}{2.167263in}}%
\pgfpathlineto{\pgfqpoint{2.474584in}{2.170635in}}%
\pgfpathlineto{\pgfqpoint{2.465819in}{2.174326in}}%
\pgfpathlineto{\pgfqpoint{2.457033in}{2.178344in}}%
\pgfpathclose%
\pgfusepath{fill}%
\end{pgfscope}%
\begin{pgfscope}%
\pgfpathrectangle{\pgfqpoint{1.254980in}{0.150000in}}{\pgfqpoint{5.490039in}{5.490039in}}%
\pgfusepath{clip}%
\pgfsetbuttcap%
\pgfsetroundjoin%
\definecolor{currentfill}{rgb}{0.280868,0.160771,0.472899}%
\pgfsetfillcolor{currentfill}%
\pgfsetfillopacity{0.700000}%
\pgfsetlinewidth{0.000000pt}%
\definecolor{currentstroke}{rgb}{0.000000,0.000000,0.000000}%
\pgfsetstrokecolor{currentstroke}%
\pgfsetdash{}{0pt}%
\pgfpathmoveto{\pgfqpoint{2.727581in}{1.852672in}}%
\pgfpathlineto{\pgfqpoint{2.741042in}{1.838759in}}%
\pgfpathlineto{\pgfqpoint{2.754498in}{1.825061in}}%
\pgfpathlineto{\pgfqpoint{2.767950in}{1.811575in}}%
\pgfpathlineto{\pgfqpoint{2.781397in}{1.798301in}}%
\pgfpathlineto{\pgfqpoint{2.789894in}{1.797995in}}%
\pgfpathlineto{\pgfqpoint{2.798375in}{1.797967in}}%
\pgfpathlineto{\pgfqpoint{2.806840in}{1.798209in}}%
\pgfpathlineto{\pgfqpoint{2.815289in}{1.798716in}}%
\pgfpathlineto{\pgfqpoint{2.801884in}{1.811393in}}%
\pgfpathlineto{\pgfqpoint{2.788474in}{1.824281in}}%
\pgfpathlineto{\pgfqpoint{2.775061in}{1.837381in}}%
\pgfpathlineto{\pgfqpoint{2.761643in}{1.850695in}}%
\pgfpathlineto{\pgfqpoint{2.753152in}{1.850774in}}%
\pgfpathlineto{\pgfqpoint{2.744645in}{1.851126in}}%
\pgfpathlineto{\pgfqpoint{2.736122in}{1.851757in}}%
\pgfpathlineto{\pgfqpoint{2.727581in}{1.852672in}}%
\pgfpathclose%
\pgfusepath{fill}%
\end{pgfscope}%
\begin{pgfscope}%
\pgfpathrectangle{\pgfqpoint{1.254980in}{0.150000in}}{\pgfqpoint{5.490039in}{5.490039in}}%
\pgfusepath{clip}%
\pgfsetbuttcap%
\pgfsetroundjoin%
\definecolor{currentfill}{rgb}{0.137770,0.537492,0.554906}%
\pgfsetfillcolor{currentfill}%
\pgfsetfillopacity{0.700000}%
\pgfsetlinewidth{0.000000pt}%
\definecolor{currentstroke}{rgb}{0.000000,0.000000,0.000000}%
\pgfsetstrokecolor{currentstroke}%
\pgfsetdash{}{0pt}%
\pgfpathmoveto{\pgfqpoint{4.933590in}{2.694713in}}%
\pgfpathlineto{\pgfqpoint{4.947534in}{2.705839in}}%
\pgfpathlineto{\pgfqpoint{4.961495in}{2.717127in}}%
\pgfpathlineto{\pgfqpoint{4.975474in}{2.728577in}}%
\pgfpathlineto{\pgfqpoint{4.989469in}{2.740188in}}%
\pgfpathlineto{\pgfqpoint{4.996931in}{2.748240in}}%
\pgfpathlineto{\pgfqpoint{5.004384in}{2.756160in}}%
\pgfpathlineto{\pgfqpoint{5.011830in}{2.763948in}}%
\pgfpathlineto{\pgfqpoint{5.019268in}{2.771608in}}%
\pgfpathlineto{\pgfqpoint{5.005279in}{2.760101in}}%
\pgfpathlineto{\pgfqpoint{4.991308in}{2.748755in}}%
\pgfpathlineto{\pgfqpoint{4.977354in}{2.737570in}}%
\pgfpathlineto{\pgfqpoint{4.963417in}{2.726546in}}%
\pgfpathlineto{\pgfqpoint{4.955972in}{2.718772in}}%
\pgfpathlineto{\pgfqpoint{4.948519in}{2.710876in}}%
\pgfpathlineto{\pgfqpoint{4.941058in}{2.702857in}}%
\pgfpathlineto{\pgfqpoint{4.933590in}{2.694713in}}%
\pgfpathclose%
\pgfusepath{fill}%
\end{pgfscope}%
\begin{pgfscope}%
\pgfpathrectangle{\pgfqpoint{1.254980in}{0.150000in}}{\pgfqpoint{5.490039in}{5.490039in}}%
\pgfusepath{clip}%
\pgfsetbuttcap%
\pgfsetroundjoin%
\definecolor{currentfill}{rgb}{0.268510,0.009605,0.335427}%
\pgfsetfillcolor{currentfill}%
\pgfsetfillopacity{0.700000}%
\pgfsetlinewidth{0.000000pt}%
\definecolor{currentstroke}{rgb}{0.000000,0.000000,0.000000}%
\pgfsetstrokecolor{currentstroke}%
\pgfsetdash{}{0pt}%
\pgfpathmoveto{\pgfqpoint{3.222561in}{1.544649in}}%
\pgfpathlineto{\pgfqpoint{3.235907in}{1.538139in}}%
\pgfpathlineto{\pgfqpoint{3.249256in}{1.531808in}}%
\pgfpathlineto{\pgfqpoint{3.262606in}{1.525656in}}%
\pgfpathlineto{\pgfqpoint{3.275958in}{1.519681in}}%
\pgfpathlineto{\pgfqpoint{3.284089in}{1.525744in}}%
\pgfpathlineto{\pgfqpoint{3.292210in}{1.531979in}}%
\pgfpathlineto{\pgfqpoint{3.300322in}{1.538380in}}%
\pgfpathlineto{\pgfqpoint{3.308424in}{1.544942in}}%
\pgfpathlineto{\pgfqpoint{3.295096in}{1.550392in}}%
\pgfpathlineto{\pgfqpoint{3.281769in}{1.556019in}}%
\pgfpathlineto{\pgfqpoint{3.268446in}{1.561824in}}%
\pgfpathlineto{\pgfqpoint{3.255124in}{1.567809in}}%
\pgfpathlineto{\pgfqpoint{3.246998in}{1.561761in}}%
\pgfpathlineto{\pgfqpoint{3.238862in}{1.555881in}}%
\pgfpathlineto{\pgfqpoint{3.230717in}{1.550175in}}%
\pgfpathlineto{\pgfqpoint{3.222561in}{1.544649in}}%
\pgfpathclose%
\pgfusepath{fill}%
\end{pgfscope}%
\begin{pgfscope}%
\pgfpathrectangle{\pgfqpoint{1.254980in}{0.150000in}}{\pgfqpoint{5.490039in}{5.490039in}}%
\pgfusepath{clip}%
\pgfsetbuttcap%
\pgfsetroundjoin%
\definecolor{currentfill}{rgb}{0.232815,0.732247,0.459277}%
\pgfsetfillcolor{currentfill}%
\pgfsetfillopacity{0.700000}%
\pgfsetlinewidth{0.000000pt}%
\definecolor{currentstroke}{rgb}{0.000000,0.000000,0.000000}%
\pgfsetstrokecolor{currentstroke}%
\pgfsetdash{}{0pt}%
\pgfpathmoveto{\pgfqpoint{5.591879in}{3.239657in}}%
\pgfpathlineto{\pgfqpoint{5.606234in}{3.253245in}}%
\pgfpathlineto{\pgfqpoint{5.620609in}{3.266994in}}%
\pgfpathlineto{\pgfqpoint{5.635005in}{3.280904in}}%
\pgfpathlineto{\pgfqpoint{5.649422in}{3.294975in}}%
\pgfpathlineto{\pgfqpoint{5.656468in}{3.297151in}}%
\pgfpathlineto{\pgfqpoint{5.663505in}{3.299244in}}%
\pgfpathlineto{\pgfqpoint{5.670532in}{3.301258in}}%
\pgfpathlineto{\pgfqpoint{5.677550in}{3.303197in}}%
\pgfpathlineto{\pgfqpoint{5.663156in}{3.289511in}}%
\pgfpathlineto{\pgfqpoint{5.648783in}{3.275986in}}%
\pgfpathlineto{\pgfqpoint{5.634430in}{3.262620in}}%
\pgfpathlineto{\pgfqpoint{5.620097in}{3.249415in}}%
\pgfpathlineto{\pgfqpoint{5.613055in}{3.247081in}}%
\pgfpathlineto{\pgfqpoint{5.606005in}{3.244680in}}%
\pgfpathlineto{\pgfqpoint{5.598946in}{3.242206in}}%
\pgfpathlineto{\pgfqpoint{5.591879in}{3.239657in}}%
\pgfpathclose%
\pgfusepath{fill}%
\end{pgfscope}%
\begin{pgfscope}%
\pgfpathrectangle{\pgfqpoint{1.254980in}{0.150000in}}{\pgfqpoint{5.490039in}{5.490039in}}%
\pgfusepath{clip}%
\pgfsetbuttcap%
\pgfsetroundjoin%
\definecolor{currentfill}{rgb}{0.267004,0.004874,0.329415}%
\pgfsetfillcolor{currentfill}%
\pgfsetfillopacity{0.700000}%
\pgfsetlinewidth{0.000000pt}%
\definecolor{currentstroke}{rgb}{0.000000,0.000000,0.000000}%
\pgfsetstrokecolor{currentstroke}%
\pgfsetdash{}{0pt}%
\pgfpathmoveto{\pgfqpoint{3.361767in}{1.524908in}}%
\pgfpathlineto{\pgfqpoint{3.375111in}{1.520336in}}%
\pgfpathlineto{\pgfqpoint{3.388458in}{1.515939in}}%
\pgfpathlineto{\pgfqpoint{3.401809in}{1.511714in}}%
\pgfpathlineto{\pgfqpoint{3.415163in}{1.507662in}}%
\pgfpathlineto{\pgfqpoint{3.423215in}{1.515394in}}%
\pgfpathlineto{\pgfqpoint{3.431258in}{1.523264in}}%
\pgfpathlineto{\pgfqpoint{3.439294in}{1.531265in}}%
\pgfpathlineto{\pgfqpoint{3.447321in}{1.539395in}}%
\pgfpathlineto{\pgfqpoint{3.433986in}{1.542951in}}%
\pgfpathlineto{\pgfqpoint{3.420654in}{1.546679in}}%
\pgfpathlineto{\pgfqpoint{3.407327in}{1.550581in}}%
\pgfpathlineto{\pgfqpoint{3.394004in}{1.554656in}}%
\pgfpathlineto{\pgfqpoint{3.385957in}{1.547012in}}%
\pgfpathlineto{\pgfqpoint{3.377902in}{1.539503in}}%
\pgfpathlineto{\pgfqpoint{3.369839in}{1.532133in}}%
\pgfpathlineto{\pgfqpoint{3.361767in}{1.524908in}}%
\pgfpathclose%
\pgfusepath{fill}%
\end{pgfscope}%
\begin{pgfscope}%
\pgfpathrectangle{\pgfqpoint{1.254980in}{0.150000in}}{\pgfqpoint{5.490039in}{5.490039in}}%
\pgfusepath{clip}%
\pgfsetbuttcap%
\pgfsetroundjoin%
\definecolor{currentfill}{rgb}{0.235526,0.309527,0.542944}%
\pgfsetfillcolor{currentfill}%
\pgfsetfillopacity{0.700000}%
\pgfsetlinewidth{0.000000pt}%
\definecolor{currentstroke}{rgb}{0.000000,0.000000,0.000000}%
\pgfsetstrokecolor{currentstroke}%
\pgfsetdash{}{0pt}%
\pgfpathmoveto{\pgfqpoint{4.329578in}{2.104000in}}%
\pgfpathlineto{\pgfqpoint{4.343187in}{2.110838in}}%
\pgfpathlineto{\pgfqpoint{4.356809in}{2.117838in}}%
\pgfpathlineto{\pgfqpoint{4.370444in}{2.124999in}}%
\pgfpathlineto{\pgfqpoint{4.384092in}{2.132321in}}%
\pgfpathlineto{\pgfqpoint{4.391800in}{2.144385in}}%
\pgfpathlineto{\pgfqpoint{4.399503in}{2.156361in}}%
\pgfpathlineto{\pgfqpoint{4.407201in}{2.168246in}}%
\pgfpathlineto{\pgfqpoint{4.414894in}{2.180040in}}%
\pgfpathlineto{\pgfqpoint{4.401247in}{2.172554in}}%
\pgfpathlineto{\pgfqpoint{4.387614in}{2.165229in}}%
\pgfpathlineto{\pgfqpoint{4.373993in}{2.158066in}}%
\pgfpathlineto{\pgfqpoint{4.360386in}{2.151063in}}%
\pgfpathlineto{\pgfqpoint{4.352692in}{2.139422in}}%
\pgfpathlineto{\pgfqpoint{4.344992in}{2.127697in}}%
\pgfpathlineto{\pgfqpoint{4.337288in}{2.115889in}}%
\pgfpathlineto{\pgfqpoint{4.329578in}{2.104000in}}%
\pgfpathclose%
\pgfusepath{fill}%
\end{pgfscope}%
\begin{pgfscope}%
\pgfpathrectangle{\pgfqpoint{1.254980in}{0.150000in}}{\pgfqpoint{5.490039in}{5.490039in}}%
\pgfusepath{clip}%
\pgfsetbuttcap%
\pgfsetroundjoin%
\definecolor{currentfill}{rgb}{0.121380,0.629492,0.531973}%
\pgfsetfillcolor{currentfill}%
\pgfsetfillopacity{0.700000}%
\pgfsetlinewidth{0.000000pt}%
\definecolor{currentstroke}{rgb}{0.000000,0.000000,0.000000}%
\pgfsetstrokecolor{currentstroke}%
\pgfsetdash{}{0pt}%
\pgfpathmoveto{\pgfqpoint{5.220230in}{2.947988in}}%
\pgfpathlineto{\pgfqpoint{5.234353in}{2.960451in}}%
\pgfpathlineto{\pgfqpoint{5.248495in}{2.973075in}}%
\pgfpathlineto{\pgfqpoint{5.262656in}{2.985860in}}%
\pgfpathlineto{\pgfqpoint{5.276836in}{2.998807in}}%
\pgfpathlineto{\pgfqpoint{5.284135in}{3.004313in}}%
\pgfpathlineto{\pgfqpoint{5.291425in}{3.009696in}}%
\pgfpathlineto{\pgfqpoint{5.298706in}{3.014960in}}%
\pgfpathlineto{\pgfqpoint{5.305979in}{3.020106in}}%
\pgfpathlineto{\pgfqpoint{5.291812in}{3.007388in}}%
\pgfpathlineto{\pgfqpoint{5.277664in}{2.994830in}}%
\pgfpathlineto{\pgfqpoint{5.263535in}{2.982433in}}%
\pgfpathlineto{\pgfqpoint{5.249424in}{2.970197in}}%
\pgfpathlineto{\pgfqpoint{5.242138in}{2.964812in}}%
\pgfpathlineto{\pgfqpoint{5.234844in}{2.959317in}}%
\pgfpathlineto{\pgfqpoint{5.227541in}{2.953710in}}%
\pgfpathlineto{\pgfqpoint{5.220230in}{2.947988in}}%
\pgfpathclose%
\pgfusepath{fill}%
\end{pgfscope}%
\begin{pgfscope}%
\pgfpathrectangle{\pgfqpoint{1.254980in}{0.150000in}}{\pgfqpoint{5.490039in}{5.490039in}}%
\pgfusepath{clip}%
\pgfsetbuttcap%
\pgfsetroundjoin%
\definecolor{currentfill}{rgb}{0.197636,0.391528,0.554969}%
\pgfsetfillcolor{currentfill}%
\pgfsetfillopacity{0.700000}%
\pgfsetlinewidth{0.000000pt}%
\definecolor{currentstroke}{rgb}{0.000000,0.000000,0.000000}%
\pgfsetstrokecolor{currentstroke}%
\pgfsetdash{}{0pt}%
\pgfpathmoveto{\pgfqpoint{4.530956in}{2.303516in}}%
\pgfpathlineto{\pgfqpoint{4.544670in}{2.312050in}}%
\pgfpathlineto{\pgfqpoint{4.558397in}{2.320744in}}%
\pgfpathlineto{\pgfqpoint{4.572140in}{2.329600in}}%
\pgfpathlineto{\pgfqpoint{4.585897in}{2.338617in}}%
\pgfpathlineto{\pgfqpoint{4.593537in}{2.349724in}}%
\pgfpathlineto{\pgfqpoint{4.601172in}{2.360718in}}%
\pgfpathlineto{\pgfqpoint{4.608800in}{2.371598in}}%
\pgfpathlineto{\pgfqpoint{4.616423in}{2.382362in}}%
\pgfpathlineto{\pgfqpoint{4.602668in}{2.373269in}}%
\pgfpathlineto{\pgfqpoint{4.588927in}{2.364336in}}%
\pgfpathlineto{\pgfqpoint{4.575202in}{2.355565in}}%
\pgfpathlineto{\pgfqpoint{4.561491in}{2.346955in}}%
\pgfpathlineto{\pgfqpoint{4.553866in}{2.336256in}}%
\pgfpathlineto{\pgfqpoint{4.546235in}{2.325449in}}%
\pgfpathlineto{\pgfqpoint{4.538599in}{2.314536in}}%
\pgfpathlineto{\pgfqpoint{4.530956in}{2.303516in}}%
\pgfpathclose%
\pgfusepath{fill}%
\end{pgfscope}%
\begin{pgfscope}%
\pgfpathrectangle{\pgfqpoint{1.254980in}{0.150000in}}{\pgfqpoint{5.490039in}{5.490039in}}%
\pgfusepath{clip}%
\pgfsetbuttcap%
\pgfsetroundjoin%
\definecolor{currentfill}{rgb}{0.165117,0.467423,0.558141}%
\pgfsetfillcolor{currentfill}%
\pgfsetfillopacity{0.700000}%
\pgfsetlinewidth{0.000000pt}%
\definecolor{currentstroke}{rgb}{0.000000,0.000000,0.000000}%
\pgfsetstrokecolor{currentstroke}%
\pgfsetdash{}{0pt}%
\pgfpathmoveto{\pgfqpoint{4.732332in}{2.502586in}}%
\pgfpathlineto{\pgfqpoint{4.746159in}{2.512551in}}%
\pgfpathlineto{\pgfqpoint{4.760001in}{2.522677in}}%
\pgfpathlineto{\pgfqpoint{4.773860in}{2.532964in}}%
\pgfpathlineto{\pgfqpoint{4.787734in}{2.543412in}}%
\pgfpathlineto{\pgfqpoint{4.795294in}{2.553141in}}%
\pgfpathlineto{\pgfqpoint{4.802846in}{2.562741in}}%
\pgfpathlineto{\pgfqpoint{4.810392in}{2.572213in}}%
\pgfpathlineto{\pgfqpoint{4.817930in}{2.581557in}}%
\pgfpathlineto{\pgfqpoint{4.804059in}{2.571121in}}%
\pgfpathlineto{\pgfqpoint{4.790205in}{2.560847in}}%
\pgfpathlineto{\pgfqpoint{4.776366in}{2.550733in}}%
\pgfpathlineto{\pgfqpoint{4.762543in}{2.540781in}}%
\pgfpathlineto{\pgfqpoint{4.755001in}{2.531413in}}%
\pgfpathlineto{\pgfqpoint{4.747451in}{2.521925in}}%
\pgfpathlineto{\pgfqpoint{4.739895in}{2.512316in}}%
\pgfpathlineto{\pgfqpoint{4.732332in}{2.502586in}}%
\pgfpathclose%
\pgfusepath{fill}%
\end{pgfscope}%
\begin{pgfscope}%
\pgfpathrectangle{\pgfqpoint{1.254980in}{0.150000in}}{\pgfqpoint{5.490039in}{5.490039in}}%
\pgfusepath{clip}%
\pgfsetbuttcap%
\pgfsetroundjoin%
\definecolor{currentfill}{rgb}{0.223925,0.334994,0.548053}%
\pgfsetfillcolor{currentfill}%
\pgfsetfillopacity{0.700000}%
\pgfsetlinewidth{0.000000pt}%
\definecolor{currentstroke}{rgb}{0.000000,0.000000,0.000000}%
\pgfsetstrokecolor{currentstroke}%
\pgfsetdash{}{0pt}%
\pgfpathmoveto{\pgfqpoint{2.402525in}{2.255026in}}%
\pgfpathlineto{\pgfqpoint{2.416167in}{2.235472in}}%
\pgfpathlineto{\pgfqpoint{2.429799in}{2.216175in}}%
\pgfpathlineto{\pgfqpoint{2.443421in}{2.197133in}}%
\pgfpathlineto{\pgfqpoint{2.457033in}{2.178344in}}%
\pgfpathlineto{\pgfqpoint{2.465819in}{2.174326in}}%
\pgfpathlineto{\pgfqpoint{2.474584in}{2.170635in}}%
\pgfpathlineto{\pgfqpoint{2.483328in}{2.167263in}}%
\pgfpathlineto{\pgfqpoint{2.492052in}{2.164203in}}%
\pgfpathlineto{\pgfqpoint{2.478493in}{2.182374in}}%
\pgfpathlineto{\pgfqpoint{2.464925in}{2.200796in}}%
\pgfpathlineto{\pgfqpoint{2.451348in}{2.219471in}}%
\pgfpathlineto{\pgfqpoint{2.437760in}{2.238403in}}%
\pgfpathlineto{\pgfqpoint{2.428983in}{2.242070in}}%
\pgfpathlineto{\pgfqpoint{2.420185in}{2.246059in}}%
\pgfpathlineto{\pgfqpoint{2.411366in}{2.250376in}}%
\pgfpathlineto{\pgfqpoint{2.402525in}{2.255026in}}%
\pgfpathclose%
\pgfusepath{fill}%
\end{pgfscope}%
\begin{pgfscope}%
\pgfpathrectangle{\pgfqpoint{1.254980in}{0.150000in}}{\pgfqpoint{5.490039in}{5.490039in}}%
\pgfusepath{clip}%
\pgfsetbuttcap%
\pgfsetroundjoin%
\definecolor{currentfill}{rgb}{0.269308,0.218818,0.509577}%
\pgfsetfillcolor{currentfill}%
\pgfsetfillopacity{0.700000}%
\pgfsetlinewidth{0.000000pt}%
\definecolor{currentstroke}{rgb}{0.000000,0.000000,0.000000}%
\pgfsetstrokecolor{currentstroke}%
\pgfsetdash{}{0pt}%
\pgfpathmoveto{\pgfqpoint{4.128210in}{1.911965in}}%
\pgfpathlineto{\pgfqpoint{4.141729in}{1.916850in}}%
\pgfpathlineto{\pgfqpoint{4.155260in}{1.921897in}}%
\pgfpathlineto{\pgfqpoint{4.168801in}{1.927105in}}%
\pgfpathlineto{\pgfqpoint{4.182354in}{1.932475in}}%
\pgfpathlineto{\pgfqpoint{4.190123in}{1.944951in}}%
\pgfpathlineto{\pgfqpoint{4.197887in}{1.957374in}}%
\pgfpathlineto{\pgfqpoint{4.205646in}{1.969742in}}%
\pgfpathlineto{\pgfqpoint{4.213401in}{1.982053in}}%
\pgfpathlineto{\pgfqpoint{4.199850in}{1.976435in}}%
\pgfpathlineto{\pgfqpoint{4.186311in}{1.970977in}}%
\pgfpathlineto{\pgfqpoint{4.172784in}{1.965682in}}%
\pgfpathlineto{\pgfqpoint{4.159267in}{1.960548in}}%
\pgfpathlineto{\pgfqpoint{4.151510in}{1.948475in}}%
\pgfpathlineto{\pgfqpoint{4.143748in}{1.936352in}}%
\pgfpathlineto{\pgfqpoint{4.135982in}{1.924181in}}%
\pgfpathlineto{\pgfqpoint{4.128210in}{1.911965in}}%
\pgfpathclose%
\pgfusepath{fill}%
\end{pgfscope}%
\begin{pgfscope}%
\pgfpathrectangle{\pgfqpoint{1.254980in}{0.150000in}}{\pgfqpoint{5.490039in}{5.490039in}}%
\pgfusepath{clip}%
\pgfsetbuttcap%
\pgfsetroundjoin%
\definecolor{currentfill}{rgb}{0.277018,0.050344,0.375715}%
\pgfsetfillcolor{currentfill}%
\pgfsetfillopacity{0.700000}%
\pgfsetlinewidth{0.000000pt}%
\definecolor{currentstroke}{rgb}{0.000000,0.000000,0.000000}%
\pgfsetstrokecolor{currentstroke}%
\pgfsetdash{}{0pt}%
\pgfpathmoveto{\pgfqpoint{3.671325in}{1.589952in}}%
\pgfpathlineto{\pgfqpoint{3.684702in}{1.589500in}}%
\pgfpathlineto{\pgfqpoint{3.698087in}{1.589212in}}%
\pgfpathlineto{\pgfqpoint{3.711478in}{1.589091in}}%
\pgfpathlineto{\pgfqpoint{3.724876in}{1.589134in}}%
\pgfpathlineto{\pgfqpoint{3.732788in}{1.599902in}}%
\pgfpathlineto{\pgfqpoint{3.740695in}{1.610725in}}%
\pgfpathlineto{\pgfqpoint{3.748596in}{1.621601in}}%
\pgfpathlineto{\pgfqpoint{3.756491in}{1.632526in}}%
\pgfpathlineto{\pgfqpoint{3.743103in}{1.632069in}}%
\pgfpathlineto{\pgfqpoint{3.729722in}{1.631777in}}%
\pgfpathlineto{\pgfqpoint{3.716348in}{1.631651in}}%
\pgfpathlineto{\pgfqpoint{3.702982in}{1.631691in}}%
\pgfpathlineto{\pgfqpoint{3.695076in}{1.621170in}}%
\pgfpathlineto{\pgfqpoint{3.687165in}{1.610703in}}%
\pgfpathlineto{\pgfqpoint{3.679248in}{1.600296in}}%
\pgfpathlineto{\pgfqpoint{3.671325in}{1.589952in}}%
\pgfpathclose%
\pgfusepath{fill}%
\end{pgfscope}%
\begin{pgfscope}%
\pgfpathrectangle{\pgfqpoint{1.254980in}{0.150000in}}{\pgfqpoint{5.490039in}{5.490039in}}%
\pgfusepath{clip}%
\pgfsetbuttcap%
\pgfsetroundjoin%
\definecolor{currentfill}{rgb}{0.282884,0.135920,0.453427}%
\pgfsetfillcolor{currentfill}%
\pgfsetfillopacity{0.700000}%
\pgfsetlinewidth{0.000000pt}%
\definecolor{currentstroke}{rgb}{0.000000,0.000000,0.000000}%
\pgfsetstrokecolor{currentstroke}%
\pgfsetdash{}{0pt}%
\pgfpathmoveto{\pgfqpoint{2.781397in}{1.798301in}}%
\pgfpathlineto{\pgfqpoint{2.794840in}{1.785237in}}%
\pgfpathlineto{\pgfqpoint{2.808279in}{1.772381in}}%
\pgfpathlineto{\pgfqpoint{2.821714in}{1.759733in}}%
\pgfpathlineto{\pgfqpoint{2.835146in}{1.747291in}}%
\pgfpathlineto{\pgfqpoint{2.843602in}{1.747592in}}%
\pgfpathlineto{\pgfqpoint{2.852042in}{1.748163in}}%
\pgfpathlineto{\pgfqpoint{2.860467in}{1.748996in}}%
\pgfpathlineto{\pgfqpoint{2.868877in}{1.750087in}}%
\pgfpathlineto{\pgfqpoint{2.855485in}{1.761934in}}%
\pgfpathlineto{\pgfqpoint{2.842090in}{1.773988in}}%
\pgfpathlineto{\pgfqpoint{2.828691in}{1.786248in}}%
\pgfpathlineto{\pgfqpoint{2.815289in}{1.798716in}}%
\pgfpathlineto{\pgfqpoint{2.806840in}{1.798209in}}%
\pgfpathlineto{\pgfqpoint{2.798375in}{1.797967in}}%
\pgfpathlineto{\pgfqpoint{2.789894in}{1.797995in}}%
\pgfpathlineto{\pgfqpoint{2.781397in}{1.798301in}}%
\pgfpathclose%
\pgfusepath{fill}%
\end{pgfscope}%
\begin{pgfscope}%
\pgfpathrectangle{\pgfqpoint{1.254980in}{0.150000in}}{\pgfqpoint{5.490039in}{5.490039in}}%
\pgfusepath{clip}%
\pgfsetbuttcap%
\pgfsetroundjoin%
\definecolor{currentfill}{rgb}{0.280267,0.073417,0.397163}%
\pgfsetfillcolor{currentfill}%
\pgfsetfillopacity{0.700000}%
\pgfsetlinewidth{0.000000pt}%
\definecolor{currentstroke}{rgb}{0.000000,0.000000,0.000000}%
\pgfsetstrokecolor{currentstroke}%
\pgfsetdash{}{0pt}%
\pgfpathmoveto{\pgfqpoint{3.756491in}{1.632526in}}%
\pgfpathlineto{\pgfqpoint{3.769887in}{1.633148in}}%
\pgfpathlineto{\pgfqpoint{3.783291in}{1.633934in}}%
\pgfpathlineto{\pgfqpoint{3.796702in}{1.634884in}}%
\pgfpathlineto{\pgfqpoint{3.810122in}{1.635998in}}%
\pgfpathlineto{\pgfqpoint{3.818003in}{1.647364in}}%
\pgfpathlineto{\pgfqpoint{3.825879in}{1.658764in}}%
\pgfpathlineto{\pgfqpoint{3.833751in}{1.670194in}}%
\pgfpathlineto{\pgfqpoint{3.841617in}{1.681651in}}%
\pgfpathlineto{\pgfqpoint{3.828205in}{1.680151in}}%
\pgfpathlineto{\pgfqpoint{3.814802in}{1.678814in}}%
\pgfpathlineto{\pgfqpoint{3.801407in}{1.677642in}}%
\pgfpathlineto{\pgfqpoint{3.788020in}{1.676635in}}%
\pgfpathlineto{\pgfqpoint{3.780146in}{1.665553in}}%
\pgfpathlineto{\pgfqpoint{3.772266in}{1.654506in}}%
\pgfpathlineto{\pgfqpoint{3.764381in}{1.643495in}}%
\pgfpathlineto{\pgfqpoint{3.756491in}{1.632526in}}%
\pgfpathclose%
\pgfusepath{fill}%
\end{pgfscope}%
\begin{pgfscope}%
\pgfpathrectangle{\pgfqpoint{1.254980in}{0.150000in}}{\pgfqpoint{5.490039in}{5.490039in}}%
\pgfusepath{clip}%
\pgfsetbuttcap%
\pgfsetroundjoin%
\definecolor{currentfill}{rgb}{0.272594,0.025563,0.353093}%
\pgfsetfillcolor{currentfill}%
\pgfsetfillopacity{0.700000}%
\pgfsetlinewidth{0.000000pt}%
\definecolor{currentstroke}{rgb}{0.000000,0.000000,0.000000}%
\pgfsetstrokecolor{currentstroke}%
\pgfsetdash{}{0pt}%
\pgfpathmoveto{\pgfqpoint{3.586078in}{1.554530in}}%
\pgfpathlineto{\pgfqpoint{3.599443in}{1.552970in}}%
\pgfpathlineto{\pgfqpoint{3.612814in}{1.551576in}}%
\pgfpathlineto{\pgfqpoint{3.626191in}{1.550350in}}%
\pgfpathlineto{\pgfqpoint{3.639574in}{1.549290in}}%
\pgfpathlineto{\pgfqpoint{3.647520in}{1.559341in}}%
\pgfpathlineto{\pgfqpoint{3.655461in}{1.569471in}}%
\pgfpathlineto{\pgfqpoint{3.663396in}{1.579676in}}%
\pgfpathlineto{\pgfqpoint{3.671325in}{1.589952in}}%
\pgfpathlineto{\pgfqpoint{3.657954in}{1.590571in}}%
\pgfpathlineto{\pgfqpoint{3.644590in}{1.591357in}}%
\pgfpathlineto{\pgfqpoint{3.631232in}{1.592310in}}%
\pgfpathlineto{\pgfqpoint{3.617881in}{1.593430in}}%
\pgfpathlineto{\pgfqpoint{3.609939in}{1.583583in}}%
\pgfpathlineto{\pgfqpoint{3.601992in}{1.573815in}}%
\pgfpathlineto{\pgfqpoint{3.594038in}{1.564129in}}%
\pgfpathlineto{\pgfqpoint{3.586078in}{1.554530in}}%
\pgfpathclose%
\pgfusepath{fill}%
\end{pgfscope}%
\begin{pgfscope}%
\pgfpathrectangle{\pgfqpoint{1.254980in}{0.150000in}}{\pgfqpoint{5.490039in}{5.490039in}}%
\pgfusepath{clip}%
\pgfsetbuttcap%
\pgfsetroundjoin%
\definecolor{currentfill}{rgb}{0.282910,0.105393,0.426902}%
\pgfsetfillcolor{currentfill}%
\pgfsetfillopacity{0.700000}%
\pgfsetlinewidth{0.000000pt}%
\definecolor{currentstroke}{rgb}{0.000000,0.000000,0.000000}%
\pgfsetstrokecolor{currentstroke}%
\pgfsetdash{}{0pt}%
\pgfpathmoveto{\pgfqpoint{3.841617in}{1.681651in}}%
\pgfpathlineto{\pgfqpoint{3.855037in}{1.683315in}}%
\pgfpathlineto{\pgfqpoint{3.868465in}{1.685143in}}%
\pgfpathlineto{\pgfqpoint{3.881902in}{1.687134in}}%
\pgfpathlineto{\pgfqpoint{3.895349in}{1.689287in}}%
\pgfpathlineto{\pgfqpoint{3.903203in}{1.701137in}}%
\pgfpathlineto{\pgfqpoint{3.911052in}{1.713000in}}%
\pgfpathlineto{\pgfqpoint{3.918897in}{1.724873in}}%
\pgfpathlineto{\pgfqpoint{3.926737in}{1.736753in}}%
\pgfpathlineto{\pgfqpoint{3.913297in}{1.734240in}}%
\pgfpathlineto{\pgfqpoint{3.899866in}{1.731890in}}%
\pgfpathlineto{\pgfqpoint{3.886444in}{1.729704in}}%
\pgfpathlineto{\pgfqpoint{3.873031in}{1.727681in}}%
\pgfpathlineto{\pgfqpoint{3.865185in}{1.716150in}}%
\pgfpathlineto{\pgfqpoint{3.857334in}{1.704633in}}%
\pgfpathlineto{\pgfqpoint{3.849478in}{1.693132in}}%
\pgfpathlineto{\pgfqpoint{3.841617in}{1.681651in}}%
\pgfpathclose%
\pgfusepath{fill}%
\end{pgfscope}%
\begin{pgfscope}%
\pgfpathrectangle{\pgfqpoint{1.254980in}{0.150000in}}{\pgfqpoint{5.490039in}{5.490039in}}%
\pgfusepath{clip}%
\pgfsetbuttcap%
\pgfsetroundjoin%
\definecolor{currentfill}{rgb}{0.281477,0.755203,0.432552}%
\pgfsetfillcolor{currentfill}%
\pgfsetfillopacity{0.700000}%
\pgfsetlinewidth{0.000000pt}%
\definecolor{currentstroke}{rgb}{0.000000,0.000000,0.000000}%
\pgfsetstrokecolor{currentstroke}%
\pgfsetdash{}{0pt}%
\pgfpathmoveto{\pgfqpoint{5.677550in}{3.303197in}}%
\pgfpathlineto{\pgfqpoint{5.691966in}{3.317043in}}%
\pgfpathlineto{\pgfqpoint{5.706402in}{3.331050in}}%
\pgfpathlineto{\pgfqpoint{5.720859in}{3.345218in}}%
\pgfpathlineto{\pgfqpoint{5.735337in}{3.359548in}}%
\pgfpathlineto{\pgfqpoint{5.742322in}{3.361011in}}%
\pgfpathlineto{\pgfqpoint{5.749298in}{3.362401in}}%
\pgfpathlineto{\pgfqpoint{5.756265in}{3.363722in}}%
\pgfpathlineto{\pgfqpoint{5.763223in}{3.364980in}}%
\pgfpathlineto{\pgfqpoint{5.748770in}{3.351068in}}%
\pgfpathlineto{\pgfqpoint{5.734338in}{3.337317in}}%
\pgfpathlineto{\pgfqpoint{5.719926in}{3.323725in}}%
\pgfpathlineto{\pgfqpoint{5.705536in}{3.310293in}}%
\pgfpathlineto{\pgfqpoint{5.698552in}{3.308609in}}%
\pgfpathlineto{\pgfqpoint{5.691561in}{3.306868in}}%
\pgfpathlineto{\pgfqpoint{5.684560in}{3.305065in}}%
\pgfpathlineto{\pgfqpoint{5.677550in}{3.303197in}}%
\pgfpathclose%
\pgfusepath{fill}%
\end{pgfscope}%
\begin{pgfscope}%
\pgfpathrectangle{\pgfqpoint{1.254980in}{0.150000in}}{\pgfqpoint{5.490039in}{5.490039in}}%
\pgfusepath{clip}%
\pgfsetbuttcap%
\pgfsetroundjoin%
\definecolor{currentfill}{rgb}{0.273809,0.031497,0.358853}%
\pgfsetfillcolor{currentfill}%
\pgfsetfillopacity{0.700000}%
\pgfsetlinewidth{0.000000pt}%
\definecolor{currentstroke}{rgb}{0.000000,0.000000,0.000000}%
\pgfsetstrokecolor{currentstroke}%
\pgfsetdash{}{0pt}%
\pgfpathmoveto{\pgfqpoint{3.082880in}{1.587508in}}%
\pgfpathlineto{\pgfqpoint{3.096247in}{1.578979in}}%
\pgfpathlineto{\pgfqpoint{3.109615in}{1.570635in}}%
\pgfpathlineto{\pgfqpoint{3.122983in}{1.562477in}}%
\pgfpathlineto{\pgfqpoint{3.136351in}{1.554504in}}%
\pgfpathlineto{\pgfqpoint{3.144578in}{1.558696in}}%
\pgfpathlineto{\pgfqpoint{3.152793in}{1.563096in}}%
\pgfpathlineto{\pgfqpoint{3.160998in}{1.567698in}}%
\pgfpathlineto{\pgfqpoint{3.169191in}{1.572496in}}%
\pgfpathlineto{\pgfqpoint{3.155851in}{1.579914in}}%
\pgfpathlineto{\pgfqpoint{3.142513in}{1.587517in}}%
\pgfpathlineto{\pgfqpoint{3.129175in}{1.595304in}}%
\pgfpathlineto{\pgfqpoint{3.115837in}{1.603276in}}%
\pgfpathlineto{\pgfqpoint{3.107615in}{1.599023in}}%
\pgfpathlineto{\pgfqpoint{3.099382in}{1.594973in}}%
\pgfpathlineto{\pgfqpoint{3.091137in}{1.591133in}}%
\pgfpathlineto{\pgfqpoint{3.082880in}{1.587508in}}%
\pgfpathclose%
\pgfusepath{fill}%
\end{pgfscope}%
\begin{pgfscope}%
\pgfpathrectangle{\pgfqpoint{1.254980in}{0.150000in}}{\pgfqpoint{5.490039in}{5.490039in}}%
\pgfusepath{clip}%
\pgfsetbuttcap%
\pgfsetroundjoin%
\definecolor{currentfill}{rgb}{0.206756,0.371758,0.553117}%
\pgfsetfillcolor{currentfill}%
\pgfsetfillopacity{0.700000}%
\pgfsetlinewidth{0.000000pt}%
\definecolor{currentstroke}{rgb}{0.000000,0.000000,0.000000}%
\pgfsetstrokecolor{currentstroke}%
\pgfsetdash{}{0pt}%
\pgfpathmoveto{\pgfqpoint{2.347847in}{2.335860in}}%
\pgfpathlineto{\pgfqpoint{2.361534in}{2.315254in}}%
\pgfpathlineto{\pgfqpoint{2.375209in}{2.294915in}}%
\pgfpathlineto{\pgfqpoint{2.388872in}{2.274840in}}%
\pgfpathlineto{\pgfqpoint{2.402525in}{2.255026in}}%
\pgfpathlineto{\pgfqpoint{2.411366in}{2.250376in}}%
\pgfpathlineto{\pgfqpoint{2.420185in}{2.246059in}}%
\pgfpathlineto{\pgfqpoint{2.428983in}{2.242070in}}%
\pgfpathlineto{\pgfqpoint{2.437760in}{2.238403in}}%
\pgfpathlineto{\pgfqpoint{2.424162in}{2.257593in}}%
\pgfpathlineto{\pgfqpoint{2.410554in}{2.277043in}}%
\pgfpathlineto{\pgfqpoint{2.396935in}{2.296756in}}%
\pgfpathlineto{\pgfqpoint{2.383306in}{2.316734in}}%
\pgfpathlineto{\pgfqpoint{2.374474in}{2.321014in}}%
\pgfpathlineto{\pgfqpoint{2.365621in}{2.325624in}}%
\pgfpathlineto{\pgfqpoint{2.356745in}{2.330571in}}%
\pgfpathlineto{\pgfqpoint{2.347847in}{2.335860in}}%
\pgfpathclose%
\pgfusepath{fill}%
\end{pgfscope}%
\begin{pgfscope}%
\pgfpathrectangle{\pgfqpoint{1.254980in}{0.150000in}}{\pgfqpoint{5.490039in}{5.490039in}}%
\pgfusepath{clip}%
\pgfsetbuttcap%
\pgfsetroundjoin%
\definecolor{currentfill}{rgb}{0.283197,0.115680,0.436115}%
\pgfsetfillcolor{currentfill}%
\pgfsetfillopacity{0.700000}%
\pgfsetlinewidth{0.000000pt}%
\definecolor{currentstroke}{rgb}{0.000000,0.000000,0.000000}%
\pgfsetstrokecolor{currentstroke}%
\pgfsetdash{}{0pt}%
\pgfpathmoveto{\pgfqpoint{2.835146in}{1.747291in}}%
\pgfpathlineto{\pgfqpoint{2.848575in}{1.735053in}}%
\pgfpathlineto{\pgfqpoint{2.862000in}{1.723020in}}%
\pgfpathlineto{\pgfqpoint{2.875422in}{1.711188in}}%
\pgfpathlineto{\pgfqpoint{2.888842in}{1.699558in}}%
\pgfpathlineto{\pgfqpoint{2.897258in}{1.700464in}}%
\pgfpathlineto{\pgfqpoint{2.905659in}{1.701631in}}%
\pgfpathlineto{\pgfqpoint{2.914045in}{1.703054in}}%
\pgfpathlineto{\pgfqpoint{2.922418in}{1.704725in}}%
\pgfpathlineto{\pgfqpoint{2.909036in}{1.715763in}}%
\pgfpathlineto{\pgfqpoint{2.895652in}{1.727002in}}%
\pgfpathlineto{\pgfqpoint{2.882266in}{1.738443in}}%
\pgfpathlineto{\pgfqpoint{2.868877in}{1.750087in}}%
\pgfpathlineto{\pgfqpoint{2.860467in}{1.748996in}}%
\pgfpathlineto{\pgfqpoint{2.852042in}{1.748163in}}%
\pgfpathlineto{\pgfqpoint{2.843602in}{1.747592in}}%
\pgfpathlineto{\pgfqpoint{2.835146in}{1.747291in}}%
\pgfpathclose%
\pgfusepath{fill}%
\end{pgfscope}%
\begin{pgfscope}%
\pgfpathrectangle{\pgfqpoint{1.254980in}{0.150000in}}{\pgfqpoint{5.490039in}{5.490039in}}%
\pgfusepath{clip}%
\pgfsetbuttcap%
\pgfsetroundjoin%
\definecolor{currentfill}{rgb}{0.282884,0.135920,0.453427}%
\pgfsetfillcolor{currentfill}%
\pgfsetfillopacity{0.700000}%
\pgfsetlinewidth{0.000000pt}%
\definecolor{currentstroke}{rgb}{0.000000,0.000000,0.000000}%
\pgfsetstrokecolor{currentstroke}%
\pgfsetdash{}{0pt}%
\pgfpathmoveto{\pgfqpoint{3.926737in}{1.736753in}}%
\pgfpathlineto{\pgfqpoint{3.940186in}{1.739428in}}%
\pgfpathlineto{\pgfqpoint{3.953644in}{1.742266in}}%
\pgfpathlineto{\pgfqpoint{3.967112in}{1.745267in}}%
\pgfpathlineto{\pgfqpoint{3.980590in}{1.748429in}}%
\pgfpathlineto{\pgfqpoint{3.988420in}{1.760654in}}%
\pgfpathlineto{\pgfqpoint{3.996245in}{1.772873in}}%
\pgfpathlineto{\pgfqpoint{4.004066in}{1.785081in}}%
\pgfpathlineto{\pgfqpoint{4.011882in}{1.797276in}}%
\pgfpathlineto{\pgfqpoint{3.998409in}{1.793782in}}%
\pgfpathlineto{\pgfqpoint{3.984946in}{1.790449in}}%
\pgfpathlineto{\pgfqpoint{3.971492in}{1.787279in}}%
\pgfpathlineto{\pgfqpoint{3.958049in}{1.784272in}}%
\pgfpathlineto{\pgfqpoint{3.950228in}{1.772398in}}%
\pgfpathlineto{\pgfqpoint{3.942402in}{1.760518in}}%
\pgfpathlineto{\pgfqpoint{3.934572in}{1.748635in}}%
\pgfpathlineto{\pgfqpoint{3.926737in}{1.736753in}}%
\pgfpathclose%
\pgfusepath{fill}%
\end{pgfscope}%
\begin{pgfscope}%
\pgfpathrectangle{\pgfqpoint{1.254980in}{0.150000in}}{\pgfqpoint{5.490039in}{5.490039in}}%
\pgfusepath{clip}%
\pgfsetbuttcap%
\pgfsetroundjoin%
\definecolor{currentfill}{rgb}{0.127568,0.566949,0.550556}%
\pgfsetfillcolor{currentfill}%
\pgfsetfillopacity{0.700000}%
\pgfsetlinewidth{0.000000pt}%
\definecolor{currentstroke}{rgb}{0.000000,0.000000,0.000000}%
\pgfsetstrokecolor{currentstroke}%
\pgfsetdash{}{0pt}%
\pgfpathmoveto{\pgfqpoint{5.019268in}{2.771608in}}%
\pgfpathlineto{\pgfqpoint{5.033274in}{2.783276in}}%
\pgfpathlineto{\pgfqpoint{5.047299in}{2.795106in}}%
\pgfpathlineto{\pgfqpoint{5.061341in}{2.807097in}}%
\pgfpathlineto{\pgfqpoint{5.075401in}{2.819250in}}%
\pgfpathlineto{\pgfqpoint{5.082823in}{2.826658in}}%
\pgfpathlineto{\pgfqpoint{5.090237in}{2.833933in}}%
\pgfpathlineto{\pgfqpoint{5.097643in}{2.841075in}}%
\pgfpathlineto{\pgfqpoint{5.105040in}{2.848088in}}%
\pgfpathlineto{\pgfqpoint{5.090988in}{2.836070in}}%
\pgfpathlineto{\pgfqpoint{5.076955in}{2.824213in}}%
\pgfpathlineto{\pgfqpoint{5.062939in}{2.812518in}}%
\pgfpathlineto{\pgfqpoint{5.048940in}{2.800983in}}%
\pgfpathlineto{\pgfqpoint{5.041534in}{2.793825in}}%
\pgfpathlineto{\pgfqpoint{5.034120in}{2.786544in}}%
\pgfpathlineto{\pgfqpoint{5.026698in}{2.779139in}}%
\pgfpathlineto{\pgfqpoint{5.019268in}{2.771608in}}%
\pgfpathclose%
\pgfusepath{fill}%
\end{pgfscope}%
\begin{pgfscope}%
\pgfpathrectangle{\pgfqpoint{1.254980in}{0.150000in}}{\pgfqpoint{5.490039in}{5.490039in}}%
\pgfusepath{clip}%
\pgfsetbuttcap%
\pgfsetroundjoin%
\definecolor{currentfill}{rgb}{0.139147,0.533812,0.555298}%
\pgfsetfillcolor{currentfill}%
\pgfsetfillopacity{0.700000}%
\pgfsetlinewidth{0.000000pt}%
\definecolor{currentstroke}{rgb}{0.000000,0.000000,0.000000}%
\pgfsetstrokecolor{currentstroke}%
\pgfsetdash{}{0pt}%
\pgfpathmoveto{\pgfqpoint{2.107952in}{2.775896in}}%
\pgfpathlineto{\pgfqpoint{2.121865in}{2.750063in}}%
\pgfpathlineto{\pgfqpoint{2.135760in}{2.724550in}}%
\pgfpathlineto{\pgfqpoint{2.149639in}{2.699356in}}%
\pgfpathlineto{\pgfqpoint{2.163501in}{2.674476in}}%
\pgfpathlineto{\pgfqpoint{2.172547in}{2.668046in}}%
\pgfpathlineto{\pgfqpoint{2.181569in}{2.661969in}}%
\pgfpathlineto{\pgfqpoint{2.190567in}{2.656238in}}%
\pgfpathlineto{\pgfqpoint{2.199541in}{2.650847in}}%
\pgfpathlineto{\pgfqpoint{2.185742in}{2.675109in}}%
\pgfpathlineto{\pgfqpoint{2.171927in}{2.699684in}}%
\pgfpathlineto{\pgfqpoint{2.158095in}{2.724574in}}%
\pgfpathlineto{\pgfqpoint{2.144246in}{2.749785in}}%
\pgfpathlineto{\pgfqpoint{2.135210in}{2.755783in}}%
\pgfpathlineto{\pgfqpoint{2.126149in}{2.762130in}}%
\pgfpathlineto{\pgfqpoint{2.117063in}{2.768832in}}%
\pgfpathlineto{\pgfqpoint{2.107952in}{2.775896in}}%
\pgfpathclose%
\pgfusepath{fill}%
\end{pgfscope}%
\begin{pgfscope}%
\pgfpathrectangle{\pgfqpoint{1.254980in}{0.150000in}}{\pgfqpoint{5.490039in}{5.490039in}}%
\pgfusepath{clip}%
\pgfsetbuttcap%
\pgfsetroundjoin%
\definecolor{currentfill}{rgb}{0.268510,0.009605,0.335427}%
\pgfsetfillcolor{currentfill}%
\pgfsetfillopacity{0.700000}%
\pgfsetlinewidth{0.000000pt}%
\definecolor{currentstroke}{rgb}{0.000000,0.000000,0.000000}%
\pgfsetstrokecolor{currentstroke}%
\pgfsetdash{}{0pt}%
\pgfpathmoveto{\pgfqpoint{3.500707in}{1.526883in}}%
\pgfpathlineto{\pgfqpoint{3.514066in}{1.524180in}}%
\pgfpathlineto{\pgfqpoint{3.527429in}{1.521647in}}%
\pgfpathlineto{\pgfqpoint{3.540798in}{1.519283in}}%
\pgfpathlineto{\pgfqpoint{3.554171in}{1.517086in}}%
\pgfpathlineto{\pgfqpoint{3.562158in}{1.526295in}}%
\pgfpathlineto{\pgfqpoint{3.570138in}{1.535609in}}%
\pgfpathlineto{\pgfqpoint{3.578111in}{1.545022in}}%
\pgfpathlineto{\pgfqpoint{3.586078in}{1.554530in}}%
\pgfpathlineto{\pgfqpoint{3.572719in}{1.556258in}}%
\pgfpathlineto{\pgfqpoint{3.559366in}{1.558155in}}%
\pgfpathlineto{\pgfqpoint{3.546018in}{1.560220in}}%
\pgfpathlineto{\pgfqpoint{3.532676in}{1.562454in}}%
\pgfpathlineto{\pgfqpoint{3.524694in}{1.553404in}}%
\pgfpathlineto{\pgfqpoint{3.516705in}{1.544455in}}%
\pgfpathlineto{\pgfqpoint{3.508710in}{1.535613in}}%
\pgfpathlineto{\pgfqpoint{3.500707in}{1.526883in}}%
\pgfpathclose%
\pgfusepath{fill}%
\end{pgfscope}%
\begin{pgfscope}%
\pgfpathrectangle{\pgfqpoint{1.254980in}{0.150000in}}{\pgfqpoint{5.490039in}{5.490039in}}%
\pgfusepath{clip}%
\pgfsetbuttcap%
\pgfsetroundjoin%
\definecolor{currentfill}{rgb}{0.134692,0.658636,0.517649}%
\pgfsetfillcolor{currentfill}%
\pgfsetfillopacity{0.700000}%
\pgfsetlinewidth{0.000000pt}%
\definecolor{currentstroke}{rgb}{0.000000,0.000000,0.000000}%
\pgfsetstrokecolor{currentstroke}%
\pgfsetdash{}{0pt}%
\pgfpathmoveto{\pgfqpoint{5.305979in}{3.020106in}}%
\pgfpathlineto{\pgfqpoint{5.320165in}{3.032986in}}%
\pgfpathlineto{\pgfqpoint{5.334370in}{3.046028in}}%
\pgfpathlineto{\pgfqpoint{5.348595in}{3.059230in}}%
\pgfpathlineto{\pgfqpoint{5.362840in}{3.072595in}}%
\pgfpathlineto{\pgfqpoint{5.370089in}{3.077379in}}%
\pgfpathlineto{\pgfqpoint{5.377330in}{3.082045in}}%
\pgfpathlineto{\pgfqpoint{5.384561in}{3.086595in}}%
\pgfpathlineto{\pgfqpoint{5.391783in}{3.091032in}}%
\pgfpathlineto{\pgfqpoint{5.377554in}{3.077928in}}%
\pgfpathlineto{\pgfqpoint{5.363343in}{3.064984in}}%
\pgfpathlineto{\pgfqpoint{5.349153in}{3.052202in}}%
\pgfpathlineto{\pgfqpoint{5.334981in}{3.039580in}}%
\pgfpathlineto{\pgfqpoint{5.327743in}{3.034873in}}%
\pgfpathlineto{\pgfqpoint{5.320497in}{3.030060in}}%
\pgfpathlineto{\pgfqpoint{5.313242in}{3.025139in}}%
\pgfpathlineto{\pgfqpoint{5.305979in}{3.020106in}}%
\pgfpathclose%
\pgfusepath{fill}%
\end{pgfscope}%
\begin{pgfscope}%
\pgfpathrectangle{\pgfqpoint{1.254980in}{0.150000in}}{\pgfqpoint{5.490039in}{5.490039in}}%
\pgfusepath{clip}%
\pgfsetbuttcap%
\pgfsetroundjoin%
\definecolor{currentfill}{rgb}{0.369214,0.788888,0.382914}%
\pgfsetfillcolor{currentfill}%
\pgfsetfillopacity{0.700000}%
\pgfsetlinewidth{0.000000pt}%
\definecolor{currentstroke}{rgb}{0.000000,0.000000,0.000000}%
\pgfsetstrokecolor{currentstroke}%
\pgfsetdash{}{0pt}%
\pgfpathmoveto{\pgfqpoint{5.848885in}{3.424928in}}%
\pgfpathlineto{\pgfqpoint{5.863418in}{3.439194in}}%
\pgfpathlineto{\pgfqpoint{5.877973in}{3.453620in}}%
\pgfpathlineto{\pgfqpoint{5.892549in}{3.468208in}}%
\pgfpathlineto{\pgfqpoint{5.899414in}{3.468400in}}%
\pgfpathlineto{\pgfqpoint{5.906270in}{3.468545in}}%
\pgfpathlineto{\pgfqpoint{5.913118in}{3.468649in}}%
\pgfpathlineto{\pgfqpoint{5.919957in}{3.468715in}}%
\pgfpathlineto{\pgfqpoint{5.905410in}{3.454609in}}%
\pgfpathlineto{\pgfqpoint{5.890886in}{3.440661in}}%
\pgfpathlineto{\pgfqpoint{5.876382in}{3.426874in}}%
\pgfpathlineto{\pgfqpoint{5.869520in}{3.426440in}}%
\pgfpathlineto{\pgfqpoint{5.862650in}{3.425974in}}%
\pgfpathlineto{\pgfqpoint{5.855772in}{3.425472in}}%
\pgfpathlineto{\pgfqpoint{5.848885in}{3.424928in}}%
\pgfpathclose%
\pgfusepath{fill}%
\end{pgfscope}%
\begin{pgfscope}%
\pgfpathrectangle{\pgfqpoint{1.254980in}{0.150000in}}{\pgfqpoint{5.490039in}{5.490039in}}%
\pgfusepath{clip}%
\pgfsetbuttcap%
\pgfsetroundjoin%
\definecolor{currentfill}{rgb}{0.327796,0.773980,0.406640}%
\pgfsetfillcolor{currentfill}%
\pgfsetfillopacity{0.700000}%
\pgfsetlinewidth{0.000000pt}%
\definecolor{currentstroke}{rgb}{0.000000,0.000000,0.000000}%
\pgfsetstrokecolor{currentstroke}%
\pgfsetdash{}{0pt}%
\pgfpathmoveto{\pgfqpoint{5.763223in}{3.364980in}}%
\pgfpathlineto{\pgfqpoint{5.777698in}{3.379052in}}%
\pgfpathlineto{\pgfqpoint{5.792193in}{3.393285in}}%
\pgfpathlineto{\pgfqpoint{5.806711in}{3.407679in}}%
\pgfpathlineto{\pgfqpoint{5.821250in}{3.422234in}}%
\pgfpathlineto{\pgfqpoint{5.828172in}{3.422995in}}%
\pgfpathlineto{\pgfqpoint{5.835085in}{3.423694in}}%
\pgfpathlineto{\pgfqpoint{5.841989in}{3.424337in}}%
\pgfpathlineto{\pgfqpoint{5.848885in}{3.424928in}}%
\pgfpathlineto{\pgfqpoint{5.834374in}{3.410822in}}%
\pgfpathlineto{\pgfqpoint{5.819884in}{3.396877in}}%
\pgfpathlineto{\pgfqpoint{5.805415in}{3.383091in}}%
\pgfpathlineto{\pgfqpoint{5.790968in}{3.369465in}}%
\pgfpathlineto{\pgfqpoint{5.784044in}{3.368416in}}%
\pgfpathlineto{\pgfqpoint{5.777112in}{3.367322in}}%
\pgfpathlineto{\pgfqpoint{5.770172in}{3.366178in}}%
\pgfpathlineto{\pgfqpoint{5.763223in}{3.364980in}}%
\pgfpathclose%
\pgfusepath{fill}%
\end{pgfscope}%
\begin{pgfscope}%
\pgfpathrectangle{\pgfqpoint{1.254980in}{0.150000in}}{\pgfqpoint{5.490039in}{5.490039in}}%
\pgfusepath{clip}%
\pgfsetbuttcap%
\pgfsetroundjoin%
\definecolor{currentfill}{rgb}{0.257322,0.256130,0.526563}%
\pgfsetfillcolor{currentfill}%
\pgfsetfillopacity{0.700000}%
\pgfsetlinewidth{0.000000pt}%
\definecolor{currentstroke}{rgb}{0.000000,0.000000,0.000000}%
\pgfsetstrokecolor{currentstroke}%
\pgfsetdash{}{0pt}%
\pgfpathmoveto{\pgfqpoint{4.213401in}{1.982053in}}%
\pgfpathlineto{\pgfqpoint{4.226964in}{1.987834in}}%
\pgfpathlineto{\pgfqpoint{4.240539in}{1.993775in}}%
\pgfpathlineto{\pgfqpoint{4.254126in}{1.999878in}}%
\pgfpathlineto{\pgfqpoint{4.267725in}{2.006141in}}%
\pgfpathlineto{\pgfqpoint{4.275473in}{2.018625in}}%
\pgfpathlineto{\pgfqpoint{4.283217in}{2.031040in}}%
\pgfpathlineto{\pgfqpoint{4.290956in}{2.043386in}}%
\pgfpathlineto{\pgfqpoint{4.298690in}{2.055659in}}%
\pgfpathlineto{\pgfqpoint{4.285092in}{2.049174in}}%
\pgfpathlineto{\pgfqpoint{4.271507in}{2.042851in}}%
\pgfpathlineto{\pgfqpoint{4.257934in}{2.036688in}}%
\pgfpathlineto{\pgfqpoint{4.244373in}{2.030688in}}%
\pgfpathlineto{\pgfqpoint{4.236637in}{2.018624in}}%
\pgfpathlineto{\pgfqpoint{4.228897in}{2.006496in}}%
\pgfpathlineto{\pgfqpoint{4.221151in}{1.994305in}}%
\pgfpathlineto{\pgfqpoint{4.213401in}{1.982053in}}%
\pgfpathclose%
\pgfusepath{fill}%
\end{pgfscope}%
\begin{pgfscope}%
\pgfpathrectangle{\pgfqpoint{1.254980in}{0.150000in}}{\pgfqpoint{5.490039in}{5.490039in}}%
\pgfusepath{clip}%
\pgfsetbuttcap%
\pgfsetroundjoin%
\definecolor{currentfill}{rgb}{0.218130,0.347432,0.550038}%
\pgfsetfillcolor{currentfill}%
\pgfsetfillopacity{0.700000}%
\pgfsetlinewidth{0.000000pt}%
\definecolor{currentstroke}{rgb}{0.000000,0.000000,0.000000}%
\pgfsetstrokecolor{currentstroke}%
\pgfsetdash{}{0pt}%
\pgfpathmoveto{\pgfqpoint{4.414894in}{2.180040in}}%
\pgfpathlineto{\pgfqpoint{4.428554in}{2.187688in}}%
\pgfpathlineto{\pgfqpoint{4.442228in}{2.195496in}}%
\pgfpathlineto{\pgfqpoint{4.455916in}{2.203466in}}%
\pgfpathlineto{\pgfqpoint{4.469618in}{2.211596in}}%
\pgfpathlineto{\pgfqpoint{4.477304in}{2.223445in}}%
\pgfpathlineto{\pgfqpoint{4.484985in}{2.235193in}}%
\pgfpathlineto{\pgfqpoint{4.492661in}{2.246839in}}%
\pgfpathlineto{\pgfqpoint{4.500331in}{2.258383in}}%
\pgfpathlineto{\pgfqpoint{4.486631in}{2.250117in}}%
\pgfpathlineto{\pgfqpoint{4.472944in}{2.242012in}}%
\pgfpathlineto{\pgfqpoint{4.459272in}{2.234068in}}%
\pgfpathlineto{\pgfqpoint{4.445613in}{2.226286in}}%
\pgfpathlineto{\pgfqpoint{4.437941in}{2.214866in}}%
\pgfpathlineto{\pgfqpoint{4.430264in}{2.203352in}}%
\pgfpathlineto{\pgfqpoint{4.422582in}{2.191743in}}%
\pgfpathlineto{\pgfqpoint{4.414894in}{2.180040in}}%
\pgfpathclose%
\pgfusepath{fill}%
\end{pgfscope}%
\begin{pgfscope}%
\pgfpathrectangle{\pgfqpoint{1.254980in}{0.150000in}}{\pgfqpoint{5.490039in}{5.490039in}}%
\pgfusepath{clip}%
\pgfsetbuttcap%
\pgfsetroundjoin%
\definecolor{currentfill}{rgb}{0.151918,0.500685,0.557587}%
\pgfsetfillcolor{currentfill}%
\pgfsetfillopacity{0.700000}%
\pgfsetlinewidth{0.000000pt}%
\definecolor{currentstroke}{rgb}{0.000000,0.000000,0.000000}%
\pgfsetstrokecolor{currentstroke}%
\pgfsetdash{}{0pt}%
\pgfpathmoveto{\pgfqpoint{4.817930in}{2.581557in}}%
\pgfpathlineto{\pgfqpoint{4.831817in}{2.592154in}}%
\pgfpathlineto{\pgfqpoint{4.845721in}{2.602913in}}%
\pgfpathlineto{\pgfqpoint{4.859642in}{2.613833in}}%
\pgfpathlineto{\pgfqpoint{4.873579in}{2.624915in}}%
\pgfpathlineto{\pgfqpoint{4.881106in}{2.634101in}}%
\pgfpathlineto{\pgfqpoint{4.888626in}{2.643153in}}%
\pgfpathlineto{\pgfqpoint{4.896139in}{2.652073in}}%
\pgfpathlineto{\pgfqpoint{4.903644in}{2.660860in}}%
\pgfpathlineto{\pgfqpoint{4.889711in}{2.649822in}}%
\pgfpathlineto{\pgfqpoint{4.875796in}{2.638945in}}%
\pgfpathlineto{\pgfqpoint{4.861896in}{2.628229in}}%
\pgfpathlineto{\pgfqpoint{4.848014in}{2.617674in}}%
\pgfpathlineto{\pgfqpoint{4.840504in}{2.608832in}}%
\pgfpathlineto{\pgfqpoint{4.832986in}{2.599866in}}%
\pgfpathlineto{\pgfqpoint{4.825462in}{2.590775in}}%
\pgfpathlineto{\pgfqpoint{4.817930in}{2.581557in}}%
\pgfpathclose%
\pgfusepath{fill}%
\end{pgfscope}%
\begin{pgfscope}%
\pgfpathrectangle{\pgfqpoint{1.254980in}{0.150000in}}{\pgfqpoint{5.490039in}{5.490039in}}%
\pgfusepath{clip}%
\pgfsetbuttcap%
\pgfsetroundjoin%
\definecolor{currentfill}{rgb}{0.182256,0.426184,0.557120}%
\pgfsetfillcolor{currentfill}%
\pgfsetfillopacity{0.700000}%
\pgfsetlinewidth{0.000000pt}%
\definecolor{currentstroke}{rgb}{0.000000,0.000000,0.000000}%
\pgfsetstrokecolor{currentstroke}%
\pgfsetdash{}{0pt}%
\pgfpathmoveto{\pgfqpoint{4.616423in}{2.382362in}}%
\pgfpathlineto{\pgfqpoint{4.630193in}{2.391617in}}%
\pgfpathlineto{\pgfqpoint{4.643979in}{2.401033in}}%
\pgfpathlineto{\pgfqpoint{4.657779in}{2.410611in}}%
\pgfpathlineto{\pgfqpoint{4.671596in}{2.420349in}}%
\pgfpathlineto{\pgfqpoint{4.679210in}{2.431058in}}%
\pgfpathlineto{\pgfqpoint{4.686818in}{2.441643in}}%
\pgfpathlineto{\pgfqpoint{4.694420in}{2.452107in}}%
\pgfpathlineto{\pgfqpoint{4.702015in}{2.462447in}}%
\pgfpathlineto{\pgfqpoint{4.688201in}{2.452662in}}%
\pgfpathlineto{\pgfqpoint{4.674403in}{2.443037in}}%
\pgfpathlineto{\pgfqpoint{4.660620in}{2.433573in}}%
\pgfpathlineto{\pgfqpoint{4.646852in}{2.424271in}}%
\pgfpathlineto{\pgfqpoint{4.639254in}{2.413967in}}%
\pgfpathlineto{\pgfqpoint{4.631650in}{2.403547in}}%
\pgfpathlineto{\pgfqpoint{4.624039in}{2.393012in}}%
\pgfpathlineto{\pgfqpoint{4.616423in}{2.382362in}}%
\pgfpathclose%
\pgfusepath{fill}%
\end{pgfscope}%
\begin{pgfscope}%
\pgfpathrectangle{\pgfqpoint{1.254980in}{0.150000in}}{\pgfqpoint{5.490039in}{5.490039in}}%
\pgfusepath{clip}%
\pgfsetbuttcap%
\pgfsetroundjoin%
\definecolor{currentfill}{rgb}{0.267004,0.004874,0.329415}%
\pgfsetfillcolor{currentfill}%
\pgfsetfillopacity{0.700000}%
\pgfsetlinewidth{0.000000pt}%
\definecolor{currentstroke}{rgb}{0.000000,0.000000,0.000000}%
\pgfsetstrokecolor{currentstroke}%
\pgfsetdash{}{0pt}%
\pgfpathmoveto{\pgfqpoint{3.275958in}{1.519681in}}%
\pgfpathlineto{\pgfqpoint{3.289313in}{1.513884in}}%
\pgfpathlineto{\pgfqpoint{3.302670in}{1.508263in}}%
\pgfpathlineto{\pgfqpoint{3.316030in}{1.502818in}}%
\pgfpathlineto{\pgfqpoint{3.329393in}{1.497548in}}%
\pgfpathlineto{\pgfqpoint{3.337500in}{1.504146in}}%
\pgfpathlineto{\pgfqpoint{3.345598in}{1.510909in}}%
\pgfpathlineto{\pgfqpoint{3.353687in}{1.517831in}}%
\pgfpathlineto{\pgfqpoint{3.361767in}{1.524908in}}%
\pgfpathlineto{\pgfqpoint{3.348427in}{1.529653in}}%
\pgfpathlineto{\pgfqpoint{3.335090in}{1.534574in}}%
\pgfpathlineto{\pgfqpoint{3.321755in}{1.539670in}}%
\pgfpathlineto{\pgfqpoint{3.308424in}{1.544942in}}%
\pgfpathlineto{\pgfqpoint{3.300322in}{1.538380in}}%
\pgfpathlineto{\pgfqpoint{3.292210in}{1.531979in}}%
\pgfpathlineto{\pgfqpoint{3.284089in}{1.525744in}}%
\pgfpathlineto{\pgfqpoint{3.275958in}{1.519681in}}%
\pgfpathclose%
\pgfusepath{fill}%
\end{pgfscope}%
\begin{pgfscope}%
\pgfpathrectangle{\pgfqpoint{1.254980in}{0.150000in}}{\pgfqpoint{5.490039in}{5.490039in}}%
\pgfusepath{clip}%
\pgfsetbuttcap%
\pgfsetroundjoin%
\definecolor{currentfill}{rgb}{0.282327,0.094955,0.417331}%
\pgfsetfillcolor{currentfill}%
\pgfsetfillopacity{0.700000}%
\pgfsetlinewidth{0.000000pt}%
\definecolor{currentstroke}{rgb}{0.000000,0.000000,0.000000}%
\pgfsetstrokecolor{currentstroke}%
\pgfsetdash{}{0pt}%
\pgfpathmoveto{\pgfqpoint{2.888842in}{1.699558in}}%
\pgfpathlineto{\pgfqpoint{2.902259in}{1.688128in}}%
\pgfpathlineto{\pgfqpoint{2.915673in}{1.676896in}}%
\pgfpathlineto{\pgfqpoint{2.929086in}{1.665862in}}%
\pgfpathlineto{\pgfqpoint{2.942496in}{1.655025in}}%
\pgfpathlineto{\pgfqpoint{2.950874in}{1.656533in}}%
\pgfpathlineto{\pgfqpoint{2.959238in}{1.658295in}}%
\pgfpathlineto{\pgfqpoint{2.967588in}{1.660304in}}%
\pgfpathlineto{\pgfqpoint{2.975924in}{1.662555in}}%
\pgfpathlineto{\pgfqpoint{2.962550in}{1.672802in}}%
\pgfpathlineto{\pgfqpoint{2.949174in}{1.683246in}}%
\pgfpathlineto{\pgfqpoint{2.935797in}{1.693886in}}%
\pgfpathlineto{\pgfqpoint{2.922418in}{1.704725in}}%
\pgfpathlineto{\pgfqpoint{2.914045in}{1.703054in}}%
\pgfpathlineto{\pgfqpoint{2.905659in}{1.701631in}}%
\pgfpathlineto{\pgfqpoint{2.897258in}{1.700464in}}%
\pgfpathlineto{\pgfqpoint{2.888842in}{1.699558in}}%
\pgfpathclose%
\pgfusepath{fill}%
\end{pgfscope}%
\begin{pgfscope}%
\pgfpathrectangle{\pgfqpoint{1.254980in}{0.150000in}}{\pgfqpoint{5.490039in}{5.490039in}}%
\pgfusepath{clip}%
\pgfsetbuttcap%
\pgfsetroundjoin%
\definecolor{currentfill}{rgb}{0.190631,0.407061,0.556089}%
\pgfsetfillcolor{currentfill}%
\pgfsetfillopacity{0.700000}%
\pgfsetlinewidth{0.000000pt}%
\definecolor{currentstroke}{rgb}{0.000000,0.000000,0.000000}%
\pgfsetstrokecolor{currentstroke}%
\pgfsetdash{}{0pt}%
\pgfpathmoveto{\pgfqpoint{2.292981in}{2.420993in}}%
\pgfpathlineto{\pgfqpoint{2.306716in}{2.399298in}}%
\pgfpathlineto{\pgfqpoint{2.320439in}{2.377879in}}%
\pgfpathlineto{\pgfqpoint{2.334149in}{2.356734in}}%
\pgfpathlineto{\pgfqpoint{2.347847in}{2.335860in}}%
\pgfpathlineto{\pgfqpoint{2.356745in}{2.330571in}}%
\pgfpathlineto{\pgfqpoint{2.365621in}{2.325624in}}%
\pgfpathlineto{\pgfqpoint{2.374474in}{2.321014in}}%
\pgfpathlineto{\pgfqpoint{2.383306in}{2.316734in}}%
\pgfpathlineto{\pgfqpoint{2.369665in}{2.336979in}}%
\pgfpathlineto{\pgfqpoint{2.356012in}{2.357494in}}%
\pgfpathlineto{\pgfqpoint{2.342348in}{2.378281in}}%
\pgfpathlineto{\pgfqpoint{2.328672in}{2.399343in}}%
\pgfpathlineto{\pgfqpoint{2.319783in}{2.404241in}}%
\pgfpathlineto{\pgfqpoint{2.310872in}{2.409478in}}%
\pgfpathlineto{\pgfqpoint{2.301938in}{2.415060in}}%
\pgfpathlineto{\pgfqpoint{2.292981in}{2.420993in}}%
\pgfpathclose%
\pgfusepath{fill}%
\end{pgfscope}%
\begin{pgfscope}%
\pgfpathrectangle{\pgfqpoint{1.254980in}{0.150000in}}{\pgfqpoint{5.490039in}{5.490039in}}%
\pgfusepath{clip}%
\pgfsetbuttcap%
\pgfsetroundjoin%
\definecolor{currentfill}{rgb}{0.279574,0.170599,0.479997}%
\pgfsetfillcolor{currentfill}%
\pgfsetfillopacity{0.700000}%
\pgfsetlinewidth{0.000000pt}%
\definecolor{currentstroke}{rgb}{0.000000,0.000000,0.000000}%
\pgfsetstrokecolor{currentstroke}%
\pgfsetdash{}{0pt}%
\pgfpathmoveto{\pgfqpoint{4.011882in}{1.797276in}}%
\pgfpathlineto{\pgfqpoint{4.025365in}{1.800933in}}%
\pgfpathlineto{\pgfqpoint{4.038859in}{1.804752in}}%
\pgfpathlineto{\pgfqpoint{4.052362in}{1.808732in}}%
\pgfpathlineto{\pgfqpoint{4.065877in}{1.812874in}}%
\pgfpathlineto{\pgfqpoint{4.073684in}{1.825369in}}%
\pgfpathlineto{\pgfqpoint{4.081487in}{1.837838in}}%
\pgfpathlineto{\pgfqpoint{4.089286in}{1.850279in}}%
\pgfpathlineto{\pgfqpoint{4.097080in}{1.862688in}}%
\pgfpathlineto{\pgfqpoint{4.083569in}{1.858241in}}%
\pgfpathlineto{\pgfqpoint{4.070069in}{1.853956in}}%
\pgfpathlineto{\pgfqpoint{4.056579in}{1.849833in}}%
\pgfpathlineto{\pgfqpoint{4.043100in}{1.845872in}}%
\pgfpathlineto{\pgfqpoint{4.035302in}{1.833757in}}%
\pgfpathlineto{\pgfqpoint{4.027500in}{1.821617in}}%
\pgfpathlineto{\pgfqpoint{4.019693in}{1.809456in}}%
\pgfpathlineto{\pgfqpoint{4.011882in}{1.797276in}}%
\pgfpathclose%
\pgfusepath{fill}%
\end{pgfscope}%
\begin{pgfscope}%
\pgfpathrectangle{\pgfqpoint{1.254980in}{0.150000in}}{\pgfqpoint{5.490039in}{5.490039in}}%
\pgfusepath{clip}%
\pgfsetbuttcap%
\pgfsetroundjoin%
\definecolor{currentfill}{rgb}{0.271305,0.019942,0.347269}%
\pgfsetfillcolor{currentfill}%
\pgfsetfillopacity{0.700000}%
\pgfsetlinewidth{0.000000pt}%
\definecolor{currentstroke}{rgb}{0.000000,0.000000,0.000000}%
\pgfsetstrokecolor{currentstroke}%
\pgfsetdash{}{0pt}%
\pgfpathmoveto{\pgfqpoint{3.136351in}{1.554504in}}%
\pgfpathlineto{\pgfqpoint{3.149720in}{1.546714in}}%
\pgfpathlineto{\pgfqpoint{3.163090in}{1.539106in}}%
\pgfpathlineto{\pgfqpoint{3.176461in}{1.531681in}}%
\pgfpathlineto{\pgfqpoint{3.189833in}{1.524436in}}%
\pgfpathlineto{\pgfqpoint{3.198031in}{1.529194in}}%
\pgfpathlineto{\pgfqpoint{3.206218in}{1.534152in}}%
\pgfpathlineto{\pgfqpoint{3.214395in}{1.539306in}}%
\pgfpathlineto{\pgfqpoint{3.222561in}{1.544649in}}%
\pgfpathlineto{\pgfqpoint{3.209216in}{1.551339in}}%
\pgfpathlineto{\pgfqpoint{3.195873in}{1.558209in}}%
\pgfpathlineto{\pgfqpoint{3.182531in}{1.565262in}}%
\pgfpathlineto{\pgfqpoint{3.169191in}{1.572496in}}%
\pgfpathlineto{\pgfqpoint{3.160998in}{1.567698in}}%
\pgfpathlineto{\pgfqpoint{3.152793in}{1.563096in}}%
\pgfpathlineto{\pgfqpoint{3.144578in}{1.558696in}}%
\pgfpathlineto{\pgfqpoint{3.136351in}{1.554504in}}%
\pgfpathclose%
\pgfusepath{fill}%
\end{pgfscope}%
\begin{pgfscope}%
\pgfpathrectangle{\pgfqpoint{1.254980in}{0.150000in}}{\pgfqpoint{5.490039in}{5.490039in}}%
\pgfusepath{clip}%
\pgfsetbuttcap%
\pgfsetroundjoin%
\definecolor{currentfill}{rgb}{0.267004,0.004874,0.329415}%
\pgfsetfillcolor{currentfill}%
\pgfsetfillopacity{0.700000}%
\pgfsetlinewidth{0.000000pt}%
\definecolor{currentstroke}{rgb}{0.000000,0.000000,0.000000}%
\pgfsetstrokecolor{currentstroke}%
\pgfsetdash{}{0pt}%
\pgfpathmoveto{\pgfqpoint{3.415163in}{1.507662in}}%
\pgfpathlineto{\pgfqpoint{3.428522in}{1.503782in}}%
\pgfpathlineto{\pgfqpoint{3.441884in}{1.500073in}}%
\pgfpathlineto{\pgfqpoint{3.455251in}{1.496534in}}%
\pgfpathlineto{\pgfqpoint{3.468622in}{1.493166in}}%
\pgfpathlineto{\pgfqpoint{3.476655in}{1.501405in}}%
\pgfpathlineto{\pgfqpoint{3.484680in}{1.509774in}}%
\pgfpathlineto{\pgfqpoint{3.492697in}{1.518268in}}%
\pgfpathlineto{\pgfqpoint{3.500707in}{1.526883in}}%
\pgfpathlineto{\pgfqpoint{3.487354in}{1.529755in}}%
\pgfpathlineto{\pgfqpoint{3.474005in}{1.532797in}}%
\pgfpathlineto{\pgfqpoint{3.460661in}{1.536010in}}%
\pgfpathlineto{\pgfqpoint{3.447321in}{1.539395in}}%
\pgfpathlineto{\pgfqpoint{3.439294in}{1.531265in}}%
\pgfpathlineto{\pgfqpoint{3.431258in}{1.523264in}}%
\pgfpathlineto{\pgfqpoint{3.423215in}{1.515394in}}%
\pgfpathlineto{\pgfqpoint{3.415163in}{1.507662in}}%
\pgfpathclose%
\pgfusepath{fill}%
\end{pgfscope}%
\begin{pgfscope}%
\pgfpathrectangle{\pgfqpoint{1.254980in}{0.150000in}}{\pgfqpoint{5.490039in}{5.490039in}}%
\pgfusepath{clip}%
\pgfsetbuttcap%
\pgfsetroundjoin%
\definecolor{currentfill}{rgb}{0.162016,0.687316,0.499129}%
\pgfsetfillcolor{currentfill}%
\pgfsetfillopacity{0.700000}%
\pgfsetlinewidth{0.000000pt}%
\definecolor{currentstroke}{rgb}{0.000000,0.000000,0.000000}%
\pgfsetstrokecolor{currentstroke}%
\pgfsetdash{}{0pt}%
\pgfpathmoveto{\pgfqpoint{5.391783in}{3.091032in}}%
\pgfpathlineto{\pgfqpoint{5.406033in}{3.104298in}}%
\pgfpathlineto{\pgfqpoint{5.420302in}{3.117725in}}%
\pgfpathlineto{\pgfqpoint{5.434591in}{3.131314in}}%
\pgfpathlineto{\pgfqpoint{5.448901in}{3.145065in}}%
\pgfpathlineto{\pgfqpoint{5.456098in}{3.149113in}}%
\pgfpathlineto{\pgfqpoint{5.463286in}{3.153047in}}%
\pgfpathlineto{\pgfqpoint{5.470464in}{3.156871in}}%
\pgfpathlineto{\pgfqpoint{5.477633in}{3.160588in}}%
\pgfpathlineto{\pgfqpoint{5.463341in}{3.147130in}}%
\pgfpathlineto{\pgfqpoint{5.449068in}{3.133832in}}%
\pgfpathlineto{\pgfqpoint{5.434816in}{3.120696in}}%
\pgfpathlineto{\pgfqpoint{5.420583in}{3.107720in}}%
\pgfpathlineto{\pgfqpoint{5.413396in}{3.103701in}}%
\pgfpathlineto{\pgfqpoint{5.406201in}{3.099582in}}%
\pgfpathlineto{\pgfqpoint{5.398997in}{3.095360in}}%
\pgfpathlineto{\pgfqpoint{5.391783in}{3.091032in}}%
\pgfpathclose%
\pgfusepath{fill}%
\end{pgfscope}%
\begin{pgfscope}%
\pgfpathrectangle{\pgfqpoint{1.254980in}{0.150000in}}{\pgfqpoint{5.490039in}{5.490039in}}%
\pgfusepath{clip}%
\pgfsetbuttcap%
\pgfsetroundjoin%
\definecolor{currentfill}{rgb}{0.120092,0.600104,0.542530}%
\pgfsetfillcolor{currentfill}%
\pgfsetfillopacity{0.700000}%
\pgfsetlinewidth{0.000000pt}%
\definecolor{currentstroke}{rgb}{0.000000,0.000000,0.000000}%
\pgfsetstrokecolor{currentstroke}%
\pgfsetdash{}{0pt}%
\pgfpathmoveto{\pgfqpoint{5.105040in}{2.848088in}}%
\pgfpathlineto{\pgfqpoint{5.119110in}{2.860267in}}%
\pgfpathlineto{\pgfqpoint{5.133198in}{2.872608in}}%
\pgfpathlineto{\pgfqpoint{5.147305in}{2.885110in}}%
\pgfpathlineto{\pgfqpoint{5.161430in}{2.897775in}}%
\pgfpathlineto{\pgfqpoint{5.168810in}{2.904504in}}%
\pgfpathlineto{\pgfqpoint{5.176182in}{2.911100in}}%
\pgfpathlineto{\pgfqpoint{5.183545in}{2.917564in}}%
\pgfpathlineto{\pgfqpoint{5.190899in}{2.923899in}}%
\pgfpathlineto{\pgfqpoint{5.176783in}{2.911401in}}%
\pgfpathlineto{\pgfqpoint{5.162686in}{2.899065in}}%
\pgfpathlineto{\pgfqpoint{5.148608in}{2.886890in}}%
\pgfpathlineto{\pgfqpoint{5.134548in}{2.874875in}}%
\pgfpathlineto{\pgfqpoint{5.127183in}{2.868364in}}%
\pgfpathlineto{\pgfqpoint{5.119810in}{2.861730in}}%
\pgfpathlineto{\pgfqpoint{5.112429in}{2.854972in}}%
\pgfpathlineto{\pgfqpoint{5.105040in}{2.848088in}}%
\pgfpathclose%
\pgfusepath{fill}%
\end{pgfscope}%
\begin{pgfscope}%
\pgfpathrectangle{\pgfqpoint{1.254980in}{0.150000in}}{\pgfqpoint{5.490039in}{5.490039in}}%
\pgfusepath{clip}%
\pgfsetbuttcap%
\pgfsetroundjoin%
\definecolor{currentfill}{rgb}{0.280267,0.073417,0.397163}%
\pgfsetfillcolor{currentfill}%
\pgfsetfillopacity{0.700000}%
\pgfsetlinewidth{0.000000pt}%
\definecolor{currentstroke}{rgb}{0.000000,0.000000,0.000000}%
\pgfsetstrokecolor{currentstroke}%
\pgfsetdash{}{0pt}%
\pgfpathmoveto{\pgfqpoint{2.942496in}{1.655025in}}%
\pgfpathlineto{\pgfqpoint{2.955905in}{1.644383in}}%
\pgfpathlineto{\pgfqpoint{2.969311in}{1.633935in}}%
\pgfpathlineto{\pgfqpoint{2.982717in}{1.623681in}}%
\pgfpathlineto{\pgfqpoint{2.996121in}{1.613619in}}%
\pgfpathlineto{\pgfqpoint{3.004462in}{1.615728in}}%
\pgfpathlineto{\pgfqpoint{3.012791in}{1.618082in}}%
\pgfpathlineto{\pgfqpoint{3.021106in}{1.620676in}}%
\pgfpathlineto{\pgfqpoint{3.029408in}{1.623504in}}%
\pgfpathlineto{\pgfqpoint{3.016038in}{1.632978in}}%
\pgfpathlineto{\pgfqpoint{3.002668in}{1.642644in}}%
\pgfpathlineto{\pgfqpoint{2.989297in}{1.652502in}}%
\pgfpathlineto{\pgfqpoint{2.975924in}{1.662555in}}%
\pgfpathlineto{\pgfqpoint{2.967588in}{1.660304in}}%
\pgfpathlineto{\pgfqpoint{2.959238in}{1.658295in}}%
\pgfpathlineto{\pgfqpoint{2.950874in}{1.656533in}}%
\pgfpathlineto{\pgfqpoint{2.942496in}{1.655025in}}%
\pgfpathclose%
\pgfusepath{fill}%
\end{pgfscope}%
\begin{pgfscope}%
\pgfpathrectangle{\pgfqpoint{1.254980in}{0.150000in}}{\pgfqpoint{5.490039in}{5.490039in}}%
\pgfusepath{clip}%
\pgfsetbuttcap%
\pgfsetroundjoin%
\definecolor{currentfill}{rgb}{0.241237,0.296485,0.539709}%
\pgfsetfillcolor{currentfill}%
\pgfsetfillopacity{0.700000}%
\pgfsetlinewidth{0.000000pt}%
\definecolor{currentstroke}{rgb}{0.000000,0.000000,0.000000}%
\pgfsetstrokecolor{currentstroke}%
\pgfsetdash{}{0pt}%
\pgfpathmoveto{\pgfqpoint{4.298690in}{2.055659in}}%
\pgfpathlineto{\pgfqpoint{4.312300in}{2.062305in}}%
\pgfpathlineto{\pgfqpoint{4.325924in}{2.069112in}}%
\pgfpathlineto{\pgfqpoint{4.339560in}{2.076080in}}%
\pgfpathlineto{\pgfqpoint{4.353209in}{2.083209in}}%
\pgfpathlineto{\pgfqpoint{4.360937in}{2.095613in}}%
\pgfpathlineto{\pgfqpoint{4.368660in}{2.107934in}}%
\pgfpathlineto{\pgfqpoint{4.376378in}{2.120170in}}%
\pgfpathlineto{\pgfqpoint{4.384092in}{2.132321in}}%
\pgfpathlineto{\pgfqpoint{4.370444in}{2.124999in}}%
\pgfpathlineto{\pgfqpoint{4.356809in}{2.117838in}}%
\pgfpathlineto{\pgfqpoint{4.343187in}{2.110838in}}%
\pgfpathlineto{\pgfqpoint{4.329578in}{2.104000in}}%
\pgfpathlineto{\pgfqpoint{4.321863in}{2.092031in}}%
\pgfpathlineto{\pgfqpoint{4.314144in}{2.079983in}}%
\pgfpathlineto{\pgfqpoint{4.306419in}{2.067859in}}%
\pgfpathlineto{\pgfqpoint{4.298690in}{2.055659in}}%
\pgfpathclose%
\pgfusepath{fill}%
\end{pgfscope}%
\begin{pgfscope}%
\pgfpathrectangle{\pgfqpoint{1.254980in}{0.150000in}}{\pgfqpoint{5.490039in}{5.490039in}}%
\pgfusepath{clip}%
\pgfsetbuttcap%
\pgfsetroundjoin%
\definecolor{currentfill}{rgb}{0.175841,0.441290,0.557685}%
\pgfsetfillcolor{currentfill}%
\pgfsetfillopacity{0.700000}%
\pgfsetlinewidth{0.000000pt}%
\definecolor{currentstroke}{rgb}{0.000000,0.000000,0.000000}%
\pgfsetstrokecolor{currentstroke}%
\pgfsetdash{}{0pt}%
\pgfpathmoveto{\pgfqpoint{2.237908in}{2.510586in}}%
\pgfpathlineto{\pgfqpoint{2.251696in}{2.487760in}}%
\pgfpathlineto{\pgfqpoint{2.265471in}{2.465222in}}%
\pgfpathlineto{\pgfqpoint{2.279233in}{2.442967in}}%
\pgfpathlineto{\pgfqpoint{2.292981in}{2.420993in}}%
\pgfpathlineto{\pgfqpoint{2.301938in}{2.415060in}}%
\pgfpathlineto{\pgfqpoint{2.310872in}{2.409478in}}%
\pgfpathlineto{\pgfqpoint{2.319783in}{2.404241in}}%
\pgfpathlineto{\pgfqpoint{2.328672in}{2.399343in}}%
\pgfpathlineto{\pgfqpoint{2.314983in}{2.420682in}}%
\pgfpathlineto{\pgfqpoint{2.301281in}{2.442301in}}%
\pgfpathlineto{\pgfqpoint{2.287567in}{2.464202in}}%
\pgfpathlineto{\pgfqpoint{2.273839in}{2.486388in}}%
\pgfpathlineto{\pgfqpoint{2.264892in}{2.491910in}}%
\pgfpathlineto{\pgfqpoint{2.255921in}{2.497779in}}%
\pgfpathlineto{\pgfqpoint{2.246927in}{2.504002in}}%
\pgfpathlineto{\pgfqpoint{2.237908in}{2.510586in}}%
\pgfpathclose%
\pgfusepath{fill}%
\end{pgfscope}%
\begin{pgfscope}%
\pgfpathrectangle{\pgfqpoint{1.254980in}{0.150000in}}{\pgfqpoint{5.490039in}{5.490039in}}%
\pgfusepath{clip}%
\pgfsetbuttcap%
\pgfsetroundjoin%
\definecolor{currentfill}{rgb}{0.273006,0.204520,0.501721}%
\pgfsetfillcolor{currentfill}%
\pgfsetfillopacity{0.700000}%
\pgfsetlinewidth{0.000000pt}%
\definecolor{currentstroke}{rgb}{0.000000,0.000000,0.000000}%
\pgfsetstrokecolor{currentstroke}%
\pgfsetdash{}{0pt}%
\pgfpathmoveto{\pgfqpoint{4.097080in}{1.862688in}}%
\pgfpathlineto{\pgfqpoint{4.110602in}{1.867297in}}%
\pgfpathlineto{\pgfqpoint{4.124135in}{1.872066in}}%
\pgfpathlineto{\pgfqpoint{4.137679in}{1.876997in}}%
\pgfpathlineto{\pgfqpoint{4.151234in}{1.882089in}}%
\pgfpathlineto{\pgfqpoint{4.159021in}{1.894753in}}%
\pgfpathlineto{\pgfqpoint{4.166803in}{1.907374in}}%
\pgfpathlineto{\pgfqpoint{4.174581in}{1.919949in}}%
\pgfpathlineto{\pgfqpoint{4.182354in}{1.932475in}}%
\pgfpathlineto{\pgfqpoint{4.168801in}{1.927105in}}%
\pgfpathlineto{\pgfqpoint{4.155260in}{1.921897in}}%
\pgfpathlineto{\pgfqpoint{4.141729in}{1.916850in}}%
\pgfpathlineto{\pgfqpoint{4.128210in}{1.911965in}}%
\pgfpathlineto{\pgfqpoint{4.120435in}{1.899705in}}%
\pgfpathlineto{\pgfqpoint{4.112654in}{1.887404in}}%
\pgfpathlineto{\pgfqpoint{4.104869in}{1.875064in}}%
\pgfpathlineto{\pgfqpoint{4.097080in}{1.862688in}}%
\pgfpathclose%
\pgfusepath{fill}%
\end{pgfscope}%
\begin{pgfscope}%
\pgfpathrectangle{\pgfqpoint{1.254980in}{0.150000in}}{\pgfqpoint{5.490039in}{5.490039in}}%
\pgfusepath{clip}%
\pgfsetbuttcap%
\pgfsetroundjoin%
\definecolor{currentfill}{rgb}{0.201239,0.383670,0.554294}%
\pgfsetfillcolor{currentfill}%
\pgfsetfillopacity{0.700000}%
\pgfsetlinewidth{0.000000pt}%
\definecolor{currentstroke}{rgb}{0.000000,0.000000,0.000000}%
\pgfsetstrokecolor{currentstroke}%
\pgfsetdash{}{0pt}%
\pgfpathmoveto{\pgfqpoint{4.500331in}{2.258383in}}%
\pgfpathlineto{\pgfqpoint{4.514046in}{2.266810in}}%
\pgfpathlineto{\pgfqpoint{4.527775in}{2.275398in}}%
\pgfpathlineto{\pgfqpoint{4.541519in}{2.284148in}}%
\pgfpathlineto{\pgfqpoint{4.555277in}{2.293058in}}%
\pgfpathlineto{\pgfqpoint{4.562941in}{2.304616in}}%
\pgfpathlineto{\pgfqpoint{4.570599in}{2.316062in}}%
\pgfpathlineto{\pgfqpoint{4.578251in}{2.327396in}}%
\pgfpathlineto{\pgfqpoint{4.585897in}{2.338617in}}%
\pgfpathlineto{\pgfqpoint{4.572140in}{2.329600in}}%
\pgfpathlineto{\pgfqpoint{4.558397in}{2.320744in}}%
\pgfpathlineto{\pgfqpoint{4.544670in}{2.312050in}}%
\pgfpathlineto{\pgfqpoint{4.530956in}{2.303516in}}%
\pgfpathlineto{\pgfqpoint{4.523308in}{2.292391in}}%
\pgfpathlineto{\pgfqpoint{4.515655in}{2.281159in}}%
\pgfpathlineto{\pgfqpoint{4.507996in}{2.269823in}}%
\pgfpathlineto{\pgfqpoint{4.500331in}{2.258383in}}%
\pgfpathclose%
\pgfusepath{fill}%
\end{pgfscope}%
\begin{pgfscope}%
\pgfpathrectangle{\pgfqpoint{1.254980in}{0.150000in}}{\pgfqpoint{5.490039in}{5.490039in}}%
\pgfusepath{clip}%
\pgfsetbuttcap%
\pgfsetroundjoin%
\definecolor{currentfill}{rgb}{0.139147,0.533812,0.555298}%
\pgfsetfillcolor{currentfill}%
\pgfsetfillopacity{0.700000}%
\pgfsetlinewidth{0.000000pt}%
\definecolor{currentstroke}{rgb}{0.000000,0.000000,0.000000}%
\pgfsetstrokecolor{currentstroke}%
\pgfsetdash{}{0pt}%
\pgfpathmoveto{\pgfqpoint{4.903644in}{2.660860in}}%
\pgfpathlineto{\pgfqpoint{4.917594in}{2.672060in}}%
\pgfpathlineto{\pgfqpoint{4.931560in}{2.683422in}}%
\pgfpathlineto{\pgfqpoint{4.945544in}{2.694945in}}%
\pgfpathlineto{\pgfqpoint{4.959546in}{2.706629in}}%
\pgfpathlineto{\pgfqpoint{4.967038in}{2.715224in}}%
\pgfpathlineto{\pgfqpoint{4.974523in}{2.723681in}}%
\pgfpathlineto{\pgfqpoint{4.982000in}{2.732002in}}%
\pgfpathlineto{\pgfqpoint{4.989469in}{2.740188in}}%
\pgfpathlineto{\pgfqpoint{4.975474in}{2.728577in}}%
\pgfpathlineto{\pgfqpoint{4.961495in}{2.717127in}}%
\pgfpathlineto{\pgfqpoint{4.947534in}{2.705839in}}%
\pgfpathlineto{\pgfqpoint{4.933590in}{2.694713in}}%
\pgfpathlineto{\pgfqpoint{4.926115in}{2.686442in}}%
\pgfpathlineto{\pgfqpoint{4.918632in}{2.678044in}}%
\pgfpathlineto{\pgfqpoint{4.911142in}{2.669517in}}%
\pgfpathlineto{\pgfqpoint{4.903644in}{2.660860in}}%
\pgfpathclose%
\pgfusepath{fill}%
\end{pgfscope}%
\begin{pgfscope}%
\pgfpathrectangle{\pgfqpoint{1.254980in}{0.150000in}}{\pgfqpoint{5.490039in}{5.490039in}}%
\pgfusepath{clip}%
\pgfsetbuttcap%
\pgfsetroundjoin%
\definecolor{currentfill}{rgb}{0.278791,0.062145,0.386592}%
\pgfsetfillcolor{currentfill}%
\pgfsetfillopacity{0.700000}%
\pgfsetlinewidth{0.000000pt}%
\definecolor{currentstroke}{rgb}{0.000000,0.000000,0.000000}%
\pgfsetstrokecolor{currentstroke}%
\pgfsetdash{}{0pt}%
\pgfpathmoveto{\pgfqpoint{3.724876in}{1.589134in}}%
\pgfpathlineto{\pgfqpoint{3.738282in}{1.589342in}}%
\pgfpathlineto{\pgfqpoint{3.751695in}{1.589714in}}%
\pgfpathlineto{\pgfqpoint{3.765116in}{1.590250in}}%
\pgfpathlineto{\pgfqpoint{3.778544in}{1.590950in}}%
\pgfpathlineto{\pgfqpoint{3.786446in}{1.602142in}}%
\pgfpathlineto{\pgfqpoint{3.794344in}{1.613383in}}%
\pgfpathlineto{\pgfqpoint{3.802235in}{1.624670in}}%
\pgfpathlineto{\pgfqpoint{3.810122in}{1.635998in}}%
\pgfpathlineto{\pgfqpoint{3.796702in}{1.634884in}}%
\pgfpathlineto{\pgfqpoint{3.783291in}{1.633934in}}%
\pgfpathlineto{\pgfqpoint{3.769887in}{1.633148in}}%
\pgfpathlineto{\pgfqpoint{3.756491in}{1.632526in}}%
\pgfpathlineto{\pgfqpoint{3.748596in}{1.621601in}}%
\pgfpathlineto{\pgfqpoint{3.740695in}{1.610725in}}%
\pgfpathlineto{\pgfqpoint{3.732788in}{1.599902in}}%
\pgfpathlineto{\pgfqpoint{3.724876in}{1.589134in}}%
\pgfpathclose%
\pgfusepath{fill}%
\end{pgfscope}%
\begin{pgfscope}%
\pgfpathrectangle{\pgfqpoint{1.254980in}{0.150000in}}{\pgfqpoint{5.490039in}{5.490039in}}%
\pgfusepath{clip}%
\pgfsetbuttcap%
\pgfsetroundjoin%
\definecolor{currentfill}{rgb}{0.168126,0.459988,0.558082}%
\pgfsetfillcolor{currentfill}%
\pgfsetfillopacity{0.700000}%
\pgfsetlinewidth{0.000000pt}%
\definecolor{currentstroke}{rgb}{0.000000,0.000000,0.000000}%
\pgfsetstrokecolor{currentstroke}%
\pgfsetdash{}{0pt}%
\pgfpathmoveto{\pgfqpoint{4.702015in}{2.462447in}}%
\pgfpathlineto{\pgfqpoint{4.715845in}{2.472394in}}%
\pgfpathlineto{\pgfqpoint{4.729690in}{2.482503in}}%
\pgfpathlineto{\pgfqpoint{4.743552in}{2.492773in}}%
\pgfpathlineto{\pgfqpoint{4.757429in}{2.503204in}}%
\pgfpathlineto{\pgfqpoint{4.765016in}{2.513451in}}%
\pgfpathlineto{\pgfqpoint{4.772595in}{2.523567in}}%
\pgfpathlineto{\pgfqpoint{4.780168in}{2.533555in}}%
\pgfpathlineto{\pgfqpoint{4.787734in}{2.543412in}}%
\pgfpathlineto{\pgfqpoint{4.773860in}{2.532964in}}%
\pgfpathlineto{\pgfqpoint{4.760001in}{2.522677in}}%
\pgfpathlineto{\pgfqpoint{4.746159in}{2.512551in}}%
\pgfpathlineto{\pgfqpoint{4.732332in}{2.502586in}}%
\pgfpathlineto{\pgfqpoint{4.724763in}{2.492735in}}%
\pgfpathlineto{\pgfqpoint{4.717187in}{2.482761in}}%
\pgfpathlineto{\pgfqpoint{4.709604in}{2.472665in}}%
\pgfpathlineto{\pgfqpoint{4.702015in}{2.462447in}}%
\pgfpathclose%
\pgfusepath{fill}%
\end{pgfscope}%
\begin{pgfscope}%
\pgfpathrectangle{\pgfqpoint{1.254980in}{0.150000in}}{\pgfqpoint{5.490039in}{5.490039in}}%
\pgfusepath{clip}%
\pgfsetbuttcap%
\pgfsetroundjoin%
\definecolor{currentfill}{rgb}{0.274952,0.037752,0.364543}%
\pgfsetfillcolor{currentfill}%
\pgfsetfillopacity{0.700000}%
\pgfsetlinewidth{0.000000pt}%
\definecolor{currentstroke}{rgb}{0.000000,0.000000,0.000000}%
\pgfsetstrokecolor{currentstroke}%
\pgfsetdash{}{0pt}%
\pgfpathmoveto{\pgfqpoint{3.639574in}{1.549290in}}%
\pgfpathlineto{\pgfqpoint{3.652963in}{1.548397in}}%
\pgfpathlineto{\pgfqpoint{3.666359in}{1.547668in}}%
\pgfpathlineto{\pgfqpoint{3.679762in}{1.547105in}}%
\pgfpathlineto{\pgfqpoint{3.693171in}{1.546707in}}%
\pgfpathlineto{\pgfqpoint{3.701106in}{1.557209in}}%
\pgfpathlineto{\pgfqpoint{3.709035in}{1.567784in}}%
\pgfpathlineto{\pgfqpoint{3.716959in}{1.578427in}}%
\pgfpathlineto{\pgfqpoint{3.724876in}{1.589134in}}%
\pgfpathlineto{\pgfqpoint{3.711478in}{1.589091in}}%
\pgfpathlineto{\pgfqpoint{3.698087in}{1.589212in}}%
\pgfpathlineto{\pgfqpoint{3.684702in}{1.589500in}}%
\pgfpathlineto{\pgfqpoint{3.671325in}{1.589952in}}%
\pgfpathlineto{\pgfqpoint{3.663396in}{1.579676in}}%
\pgfpathlineto{\pgfqpoint{3.655461in}{1.569471in}}%
\pgfpathlineto{\pgfqpoint{3.647520in}{1.559341in}}%
\pgfpathlineto{\pgfqpoint{3.639574in}{1.549290in}}%
\pgfpathclose%
\pgfusepath{fill}%
\end{pgfscope}%
\begin{pgfscope}%
\pgfpathrectangle{\pgfqpoint{1.254980in}{0.150000in}}{\pgfqpoint{5.490039in}{5.490039in}}%
\pgfusepath{clip}%
\pgfsetbuttcap%
\pgfsetroundjoin%
\definecolor{currentfill}{rgb}{0.202219,0.715272,0.476084}%
\pgfsetfillcolor{currentfill}%
\pgfsetfillopacity{0.700000}%
\pgfsetlinewidth{0.000000pt}%
\definecolor{currentstroke}{rgb}{0.000000,0.000000,0.000000}%
\pgfsetstrokecolor{currentstroke}%
\pgfsetdash{}{0pt}%
\pgfpathmoveto{\pgfqpoint{5.477633in}{3.160588in}}%
\pgfpathlineto{\pgfqpoint{5.491946in}{3.174208in}}%
\pgfpathlineto{\pgfqpoint{5.506279in}{3.187990in}}%
\pgfpathlineto{\pgfqpoint{5.520632in}{3.201933in}}%
\pgfpathlineto{\pgfqpoint{5.535006in}{3.216038in}}%
\pgfpathlineto{\pgfqpoint{5.542148in}{3.219339in}}%
\pgfpathlineto{\pgfqpoint{5.549281in}{3.222534in}}%
\pgfpathlineto{\pgfqpoint{5.556403in}{3.225625in}}%
\pgfpathlineto{\pgfqpoint{5.563517in}{3.228615in}}%
\pgfpathlineto{\pgfqpoint{5.549162in}{3.214835in}}%
\pgfpathlineto{\pgfqpoint{5.534827in}{3.201215in}}%
\pgfpathlineto{\pgfqpoint{5.520513in}{3.187756in}}%
\pgfpathlineto{\pgfqpoint{5.506219in}{3.174458in}}%
\pgfpathlineto{\pgfqpoint{5.499086in}{3.171133in}}%
\pgfpathlineto{\pgfqpoint{5.491944in}{3.167716in}}%
\pgfpathlineto{\pgfqpoint{5.484793in}{3.164202in}}%
\pgfpathlineto{\pgfqpoint{5.477633in}{3.160588in}}%
\pgfpathclose%
\pgfusepath{fill}%
\end{pgfscope}%
\begin{pgfscope}%
\pgfpathrectangle{\pgfqpoint{1.254980in}{0.150000in}}{\pgfqpoint{5.490039in}{5.490039in}}%
\pgfusepath{clip}%
\pgfsetbuttcap%
\pgfsetroundjoin%
\definecolor{currentfill}{rgb}{0.281924,0.089666,0.412415}%
\pgfsetfillcolor{currentfill}%
\pgfsetfillopacity{0.700000}%
\pgfsetlinewidth{0.000000pt}%
\definecolor{currentstroke}{rgb}{0.000000,0.000000,0.000000}%
\pgfsetstrokecolor{currentstroke}%
\pgfsetdash{}{0pt}%
\pgfpathmoveto{\pgfqpoint{3.810122in}{1.635998in}}%
\pgfpathlineto{\pgfqpoint{3.823550in}{1.637275in}}%
\pgfpathlineto{\pgfqpoint{3.836985in}{1.638716in}}%
\pgfpathlineto{\pgfqpoint{3.850430in}{1.640319in}}%
\pgfpathlineto{\pgfqpoint{3.863883in}{1.642086in}}%
\pgfpathlineto{\pgfqpoint{3.871757in}{1.653849in}}%
\pgfpathlineto{\pgfqpoint{3.879626in}{1.665639in}}%
\pgfpathlineto{\pgfqpoint{3.887489in}{1.677453in}}%
\pgfpathlineto{\pgfqpoint{3.895349in}{1.689287in}}%
\pgfpathlineto{\pgfqpoint{3.881902in}{1.687134in}}%
\pgfpathlineto{\pgfqpoint{3.868465in}{1.685143in}}%
\pgfpathlineto{\pgfqpoint{3.855037in}{1.683315in}}%
\pgfpathlineto{\pgfqpoint{3.841617in}{1.681651in}}%
\pgfpathlineto{\pgfqpoint{3.833751in}{1.670194in}}%
\pgfpathlineto{\pgfqpoint{3.825879in}{1.658764in}}%
\pgfpathlineto{\pgfqpoint{3.818003in}{1.647364in}}%
\pgfpathlineto{\pgfqpoint{3.810122in}{1.635998in}}%
\pgfpathclose%
\pgfusepath{fill}%
\end{pgfscope}%
\begin{pgfscope}%
\pgfpathrectangle{\pgfqpoint{1.254980in}{0.150000in}}{\pgfqpoint{5.490039in}{5.490039in}}%
\pgfusepath{clip}%
\pgfsetbuttcap%
\pgfsetroundjoin%
\definecolor{currentfill}{rgb}{0.267004,0.004874,0.329415}%
\pgfsetfillcolor{currentfill}%
\pgfsetfillopacity{0.700000}%
\pgfsetlinewidth{0.000000pt}%
\definecolor{currentstroke}{rgb}{0.000000,0.000000,0.000000}%
\pgfsetstrokecolor{currentstroke}%
\pgfsetdash{}{0pt}%
\pgfpathmoveto{\pgfqpoint{3.329393in}{1.497548in}}%
\pgfpathlineto{\pgfqpoint{3.342758in}{1.492452in}}%
\pgfpathlineto{\pgfqpoint{3.356127in}{1.487530in}}%
\pgfpathlineto{\pgfqpoint{3.369499in}{1.482781in}}%
\pgfpathlineto{\pgfqpoint{3.382874in}{1.478204in}}%
\pgfpathlineto{\pgfqpoint{3.390959in}{1.485338in}}%
\pgfpathlineto{\pgfqpoint{3.399036in}{1.492629in}}%
\pgfpathlineto{\pgfqpoint{3.407104in}{1.500072in}}%
\pgfpathlineto{\pgfqpoint{3.415163in}{1.507662in}}%
\pgfpathlineto{\pgfqpoint{3.401809in}{1.511714in}}%
\pgfpathlineto{\pgfqpoint{3.388458in}{1.515939in}}%
\pgfpathlineto{\pgfqpoint{3.375111in}{1.520336in}}%
\pgfpathlineto{\pgfqpoint{3.361767in}{1.524908in}}%
\pgfpathlineto{\pgfqpoint{3.353687in}{1.517831in}}%
\pgfpathlineto{\pgfqpoint{3.345598in}{1.510909in}}%
\pgfpathlineto{\pgfqpoint{3.337500in}{1.504146in}}%
\pgfpathlineto{\pgfqpoint{3.329393in}{1.497548in}}%
\pgfpathclose%
\pgfusepath{fill}%
\end{pgfscope}%
\begin{pgfscope}%
\pgfpathrectangle{\pgfqpoint{1.254980in}{0.150000in}}{\pgfqpoint{5.490039in}{5.490039in}}%
\pgfusepath{clip}%
\pgfsetbuttcap%
\pgfsetroundjoin%
\definecolor{currentfill}{rgb}{0.271305,0.019942,0.347269}%
\pgfsetfillcolor{currentfill}%
\pgfsetfillopacity{0.700000}%
\pgfsetlinewidth{0.000000pt}%
\definecolor{currentstroke}{rgb}{0.000000,0.000000,0.000000}%
\pgfsetstrokecolor{currentstroke}%
\pgfsetdash{}{0pt}%
\pgfpathmoveto{\pgfqpoint{3.554171in}{1.517086in}}%
\pgfpathlineto{\pgfqpoint{3.567551in}{1.515057in}}%
\pgfpathlineto{\pgfqpoint{3.580936in}{1.513196in}}%
\pgfpathlineto{\pgfqpoint{3.594326in}{1.511501in}}%
\pgfpathlineto{\pgfqpoint{3.607723in}{1.509972in}}%
\pgfpathlineto{\pgfqpoint{3.615695in}{1.519660in}}%
\pgfpathlineto{\pgfqpoint{3.623661in}{1.529446in}}%
\pgfpathlineto{\pgfqpoint{3.631621in}{1.539324in}}%
\pgfpathlineto{\pgfqpoint{3.639574in}{1.549290in}}%
\pgfpathlineto{\pgfqpoint{3.626191in}{1.550350in}}%
\pgfpathlineto{\pgfqpoint{3.612814in}{1.551576in}}%
\pgfpathlineto{\pgfqpoint{3.599443in}{1.552970in}}%
\pgfpathlineto{\pgfqpoint{3.586078in}{1.554530in}}%
\pgfpathlineto{\pgfqpoint{3.578111in}{1.545022in}}%
\pgfpathlineto{\pgfqpoint{3.570138in}{1.535609in}}%
\pgfpathlineto{\pgfqpoint{3.562158in}{1.526295in}}%
\pgfpathlineto{\pgfqpoint{3.554171in}{1.517086in}}%
\pgfpathclose%
\pgfusepath{fill}%
\end{pgfscope}%
\begin{pgfscope}%
\pgfpathrectangle{\pgfqpoint{1.254980in}{0.150000in}}{\pgfqpoint{5.490039in}{5.490039in}}%
\pgfusepath{clip}%
\pgfsetbuttcap%
\pgfsetroundjoin%
\definecolor{currentfill}{rgb}{0.277941,0.056324,0.381191}%
\pgfsetfillcolor{currentfill}%
\pgfsetfillopacity{0.700000}%
\pgfsetlinewidth{0.000000pt}%
\definecolor{currentstroke}{rgb}{0.000000,0.000000,0.000000}%
\pgfsetstrokecolor{currentstroke}%
\pgfsetdash{}{0pt}%
\pgfpathmoveto{\pgfqpoint{2.996121in}{1.613619in}}%
\pgfpathlineto{\pgfqpoint{3.009523in}{1.603749in}}%
\pgfpathlineto{\pgfqpoint{3.022925in}{1.594068in}}%
\pgfpathlineto{\pgfqpoint{3.036326in}{1.584577in}}%
\pgfpathlineto{\pgfqpoint{3.049727in}{1.575274in}}%
\pgfpathlineto{\pgfqpoint{3.058034in}{1.577981in}}%
\pgfpathlineto{\pgfqpoint{3.066329in}{1.580926in}}%
\pgfpathlineto{\pgfqpoint{3.074610in}{1.584104in}}%
\pgfpathlineto{\pgfqpoint{3.082880in}{1.587508in}}%
\pgfpathlineto{\pgfqpoint{3.069512in}{1.596224in}}%
\pgfpathlineto{\pgfqpoint{3.056144in}{1.605128in}}%
\pgfpathlineto{\pgfqpoint{3.042776in}{1.614221in}}%
\pgfpathlineto{\pgfqpoint{3.029408in}{1.623504in}}%
\pgfpathlineto{\pgfqpoint{3.021106in}{1.620676in}}%
\pgfpathlineto{\pgfqpoint{3.012791in}{1.618082in}}%
\pgfpathlineto{\pgfqpoint{3.004462in}{1.615728in}}%
\pgfpathlineto{\pgfqpoint{2.996121in}{1.613619in}}%
\pgfpathclose%
\pgfusepath{fill}%
\end{pgfscope}%
\begin{pgfscope}%
\pgfpathrectangle{\pgfqpoint{1.254980in}{0.150000in}}{\pgfqpoint{5.490039in}{5.490039in}}%
\pgfusepath{clip}%
\pgfsetbuttcap%
\pgfsetroundjoin%
\definecolor{currentfill}{rgb}{0.269944,0.014625,0.341379}%
\pgfsetfillcolor{currentfill}%
\pgfsetfillopacity{0.700000}%
\pgfsetlinewidth{0.000000pt}%
\definecolor{currentstroke}{rgb}{0.000000,0.000000,0.000000}%
\pgfsetstrokecolor{currentstroke}%
\pgfsetdash{}{0pt}%
\pgfpathmoveto{\pgfqpoint{3.189833in}{1.524436in}}%
\pgfpathlineto{\pgfqpoint{3.203207in}{1.517372in}}%
\pgfpathlineto{\pgfqpoint{3.216581in}{1.510487in}}%
\pgfpathlineto{\pgfqpoint{3.229958in}{1.503780in}}%
\pgfpathlineto{\pgfqpoint{3.243336in}{1.497252in}}%
\pgfpathlineto{\pgfqpoint{3.251507in}{1.502575in}}%
\pgfpathlineto{\pgfqpoint{3.259667in}{1.508092in}}%
\pgfpathlineto{\pgfqpoint{3.267818in}{1.513795in}}%
\pgfpathlineto{\pgfqpoint{3.275958in}{1.519681in}}%
\pgfpathlineto{\pgfqpoint{3.262606in}{1.525656in}}%
\pgfpathlineto{\pgfqpoint{3.249256in}{1.531808in}}%
\pgfpathlineto{\pgfqpoint{3.235907in}{1.538139in}}%
\pgfpathlineto{\pgfqpoint{3.222561in}{1.544649in}}%
\pgfpathlineto{\pgfqpoint{3.214395in}{1.539306in}}%
\pgfpathlineto{\pgfqpoint{3.206218in}{1.534152in}}%
\pgfpathlineto{\pgfqpoint{3.198031in}{1.529194in}}%
\pgfpathlineto{\pgfqpoint{3.189833in}{1.524436in}}%
\pgfpathclose%
\pgfusepath{fill}%
\end{pgfscope}%
\begin{pgfscope}%
\pgfpathrectangle{\pgfqpoint{1.254980in}{0.150000in}}{\pgfqpoint{5.490039in}{5.490039in}}%
\pgfusepath{clip}%
\pgfsetbuttcap%
\pgfsetroundjoin%
\definecolor{currentfill}{rgb}{0.283229,0.120777,0.440584}%
\pgfsetfillcolor{currentfill}%
\pgfsetfillopacity{0.700000}%
\pgfsetlinewidth{0.000000pt}%
\definecolor{currentstroke}{rgb}{0.000000,0.000000,0.000000}%
\pgfsetstrokecolor{currentstroke}%
\pgfsetdash{}{0pt}%
\pgfpathmoveto{\pgfqpoint{3.895349in}{1.689287in}}%
\pgfpathlineto{\pgfqpoint{3.908804in}{1.691603in}}%
\pgfpathlineto{\pgfqpoint{3.922268in}{1.694081in}}%
\pgfpathlineto{\pgfqpoint{3.935741in}{1.696722in}}%
\pgfpathlineto{\pgfqpoint{3.949224in}{1.699524in}}%
\pgfpathlineto{\pgfqpoint{3.957073in}{1.711744in}}%
\pgfpathlineto{\pgfqpoint{3.964917in}{1.723971in}}%
\pgfpathlineto{\pgfqpoint{3.972756in}{1.736200in}}%
\pgfpathlineto{\pgfqpoint{3.980590in}{1.748429in}}%
\pgfpathlineto{\pgfqpoint{3.967112in}{1.745267in}}%
\pgfpathlineto{\pgfqpoint{3.953644in}{1.742266in}}%
\pgfpathlineto{\pgfqpoint{3.940186in}{1.739428in}}%
\pgfpathlineto{\pgfqpoint{3.926737in}{1.736753in}}%
\pgfpathlineto{\pgfqpoint{3.918897in}{1.724873in}}%
\pgfpathlineto{\pgfqpoint{3.911052in}{1.713000in}}%
\pgfpathlineto{\pgfqpoint{3.903203in}{1.701137in}}%
\pgfpathlineto{\pgfqpoint{3.895349in}{1.689287in}}%
\pgfpathclose%
\pgfusepath{fill}%
\end{pgfscope}%
\begin{pgfscope}%
\pgfpathrectangle{\pgfqpoint{1.254980in}{0.150000in}}{\pgfqpoint{5.490039in}{5.490039in}}%
\pgfusepath{clip}%
\pgfsetbuttcap%
\pgfsetroundjoin%
\definecolor{currentfill}{rgb}{0.122312,0.633153,0.530398}%
\pgfsetfillcolor{currentfill}%
\pgfsetfillopacity{0.700000}%
\pgfsetlinewidth{0.000000pt}%
\definecolor{currentstroke}{rgb}{0.000000,0.000000,0.000000}%
\pgfsetstrokecolor{currentstroke}%
\pgfsetdash{}{0pt}%
\pgfpathmoveto{\pgfqpoint{5.190899in}{2.923899in}}%
\pgfpathlineto{\pgfqpoint{5.205033in}{2.936558in}}%
\pgfpathlineto{\pgfqpoint{5.219186in}{2.949380in}}%
\pgfpathlineto{\pgfqpoint{5.233359in}{2.962363in}}%
\pgfpathlineto{\pgfqpoint{5.247550in}{2.975508in}}%
\pgfpathlineto{\pgfqpoint{5.254885in}{2.981529in}}%
\pgfpathlineto{\pgfqpoint{5.262211in}{2.987417in}}%
\pgfpathlineto{\pgfqpoint{5.269528in}{2.993176in}}%
\pgfpathlineto{\pgfqpoint{5.276836in}{2.998807in}}%
\pgfpathlineto{\pgfqpoint{5.262656in}{2.985860in}}%
\pgfpathlineto{\pgfqpoint{5.248495in}{2.973075in}}%
\pgfpathlineto{\pgfqpoint{5.234353in}{2.960451in}}%
\pgfpathlineto{\pgfqpoint{5.220230in}{2.947988in}}%
\pgfpathlineto{\pgfqpoint{5.212910in}{2.942149in}}%
\pgfpathlineto{\pgfqpoint{5.205582in}{2.936189in}}%
\pgfpathlineto{\pgfqpoint{5.198245in}{2.930106in}}%
\pgfpathlineto{\pgfqpoint{5.190899in}{2.923899in}}%
\pgfpathclose%
\pgfusepath{fill}%
\end{pgfscope}%
\begin{pgfscope}%
\pgfpathrectangle{\pgfqpoint{1.254980in}{0.150000in}}{\pgfqpoint{5.490039in}{5.490039in}}%
\pgfusepath{clip}%
\pgfsetbuttcap%
\pgfsetroundjoin%
\definecolor{currentfill}{rgb}{0.260571,0.246922,0.522828}%
\pgfsetfillcolor{currentfill}%
\pgfsetfillopacity{0.700000}%
\pgfsetlinewidth{0.000000pt}%
\definecolor{currentstroke}{rgb}{0.000000,0.000000,0.000000}%
\pgfsetstrokecolor{currentstroke}%
\pgfsetdash{}{0pt}%
\pgfpathmoveto{\pgfqpoint{4.182354in}{1.932475in}}%
\pgfpathlineto{\pgfqpoint{4.195919in}{1.938006in}}%
\pgfpathlineto{\pgfqpoint{4.209496in}{1.943698in}}%
\pgfpathlineto{\pgfqpoint{4.223084in}{1.949550in}}%
\pgfpathlineto{\pgfqpoint{4.236685in}{1.955564in}}%
\pgfpathlineto{\pgfqpoint{4.244452in}{1.968300in}}%
\pgfpathlineto{\pgfqpoint{4.252214in}{1.980977in}}%
\pgfpathlineto{\pgfqpoint{4.259972in}{1.993591in}}%
\pgfpathlineto{\pgfqpoint{4.267725in}{2.006141in}}%
\pgfpathlineto{\pgfqpoint{4.254126in}{1.999878in}}%
\pgfpathlineto{\pgfqpoint{4.240539in}{1.993775in}}%
\pgfpathlineto{\pgfqpoint{4.226964in}{1.987834in}}%
\pgfpathlineto{\pgfqpoint{4.213401in}{1.982053in}}%
\pgfpathlineto{\pgfqpoint{4.205646in}{1.969742in}}%
\pgfpathlineto{\pgfqpoint{4.197887in}{1.957374in}}%
\pgfpathlineto{\pgfqpoint{4.190123in}{1.944951in}}%
\pgfpathlineto{\pgfqpoint{4.182354in}{1.932475in}}%
\pgfpathclose%
\pgfusepath{fill}%
\end{pgfscope}%
\begin{pgfscope}%
\pgfpathrectangle{\pgfqpoint{1.254980in}{0.150000in}}{\pgfqpoint{5.490039in}{5.490039in}}%
\pgfusepath{clip}%
\pgfsetbuttcap%
\pgfsetroundjoin%
\definecolor{currentfill}{rgb}{0.160665,0.478540,0.558115}%
\pgfsetfillcolor{currentfill}%
\pgfsetfillopacity{0.700000}%
\pgfsetlinewidth{0.000000pt}%
\definecolor{currentstroke}{rgb}{0.000000,0.000000,0.000000}%
\pgfsetstrokecolor{currentstroke}%
\pgfsetdash{}{0pt}%
\pgfpathmoveto{\pgfqpoint{2.182607in}{2.604809in}}%
\pgfpathlineto{\pgfqpoint{2.196455in}{2.580809in}}%
\pgfpathlineto{\pgfqpoint{2.210287in}{2.557107in}}%
\pgfpathlineto{\pgfqpoint{2.224105in}{2.533700in}}%
\pgfpathlineto{\pgfqpoint{2.237908in}{2.510586in}}%
\pgfpathlineto{\pgfqpoint{2.246927in}{2.504002in}}%
\pgfpathlineto{\pgfqpoint{2.255921in}{2.497779in}}%
\pgfpathlineto{\pgfqpoint{2.264892in}{2.491910in}}%
\pgfpathlineto{\pgfqpoint{2.273839in}{2.486388in}}%
\pgfpathlineto{\pgfqpoint{2.260098in}{2.508862in}}%
\pgfpathlineto{\pgfqpoint{2.246343in}{2.531626in}}%
\pgfpathlineto{\pgfqpoint{2.232573in}{2.554684in}}%
\pgfpathlineto{\pgfqpoint{2.218789in}{2.578038in}}%
\pgfpathlineto{\pgfqpoint{2.209781in}{2.584190in}}%
\pgfpathlineto{\pgfqpoint{2.200748in}{2.590698in}}%
\pgfpathlineto{\pgfqpoint{2.191691in}{2.597569in}}%
\pgfpathlineto{\pgfqpoint{2.182607in}{2.604809in}}%
\pgfpathclose%
\pgfusepath{fill}%
\end{pgfscope}%
\begin{pgfscope}%
\pgfpathrectangle{\pgfqpoint{1.254980in}{0.150000in}}{\pgfqpoint{5.490039in}{5.490039in}}%
\pgfusepath{clip}%
\pgfsetbuttcap%
\pgfsetroundjoin%
\definecolor{currentfill}{rgb}{0.223925,0.334994,0.548053}%
\pgfsetfillcolor{currentfill}%
\pgfsetfillopacity{0.700000}%
\pgfsetlinewidth{0.000000pt}%
\definecolor{currentstroke}{rgb}{0.000000,0.000000,0.000000}%
\pgfsetstrokecolor{currentstroke}%
\pgfsetdash{}{0pt}%
\pgfpathmoveto{\pgfqpoint{4.384092in}{2.132321in}}%
\pgfpathlineto{\pgfqpoint{4.397753in}{2.139804in}}%
\pgfpathlineto{\pgfqpoint{4.411428in}{2.147449in}}%
\pgfpathlineto{\pgfqpoint{4.425117in}{2.155254in}}%
\pgfpathlineto{\pgfqpoint{4.438819in}{2.163220in}}%
\pgfpathlineto{\pgfqpoint{4.446527in}{2.175459in}}%
\pgfpathlineto{\pgfqpoint{4.454229in}{2.187602in}}%
\pgfpathlineto{\pgfqpoint{4.461926in}{2.199649in}}%
\pgfpathlineto{\pgfqpoint{4.469618in}{2.211596in}}%
\pgfpathlineto{\pgfqpoint{4.455916in}{2.203466in}}%
\pgfpathlineto{\pgfqpoint{4.442228in}{2.195496in}}%
\pgfpathlineto{\pgfqpoint{4.428554in}{2.187688in}}%
\pgfpathlineto{\pgfqpoint{4.414894in}{2.180040in}}%
\pgfpathlineto{\pgfqpoint{4.407201in}{2.168246in}}%
\pgfpathlineto{\pgfqpoint{4.399503in}{2.156361in}}%
\pgfpathlineto{\pgfqpoint{4.391800in}{2.144385in}}%
\pgfpathlineto{\pgfqpoint{4.384092in}{2.132321in}}%
\pgfpathclose%
\pgfusepath{fill}%
\end{pgfscope}%
\begin{pgfscope}%
\pgfpathrectangle{\pgfqpoint{1.254980in}{0.150000in}}{\pgfqpoint{5.490039in}{5.490039in}}%
\pgfusepath{clip}%
\pgfsetbuttcap%
\pgfsetroundjoin%
\definecolor{currentfill}{rgb}{0.246070,0.738910,0.452024}%
\pgfsetfillcolor{currentfill}%
\pgfsetfillopacity{0.700000}%
\pgfsetlinewidth{0.000000pt}%
\definecolor{currentstroke}{rgb}{0.000000,0.000000,0.000000}%
\pgfsetstrokecolor{currentstroke}%
\pgfsetdash{}{0pt}%
\pgfpathmoveto{\pgfqpoint{5.563517in}{3.228615in}}%
\pgfpathlineto{\pgfqpoint{5.577893in}{3.242558in}}%
\pgfpathlineto{\pgfqpoint{5.592289in}{3.256661in}}%
\pgfpathlineto{\pgfqpoint{5.606706in}{3.270927in}}%
\pgfpathlineto{\pgfqpoint{5.621144in}{3.285355in}}%
\pgfpathlineto{\pgfqpoint{5.628228in}{3.287905in}}%
\pgfpathlineto{\pgfqpoint{5.635303in}{3.290356in}}%
\pgfpathlineto{\pgfqpoint{5.642367in}{3.292711in}}%
\pgfpathlineto{\pgfqpoint{5.649422in}{3.294975in}}%
\pgfpathlineto{\pgfqpoint{5.635005in}{3.280904in}}%
\pgfpathlineto{\pgfqpoint{5.620609in}{3.266994in}}%
\pgfpathlineto{\pgfqpoint{5.606234in}{3.253245in}}%
\pgfpathlineto{\pgfqpoint{5.591879in}{3.239657in}}%
\pgfpathlineto{\pgfqpoint{5.584802in}{3.237027in}}%
\pgfpathlineto{\pgfqpoint{5.577716in}{3.234313in}}%
\pgfpathlineto{\pgfqpoint{5.570621in}{3.231510in}}%
\pgfpathlineto{\pgfqpoint{5.563517in}{3.228615in}}%
\pgfpathclose%
\pgfusepath{fill}%
\end{pgfscope}%
\begin{pgfscope}%
\pgfpathrectangle{\pgfqpoint{1.254980in}{0.150000in}}{\pgfqpoint{5.490039in}{5.490039in}}%
\pgfusepath{clip}%
\pgfsetbuttcap%
\pgfsetroundjoin%
\definecolor{currentfill}{rgb}{0.267004,0.004874,0.329415}%
\pgfsetfillcolor{currentfill}%
\pgfsetfillopacity{0.700000}%
\pgfsetlinewidth{0.000000pt}%
\definecolor{currentstroke}{rgb}{0.000000,0.000000,0.000000}%
\pgfsetstrokecolor{currentstroke}%
\pgfsetdash{}{0pt}%
\pgfpathmoveto{\pgfqpoint{3.468622in}{1.493166in}}%
\pgfpathlineto{\pgfqpoint{3.481998in}{1.489967in}}%
\pgfpathlineto{\pgfqpoint{3.495378in}{1.486938in}}%
\pgfpathlineto{\pgfqpoint{3.508764in}{1.484077in}}%
\pgfpathlineto{\pgfqpoint{3.522154in}{1.481384in}}%
\pgfpathlineto{\pgfqpoint{3.530169in}{1.490130in}}%
\pgfpathlineto{\pgfqpoint{3.538177in}{1.498999in}}%
\pgfpathlineto{\pgfqpoint{3.546178in}{1.507986in}}%
\pgfpathlineto{\pgfqpoint{3.554171in}{1.517086in}}%
\pgfpathlineto{\pgfqpoint{3.540798in}{1.519283in}}%
\pgfpathlineto{\pgfqpoint{3.527429in}{1.521647in}}%
\pgfpathlineto{\pgfqpoint{3.514066in}{1.524180in}}%
\pgfpathlineto{\pgfqpoint{3.500707in}{1.526883in}}%
\pgfpathlineto{\pgfqpoint{3.492697in}{1.518268in}}%
\pgfpathlineto{\pgfqpoint{3.484680in}{1.509774in}}%
\pgfpathlineto{\pgfqpoint{3.476655in}{1.501405in}}%
\pgfpathlineto{\pgfqpoint{3.468622in}{1.493166in}}%
\pgfpathclose%
\pgfusepath{fill}%
\end{pgfscope}%
\begin{pgfscope}%
\pgfpathrectangle{\pgfqpoint{1.254980in}{0.150000in}}{\pgfqpoint{5.490039in}{5.490039in}}%
\pgfusepath{clip}%
\pgfsetbuttcap%
\pgfsetroundjoin%
\definecolor{currentfill}{rgb}{0.263663,0.237631,0.518762}%
\pgfsetfillcolor{currentfill}%
\pgfsetfillopacity{0.700000}%
\pgfsetlinewidth{0.000000pt}%
\definecolor{currentstroke}{rgb}{0.000000,0.000000,0.000000}%
\pgfsetstrokecolor{currentstroke}%
\pgfsetdash{}{0pt}%
\pgfpathmoveto{\pgfqpoint{2.584975in}{1.983578in}}%
\pgfpathlineto{\pgfqpoint{2.598532in}{1.967259in}}%
\pgfpathlineto{\pgfqpoint{2.612081in}{1.951167in}}%
\pgfpathlineto{\pgfqpoint{2.625623in}{1.935302in}}%
\pgfpathlineto{\pgfqpoint{2.639160in}{1.919660in}}%
\pgfpathlineto{\pgfqpoint{2.647820in}{1.916898in}}%
\pgfpathlineto{\pgfqpoint{2.656460in}{1.914454in}}%
\pgfpathlineto{\pgfqpoint{2.665083in}{1.912322in}}%
\pgfpathlineto{\pgfqpoint{2.673686in}{1.910495in}}%
\pgfpathlineto{\pgfqpoint{2.660199in}{1.925501in}}%
\pgfpathlineto{\pgfqpoint{2.646705in}{1.940730in}}%
\pgfpathlineto{\pgfqpoint{2.633205in}{1.956184in}}%
\pgfpathlineto{\pgfqpoint{2.619699in}{1.971865in}}%
\pgfpathlineto{\pgfqpoint{2.611047in}{1.974317in}}%
\pgfpathlineto{\pgfqpoint{2.602376in}{1.977082in}}%
\pgfpathlineto{\pgfqpoint{2.593686in}{1.980167in}}%
\pgfpathlineto{\pgfqpoint{2.584975in}{1.983578in}}%
\pgfpathclose%
\pgfusepath{fill}%
\end{pgfscope}%
\begin{pgfscope}%
\pgfpathrectangle{\pgfqpoint{1.254980in}{0.150000in}}{\pgfqpoint{5.490039in}{5.490039in}}%
\pgfusepath{clip}%
\pgfsetbuttcap%
\pgfsetroundjoin%
\definecolor{currentfill}{rgb}{0.281412,0.155834,0.469201}%
\pgfsetfillcolor{currentfill}%
\pgfsetfillopacity{0.700000}%
\pgfsetlinewidth{0.000000pt}%
\definecolor{currentstroke}{rgb}{0.000000,0.000000,0.000000}%
\pgfsetstrokecolor{currentstroke}%
\pgfsetdash{}{0pt}%
\pgfpathmoveto{\pgfqpoint{3.980590in}{1.748429in}}%
\pgfpathlineto{\pgfqpoint{3.994078in}{1.751754in}}%
\pgfpathlineto{\pgfqpoint{4.007575in}{1.755239in}}%
\pgfpathlineto{\pgfqpoint{4.021083in}{1.758887in}}%
\pgfpathlineto{\pgfqpoint{4.034601in}{1.762695in}}%
\pgfpathlineto{\pgfqpoint{4.042427in}{1.775264in}}%
\pgfpathlineto{\pgfqpoint{4.050248in}{1.787818in}}%
\pgfpathlineto{\pgfqpoint{4.058065in}{1.800356in}}%
\pgfpathlineto{\pgfqpoint{4.065877in}{1.812874in}}%
\pgfpathlineto{\pgfqpoint{4.052362in}{1.808732in}}%
\pgfpathlineto{\pgfqpoint{4.038859in}{1.804752in}}%
\pgfpathlineto{\pgfqpoint{4.025365in}{1.800933in}}%
\pgfpathlineto{\pgfqpoint{4.011882in}{1.797276in}}%
\pgfpathlineto{\pgfqpoint{4.004066in}{1.785081in}}%
\pgfpathlineto{\pgfqpoint{3.996245in}{1.772873in}}%
\pgfpathlineto{\pgfqpoint{3.988420in}{1.760654in}}%
\pgfpathlineto{\pgfqpoint{3.980590in}{1.748429in}}%
\pgfpathclose%
\pgfusepath{fill}%
\end{pgfscope}%
\begin{pgfscope}%
\pgfpathrectangle{\pgfqpoint{1.254980in}{0.150000in}}{\pgfqpoint{5.490039in}{5.490039in}}%
\pgfusepath{clip}%
\pgfsetbuttcap%
\pgfsetroundjoin%
\definecolor{currentfill}{rgb}{0.271828,0.209303,0.504434}%
\pgfsetfillcolor{currentfill}%
\pgfsetfillopacity{0.700000}%
\pgfsetlinewidth{0.000000pt}%
\definecolor{currentstroke}{rgb}{0.000000,0.000000,0.000000}%
\pgfsetstrokecolor{currentstroke}%
\pgfsetdash{}{0pt}%
\pgfpathmoveto{\pgfqpoint{2.639160in}{1.919660in}}%
\pgfpathlineto{\pgfqpoint{2.652689in}{1.904242in}}%
\pgfpathlineto{\pgfqpoint{2.666213in}{1.889044in}}%
\pgfpathlineto{\pgfqpoint{2.679731in}{1.874066in}}%
\pgfpathlineto{\pgfqpoint{2.693243in}{1.859305in}}%
\pgfpathlineto{\pgfqpoint{2.701855in}{1.857189in}}%
\pgfpathlineto{\pgfqpoint{2.710448in}{1.855382in}}%
\pgfpathlineto{\pgfqpoint{2.719024in}{1.853878in}}%
\pgfpathlineto{\pgfqpoint{2.727581in}{1.852672in}}%
\pgfpathlineto{\pgfqpoint{2.714116in}{1.866801in}}%
\pgfpathlineto{\pgfqpoint{2.700645in}{1.881146in}}%
\pgfpathlineto{\pgfqpoint{2.687168in}{1.895710in}}%
\pgfpathlineto{\pgfqpoint{2.673686in}{1.910495in}}%
\pgfpathlineto{\pgfqpoint{2.665083in}{1.912322in}}%
\pgfpathlineto{\pgfqpoint{2.656460in}{1.914454in}}%
\pgfpathlineto{\pgfqpoint{2.647820in}{1.916898in}}%
\pgfpathlineto{\pgfqpoint{2.639160in}{1.919660in}}%
\pgfpathclose%
\pgfusepath{fill}%
\end{pgfscope}%
\begin{pgfscope}%
\pgfpathrectangle{\pgfqpoint{1.254980in}{0.150000in}}{\pgfqpoint{5.490039in}{5.490039in}}%
\pgfusepath{clip}%
\pgfsetbuttcap%
\pgfsetroundjoin%
\definecolor{currentfill}{rgb}{0.185556,0.418570,0.556753}%
\pgfsetfillcolor{currentfill}%
\pgfsetfillopacity{0.700000}%
\pgfsetlinewidth{0.000000pt}%
\definecolor{currentstroke}{rgb}{0.000000,0.000000,0.000000}%
\pgfsetstrokecolor{currentstroke}%
\pgfsetdash{}{0pt}%
\pgfpathmoveto{\pgfqpoint{4.585897in}{2.338617in}}%
\pgfpathlineto{\pgfqpoint{4.599669in}{2.347795in}}%
\pgfpathlineto{\pgfqpoint{4.613456in}{2.357134in}}%
\pgfpathlineto{\pgfqpoint{4.627258in}{2.366634in}}%
\pgfpathlineto{\pgfqpoint{4.641076in}{2.376296in}}%
\pgfpathlineto{\pgfqpoint{4.648715in}{2.387492in}}%
\pgfpathlineto{\pgfqpoint{4.656348in}{2.398566in}}%
\pgfpathlineto{\pgfqpoint{4.663975in}{2.409519in}}%
\pgfpathlineto{\pgfqpoint{4.671596in}{2.420349in}}%
\pgfpathlineto{\pgfqpoint{4.657779in}{2.410611in}}%
\pgfpathlineto{\pgfqpoint{4.643979in}{2.401033in}}%
\pgfpathlineto{\pgfqpoint{4.630193in}{2.391617in}}%
\pgfpathlineto{\pgfqpoint{4.616423in}{2.382362in}}%
\pgfpathlineto{\pgfqpoint{4.608800in}{2.371598in}}%
\pgfpathlineto{\pgfqpoint{4.601172in}{2.360718in}}%
\pgfpathlineto{\pgfqpoint{4.593537in}{2.349724in}}%
\pgfpathlineto{\pgfqpoint{4.585897in}{2.338617in}}%
\pgfpathclose%
\pgfusepath{fill}%
\end{pgfscope}%
\begin{pgfscope}%
\pgfpathrectangle{\pgfqpoint{1.254980in}{0.150000in}}{\pgfqpoint{5.490039in}{5.490039in}}%
\pgfusepath{clip}%
\pgfsetbuttcap%
\pgfsetroundjoin%
\definecolor{currentfill}{rgb}{0.253935,0.265254,0.529983}%
\pgfsetfillcolor{currentfill}%
\pgfsetfillopacity{0.700000}%
\pgfsetlinewidth{0.000000pt}%
\definecolor{currentstroke}{rgb}{0.000000,0.000000,0.000000}%
\pgfsetstrokecolor{currentstroke}%
\pgfsetdash{}{0pt}%
\pgfpathmoveto{\pgfqpoint{2.530676in}{2.051163in}}%
\pgfpathlineto{\pgfqpoint{2.544263in}{2.033917in}}%
\pgfpathlineto{\pgfqpoint{2.557841in}{2.016905in}}%
\pgfpathlineto{\pgfqpoint{2.571412in}{2.000126in}}%
\pgfpathlineto{\pgfqpoint{2.584975in}{1.983578in}}%
\pgfpathlineto{\pgfqpoint{2.593686in}{1.980167in}}%
\pgfpathlineto{\pgfqpoint{2.602376in}{1.977082in}}%
\pgfpathlineto{\pgfqpoint{2.611047in}{1.974317in}}%
\pgfpathlineto{\pgfqpoint{2.619699in}{1.971865in}}%
\pgfpathlineto{\pgfqpoint{2.606185in}{1.987774in}}%
\pgfpathlineto{\pgfqpoint{2.592665in}{2.003913in}}%
\pgfpathlineto{\pgfqpoint{2.579138in}{2.020284in}}%
\pgfpathlineto{\pgfqpoint{2.565604in}{2.036888in}}%
\pgfpathlineto{\pgfqpoint{2.556902in}{2.039967in}}%
\pgfpathlineto{\pgfqpoint{2.548181in}{2.043369in}}%
\pgfpathlineto{\pgfqpoint{2.539439in}{2.047099in}}%
\pgfpathlineto{\pgfqpoint{2.530676in}{2.051163in}}%
\pgfpathclose%
\pgfusepath{fill}%
\end{pgfscope}%
\begin{pgfscope}%
\pgfpathrectangle{\pgfqpoint{1.254980in}{0.150000in}}{\pgfqpoint{5.490039in}{5.490039in}}%
\pgfusepath{clip}%
\pgfsetbuttcap%
\pgfsetroundjoin%
\definecolor{currentfill}{rgb}{0.127568,0.566949,0.550556}%
\pgfsetfillcolor{currentfill}%
\pgfsetfillopacity{0.700000}%
\pgfsetlinewidth{0.000000pt}%
\definecolor{currentstroke}{rgb}{0.000000,0.000000,0.000000}%
\pgfsetstrokecolor{currentstroke}%
\pgfsetdash{}{0pt}%
\pgfpathmoveto{\pgfqpoint{4.989469in}{2.740188in}}%
\pgfpathlineto{\pgfqpoint{5.003483in}{2.751960in}}%
\pgfpathlineto{\pgfqpoint{5.017514in}{2.763894in}}%
\pgfpathlineto{\pgfqpoint{5.031563in}{2.775990in}}%
\pgfpathlineto{\pgfqpoint{5.045630in}{2.788248in}}%
\pgfpathlineto{\pgfqpoint{5.053085in}{2.796207in}}%
\pgfpathlineto{\pgfqpoint{5.060532in}{2.804026in}}%
\pgfpathlineto{\pgfqpoint{5.067970in}{2.811706in}}%
\pgfpathlineto{\pgfqpoint{5.075401in}{2.819250in}}%
\pgfpathlineto{\pgfqpoint{5.061341in}{2.807097in}}%
\pgfpathlineto{\pgfqpoint{5.047299in}{2.795106in}}%
\pgfpathlineto{\pgfqpoint{5.033274in}{2.783276in}}%
\pgfpathlineto{\pgfqpoint{5.019268in}{2.771608in}}%
\pgfpathlineto{\pgfqpoint{5.011830in}{2.763948in}}%
\pgfpathlineto{\pgfqpoint{5.004384in}{2.756160in}}%
\pgfpathlineto{\pgfqpoint{4.996931in}{2.748240in}}%
\pgfpathlineto{\pgfqpoint{4.989469in}{2.740188in}}%
\pgfpathclose%
\pgfusepath{fill}%
\end{pgfscope}%
\begin{pgfscope}%
\pgfpathrectangle{\pgfqpoint{1.254980in}{0.150000in}}{\pgfqpoint{5.490039in}{5.490039in}}%
\pgfusepath{clip}%
\pgfsetbuttcap%
\pgfsetroundjoin%
\definecolor{currentfill}{rgb}{0.278012,0.180367,0.486697}%
\pgfsetfillcolor{currentfill}%
\pgfsetfillopacity{0.700000}%
\pgfsetlinewidth{0.000000pt}%
\definecolor{currentstroke}{rgb}{0.000000,0.000000,0.000000}%
\pgfsetstrokecolor{currentstroke}%
\pgfsetdash{}{0pt}%
\pgfpathmoveto{\pgfqpoint{2.693243in}{1.859305in}}%
\pgfpathlineto{\pgfqpoint{2.706750in}{1.844762in}}%
\pgfpathlineto{\pgfqpoint{2.720251in}{1.830433in}}%
\pgfpathlineto{\pgfqpoint{2.733748in}{1.816318in}}%
\pgfpathlineto{\pgfqpoint{2.747239in}{1.802415in}}%
\pgfpathlineto{\pgfqpoint{2.755805in}{1.800940in}}%
\pgfpathlineto{\pgfqpoint{2.764353in}{1.799767in}}%
\pgfpathlineto{\pgfqpoint{2.772883in}{1.798889in}}%
\pgfpathlineto{\pgfqpoint{2.781397in}{1.798301in}}%
\pgfpathlineto{\pgfqpoint{2.767950in}{1.811575in}}%
\pgfpathlineto{\pgfqpoint{2.754498in}{1.825061in}}%
\pgfpathlineto{\pgfqpoint{2.741042in}{1.838759in}}%
\pgfpathlineto{\pgfqpoint{2.727581in}{1.852672in}}%
\pgfpathlineto{\pgfqpoint{2.719024in}{1.853878in}}%
\pgfpathlineto{\pgfqpoint{2.710448in}{1.855382in}}%
\pgfpathlineto{\pgfqpoint{2.701855in}{1.857189in}}%
\pgfpathlineto{\pgfqpoint{2.693243in}{1.859305in}}%
\pgfpathclose%
\pgfusepath{fill}%
\end{pgfscope}%
\begin{pgfscope}%
\pgfpathrectangle{\pgfqpoint{1.254980in}{0.150000in}}{\pgfqpoint{5.490039in}{5.490039in}}%
\pgfusepath{clip}%
\pgfsetbuttcap%
\pgfsetroundjoin%
\definecolor{currentfill}{rgb}{0.153364,0.497000,0.557724}%
\pgfsetfillcolor{currentfill}%
\pgfsetfillopacity{0.700000}%
\pgfsetlinewidth{0.000000pt}%
\definecolor{currentstroke}{rgb}{0.000000,0.000000,0.000000}%
\pgfsetstrokecolor{currentstroke}%
\pgfsetdash{}{0pt}%
\pgfpathmoveto{\pgfqpoint{4.787734in}{2.543412in}}%
\pgfpathlineto{\pgfqpoint{4.801625in}{2.554022in}}%
\pgfpathlineto{\pgfqpoint{4.815533in}{2.564794in}}%
\pgfpathlineto{\pgfqpoint{4.829457in}{2.575727in}}%
\pgfpathlineto{\pgfqpoint{4.843398in}{2.586821in}}%
\pgfpathlineto{\pgfqpoint{4.850954in}{2.596548in}}%
\pgfpathlineto{\pgfqpoint{4.858503in}{2.606139in}}%
\pgfpathlineto{\pgfqpoint{4.866045in}{2.615594in}}%
\pgfpathlineto{\pgfqpoint{4.873579in}{2.624915in}}%
\pgfpathlineto{\pgfqpoint{4.859642in}{2.613833in}}%
\pgfpathlineto{\pgfqpoint{4.845721in}{2.602913in}}%
\pgfpathlineto{\pgfqpoint{4.831817in}{2.592154in}}%
\pgfpathlineto{\pgfqpoint{4.817930in}{2.581557in}}%
\pgfpathlineto{\pgfqpoint{4.810392in}{2.572213in}}%
\pgfpathlineto{\pgfqpoint{4.802846in}{2.562741in}}%
\pgfpathlineto{\pgfqpoint{4.795294in}{2.553141in}}%
\pgfpathlineto{\pgfqpoint{4.787734in}{2.543412in}}%
\pgfpathclose%
\pgfusepath{fill}%
\end{pgfscope}%
\begin{pgfscope}%
\pgfpathrectangle{\pgfqpoint{1.254980in}{0.150000in}}{\pgfqpoint{5.490039in}{5.490039in}}%
\pgfusepath{clip}%
\pgfsetbuttcap%
\pgfsetroundjoin%
\definecolor{currentfill}{rgb}{0.241237,0.296485,0.539709}%
\pgfsetfillcolor{currentfill}%
\pgfsetfillopacity{0.700000}%
\pgfsetlinewidth{0.000000pt}%
\definecolor{currentstroke}{rgb}{0.000000,0.000000,0.000000}%
\pgfsetstrokecolor{currentstroke}%
\pgfsetdash{}{0pt}%
\pgfpathmoveto{\pgfqpoint{2.476247in}{2.122530in}}%
\pgfpathlineto{\pgfqpoint{2.489867in}{2.104327in}}%
\pgfpathlineto{\pgfqpoint{2.503479in}{2.086367in}}%
\pgfpathlineto{\pgfqpoint{2.517082in}{2.068646in}}%
\pgfpathlineto{\pgfqpoint{2.530676in}{2.051163in}}%
\pgfpathlineto{\pgfqpoint{2.539439in}{2.047099in}}%
\pgfpathlineto{\pgfqpoint{2.548181in}{2.043369in}}%
\pgfpathlineto{\pgfqpoint{2.556902in}{2.039967in}}%
\pgfpathlineto{\pgfqpoint{2.565604in}{2.036888in}}%
\pgfpathlineto{\pgfqpoint{2.552061in}{2.053727in}}%
\pgfpathlineto{\pgfqpoint{2.538511in}{2.070803in}}%
\pgfpathlineto{\pgfqpoint{2.524953in}{2.088118in}}%
\pgfpathlineto{\pgfqpoint{2.511387in}{2.105674in}}%
\pgfpathlineto{\pgfqpoint{2.502634in}{2.109386in}}%
\pgfpathlineto{\pgfqpoint{2.493859in}{2.113428in}}%
\pgfpathlineto{\pgfqpoint{2.485064in}{2.117807in}}%
\pgfpathlineto{\pgfqpoint{2.476247in}{2.122530in}}%
\pgfpathclose%
\pgfusepath{fill}%
\end{pgfscope}%
\begin{pgfscope}%
\pgfpathrectangle{\pgfqpoint{1.254980in}{0.150000in}}{\pgfqpoint{5.490039in}{5.490039in}}%
\pgfusepath{clip}%
\pgfsetbuttcap%
\pgfsetroundjoin%
\definecolor{currentfill}{rgb}{0.276022,0.044167,0.370164}%
\pgfsetfillcolor{currentfill}%
\pgfsetfillopacity{0.700000}%
\pgfsetlinewidth{0.000000pt}%
\definecolor{currentstroke}{rgb}{0.000000,0.000000,0.000000}%
\pgfsetstrokecolor{currentstroke}%
\pgfsetdash{}{0pt}%
\pgfpathmoveto{\pgfqpoint{3.049727in}{1.575274in}}%
\pgfpathlineto{\pgfqpoint{3.063127in}{1.566159in}}%
\pgfpathlineto{\pgfqpoint{3.076526in}{1.557230in}}%
\pgfpathlineto{\pgfqpoint{3.089926in}{1.548486in}}%
\pgfpathlineto{\pgfqpoint{3.103326in}{1.539928in}}%
\pgfpathlineto{\pgfqpoint{3.111600in}{1.543231in}}%
\pgfpathlineto{\pgfqpoint{3.119863in}{1.546766in}}%
\pgfpathlineto{\pgfqpoint{3.128113in}{1.550525in}}%
\pgfpathlineto{\pgfqpoint{3.136351in}{1.554504in}}%
\pgfpathlineto{\pgfqpoint{3.122983in}{1.562477in}}%
\pgfpathlineto{\pgfqpoint{3.109615in}{1.570635in}}%
\pgfpathlineto{\pgfqpoint{3.096247in}{1.578979in}}%
\pgfpathlineto{\pgfqpoint{3.082880in}{1.587508in}}%
\pgfpathlineto{\pgfqpoint{3.074610in}{1.584104in}}%
\pgfpathlineto{\pgfqpoint{3.066329in}{1.580926in}}%
\pgfpathlineto{\pgfqpoint{3.058034in}{1.577981in}}%
\pgfpathlineto{\pgfqpoint{3.049727in}{1.575274in}}%
\pgfpathclose%
\pgfusepath{fill}%
\end{pgfscope}%
\begin{pgfscope}%
\pgfpathrectangle{\pgfqpoint{1.254980in}{0.150000in}}{\pgfqpoint{5.490039in}{5.490039in}}%
\pgfusepath{clip}%
\pgfsetbuttcap%
\pgfsetroundjoin%
\definecolor{currentfill}{rgb}{0.137339,0.662252,0.515571}%
\pgfsetfillcolor{currentfill}%
\pgfsetfillopacity{0.700000}%
\pgfsetlinewidth{0.000000pt}%
\definecolor{currentstroke}{rgb}{0.000000,0.000000,0.000000}%
\pgfsetstrokecolor{currentstroke}%
\pgfsetdash{}{0pt}%
\pgfpathmoveto{\pgfqpoint{5.276836in}{2.998807in}}%
\pgfpathlineto{\pgfqpoint{5.291035in}{3.011915in}}%
\pgfpathlineto{\pgfqpoint{5.305253in}{3.025186in}}%
\pgfpathlineto{\pgfqpoint{5.319491in}{3.038618in}}%
\pgfpathlineto{\pgfqpoint{5.333749in}{3.052213in}}%
\pgfpathlineto{\pgfqpoint{5.341036in}{3.057501in}}%
\pgfpathlineto{\pgfqpoint{5.348313in}{3.062658in}}%
\pgfpathlineto{\pgfqpoint{5.355581in}{3.067689in}}%
\pgfpathlineto{\pgfqpoint{5.362840in}{3.072595in}}%
\pgfpathlineto{\pgfqpoint{5.348595in}{3.059230in}}%
\pgfpathlineto{\pgfqpoint{5.334370in}{3.046028in}}%
\pgfpathlineto{\pgfqpoint{5.320165in}{3.032986in}}%
\pgfpathlineto{\pgfqpoint{5.305979in}{3.020106in}}%
\pgfpathlineto{\pgfqpoint{5.298706in}{3.014960in}}%
\pgfpathlineto{\pgfqpoint{5.291425in}{3.009696in}}%
\pgfpathlineto{\pgfqpoint{5.284135in}{3.004313in}}%
\pgfpathlineto{\pgfqpoint{5.276836in}{2.998807in}}%
\pgfpathclose%
\pgfusepath{fill}%
\end{pgfscope}%
\begin{pgfscope}%
\pgfpathrectangle{\pgfqpoint{1.254980in}{0.150000in}}{\pgfqpoint{5.490039in}{5.490039in}}%
\pgfusepath{clip}%
\pgfsetbuttcap%
\pgfsetroundjoin%
\definecolor{currentfill}{rgb}{0.296479,0.761561,0.424223}%
\pgfsetfillcolor{currentfill}%
\pgfsetfillopacity{0.700000}%
\pgfsetlinewidth{0.000000pt}%
\definecolor{currentstroke}{rgb}{0.000000,0.000000,0.000000}%
\pgfsetstrokecolor{currentstroke}%
\pgfsetdash{}{0pt}%
\pgfpathmoveto{\pgfqpoint{5.649422in}{3.294975in}}%
\pgfpathlineto{\pgfqpoint{5.663860in}{3.309207in}}%
\pgfpathlineto{\pgfqpoint{5.678320in}{3.323602in}}%
\pgfpathlineto{\pgfqpoint{5.692800in}{3.338158in}}%
\pgfpathlineto{\pgfqpoint{5.707302in}{3.352876in}}%
\pgfpathlineto{\pgfqpoint{5.714326in}{3.354675in}}%
\pgfpathlineto{\pgfqpoint{5.721339in}{3.356384in}}%
\pgfpathlineto{\pgfqpoint{5.728343in}{3.358007in}}%
\pgfpathlineto{\pgfqpoint{5.735337in}{3.359548in}}%
\pgfpathlineto{\pgfqpoint{5.720859in}{3.345218in}}%
\pgfpathlineto{\pgfqpoint{5.706402in}{3.331050in}}%
\pgfpathlineto{\pgfqpoint{5.691966in}{3.317043in}}%
\pgfpathlineto{\pgfqpoint{5.677550in}{3.303197in}}%
\pgfpathlineto{\pgfqpoint{5.670532in}{3.301258in}}%
\pgfpathlineto{\pgfqpoint{5.663505in}{3.299244in}}%
\pgfpathlineto{\pgfqpoint{5.656468in}{3.297151in}}%
\pgfpathlineto{\pgfqpoint{5.649422in}{3.294975in}}%
\pgfpathclose%
\pgfusepath{fill}%
\end{pgfscope}%
\begin{pgfscope}%
\pgfpathrectangle{\pgfqpoint{1.254980in}{0.150000in}}{\pgfqpoint{5.490039in}{5.490039in}}%
\pgfusepath{clip}%
\pgfsetbuttcap%
\pgfsetroundjoin%
\definecolor{currentfill}{rgb}{0.281412,0.155834,0.469201}%
\pgfsetfillcolor{currentfill}%
\pgfsetfillopacity{0.700000}%
\pgfsetlinewidth{0.000000pt}%
\definecolor{currentstroke}{rgb}{0.000000,0.000000,0.000000}%
\pgfsetstrokecolor{currentstroke}%
\pgfsetdash{}{0pt}%
\pgfpathmoveto{\pgfqpoint{2.747239in}{1.802415in}}%
\pgfpathlineto{\pgfqpoint{2.760726in}{1.788723in}}%
\pgfpathlineto{\pgfqpoint{2.774209in}{1.775240in}}%
\pgfpathlineto{\pgfqpoint{2.787687in}{1.761966in}}%
\pgfpathlineto{\pgfqpoint{2.801162in}{1.748898in}}%
\pgfpathlineto{\pgfqpoint{2.809683in}{1.748061in}}%
\pgfpathlineto{\pgfqpoint{2.818187in}{1.747519in}}%
\pgfpathlineto{\pgfqpoint{2.826675in}{1.747264in}}%
\pgfpathlineto{\pgfqpoint{2.835146in}{1.747291in}}%
\pgfpathlineto{\pgfqpoint{2.821714in}{1.759733in}}%
\pgfpathlineto{\pgfqpoint{2.808279in}{1.772381in}}%
\pgfpathlineto{\pgfqpoint{2.794840in}{1.785237in}}%
\pgfpathlineto{\pgfqpoint{2.781397in}{1.798301in}}%
\pgfpathlineto{\pgfqpoint{2.772883in}{1.798889in}}%
\pgfpathlineto{\pgfqpoint{2.764353in}{1.799767in}}%
\pgfpathlineto{\pgfqpoint{2.755805in}{1.800940in}}%
\pgfpathlineto{\pgfqpoint{2.747239in}{1.802415in}}%
\pgfpathclose%
\pgfusepath{fill}%
\end{pgfscope}%
\begin{pgfscope}%
\pgfpathrectangle{\pgfqpoint{1.254980in}{0.150000in}}{\pgfqpoint{5.490039in}{5.490039in}}%
\pgfusepath{clip}%
\pgfsetbuttcap%
\pgfsetroundjoin%
\definecolor{currentfill}{rgb}{0.225863,0.330805,0.547314}%
\pgfsetfillcolor{currentfill}%
\pgfsetfillopacity{0.700000}%
\pgfsetlinewidth{0.000000pt}%
\definecolor{currentstroke}{rgb}{0.000000,0.000000,0.000000}%
\pgfsetstrokecolor{currentstroke}%
\pgfsetdash{}{0pt}%
\pgfpathmoveto{\pgfqpoint{2.421672in}{2.197799in}}%
\pgfpathlineto{\pgfqpoint{2.435331in}{2.178609in}}%
\pgfpathlineto{\pgfqpoint{2.448979in}{2.159668in}}%
\pgfpathlineto{\pgfqpoint{2.462618in}{2.140976in}}%
\pgfpathlineto{\pgfqpoint{2.476247in}{2.122530in}}%
\pgfpathlineto{\pgfqpoint{2.485064in}{2.117807in}}%
\pgfpathlineto{\pgfqpoint{2.493859in}{2.113428in}}%
\pgfpathlineto{\pgfqpoint{2.502634in}{2.109386in}}%
\pgfpathlineto{\pgfqpoint{2.511387in}{2.105674in}}%
\pgfpathlineto{\pgfqpoint{2.497812in}{2.123472in}}%
\pgfpathlineto{\pgfqpoint{2.484228in}{2.141515in}}%
\pgfpathlineto{\pgfqpoint{2.470635in}{2.159805in}}%
\pgfpathlineto{\pgfqpoint{2.457033in}{2.178344in}}%
\pgfpathlineto{\pgfqpoint{2.448226in}{2.182693in}}%
\pgfpathlineto{\pgfqpoint{2.439397in}{2.187381in}}%
\pgfpathlineto{\pgfqpoint{2.430546in}{2.192414in}}%
\pgfpathlineto{\pgfqpoint{2.421672in}{2.197799in}}%
\pgfpathclose%
\pgfusepath{fill}%
\end{pgfscope}%
\begin{pgfscope}%
\pgfpathrectangle{\pgfqpoint{1.254980in}{0.150000in}}{\pgfqpoint{5.490039in}{5.490039in}}%
\pgfusepath{clip}%
\pgfsetbuttcap%
\pgfsetroundjoin%
\definecolor{currentfill}{rgb}{0.246811,0.283237,0.535941}%
\pgfsetfillcolor{currentfill}%
\pgfsetfillopacity{0.700000}%
\pgfsetlinewidth{0.000000pt}%
\definecolor{currentstroke}{rgb}{0.000000,0.000000,0.000000}%
\pgfsetstrokecolor{currentstroke}%
\pgfsetdash{}{0pt}%
\pgfpathmoveto{\pgfqpoint{4.267725in}{2.006141in}}%
\pgfpathlineto{\pgfqpoint{4.281337in}{2.012566in}}%
\pgfpathlineto{\pgfqpoint{4.294961in}{2.019151in}}%
\pgfpathlineto{\pgfqpoint{4.308598in}{2.025897in}}%
\pgfpathlineto{\pgfqpoint{4.322248in}{2.032804in}}%
\pgfpathlineto{\pgfqpoint{4.329995in}{2.045520in}}%
\pgfpathlineto{\pgfqpoint{4.337738in}{2.058161in}}%
\pgfpathlineto{\pgfqpoint{4.345476in}{2.070725in}}%
\pgfpathlineto{\pgfqpoint{4.353209in}{2.083209in}}%
\pgfpathlineto{\pgfqpoint{4.339560in}{2.076080in}}%
\pgfpathlineto{\pgfqpoint{4.325924in}{2.069112in}}%
\pgfpathlineto{\pgfqpoint{4.312300in}{2.062305in}}%
\pgfpathlineto{\pgfqpoint{4.298690in}{2.055659in}}%
\pgfpathlineto{\pgfqpoint{4.290956in}{2.043386in}}%
\pgfpathlineto{\pgfqpoint{4.283217in}{2.031040in}}%
\pgfpathlineto{\pgfqpoint{4.275473in}{2.018625in}}%
\pgfpathlineto{\pgfqpoint{4.267725in}{2.006141in}}%
\pgfpathclose%
\pgfusepath{fill}%
\end{pgfscope}%
\begin{pgfscope}%
\pgfpathrectangle{\pgfqpoint{1.254980in}{0.150000in}}{\pgfqpoint{5.490039in}{5.490039in}}%
\pgfusepath{clip}%
\pgfsetbuttcap%
\pgfsetroundjoin%
\definecolor{currentfill}{rgb}{0.267004,0.004874,0.329415}%
\pgfsetfillcolor{currentfill}%
\pgfsetfillopacity{0.700000}%
\pgfsetlinewidth{0.000000pt}%
\definecolor{currentstroke}{rgb}{0.000000,0.000000,0.000000}%
\pgfsetstrokecolor{currentstroke}%
\pgfsetdash{}{0pt}%
\pgfpathmoveto{\pgfqpoint{3.243336in}{1.497252in}}%
\pgfpathlineto{\pgfqpoint{3.256716in}{1.490901in}}%
\pgfpathlineto{\pgfqpoint{3.270098in}{1.484727in}}%
\pgfpathlineto{\pgfqpoint{3.283483in}{1.478728in}}%
\pgfpathlineto{\pgfqpoint{3.296869in}{1.472904in}}%
\pgfpathlineto{\pgfqpoint{3.305015in}{1.478792in}}%
\pgfpathlineto{\pgfqpoint{3.313150in}{1.484865in}}%
\pgfpathlineto{\pgfqpoint{3.321276in}{1.491119in}}%
\pgfpathlineto{\pgfqpoint{3.329393in}{1.497548in}}%
\pgfpathlineto{\pgfqpoint{3.316030in}{1.502818in}}%
\pgfpathlineto{\pgfqpoint{3.302670in}{1.508263in}}%
\pgfpathlineto{\pgfqpoint{3.289313in}{1.513884in}}%
\pgfpathlineto{\pgfqpoint{3.275958in}{1.519681in}}%
\pgfpathlineto{\pgfqpoint{3.267818in}{1.513795in}}%
\pgfpathlineto{\pgfqpoint{3.259667in}{1.508092in}}%
\pgfpathlineto{\pgfqpoint{3.251507in}{1.502575in}}%
\pgfpathlineto{\pgfqpoint{3.243336in}{1.497252in}}%
\pgfpathclose%
\pgfusepath{fill}%
\end{pgfscope}%
\begin{pgfscope}%
\pgfpathrectangle{\pgfqpoint{1.254980in}{0.150000in}}{\pgfqpoint{5.490039in}{5.490039in}}%
\pgfusepath{clip}%
\pgfsetbuttcap%
\pgfsetroundjoin%
\definecolor{currentfill}{rgb}{0.275191,0.194905,0.496005}%
\pgfsetfillcolor{currentfill}%
\pgfsetfillopacity{0.700000}%
\pgfsetlinewidth{0.000000pt}%
\definecolor{currentstroke}{rgb}{0.000000,0.000000,0.000000}%
\pgfsetstrokecolor{currentstroke}%
\pgfsetdash{}{0pt}%
\pgfpathmoveto{\pgfqpoint{4.065877in}{1.812874in}}%
\pgfpathlineto{\pgfqpoint{4.079402in}{1.817177in}}%
\pgfpathlineto{\pgfqpoint{4.092937in}{1.821641in}}%
\pgfpathlineto{\pgfqpoint{4.106484in}{1.826266in}}%
\pgfpathlineto{\pgfqpoint{4.120042in}{1.831052in}}%
\pgfpathlineto{\pgfqpoint{4.127847in}{1.843863in}}%
\pgfpathlineto{\pgfqpoint{4.135647in}{1.856642in}}%
\pgfpathlineto{\pgfqpoint{4.143443in}{1.869385in}}%
\pgfpathlineto{\pgfqpoint{4.151234in}{1.882089in}}%
\pgfpathlineto{\pgfqpoint{4.137679in}{1.876997in}}%
\pgfpathlineto{\pgfqpoint{4.124135in}{1.872066in}}%
\pgfpathlineto{\pgfqpoint{4.110602in}{1.867297in}}%
\pgfpathlineto{\pgfqpoint{4.097080in}{1.862688in}}%
\pgfpathlineto{\pgfqpoint{4.089286in}{1.850279in}}%
\pgfpathlineto{\pgfqpoint{4.081487in}{1.837838in}}%
\pgfpathlineto{\pgfqpoint{4.073684in}{1.825369in}}%
\pgfpathlineto{\pgfqpoint{4.065877in}{1.812874in}}%
\pgfpathclose%
\pgfusepath{fill}%
\end{pgfscope}%
\begin{pgfscope}%
\pgfpathrectangle{\pgfqpoint{1.254980in}{0.150000in}}{\pgfqpoint{5.490039in}{5.490039in}}%
\pgfusepath{clip}%
\pgfsetbuttcap%
\pgfsetroundjoin%
\definecolor{currentfill}{rgb}{0.144759,0.519093,0.556572}%
\pgfsetfillcolor{currentfill}%
\pgfsetfillopacity{0.700000}%
\pgfsetlinewidth{0.000000pt}%
\definecolor{currentstroke}{rgb}{0.000000,0.000000,0.000000}%
\pgfsetstrokecolor{currentstroke}%
\pgfsetdash{}{0pt}%
\pgfpathmoveto{\pgfqpoint{2.127060in}{2.703846in}}%
\pgfpathlineto{\pgfqpoint{2.140971in}{2.678625in}}%
\pgfpathlineto{\pgfqpoint{2.154866in}{2.653714in}}%
\pgfpathlineto{\pgfqpoint{2.168745in}{2.629109in}}%
\pgfpathlineto{\pgfqpoint{2.182607in}{2.604809in}}%
\pgfpathlineto{\pgfqpoint{2.191691in}{2.597569in}}%
\pgfpathlineto{\pgfqpoint{2.200748in}{2.590698in}}%
\pgfpathlineto{\pgfqpoint{2.209781in}{2.584190in}}%
\pgfpathlineto{\pgfqpoint{2.218789in}{2.578038in}}%
\pgfpathlineto{\pgfqpoint{2.204990in}{2.601691in}}%
\pgfpathlineto{\pgfqpoint{2.191176in}{2.625647in}}%
\pgfpathlineto{\pgfqpoint{2.177346in}{2.649907in}}%
\pgfpathlineto{\pgfqpoint{2.163501in}{2.674476in}}%
\pgfpathlineto{\pgfqpoint{2.154429in}{2.681264in}}%
\pgfpathlineto{\pgfqpoint{2.145332in}{2.688417in}}%
\pgfpathlineto{\pgfqpoint{2.136209in}{2.695943in}}%
\pgfpathlineto{\pgfqpoint{2.127060in}{2.703846in}}%
\pgfpathclose%
\pgfusepath{fill}%
\end{pgfscope}%
\begin{pgfscope}%
\pgfpathrectangle{\pgfqpoint{1.254980in}{0.150000in}}{\pgfqpoint{5.490039in}{5.490039in}}%
\pgfusepath{clip}%
\pgfsetbuttcap%
\pgfsetroundjoin%
\definecolor{currentfill}{rgb}{0.267004,0.004874,0.329415}%
\pgfsetfillcolor{currentfill}%
\pgfsetfillopacity{0.700000}%
\pgfsetlinewidth{0.000000pt}%
\definecolor{currentstroke}{rgb}{0.000000,0.000000,0.000000}%
\pgfsetstrokecolor{currentstroke}%
\pgfsetdash{}{0pt}%
\pgfpathmoveto{\pgfqpoint{3.382874in}{1.478204in}}%
\pgfpathlineto{\pgfqpoint{3.396253in}{1.473799in}}%
\pgfpathlineto{\pgfqpoint{3.409636in}{1.469566in}}%
\pgfpathlineto{\pgfqpoint{3.423022in}{1.465503in}}%
\pgfpathlineto{\pgfqpoint{3.436413in}{1.461610in}}%
\pgfpathlineto{\pgfqpoint{3.444477in}{1.469279in}}%
\pgfpathlineto{\pgfqpoint{3.452534in}{1.477098in}}%
\pgfpathlineto{\pgfqpoint{3.460582in}{1.485062in}}%
\pgfpathlineto{\pgfqpoint{3.468622in}{1.493166in}}%
\pgfpathlineto{\pgfqpoint{3.455251in}{1.496534in}}%
\pgfpathlineto{\pgfqpoint{3.441884in}{1.500073in}}%
\pgfpathlineto{\pgfqpoint{3.428522in}{1.503782in}}%
\pgfpathlineto{\pgfqpoint{3.415163in}{1.507662in}}%
\pgfpathlineto{\pgfqpoint{3.407104in}{1.500072in}}%
\pgfpathlineto{\pgfqpoint{3.399036in}{1.492629in}}%
\pgfpathlineto{\pgfqpoint{3.390959in}{1.485338in}}%
\pgfpathlineto{\pgfqpoint{3.382874in}{1.478204in}}%
\pgfpathclose%
\pgfusepath{fill}%
\end{pgfscope}%
\begin{pgfscope}%
\pgfpathrectangle{\pgfqpoint{1.254980in}{0.150000in}}{\pgfqpoint{5.490039in}{5.490039in}}%
\pgfusepath{clip}%
\pgfsetbuttcap%
\pgfsetroundjoin%
\definecolor{currentfill}{rgb}{0.283072,0.130895,0.449241}%
\pgfsetfillcolor{currentfill}%
\pgfsetfillopacity{0.700000}%
\pgfsetlinewidth{0.000000pt}%
\definecolor{currentstroke}{rgb}{0.000000,0.000000,0.000000}%
\pgfsetstrokecolor{currentstroke}%
\pgfsetdash{}{0pt}%
\pgfpathmoveto{\pgfqpoint{2.801162in}{1.748898in}}%
\pgfpathlineto{\pgfqpoint{2.814633in}{1.736035in}}%
\pgfpathlineto{\pgfqpoint{2.828100in}{1.723377in}}%
\pgfpathlineto{\pgfqpoint{2.841563in}{1.710922in}}%
\pgfpathlineto{\pgfqpoint{2.855023in}{1.698669in}}%
\pgfpathlineto{\pgfqpoint{2.863502in}{1.698468in}}%
\pgfpathlineto{\pgfqpoint{2.871964in}{1.698554in}}%
\pgfpathlineto{\pgfqpoint{2.880411in}{1.698919in}}%
\pgfpathlineto{\pgfqpoint{2.888842in}{1.699558in}}%
\pgfpathlineto{\pgfqpoint{2.875422in}{1.711188in}}%
\pgfpathlineto{\pgfqpoint{2.862000in}{1.723020in}}%
\pgfpathlineto{\pgfqpoint{2.848575in}{1.735053in}}%
\pgfpathlineto{\pgfqpoint{2.835146in}{1.747291in}}%
\pgfpathlineto{\pgfqpoint{2.826675in}{1.747264in}}%
\pgfpathlineto{\pgfqpoint{2.818187in}{1.747519in}}%
\pgfpathlineto{\pgfqpoint{2.809683in}{1.748061in}}%
\pgfpathlineto{\pgfqpoint{2.801162in}{1.748898in}}%
\pgfpathclose%
\pgfusepath{fill}%
\end{pgfscope}%
\begin{pgfscope}%
\pgfpathrectangle{\pgfqpoint{1.254980in}{0.150000in}}{\pgfqpoint{5.490039in}{5.490039in}}%
\pgfusepath{clip}%
\pgfsetbuttcap%
\pgfsetroundjoin%
\definecolor{currentfill}{rgb}{0.206756,0.371758,0.553117}%
\pgfsetfillcolor{currentfill}%
\pgfsetfillopacity{0.700000}%
\pgfsetlinewidth{0.000000pt}%
\definecolor{currentstroke}{rgb}{0.000000,0.000000,0.000000}%
\pgfsetstrokecolor{currentstroke}%
\pgfsetdash{}{0pt}%
\pgfpathmoveto{\pgfqpoint{4.469618in}{2.211596in}}%
\pgfpathlineto{\pgfqpoint{4.483334in}{2.219888in}}%
\pgfpathlineto{\pgfqpoint{4.497064in}{2.228341in}}%
\pgfpathlineto{\pgfqpoint{4.510808in}{2.236954in}}%
\pgfpathlineto{\pgfqpoint{4.524567in}{2.245729in}}%
\pgfpathlineto{\pgfqpoint{4.532253in}{2.257724in}}%
\pgfpathlineto{\pgfqpoint{4.539933in}{2.269611in}}%
\pgfpathlineto{\pgfqpoint{4.547608in}{2.281390in}}%
\pgfpathlineto{\pgfqpoint{4.555277in}{2.293058in}}%
\pgfpathlineto{\pgfqpoint{4.541519in}{2.284148in}}%
\pgfpathlineto{\pgfqpoint{4.527775in}{2.275398in}}%
\pgfpathlineto{\pgfqpoint{4.514046in}{2.266810in}}%
\pgfpathlineto{\pgfqpoint{4.500331in}{2.258383in}}%
\pgfpathlineto{\pgfqpoint{4.492661in}{2.246839in}}%
\pgfpathlineto{\pgfqpoint{4.484985in}{2.235193in}}%
\pgfpathlineto{\pgfqpoint{4.477304in}{2.223445in}}%
\pgfpathlineto{\pgfqpoint{4.469618in}{2.211596in}}%
\pgfpathclose%
\pgfusepath{fill}%
\end{pgfscope}%
\begin{pgfscope}%
\pgfpathrectangle{\pgfqpoint{1.254980in}{0.150000in}}{\pgfqpoint{5.490039in}{5.490039in}}%
\pgfusepath{clip}%
\pgfsetbuttcap%
\pgfsetroundjoin%
\definecolor{currentfill}{rgb}{0.210503,0.363727,0.552206}%
\pgfsetfillcolor{currentfill}%
\pgfsetfillopacity{0.700000}%
\pgfsetlinewidth{0.000000pt}%
\definecolor{currentstroke}{rgb}{0.000000,0.000000,0.000000}%
\pgfsetstrokecolor{currentstroke}%
\pgfsetdash{}{0pt}%
\pgfpathmoveto{\pgfqpoint{2.366936in}{2.277101in}}%
\pgfpathlineto{\pgfqpoint{2.380636in}{2.256890in}}%
\pgfpathlineto{\pgfqpoint{2.394325in}{2.236937in}}%
\pgfpathlineto{\pgfqpoint{2.408004in}{2.217241in}}%
\pgfpathlineto{\pgfqpoint{2.421672in}{2.197799in}}%
\pgfpathlineto{\pgfqpoint{2.430546in}{2.192414in}}%
\pgfpathlineto{\pgfqpoint{2.439397in}{2.187381in}}%
\pgfpathlineto{\pgfqpoint{2.448226in}{2.182693in}}%
\pgfpathlineto{\pgfqpoint{2.457033in}{2.178344in}}%
\pgfpathlineto{\pgfqpoint{2.443421in}{2.197133in}}%
\pgfpathlineto{\pgfqpoint{2.429799in}{2.216175in}}%
\pgfpathlineto{\pgfqpoint{2.416167in}{2.235472in}}%
\pgfpathlineto{\pgfqpoint{2.402525in}{2.255026in}}%
\pgfpathlineto{\pgfqpoint{2.393662in}{2.260017in}}%
\pgfpathlineto{\pgfqpoint{2.384776in}{2.265356in}}%
\pgfpathlineto{\pgfqpoint{2.375868in}{2.271048in}}%
\pgfpathlineto{\pgfqpoint{2.366936in}{2.277101in}}%
\pgfpathclose%
\pgfusepath{fill}%
\end{pgfscope}%
\begin{pgfscope}%
\pgfpathrectangle{\pgfqpoint{1.254980in}{0.150000in}}{\pgfqpoint{5.490039in}{5.490039in}}%
\pgfusepath{clip}%
\pgfsetbuttcap%
\pgfsetroundjoin%
\definecolor{currentfill}{rgb}{0.352360,0.783011,0.392636}%
\pgfsetfillcolor{currentfill}%
\pgfsetfillopacity{0.700000}%
\pgfsetlinewidth{0.000000pt}%
\definecolor{currentstroke}{rgb}{0.000000,0.000000,0.000000}%
\pgfsetstrokecolor{currentstroke}%
\pgfsetdash{}{0pt}%
\pgfpathmoveto{\pgfqpoint{5.735337in}{3.359548in}}%
\pgfpathlineto{\pgfqpoint{5.749837in}{3.374038in}}%
\pgfpathlineto{\pgfqpoint{5.764359in}{3.388691in}}%
\pgfpathlineto{\pgfqpoint{5.778902in}{3.403505in}}%
\pgfpathlineto{\pgfqpoint{5.793467in}{3.418482in}}%
\pgfpathlineto{\pgfqpoint{5.800427in}{3.419536in}}%
\pgfpathlineto{\pgfqpoint{5.807378in}{3.420509in}}%
\pgfpathlineto{\pgfqpoint{5.814318in}{3.421407in}}%
\pgfpathlineto{\pgfqpoint{5.821250in}{3.422234in}}%
\pgfpathlineto{\pgfqpoint{5.806711in}{3.407679in}}%
\pgfpathlineto{\pgfqpoint{5.792193in}{3.393285in}}%
\pgfpathlineto{\pgfqpoint{5.777698in}{3.379052in}}%
\pgfpathlineto{\pgfqpoint{5.763223in}{3.364980in}}%
\pgfpathlineto{\pgfqpoint{5.756265in}{3.363722in}}%
\pgfpathlineto{\pgfqpoint{5.749298in}{3.362401in}}%
\pgfpathlineto{\pgfqpoint{5.742322in}{3.361011in}}%
\pgfpathlineto{\pgfqpoint{5.735337in}{3.359548in}}%
\pgfpathclose%
\pgfusepath{fill}%
\end{pgfscope}%
\begin{pgfscope}%
\pgfpathrectangle{\pgfqpoint{1.254980in}{0.150000in}}{\pgfqpoint{5.490039in}{5.490039in}}%
\pgfusepath{clip}%
\pgfsetbuttcap%
\pgfsetroundjoin%
\definecolor{currentfill}{rgb}{0.395174,0.797475,0.367757}%
\pgfsetfillcolor{currentfill}%
\pgfsetfillopacity{0.700000}%
\pgfsetlinewidth{0.000000pt}%
\definecolor{currentstroke}{rgb}{0.000000,0.000000,0.000000}%
\pgfsetstrokecolor{currentstroke}%
\pgfsetdash{}{0pt}%
\pgfpathmoveto{\pgfqpoint{5.821250in}{3.422234in}}%
\pgfpathlineto{\pgfqpoint{5.835811in}{3.436951in}}%
\pgfpathlineto{\pgfqpoint{5.850394in}{3.451829in}}%
\pgfpathlineto{\pgfqpoint{5.864999in}{3.466869in}}%
\pgfpathlineto{\pgfqpoint{5.871900in}{3.467299in}}%
\pgfpathlineto{\pgfqpoint{5.878793in}{3.467662in}}%
\pgfpathlineto{\pgfqpoint{5.885675in}{3.467964in}}%
\pgfpathlineto{\pgfqpoint{5.892549in}{3.468208in}}%
\pgfpathlineto{\pgfqpoint{5.877973in}{3.453620in}}%
\pgfpathlineto{\pgfqpoint{5.863418in}{3.439194in}}%
\pgfpathlineto{\pgfqpoint{5.848885in}{3.424928in}}%
\pgfpathlineto{\pgfqpoint{5.841989in}{3.424337in}}%
\pgfpathlineto{\pgfqpoint{5.835085in}{3.423694in}}%
\pgfpathlineto{\pgfqpoint{5.828172in}{3.422995in}}%
\pgfpathlineto{\pgfqpoint{5.821250in}{3.422234in}}%
\pgfpathclose%
\pgfusepath{fill}%
\end{pgfscope}%
\begin{pgfscope}%
\pgfpathrectangle{\pgfqpoint{1.254980in}{0.150000in}}{\pgfqpoint{5.490039in}{5.490039in}}%
\pgfusepath{clip}%
\pgfsetbuttcap%
\pgfsetroundjoin%
\definecolor{currentfill}{rgb}{0.277018,0.050344,0.375715}%
\pgfsetfillcolor{currentfill}%
\pgfsetfillopacity{0.700000}%
\pgfsetlinewidth{0.000000pt}%
\definecolor{currentstroke}{rgb}{0.000000,0.000000,0.000000}%
\pgfsetstrokecolor{currentstroke}%
\pgfsetdash{}{0pt}%
\pgfpathmoveto{\pgfqpoint{3.693171in}{1.546707in}}%
\pgfpathlineto{\pgfqpoint{3.706588in}{1.546473in}}%
\pgfpathlineto{\pgfqpoint{3.720011in}{1.546404in}}%
\pgfpathlineto{\pgfqpoint{3.733442in}{1.546498in}}%
\pgfpathlineto{\pgfqpoint{3.746880in}{1.546755in}}%
\pgfpathlineto{\pgfqpoint{3.754805in}{1.557709in}}%
\pgfpathlineto{\pgfqpoint{3.762723in}{1.568729in}}%
\pgfpathlineto{\pgfqpoint{3.770636in}{1.579811in}}%
\pgfpathlineto{\pgfqpoint{3.778544in}{1.590950in}}%
\pgfpathlineto{\pgfqpoint{3.765116in}{1.590250in}}%
\pgfpathlineto{\pgfqpoint{3.751695in}{1.589714in}}%
\pgfpathlineto{\pgfqpoint{3.738282in}{1.589342in}}%
\pgfpathlineto{\pgfqpoint{3.724876in}{1.589134in}}%
\pgfpathlineto{\pgfqpoint{3.716959in}{1.578427in}}%
\pgfpathlineto{\pgfqpoint{3.709035in}{1.567784in}}%
\pgfpathlineto{\pgfqpoint{3.701106in}{1.557209in}}%
\pgfpathlineto{\pgfqpoint{3.693171in}{1.546707in}}%
\pgfpathclose%
\pgfusepath{fill}%
\end{pgfscope}%
\begin{pgfscope}%
\pgfpathrectangle{\pgfqpoint{1.254980in}{0.150000in}}{\pgfqpoint{5.490039in}{5.490039in}}%
\pgfusepath{clip}%
\pgfsetbuttcap%
\pgfsetroundjoin%
\definecolor{currentfill}{rgb}{0.120092,0.600104,0.542530}%
\pgfsetfillcolor{currentfill}%
\pgfsetfillopacity{0.700000}%
\pgfsetlinewidth{0.000000pt}%
\definecolor{currentstroke}{rgb}{0.000000,0.000000,0.000000}%
\pgfsetstrokecolor{currentstroke}%
\pgfsetdash{}{0pt}%
\pgfpathmoveto{\pgfqpoint{5.075401in}{2.819250in}}%
\pgfpathlineto{\pgfqpoint{5.089479in}{2.831565in}}%
\pgfpathlineto{\pgfqpoint{5.103576in}{2.844041in}}%
\pgfpathlineto{\pgfqpoint{5.117691in}{2.856680in}}%
\pgfpathlineto{\pgfqpoint{5.131824in}{2.869481in}}%
\pgfpathlineto{\pgfqpoint{5.139239in}{2.876764in}}%
\pgfpathlineto{\pgfqpoint{5.146645in}{2.883907in}}%
\pgfpathlineto{\pgfqpoint{5.154042in}{2.890909in}}%
\pgfpathlineto{\pgfqpoint{5.161430in}{2.897775in}}%
\pgfpathlineto{\pgfqpoint{5.147305in}{2.885110in}}%
\pgfpathlineto{\pgfqpoint{5.133198in}{2.872608in}}%
\pgfpathlineto{\pgfqpoint{5.119110in}{2.860267in}}%
\pgfpathlineto{\pgfqpoint{5.105040in}{2.848088in}}%
\pgfpathlineto{\pgfqpoint{5.097643in}{2.841075in}}%
\pgfpathlineto{\pgfqpoint{5.090237in}{2.833933in}}%
\pgfpathlineto{\pgfqpoint{5.082823in}{2.826658in}}%
\pgfpathlineto{\pgfqpoint{5.075401in}{2.819250in}}%
\pgfpathclose%
\pgfusepath{fill}%
\end{pgfscope}%
\begin{pgfscope}%
\pgfpathrectangle{\pgfqpoint{1.254980in}{0.150000in}}{\pgfqpoint{5.490039in}{5.490039in}}%
\pgfusepath{clip}%
\pgfsetbuttcap%
\pgfsetroundjoin%
\definecolor{currentfill}{rgb}{0.280894,0.078907,0.402329}%
\pgfsetfillcolor{currentfill}%
\pgfsetfillopacity{0.700000}%
\pgfsetlinewidth{0.000000pt}%
\definecolor{currentstroke}{rgb}{0.000000,0.000000,0.000000}%
\pgfsetstrokecolor{currentstroke}%
\pgfsetdash{}{0pt}%
\pgfpathmoveto{\pgfqpoint{3.778544in}{1.590950in}}%
\pgfpathlineto{\pgfqpoint{3.791980in}{1.591813in}}%
\pgfpathlineto{\pgfqpoint{3.805425in}{1.592838in}}%
\pgfpathlineto{\pgfqpoint{3.818877in}{1.594027in}}%
\pgfpathlineto{\pgfqpoint{3.832338in}{1.595378in}}%
\pgfpathlineto{\pgfqpoint{3.840232in}{1.606995in}}%
\pgfpathlineto{\pgfqpoint{3.848120in}{1.618655in}}%
\pgfpathlineto{\pgfqpoint{3.856004in}{1.630353in}}%
\pgfpathlineto{\pgfqpoint{3.863883in}{1.642086in}}%
\pgfpathlineto{\pgfqpoint{3.850430in}{1.640319in}}%
\pgfpathlineto{\pgfqpoint{3.836985in}{1.638716in}}%
\pgfpathlineto{\pgfqpoint{3.823550in}{1.637275in}}%
\pgfpathlineto{\pgfqpoint{3.810122in}{1.635998in}}%
\pgfpathlineto{\pgfqpoint{3.802235in}{1.624670in}}%
\pgfpathlineto{\pgfqpoint{3.794344in}{1.613383in}}%
\pgfpathlineto{\pgfqpoint{3.786446in}{1.602142in}}%
\pgfpathlineto{\pgfqpoint{3.778544in}{1.590950in}}%
\pgfpathclose%
\pgfusepath{fill}%
\end{pgfscope}%
\begin{pgfscope}%
\pgfpathrectangle{\pgfqpoint{1.254980in}{0.150000in}}{\pgfqpoint{5.490039in}{5.490039in}}%
\pgfusepath{clip}%
\pgfsetbuttcap%
\pgfsetroundjoin%
\definecolor{currentfill}{rgb}{0.171176,0.452530,0.557965}%
\pgfsetfillcolor{currentfill}%
\pgfsetfillopacity{0.700000}%
\pgfsetlinewidth{0.000000pt}%
\definecolor{currentstroke}{rgb}{0.000000,0.000000,0.000000}%
\pgfsetstrokecolor{currentstroke}%
\pgfsetdash{}{0pt}%
\pgfpathmoveto{\pgfqpoint{4.671596in}{2.420349in}}%
\pgfpathlineto{\pgfqpoint{4.685427in}{2.430249in}}%
\pgfpathlineto{\pgfqpoint{4.699275in}{2.440310in}}%
\pgfpathlineto{\pgfqpoint{4.713138in}{2.450533in}}%
\pgfpathlineto{\pgfqpoint{4.727018in}{2.460917in}}%
\pgfpathlineto{\pgfqpoint{4.734631in}{2.471684in}}%
\pgfpathlineto{\pgfqpoint{4.742237in}{2.482321in}}%
\pgfpathlineto{\pgfqpoint{4.749836in}{2.492827in}}%
\pgfpathlineto{\pgfqpoint{4.757429in}{2.503204in}}%
\pgfpathlineto{\pgfqpoint{4.743552in}{2.492773in}}%
\pgfpathlineto{\pgfqpoint{4.729690in}{2.482503in}}%
\pgfpathlineto{\pgfqpoint{4.715845in}{2.472394in}}%
\pgfpathlineto{\pgfqpoint{4.702015in}{2.462447in}}%
\pgfpathlineto{\pgfqpoint{4.694420in}{2.452107in}}%
\pgfpathlineto{\pgfqpoint{4.686818in}{2.441643in}}%
\pgfpathlineto{\pgfqpoint{4.679210in}{2.431058in}}%
\pgfpathlineto{\pgfqpoint{4.671596in}{2.420349in}}%
\pgfpathclose%
\pgfusepath{fill}%
\end{pgfscope}%
\begin{pgfscope}%
\pgfpathrectangle{\pgfqpoint{1.254980in}{0.150000in}}{\pgfqpoint{5.490039in}{5.490039in}}%
\pgfusepath{clip}%
\pgfsetbuttcap%
\pgfsetroundjoin%
\definecolor{currentfill}{rgb}{0.273809,0.031497,0.358853}%
\pgfsetfillcolor{currentfill}%
\pgfsetfillopacity{0.700000}%
\pgfsetlinewidth{0.000000pt}%
\definecolor{currentstroke}{rgb}{0.000000,0.000000,0.000000}%
\pgfsetstrokecolor{currentstroke}%
\pgfsetdash{}{0pt}%
\pgfpathmoveto{\pgfqpoint{3.103326in}{1.539928in}}%
\pgfpathlineto{\pgfqpoint{3.116726in}{1.531553in}}%
\pgfpathlineto{\pgfqpoint{3.130126in}{1.523361in}}%
\pgfpathlineto{\pgfqpoint{3.143527in}{1.515351in}}%
\pgfpathlineto{\pgfqpoint{3.156928in}{1.507522in}}%
\pgfpathlineto{\pgfqpoint{3.165172in}{1.511421in}}%
\pgfpathlineto{\pgfqpoint{3.173404in}{1.515544in}}%
\pgfpathlineto{\pgfqpoint{3.181624in}{1.519884in}}%
\pgfpathlineto{\pgfqpoint{3.189833in}{1.524436in}}%
\pgfpathlineto{\pgfqpoint{3.176461in}{1.531681in}}%
\pgfpathlineto{\pgfqpoint{3.163090in}{1.539106in}}%
\pgfpathlineto{\pgfqpoint{3.149720in}{1.546714in}}%
\pgfpathlineto{\pgfqpoint{3.136351in}{1.554504in}}%
\pgfpathlineto{\pgfqpoint{3.128113in}{1.550525in}}%
\pgfpathlineto{\pgfqpoint{3.119863in}{1.546766in}}%
\pgfpathlineto{\pgfqpoint{3.111600in}{1.543231in}}%
\pgfpathlineto{\pgfqpoint{3.103326in}{1.539928in}}%
\pgfpathclose%
\pgfusepath{fill}%
\end{pgfscope}%
\begin{pgfscope}%
\pgfpathrectangle{\pgfqpoint{1.254980in}{0.150000in}}{\pgfqpoint{5.490039in}{5.490039in}}%
\pgfusepath{clip}%
\pgfsetbuttcap%
\pgfsetroundjoin%
\definecolor{currentfill}{rgb}{0.272594,0.025563,0.353093}%
\pgfsetfillcolor{currentfill}%
\pgfsetfillopacity{0.700000}%
\pgfsetlinewidth{0.000000pt}%
\definecolor{currentstroke}{rgb}{0.000000,0.000000,0.000000}%
\pgfsetstrokecolor{currentstroke}%
\pgfsetdash{}{0pt}%
\pgfpathmoveto{\pgfqpoint{3.607723in}{1.509972in}}%
\pgfpathlineto{\pgfqpoint{3.621126in}{1.508609in}}%
\pgfpathlineto{\pgfqpoint{3.634535in}{1.507412in}}%
\pgfpathlineto{\pgfqpoint{3.647950in}{1.506380in}}%
\pgfpathlineto{\pgfqpoint{3.661372in}{1.505512in}}%
\pgfpathlineto{\pgfqpoint{3.669331in}{1.515680in}}%
\pgfpathlineto{\pgfqpoint{3.677284in}{1.525938in}}%
\pgfpathlineto{\pgfqpoint{3.685230in}{1.536282in}}%
\pgfpathlineto{\pgfqpoint{3.693171in}{1.546707in}}%
\pgfpathlineto{\pgfqpoint{3.679762in}{1.547105in}}%
\pgfpathlineto{\pgfqpoint{3.666359in}{1.547668in}}%
\pgfpathlineto{\pgfqpoint{3.652963in}{1.548397in}}%
\pgfpathlineto{\pgfqpoint{3.639574in}{1.549290in}}%
\pgfpathlineto{\pgfqpoint{3.631621in}{1.539324in}}%
\pgfpathlineto{\pgfqpoint{3.623661in}{1.529446in}}%
\pgfpathlineto{\pgfqpoint{3.615695in}{1.519660in}}%
\pgfpathlineto{\pgfqpoint{3.607723in}{1.509972in}}%
\pgfpathclose%
\pgfusepath{fill}%
\end{pgfscope}%
\begin{pgfscope}%
\pgfpathrectangle{\pgfqpoint{1.254980in}{0.150000in}}{\pgfqpoint{5.490039in}{5.490039in}}%
\pgfusepath{clip}%
\pgfsetbuttcap%
\pgfsetroundjoin%
\definecolor{currentfill}{rgb}{0.166383,0.690856,0.496502}%
\pgfsetfillcolor{currentfill}%
\pgfsetfillopacity{0.700000}%
\pgfsetlinewidth{0.000000pt}%
\definecolor{currentstroke}{rgb}{0.000000,0.000000,0.000000}%
\pgfsetstrokecolor{currentstroke}%
\pgfsetdash{}{0pt}%
\pgfpathmoveto{\pgfqpoint{5.362840in}{3.072595in}}%
\pgfpathlineto{\pgfqpoint{5.377104in}{3.086121in}}%
\pgfpathlineto{\pgfqpoint{5.391388in}{3.099809in}}%
\pgfpathlineto{\pgfqpoint{5.405693in}{3.113660in}}%
\pgfpathlineto{\pgfqpoint{5.420017in}{3.127673in}}%
\pgfpathlineto{\pgfqpoint{5.427252in}{3.132207in}}%
\pgfpathlineto{\pgfqpoint{5.434478in}{3.136615in}}%
\pgfpathlineto{\pgfqpoint{5.441694in}{3.140900in}}%
\pgfpathlineto{\pgfqpoint{5.448901in}{3.145065in}}%
\pgfpathlineto{\pgfqpoint{5.434591in}{3.131314in}}%
\pgfpathlineto{\pgfqpoint{5.420302in}{3.117725in}}%
\pgfpathlineto{\pgfqpoint{5.406033in}{3.104298in}}%
\pgfpathlineto{\pgfqpoint{5.391783in}{3.091032in}}%
\pgfpathlineto{\pgfqpoint{5.384561in}{3.086595in}}%
\pgfpathlineto{\pgfqpoint{5.377330in}{3.082045in}}%
\pgfpathlineto{\pgfqpoint{5.370089in}{3.077379in}}%
\pgfpathlineto{\pgfqpoint{5.362840in}{3.072595in}}%
\pgfpathclose%
\pgfusepath{fill}%
\end{pgfscope}%
\begin{pgfscope}%
\pgfpathrectangle{\pgfqpoint{1.254980in}{0.150000in}}{\pgfqpoint{5.490039in}{5.490039in}}%
\pgfusepath{clip}%
\pgfsetbuttcap%
\pgfsetroundjoin%
\definecolor{currentfill}{rgb}{0.283091,0.110553,0.431554}%
\pgfsetfillcolor{currentfill}%
\pgfsetfillopacity{0.700000}%
\pgfsetlinewidth{0.000000pt}%
\definecolor{currentstroke}{rgb}{0.000000,0.000000,0.000000}%
\pgfsetstrokecolor{currentstroke}%
\pgfsetdash{}{0pt}%
\pgfpathmoveto{\pgfqpoint{2.855023in}{1.698669in}}%
\pgfpathlineto{\pgfqpoint{2.868481in}{1.686616in}}%
\pgfpathlineto{\pgfqpoint{2.881935in}{1.674763in}}%
\pgfpathlineto{\pgfqpoint{2.895387in}{1.663108in}}%
\pgfpathlineto{\pgfqpoint{2.908836in}{1.651649in}}%
\pgfpathlineto{\pgfqpoint{2.917274in}{1.652082in}}%
\pgfpathlineto{\pgfqpoint{2.925696in}{1.652793in}}%
\pgfpathlineto{\pgfqpoint{2.934104in}{1.653776in}}%
\pgfpathlineto{\pgfqpoint{2.942496in}{1.655025in}}%
\pgfpathlineto{\pgfqpoint{2.929086in}{1.665862in}}%
\pgfpathlineto{\pgfqpoint{2.915673in}{1.676896in}}%
\pgfpathlineto{\pgfqpoint{2.902259in}{1.688128in}}%
\pgfpathlineto{\pgfqpoint{2.888842in}{1.699558in}}%
\pgfpathlineto{\pgfqpoint{2.880411in}{1.698919in}}%
\pgfpathlineto{\pgfqpoint{2.871964in}{1.698554in}}%
\pgfpathlineto{\pgfqpoint{2.863502in}{1.698468in}}%
\pgfpathlineto{\pgfqpoint{2.855023in}{1.698669in}}%
\pgfpathclose%
\pgfusepath{fill}%
\end{pgfscope}%
\begin{pgfscope}%
\pgfpathrectangle{\pgfqpoint{1.254980in}{0.150000in}}{\pgfqpoint{5.490039in}{5.490039in}}%
\pgfusepath{clip}%
\pgfsetbuttcap%
\pgfsetroundjoin%
\definecolor{currentfill}{rgb}{0.140536,0.530132,0.555659}%
\pgfsetfillcolor{currentfill}%
\pgfsetfillopacity{0.700000}%
\pgfsetlinewidth{0.000000pt}%
\definecolor{currentstroke}{rgb}{0.000000,0.000000,0.000000}%
\pgfsetstrokecolor{currentstroke}%
\pgfsetdash{}{0pt}%
\pgfpathmoveto{\pgfqpoint{4.873579in}{2.624915in}}%
\pgfpathlineto{\pgfqpoint{4.887533in}{2.636158in}}%
\pgfpathlineto{\pgfqpoint{4.901505in}{2.647563in}}%
\pgfpathlineto{\pgfqpoint{4.915493in}{2.659129in}}%
\pgfpathlineto{\pgfqpoint{4.929500in}{2.670858in}}%
\pgfpathlineto{\pgfqpoint{4.937023in}{2.680012in}}%
\pgfpathlineto{\pgfqpoint{4.944538in}{2.689024in}}%
\pgfpathlineto{\pgfqpoint{4.952046in}{2.697896in}}%
\pgfpathlineto{\pgfqpoint{4.959546in}{2.706629in}}%
\pgfpathlineto{\pgfqpoint{4.945544in}{2.694945in}}%
\pgfpathlineto{\pgfqpoint{4.931560in}{2.683422in}}%
\pgfpathlineto{\pgfqpoint{4.917594in}{2.672060in}}%
\pgfpathlineto{\pgfqpoint{4.903644in}{2.660860in}}%
\pgfpathlineto{\pgfqpoint{4.896139in}{2.652073in}}%
\pgfpathlineto{\pgfqpoint{4.888626in}{2.643153in}}%
\pgfpathlineto{\pgfqpoint{4.881106in}{2.634101in}}%
\pgfpathlineto{\pgfqpoint{4.873579in}{2.624915in}}%
\pgfpathclose%
\pgfusepath{fill}%
\end{pgfscope}%
\begin{pgfscope}%
\pgfpathrectangle{\pgfqpoint{1.254980in}{0.150000in}}{\pgfqpoint{5.490039in}{5.490039in}}%
\pgfusepath{clip}%
\pgfsetbuttcap%
\pgfsetroundjoin%
\definecolor{currentfill}{rgb}{0.283091,0.110553,0.431554}%
\pgfsetfillcolor{currentfill}%
\pgfsetfillopacity{0.700000}%
\pgfsetlinewidth{0.000000pt}%
\definecolor{currentstroke}{rgb}{0.000000,0.000000,0.000000}%
\pgfsetstrokecolor{currentstroke}%
\pgfsetdash{}{0pt}%
\pgfpathmoveto{\pgfqpoint{3.863883in}{1.642086in}}%
\pgfpathlineto{\pgfqpoint{3.877345in}{1.644014in}}%
\pgfpathlineto{\pgfqpoint{3.890815in}{1.646104in}}%
\pgfpathlineto{\pgfqpoint{3.904295in}{1.648356in}}%
\pgfpathlineto{\pgfqpoint{3.917783in}{1.650770in}}%
\pgfpathlineto{\pgfqpoint{3.925651in}{1.662932in}}%
\pgfpathlineto{\pgfqpoint{3.933513in}{1.675114in}}%
\pgfpathlineto{\pgfqpoint{3.941371in}{1.687312in}}%
\pgfpathlineto{\pgfqpoint{3.949224in}{1.699524in}}%
\pgfpathlineto{\pgfqpoint{3.935741in}{1.696722in}}%
\pgfpathlineto{\pgfqpoint{3.922268in}{1.694081in}}%
\pgfpathlineto{\pgfqpoint{3.908804in}{1.691603in}}%
\pgfpathlineto{\pgfqpoint{3.895349in}{1.689287in}}%
\pgfpathlineto{\pgfqpoint{3.887489in}{1.677453in}}%
\pgfpathlineto{\pgfqpoint{3.879626in}{1.665639in}}%
\pgfpathlineto{\pgfqpoint{3.871757in}{1.653849in}}%
\pgfpathlineto{\pgfqpoint{3.863883in}{1.642086in}}%
\pgfpathclose%
\pgfusepath{fill}%
\end{pgfscope}%
\begin{pgfscope}%
\pgfpathrectangle{\pgfqpoint{1.254980in}{0.150000in}}{\pgfqpoint{5.490039in}{5.490039in}}%
\pgfusepath{clip}%
\pgfsetbuttcap%
\pgfsetroundjoin%
\definecolor{currentfill}{rgb}{0.195860,0.395433,0.555276}%
\pgfsetfillcolor{currentfill}%
\pgfsetfillopacity{0.700000}%
\pgfsetlinewidth{0.000000pt}%
\definecolor{currentstroke}{rgb}{0.000000,0.000000,0.000000}%
\pgfsetstrokecolor{currentstroke}%
\pgfsetdash{}{0pt}%
\pgfpathmoveto{\pgfqpoint{2.312020in}{2.360575in}}%
\pgfpathlineto{\pgfqpoint{2.325766in}{2.339307in}}%
\pgfpathlineto{\pgfqpoint{2.339501in}{2.318307in}}%
\pgfpathlineto{\pgfqpoint{2.353224in}{2.297572in}}%
\pgfpathlineto{\pgfqpoint{2.366936in}{2.277101in}}%
\pgfpathlineto{\pgfqpoint{2.375868in}{2.271048in}}%
\pgfpathlineto{\pgfqpoint{2.384776in}{2.265356in}}%
\pgfpathlineto{\pgfqpoint{2.393662in}{2.260017in}}%
\pgfpathlineto{\pgfqpoint{2.402525in}{2.255026in}}%
\pgfpathlineto{\pgfqpoint{2.388872in}{2.274840in}}%
\pgfpathlineto{\pgfqpoint{2.375209in}{2.294915in}}%
\pgfpathlineto{\pgfqpoint{2.361534in}{2.315254in}}%
\pgfpathlineto{\pgfqpoint{2.347847in}{2.335860in}}%
\pgfpathlineto{\pgfqpoint{2.338926in}{2.341498in}}%
\pgfpathlineto{\pgfqpoint{2.329981in}{2.347492in}}%
\pgfpathlineto{\pgfqpoint{2.321013in}{2.353849in}}%
\pgfpathlineto{\pgfqpoint{2.312020in}{2.360575in}}%
\pgfpathclose%
\pgfusepath{fill}%
\end{pgfscope}%
\begin{pgfscope}%
\pgfpathrectangle{\pgfqpoint{1.254980in}{0.150000in}}{\pgfqpoint{5.490039in}{5.490039in}}%
\pgfusepath{clip}%
\pgfsetbuttcap%
\pgfsetroundjoin%
\definecolor{currentfill}{rgb}{0.265145,0.232956,0.516599}%
\pgfsetfillcolor{currentfill}%
\pgfsetfillopacity{0.700000}%
\pgfsetlinewidth{0.000000pt}%
\definecolor{currentstroke}{rgb}{0.000000,0.000000,0.000000}%
\pgfsetstrokecolor{currentstroke}%
\pgfsetdash{}{0pt}%
\pgfpathmoveto{\pgfqpoint{4.151234in}{1.882089in}}%
\pgfpathlineto{\pgfqpoint{4.164801in}{1.887342in}}%
\pgfpathlineto{\pgfqpoint{4.178379in}{1.892756in}}%
\pgfpathlineto{\pgfqpoint{4.191969in}{1.898330in}}%
\pgfpathlineto{\pgfqpoint{4.205571in}{1.904065in}}%
\pgfpathlineto{\pgfqpoint{4.213357in}{1.917018in}}%
\pgfpathlineto{\pgfqpoint{4.221137in}{1.929920in}}%
\pgfpathlineto{\pgfqpoint{4.228913in}{1.942770in}}%
\pgfpathlineto{\pgfqpoint{4.236685in}{1.955564in}}%
\pgfpathlineto{\pgfqpoint{4.223084in}{1.949550in}}%
\pgfpathlineto{\pgfqpoint{4.209496in}{1.943698in}}%
\pgfpathlineto{\pgfqpoint{4.195919in}{1.938006in}}%
\pgfpathlineto{\pgfqpoint{4.182354in}{1.932475in}}%
\pgfpathlineto{\pgfqpoint{4.174581in}{1.919949in}}%
\pgfpathlineto{\pgfqpoint{4.166803in}{1.907374in}}%
\pgfpathlineto{\pgfqpoint{4.159021in}{1.894753in}}%
\pgfpathlineto{\pgfqpoint{4.151234in}{1.882089in}}%
\pgfpathclose%
\pgfusepath{fill}%
\end{pgfscope}%
\begin{pgfscope}%
\pgfpathrectangle{\pgfqpoint{1.254980in}{0.150000in}}{\pgfqpoint{5.490039in}{5.490039in}}%
\pgfusepath{clip}%
\pgfsetbuttcap%
\pgfsetroundjoin%
\definecolor{currentfill}{rgb}{0.229739,0.322361,0.545706}%
\pgfsetfillcolor{currentfill}%
\pgfsetfillopacity{0.700000}%
\pgfsetlinewidth{0.000000pt}%
\definecolor{currentstroke}{rgb}{0.000000,0.000000,0.000000}%
\pgfsetstrokecolor{currentstroke}%
\pgfsetdash{}{0pt}%
\pgfpathmoveto{\pgfqpoint{4.353209in}{2.083209in}}%
\pgfpathlineto{\pgfqpoint{4.366871in}{2.090499in}}%
\pgfpathlineto{\pgfqpoint{4.380547in}{2.097950in}}%
\pgfpathlineto{\pgfqpoint{4.394236in}{2.105561in}}%
\pgfpathlineto{\pgfqpoint{4.407939in}{2.113333in}}%
\pgfpathlineto{\pgfqpoint{4.415666in}{2.125941in}}%
\pgfpathlineto{\pgfqpoint{4.423389in}{2.138459in}}%
\pgfpathlineto{\pgfqpoint{4.431107in}{2.150886in}}%
\pgfpathlineto{\pgfqpoint{4.438819in}{2.163220in}}%
\pgfpathlineto{\pgfqpoint{4.425117in}{2.155254in}}%
\pgfpathlineto{\pgfqpoint{4.411428in}{2.147449in}}%
\pgfpathlineto{\pgfqpoint{4.397753in}{2.139804in}}%
\pgfpathlineto{\pgfqpoint{4.384092in}{2.132321in}}%
\pgfpathlineto{\pgfqpoint{4.376378in}{2.120170in}}%
\pgfpathlineto{\pgfqpoint{4.368660in}{2.107934in}}%
\pgfpathlineto{\pgfqpoint{4.360937in}{2.095613in}}%
\pgfpathlineto{\pgfqpoint{4.353209in}{2.083209in}}%
\pgfpathclose%
\pgfusepath{fill}%
\end{pgfscope}%
\begin{pgfscope}%
\pgfpathrectangle{\pgfqpoint{1.254980in}{0.150000in}}{\pgfqpoint{5.490039in}{5.490039in}}%
\pgfusepath{clip}%
\pgfsetbuttcap%
\pgfsetroundjoin%
\definecolor{currentfill}{rgb}{0.269944,0.014625,0.341379}%
\pgfsetfillcolor{currentfill}%
\pgfsetfillopacity{0.700000}%
\pgfsetlinewidth{0.000000pt}%
\definecolor{currentstroke}{rgb}{0.000000,0.000000,0.000000}%
\pgfsetstrokecolor{currentstroke}%
\pgfsetdash{}{0pt}%
\pgfpathmoveto{\pgfqpoint{3.522154in}{1.481384in}}%
\pgfpathlineto{\pgfqpoint{3.535549in}{1.478859in}}%
\pgfpathlineto{\pgfqpoint{3.548950in}{1.476500in}}%
\pgfpathlineto{\pgfqpoint{3.562356in}{1.474309in}}%
\pgfpathlineto{\pgfqpoint{3.575767in}{1.472283in}}%
\pgfpathlineto{\pgfqpoint{3.583766in}{1.481536in}}%
\pgfpathlineto{\pgfqpoint{3.591759in}{1.490905in}}%
\pgfpathlineto{\pgfqpoint{3.599744in}{1.500385in}}%
\pgfpathlineto{\pgfqpoint{3.607723in}{1.509972in}}%
\pgfpathlineto{\pgfqpoint{3.594326in}{1.511501in}}%
\pgfpathlineto{\pgfqpoint{3.580936in}{1.513196in}}%
\pgfpathlineto{\pgfqpoint{3.567551in}{1.515057in}}%
\pgfpathlineto{\pgfqpoint{3.554171in}{1.517086in}}%
\pgfpathlineto{\pgfqpoint{3.546178in}{1.507986in}}%
\pgfpathlineto{\pgfqpoint{3.538177in}{1.498999in}}%
\pgfpathlineto{\pgfqpoint{3.530169in}{1.490130in}}%
\pgfpathlineto{\pgfqpoint{3.522154in}{1.481384in}}%
\pgfpathclose%
\pgfusepath{fill}%
\end{pgfscope}%
\begin{pgfscope}%
\pgfpathrectangle{\pgfqpoint{1.254980in}{0.150000in}}{\pgfqpoint{5.490039in}{5.490039in}}%
\pgfusepath{clip}%
\pgfsetbuttcap%
\pgfsetroundjoin%
\definecolor{currentfill}{rgb}{0.282623,0.140926,0.457517}%
\pgfsetfillcolor{currentfill}%
\pgfsetfillopacity{0.700000}%
\pgfsetlinewidth{0.000000pt}%
\definecolor{currentstroke}{rgb}{0.000000,0.000000,0.000000}%
\pgfsetstrokecolor{currentstroke}%
\pgfsetdash{}{0pt}%
\pgfpathmoveto{\pgfqpoint{3.949224in}{1.699524in}}%
\pgfpathlineto{\pgfqpoint{3.962717in}{1.702487in}}%
\pgfpathlineto{\pgfqpoint{3.976219in}{1.705612in}}%
\pgfpathlineto{\pgfqpoint{3.989731in}{1.708898in}}%
\pgfpathlineto{\pgfqpoint{4.003253in}{1.712345in}}%
\pgfpathlineto{\pgfqpoint{4.011097in}{1.724938in}}%
\pgfpathlineto{\pgfqpoint{4.018936in}{1.737529in}}%
\pgfpathlineto{\pgfqpoint{4.026771in}{1.750116in}}%
\pgfpathlineto{\pgfqpoint{4.034601in}{1.762695in}}%
\pgfpathlineto{\pgfqpoint{4.021083in}{1.758887in}}%
\pgfpathlineto{\pgfqpoint{4.007575in}{1.755239in}}%
\pgfpathlineto{\pgfqpoint{3.994078in}{1.751754in}}%
\pgfpathlineto{\pgfqpoint{3.980590in}{1.748429in}}%
\pgfpathlineto{\pgfqpoint{3.972756in}{1.736200in}}%
\pgfpathlineto{\pgfqpoint{3.964917in}{1.723971in}}%
\pgfpathlineto{\pgfqpoint{3.957073in}{1.711744in}}%
\pgfpathlineto{\pgfqpoint{3.949224in}{1.699524in}}%
\pgfpathclose%
\pgfusepath{fill}%
\end{pgfscope}%
\begin{pgfscope}%
\pgfpathrectangle{\pgfqpoint{1.254980in}{0.150000in}}{\pgfqpoint{5.490039in}{5.490039in}}%
\pgfusepath{clip}%
\pgfsetbuttcap%
\pgfsetroundjoin%
\definecolor{currentfill}{rgb}{0.129933,0.559582,0.551864}%
\pgfsetfillcolor{currentfill}%
\pgfsetfillopacity{0.700000}%
\pgfsetlinewidth{0.000000pt}%
\definecolor{currentstroke}{rgb}{0.000000,0.000000,0.000000}%
\pgfsetstrokecolor{currentstroke}%
\pgfsetdash{}{0pt}%
\pgfpathmoveto{\pgfqpoint{2.071242in}{2.807896in}}%
\pgfpathlineto{\pgfqpoint{2.085223in}{2.781403in}}%
\pgfpathlineto{\pgfqpoint{2.099186in}{2.755232in}}%
\pgfpathlineto{\pgfqpoint{2.113131in}{2.729381in}}%
\pgfpathlineto{\pgfqpoint{2.127060in}{2.703846in}}%
\pgfpathlineto{\pgfqpoint{2.136209in}{2.695943in}}%
\pgfpathlineto{\pgfqpoint{2.145332in}{2.688417in}}%
\pgfpathlineto{\pgfqpoint{2.154429in}{2.681264in}}%
\pgfpathlineto{\pgfqpoint{2.163501in}{2.674476in}}%
\pgfpathlineto{\pgfqpoint{2.149639in}{2.699356in}}%
\pgfpathlineto{\pgfqpoint{2.135760in}{2.724550in}}%
\pgfpathlineto{\pgfqpoint{2.121865in}{2.750063in}}%
\pgfpathlineto{\pgfqpoint{2.107952in}{2.775896in}}%
\pgfpathlineto{\pgfqpoint{2.098815in}{2.783328in}}%
\pgfpathlineto{\pgfqpoint{2.089651in}{2.791134in}}%
\pgfpathlineto{\pgfqpoint{2.080460in}{2.799321in}}%
\pgfpathlineto{\pgfqpoint{2.071242in}{2.807896in}}%
\pgfpathclose%
\pgfusepath{fill}%
\end{pgfscope}%
\begin{pgfscope}%
\pgfpathrectangle{\pgfqpoint{1.254980in}{0.150000in}}{\pgfqpoint{5.490039in}{5.490039in}}%
\pgfusepath{clip}%
\pgfsetbuttcap%
\pgfsetroundjoin%
\definecolor{currentfill}{rgb}{0.281924,0.089666,0.412415}%
\pgfsetfillcolor{currentfill}%
\pgfsetfillopacity{0.700000}%
\pgfsetlinewidth{0.000000pt}%
\definecolor{currentstroke}{rgb}{0.000000,0.000000,0.000000}%
\pgfsetstrokecolor{currentstroke}%
\pgfsetdash{}{0pt}%
\pgfpathmoveto{\pgfqpoint{2.908836in}{1.651649in}}%
\pgfpathlineto{\pgfqpoint{2.922283in}{1.640387in}}%
\pgfpathlineto{\pgfqpoint{2.935728in}{1.629320in}}%
\pgfpathlineto{\pgfqpoint{2.949171in}{1.618447in}}%
\pgfpathlineto{\pgfqpoint{2.962612in}{1.607766in}}%
\pgfpathlineto{\pgfqpoint{2.971011in}{1.608830in}}%
\pgfpathlineto{\pgfqpoint{2.979395in}{1.610164in}}%
\pgfpathlineto{\pgfqpoint{2.987765in}{1.611763in}}%
\pgfpathlineto{\pgfqpoint{2.996121in}{1.613619in}}%
\pgfpathlineto{\pgfqpoint{2.982717in}{1.623681in}}%
\pgfpathlineto{\pgfqpoint{2.969311in}{1.633935in}}%
\pgfpathlineto{\pgfqpoint{2.955905in}{1.644383in}}%
\pgfpathlineto{\pgfqpoint{2.942496in}{1.655025in}}%
\pgfpathlineto{\pgfqpoint{2.934104in}{1.653776in}}%
\pgfpathlineto{\pgfqpoint{2.925696in}{1.652793in}}%
\pgfpathlineto{\pgfqpoint{2.917274in}{1.652082in}}%
\pgfpathlineto{\pgfqpoint{2.908836in}{1.651649in}}%
\pgfpathclose%
\pgfusepath{fill}%
\end{pgfscope}%
\begin{pgfscope}%
\pgfpathrectangle{\pgfqpoint{1.254980in}{0.150000in}}{\pgfqpoint{5.490039in}{5.490039in}}%
\pgfusepath{clip}%
\pgfsetbuttcap%
\pgfsetroundjoin%
\definecolor{currentfill}{rgb}{0.267004,0.004874,0.329415}%
\pgfsetfillcolor{currentfill}%
\pgfsetfillopacity{0.700000}%
\pgfsetlinewidth{0.000000pt}%
\definecolor{currentstroke}{rgb}{0.000000,0.000000,0.000000}%
\pgfsetstrokecolor{currentstroke}%
\pgfsetdash{}{0pt}%
\pgfpathmoveto{\pgfqpoint{3.296869in}{1.472904in}}%
\pgfpathlineto{\pgfqpoint{3.310259in}{1.467255in}}%
\pgfpathlineto{\pgfqpoint{3.323651in}{1.461779in}}%
\pgfpathlineto{\pgfqpoint{3.337046in}{1.456477in}}%
\pgfpathlineto{\pgfqpoint{3.350444in}{1.451347in}}%
\pgfpathlineto{\pgfqpoint{3.358565in}{1.457799in}}%
\pgfpathlineto{\pgfqpoint{3.366677in}{1.464429in}}%
\pgfpathlineto{\pgfqpoint{3.374780in}{1.471233in}}%
\pgfpathlineto{\pgfqpoint{3.382874in}{1.478204in}}%
\pgfpathlineto{\pgfqpoint{3.369499in}{1.482781in}}%
\pgfpathlineto{\pgfqpoint{3.356127in}{1.487530in}}%
\pgfpathlineto{\pgfqpoint{3.342758in}{1.492452in}}%
\pgfpathlineto{\pgfqpoint{3.329393in}{1.497548in}}%
\pgfpathlineto{\pgfqpoint{3.321276in}{1.491119in}}%
\pgfpathlineto{\pgfqpoint{3.313150in}{1.484865in}}%
\pgfpathlineto{\pgfqpoint{3.305015in}{1.478792in}}%
\pgfpathlineto{\pgfqpoint{3.296869in}{1.472904in}}%
\pgfpathclose%
\pgfusepath{fill}%
\end{pgfscope}%
\begin{pgfscope}%
\pgfpathrectangle{\pgfqpoint{1.254980in}{0.150000in}}{\pgfqpoint{5.490039in}{5.490039in}}%
\pgfusepath{clip}%
\pgfsetbuttcap%
\pgfsetroundjoin%
\definecolor{currentfill}{rgb}{0.188923,0.410910,0.556326}%
\pgfsetfillcolor{currentfill}%
\pgfsetfillopacity{0.700000}%
\pgfsetlinewidth{0.000000pt}%
\definecolor{currentstroke}{rgb}{0.000000,0.000000,0.000000}%
\pgfsetstrokecolor{currentstroke}%
\pgfsetdash{}{0pt}%
\pgfpathmoveto{\pgfqpoint{4.555277in}{2.293058in}}%
\pgfpathlineto{\pgfqpoint{4.569050in}{2.302130in}}%
\pgfpathlineto{\pgfqpoint{4.582839in}{2.311362in}}%
\pgfpathlineto{\pgfqpoint{4.596642in}{2.320756in}}%
\pgfpathlineto{\pgfqpoint{4.610460in}{2.330311in}}%
\pgfpathlineto{\pgfqpoint{4.618123in}{2.341986in}}%
\pgfpathlineto{\pgfqpoint{4.625780in}{2.353543in}}%
\pgfpathlineto{\pgfqpoint{4.633431in}{2.364979in}}%
\pgfpathlineto{\pgfqpoint{4.641076in}{2.376296in}}%
\pgfpathlineto{\pgfqpoint{4.627258in}{2.366634in}}%
\pgfpathlineto{\pgfqpoint{4.613456in}{2.357134in}}%
\pgfpathlineto{\pgfqpoint{4.599669in}{2.347795in}}%
\pgfpathlineto{\pgfqpoint{4.585897in}{2.338617in}}%
\pgfpathlineto{\pgfqpoint{4.578251in}{2.327396in}}%
\pgfpathlineto{\pgfqpoint{4.570599in}{2.316062in}}%
\pgfpathlineto{\pgfqpoint{4.562941in}{2.304616in}}%
\pgfpathlineto{\pgfqpoint{4.555277in}{2.293058in}}%
\pgfpathclose%
\pgfusepath{fill}%
\end{pgfscope}%
\begin{pgfscope}%
\pgfpathrectangle{\pgfqpoint{1.254980in}{0.150000in}}{\pgfqpoint{5.490039in}{5.490039in}}%
\pgfusepath{clip}%
\pgfsetbuttcap%
\pgfsetroundjoin%
\definecolor{currentfill}{rgb}{0.208030,0.718701,0.472873}%
\pgfsetfillcolor{currentfill}%
\pgfsetfillopacity{0.700000}%
\pgfsetlinewidth{0.000000pt}%
\definecolor{currentstroke}{rgb}{0.000000,0.000000,0.000000}%
\pgfsetstrokecolor{currentstroke}%
\pgfsetdash{}{0pt}%
\pgfpathmoveto{\pgfqpoint{5.448901in}{3.145065in}}%
\pgfpathlineto{\pgfqpoint{5.463230in}{3.158977in}}%
\pgfpathlineto{\pgfqpoint{5.477580in}{3.173052in}}%
\pgfpathlineto{\pgfqpoint{5.491951in}{3.187289in}}%
\pgfpathlineto{\pgfqpoint{5.506342in}{3.201690in}}%
\pgfpathlineto{\pgfqpoint{5.513523in}{3.205455in}}%
\pgfpathlineto{\pgfqpoint{5.520694in}{3.209099in}}%
\pgfpathlineto{\pgfqpoint{5.527855in}{3.212625in}}%
\pgfpathlineto{\pgfqpoint{5.535006in}{3.216038in}}%
\pgfpathlineto{\pgfqpoint{5.520632in}{3.201933in}}%
\pgfpathlineto{\pgfqpoint{5.506279in}{3.187990in}}%
\pgfpathlineto{\pgfqpoint{5.491946in}{3.174208in}}%
\pgfpathlineto{\pgfqpoint{5.477633in}{3.160588in}}%
\pgfpathlineto{\pgfqpoint{5.470464in}{3.156871in}}%
\pgfpathlineto{\pgfqpoint{5.463286in}{3.153047in}}%
\pgfpathlineto{\pgfqpoint{5.456098in}{3.149113in}}%
\pgfpathlineto{\pgfqpoint{5.448901in}{3.145065in}}%
\pgfpathclose%
\pgfusepath{fill}%
\end{pgfscope}%
\begin{pgfscope}%
\pgfpathrectangle{\pgfqpoint{1.254980in}{0.150000in}}{\pgfqpoint{5.490039in}{5.490039in}}%
\pgfusepath{clip}%
\pgfsetbuttcap%
\pgfsetroundjoin%
\definecolor{currentfill}{rgb}{0.122312,0.633153,0.530398}%
\pgfsetfillcolor{currentfill}%
\pgfsetfillopacity{0.700000}%
\pgfsetlinewidth{0.000000pt}%
\definecolor{currentstroke}{rgb}{0.000000,0.000000,0.000000}%
\pgfsetstrokecolor{currentstroke}%
\pgfsetdash{}{0pt}%
\pgfpathmoveto{\pgfqpoint{5.161430in}{2.897775in}}%
\pgfpathlineto{\pgfqpoint{5.175575in}{2.910601in}}%
\pgfpathlineto{\pgfqpoint{5.189738in}{2.923590in}}%
\pgfpathlineto{\pgfqpoint{5.203920in}{2.936740in}}%
\pgfpathlineto{\pgfqpoint{5.218121in}{2.950054in}}%
\pgfpathlineto{\pgfqpoint{5.225492in}{2.956627in}}%
\pgfpathlineto{\pgfqpoint{5.232853in}{2.963059in}}%
\pgfpathlineto{\pgfqpoint{5.240206in}{2.969352in}}%
\pgfpathlineto{\pgfqpoint{5.247550in}{2.975508in}}%
\pgfpathlineto{\pgfqpoint{5.233359in}{2.962363in}}%
\pgfpathlineto{\pgfqpoint{5.219186in}{2.949380in}}%
\pgfpathlineto{\pgfqpoint{5.205033in}{2.936558in}}%
\pgfpathlineto{\pgfqpoint{5.190899in}{2.923899in}}%
\pgfpathlineto{\pgfqpoint{5.183545in}{2.917564in}}%
\pgfpathlineto{\pgfqpoint{5.176182in}{2.911100in}}%
\pgfpathlineto{\pgfqpoint{5.168810in}{2.904504in}}%
\pgfpathlineto{\pgfqpoint{5.161430in}{2.897775in}}%
\pgfpathclose%
\pgfusepath{fill}%
\end{pgfscope}%
\begin{pgfscope}%
\pgfpathrectangle{\pgfqpoint{1.254980in}{0.150000in}}{\pgfqpoint{5.490039in}{5.490039in}}%
\pgfusepath{clip}%
\pgfsetbuttcap%
\pgfsetroundjoin%
\definecolor{currentfill}{rgb}{0.179019,0.433756,0.557430}%
\pgfsetfillcolor{currentfill}%
\pgfsetfillopacity{0.700000}%
\pgfsetlinewidth{0.000000pt}%
\definecolor{currentstroke}{rgb}{0.000000,0.000000,0.000000}%
\pgfsetstrokecolor{currentstroke}%
\pgfsetdash{}{0pt}%
\pgfpathmoveto{\pgfqpoint{2.256907in}{2.448374in}}%
\pgfpathlineto{\pgfqpoint{2.270704in}{2.426010in}}%
\pgfpathlineto{\pgfqpoint{2.284489in}{2.403924in}}%
\pgfpathlineto{\pgfqpoint{2.298261in}{2.382114in}}%
\pgfpathlineto{\pgfqpoint{2.312020in}{2.360575in}}%
\pgfpathlineto{\pgfqpoint{2.321013in}{2.353849in}}%
\pgfpathlineto{\pgfqpoint{2.329981in}{2.347492in}}%
\pgfpathlineto{\pgfqpoint{2.338926in}{2.341498in}}%
\pgfpathlineto{\pgfqpoint{2.347847in}{2.335860in}}%
\pgfpathlineto{\pgfqpoint{2.334149in}{2.356734in}}%
\pgfpathlineto{\pgfqpoint{2.320439in}{2.377879in}}%
\pgfpathlineto{\pgfqpoint{2.306716in}{2.399298in}}%
\pgfpathlineto{\pgfqpoint{2.292981in}{2.420993in}}%
\pgfpathlineto{\pgfqpoint{2.283999in}{2.427284in}}%
\pgfpathlineto{\pgfqpoint{2.274993in}{2.433940in}}%
\pgfpathlineto{\pgfqpoint{2.265963in}{2.440968in}}%
\pgfpathlineto{\pgfqpoint{2.256907in}{2.448374in}}%
\pgfpathclose%
\pgfusepath{fill}%
\end{pgfscope}%
\begin{pgfscope}%
\pgfpathrectangle{\pgfqpoint{1.254980in}{0.150000in}}{\pgfqpoint{5.490039in}{5.490039in}}%
\pgfusepath{clip}%
\pgfsetbuttcap%
\pgfsetroundjoin%
\definecolor{currentfill}{rgb}{0.271305,0.019942,0.347269}%
\pgfsetfillcolor{currentfill}%
\pgfsetfillopacity{0.700000}%
\pgfsetlinewidth{0.000000pt}%
\definecolor{currentstroke}{rgb}{0.000000,0.000000,0.000000}%
\pgfsetstrokecolor{currentstroke}%
\pgfsetdash{}{0pt}%
\pgfpathmoveto{\pgfqpoint{3.156928in}{1.507522in}}%
\pgfpathlineto{\pgfqpoint{3.170331in}{1.499874in}}%
\pgfpathlineto{\pgfqpoint{3.183734in}{1.492405in}}%
\pgfpathlineto{\pgfqpoint{3.197139in}{1.485116in}}%
\pgfpathlineto{\pgfqpoint{3.210545in}{1.478004in}}%
\pgfpathlineto{\pgfqpoint{3.218759in}{1.482498in}}%
\pgfpathlineto{\pgfqpoint{3.226962in}{1.487208in}}%
\pgfpathlineto{\pgfqpoint{3.235154in}{1.492128in}}%
\pgfpathlineto{\pgfqpoint{3.243336in}{1.497252in}}%
\pgfpathlineto{\pgfqpoint{3.229958in}{1.503780in}}%
\pgfpathlineto{\pgfqpoint{3.216581in}{1.510487in}}%
\pgfpathlineto{\pgfqpoint{3.203207in}{1.517372in}}%
\pgfpathlineto{\pgfqpoint{3.189833in}{1.524436in}}%
\pgfpathlineto{\pgfqpoint{3.181624in}{1.519884in}}%
\pgfpathlineto{\pgfqpoint{3.173404in}{1.515544in}}%
\pgfpathlineto{\pgfqpoint{3.165172in}{1.511421in}}%
\pgfpathlineto{\pgfqpoint{3.156928in}{1.507522in}}%
\pgfpathclose%
\pgfusepath{fill}%
\end{pgfscope}%
\begin{pgfscope}%
\pgfpathrectangle{\pgfqpoint{1.254980in}{0.150000in}}{\pgfqpoint{5.490039in}{5.490039in}}%
\pgfusepath{clip}%
\pgfsetbuttcap%
\pgfsetroundjoin%
\definecolor{currentfill}{rgb}{0.156270,0.489624,0.557936}%
\pgfsetfillcolor{currentfill}%
\pgfsetfillopacity{0.700000}%
\pgfsetlinewidth{0.000000pt}%
\definecolor{currentstroke}{rgb}{0.000000,0.000000,0.000000}%
\pgfsetstrokecolor{currentstroke}%
\pgfsetdash{}{0pt}%
\pgfpathmoveto{\pgfqpoint{4.757429in}{2.503204in}}%
\pgfpathlineto{\pgfqpoint{4.771323in}{2.513797in}}%
\pgfpathlineto{\pgfqpoint{4.785234in}{2.524551in}}%
\pgfpathlineto{\pgfqpoint{4.799160in}{2.535467in}}%
\pgfpathlineto{\pgfqpoint{4.813104in}{2.546545in}}%
\pgfpathlineto{\pgfqpoint{4.820688in}{2.556820in}}%
\pgfpathlineto{\pgfqpoint{4.828265in}{2.566957in}}%
\pgfpathlineto{\pgfqpoint{4.835835in}{2.576958in}}%
\pgfpathlineto{\pgfqpoint{4.843398in}{2.586821in}}%
\pgfpathlineto{\pgfqpoint{4.829457in}{2.575727in}}%
\pgfpathlineto{\pgfqpoint{4.815533in}{2.564794in}}%
\pgfpathlineto{\pgfqpoint{4.801625in}{2.554022in}}%
\pgfpathlineto{\pgfqpoint{4.787734in}{2.543412in}}%
\pgfpathlineto{\pgfqpoint{4.780168in}{2.533555in}}%
\pgfpathlineto{\pgfqpoint{4.772595in}{2.523567in}}%
\pgfpathlineto{\pgfqpoint{4.765016in}{2.513451in}}%
\pgfpathlineto{\pgfqpoint{4.757429in}{2.503204in}}%
\pgfpathclose%
\pgfusepath{fill}%
\end{pgfscope}%
\begin{pgfscope}%
\pgfpathrectangle{\pgfqpoint{1.254980in}{0.150000in}}{\pgfqpoint{5.490039in}{5.490039in}}%
\pgfusepath{clip}%
\pgfsetbuttcap%
\pgfsetroundjoin%
\definecolor{currentfill}{rgb}{0.267004,0.004874,0.329415}%
\pgfsetfillcolor{currentfill}%
\pgfsetfillopacity{0.700000}%
\pgfsetlinewidth{0.000000pt}%
\definecolor{currentstroke}{rgb}{0.000000,0.000000,0.000000}%
\pgfsetstrokecolor{currentstroke}%
\pgfsetdash{}{0pt}%
\pgfpathmoveto{\pgfqpoint{3.436413in}{1.461610in}}%
\pgfpathlineto{\pgfqpoint{3.449807in}{1.457887in}}%
\pgfpathlineto{\pgfqpoint{3.463206in}{1.454333in}}%
\pgfpathlineto{\pgfqpoint{3.476610in}{1.450948in}}%
\pgfpathlineto{\pgfqpoint{3.490018in}{1.447730in}}%
\pgfpathlineto{\pgfqpoint{3.498063in}{1.455934in}}%
\pgfpathlineto{\pgfqpoint{3.506101in}{1.464281in}}%
\pgfpathlineto{\pgfqpoint{3.514131in}{1.472766in}}%
\pgfpathlineto{\pgfqpoint{3.522154in}{1.481384in}}%
\pgfpathlineto{\pgfqpoint{3.508764in}{1.484077in}}%
\pgfpathlineto{\pgfqpoint{3.495378in}{1.486938in}}%
\pgfpathlineto{\pgfqpoint{3.481998in}{1.489967in}}%
\pgfpathlineto{\pgfqpoint{3.468622in}{1.493166in}}%
\pgfpathlineto{\pgfqpoint{3.460582in}{1.485062in}}%
\pgfpathlineto{\pgfqpoint{3.452534in}{1.477098in}}%
\pgfpathlineto{\pgfqpoint{3.444477in}{1.469279in}}%
\pgfpathlineto{\pgfqpoint{3.436413in}{1.461610in}}%
\pgfpathclose%
\pgfusepath{fill}%
\end{pgfscope}%
\begin{pgfscope}%
\pgfpathrectangle{\pgfqpoint{1.254980in}{0.150000in}}{\pgfqpoint{5.490039in}{5.490039in}}%
\pgfusepath{clip}%
\pgfsetbuttcap%
\pgfsetroundjoin%
\definecolor{currentfill}{rgb}{0.128729,0.563265,0.551229}%
\pgfsetfillcolor{currentfill}%
\pgfsetfillopacity{0.700000}%
\pgfsetlinewidth{0.000000pt}%
\definecolor{currentstroke}{rgb}{0.000000,0.000000,0.000000}%
\pgfsetstrokecolor{currentstroke}%
\pgfsetdash{}{0pt}%
\pgfpathmoveto{\pgfqpoint{4.959546in}{2.706629in}}%
\pgfpathlineto{\pgfqpoint{4.973565in}{2.718476in}}%
\pgfpathlineto{\pgfqpoint{4.987602in}{2.730485in}}%
\pgfpathlineto{\pgfqpoint{5.001657in}{2.742655in}}%
\pgfpathlineto{\pgfqpoint{5.015730in}{2.754988in}}%
\pgfpathlineto{\pgfqpoint{5.023217in}{2.763519in}}%
\pgfpathlineto{\pgfqpoint{5.030696in}{2.771906in}}%
\pgfpathlineto{\pgfqpoint{5.038167in}{2.780148in}}%
\pgfpathlineto{\pgfqpoint{5.045630in}{2.788248in}}%
\pgfpathlineto{\pgfqpoint{5.031563in}{2.775990in}}%
\pgfpathlineto{\pgfqpoint{5.017514in}{2.763894in}}%
\pgfpathlineto{\pgfqpoint{5.003483in}{2.751960in}}%
\pgfpathlineto{\pgfqpoint{4.989469in}{2.740188in}}%
\pgfpathlineto{\pgfqpoint{4.982000in}{2.732002in}}%
\pgfpathlineto{\pgfqpoint{4.974523in}{2.723681in}}%
\pgfpathlineto{\pgfqpoint{4.967038in}{2.715224in}}%
\pgfpathlineto{\pgfqpoint{4.959546in}{2.706629in}}%
\pgfpathclose%
\pgfusepath{fill}%
\end{pgfscope}%
\begin{pgfscope}%
\pgfpathrectangle{\pgfqpoint{1.254980in}{0.150000in}}{\pgfqpoint{5.490039in}{5.490039in}}%
\pgfusepath{clip}%
\pgfsetbuttcap%
\pgfsetroundjoin%
\definecolor{currentfill}{rgb}{0.252194,0.269783,0.531579}%
\pgfsetfillcolor{currentfill}%
\pgfsetfillopacity{0.700000}%
\pgfsetlinewidth{0.000000pt}%
\definecolor{currentstroke}{rgb}{0.000000,0.000000,0.000000}%
\pgfsetstrokecolor{currentstroke}%
\pgfsetdash{}{0pt}%
\pgfpathmoveto{\pgfqpoint{4.236685in}{1.955564in}}%
\pgfpathlineto{\pgfqpoint{4.250298in}{1.961738in}}%
\pgfpathlineto{\pgfqpoint{4.263923in}{1.968073in}}%
\pgfpathlineto{\pgfqpoint{4.277561in}{1.974568in}}%
\pgfpathlineto{\pgfqpoint{4.291211in}{1.981224in}}%
\pgfpathlineto{\pgfqpoint{4.298977in}{1.994222in}}%
\pgfpathlineto{\pgfqpoint{4.306739in}{2.007153in}}%
\pgfpathlineto{\pgfqpoint{4.314496in}{2.020014in}}%
\pgfpathlineto{\pgfqpoint{4.322248in}{2.032804in}}%
\pgfpathlineto{\pgfqpoint{4.308598in}{2.025897in}}%
\pgfpathlineto{\pgfqpoint{4.294961in}{2.019151in}}%
\pgfpathlineto{\pgfqpoint{4.281337in}{2.012566in}}%
\pgfpathlineto{\pgfqpoint{4.267725in}{2.006141in}}%
\pgfpathlineto{\pgfqpoint{4.259972in}{1.993591in}}%
\pgfpathlineto{\pgfqpoint{4.252214in}{1.980977in}}%
\pgfpathlineto{\pgfqpoint{4.244452in}{1.968300in}}%
\pgfpathlineto{\pgfqpoint{4.236685in}{1.955564in}}%
\pgfpathclose%
\pgfusepath{fill}%
\end{pgfscope}%
\begin{pgfscope}%
\pgfpathrectangle{\pgfqpoint{1.254980in}{0.150000in}}{\pgfqpoint{5.490039in}{5.490039in}}%
\pgfusepath{clip}%
\pgfsetbuttcap%
\pgfsetroundjoin%
\definecolor{currentfill}{rgb}{0.278012,0.180367,0.486697}%
\pgfsetfillcolor{currentfill}%
\pgfsetfillopacity{0.700000}%
\pgfsetlinewidth{0.000000pt}%
\definecolor{currentstroke}{rgb}{0.000000,0.000000,0.000000}%
\pgfsetstrokecolor{currentstroke}%
\pgfsetdash{}{0pt}%
\pgfpathmoveto{\pgfqpoint{4.034601in}{1.762695in}}%
\pgfpathlineto{\pgfqpoint{4.048130in}{1.766665in}}%
\pgfpathlineto{\pgfqpoint{4.061668in}{1.770795in}}%
\pgfpathlineto{\pgfqpoint{4.075218in}{1.775086in}}%
\pgfpathlineto{\pgfqpoint{4.088778in}{1.779537in}}%
\pgfpathlineto{\pgfqpoint{4.096601in}{1.792450in}}%
\pgfpathlineto{\pgfqpoint{4.104419in}{1.805343in}}%
\pgfpathlineto{\pgfqpoint{4.112233in}{1.818211in}}%
\pgfpathlineto{\pgfqpoint{4.120042in}{1.831052in}}%
\pgfpathlineto{\pgfqpoint{4.106484in}{1.826266in}}%
\pgfpathlineto{\pgfqpoint{4.092937in}{1.821641in}}%
\pgfpathlineto{\pgfqpoint{4.079402in}{1.817177in}}%
\pgfpathlineto{\pgfqpoint{4.065877in}{1.812874in}}%
\pgfpathlineto{\pgfqpoint{4.058065in}{1.800356in}}%
\pgfpathlineto{\pgfqpoint{4.050248in}{1.787818in}}%
\pgfpathlineto{\pgfqpoint{4.042427in}{1.775264in}}%
\pgfpathlineto{\pgfqpoint{4.034601in}{1.762695in}}%
\pgfpathclose%
\pgfusepath{fill}%
\end{pgfscope}%
\begin{pgfscope}%
\pgfpathrectangle{\pgfqpoint{1.254980in}{0.150000in}}{\pgfqpoint{5.490039in}{5.490039in}}%
\pgfusepath{clip}%
\pgfsetbuttcap%
\pgfsetroundjoin%
\definecolor{currentfill}{rgb}{0.280267,0.073417,0.397163}%
\pgfsetfillcolor{currentfill}%
\pgfsetfillopacity{0.700000}%
\pgfsetlinewidth{0.000000pt}%
\definecolor{currentstroke}{rgb}{0.000000,0.000000,0.000000}%
\pgfsetstrokecolor{currentstroke}%
\pgfsetdash{}{0pt}%
\pgfpathmoveto{\pgfqpoint{2.962612in}{1.607766in}}%
\pgfpathlineto{\pgfqpoint{2.976052in}{1.597277in}}%
\pgfpathlineto{\pgfqpoint{2.989490in}{1.586979in}}%
\pgfpathlineto{\pgfqpoint{3.002927in}{1.576871in}}%
\pgfpathlineto{\pgfqpoint{3.016362in}{1.566951in}}%
\pgfpathlineto{\pgfqpoint{3.024724in}{1.568644in}}%
\pgfpathlineto{\pgfqpoint{3.033072in}{1.570600in}}%
\pgfpathlineto{\pgfqpoint{3.041406in}{1.572812in}}%
\pgfpathlineto{\pgfqpoint{3.049727in}{1.575274in}}%
\pgfpathlineto{\pgfqpoint{3.036326in}{1.584577in}}%
\pgfpathlineto{\pgfqpoint{3.022925in}{1.594068in}}%
\pgfpathlineto{\pgfqpoint{3.009523in}{1.603749in}}%
\pgfpathlineto{\pgfqpoint{2.996121in}{1.613619in}}%
\pgfpathlineto{\pgfqpoint{2.987765in}{1.611763in}}%
\pgfpathlineto{\pgfqpoint{2.979395in}{1.610164in}}%
\pgfpathlineto{\pgfqpoint{2.971011in}{1.608830in}}%
\pgfpathlineto{\pgfqpoint{2.962612in}{1.607766in}}%
\pgfpathclose%
\pgfusepath{fill}%
\end{pgfscope}%
\begin{pgfscope}%
\pgfpathrectangle{\pgfqpoint{1.254980in}{0.150000in}}{\pgfqpoint{5.490039in}{5.490039in}}%
\pgfusepath{clip}%
\pgfsetbuttcap%
\pgfsetroundjoin%
\definecolor{currentfill}{rgb}{0.210503,0.363727,0.552206}%
\pgfsetfillcolor{currentfill}%
\pgfsetfillopacity{0.700000}%
\pgfsetlinewidth{0.000000pt}%
\definecolor{currentstroke}{rgb}{0.000000,0.000000,0.000000}%
\pgfsetstrokecolor{currentstroke}%
\pgfsetdash{}{0pt}%
\pgfpathmoveto{\pgfqpoint{4.438819in}{2.163220in}}%
\pgfpathlineto{\pgfqpoint{4.452536in}{2.171346in}}%
\pgfpathlineto{\pgfqpoint{4.466266in}{2.179634in}}%
\pgfpathlineto{\pgfqpoint{4.480011in}{2.188082in}}%
\pgfpathlineto{\pgfqpoint{4.493770in}{2.196692in}}%
\pgfpathlineto{\pgfqpoint{4.501477in}{2.209107in}}%
\pgfpathlineto{\pgfqpoint{4.509179in}{2.221419in}}%
\pgfpathlineto{\pgfqpoint{4.516876in}{2.233627in}}%
\pgfpathlineto{\pgfqpoint{4.524567in}{2.245729in}}%
\pgfpathlineto{\pgfqpoint{4.510808in}{2.236954in}}%
\pgfpathlineto{\pgfqpoint{4.497064in}{2.228341in}}%
\pgfpathlineto{\pgfqpoint{4.483334in}{2.219888in}}%
\pgfpathlineto{\pgfqpoint{4.469618in}{2.211596in}}%
\pgfpathlineto{\pgfqpoint{4.461926in}{2.199649in}}%
\pgfpathlineto{\pgfqpoint{4.454229in}{2.187602in}}%
\pgfpathlineto{\pgfqpoint{4.446527in}{2.175459in}}%
\pgfpathlineto{\pgfqpoint{4.438819in}{2.163220in}}%
\pgfpathclose%
\pgfusepath{fill}%
\end{pgfscope}%
\begin{pgfscope}%
\pgfpathrectangle{\pgfqpoint{1.254980in}{0.150000in}}{\pgfqpoint{5.490039in}{5.490039in}}%
\pgfusepath{clip}%
\pgfsetbuttcap%
\pgfsetroundjoin%
\definecolor{currentfill}{rgb}{0.259857,0.745492,0.444467}%
\pgfsetfillcolor{currentfill}%
\pgfsetfillopacity{0.700000}%
\pgfsetlinewidth{0.000000pt}%
\definecolor{currentstroke}{rgb}{0.000000,0.000000,0.000000}%
\pgfsetstrokecolor{currentstroke}%
\pgfsetdash{}{0pt}%
\pgfpathmoveto{\pgfqpoint{5.535006in}{3.216038in}}%
\pgfpathlineto{\pgfqpoint{5.549401in}{3.230305in}}%
\pgfpathlineto{\pgfqpoint{5.563817in}{3.244735in}}%
\pgfpathlineto{\pgfqpoint{5.578253in}{3.259327in}}%
\pgfpathlineto{\pgfqpoint{5.592711in}{3.274082in}}%
\pgfpathlineto{\pgfqpoint{5.599834in}{3.277069in}}%
\pgfpathlineto{\pgfqpoint{5.606947in}{3.279940in}}%
\pgfpathlineto{\pgfqpoint{5.614051in}{3.282701in}}%
\pgfpathlineto{\pgfqpoint{5.621144in}{3.285355in}}%
\pgfpathlineto{\pgfqpoint{5.606706in}{3.270927in}}%
\pgfpathlineto{\pgfqpoint{5.592289in}{3.256661in}}%
\pgfpathlineto{\pgfqpoint{5.577893in}{3.242558in}}%
\pgfpathlineto{\pgfqpoint{5.563517in}{3.228615in}}%
\pgfpathlineto{\pgfqpoint{5.556403in}{3.225625in}}%
\pgfpathlineto{\pgfqpoint{5.549281in}{3.222534in}}%
\pgfpathlineto{\pgfqpoint{5.542148in}{3.219339in}}%
\pgfpathlineto{\pgfqpoint{5.535006in}{3.216038in}}%
\pgfpathclose%
\pgfusepath{fill}%
\end{pgfscope}%
\begin{pgfscope}%
\pgfpathrectangle{\pgfqpoint{1.254980in}{0.150000in}}{\pgfqpoint{5.490039in}{5.490039in}}%
\pgfusepath{clip}%
\pgfsetbuttcap%
\pgfsetroundjoin%
\definecolor{currentfill}{rgb}{0.165117,0.467423,0.558141}%
\pgfsetfillcolor{currentfill}%
\pgfsetfillopacity{0.700000}%
\pgfsetlinewidth{0.000000pt}%
\definecolor{currentstroke}{rgb}{0.000000,0.000000,0.000000}%
\pgfsetstrokecolor{currentstroke}%
\pgfsetdash{}{0pt}%
\pgfpathmoveto{\pgfqpoint{2.201577in}{2.540657in}}%
\pgfpathlineto{\pgfqpoint{2.215431in}{2.517156in}}%
\pgfpathlineto{\pgfqpoint{2.229270in}{2.493944in}}%
\pgfpathlineto{\pgfqpoint{2.243095in}{2.471017in}}%
\pgfpathlineto{\pgfqpoint{2.256907in}{2.448374in}}%
\pgfpathlineto{\pgfqpoint{2.265963in}{2.440968in}}%
\pgfpathlineto{\pgfqpoint{2.274993in}{2.433940in}}%
\pgfpathlineto{\pgfqpoint{2.283999in}{2.427284in}}%
\pgfpathlineto{\pgfqpoint{2.292981in}{2.420993in}}%
\pgfpathlineto{\pgfqpoint{2.279233in}{2.442967in}}%
\pgfpathlineto{\pgfqpoint{2.265471in}{2.465222in}}%
\pgfpathlineto{\pgfqpoint{2.251696in}{2.487760in}}%
\pgfpathlineto{\pgfqpoint{2.237908in}{2.510586in}}%
\pgfpathlineto{\pgfqpoint{2.228864in}{2.517536in}}%
\pgfpathlineto{\pgfqpoint{2.219794in}{2.524860in}}%
\pgfpathlineto{\pgfqpoint{2.210699in}{2.532564in}}%
\pgfpathlineto{\pgfqpoint{2.201577in}{2.540657in}}%
\pgfpathclose%
\pgfusepath{fill}%
\end{pgfscope}%
\begin{pgfscope}%
\pgfpathrectangle{\pgfqpoint{1.254980in}{0.150000in}}{\pgfqpoint{5.490039in}{5.490039in}}%
\pgfusepath{clip}%
\pgfsetbuttcap%
\pgfsetroundjoin%
\definecolor{currentfill}{rgb}{0.140210,0.665859,0.513427}%
\pgfsetfillcolor{currentfill}%
\pgfsetfillopacity{0.700000}%
\pgfsetlinewidth{0.000000pt}%
\definecolor{currentstroke}{rgb}{0.000000,0.000000,0.000000}%
\pgfsetstrokecolor{currentstroke}%
\pgfsetdash{}{0pt}%
\pgfpathmoveto{\pgfqpoint{5.247550in}{2.975508in}}%
\pgfpathlineto{\pgfqpoint{5.261761in}{2.988815in}}%
\pgfpathlineto{\pgfqpoint{5.275991in}{3.002285in}}%
\pgfpathlineto{\pgfqpoint{5.290240in}{3.015917in}}%
\pgfpathlineto{\pgfqpoint{5.304510in}{3.029712in}}%
\pgfpathlineto{\pgfqpoint{5.311834in}{3.035545in}}%
\pgfpathlineto{\pgfqpoint{5.319148in}{3.041238in}}%
\pgfpathlineto{\pgfqpoint{5.326453in}{3.046793in}}%
\pgfpathlineto{\pgfqpoint{5.333749in}{3.052213in}}%
\pgfpathlineto{\pgfqpoint{5.319491in}{3.038618in}}%
\pgfpathlineto{\pgfqpoint{5.305253in}{3.025186in}}%
\pgfpathlineto{\pgfqpoint{5.291035in}{3.011915in}}%
\pgfpathlineto{\pgfqpoint{5.276836in}{2.998807in}}%
\pgfpathlineto{\pgfqpoint{5.269528in}{2.993176in}}%
\pgfpathlineto{\pgfqpoint{5.262211in}{2.987417in}}%
\pgfpathlineto{\pgfqpoint{5.254885in}{2.981529in}}%
\pgfpathlineto{\pgfqpoint{5.247550in}{2.975508in}}%
\pgfpathclose%
\pgfusepath{fill}%
\end{pgfscope}%
\begin{pgfscope}%
\pgfpathrectangle{\pgfqpoint{1.254980in}{0.150000in}}{\pgfqpoint{5.490039in}{5.490039in}}%
\pgfusepath{clip}%
\pgfsetbuttcap%
\pgfsetroundjoin%
\definecolor{currentfill}{rgb}{0.279566,0.067836,0.391917}%
\pgfsetfillcolor{currentfill}%
\pgfsetfillopacity{0.700000}%
\pgfsetlinewidth{0.000000pt}%
\definecolor{currentstroke}{rgb}{0.000000,0.000000,0.000000}%
\pgfsetstrokecolor{currentstroke}%
\pgfsetdash{}{0pt}%
\pgfpathmoveto{\pgfqpoint{3.746880in}{1.546755in}}%
\pgfpathlineto{\pgfqpoint{3.760326in}{1.547175in}}%
\pgfpathlineto{\pgfqpoint{3.773779in}{1.547758in}}%
\pgfpathlineto{\pgfqpoint{3.787241in}{1.548504in}}%
\pgfpathlineto{\pgfqpoint{3.800710in}{1.549411in}}%
\pgfpathlineto{\pgfqpoint{3.808625in}{1.560819in}}%
\pgfpathlineto{\pgfqpoint{3.816534in}{1.572285in}}%
\pgfpathlineto{\pgfqpoint{3.824438in}{1.583806in}}%
\pgfpathlineto{\pgfqpoint{3.832338in}{1.595378in}}%
\pgfpathlineto{\pgfqpoint{3.818877in}{1.594027in}}%
\pgfpathlineto{\pgfqpoint{3.805425in}{1.592838in}}%
\pgfpathlineto{\pgfqpoint{3.791980in}{1.591813in}}%
\pgfpathlineto{\pgfqpoint{3.778544in}{1.590950in}}%
\pgfpathlineto{\pgfqpoint{3.770636in}{1.579811in}}%
\pgfpathlineto{\pgfqpoint{3.762723in}{1.568729in}}%
\pgfpathlineto{\pgfqpoint{3.754805in}{1.557709in}}%
\pgfpathlineto{\pgfqpoint{3.746880in}{1.546755in}}%
\pgfpathclose%
\pgfusepath{fill}%
\end{pgfscope}%
\begin{pgfscope}%
\pgfpathrectangle{\pgfqpoint{1.254980in}{0.150000in}}{\pgfqpoint{5.490039in}{5.490039in}}%
\pgfusepath{clip}%
\pgfsetbuttcap%
\pgfsetroundjoin%
\definecolor{currentfill}{rgb}{0.274952,0.037752,0.364543}%
\pgfsetfillcolor{currentfill}%
\pgfsetfillopacity{0.700000}%
\pgfsetlinewidth{0.000000pt}%
\definecolor{currentstroke}{rgb}{0.000000,0.000000,0.000000}%
\pgfsetstrokecolor{currentstroke}%
\pgfsetdash{}{0pt}%
\pgfpathmoveto{\pgfqpoint{3.661372in}{1.505512in}}%
\pgfpathlineto{\pgfqpoint{3.674800in}{1.504809in}}%
\pgfpathlineto{\pgfqpoint{3.688235in}{1.504269in}}%
\pgfpathlineto{\pgfqpoint{3.701677in}{1.503893in}}%
\pgfpathlineto{\pgfqpoint{3.715126in}{1.503680in}}%
\pgfpathlineto{\pgfqpoint{3.723073in}{1.514329in}}%
\pgfpathlineto{\pgfqpoint{3.731015in}{1.525060in}}%
\pgfpathlineto{\pgfqpoint{3.738950in}{1.535870in}}%
\pgfpathlineto{\pgfqpoint{3.746880in}{1.546755in}}%
\pgfpathlineto{\pgfqpoint{3.733442in}{1.546498in}}%
\pgfpathlineto{\pgfqpoint{3.720011in}{1.546404in}}%
\pgfpathlineto{\pgfqpoint{3.706588in}{1.546473in}}%
\pgfpathlineto{\pgfqpoint{3.693171in}{1.546707in}}%
\pgfpathlineto{\pgfqpoint{3.685230in}{1.536282in}}%
\pgfpathlineto{\pgfqpoint{3.677284in}{1.525938in}}%
\pgfpathlineto{\pgfqpoint{3.669331in}{1.515680in}}%
\pgfpathlineto{\pgfqpoint{3.661372in}{1.505512in}}%
\pgfpathclose%
\pgfusepath{fill}%
\end{pgfscope}%
\begin{pgfscope}%
\pgfpathrectangle{\pgfqpoint{1.254980in}{0.150000in}}{\pgfqpoint{5.490039in}{5.490039in}}%
\pgfusepath{clip}%
\pgfsetbuttcap%
\pgfsetroundjoin%
\definecolor{currentfill}{rgb}{0.174274,0.445044,0.557792}%
\pgfsetfillcolor{currentfill}%
\pgfsetfillopacity{0.700000}%
\pgfsetlinewidth{0.000000pt}%
\definecolor{currentstroke}{rgb}{0.000000,0.000000,0.000000}%
\pgfsetstrokecolor{currentstroke}%
\pgfsetdash{}{0pt}%
\pgfpathmoveto{\pgfqpoint{4.641076in}{2.376296in}}%
\pgfpathlineto{\pgfqpoint{4.654909in}{2.386119in}}%
\pgfpathlineto{\pgfqpoint{4.668758in}{2.396103in}}%
\pgfpathlineto{\pgfqpoint{4.682623in}{2.406249in}}%
\pgfpathlineto{\pgfqpoint{4.696504in}{2.416556in}}%
\pgfpathlineto{\pgfqpoint{4.704142in}{2.427840in}}%
\pgfpathlineto{\pgfqpoint{4.711773in}{2.438995in}}%
\pgfpathlineto{\pgfqpoint{4.719399in}{2.450021in}}%
\pgfpathlineto{\pgfqpoint{4.727018in}{2.460917in}}%
\pgfpathlineto{\pgfqpoint{4.713138in}{2.450533in}}%
\pgfpathlineto{\pgfqpoint{4.699275in}{2.440310in}}%
\pgfpathlineto{\pgfqpoint{4.685427in}{2.430249in}}%
\pgfpathlineto{\pgfqpoint{4.671596in}{2.420349in}}%
\pgfpathlineto{\pgfqpoint{4.663975in}{2.409519in}}%
\pgfpathlineto{\pgfqpoint{4.656348in}{2.398566in}}%
\pgfpathlineto{\pgfqpoint{4.648715in}{2.387492in}}%
\pgfpathlineto{\pgfqpoint{4.641076in}{2.376296in}}%
\pgfpathclose%
\pgfusepath{fill}%
\end{pgfscope}%
\begin{pgfscope}%
\pgfpathrectangle{\pgfqpoint{1.254980in}{0.150000in}}{\pgfqpoint{5.490039in}{5.490039in}}%
\pgfusepath{clip}%
\pgfsetbuttcap%
\pgfsetroundjoin%
\definecolor{currentfill}{rgb}{0.269308,0.218818,0.509577}%
\pgfsetfillcolor{currentfill}%
\pgfsetfillopacity{0.700000}%
\pgfsetlinewidth{0.000000pt}%
\definecolor{currentstroke}{rgb}{0.000000,0.000000,0.000000}%
\pgfsetstrokecolor{currentstroke}%
\pgfsetdash{}{0pt}%
\pgfpathmoveto{\pgfqpoint{4.120042in}{1.831052in}}%
\pgfpathlineto{\pgfqpoint{4.133611in}{1.835998in}}%
\pgfpathlineto{\pgfqpoint{4.147191in}{1.841105in}}%
\pgfpathlineto{\pgfqpoint{4.160783in}{1.846373in}}%
\pgfpathlineto{\pgfqpoint{4.174386in}{1.851800in}}%
\pgfpathlineto{\pgfqpoint{4.182189in}{1.864929in}}%
\pgfpathlineto{\pgfqpoint{4.189988in}{1.878018in}}%
\pgfpathlineto{\pgfqpoint{4.197782in}{1.891064in}}%
\pgfpathlineto{\pgfqpoint{4.205571in}{1.904065in}}%
\pgfpathlineto{\pgfqpoint{4.191969in}{1.898330in}}%
\pgfpathlineto{\pgfqpoint{4.178379in}{1.892756in}}%
\pgfpathlineto{\pgfqpoint{4.164801in}{1.887342in}}%
\pgfpathlineto{\pgfqpoint{4.151234in}{1.882089in}}%
\pgfpathlineto{\pgfqpoint{4.143443in}{1.869385in}}%
\pgfpathlineto{\pgfqpoint{4.135647in}{1.856642in}}%
\pgfpathlineto{\pgfqpoint{4.127847in}{1.843863in}}%
\pgfpathlineto{\pgfqpoint{4.120042in}{1.831052in}}%
\pgfpathclose%
\pgfusepath{fill}%
\end{pgfscope}%
\begin{pgfscope}%
\pgfpathrectangle{\pgfqpoint{1.254980in}{0.150000in}}{\pgfqpoint{5.490039in}{5.490039in}}%
\pgfusepath{clip}%
\pgfsetbuttcap%
\pgfsetroundjoin%
\definecolor{currentfill}{rgb}{0.267004,0.004874,0.329415}%
\pgfsetfillcolor{currentfill}%
\pgfsetfillopacity{0.700000}%
\pgfsetlinewidth{0.000000pt}%
\definecolor{currentstroke}{rgb}{0.000000,0.000000,0.000000}%
\pgfsetstrokecolor{currentstroke}%
\pgfsetdash{}{0pt}%
\pgfpathmoveto{\pgfqpoint{3.350444in}{1.451347in}}%
\pgfpathlineto{\pgfqpoint{3.363845in}{1.446390in}}%
\pgfpathlineto{\pgfqpoint{3.377250in}{1.441603in}}%
\pgfpathlineto{\pgfqpoint{3.390658in}{1.436987in}}%
\pgfpathlineto{\pgfqpoint{3.404069in}{1.432542in}}%
\pgfpathlineto{\pgfqpoint{3.412168in}{1.439557in}}%
\pgfpathlineto{\pgfqpoint{3.420258in}{1.446744in}}%
\pgfpathlineto{\pgfqpoint{3.428340in}{1.454097in}}%
\pgfpathlineto{\pgfqpoint{3.436413in}{1.461610in}}%
\pgfpathlineto{\pgfqpoint{3.423022in}{1.465503in}}%
\pgfpathlineto{\pgfqpoint{3.409636in}{1.469566in}}%
\pgfpathlineto{\pgfqpoint{3.396253in}{1.473799in}}%
\pgfpathlineto{\pgfqpoint{3.382874in}{1.478204in}}%
\pgfpathlineto{\pgfqpoint{3.374780in}{1.471233in}}%
\pgfpathlineto{\pgfqpoint{3.366677in}{1.464429in}}%
\pgfpathlineto{\pgfqpoint{3.358565in}{1.457799in}}%
\pgfpathlineto{\pgfqpoint{3.350444in}{1.451347in}}%
\pgfpathclose%
\pgfusepath{fill}%
\end{pgfscope}%
\begin{pgfscope}%
\pgfpathrectangle{\pgfqpoint{1.254980in}{0.150000in}}{\pgfqpoint{5.490039in}{5.490039in}}%
\pgfusepath{clip}%
\pgfsetbuttcap%
\pgfsetroundjoin%
\definecolor{currentfill}{rgb}{0.282327,0.094955,0.417331}%
\pgfsetfillcolor{currentfill}%
\pgfsetfillopacity{0.700000}%
\pgfsetlinewidth{0.000000pt}%
\definecolor{currentstroke}{rgb}{0.000000,0.000000,0.000000}%
\pgfsetstrokecolor{currentstroke}%
\pgfsetdash{}{0pt}%
\pgfpathmoveto{\pgfqpoint{3.832338in}{1.595378in}}%
\pgfpathlineto{\pgfqpoint{3.845807in}{1.596890in}}%
\pgfpathlineto{\pgfqpoint{3.859284in}{1.598565in}}%
\pgfpathlineto{\pgfqpoint{3.872771in}{1.600401in}}%
\pgfpathlineto{\pgfqpoint{3.886266in}{1.602398in}}%
\pgfpathlineto{\pgfqpoint{3.894152in}{1.614442in}}%
\pgfpathlineto{\pgfqpoint{3.902034in}{1.626521in}}%
\pgfpathlineto{\pgfqpoint{3.909911in}{1.638632in}}%
\pgfpathlineto{\pgfqpoint{3.917783in}{1.650770in}}%
\pgfpathlineto{\pgfqpoint{3.904295in}{1.648356in}}%
\pgfpathlineto{\pgfqpoint{3.890815in}{1.646104in}}%
\pgfpathlineto{\pgfqpoint{3.877345in}{1.644014in}}%
\pgfpathlineto{\pgfqpoint{3.863883in}{1.642086in}}%
\pgfpathlineto{\pgfqpoint{3.856004in}{1.630353in}}%
\pgfpathlineto{\pgfqpoint{3.848120in}{1.618655in}}%
\pgfpathlineto{\pgfqpoint{3.840232in}{1.606995in}}%
\pgfpathlineto{\pgfqpoint{3.832338in}{1.595378in}}%
\pgfpathclose%
\pgfusepath{fill}%
\end{pgfscope}%
\begin{pgfscope}%
\pgfpathrectangle{\pgfqpoint{1.254980in}{0.150000in}}{\pgfqpoint{5.490039in}{5.490039in}}%
\pgfusepath{clip}%
\pgfsetbuttcap%
\pgfsetroundjoin%
\definecolor{currentfill}{rgb}{0.277941,0.056324,0.381191}%
\pgfsetfillcolor{currentfill}%
\pgfsetfillopacity{0.700000}%
\pgfsetlinewidth{0.000000pt}%
\definecolor{currentstroke}{rgb}{0.000000,0.000000,0.000000}%
\pgfsetstrokecolor{currentstroke}%
\pgfsetdash{}{0pt}%
\pgfpathmoveto{\pgfqpoint{3.016362in}{1.566951in}}%
\pgfpathlineto{\pgfqpoint{3.029797in}{1.557220in}}%
\pgfpathlineto{\pgfqpoint{3.043232in}{1.547675in}}%
\pgfpathlineto{\pgfqpoint{3.056665in}{1.538316in}}%
\pgfpathlineto{\pgfqpoint{3.070099in}{1.529142in}}%
\pgfpathlineto{\pgfqpoint{3.078425in}{1.531461in}}%
\pgfpathlineto{\pgfqpoint{3.086738in}{1.534037in}}%
\pgfpathlineto{\pgfqpoint{3.095038in}{1.536861in}}%
\pgfpathlineto{\pgfqpoint{3.103326in}{1.539928in}}%
\pgfpathlineto{\pgfqpoint{3.089926in}{1.548486in}}%
\pgfpathlineto{\pgfqpoint{3.076526in}{1.557230in}}%
\pgfpathlineto{\pgfqpoint{3.063127in}{1.566159in}}%
\pgfpathlineto{\pgfqpoint{3.049727in}{1.575274in}}%
\pgfpathlineto{\pgfqpoint{3.041406in}{1.572812in}}%
\pgfpathlineto{\pgfqpoint{3.033072in}{1.570600in}}%
\pgfpathlineto{\pgfqpoint{3.024724in}{1.568644in}}%
\pgfpathlineto{\pgfqpoint{3.016362in}{1.566951in}}%
\pgfpathclose%
\pgfusepath{fill}%
\end{pgfscope}%
\begin{pgfscope}%
\pgfpathrectangle{\pgfqpoint{1.254980in}{0.150000in}}{\pgfqpoint{5.490039in}{5.490039in}}%
\pgfusepath{clip}%
\pgfsetbuttcap%
\pgfsetroundjoin%
\definecolor{currentfill}{rgb}{0.268510,0.009605,0.335427}%
\pgfsetfillcolor{currentfill}%
\pgfsetfillopacity{0.700000}%
\pgfsetlinewidth{0.000000pt}%
\definecolor{currentstroke}{rgb}{0.000000,0.000000,0.000000}%
\pgfsetstrokecolor{currentstroke}%
\pgfsetdash{}{0pt}%
\pgfpathmoveto{\pgfqpoint{3.210545in}{1.478004in}}%
\pgfpathlineto{\pgfqpoint{3.223953in}{1.471070in}}%
\pgfpathlineto{\pgfqpoint{3.237362in}{1.464313in}}%
\pgfpathlineto{\pgfqpoint{3.250773in}{1.457731in}}%
\pgfpathlineto{\pgfqpoint{3.264186in}{1.451325in}}%
\pgfpathlineto{\pgfqpoint{3.272373in}{1.456413in}}%
\pgfpathlineto{\pgfqpoint{3.280549in}{1.461709in}}%
\pgfpathlineto{\pgfqpoint{3.288714in}{1.467208in}}%
\pgfpathlineto{\pgfqpoint{3.296869in}{1.472904in}}%
\pgfpathlineto{\pgfqpoint{3.283483in}{1.478728in}}%
\pgfpathlineto{\pgfqpoint{3.270098in}{1.484727in}}%
\pgfpathlineto{\pgfqpoint{3.256716in}{1.490901in}}%
\pgfpathlineto{\pgfqpoint{3.243336in}{1.497252in}}%
\pgfpathlineto{\pgfqpoint{3.235154in}{1.492128in}}%
\pgfpathlineto{\pgfqpoint{3.226962in}{1.487208in}}%
\pgfpathlineto{\pgfqpoint{3.218759in}{1.482498in}}%
\pgfpathlineto{\pgfqpoint{3.210545in}{1.478004in}}%
\pgfpathclose%
\pgfusepath{fill}%
\end{pgfscope}%
\begin{pgfscope}%
\pgfpathrectangle{\pgfqpoint{1.254980in}{0.150000in}}{\pgfqpoint{5.490039in}{5.490039in}}%
\pgfusepath{clip}%
\pgfsetbuttcap%
\pgfsetroundjoin%
\definecolor{currentfill}{rgb}{0.271305,0.019942,0.347269}%
\pgfsetfillcolor{currentfill}%
\pgfsetfillopacity{0.700000}%
\pgfsetlinewidth{0.000000pt}%
\definecolor{currentstroke}{rgb}{0.000000,0.000000,0.000000}%
\pgfsetstrokecolor{currentstroke}%
\pgfsetdash{}{0pt}%
\pgfpathmoveto{\pgfqpoint{3.575767in}{1.472283in}}%
\pgfpathlineto{\pgfqpoint{3.589185in}{1.470424in}}%
\pgfpathlineto{\pgfqpoint{3.602608in}{1.468729in}}%
\pgfpathlineto{\pgfqpoint{3.616037in}{1.467200in}}%
\pgfpathlineto{\pgfqpoint{3.629472in}{1.465835in}}%
\pgfpathlineto{\pgfqpoint{3.637457in}{1.475595in}}%
\pgfpathlineto{\pgfqpoint{3.645435in}{1.485465in}}%
\pgfpathlineto{\pgfqpoint{3.653406in}{1.495439in}}%
\pgfpathlineto{\pgfqpoint{3.661372in}{1.505512in}}%
\pgfpathlineto{\pgfqpoint{3.647950in}{1.506380in}}%
\pgfpathlineto{\pgfqpoint{3.634535in}{1.507412in}}%
\pgfpathlineto{\pgfqpoint{3.621126in}{1.508609in}}%
\pgfpathlineto{\pgfqpoint{3.607723in}{1.509972in}}%
\pgfpathlineto{\pgfqpoint{3.599744in}{1.500385in}}%
\pgfpathlineto{\pgfqpoint{3.591759in}{1.490905in}}%
\pgfpathlineto{\pgfqpoint{3.583766in}{1.481536in}}%
\pgfpathlineto{\pgfqpoint{3.575767in}{1.472283in}}%
\pgfpathclose%
\pgfusepath{fill}%
\end{pgfscope}%
\begin{pgfscope}%
\pgfpathrectangle{\pgfqpoint{1.254980in}{0.150000in}}{\pgfqpoint{5.490039in}{5.490039in}}%
\pgfusepath{clip}%
\pgfsetbuttcap%
\pgfsetroundjoin%
\definecolor{currentfill}{rgb}{0.235526,0.309527,0.542944}%
\pgfsetfillcolor{currentfill}%
\pgfsetfillopacity{0.700000}%
\pgfsetlinewidth{0.000000pt}%
\definecolor{currentstroke}{rgb}{0.000000,0.000000,0.000000}%
\pgfsetstrokecolor{currentstroke}%
\pgfsetdash{}{0pt}%
\pgfpathmoveto{\pgfqpoint{4.322248in}{2.032804in}}%
\pgfpathlineto{\pgfqpoint{4.335911in}{2.039871in}}%
\pgfpathlineto{\pgfqpoint{4.349587in}{2.047099in}}%
\pgfpathlineto{\pgfqpoint{4.363276in}{2.054488in}}%
\pgfpathlineto{\pgfqpoint{4.376979in}{2.062037in}}%
\pgfpathlineto{\pgfqpoint{4.384726in}{2.074987in}}%
\pgfpathlineto{\pgfqpoint{4.392468in}{2.087854in}}%
\pgfpathlineto{\pgfqpoint{4.400206in}{2.100637in}}%
\pgfpathlineto{\pgfqpoint{4.407939in}{2.113333in}}%
\pgfpathlineto{\pgfqpoint{4.394236in}{2.105561in}}%
\pgfpathlineto{\pgfqpoint{4.380547in}{2.097950in}}%
\pgfpathlineto{\pgfqpoint{4.366871in}{2.090499in}}%
\pgfpathlineto{\pgfqpoint{4.353209in}{2.083209in}}%
\pgfpathlineto{\pgfqpoint{4.345476in}{2.070725in}}%
\pgfpathlineto{\pgfqpoint{4.337738in}{2.058161in}}%
\pgfpathlineto{\pgfqpoint{4.329995in}{2.045520in}}%
\pgfpathlineto{\pgfqpoint{4.322248in}{2.032804in}}%
\pgfpathclose%
\pgfusepath{fill}%
\end{pgfscope}%
\begin{pgfscope}%
\pgfpathrectangle{\pgfqpoint{1.254980in}{0.150000in}}{\pgfqpoint{5.490039in}{5.490039in}}%
\pgfusepath{clip}%
\pgfsetbuttcap%
\pgfsetroundjoin%
\definecolor{currentfill}{rgb}{0.319809,0.770914,0.411152}%
\pgfsetfillcolor{currentfill}%
\pgfsetfillopacity{0.700000}%
\pgfsetlinewidth{0.000000pt}%
\definecolor{currentstroke}{rgb}{0.000000,0.000000,0.000000}%
\pgfsetstrokecolor{currentstroke}%
\pgfsetdash{}{0pt}%
\pgfpathmoveto{\pgfqpoint{5.621144in}{3.285355in}}%
\pgfpathlineto{\pgfqpoint{5.635604in}{3.299945in}}%
\pgfpathlineto{\pgfqpoint{5.650085in}{3.314697in}}%
\pgfpathlineto{\pgfqpoint{5.664587in}{3.329613in}}%
\pgfpathlineto{\pgfqpoint{5.679111in}{3.344691in}}%
\pgfpathlineto{\pgfqpoint{5.686174in}{3.346894in}}%
\pgfpathlineto{\pgfqpoint{5.693227in}{3.348989in}}%
\pgfpathlineto{\pgfqpoint{5.700270in}{3.350982in}}%
\pgfpathlineto{\pgfqpoint{5.707302in}{3.352876in}}%
\pgfpathlineto{\pgfqpoint{5.692800in}{3.338158in}}%
\pgfpathlineto{\pgfqpoint{5.678320in}{3.323602in}}%
\pgfpathlineto{\pgfqpoint{5.663860in}{3.309207in}}%
\pgfpathlineto{\pgfqpoint{5.649422in}{3.294975in}}%
\pgfpathlineto{\pgfqpoint{5.642367in}{3.292711in}}%
\pgfpathlineto{\pgfqpoint{5.635303in}{3.290356in}}%
\pgfpathlineto{\pgfqpoint{5.628228in}{3.287905in}}%
\pgfpathlineto{\pgfqpoint{5.621144in}{3.285355in}}%
\pgfpathclose%
\pgfusepath{fill}%
\end{pgfscope}%
\begin{pgfscope}%
\pgfpathrectangle{\pgfqpoint{1.254980in}{0.150000in}}{\pgfqpoint{5.490039in}{5.490039in}}%
\pgfusepath{clip}%
\pgfsetbuttcap%
\pgfsetroundjoin%
\definecolor{currentfill}{rgb}{0.141935,0.526453,0.555991}%
\pgfsetfillcolor{currentfill}%
\pgfsetfillopacity{0.700000}%
\pgfsetlinewidth{0.000000pt}%
\definecolor{currentstroke}{rgb}{0.000000,0.000000,0.000000}%
\pgfsetstrokecolor{currentstroke}%
\pgfsetdash{}{0pt}%
\pgfpathmoveto{\pgfqpoint{4.843398in}{2.586821in}}%
\pgfpathlineto{\pgfqpoint{4.857356in}{2.598078in}}%
\pgfpathlineto{\pgfqpoint{4.871331in}{2.609496in}}%
\pgfpathlineto{\pgfqpoint{4.885323in}{2.621076in}}%
\pgfpathlineto{\pgfqpoint{4.899333in}{2.632818in}}%
\pgfpathlineto{\pgfqpoint{4.906886in}{2.642543in}}%
\pgfpathlineto{\pgfqpoint{4.914431in}{2.652124in}}%
\pgfpathlineto{\pgfqpoint{4.921969in}{2.661562in}}%
\pgfpathlineto{\pgfqpoint{4.929500in}{2.670858in}}%
\pgfpathlineto{\pgfqpoint{4.915493in}{2.659129in}}%
\pgfpathlineto{\pgfqpoint{4.901505in}{2.647563in}}%
\pgfpathlineto{\pgfqpoint{4.887533in}{2.636158in}}%
\pgfpathlineto{\pgfqpoint{4.873579in}{2.624915in}}%
\pgfpathlineto{\pgfqpoint{4.866045in}{2.615594in}}%
\pgfpathlineto{\pgfqpoint{4.858503in}{2.606139in}}%
\pgfpathlineto{\pgfqpoint{4.850954in}{2.596548in}}%
\pgfpathlineto{\pgfqpoint{4.843398in}{2.586821in}}%
\pgfpathclose%
\pgfusepath{fill}%
\end{pgfscope}%
\begin{pgfscope}%
\pgfpathrectangle{\pgfqpoint{1.254980in}{0.150000in}}{\pgfqpoint{5.490039in}{5.490039in}}%
\pgfusepath{clip}%
\pgfsetbuttcap%
\pgfsetroundjoin%
\definecolor{currentfill}{rgb}{0.120092,0.600104,0.542530}%
\pgfsetfillcolor{currentfill}%
\pgfsetfillopacity{0.700000}%
\pgfsetlinewidth{0.000000pt}%
\definecolor{currentstroke}{rgb}{0.000000,0.000000,0.000000}%
\pgfsetstrokecolor{currentstroke}%
\pgfsetdash{}{0pt}%
\pgfpathmoveto{\pgfqpoint{5.045630in}{2.788248in}}%
\pgfpathlineto{\pgfqpoint{5.059715in}{2.800668in}}%
\pgfpathlineto{\pgfqpoint{5.073819in}{2.813251in}}%
\pgfpathlineto{\pgfqpoint{5.087941in}{2.825996in}}%
\pgfpathlineto{\pgfqpoint{5.102082in}{2.838903in}}%
\pgfpathlineto{\pgfqpoint{5.109530in}{2.846767in}}%
\pgfpathlineto{\pgfqpoint{5.116970in}{2.854484in}}%
\pgfpathlineto{\pgfqpoint{5.124402in}{2.862055in}}%
\pgfpathlineto{\pgfqpoint{5.131824in}{2.869481in}}%
\pgfpathlineto{\pgfqpoint{5.117691in}{2.856680in}}%
\pgfpathlineto{\pgfqpoint{5.103576in}{2.844041in}}%
\pgfpathlineto{\pgfqpoint{5.089479in}{2.831565in}}%
\pgfpathlineto{\pgfqpoint{5.075401in}{2.819250in}}%
\pgfpathlineto{\pgfqpoint{5.067970in}{2.811706in}}%
\pgfpathlineto{\pgfqpoint{5.060532in}{2.804026in}}%
\pgfpathlineto{\pgfqpoint{5.053085in}{2.796207in}}%
\pgfpathlineto{\pgfqpoint{5.045630in}{2.788248in}}%
\pgfpathclose%
\pgfusepath{fill}%
\end{pgfscope}%
\begin{pgfscope}%
\pgfpathrectangle{\pgfqpoint{1.254980in}{0.150000in}}{\pgfqpoint{5.490039in}{5.490039in}}%
\pgfusepath{clip}%
\pgfsetbuttcap%
\pgfsetroundjoin%
\definecolor{currentfill}{rgb}{0.283187,0.125848,0.444960}%
\pgfsetfillcolor{currentfill}%
\pgfsetfillopacity{0.700000}%
\pgfsetlinewidth{0.000000pt}%
\definecolor{currentstroke}{rgb}{0.000000,0.000000,0.000000}%
\pgfsetstrokecolor{currentstroke}%
\pgfsetdash{}{0pt}%
\pgfpathmoveto{\pgfqpoint{3.917783in}{1.650770in}}%
\pgfpathlineto{\pgfqpoint{3.931281in}{1.653345in}}%
\pgfpathlineto{\pgfqpoint{3.944789in}{1.656081in}}%
\pgfpathlineto{\pgfqpoint{3.958306in}{1.658978in}}%
\pgfpathlineto{\pgfqpoint{3.971832in}{1.662035in}}%
\pgfpathlineto{\pgfqpoint{3.979694in}{1.674596in}}%
\pgfpathlineto{\pgfqpoint{3.987552in}{1.687171in}}%
\pgfpathlineto{\pgfqpoint{3.995405in}{1.699755in}}%
\pgfpathlineto{\pgfqpoint{4.003253in}{1.712345in}}%
\pgfpathlineto{\pgfqpoint{3.989731in}{1.708898in}}%
\pgfpathlineto{\pgfqpoint{3.976219in}{1.705612in}}%
\pgfpathlineto{\pgfqpoint{3.962717in}{1.702487in}}%
\pgfpathlineto{\pgfqpoint{3.949224in}{1.699524in}}%
\pgfpathlineto{\pgfqpoint{3.941371in}{1.687312in}}%
\pgfpathlineto{\pgfqpoint{3.933513in}{1.675114in}}%
\pgfpathlineto{\pgfqpoint{3.925651in}{1.662932in}}%
\pgfpathlineto{\pgfqpoint{3.917783in}{1.650770in}}%
\pgfpathclose%
\pgfusepath{fill}%
\end{pgfscope}%
\begin{pgfscope}%
\pgfpathrectangle{\pgfqpoint{1.254980in}{0.150000in}}{\pgfqpoint{5.490039in}{5.490039in}}%
\pgfusepath{clip}%
\pgfsetbuttcap%
\pgfsetroundjoin%
\definecolor{currentfill}{rgb}{0.265145,0.232956,0.516599}%
\pgfsetfillcolor{currentfill}%
\pgfsetfillopacity{0.700000}%
\pgfsetlinewidth{0.000000pt}%
\definecolor{currentstroke}{rgb}{0.000000,0.000000,0.000000}%
\pgfsetstrokecolor{currentstroke}%
\pgfsetdash{}{0pt}%
\pgfpathmoveto{\pgfqpoint{2.604321in}{1.934026in}}%
\pgfpathlineto{\pgfqpoint{2.617901in}{1.917940in}}%
\pgfpathlineto{\pgfqpoint{2.631476in}{1.902076in}}%
\pgfpathlineto{\pgfqpoint{2.645044in}{1.886432in}}%
\pgfpathlineto{\pgfqpoint{2.658605in}{1.871007in}}%
\pgfpathlineto{\pgfqpoint{2.667294in}{1.867583in}}%
\pgfpathlineto{\pgfqpoint{2.675963in}{1.864496in}}%
\pgfpathlineto{\pgfqpoint{2.684612in}{1.861739in}}%
\pgfpathlineto{\pgfqpoint{2.693243in}{1.859305in}}%
\pgfpathlineto{\pgfqpoint{2.679731in}{1.874066in}}%
\pgfpathlineto{\pgfqpoint{2.666213in}{1.889044in}}%
\pgfpathlineto{\pgfqpoint{2.652689in}{1.904242in}}%
\pgfpathlineto{\pgfqpoint{2.639160in}{1.919660in}}%
\pgfpathlineto{\pgfqpoint{2.630480in}{1.922748in}}%
\pgfpathlineto{\pgfqpoint{2.621781in}{1.926166in}}%
\pgfpathlineto{\pgfqpoint{2.613061in}{1.929924in}}%
\pgfpathlineto{\pgfqpoint{2.604321in}{1.934026in}}%
\pgfpathclose%
\pgfusepath{fill}%
\end{pgfscope}%
\begin{pgfscope}%
\pgfpathrectangle{\pgfqpoint{1.254980in}{0.150000in}}{\pgfqpoint{5.490039in}{5.490039in}}%
\pgfusepath{clip}%
\pgfsetbuttcap%
\pgfsetroundjoin%
\definecolor{currentfill}{rgb}{0.255645,0.260703,0.528312}%
\pgfsetfillcolor{currentfill}%
\pgfsetfillopacity{0.700000}%
\pgfsetlinewidth{0.000000pt}%
\definecolor{currentstroke}{rgb}{0.000000,0.000000,0.000000}%
\pgfsetstrokecolor{currentstroke}%
\pgfsetdash{}{0pt}%
\pgfpathmoveto{\pgfqpoint{2.549927in}{2.000624in}}%
\pgfpathlineto{\pgfqpoint{2.563536in}{1.983633in}}%
\pgfpathlineto{\pgfqpoint{2.577138in}{1.966871in}}%
\pgfpathlineto{\pgfqpoint{2.590733in}{1.950336in}}%
\pgfpathlineto{\pgfqpoint{2.604321in}{1.934026in}}%
\pgfpathlineto{\pgfqpoint{2.613061in}{1.929924in}}%
\pgfpathlineto{\pgfqpoint{2.621781in}{1.926166in}}%
\pgfpathlineto{\pgfqpoint{2.630480in}{1.922748in}}%
\pgfpathlineto{\pgfqpoint{2.639160in}{1.919660in}}%
\pgfpathlineto{\pgfqpoint{2.625623in}{1.935302in}}%
\pgfpathlineto{\pgfqpoint{2.612081in}{1.951167in}}%
\pgfpathlineto{\pgfqpoint{2.598532in}{1.967259in}}%
\pgfpathlineto{\pgfqpoint{2.584975in}{1.983578in}}%
\pgfpathlineto{\pgfqpoint{2.576245in}{1.987323in}}%
\pgfpathlineto{\pgfqpoint{2.567493in}{1.991407in}}%
\pgfpathlineto{\pgfqpoint{2.558721in}{1.995839in}}%
\pgfpathlineto{\pgfqpoint{2.549927in}{2.000624in}}%
\pgfpathclose%
\pgfusepath{fill}%
\end{pgfscope}%
\begin{pgfscope}%
\pgfpathrectangle{\pgfqpoint{1.254980in}{0.150000in}}{\pgfqpoint{5.490039in}{5.490039in}}%
\pgfusepath{clip}%
\pgfsetbuttcap%
\pgfsetroundjoin%
\definecolor{currentfill}{rgb}{0.273006,0.204520,0.501721}%
\pgfsetfillcolor{currentfill}%
\pgfsetfillopacity{0.700000}%
\pgfsetlinewidth{0.000000pt}%
\definecolor{currentstroke}{rgb}{0.000000,0.000000,0.000000}%
\pgfsetstrokecolor{currentstroke}%
\pgfsetdash{}{0pt}%
\pgfpathmoveto{\pgfqpoint{2.658605in}{1.871007in}}%
\pgfpathlineto{\pgfqpoint{2.672161in}{1.855800in}}%
\pgfpathlineto{\pgfqpoint{2.685711in}{1.840809in}}%
\pgfpathlineto{\pgfqpoint{2.699256in}{1.826032in}}%
\pgfpathlineto{\pgfqpoint{2.712795in}{1.811468in}}%
\pgfpathlineto{\pgfqpoint{2.721434in}{1.808718in}}%
\pgfpathlineto{\pgfqpoint{2.730054in}{1.806297in}}%
\pgfpathlineto{\pgfqpoint{2.738656in}{1.804199in}}%
\pgfpathlineto{\pgfqpoint{2.747239in}{1.802415in}}%
\pgfpathlineto{\pgfqpoint{2.733748in}{1.816318in}}%
\pgfpathlineto{\pgfqpoint{2.720251in}{1.830433in}}%
\pgfpathlineto{\pgfqpoint{2.706750in}{1.844762in}}%
\pgfpathlineto{\pgfqpoint{2.693243in}{1.859305in}}%
\pgfpathlineto{\pgfqpoint{2.684612in}{1.861739in}}%
\pgfpathlineto{\pgfqpoint{2.675963in}{1.864496in}}%
\pgfpathlineto{\pgfqpoint{2.667294in}{1.867583in}}%
\pgfpathlineto{\pgfqpoint{2.658605in}{1.871007in}}%
\pgfpathclose%
\pgfusepath{fill}%
\end{pgfscope}%
\begin{pgfscope}%
\pgfpathrectangle{\pgfqpoint{1.254980in}{0.150000in}}{\pgfqpoint{5.490039in}{5.490039in}}%
\pgfusepath{clip}%
\pgfsetbuttcap%
\pgfsetroundjoin%
\definecolor{currentfill}{rgb}{0.243113,0.292092,0.538516}%
\pgfsetfillcolor{currentfill}%
\pgfsetfillopacity{0.700000}%
\pgfsetlinewidth{0.000000pt}%
\definecolor{currentstroke}{rgb}{0.000000,0.000000,0.000000}%
\pgfsetstrokecolor{currentstroke}%
\pgfsetdash{}{0pt}%
\pgfpathmoveto{\pgfqpoint{2.495410in}{2.070907in}}%
\pgfpathlineto{\pgfqpoint{2.509052in}{2.052984in}}%
\pgfpathlineto{\pgfqpoint{2.522685in}{2.035298in}}%
\pgfpathlineto{\pgfqpoint{2.536310in}{2.017845in}}%
\pgfpathlineto{\pgfqpoint{2.549927in}{2.000624in}}%
\pgfpathlineto{\pgfqpoint{2.558721in}{1.995839in}}%
\pgfpathlineto{\pgfqpoint{2.567493in}{1.991407in}}%
\pgfpathlineto{\pgfqpoint{2.576245in}{1.987323in}}%
\pgfpathlineto{\pgfqpoint{2.584975in}{1.983578in}}%
\pgfpathlineto{\pgfqpoint{2.571412in}{2.000126in}}%
\pgfpathlineto{\pgfqpoint{2.557841in}{2.016905in}}%
\pgfpathlineto{\pgfqpoint{2.544263in}{2.033917in}}%
\pgfpathlineto{\pgfqpoint{2.530676in}{2.051163in}}%
\pgfpathlineto{\pgfqpoint{2.521892in}{2.055570in}}%
\pgfpathlineto{\pgfqpoint{2.513087in}{2.060324in}}%
\pgfpathlineto{\pgfqpoint{2.504260in}{2.065434in}}%
\pgfpathlineto{\pgfqpoint{2.495410in}{2.070907in}}%
\pgfpathclose%
\pgfusepath{fill}%
\end{pgfscope}%
\begin{pgfscope}%
\pgfpathrectangle{\pgfqpoint{1.254980in}{0.150000in}}{\pgfqpoint{5.490039in}{5.490039in}}%
\pgfusepath{clip}%
\pgfsetbuttcap%
\pgfsetroundjoin%
\definecolor{currentfill}{rgb}{0.149039,0.508051,0.557250}%
\pgfsetfillcolor{currentfill}%
\pgfsetfillopacity{0.700000}%
\pgfsetlinewidth{0.000000pt}%
\definecolor{currentstroke}{rgb}{0.000000,0.000000,0.000000}%
\pgfsetstrokecolor{currentstroke}%
\pgfsetdash{}{0pt}%
\pgfpathmoveto{\pgfqpoint{2.146012in}{2.637598in}}%
\pgfpathlineto{\pgfqpoint{2.159926in}{2.612916in}}%
\pgfpathlineto{\pgfqpoint{2.173825in}{2.588533in}}%
\pgfpathlineto{\pgfqpoint{2.187709in}{2.564448in}}%
\pgfpathlineto{\pgfqpoint{2.201577in}{2.540657in}}%
\pgfpathlineto{\pgfqpoint{2.210699in}{2.532564in}}%
\pgfpathlineto{\pgfqpoint{2.219794in}{2.524860in}}%
\pgfpathlineto{\pgfqpoint{2.228864in}{2.517536in}}%
\pgfpathlineto{\pgfqpoint{2.237908in}{2.510586in}}%
\pgfpathlineto{\pgfqpoint{2.224105in}{2.533700in}}%
\pgfpathlineto{\pgfqpoint{2.210287in}{2.557107in}}%
\pgfpathlineto{\pgfqpoint{2.196455in}{2.580809in}}%
\pgfpathlineto{\pgfqpoint{2.182607in}{2.604809in}}%
\pgfpathlineto{\pgfqpoint{2.173499in}{2.612425in}}%
\pgfpathlineto{\pgfqpoint{2.164363in}{2.620424in}}%
\pgfpathlineto{\pgfqpoint{2.155201in}{2.628812in}}%
\pgfpathlineto{\pgfqpoint{2.146012in}{2.637598in}}%
\pgfpathclose%
\pgfusepath{fill}%
\end{pgfscope}%
\begin{pgfscope}%
\pgfpathrectangle{\pgfqpoint{1.254980in}{0.150000in}}{\pgfqpoint{5.490039in}{5.490039in}}%
\pgfusepath{clip}%
\pgfsetbuttcap%
\pgfsetroundjoin%
\definecolor{currentfill}{rgb}{0.194100,0.399323,0.555565}%
\pgfsetfillcolor{currentfill}%
\pgfsetfillopacity{0.700000}%
\pgfsetlinewidth{0.000000pt}%
\definecolor{currentstroke}{rgb}{0.000000,0.000000,0.000000}%
\pgfsetstrokecolor{currentstroke}%
\pgfsetdash{}{0pt}%
\pgfpathmoveto{\pgfqpoint{4.524567in}{2.245729in}}%
\pgfpathlineto{\pgfqpoint{4.538341in}{2.254664in}}%
\pgfpathlineto{\pgfqpoint{4.552129in}{2.263761in}}%
\pgfpathlineto{\pgfqpoint{4.565933in}{2.273018in}}%
\pgfpathlineto{\pgfqpoint{4.579751in}{2.282437in}}%
\pgfpathlineto{\pgfqpoint{4.587437in}{2.294580in}}%
\pgfpathlineto{\pgfqpoint{4.595117in}{2.306607in}}%
\pgfpathlineto{\pgfqpoint{4.602792in}{2.318517in}}%
\pgfpathlineto{\pgfqpoint{4.610460in}{2.330311in}}%
\pgfpathlineto{\pgfqpoint{4.596642in}{2.320756in}}%
\pgfpathlineto{\pgfqpoint{4.582839in}{2.311362in}}%
\pgfpathlineto{\pgfqpoint{4.569050in}{2.302130in}}%
\pgfpathlineto{\pgfqpoint{4.555277in}{2.293058in}}%
\pgfpathlineto{\pgfqpoint{4.547608in}{2.281390in}}%
\pgfpathlineto{\pgfqpoint{4.539933in}{2.269611in}}%
\pgfpathlineto{\pgfqpoint{4.532253in}{2.257724in}}%
\pgfpathlineto{\pgfqpoint{4.524567in}{2.245729in}}%
\pgfpathclose%
\pgfusepath{fill}%
\end{pgfscope}%
\begin{pgfscope}%
\pgfpathrectangle{\pgfqpoint{1.254980in}{0.150000in}}{\pgfqpoint{5.490039in}{5.490039in}}%
\pgfusepath{clip}%
\pgfsetbuttcap%
\pgfsetroundjoin%
\definecolor{currentfill}{rgb}{0.268510,0.009605,0.335427}%
\pgfsetfillcolor{currentfill}%
\pgfsetfillopacity{0.700000}%
\pgfsetlinewidth{0.000000pt}%
\definecolor{currentstroke}{rgb}{0.000000,0.000000,0.000000}%
\pgfsetstrokecolor{currentstroke}%
\pgfsetdash{}{0pt}%
\pgfpathmoveto{\pgfqpoint{3.490018in}{1.447730in}}%
\pgfpathlineto{\pgfqpoint{3.503431in}{1.444680in}}%
\pgfpathlineto{\pgfqpoint{3.516848in}{1.441797in}}%
\pgfpathlineto{\pgfqpoint{3.530271in}{1.439080in}}%
\pgfpathlineto{\pgfqpoint{3.543699in}{1.436530in}}%
\pgfpathlineto{\pgfqpoint{3.551727in}{1.445269in}}%
\pgfpathlineto{\pgfqpoint{3.559748in}{1.454145in}}%
\pgfpathlineto{\pgfqpoint{3.567761in}{1.463151in}}%
\pgfpathlineto{\pgfqpoint{3.575767in}{1.472283in}}%
\pgfpathlineto{\pgfqpoint{3.562356in}{1.474309in}}%
\pgfpathlineto{\pgfqpoint{3.548950in}{1.476500in}}%
\pgfpathlineto{\pgfqpoint{3.535549in}{1.478859in}}%
\pgfpathlineto{\pgfqpoint{3.522154in}{1.481384in}}%
\pgfpathlineto{\pgfqpoint{3.514131in}{1.472766in}}%
\pgfpathlineto{\pgfqpoint{3.506101in}{1.464281in}}%
\pgfpathlineto{\pgfqpoint{3.498063in}{1.455934in}}%
\pgfpathlineto{\pgfqpoint{3.490018in}{1.447730in}}%
\pgfpathclose%
\pgfusepath{fill}%
\end{pgfscope}%
\begin{pgfscope}%
\pgfpathrectangle{\pgfqpoint{1.254980in}{0.150000in}}{\pgfqpoint{5.490039in}{5.490039in}}%
\pgfusepath{clip}%
\pgfsetbuttcap%
\pgfsetroundjoin%
\definecolor{currentfill}{rgb}{0.170948,0.694384,0.493803}%
\pgfsetfillcolor{currentfill}%
\pgfsetfillopacity{0.700000}%
\pgfsetlinewidth{0.000000pt}%
\definecolor{currentstroke}{rgb}{0.000000,0.000000,0.000000}%
\pgfsetstrokecolor{currentstroke}%
\pgfsetdash{}{0pt}%
\pgfpathmoveto{\pgfqpoint{5.333749in}{3.052213in}}%
\pgfpathlineto{\pgfqpoint{5.348027in}{3.065970in}}%
\pgfpathlineto{\pgfqpoint{5.362324in}{3.079890in}}%
\pgfpathlineto{\pgfqpoint{5.376642in}{3.093973in}}%
\pgfpathlineto{\pgfqpoint{5.390980in}{3.108219in}}%
\pgfpathlineto{\pgfqpoint{5.398254in}{3.113286in}}%
\pgfpathlineto{\pgfqpoint{5.405518in}{3.118215in}}%
\pgfpathlineto{\pgfqpoint{5.412772in}{3.123010in}}%
\pgfpathlineto{\pgfqpoint{5.420017in}{3.127673in}}%
\pgfpathlineto{\pgfqpoint{5.405693in}{3.113660in}}%
\pgfpathlineto{\pgfqpoint{5.391388in}{3.099809in}}%
\pgfpathlineto{\pgfqpoint{5.377104in}{3.086121in}}%
\pgfpathlineto{\pgfqpoint{5.362840in}{3.072595in}}%
\pgfpathlineto{\pgfqpoint{5.355581in}{3.067689in}}%
\pgfpathlineto{\pgfqpoint{5.348313in}{3.062658in}}%
\pgfpathlineto{\pgfqpoint{5.341036in}{3.057501in}}%
\pgfpathlineto{\pgfqpoint{5.333749in}{3.052213in}}%
\pgfpathclose%
\pgfusepath{fill}%
\end{pgfscope}%
\begin{pgfscope}%
\pgfpathrectangle{\pgfqpoint{1.254980in}{0.150000in}}{\pgfqpoint{5.490039in}{5.490039in}}%
\pgfusepath{clip}%
\pgfsetbuttcap%
\pgfsetroundjoin%
\definecolor{currentfill}{rgb}{0.278012,0.180367,0.486697}%
\pgfsetfillcolor{currentfill}%
\pgfsetfillopacity{0.700000}%
\pgfsetlinewidth{0.000000pt}%
\definecolor{currentstroke}{rgb}{0.000000,0.000000,0.000000}%
\pgfsetstrokecolor{currentstroke}%
\pgfsetdash{}{0pt}%
\pgfpathmoveto{\pgfqpoint{2.712795in}{1.811468in}}%
\pgfpathlineto{\pgfqpoint{2.726329in}{1.797116in}}%
\pgfpathlineto{\pgfqpoint{2.739859in}{1.782974in}}%
\pgfpathlineto{\pgfqpoint{2.753383in}{1.769041in}}%
\pgfpathlineto{\pgfqpoint{2.766903in}{1.755315in}}%
\pgfpathlineto{\pgfqpoint{2.775495in}{1.753237in}}%
\pgfpathlineto{\pgfqpoint{2.784068in}{1.751479in}}%
\pgfpathlineto{\pgfqpoint{2.792624in}{1.750035in}}%
\pgfpathlineto{\pgfqpoint{2.801162in}{1.748898in}}%
\pgfpathlineto{\pgfqpoint{2.787687in}{1.761966in}}%
\pgfpathlineto{\pgfqpoint{2.774209in}{1.775240in}}%
\pgfpathlineto{\pgfqpoint{2.760726in}{1.788723in}}%
\pgfpathlineto{\pgfqpoint{2.747239in}{1.802415in}}%
\pgfpathlineto{\pgfqpoint{2.738656in}{1.804199in}}%
\pgfpathlineto{\pgfqpoint{2.730054in}{1.806297in}}%
\pgfpathlineto{\pgfqpoint{2.721434in}{1.808718in}}%
\pgfpathlineto{\pgfqpoint{2.712795in}{1.811468in}}%
\pgfpathclose%
\pgfusepath{fill}%
\end{pgfscope}%
\begin{pgfscope}%
\pgfpathrectangle{\pgfqpoint{1.254980in}{0.150000in}}{\pgfqpoint{5.490039in}{5.490039in}}%
\pgfusepath{clip}%
\pgfsetbuttcap%
\pgfsetroundjoin%
\definecolor{currentfill}{rgb}{0.377779,0.791781,0.377939}%
\pgfsetfillcolor{currentfill}%
\pgfsetfillopacity{0.700000}%
\pgfsetlinewidth{0.000000pt}%
\definecolor{currentstroke}{rgb}{0.000000,0.000000,0.000000}%
\pgfsetstrokecolor{currentstroke}%
\pgfsetdash{}{0pt}%
\pgfpathmoveto{\pgfqpoint{5.707302in}{3.352876in}}%
\pgfpathlineto{\pgfqpoint{5.721826in}{3.367757in}}%
\pgfpathlineto{\pgfqpoint{5.736372in}{3.382800in}}%
\pgfpathlineto{\pgfqpoint{5.750939in}{3.398007in}}%
\pgfpathlineto{\pgfqpoint{5.765529in}{3.413376in}}%
\pgfpathlineto{\pgfqpoint{5.772529in}{3.414795in}}%
\pgfpathlineto{\pgfqpoint{5.779518in}{3.416116in}}%
\pgfpathlineto{\pgfqpoint{5.786498in}{3.417343in}}%
\pgfpathlineto{\pgfqpoint{5.793467in}{3.418482in}}%
\pgfpathlineto{\pgfqpoint{5.778902in}{3.403505in}}%
\pgfpathlineto{\pgfqpoint{5.764359in}{3.388691in}}%
\pgfpathlineto{\pgfqpoint{5.749837in}{3.374038in}}%
\pgfpathlineto{\pgfqpoint{5.735337in}{3.359548in}}%
\pgfpathlineto{\pgfqpoint{5.728343in}{3.358007in}}%
\pgfpathlineto{\pgfqpoint{5.721339in}{3.356384in}}%
\pgfpathlineto{\pgfqpoint{5.714326in}{3.354675in}}%
\pgfpathlineto{\pgfqpoint{5.707302in}{3.352876in}}%
\pgfpathclose%
\pgfusepath{fill}%
\end{pgfscope}%
\begin{pgfscope}%
\pgfpathrectangle{\pgfqpoint{1.254980in}{0.150000in}}{\pgfqpoint{5.490039in}{5.490039in}}%
\pgfusepath{clip}%
\pgfsetbuttcap%
\pgfsetroundjoin%
\definecolor{currentfill}{rgb}{0.229739,0.322361,0.545706}%
\pgfsetfillcolor{currentfill}%
\pgfsetfillopacity{0.700000}%
\pgfsetlinewidth{0.000000pt}%
\definecolor{currentstroke}{rgb}{0.000000,0.000000,0.000000}%
\pgfsetstrokecolor{currentstroke}%
\pgfsetdash{}{0pt}%
\pgfpathmoveto{\pgfqpoint{2.440754in}{2.144990in}}%
\pgfpathlineto{\pgfqpoint{2.454432in}{2.126106in}}%
\pgfpathlineto{\pgfqpoint{2.468101in}{2.107466in}}%
\pgfpathlineto{\pgfqpoint{2.481760in}{2.089067in}}%
\pgfpathlineto{\pgfqpoint{2.495410in}{2.070907in}}%
\pgfpathlineto{\pgfqpoint{2.504260in}{2.065434in}}%
\pgfpathlineto{\pgfqpoint{2.513087in}{2.060324in}}%
\pgfpathlineto{\pgfqpoint{2.521892in}{2.055570in}}%
\pgfpathlineto{\pgfqpoint{2.530676in}{2.051163in}}%
\pgfpathlineto{\pgfqpoint{2.517082in}{2.068646in}}%
\pgfpathlineto{\pgfqpoint{2.503479in}{2.086367in}}%
\pgfpathlineto{\pgfqpoint{2.489867in}{2.104327in}}%
\pgfpathlineto{\pgfqpoint{2.476247in}{2.122530in}}%
\pgfpathlineto{\pgfqpoint{2.467408in}{2.127602in}}%
\pgfpathlineto{\pgfqpoint{2.458547in}{2.133032in}}%
\pgfpathlineto{\pgfqpoint{2.449662in}{2.138825in}}%
\pgfpathlineto{\pgfqpoint{2.440754in}{2.144990in}}%
\pgfpathclose%
\pgfusepath{fill}%
\end{pgfscope}%
\begin{pgfscope}%
\pgfpathrectangle{\pgfqpoint{1.254980in}{0.150000in}}{\pgfqpoint{5.490039in}{5.490039in}}%
\pgfusepath{clip}%
\pgfsetbuttcap%
\pgfsetroundjoin%
\definecolor{currentfill}{rgb}{0.276022,0.044167,0.370164}%
\pgfsetfillcolor{currentfill}%
\pgfsetfillopacity{0.700000}%
\pgfsetlinewidth{0.000000pt}%
\definecolor{currentstroke}{rgb}{0.000000,0.000000,0.000000}%
\pgfsetstrokecolor{currentstroke}%
\pgfsetdash{}{0pt}%
\pgfpathmoveto{\pgfqpoint{3.070099in}{1.529142in}}%
\pgfpathlineto{\pgfqpoint{3.083532in}{1.520152in}}%
\pgfpathlineto{\pgfqpoint{3.096965in}{1.511345in}}%
\pgfpathlineto{\pgfqpoint{3.110398in}{1.502721in}}%
\pgfpathlineto{\pgfqpoint{3.123831in}{1.494279in}}%
\pgfpathlineto{\pgfqpoint{3.132125in}{1.497224in}}%
\pgfpathlineto{\pgfqpoint{3.140405in}{1.500417in}}%
\pgfpathlineto{\pgfqpoint{3.148673in}{1.503852in}}%
\pgfpathlineto{\pgfqpoint{3.156928in}{1.507522in}}%
\pgfpathlineto{\pgfqpoint{3.143527in}{1.515351in}}%
\pgfpathlineto{\pgfqpoint{3.130126in}{1.523361in}}%
\pgfpathlineto{\pgfqpoint{3.116726in}{1.531553in}}%
\pgfpathlineto{\pgfqpoint{3.103326in}{1.539928in}}%
\pgfpathlineto{\pgfqpoint{3.095038in}{1.536861in}}%
\pgfpathlineto{\pgfqpoint{3.086738in}{1.534037in}}%
\pgfpathlineto{\pgfqpoint{3.078425in}{1.531461in}}%
\pgfpathlineto{\pgfqpoint{3.070099in}{1.529142in}}%
\pgfpathclose%
\pgfusepath{fill}%
\end{pgfscope}%
\begin{pgfscope}%
\pgfpathrectangle{\pgfqpoint{1.254980in}{0.150000in}}{\pgfqpoint{5.490039in}{5.490039in}}%
\pgfusepath{clip}%
\pgfsetbuttcap%
\pgfsetroundjoin%
\definecolor{currentfill}{rgb}{0.257322,0.256130,0.526563}%
\pgfsetfillcolor{currentfill}%
\pgfsetfillopacity{0.700000}%
\pgfsetlinewidth{0.000000pt}%
\definecolor{currentstroke}{rgb}{0.000000,0.000000,0.000000}%
\pgfsetstrokecolor{currentstroke}%
\pgfsetdash{}{0pt}%
\pgfpathmoveto{\pgfqpoint{4.205571in}{1.904065in}}%
\pgfpathlineto{\pgfqpoint{4.219185in}{1.909960in}}%
\pgfpathlineto{\pgfqpoint{4.232812in}{1.916016in}}%
\pgfpathlineto{\pgfqpoint{4.246450in}{1.922232in}}%
\pgfpathlineto{\pgfqpoint{4.260101in}{1.928608in}}%
\pgfpathlineto{\pgfqpoint{4.267885in}{1.941851in}}%
\pgfpathlineto{\pgfqpoint{4.275665in}{1.955036in}}%
\pgfpathlineto{\pgfqpoint{4.283440in}{1.968161in}}%
\pgfpathlineto{\pgfqpoint{4.291211in}{1.981224in}}%
\pgfpathlineto{\pgfqpoint{4.277561in}{1.974568in}}%
\pgfpathlineto{\pgfqpoint{4.263923in}{1.968073in}}%
\pgfpathlineto{\pgfqpoint{4.250298in}{1.961738in}}%
\pgfpathlineto{\pgfqpoint{4.236685in}{1.955564in}}%
\pgfpathlineto{\pgfqpoint{4.228913in}{1.942770in}}%
\pgfpathlineto{\pgfqpoint{4.221137in}{1.929920in}}%
\pgfpathlineto{\pgfqpoint{4.213357in}{1.917018in}}%
\pgfpathlineto{\pgfqpoint{4.205571in}{1.904065in}}%
\pgfpathclose%
\pgfusepath{fill}%
\end{pgfscope}%
\begin{pgfscope}%
\pgfpathrectangle{\pgfqpoint{1.254980in}{0.150000in}}{\pgfqpoint{5.490039in}{5.490039in}}%
\pgfusepath{clip}%
\pgfsetbuttcap%
\pgfsetroundjoin%
\definecolor{currentfill}{rgb}{0.430983,0.808473,0.346476}%
\pgfsetfillcolor{currentfill}%
\pgfsetfillopacity{0.700000}%
\pgfsetlinewidth{0.000000pt}%
\definecolor{currentstroke}{rgb}{0.000000,0.000000,0.000000}%
\pgfsetstrokecolor{currentstroke}%
\pgfsetdash{}{0pt}%
\pgfpathmoveto{\pgfqpoint{5.793467in}{3.418482in}}%
\pgfpathlineto{\pgfqpoint{5.808055in}{3.433621in}}%
\pgfpathlineto{\pgfqpoint{5.822664in}{3.448923in}}%
\pgfpathlineto{\pgfqpoint{5.837296in}{3.464388in}}%
\pgfpathlineto{\pgfqpoint{5.844237in}{3.465132in}}%
\pgfpathlineto{\pgfqpoint{5.851167in}{3.465790in}}%
\pgfpathlineto{\pgfqpoint{5.858088in}{3.466368in}}%
\pgfpathlineto{\pgfqpoint{5.864999in}{3.466869in}}%
\pgfpathlineto{\pgfqpoint{5.850394in}{3.451829in}}%
\pgfpathlineto{\pgfqpoint{5.835811in}{3.436951in}}%
\pgfpathlineto{\pgfqpoint{5.821250in}{3.422234in}}%
\pgfpathlineto{\pgfqpoint{5.814318in}{3.421407in}}%
\pgfpathlineto{\pgfqpoint{5.807378in}{3.420509in}}%
\pgfpathlineto{\pgfqpoint{5.800427in}{3.419536in}}%
\pgfpathlineto{\pgfqpoint{5.793467in}{3.418482in}}%
\pgfpathclose%
\pgfusepath{fill}%
\end{pgfscope}%
\begin{pgfscope}%
\pgfpathrectangle{\pgfqpoint{1.254980in}{0.150000in}}{\pgfqpoint{5.490039in}{5.490039in}}%
\pgfusepath{clip}%
\pgfsetbuttcap%
\pgfsetroundjoin%
\definecolor{currentfill}{rgb}{0.280255,0.165693,0.476498}%
\pgfsetfillcolor{currentfill}%
\pgfsetfillopacity{0.700000}%
\pgfsetlinewidth{0.000000pt}%
\definecolor{currentstroke}{rgb}{0.000000,0.000000,0.000000}%
\pgfsetstrokecolor{currentstroke}%
\pgfsetdash{}{0pt}%
\pgfpathmoveto{\pgfqpoint{4.003253in}{1.712345in}}%
\pgfpathlineto{\pgfqpoint{4.016785in}{1.715953in}}%
\pgfpathlineto{\pgfqpoint{4.030328in}{1.719721in}}%
\pgfpathlineto{\pgfqpoint{4.043881in}{1.723649in}}%
\pgfpathlineto{\pgfqpoint{4.057444in}{1.727738in}}%
\pgfpathlineto{\pgfqpoint{4.065284in}{1.740703in}}%
\pgfpathlineto{\pgfqpoint{4.073120in}{1.753660in}}%
\pgfpathlineto{\pgfqpoint{4.080951in}{1.766606in}}%
\pgfpathlineto{\pgfqpoint{4.088778in}{1.779537in}}%
\pgfpathlineto{\pgfqpoint{4.075218in}{1.775086in}}%
\pgfpathlineto{\pgfqpoint{4.061668in}{1.770795in}}%
\pgfpathlineto{\pgfqpoint{4.048130in}{1.766665in}}%
\pgfpathlineto{\pgfqpoint{4.034601in}{1.762695in}}%
\pgfpathlineto{\pgfqpoint{4.026771in}{1.750116in}}%
\pgfpathlineto{\pgfqpoint{4.018936in}{1.737529in}}%
\pgfpathlineto{\pgfqpoint{4.011097in}{1.724938in}}%
\pgfpathlineto{\pgfqpoint{4.003253in}{1.712345in}}%
\pgfpathclose%
\pgfusepath{fill}%
\end{pgfscope}%
\begin{pgfscope}%
\pgfpathrectangle{\pgfqpoint{1.254980in}{0.150000in}}{\pgfqpoint{5.490039in}{5.490039in}}%
\pgfusepath{clip}%
\pgfsetbuttcap%
\pgfsetroundjoin%
\definecolor{currentfill}{rgb}{0.281412,0.155834,0.469201}%
\pgfsetfillcolor{currentfill}%
\pgfsetfillopacity{0.700000}%
\pgfsetlinewidth{0.000000pt}%
\definecolor{currentstroke}{rgb}{0.000000,0.000000,0.000000}%
\pgfsetstrokecolor{currentstroke}%
\pgfsetdash{}{0pt}%
\pgfpathmoveto{\pgfqpoint{2.766903in}{1.755315in}}%
\pgfpathlineto{\pgfqpoint{2.780419in}{1.741796in}}%
\pgfpathlineto{\pgfqpoint{2.793931in}{1.728481in}}%
\pgfpathlineto{\pgfqpoint{2.807439in}{1.715371in}}%
\pgfpathlineto{\pgfqpoint{2.820943in}{1.702463in}}%
\pgfpathlineto{\pgfqpoint{2.829489in}{1.701052in}}%
\pgfpathlineto{\pgfqpoint{2.838017in}{1.699954in}}%
\pgfpathlineto{\pgfqpoint{2.846529in}{1.699162in}}%
\pgfpathlineto{\pgfqpoint{2.855023in}{1.698669in}}%
\pgfpathlineto{\pgfqpoint{2.841563in}{1.710922in}}%
\pgfpathlineto{\pgfqpoint{2.828100in}{1.723377in}}%
\pgfpathlineto{\pgfqpoint{2.814633in}{1.736035in}}%
\pgfpathlineto{\pgfqpoint{2.801162in}{1.748898in}}%
\pgfpathlineto{\pgfqpoint{2.792624in}{1.750035in}}%
\pgfpathlineto{\pgfqpoint{2.784068in}{1.751479in}}%
\pgfpathlineto{\pgfqpoint{2.775495in}{1.753237in}}%
\pgfpathlineto{\pgfqpoint{2.766903in}{1.755315in}}%
\pgfpathclose%
\pgfusepath{fill}%
\end{pgfscope}%
\begin{pgfscope}%
\pgfpathrectangle{\pgfqpoint{1.254980in}{0.150000in}}{\pgfqpoint{5.490039in}{5.490039in}}%
\pgfusepath{clip}%
\pgfsetbuttcap%
\pgfsetroundjoin%
\definecolor{currentfill}{rgb}{0.159194,0.482237,0.558073}%
\pgfsetfillcolor{currentfill}%
\pgfsetfillopacity{0.700000}%
\pgfsetlinewidth{0.000000pt}%
\definecolor{currentstroke}{rgb}{0.000000,0.000000,0.000000}%
\pgfsetstrokecolor{currentstroke}%
\pgfsetdash{}{0pt}%
\pgfpathmoveto{\pgfqpoint{4.727018in}{2.460917in}}%
\pgfpathlineto{\pgfqpoint{4.740914in}{2.471463in}}%
\pgfpathlineto{\pgfqpoint{4.754826in}{2.482170in}}%
\pgfpathlineto{\pgfqpoint{4.768754in}{2.493039in}}%
\pgfpathlineto{\pgfqpoint{4.782700in}{2.504069in}}%
\pgfpathlineto{\pgfqpoint{4.790311in}{2.514894in}}%
\pgfpathlineto{\pgfqpoint{4.797915in}{2.525582in}}%
\pgfpathlineto{\pgfqpoint{4.805513in}{2.536132in}}%
\pgfpathlineto{\pgfqpoint{4.813104in}{2.546545in}}%
\pgfpathlineto{\pgfqpoint{4.799160in}{2.535467in}}%
\pgfpathlineto{\pgfqpoint{4.785234in}{2.524551in}}%
\pgfpathlineto{\pgfqpoint{4.771323in}{2.513797in}}%
\pgfpathlineto{\pgfqpoint{4.757429in}{2.503204in}}%
\pgfpathlineto{\pgfqpoint{4.749836in}{2.492827in}}%
\pgfpathlineto{\pgfqpoint{4.742237in}{2.482321in}}%
\pgfpathlineto{\pgfqpoint{4.734631in}{2.471684in}}%
\pgfpathlineto{\pgfqpoint{4.727018in}{2.460917in}}%
\pgfpathclose%
\pgfusepath{fill}%
\end{pgfscope}%
\begin{pgfscope}%
\pgfpathrectangle{\pgfqpoint{1.254980in}{0.150000in}}{\pgfqpoint{5.490039in}{5.490039in}}%
\pgfusepath{clip}%
\pgfsetbuttcap%
\pgfsetroundjoin%
\definecolor{currentfill}{rgb}{0.214298,0.355619,0.551184}%
\pgfsetfillcolor{currentfill}%
\pgfsetfillopacity{0.700000}%
\pgfsetlinewidth{0.000000pt}%
\definecolor{currentstroke}{rgb}{0.000000,0.000000,0.000000}%
\pgfsetstrokecolor{currentstroke}%
\pgfsetdash{}{0pt}%
\pgfpathmoveto{\pgfqpoint{2.385945in}{2.222996in}}%
\pgfpathlineto{\pgfqpoint{2.399663in}{2.203120in}}%
\pgfpathlineto{\pgfqpoint{2.413370in}{2.183495in}}%
\pgfpathlineto{\pgfqpoint{2.427067in}{2.164119in}}%
\pgfpathlineto{\pgfqpoint{2.440754in}{2.144990in}}%
\pgfpathlineto{\pgfqpoint{2.449662in}{2.138825in}}%
\pgfpathlineto{\pgfqpoint{2.458547in}{2.133032in}}%
\pgfpathlineto{\pgfqpoint{2.467408in}{2.127602in}}%
\pgfpathlineto{\pgfqpoint{2.476247in}{2.122530in}}%
\pgfpathlineto{\pgfqpoint{2.462618in}{2.140976in}}%
\pgfpathlineto{\pgfqpoint{2.448979in}{2.159668in}}%
\pgfpathlineto{\pgfqpoint{2.435331in}{2.178609in}}%
\pgfpathlineto{\pgfqpoint{2.421672in}{2.197799in}}%
\pgfpathlineto{\pgfqpoint{2.412776in}{2.203542in}}%
\pgfpathlineto{\pgfqpoint{2.403856in}{2.209652in}}%
\pgfpathlineto{\pgfqpoint{2.394913in}{2.216134in}}%
\pgfpathlineto{\pgfqpoint{2.385945in}{2.222996in}}%
\pgfpathclose%
\pgfusepath{fill}%
\end{pgfscope}%
\begin{pgfscope}%
\pgfpathrectangle{\pgfqpoint{1.254980in}{0.150000in}}{\pgfqpoint{5.490039in}{5.490039in}}%
\pgfusepath{clip}%
\pgfsetbuttcap%
\pgfsetroundjoin%
\definecolor{currentfill}{rgb}{0.216210,0.351535,0.550627}%
\pgfsetfillcolor{currentfill}%
\pgfsetfillopacity{0.700000}%
\pgfsetlinewidth{0.000000pt}%
\definecolor{currentstroke}{rgb}{0.000000,0.000000,0.000000}%
\pgfsetstrokecolor{currentstroke}%
\pgfsetdash{}{0pt}%
\pgfpathmoveto{\pgfqpoint{4.407939in}{2.113333in}}%
\pgfpathlineto{\pgfqpoint{4.421655in}{2.121266in}}%
\pgfpathlineto{\pgfqpoint{4.435386in}{2.129359in}}%
\pgfpathlineto{\pgfqpoint{4.449130in}{2.137613in}}%
\pgfpathlineto{\pgfqpoint{4.462889in}{2.146027in}}%
\pgfpathlineto{\pgfqpoint{4.470617in}{2.158841in}}%
\pgfpathlineto{\pgfqpoint{4.478340in}{2.171557in}}%
\pgfpathlineto{\pgfqpoint{4.486057in}{2.184174in}}%
\pgfpathlineto{\pgfqpoint{4.493770in}{2.196692in}}%
\pgfpathlineto{\pgfqpoint{4.480011in}{2.188082in}}%
\pgfpathlineto{\pgfqpoint{4.466266in}{2.179634in}}%
\pgfpathlineto{\pgfqpoint{4.452536in}{2.171346in}}%
\pgfpathlineto{\pgfqpoint{4.438819in}{2.163220in}}%
\pgfpathlineto{\pgfqpoint{4.431107in}{2.150886in}}%
\pgfpathlineto{\pgfqpoint{4.423389in}{2.138459in}}%
\pgfpathlineto{\pgfqpoint{4.415666in}{2.125941in}}%
\pgfpathlineto{\pgfqpoint{4.407939in}{2.113333in}}%
\pgfpathclose%
\pgfusepath{fill}%
\end{pgfscope}%
\begin{pgfscope}%
\pgfpathrectangle{\pgfqpoint{1.254980in}{0.150000in}}{\pgfqpoint{5.490039in}{5.490039in}}%
\pgfusepath{clip}%
\pgfsetbuttcap%
\pgfsetroundjoin%
\definecolor{currentfill}{rgb}{0.268510,0.009605,0.335427}%
\pgfsetfillcolor{currentfill}%
\pgfsetfillopacity{0.700000}%
\pgfsetlinewidth{0.000000pt}%
\definecolor{currentstroke}{rgb}{0.000000,0.000000,0.000000}%
\pgfsetstrokecolor{currentstroke}%
\pgfsetdash{}{0pt}%
\pgfpathmoveto{\pgfqpoint{3.264186in}{1.451325in}}%
\pgfpathlineto{\pgfqpoint{3.277601in}{1.445094in}}%
\pgfpathlineto{\pgfqpoint{3.291019in}{1.439036in}}%
\pgfpathlineto{\pgfqpoint{3.304439in}{1.433152in}}%
\pgfpathlineto{\pgfqpoint{3.317861in}{1.427440in}}%
\pgfpathlineto{\pgfqpoint{3.326022in}{1.433120in}}%
\pgfpathlineto{\pgfqpoint{3.334173in}{1.439003in}}%
\pgfpathlineto{\pgfqpoint{3.342313in}{1.445080in}}%
\pgfpathlineto{\pgfqpoint{3.350444in}{1.451347in}}%
\pgfpathlineto{\pgfqpoint{3.337046in}{1.456477in}}%
\pgfpathlineto{\pgfqpoint{3.323651in}{1.461779in}}%
\pgfpathlineto{\pgfqpoint{3.310259in}{1.467255in}}%
\pgfpathlineto{\pgfqpoint{3.296869in}{1.472904in}}%
\pgfpathlineto{\pgfqpoint{3.288714in}{1.467208in}}%
\pgfpathlineto{\pgfqpoint{3.280549in}{1.461709in}}%
\pgfpathlineto{\pgfqpoint{3.272373in}{1.456413in}}%
\pgfpathlineto{\pgfqpoint{3.264186in}{1.451325in}}%
\pgfpathclose%
\pgfusepath{fill}%
\end{pgfscope}%
\begin{pgfscope}%
\pgfpathrectangle{\pgfqpoint{1.254980in}{0.150000in}}{\pgfqpoint{5.490039in}{5.490039in}}%
\pgfusepath{clip}%
\pgfsetbuttcap%
\pgfsetroundjoin%
\definecolor{currentfill}{rgb}{0.122312,0.633153,0.530398}%
\pgfsetfillcolor{currentfill}%
\pgfsetfillopacity{0.700000}%
\pgfsetlinewidth{0.000000pt}%
\definecolor{currentstroke}{rgb}{0.000000,0.000000,0.000000}%
\pgfsetstrokecolor{currentstroke}%
\pgfsetdash{}{0pt}%
\pgfpathmoveto{\pgfqpoint{5.131824in}{2.869481in}}%
\pgfpathlineto{\pgfqpoint{5.145977in}{2.882444in}}%
\pgfpathlineto{\pgfqpoint{5.160148in}{2.895570in}}%
\pgfpathlineto{\pgfqpoint{5.174339in}{2.908859in}}%
\pgfpathlineto{\pgfqpoint{5.188549in}{2.922310in}}%
\pgfpathlineto{\pgfqpoint{5.195955in}{2.929468in}}%
\pgfpathlineto{\pgfqpoint{5.203353in}{2.936476in}}%
\pgfpathlineto{\pgfqpoint{5.210741in}{2.943338in}}%
\pgfpathlineto{\pgfqpoint{5.218121in}{2.950054in}}%
\pgfpathlineto{\pgfqpoint{5.203920in}{2.936740in}}%
\pgfpathlineto{\pgfqpoint{5.189738in}{2.923590in}}%
\pgfpathlineto{\pgfqpoint{5.175575in}{2.910601in}}%
\pgfpathlineto{\pgfqpoint{5.161430in}{2.897775in}}%
\pgfpathlineto{\pgfqpoint{5.154042in}{2.890909in}}%
\pgfpathlineto{\pgfqpoint{5.146645in}{2.883907in}}%
\pgfpathlineto{\pgfqpoint{5.139239in}{2.876764in}}%
\pgfpathlineto{\pgfqpoint{5.131824in}{2.869481in}}%
\pgfpathclose%
\pgfusepath{fill}%
\end{pgfscope}%
\begin{pgfscope}%
\pgfpathrectangle{\pgfqpoint{1.254980in}{0.150000in}}{\pgfqpoint{5.490039in}{5.490039in}}%
\pgfusepath{clip}%
\pgfsetbuttcap%
\pgfsetroundjoin%
\definecolor{currentfill}{rgb}{0.129933,0.559582,0.551864}%
\pgfsetfillcolor{currentfill}%
\pgfsetfillopacity{0.700000}%
\pgfsetlinewidth{0.000000pt}%
\definecolor{currentstroke}{rgb}{0.000000,0.000000,0.000000}%
\pgfsetstrokecolor{currentstroke}%
\pgfsetdash{}{0pt}%
\pgfpathmoveto{\pgfqpoint{4.929500in}{2.670858in}}%
\pgfpathlineto{\pgfqpoint{4.943523in}{2.682749in}}%
\pgfpathlineto{\pgfqpoint{4.957565in}{2.694801in}}%
\pgfpathlineto{\pgfqpoint{4.971624in}{2.707016in}}%
\pgfpathlineto{\pgfqpoint{4.985701in}{2.719394in}}%
\pgfpathlineto{\pgfqpoint{4.993220in}{2.728515in}}%
\pgfpathlineto{\pgfqpoint{5.000731in}{2.737487in}}%
\pgfpathlineto{\pgfqpoint{5.008234in}{2.746311in}}%
\pgfpathlineto{\pgfqpoint{5.015730in}{2.754988in}}%
\pgfpathlineto{\pgfqpoint{5.001657in}{2.742655in}}%
\pgfpathlineto{\pgfqpoint{4.987602in}{2.730485in}}%
\pgfpathlineto{\pgfqpoint{4.973565in}{2.718476in}}%
\pgfpathlineto{\pgfqpoint{4.959546in}{2.706629in}}%
\pgfpathlineto{\pgfqpoint{4.952046in}{2.697896in}}%
\pgfpathlineto{\pgfqpoint{4.944538in}{2.689024in}}%
\pgfpathlineto{\pgfqpoint{4.937023in}{2.680012in}}%
\pgfpathlineto{\pgfqpoint{4.929500in}{2.670858in}}%
\pgfpathclose%
\pgfusepath{fill}%
\end{pgfscope}%
\begin{pgfscope}%
\pgfpathrectangle{\pgfqpoint{1.254980in}{0.150000in}}{\pgfqpoint{5.490039in}{5.490039in}}%
\pgfusepath{clip}%
\pgfsetbuttcap%
\pgfsetroundjoin%
\definecolor{currentfill}{rgb}{0.283072,0.130895,0.449241}%
\pgfsetfillcolor{currentfill}%
\pgfsetfillopacity{0.700000}%
\pgfsetlinewidth{0.000000pt}%
\definecolor{currentstroke}{rgb}{0.000000,0.000000,0.000000}%
\pgfsetstrokecolor{currentstroke}%
\pgfsetdash{}{0pt}%
\pgfpathmoveto{\pgfqpoint{2.820943in}{1.702463in}}%
\pgfpathlineto{\pgfqpoint{2.834443in}{1.689756in}}%
\pgfpathlineto{\pgfqpoint{2.847940in}{1.677249in}}%
\pgfpathlineto{\pgfqpoint{2.861434in}{1.664941in}}%
\pgfpathlineto{\pgfqpoint{2.874925in}{1.652831in}}%
\pgfpathlineto{\pgfqpoint{2.883428in}{1.652086in}}%
\pgfpathlineto{\pgfqpoint{2.891913in}{1.651645in}}%
\pgfpathlineto{\pgfqpoint{2.900383in}{1.651501in}}%
\pgfpathlineto{\pgfqpoint{2.908836in}{1.651649in}}%
\pgfpathlineto{\pgfqpoint{2.895387in}{1.663108in}}%
\pgfpathlineto{\pgfqpoint{2.881935in}{1.674763in}}%
\pgfpathlineto{\pgfqpoint{2.868481in}{1.686616in}}%
\pgfpathlineto{\pgfqpoint{2.855023in}{1.698669in}}%
\pgfpathlineto{\pgfqpoint{2.846529in}{1.699162in}}%
\pgfpathlineto{\pgfqpoint{2.838017in}{1.699954in}}%
\pgfpathlineto{\pgfqpoint{2.829489in}{1.701052in}}%
\pgfpathlineto{\pgfqpoint{2.820943in}{1.702463in}}%
\pgfpathclose%
\pgfusepath{fill}%
\end{pgfscope}%
\begin{pgfscope}%
\pgfpathrectangle{\pgfqpoint{1.254980in}{0.150000in}}{\pgfqpoint{5.490039in}{5.490039in}}%
\pgfusepath{clip}%
\pgfsetbuttcap%
\pgfsetroundjoin%
\definecolor{currentfill}{rgb}{0.267004,0.004874,0.329415}%
\pgfsetfillcolor{currentfill}%
\pgfsetfillopacity{0.700000}%
\pgfsetlinewidth{0.000000pt}%
\definecolor{currentstroke}{rgb}{0.000000,0.000000,0.000000}%
\pgfsetstrokecolor{currentstroke}%
\pgfsetdash{}{0pt}%
\pgfpathmoveto{\pgfqpoint{3.404069in}{1.432542in}}%
\pgfpathlineto{\pgfqpoint{3.417485in}{1.428266in}}%
\pgfpathlineto{\pgfqpoint{3.430904in}{1.424158in}}%
\pgfpathlineto{\pgfqpoint{3.444327in}{1.420220in}}%
\pgfpathlineto{\pgfqpoint{3.457755in}{1.416449in}}%
\pgfpathlineto{\pgfqpoint{3.465833in}{1.424028in}}%
\pgfpathlineto{\pgfqpoint{3.473903in}{1.431772in}}%
\pgfpathlineto{\pgfqpoint{3.481964in}{1.439674in}}%
\pgfpathlineto{\pgfqpoint{3.490018in}{1.447730in}}%
\pgfpathlineto{\pgfqpoint{3.476610in}{1.450948in}}%
\pgfpathlineto{\pgfqpoint{3.463206in}{1.454333in}}%
\pgfpathlineto{\pgfqpoint{3.449807in}{1.457887in}}%
\pgfpathlineto{\pgfqpoint{3.436413in}{1.461610in}}%
\pgfpathlineto{\pgfqpoint{3.428340in}{1.454097in}}%
\pgfpathlineto{\pgfqpoint{3.420258in}{1.446744in}}%
\pgfpathlineto{\pgfqpoint{3.412168in}{1.439557in}}%
\pgfpathlineto{\pgfqpoint{3.404069in}{1.432542in}}%
\pgfpathclose%
\pgfusepath{fill}%
\end{pgfscope}%
\begin{pgfscope}%
\pgfpathrectangle{\pgfqpoint{1.254980in}{0.150000in}}{\pgfqpoint{5.490039in}{5.490039in}}%
\pgfusepath{clip}%
\pgfsetbuttcap%
\pgfsetroundjoin%
\definecolor{currentfill}{rgb}{0.220124,0.725509,0.466226}%
\pgfsetfillcolor{currentfill}%
\pgfsetfillopacity{0.700000}%
\pgfsetlinewidth{0.000000pt}%
\definecolor{currentstroke}{rgb}{0.000000,0.000000,0.000000}%
\pgfsetstrokecolor{currentstroke}%
\pgfsetdash{}{0pt}%
\pgfpathmoveto{\pgfqpoint{5.420017in}{3.127673in}}%
\pgfpathlineto{\pgfqpoint{5.434362in}{3.141849in}}%
\pgfpathlineto{\pgfqpoint{5.448727in}{3.156188in}}%
\pgfpathlineto{\pgfqpoint{5.463113in}{3.170690in}}%
\pgfpathlineto{\pgfqpoint{5.477520in}{3.185355in}}%
\pgfpathlineto{\pgfqpoint{5.484740in}{3.189636in}}%
\pgfpathlineto{\pgfqpoint{5.491951in}{3.193783in}}%
\pgfpathlineto{\pgfqpoint{5.499151in}{3.197800in}}%
\pgfpathlineto{\pgfqpoint{5.506342in}{3.201690in}}%
\pgfpathlineto{\pgfqpoint{5.491951in}{3.187289in}}%
\pgfpathlineto{\pgfqpoint{5.477580in}{3.173052in}}%
\pgfpathlineto{\pgfqpoint{5.463230in}{3.158977in}}%
\pgfpathlineto{\pgfqpoint{5.448901in}{3.145065in}}%
\pgfpathlineto{\pgfqpoint{5.441694in}{3.140900in}}%
\pgfpathlineto{\pgfqpoint{5.434478in}{3.136615in}}%
\pgfpathlineto{\pgfqpoint{5.427252in}{3.132207in}}%
\pgfpathlineto{\pgfqpoint{5.420017in}{3.127673in}}%
\pgfpathclose%
\pgfusepath{fill}%
\end{pgfscope}%
\begin{pgfscope}%
\pgfpathrectangle{\pgfqpoint{1.254980in}{0.150000in}}{\pgfqpoint{5.490039in}{5.490039in}}%
\pgfusepath{clip}%
\pgfsetbuttcap%
\pgfsetroundjoin%
\definecolor{currentfill}{rgb}{0.133743,0.548535,0.553541}%
\pgfsetfillcolor{currentfill}%
\pgfsetfillopacity{0.700000}%
\pgfsetlinewidth{0.000000pt}%
\definecolor{currentstroke}{rgb}{0.000000,0.000000,0.000000}%
\pgfsetstrokecolor{currentstroke}%
\pgfsetdash{}{0pt}%
\pgfpathmoveto{\pgfqpoint{2.090189in}{2.739383in}}%
\pgfpathlineto{\pgfqpoint{2.104170in}{2.713472in}}%
\pgfpathlineto{\pgfqpoint{2.118134in}{2.687872in}}%
\pgfpathlineto{\pgfqpoint{2.132081in}{2.662582in}}%
\pgfpathlineto{\pgfqpoint{2.146012in}{2.637598in}}%
\pgfpathlineto{\pgfqpoint{2.155201in}{2.628812in}}%
\pgfpathlineto{\pgfqpoint{2.164363in}{2.620424in}}%
\pgfpathlineto{\pgfqpoint{2.173499in}{2.612425in}}%
\pgfpathlineto{\pgfqpoint{2.182607in}{2.604809in}}%
\pgfpathlineto{\pgfqpoint{2.168745in}{2.629109in}}%
\pgfpathlineto{\pgfqpoint{2.154866in}{2.653714in}}%
\pgfpathlineto{\pgfqpoint{2.140971in}{2.678625in}}%
\pgfpathlineto{\pgfqpoint{2.127060in}{2.703846in}}%
\pgfpathlineto{\pgfqpoint{2.117883in}{2.712135in}}%
\pgfpathlineto{\pgfqpoint{2.108680in}{2.720816in}}%
\pgfpathlineto{\pgfqpoint{2.099449in}{2.729896in}}%
\pgfpathlineto{\pgfqpoint{2.090189in}{2.739383in}}%
\pgfpathclose%
\pgfusepath{fill}%
\end{pgfscope}%
\begin{pgfscope}%
\pgfpathrectangle{\pgfqpoint{1.254980in}{0.150000in}}{\pgfqpoint{5.490039in}{5.490039in}}%
\pgfusepath{clip}%
\pgfsetbuttcap%
\pgfsetroundjoin%
\definecolor{currentfill}{rgb}{0.197636,0.391528,0.554969}%
\pgfsetfillcolor{currentfill}%
\pgfsetfillopacity{0.700000}%
\pgfsetlinewidth{0.000000pt}%
\definecolor{currentstroke}{rgb}{0.000000,0.000000,0.000000}%
\pgfsetstrokecolor{currentstroke}%
\pgfsetdash{}{0pt}%
\pgfpathmoveto{\pgfqpoint{2.330964in}{2.305058in}}%
\pgfpathlineto{\pgfqpoint{2.344726in}{2.284155in}}%
\pgfpathlineto{\pgfqpoint{2.358477in}{2.263511in}}%
\pgfpathlineto{\pgfqpoint{2.372216in}{2.243126in}}%
\pgfpathlineto{\pgfqpoint{2.385945in}{2.222996in}}%
\pgfpathlineto{\pgfqpoint{2.394913in}{2.216134in}}%
\pgfpathlineto{\pgfqpoint{2.403856in}{2.209652in}}%
\pgfpathlineto{\pgfqpoint{2.412776in}{2.203542in}}%
\pgfpathlineto{\pgfqpoint{2.421672in}{2.197799in}}%
\pgfpathlineto{\pgfqpoint{2.408004in}{2.217241in}}%
\pgfpathlineto{\pgfqpoint{2.394325in}{2.236937in}}%
\pgfpathlineto{\pgfqpoint{2.380636in}{2.256890in}}%
\pgfpathlineto{\pgfqpoint{2.366936in}{2.277101in}}%
\pgfpathlineto{\pgfqpoint{2.357980in}{2.283521in}}%
\pgfpathlineto{\pgfqpoint{2.348999in}{2.290316in}}%
\pgfpathlineto{\pgfqpoint{2.339994in}{2.297492in}}%
\pgfpathlineto{\pgfqpoint{2.330964in}{2.305058in}}%
\pgfpathclose%
\pgfusepath{fill}%
\end{pgfscope}%
\begin{pgfscope}%
\pgfpathrectangle{\pgfqpoint{1.254980in}{0.150000in}}{\pgfqpoint{5.490039in}{5.490039in}}%
\pgfusepath{clip}%
\pgfsetbuttcap%
\pgfsetroundjoin%
\definecolor{currentfill}{rgb}{0.274128,0.199721,0.498911}%
\pgfsetfillcolor{currentfill}%
\pgfsetfillopacity{0.700000}%
\pgfsetlinewidth{0.000000pt}%
\definecolor{currentstroke}{rgb}{0.000000,0.000000,0.000000}%
\pgfsetstrokecolor{currentstroke}%
\pgfsetdash{}{0pt}%
\pgfpathmoveto{\pgfqpoint{4.088778in}{1.779537in}}%
\pgfpathlineto{\pgfqpoint{4.102350in}{1.784149in}}%
\pgfpathlineto{\pgfqpoint{4.115932in}{1.788921in}}%
\pgfpathlineto{\pgfqpoint{4.129526in}{1.793853in}}%
\pgfpathlineto{\pgfqpoint{4.143131in}{1.798945in}}%
\pgfpathlineto{\pgfqpoint{4.150951in}{1.812204in}}%
\pgfpathlineto{\pgfqpoint{4.158767in}{1.825435in}}%
\pgfpathlineto{\pgfqpoint{4.166579in}{1.838635in}}%
\pgfpathlineto{\pgfqpoint{4.174386in}{1.851800in}}%
\pgfpathlineto{\pgfqpoint{4.160783in}{1.846373in}}%
\pgfpathlineto{\pgfqpoint{4.147191in}{1.841105in}}%
\pgfpathlineto{\pgfqpoint{4.133611in}{1.835998in}}%
\pgfpathlineto{\pgfqpoint{4.120042in}{1.831052in}}%
\pgfpathlineto{\pgfqpoint{4.112233in}{1.818211in}}%
\pgfpathlineto{\pgfqpoint{4.104419in}{1.805343in}}%
\pgfpathlineto{\pgfqpoint{4.096601in}{1.792450in}}%
\pgfpathlineto{\pgfqpoint{4.088778in}{1.779537in}}%
\pgfpathclose%
\pgfusepath{fill}%
\end{pgfscope}%
\begin{pgfscope}%
\pgfpathrectangle{\pgfqpoint{1.254980in}{0.150000in}}{\pgfqpoint{5.490039in}{5.490039in}}%
\pgfusepath{clip}%
\pgfsetbuttcap%
\pgfsetroundjoin%
\definecolor{currentfill}{rgb}{0.277941,0.056324,0.381191}%
\pgfsetfillcolor{currentfill}%
\pgfsetfillopacity{0.700000}%
\pgfsetlinewidth{0.000000pt}%
\definecolor{currentstroke}{rgb}{0.000000,0.000000,0.000000}%
\pgfsetstrokecolor{currentstroke}%
\pgfsetdash{}{0pt}%
\pgfpathmoveto{\pgfqpoint{3.715126in}{1.503680in}}%
\pgfpathlineto{\pgfqpoint{3.728582in}{1.503630in}}%
\pgfpathlineto{\pgfqpoint{3.742046in}{1.503742in}}%
\pgfpathlineto{\pgfqpoint{3.755517in}{1.504017in}}%
\pgfpathlineto{\pgfqpoint{3.768996in}{1.504453in}}%
\pgfpathlineto{\pgfqpoint{3.776933in}{1.515583in}}%
\pgfpathlineto{\pgfqpoint{3.784864in}{1.526789in}}%
\pgfpathlineto{\pgfqpoint{3.792790in}{1.538066in}}%
\pgfpathlineto{\pgfqpoint{3.800710in}{1.549411in}}%
\pgfpathlineto{\pgfqpoint{3.787241in}{1.548504in}}%
\pgfpathlineto{\pgfqpoint{3.773779in}{1.547758in}}%
\pgfpathlineto{\pgfqpoint{3.760326in}{1.547175in}}%
\pgfpathlineto{\pgfqpoint{3.746880in}{1.546755in}}%
\pgfpathlineto{\pgfqpoint{3.738950in}{1.535870in}}%
\pgfpathlineto{\pgfqpoint{3.731015in}{1.525060in}}%
\pgfpathlineto{\pgfqpoint{3.723073in}{1.514329in}}%
\pgfpathlineto{\pgfqpoint{3.715126in}{1.503680in}}%
\pgfpathclose%
\pgfusepath{fill}%
\end{pgfscope}%
\begin{pgfscope}%
\pgfpathrectangle{\pgfqpoint{1.254980in}{0.150000in}}{\pgfqpoint{5.490039in}{5.490039in}}%
\pgfusepath{clip}%
\pgfsetbuttcap%
\pgfsetroundjoin%
\definecolor{currentfill}{rgb}{0.177423,0.437527,0.557565}%
\pgfsetfillcolor{currentfill}%
\pgfsetfillopacity{0.700000}%
\pgfsetlinewidth{0.000000pt}%
\definecolor{currentstroke}{rgb}{0.000000,0.000000,0.000000}%
\pgfsetstrokecolor{currentstroke}%
\pgfsetdash{}{0pt}%
\pgfpathmoveto{\pgfqpoint{4.610460in}{2.330311in}}%
\pgfpathlineto{\pgfqpoint{4.624294in}{2.340027in}}%
\pgfpathlineto{\pgfqpoint{4.638144in}{2.349904in}}%
\pgfpathlineto{\pgfqpoint{4.652009in}{2.359943in}}%
\pgfpathlineto{\pgfqpoint{4.665889in}{2.370143in}}%
\pgfpathlineto{\pgfqpoint{4.673552in}{2.381936in}}%
\pgfpathlineto{\pgfqpoint{4.681209in}{2.393603in}}%
\pgfpathlineto{\pgfqpoint{4.688859in}{2.405143in}}%
\pgfpathlineto{\pgfqpoint{4.696504in}{2.416556in}}%
\pgfpathlineto{\pgfqpoint{4.682623in}{2.406249in}}%
\pgfpathlineto{\pgfqpoint{4.668758in}{2.396103in}}%
\pgfpathlineto{\pgfqpoint{4.654909in}{2.386119in}}%
\pgfpathlineto{\pgfqpoint{4.641076in}{2.376296in}}%
\pgfpathlineto{\pgfqpoint{4.633431in}{2.364979in}}%
\pgfpathlineto{\pgfqpoint{4.625780in}{2.353543in}}%
\pgfpathlineto{\pgfqpoint{4.618123in}{2.341986in}}%
\pgfpathlineto{\pgfqpoint{4.610460in}{2.330311in}}%
\pgfpathclose%
\pgfusepath{fill}%
\end{pgfscope}%
\begin{pgfscope}%
\pgfpathrectangle{\pgfqpoint{1.254980in}{0.150000in}}{\pgfqpoint{5.490039in}{5.490039in}}%
\pgfusepath{clip}%
\pgfsetbuttcap%
\pgfsetroundjoin%
\definecolor{currentfill}{rgb}{0.273809,0.031497,0.358853}%
\pgfsetfillcolor{currentfill}%
\pgfsetfillopacity{0.700000}%
\pgfsetlinewidth{0.000000pt}%
\definecolor{currentstroke}{rgb}{0.000000,0.000000,0.000000}%
\pgfsetstrokecolor{currentstroke}%
\pgfsetdash{}{0pt}%
\pgfpathmoveto{\pgfqpoint{3.123831in}{1.494279in}}%
\pgfpathlineto{\pgfqpoint{3.137265in}{1.486017in}}%
\pgfpathlineto{\pgfqpoint{3.150700in}{1.477935in}}%
\pgfpathlineto{\pgfqpoint{3.164135in}{1.470032in}}%
\pgfpathlineto{\pgfqpoint{3.177572in}{1.462308in}}%
\pgfpathlineto{\pgfqpoint{3.185833in}{1.465878in}}%
\pgfpathlineto{\pgfqpoint{3.194082in}{1.469688in}}%
\pgfpathlineto{\pgfqpoint{3.202319in}{1.473732in}}%
\pgfpathlineto{\pgfqpoint{3.210545in}{1.478004in}}%
\pgfpathlineto{\pgfqpoint{3.197139in}{1.485116in}}%
\pgfpathlineto{\pgfqpoint{3.183734in}{1.492405in}}%
\pgfpathlineto{\pgfqpoint{3.170331in}{1.499874in}}%
\pgfpathlineto{\pgfqpoint{3.156928in}{1.507522in}}%
\pgfpathlineto{\pgfqpoint{3.148673in}{1.503852in}}%
\pgfpathlineto{\pgfqpoint{3.140405in}{1.500417in}}%
\pgfpathlineto{\pgfqpoint{3.132125in}{1.497224in}}%
\pgfpathlineto{\pgfqpoint{3.123831in}{1.494279in}}%
\pgfpathclose%
\pgfusepath{fill}%
\end{pgfscope}%
\begin{pgfscope}%
\pgfpathrectangle{\pgfqpoint{1.254980in}{0.150000in}}{\pgfqpoint{5.490039in}{5.490039in}}%
\pgfusepath{clip}%
\pgfsetbuttcap%
\pgfsetroundjoin%
\definecolor{currentfill}{rgb}{0.283091,0.110553,0.431554}%
\pgfsetfillcolor{currentfill}%
\pgfsetfillopacity{0.700000}%
\pgfsetlinewidth{0.000000pt}%
\definecolor{currentstroke}{rgb}{0.000000,0.000000,0.000000}%
\pgfsetstrokecolor{currentstroke}%
\pgfsetdash{}{0pt}%
\pgfpathmoveto{\pgfqpoint{2.874925in}{1.652831in}}%
\pgfpathlineto{\pgfqpoint{2.888414in}{1.640918in}}%
\pgfpathlineto{\pgfqpoint{2.901900in}{1.629200in}}%
\pgfpathlineto{\pgfqpoint{2.915383in}{1.617676in}}%
\pgfpathlineto{\pgfqpoint{2.928864in}{1.606346in}}%
\pgfpathlineto{\pgfqpoint{2.937324in}{1.606263in}}%
\pgfpathlineto{\pgfqpoint{2.945769in}{1.606476in}}%
\pgfpathlineto{\pgfqpoint{2.954198in}{1.606980in}}%
\pgfpathlineto{\pgfqpoint{2.962612in}{1.607766in}}%
\pgfpathlineto{\pgfqpoint{2.949171in}{1.618447in}}%
\pgfpathlineto{\pgfqpoint{2.935728in}{1.629320in}}%
\pgfpathlineto{\pgfqpoint{2.922283in}{1.640387in}}%
\pgfpathlineto{\pgfqpoint{2.908836in}{1.651649in}}%
\pgfpathlineto{\pgfqpoint{2.900383in}{1.651501in}}%
\pgfpathlineto{\pgfqpoint{2.891913in}{1.651645in}}%
\pgfpathlineto{\pgfqpoint{2.883428in}{1.652086in}}%
\pgfpathlineto{\pgfqpoint{2.874925in}{1.652831in}}%
\pgfpathclose%
\pgfusepath{fill}%
\end{pgfscope}%
\begin{pgfscope}%
\pgfpathrectangle{\pgfqpoint{1.254980in}{0.150000in}}{\pgfqpoint{5.490039in}{5.490039in}}%
\pgfusepath{clip}%
\pgfsetbuttcap%
\pgfsetroundjoin%
\definecolor{currentfill}{rgb}{0.241237,0.296485,0.539709}%
\pgfsetfillcolor{currentfill}%
\pgfsetfillopacity{0.700000}%
\pgfsetlinewidth{0.000000pt}%
\definecolor{currentstroke}{rgb}{0.000000,0.000000,0.000000}%
\pgfsetstrokecolor{currentstroke}%
\pgfsetdash{}{0pt}%
\pgfpathmoveto{\pgfqpoint{4.291211in}{1.981224in}}%
\pgfpathlineto{\pgfqpoint{4.304874in}{1.988040in}}%
\pgfpathlineto{\pgfqpoint{4.318550in}{1.995017in}}%
\pgfpathlineto{\pgfqpoint{4.332240in}{2.002153in}}%
\pgfpathlineto{\pgfqpoint{4.345942in}{2.009450in}}%
\pgfpathlineto{\pgfqpoint{4.353708in}{2.022711in}}%
\pgfpathlineto{\pgfqpoint{4.361470in}{2.035897in}}%
\pgfpathlineto{\pgfqpoint{4.369227in}{2.049006in}}%
\pgfpathlineto{\pgfqpoint{4.376979in}{2.062037in}}%
\pgfpathlineto{\pgfqpoint{4.363276in}{2.054488in}}%
\pgfpathlineto{\pgfqpoint{4.349587in}{2.047099in}}%
\pgfpathlineto{\pgfqpoint{4.335911in}{2.039871in}}%
\pgfpathlineto{\pgfqpoint{4.322248in}{2.032804in}}%
\pgfpathlineto{\pgfqpoint{4.314496in}{2.020014in}}%
\pgfpathlineto{\pgfqpoint{4.306739in}{2.007153in}}%
\pgfpathlineto{\pgfqpoint{4.298977in}{1.994222in}}%
\pgfpathlineto{\pgfqpoint{4.291211in}{1.981224in}}%
\pgfpathclose%
\pgfusepath{fill}%
\end{pgfscope}%
\begin{pgfscope}%
\pgfpathrectangle{\pgfqpoint{1.254980in}{0.150000in}}{\pgfqpoint{5.490039in}{5.490039in}}%
\pgfusepath{clip}%
\pgfsetbuttcap%
\pgfsetroundjoin%
\definecolor{currentfill}{rgb}{0.281446,0.084320,0.407414}%
\pgfsetfillcolor{currentfill}%
\pgfsetfillopacity{0.700000}%
\pgfsetlinewidth{0.000000pt}%
\definecolor{currentstroke}{rgb}{0.000000,0.000000,0.000000}%
\pgfsetstrokecolor{currentstroke}%
\pgfsetdash{}{0pt}%
\pgfpathmoveto{\pgfqpoint{3.800710in}{1.549411in}}%
\pgfpathlineto{\pgfqpoint{3.814187in}{1.550480in}}%
\pgfpathlineto{\pgfqpoint{3.827673in}{1.551711in}}%
\pgfpathlineto{\pgfqpoint{3.841166in}{1.553103in}}%
\pgfpathlineto{\pgfqpoint{3.854669in}{1.554656in}}%
\pgfpathlineto{\pgfqpoint{3.862576in}{1.566518in}}%
\pgfpathlineto{\pgfqpoint{3.870477in}{1.578432in}}%
\pgfpathlineto{\pgfqpoint{3.878374in}{1.590393in}}%
\pgfpathlineto{\pgfqpoint{3.886266in}{1.602398in}}%
\pgfpathlineto{\pgfqpoint{3.872771in}{1.600401in}}%
\pgfpathlineto{\pgfqpoint{3.859284in}{1.598565in}}%
\pgfpathlineto{\pgfqpoint{3.845807in}{1.596890in}}%
\pgfpathlineto{\pgfqpoint{3.832338in}{1.595378in}}%
\pgfpathlineto{\pgfqpoint{3.824438in}{1.583806in}}%
\pgfpathlineto{\pgfqpoint{3.816534in}{1.572285in}}%
\pgfpathlineto{\pgfqpoint{3.808625in}{1.560819in}}%
\pgfpathlineto{\pgfqpoint{3.800710in}{1.549411in}}%
\pgfpathclose%
\pgfusepath{fill}%
\end{pgfscope}%
\begin{pgfscope}%
\pgfpathrectangle{\pgfqpoint{1.254980in}{0.150000in}}{\pgfqpoint{5.490039in}{5.490039in}}%
\pgfusepath{clip}%
\pgfsetbuttcap%
\pgfsetroundjoin%
\definecolor{currentfill}{rgb}{0.273809,0.031497,0.358853}%
\pgfsetfillcolor{currentfill}%
\pgfsetfillopacity{0.700000}%
\pgfsetlinewidth{0.000000pt}%
\definecolor{currentstroke}{rgb}{0.000000,0.000000,0.000000}%
\pgfsetstrokecolor{currentstroke}%
\pgfsetdash{}{0pt}%
\pgfpathmoveto{\pgfqpoint{3.629472in}{1.465835in}}%
\pgfpathlineto{\pgfqpoint{3.642913in}{1.464634in}}%
\pgfpathlineto{\pgfqpoint{3.656361in}{1.463596in}}%
\pgfpathlineto{\pgfqpoint{3.669816in}{1.462722in}}%
\pgfpathlineto{\pgfqpoint{3.683277in}{1.462011in}}%
\pgfpathlineto{\pgfqpoint{3.691249in}{1.472280in}}%
\pgfpathlineto{\pgfqpoint{3.699214in}{1.482651in}}%
\pgfpathlineto{\pgfqpoint{3.707173in}{1.493119in}}%
\pgfpathlineto{\pgfqpoint{3.715126in}{1.503680in}}%
\pgfpathlineto{\pgfqpoint{3.701677in}{1.503893in}}%
\pgfpathlineto{\pgfqpoint{3.688235in}{1.504269in}}%
\pgfpathlineto{\pgfqpoint{3.674800in}{1.504809in}}%
\pgfpathlineto{\pgfqpoint{3.661372in}{1.505512in}}%
\pgfpathlineto{\pgfqpoint{3.653406in}{1.495439in}}%
\pgfpathlineto{\pgfqpoint{3.645435in}{1.485465in}}%
\pgfpathlineto{\pgfqpoint{3.637457in}{1.475595in}}%
\pgfpathlineto{\pgfqpoint{3.629472in}{1.465835in}}%
\pgfpathclose%
\pgfusepath{fill}%
\end{pgfscope}%
\begin{pgfscope}%
\pgfpathrectangle{\pgfqpoint{1.254980in}{0.150000in}}{\pgfqpoint{5.490039in}{5.490039in}}%
\pgfusepath{clip}%
\pgfsetbuttcap%
\pgfsetroundjoin%
\definecolor{currentfill}{rgb}{0.283197,0.115680,0.436115}%
\pgfsetfillcolor{currentfill}%
\pgfsetfillopacity{0.700000}%
\pgfsetlinewidth{0.000000pt}%
\definecolor{currentstroke}{rgb}{0.000000,0.000000,0.000000}%
\pgfsetstrokecolor{currentstroke}%
\pgfsetdash{}{0pt}%
\pgfpathmoveto{\pgfqpoint{3.886266in}{1.602398in}}%
\pgfpathlineto{\pgfqpoint{3.899770in}{1.604556in}}%
\pgfpathlineto{\pgfqpoint{3.913283in}{1.606875in}}%
\pgfpathlineto{\pgfqpoint{3.926805in}{1.609354in}}%
\pgfpathlineto{\pgfqpoint{3.940337in}{1.611994in}}%
\pgfpathlineto{\pgfqpoint{3.948218in}{1.624466in}}%
\pgfpathlineto{\pgfqpoint{3.956094in}{1.636966in}}%
\pgfpathlineto{\pgfqpoint{3.963966in}{1.649490in}}%
\pgfpathlineto{\pgfqpoint{3.971832in}{1.662035in}}%
\pgfpathlineto{\pgfqpoint{3.958306in}{1.658978in}}%
\pgfpathlineto{\pgfqpoint{3.944789in}{1.656081in}}%
\pgfpathlineto{\pgfqpoint{3.931281in}{1.653345in}}%
\pgfpathlineto{\pgfqpoint{3.917783in}{1.650770in}}%
\pgfpathlineto{\pgfqpoint{3.909911in}{1.638632in}}%
\pgfpathlineto{\pgfqpoint{3.902034in}{1.626521in}}%
\pgfpathlineto{\pgfqpoint{3.894152in}{1.614442in}}%
\pgfpathlineto{\pgfqpoint{3.886266in}{1.602398in}}%
\pgfpathclose%
\pgfusepath{fill}%
\end{pgfscope}%
\begin{pgfscope}%
\pgfpathrectangle{\pgfqpoint{1.254980in}{0.150000in}}{\pgfqpoint{5.490039in}{5.490039in}}%
\pgfusepath{clip}%
\pgfsetbuttcap%
\pgfsetroundjoin%
\definecolor{currentfill}{rgb}{0.140210,0.665859,0.513427}%
\pgfsetfillcolor{currentfill}%
\pgfsetfillopacity{0.700000}%
\pgfsetlinewidth{0.000000pt}%
\definecolor{currentstroke}{rgb}{0.000000,0.000000,0.000000}%
\pgfsetstrokecolor{currentstroke}%
\pgfsetdash{}{0pt}%
\pgfpathmoveto{\pgfqpoint{5.218121in}{2.950054in}}%
\pgfpathlineto{\pgfqpoint{5.232342in}{2.963530in}}%
\pgfpathlineto{\pgfqpoint{5.246582in}{2.977169in}}%
\pgfpathlineto{\pgfqpoint{5.260842in}{2.990971in}}%
\pgfpathlineto{\pgfqpoint{5.275121in}{3.004937in}}%
\pgfpathlineto{\pgfqpoint{5.282482in}{3.011352in}}%
\pgfpathlineto{\pgfqpoint{5.289834in}{3.017618in}}%
\pgfpathlineto{\pgfqpoint{5.297177in}{3.023737in}}%
\pgfpathlineto{\pgfqpoint{5.304510in}{3.029712in}}%
\pgfpathlineto{\pgfqpoint{5.290240in}{3.015917in}}%
\pgfpathlineto{\pgfqpoint{5.275991in}{3.002285in}}%
\pgfpathlineto{\pgfqpoint{5.261761in}{2.988815in}}%
\pgfpathlineto{\pgfqpoint{5.247550in}{2.975508in}}%
\pgfpathlineto{\pgfqpoint{5.240206in}{2.969352in}}%
\pgfpathlineto{\pgfqpoint{5.232853in}{2.963059in}}%
\pgfpathlineto{\pgfqpoint{5.225492in}{2.956627in}}%
\pgfpathlineto{\pgfqpoint{5.218121in}{2.950054in}}%
\pgfpathclose%
\pgfusepath{fill}%
\end{pgfscope}%
\begin{pgfscope}%
\pgfpathrectangle{\pgfqpoint{1.254980in}{0.150000in}}{\pgfqpoint{5.490039in}{5.490039in}}%
\pgfusepath{clip}%
\pgfsetbuttcap%
\pgfsetroundjoin%
\definecolor{currentfill}{rgb}{0.269944,0.014625,0.341379}%
\pgfsetfillcolor{currentfill}%
\pgfsetfillopacity{0.700000}%
\pgfsetlinewidth{0.000000pt}%
\definecolor{currentstroke}{rgb}{0.000000,0.000000,0.000000}%
\pgfsetstrokecolor{currentstroke}%
\pgfsetdash{}{0pt}%
\pgfpathmoveto{\pgfqpoint{3.543699in}{1.436530in}}%
\pgfpathlineto{\pgfqpoint{3.557132in}{1.434145in}}%
\pgfpathlineto{\pgfqpoint{3.570571in}{1.431925in}}%
\pgfpathlineto{\pgfqpoint{3.584015in}{1.429870in}}%
\pgfpathlineto{\pgfqpoint{3.597466in}{1.427980in}}%
\pgfpathlineto{\pgfqpoint{3.605478in}{1.437255in}}%
\pgfpathlineto{\pgfqpoint{3.613483in}{1.446659in}}%
\pgfpathlineto{\pgfqpoint{3.621481in}{1.456188in}}%
\pgfpathlineto{\pgfqpoint{3.629472in}{1.465835in}}%
\pgfpathlineto{\pgfqpoint{3.616037in}{1.467200in}}%
\pgfpathlineto{\pgfqpoint{3.602608in}{1.468729in}}%
\pgfpathlineto{\pgfqpoint{3.589185in}{1.470424in}}%
\pgfpathlineto{\pgfqpoint{3.575767in}{1.472283in}}%
\pgfpathlineto{\pgfqpoint{3.567761in}{1.463151in}}%
\pgfpathlineto{\pgfqpoint{3.559748in}{1.454145in}}%
\pgfpathlineto{\pgfqpoint{3.551727in}{1.445269in}}%
\pgfpathlineto{\pgfqpoint{3.543699in}{1.436530in}}%
\pgfpathclose%
\pgfusepath{fill}%
\end{pgfscope}%
\begin{pgfscope}%
\pgfpathrectangle{\pgfqpoint{1.254980in}{0.150000in}}{\pgfqpoint{5.490039in}{5.490039in}}%
\pgfusepath{clip}%
\pgfsetbuttcap%
\pgfsetroundjoin%
\definecolor{currentfill}{rgb}{0.144759,0.519093,0.556572}%
\pgfsetfillcolor{currentfill}%
\pgfsetfillopacity{0.700000}%
\pgfsetlinewidth{0.000000pt}%
\definecolor{currentstroke}{rgb}{0.000000,0.000000,0.000000}%
\pgfsetstrokecolor{currentstroke}%
\pgfsetdash{}{0pt}%
\pgfpathmoveto{\pgfqpoint{4.813104in}{2.546545in}}%
\pgfpathlineto{\pgfqpoint{4.827064in}{2.557784in}}%
\pgfpathlineto{\pgfqpoint{4.841042in}{2.569185in}}%
\pgfpathlineto{\pgfqpoint{4.855036in}{2.580749in}}%
\pgfpathlineto{\pgfqpoint{4.869048in}{2.592474in}}%
\pgfpathlineto{\pgfqpoint{4.876630in}{2.602778in}}%
\pgfpathlineto{\pgfqpoint{4.884205in}{2.612936in}}%
\pgfpathlineto{\pgfqpoint{4.891773in}{2.622949in}}%
\pgfpathlineto{\pgfqpoint{4.899333in}{2.632818in}}%
\pgfpathlineto{\pgfqpoint{4.885323in}{2.621076in}}%
\pgfpathlineto{\pgfqpoint{4.871331in}{2.609496in}}%
\pgfpathlineto{\pgfqpoint{4.857356in}{2.598078in}}%
\pgfpathlineto{\pgfqpoint{4.843398in}{2.586821in}}%
\pgfpathlineto{\pgfqpoint{4.835835in}{2.576958in}}%
\pgfpathlineto{\pgfqpoint{4.828265in}{2.566957in}}%
\pgfpathlineto{\pgfqpoint{4.820688in}{2.556820in}}%
\pgfpathlineto{\pgfqpoint{4.813104in}{2.546545in}}%
\pgfpathclose%
\pgfusepath{fill}%
\end{pgfscope}%
\begin{pgfscope}%
\pgfpathrectangle{\pgfqpoint{1.254980in}{0.150000in}}{\pgfqpoint{5.490039in}{5.490039in}}%
\pgfusepath{clip}%
\pgfsetbuttcap%
\pgfsetroundjoin%
\definecolor{currentfill}{rgb}{0.182256,0.426184,0.557120}%
\pgfsetfillcolor{currentfill}%
\pgfsetfillopacity{0.700000}%
\pgfsetlinewidth{0.000000pt}%
\definecolor{currentstroke}{rgb}{0.000000,0.000000,0.000000}%
\pgfsetstrokecolor{currentstroke}%
\pgfsetdash{}{0pt}%
\pgfpathmoveto{\pgfqpoint{2.275795in}{2.391316in}}%
\pgfpathlineto{\pgfqpoint{2.289606in}{2.369350in}}%
\pgfpathlineto{\pgfqpoint{2.303404in}{2.347653in}}%
\pgfpathlineto{\pgfqpoint{2.317190in}{2.326223in}}%
\pgfpathlineto{\pgfqpoint{2.330964in}{2.305058in}}%
\pgfpathlineto{\pgfqpoint{2.339994in}{2.297492in}}%
\pgfpathlineto{\pgfqpoint{2.348999in}{2.290316in}}%
\pgfpathlineto{\pgfqpoint{2.357980in}{2.283521in}}%
\pgfpathlineto{\pgfqpoint{2.366936in}{2.277101in}}%
\pgfpathlineto{\pgfqpoint{2.353224in}{2.297572in}}%
\pgfpathlineto{\pgfqpoint{2.339501in}{2.318307in}}%
\pgfpathlineto{\pgfqpoint{2.325766in}{2.339307in}}%
\pgfpathlineto{\pgfqpoint{2.312020in}{2.360575in}}%
\pgfpathlineto{\pgfqpoint{2.303002in}{2.367678in}}%
\pgfpathlineto{\pgfqpoint{2.293959in}{2.375164in}}%
\pgfpathlineto{\pgfqpoint{2.284890in}{2.383041in}}%
\pgfpathlineto{\pgfqpoint{2.275795in}{2.391316in}}%
\pgfpathclose%
\pgfusepath{fill}%
\end{pgfscope}%
\begin{pgfscope}%
\pgfpathrectangle{\pgfqpoint{1.254980in}{0.150000in}}{\pgfqpoint{5.490039in}{5.490039in}}%
\pgfusepath{clip}%
\pgfsetbuttcap%
\pgfsetroundjoin%
\definecolor{currentfill}{rgb}{0.274149,0.751988,0.436601}%
\pgfsetfillcolor{currentfill}%
\pgfsetfillopacity{0.700000}%
\pgfsetlinewidth{0.000000pt}%
\definecolor{currentstroke}{rgb}{0.000000,0.000000,0.000000}%
\pgfsetstrokecolor{currentstroke}%
\pgfsetdash{}{0pt}%
\pgfpathmoveto{\pgfqpoint{5.506342in}{3.201690in}}%
\pgfpathlineto{\pgfqpoint{5.520754in}{3.216253in}}%
\pgfpathlineto{\pgfqpoint{5.535187in}{3.230979in}}%
\pgfpathlineto{\pgfqpoint{5.549641in}{3.245868in}}%
\pgfpathlineto{\pgfqpoint{5.564117in}{3.260922in}}%
\pgfpathlineto{\pgfqpoint{5.571281in}{3.264401in}}%
\pgfpathlineto{\pgfqpoint{5.578434in}{3.267752in}}%
\pgfpathlineto{\pgfqpoint{5.585578in}{3.270978in}}%
\pgfpathlineto{\pgfqpoint{5.592711in}{3.274082in}}%
\pgfpathlineto{\pgfqpoint{5.578253in}{3.259327in}}%
\pgfpathlineto{\pgfqpoint{5.563817in}{3.244735in}}%
\pgfpathlineto{\pgfqpoint{5.549401in}{3.230305in}}%
\pgfpathlineto{\pgfqpoint{5.535006in}{3.216038in}}%
\pgfpathlineto{\pgfqpoint{5.527855in}{3.212625in}}%
\pgfpathlineto{\pgfqpoint{5.520694in}{3.209099in}}%
\pgfpathlineto{\pgfqpoint{5.513523in}{3.205455in}}%
\pgfpathlineto{\pgfqpoint{5.506342in}{3.201690in}}%
\pgfpathclose%
\pgfusepath{fill}%
\end{pgfscope}%
\begin{pgfscope}%
\pgfpathrectangle{\pgfqpoint{1.254980in}{0.150000in}}{\pgfqpoint{5.490039in}{5.490039in}}%
\pgfusepath{clip}%
\pgfsetbuttcap%
\pgfsetroundjoin%
\definecolor{currentfill}{rgb}{0.199430,0.387607,0.554642}%
\pgfsetfillcolor{currentfill}%
\pgfsetfillopacity{0.700000}%
\pgfsetlinewidth{0.000000pt}%
\definecolor{currentstroke}{rgb}{0.000000,0.000000,0.000000}%
\pgfsetstrokecolor{currentstroke}%
\pgfsetdash{}{0pt}%
\pgfpathmoveto{\pgfqpoint{4.493770in}{2.196692in}}%
\pgfpathlineto{\pgfqpoint{4.507544in}{2.205462in}}%
\pgfpathlineto{\pgfqpoint{4.521332in}{2.214392in}}%
\pgfpathlineto{\pgfqpoint{4.535135in}{2.223484in}}%
\pgfpathlineto{\pgfqpoint{4.548953in}{2.232737in}}%
\pgfpathlineto{\pgfqpoint{4.556661in}{2.245329in}}%
\pgfpathlineto{\pgfqpoint{4.564363in}{2.257810in}}%
\pgfpathlineto{\pgfqpoint{4.572060in}{2.270180in}}%
\pgfpathlineto{\pgfqpoint{4.579751in}{2.282437in}}%
\pgfpathlineto{\pgfqpoint{4.565933in}{2.273018in}}%
\pgfpathlineto{\pgfqpoint{4.552129in}{2.263761in}}%
\pgfpathlineto{\pgfqpoint{4.538341in}{2.254664in}}%
\pgfpathlineto{\pgfqpoint{4.524567in}{2.245729in}}%
\pgfpathlineto{\pgfqpoint{4.516876in}{2.233627in}}%
\pgfpathlineto{\pgfqpoint{4.509179in}{2.221419in}}%
\pgfpathlineto{\pgfqpoint{4.501477in}{2.209107in}}%
\pgfpathlineto{\pgfqpoint{4.493770in}{2.196692in}}%
\pgfpathclose%
\pgfusepath{fill}%
\end{pgfscope}%
\begin{pgfscope}%
\pgfpathrectangle{\pgfqpoint{1.254980in}{0.150000in}}{\pgfqpoint{5.490039in}{5.490039in}}%
\pgfusepath{clip}%
\pgfsetbuttcap%
\pgfsetroundjoin%
\definecolor{currentfill}{rgb}{0.281924,0.089666,0.412415}%
\pgfsetfillcolor{currentfill}%
\pgfsetfillopacity{0.700000}%
\pgfsetlinewidth{0.000000pt}%
\definecolor{currentstroke}{rgb}{0.000000,0.000000,0.000000}%
\pgfsetstrokecolor{currentstroke}%
\pgfsetdash{}{0pt}%
\pgfpathmoveto{\pgfqpoint{2.928864in}{1.606346in}}%
\pgfpathlineto{\pgfqpoint{2.942343in}{1.595208in}}%
\pgfpathlineto{\pgfqpoint{2.955820in}{1.584261in}}%
\pgfpathlineto{\pgfqpoint{2.969296in}{1.573505in}}%
\pgfpathlineto{\pgfqpoint{2.982770in}{1.562938in}}%
\pgfpathlineto{\pgfqpoint{2.991190in}{1.563514in}}%
\pgfpathlineto{\pgfqpoint{2.999596in}{1.564380in}}%
\pgfpathlineto{\pgfqpoint{3.007986in}{1.565528in}}%
\pgfpathlineto{\pgfqpoint{3.016362in}{1.566951in}}%
\pgfpathlineto{\pgfqpoint{3.002927in}{1.576871in}}%
\pgfpathlineto{\pgfqpoint{2.989490in}{1.586979in}}%
\pgfpathlineto{\pgfqpoint{2.976052in}{1.597277in}}%
\pgfpathlineto{\pgfqpoint{2.962612in}{1.607766in}}%
\pgfpathlineto{\pgfqpoint{2.954198in}{1.606980in}}%
\pgfpathlineto{\pgfqpoint{2.945769in}{1.606476in}}%
\pgfpathlineto{\pgfqpoint{2.937324in}{1.606263in}}%
\pgfpathlineto{\pgfqpoint{2.928864in}{1.606346in}}%
\pgfpathclose%
\pgfusepath{fill}%
\end{pgfscope}%
\begin{pgfscope}%
\pgfpathrectangle{\pgfqpoint{1.254980in}{0.150000in}}{\pgfqpoint{5.490039in}{5.490039in}}%
\pgfusepath{clip}%
\pgfsetbuttcap%
\pgfsetroundjoin%
\definecolor{currentfill}{rgb}{0.120565,0.596422,0.543611}%
\pgfsetfillcolor{currentfill}%
\pgfsetfillopacity{0.700000}%
\pgfsetlinewidth{0.000000pt}%
\definecolor{currentstroke}{rgb}{0.000000,0.000000,0.000000}%
\pgfsetstrokecolor{currentstroke}%
\pgfsetdash{}{0pt}%
\pgfpathmoveto{\pgfqpoint{5.015730in}{2.754988in}}%
\pgfpathlineto{\pgfqpoint{5.029821in}{2.767483in}}%
\pgfpathlineto{\pgfqpoint{5.043930in}{2.780141in}}%
\pgfpathlineto{\pgfqpoint{5.058058in}{2.792962in}}%
\pgfpathlineto{\pgfqpoint{5.072204in}{2.805945in}}%
\pgfpathlineto{\pgfqpoint{5.079686in}{2.814412in}}%
\pgfpathlineto{\pgfqpoint{5.087160in}{2.822727in}}%
\pgfpathlineto{\pgfqpoint{5.094625in}{2.830890in}}%
\pgfpathlineto{\pgfqpoint{5.102082in}{2.838903in}}%
\pgfpathlineto{\pgfqpoint{5.087941in}{2.825996in}}%
\pgfpathlineto{\pgfqpoint{5.073819in}{2.813251in}}%
\pgfpathlineto{\pgfqpoint{5.059715in}{2.800668in}}%
\pgfpathlineto{\pgfqpoint{5.045630in}{2.788248in}}%
\pgfpathlineto{\pgfqpoint{5.038167in}{2.780148in}}%
\pgfpathlineto{\pgfqpoint{5.030696in}{2.771906in}}%
\pgfpathlineto{\pgfqpoint{5.023217in}{2.763519in}}%
\pgfpathlineto{\pgfqpoint{5.015730in}{2.754988in}}%
\pgfpathclose%
\pgfusepath{fill}%
\end{pgfscope}%
\begin{pgfscope}%
\pgfpathrectangle{\pgfqpoint{1.254980in}{0.150000in}}{\pgfqpoint{5.490039in}{5.490039in}}%
\pgfusepath{clip}%
\pgfsetbuttcap%
\pgfsetroundjoin%
\definecolor{currentfill}{rgb}{0.267004,0.004874,0.329415}%
\pgfsetfillcolor{currentfill}%
\pgfsetfillopacity{0.700000}%
\pgfsetlinewidth{0.000000pt}%
\definecolor{currentstroke}{rgb}{0.000000,0.000000,0.000000}%
\pgfsetstrokecolor{currentstroke}%
\pgfsetdash{}{0pt}%
\pgfpathmoveto{\pgfqpoint{3.317861in}{1.427440in}}%
\pgfpathlineto{\pgfqpoint{3.331287in}{1.421900in}}%
\pgfpathlineto{\pgfqpoint{3.344715in}{1.416532in}}%
\pgfpathlineto{\pgfqpoint{3.358147in}{1.411334in}}%
\pgfpathlineto{\pgfqpoint{3.371581in}{1.406307in}}%
\pgfpathlineto{\pgfqpoint{3.379717in}{1.412580in}}%
\pgfpathlineto{\pgfqpoint{3.387844in}{1.419047in}}%
\pgfpathlineto{\pgfqpoint{3.395961in}{1.425703in}}%
\pgfpathlineto{\pgfqpoint{3.404069in}{1.432542in}}%
\pgfpathlineto{\pgfqpoint{3.390658in}{1.436987in}}%
\pgfpathlineto{\pgfqpoint{3.377250in}{1.441603in}}%
\pgfpathlineto{\pgfqpoint{3.363845in}{1.446390in}}%
\pgfpathlineto{\pgfqpoint{3.350444in}{1.451347in}}%
\pgfpathlineto{\pgfqpoint{3.342313in}{1.445080in}}%
\pgfpathlineto{\pgfqpoint{3.334173in}{1.439003in}}%
\pgfpathlineto{\pgfqpoint{3.326022in}{1.433120in}}%
\pgfpathlineto{\pgfqpoint{3.317861in}{1.427440in}}%
\pgfpathclose%
\pgfusepath{fill}%
\end{pgfscope}%
\begin{pgfscope}%
\pgfpathrectangle{\pgfqpoint{1.254980in}{0.150000in}}{\pgfqpoint{5.490039in}{5.490039in}}%
\pgfusepath{clip}%
\pgfsetbuttcap%
\pgfsetroundjoin%
\definecolor{currentfill}{rgb}{0.262138,0.242286,0.520837}%
\pgfsetfillcolor{currentfill}%
\pgfsetfillopacity{0.700000}%
\pgfsetlinewidth{0.000000pt}%
\definecolor{currentstroke}{rgb}{0.000000,0.000000,0.000000}%
\pgfsetstrokecolor{currentstroke}%
\pgfsetdash{}{0pt}%
\pgfpathmoveto{\pgfqpoint{4.174386in}{1.851800in}}%
\pgfpathlineto{\pgfqpoint{4.188002in}{1.857388in}}%
\pgfpathlineto{\pgfqpoint{4.201629in}{1.863136in}}%
\pgfpathlineto{\pgfqpoint{4.215268in}{1.869043in}}%
\pgfpathlineto{\pgfqpoint{4.228919in}{1.875111in}}%
\pgfpathlineto{\pgfqpoint{4.236721in}{1.888558in}}%
\pgfpathlineto{\pgfqpoint{4.244519in}{1.901959in}}%
\pgfpathlineto{\pgfqpoint{4.252312in}{1.915309in}}%
\pgfpathlineto{\pgfqpoint{4.260101in}{1.928608in}}%
\pgfpathlineto{\pgfqpoint{4.246450in}{1.922232in}}%
\pgfpathlineto{\pgfqpoint{4.232812in}{1.916016in}}%
\pgfpathlineto{\pgfqpoint{4.219185in}{1.909960in}}%
\pgfpathlineto{\pgfqpoint{4.205571in}{1.904065in}}%
\pgfpathlineto{\pgfqpoint{4.197782in}{1.891064in}}%
\pgfpathlineto{\pgfqpoint{4.189988in}{1.878018in}}%
\pgfpathlineto{\pgfqpoint{4.182189in}{1.864929in}}%
\pgfpathlineto{\pgfqpoint{4.174386in}{1.851800in}}%
\pgfpathclose%
\pgfusepath{fill}%
\end{pgfscope}%
\begin{pgfscope}%
\pgfpathrectangle{\pgfqpoint{1.254980in}{0.150000in}}{\pgfqpoint{5.490039in}{5.490039in}}%
\pgfusepath{clip}%
\pgfsetbuttcap%
\pgfsetroundjoin%
\definecolor{currentfill}{rgb}{0.281887,0.150881,0.465405}%
\pgfsetfillcolor{currentfill}%
\pgfsetfillopacity{0.700000}%
\pgfsetlinewidth{0.000000pt}%
\definecolor{currentstroke}{rgb}{0.000000,0.000000,0.000000}%
\pgfsetstrokecolor{currentstroke}%
\pgfsetdash{}{0pt}%
\pgfpathmoveto{\pgfqpoint{3.971832in}{1.662035in}}%
\pgfpathlineto{\pgfqpoint{3.985369in}{1.665253in}}%
\pgfpathlineto{\pgfqpoint{3.998915in}{1.668630in}}%
\pgfpathlineto{\pgfqpoint{4.012472in}{1.672168in}}%
\pgfpathlineto{\pgfqpoint{4.026039in}{1.675866in}}%
\pgfpathlineto{\pgfqpoint{4.033897in}{1.688829in}}%
\pgfpathlineto{\pgfqpoint{4.041750in}{1.701797in}}%
\pgfpathlineto{\pgfqpoint{4.049599in}{1.714768in}}%
\pgfpathlineto{\pgfqpoint{4.057444in}{1.727738in}}%
\pgfpathlineto{\pgfqpoint{4.043881in}{1.723649in}}%
\pgfpathlineto{\pgfqpoint{4.030328in}{1.719721in}}%
\pgfpathlineto{\pgfqpoint{4.016785in}{1.715953in}}%
\pgfpathlineto{\pgfqpoint{4.003253in}{1.712345in}}%
\pgfpathlineto{\pgfqpoint{3.995405in}{1.699755in}}%
\pgfpathlineto{\pgfqpoint{3.987552in}{1.687171in}}%
\pgfpathlineto{\pgfqpoint{3.979694in}{1.674596in}}%
\pgfpathlineto{\pgfqpoint{3.971832in}{1.662035in}}%
\pgfpathclose%
\pgfusepath{fill}%
\end{pgfscope}%
\begin{pgfscope}%
\pgfpathrectangle{\pgfqpoint{1.254980in}{0.150000in}}{\pgfqpoint{5.490039in}{5.490039in}}%
\pgfusepath{clip}%
\pgfsetbuttcap%
\pgfsetroundjoin%
\definecolor{currentfill}{rgb}{0.271305,0.019942,0.347269}%
\pgfsetfillcolor{currentfill}%
\pgfsetfillopacity{0.700000}%
\pgfsetlinewidth{0.000000pt}%
\definecolor{currentstroke}{rgb}{0.000000,0.000000,0.000000}%
\pgfsetstrokecolor{currentstroke}%
\pgfsetdash{}{0pt}%
\pgfpathmoveto{\pgfqpoint{3.177572in}{1.462308in}}%
\pgfpathlineto{\pgfqpoint{3.191009in}{1.454762in}}%
\pgfpathlineto{\pgfqpoint{3.204447in}{1.447392in}}%
\pgfpathlineto{\pgfqpoint{3.217888in}{1.440198in}}%
\pgfpathlineto{\pgfqpoint{3.231329in}{1.433180in}}%
\pgfpathlineto{\pgfqpoint{3.239560in}{1.437374in}}%
\pgfpathlineto{\pgfqpoint{3.247780in}{1.441800in}}%
\pgfpathlineto{\pgfqpoint{3.255989in}{1.446452in}}%
\pgfpathlineto{\pgfqpoint{3.264186in}{1.451325in}}%
\pgfpathlineto{\pgfqpoint{3.250773in}{1.457731in}}%
\pgfpathlineto{\pgfqpoint{3.237362in}{1.464313in}}%
\pgfpathlineto{\pgfqpoint{3.223953in}{1.471070in}}%
\pgfpathlineto{\pgfqpoint{3.210545in}{1.478004in}}%
\pgfpathlineto{\pgfqpoint{3.202319in}{1.473732in}}%
\pgfpathlineto{\pgfqpoint{3.194082in}{1.469688in}}%
\pgfpathlineto{\pgfqpoint{3.185833in}{1.465878in}}%
\pgfpathlineto{\pgfqpoint{3.177572in}{1.462308in}}%
\pgfpathclose%
\pgfusepath{fill}%
\end{pgfscope}%
\begin{pgfscope}%
\pgfpathrectangle{\pgfqpoint{1.254980in}{0.150000in}}{\pgfqpoint{5.490039in}{5.490039in}}%
\pgfusepath{clip}%
\pgfsetbuttcap%
\pgfsetroundjoin%
\definecolor{currentfill}{rgb}{0.121831,0.589055,0.545623}%
\pgfsetfillcolor{currentfill}%
\pgfsetfillopacity{0.700000}%
\pgfsetlinewidth{0.000000pt}%
\definecolor{currentstroke}{rgb}{0.000000,0.000000,0.000000}%
\pgfsetstrokecolor{currentstroke}%
\pgfsetdash{}{0pt}%
\pgfpathmoveto{\pgfqpoint{2.034086in}{2.846212in}}%
\pgfpathlineto{\pgfqpoint{2.048139in}{2.819020in}}%
\pgfpathlineto{\pgfqpoint{2.062174in}{2.792154in}}%
\pgfpathlineto{\pgfqpoint{2.076190in}{2.765609in}}%
\pgfpathlineto{\pgfqpoint{2.090189in}{2.739383in}}%
\pgfpathlineto{\pgfqpoint{2.099449in}{2.729896in}}%
\pgfpathlineto{\pgfqpoint{2.108680in}{2.720816in}}%
\pgfpathlineto{\pgfqpoint{2.117883in}{2.712135in}}%
\pgfpathlineto{\pgfqpoint{2.127060in}{2.703846in}}%
\pgfpathlineto{\pgfqpoint{2.113131in}{2.729381in}}%
\pgfpathlineto{\pgfqpoint{2.099186in}{2.755232in}}%
\pgfpathlineto{\pgfqpoint{2.085223in}{2.781403in}}%
\pgfpathlineto{\pgfqpoint{2.071242in}{2.807896in}}%
\pgfpathlineto{\pgfqpoint{2.061996in}{2.816866in}}%
\pgfpathlineto{\pgfqpoint{2.052722in}{2.826237in}}%
\pgfpathlineto{\pgfqpoint{2.043419in}{2.836016in}}%
\pgfpathlineto{\pgfqpoint{2.034086in}{2.846212in}}%
\pgfpathclose%
\pgfusepath{fill}%
\end{pgfscope}%
\begin{pgfscope}%
\pgfpathrectangle{\pgfqpoint{1.254980in}{0.150000in}}{\pgfqpoint{5.490039in}{5.490039in}}%
\pgfusepath{clip}%
\pgfsetbuttcap%
\pgfsetroundjoin%
\definecolor{currentfill}{rgb}{0.267004,0.004874,0.329415}%
\pgfsetfillcolor{currentfill}%
\pgfsetfillopacity{0.700000}%
\pgfsetlinewidth{0.000000pt}%
\definecolor{currentstroke}{rgb}{0.000000,0.000000,0.000000}%
\pgfsetstrokecolor{currentstroke}%
\pgfsetdash{}{0pt}%
\pgfpathmoveto{\pgfqpoint{3.457755in}{1.416449in}}%
\pgfpathlineto{\pgfqpoint{3.471187in}{1.412846in}}%
\pgfpathlineto{\pgfqpoint{3.484623in}{1.409410in}}%
\pgfpathlineto{\pgfqpoint{3.498064in}{1.406140in}}%
\pgfpathlineto{\pgfqpoint{3.511510in}{1.403036in}}%
\pgfpathlineto{\pgfqpoint{3.519569in}{1.411179in}}%
\pgfpathlineto{\pgfqpoint{3.527620in}{1.419480in}}%
\pgfpathlineto{\pgfqpoint{3.535664in}{1.427931in}}%
\pgfpathlineto{\pgfqpoint{3.543699in}{1.436530in}}%
\pgfpathlineto{\pgfqpoint{3.530271in}{1.439080in}}%
\pgfpathlineto{\pgfqpoint{3.516848in}{1.441797in}}%
\pgfpathlineto{\pgfqpoint{3.503431in}{1.444680in}}%
\pgfpathlineto{\pgfqpoint{3.490018in}{1.447730in}}%
\pgfpathlineto{\pgfqpoint{3.481964in}{1.439674in}}%
\pgfpathlineto{\pgfqpoint{3.473903in}{1.431772in}}%
\pgfpathlineto{\pgfqpoint{3.465833in}{1.424028in}}%
\pgfpathlineto{\pgfqpoint{3.457755in}{1.416449in}}%
\pgfpathclose%
\pgfusepath{fill}%
\end{pgfscope}%
\begin{pgfscope}%
\pgfpathrectangle{\pgfqpoint{1.254980in}{0.150000in}}{\pgfqpoint{5.490039in}{5.490039in}}%
\pgfusepath{clip}%
\pgfsetbuttcap%
\pgfsetroundjoin%
\definecolor{currentfill}{rgb}{0.221989,0.339161,0.548752}%
\pgfsetfillcolor{currentfill}%
\pgfsetfillopacity{0.700000}%
\pgfsetlinewidth{0.000000pt}%
\definecolor{currentstroke}{rgb}{0.000000,0.000000,0.000000}%
\pgfsetstrokecolor{currentstroke}%
\pgfsetdash{}{0pt}%
\pgfpathmoveto{\pgfqpoint{4.376979in}{2.062037in}}%
\pgfpathlineto{\pgfqpoint{4.390695in}{2.069746in}}%
\pgfpathlineto{\pgfqpoint{4.404425in}{2.077616in}}%
\pgfpathlineto{\pgfqpoint{4.418169in}{2.085646in}}%
\pgfpathlineto{\pgfqpoint{4.431927in}{2.093836in}}%
\pgfpathlineto{\pgfqpoint{4.439675in}{2.107021in}}%
\pgfpathlineto{\pgfqpoint{4.447418in}{2.120115in}}%
\pgfpathlineto{\pgfqpoint{4.455156in}{2.133118in}}%
\pgfpathlineto{\pgfqpoint{4.462889in}{2.146027in}}%
\pgfpathlineto{\pgfqpoint{4.449130in}{2.137613in}}%
\pgfpathlineto{\pgfqpoint{4.435386in}{2.129359in}}%
\pgfpathlineto{\pgfqpoint{4.421655in}{2.121266in}}%
\pgfpathlineto{\pgfqpoint{4.407939in}{2.113333in}}%
\pgfpathlineto{\pgfqpoint{4.400206in}{2.100637in}}%
\pgfpathlineto{\pgfqpoint{4.392468in}{2.087854in}}%
\pgfpathlineto{\pgfqpoint{4.384726in}{2.074987in}}%
\pgfpathlineto{\pgfqpoint{4.376979in}{2.062037in}}%
\pgfpathclose%
\pgfusepath{fill}%
\end{pgfscope}%
\begin{pgfscope}%
\pgfpathrectangle{\pgfqpoint{1.254980in}{0.150000in}}{\pgfqpoint{5.490039in}{5.490039in}}%
\pgfusepath{clip}%
\pgfsetbuttcap%
\pgfsetroundjoin%
\definecolor{currentfill}{rgb}{0.162142,0.474838,0.558140}%
\pgfsetfillcolor{currentfill}%
\pgfsetfillopacity{0.700000}%
\pgfsetlinewidth{0.000000pt}%
\definecolor{currentstroke}{rgb}{0.000000,0.000000,0.000000}%
\pgfsetstrokecolor{currentstroke}%
\pgfsetdash{}{0pt}%
\pgfpathmoveto{\pgfqpoint{4.696504in}{2.416556in}}%
\pgfpathlineto{\pgfqpoint{4.710400in}{2.427024in}}%
\pgfpathlineto{\pgfqpoint{4.724313in}{2.437654in}}%
\pgfpathlineto{\pgfqpoint{4.738242in}{2.448446in}}%
\pgfpathlineto{\pgfqpoint{4.752188in}{2.459399in}}%
\pgfpathlineto{\pgfqpoint{4.759826in}{2.470771in}}%
\pgfpathlineto{\pgfqpoint{4.767457in}{2.482007in}}%
\pgfpathlineto{\pgfqpoint{4.775082in}{2.493107in}}%
\pgfpathlineto{\pgfqpoint{4.782700in}{2.504069in}}%
\pgfpathlineto{\pgfqpoint{4.768754in}{2.493039in}}%
\pgfpathlineto{\pgfqpoint{4.754826in}{2.482170in}}%
\pgfpathlineto{\pgfqpoint{4.740914in}{2.471463in}}%
\pgfpathlineto{\pgfqpoint{4.727018in}{2.460917in}}%
\pgfpathlineto{\pgfqpoint{4.719399in}{2.450021in}}%
\pgfpathlineto{\pgfqpoint{4.711773in}{2.438995in}}%
\pgfpathlineto{\pgfqpoint{4.704142in}{2.427840in}}%
\pgfpathlineto{\pgfqpoint{4.696504in}{2.416556in}}%
\pgfpathclose%
\pgfusepath{fill}%
\end{pgfscope}%
\begin{pgfscope}%
\pgfpathrectangle{\pgfqpoint{1.254980in}{0.150000in}}{\pgfqpoint{5.490039in}{5.490039in}}%
\pgfusepath{clip}%
\pgfsetbuttcap%
\pgfsetroundjoin%
\definecolor{currentfill}{rgb}{0.168126,0.459988,0.558082}%
\pgfsetfillcolor{currentfill}%
\pgfsetfillopacity{0.700000}%
\pgfsetlinewidth{0.000000pt}%
\definecolor{currentstroke}{rgb}{0.000000,0.000000,0.000000}%
\pgfsetstrokecolor{currentstroke}%
\pgfsetdash{}{0pt}%
\pgfpathmoveto{\pgfqpoint{2.220420in}{2.481924in}}%
\pgfpathlineto{\pgfqpoint{2.234284in}{2.458856in}}%
\pgfpathlineto{\pgfqpoint{2.248134in}{2.436067in}}%
\pgfpathlineto{\pgfqpoint{2.261971in}{2.413554in}}%
\pgfpathlineto{\pgfqpoint{2.275795in}{2.391316in}}%
\pgfpathlineto{\pgfqpoint{2.284890in}{2.383041in}}%
\pgfpathlineto{\pgfqpoint{2.293959in}{2.375164in}}%
\pgfpathlineto{\pgfqpoint{2.303002in}{2.367678in}}%
\pgfpathlineto{\pgfqpoint{2.312020in}{2.360575in}}%
\pgfpathlineto{\pgfqpoint{2.298261in}{2.382114in}}%
\pgfpathlineto{\pgfqpoint{2.284489in}{2.403924in}}%
\pgfpathlineto{\pgfqpoint{2.270704in}{2.426010in}}%
\pgfpathlineto{\pgfqpoint{2.256907in}{2.448374in}}%
\pgfpathlineto{\pgfqpoint{2.247825in}{2.456165in}}%
\pgfpathlineto{\pgfqpoint{2.238717in}{2.464349in}}%
\pgfpathlineto{\pgfqpoint{2.229582in}{2.472933in}}%
\pgfpathlineto{\pgfqpoint{2.220420in}{2.481924in}}%
\pgfpathclose%
\pgfusepath{fill}%
\end{pgfscope}%
\begin{pgfscope}%
\pgfpathrectangle{\pgfqpoint{1.254980in}{0.150000in}}{\pgfqpoint{5.490039in}{5.490039in}}%
\pgfusepath{clip}%
\pgfsetbuttcap%
\pgfsetroundjoin%
\definecolor{currentfill}{rgb}{0.335885,0.777018,0.402049}%
\pgfsetfillcolor{currentfill}%
\pgfsetfillopacity{0.700000}%
\pgfsetlinewidth{0.000000pt}%
\definecolor{currentstroke}{rgb}{0.000000,0.000000,0.000000}%
\pgfsetstrokecolor{currentstroke}%
\pgfsetdash{}{0pt}%
\pgfpathmoveto{\pgfqpoint{5.592711in}{3.274082in}}%
\pgfpathlineto{\pgfqpoint{5.607190in}{3.289001in}}%
\pgfpathlineto{\pgfqpoint{5.621691in}{3.304083in}}%
\pgfpathlineto{\pgfqpoint{5.636213in}{3.319328in}}%
\pgfpathlineto{\pgfqpoint{5.650757in}{3.334737in}}%
\pgfpathlineto{\pgfqpoint{5.657861in}{3.337405in}}%
\pgfpathlineto{\pgfqpoint{5.664955in}{3.339951in}}%
\pgfpathlineto{\pgfqpoint{5.672038in}{3.342378in}}%
\pgfpathlineto{\pgfqpoint{5.679111in}{3.344691in}}%
\pgfpathlineto{\pgfqpoint{5.664587in}{3.329613in}}%
\pgfpathlineto{\pgfqpoint{5.650085in}{3.314697in}}%
\pgfpathlineto{\pgfqpoint{5.635604in}{3.299945in}}%
\pgfpathlineto{\pgfqpoint{5.621144in}{3.285355in}}%
\pgfpathlineto{\pgfqpoint{5.614051in}{3.282701in}}%
\pgfpathlineto{\pgfqpoint{5.606947in}{3.279940in}}%
\pgfpathlineto{\pgfqpoint{5.599834in}{3.277069in}}%
\pgfpathlineto{\pgfqpoint{5.592711in}{3.274082in}}%
\pgfpathclose%
\pgfusepath{fill}%
\end{pgfscope}%
\begin{pgfscope}%
\pgfpathrectangle{\pgfqpoint{1.254980in}{0.150000in}}{\pgfqpoint{5.490039in}{5.490039in}}%
\pgfusepath{clip}%
\pgfsetbuttcap%
\pgfsetroundjoin%
\definecolor{currentfill}{rgb}{0.280267,0.073417,0.397163}%
\pgfsetfillcolor{currentfill}%
\pgfsetfillopacity{0.700000}%
\pgfsetlinewidth{0.000000pt}%
\definecolor{currentstroke}{rgb}{0.000000,0.000000,0.000000}%
\pgfsetstrokecolor{currentstroke}%
\pgfsetdash{}{0pt}%
\pgfpathmoveto{\pgfqpoint{2.982770in}{1.562938in}}%
\pgfpathlineto{\pgfqpoint{2.996242in}{1.552559in}}%
\pgfpathlineto{\pgfqpoint{3.009714in}{1.542367in}}%
\pgfpathlineto{\pgfqpoint{3.023184in}{1.532362in}}%
\pgfpathlineto{\pgfqpoint{3.036654in}{1.522542in}}%
\pgfpathlineto{\pgfqpoint{3.045036in}{1.523777in}}%
\pgfpathlineto{\pgfqpoint{3.053404in}{1.525293in}}%
\pgfpathlineto{\pgfqpoint{3.061759in}{1.527083in}}%
\pgfpathlineto{\pgfqpoint{3.070099in}{1.529142in}}%
\pgfpathlineto{\pgfqpoint{3.056665in}{1.538316in}}%
\pgfpathlineto{\pgfqpoint{3.043232in}{1.547675in}}%
\pgfpathlineto{\pgfqpoint{3.029797in}{1.557220in}}%
\pgfpathlineto{\pgfqpoint{3.016362in}{1.566951in}}%
\pgfpathlineto{\pgfqpoint{3.007986in}{1.565528in}}%
\pgfpathlineto{\pgfqpoint{2.999596in}{1.564380in}}%
\pgfpathlineto{\pgfqpoint{2.991190in}{1.563514in}}%
\pgfpathlineto{\pgfqpoint{2.982770in}{1.562938in}}%
\pgfpathclose%
\pgfusepath{fill}%
\end{pgfscope}%
\begin{pgfscope}%
\pgfpathrectangle{\pgfqpoint{1.254980in}{0.150000in}}{\pgfqpoint{5.490039in}{5.490039in}}%
\pgfusepath{clip}%
\pgfsetbuttcap%
\pgfsetroundjoin%
\definecolor{currentfill}{rgb}{0.175707,0.697900,0.491033}%
\pgfsetfillcolor{currentfill}%
\pgfsetfillopacity{0.700000}%
\pgfsetlinewidth{0.000000pt}%
\definecolor{currentstroke}{rgb}{0.000000,0.000000,0.000000}%
\pgfsetstrokecolor{currentstroke}%
\pgfsetdash{}{0pt}%
\pgfpathmoveto{\pgfqpoint{5.304510in}{3.029712in}}%
\pgfpathlineto{\pgfqpoint{5.318799in}{3.043670in}}%
\pgfpathlineto{\pgfqpoint{5.333109in}{3.057792in}}%
\pgfpathlineto{\pgfqpoint{5.347438in}{3.072077in}}%
\pgfpathlineto{\pgfqpoint{5.361788in}{3.086525in}}%
\pgfpathlineto{\pgfqpoint{5.369101in}{3.092168in}}%
\pgfpathlineto{\pgfqpoint{5.376404in}{3.097662in}}%
\pgfpathlineto{\pgfqpoint{5.383697in}{3.103012in}}%
\pgfpathlineto{\pgfqpoint{5.390980in}{3.108219in}}%
\pgfpathlineto{\pgfqpoint{5.376642in}{3.093973in}}%
\pgfpathlineto{\pgfqpoint{5.362324in}{3.079890in}}%
\pgfpathlineto{\pgfqpoint{5.348027in}{3.065970in}}%
\pgfpathlineto{\pgfqpoint{5.333749in}{3.052213in}}%
\pgfpathlineto{\pgfqpoint{5.326453in}{3.046793in}}%
\pgfpathlineto{\pgfqpoint{5.319148in}{3.041238in}}%
\pgfpathlineto{\pgfqpoint{5.311834in}{3.035545in}}%
\pgfpathlineto{\pgfqpoint{5.304510in}{3.029712in}}%
\pgfpathclose%
\pgfusepath{fill}%
\end{pgfscope}%
\begin{pgfscope}%
\pgfpathrectangle{\pgfqpoint{1.254980in}{0.150000in}}{\pgfqpoint{5.490039in}{5.490039in}}%
\pgfusepath{clip}%
\pgfsetbuttcap%
\pgfsetroundjoin%
\definecolor{currentfill}{rgb}{0.277134,0.185228,0.489898}%
\pgfsetfillcolor{currentfill}%
\pgfsetfillopacity{0.700000}%
\pgfsetlinewidth{0.000000pt}%
\definecolor{currentstroke}{rgb}{0.000000,0.000000,0.000000}%
\pgfsetstrokecolor{currentstroke}%
\pgfsetdash{}{0pt}%
\pgfpathmoveto{\pgfqpoint{4.057444in}{1.727738in}}%
\pgfpathlineto{\pgfqpoint{4.071018in}{1.731987in}}%
\pgfpathlineto{\pgfqpoint{4.084603in}{1.736395in}}%
\pgfpathlineto{\pgfqpoint{4.098199in}{1.740963in}}%
\pgfpathlineto{\pgfqpoint{4.111806in}{1.745691in}}%
\pgfpathlineto{\pgfqpoint{4.119644in}{1.759030in}}%
\pgfpathlineto{\pgfqpoint{4.127477in}{1.772355in}}%
\pgfpathlineto{\pgfqpoint{4.135306in}{1.785660in}}%
\pgfpathlineto{\pgfqpoint{4.143131in}{1.798945in}}%
\pgfpathlineto{\pgfqpoint{4.129526in}{1.793853in}}%
\pgfpathlineto{\pgfqpoint{4.115932in}{1.788921in}}%
\pgfpathlineto{\pgfqpoint{4.102350in}{1.784149in}}%
\pgfpathlineto{\pgfqpoint{4.088778in}{1.779537in}}%
\pgfpathlineto{\pgfqpoint{4.080951in}{1.766606in}}%
\pgfpathlineto{\pgfqpoint{4.073120in}{1.753660in}}%
\pgfpathlineto{\pgfqpoint{4.065284in}{1.740703in}}%
\pgfpathlineto{\pgfqpoint{4.057444in}{1.727738in}}%
\pgfpathclose%
\pgfusepath{fill}%
\end{pgfscope}%
\begin{pgfscope}%
\pgfpathrectangle{\pgfqpoint{1.254980in}{0.150000in}}{\pgfqpoint{5.490039in}{5.490039in}}%
\pgfusepath{clip}%
\pgfsetbuttcap%
\pgfsetroundjoin%
\definecolor{currentfill}{rgb}{0.131172,0.555899,0.552459}%
\pgfsetfillcolor{currentfill}%
\pgfsetfillopacity{0.700000}%
\pgfsetlinewidth{0.000000pt}%
\definecolor{currentstroke}{rgb}{0.000000,0.000000,0.000000}%
\pgfsetstrokecolor{currentstroke}%
\pgfsetdash{}{0pt}%
\pgfpathmoveto{\pgfqpoint{4.899333in}{2.632818in}}%
\pgfpathlineto{\pgfqpoint{4.913360in}{2.644722in}}%
\pgfpathlineto{\pgfqpoint{4.927405in}{2.656789in}}%
\pgfpathlineto{\pgfqpoint{4.941467in}{2.669018in}}%
\pgfpathlineto{\pgfqpoint{4.955548in}{2.681410in}}%
\pgfpathlineto{\pgfqpoint{4.963098in}{2.691132in}}%
\pgfpathlineto{\pgfqpoint{4.970640in}{2.700703in}}%
\pgfpathlineto{\pgfqpoint{4.978174in}{2.710124in}}%
\pgfpathlineto{\pgfqpoint{4.985701in}{2.719394in}}%
\pgfpathlineto{\pgfqpoint{4.971624in}{2.707016in}}%
\pgfpathlineto{\pgfqpoint{4.957565in}{2.694801in}}%
\pgfpathlineto{\pgfqpoint{4.943523in}{2.682749in}}%
\pgfpathlineto{\pgfqpoint{4.929500in}{2.670858in}}%
\pgfpathlineto{\pgfqpoint{4.921969in}{2.661562in}}%
\pgfpathlineto{\pgfqpoint{4.914431in}{2.652124in}}%
\pgfpathlineto{\pgfqpoint{4.906886in}{2.642543in}}%
\pgfpathlineto{\pgfqpoint{4.899333in}{2.632818in}}%
\pgfpathclose%
\pgfusepath{fill}%
\end{pgfscope}%
\begin{pgfscope}%
\pgfpathrectangle{\pgfqpoint{1.254980in}{0.150000in}}{\pgfqpoint{5.490039in}{5.490039in}}%
\pgfusepath{clip}%
\pgfsetbuttcap%
\pgfsetroundjoin%
\definecolor{currentfill}{rgb}{0.246811,0.283237,0.535941}%
\pgfsetfillcolor{currentfill}%
\pgfsetfillopacity{0.700000}%
\pgfsetlinewidth{0.000000pt}%
\definecolor{currentstroke}{rgb}{0.000000,0.000000,0.000000}%
\pgfsetstrokecolor{currentstroke}%
\pgfsetdash{}{0pt}%
\pgfpathmoveto{\pgfqpoint{4.260101in}{1.928608in}}%
\pgfpathlineto{\pgfqpoint{4.273764in}{1.935144in}}%
\pgfpathlineto{\pgfqpoint{4.287441in}{1.941839in}}%
\pgfpathlineto{\pgfqpoint{4.301130in}{1.948695in}}%
\pgfpathlineto{\pgfqpoint{4.314832in}{1.955711in}}%
\pgfpathlineto{\pgfqpoint{4.322616in}{1.969246in}}%
\pgfpathlineto{\pgfqpoint{4.330396in}{1.982715in}}%
\pgfpathlineto{\pgfqpoint{4.338171in}{1.996118in}}%
\pgfpathlineto{\pgfqpoint{4.345942in}{2.009450in}}%
\pgfpathlineto{\pgfqpoint{4.332240in}{2.002153in}}%
\pgfpathlineto{\pgfqpoint{4.318550in}{1.995017in}}%
\pgfpathlineto{\pgfqpoint{4.304874in}{1.988040in}}%
\pgfpathlineto{\pgfqpoint{4.291211in}{1.981224in}}%
\pgfpathlineto{\pgfqpoint{4.283440in}{1.968161in}}%
\pgfpathlineto{\pgfqpoint{4.275665in}{1.955036in}}%
\pgfpathlineto{\pgfqpoint{4.267885in}{1.941851in}}%
\pgfpathlineto{\pgfqpoint{4.260101in}{1.928608in}}%
\pgfpathclose%
\pgfusepath{fill}%
\end{pgfscope}%
\begin{pgfscope}%
\pgfpathrectangle{\pgfqpoint{1.254980in}{0.150000in}}{\pgfqpoint{5.490039in}{5.490039in}}%
\pgfusepath{clip}%
\pgfsetbuttcap%
\pgfsetroundjoin%
\definecolor{currentfill}{rgb}{0.122312,0.633153,0.530398}%
\pgfsetfillcolor{currentfill}%
\pgfsetfillopacity{0.700000}%
\pgfsetlinewidth{0.000000pt}%
\definecolor{currentstroke}{rgb}{0.000000,0.000000,0.000000}%
\pgfsetstrokecolor{currentstroke}%
\pgfsetdash{}{0pt}%
\pgfpathmoveto{\pgfqpoint{5.102082in}{2.838903in}}%
\pgfpathlineto{\pgfqpoint{5.116241in}{2.851973in}}%
\pgfpathlineto{\pgfqpoint{5.130420in}{2.865206in}}%
\pgfpathlineto{\pgfqpoint{5.144618in}{2.878602in}}%
\pgfpathlineto{\pgfqpoint{5.158834in}{2.892161in}}%
\pgfpathlineto{\pgfqpoint{5.166276in}{2.899930in}}%
\pgfpathlineto{\pgfqpoint{5.173709in}{2.907543in}}%
\pgfpathlineto{\pgfqpoint{5.181134in}{2.915003in}}%
\pgfpathlineto{\pgfqpoint{5.188549in}{2.922310in}}%
\pgfpathlineto{\pgfqpoint{5.174339in}{2.908859in}}%
\pgfpathlineto{\pgfqpoint{5.160148in}{2.895570in}}%
\pgfpathlineto{\pgfqpoint{5.145977in}{2.882444in}}%
\pgfpathlineto{\pgfqpoint{5.131824in}{2.869481in}}%
\pgfpathlineto{\pgfqpoint{5.124402in}{2.862055in}}%
\pgfpathlineto{\pgfqpoint{5.116970in}{2.854484in}}%
\pgfpathlineto{\pgfqpoint{5.109530in}{2.846767in}}%
\pgfpathlineto{\pgfqpoint{5.102082in}{2.838903in}}%
\pgfpathclose%
\pgfusepath{fill}%
\end{pgfscope}%
\begin{pgfscope}%
\pgfpathrectangle{\pgfqpoint{1.254980in}{0.150000in}}{\pgfqpoint{5.490039in}{5.490039in}}%
\pgfusepath{clip}%
\pgfsetbuttcap%
\pgfsetroundjoin%
\definecolor{currentfill}{rgb}{0.180629,0.429975,0.557282}%
\pgfsetfillcolor{currentfill}%
\pgfsetfillopacity{0.700000}%
\pgfsetlinewidth{0.000000pt}%
\definecolor{currentstroke}{rgb}{0.000000,0.000000,0.000000}%
\pgfsetstrokecolor{currentstroke}%
\pgfsetdash{}{0pt}%
\pgfpathmoveto{\pgfqpoint{4.579751in}{2.282437in}}%
\pgfpathlineto{\pgfqpoint{4.593585in}{2.292017in}}%
\pgfpathlineto{\pgfqpoint{4.607434in}{2.301757in}}%
\pgfpathlineto{\pgfqpoint{4.621299in}{2.311659in}}%
\pgfpathlineto{\pgfqpoint{4.635179in}{2.321722in}}%
\pgfpathlineto{\pgfqpoint{4.642866in}{2.334013in}}%
\pgfpathlineto{\pgfqpoint{4.650546in}{2.346180in}}%
\pgfpathlineto{\pgfqpoint{4.658221in}{2.358224in}}%
\pgfpathlineto{\pgfqpoint{4.665889in}{2.370143in}}%
\pgfpathlineto{\pgfqpoint{4.652009in}{2.359943in}}%
\pgfpathlineto{\pgfqpoint{4.638144in}{2.349904in}}%
\pgfpathlineto{\pgfqpoint{4.624294in}{2.340027in}}%
\pgfpathlineto{\pgfqpoint{4.610460in}{2.330311in}}%
\pgfpathlineto{\pgfqpoint{4.602792in}{2.318517in}}%
\pgfpathlineto{\pgfqpoint{4.595117in}{2.306607in}}%
\pgfpathlineto{\pgfqpoint{4.587437in}{2.294580in}}%
\pgfpathlineto{\pgfqpoint{4.579751in}{2.282437in}}%
\pgfpathclose%
\pgfusepath{fill}%
\end{pgfscope}%
\begin{pgfscope}%
\pgfpathrectangle{\pgfqpoint{1.254980in}{0.150000in}}{\pgfqpoint{5.490039in}{5.490039in}}%
\pgfusepath{clip}%
\pgfsetbuttcap%
\pgfsetroundjoin%
\definecolor{currentfill}{rgb}{0.276022,0.044167,0.370164}%
\pgfsetfillcolor{currentfill}%
\pgfsetfillopacity{0.700000}%
\pgfsetlinewidth{0.000000pt}%
\definecolor{currentstroke}{rgb}{0.000000,0.000000,0.000000}%
\pgfsetstrokecolor{currentstroke}%
\pgfsetdash{}{0pt}%
\pgfpathmoveto{\pgfqpoint{3.683277in}{1.462011in}}%
\pgfpathlineto{\pgfqpoint{3.696745in}{1.461462in}}%
\pgfpathlineto{\pgfqpoint{3.710220in}{1.461076in}}%
\pgfpathlineto{\pgfqpoint{3.723702in}{1.460852in}}%
\pgfpathlineto{\pgfqpoint{3.737192in}{1.460789in}}%
\pgfpathlineto{\pgfqpoint{3.745151in}{1.471567in}}%
\pgfpathlineto{\pgfqpoint{3.753105in}{1.482440in}}%
\pgfpathlineto{\pgfqpoint{3.761054in}{1.493404in}}%
\pgfpathlineto{\pgfqpoint{3.768996in}{1.504453in}}%
\pgfpathlineto{\pgfqpoint{3.755517in}{1.504017in}}%
\pgfpathlineto{\pgfqpoint{3.742046in}{1.503742in}}%
\pgfpathlineto{\pgfqpoint{3.728582in}{1.503630in}}%
\pgfpathlineto{\pgfqpoint{3.715126in}{1.503680in}}%
\pgfpathlineto{\pgfqpoint{3.707173in}{1.493119in}}%
\pgfpathlineto{\pgfqpoint{3.699214in}{1.482651in}}%
\pgfpathlineto{\pgfqpoint{3.691249in}{1.472280in}}%
\pgfpathlineto{\pgfqpoint{3.683277in}{1.462011in}}%
\pgfpathclose%
\pgfusepath{fill}%
\end{pgfscope}%
\begin{pgfscope}%
\pgfpathrectangle{\pgfqpoint{1.254980in}{0.150000in}}{\pgfqpoint{5.490039in}{5.490039in}}%
\pgfusepath{clip}%
\pgfsetbuttcap%
\pgfsetroundjoin%
\definecolor{currentfill}{rgb}{0.395174,0.797475,0.367757}%
\pgfsetfillcolor{currentfill}%
\pgfsetfillopacity{0.700000}%
\pgfsetlinewidth{0.000000pt}%
\definecolor{currentstroke}{rgb}{0.000000,0.000000,0.000000}%
\pgfsetstrokecolor{currentstroke}%
\pgfsetdash{}{0pt}%
\pgfpathmoveto{\pgfqpoint{5.679111in}{3.344691in}}%
\pgfpathlineto{\pgfqpoint{5.693657in}{3.359933in}}%
\pgfpathlineto{\pgfqpoint{5.708225in}{3.375339in}}%
\pgfpathlineto{\pgfqpoint{5.722815in}{3.390908in}}%
\pgfpathlineto{\pgfqpoint{5.737427in}{3.406641in}}%
\pgfpathlineto{\pgfqpoint{5.744468in}{3.408492in}}%
\pgfpathlineto{\pgfqpoint{5.751499in}{3.410229in}}%
\pgfpathlineto{\pgfqpoint{5.758519in}{3.411856in}}%
\pgfpathlineto{\pgfqpoint{5.765529in}{3.413376in}}%
\pgfpathlineto{\pgfqpoint{5.750939in}{3.398007in}}%
\pgfpathlineto{\pgfqpoint{5.736372in}{3.382800in}}%
\pgfpathlineto{\pgfqpoint{5.721826in}{3.367757in}}%
\pgfpathlineto{\pgfqpoint{5.707302in}{3.352876in}}%
\pgfpathlineto{\pgfqpoint{5.700270in}{3.350982in}}%
\pgfpathlineto{\pgfqpoint{5.693227in}{3.348989in}}%
\pgfpathlineto{\pgfqpoint{5.686174in}{3.346894in}}%
\pgfpathlineto{\pgfqpoint{5.679111in}{3.344691in}}%
\pgfpathclose%
\pgfusepath{fill}%
\end{pgfscope}%
\begin{pgfscope}%
\pgfpathrectangle{\pgfqpoint{1.254980in}{0.150000in}}{\pgfqpoint{5.490039in}{5.490039in}}%
\pgfusepath{clip}%
\pgfsetbuttcap%
\pgfsetroundjoin%
\definecolor{currentfill}{rgb}{0.279566,0.067836,0.391917}%
\pgfsetfillcolor{currentfill}%
\pgfsetfillopacity{0.700000}%
\pgfsetlinewidth{0.000000pt}%
\definecolor{currentstroke}{rgb}{0.000000,0.000000,0.000000}%
\pgfsetstrokecolor{currentstroke}%
\pgfsetdash{}{0pt}%
\pgfpathmoveto{\pgfqpoint{3.768996in}{1.504453in}}%
\pgfpathlineto{\pgfqpoint{3.782482in}{1.505051in}}%
\pgfpathlineto{\pgfqpoint{3.795977in}{1.505810in}}%
\pgfpathlineto{\pgfqpoint{3.809479in}{1.506729in}}%
\pgfpathlineto{\pgfqpoint{3.822990in}{1.507810in}}%
\pgfpathlineto{\pgfqpoint{3.830917in}{1.519422in}}%
\pgfpathlineto{\pgfqpoint{3.838840in}{1.531103in}}%
\pgfpathlineto{\pgfqpoint{3.846757in}{1.542849in}}%
\pgfpathlineto{\pgfqpoint{3.854669in}{1.554656in}}%
\pgfpathlineto{\pgfqpoint{3.841166in}{1.553103in}}%
\pgfpathlineto{\pgfqpoint{3.827673in}{1.551711in}}%
\pgfpathlineto{\pgfqpoint{3.814187in}{1.550480in}}%
\pgfpathlineto{\pgfqpoint{3.800710in}{1.549411in}}%
\pgfpathlineto{\pgfqpoint{3.792790in}{1.538066in}}%
\pgfpathlineto{\pgfqpoint{3.784864in}{1.526789in}}%
\pgfpathlineto{\pgfqpoint{3.776933in}{1.515583in}}%
\pgfpathlineto{\pgfqpoint{3.768996in}{1.504453in}}%
\pgfpathclose%
\pgfusepath{fill}%
\end{pgfscope}%
\begin{pgfscope}%
\pgfpathrectangle{\pgfqpoint{1.254980in}{0.150000in}}{\pgfqpoint{5.490039in}{5.490039in}}%
\pgfusepath{clip}%
\pgfsetbuttcap%
\pgfsetroundjoin%
\definecolor{currentfill}{rgb}{0.267004,0.004874,0.329415}%
\pgfsetfillcolor{currentfill}%
\pgfsetfillopacity{0.700000}%
\pgfsetlinewidth{0.000000pt}%
\definecolor{currentstroke}{rgb}{0.000000,0.000000,0.000000}%
\pgfsetstrokecolor{currentstroke}%
\pgfsetdash{}{0pt}%
\pgfpathmoveto{\pgfqpoint{3.371581in}{1.406307in}}%
\pgfpathlineto{\pgfqpoint{3.385019in}{1.401449in}}%
\pgfpathlineto{\pgfqpoint{3.398461in}{1.396760in}}%
\pgfpathlineto{\pgfqpoint{3.411906in}{1.392240in}}%
\pgfpathlineto{\pgfqpoint{3.425355in}{1.387888in}}%
\pgfpathlineto{\pgfqpoint{3.433468in}{1.394754in}}%
\pgfpathlineto{\pgfqpoint{3.441573in}{1.401806in}}%
\pgfpathlineto{\pgfqpoint{3.449668in}{1.409040in}}%
\pgfpathlineto{\pgfqpoint{3.457755in}{1.416449in}}%
\pgfpathlineto{\pgfqpoint{3.444327in}{1.420220in}}%
\pgfpathlineto{\pgfqpoint{3.430904in}{1.424158in}}%
\pgfpathlineto{\pgfqpoint{3.417485in}{1.428266in}}%
\pgfpathlineto{\pgfqpoint{3.404069in}{1.432542in}}%
\pgfpathlineto{\pgfqpoint{3.395961in}{1.425703in}}%
\pgfpathlineto{\pgfqpoint{3.387844in}{1.419047in}}%
\pgfpathlineto{\pgfqpoint{3.379717in}{1.412580in}}%
\pgfpathlineto{\pgfqpoint{3.371581in}{1.406307in}}%
\pgfpathclose%
\pgfusepath{fill}%
\end{pgfscope}%
\begin{pgfscope}%
\pgfpathrectangle{\pgfqpoint{1.254980in}{0.150000in}}{\pgfqpoint{5.490039in}{5.490039in}}%
\pgfusepath{clip}%
\pgfsetbuttcap%
\pgfsetroundjoin%
\definecolor{currentfill}{rgb}{0.277941,0.056324,0.381191}%
\pgfsetfillcolor{currentfill}%
\pgfsetfillopacity{0.700000}%
\pgfsetlinewidth{0.000000pt}%
\definecolor{currentstroke}{rgb}{0.000000,0.000000,0.000000}%
\pgfsetstrokecolor{currentstroke}%
\pgfsetdash{}{0pt}%
\pgfpathmoveto{\pgfqpoint{3.036654in}{1.522542in}}%
\pgfpathlineto{\pgfqpoint{3.050123in}{1.512907in}}%
\pgfpathlineto{\pgfqpoint{3.063591in}{1.503456in}}%
\pgfpathlineto{\pgfqpoint{3.077059in}{1.494187in}}%
\pgfpathlineto{\pgfqpoint{3.090527in}{1.485101in}}%
\pgfpathlineto{\pgfqpoint{3.098874in}{1.486992in}}%
\pgfpathlineto{\pgfqpoint{3.107206in}{1.489156in}}%
\pgfpathlineto{\pgfqpoint{3.115526in}{1.491587in}}%
\pgfpathlineto{\pgfqpoint{3.123831in}{1.494279in}}%
\pgfpathlineto{\pgfqpoint{3.110398in}{1.502721in}}%
\pgfpathlineto{\pgfqpoint{3.096965in}{1.511345in}}%
\pgfpathlineto{\pgfqpoint{3.083532in}{1.520152in}}%
\pgfpathlineto{\pgfqpoint{3.070099in}{1.529142in}}%
\pgfpathlineto{\pgfqpoint{3.061759in}{1.527083in}}%
\pgfpathlineto{\pgfqpoint{3.053404in}{1.525293in}}%
\pgfpathlineto{\pgfqpoint{3.045036in}{1.523777in}}%
\pgfpathlineto{\pgfqpoint{3.036654in}{1.522542in}}%
\pgfpathclose%
\pgfusepath{fill}%
\end{pgfscope}%
\begin{pgfscope}%
\pgfpathrectangle{\pgfqpoint{1.254980in}{0.150000in}}{\pgfqpoint{5.490039in}{5.490039in}}%
\pgfusepath{clip}%
\pgfsetbuttcap%
\pgfsetroundjoin%
\definecolor{currentfill}{rgb}{0.153364,0.497000,0.557724}%
\pgfsetfillcolor{currentfill}%
\pgfsetfillopacity{0.700000}%
\pgfsetlinewidth{0.000000pt}%
\definecolor{currentstroke}{rgb}{0.000000,0.000000,0.000000}%
\pgfsetstrokecolor{currentstroke}%
\pgfsetdash{}{0pt}%
\pgfpathmoveto{\pgfqpoint{2.164819in}{2.577044in}}%
\pgfpathlineto{\pgfqpoint{2.178741in}{2.552832in}}%
\pgfpathlineto{\pgfqpoint{2.192649in}{2.528909in}}%
\pgfpathlineto{\pgfqpoint{2.206541in}{2.505274in}}%
\pgfpathlineto{\pgfqpoint{2.220420in}{2.481924in}}%
\pgfpathlineto{\pgfqpoint{2.229582in}{2.472933in}}%
\pgfpathlineto{\pgfqpoint{2.238717in}{2.464349in}}%
\pgfpathlineto{\pgfqpoint{2.247825in}{2.456165in}}%
\pgfpathlineto{\pgfqpoint{2.256907in}{2.448374in}}%
\pgfpathlineto{\pgfqpoint{2.243095in}{2.471017in}}%
\pgfpathlineto{\pgfqpoint{2.229270in}{2.493944in}}%
\pgfpathlineto{\pgfqpoint{2.215431in}{2.517156in}}%
\pgfpathlineto{\pgfqpoint{2.201577in}{2.540657in}}%
\pgfpathlineto{\pgfqpoint{2.192429in}{2.549143in}}%
\pgfpathlineto{\pgfqpoint{2.183253in}{2.558032in}}%
\pgfpathlineto{\pgfqpoint{2.174050in}{2.567330in}}%
\pgfpathlineto{\pgfqpoint{2.164819in}{2.577044in}}%
\pgfpathclose%
\pgfusepath{fill}%
\end{pgfscope}%
\begin{pgfscope}%
\pgfpathrectangle{\pgfqpoint{1.254980in}{0.150000in}}{\pgfqpoint{5.490039in}{5.490039in}}%
\pgfusepath{clip}%
\pgfsetbuttcap%
\pgfsetroundjoin%
\definecolor{currentfill}{rgb}{0.269944,0.014625,0.341379}%
\pgfsetfillcolor{currentfill}%
\pgfsetfillopacity{0.700000}%
\pgfsetlinewidth{0.000000pt}%
\definecolor{currentstroke}{rgb}{0.000000,0.000000,0.000000}%
\pgfsetstrokecolor{currentstroke}%
\pgfsetdash{}{0pt}%
\pgfpathmoveto{\pgfqpoint{3.231329in}{1.433180in}}%
\pgfpathlineto{\pgfqpoint{3.244772in}{1.426337in}}%
\pgfpathlineto{\pgfqpoint{3.258218in}{1.419668in}}%
\pgfpathlineto{\pgfqpoint{3.271665in}{1.413173in}}%
\pgfpathlineto{\pgfqpoint{3.285114in}{1.406850in}}%
\pgfpathlineto{\pgfqpoint{3.293317in}{1.411665in}}%
\pgfpathlineto{\pgfqpoint{3.301509in}{1.416706in}}%
\pgfpathlineto{\pgfqpoint{3.309691in}{1.421966in}}%
\pgfpathlineto{\pgfqpoint{3.317861in}{1.427440in}}%
\pgfpathlineto{\pgfqpoint{3.304439in}{1.433152in}}%
\pgfpathlineto{\pgfqpoint{3.291019in}{1.439036in}}%
\pgfpathlineto{\pgfqpoint{3.277601in}{1.445094in}}%
\pgfpathlineto{\pgfqpoint{3.264186in}{1.451325in}}%
\pgfpathlineto{\pgfqpoint{3.255989in}{1.446452in}}%
\pgfpathlineto{\pgfqpoint{3.247780in}{1.441800in}}%
\pgfpathlineto{\pgfqpoint{3.239560in}{1.437374in}}%
\pgfpathlineto{\pgfqpoint{3.231329in}{1.433180in}}%
\pgfpathclose%
\pgfusepath{fill}%
\end{pgfscope}%
\begin{pgfscope}%
\pgfpathrectangle{\pgfqpoint{1.254980in}{0.150000in}}{\pgfqpoint{5.490039in}{5.490039in}}%
\pgfusepath{clip}%
\pgfsetbuttcap%
\pgfsetroundjoin%
\definecolor{currentfill}{rgb}{0.458674,0.816363,0.329727}%
\pgfsetfillcolor{currentfill}%
\pgfsetfillopacity{0.700000}%
\pgfsetlinewidth{0.000000pt}%
\definecolor{currentstroke}{rgb}{0.000000,0.000000,0.000000}%
\pgfsetstrokecolor{currentstroke}%
\pgfsetdash{}{0pt}%
\pgfpathmoveto{\pgfqpoint{5.765529in}{3.413376in}}%
\pgfpathlineto{\pgfqpoint{5.780141in}{3.428909in}}%
\pgfpathlineto{\pgfqpoint{5.794775in}{3.444606in}}%
\pgfpathlineto{\pgfqpoint{5.809432in}{3.460467in}}%
\pgfpathlineto{\pgfqpoint{5.816414in}{3.461598in}}%
\pgfpathlineto{\pgfqpoint{5.823385in}{3.462625in}}%
\pgfpathlineto{\pgfqpoint{5.830345in}{3.463554in}}%
\pgfpathlineto{\pgfqpoint{5.837296in}{3.464388in}}%
\pgfpathlineto{\pgfqpoint{5.822664in}{3.448923in}}%
\pgfpathlineto{\pgfqpoint{5.808055in}{3.433621in}}%
\pgfpathlineto{\pgfqpoint{5.793467in}{3.418482in}}%
\pgfpathlineto{\pgfqpoint{5.786498in}{3.417343in}}%
\pgfpathlineto{\pgfqpoint{5.779518in}{3.416116in}}%
\pgfpathlineto{\pgfqpoint{5.772529in}{3.414795in}}%
\pgfpathlineto{\pgfqpoint{5.765529in}{3.413376in}}%
\pgfpathclose%
\pgfusepath{fill}%
\end{pgfscope}%
\begin{pgfscope}%
\pgfpathrectangle{\pgfqpoint{1.254980in}{0.150000in}}{\pgfqpoint{5.490039in}{5.490039in}}%
\pgfusepath{clip}%
\pgfsetbuttcap%
\pgfsetroundjoin%
\definecolor{currentfill}{rgb}{0.271305,0.019942,0.347269}%
\pgfsetfillcolor{currentfill}%
\pgfsetfillopacity{0.700000}%
\pgfsetlinewidth{0.000000pt}%
\definecolor{currentstroke}{rgb}{0.000000,0.000000,0.000000}%
\pgfsetstrokecolor{currentstroke}%
\pgfsetdash{}{0pt}%
\pgfpathmoveto{\pgfqpoint{3.597466in}{1.427980in}}%
\pgfpathlineto{\pgfqpoint{3.610922in}{1.426253in}}%
\pgfpathlineto{\pgfqpoint{3.624384in}{1.424690in}}%
\pgfpathlineto{\pgfqpoint{3.637852in}{1.423290in}}%
\pgfpathlineto{\pgfqpoint{3.651327in}{1.422052in}}%
\pgfpathlineto{\pgfqpoint{3.659324in}{1.431864in}}%
\pgfpathlineto{\pgfqpoint{3.667315in}{1.441798in}}%
\pgfpathlineto{\pgfqpoint{3.675299in}{1.451849in}}%
\pgfpathlineto{\pgfqpoint{3.683277in}{1.462011in}}%
\pgfpathlineto{\pgfqpoint{3.669816in}{1.462722in}}%
\pgfpathlineto{\pgfqpoint{3.656361in}{1.463596in}}%
\pgfpathlineto{\pgfqpoint{3.642913in}{1.464634in}}%
\pgfpathlineto{\pgfqpoint{3.629472in}{1.465835in}}%
\pgfpathlineto{\pgfqpoint{3.621481in}{1.456188in}}%
\pgfpathlineto{\pgfqpoint{3.613483in}{1.446659in}}%
\pgfpathlineto{\pgfqpoint{3.605478in}{1.437255in}}%
\pgfpathlineto{\pgfqpoint{3.597466in}{1.427980in}}%
\pgfpathclose%
\pgfusepath{fill}%
\end{pgfscope}%
\begin{pgfscope}%
\pgfpathrectangle{\pgfqpoint{1.254980in}{0.150000in}}{\pgfqpoint{5.490039in}{5.490039in}}%
\pgfusepath{clip}%
\pgfsetbuttcap%
\pgfsetroundjoin%
\definecolor{currentfill}{rgb}{0.282656,0.100196,0.422160}%
\pgfsetfillcolor{currentfill}%
\pgfsetfillopacity{0.700000}%
\pgfsetlinewidth{0.000000pt}%
\definecolor{currentstroke}{rgb}{0.000000,0.000000,0.000000}%
\pgfsetstrokecolor{currentstroke}%
\pgfsetdash{}{0pt}%
\pgfpathmoveto{\pgfqpoint{3.854669in}{1.554656in}}%
\pgfpathlineto{\pgfqpoint{3.868180in}{1.556369in}}%
\pgfpathlineto{\pgfqpoint{3.881700in}{1.558243in}}%
\pgfpathlineto{\pgfqpoint{3.895228in}{1.560277in}}%
\pgfpathlineto{\pgfqpoint{3.908766in}{1.562471in}}%
\pgfpathlineto{\pgfqpoint{3.916666in}{1.574789in}}%
\pgfpathlineto{\pgfqpoint{3.924561in}{1.587152in}}%
\pgfpathlineto{\pgfqpoint{3.932452in}{1.599555in}}%
\pgfpathlineto{\pgfqpoint{3.940337in}{1.611994in}}%
\pgfpathlineto{\pgfqpoint{3.926805in}{1.609354in}}%
\pgfpathlineto{\pgfqpoint{3.913283in}{1.606875in}}%
\pgfpathlineto{\pgfqpoint{3.899770in}{1.604556in}}%
\pgfpathlineto{\pgfqpoint{3.886266in}{1.602398in}}%
\pgfpathlineto{\pgfqpoint{3.878374in}{1.590393in}}%
\pgfpathlineto{\pgfqpoint{3.870477in}{1.578432in}}%
\pgfpathlineto{\pgfqpoint{3.862576in}{1.566518in}}%
\pgfpathlineto{\pgfqpoint{3.854669in}{1.554656in}}%
\pgfpathclose%
\pgfusepath{fill}%
\end{pgfscope}%
\begin{pgfscope}%
\pgfpathrectangle{\pgfqpoint{1.254980in}{0.150000in}}{\pgfqpoint{5.490039in}{5.490039in}}%
\pgfusepath{clip}%
\pgfsetbuttcap%
\pgfsetroundjoin%
\definecolor{currentfill}{rgb}{0.203063,0.379716,0.553925}%
\pgfsetfillcolor{currentfill}%
\pgfsetfillopacity{0.700000}%
\pgfsetlinewidth{0.000000pt}%
\definecolor{currentstroke}{rgb}{0.000000,0.000000,0.000000}%
\pgfsetstrokecolor{currentstroke}%
\pgfsetdash{}{0pt}%
\pgfpathmoveto{\pgfqpoint{4.462889in}{2.146027in}}%
\pgfpathlineto{\pgfqpoint{4.476662in}{2.154602in}}%
\pgfpathlineto{\pgfqpoint{4.490449in}{2.163338in}}%
\pgfpathlineto{\pgfqpoint{4.504251in}{2.172234in}}%
\pgfpathlineto{\pgfqpoint{4.518068in}{2.181291in}}%
\pgfpathlineto{\pgfqpoint{4.525797in}{2.194311in}}%
\pgfpathlineto{\pgfqpoint{4.533521in}{2.207226in}}%
\pgfpathlineto{\pgfqpoint{4.541240in}{2.220035in}}%
\pgfpathlineto{\pgfqpoint{4.548953in}{2.232737in}}%
\pgfpathlineto{\pgfqpoint{4.535135in}{2.223484in}}%
\pgfpathlineto{\pgfqpoint{4.521332in}{2.214392in}}%
\pgfpathlineto{\pgfqpoint{4.507544in}{2.205462in}}%
\pgfpathlineto{\pgfqpoint{4.493770in}{2.196692in}}%
\pgfpathlineto{\pgfqpoint{4.486057in}{2.184174in}}%
\pgfpathlineto{\pgfqpoint{4.478340in}{2.171557in}}%
\pgfpathlineto{\pgfqpoint{4.470617in}{2.158841in}}%
\pgfpathlineto{\pgfqpoint{4.462889in}{2.146027in}}%
\pgfpathclose%
\pgfusepath{fill}%
\end{pgfscope}%
\begin{pgfscope}%
\pgfpathrectangle{\pgfqpoint{1.254980in}{0.150000in}}{\pgfqpoint{5.490039in}{5.490039in}}%
\pgfusepath{clip}%
\pgfsetbuttcap%
\pgfsetroundjoin%
\definecolor{currentfill}{rgb}{0.226397,0.728888,0.462789}%
\pgfsetfillcolor{currentfill}%
\pgfsetfillopacity{0.700000}%
\pgfsetlinewidth{0.000000pt}%
\definecolor{currentstroke}{rgb}{0.000000,0.000000,0.000000}%
\pgfsetstrokecolor{currentstroke}%
\pgfsetdash{}{0pt}%
\pgfpathmoveto{\pgfqpoint{5.390980in}{3.108219in}}%
\pgfpathlineto{\pgfqpoint{5.405339in}{3.122628in}}%
\pgfpathlineto{\pgfqpoint{5.419718in}{3.137201in}}%
\pgfpathlineto{\pgfqpoint{5.434118in}{3.151938in}}%
\pgfpathlineto{\pgfqpoint{5.448539in}{3.166838in}}%
\pgfpathlineto{\pgfqpoint{5.455799in}{3.171682in}}%
\pgfpathlineto{\pgfqpoint{5.463049in}{3.176381in}}%
\pgfpathlineto{\pgfqpoint{5.470290in}{3.180938in}}%
\pgfpathlineto{\pgfqpoint{5.477520in}{3.185355in}}%
\pgfpathlineto{\pgfqpoint{5.463113in}{3.170690in}}%
\pgfpathlineto{\pgfqpoint{5.448727in}{3.156188in}}%
\pgfpathlineto{\pgfqpoint{5.434362in}{3.141849in}}%
\pgfpathlineto{\pgfqpoint{5.420017in}{3.127673in}}%
\pgfpathlineto{\pgfqpoint{5.412772in}{3.123010in}}%
\pgfpathlineto{\pgfqpoint{5.405518in}{3.118215in}}%
\pgfpathlineto{\pgfqpoint{5.398254in}{3.113286in}}%
\pgfpathlineto{\pgfqpoint{5.390980in}{3.108219in}}%
\pgfpathclose%
\pgfusepath{fill}%
\end{pgfscope}%
\begin{pgfscope}%
\pgfpathrectangle{\pgfqpoint{1.254980in}{0.150000in}}{\pgfqpoint{5.490039in}{5.490039in}}%
\pgfusepath{clip}%
\pgfsetbuttcap%
\pgfsetroundjoin%
\definecolor{currentfill}{rgb}{0.255645,0.260703,0.528312}%
\pgfsetfillcolor{currentfill}%
\pgfsetfillopacity{0.700000}%
\pgfsetlinewidth{0.000000pt}%
\definecolor{currentstroke}{rgb}{0.000000,0.000000,0.000000}%
\pgfsetstrokecolor{currentstroke}%
\pgfsetdash{}{0pt}%
\pgfpathmoveto{\pgfqpoint{2.569145in}{1.954034in}}%
\pgfpathlineto{\pgfqpoint{2.582780in}{1.937246in}}%
\pgfpathlineto{\pgfqpoint{2.596409in}{1.920682in}}%
\pgfpathlineto{\pgfqpoint{2.610030in}{1.904339in}}%
\pgfpathlineto{\pgfqpoint{2.623645in}{1.888216in}}%
\pgfpathlineto{\pgfqpoint{2.632417in}{1.883373in}}%
\pgfpathlineto{\pgfqpoint{2.641167in}{1.878895in}}%
\pgfpathlineto{\pgfqpoint{2.649897in}{1.874776in}}%
\pgfpathlineto{\pgfqpoint{2.658605in}{1.871007in}}%
\pgfpathlineto{\pgfqpoint{2.645044in}{1.886432in}}%
\pgfpathlineto{\pgfqpoint{2.631476in}{1.902076in}}%
\pgfpathlineto{\pgfqpoint{2.617901in}{1.917940in}}%
\pgfpathlineto{\pgfqpoint{2.604321in}{1.934026in}}%
\pgfpathlineto{\pgfqpoint{2.595559in}{1.938481in}}%
\pgfpathlineto{\pgfqpoint{2.586776in}{1.943296in}}%
\pgfpathlineto{\pgfqpoint{2.577972in}{1.948478in}}%
\pgfpathlineto{\pgfqpoint{2.569145in}{1.954034in}}%
\pgfpathclose%
\pgfusepath{fill}%
\end{pgfscope}%
\begin{pgfscope}%
\pgfpathrectangle{\pgfqpoint{1.254980in}{0.150000in}}{\pgfqpoint{5.490039in}{5.490039in}}%
\pgfusepath{clip}%
\pgfsetbuttcap%
\pgfsetroundjoin%
\definecolor{currentfill}{rgb}{0.265145,0.232956,0.516599}%
\pgfsetfillcolor{currentfill}%
\pgfsetfillopacity{0.700000}%
\pgfsetlinewidth{0.000000pt}%
\definecolor{currentstroke}{rgb}{0.000000,0.000000,0.000000}%
\pgfsetstrokecolor{currentstroke}%
\pgfsetdash{}{0pt}%
\pgfpathmoveto{\pgfqpoint{2.623645in}{1.888216in}}%
\pgfpathlineto{\pgfqpoint{2.637253in}{1.872312in}}%
\pgfpathlineto{\pgfqpoint{2.650855in}{1.856624in}}%
\pgfpathlineto{\pgfqpoint{2.664451in}{1.841153in}}%
\pgfpathlineto{\pgfqpoint{2.678041in}{1.825895in}}%
\pgfpathlineto{\pgfqpoint{2.686760in}{1.821759in}}%
\pgfpathlineto{\pgfqpoint{2.695458in}{1.817981in}}%
\pgfpathlineto{\pgfqpoint{2.704137in}{1.814553in}}%
\pgfpathlineto{\pgfqpoint{2.712795in}{1.811468in}}%
\pgfpathlineto{\pgfqpoint{2.699256in}{1.826032in}}%
\pgfpathlineto{\pgfqpoint{2.685711in}{1.840809in}}%
\pgfpathlineto{\pgfqpoint{2.672161in}{1.855800in}}%
\pgfpathlineto{\pgfqpoint{2.658605in}{1.871007in}}%
\pgfpathlineto{\pgfqpoint{2.649897in}{1.874776in}}%
\pgfpathlineto{\pgfqpoint{2.641167in}{1.878895in}}%
\pgfpathlineto{\pgfqpoint{2.632417in}{1.883373in}}%
\pgfpathlineto{\pgfqpoint{2.623645in}{1.888216in}}%
\pgfpathclose%
\pgfusepath{fill}%
\end{pgfscope}%
\begin{pgfscope}%
\pgfpathrectangle{\pgfqpoint{1.254980in}{0.150000in}}{\pgfqpoint{5.490039in}{5.490039in}}%
\pgfusepath{clip}%
\pgfsetbuttcap%
\pgfsetroundjoin%
\definecolor{currentfill}{rgb}{0.266580,0.228262,0.514349}%
\pgfsetfillcolor{currentfill}%
\pgfsetfillopacity{0.700000}%
\pgfsetlinewidth{0.000000pt}%
\definecolor{currentstroke}{rgb}{0.000000,0.000000,0.000000}%
\pgfsetstrokecolor{currentstroke}%
\pgfsetdash{}{0pt}%
\pgfpathmoveto{\pgfqpoint{4.143131in}{1.798945in}}%
\pgfpathlineto{\pgfqpoint{4.156748in}{1.804196in}}%
\pgfpathlineto{\pgfqpoint{4.170376in}{1.809607in}}%
\pgfpathlineto{\pgfqpoint{4.184016in}{1.815178in}}%
\pgfpathlineto{\pgfqpoint{4.197668in}{1.820908in}}%
\pgfpathlineto{\pgfqpoint{4.205487in}{1.834515in}}%
\pgfpathlineto{\pgfqpoint{4.213302in}{1.848086in}}%
\pgfpathlineto{\pgfqpoint{4.221113in}{1.861619in}}%
\pgfpathlineto{\pgfqpoint{4.228919in}{1.875111in}}%
\pgfpathlineto{\pgfqpoint{4.215268in}{1.869043in}}%
\pgfpathlineto{\pgfqpoint{4.201629in}{1.863136in}}%
\pgfpathlineto{\pgfqpoint{4.188002in}{1.857388in}}%
\pgfpathlineto{\pgfqpoint{4.174386in}{1.851800in}}%
\pgfpathlineto{\pgfqpoint{4.166579in}{1.838635in}}%
\pgfpathlineto{\pgfqpoint{4.158767in}{1.825435in}}%
\pgfpathlineto{\pgfqpoint{4.150951in}{1.812204in}}%
\pgfpathlineto{\pgfqpoint{4.143131in}{1.798945in}}%
\pgfpathclose%
\pgfusepath{fill}%
\end{pgfscope}%
\begin{pgfscope}%
\pgfpathrectangle{\pgfqpoint{1.254980in}{0.150000in}}{\pgfqpoint{5.490039in}{5.490039in}}%
\pgfusepath{clip}%
\pgfsetbuttcap%
\pgfsetroundjoin%
\definecolor{currentfill}{rgb}{0.147607,0.511733,0.557049}%
\pgfsetfillcolor{currentfill}%
\pgfsetfillopacity{0.700000}%
\pgfsetlinewidth{0.000000pt}%
\definecolor{currentstroke}{rgb}{0.000000,0.000000,0.000000}%
\pgfsetstrokecolor{currentstroke}%
\pgfsetdash{}{0pt}%
\pgfpathmoveto{\pgfqpoint{4.782700in}{2.504069in}}%
\pgfpathlineto{\pgfqpoint{4.796661in}{2.515262in}}%
\pgfpathlineto{\pgfqpoint{4.810640in}{2.526616in}}%
\pgfpathlineto{\pgfqpoint{4.824636in}{2.538132in}}%
\pgfpathlineto{\pgfqpoint{4.838649in}{2.549811in}}%
\pgfpathlineto{\pgfqpoint{4.846260in}{2.560694in}}%
\pgfpathlineto{\pgfqpoint{4.853863in}{2.571433in}}%
\pgfpathlineto{\pgfqpoint{4.861459in}{2.582026in}}%
\pgfpathlineto{\pgfqpoint{4.869048in}{2.592474in}}%
\pgfpathlineto{\pgfqpoint{4.855036in}{2.580749in}}%
\pgfpathlineto{\pgfqpoint{4.841042in}{2.569185in}}%
\pgfpathlineto{\pgfqpoint{4.827064in}{2.557784in}}%
\pgfpathlineto{\pgfqpoint{4.813104in}{2.546545in}}%
\pgfpathlineto{\pgfqpoint{4.805513in}{2.536132in}}%
\pgfpathlineto{\pgfqpoint{4.797915in}{2.525582in}}%
\pgfpathlineto{\pgfqpoint{4.790311in}{2.514894in}}%
\pgfpathlineto{\pgfqpoint{4.782700in}{2.504069in}}%
\pgfpathclose%
\pgfusepath{fill}%
\end{pgfscope}%
\begin{pgfscope}%
\pgfpathrectangle{\pgfqpoint{1.254980in}{0.150000in}}{\pgfqpoint{5.490039in}{5.490039in}}%
\pgfusepath{clip}%
\pgfsetbuttcap%
\pgfsetroundjoin%
\definecolor{currentfill}{rgb}{0.243113,0.292092,0.538516}%
\pgfsetfillcolor{currentfill}%
\pgfsetfillopacity{0.700000}%
\pgfsetlinewidth{0.000000pt}%
\definecolor{currentstroke}{rgb}{0.000000,0.000000,0.000000}%
\pgfsetstrokecolor{currentstroke}%
\pgfsetdash{}{0pt}%
\pgfpathmoveto{\pgfqpoint{2.514528in}{2.023448in}}%
\pgfpathlineto{\pgfqpoint{2.528194in}{2.005751in}}%
\pgfpathlineto{\pgfqpoint{2.541852in}{1.988284in}}%
\pgfpathlineto{\pgfqpoint{2.555502in}{1.971046in}}%
\pgfpathlineto{\pgfqpoint{2.569145in}{1.954034in}}%
\pgfpathlineto{\pgfqpoint{2.577972in}{1.948478in}}%
\pgfpathlineto{\pgfqpoint{2.586776in}{1.943296in}}%
\pgfpathlineto{\pgfqpoint{2.595559in}{1.938481in}}%
\pgfpathlineto{\pgfqpoint{2.604321in}{1.934026in}}%
\pgfpathlineto{\pgfqpoint{2.590733in}{1.950336in}}%
\pgfpathlineto{\pgfqpoint{2.577138in}{1.966871in}}%
\pgfpathlineto{\pgfqpoint{2.563536in}{1.983633in}}%
\pgfpathlineto{\pgfqpoint{2.549927in}{2.000624in}}%
\pgfpathlineto{\pgfqpoint{2.541111in}{2.005770in}}%
\pgfpathlineto{\pgfqpoint{2.532273in}{2.011285in}}%
\pgfpathlineto{\pgfqpoint{2.523412in}{2.017175in}}%
\pgfpathlineto{\pgfqpoint{2.514528in}{2.023448in}}%
\pgfpathclose%
\pgfusepath{fill}%
\end{pgfscope}%
\begin{pgfscope}%
\pgfpathrectangle{\pgfqpoint{1.254980in}{0.150000in}}{\pgfqpoint{5.490039in}{5.490039in}}%
\pgfusepath{clip}%
\pgfsetbuttcap%
\pgfsetroundjoin%
\definecolor{currentfill}{rgb}{0.273006,0.204520,0.501721}%
\pgfsetfillcolor{currentfill}%
\pgfsetfillopacity{0.700000}%
\pgfsetlinewidth{0.000000pt}%
\definecolor{currentstroke}{rgb}{0.000000,0.000000,0.000000}%
\pgfsetstrokecolor{currentstroke}%
\pgfsetdash{}{0pt}%
\pgfpathmoveto{\pgfqpoint{2.678041in}{1.825895in}}%
\pgfpathlineto{\pgfqpoint{2.691626in}{1.810849in}}%
\pgfpathlineto{\pgfqpoint{2.705205in}{1.796015in}}%
\pgfpathlineto{\pgfqpoint{2.718779in}{1.781391in}}%
\pgfpathlineto{\pgfqpoint{2.732348in}{1.766975in}}%
\pgfpathlineto{\pgfqpoint{2.741016in}{1.763544in}}%
\pgfpathlineto{\pgfqpoint{2.749664in}{1.760462in}}%
\pgfpathlineto{\pgfqpoint{2.758293in}{1.757721in}}%
\pgfpathlineto{\pgfqpoint{2.766903in}{1.755315in}}%
\pgfpathlineto{\pgfqpoint{2.753383in}{1.769041in}}%
\pgfpathlineto{\pgfqpoint{2.739859in}{1.782974in}}%
\pgfpathlineto{\pgfqpoint{2.726329in}{1.797116in}}%
\pgfpathlineto{\pgfqpoint{2.712795in}{1.811468in}}%
\pgfpathlineto{\pgfqpoint{2.704137in}{1.814553in}}%
\pgfpathlineto{\pgfqpoint{2.695458in}{1.817981in}}%
\pgfpathlineto{\pgfqpoint{2.686760in}{1.821759in}}%
\pgfpathlineto{\pgfqpoint{2.678041in}{1.825895in}}%
\pgfpathclose%
\pgfusepath{fill}%
\end{pgfscope}%
\begin{pgfscope}%
\pgfpathrectangle{\pgfqpoint{1.254980in}{0.150000in}}{\pgfqpoint{5.490039in}{5.490039in}}%
\pgfusepath{clip}%
\pgfsetbuttcap%
\pgfsetroundjoin%
\definecolor{currentfill}{rgb}{0.282884,0.135920,0.453427}%
\pgfsetfillcolor{currentfill}%
\pgfsetfillopacity{0.700000}%
\pgfsetlinewidth{0.000000pt}%
\definecolor{currentstroke}{rgb}{0.000000,0.000000,0.000000}%
\pgfsetstrokecolor{currentstroke}%
\pgfsetdash{}{0pt}%
\pgfpathmoveto{\pgfqpoint{3.940337in}{1.611994in}}%
\pgfpathlineto{\pgfqpoint{3.953879in}{1.614794in}}%
\pgfpathlineto{\pgfqpoint{3.967430in}{1.617753in}}%
\pgfpathlineto{\pgfqpoint{3.980991in}{1.620872in}}%
\pgfpathlineto{\pgfqpoint{3.994562in}{1.624151in}}%
\pgfpathlineto{\pgfqpoint{4.002438in}{1.637052in}}%
\pgfpathlineto{\pgfqpoint{4.010309in}{1.649974in}}%
\pgfpathlineto{\pgfqpoint{4.018176in}{1.662913in}}%
\pgfpathlineto{\pgfqpoint{4.026039in}{1.675866in}}%
\pgfpathlineto{\pgfqpoint{4.012472in}{1.672168in}}%
\pgfpathlineto{\pgfqpoint{3.998915in}{1.668630in}}%
\pgfpathlineto{\pgfqpoint{3.985369in}{1.665253in}}%
\pgfpathlineto{\pgfqpoint{3.971832in}{1.662035in}}%
\pgfpathlineto{\pgfqpoint{3.963966in}{1.649490in}}%
\pgfpathlineto{\pgfqpoint{3.956094in}{1.636966in}}%
\pgfpathlineto{\pgfqpoint{3.948218in}{1.624466in}}%
\pgfpathlineto{\pgfqpoint{3.940337in}{1.611994in}}%
\pgfpathclose%
\pgfusepath{fill}%
\end{pgfscope}%
\begin{pgfscope}%
\pgfpathrectangle{\pgfqpoint{1.254980in}{0.150000in}}{\pgfqpoint{5.490039in}{5.490039in}}%
\pgfusepath{clip}%
\pgfsetbuttcap%
\pgfsetroundjoin%
\definecolor{currentfill}{rgb}{0.268510,0.009605,0.335427}%
\pgfsetfillcolor{currentfill}%
\pgfsetfillopacity{0.700000}%
\pgfsetlinewidth{0.000000pt}%
\definecolor{currentstroke}{rgb}{0.000000,0.000000,0.000000}%
\pgfsetstrokecolor{currentstroke}%
\pgfsetdash{}{0pt}%
\pgfpathmoveto{\pgfqpoint{3.511510in}{1.403036in}}%
\pgfpathlineto{\pgfqpoint{3.524961in}{1.400098in}}%
\pgfpathlineto{\pgfqpoint{3.538417in}{1.397325in}}%
\pgfpathlineto{\pgfqpoint{3.551879in}{1.394716in}}%
\pgfpathlineto{\pgfqpoint{3.565345in}{1.392272in}}%
\pgfpathlineto{\pgfqpoint{3.573386in}{1.400979in}}%
\pgfpathlineto{\pgfqpoint{3.581420in}{1.409836in}}%
\pgfpathlineto{\pgfqpoint{3.589447in}{1.418838in}}%
\pgfpathlineto{\pgfqpoint{3.597466in}{1.427980in}}%
\pgfpathlineto{\pgfqpoint{3.584015in}{1.429870in}}%
\pgfpathlineto{\pgfqpoint{3.570571in}{1.431925in}}%
\pgfpathlineto{\pgfqpoint{3.557132in}{1.434145in}}%
\pgfpathlineto{\pgfqpoint{3.543699in}{1.436530in}}%
\pgfpathlineto{\pgfqpoint{3.535664in}{1.427931in}}%
\pgfpathlineto{\pgfqpoint{3.527620in}{1.419480in}}%
\pgfpathlineto{\pgfqpoint{3.519569in}{1.411179in}}%
\pgfpathlineto{\pgfqpoint{3.511510in}{1.403036in}}%
\pgfpathclose%
\pgfusepath{fill}%
\end{pgfscope}%
\begin{pgfscope}%
\pgfpathrectangle{\pgfqpoint{1.254980in}{0.150000in}}{\pgfqpoint{5.490039in}{5.490039in}}%
\pgfusepath{clip}%
\pgfsetbuttcap%
\pgfsetroundjoin%
\definecolor{currentfill}{rgb}{0.229739,0.322361,0.545706}%
\pgfsetfillcolor{currentfill}%
\pgfsetfillopacity{0.700000}%
\pgfsetlinewidth{0.000000pt}%
\definecolor{currentstroke}{rgb}{0.000000,0.000000,0.000000}%
\pgfsetstrokecolor{currentstroke}%
\pgfsetdash{}{0pt}%
\pgfpathmoveto{\pgfqpoint{2.459778in}{2.096566in}}%
\pgfpathlineto{\pgfqpoint{2.473479in}{2.077933in}}%
\pgfpathlineto{\pgfqpoint{2.487170in}{2.059537in}}%
\pgfpathlineto{\pgfqpoint{2.500853in}{2.041376in}}%
\pgfpathlineto{\pgfqpoint{2.514528in}{2.023448in}}%
\pgfpathlineto{\pgfqpoint{2.523412in}{2.017175in}}%
\pgfpathlineto{\pgfqpoint{2.532273in}{2.011285in}}%
\pgfpathlineto{\pgfqpoint{2.541111in}{2.005770in}}%
\pgfpathlineto{\pgfqpoint{2.549927in}{2.000624in}}%
\pgfpathlineto{\pgfqpoint{2.536310in}{2.017845in}}%
\pgfpathlineto{\pgfqpoint{2.522685in}{2.035298in}}%
\pgfpathlineto{\pgfqpoint{2.509052in}{2.052984in}}%
\pgfpathlineto{\pgfqpoint{2.495410in}{2.070907in}}%
\pgfpathlineto{\pgfqpoint{2.486537in}{2.076749in}}%
\pgfpathlineto{\pgfqpoint{2.477642in}{2.082968in}}%
\pgfpathlineto{\pgfqpoint{2.468722in}{2.089571in}}%
\pgfpathlineto{\pgfqpoint{2.459778in}{2.096566in}}%
\pgfpathclose%
\pgfusepath{fill}%
\end{pgfscope}%
\begin{pgfscope}%
\pgfpathrectangle{\pgfqpoint{1.254980in}{0.150000in}}{\pgfqpoint{5.490039in}{5.490039in}}%
\pgfusepath{clip}%
\pgfsetbuttcap%
\pgfsetroundjoin%
\definecolor{currentfill}{rgb}{0.278012,0.180367,0.486697}%
\pgfsetfillcolor{currentfill}%
\pgfsetfillopacity{0.700000}%
\pgfsetlinewidth{0.000000pt}%
\definecolor{currentstroke}{rgb}{0.000000,0.000000,0.000000}%
\pgfsetstrokecolor{currentstroke}%
\pgfsetdash{}{0pt}%
\pgfpathmoveto{\pgfqpoint{2.732348in}{1.766975in}}%
\pgfpathlineto{\pgfqpoint{2.745912in}{1.752766in}}%
\pgfpathlineto{\pgfqpoint{2.759472in}{1.738763in}}%
\pgfpathlineto{\pgfqpoint{2.773027in}{1.724965in}}%
\pgfpathlineto{\pgfqpoint{2.786578in}{1.711370in}}%
\pgfpathlineto{\pgfqpoint{2.795197in}{1.708639in}}%
\pgfpathlineto{\pgfqpoint{2.803797in}{1.706249in}}%
\pgfpathlineto{\pgfqpoint{2.812379in}{1.704193in}}%
\pgfpathlineto{\pgfqpoint{2.820943in}{1.702463in}}%
\pgfpathlineto{\pgfqpoint{2.807439in}{1.715371in}}%
\pgfpathlineto{\pgfqpoint{2.793931in}{1.728481in}}%
\pgfpathlineto{\pgfqpoint{2.780419in}{1.741796in}}%
\pgfpathlineto{\pgfqpoint{2.766903in}{1.755315in}}%
\pgfpathlineto{\pgfqpoint{2.758293in}{1.757721in}}%
\pgfpathlineto{\pgfqpoint{2.749664in}{1.760462in}}%
\pgfpathlineto{\pgfqpoint{2.741016in}{1.763544in}}%
\pgfpathlineto{\pgfqpoint{2.732348in}{1.766975in}}%
\pgfpathclose%
\pgfusepath{fill}%
\end{pgfscope}%
\begin{pgfscope}%
\pgfpathrectangle{\pgfqpoint{1.254980in}{0.150000in}}{\pgfqpoint{5.490039in}{5.490039in}}%
\pgfusepath{clip}%
\pgfsetbuttcap%
\pgfsetroundjoin%
\definecolor{currentfill}{rgb}{0.229739,0.322361,0.545706}%
\pgfsetfillcolor{currentfill}%
\pgfsetfillopacity{0.700000}%
\pgfsetlinewidth{0.000000pt}%
\definecolor{currentstroke}{rgb}{0.000000,0.000000,0.000000}%
\pgfsetstrokecolor{currentstroke}%
\pgfsetdash{}{0pt}%
\pgfpathmoveto{\pgfqpoint{4.345942in}{2.009450in}}%
\pgfpathlineto{\pgfqpoint{4.359658in}{2.016907in}}%
\pgfpathlineto{\pgfqpoint{4.373387in}{2.024524in}}%
\pgfpathlineto{\pgfqpoint{4.387130in}{2.032301in}}%
\pgfpathlineto{\pgfqpoint{4.400887in}{2.040238in}}%
\pgfpathlineto{\pgfqpoint{4.408654in}{2.053762in}}%
\pgfpathlineto{\pgfqpoint{4.416416in}{2.067205in}}%
\pgfpathlineto{\pgfqpoint{4.424174in}{2.080564in}}%
\pgfpathlineto{\pgfqpoint{4.431927in}{2.093836in}}%
\pgfpathlineto{\pgfqpoint{4.418169in}{2.085646in}}%
\pgfpathlineto{\pgfqpoint{4.404425in}{2.077616in}}%
\pgfpathlineto{\pgfqpoint{4.390695in}{2.069746in}}%
\pgfpathlineto{\pgfqpoint{4.376979in}{2.062037in}}%
\pgfpathlineto{\pgfqpoint{4.369227in}{2.049006in}}%
\pgfpathlineto{\pgfqpoint{4.361470in}{2.035897in}}%
\pgfpathlineto{\pgfqpoint{4.353708in}{2.022711in}}%
\pgfpathlineto{\pgfqpoint{4.345942in}{2.009450in}}%
\pgfpathclose%
\pgfusepath{fill}%
\end{pgfscope}%
\begin{pgfscope}%
\pgfpathrectangle{\pgfqpoint{1.254980in}{0.150000in}}{\pgfqpoint{5.490039in}{5.490039in}}%
\pgfusepath{clip}%
\pgfsetbuttcap%
\pgfsetroundjoin%
\definecolor{currentfill}{rgb}{0.121148,0.592739,0.544641}%
\pgfsetfillcolor{currentfill}%
\pgfsetfillopacity{0.700000}%
\pgfsetlinewidth{0.000000pt}%
\definecolor{currentstroke}{rgb}{0.000000,0.000000,0.000000}%
\pgfsetstrokecolor{currentstroke}%
\pgfsetdash{}{0pt}%
\pgfpathmoveto{\pgfqpoint{4.985701in}{2.719394in}}%
\pgfpathlineto{\pgfqpoint{4.999797in}{2.731934in}}%
\pgfpathlineto{\pgfqpoint{5.013910in}{2.744637in}}%
\pgfpathlineto{\pgfqpoint{5.028043in}{2.757502in}}%
\pgfpathlineto{\pgfqpoint{5.042193in}{2.770531in}}%
\pgfpathlineto{\pgfqpoint{5.049708in}{2.779618in}}%
\pgfpathlineto{\pgfqpoint{5.057215in}{2.788549in}}%
\pgfpathlineto{\pgfqpoint{5.064714in}{2.797324in}}%
\pgfpathlineto{\pgfqpoint{5.072204in}{2.805945in}}%
\pgfpathlineto{\pgfqpoint{5.058058in}{2.792962in}}%
\pgfpathlineto{\pgfqpoint{5.043930in}{2.780141in}}%
\pgfpathlineto{\pgfqpoint{5.029821in}{2.767483in}}%
\pgfpathlineto{\pgfqpoint{5.015730in}{2.754988in}}%
\pgfpathlineto{\pgfqpoint{5.008234in}{2.746311in}}%
\pgfpathlineto{\pgfqpoint{5.000731in}{2.737487in}}%
\pgfpathlineto{\pgfqpoint{4.993220in}{2.728515in}}%
\pgfpathlineto{\pgfqpoint{4.985701in}{2.719394in}}%
\pgfpathclose%
\pgfusepath{fill}%
\end{pgfscope}%
\begin{pgfscope}%
\pgfpathrectangle{\pgfqpoint{1.254980in}{0.150000in}}{\pgfqpoint{5.490039in}{5.490039in}}%
\pgfusepath{clip}%
\pgfsetbuttcap%
\pgfsetroundjoin%
\definecolor{currentfill}{rgb}{0.140210,0.665859,0.513427}%
\pgfsetfillcolor{currentfill}%
\pgfsetfillopacity{0.700000}%
\pgfsetlinewidth{0.000000pt}%
\definecolor{currentstroke}{rgb}{0.000000,0.000000,0.000000}%
\pgfsetstrokecolor{currentstroke}%
\pgfsetdash{}{0pt}%
\pgfpathmoveto{\pgfqpoint{5.188549in}{2.922310in}}%
\pgfpathlineto{\pgfqpoint{5.202778in}{2.935925in}}%
\pgfpathlineto{\pgfqpoint{5.217027in}{2.949703in}}%
\pgfpathlineto{\pgfqpoint{5.231295in}{2.963645in}}%
\pgfpathlineto{\pgfqpoint{5.245583in}{2.977750in}}%
\pgfpathlineto{\pgfqpoint{5.252982in}{2.984779in}}%
\pgfpathlineto{\pgfqpoint{5.260371in}{2.991652in}}%
\pgfpathlineto{\pgfqpoint{5.267751in}{2.998371in}}%
\pgfpathlineto{\pgfqpoint{5.275121in}{3.004937in}}%
\pgfpathlineto{\pgfqpoint{5.260842in}{2.990971in}}%
\pgfpathlineto{\pgfqpoint{5.246582in}{2.977169in}}%
\pgfpathlineto{\pgfqpoint{5.232342in}{2.963530in}}%
\pgfpathlineto{\pgfqpoint{5.218121in}{2.950054in}}%
\pgfpathlineto{\pgfqpoint{5.210741in}{2.943338in}}%
\pgfpathlineto{\pgfqpoint{5.203353in}{2.936476in}}%
\pgfpathlineto{\pgfqpoint{5.195955in}{2.929468in}}%
\pgfpathlineto{\pgfqpoint{5.188549in}{2.922310in}}%
\pgfpathclose%
\pgfusepath{fill}%
\end{pgfscope}%
\begin{pgfscope}%
\pgfpathrectangle{\pgfqpoint{1.254980in}{0.150000in}}{\pgfqpoint{5.490039in}{5.490039in}}%
\pgfusepath{clip}%
\pgfsetbuttcap%
\pgfsetroundjoin%
\definecolor{currentfill}{rgb}{0.276022,0.044167,0.370164}%
\pgfsetfillcolor{currentfill}%
\pgfsetfillopacity{0.700000}%
\pgfsetlinewidth{0.000000pt}%
\definecolor{currentstroke}{rgb}{0.000000,0.000000,0.000000}%
\pgfsetstrokecolor{currentstroke}%
\pgfsetdash{}{0pt}%
\pgfpathmoveto{\pgfqpoint{3.090527in}{1.485101in}}%
\pgfpathlineto{\pgfqpoint{3.103995in}{1.476195in}}%
\pgfpathlineto{\pgfqpoint{3.117463in}{1.467470in}}%
\pgfpathlineto{\pgfqpoint{3.130931in}{1.458924in}}%
\pgfpathlineto{\pgfqpoint{3.144400in}{1.450557in}}%
\pgfpathlineto{\pgfqpoint{3.152713in}{1.453103in}}%
\pgfpathlineto{\pgfqpoint{3.161012in}{1.455914in}}%
\pgfpathlineto{\pgfqpoint{3.169298in}{1.458985in}}%
\pgfpathlineto{\pgfqpoint{3.177572in}{1.462308in}}%
\pgfpathlineto{\pgfqpoint{3.164135in}{1.470032in}}%
\pgfpathlineto{\pgfqpoint{3.150700in}{1.477935in}}%
\pgfpathlineto{\pgfqpoint{3.137265in}{1.486017in}}%
\pgfpathlineto{\pgfqpoint{3.123831in}{1.494279in}}%
\pgfpathlineto{\pgfqpoint{3.115526in}{1.491587in}}%
\pgfpathlineto{\pgfqpoint{3.107206in}{1.489156in}}%
\pgfpathlineto{\pgfqpoint{3.098874in}{1.486992in}}%
\pgfpathlineto{\pgfqpoint{3.090527in}{1.485101in}}%
\pgfpathclose%
\pgfusepath{fill}%
\end{pgfscope}%
\begin{pgfscope}%
\pgfpathrectangle{\pgfqpoint{1.254980in}{0.150000in}}{\pgfqpoint{5.490039in}{5.490039in}}%
\pgfusepath{clip}%
\pgfsetbuttcap%
\pgfsetroundjoin%
\definecolor{currentfill}{rgb}{0.281412,0.155834,0.469201}%
\pgfsetfillcolor{currentfill}%
\pgfsetfillopacity{0.700000}%
\pgfsetlinewidth{0.000000pt}%
\definecolor{currentstroke}{rgb}{0.000000,0.000000,0.000000}%
\pgfsetstrokecolor{currentstroke}%
\pgfsetdash{}{0pt}%
\pgfpathmoveto{\pgfqpoint{2.786578in}{1.711370in}}%
\pgfpathlineto{\pgfqpoint{2.800125in}{1.697976in}}%
\pgfpathlineto{\pgfqpoint{2.813668in}{1.684784in}}%
\pgfpathlineto{\pgfqpoint{2.827207in}{1.671792in}}%
\pgfpathlineto{\pgfqpoint{2.840743in}{1.658997in}}%
\pgfpathlineto{\pgfqpoint{2.849315in}{1.656964in}}%
\pgfpathlineto{\pgfqpoint{2.857870in}{1.655264in}}%
\pgfpathlineto{\pgfqpoint{2.866406in}{1.653888in}}%
\pgfpathlineto{\pgfqpoint{2.874925in}{1.652831in}}%
\pgfpathlineto{\pgfqpoint{2.861434in}{1.664941in}}%
\pgfpathlineto{\pgfqpoint{2.847940in}{1.677249in}}%
\pgfpathlineto{\pgfqpoint{2.834443in}{1.689756in}}%
\pgfpathlineto{\pgfqpoint{2.820943in}{1.702463in}}%
\pgfpathlineto{\pgfqpoint{2.812379in}{1.704193in}}%
\pgfpathlineto{\pgfqpoint{2.803797in}{1.706249in}}%
\pgfpathlineto{\pgfqpoint{2.795197in}{1.708639in}}%
\pgfpathlineto{\pgfqpoint{2.786578in}{1.711370in}}%
\pgfpathclose%
\pgfusepath{fill}%
\end{pgfscope}%
\begin{pgfscope}%
\pgfpathrectangle{\pgfqpoint{1.254980in}{0.150000in}}{\pgfqpoint{5.490039in}{5.490039in}}%
\pgfusepath{clip}%
\pgfsetbuttcap%
\pgfsetroundjoin%
\definecolor{currentfill}{rgb}{0.214298,0.355619,0.551184}%
\pgfsetfillcolor{currentfill}%
\pgfsetfillopacity{0.700000}%
\pgfsetlinewidth{0.000000pt}%
\definecolor{currentstroke}{rgb}{0.000000,0.000000,0.000000}%
\pgfsetstrokecolor{currentstroke}%
\pgfsetdash{}{0pt}%
\pgfpathmoveto{\pgfqpoint{2.404881in}{2.173506in}}%
\pgfpathlineto{\pgfqpoint{2.418620in}{2.153907in}}%
\pgfpathlineto{\pgfqpoint{2.432349in}{2.134551in}}%
\pgfpathlineto{\pgfqpoint{2.446068in}{2.115439in}}%
\pgfpathlineto{\pgfqpoint{2.459778in}{2.096566in}}%
\pgfpathlineto{\pgfqpoint{2.468722in}{2.089571in}}%
\pgfpathlineto{\pgfqpoint{2.477642in}{2.082968in}}%
\pgfpathlineto{\pgfqpoint{2.486537in}{2.076749in}}%
\pgfpathlineto{\pgfqpoint{2.495410in}{2.070907in}}%
\pgfpathlineto{\pgfqpoint{2.481760in}{2.089067in}}%
\pgfpathlineto{\pgfqpoint{2.468101in}{2.107466in}}%
\pgfpathlineto{\pgfqpoint{2.454432in}{2.126106in}}%
\pgfpathlineto{\pgfqpoint{2.440754in}{2.144990in}}%
\pgfpathlineto{\pgfqpoint{2.431823in}{2.151533in}}%
\pgfpathlineto{\pgfqpoint{2.422867in}{2.158462in}}%
\pgfpathlineto{\pgfqpoint{2.413887in}{2.165784in}}%
\pgfpathlineto{\pgfqpoint{2.404881in}{2.173506in}}%
\pgfpathclose%
\pgfusepath{fill}%
\end{pgfscope}%
\begin{pgfscope}%
\pgfpathrectangle{\pgfqpoint{1.254980in}{0.150000in}}{\pgfqpoint{5.490039in}{5.490039in}}%
\pgfusepath{clip}%
\pgfsetbuttcap%
\pgfsetroundjoin%
\definecolor{currentfill}{rgb}{0.137770,0.537492,0.554906}%
\pgfsetfillcolor{currentfill}%
\pgfsetfillopacity{0.700000}%
\pgfsetlinewidth{0.000000pt}%
\definecolor{currentstroke}{rgb}{0.000000,0.000000,0.000000}%
\pgfsetstrokecolor{currentstroke}%
\pgfsetdash{}{0pt}%
\pgfpathmoveto{\pgfqpoint{2.108971in}{2.676852in}}%
\pgfpathlineto{\pgfqpoint{2.122957in}{2.651451in}}%
\pgfpathlineto{\pgfqpoint{2.136927in}{2.626351in}}%
\pgfpathlineto{\pgfqpoint{2.150881in}{2.601550in}}%
\pgfpathlineto{\pgfqpoint{2.164819in}{2.577044in}}%
\pgfpathlineto{\pgfqpoint{2.174050in}{2.567330in}}%
\pgfpathlineto{\pgfqpoint{2.183253in}{2.558032in}}%
\pgfpathlineto{\pgfqpoint{2.192429in}{2.549143in}}%
\pgfpathlineto{\pgfqpoint{2.201577in}{2.540657in}}%
\pgfpathlineto{\pgfqpoint{2.187709in}{2.564448in}}%
\pgfpathlineto{\pgfqpoint{2.173825in}{2.588533in}}%
\pgfpathlineto{\pgfqpoint{2.159926in}{2.612916in}}%
\pgfpathlineto{\pgfqpoint{2.146012in}{2.637598in}}%
\pgfpathlineto{\pgfqpoint{2.136795in}{2.646787in}}%
\pgfpathlineto{\pgfqpoint{2.127549in}{2.656388in}}%
\pgfpathlineto{\pgfqpoint{2.118275in}{2.666407in}}%
\pgfpathlineto{\pgfqpoint{2.108971in}{2.676852in}}%
\pgfpathclose%
\pgfusepath{fill}%
\end{pgfscope}%
\begin{pgfscope}%
\pgfpathrectangle{\pgfqpoint{1.254980in}{0.150000in}}{\pgfqpoint{5.490039in}{5.490039in}}%
\pgfusepath{clip}%
\pgfsetbuttcap%
\pgfsetroundjoin%
\definecolor{currentfill}{rgb}{0.165117,0.467423,0.558141}%
\pgfsetfillcolor{currentfill}%
\pgfsetfillopacity{0.700000}%
\pgfsetlinewidth{0.000000pt}%
\definecolor{currentstroke}{rgb}{0.000000,0.000000,0.000000}%
\pgfsetstrokecolor{currentstroke}%
\pgfsetdash{}{0pt}%
\pgfpathmoveto{\pgfqpoint{4.665889in}{2.370143in}}%
\pgfpathlineto{\pgfqpoint{4.679786in}{2.380504in}}%
\pgfpathlineto{\pgfqpoint{4.693699in}{2.391027in}}%
\pgfpathlineto{\pgfqpoint{4.707628in}{2.401711in}}%
\pgfpathlineto{\pgfqpoint{4.721574in}{2.412557in}}%
\pgfpathlineto{\pgfqpoint{4.729237in}{2.424469in}}%
\pgfpathlineto{\pgfqpoint{4.736894in}{2.436247in}}%
\pgfpathlineto{\pgfqpoint{4.744544in}{2.447891in}}%
\pgfpathlineto{\pgfqpoint{4.752188in}{2.459399in}}%
\pgfpathlineto{\pgfqpoint{4.738242in}{2.448446in}}%
\pgfpathlineto{\pgfqpoint{4.724313in}{2.437654in}}%
\pgfpathlineto{\pgfqpoint{4.710400in}{2.427024in}}%
\pgfpathlineto{\pgfqpoint{4.696504in}{2.416556in}}%
\pgfpathlineto{\pgfqpoint{4.688859in}{2.405143in}}%
\pgfpathlineto{\pgfqpoint{4.681209in}{2.393603in}}%
\pgfpathlineto{\pgfqpoint{4.673552in}{2.381936in}}%
\pgfpathlineto{\pgfqpoint{4.665889in}{2.370143in}}%
\pgfpathclose%
\pgfusepath{fill}%
\end{pgfscope}%
\begin{pgfscope}%
\pgfpathrectangle{\pgfqpoint{1.254980in}{0.150000in}}{\pgfqpoint{5.490039in}{5.490039in}}%
\pgfusepath{clip}%
\pgfsetbuttcap%
\pgfsetroundjoin%
\definecolor{currentfill}{rgb}{0.281477,0.755203,0.432552}%
\pgfsetfillcolor{currentfill}%
\pgfsetfillopacity{0.700000}%
\pgfsetlinewidth{0.000000pt}%
\definecolor{currentstroke}{rgb}{0.000000,0.000000,0.000000}%
\pgfsetstrokecolor{currentstroke}%
\pgfsetdash{}{0pt}%
\pgfpathmoveto{\pgfqpoint{5.477520in}{3.185355in}}%
\pgfpathlineto{\pgfqpoint{5.491948in}{3.200184in}}%
\pgfpathlineto{\pgfqpoint{5.506397in}{3.215177in}}%
\pgfpathlineto{\pgfqpoint{5.520867in}{3.230334in}}%
\pgfpathlineto{\pgfqpoint{5.535359in}{3.245656in}}%
\pgfpathlineto{\pgfqpoint{5.542564in}{3.249681in}}%
\pgfpathlineto{\pgfqpoint{5.549759in}{3.253565in}}%
\pgfpathlineto{\pgfqpoint{5.556943in}{3.257311in}}%
\pgfpathlineto{\pgfqpoint{5.564117in}{3.260922in}}%
\pgfpathlineto{\pgfqpoint{5.549641in}{3.245868in}}%
\pgfpathlineto{\pgfqpoint{5.535187in}{3.230979in}}%
\pgfpathlineto{\pgfqpoint{5.520754in}{3.216253in}}%
\pgfpathlineto{\pgfqpoint{5.506342in}{3.201690in}}%
\pgfpathlineto{\pgfqpoint{5.499151in}{3.197800in}}%
\pgfpathlineto{\pgfqpoint{5.491951in}{3.193783in}}%
\pgfpathlineto{\pgfqpoint{5.484740in}{3.189636in}}%
\pgfpathlineto{\pgfqpoint{5.477520in}{3.185355in}}%
\pgfpathclose%
\pgfusepath{fill}%
\end{pgfscope}%
\begin{pgfscope}%
\pgfpathrectangle{\pgfqpoint{1.254980in}{0.150000in}}{\pgfqpoint{5.490039in}{5.490039in}}%
\pgfusepath{clip}%
\pgfsetbuttcap%
\pgfsetroundjoin%
\definecolor{currentfill}{rgb}{0.279574,0.170599,0.479997}%
\pgfsetfillcolor{currentfill}%
\pgfsetfillopacity{0.700000}%
\pgfsetlinewidth{0.000000pt}%
\definecolor{currentstroke}{rgb}{0.000000,0.000000,0.000000}%
\pgfsetstrokecolor{currentstroke}%
\pgfsetdash{}{0pt}%
\pgfpathmoveto{\pgfqpoint{4.026039in}{1.675866in}}%
\pgfpathlineto{\pgfqpoint{4.039616in}{1.679723in}}%
\pgfpathlineto{\pgfqpoint{4.053204in}{1.683740in}}%
\pgfpathlineto{\pgfqpoint{4.066802in}{1.687916in}}%
\pgfpathlineto{\pgfqpoint{4.080411in}{1.692251in}}%
\pgfpathlineto{\pgfqpoint{4.088266in}{1.705616in}}%
\pgfpathlineto{\pgfqpoint{4.096117in}{1.718980in}}%
\pgfpathlineto{\pgfqpoint{4.103964in}{1.732340in}}%
\pgfpathlineto{\pgfqpoint{4.111806in}{1.745691in}}%
\pgfpathlineto{\pgfqpoint{4.098199in}{1.740963in}}%
\pgfpathlineto{\pgfqpoint{4.084603in}{1.736395in}}%
\pgfpathlineto{\pgfqpoint{4.071018in}{1.731987in}}%
\pgfpathlineto{\pgfqpoint{4.057444in}{1.727738in}}%
\pgfpathlineto{\pgfqpoint{4.049599in}{1.714768in}}%
\pgfpathlineto{\pgfqpoint{4.041750in}{1.701797in}}%
\pgfpathlineto{\pgfqpoint{4.033897in}{1.688829in}}%
\pgfpathlineto{\pgfqpoint{4.026039in}{1.675866in}}%
\pgfpathclose%
\pgfusepath{fill}%
\end{pgfscope}%
\begin{pgfscope}%
\pgfpathrectangle{\pgfqpoint{1.254980in}{0.150000in}}{\pgfqpoint{5.490039in}{5.490039in}}%
\pgfusepath{clip}%
\pgfsetbuttcap%
\pgfsetroundjoin%
\definecolor{currentfill}{rgb}{0.268510,0.009605,0.335427}%
\pgfsetfillcolor{currentfill}%
\pgfsetfillopacity{0.700000}%
\pgfsetlinewidth{0.000000pt}%
\definecolor{currentstroke}{rgb}{0.000000,0.000000,0.000000}%
\pgfsetstrokecolor{currentstroke}%
\pgfsetdash{}{0pt}%
\pgfpathmoveto{\pgfqpoint{3.285114in}{1.406850in}}%
\pgfpathlineto{\pgfqpoint{3.298566in}{1.400699in}}%
\pgfpathlineto{\pgfqpoint{3.312020in}{1.394720in}}%
\pgfpathlineto{\pgfqpoint{3.325477in}{1.388911in}}%
\pgfpathlineto{\pgfqpoint{3.338937in}{1.383273in}}%
\pgfpathlineto{\pgfqpoint{3.347113in}{1.388710in}}%
\pgfpathlineto{\pgfqpoint{3.355279in}{1.394366in}}%
\pgfpathlineto{\pgfqpoint{3.363435in}{1.400233in}}%
\pgfpathlineto{\pgfqpoint{3.371581in}{1.406307in}}%
\pgfpathlineto{\pgfqpoint{3.358147in}{1.411334in}}%
\pgfpathlineto{\pgfqpoint{3.344715in}{1.416532in}}%
\pgfpathlineto{\pgfqpoint{3.331287in}{1.421900in}}%
\pgfpathlineto{\pgfqpoint{3.317861in}{1.427440in}}%
\pgfpathlineto{\pgfqpoint{3.309691in}{1.421966in}}%
\pgfpathlineto{\pgfqpoint{3.301509in}{1.416706in}}%
\pgfpathlineto{\pgfqpoint{3.293317in}{1.411665in}}%
\pgfpathlineto{\pgfqpoint{3.285114in}{1.406850in}}%
\pgfpathclose%
\pgfusepath{fill}%
\end{pgfscope}%
\begin{pgfscope}%
\pgfpathrectangle{\pgfqpoint{1.254980in}{0.150000in}}{\pgfqpoint{5.490039in}{5.490039in}}%
\pgfusepath{clip}%
\pgfsetbuttcap%
\pgfsetroundjoin%
\definecolor{currentfill}{rgb}{0.252194,0.269783,0.531579}%
\pgfsetfillcolor{currentfill}%
\pgfsetfillopacity{0.700000}%
\pgfsetlinewidth{0.000000pt}%
\definecolor{currentstroke}{rgb}{0.000000,0.000000,0.000000}%
\pgfsetstrokecolor{currentstroke}%
\pgfsetdash{}{0pt}%
\pgfpathmoveto{\pgfqpoint{4.228919in}{1.875111in}}%
\pgfpathlineto{\pgfqpoint{4.242583in}{1.881338in}}%
\pgfpathlineto{\pgfqpoint{4.256259in}{1.887724in}}%
\pgfpathlineto{\pgfqpoint{4.269948in}{1.894271in}}%
\pgfpathlineto{\pgfqpoint{4.283650in}{1.900976in}}%
\pgfpathlineto{\pgfqpoint{4.291452in}{1.914744in}}%
\pgfpathlineto{\pgfqpoint{4.299249in}{1.928458in}}%
\pgfpathlineto{\pgfqpoint{4.307043in}{1.942114in}}%
\pgfpathlineto{\pgfqpoint{4.314832in}{1.955711in}}%
\pgfpathlineto{\pgfqpoint{4.301130in}{1.948695in}}%
\pgfpathlineto{\pgfqpoint{4.287441in}{1.941839in}}%
\pgfpathlineto{\pgfqpoint{4.273764in}{1.935144in}}%
\pgfpathlineto{\pgfqpoint{4.260101in}{1.928608in}}%
\pgfpathlineto{\pgfqpoint{4.252312in}{1.915309in}}%
\pgfpathlineto{\pgfqpoint{4.244519in}{1.901959in}}%
\pgfpathlineto{\pgfqpoint{4.236721in}{1.888558in}}%
\pgfpathlineto{\pgfqpoint{4.228919in}{1.875111in}}%
\pgfpathclose%
\pgfusepath{fill}%
\end{pgfscope}%
\begin{pgfscope}%
\pgfpathrectangle{\pgfqpoint{1.254980in}{0.150000in}}{\pgfqpoint{5.490039in}{5.490039in}}%
\pgfusepath{clip}%
\pgfsetbuttcap%
\pgfsetroundjoin%
\definecolor{currentfill}{rgb}{0.283072,0.130895,0.449241}%
\pgfsetfillcolor{currentfill}%
\pgfsetfillopacity{0.700000}%
\pgfsetlinewidth{0.000000pt}%
\definecolor{currentstroke}{rgb}{0.000000,0.000000,0.000000}%
\pgfsetstrokecolor{currentstroke}%
\pgfsetdash{}{0pt}%
\pgfpathmoveto{\pgfqpoint{2.840743in}{1.658997in}}%
\pgfpathlineto{\pgfqpoint{2.854276in}{1.646401in}}%
\pgfpathlineto{\pgfqpoint{2.867805in}{1.634000in}}%
\pgfpathlineto{\pgfqpoint{2.881332in}{1.621794in}}%
\pgfpathlineto{\pgfqpoint{2.894856in}{1.609783in}}%
\pgfpathlineto{\pgfqpoint{2.903384in}{1.608444in}}%
\pgfpathlineto{\pgfqpoint{2.911894in}{1.607430in}}%
\pgfpathlineto{\pgfqpoint{2.920387in}{1.606733in}}%
\pgfpathlineto{\pgfqpoint{2.928864in}{1.606346in}}%
\pgfpathlineto{\pgfqpoint{2.915383in}{1.617676in}}%
\pgfpathlineto{\pgfqpoint{2.901900in}{1.629200in}}%
\pgfpathlineto{\pgfqpoint{2.888414in}{1.640918in}}%
\pgfpathlineto{\pgfqpoint{2.874925in}{1.652831in}}%
\pgfpathlineto{\pgfqpoint{2.866406in}{1.653888in}}%
\pgfpathlineto{\pgfqpoint{2.857870in}{1.655264in}}%
\pgfpathlineto{\pgfqpoint{2.849315in}{1.656964in}}%
\pgfpathlineto{\pgfqpoint{2.840743in}{1.658997in}}%
\pgfpathclose%
\pgfusepath{fill}%
\end{pgfscope}%
\begin{pgfscope}%
\pgfpathrectangle{\pgfqpoint{1.254980in}{0.150000in}}{\pgfqpoint{5.490039in}{5.490039in}}%
\pgfusepath{clip}%
\pgfsetbuttcap%
\pgfsetroundjoin%
\definecolor{currentfill}{rgb}{0.199430,0.387607,0.554642}%
\pgfsetfillcolor{currentfill}%
\pgfsetfillopacity{0.700000}%
\pgfsetlinewidth{0.000000pt}%
\definecolor{currentstroke}{rgb}{0.000000,0.000000,0.000000}%
\pgfsetstrokecolor{currentstroke}%
\pgfsetdash{}{0pt}%
\pgfpathmoveto{\pgfqpoint{2.349821in}{2.254392in}}%
\pgfpathlineto{\pgfqpoint{2.363603in}{2.233794in}}%
\pgfpathlineto{\pgfqpoint{2.377373in}{2.213448in}}%
\pgfpathlineto{\pgfqpoint{2.391132in}{2.193353in}}%
\pgfpathlineto{\pgfqpoint{2.404881in}{2.173506in}}%
\pgfpathlineto{\pgfqpoint{2.413887in}{2.165784in}}%
\pgfpathlineto{\pgfqpoint{2.422867in}{2.158462in}}%
\pgfpathlineto{\pgfqpoint{2.431823in}{2.151533in}}%
\pgfpathlineto{\pgfqpoint{2.440754in}{2.144990in}}%
\pgfpathlineto{\pgfqpoint{2.427067in}{2.164119in}}%
\pgfpathlineto{\pgfqpoint{2.413370in}{2.183495in}}%
\pgfpathlineto{\pgfqpoint{2.399663in}{2.203120in}}%
\pgfpathlineto{\pgfqpoint{2.385945in}{2.222996in}}%
\pgfpathlineto{\pgfqpoint{2.376952in}{2.230246in}}%
\pgfpathlineto{\pgfqpoint{2.367934in}{2.237890in}}%
\pgfpathlineto{\pgfqpoint{2.358891in}{2.245936in}}%
\pgfpathlineto{\pgfqpoint{2.349821in}{2.254392in}}%
\pgfpathclose%
\pgfusepath{fill}%
\end{pgfscope}%
\begin{pgfscope}%
\pgfpathrectangle{\pgfqpoint{1.254980in}{0.150000in}}{\pgfqpoint{5.490039in}{5.490039in}}%
\pgfusepath{clip}%
\pgfsetbuttcap%
\pgfsetroundjoin%
\definecolor{currentfill}{rgb}{0.267004,0.004874,0.329415}%
\pgfsetfillcolor{currentfill}%
\pgfsetfillopacity{0.700000}%
\pgfsetlinewidth{0.000000pt}%
\definecolor{currentstroke}{rgb}{0.000000,0.000000,0.000000}%
\pgfsetstrokecolor{currentstroke}%
\pgfsetdash{}{0pt}%
\pgfpathmoveto{\pgfqpoint{3.425355in}{1.387888in}}%
\pgfpathlineto{\pgfqpoint{3.438808in}{1.383703in}}%
\pgfpathlineto{\pgfqpoint{3.452265in}{1.379685in}}%
\pgfpathlineto{\pgfqpoint{3.465726in}{1.375834in}}%
\pgfpathlineto{\pgfqpoint{3.479192in}{1.372148in}}%
\pgfpathlineto{\pgfqpoint{3.487284in}{1.379606in}}%
\pgfpathlineto{\pgfqpoint{3.495368in}{1.387244in}}%
\pgfpathlineto{\pgfqpoint{3.503443in}{1.395056in}}%
\pgfpathlineto{\pgfqpoint{3.511510in}{1.403036in}}%
\pgfpathlineto{\pgfqpoint{3.498064in}{1.406140in}}%
\pgfpathlineto{\pgfqpoint{3.484623in}{1.409410in}}%
\pgfpathlineto{\pgfqpoint{3.471187in}{1.412846in}}%
\pgfpathlineto{\pgfqpoint{3.457755in}{1.416449in}}%
\pgfpathlineto{\pgfqpoint{3.449668in}{1.409040in}}%
\pgfpathlineto{\pgfqpoint{3.441573in}{1.401806in}}%
\pgfpathlineto{\pgfqpoint{3.433468in}{1.394754in}}%
\pgfpathlineto{\pgfqpoint{3.425355in}{1.387888in}}%
\pgfpathclose%
\pgfusepath{fill}%
\end{pgfscope}%
\begin{pgfscope}%
\pgfpathrectangle{\pgfqpoint{1.254980in}{0.150000in}}{\pgfqpoint{5.490039in}{5.490039in}}%
\pgfusepath{clip}%
\pgfsetbuttcap%
\pgfsetroundjoin%
\definecolor{currentfill}{rgb}{0.185556,0.418570,0.556753}%
\pgfsetfillcolor{currentfill}%
\pgfsetfillopacity{0.700000}%
\pgfsetlinewidth{0.000000pt}%
\definecolor{currentstroke}{rgb}{0.000000,0.000000,0.000000}%
\pgfsetstrokecolor{currentstroke}%
\pgfsetdash{}{0pt}%
\pgfpathmoveto{\pgfqpoint{4.548953in}{2.232737in}}%
\pgfpathlineto{\pgfqpoint{4.562786in}{2.242150in}}%
\pgfpathlineto{\pgfqpoint{4.576634in}{2.251724in}}%
\pgfpathlineto{\pgfqpoint{4.590498in}{2.261459in}}%
\pgfpathlineto{\pgfqpoint{4.604377in}{2.271356in}}%
\pgfpathlineto{\pgfqpoint{4.612086in}{2.284125in}}%
\pgfpathlineto{\pgfqpoint{4.619789in}{2.296777in}}%
\pgfpathlineto{\pgfqpoint{4.627487in}{2.309310in}}%
\pgfpathlineto{\pgfqpoint{4.635179in}{2.321722in}}%
\pgfpathlineto{\pgfqpoint{4.621299in}{2.311659in}}%
\pgfpathlineto{\pgfqpoint{4.607434in}{2.301757in}}%
\pgfpathlineto{\pgfqpoint{4.593585in}{2.292017in}}%
\pgfpathlineto{\pgfqpoint{4.579751in}{2.282437in}}%
\pgfpathlineto{\pgfqpoint{4.572060in}{2.270180in}}%
\pgfpathlineto{\pgfqpoint{4.564363in}{2.257810in}}%
\pgfpathlineto{\pgfqpoint{4.556661in}{2.245329in}}%
\pgfpathlineto{\pgfqpoint{4.548953in}{2.232737in}}%
\pgfpathclose%
\pgfusepath{fill}%
\end{pgfscope}%
\begin{pgfscope}%
\pgfpathrectangle{\pgfqpoint{1.254980in}{0.150000in}}{\pgfqpoint{5.490039in}{5.490039in}}%
\pgfusepath{clip}%
\pgfsetbuttcap%
\pgfsetroundjoin%
\definecolor{currentfill}{rgb}{0.132444,0.552216,0.553018}%
\pgfsetfillcolor{currentfill}%
\pgfsetfillopacity{0.700000}%
\pgfsetlinewidth{0.000000pt}%
\definecolor{currentstroke}{rgb}{0.000000,0.000000,0.000000}%
\pgfsetstrokecolor{currentstroke}%
\pgfsetdash{}{0pt}%
\pgfpathmoveto{\pgfqpoint{4.869048in}{2.592474in}}%
\pgfpathlineto{\pgfqpoint{4.883078in}{2.604362in}}%
\pgfpathlineto{\pgfqpoint{4.897124in}{2.616412in}}%
\pgfpathlineto{\pgfqpoint{4.911189in}{2.628625in}}%
\pgfpathlineto{\pgfqpoint{4.925272in}{2.641000in}}%
\pgfpathlineto{\pgfqpoint{4.932852in}{2.651331in}}%
\pgfpathlineto{\pgfqpoint{4.940425in}{2.661510in}}%
\pgfpathlineto{\pgfqpoint{4.947990in}{2.671536in}}%
\pgfpathlineto{\pgfqpoint{4.955548in}{2.681410in}}%
\pgfpathlineto{\pgfqpoint{4.941467in}{2.669018in}}%
\pgfpathlineto{\pgfqpoint{4.927405in}{2.656789in}}%
\pgfpathlineto{\pgfqpoint{4.913360in}{2.644722in}}%
\pgfpathlineto{\pgfqpoint{4.899333in}{2.632818in}}%
\pgfpathlineto{\pgfqpoint{4.891773in}{2.622949in}}%
\pgfpathlineto{\pgfqpoint{4.884205in}{2.612936in}}%
\pgfpathlineto{\pgfqpoint{4.876630in}{2.602778in}}%
\pgfpathlineto{\pgfqpoint{4.869048in}{2.592474in}}%
\pgfpathclose%
\pgfusepath{fill}%
\end{pgfscope}%
\begin{pgfscope}%
\pgfpathrectangle{\pgfqpoint{1.254980in}{0.150000in}}{\pgfqpoint{5.490039in}{5.490039in}}%
\pgfusepath{clip}%
\pgfsetbuttcap%
\pgfsetroundjoin%
\definecolor{currentfill}{rgb}{0.273809,0.031497,0.358853}%
\pgfsetfillcolor{currentfill}%
\pgfsetfillopacity{0.700000}%
\pgfsetlinewidth{0.000000pt}%
\definecolor{currentstroke}{rgb}{0.000000,0.000000,0.000000}%
\pgfsetstrokecolor{currentstroke}%
\pgfsetdash{}{0pt}%
\pgfpathmoveto{\pgfqpoint{3.144400in}{1.450557in}}%
\pgfpathlineto{\pgfqpoint{3.157870in}{1.442368in}}%
\pgfpathlineto{\pgfqpoint{3.171340in}{1.434357in}}%
\pgfpathlineto{\pgfqpoint{3.184812in}{1.426521in}}%
\pgfpathlineto{\pgfqpoint{3.198284in}{1.418862in}}%
\pgfpathlineto{\pgfqpoint{3.206564in}{1.422061in}}%
\pgfpathlineto{\pgfqpoint{3.214831in}{1.425518in}}%
\pgfpathlineto{\pgfqpoint{3.223086in}{1.429226in}}%
\pgfpathlineto{\pgfqpoint{3.231329in}{1.433180in}}%
\pgfpathlineto{\pgfqpoint{3.217888in}{1.440198in}}%
\pgfpathlineto{\pgfqpoint{3.204447in}{1.447392in}}%
\pgfpathlineto{\pgfqpoint{3.191009in}{1.454762in}}%
\pgfpathlineto{\pgfqpoint{3.177572in}{1.462308in}}%
\pgfpathlineto{\pgfqpoint{3.169298in}{1.458985in}}%
\pgfpathlineto{\pgfqpoint{3.161012in}{1.455914in}}%
\pgfpathlineto{\pgfqpoint{3.152713in}{1.453103in}}%
\pgfpathlineto{\pgfqpoint{3.144400in}{1.450557in}}%
\pgfpathclose%
\pgfusepath{fill}%
\end{pgfscope}%
\begin{pgfscope}%
\pgfpathrectangle{\pgfqpoint{1.254980in}{0.150000in}}{\pgfqpoint{5.490039in}{5.490039in}}%
\pgfusepath{clip}%
\pgfsetbuttcap%
\pgfsetroundjoin%
\definecolor{currentfill}{rgb}{0.277941,0.056324,0.381191}%
\pgfsetfillcolor{currentfill}%
\pgfsetfillopacity{0.700000}%
\pgfsetlinewidth{0.000000pt}%
\definecolor{currentstroke}{rgb}{0.000000,0.000000,0.000000}%
\pgfsetstrokecolor{currentstroke}%
\pgfsetdash{}{0pt}%
\pgfpathmoveto{\pgfqpoint{3.737192in}{1.460789in}}%
\pgfpathlineto{\pgfqpoint{3.750688in}{1.460887in}}%
\pgfpathlineto{\pgfqpoint{3.764193in}{1.461146in}}%
\pgfpathlineto{\pgfqpoint{3.777705in}{1.461566in}}%
\pgfpathlineto{\pgfqpoint{3.791225in}{1.462146in}}%
\pgfpathlineto{\pgfqpoint{3.799174in}{1.473435in}}%
\pgfpathlineto{\pgfqpoint{3.807118in}{1.484812in}}%
\pgfpathlineto{\pgfqpoint{3.815057in}{1.496272in}}%
\pgfpathlineto{\pgfqpoint{3.822990in}{1.507810in}}%
\pgfpathlineto{\pgfqpoint{3.809479in}{1.506729in}}%
\pgfpathlineto{\pgfqpoint{3.795977in}{1.505810in}}%
\pgfpathlineto{\pgfqpoint{3.782482in}{1.505051in}}%
\pgfpathlineto{\pgfqpoint{3.768996in}{1.504453in}}%
\pgfpathlineto{\pgfqpoint{3.761054in}{1.493404in}}%
\pgfpathlineto{\pgfqpoint{3.753105in}{1.482440in}}%
\pgfpathlineto{\pgfqpoint{3.745151in}{1.471567in}}%
\pgfpathlineto{\pgfqpoint{3.737192in}{1.460789in}}%
\pgfpathclose%
\pgfusepath{fill}%
\end{pgfscope}%
\begin{pgfscope}%
\pgfpathrectangle{\pgfqpoint{1.254980in}{0.150000in}}{\pgfqpoint{5.490039in}{5.490039in}}%
\pgfusepath{clip}%
\pgfsetbuttcap%
\pgfsetroundjoin%
\definecolor{currentfill}{rgb}{0.283091,0.110553,0.431554}%
\pgfsetfillcolor{currentfill}%
\pgfsetfillopacity{0.700000}%
\pgfsetlinewidth{0.000000pt}%
\definecolor{currentstroke}{rgb}{0.000000,0.000000,0.000000}%
\pgfsetstrokecolor{currentstroke}%
\pgfsetdash{}{0pt}%
\pgfpathmoveto{\pgfqpoint{2.894856in}{1.609783in}}%
\pgfpathlineto{\pgfqpoint{2.908378in}{1.597964in}}%
\pgfpathlineto{\pgfqpoint{2.921897in}{1.586337in}}%
\pgfpathlineto{\pgfqpoint{2.935414in}{1.574901in}}%
\pgfpathlineto{\pgfqpoint{2.948929in}{1.563655in}}%
\pgfpathlineto{\pgfqpoint{2.957413in}{1.563008in}}%
\pgfpathlineto{\pgfqpoint{2.965881in}{1.562678in}}%
\pgfpathlineto{\pgfqpoint{2.974333in}{1.562657in}}%
\pgfpathlineto{\pgfqpoint{2.982770in}{1.562938in}}%
\pgfpathlineto{\pgfqpoint{2.969296in}{1.573505in}}%
\pgfpathlineto{\pgfqpoint{2.955820in}{1.584261in}}%
\pgfpathlineto{\pgfqpoint{2.942343in}{1.595208in}}%
\pgfpathlineto{\pgfqpoint{2.928864in}{1.606346in}}%
\pgfpathlineto{\pgfqpoint{2.920387in}{1.606733in}}%
\pgfpathlineto{\pgfqpoint{2.911894in}{1.607430in}}%
\pgfpathlineto{\pgfqpoint{2.903384in}{1.608444in}}%
\pgfpathlineto{\pgfqpoint{2.894856in}{1.609783in}}%
\pgfpathclose%
\pgfusepath{fill}%
\end{pgfscope}%
\begin{pgfscope}%
\pgfpathrectangle{\pgfqpoint{1.254980in}{0.150000in}}{\pgfqpoint{5.490039in}{5.490039in}}%
\pgfusepath{clip}%
\pgfsetbuttcap%
\pgfsetroundjoin%
\definecolor{currentfill}{rgb}{0.180653,0.701402,0.488189}%
\pgfsetfillcolor{currentfill}%
\pgfsetfillopacity{0.700000}%
\pgfsetlinewidth{0.000000pt}%
\definecolor{currentstroke}{rgb}{0.000000,0.000000,0.000000}%
\pgfsetstrokecolor{currentstroke}%
\pgfsetdash{}{0pt}%
\pgfpathmoveto{\pgfqpoint{5.275121in}{3.004937in}}%
\pgfpathlineto{\pgfqpoint{5.289421in}{3.019066in}}%
\pgfpathlineto{\pgfqpoint{5.303741in}{3.033358in}}%
\pgfpathlineto{\pgfqpoint{5.318080in}{3.047815in}}%
\pgfpathlineto{\pgfqpoint{5.332441in}{3.062435in}}%
\pgfpathlineto{\pgfqpoint{5.339792in}{3.068690in}}%
\pgfpathlineto{\pgfqpoint{5.347134in}{3.074789in}}%
\pgfpathlineto{\pgfqpoint{5.354466in}{3.080733in}}%
\pgfpathlineto{\pgfqpoint{5.361788in}{3.086525in}}%
\pgfpathlineto{\pgfqpoint{5.347438in}{3.072077in}}%
\pgfpathlineto{\pgfqpoint{5.333109in}{3.057792in}}%
\pgfpathlineto{\pgfqpoint{5.318799in}{3.043670in}}%
\pgfpathlineto{\pgfqpoint{5.304510in}{3.029712in}}%
\pgfpathlineto{\pgfqpoint{5.297177in}{3.023737in}}%
\pgfpathlineto{\pgfqpoint{5.289834in}{3.017618in}}%
\pgfpathlineto{\pgfqpoint{5.282482in}{3.011352in}}%
\pgfpathlineto{\pgfqpoint{5.275121in}{3.004937in}}%
\pgfpathclose%
\pgfusepath{fill}%
\end{pgfscope}%
\begin{pgfscope}%
\pgfpathrectangle{\pgfqpoint{1.254980in}{0.150000in}}{\pgfqpoint{5.490039in}{5.490039in}}%
\pgfusepath{clip}%
\pgfsetbuttcap%
\pgfsetroundjoin%
\definecolor{currentfill}{rgb}{0.273809,0.031497,0.358853}%
\pgfsetfillcolor{currentfill}%
\pgfsetfillopacity{0.700000}%
\pgfsetlinewidth{0.000000pt}%
\definecolor{currentstroke}{rgb}{0.000000,0.000000,0.000000}%
\pgfsetstrokecolor{currentstroke}%
\pgfsetdash{}{0pt}%
\pgfpathmoveto{\pgfqpoint{3.651327in}{1.422052in}}%
\pgfpathlineto{\pgfqpoint{3.664808in}{1.420977in}}%
\pgfpathlineto{\pgfqpoint{3.678296in}{1.420064in}}%
\pgfpathlineto{\pgfqpoint{3.691790in}{1.419313in}}%
\pgfpathlineto{\pgfqpoint{3.705292in}{1.418723in}}%
\pgfpathlineto{\pgfqpoint{3.713276in}{1.429072in}}%
\pgfpathlineto{\pgfqpoint{3.721254in}{1.439536in}}%
\pgfpathlineto{\pgfqpoint{3.729226in}{1.450110in}}%
\pgfpathlineto{\pgfqpoint{3.737192in}{1.460789in}}%
\pgfpathlineto{\pgfqpoint{3.723702in}{1.460852in}}%
\pgfpathlineto{\pgfqpoint{3.710220in}{1.461076in}}%
\pgfpathlineto{\pgfqpoint{3.696745in}{1.461462in}}%
\pgfpathlineto{\pgfqpoint{3.683277in}{1.462011in}}%
\pgfpathlineto{\pgfqpoint{3.675299in}{1.451849in}}%
\pgfpathlineto{\pgfqpoint{3.667315in}{1.441798in}}%
\pgfpathlineto{\pgfqpoint{3.659324in}{1.431864in}}%
\pgfpathlineto{\pgfqpoint{3.651327in}{1.422052in}}%
\pgfpathclose%
\pgfusepath{fill}%
\end{pgfscope}%
\begin{pgfscope}%
\pgfpathrectangle{\pgfqpoint{1.254980in}{0.150000in}}{\pgfqpoint{5.490039in}{5.490039in}}%
\pgfusepath{clip}%
\pgfsetbuttcap%
\pgfsetroundjoin%
\definecolor{currentfill}{rgb}{0.183898,0.422383,0.556944}%
\pgfsetfillcolor{currentfill}%
\pgfsetfillopacity{0.700000}%
\pgfsetlinewidth{0.000000pt}%
\definecolor{currentstroke}{rgb}{0.000000,0.000000,0.000000}%
\pgfsetstrokecolor{currentstroke}%
\pgfsetdash{}{0pt}%
\pgfpathmoveto{\pgfqpoint{2.294581in}{2.339358in}}%
\pgfpathlineto{\pgfqpoint{2.308409in}{2.317726in}}%
\pgfpathlineto{\pgfqpoint{2.322225in}{2.296356in}}%
\pgfpathlineto{\pgfqpoint{2.336029in}{2.275246in}}%
\pgfpathlineto{\pgfqpoint{2.349821in}{2.254392in}}%
\pgfpathlineto{\pgfqpoint{2.358891in}{2.245936in}}%
\pgfpathlineto{\pgfqpoint{2.367934in}{2.237890in}}%
\pgfpathlineto{\pgfqpoint{2.376952in}{2.230246in}}%
\pgfpathlineto{\pgfqpoint{2.385945in}{2.222996in}}%
\pgfpathlineto{\pgfqpoint{2.372216in}{2.243126in}}%
\pgfpathlineto{\pgfqpoint{2.358477in}{2.263511in}}%
\pgfpathlineto{\pgfqpoint{2.344726in}{2.284155in}}%
\pgfpathlineto{\pgfqpoint{2.330964in}{2.305058in}}%
\pgfpathlineto{\pgfqpoint{2.321908in}{2.313019in}}%
\pgfpathlineto{\pgfqpoint{2.312826in}{2.321385in}}%
\pgfpathlineto{\pgfqpoint{2.303717in}{2.330162in}}%
\pgfpathlineto{\pgfqpoint{2.294581in}{2.339358in}}%
\pgfpathclose%
\pgfusepath{fill}%
\end{pgfscope}%
\begin{pgfscope}%
\pgfpathrectangle{\pgfqpoint{1.254980in}{0.150000in}}{\pgfqpoint{5.490039in}{5.490039in}}%
\pgfusepath{clip}%
\pgfsetbuttcap%
\pgfsetroundjoin%
\definecolor{currentfill}{rgb}{0.281446,0.084320,0.407414}%
\pgfsetfillcolor{currentfill}%
\pgfsetfillopacity{0.700000}%
\pgfsetlinewidth{0.000000pt}%
\definecolor{currentstroke}{rgb}{0.000000,0.000000,0.000000}%
\pgfsetstrokecolor{currentstroke}%
\pgfsetdash{}{0pt}%
\pgfpathmoveto{\pgfqpoint{3.822990in}{1.507810in}}%
\pgfpathlineto{\pgfqpoint{3.836509in}{1.509051in}}%
\pgfpathlineto{\pgfqpoint{3.850036in}{1.510452in}}%
\pgfpathlineto{\pgfqpoint{3.863572in}{1.512013in}}%
\pgfpathlineto{\pgfqpoint{3.877117in}{1.513733in}}%
\pgfpathlineto{\pgfqpoint{3.885037in}{1.525829in}}%
\pgfpathlineto{\pgfqpoint{3.892952in}{1.537986in}}%
\pgfpathlineto{\pgfqpoint{3.900861in}{1.550202in}}%
\pgfpathlineto{\pgfqpoint{3.908766in}{1.562471in}}%
\pgfpathlineto{\pgfqpoint{3.895228in}{1.560277in}}%
\pgfpathlineto{\pgfqpoint{3.881700in}{1.558243in}}%
\pgfpathlineto{\pgfqpoint{3.868180in}{1.556369in}}%
\pgfpathlineto{\pgfqpoint{3.854669in}{1.554656in}}%
\pgfpathlineto{\pgfqpoint{3.846757in}{1.542849in}}%
\pgfpathlineto{\pgfqpoint{3.838840in}{1.531103in}}%
\pgfpathlineto{\pgfqpoint{3.830917in}{1.519422in}}%
\pgfpathlineto{\pgfqpoint{3.822990in}{1.507810in}}%
\pgfpathclose%
\pgfusepath{fill}%
\end{pgfscope}%
\begin{pgfscope}%
\pgfpathrectangle{\pgfqpoint{1.254980in}{0.150000in}}{\pgfqpoint{5.490039in}{5.490039in}}%
\pgfusepath{clip}%
\pgfsetbuttcap%
\pgfsetroundjoin%
\definecolor{currentfill}{rgb}{0.210503,0.363727,0.552206}%
\pgfsetfillcolor{currentfill}%
\pgfsetfillopacity{0.700000}%
\pgfsetlinewidth{0.000000pt}%
\definecolor{currentstroke}{rgb}{0.000000,0.000000,0.000000}%
\pgfsetstrokecolor{currentstroke}%
\pgfsetdash{}{0pt}%
\pgfpathmoveto{\pgfqpoint{4.431927in}{2.093836in}}%
\pgfpathlineto{\pgfqpoint{4.445699in}{2.102187in}}%
\pgfpathlineto{\pgfqpoint{4.459485in}{2.110698in}}%
\pgfpathlineto{\pgfqpoint{4.473286in}{2.119369in}}%
\pgfpathlineto{\pgfqpoint{4.487101in}{2.128201in}}%
\pgfpathlineto{\pgfqpoint{4.494850in}{2.141621in}}%
\pgfpathlineto{\pgfqpoint{4.502595in}{2.154944in}}%
\pgfpathlineto{\pgfqpoint{4.510334in}{2.168168in}}%
\pgfpathlineto{\pgfqpoint{4.518068in}{2.181291in}}%
\pgfpathlineto{\pgfqpoint{4.504251in}{2.172234in}}%
\pgfpathlineto{\pgfqpoint{4.490449in}{2.163338in}}%
\pgfpathlineto{\pgfqpoint{4.476662in}{2.154602in}}%
\pgfpathlineto{\pgfqpoint{4.462889in}{2.146027in}}%
\pgfpathlineto{\pgfqpoint{4.455156in}{2.133118in}}%
\pgfpathlineto{\pgfqpoint{4.447418in}{2.120115in}}%
\pgfpathlineto{\pgfqpoint{4.439675in}{2.107021in}}%
\pgfpathlineto{\pgfqpoint{4.431927in}{2.093836in}}%
\pgfpathclose%
\pgfusepath{fill}%
\end{pgfscope}%
\begin{pgfscope}%
\pgfpathrectangle{\pgfqpoint{1.254980in}{0.150000in}}{\pgfqpoint{5.490039in}{5.490039in}}%
\pgfusepath{clip}%
\pgfsetbuttcap%
\pgfsetroundjoin%
\definecolor{currentfill}{rgb}{0.121380,0.629492,0.531973}%
\pgfsetfillcolor{currentfill}%
\pgfsetfillopacity{0.700000}%
\pgfsetlinewidth{0.000000pt}%
\definecolor{currentstroke}{rgb}{0.000000,0.000000,0.000000}%
\pgfsetstrokecolor{currentstroke}%
\pgfsetdash{}{0pt}%
\pgfpathmoveto{\pgfqpoint{5.072204in}{2.805945in}}%
\pgfpathlineto{\pgfqpoint{5.086369in}{2.819091in}}%
\pgfpathlineto{\pgfqpoint{5.100554in}{2.832401in}}%
\pgfpathlineto{\pgfqpoint{5.114757in}{2.845873in}}%
\pgfpathlineto{\pgfqpoint{5.128979in}{2.859510in}}%
\pgfpathlineto{\pgfqpoint{5.136456in}{2.867912in}}%
\pgfpathlineto{\pgfqpoint{5.143924in}{2.876153in}}%
\pgfpathlineto{\pgfqpoint{5.151384in}{2.884236in}}%
\pgfpathlineto{\pgfqpoint{5.158834in}{2.892161in}}%
\pgfpathlineto{\pgfqpoint{5.144618in}{2.878602in}}%
\pgfpathlineto{\pgfqpoint{5.130420in}{2.865206in}}%
\pgfpathlineto{\pgfqpoint{5.116241in}{2.851973in}}%
\pgfpathlineto{\pgfqpoint{5.102082in}{2.838903in}}%
\pgfpathlineto{\pgfqpoint{5.094625in}{2.830890in}}%
\pgfpathlineto{\pgfqpoint{5.087160in}{2.822727in}}%
\pgfpathlineto{\pgfqpoint{5.079686in}{2.814412in}}%
\pgfpathlineto{\pgfqpoint{5.072204in}{2.805945in}}%
\pgfpathclose%
\pgfusepath{fill}%
\end{pgfscope}%
\begin{pgfscope}%
\pgfpathrectangle{\pgfqpoint{1.254980in}{0.150000in}}{\pgfqpoint{5.490039in}{5.490039in}}%
\pgfusepath{clip}%
\pgfsetbuttcap%
\pgfsetroundjoin%
\definecolor{currentfill}{rgb}{0.271828,0.209303,0.504434}%
\pgfsetfillcolor{currentfill}%
\pgfsetfillopacity{0.700000}%
\pgfsetlinewidth{0.000000pt}%
\definecolor{currentstroke}{rgb}{0.000000,0.000000,0.000000}%
\pgfsetstrokecolor{currentstroke}%
\pgfsetdash{}{0pt}%
\pgfpathmoveto{\pgfqpoint{4.111806in}{1.745691in}}%
\pgfpathlineto{\pgfqpoint{4.125424in}{1.750578in}}%
\pgfpathlineto{\pgfqpoint{4.139054in}{1.755624in}}%
\pgfpathlineto{\pgfqpoint{4.152695in}{1.760830in}}%
\pgfpathlineto{\pgfqpoint{4.166347in}{1.766194in}}%
\pgfpathlineto{\pgfqpoint{4.174184in}{1.779910in}}%
\pgfpathlineto{\pgfqpoint{4.182016in}{1.793602in}}%
\pgfpathlineto{\pgfqpoint{4.189844in}{1.807270in}}%
\pgfpathlineto{\pgfqpoint{4.197668in}{1.820908in}}%
\pgfpathlineto{\pgfqpoint{4.184016in}{1.815178in}}%
\pgfpathlineto{\pgfqpoint{4.170376in}{1.809607in}}%
\pgfpathlineto{\pgfqpoint{4.156748in}{1.804196in}}%
\pgfpathlineto{\pgfqpoint{4.143131in}{1.798945in}}%
\pgfpathlineto{\pgfqpoint{4.135306in}{1.785660in}}%
\pgfpathlineto{\pgfqpoint{4.127477in}{1.772355in}}%
\pgfpathlineto{\pgfqpoint{4.119644in}{1.759030in}}%
\pgfpathlineto{\pgfqpoint{4.111806in}{1.745691in}}%
\pgfpathclose%
\pgfusepath{fill}%
\end{pgfscope}%
\begin{pgfscope}%
\pgfpathrectangle{\pgfqpoint{1.254980in}{0.150000in}}{\pgfqpoint{5.490039in}{5.490039in}}%
\pgfusepath{clip}%
\pgfsetbuttcap%
\pgfsetroundjoin%
\definecolor{currentfill}{rgb}{0.352360,0.783011,0.392636}%
\pgfsetfillcolor{currentfill}%
\pgfsetfillopacity{0.700000}%
\pgfsetlinewidth{0.000000pt}%
\definecolor{currentstroke}{rgb}{0.000000,0.000000,0.000000}%
\pgfsetstrokecolor{currentstroke}%
\pgfsetdash{}{0pt}%
\pgfpathmoveto{\pgfqpoint{5.564117in}{3.260922in}}%
\pgfpathlineto{\pgfqpoint{5.578614in}{3.276139in}}%
\pgfpathlineto{\pgfqpoint{5.593133in}{3.291520in}}%
\pgfpathlineto{\pgfqpoint{5.607673in}{3.307066in}}%
\pgfpathlineto{\pgfqpoint{5.622236in}{3.322777in}}%
\pgfpathlineto{\pgfqpoint{5.629382in}{3.325967in}}%
\pgfpathlineto{\pgfqpoint{5.636518in}{3.329022in}}%
\pgfpathlineto{\pgfqpoint{5.643643in}{3.331944in}}%
\pgfpathlineto{\pgfqpoint{5.650757in}{3.334737in}}%
\pgfpathlineto{\pgfqpoint{5.636213in}{3.319328in}}%
\pgfpathlineto{\pgfqpoint{5.621691in}{3.304083in}}%
\pgfpathlineto{\pgfqpoint{5.607190in}{3.289001in}}%
\pgfpathlineto{\pgfqpoint{5.592711in}{3.274082in}}%
\pgfpathlineto{\pgfqpoint{5.585578in}{3.270978in}}%
\pgfpathlineto{\pgfqpoint{5.578434in}{3.267752in}}%
\pgfpathlineto{\pgfqpoint{5.571281in}{3.264401in}}%
\pgfpathlineto{\pgfqpoint{5.564117in}{3.260922in}}%
\pgfpathclose%
\pgfusepath{fill}%
\end{pgfscope}%
\begin{pgfscope}%
\pgfpathrectangle{\pgfqpoint{1.254980in}{0.150000in}}{\pgfqpoint{5.490039in}{5.490039in}}%
\pgfusepath{clip}%
\pgfsetbuttcap%
\pgfsetroundjoin%
\definecolor{currentfill}{rgb}{0.124395,0.578002,0.548287}%
\pgfsetfillcolor{currentfill}%
\pgfsetfillopacity{0.700000}%
\pgfsetlinewidth{0.000000pt}%
\definecolor{currentstroke}{rgb}{0.000000,0.000000,0.000000}%
\pgfsetstrokecolor{currentstroke}%
\pgfsetdash{}{0pt}%
\pgfpathmoveto{\pgfqpoint{2.052856in}{2.781537in}}%
\pgfpathlineto{\pgfqpoint{2.066911in}{2.754898in}}%
\pgfpathlineto{\pgfqpoint{2.080948in}{2.728573in}}%
\pgfpathlineto{\pgfqpoint{2.094968in}{2.702559in}}%
\pgfpathlineto{\pgfqpoint{2.108971in}{2.676852in}}%
\pgfpathlineto{\pgfqpoint{2.118275in}{2.666407in}}%
\pgfpathlineto{\pgfqpoint{2.127549in}{2.656388in}}%
\pgfpathlineto{\pgfqpoint{2.136795in}{2.646787in}}%
\pgfpathlineto{\pgfqpoint{2.146012in}{2.637598in}}%
\pgfpathlineto{\pgfqpoint{2.132081in}{2.662582in}}%
\pgfpathlineto{\pgfqpoint{2.118134in}{2.687872in}}%
\pgfpathlineto{\pgfqpoint{2.104170in}{2.713472in}}%
\pgfpathlineto{\pgfqpoint{2.090189in}{2.739383in}}%
\pgfpathlineto{\pgfqpoint{2.080900in}{2.749282in}}%
\pgfpathlineto{\pgfqpoint{2.071582in}{2.759603in}}%
\pgfpathlineto{\pgfqpoint{2.062234in}{2.770352in}}%
\pgfpathlineto{\pgfqpoint{2.052856in}{2.781537in}}%
\pgfpathclose%
\pgfusepath{fill}%
\end{pgfscope}%
\begin{pgfscope}%
\pgfpathrectangle{\pgfqpoint{1.254980in}{0.150000in}}{\pgfqpoint{5.490039in}{5.490039in}}%
\pgfusepath{clip}%
\pgfsetbuttcap%
\pgfsetroundjoin%
\definecolor{currentfill}{rgb}{0.269944,0.014625,0.341379}%
\pgfsetfillcolor{currentfill}%
\pgfsetfillopacity{0.700000}%
\pgfsetlinewidth{0.000000pt}%
\definecolor{currentstroke}{rgb}{0.000000,0.000000,0.000000}%
\pgfsetstrokecolor{currentstroke}%
\pgfsetdash{}{0pt}%
\pgfpathmoveto{\pgfqpoint{3.565345in}{1.392272in}}%
\pgfpathlineto{\pgfqpoint{3.578817in}{1.389991in}}%
\pgfpathlineto{\pgfqpoint{3.592295in}{1.387874in}}%
\pgfpathlineto{\pgfqpoint{3.605779in}{1.385920in}}%
\pgfpathlineto{\pgfqpoint{3.619269in}{1.384128in}}%
\pgfpathlineto{\pgfqpoint{3.627294in}{1.393400in}}%
\pgfpathlineto{\pgfqpoint{3.635312in}{1.402815in}}%
\pgfpathlineto{\pgfqpoint{3.643323in}{1.412367in}}%
\pgfpathlineto{\pgfqpoint{3.651327in}{1.422052in}}%
\pgfpathlineto{\pgfqpoint{3.637852in}{1.423290in}}%
\pgfpathlineto{\pgfqpoint{3.624384in}{1.424690in}}%
\pgfpathlineto{\pgfqpoint{3.610922in}{1.426253in}}%
\pgfpathlineto{\pgfqpoint{3.597466in}{1.427980in}}%
\pgfpathlineto{\pgfqpoint{3.589447in}{1.418838in}}%
\pgfpathlineto{\pgfqpoint{3.581420in}{1.409836in}}%
\pgfpathlineto{\pgfqpoint{3.573386in}{1.400979in}}%
\pgfpathlineto{\pgfqpoint{3.565345in}{1.392272in}}%
\pgfpathclose%
\pgfusepath{fill}%
\end{pgfscope}%
\begin{pgfscope}%
\pgfpathrectangle{\pgfqpoint{1.254980in}{0.150000in}}{\pgfqpoint{5.490039in}{5.490039in}}%
\pgfusepath{clip}%
\pgfsetbuttcap%
\pgfsetroundjoin%
\definecolor{currentfill}{rgb}{0.150476,0.504369,0.557430}%
\pgfsetfillcolor{currentfill}%
\pgfsetfillopacity{0.700000}%
\pgfsetlinewidth{0.000000pt}%
\definecolor{currentstroke}{rgb}{0.000000,0.000000,0.000000}%
\pgfsetstrokecolor{currentstroke}%
\pgfsetdash{}{0pt}%
\pgfpathmoveto{\pgfqpoint{4.752188in}{2.459399in}}%
\pgfpathlineto{\pgfqpoint{4.766151in}{2.470514in}}%
\pgfpathlineto{\pgfqpoint{4.780130in}{2.481791in}}%
\pgfpathlineto{\pgfqpoint{4.794126in}{2.493230in}}%
\pgfpathlineto{\pgfqpoint{4.808140in}{2.504831in}}%
\pgfpathlineto{\pgfqpoint{4.815777in}{2.516292in}}%
\pgfpathlineto{\pgfqpoint{4.823408in}{2.527610in}}%
\pgfpathlineto{\pgfqpoint{4.831032in}{2.538782in}}%
\pgfpathlineto{\pgfqpoint{4.838649in}{2.549811in}}%
\pgfpathlineto{\pgfqpoint{4.824636in}{2.538132in}}%
\pgfpathlineto{\pgfqpoint{4.810640in}{2.526616in}}%
\pgfpathlineto{\pgfqpoint{4.796661in}{2.515262in}}%
\pgfpathlineto{\pgfqpoint{4.782700in}{2.504069in}}%
\pgfpathlineto{\pgfqpoint{4.775082in}{2.493107in}}%
\pgfpathlineto{\pgfqpoint{4.767457in}{2.482007in}}%
\pgfpathlineto{\pgfqpoint{4.759826in}{2.470771in}}%
\pgfpathlineto{\pgfqpoint{4.752188in}{2.459399in}}%
\pgfpathclose%
\pgfusepath{fill}%
\end{pgfscope}%
\begin{pgfscope}%
\pgfpathrectangle{\pgfqpoint{1.254980in}{0.150000in}}{\pgfqpoint{5.490039in}{5.490039in}}%
\pgfusepath{clip}%
\pgfsetbuttcap%
\pgfsetroundjoin%
\definecolor{currentfill}{rgb}{0.283229,0.120777,0.440584}%
\pgfsetfillcolor{currentfill}%
\pgfsetfillopacity{0.700000}%
\pgfsetlinewidth{0.000000pt}%
\definecolor{currentstroke}{rgb}{0.000000,0.000000,0.000000}%
\pgfsetstrokecolor{currentstroke}%
\pgfsetdash{}{0pt}%
\pgfpathmoveto{\pgfqpoint{3.908766in}{1.562471in}}%
\pgfpathlineto{\pgfqpoint{3.922313in}{1.564825in}}%
\pgfpathlineto{\pgfqpoint{3.935870in}{1.567338in}}%
\pgfpathlineto{\pgfqpoint{3.949436in}{1.570010in}}%
\pgfpathlineto{\pgfqpoint{3.963011in}{1.572842in}}%
\pgfpathlineto{\pgfqpoint{3.970906in}{1.585617in}}%
\pgfpathlineto{\pgfqpoint{3.978796in}{1.598430in}}%
\pgfpathlineto{\pgfqpoint{3.986681in}{1.611276in}}%
\pgfpathlineto{\pgfqpoint{3.994562in}{1.624151in}}%
\pgfpathlineto{\pgfqpoint{3.980991in}{1.620872in}}%
\pgfpathlineto{\pgfqpoint{3.967430in}{1.617753in}}%
\pgfpathlineto{\pgfqpoint{3.953879in}{1.614794in}}%
\pgfpathlineto{\pgfqpoint{3.940337in}{1.611994in}}%
\pgfpathlineto{\pgfqpoint{3.932452in}{1.599555in}}%
\pgfpathlineto{\pgfqpoint{3.924561in}{1.587152in}}%
\pgfpathlineto{\pgfqpoint{3.916666in}{1.574789in}}%
\pgfpathlineto{\pgfqpoint{3.908766in}{1.562471in}}%
\pgfpathclose%
\pgfusepath{fill}%
\end{pgfscope}%
\begin{pgfscope}%
\pgfpathrectangle{\pgfqpoint{1.254980in}{0.150000in}}{\pgfqpoint{5.490039in}{5.490039in}}%
\pgfusepath{clip}%
\pgfsetbuttcap%
\pgfsetroundjoin%
\definecolor{currentfill}{rgb}{0.282327,0.094955,0.417331}%
\pgfsetfillcolor{currentfill}%
\pgfsetfillopacity{0.700000}%
\pgfsetlinewidth{0.000000pt}%
\definecolor{currentstroke}{rgb}{0.000000,0.000000,0.000000}%
\pgfsetstrokecolor{currentstroke}%
\pgfsetdash{}{0pt}%
\pgfpathmoveto{\pgfqpoint{2.948929in}{1.563655in}}%
\pgfpathlineto{\pgfqpoint{2.962442in}{1.552598in}}%
\pgfpathlineto{\pgfqpoint{2.975953in}{1.541729in}}%
\pgfpathlineto{\pgfqpoint{2.989463in}{1.531047in}}%
\pgfpathlineto{\pgfqpoint{3.002972in}{1.520550in}}%
\pgfpathlineto{\pgfqpoint{3.011416in}{1.520592in}}%
\pgfpathlineto{\pgfqpoint{3.019844in}{1.520943in}}%
\pgfpathlineto{\pgfqpoint{3.028256in}{1.521596in}}%
\pgfpathlineto{\pgfqpoint{3.036654in}{1.522542in}}%
\pgfpathlineto{\pgfqpoint{3.023184in}{1.532362in}}%
\pgfpathlineto{\pgfqpoint{3.009714in}{1.542367in}}%
\pgfpathlineto{\pgfqpoint{2.996242in}{1.552559in}}%
\pgfpathlineto{\pgfqpoint{2.982770in}{1.562938in}}%
\pgfpathlineto{\pgfqpoint{2.974333in}{1.562657in}}%
\pgfpathlineto{\pgfqpoint{2.965881in}{1.562678in}}%
\pgfpathlineto{\pgfqpoint{2.957413in}{1.563008in}}%
\pgfpathlineto{\pgfqpoint{2.948929in}{1.563655in}}%
\pgfpathclose%
\pgfusepath{fill}%
\end{pgfscope}%
\begin{pgfscope}%
\pgfpathrectangle{\pgfqpoint{1.254980in}{0.150000in}}{\pgfqpoint{5.490039in}{5.490039in}}%
\pgfusepath{clip}%
\pgfsetbuttcap%
\pgfsetroundjoin%
\definecolor{currentfill}{rgb}{0.235526,0.309527,0.542944}%
\pgfsetfillcolor{currentfill}%
\pgfsetfillopacity{0.700000}%
\pgfsetlinewidth{0.000000pt}%
\definecolor{currentstroke}{rgb}{0.000000,0.000000,0.000000}%
\pgfsetstrokecolor{currentstroke}%
\pgfsetdash{}{0pt}%
\pgfpathmoveto{\pgfqpoint{4.314832in}{1.955711in}}%
\pgfpathlineto{\pgfqpoint{4.328547in}{1.962886in}}%
\pgfpathlineto{\pgfqpoint{4.342275in}{1.970222in}}%
\pgfpathlineto{\pgfqpoint{4.356017in}{1.977717in}}%
\pgfpathlineto{\pgfqpoint{4.369773in}{1.985371in}}%
\pgfpathlineto{\pgfqpoint{4.377558in}{1.999199in}}%
\pgfpathlineto{\pgfqpoint{4.385339in}{2.012954in}}%
\pgfpathlineto{\pgfqpoint{4.393115in}{2.026635in}}%
\pgfpathlineto{\pgfqpoint{4.400887in}{2.040238in}}%
\pgfpathlineto{\pgfqpoint{4.387130in}{2.032301in}}%
\pgfpathlineto{\pgfqpoint{4.373387in}{2.024524in}}%
\pgfpathlineto{\pgfqpoint{4.359658in}{2.016907in}}%
\pgfpathlineto{\pgfqpoint{4.345942in}{2.009450in}}%
\pgfpathlineto{\pgfqpoint{4.338171in}{1.996118in}}%
\pgfpathlineto{\pgfqpoint{4.330396in}{1.982715in}}%
\pgfpathlineto{\pgfqpoint{4.322616in}{1.969246in}}%
\pgfpathlineto{\pgfqpoint{4.314832in}{1.955711in}}%
\pgfpathclose%
\pgfusepath{fill}%
\end{pgfscope}%
\begin{pgfscope}%
\pgfpathrectangle{\pgfqpoint{1.254980in}{0.150000in}}{\pgfqpoint{5.490039in}{5.490039in}}%
\pgfusepath{clip}%
\pgfsetbuttcap%
\pgfsetroundjoin%
\definecolor{currentfill}{rgb}{0.267004,0.004874,0.329415}%
\pgfsetfillcolor{currentfill}%
\pgfsetfillopacity{0.700000}%
\pgfsetlinewidth{0.000000pt}%
\definecolor{currentstroke}{rgb}{0.000000,0.000000,0.000000}%
\pgfsetstrokecolor{currentstroke}%
\pgfsetdash{}{0pt}%
\pgfpathmoveto{\pgfqpoint{3.338937in}{1.383273in}}%
\pgfpathlineto{\pgfqpoint{3.352400in}{1.377805in}}%
\pgfpathlineto{\pgfqpoint{3.365866in}{1.372505in}}%
\pgfpathlineto{\pgfqpoint{3.379335in}{1.367375in}}%
\pgfpathlineto{\pgfqpoint{3.392807in}{1.362412in}}%
\pgfpathlineto{\pgfqpoint{3.400959in}{1.368471in}}%
\pgfpathlineto{\pgfqpoint{3.409100in}{1.374740in}}%
\pgfpathlineto{\pgfqpoint{3.417232in}{1.381215in}}%
\pgfpathlineto{\pgfqpoint{3.425355in}{1.387888in}}%
\pgfpathlineto{\pgfqpoint{3.411906in}{1.392240in}}%
\pgfpathlineto{\pgfqpoint{3.398461in}{1.396760in}}%
\pgfpathlineto{\pgfqpoint{3.385019in}{1.401449in}}%
\pgfpathlineto{\pgfqpoint{3.371581in}{1.406307in}}%
\pgfpathlineto{\pgfqpoint{3.363435in}{1.400233in}}%
\pgfpathlineto{\pgfqpoint{3.355279in}{1.394366in}}%
\pgfpathlineto{\pgfqpoint{3.347113in}{1.388710in}}%
\pgfpathlineto{\pgfqpoint{3.338937in}{1.383273in}}%
\pgfpathclose%
\pgfusepath{fill}%
\end{pgfscope}%
\begin{pgfscope}%
\pgfpathrectangle{\pgfqpoint{1.254980in}{0.150000in}}{\pgfqpoint{5.490039in}{5.490039in}}%
\pgfusepath{clip}%
\pgfsetbuttcap%
\pgfsetroundjoin%
\definecolor{currentfill}{rgb}{0.169646,0.456262,0.558030}%
\pgfsetfillcolor{currentfill}%
\pgfsetfillopacity{0.700000}%
\pgfsetlinewidth{0.000000pt}%
\definecolor{currentstroke}{rgb}{0.000000,0.000000,0.000000}%
\pgfsetstrokecolor{currentstroke}%
\pgfsetdash{}{0pt}%
\pgfpathmoveto{\pgfqpoint{2.239143in}{2.428546in}}%
\pgfpathlineto{\pgfqpoint{2.253022in}{2.405845in}}%
\pgfpathlineto{\pgfqpoint{2.266888in}{2.383415in}}%
\pgfpathlineto{\pgfqpoint{2.280741in}{2.361253in}}%
\pgfpathlineto{\pgfqpoint{2.294581in}{2.339358in}}%
\pgfpathlineto{\pgfqpoint{2.303717in}{2.330162in}}%
\pgfpathlineto{\pgfqpoint{2.312826in}{2.321385in}}%
\pgfpathlineto{\pgfqpoint{2.321908in}{2.313019in}}%
\pgfpathlineto{\pgfqpoint{2.330964in}{2.305058in}}%
\pgfpathlineto{\pgfqpoint{2.317190in}{2.326223in}}%
\pgfpathlineto{\pgfqpoint{2.303404in}{2.347653in}}%
\pgfpathlineto{\pgfqpoint{2.289606in}{2.369350in}}%
\pgfpathlineto{\pgfqpoint{2.275795in}{2.391316in}}%
\pgfpathlineto{\pgfqpoint{2.266674in}{2.399997in}}%
\pgfpathlineto{\pgfqpoint{2.257525in}{2.409090in}}%
\pgfpathlineto{\pgfqpoint{2.248348in}{2.418604in}}%
\pgfpathlineto{\pgfqpoint{2.239143in}{2.428546in}}%
\pgfpathclose%
\pgfusepath{fill}%
\end{pgfscope}%
\begin{pgfscope}%
\pgfpathrectangle{\pgfqpoint{1.254980in}{0.150000in}}{\pgfqpoint{5.490039in}{5.490039in}}%
\pgfusepath{clip}%
\pgfsetbuttcap%
\pgfsetroundjoin%
\definecolor{currentfill}{rgb}{0.271305,0.019942,0.347269}%
\pgfsetfillcolor{currentfill}%
\pgfsetfillopacity{0.700000}%
\pgfsetlinewidth{0.000000pt}%
\definecolor{currentstroke}{rgb}{0.000000,0.000000,0.000000}%
\pgfsetstrokecolor{currentstroke}%
\pgfsetdash{}{0pt}%
\pgfpathmoveto{\pgfqpoint{3.198284in}{1.418862in}}%
\pgfpathlineto{\pgfqpoint{3.211758in}{1.411377in}}%
\pgfpathlineto{\pgfqpoint{3.225233in}{1.404067in}}%
\pgfpathlineto{\pgfqpoint{3.238710in}{1.396930in}}%
\pgfpathlineto{\pgfqpoint{3.252189in}{1.389967in}}%
\pgfpathlineto{\pgfqpoint{3.260438in}{1.393818in}}%
\pgfpathlineto{\pgfqpoint{3.268675in}{1.397920in}}%
\pgfpathlineto{\pgfqpoint{3.276900in}{1.402266in}}%
\pgfpathlineto{\pgfqpoint{3.285114in}{1.406850in}}%
\pgfpathlineto{\pgfqpoint{3.271665in}{1.413173in}}%
\pgfpathlineto{\pgfqpoint{3.258218in}{1.419668in}}%
\pgfpathlineto{\pgfqpoint{3.244772in}{1.426337in}}%
\pgfpathlineto{\pgfqpoint{3.231329in}{1.433180in}}%
\pgfpathlineto{\pgfqpoint{3.223086in}{1.429226in}}%
\pgfpathlineto{\pgfqpoint{3.214831in}{1.425518in}}%
\pgfpathlineto{\pgfqpoint{3.206564in}{1.422061in}}%
\pgfpathlineto{\pgfqpoint{3.198284in}{1.418862in}}%
\pgfpathclose%
\pgfusepath{fill}%
\end{pgfscope}%
\begin{pgfscope}%
\pgfpathrectangle{\pgfqpoint{1.254980in}{0.150000in}}{\pgfqpoint{5.490039in}{5.490039in}}%
\pgfusepath{clip}%
\pgfsetbuttcap%
\pgfsetroundjoin%
\definecolor{currentfill}{rgb}{0.421908,0.805774,0.351910}%
\pgfsetfillcolor{currentfill}%
\pgfsetfillopacity{0.700000}%
\pgfsetlinewidth{0.000000pt}%
\definecolor{currentstroke}{rgb}{0.000000,0.000000,0.000000}%
\pgfsetstrokecolor{currentstroke}%
\pgfsetdash{}{0pt}%
\pgfpathmoveto{\pgfqpoint{5.650757in}{3.334737in}}%
\pgfpathlineto{\pgfqpoint{5.665323in}{3.350311in}}%
\pgfpathlineto{\pgfqpoint{5.679912in}{3.366049in}}%
\pgfpathlineto{\pgfqpoint{5.694522in}{3.381952in}}%
\pgfpathlineto{\pgfqpoint{5.709155in}{3.398020in}}%
\pgfpathlineto{\pgfqpoint{5.716239in}{3.400366in}}%
\pgfpathlineto{\pgfqpoint{5.723313in}{3.402582in}}%
\pgfpathlineto{\pgfqpoint{5.730375in}{3.404673in}}%
\pgfpathlineto{\pgfqpoint{5.737427in}{3.406641in}}%
\pgfpathlineto{\pgfqpoint{5.722815in}{3.390908in}}%
\pgfpathlineto{\pgfqpoint{5.708225in}{3.375339in}}%
\pgfpathlineto{\pgfqpoint{5.693657in}{3.359933in}}%
\pgfpathlineto{\pgfqpoint{5.679111in}{3.344691in}}%
\pgfpathlineto{\pgfqpoint{5.672038in}{3.342378in}}%
\pgfpathlineto{\pgfqpoint{5.664955in}{3.339951in}}%
\pgfpathlineto{\pgfqpoint{5.657861in}{3.337405in}}%
\pgfpathlineto{\pgfqpoint{5.650757in}{3.334737in}}%
\pgfpathclose%
\pgfusepath{fill}%
\end{pgfscope}%
\begin{pgfscope}%
\pgfpathrectangle{\pgfqpoint{1.254980in}{0.150000in}}{\pgfqpoint{5.490039in}{5.490039in}}%
\pgfusepath{clip}%
\pgfsetbuttcap%
\pgfsetroundjoin%
\definecolor{currentfill}{rgb}{0.169646,0.456262,0.558030}%
\pgfsetfillcolor{currentfill}%
\pgfsetfillopacity{0.700000}%
\pgfsetlinewidth{0.000000pt}%
\definecolor{currentstroke}{rgb}{0.000000,0.000000,0.000000}%
\pgfsetstrokecolor{currentstroke}%
\pgfsetdash{}{0pt}%
\pgfpathmoveto{\pgfqpoint{4.635179in}{2.321722in}}%
\pgfpathlineto{\pgfqpoint{4.649076in}{2.331946in}}%
\pgfpathlineto{\pgfqpoint{4.662988in}{2.342332in}}%
\pgfpathlineto{\pgfqpoint{4.676916in}{2.352879in}}%
\pgfpathlineto{\pgfqpoint{4.690860in}{2.363587in}}%
\pgfpathlineto{\pgfqpoint{4.698548in}{2.376026in}}%
\pgfpathlineto{\pgfqpoint{4.706229in}{2.388335in}}%
\pgfpathlineto{\pgfqpoint{4.713905in}{2.400512in}}%
\pgfpathlineto{\pgfqpoint{4.721574in}{2.412557in}}%
\pgfpathlineto{\pgfqpoint{4.707628in}{2.401711in}}%
\pgfpathlineto{\pgfqpoint{4.693699in}{2.391027in}}%
\pgfpathlineto{\pgfqpoint{4.679786in}{2.380504in}}%
\pgfpathlineto{\pgfqpoint{4.665889in}{2.370143in}}%
\pgfpathlineto{\pgfqpoint{4.658221in}{2.358224in}}%
\pgfpathlineto{\pgfqpoint{4.650546in}{2.346180in}}%
\pgfpathlineto{\pgfqpoint{4.642866in}{2.334013in}}%
\pgfpathlineto{\pgfqpoint{4.635179in}{2.321722in}}%
\pgfpathclose%
\pgfusepath{fill}%
\end{pgfscope}%
\begin{pgfscope}%
\pgfpathrectangle{\pgfqpoint{1.254980in}{0.150000in}}{\pgfqpoint{5.490039in}{5.490039in}}%
\pgfusepath{clip}%
\pgfsetbuttcap%
\pgfsetroundjoin%
\definecolor{currentfill}{rgb}{0.267004,0.004874,0.329415}%
\pgfsetfillcolor{currentfill}%
\pgfsetfillopacity{0.700000}%
\pgfsetlinewidth{0.000000pt}%
\definecolor{currentstroke}{rgb}{0.000000,0.000000,0.000000}%
\pgfsetstrokecolor{currentstroke}%
\pgfsetdash{}{0pt}%
\pgfpathmoveto{\pgfqpoint{3.479192in}{1.372148in}}%
\pgfpathlineto{\pgfqpoint{3.492663in}{1.368628in}}%
\pgfpathlineto{\pgfqpoint{3.506138in}{1.365273in}}%
\pgfpathlineto{\pgfqpoint{3.519617in}{1.362083in}}%
\pgfpathlineto{\pgfqpoint{3.533102in}{1.359056in}}%
\pgfpathlineto{\pgfqpoint{3.541175in}{1.367107in}}%
\pgfpathlineto{\pgfqpoint{3.549240in}{1.375330in}}%
\pgfpathlineto{\pgfqpoint{3.557296in}{1.383721in}}%
\pgfpathlineto{\pgfqpoint{3.565345in}{1.392272in}}%
\pgfpathlineto{\pgfqpoint{3.551879in}{1.394716in}}%
\pgfpathlineto{\pgfqpoint{3.538417in}{1.397325in}}%
\pgfpathlineto{\pgfqpoint{3.524961in}{1.400098in}}%
\pgfpathlineto{\pgfqpoint{3.511510in}{1.403036in}}%
\pgfpathlineto{\pgfqpoint{3.503443in}{1.395056in}}%
\pgfpathlineto{\pgfqpoint{3.495368in}{1.387244in}}%
\pgfpathlineto{\pgfqpoint{3.487284in}{1.379606in}}%
\pgfpathlineto{\pgfqpoint{3.479192in}{1.372148in}}%
\pgfpathclose%
\pgfusepath{fill}%
\end{pgfscope}%
\begin{pgfscope}%
\pgfpathrectangle{\pgfqpoint{1.254980in}{0.150000in}}{\pgfqpoint{5.490039in}{5.490039in}}%
\pgfusepath{clip}%
\pgfsetbuttcap%
\pgfsetroundjoin%
\definecolor{currentfill}{rgb}{0.281412,0.155834,0.469201}%
\pgfsetfillcolor{currentfill}%
\pgfsetfillopacity{0.700000}%
\pgfsetlinewidth{0.000000pt}%
\definecolor{currentstroke}{rgb}{0.000000,0.000000,0.000000}%
\pgfsetstrokecolor{currentstroke}%
\pgfsetdash{}{0pt}%
\pgfpathmoveto{\pgfqpoint{3.994562in}{1.624151in}}%
\pgfpathlineto{\pgfqpoint{4.008142in}{1.627589in}}%
\pgfpathlineto{\pgfqpoint{4.021734in}{1.631186in}}%
\pgfpathlineto{\pgfqpoint{4.035335in}{1.634942in}}%
\pgfpathlineto{\pgfqpoint{4.048947in}{1.638857in}}%
\pgfpathlineto{\pgfqpoint{4.056820in}{1.652188in}}%
\pgfpathlineto{\pgfqpoint{4.064688in}{1.665533in}}%
\pgfpathlineto{\pgfqpoint{4.072552in}{1.678889in}}%
\pgfpathlineto{\pgfqpoint{4.080411in}{1.692251in}}%
\pgfpathlineto{\pgfqpoint{4.066802in}{1.687916in}}%
\pgfpathlineto{\pgfqpoint{4.053204in}{1.683740in}}%
\pgfpathlineto{\pgfqpoint{4.039616in}{1.679723in}}%
\pgfpathlineto{\pgfqpoint{4.026039in}{1.675866in}}%
\pgfpathlineto{\pgfqpoint{4.018176in}{1.662913in}}%
\pgfpathlineto{\pgfqpoint{4.010309in}{1.649974in}}%
\pgfpathlineto{\pgfqpoint{4.002438in}{1.637052in}}%
\pgfpathlineto{\pgfqpoint{3.994562in}{1.624151in}}%
\pgfpathclose%
\pgfusepath{fill}%
\end{pgfscope}%
\begin{pgfscope}%
\pgfpathrectangle{\pgfqpoint{1.254980in}{0.150000in}}{\pgfqpoint{5.490039in}{5.490039in}}%
\pgfusepath{clip}%
\pgfsetbuttcap%
\pgfsetroundjoin%
\definecolor{currentfill}{rgb}{0.232815,0.732247,0.459277}%
\pgfsetfillcolor{currentfill}%
\pgfsetfillopacity{0.700000}%
\pgfsetlinewidth{0.000000pt}%
\definecolor{currentstroke}{rgb}{0.000000,0.000000,0.000000}%
\pgfsetstrokecolor{currentstroke}%
\pgfsetdash{}{0pt}%
\pgfpathmoveto{\pgfqpoint{5.361788in}{3.086525in}}%
\pgfpathlineto{\pgfqpoint{5.376159in}{3.101138in}}%
\pgfpathlineto{\pgfqpoint{5.390550in}{3.115914in}}%
\pgfpathlineto{\pgfqpoint{5.404962in}{3.130855in}}%
\pgfpathlineto{\pgfqpoint{5.419395in}{3.145961in}}%
\pgfpathlineto{\pgfqpoint{5.426696in}{3.151410in}}%
\pgfpathlineto{\pgfqpoint{5.433987in}{3.156705in}}%
\pgfpathlineto{\pgfqpoint{5.441268in}{3.161847in}}%
\pgfpathlineto{\pgfqpoint{5.448539in}{3.166838in}}%
\pgfpathlineto{\pgfqpoint{5.434118in}{3.151938in}}%
\pgfpathlineto{\pgfqpoint{5.419718in}{3.137201in}}%
\pgfpathlineto{\pgfqpoint{5.405339in}{3.122628in}}%
\pgfpathlineto{\pgfqpoint{5.390980in}{3.108219in}}%
\pgfpathlineto{\pgfqpoint{5.383697in}{3.103012in}}%
\pgfpathlineto{\pgfqpoint{5.376404in}{3.097662in}}%
\pgfpathlineto{\pgfqpoint{5.369101in}{3.092168in}}%
\pgfpathlineto{\pgfqpoint{5.361788in}{3.086525in}}%
\pgfpathclose%
\pgfusepath{fill}%
\end{pgfscope}%
\begin{pgfscope}%
\pgfpathrectangle{\pgfqpoint{1.254980in}{0.150000in}}{\pgfqpoint{5.490039in}{5.490039in}}%
\pgfusepath{clip}%
\pgfsetbuttcap%
\pgfsetroundjoin%
\definecolor{currentfill}{rgb}{0.121831,0.589055,0.545623}%
\pgfsetfillcolor{currentfill}%
\pgfsetfillopacity{0.700000}%
\pgfsetlinewidth{0.000000pt}%
\definecolor{currentstroke}{rgb}{0.000000,0.000000,0.000000}%
\pgfsetstrokecolor{currentstroke}%
\pgfsetdash{}{0pt}%
\pgfpathmoveto{\pgfqpoint{4.955548in}{2.681410in}}%
\pgfpathlineto{\pgfqpoint{4.969646in}{2.693964in}}%
\pgfpathlineto{\pgfqpoint{4.983763in}{2.706681in}}%
\pgfpathlineto{\pgfqpoint{4.997898in}{2.719562in}}%
\pgfpathlineto{\pgfqpoint{5.012052in}{2.732605in}}%
\pgfpathlineto{\pgfqpoint{5.019599in}{2.742324in}}%
\pgfpathlineto{\pgfqpoint{5.027139in}{2.751885in}}%
\pgfpathlineto{\pgfqpoint{5.034670in}{2.761287in}}%
\pgfpathlineto{\pgfqpoint{5.042193in}{2.770531in}}%
\pgfpathlineto{\pgfqpoint{5.028043in}{2.757502in}}%
\pgfpathlineto{\pgfqpoint{5.013910in}{2.744637in}}%
\pgfpathlineto{\pgfqpoint{4.999797in}{2.731934in}}%
\pgfpathlineto{\pgfqpoint{4.985701in}{2.719394in}}%
\pgfpathlineto{\pgfqpoint{4.978174in}{2.710124in}}%
\pgfpathlineto{\pgfqpoint{4.970640in}{2.700703in}}%
\pgfpathlineto{\pgfqpoint{4.963098in}{2.691132in}}%
\pgfpathlineto{\pgfqpoint{4.955548in}{2.681410in}}%
\pgfpathclose%
\pgfusepath{fill}%
\end{pgfscope}%
\begin{pgfscope}%
\pgfpathrectangle{\pgfqpoint{1.254980in}{0.150000in}}{\pgfqpoint{5.490039in}{5.490039in}}%
\pgfusepath{clip}%
\pgfsetbuttcap%
\pgfsetroundjoin%
\definecolor{currentfill}{rgb}{0.258965,0.251537,0.524736}%
\pgfsetfillcolor{currentfill}%
\pgfsetfillopacity{0.700000}%
\pgfsetlinewidth{0.000000pt}%
\definecolor{currentstroke}{rgb}{0.000000,0.000000,0.000000}%
\pgfsetstrokecolor{currentstroke}%
\pgfsetdash{}{0pt}%
\pgfpathmoveto{\pgfqpoint{4.197668in}{1.820908in}}%
\pgfpathlineto{\pgfqpoint{4.211332in}{1.826798in}}%
\pgfpathlineto{\pgfqpoint{4.225008in}{1.832847in}}%
\pgfpathlineto{\pgfqpoint{4.238697in}{1.839055in}}%
\pgfpathlineto{\pgfqpoint{4.252398in}{1.845422in}}%
\pgfpathlineto{\pgfqpoint{4.260217in}{1.859377in}}%
\pgfpathlineto{\pgfqpoint{4.268032in}{1.873290in}}%
\pgfpathlineto{\pgfqpoint{4.275843in}{1.887157in}}%
\pgfpathlineto{\pgfqpoint{4.283650in}{1.900976in}}%
\pgfpathlineto{\pgfqpoint{4.269948in}{1.894271in}}%
\pgfpathlineto{\pgfqpoint{4.256259in}{1.887724in}}%
\pgfpathlineto{\pgfqpoint{4.242583in}{1.881338in}}%
\pgfpathlineto{\pgfqpoint{4.228919in}{1.875111in}}%
\pgfpathlineto{\pgfqpoint{4.221113in}{1.861619in}}%
\pgfpathlineto{\pgfqpoint{4.213302in}{1.848086in}}%
\pgfpathlineto{\pgfqpoint{4.205487in}{1.834515in}}%
\pgfpathlineto{\pgfqpoint{4.197668in}{1.820908in}}%
\pgfpathclose%
\pgfusepath{fill}%
\end{pgfscope}%
\begin{pgfscope}%
\pgfpathrectangle{\pgfqpoint{1.254980in}{0.150000in}}{\pgfqpoint{5.490039in}{5.490039in}}%
\pgfusepath{clip}%
\pgfsetbuttcap%
\pgfsetroundjoin%
\definecolor{currentfill}{rgb}{0.477504,0.821444,0.318195}%
\pgfsetfillcolor{currentfill}%
\pgfsetfillopacity{0.700000}%
\pgfsetlinewidth{0.000000pt}%
\definecolor{currentstroke}{rgb}{0.000000,0.000000,0.000000}%
\pgfsetstrokecolor{currentstroke}%
\pgfsetdash{}{0pt}%
\pgfpathmoveto{\pgfqpoint{5.737427in}{3.406641in}}%
\pgfpathlineto{\pgfqpoint{5.752062in}{3.422539in}}%
\pgfpathlineto{\pgfqpoint{5.766719in}{3.438602in}}%
\pgfpathlineto{\pgfqpoint{5.781399in}{3.454830in}}%
\pgfpathlineto{\pgfqpoint{5.788423in}{3.456414in}}%
\pgfpathlineto{\pgfqpoint{5.795437in}{3.457880in}}%
\pgfpathlineto{\pgfqpoint{5.802440in}{3.459229in}}%
\pgfpathlineto{\pgfqpoint{5.809432in}{3.460467in}}%
\pgfpathlineto{\pgfqpoint{5.794775in}{3.444606in}}%
\pgfpathlineto{\pgfqpoint{5.780141in}{3.428909in}}%
\pgfpathlineto{\pgfqpoint{5.765529in}{3.413376in}}%
\pgfpathlineto{\pgfqpoint{5.758519in}{3.411856in}}%
\pgfpathlineto{\pgfqpoint{5.751499in}{3.410229in}}%
\pgfpathlineto{\pgfqpoint{5.744468in}{3.408492in}}%
\pgfpathlineto{\pgfqpoint{5.737427in}{3.406641in}}%
\pgfpathclose%
\pgfusepath{fill}%
\end{pgfscope}%
\begin{pgfscope}%
\pgfpathrectangle{\pgfqpoint{1.254980in}{0.150000in}}{\pgfqpoint{5.490039in}{5.490039in}}%
\pgfusepath{clip}%
\pgfsetbuttcap%
\pgfsetroundjoin%
\definecolor{currentfill}{rgb}{0.280894,0.078907,0.402329}%
\pgfsetfillcolor{currentfill}%
\pgfsetfillopacity{0.700000}%
\pgfsetlinewidth{0.000000pt}%
\definecolor{currentstroke}{rgb}{0.000000,0.000000,0.000000}%
\pgfsetstrokecolor{currentstroke}%
\pgfsetdash{}{0pt}%
\pgfpathmoveto{\pgfqpoint{3.002972in}{1.520550in}}%
\pgfpathlineto{\pgfqpoint{3.016480in}{1.510239in}}%
\pgfpathlineto{\pgfqpoint{3.029986in}{1.500112in}}%
\pgfpathlineto{\pgfqpoint{3.043492in}{1.490168in}}%
\pgfpathlineto{\pgfqpoint{3.056997in}{1.480406in}}%
\pgfpathlineto{\pgfqpoint{3.065402in}{1.481136in}}%
\pgfpathlineto{\pgfqpoint{3.073791in}{1.482166in}}%
\pgfpathlineto{\pgfqpoint{3.082166in}{1.483490in}}%
\pgfpathlineto{\pgfqpoint{3.090527in}{1.485101in}}%
\pgfpathlineto{\pgfqpoint{3.077059in}{1.494187in}}%
\pgfpathlineto{\pgfqpoint{3.063591in}{1.503456in}}%
\pgfpathlineto{\pgfqpoint{3.050123in}{1.512907in}}%
\pgfpathlineto{\pgfqpoint{3.036654in}{1.522542in}}%
\pgfpathlineto{\pgfqpoint{3.028256in}{1.521596in}}%
\pgfpathlineto{\pgfqpoint{3.019844in}{1.520943in}}%
\pgfpathlineto{\pgfqpoint{3.011416in}{1.520592in}}%
\pgfpathlineto{\pgfqpoint{3.002972in}{1.520550in}}%
\pgfpathclose%
\pgfusepath{fill}%
\end{pgfscope}%
\begin{pgfscope}%
\pgfpathrectangle{\pgfqpoint{1.254980in}{0.150000in}}{\pgfqpoint{5.490039in}{5.490039in}}%
\pgfusepath{clip}%
\pgfsetbuttcap%
\pgfsetroundjoin%
\definecolor{currentfill}{rgb}{0.140210,0.665859,0.513427}%
\pgfsetfillcolor{currentfill}%
\pgfsetfillopacity{0.700000}%
\pgfsetlinewidth{0.000000pt}%
\definecolor{currentstroke}{rgb}{0.000000,0.000000,0.000000}%
\pgfsetstrokecolor{currentstroke}%
\pgfsetdash{}{0pt}%
\pgfpathmoveto{\pgfqpoint{5.158834in}{2.892161in}}%
\pgfpathlineto{\pgfqpoint{5.173071in}{2.905884in}}%
\pgfpathlineto{\pgfqpoint{5.187326in}{2.919770in}}%
\pgfpathlineto{\pgfqpoint{5.201602in}{2.933820in}}%
\pgfpathlineto{\pgfqpoint{5.215897in}{2.948035in}}%
\pgfpathlineto{\pgfqpoint{5.223333in}{2.955706in}}%
\pgfpathlineto{\pgfqpoint{5.230759in}{2.963215in}}%
\pgfpathlineto{\pgfqpoint{5.238176in}{2.970562in}}%
\pgfpathlineto{\pgfqpoint{5.245583in}{2.977750in}}%
\pgfpathlineto{\pgfqpoint{5.231295in}{2.963645in}}%
\pgfpathlineto{\pgfqpoint{5.217027in}{2.949703in}}%
\pgfpathlineto{\pgfqpoint{5.202778in}{2.935925in}}%
\pgfpathlineto{\pgfqpoint{5.188549in}{2.922310in}}%
\pgfpathlineto{\pgfqpoint{5.181134in}{2.915003in}}%
\pgfpathlineto{\pgfqpoint{5.173709in}{2.907543in}}%
\pgfpathlineto{\pgfqpoint{5.166276in}{2.899930in}}%
\pgfpathlineto{\pgfqpoint{5.158834in}{2.892161in}}%
\pgfpathclose%
\pgfusepath{fill}%
\end{pgfscope}%
\begin{pgfscope}%
\pgfpathrectangle{\pgfqpoint{1.254980in}{0.150000in}}{\pgfqpoint{5.490039in}{5.490039in}}%
\pgfusepath{clip}%
\pgfsetbuttcap%
\pgfsetroundjoin%
\definecolor{currentfill}{rgb}{0.190631,0.407061,0.556089}%
\pgfsetfillcolor{currentfill}%
\pgfsetfillopacity{0.700000}%
\pgfsetlinewidth{0.000000pt}%
\definecolor{currentstroke}{rgb}{0.000000,0.000000,0.000000}%
\pgfsetstrokecolor{currentstroke}%
\pgfsetdash{}{0pt}%
\pgfpathmoveto{\pgfqpoint{4.518068in}{2.181291in}}%
\pgfpathlineto{\pgfqpoint{4.531900in}{2.190508in}}%
\pgfpathlineto{\pgfqpoint{4.545747in}{2.199887in}}%
\pgfpathlineto{\pgfqpoint{4.559609in}{2.209425in}}%
\pgfpathlineto{\pgfqpoint{4.573486in}{2.219125in}}%
\pgfpathlineto{\pgfqpoint{4.581217in}{2.232352in}}%
\pgfpathlineto{\pgfqpoint{4.588942in}{2.245467in}}%
\pgfpathlineto{\pgfqpoint{4.596662in}{2.258469in}}%
\pgfpathlineto{\pgfqpoint{4.604377in}{2.271356in}}%
\pgfpathlineto{\pgfqpoint{4.590498in}{2.261459in}}%
\pgfpathlineto{\pgfqpoint{4.576634in}{2.251724in}}%
\pgfpathlineto{\pgfqpoint{4.562786in}{2.242150in}}%
\pgfpathlineto{\pgfqpoint{4.548953in}{2.232737in}}%
\pgfpathlineto{\pgfqpoint{4.541240in}{2.220035in}}%
\pgfpathlineto{\pgfqpoint{4.533521in}{2.207226in}}%
\pgfpathlineto{\pgfqpoint{4.525797in}{2.194311in}}%
\pgfpathlineto{\pgfqpoint{4.518068in}{2.181291in}}%
\pgfpathclose%
\pgfusepath{fill}%
\end{pgfscope}%
\begin{pgfscope}%
\pgfpathrectangle{\pgfqpoint{1.254980in}{0.150000in}}{\pgfqpoint{5.490039in}{5.490039in}}%
\pgfusepath{clip}%
\pgfsetbuttcap%
\pgfsetroundjoin%
\definecolor{currentfill}{rgb}{0.120081,0.622161,0.534946}%
\pgfsetfillcolor{currentfill}%
\pgfsetfillopacity{0.700000}%
\pgfsetlinewidth{0.000000pt}%
\definecolor{currentstroke}{rgb}{0.000000,0.000000,0.000000}%
\pgfsetstrokecolor{currentstroke}%
\pgfsetdash{}{0pt}%
\pgfpathmoveto{\pgfqpoint{1.996451in}{2.891299in}}%
\pgfpathlineto{\pgfqpoint{2.010580in}{2.863371in}}%
\pgfpathlineto{\pgfqpoint{2.024691in}{2.835770in}}%
\pgfpathlineto{\pgfqpoint{2.038783in}{2.808493in}}%
\pgfpathlineto{\pgfqpoint{2.052856in}{2.781537in}}%
\pgfpathlineto{\pgfqpoint{2.062234in}{2.770352in}}%
\pgfpathlineto{\pgfqpoint{2.071582in}{2.759603in}}%
\pgfpathlineto{\pgfqpoint{2.080900in}{2.749282in}}%
\pgfpathlineto{\pgfqpoint{2.090189in}{2.739383in}}%
\pgfpathlineto{\pgfqpoint{2.076190in}{2.765609in}}%
\pgfpathlineto{\pgfqpoint{2.062174in}{2.792154in}}%
\pgfpathlineto{\pgfqpoint{2.048139in}{2.819020in}}%
\pgfpathlineto{\pgfqpoint{2.034086in}{2.846212in}}%
\pgfpathlineto{\pgfqpoint{2.024723in}{2.856830in}}%
\pgfpathlineto{\pgfqpoint{2.015330in}{2.867879in}}%
\pgfpathlineto{\pgfqpoint{2.005906in}{2.879366in}}%
\pgfpathlineto{\pgfqpoint{1.996451in}{2.891299in}}%
\pgfpathclose%
\pgfusepath{fill}%
\end{pgfscope}%
\begin{pgfscope}%
\pgfpathrectangle{\pgfqpoint{1.254980in}{0.150000in}}{\pgfqpoint{5.490039in}{5.490039in}}%
\pgfusepath{clip}%
\pgfsetbuttcap%
\pgfsetroundjoin%
\definecolor{currentfill}{rgb}{0.154815,0.493313,0.557840}%
\pgfsetfillcolor{currentfill}%
\pgfsetfillopacity{0.700000}%
\pgfsetlinewidth{0.000000pt}%
\definecolor{currentstroke}{rgb}{0.000000,0.000000,0.000000}%
\pgfsetstrokecolor{currentstroke}%
\pgfsetdash{}{0pt}%
\pgfpathmoveto{\pgfqpoint{2.183489in}{2.522112in}}%
\pgfpathlineto{\pgfqpoint{2.197424in}{2.498301in}}%
\pgfpathlineto{\pgfqpoint{2.211344in}{2.474772in}}%
\pgfpathlineto{\pgfqpoint{2.225251in}{2.451521in}}%
\pgfpathlineto{\pgfqpoint{2.239143in}{2.428546in}}%
\pgfpathlineto{\pgfqpoint{2.248348in}{2.418604in}}%
\pgfpathlineto{\pgfqpoint{2.257525in}{2.409090in}}%
\pgfpathlineto{\pgfqpoint{2.266674in}{2.399997in}}%
\pgfpathlineto{\pgfqpoint{2.275795in}{2.391316in}}%
\pgfpathlineto{\pgfqpoint{2.261971in}{2.413554in}}%
\pgfpathlineto{\pgfqpoint{2.248134in}{2.436067in}}%
\pgfpathlineto{\pgfqpoint{2.234284in}{2.458856in}}%
\pgfpathlineto{\pgfqpoint{2.220420in}{2.481924in}}%
\pgfpathlineto{\pgfqpoint{2.211230in}{2.491330in}}%
\pgfpathlineto{\pgfqpoint{2.202012in}{2.501158in}}%
\pgfpathlineto{\pgfqpoint{2.192765in}{2.511416in}}%
\pgfpathlineto{\pgfqpoint{2.183489in}{2.522112in}}%
\pgfpathclose%
\pgfusepath{fill}%
\end{pgfscope}%
\begin{pgfscope}%
\pgfpathrectangle{\pgfqpoint{1.254980in}{0.150000in}}{\pgfqpoint{5.490039in}{5.490039in}}%
\pgfusepath{clip}%
\pgfsetbuttcap%
\pgfsetroundjoin%
\definecolor{currentfill}{rgb}{0.216210,0.351535,0.550627}%
\pgfsetfillcolor{currentfill}%
\pgfsetfillopacity{0.700000}%
\pgfsetlinewidth{0.000000pt}%
\definecolor{currentstroke}{rgb}{0.000000,0.000000,0.000000}%
\pgfsetstrokecolor{currentstroke}%
\pgfsetdash{}{0pt}%
\pgfpathmoveto{\pgfqpoint{4.400887in}{2.040238in}}%
\pgfpathlineto{\pgfqpoint{4.414658in}{2.048335in}}%
\pgfpathlineto{\pgfqpoint{4.428443in}{2.056592in}}%
\pgfpathlineto{\pgfqpoint{4.442242in}{2.065009in}}%
\pgfpathlineto{\pgfqpoint{4.456055in}{2.073587in}}%
\pgfpathlineto{\pgfqpoint{4.463824in}{2.087376in}}%
\pgfpathlineto{\pgfqpoint{4.471588in}{2.101076in}}%
\pgfpathlineto{\pgfqpoint{4.479347in}{2.114685in}}%
\pgfpathlineto{\pgfqpoint{4.487101in}{2.128201in}}%
\pgfpathlineto{\pgfqpoint{4.473286in}{2.119369in}}%
\pgfpathlineto{\pgfqpoint{4.459485in}{2.110698in}}%
\pgfpathlineto{\pgfqpoint{4.445699in}{2.102187in}}%
\pgfpathlineto{\pgfqpoint{4.431927in}{2.093836in}}%
\pgfpathlineto{\pgfqpoint{4.424174in}{2.080564in}}%
\pgfpathlineto{\pgfqpoint{4.416416in}{2.067205in}}%
\pgfpathlineto{\pgfqpoint{4.408654in}{2.053762in}}%
\pgfpathlineto{\pgfqpoint{4.400887in}{2.040238in}}%
\pgfpathclose%
\pgfusepath{fill}%
\end{pgfscope}%
\begin{pgfscope}%
\pgfpathrectangle{\pgfqpoint{1.254980in}{0.150000in}}{\pgfqpoint{5.490039in}{5.490039in}}%
\pgfusepath{clip}%
\pgfsetbuttcap%
\pgfsetroundjoin%
\definecolor{currentfill}{rgb}{0.135066,0.544853,0.554029}%
\pgfsetfillcolor{currentfill}%
\pgfsetfillopacity{0.700000}%
\pgfsetlinewidth{0.000000pt}%
\definecolor{currentstroke}{rgb}{0.000000,0.000000,0.000000}%
\pgfsetstrokecolor{currentstroke}%
\pgfsetdash{}{0pt}%
\pgfpathmoveto{\pgfqpoint{4.838649in}{2.549811in}}%
\pgfpathlineto{\pgfqpoint{4.852680in}{2.561652in}}%
\pgfpathlineto{\pgfqpoint{4.866728in}{2.573655in}}%
\pgfpathlineto{\pgfqpoint{4.880793in}{2.585820in}}%
\pgfpathlineto{\pgfqpoint{4.894877in}{2.598149in}}%
\pgfpathlineto{\pgfqpoint{4.902486in}{2.609091in}}%
\pgfpathlineto{\pgfqpoint{4.910089in}{2.619880in}}%
\pgfpathlineto{\pgfqpoint{4.917684in}{2.630516in}}%
\pgfpathlineto{\pgfqpoint{4.925272in}{2.641000in}}%
\pgfpathlineto{\pgfqpoint{4.911189in}{2.628625in}}%
\pgfpathlineto{\pgfqpoint{4.897124in}{2.616412in}}%
\pgfpathlineto{\pgfqpoint{4.883078in}{2.604362in}}%
\pgfpathlineto{\pgfqpoint{4.869048in}{2.592474in}}%
\pgfpathlineto{\pgfqpoint{4.861459in}{2.582026in}}%
\pgfpathlineto{\pgfqpoint{4.853863in}{2.571433in}}%
\pgfpathlineto{\pgfqpoint{4.846260in}{2.560694in}}%
\pgfpathlineto{\pgfqpoint{4.838649in}{2.549811in}}%
\pgfpathclose%
\pgfusepath{fill}%
\end{pgfscope}%
\begin{pgfscope}%
\pgfpathrectangle{\pgfqpoint{1.254980in}{0.150000in}}{\pgfqpoint{5.490039in}{5.490039in}}%
\pgfusepath{clip}%
\pgfsetbuttcap%
\pgfsetroundjoin%
\definecolor{currentfill}{rgb}{0.275191,0.194905,0.496005}%
\pgfsetfillcolor{currentfill}%
\pgfsetfillopacity{0.700000}%
\pgfsetlinewidth{0.000000pt}%
\definecolor{currentstroke}{rgb}{0.000000,0.000000,0.000000}%
\pgfsetstrokecolor{currentstroke}%
\pgfsetdash{}{0pt}%
\pgfpathmoveto{\pgfqpoint{4.080411in}{1.692251in}}%
\pgfpathlineto{\pgfqpoint{4.094032in}{1.696746in}}%
\pgfpathlineto{\pgfqpoint{4.107663in}{1.701399in}}%
\pgfpathlineto{\pgfqpoint{4.121305in}{1.706211in}}%
\pgfpathlineto{\pgfqpoint{4.134959in}{1.711182in}}%
\pgfpathlineto{\pgfqpoint{4.142813in}{1.724950in}}%
\pgfpathlineto{\pgfqpoint{4.150662in}{1.738711in}}%
\pgfpathlineto{\pgfqpoint{4.158507in}{1.752461in}}%
\pgfpathlineto{\pgfqpoint{4.166347in}{1.766194in}}%
\pgfpathlineto{\pgfqpoint{4.152695in}{1.760830in}}%
\pgfpathlineto{\pgfqpoint{4.139054in}{1.755624in}}%
\pgfpathlineto{\pgfqpoint{4.125424in}{1.750578in}}%
\pgfpathlineto{\pgfqpoint{4.111806in}{1.745691in}}%
\pgfpathlineto{\pgfqpoint{4.103964in}{1.732340in}}%
\pgfpathlineto{\pgfqpoint{4.096117in}{1.718980in}}%
\pgfpathlineto{\pgfqpoint{4.088266in}{1.705616in}}%
\pgfpathlineto{\pgfqpoint{4.080411in}{1.692251in}}%
\pgfpathclose%
\pgfusepath{fill}%
\end{pgfscope}%
\begin{pgfscope}%
\pgfpathrectangle{\pgfqpoint{1.254980in}{0.150000in}}{\pgfqpoint{5.490039in}{5.490039in}}%
\pgfusepath{clip}%
\pgfsetbuttcap%
\pgfsetroundjoin%
\definecolor{currentfill}{rgb}{0.276022,0.044167,0.370164}%
\pgfsetfillcolor{currentfill}%
\pgfsetfillopacity{0.700000}%
\pgfsetlinewidth{0.000000pt}%
\definecolor{currentstroke}{rgb}{0.000000,0.000000,0.000000}%
\pgfsetstrokecolor{currentstroke}%
\pgfsetdash{}{0pt}%
\pgfpathmoveto{\pgfqpoint{3.705292in}{1.418723in}}%
\pgfpathlineto{\pgfqpoint{3.718800in}{1.418294in}}%
\pgfpathlineto{\pgfqpoint{3.732316in}{1.418026in}}%
\pgfpathlineto{\pgfqpoint{3.745839in}{1.417918in}}%
\pgfpathlineto{\pgfqpoint{3.759369in}{1.417970in}}%
\pgfpathlineto{\pgfqpoint{3.767342in}{1.428857in}}%
\pgfpathlineto{\pgfqpoint{3.775309in}{1.439852in}}%
\pgfpathlineto{\pgfqpoint{3.783270in}{1.450951in}}%
\pgfpathlineto{\pgfqpoint{3.791225in}{1.462146in}}%
\pgfpathlineto{\pgfqpoint{3.777705in}{1.461566in}}%
\pgfpathlineto{\pgfqpoint{3.764193in}{1.461146in}}%
\pgfpathlineto{\pgfqpoint{3.750688in}{1.460887in}}%
\pgfpathlineto{\pgfqpoint{3.737192in}{1.460789in}}%
\pgfpathlineto{\pgfqpoint{3.729226in}{1.450110in}}%
\pgfpathlineto{\pgfqpoint{3.721254in}{1.439536in}}%
\pgfpathlineto{\pgfqpoint{3.713276in}{1.429072in}}%
\pgfpathlineto{\pgfqpoint{3.705292in}{1.418723in}}%
\pgfpathclose%
\pgfusepath{fill}%
\end{pgfscope}%
\begin{pgfscope}%
\pgfpathrectangle{\pgfqpoint{1.254980in}{0.150000in}}{\pgfqpoint{5.490039in}{5.490039in}}%
\pgfusepath{clip}%
\pgfsetbuttcap%
\pgfsetroundjoin%
\definecolor{currentfill}{rgb}{0.280267,0.073417,0.397163}%
\pgfsetfillcolor{currentfill}%
\pgfsetfillopacity{0.700000}%
\pgfsetlinewidth{0.000000pt}%
\definecolor{currentstroke}{rgb}{0.000000,0.000000,0.000000}%
\pgfsetstrokecolor{currentstroke}%
\pgfsetdash{}{0pt}%
\pgfpathmoveto{\pgfqpoint{3.791225in}{1.462146in}}%
\pgfpathlineto{\pgfqpoint{3.804753in}{1.462887in}}%
\pgfpathlineto{\pgfqpoint{3.818289in}{1.463787in}}%
\pgfpathlineto{\pgfqpoint{3.831833in}{1.464847in}}%
\pgfpathlineto{\pgfqpoint{3.845386in}{1.466066in}}%
\pgfpathlineto{\pgfqpoint{3.853326in}{1.477866in}}%
\pgfpathlineto{\pgfqpoint{3.861262in}{1.489747in}}%
\pgfpathlineto{\pgfqpoint{3.869192in}{1.501704in}}%
\pgfpathlineto{\pgfqpoint{3.877117in}{1.513733in}}%
\pgfpathlineto{\pgfqpoint{3.863572in}{1.512013in}}%
\pgfpathlineto{\pgfqpoint{3.850036in}{1.510452in}}%
\pgfpathlineto{\pgfqpoint{3.836509in}{1.509051in}}%
\pgfpathlineto{\pgfqpoint{3.822990in}{1.507810in}}%
\pgfpathlineto{\pgfqpoint{3.815057in}{1.496272in}}%
\pgfpathlineto{\pgfqpoint{3.807118in}{1.484812in}}%
\pgfpathlineto{\pgfqpoint{3.799174in}{1.473435in}}%
\pgfpathlineto{\pgfqpoint{3.791225in}{1.462146in}}%
\pgfpathclose%
\pgfusepath{fill}%
\end{pgfscope}%
\begin{pgfscope}%
\pgfpathrectangle{\pgfqpoint{1.254980in}{0.150000in}}{\pgfqpoint{5.490039in}{5.490039in}}%
\pgfusepath{clip}%
\pgfsetbuttcap%
\pgfsetroundjoin%
\definecolor{currentfill}{rgb}{0.267004,0.004874,0.329415}%
\pgfsetfillcolor{currentfill}%
\pgfsetfillopacity{0.700000}%
\pgfsetlinewidth{0.000000pt}%
\definecolor{currentstroke}{rgb}{0.000000,0.000000,0.000000}%
\pgfsetstrokecolor{currentstroke}%
\pgfsetdash{}{0pt}%
\pgfpathmoveto{\pgfqpoint{3.392807in}{1.362412in}}%
\pgfpathlineto{\pgfqpoint{3.406283in}{1.357617in}}%
\pgfpathlineto{\pgfqpoint{3.419763in}{1.352988in}}%
\pgfpathlineto{\pgfqpoint{3.433246in}{1.348527in}}%
\pgfpathlineto{\pgfqpoint{3.446734in}{1.344231in}}%
\pgfpathlineto{\pgfqpoint{3.454862in}{1.350911in}}%
\pgfpathlineto{\pgfqpoint{3.462981in}{1.357794in}}%
\pgfpathlineto{\pgfqpoint{3.471091in}{1.364876in}}%
\pgfpathlineto{\pgfqpoint{3.479192in}{1.372148in}}%
\pgfpathlineto{\pgfqpoint{3.465726in}{1.375834in}}%
\pgfpathlineto{\pgfqpoint{3.452265in}{1.379685in}}%
\pgfpathlineto{\pgfqpoint{3.438808in}{1.383703in}}%
\pgfpathlineto{\pgfqpoint{3.425355in}{1.387888in}}%
\pgfpathlineto{\pgfqpoint{3.417232in}{1.381215in}}%
\pgfpathlineto{\pgfqpoint{3.409100in}{1.374740in}}%
\pgfpathlineto{\pgfqpoint{3.400959in}{1.368471in}}%
\pgfpathlineto{\pgfqpoint{3.392807in}{1.362412in}}%
\pgfpathclose%
\pgfusepath{fill}%
\end{pgfscope}%
\begin{pgfscope}%
\pgfpathrectangle{\pgfqpoint{1.254980in}{0.150000in}}{\pgfqpoint{5.490039in}{5.490039in}}%
\pgfusepath{clip}%
\pgfsetbuttcap%
\pgfsetroundjoin%
\definecolor{currentfill}{rgb}{0.278791,0.062145,0.386592}%
\pgfsetfillcolor{currentfill}%
\pgfsetfillopacity{0.700000}%
\pgfsetlinewidth{0.000000pt}%
\definecolor{currentstroke}{rgb}{0.000000,0.000000,0.000000}%
\pgfsetstrokecolor{currentstroke}%
\pgfsetdash{}{0pt}%
\pgfpathmoveto{\pgfqpoint{3.056997in}{1.480406in}}%
\pgfpathlineto{\pgfqpoint{3.070502in}{1.470826in}}%
\pgfpathlineto{\pgfqpoint{3.084006in}{1.461427in}}%
\pgfpathlineto{\pgfqpoint{3.097511in}{1.452208in}}%
\pgfpathlineto{\pgfqpoint{3.111015in}{1.443168in}}%
\pgfpathlineto{\pgfqpoint{3.119382in}{1.444583in}}%
\pgfpathlineto{\pgfqpoint{3.127736in}{1.446291in}}%
\pgfpathlineto{\pgfqpoint{3.136075in}{1.448284in}}%
\pgfpathlineto{\pgfqpoint{3.144400in}{1.450557in}}%
\pgfpathlineto{\pgfqpoint{3.130931in}{1.458924in}}%
\pgfpathlineto{\pgfqpoint{3.117463in}{1.467470in}}%
\pgfpathlineto{\pgfqpoint{3.103995in}{1.476195in}}%
\pgfpathlineto{\pgfqpoint{3.090527in}{1.485101in}}%
\pgfpathlineto{\pgfqpoint{3.082166in}{1.483490in}}%
\pgfpathlineto{\pgfqpoint{3.073791in}{1.482166in}}%
\pgfpathlineto{\pgfqpoint{3.065402in}{1.481136in}}%
\pgfpathlineto{\pgfqpoint{3.056997in}{1.480406in}}%
\pgfpathclose%
\pgfusepath{fill}%
\end{pgfscope}%
\begin{pgfscope}%
\pgfpathrectangle{\pgfqpoint{1.254980in}{0.150000in}}{\pgfqpoint{5.490039in}{5.490039in}}%
\pgfusepath{clip}%
\pgfsetbuttcap%
\pgfsetroundjoin%
\definecolor{currentfill}{rgb}{0.296479,0.761561,0.424223}%
\pgfsetfillcolor{currentfill}%
\pgfsetfillopacity{0.700000}%
\pgfsetlinewidth{0.000000pt}%
\definecolor{currentstroke}{rgb}{0.000000,0.000000,0.000000}%
\pgfsetstrokecolor{currentstroke}%
\pgfsetdash{}{0pt}%
\pgfpathmoveto{\pgfqpoint{5.448539in}{3.166838in}}%
\pgfpathlineto{\pgfqpoint{5.462980in}{3.181903in}}%
\pgfpathlineto{\pgfqpoint{5.477444in}{3.197133in}}%
\pgfpathlineto{\pgfqpoint{5.491928in}{3.212527in}}%
\pgfpathlineto{\pgfqpoint{5.506434in}{3.228086in}}%
\pgfpathlineto{\pgfqpoint{5.513681in}{3.232705in}}%
\pgfpathlineto{\pgfqpoint{5.520918in}{3.237170in}}%
\pgfpathlineto{\pgfqpoint{5.528143in}{3.241487in}}%
\pgfpathlineto{\pgfqpoint{5.535359in}{3.245656in}}%
\pgfpathlineto{\pgfqpoint{5.520867in}{3.230334in}}%
\pgfpathlineto{\pgfqpoint{5.506397in}{3.215177in}}%
\pgfpathlineto{\pgfqpoint{5.491948in}{3.200184in}}%
\pgfpathlineto{\pgfqpoint{5.477520in}{3.185355in}}%
\pgfpathlineto{\pgfqpoint{5.470290in}{3.180938in}}%
\pgfpathlineto{\pgfqpoint{5.463049in}{3.176381in}}%
\pgfpathlineto{\pgfqpoint{5.455799in}{3.171682in}}%
\pgfpathlineto{\pgfqpoint{5.448539in}{3.166838in}}%
\pgfpathclose%
\pgfusepath{fill}%
\end{pgfscope}%
\begin{pgfscope}%
\pgfpathrectangle{\pgfqpoint{1.254980in}{0.150000in}}{\pgfqpoint{5.490039in}{5.490039in}}%
\pgfusepath{clip}%
\pgfsetbuttcap%
\pgfsetroundjoin%
\definecolor{currentfill}{rgb}{0.269944,0.014625,0.341379}%
\pgfsetfillcolor{currentfill}%
\pgfsetfillopacity{0.700000}%
\pgfsetlinewidth{0.000000pt}%
\definecolor{currentstroke}{rgb}{0.000000,0.000000,0.000000}%
\pgfsetstrokecolor{currentstroke}%
\pgfsetdash{}{0pt}%
\pgfpathmoveto{\pgfqpoint{3.252189in}{1.389967in}}%
\pgfpathlineto{\pgfqpoint{3.265669in}{1.383176in}}%
\pgfpathlineto{\pgfqpoint{3.279152in}{1.376556in}}%
\pgfpathlineto{\pgfqpoint{3.292637in}{1.370107in}}%
\pgfpathlineto{\pgfqpoint{3.306124in}{1.363829in}}%
\pgfpathlineto{\pgfqpoint{3.314344in}{1.368332in}}%
\pgfpathlineto{\pgfqpoint{3.322552in}{1.373077in}}%
\pgfpathlineto{\pgfqpoint{3.330750in}{1.378060in}}%
\pgfpathlineto{\pgfqpoint{3.338937in}{1.383273in}}%
\pgfpathlineto{\pgfqpoint{3.325477in}{1.388911in}}%
\pgfpathlineto{\pgfqpoint{3.312020in}{1.394720in}}%
\pgfpathlineto{\pgfqpoint{3.298566in}{1.400699in}}%
\pgfpathlineto{\pgfqpoint{3.285114in}{1.406850in}}%
\pgfpathlineto{\pgfqpoint{3.276900in}{1.402266in}}%
\pgfpathlineto{\pgfqpoint{3.268675in}{1.397920in}}%
\pgfpathlineto{\pgfqpoint{3.260438in}{1.393818in}}%
\pgfpathlineto{\pgfqpoint{3.252189in}{1.389967in}}%
\pgfpathclose%
\pgfusepath{fill}%
\end{pgfscope}%
\begin{pgfscope}%
\pgfpathrectangle{\pgfqpoint{1.254980in}{0.150000in}}{\pgfqpoint{5.490039in}{5.490039in}}%
\pgfusepath{clip}%
\pgfsetbuttcap%
\pgfsetroundjoin%
\definecolor{currentfill}{rgb}{0.272594,0.025563,0.353093}%
\pgfsetfillcolor{currentfill}%
\pgfsetfillopacity{0.700000}%
\pgfsetlinewidth{0.000000pt}%
\definecolor{currentstroke}{rgb}{0.000000,0.000000,0.000000}%
\pgfsetstrokecolor{currentstroke}%
\pgfsetdash{}{0pt}%
\pgfpathmoveto{\pgfqpoint{3.619269in}{1.384128in}}%
\pgfpathlineto{\pgfqpoint{3.632764in}{1.382499in}}%
\pgfpathlineto{\pgfqpoint{3.646266in}{1.381031in}}%
\pgfpathlineto{\pgfqpoint{3.659775in}{1.379725in}}%
\pgfpathlineto{\pgfqpoint{3.673290in}{1.378580in}}%
\pgfpathlineto{\pgfqpoint{3.681300in}{1.388417in}}%
\pgfpathlineto{\pgfqpoint{3.689304in}{1.398390in}}%
\pgfpathlineto{\pgfqpoint{3.697301in}{1.408494in}}%
\pgfpathlineto{\pgfqpoint{3.705292in}{1.418723in}}%
\pgfpathlineto{\pgfqpoint{3.691790in}{1.419313in}}%
\pgfpathlineto{\pgfqpoint{3.678296in}{1.420064in}}%
\pgfpathlineto{\pgfqpoint{3.664808in}{1.420977in}}%
\pgfpathlineto{\pgfqpoint{3.651327in}{1.422052in}}%
\pgfpathlineto{\pgfqpoint{3.643323in}{1.412367in}}%
\pgfpathlineto{\pgfqpoint{3.635312in}{1.402815in}}%
\pgfpathlineto{\pgfqpoint{3.627294in}{1.393400in}}%
\pgfpathlineto{\pgfqpoint{3.619269in}{1.384128in}}%
\pgfpathclose%
\pgfusepath{fill}%
\end{pgfscope}%
\begin{pgfscope}%
\pgfpathrectangle{\pgfqpoint{1.254980in}{0.150000in}}{\pgfqpoint{5.490039in}{5.490039in}}%
\pgfusepath{clip}%
\pgfsetbuttcap%
\pgfsetroundjoin%
\definecolor{currentfill}{rgb}{0.241237,0.296485,0.539709}%
\pgfsetfillcolor{currentfill}%
\pgfsetfillopacity{0.700000}%
\pgfsetlinewidth{0.000000pt}%
\definecolor{currentstroke}{rgb}{0.000000,0.000000,0.000000}%
\pgfsetstrokecolor{currentstroke}%
\pgfsetdash{}{0pt}%
\pgfpathmoveto{\pgfqpoint{4.283650in}{1.900976in}}%
\pgfpathlineto{\pgfqpoint{4.297364in}{1.907842in}}%
\pgfpathlineto{\pgfqpoint{4.311091in}{1.914866in}}%
\pgfpathlineto{\pgfqpoint{4.324832in}{1.922050in}}%
\pgfpathlineto{\pgfqpoint{4.338586in}{1.929393in}}%
\pgfpathlineto{\pgfqpoint{4.346389in}{1.943483in}}%
\pgfpathlineto{\pgfqpoint{4.354188in}{1.957511in}}%
\pgfpathlineto{\pgfqpoint{4.361983in}{1.971474in}}%
\pgfpathlineto{\pgfqpoint{4.369773in}{1.985371in}}%
\pgfpathlineto{\pgfqpoint{4.356017in}{1.977717in}}%
\pgfpathlineto{\pgfqpoint{4.342275in}{1.970222in}}%
\pgfpathlineto{\pgfqpoint{4.328547in}{1.962886in}}%
\pgfpathlineto{\pgfqpoint{4.314832in}{1.955711in}}%
\pgfpathlineto{\pgfqpoint{4.307043in}{1.942114in}}%
\pgfpathlineto{\pgfqpoint{4.299249in}{1.928458in}}%
\pgfpathlineto{\pgfqpoint{4.291452in}{1.914744in}}%
\pgfpathlineto{\pgfqpoint{4.283650in}{1.900976in}}%
\pgfpathclose%
\pgfusepath{fill}%
\end{pgfscope}%
\begin{pgfscope}%
\pgfpathrectangle{\pgfqpoint{1.254980in}{0.150000in}}{\pgfqpoint{5.490039in}{5.490039in}}%
\pgfusepath{clip}%
\pgfsetbuttcap%
\pgfsetroundjoin%
\definecolor{currentfill}{rgb}{0.282910,0.105393,0.426902}%
\pgfsetfillcolor{currentfill}%
\pgfsetfillopacity{0.700000}%
\pgfsetlinewidth{0.000000pt}%
\definecolor{currentstroke}{rgb}{0.000000,0.000000,0.000000}%
\pgfsetstrokecolor{currentstroke}%
\pgfsetdash{}{0pt}%
\pgfpathmoveto{\pgfqpoint{3.877117in}{1.513733in}}%
\pgfpathlineto{\pgfqpoint{3.890671in}{1.515613in}}%
\pgfpathlineto{\pgfqpoint{3.904233in}{1.517652in}}%
\pgfpathlineto{\pgfqpoint{3.917805in}{1.519850in}}%
\pgfpathlineto{\pgfqpoint{3.931386in}{1.522206in}}%
\pgfpathlineto{\pgfqpoint{3.939300in}{1.534787in}}%
\pgfpathlineto{\pgfqpoint{3.947208in}{1.547423in}}%
\pgfpathlineto{\pgfqpoint{3.955112in}{1.560109in}}%
\pgfpathlineto{\pgfqpoint{3.963011in}{1.572842in}}%
\pgfpathlineto{\pgfqpoint{3.949436in}{1.570010in}}%
\pgfpathlineto{\pgfqpoint{3.935870in}{1.567338in}}%
\pgfpathlineto{\pgfqpoint{3.922313in}{1.564825in}}%
\pgfpathlineto{\pgfqpoint{3.908766in}{1.562471in}}%
\pgfpathlineto{\pgfqpoint{3.900861in}{1.550202in}}%
\pgfpathlineto{\pgfqpoint{3.892952in}{1.537986in}}%
\pgfpathlineto{\pgfqpoint{3.885037in}{1.525829in}}%
\pgfpathlineto{\pgfqpoint{3.877117in}{1.513733in}}%
\pgfpathclose%
\pgfusepath{fill}%
\end{pgfscope}%
\begin{pgfscope}%
\pgfpathrectangle{\pgfqpoint{1.254980in}{0.150000in}}{\pgfqpoint{5.490039in}{5.490039in}}%
\pgfusepath{clip}%
\pgfsetbuttcap%
\pgfsetroundjoin%
\definecolor{currentfill}{rgb}{0.255645,0.260703,0.528312}%
\pgfsetfillcolor{currentfill}%
\pgfsetfillopacity{0.700000}%
\pgfsetlinewidth{0.000000pt}%
\definecolor{currentstroke}{rgb}{0.000000,0.000000,0.000000}%
\pgfsetstrokecolor{currentstroke}%
\pgfsetdash{}{0pt}%
\pgfpathmoveto{\pgfqpoint{2.588335in}{1.911397in}}%
\pgfpathlineto{\pgfqpoint{2.601999in}{1.894761in}}%
\pgfpathlineto{\pgfqpoint{2.615657in}{1.878344in}}%
\pgfpathlineto{\pgfqpoint{2.629308in}{1.862143in}}%
\pgfpathlineto{\pgfqpoint{2.642952in}{1.846157in}}%
\pgfpathlineto{\pgfqpoint{2.651757in}{1.840518in}}%
\pgfpathlineto{\pgfqpoint{2.660540in}{1.835266in}}%
\pgfpathlineto{\pgfqpoint{2.669301in}{1.830394in}}%
\pgfpathlineto{\pgfqpoint{2.678041in}{1.825895in}}%
\pgfpathlineto{\pgfqpoint{2.664451in}{1.841153in}}%
\pgfpathlineto{\pgfqpoint{2.650855in}{1.856624in}}%
\pgfpathlineto{\pgfqpoint{2.637253in}{1.872312in}}%
\pgfpathlineto{\pgfqpoint{2.623645in}{1.888216in}}%
\pgfpathlineto{\pgfqpoint{2.614851in}{1.893433in}}%
\pgfpathlineto{\pgfqpoint{2.606035in}{1.899030in}}%
\pgfpathlineto{\pgfqpoint{2.597196in}{1.905015in}}%
\pgfpathlineto{\pgfqpoint{2.588335in}{1.911397in}}%
\pgfpathclose%
\pgfusepath{fill}%
\end{pgfscope}%
\begin{pgfscope}%
\pgfpathrectangle{\pgfqpoint{1.254980in}{0.150000in}}{\pgfqpoint{5.490039in}{5.490039in}}%
\pgfusepath{clip}%
\pgfsetbuttcap%
\pgfsetroundjoin%
\definecolor{currentfill}{rgb}{0.265145,0.232956,0.516599}%
\pgfsetfillcolor{currentfill}%
\pgfsetfillopacity{0.700000}%
\pgfsetlinewidth{0.000000pt}%
\definecolor{currentstroke}{rgb}{0.000000,0.000000,0.000000}%
\pgfsetstrokecolor{currentstroke}%
\pgfsetdash{}{0pt}%
\pgfpathmoveto{\pgfqpoint{2.642952in}{1.846157in}}%
\pgfpathlineto{\pgfqpoint{2.656591in}{1.830385in}}%
\pgfpathlineto{\pgfqpoint{2.670223in}{1.814825in}}%
\pgfpathlineto{\pgfqpoint{2.683850in}{1.799475in}}%
\pgfpathlineto{\pgfqpoint{2.697471in}{1.784336in}}%
\pgfpathlineto{\pgfqpoint{2.706222in}{1.779435in}}%
\pgfpathlineto{\pgfqpoint{2.714951in}{1.774913in}}%
\pgfpathlineto{\pgfqpoint{2.723660in}{1.770762in}}%
\pgfpathlineto{\pgfqpoint{2.732348in}{1.766975in}}%
\pgfpathlineto{\pgfqpoint{2.718779in}{1.781391in}}%
\pgfpathlineto{\pgfqpoint{2.705205in}{1.796015in}}%
\pgfpathlineto{\pgfqpoint{2.691626in}{1.810849in}}%
\pgfpathlineto{\pgfqpoint{2.678041in}{1.825895in}}%
\pgfpathlineto{\pgfqpoint{2.669301in}{1.830394in}}%
\pgfpathlineto{\pgfqpoint{2.660540in}{1.835266in}}%
\pgfpathlineto{\pgfqpoint{2.651757in}{1.840518in}}%
\pgfpathlineto{\pgfqpoint{2.642952in}{1.846157in}}%
\pgfpathclose%
\pgfusepath{fill}%
\end{pgfscope}%
\begin{pgfscope}%
\pgfpathrectangle{\pgfqpoint{1.254980in}{0.150000in}}{\pgfqpoint{5.490039in}{5.490039in}}%
\pgfusepath{clip}%
\pgfsetbuttcap%
\pgfsetroundjoin%
\definecolor{currentfill}{rgb}{0.153364,0.497000,0.557724}%
\pgfsetfillcolor{currentfill}%
\pgfsetfillopacity{0.700000}%
\pgfsetlinewidth{0.000000pt}%
\definecolor{currentstroke}{rgb}{0.000000,0.000000,0.000000}%
\pgfsetstrokecolor{currentstroke}%
\pgfsetdash{}{0pt}%
\pgfpathmoveto{\pgfqpoint{4.721574in}{2.412557in}}%
\pgfpathlineto{\pgfqpoint{4.735536in}{2.423565in}}%
\pgfpathlineto{\pgfqpoint{4.749515in}{2.434734in}}%
\pgfpathlineto{\pgfqpoint{4.763510in}{2.446066in}}%
\pgfpathlineto{\pgfqpoint{4.777523in}{2.457559in}}%
\pgfpathlineto{\pgfqpoint{4.785187in}{2.469590in}}%
\pgfpathlineto{\pgfqpoint{4.792844in}{2.481479in}}%
\pgfpathlineto{\pgfqpoint{4.800495in}{2.493227in}}%
\pgfpathlineto{\pgfqpoint{4.808140in}{2.504831in}}%
\pgfpathlineto{\pgfqpoint{4.794126in}{2.493230in}}%
\pgfpathlineto{\pgfqpoint{4.780130in}{2.481791in}}%
\pgfpathlineto{\pgfqpoint{4.766151in}{2.470514in}}%
\pgfpathlineto{\pgfqpoint{4.752188in}{2.459399in}}%
\pgfpathlineto{\pgfqpoint{4.744544in}{2.447891in}}%
\pgfpathlineto{\pgfqpoint{4.736894in}{2.436247in}}%
\pgfpathlineto{\pgfqpoint{4.729237in}{2.424469in}}%
\pgfpathlineto{\pgfqpoint{4.721574in}{2.412557in}}%
\pgfpathclose%
\pgfusepath{fill}%
\end{pgfscope}%
\begin{pgfscope}%
\pgfpathrectangle{\pgfqpoint{1.254980in}{0.150000in}}{\pgfqpoint{5.490039in}{5.490039in}}%
\pgfusepath{clip}%
\pgfsetbuttcap%
\pgfsetroundjoin%
\definecolor{currentfill}{rgb}{0.120638,0.625828,0.533488}%
\pgfsetfillcolor{currentfill}%
\pgfsetfillopacity{0.700000}%
\pgfsetlinewidth{0.000000pt}%
\definecolor{currentstroke}{rgb}{0.000000,0.000000,0.000000}%
\pgfsetstrokecolor{currentstroke}%
\pgfsetdash{}{0pt}%
\pgfpathmoveto{\pgfqpoint{5.042193in}{2.770531in}}%
\pgfpathlineto{\pgfqpoint{5.056363in}{2.783723in}}%
\pgfpathlineto{\pgfqpoint{5.070551in}{2.797078in}}%
\pgfpathlineto{\pgfqpoint{5.084759in}{2.810597in}}%
\pgfpathlineto{\pgfqpoint{5.098985in}{2.824280in}}%
\pgfpathlineto{\pgfqpoint{5.106497in}{2.833332in}}%
\pgfpathlineto{\pgfqpoint{5.114000in}{2.842221in}}%
\pgfpathlineto{\pgfqpoint{5.121494in}{2.850946in}}%
\pgfpathlineto{\pgfqpoint{5.128979in}{2.859510in}}%
\pgfpathlineto{\pgfqpoint{5.114757in}{2.845873in}}%
\pgfpathlineto{\pgfqpoint{5.100554in}{2.832401in}}%
\pgfpathlineto{\pgfqpoint{5.086369in}{2.819091in}}%
\pgfpathlineto{\pgfqpoint{5.072204in}{2.805945in}}%
\pgfpathlineto{\pgfqpoint{5.064714in}{2.797324in}}%
\pgfpathlineto{\pgfqpoint{5.057215in}{2.788549in}}%
\pgfpathlineto{\pgfqpoint{5.049708in}{2.779618in}}%
\pgfpathlineto{\pgfqpoint{5.042193in}{2.770531in}}%
\pgfpathclose%
\pgfusepath{fill}%
\end{pgfscope}%
\begin{pgfscope}%
\pgfpathrectangle{\pgfqpoint{1.254980in}{0.150000in}}{\pgfqpoint{5.490039in}{5.490039in}}%
\pgfusepath{clip}%
\pgfsetbuttcap%
\pgfsetroundjoin%
\definecolor{currentfill}{rgb}{0.243113,0.292092,0.538516}%
\pgfsetfillcolor{currentfill}%
\pgfsetfillopacity{0.700000}%
\pgfsetlinewidth{0.000000pt}%
\definecolor{currentstroke}{rgb}{0.000000,0.000000,0.000000}%
\pgfsetstrokecolor{currentstroke}%
\pgfsetdash{}{0pt}%
\pgfpathmoveto{\pgfqpoint{2.533605in}{1.980151in}}%
\pgfpathlineto{\pgfqpoint{2.547299in}{1.962628in}}%
\pgfpathlineto{\pgfqpoint{2.560985in}{1.945329in}}%
\pgfpathlineto{\pgfqpoint{2.574663in}{1.928252in}}%
\pgfpathlineto{\pgfqpoint{2.588335in}{1.911397in}}%
\pgfpathlineto{\pgfqpoint{2.597196in}{1.905015in}}%
\pgfpathlineto{\pgfqpoint{2.606035in}{1.899030in}}%
\pgfpathlineto{\pgfqpoint{2.614851in}{1.893433in}}%
\pgfpathlineto{\pgfqpoint{2.623645in}{1.888216in}}%
\pgfpathlineto{\pgfqpoint{2.610030in}{1.904339in}}%
\pgfpathlineto{\pgfqpoint{2.596409in}{1.920682in}}%
\pgfpathlineto{\pgfqpoint{2.582780in}{1.937246in}}%
\pgfpathlineto{\pgfqpoint{2.569145in}{1.954034in}}%
\pgfpathlineto{\pgfqpoint{2.560295in}{1.959971in}}%
\pgfpathlineto{\pgfqpoint{2.551422in}{1.966298in}}%
\pgfpathlineto{\pgfqpoint{2.542526in}{1.973022in}}%
\pgfpathlineto{\pgfqpoint{2.533605in}{1.980151in}}%
\pgfpathclose%
\pgfusepath{fill}%
\end{pgfscope}%
\begin{pgfscope}%
\pgfpathrectangle{\pgfqpoint{1.254980in}{0.150000in}}{\pgfqpoint{5.490039in}{5.490039in}}%
\pgfusepath{clip}%
\pgfsetbuttcap%
\pgfsetroundjoin%
\definecolor{currentfill}{rgb}{0.273006,0.204520,0.501721}%
\pgfsetfillcolor{currentfill}%
\pgfsetfillopacity{0.700000}%
\pgfsetlinewidth{0.000000pt}%
\definecolor{currentstroke}{rgb}{0.000000,0.000000,0.000000}%
\pgfsetstrokecolor{currentstroke}%
\pgfsetdash{}{0pt}%
\pgfpathmoveto{\pgfqpoint{2.697471in}{1.784336in}}%
\pgfpathlineto{\pgfqpoint{2.711087in}{1.769404in}}%
\pgfpathlineto{\pgfqpoint{2.724698in}{1.754679in}}%
\pgfpathlineto{\pgfqpoint{2.738304in}{1.740159in}}%
\pgfpathlineto{\pgfqpoint{2.751905in}{1.725844in}}%
\pgfpathlineto{\pgfqpoint{2.760604in}{1.721678in}}%
\pgfpathlineto{\pgfqpoint{2.769281in}{1.717881in}}%
\pgfpathlineto{\pgfqpoint{2.777939in}{1.714448in}}%
\pgfpathlineto{\pgfqpoint{2.786578in}{1.711370in}}%
\pgfpathlineto{\pgfqpoint{2.773027in}{1.724965in}}%
\pgfpathlineto{\pgfqpoint{2.759472in}{1.738763in}}%
\pgfpathlineto{\pgfqpoint{2.745912in}{1.752766in}}%
\pgfpathlineto{\pgfqpoint{2.732348in}{1.766975in}}%
\pgfpathlineto{\pgfqpoint{2.723660in}{1.770762in}}%
\pgfpathlineto{\pgfqpoint{2.714951in}{1.774913in}}%
\pgfpathlineto{\pgfqpoint{2.706222in}{1.779435in}}%
\pgfpathlineto{\pgfqpoint{2.697471in}{1.784336in}}%
\pgfpathclose%
\pgfusepath{fill}%
\end{pgfscope}%
\begin{pgfscope}%
\pgfpathrectangle{\pgfqpoint{1.254980in}{0.150000in}}{\pgfqpoint{5.490039in}{5.490039in}}%
\pgfusepath{clip}%
\pgfsetbuttcap%
\pgfsetroundjoin%
\definecolor{currentfill}{rgb}{0.180653,0.701402,0.488189}%
\pgfsetfillcolor{currentfill}%
\pgfsetfillopacity{0.700000}%
\pgfsetlinewidth{0.000000pt}%
\definecolor{currentstroke}{rgb}{0.000000,0.000000,0.000000}%
\pgfsetstrokecolor{currentstroke}%
\pgfsetdash{}{0pt}%
\pgfpathmoveto{\pgfqpoint{5.245583in}{2.977750in}}%
\pgfpathlineto{\pgfqpoint{5.259892in}{2.992019in}}%
\pgfpathlineto{\pgfqpoint{5.274220in}{3.006452in}}%
\pgfpathlineto{\pgfqpoint{5.288569in}{3.021050in}}%
\pgfpathlineto{\pgfqpoint{5.302938in}{3.035812in}}%
\pgfpathlineto{\pgfqpoint{5.310328in}{3.042712in}}%
\pgfpathlineto{\pgfqpoint{5.317709in}{3.049448in}}%
\pgfpathlineto{\pgfqpoint{5.325080in}{3.056022in}}%
\pgfpathlineto{\pgfqpoint{5.332441in}{3.062435in}}%
\pgfpathlineto{\pgfqpoint{5.318080in}{3.047815in}}%
\pgfpathlineto{\pgfqpoint{5.303741in}{3.033358in}}%
\pgfpathlineto{\pgfqpoint{5.289421in}{3.019066in}}%
\pgfpathlineto{\pgfqpoint{5.275121in}{3.004937in}}%
\pgfpathlineto{\pgfqpoint{5.267751in}{2.998371in}}%
\pgfpathlineto{\pgfqpoint{5.260371in}{2.991652in}}%
\pgfpathlineto{\pgfqpoint{5.252982in}{2.984779in}}%
\pgfpathlineto{\pgfqpoint{5.245583in}{2.977750in}}%
\pgfpathclose%
\pgfusepath{fill}%
\end{pgfscope}%
\begin{pgfscope}%
\pgfpathrectangle{\pgfqpoint{1.254980in}{0.150000in}}{\pgfqpoint{5.490039in}{5.490039in}}%
\pgfusepath{clip}%
\pgfsetbuttcap%
\pgfsetroundjoin%
\definecolor{currentfill}{rgb}{0.140536,0.530132,0.555659}%
\pgfsetfillcolor{currentfill}%
\pgfsetfillopacity{0.700000}%
\pgfsetlinewidth{0.000000pt}%
\definecolor{currentstroke}{rgb}{0.000000,0.000000,0.000000}%
\pgfsetstrokecolor{currentstroke}%
\pgfsetdash{}{0pt}%
\pgfpathmoveto{\pgfqpoint{2.127599in}{2.620221in}}%
\pgfpathlineto{\pgfqpoint{2.141594in}{2.595258in}}%
\pgfpathlineto{\pgfqpoint{2.155575in}{2.570588in}}%
\pgfpathlineto{\pgfqpoint{2.169539in}{2.546207in}}%
\pgfpathlineto{\pgfqpoint{2.183489in}{2.522112in}}%
\pgfpathlineto{\pgfqpoint{2.192765in}{2.511416in}}%
\pgfpathlineto{\pgfqpoint{2.202012in}{2.501158in}}%
\pgfpathlineto{\pgfqpoint{2.211230in}{2.491330in}}%
\pgfpathlineto{\pgfqpoint{2.220420in}{2.481924in}}%
\pgfpathlineto{\pgfqpoint{2.206541in}{2.505274in}}%
\pgfpathlineto{\pgfqpoint{2.192649in}{2.528909in}}%
\pgfpathlineto{\pgfqpoint{2.178741in}{2.552832in}}%
\pgfpathlineto{\pgfqpoint{2.164819in}{2.577044in}}%
\pgfpathlineto{\pgfqpoint{2.155558in}{2.587183in}}%
\pgfpathlineto{\pgfqpoint{2.146268in}{2.597753in}}%
\pgfpathlineto{\pgfqpoint{2.136949in}{2.608763in}}%
\pgfpathlineto{\pgfqpoint{2.127599in}{2.620221in}}%
\pgfpathclose%
\pgfusepath{fill}%
\end{pgfscope}%
\begin{pgfscope}%
\pgfpathrectangle{\pgfqpoint{1.254980in}{0.150000in}}{\pgfqpoint{5.490039in}{5.490039in}}%
\pgfusepath{clip}%
\pgfsetbuttcap%
\pgfsetroundjoin%
\definecolor{currentfill}{rgb}{0.229739,0.322361,0.545706}%
\pgfsetfillcolor{currentfill}%
\pgfsetfillopacity{0.700000}%
\pgfsetlinewidth{0.000000pt}%
\definecolor{currentstroke}{rgb}{0.000000,0.000000,0.000000}%
\pgfsetstrokecolor{currentstroke}%
\pgfsetdash{}{0pt}%
\pgfpathmoveto{\pgfqpoint{2.478749in}{2.052522in}}%
\pgfpathlineto{\pgfqpoint{2.492476in}{2.034084in}}%
\pgfpathlineto{\pgfqpoint{2.506194in}{2.015878in}}%
\pgfpathlineto{\pgfqpoint{2.519903in}{1.997901in}}%
\pgfpathlineto{\pgfqpoint{2.533605in}{1.980151in}}%
\pgfpathlineto{\pgfqpoint{2.542526in}{1.973022in}}%
\pgfpathlineto{\pgfqpoint{2.551422in}{1.966298in}}%
\pgfpathlineto{\pgfqpoint{2.560295in}{1.959971in}}%
\pgfpathlineto{\pgfqpoint{2.569145in}{1.954034in}}%
\pgfpathlineto{\pgfqpoint{2.555502in}{1.971046in}}%
\pgfpathlineto{\pgfqpoint{2.541852in}{1.988284in}}%
\pgfpathlineto{\pgfqpoint{2.528194in}{2.005751in}}%
\pgfpathlineto{\pgfqpoint{2.514528in}{2.023448in}}%
\pgfpathlineto{\pgfqpoint{2.505620in}{2.030111in}}%
\pgfpathlineto{\pgfqpoint{2.496688in}{2.037173in}}%
\pgfpathlineto{\pgfqpoint{2.487731in}{2.044640in}}%
\pgfpathlineto{\pgfqpoint{2.478749in}{2.052522in}}%
\pgfpathclose%
\pgfusepath{fill}%
\end{pgfscope}%
\begin{pgfscope}%
\pgfpathrectangle{\pgfqpoint{1.254980in}{0.150000in}}{\pgfqpoint{5.490039in}{5.490039in}}%
\pgfusepath{clip}%
\pgfsetbuttcap%
\pgfsetroundjoin%
\definecolor{currentfill}{rgb}{0.268510,0.009605,0.335427}%
\pgfsetfillcolor{currentfill}%
\pgfsetfillopacity{0.700000}%
\pgfsetlinewidth{0.000000pt}%
\definecolor{currentstroke}{rgb}{0.000000,0.000000,0.000000}%
\pgfsetstrokecolor{currentstroke}%
\pgfsetdash{}{0pt}%
\pgfpathmoveto{\pgfqpoint{3.533102in}{1.359056in}}%
\pgfpathlineto{\pgfqpoint{3.546592in}{1.356193in}}%
\pgfpathlineto{\pgfqpoint{3.560088in}{1.353494in}}%
\pgfpathlineto{\pgfqpoint{3.573588in}{1.350957in}}%
\pgfpathlineto{\pgfqpoint{3.587095in}{1.348583in}}%
\pgfpathlineto{\pgfqpoint{3.595149in}{1.357227in}}%
\pgfpathlineto{\pgfqpoint{3.603197in}{1.366036in}}%
\pgfpathlineto{\pgfqpoint{3.611236in}{1.375005in}}%
\pgfpathlineto{\pgfqpoint{3.619269in}{1.384128in}}%
\pgfpathlineto{\pgfqpoint{3.605779in}{1.385920in}}%
\pgfpathlineto{\pgfqpoint{3.592295in}{1.387874in}}%
\pgfpathlineto{\pgfqpoint{3.578817in}{1.389991in}}%
\pgfpathlineto{\pgfqpoint{3.565345in}{1.392272in}}%
\pgfpathlineto{\pgfqpoint{3.557296in}{1.383721in}}%
\pgfpathlineto{\pgfqpoint{3.549240in}{1.375330in}}%
\pgfpathlineto{\pgfqpoint{3.541175in}{1.367107in}}%
\pgfpathlineto{\pgfqpoint{3.533102in}{1.359056in}}%
\pgfpathclose%
\pgfusepath{fill}%
\end{pgfscope}%
\begin{pgfscope}%
\pgfpathrectangle{\pgfqpoint{1.254980in}{0.150000in}}{\pgfqpoint{5.490039in}{5.490039in}}%
\pgfusepath{clip}%
\pgfsetbuttcap%
\pgfsetroundjoin%
\definecolor{currentfill}{rgb}{0.278012,0.180367,0.486697}%
\pgfsetfillcolor{currentfill}%
\pgfsetfillopacity{0.700000}%
\pgfsetlinewidth{0.000000pt}%
\definecolor{currentstroke}{rgb}{0.000000,0.000000,0.000000}%
\pgfsetstrokecolor{currentstroke}%
\pgfsetdash{}{0pt}%
\pgfpathmoveto{\pgfqpoint{2.751905in}{1.725844in}}%
\pgfpathlineto{\pgfqpoint{2.765502in}{1.711732in}}%
\pgfpathlineto{\pgfqpoint{2.779094in}{1.697821in}}%
\pgfpathlineto{\pgfqpoint{2.792682in}{1.684111in}}%
\pgfpathlineto{\pgfqpoint{2.806266in}{1.670600in}}%
\pgfpathlineto{\pgfqpoint{2.814915in}{1.667164in}}%
\pgfpathlineto{\pgfqpoint{2.823543in}{1.664090in}}%
\pgfpathlineto{\pgfqpoint{2.832152in}{1.661370in}}%
\pgfpathlineto{\pgfqpoint{2.840743in}{1.658997in}}%
\pgfpathlineto{\pgfqpoint{2.827207in}{1.671792in}}%
\pgfpathlineto{\pgfqpoint{2.813668in}{1.684784in}}%
\pgfpathlineto{\pgfqpoint{2.800125in}{1.697976in}}%
\pgfpathlineto{\pgfqpoint{2.786578in}{1.711370in}}%
\pgfpathlineto{\pgfqpoint{2.777939in}{1.714448in}}%
\pgfpathlineto{\pgfqpoint{2.769281in}{1.717881in}}%
\pgfpathlineto{\pgfqpoint{2.760604in}{1.721678in}}%
\pgfpathlineto{\pgfqpoint{2.751905in}{1.725844in}}%
\pgfpathclose%
\pgfusepath{fill}%
\end{pgfscope}%
\begin{pgfscope}%
\pgfpathrectangle{\pgfqpoint{1.254980in}{0.150000in}}{\pgfqpoint{5.490039in}{5.490039in}}%
\pgfusepath{clip}%
\pgfsetbuttcap%
\pgfsetroundjoin%
\definecolor{currentfill}{rgb}{0.263663,0.237631,0.518762}%
\pgfsetfillcolor{currentfill}%
\pgfsetfillopacity{0.700000}%
\pgfsetlinewidth{0.000000pt}%
\definecolor{currentstroke}{rgb}{0.000000,0.000000,0.000000}%
\pgfsetstrokecolor{currentstroke}%
\pgfsetdash{}{0pt}%
\pgfpathmoveto{\pgfqpoint{4.166347in}{1.766194in}}%
\pgfpathlineto{\pgfqpoint{4.180012in}{1.771718in}}%
\pgfpathlineto{\pgfqpoint{4.193689in}{1.777400in}}%
\pgfpathlineto{\pgfqpoint{4.207377in}{1.783242in}}%
\pgfpathlineto{\pgfqpoint{4.221078in}{1.789242in}}%
\pgfpathlineto{\pgfqpoint{4.228914in}{1.803334in}}%
\pgfpathlineto{\pgfqpoint{4.236746in}{1.817397in}}%
\pgfpathlineto{\pgfqpoint{4.244574in}{1.831427in}}%
\pgfpathlineto{\pgfqpoint{4.252398in}{1.845422in}}%
\pgfpathlineto{\pgfqpoint{4.238697in}{1.839055in}}%
\pgfpathlineto{\pgfqpoint{4.225008in}{1.832847in}}%
\pgfpathlineto{\pgfqpoint{4.211332in}{1.826798in}}%
\pgfpathlineto{\pgfqpoint{4.197668in}{1.820908in}}%
\pgfpathlineto{\pgfqpoint{4.189844in}{1.807270in}}%
\pgfpathlineto{\pgfqpoint{4.182016in}{1.793602in}}%
\pgfpathlineto{\pgfqpoint{4.174184in}{1.779910in}}%
\pgfpathlineto{\pgfqpoint{4.166347in}{1.766194in}}%
\pgfpathclose%
\pgfusepath{fill}%
\end{pgfscope}%
\begin{pgfscope}%
\pgfpathrectangle{\pgfqpoint{1.254980in}{0.150000in}}{\pgfqpoint{5.490039in}{5.490039in}}%
\pgfusepath{clip}%
\pgfsetbuttcap%
\pgfsetroundjoin%
\definecolor{currentfill}{rgb}{0.174274,0.445044,0.557792}%
\pgfsetfillcolor{currentfill}%
\pgfsetfillopacity{0.700000}%
\pgfsetlinewidth{0.000000pt}%
\definecolor{currentstroke}{rgb}{0.000000,0.000000,0.000000}%
\pgfsetstrokecolor{currentstroke}%
\pgfsetdash{}{0pt}%
\pgfpathmoveto{\pgfqpoint{4.604377in}{2.271356in}}%
\pgfpathlineto{\pgfqpoint{4.618272in}{2.281413in}}%
\pgfpathlineto{\pgfqpoint{4.632182in}{2.291631in}}%
\pgfpathlineto{\pgfqpoint{4.646109in}{2.302011in}}%
\pgfpathlineto{\pgfqpoint{4.660052in}{2.312552in}}%
\pgfpathlineto{\pgfqpoint{4.667762in}{2.325500in}}%
\pgfpathlineto{\pgfqpoint{4.675468in}{2.338323in}}%
\pgfpathlineto{\pgfqpoint{4.683167in}{2.351019in}}%
\pgfpathlineto{\pgfqpoint{4.690860in}{2.363587in}}%
\pgfpathlineto{\pgfqpoint{4.676916in}{2.352879in}}%
\pgfpathlineto{\pgfqpoint{4.662988in}{2.342332in}}%
\pgfpathlineto{\pgfqpoint{4.649076in}{2.331946in}}%
\pgfpathlineto{\pgfqpoint{4.635179in}{2.321722in}}%
\pgfpathlineto{\pgfqpoint{4.627487in}{2.309310in}}%
\pgfpathlineto{\pgfqpoint{4.619789in}{2.296777in}}%
\pgfpathlineto{\pgfqpoint{4.612086in}{2.284125in}}%
\pgfpathlineto{\pgfqpoint{4.604377in}{2.271356in}}%
\pgfpathclose%
\pgfusepath{fill}%
\end{pgfscope}%
\begin{pgfscope}%
\pgfpathrectangle{\pgfqpoint{1.254980in}{0.150000in}}{\pgfqpoint{5.490039in}{5.490039in}}%
\pgfusepath{clip}%
\pgfsetbuttcap%
\pgfsetroundjoin%
\definecolor{currentfill}{rgb}{0.282623,0.140926,0.457517}%
\pgfsetfillcolor{currentfill}%
\pgfsetfillopacity{0.700000}%
\pgfsetlinewidth{0.000000pt}%
\definecolor{currentstroke}{rgb}{0.000000,0.000000,0.000000}%
\pgfsetstrokecolor{currentstroke}%
\pgfsetdash{}{0pt}%
\pgfpathmoveto{\pgfqpoint{3.963011in}{1.572842in}}%
\pgfpathlineto{\pgfqpoint{3.976597in}{1.575833in}}%
\pgfpathlineto{\pgfqpoint{3.990192in}{1.578982in}}%
\pgfpathlineto{\pgfqpoint{4.003797in}{1.582290in}}%
\pgfpathlineto{\pgfqpoint{4.017413in}{1.585756in}}%
\pgfpathlineto{\pgfqpoint{4.025303in}{1.598989in}}%
\pgfpathlineto{\pgfqpoint{4.033189in}{1.612253in}}%
\pgfpathlineto{\pgfqpoint{4.041070in}{1.625544in}}%
\pgfpathlineto{\pgfqpoint{4.048947in}{1.638857in}}%
\pgfpathlineto{\pgfqpoint{4.035335in}{1.634942in}}%
\pgfpathlineto{\pgfqpoint{4.021734in}{1.631186in}}%
\pgfpathlineto{\pgfqpoint{4.008142in}{1.627589in}}%
\pgfpathlineto{\pgfqpoint{3.994562in}{1.624151in}}%
\pgfpathlineto{\pgfqpoint{3.986681in}{1.611276in}}%
\pgfpathlineto{\pgfqpoint{3.978796in}{1.598430in}}%
\pgfpathlineto{\pgfqpoint{3.970906in}{1.585617in}}%
\pgfpathlineto{\pgfqpoint{3.963011in}{1.572842in}}%
\pgfpathclose%
\pgfusepath{fill}%
\end{pgfscope}%
\begin{pgfscope}%
\pgfpathrectangle{\pgfqpoint{1.254980in}{0.150000in}}{\pgfqpoint{5.490039in}{5.490039in}}%
\pgfusepath{clip}%
\pgfsetbuttcap%
\pgfsetroundjoin%
\definecolor{currentfill}{rgb}{0.216210,0.351535,0.550627}%
\pgfsetfillcolor{currentfill}%
\pgfsetfillopacity{0.700000}%
\pgfsetlinewidth{0.000000pt}%
\definecolor{currentstroke}{rgb}{0.000000,0.000000,0.000000}%
\pgfsetstrokecolor{currentstroke}%
\pgfsetdash{}{0pt}%
\pgfpathmoveto{\pgfqpoint{2.423752in}{2.128618in}}%
\pgfpathlineto{\pgfqpoint{2.437515in}{2.109239in}}%
\pgfpathlineto{\pgfqpoint{2.451269in}{2.090097in}}%
\pgfpathlineto{\pgfqpoint{2.465014in}{2.071192in}}%
\pgfpathlineto{\pgfqpoint{2.478749in}{2.052522in}}%
\pgfpathlineto{\pgfqpoint{2.487731in}{2.044640in}}%
\pgfpathlineto{\pgfqpoint{2.496688in}{2.037173in}}%
\pgfpathlineto{\pgfqpoint{2.505620in}{2.030111in}}%
\pgfpathlineto{\pgfqpoint{2.514528in}{2.023448in}}%
\pgfpathlineto{\pgfqpoint{2.500853in}{2.041376in}}%
\pgfpathlineto{\pgfqpoint{2.487170in}{2.059537in}}%
\pgfpathlineto{\pgfqpoint{2.473479in}{2.077933in}}%
\pgfpathlineto{\pgfqpoint{2.459778in}{2.096566in}}%
\pgfpathlineto{\pgfqpoint{2.450810in}{2.103961in}}%
\pgfpathlineto{\pgfqpoint{2.441816in}{2.111763in}}%
\pgfpathlineto{\pgfqpoint{2.432797in}{2.119979in}}%
\pgfpathlineto{\pgfqpoint{2.423752in}{2.128618in}}%
\pgfpathclose%
\pgfusepath{fill}%
\end{pgfscope}%
\begin{pgfscope}%
\pgfpathrectangle{\pgfqpoint{1.254980in}{0.150000in}}{\pgfqpoint{5.490039in}{5.490039in}}%
\pgfusepath{clip}%
\pgfsetbuttcap%
\pgfsetroundjoin%
\definecolor{currentfill}{rgb}{0.277018,0.050344,0.375715}%
\pgfsetfillcolor{currentfill}%
\pgfsetfillopacity{0.700000}%
\pgfsetlinewidth{0.000000pt}%
\definecolor{currentstroke}{rgb}{0.000000,0.000000,0.000000}%
\pgfsetstrokecolor{currentstroke}%
\pgfsetdash{}{0pt}%
\pgfpathmoveto{\pgfqpoint{3.111015in}{1.443168in}}%
\pgfpathlineto{\pgfqpoint{3.124519in}{1.434306in}}%
\pgfpathlineto{\pgfqpoint{3.138024in}{1.425622in}}%
\pgfpathlineto{\pgfqpoint{3.151529in}{1.417115in}}%
\pgfpathlineto{\pgfqpoint{3.165035in}{1.408783in}}%
\pgfpathlineto{\pgfqpoint{3.173368in}{1.410882in}}%
\pgfpathlineto{\pgfqpoint{3.181686in}{1.413266in}}%
\pgfpathlineto{\pgfqpoint{3.189992in}{1.415928in}}%
\pgfpathlineto{\pgfqpoint{3.198284in}{1.418862in}}%
\pgfpathlineto{\pgfqpoint{3.184812in}{1.426521in}}%
\pgfpathlineto{\pgfqpoint{3.171340in}{1.434357in}}%
\pgfpathlineto{\pgfqpoint{3.157870in}{1.442368in}}%
\pgfpathlineto{\pgfqpoint{3.144400in}{1.450557in}}%
\pgfpathlineto{\pgfqpoint{3.136075in}{1.448284in}}%
\pgfpathlineto{\pgfqpoint{3.127736in}{1.446291in}}%
\pgfpathlineto{\pgfqpoint{3.119382in}{1.444583in}}%
\pgfpathlineto{\pgfqpoint{3.111015in}{1.443168in}}%
\pgfpathclose%
\pgfusepath{fill}%
\end{pgfscope}%
\begin{pgfscope}%
\pgfpathrectangle{\pgfqpoint{1.254980in}{0.150000in}}{\pgfqpoint{5.490039in}{5.490039in}}%
\pgfusepath{clip}%
\pgfsetbuttcap%
\pgfsetroundjoin%
\definecolor{currentfill}{rgb}{0.360741,0.785964,0.387814}%
\pgfsetfillcolor{currentfill}%
\pgfsetfillopacity{0.700000}%
\pgfsetlinewidth{0.000000pt}%
\definecolor{currentstroke}{rgb}{0.000000,0.000000,0.000000}%
\pgfsetstrokecolor{currentstroke}%
\pgfsetdash{}{0pt}%
\pgfpathmoveto{\pgfqpoint{5.535359in}{3.245656in}}%
\pgfpathlineto{\pgfqpoint{5.549872in}{3.261142in}}%
\pgfpathlineto{\pgfqpoint{5.564407in}{3.276793in}}%
\pgfpathlineto{\pgfqpoint{5.578964in}{3.292609in}}%
\pgfpathlineto{\pgfqpoint{5.593543in}{3.308591in}}%
\pgfpathlineto{\pgfqpoint{5.600733in}{3.312357in}}%
\pgfpathlineto{\pgfqpoint{5.607911in}{3.315975in}}%
\pgfpathlineto{\pgfqpoint{5.615079in}{3.319447in}}%
\pgfpathlineto{\pgfqpoint{5.622236in}{3.322777in}}%
\pgfpathlineto{\pgfqpoint{5.607673in}{3.307066in}}%
\pgfpathlineto{\pgfqpoint{5.593133in}{3.291520in}}%
\pgfpathlineto{\pgfqpoint{5.578614in}{3.276139in}}%
\pgfpathlineto{\pgfqpoint{5.564117in}{3.260922in}}%
\pgfpathlineto{\pgfqpoint{5.556943in}{3.257311in}}%
\pgfpathlineto{\pgfqpoint{5.549759in}{3.253565in}}%
\pgfpathlineto{\pgfqpoint{5.542564in}{3.249681in}}%
\pgfpathlineto{\pgfqpoint{5.535359in}{3.245656in}}%
\pgfpathclose%
\pgfusepath{fill}%
\end{pgfscope}%
\begin{pgfscope}%
\pgfpathrectangle{\pgfqpoint{1.254980in}{0.150000in}}{\pgfqpoint{5.490039in}{5.490039in}}%
\pgfusepath{clip}%
\pgfsetbuttcap%
\pgfsetroundjoin%
\definecolor{currentfill}{rgb}{0.281412,0.155834,0.469201}%
\pgfsetfillcolor{currentfill}%
\pgfsetfillopacity{0.700000}%
\pgfsetlinewidth{0.000000pt}%
\definecolor{currentstroke}{rgb}{0.000000,0.000000,0.000000}%
\pgfsetstrokecolor{currentstroke}%
\pgfsetdash{}{0pt}%
\pgfpathmoveto{\pgfqpoint{2.806266in}{1.670600in}}%
\pgfpathlineto{\pgfqpoint{2.819847in}{1.657287in}}%
\pgfpathlineto{\pgfqpoint{2.833424in}{1.644171in}}%
\pgfpathlineto{\pgfqpoint{2.846997in}{1.631251in}}%
\pgfpathlineto{\pgfqpoint{2.860567in}{1.618526in}}%
\pgfpathlineto{\pgfqpoint{2.869167in}{1.615817in}}%
\pgfpathlineto{\pgfqpoint{2.877748in}{1.613462in}}%
\pgfpathlineto{\pgfqpoint{2.886311in}{1.611453in}}%
\pgfpathlineto{\pgfqpoint{2.894856in}{1.609783in}}%
\pgfpathlineto{\pgfqpoint{2.881332in}{1.621794in}}%
\pgfpathlineto{\pgfqpoint{2.867805in}{1.634000in}}%
\pgfpathlineto{\pgfqpoint{2.854276in}{1.646401in}}%
\pgfpathlineto{\pgfqpoint{2.840743in}{1.658997in}}%
\pgfpathlineto{\pgfqpoint{2.832152in}{1.661370in}}%
\pgfpathlineto{\pgfqpoint{2.823543in}{1.664090in}}%
\pgfpathlineto{\pgfqpoint{2.814915in}{1.667164in}}%
\pgfpathlineto{\pgfqpoint{2.806266in}{1.670600in}}%
\pgfpathclose%
\pgfusepath{fill}%
\end{pgfscope}%
\begin{pgfscope}%
\pgfpathrectangle{\pgfqpoint{1.254980in}{0.150000in}}{\pgfqpoint{5.490039in}{5.490039in}}%
\pgfusepath{clip}%
\pgfsetbuttcap%
\pgfsetroundjoin%
\definecolor{currentfill}{rgb}{0.197636,0.391528,0.554969}%
\pgfsetfillcolor{currentfill}%
\pgfsetfillopacity{0.700000}%
\pgfsetlinewidth{0.000000pt}%
\definecolor{currentstroke}{rgb}{0.000000,0.000000,0.000000}%
\pgfsetstrokecolor{currentstroke}%
\pgfsetdash{}{0pt}%
\pgfpathmoveto{\pgfqpoint{4.487101in}{2.128201in}}%
\pgfpathlineto{\pgfqpoint{4.500931in}{2.137193in}}%
\pgfpathlineto{\pgfqpoint{4.514776in}{2.146345in}}%
\pgfpathlineto{\pgfqpoint{4.528636in}{2.155658in}}%
\pgfpathlineto{\pgfqpoint{4.542511in}{2.165132in}}%
\pgfpathlineto{\pgfqpoint{4.550262in}{2.178789in}}%
\pgfpathlineto{\pgfqpoint{4.558009in}{2.192342in}}%
\pgfpathlineto{\pgfqpoint{4.565750in}{2.205788in}}%
\pgfpathlineto{\pgfqpoint{4.573486in}{2.219125in}}%
\pgfpathlineto{\pgfqpoint{4.559609in}{2.209425in}}%
\pgfpathlineto{\pgfqpoint{4.545747in}{2.199887in}}%
\pgfpathlineto{\pgfqpoint{4.531900in}{2.190508in}}%
\pgfpathlineto{\pgfqpoint{4.518068in}{2.181291in}}%
\pgfpathlineto{\pgfqpoint{4.510334in}{2.168168in}}%
\pgfpathlineto{\pgfqpoint{4.502595in}{2.154944in}}%
\pgfpathlineto{\pgfqpoint{4.494850in}{2.141621in}}%
\pgfpathlineto{\pgfqpoint{4.487101in}{2.128201in}}%
\pgfpathclose%
\pgfusepath{fill}%
\end{pgfscope}%
\begin{pgfscope}%
\pgfpathrectangle{\pgfqpoint{1.254980in}{0.150000in}}{\pgfqpoint{5.490039in}{5.490039in}}%
\pgfusepath{clip}%
\pgfsetbuttcap%
\pgfsetroundjoin%
\definecolor{currentfill}{rgb}{0.123463,0.581687,0.547445}%
\pgfsetfillcolor{currentfill}%
\pgfsetfillopacity{0.700000}%
\pgfsetlinewidth{0.000000pt}%
\definecolor{currentstroke}{rgb}{0.000000,0.000000,0.000000}%
\pgfsetstrokecolor{currentstroke}%
\pgfsetdash{}{0pt}%
\pgfpathmoveto{\pgfqpoint{4.925272in}{2.641000in}}%
\pgfpathlineto{\pgfqpoint{4.939372in}{2.653538in}}%
\pgfpathlineto{\pgfqpoint{4.953491in}{2.666239in}}%
\pgfpathlineto{\pgfqpoint{4.967628in}{2.679103in}}%
\pgfpathlineto{\pgfqpoint{4.981784in}{2.692131in}}%
\pgfpathlineto{\pgfqpoint{4.989362in}{2.702489in}}%
\pgfpathlineto{\pgfqpoint{4.996934in}{2.712688in}}%
\pgfpathlineto{\pgfqpoint{5.004497in}{2.722726in}}%
\pgfpathlineto{\pgfqpoint{5.012052in}{2.732605in}}%
\pgfpathlineto{\pgfqpoint{4.997898in}{2.719562in}}%
\pgfpathlineto{\pgfqpoint{4.983763in}{2.706681in}}%
\pgfpathlineto{\pgfqpoint{4.969646in}{2.693964in}}%
\pgfpathlineto{\pgfqpoint{4.955548in}{2.681410in}}%
\pgfpathlineto{\pgfqpoint{4.947990in}{2.671536in}}%
\pgfpathlineto{\pgfqpoint{4.940425in}{2.661510in}}%
\pgfpathlineto{\pgfqpoint{4.932852in}{2.651331in}}%
\pgfpathlineto{\pgfqpoint{4.925272in}{2.641000in}}%
\pgfpathclose%
\pgfusepath{fill}%
\end{pgfscope}%
\begin{pgfscope}%
\pgfpathrectangle{\pgfqpoint{1.254980in}{0.150000in}}{\pgfqpoint{5.490039in}{5.490039in}}%
\pgfusepath{clip}%
\pgfsetbuttcap%
\pgfsetroundjoin%
\definecolor{currentfill}{rgb}{0.268510,0.009605,0.335427}%
\pgfsetfillcolor{currentfill}%
\pgfsetfillopacity{0.700000}%
\pgfsetlinewidth{0.000000pt}%
\definecolor{currentstroke}{rgb}{0.000000,0.000000,0.000000}%
\pgfsetstrokecolor{currentstroke}%
\pgfsetdash{}{0pt}%
\pgfpathmoveto{\pgfqpoint{3.306124in}{1.363829in}}%
\pgfpathlineto{\pgfqpoint{3.319613in}{1.357721in}}%
\pgfpathlineto{\pgfqpoint{3.333106in}{1.351782in}}%
\pgfpathlineto{\pgfqpoint{3.346601in}{1.346011in}}%
\pgfpathlineto{\pgfqpoint{3.360099in}{1.340409in}}%
\pgfpathlineto{\pgfqpoint{3.368292in}{1.345562in}}%
\pgfpathlineto{\pgfqpoint{3.376474in}{1.350951in}}%
\pgfpathlineto{\pgfqpoint{3.384646in}{1.356570in}}%
\pgfpathlineto{\pgfqpoint{3.392807in}{1.362412in}}%
\pgfpathlineto{\pgfqpoint{3.379335in}{1.367375in}}%
\pgfpathlineto{\pgfqpoint{3.365866in}{1.372505in}}%
\pgfpathlineto{\pgfqpoint{3.352400in}{1.377805in}}%
\pgfpathlineto{\pgfqpoint{3.338937in}{1.383273in}}%
\pgfpathlineto{\pgfqpoint{3.330750in}{1.378060in}}%
\pgfpathlineto{\pgfqpoint{3.322552in}{1.373077in}}%
\pgfpathlineto{\pgfqpoint{3.314344in}{1.368332in}}%
\pgfpathlineto{\pgfqpoint{3.306124in}{1.363829in}}%
\pgfpathclose%
\pgfusepath{fill}%
\end{pgfscope}%
\begin{pgfscope}%
\pgfpathrectangle{\pgfqpoint{1.254980in}{0.150000in}}{\pgfqpoint{5.490039in}{5.490039in}}%
\pgfusepath{clip}%
\pgfsetbuttcap%
\pgfsetroundjoin%
\definecolor{currentfill}{rgb}{0.201239,0.383670,0.554294}%
\pgfsetfillcolor{currentfill}%
\pgfsetfillopacity{0.700000}%
\pgfsetlinewidth{0.000000pt}%
\definecolor{currentstroke}{rgb}{0.000000,0.000000,0.000000}%
\pgfsetstrokecolor{currentstroke}%
\pgfsetdash{}{0pt}%
\pgfpathmoveto{\pgfqpoint{2.368598in}{2.208559in}}%
\pgfpathlineto{\pgfqpoint{2.382402in}{2.188207in}}%
\pgfpathlineto{\pgfqpoint{2.396196in}{2.168101in}}%
\pgfpathlineto{\pgfqpoint{2.409979in}{2.148238in}}%
\pgfpathlineto{\pgfqpoint{2.423752in}{2.128618in}}%
\pgfpathlineto{\pgfqpoint{2.432797in}{2.119979in}}%
\pgfpathlineto{\pgfqpoint{2.441816in}{2.111763in}}%
\pgfpathlineto{\pgfqpoint{2.450810in}{2.103961in}}%
\pgfpathlineto{\pgfqpoint{2.459778in}{2.096566in}}%
\pgfpathlineto{\pgfqpoint{2.446068in}{2.115439in}}%
\pgfpathlineto{\pgfqpoint{2.432349in}{2.134551in}}%
\pgfpathlineto{\pgfqpoint{2.418620in}{2.153907in}}%
\pgfpathlineto{\pgfqpoint{2.404881in}{2.173506in}}%
\pgfpathlineto{\pgfqpoint{2.395850in}{2.181637in}}%
\pgfpathlineto{\pgfqpoint{2.386793in}{2.190185in}}%
\pgfpathlineto{\pgfqpoint{2.377709in}{2.199156in}}%
\pgfpathlineto{\pgfqpoint{2.368598in}{2.208559in}}%
\pgfpathclose%
\pgfusepath{fill}%
\end{pgfscope}%
\begin{pgfscope}%
\pgfpathrectangle{\pgfqpoint{1.254980in}{0.150000in}}{\pgfqpoint{5.490039in}{5.490039in}}%
\pgfusepath{clip}%
\pgfsetbuttcap%
\pgfsetroundjoin%
\definecolor{currentfill}{rgb}{0.221989,0.339161,0.548752}%
\pgfsetfillcolor{currentfill}%
\pgfsetfillopacity{0.700000}%
\pgfsetlinewidth{0.000000pt}%
\definecolor{currentstroke}{rgb}{0.000000,0.000000,0.000000}%
\pgfsetstrokecolor{currentstroke}%
\pgfsetdash{}{0pt}%
\pgfpathmoveto{\pgfqpoint{4.369773in}{1.985371in}}%
\pgfpathlineto{\pgfqpoint{4.383542in}{1.993186in}}%
\pgfpathlineto{\pgfqpoint{4.397325in}{2.001160in}}%
\pgfpathlineto{\pgfqpoint{4.411122in}{2.009293in}}%
\pgfpathlineto{\pgfqpoint{4.424933in}{2.017587in}}%
\pgfpathlineto{\pgfqpoint{4.432720in}{2.031708in}}%
\pgfpathlineto{\pgfqpoint{4.440503in}{2.045750in}}%
\pgfpathlineto{\pgfqpoint{4.448281in}{2.059711in}}%
\pgfpathlineto{\pgfqpoint{4.456055in}{2.073587in}}%
\pgfpathlineto{\pgfqpoint{4.442242in}{2.065009in}}%
\pgfpathlineto{\pgfqpoint{4.428443in}{2.056592in}}%
\pgfpathlineto{\pgfqpoint{4.414658in}{2.048335in}}%
\pgfpathlineto{\pgfqpoint{4.400887in}{2.040238in}}%
\pgfpathlineto{\pgfqpoint{4.393115in}{2.026635in}}%
\pgfpathlineto{\pgfqpoint{4.385339in}{2.012954in}}%
\pgfpathlineto{\pgfqpoint{4.377558in}{1.999199in}}%
\pgfpathlineto{\pgfqpoint{4.369773in}{1.985371in}}%
\pgfpathclose%
\pgfusepath{fill}%
\end{pgfscope}%
\begin{pgfscope}%
\pgfpathrectangle{\pgfqpoint{1.254980in}{0.150000in}}{\pgfqpoint{5.490039in}{5.490039in}}%
\pgfusepath{clip}%
\pgfsetbuttcap%
\pgfsetroundjoin%
\definecolor{currentfill}{rgb}{0.282884,0.135920,0.453427}%
\pgfsetfillcolor{currentfill}%
\pgfsetfillopacity{0.700000}%
\pgfsetlinewidth{0.000000pt}%
\definecolor{currentstroke}{rgb}{0.000000,0.000000,0.000000}%
\pgfsetstrokecolor{currentstroke}%
\pgfsetdash{}{0pt}%
\pgfpathmoveto{\pgfqpoint{2.860567in}{1.618526in}}%
\pgfpathlineto{\pgfqpoint{2.874134in}{1.605995in}}%
\pgfpathlineto{\pgfqpoint{2.887699in}{1.593656in}}%
\pgfpathlineto{\pgfqpoint{2.901260in}{1.581508in}}%
\pgfpathlineto{\pgfqpoint{2.914819in}{1.569551in}}%
\pgfpathlineto{\pgfqpoint{2.923373in}{1.567567in}}%
\pgfpathlineto{\pgfqpoint{2.931909in}{1.565927in}}%
\pgfpathlineto{\pgfqpoint{2.940427in}{1.564626in}}%
\pgfpathlineto{\pgfqpoint{2.948929in}{1.563655in}}%
\pgfpathlineto{\pgfqpoint{2.935414in}{1.574901in}}%
\pgfpathlineto{\pgfqpoint{2.921897in}{1.586337in}}%
\pgfpathlineto{\pgfqpoint{2.908378in}{1.597964in}}%
\pgfpathlineto{\pgfqpoint{2.894856in}{1.609783in}}%
\pgfpathlineto{\pgfqpoint{2.886311in}{1.611453in}}%
\pgfpathlineto{\pgfqpoint{2.877748in}{1.613462in}}%
\pgfpathlineto{\pgfqpoint{2.869167in}{1.615817in}}%
\pgfpathlineto{\pgfqpoint{2.860567in}{1.618526in}}%
\pgfpathclose%
\pgfusepath{fill}%
\end{pgfscope}%
\begin{pgfscope}%
\pgfpathrectangle{\pgfqpoint{1.254980in}{0.150000in}}{\pgfqpoint{5.490039in}{5.490039in}}%
\pgfusepath{clip}%
\pgfsetbuttcap%
\pgfsetroundjoin%
\definecolor{currentfill}{rgb}{0.267004,0.004874,0.329415}%
\pgfsetfillcolor{currentfill}%
\pgfsetfillopacity{0.700000}%
\pgfsetlinewidth{0.000000pt}%
\definecolor{currentstroke}{rgb}{0.000000,0.000000,0.000000}%
\pgfsetstrokecolor{currentstroke}%
\pgfsetdash{}{0pt}%
\pgfpathmoveto{\pgfqpoint{3.446734in}{1.344231in}}%
\pgfpathlineto{\pgfqpoint{3.460226in}{1.340100in}}%
\pgfpathlineto{\pgfqpoint{3.473722in}{1.336135in}}%
\pgfpathlineto{\pgfqpoint{3.487222in}{1.332334in}}%
\pgfpathlineto{\pgfqpoint{3.500727in}{1.328697in}}%
\pgfpathlineto{\pgfqpoint{3.508834in}{1.335998in}}%
\pgfpathlineto{\pgfqpoint{3.516932in}{1.343496in}}%
\pgfpathlineto{\pgfqpoint{3.525021in}{1.351184in}}%
\pgfpathlineto{\pgfqpoint{3.533102in}{1.359056in}}%
\pgfpathlineto{\pgfqpoint{3.519617in}{1.362083in}}%
\pgfpathlineto{\pgfqpoint{3.506138in}{1.365273in}}%
\pgfpathlineto{\pgfqpoint{3.492663in}{1.368628in}}%
\pgfpathlineto{\pgfqpoint{3.479192in}{1.372148in}}%
\pgfpathlineto{\pgfqpoint{3.471091in}{1.364876in}}%
\pgfpathlineto{\pgfqpoint{3.462981in}{1.357794in}}%
\pgfpathlineto{\pgfqpoint{3.454862in}{1.350911in}}%
\pgfpathlineto{\pgfqpoint{3.446734in}{1.344231in}}%
\pgfpathclose%
\pgfusepath{fill}%
\end{pgfscope}%
\begin{pgfscope}%
\pgfpathrectangle{\pgfqpoint{1.254980in}{0.150000in}}{\pgfqpoint{5.490039in}{5.490039in}}%
\pgfusepath{clip}%
\pgfsetbuttcap%
\pgfsetroundjoin%
\definecolor{currentfill}{rgb}{0.278826,0.175490,0.483397}%
\pgfsetfillcolor{currentfill}%
\pgfsetfillopacity{0.700000}%
\pgfsetlinewidth{0.000000pt}%
\definecolor{currentstroke}{rgb}{0.000000,0.000000,0.000000}%
\pgfsetstrokecolor{currentstroke}%
\pgfsetdash{}{0pt}%
\pgfpathmoveto{\pgfqpoint{4.048947in}{1.638857in}}%
\pgfpathlineto{\pgfqpoint{4.062570in}{1.642930in}}%
\pgfpathlineto{\pgfqpoint{4.076204in}{1.647162in}}%
\pgfpathlineto{\pgfqpoint{4.089848in}{1.651552in}}%
\pgfpathlineto{\pgfqpoint{4.103503in}{1.656101in}}%
\pgfpathlineto{\pgfqpoint{4.111374in}{1.669864in}}%
\pgfpathlineto{\pgfqpoint{4.119240in}{1.683634in}}%
\pgfpathlineto{\pgfqpoint{4.127102in}{1.697408in}}%
\pgfpathlineto{\pgfqpoint{4.134959in}{1.711182in}}%
\pgfpathlineto{\pgfqpoint{4.121305in}{1.706211in}}%
\pgfpathlineto{\pgfqpoint{4.107663in}{1.701399in}}%
\pgfpathlineto{\pgfqpoint{4.094032in}{1.696746in}}%
\pgfpathlineto{\pgfqpoint{4.080411in}{1.692251in}}%
\pgfpathlineto{\pgfqpoint{4.072552in}{1.678889in}}%
\pgfpathlineto{\pgfqpoint{4.064688in}{1.665533in}}%
\pgfpathlineto{\pgfqpoint{4.056820in}{1.652188in}}%
\pgfpathlineto{\pgfqpoint{4.048947in}{1.638857in}}%
\pgfpathclose%
\pgfusepath{fill}%
\end{pgfscope}%
\begin{pgfscope}%
\pgfpathrectangle{\pgfqpoint{1.254980in}{0.150000in}}{\pgfqpoint{5.490039in}{5.490039in}}%
\pgfusepath{clip}%
\pgfsetbuttcap%
\pgfsetroundjoin%
\definecolor{currentfill}{rgb}{0.126453,0.570633,0.549841}%
\pgfsetfillcolor{currentfill}%
\pgfsetfillopacity{0.700000}%
\pgfsetlinewidth{0.000000pt}%
\definecolor{currentstroke}{rgb}{0.000000,0.000000,0.000000}%
\pgfsetstrokecolor{currentstroke}%
\pgfsetdash{}{0pt}%
\pgfpathmoveto{\pgfqpoint{2.071452in}{2.723050in}}%
\pgfpathlineto{\pgfqpoint{2.085514in}{2.696890in}}%
\pgfpathlineto{\pgfqpoint{2.099559in}{2.671034in}}%
\pgfpathlineto{\pgfqpoint{2.113587in}{2.645479in}}%
\pgfpathlineto{\pgfqpoint{2.127599in}{2.620221in}}%
\pgfpathlineto{\pgfqpoint{2.136949in}{2.608763in}}%
\pgfpathlineto{\pgfqpoint{2.146268in}{2.597753in}}%
\pgfpathlineto{\pgfqpoint{2.155558in}{2.587183in}}%
\pgfpathlineto{\pgfqpoint{2.164819in}{2.577044in}}%
\pgfpathlineto{\pgfqpoint{2.150881in}{2.601550in}}%
\pgfpathlineto{\pgfqpoint{2.136927in}{2.626351in}}%
\pgfpathlineto{\pgfqpoint{2.122957in}{2.651451in}}%
\pgfpathlineto{\pgfqpoint{2.108971in}{2.676852in}}%
\pgfpathlineto{\pgfqpoint{2.099638in}{2.687732in}}%
\pgfpathlineto{\pgfqpoint{2.090274in}{2.699052in}}%
\pgfpathlineto{\pgfqpoint{2.080879in}{2.710823in}}%
\pgfpathlineto{\pgfqpoint{2.071452in}{2.723050in}}%
\pgfpathclose%
\pgfusepath{fill}%
\end{pgfscope}%
\begin{pgfscope}%
\pgfpathrectangle{\pgfqpoint{1.254980in}{0.150000in}}{\pgfqpoint{5.490039in}{5.490039in}}%
\pgfusepath{clip}%
\pgfsetbuttcap%
\pgfsetroundjoin%
\definecolor{currentfill}{rgb}{0.232815,0.732247,0.459277}%
\pgfsetfillcolor{currentfill}%
\pgfsetfillopacity{0.700000}%
\pgfsetlinewidth{0.000000pt}%
\definecolor{currentstroke}{rgb}{0.000000,0.000000,0.000000}%
\pgfsetstrokecolor{currentstroke}%
\pgfsetdash{}{0pt}%
\pgfpathmoveto{\pgfqpoint{5.332441in}{3.062435in}}%
\pgfpathlineto{\pgfqpoint{5.346822in}{3.077221in}}%
\pgfpathlineto{\pgfqpoint{5.361224in}{3.092170in}}%
\pgfpathlineto{\pgfqpoint{5.375646in}{3.107285in}}%
\pgfpathlineto{\pgfqpoint{5.390090in}{3.122565in}}%
\pgfpathlineto{\pgfqpoint{5.397432in}{3.128658in}}%
\pgfpathlineto{\pgfqpoint{5.404763in}{3.134587in}}%
\pgfpathlineto{\pgfqpoint{5.412084in}{3.140354in}}%
\pgfpathlineto{\pgfqpoint{5.419395in}{3.145961in}}%
\pgfpathlineto{\pgfqpoint{5.404962in}{3.130855in}}%
\pgfpathlineto{\pgfqpoint{5.390550in}{3.115914in}}%
\pgfpathlineto{\pgfqpoint{5.376159in}{3.101138in}}%
\pgfpathlineto{\pgfqpoint{5.361788in}{3.086525in}}%
\pgfpathlineto{\pgfqpoint{5.354466in}{3.080733in}}%
\pgfpathlineto{\pgfqpoint{5.347134in}{3.074789in}}%
\pgfpathlineto{\pgfqpoint{5.339792in}{3.068690in}}%
\pgfpathlineto{\pgfqpoint{5.332441in}{3.062435in}}%
\pgfpathclose%
\pgfusepath{fill}%
\end{pgfscope}%
\begin{pgfscope}%
\pgfpathrectangle{\pgfqpoint{1.254980in}{0.150000in}}{\pgfqpoint{5.490039in}{5.490039in}}%
\pgfusepath{clip}%
\pgfsetbuttcap%
\pgfsetroundjoin%
\definecolor{currentfill}{rgb}{0.137339,0.662252,0.515571}%
\pgfsetfillcolor{currentfill}%
\pgfsetfillopacity{0.700000}%
\pgfsetlinewidth{0.000000pt}%
\definecolor{currentstroke}{rgb}{0.000000,0.000000,0.000000}%
\pgfsetstrokecolor{currentstroke}%
\pgfsetdash{}{0pt}%
\pgfpathmoveto{\pgfqpoint{5.128979in}{2.859510in}}%
\pgfpathlineto{\pgfqpoint{5.143221in}{2.873310in}}%
\pgfpathlineto{\pgfqpoint{5.157483in}{2.887274in}}%
\pgfpathlineto{\pgfqpoint{5.171764in}{2.901402in}}%
\pgfpathlineto{\pgfqpoint{5.186065in}{2.915694in}}%
\pgfpathlineto{\pgfqpoint{5.193536in}{2.924030in}}%
\pgfpathlineto{\pgfqpoint{5.200999in}{2.932198in}}%
\pgfpathlineto{\pgfqpoint{5.208453in}{2.940199in}}%
\pgfpathlineto{\pgfqpoint{5.215897in}{2.948035in}}%
\pgfpathlineto{\pgfqpoint{5.201602in}{2.933820in}}%
\pgfpathlineto{\pgfqpoint{5.187326in}{2.919770in}}%
\pgfpathlineto{\pgfqpoint{5.173071in}{2.905884in}}%
\pgfpathlineto{\pgfqpoint{5.158834in}{2.892161in}}%
\pgfpathlineto{\pgfqpoint{5.151384in}{2.884236in}}%
\pgfpathlineto{\pgfqpoint{5.143924in}{2.876153in}}%
\pgfpathlineto{\pgfqpoint{5.136456in}{2.867912in}}%
\pgfpathlineto{\pgfqpoint{5.128979in}{2.859510in}}%
\pgfpathclose%
\pgfusepath{fill}%
\end{pgfscope}%
\begin{pgfscope}%
\pgfpathrectangle{\pgfqpoint{1.254980in}{0.150000in}}{\pgfqpoint{5.490039in}{5.490039in}}%
\pgfusepath{clip}%
\pgfsetbuttcap%
\pgfsetroundjoin%
\definecolor{currentfill}{rgb}{0.137770,0.537492,0.554906}%
\pgfsetfillcolor{currentfill}%
\pgfsetfillopacity{0.700000}%
\pgfsetlinewidth{0.000000pt}%
\definecolor{currentstroke}{rgb}{0.000000,0.000000,0.000000}%
\pgfsetstrokecolor{currentstroke}%
\pgfsetdash{}{0pt}%
\pgfpathmoveto{\pgfqpoint{4.808140in}{2.504831in}}%
\pgfpathlineto{\pgfqpoint{4.822170in}{2.516595in}}%
\pgfpathlineto{\pgfqpoint{4.836218in}{2.528520in}}%
\pgfpathlineto{\pgfqpoint{4.850284in}{2.540608in}}%
\pgfpathlineto{\pgfqpoint{4.864367in}{2.552859in}}%
\pgfpathlineto{\pgfqpoint{4.872005in}{2.564409in}}%
\pgfpathlineto{\pgfqpoint{4.879636in}{2.575808in}}%
\pgfpathlineto{\pgfqpoint{4.887260in}{2.587054in}}%
\pgfpathlineto{\pgfqpoint{4.894877in}{2.598149in}}%
\pgfpathlineto{\pgfqpoint{4.880793in}{2.585820in}}%
\pgfpathlineto{\pgfqpoint{4.866728in}{2.573655in}}%
\pgfpathlineto{\pgfqpoint{4.852680in}{2.561652in}}%
\pgfpathlineto{\pgfqpoint{4.838649in}{2.549811in}}%
\pgfpathlineto{\pgfqpoint{4.831032in}{2.538782in}}%
\pgfpathlineto{\pgfqpoint{4.823408in}{2.527610in}}%
\pgfpathlineto{\pgfqpoint{4.815777in}{2.516292in}}%
\pgfpathlineto{\pgfqpoint{4.808140in}{2.504831in}}%
\pgfpathclose%
\pgfusepath{fill}%
\end{pgfscope}%
\begin{pgfscope}%
\pgfpathrectangle{\pgfqpoint{1.254980in}{0.150000in}}{\pgfqpoint{5.490039in}{5.490039in}}%
\pgfusepath{clip}%
\pgfsetbuttcap%
\pgfsetroundjoin%
\definecolor{currentfill}{rgb}{0.185556,0.418570,0.556753}%
\pgfsetfillcolor{currentfill}%
\pgfsetfillopacity{0.700000}%
\pgfsetlinewidth{0.000000pt}%
\definecolor{currentstroke}{rgb}{0.000000,0.000000,0.000000}%
\pgfsetstrokecolor{currentstroke}%
\pgfsetdash{}{0pt}%
\pgfpathmoveto{\pgfqpoint{2.313271in}{2.292471in}}%
\pgfpathlineto{\pgfqpoint{2.327120in}{2.271114in}}%
\pgfpathlineto{\pgfqpoint{2.340957in}{2.250011in}}%
\pgfpathlineto{\pgfqpoint{2.354783in}{2.229160in}}%
\pgfpathlineto{\pgfqpoint{2.368598in}{2.208559in}}%
\pgfpathlineto{\pgfqpoint{2.377709in}{2.199156in}}%
\pgfpathlineto{\pgfqpoint{2.386793in}{2.190185in}}%
\pgfpathlineto{\pgfqpoint{2.395850in}{2.181637in}}%
\pgfpathlineto{\pgfqpoint{2.404881in}{2.173506in}}%
\pgfpathlineto{\pgfqpoint{2.391132in}{2.193353in}}%
\pgfpathlineto{\pgfqpoint{2.377373in}{2.213448in}}%
\pgfpathlineto{\pgfqpoint{2.363603in}{2.233794in}}%
\pgfpathlineto{\pgfqpoint{2.349821in}{2.254392in}}%
\pgfpathlineto{\pgfqpoint{2.340725in}{2.263266in}}%
\pgfpathlineto{\pgfqpoint{2.331602in}{2.272565in}}%
\pgfpathlineto{\pgfqpoint{2.322451in}{2.282297in}}%
\pgfpathlineto{\pgfqpoint{2.313271in}{2.292471in}}%
\pgfpathclose%
\pgfusepath{fill}%
\end{pgfscope}%
\begin{pgfscope}%
\pgfpathrectangle{\pgfqpoint{1.254980in}{0.150000in}}{\pgfqpoint{5.490039in}{5.490039in}}%
\pgfusepath{clip}%
\pgfsetbuttcap%
\pgfsetroundjoin%
\definecolor{currentfill}{rgb}{0.248629,0.278775,0.534556}%
\pgfsetfillcolor{currentfill}%
\pgfsetfillopacity{0.700000}%
\pgfsetlinewidth{0.000000pt}%
\definecolor{currentstroke}{rgb}{0.000000,0.000000,0.000000}%
\pgfsetstrokecolor{currentstroke}%
\pgfsetdash{}{0pt}%
\pgfpathmoveto{\pgfqpoint{4.252398in}{1.845422in}}%
\pgfpathlineto{\pgfqpoint{4.266112in}{1.851948in}}%
\pgfpathlineto{\pgfqpoint{4.279838in}{1.858633in}}%
\pgfpathlineto{\pgfqpoint{4.293578in}{1.865477in}}%
\pgfpathlineto{\pgfqpoint{4.307330in}{1.872480in}}%
\pgfpathlineto{\pgfqpoint{4.315150in}{1.886786in}}%
\pgfpathlineto{\pgfqpoint{4.322967in}{1.901042in}}%
\pgfpathlineto{\pgfqpoint{4.330779in}{1.915245in}}%
\pgfpathlineto{\pgfqpoint{4.338586in}{1.929393in}}%
\pgfpathlineto{\pgfqpoint{4.324832in}{1.922050in}}%
\pgfpathlineto{\pgfqpoint{4.311091in}{1.914866in}}%
\pgfpathlineto{\pgfqpoint{4.297364in}{1.907842in}}%
\pgfpathlineto{\pgfqpoint{4.283650in}{1.900976in}}%
\pgfpathlineto{\pgfqpoint{4.275843in}{1.887157in}}%
\pgfpathlineto{\pgfqpoint{4.268032in}{1.873290in}}%
\pgfpathlineto{\pgfqpoint{4.260217in}{1.859377in}}%
\pgfpathlineto{\pgfqpoint{4.252398in}{1.845422in}}%
\pgfpathclose%
\pgfusepath{fill}%
\end{pgfscope}%
\begin{pgfscope}%
\pgfpathrectangle{\pgfqpoint{1.254980in}{0.150000in}}{\pgfqpoint{5.490039in}{5.490039in}}%
\pgfusepath{clip}%
\pgfsetbuttcap%
\pgfsetroundjoin%
\definecolor{currentfill}{rgb}{0.274952,0.037752,0.364543}%
\pgfsetfillcolor{currentfill}%
\pgfsetfillopacity{0.700000}%
\pgfsetlinewidth{0.000000pt}%
\definecolor{currentstroke}{rgb}{0.000000,0.000000,0.000000}%
\pgfsetstrokecolor{currentstroke}%
\pgfsetdash{}{0pt}%
\pgfpathmoveto{\pgfqpoint{3.165035in}{1.408783in}}%
\pgfpathlineto{\pgfqpoint{3.178542in}{1.400627in}}%
\pgfpathlineto{\pgfqpoint{3.192050in}{1.392646in}}%
\pgfpathlineto{\pgfqpoint{3.205559in}{1.384838in}}%
\pgfpathlineto{\pgfqpoint{3.219069in}{1.377204in}}%
\pgfpathlineto{\pgfqpoint{3.227368in}{1.379985in}}%
\pgfpathlineto{\pgfqpoint{3.235654in}{1.383044in}}%
\pgfpathlineto{\pgfqpoint{3.243928in}{1.386373in}}%
\pgfpathlineto{\pgfqpoint{3.252189in}{1.389967in}}%
\pgfpathlineto{\pgfqpoint{3.238710in}{1.396930in}}%
\pgfpathlineto{\pgfqpoint{3.225233in}{1.404067in}}%
\pgfpathlineto{\pgfqpoint{3.211758in}{1.411377in}}%
\pgfpathlineto{\pgfqpoint{3.198284in}{1.418862in}}%
\pgfpathlineto{\pgfqpoint{3.189992in}{1.415928in}}%
\pgfpathlineto{\pgfqpoint{3.181686in}{1.413266in}}%
\pgfpathlineto{\pgfqpoint{3.173368in}{1.410882in}}%
\pgfpathlineto{\pgfqpoint{3.165035in}{1.408783in}}%
\pgfpathclose%
\pgfusepath{fill}%
\end{pgfscope}%
\begin{pgfscope}%
\pgfpathrectangle{\pgfqpoint{1.254980in}{0.150000in}}{\pgfqpoint{5.490039in}{5.490039in}}%
\pgfusepath{clip}%
\pgfsetbuttcap%
\pgfsetroundjoin%
\definecolor{currentfill}{rgb}{0.440137,0.811138,0.340967}%
\pgfsetfillcolor{currentfill}%
\pgfsetfillopacity{0.700000}%
\pgfsetlinewidth{0.000000pt}%
\definecolor{currentstroke}{rgb}{0.000000,0.000000,0.000000}%
\pgfsetstrokecolor{currentstroke}%
\pgfsetdash{}{0pt}%
\pgfpathmoveto{\pgfqpoint{5.622236in}{3.322777in}}%
\pgfpathlineto{\pgfqpoint{5.636821in}{3.338652in}}%
\pgfpathlineto{\pgfqpoint{5.651428in}{3.354693in}}%
\pgfpathlineto{\pgfqpoint{5.666057in}{3.370900in}}%
\pgfpathlineto{\pgfqpoint{5.680709in}{3.387272in}}%
\pgfpathlineto{\pgfqpoint{5.687837in}{3.390171in}}%
\pgfpathlineto{\pgfqpoint{5.694954in}{3.392926in}}%
\pgfpathlineto{\pgfqpoint{5.702060in}{3.395541in}}%
\pgfpathlineto{\pgfqpoint{5.709155in}{3.398020in}}%
\pgfpathlineto{\pgfqpoint{5.694522in}{3.381952in}}%
\pgfpathlineto{\pgfqpoint{5.679912in}{3.366049in}}%
\pgfpathlineto{\pgfqpoint{5.665323in}{3.350311in}}%
\pgfpathlineto{\pgfqpoint{5.650757in}{3.334737in}}%
\pgfpathlineto{\pgfqpoint{5.643643in}{3.331944in}}%
\pgfpathlineto{\pgfqpoint{5.636518in}{3.329022in}}%
\pgfpathlineto{\pgfqpoint{5.629382in}{3.325967in}}%
\pgfpathlineto{\pgfqpoint{5.622236in}{3.322777in}}%
\pgfpathclose%
\pgfusepath{fill}%
\end{pgfscope}%
\begin{pgfscope}%
\pgfpathrectangle{\pgfqpoint{1.254980in}{0.150000in}}{\pgfqpoint{5.490039in}{5.490039in}}%
\pgfusepath{clip}%
\pgfsetbuttcap%
\pgfsetroundjoin%
\definecolor{currentfill}{rgb}{0.278791,0.062145,0.386592}%
\pgfsetfillcolor{currentfill}%
\pgfsetfillopacity{0.700000}%
\pgfsetlinewidth{0.000000pt}%
\definecolor{currentstroke}{rgb}{0.000000,0.000000,0.000000}%
\pgfsetstrokecolor{currentstroke}%
\pgfsetdash{}{0pt}%
\pgfpathmoveto{\pgfqpoint{3.759369in}{1.417970in}}%
\pgfpathlineto{\pgfqpoint{3.772907in}{1.418182in}}%
\pgfpathlineto{\pgfqpoint{3.786453in}{1.418553in}}%
\pgfpathlineto{\pgfqpoint{3.800007in}{1.419084in}}%
\pgfpathlineto{\pgfqpoint{3.813569in}{1.419773in}}%
\pgfpathlineto{\pgfqpoint{3.821531in}{1.431200in}}%
\pgfpathlineto{\pgfqpoint{3.829488in}{1.442728in}}%
\pgfpathlineto{\pgfqpoint{3.837440in}{1.454351in}}%
\pgfpathlineto{\pgfqpoint{3.845386in}{1.466066in}}%
\pgfpathlineto{\pgfqpoint{3.831833in}{1.464847in}}%
\pgfpathlineto{\pgfqpoint{3.818289in}{1.463787in}}%
\pgfpathlineto{\pgfqpoint{3.804753in}{1.462887in}}%
\pgfpathlineto{\pgfqpoint{3.791225in}{1.462146in}}%
\pgfpathlineto{\pgfqpoint{3.783270in}{1.450951in}}%
\pgfpathlineto{\pgfqpoint{3.775309in}{1.439852in}}%
\pgfpathlineto{\pgfqpoint{3.767342in}{1.428857in}}%
\pgfpathlineto{\pgfqpoint{3.759369in}{1.417970in}}%
\pgfpathclose%
\pgfusepath{fill}%
\end{pgfscope}%
\begin{pgfscope}%
\pgfpathrectangle{\pgfqpoint{1.254980in}{0.150000in}}{\pgfqpoint{5.490039in}{5.490039in}}%
\pgfusepath{clip}%
\pgfsetbuttcap%
\pgfsetroundjoin%
\definecolor{currentfill}{rgb}{0.283197,0.115680,0.436115}%
\pgfsetfillcolor{currentfill}%
\pgfsetfillopacity{0.700000}%
\pgfsetlinewidth{0.000000pt}%
\definecolor{currentstroke}{rgb}{0.000000,0.000000,0.000000}%
\pgfsetstrokecolor{currentstroke}%
\pgfsetdash{}{0pt}%
\pgfpathmoveto{\pgfqpoint{2.914819in}{1.569551in}}%
\pgfpathlineto{\pgfqpoint{2.928376in}{1.557784in}}%
\pgfpathlineto{\pgfqpoint{2.941931in}{1.546205in}}%
\pgfpathlineto{\pgfqpoint{2.955483in}{1.534814in}}%
\pgfpathlineto{\pgfqpoint{2.969034in}{1.523609in}}%
\pgfpathlineto{\pgfqpoint{2.977544in}{1.522345in}}%
\pgfpathlineto{\pgfqpoint{2.986036in}{1.521419in}}%
\pgfpathlineto{\pgfqpoint{2.994512in}{1.520823in}}%
\pgfpathlineto{\pgfqpoint{3.002972in}{1.520550in}}%
\pgfpathlineto{\pgfqpoint{2.989463in}{1.531047in}}%
\pgfpathlineto{\pgfqpoint{2.975953in}{1.541729in}}%
\pgfpathlineto{\pgfqpoint{2.962442in}{1.552598in}}%
\pgfpathlineto{\pgfqpoint{2.948929in}{1.563655in}}%
\pgfpathlineto{\pgfqpoint{2.940427in}{1.564626in}}%
\pgfpathlineto{\pgfqpoint{2.931909in}{1.565927in}}%
\pgfpathlineto{\pgfqpoint{2.923373in}{1.567567in}}%
\pgfpathlineto{\pgfqpoint{2.914819in}{1.569551in}}%
\pgfpathclose%
\pgfusepath{fill}%
\end{pgfscope}%
\begin{pgfscope}%
\pgfpathrectangle{\pgfqpoint{1.254980in}{0.150000in}}{\pgfqpoint{5.490039in}{5.490039in}}%
\pgfusepath{clip}%
\pgfsetbuttcap%
\pgfsetroundjoin%
\definecolor{currentfill}{rgb}{0.274952,0.037752,0.364543}%
\pgfsetfillcolor{currentfill}%
\pgfsetfillopacity{0.700000}%
\pgfsetlinewidth{0.000000pt}%
\definecolor{currentstroke}{rgb}{0.000000,0.000000,0.000000}%
\pgfsetstrokecolor{currentstroke}%
\pgfsetdash{}{0pt}%
\pgfpathmoveto{\pgfqpoint{3.673290in}{1.378580in}}%
\pgfpathlineto{\pgfqpoint{3.686811in}{1.377595in}}%
\pgfpathlineto{\pgfqpoint{3.700339in}{1.376771in}}%
\pgfpathlineto{\pgfqpoint{3.713875in}{1.376108in}}%
\pgfpathlineto{\pgfqpoint{3.727417in}{1.375604in}}%
\pgfpathlineto{\pgfqpoint{3.735415in}{1.386007in}}%
\pgfpathlineto{\pgfqpoint{3.743406in}{1.396539in}}%
\pgfpathlineto{\pgfqpoint{3.751390in}{1.407195in}}%
\pgfpathlineto{\pgfqpoint{3.759369in}{1.417970in}}%
\pgfpathlineto{\pgfqpoint{3.745839in}{1.417918in}}%
\pgfpathlineto{\pgfqpoint{3.732316in}{1.418026in}}%
\pgfpathlineto{\pgfqpoint{3.718800in}{1.418294in}}%
\pgfpathlineto{\pgfqpoint{3.705292in}{1.418723in}}%
\pgfpathlineto{\pgfqpoint{3.697301in}{1.408494in}}%
\pgfpathlineto{\pgfqpoint{3.689304in}{1.398390in}}%
\pgfpathlineto{\pgfqpoint{3.681300in}{1.388417in}}%
\pgfpathlineto{\pgfqpoint{3.673290in}{1.378580in}}%
\pgfpathclose%
\pgfusepath{fill}%
\end{pgfscope}%
\begin{pgfscope}%
\pgfpathrectangle{\pgfqpoint{1.254980in}{0.150000in}}{\pgfqpoint{5.490039in}{5.490039in}}%
\pgfusepath{clip}%
\pgfsetbuttcap%
\pgfsetroundjoin%
\definecolor{currentfill}{rgb}{0.281924,0.089666,0.412415}%
\pgfsetfillcolor{currentfill}%
\pgfsetfillopacity{0.700000}%
\pgfsetlinewidth{0.000000pt}%
\definecolor{currentstroke}{rgb}{0.000000,0.000000,0.000000}%
\pgfsetstrokecolor{currentstroke}%
\pgfsetdash{}{0pt}%
\pgfpathmoveto{\pgfqpoint{3.845386in}{1.466066in}}%
\pgfpathlineto{\pgfqpoint{3.858947in}{1.467444in}}%
\pgfpathlineto{\pgfqpoint{3.872517in}{1.468980in}}%
\pgfpathlineto{\pgfqpoint{3.886096in}{1.470676in}}%
\pgfpathlineto{\pgfqpoint{3.899683in}{1.472530in}}%
\pgfpathlineto{\pgfqpoint{3.907616in}{1.484842in}}%
\pgfpathlineto{\pgfqpoint{3.915545in}{1.497229in}}%
\pgfpathlineto{\pgfqpoint{3.923468in}{1.509686in}}%
\pgfpathlineto{\pgfqpoint{3.931386in}{1.522206in}}%
\pgfpathlineto{\pgfqpoint{3.917805in}{1.519850in}}%
\pgfpathlineto{\pgfqpoint{3.904233in}{1.517652in}}%
\pgfpathlineto{\pgfqpoint{3.890671in}{1.515613in}}%
\pgfpathlineto{\pgfqpoint{3.877117in}{1.513733in}}%
\pgfpathlineto{\pgfqpoint{3.869192in}{1.501704in}}%
\pgfpathlineto{\pgfqpoint{3.861262in}{1.489747in}}%
\pgfpathlineto{\pgfqpoint{3.853326in}{1.477866in}}%
\pgfpathlineto{\pgfqpoint{3.845386in}{1.466066in}}%
\pgfpathclose%
\pgfusepath{fill}%
\end{pgfscope}%
\begin{pgfscope}%
\pgfpathrectangle{\pgfqpoint{1.254980in}{0.150000in}}{\pgfqpoint{5.490039in}{5.490039in}}%
\pgfusepath{clip}%
\pgfsetbuttcap%
\pgfsetroundjoin%
\definecolor{currentfill}{rgb}{0.157729,0.485932,0.558013}%
\pgfsetfillcolor{currentfill}%
\pgfsetfillopacity{0.700000}%
\pgfsetlinewidth{0.000000pt}%
\definecolor{currentstroke}{rgb}{0.000000,0.000000,0.000000}%
\pgfsetstrokecolor{currentstroke}%
\pgfsetdash{}{0pt}%
\pgfpathmoveto{\pgfqpoint{4.690860in}{2.363587in}}%
\pgfpathlineto{\pgfqpoint{4.704821in}{2.374457in}}%
\pgfpathlineto{\pgfqpoint{4.718799in}{2.385489in}}%
\pgfpathlineto{\pgfqpoint{4.732793in}{2.396682in}}%
\pgfpathlineto{\pgfqpoint{4.746804in}{2.408037in}}%
\pgfpathlineto{\pgfqpoint{4.754493in}{2.420626in}}%
\pgfpathlineto{\pgfqpoint{4.762176in}{2.433076in}}%
\pgfpathlineto{\pgfqpoint{4.769853in}{2.445387in}}%
\pgfpathlineto{\pgfqpoint{4.777523in}{2.457559in}}%
\pgfpathlineto{\pgfqpoint{4.763510in}{2.446066in}}%
\pgfpathlineto{\pgfqpoint{4.749515in}{2.434734in}}%
\pgfpathlineto{\pgfqpoint{4.735536in}{2.423565in}}%
\pgfpathlineto{\pgfqpoint{4.721574in}{2.412557in}}%
\pgfpathlineto{\pgfqpoint{4.713905in}{2.400512in}}%
\pgfpathlineto{\pgfqpoint{4.706229in}{2.388335in}}%
\pgfpathlineto{\pgfqpoint{4.698548in}{2.376026in}}%
\pgfpathlineto{\pgfqpoint{4.690860in}{2.363587in}}%
\pgfpathclose%
\pgfusepath{fill}%
\end{pgfscope}%
\begin{pgfscope}%
\pgfpathrectangle{\pgfqpoint{1.254980in}{0.150000in}}{\pgfqpoint{5.490039in}{5.490039in}}%
\pgfusepath{clip}%
\pgfsetbuttcap%
\pgfsetroundjoin%
\definecolor{currentfill}{rgb}{0.506271,0.828786,0.300362}%
\pgfsetfillcolor{currentfill}%
\pgfsetfillopacity{0.700000}%
\pgfsetlinewidth{0.000000pt}%
\definecolor{currentstroke}{rgb}{0.000000,0.000000,0.000000}%
\pgfsetstrokecolor{currentstroke}%
\pgfsetdash{}{0pt}%
\pgfpathmoveto{\pgfqpoint{5.709155in}{3.398020in}}%
\pgfpathlineto{\pgfqpoint{5.723811in}{3.414253in}}%
\pgfpathlineto{\pgfqpoint{5.738489in}{3.430652in}}%
\pgfpathlineto{\pgfqpoint{5.753191in}{3.447217in}}%
\pgfpathlineto{\pgfqpoint{5.760259in}{3.449319in}}%
\pgfpathlineto{\pgfqpoint{5.767317in}{3.451286in}}%
\pgfpathlineto{\pgfqpoint{5.774363in}{3.453121in}}%
\pgfpathlineto{\pgfqpoint{5.781399in}{3.454830in}}%
\pgfpathlineto{\pgfqpoint{5.766719in}{3.438602in}}%
\pgfpathlineto{\pgfqpoint{5.752062in}{3.422539in}}%
\pgfpathlineto{\pgfqpoint{5.737427in}{3.406641in}}%
\pgfpathlineto{\pgfqpoint{5.730375in}{3.404673in}}%
\pgfpathlineto{\pgfqpoint{5.723313in}{3.402582in}}%
\pgfpathlineto{\pgfqpoint{5.716239in}{3.400366in}}%
\pgfpathlineto{\pgfqpoint{5.709155in}{3.398020in}}%
\pgfpathclose%
\pgfusepath{fill}%
\end{pgfscope}%
\begin{pgfscope}%
\pgfpathrectangle{\pgfqpoint{1.254980in}{0.150000in}}{\pgfqpoint{5.490039in}{5.490039in}}%
\pgfusepath{clip}%
\pgfsetbuttcap%
\pgfsetroundjoin%
\definecolor{currentfill}{rgb}{0.271305,0.019942,0.347269}%
\pgfsetfillcolor{currentfill}%
\pgfsetfillopacity{0.700000}%
\pgfsetlinewidth{0.000000pt}%
\definecolor{currentstroke}{rgb}{0.000000,0.000000,0.000000}%
\pgfsetstrokecolor{currentstroke}%
\pgfsetdash{}{0pt}%
\pgfpathmoveto{\pgfqpoint{3.587095in}{1.348583in}}%
\pgfpathlineto{\pgfqpoint{3.600607in}{1.346371in}}%
\pgfpathlineto{\pgfqpoint{3.614124in}{1.344320in}}%
\pgfpathlineto{\pgfqpoint{3.627648in}{1.342431in}}%
\pgfpathlineto{\pgfqpoint{3.641178in}{1.340703in}}%
\pgfpathlineto{\pgfqpoint{3.649217in}{1.349940in}}%
\pgfpathlineto{\pgfqpoint{3.657248in}{1.359336in}}%
\pgfpathlineto{\pgfqpoint{3.665272in}{1.368884in}}%
\pgfpathlineto{\pgfqpoint{3.673290in}{1.378580in}}%
\pgfpathlineto{\pgfqpoint{3.659775in}{1.379725in}}%
\pgfpathlineto{\pgfqpoint{3.646266in}{1.381031in}}%
\pgfpathlineto{\pgfqpoint{3.632764in}{1.382499in}}%
\pgfpathlineto{\pgfqpoint{3.619269in}{1.384128in}}%
\pgfpathlineto{\pgfqpoint{3.611236in}{1.375005in}}%
\pgfpathlineto{\pgfqpoint{3.603197in}{1.366036in}}%
\pgfpathlineto{\pgfqpoint{3.595149in}{1.357227in}}%
\pgfpathlineto{\pgfqpoint{3.587095in}{1.348583in}}%
\pgfpathclose%
\pgfusepath{fill}%
\end{pgfscope}%
\begin{pgfscope}%
\pgfpathrectangle{\pgfqpoint{1.254980in}{0.150000in}}{\pgfqpoint{5.490039in}{5.490039in}}%
\pgfusepath{clip}%
\pgfsetbuttcap%
\pgfsetroundjoin%
\definecolor{currentfill}{rgb}{0.269308,0.218818,0.509577}%
\pgfsetfillcolor{currentfill}%
\pgfsetfillopacity{0.700000}%
\pgfsetlinewidth{0.000000pt}%
\definecolor{currentstroke}{rgb}{0.000000,0.000000,0.000000}%
\pgfsetstrokecolor{currentstroke}%
\pgfsetdash{}{0pt}%
\pgfpathmoveto{\pgfqpoint{4.134959in}{1.711182in}}%
\pgfpathlineto{\pgfqpoint{4.148625in}{1.716311in}}%
\pgfpathlineto{\pgfqpoint{4.162302in}{1.721598in}}%
\pgfpathlineto{\pgfqpoint{4.175991in}{1.727044in}}%
\pgfpathlineto{\pgfqpoint{4.189692in}{1.732649in}}%
\pgfpathlineto{\pgfqpoint{4.197544in}{1.746823in}}%
\pgfpathlineto{\pgfqpoint{4.205393in}{1.760983in}}%
\pgfpathlineto{\pgfqpoint{4.213238in}{1.775123in}}%
\pgfpathlineto{\pgfqpoint{4.221078in}{1.789242in}}%
\pgfpathlineto{\pgfqpoint{4.207377in}{1.783242in}}%
\pgfpathlineto{\pgfqpoint{4.193689in}{1.777400in}}%
\pgfpathlineto{\pgfqpoint{4.180012in}{1.771718in}}%
\pgfpathlineto{\pgfqpoint{4.166347in}{1.766194in}}%
\pgfpathlineto{\pgfqpoint{4.158507in}{1.752461in}}%
\pgfpathlineto{\pgfqpoint{4.150662in}{1.738711in}}%
\pgfpathlineto{\pgfqpoint{4.142813in}{1.724950in}}%
\pgfpathlineto{\pgfqpoint{4.134959in}{1.711182in}}%
\pgfpathclose%
\pgfusepath{fill}%
\end{pgfscope}%
\begin{pgfscope}%
\pgfpathrectangle{\pgfqpoint{1.254980in}{0.150000in}}{\pgfqpoint{5.490039in}{5.490039in}}%
\pgfusepath{clip}%
\pgfsetbuttcap%
\pgfsetroundjoin%
\definecolor{currentfill}{rgb}{0.171176,0.452530,0.557965}%
\pgfsetfillcolor{currentfill}%
\pgfsetfillopacity{0.700000}%
\pgfsetlinewidth{0.000000pt}%
\definecolor{currentstroke}{rgb}{0.000000,0.000000,0.000000}%
\pgfsetstrokecolor{currentstroke}%
\pgfsetdash{}{0pt}%
\pgfpathmoveto{\pgfqpoint{2.257755in}{2.380489in}}%
\pgfpathlineto{\pgfqpoint{2.271653in}{2.358092in}}%
\pgfpathlineto{\pgfqpoint{2.285538in}{2.335958in}}%
\pgfpathlineto{\pgfqpoint{2.299411in}{2.314085in}}%
\pgfpathlineto{\pgfqpoint{2.313271in}{2.292471in}}%
\pgfpathlineto{\pgfqpoint{2.322451in}{2.282297in}}%
\pgfpathlineto{\pgfqpoint{2.331602in}{2.272565in}}%
\pgfpathlineto{\pgfqpoint{2.340725in}{2.263266in}}%
\pgfpathlineto{\pgfqpoint{2.349821in}{2.254392in}}%
\pgfpathlineto{\pgfqpoint{2.336029in}{2.275246in}}%
\pgfpathlineto{\pgfqpoint{2.322225in}{2.296356in}}%
\pgfpathlineto{\pgfqpoint{2.308409in}{2.317726in}}%
\pgfpathlineto{\pgfqpoint{2.294581in}{2.339358in}}%
\pgfpathlineto{\pgfqpoint{2.285418in}{2.348980in}}%
\pgfpathlineto{\pgfqpoint{2.276226in}{2.359038in}}%
\pgfpathlineto{\pgfqpoint{2.267005in}{2.369538in}}%
\pgfpathlineto{\pgfqpoint{2.257755in}{2.380489in}}%
\pgfpathclose%
\pgfusepath{fill}%
\end{pgfscope}%
\begin{pgfscope}%
\pgfpathrectangle{\pgfqpoint{1.254980in}{0.150000in}}{\pgfqpoint{5.490039in}{5.490039in}}%
\pgfusepath{clip}%
\pgfsetbuttcap%
\pgfsetroundjoin%
\definecolor{currentfill}{rgb}{0.179019,0.433756,0.557430}%
\pgfsetfillcolor{currentfill}%
\pgfsetfillopacity{0.700000}%
\pgfsetlinewidth{0.000000pt}%
\definecolor{currentstroke}{rgb}{0.000000,0.000000,0.000000}%
\pgfsetstrokecolor{currentstroke}%
\pgfsetdash{}{0pt}%
\pgfpathmoveto{\pgfqpoint{4.573486in}{2.219125in}}%
\pgfpathlineto{\pgfqpoint{4.587379in}{2.228985in}}%
\pgfpathlineto{\pgfqpoint{4.601287in}{2.239007in}}%
\pgfpathlineto{\pgfqpoint{4.615212in}{2.249189in}}%
\pgfpathlineto{\pgfqpoint{4.629152in}{2.259532in}}%
\pgfpathlineto{\pgfqpoint{4.636885in}{2.272968in}}%
\pgfpathlineto{\pgfqpoint{4.644613in}{2.286284in}}%
\pgfpathlineto{\pgfqpoint{4.652335in}{2.299479in}}%
\pgfpathlineto{\pgfqpoint{4.660052in}{2.312552in}}%
\pgfpathlineto{\pgfqpoint{4.646109in}{2.302011in}}%
\pgfpathlineto{\pgfqpoint{4.632182in}{2.291631in}}%
\pgfpathlineto{\pgfqpoint{4.618272in}{2.281413in}}%
\pgfpathlineto{\pgfqpoint{4.604377in}{2.271356in}}%
\pgfpathlineto{\pgfqpoint{4.596662in}{2.258469in}}%
\pgfpathlineto{\pgfqpoint{4.588942in}{2.245467in}}%
\pgfpathlineto{\pgfqpoint{4.581217in}{2.232352in}}%
\pgfpathlineto{\pgfqpoint{4.573486in}{2.219125in}}%
\pgfpathclose%
\pgfusepath{fill}%
\end{pgfscope}%
\begin{pgfscope}%
\pgfpathrectangle{\pgfqpoint{1.254980in}{0.150000in}}{\pgfqpoint{5.490039in}{5.490039in}}%
\pgfusepath{clip}%
\pgfsetbuttcap%
\pgfsetroundjoin%
\definecolor{currentfill}{rgb}{0.283187,0.125848,0.444960}%
\pgfsetfillcolor{currentfill}%
\pgfsetfillopacity{0.700000}%
\pgfsetlinewidth{0.000000pt}%
\definecolor{currentstroke}{rgb}{0.000000,0.000000,0.000000}%
\pgfsetstrokecolor{currentstroke}%
\pgfsetdash{}{0pt}%
\pgfpathmoveto{\pgfqpoint{3.931386in}{1.522206in}}%
\pgfpathlineto{\pgfqpoint{3.944977in}{1.524721in}}%
\pgfpathlineto{\pgfqpoint{3.958577in}{1.527395in}}%
\pgfpathlineto{\pgfqpoint{3.972187in}{1.530227in}}%
\pgfpathlineto{\pgfqpoint{3.985807in}{1.533216in}}%
\pgfpathlineto{\pgfqpoint{3.993715in}{1.546283in}}%
\pgfpathlineto{\pgfqpoint{4.001619in}{1.559398in}}%
\pgfpathlineto{\pgfqpoint{4.009518in}{1.572557in}}%
\pgfpathlineto{\pgfqpoint{4.017413in}{1.585756in}}%
\pgfpathlineto{\pgfqpoint{4.003797in}{1.582290in}}%
\pgfpathlineto{\pgfqpoint{3.990192in}{1.578982in}}%
\pgfpathlineto{\pgfqpoint{3.976597in}{1.575833in}}%
\pgfpathlineto{\pgfqpoint{3.963011in}{1.572842in}}%
\pgfpathlineto{\pgfqpoint{3.955112in}{1.560109in}}%
\pgfpathlineto{\pgfqpoint{3.947208in}{1.547423in}}%
\pgfpathlineto{\pgfqpoint{3.939300in}{1.534787in}}%
\pgfpathlineto{\pgfqpoint{3.931386in}{1.522206in}}%
\pgfpathclose%
\pgfusepath{fill}%
\end{pgfscope}%
\begin{pgfscope}%
\pgfpathrectangle{\pgfqpoint{1.254980in}{0.150000in}}{\pgfqpoint{5.490039in}{5.490039in}}%
\pgfusepath{clip}%
\pgfsetbuttcap%
\pgfsetroundjoin%
\definecolor{currentfill}{rgb}{0.120081,0.622161,0.534946}%
\pgfsetfillcolor{currentfill}%
\pgfsetfillopacity{0.700000}%
\pgfsetlinewidth{0.000000pt}%
\definecolor{currentstroke}{rgb}{0.000000,0.000000,0.000000}%
\pgfsetstrokecolor{currentstroke}%
\pgfsetdash{}{0pt}%
\pgfpathmoveto{\pgfqpoint{5.012052in}{2.732605in}}%
\pgfpathlineto{\pgfqpoint{5.026225in}{2.745812in}}%
\pgfpathlineto{\pgfqpoint{5.040416in}{2.759182in}}%
\pgfpathlineto{\pgfqpoint{5.054626in}{2.772716in}}%
\pgfpathlineto{\pgfqpoint{5.068856in}{2.786414in}}%
\pgfpathlineto{\pgfqpoint{5.076401in}{2.796130in}}%
\pgfpathlineto{\pgfqpoint{5.083937in}{2.805679in}}%
\pgfpathlineto{\pgfqpoint{5.091466in}{2.815062in}}%
\pgfpathlineto{\pgfqpoint{5.098985in}{2.824280in}}%
\pgfpathlineto{\pgfqpoint{5.084759in}{2.810597in}}%
\pgfpathlineto{\pgfqpoint{5.070551in}{2.797078in}}%
\pgfpathlineto{\pgfqpoint{5.056363in}{2.783723in}}%
\pgfpathlineto{\pgfqpoint{5.042193in}{2.770531in}}%
\pgfpathlineto{\pgfqpoint{5.034670in}{2.761287in}}%
\pgfpathlineto{\pgfqpoint{5.027139in}{2.751885in}}%
\pgfpathlineto{\pgfqpoint{5.019599in}{2.742324in}}%
\pgfpathlineto{\pgfqpoint{5.012052in}{2.732605in}}%
\pgfpathclose%
\pgfusepath{fill}%
\end{pgfscope}%
\begin{pgfscope}%
\pgfpathrectangle{\pgfqpoint{1.254980in}{0.150000in}}{\pgfqpoint{5.490039in}{5.490039in}}%
\pgfusepath{clip}%
\pgfsetbuttcap%
\pgfsetroundjoin%
\definecolor{currentfill}{rgb}{0.282656,0.100196,0.422160}%
\pgfsetfillcolor{currentfill}%
\pgfsetfillopacity{0.700000}%
\pgfsetlinewidth{0.000000pt}%
\definecolor{currentstroke}{rgb}{0.000000,0.000000,0.000000}%
\pgfsetstrokecolor{currentstroke}%
\pgfsetdash{}{0pt}%
\pgfpathmoveto{\pgfqpoint{2.969034in}{1.523609in}}%
\pgfpathlineto{\pgfqpoint{2.982583in}{1.512590in}}%
\pgfpathlineto{\pgfqpoint{2.996131in}{1.501756in}}%
\pgfpathlineto{\pgfqpoint{3.009677in}{1.491105in}}%
\pgfpathlineto{\pgfqpoint{3.023223in}{1.480637in}}%
\pgfpathlineto{\pgfqpoint{3.031690in}{1.480093in}}%
\pgfpathlineto{\pgfqpoint{3.040142in}{1.479878in}}%
\pgfpathlineto{\pgfqpoint{3.048577in}{1.479985in}}%
\pgfpathlineto{\pgfqpoint{3.056997in}{1.480406in}}%
\pgfpathlineto{\pgfqpoint{3.043492in}{1.490168in}}%
\pgfpathlineto{\pgfqpoint{3.029986in}{1.500112in}}%
\pgfpathlineto{\pgfqpoint{3.016480in}{1.510239in}}%
\pgfpathlineto{\pgfqpoint{3.002972in}{1.520550in}}%
\pgfpathlineto{\pgfqpoint{2.994512in}{1.520823in}}%
\pgfpathlineto{\pgfqpoint{2.986036in}{1.521419in}}%
\pgfpathlineto{\pgfqpoint{2.977544in}{1.522345in}}%
\pgfpathlineto{\pgfqpoint{2.969034in}{1.523609in}}%
\pgfpathclose%
\pgfusepath{fill}%
\end{pgfscope}%
\begin{pgfscope}%
\pgfpathrectangle{\pgfqpoint{1.254980in}{0.150000in}}{\pgfqpoint{5.490039in}{5.490039in}}%
\pgfusepath{clip}%
\pgfsetbuttcap%
\pgfsetroundjoin%
\definecolor{currentfill}{rgb}{0.268510,0.009605,0.335427}%
\pgfsetfillcolor{currentfill}%
\pgfsetfillopacity{0.700000}%
\pgfsetlinewidth{0.000000pt}%
\definecolor{currentstroke}{rgb}{0.000000,0.000000,0.000000}%
\pgfsetstrokecolor{currentstroke}%
\pgfsetdash{}{0pt}%
\pgfpathmoveto{\pgfqpoint{3.360099in}{1.340409in}}%
\pgfpathlineto{\pgfqpoint{3.373600in}{1.334974in}}%
\pgfpathlineto{\pgfqpoint{3.387105in}{1.329707in}}%
\pgfpathlineto{\pgfqpoint{3.400613in}{1.324605in}}%
\pgfpathlineto{\pgfqpoint{3.414124in}{1.319670in}}%
\pgfpathlineto{\pgfqpoint{3.422291in}{1.325474in}}%
\pgfpathlineto{\pgfqpoint{3.430449in}{1.331506in}}%
\pgfpathlineto{\pgfqpoint{3.438596in}{1.337760in}}%
\pgfpathlineto{\pgfqpoint{3.446734in}{1.344231in}}%
\pgfpathlineto{\pgfqpoint{3.433246in}{1.348527in}}%
\pgfpathlineto{\pgfqpoint{3.419763in}{1.352988in}}%
\pgfpathlineto{\pgfqpoint{3.406283in}{1.357617in}}%
\pgfpathlineto{\pgfqpoint{3.392807in}{1.362412in}}%
\pgfpathlineto{\pgfqpoint{3.384646in}{1.356570in}}%
\pgfpathlineto{\pgfqpoint{3.376474in}{1.350951in}}%
\pgfpathlineto{\pgfqpoint{3.368292in}{1.345562in}}%
\pgfpathlineto{\pgfqpoint{3.360099in}{1.340409in}}%
\pgfpathclose%
\pgfusepath{fill}%
\end{pgfscope}%
\begin{pgfscope}%
\pgfpathrectangle{\pgfqpoint{1.254980in}{0.150000in}}{\pgfqpoint{5.490039in}{5.490039in}}%
\pgfusepath{clip}%
\pgfsetbuttcap%
\pgfsetroundjoin%
\definecolor{currentfill}{rgb}{0.304148,0.764704,0.419943}%
\pgfsetfillcolor{currentfill}%
\pgfsetfillopacity{0.700000}%
\pgfsetlinewidth{0.000000pt}%
\definecolor{currentstroke}{rgb}{0.000000,0.000000,0.000000}%
\pgfsetstrokecolor{currentstroke}%
\pgfsetdash{}{0pt}%
\pgfpathmoveto{\pgfqpoint{5.419395in}{3.145961in}}%
\pgfpathlineto{\pgfqpoint{5.433850in}{3.161231in}}%
\pgfpathlineto{\pgfqpoint{5.448325in}{3.176667in}}%
\pgfpathlineto{\pgfqpoint{5.462822in}{3.192267in}}%
\pgfpathlineto{\pgfqpoint{5.477341in}{3.208034in}}%
\pgfpathlineto{\pgfqpoint{5.484630in}{3.213289in}}%
\pgfpathlineto{\pgfqpoint{5.491909in}{3.218381in}}%
\pgfpathlineto{\pgfqpoint{5.499177in}{3.223312in}}%
\pgfpathlineto{\pgfqpoint{5.506434in}{3.228086in}}%
\pgfpathlineto{\pgfqpoint{5.491928in}{3.212527in}}%
\pgfpathlineto{\pgfqpoint{5.477444in}{3.197133in}}%
\pgfpathlineto{\pgfqpoint{5.462980in}{3.181903in}}%
\pgfpathlineto{\pgfqpoint{5.448539in}{3.166838in}}%
\pgfpathlineto{\pgfqpoint{5.441268in}{3.161847in}}%
\pgfpathlineto{\pgfqpoint{5.433987in}{3.156705in}}%
\pgfpathlineto{\pgfqpoint{5.426696in}{3.151410in}}%
\pgfpathlineto{\pgfqpoint{5.419395in}{3.145961in}}%
\pgfpathclose%
\pgfusepath{fill}%
\end{pgfscope}%
\begin{pgfscope}%
\pgfpathrectangle{\pgfqpoint{1.254980in}{0.150000in}}{\pgfqpoint{5.490039in}{5.490039in}}%
\pgfusepath{clip}%
\pgfsetbuttcap%
\pgfsetroundjoin%
\definecolor{currentfill}{rgb}{0.203063,0.379716,0.553925}%
\pgfsetfillcolor{currentfill}%
\pgfsetfillopacity{0.700000}%
\pgfsetlinewidth{0.000000pt}%
\definecolor{currentstroke}{rgb}{0.000000,0.000000,0.000000}%
\pgfsetstrokecolor{currentstroke}%
\pgfsetdash{}{0pt}%
\pgfpathmoveto{\pgfqpoint{4.456055in}{2.073587in}}%
\pgfpathlineto{\pgfqpoint{4.469883in}{2.082324in}}%
\pgfpathlineto{\pgfqpoint{4.483725in}{2.091221in}}%
\pgfpathlineto{\pgfqpoint{4.497582in}{2.100278in}}%
\pgfpathlineto{\pgfqpoint{4.511455in}{2.109496in}}%
\pgfpathlineto{\pgfqpoint{4.519226in}{2.123551in}}%
\pgfpathlineto{\pgfqpoint{4.526993in}{2.137511in}}%
\pgfpathlineto{\pgfqpoint{4.534754in}{2.151372in}}%
\pgfpathlineto{\pgfqpoint{4.542511in}{2.165132in}}%
\pgfpathlineto{\pgfqpoint{4.528636in}{2.155658in}}%
\pgfpathlineto{\pgfqpoint{4.514776in}{2.146345in}}%
\pgfpathlineto{\pgfqpoint{4.500931in}{2.137193in}}%
\pgfpathlineto{\pgfqpoint{4.487101in}{2.128201in}}%
\pgfpathlineto{\pgfqpoint{4.479347in}{2.114685in}}%
\pgfpathlineto{\pgfqpoint{4.471588in}{2.101076in}}%
\pgfpathlineto{\pgfqpoint{4.463824in}{2.087376in}}%
\pgfpathlineto{\pgfqpoint{4.456055in}{2.073587in}}%
\pgfpathclose%
\pgfusepath{fill}%
\end{pgfscope}%
\begin{pgfscope}%
\pgfpathrectangle{\pgfqpoint{1.254980in}{0.150000in}}{\pgfqpoint{5.490039in}{5.490039in}}%
\pgfusepath{clip}%
\pgfsetbuttcap%
\pgfsetroundjoin%
\definecolor{currentfill}{rgb}{0.119483,0.614817,0.537692}%
\pgfsetfillcolor{currentfill}%
\pgfsetfillopacity{0.700000}%
\pgfsetlinewidth{0.000000pt}%
\definecolor{currentstroke}{rgb}{0.000000,0.000000,0.000000}%
\pgfsetstrokecolor{currentstroke}%
\pgfsetdash{}{0pt}%
\pgfpathmoveto{\pgfqpoint{2.015027in}{2.830791in}}%
\pgfpathlineto{\pgfqpoint{2.029160in}{2.803384in}}%
\pgfpathlineto{\pgfqpoint{2.043276in}{2.776294in}}%
\pgfpathlineto{\pgfqpoint{2.057373in}{2.749517in}}%
\pgfpathlineto{\pgfqpoint{2.071452in}{2.723050in}}%
\pgfpathlineto{\pgfqpoint{2.080879in}{2.710823in}}%
\pgfpathlineto{\pgfqpoint{2.090274in}{2.699052in}}%
\pgfpathlineto{\pgfqpoint{2.099638in}{2.687732in}}%
\pgfpathlineto{\pgfqpoint{2.108971in}{2.676852in}}%
\pgfpathlineto{\pgfqpoint{2.094968in}{2.702559in}}%
\pgfpathlineto{\pgfqpoint{2.080948in}{2.728573in}}%
\pgfpathlineto{\pgfqpoint{2.066911in}{2.754898in}}%
\pgfpathlineto{\pgfqpoint{2.052856in}{2.781537in}}%
\pgfpathlineto{\pgfqpoint{2.043447in}{2.793165in}}%
\pgfpathlineto{\pgfqpoint{2.034006in}{2.805245in}}%
\pgfpathlineto{\pgfqpoint{2.024533in}{2.817784in}}%
\pgfpathlineto{\pgfqpoint{2.015027in}{2.830791in}}%
\pgfpathclose%
\pgfusepath{fill}%
\end{pgfscope}%
\begin{pgfscope}%
\pgfpathrectangle{\pgfqpoint{1.254980in}{0.150000in}}{\pgfqpoint{5.490039in}{5.490039in}}%
\pgfusepath{clip}%
\pgfsetbuttcap%
\pgfsetroundjoin%
\definecolor{currentfill}{rgb}{0.180653,0.701402,0.488189}%
\pgfsetfillcolor{currentfill}%
\pgfsetfillopacity{0.700000}%
\pgfsetlinewidth{0.000000pt}%
\definecolor{currentstroke}{rgb}{0.000000,0.000000,0.000000}%
\pgfsetstrokecolor{currentstroke}%
\pgfsetdash{}{0pt}%
\pgfpathmoveto{\pgfqpoint{5.215897in}{2.948035in}}%
\pgfpathlineto{\pgfqpoint{5.230213in}{2.962413in}}%
\pgfpathlineto{\pgfqpoint{5.244548in}{2.976956in}}%
\pgfpathlineto{\pgfqpoint{5.258904in}{2.991664in}}%
\pgfpathlineto{\pgfqpoint{5.273280in}{3.006537in}}%
\pgfpathlineto{\pgfqpoint{5.280709in}{3.014110in}}%
\pgfpathlineto{\pgfqpoint{5.288128in}{3.021512in}}%
\pgfpathlineto{\pgfqpoint{5.295538in}{3.028746in}}%
\pgfpathlineto{\pgfqpoint{5.302938in}{3.035812in}}%
\pgfpathlineto{\pgfqpoint{5.288569in}{3.021050in}}%
\pgfpathlineto{\pgfqpoint{5.274220in}{3.006452in}}%
\pgfpathlineto{\pgfqpoint{5.259892in}{2.992019in}}%
\pgfpathlineto{\pgfqpoint{5.245583in}{2.977750in}}%
\pgfpathlineto{\pgfqpoint{5.238176in}{2.970562in}}%
\pgfpathlineto{\pgfqpoint{5.230759in}{2.963215in}}%
\pgfpathlineto{\pgfqpoint{5.223333in}{2.955706in}}%
\pgfpathlineto{\pgfqpoint{5.215897in}{2.948035in}}%
\pgfpathclose%
\pgfusepath{fill}%
\end{pgfscope}%
\begin{pgfscope}%
\pgfpathrectangle{\pgfqpoint{1.254980in}{0.150000in}}{\pgfqpoint{5.490039in}{5.490039in}}%
\pgfusepath{clip}%
\pgfsetbuttcap%
\pgfsetroundjoin%
\definecolor{currentfill}{rgb}{0.272594,0.025563,0.353093}%
\pgfsetfillcolor{currentfill}%
\pgfsetfillopacity{0.700000}%
\pgfsetlinewidth{0.000000pt}%
\definecolor{currentstroke}{rgb}{0.000000,0.000000,0.000000}%
\pgfsetstrokecolor{currentstroke}%
\pgfsetdash{}{0pt}%
\pgfpathmoveto{\pgfqpoint{3.219069in}{1.377204in}}%
\pgfpathlineto{\pgfqpoint{3.232581in}{1.369742in}}%
\pgfpathlineto{\pgfqpoint{3.246094in}{1.362453in}}%
\pgfpathlineto{\pgfqpoint{3.259609in}{1.355334in}}%
\pgfpathlineto{\pgfqpoint{3.273126in}{1.348386in}}%
\pgfpathlineto{\pgfqpoint{3.281394in}{1.351849in}}%
\pgfpathlineto{\pgfqpoint{3.289649in}{1.355581in}}%
\pgfpathlineto{\pgfqpoint{3.297892in}{1.359577in}}%
\pgfpathlineto{\pgfqpoint{3.306124in}{1.363829in}}%
\pgfpathlineto{\pgfqpoint{3.292637in}{1.370107in}}%
\pgfpathlineto{\pgfqpoint{3.279152in}{1.376556in}}%
\pgfpathlineto{\pgfqpoint{3.265669in}{1.383176in}}%
\pgfpathlineto{\pgfqpoint{3.252189in}{1.389967in}}%
\pgfpathlineto{\pgfqpoint{3.243928in}{1.386373in}}%
\pgfpathlineto{\pgfqpoint{3.235654in}{1.383044in}}%
\pgfpathlineto{\pgfqpoint{3.227368in}{1.379985in}}%
\pgfpathlineto{\pgfqpoint{3.219069in}{1.377204in}}%
\pgfpathclose%
\pgfusepath{fill}%
\end{pgfscope}%
\begin{pgfscope}%
\pgfpathrectangle{\pgfqpoint{1.254980in}{0.150000in}}{\pgfqpoint{5.490039in}{5.490039in}}%
\pgfusepath{clip}%
\pgfsetbuttcap%
\pgfsetroundjoin%
\definecolor{currentfill}{rgb}{0.268510,0.009605,0.335427}%
\pgfsetfillcolor{currentfill}%
\pgfsetfillopacity{0.700000}%
\pgfsetlinewidth{0.000000pt}%
\definecolor{currentstroke}{rgb}{0.000000,0.000000,0.000000}%
\pgfsetstrokecolor{currentstroke}%
\pgfsetdash{}{0pt}%
\pgfpathmoveto{\pgfqpoint{3.500727in}{1.328697in}}%
\pgfpathlineto{\pgfqpoint{3.514237in}{1.325223in}}%
\pgfpathlineto{\pgfqpoint{3.527751in}{1.321913in}}%
\pgfpathlineto{\pgfqpoint{3.541271in}{1.318765in}}%
\pgfpathlineto{\pgfqpoint{3.554796in}{1.315780in}}%
\pgfpathlineto{\pgfqpoint{3.562883in}{1.323703in}}%
\pgfpathlineto{\pgfqpoint{3.570961in}{1.331815in}}%
\pgfpathlineto{\pgfqpoint{3.579032in}{1.340110in}}%
\pgfpathlineto{\pgfqpoint{3.587095in}{1.348583in}}%
\pgfpathlineto{\pgfqpoint{3.573588in}{1.350957in}}%
\pgfpathlineto{\pgfqpoint{3.560088in}{1.353494in}}%
\pgfpathlineto{\pgfqpoint{3.546592in}{1.356193in}}%
\pgfpathlineto{\pgfqpoint{3.533102in}{1.359056in}}%
\pgfpathlineto{\pgfqpoint{3.525021in}{1.351184in}}%
\pgfpathlineto{\pgfqpoint{3.516932in}{1.343496in}}%
\pgfpathlineto{\pgfqpoint{3.508834in}{1.335998in}}%
\pgfpathlineto{\pgfqpoint{3.500727in}{1.328697in}}%
\pgfpathclose%
\pgfusepath{fill}%
\end{pgfscope}%
\begin{pgfscope}%
\pgfpathrectangle{\pgfqpoint{1.254980in}{0.150000in}}{\pgfqpoint{5.490039in}{5.490039in}}%
\pgfusepath{clip}%
\pgfsetbuttcap%
\pgfsetroundjoin%
\definecolor{currentfill}{rgb}{0.229739,0.322361,0.545706}%
\pgfsetfillcolor{currentfill}%
\pgfsetfillopacity{0.700000}%
\pgfsetlinewidth{0.000000pt}%
\definecolor{currentstroke}{rgb}{0.000000,0.000000,0.000000}%
\pgfsetstrokecolor{currentstroke}%
\pgfsetdash{}{0pt}%
\pgfpathmoveto{\pgfqpoint{4.338586in}{1.929393in}}%
\pgfpathlineto{\pgfqpoint{4.352354in}{1.936896in}}%
\pgfpathlineto{\pgfqpoint{4.366135in}{1.944558in}}%
\pgfpathlineto{\pgfqpoint{4.379929in}{1.952379in}}%
\pgfpathlineto{\pgfqpoint{4.393738in}{1.960359in}}%
\pgfpathlineto{\pgfqpoint{4.401543in}{1.974772in}}%
\pgfpathlineto{\pgfqpoint{4.409344in}{1.989116in}}%
\pgfpathlineto{\pgfqpoint{4.417141in}{2.003388in}}%
\pgfpathlineto{\pgfqpoint{4.424933in}{2.017587in}}%
\pgfpathlineto{\pgfqpoint{4.411122in}{2.009293in}}%
\pgfpathlineto{\pgfqpoint{4.397325in}{2.001160in}}%
\pgfpathlineto{\pgfqpoint{4.383542in}{1.993186in}}%
\pgfpathlineto{\pgfqpoint{4.369773in}{1.985371in}}%
\pgfpathlineto{\pgfqpoint{4.361983in}{1.971474in}}%
\pgfpathlineto{\pgfqpoint{4.354188in}{1.957511in}}%
\pgfpathlineto{\pgfqpoint{4.346389in}{1.943483in}}%
\pgfpathlineto{\pgfqpoint{4.338586in}{1.929393in}}%
\pgfpathclose%
\pgfusepath{fill}%
\end{pgfscope}%
\begin{pgfscope}%
\pgfpathrectangle{\pgfqpoint{1.254980in}{0.150000in}}{\pgfqpoint{5.490039in}{5.490039in}}%
\pgfusepath{clip}%
\pgfsetbuttcap%
\pgfsetroundjoin%
\definecolor{currentfill}{rgb}{0.280868,0.160771,0.472899}%
\pgfsetfillcolor{currentfill}%
\pgfsetfillopacity{0.700000}%
\pgfsetlinewidth{0.000000pt}%
\definecolor{currentstroke}{rgb}{0.000000,0.000000,0.000000}%
\pgfsetstrokecolor{currentstroke}%
\pgfsetdash{}{0pt}%
\pgfpathmoveto{\pgfqpoint{4.017413in}{1.585756in}}%
\pgfpathlineto{\pgfqpoint{4.031039in}{1.589380in}}%
\pgfpathlineto{\pgfqpoint{4.044675in}{1.593163in}}%
\pgfpathlineto{\pgfqpoint{4.058322in}{1.597103in}}%
\pgfpathlineto{\pgfqpoint{4.071980in}{1.601201in}}%
\pgfpathlineto{\pgfqpoint{4.079867in}{1.614895in}}%
\pgfpathlineto{\pgfqpoint{4.087750in}{1.628612in}}%
\pgfpathlineto{\pgfqpoint{4.095629in}{1.642349in}}%
\pgfpathlineto{\pgfqpoint{4.103503in}{1.656101in}}%
\pgfpathlineto{\pgfqpoint{4.089848in}{1.651552in}}%
\pgfpathlineto{\pgfqpoint{4.076204in}{1.647162in}}%
\pgfpathlineto{\pgfqpoint{4.062570in}{1.642930in}}%
\pgfpathlineto{\pgfqpoint{4.048947in}{1.638857in}}%
\pgfpathlineto{\pgfqpoint{4.041070in}{1.625544in}}%
\pgfpathlineto{\pgfqpoint{4.033189in}{1.612253in}}%
\pgfpathlineto{\pgfqpoint{4.025303in}{1.598989in}}%
\pgfpathlineto{\pgfqpoint{4.017413in}{1.585756in}}%
\pgfpathclose%
\pgfusepath{fill}%
\end{pgfscope}%
\begin{pgfscope}%
\pgfpathrectangle{\pgfqpoint{1.254980in}{0.150000in}}{\pgfqpoint{5.490039in}{5.490039in}}%
\pgfusepath{clip}%
\pgfsetbuttcap%
\pgfsetroundjoin%
\definecolor{currentfill}{rgb}{0.124395,0.578002,0.548287}%
\pgfsetfillcolor{currentfill}%
\pgfsetfillopacity{0.700000}%
\pgfsetlinewidth{0.000000pt}%
\definecolor{currentstroke}{rgb}{0.000000,0.000000,0.000000}%
\pgfsetstrokecolor{currentstroke}%
\pgfsetdash{}{0pt}%
\pgfpathmoveto{\pgfqpoint{4.894877in}{2.598149in}}%
\pgfpathlineto{\pgfqpoint{4.908978in}{2.610640in}}%
\pgfpathlineto{\pgfqpoint{4.923098in}{2.623294in}}%
\pgfpathlineto{\pgfqpoint{4.937235in}{2.636111in}}%
\pgfpathlineto{\pgfqpoint{4.951392in}{2.649092in}}%
\pgfpathlineto{\pgfqpoint{4.959001in}{2.660092in}}%
\pgfpathlineto{\pgfqpoint{4.966603in}{2.670932in}}%
\pgfpathlineto{\pgfqpoint{4.974197in}{2.681612in}}%
\pgfpathlineto{\pgfqpoint{4.981784in}{2.692131in}}%
\pgfpathlineto{\pgfqpoint{4.967628in}{2.679103in}}%
\pgfpathlineto{\pgfqpoint{4.953491in}{2.666239in}}%
\pgfpathlineto{\pgfqpoint{4.939372in}{2.653538in}}%
\pgfpathlineto{\pgfqpoint{4.925272in}{2.641000in}}%
\pgfpathlineto{\pgfqpoint{4.917684in}{2.630516in}}%
\pgfpathlineto{\pgfqpoint{4.910089in}{2.619880in}}%
\pgfpathlineto{\pgfqpoint{4.902486in}{2.609091in}}%
\pgfpathlineto{\pgfqpoint{4.894877in}{2.598149in}}%
\pgfpathclose%
\pgfusepath{fill}%
\end{pgfscope}%
\begin{pgfscope}%
\pgfpathrectangle{\pgfqpoint{1.254980in}{0.150000in}}{\pgfqpoint{5.490039in}{5.490039in}}%
\pgfusepath{clip}%
\pgfsetbuttcap%
\pgfsetroundjoin%
\definecolor{currentfill}{rgb}{0.156270,0.489624,0.557936}%
\pgfsetfillcolor{currentfill}%
\pgfsetfillopacity{0.700000}%
\pgfsetlinewidth{0.000000pt}%
\definecolor{currentstroke}{rgb}{0.000000,0.000000,0.000000}%
\pgfsetstrokecolor{currentstroke}%
\pgfsetdash{}{0pt}%
\pgfpathmoveto{\pgfqpoint{2.202031in}{2.472759in}}%
\pgfpathlineto{\pgfqpoint{2.215982in}{2.449284in}}%
\pgfpathlineto{\pgfqpoint{2.229920in}{2.426083in}}%
\pgfpathlineto{\pgfqpoint{2.243844in}{2.403152in}}%
\pgfpathlineto{\pgfqpoint{2.257755in}{2.380489in}}%
\pgfpathlineto{\pgfqpoint{2.267005in}{2.369538in}}%
\pgfpathlineto{\pgfqpoint{2.276226in}{2.359038in}}%
\pgfpathlineto{\pgfqpoint{2.285418in}{2.348980in}}%
\pgfpathlineto{\pgfqpoint{2.294581in}{2.339358in}}%
\pgfpathlineto{\pgfqpoint{2.280741in}{2.361253in}}%
\pgfpathlineto{\pgfqpoint{2.266888in}{2.383415in}}%
\pgfpathlineto{\pgfqpoint{2.253022in}{2.405845in}}%
\pgfpathlineto{\pgfqpoint{2.239143in}{2.428546in}}%
\pgfpathlineto{\pgfqpoint{2.229910in}{2.438925in}}%
\pgfpathlineto{\pgfqpoint{2.220647in}{2.449748in}}%
\pgfpathlineto{\pgfqpoint{2.211354in}{2.461023in}}%
\pgfpathlineto{\pgfqpoint{2.202031in}{2.472759in}}%
\pgfpathclose%
\pgfusepath{fill}%
\end{pgfscope}%
\begin{pgfscope}%
\pgfpathrectangle{\pgfqpoint{1.254980in}{0.150000in}}{\pgfqpoint{5.490039in}{5.490039in}}%
\pgfusepath{clip}%
\pgfsetbuttcap%
\pgfsetroundjoin%
\definecolor{currentfill}{rgb}{0.280894,0.078907,0.402329}%
\pgfsetfillcolor{currentfill}%
\pgfsetfillopacity{0.700000}%
\pgfsetlinewidth{0.000000pt}%
\definecolor{currentstroke}{rgb}{0.000000,0.000000,0.000000}%
\pgfsetstrokecolor{currentstroke}%
\pgfsetdash{}{0pt}%
\pgfpathmoveto{\pgfqpoint{3.023223in}{1.480637in}}%
\pgfpathlineto{\pgfqpoint{3.036767in}{1.470352in}}%
\pgfpathlineto{\pgfqpoint{3.050311in}{1.460248in}}%
\pgfpathlineto{\pgfqpoint{3.063854in}{1.450324in}}%
\pgfpathlineto{\pgfqpoint{3.077396in}{1.440579in}}%
\pgfpathlineto{\pgfqpoint{3.085824in}{1.440752in}}%
\pgfpathlineto{\pgfqpoint{3.094236in}{1.441245in}}%
\pgfpathlineto{\pgfqpoint{3.102633in}{1.442053in}}%
\pgfpathlineto{\pgfqpoint{3.111015in}{1.443168in}}%
\pgfpathlineto{\pgfqpoint{3.097511in}{1.452208in}}%
\pgfpathlineto{\pgfqpoint{3.084006in}{1.461427in}}%
\pgfpathlineto{\pgfqpoint{3.070502in}{1.470826in}}%
\pgfpathlineto{\pgfqpoint{3.056997in}{1.480406in}}%
\pgfpathlineto{\pgfqpoint{3.048577in}{1.479985in}}%
\pgfpathlineto{\pgfqpoint{3.040142in}{1.479878in}}%
\pgfpathlineto{\pgfqpoint{3.031690in}{1.480093in}}%
\pgfpathlineto{\pgfqpoint{3.023223in}{1.480637in}}%
\pgfpathclose%
\pgfusepath{fill}%
\end{pgfscope}%
\begin{pgfscope}%
\pgfpathrectangle{\pgfqpoint{1.254980in}{0.150000in}}{\pgfqpoint{5.490039in}{5.490039in}}%
\pgfusepath{clip}%
\pgfsetbuttcap%
\pgfsetroundjoin%
\definecolor{currentfill}{rgb}{0.255645,0.260703,0.528312}%
\pgfsetfillcolor{currentfill}%
\pgfsetfillopacity{0.700000}%
\pgfsetlinewidth{0.000000pt}%
\definecolor{currentstroke}{rgb}{0.000000,0.000000,0.000000}%
\pgfsetstrokecolor{currentstroke}%
\pgfsetdash{}{0pt}%
\pgfpathmoveto{\pgfqpoint{4.221078in}{1.789242in}}%
\pgfpathlineto{\pgfqpoint{4.234791in}{1.795400in}}%
\pgfpathlineto{\pgfqpoint{4.248517in}{1.801717in}}%
\pgfpathlineto{\pgfqpoint{4.262255in}{1.808193in}}%
\pgfpathlineto{\pgfqpoint{4.276006in}{1.814827in}}%
\pgfpathlineto{\pgfqpoint{4.283844in}{1.829299in}}%
\pgfpathlineto{\pgfqpoint{4.291677in}{1.843733in}}%
\pgfpathlineto{\pgfqpoint{4.299505in}{1.858128in}}%
\pgfpathlineto{\pgfqpoint{4.307330in}{1.872480in}}%
\pgfpathlineto{\pgfqpoint{4.293578in}{1.865477in}}%
\pgfpathlineto{\pgfqpoint{4.279838in}{1.858633in}}%
\pgfpathlineto{\pgfqpoint{4.266112in}{1.851948in}}%
\pgfpathlineto{\pgfqpoint{4.252398in}{1.845422in}}%
\pgfpathlineto{\pgfqpoint{4.244574in}{1.831427in}}%
\pgfpathlineto{\pgfqpoint{4.236746in}{1.817397in}}%
\pgfpathlineto{\pgfqpoint{4.228914in}{1.803334in}}%
\pgfpathlineto{\pgfqpoint{4.221078in}{1.789242in}}%
\pgfpathclose%
\pgfusepath{fill}%
\end{pgfscope}%
\begin{pgfscope}%
\pgfpathrectangle{\pgfqpoint{1.254980in}{0.150000in}}{\pgfqpoint{5.490039in}{5.490039in}}%
\pgfusepath{clip}%
\pgfsetbuttcap%
\pgfsetroundjoin%
\definecolor{currentfill}{rgb}{0.141935,0.526453,0.555991}%
\pgfsetfillcolor{currentfill}%
\pgfsetfillopacity{0.700000}%
\pgfsetlinewidth{0.000000pt}%
\definecolor{currentstroke}{rgb}{0.000000,0.000000,0.000000}%
\pgfsetstrokecolor{currentstroke}%
\pgfsetdash{}{0pt}%
\pgfpathmoveto{\pgfqpoint{4.777523in}{2.457559in}}%
\pgfpathlineto{\pgfqpoint{4.791553in}{2.469214in}}%
\pgfpathlineto{\pgfqpoint{4.805600in}{2.481032in}}%
\pgfpathlineto{\pgfqpoint{4.819664in}{2.493012in}}%
\pgfpathlineto{\pgfqpoint{4.833746in}{2.505155in}}%
\pgfpathlineto{\pgfqpoint{4.841412in}{2.517305in}}%
\pgfpathlineto{\pgfqpoint{4.849070in}{2.529306in}}%
\pgfpathlineto{\pgfqpoint{4.856722in}{2.541158in}}%
\pgfpathlineto{\pgfqpoint{4.864367in}{2.552859in}}%
\pgfpathlineto{\pgfqpoint{4.850284in}{2.540608in}}%
\pgfpathlineto{\pgfqpoint{4.836218in}{2.528520in}}%
\pgfpathlineto{\pgfqpoint{4.822170in}{2.516595in}}%
\pgfpathlineto{\pgfqpoint{4.808140in}{2.504831in}}%
\pgfpathlineto{\pgfqpoint{4.800495in}{2.493227in}}%
\pgfpathlineto{\pgfqpoint{4.792844in}{2.481479in}}%
\pgfpathlineto{\pgfqpoint{4.785187in}{2.469590in}}%
\pgfpathlineto{\pgfqpoint{4.777523in}{2.457559in}}%
\pgfpathclose%
\pgfusepath{fill}%
\end{pgfscope}%
\begin{pgfscope}%
\pgfpathrectangle{\pgfqpoint{1.254980in}{0.150000in}}{\pgfqpoint{5.490039in}{5.490039in}}%
\pgfusepath{clip}%
\pgfsetbuttcap%
\pgfsetroundjoin%
\definecolor{currentfill}{rgb}{0.377779,0.791781,0.377939}%
\pgfsetfillcolor{currentfill}%
\pgfsetfillopacity{0.700000}%
\pgfsetlinewidth{0.000000pt}%
\definecolor{currentstroke}{rgb}{0.000000,0.000000,0.000000}%
\pgfsetstrokecolor{currentstroke}%
\pgfsetdash{}{0pt}%
\pgfpathmoveto{\pgfqpoint{5.506434in}{3.228086in}}%
\pgfpathlineto{\pgfqpoint{5.520962in}{3.243811in}}%
\pgfpathlineto{\pgfqpoint{5.535512in}{3.259701in}}%
\pgfpathlineto{\pgfqpoint{5.550083in}{3.275757in}}%
\pgfpathlineto{\pgfqpoint{5.564677in}{3.291979in}}%
\pgfpathlineto{\pgfqpoint{5.571910in}{3.296370in}}%
\pgfpathlineto{\pgfqpoint{5.579132in}{3.300600in}}%
\pgfpathlineto{\pgfqpoint{5.586343in}{3.304672in}}%
\pgfpathlineto{\pgfqpoint{5.593543in}{3.308591in}}%
\pgfpathlineto{\pgfqpoint{5.578964in}{3.292609in}}%
\pgfpathlineto{\pgfqpoint{5.564407in}{3.276793in}}%
\pgfpathlineto{\pgfqpoint{5.549872in}{3.261142in}}%
\pgfpathlineto{\pgfqpoint{5.535359in}{3.245656in}}%
\pgfpathlineto{\pgfqpoint{5.528143in}{3.241487in}}%
\pgfpathlineto{\pgfqpoint{5.520918in}{3.237170in}}%
\pgfpathlineto{\pgfqpoint{5.513681in}{3.232705in}}%
\pgfpathlineto{\pgfqpoint{5.506434in}{3.228086in}}%
\pgfpathclose%
\pgfusepath{fill}%
\end{pgfscope}%
\begin{pgfscope}%
\pgfpathrectangle{\pgfqpoint{1.254980in}{0.150000in}}{\pgfqpoint{5.490039in}{5.490039in}}%
\pgfusepath{clip}%
\pgfsetbuttcap%
\pgfsetroundjoin%
\definecolor{currentfill}{rgb}{0.137339,0.662252,0.515571}%
\pgfsetfillcolor{currentfill}%
\pgfsetfillopacity{0.700000}%
\pgfsetlinewidth{0.000000pt}%
\definecolor{currentstroke}{rgb}{0.000000,0.000000,0.000000}%
\pgfsetstrokecolor{currentstroke}%
\pgfsetdash{}{0pt}%
\pgfpathmoveto{\pgfqpoint{5.098985in}{2.824280in}}%
\pgfpathlineto{\pgfqpoint{5.113232in}{2.838126in}}%
\pgfpathlineto{\pgfqpoint{5.127497in}{2.852137in}}%
\pgfpathlineto{\pgfqpoint{5.141782in}{2.866312in}}%
\pgfpathlineto{\pgfqpoint{5.156087in}{2.880651in}}%
\pgfpathlineto{\pgfqpoint{5.163595in}{2.889669in}}%
\pgfpathlineto{\pgfqpoint{5.171094in}{2.898515in}}%
\pgfpathlineto{\pgfqpoint{5.178584in}{2.907189in}}%
\pgfpathlineto{\pgfqpoint{5.186065in}{2.915694in}}%
\pgfpathlineto{\pgfqpoint{5.171764in}{2.901402in}}%
\pgfpathlineto{\pgfqpoint{5.157483in}{2.887274in}}%
\pgfpathlineto{\pgfqpoint{5.143221in}{2.873310in}}%
\pgfpathlineto{\pgfqpoint{5.128979in}{2.859510in}}%
\pgfpathlineto{\pgfqpoint{5.121494in}{2.850946in}}%
\pgfpathlineto{\pgfqpoint{5.114000in}{2.842221in}}%
\pgfpathlineto{\pgfqpoint{5.106497in}{2.833332in}}%
\pgfpathlineto{\pgfqpoint{5.098985in}{2.824280in}}%
\pgfpathclose%
\pgfusepath{fill}%
\end{pgfscope}%
\begin{pgfscope}%
\pgfpathrectangle{\pgfqpoint{1.254980in}{0.150000in}}{\pgfqpoint{5.490039in}{5.490039in}}%
\pgfusepath{clip}%
\pgfsetbuttcap%
\pgfsetroundjoin%
\definecolor{currentfill}{rgb}{0.277018,0.050344,0.375715}%
\pgfsetfillcolor{currentfill}%
\pgfsetfillopacity{0.700000}%
\pgfsetlinewidth{0.000000pt}%
\definecolor{currentstroke}{rgb}{0.000000,0.000000,0.000000}%
\pgfsetstrokecolor{currentstroke}%
\pgfsetdash{}{0pt}%
\pgfpathmoveto{\pgfqpoint{3.727417in}{1.375604in}}%
\pgfpathlineto{\pgfqpoint{3.740967in}{1.375259in}}%
\pgfpathlineto{\pgfqpoint{3.754524in}{1.375074in}}%
\pgfpathlineto{\pgfqpoint{3.768088in}{1.375048in}}%
\pgfpathlineto{\pgfqpoint{3.781661in}{1.375180in}}%
\pgfpathlineto{\pgfqpoint{3.789646in}{1.386151in}}%
\pgfpathlineto{\pgfqpoint{3.797626in}{1.397243in}}%
\pgfpathlineto{\pgfqpoint{3.805600in}{1.408453in}}%
\pgfpathlineto{\pgfqpoint{3.813569in}{1.419773in}}%
\pgfpathlineto{\pgfqpoint{3.800007in}{1.419084in}}%
\pgfpathlineto{\pgfqpoint{3.786453in}{1.418553in}}%
\pgfpathlineto{\pgfqpoint{3.772907in}{1.418182in}}%
\pgfpathlineto{\pgfqpoint{3.759369in}{1.417970in}}%
\pgfpathlineto{\pgfqpoint{3.751390in}{1.407195in}}%
\pgfpathlineto{\pgfqpoint{3.743406in}{1.396539in}}%
\pgfpathlineto{\pgfqpoint{3.735415in}{1.386007in}}%
\pgfpathlineto{\pgfqpoint{3.727417in}{1.375604in}}%
\pgfpathclose%
\pgfusepath{fill}%
\end{pgfscope}%
\begin{pgfscope}%
\pgfpathrectangle{\pgfqpoint{1.254980in}{0.150000in}}{\pgfqpoint{5.490039in}{5.490039in}}%
\pgfusepath{clip}%
\pgfsetbuttcap%
\pgfsetroundjoin%
\definecolor{currentfill}{rgb}{0.162142,0.474838,0.558140}%
\pgfsetfillcolor{currentfill}%
\pgfsetfillopacity{0.700000}%
\pgfsetlinewidth{0.000000pt}%
\definecolor{currentstroke}{rgb}{0.000000,0.000000,0.000000}%
\pgfsetstrokecolor{currentstroke}%
\pgfsetdash{}{0pt}%
\pgfpathmoveto{\pgfqpoint{4.660052in}{2.312552in}}%
\pgfpathlineto{\pgfqpoint{4.674011in}{2.323254in}}%
\pgfpathlineto{\pgfqpoint{4.687986in}{2.334117in}}%
\pgfpathlineto{\pgfqpoint{4.701978in}{2.345142in}}%
\pgfpathlineto{\pgfqpoint{4.715986in}{2.356329in}}%
\pgfpathlineto{\pgfqpoint{4.723699in}{2.369457in}}%
\pgfpathlineto{\pgfqpoint{4.731407in}{2.382452in}}%
\pgfpathlineto{\pgfqpoint{4.739108in}{2.395312in}}%
\pgfpathlineto{\pgfqpoint{4.746804in}{2.408037in}}%
\pgfpathlineto{\pgfqpoint{4.732793in}{2.396682in}}%
\pgfpathlineto{\pgfqpoint{4.718799in}{2.385489in}}%
\pgfpathlineto{\pgfqpoint{4.704821in}{2.374457in}}%
\pgfpathlineto{\pgfqpoint{4.690860in}{2.363587in}}%
\pgfpathlineto{\pgfqpoint{4.683167in}{2.351019in}}%
\pgfpathlineto{\pgfqpoint{4.675468in}{2.338323in}}%
\pgfpathlineto{\pgfqpoint{4.667762in}{2.325500in}}%
\pgfpathlineto{\pgfqpoint{4.660052in}{2.312552in}}%
\pgfpathclose%
\pgfusepath{fill}%
\end{pgfscope}%
\begin{pgfscope}%
\pgfpathrectangle{\pgfqpoint{1.254980in}{0.150000in}}{\pgfqpoint{5.490039in}{5.490039in}}%
\pgfusepath{clip}%
\pgfsetbuttcap%
\pgfsetroundjoin%
\definecolor{currentfill}{rgb}{0.268510,0.009605,0.335427}%
\pgfsetfillcolor{currentfill}%
\pgfsetfillopacity{0.700000}%
\pgfsetlinewidth{0.000000pt}%
\definecolor{currentstroke}{rgb}{0.000000,0.000000,0.000000}%
\pgfsetstrokecolor{currentstroke}%
\pgfsetdash{}{0pt}%
\pgfpathmoveto{\pgfqpoint{3.414124in}{1.319670in}}%
\pgfpathlineto{\pgfqpoint{3.427639in}{1.314900in}}%
\pgfpathlineto{\pgfqpoint{3.441158in}{1.310295in}}%
\pgfpathlineto{\pgfqpoint{3.454681in}{1.305855in}}%
\pgfpathlineto{\pgfqpoint{3.468208in}{1.301579in}}%
\pgfpathlineto{\pgfqpoint{3.476352in}{1.308032in}}%
\pgfpathlineto{\pgfqpoint{3.484486in}{1.314708in}}%
\pgfpathlineto{\pgfqpoint{3.492611in}{1.321598in}}%
\pgfpathlineto{\pgfqpoint{3.500727in}{1.328697in}}%
\pgfpathlineto{\pgfqpoint{3.487222in}{1.332334in}}%
\pgfpathlineto{\pgfqpoint{3.473722in}{1.336135in}}%
\pgfpathlineto{\pgfqpoint{3.460226in}{1.340100in}}%
\pgfpathlineto{\pgfqpoint{3.446734in}{1.344231in}}%
\pgfpathlineto{\pgfqpoint{3.438596in}{1.337760in}}%
\pgfpathlineto{\pgfqpoint{3.430449in}{1.331506in}}%
\pgfpathlineto{\pgfqpoint{3.422291in}{1.325474in}}%
\pgfpathlineto{\pgfqpoint{3.414124in}{1.319670in}}%
\pgfpathclose%
\pgfusepath{fill}%
\end{pgfscope}%
\begin{pgfscope}%
\pgfpathrectangle{\pgfqpoint{1.254980in}{0.150000in}}{\pgfqpoint{5.490039in}{5.490039in}}%
\pgfusepath{clip}%
\pgfsetbuttcap%
\pgfsetroundjoin%
\definecolor{currentfill}{rgb}{0.280894,0.078907,0.402329}%
\pgfsetfillcolor{currentfill}%
\pgfsetfillopacity{0.700000}%
\pgfsetlinewidth{0.000000pt}%
\definecolor{currentstroke}{rgb}{0.000000,0.000000,0.000000}%
\pgfsetstrokecolor{currentstroke}%
\pgfsetdash{}{0pt}%
\pgfpathmoveto{\pgfqpoint{3.813569in}{1.419773in}}%
\pgfpathlineto{\pgfqpoint{3.827139in}{1.420622in}}%
\pgfpathlineto{\pgfqpoint{3.840717in}{1.421629in}}%
\pgfpathlineto{\pgfqpoint{3.854304in}{1.422794in}}%
\pgfpathlineto{\pgfqpoint{3.867899in}{1.424117in}}%
\pgfpathlineto{\pgfqpoint{3.875853in}{1.436084in}}%
\pgfpathlineto{\pgfqpoint{3.883801in}{1.448145in}}%
\pgfpathlineto{\pgfqpoint{3.891745in}{1.460296in}}%
\pgfpathlineto{\pgfqpoint{3.899683in}{1.472530in}}%
\pgfpathlineto{\pgfqpoint{3.886096in}{1.470676in}}%
\pgfpathlineto{\pgfqpoint{3.872517in}{1.468980in}}%
\pgfpathlineto{\pgfqpoint{3.858947in}{1.467444in}}%
\pgfpathlineto{\pgfqpoint{3.845386in}{1.466066in}}%
\pgfpathlineto{\pgfqpoint{3.837440in}{1.454351in}}%
\pgfpathlineto{\pgfqpoint{3.829488in}{1.442728in}}%
\pgfpathlineto{\pgfqpoint{3.821531in}{1.431200in}}%
\pgfpathlineto{\pgfqpoint{3.813569in}{1.419773in}}%
\pgfpathclose%
\pgfusepath{fill}%
\end{pgfscope}%
\begin{pgfscope}%
\pgfpathrectangle{\pgfqpoint{1.254980in}{0.150000in}}{\pgfqpoint{5.490039in}{5.490039in}}%
\pgfusepath{clip}%
\pgfsetbuttcap%
\pgfsetroundjoin%
\definecolor{currentfill}{rgb}{0.274128,0.199721,0.498911}%
\pgfsetfillcolor{currentfill}%
\pgfsetfillopacity{0.700000}%
\pgfsetlinewidth{0.000000pt}%
\definecolor{currentstroke}{rgb}{0.000000,0.000000,0.000000}%
\pgfsetstrokecolor{currentstroke}%
\pgfsetdash{}{0pt}%
\pgfpathmoveto{\pgfqpoint{4.103503in}{1.656101in}}%
\pgfpathlineto{\pgfqpoint{4.117170in}{1.660807in}}%
\pgfpathlineto{\pgfqpoint{4.130848in}{1.665672in}}%
\pgfpathlineto{\pgfqpoint{4.144538in}{1.670695in}}%
\pgfpathlineto{\pgfqpoint{4.158239in}{1.675875in}}%
\pgfpathlineto{\pgfqpoint{4.166108in}{1.690072in}}%
\pgfpathlineto{\pgfqpoint{4.173974in}{1.704269in}}%
\pgfpathlineto{\pgfqpoint{4.181835in}{1.718463in}}%
\pgfpathlineto{\pgfqpoint{4.189692in}{1.732649in}}%
\pgfpathlineto{\pgfqpoint{4.175991in}{1.727044in}}%
\pgfpathlineto{\pgfqpoint{4.162302in}{1.721598in}}%
\pgfpathlineto{\pgfqpoint{4.148625in}{1.716311in}}%
\pgfpathlineto{\pgfqpoint{4.134959in}{1.711182in}}%
\pgfpathlineto{\pgfqpoint{4.127102in}{1.697408in}}%
\pgfpathlineto{\pgfqpoint{4.119240in}{1.683634in}}%
\pgfpathlineto{\pgfqpoint{4.111374in}{1.669864in}}%
\pgfpathlineto{\pgfqpoint{4.103503in}{1.656101in}}%
\pgfpathclose%
\pgfusepath{fill}%
\end{pgfscope}%
\begin{pgfscope}%
\pgfpathrectangle{\pgfqpoint{1.254980in}{0.150000in}}{\pgfqpoint{5.490039in}{5.490039in}}%
\pgfusepath{clip}%
\pgfsetbuttcap%
\pgfsetroundjoin%
\definecolor{currentfill}{rgb}{0.271305,0.019942,0.347269}%
\pgfsetfillcolor{currentfill}%
\pgfsetfillopacity{0.700000}%
\pgfsetlinewidth{0.000000pt}%
\definecolor{currentstroke}{rgb}{0.000000,0.000000,0.000000}%
\pgfsetstrokecolor{currentstroke}%
\pgfsetdash{}{0pt}%
\pgfpathmoveto{\pgfqpoint{3.273126in}{1.348386in}}%
\pgfpathlineto{\pgfqpoint{3.286645in}{1.341608in}}%
\pgfpathlineto{\pgfqpoint{3.300167in}{1.335000in}}%
\pgfpathlineto{\pgfqpoint{3.313690in}{1.328560in}}%
\pgfpathlineto{\pgfqpoint{3.327217in}{1.322289in}}%
\pgfpathlineto{\pgfqpoint{3.335454in}{1.326432in}}%
\pgfpathlineto{\pgfqpoint{3.343680in}{1.330837in}}%
\pgfpathlineto{\pgfqpoint{3.351895in}{1.335499in}}%
\pgfpathlineto{\pgfqpoint{3.360099in}{1.340409in}}%
\pgfpathlineto{\pgfqpoint{3.346601in}{1.346011in}}%
\pgfpathlineto{\pgfqpoint{3.333106in}{1.351782in}}%
\pgfpathlineto{\pgfqpoint{3.319613in}{1.357721in}}%
\pgfpathlineto{\pgfqpoint{3.306124in}{1.363829in}}%
\pgfpathlineto{\pgfqpoint{3.297892in}{1.359577in}}%
\pgfpathlineto{\pgfqpoint{3.289649in}{1.355581in}}%
\pgfpathlineto{\pgfqpoint{3.281394in}{1.351849in}}%
\pgfpathlineto{\pgfqpoint{3.273126in}{1.348386in}}%
\pgfpathclose%
\pgfusepath{fill}%
\end{pgfscope}%
\begin{pgfscope}%
\pgfpathrectangle{\pgfqpoint{1.254980in}{0.150000in}}{\pgfqpoint{5.490039in}{5.490039in}}%
\pgfusepath{clip}%
\pgfsetbuttcap%
\pgfsetroundjoin%
\definecolor{currentfill}{rgb}{0.279566,0.067836,0.391917}%
\pgfsetfillcolor{currentfill}%
\pgfsetfillopacity{0.700000}%
\pgfsetlinewidth{0.000000pt}%
\definecolor{currentstroke}{rgb}{0.000000,0.000000,0.000000}%
\pgfsetstrokecolor{currentstroke}%
\pgfsetdash{}{0pt}%
\pgfpathmoveto{\pgfqpoint{3.077396in}{1.440579in}}%
\pgfpathlineto{\pgfqpoint{3.090939in}{1.431014in}}%
\pgfpathlineto{\pgfqpoint{3.104481in}{1.421626in}}%
\pgfpathlineto{\pgfqpoint{3.118023in}{1.412416in}}%
\pgfpathlineto{\pgfqpoint{3.131565in}{1.403382in}}%
\pgfpathlineto{\pgfqpoint{3.139954in}{1.404269in}}%
\pgfpathlineto{\pgfqpoint{3.148329in}{1.405470in}}%
\pgfpathlineto{\pgfqpoint{3.156689in}{1.406977in}}%
\pgfpathlineto{\pgfqpoint{3.165035in}{1.408783in}}%
\pgfpathlineto{\pgfqpoint{3.151529in}{1.417115in}}%
\pgfpathlineto{\pgfqpoint{3.138024in}{1.425622in}}%
\pgfpathlineto{\pgfqpoint{3.124519in}{1.434306in}}%
\pgfpathlineto{\pgfqpoint{3.111015in}{1.443168in}}%
\pgfpathlineto{\pgfqpoint{3.102633in}{1.442053in}}%
\pgfpathlineto{\pgfqpoint{3.094236in}{1.441245in}}%
\pgfpathlineto{\pgfqpoint{3.085824in}{1.440752in}}%
\pgfpathlineto{\pgfqpoint{3.077396in}{1.440579in}}%
\pgfpathclose%
\pgfusepath{fill}%
\end{pgfscope}%
\begin{pgfscope}%
\pgfpathrectangle{\pgfqpoint{1.254980in}{0.150000in}}{\pgfqpoint{5.490039in}{5.490039in}}%
\pgfusepath{clip}%
\pgfsetbuttcap%
\pgfsetroundjoin%
\definecolor{currentfill}{rgb}{0.272594,0.025563,0.353093}%
\pgfsetfillcolor{currentfill}%
\pgfsetfillopacity{0.700000}%
\pgfsetlinewidth{0.000000pt}%
\definecolor{currentstroke}{rgb}{0.000000,0.000000,0.000000}%
\pgfsetstrokecolor{currentstroke}%
\pgfsetdash{}{0pt}%
\pgfpathmoveto{\pgfqpoint{3.641178in}{1.340703in}}%
\pgfpathlineto{\pgfqpoint{3.654714in}{1.339135in}}%
\pgfpathlineto{\pgfqpoint{3.668257in}{1.337727in}}%
\pgfpathlineto{\pgfqpoint{3.681806in}{1.336480in}}%
\pgfpathlineto{\pgfqpoint{3.695362in}{1.335392in}}%
\pgfpathlineto{\pgfqpoint{3.703386in}{1.345223in}}%
\pgfpathlineto{\pgfqpoint{3.711403in}{1.355206in}}%
\pgfpathlineto{\pgfqpoint{3.719413in}{1.365335in}}%
\pgfpathlineto{\pgfqpoint{3.727417in}{1.375604in}}%
\pgfpathlineto{\pgfqpoint{3.713875in}{1.376108in}}%
\pgfpathlineto{\pgfqpoint{3.700339in}{1.376771in}}%
\pgfpathlineto{\pgfqpoint{3.686811in}{1.377595in}}%
\pgfpathlineto{\pgfqpoint{3.673290in}{1.378580in}}%
\pgfpathlineto{\pgfqpoint{3.665272in}{1.368884in}}%
\pgfpathlineto{\pgfqpoint{3.657248in}{1.359336in}}%
\pgfpathlineto{\pgfqpoint{3.649217in}{1.349940in}}%
\pgfpathlineto{\pgfqpoint{3.641178in}{1.340703in}}%
\pgfpathclose%
\pgfusepath{fill}%
\end{pgfscope}%
\begin{pgfscope}%
\pgfpathrectangle{\pgfqpoint{1.254980in}{0.150000in}}{\pgfqpoint{5.490039in}{5.490039in}}%
\pgfusepath{clip}%
\pgfsetbuttcap%
\pgfsetroundjoin%
\definecolor{currentfill}{rgb}{0.239374,0.735588,0.455688}%
\pgfsetfillcolor{currentfill}%
\pgfsetfillopacity{0.700000}%
\pgfsetlinewidth{0.000000pt}%
\definecolor{currentstroke}{rgb}{0.000000,0.000000,0.000000}%
\pgfsetstrokecolor{currentstroke}%
\pgfsetdash{}{0pt}%
\pgfpathmoveto{\pgfqpoint{5.302938in}{3.035812in}}%
\pgfpathlineto{\pgfqpoint{5.317328in}{3.050739in}}%
\pgfpathlineto{\pgfqpoint{5.331738in}{3.065831in}}%
\pgfpathlineto{\pgfqpoint{5.346170in}{3.081088in}}%
\pgfpathlineto{\pgfqpoint{5.360622in}{3.096511in}}%
\pgfpathlineto{\pgfqpoint{5.368004in}{3.103281in}}%
\pgfpathlineto{\pgfqpoint{5.375376in}{3.109878in}}%
\pgfpathlineto{\pgfqpoint{5.382738in}{3.116305in}}%
\pgfpathlineto{\pgfqpoint{5.390090in}{3.122565in}}%
\pgfpathlineto{\pgfqpoint{5.375646in}{3.107285in}}%
\pgfpathlineto{\pgfqpoint{5.361224in}{3.092170in}}%
\pgfpathlineto{\pgfqpoint{5.346822in}{3.077221in}}%
\pgfpathlineto{\pgfqpoint{5.332441in}{3.062435in}}%
\pgfpathlineto{\pgfqpoint{5.325080in}{3.056022in}}%
\pgfpathlineto{\pgfqpoint{5.317709in}{3.049448in}}%
\pgfpathlineto{\pgfqpoint{5.310328in}{3.042712in}}%
\pgfpathlineto{\pgfqpoint{5.302938in}{3.035812in}}%
\pgfpathclose%
\pgfusepath{fill}%
\end{pgfscope}%
\begin{pgfscope}%
\pgfpathrectangle{\pgfqpoint{1.254980in}{0.150000in}}{\pgfqpoint{5.490039in}{5.490039in}}%
\pgfusepath{clip}%
\pgfsetbuttcap%
\pgfsetroundjoin%
\definecolor{currentfill}{rgb}{0.141935,0.526453,0.555991}%
\pgfsetfillcolor{currentfill}%
\pgfsetfillopacity{0.700000}%
\pgfsetlinewidth{0.000000pt}%
\definecolor{currentstroke}{rgb}{0.000000,0.000000,0.000000}%
\pgfsetstrokecolor{currentstroke}%
\pgfsetdash{}{0pt}%
\pgfpathmoveto{\pgfqpoint{2.146080in}{2.569437in}}%
\pgfpathlineto{\pgfqpoint{2.160090in}{2.544845in}}%
\pgfpathlineto{\pgfqpoint{2.174085in}{2.520537in}}%
\pgfpathlineto{\pgfqpoint{2.188065in}{2.496509in}}%
\pgfpathlineto{\pgfqpoint{2.202031in}{2.472759in}}%
\pgfpathlineto{\pgfqpoint{2.211354in}{2.461023in}}%
\pgfpathlineto{\pgfqpoint{2.220647in}{2.449748in}}%
\pgfpathlineto{\pgfqpoint{2.229910in}{2.438925in}}%
\pgfpathlineto{\pgfqpoint{2.239143in}{2.428546in}}%
\pgfpathlineto{\pgfqpoint{2.225251in}{2.451521in}}%
\pgfpathlineto{\pgfqpoint{2.211344in}{2.474772in}}%
\pgfpathlineto{\pgfqpoint{2.197424in}{2.498301in}}%
\pgfpathlineto{\pgfqpoint{2.183489in}{2.522112in}}%
\pgfpathlineto{\pgfqpoint{2.174183in}{2.533254in}}%
\pgfpathlineto{\pgfqpoint{2.164846in}{2.544850in}}%
\pgfpathlineto{\pgfqpoint{2.155479in}{2.556908in}}%
\pgfpathlineto{\pgfqpoint{2.146080in}{2.569437in}}%
\pgfpathclose%
\pgfusepath{fill}%
\end{pgfscope}%
\begin{pgfscope}%
\pgfpathrectangle{\pgfqpoint{1.254980in}{0.150000in}}{\pgfqpoint{5.490039in}{5.490039in}}%
\pgfusepath{clip}%
\pgfsetbuttcap%
\pgfsetroundjoin%
\definecolor{currentfill}{rgb}{0.183898,0.422383,0.556944}%
\pgfsetfillcolor{currentfill}%
\pgfsetfillopacity{0.700000}%
\pgfsetlinewidth{0.000000pt}%
\definecolor{currentstroke}{rgb}{0.000000,0.000000,0.000000}%
\pgfsetstrokecolor{currentstroke}%
\pgfsetdash{}{0pt}%
\pgfpathmoveto{\pgfqpoint{4.542511in}{2.165132in}}%
\pgfpathlineto{\pgfqpoint{4.556401in}{2.174766in}}%
\pgfpathlineto{\pgfqpoint{4.570307in}{2.184560in}}%
\pgfpathlineto{\pgfqpoint{4.584228in}{2.194515in}}%
\pgfpathlineto{\pgfqpoint{4.598165in}{2.204631in}}%
\pgfpathlineto{\pgfqpoint{4.605920in}{2.218526in}}%
\pgfpathlineto{\pgfqpoint{4.613669in}{2.232310in}}%
\pgfpathlineto{\pgfqpoint{4.621413in}{2.245979in}}%
\pgfpathlineto{\pgfqpoint{4.629152in}{2.259532in}}%
\pgfpathlineto{\pgfqpoint{4.615212in}{2.249189in}}%
\pgfpathlineto{\pgfqpoint{4.601287in}{2.239007in}}%
\pgfpathlineto{\pgfqpoint{4.587379in}{2.228985in}}%
\pgfpathlineto{\pgfqpoint{4.573486in}{2.219125in}}%
\pgfpathlineto{\pgfqpoint{4.565750in}{2.205788in}}%
\pgfpathlineto{\pgfqpoint{4.558009in}{2.192342in}}%
\pgfpathlineto{\pgfqpoint{4.550262in}{2.178789in}}%
\pgfpathlineto{\pgfqpoint{4.542511in}{2.165132in}}%
\pgfpathclose%
\pgfusepath{fill}%
\end{pgfscope}%
\begin{pgfscope}%
\pgfpathrectangle{\pgfqpoint{1.254980in}{0.150000in}}{\pgfqpoint{5.490039in}{5.490039in}}%
\pgfusepath{clip}%
\pgfsetbuttcap%
\pgfsetroundjoin%
\definecolor{currentfill}{rgb}{0.134692,0.658636,0.517649}%
\pgfsetfillcolor{currentfill}%
\pgfsetfillopacity{0.700000}%
\pgfsetlinewidth{0.000000pt}%
\definecolor{currentstroke}{rgb}{0.000000,0.000000,0.000000}%
\pgfsetstrokecolor{currentstroke}%
\pgfsetdash{}{0pt}%
\pgfpathmoveto{\pgfqpoint{1.958300in}{2.943647in}}%
\pgfpathlineto{\pgfqpoint{1.972512in}{2.914942in}}%
\pgfpathlineto{\pgfqpoint{1.986703in}{2.886566in}}%
\pgfpathlineto{\pgfqpoint{2.000874in}{2.858517in}}%
\pgfpathlineto{\pgfqpoint{2.015027in}{2.830791in}}%
\pgfpathlineto{\pgfqpoint{2.024533in}{2.817784in}}%
\pgfpathlineto{\pgfqpoint{2.034006in}{2.805245in}}%
\pgfpathlineto{\pgfqpoint{2.043447in}{2.793165in}}%
\pgfpathlineto{\pgfqpoint{2.052856in}{2.781537in}}%
\pgfpathlineto{\pgfqpoint{2.038783in}{2.808493in}}%
\pgfpathlineto{\pgfqpoint{2.024691in}{2.835770in}}%
\pgfpathlineto{\pgfqpoint{2.010580in}{2.863371in}}%
\pgfpathlineto{\pgfqpoint{1.996451in}{2.891299in}}%
\pgfpathlineto{\pgfqpoint{1.986963in}{2.903686in}}%
\pgfpathlineto{\pgfqpoint{1.977442in}{2.916534in}}%
\pgfpathlineto{\pgfqpoint{1.967888in}{2.929852in}}%
\pgfpathlineto{\pgfqpoint{1.958300in}{2.943647in}}%
\pgfpathclose%
\pgfusepath{fill}%
\end{pgfscope}%
\begin{pgfscope}%
\pgfpathrectangle{\pgfqpoint{1.254980in}{0.150000in}}{\pgfqpoint{5.490039in}{5.490039in}}%
\pgfusepath{clip}%
\pgfsetbuttcap%
\pgfsetroundjoin%
\definecolor{currentfill}{rgb}{0.283091,0.110553,0.431554}%
\pgfsetfillcolor{currentfill}%
\pgfsetfillopacity{0.700000}%
\pgfsetlinewidth{0.000000pt}%
\definecolor{currentstroke}{rgb}{0.000000,0.000000,0.000000}%
\pgfsetstrokecolor{currentstroke}%
\pgfsetdash{}{0pt}%
\pgfpathmoveto{\pgfqpoint{3.899683in}{1.472530in}}%
\pgfpathlineto{\pgfqpoint{3.913280in}{1.474541in}}%
\pgfpathlineto{\pgfqpoint{3.926886in}{1.476711in}}%
\pgfpathlineto{\pgfqpoint{3.940501in}{1.479039in}}%
\pgfpathlineto{\pgfqpoint{3.954126in}{1.481524in}}%
\pgfpathlineto{\pgfqpoint{3.962053in}{1.494351in}}%
\pgfpathlineto{\pgfqpoint{3.969976in}{1.507246in}}%
\pgfpathlineto{\pgfqpoint{3.977894in}{1.520202in}}%
\pgfpathlineto{\pgfqpoint{3.985807in}{1.533216in}}%
\pgfpathlineto{\pgfqpoint{3.972187in}{1.530227in}}%
\pgfpathlineto{\pgfqpoint{3.958577in}{1.527395in}}%
\pgfpathlineto{\pgfqpoint{3.944977in}{1.524721in}}%
\pgfpathlineto{\pgfqpoint{3.931386in}{1.522206in}}%
\pgfpathlineto{\pgfqpoint{3.923468in}{1.509686in}}%
\pgfpathlineto{\pgfqpoint{3.915545in}{1.497229in}}%
\pgfpathlineto{\pgfqpoint{3.907616in}{1.484842in}}%
\pgfpathlineto{\pgfqpoint{3.899683in}{1.472530in}}%
\pgfpathclose%
\pgfusepath{fill}%
\end{pgfscope}%
\begin{pgfscope}%
\pgfpathrectangle{\pgfqpoint{1.254980in}{0.150000in}}{\pgfqpoint{5.490039in}{5.490039in}}%
\pgfusepath{clip}%
\pgfsetbuttcap%
\pgfsetroundjoin%
\definecolor{currentfill}{rgb}{0.210503,0.363727,0.552206}%
\pgfsetfillcolor{currentfill}%
\pgfsetfillopacity{0.700000}%
\pgfsetlinewidth{0.000000pt}%
\definecolor{currentstroke}{rgb}{0.000000,0.000000,0.000000}%
\pgfsetstrokecolor{currentstroke}%
\pgfsetdash{}{0pt}%
\pgfpathmoveto{\pgfqpoint{4.424933in}{2.017587in}}%
\pgfpathlineto{\pgfqpoint{4.438758in}{2.026040in}}%
\pgfpathlineto{\pgfqpoint{4.452598in}{2.034652in}}%
\pgfpathlineto{\pgfqpoint{4.466452in}{2.043425in}}%
\pgfpathlineto{\pgfqpoint{4.480321in}{2.052357in}}%
\pgfpathlineto{\pgfqpoint{4.488112in}{2.066774in}}%
\pgfpathlineto{\pgfqpoint{4.495897in}{2.081105in}}%
\pgfpathlineto{\pgfqpoint{4.503678in}{2.095346in}}%
\pgfpathlineto{\pgfqpoint{4.511455in}{2.109496in}}%
\pgfpathlineto{\pgfqpoint{4.497582in}{2.100278in}}%
\pgfpathlineto{\pgfqpoint{4.483725in}{2.091221in}}%
\pgfpathlineto{\pgfqpoint{4.469883in}{2.082324in}}%
\pgfpathlineto{\pgfqpoint{4.456055in}{2.073587in}}%
\pgfpathlineto{\pgfqpoint{4.448281in}{2.059711in}}%
\pgfpathlineto{\pgfqpoint{4.440503in}{2.045750in}}%
\pgfpathlineto{\pgfqpoint{4.432720in}{2.031708in}}%
\pgfpathlineto{\pgfqpoint{4.424933in}{2.017587in}}%
\pgfpathclose%
\pgfusepath{fill}%
\end{pgfscope}%
\begin{pgfscope}%
\pgfpathrectangle{\pgfqpoint{1.254980in}{0.150000in}}{\pgfqpoint{5.490039in}{5.490039in}}%
\pgfusepath{clip}%
\pgfsetbuttcap%
\pgfsetroundjoin%
\definecolor{currentfill}{rgb}{0.269944,0.014625,0.341379}%
\pgfsetfillcolor{currentfill}%
\pgfsetfillopacity{0.700000}%
\pgfsetlinewidth{0.000000pt}%
\definecolor{currentstroke}{rgb}{0.000000,0.000000,0.000000}%
\pgfsetstrokecolor{currentstroke}%
\pgfsetdash{}{0pt}%
\pgfpathmoveto{\pgfqpoint{3.554796in}{1.315780in}}%
\pgfpathlineto{\pgfqpoint{3.568326in}{1.312957in}}%
\pgfpathlineto{\pgfqpoint{3.581861in}{1.310295in}}%
\pgfpathlineto{\pgfqpoint{3.595402in}{1.307795in}}%
\pgfpathlineto{\pgfqpoint{3.608949in}{1.305455in}}%
\pgfpathlineto{\pgfqpoint{3.617018in}{1.314000in}}%
\pgfpathlineto{\pgfqpoint{3.625079in}{1.322726in}}%
\pgfpathlineto{\pgfqpoint{3.633132in}{1.331629in}}%
\pgfpathlineto{\pgfqpoint{3.641178in}{1.340703in}}%
\pgfpathlineto{\pgfqpoint{3.627648in}{1.342431in}}%
\pgfpathlineto{\pgfqpoint{3.614124in}{1.344320in}}%
\pgfpathlineto{\pgfqpoint{3.600607in}{1.346371in}}%
\pgfpathlineto{\pgfqpoint{3.587095in}{1.348583in}}%
\pgfpathlineto{\pgfqpoint{3.579032in}{1.340110in}}%
\pgfpathlineto{\pgfqpoint{3.570961in}{1.331815in}}%
\pgfpathlineto{\pgfqpoint{3.562883in}{1.323703in}}%
\pgfpathlineto{\pgfqpoint{3.554796in}{1.315780in}}%
\pgfpathclose%
\pgfusepath{fill}%
\end{pgfscope}%
\begin{pgfscope}%
\pgfpathrectangle{\pgfqpoint{1.254980in}{0.150000in}}{\pgfqpoint{5.490039in}{5.490039in}}%
\pgfusepath{clip}%
\pgfsetbuttcap%
\pgfsetroundjoin%
\definecolor{currentfill}{rgb}{0.119483,0.614817,0.537692}%
\pgfsetfillcolor{currentfill}%
\pgfsetfillopacity{0.700000}%
\pgfsetlinewidth{0.000000pt}%
\definecolor{currentstroke}{rgb}{0.000000,0.000000,0.000000}%
\pgfsetstrokecolor{currentstroke}%
\pgfsetdash{}{0pt}%
\pgfpathmoveto{\pgfqpoint{4.981784in}{2.692131in}}%
\pgfpathlineto{\pgfqpoint{4.995958in}{2.705321in}}%
\pgfpathlineto{\pgfqpoint{5.010151in}{2.718676in}}%
\pgfpathlineto{\pgfqpoint{5.024363in}{2.732194in}}%
\pgfpathlineto{\pgfqpoint{5.038594in}{2.745876in}}%
\pgfpathlineto{\pgfqpoint{5.046172in}{2.756262in}}%
\pgfpathlineto{\pgfqpoint{5.053741in}{2.766481in}}%
\pgfpathlineto{\pgfqpoint{5.061303in}{2.776531in}}%
\pgfpathlineto{\pgfqpoint{5.068856in}{2.786414in}}%
\pgfpathlineto{\pgfqpoint{5.054626in}{2.772716in}}%
\pgfpathlineto{\pgfqpoint{5.040416in}{2.759182in}}%
\pgfpathlineto{\pgfqpoint{5.026225in}{2.745812in}}%
\pgfpathlineto{\pgfqpoint{5.012052in}{2.732605in}}%
\pgfpathlineto{\pgfqpoint{5.004497in}{2.722726in}}%
\pgfpathlineto{\pgfqpoint{4.996934in}{2.712688in}}%
\pgfpathlineto{\pgfqpoint{4.989362in}{2.702489in}}%
\pgfpathlineto{\pgfqpoint{4.981784in}{2.692131in}}%
\pgfpathclose%
\pgfusepath{fill}%
\end{pgfscope}%
\begin{pgfscope}%
\pgfpathrectangle{\pgfqpoint{1.254980in}{0.150000in}}{\pgfqpoint{5.490039in}{5.490039in}}%
\pgfusepath{clip}%
\pgfsetbuttcap%
\pgfsetroundjoin%
\definecolor{currentfill}{rgb}{0.237441,0.305202,0.541921}%
\pgfsetfillcolor{currentfill}%
\pgfsetfillopacity{0.700000}%
\pgfsetlinewidth{0.000000pt}%
\definecolor{currentstroke}{rgb}{0.000000,0.000000,0.000000}%
\pgfsetstrokecolor{currentstroke}%
\pgfsetdash{}{0pt}%
\pgfpathmoveto{\pgfqpoint{4.307330in}{1.872480in}}%
\pgfpathlineto{\pgfqpoint{4.321096in}{1.879642in}}%
\pgfpathlineto{\pgfqpoint{4.334875in}{1.886963in}}%
\pgfpathlineto{\pgfqpoint{4.348667in}{1.894443in}}%
\pgfpathlineto{\pgfqpoint{4.362474in}{1.902081in}}%
\pgfpathlineto{\pgfqpoint{4.370296in}{1.916739in}}%
\pgfpathlineto{\pgfqpoint{4.378114in}{1.931340in}}%
\pgfpathlineto{\pgfqpoint{4.385928in}{1.945881in}}%
\pgfpathlineto{\pgfqpoint{4.393738in}{1.960359in}}%
\pgfpathlineto{\pgfqpoint{4.379929in}{1.952379in}}%
\pgfpathlineto{\pgfqpoint{4.366135in}{1.944558in}}%
\pgfpathlineto{\pgfqpoint{4.352354in}{1.936896in}}%
\pgfpathlineto{\pgfqpoint{4.338586in}{1.929393in}}%
\pgfpathlineto{\pgfqpoint{4.330779in}{1.915245in}}%
\pgfpathlineto{\pgfqpoint{4.322967in}{1.901042in}}%
\pgfpathlineto{\pgfqpoint{4.315150in}{1.886786in}}%
\pgfpathlineto{\pgfqpoint{4.307330in}{1.872480in}}%
\pgfpathclose%
\pgfusepath{fill}%
\end{pgfscope}%
\begin{pgfscope}%
\pgfpathrectangle{\pgfqpoint{1.254980in}{0.150000in}}{\pgfqpoint{5.490039in}{5.490039in}}%
\pgfusepath{clip}%
\pgfsetbuttcap%
\pgfsetroundjoin%
\definecolor{currentfill}{rgb}{0.458674,0.816363,0.329727}%
\pgfsetfillcolor{currentfill}%
\pgfsetfillopacity{0.700000}%
\pgfsetlinewidth{0.000000pt}%
\definecolor{currentstroke}{rgb}{0.000000,0.000000,0.000000}%
\pgfsetstrokecolor{currentstroke}%
\pgfsetdash{}{0pt}%
\pgfpathmoveto{\pgfqpoint{5.593543in}{3.308591in}}%
\pgfpathlineto{\pgfqpoint{5.608145in}{3.324738in}}%
\pgfpathlineto{\pgfqpoint{5.622769in}{3.341051in}}%
\pgfpathlineto{\pgfqpoint{5.637415in}{3.357531in}}%
\pgfpathlineto{\pgfqpoint{5.652084in}{3.374178in}}%
\pgfpathlineto{\pgfqpoint{5.659257in}{3.377683in}}%
\pgfpathlineto{\pgfqpoint{5.666419in}{3.381031in}}%
\pgfpathlineto{\pgfqpoint{5.673569in}{3.384227in}}%
\pgfpathlineto{\pgfqpoint{5.680709in}{3.387272in}}%
\pgfpathlineto{\pgfqpoint{5.666057in}{3.370900in}}%
\pgfpathlineto{\pgfqpoint{5.651428in}{3.354693in}}%
\pgfpathlineto{\pgfqpoint{5.636821in}{3.338652in}}%
\pgfpathlineto{\pgfqpoint{5.622236in}{3.322777in}}%
\pgfpathlineto{\pgfqpoint{5.615079in}{3.319447in}}%
\pgfpathlineto{\pgfqpoint{5.607911in}{3.315975in}}%
\pgfpathlineto{\pgfqpoint{5.600733in}{3.312357in}}%
\pgfpathlineto{\pgfqpoint{5.593543in}{3.308591in}}%
\pgfpathclose%
\pgfusepath{fill}%
\end{pgfscope}%
\begin{pgfscope}%
\pgfpathrectangle{\pgfqpoint{1.254980in}{0.150000in}}{\pgfqpoint{5.490039in}{5.490039in}}%
\pgfusepath{clip}%
\pgfsetbuttcap%
\pgfsetroundjoin%
\definecolor{currentfill}{rgb}{0.282290,0.145912,0.461510}%
\pgfsetfillcolor{currentfill}%
\pgfsetfillopacity{0.700000}%
\pgfsetlinewidth{0.000000pt}%
\definecolor{currentstroke}{rgb}{0.000000,0.000000,0.000000}%
\pgfsetstrokecolor{currentstroke}%
\pgfsetdash{}{0pt}%
\pgfpathmoveto{\pgfqpoint{3.985807in}{1.533216in}}%
\pgfpathlineto{\pgfqpoint{3.999436in}{1.536364in}}%
\pgfpathlineto{\pgfqpoint{4.013076in}{1.539669in}}%
\pgfpathlineto{\pgfqpoint{4.026726in}{1.543132in}}%
\pgfpathlineto{\pgfqpoint{4.040387in}{1.546752in}}%
\pgfpathlineto{\pgfqpoint{4.048292in}{1.560307in}}%
\pgfpathlineto{\pgfqpoint{4.056192in}{1.573903in}}%
\pgfpathlineto{\pgfqpoint{4.064088in}{1.587536in}}%
\pgfpathlineto{\pgfqpoint{4.071980in}{1.601201in}}%
\pgfpathlineto{\pgfqpoint{4.058322in}{1.597103in}}%
\pgfpathlineto{\pgfqpoint{4.044675in}{1.593163in}}%
\pgfpathlineto{\pgfqpoint{4.031039in}{1.589380in}}%
\pgfpathlineto{\pgfqpoint{4.017413in}{1.585756in}}%
\pgfpathlineto{\pgfqpoint{4.009518in}{1.572557in}}%
\pgfpathlineto{\pgfqpoint{4.001619in}{1.559398in}}%
\pgfpathlineto{\pgfqpoint{3.993715in}{1.546283in}}%
\pgfpathlineto{\pgfqpoint{3.985807in}{1.533216in}}%
\pgfpathclose%
\pgfusepath{fill}%
\end{pgfscope}%
\begin{pgfscope}%
\pgfpathrectangle{\pgfqpoint{1.254980in}{0.150000in}}{\pgfqpoint{5.490039in}{5.490039in}}%
\pgfusepath{clip}%
\pgfsetbuttcap%
\pgfsetroundjoin%
\definecolor{currentfill}{rgb}{0.277018,0.050344,0.375715}%
\pgfsetfillcolor{currentfill}%
\pgfsetfillopacity{0.700000}%
\pgfsetlinewidth{0.000000pt}%
\definecolor{currentstroke}{rgb}{0.000000,0.000000,0.000000}%
\pgfsetstrokecolor{currentstroke}%
\pgfsetdash{}{0pt}%
\pgfpathmoveto{\pgfqpoint{3.131565in}{1.403382in}}%
\pgfpathlineto{\pgfqpoint{3.145108in}{1.394524in}}%
\pgfpathlineto{\pgfqpoint{3.158651in}{1.385841in}}%
\pgfpathlineto{\pgfqpoint{3.172195in}{1.377332in}}%
\pgfpathlineto{\pgfqpoint{3.185739in}{1.368997in}}%
\pgfpathlineto{\pgfqpoint{3.194093in}{1.370597in}}%
\pgfpathlineto{\pgfqpoint{3.202432in}{1.372503in}}%
\pgfpathlineto{\pgfqpoint{3.210757in}{1.374708in}}%
\pgfpathlineto{\pgfqpoint{3.219069in}{1.377204in}}%
\pgfpathlineto{\pgfqpoint{3.205559in}{1.384838in}}%
\pgfpathlineto{\pgfqpoint{3.192050in}{1.392646in}}%
\pgfpathlineto{\pgfqpoint{3.178542in}{1.400627in}}%
\pgfpathlineto{\pgfqpoint{3.165035in}{1.408783in}}%
\pgfpathlineto{\pgfqpoint{3.156689in}{1.406977in}}%
\pgfpathlineto{\pgfqpoint{3.148329in}{1.405470in}}%
\pgfpathlineto{\pgfqpoint{3.139954in}{1.404269in}}%
\pgfpathlineto{\pgfqpoint{3.131565in}{1.403382in}}%
\pgfpathclose%
\pgfusepath{fill}%
\end{pgfscope}%
\begin{pgfscope}%
\pgfpathrectangle{\pgfqpoint{1.254980in}{0.150000in}}{\pgfqpoint{5.490039in}{5.490039in}}%
\pgfusepath{clip}%
\pgfsetbuttcap%
\pgfsetroundjoin%
\definecolor{currentfill}{rgb}{0.262138,0.242286,0.520837}%
\pgfsetfillcolor{currentfill}%
\pgfsetfillopacity{0.700000}%
\pgfsetlinewidth{0.000000pt}%
\definecolor{currentstroke}{rgb}{0.000000,0.000000,0.000000}%
\pgfsetstrokecolor{currentstroke}%
\pgfsetdash{}{0pt}%
\pgfpathmoveto{\pgfqpoint{4.189692in}{1.732649in}}%
\pgfpathlineto{\pgfqpoint{4.203404in}{1.738411in}}%
\pgfpathlineto{\pgfqpoint{4.217130in}{1.744332in}}%
\pgfpathlineto{\pgfqpoint{4.230867in}{1.750411in}}%
\pgfpathlineto{\pgfqpoint{4.244617in}{1.756648in}}%
\pgfpathlineto{\pgfqpoint{4.252470in}{1.771229in}}%
\pgfpathlineto{\pgfqpoint{4.260320in}{1.785789in}}%
\pgfpathlineto{\pgfqpoint{4.268165in}{1.800323in}}%
\pgfpathlineto{\pgfqpoint{4.276006in}{1.814827in}}%
\pgfpathlineto{\pgfqpoint{4.262255in}{1.808193in}}%
\pgfpathlineto{\pgfqpoint{4.248517in}{1.801717in}}%
\pgfpathlineto{\pgfqpoint{4.234791in}{1.795400in}}%
\pgfpathlineto{\pgfqpoint{4.221078in}{1.789242in}}%
\pgfpathlineto{\pgfqpoint{4.213238in}{1.775123in}}%
\pgfpathlineto{\pgfqpoint{4.205393in}{1.760983in}}%
\pgfpathlineto{\pgfqpoint{4.197544in}{1.746823in}}%
\pgfpathlineto{\pgfqpoint{4.189692in}{1.732649in}}%
\pgfpathclose%
\pgfusepath{fill}%
\end{pgfscope}%
\begin{pgfscope}%
\pgfpathrectangle{\pgfqpoint{1.254980in}{0.150000in}}{\pgfqpoint{5.490039in}{5.490039in}}%
\pgfusepath{clip}%
\pgfsetbuttcap%
\pgfsetroundjoin%
\definecolor{currentfill}{rgb}{0.127568,0.566949,0.550556}%
\pgfsetfillcolor{currentfill}%
\pgfsetfillopacity{0.700000}%
\pgfsetlinewidth{0.000000pt}%
\definecolor{currentstroke}{rgb}{0.000000,0.000000,0.000000}%
\pgfsetstrokecolor{currentstroke}%
\pgfsetdash{}{0pt}%
\pgfpathmoveto{\pgfqpoint{4.864367in}{2.552859in}}%
\pgfpathlineto{\pgfqpoint{4.878468in}{2.565273in}}%
\pgfpathlineto{\pgfqpoint{4.892587in}{2.577849in}}%
\pgfpathlineto{\pgfqpoint{4.906724in}{2.590588in}}%
\pgfpathlineto{\pgfqpoint{4.920880in}{2.603491in}}%
\pgfpathlineto{\pgfqpoint{4.928519in}{2.615130in}}%
\pgfpathlineto{\pgfqpoint{4.936150in}{2.626610in}}%
\pgfpathlineto{\pgfqpoint{4.943775in}{2.637931in}}%
\pgfpathlineto{\pgfqpoint{4.951392in}{2.649092in}}%
\pgfpathlineto{\pgfqpoint{4.937235in}{2.636111in}}%
\pgfpathlineto{\pgfqpoint{4.923098in}{2.623294in}}%
\pgfpathlineto{\pgfqpoint{4.908978in}{2.610640in}}%
\pgfpathlineto{\pgfqpoint{4.894877in}{2.598149in}}%
\pgfpathlineto{\pgfqpoint{4.887260in}{2.587054in}}%
\pgfpathlineto{\pgfqpoint{4.879636in}{2.575808in}}%
\pgfpathlineto{\pgfqpoint{4.872005in}{2.564409in}}%
\pgfpathlineto{\pgfqpoint{4.864367in}{2.552859in}}%
\pgfpathclose%
\pgfusepath{fill}%
\end{pgfscope}%
\begin{pgfscope}%
\pgfpathrectangle{\pgfqpoint{1.254980in}{0.150000in}}{\pgfqpoint{5.490039in}{5.490039in}}%
\pgfusepath{clip}%
\pgfsetbuttcap%
\pgfsetroundjoin%
\definecolor{currentfill}{rgb}{0.127568,0.566949,0.550556}%
\pgfsetfillcolor{currentfill}%
\pgfsetfillopacity{0.700000}%
\pgfsetlinewidth{0.000000pt}%
\definecolor{currentstroke}{rgb}{0.000000,0.000000,0.000000}%
\pgfsetstrokecolor{currentstroke}%
\pgfsetdash{}{0pt}%
\pgfpathmoveto{\pgfqpoint{2.089883in}{2.670691in}}%
\pgfpathlineto{\pgfqpoint{2.103956in}{2.644939in}}%
\pgfpathlineto{\pgfqpoint{2.118013in}{2.619481in}}%
\pgfpathlineto{\pgfqpoint{2.132055in}{2.594315in}}%
\pgfpathlineto{\pgfqpoint{2.146080in}{2.569437in}}%
\pgfpathlineto{\pgfqpoint{2.155479in}{2.556908in}}%
\pgfpathlineto{\pgfqpoint{2.164846in}{2.544850in}}%
\pgfpathlineto{\pgfqpoint{2.174183in}{2.533254in}}%
\pgfpathlineto{\pgfqpoint{2.183489in}{2.522112in}}%
\pgfpathlineto{\pgfqpoint{2.169539in}{2.546207in}}%
\pgfpathlineto{\pgfqpoint{2.155575in}{2.570588in}}%
\pgfpathlineto{\pgfqpoint{2.141594in}{2.595258in}}%
\pgfpathlineto{\pgfqpoint{2.127599in}{2.620221in}}%
\pgfpathlineto{\pgfqpoint{2.118218in}{2.632134in}}%
\pgfpathlineto{\pgfqpoint{2.108805in}{2.644512in}}%
\pgfpathlineto{\pgfqpoint{2.099360in}{2.657361in}}%
\pgfpathlineto{\pgfqpoint{2.089883in}{2.670691in}}%
\pgfpathclose%
\pgfusepath{fill}%
\end{pgfscope}%
\begin{pgfscope}%
\pgfpathrectangle{\pgfqpoint{1.254980in}{0.150000in}}{\pgfqpoint{5.490039in}{5.490039in}}%
\pgfusepath{clip}%
\pgfsetbuttcap%
\pgfsetroundjoin%
\definecolor{currentfill}{rgb}{0.525776,0.833491,0.288127}%
\pgfsetfillcolor{currentfill}%
\pgfsetfillopacity{0.700000}%
\pgfsetlinewidth{0.000000pt}%
\definecolor{currentstroke}{rgb}{0.000000,0.000000,0.000000}%
\pgfsetstrokecolor{currentstroke}%
\pgfsetdash{}{0pt}%
\pgfpathmoveto{\pgfqpoint{5.680709in}{3.387272in}}%
\pgfpathlineto{\pgfqpoint{5.695384in}{3.403811in}}%
\pgfpathlineto{\pgfqpoint{5.710081in}{3.420516in}}%
\pgfpathlineto{\pgfqpoint{5.724802in}{3.437387in}}%
\pgfpathlineto{\pgfqpoint{5.731916in}{3.440065in}}%
\pgfpathlineto{\pgfqpoint{5.739019in}{3.442593in}}%
\pgfpathlineto{\pgfqpoint{5.746110in}{3.444976in}}%
\pgfpathlineto{\pgfqpoint{5.753191in}{3.447217in}}%
\pgfpathlineto{\pgfqpoint{5.738489in}{3.430652in}}%
\pgfpathlineto{\pgfqpoint{5.723811in}{3.414253in}}%
\pgfpathlineto{\pgfqpoint{5.709155in}{3.398020in}}%
\pgfpathlineto{\pgfqpoint{5.702060in}{3.395541in}}%
\pgfpathlineto{\pgfqpoint{5.694954in}{3.392926in}}%
\pgfpathlineto{\pgfqpoint{5.687837in}{3.390171in}}%
\pgfpathlineto{\pgfqpoint{5.680709in}{3.387272in}}%
\pgfpathclose%
\pgfusepath{fill}%
\end{pgfscope}%
\begin{pgfscope}%
\pgfpathrectangle{\pgfqpoint{1.254980in}{0.150000in}}{\pgfqpoint{5.490039in}{5.490039in}}%
\pgfusepath{clip}%
\pgfsetbuttcap%
\pgfsetroundjoin%
\definecolor{currentfill}{rgb}{0.175707,0.697900,0.491033}%
\pgfsetfillcolor{currentfill}%
\pgfsetfillopacity{0.700000}%
\pgfsetlinewidth{0.000000pt}%
\definecolor{currentstroke}{rgb}{0.000000,0.000000,0.000000}%
\pgfsetstrokecolor{currentstroke}%
\pgfsetdash{}{0pt}%
\pgfpathmoveto{\pgfqpoint{5.186065in}{2.915694in}}%
\pgfpathlineto{\pgfqpoint{5.200386in}{2.930151in}}%
\pgfpathlineto{\pgfqpoint{5.214727in}{2.944773in}}%
\pgfpathlineto{\pgfqpoint{5.229088in}{2.959560in}}%
\pgfpathlineto{\pgfqpoint{5.243470in}{2.974512in}}%
\pgfpathlineto{\pgfqpoint{5.250937in}{2.982780in}}%
\pgfpathlineto{\pgfqpoint{5.258394in}{2.990873in}}%
\pgfpathlineto{\pgfqpoint{5.265842in}{2.998792in}}%
\pgfpathlineto{\pgfqpoint{5.273280in}{3.006537in}}%
\pgfpathlineto{\pgfqpoint{5.258904in}{2.991664in}}%
\pgfpathlineto{\pgfqpoint{5.244548in}{2.976956in}}%
\pgfpathlineto{\pgfqpoint{5.230213in}{2.962413in}}%
\pgfpathlineto{\pgfqpoint{5.215897in}{2.948035in}}%
\pgfpathlineto{\pgfqpoint{5.208453in}{2.940199in}}%
\pgfpathlineto{\pgfqpoint{5.200999in}{2.932198in}}%
\pgfpathlineto{\pgfqpoint{5.193536in}{2.924030in}}%
\pgfpathlineto{\pgfqpoint{5.186065in}{2.915694in}}%
\pgfpathclose%
\pgfusepath{fill}%
\end{pgfscope}%
\begin{pgfscope}%
\pgfpathrectangle{\pgfqpoint{1.254980in}{0.150000in}}{\pgfqpoint{5.490039in}{5.490039in}}%
\pgfusepath{clip}%
\pgfsetbuttcap%
\pgfsetroundjoin%
\definecolor{currentfill}{rgb}{0.269944,0.014625,0.341379}%
\pgfsetfillcolor{currentfill}%
\pgfsetfillopacity{0.700000}%
\pgfsetlinewidth{0.000000pt}%
\definecolor{currentstroke}{rgb}{0.000000,0.000000,0.000000}%
\pgfsetstrokecolor{currentstroke}%
\pgfsetdash{}{0pt}%
\pgfpathmoveto{\pgfqpoint{3.327217in}{1.322289in}}%
\pgfpathlineto{\pgfqpoint{3.340745in}{1.316185in}}%
\pgfpathlineto{\pgfqpoint{3.354277in}{1.310249in}}%
\pgfpathlineto{\pgfqpoint{3.367812in}{1.304479in}}%
\pgfpathlineto{\pgfqpoint{3.381349in}{1.298875in}}%
\pgfpathlineto{\pgfqpoint{3.389559in}{1.303697in}}%
\pgfpathlineto{\pgfqpoint{3.397758in}{1.308775in}}%
\pgfpathlineto{\pgfqpoint{3.405946in}{1.314102in}}%
\pgfpathlineto{\pgfqpoint{3.414124in}{1.319670in}}%
\pgfpathlineto{\pgfqpoint{3.400613in}{1.324605in}}%
\pgfpathlineto{\pgfqpoint{3.387105in}{1.329707in}}%
\pgfpathlineto{\pgfqpoint{3.373600in}{1.334974in}}%
\pgfpathlineto{\pgfqpoint{3.360099in}{1.340409in}}%
\pgfpathlineto{\pgfqpoint{3.351895in}{1.335499in}}%
\pgfpathlineto{\pgfqpoint{3.343680in}{1.330837in}}%
\pgfpathlineto{\pgfqpoint{3.335454in}{1.326432in}}%
\pgfpathlineto{\pgfqpoint{3.327217in}{1.322289in}}%
\pgfpathclose%
\pgfusepath{fill}%
\end{pgfscope}%
\begin{pgfscope}%
\pgfpathrectangle{\pgfqpoint{1.254980in}{0.150000in}}{\pgfqpoint{5.490039in}{5.490039in}}%
\pgfusepath{clip}%
\pgfsetbuttcap%
\pgfsetroundjoin%
\definecolor{currentfill}{rgb}{0.304148,0.764704,0.419943}%
\pgfsetfillcolor{currentfill}%
\pgfsetfillopacity{0.700000}%
\pgfsetlinewidth{0.000000pt}%
\definecolor{currentstroke}{rgb}{0.000000,0.000000,0.000000}%
\pgfsetstrokecolor{currentstroke}%
\pgfsetdash{}{0pt}%
\pgfpathmoveto{\pgfqpoint{5.390090in}{3.122565in}}%
\pgfpathlineto{\pgfqpoint{5.404555in}{3.138010in}}%
\pgfpathlineto{\pgfqpoint{5.419041in}{3.153620in}}%
\pgfpathlineto{\pgfqpoint{5.433549in}{3.169397in}}%
\pgfpathlineto{\pgfqpoint{5.448078in}{3.185340in}}%
\pgfpathlineto{\pgfqpoint{5.455410in}{3.191269in}}%
\pgfpathlineto{\pgfqpoint{5.462731in}{3.197026in}}%
\pgfpathlineto{\pgfqpoint{5.470041in}{3.202614in}}%
\pgfpathlineto{\pgfqpoint{5.477341in}{3.208034in}}%
\pgfpathlineto{\pgfqpoint{5.462822in}{3.192267in}}%
\pgfpathlineto{\pgfqpoint{5.448325in}{3.176667in}}%
\pgfpathlineto{\pgfqpoint{5.433850in}{3.161231in}}%
\pgfpathlineto{\pgfqpoint{5.419395in}{3.145961in}}%
\pgfpathlineto{\pgfqpoint{5.412084in}{3.140354in}}%
\pgfpathlineto{\pgfqpoint{5.404763in}{3.134587in}}%
\pgfpathlineto{\pgfqpoint{5.397432in}{3.128658in}}%
\pgfpathlineto{\pgfqpoint{5.390090in}{3.122565in}}%
\pgfpathclose%
\pgfusepath{fill}%
\end{pgfscope}%
\begin{pgfscope}%
\pgfpathrectangle{\pgfqpoint{1.254980in}{0.150000in}}{\pgfqpoint{5.490039in}{5.490039in}}%
\pgfusepath{clip}%
\pgfsetbuttcap%
\pgfsetroundjoin%
\definecolor{currentfill}{rgb}{0.268510,0.009605,0.335427}%
\pgfsetfillcolor{currentfill}%
\pgfsetfillopacity{0.700000}%
\pgfsetlinewidth{0.000000pt}%
\definecolor{currentstroke}{rgb}{0.000000,0.000000,0.000000}%
\pgfsetstrokecolor{currentstroke}%
\pgfsetdash{}{0pt}%
\pgfpathmoveto{\pgfqpoint{3.468208in}{1.301579in}}%
\pgfpathlineto{\pgfqpoint{3.481740in}{1.297466in}}%
\pgfpathlineto{\pgfqpoint{3.495276in}{1.293516in}}%
\pgfpathlineto{\pgfqpoint{3.508816in}{1.289729in}}%
\pgfpathlineto{\pgfqpoint{3.522362in}{1.286105in}}%
\pgfpathlineto{\pgfqpoint{3.530483in}{1.293209in}}%
\pgfpathlineto{\pgfqpoint{3.538596in}{1.300527in}}%
\pgfpathlineto{\pgfqpoint{3.546700in}{1.308053in}}%
\pgfpathlineto{\pgfqpoint{3.554796in}{1.315780in}}%
\pgfpathlineto{\pgfqpoint{3.541271in}{1.318765in}}%
\pgfpathlineto{\pgfqpoint{3.527751in}{1.321913in}}%
\pgfpathlineto{\pgfqpoint{3.514237in}{1.325223in}}%
\pgfpathlineto{\pgfqpoint{3.500727in}{1.328697in}}%
\pgfpathlineto{\pgfqpoint{3.492611in}{1.321598in}}%
\pgfpathlineto{\pgfqpoint{3.484486in}{1.314708in}}%
\pgfpathlineto{\pgfqpoint{3.476352in}{1.308032in}}%
\pgfpathlineto{\pgfqpoint{3.468208in}{1.301579in}}%
\pgfpathclose%
\pgfusepath{fill}%
\end{pgfscope}%
\begin{pgfscope}%
\pgfpathrectangle{\pgfqpoint{1.254980in}{0.150000in}}{\pgfqpoint{5.490039in}{5.490039in}}%
\pgfusepath{clip}%
\pgfsetbuttcap%
\pgfsetroundjoin%
\definecolor{currentfill}{rgb}{0.146180,0.515413,0.556823}%
\pgfsetfillcolor{currentfill}%
\pgfsetfillopacity{0.700000}%
\pgfsetlinewidth{0.000000pt}%
\definecolor{currentstroke}{rgb}{0.000000,0.000000,0.000000}%
\pgfsetstrokecolor{currentstroke}%
\pgfsetdash{}{0pt}%
\pgfpathmoveto{\pgfqpoint{4.746804in}{2.408037in}}%
\pgfpathlineto{\pgfqpoint{4.760832in}{2.419554in}}%
\pgfpathlineto{\pgfqpoint{4.774877in}{2.431234in}}%
\pgfpathlineto{\pgfqpoint{4.788939in}{2.443075in}}%
\pgfpathlineto{\pgfqpoint{4.803019in}{2.455079in}}%
\pgfpathlineto{\pgfqpoint{4.810710in}{2.467817in}}%
\pgfpathlineto{\pgfqpoint{4.818396in}{2.480410in}}%
\pgfpathlineto{\pgfqpoint{4.826074in}{2.492856in}}%
\pgfpathlineto{\pgfqpoint{4.833746in}{2.505155in}}%
\pgfpathlineto{\pgfqpoint{4.819664in}{2.493012in}}%
\pgfpathlineto{\pgfqpoint{4.805600in}{2.481032in}}%
\pgfpathlineto{\pgfqpoint{4.791553in}{2.469214in}}%
\pgfpathlineto{\pgfqpoint{4.777523in}{2.457559in}}%
\pgfpathlineto{\pgfqpoint{4.769853in}{2.445387in}}%
\pgfpathlineto{\pgfqpoint{4.762176in}{2.433076in}}%
\pgfpathlineto{\pgfqpoint{4.754493in}{2.420626in}}%
\pgfpathlineto{\pgfqpoint{4.746804in}{2.408037in}}%
\pgfpathclose%
\pgfusepath{fill}%
\end{pgfscope}%
\begin{pgfscope}%
\pgfpathrectangle{\pgfqpoint{1.254980in}{0.150000in}}{\pgfqpoint{5.490039in}{5.490039in}}%
\pgfusepath{clip}%
\pgfsetbuttcap%
\pgfsetroundjoin%
\definecolor{currentfill}{rgb}{0.277134,0.185228,0.489898}%
\pgfsetfillcolor{currentfill}%
\pgfsetfillopacity{0.700000}%
\pgfsetlinewidth{0.000000pt}%
\definecolor{currentstroke}{rgb}{0.000000,0.000000,0.000000}%
\pgfsetstrokecolor{currentstroke}%
\pgfsetdash{}{0pt}%
\pgfpathmoveto{\pgfqpoint{4.071980in}{1.601201in}}%
\pgfpathlineto{\pgfqpoint{4.085648in}{1.605457in}}%
\pgfpathlineto{\pgfqpoint{4.099328in}{1.609871in}}%
\pgfpathlineto{\pgfqpoint{4.113019in}{1.614442in}}%
\pgfpathlineto{\pgfqpoint{4.126721in}{1.619170in}}%
\pgfpathlineto{\pgfqpoint{4.134607in}{1.633326in}}%
\pgfpathlineto{\pgfqpoint{4.142488in}{1.647498in}}%
\pgfpathlineto{\pgfqpoint{4.150366in}{1.661682in}}%
\pgfpathlineto{\pgfqpoint{4.158239in}{1.675875in}}%
\pgfpathlineto{\pgfqpoint{4.144538in}{1.670695in}}%
\pgfpathlineto{\pgfqpoint{4.130848in}{1.665672in}}%
\pgfpathlineto{\pgfqpoint{4.117170in}{1.660807in}}%
\pgfpathlineto{\pgfqpoint{4.103503in}{1.656101in}}%
\pgfpathlineto{\pgfqpoint{4.095629in}{1.642349in}}%
\pgfpathlineto{\pgfqpoint{4.087750in}{1.628612in}}%
\pgfpathlineto{\pgfqpoint{4.079867in}{1.614895in}}%
\pgfpathlineto{\pgfqpoint{4.071980in}{1.601201in}}%
\pgfpathclose%
\pgfusepath{fill}%
\end{pgfscope}%
\begin{pgfscope}%
\pgfpathrectangle{\pgfqpoint{1.254980in}{0.150000in}}{\pgfqpoint{5.490039in}{5.490039in}}%
\pgfusepath{clip}%
\pgfsetbuttcap%
\pgfsetroundjoin%
\definecolor{currentfill}{rgb}{0.166617,0.463708,0.558119}%
\pgfsetfillcolor{currentfill}%
\pgfsetfillopacity{0.700000}%
\pgfsetlinewidth{0.000000pt}%
\definecolor{currentstroke}{rgb}{0.000000,0.000000,0.000000}%
\pgfsetstrokecolor{currentstroke}%
\pgfsetdash{}{0pt}%
\pgfpathmoveto{\pgfqpoint{4.629152in}{2.259532in}}%
\pgfpathlineto{\pgfqpoint{4.643108in}{2.270036in}}%
\pgfpathlineto{\pgfqpoint{4.657080in}{2.280702in}}%
\pgfpathlineto{\pgfqpoint{4.671069in}{2.291529in}}%
\pgfpathlineto{\pgfqpoint{4.685074in}{2.302517in}}%
\pgfpathlineto{\pgfqpoint{4.692811in}{2.316162in}}%
\pgfpathlineto{\pgfqpoint{4.700542in}{2.329680in}}%
\pgfpathlineto{\pgfqpoint{4.708267in}{2.343070in}}%
\pgfpathlineto{\pgfqpoint{4.715986in}{2.356329in}}%
\pgfpathlineto{\pgfqpoint{4.701978in}{2.345142in}}%
\pgfpathlineto{\pgfqpoint{4.687986in}{2.334117in}}%
\pgfpathlineto{\pgfqpoint{4.674011in}{2.323254in}}%
\pgfpathlineto{\pgfqpoint{4.660052in}{2.312552in}}%
\pgfpathlineto{\pgfqpoint{4.652335in}{2.299479in}}%
\pgfpathlineto{\pgfqpoint{4.644613in}{2.286284in}}%
\pgfpathlineto{\pgfqpoint{4.636885in}{2.272968in}}%
\pgfpathlineto{\pgfqpoint{4.629152in}{2.259532in}}%
\pgfpathclose%
\pgfusepath{fill}%
\end{pgfscope}%
\begin{pgfscope}%
\pgfpathrectangle{\pgfqpoint{1.254980in}{0.150000in}}{\pgfqpoint{5.490039in}{5.490039in}}%
\pgfusepath{clip}%
\pgfsetbuttcap%
\pgfsetroundjoin%
\definecolor{currentfill}{rgb}{0.190631,0.407061,0.556089}%
\pgfsetfillcolor{currentfill}%
\pgfsetfillopacity{0.700000}%
\pgfsetlinewidth{0.000000pt}%
\definecolor{currentstroke}{rgb}{0.000000,0.000000,0.000000}%
\pgfsetstrokecolor{currentstroke}%
\pgfsetdash{}{0pt}%
\pgfpathmoveto{\pgfqpoint{4.511455in}{2.109496in}}%
\pgfpathlineto{\pgfqpoint{4.525342in}{2.118874in}}%
\pgfpathlineto{\pgfqpoint{4.539245in}{2.128411in}}%
\pgfpathlineto{\pgfqpoint{4.553162in}{2.138110in}}%
\pgfpathlineto{\pgfqpoint{4.567096in}{2.147968in}}%
\pgfpathlineto{\pgfqpoint{4.574871in}{2.162291in}}%
\pgfpathlineto{\pgfqpoint{4.582641in}{2.176511in}}%
\pgfpathlineto{\pgfqpoint{4.590405in}{2.190625in}}%
\pgfpathlineto{\pgfqpoint{4.598165in}{2.204631in}}%
\pgfpathlineto{\pgfqpoint{4.584228in}{2.194515in}}%
\pgfpathlineto{\pgfqpoint{4.570307in}{2.184560in}}%
\pgfpathlineto{\pgfqpoint{4.556401in}{2.174766in}}%
\pgfpathlineto{\pgfqpoint{4.542511in}{2.165132in}}%
\pgfpathlineto{\pgfqpoint{4.534754in}{2.151372in}}%
\pgfpathlineto{\pgfqpoint{4.526993in}{2.137511in}}%
\pgfpathlineto{\pgfqpoint{4.519226in}{2.123551in}}%
\pgfpathlineto{\pgfqpoint{4.511455in}{2.109496in}}%
\pgfpathclose%
\pgfusepath{fill}%
\end{pgfscope}%
\begin{pgfscope}%
\pgfpathrectangle{\pgfqpoint{1.254980in}{0.150000in}}{\pgfqpoint{5.490039in}{5.490039in}}%
\pgfusepath{clip}%
\pgfsetbuttcap%
\pgfsetroundjoin%
\definecolor{currentfill}{rgb}{0.278791,0.062145,0.386592}%
\pgfsetfillcolor{currentfill}%
\pgfsetfillopacity{0.700000}%
\pgfsetlinewidth{0.000000pt}%
\definecolor{currentstroke}{rgb}{0.000000,0.000000,0.000000}%
\pgfsetstrokecolor{currentstroke}%
\pgfsetdash{}{0pt}%
\pgfpathmoveto{\pgfqpoint{3.781661in}{1.375180in}}%
\pgfpathlineto{\pgfqpoint{3.795241in}{1.375471in}}%
\pgfpathlineto{\pgfqpoint{3.808828in}{1.375920in}}%
\pgfpathlineto{\pgfqpoint{3.822424in}{1.376527in}}%
\pgfpathlineto{\pgfqpoint{3.836029in}{1.377292in}}%
\pgfpathlineto{\pgfqpoint{3.844004in}{1.388831in}}%
\pgfpathlineto{\pgfqpoint{3.851975in}{1.400485in}}%
\pgfpathlineto{\pgfqpoint{3.859940in}{1.412249in}}%
\pgfpathlineto{\pgfqpoint{3.867899in}{1.424117in}}%
\pgfpathlineto{\pgfqpoint{3.854304in}{1.422794in}}%
\pgfpathlineto{\pgfqpoint{3.840717in}{1.421629in}}%
\pgfpathlineto{\pgfqpoint{3.827139in}{1.420622in}}%
\pgfpathlineto{\pgfqpoint{3.813569in}{1.419773in}}%
\pgfpathlineto{\pgfqpoint{3.805600in}{1.408453in}}%
\pgfpathlineto{\pgfqpoint{3.797626in}{1.397243in}}%
\pgfpathlineto{\pgfqpoint{3.789646in}{1.386151in}}%
\pgfpathlineto{\pgfqpoint{3.781661in}{1.375180in}}%
\pgfpathclose%
\pgfusepath{fill}%
\end{pgfscope}%
\begin{pgfscope}%
\pgfpathrectangle{\pgfqpoint{1.254980in}{0.150000in}}{\pgfqpoint{5.490039in}{5.490039in}}%
\pgfusepath{clip}%
\pgfsetbuttcap%
\pgfsetroundjoin%
\definecolor{currentfill}{rgb}{0.274952,0.037752,0.364543}%
\pgfsetfillcolor{currentfill}%
\pgfsetfillopacity{0.700000}%
\pgfsetlinewidth{0.000000pt}%
\definecolor{currentstroke}{rgb}{0.000000,0.000000,0.000000}%
\pgfsetstrokecolor{currentstroke}%
\pgfsetdash{}{0pt}%
\pgfpathmoveto{\pgfqpoint{3.185739in}{1.368997in}}%
\pgfpathlineto{\pgfqpoint{3.199285in}{1.360834in}}%
\pgfpathlineto{\pgfqpoint{3.212832in}{1.352844in}}%
\pgfpathlineto{\pgfqpoint{3.226380in}{1.345025in}}%
\pgfpathlineto{\pgfqpoint{3.239929in}{1.337377in}}%
\pgfpathlineto{\pgfqpoint{3.248248in}{1.339689in}}%
\pgfpathlineto{\pgfqpoint{3.256554in}{1.342299in}}%
\pgfpathlineto{\pgfqpoint{3.264847in}{1.345201in}}%
\pgfpathlineto{\pgfqpoint{3.273126in}{1.348386in}}%
\pgfpathlineto{\pgfqpoint{3.259609in}{1.355334in}}%
\pgfpathlineto{\pgfqpoint{3.246094in}{1.362453in}}%
\pgfpathlineto{\pgfqpoint{3.232581in}{1.369742in}}%
\pgfpathlineto{\pgfqpoint{3.219069in}{1.377204in}}%
\pgfpathlineto{\pgfqpoint{3.210757in}{1.374708in}}%
\pgfpathlineto{\pgfqpoint{3.202432in}{1.372503in}}%
\pgfpathlineto{\pgfqpoint{3.194093in}{1.370597in}}%
\pgfpathlineto{\pgfqpoint{3.185739in}{1.368997in}}%
\pgfpathclose%
\pgfusepath{fill}%
\end{pgfscope}%
\begin{pgfscope}%
\pgfpathrectangle{\pgfqpoint{1.254980in}{0.150000in}}{\pgfqpoint{5.490039in}{5.490039in}}%
\pgfusepath{clip}%
\pgfsetbuttcap%
\pgfsetroundjoin%
\definecolor{currentfill}{rgb}{0.216210,0.351535,0.550627}%
\pgfsetfillcolor{currentfill}%
\pgfsetfillopacity{0.700000}%
\pgfsetlinewidth{0.000000pt}%
\definecolor{currentstroke}{rgb}{0.000000,0.000000,0.000000}%
\pgfsetstrokecolor{currentstroke}%
\pgfsetdash{}{0pt}%
\pgfpathmoveto{\pgfqpoint{4.393738in}{1.960359in}}%
\pgfpathlineto{\pgfqpoint{4.407561in}{1.968499in}}%
\pgfpathlineto{\pgfqpoint{4.421398in}{1.976798in}}%
\pgfpathlineto{\pgfqpoint{4.435249in}{1.985256in}}%
\pgfpathlineto{\pgfqpoint{4.449115in}{1.993874in}}%
\pgfpathlineto{\pgfqpoint{4.456923in}{2.008611in}}%
\pgfpathlineto{\pgfqpoint{4.464727in}{2.023273in}}%
\pgfpathlineto{\pgfqpoint{4.472527in}{2.037856in}}%
\pgfpathlineto{\pgfqpoint{4.480321in}{2.052357in}}%
\pgfpathlineto{\pgfqpoint{4.466452in}{2.043425in}}%
\pgfpathlineto{\pgfqpoint{4.452598in}{2.034652in}}%
\pgfpathlineto{\pgfqpoint{4.438758in}{2.026040in}}%
\pgfpathlineto{\pgfqpoint{4.424933in}{2.017587in}}%
\pgfpathlineto{\pgfqpoint{4.417141in}{2.003388in}}%
\pgfpathlineto{\pgfqpoint{4.409344in}{1.989116in}}%
\pgfpathlineto{\pgfqpoint{4.401543in}{1.974772in}}%
\pgfpathlineto{\pgfqpoint{4.393738in}{1.960359in}}%
\pgfpathclose%
\pgfusepath{fill}%
\end{pgfscope}%
\begin{pgfscope}%
\pgfpathrectangle{\pgfqpoint{1.254980in}{0.150000in}}{\pgfqpoint{5.490039in}{5.490039in}}%
\pgfusepath{clip}%
\pgfsetbuttcap%
\pgfsetroundjoin%
\definecolor{currentfill}{rgb}{0.274952,0.037752,0.364543}%
\pgfsetfillcolor{currentfill}%
\pgfsetfillopacity{0.700000}%
\pgfsetlinewidth{0.000000pt}%
\definecolor{currentstroke}{rgb}{0.000000,0.000000,0.000000}%
\pgfsetstrokecolor{currentstroke}%
\pgfsetdash{}{0pt}%
\pgfpathmoveto{\pgfqpoint{3.695362in}{1.335392in}}%
\pgfpathlineto{\pgfqpoint{3.708925in}{1.334463in}}%
\pgfpathlineto{\pgfqpoint{3.722494in}{1.333693in}}%
\pgfpathlineto{\pgfqpoint{3.736071in}{1.333082in}}%
\pgfpathlineto{\pgfqpoint{3.749655in}{1.332629in}}%
\pgfpathlineto{\pgfqpoint{3.757666in}{1.343056in}}%
\pgfpathlineto{\pgfqpoint{3.765670in}{1.353627in}}%
\pgfpathlineto{\pgfqpoint{3.773669in}{1.364337in}}%
\pgfpathlineto{\pgfqpoint{3.781661in}{1.375180in}}%
\pgfpathlineto{\pgfqpoint{3.768088in}{1.375048in}}%
\pgfpathlineto{\pgfqpoint{3.754524in}{1.375074in}}%
\pgfpathlineto{\pgfqpoint{3.740967in}{1.375259in}}%
\pgfpathlineto{\pgfqpoint{3.727417in}{1.375604in}}%
\pgfpathlineto{\pgfqpoint{3.719413in}{1.365335in}}%
\pgfpathlineto{\pgfqpoint{3.711403in}{1.355206in}}%
\pgfpathlineto{\pgfqpoint{3.703386in}{1.345223in}}%
\pgfpathlineto{\pgfqpoint{3.695362in}{1.335392in}}%
\pgfpathclose%
\pgfusepath{fill}%
\end{pgfscope}%
\begin{pgfscope}%
\pgfpathrectangle{\pgfqpoint{1.254980in}{0.150000in}}{\pgfqpoint{5.490039in}{5.490039in}}%
\pgfusepath{clip}%
\pgfsetbuttcap%
\pgfsetroundjoin%
\definecolor{currentfill}{rgb}{0.132268,0.655014,0.519661}%
\pgfsetfillcolor{currentfill}%
\pgfsetfillopacity{0.700000}%
\pgfsetlinewidth{0.000000pt}%
\definecolor{currentstroke}{rgb}{0.000000,0.000000,0.000000}%
\pgfsetstrokecolor{currentstroke}%
\pgfsetdash{}{0pt}%
\pgfpathmoveto{\pgfqpoint{5.068856in}{2.786414in}}%
\pgfpathlineto{\pgfqpoint{5.083105in}{2.800276in}}%
\pgfpathlineto{\pgfqpoint{5.097373in}{2.814302in}}%
\pgfpathlineto{\pgfqpoint{5.111661in}{2.828493in}}%
\pgfpathlineto{\pgfqpoint{5.125969in}{2.842848in}}%
\pgfpathlineto{\pgfqpoint{5.133512in}{2.852560in}}%
\pgfpathlineto{\pgfqpoint{5.141046in}{2.862098in}}%
\pgfpathlineto{\pgfqpoint{5.148571in}{2.871461in}}%
\pgfpathlineto{\pgfqpoint{5.156087in}{2.880651in}}%
\pgfpathlineto{\pgfqpoint{5.141782in}{2.866312in}}%
\pgfpathlineto{\pgfqpoint{5.127497in}{2.852137in}}%
\pgfpathlineto{\pgfqpoint{5.113232in}{2.838126in}}%
\pgfpathlineto{\pgfqpoint{5.098985in}{2.824280in}}%
\pgfpathlineto{\pgfqpoint{5.091466in}{2.815062in}}%
\pgfpathlineto{\pgfqpoint{5.083937in}{2.805679in}}%
\pgfpathlineto{\pgfqpoint{5.076401in}{2.796130in}}%
\pgfpathlineto{\pgfqpoint{5.068856in}{2.786414in}}%
\pgfpathclose%
\pgfusepath{fill}%
\end{pgfscope}%
\begin{pgfscope}%
\pgfpathrectangle{\pgfqpoint{1.254980in}{0.150000in}}{\pgfqpoint{5.490039in}{5.490039in}}%
\pgfusepath{clip}%
\pgfsetbuttcap%
\pgfsetroundjoin%
\definecolor{currentfill}{rgb}{0.282327,0.094955,0.417331}%
\pgfsetfillcolor{currentfill}%
\pgfsetfillopacity{0.700000}%
\pgfsetlinewidth{0.000000pt}%
\definecolor{currentstroke}{rgb}{0.000000,0.000000,0.000000}%
\pgfsetstrokecolor{currentstroke}%
\pgfsetdash{}{0pt}%
\pgfpathmoveto{\pgfqpoint{3.867899in}{1.424117in}}%
\pgfpathlineto{\pgfqpoint{3.881503in}{1.425598in}}%
\pgfpathlineto{\pgfqpoint{3.895116in}{1.427236in}}%
\pgfpathlineto{\pgfqpoint{3.908738in}{1.429032in}}%
\pgfpathlineto{\pgfqpoint{3.922369in}{1.430985in}}%
\pgfpathlineto{\pgfqpoint{3.930315in}{1.443494in}}%
\pgfpathlineto{\pgfqpoint{3.938257in}{1.456091in}}%
\pgfpathlineto{\pgfqpoint{3.946194in}{1.468769in}}%
\pgfpathlineto{\pgfqpoint{3.954126in}{1.481524in}}%
\pgfpathlineto{\pgfqpoint{3.940501in}{1.479039in}}%
\pgfpathlineto{\pgfqpoint{3.926886in}{1.476711in}}%
\pgfpathlineto{\pgfqpoint{3.913280in}{1.474541in}}%
\pgfpathlineto{\pgfqpoint{3.899683in}{1.472530in}}%
\pgfpathlineto{\pgfqpoint{3.891745in}{1.460296in}}%
\pgfpathlineto{\pgfqpoint{3.883801in}{1.448145in}}%
\pgfpathlineto{\pgfqpoint{3.875853in}{1.436084in}}%
\pgfpathlineto{\pgfqpoint{3.867899in}{1.424117in}}%
\pgfpathclose%
\pgfusepath{fill}%
\end{pgfscope}%
\begin{pgfscope}%
\pgfpathrectangle{\pgfqpoint{1.254980in}{0.150000in}}{\pgfqpoint{5.490039in}{5.490039in}}%
\pgfusepath{clip}%
\pgfsetbuttcap%
\pgfsetroundjoin%
\definecolor{currentfill}{rgb}{0.244972,0.287675,0.537260}%
\pgfsetfillcolor{currentfill}%
\pgfsetfillopacity{0.700000}%
\pgfsetlinewidth{0.000000pt}%
\definecolor{currentstroke}{rgb}{0.000000,0.000000,0.000000}%
\pgfsetstrokecolor{currentstroke}%
\pgfsetdash{}{0pt}%
\pgfpathmoveto{\pgfqpoint{4.276006in}{1.814827in}}%
\pgfpathlineto{\pgfqpoint{4.289771in}{1.821620in}}%
\pgfpathlineto{\pgfqpoint{4.303548in}{1.828571in}}%
\pgfpathlineto{\pgfqpoint{4.317338in}{1.835680in}}%
\pgfpathlineto{\pgfqpoint{4.331142in}{1.842948in}}%
\pgfpathlineto{\pgfqpoint{4.338981in}{1.857800in}}%
\pgfpathlineto{\pgfqpoint{4.346816in}{1.872609in}}%
\pgfpathlineto{\pgfqpoint{4.354647in}{1.887370in}}%
\pgfpathlineto{\pgfqpoint{4.362474in}{1.902081in}}%
\pgfpathlineto{\pgfqpoint{4.348667in}{1.894443in}}%
\pgfpathlineto{\pgfqpoint{4.334875in}{1.886963in}}%
\pgfpathlineto{\pgfqpoint{4.321096in}{1.879642in}}%
\pgfpathlineto{\pgfqpoint{4.307330in}{1.872480in}}%
\pgfpathlineto{\pgfqpoint{4.299505in}{1.858128in}}%
\pgfpathlineto{\pgfqpoint{4.291677in}{1.843733in}}%
\pgfpathlineto{\pgfqpoint{4.283844in}{1.829299in}}%
\pgfpathlineto{\pgfqpoint{4.276006in}{1.814827in}}%
\pgfpathclose%
\pgfusepath{fill}%
\end{pgfscope}%
\begin{pgfscope}%
\pgfpathrectangle{\pgfqpoint{1.254980in}{0.150000in}}{\pgfqpoint{5.490039in}{5.490039in}}%
\pgfusepath{clip}%
\pgfsetbuttcap%
\pgfsetroundjoin%
\definecolor{currentfill}{rgb}{0.271305,0.019942,0.347269}%
\pgfsetfillcolor{currentfill}%
\pgfsetfillopacity{0.700000}%
\pgfsetlinewidth{0.000000pt}%
\definecolor{currentstroke}{rgb}{0.000000,0.000000,0.000000}%
\pgfsetstrokecolor{currentstroke}%
\pgfsetdash{}{0pt}%
\pgfpathmoveto{\pgfqpoint{3.608949in}{1.305455in}}%
\pgfpathlineto{\pgfqpoint{3.622502in}{1.303275in}}%
\pgfpathlineto{\pgfqpoint{3.636060in}{1.301256in}}%
\pgfpathlineto{\pgfqpoint{3.649625in}{1.299396in}}%
\pgfpathlineto{\pgfqpoint{3.663196in}{1.297696in}}%
\pgfpathlineto{\pgfqpoint{3.671248in}{1.306863in}}%
\pgfpathlineto{\pgfqpoint{3.679293in}{1.316206in}}%
\pgfpathlineto{\pgfqpoint{3.687331in}{1.325717in}}%
\pgfpathlineto{\pgfqpoint{3.695362in}{1.335392in}}%
\pgfpathlineto{\pgfqpoint{3.681806in}{1.336480in}}%
\pgfpathlineto{\pgfqpoint{3.668257in}{1.337727in}}%
\pgfpathlineto{\pgfqpoint{3.654714in}{1.339135in}}%
\pgfpathlineto{\pgfqpoint{3.641178in}{1.340703in}}%
\pgfpathlineto{\pgfqpoint{3.633132in}{1.331629in}}%
\pgfpathlineto{\pgfqpoint{3.625079in}{1.322726in}}%
\pgfpathlineto{\pgfqpoint{3.617018in}{1.314000in}}%
\pgfpathlineto{\pgfqpoint{3.608949in}{1.305455in}}%
\pgfpathclose%
\pgfusepath{fill}%
\end{pgfscope}%
\begin{pgfscope}%
\pgfpathrectangle{\pgfqpoint{1.254980in}{0.150000in}}{\pgfqpoint{5.490039in}{5.490039in}}%
\pgfusepath{clip}%
\pgfsetbuttcap%
\pgfsetroundjoin%
\definecolor{currentfill}{rgb}{0.119512,0.607464,0.540218}%
\pgfsetfillcolor{currentfill}%
\pgfsetfillopacity{0.700000}%
\pgfsetlinewidth{0.000000pt}%
\definecolor{currentstroke}{rgb}{0.000000,0.000000,0.000000}%
\pgfsetstrokecolor{currentstroke}%
\pgfsetdash{}{0pt}%
\pgfpathmoveto{\pgfqpoint{2.033418in}{2.776701in}}%
\pgfpathlineto{\pgfqpoint{2.047560in}{2.749742in}}%
\pgfpathlineto{\pgfqpoint{2.061685in}{2.723090in}}%
\pgfpathlineto{\pgfqpoint{2.075792in}{2.696740in}}%
\pgfpathlineto{\pgfqpoint{2.089883in}{2.670691in}}%
\pgfpathlineto{\pgfqpoint{2.099360in}{2.657361in}}%
\pgfpathlineto{\pgfqpoint{2.108805in}{2.644512in}}%
\pgfpathlineto{\pgfqpoint{2.118218in}{2.632134in}}%
\pgfpathlineto{\pgfqpoint{2.127599in}{2.620221in}}%
\pgfpathlineto{\pgfqpoint{2.113587in}{2.645479in}}%
\pgfpathlineto{\pgfqpoint{2.099559in}{2.671034in}}%
\pgfpathlineto{\pgfqpoint{2.085514in}{2.696890in}}%
\pgfpathlineto{\pgfqpoint{2.071452in}{2.723050in}}%
\pgfpathlineto{\pgfqpoint{2.061993in}{2.735744in}}%
\pgfpathlineto{\pgfqpoint{2.052502in}{2.748911in}}%
\pgfpathlineto{\pgfqpoint{2.042977in}{2.762560in}}%
\pgfpathlineto{\pgfqpoint{2.033418in}{2.776701in}}%
\pgfpathclose%
\pgfusepath{fill}%
\end{pgfscope}%
\begin{pgfscope}%
\pgfpathrectangle{\pgfqpoint{1.254980in}{0.150000in}}{\pgfqpoint{5.490039in}{5.490039in}}%
\pgfusepath{clip}%
\pgfsetbuttcap%
\pgfsetroundjoin%
\definecolor{currentfill}{rgb}{0.386433,0.794644,0.372886}%
\pgfsetfillcolor{currentfill}%
\pgfsetfillopacity{0.700000}%
\pgfsetlinewidth{0.000000pt}%
\definecolor{currentstroke}{rgb}{0.000000,0.000000,0.000000}%
\pgfsetstrokecolor{currentstroke}%
\pgfsetdash{}{0pt}%
\pgfpathmoveto{\pgfqpoint{5.477341in}{3.208034in}}%
\pgfpathlineto{\pgfqpoint{5.491881in}{3.223966in}}%
\pgfpathlineto{\pgfqpoint{5.506444in}{3.240065in}}%
\pgfpathlineto{\pgfqpoint{5.521028in}{3.256330in}}%
\pgfpathlineto{\pgfqpoint{5.535635in}{3.272762in}}%
\pgfpathlineto{\pgfqpoint{5.542912in}{3.277820in}}%
\pgfpathlineto{\pgfqpoint{5.550178in}{3.282707in}}%
\pgfpathlineto{\pgfqpoint{5.557433in}{3.287426in}}%
\pgfpathlineto{\pgfqpoint{5.564677in}{3.291979in}}%
\pgfpathlineto{\pgfqpoint{5.550083in}{3.275757in}}%
\pgfpathlineto{\pgfqpoint{5.535512in}{3.259701in}}%
\pgfpathlineto{\pgfqpoint{5.520962in}{3.243811in}}%
\pgfpathlineto{\pgfqpoint{5.506434in}{3.228086in}}%
\pgfpathlineto{\pgfqpoint{5.499177in}{3.223312in}}%
\pgfpathlineto{\pgfqpoint{5.491909in}{3.218381in}}%
\pgfpathlineto{\pgfqpoint{5.484630in}{3.213289in}}%
\pgfpathlineto{\pgfqpoint{5.477341in}{3.208034in}}%
\pgfpathclose%
\pgfusepath{fill}%
\end{pgfscope}%
\begin{pgfscope}%
\pgfpathrectangle{\pgfqpoint{1.254980in}{0.150000in}}{\pgfqpoint{5.490039in}{5.490039in}}%
\pgfusepath{clip}%
\pgfsetbuttcap%
\pgfsetroundjoin%
\definecolor{currentfill}{rgb}{0.239374,0.735588,0.455688}%
\pgfsetfillcolor{currentfill}%
\pgfsetfillopacity{0.700000}%
\pgfsetlinewidth{0.000000pt}%
\definecolor{currentstroke}{rgb}{0.000000,0.000000,0.000000}%
\pgfsetstrokecolor{currentstroke}%
\pgfsetdash{}{0pt}%
\pgfpathmoveto{\pgfqpoint{5.273280in}{3.006537in}}%
\pgfpathlineto{\pgfqpoint{5.287677in}{3.021575in}}%
\pgfpathlineto{\pgfqpoint{5.302095in}{3.036778in}}%
\pgfpathlineto{\pgfqpoint{5.316534in}{3.052147in}}%
\pgfpathlineto{\pgfqpoint{5.330994in}{3.067682in}}%
\pgfpathlineto{\pgfqpoint{5.338416in}{3.075155in}}%
\pgfpathlineto{\pgfqpoint{5.345828in}{3.082450in}}%
\pgfpathlineto{\pgfqpoint{5.353230in}{3.089568in}}%
\pgfpathlineto{\pgfqpoint{5.360622in}{3.096511in}}%
\pgfpathlineto{\pgfqpoint{5.346170in}{3.081088in}}%
\pgfpathlineto{\pgfqpoint{5.331738in}{3.065831in}}%
\pgfpathlineto{\pgfqpoint{5.317328in}{3.050739in}}%
\pgfpathlineto{\pgfqpoint{5.302938in}{3.035812in}}%
\pgfpathlineto{\pgfqpoint{5.295538in}{3.028746in}}%
\pgfpathlineto{\pgfqpoint{5.288128in}{3.021512in}}%
\pgfpathlineto{\pgfqpoint{5.280709in}{3.014110in}}%
\pgfpathlineto{\pgfqpoint{5.273280in}{3.006537in}}%
\pgfpathclose%
\pgfusepath{fill}%
\end{pgfscope}%
\begin{pgfscope}%
\pgfpathrectangle{\pgfqpoint{1.254980in}{0.150000in}}{\pgfqpoint{5.490039in}{5.490039in}}%
\pgfusepath{clip}%
\pgfsetbuttcap%
\pgfsetroundjoin%
\definecolor{currentfill}{rgb}{0.283072,0.130895,0.449241}%
\pgfsetfillcolor{currentfill}%
\pgfsetfillopacity{0.700000}%
\pgfsetlinewidth{0.000000pt}%
\definecolor{currentstroke}{rgb}{0.000000,0.000000,0.000000}%
\pgfsetstrokecolor{currentstroke}%
\pgfsetdash{}{0pt}%
\pgfpathmoveto{\pgfqpoint{3.954126in}{1.481524in}}%
\pgfpathlineto{\pgfqpoint{3.967761in}{1.484167in}}%
\pgfpathlineto{\pgfqpoint{3.981405in}{1.486967in}}%
\pgfpathlineto{\pgfqpoint{3.995059in}{1.489924in}}%
\pgfpathlineto{\pgfqpoint{4.008724in}{1.493038in}}%
\pgfpathlineto{\pgfqpoint{4.016646in}{1.506381in}}%
\pgfpathlineto{\pgfqpoint{4.024564in}{1.519784in}}%
\pgfpathlineto{\pgfqpoint{4.032478in}{1.533243in}}%
\pgfpathlineto{\pgfqpoint{4.040387in}{1.546752in}}%
\pgfpathlineto{\pgfqpoint{4.026726in}{1.543132in}}%
\pgfpathlineto{\pgfqpoint{4.013076in}{1.539669in}}%
\pgfpathlineto{\pgfqpoint{3.999436in}{1.536364in}}%
\pgfpathlineto{\pgfqpoint{3.985807in}{1.533216in}}%
\pgfpathlineto{\pgfqpoint{3.977894in}{1.520202in}}%
\pgfpathlineto{\pgfqpoint{3.969976in}{1.507246in}}%
\pgfpathlineto{\pgfqpoint{3.962053in}{1.494351in}}%
\pgfpathlineto{\pgfqpoint{3.954126in}{1.481524in}}%
\pgfpathclose%
\pgfusepath{fill}%
\end{pgfscope}%
\begin{pgfscope}%
\pgfpathrectangle{\pgfqpoint{1.254980in}{0.150000in}}{\pgfqpoint{5.490039in}{5.490039in}}%
\pgfusepath{clip}%
\pgfsetbuttcap%
\pgfsetroundjoin%
\definecolor{currentfill}{rgb}{0.268510,0.009605,0.335427}%
\pgfsetfillcolor{currentfill}%
\pgfsetfillopacity{0.700000}%
\pgfsetlinewidth{0.000000pt}%
\definecolor{currentstroke}{rgb}{0.000000,0.000000,0.000000}%
\pgfsetstrokecolor{currentstroke}%
\pgfsetdash{}{0pt}%
\pgfpathmoveto{\pgfqpoint{3.381349in}{1.298875in}}%
\pgfpathlineto{\pgfqpoint{3.394890in}{1.293436in}}%
\pgfpathlineto{\pgfqpoint{3.408435in}{1.288163in}}%
\pgfpathlineto{\pgfqpoint{3.421983in}{1.283054in}}%
\pgfpathlineto{\pgfqpoint{3.435534in}{1.278110in}}%
\pgfpathlineto{\pgfqpoint{3.443718in}{1.283612in}}%
\pgfpathlineto{\pgfqpoint{3.451892in}{1.289362in}}%
\pgfpathlineto{\pgfqpoint{3.460055in}{1.295353in}}%
\pgfpathlineto{\pgfqpoint{3.468208in}{1.301579in}}%
\pgfpathlineto{\pgfqpoint{3.454681in}{1.305855in}}%
\pgfpathlineto{\pgfqpoint{3.441158in}{1.310295in}}%
\pgfpathlineto{\pgfqpoint{3.427639in}{1.314900in}}%
\pgfpathlineto{\pgfqpoint{3.414124in}{1.319670in}}%
\pgfpathlineto{\pgfqpoint{3.405946in}{1.314102in}}%
\pgfpathlineto{\pgfqpoint{3.397758in}{1.308775in}}%
\pgfpathlineto{\pgfqpoint{3.389559in}{1.303697in}}%
\pgfpathlineto{\pgfqpoint{3.381349in}{1.298875in}}%
\pgfpathclose%
\pgfusepath{fill}%
\end{pgfscope}%
\begin{pgfscope}%
\pgfpathrectangle{\pgfqpoint{1.254980in}{0.150000in}}{\pgfqpoint{5.490039in}{5.490039in}}%
\pgfusepath{clip}%
\pgfsetbuttcap%
\pgfsetroundjoin%
\definecolor{currentfill}{rgb}{0.266580,0.228262,0.514349}%
\pgfsetfillcolor{currentfill}%
\pgfsetfillopacity{0.700000}%
\pgfsetlinewidth{0.000000pt}%
\definecolor{currentstroke}{rgb}{0.000000,0.000000,0.000000}%
\pgfsetstrokecolor{currentstroke}%
\pgfsetdash{}{0pt}%
\pgfpathmoveto{\pgfqpoint{4.158239in}{1.675875in}}%
\pgfpathlineto{\pgfqpoint{4.171952in}{1.681213in}}%
\pgfpathlineto{\pgfqpoint{4.185677in}{1.686709in}}%
\pgfpathlineto{\pgfqpoint{4.199414in}{1.692363in}}%
\pgfpathlineto{\pgfqpoint{4.213163in}{1.698174in}}%
\pgfpathlineto{\pgfqpoint{4.221033in}{1.712806in}}%
\pgfpathlineto{\pgfqpoint{4.228898in}{1.727432in}}%
\pgfpathlineto{\pgfqpoint{4.236760in}{1.742047in}}%
\pgfpathlineto{\pgfqpoint{4.244617in}{1.756648in}}%
\pgfpathlineto{\pgfqpoint{4.230867in}{1.750411in}}%
\pgfpathlineto{\pgfqpoint{4.217130in}{1.744332in}}%
\pgfpathlineto{\pgfqpoint{4.203404in}{1.738411in}}%
\pgfpathlineto{\pgfqpoint{4.189692in}{1.732649in}}%
\pgfpathlineto{\pgfqpoint{4.181835in}{1.718463in}}%
\pgfpathlineto{\pgfqpoint{4.173974in}{1.704269in}}%
\pgfpathlineto{\pgfqpoint{4.166108in}{1.690072in}}%
\pgfpathlineto{\pgfqpoint{4.158239in}{1.675875in}}%
\pgfpathclose%
\pgfusepath{fill}%
\end{pgfscope}%
\begin{pgfscope}%
\pgfpathrectangle{\pgfqpoint{1.254980in}{0.150000in}}{\pgfqpoint{5.490039in}{5.490039in}}%
\pgfusepath{clip}%
\pgfsetbuttcap%
\pgfsetroundjoin%
\definecolor{currentfill}{rgb}{0.119423,0.611141,0.538982}%
\pgfsetfillcolor{currentfill}%
\pgfsetfillopacity{0.700000}%
\pgfsetlinewidth{0.000000pt}%
\definecolor{currentstroke}{rgb}{0.000000,0.000000,0.000000}%
\pgfsetstrokecolor{currentstroke}%
\pgfsetdash{}{0pt}%
\pgfpathmoveto{\pgfqpoint{4.951392in}{2.649092in}}%
\pgfpathlineto{\pgfqpoint{4.965566in}{2.662235in}}%
\pgfpathlineto{\pgfqpoint{4.979760in}{2.675543in}}%
\pgfpathlineto{\pgfqpoint{4.993972in}{2.689014in}}%
\pgfpathlineto{\pgfqpoint{5.008203in}{2.702650in}}%
\pgfpathlineto{\pgfqpoint{5.015813in}{2.713708in}}%
\pgfpathlineto{\pgfqpoint{5.023414in}{2.724599in}}%
\pgfpathlineto{\pgfqpoint{5.031008in}{2.735322in}}%
\pgfpathlineto{\pgfqpoint{5.038594in}{2.745876in}}%
\pgfpathlineto{\pgfqpoint{5.024363in}{2.732194in}}%
\pgfpathlineto{\pgfqpoint{5.010151in}{2.718676in}}%
\pgfpathlineto{\pgfqpoint{4.995958in}{2.705321in}}%
\pgfpathlineto{\pgfqpoint{4.981784in}{2.692131in}}%
\pgfpathlineto{\pgfqpoint{4.974197in}{2.681612in}}%
\pgfpathlineto{\pgfqpoint{4.966603in}{2.670932in}}%
\pgfpathlineto{\pgfqpoint{4.959001in}{2.660092in}}%
\pgfpathlineto{\pgfqpoint{4.951392in}{2.649092in}}%
\pgfpathclose%
\pgfusepath{fill}%
\end{pgfscope}%
\begin{pgfscope}%
\pgfpathrectangle{\pgfqpoint{1.254980in}{0.150000in}}{\pgfqpoint{5.490039in}{5.490039in}}%
\pgfusepath{clip}%
\pgfsetbuttcap%
\pgfsetroundjoin%
\definecolor{currentfill}{rgb}{0.268510,0.009605,0.335427}%
\pgfsetfillcolor{currentfill}%
\pgfsetfillopacity{0.700000}%
\pgfsetlinewidth{0.000000pt}%
\definecolor{currentstroke}{rgb}{0.000000,0.000000,0.000000}%
\pgfsetstrokecolor{currentstroke}%
\pgfsetdash{}{0pt}%
\pgfpathmoveto{\pgfqpoint{3.522362in}{1.286105in}}%
\pgfpathlineto{\pgfqpoint{3.535912in}{1.282642in}}%
\pgfpathlineto{\pgfqpoint{3.549467in}{1.279340in}}%
\pgfpathlineto{\pgfqpoint{3.563027in}{1.276200in}}%
\pgfpathlineto{\pgfqpoint{3.576592in}{1.273220in}}%
\pgfpathlineto{\pgfqpoint{3.584694in}{1.280975in}}%
\pgfpathlineto{\pgfqpoint{3.592787in}{1.288936in}}%
\pgfpathlineto{\pgfqpoint{3.600872in}{1.297098in}}%
\pgfpathlineto{\pgfqpoint{3.608949in}{1.305455in}}%
\pgfpathlineto{\pgfqpoint{3.595402in}{1.307795in}}%
\pgfpathlineto{\pgfqpoint{3.581861in}{1.310295in}}%
\pgfpathlineto{\pgfqpoint{3.568326in}{1.312957in}}%
\pgfpathlineto{\pgfqpoint{3.554796in}{1.315780in}}%
\pgfpathlineto{\pgfqpoint{3.546700in}{1.308053in}}%
\pgfpathlineto{\pgfqpoint{3.538596in}{1.300527in}}%
\pgfpathlineto{\pgfqpoint{3.530483in}{1.293209in}}%
\pgfpathlineto{\pgfqpoint{3.522362in}{1.286105in}}%
\pgfpathclose%
\pgfusepath{fill}%
\end{pgfscope}%
\begin{pgfscope}%
\pgfpathrectangle{\pgfqpoint{1.254980in}{0.150000in}}{\pgfqpoint{5.490039in}{5.490039in}}%
\pgfusepath{clip}%
\pgfsetbuttcap%
\pgfsetroundjoin%
\definecolor{currentfill}{rgb}{0.273809,0.031497,0.358853}%
\pgfsetfillcolor{currentfill}%
\pgfsetfillopacity{0.700000}%
\pgfsetlinewidth{0.000000pt}%
\definecolor{currentstroke}{rgb}{0.000000,0.000000,0.000000}%
\pgfsetstrokecolor{currentstroke}%
\pgfsetdash{}{0pt}%
\pgfpathmoveto{\pgfqpoint{3.239929in}{1.337377in}}%
\pgfpathlineto{\pgfqpoint{3.253480in}{1.329900in}}%
\pgfpathlineto{\pgfqpoint{3.267033in}{1.322592in}}%
\pgfpathlineto{\pgfqpoint{3.280588in}{1.315453in}}%
\pgfpathlineto{\pgfqpoint{3.294145in}{1.308483in}}%
\pgfpathlineto{\pgfqpoint{3.302431in}{1.311505in}}%
\pgfpathlineto{\pgfqpoint{3.310705in}{1.314819in}}%
\pgfpathlineto{\pgfqpoint{3.318967in}{1.318416in}}%
\pgfpathlineto{\pgfqpoint{3.327217in}{1.322289in}}%
\pgfpathlineto{\pgfqpoint{3.313690in}{1.328560in}}%
\pgfpathlineto{\pgfqpoint{3.300167in}{1.335000in}}%
\pgfpathlineto{\pgfqpoint{3.286645in}{1.341608in}}%
\pgfpathlineto{\pgfqpoint{3.273126in}{1.348386in}}%
\pgfpathlineto{\pgfqpoint{3.264847in}{1.345201in}}%
\pgfpathlineto{\pgfqpoint{3.256554in}{1.342299in}}%
\pgfpathlineto{\pgfqpoint{3.248248in}{1.339689in}}%
\pgfpathlineto{\pgfqpoint{3.239929in}{1.337377in}}%
\pgfpathclose%
\pgfusepath{fill}%
\end{pgfscope}%
\begin{pgfscope}%
\pgfpathrectangle{\pgfqpoint{1.254980in}{0.150000in}}{\pgfqpoint{5.490039in}{5.490039in}}%
\pgfusepath{clip}%
\pgfsetbuttcap%
\pgfsetroundjoin%
\definecolor{currentfill}{rgb}{0.129933,0.559582,0.551864}%
\pgfsetfillcolor{currentfill}%
\pgfsetfillopacity{0.700000}%
\pgfsetlinewidth{0.000000pt}%
\definecolor{currentstroke}{rgb}{0.000000,0.000000,0.000000}%
\pgfsetstrokecolor{currentstroke}%
\pgfsetdash{}{0pt}%
\pgfpathmoveto{\pgfqpoint{4.833746in}{2.505155in}}%
\pgfpathlineto{\pgfqpoint{4.847846in}{2.517460in}}%
\pgfpathlineto{\pgfqpoint{4.861963in}{2.529928in}}%
\pgfpathlineto{\pgfqpoint{4.876099in}{2.542559in}}%
\pgfpathlineto{\pgfqpoint{4.890253in}{2.555353in}}%
\pgfpathlineto{\pgfqpoint{4.897920in}{2.567623in}}%
\pgfpathlineto{\pgfqpoint{4.905580in}{2.579737in}}%
\pgfpathlineto{\pgfqpoint{4.913234in}{2.591693in}}%
\pgfpathlineto{\pgfqpoint{4.920880in}{2.603491in}}%
\pgfpathlineto{\pgfqpoint{4.906724in}{2.590588in}}%
\pgfpathlineto{\pgfqpoint{4.892587in}{2.577849in}}%
\pgfpathlineto{\pgfqpoint{4.878468in}{2.565273in}}%
\pgfpathlineto{\pgfqpoint{4.864367in}{2.552859in}}%
\pgfpathlineto{\pgfqpoint{4.856722in}{2.541158in}}%
\pgfpathlineto{\pgfqpoint{4.849070in}{2.529306in}}%
\pgfpathlineto{\pgfqpoint{4.841412in}{2.517305in}}%
\pgfpathlineto{\pgfqpoint{4.833746in}{2.505155in}}%
\pgfpathclose%
\pgfusepath{fill}%
\end{pgfscope}%
\begin{pgfscope}%
\pgfpathrectangle{\pgfqpoint{1.254980in}{0.150000in}}{\pgfqpoint{5.490039in}{5.490039in}}%
\pgfusepath{clip}%
\pgfsetbuttcap%
\pgfsetroundjoin%
\definecolor{currentfill}{rgb}{0.150476,0.504369,0.557430}%
\pgfsetfillcolor{currentfill}%
\pgfsetfillopacity{0.700000}%
\pgfsetlinewidth{0.000000pt}%
\definecolor{currentstroke}{rgb}{0.000000,0.000000,0.000000}%
\pgfsetstrokecolor{currentstroke}%
\pgfsetdash{}{0pt}%
\pgfpathmoveto{\pgfqpoint{4.715986in}{2.356329in}}%
\pgfpathlineto{\pgfqpoint{4.730011in}{2.367678in}}%
\pgfpathlineto{\pgfqpoint{4.744054in}{2.379188in}}%
\pgfpathlineto{\pgfqpoint{4.758113in}{2.390860in}}%
\pgfpathlineto{\pgfqpoint{4.772190in}{2.402695in}}%
\pgfpathlineto{\pgfqpoint{4.779906in}{2.416003in}}%
\pgfpathlineto{\pgfqpoint{4.787617in}{2.429171in}}%
\pgfpathlineto{\pgfqpoint{4.795321in}{2.442196in}}%
\pgfpathlineto{\pgfqpoint{4.803019in}{2.455079in}}%
\pgfpathlineto{\pgfqpoint{4.788939in}{2.443075in}}%
\pgfpathlineto{\pgfqpoint{4.774877in}{2.431234in}}%
\pgfpathlineto{\pgfqpoint{4.760832in}{2.419554in}}%
\pgfpathlineto{\pgfqpoint{4.746804in}{2.408037in}}%
\pgfpathlineto{\pgfqpoint{4.739108in}{2.395312in}}%
\pgfpathlineto{\pgfqpoint{4.731407in}{2.382452in}}%
\pgfpathlineto{\pgfqpoint{4.723699in}{2.369457in}}%
\pgfpathlineto{\pgfqpoint{4.715986in}{2.356329in}}%
\pgfpathclose%
\pgfusepath{fill}%
\end{pgfscope}%
\begin{pgfscope}%
\pgfpathrectangle{\pgfqpoint{1.254980in}{0.150000in}}{\pgfqpoint{5.490039in}{5.490039in}}%
\pgfusepath{clip}%
\pgfsetbuttcap%
\pgfsetroundjoin%
\definecolor{currentfill}{rgb}{0.280255,0.165693,0.476498}%
\pgfsetfillcolor{currentfill}%
\pgfsetfillopacity{0.700000}%
\pgfsetlinewidth{0.000000pt}%
\definecolor{currentstroke}{rgb}{0.000000,0.000000,0.000000}%
\pgfsetstrokecolor{currentstroke}%
\pgfsetdash{}{0pt}%
\pgfpathmoveto{\pgfqpoint{4.040387in}{1.546752in}}%
\pgfpathlineto{\pgfqpoint{4.054058in}{1.550529in}}%
\pgfpathlineto{\pgfqpoint{4.067740in}{1.554463in}}%
\pgfpathlineto{\pgfqpoint{4.081433in}{1.558555in}}%
\pgfpathlineto{\pgfqpoint{4.095136in}{1.562803in}}%
\pgfpathlineto{\pgfqpoint{4.103039in}{1.576848in}}%
\pgfpathlineto{\pgfqpoint{4.110937in}{1.590927in}}%
\pgfpathlineto{\pgfqpoint{4.118831in}{1.605036in}}%
\pgfpathlineto{\pgfqpoint{4.126721in}{1.619170in}}%
\pgfpathlineto{\pgfqpoint{4.113019in}{1.614442in}}%
\pgfpathlineto{\pgfqpoint{4.099328in}{1.609871in}}%
\pgfpathlineto{\pgfqpoint{4.085648in}{1.605457in}}%
\pgfpathlineto{\pgfqpoint{4.071980in}{1.601201in}}%
\pgfpathlineto{\pgfqpoint{4.064088in}{1.587536in}}%
\pgfpathlineto{\pgfqpoint{4.056192in}{1.573903in}}%
\pgfpathlineto{\pgfqpoint{4.048292in}{1.560307in}}%
\pgfpathlineto{\pgfqpoint{4.040387in}{1.546752in}}%
\pgfpathclose%
\pgfusepath{fill}%
\end{pgfscope}%
\begin{pgfscope}%
\pgfpathrectangle{\pgfqpoint{1.254980in}{0.150000in}}{\pgfqpoint{5.490039in}{5.490039in}}%
\pgfusepath{clip}%
\pgfsetbuttcap%
\pgfsetroundjoin%
\definecolor{currentfill}{rgb}{0.171176,0.452530,0.557965}%
\pgfsetfillcolor{currentfill}%
\pgfsetfillopacity{0.700000}%
\pgfsetlinewidth{0.000000pt}%
\definecolor{currentstroke}{rgb}{0.000000,0.000000,0.000000}%
\pgfsetstrokecolor{currentstroke}%
\pgfsetdash{}{0pt}%
\pgfpathmoveto{\pgfqpoint{4.598165in}{2.204631in}}%
\pgfpathlineto{\pgfqpoint{4.612118in}{2.214907in}}%
\pgfpathlineto{\pgfqpoint{4.626087in}{2.225345in}}%
\pgfpathlineto{\pgfqpoint{4.640072in}{2.235943in}}%
\pgfpathlineto{\pgfqpoint{4.654073in}{2.246702in}}%
\pgfpathlineto{\pgfqpoint{4.661832in}{2.260837in}}%
\pgfpathlineto{\pgfqpoint{4.669585in}{2.274853in}}%
\pgfpathlineto{\pgfqpoint{4.677332in}{2.288746in}}%
\pgfpathlineto{\pgfqpoint{4.685074in}{2.302517in}}%
\pgfpathlineto{\pgfqpoint{4.671069in}{2.291529in}}%
\pgfpathlineto{\pgfqpoint{4.657080in}{2.280702in}}%
\pgfpathlineto{\pgfqpoint{4.643108in}{2.270036in}}%
\pgfpathlineto{\pgfqpoint{4.629152in}{2.259532in}}%
\pgfpathlineto{\pgfqpoint{4.621413in}{2.245979in}}%
\pgfpathlineto{\pgfqpoint{4.613669in}{2.232310in}}%
\pgfpathlineto{\pgfqpoint{4.605920in}{2.218526in}}%
\pgfpathlineto{\pgfqpoint{4.598165in}{2.204631in}}%
\pgfpathclose%
\pgfusepath{fill}%
\end{pgfscope}%
\begin{pgfscope}%
\pgfpathrectangle{\pgfqpoint{1.254980in}{0.150000in}}{\pgfqpoint{5.490039in}{5.490039in}}%
\pgfusepath{clip}%
\pgfsetbuttcap%
\pgfsetroundjoin%
\definecolor{currentfill}{rgb}{0.197636,0.391528,0.554969}%
\pgfsetfillcolor{currentfill}%
\pgfsetfillopacity{0.700000}%
\pgfsetlinewidth{0.000000pt}%
\definecolor{currentstroke}{rgb}{0.000000,0.000000,0.000000}%
\pgfsetstrokecolor{currentstroke}%
\pgfsetdash{}{0pt}%
\pgfpathmoveto{\pgfqpoint{4.480321in}{2.052357in}}%
\pgfpathlineto{\pgfqpoint{4.494205in}{2.061449in}}%
\pgfpathlineto{\pgfqpoint{4.508104in}{2.070701in}}%
\pgfpathlineto{\pgfqpoint{4.522018in}{2.080113in}}%
\pgfpathlineto{\pgfqpoint{4.535948in}{2.089684in}}%
\pgfpathlineto{\pgfqpoint{4.543742in}{2.104399in}}%
\pgfpathlineto{\pgfqpoint{4.551531in}{2.119019in}}%
\pgfpathlineto{\pgfqpoint{4.559316in}{2.133543in}}%
\pgfpathlineto{\pgfqpoint{4.567096in}{2.147968in}}%
\pgfpathlineto{\pgfqpoint{4.553162in}{2.138110in}}%
\pgfpathlineto{\pgfqpoint{4.539245in}{2.128411in}}%
\pgfpathlineto{\pgfqpoint{4.525342in}{2.118874in}}%
\pgfpathlineto{\pgfqpoint{4.511455in}{2.109496in}}%
\pgfpathlineto{\pgfqpoint{4.503678in}{2.095346in}}%
\pgfpathlineto{\pgfqpoint{4.495897in}{2.081105in}}%
\pgfpathlineto{\pgfqpoint{4.488112in}{2.066774in}}%
\pgfpathlineto{\pgfqpoint{4.480321in}{2.052357in}}%
\pgfpathclose%
\pgfusepath{fill}%
\end{pgfscope}%
\begin{pgfscope}%
\pgfpathrectangle{\pgfqpoint{1.254980in}{0.150000in}}{\pgfqpoint{5.490039in}{5.490039in}}%
\pgfusepath{clip}%
\pgfsetbuttcap%
\pgfsetroundjoin%
\definecolor{currentfill}{rgb}{0.170948,0.694384,0.493803}%
\pgfsetfillcolor{currentfill}%
\pgfsetfillopacity{0.700000}%
\pgfsetlinewidth{0.000000pt}%
\definecolor{currentstroke}{rgb}{0.000000,0.000000,0.000000}%
\pgfsetstrokecolor{currentstroke}%
\pgfsetdash{}{0pt}%
\pgfpathmoveto{\pgfqpoint{5.156087in}{2.880651in}}%
\pgfpathlineto{\pgfqpoint{5.170412in}{2.895156in}}%
\pgfpathlineto{\pgfqpoint{5.184758in}{2.909825in}}%
\pgfpathlineto{\pgfqpoint{5.199123in}{2.924659in}}%
\pgfpathlineto{\pgfqpoint{5.213509in}{2.939659in}}%
\pgfpathlineto{\pgfqpoint{5.221013in}{2.948641in}}%
\pgfpathlineto{\pgfqpoint{5.228508in}{2.957443in}}%
\pgfpathlineto{\pgfqpoint{5.235994in}{2.966066in}}%
\pgfpathlineto{\pgfqpoint{5.243470in}{2.974512in}}%
\pgfpathlineto{\pgfqpoint{5.229088in}{2.959560in}}%
\pgfpathlineto{\pgfqpoint{5.214727in}{2.944773in}}%
\pgfpathlineto{\pgfqpoint{5.200386in}{2.930151in}}%
\pgfpathlineto{\pgfqpoint{5.186065in}{2.915694in}}%
\pgfpathlineto{\pgfqpoint{5.178584in}{2.907189in}}%
\pgfpathlineto{\pgfqpoint{5.171094in}{2.898515in}}%
\pgfpathlineto{\pgfqpoint{5.163595in}{2.889669in}}%
\pgfpathlineto{\pgfqpoint{5.156087in}{2.880651in}}%
\pgfpathclose%
\pgfusepath{fill}%
\end{pgfscope}%
\begin{pgfscope}%
\pgfpathrectangle{\pgfqpoint{1.254980in}{0.150000in}}{\pgfqpoint{5.490039in}{5.490039in}}%
\pgfusepath{clip}%
\pgfsetbuttcap%
\pgfsetroundjoin%
\definecolor{currentfill}{rgb}{0.223925,0.334994,0.548053}%
\pgfsetfillcolor{currentfill}%
\pgfsetfillopacity{0.700000}%
\pgfsetlinewidth{0.000000pt}%
\definecolor{currentstroke}{rgb}{0.000000,0.000000,0.000000}%
\pgfsetstrokecolor{currentstroke}%
\pgfsetdash{}{0pt}%
\pgfpathmoveto{\pgfqpoint{4.362474in}{1.902081in}}%
\pgfpathlineto{\pgfqpoint{4.376294in}{1.909879in}}%
\pgfpathlineto{\pgfqpoint{4.390127in}{1.917835in}}%
\pgfpathlineto{\pgfqpoint{4.403975in}{1.925950in}}%
\pgfpathlineto{\pgfqpoint{4.417838in}{1.934224in}}%
\pgfpathlineto{\pgfqpoint{4.425663in}{1.949235in}}%
\pgfpathlineto{\pgfqpoint{4.433485in}{1.964183in}}%
\pgfpathlineto{\pgfqpoint{4.441302in}{1.979064in}}%
\pgfpathlineto{\pgfqpoint{4.449115in}{1.993874in}}%
\pgfpathlineto{\pgfqpoint{4.435249in}{1.985256in}}%
\pgfpathlineto{\pgfqpoint{4.421398in}{1.976798in}}%
\pgfpathlineto{\pgfqpoint{4.407561in}{1.968499in}}%
\pgfpathlineto{\pgfqpoint{4.393738in}{1.960359in}}%
\pgfpathlineto{\pgfqpoint{4.385928in}{1.945881in}}%
\pgfpathlineto{\pgfqpoint{4.378114in}{1.931340in}}%
\pgfpathlineto{\pgfqpoint{4.370296in}{1.916739in}}%
\pgfpathlineto{\pgfqpoint{4.362474in}{1.902081in}}%
\pgfpathclose%
\pgfusepath{fill}%
\end{pgfscope}%
\begin{pgfscope}%
\pgfpathrectangle{\pgfqpoint{1.254980in}{0.150000in}}{\pgfqpoint{5.490039in}{5.490039in}}%
\pgfusepath{clip}%
\pgfsetbuttcap%
\pgfsetroundjoin%
\definecolor{currentfill}{rgb}{0.468053,0.818921,0.323998}%
\pgfsetfillcolor{currentfill}%
\pgfsetfillopacity{0.700000}%
\pgfsetlinewidth{0.000000pt}%
\definecolor{currentstroke}{rgb}{0.000000,0.000000,0.000000}%
\pgfsetstrokecolor{currentstroke}%
\pgfsetdash{}{0pt}%
\pgfpathmoveto{\pgfqpoint{5.564677in}{3.291979in}}%
\pgfpathlineto{\pgfqpoint{5.579293in}{3.308368in}}%
\pgfpathlineto{\pgfqpoint{5.593932in}{3.324923in}}%
\pgfpathlineto{\pgfqpoint{5.608593in}{3.341646in}}%
\pgfpathlineto{\pgfqpoint{5.623277in}{3.358535in}}%
\pgfpathlineto{\pgfqpoint{5.630496in}{3.362695in}}%
\pgfpathlineto{\pgfqpoint{5.637703in}{3.366687in}}%
\pgfpathlineto{\pgfqpoint{5.644899in}{3.370513in}}%
\pgfpathlineto{\pgfqpoint{5.652084in}{3.374178in}}%
\pgfpathlineto{\pgfqpoint{5.637415in}{3.357531in}}%
\pgfpathlineto{\pgfqpoint{5.622769in}{3.341051in}}%
\pgfpathlineto{\pgfqpoint{5.608145in}{3.324738in}}%
\pgfpathlineto{\pgfqpoint{5.593543in}{3.308591in}}%
\pgfpathlineto{\pgfqpoint{5.586343in}{3.304672in}}%
\pgfpathlineto{\pgfqpoint{5.579132in}{3.300600in}}%
\pgfpathlineto{\pgfqpoint{5.571910in}{3.296370in}}%
\pgfpathlineto{\pgfqpoint{5.564677in}{3.291979in}}%
\pgfpathclose%
\pgfusepath{fill}%
\end{pgfscope}%
\begin{pgfscope}%
\pgfpathrectangle{\pgfqpoint{1.254980in}{0.150000in}}{\pgfqpoint{5.490039in}{5.490039in}}%
\pgfusepath{clip}%
\pgfsetbuttcap%
\pgfsetroundjoin%
\definecolor{currentfill}{rgb}{0.130067,0.651384,0.521608}%
\pgfsetfillcolor{currentfill}%
\pgfsetfillopacity{0.700000}%
\pgfsetlinewidth{0.000000pt}%
\definecolor{currentstroke}{rgb}{0.000000,0.000000,0.000000}%
\pgfsetstrokecolor{currentstroke}%
\pgfsetdash{}{0pt}%
\pgfpathmoveto{\pgfqpoint{1.976664in}{2.887660in}}%
\pgfpathlineto{\pgfqpoint{1.990881in}{2.859445in}}%
\pgfpathlineto{\pgfqpoint{2.005079in}{2.831549in}}%
\pgfpathlineto{\pgfqpoint{2.019258in}{2.803969in}}%
\pgfpathlineto{\pgfqpoint{2.033418in}{2.776701in}}%
\pgfpathlineto{\pgfqpoint{2.042977in}{2.762560in}}%
\pgfpathlineto{\pgfqpoint{2.052502in}{2.748911in}}%
\pgfpathlineto{\pgfqpoint{2.061993in}{2.735744in}}%
\pgfpathlineto{\pgfqpoint{2.071452in}{2.723050in}}%
\pgfpathlineto{\pgfqpoint{2.057373in}{2.749517in}}%
\pgfpathlineto{\pgfqpoint{2.043276in}{2.776294in}}%
\pgfpathlineto{\pgfqpoint{2.029160in}{2.803384in}}%
\pgfpathlineto{\pgfqpoint{2.015027in}{2.830791in}}%
\pgfpathlineto{\pgfqpoint{2.005488in}{2.844273in}}%
\pgfpathlineto{\pgfqpoint{1.995915in}{2.858240in}}%
\pgfpathlineto{\pgfqpoint{1.986307in}{2.872699in}}%
\pgfpathlineto{\pgfqpoint{1.976664in}{2.887660in}}%
\pgfpathclose%
\pgfusepath{fill}%
\end{pgfscope}%
\begin{pgfscope}%
\pgfpathrectangle{\pgfqpoint{1.254980in}{0.150000in}}{\pgfqpoint{5.490039in}{5.490039in}}%
\pgfusepath{clip}%
\pgfsetbuttcap%
\pgfsetroundjoin%
\definecolor{currentfill}{rgb}{0.252194,0.269783,0.531579}%
\pgfsetfillcolor{currentfill}%
\pgfsetfillopacity{0.700000}%
\pgfsetlinewidth{0.000000pt}%
\definecolor{currentstroke}{rgb}{0.000000,0.000000,0.000000}%
\pgfsetstrokecolor{currentstroke}%
\pgfsetdash{}{0pt}%
\pgfpathmoveto{\pgfqpoint{4.244617in}{1.756648in}}%
\pgfpathlineto{\pgfqpoint{4.258380in}{1.763042in}}%
\pgfpathlineto{\pgfqpoint{4.272155in}{1.769595in}}%
\pgfpathlineto{\pgfqpoint{4.285943in}{1.776306in}}%
\pgfpathlineto{\pgfqpoint{4.299745in}{1.783174in}}%
\pgfpathlineto{\pgfqpoint{4.307600in}{1.798165in}}%
\pgfpathlineto{\pgfqpoint{4.315451in}{1.813127in}}%
\pgfpathlineto{\pgfqpoint{4.323299in}{1.828056in}}%
\pgfpathlineto{\pgfqpoint{4.331142in}{1.842948in}}%
\pgfpathlineto{\pgfqpoint{4.317338in}{1.835680in}}%
\pgfpathlineto{\pgfqpoint{4.303548in}{1.828571in}}%
\pgfpathlineto{\pgfqpoint{4.289771in}{1.821620in}}%
\pgfpathlineto{\pgfqpoint{4.276006in}{1.814827in}}%
\pgfpathlineto{\pgfqpoint{4.268165in}{1.800323in}}%
\pgfpathlineto{\pgfqpoint{4.260320in}{1.785789in}}%
\pgfpathlineto{\pgfqpoint{4.252470in}{1.771229in}}%
\pgfpathlineto{\pgfqpoint{4.244617in}{1.756648in}}%
\pgfpathclose%
\pgfusepath{fill}%
\end{pgfscope}%
\begin{pgfscope}%
\pgfpathrectangle{\pgfqpoint{1.254980in}{0.150000in}}{\pgfqpoint{5.490039in}{5.490039in}}%
\pgfusepath{clip}%
\pgfsetbuttcap%
\pgfsetroundjoin%
\definecolor{currentfill}{rgb}{0.311925,0.767822,0.415586}%
\pgfsetfillcolor{currentfill}%
\pgfsetfillopacity{0.700000}%
\pgfsetlinewidth{0.000000pt}%
\definecolor{currentstroke}{rgb}{0.000000,0.000000,0.000000}%
\pgfsetstrokecolor{currentstroke}%
\pgfsetdash{}{0pt}%
\pgfpathmoveto{\pgfqpoint{5.360622in}{3.096511in}}%
\pgfpathlineto{\pgfqpoint{5.375096in}{3.112100in}}%
\pgfpathlineto{\pgfqpoint{5.389591in}{3.127855in}}%
\pgfpathlineto{\pgfqpoint{5.404108in}{3.143776in}}%
\pgfpathlineto{\pgfqpoint{5.418647in}{3.159864in}}%
\pgfpathlineto{\pgfqpoint{5.426020in}{3.166501in}}%
\pgfpathlineto{\pgfqpoint{5.433384in}{3.172958in}}%
\pgfpathlineto{\pgfqpoint{5.440736in}{3.179237in}}%
\pgfpathlineto{\pgfqpoint{5.448078in}{3.185340in}}%
\pgfpathlineto{\pgfqpoint{5.433549in}{3.169397in}}%
\pgfpathlineto{\pgfqpoint{5.419041in}{3.153620in}}%
\pgfpathlineto{\pgfqpoint{5.404555in}{3.138010in}}%
\pgfpathlineto{\pgfqpoint{5.390090in}{3.122565in}}%
\pgfpathlineto{\pgfqpoint{5.382738in}{3.116305in}}%
\pgfpathlineto{\pgfqpoint{5.375376in}{3.109878in}}%
\pgfpathlineto{\pgfqpoint{5.368004in}{3.103281in}}%
\pgfpathlineto{\pgfqpoint{5.360622in}{3.096511in}}%
\pgfpathclose%
\pgfusepath{fill}%
\end{pgfscope}%
\begin{pgfscope}%
\pgfpathrectangle{\pgfqpoint{1.254980in}{0.150000in}}{\pgfqpoint{5.490039in}{5.490039in}}%
\pgfusepath{clip}%
\pgfsetbuttcap%
\pgfsetroundjoin%
\definecolor{currentfill}{rgb}{0.277018,0.050344,0.375715}%
\pgfsetfillcolor{currentfill}%
\pgfsetfillopacity{0.700000}%
\pgfsetlinewidth{0.000000pt}%
\definecolor{currentstroke}{rgb}{0.000000,0.000000,0.000000}%
\pgfsetstrokecolor{currentstroke}%
\pgfsetdash{}{0pt}%
\pgfpathmoveto{\pgfqpoint{3.749655in}{1.332629in}}%
\pgfpathlineto{\pgfqpoint{3.763247in}{1.332335in}}%
\pgfpathlineto{\pgfqpoint{3.776846in}{1.332198in}}%
\pgfpathlineto{\pgfqpoint{3.790452in}{1.332219in}}%
\pgfpathlineto{\pgfqpoint{3.804067in}{1.332397in}}%
\pgfpathlineto{\pgfqpoint{3.812066in}{1.343420in}}%
\pgfpathlineto{\pgfqpoint{3.820060in}{1.354581in}}%
\pgfpathlineto{\pgfqpoint{3.828047in}{1.365873in}}%
\pgfpathlineto{\pgfqpoint{3.836029in}{1.377292in}}%
\pgfpathlineto{\pgfqpoint{3.822424in}{1.376527in}}%
\pgfpathlineto{\pgfqpoint{3.808828in}{1.375920in}}%
\pgfpathlineto{\pgfqpoint{3.795241in}{1.375471in}}%
\pgfpathlineto{\pgfqpoint{3.781661in}{1.375180in}}%
\pgfpathlineto{\pgfqpoint{3.773669in}{1.364337in}}%
\pgfpathlineto{\pgfqpoint{3.765670in}{1.353627in}}%
\pgfpathlineto{\pgfqpoint{3.757666in}{1.343056in}}%
\pgfpathlineto{\pgfqpoint{3.749655in}{1.332629in}}%
\pgfpathclose%
\pgfusepath{fill}%
\end{pgfscope}%
\begin{pgfscope}%
\pgfpathrectangle{\pgfqpoint{1.254980in}{0.150000in}}{\pgfqpoint{5.490039in}{5.490039in}}%
\pgfusepath{clip}%
\pgfsetbuttcap%
\pgfsetroundjoin%
\definecolor{currentfill}{rgb}{0.268510,0.009605,0.335427}%
\pgfsetfillcolor{currentfill}%
\pgfsetfillopacity{0.700000}%
\pgfsetlinewidth{0.000000pt}%
\definecolor{currentstroke}{rgb}{0.000000,0.000000,0.000000}%
\pgfsetstrokecolor{currentstroke}%
\pgfsetdash{}{0pt}%
\pgfpathmoveto{\pgfqpoint{3.435534in}{1.278110in}}%
\pgfpathlineto{\pgfqpoint{3.449090in}{1.273328in}}%
\pgfpathlineto{\pgfqpoint{3.462649in}{1.268710in}}%
\pgfpathlineto{\pgfqpoint{3.476213in}{1.264255in}}%
\pgfpathlineto{\pgfqpoint{3.489780in}{1.259962in}}%
\pgfpathlineto{\pgfqpoint{3.497940in}{1.266143in}}%
\pgfpathlineto{\pgfqpoint{3.506090in}{1.272566in}}%
\pgfpathlineto{\pgfqpoint{3.514231in}{1.279221in}}%
\pgfpathlineto{\pgfqpoint{3.522362in}{1.286105in}}%
\pgfpathlineto{\pgfqpoint{3.508816in}{1.289729in}}%
\pgfpathlineto{\pgfqpoint{3.495276in}{1.293516in}}%
\pgfpathlineto{\pgfqpoint{3.481740in}{1.297466in}}%
\pgfpathlineto{\pgfqpoint{3.468208in}{1.301579in}}%
\pgfpathlineto{\pgfqpoint{3.460055in}{1.295353in}}%
\pgfpathlineto{\pgfqpoint{3.451892in}{1.289362in}}%
\pgfpathlineto{\pgfqpoint{3.443718in}{1.283612in}}%
\pgfpathlineto{\pgfqpoint{3.435534in}{1.278110in}}%
\pgfpathclose%
\pgfusepath{fill}%
\end{pgfscope}%
\begin{pgfscope}%
\pgfpathrectangle{\pgfqpoint{1.254980in}{0.150000in}}{\pgfqpoint{5.490039in}{5.490039in}}%
\pgfusepath{clip}%
\pgfsetbuttcap%
\pgfsetroundjoin%
\definecolor{currentfill}{rgb}{0.280894,0.078907,0.402329}%
\pgfsetfillcolor{currentfill}%
\pgfsetfillopacity{0.700000}%
\pgfsetlinewidth{0.000000pt}%
\definecolor{currentstroke}{rgb}{0.000000,0.000000,0.000000}%
\pgfsetstrokecolor{currentstroke}%
\pgfsetdash{}{0pt}%
\pgfpathmoveto{\pgfqpoint{3.836029in}{1.377292in}}%
\pgfpathlineto{\pgfqpoint{3.849641in}{1.378214in}}%
\pgfpathlineto{\pgfqpoint{3.863262in}{1.379293in}}%
\pgfpathlineto{\pgfqpoint{3.876892in}{1.380529in}}%
\pgfpathlineto{\pgfqpoint{3.890530in}{1.381922in}}%
\pgfpathlineto{\pgfqpoint{3.898498in}{1.394031in}}%
\pgfpathlineto{\pgfqpoint{3.906460in}{1.406248in}}%
\pgfpathlineto{\pgfqpoint{3.914417in}{1.418568in}}%
\pgfpathlineto{\pgfqpoint{3.922369in}{1.430985in}}%
\pgfpathlineto{\pgfqpoint{3.908738in}{1.429032in}}%
\pgfpathlineto{\pgfqpoint{3.895116in}{1.427236in}}%
\pgfpathlineto{\pgfqpoint{3.881503in}{1.425598in}}%
\pgfpathlineto{\pgfqpoint{3.867899in}{1.424117in}}%
\pgfpathlineto{\pgfqpoint{3.859940in}{1.412249in}}%
\pgfpathlineto{\pgfqpoint{3.851975in}{1.400485in}}%
\pgfpathlineto{\pgfqpoint{3.844004in}{1.388831in}}%
\pgfpathlineto{\pgfqpoint{3.836029in}{1.377292in}}%
\pgfpathclose%
\pgfusepath{fill}%
\end{pgfscope}%
\begin{pgfscope}%
\pgfpathrectangle{\pgfqpoint{1.254980in}{0.150000in}}{\pgfqpoint{5.490039in}{5.490039in}}%
\pgfusepath{clip}%
\pgfsetbuttcap%
\pgfsetroundjoin%
\definecolor{currentfill}{rgb}{0.545524,0.838039,0.275626}%
\pgfsetfillcolor{currentfill}%
\pgfsetfillopacity{0.700000}%
\pgfsetlinewidth{0.000000pt}%
\definecolor{currentstroke}{rgb}{0.000000,0.000000,0.000000}%
\pgfsetstrokecolor{currentstroke}%
\pgfsetdash{}{0pt}%
\pgfpathmoveto{\pgfqpoint{5.652084in}{3.374178in}}%
\pgfpathlineto{\pgfqpoint{5.666776in}{3.390991in}}%
\pgfpathlineto{\pgfqpoint{5.681491in}{3.407972in}}%
\pgfpathlineto{\pgfqpoint{5.696229in}{3.425120in}}%
\pgfpathlineto{\pgfqpoint{5.703390in}{3.428427in}}%
\pgfpathlineto{\pgfqpoint{5.710539in}{3.431572in}}%
\pgfpathlineto{\pgfqpoint{5.717676in}{3.434557in}}%
\pgfpathlineto{\pgfqpoint{5.724802in}{3.437387in}}%
\pgfpathlineto{\pgfqpoint{5.710081in}{3.420516in}}%
\pgfpathlineto{\pgfqpoint{5.695384in}{3.403811in}}%
\pgfpathlineto{\pgfqpoint{5.680709in}{3.387272in}}%
\pgfpathlineto{\pgfqpoint{5.673569in}{3.384227in}}%
\pgfpathlineto{\pgfqpoint{5.666419in}{3.381031in}}%
\pgfpathlineto{\pgfqpoint{5.659257in}{3.377683in}}%
\pgfpathlineto{\pgfqpoint{5.652084in}{3.374178in}}%
\pgfpathclose%
\pgfusepath{fill}%
\end{pgfscope}%
\begin{pgfscope}%
\pgfpathrectangle{\pgfqpoint{1.254980in}{0.150000in}}{\pgfqpoint{5.490039in}{5.490039in}}%
\pgfusepath{clip}%
\pgfsetbuttcap%
\pgfsetroundjoin%
\definecolor{currentfill}{rgb}{0.273809,0.031497,0.358853}%
\pgfsetfillcolor{currentfill}%
\pgfsetfillopacity{0.700000}%
\pgfsetlinewidth{0.000000pt}%
\definecolor{currentstroke}{rgb}{0.000000,0.000000,0.000000}%
\pgfsetstrokecolor{currentstroke}%
\pgfsetdash{}{0pt}%
\pgfpathmoveto{\pgfqpoint{3.663196in}{1.297696in}}%
\pgfpathlineto{\pgfqpoint{3.676773in}{1.296155in}}%
\pgfpathlineto{\pgfqpoint{3.690358in}{1.294772in}}%
\pgfpathlineto{\pgfqpoint{3.703948in}{1.293548in}}%
\pgfpathlineto{\pgfqpoint{3.717546in}{1.292482in}}%
\pgfpathlineto{\pgfqpoint{3.725583in}{1.302273in}}%
\pgfpathlineto{\pgfqpoint{3.733614in}{1.312232in}}%
\pgfpathlineto{\pgfqpoint{3.741638in}{1.322352in}}%
\pgfpathlineto{\pgfqpoint{3.749655in}{1.332629in}}%
\pgfpathlineto{\pgfqpoint{3.736071in}{1.333082in}}%
\pgfpathlineto{\pgfqpoint{3.722494in}{1.333693in}}%
\pgfpathlineto{\pgfqpoint{3.708925in}{1.334463in}}%
\pgfpathlineto{\pgfqpoint{3.695362in}{1.335392in}}%
\pgfpathlineto{\pgfqpoint{3.687331in}{1.325717in}}%
\pgfpathlineto{\pgfqpoint{3.679293in}{1.316206in}}%
\pgfpathlineto{\pgfqpoint{3.671248in}{1.306863in}}%
\pgfpathlineto{\pgfqpoint{3.663196in}{1.297696in}}%
\pgfpathclose%
\pgfusepath{fill}%
\end{pgfscope}%
\begin{pgfscope}%
\pgfpathrectangle{\pgfqpoint{1.254980in}{0.150000in}}{\pgfqpoint{5.490039in}{5.490039in}}%
\pgfusepath{clip}%
\pgfsetbuttcap%
\pgfsetroundjoin%
\definecolor{currentfill}{rgb}{0.272594,0.025563,0.353093}%
\pgfsetfillcolor{currentfill}%
\pgfsetfillopacity{0.700000}%
\pgfsetlinewidth{0.000000pt}%
\definecolor{currentstroke}{rgb}{0.000000,0.000000,0.000000}%
\pgfsetstrokecolor{currentstroke}%
\pgfsetdash{}{0pt}%
\pgfpathmoveto{\pgfqpoint{3.294145in}{1.308483in}}%
\pgfpathlineto{\pgfqpoint{3.307704in}{1.301680in}}%
\pgfpathlineto{\pgfqpoint{3.321265in}{1.295045in}}%
\pgfpathlineto{\pgfqpoint{3.334829in}{1.288577in}}%
\pgfpathlineto{\pgfqpoint{3.348395in}{1.282275in}}%
\pgfpathlineto{\pgfqpoint{3.356651in}{1.286007in}}%
\pgfpathlineto{\pgfqpoint{3.364896in}{1.290022in}}%
\pgfpathlineto{\pgfqpoint{3.373128in}{1.294314in}}%
\pgfpathlineto{\pgfqpoint{3.381349in}{1.298875in}}%
\pgfpathlineto{\pgfqpoint{3.367812in}{1.304479in}}%
\pgfpathlineto{\pgfqpoint{3.354277in}{1.310249in}}%
\pgfpathlineto{\pgfqpoint{3.340745in}{1.316185in}}%
\pgfpathlineto{\pgfqpoint{3.327217in}{1.322289in}}%
\pgfpathlineto{\pgfqpoint{3.318967in}{1.318416in}}%
\pgfpathlineto{\pgfqpoint{3.310705in}{1.314819in}}%
\pgfpathlineto{\pgfqpoint{3.302431in}{1.311505in}}%
\pgfpathlineto{\pgfqpoint{3.294145in}{1.308483in}}%
\pgfpathclose%
\pgfusepath{fill}%
\end{pgfscope}%
\begin{pgfscope}%
\pgfpathrectangle{\pgfqpoint{1.254980in}{0.150000in}}{\pgfqpoint{5.490039in}{5.490039in}}%
\pgfusepath{clip}%
\pgfsetbuttcap%
\pgfsetroundjoin%
\definecolor{currentfill}{rgb}{0.130067,0.651384,0.521608}%
\pgfsetfillcolor{currentfill}%
\pgfsetfillopacity{0.700000}%
\pgfsetlinewidth{0.000000pt}%
\definecolor{currentstroke}{rgb}{0.000000,0.000000,0.000000}%
\pgfsetstrokecolor{currentstroke}%
\pgfsetdash{}{0pt}%
\pgfpathmoveto{\pgfqpoint{5.038594in}{2.745876in}}%
\pgfpathlineto{\pgfqpoint{5.052844in}{2.759722in}}%
\pgfpathlineto{\pgfqpoint{5.067114in}{2.773733in}}%
\pgfpathlineto{\pgfqpoint{5.081403in}{2.787908in}}%
\pgfpathlineto{\pgfqpoint{5.095712in}{2.802248in}}%
\pgfpathlineto{\pgfqpoint{5.103289in}{2.812662in}}%
\pgfpathlineto{\pgfqpoint{5.110857in}{2.822899in}}%
\pgfpathlineto{\pgfqpoint{5.118417in}{2.832962in}}%
\pgfpathlineto{\pgfqpoint{5.125969in}{2.842848in}}%
\pgfpathlineto{\pgfqpoint{5.111661in}{2.828493in}}%
\pgfpathlineto{\pgfqpoint{5.097373in}{2.814302in}}%
\pgfpathlineto{\pgfqpoint{5.083105in}{2.800276in}}%
\pgfpathlineto{\pgfqpoint{5.068856in}{2.786414in}}%
\pgfpathlineto{\pgfqpoint{5.061303in}{2.776531in}}%
\pgfpathlineto{\pgfqpoint{5.053741in}{2.766481in}}%
\pgfpathlineto{\pgfqpoint{5.046172in}{2.756262in}}%
\pgfpathlineto{\pgfqpoint{5.038594in}{2.745876in}}%
\pgfpathclose%
\pgfusepath{fill}%
\end{pgfscope}%
\begin{pgfscope}%
\pgfpathrectangle{\pgfqpoint{1.254980in}{0.150000in}}{\pgfqpoint{5.490039in}{5.490039in}}%
\pgfusepath{clip}%
\pgfsetbuttcap%
\pgfsetroundjoin%
\definecolor{currentfill}{rgb}{0.271828,0.209303,0.504434}%
\pgfsetfillcolor{currentfill}%
\pgfsetfillopacity{0.700000}%
\pgfsetlinewidth{0.000000pt}%
\definecolor{currentstroke}{rgb}{0.000000,0.000000,0.000000}%
\pgfsetstrokecolor{currentstroke}%
\pgfsetdash{}{0pt}%
\pgfpathmoveto{\pgfqpoint{4.126721in}{1.619170in}}%
\pgfpathlineto{\pgfqpoint{4.140434in}{1.624056in}}%
\pgfpathlineto{\pgfqpoint{4.154159in}{1.629099in}}%
\pgfpathlineto{\pgfqpoint{4.167896in}{1.634299in}}%
\pgfpathlineto{\pgfqpoint{4.181645in}{1.639656in}}%
\pgfpathlineto{\pgfqpoint{4.189531in}{1.654275in}}%
\pgfpathlineto{\pgfqpoint{4.197412in}{1.668904in}}%
\pgfpathlineto{\pgfqpoint{4.205290in}{1.683538in}}%
\pgfpathlineto{\pgfqpoint{4.213163in}{1.698174in}}%
\pgfpathlineto{\pgfqpoint{4.199414in}{1.692363in}}%
\pgfpathlineto{\pgfqpoint{4.185677in}{1.686709in}}%
\pgfpathlineto{\pgfqpoint{4.171952in}{1.681213in}}%
\pgfpathlineto{\pgfqpoint{4.158239in}{1.675875in}}%
\pgfpathlineto{\pgfqpoint{4.150366in}{1.661682in}}%
\pgfpathlineto{\pgfqpoint{4.142488in}{1.647498in}}%
\pgfpathlineto{\pgfqpoint{4.134607in}{1.633326in}}%
\pgfpathlineto{\pgfqpoint{4.126721in}{1.619170in}}%
\pgfpathclose%
\pgfusepath{fill}%
\end{pgfscope}%
\begin{pgfscope}%
\pgfpathrectangle{\pgfqpoint{1.254980in}{0.150000in}}{\pgfqpoint{5.490039in}{5.490039in}}%
\pgfusepath{clip}%
\pgfsetbuttcap%
\pgfsetroundjoin%
\definecolor{currentfill}{rgb}{0.283197,0.115680,0.436115}%
\pgfsetfillcolor{currentfill}%
\pgfsetfillopacity{0.700000}%
\pgfsetlinewidth{0.000000pt}%
\definecolor{currentstroke}{rgb}{0.000000,0.000000,0.000000}%
\pgfsetstrokecolor{currentstroke}%
\pgfsetdash{}{0pt}%
\pgfpathmoveto{\pgfqpoint{3.922369in}{1.430985in}}%
\pgfpathlineto{\pgfqpoint{3.936009in}{1.433095in}}%
\pgfpathlineto{\pgfqpoint{3.949659in}{1.435362in}}%
\pgfpathlineto{\pgfqpoint{3.963318in}{1.437786in}}%
\pgfpathlineto{\pgfqpoint{3.976987in}{1.440366in}}%
\pgfpathlineto{\pgfqpoint{3.984928in}{1.453419in}}%
\pgfpathlineto{\pgfqpoint{3.992865in}{1.466552in}}%
\pgfpathlineto{\pgfqpoint{4.000796in}{1.479760in}}%
\pgfpathlineto{\pgfqpoint{4.008724in}{1.493038in}}%
\pgfpathlineto{\pgfqpoint{3.995059in}{1.489924in}}%
\pgfpathlineto{\pgfqpoint{3.981405in}{1.486967in}}%
\pgfpathlineto{\pgfqpoint{3.967761in}{1.484167in}}%
\pgfpathlineto{\pgfqpoint{3.954126in}{1.481524in}}%
\pgfpathlineto{\pgfqpoint{3.946194in}{1.468769in}}%
\pgfpathlineto{\pgfqpoint{3.938257in}{1.456091in}}%
\pgfpathlineto{\pgfqpoint{3.930315in}{1.443494in}}%
\pgfpathlineto{\pgfqpoint{3.922369in}{1.430985in}}%
\pgfpathclose%
\pgfusepath{fill}%
\end{pgfscope}%
\begin{pgfscope}%
\pgfpathrectangle{\pgfqpoint{1.254980in}{0.150000in}}{\pgfqpoint{5.490039in}{5.490039in}}%
\pgfusepath{clip}%
\pgfsetbuttcap%
\pgfsetroundjoin%
\definecolor{currentfill}{rgb}{0.120092,0.600104,0.542530}%
\pgfsetfillcolor{currentfill}%
\pgfsetfillopacity{0.700000}%
\pgfsetlinewidth{0.000000pt}%
\definecolor{currentstroke}{rgb}{0.000000,0.000000,0.000000}%
\pgfsetstrokecolor{currentstroke}%
\pgfsetdash{}{0pt}%
\pgfpathmoveto{\pgfqpoint{4.920880in}{2.603491in}}%
\pgfpathlineto{\pgfqpoint{4.935054in}{2.616557in}}%
\pgfpathlineto{\pgfqpoint{4.949246in}{2.629787in}}%
\pgfpathlineto{\pgfqpoint{4.963458in}{2.643180in}}%
\pgfpathlineto{\pgfqpoint{4.977688in}{2.656737in}}%
\pgfpathlineto{\pgfqpoint{4.985328in}{2.668466in}}%
\pgfpathlineto{\pgfqpoint{4.992961in}{2.680028in}}%
\pgfpathlineto{\pgfqpoint{5.000586in}{2.691423in}}%
\pgfpathlineto{\pgfqpoint{5.008203in}{2.702650in}}%
\pgfpathlineto{\pgfqpoint{4.993972in}{2.689014in}}%
\pgfpathlineto{\pgfqpoint{4.979760in}{2.675543in}}%
\pgfpathlineto{\pgfqpoint{4.965566in}{2.662235in}}%
\pgfpathlineto{\pgfqpoint{4.951392in}{2.649092in}}%
\pgfpathlineto{\pgfqpoint{4.943775in}{2.637931in}}%
\pgfpathlineto{\pgfqpoint{4.936150in}{2.626610in}}%
\pgfpathlineto{\pgfqpoint{4.928519in}{2.615130in}}%
\pgfpathlineto{\pgfqpoint{4.920880in}{2.603491in}}%
\pgfpathclose%
\pgfusepath{fill}%
\end{pgfscope}%
\begin{pgfscope}%
\pgfpathrectangle{\pgfqpoint{1.254980in}{0.150000in}}{\pgfqpoint{5.490039in}{5.490039in}}%
\pgfusepath{clip}%
\pgfsetbuttcap%
\pgfsetroundjoin%
\definecolor{currentfill}{rgb}{0.269944,0.014625,0.341379}%
\pgfsetfillcolor{currentfill}%
\pgfsetfillopacity{0.700000}%
\pgfsetlinewidth{0.000000pt}%
\definecolor{currentstroke}{rgb}{0.000000,0.000000,0.000000}%
\pgfsetstrokecolor{currentstroke}%
\pgfsetdash{}{0pt}%
\pgfpathmoveto{\pgfqpoint{3.576592in}{1.273220in}}%
\pgfpathlineto{\pgfqpoint{3.590163in}{1.270401in}}%
\pgfpathlineto{\pgfqpoint{3.603740in}{1.267741in}}%
\pgfpathlineto{\pgfqpoint{3.617322in}{1.265241in}}%
\pgfpathlineto{\pgfqpoint{3.630910in}{1.262900in}}%
\pgfpathlineto{\pgfqpoint{3.638993in}{1.271306in}}%
\pgfpathlineto{\pgfqpoint{3.647069in}{1.279911in}}%
\pgfpathlineto{\pgfqpoint{3.655136in}{1.288710in}}%
\pgfpathlineto{\pgfqpoint{3.663196in}{1.297696in}}%
\pgfpathlineto{\pgfqpoint{3.649625in}{1.299396in}}%
\pgfpathlineto{\pgfqpoint{3.636060in}{1.301256in}}%
\pgfpathlineto{\pgfqpoint{3.622502in}{1.303275in}}%
\pgfpathlineto{\pgfqpoint{3.608949in}{1.305455in}}%
\pgfpathlineto{\pgfqpoint{3.600872in}{1.297098in}}%
\pgfpathlineto{\pgfqpoint{3.592787in}{1.288936in}}%
\pgfpathlineto{\pgfqpoint{3.584694in}{1.280975in}}%
\pgfpathlineto{\pgfqpoint{3.576592in}{1.273220in}}%
\pgfpathclose%
\pgfusepath{fill}%
\end{pgfscope}%
\begin{pgfscope}%
\pgfpathrectangle{\pgfqpoint{1.254980in}{0.150000in}}{\pgfqpoint{5.490039in}{5.490039in}}%
\pgfusepath{clip}%
\pgfsetbuttcap%
\pgfsetroundjoin%
\definecolor{currentfill}{rgb}{0.133743,0.548535,0.553541}%
\pgfsetfillcolor{currentfill}%
\pgfsetfillopacity{0.700000}%
\pgfsetlinewidth{0.000000pt}%
\definecolor{currentstroke}{rgb}{0.000000,0.000000,0.000000}%
\pgfsetstrokecolor{currentstroke}%
\pgfsetdash{}{0pt}%
\pgfpathmoveto{\pgfqpoint{4.803019in}{2.455079in}}%
\pgfpathlineto{\pgfqpoint{4.817116in}{2.467245in}}%
\pgfpathlineto{\pgfqpoint{4.831231in}{2.479574in}}%
\pgfpathlineto{\pgfqpoint{4.845364in}{2.492065in}}%
\pgfpathlineto{\pgfqpoint{4.859515in}{2.504720in}}%
\pgfpathlineto{\pgfqpoint{4.867210in}{2.517609in}}%
\pgfpathlineto{\pgfqpoint{4.874898in}{2.530345in}}%
\pgfpathlineto{\pgfqpoint{4.882579in}{2.542926in}}%
\pgfpathlineto{\pgfqpoint{4.890253in}{2.555353in}}%
\pgfpathlineto{\pgfqpoint{4.876099in}{2.542559in}}%
\pgfpathlineto{\pgfqpoint{4.861963in}{2.529928in}}%
\pgfpathlineto{\pgfqpoint{4.847846in}{2.517460in}}%
\pgfpathlineto{\pgfqpoint{4.833746in}{2.505155in}}%
\pgfpathlineto{\pgfqpoint{4.826074in}{2.492856in}}%
\pgfpathlineto{\pgfqpoint{4.818396in}{2.480410in}}%
\pgfpathlineto{\pgfqpoint{4.810710in}{2.467817in}}%
\pgfpathlineto{\pgfqpoint{4.803019in}{2.455079in}}%
\pgfpathclose%
\pgfusepath{fill}%
\end{pgfscope}%
\begin{pgfscope}%
\pgfpathrectangle{\pgfqpoint{1.254980in}{0.150000in}}{\pgfqpoint{5.490039in}{5.490039in}}%
\pgfusepath{clip}%
\pgfsetbuttcap%
\pgfsetroundjoin%
\definecolor{currentfill}{rgb}{0.232815,0.732247,0.459277}%
\pgfsetfillcolor{currentfill}%
\pgfsetfillopacity{0.700000}%
\pgfsetlinewidth{0.000000pt}%
\definecolor{currentstroke}{rgb}{0.000000,0.000000,0.000000}%
\pgfsetstrokecolor{currentstroke}%
\pgfsetdash{}{0pt}%
\pgfpathmoveto{\pgfqpoint{5.243470in}{2.974512in}}%
\pgfpathlineto{\pgfqpoint{5.257872in}{2.989629in}}%
\pgfpathlineto{\pgfqpoint{5.272296in}{3.004913in}}%
\pgfpathlineto{\pgfqpoint{5.286740in}{3.020362in}}%
\pgfpathlineto{\pgfqpoint{5.301205in}{3.035977in}}%
\pgfpathlineto{\pgfqpoint{5.308667in}{3.044178in}}%
\pgfpathlineto{\pgfqpoint{5.316119in}{3.052194in}}%
\pgfpathlineto{\pgfqpoint{5.323561in}{3.060029in}}%
\pgfpathlineto{\pgfqpoint{5.330994in}{3.067682in}}%
\pgfpathlineto{\pgfqpoint{5.316534in}{3.052147in}}%
\pgfpathlineto{\pgfqpoint{5.302095in}{3.036778in}}%
\pgfpathlineto{\pgfqpoint{5.287677in}{3.021575in}}%
\pgfpathlineto{\pgfqpoint{5.273280in}{3.006537in}}%
\pgfpathlineto{\pgfqpoint{5.265842in}{2.998792in}}%
\pgfpathlineto{\pgfqpoint{5.258394in}{2.990873in}}%
\pgfpathlineto{\pgfqpoint{5.250937in}{2.982780in}}%
\pgfpathlineto{\pgfqpoint{5.243470in}{2.974512in}}%
\pgfpathclose%
\pgfusepath{fill}%
\end{pgfscope}%
\begin{pgfscope}%
\pgfpathrectangle{\pgfqpoint{1.254980in}{0.150000in}}{\pgfqpoint{5.490039in}{5.490039in}}%
\pgfusepath{clip}%
\pgfsetbuttcap%
\pgfsetroundjoin%
\definecolor{currentfill}{rgb}{0.154815,0.493313,0.557840}%
\pgfsetfillcolor{currentfill}%
\pgfsetfillopacity{0.700000}%
\pgfsetlinewidth{0.000000pt}%
\definecolor{currentstroke}{rgb}{0.000000,0.000000,0.000000}%
\pgfsetstrokecolor{currentstroke}%
\pgfsetdash{}{0pt}%
\pgfpathmoveto{\pgfqpoint{4.685074in}{2.302517in}}%
\pgfpathlineto{\pgfqpoint{4.699096in}{2.313666in}}%
\pgfpathlineto{\pgfqpoint{4.713135in}{2.324977in}}%
\pgfpathlineto{\pgfqpoint{4.727191in}{2.336450in}}%
\pgfpathlineto{\pgfqpoint{4.741264in}{2.348085in}}%
\pgfpathlineto{\pgfqpoint{4.749004in}{2.361941in}}%
\pgfpathlineto{\pgfqpoint{4.756739in}{2.375662in}}%
\pgfpathlineto{\pgfqpoint{4.764467in}{2.389247in}}%
\pgfpathlineto{\pgfqpoint{4.772190in}{2.402695in}}%
\pgfpathlineto{\pgfqpoint{4.758113in}{2.390860in}}%
\pgfpathlineto{\pgfqpoint{4.744054in}{2.379188in}}%
\pgfpathlineto{\pgfqpoint{4.730011in}{2.367678in}}%
\pgfpathlineto{\pgfqpoint{4.715986in}{2.356329in}}%
\pgfpathlineto{\pgfqpoint{4.708267in}{2.343070in}}%
\pgfpathlineto{\pgfqpoint{4.700542in}{2.329680in}}%
\pgfpathlineto{\pgfqpoint{4.692811in}{2.316162in}}%
\pgfpathlineto{\pgfqpoint{4.685074in}{2.302517in}}%
\pgfpathclose%
\pgfusepath{fill}%
\end{pgfscope}%
\begin{pgfscope}%
\pgfpathrectangle{\pgfqpoint{1.254980in}{0.150000in}}{\pgfqpoint{5.490039in}{5.490039in}}%
\pgfusepath{clip}%
\pgfsetbuttcap%
\pgfsetroundjoin%
\definecolor{currentfill}{rgb}{0.177423,0.437527,0.557565}%
\pgfsetfillcolor{currentfill}%
\pgfsetfillopacity{0.700000}%
\pgfsetlinewidth{0.000000pt}%
\definecolor{currentstroke}{rgb}{0.000000,0.000000,0.000000}%
\pgfsetstrokecolor{currentstroke}%
\pgfsetdash{}{0pt}%
\pgfpathmoveto{\pgfqpoint{4.567096in}{2.147968in}}%
\pgfpathlineto{\pgfqpoint{4.581045in}{2.157987in}}%
\pgfpathlineto{\pgfqpoint{4.595010in}{2.168166in}}%
\pgfpathlineto{\pgfqpoint{4.608991in}{2.178506in}}%
\pgfpathlineto{\pgfqpoint{4.622987in}{2.189007in}}%
\pgfpathlineto{\pgfqpoint{4.630767in}{2.203600in}}%
\pgfpathlineto{\pgfqpoint{4.638541in}{2.218081in}}%
\pgfpathlineto{\pgfqpoint{4.646310in}{2.232450in}}%
\pgfpathlineto{\pgfqpoint{4.654073in}{2.246702in}}%
\pgfpathlineto{\pgfqpoint{4.640072in}{2.235943in}}%
\pgfpathlineto{\pgfqpoint{4.626087in}{2.225345in}}%
\pgfpathlineto{\pgfqpoint{4.612118in}{2.214907in}}%
\pgfpathlineto{\pgfqpoint{4.598165in}{2.204631in}}%
\pgfpathlineto{\pgfqpoint{4.590405in}{2.190625in}}%
\pgfpathlineto{\pgfqpoint{4.582641in}{2.176511in}}%
\pgfpathlineto{\pgfqpoint{4.574871in}{2.162291in}}%
\pgfpathlineto{\pgfqpoint{4.567096in}{2.147968in}}%
\pgfpathclose%
\pgfusepath{fill}%
\end{pgfscope}%
\begin{pgfscope}%
\pgfpathrectangle{\pgfqpoint{1.254980in}{0.150000in}}{\pgfqpoint{5.490039in}{5.490039in}}%
\pgfusepath{clip}%
\pgfsetbuttcap%
\pgfsetroundjoin%
\definecolor{currentfill}{rgb}{0.281887,0.150881,0.465405}%
\pgfsetfillcolor{currentfill}%
\pgfsetfillopacity{0.700000}%
\pgfsetlinewidth{0.000000pt}%
\definecolor{currentstroke}{rgb}{0.000000,0.000000,0.000000}%
\pgfsetstrokecolor{currentstroke}%
\pgfsetdash{}{0pt}%
\pgfpathmoveto{\pgfqpoint{4.008724in}{1.493038in}}%
\pgfpathlineto{\pgfqpoint{4.022398in}{1.496309in}}%
\pgfpathlineto{\pgfqpoint{4.036083in}{1.499736in}}%
\pgfpathlineto{\pgfqpoint{4.049778in}{1.503320in}}%
\pgfpathlineto{\pgfqpoint{4.063484in}{1.507061in}}%
\pgfpathlineto{\pgfqpoint{4.071404in}{1.520921in}}%
\pgfpathlineto{\pgfqpoint{4.079319in}{1.534835in}}%
\pgfpathlineto{\pgfqpoint{4.087230in}{1.548797in}}%
\pgfpathlineto{\pgfqpoint{4.095136in}{1.562803in}}%
\pgfpathlineto{\pgfqpoint{4.081433in}{1.558555in}}%
\pgfpathlineto{\pgfqpoint{4.067740in}{1.554463in}}%
\pgfpathlineto{\pgfqpoint{4.054058in}{1.550529in}}%
\pgfpathlineto{\pgfqpoint{4.040387in}{1.546752in}}%
\pgfpathlineto{\pgfqpoint{4.032478in}{1.533243in}}%
\pgfpathlineto{\pgfqpoint{4.024564in}{1.519784in}}%
\pgfpathlineto{\pgfqpoint{4.016646in}{1.506381in}}%
\pgfpathlineto{\pgfqpoint{4.008724in}{1.493038in}}%
\pgfpathclose%
\pgfusepath{fill}%
\end{pgfscope}%
\begin{pgfscope}%
\pgfpathrectangle{\pgfqpoint{1.254980in}{0.150000in}}{\pgfqpoint{5.490039in}{5.490039in}}%
\pgfusepath{clip}%
\pgfsetbuttcap%
\pgfsetroundjoin%
\definecolor{currentfill}{rgb}{0.203063,0.379716,0.553925}%
\pgfsetfillcolor{currentfill}%
\pgfsetfillopacity{0.700000}%
\pgfsetlinewidth{0.000000pt}%
\definecolor{currentstroke}{rgb}{0.000000,0.000000,0.000000}%
\pgfsetstrokecolor{currentstroke}%
\pgfsetdash{}{0pt}%
\pgfpathmoveto{\pgfqpoint{4.449115in}{1.993874in}}%
\pgfpathlineto{\pgfqpoint{4.462995in}{2.002651in}}%
\pgfpathlineto{\pgfqpoint{4.476890in}{2.011587in}}%
\pgfpathlineto{\pgfqpoint{4.490800in}{2.020683in}}%
\pgfpathlineto{\pgfqpoint{4.504725in}{2.029938in}}%
\pgfpathlineto{\pgfqpoint{4.512537in}{2.045003in}}%
\pgfpathlineto{\pgfqpoint{4.520345in}{2.059983in}}%
\pgfpathlineto{\pgfqpoint{4.528149in}{2.074878in}}%
\pgfpathlineto{\pgfqpoint{4.535948in}{2.089684in}}%
\pgfpathlineto{\pgfqpoint{4.522018in}{2.080113in}}%
\pgfpathlineto{\pgfqpoint{4.508104in}{2.070701in}}%
\pgfpathlineto{\pgfqpoint{4.494205in}{2.061449in}}%
\pgfpathlineto{\pgfqpoint{4.480321in}{2.052357in}}%
\pgfpathlineto{\pgfqpoint{4.472527in}{2.037856in}}%
\pgfpathlineto{\pgfqpoint{4.464727in}{2.023273in}}%
\pgfpathlineto{\pgfqpoint{4.456923in}{2.008611in}}%
\pgfpathlineto{\pgfqpoint{4.449115in}{1.993874in}}%
\pgfpathclose%
\pgfusepath{fill}%
\end{pgfscope}%
\begin{pgfscope}%
\pgfpathrectangle{\pgfqpoint{1.254980in}{0.150000in}}{\pgfqpoint{5.490039in}{5.490039in}}%
\pgfusepath{clip}%
\pgfsetbuttcap%
\pgfsetroundjoin%
\definecolor{currentfill}{rgb}{0.233603,0.313828,0.543914}%
\pgfsetfillcolor{currentfill}%
\pgfsetfillopacity{0.700000}%
\pgfsetlinewidth{0.000000pt}%
\definecolor{currentstroke}{rgb}{0.000000,0.000000,0.000000}%
\pgfsetstrokecolor{currentstroke}%
\pgfsetdash{}{0pt}%
\pgfpathmoveto{\pgfqpoint{4.331142in}{1.842948in}}%
\pgfpathlineto{\pgfqpoint{4.344959in}{1.850374in}}%
\pgfpathlineto{\pgfqpoint{4.358790in}{1.857959in}}%
\pgfpathlineto{\pgfqpoint{4.372634in}{1.865702in}}%
\pgfpathlineto{\pgfqpoint{4.386493in}{1.873603in}}%
\pgfpathlineto{\pgfqpoint{4.394335in}{1.888838in}}%
\pgfpathlineto{\pgfqpoint{4.402174in}{1.904022in}}%
\pgfpathlineto{\pgfqpoint{4.410008in}{1.919152in}}%
\pgfpathlineto{\pgfqpoint{4.417838in}{1.934224in}}%
\pgfpathlineto{\pgfqpoint{4.403975in}{1.925950in}}%
\pgfpathlineto{\pgfqpoint{4.390127in}{1.917835in}}%
\pgfpathlineto{\pgfqpoint{4.376294in}{1.909879in}}%
\pgfpathlineto{\pgfqpoint{4.362474in}{1.902081in}}%
\pgfpathlineto{\pgfqpoint{4.354647in}{1.887370in}}%
\pgfpathlineto{\pgfqpoint{4.346816in}{1.872609in}}%
\pgfpathlineto{\pgfqpoint{4.338981in}{1.857800in}}%
\pgfpathlineto{\pgfqpoint{4.331142in}{1.842948in}}%
\pgfpathclose%
\pgfusepath{fill}%
\end{pgfscope}%
\begin{pgfscope}%
\pgfpathrectangle{\pgfqpoint{1.254980in}{0.150000in}}{\pgfqpoint{5.490039in}{5.490039in}}%
\pgfusepath{clip}%
\pgfsetbuttcap%
\pgfsetroundjoin%
\definecolor{currentfill}{rgb}{0.395174,0.797475,0.367757}%
\pgfsetfillcolor{currentfill}%
\pgfsetfillopacity{0.700000}%
\pgfsetlinewidth{0.000000pt}%
\definecolor{currentstroke}{rgb}{0.000000,0.000000,0.000000}%
\pgfsetstrokecolor{currentstroke}%
\pgfsetdash{}{0pt}%
\pgfpathmoveto{\pgfqpoint{5.448078in}{3.185340in}}%
\pgfpathlineto{\pgfqpoint{5.462630in}{3.201449in}}%
\pgfpathlineto{\pgfqpoint{5.477203in}{3.217725in}}%
\pgfpathlineto{\pgfqpoint{5.491798in}{3.234168in}}%
\pgfpathlineto{\pgfqpoint{5.506416in}{3.250778in}}%
\pgfpathlineto{\pgfqpoint{5.513737in}{3.256542in}}%
\pgfpathlineto{\pgfqpoint{5.521048in}{3.262125in}}%
\pgfpathlineto{\pgfqpoint{5.528347in}{3.267531in}}%
\pgfpathlineto{\pgfqpoint{5.535635in}{3.272762in}}%
\pgfpathlineto{\pgfqpoint{5.521028in}{3.256330in}}%
\pgfpathlineto{\pgfqpoint{5.506444in}{3.240065in}}%
\pgfpathlineto{\pgfqpoint{5.491881in}{3.223966in}}%
\pgfpathlineto{\pgfqpoint{5.477341in}{3.208034in}}%
\pgfpathlineto{\pgfqpoint{5.470041in}{3.202614in}}%
\pgfpathlineto{\pgfqpoint{5.462731in}{3.197026in}}%
\pgfpathlineto{\pgfqpoint{5.455410in}{3.191269in}}%
\pgfpathlineto{\pgfqpoint{5.448078in}{3.185340in}}%
\pgfpathclose%
\pgfusepath{fill}%
\end{pgfscope}%
\begin{pgfscope}%
\pgfpathrectangle{\pgfqpoint{1.254980in}{0.150000in}}{\pgfqpoint{5.490039in}{5.490039in}}%
\pgfusepath{clip}%
\pgfsetbuttcap%
\pgfsetroundjoin%
\definecolor{currentfill}{rgb}{0.175707,0.697900,0.491033}%
\pgfsetfillcolor{currentfill}%
\pgfsetfillopacity{0.700000}%
\pgfsetlinewidth{0.000000pt}%
\definecolor{currentstroke}{rgb}{0.000000,0.000000,0.000000}%
\pgfsetstrokecolor{currentstroke}%
\pgfsetdash{}{0pt}%
\pgfpathmoveto{\pgfqpoint{1.919597in}{3.003775in}}%
\pgfpathlineto{\pgfqpoint{1.933894in}{2.974251in}}%
\pgfpathlineto{\pgfqpoint{1.948171in}{2.945059in}}%
\pgfpathlineto{\pgfqpoint{1.962427in}{2.916197in}}%
\pgfpathlineto{\pgfqpoint{1.976664in}{2.887660in}}%
\pgfpathlineto{\pgfqpoint{1.986307in}{2.872699in}}%
\pgfpathlineto{\pgfqpoint{1.995915in}{2.858240in}}%
\pgfpathlineto{\pgfqpoint{2.005488in}{2.844273in}}%
\pgfpathlineto{\pgfqpoint{2.015027in}{2.830791in}}%
\pgfpathlineto{\pgfqpoint{2.000874in}{2.858517in}}%
\pgfpathlineto{\pgfqpoint{1.986703in}{2.886566in}}%
\pgfpathlineto{\pgfqpoint{1.972512in}{2.914942in}}%
\pgfpathlineto{\pgfqpoint{1.958300in}{2.943647in}}%
\pgfpathlineto{\pgfqpoint{1.948678in}{2.957929in}}%
\pgfpathlineto{\pgfqpoint{1.939020in}{2.972705in}}%
\pgfpathlineto{\pgfqpoint{1.929327in}{2.987984in}}%
\pgfpathlineto{\pgfqpoint{1.919597in}{3.003775in}}%
\pgfpathclose%
\pgfusepath{fill}%
\end{pgfscope}%
\begin{pgfscope}%
\pgfpathrectangle{\pgfqpoint{1.254980in}{0.150000in}}{\pgfqpoint{5.490039in}{5.490039in}}%
\pgfusepath{clip}%
\pgfsetbuttcap%
\pgfsetroundjoin%
\definecolor{currentfill}{rgb}{0.258965,0.251537,0.524736}%
\pgfsetfillcolor{currentfill}%
\pgfsetfillopacity{0.700000}%
\pgfsetlinewidth{0.000000pt}%
\definecolor{currentstroke}{rgb}{0.000000,0.000000,0.000000}%
\pgfsetstrokecolor{currentstroke}%
\pgfsetdash{}{0pt}%
\pgfpathmoveto{\pgfqpoint{4.213163in}{1.698174in}}%
\pgfpathlineto{\pgfqpoint{4.226924in}{1.704142in}}%
\pgfpathlineto{\pgfqpoint{4.240698in}{1.710268in}}%
\pgfpathlineto{\pgfqpoint{4.254485in}{1.716552in}}%
\pgfpathlineto{\pgfqpoint{4.268284in}{1.722992in}}%
\pgfpathlineto{\pgfqpoint{4.276155in}{1.738063in}}%
\pgfpathlineto{\pgfqpoint{4.284022in}{1.753119in}}%
\pgfpathlineto{\pgfqpoint{4.291886in}{1.768158in}}%
\pgfpathlineto{\pgfqpoint{4.299745in}{1.783174in}}%
\pgfpathlineto{\pgfqpoint{4.285943in}{1.776306in}}%
\pgfpathlineto{\pgfqpoint{4.272155in}{1.769595in}}%
\pgfpathlineto{\pgfqpoint{4.258380in}{1.763042in}}%
\pgfpathlineto{\pgfqpoint{4.244617in}{1.756648in}}%
\pgfpathlineto{\pgfqpoint{4.236760in}{1.742047in}}%
\pgfpathlineto{\pgfqpoint{4.228898in}{1.727432in}}%
\pgfpathlineto{\pgfqpoint{4.221033in}{1.712806in}}%
\pgfpathlineto{\pgfqpoint{4.213163in}{1.698174in}}%
\pgfpathclose%
\pgfusepath{fill}%
\end{pgfscope}%
\begin{pgfscope}%
\pgfpathrectangle{\pgfqpoint{1.254980in}{0.150000in}}{\pgfqpoint{5.490039in}{5.490039in}}%
\pgfusepath{clip}%
\pgfsetbuttcap%
\pgfsetroundjoin%
\definecolor{currentfill}{rgb}{0.271305,0.019942,0.347269}%
\pgfsetfillcolor{currentfill}%
\pgfsetfillopacity{0.700000}%
\pgfsetlinewidth{0.000000pt}%
\definecolor{currentstroke}{rgb}{0.000000,0.000000,0.000000}%
\pgfsetstrokecolor{currentstroke}%
\pgfsetdash{}{0pt}%
\pgfpathmoveto{\pgfqpoint{3.348395in}{1.282275in}}%
\pgfpathlineto{\pgfqpoint{3.361964in}{1.276138in}}%
\pgfpathlineto{\pgfqpoint{3.375537in}{1.270167in}}%
\pgfpathlineto{\pgfqpoint{3.389112in}{1.264360in}}%
\pgfpathlineto{\pgfqpoint{3.402690in}{1.258717in}}%
\pgfpathlineto{\pgfqpoint{3.410918in}{1.263159in}}%
\pgfpathlineto{\pgfqpoint{3.419134in}{1.267876in}}%
\pgfpathlineto{\pgfqpoint{3.427340in}{1.272862in}}%
\pgfpathlineto{\pgfqpoint{3.435534in}{1.278110in}}%
\pgfpathlineto{\pgfqpoint{3.421983in}{1.283054in}}%
\pgfpathlineto{\pgfqpoint{3.408435in}{1.288163in}}%
\pgfpathlineto{\pgfqpoint{3.394890in}{1.293436in}}%
\pgfpathlineto{\pgfqpoint{3.381349in}{1.298875in}}%
\pgfpathlineto{\pgfqpoint{3.373128in}{1.294314in}}%
\pgfpathlineto{\pgfqpoint{3.364896in}{1.290022in}}%
\pgfpathlineto{\pgfqpoint{3.356651in}{1.286007in}}%
\pgfpathlineto{\pgfqpoint{3.348395in}{1.282275in}}%
\pgfpathclose%
\pgfusepath{fill}%
\end{pgfscope}%
\begin{pgfscope}%
\pgfpathrectangle{\pgfqpoint{1.254980in}{0.150000in}}{\pgfqpoint{5.490039in}{5.490039in}}%
\pgfusepath{clip}%
\pgfsetbuttcap%
\pgfsetroundjoin%
\definecolor{currentfill}{rgb}{0.269944,0.014625,0.341379}%
\pgfsetfillcolor{currentfill}%
\pgfsetfillopacity{0.700000}%
\pgfsetlinewidth{0.000000pt}%
\definecolor{currentstroke}{rgb}{0.000000,0.000000,0.000000}%
\pgfsetstrokecolor{currentstroke}%
\pgfsetdash{}{0pt}%
\pgfpathmoveto{\pgfqpoint{3.489780in}{1.259962in}}%
\pgfpathlineto{\pgfqpoint{3.503353in}{1.255831in}}%
\pgfpathlineto{\pgfqpoint{3.516930in}{1.251861in}}%
\pgfpathlineto{\pgfqpoint{3.530511in}{1.248052in}}%
\pgfpathlineto{\pgfqpoint{3.544097in}{1.244403in}}%
\pgfpathlineto{\pgfqpoint{3.552235in}{1.251264in}}%
\pgfpathlineto{\pgfqpoint{3.560363in}{1.258358in}}%
\pgfpathlineto{\pgfqpoint{3.568482in}{1.265679in}}%
\pgfpathlineto{\pgfqpoint{3.576592in}{1.273220in}}%
\pgfpathlineto{\pgfqpoint{3.563027in}{1.276200in}}%
\pgfpathlineto{\pgfqpoint{3.549467in}{1.279340in}}%
\pgfpathlineto{\pgfqpoint{3.535912in}{1.282642in}}%
\pgfpathlineto{\pgfqpoint{3.522362in}{1.286105in}}%
\pgfpathlineto{\pgfqpoint{3.514231in}{1.279221in}}%
\pgfpathlineto{\pgfqpoint{3.506090in}{1.272566in}}%
\pgfpathlineto{\pgfqpoint{3.497940in}{1.266143in}}%
\pgfpathlineto{\pgfqpoint{3.489780in}{1.259962in}}%
\pgfpathclose%
\pgfusepath{fill}%
\end{pgfscope}%
\begin{pgfscope}%
\pgfpathrectangle{\pgfqpoint{1.254980in}{0.150000in}}{\pgfqpoint{5.490039in}{5.490039in}}%
\pgfusepath{clip}%
\pgfsetbuttcap%
\pgfsetroundjoin%
\definecolor{currentfill}{rgb}{0.166383,0.690856,0.496502}%
\pgfsetfillcolor{currentfill}%
\pgfsetfillopacity{0.700000}%
\pgfsetlinewidth{0.000000pt}%
\definecolor{currentstroke}{rgb}{0.000000,0.000000,0.000000}%
\pgfsetstrokecolor{currentstroke}%
\pgfsetdash{}{0pt}%
\pgfpathmoveto{\pgfqpoint{5.125969in}{2.842848in}}%
\pgfpathlineto{\pgfqpoint{5.140296in}{2.857369in}}%
\pgfpathlineto{\pgfqpoint{5.154644in}{2.872054in}}%
\pgfpathlineto{\pgfqpoint{5.169012in}{2.886905in}}%
\pgfpathlineto{\pgfqpoint{5.183400in}{2.901922in}}%
\pgfpathlineto{\pgfqpoint{5.190941in}{2.911629in}}%
\pgfpathlineto{\pgfqpoint{5.198473in}{2.921154in}}%
\pgfpathlineto{\pgfqpoint{5.205996in}{2.930497in}}%
\pgfpathlineto{\pgfqpoint{5.213509in}{2.939659in}}%
\pgfpathlineto{\pgfqpoint{5.199123in}{2.924659in}}%
\pgfpathlineto{\pgfqpoint{5.184758in}{2.909825in}}%
\pgfpathlineto{\pgfqpoint{5.170412in}{2.895156in}}%
\pgfpathlineto{\pgfqpoint{5.156087in}{2.880651in}}%
\pgfpathlineto{\pgfqpoint{5.148571in}{2.871461in}}%
\pgfpathlineto{\pgfqpoint{5.141046in}{2.862098in}}%
\pgfpathlineto{\pgfqpoint{5.133512in}{2.852560in}}%
\pgfpathlineto{\pgfqpoint{5.125969in}{2.842848in}}%
\pgfpathclose%
\pgfusepath{fill}%
\end{pgfscope}%
\begin{pgfscope}%
\pgfpathrectangle{\pgfqpoint{1.254980in}{0.150000in}}{\pgfqpoint{5.490039in}{5.490039in}}%
\pgfusepath{clip}%
\pgfsetbuttcap%
\pgfsetroundjoin%
\definecolor{currentfill}{rgb}{0.276194,0.190074,0.493001}%
\pgfsetfillcolor{currentfill}%
\pgfsetfillopacity{0.700000}%
\pgfsetlinewidth{0.000000pt}%
\definecolor{currentstroke}{rgb}{0.000000,0.000000,0.000000}%
\pgfsetstrokecolor{currentstroke}%
\pgfsetdash{}{0pt}%
\pgfpathmoveto{\pgfqpoint{4.095136in}{1.562803in}}%
\pgfpathlineto{\pgfqpoint{4.108851in}{1.567208in}}%
\pgfpathlineto{\pgfqpoint{4.122577in}{1.571770in}}%
\pgfpathlineto{\pgfqpoint{4.136314in}{1.576489in}}%
\pgfpathlineto{\pgfqpoint{4.150063in}{1.581364in}}%
\pgfpathlineto{\pgfqpoint{4.157965in}{1.595900in}}%
\pgfpathlineto{\pgfqpoint{4.165862in}{1.610464in}}%
\pgfpathlineto{\pgfqpoint{4.173756in}{1.625051in}}%
\pgfpathlineto{\pgfqpoint{4.181645in}{1.639656in}}%
\pgfpathlineto{\pgfqpoint{4.167896in}{1.634299in}}%
\pgfpathlineto{\pgfqpoint{4.154159in}{1.629099in}}%
\pgfpathlineto{\pgfqpoint{4.140434in}{1.624056in}}%
\pgfpathlineto{\pgfqpoint{4.126721in}{1.619170in}}%
\pgfpathlineto{\pgfqpoint{4.118831in}{1.605036in}}%
\pgfpathlineto{\pgfqpoint{4.110937in}{1.590927in}}%
\pgfpathlineto{\pgfqpoint{4.103039in}{1.576848in}}%
\pgfpathlineto{\pgfqpoint{4.095136in}{1.562803in}}%
\pgfpathclose%
\pgfusepath{fill}%
\end{pgfscope}%
\begin{pgfscope}%
\pgfpathrectangle{\pgfqpoint{1.254980in}{0.150000in}}{\pgfqpoint{5.490039in}{5.490039in}}%
\pgfusepath{clip}%
\pgfsetbuttcap%
\pgfsetroundjoin%
\definecolor{currentfill}{rgb}{0.279566,0.067836,0.391917}%
\pgfsetfillcolor{currentfill}%
\pgfsetfillopacity{0.700000}%
\pgfsetlinewidth{0.000000pt}%
\definecolor{currentstroke}{rgb}{0.000000,0.000000,0.000000}%
\pgfsetstrokecolor{currentstroke}%
\pgfsetdash{}{0pt}%
\pgfpathmoveto{\pgfqpoint{3.804067in}{1.332397in}}%
\pgfpathlineto{\pgfqpoint{3.817689in}{1.332732in}}%
\pgfpathlineto{\pgfqpoint{3.831320in}{1.333224in}}%
\pgfpathlineto{\pgfqpoint{3.844959in}{1.333873in}}%
\pgfpathlineto{\pgfqpoint{3.858606in}{1.334678in}}%
\pgfpathlineto{\pgfqpoint{3.866595in}{1.346299in}}%
\pgfpathlineto{\pgfqpoint{3.874579in}{1.358051in}}%
\pgfpathlineto{\pgfqpoint{3.882557in}{1.369927in}}%
\pgfpathlineto{\pgfqpoint{3.890530in}{1.381922in}}%
\pgfpathlineto{\pgfqpoint{3.876892in}{1.380529in}}%
\pgfpathlineto{\pgfqpoint{3.863262in}{1.379293in}}%
\pgfpathlineto{\pgfqpoint{3.849641in}{1.378214in}}%
\pgfpathlineto{\pgfqpoint{3.836029in}{1.377292in}}%
\pgfpathlineto{\pgfqpoint{3.828047in}{1.365873in}}%
\pgfpathlineto{\pgfqpoint{3.820060in}{1.354581in}}%
\pgfpathlineto{\pgfqpoint{3.812066in}{1.343420in}}%
\pgfpathlineto{\pgfqpoint{3.804067in}{1.332397in}}%
\pgfpathclose%
\pgfusepath{fill}%
\end{pgfscope}%
\begin{pgfscope}%
\pgfpathrectangle{\pgfqpoint{1.254980in}{0.150000in}}{\pgfqpoint{5.490039in}{5.490039in}}%
\pgfusepath{clip}%
\pgfsetbuttcap%
\pgfsetroundjoin%
\definecolor{currentfill}{rgb}{0.276022,0.044167,0.370164}%
\pgfsetfillcolor{currentfill}%
\pgfsetfillopacity{0.700000}%
\pgfsetlinewidth{0.000000pt}%
\definecolor{currentstroke}{rgb}{0.000000,0.000000,0.000000}%
\pgfsetstrokecolor{currentstroke}%
\pgfsetdash{}{0pt}%
\pgfpathmoveto{\pgfqpoint{3.717546in}{1.292482in}}%
\pgfpathlineto{\pgfqpoint{3.731150in}{1.291574in}}%
\pgfpathlineto{\pgfqpoint{3.744762in}{1.290824in}}%
\pgfpathlineto{\pgfqpoint{3.758381in}{1.290231in}}%
\pgfpathlineto{\pgfqpoint{3.772007in}{1.289795in}}%
\pgfpathlineto{\pgfqpoint{3.780032in}{1.300210in}}%
\pgfpathlineto{\pgfqpoint{3.788050in}{1.310786in}}%
\pgfpathlineto{\pgfqpoint{3.796061in}{1.321517in}}%
\pgfpathlineto{\pgfqpoint{3.804067in}{1.332397in}}%
\pgfpathlineto{\pgfqpoint{3.790452in}{1.332219in}}%
\pgfpathlineto{\pgfqpoint{3.776846in}{1.332198in}}%
\pgfpathlineto{\pgfqpoint{3.763247in}{1.332335in}}%
\pgfpathlineto{\pgfqpoint{3.749655in}{1.332629in}}%
\pgfpathlineto{\pgfqpoint{3.741638in}{1.322352in}}%
\pgfpathlineto{\pgfqpoint{3.733614in}{1.312232in}}%
\pgfpathlineto{\pgfqpoint{3.725583in}{1.302273in}}%
\pgfpathlineto{\pgfqpoint{3.717546in}{1.292482in}}%
\pgfpathclose%
\pgfusepath{fill}%
\end{pgfscope}%
\begin{pgfscope}%
\pgfpathrectangle{\pgfqpoint{1.254980in}{0.150000in}}{\pgfqpoint{5.490039in}{5.490039in}}%
\pgfusepath{clip}%
\pgfsetbuttcap%
\pgfsetroundjoin%
\definecolor{currentfill}{rgb}{0.282656,0.100196,0.422160}%
\pgfsetfillcolor{currentfill}%
\pgfsetfillopacity{0.700000}%
\pgfsetlinewidth{0.000000pt}%
\definecolor{currentstroke}{rgb}{0.000000,0.000000,0.000000}%
\pgfsetstrokecolor{currentstroke}%
\pgfsetdash{}{0pt}%
\pgfpathmoveto{\pgfqpoint{3.890530in}{1.381922in}}%
\pgfpathlineto{\pgfqpoint{3.904177in}{1.383472in}}%
\pgfpathlineto{\pgfqpoint{3.917834in}{1.385178in}}%
\pgfpathlineto{\pgfqpoint{3.931499in}{1.387040in}}%
\pgfpathlineto{\pgfqpoint{3.945174in}{1.389059in}}%
\pgfpathlineto{\pgfqpoint{3.953135in}{1.401739in}}%
\pgfpathlineto{\pgfqpoint{3.961090in}{1.414521in}}%
\pgfpathlineto{\pgfqpoint{3.969041in}{1.427398in}}%
\pgfpathlineto{\pgfqpoint{3.976987in}{1.440366in}}%
\pgfpathlineto{\pgfqpoint{3.963318in}{1.437786in}}%
\pgfpathlineto{\pgfqpoint{3.949659in}{1.435362in}}%
\pgfpathlineto{\pgfqpoint{3.936009in}{1.433095in}}%
\pgfpathlineto{\pgfqpoint{3.922369in}{1.430985in}}%
\pgfpathlineto{\pgfqpoint{3.914417in}{1.418568in}}%
\pgfpathlineto{\pgfqpoint{3.906460in}{1.406248in}}%
\pgfpathlineto{\pgfqpoint{3.898498in}{1.394031in}}%
\pgfpathlineto{\pgfqpoint{3.890530in}{1.381922in}}%
\pgfpathclose%
\pgfusepath{fill}%
\end{pgfscope}%
\begin{pgfscope}%
\pgfpathrectangle{\pgfqpoint{1.254980in}{0.150000in}}{\pgfqpoint{5.490039in}{5.490039in}}%
\pgfusepath{clip}%
\pgfsetbuttcap%
\pgfsetroundjoin%
\definecolor{currentfill}{rgb}{0.126326,0.644107,0.525311}%
\pgfsetfillcolor{currentfill}%
\pgfsetfillopacity{0.700000}%
\pgfsetlinewidth{0.000000pt}%
\definecolor{currentstroke}{rgb}{0.000000,0.000000,0.000000}%
\pgfsetstrokecolor{currentstroke}%
\pgfsetdash{}{0pt}%
\pgfpathmoveto{\pgfqpoint{5.008203in}{2.702650in}}%
\pgfpathlineto{\pgfqpoint{5.022454in}{2.716449in}}%
\pgfpathlineto{\pgfqpoint{5.036723in}{2.730413in}}%
\pgfpathlineto{\pgfqpoint{5.051012in}{2.744541in}}%
\pgfpathlineto{\pgfqpoint{5.065321in}{2.758834in}}%
\pgfpathlineto{\pgfqpoint{5.072931in}{2.769952in}}%
\pgfpathlineto{\pgfqpoint{5.080533in}{2.780893in}}%
\pgfpathlineto{\pgfqpoint{5.088127in}{2.791659in}}%
\pgfpathlineto{\pgfqpoint{5.095712in}{2.802248in}}%
\pgfpathlineto{\pgfqpoint{5.081403in}{2.787908in}}%
\pgfpathlineto{\pgfqpoint{5.067114in}{2.773733in}}%
\pgfpathlineto{\pgfqpoint{5.052844in}{2.759722in}}%
\pgfpathlineto{\pgfqpoint{5.038594in}{2.745876in}}%
\pgfpathlineto{\pgfqpoint{5.031008in}{2.735322in}}%
\pgfpathlineto{\pgfqpoint{5.023414in}{2.724599in}}%
\pgfpathlineto{\pgfqpoint{5.015813in}{2.713708in}}%
\pgfpathlineto{\pgfqpoint{5.008203in}{2.702650in}}%
\pgfpathclose%
\pgfusepath{fill}%
\end{pgfscope}%
\begin{pgfscope}%
\pgfpathrectangle{\pgfqpoint{1.254980in}{0.150000in}}{\pgfqpoint{5.490039in}{5.490039in}}%
\pgfusepath{clip}%
\pgfsetbuttcap%
\pgfsetroundjoin%
\definecolor{currentfill}{rgb}{0.487026,0.823929,0.312321}%
\pgfsetfillcolor{currentfill}%
\pgfsetfillopacity{0.700000}%
\pgfsetlinewidth{0.000000pt}%
\definecolor{currentstroke}{rgb}{0.000000,0.000000,0.000000}%
\pgfsetstrokecolor{currentstroke}%
\pgfsetdash{}{0pt}%
\pgfpathmoveto{\pgfqpoint{5.535635in}{3.272762in}}%
\pgfpathlineto{\pgfqpoint{5.550264in}{3.289361in}}%
\pgfpathlineto{\pgfqpoint{5.564916in}{3.306127in}}%
\pgfpathlineto{\pgfqpoint{5.579590in}{3.323061in}}%
\pgfpathlineto{\pgfqpoint{5.594288in}{3.340163in}}%
\pgfpathlineto{\pgfqpoint{5.601552in}{3.345021in}}%
\pgfpathlineto{\pgfqpoint{5.608805in}{3.349701in}}%
\pgfpathlineto{\pgfqpoint{5.616047in}{3.354205in}}%
\pgfpathlineto{\pgfqpoint{5.623277in}{3.358535in}}%
\pgfpathlineto{\pgfqpoint{5.608593in}{3.341646in}}%
\pgfpathlineto{\pgfqpoint{5.593932in}{3.324923in}}%
\pgfpathlineto{\pgfqpoint{5.579293in}{3.308368in}}%
\pgfpathlineto{\pgfqpoint{5.564677in}{3.291979in}}%
\pgfpathlineto{\pgfqpoint{5.557433in}{3.287426in}}%
\pgfpathlineto{\pgfqpoint{5.550178in}{3.282707in}}%
\pgfpathlineto{\pgfqpoint{5.542912in}{3.277820in}}%
\pgfpathlineto{\pgfqpoint{5.535635in}{3.272762in}}%
\pgfpathclose%
\pgfusepath{fill}%
\end{pgfscope}%
\begin{pgfscope}%
\pgfpathrectangle{\pgfqpoint{1.254980in}{0.150000in}}{\pgfqpoint{5.490039in}{5.490039in}}%
\pgfusepath{clip}%
\pgfsetbuttcap%
\pgfsetroundjoin%
\definecolor{currentfill}{rgb}{0.311925,0.767822,0.415586}%
\pgfsetfillcolor{currentfill}%
\pgfsetfillopacity{0.700000}%
\pgfsetlinewidth{0.000000pt}%
\definecolor{currentstroke}{rgb}{0.000000,0.000000,0.000000}%
\pgfsetstrokecolor{currentstroke}%
\pgfsetdash{}{0pt}%
\pgfpathmoveto{\pgfqpoint{5.330994in}{3.067682in}}%
\pgfpathlineto{\pgfqpoint{5.345475in}{3.083383in}}%
\pgfpathlineto{\pgfqpoint{5.359977in}{3.099251in}}%
\pgfpathlineto{\pgfqpoint{5.374501in}{3.115285in}}%
\pgfpathlineto{\pgfqpoint{5.389047in}{3.131486in}}%
\pgfpathlineto{\pgfqpoint{5.396462in}{3.138858in}}%
\pgfpathlineto{\pgfqpoint{5.403868in}{3.146044in}}%
\pgfpathlineto{\pgfqpoint{5.411262in}{3.153046in}}%
\pgfpathlineto{\pgfqpoint{5.418647in}{3.159864in}}%
\pgfpathlineto{\pgfqpoint{5.404108in}{3.143776in}}%
\pgfpathlineto{\pgfqpoint{5.389591in}{3.127855in}}%
\pgfpathlineto{\pgfqpoint{5.375096in}{3.112100in}}%
\pgfpathlineto{\pgfqpoint{5.360622in}{3.096511in}}%
\pgfpathlineto{\pgfqpoint{5.353230in}{3.089568in}}%
\pgfpathlineto{\pgfqpoint{5.345828in}{3.082450in}}%
\pgfpathlineto{\pgfqpoint{5.338416in}{3.075155in}}%
\pgfpathlineto{\pgfqpoint{5.330994in}{3.067682in}}%
\pgfpathclose%
\pgfusepath{fill}%
\end{pgfscope}%
\begin{pgfscope}%
\pgfpathrectangle{\pgfqpoint{1.254980in}{0.150000in}}{\pgfqpoint{5.490039in}{5.490039in}}%
\pgfusepath{clip}%
\pgfsetbuttcap%
\pgfsetroundjoin%
\definecolor{currentfill}{rgb}{0.272594,0.025563,0.353093}%
\pgfsetfillcolor{currentfill}%
\pgfsetfillopacity{0.700000}%
\pgfsetlinewidth{0.000000pt}%
\definecolor{currentstroke}{rgb}{0.000000,0.000000,0.000000}%
\pgfsetstrokecolor{currentstroke}%
\pgfsetdash{}{0pt}%
\pgfpathmoveto{\pgfqpoint{3.630910in}{1.262900in}}%
\pgfpathlineto{\pgfqpoint{3.644504in}{1.260718in}}%
\pgfpathlineto{\pgfqpoint{3.658104in}{1.258695in}}%
\pgfpathlineto{\pgfqpoint{3.671711in}{1.256830in}}%
\pgfpathlineto{\pgfqpoint{3.685324in}{1.255123in}}%
\pgfpathlineto{\pgfqpoint{3.693390in}{1.264180in}}%
\pgfpathlineto{\pgfqpoint{3.701450in}{1.273429in}}%
\pgfpathlineto{\pgfqpoint{3.709501in}{1.282866in}}%
\pgfpathlineto{\pgfqpoint{3.717546in}{1.292482in}}%
\pgfpathlineto{\pgfqpoint{3.703948in}{1.293548in}}%
\pgfpathlineto{\pgfqpoint{3.690358in}{1.294772in}}%
\pgfpathlineto{\pgfqpoint{3.676773in}{1.296155in}}%
\pgfpathlineto{\pgfqpoint{3.663196in}{1.297696in}}%
\pgfpathlineto{\pgfqpoint{3.655136in}{1.288710in}}%
\pgfpathlineto{\pgfqpoint{3.647069in}{1.279911in}}%
\pgfpathlineto{\pgfqpoint{3.638993in}{1.271306in}}%
\pgfpathlineto{\pgfqpoint{3.630910in}{1.262900in}}%
\pgfpathclose%
\pgfusepath{fill}%
\end{pgfscope}%
\begin{pgfscope}%
\pgfpathrectangle{\pgfqpoint{1.254980in}{0.150000in}}{\pgfqpoint{5.490039in}{5.490039in}}%
\pgfusepath{clip}%
\pgfsetbuttcap%
\pgfsetroundjoin%
\definecolor{currentfill}{rgb}{0.212395,0.359683,0.551710}%
\pgfsetfillcolor{currentfill}%
\pgfsetfillopacity{0.700000}%
\pgfsetlinewidth{0.000000pt}%
\definecolor{currentstroke}{rgb}{0.000000,0.000000,0.000000}%
\pgfsetstrokecolor{currentstroke}%
\pgfsetdash{}{0pt}%
\pgfpathmoveto{\pgfqpoint{4.417838in}{1.934224in}}%
\pgfpathlineto{\pgfqpoint{4.431714in}{1.942657in}}%
\pgfpathlineto{\pgfqpoint{4.445605in}{1.951249in}}%
\pgfpathlineto{\pgfqpoint{4.459511in}{1.959999in}}%
\pgfpathlineto{\pgfqpoint{4.473431in}{1.968909in}}%
\pgfpathlineto{\pgfqpoint{4.481261in}{1.984276in}}%
\pgfpathlineto{\pgfqpoint{4.489087in}{1.999572in}}%
\pgfpathlineto{\pgfqpoint{4.496908in}{2.014794in}}%
\pgfpathlineto{\pgfqpoint{4.504725in}{2.029938in}}%
\pgfpathlineto{\pgfqpoint{4.490800in}{2.020683in}}%
\pgfpathlineto{\pgfqpoint{4.476890in}{2.011587in}}%
\pgfpathlineto{\pgfqpoint{4.462995in}{2.002651in}}%
\pgfpathlineto{\pgfqpoint{4.449115in}{1.993874in}}%
\pgfpathlineto{\pgfqpoint{4.441302in}{1.979064in}}%
\pgfpathlineto{\pgfqpoint{4.433485in}{1.964183in}}%
\pgfpathlineto{\pgfqpoint{4.425663in}{1.949235in}}%
\pgfpathlineto{\pgfqpoint{4.417838in}{1.934224in}}%
\pgfpathclose%
\pgfusepath{fill}%
\end{pgfscope}%
\begin{pgfscope}%
\pgfpathrectangle{\pgfqpoint{1.254980in}{0.150000in}}{\pgfqpoint{5.490039in}{5.490039in}}%
\pgfusepath{clip}%
\pgfsetbuttcap%
\pgfsetroundjoin%
\definecolor{currentfill}{rgb}{0.121148,0.592739,0.544641}%
\pgfsetfillcolor{currentfill}%
\pgfsetfillopacity{0.700000}%
\pgfsetlinewidth{0.000000pt}%
\definecolor{currentstroke}{rgb}{0.000000,0.000000,0.000000}%
\pgfsetstrokecolor{currentstroke}%
\pgfsetdash{}{0pt}%
\pgfpathmoveto{\pgfqpoint{4.890253in}{2.555353in}}%
\pgfpathlineto{\pgfqpoint{4.904425in}{2.568310in}}%
\pgfpathlineto{\pgfqpoint{4.918616in}{2.581431in}}%
\pgfpathlineto{\pgfqpoint{4.932825in}{2.594715in}}%
\pgfpathlineto{\pgfqpoint{4.947053in}{2.608163in}}%
\pgfpathlineto{\pgfqpoint{4.954723in}{2.620554in}}%
\pgfpathlineto{\pgfqpoint{4.962385in}{2.632781in}}%
\pgfpathlineto{\pgfqpoint{4.970040in}{2.644842in}}%
\pgfpathlineto{\pgfqpoint{4.977688in}{2.656737in}}%
\pgfpathlineto{\pgfqpoint{4.963458in}{2.643180in}}%
\pgfpathlineto{\pgfqpoint{4.949246in}{2.629787in}}%
\pgfpathlineto{\pgfqpoint{4.935054in}{2.616557in}}%
\pgfpathlineto{\pgfqpoint{4.920880in}{2.603491in}}%
\pgfpathlineto{\pgfqpoint{4.913234in}{2.591693in}}%
\pgfpathlineto{\pgfqpoint{4.905580in}{2.579737in}}%
\pgfpathlineto{\pgfqpoint{4.897920in}{2.567623in}}%
\pgfpathlineto{\pgfqpoint{4.890253in}{2.555353in}}%
\pgfpathclose%
\pgfusepath{fill}%
\end{pgfscope}%
\begin{pgfscope}%
\pgfpathrectangle{\pgfqpoint{1.254980in}{0.150000in}}{\pgfqpoint{5.490039in}{5.490039in}}%
\pgfusepath{clip}%
\pgfsetbuttcap%
\pgfsetroundjoin%
\definecolor{currentfill}{rgb}{0.183898,0.422383,0.556944}%
\pgfsetfillcolor{currentfill}%
\pgfsetfillopacity{0.700000}%
\pgfsetlinewidth{0.000000pt}%
\definecolor{currentstroke}{rgb}{0.000000,0.000000,0.000000}%
\pgfsetstrokecolor{currentstroke}%
\pgfsetdash{}{0pt}%
\pgfpathmoveto{\pgfqpoint{4.535948in}{2.089684in}}%
\pgfpathlineto{\pgfqpoint{4.549893in}{2.099416in}}%
\pgfpathlineto{\pgfqpoint{4.563853in}{2.109308in}}%
\pgfpathlineto{\pgfqpoint{4.577829in}{2.119359in}}%
\pgfpathlineto{\pgfqpoint{4.591821in}{2.129572in}}%
\pgfpathlineto{\pgfqpoint{4.599620in}{2.144585in}}%
\pgfpathlineto{\pgfqpoint{4.607414in}{2.159497in}}%
\pgfpathlineto{\pgfqpoint{4.615203in}{2.174305in}}%
\pgfpathlineto{\pgfqpoint{4.622987in}{2.189007in}}%
\pgfpathlineto{\pgfqpoint{4.608991in}{2.178506in}}%
\pgfpathlineto{\pgfqpoint{4.595010in}{2.168166in}}%
\pgfpathlineto{\pgfqpoint{4.581045in}{2.157987in}}%
\pgfpathlineto{\pgfqpoint{4.567096in}{2.147968in}}%
\pgfpathlineto{\pgfqpoint{4.559316in}{2.133543in}}%
\pgfpathlineto{\pgfqpoint{4.551531in}{2.119019in}}%
\pgfpathlineto{\pgfqpoint{4.543742in}{2.104399in}}%
\pgfpathlineto{\pgfqpoint{4.535948in}{2.089684in}}%
\pgfpathclose%
\pgfusepath{fill}%
\end{pgfscope}%
\begin{pgfscope}%
\pgfpathrectangle{\pgfqpoint{1.254980in}{0.150000in}}{\pgfqpoint{5.490039in}{5.490039in}}%
\pgfusepath{clip}%
\pgfsetbuttcap%
\pgfsetroundjoin%
\definecolor{currentfill}{rgb}{0.241237,0.296485,0.539709}%
\pgfsetfillcolor{currentfill}%
\pgfsetfillopacity{0.700000}%
\pgfsetlinewidth{0.000000pt}%
\definecolor{currentstroke}{rgb}{0.000000,0.000000,0.000000}%
\pgfsetstrokecolor{currentstroke}%
\pgfsetdash{}{0pt}%
\pgfpathmoveto{\pgfqpoint{4.299745in}{1.783174in}}%
\pgfpathlineto{\pgfqpoint{4.313559in}{1.790201in}}%
\pgfpathlineto{\pgfqpoint{4.327387in}{1.797385in}}%
\pgfpathlineto{\pgfqpoint{4.341229in}{1.804727in}}%
\pgfpathlineto{\pgfqpoint{4.355084in}{1.812227in}}%
\pgfpathlineto{\pgfqpoint{4.362942in}{1.827630in}}%
\pgfpathlineto{\pgfqpoint{4.370796in}{1.842995in}}%
\pgfpathlineto{\pgfqpoint{4.378647in}{1.858321in}}%
\pgfpathlineto{\pgfqpoint{4.386493in}{1.873603in}}%
\pgfpathlineto{\pgfqpoint{4.372634in}{1.865702in}}%
\pgfpathlineto{\pgfqpoint{4.358790in}{1.857959in}}%
\pgfpathlineto{\pgfqpoint{4.344959in}{1.850374in}}%
\pgfpathlineto{\pgfqpoint{4.331142in}{1.842948in}}%
\pgfpathlineto{\pgfqpoint{4.323299in}{1.828056in}}%
\pgfpathlineto{\pgfqpoint{4.315451in}{1.813127in}}%
\pgfpathlineto{\pgfqpoint{4.307600in}{1.798165in}}%
\pgfpathlineto{\pgfqpoint{4.299745in}{1.783174in}}%
\pgfpathclose%
\pgfusepath{fill}%
\end{pgfscope}%
\begin{pgfscope}%
\pgfpathrectangle{\pgfqpoint{1.254980in}{0.150000in}}{\pgfqpoint{5.490039in}{5.490039in}}%
\pgfusepath{clip}%
\pgfsetbuttcap%
\pgfsetroundjoin%
\definecolor{currentfill}{rgb}{0.159194,0.482237,0.558073}%
\pgfsetfillcolor{currentfill}%
\pgfsetfillopacity{0.700000}%
\pgfsetlinewidth{0.000000pt}%
\definecolor{currentstroke}{rgb}{0.000000,0.000000,0.000000}%
\pgfsetstrokecolor{currentstroke}%
\pgfsetdash{}{0pt}%
\pgfpathmoveto{\pgfqpoint{4.654073in}{2.246702in}}%
\pgfpathlineto{\pgfqpoint{4.668091in}{2.257623in}}%
\pgfpathlineto{\pgfqpoint{4.682126in}{2.268704in}}%
\pgfpathlineto{\pgfqpoint{4.696177in}{2.279947in}}%
\pgfpathlineto{\pgfqpoint{4.710245in}{2.291352in}}%
\pgfpathlineto{\pgfqpoint{4.718008in}{2.305728in}}%
\pgfpathlineto{\pgfqpoint{4.725765in}{2.319977in}}%
\pgfpathlineto{\pgfqpoint{4.733517in}{2.334096in}}%
\pgfpathlineto{\pgfqpoint{4.741264in}{2.348085in}}%
\pgfpathlineto{\pgfqpoint{4.727191in}{2.336450in}}%
\pgfpathlineto{\pgfqpoint{4.713135in}{2.324977in}}%
\pgfpathlineto{\pgfqpoint{4.699096in}{2.313666in}}%
\pgfpathlineto{\pgfqpoint{4.685074in}{2.302517in}}%
\pgfpathlineto{\pgfqpoint{4.677332in}{2.288746in}}%
\pgfpathlineto{\pgfqpoint{4.669585in}{2.274853in}}%
\pgfpathlineto{\pgfqpoint{4.661832in}{2.260837in}}%
\pgfpathlineto{\pgfqpoint{4.654073in}{2.246702in}}%
\pgfpathclose%
\pgfusepath{fill}%
\end{pgfscope}%
\begin{pgfscope}%
\pgfpathrectangle{\pgfqpoint{1.254980in}{0.150000in}}{\pgfqpoint{5.490039in}{5.490039in}}%
\pgfusepath{clip}%
\pgfsetbuttcap%
\pgfsetroundjoin%
\definecolor{currentfill}{rgb}{0.137770,0.537492,0.554906}%
\pgfsetfillcolor{currentfill}%
\pgfsetfillopacity{0.700000}%
\pgfsetlinewidth{0.000000pt}%
\definecolor{currentstroke}{rgb}{0.000000,0.000000,0.000000}%
\pgfsetstrokecolor{currentstroke}%
\pgfsetdash{}{0pt}%
\pgfpathmoveto{\pgfqpoint{4.772190in}{2.402695in}}%
\pgfpathlineto{\pgfqpoint{4.786284in}{2.414691in}}%
\pgfpathlineto{\pgfqpoint{4.800396in}{2.426850in}}%
\pgfpathlineto{\pgfqpoint{4.814525in}{2.439172in}}%
\pgfpathlineto{\pgfqpoint{4.828672in}{2.451656in}}%
\pgfpathlineto{\pgfqpoint{4.836393in}{2.465146in}}%
\pgfpathlineto{\pgfqpoint{4.844107in}{2.478487in}}%
\pgfpathlineto{\pgfqpoint{4.851814in}{2.491679in}}%
\pgfpathlineto{\pgfqpoint{4.859515in}{2.504720in}}%
\pgfpathlineto{\pgfqpoint{4.845364in}{2.492065in}}%
\pgfpathlineto{\pgfqpoint{4.831231in}{2.479574in}}%
\pgfpathlineto{\pgfqpoint{4.817116in}{2.467245in}}%
\pgfpathlineto{\pgfqpoint{4.803019in}{2.455079in}}%
\pgfpathlineto{\pgfqpoint{4.795321in}{2.442196in}}%
\pgfpathlineto{\pgfqpoint{4.787617in}{2.429171in}}%
\pgfpathlineto{\pgfqpoint{4.779906in}{2.416003in}}%
\pgfpathlineto{\pgfqpoint{4.772190in}{2.402695in}}%
\pgfpathclose%
\pgfusepath{fill}%
\end{pgfscope}%
\begin{pgfscope}%
\pgfpathrectangle{\pgfqpoint{1.254980in}{0.150000in}}{\pgfqpoint{5.490039in}{5.490039in}}%
\pgfusepath{clip}%
\pgfsetbuttcap%
\pgfsetroundjoin%
\definecolor{currentfill}{rgb}{0.283072,0.130895,0.449241}%
\pgfsetfillcolor{currentfill}%
\pgfsetfillopacity{0.700000}%
\pgfsetlinewidth{0.000000pt}%
\definecolor{currentstroke}{rgb}{0.000000,0.000000,0.000000}%
\pgfsetstrokecolor{currentstroke}%
\pgfsetdash{}{0pt}%
\pgfpathmoveto{\pgfqpoint{3.976987in}{1.440366in}}%
\pgfpathlineto{\pgfqpoint{3.990666in}{1.443102in}}%
\pgfpathlineto{\pgfqpoint{4.004355in}{1.445995in}}%
\pgfpathlineto{\pgfqpoint{4.018054in}{1.449044in}}%
\pgfpathlineto{\pgfqpoint{4.031763in}{1.452248in}}%
\pgfpathlineto{\pgfqpoint{4.039700in}{1.465847in}}%
\pgfpathlineto{\pgfqpoint{4.047632in}{1.479518in}}%
\pgfpathlineto{\pgfqpoint{4.055560in}{1.493258in}}%
\pgfpathlineto{\pgfqpoint{4.063484in}{1.507061in}}%
\pgfpathlineto{\pgfqpoint{4.049778in}{1.503320in}}%
\pgfpathlineto{\pgfqpoint{4.036083in}{1.499736in}}%
\pgfpathlineto{\pgfqpoint{4.022398in}{1.496309in}}%
\pgfpathlineto{\pgfqpoint{4.008724in}{1.493038in}}%
\pgfpathlineto{\pgfqpoint{4.000796in}{1.479760in}}%
\pgfpathlineto{\pgfqpoint{3.992865in}{1.466552in}}%
\pgfpathlineto{\pgfqpoint{3.984928in}{1.453419in}}%
\pgfpathlineto{\pgfqpoint{3.976987in}{1.440366in}}%
\pgfpathclose%
\pgfusepath{fill}%
\end{pgfscope}%
\begin{pgfscope}%
\pgfpathrectangle{\pgfqpoint{1.254980in}{0.150000in}}{\pgfqpoint{5.490039in}{5.490039in}}%
\pgfusepath{clip}%
\pgfsetbuttcap%
\pgfsetroundjoin%
\definecolor{currentfill}{rgb}{0.269944,0.014625,0.341379}%
\pgfsetfillcolor{currentfill}%
\pgfsetfillopacity{0.700000}%
\pgfsetlinewidth{0.000000pt}%
\definecolor{currentstroke}{rgb}{0.000000,0.000000,0.000000}%
\pgfsetstrokecolor{currentstroke}%
\pgfsetdash{}{0pt}%
\pgfpathmoveto{\pgfqpoint{3.402690in}{1.258717in}}%
\pgfpathlineto{\pgfqpoint{3.416272in}{1.253238in}}%
\pgfpathlineto{\pgfqpoint{3.429858in}{1.247923in}}%
\pgfpathlineto{\pgfqpoint{3.443447in}{1.242770in}}%
\pgfpathlineto{\pgfqpoint{3.457040in}{1.237779in}}%
\pgfpathlineto{\pgfqpoint{3.465241in}{1.242929in}}%
\pgfpathlineto{\pgfqpoint{3.473431in}{1.248348in}}%
\pgfpathlineto{\pgfqpoint{3.481611in}{1.254028in}}%
\pgfpathlineto{\pgfqpoint{3.489780in}{1.259962in}}%
\pgfpathlineto{\pgfqpoint{3.476213in}{1.264255in}}%
\pgfpathlineto{\pgfqpoint{3.462649in}{1.268710in}}%
\pgfpathlineto{\pgfqpoint{3.449090in}{1.273328in}}%
\pgfpathlineto{\pgfqpoint{3.435534in}{1.278110in}}%
\pgfpathlineto{\pgfqpoint{3.427340in}{1.272862in}}%
\pgfpathlineto{\pgfqpoint{3.419134in}{1.267876in}}%
\pgfpathlineto{\pgfqpoint{3.410918in}{1.263159in}}%
\pgfpathlineto{\pgfqpoint{3.402690in}{1.258717in}}%
\pgfpathclose%
\pgfusepath{fill}%
\end{pgfscope}%
\begin{pgfscope}%
\pgfpathrectangle{\pgfqpoint{1.254980in}{0.150000in}}{\pgfqpoint{5.490039in}{5.490039in}}%
\pgfusepath{clip}%
\pgfsetbuttcap%
\pgfsetroundjoin%
\definecolor{currentfill}{rgb}{0.265145,0.232956,0.516599}%
\pgfsetfillcolor{currentfill}%
\pgfsetfillopacity{0.700000}%
\pgfsetlinewidth{0.000000pt}%
\definecolor{currentstroke}{rgb}{0.000000,0.000000,0.000000}%
\pgfsetstrokecolor{currentstroke}%
\pgfsetdash{}{0pt}%
\pgfpathmoveto{\pgfqpoint{4.181645in}{1.639656in}}%
\pgfpathlineto{\pgfqpoint{4.195406in}{1.645170in}}%
\pgfpathlineto{\pgfqpoint{4.209179in}{1.650841in}}%
\pgfpathlineto{\pgfqpoint{4.222964in}{1.656669in}}%
\pgfpathlineto{\pgfqpoint{4.236762in}{1.662654in}}%
\pgfpathlineto{\pgfqpoint{4.244648in}{1.677739in}}%
\pgfpathlineto{\pgfqpoint{4.252531in}{1.692826in}}%
\pgfpathlineto{\pgfqpoint{4.260409in}{1.707912in}}%
\pgfpathlineto{\pgfqpoint{4.268284in}{1.722992in}}%
\pgfpathlineto{\pgfqpoint{4.254485in}{1.716552in}}%
\pgfpathlineto{\pgfqpoint{4.240698in}{1.710268in}}%
\pgfpathlineto{\pgfqpoint{4.226924in}{1.704142in}}%
\pgfpathlineto{\pgfqpoint{4.213163in}{1.698174in}}%
\pgfpathlineto{\pgfqpoint{4.205290in}{1.683538in}}%
\pgfpathlineto{\pgfqpoint{4.197412in}{1.668904in}}%
\pgfpathlineto{\pgfqpoint{4.189531in}{1.654275in}}%
\pgfpathlineto{\pgfqpoint{4.181645in}{1.639656in}}%
\pgfpathclose%
\pgfusepath{fill}%
\end{pgfscope}%
\begin{pgfscope}%
\pgfpathrectangle{\pgfqpoint{1.254980in}{0.150000in}}{\pgfqpoint{5.490039in}{5.490039in}}%
\pgfusepath{clip}%
\pgfsetbuttcap%
\pgfsetroundjoin%
\definecolor{currentfill}{rgb}{0.565498,0.842430,0.262877}%
\pgfsetfillcolor{currentfill}%
\pgfsetfillopacity{0.700000}%
\pgfsetlinewidth{0.000000pt}%
\definecolor{currentstroke}{rgb}{0.000000,0.000000,0.000000}%
\pgfsetstrokecolor{currentstroke}%
\pgfsetdash{}{0pt}%
\pgfpathmoveto{\pgfqpoint{5.623277in}{3.358535in}}%
\pgfpathlineto{\pgfqpoint{5.637985in}{3.375593in}}%
\pgfpathlineto{\pgfqpoint{5.652715in}{3.392818in}}%
\pgfpathlineto{\pgfqpoint{5.667468in}{3.410212in}}%
\pgfpathlineto{\pgfqpoint{5.674676in}{3.414197in}}%
\pgfpathlineto{\pgfqpoint{5.681872in}{3.418008in}}%
\pgfpathlineto{\pgfqpoint{5.689056in}{3.421648in}}%
\pgfpathlineto{\pgfqpoint{5.696229in}{3.425120in}}%
\pgfpathlineto{\pgfqpoint{5.681491in}{3.407972in}}%
\pgfpathlineto{\pgfqpoint{5.666776in}{3.390991in}}%
\pgfpathlineto{\pgfqpoint{5.652084in}{3.374178in}}%
\pgfpathlineto{\pgfqpoint{5.644899in}{3.370513in}}%
\pgfpathlineto{\pgfqpoint{5.637703in}{3.366687in}}%
\pgfpathlineto{\pgfqpoint{5.630496in}{3.362695in}}%
\pgfpathlineto{\pgfqpoint{5.623277in}{3.358535in}}%
\pgfpathclose%
\pgfusepath{fill}%
\end{pgfscope}%
\begin{pgfscope}%
\pgfpathrectangle{\pgfqpoint{1.254980in}{0.150000in}}{\pgfqpoint{5.490039in}{5.490039in}}%
\pgfusepath{clip}%
\pgfsetbuttcap%
\pgfsetroundjoin%
\definecolor{currentfill}{rgb}{0.269944,0.014625,0.341379}%
\pgfsetfillcolor{currentfill}%
\pgfsetfillopacity{0.700000}%
\pgfsetlinewidth{0.000000pt}%
\definecolor{currentstroke}{rgb}{0.000000,0.000000,0.000000}%
\pgfsetstrokecolor{currentstroke}%
\pgfsetdash{}{0pt}%
\pgfpathmoveto{\pgfqpoint{3.544097in}{1.244403in}}%
\pgfpathlineto{\pgfqpoint{3.557689in}{1.240915in}}%
\pgfpathlineto{\pgfqpoint{3.571285in}{1.237587in}}%
\pgfpathlineto{\pgfqpoint{3.584887in}{1.234418in}}%
\pgfpathlineto{\pgfqpoint{3.598494in}{1.231408in}}%
\pgfpathlineto{\pgfqpoint{3.606611in}{1.238948in}}%
\pgfpathlineto{\pgfqpoint{3.614719in}{1.246715in}}%
\pgfpathlineto{\pgfqpoint{3.622819in}{1.254701in}}%
\pgfpathlineto{\pgfqpoint{3.630910in}{1.262900in}}%
\pgfpathlineto{\pgfqpoint{3.617322in}{1.265241in}}%
\pgfpathlineto{\pgfqpoint{3.603740in}{1.267741in}}%
\pgfpathlineto{\pgfqpoint{3.590163in}{1.270401in}}%
\pgfpathlineto{\pgfqpoint{3.576592in}{1.273220in}}%
\pgfpathlineto{\pgfqpoint{3.568482in}{1.265679in}}%
\pgfpathlineto{\pgfqpoint{3.560363in}{1.258358in}}%
\pgfpathlineto{\pgfqpoint{3.552235in}{1.251264in}}%
\pgfpathlineto{\pgfqpoint{3.544097in}{1.244403in}}%
\pgfpathclose%
\pgfusepath{fill}%
\end{pgfscope}%
\begin{pgfscope}%
\pgfpathrectangle{\pgfqpoint{1.254980in}{0.150000in}}{\pgfqpoint{5.490039in}{5.490039in}}%
\pgfusepath{clip}%
\pgfsetbuttcap%
\pgfsetroundjoin%
\definecolor{currentfill}{rgb}{0.226397,0.728888,0.462789}%
\pgfsetfillcolor{currentfill}%
\pgfsetfillopacity{0.700000}%
\pgfsetlinewidth{0.000000pt}%
\definecolor{currentstroke}{rgb}{0.000000,0.000000,0.000000}%
\pgfsetstrokecolor{currentstroke}%
\pgfsetdash{}{0pt}%
\pgfpathmoveto{\pgfqpoint{5.213509in}{2.939659in}}%
\pgfpathlineto{\pgfqpoint{5.227915in}{2.954825in}}%
\pgfpathlineto{\pgfqpoint{5.242343in}{2.970157in}}%
\pgfpathlineto{\pgfqpoint{5.256791in}{2.985655in}}%
\pgfpathlineto{\pgfqpoint{5.271260in}{3.001319in}}%
\pgfpathlineto{\pgfqpoint{5.278761in}{3.010264in}}%
\pgfpathlineto{\pgfqpoint{5.286252in}{3.019022in}}%
\pgfpathlineto{\pgfqpoint{5.293734in}{3.027592in}}%
\pgfpathlineto{\pgfqpoint{5.301205in}{3.035977in}}%
\pgfpathlineto{\pgfqpoint{5.286740in}{3.020362in}}%
\pgfpathlineto{\pgfqpoint{5.272296in}{3.004913in}}%
\pgfpathlineto{\pgfqpoint{5.257872in}{2.989629in}}%
\pgfpathlineto{\pgfqpoint{5.243470in}{2.974512in}}%
\pgfpathlineto{\pgfqpoint{5.235994in}{2.966066in}}%
\pgfpathlineto{\pgfqpoint{5.228508in}{2.957443in}}%
\pgfpathlineto{\pgfqpoint{5.221013in}{2.948641in}}%
\pgfpathlineto{\pgfqpoint{5.213509in}{2.939659in}}%
\pgfpathclose%
\pgfusepath{fill}%
\end{pgfscope}%
\begin{pgfscope}%
\pgfpathrectangle{\pgfqpoint{1.254980in}{0.150000in}}{\pgfqpoint{5.490039in}{5.490039in}}%
\pgfusepath{clip}%
\pgfsetbuttcap%
\pgfsetroundjoin%
\definecolor{currentfill}{rgb}{0.279574,0.170599,0.479997}%
\pgfsetfillcolor{currentfill}%
\pgfsetfillopacity{0.700000}%
\pgfsetlinewidth{0.000000pt}%
\definecolor{currentstroke}{rgb}{0.000000,0.000000,0.000000}%
\pgfsetstrokecolor{currentstroke}%
\pgfsetdash{}{0pt}%
\pgfpathmoveto{\pgfqpoint{4.063484in}{1.507061in}}%
\pgfpathlineto{\pgfqpoint{4.077201in}{1.510957in}}%
\pgfpathlineto{\pgfqpoint{4.090928in}{1.515010in}}%
\pgfpathlineto{\pgfqpoint{4.104667in}{1.519219in}}%
\pgfpathlineto{\pgfqpoint{4.118417in}{1.523585in}}%
\pgfpathlineto{\pgfqpoint{4.126335in}{1.537965in}}%
\pgfpathlineto{\pgfqpoint{4.134248in}{1.552391in}}%
\pgfpathlineto{\pgfqpoint{4.142158in}{1.566859in}}%
\pgfpathlineto{\pgfqpoint{4.150063in}{1.581364in}}%
\pgfpathlineto{\pgfqpoint{4.136314in}{1.576489in}}%
\pgfpathlineto{\pgfqpoint{4.122577in}{1.571770in}}%
\pgfpathlineto{\pgfqpoint{4.108851in}{1.567208in}}%
\pgfpathlineto{\pgfqpoint{4.095136in}{1.562803in}}%
\pgfpathlineto{\pgfqpoint{4.087230in}{1.548797in}}%
\pgfpathlineto{\pgfqpoint{4.079319in}{1.534835in}}%
\pgfpathlineto{\pgfqpoint{4.071404in}{1.520921in}}%
\pgfpathlineto{\pgfqpoint{4.063484in}{1.507061in}}%
\pgfpathclose%
\pgfusepath{fill}%
\end{pgfscope}%
\begin{pgfscope}%
\pgfpathrectangle{\pgfqpoint{1.254980in}{0.150000in}}{\pgfqpoint{5.490039in}{5.490039in}}%
\pgfusepath{clip}%
\pgfsetbuttcap%
\pgfsetroundjoin%
\definecolor{currentfill}{rgb}{0.395174,0.797475,0.367757}%
\pgfsetfillcolor{currentfill}%
\pgfsetfillopacity{0.700000}%
\pgfsetlinewidth{0.000000pt}%
\definecolor{currentstroke}{rgb}{0.000000,0.000000,0.000000}%
\pgfsetstrokecolor{currentstroke}%
\pgfsetdash{}{0pt}%
\pgfpathmoveto{\pgfqpoint{5.418647in}{3.159864in}}%
\pgfpathlineto{\pgfqpoint{5.433207in}{3.176119in}}%
\pgfpathlineto{\pgfqpoint{5.447789in}{3.192541in}}%
\pgfpathlineto{\pgfqpoint{5.462394in}{3.209130in}}%
\pgfpathlineto{\pgfqpoint{5.477021in}{3.225887in}}%
\pgfpathlineto{\pgfqpoint{5.484386in}{3.232390in}}%
\pgfpathlineto{\pgfqpoint{5.491741in}{3.238704in}}%
\pgfpathlineto{\pgfqpoint{5.499084in}{3.244833in}}%
\pgfpathlineto{\pgfqpoint{5.506416in}{3.250778in}}%
\pgfpathlineto{\pgfqpoint{5.491798in}{3.234168in}}%
\pgfpathlineto{\pgfqpoint{5.477203in}{3.217725in}}%
\pgfpathlineto{\pgfqpoint{5.462630in}{3.201449in}}%
\pgfpathlineto{\pgfqpoint{5.448078in}{3.185340in}}%
\pgfpathlineto{\pgfqpoint{5.440736in}{3.179237in}}%
\pgfpathlineto{\pgfqpoint{5.433384in}{3.172958in}}%
\pgfpathlineto{\pgfqpoint{5.426020in}{3.166501in}}%
\pgfpathlineto{\pgfqpoint{5.418647in}{3.159864in}}%
\pgfpathclose%
\pgfusepath{fill}%
\end{pgfscope}%
\begin{pgfscope}%
\pgfpathrectangle{\pgfqpoint{1.254980in}{0.150000in}}{\pgfqpoint{5.490039in}{5.490039in}}%
\pgfusepath{clip}%
\pgfsetbuttcap%
\pgfsetroundjoin%
\definecolor{currentfill}{rgb}{0.162016,0.687316,0.499129}%
\pgfsetfillcolor{currentfill}%
\pgfsetfillopacity{0.700000}%
\pgfsetlinewidth{0.000000pt}%
\definecolor{currentstroke}{rgb}{0.000000,0.000000,0.000000}%
\pgfsetstrokecolor{currentstroke}%
\pgfsetdash{}{0pt}%
\pgfpathmoveto{\pgfqpoint{5.095712in}{2.802248in}}%
\pgfpathlineto{\pgfqpoint{5.110041in}{2.816753in}}%
\pgfpathlineto{\pgfqpoint{5.124390in}{2.831424in}}%
\pgfpathlineto{\pgfqpoint{5.138759in}{2.846259in}}%
\pgfpathlineto{\pgfqpoint{5.153148in}{2.861261in}}%
\pgfpathlineto{\pgfqpoint{5.160724in}{2.871701in}}%
\pgfpathlineto{\pgfqpoint{5.168292in}{2.881958in}}%
\pgfpathlineto{\pgfqpoint{5.175851in}{2.892031in}}%
\pgfpathlineto{\pgfqpoint{5.183400in}{2.901922in}}%
\pgfpathlineto{\pgfqpoint{5.169012in}{2.886905in}}%
\pgfpathlineto{\pgfqpoint{5.154644in}{2.872054in}}%
\pgfpathlineto{\pgfqpoint{5.140296in}{2.857369in}}%
\pgfpathlineto{\pgfqpoint{5.125969in}{2.842848in}}%
\pgfpathlineto{\pgfqpoint{5.118417in}{2.832962in}}%
\pgfpathlineto{\pgfqpoint{5.110857in}{2.822899in}}%
\pgfpathlineto{\pgfqpoint{5.103289in}{2.812662in}}%
\pgfpathlineto{\pgfqpoint{5.095712in}{2.802248in}}%
\pgfpathclose%
\pgfusepath{fill}%
\end{pgfscope}%
\begin{pgfscope}%
\pgfpathrectangle{\pgfqpoint{1.254980in}{0.150000in}}{\pgfqpoint{5.490039in}{5.490039in}}%
\pgfusepath{clip}%
\pgfsetbuttcap%
\pgfsetroundjoin%
\definecolor{currentfill}{rgb}{0.277941,0.056324,0.381191}%
\pgfsetfillcolor{currentfill}%
\pgfsetfillopacity{0.700000}%
\pgfsetlinewidth{0.000000pt}%
\definecolor{currentstroke}{rgb}{0.000000,0.000000,0.000000}%
\pgfsetstrokecolor{currentstroke}%
\pgfsetdash{}{0pt}%
\pgfpathmoveto{\pgfqpoint{3.772007in}{1.289795in}}%
\pgfpathlineto{\pgfqpoint{3.785641in}{1.289515in}}%
\pgfpathlineto{\pgfqpoint{3.799283in}{1.289393in}}%
\pgfpathlineto{\pgfqpoint{3.812932in}{1.289426in}}%
\pgfpathlineto{\pgfqpoint{3.826590in}{1.289616in}}%
\pgfpathlineto{\pgfqpoint{3.834603in}{1.300657in}}%
\pgfpathlineto{\pgfqpoint{3.842610in}{1.311851in}}%
\pgfpathlineto{\pgfqpoint{3.850611in}{1.323194in}}%
\pgfpathlineto{\pgfqpoint{3.858606in}{1.334678in}}%
\pgfpathlineto{\pgfqpoint{3.844959in}{1.333873in}}%
\pgfpathlineto{\pgfqpoint{3.831320in}{1.333224in}}%
\pgfpathlineto{\pgfqpoint{3.817689in}{1.332732in}}%
\pgfpathlineto{\pgfqpoint{3.804067in}{1.332397in}}%
\pgfpathlineto{\pgfqpoint{3.796061in}{1.321517in}}%
\pgfpathlineto{\pgfqpoint{3.788050in}{1.310786in}}%
\pgfpathlineto{\pgfqpoint{3.780032in}{1.300210in}}%
\pgfpathlineto{\pgfqpoint{3.772007in}{1.289795in}}%
\pgfpathclose%
\pgfusepath{fill}%
\end{pgfscope}%
\begin{pgfscope}%
\pgfpathrectangle{\pgfqpoint{1.254980in}{0.150000in}}{\pgfqpoint{5.490039in}{5.490039in}}%
\pgfusepath{clip}%
\pgfsetbuttcap%
\pgfsetroundjoin%
\definecolor{currentfill}{rgb}{0.220057,0.343307,0.549413}%
\pgfsetfillcolor{currentfill}%
\pgfsetfillopacity{0.700000}%
\pgfsetlinewidth{0.000000pt}%
\definecolor{currentstroke}{rgb}{0.000000,0.000000,0.000000}%
\pgfsetstrokecolor{currentstroke}%
\pgfsetdash{}{0pt}%
\pgfpathmoveto{\pgfqpoint{4.386493in}{1.873603in}}%
\pgfpathlineto{\pgfqpoint{4.400366in}{1.881663in}}%
\pgfpathlineto{\pgfqpoint{4.414253in}{1.889881in}}%
\pgfpathlineto{\pgfqpoint{4.428154in}{1.898258in}}%
\pgfpathlineto{\pgfqpoint{4.442069in}{1.906793in}}%
\pgfpathlineto{\pgfqpoint{4.449916in}{1.922412in}}%
\pgfpathlineto{\pgfqpoint{4.457758in}{1.937974in}}%
\pgfpathlineto{\pgfqpoint{4.465597in}{1.953474in}}%
\pgfpathlineto{\pgfqpoint{4.473431in}{1.968909in}}%
\pgfpathlineto{\pgfqpoint{4.459511in}{1.959999in}}%
\pgfpathlineto{\pgfqpoint{4.445605in}{1.951249in}}%
\pgfpathlineto{\pgfqpoint{4.431714in}{1.942657in}}%
\pgfpathlineto{\pgfqpoint{4.417838in}{1.934224in}}%
\pgfpathlineto{\pgfqpoint{4.410008in}{1.919152in}}%
\pgfpathlineto{\pgfqpoint{4.402174in}{1.904022in}}%
\pgfpathlineto{\pgfqpoint{4.394335in}{1.888838in}}%
\pgfpathlineto{\pgfqpoint{4.386493in}{1.873603in}}%
\pgfpathclose%
\pgfusepath{fill}%
\end{pgfscope}%
\begin{pgfscope}%
\pgfpathrectangle{\pgfqpoint{1.254980in}{0.150000in}}{\pgfqpoint{5.490039in}{5.490039in}}%
\pgfusepath{clip}%
\pgfsetbuttcap%
\pgfsetroundjoin%
\definecolor{currentfill}{rgb}{0.281446,0.084320,0.407414}%
\pgfsetfillcolor{currentfill}%
\pgfsetfillopacity{0.700000}%
\pgfsetlinewidth{0.000000pt}%
\definecolor{currentstroke}{rgb}{0.000000,0.000000,0.000000}%
\pgfsetstrokecolor{currentstroke}%
\pgfsetdash{}{0pt}%
\pgfpathmoveto{\pgfqpoint{3.858606in}{1.334678in}}%
\pgfpathlineto{\pgfqpoint{3.872261in}{1.335640in}}%
\pgfpathlineto{\pgfqpoint{3.885926in}{1.336758in}}%
\pgfpathlineto{\pgfqpoint{3.899599in}{1.338031in}}%
\pgfpathlineto{\pgfqpoint{3.913281in}{1.339460in}}%
\pgfpathlineto{\pgfqpoint{3.921262in}{1.351680in}}%
\pgfpathlineto{\pgfqpoint{3.929238in}{1.364024in}}%
\pgfpathlineto{\pgfqpoint{3.937209in}{1.376485in}}%
\pgfpathlineto{\pgfqpoint{3.945174in}{1.389059in}}%
\pgfpathlineto{\pgfqpoint{3.931499in}{1.387040in}}%
\pgfpathlineto{\pgfqpoint{3.917834in}{1.385178in}}%
\pgfpathlineto{\pgfqpoint{3.904177in}{1.383472in}}%
\pgfpathlineto{\pgfqpoint{3.890530in}{1.381922in}}%
\pgfpathlineto{\pgfqpoint{3.882557in}{1.369927in}}%
\pgfpathlineto{\pgfqpoint{3.874579in}{1.358051in}}%
\pgfpathlineto{\pgfqpoint{3.866595in}{1.346299in}}%
\pgfpathlineto{\pgfqpoint{3.858606in}{1.334678in}}%
\pgfpathclose%
\pgfusepath{fill}%
\end{pgfscope}%
\begin{pgfscope}%
\pgfpathrectangle{\pgfqpoint{1.254980in}{0.150000in}}{\pgfqpoint{5.490039in}{5.490039in}}%
\pgfusepath{clip}%
\pgfsetbuttcap%
\pgfsetroundjoin%
\definecolor{currentfill}{rgb}{0.190631,0.407061,0.556089}%
\pgfsetfillcolor{currentfill}%
\pgfsetfillopacity{0.700000}%
\pgfsetlinewidth{0.000000pt}%
\definecolor{currentstroke}{rgb}{0.000000,0.000000,0.000000}%
\pgfsetstrokecolor{currentstroke}%
\pgfsetdash{}{0pt}%
\pgfpathmoveto{\pgfqpoint{4.504725in}{2.029938in}}%
\pgfpathlineto{\pgfqpoint{4.518665in}{2.039353in}}%
\pgfpathlineto{\pgfqpoint{4.532621in}{2.048928in}}%
\pgfpathlineto{\pgfqpoint{4.546592in}{2.058662in}}%
\pgfpathlineto{\pgfqpoint{4.560578in}{2.068556in}}%
\pgfpathlineto{\pgfqpoint{4.568396in}{2.083949in}}%
\pgfpathlineto{\pgfqpoint{4.576209in}{2.099251in}}%
\pgfpathlineto{\pgfqpoint{4.584017in}{2.114459in}}%
\pgfpathlineto{\pgfqpoint{4.591821in}{2.129572in}}%
\pgfpathlineto{\pgfqpoint{4.577829in}{2.119359in}}%
\pgfpathlineto{\pgfqpoint{4.563853in}{2.109308in}}%
\pgfpathlineto{\pgfqpoint{4.549893in}{2.099416in}}%
\pgfpathlineto{\pgfqpoint{4.535948in}{2.089684in}}%
\pgfpathlineto{\pgfqpoint{4.528149in}{2.074878in}}%
\pgfpathlineto{\pgfqpoint{4.520345in}{2.059983in}}%
\pgfpathlineto{\pgfqpoint{4.512537in}{2.045003in}}%
\pgfpathlineto{\pgfqpoint{4.504725in}{2.029938in}}%
\pgfpathclose%
\pgfusepath{fill}%
\end{pgfscope}%
\begin{pgfscope}%
\pgfpathrectangle{\pgfqpoint{1.254980in}{0.150000in}}{\pgfqpoint{5.490039in}{5.490039in}}%
\pgfusepath{clip}%
\pgfsetbuttcap%
\pgfsetroundjoin%
\definecolor{currentfill}{rgb}{0.248629,0.278775,0.534556}%
\pgfsetfillcolor{currentfill}%
\pgfsetfillopacity{0.700000}%
\pgfsetlinewidth{0.000000pt}%
\definecolor{currentstroke}{rgb}{0.000000,0.000000,0.000000}%
\pgfsetstrokecolor{currentstroke}%
\pgfsetdash{}{0pt}%
\pgfpathmoveto{\pgfqpoint{4.268284in}{1.722992in}}%
\pgfpathlineto{\pgfqpoint{4.282096in}{1.729590in}}%
\pgfpathlineto{\pgfqpoint{4.295922in}{1.736346in}}%
\pgfpathlineto{\pgfqpoint{4.309760in}{1.743259in}}%
\pgfpathlineto{\pgfqpoint{4.323612in}{1.750329in}}%
\pgfpathlineto{\pgfqpoint{4.331486in}{1.765839in}}%
\pgfpathlineto{\pgfqpoint{4.339356in}{1.781328in}}%
\pgfpathlineto{\pgfqpoint{4.347222in}{1.796792in}}%
\pgfpathlineto{\pgfqpoint{4.355084in}{1.812227in}}%
\pgfpathlineto{\pgfqpoint{4.341229in}{1.804727in}}%
\pgfpathlineto{\pgfqpoint{4.327387in}{1.797385in}}%
\pgfpathlineto{\pgfqpoint{4.313559in}{1.790201in}}%
\pgfpathlineto{\pgfqpoint{4.299745in}{1.783174in}}%
\pgfpathlineto{\pgfqpoint{4.291886in}{1.768158in}}%
\pgfpathlineto{\pgfqpoint{4.284022in}{1.753119in}}%
\pgfpathlineto{\pgfqpoint{4.276155in}{1.738063in}}%
\pgfpathlineto{\pgfqpoint{4.268284in}{1.722992in}}%
\pgfpathclose%
\pgfusepath{fill}%
\end{pgfscope}%
\begin{pgfscope}%
\pgfpathrectangle{\pgfqpoint{1.254980in}{0.150000in}}{\pgfqpoint{5.490039in}{5.490039in}}%
\pgfusepath{clip}%
\pgfsetbuttcap%
\pgfsetroundjoin%
\definecolor{currentfill}{rgb}{0.269944,0.014625,0.341379}%
\pgfsetfillcolor{currentfill}%
\pgfsetfillopacity{0.700000}%
\pgfsetlinewidth{0.000000pt}%
\definecolor{currentstroke}{rgb}{0.000000,0.000000,0.000000}%
\pgfsetstrokecolor{currentstroke}%
\pgfsetdash{}{0pt}%
\pgfpathmoveto{\pgfqpoint{3.457040in}{1.237779in}}%
\pgfpathlineto{\pgfqpoint{3.470636in}{1.232950in}}%
\pgfpathlineto{\pgfqpoint{3.484237in}{1.228283in}}%
\pgfpathlineto{\pgfqpoint{3.497842in}{1.223776in}}%
\pgfpathlineto{\pgfqpoint{3.511452in}{1.219430in}}%
\pgfpathlineto{\pgfqpoint{3.519628in}{1.225289in}}%
\pgfpathlineto{\pgfqpoint{3.527794in}{1.231409in}}%
\pgfpathlineto{\pgfqpoint{3.535951in}{1.237783in}}%
\pgfpathlineto{\pgfqpoint{3.544097in}{1.244403in}}%
\pgfpathlineto{\pgfqpoint{3.530511in}{1.248052in}}%
\pgfpathlineto{\pgfqpoint{3.516930in}{1.251861in}}%
\pgfpathlineto{\pgfqpoint{3.503353in}{1.255831in}}%
\pgfpathlineto{\pgfqpoint{3.489780in}{1.259962in}}%
\pgfpathlineto{\pgfqpoint{3.481611in}{1.254028in}}%
\pgfpathlineto{\pgfqpoint{3.473431in}{1.248348in}}%
\pgfpathlineto{\pgfqpoint{3.465241in}{1.242929in}}%
\pgfpathlineto{\pgfqpoint{3.457040in}{1.237779in}}%
\pgfpathclose%
\pgfusepath{fill}%
\end{pgfscope}%
\begin{pgfscope}%
\pgfpathrectangle{\pgfqpoint{1.254980in}{0.150000in}}{\pgfqpoint{5.490039in}{5.490039in}}%
\pgfusepath{clip}%
\pgfsetbuttcap%
\pgfsetroundjoin%
\definecolor{currentfill}{rgb}{0.273809,0.031497,0.358853}%
\pgfsetfillcolor{currentfill}%
\pgfsetfillopacity{0.700000}%
\pgfsetlinewidth{0.000000pt}%
\definecolor{currentstroke}{rgb}{0.000000,0.000000,0.000000}%
\pgfsetstrokecolor{currentstroke}%
\pgfsetdash{}{0pt}%
\pgfpathmoveto{\pgfqpoint{3.685324in}{1.255123in}}%
\pgfpathlineto{\pgfqpoint{3.698943in}{1.253573in}}%
\pgfpathlineto{\pgfqpoint{3.712570in}{1.252181in}}%
\pgfpathlineto{\pgfqpoint{3.726203in}{1.250946in}}%
\pgfpathlineto{\pgfqpoint{3.739842in}{1.249867in}}%
\pgfpathlineto{\pgfqpoint{3.747894in}{1.259577in}}%
\pgfpathlineto{\pgfqpoint{3.755939in}{1.269472in}}%
\pgfpathlineto{\pgfqpoint{3.763976in}{1.279547in}}%
\pgfpathlineto{\pgfqpoint{3.772007in}{1.289795in}}%
\pgfpathlineto{\pgfqpoint{3.758381in}{1.290231in}}%
\pgfpathlineto{\pgfqpoint{3.744762in}{1.290824in}}%
\pgfpathlineto{\pgfqpoint{3.731150in}{1.291574in}}%
\pgfpathlineto{\pgfqpoint{3.717546in}{1.292482in}}%
\pgfpathlineto{\pgfqpoint{3.709501in}{1.282866in}}%
\pgfpathlineto{\pgfqpoint{3.701450in}{1.273429in}}%
\pgfpathlineto{\pgfqpoint{3.693390in}{1.264180in}}%
\pgfpathlineto{\pgfqpoint{3.685324in}{1.255123in}}%
\pgfpathclose%
\pgfusepath{fill}%
\end{pgfscope}%
\begin{pgfscope}%
\pgfpathrectangle{\pgfqpoint{1.254980in}{0.150000in}}{\pgfqpoint{5.490039in}{5.490039in}}%
\pgfusepath{clip}%
\pgfsetbuttcap%
\pgfsetroundjoin%
\definecolor{currentfill}{rgb}{0.123444,0.636809,0.528763}%
\pgfsetfillcolor{currentfill}%
\pgfsetfillopacity{0.700000}%
\pgfsetlinewidth{0.000000pt}%
\definecolor{currentstroke}{rgb}{0.000000,0.000000,0.000000}%
\pgfsetstrokecolor{currentstroke}%
\pgfsetdash{}{0pt}%
\pgfpathmoveto{\pgfqpoint{4.977688in}{2.656737in}}%
\pgfpathlineto{\pgfqpoint{4.991937in}{2.670459in}}%
\pgfpathlineto{\pgfqpoint{5.006206in}{2.684345in}}%
\pgfpathlineto{\pgfqpoint{5.020494in}{2.698395in}}%
\pgfpathlineto{\pgfqpoint{5.034801in}{2.712610in}}%
\pgfpathlineto{\pgfqpoint{5.042443in}{2.724429in}}%
\pgfpathlineto{\pgfqpoint{5.050077in}{2.736072in}}%
\pgfpathlineto{\pgfqpoint{5.057703in}{2.747541in}}%
\pgfpathlineto{\pgfqpoint{5.065321in}{2.758834in}}%
\pgfpathlineto{\pgfqpoint{5.051012in}{2.744541in}}%
\pgfpathlineto{\pgfqpoint{5.036723in}{2.730413in}}%
\pgfpathlineto{\pgfqpoint{5.022454in}{2.716449in}}%
\pgfpathlineto{\pgfqpoint{5.008203in}{2.702650in}}%
\pgfpathlineto{\pgfqpoint{5.000586in}{2.691423in}}%
\pgfpathlineto{\pgfqpoint{4.992961in}{2.680028in}}%
\pgfpathlineto{\pgfqpoint{4.985328in}{2.668466in}}%
\pgfpathlineto{\pgfqpoint{4.977688in}{2.656737in}}%
\pgfpathclose%
\pgfusepath{fill}%
\end{pgfscope}%
\begin{pgfscope}%
\pgfpathrectangle{\pgfqpoint{1.254980in}{0.150000in}}{\pgfqpoint{5.490039in}{5.490039in}}%
\pgfusepath{clip}%
\pgfsetbuttcap%
\pgfsetroundjoin%
\definecolor{currentfill}{rgb}{0.165117,0.467423,0.558141}%
\pgfsetfillcolor{currentfill}%
\pgfsetfillopacity{0.700000}%
\pgfsetlinewidth{0.000000pt}%
\definecolor{currentstroke}{rgb}{0.000000,0.000000,0.000000}%
\pgfsetstrokecolor{currentstroke}%
\pgfsetdash{}{0pt}%
\pgfpathmoveto{\pgfqpoint{4.622987in}{2.189007in}}%
\pgfpathlineto{\pgfqpoint{4.637001in}{2.199668in}}%
\pgfpathlineto{\pgfqpoint{4.651030in}{2.210491in}}%
\pgfpathlineto{\pgfqpoint{4.665076in}{2.221474in}}%
\pgfpathlineto{\pgfqpoint{4.679139in}{2.232618in}}%
\pgfpathlineto{\pgfqpoint{4.686923in}{2.247482in}}%
\pgfpathlineto{\pgfqpoint{4.694703in}{2.262227in}}%
\pgfpathlineto{\pgfqpoint{4.702476in}{2.276851in}}%
\pgfpathlineto{\pgfqpoint{4.710245in}{2.291352in}}%
\pgfpathlineto{\pgfqpoint{4.696177in}{2.279947in}}%
\pgfpathlineto{\pgfqpoint{4.682126in}{2.268704in}}%
\pgfpathlineto{\pgfqpoint{4.668091in}{2.257623in}}%
\pgfpathlineto{\pgfqpoint{4.654073in}{2.246702in}}%
\pgfpathlineto{\pgfqpoint{4.646310in}{2.232450in}}%
\pgfpathlineto{\pgfqpoint{4.638541in}{2.218081in}}%
\pgfpathlineto{\pgfqpoint{4.630767in}{2.203600in}}%
\pgfpathlineto{\pgfqpoint{4.622987in}{2.189007in}}%
\pgfpathclose%
\pgfusepath{fill}%
\end{pgfscope}%
\begin{pgfscope}%
\pgfpathrectangle{\pgfqpoint{1.254980in}{0.150000in}}{\pgfqpoint{5.490039in}{5.490039in}}%
\pgfusepath{clip}%
\pgfsetbuttcap%
\pgfsetroundjoin%
\definecolor{currentfill}{rgb}{0.141935,0.526453,0.555991}%
\pgfsetfillcolor{currentfill}%
\pgfsetfillopacity{0.700000}%
\pgfsetlinewidth{0.000000pt}%
\definecolor{currentstroke}{rgb}{0.000000,0.000000,0.000000}%
\pgfsetstrokecolor{currentstroke}%
\pgfsetdash{}{0pt}%
\pgfpathmoveto{\pgfqpoint{4.741264in}{2.348085in}}%
\pgfpathlineto{\pgfqpoint{4.755354in}{2.359881in}}%
\pgfpathlineto{\pgfqpoint{4.769461in}{2.371840in}}%
\pgfpathlineto{\pgfqpoint{4.783586in}{2.383961in}}%
\pgfpathlineto{\pgfqpoint{4.797729in}{2.396244in}}%
\pgfpathlineto{\pgfqpoint{4.805474in}{2.410312in}}%
\pgfpathlineto{\pgfqpoint{4.813213in}{2.424238in}}%
\pgfpathlineto{\pgfqpoint{4.820946in}{2.438020in}}%
\pgfpathlineto{\pgfqpoint{4.828672in}{2.451656in}}%
\pgfpathlineto{\pgfqpoint{4.814525in}{2.439172in}}%
\pgfpathlineto{\pgfqpoint{4.800396in}{2.426850in}}%
\pgfpathlineto{\pgfqpoint{4.786284in}{2.414691in}}%
\pgfpathlineto{\pgfqpoint{4.772190in}{2.402695in}}%
\pgfpathlineto{\pgfqpoint{4.764467in}{2.389247in}}%
\pgfpathlineto{\pgfqpoint{4.756739in}{2.375662in}}%
\pgfpathlineto{\pgfqpoint{4.749004in}{2.361941in}}%
\pgfpathlineto{\pgfqpoint{4.741264in}{2.348085in}}%
\pgfpathclose%
\pgfusepath{fill}%
\end{pgfscope}%
\begin{pgfscope}%
\pgfpathrectangle{\pgfqpoint{1.254980in}{0.150000in}}{\pgfqpoint{5.490039in}{5.490039in}}%
\pgfusepath{clip}%
\pgfsetbuttcap%
\pgfsetroundjoin%
\definecolor{currentfill}{rgb}{0.123463,0.581687,0.547445}%
\pgfsetfillcolor{currentfill}%
\pgfsetfillopacity{0.700000}%
\pgfsetlinewidth{0.000000pt}%
\definecolor{currentstroke}{rgb}{0.000000,0.000000,0.000000}%
\pgfsetstrokecolor{currentstroke}%
\pgfsetdash{}{0pt}%
\pgfpathmoveto{\pgfqpoint{4.859515in}{2.504720in}}%
\pgfpathlineto{\pgfqpoint{4.873685in}{2.517538in}}%
\pgfpathlineto{\pgfqpoint{4.887872in}{2.530518in}}%
\pgfpathlineto{\pgfqpoint{4.902078in}{2.543663in}}%
\pgfpathlineto{\pgfqpoint{4.916303in}{2.556971in}}%
\pgfpathlineto{\pgfqpoint{4.924001in}{2.570011in}}%
\pgfpathlineto{\pgfqpoint{4.931692in}{2.582891in}}%
\pgfpathlineto{\pgfqpoint{4.939376in}{2.595608in}}%
\pgfpathlineto{\pgfqpoint{4.947053in}{2.608163in}}%
\pgfpathlineto{\pgfqpoint{4.932825in}{2.594715in}}%
\pgfpathlineto{\pgfqpoint{4.918616in}{2.581431in}}%
\pgfpathlineto{\pgfqpoint{4.904425in}{2.568310in}}%
\pgfpathlineto{\pgfqpoint{4.890253in}{2.555353in}}%
\pgfpathlineto{\pgfqpoint{4.882579in}{2.542926in}}%
\pgfpathlineto{\pgfqpoint{4.874898in}{2.530345in}}%
\pgfpathlineto{\pgfqpoint{4.867210in}{2.517609in}}%
\pgfpathlineto{\pgfqpoint{4.859515in}{2.504720in}}%
\pgfpathclose%
\pgfusepath{fill}%
\end{pgfscope}%
\begin{pgfscope}%
\pgfpathrectangle{\pgfqpoint{1.254980in}{0.150000in}}{\pgfqpoint{5.490039in}{5.490039in}}%
\pgfusepath{clip}%
\pgfsetbuttcap%
\pgfsetroundjoin%
\definecolor{currentfill}{rgb}{0.270595,0.214069,0.507052}%
\pgfsetfillcolor{currentfill}%
\pgfsetfillopacity{0.700000}%
\pgfsetlinewidth{0.000000pt}%
\definecolor{currentstroke}{rgb}{0.000000,0.000000,0.000000}%
\pgfsetstrokecolor{currentstroke}%
\pgfsetdash{}{0pt}%
\pgfpathmoveto{\pgfqpoint{4.150063in}{1.581364in}}%
\pgfpathlineto{\pgfqpoint{4.163824in}{1.586395in}}%
\pgfpathlineto{\pgfqpoint{4.177596in}{1.591583in}}%
\pgfpathlineto{\pgfqpoint{4.191381in}{1.596928in}}%
\pgfpathlineto{\pgfqpoint{4.205177in}{1.602428in}}%
\pgfpathlineto{\pgfqpoint{4.213079in}{1.617458in}}%
\pgfpathlineto{\pgfqpoint{4.220977in}{1.632509in}}%
\pgfpathlineto{\pgfqpoint{4.228871in}{1.647576in}}%
\pgfpathlineto{\pgfqpoint{4.236762in}{1.662654in}}%
\pgfpathlineto{\pgfqpoint{4.222964in}{1.656669in}}%
\pgfpathlineto{\pgfqpoint{4.209179in}{1.650841in}}%
\pgfpathlineto{\pgfqpoint{4.195406in}{1.645170in}}%
\pgfpathlineto{\pgfqpoint{4.181645in}{1.639656in}}%
\pgfpathlineto{\pgfqpoint{4.173756in}{1.625051in}}%
\pgfpathlineto{\pgfqpoint{4.165862in}{1.610464in}}%
\pgfpathlineto{\pgfqpoint{4.157965in}{1.595900in}}%
\pgfpathlineto{\pgfqpoint{4.150063in}{1.581364in}}%
\pgfpathclose%
\pgfusepath{fill}%
\end{pgfscope}%
\begin{pgfscope}%
\pgfpathrectangle{\pgfqpoint{1.254980in}{0.150000in}}{\pgfqpoint{5.490039in}{5.490039in}}%
\pgfusepath{clip}%
\pgfsetbuttcap%
\pgfsetroundjoin%
\definecolor{currentfill}{rgb}{0.283197,0.115680,0.436115}%
\pgfsetfillcolor{currentfill}%
\pgfsetfillopacity{0.700000}%
\pgfsetlinewidth{0.000000pt}%
\definecolor{currentstroke}{rgb}{0.000000,0.000000,0.000000}%
\pgfsetstrokecolor{currentstroke}%
\pgfsetdash{}{0pt}%
\pgfpathmoveto{\pgfqpoint{3.945174in}{1.389059in}}%
\pgfpathlineto{\pgfqpoint{3.958858in}{1.391233in}}%
\pgfpathlineto{\pgfqpoint{3.972552in}{1.393563in}}%
\pgfpathlineto{\pgfqpoint{3.986256in}{1.396049in}}%
\pgfpathlineto{\pgfqpoint{3.999969in}{1.398690in}}%
\pgfpathlineto{\pgfqpoint{4.007925in}{1.411944in}}%
\pgfpathlineto{\pgfqpoint{4.015875in}{1.425292in}}%
\pgfpathlineto{\pgfqpoint{4.023821in}{1.438728in}}%
\pgfpathlineto{\pgfqpoint{4.031763in}{1.452248in}}%
\pgfpathlineto{\pgfqpoint{4.018054in}{1.449044in}}%
\pgfpathlineto{\pgfqpoint{4.004355in}{1.445995in}}%
\pgfpathlineto{\pgfqpoint{3.990666in}{1.443102in}}%
\pgfpathlineto{\pgfqpoint{3.976987in}{1.440366in}}%
\pgfpathlineto{\pgfqpoint{3.969041in}{1.427398in}}%
\pgfpathlineto{\pgfqpoint{3.961090in}{1.414521in}}%
\pgfpathlineto{\pgfqpoint{3.953135in}{1.401739in}}%
\pgfpathlineto{\pgfqpoint{3.945174in}{1.389059in}}%
\pgfpathclose%
\pgfusepath{fill}%
\end{pgfscope}%
\begin{pgfscope}%
\pgfpathrectangle{\pgfqpoint{1.254980in}{0.150000in}}{\pgfqpoint{5.490039in}{5.490039in}}%
\pgfusepath{clip}%
\pgfsetbuttcap%
\pgfsetroundjoin%
\definecolor{currentfill}{rgb}{0.271305,0.019942,0.347269}%
\pgfsetfillcolor{currentfill}%
\pgfsetfillopacity{0.700000}%
\pgfsetlinewidth{0.000000pt}%
\definecolor{currentstroke}{rgb}{0.000000,0.000000,0.000000}%
\pgfsetstrokecolor{currentstroke}%
\pgfsetdash{}{0pt}%
\pgfpathmoveto{\pgfqpoint{3.598494in}{1.231408in}}%
\pgfpathlineto{\pgfqpoint{3.612107in}{1.228557in}}%
\pgfpathlineto{\pgfqpoint{3.625725in}{1.225864in}}%
\pgfpathlineto{\pgfqpoint{3.639350in}{1.223330in}}%
\pgfpathlineto{\pgfqpoint{3.652980in}{1.220953in}}%
\pgfpathlineto{\pgfqpoint{3.661078in}{1.229173in}}%
\pgfpathlineto{\pgfqpoint{3.669168in}{1.237613in}}%
\pgfpathlineto{\pgfqpoint{3.677250in}{1.246265in}}%
\pgfpathlineto{\pgfqpoint{3.685324in}{1.255123in}}%
\pgfpathlineto{\pgfqpoint{3.671711in}{1.256830in}}%
\pgfpathlineto{\pgfqpoint{3.658104in}{1.258695in}}%
\pgfpathlineto{\pgfqpoint{3.644504in}{1.260718in}}%
\pgfpathlineto{\pgfqpoint{3.630910in}{1.262900in}}%
\pgfpathlineto{\pgfqpoint{3.622819in}{1.254701in}}%
\pgfpathlineto{\pgfqpoint{3.614719in}{1.246715in}}%
\pgfpathlineto{\pgfqpoint{3.606611in}{1.238948in}}%
\pgfpathlineto{\pgfqpoint{3.598494in}{1.231408in}}%
\pgfpathclose%
\pgfusepath{fill}%
\end{pgfscope}%
\begin{pgfscope}%
\pgfpathrectangle{\pgfqpoint{1.254980in}{0.150000in}}{\pgfqpoint{5.490039in}{5.490039in}}%
\pgfusepath{clip}%
\pgfsetbuttcap%
\pgfsetroundjoin%
\definecolor{currentfill}{rgb}{0.304148,0.764704,0.419943}%
\pgfsetfillcolor{currentfill}%
\pgfsetfillopacity{0.700000}%
\pgfsetlinewidth{0.000000pt}%
\definecolor{currentstroke}{rgb}{0.000000,0.000000,0.000000}%
\pgfsetstrokecolor{currentstroke}%
\pgfsetdash{}{0pt}%
\pgfpathmoveto{\pgfqpoint{5.301205in}{3.035977in}}%
\pgfpathlineto{\pgfqpoint{5.315692in}{3.051759in}}%
\pgfpathlineto{\pgfqpoint{5.330200in}{3.067708in}}%
\pgfpathlineto{\pgfqpoint{5.344730in}{3.083824in}}%
\pgfpathlineto{\pgfqpoint{5.359281in}{3.100107in}}%
\pgfpathlineto{\pgfqpoint{5.366738in}{3.108238in}}%
\pgfpathlineto{\pgfqpoint{5.374184in}{3.116177in}}%
\pgfpathlineto{\pgfqpoint{5.381621in}{3.123926in}}%
\pgfpathlineto{\pgfqpoint{5.389047in}{3.131486in}}%
\pgfpathlineto{\pgfqpoint{5.374501in}{3.115285in}}%
\pgfpathlineto{\pgfqpoint{5.359977in}{3.099251in}}%
\pgfpathlineto{\pgfqpoint{5.345475in}{3.083383in}}%
\pgfpathlineto{\pgfqpoint{5.330994in}{3.067682in}}%
\pgfpathlineto{\pgfqpoint{5.323561in}{3.060029in}}%
\pgfpathlineto{\pgfqpoint{5.316119in}{3.052194in}}%
\pgfpathlineto{\pgfqpoint{5.308667in}{3.044178in}}%
\pgfpathlineto{\pgfqpoint{5.301205in}{3.035977in}}%
\pgfpathclose%
\pgfusepath{fill}%
\end{pgfscope}%
\begin{pgfscope}%
\pgfpathrectangle{\pgfqpoint{1.254980in}{0.150000in}}{\pgfqpoint{5.490039in}{5.490039in}}%
\pgfusepath{clip}%
\pgfsetbuttcap%
\pgfsetroundjoin%
\definecolor{currentfill}{rgb}{0.487026,0.823929,0.312321}%
\pgfsetfillcolor{currentfill}%
\pgfsetfillopacity{0.700000}%
\pgfsetlinewidth{0.000000pt}%
\definecolor{currentstroke}{rgb}{0.000000,0.000000,0.000000}%
\pgfsetstrokecolor{currentstroke}%
\pgfsetdash{}{0pt}%
\pgfpathmoveto{\pgfqpoint{5.506416in}{3.250778in}}%
\pgfpathlineto{\pgfqpoint{5.521056in}{3.267556in}}%
\pgfpathlineto{\pgfqpoint{5.535719in}{3.284502in}}%
\pgfpathlineto{\pgfqpoint{5.550405in}{3.301616in}}%
\pgfpathlineto{\pgfqpoint{5.565114in}{3.318899in}}%
\pgfpathlineto{\pgfqpoint{5.572424in}{3.324494in}}%
\pgfpathlineto{\pgfqpoint{5.579724in}{3.329902in}}%
\pgfpathlineto{\pgfqpoint{5.587011in}{3.335124in}}%
\pgfpathlineto{\pgfqpoint{5.594288in}{3.340163in}}%
\pgfpathlineto{\pgfqpoint{5.579590in}{3.323061in}}%
\pgfpathlineto{\pgfqpoint{5.564916in}{3.306127in}}%
\pgfpathlineto{\pgfqpoint{5.550264in}{3.289361in}}%
\pgfpathlineto{\pgfqpoint{5.535635in}{3.272762in}}%
\pgfpathlineto{\pgfqpoint{5.528347in}{3.267531in}}%
\pgfpathlineto{\pgfqpoint{5.521048in}{3.262125in}}%
\pgfpathlineto{\pgfqpoint{5.513737in}{3.256542in}}%
\pgfpathlineto{\pgfqpoint{5.506416in}{3.250778in}}%
\pgfpathclose%
\pgfusepath{fill}%
\end{pgfscope}%
\begin{pgfscope}%
\pgfpathrectangle{\pgfqpoint{1.254980in}{0.150000in}}{\pgfqpoint{5.490039in}{5.490039in}}%
\pgfusepath{clip}%
\pgfsetbuttcap%
\pgfsetroundjoin%
\definecolor{currentfill}{rgb}{0.281412,0.155834,0.469201}%
\pgfsetfillcolor{currentfill}%
\pgfsetfillopacity{0.700000}%
\pgfsetlinewidth{0.000000pt}%
\definecolor{currentstroke}{rgb}{0.000000,0.000000,0.000000}%
\pgfsetstrokecolor{currentstroke}%
\pgfsetdash{}{0pt}%
\pgfpathmoveto{\pgfqpoint{4.031763in}{1.452248in}}%
\pgfpathlineto{\pgfqpoint{4.045482in}{1.455609in}}%
\pgfpathlineto{\pgfqpoint{4.059212in}{1.459125in}}%
\pgfpathlineto{\pgfqpoint{4.072953in}{1.462797in}}%
\pgfpathlineto{\pgfqpoint{4.086705in}{1.466624in}}%
\pgfpathlineto{\pgfqpoint{4.094639in}{1.480770in}}%
\pgfpathlineto{\pgfqpoint{4.102569in}{1.494982in}}%
\pgfpathlineto{\pgfqpoint{4.110495in}{1.509255in}}%
\pgfpathlineto{\pgfqpoint{4.118417in}{1.523585in}}%
\pgfpathlineto{\pgfqpoint{4.104667in}{1.519219in}}%
\pgfpathlineto{\pgfqpoint{4.090928in}{1.515010in}}%
\pgfpathlineto{\pgfqpoint{4.077201in}{1.510957in}}%
\pgfpathlineto{\pgfqpoint{4.063484in}{1.507061in}}%
\pgfpathlineto{\pgfqpoint{4.055560in}{1.493258in}}%
\pgfpathlineto{\pgfqpoint{4.047632in}{1.479518in}}%
\pgfpathlineto{\pgfqpoint{4.039700in}{1.465847in}}%
\pgfpathlineto{\pgfqpoint{4.031763in}{1.452248in}}%
\pgfpathclose%
\pgfusepath{fill}%
\end{pgfscope}%
\begin{pgfscope}%
\pgfpathrectangle{\pgfqpoint{1.254980in}{0.150000in}}{\pgfqpoint{5.490039in}{5.490039in}}%
\pgfusepath{clip}%
\pgfsetbuttcap%
\pgfsetroundjoin%
\definecolor{currentfill}{rgb}{0.220124,0.725509,0.466226}%
\pgfsetfillcolor{currentfill}%
\pgfsetfillopacity{0.700000}%
\pgfsetlinewidth{0.000000pt}%
\definecolor{currentstroke}{rgb}{0.000000,0.000000,0.000000}%
\pgfsetstrokecolor{currentstroke}%
\pgfsetdash{}{0pt}%
\pgfpathmoveto{\pgfqpoint{5.183400in}{2.901922in}}%
\pgfpathlineto{\pgfqpoint{5.197809in}{2.917104in}}%
\pgfpathlineto{\pgfqpoint{5.212239in}{2.932453in}}%
\pgfpathlineto{\pgfqpoint{5.226690in}{2.947967in}}%
\pgfpathlineto{\pgfqpoint{5.241161in}{2.963649in}}%
\pgfpathlineto{\pgfqpoint{5.248700in}{2.973351in}}%
\pgfpathlineto{\pgfqpoint{5.256230in}{2.982863in}}%
\pgfpathlineto{\pgfqpoint{5.263750in}{2.992186in}}%
\pgfpathlineto{\pgfqpoint{5.271260in}{3.001319in}}%
\pgfpathlineto{\pgfqpoint{5.256791in}{2.985655in}}%
\pgfpathlineto{\pgfqpoint{5.242343in}{2.970157in}}%
\pgfpathlineto{\pgfqpoint{5.227915in}{2.954825in}}%
\pgfpathlineto{\pgfqpoint{5.213509in}{2.939659in}}%
\pgfpathlineto{\pgfqpoint{5.205996in}{2.930497in}}%
\pgfpathlineto{\pgfqpoint{5.198473in}{2.921154in}}%
\pgfpathlineto{\pgfqpoint{5.190941in}{2.911629in}}%
\pgfpathlineto{\pgfqpoint{5.183400in}{2.901922in}}%
\pgfpathclose%
\pgfusepath{fill}%
\end{pgfscope}%
\begin{pgfscope}%
\pgfpathrectangle{\pgfqpoint{1.254980in}{0.150000in}}{\pgfqpoint{5.490039in}{5.490039in}}%
\pgfusepath{clip}%
\pgfsetbuttcap%
\pgfsetroundjoin%
\definecolor{currentfill}{rgb}{0.227802,0.326594,0.546532}%
\pgfsetfillcolor{currentfill}%
\pgfsetfillopacity{0.700000}%
\pgfsetlinewidth{0.000000pt}%
\definecolor{currentstroke}{rgb}{0.000000,0.000000,0.000000}%
\pgfsetstrokecolor{currentstroke}%
\pgfsetdash{}{0pt}%
\pgfpathmoveto{\pgfqpoint{4.355084in}{1.812227in}}%
\pgfpathlineto{\pgfqpoint{4.368953in}{1.819885in}}%
\pgfpathlineto{\pgfqpoint{4.382835in}{1.827701in}}%
\pgfpathlineto{\pgfqpoint{4.396732in}{1.835674in}}%
\pgfpathlineto{\pgfqpoint{4.410643in}{1.843806in}}%
\pgfpathlineto{\pgfqpoint{4.418506in}{1.859622in}}%
\pgfpathlineto{\pgfqpoint{4.426364in}{1.875394in}}%
\pgfpathlineto{\pgfqpoint{4.434219in}{1.891119in}}%
\pgfpathlineto{\pgfqpoint{4.442069in}{1.906793in}}%
\pgfpathlineto{\pgfqpoint{4.428154in}{1.898258in}}%
\pgfpathlineto{\pgfqpoint{4.414253in}{1.889881in}}%
\pgfpathlineto{\pgfqpoint{4.400366in}{1.881663in}}%
\pgfpathlineto{\pgfqpoint{4.386493in}{1.873603in}}%
\pgfpathlineto{\pgfqpoint{4.378647in}{1.858321in}}%
\pgfpathlineto{\pgfqpoint{4.370796in}{1.842995in}}%
\pgfpathlineto{\pgfqpoint{4.362942in}{1.827630in}}%
\pgfpathlineto{\pgfqpoint{4.355084in}{1.812227in}}%
\pgfpathclose%
\pgfusepath{fill}%
\end{pgfscope}%
\begin{pgfscope}%
\pgfpathrectangle{\pgfqpoint{1.254980in}{0.150000in}}{\pgfqpoint{5.490039in}{5.490039in}}%
\pgfusepath{clip}%
\pgfsetbuttcap%
\pgfsetroundjoin%
\definecolor{currentfill}{rgb}{0.197636,0.391528,0.554969}%
\pgfsetfillcolor{currentfill}%
\pgfsetfillopacity{0.700000}%
\pgfsetlinewidth{0.000000pt}%
\definecolor{currentstroke}{rgb}{0.000000,0.000000,0.000000}%
\pgfsetstrokecolor{currentstroke}%
\pgfsetdash{}{0pt}%
\pgfpathmoveto{\pgfqpoint{4.473431in}{1.968909in}}%
\pgfpathlineto{\pgfqpoint{4.487366in}{1.977978in}}%
\pgfpathlineto{\pgfqpoint{4.501317in}{1.987206in}}%
\pgfpathlineto{\pgfqpoint{4.515282in}{1.996593in}}%
\pgfpathlineto{\pgfqpoint{4.529263in}{2.006139in}}%
\pgfpathlineto{\pgfqpoint{4.537098in}{2.021864in}}%
\pgfpathlineto{\pgfqpoint{4.544929in}{2.037511in}}%
\pgfpathlineto{\pgfqpoint{4.552756in}{2.053076in}}%
\pgfpathlineto{\pgfqpoint{4.560578in}{2.068556in}}%
\pgfpathlineto{\pgfqpoint{4.546592in}{2.058662in}}%
\pgfpathlineto{\pgfqpoint{4.532621in}{2.048928in}}%
\pgfpathlineto{\pgfqpoint{4.518665in}{2.039353in}}%
\pgfpathlineto{\pgfqpoint{4.504725in}{2.029938in}}%
\pgfpathlineto{\pgfqpoint{4.496908in}{2.014794in}}%
\pgfpathlineto{\pgfqpoint{4.489087in}{1.999572in}}%
\pgfpathlineto{\pgfqpoint{4.481261in}{1.984276in}}%
\pgfpathlineto{\pgfqpoint{4.473431in}{1.968909in}}%
\pgfpathclose%
\pgfusepath{fill}%
\end{pgfscope}%
\begin{pgfscope}%
\pgfpathrectangle{\pgfqpoint{1.254980in}{0.150000in}}{\pgfqpoint{5.490039in}{5.490039in}}%
\pgfusepath{clip}%
\pgfsetbuttcap%
\pgfsetroundjoin%
\definecolor{currentfill}{rgb}{0.255645,0.260703,0.528312}%
\pgfsetfillcolor{currentfill}%
\pgfsetfillopacity{0.700000}%
\pgfsetlinewidth{0.000000pt}%
\definecolor{currentstroke}{rgb}{0.000000,0.000000,0.000000}%
\pgfsetstrokecolor{currentstroke}%
\pgfsetdash{}{0pt}%
\pgfpathmoveto{\pgfqpoint{4.236762in}{1.662654in}}%
\pgfpathlineto{\pgfqpoint{4.250572in}{1.668795in}}%
\pgfpathlineto{\pgfqpoint{4.264395in}{1.675094in}}%
\pgfpathlineto{\pgfqpoint{4.278230in}{1.681549in}}%
\pgfpathlineto{\pgfqpoint{4.292079in}{1.688160in}}%
\pgfpathlineto{\pgfqpoint{4.299968in}{1.703713in}}%
\pgfpathlineto{\pgfqpoint{4.307853in}{1.719262in}}%
\pgfpathlineto{\pgfqpoint{4.315734in}{1.734802in}}%
\pgfpathlineto{\pgfqpoint{4.323612in}{1.750329in}}%
\pgfpathlineto{\pgfqpoint{4.309760in}{1.743259in}}%
\pgfpathlineto{\pgfqpoint{4.295922in}{1.736346in}}%
\pgfpathlineto{\pgfqpoint{4.282096in}{1.729590in}}%
\pgfpathlineto{\pgfqpoint{4.268284in}{1.722992in}}%
\pgfpathlineto{\pgfqpoint{4.260409in}{1.707912in}}%
\pgfpathlineto{\pgfqpoint{4.252531in}{1.692826in}}%
\pgfpathlineto{\pgfqpoint{4.244648in}{1.677739in}}%
\pgfpathlineto{\pgfqpoint{4.236762in}{1.662654in}}%
\pgfpathclose%
\pgfusepath{fill}%
\end{pgfscope}%
\begin{pgfscope}%
\pgfpathrectangle{\pgfqpoint{1.254980in}{0.150000in}}{\pgfqpoint{5.490039in}{5.490039in}}%
\pgfusepath{clip}%
\pgfsetbuttcap%
\pgfsetroundjoin%
\definecolor{currentfill}{rgb}{0.269944,0.014625,0.341379}%
\pgfsetfillcolor{currentfill}%
\pgfsetfillopacity{0.700000}%
\pgfsetlinewidth{0.000000pt}%
\definecolor{currentstroke}{rgb}{0.000000,0.000000,0.000000}%
\pgfsetstrokecolor{currentstroke}%
\pgfsetdash{}{0pt}%
\pgfpathmoveto{\pgfqpoint{3.511452in}{1.219430in}}%
\pgfpathlineto{\pgfqpoint{3.525066in}{1.215244in}}%
\pgfpathlineto{\pgfqpoint{3.538685in}{1.211218in}}%
\pgfpathlineto{\pgfqpoint{3.552308in}{1.207351in}}%
\pgfpathlineto{\pgfqpoint{3.565937in}{1.203644in}}%
\pgfpathlineto{\pgfqpoint{3.574090in}{1.210211in}}%
\pgfpathlineto{\pgfqpoint{3.582234in}{1.217032in}}%
\pgfpathlineto{\pgfqpoint{3.590369in}{1.224100in}}%
\pgfpathlineto{\pgfqpoint{3.598494in}{1.231408in}}%
\pgfpathlineto{\pgfqpoint{3.584887in}{1.234418in}}%
\pgfpathlineto{\pgfqpoint{3.571285in}{1.237587in}}%
\pgfpathlineto{\pgfqpoint{3.557689in}{1.240915in}}%
\pgfpathlineto{\pgfqpoint{3.544097in}{1.244403in}}%
\pgfpathlineto{\pgfqpoint{3.535951in}{1.237783in}}%
\pgfpathlineto{\pgfqpoint{3.527794in}{1.231409in}}%
\pgfpathlineto{\pgfqpoint{3.519628in}{1.225289in}}%
\pgfpathlineto{\pgfqpoint{3.511452in}{1.219430in}}%
\pgfpathclose%
\pgfusepath{fill}%
\end{pgfscope}%
\begin{pgfscope}%
\pgfpathrectangle{\pgfqpoint{1.254980in}{0.150000in}}{\pgfqpoint{5.490039in}{5.490039in}}%
\pgfusepath{clip}%
\pgfsetbuttcap%
\pgfsetroundjoin%
\definecolor{currentfill}{rgb}{0.575563,0.844566,0.256415}%
\pgfsetfillcolor{currentfill}%
\pgfsetfillopacity{0.700000}%
\pgfsetlinewidth{0.000000pt}%
\definecolor{currentstroke}{rgb}{0.000000,0.000000,0.000000}%
\pgfsetstrokecolor{currentstroke}%
\pgfsetdash{}{0pt}%
\pgfpathmoveto{\pgfqpoint{5.594288in}{3.340163in}}%
\pgfpathlineto{\pgfqpoint{5.609008in}{3.357434in}}%
\pgfpathlineto{\pgfqpoint{5.623752in}{3.374872in}}%
\pgfpathlineto{\pgfqpoint{5.638519in}{3.392480in}}%
\pgfpathlineto{\pgfqpoint{5.645774in}{3.397187in}}%
\pgfpathlineto{\pgfqpoint{5.653017in}{3.401710in}}%
\pgfpathlineto{\pgfqpoint{5.660249in}{3.406051in}}%
\pgfpathlineto{\pgfqpoint{5.667468in}{3.410212in}}%
\pgfpathlineto{\pgfqpoint{5.652715in}{3.392818in}}%
\pgfpathlineto{\pgfqpoint{5.637985in}{3.375593in}}%
\pgfpathlineto{\pgfqpoint{5.623277in}{3.358535in}}%
\pgfpathlineto{\pgfqpoint{5.616047in}{3.354205in}}%
\pgfpathlineto{\pgfqpoint{5.608805in}{3.349701in}}%
\pgfpathlineto{\pgfqpoint{5.601552in}{3.345021in}}%
\pgfpathlineto{\pgfqpoint{5.594288in}{3.340163in}}%
\pgfpathclose%
\pgfusepath{fill}%
\end{pgfscope}%
\begin{pgfscope}%
\pgfpathrectangle{\pgfqpoint{1.254980in}{0.150000in}}{\pgfqpoint{5.490039in}{5.490039in}}%
\pgfusepath{clip}%
\pgfsetbuttcap%
\pgfsetroundjoin%
\definecolor{currentfill}{rgb}{0.171176,0.452530,0.557965}%
\pgfsetfillcolor{currentfill}%
\pgfsetfillopacity{0.700000}%
\pgfsetlinewidth{0.000000pt}%
\definecolor{currentstroke}{rgb}{0.000000,0.000000,0.000000}%
\pgfsetstrokecolor{currentstroke}%
\pgfsetdash{}{0pt}%
\pgfpathmoveto{\pgfqpoint{4.591821in}{2.129572in}}%
\pgfpathlineto{\pgfqpoint{4.605829in}{2.139944in}}%
\pgfpathlineto{\pgfqpoint{4.619853in}{2.150477in}}%
\pgfpathlineto{\pgfqpoint{4.633893in}{2.161170in}}%
\pgfpathlineto{\pgfqpoint{4.647950in}{2.172024in}}%
\pgfpathlineto{\pgfqpoint{4.655755in}{2.187339in}}%
\pgfpathlineto{\pgfqpoint{4.663555in}{2.202544in}}%
\pgfpathlineto{\pgfqpoint{4.671349in}{2.217638in}}%
\pgfpathlineto{\pgfqpoint{4.679139in}{2.232618in}}%
\pgfpathlineto{\pgfqpoint{4.665076in}{2.221474in}}%
\pgfpathlineto{\pgfqpoint{4.651030in}{2.210491in}}%
\pgfpathlineto{\pgfqpoint{4.637001in}{2.199668in}}%
\pgfpathlineto{\pgfqpoint{4.622987in}{2.189007in}}%
\pgfpathlineto{\pgfqpoint{4.615203in}{2.174305in}}%
\pgfpathlineto{\pgfqpoint{4.607414in}{2.159497in}}%
\pgfpathlineto{\pgfqpoint{4.599620in}{2.144585in}}%
\pgfpathlineto{\pgfqpoint{4.591821in}{2.129572in}}%
\pgfpathclose%
\pgfusepath{fill}%
\end{pgfscope}%
\begin{pgfscope}%
\pgfpathrectangle{\pgfqpoint{1.254980in}{0.150000in}}{\pgfqpoint{5.490039in}{5.490039in}}%
\pgfusepath{clip}%
\pgfsetbuttcap%
\pgfsetroundjoin%
\definecolor{currentfill}{rgb}{0.153894,0.680203,0.504172}%
\pgfsetfillcolor{currentfill}%
\pgfsetfillopacity{0.700000}%
\pgfsetlinewidth{0.000000pt}%
\definecolor{currentstroke}{rgb}{0.000000,0.000000,0.000000}%
\pgfsetstrokecolor{currentstroke}%
\pgfsetdash{}{0pt}%
\pgfpathmoveto{\pgfqpoint{5.065321in}{2.758834in}}%
\pgfpathlineto{\pgfqpoint{5.079650in}{2.773293in}}%
\pgfpathlineto{\pgfqpoint{5.093998in}{2.787916in}}%
\pgfpathlineto{\pgfqpoint{5.108367in}{2.802705in}}%
\pgfpathlineto{\pgfqpoint{5.122756in}{2.817660in}}%
\pgfpathlineto{\pgfqpoint{5.130367in}{2.828836in}}%
\pgfpathlineto{\pgfqpoint{5.137969in}{2.839828in}}%
\pgfpathlineto{\pgfqpoint{5.145563in}{2.850636in}}%
\pgfpathlineto{\pgfqpoint{5.153148in}{2.861261in}}%
\pgfpathlineto{\pgfqpoint{5.138759in}{2.846259in}}%
\pgfpathlineto{\pgfqpoint{5.124390in}{2.831424in}}%
\pgfpathlineto{\pgfqpoint{5.110041in}{2.816753in}}%
\pgfpathlineto{\pgfqpoint{5.095712in}{2.802248in}}%
\pgfpathlineto{\pgfqpoint{5.088127in}{2.791659in}}%
\pgfpathlineto{\pgfqpoint{5.080533in}{2.780893in}}%
\pgfpathlineto{\pgfqpoint{5.072931in}{2.769952in}}%
\pgfpathlineto{\pgfqpoint{5.065321in}{2.758834in}}%
\pgfpathclose%
\pgfusepath{fill}%
\end{pgfscope}%
\begin{pgfscope}%
\pgfpathrectangle{\pgfqpoint{1.254980in}{0.150000in}}{\pgfqpoint{5.490039in}{5.490039in}}%
\pgfusepath{clip}%
\pgfsetbuttcap%
\pgfsetroundjoin%
\definecolor{currentfill}{rgb}{0.147607,0.511733,0.557049}%
\pgfsetfillcolor{currentfill}%
\pgfsetfillopacity{0.700000}%
\pgfsetlinewidth{0.000000pt}%
\definecolor{currentstroke}{rgb}{0.000000,0.000000,0.000000}%
\pgfsetstrokecolor{currentstroke}%
\pgfsetdash{}{0pt}%
\pgfpathmoveto{\pgfqpoint{4.710245in}{2.291352in}}%
\pgfpathlineto{\pgfqpoint{4.724330in}{2.302918in}}%
\pgfpathlineto{\pgfqpoint{4.738432in}{2.314646in}}%
\pgfpathlineto{\pgfqpoint{4.752552in}{2.326536in}}%
\pgfpathlineto{\pgfqpoint{4.766689in}{2.338588in}}%
\pgfpathlineto{\pgfqpoint{4.774458in}{2.353206in}}%
\pgfpathlineto{\pgfqpoint{4.782221in}{2.367690in}}%
\pgfpathlineto{\pgfqpoint{4.789977in}{2.382036in}}%
\pgfpathlineto{\pgfqpoint{4.797729in}{2.396244in}}%
\pgfpathlineto{\pgfqpoint{4.783586in}{2.383961in}}%
\pgfpathlineto{\pgfqpoint{4.769461in}{2.371840in}}%
\pgfpathlineto{\pgfqpoint{4.755354in}{2.359881in}}%
\pgfpathlineto{\pgfqpoint{4.741264in}{2.348085in}}%
\pgfpathlineto{\pgfqpoint{4.733517in}{2.334096in}}%
\pgfpathlineto{\pgfqpoint{4.725765in}{2.319977in}}%
\pgfpathlineto{\pgfqpoint{4.718008in}{2.305728in}}%
\pgfpathlineto{\pgfqpoint{4.710245in}{2.291352in}}%
\pgfpathclose%
\pgfusepath{fill}%
\end{pgfscope}%
\begin{pgfscope}%
\pgfpathrectangle{\pgfqpoint{1.254980in}{0.150000in}}{\pgfqpoint{5.490039in}{5.490039in}}%
\pgfusepath{clip}%
\pgfsetbuttcap%
\pgfsetroundjoin%
\definecolor{currentfill}{rgb}{0.275191,0.194905,0.496005}%
\pgfsetfillcolor{currentfill}%
\pgfsetfillopacity{0.700000}%
\pgfsetlinewidth{0.000000pt}%
\definecolor{currentstroke}{rgb}{0.000000,0.000000,0.000000}%
\pgfsetstrokecolor{currentstroke}%
\pgfsetdash{}{0pt}%
\pgfpathmoveto{\pgfqpoint{4.118417in}{1.523585in}}%
\pgfpathlineto{\pgfqpoint{4.132178in}{1.528106in}}%
\pgfpathlineto{\pgfqpoint{4.145951in}{1.532783in}}%
\pgfpathlineto{\pgfqpoint{4.159735in}{1.537615in}}%
\pgfpathlineto{\pgfqpoint{4.173531in}{1.542604in}}%
\pgfpathlineto{\pgfqpoint{4.181448in}{1.557506in}}%
\pgfpathlineto{\pgfqpoint{4.189362in}{1.572447in}}%
\pgfpathlineto{\pgfqpoint{4.197271in}{1.587423in}}%
\pgfpathlineto{\pgfqpoint{4.205177in}{1.602428in}}%
\pgfpathlineto{\pgfqpoint{4.191381in}{1.596928in}}%
\pgfpathlineto{\pgfqpoint{4.177596in}{1.591583in}}%
\pgfpathlineto{\pgfqpoint{4.163824in}{1.586395in}}%
\pgfpathlineto{\pgfqpoint{4.150063in}{1.581364in}}%
\pgfpathlineto{\pgfqpoint{4.142158in}{1.566859in}}%
\pgfpathlineto{\pgfqpoint{4.134248in}{1.552391in}}%
\pgfpathlineto{\pgfqpoint{4.126335in}{1.537965in}}%
\pgfpathlineto{\pgfqpoint{4.118417in}{1.523585in}}%
\pgfpathclose%
\pgfusepath{fill}%
\end{pgfscope}%
\begin{pgfscope}%
\pgfpathrectangle{\pgfqpoint{1.254980in}{0.150000in}}{\pgfqpoint{5.490039in}{5.490039in}}%
\pgfusepath{clip}%
\pgfsetbuttcap%
\pgfsetroundjoin%
\definecolor{currentfill}{rgb}{0.280267,0.073417,0.397163}%
\pgfsetfillcolor{currentfill}%
\pgfsetfillopacity{0.700000}%
\pgfsetlinewidth{0.000000pt}%
\definecolor{currentstroke}{rgb}{0.000000,0.000000,0.000000}%
\pgfsetstrokecolor{currentstroke}%
\pgfsetdash{}{0pt}%
\pgfpathmoveto{\pgfqpoint{3.826590in}{1.289616in}}%
\pgfpathlineto{\pgfqpoint{3.840255in}{1.289962in}}%
\pgfpathlineto{\pgfqpoint{3.853929in}{1.290463in}}%
\pgfpathlineto{\pgfqpoint{3.867611in}{1.291119in}}%
\pgfpathlineto{\pgfqpoint{3.881301in}{1.291931in}}%
\pgfpathlineto{\pgfqpoint{3.889304in}{1.303599in}}%
\pgfpathlineto{\pgfqpoint{3.897302in}{1.315414in}}%
\pgfpathlineto{\pgfqpoint{3.905294in}{1.327369in}}%
\pgfpathlineto{\pgfqpoint{3.913281in}{1.339460in}}%
\pgfpathlineto{\pgfqpoint{3.899599in}{1.338031in}}%
\pgfpathlineto{\pgfqpoint{3.885926in}{1.336758in}}%
\pgfpathlineto{\pgfqpoint{3.872261in}{1.335640in}}%
\pgfpathlineto{\pgfqpoint{3.858606in}{1.334678in}}%
\pgfpathlineto{\pgfqpoint{3.850611in}{1.323194in}}%
\pgfpathlineto{\pgfqpoint{3.842610in}{1.311851in}}%
\pgfpathlineto{\pgfqpoint{3.834603in}{1.300657in}}%
\pgfpathlineto{\pgfqpoint{3.826590in}{1.289616in}}%
\pgfpathclose%
\pgfusepath{fill}%
\end{pgfscope}%
\begin{pgfscope}%
\pgfpathrectangle{\pgfqpoint{1.254980in}{0.150000in}}{\pgfqpoint{5.490039in}{5.490039in}}%
\pgfusepath{clip}%
\pgfsetbuttcap%
\pgfsetroundjoin%
\definecolor{currentfill}{rgb}{0.276022,0.044167,0.370164}%
\pgfsetfillcolor{currentfill}%
\pgfsetfillopacity{0.700000}%
\pgfsetlinewidth{0.000000pt}%
\definecolor{currentstroke}{rgb}{0.000000,0.000000,0.000000}%
\pgfsetstrokecolor{currentstroke}%
\pgfsetdash{}{0pt}%
\pgfpathmoveto{\pgfqpoint{3.739842in}{1.249867in}}%
\pgfpathlineto{\pgfqpoint{3.753490in}{1.248945in}}%
\pgfpathlineto{\pgfqpoint{3.767144in}{1.248180in}}%
\pgfpathlineto{\pgfqpoint{3.780806in}{1.247570in}}%
\pgfpathlineto{\pgfqpoint{3.794475in}{1.247116in}}%
\pgfpathlineto{\pgfqpoint{3.802513in}{1.257479in}}%
\pgfpathlineto{\pgfqpoint{3.810545in}{1.268021in}}%
\pgfpathlineto{\pgfqpoint{3.818570in}{1.278735in}}%
\pgfpathlineto{\pgfqpoint{3.826590in}{1.289616in}}%
\pgfpathlineto{\pgfqpoint{3.812932in}{1.289426in}}%
\pgfpathlineto{\pgfqpoint{3.799283in}{1.289393in}}%
\pgfpathlineto{\pgfqpoint{3.785641in}{1.289515in}}%
\pgfpathlineto{\pgfqpoint{3.772007in}{1.289795in}}%
\pgfpathlineto{\pgfqpoint{3.763976in}{1.279547in}}%
\pgfpathlineto{\pgfqpoint{3.755939in}{1.269472in}}%
\pgfpathlineto{\pgfqpoint{3.747894in}{1.259577in}}%
\pgfpathlineto{\pgfqpoint{3.739842in}{1.249867in}}%
\pgfpathclose%
\pgfusepath{fill}%
\end{pgfscope}%
\begin{pgfscope}%
\pgfpathrectangle{\pgfqpoint{1.254980in}{0.150000in}}{\pgfqpoint{5.490039in}{5.490039in}}%
\pgfusepath{clip}%
\pgfsetbuttcap%
\pgfsetroundjoin%
\definecolor{currentfill}{rgb}{0.120638,0.625828,0.533488}%
\pgfsetfillcolor{currentfill}%
\pgfsetfillopacity{0.700000}%
\pgfsetlinewidth{0.000000pt}%
\definecolor{currentstroke}{rgb}{0.000000,0.000000,0.000000}%
\pgfsetstrokecolor{currentstroke}%
\pgfsetdash{}{0pt}%
\pgfpathmoveto{\pgfqpoint{4.947053in}{2.608163in}}%
\pgfpathlineto{\pgfqpoint{4.961300in}{2.621775in}}%
\pgfpathlineto{\pgfqpoint{4.975566in}{2.635552in}}%
\pgfpathlineto{\pgfqpoint{4.989851in}{2.649493in}}%
\pgfpathlineto{\pgfqpoint{5.004156in}{2.663598in}}%
\pgfpathlineto{\pgfqpoint{5.011828in}{2.676110in}}%
\pgfpathlineto{\pgfqpoint{5.019494in}{2.688450in}}%
\pgfpathlineto{\pgfqpoint{5.027151in}{2.700617in}}%
\pgfpathlineto{\pgfqpoint{5.034801in}{2.712610in}}%
\pgfpathlineto{\pgfqpoint{5.020494in}{2.698395in}}%
\pgfpathlineto{\pgfqpoint{5.006206in}{2.684345in}}%
\pgfpathlineto{\pgfqpoint{4.991937in}{2.670459in}}%
\pgfpathlineto{\pgfqpoint{4.977688in}{2.656737in}}%
\pgfpathlineto{\pgfqpoint{4.970040in}{2.644842in}}%
\pgfpathlineto{\pgfqpoint{4.962385in}{2.632781in}}%
\pgfpathlineto{\pgfqpoint{4.954723in}{2.620554in}}%
\pgfpathlineto{\pgfqpoint{4.947053in}{2.608163in}}%
\pgfpathclose%
\pgfusepath{fill}%
\end{pgfscope}%
\begin{pgfscope}%
\pgfpathrectangle{\pgfqpoint{1.254980in}{0.150000in}}{\pgfqpoint{5.490039in}{5.490039in}}%
\pgfusepath{clip}%
\pgfsetbuttcap%
\pgfsetroundjoin%
\definecolor{currentfill}{rgb}{0.126453,0.570633,0.549841}%
\pgfsetfillcolor{currentfill}%
\pgfsetfillopacity{0.700000}%
\pgfsetlinewidth{0.000000pt}%
\definecolor{currentstroke}{rgb}{0.000000,0.000000,0.000000}%
\pgfsetstrokecolor{currentstroke}%
\pgfsetdash{}{0pt}%
\pgfpathmoveto{\pgfqpoint{4.828672in}{2.451656in}}%
\pgfpathlineto{\pgfqpoint{4.842838in}{2.464303in}}%
\pgfpathlineto{\pgfqpoint{4.857021in}{2.477113in}}%
\pgfpathlineto{\pgfqpoint{4.871223in}{2.490087in}}%
\pgfpathlineto{\pgfqpoint{4.885443in}{2.503223in}}%
\pgfpathlineto{\pgfqpoint{4.893168in}{2.516896in}}%
\pgfpathlineto{\pgfqpoint{4.900886in}{2.530412in}}%
\pgfpathlineto{\pgfqpoint{4.908598in}{2.543771in}}%
\pgfpathlineto{\pgfqpoint{4.916303in}{2.556971in}}%
\pgfpathlineto{\pgfqpoint{4.902078in}{2.543663in}}%
\pgfpathlineto{\pgfqpoint{4.887872in}{2.530518in}}%
\pgfpathlineto{\pgfqpoint{4.873685in}{2.517538in}}%
\pgfpathlineto{\pgfqpoint{4.859515in}{2.504720in}}%
\pgfpathlineto{\pgfqpoint{4.851814in}{2.491679in}}%
\pgfpathlineto{\pgfqpoint{4.844107in}{2.478487in}}%
\pgfpathlineto{\pgfqpoint{4.836393in}{2.465146in}}%
\pgfpathlineto{\pgfqpoint{4.828672in}{2.451656in}}%
\pgfpathclose%
\pgfusepath{fill}%
\end{pgfscope}%
\begin{pgfscope}%
\pgfpathrectangle{\pgfqpoint{1.254980in}{0.150000in}}{\pgfqpoint{5.490039in}{5.490039in}}%
\pgfusepath{clip}%
\pgfsetbuttcap%
\pgfsetroundjoin%
\definecolor{currentfill}{rgb}{0.395174,0.797475,0.367757}%
\pgfsetfillcolor{currentfill}%
\pgfsetfillopacity{0.700000}%
\pgfsetlinewidth{0.000000pt}%
\definecolor{currentstroke}{rgb}{0.000000,0.000000,0.000000}%
\pgfsetstrokecolor{currentstroke}%
\pgfsetdash{}{0pt}%
\pgfpathmoveto{\pgfqpoint{5.389047in}{3.131486in}}%
\pgfpathlineto{\pgfqpoint{5.403615in}{3.147855in}}%
\pgfpathlineto{\pgfqpoint{5.418204in}{3.164391in}}%
\pgfpathlineto{\pgfqpoint{5.432816in}{3.181095in}}%
\pgfpathlineto{\pgfqpoint{5.447451in}{3.197968in}}%
\pgfpathlineto{\pgfqpoint{5.454860in}{3.205237in}}%
\pgfpathlineto{\pgfqpoint{5.462258in}{3.212313in}}%
\pgfpathlineto{\pgfqpoint{5.469645in}{3.219196in}}%
\pgfpathlineto{\pgfqpoint{5.477021in}{3.225887in}}%
\pgfpathlineto{\pgfqpoint{5.462394in}{3.209130in}}%
\pgfpathlineto{\pgfqpoint{5.447789in}{3.192541in}}%
\pgfpathlineto{\pgfqpoint{5.433207in}{3.176119in}}%
\pgfpathlineto{\pgfqpoint{5.418647in}{3.159864in}}%
\pgfpathlineto{\pgfqpoint{5.411262in}{3.153046in}}%
\pgfpathlineto{\pgfqpoint{5.403868in}{3.146044in}}%
\pgfpathlineto{\pgfqpoint{5.396462in}{3.138858in}}%
\pgfpathlineto{\pgfqpoint{5.389047in}{3.131486in}}%
\pgfpathclose%
\pgfusepath{fill}%
\end{pgfscope}%
\begin{pgfscope}%
\pgfpathrectangle{\pgfqpoint{1.254980in}{0.150000in}}{\pgfqpoint{5.490039in}{5.490039in}}%
\pgfusepath{clip}%
\pgfsetbuttcap%
\pgfsetroundjoin%
\definecolor{currentfill}{rgb}{0.282656,0.100196,0.422160}%
\pgfsetfillcolor{currentfill}%
\pgfsetfillopacity{0.700000}%
\pgfsetlinewidth{0.000000pt}%
\definecolor{currentstroke}{rgb}{0.000000,0.000000,0.000000}%
\pgfsetstrokecolor{currentstroke}%
\pgfsetdash{}{0pt}%
\pgfpathmoveto{\pgfqpoint{3.913281in}{1.339460in}}%
\pgfpathlineto{\pgfqpoint{3.926972in}{1.341045in}}%
\pgfpathlineto{\pgfqpoint{3.940672in}{1.342785in}}%
\pgfpathlineto{\pgfqpoint{3.954381in}{1.344680in}}%
\pgfpathlineto{\pgfqpoint{3.968100in}{1.346730in}}%
\pgfpathlineto{\pgfqpoint{3.976075in}{1.359551in}}%
\pgfpathlineto{\pgfqpoint{3.984045in}{1.372488in}}%
\pgfpathlineto{\pgfqpoint{3.992009in}{1.385537in}}%
\pgfpathlineto{\pgfqpoint{3.999969in}{1.398690in}}%
\pgfpathlineto{\pgfqpoint{3.986256in}{1.396049in}}%
\pgfpathlineto{\pgfqpoint{3.972552in}{1.393563in}}%
\pgfpathlineto{\pgfqpoint{3.958858in}{1.391233in}}%
\pgfpathlineto{\pgfqpoint{3.945174in}{1.389059in}}%
\pgfpathlineto{\pgfqpoint{3.937209in}{1.376485in}}%
\pgfpathlineto{\pgfqpoint{3.929238in}{1.364024in}}%
\pgfpathlineto{\pgfqpoint{3.921262in}{1.351680in}}%
\pgfpathlineto{\pgfqpoint{3.913281in}{1.339460in}}%
\pgfpathclose%
\pgfusepath{fill}%
\end{pgfscope}%
\begin{pgfscope}%
\pgfpathrectangle{\pgfqpoint{1.254980in}{0.150000in}}{\pgfqpoint{5.490039in}{5.490039in}}%
\pgfusepath{clip}%
\pgfsetbuttcap%
\pgfsetroundjoin%
\definecolor{currentfill}{rgb}{0.272594,0.025563,0.353093}%
\pgfsetfillcolor{currentfill}%
\pgfsetfillopacity{0.700000}%
\pgfsetlinewidth{0.000000pt}%
\definecolor{currentstroke}{rgb}{0.000000,0.000000,0.000000}%
\pgfsetstrokecolor{currentstroke}%
\pgfsetdash{}{0pt}%
\pgfpathmoveto{\pgfqpoint{3.652980in}{1.220953in}}%
\pgfpathlineto{\pgfqpoint{3.666616in}{1.218733in}}%
\pgfpathlineto{\pgfqpoint{3.680259in}{1.216671in}}%
\pgfpathlineto{\pgfqpoint{3.693908in}{1.214765in}}%
\pgfpathlineto{\pgfqpoint{3.707563in}{1.213016in}}%
\pgfpathlineto{\pgfqpoint{3.715644in}{1.221918in}}%
\pgfpathlineto{\pgfqpoint{3.723718in}{1.231031in}}%
\pgfpathlineto{\pgfqpoint{3.731784in}{1.240350in}}%
\pgfpathlineto{\pgfqpoint{3.739842in}{1.249867in}}%
\pgfpathlineto{\pgfqpoint{3.726203in}{1.250946in}}%
\pgfpathlineto{\pgfqpoint{3.712570in}{1.252181in}}%
\pgfpathlineto{\pgfqpoint{3.698943in}{1.253573in}}%
\pgfpathlineto{\pgfqpoint{3.685324in}{1.255123in}}%
\pgfpathlineto{\pgfqpoint{3.677250in}{1.246265in}}%
\pgfpathlineto{\pgfqpoint{3.669168in}{1.237613in}}%
\pgfpathlineto{\pgfqpoint{3.661078in}{1.229173in}}%
\pgfpathlineto{\pgfqpoint{3.652980in}{1.220953in}}%
\pgfpathclose%
\pgfusepath{fill}%
\end{pgfscope}%
\begin{pgfscope}%
\pgfpathrectangle{\pgfqpoint{1.254980in}{0.150000in}}{\pgfqpoint{5.490039in}{5.490039in}}%
\pgfusepath{clip}%
\pgfsetbuttcap%
\pgfsetroundjoin%
\definecolor{currentfill}{rgb}{0.282884,0.135920,0.453427}%
\pgfsetfillcolor{currentfill}%
\pgfsetfillopacity{0.700000}%
\pgfsetlinewidth{0.000000pt}%
\definecolor{currentstroke}{rgb}{0.000000,0.000000,0.000000}%
\pgfsetstrokecolor{currentstroke}%
\pgfsetdash{}{0pt}%
\pgfpathmoveto{\pgfqpoint{3.999969in}{1.398690in}}%
\pgfpathlineto{\pgfqpoint{4.013693in}{1.401487in}}%
\pgfpathlineto{\pgfqpoint{4.027427in}{1.404439in}}%
\pgfpathlineto{\pgfqpoint{4.041170in}{1.407546in}}%
\pgfpathlineto{\pgfqpoint{4.054925in}{1.410808in}}%
\pgfpathlineto{\pgfqpoint{4.062876in}{1.424636in}}%
\pgfpathlineto{\pgfqpoint{4.070824in}{1.438552in}}%
\pgfpathlineto{\pgfqpoint{4.078766in}{1.452550in}}%
\pgfpathlineto{\pgfqpoint{4.086705in}{1.466624in}}%
\pgfpathlineto{\pgfqpoint{4.072953in}{1.462797in}}%
\pgfpathlineto{\pgfqpoint{4.059212in}{1.459125in}}%
\pgfpathlineto{\pgfqpoint{4.045482in}{1.455609in}}%
\pgfpathlineto{\pgfqpoint{4.031763in}{1.452248in}}%
\pgfpathlineto{\pgfqpoint{4.023821in}{1.438728in}}%
\pgfpathlineto{\pgfqpoint{4.015875in}{1.425292in}}%
\pgfpathlineto{\pgfqpoint{4.007925in}{1.411944in}}%
\pgfpathlineto{\pgfqpoint{3.999969in}{1.398690in}}%
\pgfpathclose%
\pgfusepath{fill}%
\end{pgfscope}%
\begin{pgfscope}%
\pgfpathrectangle{\pgfqpoint{1.254980in}{0.150000in}}{\pgfqpoint{5.490039in}{5.490039in}}%
\pgfusepath{clip}%
\pgfsetbuttcap%
\pgfsetroundjoin%
\definecolor{currentfill}{rgb}{0.237441,0.305202,0.541921}%
\pgfsetfillcolor{currentfill}%
\pgfsetfillopacity{0.700000}%
\pgfsetlinewidth{0.000000pt}%
\definecolor{currentstroke}{rgb}{0.000000,0.000000,0.000000}%
\pgfsetstrokecolor{currentstroke}%
\pgfsetdash{}{0pt}%
\pgfpathmoveto{\pgfqpoint{4.323612in}{1.750329in}}%
\pgfpathlineto{\pgfqpoint{4.337477in}{1.757556in}}%
\pgfpathlineto{\pgfqpoint{4.351356in}{1.764941in}}%
\pgfpathlineto{\pgfqpoint{4.365249in}{1.772483in}}%
\pgfpathlineto{\pgfqpoint{4.379155in}{1.780183in}}%
\pgfpathlineto{\pgfqpoint{4.387033in}{1.796135in}}%
\pgfpathlineto{\pgfqpoint{4.394907in}{1.812058in}}%
\pgfpathlineto{\pgfqpoint{4.402777in}{1.827950in}}%
\pgfpathlineto{\pgfqpoint{4.410643in}{1.843806in}}%
\pgfpathlineto{\pgfqpoint{4.396732in}{1.835674in}}%
\pgfpathlineto{\pgfqpoint{4.382835in}{1.827701in}}%
\pgfpathlineto{\pgfqpoint{4.368953in}{1.819885in}}%
\pgfpathlineto{\pgfqpoint{4.355084in}{1.812227in}}%
\pgfpathlineto{\pgfqpoint{4.347222in}{1.796792in}}%
\pgfpathlineto{\pgfqpoint{4.339356in}{1.781328in}}%
\pgfpathlineto{\pgfqpoint{4.331486in}{1.765839in}}%
\pgfpathlineto{\pgfqpoint{4.323612in}{1.750329in}}%
\pgfpathclose%
\pgfusepath{fill}%
\end{pgfscope}%
\begin{pgfscope}%
\pgfpathrectangle{\pgfqpoint{1.254980in}{0.150000in}}{\pgfqpoint{5.490039in}{5.490039in}}%
\pgfusepath{clip}%
\pgfsetbuttcap%
\pgfsetroundjoin%
\definecolor{currentfill}{rgb}{0.304148,0.764704,0.419943}%
\pgfsetfillcolor{currentfill}%
\pgfsetfillopacity{0.700000}%
\pgfsetlinewidth{0.000000pt}%
\definecolor{currentstroke}{rgb}{0.000000,0.000000,0.000000}%
\pgfsetstrokecolor{currentstroke}%
\pgfsetdash{}{0pt}%
\pgfpathmoveto{\pgfqpoint{5.271260in}{3.001319in}}%
\pgfpathlineto{\pgfqpoint{5.285751in}{3.017150in}}%
\pgfpathlineto{\pgfqpoint{5.300262in}{3.033148in}}%
\pgfpathlineto{\pgfqpoint{5.314796in}{3.049314in}}%
\pgfpathlineto{\pgfqpoint{5.329351in}{3.065647in}}%
\pgfpathlineto{\pgfqpoint{5.336849in}{3.074554in}}%
\pgfpathlineto{\pgfqpoint{5.344336in}{3.083266in}}%
\pgfpathlineto{\pgfqpoint{5.351814in}{3.091783in}}%
\pgfpathlineto{\pgfqpoint{5.359281in}{3.100107in}}%
\pgfpathlineto{\pgfqpoint{5.344730in}{3.083824in}}%
\pgfpathlineto{\pgfqpoint{5.330200in}{3.067708in}}%
\pgfpathlineto{\pgfqpoint{5.315692in}{3.051759in}}%
\pgfpathlineto{\pgfqpoint{5.301205in}{3.035977in}}%
\pgfpathlineto{\pgfqpoint{5.293734in}{3.027592in}}%
\pgfpathlineto{\pgfqpoint{5.286252in}{3.019022in}}%
\pgfpathlineto{\pgfqpoint{5.278761in}{3.010264in}}%
\pgfpathlineto{\pgfqpoint{5.271260in}{3.001319in}}%
\pgfpathclose%
\pgfusepath{fill}%
\end{pgfscope}%
\begin{pgfscope}%
\pgfpathrectangle{\pgfqpoint{1.254980in}{0.150000in}}{\pgfqpoint{5.490039in}{5.490039in}}%
\pgfusepath{clip}%
\pgfsetbuttcap%
\pgfsetroundjoin%
\definecolor{currentfill}{rgb}{0.206756,0.371758,0.553117}%
\pgfsetfillcolor{currentfill}%
\pgfsetfillopacity{0.700000}%
\pgfsetlinewidth{0.000000pt}%
\definecolor{currentstroke}{rgb}{0.000000,0.000000,0.000000}%
\pgfsetstrokecolor{currentstroke}%
\pgfsetdash{}{0pt}%
\pgfpathmoveto{\pgfqpoint{4.442069in}{1.906793in}}%
\pgfpathlineto{\pgfqpoint{4.456000in}{1.915487in}}%
\pgfpathlineto{\pgfqpoint{4.469945in}{1.924339in}}%
\pgfpathlineto{\pgfqpoint{4.483905in}{1.933350in}}%
\pgfpathlineto{\pgfqpoint{4.497880in}{1.942519in}}%
\pgfpathlineto{\pgfqpoint{4.505732in}{1.958526in}}%
\pgfpathlineto{\pgfqpoint{4.513580in}{1.974467in}}%
\pgfpathlineto{\pgfqpoint{4.521424in}{1.990339in}}%
\pgfpathlineto{\pgfqpoint{4.529263in}{2.006139in}}%
\pgfpathlineto{\pgfqpoint{4.515282in}{1.996593in}}%
\pgfpathlineto{\pgfqpoint{4.501317in}{1.987206in}}%
\pgfpathlineto{\pgfqpoint{4.487366in}{1.977978in}}%
\pgfpathlineto{\pgfqpoint{4.473431in}{1.968909in}}%
\pgfpathlineto{\pgfqpoint{4.465597in}{1.953474in}}%
\pgfpathlineto{\pgfqpoint{4.457758in}{1.937974in}}%
\pgfpathlineto{\pgfqpoint{4.449916in}{1.922412in}}%
\pgfpathlineto{\pgfqpoint{4.442069in}{1.906793in}}%
\pgfpathclose%
\pgfusepath{fill}%
\end{pgfscope}%
\begin{pgfscope}%
\pgfpathrectangle{\pgfqpoint{1.254980in}{0.150000in}}{\pgfqpoint{5.490039in}{5.490039in}}%
\pgfusepath{clip}%
\pgfsetbuttcap%
\pgfsetroundjoin%
\definecolor{currentfill}{rgb}{0.262138,0.242286,0.520837}%
\pgfsetfillcolor{currentfill}%
\pgfsetfillopacity{0.700000}%
\pgfsetlinewidth{0.000000pt}%
\definecolor{currentstroke}{rgb}{0.000000,0.000000,0.000000}%
\pgfsetstrokecolor{currentstroke}%
\pgfsetdash{}{0pt}%
\pgfpathmoveto{\pgfqpoint{4.205177in}{1.602428in}}%
\pgfpathlineto{\pgfqpoint{4.218986in}{1.608085in}}%
\pgfpathlineto{\pgfqpoint{4.232807in}{1.613898in}}%
\pgfpathlineto{\pgfqpoint{4.246640in}{1.619867in}}%
\pgfpathlineto{\pgfqpoint{4.260487in}{1.625993in}}%
\pgfpathlineto{\pgfqpoint{4.268390in}{1.641519in}}%
\pgfpathlineto{\pgfqpoint{4.276290in}{1.657059in}}%
\pgfpathlineto{\pgfqpoint{4.284187in}{1.672608in}}%
\pgfpathlineto{\pgfqpoint{4.292079in}{1.688160in}}%
\pgfpathlineto{\pgfqpoint{4.278230in}{1.681549in}}%
\pgfpathlineto{\pgfqpoint{4.264395in}{1.675094in}}%
\pgfpathlineto{\pgfqpoint{4.250572in}{1.668795in}}%
\pgfpathlineto{\pgfqpoint{4.236762in}{1.662654in}}%
\pgfpathlineto{\pgfqpoint{4.228871in}{1.647576in}}%
\pgfpathlineto{\pgfqpoint{4.220977in}{1.632509in}}%
\pgfpathlineto{\pgfqpoint{4.213079in}{1.617458in}}%
\pgfpathlineto{\pgfqpoint{4.205177in}{1.602428in}}%
\pgfpathclose%
\pgfusepath{fill}%
\end{pgfscope}%
\begin{pgfscope}%
\pgfpathrectangle{\pgfqpoint{1.254980in}{0.150000in}}{\pgfqpoint{5.490039in}{5.490039in}}%
\pgfusepath{clip}%
\pgfsetbuttcap%
\pgfsetroundjoin%
\definecolor{currentfill}{rgb}{0.271305,0.019942,0.347269}%
\pgfsetfillcolor{currentfill}%
\pgfsetfillopacity{0.700000}%
\pgfsetlinewidth{0.000000pt}%
\definecolor{currentstroke}{rgb}{0.000000,0.000000,0.000000}%
\pgfsetstrokecolor{currentstroke}%
\pgfsetdash{}{0pt}%
\pgfpathmoveto{\pgfqpoint{3.565937in}{1.203644in}}%
\pgfpathlineto{\pgfqpoint{3.579570in}{1.200095in}}%
\pgfpathlineto{\pgfqpoint{3.593209in}{1.196704in}}%
\pgfpathlineto{\pgfqpoint{3.606853in}{1.193471in}}%
\pgfpathlineto{\pgfqpoint{3.620503in}{1.190396in}}%
\pgfpathlineto{\pgfqpoint{3.628635in}{1.197672in}}%
\pgfpathlineto{\pgfqpoint{3.636759in}{1.205195in}}%
\pgfpathlineto{\pgfqpoint{3.644874in}{1.212958in}}%
\pgfpathlineto{\pgfqpoint{3.652980in}{1.220953in}}%
\pgfpathlineto{\pgfqpoint{3.639350in}{1.223330in}}%
\pgfpathlineto{\pgfqpoint{3.625725in}{1.225864in}}%
\pgfpathlineto{\pgfqpoint{3.612107in}{1.228557in}}%
\pgfpathlineto{\pgfqpoint{3.598494in}{1.231408in}}%
\pgfpathlineto{\pgfqpoint{3.590369in}{1.224100in}}%
\pgfpathlineto{\pgfqpoint{3.582234in}{1.217032in}}%
\pgfpathlineto{\pgfqpoint{3.574090in}{1.210211in}}%
\pgfpathlineto{\pgfqpoint{3.565937in}{1.203644in}}%
\pgfpathclose%
\pgfusepath{fill}%
\end{pgfscope}%
\begin{pgfscope}%
\pgfpathrectangle{\pgfqpoint{1.254980in}{0.150000in}}{\pgfqpoint{5.490039in}{5.490039in}}%
\pgfusepath{clip}%
\pgfsetbuttcap%
\pgfsetroundjoin%
\definecolor{currentfill}{rgb}{0.179019,0.433756,0.557430}%
\pgfsetfillcolor{currentfill}%
\pgfsetfillopacity{0.700000}%
\pgfsetlinewidth{0.000000pt}%
\definecolor{currentstroke}{rgb}{0.000000,0.000000,0.000000}%
\pgfsetstrokecolor{currentstroke}%
\pgfsetdash{}{0pt}%
\pgfpathmoveto{\pgfqpoint{4.560578in}{2.068556in}}%
\pgfpathlineto{\pgfqpoint{4.574580in}{2.078610in}}%
\pgfpathlineto{\pgfqpoint{4.588599in}{2.088824in}}%
\pgfpathlineto{\pgfqpoint{4.602633in}{2.099198in}}%
\pgfpathlineto{\pgfqpoint{4.616684in}{2.109732in}}%
\pgfpathlineto{\pgfqpoint{4.624507in}{2.125455in}}%
\pgfpathlineto{\pgfqpoint{4.632326in}{2.141080in}}%
\pgfpathlineto{\pgfqpoint{4.640141in}{2.156604in}}%
\pgfpathlineto{\pgfqpoint{4.647950in}{2.172024in}}%
\pgfpathlineto{\pgfqpoint{4.633893in}{2.161170in}}%
\pgfpathlineto{\pgfqpoint{4.619853in}{2.150477in}}%
\pgfpathlineto{\pgfqpoint{4.605829in}{2.139944in}}%
\pgfpathlineto{\pgfqpoint{4.591821in}{2.129572in}}%
\pgfpathlineto{\pgfqpoint{4.584017in}{2.114459in}}%
\pgfpathlineto{\pgfqpoint{4.576209in}{2.099251in}}%
\pgfpathlineto{\pgfqpoint{4.568396in}{2.083949in}}%
\pgfpathlineto{\pgfqpoint{4.560578in}{2.068556in}}%
\pgfpathclose%
\pgfusepath{fill}%
\end{pgfscope}%
\begin{pgfscope}%
\pgfpathrectangle{\pgfqpoint{1.254980in}{0.150000in}}{\pgfqpoint{5.490039in}{5.490039in}}%
\pgfusepath{clip}%
\pgfsetbuttcap%
\pgfsetroundjoin%
\definecolor{currentfill}{rgb}{0.153364,0.497000,0.557724}%
\pgfsetfillcolor{currentfill}%
\pgfsetfillopacity{0.700000}%
\pgfsetlinewidth{0.000000pt}%
\definecolor{currentstroke}{rgb}{0.000000,0.000000,0.000000}%
\pgfsetstrokecolor{currentstroke}%
\pgfsetdash{}{0pt}%
\pgfpathmoveto{\pgfqpoint{4.679139in}{2.232618in}}%
\pgfpathlineto{\pgfqpoint{4.693219in}{2.243924in}}%
\pgfpathlineto{\pgfqpoint{4.707315in}{2.255391in}}%
\pgfpathlineto{\pgfqpoint{4.721429in}{2.267019in}}%
\pgfpathlineto{\pgfqpoint{4.735560in}{2.278809in}}%
\pgfpathlineto{\pgfqpoint{4.743350in}{2.293945in}}%
\pgfpathlineto{\pgfqpoint{4.751136in}{2.308955in}}%
\pgfpathlineto{\pgfqpoint{4.758915in}{2.323837in}}%
\pgfpathlineto{\pgfqpoint{4.766689in}{2.338588in}}%
\pgfpathlineto{\pgfqpoint{4.752552in}{2.326536in}}%
\pgfpathlineto{\pgfqpoint{4.738432in}{2.314646in}}%
\pgfpathlineto{\pgfqpoint{4.724330in}{2.302918in}}%
\pgfpathlineto{\pgfqpoint{4.710245in}{2.291352in}}%
\pgfpathlineto{\pgfqpoint{4.702476in}{2.276851in}}%
\pgfpathlineto{\pgfqpoint{4.694703in}{2.262227in}}%
\pgfpathlineto{\pgfqpoint{4.686923in}{2.247482in}}%
\pgfpathlineto{\pgfqpoint{4.679139in}{2.232618in}}%
\pgfpathclose%
\pgfusepath{fill}%
\end{pgfscope}%
\begin{pgfscope}%
\pgfpathrectangle{\pgfqpoint{1.254980in}{0.150000in}}{\pgfqpoint{5.490039in}{5.490039in}}%
\pgfusepath{clip}%
\pgfsetbuttcap%
\pgfsetroundjoin%
\definecolor{currentfill}{rgb}{0.214000,0.722114,0.469588}%
\pgfsetfillcolor{currentfill}%
\pgfsetfillopacity{0.700000}%
\pgfsetlinewidth{0.000000pt}%
\definecolor{currentstroke}{rgb}{0.000000,0.000000,0.000000}%
\pgfsetstrokecolor{currentstroke}%
\pgfsetdash{}{0pt}%
\pgfpathmoveto{\pgfqpoint{5.153148in}{2.861261in}}%
\pgfpathlineto{\pgfqpoint{5.167558in}{2.876428in}}%
\pgfpathlineto{\pgfqpoint{5.181988in}{2.891762in}}%
\pgfpathlineto{\pgfqpoint{5.196440in}{2.907262in}}%
\pgfpathlineto{\pgfqpoint{5.210912in}{2.922929in}}%
\pgfpathlineto{\pgfqpoint{5.218488in}{2.933396in}}%
\pgfpathlineto{\pgfqpoint{5.226055in}{2.943672in}}%
\pgfpathlineto{\pgfqpoint{5.233613in}{2.953756in}}%
\pgfpathlineto{\pgfqpoint{5.241161in}{2.963649in}}%
\pgfpathlineto{\pgfqpoint{5.226690in}{2.947967in}}%
\pgfpathlineto{\pgfqpoint{5.212239in}{2.932453in}}%
\pgfpathlineto{\pgfqpoint{5.197809in}{2.917104in}}%
\pgfpathlineto{\pgfqpoint{5.183400in}{2.901922in}}%
\pgfpathlineto{\pgfqpoint{5.175851in}{2.892031in}}%
\pgfpathlineto{\pgfqpoint{5.168292in}{2.881958in}}%
\pgfpathlineto{\pgfqpoint{5.160724in}{2.871701in}}%
\pgfpathlineto{\pgfqpoint{5.153148in}{2.861261in}}%
\pgfpathclose%
\pgfusepath{fill}%
\end{pgfscope}%
\begin{pgfscope}%
\pgfpathrectangle{\pgfqpoint{1.254980in}{0.150000in}}{\pgfqpoint{5.490039in}{5.490039in}}%
\pgfusepath{clip}%
\pgfsetbuttcap%
\pgfsetroundjoin%
\definecolor{currentfill}{rgb}{0.496615,0.826376,0.306377}%
\pgfsetfillcolor{currentfill}%
\pgfsetfillopacity{0.700000}%
\pgfsetlinewidth{0.000000pt}%
\definecolor{currentstroke}{rgb}{0.000000,0.000000,0.000000}%
\pgfsetstrokecolor{currentstroke}%
\pgfsetdash{}{0pt}%
\pgfpathmoveto{\pgfqpoint{5.477021in}{3.225887in}}%
\pgfpathlineto{\pgfqpoint{5.491671in}{3.242813in}}%
\pgfpathlineto{\pgfqpoint{5.506343in}{3.259906in}}%
\pgfpathlineto{\pgfqpoint{5.521038in}{3.277169in}}%
\pgfpathlineto{\pgfqpoint{5.535756in}{3.294600in}}%
\pgfpathlineto{\pgfqpoint{5.543113in}{3.300966in}}%
\pgfpathlineto{\pgfqpoint{5.550458in}{3.307137in}}%
\pgfpathlineto{\pgfqpoint{5.557791in}{3.313114in}}%
\pgfpathlineto{\pgfqpoint{5.565114in}{3.318899in}}%
\pgfpathlineto{\pgfqpoint{5.550405in}{3.301616in}}%
\pgfpathlineto{\pgfqpoint{5.535719in}{3.284502in}}%
\pgfpathlineto{\pgfqpoint{5.521056in}{3.267556in}}%
\pgfpathlineto{\pgfqpoint{5.506416in}{3.250778in}}%
\pgfpathlineto{\pgfqpoint{5.499084in}{3.244833in}}%
\pgfpathlineto{\pgfqpoint{5.491741in}{3.238704in}}%
\pgfpathlineto{\pgfqpoint{5.484386in}{3.232390in}}%
\pgfpathlineto{\pgfqpoint{5.477021in}{3.225887in}}%
\pgfpathclose%
\pgfusepath{fill}%
\end{pgfscope}%
\begin{pgfscope}%
\pgfpathrectangle{\pgfqpoint{1.254980in}{0.150000in}}{\pgfqpoint{5.490039in}{5.490039in}}%
\pgfusepath{clip}%
\pgfsetbuttcap%
\pgfsetroundjoin%
\definecolor{currentfill}{rgb}{0.278826,0.175490,0.483397}%
\pgfsetfillcolor{currentfill}%
\pgfsetfillopacity{0.700000}%
\pgfsetlinewidth{0.000000pt}%
\definecolor{currentstroke}{rgb}{0.000000,0.000000,0.000000}%
\pgfsetstrokecolor{currentstroke}%
\pgfsetdash{}{0pt}%
\pgfpathmoveto{\pgfqpoint{4.086705in}{1.466624in}}%
\pgfpathlineto{\pgfqpoint{4.100467in}{1.470607in}}%
\pgfpathlineto{\pgfqpoint{4.114241in}{1.474745in}}%
\pgfpathlineto{\pgfqpoint{4.128026in}{1.479039in}}%
\pgfpathlineto{\pgfqpoint{4.141822in}{1.483487in}}%
\pgfpathlineto{\pgfqpoint{4.149755in}{1.498182in}}%
\pgfpathlineto{\pgfqpoint{4.157684in}{1.512937in}}%
\pgfpathlineto{\pgfqpoint{4.165610in}{1.527746in}}%
\pgfpathlineto{\pgfqpoint{4.173531in}{1.542604in}}%
\pgfpathlineto{\pgfqpoint{4.159735in}{1.537615in}}%
\pgfpathlineto{\pgfqpoint{4.145951in}{1.532783in}}%
\pgfpathlineto{\pgfqpoint{4.132178in}{1.528106in}}%
\pgfpathlineto{\pgfqpoint{4.118417in}{1.523585in}}%
\pgfpathlineto{\pgfqpoint{4.110495in}{1.509255in}}%
\pgfpathlineto{\pgfqpoint{4.102569in}{1.494982in}}%
\pgfpathlineto{\pgfqpoint{4.094639in}{1.480770in}}%
\pgfpathlineto{\pgfqpoint{4.086705in}{1.466624in}}%
\pgfpathclose%
\pgfusepath{fill}%
\end{pgfscope}%
\begin{pgfscope}%
\pgfpathrectangle{\pgfqpoint{1.254980in}{0.150000in}}{\pgfqpoint{5.490039in}{5.490039in}}%
\pgfusepath{clip}%
\pgfsetbuttcap%
\pgfsetroundjoin%
\definecolor{currentfill}{rgb}{0.131172,0.555899,0.552459}%
\pgfsetfillcolor{currentfill}%
\pgfsetfillopacity{0.700000}%
\pgfsetlinewidth{0.000000pt}%
\definecolor{currentstroke}{rgb}{0.000000,0.000000,0.000000}%
\pgfsetstrokecolor{currentstroke}%
\pgfsetdash{}{0pt}%
\pgfpathmoveto{\pgfqpoint{4.797729in}{2.396244in}}%
\pgfpathlineto{\pgfqpoint{4.811889in}{2.408690in}}%
\pgfpathlineto{\pgfqpoint{4.826067in}{2.421299in}}%
\pgfpathlineto{\pgfqpoint{4.840264in}{2.434070in}}%
\pgfpathlineto{\pgfqpoint{4.854479in}{2.447005in}}%
\pgfpathlineto{\pgfqpoint{4.862230in}{2.461286in}}%
\pgfpathlineto{\pgfqpoint{4.869974in}{2.475417in}}%
\pgfpathlineto{\pgfqpoint{4.877712in}{2.489397in}}%
\pgfpathlineto{\pgfqpoint{4.885443in}{2.503223in}}%
\pgfpathlineto{\pgfqpoint{4.871223in}{2.490087in}}%
\pgfpathlineto{\pgfqpoint{4.857021in}{2.477113in}}%
\pgfpathlineto{\pgfqpoint{4.842838in}{2.464303in}}%
\pgfpathlineto{\pgfqpoint{4.828672in}{2.451656in}}%
\pgfpathlineto{\pgfqpoint{4.820946in}{2.438020in}}%
\pgfpathlineto{\pgfqpoint{4.813213in}{2.424238in}}%
\pgfpathlineto{\pgfqpoint{4.805474in}{2.410312in}}%
\pgfpathlineto{\pgfqpoint{4.797729in}{2.396244in}}%
\pgfpathclose%
\pgfusepath{fill}%
\end{pgfscope}%
\begin{pgfscope}%
\pgfpathrectangle{\pgfqpoint{1.254980in}{0.150000in}}{\pgfqpoint{5.490039in}{5.490039in}}%
\pgfusepath{clip}%
\pgfsetbuttcap%
\pgfsetroundjoin%
\definecolor{currentfill}{rgb}{0.143303,0.669459,0.511215}%
\pgfsetfillcolor{currentfill}%
\pgfsetfillopacity{0.700000}%
\pgfsetlinewidth{0.000000pt}%
\definecolor{currentstroke}{rgb}{0.000000,0.000000,0.000000}%
\pgfsetstrokecolor{currentstroke}%
\pgfsetdash{}{0pt}%
\pgfpathmoveto{\pgfqpoint{5.034801in}{2.712610in}}%
\pgfpathlineto{\pgfqpoint{5.049128in}{2.726990in}}%
\pgfpathlineto{\pgfqpoint{5.063475in}{2.741535in}}%
\pgfpathlineto{\pgfqpoint{5.077842in}{2.756245in}}%
\pgfpathlineto{\pgfqpoint{5.092229in}{2.771122in}}%
\pgfpathlineto{\pgfqpoint{5.099873in}{2.783031in}}%
\pgfpathlineto{\pgfqpoint{5.107509in}{2.794757in}}%
\pgfpathlineto{\pgfqpoint{5.115137in}{2.806300in}}%
\pgfpathlineto{\pgfqpoint{5.122756in}{2.817660in}}%
\pgfpathlineto{\pgfqpoint{5.108367in}{2.802705in}}%
\pgfpathlineto{\pgfqpoint{5.093998in}{2.787916in}}%
\pgfpathlineto{\pgfqpoint{5.079650in}{2.773293in}}%
\pgfpathlineto{\pgfqpoint{5.065321in}{2.758834in}}%
\pgfpathlineto{\pgfqpoint{5.057703in}{2.747541in}}%
\pgfpathlineto{\pgfqpoint{5.050077in}{2.736072in}}%
\pgfpathlineto{\pgfqpoint{5.042443in}{2.724429in}}%
\pgfpathlineto{\pgfqpoint{5.034801in}{2.712610in}}%
\pgfpathclose%
\pgfusepath{fill}%
\end{pgfscope}%
\begin{pgfscope}%
\pgfpathrectangle{\pgfqpoint{1.254980in}{0.150000in}}{\pgfqpoint{5.490039in}{5.490039in}}%
\pgfusepath{clip}%
\pgfsetbuttcap%
\pgfsetroundjoin%
\definecolor{currentfill}{rgb}{0.119483,0.614817,0.537692}%
\pgfsetfillcolor{currentfill}%
\pgfsetfillopacity{0.700000}%
\pgfsetlinewidth{0.000000pt}%
\definecolor{currentstroke}{rgb}{0.000000,0.000000,0.000000}%
\pgfsetstrokecolor{currentstroke}%
\pgfsetdash{}{0pt}%
\pgfpathmoveto{\pgfqpoint{4.916303in}{2.556971in}}%
\pgfpathlineto{\pgfqpoint{4.930546in}{2.570443in}}%
\pgfpathlineto{\pgfqpoint{4.944809in}{2.584078in}}%
\pgfpathlineto{\pgfqpoint{4.959090in}{2.597878in}}%
\pgfpathlineto{\pgfqpoint{4.973391in}{2.611843in}}%
\pgfpathlineto{\pgfqpoint{4.981093in}{2.625036in}}%
\pgfpathlineto{\pgfqpoint{4.988788in}{2.638060in}}%
\pgfpathlineto{\pgfqpoint{4.996476in}{2.650914in}}%
\pgfpathlineto{\pgfqpoint{5.004156in}{2.663598in}}%
\pgfpathlineto{\pgfqpoint{4.989851in}{2.649493in}}%
\pgfpathlineto{\pgfqpoint{4.975566in}{2.635552in}}%
\pgfpathlineto{\pgfqpoint{4.961300in}{2.621775in}}%
\pgfpathlineto{\pgfqpoint{4.947053in}{2.608163in}}%
\pgfpathlineto{\pgfqpoint{4.939376in}{2.595608in}}%
\pgfpathlineto{\pgfqpoint{4.931692in}{2.582891in}}%
\pgfpathlineto{\pgfqpoint{4.924001in}{2.570011in}}%
\pgfpathlineto{\pgfqpoint{4.916303in}{2.556971in}}%
\pgfpathclose%
\pgfusepath{fill}%
\end{pgfscope}%
\begin{pgfscope}%
\pgfpathrectangle{\pgfqpoint{1.254980in}{0.150000in}}{\pgfqpoint{5.490039in}{5.490039in}}%
\pgfusepath{clip}%
\pgfsetbuttcap%
\pgfsetroundjoin%
\definecolor{currentfill}{rgb}{0.278791,0.062145,0.386592}%
\pgfsetfillcolor{currentfill}%
\pgfsetfillopacity{0.700000}%
\pgfsetlinewidth{0.000000pt}%
\definecolor{currentstroke}{rgb}{0.000000,0.000000,0.000000}%
\pgfsetstrokecolor{currentstroke}%
\pgfsetdash{}{0pt}%
\pgfpathmoveto{\pgfqpoint{3.794475in}{1.247116in}}%
\pgfpathlineto{\pgfqpoint{3.808152in}{1.246818in}}%
\pgfpathlineto{\pgfqpoint{3.821836in}{1.246675in}}%
\pgfpathlineto{\pgfqpoint{3.835529in}{1.246687in}}%
\pgfpathlineto{\pgfqpoint{3.849230in}{1.246854in}}%
\pgfpathlineto{\pgfqpoint{3.857257in}{1.257872in}}%
\pgfpathlineto{\pgfqpoint{3.865277in}{1.269062in}}%
\pgfpathlineto{\pgfqpoint{3.873292in}{1.280417in}}%
\pgfpathlineto{\pgfqpoint{3.881301in}{1.291931in}}%
\pgfpathlineto{\pgfqpoint{3.867611in}{1.291119in}}%
\pgfpathlineto{\pgfqpoint{3.853929in}{1.290463in}}%
\pgfpathlineto{\pgfqpoint{3.840255in}{1.289962in}}%
\pgfpathlineto{\pgfqpoint{3.826590in}{1.289616in}}%
\pgfpathlineto{\pgfqpoint{3.818570in}{1.278735in}}%
\pgfpathlineto{\pgfqpoint{3.810545in}{1.268021in}}%
\pgfpathlineto{\pgfqpoint{3.802513in}{1.257479in}}%
\pgfpathlineto{\pgfqpoint{3.794475in}{1.247116in}}%
\pgfpathclose%
\pgfusepath{fill}%
\end{pgfscope}%
\begin{pgfscope}%
\pgfpathrectangle{\pgfqpoint{1.254980in}{0.150000in}}{\pgfqpoint{5.490039in}{5.490039in}}%
\pgfusepath{clip}%
\pgfsetbuttcap%
\pgfsetroundjoin%
\definecolor{currentfill}{rgb}{0.281924,0.089666,0.412415}%
\pgfsetfillcolor{currentfill}%
\pgfsetfillopacity{0.700000}%
\pgfsetlinewidth{0.000000pt}%
\definecolor{currentstroke}{rgb}{0.000000,0.000000,0.000000}%
\pgfsetstrokecolor{currentstroke}%
\pgfsetdash{}{0pt}%
\pgfpathmoveto{\pgfqpoint{3.881301in}{1.291931in}}%
\pgfpathlineto{\pgfqpoint{3.895000in}{1.292898in}}%
\pgfpathlineto{\pgfqpoint{3.908708in}{1.294020in}}%
\pgfpathlineto{\pgfqpoint{3.922425in}{1.295297in}}%
\pgfpathlineto{\pgfqpoint{3.936151in}{1.296728in}}%
\pgfpathlineto{\pgfqpoint{3.944146in}{1.309025in}}%
\pgfpathlineto{\pgfqpoint{3.952136in}{1.321461in}}%
\pgfpathlineto{\pgfqpoint{3.960121in}{1.334031in}}%
\pgfpathlineto{\pgfqpoint{3.968100in}{1.346730in}}%
\pgfpathlineto{\pgfqpoint{3.954381in}{1.344680in}}%
\pgfpathlineto{\pgfqpoint{3.940672in}{1.342785in}}%
\pgfpathlineto{\pgfqpoint{3.926972in}{1.341045in}}%
\pgfpathlineto{\pgfqpoint{3.913281in}{1.339460in}}%
\pgfpathlineto{\pgfqpoint{3.905294in}{1.327369in}}%
\pgfpathlineto{\pgfqpoint{3.897302in}{1.315414in}}%
\pgfpathlineto{\pgfqpoint{3.889304in}{1.303599in}}%
\pgfpathlineto{\pgfqpoint{3.881301in}{1.291931in}}%
\pgfpathclose%
\pgfusepath{fill}%
\end{pgfscope}%
\begin{pgfscope}%
\pgfpathrectangle{\pgfqpoint{1.254980in}{0.150000in}}{\pgfqpoint{5.490039in}{5.490039in}}%
\pgfusepath{clip}%
\pgfsetbuttcap%
\pgfsetroundjoin%
\definecolor{currentfill}{rgb}{0.274952,0.037752,0.364543}%
\pgfsetfillcolor{currentfill}%
\pgfsetfillopacity{0.700000}%
\pgfsetlinewidth{0.000000pt}%
\definecolor{currentstroke}{rgb}{0.000000,0.000000,0.000000}%
\pgfsetstrokecolor{currentstroke}%
\pgfsetdash{}{0pt}%
\pgfpathmoveto{\pgfqpoint{3.707563in}{1.213016in}}%
\pgfpathlineto{\pgfqpoint{3.721225in}{1.211424in}}%
\pgfpathlineto{\pgfqpoint{3.734894in}{1.209987in}}%
\pgfpathlineto{\pgfqpoint{3.748570in}{1.208706in}}%
\pgfpathlineto{\pgfqpoint{3.762253in}{1.207581in}}%
\pgfpathlineto{\pgfqpoint{3.770319in}{1.217164in}}%
\pgfpathlineto{\pgfqpoint{3.778378in}{1.226952in}}%
\pgfpathlineto{\pgfqpoint{3.786430in}{1.236938in}}%
\pgfpathlineto{\pgfqpoint{3.794475in}{1.247116in}}%
\pgfpathlineto{\pgfqpoint{3.780806in}{1.247570in}}%
\pgfpathlineto{\pgfqpoint{3.767144in}{1.248180in}}%
\pgfpathlineto{\pgfqpoint{3.753490in}{1.248945in}}%
\pgfpathlineto{\pgfqpoint{3.739842in}{1.249867in}}%
\pgfpathlineto{\pgfqpoint{3.731784in}{1.240350in}}%
\pgfpathlineto{\pgfqpoint{3.723718in}{1.231031in}}%
\pgfpathlineto{\pgfqpoint{3.715644in}{1.221918in}}%
\pgfpathlineto{\pgfqpoint{3.707563in}{1.213016in}}%
\pgfpathclose%
\pgfusepath{fill}%
\end{pgfscope}%
\begin{pgfscope}%
\pgfpathrectangle{\pgfqpoint{1.254980in}{0.150000in}}{\pgfqpoint{5.490039in}{5.490039in}}%
\pgfusepath{clip}%
\pgfsetbuttcap%
\pgfsetroundjoin%
\definecolor{currentfill}{rgb}{0.244972,0.287675,0.537260}%
\pgfsetfillcolor{currentfill}%
\pgfsetfillopacity{0.700000}%
\pgfsetlinewidth{0.000000pt}%
\definecolor{currentstroke}{rgb}{0.000000,0.000000,0.000000}%
\pgfsetstrokecolor{currentstroke}%
\pgfsetdash{}{0pt}%
\pgfpathmoveto{\pgfqpoint{4.292079in}{1.688160in}}%
\pgfpathlineto{\pgfqpoint{4.305941in}{1.694929in}}%
\pgfpathlineto{\pgfqpoint{4.319816in}{1.701854in}}%
\pgfpathlineto{\pgfqpoint{4.333705in}{1.708937in}}%
\pgfpathlineto{\pgfqpoint{4.347607in}{1.716175in}}%
\pgfpathlineto{\pgfqpoint{4.355499in}{1.732199in}}%
\pgfpathlineto{\pgfqpoint{4.363388in}{1.748210in}}%
\pgfpathlineto{\pgfqpoint{4.371274in}{1.764206in}}%
\pgfpathlineto{\pgfqpoint{4.379155in}{1.780183in}}%
\pgfpathlineto{\pgfqpoint{4.365249in}{1.772483in}}%
\pgfpathlineto{\pgfqpoint{4.351356in}{1.764941in}}%
\pgfpathlineto{\pgfqpoint{4.337477in}{1.757556in}}%
\pgfpathlineto{\pgfqpoint{4.323612in}{1.750329in}}%
\pgfpathlineto{\pgfqpoint{4.315734in}{1.734802in}}%
\pgfpathlineto{\pgfqpoint{4.307853in}{1.719262in}}%
\pgfpathlineto{\pgfqpoint{4.299968in}{1.703713in}}%
\pgfpathlineto{\pgfqpoint{4.292079in}{1.688160in}}%
\pgfpathclose%
\pgfusepath{fill}%
\end{pgfscope}%
\begin{pgfscope}%
\pgfpathrectangle{\pgfqpoint{1.254980in}{0.150000in}}{\pgfqpoint{5.490039in}{5.490039in}}%
\pgfusepath{clip}%
\pgfsetbuttcap%
\pgfsetroundjoin%
\definecolor{currentfill}{rgb}{0.585678,0.846661,0.249897}%
\pgfsetfillcolor{currentfill}%
\pgfsetfillopacity{0.700000}%
\pgfsetlinewidth{0.000000pt}%
\definecolor{currentstroke}{rgb}{0.000000,0.000000,0.000000}%
\pgfsetstrokecolor{currentstroke}%
\pgfsetdash{}{0pt}%
\pgfpathmoveto{\pgfqpoint{5.565114in}{3.318899in}}%
\pgfpathlineto{\pgfqpoint{5.579846in}{3.336351in}}%
\pgfpathlineto{\pgfqpoint{5.594601in}{3.353971in}}%
\pgfpathlineto{\pgfqpoint{5.609379in}{3.371762in}}%
\pgfpathlineto{\pgfqpoint{5.616682in}{3.377229in}}%
\pgfpathlineto{\pgfqpoint{5.623973in}{3.382504in}}%
\pgfpathlineto{\pgfqpoint{5.631252in}{3.387587in}}%
\pgfpathlineto{\pgfqpoint{5.638519in}{3.392480in}}%
\pgfpathlineto{\pgfqpoint{5.623752in}{3.374872in}}%
\pgfpathlineto{\pgfqpoint{5.609008in}{3.357434in}}%
\pgfpathlineto{\pgfqpoint{5.594288in}{3.340163in}}%
\pgfpathlineto{\pgfqpoint{5.587011in}{3.335124in}}%
\pgfpathlineto{\pgfqpoint{5.579724in}{3.329902in}}%
\pgfpathlineto{\pgfqpoint{5.572424in}{3.324494in}}%
\pgfpathlineto{\pgfqpoint{5.565114in}{3.318899in}}%
\pgfpathclose%
\pgfusepath{fill}%
\end{pgfscope}%
\begin{pgfscope}%
\pgfpathrectangle{\pgfqpoint{1.254980in}{0.150000in}}{\pgfqpoint{5.490039in}{5.490039in}}%
\pgfusepath{clip}%
\pgfsetbuttcap%
\pgfsetroundjoin%
\definecolor{currentfill}{rgb}{0.214298,0.355619,0.551184}%
\pgfsetfillcolor{currentfill}%
\pgfsetfillopacity{0.700000}%
\pgfsetlinewidth{0.000000pt}%
\definecolor{currentstroke}{rgb}{0.000000,0.000000,0.000000}%
\pgfsetstrokecolor{currentstroke}%
\pgfsetdash{}{0pt}%
\pgfpathmoveto{\pgfqpoint{4.410643in}{1.843806in}}%
\pgfpathlineto{\pgfqpoint{4.424569in}{1.852096in}}%
\pgfpathlineto{\pgfqpoint{4.438508in}{1.860543in}}%
\pgfpathlineto{\pgfqpoint{4.452463in}{1.869149in}}%
\pgfpathlineto{\pgfqpoint{4.466432in}{1.877913in}}%
\pgfpathlineto{\pgfqpoint{4.474300in}{1.894144in}}%
\pgfpathlineto{\pgfqpoint{4.482164in}{1.910325in}}%
\pgfpathlineto{\pgfqpoint{4.490024in}{1.926451in}}%
\pgfpathlineto{\pgfqpoint{4.497880in}{1.942519in}}%
\pgfpathlineto{\pgfqpoint{4.483905in}{1.933350in}}%
\pgfpathlineto{\pgfqpoint{4.469945in}{1.924339in}}%
\pgfpathlineto{\pgfqpoint{4.456000in}{1.915487in}}%
\pgfpathlineto{\pgfqpoint{4.442069in}{1.906793in}}%
\pgfpathlineto{\pgfqpoint{4.434219in}{1.891119in}}%
\pgfpathlineto{\pgfqpoint{4.426364in}{1.875394in}}%
\pgfpathlineto{\pgfqpoint{4.418506in}{1.859622in}}%
\pgfpathlineto{\pgfqpoint{4.410643in}{1.843806in}}%
\pgfpathclose%
\pgfusepath{fill}%
\end{pgfscope}%
\begin{pgfscope}%
\pgfpathrectangle{\pgfqpoint{1.254980in}{0.150000in}}{\pgfqpoint{5.490039in}{5.490039in}}%
\pgfusepath{clip}%
\pgfsetbuttcap%
\pgfsetroundjoin%
\definecolor{currentfill}{rgb}{0.283229,0.120777,0.440584}%
\pgfsetfillcolor{currentfill}%
\pgfsetfillopacity{0.700000}%
\pgfsetlinewidth{0.000000pt}%
\definecolor{currentstroke}{rgb}{0.000000,0.000000,0.000000}%
\pgfsetstrokecolor{currentstroke}%
\pgfsetdash{}{0pt}%
\pgfpathmoveto{\pgfqpoint{3.968100in}{1.346730in}}%
\pgfpathlineto{\pgfqpoint{3.981829in}{1.348935in}}%
\pgfpathlineto{\pgfqpoint{3.995567in}{1.351294in}}%
\pgfpathlineto{\pgfqpoint{4.009316in}{1.353809in}}%
\pgfpathlineto{\pgfqpoint{4.023074in}{1.356478in}}%
\pgfpathlineto{\pgfqpoint{4.031044in}{1.369901in}}%
\pgfpathlineto{\pgfqpoint{4.039008in}{1.383435in}}%
\pgfpathlineto{\pgfqpoint{4.046969in}{1.397072in}}%
\pgfpathlineto{\pgfqpoint{4.054925in}{1.410808in}}%
\pgfpathlineto{\pgfqpoint{4.041170in}{1.407546in}}%
\pgfpathlineto{\pgfqpoint{4.027427in}{1.404439in}}%
\pgfpathlineto{\pgfqpoint{4.013693in}{1.401487in}}%
\pgfpathlineto{\pgfqpoint{3.999969in}{1.398690in}}%
\pgfpathlineto{\pgfqpoint{3.992009in}{1.385537in}}%
\pgfpathlineto{\pgfqpoint{3.984045in}{1.372488in}}%
\pgfpathlineto{\pgfqpoint{3.976075in}{1.359551in}}%
\pgfpathlineto{\pgfqpoint{3.968100in}{1.346730in}}%
\pgfpathclose%
\pgfusepath{fill}%
\end{pgfscope}%
\begin{pgfscope}%
\pgfpathrectangle{\pgfqpoint{1.254980in}{0.150000in}}{\pgfqpoint{5.490039in}{5.490039in}}%
\pgfusepath{clip}%
\pgfsetbuttcap%
\pgfsetroundjoin%
\definecolor{currentfill}{rgb}{0.395174,0.797475,0.367757}%
\pgfsetfillcolor{currentfill}%
\pgfsetfillopacity{0.700000}%
\pgfsetlinewidth{0.000000pt}%
\definecolor{currentstroke}{rgb}{0.000000,0.000000,0.000000}%
\pgfsetstrokecolor{currentstroke}%
\pgfsetdash{}{0pt}%
\pgfpathmoveto{\pgfqpoint{5.359281in}{3.100107in}}%
\pgfpathlineto{\pgfqpoint{5.373854in}{3.116558in}}%
\pgfpathlineto{\pgfqpoint{5.388449in}{3.133176in}}%
\pgfpathlineto{\pgfqpoint{5.403067in}{3.149963in}}%
\pgfpathlineto{\pgfqpoint{5.417707in}{3.166919in}}%
\pgfpathlineto{\pgfqpoint{5.425159in}{3.174979in}}%
\pgfpathlineto{\pgfqpoint{5.432600in}{3.182840in}}%
\pgfpathlineto{\pgfqpoint{5.440031in}{3.190502in}}%
\pgfpathlineto{\pgfqpoint{5.447451in}{3.197968in}}%
\pgfpathlineto{\pgfqpoint{5.432816in}{3.181095in}}%
\pgfpathlineto{\pgfqpoint{5.418204in}{3.164391in}}%
\pgfpathlineto{\pgfqpoint{5.403615in}{3.147855in}}%
\pgfpathlineto{\pgfqpoint{5.389047in}{3.131486in}}%
\pgfpathlineto{\pgfqpoint{5.381621in}{3.123926in}}%
\pgfpathlineto{\pgfqpoint{5.374184in}{3.116177in}}%
\pgfpathlineto{\pgfqpoint{5.366738in}{3.108238in}}%
\pgfpathlineto{\pgfqpoint{5.359281in}{3.100107in}}%
\pgfpathclose%
\pgfusepath{fill}%
\end{pgfscope}%
\begin{pgfscope}%
\pgfpathrectangle{\pgfqpoint{1.254980in}{0.150000in}}{\pgfqpoint{5.490039in}{5.490039in}}%
\pgfusepath{clip}%
\pgfsetbuttcap%
\pgfsetroundjoin%
\definecolor{currentfill}{rgb}{0.269308,0.218818,0.509577}%
\pgfsetfillcolor{currentfill}%
\pgfsetfillopacity{0.700000}%
\pgfsetlinewidth{0.000000pt}%
\definecolor{currentstroke}{rgb}{0.000000,0.000000,0.000000}%
\pgfsetstrokecolor{currentstroke}%
\pgfsetdash{}{0pt}%
\pgfpathmoveto{\pgfqpoint{4.173531in}{1.542604in}}%
\pgfpathlineto{\pgfqpoint{4.187339in}{1.547748in}}%
\pgfpathlineto{\pgfqpoint{4.201158in}{1.553048in}}%
\pgfpathlineto{\pgfqpoint{4.214990in}{1.558503in}}%
\pgfpathlineto{\pgfqpoint{4.228835in}{1.564114in}}%
\pgfpathlineto{\pgfqpoint{4.236753in}{1.579540in}}%
\pgfpathlineto{\pgfqpoint{4.244668in}{1.594998in}}%
\pgfpathlineto{\pgfqpoint{4.252579in}{1.610484in}}%
\pgfpathlineto{\pgfqpoint{4.260487in}{1.625993in}}%
\pgfpathlineto{\pgfqpoint{4.246640in}{1.619867in}}%
\pgfpathlineto{\pgfqpoint{4.232807in}{1.613898in}}%
\pgfpathlineto{\pgfqpoint{4.218986in}{1.608085in}}%
\pgfpathlineto{\pgfqpoint{4.205177in}{1.602428in}}%
\pgfpathlineto{\pgfqpoint{4.197271in}{1.587423in}}%
\pgfpathlineto{\pgfqpoint{4.189362in}{1.572447in}}%
\pgfpathlineto{\pgfqpoint{4.181448in}{1.557506in}}%
\pgfpathlineto{\pgfqpoint{4.173531in}{1.542604in}}%
\pgfpathclose%
\pgfusepath{fill}%
\end{pgfscope}%
\begin{pgfscope}%
\pgfpathrectangle{\pgfqpoint{1.254980in}{0.150000in}}{\pgfqpoint{5.490039in}{5.490039in}}%
\pgfusepath{clip}%
\pgfsetbuttcap%
\pgfsetroundjoin%
\definecolor{currentfill}{rgb}{0.185556,0.418570,0.556753}%
\pgfsetfillcolor{currentfill}%
\pgfsetfillopacity{0.700000}%
\pgfsetlinewidth{0.000000pt}%
\definecolor{currentstroke}{rgb}{0.000000,0.000000,0.000000}%
\pgfsetstrokecolor{currentstroke}%
\pgfsetdash{}{0pt}%
\pgfpathmoveto{\pgfqpoint{4.529263in}{2.006139in}}%
\pgfpathlineto{\pgfqpoint{4.543260in}{2.015845in}}%
\pgfpathlineto{\pgfqpoint{4.557272in}{2.025710in}}%
\pgfpathlineto{\pgfqpoint{4.571300in}{2.035735in}}%
\pgfpathlineto{\pgfqpoint{4.585343in}{2.045919in}}%
\pgfpathlineto{\pgfqpoint{4.593185in}{2.062004in}}%
\pgfpathlineto{\pgfqpoint{4.601022in}{2.078003in}}%
\pgfpathlineto{\pgfqpoint{4.608855in}{2.093914in}}%
\pgfpathlineto{\pgfqpoint{4.616684in}{2.109732in}}%
\pgfpathlineto{\pgfqpoint{4.602633in}{2.099198in}}%
\pgfpathlineto{\pgfqpoint{4.588599in}{2.088824in}}%
\pgfpathlineto{\pgfqpoint{4.574580in}{2.078610in}}%
\pgfpathlineto{\pgfqpoint{4.560578in}{2.068556in}}%
\pgfpathlineto{\pgfqpoint{4.552756in}{2.053076in}}%
\pgfpathlineto{\pgfqpoint{4.544929in}{2.037511in}}%
\pgfpathlineto{\pgfqpoint{4.537098in}{2.021864in}}%
\pgfpathlineto{\pgfqpoint{4.529263in}{2.006139in}}%
\pgfpathclose%
\pgfusepath{fill}%
\end{pgfscope}%
\begin{pgfscope}%
\pgfpathrectangle{\pgfqpoint{1.254980in}{0.150000in}}{\pgfqpoint{5.490039in}{5.490039in}}%
\pgfusepath{clip}%
\pgfsetbuttcap%
\pgfsetroundjoin%
\definecolor{currentfill}{rgb}{0.272594,0.025563,0.353093}%
\pgfsetfillcolor{currentfill}%
\pgfsetfillopacity{0.700000}%
\pgfsetlinewidth{0.000000pt}%
\definecolor{currentstroke}{rgb}{0.000000,0.000000,0.000000}%
\pgfsetstrokecolor{currentstroke}%
\pgfsetdash{}{0pt}%
\pgfpathmoveto{\pgfqpoint{3.620503in}{1.190396in}}%
\pgfpathlineto{\pgfqpoint{3.634158in}{1.187478in}}%
\pgfpathlineto{\pgfqpoint{3.647819in}{1.184717in}}%
\pgfpathlineto{\pgfqpoint{3.661487in}{1.182113in}}%
\pgfpathlineto{\pgfqpoint{3.675160in}{1.179665in}}%
\pgfpathlineto{\pgfqpoint{3.683273in}{1.187651in}}%
\pgfpathlineto{\pgfqpoint{3.691378in}{1.195876in}}%
\pgfpathlineto{\pgfqpoint{3.699474in}{1.204334in}}%
\pgfpathlineto{\pgfqpoint{3.707563in}{1.213016in}}%
\pgfpathlineto{\pgfqpoint{3.693908in}{1.214765in}}%
\pgfpathlineto{\pgfqpoint{3.680259in}{1.216671in}}%
\pgfpathlineto{\pgfqpoint{3.666616in}{1.218733in}}%
\pgfpathlineto{\pgfqpoint{3.652980in}{1.220953in}}%
\pgfpathlineto{\pgfqpoint{3.644874in}{1.212958in}}%
\pgfpathlineto{\pgfqpoint{3.636759in}{1.205195in}}%
\pgfpathlineto{\pgfqpoint{3.628635in}{1.197672in}}%
\pgfpathlineto{\pgfqpoint{3.620503in}{1.190396in}}%
\pgfpathclose%
\pgfusepath{fill}%
\end{pgfscope}%
\begin{pgfscope}%
\pgfpathrectangle{\pgfqpoint{1.254980in}{0.150000in}}{\pgfqpoint{5.490039in}{5.490039in}}%
\pgfusepath{clip}%
\pgfsetbuttcap%
\pgfsetroundjoin%
\definecolor{currentfill}{rgb}{0.159194,0.482237,0.558073}%
\pgfsetfillcolor{currentfill}%
\pgfsetfillopacity{0.700000}%
\pgfsetlinewidth{0.000000pt}%
\definecolor{currentstroke}{rgb}{0.000000,0.000000,0.000000}%
\pgfsetstrokecolor{currentstroke}%
\pgfsetdash{}{0pt}%
\pgfpathmoveto{\pgfqpoint{4.647950in}{2.172024in}}%
\pgfpathlineto{\pgfqpoint{4.662024in}{2.183039in}}%
\pgfpathlineto{\pgfqpoint{4.676114in}{2.194215in}}%
\pgfpathlineto{\pgfqpoint{4.690221in}{2.205552in}}%
\pgfpathlineto{\pgfqpoint{4.704345in}{2.217050in}}%
\pgfpathlineto{\pgfqpoint{4.712157in}{2.232667in}}%
\pgfpathlineto{\pgfqpoint{4.719963in}{2.248167in}}%
\pgfpathlineto{\pgfqpoint{4.727764in}{2.263549in}}%
\pgfpathlineto{\pgfqpoint{4.735560in}{2.278809in}}%
\pgfpathlineto{\pgfqpoint{4.721429in}{2.267019in}}%
\pgfpathlineto{\pgfqpoint{4.707315in}{2.255391in}}%
\pgfpathlineto{\pgfqpoint{4.693219in}{2.243924in}}%
\pgfpathlineto{\pgfqpoint{4.679139in}{2.232618in}}%
\pgfpathlineto{\pgfqpoint{4.671349in}{2.217638in}}%
\pgfpathlineto{\pgfqpoint{4.663555in}{2.202544in}}%
\pgfpathlineto{\pgfqpoint{4.655755in}{2.187339in}}%
\pgfpathlineto{\pgfqpoint{4.647950in}{2.172024in}}%
\pgfpathclose%
\pgfusepath{fill}%
\end{pgfscope}%
\begin{pgfscope}%
\pgfpathrectangle{\pgfqpoint{1.254980in}{0.150000in}}{\pgfqpoint{5.490039in}{5.490039in}}%
\pgfusepath{clip}%
\pgfsetbuttcap%
\pgfsetroundjoin%
\definecolor{currentfill}{rgb}{0.280868,0.160771,0.472899}%
\pgfsetfillcolor{currentfill}%
\pgfsetfillopacity{0.700000}%
\pgfsetlinewidth{0.000000pt}%
\definecolor{currentstroke}{rgb}{0.000000,0.000000,0.000000}%
\pgfsetstrokecolor{currentstroke}%
\pgfsetdash{}{0pt}%
\pgfpathmoveto{\pgfqpoint{4.054925in}{1.410808in}}%
\pgfpathlineto{\pgfqpoint{4.068690in}{1.414224in}}%
\pgfpathlineto{\pgfqpoint{4.082465in}{1.417796in}}%
\pgfpathlineto{\pgfqpoint{4.096252in}{1.421522in}}%
\pgfpathlineto{\pgfqpoint{4.110049in}{1.425403in}}%
\pgfpathlineto{\pgfqpoint{4.117999in}{1.439809in}}%
\pgfpathlineto{\pgfqpoint{4.125944in}{1.454295in}}%
\pgfpathlineto{\pgfqpoint{4.133885in}{1.468856in}}%
\pgfpathlineto{\pgfqpoint{4.141822in}{1.483487in}}%
\pgfpathlineto{\pgfqpoint{4.128026in}{1.479039in}}%
\pgfpathlineto{\pgfqpoint{4.114241in}{1.474745in}}%
\pgfpathlineto{\pgfqpoint{4.100467in}{1.470607in}}%
\pgfpathlineto{\pgfqpoint{4.086705in}{1.466624in}}%
\pgfpathlineto{\pgfqpoint{4.078766in}{1.452550in}}%
\pgfpathlineto{\pgfqpoint{4.070824in}{1.438552in}}%
\pgfpathlineto{\pgfqpoint{4.062876in}{1.424636in}}%
\pgfpathlineto{\pgfqpoint{4.054925in}{1.410808in}}%
\pgfpathclose%
\pgfusepath{fill}%
\end{pgfscope}%
\begin{pgfscope}%
\pgfpathrectangle{\pgfqpoint{1.254980in}{0.150000in}}{\pgfqpoint{5.490039in}{5.490039in}}%
\pgfusepath{clip}%
\pgfsetbuttcap%
\pgfsetroundjoin%
\definecolor{currentfill}{rgb}{0.288921,0.758394,0.428426}%
\pgfsetfillcolor{currentfill}%
\pgfsetfillopacity{0.700000}%
\pgfsetlinewidth{0.000000pt}%
\definecolor{currentstroke}{rgb}{0.000000,0.000000,0.000000}%
\pgfsetstrokecolor{currentstroke}%
\pgfsetdash{}{0pt}%
\pgfpathmoveto{\pgfqpoint{5.241161in}{2.963649in}}%
\pgfpathlineto{\pgfqpoint{5.255654in}{2.979497in}}%
\pgfpathlineto{\pgfqpoint{5.270168in}{2.995513in}}%
\pgfpathlineto{\pgfqpoint{5.284703in}{3.011696in}}%
\pgfpathlineto{\pgfqpoint{5.299261in}{3.028047in}}%
\pgfpathlineto{\pgfqpoint{5.306798in}{3.037743in}}%
\pgfpathlineto{\pgfqpoint{5.314326in}{3.047242in}}%
\pgfpathlineto{\pgfqpoint{5.321843in}{3.056543in}}%
\pgfpathlineto{\pgfqpoint{5.329351in}{3.065647in}}%
\pgfpathlineto{\pgfqpoint{5.314796in}{3.049314in}}%
\pgfpathlineto{\pgfqpoint{5.300262in}{3.033148in}}%
\pgfpathlineto{\pgfqpoint{5.285751in}{3.017150in}}%
\pgfpathlineto{\pgfqpoint{5.271260in}{3.001319in}}%
\pgfpathlineto{\pgfqpoint{5.263750in}{2.992186in}}%
\pgfpathlineto{\pgfqpoint{5.256230in}{2.982863in}}%
\pgfpathlineto{\pgfqpoint{5.248700in}{2.973351in}}%
\pgfpathlineto{\pgfqpoint{5.241161in}{2.963649in}}%
\pgfpathclose%
\pgfusepath{fill}%
\end{pgfscope}%
\begin{pgfscope}%
\pgfpathrectangle{\pgfqpoint{1.254980in}{0.150000in}}{\pgfqpoint{5.490039in}{5.490039in}}%
\pgfusepath{clip}%
\pgfsetbuttcap%
\pgfsetroundjoin%
\definecolor{currentfill}{rgb}{0.135066,0.544853,0.554029}%
\pgfsetfillcolor{currentfill}%
\pgfsetfillopacity{0.700000}%
\pgfsetlinewidth{0.000000pt}%
\definecolor{currentstroke}{rgb}{0.000000,0.000000,0.000000}%
\pgfsetstrokecolor{currentstroke}%
\pgfsetdash{}{0pt}%
\pgfpathmoveto{\pgfqpoint{4.766689in}{2.338588in}}%
\pgfpathlineto{\pgfqpoint{4.780844in}{2.350802in}}%
\pgfpathlineto{\pgfqpoint{4.795017in}{2.363178in}}%
\pgfpathlineto{\pgfqpoint{4.809207in}{2.375717in}}%
\pgfpathlineto{\pgfqpoint{4.823416in}{2.388419in}}%
\pgfpathlineto{\pgfqpoint{4.831191in}{2.403281in}}%
\pgfpathlineto{\pgfqpoint{4.838960in}{2.418001in}}%
\pgfpathlineto{\pgfqpoint{4.846722in}{2.432576in}}%
\pgfpathlineto{\pgfqpoint{4.854479in}{2.447005in}}%
\pgfpathlineto{\pgfqpoint{4.840264in}{2.434070in}}%
\pgfpathlineto{\pgfqpoint{4.826067in}{2.421299in}}%
\pgfpathlineto{\pgfqpoint{4.811889in}{2.408690in}}%
\pgfpathlineto{\pgfqpoint{4.797729in}{2.396244in}}%
\pgfpathlineto{\pgfqpoint{4.789977in}{2.382036in}}%
\pgfpathlineto{\pgfqpoint{4.782221in}{2.367690in}}%
\pgfpathlineto{\pgfqpoint{4.774458in}{2.353206in}}%
\pgfpathlineto{\pgfqpoint{4.766689in}{2.338588in}}%
\pgfpathclose%
\pgfusepath{fill}%
\end{pgfscope}%
\begin{pgfscope}%
\pgfpathrectangle{\pgfqpoint{1.254980in}{0.150000in}}{\pgfqpoint{5.490039in}{5.490039in}}%
\pgfusepath{clip}%
\pgfsetbuttcap%
\pgfsetroundjoin%
\definecolor{currentfill}{rgb}{0.202219,0.715272,0.476084}%
\pgfsetfillcolor{currentfill}%
\pgfsetfillopacity{0.700000}%
\pgfsetlinewidth{0.000000pt}%
\definecolor{currentstroke}{rgb}{0.000000,0.000000,0.000000}%
\pgfsetstrokecolor{currentstroke}%
\pgfsetdash{}{0pt}%
\pgfpathmoveto{\pgfqpoint{5.122756in}{2.817660in}}%
\pgfpathlineto{\pgfqpoint{5.137165in}{2.832780in}}%
\pgfpathlineto{\pgfqpoint{5.151595in}{2.848067in}}%
\pgfpathlineto{\pgfqpoint{5.166046in}{2.863521in}}%
\pgfpathlineto{\pgfqpoint{5.180517in}{2.879141in}}%
\pgfpathlineto{\pgfqpoint{5.188129in}{2.890375in}}%
\pgfpathlineto{\pgfqpoint{5.195733in}{2.901418in}}%
\pgfpathlineto{\pgfqpoint{5.203327in}{2.912269in}}%
\pgfpathlineto{\pgfqpoint{5.210912in}{2.922929in}}%
\pgfpathlineto{\pgfqpoint{5.196440in}{2.907262in}}%
\pgfpathlineto{\pgfqpoint{5.181988in}{2.891762in}}%
\pgfpathlineto{\pgfqpoint{5.167558in}{2.876428in}}%
\pgfpathlineto{\pgfqpoint{5.153148in}{2.861261in}}%
\pgfpathlineto{\pgfqpoint{5.145563in}{2.850636in}}%
\pgfpathlineto{\pgfqpoint{5.137969in}{2.839828in}}%
\pgfpathlineto{\pgfqpoint{5.130367in}{2.828836in}}%
\pgfpathlineto{\pgfqpoint{5.122756in}{2.817660in}}%
\pgfpathclose%
\pgfusepath{fill}%
\end{pgfscope}%
\begin{pgfscope}%
\pgfpathrectangle{\pgfqpoint{1.254980in}{0.150000in}}{\pgfqpoint{5.490039in}{5.490039in}}%
\pgfusepath{clip}%
\pgfsetbuttcap%
\pgfsetroundjoin%
\definecolor{currentfill}{rgb}{0.119738,0.603785,0.541400}%
\pgfsetfillcolor{currentfill}%
\pgfsetfillopacity{0.700000}%
\pgfsetlinewidth{0.000000pt}%
\definecolor{currentstroke}{rgb}{0.000000,0.000000,0.000000}%
\pgfsetstrokecolor{currentstroke}%
\pgfsetdash{}{0pt}%
\pgfpathmoveto{\pgfqpoint{4.885443in}{2.503223in}}%
\pgfpathlineto{\pgfqpoint{4.899682in}{2.516524in}}%
\pgfpathlineto{\pgfqpoint{4.913940in}{2.529988in}}%
\pgfpathlineto{\pgfqpoint{4.928216in}{2.543616in}}%
\pgfpathlineto{\pgfqpoint{4.942512in}{2.557408in}}%
\pgfpathlineto{\pgfqpoint{4.950242in}{2.571263in}}%
\pgfpathlineto{\pgfqpoint{4.957966in}{2.584955in}}%
\pgfpathlineto{\pgfqpoint{4.965682in}{2.598482in}}%
\pgfpathlineto{\pgfqpoint{4.973391in}{2.611843in}}%
\pgfpathlineto{\pgfqpoint{4.959090in}{2.597878in}}%
\pgfpathlineto{\pgfqpoint{4.944809in}{2.584078in}}%
\pgfpathlineto{\pgfqpoint{4.930546in}{2.570443in}}%
\pgfpathlineto{\pgfqpoint{4.916303in}{2.556971in}}%
\pgfpathlineto{\pgfqpoint{4.908598in}{2.543771in}}%
\pgfpathlineto{\pgfqpoint{4.900886in}{2.530412in}}%
\pgfpathlineto{\pgfqpoint{4.893168in}{2.516896in}}%
\pgfpathlineto{\pgfqpoint{4.885443in}{2.503223in}}%
\pgfpathclose%
\pgfusepath{fill}%
\end{pgfscope}%
\begin{pgfscope}%
\pgfpathrectangle{\pgfqpoint{1.254980in}{0.150000in}}{\pgfqpoint{5.490039in}{5.490039in}}%
\pgfusepath{clip}%
\pgfsetbuttcap%
\pgfsetroundjoin%
\definecolor{currentfill}{rgb}{0.137339,0.662252,0.515571}%
\pgfsetfillcolor{currentfill}%
\pgfsetfillopacity{0.700000}%
\pgfsetlinewidth{0.000000pt}%
\definecolor{currentstroke}{rgb}{0.000000,0.000000,0.000000}%
\pgfsetstrokecolor{currentstroke}%
\pgfsetdash{}{0pt}%
\pgfpathmoveto{\pgfqpoint{5.004156in}{2.663598in}}%
\pgfpathlineto{\pgfqpoint{5.018480in}{2.677868in}}%
\pgfpathlineto{\pgfqpoint{5.032824in}{2.692303in}}%
\pgfpathlineto{\pgfqpoint{5.047188in}{2.706904in}}%
\pgfpathlineto{\pgfqpoint{5.061572in}{2.721670in}}%
\pgfpathlineto{\pgfqpoint{5.069248in}{2.734304in}}%
\pgfpathlineto{\pgfqpoint{5.076916in}{2.746758in}}%
\pgfpathlineto{\pgfqpoint{5.084576in}{2.759031in}}%
\pgfpathlineto{\pgfqpoint{5.092229in}{2.771122in}}%
\pgfpathlineto{\pgfqpoint{5.077842in}{2.756245in}}%
\pgfpathlineto{\pgfqpoint{5.063475in}{2.741535in}}%
\pgfpathlineto{\pgfqpoint{5.049128in}{2.726990in}}%
\pgfpathlineto{\pgfqpoint{5.034801in}{2.712610in}}%
\pgfpathlineto{\pgfqpoint{5.027151in}{2.700617in}}%
\pgfpathlineto{\pgfqpoint{5.019494in}{2.688450in}}%
\pgfpathlineto{\pgfqpoint{5.011828in}{2.676110in}}%
\pgfpathlineto{\pgfqpoint{5.004156in}{2.663598in}}%
\pgfpathclose%
\pgfusepath{fill}%
\end{pgfscope}%
\begin{pgfscope}%
\pgfpathrectangle{\pgfqpoint{1.254980in}{0.150000in}}{\pgfqpoint{5.490039in}{5.490039in}}%
\pgfusepath{clip}%
\pgfsetbuttcap%
\pgfsetroundjoin%
\definecolor{currentfill}{rgb}{0.253935,0.265254,0.529983}%
\pgfsetfillcolor{currentfill}%
\pgfsetfillopacity{0.700000}%
\pgfsetlinewidth{0.000000pt}%
\definecolor{currentstroke}{rgb}{0.000000,0.000000,0.000000}%
\pgfsetstrokecolor{currentstroke}%
\pgfsetdash{}{0pt}%
\pgfpathmoveto{\pgfqpoint{4.260487in}{1.625993in}}%
\pgfpathlineto{\pgfqpoint{4.274346in}{1.632274in}}%
\pgfpathlineto{\pgfqpoint{4.288218in}{1.638712in}}%
\pgfpathlineto{\pgfqpoint{4.302103in}{1.645306in}}%
\pgfpathlineto{\pgfqpoint{4.316001in}{1.652056in}}%
\pgfpathlineto{\pgfqpoint{4.323908in}{1.668081in}}%
\pgfpathlineto{\pgfqpoint{4.331811in}{1.684112in}}%
\pgfpathlineto{\pgfqpoint{4.339711in}{1.700145in}}%
\pgfpathlineto{\pgfqpoint{4.347607in}{1.716175in}}%
\pgfpathlineto{\pgfqpoint{4.333705in}{1.708937in}}%
\pgfpathlineto{\pgfqpoint{4.319816in}{1.701854in}}%
\pgfpathlineto{\pgfqpoint{4.305941in}{1.694929in}}%
\pgfpathlineto{\pgfqpoint{4.292079in}{1.688160in}}%
\pgfpathlineto{\pgfqpoint{4.284187in}{1.672608in}}%
\pgfpathlineto{\pgfqpoint{4.276290in}{1.657059in}}%
\pgfpathlineto{\pgfqpoint{4.268390in}{1.641519in}}%
\pgfpathlineto{\pgfqpoint{4.260487in}{1.625993in}}%
\pgfpathclose%
\pgfusepath{fill}%
\end{pgfscope}%
\begin{pgfscope}%
\pgfpathrectangle{\pgfqpoint{1.254980in}{0.150000in}}{\pgfqpoint{5.490039in}{5.490039in}}%
\pgfusepath{clip}%
\pgfsetbuttcap%
\pgfsetroundjoin%
\definecolor{currentfill}{rgb}{0.223925,0.334994,0.548053}%
\pgfsetfillcolor{currentfill}%
\pgfsetfillopacity{0.700000}%
\pgfsetlinewidth{0.000000pt}%
\definecolor{currentstroke}{rgb}{0.000000,0.000000,0.000000}%
\pgfsetstrokecolor{currentstroke}%
\pgfsetdash{}{0pt}%
\pgfpathmoveto{\pgfqpoint{4.379155in}{1.780183in}}%
\pgfpathlineto{\pgfqpoint{4.393075in}{1.788039in}}%
\pgfpathlineto{\pgfqpoint{4.407010in}{1.796054in}}%
\pgfpathlineto{\pgfqpoint{4.420959in}{1.804225in}}%
\pgfpathlineto{\pgfqpoint{4.434923in}{1.812555in}}%
\pgfpathlineto{\pgfqpoint{4.442806in}{1.828951in}}%
\pgfpathlineto{\pgfqpoint{4.450685in}{1.845312in}}%
\pgfpathlineto{\pgfqpoint{4.458560in}{1.861634in}}%
\pgfpathlineto{\pgfqpoint{4.466432in}{1.877913in}}%
\pgfpathlineto{\pgfqpoint{4.452463in}{1.869149in}}%
\pgfpathlineto{\pgfqpoint{4.438508in}{1.860543in}}%
\pgfpathlineto{\pgfqpoint{4.424569in}{1.852096in}}%
\pgfpathlineto{\pgfqpoint{4.410643in}{1.843806in}}%
\pgfpathlineto{\pgfqpoint{4.402777in}{1.827950in}}%
\pgfpathlineto{\pgfqpoint{4.394907in}{1.812058in}}%
\pgfpathlineto{\pgfqpoint{4.387033in}{1.796135in}}%
\pgfpathlineto{\pgfqpoint{4.379155in}{1.780183in}}%
\pgfpathclose%
\pgfusepath{fill}%
\end{pgfscope}%
\begin{pgfscope}%
\pgfpathrectangle{\pgfqpoint{1.254980in}{0.150000in}}{\pgfqpoint{5.490039in}{5.490039in}}%
\pgfusepath{clip}%
\pgfsetbuttcap%
\pgfsetroundjoin%
\definecolor{currentfill}{rgb}{0.280267,0.073417,0.397163}%
\pgfsetfillcolor{currentfill}%
\pgfsetfillopacity{0.700000}%
\pgfsetlinewidth{0.000000pt}%
\definecolor{currentstroke}{rgb}{0.000000,0.000000,0.000000}%
\pgfsetstrokecolor{currentstroke}%
\pgfsetdash{}{0pt}%
\pgfpathmoveto{\pgfqpoint{3.849230in}{1.246854in}}%
\pgfpathlineto{\pgfqpoint{3.862939in}{1.247176in}}%
\pgfpathlineto{\pgfqpoint{3.876656in}{1.247652in}}%
\pgfpathlineto{\pgfqpoint{3.890382in}{1.248283in}}%
\pgfpathlineto{\pgfqpoint{3.904116in}{1.249068in}}%
\pgfpathlineto{\pgfqpoint{3.912133in}{1.260742in}}%
\pgfpathlineto{\pgfqpoint{3.920145in}{1.272581in}}%
\pgfpathlineto{\pgfqpoint{3.928151in}{1.284578in}}%
\pgfpathlineto{\pgfqpoint{3.936151in}{1.296728in}}%
\pgfpathlineto{\pgfqpoint{3.922425in}{1.295297in}}%
\pgfpathlineto{\pgfqpoint{3.908708in}{1.294020in}}%
\pgfpathlineto{\pgfqpoint{3.895000in}{1.292898in}}%
\pgfpathlineto{\pgfqpoint{3.881301in}{1.291931in}}%
\pgfpathlineto{\pgfqpoint{3.873292in}{1.280417in}}%
\pgfpathlineto{\pgfqpoint{3.865277in}{1.269062in}}%
\pgfpathlineto{\pgfqpoint{3.857257in}{1.257872in}}%
\pgfpathlineto{\pgfqpoint{3.849230in}{1.246854in}}%
\pgfpathclose%
\pgfusepath{fill}%
\end{pgfscope}%
\begin{pgfscope}%
\pgfpathrectangle{\pgfqpoint{1.254980in}{0.150000in}}{\pgfqpoint{5.490039in}{5.490039in}}%
\pgfusepath{clip}%
\pgfsetbuttcap%
\pgfsetroundjoin%
\definecolor{currentfill}{rgb}{0.496615,0.826376,0.306377}%
\pgfsetfillcolor{currentfill}%
\pgfsetfillopacity{0.700000}%
\pgfsetlinewidth{0.000000pt}%
\definecolor{currentstroke}{rgb}{0.000000,0.000000,0.000000}%
\pgfsetstrokecolor{currentstroke}%
\pgfsetdash{}{0pt}%
\pgfpathmoveto{\pgfqpoint{5.447451in}{3.197968in}}%
\pgfpathlineto{\pgfqpoint{5.462107in}{3.215008in}}%
\pgfpathlineto{\pgfqpoint{5.476787in}{3.232218in}}%
\pgfpathlineto{\pgfqpoint{5.491490in}{3.249597in}}%
\pgfpathlineto{\pgfqpoint{5.506215in}{3.267145in}}%
\pgfpathlineto{\pgfqpoint{5.513617in}{3.274311in}}%
\pgfpathlineto{\pgfqpoint{5.521008in}{3.281274in}}%
\pgfpathlineto{\pgfqpoint{5.528388in}{3.288037in}}%
\pgfpathlineto{\pgfqpoint{5.535756in}{3.294600in}}%
\pgfpathlineto{\pgfqpoint{5.521038in}{3.277169in}}%
\pgfpathlineto{\pgfqpoint{5.506343in}{3.259906in}}%
\pgfpathlineto{\pgfqpoint{5.491671in}{3.242813in}}%
\pgfpathlineto{\pgfqpoint{5.477021in}{3.225887in}}%
\pgfpathlineto{\pgfqpoint{5.469645in}{3.219196in}}%
\pgfpathlineto{\pgfqpoint{5.462258in}{3.212313in}}%
\pgfpathlineto{\pgfqpoint{5.454860in}{3.205237in}}%
\pgfpathlineto{\pgfqpoint{5.447451in}{3.197968in}}%
\pgfpathclose%
\pgfusepath{fill}%
\end{pgfscope}%
\begin{pgfscope}%
\pgfpathrectangle{\pgfqpoint{1.254980in}{0.150000in}}{\pgfqpoint{5.490039in}{5.490039in}}%
\pgfusepath{clip}%
\pgfsetbuttcap%
\pgfsetroundjoin%
\definecolor{currentfill}{rgb}{0.277018,0.050344,0.375715}%
\pgfsetfillcolor{currentfill}%
\pgfsetfillopacity{0.700000}%
\pgfsetlinewidth{0.000000pt}%
\definecolor{currentstroke}{rgb}{0.000000,0.000000,0.000000}%
\pgfsetstrokecolor{currentstroke}%
\pgfsetdash{}{0pt}%
\pgfpathmoveto{\pgfqpoint{3.762253in}{1.207581in}}%
\pgfpathlineto{\pgfqpoint{3.775943in}{1.206611in}}%
\pgfpathlineto{\pgfqpoint{3.789641in}{1.205795in}}%
\pgfpathlineto{\pgfqpoint{3.803346in}{1.205135in}}%
\pgfpathlineto{\pgfqpoint{3.817059in}{1.204629in}}%
\pgfpathlineto{\pgfqpoint{3.825111in}{1.214895in}}%
\pgfpathlineto{\pgfqpoint{3.833157in}{1.225359in}}%
\pgfpathlineto{\pgfqpoint{3.841197in}{1.236014in}}%
\pgfpathlineto{\pgfqpoint{3.849230in}{1.246854in}}%
\pgfpathlineto{\pgfqpoint{3.835529in}{1.246687in}}%
\pgfpathlineto{\pgfqpoint{3.821836in}{1.246675in}}%
\pgfpathlineto{\pgfqpoint{3.808152in}{1.246818in}}%
\pgfpathlineto{\pgfqpoint{3.794475in}{1.247116in}}%
\pgfpathlineto{\pgfqpoint{3.786430in}{1.236938in}}%
\pgfpathlineto{\pgfqpoint{3.778378in}{1.226952in}}%
\pgfpathlineto{\pgfqpoint{3.770319in}{1.217164in}}%
\pgfpathlineto{\pgfqpoint{3.762253in}{1.207581in}}%
\pgfpathclose%
\pgfusepath{fill}%
\end{pgfscope}%
\begin{pgfscope}%
\pgfpathrectangle{\pgfqpoint{1.254980in}{0.150000in}}{\pgfqpoint{5.490039in}{5.490039in}}%
\pgfusepath{clip}%
\pgfsetbuttcap%
\pgfsetroundjoin%
\definecolor{currentfill}{rgb}{0.274128,0.199721,0.498911}%
\pgfsetfillcolor{currentfill}%
\pgfsetfillopacity{0.700000}%
\pgfsetlinewidth{0.000000pt}%
\definecolor{currentstroke}{rgb}{0.000000,0.000000,0.000000}%
\pgfsetstrokecolor{currentstroke}%
\pgfsetdash{}{0pt}%
\pgfpathmoveto{\pgfqpoint{4.141822in}{1.483487in}}%
\pgfpathlineto{\pgfqpoint{4.155630in}{1.488091in}}%
\pgfpathlineto{\pgfqpoint{4.169449in}{1.492850in}}%
\pgfpathlineto{\pgfqpoint{4.183280in}{1.497763in}}%
\pgfpathlineto{\pgfqpoint{4.197123in}{1.502832in}}%
\pgfpathlineto{\pgfqpoint{4.205057in}{1.518079in}}%
\pgfpathlineto{\pgfqpoint{4.212987in}{1.533379in}}%
\pgfpathlineto{\pgfqpoint{4.220912in}{1.548725in}}%
\pgfpathlineto{\pgfqpoint{4.228835in}{1.564114in}}%
\pgfpathlineto{\pgfqpoint{4.214990in}{1.558503in}}%
\pgfpathlineto{\pgfqpoint{4.201158in}{1.553048in}}%
\pgfpathlineto{\pgfqpoint{4.187339in}{1.547748in}}%
\pgfpathlineto{\pgfqpoint{4.173531in}{1.542604in}}%
\pgfpathlineto{\pgfqpoint{4.165610in}{1.527746in}}%
\pgfpathlineto{\pgfqpoint{4.157684in}{1.512937in}}%
\pgfpathlineto{\pgfqpoint{4.149755in}{1.498182in}}%
\pgfpathlineto{\pgfqpoint{4.141822in}{1.483487in}}%
\pgfpathclose%
\pgfusepath{fill}%
\end{pgfscope}%
\begin{pgfscope}%
\pgfpathrectangle{\pgfqpoint{1.254980in}{0.150000in}}{\pgfqpoint{5.490039in}{5.490039in}}%
\pgfusepath{clip}%
\pgfsetbuttcap%
\pgfsetroundjoin%
\definecolor{currentfill}{rgb}{0.194100,0.399323,0.555565}%
\pgfsetfillcolor{currentfill}%
\pgfsetfillopacity{0.700000}%
\pgfsetlinewidth{0.000000pt}%
\definecolor{currentstroke}{rgb}{0.000000,0.000000,0.000000}%
\pgfsetstrokecolor{currentstroke}%
\pgfsetdash{}{0pt}%
\pgfpathmoveto{\pgfqpoint{4.497880in}{1.942519in}}%
\pgfpathlineto{\pgfqpoint{4.511870in}{1.951848in}}%
\pgfpathlineto{\pgfqpoint{4.525876in}{1.961335in}}%
\pgfpathlineto{\pgfqpoint{4.539897in}{1.970981in}}%
\pgfpathlineto{\pgfqpoint{4.553934in}{1.980787in}}%
\pgfpathlineto{\pgfqpoint{4.561793in}{1.997182in}}%
\pgfpathlineto{\pgfqpoint{4.569647in}{2.013505in}}%
\pgfpathlineto{\pgfqpoint{4.577497in}{2.029752in}}%
\pgfpathlineto{\pgfqpoint{4.585343in}{2.045919in}}%
\pgfpathlineto{\pgfqpoint{4.571300in}{2.035735in}}%
\pgfpathlineto{\pgfqpoint{4.557272in}{2.025710in}}%
\pgfpathlineto{\pgfqpoint{4.543260in}{2.015845in}}%
\pgfpathlineto{\pgfqpoint{4.529263in}{2.006139in}}%
\pgfpathlineto{\pgfqpoint{4.521424in}{1.990339in}}%
\pgfpathlineto{\pgfqpoint{4.513580in}{1.974467in}}%
\pgfpathlineto{\pgfqpoint{4.505732in}{1.958526in}}%
\pgfpathlineto{\pgfqpoint{4.497880in}{1.942519in}}%
\pgfpathclose%
\pgfusepath{fill}%
\end{pgfscope}%
\begin{pgfscope}%
\pgfpathrectangle{\pgfqpoint{1.254980in}{0.150000in}}{\pgfqpoint{5.490039in}{5.490039in}}%
\pgfusepath{clip}%
\pgfsetbuttcap%
\pgfsetroundjoin%
\definecolor{currentfill}{rgb}{0.282910,0.105393,0.426902}%
\pgfsetfillcolor{currentfill}%
\pgfsetfillopacity{0.700000}%
\pgfsetlinewidth{0.000000pt}%
\definecolor{currentstroke}{rgb}{0.000000,0.000000,0.000000}%
\pgfsetstrokecolor{currentstroke}%
\pgfsetdash{}{0pt}%
\pgfpathmoveto{\pgfqpoint{3.936151in}{1.296728in}}%
\pgfpathlineto{\pgfqpoint{3.949886in}{1.298314in}}%
\pgfpathlineto{\pgfqpoint{3.963631in}{1.300054in}}%
\pgfpathlineto{\pgfqpoint{3.977385in}{1.301948in}}%
\pgfpathlineto{\pgfqpoint{3.991148in}{1.303996in}}%
\pgfpathlineto{\pgfqpoint{3.999137in}{1.316923in}}%
\pgfpathlineto{\pgfqpoint{4.007121in}{1.329983in}}%
\pgfpathlineto{\pgfqpoint{4.015100in}{1.343169in}}%
\pgfpathlineto{\pgfqpoint{4.023074in}{1.356478in}}%
\pgfpathlineto{\pgfqpoint{4.009316in}{1.353809in}}%
\pgfpathlineto{\pgfqpoint{3.995567in}{1.351294in}}%
\pgfpathlineto{\pgfqpoint{3.981829in}{1.348935in}}%
\pgfpathlineto{\pgfqpoint{3.968100in}{1.346730in}}%
\pgfpathlineto{\pgfqpoint{3.960121in}{1.334031in}}%
\pgfpathlineto{\pgfqpoint{3.952136in}{1.321461in}}%
\pgfpathlineto{\pgfqpoint{3.944146in}{1.309025in}}%
\pgfpathlineto{\pgfqpoint{3.936151in}{1.296728in}}%
\pgfpathclose%
\pgfusepath{fill}%
\end{pgfscope}%
\begin{pgfscope}%
\pgfpathrectangle{\pgfqpoint{1.254980in}{0.150000in}}{\pgfqpoint{5.490039in}{5.490039in}}%
\pgfusepath{clip}%
\pgfsetbuttcap%
\pgfsetroundjoin%
\definecolor{currentfill}{rgb}{0.166617,0.463708,0.558119}%
\pgfsetfillcolor{currentfill}%
\pgfsetfillopacity{0.700000}%
\pgfsetlinewidth{0.000000pt}%
\definecolor{currentstroke}{rgb}{0.000000,0.000000,0.000000}%
\pgfsetstrokecolor{currentstroke}%
\pgfsetdash{}{0pt}%
\pgfpathmoveto{\pgfqpoint{4.616684in}{2.109732in}}%
\pgfpathlineto{\pgfqpoint{4.630750in}{2.120426in}}%
\pgfpathlineto{\pgfqpoint{4.644834in}{2.131281in}}%
\pgfpathlineto{\pgfqpoint{4.658934in}{2.142296in}}%
\pgfpathlineto{\pgfqpoint{4.673051in}{2.153472in}}%
\pgfpathlineto{\pgfqpoint{4.680882in}{2.169527in}}%
\pgfpathlineto{\pgfqpoint{4.688708in}{2.185477in}}%
\pgfpathlineto{\pgfqpoint{4.696529in}{2.201319in}}%
\pgfpathlineto{\pgfqpoint{4.704345in}{2.217050in}}%
\pgfpathlineto{\pgfqpoint{4.690221in}{2.205552in}}%
\pgfpathlineto{\pgfqpoint{4.676114in}{2.194215in}}%
\pgfpathlineto{\pgfqpoint{4.662024in}{2.183039in}}%
\pgfpathlineto{\pgfqpoint{4.647950in}{2.172024in}}%
\pgfpathlineto{\pgfqpoint{4.640141in}{2.156604in}}%
\pgfpathlineto{\pgfqpoint{4.632326in}{2.141080in}}%
\pgfpathlineto{\pgfqpoint{4.624507in}{2.125455in}}%
\pgfpathlineto{\pgfqpoint{4.616684in}{2.109732in}}%
\pgfpathclose%
\pgfusepath{fill}%
\end{pgfscope}%
\begin{pgfscope}%
\pgfpathrectangle{\pgfqpoint{1.254980in}{0.150000in}}{\pgfqpoint{5.490039in}{5.490039in}}%
\pgfusepath{clip}%
\pgfsetbuttcap%
\pgfsetroundjoin%
\definecolor{currentfill}{rgb}{0.273809,0.031497,0.358853}%
\pgfsetfillcolor{currentfill}%
\pgfsetfillopacity{0.700000}%
\pgfsetlinewidth{0.000000pt}%
\definecolor{currentstroke}{rgb}{0.000000,0.000000,0.000000}%
\pgfsetstrokecolor{currentstroke}%
\pgfsetdash{}{0pt}%
\pgfpathmoveto{\pgfqpoint{3.675160in}{1.179665in}}%
\pgfpathlineto{\pgfqpoint{3.688839in}{1.177373in}}%
\pgfpathlineto{\pgfqpoint{3.702525in}{1.175237in}}%
\pgfpathlineto{\pgfqpoint{3.716217in}{1.173257in}}%
\pgfpathlineto{\pgfqpoint{3.729915in}{1.171432in}}%
\pgfpathlineto{\pgfqpoint{3.738011in}{1.180128in}}%
\pgfpathlineto{\pgfqpoint{3.746100in}{1.189056in}}%
\pgfpathlineto{\pgfqpoint{3.754180in}{1.198209in}}%
\pgfpathlineto{\pgfqpoint{3.762253in}{1.207581in}}%
\pgfpathlineto{\pgfqpoint{3.748570in}{1.208706in}}%
\pgfpathlineto{\pgfqpoint{3.734894in}{1.209987in}}%
\pgfpathlineto{\pgfqpoint{3.721225in}{1.211424in}}%
\pgfpathlineto{\pgfqpoint{3.707563in}{1.213016in}}%
\pgfpathlineto{\pgfqpoint{3.699474in}{1.204334in}}%
\pgfpathlineto{\pgfqpoint{3.691378in}{1.195876in}}%
\pgfpathlineto{\pgfqpoint{3.683273in}{1.187651in}}%
\pgfpathlineto{\pgfqpoint{3.675160in}{1.179665in}}%
\pgfpathclose%
\pgfusepath{fill}%
\end{pgfscope}%
\begin{pgfscope}%
\pgfpathrectangle{\pgfqpoint{1.254980in}{0.150000in}}{\pgfqpoint{5.490039in}{5.490039in}}%
\pgfusepath{clip}%
\pgfsetbuttcap%
\pgfsetroundjoin%
\definecolor{currentfill}{rgb}{0.282623,0.140926,0.457517}%
\pgfsetfillcolor{currentfill}%
\pgfsetfillopacity{0.700000}%
\pgfsetlinewidth{0.000000pt}%
\definecolor{currentstroke}{rgb}{0.000000,0.000000,0.000000}%
\pgfsetstrokecolor{currentstroke}%
\pgfsetdash{}{0pt}%
\pgfpathmoveto{\pgfqpoint{4.023074in}{1.356478in}}%
\pgfpathlineto{\pgfqpoint{4.036842in}{1.359301in}}%
\pgfpathlineto{\pgfqpoint{4.050621in}{1.362278in}}%
\pgfpathlineto{\pgfqpoint{4.064410in}{1.365410in}}%
\pgfpathlineto{\pgfqpoint{4.078210in}{1.368696in}}%
\pgfpathlineto{\pgfqpoint{4.086176in}{1.382724in}}%
\pgfpathlineto{\pgfqpoint{4.094138in}{1.396855in}}%
\pgfpathlineto{\pgfqpoint{4.102096in}{1.411084in}}%
\pgfpathlineto{\pgfqpoint{4.110049in}{1.425403in}}%
\pgfpathlineto{\pgfqpoint{4.096252in}{1.421522in}}%
\pgfpathlineto{\pgfqpoint{4.082465in}{1.417796in}}%
\pgfpathlineto{\pgfqpoint{4.068690in}{1.414224in}}%
\pgfpathlineto{\pgfqpoint{4.054925in}{1.410808in}}%
\pgfpathlineto{\pgfqpoint{4.046969in}{1.397072in}}%
\pgfpathlineto{\pgfqpoint{4.039008in}{1.383435in}}%
\pgfpathlineto{\pgfqpoint{4.031044in}{1.369901in}}%
\pgfpathlineto{\pgfqpoint{4.023074in}{1.356478in}}%
\pgfpathclose%
\pgfusepath{fill}%
\end{pgfscope}%
\begin{pgfscope}%
\pgfpathrectangle{\pgfqpoint{1.254980in}{0.150000in}}{\pgfqpoint{5.490039in}{5.490039in}}%
\pgfusepath{clip}%
\pgfsetbuttcap%
\pgfsetroundjoin%
\definecolor{currentfill}{rgb}{0.386433,0.794644,0.372886}%
\pgfsetfillcolor{currentfill}%
\pgfsetfillopacity{0.700000}%
\pgfsetlinewidth{0.000000pt}%
\definecolor{currentstroke}{rgb}{0.000000,0.000000,0.000000}%
\pgfsetstrokecolor{currentstroke}%
\pgfsetdash{}{0pt}%
\pgfpathmoveto{\pgfqpoint{5.329351in}{3.065647in}}%
\pgfpathlineto{\pgfqpoint{5.343928in}{3.082148in}}%
\pgfpathlineto{\pgfqpoint{5.358527in}{3.098817in}}%
\pgfpathlineto{\pgfqpoint{5.373148in}{3.115654in}}%
\pgfpathlineto{\pgfqpoint{5.387792in}{3.132660in}}%
\pgfpathlineto{\pgfqpoint{5.395286in}{3.141529in}}%
\pgfpathlineto{\pgfqpoint{5.402770in}{3.150195in}}%
\pgfpathlineto{\pgfqpoint{5.410244in}{3.158658in}}%
\pgfpathlineto{\pgfqpoint{5.417707in}{3.166919in}}%
\pgfpathlineto{\pgfqpoint{5.403067in}{3.149963in}}%
\pgfpathlineto{\pgfqpoint{5.388449in}{3.133176in}}%
\pgfpathlineto{\pgfqpoint{5.373854in}{3.116558in}}%
\pgfpathlineto{\pgfqpoint{5.359281in}{3.100107in}}%
\pgfpathlineto{\pgfqpoint{5.351814in}{3.091783in}}%
\pgfpathlineto{\pgfqpoint{5.344336in}{3.083266in}}%
\pgfpathlineto{\pgfqpoint{5.336849in}{3.074554in}}%
\pgfpathlineto{\pgfqpoint{5.329351in}{3.065647in}}%
\pgfpathclose%
\pgfusepath{fill}%
\end{pgfscope}%
\begin{pgfscope}%
\pgfpathrectangle{\pgfqpoint{1.254980in}{0.150000in}}{\pgfqpoint{5.490039in}{5.490039in}}%
\pgfusepath{clip}%
\pgfsetbuttcap%
\pgfsetroundjoin%
\definecolor{currentfill}{rgb}{0.140536,0.530132,0.555659}%
\pgfsetfillcolor{currentfill}%
\pgfsetfillopacity{0.700000}%
\pgfsetlinewidth{0.000000pt}%
\definecolor{currentstroke}{rgb}{0.000000,0.000000,0.000000}%
\pgfsetstrokecolor{currentstroke}%
\pgfsetdash{}{0pt}%
\pgfpathmoveto{\pgfqpoint{4.735560in}{2.278809in}}%
\pgfpathlineto{\pgfqpoint{4.749708in}{2.290761in}}%
\pgfpathlineto{\pgfqpoint{4.763874in}{2.302874in}}%
\pgfpathlineto{\pgfqpoint{4.778058in}{2.315150in}}%
\pgfpathlineto{\pgfqpoint{4.792260in}{2.327588in}}%
\pgfpathlineto{\pgfqpoint{4.800057in}{2.342999in}}%
\pgfpathlineto{\pgfqpoint{4.807849in}{2.358276in}}%
\pgfpathlineto{\pgfqpoint{4.815635in}{2.373416in}}%
\pgfpathlineto{\pgfqpoint{4.823416in}{2.388419in}}%
\pgfpathlineto{\pgfqpoint{4.809207in}{2.375717in}}%
\pgfpathlineto{\pgfqpoint{4.795017in}{2.363178in}}%
\pgfpathlineto{\pgfqpoint{4.780844in}{2.350802in}}%
\pgfpathlineto{\pgfqpoint{4.766689in}{2.338588in}}%
\pgfpathlineto{\pgfqpoint{4.758915in}{2.323837in}}%
\pgfpathlineto{\pgfqpoint{4.751136in}{2.308955in}}%
\pgfpathlineto{\pgfqpoint{4.743350in}{2.293945in}}%
\pgfpathlineto{\pgfqpoint{4.735560in}{2.278809in}}%
\pgfpathclose%
\pgfusepath{fill}%
\end{pgfscope}%
\begin{pgfscope}%
\pgfpathrectangle{\pgfqpoint{1.254980in}{0.150000in}}{\pgfqpoint{5.490039in}{5.490039in}}%
\pgfusepath{clip}%
\pgfsetbuttcap%
\pgfsetroundjoin%
\definecolor{currentfill}{rgb}{0.595839,0.848717,0.243329}%
\pgfsetfillcolor{currentfill}%
\pgfsetfillopacity{0.700000}%
\pgfsetlinewidth{0.000000pt}%
\definecolor{currentstroke}{rgb}{0.000000,0.000000,0.000000}%
\pgfsetstrokecolor{currentstroke}%
\pgfsetdash{}{0pt}%
\pgfpathmoveto{\pgfqpoint{5.535756in}{3.294600in}}%
\pgfpathlineto{\pgfqpoint{5.550497in}{3.312201in}}%
\pgfpathlineto{\pgfqpoint{5.565262in}{3.329972in}}%
\pgfpathlineto{\pgfqpoint{5.580050in}{3.347912in}}%
\pgfpathlineto{\pgfqpoint{5.587400in}{3.354175in}}%
\pgfpathlineto{\pgfqpoint{5.594738in}{3.360236in}}%
\pgfpathlineto{\pgfqpoint{5.602065in}{3.366098in}}%
\pgfpathlineto{\pgfqpoint{5.609379in}{3.371762in}}%
\pgfpathlineto{\pgfqpoint{5.594601in}{3.353971in}}%
\pgfpathlineto{\pgfqpoint{5.579846in}{3.336351in}}%
\pgfpathlineto{\pgfqpoint{5.565114in}{3.318899in}}%
\pgfpathlineto{\pgfqpoint{5.557791in}{3.313114in}}%
\pgfpathlineto{\pgfqpoint{5.550458in}{3.307137in}}%
\pgfpathlineto{\pgfqpoint{5.543113in}{3.300966in}}%
\pgfpathlineto{\pgfqpoint{5.535756in}{3.294600in}}%
\pgfpathclose%
\pgfusepath{fill}%
\end{pgfscope}%
\begin{pgfscope}%
\pgfpathrectangle{\pgfqpoint{1.254980in}{0.150000in}}{\pgfqpoint{5.490039in}{5.490039in}}%
\pgfusepath{clip}%
\pgfsetbuttcap%
\pgfsetroundjoin%
\definecolor{currentfill}{rgb}{0.233603,0.313828,0.543914}%
\pgfsetfillcolor{currentfill}%
\pgfsetfillopacity{0.700000}%
\pgfsetlinewidth{0.000000pt}%
\definecolor{currentstroke}{rgb}{0.000000,0.000000,0.000000}%
\pgfsetstrokecolor{currentstroke}%
\pgfsetdash{}{0pt}%
\pgfpathmoveto{\pgfqpoint{4.347607in}{1.716175in}}%
\pgfpathlineto{\pgfqpoint{4.361523in}{1.723571in}}%
\pgfpathlineto{\pgfqpoint{4.375453in}{1.731124in}}%
\pgfpathlineto{\pgfqpoint{4.389396in}{1.738833in}}%
\pgfpathlineto{\pgfqpoint{4.403354in}{1.746699in}}%
\pgfpathlineto{\pgfqpoint{4.411252in}{1.763195in}}%
\pgfpathlineto{\pgfqpoint{4.419146in}{1.779673in}}%
\pgfpathlineto{\pgfqpoint{4.427036in}{1.796127in}}%
\pgfpathlineto{\pgfqpoint{4.434923in}{1.812555in}}%
\pgfpathlineto{\pgfqpoint{4.420959in}{1.804225in}}%
\pgfpathlineto{\pgfqpoint{4.407010in}{1.796054in}}%
\pgfpathlineto{\pgfqpoint{4.393075in}{1.788039in}}%
\pgfpathlineto{\pgfqpoint{4.379155in}{1.780183in}}%
\pgfpathlineto{\pgfqpoint{4.371274in}{1.764206in}}%
\pgfpathlineto{\pgfqpoint{4.363388in}{1.748210in}}%
\pgfpathlineto{\pgfqpoint{4.355499in}{1.732199in}}%
\pgfpathlineto{\pgfqpoint{4.347607in}{1.716175in}}%
\pgfpathclose%
\pgfusepath{fill}%
\end{pgfscope}%
\begin{pgfscope}%
\pgfpathrectangle{\pgfqpoint{1.254980in}{0.150000in}}{\pgfqpoint{5.490039in}{5.490039in}}%
\pgfusepath{clip}%
\pgfsetbuttcap%
\pgfsetroundjoin%
\definecolor{currentfill}{rgb}{0.121831,0.589055,0.545623}%
\pgfsetfillcolor{currentfill}%
\pgfsetfillopacity{0.700000}%
\pgfsetlinewidth{0.000000pt}%
\definecolor{currentstroke}{rgb}{0.000000,0.000000,0.000000}%
\pgfsetstrokecolor{currentstroke}%
\pgfsetdash{}{0pt}%
\pgfpathmoveto{\pgfqpoint{4.854479in}{2.447005in}}%
\pgfpathlineto{\pgfqpoint{4.868712in}{2.460103in}}%
\pgfpathlineto{\pgfqpoint{4.882964in}{2.473364in}}%
\pgfpathlineto{\pgfqpoint{4.897235in}{2.486789in}}%
\pgfpathlineto{\pgfqpoint{4.911525in}{2.500377in}}%
\pgfpathlineto{\pgfqpoint{4.919281in}{2.514873in}}%
\pgfpathlineto{\pgfqpoint{4.927032in}{2.529211in}}%
\pgfpathlineto{\pgfqpoint{4.934775in}{2.543390in}}%
\pgfpathlineto{\pgfqpoint{4.942512in}{2.557408in}}%
\pgfpathlineto{\pgfqpoint{4.928216in}{2.543616in}}%
\pgfpathlineto{\pgfqpoint{4.913940in}{2.529988in}}%
\pgfpathlineto{\pgfqpoint{4.899682in}{2.516524in}}%
\pgfpathlineto{\pgfqpoint{4.885443in}{2.503223in}}%
\pgfpathlineto{\pgfqpoint{4.877712in}{2.489397in}}%
\pgfpathlineto{\pgfqpoint{4.869974in}{2.475417in}}%
\pgfpathlineto{\pgfqpoint{4.862230in}{2.461286in}}%
\pgfpathlineto{\pgfqpoint{4.854479in}{2.447005in}}%
\pgfpathclose%
\pgfusepath{fill}%
\end{pgfscope}%
\begin{pgfscope}%
\pgfpathrectangle{\pgfqpoint{1.254980in}{0.150000in}}{\pgfqpoint{5.490039in}{5.490039in}}%
\pgfusepath{clip}%
\pgfsetbuttcap%
\pgfsetroundjoin%
\definecolor{currentfill}{rgb}{0.260571,0.246922,0.522828}%
\pgfsetfillcolor{currentfill}%
\pgfsetfillopacity{0.700000}%
\pgfsetlinewidth{0.000000pt}%
\definecolor{currentstroke}{rgb}{0.000000,0.000000,0.000000}%
\pgfsetstrokecolor{currentstroke}%
\pgfsetdash{}{0pt}%
\pgfpathmoveto{\pgfqpoint{4.228835in}{1.564114in}}%
\pgfpathlineto{\pgfqpoint{4.242691in}{1.569881in}}%
\pgfpathlineto{\pgfqpoint{4.256561in}{1.575803in}}%
\pgfpathlineto{\pgfqpoint{4.270443in}{1.581880in}}%
\pgfpathlineto{\pgfqpoint{4.284338in}{1.588114in}}%
\pgfpathlineto{\pgfqpoint{4.292259in}{1.604066in}}%
\pgfpathlineto{\pgfqpoint{4.300176in}{1.620044in}}%
\pgfpathlineto{\pgfqpoint{4.308090in}{1.636042in}}%
\pgfpathlineto{\pgfqpoint{4.316001in}{1.652056in}}%
\pgfpathlineto{\pgfqpoint{4.302103in}{1.645306in}}%
\pgfpathlineto{\pgfqpoint{4.288218in}{1.638712in}}%
\pgfpathlineto{\pgfqpoint{4.274346in}{1.632274in}}%
\pgfpathlineto{\pgfqpoint{4.260487in}{1.625993in}}%
\pgfpathlineto{\pgfqpoint{4.252579in}{1.610484in}}%
\pgfpathlineto{\pgfqpoint{4.244668in}{1.594998in}}%
\pgfpathlineto{\pgfqpoint{4.236753in}{1.579540in}}%
\pgfpathlineto{\pgfqpoint{4.228835in}{1.564114in}}%
\pgfpathclose%
\pgfusepath{fill}%
\end{pgfscope}%
\begin{pgfscope}%
\pgfpathrectangle{\pgfqpoint{1.254980in}{0.150000in}}{\pgfqpoint{5.490039in}{5.490039in}}%
\pgfusepath{clip}%
\pgfsetbuttcap%
\pgfsetroundjoin%
\definecolor{currentfill}{rgb}{0.281477,0.755203,0.432552}%
\pgfsetfillcolor{currentfill}%
\pgfsetfillopacity{0.700000}%
\pgfsetlinewidth{0.000000pt}%
\definecolor{currentstroke}{rgb}{0.000000,0.000000,0.000000}%
\pgfsetstrokecolor{currentstroke}%
\pgfsetdash{}{0pt}%
\pgfpathmoveto{\pgfqpoint{5.210912in}{2.922929in}}%
\pgfpathlineto{\pgfqpoint{5.225405in}{2.938762in}}%
\pgfpathlineto{\pgfqpoint{5.239920in}{2.954763in}}%
\pgfpathlineto{\pgfqpoint{5.254456in}{2.970932in}}%
\pgfpathlineto{\pgfqpoint{5.269014in}{2.987268in}}%
\pgfpathlineto{\pgfqpoint{5.276590in}{2.997762in}}%
\pgfpathlineto{\pgfqpoint{5.284157in}{3.008056in}}%
\pgfpathlineto{\pgfqpoint{5.291714in}{3.018151in}}%
\pgfpathlineto{\pgfqpoint{5.299261in}{3.028047in}}%
\pgfpathlineto{\pgfqpoint{5.284703in}{3.011696in}}%
\pgfpathlineto{\pgfqpoint{5.270168in}{2.995513in}}%
\pgfpathlineto{\pgfqpoint{5.255654in}{2.979497in}}%
\pgfpathlineto{\pgfqpoint{5.241161in}{2.963649in}}%
\pgfpathlineto{\pgfqpoint{5.233613in}{2.953756in}}%
\pgfpathlineto{\pgfqpoint{5.226055in}{2.943672in}}%
\pgfpathlineto{\pgfqpoint{5.218488in}{2.933396in}}%
\pgfpathlineto{\pgfqpoint{5.210912in}{2.922929in}}%
\pgfpathclose%
\pgfusepath{fill}%
\end{pgfscope}%
\begin{pgfscope}%
\pgfpathrectangle{\pgfqpoint{1.254980in}{0.150000in}}{\pgfqpoint{5.490039in}{5.490039in}}%
\pgfusepath{clip}%
\pgfsetbuttcap%
\pgfsetroundjoin%
\definecolor{currentfill}{rgb}{0.130067,0.651384,0.521608}%
\pgfsetfillcolor{currentfill}%
\pgfsetfillopacity{0.700000}%
\pgfsetlinewidth{0.000000pt}%
\definecolor{currentstroke}{rgb}{0.000000,0.000000,0.000000}%
\pgfsetstrokecolor{currentstroke}%
\pgfsetdash{}{0pt}%
\pgfpathmoveto{\pgfqpoint{4.973391in}{2.611843in}}%
\pgfpathlineto{\pgfqpoint{4.987711in}{2.625972in}}%
\pgfpathlineto{\pgfqpoint{5.002051in}{2.640266in}}%
\pgfpathlineto{\pgfqpoint{5.016410in}{2.654725in}}%
\pgfpathlineto{\pgfqpoint{5.030790in}{2.669349in}}%
\pgfpathlineto{\pgfqpoint{5.038497in}{2.682695in}}%
\pgfpathlineto{\pgfqpoint{5.046196in}{2.695865in}}%
\pgfpathlineto{\pgfqpoint{5.053888in}{2.708857in}}%
\pgfpathlineto{\pgfqpoint{5.061572in}{2.721670in}}%
\pgfpathlineto{\pgfqpoint{5.047188in}{2.706904in}}%
\pgfpathlineto{\pgfqpoint{5.032824in}{2.692303in}}%
\pgfpathlineto{\pgfqpoint{5.018480in}{2.677868in}}%
\pgfpathlineto{\pgfqpoint{5.004156in}{2.663598in}}%
\pgfpathlineto{\pgfqpoint{4.996476in}{2.650914in}}%
\pgfpathlineto{\pgfqpoint{4.988788in}{2.638060in}}%
\pgfpathlineto{\pgfqpoint{4.981093in}{2.625036in}}%
\pgfpathlineto{\pgfqpoint{4.973391in}{2.611843in}}%
\pgfpathclose%
\pgfusepath{fill}%
\end{pgfscope}%
\begin{pgfscope}%
\pgfpathrectangle{\pgfqpoint{1.254980in}{0.150000in}}{\pgfqpoint{5.490039in}{5.490039in}}%
\pgfusepath{clip}%
\pgfsetbuttcap%
\pgfsetroundjoin%
\definecolor{currentfill}{rgb}{0.201239,0.383670,0.554294}%
\pgfsetfillcolor{currentfill}%
\pgfsetfillopacity{0.700000}%
\pgfsetlinewidth{0.000000pt}%
\definecolor{currentstroke}{rgb}{0.000000,0.000000,0.000000}%
\pgfsetstrokecolor{currentstroke}%
\pgfsetdash{}{0pt}%
\pgfpathmoveto{\pgfqpoint{4.466432in}{1.877913in}}%
\pgfpathlineto{\pgfqpoint{4.480416in}{1.886835in}}%
\pgfpathlineto{\pgfqpoint{4.494416in}{1.895915in}}%
\pgfpathlineto{\pgfqpoint{4.508430in}{1.905154in}}%
\pgfpathlineto{\pgfqpoint{4.522460in}{1.914551in}}%
\pgfpathlineto{\pgfqpoint{4.530334in}{1.931201in}}%
\pgfpathlineto{\pgfqpoint{4.538205in}{1.947793in}}%
\pgfpathlineto{\pgfqpoint{4.546072in}{1.964322in}}%
\pgfpathlineto{\pgfqpoint{4.553934in}{1.980787in}}%
\pgfpathlineto{\pgfqpoint{4.539897in}{1.970981in}}%
\pgfpathlineto{\pgfqpoint{4.525876in}{1.961335in}}%
\pgfpathlineto{\pgfqpoint{4.511870in}{1.951848in}}%
\pgfpathlineto{\pgfqpoint{4.497880in}{1.942519in}}%
\pgfpathlineto{\pgfqpoint{4.490024in}{1.926451in}}%
\pgfpathlineto{\pgfqpoint{4.482164in}{1.910325in}}%
\pgfpathlineto{\pgfqpoint{4.474300in}{1.894144in}}%
\pgfpathlineto{\pgfqpoint{4.466432in}{1.877913in}}%
\pgfpathclose%
\pgfusepath{fill}%
\end{pgfscope}%
\begin{pgfscope}%
\pgfpathrectangle{\pgfqpoint{1.254980in}{0.150000in}}{\pgfqpoint{5.490039in}{5.490039in}}%
\pgfusepath{clip}%
\pgfsetbuttcap%
\pgfsetroundjoin%
\definecolor{currentfill}{rgb}{0.185783,0.704891,0.485273}%
\pgfsetfillcolor{currentfill}%
\pgfsetfillopacity{0.700000}%
\pgfsetlinewidth{0.000000pt}%
\definecolor{currentstroke}{rgb}{0.000000,0.000000,0.000000}%
\pgfsetstrokecolor{currentstroke}%
\pgfsetdash{}{0pt}%
\pgfpathmoveto{\pgfqpoint{5.092229in}{2.771122in}}%
\pgfpathlineto{\pgfqpoint{5.106636in}{2.786164in}}%
\pgfpathlineto{\pgfqpoint{5.121064in}{2.801372in}}%
\pgfpathlineto{\pgfqpoint{5.135512in}{2.816747in}}%
\pgfpathlineto{\pgfqpoint{5.149981in}{2.832288in}}%
\pgfpathlineto{\pgfqpoint{5.157628in}{2.844287in}}%
\pgfpathlineto{\pgfqpoint{5.165267in}{2.856096in}}%
\pgfpathlineto{\pgfqpoint{5.172896in}{2.867714in}}%
\pgfpathlineto{\pgfqpoint{5.180517in}{2.879141in}}%
\pgfpathlineto{\pgfqpoint{5.166046in}{2.863521in}}%
\pgfpathlineto{\pgfqpoint{5.151595in}{2.848067in}}%
\pgfpathlineto{\pgfqpoint{5.137165in}{2.832780in}}%
\pgfpathlineto{\pgfqpoint{5.122756in}{2.817660in}}%
\pgfpathlineto{\pgfqpoint{5.115137in}{2.806300in}}%
\pgfpathlineto{\pgfqpoint{5.107509in}{2.794757in}}%
\pgfpathlineto{\pgfqpoint{5.099873in}{2.783031in}}%
\pgfpathlineto{\pgfqpoint{5.092229in}{2.771122in}}%
\pgfpathclose%
\pgfusepath{fill}%
\end{pgfscope}%
\begin{pgfscope}%
\pgfpathrectangle{\pgfqpoint{1.254980in}{0.150000in}}{\pgfqpoint{5.490039in}{5.490039in}}%
\pgfusepath{clip}%
\pgfsetbuttcap%
\pgfsetroundjoin%
\definecolor{currentfill}{rgb}{0.278012,0.180367,0.486697}%
\pgfsetfillcolor{currentfill}%
\pgfsetfillopacity{0.700000}%
\pgfsetlinewidth{0.000000pt}%
\definecolor{currentstroke}{rgb}{0.000000,0.000000,0.000000}%
\pgfsetstrokecolor{currentstroke}%
\pgfsetdash{}{0pt}%
\pgfpathmoveto{\pgfqpoint{4.110049in}{1.425403in}}%
\pgfpathlineto{\pgfqpoint{4.123858in}{1.429439in}}%
\pgfpathlineto{\pgfqpoint{4.137678in}{1.433629in}}%
\pgfpathlineto{\pgfqpoint{4.151509in}{1.437973in}}%
\pgfpathlineto{\pgfqpoint{4.165352in}{1.442472in}}%
\pgfpathlineto{\pgfqpoint{4.173300in}{1.457457in}}%
\pgfpathlineto{\pgfqpoint{4.181245in}{1.472516in}}%
\pgfpathlineto{\pgfqpoint{4.189186in}{1.487643in}}%
\pgfpathlineto{\pgfqpoint{4.197123in}{1.502832in}}%
\pgfpathlineto{\pgfqpoint{4.183280in}{1.497763in}}%
\pgfpathlineto{\pgfqpoint{4.169449in}{1.492850in}}%
\pgfpathlineto{\pgfqpoint{4.155630in}{1.488091in}}%
\pgfpathlineto{\pgfqpoint{4.141822in}{1.483487in}}%
\pgfpathlineto{\pgfqpoint{4.133885in}{1.468856in}}%
\pgfpathlineto{\pgfqpoint{4.125944in}{1.454295in}}%
\pgfpathlineto{\pgfqpoint{4.117999in}{1.439809in}}%
\pgfpathlineto{\pgfqpoint{4.110049in}{1.425403in}}%
\pgfpathclose%
\pgfusepath{fill}%
\end{pgfscope}%
\begin{pgfscope}%
\pgfpathrectangle{\pgfqpoint{1.254980in}{0.150000in}}{\pgfqpoint{5.490039in}{5.490039in}}%
\pgfusepath{clip}%
\pgfsetbuttcap%
\pgfsetroundjoin%
\definecolor{currentfill}{rgb}{0.172719,0.448791,0.557885}%
\pgfsetfillcolor{currentfill}%
\pgfsetfillopacity{0.700000}%
\pgfsetlinewidth{0.000000pt}%
\definecolor{currentstroke}{rgb}{0.000000,0.000000,0.000000}%
\pgfsetstrokecolor{currentstroke}%
\pgfsetdash{}{0pt}%
\pgfpathmoveto{\pgfqpoint{4.585343in}{2.045919in}}%
\pgfpathlineto{\pgfqpoint{4.599403in}{2.056264in}}%
\pgfpathlineto{\pgfqpoint{4.613480in}{2.066767in}}%
\pgfpathlineto{\pgfqpoint{4.627572in}{2.077431in}}%
\pgfpathlineto{\pgfqpoint{4.641681in}{2.088255in}}%
\pgfpathlineto{\pgfqpoint{4.649530in}{2.104702in}}%
\pgfpathlineto{\pgfqpoint{4.657375in}{2.121056in}}%
\pgfpathlineto{\pgfqpoint{4.665215in}{2.137314in}}%
\pgfpathlineto{\pgfqpoint{4.673051in}{2.153472in}}%
\pgfpathlineto{\pgfqpoint{4.658934in}{2.142296in}}%
\pgfpathlineto{\pgfqpoint{4.644834in}{2.131281in}}%
\pgfpathlineto{\pgfqpoint{4.630750in}{2.120426in}}%
\pgfpathlineto{\pgfqpoint{4.616684in}{2.109732in}}%
\pgfpathlineto{\pgfqpoint{4.608855in}{2.093914in}}%
\pgfpathlineto{\pgfqpoint{4.601022in}{2.078003in}}%
\pgfpathlineto{\pgfqpoint{4.593185in}{2.062004in}}%
\pgfpathlineto{\pgfqpoint{4.585343in}{2.045919in}}%
\pgfpathclose%
\pgfusepath{fill}%
\end{pgfscope}%
\begin{pgfscope}%
\pgfpathrectangle{\pgfqpoint{1.254980in}{0.150000in}}{\pgfqpoint{5.490039in}{5.490039in}}%
\pgfusepath{clip}%
\pgfsetbuttcap%
\pgfsetroundjoin%
\definecolor{currentfill}{rgb}{0.278791,0.062145,0.386592}%
\pgfsetfillcolor{currentfill}%
\pgfsetfillopacity{0.700000}%
\pgfsetlinewidth{0.000000pt}%
\definecolor{currentstroke}{rgb}{0.000000,0.000000,0.000000}%
\pgfsetstrokecolor{currentstroke}%
\pgfsetdash{}{0pt}%
\pgfpathmoveto{\pgfqpoint{3.817059in}{1.204629in}}%
\pgfpathlineto{\pgfqpoint{3.830779in}{1.204278in}}%
\pgfpathlineto{\pgfqpoint{3.844507in}{1.204081in}}%
\pgfpathlineto{\pgfqpoint{3.858244in}{1.204037in}}%
\pgfpathlineto{\pgfqpoint{3.871988in}{1.204148in}}%
\pgfpathlineto{\pgfqpoint{3.880029in}{1.215098in}}%
\pgfpathlineto{\pgfqpoint{3.888064in}{1.226239in}}%
\pgfpathlineto{\pgfqpoint{3.896093in}{1.237565in}}%
\pgfpathlineto{\pgfqpoint{3.904116in}{1.249068in}}%
\pgfpathlineto{\pgfqpoint{3.890382in}{1.248283in}}%
\pgfpathlineto{\pgfqpoint{3.876656in}{1.247652in}}%
\pgfpathlineto{\pgfqpoint{3.862939in}{1.247176in}}%
\pgfpathlineto{\pgfqpoint{3.849230in}{1.246854in}}%
\pgfpathlineto{\pgfqpoint{3.841197in}{1.236014in}}%
\pgfpathlineto{\pgfqpoint{3.833157in}{1.225359in}}%
\pgfpathlineto{\pgfqpoint{3.825111in}{1.214895in}}%
\pgfpathlineto{\pgfqpoint{3.817059in}{1.204629in}}%
\pgfpathclose%
\pgfusepath{fill}%
\end{pgfscope}%
\begin{pgfscope}%
\pgfpathrectangle{\pgfqpoint{1.254980in}{0.150000in}}{\pgfqpoint{5.490039in}{5.490039in}}%
\pgfusepath{clip}%
\pgfsetbuttcap%
\pgfsetroundjoin%
\definecolor{currentfill}{rgb}{0.281924,0.089666,0.412415}%
\pgfsetfillcolor{currentfill}%
\pgfsetfillopacity{0.700000}%
\pgfsetlinewidth{0.000000pt}%
\definecolor{currentstroke}{rgb}{0.000000,0.000000,0.000000}%
\pgfsetstrokecolor{currentstroke}%
\pgfsetdash{}{0pt}%
\pgfpathmoveto{\pgfqpoint{3.904116in}{1.249068in}}%
\pgfpathlineto{\pgfqpoint{3.917859in}{1.250006in}}%
\pgfpathlineto{\pgfqpoint{3.931611in}{1.251099in}}%
\pgfpathlineto{\pgfqpoint{3.945372in}{1.252345in}}%
\pgfpathlineto{\pgfqpoint{3.959143in}{1.253745in}}%
\pgfpathlineto{\pgfqpoint{3.967152in}{1.266077in}}%
\pgfpathlineto{\pgfqpoint{3.975156in}{1.278567in}}%
\pgfpathlineto{\pgfqpoint{3.983154in}{1.291209in}}%
\pgfpathlineto{\pgfqpoint{3.991148in}{1.303996in}}%
\pgfpathlineto{\pgfqpoint{3.977385in}{1.301948in}}%
\pgfpathlineto{\pgfqpoint{3.963631in}{1.300054in}}%
\pgfpathlineto{\pgfqpoint{3.949886in}{1.298314in}}%
\pgfpathlineto{\pgfqpoint{3.936151in}{1.296728in}}%
\pgfpathlineto{\pgfqpoint{3.928151in}{1.284578in}}%
\pgfpathlineto{\pgfqpoint{3.920145in}{1.272581in}}%
\pgfpathlineto{\pgfqpoint{3.912133in}{1.260742in}}%
\pgfpathlineto{\pgfqpoint{3.904116in}{1.249068in}}%
\pgfpathclose%
\pgfusepath{fill}%
\end{pgfscope}%
\begin{pgfscope}%
\pgfpathrectangle{\pgfqpoint{1.254980in}{0.150000in}}{\pgfqpoint{5.490039in}{5.490039in}}%
\pgfusepath{clip}%
\pgfsetbuttcap%
\pgfsetroundjoin%
\definecolor{currentfill}{rgb}{0.276022,0.044167,0.370164}%
\pgfsetfillcolor{currentfill}%
\pgfsetfillopacity{0.700000}%
\pgfsetlinewidth{0.000000pt}%
\definecolor{currentstroke}{rgb}{0.000000,0.000000,0.000000}%
\pgfsetstrokecolor{currentstroke}%
\pgfsetdash{}{0pt}%
\pgfpathmoveto{\pgfqpoint{3.729915in}{1.171432in}}%
\pgfpathlineto{\pgfqpoint{3.743621in}{1.169761in}}%
\pgfpathlineto{\pgfqpoint{3.757333in}{1.168246in}}%
\pgfpathlineto{\pgfqpoint{3.771053in}{1.166885in}}%
\pgfpathlineto{\pgfqpoint{3.784779in}{1.165678in}}%
\pgfpathlineto{\pgfqpoint{3.792860in}{1.175085in}}%
\pgfpathlineto{\pgfqpoint{3.800933in}{1.184717in}}%
\pgfpathlineto{\pgfqpoint{3.808999in}{1.194568in}}%
\pgfpathlineto{\pgfqpoint{3.817059in}{1.204629in}}%
\pgfpathlineto{\pgfqpoint{3.803346in}{1.205135in}}%
\pgfpathlineto{\pgfqpoint{3.789641in}{1.205795in}}%
\pgfpathlineto{\pgfqpoint{3.775943in}{1.206611in}}%
\pgfpathlineto{\pgfqpoint{3.762253in}{1.207581in}}%
\pgfpathlineto{\pgfqpoint{3.754180in}{1.198209in}}%
\pgfpathlineto{\pgfqpoint{3.746100in}{1.189056in}}%
\pgfpathlineto{\pgfqpoint{3.738011in}{1.180128in}}%
\pgfpathlineto{\pgfqpoint{3.729915in}{1.171432in}}%
\pgfpathclose%
\pgfusepath{fill}%
\end{pgfscope}%
\begin{pgfscope}%
\pgfpathrectangle{\pgfqpoint{1.254980in}{0.150000in}}{\pgfqpoint{5.490039in}{5.490039in}}%
\pgfusepath{clip}%
\pgfsetbuttcap%
\pgfsetroundjoin%
\definecolor{currentfill}{rgb}{0.147607,0.511733,0.557049}%
\pgfsetfillcolor{currentfill}%
\pgfsetfillopacity{0.700000}%
\pgfsetlinewidth{0.000000pt}%
\definecolor{currentstroke}{rgb}{0.000000,0.000000,0.000000}%
\pgfsetstrokecolor{currentstroke}%
\pgfsetdash{}{0pt}%
\pgfpathmoveto{\pgfqpoint{4.704345in}{2.217050in}}%
\pgfpathlineto{\pgfqpoint{4.718487in}{2.228709in}}%
\pgfpathlineto{\pgfqpoint{4.732645in}{2.240529in}}%
\pgfpathlineto{\pgfqpoint{4.746822in}{2.252512in}}%
\pgfpathlineto{\pgfqpoint{4.761015in}{2.264656in}}%
\pgfpathlineto{\pgfqpoint{4.768834in}{2.280577in}}%
\pgfpathlineto{\pgfqpoint{4.776648in}{2.296375in}}%
\pgfpathlineto{\pgfqpoint{4.784457in}{2.312046in}}%
\pgfpathlineto{\pgfqpoint{4.792260in}{2.327588in}}%
\pgfpathlineto{\pgfqpoint{4.778058in}{2.315150in}}%
\pgfpathlineto{\pgfqpoint{4.763874in}{2.302874in}}%
\pgfpathlineto{\pgfqpoint{4.749708in}{2.290761in}}%
\pgfpathlineto{\pgfqpoint{4.735560in}{2.278809in}}%
\pgfpathlineto{\pgfqpoint{4.727764in}{2.263549in}}%
\pgfpathlineto{\pgfqpoint{4.719963in}{2.248167in}}%
\pgfpathlineto{\pgfqpoint{4.712157in}{2.232667in}}%
\pgfpathlineto{\pgfqpoint{4.704345in}{2.217050in}}%
\pgfpathclose%
\pgfusepath{fill}%
\end{pgfscope}%
\begin{pgfscope}%
\pgfpathrectangle{\pgfqpoint{1.254980in}{0.150000in}}{\pgfqpoint{5.490039in}{5.490039in}}%
\pgfusepath{clip}%
\pgfsetbuttcap%
\pgfsetroundjoin%
\definecolor{currentfill}{rgb}{0.496615,0.826376,0.306377}%
\pgfsetfillcolor{currentfill}%
\pgfsetfillopacity{0.700000}%
\pgfsetlinewidth{0.000000pt}%
\definecolor{currentstroke}{rgb}{0.000000,0.000000,0.000000}%
\pgfsetstrokecolor{currentstroke}%
\pgfsetdash{}{0pt}%
\pgfpathmoveto{\pgfqpoint{5.417707in}{3.166919in}}%
\pgfpathlineto{\pgfqpoint{5.432369in}{3.184043in}}%
\pgfpathlineto{\pgfqpoint{5.447054in}{3.201336in}}%
\pgfpathlineto{\pgfqpoint{5.461762in}{3.218799in}}%
\pgfpathlineto{\pgfqpoint{5.476493in}{3.236432in}}%
\pgfpathlineto{\pgfqpoint{5.483941in}{3.244420in}}%
\pgfpathlineto{\pgfqpoint{5.491377in}{3.252201in}}%
\pgfpathlineto{\pgfqpoint{5.498802in}{3.259776in}}%
\pgfpathlineto{\pgfqpoint{5.506215in}{3.267145in}}%
\pgfpathlineto{\pgfqpoint{5.491490in}{3.249597in}}%
\pgfpathlineto{\pgfqpoint{5.476787in}{3.232218in}}%
\pgfpathlineto{\pgfqpoint{5.462107in}{3.215008in}}%
\pgfpathlineto{\pgfqpoint{5.447451in}{3.197968in}}%
\pgfpathlineto{\pgfqpoint{5.440031in}{3.190502in}}%
\pgfpathlineto{\pgfqpoint{5.432600in}{3.182840in}}%
\pgfpathlineto{\pgfqpoint{5.425159in}{3.174979in}}%
\pgfpathlineto{\pgfqpoint{5.417707in}{3.166919in}}%
\pgfpathclose%
\pgfusepath{fill}%
\end{pgfscope}%
\begin{pgfscope}%
\pgfpathrectangle{\pgfqpoint{1.254980in}{0.150000in}}{\pgfqpoint{5.490039in}{5.490039in}}%
\pgfusepath{clip}%
\pgfsetbuttcap%
\pgfsetroundjoin%
\definecolor{currentfill}{rgb}{0.283187,0.125848,0.444960}%
\pgfsetfillcolor{currentfill}%
\pgfsetfillopacity{0.700000}%
\pgfsetlinewidth{0.000000pt}%
\definecolor{currentstroke}{rgb}{0.000000,0.000000,0.000000}%
\pgfsetstrokecolor{currentstroke}%
\pgfsetdash{}{0pt}%
\pgfpathmoveto{\pgfqpoint{3.991148in}{1.303996in}}%
\pgfpathlineto{\pgfqpoint{4.004921in}{1.306198in}}%
\pgfpathlineto{\pgfqpoint{4.018705in}{1.308554in}}%
\pgfpathlineto{\pgfqpoint{4.032498in}{1.311064in}}%
\pgfpathlineto{\pgfqpoint{4.046301in}{1.313727in}}%
\pgfpathlineto{\pgfqpoint{4.054285in}{1.327285in}}%
\pgfpathlineto{\pgfqpoint{4.062265in}{1.340971in}}%
\pgfpathlineto{\pgfqpoint{4.070239in}{1.354776in}}%
\pgfpathlineto{\pgfqpoint{4.078210in}{1.368696in}}%
\pgfpathlineto{\pgfqpoint{4.064410in}{1.365410in}}%
\pgfpathlineto{\pgfqpoint{4.050621in}{1.362278in}}%
\pgfpathlineto{\pgfqpoint{4.036842in}{1.359301in}}%
\pgfpathlineto{\pgfqpoint{4.023074in}{1.356478in}}%
\pgfpathlineto{\pgfqpoint{4.015100in}{1.343169in}}%
\pgfpathlineto{\pgfqpoint{4.007121in}{1.329983in}}%
\pgfpathlineto{\pgfqpoint{3.999137in}{1.316923in}}%
\pgfpathlineto{\pgfqpoint{3.991148in}{1.303996in}}%
\pgfpathclose%
\pgfusepath{fill}%
\end{pgfscope}%
\begin{pgfscope}%
\pgfpathrectangle{\pgfqpoint{1.254980in}{0.150000in}}{\pgfqpoint{5.490039in}{5.490039in}}%
\pgfusepath{clip}%
\pgfsetbuttcap%
\pgfsetroundjoin%
\definecolor{currentfill}{rgb}{0.243113,0.292092,0.538516}%
\pgfsetfillcolor{currentfill}%
\pgfsetfillopacity{0.700000}%
\pgfsetlinewidth{0.000000pt}%
\definecolor{currentstroke}{rgb}{0.000000,0.000000,0.000000}%
\pgfsetstrokecolor{currentstroke}%
\pgfsetdash{}{0pt}%
\pgfpathmoveto{\pgfqpoint{4.316001in}{1.652056in}}%
\pgfpathlineto{\pgfqpoint{4.329912in}{1.658962in}}%
\pgfpathlineto{\pgfqpoint{4.343838in}{1.666024in}}%
\pgfpathlineto{\pgfqpoint{4.357776in}{1.673243in}}%
\pgfpathlineto{\pgfqpoint{4.371729in}{1.680618in}}%
\pgfpathlineto{\pgfqpoint{4.379640in}{1.697144in}}%
\pgfpathlineto{\pgfqpoint{4.387548in}{1.713669in}}%
\pgfpathlineto{\pgfqpoint{4.395453in}{1.730189in}}%
\pgfpathlineto{\pgfqpoint{4.403354in}{1.746699in}}%
\pgfpathlineto{\pgfqpoint{4.389396in}{1.738833in}}%
\pgfpathlineto{\pgfqpoint{4.375453in}{1.731124in}}%
\pgfpathlineto{\pgfqpoint{4.361523in}{1.723571in}}%
\pgfpathlineto{\pgfqpoint{4.347607in}{1.716175in}}%
\pgfpathlineto{\pgfqpoint{4.339711in}{1.700145in}}%
\pgfpathlineto{\pgfqpoint{4.331811in}{1.684112in}}%
\pgfpathlineto{\pgfqpoint{4.323908in}{1.668081in}}%
\pgfpathlineto{\pgfqpoint{4.316001in}{1.652056in}}%
\pgfpathclose%
\pgfusepath{fill}%
\end{pgfscope}%
\begin{pgfscope}%
\pgfpathrectangle{\pgfqpoint{1.254980in}{0.150000in}}{\pgfqpoint{5.490039in}{5.490039in}}%
\pgfusepath{clip}%
\pgfsetbuttcap%
\pgfsetroundjoin%
\definecolor{currentfill}{rgb}{0.267968,0.223549,0.512008}%
\pgfsetfillcolor{currentfill}%
\pgfsetfillopacity{0.700000}%
\pgfsetlinewidth{0.000000pt}%
\definecolor{currentstroke}{rgb}{0.000000,0.000000,0.000000}%
\pgfsetstrokecolor{currentstroke}%
\pgfsetdash{}{0pt}%
\pgfpathmoveto{\pgfqpoint{4.197123in}{1.502832in}}%
\pgfpathlineto{\pgfqpoint{4.210978in}{1.508056in}}%
\pgfpathlineto{\pgfqpoint{4.224846in}{1.513434in}}%
\pgfpathlineto{\pgfqpoint{4.238725in}{1.518968in}}%
\pgfpathlineto{\pgfqpoint{4.252618in}{1.524656in}}%
\pgfpathlineto{\pgfqpoint{4.260553in}{1.540458in}}%
\pgfpathlineto{\pgfqpoint{4.268485in}{1.556304in}}%
\pgfpathlineto{\pgfqpoint{4.276413in}{1.572191in}}%
\pgfpathlineto{\pgfqpoint{4.284338in}{1.588114in}}%
\pgfpathlineto{\pgfqpoint{4.270443in}{1.581880in}}%
\pgfpathlineto{\pgfqpoint{4.256561in}{1.575803in}}%
\pgfpathlineto{\pgfqpoint{4.242691in}{1.569881in}}%
\pgfpathlineto{\pgfqpoint{4.228835in}{1.564114in}}%
\pgfpathlineto{\pgfqpoint{4.220912in}{1.548725in}}%
\pgfpathlineto{\pgfqpoint{4.212987in}{1.533379in}}%
\pgfpathlineto{\pgfqpoint{4.205057in}{1.518079in}}%
\pgfpathlineto{\pgfqpoint{4.197123in}{1.502832in}}%
\pgfpathclose%
\pgfusepath{fill}%
\end{pgfscope}%
\begin{pgfscope}%
\pgfpathrectangle{\pgfqpoint{1.254980in}{0.150000in}}{\pgfqpoint{5.490039in}{5.490039in}}%
\pgfusepath{clip}%
\pgfsetbuttcap%
\pgfsetroundjoin%
\definecolor{currentfill}{rgb}{0.124395,0.578002,0.548287}%
\pgfsetfillcolor{currentfill}%
\pgfsetfillopacity{0.700000}%
\pgfsetlinewidth{0.000000pt}%
\definecolor{currentstroke}{rgb}{0.000000,0.000000,0.000000}%
\pgfsetstrokecolor{currentstroke}%
\pgfsetdash{}{0pt}%
\pgfpathmoveto{\pgfqpoint{4.823416in}{2.388419in}}%
\pgfpathlineto{\pgfqpoint{4.837643in}{2.401283in}}%
\pgfpathlineto{\pgfqpoint{4.851888in}{2.414311in}}%
\pgfpathlineto{\pgfqpoint{4.866152in}{2.427502in}}%
\pgfpathlineto{\pgfqpoint{4.880435in}{2.440856in}}%
\pgfpathlineto{\pgfqpoint{4.888217in}{2.455963in}}%
\pgfpathlineto{\pgfqpoint{4.895992in}{2.470921in}}%
\pgfpathlineto{\pgfqpoint{4.903762in}{2.485726in}}%
\pgfpathlineto{\pgfqpoint{4.911525in}{2.500377in}}%
\pgfpathlineto{\pgfqpoint{4.897235in}{2.486789in}}%
\pgfpathlineto{\pgfqpoint{4.882964in}{2.473364in}}%
\pgfpathlineto{\pgfqpoint{4.868712in}{2.460103in}}%
\pgfpathlineto{\pgfqpoint{4.854479in}{2.447005in}}%
\pgfpathlineto{\pgfqpoint{4.846722in}{2.432576in}}%
\pgfpathlineto{\pgfqpoint{4.838960in}{2.418001in}}%
\pgfpathlineto{\pgfqpoint{4.831191in}{2.403281in}}%
\pgfpathlineto{\pgfqpoint{4.823416in}{2.388419in}}%
\pgfpathclose%
\pgfusepath{fill}%
\end{pgfscope}%
\begin{pgfscope}%
\pgfpathrectangle{\pgfqpoint{1.254980in}{0.150000in}}{\pgfqpoint{5.490039in}{5.490039in}}%
\pgfusepath{clip}%
\pgfsetbuttcap%
\pgfsetroundjoin%
\definecolor{currentfill}{rgb}{0.210503,0.363727,0.552206}%
\pgfsetfillcolor{currentfill}%
\pgfsetfillopacity{0.700000}%
\pgfsetlinewidth{0.000000pt}%
\definecolor{currentstroke}{rgb}{0.000000,0.000000,0.000000}%
\pgfsetstrokecolor{currentstroke}%
\pgfsetdash{}{0pt}%
\pgfpathmoveto{\pgfqpoint{4.434923in}{1.812555in}}%
\pgfpathlineto{\pgfqpoint{4.448901in}{1.821042in}}%
\pgfpathlineto{\pgfqpoint{4.462893in}{1.829686in}}%
\pgfpathlineto{\pgfqpoint{4.476901in}{1.838489in}}%
\pgfpathlineto{\pgfqpoint{4.490924in}{1.847449in}}%
\pgfpathlineto{\pgfqpoint{4.498813in}{1.864292in}}%
\pgfpathlineto{\pgfqpoint{4.506699in}{1.881093in}}%
\pgfpathlineto{\pgfqpoint{4.514581in}{1.897847in}}%
\pgfpathlineto{\pgfqpoint{4.522460in}{1.914551in}}%
\pgfpathlineto{\pgfqpoint{4.508430in}{1.905154in}}%
\pgfpathlineto{\pgfqpoint{4.494416in}{1.895915in}}%
\pgfpathlineto{\pgfqpoint{4.480416in}{1.886835in}}%
\pgfpathlineto{\pgfqpoint{4.466432in}{1.877913in}}%
\pgfpathlineto{\pgfqpoint{4.458560in}{1.861634in}}%
\pgfpathlineto{\pgfqpoint{4.450685in}{1.845312in}}%
\pgfpathlineto{\pgfqpoint{4.442806in}{1.828951in}}%
\pgfpathlineto{\pgfqpoint{4.434923in}{1.812555in}}%
\pgfpathclose%
\pgfusepath{fill}%
\end{pgfscope}%
\begin{pgfscope}%
\pgfpathrectangle{\pgfqpoint{1.254980in}{0.150000in}}{\pgfqpoint{5.490039in}{5.490039in}}%
\pgfusepath{clip}%
\pgfsetbuttcap%
\pgfsetroundjoin%
\definecolor{currentfill}{rgb}{0.377779,0.791781,0.377939}%
\pgfsetfillcolor{currentfill}%
\pgfsetfillopacity{0.700000}%
\pgfsetlinewidth{0.000000pt}%
\definecolor{currentstroke}{rgb}{0.000000,0.000000,0.000000}%
\pgfsetstrokecolor{currentstroke}%
\pgfsetdash{}{0pt}%
\pgfpathmoveto{\pgfqpoint{5.299261in}{3.028047in}}%
\pgfpathlineto{\pgfqpoint{5.313840in}{3.044565in}}%
\pgfpathlineto{\pgfqpoint{5.328441in}{3.061252in}}%
\pgfpathlineto{\pgfqpoint{5.343064in}{3.078108in}}%
\pgfpathlineto{\pgfqpoint{5.357710in}{3.095133in}}%
\pgfpathlineto{\pgfqpoint{5.365246in}{3.104824in}}%
\pgfpathlineto{\pgfqpoint{5.372772in}{3.114308in}}%
\pgfpathlineto{\pgfqpoint{5.380287in}{3.123587in}}%
\pgfpathlineto{\pgfqpoint{5.387792in}{3.132660in}}%
\pgfpathlineto{\pgfqpoint{5.373148in}{3.115654in}}%
\pgfpathlineto{\pgfqpoint{5.358527in}{3.098817in}}%
\pgfpathlineto{\pgfqpoint{5.343928in}{3.082148in}}%
\pgfpathlineto{\pgfqpoint{5.329351in}{3.065647in}}%
\pgfpathlineto{\pgfqpoint{5.321843in}{3.056543in}}%
\pgfpathlineto{\pgfqpoint{5.314326in}{3.047242in}}%
\pgfpathlineto{\pgfqpoint{5.306798in}{3.037743in}}%
\pgfpathlineto{\pgfqpoint{5.299261in}{3.028047in}}%
\pgfpathclose%
\pgfusepath{fill}%
\end{pgfscope}%
\begin{pgfscope}%
\pgfpathrectangle{\pgfqpoint{1.254980in}{0.150000in}}{\pgfqpoint{5.490039in}{5.490039in}}%
\pgfusepath{clip}%
\pgfsetbuttcap%
\pgfsetroundjoin%
\definecolor{currentfill}{rgb}{0.124780,0.640461,0.527068}%
\pgfsetfillcolor{currentfill}%
\pgfsetfillopacity{0.700000}%
\pgfsetlinewidth{0.000000pt}%
\definecolor{currentstroke}{rgb}{0.000000,0.000000,0.000000}%
\pgfsetstrokecolor{currentstroke}%
\pgfsetdash{}{0pt}%
\pgfpathmoveto{\pgfqpoint{4.942512in}{2.557408in}}%
\pgfpathlineto{\pgfqpoint{4.956827in}{2.571364in}}%
\pgfpathlineto{\pgfqpoint{4.971161in}{2.585485in}}%
\pgfpathlineto{\pgfqpoint{4.985515in}{2.599771in}}%
\pgfpathlineto{\pgfqpoint{4.999889in}{2.614222in}}%
\pgfpathlineto{\pgfqpoint{5.007625in}{2.628263in}}%
\pgfpathlineto{\pgfqpoint{5.015354in}{2.642131in}}%
\pgfpathlineto{\pgfqpoint{5.023075in}{2.655827in}}%
\pgfpathlineto{\pgfqpoint{5.030790in}{2.669349in}}%
\pgfpathlineto{\pgfqpoint{5.016410in}{2.654725in}}%
\pgfpathlineto{\pgfqpoint{5.002051in}{2.640266in}}%
\pgfpathlineto{\pgfqpoint{4.987711in}{2.625972in}}%
\pgfpathlineto{\pgfqpoint{4.973391in}{2.611843in}}%
\pgfpathlineto{\pgfqpoint{4.965682in}{2.598482in}}%
\pgfpathlineto{\pgfqpoint{4.957966in}{2.584955in}}%
\pgfpathlineto{\pgfqpoint{4.950242in}{2.571263in}}%
\pgfpathlineto{\pgfqpoint{4.942512in}{2.557408in}}%
\pgfpathclose%
\pgfusepath{fill}%
\end{pgfscope}%
\begin{pgfscope}%
\pgfpathrectangle{\pgfqpoint{1.254980in}{0.150000in}}{\pgfqpoint{5.490039in}{5.490039in}}%
\pgfusepath{clip}%
\pgfsetbuttcap%
\pgfsetroundjoin%
\definecolor{currentfill}{rgb}{0.280868,0.160771,0.472899}%
\pgfsetfillcolor{currentfill}%
\pgfsetfillopacity{0.700000}%
\pgfsetlinewidth{0.000000pt}%
\definecolor{currentstroke}{rgb}{0.000000,0.000000,0.000000}%
\pgfsetstrokecolor{currentstroke}%
\pgfsetdash{}{0pt}%
\pgfpathmoveto{\pgfqpoint{4.078210in}{1.368696in}}%
\pgfpathlineto{\pgfqpoint{4.092020in}{1.372135in}}%
\pgfpathlineto{\pgfqpoint{4.105842in}{1.375729in}}%
\pgfpathlineto{\pgfqpoint{4.119674in}{1.379477in}}%
\pgfpathlineto{\pgfqpoint{4.133518in}{1.383378in}}%
\pgfpathlineto{\pgfqpoint{4.141482in}{1.398013in}}%
\pgfpathlineto{\pgfqpoint{4.149442in}{1.412745in}}%
\pgfpathlineto{\pgfqpoint{4.157399in}{1.427566in}}%
\pgfpathlineto{\pgfqpoint{4.165352in}{1.442472in}}%
\pgfpathlineto{\pgfqpoint{4.151509in}{1.437973in}}%
\pgfpathlineto{\pgfqpoint{4.137678in}{1.433629in}}%
\pgfpathlineto{\pgfqpoint{4.123858in}{1.429439in}}%
\pgfpathlineto{\pgfqpoint{4.110049in}{1.425403in}}%
\pgfpathlineto{\pgfqpoint{4.102096in}{1.411084in}}%
\pgfpathlineto{\pgfqpoint{4.094138in}{1.396855in}}%
\pgfpathlineto{\pgfqpoint{4.086176in}{1.382724in}}%
\pgfpathlineto{\pgfqpoint{4.078210in}{1.368696in}}%
\pgfpathclose%
\pgfusepath{fill}%
\end{pgfscope}%
\begin{pgfscope}%
\pgfpathrectangle{\pgfqpoint{1.254980in}{0.150000in}}{\pgfqpoint{5.490039in}{5.490039in}}%
\pgfusepath{clip}%
\pgfsetbuttcap%
\pgfsetroundjoin%
\definecolor{currentfill}{rgb}{0.180629,0.429975,0.557282}%
\pgfsetfillcolor{currentfill}%
\pgfsetfillopacity{0.700000}%
\pgfsetlinewidth{0.000000pt}%
\definecolor{currentstroke}{rgb}{0.000000,0.000000,0.000000}%
\pgfsetstrokecolor{currentstroke}%
\pgfsetdash{}{0pt}%
\pgfpathmoveto{\pgfqpoint{4.553934in}{1.980787in}}%
\pgfpathlineto{\pgfqpoint{4.567987in}{1.990751in}}%
\pgfpathlineto{\pgfqpoint{4.582056in}{2.000875in}}%
\pgfpathlineto{\pgfqpoint{4.596140in}{2.011158in}}%
\pgfpathlineto{\pgfqpoint{4.610241in}{2.021601in}}%
\pgfpathlineto{\pgfqpoint{4.618108in}{2.038388in}}%
\pgfpathlineto{\pgfqpoint{4.625970in}{2.055095in}}%
\pgfpathlineto{\pgfqpoint{4.633828in}{2.071718in}}%
\pgfpathlineto{\pgfqpoint{4.641681in}{2.088255in}}%
\pgfpathlineto{\pgfqpoint{4.627572in}{2.077431in}}%
\pgfpathlineto{\pgfqpoint{4.613480in}{2.066767in}}%
\pgfpathlineto{\pgfqpoint{4.599403in}{2.056264in}}%
\pgfpathlineto{\pgfqpoint{4.585343in}{2.045919in}}%
\pgfpathlineto{\pgfqpoint{4.577497in}{2.029752in}}%
\pgfpathlineto{\pgfqpoint{4.569647in}{2.013505in}}%
\pgfpathlineto{\pgfqpoint{4.561793in}{1.997182in}}%
\pgfpathlineto{\pgfqpoint{4.553934in}{1.980787in}}%
\pgfpathclose%
\pgfusepath{fill}%
\end{pgfscope}%
\begin{pgfscope}%
\pgfpathrectangle{\pgfqpoint{1.254980in}{0.150000in}}{\pgfqpoint{5.490039in}{5.490039in}}%
\pgfusepath{clip}%
\pgfsetbuttcap%
\pgfsetroundjoin%
\definecolor{currentfill}{rgb}{0.266941,0.748751,0.440573}%
\pgfsetfillcolor{currentfill}%
\pgfsetfillopacity{0.700000}%
\pgfsetlinewidth{0.000000pt}%
\definecolor{currentstroke}{rgb}{0.000000,0.000000,0.000000}%
\pgfsetstrokecolor{currentstroke}%
\pgfsetdash{}{0pt}%
\pgfpathmoveto{\pgfqpoint{5.180517in}{2.879141in}}%
\pgfpathlineto{\pgfqpoint{5.195010in}{2.894928in}}%
\pgfpathlineto{\pgfqpoint{5.209523in}{2.910882in}}%
\pgfpathlineto{\pgfqpoint{5.224058in}{2.927004in}}%
\pgfpathlineto{\pgfqpoint{5.238615in}{2.943293in}}%
\pgfpathlineto{\pgfqpoint{5.246229in}{2.954587in}}%
\pgfpathlineto{\pgfqpoint{5.253833in}{2.965680in}}%
\pgfpathlineto{\pgfqpoint{5.261428in}{2.976574in}}%
\pgfpathlineto{\pgfqpoint{5.269014in}{2.987268in}}%
\pgfpathlineto{\pgfqpoint{5.254456in}{2.970932in}}%
\pgfpathlineto{\pgfqpoint{5.239920in}{2.954763in}}%
\pgfpathlineto{\pgfqpoint{5.225405in}{2.938762in}}%
\pgfpathlineto{\pgfqpoint{5.210912in}{2.922929in}}%
\pgfpathlineto{\pgfqpoint{5.203327in}{2.912269in}}%
\pgfpathlineto{\pgfqpoint{5.195733in}{2.901418in}}%
\pgfpathlineto{\pgfqpoint{5.188129in}{2.890375in}}%
\pgfpathlineto{\pgfqpoint{5.180517in}{2.879141in}}%
\pgfpathclose%
\pgfusepath{fill}%
\end{pgfscope}%
\begin{pgfscope}%
\pgfpathrectangle{\pgfqpoint{1.254980in}{0.150000in}}{\pgfqpoint{5.490039in}{5.490039in}}%
\pgfusepath{clip}%
\pgfsetbuttcap%
\pgfsetroundjoin%
\definecolor{currentfill}{rgb}{0.175707,0.697900,0.491033}%
\pgfsetfillcolor{currentfill}%
\pgfsetfillopacity{0.700000}%
\pgfsetlinewidth{0.000000pt}%
\definecolor{currentstroke}{rgb}{0.000000,0.000000,0.000000}%
\pgfsetstrokecolor{currentstroke}%
\pgfsetdash{}{0pt}%
\pgfpathmoveto{\pgfqpoint{5.061572in}{2.721670in}}%
\pgfpathlineto{\pgfqpoint{5.075976in}{2.736602in}}%
\pgfpathlineto{\pgfqpoint{5.090400in}{2.751700in}}%
\pgfpathlineto{\pgfqpoint{5.104845in}{2.766964in}}%
\pgfpathlineto{\pgfqpoint{5.119310in}{2.782394in}}%
\pgfpathlineto{\pgfqpoint{5.126990in}{2.795151in}}%
\pgfpathlineto{\pgfqpoint{5.134662in}{2.807719in}}%
\pgfpathlineto{\pgfqpoint{5.142326in}{2.820098in}}%
\pgfpathlineto{\pgfqpoint{5.149981in}{2.832288in}}%
\pgfpathlineto{\pgfqpoint{5.135512in}{2.816747in}}%
\pgfpathlineto{\pgfqpoint{5.121064in}{2.801372in}}%
\pgfpathlineto{\pgfqpoint{5.106636in}{2.786164in}}%
\pgfpathlineto{\pgfqpoint{5.092229in}{2.771122in}}%
\pgfpathlineto{\pgfqpoint{5.084576in}{2.759031in}}%
\pgfpathlineto{\pgfqpoint{5.076916in}{2.746758in}}%
\pgfpathlineto{\pgfqpoint{5.069248in}{2.734304in}}%
\pgfpathlineto{\pgfqpoint{5.061572in}{2.721670in}}%
\pgfpathclose%
\pgfusepath{fill}%
\end{pgfscope}%
\begin{pgfscope}%
\pgfpathrectangle{\pgfqpoint{1.254980in}{0.150000in}}{\pgfqpoint{5.490039in}{5.490039in}}%
\pgfusepath{clip}%
\pgfsetbuttcap%
\pgfsetroundjoin%
\definecolor{currentfill}{rgb}{0.595839,0.848717,0.243329}%
\pgfsetfillcolor{currentfill}%
\pgfsetfillopacity{0.700000}%
\pgfsetlinewidth{0.000000pt}%
\definecolor{currentstroke}{rgb}{0.000000,0.000000,0.000000}%
\pgfsetstrokecolor{currentstroke}%
\pgfsetdash{}{0pt}%
\pgfpathmoveto{\pgfqpoint{5.506215in}{3.267145in}}%
\pgfpathlineto{\pgfqpoint{5.520964in}{3.284863in}}%
\pgfpathlineto{\pgfqpoint{5.535736in}{3.302751in}}%
\pgfpathlineto{\pgfqpoint{5.550532in}{3.320810in}}%
\pgfpathlineto{\pgfqpoint{5.557929in}{3.327897in}}%
\pgfpathlineto{\pgfqpoint{5.565314in}{3.334775in}}%
\pgfpathlineto{\pgfqpoint{5.572688in}{3.341446in}}%
\pgfpathlineto{\pgfqpoint{5.580050in}{3.347912in}}%
\pgfpathlineto{\pgfqpoint{5.565262in}{3.329972in}}%
\pgfpathlineto{\pgfqpoint{5.550497in}{3.312201in}}%
\pgfpathlineto{\pgfqpoint{5.535756in}{3.294600in}}%
\pgfpathlineto{\pgfqpoint{5.528388in}{3.288037in}}%
\pgfpathlineto{\pgfqpoint{5.521008in}{3.281274in}}%
\pgfpathlineto{\pgfqpoint{5.513617in}{3.274311in}}%
\pgfpathlineto{\pgfqpoint{5.506215in}{3.267145in}}%
\pgfpathclose%
\pgfusepath{fill}%
\end{pgfscope}%
\begin{pgfscope}%
\pgfpathrectangle{\pgfqpoint{1.254980in}{0.150000in}}{\pgfqpoint{5.490039in}{5.490039in}}%
\pgfusepath{clip}%
\pgfsetbuttcap%
\pgfsetroundjoin%
\definecolor{currentfill}{rgb}{0.153364,0.497000,0.557724}%
\pgfsetfillcolor{currentfill}%
\pgfsetfillopacity{0.700000}%
\pgfsetlinewidth{0.000000pt}%
\definecolor{currentstroke}{rgb}{0.000000,0.000000,0.000000}%
\pgfsetstrokecolor{currentstroke}%
\pgfsetdash{}{0pt}%
\pgfpathmoveto{\pgfqpoint{4.673051in}{2.153472in}}%
\pgfpathlineto{\pgfqpoint{4.687184in}{2.164808in}}%
\pgfpathlineto{\pgfqpoint{4.701335in}{2.176306in}}%
\pgfpathlineto{\pgfqpoint{4.715504in}{2.187964in}}%
\pgfpathlineto{\pgfqpoint{4.729689in}{2.199784in}}%
\pgfpathlineto{\pgfqpoint{4.737528in}{2.216174in}}%
\pgfpathlineto{\pgfqpoint{4.745362in}{2.232452in}}%
\pgfpathlineto{\pgfqpoint{4.753191in}{2.248613in}}%
\pgfpathlineto{\pgfqpoint{4.761015in}{2.264656in}}%
\pgfpathlineto{\pgfqpoint{4.746822in}{2.252512in}}%
\pgfpathlineto{\pgfqpoint{4.732645in}{2.240529in}}%
\pgfpathlineto{\pgfqpoint{4.718487in}{2.228709in}}%
\pgfpathlineto{\pgfqpoint{4.704345in}{2.217050in}}%
\pgfpathlineto{\pgfqpoint{4.696529in}{2.201319in}}%
\pgfpathlineto{\pgfqpoint{4.688708in}{2.185477in}}%
\pgfpathlineto{\pgfqpoint{4.680882in}{2.169527in}}%
\pgfpathlineto{\pgfqpoint{4.673051in}{2.153472in}}%
\pgfpathclose%
\pgfusepath{fill}%
\end{pgfscope}%
\begin{pgfscope}%
\pgfpathrectangle{\pgfqpoint{1.254980in}{0.150000in}}{\pgfqpoint{5.490039in}{5.490039in}}%
\pgfusepath{clip}%
\pgfsetbuttcap%
\pgfsetroundjoin%
\definecolor{currentfill}{rgb}{0.250425,0.274290,0.533103}%
\pgfsetfillcolor{currentfill}%
\pgfsetfillopacity{0.700000}%
\pgfsetlinewidth{0.000000pt}%
\definecolor{currentstroke}{rgb}{0.000000,0.000000,0.000000}%
\pgfsetstrokecolor{currentstroke}%
\pgfsetdash{}{0pt}%
\pgfpathmoveto{\pgfqpoint{4.284338in}{1.588114in}}%
\pgfpathlineto{\pgfqpoint{4.298245in}{1.594502in}}%
\pgfpathlineto{\pgfqpoint{4.312166in}{1.601046in}}%
\pgfpathlineto{\pgfqpoint{4.326101in}{1.607746in}}%
\pgfpathlineto{\pgfqpoint{4.340048in}{1.614602in}}%
\pgfpathlineto{\pgfqpoint{4.347974in}{1.631083in}}%
\pgfpathlineto{\pgfqpoint{4.355895in}{1.647583in}}%
\pgfpathlineto{\pgfqpoint{4.363814in}{1.664096in}}%
\pgfpathlineto{\pgfqpoint{4.371729in}{1.680618in}}%
\pgfpathlineto{\pgfqpoint{4.357776in}{1.673243in}}%
\pgfpathlineto{\pgfqpoint{4.343838in}{1.666024in}}%
\pgfpathlineto{\pgfqpoint{4.329912in}{1.658962in}}%
\pgfpathlineto{\pgfqpoint{4.316001in}{1.652056in}}%
\pgfpathlineto{\pgfqpoint{4.308090in}{1.636042in}}%
\pgfpathlineto{\pgfqpoint{4.300176in}{1.620044in}}%
\pgfpathlineto{\pgfqpoint{4.292259in}{1.604066in}}%
\pgfpathlineto{\pgfqpoint{4.284338in}{1.588114in}}%
\pgfpathclose%
\pgfusepath{fill}%
\end{pgfscope}%
\begin{pgfscope}%
\pgfpathrectangle{\pgfqpoint{1.254980in}{0.150000in}}{\pgfqpoint{5.490039in}{5.490039in}}%
\pgfusepath{clip}%
\pgfsetbuttcap%
\pgfsetroundjoin%
\definecolor{currentfill}{rgb}{0.280894,0.078907,0.402329}%
\pgfsetfillcolor{currentfill}%
\pgfsetfillopacity{0.700000}%
\pgfsetlinewidth{0.000000pt}%
\definecolor{currentstroke}{rgb}{0.000000,0.000000,0.000000}%
\pgfsetstrokecolor{currentstroke}%
\pgfsetdash{}{0pt}%
\pgfpathmoveto{\pgfqpoint{3.871988in}{1.204148in}}%
\pgfpathlineto{\pgfqpoint{3.885741in}{1.204412in}}%
\pgfpathlineto{\pgfqpoint{3.899503in}{1.204829in}}%
\pgfpathlineto{\pgfqpoint{3.913273in}{1.205400in}}%
\pgfpathlineto{\pgfqpoint{3.927051in}{1.206124in}}%
\pgfpathlineto{\pgfqpoint{3.935082in}{1.217760in}}%
\pgfpathlineto{\pgfqpoint{3.943108in}{1.229580in}}%
\pgfpathlineto{\pgfqpoint{3.951128in}{1.241577in}}%
\pgfpathlineto{\pgfqpoint{3.959143in}{1.253745in}}%
\pgfpathlineto{\pgfqpoint{3.945372in}{1.252345in}}%
\pgfpathlineto{\pgfqpoint{3.931611in}{1.251099in}}%
\pgfpathlineto{\pgfqpoint{3.917859in}{1.250006in}}%
\pgfpathlineto{\pgfqpoint{3.904116in}{1.249068in}}%
\pgfpathlineto{\pgfqpoint{3.896093in}{1.237565in}}%
\pgfpathlineto{\pgfqpoint{3.888064in}{1.226239in}}%
\pgfpathlineto{\pgfqpoint{3.880029in}{1.215098in}}%
\pgfpathlineto{\pgfqpoint{3.871988in}{1.204148in}}%
\pgfpathclose%
\pgfusepath{fill}%
\end{pgfscope}%
\begin{pgfscope}%
\pgfpathrectangle{\pgfqpoint{1.254980in}{0.150000in}}{\pgfqpoint{5.490039in}{5.490039in}}%
\pgfusepath{clip}%
\pgfsetbuttcap%
\pgfsetroundjoin%
\definecolor{currentfill}{rgb}{0.277941,0.056324,0.381191}%
\pgfsetfillcolor{currentfill}%
\pgfsetfillopacity{0.700000}%
\pgfsetlinewidth{0.000000pt}%
\definecolor{currentstroke}{rgb}{0.000000,0.000000,0.000000}%
\pgfsetstrokecolor{currentstroke}%
\pgfsetdash{}{0pt}%
\pgfpathmoveto{\pgfqpoint{3.784779in}{1.165678in}}%
\pgfpathlineto{\pgfqpoint{3.798513in}{1.164626in}}%
\pgfpathlineto{\pgfqpoint{3.812255in}{1.163727in}}%
\pgfpathlineto{\pgfqpoint{3.826004in}{1.162982in}}%
\pgfpathlineto{\pgfqpoint{3.839760in}{1.162390in}}%
\pgfpathlineto{\pgfqpoint{3.847827in}{1.172509in}}%
\pgfpathlineto{\pgfqpoint{3.855887in}{1.182847in}}%
\pgfpathlineto{\pgfqpoint{3.863941in}{1.193395in}}%
\pgfpathlineto{\pgfqpoint{3.871988in}{1.204148in}}%
\pgfpathlineto{\pgfqpoint{3.858244in}{1.204037in}}%
\pgfpathlineto{\pgfqpoint{3.844507in}{1.204081in}}%
\pgfpathlineto{\pgfqpoint{3.830779in}{1.204278in}}%
\pgfpathlineto{\pgfqpoint{3.817059in}{1.204629in}}%
\pgfpathlineto{\pgfqpoint{3.808999in}{1.194568in}}%
\pgfpathlineto{\pgfqpoint{3.800933in}{1.184717in}}%
\pgfpathlineto{\pgfqpoint{3.792860in}{1.175085in}}%
\pgfpathlineto{\pgfqpoint{3.784779in}{1.165678in}}%
\pgfpathclose%
\pgfusepath{fill}%
\end{pgfscope}%
\begin{pgfscope}%
\pgfpathrectangle{\pgfqpoint{1.254980in}{0.150000in}}{\pgfqpoint{5.490039in}{5.490039in}}%
\pgfusepath{clip}%
\pgfsetbuttcap%
\pgfsetroundjoin%
\definecolor{currentfill}{rgb}{0.220057,0.343307,0.549413}%
\pgfsetfillcolor{currentfill}%
\pgfsetfillopacity{0.700000}%
\pgfsetlinewidth{0.000000pt}%
\definecolor{currentstroke}{rgb}{0.000000,0.000000,0.000000}%
\pgfsetstrokecolor{currentstroke}%
\pgfsetdash{}{0pt}%
\pgfpathmoveto{\pgfqpoint{4.403354in}{1.746699in}}%
\pgfpathlineto{\pgfqpoint{4.417326in}{1.754722in}}%
\pgfpathlineto{\pgfqpoint{4.431313in}{1.762902in}}%
\pgfpathlineto{\pgfqpoint{4.445314in}{1.771240in}}%
\pgfpathlineto{\pgfqpoint{4.459330in}{1.779734in}}%
\pgfpathlineto{\pgfqpoint{4.467234in}{1.796705in}}%
\pgfpathlineto{\pgfqpoint{4.475134in}{1.813651in}}%
\pgfpathlineto{\pgfqpoint{4.483031in}{1.830567in}}%
\pgfpathlineto{\pgfqpoint{4.490924in}{1.847449in}}%
\pgfpathlineto{\pgfqpoint{4.476901in}{1.838489in}}%
\pgfpathlineto{\pgfqpoint{4.462893in}{1.829686in}}%
\pgfpathlineto{\pgfqpoint{4.448901in}{1.821042in}}%
\pgfpathlineto{\pgfqpoint{4.434923in}{1.812555in}}%
\pgfpathlineto{\pgfqpoint{4.427036in}{1.796127in}}%
\pgfpathlineto{\pgfqpoint{4.419146in}{1.779673in}}%
\pgfpathlineto{\pgfqpoint{4.411252in}{1.763195in}}%
\pgfpathlineto{\pgfqpoint{4.403354in}{1.746699in}}%
\pgfpathclose%
\pgfusepath{fill}%
\end{pgfscope}%
\begin{pgfscope}%
\pgfpathrectangle{\pgfqpoint{1.254980in}{0.150000in}}{\pgfqpoint{5.490039in}{5.490039in}}%
\pgfusepath{clip}%
\pgfsetbuttcap%
\pgfsetroundjoin%
\definecolor{currentfill}{rgb}{0.273006,0.204520,0.501721}%
\pgfsetfillcolor{currentfill}%
\pgfsetfillopacity{0.700000}%
\pgfsetlinewidth{0.000000pt}%
\definecolor{currentstroke}{rgb}{0.000000,0.000000,0.000000}%
\pgfsetstrokecolor{currentstroke}%
\pgfsetdash{}{0pt}%
\pgfpathmoveto{\pgfqpoint{4.165352in}{1.442472in}}%
\pgfpathlineto{\pgfqpoint{4.179206in}{1.447125in}}%
\pgfpathlineto{\pgfqpoint{4.193072in}{1.451933in}}%
\pgfpathlineto{\pgfqpoint{4.206950in}{1.456894in}}%
\pgfpathlineto{\pgfqpoint{4.220841in}{1.462010in}}%
\pgfpathlineto{\pgfqpoint{4.228790in}{1.477577in}}%
\pgfpathlineto{\pgfqpoint{4.236736in}{1.493211in}}%
\pgfpathlineto{\pgfqpoint{4.244679in}{1.508906in}}%
\pgfpathlineto{\pgfqpoint{4.252618in}{1.524656in}}%
\pgfpathlineto{\pgfqpoint{4.238725in}{1.518968in}}%
\pgfpathlineto{\pgfqpoint{4.224846in}{1.513434in}}%
\pgfpathlineto{\pgfqpoint{4.210978in}{1.508056in}}%
\pgfpathlineto{\pgfqpoint{4.197123in}{1.502832in}}%
\pgfpathlineto{\pgfqpoint{4.189186in}{1.487643in}}%
\pgfpathlineto{\pgfqpoint{4.181245in}{1.472516in}}%
\pgfpathlineto{\pgfqpoint{4.173300in}{1.457457in}}%
\pgfpathlineto{\pgfqpoint{4.165352in}{1.442472in}}%
\pgfpathclose%
\pgfusepath{fill}%
\end{pgfscope}%
\begin{pgfscope}%
\pgfpathrectangle{\pgfqpoint{1.254980in}{0.150000in}}{\pgfqpoint{5.490039in}{5.490039in}}%
\pgfusepath{clip}%
\pgfsetbuttcap%
\pgfsetroundjoin%
\definecolor{currentfill}{rgb}{0.283091,0.110553,0.431554}%
\pgfsetfillcolor{currentfill}%
\pgfsetfillopacity{0.700000}%
\pgfsetlinewidth{0.000000pt}%
\definecolor{currentstroke}{rgb}{0.000000,0.000000,0.000000}%
\pgfsetstrokecolor{currentstroke}%
\pgfsetdash{}{0pt}%
\pgfpathmoveto{\pgfqpoint{3.959143in}{1.253745in}}%
\pgfpathlineto{\pgfqpoint{3.972922in}{1.255298in}}%
\pgfpathlineto{\pgfqpoint{3.986711in}{1.257005in}}%
\pgfpathlineto{\pgfqpoint{4.000510in}{1.258865in}}%
\pgfpathlineto{\pgfqpoint{4.014318in}{1.260877in}}%
\pgfpathlineto{\pgfqpoint{4.022321in}{1.273869in}}%
\pgfpathlineto{\pgfqpoint{4.030319in}{1.287012in}}%
\pgfpathlineto{\pgfqpoint{4.038312in}{1.300300in}}%
\pgfpathlineto{\pgfqpoint{4.046301in}{1.313727in}}%
\pgfpathlineto{\pgfqpoint{4.032498in}{1.311064in}}%
\pgfpathlineto{\pgfqpoint{4.018705in}{1.308554in}}%
\pgfpathlineto{\pgfqpoint{4.004921in}{1.306198in}}%
\pgfpathlineto{\pgfqpoint{3.991148in}{1.303996in}}%
\pgfpathlineto{\pgfqpoint{3.983154in}{1.291209in}}%
\pgfpathlineto{\pgfqpoint{3.975156in}{1.278567in}}%
\pgfpathlineto{\pgfqpoint{3.967152in}{1.266077in}}%
\pgfpathlineto{\pgfqpoint{3.959143in}{1.253745in}}%
\pgfpathclose%
\pgfusepath{fill}%
\end{pgfscope}%
\begin{pgfscope}%
\pgfpathrectangle{\pgfqpoint{1.254980in}{0.150000in}}{\pgfqpoint{5.490039in}{5.490039in}}%
\pgfusepath{clip}%
\pgfsetbuttcap%
\pgfsetroundjoin%
\definecolor{currentfill}{rgb}{0.129933,0.559582,0.551864}%
\pgfsetfillcolor{currentfill}%
\pgfsetfillopacity{0.700000}%
\pgfsetlinewidth{0.000000pt}%
\definecolor{currentstroke}{rgb}{0.000000,0.000000,0.000000}%
\pgfsetstrokecolor{currentstroke}%
\pgfsetdash{}{0pt}%
\pgfpathmoveto{\pgfqpoint{4.792260in}{2.327588in}}%
\pgfpathlineto{\pgfqpoint{4.806479in}{2.340189in}}%
\pgfpathlineto{\pgfqpoint{4.820717in}{2.352952in}}%
\pgfpathlineto{\pgfqpoint{4.834973in}{2.365878in}}%
\pgfpathlineto{\pgfqpoint{4.849248in}{2.378966in}}%
\pgfpathlineto{\pgfqpoint{4.857053in}{2.394653in}}%
\pgfpathlineto{\pgfqpoint{4.864853in}{2.410198in}}%
\pgfpathlineto{\pgfqpoint{4.872647in}{2.425600in}}%
\pgfpathlineto{\pgfqpoint{4.880435in}{2.440856in}}%
\pgfpathlineto{\pgfqpoint{4.866152in}{2.427502in}}%
\pgfpathlineto{\pgfqpoint{4.851888in}{2.414311in}}%
\pgfpathlineto{\pgfqpoint{4.837643in}{2.401283in}}%
\pgfpathlineto{\pgfqpoint{4.823416in}{2.388419in}}%
\pgfpathlineto{\pgfqpoint{4.815635in}{2.373416in}}%
\pgfpathlineto{\pgfqpoint{4.807849in}{2.358276in}}%
\pgfpathlineto{\pgfqpoint{4.800057in}{2.342999in}}%
\pgfpathlineto{\pgfqpoint{4.792260in}{2.327588in}}%
\pgfpathclose%
\pgfusepath{fill}%
\end{pgfscope}%
\begin{pgfscope}%
\pgfpathrectangle{\pgfqpoint{1.254980in}{0.150000in}}{\pgfqpoint{5.490039in}{5.490039in}}%
\pgfusepath{clip}%
\pgfsetbuttcap%
\pgfsetroundjoin%
\definecolor{currentfill}{rgb}{0.188923,0.410910,0.556326}%
\pgfsetfillcolor{currentfill}%
\pgfsetfillopacity{0.700000}%
\pgfsetlinewidth{0.000000pt}%
\definecolor{currentstroke}{rgb}{0.000000,0.000000,0.000000}%
\pgfsetstrokecolor{currentstroke}%
\pgfsetdash{}{0pt}%
\pgfpathmoveto{\pgfqpoint{4.522460in}{1.914551in}}%
\pgfpathlineto{\pgfqpoint{4.536505in}{1.924107in}}%
\pgfpathlineto{\pgfqpoint{4.550566in}{1.933821in}}%
\pgfpathlineto{\pgfqpoint{4.564643in}{1.943694in}}%
\pgfpathlineto{\pgfqpoint{4.578736in}{1.953726in}}%
\pgfpathlineto{\pgfqpoint{4.586618in}{1.970796in}}%
\pgfpathlineto{\pgfqpoint{4.594496in}{1.987801in}}%
\pgfpathlineto{\pgfqpoint{4.602371in}{2.004737in}}%
\pgfpathlineto{\pgfqpoint{4.610241in}{2.021601in}}%
\pgfpathlineto{\pgfqpoint{4.596140in}{2.011158in}}%
\pgfpathlineto{\pgfqpoint{4.582056in}{2.000875in}}%
\pgfpathlineto{\pgfqpoint{4.567987in}{1.990751in}}%
\pgfpathlineto{\pgfqpoint{4.553934in}{1.980787in}}%
\pgfpathlineto{\pgfqpoint{4.546072in}{1.964322in}}%
\pgfpathlineto{\pgfqpoint{4.538205in}{1.947793in}}%
\pgfpathlineto{\pgfqpoint{4.530334in}{1.931201in}}%
\pgfpathlineto{\pgfqpoint{4.522460in}{1.914551in}}%
\pgfpathclose%
\pgfusepath{fill}%
\end{pgfscope}%
\begin{pgfscope}%
\pgfpathrectangle{\pgfqpoint{1.254980in}{0.150000in}}{\pgfqpoint{5.490039in}{5.490039in}}%
\pgfusepath{clip}%
\pgfsetbuttcap%
\pgfsetroundjoin%
\definecolor{currentfill}{rgb}{0.487026,0.823929,0.312321}%
\pgfsetfillcolor{currentfill}%
\pgfsetfillopacity{0.700000}%
\pgfsetlinewidth{0.000000pt}%
\definecolor{currentstroke}{rgb}{0.000000,0.000000,0.000000}%
\pgfsetstrokecolor{currentstroke}%
\pgfsetdash{}{0pt}%
\pgfpathmoveto{\pgfqpoint{5.387792in}{3.132660in}}%
\pgfpathlineto{\pgfqpoint{5.402458in}{3.149836in}}%
\pgfpathlineto{\pgfqpoint{5.417147in}{3.167180in}}%
\pgfpathlineto{\pgfqpoint{5.431859in}{3.184695in}}%
\pgfpathlineto{\pgfqpoint{5.446593in}{3.202379in}}%
\pgfpathlineto{\pgfqpoint{5.454085in}{3.211209in}}%
\pgfpathlineto{\pgfqpoint{5.461565in}{3.219827in}}%
\pgfpathlineto{\pgfqpoint{5.469035in}{3.228234in}}%
\pgfpathlineto{\pgfqpoint{5.476493in}{3.236432in}}%
\pgfpathlineto{\pgfqpoint{5.461762in}{3.218799in}}%
\pgfpathlineto{\pgfqpoint{5.447054in}{3.201336in}}%
\pgfpathlineto{\pgfqpoint{5.432369in}{3.184043in}}%
\pgfpathlineto{\pgfqpoint{5.417707in}{3.166919in}}%
\pgfpathlineto{\pgfqpoint{5.410244in}{3.158658in}}%
\pgfpathlineto{\pgfqpoint{5.402770in}{3.150195in}}%
\pgfpathlineto{\pgfqpoint{5.395286in}{3.141529in}}%
\pgfpathlineto{\pgfqpoint{5.387792in}{3.132660in}}%
\pgfpathclose%
\pgfusepath{fill}%
\end{pgfscope}%
\begin{pgfscope}%
\pgfpathrectangle{\pgfqpoint{1.254980in}{0.150000in}}{\pgfqpoint{5.490039in}{5.490039in}}%
\pgfusepath{clip}%
\pgfsetbuttcap%
\pgfsetroundjoin%
\definecolor{currentfill}{rgb}{0.120638,0.625828,0.533488}%
\pgfsetfillcolor{currentfill}%
\pgfsetfillopacity{0.700000}%
\pgfsetlinewidth{0.000000pt}%
\definecolor{currentstroke}{rgb}{0.000000,0.000000,0.000000}%
\pgfsetstrokecolor{currentstroke}%
\pgfsetdash{}{0pt}%
\pgfpathmoveto{\pgfqpoint{4.911525in}{2.500377in}}%
\pgfpathlineto{\pgfqpoint{4.925833in}{2.514130in}}%
\pgfpathlineto{\pgfqpoint{4.940161in}{2.528047in}}%
\pgfpathlineto{\pgfqpoint{4.954509in}{2.542129in}}%
\pgfpathlineto{\pgfqpoint{4.968876in}{2.556375in}}%
\pgfpathlineto{\pgfqpoint{4.976639in}{2.571086in}}%
\pgfpathlineto{\pgfqpoint{4.984396in}{2.585632in}}%
\pgfpathlineto{\pgfqpoint{4.992146in}{2.600012in}}%
\pgfpathlineto{\pgfqpoint{4.999889in}{2.614222in}}%
\pgfpathlineto{\pgfqpoint{4.985515in}{2.599771in}}%
\pgfpathlineto{\pgfqpoint{4.971161in}{2.585485in}}%
\pgfpathlineto{\pgfqpoint{4.956827in}{2.571364in}}%
\pgfpathlineto{\pgfqpoint{4.942512in}{2.557408in}}%
\pgfpathlineto{\pgfqpoint{4.934775in}{2.543390in}}%
\pgfpathlineto{\pgfqpoint{4.927032in}{2.529211in}}%
\pgfpathlineto{\pgfqpoint{4.919281in}{2.514873in}}%
\pgfpathlineto{\pgfqpoint{4.911525in}{2.500377in}}%
\pgfpathclose%
\pgfusepath{fill}%
\end{pgfscope}%
\begin{pgfscope}%
\pgfpathrectangle{\pgfqpoint{1.254980in}{0.150000in}}{\pgfqpoint{5.490039in}{5.490039in}}%
\pgfusepath{clip}%
\pgfsetbuttcap%
\pgfsetroundjoin%
\definecolor{currentfill}{rgb}{0.282290,0.145912,0.461510}%
\pgfsetfillcolor{currentfill}%
\pgfsetfillopacity{0.700000}%
\pgfsetlinewidth{0.000000pt}%
\definecolor{currentstroke}{rgb}{0.000000,0.000000,0.000000}%
\pgfsetstrokecolor{currentstroke}%
\pgfsetdash{}{0pt}%
\pgfpathmoveto{\pgfqpoint{4.046301in}{1.313727in}}%
\pgfpathlineto{\pgfqpoint{4.060115in}{1.316543in}}%
\pgfpathlineto{\pgfqpoint{4.073939in}{1.319513in}}%
\pgfpathlineto{\pgfqpoint{4.087773in}{1.322636in}}%
\pgfpathlineto{\pgfqpoint{4.101619in}{1.325912in}}%
\pgfpathlineto{\pgfqpoint{4.109600in}{1.340106in}}%
\pgfpathlineto{\pgfqpoint{4.117576in}{1.354418in}}%
\pgfpathlineto{\pgfqpoint{4.125549in}{1.368844in}}%
\pgfpathlineto{\pgfqpoint{4.133518in}{1.383378in}}%
\pgfpathlineto{\pgfqpoint{4.119674in}{1.379477in}}%
\pgfpathlineto{\pgfqpoint{4.105842in}{1.375729in}}%
\pgfpathlineto{\pgfqpoint{4.092020in}{1.372135in}}%
\pgfpathlineto{\pgfqpoint{4.078210in}{1.368696in}}%
\pgfpathlineto{\pgfqpoint{4.070239in}{1.354776in}}%
\pgfpathlineto{\pgfqpoint{4.062265in}{1.340971in}}%
\pgfpathlineto{\pgfqpoint{4.054285in}{1.327285in}}%
\pgfpathlineto{\pgfqpoint{4.046301in}{1.313727in}}%
\pgfpathclose%
\pgfusepath{fill}%
\end{pgfscope}%
\begin{pgfscope}%
\pgfpathrectangle{\pgfqpoint{1.254980in}{0.150000in}}{\pgfqpoint{5.490039in}{5.490039in}}%
\pgfusepath{clip}%
\pgfsetbuttcap%
\pgfsetroundjoin%
\definecolor{currentfill}{rgb}{0.162016,0.687316,0.499129}%
\pgfsetfillcolor{currentfill}%
\pgfsetfillopacity{0.700000}%
\pgfsetlinewidth{0.000000pt}%
\definecolor{currentstroke}{rgb}{0.000000,0.000000,0.000000}%
\pgfsetstrokecolor{currentstroke}%
\pgfsetdash{}{0pt}%
\pgfpathmoveto{\pgfqpoint{5.030790in}{2.669349in}}%
\pgfpathlineto{\pgfqpoint{5.045189in}{2.684139in}}%
\pgfpathlineto{\pgfqpoint{5.059609in}{2.699094in}}%
\pgfpathlineto{\pgfqpoint{5.074049in}{2.714216in}}%
\pgfpathlineto{\pgfqpoint{5.088509in}{2.729504in}}%
\pgfpathlineto{\pgfqpoint{5.096221in}{2.743004in}}%
\pgfpathlineto{\pgfqpoint{5.103925in}{2.756320in}}%
\pgfpathlineto{\pgfqpoint{5.111622in}{2.769450in}}%
\pgfpathlineto{\pgfqpoint{5.119310in}{2.782394in}}%
\pgfpathlineto{\pgfqpoint{5.104845in}{2.766964in}}%
\pgfpathlineto{\pgfqpoint{5.090400in}{2.751700in}}%
\pgfpathlineto{\pgfqpoint{5.075976in}{2.736602in}}%
\pgfpathlineto{\pgfqpoint{5.061572in}{2.721670in}}%
\pgfpathlineto{\pgfqpoint{5.053888in}{2.708857in}}%
\pgfpathlineto{\pgfqpoint{5.046196in}{2.695865in}}%
\pgfpathlineto{\pgfqpoint{5.038497in}{2.682695in}}%
\pgfpathlineto{\pgfqpoint{5.030790in}{2.669349in}}%
\pgfpathclose%
\pgfusepath{fill}%
\end{pgfscope}%
\begin{pgfscope}%
\pgfpathrectangle{\pgfqpoint{1.254980in}{0.150000in}}{\pgfqpoint{5.490039in}{5.490039in}}%
\pgfusepath{clip}%
\pgfsetbuttcap%
\pgfsetroundjoin%
\definecolor{currentfill}{rgb}{0.160665,0.478540,0.558115}%
\pgfsetfillcolor{currentfill}%
\pgfsetfillopacity{0.700000}%
\pgfsetlinewidth{0.000000pt}%
\definecolor{currentstroke}{rgb}{0.000000,0.000000,0.000000}%
\pgfsetstrokecolor{currentstroke}%
\pgfsetdash{}{0pt}%
\pgfpathmoveto{\pgfqpoint{4.641681in}{2.088255in}}%
\pgfpathlineto{\pgfqpoint{4.655807in}{2.099240in}}%
\pgfpathlineto{\pgfqpoint{4.669949in}{2.110384in}}%
\pgfpathlineto{\pgfqpoint{4.684109in}{2.121689in}}%
\pgfpathlineto{\pgfqpoint{4.698286in}{2.133155in}}%
\pgfpathlineto{\pgfqpoint{4.706143in}{2.149966in}}%
\pgfpathlineto{\pgfqpoint{4.713997in}{2.166677in}}%
\pgfpathlineto{\pgfqpoint{4.721845in}{2.183284in}}%
\pgfpathlineto{\pgfqpoint{4.729689in}{2.199784in}}%
\pgfpathlineto{\pgfqpoint{4.715504in}{2.187964in}}%
\pgfpathlineto{\pgfqpoint{4.701335in}{2.176306in}}%
\pgfpathlineto{\pgfqpoint{4.687184in}{2.164808in}}%
\pgfpathlineto{\pgfqpoint{4.673051in}{2.153472in}}%
\pgfpathlineto{\pgfqpoint{4.665215in}{2.137314in}}%
\pgfpathlineto{\pgfqpoint{4.657375in}{2.121056in}}%
\pgfpathlineto{\pgfqpoint{4.649530in}{2.104702in}}%
\pgfpathlineto{\pgfqpoint{4.641681in}{2.088255in}}%
\pgfpathclose%
\pgfusepath{fill}%
\end{pgfscope}%
\begin{pgfscope}%
\pgfpathrectangle{\pgfqpoint{1.254980in}{0.150000in}}{\pgfqpoint{5.490039in}{5.490039in}}%
\pgfusepath{clip}%
\pgfsetbuttcap%
\pgfsetroundjoin%
\definecolor{currentfill}{rgb}{0.369214,0.788888,0.382914}%
\pgfsetfillcolor{currentfill}%
\pgfsetfillopacity{0.700000}%
\pgfsetlinewidth{0.000000pt}%
\definecolor{currentstroke}{rgb}{0.000000,0.000000,0.000000}%
\pgfsetstrokecolor{currentstroke}%
\pgfsetdash{}{0pt}%
\pgfpathmoveto{\pgfqpoint{5.269014in}{2.987268in}}%
\pgfpathlineto{\pgfqpoint{5.283593in}{3.003772in}}%
\pgfpathlineto{\pgfqpoint{5.298195in}{3.020445in}}%
\pgfpathlineto{\pgfqpoint{5.312818in}{3.037287in}}%
\pgfpathlineto{\pgfqpoint{5.327464in}{3.054297in}}%
\pgfpathlineto{\pgfqpoint{5.335041in}{3.064818in}}%
\pgfpathlineto{\pgfqpoint{5.342607in}{3.075130in}}%
\pgfpathlineto{\pgfqpoint{5.350164in}{3.085235in}}%
\pgfpathlineto{\pgfqpoint{5.357710in}{3.095133in}}%
\pgfpathlineto{\pgfqpoint{5.343064in}{3.078108in}}%
\pgfpathlineto{\pgfqpoint{5.328441in}{3.061252in}}%
\pgfpathlineto{\pgfqpoint{5.313840in}{3.044565in}}%
\pgfpathlineto{\pgfqpoint{5.299261in}{3.028047in}}%
\pgfpathlineto{\pgfqpoint{5.291714in}{3.018151in}}%
\pgfpathlineto{\pgfqpoint{5.284157in}{3.008056in}}%
\pgfpathlineto{\pgfqpoint{5.276590in}{2.997762in}}%
\pgfpathlineto{\pgfqpoint{5.269014in}{2.987268in}}%
\pgfpathclose%
\pgfusepath{fill}%
\end{pgfscope}%
\begin{pgfscope}%
\pgfpathrectangle{\pgfqpoint{1.254980in}{0.150000in}}{\pgfqpoint{5.490039in}{5.490039in}}%
\pgfusepath{clip}%
\pgfsetbuttcap%
\pgfsetroundjoin%
\definecolor{currentfill}{rgb}{0.252899,0.742211,0.448284}%
\pgfsetfillcolor{currentfill}%
\pgfsetfillopacity{0.700000}%
\pgfsetlinewidth{0.000000pt}%
\definecolor{currentstroke}{rgb}{0.000000,0.000000,0.000000}%
\pgfsetstrokecolor{currentstroke}%
\pgfsetdash{}{0pt}%
\pgfpathmoveto{\pgfqpoint{5.149981in}{2.832288in}}%
\pgfpathlineto{\pgfqpoint{5.164472in}{2.847996in}}%
\pgfpathlineto{\pgfqpoint{5.178983in}{2.863872in}}%
\pgfpathlineto{\pgfqpoint{5.193515in}{2.879914in}}%
\pgfpathlineto{\pgfqpoint{5.208069in}{2.896125in}}%
\pgfpathlineto{\pgfqpoint{5.215719in}{2.908216in}}%
\pgfpathlineto{\pgfqpoint{5.223360in}{2.920107in}}%
\pgfpathlineto{\pgfqpoint{5.230992in}{2.931800in}}%
\pgfpathlineto{\pgfqpoint{5.238615in}{2.943293in}}%
\pgfpathlineto{\pgfqpoint{5.224058in}{2.927004in}}%
\pgfpathlineto{\pgfqpoint{5.209523in}{2.910882in}}%
\pgfpathlineto{\pgfqpoint{5.195010in}{2.894928in}}%
\pgfpathlineto{\pgfqpoint{5.180517in}{2.879141in}}%
\pgfpathlineto{\pgfqpoint{5.172896in}{2.867714in}}%
\pgfpathlineto{\pgfqpoint{5.165267in}{2.856096in}}%
\pgfpathlineto{\pgfqpoint{5.157628in}{2.844287in}}%
\pgfpathlineto{\pgfqpoint{5.149981in}{2.832288in}}%
\pgfpathclose%
\pgfusepath{fill}%
\end{pgfscope}%
\begin{pgfscope}%
\pgfpathrectangle{\pgfqpoint{1.254980in}{0.150000in}}{\pgfqpoint{5.490039in}{5.490039in}}%
\pgfusepath{clip}%
\pgfsetbuttcap%
\pgfsetroundjoin%
\definecolor{currentfill}{rgb}{0.258965,0.251537,0.524736}%
\pgfsetfillcolor{currentfill}%
\pgfsetfillopacity{0.700000}%
\pgfsetlinewidth{0.000000pt}%
\definecolor{currentstroke}{rgb}{0.000000,0.000000,0.000000}%
\pgfsetstrokecolor{currentstroke}%
\pgfsetdash{}{0pt}%
\pgfpathmoveto{\pgfqpoint{4.252618in}{1.524656in}}%
\pgfpathlineto{\pgfqpoint{4.266522in}{1.530500in}}%
\pgfpathlineto{\pgfqpoint{4.280440in}{1.536498in}}%
\pgfpathlineto{\pgfqpoint{4.294370in}{1.542651in}}%
\pgfpathlineto{\pgfqpoint{4.308314in}{1.548959in}}%
\pgfpathlineto{\pgfqpoint{4.316253in}{1.565317in}}%
\pgfpathlineto{\pgfqpoint{4.324188in}{1.581713in}}%
\pgfpathlineto{\pgfqpoint{4.332120in}{1.598143in}}%
\pgfpathlineto{\pgfqpoint{4.340048in}{1.614602in}}%
\pgfpathlineto{\pgfqpoint{4.326101in}{1.607746in}}%
\pgfpathlineto{\pgfqpoint{4.312166in}{1.601046in}}%
\pgfpathlineto{\pgfqpoint{4.298245in}{1.594502in}}%
\pgfpathlineto{\pgfqpoint{4.284338in}{1.588114in}}%
\pgfpathlineto{\pgfqpoint{4.276413in}{1.572191in}}%
\pgfpathlineto{\pgfqpoint{4.268485in}{1.556304in}}%
\pgfpathlineto{\pgfqpoint{4.260553in}{1.540458in}}%
\pgfpathlineto{\pgfqpoint{4.252618in}{1.524656in}}%
\pgfpathclose%
\pgfusepath{fill}%
\end{pgfscope}%
\begin{pgfscope}%
\pgfpathrectangle{\pgfqpoint{1.254980in}{0.150000in}}{\pgfqpoint{5.490039in}{5.490039in}}%
\pgfusepath{clip}%
\pgfsetbuttcap%
\pgfsetroundjoin%
\definecolor{currentfill}{rgb}{0.229739,0.322361,0.545706}%
\pgfsetfillcolor{currentfill}%
\pgfsetfillopacity{0.700000}%
\pgfsetlinewidth{0.000000pt}%
\definecolor{currentstroke}{rgb}{0.000000,0.000000,0.000000}%
\pgfsetstrokecolor{currentstroke}%
\pgfsetdash{}{0pt}%
\pgfpathmoveto{\pgfqpoint{4.371729in}{1.680618in}}%
\pgfpathlineto{\pgfqpoint{4.385695in}{1.688149in}}%
\pgfpathlineto{\pgfqpoint{4.399676in}{1.695836in}}%
\pgfpathlineto{\pgfqpoint{4.413671in}{1.703680in}}%
\pgfpathlineto{\pgfqpoint{4.427680in}{1.711680in}}%
\pgfpathlineto{\pgfqpoint{4.435597in}{1.728710in}}%
\pgfpathlineto{\pgfqpoint{4.443512in}{1.745732in}}%
\pgfpathlineto{\pgfqpoint{4.451422in}{1.762741in}}%
\pgfpathlineto{\pgfqpoint{4.459330in}{1.779734in}}%
\pgfpathlineto{\pgfqpoint{4.445314in}{1.771240in}}%
\pgfpathlineto{\pgfqpoint{4.431313in}{1.762902in}}%
\pgfpathlineto{\pgfqpoint{4.417326in}{1.754722in}}%
\pgfpathlineto{\pgfqpoint{4.403354in}{1.746699in}}%
\pgfpathlineto{\pgfqpoint{4.395453in}{1.730189in}}%
\pgfpathlineto{\pgfqpoint{4.387548in}{1.713669in}}%
\pgfpathlineto{\pgfqpoint{4.379640in}{1.697144in}}%
\pgfpathlineto{\pgfqpoint{4.371729in}{1.680618in}}%
\pgfpathclose%
\pgfusepath{fill}%
\end{pgfscope}%
\begin{pgfscope}%
\pgfpathrectangle{\pgfqpoint{1.254980in}{0.150000in}}{\pgfqpoint{5.490039in}{5.490039in}}%
\pgfusepath{clip}%
\pgfsetbuttcap%
\pgfsetroundjoin%
\definecolor{currentfill}{rgb}{0.277134,0.185228,0.489898}%
\pgfsetfillcolor{currentfill}%
\pgfsetfillopacity{0.700000}%
\pgfsetlinewidth{0.000000pt}%
\definecolor{currentstroke}{rgb}{0.000000,0.000000,0.000000}%
\pgfsetstrokecolor{currentstroke}%
\pgfsetdash{}{0pt}%
\pgfpathmoveto{\pgfqpoint{4.133518in}{1.383378in}}%
\pgfpathlineto{\pgfqpoint{4.147372in}{1.387433in}}%
\pgfpathlineto{\pgfqpoint{4.161238in}{1.391642in}}%
\pgfpathlineto{\pgfqpoint{4.175116in}{1.396004in}}%
\pgfpathlineto{\pgfqpoint{4.189005in}{1.400520in}}%
\pgfpathlineto{\pgfqpoint{4.196970in}{1.415765in}}%
\pgfpathlineto{\pgfqpoint{4.204930in}{1.431099in}}%
\pgfpathlineto{\pgfqpoint{4.212887in}{1.446516in}}%
\pgfpathlineto{\pgfqpoint{4.220841in}{1.462010in}}%
\pgfpathlineto{\pgfqpoint{4.206950in}{1.456894in}}%
\pgfpathlineto{\pgfqpoint{4.193072in}{1.451933in}}%
\pgfpathlineto{\pgfqpoint{4.179206in}{1.447125in}}%
\pgfpathlineto{\pgfqpoint{4.165352in}{1.442472in}}%
\pgfpathlineto{\pgfqpoint{4.157399in}{1.427566in}}%
\pgfpathlineto{\pgfqpoint{4.149442in}{1.412745in}}%
\pgfpathlineto{\pgfqpoint{4.141482in}{1.398013in}}%
\pgfpathlineto{\pgfqpoint{4.133518in}{1.383378in}}%
\pgfpathclose%
\pgfusepath{fill}%
\end{pgfscope}%
\begin{pgfscope}%
\pgfpathrectangle{\pgfqpoint{1.254980in}{0.150000in}}{\pgfqpoint{5.490039in}{5.490039in}}%
\pgfusepath{clip}%
\pgfsetbuttcap%
\pgfsetroundjoin%
\definecolor{currentfill}{rgb}{0.135066,0.544853,0.554029}%
\pgfsetfillcolor{currentfill}%
\pgfsetfillopacity{0.700000}%
\pgfsetlinewidth{0.000000pt}%
\definecolor{currentstroke}{rgb}{0.000000,0.000000,0.000000}%
\pgfsetstrokecolor{currentstroke}%
\pgfsetdash{}{0pt}%
\pgfpathmoveto{\pgfqpoint{4.761015in}{2.264656in}}%
\pgfpathlineto{\pgfqpoint{4.775227in}{2.276962in}}%
\pgfpathlineto{\pgfqpoint{4.789457in}{2.289430in}}%
\pgfpathlineto{\pgfqpoint{4.803705in}{2.302060in}}%
\pgfpathlineto{\pgfqpoint{4.817971in}{2.314853in}}%
\pgfpathlineto{\pgfqpoint{4.825798in}{2.331081in}}%
\pgfpathlineto{\pgfqpoint{4.833620in}{2.347178in}}%
\pgfpathlineto{\pgfqpoint{4.841437in}{2.363141in}}%
\pgfpathlineto{\pgfqpoint{4.849248in}{2.378966in}}%
\pgfpathlineto{\pgfqpoint{4.834973in}{2.365878in}}%
\pgfpathlineto{\pgfqpoint{4.820717in}{2.352952in}}%
\pgfpathlineto{\pgfqpoint{4.806479in}{2.340189in}}%
\pgfpathlineto{\pgfqpoint{4.792260in}{2.327588in}}%
\pgfpathlineto{\pgfqpoint{4.784457in}{2.312046in}}%
\pgfpathlineto{\pgfqpoint{4.776648in}{2.296375in}}%
\pgfpathlineto{\pgfqpoint{4.768834in}{2.280577in}}%
\pgfpathlineto{\pgfqpoint{4.761015in}{2.264656in}}%
\pgfpathclose%
\pgfusepath{fill}%
\end{pgfscope}%
\begin{pgfscope}%
\pgfpathrectangle{\pgfqpoint{1.254980in}{0.150000in}}{\pgfqpoint{5.490039in}{5.490039in}}%
\pgfusepath{clip}%
\pgfsetbuttcap%
\pgfsetroundjoin%
\definecolor{currentfill}{rgb}{0.197636,0.391528,0.554969}%
\pgfsetfillcolor{currentfill}%
\pgfsetfillopacity{0.700000}%
\pgfsetlinewidth{0.000000pt}%
\definecolor{currentstroke}{rgb}{0.000000,0.000000,0.000000}%
\pgfsetstrokecolor{currentstroke}%
\pgfsetdash{}{0pt}%
\pgfpathmoveto{\pgfqpoint{4.490924in}{1.847449in}}%
\pgfpathlineto{\pgfqpoint{4.504962in}{1.856567in}}%
\pgfpathlineto{\pgfqpoint{4.519015in}{1.865842in}}%
\pgfpathlineto{\pgfqpoint{4.533084in}{1.875276in}}%
\pgfpathlineto{\pgfqpoint{4.547168in}{1.884869in}}%
\pgfpathlineto{\pgfqpoint{4.555066in}{1.902161in}}%
\pgfpathlineto{\pgfqpoint{4.562959in}{1.919405in}}%
\pgfpathlineto{\pgfqpoint{4.570849in}{1.936594in}}%
\pgfpathlineto{\pgfqpoint{4.578736in}{1.953726in}}%
\pgfpathlineto{\pgfqpoint{4.564643in}{1.943694in}}%
\pgfpathlineto{\pgfqpoint{4.550566in}{1.933821in}}%
\pgfpathlineto{\pgfqpoint{4.536505in}{1.924107in}}%
\pgfpathlineto{\pgfqpoint{4.522460in}{1.914551in}}%
\pgfpathlineto{\pgfqpoint{4.514581in}{1.897847in}}%
\pgfpathlineto{\pgfqpoint{4.506699in}{1.881093in}}%
\pgfpathlineto{\pgfqpoint{4.498813in}{1.864292in}}%
\pgfpathlineto{\pgfqpoint{4.490924in}{1.847449in}}%
\pgfpathclose%
\pgfusepath{fill}%
\end{pgfscope}%
\begin{pgfscope}%
\pgfpathrectangle{\pgfqpoint{1.254980in}{0.150000in}}{\pgfqpoint{5.490039in}{5.490039in}}%
\pgfusepath{clip}%
\pgfsetbuttcap%
\pgfsetroundjoin%
\definecolor{currentfill}{rgb}{0.595839,0.848717,0.243329}%
\pgfsetfillcolor{currentfill}%
\pgfsetfillopacity{0.700000}%
\pgfsetlinewidth{0.000000pt}%
\definecolor{currentstroke}{rgb}{0.000000,0.000000,0.000000}%
\pgfsetstrokecolor{currentstroke}%
\pgfsetdash{}{0pt}%
\pgfpathmoveto{\pgfqpoint{5.476493in}{3.236432in}}%
\pgfpathlineto{\pgfqpoint{5.491248in}{3.254235in}}%
\pgfpathlineto{\pgfqpoint{5.506025in}{3.272208in}}%
\pgfpathlineto{\pgfqpoint{5.520827in}{3.290352in}}%
\pgfpathlineto{\pgfqpoint{5.528270in}{3.298286in}}%
\pgfpathlineto{\pgfqpoint{5.535702in}{3.306006in}}%
\pgfpathlineto{\pgfqpoint{5.543123in}{3.313514in}}%
\pgfpathlineto{\pgfqpoint{5.550532in}{3.320810in}}%
\pgfpathlineto{\pgfqpoint{5.535736in}{3.302751in}}%
\pgfpathlineto{\pgfqpoint{5.520964in}{3.284863in}}%
\pgfpathlineto{\pgfqpoint{5.506215in}{3.267145in}}%
\pgfpathlineto{\pgfqpoint{5.498802in}{3.259776in}}%
\pgfpathlineto{\pgfqpoint{5.491377in}{3.252201in}}%
\pgfpathlineto{\pgfqpoint{5.483941in}{3.244420in}}%
\pgfpathlineto{\pgfqpoint{5.476493in}{3.236432in}}%
\pgfpathclose%
\pgfusepath{fill}%
\end{pgfscope}%
\begin{pgfscope}%
\pgfpathrectangle{\pgfqpoint{1.254980in}{0.150000in}}{\pgfqpoint{5.490039in}{5.490039in}}%
\pgfusepath{clip}%
\pgfsetbuttcap%
\pgfsetroundjoin%
\definecolor{currentfill}{rgb}{0.279566,0.067836,0.391917}%
\pgfsetfillcolor{currentfill}%
\pgfsetfillopacity{0.700000}%
\pgfsetlinewidth{0.000000pt}%
\definecolor{currentstroke}{rgb}{0.000000,0.000000,0.000000}%
\pgfsetstrokecolor{currentstroke}%
\pgfsetdash{}{0pt}%
\pgfpathmoveto{\pgfqpoint{3.839760in}{1.162390in}}%
\pgfpathlineto{\pgfqpoint{3.853525in}{1.161951in}}%
\pgfpathlineto{\pgfqpoint{3.867297in}{1.161666in}}%
\pgfpathlineto{\pgfqpoint{3.881078in}{1.161533in}}%
\pgfpathlineto{\pgfqpoint{3.894867in}{1.161554in}}%
\pgfpathlineto{\pgfqpoint{3.902922in}{1.172387in}}%
\pgfpathlineto{\pgfqpoint{3.910971in}{1.183430in}}%
\pgfpathlineto{\pgfqpoint{3.919014in}{1.194679in}}%
\pgfpathlineto{\pgfqpoint{3.927051in}{1.206124in}}%
\pgfpathlineto{\pgfqpoint{3.913273in}{1.205400in}}%
\pgfpathlineto{\pgfqpoint{3.899503in}{1.204829in}}%
\pgfpathlineto{\pgfqpoint{3.885741in}{1.204412in}}%
\pgfpathlineto{\pgfqpoint{3.871988in}{1.204148in}}%
\pgfpathlineto{\pgfqpoint{3.863941in}{1.193395in}}%
\pgfpathlineto{\pgfqpoint{3.855887in}{1.182847in}}%
\pgfpathlineto{\pgfqpoint{3.847827in}{1.172509in}}%
\pgfpathlineto{\pgfqpoint{3.839760in}{1.162390in}}%
\pgfpathclose%
\pgfusepath{fill}%
\end{pgfscope}%
\begin{pgfscope}%
\pgfpathrectangle{\pgfqpoint{1.254980in}{0.150000in}}{\pgfqpoint{5.490039in}{5.490039in}}%
\pgfusepath{clip}%
\pgfsetbuttcap%
\pgfsetroundjoin%
\definecolor{currentfill}{rgb}{0.282327,0.094955,0.417331}%
\pgfsetfillcolor{currentfill}%
\pgfsetfillopacity{0.700000}%
\pgfsetlinewidth{0.000000pt}%
\definecolor{currentstroke}{rgb}{0.000000,0.000000,0.000000}%
\pgfsetstrokecolor{currentstroke}%
\pgfsetdash{}{0pt}%
\pgfpathmoveto{\pgfqpoint{3.927051in}{1.206124in}}%
\pgfpathlineto{\pgfqpoint{3.940839in}{1.207001in}}%
\pgfpathlineto{\pgfqpoint{3.954635in}{1.208031in}}%
\pgfpathlineto{\pgfqpoint{3.968441in}{1.209213in}}%
\pgfpathlineto{\pgfqpoint{3.982256in}{1.210548in}}%
\pgfpathlineto{\pgfqpoint{3.990279in}{1.222871in}}%
\pgfpathlineto{\pgfqpoint{3.998297in}{1.235372in}}%
\pgfpathlineto{\pgfqpoint{4.006310in}{1.248043in}}%
\pgfpathlineto{\pgfqpoint{4.014318in}{1.260877in}}%
\pgfpathlineto{\pgfqpoint{4.000510in}{1.258865in}}%
\pgfpathlineto{\pgfqpoint{3.986711in}{1.257005in}}%
\pgfpathlineto{\pgfqpoint{3.972922in}{1.255298in}}%
\pgfpathlineto{\pgfqpoint{3.959143in}{1.253745in}}%
\pgfpathlineto{\pgfqpoint{3.951128in}{1.241577in}}%
\pgfpathlineto{\pgfqpoint{3.943108in}{1.229580in}}%
\pgfpathlineto{\pgfqpoint{3.935082in}{1.217760in}}%
\pgfpathlineto{\pgfqpoint{3.927051in}{1.206124in}}%
\pgfpathclose%
\pgfusepath{fill}%
\end{pgfscope}%
\begin{pgfscope}%
\pgfpathrectangle{\pgfqpoint{1.254980in}{0.150000in}}{\pgfqpoint{5.490039in}{5.490039in}}%
\pgfusepath{clip}%
\pgfsetbuttcap%
\pgfsetroundjoin%
\definecolor{currentfill}{rgb}{0.119423,0.611141,0.538982}%
\pgfsetfillcolor{currentfill}%
\pgfsetfillopacity{0.700000}%
\pgfsetlinewidth{0.000000pt}%
\definecolor{currentstroke}{rgb}{0.000000,0.000000,0.000000}%
\pgfsetstrokecolor{currentstroke}%
\pgfsetdash{}{0pt}%
\pgfpathmoveto{\pgfqpoint{4.880435in}{2.440856in}}%
\pgfpathlineto{\pgfqpoint{4.894736in}{2.454374in}}%
\pgfpathlineto{\pgfqpoint{4.909057in}{2.468055in}}%
\pgfpathlineto{\pgfqpoint{4.923396in}{2.481901in}}%
\pgfpathlineto{\pgfqpoint{4.937755in}{2.495911in}}%
\pgfpathlineto{\pgfqpoint{4.945545in}{2.511266in}}%
\pgfpathlineto{\pgfqpoint{4.953328in}{2.526463in}}%
\pgfpathlineto{\pgfqpoint{4.961105in}{2.541500in}}%
\pgfpathlineto{\pgfqpoint{4.968876in}{2.556375in}}%
\pgfpathlineto{\pgfqpoint{4.954509in}{2.542129in}}%
\pgfpathlineto{\pgfqpoint{4.940161in}{2.528047in}}%
\pgfpathlineto{\pgfqpoint{4.925833in}{2.514130in}}%
\pgfpathlineto{\pgfqpoint{4.911525in}{2.500377in}}%
\pgfpathlineto{\pgfqpoint{4.903762in}{2.485726in}}%
\pgfpathlineto{\pgfqpoint{4.895992in}{2.470921in}}%
\pgfpathlineto{\pgfqpoint{4.888217in}{2.455963in}}%
\pgfpathlineto{\pgfqpoint{4.880435in}{2.440856in}}%
\pgfpathclose%
\pgfusepath{fill}%
\end{pgfscope}%
\begin{pgfscope}%
\pgfpathrectangle{\pgfqpoint{1.254980in}{0.150000in}}{\pgfqpoint{5.490039in}{5.490039in}}%
\pgfusepath{clip}%
\pgfsetbuttcap%
\pgfsetroundjoin%
\definecolor{currentfill}{rgb}{0.168126,0.459988,0.558082}%
\pgfsetfillcolor{currentfill}%
\pgfsetfillopacity{0.700000}%
\pgfsetlinewidth{0.000000pt}%
\definecolor{currentstroke}{rgb}{0.000000,0.000000,0.000000}%
\pgfsetstrokecolor{currentstroke}%
\pgfsetdash{}{0pt}%
\pgfpathmoveto{\pgfqpoint{4.610241in}{2.021601in}}%
\pgfpathlineto{\pgfqpoint{4.624359in}{2.032203in}}%
\pgfpathlineto{\pgfqpoint{4.638493in}{2.042965in}}%
\pgfpathlineto{\pgfqpoint{4.652643in}{2.053886in}}%
\pgfpathlineto{\pgfqpoint{4.666811in}{2.064968in}}%
\pgfpathlineto{\pgfqpoint{4.674686in}{2.082149in}}%
\pgfpathlineto{\pgfqpoint{4.682557in}{2.099243in}}%
\pgfpathlineto{\pgfqpoint{4.690423in}{2.116246in}}%
\pgfpathlineto{\pgfqpoint{4.698286in}{2.133155in}}%
\pgfpathlineto{\pgfqpoint{4.684109in}{2.121689in}}%
\pgfpathlineto{\pgfqpoint{4.669949in}{2.110384in}}%
\pgfpathlineto{\pgfqpoint{4.655807in}{2.099240in}}%
\pgfpathlineto{\pgfqpoint{4.641681in}{2.088255in}}%
\pgfpathlineto{\pgfqpoint{4.633828in}{2.071718in}}%
\pgfpathlineto{\pgfqpoint{4.625970in}{2.055095in}}%
\pgfpathlineto{\pgfqpoint{4.618108in}{2.038388in}}%
\pgfpathlineto{\pgfqpoint{4.610241in}{2.021601in}}%
\pgfpathclose%
\pgfusepath{fill}%
\end{pgfscope}%
\begin{pgfscope}%
\pgfpathrectangle{\pgfqpoint{1.254980in}{0.150000in}}{\pgfqpoint{5.490039in}{5.490039in}}%
\pgfusepath{clip}%
\pgfsetbuttcap%
\pgfsetroundjoin%
\definecolor{currentfill}{rgb}{0.283187,0.125848,0.444960}%
\pgfsetfillcolor{currentfill}%
\pgfsetfillopacity{0.700000}%
\pgfsetlinewidth{0.000000pt}%
\definecolor{currentstroke}{rgb}{0.000000,0.000000,0.000000}%
\pgfsetstrokecolor{currentstroke}%
\pgfsetdash{}{0pt}%
\pgfpathmoveto{\pgfqpoint{4.014318in}{1.260877in}}%
\pgfpathlineto{\pgfqpoint{4.028136in}{1.263043in}}%
\pgfpathlineto{\pgfqpoint{4.041964in}{1.265361in}}%
\pgfpathlineto{\pgfqpoint{4.055803in}{1.267833in}}%
\pgfpathlineto{\pgfqpoint{4.069651in}{1.270456in}}%
\pgfpathlineto{\pgfqpoint{4.077650in}{1.284110in}}%
\pgfpathlineto{\pgfqpoint{4.085644in}{1.297909in}}%
\pgfpathlineto{\pgfqpoint{4.093633in}{1.311845in}}%
\pgfpathlineto{\pgfqpoint{4.101619in}{1.325912in}}%
\pgfpathlineto{\pgfqpoint{4.087773in}{1.322636in}}%
\pgfpathlineto{\pgfqpoint{4.073939in}{1.319513in}}%
\pgfpathlineto{\pgfqpoint{4.060115in}{1.316543in}}%
\pgfpathlineto{\pgfqpoint{4.046301in}{1.313727in}}%
\pgfpathlineto{\pgfqpoint{4.038312in}{1.300300in}}%
\pgfpathlineto{\pgfqpoint{4.030319in}{1.287012in}}%
\pgfpathlineto{\pgfqpoint{4.022321in}{1.273869in}}%
\pgfpathlineto{\pgfqpoint{4.014318in}{1.260877in}}%
\pgfpathclose%
\pgfusepath{fill}%
\end{pgfscope}%
\begin{pgfscope}%
\pgfpathrectangle{\pgfqpoint{1.254980in}{0.150000in}}{\pgfqpoint{5.490039in}{5.490039in}}%
\pgfusepath{clip}%
\pgfsetbuttcap%
\pgfsetroundjoin%
\definecolor{currentfill}{rgb}{0.266580,0.228262,0.514349}%
\pgfsetfillcolor{currentfill}%
\pgfsetfillopacity{0.700000}%
\pgfsetlinewidth{0.000000pt}%
\definecolor{currentstroke}{rgb}{0.000000,0.000000,0.000000}%
\pgfsetstrokecolor{currentstroke}%
\pgfsetdash{}{0pt}%
\pgfpathmoveto{\pgfqpoint{4.220841in}{1.462010in}}%
\pgfpathlineto{\pgfqpoint{4.234743in}{1.467281in}}%
\pgfpathlineto{\pgfqpoint{4.248658in}{1.472705in}}%
\pgfpathlineto{\pgfqpoint{4.262585in}{1.478284in}}%
\pgfpathlineto{\pgfqpoint{4.276525in}{1.484016in}}%
\pgfpathlineto{\pgfqpoint{4.284478in}{1.500168in}}%
\pgfpathlineto{\pgfqpoint{4.292426in}{1.516379in}}%
\pgfpathlineto{\pgfqpoint{4.300372in}{1.532644in}}%
\pgfpathlineto{\pgfqpoint{4.308314in}{1.548959in}}%
\pgfpathlineto{\pgfqpoint{4.294370in}{1.542651in}}%
\pgfpathlineto{\pgfqpoint{4.280440in}{1.536498in}}%
\pgfpathlineto{\pgfqpoint{4.266522in}{1.530500in}}%
\pgfpathlineto{\pgfqpoint{4.252618in}{1.524656in}}%
\pgfpathlineto{\pgfqpoint{4.244679in}{1.508906in}}%
\pgfpathlineto{\pgfqpoint{4.236736in}{1.493211in}}%
\pgfpathlineto{\pgfqpoint{4.228790in}{1.477577in}}%
\pgfpathlineto{\pgfqpoint{4.220841in}{1.462010in}}%
\pgfpathclose%
\pgfusepath{fill}%
\end{pgfscope}%
\begin{pgfscope}%
\pgfpathrectangle{\pgfqpoint{1.254980in}{0.150000in}}{\pgfqpoint{5.490039in}{5.490039in}}%
\pgfusepath{clip}%
\pgfsetbuttcap%
\pgfsetroundjoin%
\definecolor{currentfill}{rgb}{0.239346,0.300855,0.540844}%
\pgfsetfillcolor{currentfill}%
\pgfsetfillopacity{0.700000}%
\pgfsetlinewidth{0.000000pt}%
\definecolor{currentstroke}{rgb}{0.000000,0.000000,0.000000}%
\pgfsetstrokecolor{currentstroke}%
\pgfsetdash{}{0pt}%
\pgfpathmoveto{\pgfqpoint{4.340048in}{1.614602in}}%
\pgfpathlineto{\pgfqpoint{4.354010in}{1.621612in}}%
\pgfpathlineto{\pgfqpoint{4.367985in}{1.628779in}}%
\pgfpathlineto{\pgfqpoint{4.381974in}{1.636101in}}%
\pgfpathlineto{\pgfqpoint{4.395976in}{1.643579in}}%
\pgfpathlineto{\pgfqpoint{4.403907in}{1.660592in}}%
\pgfpathlineto{\pgfqpoint{4.411835in}{1.677617in}}%
\pgfpathlineto{\pgfqpoint{4.419759in}{1.694648in}}%
\pgfpathlineto{\pgfqpoint{4.427680in}{1.711680in}}%
\pgfpathlineto{\pgfqpoint{4.413671in}{1.703680in}}%
\pgfpathlineto{\pgfqpoint{4.399676in}{1.695836in}}%
\pgfpathlineto{\pgfqpoint{4.385695in}{1.688149in}}%
\pgfpathlineto{\pgfqpoint{4.371729in}{1.680618in}}%
\pgfpathlineto{\pgfqpoint{4.363814in}{1.664096in}}%
\pgfpathlineto{\pgfqpoint{4.355895in}{1.647583in}}%
\pgfpathlineto{\pgfqpoint{4.347974in}{1.631083in}}%
\pgfpathlineto{\pgfqpoint{4.340048in}{1.614602in}}%
\pgfpathclose%
\pgfusepath{fill}%
\end{pgfscope}%
\begin{pgfscope}%
\pgfpathrectangle{\pgfqpoint{1.254980in}{0.150000in}}{\pgfqpoint{5.490039in}{5.490039in}}%
\pgfusepath{clip}%
\pgfsetbuttcap%
\pgfsetroundjoin%
\definecolor{currentfill}{rgb}{0.146616,0.673050,0.508936}%
\pgfsetfillcolor{currentfill}%
\pgfsetfillopacity{0.700000}%
\pgfsetlinewidth{0.000000pt}%
\definecolor{currentstroke}{rgb}{0.000000,0.000000,0.000000}%
\pgfsetstrokecolor{currentstroke}%
\pgfsetdash{}{0pt}%
\pgfpathmoveto{\pgfqpoint{4.999889in}{2.614222in}}%
\pgfpathlineto{\pgfqpoint{5.014283in}{2.628838in}}%
\pgfpathlineto{\pgfqpoint{5.028696in}{2.643620in}}%
\pgfpathlineto{\pgfqpoint{5.043130in}{2.658568in}}%
\pgfpathlineto{\pgfqpoint{5.057584in}{2.673681in}}%
\pgfpathlineto{\pgfqpoint{5.065327in}{2.687907in}}%
\pgfpathlineto{\pgfqpoint{5.073062in}{2.701954in}}%
\pgfpathlineto{\pgfqpoint{5.080789in}{2.715820in}}%
\pgfpathlineto{\pgfqpoint{5.088509in}{2.729504in}}%
\pgfpathlineto{\pgfqpoint{5.074049in}{2.714216in}}%
\pgfpathlineto{\pgfqpoint{5.059609in}{2.699094in}}%
\pgfpathlineto{\pgfqpoint{5.045189in}{2.684139in}}%
\pgfpathlineto{\pgfqpoint{5.030790in}{2.669349in}}%
\pgfpathlineto{\pgfqpoint{5.023075in}{2.655827in}}%
\pgfpathlineto{\pgfqpoint{5.015354in}{2.642131in}}%
\pgfpathlineto{\pgfqpoint{5.007625in}{2.628263in}}%
\pgfpathlineto{\pgfqpoint{4.999889in}{2.614222in}}%
\pgfpathclose%
\pgfusepath{fill}%
\end{pgfscope}%
\begin{pgfscope}%
\pgfpathrectangle{\pgfqpoint{1.254980in}{0.150000in}}{\pgfqpoint{5.490039in}{5.490039in}}%
\pgfusepath{clip}%
\pgfsetbuttcap%
\pgfsetroundjoin%
\definecolor{currentfill}{rgb}{0.477504,0.821444,0.318195}%
\pgfsetfillcolor{currentfill}%
\pgfsetfillopacity{0.700000}%
\pgfsetlinewidth{0.000000pt}%
\definecolor{currentstroke}{rgb}{0.000000,0.000000,0.000000}%
\pgfsetstrokecolor{currentstroke}%
\pgfsetdash{}{0pt}%
\pgfpathmoveto{\pgfqpoint{5.357710in}{3.095133in}}%
\pgfpathlineto{\pgfqpoint{5.372378in}{3.112327in}}%
\pgfpathlineto{\pgfqpoint{5.387069in}{3.129690in}}%
\pgfpathlineto{\pgfqpoint{5.401782in}{3.147224in}}%
\pgfpathlineto{\pgfqpoint{5.416519in}{3.164927in}}%
\pgfpathlineto{\pgfqpoint{5.424054in}{3.174611in}}%
\pgfpathlineto{\pgfqpoint{5.431578in}{3.184081in}}%
\pgfpathlineto{\pgfqpoint{5.439091in}{3.193337in}}%
\pgfpathlineto{\pgfqpoint{5.446593in}{3.202379in}}%
\pgfpathlineto{\pgfqpoint{5.431859in}{3.184695in}}%
\pgfpathlineto{\pgfqpoint{5.417147in}{3.167180in}}%
\pgfpathlineto{\pgfqpoint{5.402458in}{3.149836in}}%
\pgfpathlineto{\pgfqpoint{5.387792in}{3.132660in}}%
\pgfpathlineto{\pgfqpoint{5.380287in}{3.123587in}}%
\pgfpathlineto{\pgfqpoint{5.372772in}{3.114308in}}%
\pgfpathlineto{\pgfqpoint{5.365246in}{3.104824in}}%
\pgfpathlineto{\pgfqpoint{5.357710in}{3.095133in}}%
\pgfpathclose%
\pgfusepath{fill}%
\end{pgfscope}%
\begin{pgfscope}%
\pgfpathrectangle{\pgfqpoint{1.254980in}{0.150000in}}{\pgfqpoint{5.490039in}{5.490039in}}%
\pgfusepath{clip}%
\pgfsetbuttcap%
\pgfsetroundjoin%
\definecolor{currentfill}{rgb}{0.232815,0.732247,0.459277}%
\pgfsetfillcolor{currentfill}%
\pgfsetfillopacity{0.700000}%
\pgfsetlinewidth{0.000000pt}%
\definecolor{currentstroke}{rgb}{0.000000,0.000000,0.000000}%
\pgfsetstrokecolor{currentstroke}%
\pgfsetdash{}{0pt}%
\pgfpathmoveto{\pgfqpoint{5.119310in}{2.782394in}}%
\pgfpathlineto{\pgfqpoint{5.133796in}{2.797992in}}%
\pgfpathlineto{\pgfqpoint{5.148304in}{2.813756in}}%
\pgfpathlineto{\pgfqpoint{5.162832in}{2.829688in}}%
\pgfpathlineto{\pgfqpoint{5.177382in}{2.845787in}}%
\pgfpathlineto{\pgfqpoint{5.185067in}{2.858667in}}%
\pgfpathlineto{\pgfqpoint{5.192743in}{2.871350in}}%
\pgfpathlineto{\pgfqpoint{5.200410in}{2.883836in}}%
\pgfpathlineto{\pgfqpoint{5.208069in}{2.896125in}}%
\pgfpathlineto{\pgfqpoint{5.193515in}{2.879914in}}%
\pgfpathlineto{\pgfqpoint{5.178983in}{2.863872in}}%
\pgfpathlineto{\pgfqpoint{5.164472in}{2.847996in}}%
\pgfpathlineto{\pgfqpoint{5.149981in}{2.832288in}}%
\pgfpathlineto{\pgfqpoint{5.142326in}{2.820098in}}%
\pgfpathlineto{\pgfqpoint{5.134662in}{2.807719in}}%
\pgfpathlineto{\pgfqpoint{5.126990in}{2.795151in}}%
\pgfpathlineto{\pgfqpoint{5.119310in}{2.782394in}}%
\pgfpathclose%
\pgfusepath{fill}%
\end{pgfscope}%
\begin{pgfscope}%
\pgfpathrectangle{\pgfqpoint{1.254980in}{0.150000in}}{\pgfqpoint{5.490039in}{5.490039in}}%
\pgfusepath{clip}%
\pgfsetbuttcap%
\pgfsetroundjoin%
\definecolor{currentfill}{rgb}{0.141935,0.526453,0.555991}%
\pgfsetfillcolor{currentfill}%
\pgfsetfillopacity{0.700000}%
\pgfsetlinewidth{0.000000pt}%
\definecolor{currentstroke}{rgb}{0.000000,0.000000,0.000000}%
\pgfsetstrokecolor{currentstroke}%
\pgfsetdash{}{0pt}%
\pgfpathmoveto{\pgfqpoint{4.729689in}{2.199784in}}%
\pgfpathlineto{\pgfqpoint{4.743892in}{2.211765in}}%
\pgfpathlineto{\pgfqpoint{4.758113in}{2.223908in}}%
\pgfpathlineto{\pgfqpoint{4.772352in}{2.236213in}}%
\pgfpathlineto{\pgfqpoint{4.786608in}{2.248679in}}%
\pgfpathlineto{\pgfqpoint{4.794456in}{2.265406in}}%
\pgfpathlineto{\pgfqpoint{4.802300in}{2.282013in}}%
\pgfpathlineto{\pgfqpoint{4.810138in}{2.298496in}}%
\pgfpathlineto{\pgfqpoint{4.817971in}{2.314853in}}%
\pgfpathlineto{\pgfqpoint{4.803705in}{2.302060in}}%
\pgfpathlineto{\pgfqpoint{4.789457in}{2.289430in}}%
\pgfpathlineto{\pgfqpoint{4.775227in}{2.276962in}}%
\pgfpathlineto{\pgfqpoint{4.761015in}{2.264656in}}%
\pgfpathlineto{\pgfqpoint{4.753191in}{2.248613in}}%
\pgfpathlineto{\pgfqpoint{4.745362in}{2.232452in}}%
\pgfpathlineto{\pgfqpoint{4.737528in}{2.216174in}}%
\pgfpathlineto{\pgfqpoint{4.729689in}{2.199784in}}%
\pgfpathclose%
\pgfusepath{fill}%
\end{pgfscope}%
\begin{pgfscope}%
\pgfpathrectangle{\pgfqpoint{1.254980in}{0.150000in}}{\pgfqpoint{5.490039in}{5.490039in}}%
\pgfusepath{clip}%
\pgfsetbuttcap%
\pgfsetroundjoin%
\definecolor{currentfill}{rgb}{0.206756,0.371758,0.553117}%
\pgfsetfillcolor{currentfill}%
\pgfsetfillopacity{0.700000}%
\pgfsetlinewidth{0.000000pt}%
\definecolor{currentstroke}{rgb}{0.000000,0.000000,0.000000}%
\pgfsetstrokecolor{currentstroke}%
\pgfsetdash{}{0pt}%
\pgfpathmoveto{\pgfqpoint{4.459330in}{1.779734in}}%
\pgfpathlineto{\pgfqpoint{4.473360in}{1.788386in}}%
\pgfpathlineto{\pgfqpoint{4.487406in}{1.797194in}}%
\pgfpathlineto{\pgfqpoint{4.501467in}{1.806161in}}%
\pgfpathlineto{\pgfqpoint{4.515543in}{1.815284in}}%
\pgfpathlineto{\pgfqpoint{4.523454in}{1.832734in}}%
\pgfpathlineto{\pgfqpoint{4.531363in}{1.850151in}}%
\pgfpathlineto{\pgfqpoint{4.539267in}{1.867530in}}%
\pgfpathlineto{\pgfqpoint{4.547168in}{1.884869in}}%
\pgfpathlineto{\pgfqpoint{4.533084in}{1.875276in}}%
\pgfpathlineto{\pgfqpoint{4.519015in}{1.865842in}}%
\pgfpathlineto{\pgfqpoint{4.504962in}{1.856567in}}%
\pgfpathlineto{\pgfqpoint{4.490924in}{1.847449in}}%
\pgfpathlineto{\pgfqpoint{4.483031in}{1.830567in}}%
\pgfpathlineto{\pgfqpoint{4.475134in}{1.813651in}}%
\pgfpathlineto{\pgfqpoint{4.467234in}{1.796705in}}%
\pgfpathlineto{\pgfqpoint{4.459330in}{1.779734in}}%
\pgfpathclose%
\pgfusepath{fill}%
\end{pgfscope}%
\begin{pgfscope}%
\pgfpathrectangle{\pgfqpoint{1.254980in}{0.150000in}}{\pgfqpoint{5.490039in}{5.490039in}}%
\pgfusepath{clip}%
\pgfsetbuttcap%
\pgfsetroundjoin%
\definecolor{currentfill}{rgb}{0.352360,0.783011,0.392636}%
\pgfsetfillcolor{currentfill}%
\pgfsetfillopacity{0.700000}%
\pgfsetlinewidth{0.000000pt}%
\definecolor{currentstroke}{rgb}{0.000000,0.000000,0.000000}%
\pgfsetstrokecolor{currentstroke}%
\pgfsetdash{}{0pt}%
\pgfpathmoveto{\pgfqpoint{5.238615in}{2.943293in}}%
\pgfpathlineto{\pgfqpoint{5.253193in}{2.959751in}}%
\pgfpathlineto{\pgfqpoint{5.267793in}{2.976377in}}%
\pgfpathlineto{\pgfqpoint{5.282416in}{2.993172in}}%
\pgfpathlineto{\pgfqpoint{5.297060in}{3.010136in}}%
\pgfpathlineto{\pgfqpoint{5.304676in}{3.021488in}}%
\pgfpathlineto{\pgfqpoint{5.312282in}{3.032633in}}%
\pgfpathlineto{\pgfqpoint{5.319878in}{3.043569in}}%
\pgfpathlineto{\pgfqpoint{5.327464in}{3.054297in}}%
\pgfpathlineto{\pgfqpoint{5.312818in}{3.037287in}}%
\pgfpathlineto{\pgfqpoint{5.298195in}{3.020445in}}%
\pgfpathlineto{\pgfqpoint{5.283593in}{3.003772in}}%
\pgfpathlineto{\pgfqpoint{5.269014in}{2.987268in}}%
\pgfpathlineto{\pgfqpoint{5.261428in}{2.976574in}}%
\pgfpathlineto{\pgfqpoint{5.253833in}{2.965680in}}%
\pgfpathlineto{\pgfqpoint{5.246229in}{2.954587in}}%
\pgfpathlineto{\pgfqpoint{5.238615in}{2.943293in}}%
\pgfpathclose%
\pgfusepath{fill}%
\end{pgfscope}%
\begin{pgfscope}%
\pgfpathrectangle{\pgfqpoint{1.254980in}{0.150000in}}{\pgfqpoint{5.490039in}{5.490039in}}%
\pgfusepath{clip}%
\pgfsetbuttcap%
\pgfsetroundjoin%
\definecolor{currentfill}{rgb}{0.280255,0.165693,0.476498}%
\pgfsetfillcolor{currentfill}%
\pgfsetfillopacity{0.700000}%
\pgfsetlinewidth{0.000000pt}%
\definecolor{currentstroke}{rgb}{0.000000,0.000000,0.000000}%
\pgfsetstrokecolor{currentstroke}%
\pgfsetdash{}{0pt}%
\pgfpathmoveto{\pgfqpoint{4.101619in}{1.325912in}}%
\pgfpathlineto{\pgfqpoint{4.115475in}{1.329342in}}%
\pgfpathlineto{\pgfqpoint{4.129342in}{1.332924in}}%
\pgfpathlineto{\pgfqpoint{4.143220in}{1.336660in}}%
\pgfpathlineto{\pgfqpoint{4.157110in}{1.340548in}}%
\pgfpathlineto{\pgfqpoint{4.165090in}{1.355378in}}%
\pgfpathlineto{\pgfqpoint{4.173065in}{1.370321in}}%
\pgfpathlineto{\pgfqpoint{4.181037in}{1.385370in}}%
\pgfpathlineto{\pgfqpoint{4.189005in}{1.400520in}}%
\pgfpathlineto{\pgfqpoint{4.175116in}{1.396004in}}%
\pgfpathlineto{\pgfqpoint{4.161238in}{1.391642in}}%
\pgfpathlineto{\pgfqpoint{4.147372in}{1.387433in}}%
\pgfpathlineto{\pgfqpoint{4.133518in}{1.383378in}}%
\pgfpathlineto{\pgfqpoint{4.125549in}{1.368844in}}%
\pgfpathlineto{\pgfqpoint{4.117576in}{1.354418in}}%
\pgfpathlineto{\pgfqpoint{4.109600in}{1.340106in}}%
\pgfpathlineto{\pgfqpoint{4.101619in}{1.325912in}}%
\pgfpathclose%
\pgfusepath{fill}%
\end{pgfscope}%
\begin{pgfscope}%
\pgfpathrectangle{\pgfqpoint{1.254980in}{0.150000in}}{\pgfqpoint{5.490039in}{5.490039in}}%
\pgfusepath{clip}%
\pgfsetbuttcap%
\pgfsetroundjoin%
\definecolor{currentfill}{rgb}{0.175841,0.441290,0.557685}%
\pgfsetfillcolor{currentfill}%
\pgfsetfillopacity{0.700000}%
\pgfsetlinewidth{0.000000pt}%
\definecolor{currentstroke}{rgb}{0.000000,0.000000,0.000000}%
\pgfsetstrokecolor{currentstroke}%
\pgfsetdash{}{0pt}%
\pgfpathmoveto{\pgfqpoint{4.578736in}{1.953726in}}%
\pgfpathlineto{\pgfqpoint{4.592844in}{1.963917in}}%
\pgfpathlineto{\pgfqpoint{4.606969in}{1.974267in}}%
\pgfpathlineto{\pgfqpoint{4.621111in}{1.984776in}}%
\pgfpathlineto{\pgfqpoint{4.635269in}{1.995444in}}%
\pgfpathlineto{\pgfqpoint{4.643160in}{2.012938in}}%
\pgfpathlineto{\pgfqpoint{4.651048in}{2.030359in}}%
\pgfpathlineto{\pgfqpoint{4.658931in}{2.047704in}}%
\pgfpathlineto{\pgfqpoint{4.666811in}{2.064968in}}%
\pgfpathlineto{\pgfqpoint{4.652643in}{2.053886in}}%
\pgfpathlineto{\pgfqpoint{4.638493in}{2.042965in}}%
\pgfpathlineto{\pgfqpoint{4.624359in}{2.032203in}}%
\pgfpathlineto{\pgfqpoint{4.610241in}{2.021601in}}%
\pgfpathlineto{\pgfqpoint{4.602371in}{2.004737in}}%
\pgfpathlineto{\pgfqpoint{4.594496in}{1.987801in}}%
\pgfpathlineto{\pgfqpoint{4.586618in}{1.970796in}}%
\pgfpathlineto{\pgfqpoint{4.578736in}{1.953726in}}%
\pgfpathclose%
\pgfusepath{fill}%
\end{pgfscope}%
\begin{pgfscope}%
\pgfpathrectangle{\pgfqpoint{1.254980in}{0.150000in}}{\pgfqpoint{5.490039in}{5.490039in}}%
\pgfusepath{clip}%
\pgfsetbuttcap%
\pgfsetroundjoin%
\definecolor{currentfill}{rgb}{0.120565,0.596422,0.543611}%
\pgfsetfillcolor{currentfill}%
\pgfsetfillopacity{0.700000}%
\pgfsetlinewidth{0.000000pt}%
\definecolor{currentstroke}{rgb}{0.000000,0.000000,0.000000}%
\pgfsetstrokecolor{currentstroke}%
\pgfsetdash{}{0pt}%
\pgfpathmoveto{\pgfqpoint{4.849248in}{2.378966in}}%
\pgfpathlineto{\pgfqpoint{4.863541in}{2.392218in}}%
\pgfpathlineto{\pgfqpoint{4.877854in}{2.405634in}}%
\pgfpathlineto{\pgfqpoint{4.892185in}{2.419213in}}%
\pgfpathlineto{\pgfqpoint{4.906535in}{2.432956in}}%
\pgfpathlineto{\pgfqpoint{4.914349in}{2.448921in}}%
\pgfpathlineto{\pgfqpoint{4.922157in}{2.464736in}}%
\pgfpathlineto{\pgfqpoint{4.929959in}{2.480400in}}%
\pgfpathlineto{\pgfqpoint{4.937755in}{2.495911in}}%
\pgfpathlineto{\pgfqpoint{4.923396in}{2.481901in}}%
\pgfpathlineto{\pgfqpoint{4.909057in}{2.468055in}}%
\pgfpathlineto{\pgfqpoint{4.894736in}{2.454374in}}%
\pgfpathlineto{\pgfqpoint{4.880435in}{2.440856in}}%
\pgfpathlineto{\pgfqpoint{4.872647in}{2.425600in}}%
\pgfpathlineto{\pgfqpoint{4.864853in}{2.410198in}}%
\pgfpathlineto{\pgfqpoint{4.857053in}{2.394653in}}%
\pgfpathlineto{\pgfqpoint{4.849248in}{2.378966in}}%
\pgfpathclose%
\pgfusepath{fill}%
\end{pgfscope}%
\begin{pgfscope}%
\pgfpathrectangle{\pgfqpoint{1.254980in}{0.150000in}}{\pgfqpoint{5.490039in}{5.490039in}}%
\pgfusepath{clip}%
\pgfsetbuttcap%
\pgfsetroundjoin%
\definecolor{currentfill}{rgb}{0.281446,0.084320,0.407414}%
\pgfsetfillcolor{currentfill}%
\pgfsetfillopacity{0.700000}%
\pgfsetlinewidth{0.000000pt}%
\definecolor{currentstroke}{rgb}{0.000000,0.000000,0.000000}%
\pgfsetstrokecolor{currentstroke}%
\pgfsetdash{}{0pt}%
\pgfpathmoveto{\pgfqpoint{3.894867in}{1.161554in}}%
\pgfpathlineto{\pgfqpoint{3.908664in}{1.161726in}}%
\pgfpathlineto{\pgfqpoint{3.922470in}{1.162051in}}%
\pgfpathlineto{\pgfqpoint{3.936284in}{1.162529in}}%
\pgfpathlineto{\pgfqpoint{3.950108in}{1.163158in}}%
\pgfpathlineto{\pgfqpoint{3.958153in}{1.174706in}}%
\pgfpathlineto{\pgfqpoint{3.966193in}{1.186459in}}%
\pgfpathlineto{\pgfqpoint{3.974227in}{1.198408in}}%
\pgfpathlineto{\pgfqpoint{3.982256in}{1.210548in}}%
\pgfpathlineto{\pgfqpoint{3.968441in}{1.209213in}}%
\pgfpathlineto{\pgfqpoint{3.954635in}{1.208031in}}%
\pgfpathlineto{\pgfqpoint{3.940839in}{1.207001in}}%
\pgfpathlineto{\pgfqpoint{3.927051in}{1.206124in}}%
\pgfpathlineto{\pgfqpoint{3.919014in}{1.194679in}}%
\pgfpathlineto{\pgfqpoint{3.910971in}{1.183430in}}%
\pgfpathlineto{\pgfqpoint{3.902922in}{1.172387in}}%
\pgfpathlineto{\pgfqpoint{3.894867in}{1.161554in}}%
\pgfpathclose%
\pgfusepath{fill}%
\end{pgfscope}%
\begin{pgfscope}%
\pgfpathrectangle{\pgfqpoint{1.254980in}{0.150000in}}{\pgfqpoint{5.490039in}{5.490039in}}%
\pgfusepath{clip}%
\pgfsetbuttcap%
\pgfsetroundjoin%
\definecolor{currentfill}{rgb}{0.248629,0.278775,0.534556}%
\pgfsetfillcolor{currentfill}%
\pgfsetfillopacity{0.700000}%
\pgfsetlinewidth{0.000000pt}%
\definecolor{currentstroke}{rgb}{0.000000,0.000000,0.000000}%
\pgfsetstrokecolor{currentstroke}%
\pgfsetdash{}{0pt}%
\pgfpathmoveto{\pgfqpoint{4.308314in}{1.548959in}}%
\pgfpathlineto{\pgfqpoint{4.322271in}{1.555422in}}%
\pgfpathlineto{\pgfqpoint{4.336241in}{1.562039in}}%
\pgfpathlineto{\pgfqpoint{4.350224in}{1.568812in}}%
\pgfpathlineto{\pgfqpoint{4.364221in}{1.575740in}}%
\pgfpathlineto{\pgfqpoint{4.372165in}{1.592657in}}%
\pgfpathlineto{\pgfqpoint{4.380105in}{1.609607in}}%
\pgfpathlineto{\pgfqpoint{4.388042in}{1.626582in}}%
\pgfpathlineto{\pgfqpoint{4.395976in}{1.643579in}}%
\pgfpathlineto{\pgfqpoint{4.381974in}{1.636101in}}%
\pgfpathlineto{\pgfqpoint{4.367985in}{1.628779in}}%
\pgfpathlineto{\pgfqpoint{4.354010in}{1.621612in}}%
\pgfpathlineto{\pgfqpoint{4.340048in}{1.614602in}}%
\pgfpathlineto{\pgfqpoint{4.332120in}{1.598143in}}%
\pgfpathlineto{\pgfqpoint{4.324188in}{1.581713in}}%
\pgfpathlineto{\pgfqpoint{4.316253in}{1.565317in}}%
\pgfpathlineto{\pgfqpoint{4.308314in}{1.548959in}}%
\pgfpathclose%
\pgfusepath{fill}%
\end{pgfscope}%
\begin{pgfscope}%
\pgfpathrectangle{\pgfqpoint{1.254980in}{0.150000in}}{\pgfqpoint{5.490039in}{5.490039in}}%
\pgfusepath{clip}%
\pgfsetbuttcap%
\pgfsetroundjoin%
\definecolor{currentfill}{rgb}{0.585678,0.846661,0.249897}%
\pgfsetfillcolor{currentfill}%
\pgfsetfillopacity{0.700000}%
\pgfsetlinewidth{0.000000pt}%
\definecolor{currentstroke}{rgb}{0.000000,0.000000,0.000000}%
\pgfsetstrokecolor{currentstroke}%
\pgfsetdash{}{0pt}%
\pgfpathmoveto{\pgfqpoint{5.446593in}{3.202379in}}%
\pgfpathlineto{\pgfqpoint{5.461351in}{3.220234in}}%
\pgfpathlineto{\pgfqpoint{5.476133in}{3.238260in}}%
\pgfpathlineto{\pgfqpoint{5.490937in}{3.256456in}}%
\pgfpathlineto{\pgfqpoint{5.498427in}{3.265256in}}%
\pgfpathlineto{\pgfqpoint{5.505905in}{3.273838in}}%
\pgfpathlineto{\pgfqpoint{5.513371in}{3.282203in}}%
\pgfpathlineto{\pgfqpoint{5.520827in}{3.290352in}}%
\pgfpathlineto{\pgfqpoint{5.506025in}{3.272208in}}%
\pgfpathlineto{\pgfqpoint{5.491248in}{3.254235in}}%
\pgfpathlineto{\pgfqpoint{5.476493in}{3.236432in}}%
\pgfpathlineto{\pgfqpoint{5.469035in}{3.228234in}}%
\pgfpathlineto{\pgfqpoint{5.461565in}{3.219827in}}%
\pgfpathlineto{\pgfqpoint{5.454085in}{3.211209in}}%
\pgfpathlineto{\pgfqpoint{5.446593in}{3.202379in}}%
\pgfpathclose%
\pgfusepath{fill}%
\end{pgfscope}%
\begin{pgfscope}%
\pgfpathrectangle{\pgfqpoint{1.254980in}{0.150000in}}{\pgfqpoint{5.490039in}{5.490039in}}%
\pgfusepath{clip}%
\pgfsetbuttcap%
\pgfsetroundjoin%
\definecolor{currentfill}{rgb}{0.271828,0.209303,0.504434}%
\pgfsetfillcolor{currentfill}%
\pgfsetfillopacity{0.700000}%
\pgfsetlinewidth{0.000000pt}%
\definecolor{currentstroke}{rgb}{0.000000,0.000000,0.000000}%
\pgfsetstrokecolor{currentstroke}%
\pgfsetdash{}{0pt}%
\pgfpathmoveto{\pgfqpoint{4.189005in}{1.400520in}}%
\pgfpathlineto{\pgfqpoint{4.202907in}{1.405190in}}%
\pgfpathlineto{\pgfqpoint{4.216820in}{1.410013in}}%
\pgfpathlineto{\pgfqpoint{4.230745in}{1.414989in}}%
\pgfpathlineto{\pgfqpoint{4.244682in}{1.420119in}}%
\pgfpathlineto{\pgfqpoint{4.252648in}{1.435976in}}%
\pgfpathlineto{\pgfqpoint{4.260611in}{1.451915in}}%
\pgfpathlineto{\pgfqpoint{4.268570in}{1.467930in}}%
\pgfpathlineto{\pgfqpoint{4.276525in}{1.484016in}}%
\pgfpathlineto{\pgfqpoint{4.262585in}{1.478284in}}%
\pgfpathlineto{\pgfqpoint{4.248658in}{1.472705in}}%
\pgfpathlineto{\pgfqpoint{4.234743in}{1.467281in}}%
\pgfpathlineto{\pgfqpoint{4.220841in}{1.462010in}}%
\pgfpathlineto{\pgfqpoint{4.212887in}{1.446516in}}%
\pgfpathlineto{\pgfqpoint{4.204930in}{1.431099in}}%
\pgfpathlineto{\pgfqpoint{4.196970in}{1.415765in}}%
\pgfpathlineto{\pgfqpoint{4.189005in}{1.400520in}}%
\pgfpathclose%
\pgfusepath{fill}%
\end{pgfscope}%
\begin{pgfscope}%
\pgfpathrectangle{\pgfqpoint{1.254980in}{0.150000in}}{\pgfqpoint{5.490039in}{5.490039in}}%
\pgfusepath{clip}%
\pgfsetbuttcap%
\pgfsetroundjoin%
\definecolor{currentfill}{rgb}{0.283091,0.110553,0.431554}%
\pgfsetfillcolor{currentfill}%
\pgfsetfillopacity{0.700000}%
\pgfsetlinewidth{0.000000pt}%
\definecolor{currentstroke}{rgb}{0.000000,0.000000,0.000000}%
\pgfsetstrokecolor{currentstroke}%
\pgfsetdash{}{0pt}%
\pgfpathmoveto{\pgfqpoint{3.982256in}{1.210548in}}%
\pgfpathlineto{\pgfqpoint{3.996080in}{1.212035in}}%
\pgfpathlineto{\pgfqpoint{4.009914in}{1.213675in}}%
\pgfpathlineto{\pgfqpoint{4.023758in}{1.215467in}}%
\pgfpathlineto{\pgfqpoint{4.037611in}{1.217411in}}%
\pgfpathlineto{\pgfqpoint{4.045628in}{1.230424in}}%
\pgfpathlineto{\pgfqpoint{4.053641in}{1.243607in}}%
\pgfpathlineto{\pgfqpoint{4.061648in}{1.256953in}}%
\pgfpathlineto{\pgfqpoint{4.069651in}{1.270456in}}%
\pgfpathlineto{\pgfqpoint{4.055803in}{1.267833in}}%
\pgfpathlineto{\pgfqpoint{4.041964in}{1.265361in}}%
\pgfpathlineto{\pgfqpoint{4.028136in}{1.263043in}}%
\pgfpathlineto{\pgfqpoint{4.014318in}{1.260877in}}%
\pgfpathlineto{\pgfqpoint{4.006310in}{1.248043in}}%
\pgfpathlineto{\pgfqpoint{3.998297in}{1.235372in}}%
\pgfpathlineto{\pgfqpoint{3.990279in}{1.222871in}}%
\pgfpathlineto{\pgfqpoint{3.982256in}{1.210548in}}%
\pgfpathclose%
\pgfusepath{fill}%
\end{pgfscope}%
\begin{pgfscope}%
\pgfpathrectangle{\pgfqpoint{1.254980in}{0.150000in}}{\pgfqpoint{5.490039in}{5.490039in}}%
\pgfusepath{clip}%
\pgfsetbuttcap%
\pgfsetroundjoin%
\definecolor{currentfill}{rgb}{0.137339,0.662252,0.515571}%
\pgfsetfillcolor{currentfill}%
\pgfsetfillopacity{0.700000}%
\pgfsetlinewidth{0.000000pt}%
\definecolor{currentstroke}{rgb}{0.000000,0.000000,0.000000}%
\pgfsetstrokecolor{currentstroke}%
\pgfsetdash{}{0pt}%
\pgfpathmoveto{\pgfqpoint{4.968876in}{2.556375in}}%
\pgfpathlineto{\pgfqpoint{4.983262in}{2.570786in}}%
\pgfpathlineto{\pgfqpoint{4.997669in}{2.585362in}}%
\pgfpathlineto{\pgfqpoint{5.012095in}{2.600104in}}%
\pgfpathlineto{\pgfqpoint{5.026541in}{2.615011in}}%
\pgfpathlineto{\pgfqpoint{5.034313in}{2.629940in}}%
\pgfpathlineto{\pgfqpoint{5.042077in}{2.644696in}}%
\pgfpathlineto{\pgfqpoint{5.049834in}{2.659277in}}%
\pgfpathlineto{\pgfqpoint{5.057584in}{2.673681in}}%
\pgfpathlineto{\pgfqpoint{5.043130in}{2.658568in}}%
\pgfpathlineto{\pgfqpoint{5.028696in}{2.643620in}}%
\pgfpathlineto{\pgfqpoint{5.014283in}{2.628838in}}%
\pgfpathlineto{\pgfqpoint{4.999889in}{2.614222in}}%
\pgfpathlineto{\pgfqpoint{4.992146in}{2.600012in}}%
\pgfpathlineto{\pgfqpoint{4.984396in}{2.585632in}}%
\pgfpathlineto{\pgfqpoint{4.976639in}{2.571086in}}%
\pgfpathlineto{\pgfqpoint{4.968876in}{2.556375in}}%
\pgfpathclose%
\pgfusepath{fill}%
\end{pgfscope}%
\begin{pgfscope}%
\pgfpathrectangle{\pgfqpoint{1.254980in}{0.150000in}}{\pgfqpoint{5.490039in}{5.490039in}}%
\pgfusepath{clip}%
\pgfsetbuttcap%
\pgfsetroundjoin%
\definecolor{currentfill}{rgb}{0.216210,0.351535,0.550627}%
\pgfsetfillcolor{currentfill}%
\pgfsetfillopacity{0.700000}%
\pgfsetlinewidth{0.000000pt}%
\definecolor{currentstroke}{rgb}{0.000000,0.000000,0.000000}%
\pgfsetstrokecolor{currentstroke}%
\pgfsetdash{}{0pt}%
\pgfpathmoveto{\pgfqpoint{4.427680in}{1.711680in}}%
\pgfpathlineto{\pgfqpoint{4.441703in}{1.719837in}}%
\pgfpathlineto{\pgfqpoint{4.455742in}{1.728150in}}%
\pgfpathlineto{\pgfqpoint{4.469795in}{1.736620in}}%
\pgfpathlineto{\pgfqpoint{4.483863in}{1.745247in}}%
\pgfpathlineto{\pgfqpoint{4.491788in}{1.762783in}}%
\pgfpathlineto{\pgfqpoint{4.499709in}{1.780304in}}%
\pgfpathlineto{\pgfqpoint{4.507628in}{1.797806in}}%
\pgfpathlineto{\pgfqpoint{4.515543in}{1.815284in}}%
\pgfpathlineto{\pgfqpoint{4.501467in}{1.806161in}}%
\pgfpathlineto{\pgfqpoint{4.487406in}{1.797194in}}%
\pgfpathlineto{\pgfqpoint{4.473360in}{1.788386in}}%
\pgfpathlineto{\pgfqpoint{4.459330in}{1.779734in}}%
\pgfpathlineto{\pgfqpoint{4.451422in}{1.762741in}}%
\pgfpathlineto{\pgfqpoint{4.443512in}{1.745732in}}%
\pgfpathlineto{\pgfqpoint{4.435597in}{1.728710in}}%
\pgfpathlineto{\pgfqpoint{4.427680in}{1.711680in}}%
\pgfpathclose%
\pgfusepath{fill}%
\end{pgfscope}%
\begin{pgfscope}%
\pgfpathrectangle{\pgfqpoint{1.254980in}{0.150000in}}{\pgfqpoint{5.490039in}{5.490039in}}%
\pgfusepath{clip}%
\pgfsetbuttcap%
\pgfsetroundjoin%
\definecolor{currentfill}{rgb}{0.149039,0.508051,0.557250}%
\pgfsetfillcolor{currentfill}%
\pgfsetfillopacity{0.700000}%
\pgfsetlinewidth{0.000000pt}%
\definecolor{currentstroke}{rgb}{0.000000,0.000000,0.000000}%
\pgfsetstrokecolor{currentstroke}%
\pgfsetdash{}{0pt}%
\pgfpathmoveto{\pgfqpoint{4.698286in}{2.133155in}}%
\pgfpathlineto{\pgfqpoint{4.712480in}{2.144781in}}%
\pgfpathlineto{\pgfqpoint{4.726691in}{2.156569in}}%
\pgfpathlineto{\pgfqpoint{4.740920in}{2.168517in}}%
\pgfpathlineto{\pgfqpoint{4.755167in}{2.180627in}}%
\pgfpathlineto{\pgfqpoint{4.763034in}{2.197805in}}%
\pgfpathlineto{\pgfqpoint{4.770897in}{2.214876in}}%
\pgfpathlineto{\pgfqpoint{4.778755in}{2.231835in}}%
\pgfpathlineto{\pgfqpoint{4.786608in}{2.248679in}}%
\pgfpathlineto{\pgfqpoint{4.772352in}{2.236213in}}%
\pgfpathlineto{\pgfqpoint{4.758113in}{2.223908in}}%
\pgfpathlineto{\pgfqpoint{4.743892in}{2.211765in}}%
\pgfpathlineto{\pgfqpoint{4.729689in}{2.199784in}}%
\pgfpathlineto{\pgfqpoint{4.721845in}{2.183284in}}%
\pgfpathlineto{\pgfqpoint{4.713997in}{2.166677in}}%
\pgfpathlineto{\pgfqpoint{4.706143in}{2.149966in}}%
\pgfpathlineto{\pgfqpoint{4.698286in}{2.133155in}}%
\pgfpathclose%
\pgfusepath{fill}%
\end{pgfscope}%
\begin{pgfscope}%
\pgfpathrectangle{\pgfqpoint{1.254980in}{0.150000in}}{\pgfqpoint{5.490039in}{5.490039in}}%
\pgfusepath{clip}%
\pgfsetbuttcap%
\pgfsetroundjoin%
\definecolor{currentfill}{rgb}{0.214000,0.722114,0.469588}%
\pgfsetfillcolor{currentfill}%
\pgfsetfillopacity{0.700000}%
\pgfsetlinewidth{0.000000pt}%
\definecolor{currentstroke}{rgb}{0.000000,0.000000,0.000000}%
\pgfsetstrokecolor{currentstroke}%
\pgfsetdash{}{0pt}%
\pgfpathmoveto{\pgfqpoint{5.088509in}{2.729504in}}%
\pgfpathlineto{\pgfqpoint{5.102990in}{2.744958in}}%
\pgfpathlineto{\pgfqpoint{5.117492in}{2.760580in}}%
\pgfpathlineto{\pgfqpoint{5.132015in}{2.776368in}}%
\pgfpathlineto{\pgfqpoint{5.146559in}{2.792324in}}%
\pgfpathlineto{\pgfqpoint{5.154277in}{2.805980in}}%
\pgfpathlineto{\pgfqpoint{5.161987in}{2.819443in}}%
\pgfpathlineto{\pgfqpoint{5.169689in}{2.832712in}}%
\pgfpathlineto{\pgfqpoint{5.177382in}{2.845787in}}%
\pgfpathlineto{\pgfqpoint{5.162832in}{2.829688in}}%
\pgfpathlineto{\pgfqpoint{5.148304in}{2.813756in}}%
\pgfpathlineto{\pgfqpoint{5.133796in}{2.797992in}}%
\pgfpathlineto{\pgfqpoint{5.119310in}{2.782394in}}%
\pgfpathlineto{\pgfqpoint{5.111622in}{2.769450in}}%
\pgfpathlineto{\pgfqpoint{5.103925in}{2.756320in}}%
\pgfpathlineto{\pgfqpoint{5.096221in}{2.743004in}}%
\pgfpathlineto{\pgfqpoint{5.088509in}{2.729504in}}%
\pgfpathclose%
\pgfusepath{fill}%
\end{pgfscope}%
\begin{pgfscope}%
\pgfpathrectangle{\pgfqpoint{1.254980in}{0.150000in}}{\pgfqpoint{5.490039in}{5.490039in}}%
\pgfusepath{clip}%
\pgfsetbuttcap%
\pgfsetroundjoin%
\definecolor{currentfill}{rgb}{0.282290,0.145912,0.461510}%
\pgfsetfillcolor{currentfill}%
\pgfsetfillopacity{0.700000}%
\pgfsetlinewidth{0.000000pt}%
\definecolor{currentstroke}{rgb}{0.000000,0.000000,0.000000}%
\pgfsetstrokecolor{currentstroke}%
\pgfsetdash{}{0pt}%
\pgfpathmoveto{\pgfqpoint{4.069651in}{1.270456in}}%
\pgfpathlineto{\pgfqpoint{4.083510in}{1.273233in}}%
\pgfpathlineto{\pgfqpoint{4.097380in}{1.276162in}}%
\pgfpathlineto{\pgfqpoint{4.111260in}{1.279243in}}%
\pgfpathlineto{\pgfqpoint{4.125151in}{1.282477in}}%
\pgfpathlineto{\pgfqpoint{4.133147in}{1.296795in}}%
\pgfpathlineto{\pgfqpoint{4.141139in}{1.311251in}}%
\pgfpathlineto{\pgfqpoint{4.149126in}{1.325837in}}%
\pgfpathlineto{\pgfqpoint{4.157110in}{1.340548in}}%
\pgfpathlineto{\pgfqpoint{4.143220in}{1.336660in}}%
\pgfpathlineto{\pgfqpoint{4.129342in}{1.332924in}}%
\pgfpathlineto{\pgfqpoint{4.115475in}{1.329342in}}%
\pgfpathlineto{\pgfqpoint{4.101619in}{1.325912in}}%
\pgfpathlineto{\pgfqpoint{4.093633in}{1.311845in}}%
\pgfpathlineto{\pgfqpoint{4.085644in}{1.297909in}}%
\pgfpathlineto{\pgfqpoint{4.077650in}{1.284110in}}%
\pgfpathlineto{\pgfqpoint{4.069651in}{1.270456in}}%
\pgfpathclose%
\pgfusepath{fill}%
\end{pgfscope}%
\begin{pgfscope}%
\pgfpathrectangle{\pgfqpoint{1.254980in}{0.150000in}}{\pgfqpoint{5.490039in}{5.490039in}}%
\pgfusepath{clip}%
\pgfsetbuttcap%
\pgfsetroundjoin%
\definecolor{currentfill}{rgb}{0.185556,0.418570,0.556753}%
\pgfsetfillcolor{currentfill}%
\pgfsetfillopacity{0.700000}%
\pgfsetlinewidth{0.000000pt}%
\definecolor{currentstroke}{rgb}{0.000000,0.000000,0.000000}%
\pgfsetstrokecolor{currentstroke}%
\pgfsetdash{}{0pt}%
\pgfpathmoveto{\pgfqpoint{4.547168in}{1.884869in}}%
\pgfpathlineto{\pgfqpoint{4.561268in}{1.894619in}}%
\pgfpathlineto{\pgfqpoint{4.575384in}{1.904528in}}%
\pgfpathlineto{\pgfqpoint{4.589517in}{1.914595in}}%
\pgfpathlineto{\pgfqpoint{4.603665in}{1.924821in}}%
\pgfpathlineto{\pgfqpoint{4.611572in}{1.942566in}}%
\pgfpathlineto{\pgfqpoint{4.619474in}{1.960254in}}%
\pgfpathlineto{\pgfqpoint{4.627374in}{1.977882in}}%
\pgfpathlineto{\pgfqpoint{4.635269in}{1.995444in}}%
\pgfpathlineto{\pgfqpoint{4.621111in}{1.984776in}}%
\pgfpathlineto{\pgfqpoint{4.606969in}{1.974267in}}%
\pgfpathlineto{\pgfqpoint{4.592844in}{1.963917in}}%
\pgfpathlineto{\pgfqpoint{4.578736in}{1.953726in}}%
\pgfpathlineto{\pgfqpoint{4.570849in}{1.936594in}}%
\pgfpathlineto{\pgfqpoint{4.562959in}{1.919405in}}%
\pgfpathlineto{\pgfqpoint{4.555066in}{1.902161in}}%
\pgfpathlineto{\pgfqpoint{4.547168in}{1.884869in}}%
\pgfpathclose%
\pgfusepath{fill}%
\end{pgfscope}%
\begin{pgfscope}%
\pgfpathrectangle{\pgfqpoint{1.254980in}{0.150000in}}{\pgfqpoint{5.490039in}{5.490039in}}%
\pgfusepath{clip}%
\pgfsetbuttcap%
\pgfsetroundjoin%
\definecolor{currentfill}{rgb}{0.468053,0.818921,0.323998}%
\pgfsetfillcolor{currentfill}%
\pgfsetfillopacity{0.700000}%
\pgfsetlinewidth{0.000000pt}%
\definecolor{currentstroke}{rgb}{0.000000,0.000000,0.000000}%
\pgfsetstrokecolor{currentstroke}%
\pgfsetdash{}{0pt}%
\pgfpathmoveto{\pgfqpoint{5.327464in}{3.054297in}}%
\pgfpathlineto{\pgfqpoint{5.342132in}{3.071477in}}%
\pgfpathlineto{\pgfqpoint{5.356823in}{3.088827in}}%
\pgfpathlineto{\pgfqpoint{5.371537in}{3.106346in}}%
\pgfpathlineto{\pgfqpoint{5.386273in}{3.124036in}}%
\pgfpathlineto{\pgfqpoint{5.393850in}{3.134583in}}%
\pgfpathlineto{\pgfqpoint{5.401417in}{3.144913in}}%
\pgfpathlineto{\pgfqpoint{5.408973in}{3.155028in}}%
\pgfpathlineto{\pgfqpoint{5.416519in}{3.164927in}}%
\pgfpathlineto{\pgfqpoint{5.401782in}{3.147224in}}%
\pgfpathlineto{\pgfqpoint{5.387069in}{3.129690in}}%
\pgfpathlineto{\pgfqpoint{5.372378in}{3.112327in}}%
\pgfpathlineto{\pgfqpoint{5.357710in}{3.095133in}}%
\pgfpathlineto{\pgfqpoint{5.350164in}{3.085235in}}%
\pgfpathlineto{\pgfqpoint{5.342607in}{3.075130in}}%
\pgfpathlineto{\pgfqpoint{5.335041in}{3.064818in}}%
\pgfpathlineto{\pgfqpoint{5.327464in}{3.054297in}}%
\pgfpathclose%
\pgfusepath{fill}%
\end{pgfscope}%
\begin{pgfscope}%
\pgfpathrectangle{\pgfqpoint{1.254980in}{0.150000in}}{\pgfqpoint{5.490039in}{5.490039in}}%
\pgfusepath{clip}%
\pgfsetbuttcap%
\pgfsetroundjoin%
\definecolor{currentfill}{rgb}{0.327796,0.773980,0.406640}%
\pgfsetfillcolor{currentfill}%
\pgfsetfillopacity{0.700000}%
\pgfsetlinewidth{0.000000pt}%
\definecolor{currentstroke}{rgb}{0.000000,0.000000,0.000000}%
\pgfsetstrokecolor{currentstroke}%
\pgfsetdash{}{0pt}%
\pgfpathmoveto{\pgfqpoint{5.208069in}{2.896125in}}%
\pgfpathlineto{\pgfqpoint{5.222645in}{2.912504in}}%
\pgfpathlineto{\pgfqpoint{5.237242in}{2.929051in}}%
\pgfpathlineto{\pgfqpoint{5.251861in}{2.945766in}}%
\pgfpathlineto{\pgfqpoint{5.266502in}{2.962651in}}%
\pgfpathlineto{\pgfqpoint{5.274156in}{2.974833in}}%
\pgfpathlineto{\pgfqpoint{5.281800in}{2.986808in}}%
\pgfpathlineto{\pgfqpoint{5.289435in}{2.998576in}}%
\pgfpathlineto{\pgfqpoint{5.297060in}{3.010136in}}%
\pgfpathlineto{\pgfqpoint{5.282416in}{2.993172in}}%
\pgfpathlineto{\pgfqpoint{5.267793in}{2.976377in}}%
\pgfpathlineto{\pgfqpoint{5.253193in}{2.959751in}}%
\pgfpathlineto{\pgfqpoint{5.238615in}{2.943293in}}%
\pgfpathlineto{\pgfqpoint{5.230992in}{2.931800in}}%
\pgfpathlineto{\pgfqpoint{5.223360in}{2.920107in}}%
\pgfpathlineto{\pgfqpoint{5.215719in}{2.908216in}}%
\pgfpathlineto{\pgfqpoint{5.208069in}{2.896125in}}%
\pgfpathclose%
\pgfusepath{fill}%
\end{pgfscope}%
\begin{pgfscope}%
\pgfpathrectangle{\pgfqpoint{1.254980in}{0.150000in}}{\pgfqpoint{5.490039in}{5.490039in}}%
\pgfusepath{clip}%
\pgfsetbuttcap%
\pgfsetroundjoin%
\definecolor{currentfill}{rgb}{0.124395,0.578002,0.548287}%
\pgfsetfillcolor{currentfill}%
\pgfsetfillopacity{0.700000}%
\pgfsetlinewidth{0.000000pt}%
\definecolor{currentstroke}{rgb}{0.000000,0.000000,0.000000}%
\pgfsetstrokecolor{currentstroke}%
\pgfsetdash{}{0pt}%
\pgfpathmoveto{\pgfqpoint{4.817971in}{2.314853in}}%
\pgfpathlineto{\pgfqpoint{4.832255in}{2.327809in}}%
\pgfpathlineto{\pgfqpoint{4.846558in}{2.340927in}}%
\pgfpathlineto{\pgfqpoint{4.860880in}{2.354209in}}%
\pgfpathlineto{\pgfqpoint{4.875220in}{2.367653in}}%
\pgfpathlineto{\pgfqpoint{4.883057in}{2.384190in}}%
\pgfpathlineto{\pgfqpoint{4.890889in}{2.400588in}}%
\pgfpathlineto{\pgfqpoint{4.898715in}{2.416844in}}%
\pgfpathlineto{\pgfqpoint{4.906535in}{2.432956in}}%
\pgfpathlineto{\pgfqpoint{4.892185in}{2.419213in}}%
\pgfpathlineto{\pgfqpoint{4.877854in}{2.405634in}}%
\pgfpathlineto{\pgfqpoint{4.863541in}{2.392218in}}%
\pgfpathlineto{\pgfqpoint{4.849248in}{2.378966in}}%
\pgfpathlineto{\pgfqpoint{4.841437in}{2.363141in}}%
\pgfpathlineto{\pgfqpoint{4.833620in}{2.347178in}}%
\pgfpathlineto{\pgfqpoint{4.825798in}{2.331081in}}%
\pgfpathlineto{\pgfqpoint{4.817971in}{2.314853in}}%
\pgfpathclose%
\pgfusepath{fill}%
\end{pgfscope}%
\begin{pgfscope}%
\pgfpathrectangle{\pgfqpoint{1.254980in}{0.150000in}}{\pgfqpoint{5.490039in}{5.490039in}}%
\pgfusepath{clip}%
\pgfsetbuttcap%
\pgfsetroundjoin%
\definecolor{currentfill}{rgb}{0.257322,0.256130,0.526563}%
\pgfsetfillcolor{currentfill}%
\pgfsetfillopacity{0.700000}%
\pgfsetlinewidth{0.000000pt}%
\definecolor{currentstroke}{rgb}{0.000000,0.000000,0.000000}%
\pgfsetstrokecolor{currentstroke}%
\pgfsetdash{}{0pt}%
\pgfpathmoveto{\pgfqpoint{4.276525in}{1.484016in}}%
\pgfpathlineto{\pgfqpoint{4.290478in}{1.489903in}}%
\pgfpathlineto{\pgfqpoint{4.304444in}{1.495945in}}%
\pgfpathlineto{\pgfqpoint{4.318423in}{1.502140in}}%
\pgfpathlineto{\pgfqpoint{4.332415in}{1.508490in}}%
\pgfpathlineto{\pgfqpoint{4.340371in}{1.525228in}}%
\pgfpathlineto{\pgfqpoint{4.348324in}{1.542020in}}%
\pgfpathlineto{\pgfqpoint{4.356274in}{1.558859in}}%
\pgfpathlineto{\pgfqpoint{4.364221in}{1.575740in}}%
\pgfpathlineto{\pgfqpoint{4.350224in}{1.568812in}}%
\pgfpathlineto{\pgfqpoint{4.336241in}{1.562039in}}%
\pgfpathlineto{\pgfqpoint{4.322271in}{1.555422in}}%
\pgfpathlineto{\pgfqpoint{4.308314in}{1.548959in}}%
\pgfpathlineto{\pgfqpoint{4.300372in}{1.532644in}}%
\pgfpathlineto{\pgfqpoint{4.292426in}{1.516379in}}%
\pgfpathlineto{\pgfqpoint{4.284478in}{1.500168in}}%
\pgfpathlineto{\pgfqpoint{4.276525in}{1.484016in}}%
\pgfpathclose%
\pgfusepath{fill}%
\end{pgfscope}%
\begin{pgfscope}%
\pgfpathrectangle{\pgfqpoint{1.254980in}{0.150000in}}{\pgfqpoint{5.490039in}{5.490039in}}%
\pgfusepath{clip}%
\pgfsetbuttcap%
\pgfsetroundjoin%
\definecolor{currentfill}{rgb}{0.227802,0.326594,0.546532}%
\pgfsetfillcolor{currentfill}%
\pgfsetfillopacity{0.700000}%
\pgfsetlinewidth{0.000000pt}%
\definecolor{currentstroke}{rgb}{0.000000,0.000000,0.000000}%
\pgfsetstrokecolor{currentstroke}%
\pgfsetdash{}{0pt}%
\pgfpathmoveto{\pgfqpoint{4.395976in}{1.643579in}}%
\pgfpathlineto{\pgfqpoint{4.409993in}{1.651213in}}%
\pgfpathlineto{\pgfqpoint{4.424025in}{1.659002in}}%
\pgfpathlineto{\pgfqpoint{4.438070in}{1.666948in}}%
\pgfpathlineto{\pgfqpoint{4.452131in}{1.675049in}}%
\pgfpathlineto{\pgfqpoint{4.460068in}{1.692597in}}%
\pgfpathlineto{\pgfqpoint{4.468003in}{1.710149in}}%
\pgfpathlineto{\pgfqpoint{4.475935in}{1.727701in}}%
\pgfpathlineto{\pgfqpoint{4.483863in}{1.745247in}}%
\pgfpathlineto{\pgfqpoint{4.469795in}{1.736620in}}%
\pgfpathlineto{\pgfqpoint{4.455742in}{1.728150in}}%
\pgfpathlineto{\pgfqpoint{4.441703in}{1.719837in}}%
\pgfpathlineto{\pgfqpoint{4.427680in}{1.711680in}}%
\pgfpathlineto{\pgfqpoint{4.419759in}{1.694648in}}%
\pgfpathlineto{\pgfqpoint{4.411835in}{1.677617in}}%
\pgfpathlineto{\pgfqpoint{4.403907in}{1.660592in}}%
\pgfpathlineto{\pgfqpoint{4.395976in}{1.643579in}}%
\pgfpathclose%
\pgfusepath{fill}%
\end{pgfscope}%
\begin{pgfscope}%
\pgfpathrectangle{\pgfqpoint{1.254980in}{0.150000in}}{\pgfqpoint{5.490039in}{5.490039in}}%
\pgfusepath{clip}%
\pgfsetbuttcap%
\pgfsetroundjoin%
\definecolor{currentfill}{rgb}{0.276194,0.190074,0.493001}%
\pgfsetfillcolor{currentfill}%
\pgfsetfillopacity{0.700000}%
\pgfsetlinewidth{0.000000pt}%
\definecolor{currentstroke}{rgb}{0.000000,0.000000,0.000000}%
\pgfsetstrokecolor{currentstroke}%
\pgfsetdash{}{0pt}%
\pgfpathmoveto{\pgfqpoint{4.157110in}{1.340548in}}%
\pgfpathlineto{\pgfqpoint{4.171011in}{1.344590in}}%
\pgfpathlineto{\pgfqpoint{4.184923in}{1.348784in}}%
\pgfpathlineto{\pgfqpoint{4.198847in}{1.353131in}}%
\pgfpathlineto{\pgfqpoint{4.212783in}{1.357631in}}%
\pgfpathlineto{\pgfqpoint{4.220764in}{1.373101in}}%
\pgfpathlineto{\pgfqpoint{4.228740in}{1.388676in}}%
\pgfpathlineto{\pgfqpoint{4.236713in}{1.404351in}}%
\pgfpathlineto{\pgfqpoint{4.244682in}{1.420119in}}%
\pgfpathlineto{\pgfqpoint{4.230745in}{1.414989in}}%
\pgfpathlineto{\pgfqpoint{4.216820in}{1.410013in}}%
\pgfpathlineto{\pgfqpoint{4.202907in}{1.405190in}}%
\pgfpathlineto{\pgfqpoint{4.189005in}{1.400520in}}%
\pgfpathlineto{\pgfqpoint{4.181037in}{1.385370in}}%
\pgfpathlineto{\pgfqpoint{4.173065in}{1.370321in}}%
\pgfpathlineto{\pgfqpoint{4.165090in}{1.355378in}}%
\pgfpathlineto{\pgfqpoint{4.157110in}{1.340548in}}%
\pgfpathclose%
\pgfusepath{fill}%
\end{pgfscope}%
\begin{pgfscope}%
\pgfpathrectangle{\pgfqpoint{1.254980in}{0.150000in}}{\pgfqpoint{5.490039in}{5.490039in}}%
\pgfusepath{clip}%
\pgfsetbuttcap%
\pgfsetroundjoin%
\definecolor{currentfill}{rgb}{0.156270,0.489624,0.557936}%
\pgfsetfillcolor{currentfill}%
\pgfsetfillopacity{0.700000}%
\pgfsetlinewidth{0.000000pt}%
\definecolor{currentstroke}{rgb}{0.000000,0.000000,0.000000}%
\pgfsetstrokecolor{currentstroke}%
\pgfsetdash{}{0pt}%
\pgfpathmoveto{\pgfqpoint{4.666811in}{2.064968in}}%
\pgfpathlineto{\pgfqpoint{4.680995in}{2.076210in}}%
\pgfpathlineto{\pgfqpoint{4.695197in}{2.087612in}}%
\pgfpathlineto{\pgfqpoint{4.709416in}{2.099175in}}%
\pgfpathlineto{\pgfqpoint{4.723652in}{2.110898in}}%
\pgfpathlineto{\pgfqpoint{4.731537in}{2.128476in}}%
\pgfpathlineto{\pgfqpoint{4.739418in}{2.145959in}}%
\pgfpathlineto{\pgfqpoint{4.747295in}{2.163344in}}%
\pgfpathlineto{\pgfqpoint{4.755167in}{2.180627in}}%
\pgfpathlineto{\pgfqpoint{4.740920in}{2.168517in}}%
\pgfpathlineto{\pgfqpoint{4.726691in}{2.156569in}}%
\pgfpathlineto{\pgfqpoint{4.712480in}{2.144781in}}%
\pgfpathlineto{\pgfqpoint{4.698286in}{2.133155in}}%
\pgfpathlineto{\pgfqpoint{4.690423in}{2.116246in}}%
\pgfpathlineto{\pgfqpoint{4.682557in}{2.099243in}}%
\pgfpathlineto{\pgfqpoint{4.674686in}{2.082149in}}%
\pgfpathlineto{\pgfqpoint{4.666811in}{2.064968in}}%
\pgfpathclose%
\pgfusepath{fill}%
\end{pgfscope}%
\begin{pgfscope}%
\pgfpathrectangle{\pgfqpoint{1.254980in}{0.150000in}}{\pgfqpoint{5.490039in}{5.490039in}}%
\pgfusepath{clip}%
\pgfsetbuttcap%
\pgfsetroundjoin%
\definecolor{currentfill}{rgb}{0.128087,0.647749,0.523491}%
\pgfsetfillcolor{currentfill}%
\pgfsetfillopacity{0.700000}%
\pgfsetlinewidth{0.000000pt}%
\definecolor{currentstroke}{rgb}{0.000000,0.000000,0.000000}%
\pgfsetstrokecolor{currentstroke}%
\pgfsetdash{}{0pt}%
\pgfpathmoveto{\pgfqpoint{4.937755in}{2.495911in}}%
\pgfpathlineto{\pgfqpoint{4.952134in}{2.510086in}}%
\pgfpathlineto{\pgfqpoint{4.966532in}{2.524425in}}%
\pgfpathlineto{\pgfqpoint{4.980950in}{2.538929in}}%
\pgfpathlineto{\pgfqpoint{4.995388in}{2.553599in}}%
\pgfpathlineto{\pgfqpoint{5.003186in}{2.569202in}}%
\pgfpathlineto{\pgfqpoint{5.010978in}{2.584640in}}%
\pgfpathlineto{\pgfqpoint{5.018763in}{2.599910in}}%
\pgfpathlineto{\pgfqpoint{5.026541in}{2.615011in}}%
\pgfpathlineto{\pgfqpoint{5.012095in}{2.600104in}}%
\pgfpathlineto{\pgfqpoint{4.997669in}{2.585362in}}%
\pgfpathlineto{\pgfqpoint{4.983262in}{2.570786in}}%
\pgfpathlineto{\pgfqpoint{4.968876in}{2.556375in}}%
\pgfpathlineto{\pgfqpoint{4.961105in}{2.541500in}}%
\pgfpathlineto{\pgfqpoint{4.953328in}{2.526463in}}%
\pgfpathlineto{\pgfqpoint{4.945545in}{2.511266in}}%
\pgfpathlineto{\pgfqpoint{4.937755in}{2.495911in}}%
\pgfpathclose%
\pgfusepath{fill}%
\end{pgfscope}%
\begin{pgfscope}%
\pgfpathrectangle{\pgfqpoint{1.254980in}{0.150000in}}{\pgfqpoint{5.490039in}{5.490039in}}%
\pgfusepath{clip}%
\pgfsetbuttcap%
\pgfsetroundjoin%
\definecolor{currentfill}{rgb}{0.282656,0.100196,0.422160}%
\pgfsetfillcolor{currentfill}%
\pgfsetfillopacity{0.700000}%
\pgfsetlinewidth{0.000000pt}%
\definecolor{currentstroke}{rgb}{0.000000,0.000000,0.000000}%
\pgfsetstrokecolor{currentstroke}%
\pgfsetdash{}{0pt}%
\pgfpathmoveto{\pgfqpoint{3.950108in}{1.163158in}}%
\pgfpathlineto{\pgfqpoint{3.963940in}{1.163940in}}%
\pgfpathlineto{\pgfqpoint{3.977781in}{1.164873in}}%
\pgfpathlineto{\pgfqpoint{3.991632in}{1.165958in}}%
\pgfpathlineto{\pgfqpoint{4.005492in}{1.167194in}}%
\pgfpathlineto{\pgfqpoint{4.013529in}{1.179459in}}%
\pgfpathlineto{\pgfqpoint{4.021562in}{1.191922in}}%
\pgfpathlineto{\pgfqpoint{4.029589in}{1.204574in}}%
\pgfpathlineto{\pgfqpoint{4.037611in}{1.217411in}}%
\pgfpathlineto{\pgfqpoint{4.023758in}{1.215467in}}%
\pgfpathlineto{\pgfqpoint{4.009914in}{1.213675in}}%
\pgfpathlineto{\pgfqpoint{3.996080in}{1.212035in}}%
\pgfpathlineto{\pgfqpoint{3.982256in}{1.210548in}}%
\pgfpathlineto{\pgfqpoint{3.974227in}{1.198408in}}%
\pgfpathlineto{\pgfqpoint{3.966193in}{1.186459in}}%
\pgfpathlineto{\pgfqpoint{3.958153in}{1.174706in}}%
\pgfpathlineto{\pgfqpoint{3.950108in}{1.163158in}}%
\pgfpathclose%
\pgfusepath{fill}%
\end{pgfscope}%
\begin{pgfscope}%
\pgfpathrectangle{\pgfqpoint{1.254980in}{0.150000in}}{\pgfqpoint{5.490039in}{5.490039in}}%
\pgfusepath{clip}%
\pgfsetbuttcap%
\pgfsetroundjoin%
\definecolor{currentfill}{rgb}{0.194100,0.399323,0.555565}%
\pgfsetfillcolor{currentfill}%
\pgfsetfillopacity{0.700000}%
\pgfsetlinewidth{0.000000pt}%
\definecolor{currentstroke}{rgb}{0.000000,0.000000,0.000000}%
\pgfsetstrokecolor{currentstroke}%
\pgfsetdash{}{0pt}%
\pgfpathmoveto{\pgfqpoint{4.515543in}{1.815284in}}%
\pgfpathlineto{\pgfqpoint{4.529634in}{1.824565in}}%
\pgfpathlineto{\pgfqpoint{4.543742in}{1.834004in}}%
\pgfpathlineto{\pgfqpoint{4.557864in}{1.843601in}}%
\pgfpathlineto{\pgfqpoint{4.572003in}{1.853355in}}%
\pgfpathlineto{\pgfqpoint{4.579924in}{1.871286in}}%
\pgfpathlineto{\pgfqpoint{4.587841in}{1.889177in}}%
\pgfpathlineto{\pgfqpoint{4.595755in}{1.907023in}}%
\pgfpathlineto{\pgfqpoint{4.603665in}{1.924821in}}%
\pgfpathlineto{\pgfqpoint{4.589517in}{1.914595in}}%
\pgfpathlineto{\pgfqpoint{4.575384in}{1.904528in}}%
\pgfpathlineto{\pgfqpoint{4.561268in}{1.894619in}}%
\pgfpathlineto{\pgfqpoint{4.547168in}{1.884869in}}%
\pgfpathlineto{\pgfqpoint{4.539267in}{1.867530in}}%
\pgfpathlineto{\pgfqpoint{4.531363in}{1.850151in}}%
\pgfpathlineto{\pgfqpoint{4.523454in}{1.832734in}}%
\pgfpathlineto{\pgfqpoint{4.515543in}{1.815284in}}%
\pgfpathclose%
\pgfusepath{fill}%
\end{pgfscope}%
\begin{pgfscope}%
\pgfpathrectangle{\pgfqpoint{1.254980in}{0.150000in}}{\pgfqpoint{5.490039in}{5.490039in}}%
\pgfusepath{clip}%
\pgfsetbuttcap%
\pgfsetroundjoin%
\definecolor{currentfill}{rgb}{0.575563,0.844566,0.256415}%
\pgfsetfillcolor{currentfill}%
\pgfsetfillopacity{0.700000}%
\pgfsetlinewidth{0.000000pt}%
\definecolor{currentstroke}{rgb}{0.000000,0.000000,0.000000}%
\pgfsetstrokecolor{currentstroke}%
\pgfsetdash{}{0pt}%
\pgfpathmoveto{\pgfqpoint{5.416519in}{3.164927in}}%
\pgfpathlineto{\pgfqpoint{5.431278in}{3.182801in}}%
\pgfpathlineto{\pgfqpoint{5.446061in}{3.200846in}}%
\pgfpathlineto{\pgfqpoint{5.460868in}{3.219063in}}%
\pgfpathlineto{\pgfqpoint{5.468402in}{3.228742in}}%
\pgfpathlineto{\pgfqpoint{5.475925in}{3.238200in}}%
\pgfpathlineto{\pgfqpoint{5.483437in}{3.247438in}}%
\pgfpathlineto{\pgfqpoint{5.490937in}{3.256456in}}%
\pgfpathlineto{\pgfqpoint{5.476133in}{3.238260in}}%
\pgfpathlineto{\pgfqpoint{5.461351in}{3.220234in}}%
\pgfpathlineto{\pgfqpoint{5.446593in}{3.202379in}}%
\pgfpathlineto{\pgfqpoint{5.439091in}{3.193337in}}%
\pgfpathlineto{\pgfqpoint{5.431578in}{3.184081in}}%
\pgfpathlineto{\pgfqpoint{5.424054in}{3.174611in}}%
\pgfpathlineto{\pgfqpoint{5.416519in}{3.164927in}}%
\pgfpathclose%
\pgfusepath{fill}%
\end{pgfscope}%
\begin{pgfscope}%
\pgfpathrectangle{\pgfqpoint{1.254980in}{0.150000in}}{\pgfqpoint{5.490039in}{5.490039in}}%
\pgfusepath{clip}%
\pgfsetbuttcap%
\pgfsetroundjoin%
\definecolor{currentfill}{rgb}{0.191090,0.708366,0.482284}%
\pgfsetfillcolor{currentfill}%
\pgfsetfillopacity{0.700000}%
\pgfsetlinewidth{0.000000pt}%
\definecolor{currentstroke}{rgb}{0.000000,0.000000,0.000000}%
\pgfsetstrokecolor{currentstroke}%
\pgfsetdash{}{0pt}%
\pgfpathmoveto{\pgfqpoint{5.057584in}{2.673681in}}%
\pgfpathlineto{\pgfqpoint{5.072059in}{2.688961in}}%
\pgfpathlineto{\pgfqpoint{5.086554in}{2.704407in}}%
\pgfpathlineto{\pgfqpoint{5.101070in}{2.720020in}}%
\pgfpathlineto{\pgfqpoint{5.115607in}{2.735800in}}%
\pgfpathlineto{\pgfqpoint{5.123357in}{2.750214in}}%
\pgfpathlineto{\pgfqpoint{5.131099in}{2.764440in}}%
\pgfpathlineto{\pgfqpoint{5.138833in}{2.778477in}}%
\pgfpathlineto{\pgfqpoint{5.146559in}{2.792324in}}%
\pgfpathlineto{\pgfqpoint{5.132015in}{2.776368in}}%
\pgfpathlineto{\pgfqpoint{5.117492in}{2.760580in}}%
\pgfpathlineto{\pgfqpoint{5.102990in}{2.744958in}}%
\pgfpathlineto{\pgfqpoint{5.088509in}{2.729504in}}%
\pgfpathlineto{\pgfqpoint{5.080789in}{2.715820in}}%
\pgfpathlineto{\pgfqpoint{5.073062in}{2.701954in}}%
\pgfpathlineto{\pgfqpoint{5.065327in}{2.687907in}}%
\pgfpathlineto{\pgfqpoint{5.057584in}{2.673681in}}%
\pgfpathclose%
\pgfusepath{fill}%
\end{pgfscope}%
\begin{pgfscope}%
\pgfpathrectangle{\pgfqpoint{1.254980in}{0.150000in}}{\pgfqpoint{5.490039in}{5.490039in}}%
\pgfusepath{clip}%
\pgfsetbuttcap%
\pgfsetroundjoin%
\definecolor{currentfill}{rgb}{0.283072,0.130895,0.449241}%
\pgfsetfillcolor{currentfill}%
\pgfsetfillopacity{0.700000}%
\pgfsetlinewidth{0.000000pt}%
\definecolor{currentstroke}{rgb}{0.000000,0.000000,0.000000}%
\pgfsetstrokecolor{currentstroke}%
\pgfsetdash{}{0pt}%
\pgfpathmoveto{\pgfqpoint{4.037611in}{1.217411in}}%
\pgfpathlineto{\pgfqpoint{4.051474in}{1.219507in}}%
\pgfpathlineto{\pgfqpoint{4.065348in}{1.221754in}}%
\pgfpathlineto{\pgfqpoint{4.079231in}{1.224154in}}%
\pgfpathlineto{\pgfqpoint{4.093125in}{1.226706in}}%
\pgfpathlineto{\pgfqpoint{4.101138in}{1.240410in}}%
\pgfpathlineto{\pgfqpoint{4.109147in}{1.254278in}}%
\pgfpathlineto{\pgfqpoint{4.117151in}{1.268302in}}%
\pgfpathlineto{\pgfqpoint{4.125151in}{1.282477in}}%
\pgfpathlineto{\pgfqpoint{4.111260in}{1.279243in}}%
\pgfpathlineto{\pgfqpoint{4.097380in}{1.276162in}}%
\pgfpathlineto{\pgfqpoint{4.083510in}{1.273233in}}%
\pgfpathlineto{\pgfqpoint{4.069651in}{1.270456in}}%
\pgfpathlineto{\pgfqpoint{4.061648in}{1.256953in}}%
\pgfpathlineto{\pgfqpoint{4.053641in}{1.243607in}}%
\pgfpathlineto{\pgfqpoint{4.045628in}{1.230424in}}%
\pgfpathlineto{\pgfqpoint{4.037611in}{1.217411in}}%
\pgfpathclose%
\pgfusepath{fill}%
\end{pgfscope}%
\begin{pgfscope}%
\pgfpathrectangle{\pgfqpoint{1.254980in}{0.150000in}}{\pgfqpoint{5.490039in}{5.490039in}}%
\pgfusepath{clip}%
\pgfsetbuttcap%
\pgfsetroundjoin%
\definecolor{currentfill}{rgb}{0.129933,0.559582,0.551864}%
\pgfsetfillcolor{currentfill}%
\pgfsetfillopacity{0.700000}%
\pgfsetlinewidth{0.000000pt}%
\definecolor{currentstroke}{rgb}{0.000000,0.000000,0.000000}%
\pgfsetstrokecolor{currentstroke}%
\pgfsetdash{}{0pt}%
\pgfpathmoveto{\pgfqpoint{4.786608in}{2.248679in}}%
\pgfpathlineto{\pgfqpoint{4.800883in}{2.261308in}}%
\pgfpathlineto{\pgfqpoint{4.815176in}{2.274099in}}%
\pgfpathlineto{\pgfqpoint{4.829488in}{2.287052in}}%
\pgfpathlineto{\pgfqpoint{4.843818in}{2.300168in}}%
\pgfpathlineto{\pgfqpoint{4.851676in}{2.317234in}}%
\pgfpathlineto{\pgfqpoint{4.859530in}{2.334172in}}%
\pgfpathlineto{\pgfqpoint{4.867378in}{2.350980in}}%
\pgfpathlineto{\pgfqpoint{4.875220in}{2.367653in}}%
\pgfpathlineto{\pgfqpoint{4.860880in}{2.354209in}}%
\pgfpathlineto{\pgfqpoint{4.846558in}{2.340927in}}%
\pgfpathlineto{\pgfqpoint{4.832255in}{2.327809in}}%
\pgfpathlineto{\pgfqpoint{4.817971in}{2.314853in}}%
\pgfpathlineto{\pgfqpoint{4.810138in}{2.298496in}}%
\pgfpathlineto{\pgfqpoint{4.802300in}{2.282013in}}%
\pgfpathlineto{\pgfqpoint{4.794456in}{2.265406in}}%
\pgfpathlineto{\pgfqpoint{4.786608in}{2.248679in}}%
\pgfpathclose%
\pgfusepath{fill}%
\end{pgfscope}%
\begin{pgfscope}%
\pgfpathrectangle{\pgfqpoint{1.254980in}{0.150000in}}{\pgfqpoint{5.490039in}{5.490039in}}%
\pgfusepath{clip}%
\pgfsetbuttcap%
\pgfsetroundjoin%
\definecolor{currentfill}{rgb}{0.265145,0.232956,0.516599}%
\pgfsetfillcolor{currentfill}%
\pgfsetfillopacity{0.700000}%
\pgfsetlinewidth{0.000000pt}%
\definecolor{currentstroke}{rgb}{0.000000,0.000000,0.000000}%
\pgfsetstrokecolor{currentstroke}%
\pgfsetdash{}{0pt}%
\pgfpathmoveto{\pgfqpoint{4.244682in}{1.420119in}}%
\pgfpathlineto{\pgfqpoint{4.258632in}{1.425403in}}%
\pgfpathlineto{\pgfqpoint{4.272595in}{1.430840in}}%
\pgfpathlineto{\pgfqpoint{4.286570in}{1.436431in}}%
\pgfpathlineto{\pgfqpoint{4.300557in}{1.442175in}}%
\pgfpathlineto{\pgfqpoint{4.308527in}{1.458647in}}%
\pgfpathlineto{\pgfqpoint{4.316493in}{1.475193in}}%
\pgfpathlineto{\pgfqpoint{4.324455in}{1.491809in}}%
\pgfpathlineto{\pgfqpoint{4.332415in}{1.508490in}}%
\pgfpathlineto{\pgfqpoint{4.318423in}{1.502140in}}%
\pgfpathlineto{\pgfqpoint{4.304444in}{1.495945in}}%
\pgfpathlineto{\pgfqpoint{4.290478in}{1.489903in}}%
\pgfpathlineto{\pgfqpoint{4.276525in}{1.484016in}}%
\pgfpathlineto{\pgfqpoint{4.268570in}{1.467930in}}%
\pgfpathlineto{\pgfqpoint{4.260611in}{1.451915in}}%
\pgfpathlineto{\pgfqpoint{4.252648in}{1.435976in}}%
\pgfpathlineto{\pgfqpoint{4.244682in}{1.420119in}}%
\pgfpathclose%
\pgfusepath{fill}%
\end{pgfscope}%
\begin{pgfscope}%
\pgfpathrectangle{\pgfqpoint{1.254980in}{0.150000in}}{\pgfqpoint{5.490039in}{5.490039in}}%
\pgfusepath{clip}%
\pgfsetbuttcap%
\pgfsetroundjoin%
\definecolor{currentfill}{rgb}{0.311925,0.767822,0.415586}%
\pgfsetfillcolor{currentfill}%
\pgfsetfillopacity{0.700000}%
\pgfsetlinewidth{0.000000pt}%
\definecolor{currentstroke}{rgb}{0.000000,0.000000,0.000000}%
\pgfsetstrokecolor{currentstroke}%
\pgfsetdash{}{0pt}%
\pgfpathmoveto{\pgfqpoint{5.177382in}{2.845787in}}%
\pgfpathlineto{\pgfqpoint{5.191953in}{2.862054in}}%
\pgfpathlineto{\pgfqpoint{5.206546in}{2.878490in}}%
\pgfpathlineto{\pgfqpoint{5.221160in}{2.895094in}}%
\pgfpathlineto{\pgfqpoint{5.235797in}{2.911867in}}%
\pgfpathlineto{\pgfqpoint{5.243487in}{2.924870in}}%
\pgfpathlineto{\pgfqpoint{5.251168in}{2.937669in}}%
\pgfpathlineto{\pgfqpoint{5.258840in}{2.950263in}}%
\pgfpathlineto{\pgfqpoint{5.266502in}{2.962651in}}%
\pgfpathlineto{\pgfqpoint{5.251861in}{2.945766in}}%
\pgfpathlineto{\pgfqpoint{5.237242in}{2.929051in}}%
\pgfpathlineto{\pgfqpoint{5.222645in}{2.912504in}}%
\pgfpathlineto{\pgfqpoint{5.208069in}{2.896125in}}%
\pgfpathlineto{\pgfqpoint{5.200410in}{2.883836in}}%
\pgfpathlineto{\pgfqpoint{5.192743in}{2.871350in}}%
\pgfpathlineto{\pgfqpoint{5.185067in}{2.858667in}}%
\pgfpathlineto{\pgfqpoint{5.177382in}{2.845787in}}%
\pgfpathclose%
\pgfusepath{fill}%
\end{pgfscope}%
\begin{pgfscope}%
\pgfpathrectangle{\pgfqpoint{1.254980in}{0.150000in}}{\pgfqpoint{5.490039in}{5.490039in}}%
\pgfusepath{clip}%
\pgfsetbuttcap%
\pgfsetroundjoin%
\definecolor{currentfill}{rgb}{0.237441,0.305202,0.541921}%
\pgfsetfillcolor{currentfill}%
\pgfsetfillopacity{0.700000}%
\pgfsetlinewidth{0.000000pt}%
\definecolor{currentstroke}{rgb}{0.000000,0.000000,0.000000}%
\pgfsetstrokecolor{currentstroke}%
\pgfsetdash{}{0pt}%
\pgfpathmoveto{\pgfqpoint{4.364221in}{1.575740in}}%
\pgfpathlineto{\pgfqpoint{4.378232in}{1.582822in}}%
\pgfpathlineto{\pgfqpoint{4.392257in}{1.590060in}}%
\pgfpathlineto{\pgfqpoint{4.406296in}{1.597453in}}%
\pgfpathlineto{\pgfqpoint{4.420349in}{1.605001in}}%
\pgfpathlineto{\pgfqpoint{4.428299in}{1.622482in}}%
\pgfpathlineto{\pgfqpoint{4.436246in}{1.639986in}}%
\pgfpathlineto{\pgfqpoint{4.444190in}{1.657511in}}%
\pgfpathlineto{\pgfqpoint{4.452131in}{1.675049in}}%
\pgfpathlineto{\pgfqpoint{4.438070in}{1.666948in}}%
\pgfpathlineto{\pgfqpoint{4.424025in}{1.659002in}}%
\pgfpathlineto{\pgfqpoint{4.409993in}{1.651213in}}%
\pgfpathlineto{\pgfqpoint{4.395976in}{1.643579in}}%
\pgfpathlineto{\pgfqpoint{4.388042in}{1.626582in}}%
\pgfpathlineto{\pgfqpoint{4.380105in}{1.609607in}}%
\pgfpathlineto{\pgfqpoint{4.372165in}{1.592657in}}%
\pgfpathlineto{\pgfqpoint{4.364221in}{1.575740in}}%
\pgfpathclose%
\pgfusepath{fill}%
\end{pgfscope}%
\begin{pgfscope}%
\pgfpathrectangle{\pgfqpoint{1.254980in}{0.150000in}}{\pgfqpoint{5.490039in}{5.490039in}}%
\pgfusepath{clip}%
\pgfsetbuttcap%
\pgfsetroundjoin%
\definecolor{currentfill}{rgb}{0.163625,0.471133,0.558148}%
\pgfsetfillcolor{currentfill}%
\pgfsetfillopacity{0.700000}%
\pgfsetlinewidth{0.000000pt}%
\definecolor{currentstroke}{rgb}{0.000000,0.000000,0.000000}%
\pgfsetstrokecolor{currentstroke}%
\pgfsetdash{}{0pt}%
\pgfpathmoveto{\pgfqpoint{4.635269in}{1.995444in}}%
\pgfpathlineto{\pgfqpoint{4.649444in}{2.006272in}}%
\pgfpathlineto{\pgfqpoint{4.663635in}{2.017260in}}%
\pgfpathlineto{\pgfqpoint{4.677843in}{2.028407in}}%
\pgfpathlineto{\pgfqpoint{4.692069in}{2.039714in}}%
\pgfpathlineto{\pgfqpoint{4.699971in}{2.057634in}}%
\pgfpathlineto{\pgfqpoint{4.707869in}{2.075474in}}%
\pgfpathlineto{\pgfqpoint{4.715762in}{2.093230in}}%
\pgfpathlineto{\pgfqpoint{4.723652in}{2.110898in}}%
\pgfpathlineto{\pgfqpoint{4.709416in}{2.099175in}}%
\pgfpathlineto{\pgfqpoint{4.695197in}{2.087612in}}%
\pgfpathlineto{\pgfqpoint{4.680995in}{2.076210in}}%
\pgfpathlineto{\pgfqpoint{4.666811in}{2.064968in}}%
\pgfpathlineto{\pgfqpoint{4.658931in}{2.047704in}}%
\pgfpathlineto{\pgfqpoint{4.651048in}{2.030359in}}%
\pgfpathlineto{\pgfqpoint{4.643160in}{2.012938in}}%
\pgfpathlineto{\pgfqpoint{4.635269in}{1.995444in}}%
\pgfpathclose%
\pgfusepath{fill}%
\end{pgfscope}%
\begin{pgfscope}%
\pgfpathrectangle{\pgfqpoint{1.254980in}{0.150000in}}{\pgfqpoint{5.490039in}{5.490039in}}%
\pgfusepath{clip}%
\pgfsetbuttcap%
\pgfsetroundjoin%
\definecolor{currentfill}{rgb}{0.449368,0.813768,0.335384}%
\pgfsetfillcolor{currentfill}%
\pgfsetfillopacity{0.700000}%
\pgfsetlinewidth{0.000000pt}%
\definecolor{currentstroke}{rgb}{0.000000,0.000000,0.000000}%
\pgfsetstrokecolor{currentstroke}%
\pgfsetdash{}{0pt}%
\pgfpathmoveto{\pgfqpoint{5.297060in}{3.010136in}}%
\pgfpathlineto{\pgfqpoint{5.311727in}{3.027269in}}%
\pgfpathlineto{\pgfqpoint{5.326416in}{3.044572in}}%
\pgfpathlineto{\pgfqpoint{5.341128in}{3.062045in}}%
\pgfpathlineto{\pgfqpoint{5.355862in}{3.079688in}}%
\pgfpathlineto{\pgfqpoint{5.363480in}{3.091100in}}%
\pgfpathlineto{\pgfqpoint{5.371088in}{3.102295in}}%
\pgfpathlineto{\pgfqpoint{5.378686in}{3.113274in}}%
\pgfpathlineto{\pgfqpoint{5.386273in}{3.124036in}}%
\pgfpathlineto{\pgfqpoint{5.371537in}{3.106346in}}%
\pgfpathlineto{\pgfqpoint{5.356823in}{3.088827in}}%
\pgfpathlineto{\pgfqpoint{5.342132in}{3.071477in}}%
\pgfpathlineto{\pgfqpoint{5.327464in}{3.054297in}}%
\pgfpathlineto{\pgfqpoint{5.319878in}{3.043569in}}%
\pgfpathlineto{\pgfqpoint{5.312282in}{3.032633in}}%
\pgfpathlineto{\pgfqpoint{5.304676in}{3.021488in}}%
\pgfpathlineto{\pgfqpoint{5.297060in}{3.010136in}}%
\pgfpathclose%
\pgfusepath{fill}%
\end{pgfscope}%
\begin{pgfscope}%
\pgfpathrectangle{\pgfqpoint{1.254980in}{0.150000in}}{\pgfqpoint{5.490039in}{5.490039in}}%
\pgfusepath{clip}%
\pgfsetbuttcap%
\pgfsetroundjoin%
\definecolor{currentfill}{rgb}{0.279574,0.170599,0.479997}%
\pgfsetfillcolor{currentfill}%
\pgfsetfillopacity{0.700000}%
\pgfsetlinewidth{0.000000pt}%
\definecolor{currentstroke}{rgb}{0.000000,0.000000,0.000000}%
\pgfsetstrokecolor{currentstroke}%
\pgfsetdash{}{0pt}%
\pgfpathmoveto{\pgfqpoint{4.125151in}{1.282477in}}%
\pgfpathlineto{\pgfqpoint{4.139053in}{1.285863in}}%
\pgfpathlineto{\pgfqpoint{4.152966in}{1.289401in}}%
\pgfpathlineto{\pgfqpoint{4.166891in}{1.293092in}}%
\pgfpathlineto{\pgfqpoint{4.180826in}{1.296934in}}%
\pgfpathlineto{\pgfqpoint{4.188821in}{1.311919in}}%
\pgfpathlineto{\pgfqpoint{4.196812in}{1.327034in}}%
\pgfpathlineto{\pgfqpoint{4.204800in}{1.342274in}}%
\pgfpathlineto{\pgfqpoint{4.212783in}{1.357631in}}%
\pgfpathlineto{\pgfqpoint{4.198847in}{1.353131in}}%
\pgfpathlineto{\pgfqpoint{4.184923in}{1.348784in}}%
\pgfpathlineto{\pgfqpoint{4.171011in}{1.344590in}}%
\pgfpathlineto{\pgfqpoint{4.157110in}{1.340548in}}%
\pgfpathlineto{\pgfqpoint{4.149126in}{1.325837in}}%
\pgfpathlineto{\pgfqpoint{4.141139in}{1.311251in}}%
\pgfpathlineto{\pgfqpoint{4.133147in}{1.296795in}}%
\pgfpathlineto{\pgfqpoint{4.125151in}{1.282477in}}%
\pgfpathclose%
\pgfusepath{fill}%
\end{pgfscope}%
\begin{pgfscope}%
\pgfpathrectangle{\pgfqpoint{1.254980in}{0.150000in}}{\pgfqpoint{5.490039in}{5.490039in}}%
\pgfusepath{clip}%
\pgfsetbuttcap%
\pgfsetroundjoin%
\definecolor{currentfill}{rgb}{0.121380,0.629492,0.531973}%
\pgfsetfillcolor{currentfill}%
\pgfsetfillopacity{0.700000}%
\pgfsetlinewidth{0.000000pt}%
\definecolor{currentstroke}{rgb}{0.000000,0.000000,0.000000}%
\pgfsetstrokecolor{currentstroke}%
\pgfsetdash{}{0pt}%
\pgfpathmoveto{\pgfqpoint{4.906535in}{2.432956in}}%
\pgfpathlineto{\pgfqpoint{4.920904in}{2.446863in}}%
\pgfpathlineto{\pgfqpoint{4.935293in}{2.460934in}}%
\pgfpathlineto{\pgfqpoint{4.949702in}{2.475169in}}%
\pgfpathlineto{\pgfqpoint{4.964130in}{2.489570in}}%
\pgfpathlineto{\pgfqpoint{4.971954in}{2.505815in}}%
\pgfpathlineto{\pgfqpoint{4.979772in}{2.521903in}}%
\pgfpathlineto{\pgfqpoint{4.987583in}{2.537831in}}%
\pgfpathlineto{\pgfqpoint{4.995388in}{2.553599in}}%
\pgfpathlineto{\pgfqpoint{4.980950in}{2.538929in}}%
\pgfpathlineto{\pgfqpoint{4.966532in}{2.524425in}}%
\pgfpathlineto{\pgfqpoint{4.952134in}{2.510086in}}%
\pgfpathlineto{\pgfqpoint{4.937755in}{2.495911in}}%
\pgfpathlineto{\pgfqpoint{4.929959in}{2.480400in}}%
\pgfpathlineto{\pgfqpoint{4.922157in}{2.464736in}}%
\pgfpathlineto{\pgfqpoint{4.914349in}{2.448921in}}%
\pgfpathlineto{\pgfqpoint{4.906535in}{2.432956in}}%
\pgfpathclose%
\pgfusepath{fill}%
\end{pgfscope}%
\begin{pgfscope}%
\pgfpathrectangle{\pgfqpoint{1.254980in}{0.150000in}}{\pgfqpoint{5.490039in}{5.490039in}}%
\pgfusepath{clip}%
\pgfsetbuttcap%
\pgfsetroundjoin%
\definecolor{currentfill}{rgb}{0.203063,0.379716,0.553925}%
\pgfsetfillcolor{currentfill}%
\pgfsetfillopacity{0.700000}%
\pgfsetlinewidth{0.000000pt}%
\definecolor{currentstroke}{rgb}{0.000000,0.000000,0.000000}%
\pgfsetstrokecolor{currentstroke}%
\pgfsetdash{}{0pt}%
\pgfpathmoveto{\pgfqpoint{4.483863in}{1.745247in}}%
\pgfpathlineto{\pgfqpoint{4.497946in}{1.754030in}}%
\pgfpathlineto{\pgfqpoint{4.512044in}{1.762971in}}%
\pgfpathlineto{\pgfqpoint{4.526158in}{1.772068in}}%
\pgfpathlineto{\pgfqpoint{4.540287in}{1.781323in}}%
\pgfpathlineto{\pgfqpoint{4.548221in}{1.799368in}}%
\pgfpathlineto{\pgfqpoint{4.556152in}{1.817392in}}%
\pgfpathlineto{\pgfqpoint{4.564079in}{1.835389in}}%
\pgfpathlineto{\pgfqpoint{4.572003in}{1.853355in}}%
\pgfpathlineto{\pgfqpoint{4.557864in}{1.843601in}}%
\pgfpathlineto{\pgfqpoint{4.543742in}{1.834004in}}%
\pgfpathlineto{\pgfqpoint{4.529634in}{1.824565in}}%
\pgfpathlineto{\pgfqpoint{4.515543in}{1.815284in}}%
\pgfpathlineto{\pgfqpoint{4.507628in}{1.797806in}}%
\pgfpathlineto{\pgfqpoint{4.499709in}{1.780304in}}%
\pgfpathlineto{\pgfqpoint{4.491788in}{1.762783in}}%
\pgfpathlineto{\pgfqpoint{4.483863in}{1.745247in}}%
\pgfpathclose%
\pgfusepath{fill}%
\end{pgfscope}%
\begin{pgfscope}%
\pgfpathrectangle{\pgfqpoint{1.254980in}{0.150000in}}{\pgfqpoint{5.490039in}{5.490039in}}%
\pgfusepath{clip}%
\pgfsetbuttcap%
\pgfsetroundjoin%
\definecolor{currentfill}{rgb}{0.136408,0.541173,0.554483}%
\pgfsetfillcolor{currentfill}%
\pgfsetfillopacity{0.700000}%
\pgfsetlinewidth{0.000000pt}%
\definecolor{currentstroke}{rgb}{0.000000,0.000000,0.000000}%
\pgfsetstrokecolor{currentstroke}%
\pgfsetdash{}{0pt}%
\pgfpathmoveto{\pgfqpoint{4.755167in}{2.180627in}}%
\pgfpathlineto{\pgfqpoint{4.769431in}{2.192898in}}%
\pgfpathlineto{\pgfqpoint{4.783714in}{2.205331in}}%
\pgfpathlineto{\pgfqpoint{4.798015in}{2.217926in}}%
\pgfpathlineto{\pgfqpoint{4.812334in}{2.230683in}}%
\pgfpathlineto{\pgfqpoint{4.820212in}{2.248231in}}%
\pgfpathlineto{\pgfqpoint{4.828086in}{2.265663in}}%
\pgfpathlineto{\pgfqpoint{4.835954in}{2.282976in}}%
\pgfpathlineto{\pgfqpoint{4.843818in}{2.300168in}}%
\pgfpathlineto{\pgfqpoint{4.829488in}{2.287052in}}%
\pgfpathlineto{\pgfqpoint{4.815176in}{2.274099in}}%
\pgfpathlineto{\pgfqpoint{4.800883in}{2.261308in}}%
\pgfpathlineto{\pgfqpoint{4.786608in}{2.248679in}}%
\pgfpathlineto{\pgfqpoint{4.778755in}{2.231835in}}%
\pgfpathlineto{\pgfqpoint{4.770897in}{2.214876in}}%
\pgfpathlineto{\pgfqpoint{4.763034in}{2.197805in}}%
\pgfpathlineto{\pgfqpoint{4.755167in}{2.180627in}}%
\pgfpathclose%
\pgfusepath{fill}%
\end{pgfscope}%
\begin{pgfscope}%
\pgfpathrectangle{\pgfqpoint{1.254980in}{0.150000in}}{\pgfqpoint{5.490039in}{5.490039in}}%
\pgfusepath{clip}%
\pgfsetbuttcap%
\pgfsetroundjoin%
\definecolor{currentfill}{rgb}{0.175707,0.697900,0.491033}%
\pgfsetfillcolor{currentfill}%
\pgfsetfillopacity{0.700000}%
\pgfsetlinewidth{0.000000pt}%
\definecolor{currentstroke}{rgb}{0.000000,0.000000,0.000000}%
\pgfsetstrokecolor{currentstroke}%
\pgfsetdash{}{0pt}%
\pgfpathmoveto{\pgfqpoint{5.026541in}{2.615011in}}%
\pgfpathlineto{\pgfqpoint{5.041008in}{2.630084in}}%
\pgfpathlineto{\pgfqpoint{5.055496in}{2.645323in}}%
\pgfpathlineto{\pgfqpoint{5.070004in}{2.660729in}}%
\pgfpathlineto{\pgfqpoint{5.084533in}{2.676302in}}%
\pgfpathlineto{\pgfqpoint{5.092312in}{2.691450in}}%
\pgfpathlineto{\pgfqpoint{5.100085in}{2.706417in}}%
\pgfpathlineto{\pgfqpoint{5.107850in}{2.721201in}}%
\pgfpathlineto{\pgfqpoint{5.115607in}{2.735800in}}%
\pgfpathlineto{\pgfqpoint{5.101070in}{2.720020in}}%
\pgfpathlineto{\pgfqpoint{5.086554in}{2.704407in}}%
\pgfpathlineto{\pgfqpoint{5.072059in}{2.688961in}}%
\pgfpathlineto{\pgfqpoint{5.057584in}{2.673681in}}%
\pgfpathlineto{\pgfqpoint{5.049834in}{2.659277in}}%
\pgfpathlineto{\pgfqpoint{5.042077in}{2.644696in}}%
\pgfpathlineto{\pgfqpoint{5.034313in}{2.629940in}}%
\pgfpathlineto{\pgfqpoint{5.026541in}{2.615011in}}%
\pgfpathclose%
\pgfusepath{fill}%
\end{pgfscope}%
\begin{pgfscope}%
\pgfpathrectangle{\pgfqpoint{1.254980in}{0.150000in}}{\pgfqpoint{5.490039in}{5.490039in}}%
\pgfusepath{clip}%
\pgfsetbuttcap%
\pgfsetroundjoin%
\definecolor{currentfill}{rgb}{0.172719,0.448791,0.557885}%
\pgfsetfillcolor{currentfill}%
\pgfsetfillopacity{0.700000}%
\pgfsetlinewidth{0.000000pt}%
\definecolor{currentstroke}{rgb}{0.000000,0.000000,0.000000}%
\pgfsetstrokecolor{currentstroke}%
\pgfsetdash{}{0pt}%
\pgfpathmoveto{\pgfqpoint{4.603665in}{1.924821in}}%
\pgfpathlineto{\pgfqpoint{4.617830in}{1.935206in}}%
\pgfpathlineto{\pgfqpoint{4.632011in}{1.945749in}}%
\pgfpathlineto{\pgfqpoint{4.646209in}{1.956452in}}%
\pgfpathlineto{\pgfqpoint{4.660423in}{1.967313in}}%
\pgfpathlineto{\pgfqpoint{4.668340in}{1.985514in}}%
\pgfpathlineto{\pgfqpoint{4.676254in}{2.003650in}}%
\pgfpathlineto{\pgfqpoint{4.684163in}{2.021718in}}%
\pgfpathlineto{\pgfqpoint{4.692069in}{2.039714in}}%
\pgfpathlineto{\pgfqpoint{4.677843in}{2.028407in}}%
\pgfpathlineto{\pgfqpoint{4.663635in}{2.017260in}}%
\pgfpathlineto{\pgfqpoint{4.649444in}{2.006272in}}%
\pgfpathlineto{\pgfqpoint{4.635269in}{1.995444in}}%
\pgfpathlineto{\pgfqpoint{4.627374in}{1.977882in}}%
\pgfpathlineto{\pgfqpoint{4.619474in}{1.960254in}}%
\pgfpathlineto{\pgfqpoint{4.611572in}{1.942566in}}%
\pgfpathlineto{\pgfqpoint{4.603665in}{1.924821in}}%
\pgfpathclose%
\pgfusepath{fill}%
\end{pgfscope}%
\begin{pgfscope}%
\pgfpathrectangle{\pgfqpoint{1.254980in}{0.150000in}}{\pgfqpoint{5.490039in}{5.490039in}}%
\pgfusepath{clip}%
\pgfsetbuttcap%
\pgfsetroundjoin%
\definecolor{currentfill}{rgb}{0.246811,0.283237,0.535941}%
\pgfsetfillcolor{currentfill}%
\pgfsetfillopacity{0.700000}%
\pgfsetlinewidth{0.000000pt}%
\definecolor{currentstroke}{rgb}{0.000000,0.000000,0.000000}%
\pgfsetstrokecolor{currentstroke}%
\pgfsetdash{}{0pt}%
\pgfpathmoveto{\pgfqpoint{4.332415in}{1.508490in}}%
\pgfpathlineto{\pgfqpoint{4.346420in}{1.514994in}}%
\pgfpathlineto{\pgfqpoint{4.360439in}{1.521652in}}%
\pgfpathlineto{\pgfqpoint{4.374472in}{1.528465in}}%
\pgfpathlineto{\pgfqpoint{4.388518in}{1.535432in}}%
\pgfpathlineto{\pgfqpoint{4.396480in}{1.552761in}}%
\pgfpathlineto{\pgfqpoint{4.404439in}{1.570136in}}%
\pgfpathlineto{\pgfqpoint{4.412395in}{1.587551in}}%
\pgfpathlineto{\pgfqpoint{4.420349in}{1.605001in}}%
\pgfpathlineto{\pgfqpoint{4.406296in}{1.597453in}}%
\pgfpathlineto{\pgfqpoint{4.392257in}{1.590060in}}%
\pgfpathlineto{\pgfqpoint{4.378232in}{1.582822in}}%
\pgfpathlineto{\pgfqpoint{4.364221in}{1.575740in}}%
\pgfpathlineto{\pgfqpoint{4.356274in}{1.558859in}}%
\pgfpathlineto{\pgfqpoint{4.348324in}{1.542020in}}%
\pgfpathlineto{\pgfqpoint{4.340371in}{1.525228in}}%
\pgfpathlineto{\pgfqpoint{4.332415in}{1.508490in}}%
\pgfpathclose%
\pgfusepath{fill}%
\end{pgfscope}%
\begin{pgfscope}%
\pgfpathrectangle{\pgfqpoint{1.254980in}{0.150000in}}{\pgfqpoint{5.490039in}{5.490039in}}%
\pgfusepath{clip}%
\pgfsetbuttcap%
\pgfsetroundjoin%
\definecolor{currentfill}{rgb}{0.270595,0.214069,0.507052}%
\pgfsetfillcolor{currentfill}%
\pgfsetfillopacity{0.700000}%
\pgfsetlinewidth{0.000000pt}%
\definecolor{currentstroke}{rgb}{0.000000,0.000000,0.000000}%
\pgfsetstrokecolor{currentstroke}%
\pgfsetdash{}{0pt}%
\pgfpathmoveto{\pgfqpoint{4.212783in}{1.357631in}}%
\pgfpathlineto{\pgfqpoint{4.226731in}{1.362284in}}%
\pgfpathlineto{\pgfqpoint{4.240691in}{1.367090in}}%
\pgfpathlineto{\pgfqpoint{4.254664in}{1.372049in}}%
\pgfpathlineto{\pgfqpoint{4.268648in}{1.377160in}}%
\pgfpathlineto{\pgfqpoint{4.276631in}{1.393271in}}%
\pgfpathlineto{\pgfqpoint{4.284609in}{1.409482in}}%
\pgfpathlineto{\pgfqpoint{4.292585in}{1.425785in}}%
\pgfpathlineto{\pgfqpoint{4.300557in}{1.442175in}}%
\pgfpathlineto{\pgfqpoint{4.286570in}{1.436431in}}%
\pgfpathlineto{\pgfqpoint{4.272595in}{1.430840in}}%
\pgfpathlineto{\pgfqpoint{4.258632in}{1.425403in}}%
\pgfpathlineto{\pgfqpoint{4.244682in}{1.420119in}}%
\pgfpathlineto{\pgfqpoint{4.236713in}{1.404351in}}%
\pgfpathlineto{\pgfqpoint{4.228740in}{1.388676in}}%
\pgfpathlineto{\pgfqpoint{4.220764in}{1.373101in}}%
\pgfpathlineto{\pgfqpoint{4.212783in}{1.357631in}}%
\pgfpathclose%
\pgfusepath{fill}%
\end{pgfscope}%
\begin{pgfscope}%
\pgfpathrectangle{\pgfqpoint{1.254980in}{0.150000in}}{\pgfqpoint{5.490039in}{5.490039in}}%
\pgfusepath{clip}%
\pgfsetbuttcap%
\pgfsetroundjoin%
\definecolor{currentfill}{rgb}{0.283197,0.115680,0.436115}%
\pgfsetfillcolor{currentfill}%
\pgfsetfillopacity{0.700000}%
\pgfsetlinewidth{0.000000pt}%
\definecolor{currentstroke}{rgb}{0.000000,0.000000,0.000000}%
\pgfsetstrokecolor{currentstroke}%
\pgfsetdash{}{0pt}%
\pgfpathmoveto{\pgfqpoint{4.005492in}{1.167194in}}%
\pgfpathlineto{\pgfqpoint{4.019361in}{1.168582in}}%
\pgfpathlineto{\pgfqpoint{4.033240in}{1.170122in}}%
\pgfpathlineto{\pgfqpoint{4.047129in}{1.171813in}}%
\pgfpathlineto{\pgfqpoint{4.061027in}{1.173655in}}%
\pgfpathlineto{\pgfqpoint{4.069059in}{1.186639in}}%
\pgfpathlineto{\pgfqpoint{4.077086in}{1.199813in}}%
\pgfpathlineto{\pgfqpoint{4.085108in}{1.213171in}}%
\pgfpathlineto{\pgfqpoint{4.093125in}{1.226706in}}%
\pgfpathlineto{\pgfqpoint{4.079231in}{1.224154in}}%
\pgfpathlineto{\pgfqpoint{4.065348in}{1.221754in}}%
\pgfpathlineto{\pgfqpoint{4.051474in}{1.219507in}}%
\pgfpathlineto{\pgfqpoint{4.037611in}{1.217411in}}%
\pgfpathlineto{\pgfqpoint{4.029589in}{1.204574in}}%
\pgfpathlineto{\pgfqpoint{4.021562in}{1.191922in}}%
\pgfpathlineto{\pgfqpoint{4.013529in}{1.179459in}}%
\pgfpathlineto{\pgfqpoint{4.005492in}{1.167194in}}%
\pgfpathclose%
\pgfusepath{fill}%
\end{pgfscope}%
\begin{pgfscope}%
\pgfpathrectangle{\pgfqpoint{1.254980in}{0.150000in}}{\pgfqpoint{5.490039in}{5.490039in}}%
\pgfusepath{clip}%
\pgfsetbuttcap%
\pgfsetroundjoin%
\definecolor{currentfill}{rgb}{0.555484,0.840254,0.269281}%
\pgfsetfillcolor{currentfill}%
\pgfsetfillopacity{0.700000}%
\pgfsetlinewidth{0.000000pt}%
\definecolor{currentstroke}{rgb}{0.000000,0.000000,0.000000}%
\pgfsetstrokecolor{currentstroke}%
\pgfsetdash{}{0pt}%
\pgfpathmoveto{\pgfqpoint{5.386273in}{3.124036in}}%
\pgfpathlineto{\pgfqpoint{5.401033in}{3.141897in}}%
\pgfpathlineto{\pgfqpoint{5.415816in}{3.159928in}}%
\pgfpathlineto{\pgfqpoint{5.430622in}{3.178131in}}%
\pgfpathlineto{\pgfqpoint{5.438200in}{3.188697in}}%
\pgfpathlineto{\pgfqpoint{5.445767in}{3.199041in}}%
\pgfpathlineto{\pgfqpoint{5.453323in}{3.209163in}}%
\pgfpathlineto{\pgfqpoint{5.460868in}{3.219063in}}%
\pgfpathlineto{\pgfqpoint{5.446061in}{3.200846in}}%
\pgfpathlineto{\pgfqpoint{5.431278in}{3.182801in}}%
\pgfpathlineto{\pgfqpoint{5.416519in}{3.164927in}}%
\pgfpathlineto{\pgfqpoint{5.408973in}{3.155028in}}%
\pgfpathlineto{\pgfqpoint{5.401417in}{3.144913in}}%
\pgfpathlineto{\pgfqpoint{5.393850in}{3.134583in}}%
\pgfpathlineto{\pgfqpoint{5.386273in}{3.124036in}}%
\pgfpathclose%
\pgfusepath{fill}%
\end{pgfscope}%
\begin{pgfscope}%
\pgfpathrectangle{\pgfqpoint{1.254980in}{0.150000in}}{\pgfqpoint{5.490039in}{5.490039in}}%
\pgfusepath{clip}%
\pgfsetbuttcap%
\pgfsetroundjoin%
\definecolor{currentfill}{rgb}{0.214298,0.355619,0.551184}%
\pgfsetfillcolor{currentfill}%
\pgfsetfillopacity{0.700000}%
\pgfsetlinewidth{0.000000pt}%
\definecolor{currentstroke}{rgb}{0.000000,0.000000,0.000000}%
\pgfsetstrokecolor{currentstroke}%
\pgfsetdash{}{0pt}%
\pgfpathmoveto{\pgfqpoint{4.452131in}{1.675049in}}%
\pgfpathlineto{\pgfqpoint{4.466206in}{1.683307in}}%
\pgfpathlineto{\pgfqpoint{4.480295in}{1.691720in}}%
\pgfpathlineto{\pgfqpoint{4.494400in}{1.700290in}}%
\pgfpathlineto{\pgfqpoint{4.508520in}{1.709016in}}%
\pgfpathlineto{\pgfqpoint{4.516467in}{1.727102in}}%
\pgfpathlineto{\pgfqpoint{4.524410in}{1.745185in}}%
\pgfpathlineto{\pgfqpoint{4.532350in}{1.763260in}}%
\pgfpathlineto{\pgfqpoint{4.540287in}{1.781323in}}%
\pgfpathlineto{\pgfqpoint{4.526158in}{1.772068in}}%
\pgfpathlineto{\pgfqpoint{4.512044in}{1.762971in}}%
\pgfpathlineto{\pgfqpoint{4.497946in}{1.754030in}}%
\pgfpathlineto{\pgfqpoint{4.483863in}{1.745247in}}%
\pgfpathlineto{\pgfqpoint{4.475935in}{1.727701in}}%
\pgfpathlineto{\pgfqpoint{4.468003in}{1.710149in}}%
\pgfpathlineto{\pgfqpoint{4.460068in}{1.692597in}}%
\pgfpathlineto{\pgfqpoint{4.452131in}{1.675049in}}%
\pgfpathclose%
\pgfusepath{fill}%
\end{pgfscope}%
\begin{pgfscope}%
\pgfpathrectangle{\pgfqpoint{1.254980in}{0.150000in}}{\pgfqpoint{5.490039in}{5.490039in}}%
\pgfusepath{clip}%
\pgfsetbuttcap%
\pgfsetroundjoin%
\definecolor{currentfill}{rgb}{0.281477,0.755203,0.432552}%
\pgfsetfillcolor{currentfill}%
\pgfsetfillopacity{0.700000}%
\pgfsetlinewidth{0.000000pt}%
\definecolor{currentstroke}{rgb}{0.000000,0.000000,0.000000}%
\pgfsetstrokecolor{currentstroke}%
\pgfsetdash{}{0pt}%
\pgfpathmoveto{\pgfqpoint{5.146559in}{2.792324in}}%
\pgfpathlineto{\pgfqpoint{5.161124in}{2.808448in}}%
\pgfpathlineto{\pgfqpoint{5.175711in}{2.824739in}}%
\pgfpathlineto{\pgfqpoint{5.190320in}{2.841199in}}%
\pgfpathlineto{\pgfqpoint{5.204950in}{2.857827in}}%
\pgfpathlineto{\pgfqpoint{5.212675in}{2.871639in}}%
\pgfpathlineto{\pgfqpoint{5.220391in}{2.885250in}}%
\pgfpathlineto{\pgfqpoint{5.228098in}{2.898660in}}%
\pgfpathlineto{\pgfqpoint{5.235797in}{2.911867in}}%
\pgfpathlineto{\pgfqpoint{5.221160in}{2.895094in}}%
\pgfpathlineto{\pgfqpoint{5.206546in}{2.878490in}}%
\pgfpathlineto{\pgfqpoint{5.191953in}{2.862054in}}%
\pgfpathlineto{\pgfqpoint{5.177382in}{2.845787in}}%
\pgfpathlineto{\pgfqpoint{5.169689in}{2.832712in}}%
\pgfpathlineto{\pgfqpoint{5.161987in}{2.819443in}}%
\pgfpathlineto{\pgfqpoint{5.154277in}{2.805980in}}%
\pgfpathlineto{\pgfqpoint{5.146559in}{2.792324in}}%
\pgfpathclose%
\pgfusepath{fill}%
\end{pgfscope}%
\begin{pgfscope}%
\pgfpathrectangle{\pgfqpoint{1.254980in}{0.150000in}}{\pgfqpoint{5.490039in}{5.490039in}}%
\pgfusepath{clip}%
\pgfsetbuttcap%
\pgfsetroundjoin%
\definecolor{currentfill}{rgb}{0.119423,0.611141,0.538982}%
\pgfsetfillcolor{currentfill}%
\pgfsetfillopacity{0.700000}%
\pgfsetlinewidth{0.000000pt}%
\definecolor{currentstroke}{rgb}{0.000000,0.000000,0.000000}%
\pgfsetstrokecolor{currentstroke}%
\pgfsetdash{}{0pt}%
\pgfpathmoveto{\pgfqpoint{4.875220in}{2.367653in}}%
\pgfpathlineto{\pgfqpoint{4.889580in}{2.381262in}}%
\pgfpathlineto{\pgfqpoint{4.903959in}{2.395033in}}%
\pgfpathlineto{\pgfqpoint{4.918357in}{2.408969in}}%
\pgfpathlineto{\pgfqpoint{4.932774in}{2.423070in}}%
\pgfpathlineto{\pgfqpoint{4.940622in}{2.439918in}}%
\pgfpathlineto{\pgfqpoint{4.948464in}{2.456619in}}%
\pgfpathlineto{\pgfqpoint{4.956300in}{2.473170in}}%
\pgfpathlineto{\pgfqpoint{4.964130in}{2.489570in}}%
\pgfpathlineto{\pgfqpoint{4.949702in}{2.475169in}}%
\pgfpathlineto{\pgfqpoint{4.935293in}{2.460934in}}%
\pgfpathlineto{\pgfqpoint{4.920904in}{2.446863in}}%
\pgfpathlineto{\pgfqpoint{4.906535in}{2.432956in}}%
\pgfpathlineto{\pgfqpoint{4.898715in}{2.416844in}}%
\pgfpathlineto{\pgfqpoint{4.890889in}{2.400588in}}%
\pgfpathlineto{\pgfqpoint{4.883057in}{2.384190in}}%
\pgfpathlineto{\pgfqpoint{4.875220in}{2.367653in}}%
\pgfpathclose%
\pgfusepath{fill}%
\end{pgfscope}%
\begin{pgfscope}%
\pgfpathrectangle{\pgfqpoint{1.254980in}{0.150000in}}{\pgfqpoint{5.490039in}{5.490039in}}%
\pgfusepath{clip}%
\pgfsetbuttcap%
\pgfsetroundjoin%
\definecolor{currentfill}{rgb}{0.281887,0.150881,0.465405}%
\pgfsetfillcolor{currentfill}%
\pgfsetfillopacity{0.700000}%
\pgfsetlinewidth{0.000000pt}%
\definecolor{currentstroke}{rgb}{0.000000,0.000000,0.000000}%
\pgfsetstrokecolor{currentstroke}%
\pgfsetdash{}{0pt}%
\pgfpathmoveto{\pgfqpoint{4.093125in}{1.226706in}}%
\pgfpathlineto{\pgfqpoint{4.107030in}{1.229409in}}%
\pgfpathlineto{\pgfqpoint{4.120945in}{1.232264in}}%
\pgfpathlineto{\pgfqpoint{4.134871in}{1.235270in}}%
\pgfpathlineto{\pgfqpoint{4.148808in}{1.238428in}}%
\pgfpathlineto{\pgfqpoint{4.156819in}{1.252827in}}%
\pgfpathlineto{\pgfqpoint{4.164825in}{1.267382in}}%
\pgfpathlineto{\pgfqpoint{4.172828in}{1.282086in}}%
\pgfpathlineto{\pgfqpoint{4.180826in}{1.296934in}}%
\pgfpathlineto{\pgfqpoint{4.166891in}{1.293092in}}%
\pgfpathlineto{\pgfqpoint{4.152966in}{1.289401in}}%
\pgfpathlineto{\pgfqpoint{4.139053in}{1.285863in}}%
\pgfpathlineto{\pgfqpoint{4.125151in}{1.282477in}}%
\pgfpathlineto{\pgfqpoint{4.117151in}{1.268302in}}%
\pgfpathlineto{\pgfqpoint{4.109147in}{1.254278in}}%
\pgfpathlineto{\pgfqpoint{4.101138in}{1.240410in}}%
\pgfpathlineto{\pgfqpoint{4.093125in}{1.226706in}}%
\pgfpathclose%
\pgfusepath{fill}%
\end{pgfscope}%
\begin{pgfscope}%
\pgfpathrectangle{\pgfqpoint{1.254980in}{0.150000in}}{\pgfqpoint{5.490039in}{5.490039in}}%
\pgfusepath{clip}%
\pgfsetbuttcap%
\pgfsetroundjoin%
\definecolor{currentfill}{rgb}{0.421908,0.805774,0.351910}%
\pgfsetfillcolor{currentfill}%
\pgfsetfillopacity{0.700000}%
\pgfsetlinewidth{0.000000pt}%
\definecolor{currentstroke}{rgb}{0.000000,0.000000,0.000000}%
\pgfsetstrokecolor{currentstroke}%
\pgfsetdash{}{0pt}%
\pgfpathmoveto{\pgfqpoint{5.266502in}{2.962651in}}%
\pgfpathlineto{\pgfqpoint{5.281166in}{2.979705in}}%
\pgfpathlineto{\pgfqpoint{5.295852in}{2.996928in}}%
\pgfpathlineto{\pgfqpoint{5.310560in}{3.014322in}}%
\pgfpathlineto{\pgfqpoint{5.325291in}{3.031885in}}%
\pgfpathlineto{\pgfqpoint{5.332949in}{3.044159in}}%
\pgfpathlineto{\pgfqpoint{5.340596in}{3.056218in}}%
\pgfpathlineto{\pgfqpoint{5.348234in}{3.068061in}}%
\pgfpathlineto{\pgfqpoint{5.355862in}{3.079688in}}%
\pgfpathlineto{\pgfqpoint{5.341128in}{3.062045in}}%
\pgfpathlineto{\pgfqpoint{5.326416in}{3.044572in}}%
\pgfpathlineto{\pgfqpoint{5.311727in}{3.027269in}}%
\pgfpathlineto{\pgfqpoint{5.297060in}{3.010136in}}%
\pgfpathlineto{\pgfqpoint{5.289435in}{2.998576in}}%
\pgfpathlineto{\pgfqpoint{5.281800in}{2.986808in}}%
\pgfpathlineto{\pgfqpoint{5.274156in}{2.974833in}}%
\pgfpathlineto{\pgfqpoint{5.266502in}{2.962651in}}%
\pgfpathclose%
\pgfusepath{fill}%
\end{pgfscope}%
\begin{pgfscope}%
\pgfpathrectangle{\pgfqpoint{1.254980in}{0.150000in}}{\pgfqpoint{5.490039in}{5.490039in}}%
\pgfusepath{clip}%
\pgfsetbuttcap%
\pgfsetroundjoin%
\definecolor{currentfill}{rgb}{0.143343,0.522773,0.556295}%
\pgfsetfillcolor{currentfill}%
\pgfsetfillopacity{0.700000}%
\pgfsetlinewidth{0.000000pt}%
\definecolor{currentstroke}{rgb}{0.000000,0.000000,0.000000}%
\pgfsetstrokecolor{currentstroke}%
\pgfsetdash{}{0pt}%
\pgfpathmoveto{\pgfqpoint{4.723652in}{2.110898in}}%
\pgfpathlineto{\pgfqpoint{4.737906in}{2.122783in}}%
\pgfpathlineto{\pgfqpoint{4.752177in}{2.134828in}}%
\pgfpathlineto{\pgfqpoint{4.766467in}{2.147034in}}%
\pgfpathlineto{\pgfqpoint{4.780775in}{2.159402in}}%
\pgfpathlineto{\pgfqpoint{4.788671in}{2.177379in}}%
\pgfpathlineto{\pgfqpoint{4.796563in}{2.195253in}}%
\pgfpathlineto{\pgfqpoint{4.804451in}{2.213023in}}%
\pgfpathlineto{\pgfqpoint{4.812334in}{2.230683in}}%
\pgfpathlineto{\pgfqpoint{4.798015in}{2.217926in}}%
\pgfpathlineto{\pgfqpoint{4.783714in}{2.205331in}}%
\pgfpathlineto{\pgfqpoint{4.769431in}{2.192898in}}%
\pgfpathlineto{\pgfqpoint{4.755167in}{2.180627in}}%
\pgfpathlineto{\pgfqpoint{4.747295in}{2.163344in}}%
\pgfpathlineto{\pgfqpoint{4.739418in}{2.145959in}}%
\pgfpathlineto{\pgfqpoint{4.731537in}{2.128476in}}%
\pgfpathlineto{\pgfqpoint{4.723652in}{2.110898in}}%
\pgfpathclose%
\pgfusepath{fill}%
\end{pgfscope}%
\begin{pgfscope}%
\pgfpathrectangle{\pgfqpoint{1.254980in}{0.150000in}}{\pgfqpoint{5.490039in}{5.490039in}}%
\pgfusepath{clip}%
\pgfsetbuttcap%
\pgfsetroundjoin%
\definecolor{currentfill}{rgb}{0.180629,0.429975,0.557282}%
\pgfsetfillcolor{currentfill}%
\pgfsetfillopacity{0.700000}%
\pgfsetlinewidth{0.000000pt}%
\definecolor{currentstroke}{rgb}{0.000000,0.000000,0.000000}%
\pgfsetstrokecolor{currentstroke}%
\pgfsetdash{}{0pt}%
\pgfpathmoveto{\pgfqpoint{4.572003in}{1.853355in}}%
\pgfpathlineto{\pgfqpoint{4.586158in}{1.863268in}}%
\pgfpathlineto{\pgfqpoint{4.600329in}{1.873338in}}%
\pgfpathlineto{\pgfqpoint{4.614516in}{1.883567in}}%
\pgfpathlineto{\pgfqpoint{4.628720in}{1.893955in}}%
\pgfpathlineto{\pgfqpoint{4.636651in}{1.912369in}}%
\pgfpathlineto{\pgfqpoint{4.644578in}{1.930737in}}%
\pgfpathlineto{\pgfqpoint{4.652503in}{1.949053in}}%
\pgfpathlineto{\pgfqpoint{4.660423in}{1.967313in}}%
\pgfpathlineto{\pgfqpoint{4.646209in}{1.956452in}}%
\pgfpathlineto{\pgfqpoint{4.632011in}{1.945749in}}%
\pgfpathlineto{\pgfqpoint{4.617830in}{1.935206in}}%
\pgfpathlineto{\pgfqpoint{4.603665in}{1.924821in}}%
\pgfpathlineto{\pgfqpoint{4.595755in}{1.907023in}}%
\pgfpathlineto{\pgfqpoint{4.587841in}{1.889177in}}%
\pgfpathlineto{\pgfqpoint{4.579924in}{1.871286in}}%
\pgfpathlineto{\pgfqpoint{4.572003in}{1.853355in}}%
\pgfpathclose%
\pgfusepath{fill}%
\end{pgfscope}%
\begin{pgfscope}%
\pgfpathrectangle{\pgfqpoint{1.254980in}{0.150000in}}{\pgfqpoint{5.490039in}{5.490039in}}%
\pgfusepath{clip}%
\pgfsetbuttcap%
\pgfsetroundjoin%
\definecolor{currentfill}{rgb}{0.255645,0.260703,0.528312}%
\pgfsetfillcolor{currentfill}%
\pgfsetfillopacity{0.700000}%
\pgfsetlinewidth{0.000000pt}%
\definecolor{currentstroke}{rgb}{0.000000,0.000000,0.000000}%
\pgfsetstrokecolor{currentstroke}%
\pgfsetdash{}{0pt}%
\pgfpathmoveto{\pgfqpoint{4.300557in}{1.442175in}}%
\pgfpathlineto{\pgfqpoint{4.314558in}{1.448073in}}%
\pgfpathlineto{\pgfqpoint{4.328572in}{1.454124in}}%
\pgfpathlineto{\pgfqpoint{4.342599in}{1.460329in}}%
\pgfpathlineto{\pgfqpoint{4.356640in}{1.466688in}}%
\pgfpathlineto{\pgfqpoint{4.364614in}{1.483777in}}%
\pgfpathlineto{\pgfqpoint{4.372585in}{1.500934in}}%
\pgfpathlineto{\pgfqpoint{4.380553in}{1.518155in}}%
\pgfpathlineto{\pgfqpoint{4.388518in}{1.535432in}}%
\pgfpathlineto{\pgfqpoint{4.374472in}{1.528465in}}%
\pgfpathlineto{\pgfqpoint{4.360439in}{1.521652in}}%
\pgfpathlineto{\pgfqpoint{4.346420in}{1.514994in}}%
\pgfpathlineto{\pgfqpoint{4.332415in}{1.508490in}}%
\pgfpathlineto{\pgfqpoint{4.324455in}{1.491809in}}%
\pgfpathlineto{\pgfqpoint{4.316493in}{1.475193in}}%
\pgfpathlineto{\pgfqpoint{4.308527in}{1.458647in}}%
\pgfpathlineto{\pgfqpoint{4.300557in}{1.442175in}}%
\pgfpathclose%
\pgfusepath{fill}%
\end{pgfscope}%
\begin{pgfscope}%
\pgfpathrectangle{\pgfqpoint{1.254980in}{0.150000in}}{\pgfqpoint{5.490039in}{5.490039in}}%
\pgfusepath{clip}%
\pgfsetbuttcap%
\pgfsetroundjoin%
\definecolor{currentfill}{rgb}{0.153894,0.680203,0.504172}%
\pgfsetfillcolor{currentfill}%
\pgfsetfillopacity{0.700000}%
\pgfsetlinewidth{0.000000pt}%
\definecolor{currentstroke}{rgb}{0.000000,0.000000,0.000000}%
\pgfsetstrokecolor{currentstroke}%
\pgfsetdash{}{0pt}%
\pgfpathmoveto{\pgfqpoint{4.995388in}{2.553599in}}%
\pgfpathlineto{\pgfqpoint{5.009846in}{2.568434in}}%
\pgfpathlineto{\pgfqpoint{5.024324in}{2.583434in}}%
\pgfpathlineto{\pgfqpoint{5.038823in}{2.598601in}}%
\pgfpathlineto{\pgfqpoint{5.053342in}{2.613934in}}%
\pgfpathlineto{\pgfqpoint{5.061150in}{2.629788in}}%
\pgfpathlineto{\pgfqpoint{5.068951in}{2.645469in}}%
\pgfpathlineto{\pgfqpoint{5.076746in}{2.660974in}}%
\pgfpathlineto{\pgfqpoint{5.084533in}{2.676302in}}%
\pgfpathlineto{\pgfqpoint{5.070004in}{2.660729in}}%
\pgfpathlineto{\pgfqpoint{5.055496in}{2.645323in}}%
\pgfpathlineto{\pgfqpoint{5.041008in}{2.630084in}}%
\pgfpathlineto{\pgfqpoint{5.026541in}{2.615011in}}%
\pgfpathlineto{\pgfqpoint{5.018763in}{2.599910in}}%
\pgfpathlineto{\pgfqpoint{5.010978in}{2.584640in}}%
\pgfpathlineto{\pgfqpoint{5.003186in}{2.569202in}}%
\pgfpathlineto{\pgfqpoint{4.995388in}{2.553599in}}%
\pgfpathclose%
\pgfusepath{fill}%
\end{pgfscope}%
\begin{pgfscope}%
\pgfpathrectangle{\pgfqpoint{1.254980in}{0.150000in}}{\pgfqpoint{5.490039in}{5.490039in}}%
\pgfusepath{clip}%
\pgfsetbuttcap%
\pgfsetroundjoin%
\definecolor{currentfill}{rgb}{0.276194,0.190074,0.493001}%
\pgfsetfillcolor{currentfill}%
\pgfsetfillopacity{0.700000}%
\pgfsetlinewidth{0.000000pt}%
\definecolor{currentstroke}{rgb}{0.000000,0.000000,0.000000}%
\pgfsetstrokecolor{currentstroke}%
\pgfsetdash{}{0pt}%
\pgfpathmoveto{\pgfqpoint{4.180826in}{1.296934in}}%
\pgfpathlineto{\pgfqpoint{4.194774in}{1.300929in}}%
\pgfpathlineto{\pgfqpoint{4.208733in}{1.305076in}}%
\pgfpathlineto{\pgfqpoint{4.222703in}{1.309375in}}%
\pgfpathlineto{\pgfqpoint{4.236686in}{1.313826in}}%
\pgfpathlineto{\pgfqpoint{4.244682in}{1.329480in}}%
\pgfpathlineto{\pgfqpoint{4.252674in}{1.345258in}}%
\pgfpathlineto{\pgfqpoint{4.260663in}{1.361153in}}%
\pgfpathlineto{\pgfqpoint{4.268648in}{1.377160in}}%
\pgfpathlineto{\pgfqpoint{4.254664in}{1.372049in}}%
\pgfpathlineto{\pgfqpoint{4.240691in}{1.367090in}}%
\pgfpathlineto{\pgfqpoint{4.226731in}{1.362284in}}%
\pgfpathlineto{\pgfqpoint{4.212783in}{1.357631in}}%
\pgfpathlineto{\pgfqpoint{4.204800in}{1.342274in}}%
\pgfpathlineto{\pgfqpoint{4.196812in}{1.327034in}}%
\pgfpathlineto{\pgfqpoint{4.188821in}{1.311919in}}%
\pgfpathlineto{\pgfqpoint{4.180826in}{1.296934in}}%
\pgfpathclose%
\pgfusepath{fill}%
\end{pgfscope}%
\begin{pgfscope}%
\pgfpathrectangle{\pgfqpoint{1.254980in}{0.150000in}}{\pgfqpoint{5.490039in}{5.490039in}}%
\pgfusepath{clip}%
\pgfsetbuttcap%
\pgfsetroundjoin%
\definecolor{currentfill}{rgb}{0.223925,0.334994,0.548053}%
\pgfsetfillcolor{currentfill}%
\pgfsetfillopacity{0.700000}%
\pgfsetlinewidth{0.000000pt}%
\definecolor{currentstroke}{rgb}{0.000000,0.000000,0.000000}%
\pgfsetstrokecolor{currentstroke}%
\pgfsetdash{}{0pt}%
\pgfpathmoveto{\pgfqpoint{4.420349in}{1.605001in}}%
\pgfpathlineto{\pgfqpoint{4.434416in}{1.612705in}}%
\pgfpathlineto{\pgfqpoint{4.448498in}{1.620564in}}%
\pgfpathlineto{\pgfqpoint{4.462594in}{1.628578in}}%
\pgfpathlineto{\pgfqpoint{4.476705in}{1.636748in}}%
\pgfpathlineto{\pgfqpoint{4.484663in}{1.654793in}}%
\pgfpathlineto{\pgfqpoint{4.492619in}{1.672857in}}%
\pgfpathlineto{\pgfqpoint{4.500571in}{1.690933in}}%
\pgfpathlineto{\pgfqpoint{4.508520in}{1.709016in}}%
\pgfpathlineto{\pgfqpoint{4.494400in}{1.700290in}}%
\pgfpathlineto{\pgfqpoint{4.480295in}{1.691720in}}%
\pgfpathlineto{\pgfqpoint{4.466206in}{1.683307in}}%
\pgfpathlineto{\pgfqpoint{4.452131in}{1.675049in}}%
\pgfpathlineto{\pgfqpoint{4.444190in}{1.657511in}}%
\pgfpathlineto{\pgfqpoint{4.436246in}{1.639986in}}%
\pgfpathlineto{\pgfqpoint{4.428299in}{1.622482in}}%
\pgfpathlineto{\pgfqpoint{4.420349in}{1.605001in}}%
\pgfpathclose%
\pgfusepath{fill}%
\end{pgfscope}%
\begin{pgfscope}%
\pgfpathrectangle{\pgfqpoint{1.254980in}{0.150000in}}{\pgfqpoint{5.490039in}{5.490039in}}%
\pgfusepath{clip}%
\pgfsetbuttcap%
\pgfsetroundjoin%
\definecolor{currentfill}{rgb}{0.121148,0.592739,0.544641}%
\pgfsetfillcolor{currentfill}%
\pgfsetfillopacity{0.700000}%
\pgfsetlinewidth{0.000000pt}%
\definecolor{currentstroke}{rgb}{0.000000,0.000000,0.000000}%
\pgfsetstrokecolor{currentstroke}%
\pgfsetdash{}{0pt}%
\pgfpathmoveto{\pgfqpoint{4.843818in}{2.300168in}}%
\pgfpathlineto{\pgfqpoint{4.858167in}{2.313447in}}%
\pgfpathlineto{\pgfqpoint{4.872535in}{2.326889in}}%
\pgfpathlineto{\pgfqpoint{4.886922in}{2.340494in}}%
\pgfpathlineto{\pgfqpoint{4.901328in}{2.354263in}}%
\pgfpathlineto{\pgfqpoint{4.909198in}{2.371671in}}%
\pgfpathlineto{\pgfqpoint{4.917062in}{2.388943in}}%
\pgfpathlineto{\pgfqpoint{4.924921in}{2.406077in}}%
\pgfpathlineto{\pgfqpoint{4.932774in}{2.423070in}}%
\pgfpathlineto{\pgfqpoint{4.918357in}{2.408969in}}%
\pgfpathlineto{\pgfqpoint{4.903959in}{2.395033in}}%
\pgfpathlineto{\pgfqpoint{4.889580in}{2.381262in}}%
\pgfpathlineto{\pgfqpoint{4.875220in}{2.367653in}}%
\pgfpathlineto{\pgfqpoint{4.867378in}{2.350980in}}%
\pgfpathlineto{\pgfqpoint{4.859530in}{2.334172in}}%
\pgfpathlineto{\pgfqpoint{4.851676in}{2.317234in}}%
\pgfpathlineto{\pgfqpoint{4.843818in}{2.300168in}}%
\pgfpathclose%
\pgfusepath{fill}%
\end{pgfscope}%
\begin{pgfscope}%
\pgfpathrectangle{\pgfqpoint{1.254980in}{0.150000in}}{\pgfqpoint{5.490039in}{5.490039in}}%
\pgfusepath{clip}%
\pgfsetbuttcap%
\pgfsetroundjoin%
\definecolor{currentfill}{rgb}{0.151918,0.500685,0.557587}%
\pgfsetfillcolor{currentfill}%
\pgfsetfillopacity{0.700000}%
\pgfsetlinewidth{0.000000pt}%
\definecolor{currentstroke}{rgb}{0.000000,0.000000,0.000000}%
\pgfsetstrokecolor{currentstroke}%
\pgfsetdash{}{0pt}%
\pgfpathmoveto{\pgfqpoint{4.692069in}{2.039714in}}%
\pgfpathlineto{\pgfqpoint{4.706312in}{2.051181in}}%
\pgfpathlineto{\pgfqpoint{4.720572in}{2.062809in}}%
\pgfpathlineto{\pgfqpoint{4.734850in}{2.074597in}}%
\pgfpathlineto{\pgfqpoint{4.749146in}{2.086546in}}%
\pgfpathlineto{\pgfqpoint{4.757059in}{2.104894in}}%
\pgfpathlineto{\pgfqpoint{4.764969in}{2.123156in}}%
\pgfpathlineto{\pgfqpoint{4.772874in}{2.141326in}}%
\pgfpathlineto{\pgfqpoint{4.780775in}{2.159402in}}%
\pgfpathlineto{\pgfqpoint{4.766467in}{2.147034in}}%
\pgfpathlineto{\pgfqpoint{4.752177in}{2.134828in}}%
\pgfpathlineto{\pgfqpoint{4.737906in}{2.122783in}}%
\pgfpathlineto{\pgfqpoint{4.723652in}{2.110898in}}%
\pgfpathlineto{\pgfqpoint{4.715762in}{2.093230in}}%
\pgfpathlineto{\pgfqpoint{4.707869in}{2.075474in}}%
\pgfpathlineto{\pgfqpoint{4.699971in}{2.057634in}}%
\pgfpathlineto{\pgfqpoint{4.692069in}{2.039714in}}%
\pgfpathclose%
\pgfusepath{fill}%
\end{pgfscope}%
\begin{pgfscope}%
\pgfpathrectangle{\pgfqpoint{1.254980in}{0.150000in}}{\pgfqpoint{5.490039in}{5.490039in}}%
\pgfusepath{clip}%
\pgfsetbuttcap%
\pgfsetroundjoin%
\definecolor{currentfill}{rgb}{0.259857,0.745492,0.444467}%
\pgfsetfillcolor{currentfill}%
\pgfsetfillopacity{0.700000}%
\pgfsetlinewidth{0.000000pt}%
\definecolor{currentstroke}{rgb}{0.000000,0.000000,0.000000}%
\pgfsetstrokecolor{currentstroke}%
\pgfsetdash{}{0pt}%
\pgfpathmoveto{\pgfqpoint{5.115607in}{2.735800in}}%
\pgfpathlineto{\pgfqpoint{5.130165in}{2.751748in}}%
\pgfpathlineto{\pgfqpoint{5.144745in}{2.767863in}}%
\pgfpathlineto{\pgfqpoint{5.159346in}{2.784146in}}%
\pgfpathlineto{\pgfqpoint{5.173969in}{2.800598in}}%
\pgfpathlineto{\pgfqpoint{5.181726in}{2.815200in}}%
\pgfpathlineto{\pgfqpoint{5.189476in}{2.829606in}}%
\pgfpathlineto{\pgfqpoint{5.197217in}{2.843816in}}%
\pgfpathlineto{\pgfqpoint{5.204950in}{2.857827in}}%
\pgfpathlineto{\pgfqpoint{5.190320in}{2.841199in}}%
\pgfpathlineto{\pgfqpoint{5.175711in}{2.824739in}}%
\pgfpathlineto{\pgfqpoint{5.161124in}{2.808448in}}%
\pgfpathlineto{\pgfqpoint{5.146559in}{2.792324in}}%
\pgfpathlineto{\pgfqpoint{5.138833in}{2.778477in}}%
\pgfpathlineto{\pgfqpoint{5.131099in}{2.764440in}}%
\pgfpathlineto{\pgfqpoint{5.123357in}{2.750214in}}%
\pgfpathlineto{\pgfqpoint{5.115607in}{2.735800in}}%
\pgfpathclose%
\pgfusepath{fill}%
\end{pgfscope}%
\begin{pgfscope}%
\pgfpathrectangle{\pgfqpoint{1.254980in}{0.150000in}}{\pgfqpoint{5.490039in}{5.490039in}}%
\pgfusepath{clip}%
\pgfsetbuttcap%
\pgfsetroundjoin%
\definecolor{currentfill}{rgb}{0.190631,0.407061,0.556089}%
\pgfsetfillcolor{currentfill}%
\pgfsetfillopacity{0.700000}%
\pgfsetlinewidth{0.000000pt}%
\definecolor{currentstroke}{rgb}{0.000000,0.000000,0.000000}%
\pgfsetstrokecolor{currentstroke}%
\pgfsetdash{}{0pt}%
\pgfpathmoveto{\pgfqpoint{4.540287in}{1.781323in}}%
\pgfpathlineto{\pgfqpoint{4.554432in}{1.790734in}}%
\pgfpathlineto{\pgfqpoint{4.568593in}{1.800303in}}%
\pgfpathlineto{\pgfqpoint{4.582769in}{1.810030in}}%
\pgfpathlineto{\pgfqpoint{4.596962in}{1.819914in}}%
\pgfpathlineto{\pgfqpoint{4.604906in}{1.838472in}}%
\pgfpathlineto{\pgfqpoint{4.612847in}{1.857001in}}%
\pgfpathlineto{\pgfqpoint{4.620785in}{1.875497in}}%
\pgfpathlineto{\pgfqpoint{4.628720in}{1.893955in}}%
\pgfpathlineto{\pgfqpoint{4.614516in}{1.883567in}}%
\pgfpathlineto{\pgfqpoint{4.600329in}{1.873338in}}%
\pgfpathlineto{\pgfqpoint{4.586158in}{1.863268in}}%
\pgfpathlineto{\pgfqpoint{4.572003in}{1.853355in}}%
\pgfpathlineto{\pgfqpoint{4.564079in}{1.835389in}}%
\pgfpathlineto{\pgfqpoint{4.556152in}{1.817392in}}%
\pgfpathlineto{\pgfqpoint{4.548221in}{1.799368in}}%
\pgfpathlineto{\pgfqpoint{4.540287in}{1.781323in}}%
\pgfpathclose%
\pgfusepath{fill}%
\end{pgfscope}%
\begin{pgfscope}%
\pgfpathrectangle{\pgfqpoint{1.254980in}{0.150000in}}{\pgfqpoint{5.490039in}{5.490039in}}%
\pgfusepath{clip}%
\pgfsetbuttcap%
\pgfsetroundjoin%
\definecolor{currentfill}{rgb}{0.535621,0.835785,0.281908}%
\pgfsetfillcolor{currentfill}%
\pgfsetfillopacity{0.700000}%
\pgfsetlinewidth{0.000000pt}%
\definecolor{currentstroke}{rgb}{0.000000,0.000000,0.000000}%
\pgfsetstrokecolor{currentstroke}%
\pgfsetdash{}{0pt}%
\pgfpathmoveto{\pgfqpoint{5.355862in}{3.079688in}}%
\pgfpathlineto{\pgfqpoint{5.370620in}{3.097502in}}%
\pgfpathlineto{\pgfqpoint{5.385401in}{3.115487in}}%
\pgfpathlineto{\pgfqpoint{5.400205in}{3.133643in}}%
\pgfpathlineto{\pgfqpoint{5.407825in}{3.145099in}}%
\pgfpathlineto{\pgfqpoint{5.415435in}{3.156332in}}%
\pgfpathlineto{\pgfqpoint{5.423034in}{3.167343in}}%
\pgfpathlineto{\pgfqpoint{5.430622in}{3.178131in}}%
\pgfpathlineto{\pgfqpoint{5.415816in}{3.159928in}}%
\pgfpathlineto{\pgfqpoint{5.401033in}{3.141897in}}%
\pgfpathlineto{\pgfqpoint{5.386273in}{3.124036in}}%
\pgfpathlineto{\pgfqpoint{5.378686in}{3.113274in}}%
\pgfpathlineto{\pgfqpoint{5.371088in}{3.102295in}}%
\pgfpathlineto{\pgfqpoint{5.363480in}{3.091100in}}%
\pgfpathlineto{\pgfqpoint{5.355862in}{3.079688in}}%
\pgfpathclose%
\pgfusepath{fill}%
\end{pgfscope}%
\begin{pgfscope}%
\pgfpathrectangle{\pgfqpoint{1.254980in}{0.150000in}}{\pgfqpoint{5.490039in}{5.490039in}}%
\pgfusepath{clip}%
\pgfsetbuttcap%
\pgfsetroundjoin%
\definecolor{currentfill}{rgb}{0.282884,0.135920,0.453427}%
\pgfsetfillcolor{currentfill}%
\pgfsetfillopacity{0.700000}%
\pgfsetlinewidth{0.000000pt}%
\definecolor{currentstroke}{rgb}{0.000000,0.000000,0.000000}%
\pgfsetstrokecolor{currentstroke}%
\pgfsetdash{}{0pt}%
\pgfpathmoveto{\pgfqpoint{4.061027in}{1.173655in}}%
\pgfpathlineto{\pgfqpoint{4.074936in}{1.175648in}}%
\pgfpathlineto{\pgfqpoint{4.088855in}{1.177792in}}%
\pgfpathlineto{\pgfqpoint{4.102784in}{1.180087in}}%
\pgfpathlineto{\pgfqpoint{4.116724in}{1.182533in}}%
\pgfpathlineto{\pgfqpoint{4.124751in}{1.196239in}}%
\pgfpathlineto{\pgfqpoint{4.132774in}{1.210127in}}%
\pgfpathlineto{\pgfqpoint{4.140793in}{1.224193in}}%
\pgfpathlineto{\pgfqpoint{4.148808in}{1.238428in}}%
\pgfpathlineto{\pgfqpoint{4.134871in}{1.235270in}}%
\pgfpathlineto{\pgfqpoint{4.120945in}{1.232264in}}%
\pgfpathlineto{\pgfqpoint{4.107030in}{1.229409in}}%
\pgfpathlineto{\pgfqpoint{4.093125in}{1.226706in}}%
\pgfpathlineto{\pgfqpoint{4.085108in}{1.213171in}}%
\pgfpathlineto{\pgfqpoint{4.077086in}{1.199813in}}%
\pgfpathlineto{\pgfqpoint{4.069059in}{1.186639in}}%
\pgfpathlineto{\pgfqpoint{4.061027in}{1.173655in}}%
\pgfpathclose%
\pgfusepath{fill}%
\end{pgfscope}%
\begin{pgfscope}%
\pgfpathrectangle{\pgfqpoint{1.254980in}{0.150000in}}{\pgfqpoint{5.490039in}{5.490039in}}%
\pgfusepath{clip}%
\pgfsetbuttcap%
\pgfsetroundjoin%
\definecolor{currentfill}{rgb}{0.395174,0.797475,0.367757}%
\pgfsetfillcolor{currentfill}%
\pgfsetfillopacity{0.700000}%
\pgfsetlinewidth{0.000000pt}%
\definecolor{currentstroke}{rgb}{0.000000,0.000000,0.000000}%
\pgfsetstrokecolor{currentstroke}%
\pgfsetdash{}{0pt}%
\pgfpathmoveto{\pgfqpoint{5.235797in}{2.911867in}}%
\pgfpathlineto{\pgfqpoint{5.250456in}{2.928808in}}%
\pgfpathlineto{\pgfqpoint{5.265137in}{2.945920in}}%
\pgfpathlineto{\pgfqpoint{5.279840in}{2.963201in}}%
\pgfpathlineto{\pgfqpoint{5.294566in}{2.980652in}}%
\pgfpathlineto{\pgfqpoint{5.302261in}{2.993780in}}%
\pgfpathlineto{\pgfqpoint{5.309947in}{3.006695in}}%
\pgfpathlineto{\pgfqpoint{5.317624in}{3.019397in}}%
\pgfpathlineto{\pgfqpoint{5.325291in}{3.031885in}}%
\pgfpathlineto{\pgfqpoint{5.310560in}{3.014322in}}%
\pgfpathlineto{\pgfqpoint{5.295852in}{2.996928in}}%
\pgfpathlineto{\pgfqpoint{5.281166in}{2.979705in}}%
\pgfpathlineto{\pgfqpoint{5.266502in}{2.962651in}}%
\pgfpathlineto{\pgfqpoint{5.258840in}{2.950263in}}%
\pgfpathlineto{\pgfqpoint{5.251168in}{2.937669in}}%
\pgfpathlineto{\pgfqpoint{5.243487in}{2.924870in}}%
\pgfpathlineto{\pgfqpoint{5.235797in}{2.911867in}}%
\pgfpathclose%
\pgfusepath{fill}%
\end{pgfscope}%
\begin{pgfscope}%
\pgfpathrectangle{\pgfqpoint{1.254980in}{0.150000in}}{\pgfqpoint{5.490039in}{5.490039in}}%
\pgfusepath{clip}%
\pgfsetbuttcap%
\pgfsetroundjoin%
\definecolor{currentfill}{rgb}{0.263663,0.237631,0.518762}%
\pgfsetfillcolor{currentfill}%
\pgfsetfillopacity{0.700000}%
\pgfsetlinewidth{0.000000pt}%
\definecolor{currentstroke}{rgb}{0.000000,0.000000,0.000000}%
\pgfsetstrokecolor{currentstroke}%
\pgfsetdash{}{0pt}%
\pgfpathmoveto{\pgfqpoint{4.268648in}{1.377160in}}%
\pgfpathlineto{\pgfqpoint{4.282646in}{1.382424in}}%
\pgfpathlineto{\pgfqpoint{4.296655in}{1.387841in}}%
\pgfpathlineto{\pgfqpoint{4.310678in}{1.393411in}}%
\pgfpathlineto{\pgfqpoint{4.324714in}{1.399134in}}%
\pgfpathlineto{\pgfqpoint{4.332700in}{1.415890in}}%
\pgfpathlineto{\pgfqpoint{4.340683in}{1.432739in}}%
\pgfpathlineto{\pgfqpoint{4.348663in}{1.449673in}}%
\pgfpathlineto{\pgfqpoint{4.356640in}{1.466688in}}%
\pgfpathlineto{\pgfqpoint{4.342599in}{1.460329in}}%
\pgfpathlineto{\pgfqpoint{4.328572in}{1.454124in}}%
\pgfpathlineto{\pgfqpoint{4.314558in}{1.448073in}}%
\pgfpathlineto{\pgfqpoint{4.300557in}{1.442175in}}%
\pgfpathlineto{\pgfqpoint{4.292585in}{1.425785in}}%
\pgfpathlineto{\pgfqpoint{4.284609in}{1.409482in}}%
\pgfpathlineto{\pgfqpoint{4.276631in}{1.393271in}}%
\pgfpathlineto{\pgfqpoint{4.268648in}{1.377160in}}%
\pgfpathclose%
\pgfusepath{fill}%
\end{pgfscope}%
\begin{pgfscope}%
\pgfpathrectangle{\pgfqpoint{1.254980in}{0.150000in}}{\pgfqpoint{5.490039in}{5.490039in}}%
\pgfusepath{clip}%
\pgfsetbuttcap%
\pgfsetroundjoin%
\definecolor{currentfill}{rgb}{0.140210,0.665859,0.513427}%
\pgfsetfillcolor{currentfill}%
\pgfsetfillopacity{0.700000}%
\pgfsetlinewidth{0.000000pt}%
\definecolor{currentstroke}{rgb}{0.000000,0.000000,0.000000}%
\pgfsetstrokecolor{currentstroke}%
\pgfsetdash{}{0pt}%
\pgfpathmoveto{\pgfqpoint{4.964130in}{2.489570in}}%
\pgfpathlineto{\pgfqpoint{4.978578in}{2.504135in}}%
\pgfpathlineto{\pgfqpoint{4.993046in}{2.518866in}}%
\pgfpathlineto{\pgfqpoint{5.007535in}{2.533762in}}%
\pgfpathlineto{\pgfqpoint{5.022043in}{2.548824in}}%
\pgfpathlineto{\pgfqpoint{5.029878in}{2.565351in}}%
\pgfpathlineto{\pgfqpoint{5.037706in}{2.581713in}}%
\pgfpathlineto{\pgfqpoint{5.045528in}{2.597908in}}%
\pgfpathlineto{\pgfqpoint{5.053342in}{2.613934in}}%
\pgfpathlineto{\pgfqpoint{5.038823in}{2.598601in}}%
\pgfpathlineto{\pgfqpoint{5.024324in}{2.583434in}}%
\pgfpathlineto{\pgfqpoint{5.009846in}{2.568434in}}%
\pgfpathlineto{\pgfqpoint{4.995388in}{2.553599in}}%
\pgfpathlineto{\pgfqpoint{4.987583in}{2.537831in}}%
\pgfpathlineto{\pgfqpoint{4.979772in}{2.521903in}}%
\pgfpathlineto{\pgfqpoint{4.971954in}{2.505815in}}%
\pgfpathlineto{\pgfqpoint{4.964130in}{2.489570in}}%
\pgfpathclose%
\pgfusepath{fill}%
\end{pgfscope}%
\begin{pgfscope}%
\pgfpathrectangle{\pgfqpoint{1.254980in}{0.150000in}}{\pgfqpoint{5.490039in}{5.490039in}}%
\pgfusepath{clip}%
\pgfsetbuttcap%
\pgfsetroundjoin%
\definecolor{currentfill}{rgb}{0.235526,0.309527,0.542944}%
\pgfsetfillcolor{currentfill}%
\pgfsetfillopacity{0.700000}%
\pgfsetlinewidth{0.000000pt}%
\definecolor{currentstroke}{rgb}{0.000000,0.000000,0.000000}%
\pgfsetstrokecolor{currentstroke}%
\pgfsetdash{}{0pt}%
\pgfpathmoveto{\pgfqpoint{4.388518in}{1.535432in}}%
\pgfpathlineto{\pgfqpoint{4.402578in}{1.542554in}}%
\pgfpathlineto{\pgfqpoint{4.416653in}{1.549830in}}%
\pgfpathlineto{\pgfqpoint{4.430741in}{1.557261in}}%
\pgfpathlineto{\pgfqpoint{4.444844in}{1.564846in}}%
\pgfpathlineto{\pgfqpoint{4.452814in}{1.582768in}}%
\pgfpathlineto{\pgfqpoint{4.460780in}{1.600730in}}%
\pgfpathlineto{\pgfqpoint{4.468744in}{1.618725in}}%
\pgfpathlineto{\pgfqpoint{4.476705in}{1.636748in}}%
\pgfpathlineto{\pgfqpoint{4.462594in}{1.628578in}}%
\pgfpathlineto{\pgfqpoint{4.448498in}{1.620564in}}%
\pgfpathlineto{\pgfqpoint{4.434416in}{1.612705in}}%
\pgfpathlineto{\pgfqpoint{4.420349in}{1.605001in}}%
\pgfpathlineto{\pgfqpoint{4.412395in}{1.587551in}}%
\pgfpathlineto{\pgfqpoint{4.404439in}{1.570136in}}%
\pgfpathlineto{\pgfqpoint{4.396480in}{1.552761in}}%
\pgfpathlineto{\pgfqpoint{4.388518in}{1.535432in}}%
\pgfpathclose%
\pgfusepath{fill}%
\end{pgfscope}%
\begin{pgfscope}%
\pgfpathrectangle{\pgfqpoint{1.254980in}{0.150000in}}{\pgfqpoint{5.490039in}{5.490039in}}%
\pgfusepath{clip}%
\pgfsetbuttcap%
\pgfsetroundjoin%
\definecolor{currentfill}{rgb}{0.279574,0.170599,0.479997}%
\pgfsetfillcolor{currentfill}%
\pgfsetfillopacity{0.700000}%
\pgfsetlinewidth{0.000000pt}%
\definecolor{currentstroke}{rgb}{0.000000,0.000000,0.000000}%
\pgfsetstrokecolor{currentstroke}%
\pgfsetdash{}{0pt}%
\pgfpathmoveto{\pgfqpoint{4.148808in}{1.238428in}}%
\pgfpathlineto{\pgfqpoint{4.162756in}{1.241738in}}%
\pgfpathlineto{\pgfqpoint{4.176715in}{1.245199in}}%
\pgfpathlineto{\pgfqpoint{4.190686in}{1.248811in}}%
\pgfpathlineto{\pgfqpoint{4.204668in}{1.252575in}}%
\pgfpathlineto{\pgfqpoint{4.212678in}{1.267670in}}%
\pgfpathlineto{\pgfqpoint{4.220684in}{1.282915in}}%
\pgfpathlineto{\pgfqpoint{4.228687in}{1.298302in}}%
\pgfpathlineto{\pgfqpoint{4.236686in}{1.313826in}}%
\pgfpathlineto{\pgfqpoint{4.222703in}{1.309375in}}%
\pgfpathlineto{\pgfqpoint{4.208733in}{1.305076in}}%
\pgfpathlineto{\pgfqpoint{4.194774in}{1.300929in}}%
\pgfpathlineto{\pgfqpoint{4.180826in}{1.296934in}}%
\pgfpathlineto{\pgfqpoint{4.172828in}{1.282086in}}%
\pgfpathlineto{\pgfqpoint{4.164825in}{1.267382in}}%
\pgfpathlineto{\pgfqpoint{4.156819in}{1.252827in}}%
\pgfpathlineto{\pgfqpoint{4.148808in}{1.238428in}}%
\pgfpathclose%
\pgfusepath{fill}%
\end{pgfscope}%
\begin{pgfscope}%
\pgfpathrectangle{\pgfqpoint{1.254980in}{0.150000in}}{\pgfqpoint{5.490039in}{5.490039in}}%
\pgfusepath{clip}%
\pgfsetbuttcap%
\pgfsetroundjoin%
\definecolor{currentfill}{rgb}{0.125394,0.574318,0.549086}%
\pgfsetfillcolor{currentfill}%
\pgfsetfillopacity{0.700000}%
\pgfsetlinewidth{0.000000pt}%
\definecolor{currentstroke}{rgb}{0.000000,0.000000,0.000000}%
\pgfsetstrokecolor{currentstroke}%
\pgfsetdash{}{0pt}%
\pgfpathmoveto{\pgfqpoint{4.812334in}{2.230683in}}%
\pgfpathlineto{\pgfqpoint{4.826672in}{2.243602in}}%
\pgfpathlineto{\pgfqpoint{4.841028in}{2.256684in}}%
\pgfpathlineto{\pgfqpoint{4.855403in}{2.269928in}}%
\pgfpathlineto{\pgfqpoint{4.869797in}{2.283335in}}%
\pgfpathlineto{\pgfqpoint{4.877687in}{2.301255in}}%
\pgfpathlineto{\pgfqpoint{4.885572in}{2.319052in}}%
\pgfpathlineto{\pgfqpoint{4.893453in}{2.336722in}}%
\pgfpathlineto{\pgfqpoint{4.901328in}{2.354263in}}%
\pgfpathlineto{\pgfqpoint{4.886922in}{2.340494in}}%
\pgfpathlineto{\pgfqpoint{4.872535in}{2.326889in}}%
\pgfpathlineto{\pgfqpoint{4.858167in}{2.313447in}}%
\pgfpathlineto{\pgfqpoint{4.843818in}{2.300168in}}%
\pgfpathlineto{\pgfqpoint{4.835954in}{2.282976in}}%
\pgfpathlineto{\pgfqpoint{4.828086in}{2.265663in}}%
\pgfpathlineto{\pgfqpoint{4.820212in}{2.248231in}}%
\pgfpathlineto{\pgfqpoint{4.812334in}{2.230683in}}%
\pgfpathclose%
\pgfusepath{fill}%
\end{pgfscope}%
\begin{pgfscope}%
\pgfpathrectangle{\pgfqpoint{1.254980in}{0.150000in}}{\pgfqpoint{5.490039in}{5.490039in}}%
\pgfusepath{clip}%
\pgfsetbuttcap%
\pgfsetroundjoin%
\definecolor{currentfill}{rgb}{0.160665,0.478540,0.558115}%
\pgfsetfillcolor{currentfill}%
\pgfsetfillopacity{0.700000}%
\pgfsetlinewidth{0.000000pt}%
\definecolor{currentstroke}{rgb}{0.000000,0.000000,0.000000}%
\pgfsetstrokecolor{currentstroke}%
\pgfsetdash{}{0pt}%
\pgfpathmoveto{\pgfqpoint{4.660423in}{1.967313in}}%
\pgfpathlineto{\pgfqpoint{4.674655in}{1.978335in}}%
\pgfpathlineto{\pgfqpoint{4.688904in}{1.989515in}}%
\pgfpathlineto{\pgfqpoint{4.703170in}{2.000855in}}%
\pgfpathlineto{\pgfqpoint{4.717453in}{2.012355in}}%
\pgfpathlineto{\pgfqpoint{4.725382in}{2.031014in}}%
\pgfpathlineto{\pgfqpoint{4.733307in}{2.049601in}}%
\pgfpathlineto{\pgfqpoint{4.741228in}{2.068113in}}%
\pgfpathlineto{\pgfqpoint{4.749146in}{2.086546in}}%
\pgfpathlineto{\pgfqpoint{4.734850in}{2.074597in}}%
\pgfpathlineto{\pgfqpoint{4.720572in}{2.062809in}}%
\pgfpathlineto{\pgfqpoint{4.706312in}{2.051181in}}%
\pgfpathlineto{\pgfqpoint{4.692069in}{2.039714in}}%
\pgfpathlineto{\pgfqpoint{4.684163in}{2.021718in}}%
\pgfpathlineto{\pgfqpoint{4.676254in}{2.003650in}}%
\pgfpathlineto{\pgfqpoint{4.668340in}{1.985514in}}%
\pgfpathlineto{\pgfqpoint{4.660423in}{1.967313in}}%
\pgfpathclose%
\pgfusepath{fill}%
\end{pgfscope}%
\begin{pgfscope}%
\pgfpathrectangle{\pgfqpoint{1.254980in}{0.150000in}}{\pgfqpoint{5.490039in}{5.490039in}}%
\pgfusepath{clip}%
\pgfsetbuttcap%
\pgfsetroundjoin%
\definecolor{currentfill}{rgb}{0.201239,0.383670,0.554294}%
\pgfsetfillcolor{currentfill}%
\pgfsetfillopacity{0.700000}%
\pgfsetlinewidth{0.000000pt}%
\definecolor{currentstroke}{rgb}{0.000000,0.000000,0.000000}%
\pgfsetstrokecolor{currentstroke}%
\pgfsetdash{}{0pt}%
\pgfpathmoveto{\pgfqpoint{4.508520in}{1.709016in}}%
\pgfpathlineto{\pgfqpoint{4.522655in}{1.717899in}}%
\pgfpathlineto{\pgfqpoint{4.536806in}{1.726938in}}%
\pgfpathlineto{\pgfqpoint{4.550972in}{1.736133in}}%
\pgfpathlineto{\pgfqpoint{4.565155in}{1.745486in}}%
\pgfpathlineto{\pgfqpoint{4.573111in}{1.764112in}}%
\pgfpathlineto{\pgfqpoint{4.581064in}{1.782729in}}%
\pgfpathlineto{\pgfqpoint{4.589015in}{1.801331in}}%
\pgfpathlineto{\pgfqpoint{4.596962in}{1.819914in}}%
\pgfpathlineto{\pgfqpoint{4.582769in}{1.810030in}}%
\pgfpathlineto{\pgfqpoint{4.568593in}{1.800303in}}%
\pgfpathlineto{\pgfqpoint{4.554432in}{1.790734in}}%
\pgfpathlineto{\pgfqpoint{4.540287in}{1.781323in}}%
\pgfpathlineto{\pgfqpoint{4.532350in}{1.763260in}}%
\pgfpathlineto{\pgfqpoint{4.524410in}{1.745185in}}%
\pgfpathlineto{\pgfqpoint{4.516467in}{1.727102in}}%
\pgfpathlineto{\pgfqpoint{4.508520in}{1.709016in}}%
\pgfpathclose%
\pgfusepath{fill}%
\end{pgfscope}%
\begin{pgfscope}%
\pgfpathrectangle{\pgfqpoint{1.254980in}{0.150000in}}{\pgfqpoint{5.490039in}{5.490039in}}%
\pgfusepath{clip}%
\pgfsetbuttcap%
\pgfsetroundjoin%
\definecolor{currentfill}{rgb}{0.232815,0.732247,0.459277}%
\pgfsetfillcolor{currentfill}%
\pgfsetfillopacity{0.700000}%
\pgfsetlinewidth{0.000000pt}%
\definecolor{currentstroke}{rgb}{0.000000,0.000000,0.000000}%
\pgfsetstrokecolor{currentstroke}%
\pgfsetdash{}{0pt}%
\pgfpathmoveto{\pgfqpoint{5.084533in}{2.676302in}}%
\pgfpathlineto{\pgfqpoint{5.099082in}{2.692041in}}%
\pgfpathlineto{\pgfqpoint{5.113653in}{2.707948in}}%
\pgfpathlineto{\pgfqpoint{5.128245in}{2.724022in}}%
\pgfpathlineto{\pgfqpoint{5.142859in}{2.740265in}}%
\pgfpathlineto{\pgfqpoint{5.150648in}{2.755633in}}%
\pgfpathlineto{\pgfqpoint{5.158429in}{2.770813in}}%
\pgfpathlineto{\pgfqpoint{5.166203in}{2.785802in}}%
\pgfpathlineto{\pgfqpoint{5.173969in}{2.800598in}}%
\pgfpathlineto{\pgfqpoint{5.159346in}{2.784146in}}%
\pgfpathlineto{\pgfqpoint{5.144745in}{2.767863in}}%
\pgfpathlineto{\pgfqpoint{5.130165in}{2.751748in}}%
\pgfpathlineto{\pgfqpoint{5.115607in}{2.735800in}}%
\pgfpathlineto{\pgfqpoint{5.107850in}{2.721201in}}%
\pgfpathlineto{\pgfqpoint{5.100085in}{2.706417in}}%
\pgfpathlineto{\pgfqpoint{5.092312in}{2.691450in}}%
\pgfpathlineto{\pgfqpoint{5.084533in}{2.676302in}}%
\pgfpathclose%
\pgfusepath{fill}%
\end{pgfscope}%
\begin{pgfscope}%
\pgfpathrectangle{\pgfqpoint{1.254980in}{0.150000in}}{\pgfqpoint{5.490039in}{5.490039in}}%
\pgfusepath{clip}%
\pgfsetbuttcap%
\pgfsetroundjoin%
\definecolor{currentfill}{rgb}{0.270595,0.214069,0.507052}%
\pgfsetfillcolor{currentfill}%
\pgfsetfillopacity{0.700000}%
\pgfsetlinewidth{0.000000pt}%
\definecolor{currentstroke}{rgb}{0.000000,0.000000,0.000000}%
\pgfsetstrokecolor{currentstroke}%
\pgfsetdash{}{0pt}%
\pgfpathmoveto{\pgfqpoint{4.236686in}{1.313826in}}%
\pgfpathlineto{\pgfqpoint{4.250681in}{1.318430in}}%
\pgfpathlineto{\pgfqpoint{4.264688in}{1.323185in}}%
\pgfpathlineto{\pgfqpoint{4.278707in}{1.328092in}}%
\pgfpathlineto{\pgfqpoint{4.292739in}{1.333152in}}%
\pgfpathlineto{\pgfqpoint{4.300737in}{1.349478in}}%
\pgfpathlineto{\pgfqpoint{4.308732in}{1.365922in}}%
\pgfpathlineto{\pgfqpoint{4.316724in}{1.382475in}}%
\pgfpathlineto{\pgfqpoint{4.324714in}{1.399134in}}%
\pgfpathlineto{\pgfqpoint{4.310678in}{1.393411in}}%
\pgfpathlineto{\pgfqpoint{4.296655in}{1.387841in}}%
\pgfpathlineto{\pgfqpoint{4.282646in}{1.382424in}}%
\pgfpathlineto{\pgfqpoint{4.268648in}{1.377160in}}%
\pgfpathlineto{\pgfqpoint{4.260663in}{1.361153in}}%
\pgfpathlineto{\pgfqpoint{4.252674in}{1.345258in}}%
\pgfpathlineto{\pgfqpoint{4.244682in}{1.329480in}}%
\pgfpathlineto{\pgfqpoint{4.236686in}{1.313826in}}%
\pgfpathclose%
\pgfusepath{fill}%
\end{pgfscope}%
\begin{pgfscope}%
\pgfpathrectangle{\pgfqpoint{1.254980in}{0.150000in}}{\pgfqpoint{5.490039in}{5.490039in}}%
\pgfusepath{clip}%
\pgfsetbuttcap%
\pgfsetroundjoin%
\definecolor{currentfill}{rgb}{0.244972,0.287675,0.537260}%
\pgfsetfillcolor{currentfill}%
\pgfsetfillopacity{0.700000}%
\pgfsetlinewidth{0.000000pt}%
\definecolor{currentstroke}{rgb}{0.000000,0.000000,0.000000}%
\pgfsetstrokecolor{currentstroke}%
\pgfsetdash{}{0pt}%
\pgfpathmoveto{\pgfqpoint{4.356640in}{1.466688in}}%
\pgfpathlineto{\pgfqpoint{4.370694in}{1.473200in}}%
\pgfpathlineto{\pgfqpoint{4.384761in}{1.479866in}}%
\pgfpathlineto{\pgfqpoint{4.398843in}{1.486686in}}%
\pgfpathlineto{\pgfqpoint{4.412938in}{1.493660in}}%
\pgfpathlineto{\pgfqpoint{4.420919in}{1.511370in}}%
\pgfpathlineto{\pgfqpoint{4.428897in}{1.529141in}}%
\pgfpathlineto{\pgfqpoint{4.436872in}{1.546969in}}%
\pgfpathlineto{\pgfqpoint{4.444844in}{1.564846in}}%
\pgfpathlineto{\pgfqpoint{4.430741in}{1.557261in}}%
\pgfpathlineto{\pgfqpoint{4.416653in}{1.549830in}}%
\pgfpathlineto{\pgfqpoint{4.402578in}{1.542554in}}%
\pgfpathlineto{\pgfqpoint{4.388518in}{1.535432in}}%
\pgfpathlineto{\pgfqpoint{4.380553in}{1.518155in}}%
\pgfpathlineto{\pgfqpoint{4.372585in}{1.500934in}}%
\pgfpathlineto{\pgfqpoint{4.364614in}{1.483777in}}%
\pgfpathlineto{\pgfqpoint{4.356640in}{1.466688in}}%
\pgfpathclose%
\pgfusepath{fill}%
\end{pgfscope}%
\begin{pgfscope}%
\pgfpathrectangle{\pgfqpoint{1.254980in}{0.150000in}}{\pgfqpoint{5.490039in}{5.490039in}}%
\pgfusepath{clip}%
\pgfsetbuttcap%
\pgfsetroundjoin%
\definecolor{currentfill}{rgb}{0.128087,0.647749,0.523491}%
\pgfsetfillcolor{currentfill}%
\pgfsetfillopacity{0.700000}%
\pgfsetlinewidth{0.000000pt}%
\definecolor{currentstroke}{rgb}{0.000000,0.000000,0.000000}%
\pgfsetstrokecolor{currentstroke}%
\pgfsetdash{}{0pt}%
\pgfpathmoveto{\pgfqpoint{4.932774in}{2.423070in}}%
\pgfpathlineto{\pgfqpoint{4.947212in}{2.437334in}}%
\pgfpathlineto{\pgfqpoint{4.961669in}{2.451763in}}%
\pgfpathlineto{\pgfqpoint{4.976146in}{2.466357in}}%
\pgfpathlineto{\pgfqpoint{4.990643in}{2.481117in}}%
\pgfpathlineto{\pgfqpoint{4.998502in}{2.498278in}}%
\pgfpathlineto{\pgfqpoint{5.006355in}{2.515285in}}%
\pgfpathlineto{\pgfqpoint{5.014203in}{2.532134in}}%
\pgfpathlineto{\pgfqpoint{5.022043in}{2.548824in}}%
\pgfpathlineto{\pgfqpoint{5.007535in}{2.533762in}}%
\pgfpathlineto{\pgfqpoint{4.993046in}{2.518866in}}%
\pgfpathlineto{\pgfqpoint{4.978578in}{2.504135in}}%
\pgfpathlineto{\pgfqpoint{4.964130in}{2.489570in}}%
\pgfpathlineto{\pgfqpoint{4.956300in}{2.473170in}}%
\pgfpathlineto{\pgfqpoint{4.948464in}{2.456619in}}%
\pgfpathlineto{\pgfqpoint{4.940622in}{2.439918in}}%
\pgfpathlineto{\pgfqpoint{4.932774in}{2.423070in}}%
\pgfpathclose%
\pgfusepath{fill}%
\end{pgfscope}%
\begin{pgfscope}%
\pgfpathrectangle{\pgfqpoint{1.254980in}{0.150000in}}{\pgfqpoint{5.490039in}{5.490039in}}%
\pgfusepath{clip}%
\pgfsetbuttcap%
\pgfsetroundjoin%
\definecolor{currentfill}{rgb}{0.515992,0.831158,0.294279}%
\pgfsetfillcolor{currentfill}%
\pgfsetfillopacity{0.700000}%
\pgfsetlinewidth{0.000000pt}%
\definecolor{currentstroke}{rgb}{0.000000,0.000000,0.000000}%
\pgfsetstrokecolor{currentstroke}%
\pgfsetdash{}{0pt}%
\pgfpathmoveto{\pgfqpoint{5.325291in}{3.031885in}}%
\pgfpathlineto{\pgfqpoint{5.340045in}{3.049620in}}%
\pgfpathlineto{\pgfqpoint{5.354822in}{3.067525in}}%
\pgfpathlineto{\pgfqpoint{5.369623in}{3.085601in}}%
\pgfpathlineto{\pgfqpoint{5.377284in}{3.097944in}}%
\pgfpathlineto{\pgfqpoint{5.384934in}{3.110065in}}%
\pgfpathlineto{\pgfqpoint{5.392575in}{3.121965in}}%
\pgfpathlineto{\pgfqpoint{5.400205in}{3.133643in}}%
\pgfpathlineto{\pgfqpoint{5.385401in}{3.115487in}}%
\pgfpathlineto{\pgfqpoint{5.370620in}{3.097502in}}%
\pgfpathlineto{\pgfqpoint{5.355862in}{3.079688in}}%
\pgfpathlineto{\pgfqpoint{5.348234in}{3.068061in}}%
\pgfpathlineto{\pgfqpoint{5.340596in}{3.056218in}}%
\pgfpathlineto{\pgfqpoint{5.332949in}{3.044159in}}%
\pgfpathlineto{\pgfqpoint{5.325291in}{3.031885in}}%
\pgfpathclose%
\pgfusepath{fill}%
\end{pgfscope}%
\begin{pgfscope}%
\pgfpathrectangle{\pgfqpoint{1.254980in}{0.150000in}}{\pgfqpoint{5.490039in}{5.490039in}}%
\pgfusepath{clip}%
\pgfsetbuttcap%
\pgfsetroundjoin%
\definecolor{currentfill}{rgb}{0.369214,0.788888,0.382914}%
\pgfsetfillcolor{currentfill}%
\pgfsetfillopacity{0.700000}%
\pgfsetlinewidth{0.000000pt}%
\definecolor{currentstroke}{rgb}{0.000000,0.000000,0.000000}%
\pgfsetstrokecolor{currentstroke}%
\pgfsetdash{}{0pt}%
\pgfpathmoveto{\pgfqpoint{5.204950in}{2.857827in}}%
\pgfpathlineto{\pgfqpoint{5.219602in}{2.874624in}}%
\pgfpathlineto{\pgfqpoint{5.234277in}{2.891591in}}%
\pgfpathlineto{\pgfqpoint{5.248974in}{2.908726in}}%
\pgfpathlineto{\pgfqpoint{5.263693in}{2.926032in}}%
\pgfpathlineto{\pgfqpoint{5.271425in}{2.940001in}}%
\pgfpathlineto{\pgfqpoint{5.279147in}{2.953762in}}%
\pgfpathlineto{\pgfqpoint{5.286861in}{2.967312in}}%
\pgfpathlineto{\pgfqpoint{5.294566in}{2.980652in}}%
\pgfpathlineto{\pgfqpoint{5.279840in}{2.963201in}}%
\pgfpathlineto{\pgfqpoint{5.265137in}{2.945920in}}%
\pgfpathlineto{\pgfqpoint{5.250456in}{2.928808in}}%
\pgfpathlineto{\pgfqpoint{5.235797in}{2.911867in}}%
\pgfpathlineto{\pgfqpoint{5.228098in}{2.898660in}}%
\pgfpathlineto{\pgfqpoint{5.220391in}{2.885250in}}%
\pgfpathlineto{\pgfqpoint{5.212675in}{2.871639in}}%
\pgfpathlineto{\pgfqpoint{5.204950in}{2.857827in}}%
\pgfpathclose%
\pgfusepath{fill}%
\end{pgfscope}%
\begin{pgfscope}%
\pgfpathrectangle{\pgfqpoint{1.254980in}{0.150000in}}{\pgfqpoint{5.490039in}{5.490039in}}%
\pgfusepath{clip}%
\pgfsetbuttcap%
\pgfsetroundjoin%
\definecolor{currentfill}{rgb}{0.131172,0.555899,0.552459}%
\pgfsetfillcolor{currentfill}%
\pgfsetfillopacity{0.700000}%
\pgfsetlinewidth{0.000000pt}%
\definecolor{currentstroke}{rgb}{0.000000,0.000000,0.000000}%
\pgfsetstrokecolor{currentstroke}%
\pgfsetdash{}{0pt}%
\pgfpathmoveto{\pgfqpoint{4.780775in}{2.159402in}}%
\pgfpathlineto{\pgfqpoint{4.795100in}{2.171931in}}%
\pgfpathlineto{\pgfqpoint{4.809445in}{2.184622in}}%
\pgfpathlineto{\pgfqpoint{4.823807in}{2.197475in}}%
\pgfpathlineto{\pgfqpoint{4.838189in}{2.210490in}}%
\pgfpathlineto{\pgfqpoint{4.846098in}{2.228869in}}%
\pgfpathlineto{\pgfqpoint{4.854002in}{2.247139in}}%
\pgfpathlineto{\pgfqpoint{4.861902in}{2.265295in}}%
\pgfpathlineto{\pgfqpoint{4.869797in}{2.283335in}}%
\pgfpathlineto{\pgfqpoint{4.855403in}{2.269928in}}%
\pgfpathlineto{\pgfqpoint{4.841028in}{2.256684in}}%
\pgfpathlineto{\pgfqpoint{4.826672in}{2.243602in}}%
\pgfpathlineto{\pgfqpoint{4.812334in}{2.230683in}}%
\pgfpathlineto{\pgfqpoint{4.804451in}{2.213023in}}%
\pgfpathlineto{\pgfqpoint{4.796563in}{2.195253in}}%
\pgfpathlineto{\pgfqpoint{4.788671in}{2.177379in}}%
\pgfpathlineto{\pgfqpoint{4.780775in}{2.159402in}}%
\pgfpathclose%
\pgfusepath{fill}%
\end{pgfscope}%
\begin{pgfscope}%
\pgfpathrectangle{\pgfqpoint{1.254980in}{0.150000in}}{\pgfqpoint{5.490039in}{5.490039in}}%
\pgfusepath{clip}%
\pgfsetbuttcap%
\pgfsetroundjoin%
\definecolor{currentfill}{rgb}{0.168126,0.459988,0.558082}%
\pgfsetfillcolor{currentfill}%
\pgfsetfillopacity{0.700000}%
\pgfsetlinewidth{0.000000pt}%
\definecolor{currentstroke}{rgb}{0.000000,0.000000,0.000000}%
\pgfsetstrokecolor{currentstroke}%
\pgfsetdash{}{0pt}%
\pgfpathmoveto{\pgfqpoint{4.628720in}{1.893955in}}%
\pgfpathlineto{\pgfqpoint{4.642940in}{1.904500in}}%
\pgfpathlineto{\pgfqpoint{4.657177in}{1.915205in}}%
\pgfpathlineto{\pgfqpoint{4.671431in}{1.926068in}}%
\pgfpathlineto{\pgfqpoint{4.685702in}{1.937091in}}%
\pgfpathlineto{\pgfqpoint{4.693645in}{1.955993in}}%
\pgfpathlineto{\pgfqpoint{4.701585in}{1.974841in}}%
\pgfpathlineto{\pgfqpoint{4.709521in}{1.993630in}}%
\pgfpathlineto{\pgfqpoint{4.717453in}{2.012355in}}%
\pgfpathlineto{\pgfqpoint{4.703170in}{2.000855in}}%
\pgfpathlineto{\pgfqpoint{4.688904in}{1.989515in}}%
\pgfpathlineto{\pgfqpoint{4.674655in}{1.978335in}}%
\pgfpathlineto{\pgfqpoint{4.660423in}{1.967313in}}%
\pgfpathlineto{\pgfqpoint{4.652503in}{1.949053in}}%
\pgfpathlineto{\pgfqpoint{4.644578in}{1.930737in}}%
\pgfpathlineto{\pgfqpoint{4.636651in}{1.912369in}}%
\pgfpathlineto{\pgfqpoint{4.628720in}{1.893955in}}%
\pgfpathclose%
\pgfusepath{fill}%
\end{pgfscope}%
\begin{pgfscope}%
\pgfpathrectangle{\pgfqpoint{1.254980in}{0.150000in}}{\pgfqpoint{5.490039in}{5.490039in}}%
\pgfusepath{clip}%
\pgfsetbuttcap%
\pgfsetroundjoin%
\definecolor{currentfill}{rgb}{0.212395,0.359683,0.551710}%
\pgfsetfillcolor{currentfill}%
\pgfsetfillopacity{0.700000}%
\pgfsetlinewidth{0.000000pt}%
\definecolor{currentstroke}{rgb}{0.000000,0.000000,0.000000}%
\pgfsetstrokecolor{currentstroke}%
\pgfsetdash{}{0pt}%
\pgfpathmoveto{\pgfqpoint{4.476705in}{1.636748in}}%
\pgfpathlineto{\pgfqpoint{4.490831in}{1.645073in}}%
\pgfpathlineto{\pgfqpoint{4.504972in}{1.653554in}}%
\pgfpathlineto{\pgfqpoint{4.519129in}{1.662191in}}%
\pgfpathlineto{\pgfqpoint{4.533300in}{1.670983in}}%
\pgfpathlineto{\pgfqpoint{4.541268in}{1.689598in}}%
\pgfpathlineto{\pgfqpoint{4.549233in}{1.708224in}}%
\pgfpathlineto{\pgfqpoint{4.557195in}{1.726855in}}%
\pgfpathlineto{\pgfqpoint{4.565155in}{1.745486in}}%
\pgfpathlineto{\pgfqpoint{4.550972in}{1.736133in}}%
\pgfpathlineto{\pgfqpoint{4.536806in}{1.726938in}}%
\pgfpathlineto{\pgfqpoint{4.522655in}{1.717899in}}%
\pgfpathlineto{\pgfqpoint{4.508520in}{1.709016in}}%
\pgfpathlineto{\pgfqpoint{4.500571in}{1.690933in}}%
\pgfpathlineto{\pgfqpoint{4.492619in}{1.672857in}}%
\pgfpathlineto{\pgfqpoint{4.484663in}{1.654793in}}%
\pgfpathlineto{\pgfqpoint{4.476705in}{1.636748in}}%
\pgfpathclose%
\pgfusepath{fill}%
\end{pgfscope}%
\begin{pgfscope}%
\pgfpathrectangle{\pgfqpoint{1.254980in}{0.150000in}}{\pgfqpoint{5.490039in}{5.490039in}}%
\pgfusepath{clip}%
\pgfsetbuttcap%
\pgfsetroundjoin%
\definecolor{currentfill}{rgb}{0.281412,0.155834,0.469201}%
\pgfsetfillcolor{currentfill}%
\pgfsetfillopacity{0.700000}%
\pgfsetlinewidth{0.000000pt}%
\definecolor{currentstroke}{rgb}{0.000000,0.000000,0.000000}%
\pgfsetstrokecolor{currentstroke}%
\pgfsetdash{}{0pt}%
\pgfpathmoveto{\pgfqpoint{4.116724in}{1.182533in}}%
\pgfpathlineto{\pgfqpoint{4.130674in}{1.185130in}}%
\pgfpathlineto{\pgfqpoint{4.144635in}{1.187878in}}%
\pgfpathlineto{\pgfqpoint{4.158607in}{1.190777in}}%
\pgfpathlineto{\pgfqpoint{4.172590in}{1.193826in}}%
\pgfpathlineto{\pgfqpoint{4.180615in}{1.208255in}}%
\pgfpathlineto{\pgfqpoint{4.188636in}{1.222861in}}%
\pgfpathlineto{\pgfqpoint{4.196654in}{1.237637in}}%
\pgfpathlineto{\pgfqpoint{4.204668in}{1.252575in}}%
\pgfpathlineto{\pgfqpoint{4.190686in}{1.248811in}}%
\pgfpathlineto{\pgfqpoint{4.176715in}{1.245199in}}%
\pgfpathlineto{\pgfqpoint{4.162756in}{1.241738in}}%
\pgfpathlineto{\pgfqpoint{4.148808in}{1.238428in}}%
\pgfpathlineto{\pgfqpoint{4.140793in}{1.224193in}}%
\pgfpathlineto{\pgfqpoint{4.132774in}{1.210127in}}%
\pgfpathlineto{\pgfqpoint{4.124751in}{1.196239in}}%
\pgfpathlineto{\pgfqpoint{4.116724in}{1.182533in}}%
\pgfpathclose%
\pgfusepath{fill}%
\end{pgfscope}%
\begin{pgfscope}%
\pgfpathrectangle{\pgfqpoint{1.254980in}{0.150000in}}{\pgfqpoint{5.490039in}{5.490039in}}%
\pgfusepath{clip}%
\pgfsetbuttcap%
\pgfsetroundjoin%
\definecolor{currentfill}{rgb}{0.208030,0.718701,0.472873}%
\pgfsetfillcolor{currentfill}%
\pgfsetfillopacity{0.700000}%
\pgfsetlinewidth{0.000000pt}%
\definecolor{currentstroke}{rgb}{0.000000,0.000000,0.000000}%
\pgfsetstrokecolor{currentstroke}%
\pgfsetdash{}{0pt}%
\pgfpathmoveto{\pgfqpoint{5.053342in}{2.613934in}}%
\pgfpathlineto{\pgfqpoint{5.067882in}{2.629434in}}%
\pgfpathlineto{\pgfqpoint{5.082443in}{2.645100in}}%
\pgfpathlineto{\pgfqpoint{5.097026in}{2.660934in}}%
\pgfpathlineto{\pgfqpoint{5.111629in}{2.676934in}}%
\pgfpathlineto{\pgfqpoint{5.119447in}{2.693041in}}%
\pgfpathlineto{\pgfqpoint{5.127259in}{2.708967in}}%
\pgfpathlineto{\pgfqpoint{5.135063in}{2.724708in}}%
\pgfpathlineto{\pgfqpoint{5.142859in}{2.740265in}}%
\pgfpathlineto{\pgfqpoint{5.128245in}{2.724022in}}%
\pgfpathlineto{\pgfqpoint{5.113653in}{2.707948in}}%
\pgfpathlineto{\pgfqpoint{5.099082in}{2.692041in}}%
\pgfpathlineto{\pgfqpoint{5.084533in}{2.676302in}}%
\pgfpathlineto{\pgfqpoint{5.076746in}{2.660974in}}%
\pgfpathlineto{\pgfqpoint{5.068951in}{2.645469in}}%
\pgfpathlineto{\pgfqpoint{5.061150in}{2.629788in}}%
\pgfpathlineto{\pgfqpoint{5.053342in}{2.613934in}}%
\pgfpathclose%
\pgfusepath{fill}%
\end{pgfscope}%
\begin{pgfscope}%
\pgfpathrectangle{\pgfqpoint{1.254980in}{0.150000in}}{\pgfqpoint{5.490039in}{5.490039in}}%
\pgfusepath{clip}%
\pgfsetbuttcap%
\pgfsetroundjoin%
\definecolor{currentfill}{rgb}{0.253935,0.265254,0.529983}%
\pgfsetfillcolor{currentfill}%
\pgfsetfillopacity{0.700000}%
\pgfsetlinewidth{0.000000pt}%
\definecolor{currentstroke}{rgb}{0.000000,0.000000,0.000000}%
\pgfsetstrokecolor{currentstroke}%
\pgfsetdash{}{0pt}%
\pgfpathmoveto{\pgfqpoint{4.324714in}{1.399134in}}%
\pgfpathlineto{\pgfqpoint{4.338762in}{1.405009in}}%
\pgfpathlineto{\pgfqpoint{4.352824in}{1.411038in}}%
\pgfpathlineto{\pgfqpoint{4.366899in}{1.417220in}}%
\pgfpathlineto{\pgfqpoint{4.380988in}{1.423554in}}%
\pgfpathlineto{\pgfqpoint{4.388980in}{1.440959in}}%
\pgfpathlineto{\pgfqpoint{4.396969in}{1.458449in}}%
\pgfpathlineto{\pgfqpoint{4.404955in}{1.476018in}}%
\pgfpathlineto{\pgfqpoint{4.412938in}{1.493660in}}%
\pgfpathlineto{\pgfqpoint{4.398843in}{1.486686in}}%
\pgfpathlineto{\pgfqpoint{4.384761in}{1.479866in}}%
\pgfpathlineto{\pgfqpoint{4.370694in}{1.473200in}}%
\pgfpathlineto{\pgfqpoint{4.356640in}{1.466688in}}%
\pgfpathlineto{\pgfqpoint{4.348663in}{1.449673in}}%
\pgfpathlineto{\pgfqpoint{4.340683in}{1.432739in}}%
\pgfpathlineto{\pgfqpoint{4.332700in}{1.415890in}}%
\pgfpathlineto{\pgfqpoint{4.324714in}{1.399134in}}%
\pgfpathclose%
\pgfusepath{fill}%
\end{pgfscope}%
\begin{pgfscope}%
\pgfpathrectangle{\pgfqpoint{1.254980in}{0.150000in}}{\pgfqpoint{5.490039in}{5.490039in}}%
\pgfusepath{clip}%
\pgfsetbuttcap%
\pgfsetroundjoin%
\definecolor{currentfill}{rgb}{0.121380,0.629492,0.531973}%
\pgfsetfillcolor{currentfill}%
\pgfsetfillopacity{0.700000}%
\pgfsetlinewidth{0.000000pt}%
\definecolor{currentstroke}{rgb}{0.000000,0.000000,0.000000}%
\pgfsetstrokecolor{currentstroke}%
\pgfsetdash{}{0pt}%
\pgfpathmoveto{\pgfqpoint{4.901328in}{2.354263in}}%
\pgfpathlineto{\pgfqpoint{4.915753in}{2.368196in}}%
\pgfpathlineto{\pgfqpoint{4.930198in}{2.382293in}}%
\pgfpathlineto{\pgfqpoint{4.944663in}{2.396554in}}%
\pgfpathlineto{\pgfqpoint{4.959148in}{2.410979in}}%
\pgfpathlineto{\pgfqpoint{4.967030in}{2.428732in}}%
\pgfpathlineto{\pgfqpoint{4.974907in}{2.446341in}}%
\pgfpathlineto{\pgfqpoint{4.982778in}{2.463803in}}%
\pgfpathlineto{\pgfqpoint{4.990643in}{2.481117in}}%
\pgfpathlineto{\pgfqpoint{4.976146in}{2.466357in}}%
\pgfpathlineto{\pgfqpoint{4.961669in}{2.451763in}}%
\pgfpathlineto{\pgfqpoint{4.947212in}{2.437334in}}%
\pgfpathlineto{\pgfqpoint{4.932774in}{2.423070in}}%
\pgfpathlineto{\pgfqpoint{4.924921in}{2.406077in}}%
\pgfpathlineto{\pgfqpoint{4.917062in}{2.388943in}}%
\pgfpathlineto{\pgfqpoint{4.909198in}{2.371671in}}%
\pgfpathlineto{\pgfqpoint{4.901328in}{2.354263in}}%
\pgfpathclose%
\pgfusepath{fill}%
\end{pgfscope}%
\begin{pgfscope}%
\pgfpathrectangle{\pgfqpoint{1.254980in}{0.150000in}}{\pgfqpoint{5.490039in}{5.490039in}}%
\pgfusepath{clip}%
\pgfsetbuttcap%
\pgfsetroundjoin%
\definecolor{currentfill}{rgb}{0.139147,0.533812,0.555298}%
\pgfsetfillcolor{currentfill}%
\pgfsetfillopacity{0.700000}%
\pgfsetlinewidth{0.000000pt}%
\definecolor{currentstroke}{rgb}{0.000000,0.000000,0.000000}%
\pgfsetstrokecolor{currentstroke}%
\pgfsetdash{}{0pt}%
\pgfpathmoveto{\pgfqpoint{4.749146in}{2.086546in}}%
\pgfpathlineto{\pgfqpoint{4.763459in}{2.098655in}}%
\pgfpathlineto{\pgfqpoint{4.777791in}{2.110925in}}%
\pgfpathlineto{\pgfqpoint{4.792141in}{2.123357in}}%
\pgfpathlineto{\pgfqpoint{4.806509in}{2.135950in}}%
\pgfpathlineto{\pgfqpoint{4.814435in}{2.154731in}}%
\pgfpathlineto{\pgfqpoint{4.822357in}{2.173417in}}%
\pgfpathlineto{\pgfqpoint{4.830275in}{2.192005in}}%
\pgfpathlineto{\pgfqpoint{4.838189in}{2.210490in}}%
\pgfpathlineto{\pgfqpoint{4.823807in}{2.197475in}}%
\pgfpathlineto{\pgfqpoint{4.809445in}{2.184622in}}%
\pgfpathlineto{\pgfqpoint{4.795100in}{2.171931in}}%
\pgfpathlineto{\pgfqpoint{4.780775in}{2.159402in}}%
\pgfpathlineto{\pgfqpoint{4.772874in}{2.141326in}}%
\pgfpathlineto{\pgfqpoint{4.764969in}{2.123156in}}%
\pgfpathlineto{\pgfqpoint{4.757059in}{2.104894in}}%
\pgfpathlineto{\pgfqpoint{4.749146in}{2.086546in}}%
\pgfpathclose%
\pgfusepath{fill}%
\end{pgfscope}%
\begin{pgfscope}%
\pgfpathrectangle{\pgfqpoint{1.254980in}{0.150000in}}{\pgfqpoint{5.490039in}{5.490039in}}%
\pgfusepath{clip}%
\pgfsetbuttcap%
\pgfsetroundjoin%
\definecolor{currentfill}{rgb}{0.177423,0.437527,0.557565}%
\pgfsetfillcolor{currentfill}%
\pgfsetfillopacity{0.700000}%
\pgfsetlinewidth{0.000000pt}%
\definecolor{currentstroke}{rgb}{0.000000,0.000000,0.000000}%
\pgfsetstrokecolor{currentstroke}%
\pgfsetdash{}{0pt}%
\pgfpathmoveto{\pgfqpoint{4.596962in}{1.819914in}}%
\pgfpathlineto{\pgfqpoint{4.611171in}{1.829956in}}%
\pgfpathlineto{\pgfqpoint{4.625396in}{1.840155in}}%
\pgfpathlineto{\pgfqpoint{4.639638in}{1.850513in}}%
\pgfpathlineto{\pgfqpoint{4.653897in}{1.861029in}}%
\pgfpathlineto{\pgfqpoint{4.661853in}{1.880103in}}%
\pgfpathlineto{\pgfqpoint{4.669806in}{1.899141in}}%
\pgfpathlineto{\pgfqpoint{4.677756in}{1.918139in}}%
\pgfpathlineto{\pgfqpoint{4.685702in}{1.937091in}}%
\pgfpathlineto{\pgfqpoint{4.671431in}{1.926068in}}%
\pgfpathlineto{\pgfqpoint{4.657177in}{1.915205in}}%
\pgfpathlineto{\pgfqpoint{4.642940in}{1.904500in}}%
\pgfpathlineto{\pgfqpoint{4.628720in}{1.893955in}}%
\pgfpathlineto{\pgfqpoint{4.620785in}{1.875497in}}%
\pgfpathlineto{\pgfqpoint{4.612847in}{1.857001in}}%
\pgfpathlineto{\pgfqpoint{4.604906in}{1.838472in}}%
\pgfpathlineto{\pgfqpoint{4.596962in}{1.819914in}}%
\pgfpathclose%
\pgfusepath{fill}%
\end{pgfscope}%
\begin{pgfscope}%
\pgfpathrectangle{\pgfqpoint{1.254980in}{0.150000in}}{\pgfqpoint{5.490039in}{5.490039in}}%
\pgfusepath{clip}%
\pgfsetbuttcap%
\pgfsetroundjoin%
\definecolor{currentfill}{rgb}{0.275191,0.194905,0.496005}%
\pgfsetfillcolor{currentfill}%
\pgfsetfillopacity{0.700000}%
\pgfsetlinewidth{0.000000pt}%
\definecolor{currentstroke}{rgb}{0.000000,0.000000,0.000000}%
\pgfsetstrokecolor{currentstroke}%
\pgfsetdash{}{0pt}%
\pgfpathmoveto{\pgfqpoint{4.204668in}{1.252575in}}%
\pgfpathlineto{\pgfqpoint{4.218661in}{1.256490in}}%
\pgfpathlineto{\pgfqpoint{4.232667in}{1.260557in}}%
\pgfpathlineto{\pgfqpoint{4.246684in}{1.264775in}}%
\pgfpathlineto{\pgfqpoint{4.260713in}{1.269144in}}%
\pgfpathlineto{\pgfqpoint{4.268724in}{1.284939in}}%
\pgfpathlineto{\pgfqpoint{4.276732in}{1.300876in}}%
\pgfpathlineto{\pgfqpoint{4.284737in}{1.316949in}}%
\pgfpathlineto{\pgfqpoint{4.292739in}{1.333152in}}%
\pgfpathlineto{\pgfqpoint{4.278707in}{1.328092in}}%
\pgfpathlineto{\pgfqpoint{4.264688in}{1.323185in}}%
\pgfpathlineto{\pgfqpoint{4.250681in}{1.318430in}}%
\pgfpathlineto{\pgfqpoint{4.236686in}{1.313826in}}%
\pgfpathlineto{\pgfqpoint{4.228687in}{1.298302in}}%
\pgfpathlineto{\pgfqpoint{4.220684in}{1.282915in}}%
\pgfpathlineto{\pgfqpoint{4.212678in}{1.267670in}}%
\pgfpathlineto{\pgfqpoint{4.204668in}{1.252575in}}%
\pgfpathclose%
\pgfusepath{fill}%
\end{pgfscope}%
\begin{pgfscope}%
\pgfpathrectangle{\pgfqpoint{1.254980in}{0.150000in}}{\pgfqpoint{5.490039in}{5.490039in}}%
\pgfusepath{clip}%
\pgfsetbuttcap%
\pgfsetroundjoin%
\definecolor{currentfill}{rgb}{0.221989,0.339161,0.548752}%
\pgfsetfillcolor{currentfill}%
\pgfsetfillopacity{0.700000}%
\pgfsetlinewidth{0.000000pt}%
\definecolor{currentstroke}{rgb}{0.000000,0.000000,0.000000}%
\pgfsetstrokecolor{currentstroke}%
\pgfsetdash{}{0pt}%
\pgfpathmoveto{\pgfqpoint{4.444844in}{1.564846in}}%
\pgfpathlineto{\pgfqpoint{4.458961in}{1.572587in}}%
\pgfpathlineto{\pgfqpoint{4.473093in}{1.580482in}}%
\pgfpathlineto{\pgfqpoint{4.487240in}{1.588532in}}%
\pgfpathlineto{\pgfqpoint{4.501402in}{1.596737in}}%
\pgfpathlineto{\pgfqpoint{4.509381in}{1.615256in}}%
\pgfpathlineto{\pgfqpoint{4.517357in}{1.633807in}}%
\pgfpathlineto{\pgfqpoint{4.525330in}{1.652385in}}%
\pgfpathlineto{\pgfqpoint{4.533300in}{1.670983in}}%
\pgfpathlineto{\pgfqpoint{4.519129in}{1.662191in}}%
\pgfpathlineto{\pgfqpoint{4.504972in}{1.653554in}}%
\pgfpathlineto{\pgfqpoint{4.490831in}{1.645073in}}%
\pgfpathlineto{\pgfqpoint{4.476705in}{1.636748in}}%
\pgfpathlineto{\pgfqpoint{4.468744in}{1.618725in}}%
\pgfpathlineto{\pgfqpoint{4.460780in}{1.600730in}}%
\pgfpathlineto{\pgfqpoint{4.452814in}{1.582768in}}%
\pgfpathlineto{\pgfqpoint{4.444844in}{1.564846in}}%
\pgfpathclose%
\pgfusepath{fill}%
\end{pgfscope}%
\begin{pgfscope}%
\pgfpathrectangle{\pgfqpoint{1.254980in}{0.150000in}}{\pgfqpoint{5.490039in}{5.490039in}}%
\pgfusepath{clip}%
\pgfsetbuttcap%
\pgfsetroundjoin%
\definecolor{currentfill}{rgb}{0.344074,0.780029,0.397381}%
\pgfsetfillcolor{currentfill}%
\pgfsetfillopacity{0.700000}%
\pgfsetlinewidth{0.000000pt}%
\definecolor{currentstroke}{rgb}{0.000000,0.000000,0.000000}%
\pgfsetstrokecolor{currentstroke}%
\pgfsetdash{}{0pt}%
\pgfpathmoveto{\pgfqpoint{5.173969in}{2.800598in}}%
\pgfpathlineto{\pgfqpoint{5.188613in}{2.817218in}}%
\pgfpathlineto{\pgfqpoint{5.203279in}{2.834007in}}%
\pgfpathlineto{\pgfqpoint{5.217968in}{2.850965in}}%
\pgfpathlineto{\pgfqpoint{5.232679in}{2.868092in}}%
\pgfpathlineto{\pgfqpoint{5.240445in}{2.882884in}}%
\pgfpathlineto{\pgfqpoint{5.248203in}{2.897472in}}%
\pgfpathlineto{\pgfqpoint{5.255952in}{2.911855in}}%
\pgfpathlineto{\pgfqpoint{5.263693in}{2.926032in}}%
\pgfpathlineto{\pgfqpoint{5.248974in}{2.908726in}}%
\pgfpathlineto{\pgfqpoint{5.234277in}{2.891591in}}%
\pgfpathlineto{\pgfqpoint{5.219602in}{2.874624in}}%
\pgfpathlineto{\pgfqpoint{5.204950in}{2.857827in}}%
\pgfpathlineto{\pgfqpoint{5.197217in}{2.843816in}}%
\pgfpathlineto{\pgfqpoint{5.189476in}{2.829606in}}%
\pgfpathlineto{\pgfqpoint{5.181726in}{2.815200in}}%
\pgfpathlineto{\pgfqpoint{5.173969in}{2.800598in}}%
\pgfpathclose%
\pgfusepath{fill}%
\end{pgfscope}%
\begin{pgfscope}%
\pgfpathrectangle{\pgfqpoint{1.254980in}{0.150000in}}{\pgfqpoint{5.490039in}{5.490039in}}%
\pgfusepath{clip}%
\pgfsetbuttcap%
\pgfsetroundjoin%
\definecolor{currentfill}{rgb}{0.487026,0.823929,0.312321}%
\pgfsetfillcolor{currentfill}%
\pgfsetfillopacity{0.700000}%
\pgfsetlinewidth{0.000000pt}%
\definecolor{currentstroke}{rgb}{0.000000,0.000000,0.000000}%
\pgfsetstrokecolor{currentstroke}%
\pgfsetdash{}{0pt}%
\pgfpathmoveto{\pgfqpoint{5.294566in}{2.980652in}}%
\pgfpathlineto{\pgfqpoint{5.309315in}{2.998273in}}%
\pgfpathlineto{\pgfqpoint{5.324086in}{3.016066in}}%
\pgfpathlineto{\pgfqpoint{5.338881in}{3.034029in}}%
\pgfpathlineto{\pgfqpoint{5.346581in}{3.047251in}}%
\pgfpathlineto{\pgfqpoint{5.354272in}{3.060254in}}%
\pgfpathlineto{\pgfqpoint{5.361952in}{3.073038in}}%
\pgfpathlineto{\pgfqpoint{5.369623in}{3.085601in}}%
\pgfpathlineto{\pgfqpoint{5.354822in}{3.067525in}}%
\pgfpathlineto{\pgfqpoint{5.340045in}{3.049620in}}%
\pgfpathlineto{\pgfqpoint{5.325291in}{3.031885in}}%
\pgfpathlineto{\pgfqpoint{5.317624in}{3.019397in}}%
\pgfpathlineto{\pgfqpoint{5.309947in}{3.006695in}}%
\pgfpathlineto{\pgfqpoint{5.302261in}{2.993780in}}%
\pgfpathlineto{\pgfqpoint{5.294566in}{2.980652in}}%
\pgfpathclose%
\pgfusepath{fill}%
\end{pgfscope}%
\begin{pgfscope}%
\pgfpathrectangle{\pgfqpoint{1.254980in}{0.150000in}}{\pgfqpoint{5.490039in}{5.490039in}}%
\pgfusepath{clip}%
\pgfsetbuttcap%
\pgfsetroundjoin%
\definecolor{currentfill}{rgb}{0.180653,0.701402,0.488189}%
\pgfsetfillcolor{currentfill}%
\pgfsetfillopacity{0.700000}%
\pgfsetlinewidth{0.000000pt}%
\definecolor{currentstroke}{rgb}{0.000000,0.000000,0.000000}%
\pgfsetstrokecolor{currentstroke}%
\pgfsetdash{}{0pt}%
\pgfpathmoveto{\pgfqpoint{5.022043in}{2.548824in}}%
\pgfpathlineto{\pgfqpoint{5.036573in}{2.564052in}}%
\pgfpathlineto{\pgfqpoint{5.051123in}{2.579446in}}%
\pgfpathlineto{\pgfqpoint{5.065694in}{2.595007in}}%
\pgfpathlineto{\pgfqpoint{5.080285in}{2.610734in}}%
\pgfpathlineto{\pgfqpoint{5.088132in}{2.627546in}}%
\pgfpathlineto{\pgfqpoint{5.095971in}{2.644184in}}%
\pgfpathlineto{\pgfqpoint{5.103803in}{2.660648in}}%
\pgfpathlineto{\pgfqpoint{5.111629in}{2.676934in}}%
\pgfpathlineto{\pgfqpoint{5.097026in}{2.660934in}}%
\pgfpathlineto{\pgfqpoint{5.082443in}{2.645100in}}%
\pgfpathlineto{\pgfqpoint{5.067882in}{2.629434in}}%
\pgfpathlineto{\pgfqpoint{5.053342in}{2.613934in}}%
\pgfpathlineto{\pgfqpoint{5.045528in}{2.597908in}}%
\pgfpathlineto{\pgfqpoint{5.037706in}{2.581713in}}%
\pgfpathlineto{\pgfqpoint{5.029878in}{2.565351in}}%
\pgfpathlineto{\pgfqpoint{5.022043in}{2.548824in}}%
\pgfpathclose%
\pgfusepath{fill}%
\end{pgfscope}%
\begin{pgfscope}%
\pgfpathrectangle{\pgfqpoint{1.254980in}{0.150000in}}{\pgfqpoint{5.490039in}{5.490039in}}%
\pgfusepath{clip}%
\pgfsetbuttcap%
\pgfsetroundjoin%
\definecolor{currentfill}{rgb}{0.187231,0.414746,0.556547}%
\pgfsetfillcolor{currentfill}%
\pgfsetfillopacity{0.700000}%
\pgfsetlinewidth{0.000000pt}%
\definecolor{currentstroke}{rgb}{0.000000,0.000000,0.000000}%
\pgfsetstrokecolor{currentstroke}%
\pgfsetdash{}{0pt}%
\pgfpathmoveto{\pgfqpoint{4.565155in}{1.745486in}}%
\pgfpathlineto{\pgfqpoint{4.579352in}{1.754995in}}%
\pgfpathlineto{\pgfqpoint{4.593566in}{1.764662in}}%
\pgfpathlineto{\pgfqpoint{4.607796in}{1.774485in}}%
\pgfpathlineto{\pgfqpoint{4.622043in}{1.784466in}}%
\pgfpathlineto{\pgfqpoint{4.630011in}{1.803637in}}%
\pgfpathlineto{\pgfqpoint{4.637976in}{1.822791in}}%
\pgfpathlineto{\pgfqpoint{4.645938in}{1.841923in}}%
\pgfpathlineto{\pgfqpoint{4.653897in}{1.861029in}}%
\pgfpathlineto{\pgfqpoint{4.639638in}{1.850513in}}%
\pgfpathlineto{\pgfqpoint{4.625396in}{1.840155in}}%
\pgfpathlineto{\pgfqpoint{4.611171in}{1.829956in}}%
\pgfpathlineto{\pgfqpoint{4.596962in}{1.819914in}}%
\pgfpathlineto{\pgfqpoint{4.589015in}{1.801331in}}%
\pgfpathlineto{\pgfqpoint{4.581064in}{1.782729in}}%
\pgfpathlineto{\pgfqpoint{4.573111in}{1.764112in}}%
\pgfpathlineto{\pgfqpoint{4.565155in}{1.745486in}}%
\pgfpathclose%
\pgfusepath{fill}%
\end{pgfscope}%
\begin{pgfscope}%
\pgfpathrectangle{\pgfqpoint{1.254980in}{0.150000in}}{\pgfqpoint{5.490039in}{5.490039in}}%
\pgfusepath{clip}%
\pgfsetbuttcap%
\pgfsetroundjoin%
\definecolor{currentfill}{rgb}{0.147607,0.511733,0.557049}%
\pgfsetfillcolor{currentfill}%
\pgfsetfillopacity{0.700000}%
\pgfsetlinewidth{0.000000pt}%
\definecolor{currentstroke}{rgb}{0.000000,0.000000,0.000000}%
\pgfsetstrokecolor{currentstroke}%
\pgfsetdash{}{0pt}%
\pgfpathmoveto{\pgfqpoint{4.717453in}{2.012355in}}%
\pgfpathlineto{\pgfqpoint{4.731754in}{2.024016in}}%
\pgfpathlineto{\pgfqpoint{4.746073in}{2.035836in}}%
\pgfpathlineto{\pgfqpoint{4.760409in}{2.047817in}}%
\pgfpathlineto{\pgfqpoint{4.774764in}{2.059958in}}%
\pgfpathlineto{\pgfqpoint{4.782706in}{2.079078in}}%
\pgfpathlineto{\pgfqpoint{4.790644in}{2.098120in}}%
\pgfpathlineto{\pgfqpoint{4.798579in}{2.117078in}}%
\pgfpathlineto{\pgfqpoint{4.806509in}{2.135950in}}%
\pgfpathlineto{\pgfqpoint{4.792141in}{2.123357in}}%
\pgfpathlineto{\pgfqpoint{4.777791in}{2.110925in}}%
\pgfpathlineto{\pgfqpoint{4.763459in}{2.098655in}}%
\pgfpathlineto{\pgfqpoint{4.749146in}{2.086546in}}%
\pgfpathlineto{\pgfqpoint{4.741228in}{2.068113in}}%
\pgfpathlineto{\pgfqpoint{4.733307in}{2.049601in}}%
\pgfpathlineto{\pgfqpoint{4.725382in}{2.031014in}}%
\pgfpathlineto{\pgfqpoint{4.717453in}{2.012355in}}%
\pgfpathclose%
\pgfusepath{fill}%
\end{pgfscope}%
\begin{pgfscope}%
\pgfpathrectangle{\pgfqpoint{1.254980in}{0.150000in}}{\pgfqpoint{5.490039in}{5.490039in}}%
\pgfusepath{clip}%
\pgfsetbuttcap%
\pgfsetroundjoin%
\definecolor{currentfill}{rgb}{0.119512,0.607464,0.540218}%
\pgfsetfillcolor{currentfill}%
\pgfsetfillopacity{0.700000}%
\pgfsetlinewidth{0.000000pt}%
\definecolor{currentstroke}{rgb}{0.000000,0.000000,0.000000}%
\pgfsetstrokecolor{currentstroke}%
\pgfsetdash{}{0pt}%
\pgfpathmoveto{\pgfqpoint{4.869797in}{2.283335in}}%
\pgfpathlineto{\pgfqpoint{4.884210in}{2.296905in}}%
\pgfpathlineto{\pgfqpoint{4.898643in}{2.310639in}}%
\pgfpathlineto{\pgfqpoint{4.913094in}{2.324536in}}%
\pgfpathlineto{\pgfqpoint{4.927566in}{2.338597in}}%
\pgfpathlineto{\pgfqpoint{4.935469in}{2.356892in}}%
\pgfpathlineto{\pgfqpoint{4.943367in}{2.375057in}}%
\pgfpathlineto{\pgfqpoint{4.951260in}{2.393087in}}%
\pgfpathlineto{\pgfqpoint{4.959148in}{2.410979in}}%
\pgfpathlineto{\pgfqpoint{4.944663in}{2.396554in}}%
\pgfpathlineto{\pgfqpoint{4.930198in}{2.382293in}}%
\pgfpathlineto{\pgfqpoint{4.915753in}{2.368196in}}%
\pgfpathlineto{\pgfqpoint{4.901328in}{2.354263in}}%
\pgfpathlineto{\pgfqpoint{4.893453in}{2.336722in}}%
\pgfpathlineto{\pgfqpoint{4.885572in}{2.319052in}}%
\pgfpathlineto{\pgfqpoint{4.877687in}{2.301255in}}%
\pgfpathlineto{\pgfqpoint{4.869797in}{2.283335in}}%
\pgfpathclose%
\pgfusepath{fill}%
\end{pgfscope}%
\begin{pgfscope}%
\pgfpathrectangle{\pgfqpoint{1.254980in}{0.150000in}}{\pgfqpoint{5.490039in}{5.490039in}}%
\pgfusepath{clip}%
\pgfsetbuttcap%
\pgfsetroundjoin%
\definecolor{currentfill}{rgb}{0.262138,0.242286,0.520837}%
\pgfsetfillcolor{currentfill}%
\pgfsetfillopacity{0.700000}%
\pgfsetlinewidth{0.000000pt}%
\definecolor{currentstroke}{rgb}{0.000000,0.000000,0.000000}%
\pgfsetstrokecolor{currentstroke}%
\pgfsetdash{}{0pt}%
\pgfpathmoveto{\pgfqpoint{4.292739in}{1.333152in}}%
\pgfpathlineto{\pgfqpoint{4.306783in}{1.338364in}}%
\pgfpathlineto{\pgfqpoint{4.320840in}{1.343728in}}%
\pgfpathlineto{\pgfqpoint{4.334910in}{1.349244in}}%
\pgfpathlineto{\pgfqpoint{4.348993in}{1.354912in}}%
\pgfpathlineto{\pgfqpoint{4.356996in}{1.371914in}}%
\pgfpathlineto{\pgfqpoint{4.364996in}{1.389026in}}%
\pgfpathlineto{\pgfqpoint{4.372994in}{1.406241in}}%
\pgfpathlineto{\pgfqpoint{4.380988in}{1.423554in}}%
\pgfpathlineto{\pgfqpoint{4.366899in}{1.417220in}}%
\pgfpathlineto{\pgfqpoint{4.352824in}{1.411038in}}%
\pgfpathlineto{\pgfqpoint{4.338762in}{1.405009in}}%
\pgfpathlineto{\pgfqpoint{4.324714in}{1.399134in}}%
\pgfpathlineto{\pgfqpoint{4.316724in}{1.382475in}}%
\pgfpathlineto{\pgfqpoint{4.308732in}{1.365922in}}%
\pgfpathlineto{\pgfqpoint{4.300737in}{1.349478in}}%
\pgfpathlineto{\pgfqpoint{4.292739in}{1.333152in}}%
\pgfpathclose%
\pgfusepath{fill}%
\end{pgfscope}%
\begin{pgfscope}%
\pgfpathrectangle{\pgfqpoint{1.254980in}{0.150000in}}{\pgfqpoint{5.490039in}{5.490039in}}%
\pgfusepath{clip}%
\pgfsetbuttcap%
\pgfsetroundjoin%
\definecolor{currentfill}{rgb}{0.233603,0.313828,0.543914}%
\pgfsetfillcolor{currentfill}%
\pgfsetfillopacity{0.700000}%
\pgfsetlinewidth{0.000000pt}%
\definecolor{currentstroke}{rgb}{0.000000,0.000000,0.000000}%
\pgfsetstrokecolor{currentstroke}%
\pgfsetdash{}{0pt}%
\pgfpathmoveto{\pgfqpoint{4.412938in}{1.493660in}}%
\pgfpathlineto{\pgfqpoint{4.427048in}{1.500788in}}%
\pgfpathlineto{\pgfqpoint{4.441171in}{1.508070in}}%
\pgfpathlineto{\pgfqpoint{4.455309in}{1.515505in}}%
\pgfpathlineto{\pgfqpoint{4.469462in}{1.523095in}}%
\pgfpathlineto{\pgfqpoint{4.477451in}{1.541429in}}%
\pgfpathlineto{\pgfqpoint{4.485437in}{1.559818in}}%
\pgfpathlineto{\pgfqpoint{4.493421in}{1.578256in}}%
\pgfpathlineto{\pgfqpoint{4.501402in}{1.596737in}}%
\pgfpathlineto{\pgfqpoint{4.487240in}{1.588532in}}%
\pgfpathlineto{\pgfqpoint{4.473093in}{1.580482in}}%
\pgfpathlineto{\pgfqpoint{4.458961in}{1.572587in}}%
\pgfpathlineto{\pgfqpoint{4.444844in}{1.564846in}}%
\pgfpathlineto{\pgfqpoint{4.436872in}{1.546969in}}%
\pgfpathlineto{\pgfqpoint{4.428897in}{1.529141in}}%
\pgfpathlineto{\pgfqpoint{4.420919in}{1.511370in}}%
\pgfpathlineto{\pgfqpoint{4.412938in}{1.493660in}}%
\pgfpathclose%
\pgfusepath{fill}%
\end{pgfscope}%
\begin{pgfscope}%
\pgfpathrectangle{\pgfqpoint{1.254980in}{0.150000in}}{\pgfqpoint{5.490039in}{5.490039in}}%
\pgfusepath{clip}%
\pgfsetbuttcap%
\pgfsetroundjoin%
\definecolor{currentfill}{rgb}{0.278826,0.175490,0.483397}%
\pgfsetfillcolor{currentfill}%
\pgfsetfillopacity{0.700000}%
\pgfsetlinewidth{0.000000pt}%
\definecolor{currentstroke}{rgb}{0.000000,0.000000,0.000000}%
\pgfsetstrokecolor{currentstroke}%
\pgfsetdash{}{0pt}%
\pgfpathmoveto{\pgfqpoint{4.172590in}{1.193826in}}%
\pgfpathlineto{\pgfqpoint{4.186584in}{1.197026in}}%
\pgfpathlineto{\pgfqpoint{4.200589in}{1.200377in}}%
\pgfpathlineto{\pgfqpoint{4.214606in}{1.203878in}}%
\pgfpathlineto{\pgfqpoint{4.228634in}{1.207531in}}%
\pgfpathlineto{\pgfqpoint{4.236659in}{1.222686in}}%
\pgfpathlineto{\pgfqpoint{4.244680in}{1.238011in}}%
\pgfpathlineto{\pgfqpoint{4.252698in}{1.253500in}}%
\pgfpathlineto{\pgfqpoint{4.260713in}{1.269144in}}%
\pgfpathlineto{\pgfqpoint{4.246684in}{1.264775in}}%
\pgfpathlineto{\pgfqpoint{4.232667in}{1.260557in}}%
\pgfpathlineto{\pgfqpoint{4.218661in}{1.256490in}}%
\pgfpathlineto{\pgfqpoint{4.204668in}{1.252575in}}%
\pgfpathlineto{\pgfqpoint{4.196654in}{1.237637in}}%
\pgfpathlineto{\pgfqpoint{4.188636in}{1.222861in}}%
\pgfpathlineto{\pgfqpoint{4.180615in}{1.208255in}}%
\pgfpathlineto{\pgfqpoint{4.172590in}{1.193826in}}%
\pgfpathclose%
\pgfusepath{fill}%
\end{pgfscope}%
\begin{pgfscope}%
\pgfpathrectangle{\pgfqpoint{1.254980in}{0.150000in}}{\pgfqpoint{5.490039in}{5.490039in}}%
\pgfusepath{clip}%
\pgfsetbuttcap%
\pgfsetroundjoin%
\definecolor{currentfill}{rgb}{0.311925,0.767822,0.415586}%
\pgfsetfillcolor{currentfill}%
\pgfsetfillopacity{0.700000}%
\pgfsetlinewidth{0.000000pt}%
\definecolor{currentstroke}{rgb}{0.000000,0.000000,0.000000}%
\pgfsetstrokecolor{currentstroke}%
\pgfsetdash{}{0pt}%
\pgfpathmoveto{\pgfqpoint{5.142859in}{2.740265in}}%
\pgfpathlineto{\pgfqpoint{5.157494in}{2.756675in}}%
\pgfpathlineto{\pgfqpoint{5.172151in}{2.773254in}}%
\pgfpathlineto{\pgfqpoint{5.186830in}{2.790002in}}%
\pgfpathlineto{\pgfqpoint{5.201531in}{2.806918in}}%
\pgfpathlineto{\pgfqpoint{5.209330in}{2.822509in}}%
\pgfpathlineto{\pgfqpoint{5.217121in}{2.837903in}}%
\pgfpathlineto{\pgfqpoint{5.224904in}{2.853098in}}%
\pgfpathlineto{\pgfqpoint{5.232679in}{2.868092in}}%
\pgfpathlineto{\pgfqpoint{5.217968in}{2.850965in}}%
\pgfpathlineto{\pgfqpoint{5.203279in}{2.834007in}}%
\pgfpathlineto{\pgfqpoint{5.188613in}{2.817218in}}%
\pgfpathlineto{\pgfqpoint{5.173969in}{2.800598in}}%
\pgfpathlineto{\pgfqpoint{5.166203in}{2.785802in}}%
\pgfpathlineto{\pgfqpoint{5.158429in}{2.770813in}}%
\pgfpathlineto{\pgfqpoint{5.150648in}{2.755633in}}%
\pgfpathlineto{\pgfqpoint{5.142859in}{2.740265in}}%
\pgfpathclose%
\pgfusepath{fill}%
\end{pgfscope}%
\begin{pgfscope}%
\pgfpathrectangle{\pgfqpoint{1.254980in}{0.150000in}}{\pgfqpoint{5.490039in}{5.490039in}}%
\pgfusepath{clip}%
\pgfsetbuttcap%
\pgfsetroundjoin%
\definecolor{currentfill}{rgb}{0.197636,0.391528,0.554969}%
\pgfsetfillcolor{currentfill}%
\pgfsetfillopacity{0.700000}%
\pgfsetlinewidth{0.000000pt}%
\definecolor{currentstroke}{rgb}{0.000000,0.000000,0.000000}%
\pgfsetstrokecolor{currentstroke}%
\pgfsetdash{}{0pt}%
\pgfpathmoveto{\pgfqpoint{4.533300in}{1.670983in}}%
\pgfpathlineto{\pgfqpoint{4.547487in}{1.679932in}}%
\pgfpathlineto{\pgfqpoint{4.561690in}{1.689037in}}%
\pgfpathlineto{\pgfqpoint{4.575909in}{1.698298in}}%
\pgfpathlineto{\pgfqpoint{4.590143in}{1.707716in}}%
\pgfpathlineto{\pgfqpoint{4.598122in}{1.726903in}}%
\pgfpathlineto{\pgfqpoint{4.606099in}{1.746094in}}%
\pgfpathlineto{\pgfqpoint{4.614072in}{1.765283in}}%
\pgfpathlineto{\pgfqpoint{4.622043in}{1.784466in}}%
\pgfpathlineto{\pgfqpoint{4.607796in}{1.774485in}}%
\pgfpathlineto{\pgfqpoint{4.593566in}{1.764662in}}%
\pgfpathlineto{\pgfqpoint{4.579352in}{1.754995in}}%
\pgfpathlineto{\pgfqpoint{4.565155in}{1.745486in}}%
\pgfpathlineto{\pgfqpoint{4.557195in}{1.726855in}}%
\pgfpathlineto{\pgfqpoint{4.549233in}{1.708224in}}%
\pgfpathlineto{\pgfqpoint{4.541268in}{1.689598in}}%
\pgfpathlineto{\pgfqpoint{4.533300in}{1.670983in}}%
\pgfpathclose%
\pgfusepath{fill}%
\end{pgfscope}%
\begin{pgfscope}%
\pgfpathrectangle{\pgfqpoint{1.254980in}{0.150000in}}{\pgfqpoint{5.490039in}{5.490039in}}%
\pgfusepath{clip}%
\pgfsetbuttcap%
\pgfsetroundjoin%
\definecolor{currentfill}{rgb}{0.156270,0.489624,0.557936}%
\pgfsetfillcolor{currentfill}%
\pgfsetfillopacity{0.700000}%
\pgfsetlinewidth{0.000000pt}%
\definecolor{currentstroke}{rgb}{0.000000,0.000000,0.000000}%
\pgfsetstrokecolor{currentstroke}%
\pgfsetdash{}{0pt}%
\pgfpathmoveto{\pgfqpoint{4.685702in}{1.937091in}}%
\pgfpathlineto{\pgfqpoint{4.699990in}{1.948272in}}%
\pgfpathlineto{\pgfqpoint{4.714296in}{1.959613in}}%
\pgfpathlineto{\pgfqpoint{4.728619in}{1.971114in}}%
\pgfpathlineto{\pgfqpoint{4.742960in}{1.982774in}}%
\pgfpathlineto{\pgfqpoint{4.750916in}{2.002167in}}%
\pgfpathlineto{\pgfqpoint{4.758869in}{2.021498in}}%
\pgfpathlineto{\pgfqpoint{4.766818in}{2.040763in}}%
\pgfpathlineto{\pgfqpoint{4.774764in}{2.059958in}}%
\pgfpathlineto{\pgfqpoint{4.760409in}{2.047817in}}%
\pgfpathlineto{\pgfqpoint{4.746073in}{2.035836in}}%
\pgfpathlineto{\pgfqpoint{4.731754in}{2.024016in}}%
\pgfpathlineto{\pgfqpoint{4.717453in}{2.012355in}}%
\pgfpathlineto{\pgfqpoint{4.709521in}{1.993630in}}%
\pgfpathlineto{\pgfqpoint{4.701585in}{1.974841in}}%
\pgfpathlineto{\pgfqpoint{4.693645in}{1.955993in}}%
\pgfpathlineto{\pgfqpoint{4.685702in}{1.937091in}}%
\pgfpathclose%
\pgfusepath{fill}%
\end{pgfscope}%
\begin{pgfscope}%
\pgfpathrectangle{\pgfqpoint{1.254980in}{0.150000in}}{\pgfqpoint{5.490039in}{5.490039in}}%
\pgfusepath{clip}%
\pgfsetbuttcap%
\pgfsetroundjoin%
\definecolor{currentfill}{rgb}{0.458674,0.816363,0.329727}%
\pgfsetfillcolor{currentfill}%
\pgfsetfillopacity{0.700000}%
\pgfsetlinewidth{0.000000pt}%
\definecolor{currentstroke}{rgb}{0.000000,0.000000,0.000000}%
\pgfsetstrokecolor{currentstroke}%
\pgfsetdash{}{0pt}%
\pgfpathmoveto{\pgfqpoint{5.263693in}{2.926032in}}%
\pgfpathlineto{\pgfqpoint{5.278435in}{2.943508in}}%
\pgfpathlineto{\pgfqpoint{5.293199in}{2.961154in}}%
\pgfpathlineto{\pgfqpoint{5.307987in}{2.978972in}}%
\pgfpathlineto{\pgfqpoint{5.315724in}{2.993060in}}%
\pgfpathlineto{\pgfqpoint{5.323453in}{3.006932in}}%
\pgfpathlineto{\pgfqpoint{5.331172in}{3.020589in}}%
\pgfpathlineto{\pgfqpoint{5.338881in}{3.034029in}}%
\pgfpathlineto{\pgfqpoint{5.324086in}{3.016066in}}%
\pgfpathlineto{\pgfqpoint{5.309315in}{2.998273in}}%
\pgfpathlineto{\pgfqpoint{5.294566in}{2.980652in}}%
\pgfpathlineto{\pgfqpoint{5.286861in}{2.967312in}}%
\pgfpathlineto{\pgfqpoint{5.279147in}{2.953762in}}%
\pgfpathlineto{\pgfqpoint{5.271425in}{2.940001in}}%
\pgfpathlineto{\pgfqpoint{5.263693in}{2.926032in}}%
\pgfpathclose%
\pgfusepath{fill}%
\end{pgfscope}%
\begin{pgfscope}%
\pgfpathrectangle{\pgfqpoint{1.254980in}{0.150000in}}{\pgfqpoint{5.490039in}{5.490039in}}%
\pgfusepath{clip}%
\pgfsetbuttcap%
\pgfsetroundjoin%
\definecolor{currentfill}{rgb}{0.121831,0.589055,0.545623}%
\pgfsetfillcolor{currentfill}%
\pgfsetfillopacity{0.700000}%
\pgfsetlinewidth{0.000000pt}%
\definecolor{currentstroke}{rgb}{0.000000,0.000000,0.000000}%
\pgfsetstrokecolor{currentstroke}%
\pgfsetdash{}{0pt}%
\pgfpathmoveto{\pgfqpoint{4.838189in}{2.210490in}}%
\pgfpathlineto{\pgfqpoint{4.852589in}{2.223667in}}%
\pgfpathlineto{\pgfqpoint{4.867008in}{2.237007in}}%
\pgfpathlineto{\pgfqpoint{4.881446in}{2.250510in}}%
\pgfpathlineto{\pgfqpoint{4.895904in}{2.264176in}}%
\pgfpathlineto{\pgfqpoint{4.903826in}{2.282961in}}%
\pgfpathlineto{\pgfqpoint{4.911744in}{2.301628in}}%
\pgfpathlineto{\pgfqpoint{4.919658in}{2.320175in}}%
\pgfpathlineto{\pgfqpoint{4.927566in}{2.338597in}}%
\pgfpathlineto{\pgfqpoint{4.913094in}{2.324536in}}%
\pgfpathlineto{\pgfqpoint{4.898643in}{2.310639in}}%
\pgfpathlineto{\pgfqpoint{4.884210in}{2.296905in}}%
\pgfpathlineto{\pgfqpoint{4.869797in}{2.283335in}}%
\pgfpathlineto{\pgfqpoint{4.861902in}{2.265295in}}%
\pgfpathlineto{\pgfqpoint{4.854002in}{2.247139in}}%
\pgfpathlineto{\pgfqpoint{4.846098in}{2.228869in}}%
\pgfpathlineto{\pgfqpoint{4.838189in}{2.210490in}}%
\pgfpathclose%
\pgfusepath{fill}%
\end{pgfscope}%
\begin{pgfscope}%
\pgfpathrectangle{\pgfqpoint{1.254980in}{0.150000in}}{\pgfqpoint{5.490039in}{5.490039in}}%
\pgfusepath{clip}%
\pgfsetbuttcap%
\pgfsetroundjoin%
\definecolor{currentfill}{rgb}{0.157851,0.683765,0.501686}%
\pgfsetfillcolor{currentfill}%
\pgfsetfillopacity{0.700000}%
\pgfsetlinewidth{0.000000pt}%
\definecolor{currentstroke}{rgb}{0.000000,0.000000,0.000000}%
\pgfsetstrokecolor{currentstroke}%
\pgfsetdash{}{0pt}%
\pgfpathmoveto{\pgfqpoint{4.990643in}{2.481117in}}%
\pgfpathlineto{\pgfqpoint{5.005160in}{2.496041in}}%
\pgfpathlineto{\pgfqpoint{5.019698in}{2.511132in}}%
\pgfpathlineto{\pgfqpoint{5.034257in}{2.526388in}}%
\pgfpathlineto{\pgfqpoint{5.048836in}{2.541811in}}%
\pgfpathlineto{\pgfqpoint{5.056708in}{2.559288in}}%
\pgfpathlineto{\pgfqpoint{5.064574in}{2.576603in}}%
\pgfpathlineto{\pgfqpoint{5.072433in}{2.593752in}}%
\pgfpathlineto{\pgfqpoint{5.080285in}{2.610734in}}%
\pgfpathlineto{\pgfqpoint{5.065694in}{2.595007in}}%
\pgfpathlineto{\pgfqpoint{5.051123in}{2.579446in}}%
\pgfpathlineto{\pgfqpoint{5.036573in}{2.564052in}}%
\pgfpathlineto{\pgfqpoint{5.022043in}{2.548824in}}%
\pgfpathlineto{\pgfqpoint{5.014203in}{2.532134in}}%
\pgfpathlineto{\pgfqpoint{5.006355in}{2.515285in}}%
\pgfpathlineto{\pgfqpoint{4.998502in}{2.498278in}}%
\pgfpathlineto{\pgfqpoint{4.990643in}{2.481117in}}%
\pgfpathclose%
\pgfusepath{fill}%
\end{pgfscope}%
\begin{pgfscope}%
\pgfpathrectangle{\pgfqpoint{1.254980in}{0.150000in}}{\pgfqpoint{5.490039in}{5.490039in}}%
\pgfusepath{clip}%
\pgfsetbuttcap%
\pgfsetroundjoin%
\definecolor{currentfill}{rgb}{0.269308,0.218818,0.509577}%
\pgfsetfillcolor{currentfill}%
\pgfsetfillopacity{0.700000}%
\pgfsetlinewidth{0.000000pt}%
\definecolor{currentstroke}{rgb}{0.000000,0.000000,0.000000}%
\pgfsetstrokecolor{currentstroke}%
\pgfsetdash{}{0pt}%
\pgfpathmoveto{\pgfqpoint{4.260713in}{1.269144in}}%
\pgfpathlineto{\pgfqpoint{4.274754in}{1.273665in}}%
\pgfpathlineto{\pgfqpoint{4.288808in}{1.278338in}}%
\pgfpathlineto{\pgfqpoint{4.302874in}{1.283161in}}%
\pgfpathlineto{\pgfqpoint{4.316953in}{1.288136in}}%
\pgfpathlineto{\pgfqpoint{4.324968in}{1.304633in}}%
\pgfpathlineto{\pgfqpoint{4.332979in}{1.321265in}}%
\pgfpathlineto{\pgfqpoint{4.340988in}{1.338027in}}%
\pgfpathlineto{\pgfqpoint{4.348993in}{1.354912in}}%
\pgfpathlineto{\pgfqpoint{4.334910in}{1.349244in}}%
\pgfpathlineto{\pgfqpoint{4.320840in}{1.343728in}}%
\pgfpathlineto{\pgfqpoint{4.306783in}{1.338364in}}%
\pgfpathlineto{\pgfqpoint{4.292739in}{1.333152in}}%
\pgfpathlineto{\pgfqpoint{4.284737in}{1.316949in}}%
\pgfpathlineto{\pgfqpoint{4.276732in}{1.300876in}}%
\pgfpathlineto{\pgfqpoint{4.268724in}{1.284939in}}%
\pgfpathlineto{\pgfqpoint{4.260713in}{1.269144in}}%
\pgfpathclose%
\pgfusepath{fill}%
\end{pgfscope}%
\begin{pgfscope}%
\pgfpathrectangle{\pgfqpoint{1.254980in}{0.150000in}}{\pgfqpoint{5.490039in}{5.490039in}}%
\pgfusepath{clip}%
\pgfsetbuttcap%
\pgfsetroundjoin%
\definecolor{currentfill}{rgb}{0.244972,0.287675,0.537260}%
\pgfsetfillcolor{currentfill}%
\pgfsetfillopacity{0.700000}%
\pgfsetlinewidth{0.000000pt}%
\definecolor{currentstroke}{rgb}{0.000000,0.000000,0.000000}%
\pgfsetstrokecolor{currentstroke}%
\pgfsetdash{}{0pt}%
\pgfpathmoveto{\pgfqpoint{4.380988in}{1.423554in}}%
\pgfpathlineto{\pgfqpoint{4.395090in}{1.430042in}}%
\pgfpathlineto{\pgfqpoint{4.409207in}{1.436683in}}%
\pgfpathlineto{\pgfqpoint{4.423337in}{1.443477in}}%
\pgfpathlineto{\pgfqpoint{4.437481in}{1.450425in}}%
\pgfpathlineto{\pgfqpoint{4.445480in}{1.468481in}}%
\pgfpathlineto{\pgfqpoint{4.453476in}{1.486615in}}%
\pgfpathlineto{\pgfqpoint{4.461470in}{1.504822in}}%
\pgfpathlineto{\pgfqpoint{4.469462in}{1.523095in}}%
\pgfpathlineto{\pgfqpoint{4.455309in}{1.515505in}}%
\pgfpathlineto{\pgfqpoint{4.441171in}{1.508070in}}%
\pgfpathlineto{\pgfqpoint{4.427048in}{1.500788in}}%
\pgfpathlineto{\pgfqpoint{4.412938in}{1.493660in}}%
\pgfpathlineto{\pgfqpoint{4.404955in}{1.476018in}}%
\pgfpathlineto{\pgfqpoint{4.396969in}{1.458449in}}%
\pgfpathlineto{\pgfqpoint{4.388980in}{1.440959in}}%
\pgfpathlineto{\pgfqpoint{4.380988in}{1.423554in}}%
\pgfpathclose%
\pgfusepath{fill}%
\end{pgfscope}%
\begin{pgfscope}%
\pgfpathrectangle{\pgfqpoint{1.254980in}{0.150000in}}{\pgfqpoint{5.490039in}{5.490039in}}%
\pgfusepath{clip}%
\pgfsetbuttcap%
\pgfsetroundjoin%
\definecolor{currentfill}{rgb}{0.165117,0.467423,0.558141}%
\pgfsetfillcolor{currentfill}%
\pgfsetfillopacity{0.700000}%
\pgfsetlinewidth{0.000000pt}%
\definecolor{currentstroke}{rgb}{0.000000,0.000000,0.000000}%
\pgfsetstrokecolor{currentstroke}%
\pgfsetdash{}{0pt}%
\pgfpathmoveto{\pgfqpoint{4.653897in}{1.861029in}}%
\pgfpathlineto{\pgfqpoint{4.668173in}{1.871703in}}%
\pgfpathlineto{\pgfqpoint{4.682465in}{1.882536in}}%
\pgfpathlineto{\pgfqpoint{4.696775in}{1.893527in}}%
\pgfpathlineto{\pgfqpoint{4.711102in}{1.904677in}}%
\pgfpathlineto{\pgfqpoint{4.719071in}{1.924271in}}%
\pgfpathlineto{\pgfqpoint{4.727037in}{1.943822in}}%
\pgfpathlineto{\pgfqpoint{4.735000in}{1.963324in}}%
\pgfpathlineto{\pgfqpoint{4.742960in}{1.982774in}}%
\pgfpathlineto{\pgfqpoint{4.728619in}{1.971114in}}%
\pgfpathlineto{\pgfqpoint{4.714296in}{1.959613in}}%
\pgfpathlineto{\pgfqpoint{4.699990in}{1.948272in}}%
\pgfpathlineto{\pgfqpoint{4.685702in}{1.937091in}}%
\pgfpathlineto{\pgfqpoint{4.677756in}{1.918139in}}%
\pgfpathlineto{\pgfqpoint{4.669806in}{1.899141in}}%
\pgfpathlineto{\pgfqpoint{4.661853in}{1.880103in}}%
\pgfpathlineto{\pgfqpoint{4.653897in}{1.861029in}}%
\pgfpathclose%
\pgfusepath{fill}%
\end{pgfscope}%
\begin{pgfscope}%
\pgfpathrectangle{\pgfqpoint{1.254980in}{0.150000in}}{\pgfqpoint{5.490039in}{5.490039in}}%
\pgfusepath{clip}%
\pgfsetbuttcap%
\pgfsetroundjoin%
\definecolor{currentfill}{rgb}{0.208623,0.367752,0.552675}%
\pgfsetfillcolor{currentfill}%
\pgfsetfillopacity{0.700000}%
\pgfsetlinewidth{0.000000pt}%
\definecolor{currentstroke}{rgb}{0.000000,0.000000,0.000000}%
\pgfsetstrokecolor{currentstroke}%
\pgfsetdash{}{0pt}%
\pgfpathmoveto{\pgfqpoint{4.501402in}{1.596737in}}%
\pgfpathlineto{\pgfqpoint{4.515579in}{1.605097in}}%
\pgfpathlineto{\pgfqpoint{4.529771in}{1.613613in}}%
\pgfpathlineto{\pgfqpoint{4.543978in}{1.622284in}}%
\pgfpathlineto{\pgfqpoint{4.558201in}{1.631111in}}%
\pgfpathlineto{\pgfqpoint{4.566191in}{1.650229in}}%
\pgfpathlineto{\pgfqpoint{4.574177in}{1.669374in}}%
\pgfpathlineto{\pgfqpoint{4.582162in}{1.688538in}}%
\pgfpathlineto{\pgfqpoint{4.590143in}{1.707716in}}%
\pgfpathlineto{\pgfqpoint{4.575909in}{1.698298in}}%
\pgfpathlineto{\pgfqpoint{4.561690in}{1.689037in}}%
\pgfpathlineto{\pgfqpoint{4.547487in}{1.679932in}}%
\pgfpathlineto{\pgfqpoint{4.533300in}{1.670983in}}%
\pgfpathlineto{\pgfqpoint{4.525330in}{1.652385in}}%
\pgfpathlineto{\pgfqpoint{4.517357in}{1.633807in}}%
\pgfpathlineto{\pgfqpoint{4.509381in}{1.615256in}}%
\pgfpathlineto{\pgfqpoint{4.501402in}{1.596737in}}%
\pgfpathclose%
\pgfusepath{fill}%
\end{pgfscope}%
\begin{pgfscope}%
\pgfpathrectangle{\pgfqpoint{1.254980in}{0.150000in}}{\pgfqpoint{5.490039in}{5.490039in}}%
\pgfusepath{clip}%
\pgfsetbuttcap%
\pgfsetroundjoin%
\definecolor{currentfill}{rgb}{0.127568,0.566949,0.550556}%
\pgfsetfillcolor{currentfill}%
\pgfsetfillopacity{0.700000}%
\pgfsetlinewidth{0.000000pt}%
\definecolor{currentstroke}{rgb}{0.000000,0.000000,0.000000}%
\pgfsetstrokecolor{currentstroke}%
\pgfsetdash{}{0pt}%
\pgfpathmoveto{\pgfqpoint{4.806509in}{2.135950in}}%
\pgfpathlineto{\pgfqpoint{4.820895in}{2.148705in}}%
\pgfpathlineto{\pgfqpoint{4.835301in}{2.161621in}}%
\pgfpathlineto{\pgfqpoint{4.849725in}{2.174700in}}%
\pgfpathlineto{\pgfqpoint{4.864168in}{2.187940in}}%
\pgfpathlineto{\pgfqpoint{4.872108in}{2.207156in}}%
\pgfpathlineto{\pgfqpoint{4.880045in}{2.226270in}}%
\pgfpathlineto{\pgfqpoint{4.887976in}{2.245278in}}%
\pgfpathlineto{\pgfqpoint{4.895904in}{2.264176in}}%
\pgfpathlineto{\pgfqpoint{4.881446in}{2.250510in}}%
\pgfpathlineto{\pgfqpoint{4.867008in}{2.237007in}}%
\pgfpathlineto{\pgfqpoint{4.852589in}{2.223667in}}%
\pgfpathlineto{\pgfqpoint{4.838189in}{2.210490in}}%
\pgfpathlineto{\pgfqpoint{4.830275in}{2.192005in}}%
\pgfpathlineto{\pgfqpoint{4.822357in}{2.173417in}}%
\pgfpathlineto{\pgfqpoint{4.814435in}{2.154731in}}%
\pgfpathlineto{\pgfqpoint{4.806509in}{2.135950in}}%
\pgfpathclose%
\pgfusepath{fill}%
\end{pgfscope}%
\begin{pgfscope}%
\pgfpathrectangle{\pgfqpoint{1.254980in}{0.150000in}}{\pgfqpoint{5.490039in}{5.490039in}}%
\pgfusepath{clip}%
\pgfsetbuttcap%
\pgfsetroundjoin%
\definecolor{currentfill}{rgb}{0.274149,0.751988,0.436601}%
\pgfsetfillcolor{currentfill}%
\pgfsetfillopacity{0.700000}%
\pgfsetlinewidth{0.000000pt}%
\definecolor{currentstroke}{rgb}{0.000000,0.000000,0.000000}%
\pgfsetstrokecolor{currentstroke}%
\pgfsetdash{}{0pt}%
\pgfpathmoveto{\pgfqpoint{5.111629in}{2.676934in}}%
\pgfpathlineto{\pgfqpoint{5.126254in}{2.693103in}}%
\pgfpathlineto{\pgfqpoint{5.140900in}{2.709440in}}%
\pgfpathlineto{\pgfqpoint{5.155568in}{2.725945in}}%
\pgfpathlineto{\pgfqpoint{5.170258in}{2.742618in}}%
\pgfpathlineto{\pgfqpoint{5.178088in}{2.758980in}}%
\pgfpathlineto{\pgfqpoint{5.185910in}{2.775151in}}%
\pgfpathlineto{\pgfqpoint{5.193725in}{2.791132in}}%
\pgfpathlineto{\pgfqpoint{5.201531in}{2.806918in}}%
\pgfpathlineto{\pgfqpoint{5.186830in}{2.790002in}}%
\pgfpathlineto{\pgfqpoint{5.172151in}{2.773254in}}%
\pgfpathlineto{\pgfqpoint{5.157494in}{2.756675in}}%
\pgfpathlineto{\pgfqpoint{5.142859in}{2.740265in}}%
\pgfpathlineto{\pgfqpoint{5.135063in}{2.724708in}}%
\pgfpathlineto{\pgfqpoint{5.127259in}{2.708967in}}%
\pgfpathlineto{\pgfqpoint{5.119447in}{2.693041in}}%
\pgfpathlineto{\pgfqpoint{5.111629in}{2.676934in}}%
\pgfpathclose%
\pgfusepath{fill}%
\end{pgfscope}%
\begin{pgfscope}%
\pgfpathrectangle{\pgfqpoint{1.254980in}{0.150000in}}{\pgfqpoint{5.490039in}{5.490039in}}%
\pgfusepath{clip}%
\pgfsetbuttcap%
\pgfsetroundjoin%
\definecolor{currentfill}{rgb}{0.140210,0.665859,0.513427}%
\pgfsetfillcolor{currentfill}%
\pgfsetfillopacity{0.700000}%
\pgfsetlinewidth{0.000000pt}%
\definecolor{currentstroke}{rgb}{0.000000,0.000000,0.000000}%
\pgfsetstrokecolor{currentstroke}%
\pgfsetdash{}{0pt}%
\pgfpathmoveto{\pgfqpoint{4.959148in}{2.410979in}}%
\pgfpathlineto{\pgfqpoint{4.973653in}{2.425570in}}%
\pgfpathlineto{\pgfqpoint{4.988178in}{2.440325in}}%
\pgfpathlineto{\pgfqpoint{5.002723in}{2.455246in}}%
\pgfpathlineto{\pgfqpoint{5.017289in}{2.470332in}}%
\pgfpathlineto{\pgfqpoint{5.025185in}{2.488432in}}%
\pgfpathlineto{\pgfqpoint{5.033074in}{2.506380in}}%
\pgfpathlineto{\pgfqpoint{5.040958in}{2.524174in}}%
\pgfpathlineto{\pgfqpoint{5.048836in}{2.541811in}}%
\pgfpathlineto{\pgfqpoint{5.034257in}{2.526388in}}%
\pgfpathlineto{\pgfqpoint{5.019698in}{2.511132in}}%
\pgfpathlineto{\pgfqpoint{5.005160in}{2.496041in}}%
\pgfpathlineto{\pgfqpoint{4.990643in}{2.481117in}}%
\pgfpathlineto{\pgfqpoint{4.982778in}{2.463803in}}%
\pgfpathlineto{\pgfqpoint{4.974907in}{2.446341in}}%
\pgfpathlineto{\pgfqpoint{4.967030in}{2.428732in}}%
\pgfpathlineto{\pgfqpoint{4.959148in}{2.410979in}}%
\pgfpathclose%
\pgfusepath{fill}%
\end{pgfscope}%
\begin{pgfscope}%
\pgfpathrectangle{\pgfqpoint{1.254980in}{0.150000in}}{\pgfqpoint{5.490039in}{5.490039in}}%
\pgfusepath{clip}%
\pgfsetbuttcap%
\pgfsetroundjoin%
\definecolor{currentfill}{rgb}{0.253935,0.265254,0.529983}%
\pgfsetfillcolor{currentfill}%
\pgfsetfillopacity{0.700000}%
\pgfsetlinewidth{0.000000pt}%
\definecolor{currentstroke}{rgb}{0.000000,0.000000,0.000000}%
\pgfsetstrokecolor{currentstroke}%
\pgfsetdash{}{0pt}%
\pgfpathmoveto{\pgfqpoint{4.348993in}{1.354912in}}%
\pgfpathlineto{\pgfqpoint{4.363090in}{1.360733in}}%
\pgfpathlineto{\pgfqpoint{4.377199in}{1.366706in}}%
\pgfpathlineto{\pgfqpoint{4.391322in}{1.372832in}}%
\pgfpathlineto{\pgfqpoint{4.405459in}{1.379110in}}%
\pgfpathlineto{\pgfqpoint{4.413468in}{1.396790in}}%
\pgfpathlineto{\pgfqpoint{4.421475in}{1.414573in}}%
\pgfpathlineto{\pgfqpoint{4.429479in}{1.432453in}}%
\pgfpathlineto{\pgfqpoint{4.437481in}{1.450425in}}%
\pgfpathlineto{\pgfqpoint{4.423337in}{1.443477in}}%
\pgfpathlineto{\pgfqpoint{4.409207in}{1.436683in}}%
\pgfpathlineto{\pgfqpoint{4.395090in}{1.430042in}}%
\pgfpathlineto{\pgfqpoint{4.380988in}{1.423554in}}%
\pgfpathlineto{\pgfqpoint{4.372994in}{1.406241in}}%
\pgfpathlineto{\pgfqpoint{4.364996in}{1.389026in}}%
\pgfpathlineto{\pgfqpoint{4.356996in}{1.371914in}}%
\pgfpathlineto{\pgfqpoint{4.348993in}{1.354912in}}%
\pgfpathclose%
\pgfusepath{fill}%
\end{pgfscope}%
\begin{pgfscope}%
\pgfpathrectangle{\pgfqpoint{1.254980in}{0.150000in}}{\pgfqpoint{5.490039in}{5.490039in}}%
\pgfusepath{clip}%
\pgfsetbuttcap%
\pgfsetroundjoin%
\definecolor{currentfill}{rgb}{0.421908,0.805774,0.351910}%
\pgfsetfillcolor{currentfill}%
\pgfsetfillopacity{0.700000}%
\pgfsetlinewidth{0.000000pt}%
\definecolor{currentstroke}{rgb}{0.000000,0.000000,0.000000}%
\pgfsetstrokecolor{currentstroke}%
\pgfsetdash{}{0pt}%
\pgfpathmoveto{\pgfqpoint{5.232679in}{2.868092in}}%
\pgfpathlineto{\pgfqpoint{5.247412in}{2.885390in}}%
\pgfpathlineto{\pgfqpoint{5.262168in}{2.902857in}}%
\pgfpathlineto{\pgfqpoint{5.276947in}{2.920495in}}%
\pgfpathlineto{\pgfqpoint{5.284720in}{2.935430in}}%
\pgfpathlineto{\pgfqpoint{5.292485in}{2.950156in}}%
\pgfpathlineto{\pgfqpoint{5.300240in}{2.964670in}}%
\pgfpathlineto{\pgfqpoint{5.307987in}{2.978972in}}%
\pgfpathlineto{\pgfqpoint{5.293199in}{2.961154in}}%
\pgfpathlineto{\pgfqpoint{5.278435in}{2.943508in}}%
\pgfpathlineto{\pgfqpoint{5.263693in}{2.926032in}}%
\pgfpathlineto{\pgfqpoint{5.255952in}{2.911855in}}%
\pgfpathlineto{\pgfqpoint{5.248203in}{2.897472in}}%
\pgfpathlineto{\pgfqpoint{5.240445in}{2.882884in}}%
\pgfpathlineto{\pgfqpoint{5.232679in}{2.868092in}}%
\pgfpathclose%
\pgfusepath{fill}%
\end{pgfscope}%
\begin{pgfscope}%
\pgfpathrectangle{\pgfqpoint{1.254980in}{0.150000in}}{\pgfqpoint{5.490039in}{5.490039in}}%
\pgfusepath{clip}%
\pgfsetbuttcap%
\pgfsetroundjoin%
\definecolor{currentfill}{rgb}{0.275191,0.194905,0.496005}%
\pgfsetfillcolor{currentfill}%
\pgfsetfillopacity{0.700000}%
\pgfsetlinewidth{0.000000pt}%
\definecolor{currentstroke}{rgb}{0.000000,0.000000,0.000000}%
\pgfsetstrokecolor{currentstroke}%
\pgfsetdash{}{0pt}%
\pgfpathmoveto{\pgfqpoint{4.228634in}{1.207531in}}%
\pgfpathlineto{\pgfqpoint{4.242674in}{1.211333in}}%
\pgfpathlineto{\pgfqpoint{4.256725in}{1.215287in}}%
\pgfpathlineto{\pgfqpoint{4.270789in}{1.219391in}}%
\pgfpathlineto{\pgfqpoint{4.284865in}{1.223646in}}%
\pgfpathlineto{\pgfqpoint{4.292892in}{1.239531in}}%
\pgfpathlineto{\pgfqpoint{4.300915in}{1.255578in}}%
\pgfpathlineto{\pgfqpoint{4.308936in}{1.271783in}}%
\pgfpathlineto{\pgfqpoint{4.316953in}{1.288136in}}%
\pgfpathlineto{\pgfqpoint{4.302874in}{1.283161in}}%
\pgfpathlineto{\pgfqpoint{4.288808in}{1.278338in}}%
\pgfpathlineto{\pgfqpoint{4.274754in}{1.273665in}}%
\pgfpathlineto{\pgfqpoint{4.260713in}{1.269144in}}%
\pgfpathlineto{\pgfqpoint{4.252698in}{1.253500in}}%
\pgfpathlineto{\pgfqpoint{4.244680in}{1.238011in}}%
\pgfpathlineto{\pgfqpoint{4.236659in}{1.222686in}}%
\pgfpathlineto{\pgfqpoint{4.228634in}{1.207531in}}%
\pgfpathclose%
\pgfusepath{fill}%
\end{pgfscope}%
\begin{pgfscope}%
\pgfpathrectangle{\pgfqpoint{1.254980in}{0.150000in}}{\pgfqpoint{5.490039in}{5.490039in}}%
\pgfusepath{clip}%
\pgfsetbuttcap%
\pgfsetroundjoin%
\definecolor{currentfill}{rgb}{0.175841,0.441290,0.557685}%
\pgfsetfillcolor{currentfill}%
\pgfsetfillopacity{0.700000}%
\pgfsetlinewidth{0.000000pt}%
\definecolor{currentstroke}{rgb}{0.000000,0.000000,0.000000}%
\pgfsetstrokecolor{currentstroke}%
\pgfsetdash{}{0pt}%
\pgfpathmoveto{\pgfqpoint{4.622043in}{1.784466in}}%
\pgfpathlineto{\pgfqpoint{4.636306in}{1.794604in}}%
\pgfpathlineto{\pgfqpoint{4.650585in}{1.804900in}}%
\pgfpathlineto{\pgfqpoint{4.664882in}{1.815354in}}%
\pgfpathlineto{\pgfqpoint{4.679195in}{1.825965in}}%
\pgfpathlineto{\pgfqpoint{4.687176in}{1.845684in}}%
\pgfpathlineto{\pgfqpoint{4.695154in}{1.865379in}}%
\pgfpathlineto{\pgfqpoint{4.703129in}{1.885045in}}%
\pgfpathlineto{\pgfqpoint{4.711102in}{1.904677in}}%
\pgfpathlineto{\pgfqpoint{4.696775in}{1.893527in}}%
\pgfpathlineto{\pgfqpoint{4.682465in}{1.882536in}}%
\pgfpathlineto{\pgfqpoint{4.668173in}{1.871703in}}%
\pgfpathlineto{\pgfqpoint{4.653897in}{1.861029in}}%
\pgfpathlineto{\pgfqpoint{4.645938in}{1.841923in}}%
\pgfpathlineto{\pgfqpoint{4.637976in}{1.822791in}}%
\pgfpathlineto{\pgfqpoint{4.630011in}{1.803637in}}%
\pgfpathlineto{\pgfqpoint{4.622043in}{1.784466in}}%
\pgfpathclose%
\pgfusepath{fill}%
\end{pgfscope}%
\begin{pgfscope}%
\pgfpathrectangle{\pgfqpoint{1.254980in}{0.150000in}}{\pgfqpoint{5.490039in}{5.490039in}}%
\pgfusepath{clip}%
\pgfsetbuttcap%
\pgfsetroundjoin%
\definecolor{currentfill}{rgb}{0.220057,0.343307,0.549413}%
\pgfsetfillcolor{currentfill}%
\pgfsetfillopacity{0.700000}%
\pgfsetlinewidth{0.000000pt}%
\definecolor{currentstroke}{rgb}{0.000000,0.000000,0.000000}%
\pgfsetstrokecolor{currentstroke}%
\pgfsetdash{}{0pt}%
\pgfpathmoveto{\pgfqpoint{4.469462in}{1.523095in}}%
\pgfpathlineto{\pgfqpoint{4.483629in}{1.530840in}}%
\pgfpathlineto{\pgfqpoint{4.497811in}{1.538738in}}%
\pgfpathlineto{\pgfqpoint{4.512008in}{1.546792in}}%
\pgfpathlineto{\pgfqpoint{4.526220in}{1.554999in}}%
\pgfpathlineto{\pgfqpoint{4.534219in}{1.573961in}}%
\pgfpathlineto{\pgfqpoint{4.542216in}{1.592971in}}%
\pgfpathlineto{\pgfqpoint{4.550210in}{1.612022in}}%
\pgfpathlineto{\pgfqpoint{4.558201in}{1.631111in}}%
\pgfpathlineto{\pgfqpoint{4.543978in}{1.622284in}}%
\pgfpathlineto{\pgfqpoint{4.529771in}{1.613613in}}%
\pgfpathlineto{\pgfqpoint{4.515579in}{1.605097in}}%
\pgfpathlineto{\pgfqpoint{4.501402in}{1.596737in}}%
\pgfpathlineto{\pgfqpoint{4.493421in}{1.578256in}}%
\pgfpathlineto{\pgfqpoint{4.485437in}{1.559818in}}%
\pgfpathlineto{\pgfqpoint{4.477451in}{1.541429in}}%
\pgfpathlineto{\pgfqpoint{4.469462in}{1.523095in}}%
\pgfpathclose%
\pgfusepath{fill}%
\end{pgfscope}%
\begin{pgfscope}%
\pgfpathrectangle{\pgfqpoint{1.254980in}{0.150000in}}{\pgfqpoint{5.490039in}{5.490039in}}%
\pgfusepath{clip}%
\pgfsetbuttcap%
\pgfsetroundjoin%
\definecolor{currentfill}{rgb}{0.135066,0.544853,0.554029}%
\pgfsetfillcolor{currentfill}%
\pgfsetfillopacity{0.700000}%
\pgfsetlinewidth{0.000000pt}%
\definecolor{currentstroke}{rgb}{0.000000,0.000000,0.000000}%
\pgfsetstrokecolor{currentstroke}%
\pgfsetdash{}{0pt}%
\pgfpathmoveto{\pgfqpoint{4.774764in}{2.059958in}}%
\pgfpathlineto{\pgfqpoint{4.789137in}{2.072260in}}%
\pgfpathlineto{\pgfqpoint{4.803528in}{2.084723in}}%
\pgfpathlineto{\pgfqpoint{4.817937in}{2.097347in}}%
\pgfpathlineto{\pgfqpoint{4.832366in}{2.110133in}}%
\pgfpathlineto{\pgfqpoint{4.840322in}{2.129718in}}%
\pgfpathlineto{\pgfqpoint{4.848275in}{2.149217in}}%
\pgfpathlineto{\pgfqpoint{4.856223in}{2.168626in}}%
\pgfpathlineto{\pgfqpoint{4.864168in}{2.187940in}}%
\pgfpathlineto{\pgfqpoint{4.849725in}{2.174700in}}%
\pgfpathlineto{\pgfqpoint{4.835301in}{2.161621in}}%
\pgfpathlineto{\pgfqpoint{4.820895in}{2.148705in}}%
\pgfpathlineto{\pgfqpoint{4.806509in}{2.135950in}}%
\pgfpathlineto{\pgfqpoint{4.798579in}{2.117078in}}%
\pgfpathlineto{\pgfqpoint{4.790644in}{2.098120in}}%
\pgfpathlineto{\pgfqpoint{4.782706in}{2.079078in}}%
\pgfpathlineto{\pgfqpoint{4.774764in}{2.059958in}}%
\pgfpathclose%
\pgfusepath{fill}%
\end{pgfscope}%
\begin{pgfscope}%
\pgfpathrectangle{\pgfqpoint{1.254980in}{0.150000in}}{\pgfqpoint{5.490039in}{5.490039in}}%
\pgfusepath{clip}%
\pgfsetbuttcap%
\pgfsetroundjoin%
\definecolor{currentfill}{rgb}{0.126326,0.644107,0.525311}%
\pgfsetfillcolor{currentfill}%
\pgfsetfillopacity{0.700000}%
\pgfsetlinewidth{0.000000pt}%
\definecolor{currentstroke}{rgb}{0.000000,0.000000,0.000000}%
\pgfsetstrokecolor{currentstroke}%
\pgfsetdash{}{0pt}%
\pgfpathmoveto{\pgfqpoint{4.927566in}{2.338597in}}%
\pgfpathlineto{\pgfqpoint{4.942057in}{2.352823in}}%
\pgfpathlineto{\pgfqpoint{4.956568in}{2.367212in}}%
\pgfpathlineto{\pgfqpoint{4.971099in}{2.381766in}}%
\pgfpathlineto{\pgfqpoint{4.985651in}{2.396485in}}%
\pgfpathlineto{\pgfqpoint{4.993568in}{2.415158in}}%
\pgfpathlineto{\pgfqpoint{5.001481in}{2.433692in}}%
\pgfpathlineto{\pgfqpoint{5.009388in}{2.452085in}}%
\pgfpathlineto{\pgfqpoint{5.017289in}{2.470332in}}%
\pgfpathlineto{\pgfqpoint{5.002723in}{2.455246in}}%
\pgfpathlineto{\pgfqpoint{4.988178in}{2.440325in}}%
\pgfpathlineto{\pgfqpoint{4.973653in}{2.425570in}}%
\pgfpathlineto{\pgfqpoint{4.959148in}{2.410979in}}%
\pgfpathlineto{\pgfqpoint{4.951260in}{2.393087in}}%
\pgfpathlineto{\pgfqpoint{4.943367in}{2.375057in}}%
\pgfpathlineto{\pgfqpoint{4.935469in}{2.356892in}}%
\pgfpathlineto{\pgfqpoint{4.927566in}{2.338597in}}%
\pgfpathclose%
\pgfusepath{fill}%
\end{pgfscope}%
\begin{pgfscope}%
\pgfpathrectangle{\pgfqpoint{1.254980in}{0.150000in}}{\pgfqpoint{5.490039in}{5.490039in}}%
\pgfusepath{clip}%
\pgfsetbuttcap%
\pgfsetroundjoin%
\definecolor{currentfill}{rgb}{0.239374,0.735588,0.455688}%
\pgfsetfillcolor{currentfill}%
\pgfsetfillopacity{0.700000}%
\pgfsetlinewidth{0.000000pt}%
\definecolor{currentstroke}{rgb}{0.000000,0.000000,0.000000}%
\pgfsetstrokecolor{currentstroke}%
\pgfsetdash{}{0pt}%
\pgfpathmoveto{\pgfqpoint{5.080285in}{2.610734in}}%
\pgfpathlineto{\pgfqpoint{5.094899in}{2.626629in}}%
\pgfpathlineto{\pgfqpoint{5.109533in}{2.642691in}}%
\pgfpathlineto{\pgfqpoint{5.124189in}{2.658921in}}%
\pgfpathlineto{\pgfqpoint{5.138866in}{2.675319in}}%
\pgfpathlineto{\pgfqpoint{5.146725in}{2.692417in}}%
\pgfpathlineto{\pgfqpoint{5.154577in}{2.709335in}}%
\pgfpathlineto{\pgfqpoint{5.162421in}{2.726069in}}%
\pgfpathlineto{\pgfqpoint{5.170258in}{2.742618in}}%
\pgfpathlineto{\pgfqpoint{5.155568in}{2.725945in}}%
\pgfpathlineto{\pgfqpoint{5.140900in}{2.709440in}}%
\pgfpathlineto{\pgfqpoint{5.126254in}{2.693103in}}%
\pgfpathlineto{\pgfqpoint{5.111629in}{2.676934in}}%
\pgfpathlineto{\pgfqpoint{5.103803in}{2.660648in}}%
\pgfpathlineto{\pgfqpoint{5.095971in}{2.644184in}}%
\pgfpathlineto{\pgfqpoint{5.088132in}{2.627546in}}%
\pgfpathlineto{\pgfqpoint{5.080285in}{2.610734in}}%
\pgfpathclose%
\pgfusepath{fill}%
\end{pgfscope}%
\begin{pgfscope}%
\pgfpathrectangle{\pgfqpoint{1.254980in}{0.150000in}}{\pgfqpoint{5.490039in}{5.490039in}}%
\pgfusepath{clip}%
\pgfsetbuttcap%
\pgfsetroundjoin%
\definecolor{currentfill}{rgb}{0.262138,0.242286,0.520837}%
\pgfsetfillcolor{currentfill}%
\pgfsetfillopacity{0.700000}%
\pgfsetlinewidth{0.000000pt}%
\definecolor{currentstroke}{rgb}{0.000000,0.000000,0.000000}%
\pgfsetstrokecolor{currentstroke}%
\pgfsetdash{}{0pt}%
\pgfpathmoveto{\pgfqpoint{4.316953in}{1.288136in}}%
\pgfpathlineto{\pgfqpoint{4.331045in}{1.293263in}}%
\pgfpathlineto{\pgfqpoint{4.345149in}{1.298541in}}%
\pgfpathlineto{\pgfqpoint{4.359266in}{1.303971in}}%
\pgfpathlineto{\pgfqpoint{4.373397in}{1.309552in}}%
\pgfpathlineto{\pgfqpoint{4.381416in}{1.326754in}}%
\pgfpathlineto{\pgfqpoint{4.389433in}{1.344085in}}%
\pgfpathlineto{\pgfqpoint{4.397447in}{1.361539in}}%
\pgfpathlineto{\pgfqpoint{4.405459in}{1.379110in}}%
\pgfpathlineto{\pgfqpoint{4.391322in}{1.372832in}}%
\pgfpathlineto{\pgfqpoint{4.377199in}{1.366706in}}%
\pgfpathlineto{\pgfqpoint{4.363090in}{1.360733in}}%
\pgfpathlineto{\pgfqpoint{4.348993in}{1.354912in}}%
\pgfpathlineto{\pgfqpoint{4.340988in}{1.338027in}}%
\pgfpathlineto{\pgfqpoint{4.332979in}{1.321265in}}%
\pgfpathlineto{\pgfqpoint{4.324968in}{1.304633in}}%
\pgfpathlineto{\pgfqpoint{4.316953in}{1.288136in}}%
\pgfpathclose%
\pgfusepath{fill}%
\end{pgfscope}%
\begin{pgfscope}%
\pgfpathrectangle{\pgfqpoint{1.254980in}{0.150000in}}{\pgfqpoint{5.490039in}{5.490039in}}%
\pgfusepath{clip}%
\pgfsetbuttcap%
\pgfsetroundjoin%
\definecolor{currentfill}{rgb}{0.185556,0.418570,0.556753}%
\pgfsetfillcolor{currentfill}%
\pgfsetfillopacity{0.700000}%
\pgfsetlinewidth{0.000000pt}%
\definecolor{currentstroke}{rgb}{0.000000,0.000000,0.000000}%
\pgfsetstrokecolor{currentstroke}%
\pgfsetdash{}{0pt}%
\pgfpathmoveto{\pgfqpoint{4.590143in}{1.707716in}}%
\pgfpathlineto{\pgfqpoint{4.604394in}{1.717290in}}%
\pgfpathlineto{\pgfqpoint{4.618661in}{1.727021in}}%
\pgfpathlineto{\pgfqpoint{4.632944in}{1.736909in}}%
\pgfpathlineto{\pgfqpoint{4.647243in}{1.746953in}}%
\pgfpathlineto{\pgfqpoint{4.655235in}{1.766717in}}%
\pgfpathlineto{\pgfqpoint{4.663224in}{1.786477in}}%
\pgfpathlineto{\pgfqpoint{4.671211in}{1.806228in}}%
\pgfpathlineto{\pgfqpoint{4.679195in}{1.825965in}}%
\pgfpathlineto{\pgfqpoint{4.664882in}{1.815354in}}%
\pgfpathlineto{\pgfqpoint{4.650585in}{1.804900in}}%
\pgfpathlineto{\pgfqpoint{4.636306in}{1.794604in}}%
\pgfpathlineto{\pgfqpoint{4.622043in}{1.784466in}}%
\pgfpathlineto{\pgfqpoint{4.614072in}{1.765283in}}%
\pgfpathlineto{\pgfqpoint{4.606099in}{1.746094in}}%
\pgfpathlineto{\pgfqpoint{4.598122in}{1.726903in}}%
\pgfpathlineto{\pgfqpoint{4.590143in}{1.707716in}}%
\pgfpathclose%
\pgfusepath{fill}%
\end{pgfscope}%
\begin{pgfscope}%
\pgfpathrectangle{\pgfqpoint{1.254980in}{0.150000in}}{\pgfqpoint{5.490039in}{5.490039in}}%
\pgfusepath{clip}%
\pgfsetbuttcap%
\pgfsetroundjoin%
\definecolor{currentfill}{rgb}{0.386433,0.794644,0.372886}%
\pgfsetfillcolor{currentfill}%
\pgfsetfillopacity{0.700000}%
\pgfsetlinewidth{0.000000pt}%
\definecolor{currentstroke}{rgb}{0.000000,0.000000,0.000000}%
\pgfsetstrokecolor{currentstroke}%
\pgfsetdash{}{0pt}%
\pgfpathmoveto{\pgfqpoint{5.201531in}{2.806918in}}%
\pgfpathlineto{\pgfqpoint{5.216255in}{2.824005in}}%
\pgfpathlineto{\pgfqpoint{5.231001in}{2.841260in}}%
\pgfpathlineto{\pgfqpoint{5.245769in}{2.858687in}}%
\pgfpathlineto{\pgfqpoint{5.253576in}{2.874446in}}%
\pgfpathlineto{\pgfqpoint{5.261375in}{2.890001in}}%
\pgfpathlineto{\pgfqpoint{5.269165in}{2.905352in}}%
\pgfpathlineto{\pgfqpoint{5.276947in}{2.920495in}}%
\pgfpathlineto{\pgfqpoint{5.262168in}{2.902857in}}%
\pgfpathlineto{\pgfqpoint{5.247412in}{2.885390in}}%
\pgfpathlineto{\pgfqpoint{5.232679in}{2.868092in}}%
\pgfpathlineto{\pgfqpoint{5.224904in}{2.853098in}}%
\pgfpathlineto{\pgfqpoint{5.217121in}{2.837903in}}%
\pgfpathlineto{\pgfqpoint{5.209330in}{2.822509in}}%
\pgfpathlineto{\pgfqpoint{5.201531in}{2.806918in}}%
\pgfpathclose%
\pgfusepath{fill}%
\end{pgfscope}%
\begin{pgfscope}%
\pgfpathrectangle{\pgfqpoint{1.254980in}{0.150000in}}{\pgfqpoint{5.490039in}{5.490039in}}%
\pgfusepath{clip}%
\pgfsetbuttcap%
\pgfsetroundjoin%
\definecolor{currentfill}{rgb}{0.231674,0.318106,0.544834}%
\pgfsetfillcolor{currentfill}%
\pgfsetfillopacity{0.700000}%
\pgfsetlinewidth{0.000000pt}%
\definecolor{currentstroke}{rgb}{0.000000,0.000000,0.000000}%
\pgfsetstrokecolor{currentstroke}%
\pgfsetdash{}{0pt}%
\pgfpathmoveto{\pgfqpoint{4.437481in}{1.450425in}}%
\pgfpathlineto{\pgfqpoint{4.451639in}{1.457526in}}%
\pgfpathlineto{\pgfqpoint{4.465812in}{1.464780in}}%
\pgfpathlineto{\pgfqpoint{4.479999in}{1.472188in}}%
\pgfpathlineto{\pgfqpoint{4.494201in}{1.479750in}}%
\pgfpathlineto{\pgfqpoint{4.502209in}{1.498461in}}%
\pgfpathlineto{\pgfqpoint{4.510215in}{1.517243in}}%
\pgfpathlineto{\pgfqpoint{4.518219in}{1.536091in}}%
\pgfpathlineto{\pgfqpoint{4.526220in}{1.554999in}}%
\pgfpathlineto{\pgfqpoint{4.512008in}{1.546792in}}%
\pgfpathlineto{\pgfqpoint{4.497811in}{1.538738in}}%
\pgfpathlineto{\pgfqpoint{4.483629in}{1.530840in}}%
\pgfpathlineto{\pgfqpoint{4.469462in}{1.523095in}}%
\pgfpathlineto{\pgfqpoint{4.461470in}{1.504822in}}%
\pgfpathlineto{\pgfqpoint{4.453476in}{1.486615in}}%
\pgfpathlineto{\pgfqpoint{4.445480in}{1.468481in}}%
\pgfpathlineto{\pgfqpoint{4.437481in}{1.450425in}}%
\pgfpathclose%
\pgfusepath{fill}%
\end{pgfscope}%
\begin{pgfscope}%
\pgfpathrectangle{\pgfqpoint{1.254980in}{0.150000in}}{\pgfqpoint{5.490039in}{5.490039in}}%
\pgfusepath{clip}%
\pgfsetbuttcap%
\pgfsetroundjoin%
\definecolor{currentfill}{rgb}{0.144759,0.519093,0.556572}%
\pgfsetfillcolor{currentfill}%
\pgfsetfillopacity{0.700000}%
\pgfsetlinewidth{0.000000pt}%
\definecolor{currentstroke}{rgb}{0.000000,0.000000,0.000000}%
\pgfsetstrokecolor{currentstroke}%
\pgfsetdash{}{0pt}%
\pgfpathmoveto{\pgfqpoint{4.742960in}{1.982774in}}%
\pgfpathlineto{\pgfqpoint{4.757318in}{1.994594in}}%
\pgfpathlineto{\pgfqpoint{4.771695in}{2.006574in}}%
\pgfpathlineto{\pgfqpoint{4.786090in}{2.018715in}}%
\pgfpathlineto{\pgfqpoint{4.800503in}{2.031016in}}%
\pgfpathlineto{\pgfqpoint{4.808474in}{2.050903in}}%
\pgfpathlineto{\pgfqpoint{4.816441in}{2.070721in}}%
\pgfpathlineto{\pgfqpoint{4.824405in}{2.090466in}}%
\pgfpathlineto{\pgfqpoint{4.832366in}{2.110133in}}%
\pgfpathlineto{\pgfqpoint{4.817937in}{2.097347in}}%
\pgfpathlineto{\pgfqpoint{4.803528in}{2.084723in}}%
\pgfpathlineto{\pgfqpoint{4.789137in}{2.072260in}}%
\pgfpathlineto{\pgfqpoint{4.774764in}{2.059958in}}%
\pgfpathlineto{\pgfqpoint{4.766818in}{2.040763in}}%
\pgfpathlineto{\pgfqpoint{4.758869in}{2.021498in}}%
\pgfpathlineto{\pgfqpoint{4.750916in}{2.002167in}}%
\pgfpathlineto{\pgfqpoint{4.742960in}{1.982774in}}%
\pgfpathclose%
\pgfusepath{fill}%
\end{pgfscope}%
\begin{pgfscope}%
\pgfpathrectangle{\pgfqpoint{1.254980in}{0.150000in}}{\pgfqpoint{5.490039in}{5.490039in}}%
\pgfusepath{clip}%
\pgfsetbuttcap%
\pgfsetroundjoin%
\definecolor{currentfill}{rgb}{0.120081,0.622161,0.534946}%
\pgfsetfillcolor{currentfill}%
\pgfsetfillopacity{0.700000}%
\pgfsetlinewidth{0.000000pt}%
\definecolor{currentstroke}{rgb}{0.000000,0.000000,0.000000}%
\pgfsetstrokecolor{currentstroke}%
\pgfsetdash{}{0pt}%
\pgfpathmoveto{\pgfqpoint{4.895904in}{2.264176in}}%
\pgfpathlineto{\pgfqpoint{4.910381in}{2.278006in}}%
\pgfpathlineto{\pgfqpoint{4.924877in}{2.291999in}}%
\pgfpathlineto{\pgfqpoint{4.939394in}{2.306155in}}%
\pgfpathlineto{\pgfqpoint{4.953930in}{2.320476in}}%
\pgfpathlineto{\pgfqpoint{4.961867in}{2.339669in}}%
\pgfpathlineto{\pgfqpoint{4.969800in}{2.358737in}}%
\pgfpathlineto{\pgfqpoint{4.977728in}{2.377677in}}%
\pgfpathlineto{\pgfqpoint{4.985651in}{2.396485in}}%
\pgfpathlineto{\pgfqpoint{4.971099in}{2.381766in}}%
\pgfpathlineto{\pgfqpoint{4.956568in}{2.367212in}}%
\pgfpathlineto{\pgfqpoint{4.942057in}{2.352823in}}%
\pgfpathlineto{\pgfqpoint{4.927566in}{2.338597in}}%
\pgfpathlineto{\pgfqpoint{4.919658in}{2.320175in}}%
\pgfpathlineto{\pgfqpoint{4.911744in}{2.301628in}}%
\pgfpathlineto{\pgfqpoint{4.903826in}{2.282961in}}%
\pgfpathlineto{\pgfqpoint{4.895904in}{2.264176in}}%
\pgfpathclose%
\pgfusepath{fill}%
\end{pgfscope}%
\begin{pgfscope}%
\pgfpathrectangle{\pgfqpoint{1.254980in}{0.150000in}}{\pgfqpoint{5.490039in}{5.490039in}}%
\pgfusepath{clip}%
\pgfsetbuttcap%
\pgfsetroundjoin%
\definecolor{currentfill}{rgb}{0.208030,0.718701,0.472873}%
\pgfsetfillcolor{currentfill}%
\pgfsetfillopacity{0.700000}%
\pgfsetlinewidth{0.000000pt}%
\definecolor{currentstroke}{rgb}{0.000000,0.000000,0.000000}%
\pgfsetstrokecolor{currentstroke}%
\pgfsetdash{}{0pt}%
\pgfpathmoveto{\pgfqpoint{5.048836in}{2.541811in}}%
\pgfpathlineto{\pgfqpoint{5.063437in}{2.557400in}}%
\pgfpathlineto{\pgfqpoint{5.078058in}{2.573156in}}%
\pgfpathlineto{\pgfqpoint{5.092701in}{2.589080in}}%
\pgfpathlineto{\pgfqpoint{5.107365in}{2.605170in}}%
\pgfpathlineto{\pgfqpoint{5.115250in}{2.622966in}}%
\pgfpathlineto{\pgfqpoint{5.123129in}{2.640591in}}%
\pgfpathlineto{\pgfqpoint{5.131001in}{2.658043in}}%
\pgfpathlineto{\pgfqpoint{5.138866in}{2.675319in}}%
\pgfpathlineto{\pgfqpoint{5.124189in}{2.658921in}}%
\pgfpathlineto{\pgfqpoint{5.109533in}{2.642691in}}%
\pgfpathlineto{\pgfqpoint{5.094899in}{2.626629in}}%
\pgfpathlineto{\pgfqpoint{5.080285in}{2.610734in}}%
\pgfpathlineto{\pgfqpoint{5.072433in}{2.593752in}}%
\pgfpathlineto{\pgfqpoint{5.064574in}{2.576603in}}%
\pgfpathlineto{\pgfqpoint{5.056708in}{2.559288in}}%
\pgfpathlineto{\pgfqpoint{5.048836in}{2.541811in}}%
\pgfpathclose%
\pgfusepath{fill}%
\end{pgfscope}%
\begin{pgfscope}%
\pgfpathrectangle{\pgfqpoint{1.254980in}{0.150000in}}{\pgfqpoint{5.490039in}{5.490039in}}%
\pgfusepath{clip}%
\pgfsetbuttcap%
\pgfsetroundjoin%
\definecolor{currentfill}{rgb}{0.195860,0.395433,0.555276}%
\pgfsetfillcolor{currentfill}%
\pgfsetfillopacity{0.700000}%
\pgfsetlinewidth{0.000000pt}%
\definecolor{currentstroke}{rgb}{0.000000,0.000000,0.000000}%
\pgfsetstrokecolor{currentstroke}%
\pgfsetdash{}{0pt}%
\pgfpathmoveto{\pgfqpoint{4.558201in}{1.631111in}}%
\pgfpathlineto{\pgfqpoint{4.572440in}{1.640093in}}%
\pgfpathlineto{\pgfqpoint{4.586694in}{1.649230in}}%
\pgfpathlineto{\pgfqpoint{4.600965in}{1.658524in}}%
\pgfpathlineto{\pgfqpoint{4.615251in}{1.667974in}}%
\pgfpathlineto{\pgfqpoint{4.623253in}{1.687697in}}%
\pgfpathlineto{\pgfqpoint{4.631252in}{1.707438in}}%
\pgfpathlineto{\pgfqpoint{4.639249in}{1.727192in}}%
\pgfpathlineto{\pgfqpoint{4.647243in}{1.746953in}}%
\pgfpathlineto{\pgfqpoint{4.632944in}{1.736909in}}%
\pgfpathlineto{\pgfqpoint{4.618661in}{1.727021in}}%
\pgfpathlineto{\pgfqpoint{4.604394in}{1.717290in}}%
\pgfpathlineto{\pgfqpoint{4.590143in}{1.707716in}}%
\pgfpathlineto{\pgfqpoint{4.582162in}{1.688538in}}%
\pgfpathlineto{\pgfqpoint{4.574177in}{1.669374in}}%
\pgfpathlineto{\pgfqpoint{4.566191in}{1.650229in}}%
\pgfpathlineto{\pgfqpoint{4.558201in}{1.631111in}}%
\pgfpathclose%
\pgfusepath{fill}%
\end{pgfscope}%
\begin{pgfscope}%
\pgfpathrectangle{\pgfqpoint{1.254980in}{0.150000in}}{\pgfqpoint{5.490039in}{5.490039in}}%
\pgfusepath{clip}%
\pgfsetbuttcap%
\pgfsetroundjoin%
\definecolor{currentfill}{rgb}{0.153364,0.497000,0.557724}%
\pgfsetfillcolor{currentfill}%
\pgfsetfillopacity{0.700000}%
\pgfsetlinewidth{0.000000pt}%
\definecolor{currentstroke}{rgb}{0.000000,0.000000,0.000000}%
\pgfsetstrokecolor{currentstroke}%
\pgfsetdash{}{0pt}%
\pgfpathmoveto{\pgfqpoint{4.711102in}{1.904677in}}%
\pgfpathlineto{\pgfqpoint{4.725446in}{1.915987in}}%
\pgfpathlineto{\pgfqpoint{4.739808in}{1.927455in}}%
\pgfpathlineto{\pgfqpoint{4.754188in}{1.939083in}}%
\pgfpathlineto{\pgfqpoint{4.768586in}{1.950871in}}%
\pgfpathlineto{\pgfqpoint{4.776570in}{1.970987in}}%
\pgfpathlineto{\pgfqpoint{4.784551in}{1.991054in}}%
\pgfpathlineto{\pgfqpoint{4.792528in}{2.011065in}}%
\pgfpathlineto{\pgfqpoint{4.800503in}{2.031016in}}%
\pgfpathlineto{\pgfqpoint{4.786090in}{2.018715in}}%
\pgfpathlineto{\pgfqpoint{4.771695in}{2.006574in}}%
\pgfpathlineto{\pgfqpoint{4.757318in}{1.994594in}}%
\pgfpathlineto{\pgfqpoint{4.742960in}{1.982774in}}%
\pgfpathlineto{\pgfqpoint{4.735000in}{1.963324in}}%
\pgfpathlineto{\pgfqpoint{4.727037in}{1.943822in}}%
\pgfpathlineto{\pgfqpoint{4.719071in}{1.924271in}}%
\pgfpathlineto{\pgfqpoint{4.711102in}{1.904677in}}%
\pgfpathclose%
\pgfusepath{fill}%
\end{pgfscope}%
\begin{pgfscope}%
\pgfpathrectangle{\pgfqpoint{1.254980in}{0.150000in}}{\pgfqpoint{5.490039in}{5.490039in}}%
\pgfusepath{clip}%
\pgfsetbuttcap%
\pgfsetroundjoin%
\definecolor{currentfill}{rgb}{0.269308,0.218818,0.509577}%
\pgfsetfillcolor{currentfill}%
\pgfsetfillopacity{0.700000}%
\pgfsetlinewidth{0.000000pt}%
\definecolor{currentstroke}{rgb}{0.000000,0.000000,0.000000}%
\pgfsetstrokecolor{currentstroke}%
\pgfsetdash{}{0pt}%
\pgfpathmoveto{\pgfqpoint{4.284865in}{1.223646in}}%
\pgfpathlineto{\pgfqpoint{4.298953in}{1.228052in}}%
\pgfpathlineto{\pgfqpoint{4.313053in}{1.232608in}}%
\pgfpathlineto{\pgfqpoint{4.327166in}{1.237315in}}%
\pgfpathlineto{\pgfqpoint{4.341292in}{1.242173in}}%
\pgfpathlineto{\pgfqpoint{4.349322in}{1.258790in}}%
\pgfpathlineto{\pgfqpoint{4.357350in}{1.275563in}}%
\pgfpathlineto{\pgfqpoint{4.365375in}{1.292486in}}%
\pgfpathlineto{\pgfqpoint{4.373397in}{1.309552in}}%
\pgfpathlineto{\pgfqpoint{4.359266in}{1.303971in}}%
\pgfpathlineto{\pgfqpoint{4.345149in}{1.298541in}}%
\pgfpathlineto{\pgfqpoint{4.331045in}{1.293263in}}%
\pgfpathlineto{\pgfqpoint{4.316953in}{1.288136in}}%
\pgfpathlineto{\pgfqpoint{4.308936in}{1.271783in}}%
\pgfpathlineto{\pgfqpoint{4.300915in}{1.255578in}}%
\pgfpathlineto{\pgfqpoint{4.292892in}{1.239531in}}%
\pgfpathlineto{\pgfqpoint{4.284865in}{1.223646in}}%
\pgfpathclose%
\pgfusepath{fill}%
\end{pgfscope}%
\begin{pgfscope}%
\pgfpathrectangle{\pgfqpoint{1.254980in}{0.150000in}}{\pgfqpoint{5.490039in}{5.490039in}}%
\pgfusepath{clip}%
\pgfsetbuttcap%
\pgfsetroundjoin%
\definecolor{currentfill}{rgb}{0.243113,0.292092,0.538516}%
\pgfsetfillcolor{currentfill}%
\pgfsetfillopacity{0.700000}%
\pgfsetlinewidth{0.000000pt}%
\definecolor{currentstroke}{rgb}{0.000000,0.000000,0.000000}%
\pgfsetstrokecolor{currentstroke}%
\pgfsetdash{}{0pt}%
\pgfpathmoveto{\pgfqpoint{4.405459in}{1.379110in}}%
\pgfpathlineto{\pgfqpoint{4.419610in}{1.385540in}}%
\pgfpathlineto{\pgfqpoint{4.433774in}{1.392123in}}%
\pgfpathlineto{\pgfqpoint{4.447953in}{1.398859in}}%
\pgfpathlineto{\pgfqpoint{4.462145in}{1.405747in}}%
\pgfpathlineto{\pgfqpoint{4.470163in}{1.424109in}}%
\pgfpathlineto{\pgfqpoint{4.478178in}{1.442568in}}%
\pgfpathlineto{\pgfqpoint{4.486190in}{1.461117in}}%
\pgfpathlineto{\pgfqpoint{4.494201in}{1.479750in}}%
\pgfpathlineto{\pgfqpoint{4.479999in}{1.472188in}}%
\pgfpathlineto{\pgfqpoint{4.465812in}{1.464780in}}%
\pgfpathlineto{\pgfqpoint{4.451639in}{1.457526in}}%
\pgfpathlineto{\pgfqpoint{4.437481in}{1.450425in}}%
\pgfpathlineto{\pgfqpoint{4.429479in}{1.432453in}}%
\pgfpathlineto{\pgfqpoint{4.421475in}{1.414573in}}%
\pgfpathlineto{\pgfqpoint{4.413468in}{1.396790in}}%
\pgfpathlineto{\pgfqpoint{4.405459in}{1.379110in}}%
\pgfpathclose%
\pgfusepath{fill}%
\end{pgfscope}%
\begin{pgfscope}%
\pgfpathrectangle{\pgfqpoint{1.254980in}{0.150000in}}{\pgfqpoint{5.490039in}{5.490039in}}%
\pgfusepath{clip}%
\pgfsetbuttcap%
\pgfsetroundjoin%
\definecolor{currentfill}{rgb}{0.352360,0.783011,0.392636}%
\pgfsetfillcolor{currentfill}%
\pgfsetfillopacity{0.700000}%
\pgfsetlinewidth{0.000000pt}%
\definecolor{currentstroke}{rgb}{0.000000,0.000000,0.000000}%
\pgfsetstrokecolor{currentstroke}%
\pgfsetdash{}{0pt}%
\pgfpathmoveto{\pgfqpoint{5.170258in}{2.742618in}}%
\pgfpathlineto{\pgfqpoint{5.184970in}{2.759461in}}%
\pgfpathlineto{\pgfqpoint{5.199704in}{2.776472in}}%
\pgfpathlineto{\pgfqpoint{5.214461in}{2.793654in}}%
\pgfpathlineto{\pgfqpoint{5.222300in}{2.810208in}}%
\pgfpathlineto{\pgfqpoint{5.230131in}{2.826566in}}%
\pgfpathlineto{\pgfqpoint{5.237954in}{2.842726in}}%
\pgfpathlineto{\pgfqpoint{5.245769in}{2.858687in}}%
\pgfpathlineto{\pgfqpoint{5.231001in}{2.841260in}}%
\pgfpathlineto{\pgfqpoint{5.216255in}{2.824005in}}%
\pgfpathlineto{\pgfqpoint{5.201531in}{2.806918in}}%
\pgfpathlineto{\pgfqpoint{5.193725in}{2.791132in}}%
\pgfpathlineto{\pgfqpoint{5.185910in}{2.775151in}}%
\pgfpathlineto{\pgfqpoint{5.178088in}{2.758980in}}%
\pgfpathlineto{\pgfqpoint{5.170258in}{2.742618in}}%
\pgfpathclose%
\pgfusepath{fill}%
\end{pgfscope}%
\begin{pgfscope}%
\pgfpathrectangle{\pgfqpoint{1.254980in}{0.150000in}}{\pgfqpoint{5.490039in}{5.490039in}}%
\pgfusepath{clip}%
\pgfsetbuttcap%
\pgfsetroundjoin%
\definecolor{currentfill}{rgb}{0.120092,0.600104,0.542530}%
\pgfsetfillcolor{currentfill}%
\pgfsetfillopacity{0.700000}%
\pgfsetlinewidth{0.000000pt}%
\definecolor{currentstroke}{rgb}{0.000000,0.000000,0.000000}%
\pgfsetstrokecolor{currentstroke}%
\pgfsetdash{}{0pt}%
\pgfpathmoveto{\pgfqpoint{4.864168in}{2.187940in}}%
\pgfpathlineto{\pgfqpoint{4.878630in}{2.201344in}}%
\pgfpathlineto{\pgfqpoint{4.893112in}{2.214910in}}%
\pgfpathlineto{\pgfqpoint{4.907613in}{2.228639in}}%
\pgfpathlineto{\pgfqpoint{4.922133in}{2.242531in}}%
\pgfpathlineto{\pgfqpoint{4.930089in}{2.262186in}}%
\pgfpathlineto{\pgfqpoint{4.938040in}{2.281731in}}%
\pgfpathlineto{\pgfqpoint{4.945987in}{2.301162in}}%
\pgfpathlineto{\pgfqpoint{4.953930in}{2.320476in}}%
\pgfpathlineto{\pgfqpoint{4.939394in}{2.306155in}}%
\pgfpathlineto{\pgfqpoint{4.924877in}{2.291999in}}%
\pgfpathlineto{\pgfqpoint{4.910381in}{2.278006in}}%
\pgfpathlineto{\pgfqpoint{4.895904in}{2.264176in}}%
\pgfpathlineto{\pgfqpoint{4.887976in}{2.245278in}}%
\pgfpathlineto{\pgfqpoint{4.880045in}{2.226270in}}%
\pgfpathlineto{\pgfqpoint{4.872108in}{2.207156in}}%
\pgfpathlineto{\pgfqpoint{4.864168in}{2.187940in}}%
\pgfpathclose%
\pgfusepath{fill}%
\end{pgfscope}%
\begin{pgfscope}%
\pgfpathrectangle{\pgfqpoint{1.254980in}{0.150000in}}{\pgfqpoint{5.490039in}{5.490039in}}%
\pgfusepath{clip}%
\pgfsetbuttcap%
\pgfsetroundjoin%
\definecolor{currentfill}{rgb}{0.208623,0.367752,0.552675}%
\pgfsetfillcolor{currentfill}%
\pgfsetfillopacity{0.700000}%
\pgfsetlinewidth{0.000000pt}%
\definecolor{currentstroke}{rgb}{0.000000,0.000000,0.000000}%
\pgfsetstrokecolor{currentstroke}%
\pgfsetdash{}{0pt}%
\pgfpathmoveto{\pgfqpoint{4.526220in}{1.554999in}}%
\pgfpathlineto{\pgfqpoint{4.540447in}{1.563362in}}%
\pgfpathlineto{\pgfqpoint{4.554690in}{1.571879in}}%
\pgfpathlineto{\pgfqpoint{4.568948in}{1.580551in}}%
\pgfpathlineto{\pgfqpoint{4.583222in}{1.589378in}}%
\pgfpathlineto{\pgfqpoint{4.591233in}{1.608972in}}%
\pgfpathlineto{\pgfqpoint{4.599241in}{1.628606in}}%
\pgfpathlineto{\pgfqpoint{4.607247in}{1.648275in}}%
\pgfpathlineto{\pgfqpoint{4.615251in}{1.667974in}}%
\pgfpathlineto{\pgfqpoint{4.600965in}{1.658524in}}%
\pgfpathlineto{\pgfqpoint{4.586694in}{1.649230in}}%
\pgfpathlineto{\pgfqpoint{4.572440in}{1.640093in}}%
\pgfpathlineto{\pgfqpoint{4.558201in}{1.631111in}}%
\pgfpathlineto{\pgfqpoint{4.550210in}{1.612022in}}%
\pgfpathlineto{\pgfqpoint{4.542216in}{1.592971in}}%
\pgfpathlineto{\pgfqpoint{4.534219in}{1.573961in}}%
\pgfpathlineto{\pgfqpoint{4.526220in}{1.554999in}}%
\pgfpathclose%
\pgfusepath{fill}%
\end{pgfscope}%
\begin{pgfscope}%
\pgfpathrectangle{\pgfqpoint{1.254980in}{0.150000in}}{\pgfqpoint{5.490039in}{5.490039in}}%
\pgfusepath{clip}%
\pgfsetbuttcap%
\pgfsetroundjoin%
\definecolor{currentfill}{rgb}{0.180653,0.701402,0.488189}%
\pgfsetfillcolor{currentfill}%
\pgfsetfillopacity{0.700000}%
\pgfsetlinewidth{0.000000pt}%
\definecolor{currentstroke}{rgb}{0.000000,0.000000,0.000000}%
\pgfsetstrokecolor{currentstroke}%
\pgfsetdash{}{0pt}%
\pgfpathmoveto{\pgfqpoint{5.017289in}{2.470332in}}%
\pgfpathlineto{\pgfqpoint{5.031875in}{2.485585in}}%
\pgfpathlineto{\pgfqpoint{5.046483in}{2.501003in}}%
\pgfpathlineto{\pgfqpoint{5.061111in}{2.516588in}}%
\pgfpathlineto{\pgfqpoint{5.075760in}{2.532340in}}%
\pgfpathlineto{\pgfqpoint{5.083671in}{2.550789in}}%
\pgfpathlineto{\pgfqpoint{5.091575in}{2.569079in}}%
\pgfpathlineto{\pgfqpoint{5.099473in}{2.587207in}}%
\pgfpathlineto{\pgfqpoint{5.107365in}{2.605170in}}%
\pgfpathlineto{\pgfqpoint{5.092701in}{2.589080in}}%
\pgfpathlineto{\pgfqpoint{5.078058in}{2.573156in}}%
\pgfpathlineto{\pgfqpoint{5.063437in}{2.557400in}}%
\pgfpathlineto{\pgfqpoint{5.048836in}{2.541811in}}%
\pgfpathlineto{\pgfqpoint{5.040958in}{2.524174in}}%
\pgfpathlineto{\pgfqpoint{5.033074in}{2.506380in}}%
\pgfpathlineto{\pgfqpoint{5.025185in}{2.488432in}}%
\pgfpathlineto{\pgfqpoint{5.017289in}{2.470332in}}%
\pgfpathclose%
\pgfusepath{fill}%
\end{pgfscope}%
\begin{pgfscope}%
\pgfpathrectangle{\pgfqpoint{1.254980in}{0.150000in}}{\pgfqpoint{5.490039in}{5.490039in}}%
\pgfusepath{clip}%
\pgfsetbuttcap%
\pgfsetroundjoin%
\definecolor{currentfill}{rgb}{0.163625,0.471133,0.558148}%
\pgfsetfillcolor{currentfill}%
\pgfsetfillopacity{0.700000}%
\pgfsetlinewidth{0.000000pt}%
\definecolor{currentstroke}{rgb}{0.000000,0.000000,0.000000}%
\pgfsetstrokecolor{currentstroke}%
\pgfsetdash{}{0pt}%
\pgfpathmoveto{\pgfqpoint{4.679195in}{1.825965in}}%
\pgfpathlineto{\pgfqpoint{4.693525in}{1.836735in}}%
\pgfpathlineto{\pgfqpoint{4.707873in}{1.847663in}}%
\pgfpathlineto{\pgfqpoint{4.722237in}{1.858750in}}%
\pgfpathlineto{\pgfqpoint{4.736620in}{1.869995in}}%
\pgfpathlineto{\pgfqpoint{4.744616in}{1.890265in}}%
\pgfpathlineto{\pgfqpoint{4.752609in}{1.910504in}}%
\pgfpathlineto{\pgfqpoint{4.760599in}{1.930708in}}%
\pgfpathlineto{\pgfqpoint{4.768586in}{1.950871in}}%
\pgfpathlineto{\pgfqpoint{4.754188in}{1.939083in}}%
\pgfpathlineto{\pgfqpoint{4.739808in}{1.927455in}}%
\pgfpathlineto{\pgfqpoint{4.725446in}{1.915987in}}%
\pgfpathlineto{\pgfqpoint{4.711102in}{1.904677in}}%
\pgfpathlineto{\pgfqpoint{4.703129in}{1.885045in}}%
\pgfpathlineto{\pgfqpoint{4.695154in}{1.865379in}}%
\pgfpathlineto{\pgfqpoint{4.687176in}{1.845684in}}%
\pgfpathlineto{\pgfqpoint{4.679195in}{1.825965in}}%
\pgfpathclose%
\pgfusepath{fill}%
\end{pgfscope}%
\begin{pgfscope}%
\pgfpathrectangle{\pgfqpoint{1.254980in}{0.150000in}}{\pgfqpoint{5.490039in}{5.490039in}}%
\pgfusepath{clip}%
\pgfsetbuttcap%
\pgfsetroundjoin%
\definecolor{currentfill}{rgb}{0.252194,0.269783,0.531579}%
\pgfsetfillcolor{currentfill}%
\pgfsetfillopacity{0.700000}%
\pgfsetlinewidth{0.000000pt}%
\definecolor{currentstroke}{rgb}{0.000000,0.000000,0.000000}%
\pgfsetstrokecolor{currentstroke}%
\pgfsetdash{}{0pt}%
\pgfpathmoveto{\pgfqpoint{4.373397in}{1.309552in}}%
\pgfpathlineto{\pgfqpoint{4.387541in}{1.315285in}}%
\pgfpathlineto{\pgfqpoint{4.401698in}{1.321170in}}%
\pgfpathlineto{\pgfqpoint{4.415869in}{1.327206in}}%
\pgfpathlineto{\pgfqpoint{4.430053in}{1.333394in}}%
\pgfpathlineto{\pgfqpoint{4.438080in}{1.351305in}}%
\pgfpathlineto{\pgfqpoint{4.446104in}{1.369339in}}%
\pgfpathlineto{\pgfqpoint{4.454126in}{1.387488in}}%
\pgfpathlineto{\pgfqpoint{4.462145in}{1.405747in}}%
\pgfpathlineto{\pgfqpoint{4.447953in}{1.398859in}}%
\pgfpathlineto{\pgfqpoint{4.433774in}{1.392123in}}%
\pgfpathlineto{\pgfqpoint{4.419610in}{1.385540in}}%
\pgfpathlineto{\pgfqpoint{4.405459in}{1.379110in}}%
\pgfpathlineto{\pgfqpoint{4.397447in}{1.361539in}}%
\pgfpathlineto{\pgfqpoint{4.389433in}{1.344085in}}%
\pgfpathlineto{\pgfqpoint{4.381416in}{1.326754in}}%
\pgfpathlineto{\pgfqpoint{4.373397in}{1.309552in}}%
\pgfpathclose%
\pgfusepath{fill}%
\end{pgfscope}%
\begin{pgfscope}%
\pgfpathrectangle{\pgfqpoint{1.254980in}{0.150000in}}{\pgfqpoint{5.490039in}{5.490039in}}%
\pgfusepath{clip}%
\pgfsetbuttcap%
\pgfsetroundjoin%
\definecolor{currentfill}{rgb}{0.124395,0.578002,0.548287}%
\pgfsetfillcolor{currentfill}%
\pgfsetfillopacity{0.700000}%
\pgfsetlinewidth{0.000000pt}%
\definecolor{currentstroke}{rgb}{0.000000,0.000000,0.000000}%
\pgfsetstrokecolor{currentstroke}%
\pgfsetdash{}{0pt}%
\pgfpathmoveto{\pgfqpoint{4.832366in}{2.110133in}}%
\pgfpathlineto{\pgfqpoint{4.846813in}{2.123080in}}%
\pgfpathlineto{\pgfqpoint{4.861278in}{2.136190in}}%
\pgfpathlineto{\pgfqpoint{4.875764in}{2.149461in}}%
\pgfpathlineto{\pgfqpoint{4.890268in}{2.162895in}}%
\pgfpathlineto{\pgfqpoint{4.898240in}{2.182948in}}%
\pgfpathlineto{\pgfqpoint{4.906209in}{2.202908in}}%
\pgfpathlineto{\pgfqpoint{4.914173in}{2.222770in}}%
\pgfpathlineto{\pgfqpoint{4.922133in}{2.242531in}}%
\pgfpathlineto{\pgfqpoint{4.907613in}{2.228639in}}%
\pgfpathlineto{\pgfqpoint{4.893112in}{2.214910in}}%
\pgfpathlineto{\pgfqpoint{4.878630in}{2.201344in}}%
\pgfpathlineto{\pgfqpoint{4.864168in}{2.187940in}}%
\pgfpathlineto{\pgfqpoint{4.856223in}{2.168626in}}%
\pgfpathlineto{\pgfqpoint{4.848275in}{2.149217in}}%
\pgfpathlineto{\pgfqpoint{4.840322in}{2.129718in}}%
\pgfpathlineto{\pgfqpoint{4.832366in}{2.110133in}}%
\pgfpathclose%
\pgfusepath{fill}%
\end{pgfscope}%
\begin{pgfscope}%
\pgfpathrectangle{\pgfqpoint{1.254980in}{0.150000in}}{\pgfqpoint{5.490039in}{5.490039in}}%
\pgfusepath{clip}%
\pgfsetbuttcap%
\pgfsetroundjoin%
\definecolor{currentfill}{rgb}{0.311925,0.767822,0.415586}%
\pgfsetfillcolor{currentfill}%
\pgfsetfillopacity{0.700000}%
\pgfsetlinewidth{0.000000pt}%
\definecolor{currentstroke}{rgb}{0.000000,0.000000,0.000000}%
\pgfsetstrokecolor{currentstroke}%
\pgfsetdash{}{0pt}%
\pgfpathmoveto{\pgfqpoint{5.138866in}{2.675319in}}%
\pgfpathlineto{\pgfqpoint{5.153566in}{2.691886in}}%
\pgfpathlineto{\pgfqpoint{5.168287in}{2.708621in}}%
\pgfpathlineto{\pgfqpoint{5.183031in}{2.725525in}}%
\pgfpathlineto{\pgfqpoint{5.190900in}{2.742840in}}%
\pgfpathlineto{\pgfqpoint{5.198761in}{2.759968in}}%
\pgfpathlineto{\pgfqpoint{5.206615in}{2.776906in}}%
\pgfpathlineto{\pgfqpoint{5.214461in}{2.793654in}}%
\pgfpathlineto{\pgfqpoint{5.199704in}{2.776472in}}%
\pgfpathlineto{\pgfqpoint{5.184970in}{2.759461in}}%
\pgfpathlineto{\pgfqpoint{5.170258in}{2.742618in}}%
\pgfpathlineto{\pgfqpoint{5.162421in}{2.726069in}}%
\pgfpathlineto{\pgfqpoint{5.154577in}{2.709335in}}%
\pgfpathlineto{\pgfqpoint{5.146725in}{2.692417in}}%
\pgfpathlineto{\pgfqpoint{5.138866in}{2.675319in}}%
\pgfpathclose%
\pgfusepath{fill}%
\end{pgfscope}%
\begin{pgfscope}%
\pgfpathrectangle{\pgfqpoint{1.254980in}{0.150000in}}{\pgfqpoint{5.490039in}{5.490039in}}%
\pgfusepath{clip}%
\pgfsetbuttcap%
\pgfsetroundjoin%
\definecolor{currentfill}{rgb}{0.220057,0.343307,0.549413}%
\pgfsetfillcolor{currentfill}%
\pgfsetfillopacity{0.700000}%
\pgfsetlinewidth{0.000000pt}%
\definecolor{currentstroke}{rgb}{0.000000,0.000000,0.000000}%
\pgfsetstrokecolor{currentstroke}%
\pgfsetdash{}{0pt}%
\pgfpathmoveto{\pgfqpoint{4.494201in}{1.479750in}}%
\pgfpathlineto{\pgfqpoint{4.508418in}{1.487465in}}%
\pgfpathlineto{\pgfqpoint{4.522649in}{1.495334in}}%
\pgfpathlineto{\pgfqpoint{4.536896in}{1.503358in}}%
\pgfpathlineto{\pgfqpoint{4.551158in}{1.511535in}}%
\pgfpathlineto{\pgfqpoint{4.559177in}{1.530905in}}%
\pgfpathlineto{\pgfqpoint{4.567194in}{1.550339in}}%
\pgfpathlineto{\pgfqpoint{4.575209in}{1.569832in}}%
\pgfpathlineto{\pgfqpoint{4.583222in}{1.589378in}}%
\pgfpathlineto{\pgfqpoint{4.568948in}{1.580551in}}%
\pgfpathlineto{\pgfqpoint{4.554690in}{1.571879in}}%
\pgfpathlineto{\pgfqpoint{4.540447in}{1.563362in}}%
\pgfpathlineto{\pgfqpoint{4.526220in}{1.554999in}}%
\pgfpathlineto{\pgfqpoint{4.518219in}{1.536091in}}%
\pgfpathlineto{\pgfqpoint{4.510215in}{1.517243in}}%
\pgfpathlineto{\pgfqpoint{4.502209in}{1.498461in}}%
\pgfpathlineto{\pgfqpoint{4.494201in}{1.479750in}}%
\pgfpathclose%
\pgfusepath{fill}%
\end{pgfscope}%
\begin{pgfscope}%
\pgfpathrectangle{\pgfqpoint{1.254980in}{0.150000in}}{\pgfqpoint{5.490039in}{5.490039in}}%
\pgfusepath{clip}%
\pgfsetbuttcap%
\pgfsetroundjoin%
\definecolor{currentfill}{rgb}{0.172719,0.448791,0.557885}%
\pgfsetfillcolor{currentfill}%
\pgfsetfillopacity{0.700000}%
\pgfsetlinewidth{0.000000pt}%
\definecolor{currentstroke}{rgb}{0.000000,0.000000,0.000000}%
\pgfsetstrokecolor{currentstroke}%
\pgfsetdash{}{0pt}%
\pgfpathmoveto{\pgfqpoint{4.647243in}{1.746953in}}%
\pgfpathlineto{\pgfqpoint{4.661560in}{1.757155in}}%
\pgfpathlineto{\pgfqpoint{4.675893in}{1.767514in}}%
\pgfpathlineto{\pgfqpoint{4.690243in}{1.778031in}}%
\pgfpathlineto{\pgfqpoint{4.704610in}{1.788705in}}%
\pgfpathlineto{\pgfqpoint{4.712617in}{1.809048in}}%
\pgfpathlineto{\pgfqpoint{4.720620in}{1.829381in}}%
\pgfpathlineto{\pgfqpoint{4.728621in}{1.849698in}}%
\pgfpathlineto{\pgfqpoint{4.736620in}{1.869995in}}%
\pgfpathlineto{\pgfqpoint{4.722237in}{1.858750in}}%
\pgfpathlineto{\pgfqpoint{4.707873in}{1.847663in}}%
\pgfpathlineto{\pgfqpoint{4.693525in}{1.836735in}}%
\pgfpathlineto{\pgfqpoint{4.679195in}{1.825965in}}%
\pgfpathlineto{\pgfqpoint{4.671211in}{1.806228in}}%
\pgfpathlineto{\pgfqpoint{4.663224in}{1.786477in}}%
\pgfpathlineto{\pgfqpoint{4.655235in}{1.766717in}}%
\pgfpathlineto{\pgfqpoint{4.647243in}{1.746953in}}%
\pgfpathclose%
\pgfusepath{fill}%
\end{pgfscope}%
\begin{pgfscope}%
\pgfpathrectangle{\pgfqpoint{1.254980in}{0.150000in}}{\pgfqpoint{5.490039in}{5.490039in}}%
\pgfusepath{clip}%
\pgfsetbuttcap%
\pgfsetroundjoin%
\definecolor{currentfill}{rgb}{0.153894,0.680203,0.504172}%
\pgfsetfillcolor{currentfill}%
\pgfsetfillopacity{0.700000}%
\pgfsetlinewidth{0.000000pt}%
\definecolor{currentstroke}{rgb}{0.000000,0.000000,0.000000}%
\pgfsetstrokecolor{currentstroke}%
\pgfsetdash{}{0pt}%
\pgfpathmoveto{\pgfqpoint{4.985651in}{2.396485in}}%
\pgfpathlineto{\pgfqpoint{5.000223in}{2.411369in}}%
\pgfpathlineto{\pgfqpoint{5.014815in}{2.426419in}}%
\pgfpathlineto{\pgfqpoint{5.029428in}{2.441634in}}%
\pgfpathlineto{\pgfqpoint{5.044062in}{2.457015in}}%
\pgfpathlineto{\pgfqpoint{5.051995in}{2.476069in}}%
\pgfpathlineto{\pgfqpoint{5.059923in}{2.494977in}}%
\pgfpathlineto{\pgfqpoint{5.067844in}{2.513734in}}%
\pgfpathlineto{\pgfqpoint{5.075760in}{2.532340in}}%
\pgfpathlineto{\pgfqpoint{5.061111in}{2.516588in}}%
\pgfpathlineto{\pgfqpoint{5.046483in}{2.501003in}}%
\pgfpathlineto{\pgfqpoint{5.031875in}{2.485585in}}%
\pgfpathlineto{\pgfqpoint{5.017289in}{2.470332in}}%
\pgfpathlineto{\pgfqpoint{5.009388in}{2.452085in}}%
\pgfpathlineto{\pgfqpoint{5.001481in}{2.433692in}}%
\pgfpathlineto{\pgfqpoint{4.993568in}{2.415158in}}%
\pgfpathlineto{\pgfqpoint{4.985651in}{2.396485in}}%
\pgfpathclose%
\pgfusepath{fill}%
\end{pgfscope}%
\begin{pgfscope}%
\pgfpathrectangle{\pgfqpoint{1.254980in}{0.150000in}}{\pgfqpoint{5.490039in}{5.490039in}}%
\pgfusepath{clip}%
\pgfsetbuttcap%
\pgfsetroundjoin%
\definecolor{currentfill}{rgb}{0.132444,0.552216,0.553018}%
\pgfsetfillcolor{currentfill}%
\pgfsetfillopacity{0.700000}%
\pgfsetlinewidth{0.000000pt}%
\definecolor{currentstroke}{rgb}{0.000000,0.000000,0.000000}%
\pgfsetstrokecolor{currentstroke}%
\pgfsetdash{}{0pt}%
\pgfpathmoveto{\pgfqpoint{4.800503in}{2.031016in}}%
\pgfpathlineto{\pgfqpoint{4.814934in}{2.043478in}}%
\pgfpathlineto{\pgfqpoint{4.829384in}{2.056101in}}%
\pgfpathlineto{\pgfqpoint{4.843853in}{2.068885in}}%
\pgfpathlineto{\pgfqpoint{4.858341in}{2.081830in}}%
\pgfpathlineto{\pgfqpoint{4.866328in}{2.102215in}}%
\pgfpathlineto{\pgfqpoint{4.874312in}{2.122524in}}%
\pgfpathlineto{\pgfqpoint{4.882292in}{2.142752in}}%
\pgfpathlineto{\pgfqpoint{4.890268in}{2.162895in}}%
\pgfpathlineto{\pgfqpoint{4.875764in}{2.149461in}}%
\pgfpathlineto{\pgfqpoint{4.861278in}{2.136190in}}%
\pgfpathlineto{\pgfqpoint{4.846813in}{2.123080in}}%
\pgfpathlineto{\pgfqpoint{4.832366in}{2.110133in}}%
\pgfpathlineto{\pgfqpoint{4.824405in}{2.090466in}}%
\pgfpathlineto{\pgfqpoint{4.816441in}{2.070721in}}%
\pgfpathlineto{\pgfqpoint{4.808474in}{2.050903in}}%
\pgfpathlineto{\pgfqpoint{4.800503in}{2.031016in}}%
\pgfpathclose%
\pgfusepath{fill}%
\end{pgfscope}%
\begin{pgfscope}%
\pgfpathrectangle{\pgfqpoint{1.254980in}{0.150000in}}{\pgfqpoint{5.490039in}{5.490039in}}%
\pgfusepath{clip}%
\pgfsetbuttcap%
\pgfsetroundjoin%
\definecolor{currentfill}{rgb}{0.262138,0.242286,0.520837}%
\pgfsetfillcolor{currentfill}%
\pgfsetfillopacity{0.700000}%
\pgfsetlinewidth{0.000000pt}%
\definecolor{currentstroke}{rgb}{0.000000,0.000000,0.000000}%
\pgfsetstrokecolor{currentstroke}%
\pgfsetdash{}{0pt}%
\pgfpathmoveto{\pgfqpoint{4.341292in}{1.242173in}}%
\pgfpathlineto{\pgfqpoint{4.355430in}{1.247181in}}%
\pgfpathlineto{\pgfqpoint{4.369582in}{1.252341in}}%
\pgfpathlineto{\pgfqpoint{4.383746in}{1.257651in}}%
\pgfpathlineto{\pgfqpoint{4.397924in}{1.263113in}}%
\pgfpathlineto{\pgfqpoint{4.405960in}{1.280465in}}%
\pgfpathlineto{\pgfqpoint{4.413993in}{1.297968in}}%
\pgfpathlineto{\pgfqpoint{4.422024in}{1.315613in}}%
\pgfpathlineto{\pgfqpoint{4.430053in}{1.333394in}}%
\pgfpathlineto{\pgfqpoint{4.415869in}{1.327206in}}%
\pgfpathlineto{\pgfqpoint{4.401698in}{1.321170in}}%
\pgfpathlineto{\pgfqpoint{4.387541in}{1.315285in}}%
\pgfpathlineto{\pgfqpoint{4.373397in}{1.309552in}}%
\pgfpathlineto{\pgfqpoint{4.365375in}{1.292486in}}%
\pgfpathlineto{\pgfqpoint{4.357350in}{1.275563in}}%
\pgfpathlineto{\pgfqpoint{4.349322in}{1.258790in}}%
\pgfpathlineto{\pgfqpoint{4.341292in}{1.242173in}}%
\pgfpathclose%
\pgfusepath{fill}%
\end{pgfscope}%
\begin{pgfscope}%
\pgfpathrectangle{\pgfqpoint{1.254980in}{0.150000in}}{\pgfqpoint{5.490039in}{5.490039in}}%
\pgfusepath{clip}%
\pgfsetbuttcap%
\pgfsetroundjoin%
\definecolor{currentfill}{rgb}{0.183898,0.422383,0.556944}%
\pgfsetfillcolor{currentfill}%
\pgfsetfillopacity{0.700000}%
\pgfsetlinewidth{0.000000pt}%
\definecolor{currentstroke}{rgb}{0.000000,0.000000,0.000000}%
\pgfsetstrokecolor{currentstroke}%
\pgfsetdash{}{0pt}%
\pgfpathmoveto{\pgfqpoint{4.615251in}{1.667974in}}%
\pgfpathlineto{\pgfqpoint{4.629554in}{1.677580in}}%
\pgfpathlineto{\pgfqpoint{4.643873in}{1.687342in}}%
\pgfpathlineto{\pgfqpoint{4.658209in}{1.697261in}}%
\pgfpathlineto{\pgfqpoint{4.672561in}{1.707337in}}%
\pgfpathlineto{\pgfqpoint{4.680577in}{1.727667in}}%
\pgfpathlineto{\pgfqpoint{4.688590in}{1.748009in}}%
\pgfpathlineto{\pgfqpoint{4.696602in}{1.768357in}}%
\pgfpathlineto{\pgfqpoint{4.704610in}{1.788705in}}%
\pgfpathlineto{\pgfqpoint{4.690243in}{1.778031in}}%
\pgfpathlineto{\pgfqpoint{4.675893in}{1.767514in}}%
\pgfpathlineto{\pgfqpoint{4.661560in}{1.757155in}}%
\pgfpathlineto{\pgfqpoint{4.647243in}{1.746953in}}%
\pgfpathlineto{\pgfqpoint{4.639249in}{1.727192in}}%
\pgfpathlineto{\pgfqpoint{4.631252in}{1.707438in}}%
\pgfpathlineto{\pgfqpoint{4.623253in}{1.687697in}}%
\pgfpathlineto{\pgfqpoint{4.615251in}{1.667974in}}%
\pgfpathclose%
\pgfusepath{fill}%
\end{pgfscope}%
\begin{pgfscope}%
\pgfpathrectangle{\pgfqpoint{1.254980in}{0.150000in}}{\pgfqpoint{5.490039in}{5.490039in}}%
\pgfusepath{clip}%
\pgfsetbuttcap%
\pgfsetroundjoin%
\definecolor{currentfill}{rgb}{0.266941,0.748751,0.440573}%
\pgfsetfillcolor{currentfill}%
\pgfsetfillopacity{0.700000}%
\pgfsetlinewidth{0.000000pt}%
\definecolor{currentstroke}{rgb}{0.000000,0.000000,0.000000}%
\pgfsetstrokecolor{currentstroke}%
\pgfsetdash{}{0pt}%
\pgfpathmoveto{\pgfqpoint{5.107365in}{2.605170in}}%
\pgfpathlineto{\pgfqpoint{5.122050in}{2.621429in}}%
\pgfpathlineto{\pgfqpoint{5.136758in}{2.637855in}}%
\pgfpathlineto{\pgfqpoint{5.151487in}{2.654450in}}%
\pgfpathlineto{\pgfqpoint{5.159383in}{2.672486in}}%
\pgfpathlineto{\pgfqpoint{5.167273in}{2.690346in}}%
\pgfpathlineto{\pgfqpoint{5.175155in}{2.708026in}}%
\pgfpathlineto{\pgfqpoint{5.183031in}{2.725525in}}%
\pgfpathlineto{\pgfqpoint{5.168287in}{2.708621in}}%
\pgfpathlineto{\pgfqpoint{5.153566in}{2.691886in}}%
\pgfpathlineto{\pgfqpoint{5.138866in}{2.675319in}}%
\pgfpathlineto{\pgfqpoint{5.131001in}{2.658043in}}%
\pgfpathlineto{\pgfqpoint{5.123129in}{2.640591in}}%
\pgfpathlineto{\pgfqpoint{5.115250in}{2.622966in}}%
\pgfpathlineto{\pgfqpoint{5.107365in}{2.605170in}}%
\pgfpathclose%
\pgfusepath{fill}%
\end{pgfscope}%
\begin{pgfscope}%
\pgfpathrectangle{\pgfqpoint{1.254980in}{0.150000in}}{\pgfqpoint{5.490039in}{5.490039in}}%
\pgfusepath{clip}%
\pgfsetbuttcap%
\pgfsetroundjoin%
\definecolor{currentfill}{rgb}{0.231674,0.318106,0.544834}%
\pgfsetfillcolor{currentfill}%
\pgfsetfillopacity{0.700000}%
\pgfsetlinewidth{0.000000pt}%
\definecolor{currentstroke}{rgb}{0.000000,0.000000,0.000000}%
\pgfsetstrokecolor{currentstroke}%
\pgfsetdash{}{0pt}%
\pgfpathmoveto{\pgfqpoint{4.462145in}{1.405747in}}%
\pgfpathlineto{\pgfqpoint{4.476352in}{1.412788in}}%
\pgfpathlineto{\pgfqpoint{4.490574in}{1.419983in}}%
\pgfpathlineto{\pgfqpoint{4.504810in}{1.427330in}}%
\pgfpathlineto{\pgfqpoint{4.519061in}{1.434830in}}%
\pgfpathlineto{\pgfqpoint{4.527088in}{1.453878in}}%
\pgfpathlineto{\pgfqpoint{4.535113in}{1.473016in}}%
\pgfpathlineto{\pgfqpoint{4.543137in}{1.492237in}}%
\pgfpathlineto{\pgfqpoint{4.551158in}{1.511535in}}%
\pgfpathlineto{\pgfqpoint{4.536896in}{1.503358in}}%
\pgfpathlineto{\pgfqpoint{4.522649in}{1.495334in}}%
\pgfpathlineto{\pgfqpoint{4.508418in}{1.487465in}}%
\pgfpathlineto{\pgfqpoint{4.494201in}{1.479750in}}%
\pgfpathlineto{\pgfqpoint{4.486190in}{1.461117in}}%
\pgfpathlineto{\pgfqpoint{4.478178in}{1.442568in}}%
\pgfpathlineto{\pgfqpoint{4.470163in}{1.424109in}}%
\pgfpathlineto{\pgfqpoint{4.462145in}{1.405747in}}%
\pgfpathclose%
\pgfusepath{fill}%
\end{pgfscope}%
\begin{pgfscope}%
\pgfpathrectangle{\pgfqpoint{1.254980in}{0.150000in}}{\pgfqpoint{5.490039in}{5.490039in}}%
\pgfusepath{clip}%
\pgfsetbuttcap%
\pgfsetroundjoin%
\definecolor{currentfill}{rgb}{0.134692,0.658636,0.517649}%
\pgfsetfillcolor{currentfill}%
\pgfsetfillopacity{0.700000}%
\pgfsetlinewidth{0.000000pt}%
\definecolor{currentstroke}{rgb}{0.000000,0.000000,0.000000}%
\pgfsetstrokecolor{currentstroke}%
\pgfsetdash{}{0pt}%
\pgfpathmoveto{\pgfqpoint{4.953930in}{2.320476in}}%
\pgfpathlineto{\pgfqpoint{4.968486in}{2.334961in}}%
\pgfpathlineto{\pgfqpoint{4.983063in}{2.349611in}}%
\pgfpathlineto{\pgfqpoint{4.997660in}{2.364426in}}%
\pgfpathlineto{\pgfqpoint{5.012277in}{2.379406in}}%
\pgfpathlineto{\pgfqpoint{5.020231in}{2.399010in}}%
\pgfpathlineto{\pgfqpoint{5.028180in}{2.418482in}}%
\pgfpathlineto{\pgfqpoint{5.036124in}{2.437819in}}%
\pgfpathlineto{\pgfqpoint{5.044062in}{2.457015in}}%
\pgfpathlineto{\pgfqpoint{5.029428in}{2.441634in}}%
\pgfpathlineto{\pgfqpoint{5.014815in}{2.426419in}}%
\pgfpathlineto{\pgfqpoint{5.000223in}{2.411369in}}%
\pgfpathlineto{\pgfqpoint{4.985651in}{2.396485in}}%
\pgfpathlineto{\pgfqpoint{4.977728in}{2.377677in}}%
\pgfpathlineto{\pgfqpoint{4.969800in}{2.358737in}}%
\pgfpathlineto{\pgfqpoint{4.961867in}{2.339669in}}%
\pgfpathlineto{\pgfqpoint{4.953930in}{2.320476in}}%
\pgfpathclose%
\pgfusepath{fill}%
\end{pgfscope}%
\begin{pgfscope}%
\pgfpathrectangle{\pgfqpoint{1.254980in}{0.150000in}}{\pgfqpoint{5.490039in}{5.490039in}}%
\pgfusepath{clip}%
\pgfsetbuttcap%
\pgfsetroundjoin%
\definecolor{currentfill}{rgb}{0.141935,0.526453,0.555991}%
\pgfsetfillcolor{currentfill}%
\pgfsetfillopacity{0.700000}%
\pgfsetlinewidth{0.000000pt}%
\definecolor{currentstroke}{rgb}{0.000000,0.000000,0.000000}%
\pgfsetstrokecolor{currentstroke}%
\pgfsetdash{}{0pt}%
\pgfpathmoveto{\pgfqpoint{4.768586in}{1.950871in}}%
\pgfpathlineto{\pgfqpoint{4.783002in}{1.962818in}}%
\pgfpathlineto{\pgfqpoint{4.797436in}{1.974925in}}%
\pgfpathlineto{\pgfqpoint{4.811888in}{1.987193in}}%
\pgfpathlineto{\pgfqpoint{4.826359in}{1.999621in}}%
\pgfpathlineto{\pgfqpoint{4.834359in}{2.020264in}}%
\pgfpathlineto{\pgfqpoint{4.842356in}{2.040850in}}%
\pgfpathlineto{\pgfqpoint{4.850350in}{2.061374in}}%
\pgfpathlineto{\pgfqpoint{4.858341in}{2.081830in}}%
\pgfpathlineto{\pgfqpoint{4.843853in}{2.068885in}}%
\pgfpathlineto{\pgfqpoint{4.829384in}{2.056101in}}%
\pgfpathlineto{\pgfqpoint{4.814934in}{2.043478in}}%
\pgfpathlineto{\pgfqpoint{4.800503in}{2.031016in}}%
\pgfpathlineto{\pgfqpoint{4.792528in}{2.011065in}}%
\pgfpathlineto{\pgfqpoint{4.784551in}{1.991054in}}%
\pgfpathlineto{\pgfqpoint{4.776570in}{1.970987in}}%
\pgfpathlineto{\pgfqpoint{4.768586in}{1.950871in}}%
\pgfpathclose%
\pgfusepath{fill}%
\end{pgfscope}%
\begin{pgfscope}%
\pgfpathrectangle{\pgfqpoint{1.254980in}{0.150000in}}{\pgfqpoint{5.490039in}{5.490039in}}%
\pgfusepath{clip}%
\pgfsetbuttcap%
\pgfsetroundjoin%
\definecolor{currentfill}{rgb}{0.194100,0.399323,0.555565}%
\pgfsetfillcolor{currentfill}%
\pgfsetfillopacity{0.700000}%
\pgfsetlinewidth{0.000000pt}%
\definecolor{currentstroke}{rgb}{0.000000,0.000000,0.000000}%
\pgfsetstrokecolor{currentstroke}%
\pgfsetdash{}{0pt}%
\pgfpathmoveto{\pgfqpoint{4.583222in}{1.589378in}}%
\pgfpathlineto{\pgfqpoint{4.597512in}{1.598361in}}%
\pgfpathlineto{\pgfqpoint{4.611817in}{1.607499in}}%
\pgfpathlineto{\pgfqpoint{4.626139in}{1.616792in}}%
\pgfpathlineto{\pgfqpoint{4.640477in}{1.626241in}}%
\pgfpathlineto{\pgfqpoint{4.648501in}{1.646469in}}%
\pgfpathlineto{\pgfqpoint{4.656524in}{1.666732in}}%
\pgfpathlineto{\pgfqpoint{4.664544in}{1.687023in}}%
\pgfpathlineto{\pgfqpoint{4.672561in}{1.707337in}}%
\pgfpathlineto{\pgfqpoint{4.658209in}{1.697261in}}%
\pgfpathlineto{\pgfqpoint{4.643873in}{1.687342in}}%
\pgfpathlineto{\pgfqpoint{4.629554in}{1.677580in}}%
\pgfpathlineto{\pgfqpoint{4.615251in}{1.667974in}}%
\pgfpathlineto{\pgfqpoint{4.607247in}{1.648275in}}%
\pgfpathlineto{\pgfqpoint{4.599241in}{1.628606in}}%
\pgfpathlineto{\pgfqpoint{4.591233in}{1.608972in}}%
\pgfpathlineto{\pgfqpoint{4.583222in}{1.589378in}}%
\pgfpathclose%
\pgfusepath{fill}%
\end{pgfscope}%
\begin{pgfscope}%
\pgfpathrectangle{\pgfqpoint{1.254980in}{0.150000in}}{\pgfqpoint{5.490039in}{5.490039in}}%
\pgfusepath{clip}%
\pgfsetbuttcap%
\pgfsetroundjoin%
\definecolor{currentfill}{rgb}{0.123444,0.636809,0.528763}%
\pgfsetfillcolor{currentfill}%
\pgfsetfillopacity{0.700000}%
\pgfsetlinewidth{0.000000pt}%
\definecolor{currentstroke}{rgb}{0.000000,0.000000,0.000000}%
\pgfsetstrokecolor{currentstroke}%
\pgfsetdash{}{0pt}%
\pgfpathmoveto{\pgfqpoint{4.922133in}{2.242531in}}%
\pgfpathlineto{\pgfqpoint{4.936673in}{2.256587in}}%
\pgfpathlineto{\pgfqpoint{4.951234in}{2.270806in}}%
\pgfpathlineto{\pgfqpoint{4.965814in}{2.285190in}}%
\pgfpathlineto{\pgfqpoint{4.980414in}{2.299738in}}%
\pgfpathlineto{\pgfqpoint{4.988387in}{2.319834in}}%
\pgfpathlineto{\pgfqpoint{4.996355in}{2.339814in}}%
\pgfpathlineto{\pgfqpoint{5.004319in}{2.359672in}}%
\pgfpathlineto{\pgfqpoint{5.012277in}{2.379406in}}%
\pgfpathlineto{\pgfqpoint{4.997660in}{2.364426in}}%
\pgfpathlineto{\pgfqpoint{4.983063in}{2.349611in}}%
\pgfpathlineto{\pgfqpoint{4.968486in}{2.334961in}}%
\pgfpathlineto{\pgfqpoint{4.953930in}{2.320476in}}%
\pgfpathlineto{\pgfqpoint{4.945987in}{2.301162in}}%
\pgfpathlineto{\pgfqpoint{4.938040in}{2.281731in}}%
\pgfpathlineto{\pgfqpoint{4.930089in}{2.262186in}}%
\pgfpathlineto{\pgfqpoint{4.922133in}{2.242531in}}%
\pgfpathclose%
\pgfusepath{fill}%
\end{pgfscope}%
\begin{pgfscope}%
\pgfpathrectangle{\pgfqpoint{1.254980in}{0.150000in}}{\pgfqpoint{5.490039in}{5.490039in}}%
\pgfusepath{clip}%
\pgfsetbuttcap%
\pgfsetroundjoin%
\definecolor{currentfill}{rgb}{0.243113,0.292092,0.538516}%
\pgfsetfillcolor{currentfill}%
\pgfsetfillopacity{0.700000}%
\pgfsetlinewidth{0.000000pt}%
\definecolor{currentstroke}{rgb}{0.000000,0.000000,0.000000}%
\pgfsetstrokecolor{currentstroke}%
\pgfsetdash{}{0pt}%
\pgfpathmoveto{\pgfqpoint{4.430053in}{1.333394in}}%
\pgfpathlineto{\pgfqpoint{4.444252in}{1.339735in}}%
\pgfpathlineto{\pgfqpoint{4.458464in}{1.346227in}}%
\pgfpathlineto{\pgfqpoint{4.472690in}{1.352871in}}%
\pgfpathlineto{\pgfqpoint{4.486931in}{1.359668in}}%
\pgfpathlineto{\pgfqpoint{4.494967in}{1.378291in}}%
\pgfpathlineto{\pgfqpoint{4.503000in}{1.397030in}}%
\pgfpathlineto{\pgfqpoint{4.511031in}{1.415879in}}%
\pgfpathlineto{\pgfqpoint{4.519061in}{1.434830in}}%
\pgfpathlineto{\pgfqpoint{4.504810in}{1.427330in}}%
\pgfpathlineto{\pgfqpoint{4.490574in}{1.419983in}}%
\pgfpathlineto{\pgfqpoint{4.476352in}{1.412788in}}%
\pgfpathlineto{\pgfqpoint{4.462145in}{1.405747in}}%
\pgfpathlineto{\pgfqpoint{4.454126in}{1.387488in}}%
\pgfpathlineto{\pgfqpoint{4.446104in}{1.369339in}}%
\pgfpathlineto{\pgfqpoint{4.438080in}{1.351305in}}%
\pgfpathlineto{\pgfqpoint{4.430053in}{1.333394in}}%
\pgfpathclose%
\pgfusepath{fill}%
\end{pgfscope}%
\begin{pgfscope}%
\pgfpathrectangle{\pgfqpoint{1.254980in}{0.150000in}}{\pgfqpoint{5.490039in}{5.490039in}}%
\pgfusepath{clip}%
\pgfsetbuttcap%
\pgfsetroundjoin%
\definecolor{currentfill}{rgb}{0.232815,0.732247,0.459277}%
\pgfsetfillcolor{currentfill}%
\pgfsetfillopacity{0.700000}%
\pgfsetlinewidth{0.000000pt}%
\definecolor{currentstroke}{rgb}{0.000000,0.000000,0.000000}%
\pgfsetstrokecolor{currentstroke}%
\pgfsetdash{}{0pt}%
\pgfpathmoveto{\pgfqpoint{5.075760in}{2.532340in}}%
\pgfpathlineto{\pgfqpoint{5.090431in}{2.548258in}}%
\pgfpathlineto{\pgfqpoint{5.105123in}{2.564344in}}%
\pgfpathlineto{\pgfqpoint{5.119837in}{2.580598in}}%
\pgfpathlineto{\pgfqpoint{5.127759in}{2.599311in}}%
\pgfpathlineto{\pgfqpoint{5.135675in}{2.617860in}}%
\pgfpathlineto{\pgfqpoint{5.143584in}{2.636241in}}%
\pgfpathlineto{\pgfqpoint{5.151487in}{2.654450in}}%
\pgfpathlineto{\pgfqpoint{5.136758in}{2.637855in}}%
\pgfpathlineto{\pgfqpoint{5.122050in}{2.621429in}}%
\pgfpathlineto{\pgfqpoint{5.107365in}{2.605170in}}%
\pgfpathlineto{\pgfqpoint{5.099473in}{2.587207in}}%
\pgfpathlineto{\pgfqpoint{5.091575in}{2.569079in}}%
\pgfpathlineto{\pgfqpoint{5.083671in}{2.550789in}}%
\pgfpathlineto{\pgfqpoint{5.075760in}{2.532340in}}%
\pgfpathclose%
\pgfusepath{fill}%
\end{pgfscope}%
\begin{pgfscope}%
\pgfpathrectangle{\pgfqpoint{1.254980in}{0.150000in}}{\pgfqpoint{5.490039in}{5.490039in}}%
\pgfusepath{clip}%
\pgfsetbuttcap%
\pgfsetroundjoin%
\definecolor{currentfill}{rgb}{0.150476,0.504369,0.557430}%
\pgfsetfillcolor{currentfill}%
\pgfsetfillopacity{0.700000}%
\pgfsetlinewidth{0.000000pt}%
\definecolor{currentstroke}{rgb}{0.000000,0.000000,0.000000}%
\pgfsetstrokecolor{currentstroke}%
\pgfsetdash{}{0pt}%
\pgfpathmoveto{\pgfqpoint{4.736620in}{1.869995in}}%
\pgfpathlineto{\pgfqpoint{4.751020in}{1.881399in}}%
\pgfpathlineto{\pgfqpoint{4.765438in}{1.892961in}}%
\pgfpathlineto{\pgfqpoint{4.779874in}{1.904684in}}%
\pgfpathlineto{\pgfqpoint{4.794328in}{1.916565in}}%
\pgfpathlineto{\pgfqpoint{4.802340in}{1.937391in}}%
\pgfpathlineto{\pgfqpoint{4.810349in}{1.958179in}}%
\pgfpathlineto{\pgfqpoint{4.818355in}{1.978924in}}%
\pgfpathlineto{\pgfqpoint{4.826359in}{1.999621in}}%
\pgfpathlineto{\pgfqpoint{4.811888in}{1.987193in}}%
\pgfpathlineto{\pgfqpoint{4.797436in}{1.974925in}}%
\pgfpathlineto{\pgfqpoint{4.783002in}{1.962818in}}%
\pgfpathlineto{\pgfqpoint{4.768586in}{1.950871in}}%
\pgfpathlineto{\pgfqpoint{4.760599in}{1.930708in}}%
\pgfpathlineto{\pgfqpoint{4.752609in}{1.910504in}}%
\pgfpathlineto{\pgfqpoint{4.744616in}{1.890265in}}%
\pgfpathlineto{\pgfqpoint{4.736620in}{1.869995in}}%
\pgfpathclose%
\pgfusepath{fill}%
\end{pgfscope}%
\begin{pgfscope}%
\pgfpathrectangle{\pgfqpoint{1.254980in}{0.150000in}}{\pgfqpoint{5.490039in}{5.490039in}}%
\pgfusepath{clip}%
\pgfsetbuttcap%
\pgfsetroundjoin%
\definecolor{currentfill}{rgb}{0.206756,0.371758,0.553117}%
\pgfsetfillcolor{currentfill}%
\pgfsetfillopacity{0.700000}%
\pgfsetlinewidth{0.000000pt}%
\definecolor{currentstroke}{rgb}{0.000000,0.000000,0.000000}%
\pgfsetstrokecolor{currentstroke}%
\pgfsetdash{}{0pt}%
\pgfpathmoveto{\pgfqpoint{4.551158in}{1.511535in}}%
\pgfpathlineto{\pgfqpoint{4.565435in}{1.519867in}}%
\pgfpathlineto{\pgfqpoint{4.579728in}{1.528353in}}%
\pgfpathlineto{\pgfqpoint{4.594036in}{1.536993in}}%
\pgfpathlineto{\pgfqpoint{4.608361in}{1.545788in}}%
\pgfpathlineto{\pgfqpoint{4.616393in}{1.565820in}}%
\pgfpathlineto{\pgfqpoint{4.624423in}{1.585911in}}%
\pgfpathlineto{\pgfqpoint{4.632451in}{1.606053in}}%
\pgfpathlineto{\pgfqpoint{4.640477in}{1.626241in}}%
\pgfpathlineto{\pgfqpoint{4.626139in}{1.616792in}}%
\pgfpathlineto{\pgfqpoint{4.611817in}{1.607499in}}%
\pgfpathlineto{\pgfqpoint{4.597512in}{1.598361in}}%
\pgfpathlineto{\pgfqpoint{4.583222in}{1.589378in}}%
\pgfpathlineto{\pgfqpoint{4.575209in}{1.569832in}}%
\pgfpathlineto{\pgfqpoint{4.567194in}{1.550339in}}%
\pgfpathlineto{\pgfqpoint{4.559177in}{1.530905in}}%
\pgfpathlineto{\pgfqpoint{4.551158in}{1.511535in}}%
\pgfpathclose%
\pgfusepath{fill}%
\end{pgfscope}%
\begin{pgfscope}%
\pgfpathrectangle{\pgfqpoint{1.254980in}{0.150000in}}{\pgfqpoint{5.490039in}{5.490039in}}%
\pgfusepath{clip}%
\pgfsetbuttcap%
\pgfsetroundjoin%
\definecolor{currentfill}{rgb}{0.119423,0.611141,0.538982}%
\pgfsetfillcolor{currentfill}%
\pgfsetfillopacity{0.700000}%
\pgfsetlinewidth{0.000000pt}%
\definecolor{currentstroke}{rgb}{0.000000,0.000000,0.000000}%
\pgfsetstrokecolor{currentstroke}%
\pgfsetdash{}{0pt}%
\pgfpathmoveto{\pgfqpoint{4.890268in}{2.162895in}}%
\pgfpathlineto{\pgfqpoint{4.904792in}{2.176491in}}%
\pgfpathlineto{\pgfqpoint{4.919335in}{2.190250in}}%
\pgfpathlineto{\pgfqpoint{4.933898in}{2.204172in}}%
\pgfpathlineto{\pgfqpoint{4.948481in}{2.218258in}}%
\pgfpathlineto{\pgfqpoint{4.956470in}{2.238783in}}%
\pgfpathlineto{\pgfqpoint{4.964456in}{2.259208in}}%
\pgfpathlineto{\pgfqpoint{4.972437in}{2.279528in}}%
\pgfpathlineto{\pgfqpoint{4.980414in}{2.299738in}}%
\pgfpathlineto{\pgfqpoint{4.965814in}{2.285190in}}%
\pgfpathlineto{\pgfqpoint{4.951234in}{2.270806in}}%
\pgfpathlineto{\pgfqpoint{4.936673in}{2.256587in}}%
\pgfpathlineto{\pgfqpoint{4.922133in}{2.242531in}}%
\pgfpathlineto{\pgfqpoint{4.914173in}{2.222770in}}%
\pgfpathlineto{\pgfqpoint{4.906209in}{2.202908in}}%
\pgfpathlineto{\pgfqpoint{4.898240in}{2.182948in}}%
\pgfpathlineto{\pgfqpoint{4.890268in}{2.162895in}}%
\pgfpathclose%
\pgfusepath{fill}%
\end{pgfscope}%
\begin{pgfscope}%
\pgfpathrectangle{\pgfqpoint{1.254980in}{0.150000in}}{\pgfqpoint{5.490039in}{5.490039in}}%
\pgfusepath{clip}%
\pgfsetbuttcap%
\pgfsetroundjoin%
\definecolor{currentfill}{rgb}{0.160665,0.478540,0.558115}%
\pgfsetfillcolor{currentfill}%
\pgfsetfillopacity{0.700000}%
\pgfsetlinewidth{0.000000pt}%
\definecolor{currentstroke}{rgb}{0.000000,0.000000,0.000000}%
\pgfsetstrokecolor{currentstroke}%
\pgfsetdash{}{0pt}%
\pgfpathmoveto{\pgfqpoint{4.704610in}{1.788705in}}%
\pgfpathlineto{\pgfqpoint{4.718995in}{1.799537in}}%
\pgfpathlineto{\pgfqpoint{4.733397in}{1.810527in}}%
\pgfpathlineto{\pgfqpoint{4.747816in}{1.821676in}}%
\pgfpathlineto{\pgfqpoint{4.762254in}{1.832983in}}%
\pgfpathlineto{\pgfqpoint{4.770276in}{1.853910in}}%
\pgfpathlineto{\pgfqpoint{4.778296in}{1.874819in}}%
\pgfpathlineto{\pgfqpoint{4.786313in}{1.895706in}}%
\pgfpathlineto{\pgfqpoint{4.794328in}{1.916565in}}%
\pgfpathlineto{\pgfqpoint{4.779874in}{1.904684in}}%
\pgfpathlineto{\pgfqpoint{4.765438in}{1.892961in}}%
\pgfpathlineto{\pgfqpoint{4.751020in}{1.881399in}}%
\pgfpathlineto{\pgfqpoint{4.736620in}{1.869995in}}%
\pgfpathlineto{\pgfqpoint{4.728621in}{1.849698in}}%
\pgfpathlineto{\pgfqpoint{4.720620in}{1.829381in}}%
\pgfpathlineto{\pgfqpoint{4.712617in}{1.809048in}}%
\pgfpathlineto{\pgfqpoint{4.704610in}{1.788705in}}%
\pgfpathclose%
\pgfusepath{fill}%
\end{pgfscope}%
\begin{pgfscope}%
\pgfpathrectangle{\pgfqpoint{1.254980in}{0.150000in}}{\pgfqpoint{5.490039in}{5.490039in}}%
\pgfusepath{clip}%
\pgfsetbuttcap%
\pgfsetroundjoin%
\definecolor{currentfill}{rgb}{0.196571,0.711827,0.479221}%
\pgfsetfillcolor{currentfill}%
\pgfsetfillopacity{0.700000}%
\pgfsetlinewidth{0.000000pt}%
\definecolor{currentstroke}{rgb}{0.000000,0.000000,0.000000}%
\pgfsetstrokecolor{currentstroke}%
\pgfsetdash{}{0pt}%
\pgfpathmoveto{\pgfqpoint{5.044062in}{2.457015in}}%
\pgfpathlineto{\pgfqpoint{5.058717in}{2.472563in}}%
\pgfpathlineto{\pgfqpoint{5.073393in}{2.488277in}}%
\pgfpathlineto{\pgfqpoint{5.088091in}{2.504158in}}%
\pgfpathlineto{\pgfqpoint{5.096036in}{2.523500in}}%
\pgfpathlineto{\pgfqpoint{5.103976in}{2.542689in}}%
\pgfpathlineto{\pgfqpoint{5.111910in}{2.561723in}}%
\pgfpathlineto{\pgfqpoint{5.119837in}{2.580598in}}%
\pgfpathlineto{\pgfqpoint{5.105123in}{2.564344in}}%
\pgfpathlineto{\pgfqpoint{5.090431in}{2.548258in}}%
\pgfpathlineto{\pgfqpoint{5.075760in}{2.532340in}}%
\pgfpathlineto{\pgfqpoint{5.067844in}{2.513734in}}%
\pgfpathlineto{\pgfqpoint{5.059923in}{2.494977in}}%
\pgfpathlineto{\pgfqpoint{5.051995in}{2.476069in}}%
\pgfpathlineto{\pgfqpoint{5.044062in}{2.457015in}}%
\pgfpathclose%
\pgfusepath{fill}%
\end{pgfscope}%
\begin{pgfscope}%
\pgfpathrectangle{\pgfqpoint{1.254980in}{0.150000in}}{\pgfqpoint{5.490039in}{5.490039in}}%
\pgfusepath{clip}%
\pgfsetbuttcap%
\pgfsetroundjoin%
\definecolor{currentfill}{rgb}{0.252194,0.269783,0.531579}%
\pgfsetfillcolor{currentfill}%
\pgfsetfillopacity{0.700000}%
\pgfsetlinewidth{0.000000pt}%
\definecolor{currentstroke}{rgb}{0.000000,0.000000,0.000000}%
\pgfsetstrokecolor{currentstroke}%
\pgfsetdash{}{0pt}%
\pgfpathmoveto{\pgfqpoint{4.397924in}{1.263113in}}%
\pgfpathlineto{\pgfqpoint{4.412115in}{1.268725in}}%
\pgfpathlineto{\pgfqpoint{4.426319in}{1.274489in}}%
\pgfpathlineto{\pgfqpoint{4.440538in}{1.280403in}}%
\pgfpathlineto{\pgfqpoint{4.454770in}{1.286469in}}%
\pgfpathlineto{\pgfqpoint{4.462813in}{1.304561in}}%
\pgfpathlineto{\pgfqpoint{4.470855in}{1.322796in}}%
\pgfpathlineto{\pgfqpoint{4.478894in}{1.341167in}}%
\pgfpathlineto{\pgfqpoint{4.486931in}{1.359668in}}%
\pgfpathlineto{\pgfqpoint{4.472690in}{1.352871in}}%
\pgfpathlineto{\pgfqpoint{4.458464in}{1.346227in}}%
\pgfpathlineto{\pgfqpoint{4.444252in}{1.339735in}}%
\pgfpathlineto{\pgfqpoint{4.430053in}{1.333394in}}%
\pgfpathlineto{\pgfqpoint{4.422024in}{1.315613in}}%
\pgfpathlineto{\pgfqpoint{4.413993in}{1.297968in}}%
\pgfpathlineto{\pgfqpoint{4.405960in}{1.280465in}}%
\pgfpathlineto{\pgfqpoint{4.397924in}{1.263113in}}%
\pgfpathclose%
\pgfusepath{fill}%
\end{pgfscope}%
\begin{pgfscope}%
\pgfpathrectangle{\pgfqpoint{1.254980in}{0.150000in}}{\pgfqpoint{5.490039in}{5.490039in}}%
\pgfusepath{clip}%
\pgfsetbuttcap%
\pgfsetroundjoin%
\definecolor{currentfill}{rgb}{0.218130,0.347432,0.550038}%
\pgfsetfillcolor{currentfill}%
\pgfsetfillopacity{0.700000}%
\pgfsetlinewidth{0.000000pt}%
\definecolor{currentstroke}{rgb}{0.000000,0.000000,0.000000}%
\pgfsetstrokecolor{currentstroke}%
\pgfsetdash{}{0pt}%
\pgfpathmoveto{\pgfqpoint{4.519061in}{1.434830in}}%
\pgfpathlineto{\pgfqpoint{4.533326in}{1.442484in}}%
\pgfpathlineto{\pgfqpoint{4.547607in}{1.450291in}}%
\pgfpathlineto{\pgfqpoint{4.561903in}{1.458252in}}%
\pgfpathlineto{\pgfqpoint{4.576215in}{1.466366in}}%
\pgfpathlineto{\pgfqpoint{4.584254in}{1.486103in}}%
\pgfpathlineto{\pgfqpoint{4.592291in}{1.505923in}}%
\pgfpathlineto{\pgfqpoint{4.600327in}{1.525821in}}%
\pgfpathlineto{\pgfqpoint{4.608361in}{1.545788in}}%
\pgfpathlineto{\pgfqpoint{4.594036in}{1.536993in}}%
\pgfpathlineto{\pgfqpoint{4.579728in}{1.528353in}}%
\pgfpathlineto{\pgfqpoint{4.565435in}{1.519867in}}%
\pgfpathlineto{\pgfqpoint{4.551158in}{1.511535in}}%
\pgfpathlineto{\pgfqpoint{4.543137in}{1.492237in}}%
\pgfpathlineto{\pgfqpoint{4.535113in}{1.473016in}}%
\pgfpathlineto{\pgfqpoint{4.527088in}{1.453878in}}%
\pgfpathlineto{\pgfqpoint{4.519061in}{1.434830in}}%
\pgfpathclose%
\pgfusepath{fill}%
\end{pgfscope}%
\begin{pgfscope}%
\pgfpathrectangle{\pgfqpoint{1.254980in}{0.150000in}}{\pgfqpoint{5.490039in}{5.490039in}}%
\pgfusepath{clip}%
\pgfsetbuttcap%
\pgfsetroundjoin%
\definecolor{currentfill}{rgb}{0.122606,0.585371,0.546557}%
\pgfsetfillcolor{currentfill}%
\pgfsetfillopacity{0.700000}%
\pgfsetlinewidth{0.000000pt}%
\definecolor{currentstroke}{rgb}{0.000000,0.000000,0.000000}%
\pgfsetstrokecolor{currentstroke}%
\pgfsetdash{}{0pt}%
\pgfpathmoveto{\pgfqpoint{4.858341in}{2.081830in}}%
\pgfpathlineto{\pgfqpoint{4.872848in}{2.094938in}}%
\pgfpathlineto{\pgfqpoint{4.887374in}{2.108207in}}%
\pgfpathlineto{\pgfqpoint{4.901919in}{2.121638in}}%
\pgfpathlineto{\pgfqpoint{4.916484in}{2.135231in}}%
\pgfpathlineto{\pgfqpoint{4.924489in}{2.156118in}}%
\pgfpathlineto{\pgfqpoint{4.932490in}{2.176921in}}%
\pgfpathlineto{\pgfqpoint{4.940487in}{2.197635in}}%
\pgfpathlineto{\pgfqpoint{4.948481in}{2.218258in}}%
\pgfpathlineto{\pgfqpoint{4.933898in}{2.204172in}}%
\pgfpathlineto{\pgfqpoint{4.919335in}{2.190250in}}%
\pgfpathlineto{\pgfqpoint{4.904792in}{2.176491in}}%
\pgfpathlineto{\pgfqpoint{4.890268in}{2.162895in}}%
\pgfpathlineto{\pgfqpoint{4.882292in}{2.142752in}}%
\pgfpathlineto{\pgfqpoint{4.874312in}{2.122524in}}%
\pgfpathlineto{\pgfqpoint{4.866328in}{2.102215in}}%
\pgfpathlineto{\pgfqpoint{4.858341in}{2.081830in}}%
\pgfpathclose%
\pgfusepath{fill}%
\end{pgfscope}%
\begin{pgfscope}%
\pgfpathrectangle{\pgfqpoint{1.254980in}{0.150000in}}{\pgfqpoint{5.490039in}{5.490039in}}%
\pgfusepath{clip}%
\pgfsetbuttcap%
\pgfsetroundjoin%
\definecolor{currentfill}{rgb}{0.171176,0.452530,0.557965}%
\pgfsetfillcolor{currentfill}%
\pgfsetfillopacity{0.700000}%
\pgfsetlinewidth{0.000000pt}%
\definecolor{currentstroke}{rgb}{0.000000,0.000000,0.000000}%
\pgfsetstrokecolor{currentstroke}%
\pgfsetdash{}{0pt}%
\pgfpathmoveto{\pgfqpoint{4.672561in}{1.707337in}}%
\pgfpathlineto{\pgfqpoint{4.686931in}{1.717569in}}%
\pgfpathlineto{\pgfqpoint{4.701317in}{1.727958in}}%
\pgfpathlineto{\pgfqpoint{4.715720in}{1.738505in}}%
\pgfpathlineto{\pgfqpoint{4.730141in}{1.749209in}}%
\pgfpathlineto{\pgfqpoint{4.738173in}{1.770151in}}%
\pgfpathlineto{\pgfqpoint{4.746202in}{1.791098in}}%
\pgfpathlineto{\pgfqpoint{4.754229in}{1.812044in}}%
\pgfpathlineto{\pgfqpoint{4.762254in}{1.832983in}}%
\pgfpathlineto{\pgfqpoint{4.747816in}{1.821676in}}%
\pgfpathlineto{\pgfqpoint{4.733397in}{1.810527in}}%
\pgfpathlineto{\pgfqpoint{4.718995in}{1.799537in}}%
\pgfpathlineto{\pgfqpoint{4.704610in}{1.788705in}}%
\pgfpathlineto{\pgfqpoint{4.696602in}{1.768357in}}%
\pgfpathlineto{\pgfqpoint{4.688590in}{1.748009in}}%
\pgfpathlineto{\pgfqpoint{4.680577in}{1.727667in}}%
\pgfpathlineto{\pgfqpoint{4.672561in}{1.707337in}}%
\pgfpathclose%
\pgfusepath{fill}%
\end{pgfscope}%
\begin{pgfscope}%
\pgfpathrectangle{\pgfqpoint{1.254980in}{0.150000in}}{\pgfqpoint{5.490039in}{5.490039in}}%
\pgfusepath{clip}%
\pgfsetbuttcap%
\pgfsetroundjoin%
\definecolor{currentfill}{rgb}{0.166383,0.690856,0.496502}%
\pgfsetfillcolor{currentfill}%
\pgfsetfillopacity{0.700000}%
\pgfsetlinewidth{0.000000pt}%
\definecolor{currentstroke}{rgb}{0.000000,0.000000,0.000000}%
\pgfsetstrokecolor{currentstroke}%
\pgfsetdash{}{0pt}%
\pgfpathmoveto{\pgfqpoint{5.012277in}{2.379406in}}%
\pgfpathlineto{\pgfqpoint{5.026916in}{2.394551in}}%
\pgfpathlineto{\pgfqpoint{5.041575in}{2.409862in}}%
\pgfpathlineto{\pgfqpoint{5.056255in}{2.425339in}}%
\pgfpathlineto{\pgfqpoint{5.064222in}{2.445255in}}%
\pgfpathlineto{\pgfqpoint{5.072183in}{2.465032in}}%
\pgfpathlineto{\pgfqpoint{5.080140in}{2.484668in}}%
\pgfpathlineto{\pgfqpoint{5.088091in}{2.504158in}}%
\pgfpathlineto{\pgfqpoint{5.073393in}{2.488277in}}%
\pgfpathlineto{\pgfqpoint{5.058717in}{2.472563in}}%
\pgfpathlineto{\pgfqpoint{5.044062in}{2.457015in}}%
\pgfpathlineto{\pgfqpoint{5.036124in}{2.437819in}}%
\pgfpathlineto{\pgfqpoint{5.028180in}{2.418482in}}%
\pgfpathlineto{\pgfqpoint{5.020231in}{2.399010in}}%
\pgfpathlineto{\pgfqpoint{5.012277in}{2.379406in}}%
\pgfpathclose%
\pgfusepath{fill}%
\end{pgfscope}%
\begin{pgfscope}%
\pgfpathrectangle{\pgfqpoint{1.254980in}{0.150000in}}{\pgfqpoint{5.490039in}{5.490039in}}%
\pgfusepath{clip}%
\pgfsetbuttcap%
\pgfsetroundjoin%
\definecolor{currentfill}{rgb}{0.129933,0.559582,0.551864}%
\pgfsetfillcolor{currentfill}%
\pgfsetfillopacity{0.700000}%
\pgfsetlinewidth{0.000000pt}%
\definecolor{currentstroke}{rgb}{0.000000,0.000000,0.000000}%
\pgfsetstrokecolor{currentstroke}%
\pgfsetdash{}{0pt}%
\pgfpathmoveto{\pgfqpoint{4.826359in}{1.999621in}}%
\pgfpathlineto{\pgfqpoint{4.840848in}{2.012209in}}%
\pgfpathlineto{\pgfqpoint{4.855357in}{2.024959in}}%
\pgfpathlineto{\pgfqpoint{4.869884in}{2.037870in}}%
\pgfpathlineto{\pgfqpoint{4.884431in}{2.050942in}}%
\pgfpathlineto{\pgfqpoint{4.892449in}{2.072116in}}%
\pgfpathlineto{\pgfqpoint{4.900464in}{2.093226in}}%
\pgfpathlineto{\pgfqpoint{4.908476in}{2.114266in}}%
\pgfpathlineto{\pgfqpoint{4.916484in}{2.135231in}}%
\pgfpathlineto{\pgfqpoint{4.901919in}{2.121638in}}%
\pgfpathlineto{\pgfqpoint{4.887374in}{2.108207in}}%
\pgfpathlineto{\pgfqpoint{4.872848in}{2.094938in}}%
\pgfpathlineto{\pgfqpoint{4.858341in}{2.081830in}}%
\pgfpathlineto{\pgfqpoint{4.850350in}{2.061374in}}%
\pgfpathlineto{\pgfqpoint{4.842356in}{2.040850in}}%
\pgfpathlineto{\pgfqpoint{4.834359in}{2.020264in}}%
\pgfpathlineto{\pgfqpoint{4.826359in}{1.999621in}}%
\pgfpathclose%
\pgfusepath{fill}%
\end{pgfscope}%
\begin{pgfscope}%
\pgfpathrectangle{\pgfqpoint{1.254980in}{0.150000in}}{\pgfqpoint{5.490039in}{5.490039in}}%
\pgfusepath{clip}%
\pgfsetbuttcap%
\pgfsetroundjoin%
\definecolor{currentfill}{rgb}{0.229739,0.322361,0.545706}%
\pgfsetfillcolor{currentfill}%
\pgfsetfillopacity{0.700000}%
\pgfsetlinewidth{0.000000pt}%
\definecolor{currentstroke}{rgb}{0.000000,0.000000,0.000000}%
\pgfsetstrokecolor{currentstroke}%
\pgfsetdash{}{0pt}%
\pgfpathmoveto{\pgfqpoint{4.486931in}{1.359668in}}%
\pgfpathlineto{\pgfqpoint{4.501187in}{1.366617in}}%
\pgfpathlineto{\pgfqpoint{4.515456in}{1.373718in}}%
\pgfpathlineto{\pgfqpoint{4.529741in}{1.380972in}}%
\pgfpathlineto{\pgfqpoint{4.544040in}{1.388378in}}%
\pgfpathlineto{\pgfqpoint{4.552086in}{1.407718in}}%
\pgfpathlineto{\pgfqpoint{4.560131in}{1.427167in}}%
\pgfpathlineto{\pgfqpoint{4.568174in}{1.446718in}}%
\pgfpathlineto{\pgfqpoint{4.576215in}{1.466366in}}%
\pgfpathlineto{\pgfqpoint{4.561903in}{1.458252in}}%
\pgfpathlineto{\pgfqpoint{4.547607in}{1.450291in}}%
\pgfpathlineto{\pgfqpoint{4.533326in}{1.442484in}}%
\pgfpathlineto{\pgfqpoint{4.519061in}{1.434830in}}%
\pgfpathlineto{\pgfqpoint{4.511031in}{1.415879in}}%
\pgfpathlineto{\pgfqpoint{4.503000in}{1.397030in}}%
\pgfpathlineto{\pgfqpoint{4.494967in}{1.378291in}}%
\pgfpathlineto{\pgfqpoint{4.486931in}{1.359668in}}%
\pgfpathclose%
\pgfusepath{fill}%
\end{pgfscope}%
\begin{pgfscope}%
\pgfpathrectangle{\pgfqpoint{1.254980in}{0.150000in}}{\pgfqpoint{5.490039in}{5.490039in}}%
\pgfusepath{clip}%
\pgfsetbuttcap%
\pgfsetroundjoin%
\definecolor{currentfill}{rgb}{0.182256,0.426184,0.557120}%
\pgfsetfillcolor{currentfill}%
\pgfsetfillopacity{0.700000}%
\pgfsetlinewidth{0.000000pt}%
\definecolor{currentstroke}{rgb}{0.000000,0.000000,0.000000}%
\pgfsetstrokecolor{currentstroke}%
\pgfsetdash{}{0pt}%
\pgfpathmoveto{\pgfqpoint{4.640477in}{1.626241in}}%
\pgfpathlineto{\pgfqpoint{4.654832in}{1.635846in}}%
\pgfpathlineto{\pgfqpoint{4.669203in}{1.645607in}}%
\pgfpathlineto{\pgfqpoint{4.683590in}{1.655524in}}%
\pgfpathlineto{\pgfqpoint{4.697995in}{1.665597in}}%
\pgfpathlineto{\pgfqpoint{4.706034in}{1.686465in}}%
\pgfpathlineto{\pgfqpoint{4.714072in}{1.707360in}}%
\pgfpathlineto{\pgfqpoint{4.722108in}{1.728276in}}%
\pgfpathlineto{\pgfqpoint{4.730141in}{1.749209in}}%
\pgfpathlineto{\pgfqpoint{4.715720in}{1.738505in}}%
\pgfpathlineto{\pgfqpoint{4.701317in}{1.727958in}}%
\pgfpathlineto{\pgfqpoint{4.686931in}{1.717569in}}%
\pgfpathlineto{\pgfqpoint{4.672561in}{1.707337in}}%
\pgfpathlineto{\pgfqpoint{4.664544in}{1.687023in}}%
\pgfpathlineto{\pgfqpoint{4.656524in}{1.666732in}}%
\pgfpathlineto{\pgfqpoint{4.648501in}{1.646469in}}%
\pgfpathlineto{\pgfqpoint{4.640477in}{1.626241in}}%
\pgfpathclose%
\pgfusepath{fill}%
\end{pgfscope}%
\begin{pgfscope}%
\pgfpathrectangle{\pgfqpoint{1.254980in}{0.150000in}}{\pgfqpoint{5.490039in}{5.490039in}}%
\pgfusepath{clip}%
\pgfsetbuttcap%
\pgfsetroundjoin%
\definecolor{currentfill}{rgb}{0.140210,0.665859,0.513427}%
\pgfsetfillcolor{currentfill}%
\pgfsetfillopacity{0.700000}%
\pgfsetlinewidth{0.000000pt}%
\definecolor{currentstroke}{rgb}{0.000000,0.000000,0.000000}%
\pgfsetstrokecolor{currentstroke}%
\pgfsetdash{}{0pt}%
\pgfpathmoveto{\pgfqpoint{4.980414in}{2.299738in}}%
\pgfpathlineto{\pgfqpoint{4.995035in}{2.314450in}}%
\pgfpathlineto{\pgfqpoint{5.009677in}{2.329327in}}%
\pgfpathlineto{\pgfqpoint{5.024339in}{2.344370in}}%
\pgfpathlineto{\pgfqpoint{5.032325in}{2.364801in}}%
\pgfpathlineto{\pgfqpoint{5.040307in}{2.385108in}}%
\pgfpathlineto{\pgfqpoint{5.048283in}{2.405289in}}%
\pgfpathlineto{\pgfqpoint{5.056255in}{2.425339in}}%
\pgfpathlineto{\pgfqpoint{5.041575in}{2.409862in}}%
\pgfpathlineto{\pgfqpoint{5.026916in}{2.394551in}}%
\pgfpathlineto{\pgfqpoint{5.012277in}{2.379406in}}%
\pgfpathlineto{\pgfqpoint{5.004319in}{2.359672in}}%
\pgfpathlineto{\pgfqpoint{4.996355in}{2.339814in}}%
\pgfpathlineto{\pgfqpoint{4.988387in}{2.319834in}}%
\pgfpathlineto{\pgfqpoint{4.980414in}{2.299738in}}%
\pgfpathclose%
\pgfusepath{fill}%
\end{pgfscope}%
\begin{pgfscope}%
\pgfpathrectangle{\pgfqpoint{1.254980in}{0.150000in}}{\pgfqpoint{5.490039in}{5.490039in}}%
\pgfusepath{clip}%
\pgfsetbuttcap%
\pgfsetroundjoin%
\definecolor{currentfill}{rgb}{0.139147,0.533812,0.555298}%
\pgfsetfillcolor{currentfill}%
\pgfsetfillopacity{0.700000}%
\pgfsetlinewidth{0.000000pt}%
\definecolor{currentstroke}{rgb}{0.000000,0.000000,0.000000}%
\pgfsetstrokecolor{currentstroke}%
\pgfsetdash{}{0pt}%
\pgfpathmoveto{\pgfqpoint{4.794328in}{1.916565in}}%
\pgfpathlineto{\pgfqpoint{4.808800in}{1.928607in}}%
\pgfpathlineto{\pgfqpoint{4.823291in}{1.940808in}}%
\pgfpathlineto{\pgfqpoint{4.837800in}{1.953169in}}%
\pgfpathlineto{\pgfqpoint{4.852328in}{1.965691in}}%
\pgfpathlineto{\pgfqpoint{4.860358in}{1.987076in}}%
\pgfpathlineto{\pgfqpoint{4.868385in}{2.008417in}}%
\pgfpathlineto{\pgfqpoint{4.876410in}{2.029707in}}%
\pgfpathlineto{\pgfqpoint{4.884431in}{2.050942in}}%
\pgfpathlineto{\pgfqpoint{4.869884in}{2.037870in}}%
\pgfpathlineto{\pgfqpoint{4.855357in}{2.024959in}}%
\pgfpathlineto{\pgfqpoint{4.840848in}{2.012209in}}%
\pgfpathlineto{\pgfqpoint{4.826359in}{1.999621in}}%
\pgfpathlineto{\pgfqpoint{4.818355in}{1.978924in}}%
\pgfpathlineto{\pgfqpoint{4.810349in}{1.958179in}}%
\pgfpathlineto{\pgfqpoint{4.802340in}{1.937391in}}%
\pgfpathlineto{\pgfqpoint{4.794328in}{1.916565in}}%
\pgfpathclose%
\pgfusepath{fill}%
\end{pgfscope}%
\begin{pgfscope}%
\pgfpathrectangle{\pgfqpoint{1.254980in}{0.150000in}}{\pgfqpoint{5.490039in}{5.490039in}}%
\pgfusepath{clip}%
\pgfsetbuttcap%
\pgfsetroundjoin%
\definecolor{currentfill}{rgb}{0.194100,0.399323,0.555565}%
\pgfsetfillcolor{currentfill}%
\pgfsetfillopacity{0.700000}%
\pgfsetlinewidth{0.000000pt}%
\definecolor{currentstroke}{rgb}{0.000000,0.000000,0.000000}%
\pgfsetstrokecolor{currentstroke}%
\pgfsetdash{}{0pt}%
\pgfpathmoveto{\pgfqpoint{4.608361in}{1.545788in}}%
\pgfpathlineto{\pgfqpoint{4.622701in}{1.554738in}}%
\pgfpathlineto{\pgfqpoint{4.637057in}{1.563843in}}%
\pgfpathlineto{\pgfqpoint{4.651430in}{1.573103in}}%
\pgfpathlineto{\pgfqpoint{4.665819in}{1.582518in}}%
\pgfpathlineto{\pgfqpoint{4.673866in}{1.603217in}}%
\pgfpathlineto{\pgfqpoint{4.681911in}{1.623967in}}%
\pgfpathlineto{\pgfqpoint{4.689954in}{1.644762in}}%
\pgfpathlineto{\pgfqpoint{4.697995in}{1.665597in}}%
\pgfpathlineto{\pgfqpoint{4.683590in}{1.655524in}}%
\pgfpathlineto{\pgfqpoint{4.669203in}{1.645607in}}%
\pgfpathlineto{\pgfqpoint{4.654832in}{1.635846in}}%
\pgfpathlineto{\pgfqpoint{4.640477in}{1.626241in}}%
\pgfpathlineto{\pgfqpoint{4.632451in}{1.606053in}}%
\pgfpathlineto{\pgfqpoint{4.624423in}{1.585911in}}%
\pgfpathlineto{\pgfqpoint{4.616393in}{1.565820in}}%
\pgfpathlineto{\pgfqpoint{4.608361in}{1.545788in}}%
\pgfpathclose%
\pgfusepath{fill}%
\end{pgfscope}%
\begin{pgfscope}%
\pgfpathrectangle{\pgfqpoint{1.254980in}{0.150000in}}{\pgfqpoint{5.490039in}{5.490039in}}%
\pgfusepath{clip}%
\pgfsetbuttcap%
\pgfsetroundjoin%
\definecolor{currentfill}{rgb}{0.241237,0.296485,0.539709}%
\pgfsetfillcolor{currentfill}%
\pgfsetfillopacity{0.700000}%
\pgfsetlinewidth{0.000000pt}%
\definecolor{currentstroke}{rgb}{0.000000,0.000000,0.000000}%
\pgfsetstrokecolor{currentstroke}%
\pgfsetdash{}{0pt}%
\pgfpathmoveto{\pgfqpoint{4.454770in}{1.286469in}}%
\pgfpathlineto{\pgfqpoint{4.469016in}{1.292687in}}%
\pgfpathlineto{\pgfqpoint{4.483276in}{1.299056in}}%
\pgfpathlineto{\pgfqpoint{4.497550in}{1.305576in}}%
\pgfpathlineto{\pgfqpoint{4.511839in}{1.312248in}}%
\pgfpathlineto{\pgfqpoint{4.519892in}{1.331083in}}%
\pgfpathlineto{\pgfqpoint{4.527943in}{1.350054in}}%
\pgfpathlineto{\pgfqpoint{4.535992in}{1.369155in}}%
\pgfpathlineto{\pgfqpoint{4.544040in}{1.388378in}}%
\pgfpathlineto{\pgfqpoint{4.529741in}{1.380972in}}%
\pgfpathlineto{\pgfqpoint{4.515456in}{1.373718in}}%
\pgfpathlineto{\pgfqpoint{4.501187in}{1.366617in}}%
\pgfpathlineto{\pgfqpoint{4.486931in}{1.359668in}}%
\pgfpathlineto{\pgfqpoint{4.478894in}{1.341167in}}%
\pgfpathlineto{\pgfqpoint{4.470855in}{1.322796in}}%
\pgfpathlineto{\pgfqpoint{4.462813in}{1.304561in}}%
\pgfpathlineto{\pgfqpoint{4.454770in}{1.286469in}}%
\pgfpathclose%
\pgfusepath{fill}%
\end{pgfscope}%
\begin{pgfscope}%
\pgfpathrectangle{\pgfqpoint{1.254980in}{0.150000in}}{\pgfqpoint{5.490039in}{5.490039in}}%
\pgfusepath{clip}%
\pgfsetbuttcap%
\pgfsetroundjoin%
\definecolor{currentfill}{rgb}{0.124780,0.640461,0.527068}%
\pgfsetfillcolor{currentfill}%
\pgfsetfillopacity{0.700000}%
\pgfsetlinewidth{0.000000pt}%
\definecolor{currentstroke}{rgb}{0.000000,0.000000,0.000000}%
\pgfsetstrokecolor{currentstroke}%
\pgfsetdash{}{0pt}%
\pgfpathmoveto{\pgfqpoint{4.948481in}{2.218258in}}%
\pgfpathlineto{\pgfqpoint{4.963084in}{2.232507in}}%
\pgfpathlineto{\pgfqpoint{4.977707in}{2.246920in}}%
\pgfpathlineto{\pgfqpoint{4.992350in}{2.261498in}}%
\pgfpathlineto{\pgfqpoint{5.000354in}{2.282380in}}%
\pgfpathlineto{\pgfqpoint{5.008353in}{2.303155in}}%
\pgfpathlineto{\pgfqpoint{5.016348in}{2.323820in}}%
\pgfpathlineto{\pgfqpoint{5.024339in}{2.344370in}}%
\pgfpathlineto{\pgfqpoint{5.009677in}{2.329327in}}%
\pgfpathlineto{\pgfqpoint{4.995035in}{2.314450in}}%
\pgfpathlineto{\pgfqpoint{4.980414in}{2.299738in}}%
\pgfpathlineto{\pgfqpoint{4.972437in}{2.279528in}}%
\pgfpathlineto{\pgfqpoint{4.964456in}{2.259208in}}%
\pgfpathlineto{\pgfqpoint{4.956470in}{2.238783in}}%
\pgfpathlineto{\pgfqpoint{4.948481in}{2.218258in}}%
\pgfpathclose%
\pgfusepath{fill}%
\end{pgfscope}%
\begin{pgfscope}%
\pgfpathrectangle{\pgfqpoint{1.254980in}{0.150000in}}{\pgfqpoint{5.490039in}{5.490039in}}%
\pgfusepath{clip}%
\pgfsetbuttcap%
\pgfsetroundjoin%
\definecolor{currentfill}{rgb}{0.149039,0.508051,0.557250}%
\pgfsetfillcolor{currentfill}%
\pgfsetfillopacity{0.700000}%
\pgfsetlinewidth{0.000000pt}%
\definecolor{currentstroke}{rgb}{0.000000,0.000000,0.000000}%
\pgfsetstrokecolor{currentstroke}%
\pgfsetdash{}{0pt}%
\pgfpathmoveto{\pgfqpoint{4.762254in}{1.832983in}}%
\pgfpathlineto{\pgfqpoint{4.776709in}{1.844448in}}%
\pgfpathlineto{\pgfqpoint{4.791182in}{1.856072in}}%
\pgfpathlineto{\pgfqpoint{4.805673in}{1.867856in}}%
\pgfpathlineto{\pgfqpoint{4.820183in}{1.879798in}}%
\pgfpathlineto{\pgfqpoint{4.828223in}{1.901313in}}%
\pgfpathlineto{\pgfqpoint{4.836261in}{1.922804in}}%
\pgfpathlineto{\pgfqpoint{4.844296in}{1.944265in}}%
\pgfpathlineto{\pgfqpoint{4.852328in}{1.965691in}}%
\pgfpathlineto{\pgfqpoint{4.837800in}{1.953169in}}%
\pgfpathlineto{\pgfqpoint{4.823291in}{1.940808in}}%
\pgfpathlineto{\pgfqpoint{4.808800in}{1.928607in}}%
\pgfpathlineto{\pgfqpoint{4.794328in}{1.916565in}}%
\pgfpathlineto{\pgfqpoint{4.786313in}{1.895706in}}%
\pgfpathlineto{\pgfqpoint{4.778296in}{1.874819in}}%
\pgfpathlineto{\pgfqpoint{4.770276in}{1.853910in}}%
\pgfpathlineto{\pgfqpoint{4.762254in}{1.832983in}}%
\pgfpathclose%
\pgfusepath{fill}%
\end{pgfscope}%
\begin{pgfscope}%
\pgfpathrectangle{\pgfqpoint{1.254980in}{0.150000in}}{\pgfqpoint{5.490039in}{5.490039in}}%
\pgfusepath{clip}%
\pgfsetbuttcap%
\pgfsetroundjoin%
\definecolor{currentfill}{rgb}{0.204903,0.375746,0.553533}%
\pgfsetfillcolor{currentfill}%
\pgfsetfillopacity{0.700000}%
\pgfsetlinewidth{0.000000pt}%
\definecolor{currentstroke}{rgb}{0.000000,0.000000,0.000000}%
\pgfsetstrokecolor{currentstroke}%
\pgfsetdash{}{0pt}%
\pgfpathmoveto{\pgfqpoint{4.576215in}{1.466366in}}%
\pgfpathlineto{\pgfqpoint{4.590541in}{1.474634in}}%
\pgfpathlineto{\pgfqpoint{4.604884in}{1.483056in}}%
\pgfpathlineto{\pgfqpoint{4.619242in}{1.491631in}}%
\pgfpathlineto{\pgfqpoint{4.633617in}{1.500362in}}%
\pgfpathlineto{\pgfqpoint{4.641670in}{1.520792in}}%
\pgfpathlineto{\pgfqpoint{4.649721in}{1.541300in}}%
\pgfpathlineto{\pgfqpoint{4.657771in}{1.561877in}}%
\pgfpathlineto{\pgfqpoint{4.665819in}{1.582518in}}%
\pgfpathlineto{\pgfqpoint{4.651430in}{1.573103in}}%
\pgfpathlineto{\pgfqpoint{4.637057in}{1.563843in}}%
\pgfpathlineto{\pgfqpoint{4.622701in}{1.554738in}}%
\pgfpathlineto{\pgfqpoint{4.608361in}{1.545788in}}%
\pgfpathlineto{\pgfqpoint{4.600327in}{1.525821in}}%
\pgfpathlineto{\pgfqpoint{4.592291in}{1.505923in}}%
\pgfpathlineto{\pgfqpoint{4.584254in}{1.486103in}}%
\pgfpathlineto{\pgfqpoint{4.576215in}{1.466366in}}%
\pgfpathclose%
\pgfusepath{fill}%
\end{pgfscope}%
\begin{pgfscope}%
\pgfpathrectangle{\pgfqpoint{1.254980in}{0.150000in}}{\pgfqpoint{5.490039in}{5.490039in}}%
\pgfusepath{clip}%
\pgfsetbuttcap%
\pgfsetroundjoin%
\definecolor{currentfill}{rgb}{0.119699,0.618490,0.536347}%
\pgfsetfillcolor{currentfill}%
\pgfsetfillopacity{0.700000}%
\pgfsetlinewidth{0.000000pt}%
\definecolor{currentstroke}{rgb}{0.000000,0.000000,0.000000}%
\pgfsetstrokecolor{currentstroke}%
\pgfsetdash{}{0pt}%
\pgfpathmoveto{\pgfqpoint{4.916484in}{2.135231in}}%
\pgfpathlineto{\pgfqpoint{4.931068in}{2.148988in}}%
\pgfpathlineto{\pgfqpoint{4.945673in}{2.162907in}}%
\pgfpathlineto{\pgfqpoint{4.960297in}{2.176989in}}%
\pgfpathlineto{\pgfqpoint{4.968316in}{2.198254in}}%
\pgfpathlineto{\pgfqpoint{4.976332in}{2.219431in}}%
\pgfpathlineto{\pgfqpoint{4.984343in}{2.240513in}}%
\pgfpathlineto{\pgfqpoint{4.992350in}{2.261498in}}%
\pgfpathlineto{\pgfqpoint{4.977707in}{2.246920in}}%
\pgfpathlineto{\pgfqpoint{4.963084in}{2.232507in}}%
\pgfpathlineto{\pgfqpoint{4.948481in}{2.218258in}}%
\pgfpathlineto{\pgfqpoint{4.940487in}{2.197635in}}%
\pgfpathlineto{\pgfqpoint{4.932490in}{2.176921in}}%
\pgfpathlineto{\pgfqpoint{4.924489in}{2.156118in}}%
\pgfpathlineto{\pgfqpoint{4.916484in}{2.135231in}}%
\pgfpathclose%
\pgfusepath{fill}%
\end{pgfscope}%
\begin{pgfscope}%
\pgfpathrectangle{\pgfqpoint{1.254980in}{0.150000in}}{\pgfqpoint{5.490039in}{5.490039in}}%
\pgfusepath{clip}%
\pgfsetbuttcap%
\pgfsetroundjoin%
\definecolor{currentfill}{rgb}{0.159194,0.482237,0.558073}%
\pgfsetfillcolor{currentfill}%
\pgfsetfillopacity{0.700000}%
\pgfsetlinewidth{0.000000pt}%
\definecolor{currentstroke}{rgb}{0.000000,0.000000,0.000000}%
\pgfsetstrokecolor{currentstroke}%
\pgfsetdash{}{0pt}%
\pgfpathmoveto{\pgfqpoint{4.730141in}{1.749209in}}%
\pgfpathlineto{\pgfqpoint{4.744579in}{1.760070in}}%
\pgfpathlineto{\pgfqpoint{4.759035in}{1.771089in}}%
\pgfpathlineto{\pgfqpoint{4.773509in}{1.782267in}}%
\pgfpathlineto{\pgfqpoint{4.788000in}{1.793602in}}%
\pgfpathlineto{\pgfqpoint{4.796049in}{1.815160in}}%
\pgfpathlineto{\pgfqpoint{4.804096in}{1.836716in}}%
\pgfpathlineto{\pgfqpoint{4.812141in}{1.858264in}}%
\pgfpathlineto{\pgfqpoint{4.820183in}{1.879798in}}%
\pgfpathlineto{\pgfqpoint{4.805673in}{1.867856in}}%
\pgfpathlineto{\pgfqpoint{4.791182in}{1.856072in}}%
\pgfpathlineto{\pgfqpoint{4.776709in}{1.844448in}}%
\pgfpathlineto{\pgfqpoint{4.762254in}{1.832983in}}%
\pgfpathlineto{\pgfqpoint{4.754229in}{1.812044in}}%
\pgfpathlineto{\pgfqpoint{4.746202in}{1.791098in}}%
\pgfpathlineto{\pgfqpoint{4.738173in}{1.770151in}}%
\pgfpathlineto{\pgfqpoint{4.730141in}{1.749209in}}%
\pgfpathclose%
\pgfusepath{fill}%
\end{pgfscope}%
\begin{pgfscope}%
\pgfpathrectangle{\pgfqpoint{1.254980in}{0.150000in}}{\pgfqpoint{5.490039in}{5.490039in}}%
\pgfusepath{clip}%
\pgfsetbuttcap%
\pgfsetroundjoin%
\definecolor{currentfill}{rgb}{0.218130,0.347432,0.550038}%
\pgfsetfillcolor{currentfill}%
\pgfsetfillopacity{0.700000}%
\pgfsetlinewidth{0.000000pt}%
\definecolor{currentstroke}{rgb}{0.000000,0.000000,0.000000}%
\pgfsetstrokecolor{currentstroke}%
\pgfsetdash{}{0pt}%
\pgfpathmoveto{\pgfqpoint{4.544040in}{1.388378in}}%
\pgfpathlineto{\pgfqpoint{4.558355in}{1.395937in}}%
\pgfpathlineto{\pgfqpoint{4.572684in}{1.403649in}}%
\pgfpathlineto{\pgfqpoint{4.587029in}{1.411514in}}%
\pgfpathlineto{\pgfqpoint{4.601390in}{1.419532in}}%
\pgfpathlineto{\pgfqpoint{4.609448in}{1.439592in}}%
\pgfpathlineto{\pgfqpoint{4.617506in}{1.459755in}}%
\pgfpathlineto{\pgfqpoint{4.625562in}{1.480014in}}%
\pgfpathlineto{\pgfqpoint{4.633617in}{1.500362in}}%
\pgfpathlineto{\pgfqpoint{4.619242in}{1.491631in}}%
\pgfpathlineto{\pgfqpoint{4.604884in}{1.483056in}}%
\pgfpathlineto{\pgfqpoint{4.590541in}{1.474634in}}%
\pgfpathlineto{\pgfqpoint{4.576215in}{1.466366in}}%
\pgfpathlineto{\pgfqpoint{4.568174in}{1.446718in}}%
\pgfpathlineto{\pgfqpoint{4.560131in}{1.427167in}}%
\pgfpathlineto{\pgfqpoint{4.552086in}{1.407718in}}%
\pgfpathlineto{\pgfqpoint{4.544040in}{1.388378in}}%
\pgfpathclose%
\pgfusepath{fill}%
\end{pgfscope}%
\begin{pgfscope}%
\pgfpathrectangle{\pgfqpoint{1.254980in}{0.150000in}}{\pgfqpoint{5.490039in}{5.490039in}}%
\pgfusepath{clip}%
\pgfsetbuttcap%
\pgfsetroundjoin%
\definecolor{currentfill}{rgb}{0.121831,0.589055,0.545623}%
\pgfsetfillcolor{currentfill}%
\pgfsetfillopacity{0.700000}%
\pgfsetlinewidth{0.000000pt}%
\definecolor{currentstroke}{rgb}{0.000000,0.000000,0.000000}%
\pgfsetstrokecolor{currentstroke}%
\pgfsetdash{}{0pt}%
\pgfpathmoveto{\pgfqpoint{4.884431in}{2.050942in}}%
\pgfpathlineto{\pgfqpoint{4.898997in}{2.064175in}}%
\pgfpathlineto{\pgfqpoint{4.913582in}{2.077571in}}%
\pgfpathlineto{\pgfqpoint{4.928187in}{2.091129in}}%
\pgfpathlineto{\pgfqpoint{4.936220in}{2.112704in}}%
\pgfpathlineto{\pgfqpoint{4.944249in}{2.134209in}}%
\pgfpathlineto{\pgfqpoint{4.952275in}{2.155639in}}%
\pgfpathlineto{\pgfqpoint{4.960297in}{2.176989in}}%
\pgfpathlineto{\pgfqpoint{4.945673in}{2.162907in}}%
\pgfpathlineto{\pgfqpoint{4.931068in}{2.148988in}}%
\pgfpathlineto{\pgfqpoint{4.916484in}{2.135231in}}%
\pgfpathlineto{\pgfqpoint{4.908476in}{2.114266in}}%
\pgfpathlineto{\pgfqpoint{4.900464in}{2.093226in}}%
\pgfpathlineto{\pgfqpoint{4.892449in}{2.072116in}}%
\pgfpathlineto{\pgfqpoint{4.884431in}{2.050942in}}%
\pgfpathclose%
\pgfusepath{fill}%
\end{pgfscope}%
\begin{pgfscope}%
\pgfpathrectangle{\pgfqpoint{1.254980in}{0.150000in}}{\pgfqpoint{5.490039in}{5.490039in}}%
\pgfusepath{clip}%
\pgfsetbuttcap%
\pgfsetroundjoin%
\definecolor{currentfill}{rgb}{0.169646,0.456262,0.558030}%
\pgfsetfillcolor{currentfill}%
\pgfsetfillopacity{0.700000}%
\pgfsetlinewidth{0.000000pt}%
\definecolor{currentstroke}{rgb}{0.000000,0.000000,0.000000}%
\pgfsetstrokecolor{currentstroke}%
\pgfsetdash{}{0pt}%
\pgfpathmoveto{\pgfqpoint{4.697995in}{1.665597in}}%
\pgfpathlineto{\pgfqpoint{4.712417in}{1.675827in}}%
\pgfpathlineto{\pgfqpoint{4.726855in}{1.686213in}}%
\pgfpathlineto{\pgfqpoint{4.741311in}{1.696756in}}%
\pgfpathlineto{\pgfqpoint{4.755785in}{1.707457in}}%
\pgfpathlineto{\pgfqpoint{4.763842in}{1.728968in}}%
\pgfpathlineto{\pgfqpoint{4.771896in}{1.750500in}}%
\pgfpathlineto{\pgfqpoint{4.779949in}{1.772047in}}%
\pgfpathlineto{\pgfqpoint{4.788000in}{1.793602in}}%
\pgfpathlineto{\pgfqpoint{4.773509in}{1.782267in}}%
\pgfpathlineto{\pgfqpoint{4.759035in}{1.771089in}}%
\pgfpathlineto{\pgfqpoint{4.744579in}{1.760070in}}%
\pgfpathlineto{\pgfqpoint{4.730141in}{1.749209in}}%
\pgfpathlineto{\pgfqpoint{4.722108in}{1.728276in}}%
\pgfpathlineto{\pgfqpoint{4.714072in}{1.707360in}}%
\pgfpathlineto{\pgfqpoint{4.706034in}{1.686465in}}%
\pgfpathlineto{\pgfqpoint{4.697995in}{1.665597in}}%
\pgfpathclose%
\pgfusepath{fill}%
\end{pgfscope}%
\begin{pgfscope}%
\pgfpathrectangle{\pgfqpoint{1.254980in}{0.150000in}}{\pgfqpoint{5.490039in}{5.490039in}}%
\pgfusepath{clip}%
\pgfsetbuttcap%
\pgfsetroundjoin%
\definecolor{currentfill}{rgb}{0.128729,0.563265,0.551229}%
\pgfsetfillcolor{currentfill}%
\pgfsetfillopacity{0.700000}%
\pgfsetlinewidth{0.000000pt}%
\definecolor{currentstroke}{rgb}{0.000000,0.000000,0.000000}%
\pgfsetstrokecolor{currentstroke}%
\pgfsetdash{}{0pt}%
\pgfpathmoveto{\pgfqpoint{4.852328in}{1.965691in}}%
\pgfpathlineto{\pgfqpoint{4.866876in}{1.978373in}}%
\pgfpathlineto{\pgfqpoint{4.881442in}{1.991216in}}%
\pgfpathlineto{\pgfqpoint{4.896027in}{2.004220in}}%
\pgfpathlineto{\pgfqpoint{4.904071in}{2.026029in}}%
\pgfpathlineto{\pgfqpoint{4.912113in}{2.047786in}}%
\pgfpathlineto{\pgfqpoint{4.920151in}{2.069488in}}%
\pgfpathlineto{\pgfqpoint{4.928187in}{2.091129in}}%
\pgfpathlineto{\pgfqpoint{4.913582in}{2.077571in}}%
\pgfpathlineto{\pgfqpoint{4.898997in}{2.064175in}}%
\pgfpathlineto{\pgfqpoint{4.884431in}{2.050942in}}%
\pgfpathlineto{\pgfqpoint{4.876410in}{2.029707in}}%
\pgfpathlineto{\pgfqpoint{4.868385in}{2.008417in}}%
\pgfpathlineto{\pgfqpoint{4.860358in}{1.987076in}}%
\pgfpathlineto{\pgfqpoint{4.852328in}{1.965691in}}%
\pgfpathclose%
\pgfusepath{fill}%
\end{pgfscope}%
\begin{pgfscope}%
\pgfpathrectangle{\pgfqpoint{1.254980in}{0.150000in}}{\pgfqpoint{5.490039in}{5.490039in}}%
\pgfusepath{clip}%
\pgfsetbuttcap%
\pgfsetroundjoin%
\definecolor{currentfill}{rgb}{0.229739,0.322361,0.545706}%
\pgfsetfillcolor{currentfill}%
\pgfsetfillopacity{0.700000}%
\pgfsetlinewidth{0.000000pt}%
\definecolor{currentstroke}{rgb}{0.000000,0.000000,0.000000}%
\pgfsetstrokecolor{currentstroke}%
\pgfsetdash{}{0pt}%
\pgfpathmoveto{\pgfqpoint{4.511839in}{1.312248in}}%
\pgfpathlineto{\pgfqpoint{4.526142in}{1.319072in}}%
\pgfpathlineto{\pgfqpoint{4.540459in}{1.326047in}}%
\pgfpathlineto{\pgfqpoint{4.554792in}{1.333175in}}%
\pgfpathlineto{\pgfqpoint{4.569140in}{1.340454in}}%
\pgfpathlineto{\pgfqpoint{4.577204in}{1.360036in}}%
\pgfpathlineto{\pgfqpoint{4.585267in}{1.379747in}}%
\pgfpathlineto{\pgfqpoint{4.593329in}{1.399582in}}%
\pgfpathlineto{\pgfqpoint{4.601390in}{1.419532in}}%
\pgfpathlineto{\pgfqpoint{4.587029in}{1.411514in}}%
\pgfpathlineto{\pgfqpoint{4.572684in}{1.403649in}}%
\pgfpathlineto{\pgfqpoint{4.558355in}{1.395937in}}%
\pgfpathlineto{\pgfqpoint{4.544040in}{1.388378in}}%
\pgfpathlineto{\pgfqpoint{4.535992in}{1.369155in}}%
\pgfpathlineto{\pgfqpoint{4.527943in}{1.350054in}}%
\pgfpathlineto{\pgfqpoint{4.519892in}{1.331083in}}%
\pgfpathlineto{\pgfqpoint{4.511839in}{1.312248in}}%
\pgfpathclose%
\pgfusepath{fill}%
\end{pgfscope}%
\begin{pgfscope}%
\pgfpathrectangle{\pgfqpoint{1.254980in}{0.150000in}}{\pgfqpoint{5.490039in}{5.490039in}}%
\pgfusepath{clip}%
\pgfsetbuttcap%
\pgfsetroundjoin%
\definecolor{currentfill}{rgb}{0.180629,0.429975,0.557282}%
\pgfsetfillcolor{currentfill}%
\pgfsetfillopacity{0.700000}%
\pgfsetlinewidth{0.000000pt}%
\definecolor{currentstroke}{rgb}{0.000000,0.000000,0.000000}%
\pgfsetstrokecolor{currentstroke}%
\pgfsetdash{}{0pt}%
\pgfpathmoveto{\pgfqpoint{4.665819in}{1.582518in}}%
\pgfpathlineto{\pgfqpoint{4.680225in}{1.592089in}}%
\pgfpathlineto{\pgfqpoint{4.694647in}{1.601815in}}%
\pgfpathlineto{\pgfqpoint{4.709086in}{1.611697in}}%
\pgfpathlineto{\pgfqpoint{4.723542in}{1.621735in}}%
\pgfpathlineto{\pgfqpoint{4.731605in}{1.643105in}}%
\pgfpathlineto{\pgfqpoint{4.739667in}{1.664519in}}%
\pgfpathlineto{\pgfqpoint{4.747727in}{1.685972in}}%
\pgfpathlineto{\pgfqpoint{4.755785in}{1.707457in}}%
\pgfpathlineto{\pgfqpoint{4.741311in}{1.696756in}}%
\pgfpathlineto{\pgfqpoint{4.726855in}{1.686213in}}%
\pgfpathlineto{\pgfqpoint{4.712417in}{1.675827in}}%
\pgfpathlineto{\pgfqpoint{4.697995in}{1.665597in}}%
\pgfpathlineto{\pgfqpoint{4.689954in}{1.644762in}}%
\pgfpathlineto{\pgfqpoint{4.681911in}{1.623967in}}%
\pgfpathlineto{\pgfqpoint{4.673866in}{1.603217in}}%
\pgfpathlineto{\pgfqpoint{4.665819in}{1.582518in}}%
\pgfpathclose%
\pgfusepath{fill}%
\end{pgfscope}%
\begin{pgfscope}%
\pgfpathrectangle{\pgfqpoint{1.254980in}{0.150000in}}{\pgfqpoint{5.490039in}{5.490039in}}%
\pgfusepath{clip}%
\pgfsetbuttcap%
\pgfsetroundjoin%
\definecolor{currentfill}{rgb}{0.137770,0.537492,0.554906}%
\pgfsetfillcolor{currentfill}%
\pgfsetfillopacity{0.700000}%
\pgfsetlinewidth{0.000000pt}%
\definecolor{currentstroke}{rgb}{0.000000,0.000000,0.000000}%
\pgfsetstrokecolor{currentstroke}%
\pgfsetdash{}{0pt}%
\pgfpathmoveto{\pgfqpoint{4.820183in}{1.879798in}}%
\pgfpathlineto{\pgfqpoint{4.834711in}{1.891901in}}%
\pgfpathlineto{\pgfqpoint{4.849258in}{1.904162in}}%
\pgfpathlineto{\pgfqpoint{4.863824in}{1.916584in}}%
\pgfpathlineto{\pgfqpoint{4.871878in}{1.938543in}}%
\pgfpathlineto{\pgfqpoint{4.879930in}{1.960472in}}%
\pgfpathlineto{\pgfqpoint{4.887980in}{1.982367in}}%
\pgfpathlineto{\pgfqpoint{4.896027in}{2.004220in}}%
\pgfpathlineto{\pgfqpoint{4.881442in}{1.991216in}}%
\pgfpathlineto{\pgfqpoint{4.866876in}{1.978373in}}%
\pgfpathlineto{\pgfqpoint{4.852328in}{1.965691in}}%
\pgfpathlineto{\pgfqpoint{4.844296in}{1.944265in}}%
\pgfpathlineto{\pgfqpoint{4.836261in}{1.922804in}}%
\pgfpathlineto{\pgfqpoint{4.828223in}{1.901313in}}%
\pgfpathlineto{\pgfqpoint{4.820183in}{1.879798in}}%
\pgfpathclose%
\pgfusepath{fill}%
\end{pgfscope}%
\begin{pgfscope}%
\pgfpathrectangle{\pgfqpoint{1.254980in}{0.150000in}}{\pgfqpoint{5.490039in}{5.490039in}}%
\pgfusepath{clip}%
\pgfsetbuttcap%
\pgfsetroundjoin%
\definecolor{currentfill}{rgb}{0.192357,0.403199,0.555836}%
\pgfsetfillcolor{currentfill}%
\pgfsetfillopacity{0.700000}%
\pgfsetlinewidth{0.000000pt}%
\definecolor{currentstroke}{rgb}{0.000000,0.000000,0.000000}%
\pgfsetstrokecolor{currentstroke}%
\pgfsetdash{}{0pt}%
\pgfpathmoveto{\pgfqpoint{4.633617in}{1.500362in}}%
\pgfpathlineto{\pgfqpoint{4.648007in}{1.509246in}}%
\pgfpathlineto{\pgfqpoint{4.662413in}{1.518285in}}%
\pgfpathlineto{\pgfqpoint{4.676836in}{1.527478in}}%
\pgfpathlineto{\pgfqpoint{4.691276in}{1.536827in}}%
\pgfpathlineto{\pgfqpoint{4.699344in}{1.557956in}}%
\pgfpathlineto{\pgfqpoint{4.707412in}{1.579154in}}%
\pgfpathlineto{\pgfqpoint{4.715478in}{1.600416in}}%
\pgfpathlineto{\pgfqpoint{4.723542in}{1.621735in}}%
\pgfpathlineto{\pgfqpoint{4.709086in}{1.611697in}}%
\pgfpathlineto{\pgfqpoint{4.694647in}{1.601815in}}%
\pgfpathlineto{\pgfqpoint{4.680225in}{1.592089in}}%
\pgfpathlineto{\pgfqpoint{4.665819in}{1.582518in}}%
\pgfpathlineto{\pgfqpoint{4.657771in}{1.561877in}}%
\pgfpathlineto{\pgfqpoint{4.649721in}{1.541300in}}%
\pgfpathlineto{\pgfqpoint{4.641670in}{1.520792in}}%
\pgfpathlineto{\pgfqpoint{4.633617in}{1.500362in}}%
\pgfpathclose%
\pgfusepath{fill}%
\end{pgfscope}%
\begin{pgfscope}%
\pgfpathrectangle{\pgfqpoint{1.254980in}{0.150000in}}{\pgfqpoint{5.490039in}{5.490039in}}%
\pgfusepath{clip}%
\pgfsetbuttcap%
\pgfsetroundjoin%
\definecolor{currentfill}{rgb}{0.149039,0.508051,0.557250}%
\pgfsetfillcolor{currentfill}%
\pgfsetfillopacity{0.700000}%
\pgfsetlinewidth{0.000000pt}%
\definecolor{currentstroke}{rgb}{0.000000,0.000000,0.000000}%
\pgfsetstrokecolor{currentstroke}%
\pgfsetdash{}{0pt}%
\pgfpathmoveto{\pgfqpoint{4.788000in}{1.793602in}}%
\pgfpathlineto{\pgfqpoint{4.802510in}{1.805096in}}%
\pgfpathlineto{\pgfqpoint{4.817037in}{1.816748in}}%
\pgfpathlineto{\pgfqpoint{4.831584in}{1.828559in}}%
\pgfpathlineto{\pgfqpoint{4.839647in}{1.850583in}}%
\pgfpathlineto{\pgfqpoint{4.847708in}{1.872599in}}%
\pgfpathlineto{\pgfqpoint{4.855767in}{1.894601in}}%
\pgfpathlineto{\pgfqpoint{4.863824in}{1.916584in}}%
\pgfpathlineto{\pgfqpoint{4.849258in}{1.904162in}}%
\pgfpathlineto{\pgfqpoint{4.834711in}{1.891901in}}%
\pgfpathlineto{\pgfqpoint{4.820183in}{1.879798in}}%
\pgfpathlineto{\pgfqpoint{4.812141in}{1.858264in}}%
\pgfpathlineto{\pgfqpoint{4.804096in}{1.836716in}}%
\pgfpathlineto{\pgfqpoint{4.796049in}{1.815160in}}%
\pgfpathlineto{\pgfqpoint{4.788000in}{1.793602in}}%
\pgfpathclose%
\pgfusepath{fill}%
\end{pgfscope}%
\begin{pgfscope}%
\pgfpathrectangle{\pgfqpoint{1.254980in}{0.150000in}}{\pgfqpoint{5.490039in}{5.490039in}}%
\pgfusepath{clip}%
\pgfsetbuttcap%
\pgfsetroundjoin%
\definecolor{currentfill}{rgb}{0.204903,0.375746,0.553533}%
\pgfsetfillcolor{currentfill}%
\pgfsetfillopacity{0.700000}%
\pgfsetlinewidth{0.000000pt}%
\definecolor{currentstroke}{rgb}{0.000000,0.000000,0.000000}%
\pgfsetstrokecolor{currentstroke}%
\pgfsetdash{}{0pt}%
\pgfpathmoveto{\pgfqpoint{4.601390in}{1.419532in}}%
\pgfpathlineto{\pgfqpoint{4.615765in}{1.427704in}}%
\pgfpathlineto{\pgfqpoint{4.630157in}{1.436029in}}%
\pgfpathlineto{\pgfqpoint{4.644565in}{1.444507in}}%
\pgfpathlineto{\pgfqpoint{4.658988in}{1.453139in}}%
\pgfpathlineto{\pgfqpoint{4.667062in}{1.473924in}}%
\pgfpathlineto{\pgfqpoint{4.675134in}{1.494804in}}%
\pgfpathlineto{\pgfqpoint{4.683206in}{1.515774in}}%
\pgfpathlineto{\pgfqpoint{4.691276in}{1.536827in}}%
\pgfpathlineto{\pgfqpoint{4.676836in}{1.527478in}}%
\pgfpathlineto{\pgfqpoint{4.662413in}{1.518285in}}%
\pgfpathlineto{\pgfqpoint{4.648007in}{1.509246in}}%
\pgfpathlineto{\pgfqpoint{4.633617in}{1.500362in}}%
\pgfpathlineto{\pgfqpoint{4.625562in}{1.480014in}}%
\pgfpathlineto{\pgfqpoint{4.617506in}{1.459755in}}%
\pgfpathlineto{\pgfqpoint{4.609448in}{1.439592in}}%
\pgfpathlineto{\pgfqpoint{4.601390in}{1.419532in}}%
\pgfpathclose%
\pgfusepath{fill}%
\end{pgfscope}%
\begin{pgfscope}%
\pgfpathrectangle{\pgfqpoint{1.254980in}{0.150000in}}{\pgfqpoint{5.490039in}{5.490039in}}%
\pgfusepath{clip}%
\pgfsetbuttcap%
\pgfsetroundjoin%
\definecolor{currentfill}{rgb}{0.159194,0.482237,0.558073}%
\pgfsetfillcolor{currentfill}%
\pgfsetfillopacity{0.700000}%
\pgfsetlinewidth{0.000000pt}%
\definecolor{currentstroke}{rgb}{0.000000,0.000000,0.000000}%
\pgfsetstrokecolor{currentstroke}%
\pgfsetdash{}{0pt}%
\pgfpathmoveto{\pgfqpoint{4.755785in}{1.707457in}}%
\pgfpathlineto{\pgfqpoint{4.770276in}{1.718314in}}%
\pgfpathlineto{\pgfqpoint{4.784785in}{1.729330in}}%
\pgfpathlineto{\pgfqpoint{4.799312in}{1.740502in}}%
\pgfpathlineto{\pgfqpoint{4.807383in}{1.762500in}}%
\pgfpathlineto{\pgfqpoint{4.815451in}{1.784512in}}%
\pgfpathlineto{\pgfqpoint{4.823518in}{1.806534in}}%
\pgfpathlineto{\pgfqpoint{4.831584in}{1.828559in}}%
\pgfpathlineto{\pgfqpoint{4.817037in}{1.816748in}}%
\pgfpathlineto{\pgfqpoint{4.802510in}{1.805096in}}%
\pgfpathlineto{\pgfqpoint{4.788000in}{1.793602in}}%
\pgfpathlineto{\pgfqpoint{4.779949in}{1.772047in}}%
\pgfpathlineto{\pgfqpoint{4.771896in}{1.750500in}}%
\pgfpathlineto{\pgfqpoint{4.763842in}{1.728968in}}%
\pgfpathlineto{\pgfqpoint{4.755785in}{1.707457in}}%
\pgfpathclose%
\pgfusepath{fill}%
\end{pgfscope}%
\begin{pgfscope}%
\pgfpathrectangle{\pgfqpoint{1.254980in}{0.150000in}}{\pgfqpoint{5.490039in}{5.490039in}}%
\pgfusepath{clip}%
\pgfsetbuttcap%
\pgfsetroundjoin%
\definecolor{currentfill}{rgb}{0.218130,0.347432,0.550038}%
\pgfsetfillcolor{currentfill}%
\pgfsetfillopacity{0.700000}%
\pgfsetlinewidth{0.000000pt}%
\definecolor{currentstroke}{rgb}{0.000000,0.000000,0.000000}%
\pgfsetstrokecolor{currentstroke}%
\pgfsetdash{}{0pt}%
\pgfpathmoveto{\pgfqpoint{4.569140in}{1.340454in}}%
\pgfpathlineto{\pgfqpoint{4.583502in}{1.347886in}}%
\pgfpathlineto{\pgfqpoint{4.597880in}{1.355470in}}%
\pgfpathlineto{\pgfqpoint{4.612273in}{1.363207in}}%
\pgfpathlineto{\pgfqpoint{4.626682in}{1.371096in}}%
\pgfpathlineto{\pgfqpoint{4.634760in}{1.391429in}}%
\pgfpathlineto{\pgfqpoint{4.642837in}{1.411885in}}%
\pgfpathlineto{\pgfqpoint{4.650913in}{1.432457in}}%
\pgfpathlineto{\pgfqpoint{4.658988in}{1.453139in}}%
\pgfpathlineto{\pgfqpoint{4.644565in}{1.444507in}}%
\pgfpathlineto{\pgfqpoint{4.630157in}{1.436029in}}%
\pgfpathlineto{\pgfqpoint{4.615765in}{1.427704in}}%
\pgfpathlineto{\pgfqpoint{4.601390in}{1.419532in}}%
\pgfpathlineto{\pgfqpoint{4.593329in}{1.399582in}}%
\pgfpathlineto{\pgfqpoint{4.585267in}{1.379747in}}%
\pgfpathlineto{\pgfqpoint{4.577204in}{1.360036in}}%
\pgfpathlineto{\pgfqpoint{4.569140in}{1.340454in}}%
\pgfpathclose%
\pgfusepath{fill}%
\end{pgfscope}%
\begin{pgfscope}%
\pgfpathrectangle{\pgfqpoint{1.254980in}{0.150000in}}{\pgfqpoint{5.490039in}{5.490039in}}%
\pgfusepath{clip}%
\pgfsetbuttcap%
\pgfsetroundjoin%
\definecolor{currentfill}{rgb}{0.171176,0.452530,0.557965}%
\pgfsetfillcolor{currentfill}%
\pgfsetfillopacity{0.700000}%
\pgfsetlinewidth{0.000000pt}%
\definecolor{currentstroke}{rgb}{0.000000,0.000000,0.000000}%
\pgfsetstrokecolor{currentstroke}%
\pgfsetdash{}{0pt}%
\pgfpathmoveto{\pgfqpoint{4.723542in}{1.621735in}}%
\pgfpathlineto{\pgfqpoint{4.738016in}{1.631929in}}%
\pgfpathlineto{\pgfqpoint{4.752506in}{1.642280in}}%
\pgfpathlineto{\pgfqpoint{4.767014in}{1.652786in}}%
\pgfpathlineto{\pgfqpoint{4.775091in}{1.674662in}}%
\pgfpathlineto{\pgfqpoint{4.783166in}{1.696578in}}%
\pgfpathlineto{\pgfqpoint{4.791240in}{1.718526in}}%
\pgfpathlineto{\pgfqpoint{4.799312in}{1.740502in}}%
\pgfpathlineto{\pgfqpoint{4.784785in}{1.729330in}}%
\pgfpathlineto{\pgfqpoint{4.770276in}{1.718314in}}%
\pgfpathlineto{\pgfqpoint{4.755785in}{1.707457in}}%
\pgfpathlineto{\pgfqpoint{4.747727in}{1.685972in}}%
\pgfpathlineto{\pgfqpoint{4.739667in}{1.664519in}}%
\pgfpathlineto{\pgfqpoint{4.731605in}{1.643105in}}%
\pgfpathlineto{\pgfqpoint{4.723542in}{1.621735in}}%
\pgfpathclose%
\pgfusepath{fill}%
\end{pgfscope}%
\begin{pgfscope}%
\pgfpathrectangle{\pgfqpoint{1.254980in}{0.150000in}}{\pgfqpoint{5.490039in}{5.490039in}}%
\pgfusepath{clip}%
\pgfsetbuttcap%
\pgfsetroundjoin%
\definecolor{currentfill}{rgb}{0.182256,0.426184,0.557120}%
\pgfsetfillcolor{currentfill}%
\pgfsetfillopacity{0.700000}%
\pgfsetlinewidth{0.000000pt}%
\definecolor{currentstroke}{rgb}{0.000000,0.000000,0.000000}%
\pgfsetstrokecolor{currentstroke}%
\pgfsetdash{}{0pt}%
\pgfpathmoveto{\pgfqpoint{4.691276in}{1.536827in}}%
\pgfpathlineto{\pgfqpoint{4.705732in}{1.546330in}}%
\pgfpathlineto{\pgfqpoint{4.720205in}{1.555989in}}%
\pgfpathlineto{\pgfqpoint{4.734695in}{1.565803in}}%
\pgfpathlineto{\pgfqpoint{4.742777in}{1.587458in}}%
\pgfpathlineto{\pgfqpoint{4.750857in}{1.609178in}}%
\pgfpathlineto{\pgfqpoint{4.758936in}{1.630956in}}%
\pgfpathlineto{\pgfqpoint{4.767014in}{1.652786in}}%
\pgfpathlineto{\pgfqpoint{4.752506in}{1.642280in}}%
\pgfpathlineto{\pgfqpoint{4.738016in}{1.631929in}}%
\pgfpathlineto{\pgfqpoint{4.723542in}{1.621735in}}%
\pgfpathlineto{\pgfqpoint{4.715478in}{1.600416in}}%
\pgfpathlineto{\pgfqpoint{4.707412in}{1.579154in}}%
\pgfpathlineto{\pgfqpoint{4.699344in}{1.557956in}}%
\pgfpathlineto{\pgfqpoint{4.691276in}{1.536827in}}%
\pgfpathclose%
\pgfusepath{fill}%
\end{pgfscope}%
\begin{pgfscope}%
\pgfpathrectangle{\pgfqpoint{1.254980in}{0.150000in}}{\pgfqpoint{5.490039in}{5.490039in}}%
\pgfusepath{clip}%
\pgfsetbuttcap%
\pgfsetroundjoin%
\definecolor{currentfill}{rgb}{0.194100,0.399323,0.555565}%
\pgfsetfillcolor{currentfill}%
\pgfsetfillopacity{0.700000}%
\pgfsetlinewidth{0.000000pt}%
\definecolor{currentstroke}{rgb}{0.000000,0.000000,0.000000}%
\pgfsetstrokecolor{currentstroke}%
\pgfsetdash{}{0pt}%
\pgfpathmoveto{\pgfqpoint{4.658988in}{1.453139in}}%
\pgfpathlineto{\pgfqpoint{4.673428in}{1.461925in}}%
\pgfpathlineto{\pgfqpoint{4.687884in}{1.470865in}}%
\pgfpathlineto{\pgfqpoint{4.702357in}{1.479959in}}%
\pgfpathlineto{\pgfqpoint{4.710443in}{1.501290in}}%
\pgfpathlineto{\pgfqpoint{4.718528in}{1.522712in}}%
\pgfpathlineto{\pgfqpoint{4.726612in}{1.544219in}}%
\pgfpathlineto{\pgfqpoint{4.734695in}{1.565803in}}%
\pgfpathlineto{\pgfqpoint{4.720205in}{1.555989in}}%
\pgfpathlineto{\pgfqpoint{4.705732in}{1.546330in}}%
\pgfpathlineto{\pgfqpoint{4.691276in}{1.536827in}}%
\pgfpathlineto{\pgfqpoint{4.683206in}{1.515774in}}%
\pgfpathlineto{\pgfqpoint{4.675134in}{1.494804in}}%
\pgfpathlineto{\pgfqpoint{4.667062in}{1.473924in}}%
\pgfpathlineto{\pgfqpoint{4.658988in}{1.453139in}}%
\pgfpathclose%
\pgfusepath{fill}%
\end{pgfscope}%
\begin{pgfscope}%
\pgfpathrectangle{\pgfqpoint{1.254980in}{0.150000in}}{\pgfqpoint{5.490039in}{5.490039in}}%
\pgfusepath{clip}%
\pgfsetbuttcap%
\pgfsetroundjoin%
\definecolor{currentfill}{rgb}{0.206756,0.371758,0.553117}%
\pgfsetfillcolor{currentfill}%
\pgfsetfillopacity{0.700000}%
\pgfsetlinewidth{0.000000pt}%
\definecolor{currentstroke}{rgb}{0.000000,0.000000,0.000000}%
\pgfsetstrokecolor{currentstroke}%
\pgfsetdash{}{0pt}%
\pgfpathmoveto{\pgfqpoint{4.626682in}{1.371096in}}%
\pgfpathlineto{\pgfqpoint{4.641107in}{1.379138in}}%
\pgfpathlineto{\pgfqpoint{4.655547in}{1.387333in}}%
\pgfpathlineto{\pgfqpoint{4.670003in}{1.395681in}}%
\pgfpathlineto{\pgfqpoint{4.678093in}{1.416580in}}%
\pgfpathlineto{\pgfqpoint{4.686182in}{1.437597in}}%
\pgfpathlineto{\pgfqpoint{4.694270in}{1.458726in}}%
\pgfpathlineto{\pgfqpoint{4.702357in}{1.479959in}}%
\pgfpathlineto{\pgfqpoint{4.687884in}{1.470865in}}%
\pgfpathlineto{\pgfqpoint{4.673428in}{1.461925in}}%
\pgfpathlineto{\pgfqpoint{4.658988in}{1.453139in}}%
\pgfpathlineto{\pgfqpoint{4.650913in}{1.432457in}}%
\pgfpathlineto{\pgfqpoint{4.642837in}{1.411885in}}%
\pgfpathlineto{\pgfqpoint{4.634760in}{1.391429in}}%
\pgfpathlineto{\pgfqpoint{4.626682in}{1.371096in}}%
\pgfpathclose%
\pgfusepath{fill}%
\end{pgfscope}%
\begin{pgfscope}%
\pgfsetbuttcap%
\pgfsetmiterjoin%
\definecolor{currentfill}{rgb}{1.000000,1.000000,1.000000}%
\pgfsetfillcolor{currentfill}%
\pgfsetfillopacity{0.800000}%
\pgfsetlinewidth{1.003750pt}%
\definecolor{currentstroke}{rgb}{0.800000,0.800000,0.800000}%
\pgfsetstrokecolor{currentstroke}%
\pgfsetstrokeopacity{0.800000}%
\pgfsetdash{}{0pt}%
\pgfpathmoveto{\pgfqpoint{5.541867in}{5.121213in}}%
\pgfpathlineto{\pgfqpoint{6.647797in}{5.121213in}}%
\pgfpathquadraticcurveto{\pgfqpoint{6.675575in}{5.121213in}}{\pgfqpoint{6.675575in}{5.148991in}}%
\pgfpathlineto{\pgfqpoint{6.675575in}{5.542817in}}%
\pgfpathquadraticcurveto{\pgfqpoint{6.675575in}{5.570595in}}{\pgfqpoint{6.647797in}{5.570595in}}%
\pgfpathlineto{\pgfqpoint{5.541867in}{5.570595in}}%
\pgfpathquadraticcurveto{\pgfqpoint{5.514090in}{5.570595in}}{\pgfqpoint{5.514090in}{5.542817in}}%
\pgfpathlineto{\pgfqpoint{5.514090in}{5.148991in}}%
\pgfpathquadraticcurveto{\pgfqpoint{5.514090in}{5.121213in}}{\pgfqpoint{5.541867in}{5.121213in}}%
\pgfpathlineto{\pgfqpoint{5.541867in}{5.121213in}}%
\pgfpathclose%
\pgfusepath{stroke,fill}%
\end{pgfscope}%
\begin{pgfscope}%
\pgfsetrectcap%
\pgfsetroundjoin%
\pgfsetlinewidth{1.505625pt}%
\definecolor{currentstroke}{rgb}{1.000000,0.000000,0.000000}%
\pgfsetstrokecolor{currentstroke}%
\pgfsetdash{}{0pt}%
\pgfpathmoveto{\pgfqpoint{5.569645in}{5.458127in}}%
\pgfpathlineto{\pgfqpoint{5.708534in}{5.458127in}}%
\pgfpathlineto{\pgfqpoint{5.847423in}{5.458127in}}%
\pgfusepath{stroke}%
\end{pgfscope}%
\begin{pgfscope}%
\pgfsetbuttcap%
\pgfsetroundjoin%
\definecolor{currentfill}{rgb}{1.000000,0.000000,0.000000}%
\pgfsetfillcolor{currentfill}%
\pgfsetlinewidth{1.003750pt}%
\definecolor{currentstroke}{rgb}{1.000000,0.000000,0.000000}%
\pgfsetstrokecolor{currentstroke}%
\pgfsetdash{}{0pt}%
\pgfsys@defobject{currentmarker}{\pgfqpoint{-0.041667in}{-0.041667in}}{\pgfqpoint{0.041667in}{0.041667in}}{%
\pgfpathmoveto{\pgfqpoint{0.000000in}{-0.041667in}}%
\pgfpathcurveto{\pgfqpoint{0.011050in}{-0.041667in}}{\pgfqpoint{0.021649in}{-0.037276in}}{\pgfqpoint{0.029463in}{-0.029463in}}%
\pgfpathcurveto{\pgfqpoint{0.037276in}{-0.021649in}}{\pgfqpoint{0.041667in}{-0.011050in}}{\pgfqpoint{0.041667in}{0.000000in}}%
\pgfpathcurveto{\pgfqpoint{0.041667in}{0.011050in}}{\pgfqpoint{0.037276in}{0.021649in}}{\pgfqpoint{0.029463in}{0.029463in}}%
\pgfpathcurveto{\pgfqpoint{0.021649in}{0.037276in}}{\pgfqpoint{0.011050in}{0.041667in}}{\pgfqpoint{0.000000in}{0.041667in}}%
\pgfpathcurveto{\pgfqpoint{-0.011050in}{0.041667in}}{\pgfqpoint{-0.021649in}{0.037276in}}{\pgfqpoint{-0.029463in}{0.029463in}}%
\pgfpathcurveto{\pgfqpoint{-0.037276in}{0.021649in}}{\pgfqpoint{-0.041667in}{0.011050in}}{\pgfqpoint{-0.041667in}{0.000000in}}%
\pgfpathcurveto{\pgfqpoint{-0.041667in}{-0.011050in}}{\pgfqpoint{-0.037276in}{-0.021649in}}{\pgfqpoint{-0.029463in}{-0.029463in}}%
\pgfpathcurveto{\pgfqpoint{-0.021649in}{-0.037276in}}{\pgfqpoint{-0.011050in}{-0.041667in}}{\pgfqpoint{0.000000in}{-0.041667in}}%
\pgfpathlineto{\pgfqpoint{0.000000in}{-0.041667in}}%
\pgfpathclose%
\pgfusepath{stroke,fill}%
}%
\begin{pgfscope}%
\pgfsys@transformshift{5.708534in}{5.458127in}%
\pgfsys@useobject{currentmarker}{}%
\end{pgfscope}%
\end{pgfscope}%
\begin{pgfscope}%
\definecolor{textcolor}{rgb}{0.000000,0.000000,0.000000}%
\pgfsetstrokecolor{textcolor}%
\pgfsetfillcolor{textcolor}%
\pgftext[x=5.958534in,y=5.409516in,left,base]{\color{textcolor}\sffamily\fontsize{10.000000}{12.000000}\selectfont Iterations}%
\end{pgfscope}%
\begin{pgfscope}%
\pgfsetbuttcap%
\pgfsetroundjoin%
\definecolor{currentfill}{rgb}{0.000000,0.000000,1.000000}%
\pgfsetfillcolor{currentfill}%
\pgfsetlinewidth{1.003750pt}%
\definecolor{currentstroke}{rgb}{0.000000,0.000000,1.000000}%
\pgfsetstrokecolor{currentstroke}%
\pgfsetdash{}{0pt}%
\pgfsys@defobject{currentmarker}{\pgfqpoint{-0.069444in}{-0.069444in}}{\pgfqpoint{0.069444in}{0.069444in}}{%
\pgfpathmoveto{\pgfqpoint{0.000000in}{-0.069444in}}%
\pgfpathcurveto{\pgfqpoint{0.018417in}{-0.069444in}}{\pgfqpoint{0.036082in}{-0.062127in}}{\pgfqpoint{0.049105in}{-0.049105in}}%
\pgfpathcurveto{\pgfqpoint{0.062127in}{-0.036082in}}{\pgfqpoint{0.069444in}{-0.018417in}}{\pgfqpoint{0.069444in}{0.000000in}}%
\pgfpathcurveto{\pgfqpoint{0.069444in}{0.018417in}}{\pgfqpoint{0.062127in}{0.036082in}}{\pgfqpoint{0.049105in}{0.049105in}}%
\pgfpathcurveto{\pgfqpoint{0.036082in}{0.062127in}}{\pgfqpoint{0.018417in}{0.069444in}}{\pgfqpoint{0.000000in}{0.069444in}}%
\pgfpathcurveto{\pgfqpoint{-0.018417in}{0.069444in}}{\pgfqpoint{-0.036082in}{0.062127in}}{\pgfqpoint{-0.049105in}{0.049105in}}%
\pgfpathcurveto{\pgfqpoint{-0.062127in}{0.036082in}}{\pgfqpoint{-0.069444in}{0.018417in}}{\pgfqpoint{-0.069444in}{0.000000in}}%
\pgfpathcurveto{\pgfqpoint{-0.069444in}{-0.018417in}}{\pgfqpoint{-0.062127in}{-0.036082in}}{\pgfqpoint{-0.049105in}{-0.049105in}}%
\pgfpathcurveto{\pgfqpoint{-0.036082in}{-0.062127in}}{\pgfqpoint{-0.018417in}{-0.069444in}}{\pgfqpoint{0.000000in}{-0.069444in}}%
\pgfpathlineto{\pgfqpoint{0.000000in}{-0.069444in}}%
\pgfpathclose%
\pgfusepath{stroke,fill}%
}%
\begin{pgfscope}%
\pgfsys@transformshift{5.708534in}{5.242117in}%
\pgfsys@useobject{currentmarker}{}%
\end{pgfscope}%
\end{pgfscope}%
\begin{pgfscope}%
\definecolor{textcolor}{rgb}{0.000000,0.000000,0.000000}%
\pgfsetstrokecolor{textcolor}%
\pgfsetfillcolor{textcolor}%
\pgftext[x=5.958534in,y=5.205659in,left,base]{\color{textcolor}\sffamily\fontsize{10.000000}{12.000000}\selectfont Minimum}%
\end{pgfscope}%
\end{pgfpicture}%
\makeatother%
\endgroup%
}
        \caption{3D graf funkcie}
        \label{fig:newton_vpravo}
    \end{subfigure}

    \label{fig:newton_komplet}
\end{figure}

Na vrstevnicovom grafe je pekne vidieť, ako po prvotných dlhých skokoch metóda spomalí a už len jemne dolaďuje polohu v okolí minima. Hoci sme potrebovali viac krokov než pri Newtonovej metóde, výhodou je, že výpočet každej iterácie bol výrazne jednoduchší bez potreby invertovania matice.

\newpage
\noindent \textbf{Počiatočný bod} $x^{[0]} = [2; -2]$ \\
Z tohto bodu sa metóda potrebuje vymotať z údolia.

\begin{table}[H]
    \centering
    \begin{tabular}{cccc}
        \toprule
        \textbf{Iterácia} & \textbf{Bod } $x^{[k]} = [x;y]$ & \textbf{Hodnota } $f(x^{[k]})$ & \textbf{Norma } $\|\nabla f\|$ \\
        \midrule
        0  & $[0.000000;\; 0.000000]$   & 2.000000 & -- \\
        1  & $[0.000000;\; -0.500000]$  & 1.669031 & 1.000000 \\
        2  & $[0.250000;\; -0.553265]$  & 1.607017 & 0.511223 \\
        \dots & \dots & \dots & \dots \\
        13 & $[0.388215;\; -0.741333]$  & 1.568997 & 0.015403 \\
        14 & $[0.388461;\; -0.740832]$  & 1.568997 & 0.002233 \\
        15 & $[0.388004;\; -0.740889]$  & 1.568996 & 0.001887 \\
        \bottomrule
    \end{tabular}
    \caption{Priebeh MSG pre $x^{[0]} = [0;0]$.}
\end{table}

Na grafe vidíme výrazný cik-cak pohyb, pretože metóda musela mnohokrát korigovať smer, aby sa udržala v klesajúcom koryte funkcie. Kým Newtonova metóda vďaka znalosti zakrivenia (Hessiánu) prekonala tento úsek v podstate jedným skokom, MSG sa k minimu musela prepracovať postupne, čo vysvetľuje vyšší počet iterácií.

\begin{figure}[H]
    \centering

    \begin{subfigure}{0.48\textwidth}
        \centering
        \resizebox{\linewidth}{!}{%% Creator: Matplotlib, PGF backend
%%
%% To include the figure in your LaTeX document, write
%%   \input{<filename>.pgf}
%%
%% Make sure the required packages are loaded in your preamble
%%   \usepackage{pgf}
%%
%% Also ensure that all the required font packages are loaded; for instance,
%% the lmodern package is sometimes necessary when using math font.
%%   \usepackage{lmodern}
%%
%% Figures using additional raster images can only be included by \input if
%% they are in the same directory as the main LaTeX file. For loading figures
%% from other directories you can use the `import` package
%%   \usepackage{import}
%%
%% and then include the figures with
%%   \import{<path to file>}{<filename>.pgf}
%%
%% Matplotlib used the following preamble
%%   
%%   \usepackage{fontspec}
%%   \setmainfont{DejaVuSerif.ttf}[Path=\detokenize{/home/radimek/Documents/projekt_mat_prog/mat_prog_kernel/lib/python3.12/site-packages/matplotlib/mpl-data/fonts/ttf/}]
%%   \setsansfont{DejaVuSans.ttf}[Path=\detokenize{/home/radimek/Documents/projekt_mat_prog/mat_prog_kernel/lib/python3.12/site-packages/matplotlib/mpl-data/fonts/ttf/}]
%%   \setmonofont{DejaVuSansMono.ttf}[Path=\detokenize{/home/radimek/Documents/projekt_mat_prog/mat_prog_kernel/lib/python3.12/site-packages/matplotlib/mpl-data/fonts/ttf/}]
%%   \makeatletter\@ifpackageloaded{underscore}{}{\usepackage[strings]{underscore}}\makeatother
%%
\begingroup%
\makeatletter%
\begin{pgfpicture}%
\pgfpathrectangle{\pgfpointorigin}{\pgfqpoint{7.000000in}{6.000000in}}%
\pgfusepath{use as bounding box, clip}%
\begin{pgfscope}%
\pgfsetbuttcap%
\pgfsetmiterjoin%
\definecolor{currentfill}{rgb}{1.000000,1.000000,1.000000}%
\pgfsetfillcolor{currentfill}%
\pgfsetlinewidth{0.000000pt}%
\definecolor{currentstroke}{rgb}{1.000000,1.000000,1.000000}%
\pgfsetstrokecolor{currentstroke}%
\pgfsetdash{}{0pt}%
\pgfpathmoveto{\pgfqpoint{0.000000in}{0.000000in}}%
\pgfpathlineto{\pgfqpoint{7.000000in}{0.000000in}}%
\pgfpathlineto{\pgfqpoint{7.000000in}{6.000000in}}%
\pgfpathlineto{\pgfqpoint{0.000000in}{6.000000in}}%
\pgfpathlineto{\pgfqpoint{0.000000in}{0.000000in}}%
\pgfpathclose%
\pgfusepath{fill}%
\end{pgfscope}%
\begin{pgfscope}%
\pgfsetbuttcap%
\pgfsetmiterjoin%
\definecolor{currentfill}{rgb}{1.000000,1.000000,1.000000}%
\pgfsetfillcolor{currentfill}%
\pgfsetlinewidth{0.000000pt}%
\definecolor{currentstroke}{rgb}{0.000000,0.000000,0.000000}%
\pgfsetstrokecolor{currentstroke}%
\pgfsetstrokeopacity{0.000000}%
\pgfsetdash{}{0pt}%
\pgfpathmoveto{\pgfqpoint{0.766095in}{0.571603in}}%
\pgfpathlineto{\pgfqpoint{6.739560in}{0.571603in}}%
\pgfpathlineto{\pgfqpoint{6.739560in}{5.640039in}}%
\pgfpathlineto{\pgfqpoint{0.766095in}{5.640039in}}%
\pgfpathlineto{\pgfqpoint{0.766095in}{0.571603in}}%
\pgfpathclose%
\pgfusepath{fill}%
\end{pgfscope}%
\begin{pgfscope}%
\pgfpathrectangle{\pgfqpoint{0.766095in}{0.571603in}}{\pgfqpoint{5.973465in}{5.068436in}}%
\pgfusepath{clip}%
\pgfsetbuttcap%
\pgfsetroundjoin%
\definecolor{currentfill}{rgb}{0.000000,0.000000,1.000000}%
\pgfsetfillcolor{currentfill}%
\pgfsetlinewidth{1.003750pt}%
\definecolor{currentstroke}{rgb}{0.000000,0.000000,1.000000}%
\pgfsetstrokecolor{currentstroke}%
\pgfsetdash{}{0pt}%
\pgfsys@defobject{currentmarker}{\pgfqpoint{-0.069444in}{-0.069444in}}{\pgfqpoint{0.069444in}{0.069444in}}{%
\pgfpathmoveto{\pgfqpoint{0.000000in}{-0.069444in}}%
\pgfpathcurveto{\pgfqpoint{0.018417in}{-0.069444in}}{\pgfqpoint{0.036082in}{-0.062127in}}{\pgfqpoint{0.049105in}{-0.049105in}}%
\pgfpathcurveto{\pgfqpoint{0.062127in}{-0.036082in}}{\pgfqpoint{0.069444in}{-0.018417in}}{\pgfqpoint{0.069444in}{0.000000in}}%
\pgfpathcurveto{\pgfqpoint{0.069444in}{0.018417in}}{\pgfqpoint{0.062127in}{0.036082in}}{\pgfqpoint{0.049105in}{0.049105in}}%
\pgfpathcurveto{\pgfqpoint{0.036082in}{0.062127in}}{\pgfqpoint{0.018417in}{0.069444in}}{\pgfqpoint{0.000000in}{0.069444in}}%
\pgfpathcurveto{\pgfqpoint{-0.018417in}{0.069444in}}{\pgfqpoint{-0.036082in}{0.062127in}}{\pgfqpoint{-0.049105in}{0.049105in}}%
\pgfpathcurveto{\pgfqpoint{-0.062127in}{0.036082in}}{\pgfqpoint{-0.069444in}{0.018417in}}{\pgfqpoint{-0.069444in}{0.000000in}}%
\pgfpathcurveto{\pgfqpoint{-0.069444in}{-0.018417in}}{\pgfqpoint{-0.062127in}{-0.036082in}}{\pgfqpoint{-0.049105in}{-0.049105in}}%
\pgfpathcurveto{\pgfqpoint{-0.036082in}{-0.062127in}}{\pgfqpoint{-0.018417in}{-0.069444in}}{\pgfqpoint{0.000000in}{-0.069444in}}%
\pgfpathlineto{\pgfqpoint{0.000000in}{-0.069444in}}%
\pgfpathclose%
\pgfusepath{stroke,fill}%
}%
\begin{pgfscope}%
\pgfsys@transformshift{3.135086in}{3.118854in}%
\pgfsys@useobject{currentmarker}{}%
\end{pgfscope}%
\end{pgfscope}%
\begin{pgfscope}%
\pgfsetbuttcap%
\pgfsetroundjoin%
\definecolor{currentfill}{rgb}{0.000000,0.000000,0.000000}%
\pgfsetfillcolor{currentfill}%
\pgfsetlinewidth{0.803000pt}%
\definecolor{currentstroke}{rgb}{0.000000,0.000000,0.000000}%
\pgfsetstrokecolor{currentstroke}%
\pgfsetdash{}{0pt}%
\pgfsys@defobject{currentmarker}{\pgfqpoint{0.000000in}{-0.048611in}}{\pgfqpoint{0.000000in}{0.000000in}}{%
\pgfpathmoveto{\pgfqpoint{0.000000in}{0.000000in}}%
\pgfpathlineto{\pgfqpoint{0.000000in}{-0.048611in}}%
\pgfusepath{stroke,fill}%
}%
\begin{pgfscope}%
\pgfsys@transformshift{0.766095in}{0.571603in}%
\pgfsys@useobject{currentmarker}{}%
\end{pgfscope}%
\end{pgfscope}%
\begin{pgfscope}%
\definecolor{textcolor}{rgb}{0.000000,0.000000,0.000000}%
\pgfsetstrokecolor{textcolor}%
\pgfsetfillcolor{textcolor}%
\pgftext[x=0.766095in,y=0.474381in,,top]{\color{textcolor}\sffamily\fontsize{10.000000}{12.000000}\selectfont \ensuremath{-}1.0}%
\end{pgfscope}%
\begin{pgfscope}%
\pgfsetbuttcap%
\pgfsetroundjoin%
\definecolor{currentfill}{rgb}{0.000000,0.000000,0.000000}%
\pgfsetfillcolor{currentfill}%
\pgfsetlinewidth{0.803000pt}%
\definecolor{currentstroke}{rgb}{0.000000,0.000000,0.000000}%
\pgfsetstrokecolor{currentstroke}%
\pgfsetdash{}{0pt}%
\pgfsys@defobject{currentmarker}{\pgfqpoint{0.000000in}{-0.048611in}}{\pgfqpoint{0.000000in}{0.000000in}}{%
\pgfpathmoveto{\pgfqpoint{0.000000in}{0.000000in}}%
\pgfpathlineto{\pgfqpoint{0.000000in}{-0.048611in}}%
\pgfusepath{stroke,fill}%
}%
\begin{pgfscope}%
\pgfsys@transformshift{1.619447in}{0.571603in}%
\pgfsys@useobject{currentmarker}{}%
\end{pgfscope}%
\end{pgfscope}%
\begin{pgfscope}%
\definecolor{textcolor}{rgb}{0.000000,0.000000,0.000000}%
\pgfsetstrokecolor{textcolor}%
\pgfsetfillcolor{textcolor}%
\pgftext[x=1.619447in,y=0.474381in,,top]{\color{textcolor}\sffamily\fontsize{10.000000}{12.000000}\selectfont \ensuremath{-}0.5}%
\end{pgfscope}%
\begin{pgfscope}%
\pgfsetbuttcap%
\pgfsetroundjoin%
\definecolor{currentfill}{rgb}{0.000000,0.000000,0.000000}%
\pgfsetfillcolor{currentfill}%
\pgfsetlinewidth{0.803000pt}%
\definecolor{currentstroke}{rgb}{0.000000,0.000000,0.000000}%
\pgfsetstrokecolor{currentstroke}%
\pgfsetdash{}{0pt}%
\pgfsys@defobject{currentmarker}{\pgfqpoint{0.000000in}{-0.048611in}}{\pgfqpoint{0.000000in}{0.000000in}}{%
\pgfpathmoveto{\pgfqpoint{0.000000in}{0.000000in}}%
\pgfpathlineto{\pgfqpoint{0.000000in}{-0.048611in}}%
\pgfusepath{stroke,fill}%
}%
\begin{pgfscope}%
\pgfsys@transformshift{2.472799in}{0.571603in}%
\pgfsys@useobject{currentmarker}{}%
\end{pgfscope}%
\end{pgfscope}%
\begin{pgfscope}%
\definecolor{textcolor}{rgb}{0.000000,0.000000,0.000000}%
\pgfsetstrokecolor{textcolor}%
\pgfsetfillcolor{textcolor}%
\pgftext[x=2.472799in,y=0.474381in,,top]{\color{textcolor}\sffamily\fontsize{10.000000}{12.000000}\selectfont 0.0}%
\end{pgfscope}%
\begin{pgfscope}%
\pgfsetbuttcap%
\pgfsetroundjoin%
\definecolor{currentfill}{rgb}{0.000000,0.000000,0.000000}%
\pgfsetfillcolor{currentfill}%
\pgfsetlinewidth{0.803000pt}%
\definecolor{currentstroke}{rgb}{0.000000,0.000000,0.000000}%
\pgfsetstrokecolor{currentstroke}%
\pgfsetdash{}{0pt}%
\pgfsys@defobject{currentmarker}{\pgfqpoint{0.000000in}{-0.048611in}}{\pgfqpoint{0.000000in}{0.000000in}}{%
\pgfpathmoveto{\pgfqpoint{0.000000in}{0.000000in}}%
\pgfpathlineto{\pgfqpoint{0.000000in}{-0.048611in}}%
\pgfusepath{stroke,fill}%
}%
\begin{pgfscope}%
\pgfsys@transformshift{3.326152in}{0.571603in}%
\pgfsys@useobject{currentmarker}{}%
\end{pgfscope}%
\end{pgfscope}%
\begin{pgfscope}%
\definecolor{textcolor}{rgb}{0.000000,0.000000,0.000000}%
\pgfsetstrokecolor{textcolor}%
\pgfsetfillcolor{textcolor}%
\pgftext[x=3.326152in,y=0.474381in,,top]{\color{textcolor}\sffamily\fontsize{10.000000}{12.000000}\selectfont 0.5}%
\end{pgfscope}%
\begin{pgfscope}%
\pgfsetbuttcap%
\pgfsetroundjoin%
\definecolor{currentfill}{rgb}{0.000000,0.000000,0.000000}%
\pgfsetfillcolor{currentfill}%
\pgfsetlinewidth{0.803000pt}%
\definecolor{currentstroke}{rgb}{0.000000,0.000000,0.000000}%
\pgfsetstrokecolor{currentstroke}%
\pgfsetdash{}{0pt}%
\pgfsys@defobject{currentmarker}{\pgfqpoint{0.000000in}{-0.048611in}}{\pgfqpoint{0.000000in}{0.000000in}}{%
\pgfpathmoveto{\pgfqpoint{0.000000in}{0.000000in}}%
\pgfpathlineto{\pgfqpoint{0.000000in}{-0.048611in}}%
\pgfusepath{stroke,fill}%
}%
\begin{pgfscope}%
\pgfsys@transformshift{4.179504in}{0.571603in}%
\pgfsys@useobject{currentmarker}{}%
\end{pgfscope}%
\end{pgfscope}%
\begin{pgfscope}%
\definecolor{textcolor}{rgb}{0.000000,0.000000,0.000000}%
\pgfsetstrokecolor{textcolor}%
\pgfsetfillcolor{textcolor}%
\pgftext[x=4.179504in,y=0.474381in,,top]{\color{textcolor}\sffamily\fontsize{10.000000}{12.000000}\selectfont 1.0}%
\end{pgfscope}%
\begin{pgfscope}%
\pgfsetbuttcap%
\pgfsetroundjoin%
\definecolor{currentfill}{rgb}{0.000000,0.000000,0.000000}%
\pgfsetfillcolor{currentfill}%
\pgfsetlinewidth{0.803000pt}%
\definecolor{currentstroke}{rgb}{0.000000,0.000000,0.000000}%
\pgfsetstrokecolor{currentstroke}%
\pgfsetdash{}{0pt}%
\pgfsys@defobject{currentmarker}{\pgfqpoint{0.000000in}{-0.048611in}}{\pgfqpoint{0.000000in}{0.000000in}}{%
\pgfpathmoveto{\pgfqpoint{0.000000in}{0.000000in}}%
\pgfpathlineto{\pgfqpoint{0.000000in}{-0.048611in}}%
\pgfusepath{stroke,fill}%
}%
\begin{pgfscope}%
\pgfsys@transformshift{5.032856in}{0.571603in}%
\pgfsys@useobject{currentmarker}{}%
\end{pgfscope}%
\end{pgfscope}%
\begin{pgfscope}%
\definecolor{textcolor}{rgb}{0.000000,0.000000,0.000000}%
\pgfsetstrokecolor{textcolor}%
\pgfsetfillcolor{textcolor}%
\pgftext[x=5.032856in,y=0.474381in,,top]{\color{textcolor}\sffamily\fontsize{10.000000}{12.000000}\selectfont 1.5}%
\end{pgfscope}%
\begin{pgfscope}%
\pgfsetbuttcap%
\pgfsetroundjoin%
\definecolor{currentfill}{rgb}{0.000000,0.000000,0.000000}%
\pgfsetfillcolor{currentfill}%
\pgfsetlinewidth{0.803000pt}%
\definecolor{currentstroke}{rgb}{0.000000,0.000000,0.000000}%
\pgfsetstrokecolor{currentstroke}%
\pgfsetdash{}{0pt}%
\pgfsys@defobject{currentmarker}{\pgfqpoint{0.000000in}{-0.048611in}}{\pgfqpoint{0.000000in}{0.000000in}}{%
\pgfpathmoveto{\pgfqpoint{0.000000in}{0.000000in}}%
\pgfpathlineto{\pgfqpoint{0.000000in}{-0.048611in}}%
\pgfusepath{stroke,fill}%
}%
\begin{pgfscope}%
\pgfsys@transformshift{5.886208in}{0.571603in}%
\pgfsys@useobject{currentmarker}{}%
\end{pgfscope}%
\end{pgfscope}%
\begin{pgfscope}%
\definecolor{textcolor}{rgb}{0.000000,0.000000,0.000000}%
\pgfsetstrokecolor{textcolor}%
\pgfsetfillcolor{textcolor}%
\pgftext[x=5.886208in,y=0.474381in,,top]{\color{textcolor}\sffamily\fontsize{10.000000}{12.000000}\selectfont 2.0}%
\end{pgfscope}%
\begin{pgfscope}%
\pgfsetbuttcap%
\pgfsetroundjoin%
\definecolor{currentfill}{rgb}{0.000000,0.000000,0.000000}%
\pgfsetfillcolor{currentfill}%
\pgfsetlinewidth{0.803000pt}%
\definecolor{currentstroke}{rgb}{0.000000,0.000000,0.000000}%
\pgfsetstrokecolor{currentstroke}%
\pgfsetdash{}{0pt}%
\pgfsys@defobject{currentmarker}{\pgfqpoint{0.000000in}{-0.048611in}}{\pgfqpoint{0.000000in}{0.000000in}}{%
\pgfpathmoveto{\pgfqpoint{0.000000in}{0.000000in}}%
\pgfpathlineto{\pgfqpoint{0.000000in}{-0.048611in}}%
\pgfusepath{stroke,fill}%
}%
\begin{pgfscope}%
\pgfsys@transformshift{6.739560in}{0.571603in}%
\pgfsys@useobject{currentmarker}{}%
\end{pgfscope}%
\end{pgfscope}%
\begin{pgfscope}%
\definecolor{textcolor}{rgb}{0.000000,0.000000,0.000000}%
\pgfsetstrokecolor{textcolor}%
\pgfsetfillcolor{textcolor}%
\pgftext[x=6.739560in,y=0.474381in,,top]{\color{textcolor}\sffamily\fontsize{10.000000}{12.000000}\selectfont 2.5}%
\end{pgfscope}%
\begin{pgfscope}%
\definecolor{textcolor}{rgb}{0.000000,0.000000,0.000000}%
\pgfsetstrokecolor{textcolor}%
\pgfsetfillcolor{textcolor}%
\pgftext[x=3.752828in,y=0.284413in,,top]{\color{textcolor}\sffamily\fontsize{10.000000}{12.000000}\selectfont x}%
\end{pgfscope}%
\begin{pgfscope}%
\pgfsetbuttcap%
\pgfsetroundjoin%
\definecolor{currentfill}{rgb}{0.000000,0.000000,0.000000}%
\pgfsetfillcolor{currentfill}%
\pgfsetlinewidth{0.803000pt}%
\definecolor{currentstroke}{rgb}{0.000000,0.000000,0.000000}%
\pgfsetstrokecolor{currentstroke}%
\pgfsetdash{}{0pt}%
\pgfsys@defobject{currentmarker}{\pgfqpoint{-0.048611in}{0.000000in}}{\pgfqpoint{-0.000000in}{0.000000in}}{%
\pgfpathmoveto{\pgfqpoint{-0.000000in}{0.000000in}}%
\pgfpathlineto{\pgfqpoint{-0.048611in}{0.000000in}}%
\pgfusepath{stroke,fill}%
}%
\begin{pgfscope}%
\pgfsys@transformshift{0.766095in}{0.571603in}%
\pgfsys@useobject{currentmarker}{}%
\end{pgfscope}%
\end{pgfscope}%
\begin{pgfscope}%
\definecolor{textcolor}{rgb}{0.000000,0.000000,0.000000}%
\pgfsetstrokecolor{textcolor}%
\pgfsetfillcolor{textcolor}%
\pgftext[x=0.339968in, y=0.518842in, left, base]{\color{textcolor}\sffamily\fontsize{10.000000}{12.000000}\selectfont \ensuremath{-}2.5}%
\end{pgfscope}%
\begin{pgfscope}%
\pgfsetbuttcap%
\pgfsetroundjoin%
\definecolor{currentfill}{rgb}{0.000000,0.000000,0.000000}%
\pgfsetfillcolor{currentfill}%
\pgfsetlinewidth{0.803000pt}%
\definecolor{currentstroke}{rgb}{0.000000,0.000000,0.000000}%
\pgfsetstrokecolor{currentstroke}%
\pgfsetdash{}{0pt}%
\pgfsys@defobject{currentmarker}{\pgfqpoint{-0.048611in}{0.000000in}}{\pgfqpoint{-0.000000in}{0.000000in}}{%
\pgfpathmoveto{\pgfqpoint{-0.000000in}{0.000000in}}%
\pgfpathlineto{\pgfqpoint{-0.048611in}{0.000000in}}%
\pgfusepath{stroke,fill}%
}%
\begin{pgfscope}%
\pgfsys@transformshift{0.766095in}{1.295666in}%
\pgfsys@useobject{currentmarker}{}%
\end{pgfscope}%
\end{pgfscope}%
\begin{pgfscope}%
\definecolor{textcolor}{rgb}{0.000000,0.000000,0.000000}%
\pgfsetstrokecolor{textcolor}%
\pgfsetfillcolor{textcolor}%
\pgftext[x=0.339968in, y=1.242904in, left, base]{\color{textcolor}\sffamily\fontsize{10.000000}{12.000000}\selectfont \ensuremath{-}2.0}%
\end{pgfscope}%
\begin{pgfscope}%
\pgfsetbuttcap%
\pgfsetroundjoin%
\definecolor{currentfill}{rgb}{0.000000,0.000000,0.000000}%
\pgfsetfillcolor{currentfill}%
\pgfsetlinewidth{0.803000pt}%
\definecolor{currentstroke}{rgb}{0.000000,0.000000,0.000000}%
\pgfsetstrokecolor{currentstroke}%
\pgfsetdash{}{0pt}%
\pgfsys@defobject{currentmarker}{\pgfqpoint{-0.048611in}{0.000000in}}{\pgfqpoint{-0.000000in}{0.000000in}}{%
\pgfpathmoveto{\pgfqpoint{-0.000000in}{0.000000in}}%
\pgfpathlineto{\pgfqpoint{-0.048611in}{0.000000in}}%
\pgfusepath{stroke,fill}%
}%
\begin{pgfscope}%
\pgfsys@transformshift{0.766095in}{2.019728in}%
\pgfsys@useobject{currentmarker}{}%
\end{pgfscope}%
\end{pgfscope}%
\begin{pgfscope}%
\definecolor{textcolor}{rgb}{0.000000,0.000000,0.000000}%
\pgfsetstrokecolor{textcolor}%
\pgfsetfillcolor{textcolor}%
\pgftext[x=0.339968in, y=1.966966in, left, base]{\color{textcolor}\sffamily\fontsize{10.000000}{12.000000}\selectfont \ensuremath{-}1.5}%
\end{pgfscope}%
\begin{pgfscope}%
\pgfsetbuttcap%
\pgfsetroundjoin%
\definecolor{currentfill}{rgb}{0.000000,0.000000,0.000000}%
\pgfsetfillcolor{currentfill}%
\pgfsetlinewidth{0.803000pt}%
\definecolor{currentstroke}{rgb}{0.000000,0.000000,0.000000}%
\pgfsetstrokecolor{currentstroke}%
\pgfsetdash{}{0pt}%
\pgfsys@defobject{currentmarker}{\pgfqpoint{-0.048611in}{0.000000in}}{\pgfqpoint{-0.000000in}{0.000000in}}{%
\pgfpathmoveto{\pgfqpoint{-0.000000in}{0.000000in}}%
\pgfpathlineto{\pgfqpoint{-0.048611in}{0.000000in}}%
\pgfusepath{stroke,fill}%
}%
\begin{pgfscope}%
\pgfsys@transformshift{0.766095in}{2.743790in}%
\pgfsys@useobject{currentmarker}{}%
\end{pgfscope}%
\end{pgfscope}%
\begin{pgfscope}%
\definecolor{textcolor}{rgb}{0.000000,0.000000,0.000000}%
\pgfsetstrokecolor{textcolor}%
\pgfsetfillcolor{textcolor}%
\pgftext[x=0.339968in, y=2.691029in, left, base]{\color{textcolor}\sffamily\fontsize{10.000000}{12.000000}\selectfont \ensuremath{-}1.0}%
\end{pgfscope}%
\begin{pgfscope}%
\pgfsetbuttcap%
\pgfsetroundjoin%
\definecolor{currentfill}{rgb}{0.000000,0.000000,0.000000}%
\pgfsetfillcolor{currentfill}%
\pgfsetlinewidth{0.803000pt}%
\definecolor{currentstroke}{rgb}{0.000000,0.000000,0.000000}%
\pgfsetstrokecolor{currentstroke}%
\pgfsetdash{}{0pt}%
\pgfsys@defobject{currentmarker}{\pgfqpoint{-0.048611in}{0.000000in}}{\pgfqpoint{-0.000000in}{0.000000in}}{%
\pgfpathmoveto{\pgfqpoint{-0.000000in}{0.000000in}}%
\pgfpathlineto{\pgfqpoint{-0.048611in}{0.000000in}}%
\pgfusepath{stroke,fill}%
}%
\begin{pgfscope}%
\pgfsys@transformshift{0.766095in}{3.467852in}%
\pgfsys@useobject{currentmarker}{}%
\end{pgfscope}%
\end{pgfscope}%
\begin{pgfscope}%
\definecolor{textcolor}{rgb}{0.000000,0.000000,0.000000}%
\pgfsetstrokecolor{textcolor}%
\pgfsetfillcolor{textcolor}%
\pgftext[x=0.339968in, y=3.415091in, left, base]{\color{textcolor}\sffamily\fontsize{10.000000}{12.000000}\selectfont \ensuremath{-}0.5}%
\end{pgfscope}%
\begin{pgfscope}%
\pgfsetbuttcap%
\pgfsetroundjoin%
\definecolor{currentfill}{rgb}{0.000000,0.000000,0.000000}%
\pgfsetfillcolor{currentfill}%
\pgfsetlinewidth{0.803000pt}%
\definecolor{currentstroke}{rgb}{0.000000,0.000000,0.000000}%
\pgfsetstrokecolor{currentstroke}%
\pgfsetdash{}{0pt}%
\pgfsys@defobject{currentmarker}{\pgfqpoint{-0.048611in}{0.000000in}}{\pgfqpoint{-0.000000in}{0.000000in}}{%
\pgfpathmoveto{\pgfqpoint{-0.000000in}{0.000000in}}%
\pgfpathlineto{\pgfqpoint{-0.048611in}{0.000000in}}%
\pgfusepath{stroke,fill}%
}%
\begin{pgfscope}%
\pgfsys@transformshift{0.766095in}{4.191915in}%
\pgfsys@useobject{currentmarker}{}%
\end{pgfscope}%
\end{pgfscope}%
\begin{pgfscope}%
\definecolor{textcolor}{rgb}{0.000000,0.000000,0.000000}%
\pgfsetstrokecolor{textcolor}%
\pgfsetfillcolor{textcolor}%
\pgftext[x=0.447993in, y=4.139153in, left, base]{\color{textcolor}\sffamily\fontsize{10.000000}{12.000000}\selectfont 0.0}%
\end{pgfscope}%
\begin{pgfscope}%
\pgfsetbuttcap%
\pgfsetroundjoin%
\definecolor{currentfill}{rgb}{0.000000,0.000000,0.000000}%
\pgfsetfillcolor{currentfill}%
\pgfsetlinewidth{0.803000pt}%
\definecolor{currentstroke}{rgb}{0.000000,0.000000,0.000000}%
\pgfsetstrokecolor{currentstroke}%
\pgfsetdash{}{0pt}%
\pgfsys@defobject{currentmarker}{\pgfqpoint{-0.048611in}{0.000000in}}{\pgfqpoint{-0.000000in}{0.000000in}}{%
\pgfpathmoveto{\pgfqpoint{-0.000000in}{0.000000in}}%
\pgfpathlineto{\pgfqpoint{-0.048611in}{0.000000in}}%
\pgfusepath{stroke,fill}%
}%
\begin{pgfscope}%
\pgfsys@transformshift{0.766095in}{4.915977in}%
\pgfsys@useobject{currentmarker}{}%
\end{pgfscope}%
\end{pgfscope}%
\begin{pgfscope}%
\definecolor{textcolor}{rgb}{0.000000,0.000000,0.000000}%
\pgfsetstrokecolor{textcolor}%
\pgfsetfillcolor{textcolor}%
\pgftext[x=0.447993in, y=4.863215in, left, base]{\color{textcolor}\sffamily\fontsize{10.000000}{12.000000}\selectfont 0.5}%
\end{pgfscope}%
\begin{pgfscope}%
\pgfsetbuttcap%
\pgfsetroundjoin%
\definecolor{currentfill}{rgb}{0.000000,0.000000,0.000000}%
\pgfsetfillcolor{currentfill}%
\pgfsetlinewidth{0.803000pt}%
\definecolor{currentstroke}{rgb}{0.000000,0.000000,0.000000}%
\pgfsetstrokecolor{currentstroke}%
\pgfsetdash{}{0pt}%
\pgfsys@defobject{currentmarker}{\pgfqpoint{-0.048611in}{0.000000in}}{\pgfqpoint{-0.000000in}{0.000000in}}{%
\pgfpathmoveto{\pgfqpoint{-0.000000in}{0.000000in}}%
\pgfpathlineto{\pgfqpoint{-0.048611in}{0.000000in}}%
\pgfusepath{stroke,fill}%
}%
\begin{pgfscope}%
\pgfsys@transformshift{0.766095in}{5.640039in}%
\pgfsys@useobject{currentmarker}{}%
\end{pgfscope}%
\end{pgfscope}%
\begin{pgfscope}%
\definecolor{textcolor}{rgb}{0.000000,0.000000,0.000000}%
\pgfsetstrokecolor{textcolor}%
\pgfsetfillcolor{textcolor}%
\pgftext[x=0.447993in, y=5.587277in, left, base]{\color{textcolor}\sffamily\fontsize{10.000000}{12.000000}\selectfont 1.0}%
\end{pgfscope}%
\begin{pgfscope}%
\definecolor{textcolor}{rgb}{0.000000,0.000000,0.000000}%
\pgfsetstrokecolor{textcolor}%
\pgfsetfillcolor{textcolor}%
\pgftext[x=0.284413in,y=3.105821in,,bottom,rotate=90.000000]{\color{textcolor}\sffamily\fontsize{10.000000}{12.000000}\selectfont y}%
\end{pgfscope}%
\begin{pgfscope}%
\pgfpathrectangle{\pgfqpoint{0.766095in}{0.571603in}}{\pgfqpoint{5.973465in}{5.068436in}}%
\pgfusepath{clip}%
\pgfsetbuttcap%
\pgfsetroundjoin%
\pgfsetlinewidth{1.505625pt}%
\definecolor{currentstroke}{rgb}{0.276022,0.044167,0.370164}%
\pgfsetstrokecolor{currentstroke}%
\pgfsetdash{}{0pt}%
\pgfpathmoveto{\pgfqpoint{6.739560in}{2.105939in}}%
\pgfpathlineto{\pgfqpoint{6.611799in}{2.176183in}}%
\pgfpathlineto{\pgfqpoint{6.019142in}{2.491021in}}%
\pgfpathlineto{\pgfqpoint{5.748986in}{2.638629in}}%
\pgfpathlineto{\pgfqpoint{5.598899in}{2.723384in}}%
\pgfpathlineto{\pgfqpoint{5.445449in}{2.812922in}}%
\pgfpathlineto{\pgfqpoint{5.318779in}{2.889330in}}%
\pgfpathlineto{\pgfqpoint{5.196488in}{2.965739in}}%
\pgfpathlineto{\pgfqpoint{5.078790in}{3.042147in}}%
\pgfpathlineto{\pgfqpoint{4.965893in}{3.118556in}}%
\pgfpathlineto{\pgfqpoint{4.857862in}{3.194965in}}%
\pgfpathlineto{\pgfqpoint{4.758411in}{3.268754in}}%
\pgfpathlineto{\pgfqpoint{4.689150in}{3.322312in}}%
\pgfpathlineto{\pgfqpoint{4.594840in}{3.398721in}}%
\pgfpathlineto{\pgfqpoint{4.518272in}{3.464230in}}%
\pgfpathlineto{\pgfqpoint{4.458237in}{3.518030in}}%
\pgfpathlineto{\pgfqpoint{4.422109in}{3.551538in}}%
\pgfpathlineto{\pgfqpoint{4.368185in}{3.603576in}}%
\pgfpathlineto{\pgfqpoint{4.308150in}{3.664528in}}%
\pgfpathlineto{\pgfqpoint{4.270617in}{3.704355in}}%
\pgfpathlineto{\pgfqpoint{4.218098in}{3.763115in}}%
\pgfpathlineto{\pgfqpoint{4.181357in}{3.806233in}}%
\pgfpathlineto{\pgfqpoint{4.128045in}{3.872891in}}%
\pgfpathlineto{\pgfqpoint{4.098028in}{3.912615in}}%
\pgfpathlineto{\pgfqpoint{4.064608in}{3.959050in}}%
\pgfpathlineto{\pgfqpoint{4.030070in}{4.009989in}}%
\pgfpathlineto{\pgfqpoint{3.997659in}{4.060928in}}%
\pgfpathlineto{\pgfqpoint{3.967296in}{4.111867in}}%
\pgfpathlineto{\pgfqpoint{3.938894in}{4.162807in}}%
\pgfpathlineto{\pgfqpoint{3.912361in}{4.213746in}}%
\pgfpathlineto{\pgfqpoint{3.887592in}{4.264685in}}%
\pgfpathlineto{\pgfqpoint{3.857889in}{4.330645in}}%
\pgfpathlineto{\pgfqpoint{3.842549in}{4.366563in}}%
\pgfpathlineto{\pgfqpoint{3.812182in}{4.442971in}}%
\pgfpathlineto{\pgfqpoint{3.774687in}{4.544849in}}%
\pgfpathlineto{\pgfqpoint{3.728680in}{4.672197in}}%
\pgfpathlineto{\pgfqpoint{3.698497in}{4.748606in}}%
\pgfpathlineto{\pgfqpoint{3.675745in}{4.799545in}}%
\pgfpathlineto{\pgfqpoint{3.647767in}{4.853377in}}%
\pgfpathlineto{\pgfqpoint{3.634298in}{4.875953in}}%
\pgfpathlineto{\pgfqpoint{3.617749in}{4.901510in}}%
\pgfpathlineto{\pgfqpoint{3.587732in}{4.941149in}}%
\pgfpathlineto{\pgfqpoint{3.578168in}{4.952362in}}%
\pgfpathlineto{\pgfqpoint{3.554241in}{4.977831in}}%
\pgfpathlineto{\pgfqpoint{3.526438in}{5.003301in}}%
\pgfpathlineto{\pgfqpoint{3.493496in}{5.028770in}}%
\pgfpathlineto{\pgfqpoint{3.453611in}{5.054240in}}%
\pgfpathlineto{\pgfqpoint{3.437645in}{5.063053in}}%
\pgfpathlineto{\pgfqpoint{3.403900in}{5.079709in}}%
\pgfpathlineto{\pgfqpoint{3.377610in}{5.090714in}}%
\pgfpathlineto{\pgfqpoint{3.337633in}{5.105179in}}%
\pgfpathlineto{\pgfqpoint{3.287558in}{5.119511in}}%
\pgfpathlineto{\pgfqpoint{3.227523in}{5.132565in}}%
\pgfpathlineto{\pgfqpoint{3.167488in}{5.141538in}}%
\pgfpathlineto{\pgfqpoint{3.107453in}{5.147428in}}%
\pgfpathlineto{\pgfqpoint{3.047418in}{5.150605in}}%
\pgfpathlineto{\pgfqpoint{2.987384in}{5.151383in}}%
\pgfpathlineto{\pgfqpoint{2.927349in}{5.150024in}}%
\pgfpathlineto{\pgfqpoint{2.867314in}{5.146749in}}%
\pgfpathlineto{\pgfqpoint{2.807279in}{5.141747in}}%
\pgfpathlineto{\pgfqpoint{2.717227in}{5.131356in}}%
\pgfpathlineto{\pgfqpoint{2.657192in}{5.122357in}}%
\pgfpathlineto{\pgfqpoint{2.560926in}{5.105179in}}%
\pgfpathlineto{\pgfqpoint{2.477088in}{5.087178in}}%
\pgfpathlineto{\pgfqpoint{2.417053in}{5.072705in}}%
\pgfpathlineto{\pgfqpoint{2.327000in}{5.048425in}}%
\pgfpathlineto{\pgfqpoint{2.236948in}{5.020961in}}%
\pgfpathlineto{\pgfqpoint{2.146896in}{4.990210in}}%
\pgfpathlineto{\pgfqpoint{2.086861in}{4.967802in}}%
\pgfpathlineto{\pgfqpoint{2.026826in}{4.943762in}}%
\pgfpathlineto{\pgfqpoint{1.966792in}{4.917978in}}%
\pgfpathlineto{\pgfqpoint{1.906757in}{4.890327in}}%
\pgfpathlineto{\pgfqpoint{1.846722in}{4.860675in}}%
\pgfpathlineto{\pgfqpoint{1.779686in}{4.825014in}}%
\pgfpathlineto{\pgfqpoint{1.726652in}{4.794582in}}%
\pgfpathlineto{\pgfqpoint{1.666617in}{4.757549in}}%
\pgfpathlineto{\pgfqpoint{1.614714in}{4.723136in}}%
\pgfpathlineto{\pgfqpoint{1.576565in}{4.696217in}}%
\pgfpathlineto{\pgfqpoint{1.511731in}{4.646728in}}%
\pgfpathlineto{\pgfqpoint{1.456496in}{4.600333in}}%
\pgfpathlineto{\pgfqpoint{1.423339in}{4.570319in}}%
\pgfpathlineto{\pgfqpoint{1.371541in}{4.519380in}}%
\pgfpathlineto{\pgfqpoint{1.336426in}{4.481507in}}%
\pgfpathlineto{\pgfqpoint{1.303309in}{4.442971in}}%
\pgfpathlineto{\pgfqpoint{1.263727in}{4.392032in}}%
\pgfpathlineto{\pgfqpoint{1.228406in}{4.341093in}}%
\pgfpathlineto{\pgfqpoint{1.197176in}{4.290154in}}%
\pgfpathlineto{\pgfqpoint{1.169856in}{4.239215in}}%
\pgfpathlineto{\pgfqpoint{1.146259in}{4.188276in}}%
\pgfpathlineto{\pgfqpoint{1.126194in}{4.137337in}}%
\pgfpathlineto{\pgfqpoint{1.109817in}{4.086398in}}%
\pgfpathlineto{\pgfqpoint{1.096287in}{4.033611in}}%
\pgfpathlineto{\pgfqpoint{1.087066in}{3.984520in}}%
\pgfpathlineto{\pgfqpoint{1.080717in}{3.933581in}}%
\pgfpathlineto{\pgfqpoint{1.077651in}{3.882642in}}%
\pgfpathlineto{\pgfqpoint{1.077887in}{3.831703in}}%
\pgfpathlineto{\pgfqpoint{1.081436in}{3.780764in}}%
\pgfpathlineto{\pgfqpoint{1.088294in}{3.729825in}}%
\pgfpathlineto{\pgfqpoint{1.098501in}{3.678886in}}%
\pgfpathlineto{\pgfqpoint{1.112250in}{3.627946in}}%
\pgfpathlineto{\pgfqpoint{1.129378in}{3.577007in}}%
\pgfpathlineto{\pgfqpoint{1.150170in}{3.526068in}}%
\pgfpathlineto{\pgfqpoint{1.174642in}{3.475129in}}%
\pgfpathlineto{\pgfqpoint{1.202880in}{3.424190in}}%
\pgfpathlineto{\pgfqpoint{1.235022in}{3.373251in}}%
\pgfpathlineto{\pgfqpoint{1.271187in}{3.322312in}}%
\pgfpathlineto{\pgfqpoint{1.311581in}{3.271373in}}%
\pgfpathlineto{\pgfqpoint{1.356365in}{3.220434in}}%
\pgfpathlineto{\pgfqpoint{1.405658in}{3.169495in}}%
\pgfpathlineto{\pgfqpoint{1.459648in}{3.118556in}}%
\pgfpathlineto{\pgfqpoint{1.518604in}{3.067617in}}%
\pgfpathlineto{\pgfqpoint{1.582758in}{3.016678in}}%
\pgfpathlineto{\pgfqpoint{1.652298in}{2.965739in}}%
\pgfpathlineto{\pgfqpoint{1.727378in}{2.914800in}}%
\pgfpathlineto{\pgfqpoint{1.816704in}{2.858967in}}%
\pgfpathlineto{\pgfqpoint{1.895733in}{2.812922in}}%
\pgfpathlineto{\pgfqpoint{1.996809in}{2.758171in}}%
\pgfpathlineto{\pgfqpoint{2.089888in}{2.711044in}}%
\pgfpathlineto{\pgfqpoint{2.206931in}{2.655868in}}%
\pgfpathlineto{\pgfqpoint{2.312572in}{2.609165in}}%
\pgfpathlineto{\pgfqpoint{2.447070in}{2.553624in}}%
\pgfpathlineto{\pgfqpoint{2.567140in}{2.506998in}}%
\pgfpathlineto{\pgfqpoint{2.717227in}{2.452449in}}%
\pgfpathlineto{\pgfqpoint{2.867314in}{2.401254in}}%
\pgfpathlineto{\pgfqpoint{3.017401in}{2.352985in}}%
\pgfpathlineto{\pgfqpoint{3.197505in}{2.298637in}}%
\pgfpathlineto{\pgfqpoint{3.377610in}{2.247527in}}%
\pgfpathlineto{\pgfqpoint{3.557714in}{2.199277in}}%
\pgfpathlineto{\pgfqpoint{3.767836in}{2.146356in}}%
\pgfpathlineto{\pgfqpoint{3.977958in}{2.096533in}}%
\pgfpathlineto{\pgfqpoint{4.218098in}{2.043146in}}%
\pgfpathlineto{\pgfqpoint{4.458237in}{1.993093in}}%
\pgfpathlineto{\pgfqpoint{4.728394in}{1.940592in}}%
\pgfpathlineto{\pgfqpoint{4.973772in}{1.896019in}}%
\pgfpathlineto{\pgfqpoint{5.278740in}{1.845080in}}%
\pgfpathlineto{\pgfqpoint{5.568881in}{1.801302in}}%
\pgfpathlineto{\pgfqpoint{5.718968in}{1.780461in}}%
\pgfpathlineto{\pgfqpoint{5.899073in}{1.757218in}}%
\pgfpathlineto{\pgfqpoint{6.079177in}{1.736147in}}%
\pgfpathlineto{\pgfqpoint{6.259282in}{1.717499in}}%
\pgfpathlineto{\pgfqpoint{6.439386in}{1.702650in}}%
\pgfpathlineto{\pgfqpoint{6.559456in}{1.694636in}}%
\pgfpathlineto{\pgfqpoint{6.649508in}{1.690350in}}%
\pgfpathlineto{\pgfqpoint{6.739560in}{1.687958in}}%
\pgfpathlineto{\pgfqpoint{6.739560in}{1.687958in}}%
\pgfusepath{stroke}%
\end{pgfscope}%
\begin{pgfscope}%
\pgfpathrectangle{\pgfqpoint{0.766095in}{0.571603in}}{\pgfqpoint{5.973465in}{5.068436in}}%
\pgfusepath{clip}%
\pgfsetbuttcap%
\pgfsetroundjoin%
\pgfsetlinewidth{1.505625pt}%
\definecolor{currentstroke}{rgb}{0.281446,0.084320,0.407414}%
\pgfsetstrokecolor{currentstroke}%
\pgfsetdash{}{0pt}%
\pgfpathmoveto{\pgfqpoint{2.677294in}{5.640039in}}%
\pgfpathlineto{\pgfqpoint{2.657192in}{5.635455in}}%
\pgfpathlineto{\pgfqpoint{2.627175in}{5.628529in}}%
\pgfpathlineto{\pgfqpoint{2.597157in}{5.621550in}}%
\pgfpathlineto{\pgfqpoint{2.567366in}{5.614570in}}%
\pgfpathlineto{\pgfqpoint{2.567140in}{5.614515in}}%
\pgfpathlineto{\pgfqpoint{2.537122in}{5.607146in}}%
\pgfpathlineto{\pgfqpoint{2.507105in}{5.599726in}}%
\pgfpathlineto{\pgfqpoint{2.477088in}{5.592256in}}%
\pgfpathlineto{\pgfqpoint{2.464569in}{5.589100in}}%
\pgfpathlineto{\pgfqpoint{2.447070in}{5.584563in}}%
\pgfpathlineto{\pgfqpoint{2.417053in}{5.576696in}}%
\pgfpathlineto{\pgfqpoint{2.387035in}{5.568782in}}%
\pgfpathlineto{\pgfqpoint{2.367687in}{5.563630in}}%
\pgfpathlineto{\pgfqpoint{2.357018in}{5.560710in}}%
\pgfpathlineto{\pgfqpoint{2.327000in}{5.552389in}}%
\pgfpathlineto{\pgfqpoint{2.296983in}{5.544023in}}%
\pgfpathlineto{\pgfqpoint{2.276131in}{5.538161in}}%
\pgfpathlineto{\pgfqpoint{2.266966in}{5.535512in}}%
\pgfpathlineto{\pgfqpoint{2.236948in}{5.526730in}}%
\pgfpathlineto{\pgfqpoint{2.206931in}{5.517903in}}%
\pgfpathlineto{\pgfqpoint{2.189370in}{5.512691in}}%
\pgfpathlineto{\pgfqpoint{2.176913in}{5.508893in}}%
\pgfpathlineto{\pgfqpoint{2.146896in}{5.499639in}}%
\pgfpathlineto{\pgfqpoint{2.116879in}{5.490343in}}%
\pgfpathlineto{\pgfqpoint{2.106916in}{5.487222in}}%
\pgfpathlineto{\pgfqpoint{2.086861in}{5.480768in}}%
\pgfpathlineto{\pgfqpoint{2.056844in}{5.471033in}}%
\pgfpathlineto{\pgfqpoint{2.028364in}{5.461752in}}%
\pgfpathlineto{\pgfqpoint{2.026826in}{5.461238in}}%
\pgfpathlineto{\pgfqpoint{1.996809in}{5.451051in}}%
\pgfpathlineto{\pgfqpoint{1.966792in}{5.440823in}}%
\pgfpathlineto{\pgfqpoint{1.953593in}{5.436283in}}%
\pgfpathlineto{\pgfqpoint{1.936774in}{5.430341in}}%
\pgfpathlineto{\pgfqpoint{1.906757in}{5.419648in}}%
\pgfpathlineto{\pgfqpoint{1.882091in}{5.410813in}}%
\pgfpathlineto{\pgfqpoint{1.876739in}{5.408845in}}%
\pgfpathlineto{\pgfqpoint{1.846722in}{5.397672in}}%
\pgfpathlineto{\pgfqpoint{1.816704in}{5.386460in}}%
\pgfpathlineto{\pgfqpoint{1.813752in}{5.385344in}}%
\pgfpathlineto{\pgfqpoint{1.786687in}{5.374834in}}%
\pgfpathlineto{\pgfqpoint{1.756670in}{5.363127in}}%
\pgfpathlineto{\pgfqpoint{1.748416in}{5.359874in}}%
\pgfpathlineto{\pgfqpoint{1.726652in}{5.351069in}}%
\pgfpathlineto{\pgfqpoint{1.696635in}{5.338851in}}%
\pgfpathlineto{\pgfqpoint{1.685815in}{5.334405in}}%
\pgfpathlineto{\pgfqpoint{1.666617in}{5.326307in}}%
\pgfpathlineto{\pgfqpoint{1.636600in}{5.313561in}}%
\pgfpathlineto{\pgfqpoint{1.625809in}{5.308935in}}%
\pgfpathlineto{\pgfqpoint{1.606583in}{5.300477in}}%
\pgfpathlineto{\pgfqpoint{1.576565in}{5.287184in}}%
\pgfpathlineto{\pgfqpoint{1.568253in}{5.283466in}}%
\pgfpathlineto{\pgfqpoint{1.546548in}{5.273500in}}%
\pgfpathlineto{\pgfqpoint{1.516530in}{5.259641in}}%
\pgfpathlineto{\pgfqpoint{1.513007in}{5.257996in}}%
\pgfpathlineto{\pgfqpoint{1.486513in}{5.245297in}}%
\pgfpathlineto{\pgfqpoint{1.460007in}{5.232527in}}%
\pgfpathlineto{\pgfqpoint{1.456496in}{5.230790in}}%
\pgfpathlineto{\pgfqpoint{1.426478in}{5.215778in}}%
\pgfpathlineto{\pgfqpoint{1.409160in}{5.207057in}}%
\pgfpathlineto{\pgfqpoint{1.396461in}{5.200494in}}%
\pgfpathlineto{\pgfqpoint{1.366443in}{5.184851in}}%
\pgfpathlineto{\pgfqpoint{1.360246in}{5.181588in}}%
\pgfpathlineto{\pgfqpoint{1.336426in}{5.168715in}}%
\pgfpathlineto{\pgfqpoint{1.313277in}{5.156118in}}%
\pgfpathlineto{\pgfqpoint{1.306408in}{5.152281in}}%
\pgfpathlineto{\pgfqpoint{1.276391in}{5.135346in}}%
\pgfpathlineto{\pgfqpoint{1.268144in}{5.130649in}}%
\pgfpathlineto{\pgfqpoint{1.246374in}{5.117919in}}%
\pgfpathlineto{\pgfqpoint{1.224757in}{5.105179in}}%
\pgfpathlineto{\pgfqpoint{1.216356in}{5.100096in}}%
\pgfpathlineto{\pgfqpoint{1.186339in}{5.081761in}}%
\pgfpathlineto{\pgfqpoint{1.183017in}{5.079709in}}%
\pgfpathlineto{\pgfqpoint{1.156321in}{5.062782in}}%
\pgfpathlineto{\pgfqpoint{1.142971in}{5.054240in}}%
\pgfpathlineto{\pgfqpoint{1.126304in}{5.043290in}}%
\pgfpathlineto{\pgfqpoint{1.104410in}{5.028770in}}%
\pgfpathlineto{\pgfqpoint{1.096287in}{5.023238in}}%
\pgfpathlineto{\pgfqpoint{1.067303in}{5.003301in}}%
\pgfpathlineto{\pgfqpoint{1.066269in}{5.002570in}}%
\pgfpathlineto{\pgfqpoint{1.036252in}{4.981110in}}%
\pgfpathlineto{\pgfqpoint{1.031714in}{4.977831in}}%
\pgfpathlineto{\pgfqpoint{1.006234in}{4.958915in}}%
\pgfpathlineto{\pgfqpoint{0.997501in}{4.952362in}}%
\pgfpathlineto{\pgfqpoint{0.976217in}{4.935947in}}%
\pgfpathlineto{\pgfqpoint{0.964607in}{4.926892in}}%
\pgfpathlineto{\pgfqpoint{0.946199in}{4.912134in}}%
\pgfpathlineto{\pgfqpoint{0.932993in}{4.901423in}}%
\pgfpathlineto{\pgfqpoint{0.916182in}{4.887401in}}%
\pgfpathlineto{\pgfqpoint{0.902621in}{4.875953in}}%
\pgfpathlineto{\pgfqpoint{0.886165in}{4.861662in}}%
\pgfpathlineto{\pgfqpoint{0.873450in}{4.850484in}}%
\pgfpathlineto{\pgfqpoint{0.856147in}{4.834826in}}%
\pgfpathlineto{\pgfqpoint{0.845441in}{4.825014in}}%
\pgfpathlineto{\pgfqpoint{0.826130in}{4.806790in}}%
\pgfpathlineto{\pgfqpoint{0.818551in}{4.799545in}}%
\pgfpathlineto{\pgfqpoint{0.796112in}{4.777442in}}%
\pgfpathlineto{\pgfqpoint{0.792740in}{4.774075in}}%
\pgfpathlineto{\pgfqpoint{0.768006in}{4.748606in}}%
\pgfpathlineto{\pgfqpoint{0.766095in}{4.746575in}}%
\pgfusepath{stroke}%
\end{pgfscope}%
\begin{pgfscope}%
\pgfpathrectangle{\pgfqpoint{0.766095in}{0.571603in}}{\pgfqpoint{5.973465in}{5.068436in}}%
\pgfusepath{clip}%
\pgfsetbuttcap%
\pgfsetroundjoin%
\pgfsetlinewidth{1.505625pt}%
\definecolor{currentstroke}{rgb}{0.281446,0.084320,0.407414}%
\pgfsetstrokecolor{currentstroke}%
\pgfsetdash{}{0pt}%
\pgfpathmoveto{\pgfqpoint{0.766095in}{3.153363in}}%
\pgfpathlineto{\pgfqpoint{0.800267in}{3.118556in}}%
\pgfpathlineto{\pgfqpoint{0.856147in}{3.065641in}}%
\pgfpathlineto{\pgfqpoint{0.916182in}{3.013213in}}%
\pgfpathlineto{\pgfqpoint{0.976217in}{2.964502in}}%
\pgfpathlineto{\pgfqpoint{1.042028in}{2.914800in}}%
\pgfpathlineto{\pgfqpoint{1.114420in}{2.863861in}}%
\pgfpathlineto{\pgfqpoint{1.191984in}{2.812922in}}%
\pgfpathlineto{\pgfqpoint{1.276391in}{2.761166in}}%
\pgfpathlineto{\pgfqpoint{1.366443in}{2.709565in}}%
\pgfpathlineto{\pgfqpoint{1.458438in}{2.660104in}}%
\pgfpathlineto{\pgfqpoint{1.559397in}{2.609165in}}%
\pgfpathlineto{\pgfqpoint{1.666742in}{2.558226in}}%
\pgfpathlineto{\pgfqpoint{1.786687in}{2.504830in}}%
\pgfpathlineto{\pgfqpoint{1.906757in}{2.454481in}}%
\pgfpathlineto{\pgfqpoint{2.030608in}{2.405409in}}%
\pgfpathlineto{\pgfqpoint{2.176913in}{2.350808in}}%
\pgfpathlineto{\pgfqpoint{2.327000in}{2.297980in}}%
\pgfpathlineto{\pgfqpoint{2.477088in}{2.247955in}}%
\pgfpathlineto{\pgfqpoint{2.657192in}{2.191255in}}%
\pgfpathlineto{\pgfqpoint{2.807279in}{2.146307in}}%
\pgfpathlineto{\pgfqpoint{2.987384in}{2.094967in}}%
\pgfpathlineto{\pgfqpoint{3.197505in}{2.038189in}}%
\pgfpathlineto{\pgfqpoint{3.377610in}{1.991776in}}%
\pgfpathlineto{\pgfqpoint{3.587732in}{1.940079in}}%
\pgfpathlineto{\pgfqpoint{3.827871in}{1.883833in}}%
\pgfpathlineto{\pgfqpoint{4.037993in}{1.836772in}}%
\pgfpathlineto{\pgfqpoint{4.278132in}{1.785257in}}%
\pgfpathlineto{\pgfqpoint{4.548289in}{1.729885in}}%
\pgfpathlineto{\pgfqpoint{4.758411in}{1.688446in}}%
\pgfpathlineto{\pgfqpoint{5.144672in}{1.615854in}}%
\pgfpathlineto{\pgfqpoint{5.568881in}{1.540993in}}%
\pgfpathlineto{\pgfqpoint{5.748986in}{1.510620in}}%
\pgfpathlineto{\pgfqpoint{6.201284in}{1.437567in}}%
\pgfpathlineto{\pgfqpoint{6.709543in}{1.361124in}}%
\pgfpathlineto{\pgfqpoint{6.739560in}{1.356899in}}%
\pgfpathlineto{\pgfqpoint{6.739560in}{1.356899in}}%
\pgfusepath{stroke}%
\end{pgfscope}%
\begin{pgfscope}%
\pgfpathrectangle{\pgfqpoint{0.766095in}{0.571603in}}{\pgfqpoint{5.973465in}{5.068436in}}%
\pgfusepath{clip}%
\pgfsetbuttcap%
\pgfsetroundjoin%
\pgfsetlinewidth{1.505625pt}%
\definecolor{currentstroke}{rgb}{0.281446,0.084320,0.407414}%
\pgfsetstrokecolor{currentstroke}%
\pgfsetdash{}{0pt}%
\pgfpathmoveto{\pgfqpoint{6.739560in}{2.579795in}}%
\pgfpathlineto{\pgfqpoint{6.579104in}{2.660104in}}%
\pgfpathlineto{\pgfqpoint{6.469404in}{2.716879in}}%
\pgfpathlineto{\pgfqpoint{6.337491in}{2.787452in}}%
\pgfpathlineto{\pgfqpoint{6.229264in}{2.847454in}}%
\pgfpathlineto{\pgfqpoint{6.109195in}{2.916552in}}%
\pgfpathlineto{\pgfqpoint{5.989125in}{2.988660in}}%
\pgfpathlineto{\pgfqpoint{5.899073in}{3.045016in}}%
\pgfpathlineto{\pgfqpoint{5.809020in}{3.103597in}}%
\pgfpathlineto{\pgfqpoint{5.712037in}{3.169495in}}%
\pgfpathlineto{\pgfqpoint{5.628916in}{3.228732in}}%
\pgfpathlineto{\pgfqpoint{5.568881in}{3.273250in}}%
\pgfpathlineto{\pgfqpoint{5.478829in}{3.343279in}}%
\pgfpathlineto{\pgfqpoint{5.418794in}{3.392442in}}%
\pgfpathlineto{\pgfqpoint{5.381353in}{3.424190in}}%
\pgfpathlineto{\pgfqpoint{5.323692in}{3.475129in}}%
\pgfpathlineto{\pgfqpoint{5.268707in}{3.526238in}}%
\pgfpathlineto{\pgfqpoint{5.208672in}{3.585384in}}%
\pgfpathlineto{\pgfqpoint{5.167702in}{3.627946in}}%
\pgfpathlineto{\pgfqpoint{5.118620in}{3.682109in}}%
\pgfpathlineto{\pgfqpoint{5.077902in}{3.729825in}}%
\pgfpathlineto{\pgfqpoint{5.028568in}{3.792262in}}%
\pgfpathlineto{\pgfqpoint{4.998550in}{3.832956in}}%
\pgfpathlineto{\pgfqpoint{4.964364in}{3.882642in}}%
\pgfpathlineto{\pgfqpoint{4.932093in}{3.933581in}}%
\pgfpathlineto{\pgfqpoint{4.902595in}{3.984520in}}%
\pgfpathlineto{\pgfqpoint{4.875848in}{4.035459in}}%
\pgfpathlineto{\pgfqpoint{4.848463in}{4.094151in}}%
\pgfpathlineto{\pgfqpoint{4.830408in}{4.137337in}}%
\pgfpathlineto{\pgfqpoint{4.811699in}{4.188276in}}%
\pgfpathlineto{\pgfqpoint{4.795581in}{4.239215in}}%
\pgfpathlineto{\pgfqpoint{4.782024in}{4.290154in}}%
\pgfpathlineto{\pgfqpoint{4.770957in}{4.341093in}}%
\pgfpathlineto{\pgfqpoint{4.762374in}{4.392032in}}%
\pgfpathlineto{\pgfqpoint{4.756159in}{4.442971in}}%
\pgfpathlineto{\pgfqpoint{4.752229in}{4.493910in}}%
\pgfpathlineto{\pgfqpoint{4.750539in}{4.544849in}}%
\pgfpathlineto{\pgfqpoint{4.750994in}{4.595788in}}%
\pgfpathlineto{\pgfqpoint{4.753494in}{4.646728in}}%
\pgfpathlineto{\pgfqpoint{4.758411in}{4.701875in}}%
\pgfpathlineto{\pgfqpoint{4.764123in}{4.748606in}}%
\pgfpathlineto{\pgfqpoint{4.771979in}{4.799545in}}%
\pgfpathlineto{\pgfqpoint{4.786585in}{4.875953in}}%
\pgfpathlineto{\pgfqpoint{4.803858in}{4.952362in}}%
\pgfpathlineto{\pgfqpoint{4.829870in}{5.054240in}}%
\pgfpathlineto{\pgfqpoint{4.876382in}{5.232527in}}%
\pgfpathlineto{\pgfqpoint{4.887704in}{5.283466in}}%
\pgfpathlineto{\pgfqpoint{4.897059in}{5.334405in}}%
\pgfpathlineto{\pgfqpoint{4.903574in}{5.385344in}}%
\pgfpathlineto{\pgfqpoint{4.905414in}{5.410813in}}%
\pgfpathlineto{\pgfqpoint{4.906081in}{5.436283in}}%
\pgfpathlineto{\pgfqpoint{4.905369in}{5.461752in}}%
\pgfpathlineto{\pgfqpoint{4.903044in}{5.487222in}}%
\pgfpathlineto{\pgfqpoint{4.898830in}{5.512691in}}%
\pgfpathlineto{\pgfqpoint{4.892407in}{5.538161in}}%
\pgfpathlineto{\pgfqpoint{4.878481in}{5.574252in}}%
\pgfpathlineto{\pgfqpoint{4.871178in}{5.589100in}}%
\pgfpathlineto{\pgfqpoint{4.848463in}{5.623195in}}%
\pgfpathlineto{\pgfqpoint{4.834430in}{5.640039in}}%
\pgfpathlineto{\pgfqpoint{4.834430in}{5.640039in}}%
\pgfusepath{stroke}%
\end{pgfscope}%
\begin{pgfscope}%
\pgfpathrectangle{\pgfqpoint{0.766095in}{0.571603in}}{\pgfqpoint{5.973465in}{5.068436in}}%
\pgfusepath{clip}%
\pgfsetbuttcap%
\pgfsetroundjoin%
\pgfsetlinewidth{1.505625pt}%
\definecolor{currentstroke}{rgb}{0.283229,0.120777,0.440584}%
\pgfsetstrokecolor{currentstroke}%
\pgfsetdash{}{0pt}%
\pgfpathmoveto{\pgfqpoint{1.675133in}{5.640039in}}%
\pgfpathlineto{\pgfqpoint{1.666617in}{5.637037in}}%
\pgfpathlineto{\pgfqpoint{1.636600in}{5.626374in}}%
\pgfpathlineto{\pgfqpoint{1.606583in}{5.615696in}}%
\pgfpathlineto{\pgfqpoint{1.603444in}{5.614570in}}%
\pgfpathlineto{\pgfqpoint{1.576565in}{5.604672in}}%
\pgfpathlineto{\pgfqpoint{1.546548in}{5.593592in}}%
\pgfpathlineto{\pgfqpoint{1.534454in}{5.589100in}}%
\pgfpathlineto{\pgfqpoint{1.516530in}{5.582270in}}%
\pgfpathlineto{\pgfqpoint{1.486513in}{5.570776in}}%
\pgfpathlineto{\pgfqpoint{1.467937in}{5.563630in}}%
\pgfpathlineto{\pgfqpoint{1.456496in}{5.559115in}}%
\pgfpathlineto{\pgfqpoint{1.426478in}{5.547193in}}%
\pgfpathlineto{\pgfqpoint{1.403816in}{5.538161in}}%
\pgfpathlineto{\pgfqpoint{1.396461in}{5.535154in}}%
\pgfpathlineto{\pgfqpoint{1.366443in}{5.522788in}}%
\pgfpathlineto{\pgfqpoint{1.342007in}{5.512691in}}%
\pgfpathlineto{\pgfqpoint{1.336426in}{5.510326in}}%
\pgfpathlineto{\pgfqpoint{1.306408in}{5.497501in}}%
\pgfpathlineto{\pgfqpoint{1.282423in}{5.487222in}}%
\pgfpathlineto{\pgfqpoint{1.276391in}{5.484570in}}%
\pgfpathlineto{\pgfqpoint{1.246374in}{5.471270in}}%
\pgfpathlineto{\pgfqpoint{1.224974in}{5.461752in}}%
\pgfpathlineto{\pgfqpoint{1.216356in}{5.457821in}}%
\pgfpathlineto{\pgfqpoint{1.186339in}{5.444028in}}%
\pgfpathlineto{\pgfqpoint{1.169568in}{5.436283in}}%
\pgfpathlineto{\pgfqpoint{1.156321in}{5.430008in}}%
\pgfpathlineto{\pgfqpoint{1.126304in}{5.415704in}}%
\pgfpathlineto{\pgfqpoint{1.116110in}{5.410813in}}%
\pgfpathlineto{\pgfqpoint{1.096287in}{5.401058in}}%
\pgfpathlineto{\pgfqpoint{1.066269in}{5.386224in}}%
\pgfpathlineto{\pgfqpoint{1.064503in}{5.385344in}}%
\pgfpathlineto{\pgfqpoint{1.036252in}{5.370893in}}%
\pgfpathlineto{\pgfqpoint{1.014801in}{5.359874in}}%
\pgfpathlineto{\pgfqpoint{1.006234in}{5.355360in}}%
\pgfpathlineto{\pgfqpoint{0.976217in}{5.339429in}}%
\pgfpathlineto{\pgfqpoint{0.966818in}{5.334405in}}%
\pgfpathlineto{\pgfqpoint{0.946199in}{5.323100in}}%
\pgfpathlineto{\pgfqpoint{0.920498in}{5.308935in}}%
\pgfpathlineto{\pgfqpoint{0.916182in}{5.306495in}}%
\pgfpathlineto{\pgfqpoint{0.886165in}{5.289381in}}%
\pgfpathlineto{\pgfqpoint{0.875861in}{5.283466in}}%
\pgfpathlineto{\pgfqpoint{0.856147in}{5.271856in}}%
\pgfpathlineto{\pgfqpoint{0.832758in}{5.257996in}}%
\pgfpathlineto{\pgfqpoint{0.826130in}{5.253967in}}%
\pgfpathlineto{\pgfqpoint{0.796112in}{5.235574in}}%
\pgfpathlineto{\pgfqpoint{0.791180in}{5.232527in}}%
\pgfpathlineto{\pgfqpoint{0.766095in}{5.216624in}}%
\pgfusepath{stroke}%
\end{pgfscope}%
\begin{pgfscope}%
\pgfpathrectangle{\pgfqpoint{0.766095in}{0.571603in}}{\pgfqpoint{5.973465in}{5.068436in}}%
\pgfusepath{clip}%
\pgfsetbuttcap%
\pgfsetroundjoin%
\pgfsetlinewidth{1.505625pt}%
\definecolor{currentstroke}{rgb}{0.283229,0.120777,0.440584}%
\pgfsetstrokecolor{currentstroke}%
\pgfsetdash{}{0pt}%
\pgfpathmoveto{\pgfqpoint{6.739560in}{2.923245in}}%
\pgfpathlineto{\pgfqpoint{6.619491in}{2.993711in}}%
\pgfpathlineto{\pgfqpoint{6.529438in}{3.048918in}}%
\pgfpathlineto{\pgfqpoint{6.439386in}{3.106429in}}%
\pgfpathlineto{\pgfqpoint{6.349334in}{3.166551in}}%
\pgfpathlineto{\pgfqpoint{6.289299in}{3.208317in}}%
\pgfpathlineto{\pgfqpoint{6.229264in}{3.251552in}}%
\pgfpathlineto{\pgfqpoint{6.168700in}{3.296843in}}%
\pgfpathlineto{\pgfqpoint{6.103482in}{3.347782in}}%
\pgfpathlineto{\pgfqpoint{6.041327in}{3.398721in}}%
\pgfpathlineto{\pgfqpoint{5.982230in}{3.449660in}}%
\pgfpathlineto{\pgfqpoint{5.926198in}{3.500599in}}%
\pgfpathlineto{\pgfqpoint{5.869055in}{3.555709in}}%
\pgfpathlineto{\pgfqpoint{5.823249in}{3.602477in}}%
\pgfpathlineto{\pgfqpoint{5.776382in}{3.653416in}}%
\pgfpathlineto{\pgfqpoint{5.732517in}{3.704355in}}%
\pgfpathlineto{\pgfqpoint{5.688951in}{3.758981in}}%
\pgfpathlineto{\pgfqpoint{5.653953in}{3.806233in}}%
\pgfpathlineto{\pgfqpoint{5.619189in}{3.857172in}}%
\pgfpathlineto{\pgfqpoint{5.587444in}{3.908111in}}%
\pgfpathlineto{\pgfqpoint{5.558697in}{3.959050in}}%
\pgfpathlineto{\pgfqpoint{5.532937in}{4.009989in}}%
\pgfpathlineto{\pgfqpoint{5.508846in}{4.064151in}}%
\pgfpathlineto{\pgfqpoint{5.490250in}{4.111867in}}%
\pgfpathlineto{\pgfqpoint{5.473310in}{4.162807in}}%
\pgfpathlineto{\pgfqpoint{5.459238in}{4.213746in}}%
\pgfpathlineto{\pgfqpoint{5.448064in}{4.264685in}}%
\pgfpathlineto{\pgfqpoint{5.439667in}{4.315624in}}%
\pgfpathlineto{\pgfqpoint{5.434081in}{4.366563in}}%
\pgfpathlineto{\pgfqpoint{5.431254in}{4.417502in}}%
\pgfpathlineto{\pgfqpoint{5.431135in}{4.468441in}}%
\pgfpathlineto{\pgfqpoint{5.433677in}{4.519380in}}%
\pgfpathlineto{\pgfqpoint{5.438833in}{4.570319in}}%
\pgfpathlineto{\pgfqpoint{5.446559in}{4.621258in}}%
\pgfpathlineto{\pgfqpoint{5.456736in}{4.672197in}}%
\pgfpathlineto{\pgfqpoint{5.469349in}{4.723136in}}%
\pgfpathlineto{\pgfqpoint{5.484322in}{4.774075in}}%
\pgfpathlineto{\pgfqpoint{5.501547in}{4.825014in}}%
\pgfpathlineto{\pgfqpoint{5.520946in}{4.875953in}}%
\pgfpathlineto{\pgfqpoint{5.542452in}{4.926892in}}%
\pgfpathlineto{\pgfqpoint{5.568881in}{4.983922in}}%
\pgfpathlineto{\pgfqpoint{5.598899in}{5.043479in}}%
\pgfpathlineto{\pgfqpoint{5.632381in}{5.105179in}}%
\pgfpathlineto{\pgfqpoint{5.676890in}{5.181588in}}%
\pgfpathlineto{\pgfqpoint{5.724220in}{5.257996in}}%
\pgfpathlineto{\pgfqpoint{5.790389in}{5.359874in}}%
\pgfpathlineto{\pgfqpoint{5.929090in}{5.570456in}}%
\pgfpathlineto{\pgfqpoint{5.971775in}{5.640039in}}%
\pgfpathlineto{\pgfqpoint{5.971775in}{5.640039in}}%
\pgfusepath{stroke}%
\end{pgfscope}%
\begin{pgfscope}%
\pgfpathrectangle{\pgfqpoint{0.766095in}{0.571603in}}{\pgfqpoint{5.973465in}{5.068436in}}%
\pgfusepath{clip}%
\pgfsetbuttcap%
\pgfsetroundjoin%
\pgfsetlinewidth{1.505625pt}%
\definecolor{currentstroke}{rgb}{0.283229,0.120777,0.440584}%
\pgfsetstrokecolor{currentstroke}%
\pgfsetdash{}{0pt}%
\pgfpathmoveto{\pgfqpoint{0.766095in}{2.760369in}}%
\pgfpathlineto{\pgfqpoint{0.856147in}{2.704284in}}%
\pgfpathlineto{\pgfqpoint{0.931502in}{2.660104in}}%
\pgfpathlineto{\pgfqpoint{1.023769in}{2.609165in}}%
\pgfpathlineto{\pgfqpoint{1.126304in}{2.556033in}}%
\pgfpathlineto{\pgfqpoint{1.226199in}{2.507287in}}%
\pgfpathlineto{\pgfqpoint{1.336834in}{2.456348in}}%
\pgfpathlineto{\pgfqpoint{1.456496in}{2.404441in}}%
\pgfpathlineto{\pgfqpoint{1.578406in}{2.354470in}}%
\pgfpathlineto{\pgfqpoint{1.726652in}{2.297278in}}%
\pgfpathlineto{\pgfqpoint{1.848593in}{2.252592in}}%
\pgfpathlineto{\pgfqpoint{2.026826in}{2.191044in}}%
\pgfpathlineto{\pgfqpoint{2.149314in}{2.150714in}}%
\pgfpathlineto{\pgfqpoint{2.327000in}{2.095157in}}%
\pgfpathlineto{\pgfqpoint{2.507105in}{2.041711in}}%
\pgfpathlineto{\pgfqpoint{2.687209in}{1.990806in}}%
\pgfpathlineto{\pgfqpoint{2.897331in}{1.934283in}}%
\pgfpathlineto{\pgfqpoint{3.077436in}{1.887951in}}%
\pgfpathlineto{\pgfqpoint{3.287558in}{1.836144in}}%
\pgfpathlineto{\pgfqpoint{3.497680in}{1.786454in}}%
\pgfpathlineto{\pgfqpoint{3.737819in}{1.732008in}}%
\pgfpathlineto{\pgfqpoint{3.977958in}{1.679746in}}%
\pgfpathlineto{\pgfqpoint{4.218098in}{1.629441in}}%
\pgfpathlineto{\pgfqpoint{4.458237in}{1.580892in}}%
\pgfpathlineto{\pgfqpoint{4.758411in}{1.522435in}}%
\pgfpathlineto{\pgfqpoint{4.968533in}{1.482833in}}%
\pgfpathlineto{\pgfqpoint{5.358759in}{1.411874in}}%
\pgfpathlineto{\pgfqpoint{5.869055in}{1.323947in}}%
\pgfpathlineto{\pgfqpoint{6.199247in}{1.269451in}}%
\pgfpathlineto{\pgfqpoint{6.499421in}{1.221475in}}%
\pgfpathlineto{\pgfqpoint{6.739560in}{1.183966in}}%
\pgfpathlineto{\pgfqpoint{6.739560in}{1.183966in}}%
\pgfusepath{stroke}%
\end{pgfscope}%
\begin{pgfscope}%
\pgfpathrectangle{\pgfqpoint{0.766095in}{0.571603in}}{\pgfqpoint{5.973465in}{5.068436in}}%
\pgfusepath{clip}%
\pgfsetbuttcap%
\pgfsetroundjoin%
\pgfsetlinewidth{1.505625pt}%
\definecolor{currentstroke}{rgb}{0.281412,0.155834,0.469201}%
\pgfsetstrokecolor{currentstroke}%
\pgfsetdash{}{0pt}%
\pgfpathmoveto{\pgfqpoint{1.066776in}{5.640039in}}%
\pgfpathlineto{\pgfqpoint{1.066269in}{5.639826in}}%
\pgfpathlineto{\pgfqpoint{1.036252in}{5.627087in}}%
\pgfpathlineto{\pgfqpoint{1.006796in}{5.614570in}}%
\pgfpathlineto{\pgfqpoint{1.006234in}{5.614325in}}%
\pgfpathlineto{\pgfqpoint{0.976217in}{5.601144in}}%
\pgfpathlineto{\pgfqpoint{0.948842in}{5.589100in}}%
\pgfpathlineto{\pgfqpoint{0.946199in}{5.587909in}}%
\pgfpathlineto{\pgfqpoint{0.916182in}{5.574269in}}%
\pgfpathlineto{\pgfqpoint{0.892836in}{5.563630in}}%
\pgfpathlineto{\pgfqpoint{0.886165in}{5.560514in}}%
\pgfpathlineto{\pgfqpoint{0.856147in}{5.546399in}}%
\pgfpathlineto{\pgfqpoint{0.838703in}{5.538161in}}%
\pgfpathlineto{\pgfqpoint{0.826130in}{5.532075in}}%
\pgfpathlineto{\pgfqpoint{0.796112in}{5.517466in}}%
\pgfpathlineto{\pgfqpoint{0.786361in}{5.512691in}}%
\pgfpathlineto{\pgfqpoint{0.766095in}{5.502520in}}%
\pgfusepath{stroke}%
\end{pgfscope}%
\begin{pgfscope}%
\pgfpathrectangle{\pgfqpoint{0.766095in}{0.571603in}}{\pgfqpoint{5.973465in}{5.068436in}}%
\pgfusepath{clip}%
\pgfsetbuttcap%
\pgfsetroundjoin%
\pgfsetlinewidth{1.505625pt}%
\definecolor{currentstroke}{rgb}{0.281412,0.155834,0.469201}%
\pgfsetstrokecolor{currentstroke}%
\pgfsetdash{}{0pt}%
\pgfpathmoveto{\pgfqpoint{6.739560in}{3.270265in}}%
\pgfpathlineto{\pgfqpoint{6.704018in}{3.296843in}}%
\pgfpathlineto{\pgfqpoint{6.619491in}{3.362975in}}%
\pgfpathlineto{\pgfqpoint{6.559456in}{3.412670in}}%
\pgfpathlineto{\pgfqpoint{6.516641in}{3.449660in}}%
\pgfpathlineto{\pgfqpoint{6.460569in}{3.500599in}}%
\pgfpathlineto{\pgfqpoint{6.407697in}{3.551538in}}%
\pgfpathlineto{\pgfqpoint{6.349334in}{3.611791in}}%
\pgfpathlineto{\pgfqpoint{6.311458in}{3.653416in}}%
\pgfpathlineto{\pgfqpoint{6.259282in}{3.715348in}}%
\pgfpathlineto{\pgfqpoint{6.227952in}{3.755294in}}%
\pgfpathlineto{\pgfqpoint{6.190921in}{3.806233in}}%
\pgfpathlineto{\pgfqpoint{6.157056in}{3.857172in}}%
\pgfpathlineto{\pgfqpoint{6.126337in}{3.908111in}}%
\pgfpathlineto{\pgfqpoint{6.098752in}{3.959050in}}%
\pgfpathlineto{\pgfqpoint{6.074292in}{4.009989in}}%
\pgfpathlineto{\pgfqpoint{6.052926in}{4.060928in}}%
\pgfpathlineto{\pgfqpoint{6.034627in}{4.111867in}}%
\pgfpathlineto{\pgfqpoint{6.019142in}{4.163971in}}%
\pgfpathlineto{\pgfqpoint{6.007239in}{4.213746in}}%
\pgfpathlineto{\pgfqpoint{5.998091in}{4.264685in}}%
\pgfpathlineto{\pgfqpoint{5.991952in}{4.315624in}}%
\pgfpathlineto{\pgfqpoint{5.988775in}{4.366563in}}%
\pgfpathlineto{\pgfqpoint{5.988529in}{4.417502in}}%
\pgfpathlineto{\pgfqpoint{5.991184in}{4.468441in}}%
\pgfpathlineto{\pgfqpoint{5.996701in}{4.519380in}}%
\pgfpathlineto{\pgfqpoint{6.005057in}{4.570319in}}%
\pgfpathlineto{\pgfqpoint{6.016230in}{4.621258in}}%
\pgfpathlineto{\pgfqpoint{6.030112in}{4.672197in}}%
\pgfpathlineto{\pgfqpoint{6.049160in}{4.729803in}}%
\pgfpathlineto{\pgfqpoint{6.065946in}{4.774075in}}%
\pgfpathlineto{\pgfqpoint{6.087778in}{4.825014in}}%
\pgfpathlineto{\pgfqpoint{6.112144in}{4.875953in}}%
\pgfpathlineto{\pgfqpoint{6.139212in}{4.927332in}}%
\pgfpathlineto{\pgfqpoint{6.169229in}{4.979593in}}%
\pgfpathlineto{\pgfqpoint{6.199673in}{5.028770in}}%
\pgfpathlineto{\pgfqpoint{6.233393in}{5.079709in}}%
\pgfpathlineto{\pgfqpoint{6.269247in}{5.130649in}}%
\pgfpathlineto{\pgfqpoint{6.319316in}{5.197353in}}%
\pgfpathlineto{\pgfqpoint{6.367563in}{5.257996in}}%
\pgfpathlineto{\pgfqpoint{6.431901in}{5.334405in}}%
\pgfpathlineto{\pgfqpoint{6.499692in}{5.410813in}}%
\pgfpathlineto{\pgfqpoint{6.570311in}{5.487222in}}%
\pgfpathlineto{\pgfqpoint{6.667841in}{5.589100in}}%
\pgfpathlineto{\pgfqpoint{6.717517in}{5.640039in}}%
\pgfpathlineto{\pgfqpoint{6.717517in}{5.640039in}}%
\pgfusepath{stroke}%
\end{pgfscope}%
\begin{pgfscope}%
\pgfpathrectangle{\pgfqpoint{0.766095in}{0.571603in}}{\pgfqpoint{5.973465in}{5.068436in}}%
\pgfusepath{clip}%
\pgfsetbuttcap%
\pgfsetroundjoin%
\pgfsetlinewidth{1.505625pt}%
\definecolor{currentstroke}{rgb}{0.281412,0.155834,0.469201}%
\pgfsetstrokecolor{currentstroke}%
\pgfsetdash{}{0pt}%
\pgfpathmoveto{\pgfqpoint{0.766095in}{2.516810in}}%
\pgfpathlineto{\pgfqpoint{0.835088in}{2.481818in}}%
\pgfpathlineto{\pgfqpoint{0.946199in}{2.428427in}}%
\pgfpathlineto{\pgfqpoint{1.066269in}{2.374191in}}%
\pgfpathlineto{\pgfqpoint{1.186339in}{2.323031in}}%
\pgfpathlineto{\pgfqpoint{1.306408in}{2.274568in}}%
\pgfpathlineto{\pgfqpoint{1.430138in}{2.227123in}}%
\pgfpathlineto{\pgfqpoint{1.576565in}{2.173953in}}%
\pgfpathlineto{\pgfqpoint{1.726652in}{2.122305in}}%
\pgfpathlineto{\pgfqpoint{1.906757in}{2.063716in}}%
\pgfpathlineto{\pgfqpoint{2.056844in}{2.017259in}}%
\pgfpathlineto{\pgfqpoint{2.236948in}{1.964127in}}%
\pgfpathlineto{\pgfqpoint{2.417053in}{1.913455in}}%
\pgfpathlineto{\pgfqpoint{2.597157in}{1.864953in}}%
\pgfpathlineto{\pgfqpoint{2.807279in}{1.810856in}}%
\pgfpathlineto{\pgfqpoint{3.017401in}{1.759066in}}%
\pgfpathlineto{\pgfqpoint{3.227523in}{1.709322in}}%
\pgfpathlineto{\pgfqpoint{3.467662in}{1.654723in}}%
\pgfpathlineto{\pgfqpoint{3.707802in}{1.602223in}}%
\pgfpathlineto{\pgfqpoint{3.947941in}{1.551603in}}%
\pgfpathlineto{\pgfqpoint{4.188080in}{1.502663in}}%
\pgfpathlineto{\pgfqpoint{4.488254in}{1.443573in}}%
\pgfpathlineto{\pgfqpoint{4.698376in}{1.403518in}}%
\pgfpathlineto{\pgfqpoint{5.065180in}{1.335689in}}%
\pgfpathlineto{\pgfqpoint{5.598899in}{1.241731in}}%
\pgfpathlineto{\pgfqpoint{5.869055in}{1.195934in}}%
\pgfpathlineto{\pgfqpoint{6.259282in}{1.131540in}}%
\pgfpathlineto{\pgfqpoint{6.739560in}{1.055255in}}%
\pgfpathlineto{\pgfqpoint{6.739560in}{1.055255in}}%
\pgfusepath{stroke}%
\end{pgfscope}%
\begin{pgfscope}%
\pgfpathrectangle{\pgfqpoint{0.766095in}{0.571603in}}{\pgfqpoint{5.973465in}{5.068436in}}%
\pgfusepath{clip}%
\pgfsetbuttcap%
\pgfsetroundjoin%
\pgfsetlinewidth{1.505625pt}%
\definecolor{currentstroke}{rgb}{0.276194,0.190074,0.493001}%
\pgfsetstrokecolor{currentstroke}%
\pgfsetdash{}{0pt}%
\pgfpathmoveto{\pgfqpoint{6.739560in}{3.707249in}}%
\pgfpathlineto{\pgfqpoint{6.721538in}{3.729825in}}%
\pgfpathlineto{\pgfqpoint{6.709543in}{3.745514in}}%
\pgfpathlineto{\pgfqpoint{6.702047in}{3.755294in}}%
\pgfpathlineto{\pgfqpoint{6.683372in}{3.780764in}}%
\pgfpathlineto{\pgfqpoint{6.679525in}{3.786255in}}%
\pgfpathlineto{\pgfqpoint{6.665489in}{3.806233in}}%
\pgfpathlineto{\pgfqpoint{6.649508in}{3.830130in}}%
\pgfpathlineto{\pgfqpoint{6.648453in}{3.831703in}}%
\pgfpathlineto{\pgfqpoint{6.632185in}{3.857172in}}%
\pgfpathlineto{\pgfqpoint{6.619491in}{3.878152in}}%
\pgfpathlineto{\pgfqpoint{6.616764in}{3.882642in}}%
\pgfpathlineto{\pgfqpoint{6.602119in}{3.908111in}}%
\pgfpathlineto{\pgfqpoint{6.589473in}{3.931454in}}%
\pgfpathlineto{\pgfqpoint{6.588316in}{3.933581in}}%
\pgfpathlineto{\pgfqpoint{6.575278in}{3.959050in}}%
\pgfpathlineto{\pgfqpoint{6.563079in}{3.984520in}}%
\pgfpathlineto{\pgfqpoint{6.559456in}{3.992627in}}%
\pgfpathlineto{\pgfqpoint{6.551659in}{4.009989in}}%
\pgfpathlineto{\pgfqpoint{6.541043in}{4.035459in}}%
\pgfpathlineto{\pgfqpoint{6.531248in}{4.060928in}}%
\pgfpathlineto{\pgfqpoint{6.529438in}{4.066057in}}%
\pgfpathlineto{\pgfqpoint{6.522220in}{4.086398in}}%
\pgfpathlineto{\pgfqpoint{6.513992in}{4.111867in}}%
\pgfpathlineto{\pgfqpoint{6.506570in}{4.137337in}}%
\pgfpathlineto{\pgfqpoint{6.499948in}{4.162807in}}%
\pgfpathlineto{\pgfqpoint{6.499421in}{4.165111in}}%
\pgfpathlineto{\pgfqpoint{6.494086in}{4.188276in}}%
\pgfpathlineto{\pgfqpoint{6.489014in}{4.213746in}}%
\pgfpathlineto{\pgfqpoint{6.484731in}{4.239215in}}%
\pgfpathlineto{\pgfqpoint{6.481232in}{4.264685in}}%
\pgfpathlineto{\pgfqpoint{6.478512in}{4.290154in}}%
\pgfpathlineto{\pgfqpoint{6.476569in}{4.315624in}}%
\pgfpathlineto{\pgfqpoint{6.475397in}{4.341093in}}%
\pgfpathlineto{\pgfqpoint{6.474992in}{4.366563in}}%
\pgfpathlineto{\pgfqpoint{6.475353in}{4.392032in}}%
\pgfpathlineto{\pgfqpoint{6.476475in}{4.417502in}}%
\pgfpathlineto{\pgfqpoint{6.478356in}{4.442971in}}%
\pgfpathlineto{\pgfqpoint{6.480993in}{4.468441in}}%
\pgfpathlineto{\pgfqpoint{6.484384in}{4.493910in}}%
\pgfpathlineto{\pgfqpoint{6.488526in}{4.519380in}}%
\pgfpathlineto{\pgfqpoint{6.493418in}{4.544849in}}%
\pgfpathlineto{\pgfqpoint{6.499059in}{4.570319in}}%
\pgfpathlineto{\pgfqpoint{6.499421in}{4.571767in}}%
\pgfpathlineto{\pgfqpoint{6.505406in}{4.595788in}}%
\pgfpathlineto{\pgfqpoint{6.512491in}{4.621258in}}%
\pgfpathlineto{\pgfqpoint{6.520315in}{4.646728in}}%
\pgfpathlineto{\pgfqpoint{6.528879in}{4.672197in}}%
\pgfpathlineto{\pgfqpoint{6.529438in}{4.673736in}}%
\pgfpathlineto{\pgfqpoint{6.538121in}{4.697667in}}%
\pgfpathlineto{\pgfqpoint{6.548092in}{4.723136in}}%
\pgfpathlineto{\pgfqpoint{6.558795in}{4.748606in}}%
\pgfpathlineto{\pgfqpoint{6.559456in}{4.750084in}}%
\pgfpathlineto{\pgfqpoint{6.570157in}{4.774075in}}%
\pgfpathlineto{\pgfqpoint{6.582242in}{4.799545in}}%
\pgfpathlineto{\pgfqpoint{6.589473in}{4.813955in}}%
\pgfpathlineto{\pgfqpoint{6.595016in}{4.825014in}}%
\pgfpathlineto{\pgfqpoint{6.608462in}{4.850484in}}%
\pgfpathlineto{\pgfqpoint{6.619491in}{4.870335in}}%
\pgfpathlineto{\pgfqpoint{6.622610in}{4.875953in}}%
\pgfpathlineto{\pgfqpoint{6.637399in}{4.901423in}}%
\pgfpathlineto{\pgfqpoint{6.649508in}{4.921338in}}%
\pgfpathlineto{\pgfqpoint{6.652884in}{4.926892in}}%
\pgfpathlineto{\pgfqpoint{6.668999in}{4.952362in}}%
\pgfpathlineto{\pgfqpoint{6.679525in}{4.968336in}}%
\pgfpathlineto{\pgfqpoint{6.685784in}{4.977831in}}%
\pgfpathlineto{\pgfqpoint{6.703207in}{5.003301in}}%
\pgfpathlineto{\pgfqpoint{6.709543in}{5.012245in}}%
\pgfpathlineto{\pgfqpoint{6.721258in}{5.028770in}}%
\pgfpathlineto{\pgfqpoint{6.739560in}{5.053685in}}%
\pgfusepath{stroke}%
\end{pgfscope}%
\begin{pgfscope}%
\pgfpathrectangle{\pgfqpoint{0.766095in}{0.571603in}}{\pgfqpoint{5.973465in}{5.068436in}}%
\pgfusepath{clip}%
\pgfsetbuttcap%
\pgfsetroundjoin%
\pgfsetlinewidth{1.505625pt}%
\definecolor{currentstroke}{rgb}{0.276194,0.190074,0.493001}%
\pgfsetstrokecolor{currentstroke}%
\pgfsetdash{}{0pt}%
\pgfpathmoveto{\pgfqpoint{0.766095in}{2.333735in}}%
\pgfpathlineto{\pgfqpoint{0.856147in}{2.294385in}}%
\pgfpathlineto{\pgfqpoint{0.955895in}{2.252592in}}%
\pgfpathlineto{\pgfqpoint{1.096287in}{2.196949in}}%
\pgfpathlineto{\pgfqpoint{1.219187in}{2.150714in}}%
\pgfpathlineto{\pgfqpoint{1.366443in}{2.098175in}}%
\pgfpathlineto{\pgfqpoint{1.516530in}{2.047351in}}%
\pgfpathlineto{\pgfqpoint{1.696635in}{1.989628in}}%
\pgfpathlineto{\pgfqpoint{1.846722in}{1.943769in}}%
\pgfpathlineto{\pgfqpoint{2.056844in}{1.882808in}}%
\pgfpathlineto{\pgfqpoint{2.206931in}{1.841151in}}%
\pgfpathlineto{\pgfqpoint{2.417053in}{1.785366in}}%
\pgfpathlineto{\pgfqpoint{2.627175in}{1.732059in}}%
\pgfpathlineto{\pgfqpoint{2.837297in}{1.680949in}}%
\pgfpathlineto{\pgfqpoint{3.047418in}{1.631787in}}%
\pgfpathlineto{\pgfqpoint{3.287558in}{1.577734in}}%
\pgfpathlineto{\pgfqpoint{3.527697in}{1.525681in}}%
\pgfpathlineto{\pgfqpoint{3.767836in}{1.475413in}}%
\pgfpathlineto{\pgfqpoint{4.037993in}{1.420751in}}%
\pgfpathlineto{\pgfqpoint{4.278132in}{1.373699in}}%
\pgfpathlineto{\pgfqpoint{4.578307in}{1.316639in}}%
\pgfpathlineto{\pgfqpoint{4.818446in}{1.272342in}}%
\pgfpathlineto{\pgfqpoint{5.174970in}{1.208341in}}%
\pgfpathlineto{\pgfqpoint{5.718968in}{1.114865in}}%
\pgfpathlineto{\pgfqpoint{6.019142in}{1.065033in}}%
\pgfpathlineto{\pgfqpoint{6.349334in}{1.011520in}}%
\pgfpathlineto{\pgfqpoint{6.649508in}{0.964028in}}%
\pgfpathlineto{\pgfqpoint{6.739560in}{0.949913in}}%
\pgfpathlineto{\pgfqpoint{6.739560in}{0.949913in}}%
\pgfusepath{stroke}%
\end{pgfscope}%
\begin{pgfscope}%
\pgfpathrectangle{\pgfqpoint{0.766095in}{0.571603in}}{\pgfqpoint{5.973465in}{5.068436in}}%
\pgfusepath{clip}%
\pgfsetbuttcap%
\pgfsetroundjoin%
\pgfsetlinewidth{1.505625pt}%
\definecolor{currentstroke}{rgb}{0.267968,0.223549,0.512008}%
\pgfsetstrokecolor{currentstroke}%
\pgfsetdash{}{0pt}%
\pgfpathmoveto{\pgfqpoint{0.766095in}{2.184952in}}%
\pgfpathlineto{\pgfqpoint{0.856147in}{2.149340in}}%
\pgfpathlineto{\pgfqpoint{1.006234in}{2.092980in}}%
\pgfpathlineto{\pgfqpoint{1.156321in}{2.039705in}}%
\pgfpathlineto{\pgfqpoint{1.306408in}{1.989152in}}%
\pgfpathlineto{\pgfqpoint{1.456496in}{1.940994in}}%
\pgfpathlineto{\pgfqpoint{1.636600in}{1.886051in}}%
\pgfpathlineto{\pgfqpoint{1.786687in}{1.842260in}}%
\pgfpathlineto{\pgfqpoint{1.996809in}{1.783856in}}%
\pgfpathlineto{\pgfqpoint{2.176913in}{1.736011in}}%
\pgfpathlineto{\pgfqpoint{2.387035in}{1.682555in}}%
\pgfpathlineto{\pgfqpoint{2.597157in}{1.631307in}}%
\pgfpathlineto{\pgfqpoint{2.837297in}{1.575127in}}%
\pgfpathlineto{\pgfqpoint{3.047418in}{1.527823in}}%
\pgfpathlineto{\pgfqpoint{3.287558in}{1.475632in}}%
\pgfpathlineto{\pgfqpoint{3.527697in}{1.425223in}}%
\pgfpathlineto{\pgfqpoint{3.797854in}{1.370398in}}%
\pgfpathlineto{\pgfqpoint{4.037993in}{1.323185in}}%
\pgfpathlineto{\pgfqpoint{4.338167in}{1.265916in}}%
\pgfpathlineto{\pgfqpoint{4.578307in}{1.221429in}}%
\pgfpathlineto{\pgfqpoint{4.933291in}{1.157402in}}%
\pgfpathlineto{\pgfqpoint{5.478829in}{1.063094in}}%
\pgfpathlineto{\pgfqpoint{5.779003in}{1.012913in}}%
\pgfpathlineto{\pgfqpoint{6.109195in}{0.958970in}}%
\pgfpathlineto{\pgfqpoint{6.379351in}{0.915823in}}%
\pgfpathlineto{\pgfqpoint{6.739560in}{0.859429in}}%
\pgfpathlineto{\pgfqpoint{6.739560in}{0.859429in}}%
\pgfusepath{stroke}%
\end{pgfscope}%
\begin{pgfscope}%
\pgfpathrectangle{\pgfqpoint{0.766095in}{0.571603in}}{\pgfqpoint{5.973465in}{5.068436in}}%
\pgfusepath{clip}%
\pgfsetbuttcap%
\pgfsetroundjoin%
\pgfsetlinewidth{1.505625pt}%
\definecolor{currentstroke}{rgb}{0.257322,0.256130,0.526563}%
\pgfsetstrokecolor{currentstroke}%
\pgfsetdash{}{0pt}%
\pgfpathmoveto{\pgfqpoint{0.766095in}{2.058498in}}%
\pgfpathlineto{\pgfqpoint{0.886165in}{2.015004in}}%
\pgfpathlineto{\pgfqpoint{1.036252in}{1.963246in}}%
\pgfpathlineto{\pgfqpoint{1.186339in}{1.914038in}}%
\pgfpathlineto{\pgfqpoint{1.336426in}{1.867071in}}%
\pgfpathlineto{\pgfqpoint{1.516530in}{1.813415in}}%
\pgfpathlineto{\pgfqpoint{1.696635in}{1.762276in}}%
\pgfpathlineto{\pgfqpoint{1.906757in}{1.705445in}}%
\pgfpathlineto{\pgfqpoint{2.086861in}{1.658822in}}%
\pgfpathlineto{\pgfqpoint{2.296983in}{1.606626in}}%
\pgfpathlineto{\pgfqpoint{2.537122in}{1.549501in}}%
\pgfpathlineto{\pgfqpoint{2.747244in}{1.501471in}}%
\pgfpathlineto{\pgfqpoint{2.987384in}{1.448540in}}%
\pgfpathlineto{\pgfqpoint{3.227523in}{1.397474in}}%
\pgfpathlineto{\pgfqpoint{3.467662in}{1.348067in}}%
\pgfpathlineto{\pgfqpoint{3.737819in}{1.294240in}}%
\pgfpathlineto{\pgfqpoint{4.007976in}{1.242081in}}%
\pgfpathlineto{\pgfqpoint{4.278132in}{1.191420in}}%
\pgfpathlineto{\pgfqpoint{4.548289in}{1.142108in}}%
\pgfpathlineto{\pgfqpoint{4.878481in}{1.083410in}}%
\pgfpathlineto{\pgfqpoint{5.118620in}{1.041864in}}%
\pgfpathlineto{\pgfqpoint{5.489070in}{0.979116in}}%
\pgfpathlineto{\pgfqpoint{6.139212in}{0.873236in}}%
\pgfpathlineto{\pgfqpoint{6.739560in}{0.779377in}}%
\pgfpathlineto{\pgfqpoint{6.739560in}{0.779377in}}%
\pgfusepath{stroke}%
\end{pgfscope}%
\begin{pgfscope}%
\pgfpathrectangle{\pgfqpoint{0.766095in}{0.571603in}}{\pgfqpoint{5.973465in}{5.068436in}}%
\pgfusepath{clip}%
\pgfsetbuttcap%
\pgfsetroundjoin%
\pgfsetlinewidth{1.505625pt}%
\definecolor{currentstroke}{rgb}{0.244972,0.287675,0.537260}%
\pgfsetstrokecolor{currentstroke}%
\pgfsetdash{}{0pt}%
\pgfpathmoveto{\pgfqpoint{0.766095in}{1.947856in}}%
\pgfpathlineto{\pgfqpoint{0.856147in}{1.917185in}}%
\pgfpathlineto{\pgfqpoint{1.006234in}{1.868007in}}%
\pgfpathlineto{\pgfqpoint{1.186339in}{1.812034in}}%
\pgfpathlineto{\pgfqpoint{1.366443in}{1.758878in}}%
\pgfpathlineto{\pgfqpoint{1.546548in}{1.708205in}}%
\pgfpathlineto{\pgfqpoint{1.726652in}{1.659719in}}%
\pgfpathlineto{\pgfqpoint{1.936774in}{1.605628in}}%
\pgfpathlineto{\pgfqpoint{2.146896in}{1.553847in}}%
\pgfpathlineto{\pgfqpoint{2.357018in}{1.504104in}}%
\pgfpathlineto{\pgfqpoint{2.597157in}{1.449469in}}%
\pgfpathlineto{\pgfqpoint{2.837297in}{1.396900in}}%
\pgfpathlineto{\pgfqpoint{3.077436in}{1.346166in}}%
\pgfpathlineto{\pgfqpoint{3.317575in}{1.297062in}}%
\pgfpathlineto{\pgfqpoint{3.587732in}{1.243541in}}%
\pgfpathlineto{\pgfqpoint{3.857889in}{1.191656in}}%
\pgfpathlineto{\pgfqpoint{4.128045in}{1.141238in}}%
\pgfpathlineto{\pgfqpoint{4.428219in}{1.086729in}}%
\pgfpathlineto{\pgfqpoint{4.698376in}{1.038959in}}%
\pgfpathlineto{\pgfqpoint{5.028568in}{0.981946in}}%
\pgfpathlineto{\pgfqpoint{5.298724in}{0.936469in}}%
\pgfpathlineto{\pgfqpoint{5.657817in}{0.877238in}}%
\pgfpathlineto{\pgfqpoint{6.319316in}{0.771976in}}%
\pgfpathlineto{\pgfqpoint{6.739560in}{0.707302in}}%
\pgfpathlineto{\pgfqpoint{6.739560in}{0.707302in}}%
\pgfusepath{stroke}%
\end{pgfscope}%
\begin{pgfscope}%
\pgfpathrectangle{\pgfqpoint{0.766095in}{0.571603in}}{\pgfqpoint{5.973465in}{5.068436in}}%
\pgfusepath{clip}%
\pgfsetbuttcap%
\pgfsetroundjoin%
\pgfsetlinewidth{1.505625pt}%
\definecolor{currentstroke}{rgb}{0.229739,0.322361,0.545706}%
\pgfsetstrokecolor{currentstroke}%
\pgfsetdash{}{0pt}%
\pgfpathmoveto{\pgfqpoint{0.766095in}{1.849267in}}%
\pgfpathlineto{\pgfqpoint{0.886165in}{1.810750in}}%
\pgfpathlineto{\pgfqpoint{1.036252in}{1.764502in}}%
\pgfpathlineto{\pgfqpoint{1.216356in}{1.711632in}}%
\pgfpathlineto{\pgfqpoint{1.396461in}{1.661221in}}%
\pgfpathlineto{\pgfqpoint{1.606583in}{1.605183in}}%
\pgfpathlineto{\pgfqpoint{1.786687in}{1.559177in}}%
\pgfpathlineto{\pgfqpoint{2.026826in}{1.500498in}}%
\pgfpathlineto{\pgfqpoint{2.236948in}{1.451282in}}%
\pgfpathlineto{\pgfqpoint{2.477088in}{1.397186in}}%
\pgfpathlineto{\pgfqpoint{2.687209in}{1.351522in}}%
\pgfpathlineto{\pgfqpoint{2.957366in}{1.294838in}}%
\pgfpathlineto{\pgfqpoint{3.197505in}{1.246151in}}%
\pgfpathlineto{\pgfqpoint{3.467662in}{1.193059in}}%
\pgfpathlineto{\pgfqpoint{3.737819in}{1.141565in}}%
\pgfpathlineto{\pgfqpoint{4.037993in}{1.085987in}}%
\pgfpathlineto{\pgfqpoint{4.308150in}{1.037352in}}%
\pgfpathlineto{\pgfqpoint{4.608324in}{0.984622in}}%
\pgfpathlineto{\pgfqpoint{4.878481in}{0.938300in}}%
\pgfpathlineto{\pgfqpoint{5.242545in}{0.877238in}}%
\pgfpathlineto{\pgfqpoint{5.899073in}{0.771144in}}%
\pgfpathlineto{\pgfqpoint{6.529438in}{0.673174in}}%
\pgfpathlineto{\pgfqpoint{6.739560in}{0.641381in}}%
\pgfpathlineto{\pgfqpoint{6.739560in}{0.641381in}}%
\pgfusepath{stroke}%
\end{pgfscope}%
\begin{pgfscope}%
\pgfpathrectangle{\pgfqpoint{0.766095in}{0.571603in}}{\pgfqpoint{5.973465in}{5.068436in}}%
\pgfusepath{clip}%
\pgfsetbuttcap%
\pgfsetroundjoin%
\pgfsetlinewidth{1.505625pt}%
\definecolor{currentstroke}{rgb}{0.216210,0.351535,0.550627}%
\pgfsetstrokecolor{currentstroke}%
\pgfsetdash{}{0pt}%
\pgfpathmoveto{\pgfqpoint{0.766095in}{1.760015in}}%
\pgfpathlineto{\pgfqpoint{0.916182in}{1.714239in}}%
\pgfpathlineto{\pgfqpoint{1.096287in}{1.661890in}}%
\pgfpathlineto{\pgfqpoint{1.276391in}{1.611957in}}%
\pgfpathlineto{\pgfqpoint{1.486513in}{1.556442in}}%
\pgfpathlineto{\pgfqpoint{1.696635in}{1.503449in}}%
\pgfpathlineto{\pgfqpoint{1.906757in}{1.452677in}}%
\pgfpathlineto{\pgfqpoint{2.146896in}{1.397045in}}%
\pgfpathlineto{\pgfqpoint{2.357018in}{1.350212in}}%
\pgfpathlineto{\pgfqpoint{2.597157in}{1.298547in}}%
\pgfpathlineto{\pgfqpoint{2.867314in}{1.242508in}}%
\pgfpathlineto{\pgfqpoint{3.107453in}{1.194355in}}%
\pgfpathlineto{\pgfqpoint{3.377610in}{1.141809in}}%
\pgfpathlineto{\pgfqpoint{3.647767in}{1.090814in}}%
\pgfpathlineto{\pgfqpoint{3.947941in}{1.035744in}}%
\pgfpathlineto{\pgfqpoint{4.218098in}{0.987522in}}%
\pgfpathlineto{\pgfqpoint{4.548289in}{0.930008in}}%
\pgfpathlineto{\pgfqpoint{4.818446in}{0.884156in}}%
\pgfpathlineto{\pgfqpoint{5.148637in}{0.829252in}}%
\pgfpathlineto{\pgfqpoint{5.448811in}{0.780467in}}%
\pgfpathlineto{\pgfqpoint{5.779003in}{0.727845in}}%
\pgfpathlineto{\pgfqpoint{6.109195in}{0.676276in}}%
\pgfpathlineto{\pgfqpoint{6.439386in}{0.625682in}}%
\pgfpathlineto{\pgfqpoint{6.739560in}{0.580511in}}%
\pgfpathlineto{\pgfqpoint{6.739560in}{0.580511in}}%
\pgfusepath{stroke}%
\end{pgfscope}%
\begin{pgfscope}%
\pgfpathrectangle{\pgfqpoint{0.766095in}{0.571603in}}{\pgfqpoint{5.973465in}{5.068436in}}%
\pgfusepath{clip}%
\pgfsetbuttcap%
\pgfsetroundjoin%
\pgfsetlinewidth{1.505625pt}%
\definecolor{currentstroke}{rgb}{0.203063,0.379716,0.553925}%
\pgfsetstrokecolor{currentstroke}%
\pgfsetdash{}{0pt}%
\pgfpathmoveto{\pgfqpoint{0.766095in}{1.678274in}}%
\pgfpathlineto{\pgfqpoint{0.916182in}{1.634381in}}%
\pgfpathlineto{\pgfqpoint{1.096287in}{1.584019in}}%
\pgfpathlineto{\pgfqpoint{1.306408in}{1.528075in}}%
\pgfpathlineto{\pgfqpoint{1.486513in}{1.482190in}}%
\pgfpathlineto{\pgfqpoint{1.726652in}{1.423694in}}%
\pgfpathlineto{\pgfqpoint{1.936774in}{1.374669in}}%
\pgfpathlineto{\pgfqpoint{2.176913in}{1.320811in}}%
\pgfpathlineto{\pgfqpoint{2.387035in}{1.275366in}}%
\pgfpathlineto{\pgfqpoint{2.657192in}{1.218976in}}%
\pgfpathlineto{\pgfqpoint{2.897331in}{1.170552in}}%
\pgfpathlineto{\pgfqpoint{3.197505in}{1.111969in}}%
\pgfpathlineto{\pgfqpoint{3.437645in}{1.066557in}}%
\pgfpathlineto{\pgfqpoint{3.737819in}{1.011303in}}%
\pgfpathlineto{\pgfqpoint{4.007976in}{0.962933in}}%
\pgfpathlineto{\pgfqpoint{4.338167in}{0.905278in}}%
\pgfpathlineto{\pgfqpoint{4.608324in}{0.859315in}}%
\pgfpathlineto{\pgfqpoint{4.959390in}{0.800829in}}%
\pgfpathlineto{\pgfqpoint{5.598899in}{0.697933in}}%
\pgfpathlineto{\pgfqpoint{6.319316in}{0.586737in}}%
\pgfpathlineto{\pgfqpoint{6.418960in}{0.571603in}}%
\pgfpathlineto{\pgfqpoint{6.418960in}{0.571603in}}%
\pgfusepath{stroke}%
\end{pgfscope}%
\begin{pgfscope}%
\pgfpathrectangle{\pgfqpoint{0.766095in}{0.571603in}}{\pgfqpoint{5.973465in}{5.068436in}}%
\pgfusepath{clip}%
\pgfsetbuttcap%
\pgfsetroundjoin%
\pgfsetlinewidth{1.505625pt}%
\definecolor{currentstroke}{rgb}{0.190631,0.407061,0.556089}%
\pgfsetstrokecolor{currentstroke}%
\pgfsetdash{}{0pt}%
\pgfpathmoveto{\pgfqpoint{0.766095in}{1.602744in}}%
\pgfpathlineto{\pgfqpoint{0.916182in}{1.560429in}}%
\pgfpathlineto{\pgfqpoint{1.126304in}{1.503924in}}%
\pgfpathlineto{\pgfqpoint{1.306408in}{1.457627in}}%
\pgfpathlineto{\pgfqpoint{1.546548in}{1.398689in}}%
\pgfpathlineto{\pgfqpoint{1.756670in}{1.349345in}}%
\pgfpathlineto{\pgfqpoint{1.996809in}{1.295194in}}%
\pgfpathlineto{\pgfqpoint{2.206931in}{1.249544in}}%
\pgfpathlineto{\pgfqpoint{2.477088in}{1.192943in}}%
\pgfpathlineto{\pgfqpoint{2.717227in}{1.144376in}}%
\pgfpathlineto{\pgfqpoint{2.987384in}{1.091460in}}%
\pgfpathlineto{\pgfqpoint{3.287558in}{1.034528in}}%
\pgfpathlineto{\pgfqpoint{3.527697in}{0.990318in}}%
\pgfpathlineto{\pgfqpoint{3.874151in}{0.928177in}}%
\pgfpathlineto{\pgfqpoint{4.398202in}{0.837837in}}%
\pgfpathlineto{\pgfqpoint{4.788428in}{0.772786in}}%
\pgfpathlineto{\pgfqpoint{5.405233in}{0.673481in}}%
\pgfpathlineto{\pgfqpoint{6.063118in}{0.571603in}}%
\pgfpathlineto{\pgfqpoint{6.063118in}{0.571603in}}%
\pgfusepath{stroke}%
\end{pgfscope}%
\begin{pgfscope}%
\pgfpathrectangle{\pgfqpoint{0.766095in}{0.571603in}}{\pgfqpoint{5.973465in}{5.068436in}}%
\pgfusepath{clip}%
\pgfsetbuttcap%
\pgfsetroundjoin%
\pgfsetlinewidth{1.505625pt}%
\definecolor{currentstroke}{rgb}{0.179019,0.433756,0.557430}%
\pgfsetstrokecolor{currentstroke}%
\pgfsetdash{}{0pt}%
\pgfpathmoveto{\pgfqpoint{0.766095in}{1.532426in}}%
\pgfpathlineto{\pgfqpoint{0.946199in}{1.483522in}}%
\pgfpathlineto{\pgfqpoint{1.156321in}{1.429130in}}%
\pgfpathlineto{\pgfqpoint{1.366443in}{1.377203in}}%
\pgfpathlineto{\pgfqpoint{1.576565in}{1.327439in}}%
\pgfpathlineto{\pgfqpoint{1.816704in}{1.272909in}}%
\pgfpathlineto{\pgfqpoint{2.056844in}{1.220547in}}%
\pgfpathlineto{\pgfqpoint{2.296983in}{1.170091in}}%
\pgfpathlineto{\pgfqpoint{2.567140in}{1.115322in}}%
\pgfpathlineto{\pgfqpoint{2.807279in}{1.068224in}}%
\pgfpathlineto{\pgfqpoint{3.107453in}{1.011153in}}%
\pgfpathlineto{\pgfqpoint{3.377610in}{0.961355in}}%
\pgfpathlineto{\pgfqpoint{3.677784in}{0.907516in}}%
\pgfpathlineto{\pgfqpoint{3.947941in}{0.860324in}}%
\pgfpathlineto{\pgfqpoint{4.296699in}{0.800829in}}%
\pgfpathlineto{\pgfqpoint{4.908498in}{0.700227in}}%
\pgfpathlineto{\pgfqpoint{5.238690in}{0.647593in}}%
\pgfpathlineto{\pgfqpoint{5.727338in}{0.571603in}}%
\pgfpathlineto{\pgfqpoint{5.727338in}{0.571603in}}%
\pgfusepath{stroke}%
\end{pgfscope}%
\begin{pgfscope}%
\pgfpathrectangle{\pgfqpoint{0.766095in}{0.571603in}}{\pgfqpoint{5.973465in}{5.068436in}}%
\pgfusepath{clip}%
\pgfsetbuttcap%
\pgfsetroundjoin%
\pgfsetlinewidth{1.505625pt}%
\definecolor{currentstroke}{rgb}{0.168126,0.459988,0.558082}%
\pgfsetstrokecolor{currentstroke}%
\pgfsetdash{}{0pt}%
\pgfpathmoveto{\pgfqpoint{0.766095in}{1.466565in}}%
\pgfpathlineto{\pgfqpoint{0.916182in}{1.426929in}}%
\pgfpathlineto{\pgfqpoint{1.096287in}{1.381106in}}%
\pgfpathlineto{\pgfqpoint{1.336426in}{1.322780in}}%
\pgfpathlineto{\pgfqpoint{1.546548in}{1.273953in}}%
\pgfpathlineto{\pgfqpoint{1.786687in}{1.220380in}}%
\pgfpathlineto{\pgfqpoint{2.026826in}{1.168886in}}%
\pgfpathlineto{\pgfqpoint{2.266966in}{1.119221in}}%
\pgfpathlineto{\pgfqpoint{2.537122in}{1.065261in}}%
\pgfpathlineto{\pgfqpoint{2.807279in}{1.013089in}}%
\pgfpathlineto{\pgfqpoint{3.077436in}{0.962500in}}%
\pgfpathlineto{\pgfqpoint{3.377610in}{0.907903in}}%
\pgfpathlineto{\pgfqpoint{3.647767in}{0.860113in}}%
\pgfpathlineto{\pgfqpoint{3.977958in}{0.803135in}}%
\pgfpathlineto{\pgfqpoint{4.278132in}{0.752664in}}%
\pgfpathlineto{\pgfqpoint{4.578307in}{0.703303in}}%
\pgfpathlineto{\pgfqpoint{4.921579in}{0.648012in}}%
\pgfpathlineto{\pgfqpoint{5.408771in}{0.571603in}}%
\pgfpathlineto{\pgfqpoint{5.408771in}{0.571603in}}%
\pgfusepath{stroke}%
\end{pgfscope}%
\begin{pgfscope}%
\pgfpathrectangle{\pgfqpoint{0.766095in}{0.571603in}}{\pgfqpoint{5.973465in}{5.068436in}}%
\pgfusepath{clip}%
\pgfsetbuttcap%
\pgfsetroundjoin%
\pgfsetlinewidth{1.505625pt}%
\definecolor{currentstroke}{rgb}{0.157729,0.485932,0.558013}%
\pgfsetstrokecolor{currentstroke}%
\pgfsetdash{}{0pt}%
\pgfpathmoveto{\pgfqpoint{0.766095in}{1.404621in}}%
\pgfpathlineto{\pgfqpoint{0.976217in}{1.350974in}}%
\pgfpathlineto{\pgfqpoint{1.186339in}{1.299754in}}%
\pgfpathlineto{\pgfqpoint{1.396461in}{1.250663in}}%
\pgfpathlineto{\pgfqpoint{1.636600in}{1.196861in}}%
\pgfpathlineto{\pgfqpoint{1.876739in}{1.145191in}}%
\pgfpathlineto{\pgfqpoint{2.146896in}{1.089278in}}%
\pgfpathlineto{\pgfqpoint{2.387035in}{1.041331in}}%
\pgfpathlineto{\pgfqpoint{2.657192in}{0.989094in}}%
\pgfpathlineto{\pgfqpoint{2.957366in}{0.932905in}}%
\pgfpathlineto{\pgfqpoint{3.227523in}{0.883860in}}%
\pgfpathlineto{\pgfqpoint{3.527697in}{0.830802in}}%
\pgfpathlineto{\pgfqpoint{3.827871in}{0.779126in}}%
\pgfpathlineto{\pgfqpoint{4.128045in}{0.728690in}}%
\pgfpathlineto{\pgfqpoint{4.458237in}{0.674446in}}%
\pgfpathlineto{\pgfqpoint{4.781424in}{0.622542in}}%
\pgfpathlineto{\pgfqpoint{5.105207in}{0.571603in}}%
\pgfpathlineto{\pgfqpoint{5.105207in}{0.571603in}}%
\pgfusepath{stroke}%
\end{pgfscope}%
\begin{pgfscope}%
\pgfpathrectangle{\pgfqpoint{0.766095in}{0.571603in}}{\pgfqpoint{5.973465in}{5.068436in}}%
\pgfusepath{clip}%
\pgfsetbuttcap%
\pgfsetroundjoin%
\pgfsetlinewidth{1.505625pt}%
\definecolor{currentstroke}{rgb}{0.147607,0.511733,0.557049}%
\pgfsetstrokecolor{currentstroke}%
\pgfsetdash{}{0pt}%
\pgfpathmoveto{\pgfqpoint{0.766095in}{1.346039in}}%
\pgfpathlineto{\pgfqpoint{0.946199in}{1.301045in}}%
\pgfpathlineto{\pgfqpoint{1.156321in}{1.250712in}}%
\pgfpathlineto{\pgfqpoint{1.396461in}{1.195694in}}%
\pgfpathlineto{\pgfqpoint{1.636600in}{1.142989in}}%
\pgfpathlineto{\pgfqpoint{1.876739in}{1.092312in}}%
\pgfpathlineto{\pgfqpoint{2.116879in}{1.043412in}}%
\pgfpathlineto{\pgfqpoint{2.387035in}{0.990264in}}%
\pgfpathlineto{\pgfqpoint{2.657192in}{0.938855in}}%
\pgfpathlineto{\pgfqpoint{2.957366in}{0.883506in}}%
\pgfpathlineto{\pgfqpoint{3.227523in}{0.835143in}}%
\pgfpathlineto{\pgfqpoint{3.557714in}{0.777587in}}%
\pgfpathlineto{\pgfqpoint{3.857889in}{0.726678in}}%
\pgfpathlineto{\pgfqpoint{4.158063in}{0.676945in}}%
\pgfpathlineto{\pgfqpoint{4.493633in}{0.622542in}}%
\pgfpathlineto{\pgfqpoint{4.814891in}{0.571603in}}%
\pgfpathlineto{\pgfqpoint{4.814891in}{0.571603in}}%
\pgfusepath{stroke}%
\end{pgfscope}%
\begin{pgfscope}%
\pgfpathrectangle{\pgfqpoint{0.766095in}{0.571603in}}{\pgfqpoint{5.973465in}{5.068436in}}%
\pgfusepath{clip}%
\pgfsetbuttcap%
\pgfsetroundjoin%
\pgfsetlinewidth{1.505625pt}%
\definecolor{currentstroke}{rgb}{0.137770,0.537492,0.554906}%
\pgfsetstrokecolor{currentstroke}%
\pgfsetdash{}{0pt}%
\pgfpathmoveto{\pgfqpoint{0.766095in}{1.290411in}}%
\pgfpathlineto{\pgfqpoint{0.916182in}{1.253674in}}%
\pgfpathlineto{\pgfqpoint{1.156321in}{1.197261in}}%
\pgfpathlineto{\pgfqpoint{1.366443in}{1.149964in}}%
\pgfpathlineto{\pgfqpoint{1.636600in}{1.091671in}}%
\pgfpathlineto{\pgfqpoint{1.876739in}{1.041905in}}%
\pgfpathlineto{\pgfqpoint{2.146896in}{0.987922in}}%
\pgfpathlineto{\pgfqpoint{2.417053in}{0.935808in}}%
\pgfpathlineto{\pgfqpoint{2.687209in}{0.885338in}}%
\pgfpathlineto{\pgfqpoint{2.987384in}{0.830925in}}%
\pgfpathlineto{\pgfqpoint{3.257540in}{0.783338in}}%
\pgfpathlineto{\pgfqpoint{3.587732in}{0.726641in}}%
\pgfpathlineto{\pgfqpoint{3.887906in}{0.676446in}}%
\pgfpathlineto{\pgfqpoint{4.218098in}{0.622471in}}%
\pgfpathlineto{\pgfqpoint{4.536450in}{0.571603in}}%
\pgfpathlineto{\pgfqpoint{4.536450in}{0.571603in}}%
\pgfusepath{stroke}%
\end{pgfscope}%
\begin{pgfscope}%
\pgfpathrectangle{\pgfqpoint{0.766095in}{0.571603in}}{\pgfqpoint{5.973465in}{5.068436in}}%
\pgfusepath{clip}%
\pgfsetbuttcap%
\pgfsetroundjoin%
\pgfsetlinewidth{1.505625pt}%
\definecolor{currentstroke}{rgb}{0.127568,0.566949,0.550556}%
\pgfsetstrokecolor{currentstroke}%
\pgfsetdash{}{0pt}%
\pgfpathmoveto{\pgfqpoint{0.766095in}{1.237446in}}%
\pgfpathlineto{\pgfqpoint{0.916182in}{1.201531in}}%
\pgfpathlineto{\pgfqpoint{1.156321in}{1.146271in}}%
\pgfpathlineto{\pgfqpoint{1.396461in}{1.093410in}}%
\pgfpathlineto{\pgfqpoint{1.636600in}{1.042646in}}%
\pgfpathlineto{\pgfqpoint{1.876739in}{0.993715in}}%
\pgfpathlineto{\pgfqpoint{2.176913in}{0.934789in}}%
\pgfpathlineto{\pgfqpoint{2.417053in}{0.889256in}}%
\pgfpathlineto{\pgfqpoint{2.717227in}{0.834034in}}%
\pgfpathlineto{\pgfqpoint{3.017401in}{0.780498in}}%
\pgfpathlineto{\pgfqpoint{3.317575in}{0.728455in}}%
\pgfpathlineto{\pgfqpoint{3.617749in}{0.677740in}}%
\pgfpathlineto{\pgfqpoint{3.947941in}{0.623273in}}%
\pgfpathlineto{\pgfqpoint{4.268709in}{0.571603in}}%
\pgfpathlineto{\pgfqpoint{4.268709in}{0.571603in}}%
\pgfusepath{stroke}%
\end{pgfscope}%
\begin{pgfscope}%
\pgfpathrectangle{\pgfqpoint{0.766095in}{0.571603in}}{\pgfqpoint{5.973465in}{5.068436in}}%
\pgfusepath{clip}%
\pgfsetbuttcap%
\pgfsetroundjoin%
\pgfsetlinewidth{1.505625pt}%
\definecolor{currentstroke}{rgb}{0.121148,0.592739,0.544641}%
\pgfsetstrokecolor{currentstroke}%
\pgfsetdash{}{0pt}%
\pgfpathmoveto{\pgfqpoint{0.766095in}{1.186879in}}%
\pgfpathlineto{\pgfqpoint{0.946199in}{1.144796in}}%
\pgfpathlineto{\pgfqpoint{1.156321in}{1.097495in}}%
\pgfpathlineto{\pgfqpoint{1.396461in}{1.045588in}}%
\pgfpathlineto{\pgfqpoint{1.666617in}{0.989571in}}%
\pgfpathlineto{\pgfqpoint{1.906757in}{0.941625in}}%
\pgfpathlineto{\pgfqpoint{2.206931in}{0.883795in}}%
\pgfpathlineto{\pgfqpoint{2.477088in}{0.833531in}}%
\pgfpathlineto{\pgfqpoint{2.747244in}{0.784736in}}%
\pgfpathlineto{\pgfqpoint{3.077436in}{0.726792in}}%
\pgfpathlineto{\pgfqpoint{3.377610in}{0.675626in}}%
\pgfpathlineto{\pgfqpoint{3.677784in}{0.625712in}}%
\pgfpathlineto{\pgfqpoint{4.010722in}{0.571603in}}%
\pgfpathlineto{\pgfqpoint{4.010722in}{0.571603in}}%
\pgfusepath{stroke}%
\end{pgfscope}%
\begin{pgfscope}%
\pgfpathrectangle{\pgfqpoint{0.766095in}{0.571603in}}{\pgfqpoint{5.973465in}{5.068436in}}%
\pgfusepath{clip}%
\pgfsetbuttcap%
\pgfsetroundjoin%
\pgfsetlinewidth{1.505625pt}%
\definecolor{currentstroke}{rgb}{0.119699,0.618490,0.536347}%
\pgfsetstrokecolor{currentstroke}%
\pgfsetdash{}{0pt}%
\pgfpathmoveto{\pgfqpoint{0.766095in}{1.138475in}}%
\pgfpathlineto{\pgfqpoint{0.793965in}{1.131933in}}%
\pgfpathlineto{\pgfqpoint{0.796112in}{1.131438in}}%
\pgfpathlineto{\pgfqpoint{0.826130in}{1.124544in}}%
\pgfpathlineto{\pgfqpoint{0.856147in}{1.117650in}}%
\pgfpathlineto{\pgfqpoint{0.886165in}{1.110756in}}%
\pgfpathlineto{\pgfqpoint{0.904878in}{1.106463in}}%
\pgfpathlineto{\pgfqpoint{0.916182in}{1.103919in}}%
\pgfpathlineto{\pgfqpoint{0.946199in}{1.097174in}}%
\pgfpathlineto{\pgfqpoint{0.976217in}{1.090429in}}%
\pgfpathlineto{\pgfqpoint{1.006234in}{1.083682in}}%
\pgfpathlineto{\pgfqpoint{1.018217in}{1.080994in}}%
\pgfpathlineto{\pgfqpoint{1.036252in}{1.077023in}}%
\pgfpathlineto{\pgfqpoint{1.066269in}{1.070421in}}%
\pgfpathlineto{\pgfqpoint{1.096287in}{1.063817in}}%
\pgfpathlineto{\pgfqpoint{1.126304in}{1.057211in}}%
\pgfpathlineto{\pgfqpoint{1.133984in}{1.055524in}}%
\pgfpathlineto{\pgfqpoint{1.156321in}{1.050710in}}%
\pgfpathlineto{\pgfqpoint{1.186339in}{1.044244in}}%
\pgfpathlineto{\pgfqpoint{1.216356in}{1.037775in}}%
\pgfpathlineto{\pgfqpoint{1.246374in}{1.031303in}}%
\pgfpathlineto{\pgfqpoint{1.252174in}{1.030055in}}%
\pgfpathlineto{\pgfqpoint{1.276391in}{1.024942in}}%
\pgfpathlineto{\pgfqpoint{1.306408in}{1.018605in}}%
\pgfpathlineto{\pgfqpoint{1.336426in}{1.012265in}}%
\pgfpathlineto{\pgfqpoint{1.366443in}{1.005920in}}%
\pgfpathlineto{\pgfqpoint{1.372771in}{1.004585in}}%
\pgfpathlineto{\pgfqpoint{1.396461in}{0.999682in}}%
\pgfpathlineto{\pgfqpoint{1.426478in}{0.993469in}}%
\pgfpathlineto{\pgfqpoint{1.456496in}{0.987251in}}%
\pgfpathlineto{\pgfqpoint{1.486513in}{0.981029in}}%
\pgfpathlineto{\pgfqpoint{1.495756in}{0.979116in}}%
\pgfpathlineto{\pgfqpoint{1.516530in}{0.974897in}}%
\pgfpathlineto{\pgfqpoint{1.546548in}{0.968802in}}%
\pgfpathlineto{\pgfqpoint{1.576565in}{0.962702in}}%
\pgfpathlineto{\pgfqpoint{1.606583in}{0.956597in}}%
\pgfpathlineto{\pgfqpoint{1.621101in}{0.953646in}}%
\pgfpathlineto{\pgfqpoint{1.636600in}{0.950556in}}%
\pgfpathlineto{\pgfqpoint{1.666617in}{0.944574in}}%
\pgfpathlineto{\pgfqpoint{1.696635in}{0.938587in}}%
\pgfpathlineto{\pgfqpoint{1.726652in}{0.932594in}}%
\pgfpathlineto{\pgfqpoint{1.748771in}{0.928177in}}%
\pgfpathlineto{\pgfqpoint{1.756670in}{0.926629in}}%
\pgfpathlineto{\pgfqpoint{1.786687in}{0.920756in}}%
\pgfpathlineto{\pgfqpoint{1.816704in}{0.914877in}}%
\pgfpathlineto{\pgfqpoint{1.846722in}{0.908992in}}%
\pgfpathlineto{\pgfqpoint{1.876739in}{0.903099in}}%
\pgfpathlineto{\pgfqpoint{1.878742in}{0.902707in}}%
\pgfpathlineto{\pgfqpoint{1.906757in}{0.897322in}}%
\pgfpathlineto{\pgfqpoint{1.936774in}{0.891547in}}%
\pgfpathlineto{\pgfqpoint{1.966792in}{0.885765in}}%
\pgfpathlineto{\pgfqpoint{1.996809in}{0.879975in}}%
\pgfpathlineto{\pgfqpoint{2.011010in}{0.877238in}}%
\pgfpathlineto{\pgfqpoint{2.026826in}{0.874246in}}%
\pgfpathlineto{\pgfqpoint{2.056844in}{0.868571in}}%
\pgfpathlineto{\pgfqpoint{2.086861in}{0.862888in}}%
\pgfpathlineto{\pgfqpoint{2.116879in}{0.857198in}}%
\pgfpathlineto{\pgfqpoint{2.145487in}{0.851768in}}%
\pgfpathlineto{\pgfqpoint{2.146896in}{0.851506in}}%
\pgfpathlineto{\pgfqpoint{2.176913in}{0.845926in}}%
\pgfpathlineto{\pgfqpoint{2.206931in}{0.840339in}}%
\pgfpathlineto{\pgfqpoint{2.236948in}{0.834745in}}%
\pgfpathlineto{\pgfqpoint{2.266966in}{0.829142in}}%
\pgfpathlineto{\pgfqpoint{2.282198in}{0.826299in}}%
\pgfpathlineto{\pgfqpoint{2.296983in}{0.823592in}}%
\pgfpathlineto{\pgfqpoint{2.327000in}{0.818097in}}%
\pgfpathlineto{\pgfqpoint{2.357018in}{0.812594in}}%
\pgfpathlineto{\pgfqpoint{2.387035in}{0.807083in}}%
\pgfpathlineto{\pgfqpoint{2.417053in}{0.801563in}}%
\pgfpathlineto{\pgfqpoint{2.421049in}{0.800829in}}%
\pgfpathlineto{\pgfqpoint{2.447070in}{0.796141in}}%
\pgfpathlineto{\pgfqpoint{2.477088in}{0.790727in}}%
\pgfpathlineto{\pgfqpoint{2.507105in}{0.785305in}}%
\pgfpathlineto{\pgfqpoint{2.537122in}{0.779873in}}%
\pgfpathlineto{\pgfqpoint{2.562040in}{0.775360in}}%
\pgfpathlineto{\pgfqpoint{2.567140in}{0.774453in}}%
\pgfpathlineto{\pgfqpoint{2.597157in}{0.769125in}}%
\pgfpathlineto{\pgfqpoint{2.627175in}{0.763789in}}%
\pgfpathlineto{\pgfqpoint{2.657192in}{0.758443in}}%
\pgfpathlineto{\pgfqpoint{2.687209in}{0.753087in}}%
\pgfpathlineto{\pgfqpoint{2.705122in}{0.749890in}}%
\pgfpathlineto{\pgfqpoint{2.717227in}{0.747771in}}%
\pgfpathlineto{\pgfqpoint{2.747244in}{0.742517in}}%
\pgfpathlineto{\pgfqpoint{2.777262in}{0.737254in}}%
\pgfpathlineto{\pgfqpoint{2.807279in}{0.731982in}}%
\pgfpathlineto{\pgfqpoint{2.837297in}{0.726699in}}%
\pgfpathlineto{\pgfqpoint{2.850245in}{0.724420in}}%
\pgfpathlineto{\pgfqpoint{2.867314in}{0.721475in}}%
\pgfpathlineto{\pgfqpoint{2.897331in}{0.716292in}}%
\pgfpathlineto{\pgfqpoint{2.927349in}{0.711100in}}%
\pgfpathlineto{\pgfqpoint{2.957366in}{0.705897in}}%
\pgfpathlineto{\pgfqpoint{2.987384in}{0.700685in}}%
\pgfpathlineto{\pgfqpoint{2.997373in}{0.698951in}}%
\pgfpathlineto{\pgfqpoint{3.017401in}{0.695541in}}%
\pgfpathlineto{\pgfqpoint{3.047418in}{0.690427in}}%
\pgfpathlineto{\pgfqpoint{3.077436in}{0.685302in}}%
\pgfpathlineto{\pgfqpoint{3.107453in}{0.680168in}}%
\pgfpathlineto{\pgfqpoint{3.137471in}{0.675023in}}%
\pgfpathlineto{\pgfqpoint{3.146468in}{0.673481in}}%
\pgfpathlineto{\pgfqpoint{3.167488in}{0.669950in}}%
\pgfpathlineto{\pgfqpoint{3.197505in}{0.664901in}}%
\pgfpathlineto{\pgfqpoint{3.227523in}{0.659842in}}%
\pgfpathlineto{\pgfqpoint{3.257540in}{0.654773in}}%
\pgfpathlineto{\pgfqpoint{3.287558in}{0.649693in}}%
\pgfpathlineto{\pgfqpoint{3.297491in}{0.648012in}}%
\pgfpathlineto{\pgfqpoint{3.317575in}{0.644680in}}%
\pgfpathlineto{\pgfqpoint{3.347593in}{0.639694in}}%
\pgfpathlineto{\pgfqpoint{3.377610in}{0.634698in}}%
\pgfpathlineto{\pgfqpoint{3.407627in}{0.629692in}}%
\pgfpathlineto{\pgfqpoint{3.437645in}{0.624675in}}%
\pgfpathlineto{\pgfqpoint{3.450401in}{0.622542in}}%
\pgfpathlineto{\pgfqpoint{3.467662in}{0.619713in}}%
\pgfpathlineto{\pgfqpoint{3.497680in}{0.614789in}}%
\pgfpathlineto{\pgfqpoint{3.527697in}{0.609854in}}%
\pgfpathlineto{\pgfqpoint{3.557714in}{0.604909in}}%
\pgfpathlineto{\pgfqpoint{3.587732in}{0.599952in}}%
\pgfpathlineto{\pgfqpoint{3.605157in}{0.597073in}}%
\pgfpathlineto{\pgfqpoint{3.617749in}{0.595033in}}%
\pgfpathlineto{\pgfqpoint{3.647767in}{0.590168in}}%
\pgfpathlineto{\pgfqpoint{3.677784in}{0.585292in}}%
\pgfpathlineto{\pgfqpoint{3.707802in}{0.580406in}}%
\pgfpathlineto{\pgfqpoint{3.737819in}{0.575509in}}%
\pgfpathlineto{\pgfqpoint{3.761717in}{0.571603in}}%
\pgfusepath{stroke}%
\end{pgfscope}%
\begin{pgfscope}%
\pgfpathrectangle{\pgfqpoint{0.766095in}{0.571603in}}{\pgfqpoint{5.973465in}{5.068436in}}%
\pgfusepath{clip}%
\pgfsetbuttcap%
\pgfsetroundjoin%
\pgfsetlinewidth{1.505625pt}%
\definecolor{currentstroke}{rgb}{0.126326,0.644107,0.525311}%
\pgfsetstrokecolor{currentstroke}%
\pgfsetdash{}{0pt}%
\pgfpathmoveto{\pgfqpoint{0.766095in}{1.092020in}}%
\pgfpathlineto{\pgfqpoint{0.796112in}{1.085129in}}%
\pgfpathlineto{\pgfqpoint{0.814148in}{1.080994in}}%
\pgfpathlineto{\pgfqpoint{0.826130in}{1.078297in}}%
\pgfpathlineto{\pgfqpoint{0.856147in}{1.071556in}}%
\pgfpathlineto{\pgfqpoint{0.886165in}{1.064814in}}%
\pgfpathlineto{\pgfqpoint{0.916182in}{1.058072in}}%
\pgfpathlineto{\pgfqpoint{0.927545in}{1.055524in}}%
\pgfpathlineto{\pgfqpoint{0.946199in}{1.051420in}}%
\pgfpathlineto{\pgfqpoint{0.976217in}{1.044822in}}%
\pgfpathlineto{\pgfqpoint{1.006234in}{1.038223in}}%
\pgfpathlineto{\pgfqpoint{1.036252in}{1.031623in}}%
\pgfpathlineto{\pgfqpoint{1.043401in}{1.030055in}}%
\pgfpathlineto{\pgfqpoint{1.066269in}{1.025131in}}%
\pgfpathlineto{\pgfqpoint{1.096287in}{1.018670in}}%
\pgfpathlineto{\pgfqpoint{1.126304in}{1.012208in}}%
\pgfpathlineto{\pgfqpoint{1.156321in}{1.005743in}}%
\pgfpathlineto{\pgfqpoint{1.161710in}{1.004585in}}%
\pgfpathlineto{\pgfqpoint{1.186339in}{0.999391in}}%
\pgfpathlineto{\pgfqpoint{1.216356in}{0.993061in}}%
\pgfpathlineto{\pgfqpoint{1.246374in}{0.986729in}}%
\pgfpathlineto{\pgfqpoint{1.276391in}{0.980394in}}%
\pgfpathlineto{\pgfqpoint{1.282457in}{0.979116in}}%
\pgfpathlineto{\pgfqpoint{1.306408in}{0.974164in}}%
\pgfpathlineto{\pgfqpoint{1.336426in}{0.967960in}}%
\pgfpathlineto{\pgfqpoint{1.366443in}{0.961752in}}%
\pgfpathlineto{\pgfqpoint{1.396461in}{0.955539in}}%
\pgfpathlineto{\pgfqpoint{1.405623in}{0.953646in}}%
\pgfpathlineto{\pgfqpoint{1.426478in}{0.949417in}}%
\pgfpathlineto{\pgfqpoint{1.456496in}{0.943332in}}%
\pgfpathlineto{\pgfqpoint{1.486513in}{0.937242in}}%
\pgfpathlineto{\pgfqpoint{1.516530in}{0.931148in}}%
\pgfpathlineto{\pgfqpoint{1.531178in}{0.928177in}}%
\pgfpathlineto{\pgfqpoint{1.546548in}{0.925117in}}%
\pgfpathlineto{\pgfqpoint{1.576565in}{0.919146in}}%
\pgfpathlineto{\pgfqpoint{1.606583in}{0.913170in}}%
\pgfpathlineto{\pgfqpoint{1.636600in}{0.907189in}}%
\pgfpathlineto{\pgfqpoint{1.659089in}{0.902707in}}%
\pgfpathlineto{\pgfqpoint{1.666617in}{0.901235in}}%
\pgfpathlineto{\pgfqpoint{1.696635in}{0.895373in}}%
\pgfpathlineto{\pgfqpoint{1.726652in}{0.889506in}}%
\pgfpathlineto{\pgfqpoint{1.756670in}{0.883634in}}%
\pgfpathlineto{\pgfqpoint{1.786687in}{0.877755in}}%
\pgfpathlineto{\pgfqpoint{1.789332in}{0.877238in}}%
\pgfpathlineto{\pgfqpoint{1.816704in}{0.871987in}}%
\pgfpathlineto{\pgfqpoint{1.846722in}{0.866224in}}%
\pgfpathlineto{\pgfqpoint{1.876739in}{0.860456in}}%
\pgfpathlineto{\pgfqpoint{1.906757in}{0.854680in}}%
\pgfpathlineto{\pgfqpoint{1.921900in}{0.851768in}}%
\pgfpathlineto{\pgfqpoint{1.936774in}{0.848961in}}%
\pgfpathlineto{\pgfqpoint{1.966792in}{0.843299in}}%
\pgfpathlineto{\pgfqpoint{1.996809in}{0.837630in}}%
\pgfpathlineto{\pgfqpoint{2.026826in}{0.831955in}}%
\pgfpathlineto{\pgfqpoint{2.056703in}{0.826299in}}%
\pgfpathlineto{\pgfqpoint{2.056844in}{0.826272in}}%
\pgfpathlineto{\pgfqpoint{2.086861in}{0.820707in}}%
\pgfpathlineto{\pgfqpoint{2.116879in}{0.815134in}}%
\pgfpathlineto{\pgfqpoint{2.146896in}{0.809555in}}%
\pgfpathlineto{\pgfqpoint{2.176913in}{0.803967in}}%
\pgfpathlineto{\pgfqpoint{2.193773in}{0.800829in}}%
\pgfpathlineto{\pgfqpoint{2.206931in}{0.798426in}}%
\pgfpathlineto{\pgfqpoint{2.236948in}{0.792946in}}%
\pgfpathlineto{\pgfqpoint{2.266966in}{0.787459in}}%
\pgfpathlineto{\pgfqpoint{2.296983in}{0.781964in}}%
\pgfpathlineto{\pgfqpoint{2.327000in}{0.776461in}}%
\pgfpathlineto{\pgfqpoint{2.333012in}{0.775360in}}%
\pgfpathlineto{\pgfqpoint{2.357018in}{0.771046in}}%
\pgfpathlineto{\pgfqpoint{2.387035in}{0.765648in}}%
\pgfpathlineto{\pgfqpoint{2.417053in}{0.760242in}}%
\pgfpathlineto{\pgfqpoint{2.447070in}{0.754827in}}%
\pgfpathlineto{\pgfqpoint{2.474405in}{0.749890in}}%
\pgfpathlineto{\pgfqpoint{2.477088in}{0.749415in}}%
\pgfpathlineto{\pgfqpoint{2.507105in}{0.744103in}}%
\pgfpathlineto{\pgfqpoint{2.537122in}{0.738782in}}%
\pgfpathlineto{\pgfqpoint{2.567140in}{0.733453in}}%
\pgfpathlineto{\pgfqpoint{2.597157in}{0.728116in}}%
\pgfpathlineto{\pgfqpoint{2.617924in}{0.724420in}}%
\pgfpathlineto{\pgfqpoint{2.627175in}{0.722806in}}%
\pgfpathlineto{\pgfqpoint{2.657192in}{0.717568in}}%
\pgfpathlineto{\pgfqpoint{2.687209in}{0.712323in}}%
\pgfpathlineto{\pgfqpoint{2.717227in}{0.707068in}}%
\pgfpathlineto{\pgfqpoint{2.747244in}{0.701804in}}%
\pgfpathlineto{\pgfqpoint{2.763506in}{0.698951in}}%
\pgfpathlineto{\pgfqpoint{2.777262in}{0.696584in}}%
\pgfpathlineto{\pgfqpoint{2.807279in}{0.691418in}}%
\pgfpathlineto{\pgfqpoint{2.837297in}{0.686244in}}%
\pgfpathlineto{\pgfqpoint{2.867314in}{0.681060in}}%
\pgfpathlineto{\pgfqpoint{2.897331in}{0.675867in}}%
\pgfpathlineto{\pgfqpoint{2.911116in}{0.673481in}}%
\pgfpathlineto{\pgfqpoint{2.927349in}{0.670726in}}%
\pgfpathlineto{\pgfqpoint{2.957366in}{0.665629in}}%
\pgfpathlineto{\pgfqpoint{2.987384in}{0.660523in}}%
\pgfpathlineto{\pgfqpoint{3.017401in}{0.655408in}}%
\pgfpathlineto{\pgfqpoint{3.047418in}{0.650282in}}%
\pgfpathlineto{\pgfqpoint{3.060712in}{0.648012in}}%
\pgfpathlineto{\pgfqpoint{3.077436in}{0.645211in}}%
\pgfpathlineto{\pgfqpoint{3.107453in}{0.640180in}}%
\pgfpathlineto{\pgfqpoint{3.137471in}{0.635140in}}%
\pgfpathlineto{\pgfqpoint{3.167488in}{0.630090in}}%
\pgfpathlineto{\pgfqpoint{3.197505in}{0.625030in}}%
\pgfpathlineto{\pgfqpoint{3.212253in}{0.622542in}}%
\pgfpathlineto{\pgfqpoint{3.227523in}{0.620017in}}%
\pgfpathlineto{\pgfqpoint{3.257540in}{0.615050in}}%
\pgfpathlineto{\pgfqpoint{3.287558in}{0.610073in}}%
\pgfpathlineto{\pgfqpoint{3.317575in}{0.605087in}}%
\pgfpathlineto{\pgfqpoint{3.347593in}{0.600090in}}%
\pgfpathlineto{\pgfqpoint{3.365699in}{0.597073in}}%
\pgfpathlineto{\pgfqpoint{3.377610in}{0.595127in}}%
\pgfpathlineto{\pgfqpoint{3.407627in}{0.590221in}}%
\pgfpathlineto{\pgfqpoint{3.437645in}{0.585306in}}%
\pgfpathlineto{\pgfqpoint{3.467662in}{0.580380in}}%
\pgfpathlineto{\pgfqpoint{3.497680in}{0.575444in}}%
\pgfpathlineto{\pgfqpoint{3.521005in}{0.571603in}}%
\pgfusepath{stroke}%
\end{pgfscope}%
\begin{pgfscope}%
\pgfpathrectangle{\pgfqpoint{0.766095in}{0.571603in}}{\pgfqpoint{5.973465in}{5.068436in}}%
\pgfusepath{clip}%
\pgfsetbuttcap%
\pgfsetroundjoin%
\pgfsetlinewidth{1.505625pt}%
\definecolor{currentstroke}{rgb}{0.143303,0.669459,0.511215}%
\pgfsetstrokecolor{currentstroke}%
\pgfsetdash{}{0pt}%
\pgfpathmoveto{\pgfqpoint{0.766095in}{1.047320in}}%
\pgfpathlineto{\pgfqpoint{0.796112in}{1.040585in}}%
\pgfpathlineto{\pgfqpoint{0.826130in}{1.033851in}}%
\pgfpathlineto{\pgfqpoint{0.843071in}{1.030055in}}%
\pgfpathlineto{\pgfqpoint{0.856147in}{1.027179in}}%
\pgfpathlineto{\pgfqpoint{0.886165in}{1.020588in}}%
\pgfpathlineto{\pgfqpoint{0.916182in}{1.013998in}}%
\pgfpathlineto{\pgfqpoint{0.946199in}{1.007407in}}%
\pgfpathlineto{\pgfqpoint{0.959073in}{1.004585in}}%
\pgfpathlineto{\pgfqpoint{0.976217in}{1.000896in}}%
\pgfpathlineto{\pgfqpoint{1.006234in}{0.994444in}}%
\pgfpathlineto{\pgfqpoint{1.036252in}{0.987992in}}%
\pgfpathlineto{\pgfqpoint{1.066269in}{0.981538in}}%
\pgfpathlineto{\pgfqpoint{1.077553in}{0.979116in}}%
\pgfpathlineto{\pgfqpoint{1.096287in}{0.975168in}}%
\pgfpathlineto{\pgfqpoint{1.126304in}{0.968849in}}%
\pgfpathlineto{\pgfqpoint{1.156321in}{0.962528in}}%
\pgfpathlineto{\pgfqpoint{1.186339in}{0.956204in}}%
\pgfpathlineto{\pgfqpoint{1.198499in}{0.953646in}}%
\pgfpathlineto{\pgfqpoint{1.216356in}{0.949959in}}%
\pgfpathlineto{\pgfqpoint{1.246374in}{0.943766in}}%
\pgfpathlineto{\pgfqpoint{1.276391in}{0.937570in}}%
\pgfpathlineto{\pgfqpoint{1.306408in}{0.931371in}}%
\pgfpathlineto{\pgfqpoint{1.321891in}{0.928177in}}%
\pgfpathlineto{\pgfqpoint{1.336426in}{0.925233in}}%
\pgfpathlineto{\pgfqpoint{1.366443in}{0.919160in}}%
\pgfpathlineto{\pgfqpoint{1.396461in}{0.913084in}}%
\pgfpathlineto{\pgfqpoint{1.426478in}{0.907004in}}%
\pgfpathlineto{\pgfqpoint{1.447699in}{0.902707in}}%
\pgfpathlineto{\pgfqpoint{1.456496in}{0.900959in}}%
\pgfpathlineto{\pgfqpoint{1.486513in}{0.895001in}}%
\pgfpathlineto{\pgfqpoint{1.516530in}{0.889040in}}%
\pgfpathlineto{\pgfqpoint{1.546548in}{0.883074in}}%
\pgfpathlineto{\pgfqpoint{1.575890in}{0.877238in}}%
\pgfpathlineto{\pgfqpoint{1.576565in}{0.877106in}}%
\pgfpathlineto{\pgfqpoint{1.606583in}{0.871259in}}%
\pgfpathlineto{\pgfqpoint{1.636600in}{0.865407in}}%
\pgfpathlineto{\pgfqpoint{1.666617in}{0.859550in}}%
\pgfpathlineto{\pgfqpoint{1.696635in}{0.853688in}}%
\pgfpathlineto{\pgfqpoint{1.706476in}{0.851768in}}%
\pgfpathlineto{\pgfqpoint{1.726652in}{0.847905in}}%
\pgfpathlineto{\pgfqpoint{1.756670in}{0.842159in}}%
\pgfpathlineto{\pgfqpoint{1.786687in}{0.836406in}}%
\pgfpathlineto{\pgfqpoint{1.816704in}{0.830648in}}%
\pgfpathlineto{\pgfqpoint{1.839374in}{0.826299in}}%
\pgfpathlineto{\pgfqpoint{1.846722in}{0.824915in}}%
\pgfpathlineto{\pgfqpoint{1.876739in}{0.819269in}}%
\pgfpathlineto{\pgfqpoint{1.906757in}{0.813618in}}%
\pgfpathlineto{\pgfqpoint{1.936774in}{0.807960in}}%
\pgfpathlineto{\pgfqpoint{1.966792in}{0.802296in}}%
\pgfpathlineto{\pgfqpoint{1.974572in}{0.800829in}}%
\pgfpathlineto{\pgfqpoint{1.996809in}{0.796716in}}%
\pgfpathlineto{\pgfqpoint{2.026826in}{0.791161in}}%
\pgfpathlineto{\pgfqpoint{2.056844in}{0.785599in}}%
\pgfpathlineto{\pgfqpoint{2.086861in}{0.780031in}}%
\pgfpathlineto{\pgfqpoint{2.112023in}{0.775360in}}%
\pgfpathlineto{\pgfqpoint{2.116879in}{0.774475in}}%
\pgfpathlineto{\pgfqpoint{2.146896in}{0.769013in}}%
\pgfpathlineto{\pgfqpoint{2.176913in}{0.763545in}}%
\pgfpathlineto{\pgfqpoint{2.206931in}{0.758069in}}%
\pgfpathlineto{\pgfqpoint{2.236948in}{0.752585in}}%
\pgfpathlineto{\pgfqpoint{2.251705in}{0.749890in}}%
\pgfpathlineto{\pgfqpoint{2.266966in}{0.747155in}}%
\pgfpathlineto{\pgfqpoint{2.296983in}{0.741776in}}%
\pgfpathlineto{\pgfqpoint{2.327000in}{0.736389in}}%
\pgfpathlineto{\pgfqpoint{2.357018in}{0.730995in}}%
\pgfpathlineto{\pgfqpoint{2.387035in}{0.725592in}}%
\pgfpathlineto{\pgfqpoint{2.393551in}{0.724420in}}%
\pgfpathlineto{\pgfqpoint{2.417053in}{0.720274in}}%
\pgfpathlineto{\pgfqpoint{2.447070in}{0.714973in}}%
\pgfpathlineto{\pgfqpoint{2.477088in}{0.709665in}}%
\pgfpathlineto{\pgfqpoint{2.507105in}{0.704348in}}%
\pgfpathlineto{\pgfqpoint{2.537122in}{0.699023in}}%
\pgfpathlineto{\pgfqpoint{2.537529in}{0.698951in}}%
\pgfpathlineto{\pgfqpoint{2.567140in}{0.693804in}}%
\pgfpathlineto{\pgfqpoint{2.597157in}{0.688579in}}%
\pgfpathlineto{\pgfqpoint{2.627175in}{0.683345in}}%
\pgfpathlineto{\pgfqpoint{2.657192in}{0.678102in}}%
\pgfpathlineto{\pgfqpoint{2.683616in}{0.673481in}}%
\pgfpathlineto{\pgfqpoint{2.687209in}{0.672865in}}%
\pgfpathlineto{\pgfqpoint{2.717227in}{0.667720in}}%
\pgfpathlineto{\pgfqpoint{2.747244in}{0.662566in}}%
\pgfpathlineto{\pgfqpoint{2.777262in}{0.657404in}}%
\pgfpathlineto{\pgfqpoint{2.807279in}{0.652233in}}%
\pgfpathlineto{\pgfqpoint{2.831750in}{0.648012in}}%
\pgfpathlineto{\pgfqpoint{2.837297in}{0.647073in}}%
\pgfpathlineto{\pgfqpoint{2.867314in}{0.641998in}}%
\pgfpathlineto{\pgfqpoint{2.897331in}{0.636913in}}%
\pgfpathlineto{\pgfqpoint{2.927349in}{0.631819in}}%
\pgfpathlineto{\pgfqpoint{2.957366in}{0.626717in}}%
\pgfpathlineto{\pgfqpoint{2.981889in}{0.622542in}}%
\pgfpathlineto{\pgfqpoint{2.987384in}{0.621625in}}%
\pgfpathlineto{\pgfqpoint{3.017401in}{0.616616in}}%
\pgfpathlineto{\pgfqpoint{3.047418in}{0.611597in}}%
\pgfpathlineto{\pgfqpoint{3.077436in}{0.606570in}}%
\pgfpathlineto{\pgfqpoint{3.107453in}{0.601533in}}%
\pgfpathlineto{\pgfqpoint{3.133988in}{0.597073in}}%
\pgfpathlineto{\pgfqpoint{3.137471in}{0.596499in}}%
\pgfpathlineto{\pgfqpoint{3.167488in}{0.591554in}}%
\pgfpathlineto{\pgfqpoint{3.197505in}{0.586599in}}%
\pgfpathlineto{\pgfqpoint{3.227523in}{0.581635in}}%
\pgfpathlineto{\pgfqpoint{3.257540in}{0.576662in}}%
\pgfpathlineto{\pgfqpoint{3.287558in}{0.571678in}}%
\pgfpathlineto{\pgfqpoint{3.288007in}{0.571603in}}%
\pgfusepath{stroke}%
\end{pgfscope}%
\begin{pgfscope}%
\pgfpathrectangle{\pgfqpoint{0.766095in}{0.571603in}}{\pgfqpoint{5.973465in}{5.068436in}}%
\pgfusepath{clip}%
\pgfsetbuttcap%
\pgfsetroundjoin%
\pgfsetlinewidth{1.505625pt}%
\definecolor{currentstroke}{rgb}{0.170948,0.694384,0.493803}%
\pgfsetstrokecolor{currentstroke}%
\pgfsetdash{}{0pt}%
\pgfpathmoveto{\pgfqpoint{0.766095in}{1.004202in}}%
\pgfpathlineto{\pgfqpoint{0.796112in}{0.997621in}}%
\pgfpathlineto{\pgfqpoint{0.826130in}{0.991041in}}%
\pgfpathlineto{\pgfqpoint{0.856147in}{0.984462in}}%
\pgfpathlineto{\pgfqpoint{0.880551in}{0.979116in}}%
\pgfpathlineto{\pgfqpoint{0.886165in}{0.977908in}}%
\pgfpathlineto{\pgfqpoint{0.916182in}{0.971468in}}%
\pgfpathlineto{\pgfqpoint{0.946199in}{0.965028in}}%
\pgfpathlineto{\pgfqpoint{0.976217in}{0.958587in}}%
\pgfpathlineto{\pgfqpoint{0.999254in}{0.953646in}}%
\pgfpathlineto{\pgfqpoint{1.006234in}{0.952176in}}%
\pgfpathlineto{\pgfqpoint{1.036252in}{0.945870in}}%
\pgfpathlineto{\pgfqpoint{1.066269in}{0.939562in}}%
\pgfpathlineto{\pgfqpoint{1.096287in}{0.933253in}}%
\pgfpathlineto{\pgfqpoint{1.120446in}{0.928177in}}%
\pgfpathlineto{\pgfqpoint{1.126304in}{0.926968in}}%
\pgfpathlineto{\pgfqpoint{1.156321in}{0.920789in}}%
\pgfpathlineto{\pgfqpoint{1.186339in}{0.914608in}}%
\pgfpathlineto{\pgfqpoint{1.216356in}{0.908425in}}%
\pgfpathlineto{\pgfqpoint{1.244107in}{0.902707in}}%
\pgfpathlineto{\pgfqpoint{1.246374in}{0.902249in}}%
\pgfpathlineto{\pgfqpoint{1.276391in}{0.896191in}}%
\pgfpathlineto{\pgfqpoint{1.306408in}{0.890131in}}%
\pgfpathlineto{\pgfqpoint{1.336426in}{0.884068in}}%
\pgfpathlineto{\pgfqpoint{1.366443in}{0.878001in}}%
\pgfpathlineto{\pgfqpoint{1.370230in}{0.877238in}}%
\pgfpathlineto{\pgfqpoint{1.396461in}{0.872044in}}%
\pgfpathlineto{\pgfqpoint{1.426478in}{0.866099in}}%
\pgfpathlineto{\pgfqpoint{1.456496in}{0.860151in}}%
\pgfpathlineto{\pgfqpoint{1.486513in}{0.854199in}}%
\pgfpathlineto{\pgfqpoint{1.498782in}{0.851768in}}%
\pgfpathlineto{\pgfqpoint{1.516530in}{0.848317in}}%
\pgfpathlineto{\pgfqpoint{1.546548in}{0.842483in}}%
\pgfpathlineto{\pgfqpoint{1.576565in}{0.836644in}}%
\pgfpathlineto{\pgfqpoint{1.606583in}{0.830801in}}%
\pgfpathlineto{\pgfqpoint{1.629712in}{0.826299in}}%
\pgfpathlineto{\pgfqpoint{1.636600in}{0.824982in}}%
\pgfpathlineto{\pgfqpoint{1.666617in}{0.819254in}}%
\pgfpathlineto{\pgfqpoint{1.696635in}{0.813521in}}%
\pgfpathlineto{\pgfqpoint{1.726652in}{0.807783in}}%
\pgfpathlineto{\pgfqpoint{1.756670in}{0.802040in}}%
\pgfpathlineto{\pgfqpoint{1.763005in}{0.800829in}}%
\pgfpathlineto{\pgfqpoint{1.786687in}{0.796387in}}%
\pgfpathlineto{\pgfqpoint{1.816704in}{0.790756in}}%
\pgfpathlineto{\pgfqpoint{1.846722in}{0.785118in}}%
\pgfpathlineto{\pgfqpoint{1.876739in}{0.779475in}}%
\pgfpathlineto{\pgfqpoint{1.898625in}{0.775360in}}%
\pgfpathlineto{\pgfqpoint{1.906757in}{0.773859in}}%
\pgfpathlineto{\pgfqpoint{1.936774in}{0.768324in}}%
\pgfpathlineto{\pgfqpoint{1.966792in}{0.762783in}}%
\pgfpathlineto{\pgfqpoint{1.996809in}{0.757237in}}%
\pgfpathlineto{\pgfqpoint{2.026826in}{0.751683in}}%
\pgfpathlineto{\pgfqpoint{2.036528in}{0.749890in}}%
\pgfpathlineto{\pgfqpoint{2.056844in}{0.746204in}}%
\pgfpathlineto{\pgfqpoint{2.086861in}{0.740757in}}%
\pgfpathlineto{\pgfqpoint{2.116879in}{0.735303in}}%
\pgfpathlineto{\pgfqpoint{2.146896in}{0.729842in}}%
\pgfpathlineto{\pgfqpoint{2.176659in}{0.724420in}}%
\pgfpathlineto{\pgfqpoint{2.176913in}{0.724375in}}%
\pgfpathlineto{\pgfqpoint{2.206931in}{0.719017in}}%
\pgfpathlineto{\pgfqpoint{2.236948in}{0.713653in}}%
\pgfpathlineto{\pgfqpoint{2.266966in}{0.708281in}}%
\pgfpathlineto{\pgfqpoint{2.296983in}{0.702902in}}%
\pgfpathlineto{\pgfqpoint{2.319020in}{0.698951in}}%
\pgfpathlineto{\pgfqpoint{2.327000in}{0.697547in}}%
\pgfpathlineto{\pgfqpoint{2.357018in}{0.692268in}}%
\pgfpathlineto{\pgfqpoint{2.387035in}{0.686983in}}%
\pgfpathlineto{\pgfqpoint{2.417053in}{0.681690in}}%
\pgfpathlineto{\pgfqpoint{2.447070in}{0.676389in}}%
\pgfpathlineto{\pgfqpoint{2.463531in}{0.673481in}}%
\pgfpathlineto{\pgfqpoint{2.477088in}{0.671132in}}%
\pgfpathlineto{\pgfqpoint{2.507105in}{0.665929in}}%
\pgfpathlineto{\pgfqpoint{2.537122in}{0.660719in}}%
\pgfpathlineto{\pgfqpoint{2.567140in}{0.655501in}}%
\pgfpathlineto{\pgfqpoint{2.597157in}{0.650275in}}%
\pgfpathlineto{\pgfqpoint{2.610154in}{0.648012in}}%
\pgfpathlineto{\pgfqpoint{2.627175in}{0.645104in}}%
\pgfpathlineto{\pgfqpoint{2.657192in}{0.639974in}}%
\pgfpathlineto{\pgfqpoint{2.687209in}{0.634836in}}%
\pgfpathlineto{\pgfqpoint{2.717227in}{0.629690in}}%
\pgfpathlineto{\pgfqpoint{2.747244in}{0.624535in}}%
\pgfpathlineto{\pgfqpoint{2.758847in}{0.622542in}}%
\pgfpathlineto{\pgfqpoint{2.777262in}{0.619440in}}%
\pgfpathlineto{\pgfqpoint{2.807279in}{0.614379in}}%
\pgfpathlineto{\pgfqpoint{2.837297in}{0.609310in}}%
\pgfpathlineto{\pgfqpoint{2.867314in}{0.604233in}}%
\pgfpathlineto{\pgfqpoint{2.897331in}{0.599146in}}%
\pgfpathlineto{\pgfqpoint{2.909567in}{0.597073in}}%
\pgfpathlineto{\pgfqpoint{2.927349in}{0.594116in}}%
\pgfpathlineto{\pgfqpoint{2.957366in}{0.589123in}}%
\pgfpathlineto{\pgfqpoint{2.987384in}{0.584120in}}%
\pgfpathlineto{\pgfqpoint{3.017401in}{0.579109in}}%
\pgfpathlineto{\pgfqpoint{3.047418in}{0.574088in}}%
\pgfpathlineto{\pgfqpoint{3.062270in}{0.571603in}}%
\pgfusepath{stroke}%
\end{pgfscope}%
\begin{pgfscope}%
\pgfpathrectangle{\pgfqpoint{0.766095in}{0.571603in}}{\pgfqpoint{5.973465in}{5.068436in}}%
\pgfusepath{clip}%
\pgfsetbuttcap%
\pgfsetroundjoin%
\pgfsetlinewidth{1.505625pt}%
\definecolor{currentstroke}{rgb}{0.208030,0.718701,0.472873}%
\pgfsetstrokecolor{currentstroke}%
\pgfsetdash{}{0pt}%
\pgfpathmoveto{\pgfqpoint{0.766095in}{0.962695in}}%
\pgfpathlineto{\pgfqpoint{0.796112in}{0.956130in}}%
\pgfpathlineto{\pgfqpoint{0.807491in}{0.953646in}}%
\pgfpathlineto{\pgfqpoint{0.826130in}{0.949651in}}%
\pgfpathlineto{\pgfqpoint{0.856147in}{0.943224in}}%
\pgfpathlineto{\pgfqpoint{0.886165in}{0.936798in}}%
\pgfpathlineto{\pgfqpoint{0.916182in}{0.930372in}}%
\pgfpathlineto{\pgfqpoint{0.926457in}{0.928177in}}%
\pgfpathlineto{\pgfqpoint{0.946199in}{0.924034in}}%
\pgfpathlineto{\pgfqpoint{0.976217in}{0.917742in}}%
\pgfpathlineto{\pgfqpoint{1.006234in}{0.911449in}}%
\pgfpathlineto{\pgfqpoint{1.036252in}{0.905156in}}%
\pgfpathlineto{\pgfqpoint{1.047948in}{0.902707in}}%
\pgfpathlineto{\pgfqpoint{1.066269in}{0.898941in}}%
\pgfpathlineto{\pgfqpoint{1.096287in}{0.892777in}}%
\pgfpathlineto{\pgfqpoint{1.126304in}{0.886611in}}%
\pgfpathlineto{\pgfqpoint{1.156321in}{0.880443in}}%
\pgfpathlineto{\pgfqpoint{1.171941in}{0.877238in}}%
\pgfpathlineto{\pgfqpoint{1.186339in}{0.874336in}}%
\pgfpathlineto{\pgfqpoint{1.216356in}{0.868294in}}%
\pgfpathlineto{\pgfqpoint{1.246374in}{0.862249in}}%
\pgfpathlineto{\pgfqpoint{1.276391in}{0.856202in}}%
\pgfpathlineto{\pgfqpoint{1.298408in}{0.851768in}}%
\pgfpathlineto{\pgfqpoint{1.306408in}{0.850186in}}%
\pgfpathlineto{\pgfqpoint{1.336426in}{0.844260in}}%
\pgfpathlineto{\pgfqpoint{1.366443in}{0.838332in}}%
\pgfpathlineto{\pgfqpoint{1.396461in}{0.832400in}}%
\pgfpathlineto{\pgfqpoint{1.426478in}{0.826464in}}%
\pgfpathlineto{\pgfqpoint{1.427318in}{0.826299in}}%
\pgfpathlineto{\pgfqpoint{1.456496in}{0.820647in}}%
\pgfpathlineto{\pgfqpoint{1.486513in}{0.814829in}}%
\pgfpathlineto{\pgfqpoint{1.516530in}{0.809007in}}%
\pgfpathlineto{\pgfqpoint{1.546548in}{0.803181in}}%
\pgfpathlineto{\pgfqpoint{1.558679in}{0.800829in}}%
\pgfpathlineto{\pgfqpoint{1.576565in}{0.797424in}}%
\pgfpathlineto{\pgfqpoint{1.606583in}{0.791712in}}%
\pgfpathlineto{\pgfqpoint{1.636600in}{0.785996in}}%
\pgfpathlineto{\pgfqpoint{1.666617in}{0.780276in}}%
\pgfpathlineto{\pgfqpoint{1.692401in}{0.775360in}}%
\pgfpathlineto{\pgfqpoint{1.696635in}{0.774567in}}%
\pgfpathlineto{\pgfqpoint{1.726652in}{0.768957in}}%
\pgfpathlineto{\pgfqpoint{1.756670in}{0.763342in}}%
\pgfpathlineto{\pgfqpoint{1.786687in}{0.757722in}}%
\pgfpathlineto{\pgfqpoint{1.816704in}{0.752097in}}%
\pgfpathlineto{\pgfqpoint{1.828490in}{0.749890in}}%
\pgfpathlineto{\pgfqpoint{1.846722in}{0.746538in}}%
\pgfpathlineto{\pgfqpoint{1.876739in}{0.741021in}}%
\pgfpathlineto{\pgfqpoint{1.906757in}{0.735498in}}%
\pgfpathlineto{\pgfqpoint{1.936774in}{0.729969in}}%
\pgfpathlineto{\pgfqpoint{1.966792in}{0.724434in}}%
\pgfpathlineto{\pgfqpoint{1.966863in}{0.724420in}}%
\pgfpathlineto{\pgfqpoint{1.996809in}{0.719009in}}%
\pgfpathlineto{\pgfqpoint{2.026826in}{0.713580in}}%
\pgfpathlineto{\pgfqpoint{2.056844in}{0.708144in}}%
\pgfpathlineto{\pgfqpoint{2.086861in}{0.702701in}}%
\pgfpathlineto{\pgfqpoint{2.107538in}{0.698951in}}%
\pgfpathlineto{\pgfqpoint{2.116879in}{0.697288in}}%
\pgfpathlineto{\pgfqpoint{2.146896in}{0.691948in}}%
\pgfpathlineto{\pgfqpoint{2.176913in}{0.686601in}}%
\pgfpathlineto{\pgfqpoint{2.206931in}{0.681248in}}%
\pgfpathlineto{\pgfqpoint{2.236948in}{0.675887in}}%
\pgfpathlineto{\pgfqpoint{2.250425in}{0.673481in}}%
\pgfpathlineto{\pgfqpoint{2.266966in}{0.670583in}}%
\pgfpathlineto{\pgfqpoint{2.296983in}{0.665322in}}%
\pgfpathlineto{\pgfqpoint{2.327000in}{0.660055in}}%
\pgfpathlineto{\pgfqpoint{2.357018in}{0.654780in}}%
\pgfpathlineto{\pgfqpoint{2.387035in}{0.649498in}}%
\pgfpathlineto{\pgfqpoint{2.395489in}{0.648012in}}%
\pgfpathlineto{\pgfqpoint{2.417053in}{0.644290in}}%
\pgfpathlineto{\pgfqpoint{2.447070in}{0.639106in}}%
\pgfpathlineto{\pgfqpoint{2.477088in}{0.633914in}}%
\pgfpathlineto{\pgfqpoint{2.507105in}{0.628714in}}%
\pgfpathlineto{\pgfqpoint{2.537122in}{0.623507in}}%
\pgfpathlineto{\pgfqpoint{2.542689in}{0.622542in}}%
\pgfpathlineto{\pgfqpoint{2.567140in}{0.618383in}}%
\pgfpathlineto{\pgfqpoint{2.597157in}{0.613271in}}%
\pgfpathlineto{\pgfqpoint{2.627175in}{0.608152in}}%
\pgfpathlineto{\pgfqpoint{2.657192in}{0.603024in}}%
\pgfpathlineto{\pgfqpoint{2.687209in}{0.597889in}}%
\pgfpathlineto{\pgfqpoint{2.691982in}{0.597073in}}%
\pgfpathlineto{\pgfqpoint{2.717227in}{0.592837in}}%
\pgfpathlineto{\pgfqpoint{2.747244in}{0.587795in}}%
\pgfpathlineto{\pgfqpoint{2.777262in}{0.582745in}}%
\pgfpathlineto{\pgfqpoint{2.807279in}{0.577687in}}%
\pgfpathlineto{\pgfqpoint{2.837297in}{0.572620in}}%
\pgfpathlineto{\pgfqpoint{2.843323in}{0.571603in}}%
\pgfusepath{stroke}%
\end{pgfscope}%
\begin{pgfscope}%
\pgfpathrectangle{\pgfqpoint{0.766095in}{0.571603in}}{\pgfqpoint{5.973465in}{5.068436in}}%
\pgfusepath{clip}%
\pgfsetbuttcap%
\pgfsetroundjoin%
\pgfsetlinewidth{1.505625pt}%
\definecolor{currentstroke}{rgb}{0.252899,0.742211,0.448284}%
\pgfsetstrokecolor{currentstroke}%
\pgfsetdash{}{0pt}%
\pgfpathmoveto{\pgfqpoint{0.766095in}{0.922499in}}%
\pgfpathlineto{\pgfqpoint{0.796112in}{0.916089in}}%
\pgfpathlineto{\pgfqpoint{0.826130in}{0.909681in}}%
\pgfpathlineto{\pgfqpoint{0.856147in}{0.903272in}}%
\pgfpathlineto{\pgfqpoint{0.858801in}{0.902707in}}%
\pgfpathlineto{\pgfqpoint{0.886165in}{0.896985in}}%
\pgfpathlineto{\pgfqpoint{0.916182in}{0.890710in}}%
\pgfpathlineto{\pgfqpoint{0.946199in}{0.884435in}}%
\pgfpathlineto{\pgfqpoint{0.976217in}{0.878159in}}%
\pgfpathlineto{\pgfqpoint{0.980636in}{0.877238in}}%
\pgfpathlineto{\pgfqpoint{1.006234in}{0.871994in}}%
\pgfpathlineto{\pgfqpoint{1.036252in}{0.865847in}}%
\pgfpathlineto{\pgfqpoint{1.066269in}{0.859700in}}%
\pgfpathlineto{\pgfqpoint{1.096287in}{0.853551in}}%
\pgfpathlineto{\pgfqpoint{1.105005in}{0.851768in}}%
\pgfpathlineto{\pgfqpoint{1.126304in}{0.847491in}}%
\pgfpathlineto{\pgfqpoint{1.156321in}{0.841467in}}%
\pgfpathlineto{\pgfqpoint{1.186339in}{0.835441in}}%
\pgfpathlineto{\pgfqpoint{1.216356in}{0.829413in}}%
\pgfpathlineto{\pgfqpoint{1.231878in}{0.826299in}}%
\pgfpathlineto{\pgfqpoint{1.246374in}{0.823443in}}%
\pgfpathlineto{\pgfqpoint{1.276391in}{0.817535in}}%
\pgfpathlineto{\pgfqpoint{1.306408in}{0.811625in}}%
\pgfpathlineto{\pgfqpoint{1.336426in}{0.805713in}}%
\pgfpathlineto{\pgfqpoint{1.361219in}{0.800829in}}%
\pgfpathlineto{\pgfqpoint{1.366443in}{0.799818in}}%
\pgfpathlineto{\pgfqpoint{1.396461in}{0.794023in}}%
\pgfpathlineto{\pgfqpoint{1.426478in}{0.788224in}}%
\pgfpathlineto{\pgfqpoint{1.456496in}{0.782422in}}%
\pgfpathlineto{\pgfqpoint{1.486513in}{0.776616in}}%
\pgfpathlineto{\pgfqpoint{1.493017in}{0.775360in}}%
\pgfpathlineto{\pgfqpoint{1.516530in}{0.770901in}}%
\pgfpathlineto{\pgfqpoint{1.546548in}{0.765208in}}%
\pgfpathlineto{\pgfqpoint{1.576565in}{0.759512in}}%
\pgfpathlineto{\pgfqpoint{1.606583in}{0.753811in}}%
\pgfpathlineto{\pgfqpoint{1.627231in}{0.749890in}}%
\pgfpathlineto{\pgfqpoint{1.636600in}{0.748143in}}%
\pgfpathlineto{\pgfqpoint{1.666617in}{0.742552in}}%
\pgfpathlineto{\pgfqpoint{1.696635in}{0.736957in}}%
\pgfpathlineto{\pgfqpoint{1.726652in}{0.731357in}}%
\pgfpathlineto{\pgfqpoint{1.756670in}{0.725753in}}%
\pgfpathlineto{\pgfqpoint{1.763813in}{0.724420in}}%
\pgfpathlineto{\pgfqpoint{1.786687in}{0.720232in}}%
\pgfpathlineto{\pgfqpoint{1.816704in}{0.714734in}}%
\pgfpathlineto{\pgfqpoint{1.846722in}{0.709231in}}%
\pgfpathlineto{\pgfqpoint{1.876739in}{0.703723in}}%
\pgfpathlineto{\pgfqpoint{1.902729in}{0.698951in}}%
\pgfpathlineto{\pgfqpoint{1.906757in}{0.698225in}}%
\pgfpathlineto{\pgfqpoint{1.936774in}{0.692821in}}%
\pgfpathlineto{\pgfqpoint{1.966792in}{0.687411in}}%
\pgfpathlineto{\pgfqpoint{1.996809in}{0.681996in}}%
\pgfpathlineto{\pgfqpoint{2.026826in}{0.676574in}}%
\pgfpathlineto{\pgfqpoint{2.043951in}{0.673481in}}%
\pgfpathlineto{\pgfqpoint{2.056844in}{0.671195in}}%
\pgfpathlineto{\pgfqpoint{2.086861in}{0.665875in}}%
\pgfpathlineto{\pgfqpoint{2.116879in}{0.660549in}}%
\pgfpathlineto{\pgfqpoint{2.146896in}{0.655217in}}%
\pgfpathlineto{\pgfqpoint{2.176913in}{0.649878in}}%
\pgfpathlineto{\pgfqpoint{2.187411in}{0.648012in}}%
\pgfpathlineto{\pgfqpoint{2.206931in}{0.644605in}}%
\pgfpathlineto{\pgfqpoint{2.236948in}{0.639366in}}%
\pgfpathlineto{\pgfqpoint{2.266966in}{0.634119in}}%
\pgfpathlineto{\pgfqpoint{2.296983in}{0.628866in}}%
\pgfpathlineto{\pgfqpoint{2.327000in}{0.623605in}}%
\pgfpathlineto{\pgfqpoint{2.333072in}{0.622542in}}%
\pgfpathlineto{\pgfqpoint{2.357018in}{0.618426in}}%
\pgfpathlineto{\pgfqpoint{2.387035in}{0.613263in}}%
\pgfpathlineto{\pgfqpoint{2.417053in}{0.608092in}}%
\pgfpathlineto{\pgfqpoint{2.447070in}{0.602914in}}%
\pgfpathlineto{\pgfqpoint{2.477088in}{0.597729in}}%
\pgfpathlineto{\pgfqpoint{2.480890in}{0.597073in}}%
\pgfpathlineto{\pgfqpoint{2.507105in}{0.592632in}}%
\pgfpathlineto{\pgfqpoint{2.537122in}{0.587541in}}%
\pgfpathlineto{\pgfqpoint{2.567140in}{0.582443in}}%
\pgfpathlineto{\pgfqpoint{2.597157in}{0.577338in}}%
\pgfpathlineto{\pgfqpoint{2.627175in}{0.572224in}}%
\pgfpathlineto{\pgfqpoint{2.630821in}{0.571603in}}%
\pgfusepath{stroke}%
\end{pgfscope}%
\begin{pgfscope}%
\pgfpathrectangle{\pgfqpoint{0.766095in}{0.571603in}}{\pgfqpoint{5.973465in}{5.068436in}}%
\pgfusepath{clip}%
\pgfsetbuttcap%
\pgfsetroundjoin%
\pgfsetlinewidth{1.505625pt}%
\definecolor{currentstroke}{rgb}{0.304148,0.764704,0.419943}%
\pgfsetstrokecolor{currentstroke}%
\pgfsetdash{}{0pt}%
\pgfpathmoveto{\pgfqpoint{0.766095in}{0.883604in}}%
\pgfpathlineto{\pgfqpoint{0.796010in}{0.877238in}}%
\pgfpathlineto{\pgfqpoint{0.796112in}{0.877216in}}%
\pgfpathlineto{\pgfqpoint{0.826130in}{0.870960in}}%
\pgfpathlineto{\pgfqpoint{0.856147in}{0.864704in}}%
\pgfpathlineto{\pgfqpoint{0.886165in}{0.858449in}}%
\pgfpathlineto{\pgfqpoint{0.916182in}{0.852194in}}%
\pgfpathlineto{\pgfqpoint{0.918234in}{0.851768in}}%
\pgfpathlineto{\pgfqpoint{0.946199in}{0.846059in}}%
\pgfpathlineto{\pgfqpoint{0.976217in}{0.839932in}}%
\pgfpathlineto{\pgfqpoint{1.006234in}{0.833805in}}%
\pgfpathlineto{\pgfqpoint{1.036252in}{0.827677in}}%
\pgfpathlineto{\pgfqpoint{1.043019in}{0.826299in}}%
\pgfpathlineto{\pgfqpoint{1.066269in}{0.821646in}}%
\pgfpathlineto{\pgfqpoint{1.096287in}{0.815642in}}%
\pgfpathlineto{\pgfqpoint{1.126304in}{0.809637in}}%
\pgfpathlineto{\pgfqpoint{1.156321in}{0.803630in}}%
\pgfpathlineto{\pgfqpoint{1.170336in}{0.800829in}}%
\pgfpathlineto{\pgfqpoint{1.186339in}{0.797688in}}%
\pgfpathlineto{\pgfqpoint{1.216356in}{0.791801in}}%
\pgfpathlineto{\pgfqpoint{1.246374in}{0.785912in}}%
\pgfpathlineto{\pgfqpoint{1.276391in}{0.780021in}}%
\pgfpathlineto{\pgfqpoint{1.300147in}{0.775360in}}%
\pgfpathlineto{\pgfqpoint{1.306408in}{0.774153in}}%
\pgfpathlineto{\pgfqpoint{1.336426in}{0.768378in}}%
\pgfpathlineto{\pgfqpoint{1.366443in}{0.762600in}}%
\pgfpathlineto{\pgfqpoint{1.396461in}{0.756820in}}%
\pgfpathlineto{\pgfqpoint{1.426478in}{0.751036in}}%
\pgfpathlineto{\pgfqpoint{1.432438in}{0.749890in}}%
\pgfpathlineto{\pgfqpoint{1.456496in}{0.745345in}}%
\pgfpathlineto{\pgfqpoint{1.486513in}{0.739674in}}%
\pgfpathlineto{\pgfqpoint{1.516530in}{0.734000in}}%
\pgfpathlineto{\pgfqpoint{1.546548in}{0.728321in}}%
\pgfpathlineto{\pgfqpoint{1.567173in}{0.724420in}}%
\pgfpathlineto{\pgfqpoint{1.576565in}{0.722676in}}%
\pgfpathlineto{\pgfqpoint{1.606583in}{0.717107in}}%
\pgfpathlineto{\pgfqpoint{1.636600in}{0.711534in}}%
\pgfpathlineto{\pgfqpoint{1.666617in}{0.705957in}}%
\pgfpathlineto{\pgfqpoint{1.696635in}{0.700375in}}%
\pgfpathlineto{\pgfqpoint{1.704302in}{0.698951in}}%
\pgfpathlineto{\pgfqpoint{1.726652in}{0.694874in}}%
\pgfpathlineto{\pgfqpoint{1.756670in}{0.689399in}}%
\pgfpathlineto{\pgfqpoint{1.786687in}{0.683919in}}%
\pgfpathlineto{\pgfqpoint{1.816704in}{0.678433in}}%
\pgfpathlineto{\pgfqpoint{1.843784in}{0.673481in}}%
\pgfpathlineto{\pgfqpoint{1.846722in}{0.672954in}}%
\pgfpathlineto{\pgfqpoint{1.876739in}{0.667572in}}%
\pgfpathlineto{\pgfqpoint{1.906757in}{0.662185in}}%
\pgfpathlineto{\pgfqpoint{1.936774in}{0.656792in}}%
\pgfpathlineto{\pgfqpoint{1.966792in}{0.651394in}}%
\pgfpathlineto{\pgfqpoint{1.985600in}{0.648012in}}%
\pgfpathlineto{\pgfqpoint{1.996809in}{0.646032in}}%
\pgfpathlineto{\pgfqpoint{2.026826in}{0.640735in}}%
\pgfpathlineto{\pgfqpoint{2.056844in}{0.635432in}}%
\pgfpathlineto{\pgfqpoint{2.086861in}{0.630123in}}%
\pgfpathlineto{\pgfqpoint{2.116879in}{0.624808in}}%
\pgfpathlineto{\pgfqpoint{2.129677in}{0.622542in}}%
\pgfpathlineto{\pgfqpoint{2.146896in}{0.619550in}}%
\pgfpathlineto{\pgfqpoint{2.176913in}{0.614333in}}%
\pgfpathlineto{\pgfqpoint{2.206931in}{0.609109in}}%
\pgfpathlineto{\pgfqpoint{2.236948in}{0.603880in}}%
\pgfpathlineto{\pgfqpoint{2.266966in}{0.598643in}}%
\pgfpathlineto{\pgfqpoint{2.275974in}{0.597073in}}%
\pgfpathlineto{\pgfqpoint{2.296983in}{0.593477in}}%
\pgfpathlineto{\pgfqpoint{2.327000in}{0.588336in}}%
\pgfpathlineto{\pgfqpoint{2.357018in}{0.583189in}}%
\pgfpathlineto{\pgfqpoint{2.387035in}{0.578035in}}%
\pgfpathlineto{\pgfqpoint{2.417053in}{0.572874in}}%
\pgfpathlineto{\pgfqpoint{2.424449in}{0.571603in}}%
\pgfusepath{stroke}%
\end{pgfscope}%
\begin{pgfscope}%
\pgfpathrectangle{\pgfqpoint{0.766095in}{0.571603in}}{\pgfqpoint{5.973465in}{5.068436in}}%
\pgfusepath{clip}%
\pgfsetbuttcap%
\pgfsetroundjoin%
\pgfsetlinewidth{1.505625pt}%
\definecolor{currentstroke}{rgb}{0.369214,0.788888,0.382914}%
\pgfsetstrokecolor{currentstroke}%
\pgfsetdash{}{0pt}%
\pgfpathmoveto{\pgfqpoint{0.766095in}{0.845894in}}%
\pgfpathlineto{\pgfqpoint{0.796112in}{0.839661in}}%
\pgfpathlineto{\pgfqpoint{0.826130in}{0.833428in}}%
\pgfpathlineto{\pgfqpoint{0.856147in}{0.827195in}}%
\pgfpathlineto{\pgfqpoint{0.860478in}{0.826299in}}%
\pgfpathlineto{\pgfqpoint{0.886165in}{0.821072in}}%
\pgfpathlineto{\pgfqpoint{0.916182in}{0.814967in}}%
\pgfpathlineto{\pgfqpoint{0.946199in}{0.808863in}}%
\pgfpathlineto{\pgfqpoint{0.976217in}{0.802758in}}%
\pgfpathlineto{\pgfqpoint{0.985720in}{0.800829in}}%
\pgfpathlineto{\pgfqpoint{1.006234in}{0.796738in}}%
\pgfpathlineto{\pgfqpoint{1.036252in}{0.790756in}}%
\pgfpathlineto{\pgfqpoint{1.066269in}{0.784774in}}%
\pgfpathlineto{\pgfqpoint{1.096287in}{0.778791in}}%
\pgfpathlineto{\pgfqpoint{1.113517in}{0.775360in}}%
\pgfpathlineto{\pgfqpoint{1.126304in}{0.772858in}}%
\pgfpathlineto{\pgfqpoint{1.156321in}{0.766994in}}%
\pgfpathlineto{\pgfqpoint{1.186339in}{0.761128in}}%
\pgfpathlineto{\pgfqpoint{1.216356in}{0.755261in}}%
\pgfpathlineto{\pgfqpoint{1.243832in}{0.749890in}}%
\pgfpathlineto{\pgfqpoint{1.246374in}{0.749402in}}%
\pgfpathlineto{\pgfqpoint{1.276391in}{0.743650in}}%
\pgfpathlineto{\pgfqpoint{1.306408in}{0.737896in}}%
\pgfpathlineto{\pgfqpoint{1.336426in}{0.732139in}}%
\pgfpathlineto{\pgfqpoint{1.366443in}{0.726380in}}%
\pgfpathlineto{\pgfqpoint{1.376671in}{0.724420in}}%
\pgfpathlineto{\pgfqpoint{1.396461in}{0.720696in}}%
\pgfpathlineto{\pgfqpoint{1.426478in}{0.715048in}}%
\pgfpathlineto{\pgfqpoint{1.456496in}{0.709398in}}%
\pgfpathlineto{\pgfqpoint{1.486513in}{0.703744in}}%
\pgfpathlineto{\pgfqpoint{1.511956in}{0.698951in}}%
\pgfpathlineto{\pgfqpoint{1.516530in}{0.698104in}}%
\pgfpathlineto{\pgfqpoint{1.546548in}{0.692559in}}%
\pgfpathlineto{\pgfqpoint{1.576565in}{0.687010in}}%
\pgfpathlineto{\pgfqpoint{1.606583in}{0.681458in}}%
\pgfpathlineto{\pgfqpoint{1.636600in}{0.675901in}}%
\pgfpathlineto{\pgfqpoint{1.649680in}{0.673481in}}%
\pgfpathlineto{\pgfqpoint{1.666617in}{0.670404in}}%
\pgfpathlineto{\pgfqpoint{1.696635in}{0.664953in}}%
\pgfpathlineto{\pgfqpoint{1.726652in}{0.659497in}}%
\pgfpathlineto{\pgfqpoint{1.756670in}{0.654037in}}%
\pgfpathlineto{\pgfqpoint{1.786687in}{0.648572in}}%
\pgfpathlineto{\pgfqpoint{1.789768in}{0.648012in}}%
\pgfpathlineto{\pgfqpoint{1.816704in}{0.643203in}}%
\pgfpathlineto{\pgfqpoint{1.846722in}{0.637840in}}%
\pgfpathlineto{\pgfqpoint{1.876739in}{0.632473in}}%
\pgfpathlineto{\pgfqpoint{1.906757in}{0.627100in}}%
\pgfpathlineto{\pgfqpoint{1.932207in}{0.622542in}}%
\pgfpathlineto{\pgfqpoint{1.936774in}{0.621739in}}%
\pgfpathlineto{\pgfqpoint{1.966792in}{0.616466in}}%
\pgfpathlineto{\pgfqpoint{1.996809in}{0.611188in}}%
\pgfpathlineto{\pgfqpoint{2.026826in}{0.605904in}}%
\pgfpathlineto{\pgfqpoint{2.056844in}{0.600615in}}%
\pgfpathlineto{\pgfqpoint{2.076940in}{0.597073in}}%
\pgfpathlineto{\pgfqpoint{2.086861in}{0.595356in}}%
\pgfpathlineto{\pgfqpoint{2.116879in}{0.590163in}}%
\pgfpathlineto{\pgfqpoint{2.146896in}{0.584965in}}%
\pgfpathlineto{\pgfqpoint{2.176913in}{0.579761in}}%
\pgfpathlineto{\pgfqpoint{2.206931in}{0.574551in}}%
\pgfpathlineto{\pgfqpoint{2.223913in}{0.571603in}}%
\pgfusepath{stroke}%
\end{pgfscope}%
\begin{pgfscope}%
\pgfpathrectangle{\pgfqpoint{0.766095in}{0.571603in}}{\pgfqpoint{5.973465in}{5.068436in}}%
\pgfusepath{clip}%
\pgfsetbuttcap%
\pgfsetroundjoin%
\pgfsetlinewidth{1.505625pt}%
\definecolor{currentstroke}{rgb}{0.430983,0.808473,0.346476}%
\pgfsetstrokecolor{currentstroke}%
\pgfsetdash{}{0pt}%
\pgfpathmoveto{\pgfqpoint{0.766095in}{0.809309in}}%
\pgfpathlineto{\pgfqpoint{0.796112in}{0.803102in}}%
\pgfpathlineto{\pgfqpoint{0.807122in}{0.800829in}}%
\pgfpathlineto{\pgfqpoint{0.826130in}{0.796974in}}%
\pgfpathlineto{\pgfqpoint{0.856147in}{0.790893in}}%
\pgfpathlineto{\pgfqpoint{0.886165in}{0.784813in}}%
\pgfpathlineto{\pgfqpoint{0.916182in}{0.778733in}}%
\pgfpathlineto{\pgfqpoint{0.932858in}{0.775360in}}%
\pgfpathlineto{\pgfqpoint{0.946199in}{0.772708in}}%
\pgfpathlineto{\pgfqpoint{0.976217in}{0.766750in}}%
\pgfpathlineto{\pgfqpoint{1.006234in}{0.760793in}}%
\pgfpathlineto{\pgfqpoint{1.036252in}{0.754835in}}%
\pgfpathlineto{\pgfqpoint{1.061171in}{0.749890in}}%
\pgfpathlineto{\pgfqpoint{1.066269in}{0.748896in}}%
\pgfpathlineto{\pgfqpoint{1.096287in}{0.743056in}}%
\pgfpathlineto{\pgfqpoint{1.126304in}{0.737216in}}%
\pgfpathlineto{\pgfqpoint{1.156321in}{0.731374in}}%
\pgfpathlineto{\pgfqpoint{1.186339in}{0.725531in}}%
\pgfpathlineto{\pgfqpoint{1.192052in}{0.724420in}}%
\pgfpathlineto{\pgfqpoint{1.216356in}{0.719781in}}%
\pgfpathlineto{\pgfqpoint{1.246374in}{0.714052in}}%
\pgfpathlineto{\pgfqpoint{1.276391in}{0.708321in}}%
\pgfpathlineto{\pgfqpoint{1.306408in}{0.702588in}}%
\pgfpathlineto{\pgfqpoint{1.325462in}{0.698951in}}%
\pgfpathlineto{\pgfqpoint{1.336426in}{0.696895in}}%
\pgfpathlineto{\pgfqpoint{1.366443in}{0.691273in}}%
\pgfpathlineto{\pgfqpoint{1.396461in}{0.685648in}}%
\pgfpathlineto{\pgfqpoint{1.426478in}{0.680021in}}%
\pgfpathlineto{\pgfqpoint{1.456496in}{0.674390in}}%
\pgfpathlineto{\pgfqpoint{1.461348in}{0.673481in}}%
\pgfpathlineto{\pgfqpoint{1.486513in}{0.668852in}}%
\pgfpathlineto{\pgfqpoint{1.516530in}{0.663329in}}%
\pgfpathlineto{\pgfqpoint{1.546548in}{0.657803in}}%
\pgfpathlineto{\pgfqpoint{1.576565in}{0.652273in}}%
\pgfpathlineto{\pgfqpoint{1.599691in}{0.648012in}}%
\pgfpathlineto{\pgfqpoint{1.606583in}{0.646765in}}%
\pgfpathlineto{\pgfqpoint{1.636600in}{0.641339in}}%
\pgfpathlineto{\pgfqpoint{1.666617in}{0.635910in}}%
\pgfpathlineto{\pgfqpoint{1.696635in}{0.630476in}}%
\pgfpathlineto{\pgfqpoint{1.726652in}{0.625039in}}%
\pgfpathlineto{\pgfqpoint{1.740441in}{0.622542in}}%
\pgfpathlineto{\pgfqpoint{1.756670in}{0.619656in}}%
\pgfpathlineto{\pgfqpoint{1.786687in}{0.614320in}}%
\pgfpathlineto{\pgfqpoint{1.816704in}{0.608979in}}%
\pgfpathlineto{\pgfqpoint{1.846722in}{0.603634in}}%
\pgfpathlineto{\pgfqpoint{1.876739in}{0.598284in}}%
\pgfpathlineto{\pgfqpoint{1.883539in}{0.597073in}}%
\pgfpathlineto{\pgfqpoint{1.906757in}{0.593012in}}%
\pgfpathlineto{\pgfqpoint{1.936774in}{0.587761in}}%
\pgfpathlineto{\pgfqpoint{1.966792in}{0.582504in}}%
\pgfpathlineto{\pgfqpoint{1.996809in}{0.577242in}}%
\pgfpathlineto{\pgfqpoint{2.026826in}{0.571975in}}%
\pgfpathlineto{\pgfqpoint{2.028947in}{0.571603in}}%
\pgfusepath{stroke}%
\end{pgfscope}%
\begin{pgfscope}%
\pgfpathrectangle{\pgfqpoint{0.766095in}{0.571603in}}{\pgfqpoint{5.973465in}{5.068436in}}%
\pgfusepath{clip}%
\pgfsetbuttcap%
\pgfsetroundjoin%
\pgfsetlinewidth{1.505625pt}%
\definecolor{currentstroke}{rgb}{0.496615,0.826376,0.306377}%
\pgfsetstrokecolor{currentstroke}%
\pgfsetdash{}{0pt}%
\pgfpathmoveto{\pgfqpoint{0.766095in}{0.773710in}}%
\pgfpathlineto{\pgfqpoint{0.796112in}{0.767654in}}%
\pgfpathlineto{\pgfqpoint{0.826130in}{0.761600in}}%
\pgfpathlineto{\pgfqpoint{0.856147in}{0.755546in}}%
\pgfpathlineto{\pgfqpoint{0.884201in}{0.749890in}}%
\pgfpathlineto{\pgfqpoint{0.886165in}{0.749501in}}%
\pgfpathlineto{\pgfqpoint{0.916182in}{0.743569in}}%
\pgfpathlineto{\pgfqpoint{0.946199in}{0.737638in}}%
\pgfpathlineto{\pgfqpoint{0.976217in}{0.731706in}}%
\pgfpathlineto{\pgfqpoint{1.006234in}{0.725774in}}%
\pgfpathlineto{\pgfqpoint{1.013099in}{0.724420in}}%
\pgfpathlineto{\pgfqpoint{1.036252in}{0.719934in}}%
\pgfpathlineto{\pgfqpoint{1.066269in}{0.714119in}}%
\pgfpathlineto{\pgfqpoint{1.096287in}{0.708304in}}%
\pgfpathlineto{\pgfqpoint{1.126304in}{0.702488in}}%
\pgfpathlineto{\pgfqpoint{1.144577in}{0.698951in}}%
\pgfpathlineto{\pgfqpoint{1.156321in}{0.696717in}}%
\pgfpathlineto{\pgfqpoint{1.186339in}{0.691014in}}%
\pgfpathlineto{\pgfqpoint{1.216356in}{0.685311in}}%
\pgfpathlineto{\pgfqpoint{1.246374in}{0.679605in}}%
\pgfpathlineto{\pgfqpoint{1.276391in}{0.673898in}}%
\pgfpathlineto{\pgfqpoint{1.278585in}{0.673481in}}%
\pgfpathlineto{\pgfqpoint{1.306408in}{0.668294in}}%
\pgfpathlineto{\pgfqpoint{1.336426in}{0.662697in}}%
\pgfpathlineto{\pgfqpoint{1.366443in}{0.657097in}}%
\pgfpathlineto{\pgfqpoint{1.396461in}{0.651495in}}%
\pgfpathlineto{\pgfqpoint{1.415130in}{0.648012in}}%
\pgfpathlineto{\pgfqpoint{1.426478in}{0.645932in}}%
\pgfpathlineto{\pgfqpoint{1.456496in}{0.640436in}}%
\pgfpathlineto{\pgfqpoint{1.486513in}{0.634938in}}%
\pgfpathlineto{\pgfqpoint{1.516530in}{0.629436in}}%
\pgfpathlineto{\pgfqpoint{1.546548in}{0.623931in}}%
\pgfpathlineto{\pgfqpoint{1.554129in}{0.622542in}}%
\pgfpathlineto{\pgfqpoint{1.576565in}{0.618505in}}%
\pgfpathlineto{\pgfqpoint{1.606583in}{0.613103in}}%
\pgfpathlineto{\pgfqpoint{1.636600in}{0.607698in}}%
\pgfpathlineto{\pgfqpoint{1.666617in}{0.602289in}}%
\pgfpathlineto{\pgfqpoint{1.695546in}{0.597073in}}%
\pgfpathlineto{\pgfqpoint{1.696635in}{0.596880in}}%
\pgfpathlineto{\pgfqpoint{1.726652in}{0.591572in}}%
\pgfpathlineto{\pgfqpoint{1.756670in}{0.586259in}}%
\pgfpathlineto{\pgfqpoint{1.786687in}{0.580943in}}%
\pgfpathlineto{\pgfqpoint{1.816704in}{0.575621in}}%
\pgfpathlineto{\pgfqpoint{1.839364in}{0.571603in}}%
\pgfusepath{stroke}%
\end{pgfscope}%
\begin{pgfscope}%
\pgfpathrectangle{\pgfqpoint{0.766095in}{0.571603in}}{\pgfqpoint{5.973465in}{5.068436in}}%
\pgfusepath{clip}%
\pgfsetbuttcap%
\pgfsetroundjoin%
\pgfsetlinewidth{1.505625pt}%
\definecolor{currentstroke}{rgb}{0.565498,0.842430,0.262877}%
\pgfsetstrokecolor{currentstroke}%
\pgfsetdash{}{0pt}%
\pgfpathmoveto{\pgfqpoint{0.766095in}{0.739171in}}%
\pgfpathlineto{\pgfqpoint{0.796112in}{0.733145in}}%
\pgfpathlineto{\pgfqpoint{0.826130in}{0.727120in}}%
\pgfpathlineto{\pgfqpoint{0.839601in}{0.724420in}}%
\pgfpathlineto{\pgfqpoint{0.856147in}{0.721162in}}%
\pgfpathlineto{\pgfqpoint{0.886165in}{0.715258in}}%
\pgfpathlineto{\pgfqpoint{0.916182in}{0.709354in}}%
\pgfpathlineto{\pgfqpoint{0.946199in}{0.703451in}}%
\pgfpathlineto{\pgfqpoint{0.969093in}{0.698951in}}%
\pgfpathlineto{\pgfqpoint{0.976217in}{0.697575in}}%
\pgfpathlineto{\pgfqpoint{1.006234in}{0.691788in}}%
\pgfpathlineto{\pgfqpoint{1.036252in}{0.686002in}}%
\pgfpathlineto{\pgfqpoint{1.066269in}{0.680214in}}%
\pgfpathlineto{\pgfqpoint{1.096287in}{0.674426in}}%
\pgfpathlineto{\pgfqpoint{1.101194in}{0.673481in}}%
\pgfpathlineto{\pgfqpoint{1.126304in}{0.668733in}}%
\pgfpathlineto{\pgfqpoint{1.156321in}{0.663057in}}%
\pgfpathlineto{\pgfqpoint{1.186339in}{0.657381in}}%
\pgfpathlineto{\pgfqpoint{1.216356in}{0.651703in}}%
\pgfpathlineto{\pgfqpoint{1.235878in}{0.648012in}}%
\pgfpathlineto{\pgfqpoint{1.246374in}{0.646062in}}%
\pgfpathlineto{\pgfqpoint{1.276391in}{0.640493in}}%
\pgfpathlineto{\pgfqpoint{1.306408in}{0.634922in}}%
\pgfpathlineto{\pgfqpoint{1.336426in}{0.629349in}}%
\pgfpathlineto{\pgfqpoint{1.366443in}{0.623774in}}%
\pgfpathlineto{\pgfqpoint{1.373085in}{0.622542in}}%
\pgfpathlineto{\pgfqpoint{1.396461in}{0.618282in}}%
\pgfpathlineto{\pgfqpoint{1.426478in}{0.612813in}}%
\pgfpathlineto{\pgfqpoint{1.456496in}{0.607341in}}%
\pgfpathlineto{\pgfqpoint{1.486513in}{0.601866in}}%
\pgfpathlineto{\pgfqpoint{1.512782in}{0.597073in}}%
\pgfpathlineto{\pgfqpoint{1.516530in}{0.596401in}}%
\pgfpathlineto{\pgfqpoint{1.546548in}{0.591028in}}%
\pgfpathlineto{\pgfqpoint{1.576565in}{0.585653in}}%
\pgfpathlineto{\pgfqpoint{1.606583in}{0.580274in}}%
\pgfpathlineto{\pgfqpoint{1.636600in}{0.574891in}}%
\pgfpathlineto{\pgfqpoint{1.654938in}{0.571603in}}%
\pgfusepath{stroke}%
\end{pgfscope}%
\begin{pgfscope}%
\pgfpathrectangle{\pgfqpoint{0.766095in}{0.571603in}}{\pgfqpoint{5.973465in}{5.068436in}}%
\pgfusepath{clip}%
\pgfsetbuttcap%
\pgfsetroundjoin%
\pgfsetlinewidth{1.505625pt}%
\definecolor{currentstroke}{rgb}{0.636902,0.856542,0.216620}%
\pgfsetstrokecolor{currentstroke}%
\pgfsetdash{}{0pt}%
\pgfpathmoveto{\pgfqpoint{0.766095in}{0.705486in}}%
\pgfpathlineto{\pgfqpoint{0.796112in}{0.699491in}}%
\pgfpathlineto{\pgfqpoint{0.798823in}{0.698951in}}%
\pgfpathlineto{\pgfqpoint{0.826130in}{0.693604in}}%
\pgfpathlineto{\pgfqpoint{0.856147in}{0.687730in}}%
\pgfpathlineto{\pgfqpoint{0.886165in}{0.681856in}}%
\pgfpathlineto{\pgfqpoint{0.916182in}{0.675982in}}%
\pgfpathlineto{\pgfqpoint{0.928982in}{0.673481in}}%
\pgfpathlineto{\pgfqpoint{0.946199in}{0.670175in}}%
\pgfpathlineto{\pgfqpoint{0.976217in}{0.664417in}}%
\pgfpathlineto{\pgfqpoint{1.006234in}{0.658660in}}%
\pgfpathlineto{\pgfqpoint{1.036252in}{0.652902in}}%
\pgfpathlineto{\pgfqpoint{1.061749in}{0.648012in}}%
\pgfpathlineto{\pgfqpoint{1.066269in}{0.647160in}}%
\pgfpathlineto{\pgfqpoint{1.096287in}{0.641514in}}%
\pgfpathlineto{\pgfqpoint{1.126304in}{0.635867in}}%
\pgfpathlineto{\pgfqpoint{1.156321in}{0.630219in}}%
\pgfpathlineto{\pgfqpoint{1.186339in}{0.624570in}}%
\pgfpathlineto{\pgfqpoint{1.197127in}{0.622542in}}%
\pgfpathlineto{\pgfqpoint{1.216356in}{0.618991in}}%
\pgfpathlineto{\pgfqpoint{1.246374in}{0.613450in}}%
\pgfpathlineto{\pgfqpoint{1.276391in}{0.607907in}}%
\pgfpathlineto{\pgfqpoint{1.306408in}{0.602363in}}%
\pgfpathlineto{\pgfqpoint{1.335040in}{0.597073in}}%
\pgfpathlineto{\pgfqpoint{1.336426in}{0.596821in}}%
\pgfpathlineto{\pgfqpoint{1.366443in}{0.591382in}}%
\pgfpathlineto{\pgfqpoint{1.396461in}{0.585940in}}%
\pgfpathlineto{\pgfqpoint{1.426478in}{0.580496in}}%
\pgfpathlineto{\pgfqpoint{1.456496in}{0.575049in}}%
\pgfpathlineto{\pgfqpoint{1.475491in}{0.571603in}}%
\pgfusepath{stroke}%
\end{pgfscope}%
\begin{pgfscope}%
\pgfpathrectangle{\pgfqpoint{0.766095in}{0.571603in}}{\pgfqpoint{5.973465in}{5.068436in}}%
\pgfusepath{clip}%
\pgfsetbuttcap%
\pgfsetroundjoin%
\pgfsetlinewidth{1.505625pt}%
\definecolor{currentstroke}{rgb}{0.709898,0.868751,0.169257}%
\pgfsetstrokecolor{currentstroke}%
\pgfsetdash{}{0pt}%
\pgfpathmoveto{\pgfqpoint{0.766095in}{0.672634in}}%
\pgfpathlineto{\pgfqpoint{0.796112in}{0.666789in}}%
\pgfpathlineto{\pgfqpoint{0.826130in}{0.660945in}}%
\pgfpathlineto{\pgfqpoint{0.856147in}{0.655102in}}%
\pgfpathlineto{\pgfqpoint{0.886165in}{0.649260in}}%
\pgfpathlineto{\pgfqpoint{0.892589in}{0.648012in}}%
\pgfpathlineto{\pgfqpoint{0.916182in}{0.643508in}}%
\pgfpathlineto{\pgfqpoint{0.946199in}{0.637780in}}%
\pgfpathlineto{\pgfqpoint{0.976217in}{0.632053in}}%
\pgfpathlineto{\pgfqpoint{1.006234in}{0.626326in}}%
\pgfpathlineto{\pgfqpoint{1.026080in}{0.622542in}}%
\pgfpathlineto{\pgfqpoint{1.036252in}{0.620636in}}%
\pgfpathlineto{\pgfqpoint{1.066269in}{0.615020in}}%
\pgfpathlineto{\pgfqpoint{1.096287in}{0.609403in}}%
\pgfpathlineto{\pgfqpoint{1.126304in}{0.603786in}}%
\pgfpathlineto{\pgfqpoint{1.156321in}{0.598167in}}%
\pgfpathlineto{\pgfqpoint{1.162177in}{0.597073in}}%
\pgfpathlineto{\pgfqpoint{1.186339in}{0.592636in}}%
\pgfpathlineto{\pgfqpoint{1.216356in}{0.587124in}}%
\pgfpathlineto{\pgfqpoint{1.246374in}{0.581612in}}%
\pgfpathlineto{\pgfqpoint{1.276391in}{0.576097in}}%
\pgfpathlineto{\pgfqpoint{1.300855in}{0.571603in}}%
\pgfusepath{stroke}%
\end{pgfscope}%
\begin{pgfscope}%
\pgfpathrectangle{\pgfqpoint{0.766095in}{0.571603in}}{\pgfqpoint{5.973465in}{5.068436in}}%
\pgfusepath{clip}%
\pgfsetbuttcap%
\pgfsetroundjoin%
\pgfsetlinewidth{1.505625pt}%
\definecolor{currentstroke}{rgb}{0.783315,0.879285,0.125405}%
\pgfsetstrokecolor{currentstroke}%
\pgfsetdash{}{0pt}%
\pgfpathmoveto{\pgfqpoint{0.766095in}{0.640676in}}%
\pgfpathlineto{\pgfqpoint{0.796112in}{0.634864in}}%
\pgfpathlineto{\pgfqpoint{0.826130in}{0.629054in}}%
\pgfpathlineto{\pgfqpoint{0.856147in}{0.623244in}}%
\pgfpathlineto{\pgfqpoint{0.859782in}{0.622542in}}%
\pgfpathlineto{\pgfqpoint{0.886165in}{0.617534in}}%
\pgfpathlineto{\pgfqpoint{0.916182in}{0.611839in}}%
\pgfpathlineto{\pgfqpoint{0.946199in}{0.606144in}}%
\pgfpathlineto{\pgfqpoint{0.976217in}{0.600448in}}%
\pgfpathlineto{\pgfqpoint{0.994027in}{0.597073in}}%
\pgfpathlineto{\pgfqpoint{1.006234in}{0.594798in}}%
\pgfpathlineto{\pgfqpoint{1.036252in}{0.589213in}}%
\pgfpathlineto{\pgfqpoint{1.066269in}{0.583628in}}%
\pgfpathlineto{\pgfqpoint{1.096287in}{0.578042in}}%
\pgfpathlineto{\pgfqpoint{1.126304in}{0.572455in}}%
\pgfpathlineto{\pgfqpoint{1.130890in}{0.571603in}}%
\pgfusepath{stroke}%
\end{pgfscope}%
\begin{pgfscope}%
\pgfpathrectangle{\pgfqpoint{0.766095in}{0.571603in}}{\pgfqpoint{5.973465in}{5.068436in}}%
\pgfusepath{clip}%
\pgfsetbuttcap%
\pgfsetroundjoin%
\pgfsetlinewidth{1.505625pt}%
\definecolor{currentstroke}{rgb}{0.855810,0.888601,0.097452}%
\pgfsetstrokecolor{currentstroke}%
\pgfsetdash{}{0pt}%
\pgfpathmoveto{\pgfqpoint{0.766095in}{0.609451in}}%
\pgfpathlineto{\pgfqpoint{0.796112in}{0.603674in}}%
\pgfpathlineto{\pgfqpoint{0.826130in}{0.597899in}}%
\pgfpathlineto{\pgfqpoint{0.830433in}{0.597073in}}%
\pgfpathlineto{\pgfqpoint{0.856147in}{0.592219in}}%
\pgfpathlineto{\pgfqpoint{0.886165in}{0.586557in}}%
\pgfpathlineto{\pgfqpoint{0.916182in}{0.580895in}}%
\pgfpathlineto{\pgfqpoint{0.946199in}{0.575233in}}%
\pgfpathlineto{\pgfqpoint{0.965461in}{0.571603in}}%
\pgfusepath{stroke}%
\end{pgfscope}%
\begin{pgfscope}%
\pgfpathrectangle{\pgfqpoint{0.766095in}{0.571603in}}{\pgfqpoint{5.973465in}{5.068436in}}%
\pgfusepath{clip}%
\pgfsetbuttcap%
\pgfsetroundjoin%
\pgfsetlinewidth{1.505625pt}%
\definecolor{currentstroke}{rgb}{0.926106,0.897330,0.104071}%
\pgfsetstrokecolor{currentstroke}%
\pgfsetdash{}{0pt}%
\pgfpathmoveto{\pgfqpoint{0.766095in}{0.578930in}}%
\pgfpathlineto{\pgfqpoint{0.796112in}{0.573189in}}%
\pgfpathlineto{\pgfqpoint{0.804421in}{0.571603in}}%
\pgfusepath{stroke}%
\end{pgfscope}%
\begin{pgfscope}%
\pgfpathrectangle{\pgfqpoint{0.766095in}{0.571603in}}{\pgfqpoint{5.973465in}{5.068436in}}%
\pgfusepath{clip}%
\pgfsetrectcap%
\pgfsetroundjoin%
\pgfsetlinewidth{1.505625pt}%
\definecolor{currentstroke}{rgb}{1.000000,0.000000,0.000000}%
\pgfsetstrokecolor{currentstroke}%
\pgfsetdash{}{0pt}%
\pgfpathmoveto{\pgfqpoint{5.886208in}{1.295666in}}%
\pgfpathlineto{\pgfqpoint{6.220651in}{2.731541in}}%
\pgfpathlineto{\pgfqpoint{5.528009in}{1.804642in}}%
\pgfpathlineto{\pgfqpoint{1.825266in}{3.093923in}}%
\pgfpathlineto{\pgfqpoint{2.836493in}{3.972121in}}%
\pgfpathlineto{\pgfqpoint{2.572898in}{3.461492in}}%
\pgfpathlineto{\pgfqpoint{3.143893in}{3.158964in}}%
\pgfpathlineto{\pgfqpoint{3.135464in}{3.084190in}}%
\pgfpathlineto{\pgfqpoint{3.166161in}{3.133778in}}%
\pgfpathlineto{\pgfqpoint{3.124094in}{3.125317in}}%
\pgfpathlineto{\pgfqpoint{3.131801in}{3.118364in}}%
\pgfpathlineto{\pgfqpoint{3.139555in}{3.116890in}}%
\pgfpathlineto{\pgfqpoint{3.135210in}{3.117438in}}%
\pgfpathlineto{\pgfqpoint{3.134475in}{3.117816in}}%
\pgfpathlineto{\pgfqpoint{3.135979in}{3.119872in}}%
\pgfpathlineto{\pgfqpoint{3.135086in}{3.118854in}}%
\pgfusepath{stroke}%
\end{pgfscope}%
\begin{pgfscope}%
\pgfpathrectangle{\pgfqpoint{0.766095in}{0.571603in}}{\pgfqpoint{5.973465in}{5.068436in}}%
\pgfusepath{clip}%
\pgfsetbuttcap%
\pgfsetroundjoin%
\definecolor{currentfill}{rgb}{1.000000,0.000000,0.000000}%
\pgfsetfillcolor{currentfill}%
\pgfsetlinewidth{1.003750pt}%
\definecolor{currentstroke}{rgb}{1.000000,0.000000,0.000000}%
\pgfsetstrokecolor{currentstroke}%
\pgfsetdash{}{0pt}%
\pgfsys@defobject{currentmarker}{\pgfqpoint{-0.041667in}{-0.041667in}}{\pgfqpoint{0.041667in}{0.041667in}}{%
\pgfpathmoveto{\pgfqpoint{0.000000in}{-0.041667in}}%
\pgfpathcurveto{\pgfqpoint{0.011050in}{-0.041667in}}{\pgfqpoint{0.021649in}{-0.037276in}}{\pgfqpoint{0.029463in}{-0.029463in}}%
\pgfpathcurveto{\pgfqpoint{0.037276in}{-0.021649in}}{\pgfqpoint{0.041667in}{-0.011050in}}{\pgfqpoint{0.041667in}{0.000000in}}%
\pgfpathcurveto{\pgfqpoint{0.041667in}{0.011050in}}{\pgfqpoint{0.037276in}{0.021649in}}{\pgfqpoint{0.029463in}{0.029463in}}%
\pgfpathcurveto{\pgfqpoint{0.021649in}{0.037276in}}{\pgfqpoint{0.011050in}{0.041667in}}{\pgfqpoint{0.000000in}{0.041667in}}%
\pgfpathcurveto{\pgfqpoint{-0.011050in}{0.041667in}}{\pgfqpoint{-0.021649in}{0.037276in}}{\pgfqpoint{-0.029463in}{0.029463in}}%
\pgfpathcurveto{\pgfqpoint{-0.037276in}{0.021649in}}{\pgfqpoint{-0.041667in}{0.011050in}}{\pgfqpoint{-0.041667in}{0.000000in}}%
\pgfpathcurveto{\pgfqpoint{-0.041667in}{-0.011050in}}{\pgfqpoint{-0.037276in}{-0.021649in}}{\pgfqpoint{-0.029463in}{-0.029463in}}%
\pgfpathcurveto{\pgfqpoint{-0.021649in}{-0.037276in}}{\pgfqpoint{-0.011050in}{-0.041667in}}{\pgfqpoint{0.000000in}{-0.041667in}}%
\pgfpathlineto{\pgfqpoint{0.000000in}{-0.041667in}}%
\pgfpathclose%
\pgfusepath{stroke,fill}%
}%
\begin{pgfscope}%
\pgfsys@transformshift{5.886208in}{1.295666in}%
\pgfsys@useobject{currentmarker}{}%
\end{pgfscope}%
\begin{pgfscope}%
\pgfsys@transformshift{6.220651in}{2.731541in}%
\pgfsys@useobject{currentmarker}{}%
\end{pgfscope}%
\begin{pgfscope}%
\pgfsys@transformshift{5.528009in}{1.804642in}%
\pgfsys@useobject{currentmarker}{}%
\end{pgfscope}%
\begin{pgfscope}%
\pgfsys@transformshift{1.825266in}{3.093923in}%
\pgfsys@useobject{currentmarker}{}%
\end{pgfscope}%
\begin{pgfscope}%
\pgfsys@transformshift{2.836493in}{3.972121in}%
\pgfsys@useobject{currentmarker}{}%
\end{pgfscope}%
\begin{pgfscope}%
\pgfsys@transformshift{2.572898in}{3.461492in}%
\pgfsys@useobject{currentmarker}{}%
\end{pgfscope}%
\begin{pgfscope}%
\pgfsys@transformshift{3.143893in}{3.158964in}%
\pgfsys@useobject{currentmarker}{}%
\end{pgfscope}%
\begin{pgfscope}%
\pgfsys@transformshift{3.135464in}{3.084190in}%
\pgfsys@useobject{currentmarker}{}%
\end{pgfscope}%
\begin{pgfscope}%
\pgfsys@transformshift{3.166161in}{3.133778in}%
\pgfsys@useobject{currentmarker}{}%
\end{pgfscope}%
\begin{pgfscope}%
\pgfsys@transformshift{3.124094in}{3.125317in}%
\pgfsys@useobject{currentmarker}{}%
\end{pgfscope}%
\begin{pgfscope}%
\pgfsys@transformshift{3.131801in}{3.118364in}%
\pgfsys@useobject{currentmarker}{}%
\end{pgfscope}%
\begin{pgfscope}%
\pgfsys@transformshift{3.139555in}{3.116890in}%
\pgfsys@useobject{currentmarker}{}%
\end{pgfscope}%
\begin{pgfscope}%
\pgfsys@transformshift{3.135210in}{3.117438in}%
\pgfsys@useobject{currentmarker}{}%
\end{pgfscope}%
\begin{pgfscope}%
\pgfsys@transformshift{3.134475in}{3.117816in}%
\pgfsys@useobject{currentmarker}{}%
\end{pgfscope}%
\begin{pgfscope}%
\pgfsys@transformshift{3.135979in}{3.119872in}%
\pgfsys@useobject{currentmarker}{}%
\end{pgfscope}%
\begin{pgfscope}%
\pgfsys@transformshift{3.135086in}{3.118854in}%
\pgfsys@useobject{currentmarker}{}%
\end{pgfscope}%
\end{pgfscope}%
\begin{pgfscope}%
\pgfsetrectcap%
\pgfsetmiterjoin%
\pgfsetlinewidth{0.803000pt}%
\definecolor{currentstroke}{rgb}{0.000000,0.000000,0.000000}%
\pgfsetstrokecolor{currentstroke}%
\pgfsetdash{}{0pt}%
\pgfpathmoveto{\pgfqpoint{0.766095in}{0.571603in}}%
\pgfpathlineto{\pgfqpoint{0.766095in}{5.640039in}}%
\pgfusepath{stroke}%
\end{pgfscope}%
\begin{pgfscope}%
\pgfsetrectcap%
\pgfsetmiterjoin%
\pgfsetlinewidth{0.803000pt}%
\definecolor{currentstroke}{rgb}{0.000000,0.000000,0.000000}%
\pgfsetstrokecolor{currentstroke}%
\pgfsetdash{}{0pt}%
\pgfpathmoveto{\pgfqpoint{6.739560in}{0.571603in}}%
\pgfpathlineto{\pgfqpoint{6.739560in}{5.640039in}}%
\pgfusepath{stroke}%
\end{pgfscope}%
\begin{pgfscope}%
\pgfsetrectcap%
\pgfsetmiterjoin%
\pgfsetlinewidth{0.803000pt}%
\definecolor{currentstroke}{rgb}{0.000000,0.000000,0.000000}%
\pgfsetstrokecolor{currentstroke}%
\pgfsetdash{}{0pt}%
\pgfpathmoveto{\pgfqpoint{0.766095in}{0.571603in}}%
\pgfpathlineto{\pgfqpoint{6.739560in}{0.571603in}}%
\pgfusepath{stroke}%
\end{pgfscope}%
\begin{pgfscope}%
\pgfsetrectcap%
\pgfsetmiterjoin%
\pgfsetlinewidth{0.803000pt}%
\definecolor{currentstroke}{rgb}{0.000000,0.000000,0.000000}%
\pgfsetstrokecolor{currentstroke}%
\pgfsetdash{}{0pt}%
\pgfpathmoveto{\pgfqpoint{0.766095in}{5.640039in}}%
\pgfpathlineto{\pgfqpoint{6.739560in}{5.640039in}}%
\pgfusepath{stroke}%
\end{pgfscope}%
\begin{pgfscope}%
\pgfsetbuttcap%
\pgfsetmiterjoin%
\definecolor{currentfill}{rgb}{1.000000,1.000000,1.000000}%
\pgfsetfillcolor{currentfill}%
\pgfsetfillopacity{0.800000}%
\pgfsetlinewidth{1.003750pt}%
\definecolor{currentstroke}{rgb}{0.800000,0.800000,0.800000}%
\pgfsetstrokecolor{currentstroke}%
\pgfsetstrokeopacity{0.800000}%
\pgfsetdash{}{0pt}%
\pgfpathmoveto{\pgfqpoint{5.536408in}{5.121213in}}%
\pgfpathlineto{\pgfqpoint{6.642338in}{5.121213in}}%
\pgfpathquadraticcurveto{\pgfqpoint{6.670116in}{5.121213in}}{\pgfqpoint{6.670116in}{5.148991in}}%
\pgfpathlineto{\pgfqpoint{6.670116in}{5.542817in}}%
\pgfpathquadraticcurveto{\pgfqpoint{6.670116in}{5.570595in}}{\pgfqpoint{6.642338in}{5.570595in}}%
\pgfpathlineto{\pgfqpoint{5.536408in}{5.570595in}}%
\pgfpathquadraticcurveto{\pgfqpoint{5.508630in}{5.570595in}}{\pgfqpoint{5.508630in}{5.542817in}}%
\pgfpathlineto{\pgfqpoint{5.508630in}{5.148991in}}%
\pgfpathquadraticcurveto{\pgfqpoint{5.508630in}{5.121213in}}{\pgfqpoint{5.536408in}{5.121213in}}%
\pgfpathlineto{\pgfqpoint{5.536408in}{5.121213in}}%
\pgfpathclose%
\pgfusepath{stroke,fill}%
\end{pgfscope}%
\begin{pgfscope}%
\pgfsetrectcap%
\pgfsetroundjoin%
\pgfsetlinewidth{1.505625pt}%
\definecolor{currentstroke}{rgb}{1.000000,0.000000,0.000000}%
\pgfsetstrokecolor{currentstroke}%
\pgfsetdash{}{0pt}%
\pgfpathmoveto{\pgfqpoint{5.564186in}{5.458127in}}%
\pgfpathlineto{\pgfqpoint{5.703075in}{5.458127in}}%
\pgfpathlineto{\pgfqpoint{5.841964in}{5.458127in}}%
\pgfusepath{stroke}%
\end{pgfscope}%
\begin{pgfscope}%
\pgfsetbuttcap%
\pgfsetroundjoin%
\definecolor{currentfill}{rgb}{1.000000,0.000000,0.000000}%
\pgfsetfillcolor{currentfill}%
\pgfsetlinewidth{1.003750pt}%
\definecolor{currentstroke}{rgb}{1.000000,0.000000,0.000000}%
\pgfsetstrokecolor{currentstroke}%
\pgfsetdash{}{0pt}%
\pgfsys@defobject{currentmarker}{\pgfqpoint{-0.041667in}{-0.041667in}}{\pgfqpoint{0.041667in}{0.041667in}}{%
\pgfpathmoveto{\pgfqpoint{0.000000in}{-0.041667in}}%
\pgfpathcurveto{\pgfqpoint{0.011050in}{-0.041667in}}{\pgfqpoint{0.021649in}{-0.037276in}}{\pgfqpoint{0.029463in}{-0.029463in}}%
\pgfpathcurveto{\pgfqpoint{0.037276in}{-0.021649in}}{\pgfqpoint{0.041667in}{-0.011050in}}{\pgfqpoint{0.041667in}{0.000000in}}%
\pgfpathcurveto{\pgfqpoint{0.041667in}{0.011050in}}{\pgfqpoint{0.037276in}{0.021649in}}{\pgfqpoint{0.029463in}{0.029463in}}%
\pgfpathcurveto{\pgfqpoint{0.021649in}{0.037276in}}{\pgfqpoint{0.011050in}{0.041667in}}{\pgfqpoint{0.000000in}{0.041667in}}%
\pgfpathcurveto{\pgfqpoint{-0.011050in}{0.041667in}}{\pgfqpoint{-0.021649in}{0.037276in}}{\pgfqpoint{-0.029463in}{0.029463in}}%
\pgfpathcurveto{\pgfqpoint{-0.037276in}{0.021649in}}{\pgfqpoint{-0.041667in}{0.011050in}}{\pgfqpoint{-0.041667in}{0.000000in}}%
\pgfpathcurveto{\pgfqpoint{-0.041667in}{-0.011050in}}{\pgfqpoint{-0.037276in}{-0.021649in}}{\pgfqpoint{-0.029463in}{-0.029463in}}%
\pgfpathcurveto{\pgfqpoint{-0.021649in}{-0.037276in}}{\pgfqpoint{-0.011050in}{-0.041667in}}{\pgfqpoint{0.000000in}{-0.041667in}}%
\pgfpathlineto{\pgfqpoint{0.000000in}{-0.041667in}}%
\pgfpathclose%
\pgfusepath{stroke,fill}%
}%
\begin{pgfscope}%
\pgfsys@transformshift{5.703075in}{5.458127in}%
\pgfsys@useobject{currentmarker}{}%
\end{pgfscope}%
\end{pgfscope}%
\begin{pgfscope}%
\definecolor{textcolor}{rgb}{0.000000,0.000000,0.000000}%
\pgfsetstrokecolor{textcolor}%
\pgfsetfillcolor{textcolor}%
\pgftext[x=5.953075in,y=5.409516in,left,base]{\color{textcolor}\sffamily\fontsize{10.000000}{12.000000}\selectfont Iterations}%
\end{pgfscope}%
\begin{pgfscope}%
\pgfsetbuttcap%
\pgfsetroundjoin%
\definecolor{currentfill}{rgb}{0.000000,0.000000,1.000000}%
\pgfsetfillcolor{currentfill}%
\pgfsetlinewidth{1.003750pt}%
\definecolor{currentstroke}{rgb}{0.000000,0.000000,1.000000}%
\pgfsetstrokecolor{currentstroke}%
\pgfsetdash{}{0pt}%
\pgfsys@defobject{currentmarker}{\pgfqpoint{-0.069444in}{-0.069444in}}{\pgfqpoint{0.069444in}{0.069444in}}{%
\pgfpathmoveto{\pgfqpoint{0.000000in}{-0.069444in}}%
\pgfpathcurveto{\pgfqpoint{0.018417in}{-0.069444in}}{\pgfqpoint{0.036082in}{-0.062127in}}{\pgfqpoint{0.049105in}{-0.049105in}}%
\pgfpathcurveto{\pgfqpoint{0.062127in}{-0.036082in}}{\pgfqpoint{0.069444in}{-0.018417in}}{\pgfqpoint{0.069444in}{0.000000in}}%
\pgfpathcurveto{\pgfqpoint{0.069444in}{0.018417in}}{\pgfqpoint{0.062127in}{0.036082in}}{\pgfqpoint{0.049105in}{0.049105in}}%
\pgfpathcurveto{\pgfqpoint{0.036082in}{0.062127in}}{\pgfqpoint{0.018417in}{0.069444in}}{\pgfqpoint{0.000000in}{0.069444in}}%
\pgfpathcurveto{\pgfqpoint{-0.018417in}{0.069444in}}{\pgfqpoint{-0.036082in}{0.062127in}}{\pgfqpoint{-0.049105in}{0.049105in}}%
\pgfpathcurveto{\pgfqpoint{-0.062127in}{0.036082in}}{\pgfqpoint{-0.069444in}{0.018417in}}{\pgfqpoint{-0.069444in}{0.000000in}}%
\pgfpathcurveto{\pgfqpoint{-0.069444in}{-0.018417in}}{\pgfqpoint{-0.062127in}{-0.036082in}}{\pgfqpoint{-0.049105in}{-0.049105in}}%
\pgfpathcurveto{\pgfqpoint{-0.036082in}{-0.062127in}}{\pgfqpoint{-0.018417in}{-0.069444in}}{\pgfqpoint{0.000000in}{-0.069444in}}%
\pgfpathlineto{\pgfqpoint{0.000000in}{-0.069444in}}%
\pgfpathclose%
\pgfusepath{stroke,fill}%
}%
\begin{pgfscope}%
\pgfsys@transformshift{5.703075in}{5.242117in}%
\pgfsys@useobject{currentmarker}{}%
\end{pgfscope}%
\end{pgfscope}%
\begin{pgfscope}%
\definecolor{textcolor}{rgb}{0.000000,0.000000,0.000000}%
\pgfsetstrokecolor{textcolor}%
\pgfsetfillcolor{textcolor}%
\pgftext[x=5.953075in,y=5.205659in,left,base]{\color{textcolor}\sffamily\fontsize{10.000000}{12.000000}\selectfont Minimum}%
\end{pgfscope}%
\end{pgfpicture}%
\makeatother%
\endgroup%
}
        \caption{Pohľad zhora (Vrstevnice)}
        \label{fig:newton_vlavo}
    \end{subfigure}
    \hfill
    \begin{subfigure}{0.48\textwidth}
        \centering
        \resizebox{\linewidth}{!}{%% Creator: Matplotlib, PGF backend
%%
%% To include the figure in your LaTeX document, write
%%   \input{<filename>.pgf}
%%
%% Make sure the required packages are loaded in your preamble
%%   \usepackage{pgf}
%%
%% Also ensure that all the required font packages are loaded; for instance,
%% the lmodern package is sometimes necessary when using math font.
%%   \usepackage{lmodern}
%%
%% Figures using additional raster images can only be included by \input if
%% they are in the same directory as the main LaTeX file. For loading figures
%% from other directories you can use the `import` package
%%   \usepackage{import}
%%
%% and then include the figures with
%%   \import{<path to file>}{<filename>.pgf}
%%
%% Matplotlib used the following preamble
%%   
%%   \usepackage{fontspec}
%%   \setmainfont{DejaVuSerif.ttf}[Path=\detokenize{/home/radimek/Documents/projekt_mat_prog/mat_prog_kernel/lib/python3.12/site-packages/matplotlib/mpl-data/fonts/ttf/}]
%%   \setsansfont{DejaVuSans.ttf}[Path=\detokenize{/home/radimek/Documents/projekt_mat_prog/mat_prog_kernel/lib/python3.12/site-packages/matplotlib/mpl-data/fonts/ttf/}]
%%   \setmonofont{DejaVuSansMono.ttf}[Path=\detokenize{/home/radimek/Documents/projekt_mat_prog/mat_prog_kernel/lib/python3.12/site-packages/matplotlib/mpl-data/fonts/ttf/}]
%%   \makeatletter\@ifpackageloaded{underscore}{}{\usepackage[strings]{underscore}}\makeatother
%%
\begingroup%
\makeatletter%
\begin{pgfpicture}%
\pgfpathrectangle{\pgfpointorigin}{\pgfqpoint{8.000000in}{6.000000in}}%
\pgfusepath{use as bounding box, clip}%
\begin{pgfscope}%
\pgfsetbuttcap%
\pgfsetmiterjoin%
\definecolor{currentfill}{rgb}{1.000000,1.000000,1.000000}%
\pgfsetfillcolor{currentfill}%
\pgfsetlinewidth{0.000000pt}%
\definecolor{currentstroke}{rgb}{1.000000,1.000000,1.000000}%
\pgfsetstrokecolor{currentstroke}%
\pgfsetdash{}{0pt}%
\pgfpathmoveto{\pgfqpoint{0.000000in}{0.000000in}}%
\pgfpathlineto{\pgfqpoint{8.000000in}{0.000000in}}%
\pgfpathlineto{\pgfqpoint{8.000000in}{6.000000in}}%
\pgfpathlineto{\pgfqpoint{0.000000in}{6.000000in}}%
\pgfpathlineto{\pgfqpoint{0.000000in}{0.000000in}}%
\pgfpathclose%
\pgfusepath{fill}%
\end{pgfscope}%
\begin{pgfscope}%
\pgfsetbuttcap%
\pgfsetmiterjoin%
\definecolor{currentfill}{rgb}{1.000000,1.000000,1.000000}%
\pgfsetfillcolor{currentfill}%
\pgfsetlinewidth{0.000000pt}%
\definecolor{currentstroke}{rgb}{0.000000,0.000000,0.000000}%
\pgfsetstrokecolor{currentstroke}%
\pgfsetstrokeopacity{0.000000}%
\pgfsetdash{}{0pt}%
\pgfpathmoveto{\pgfqpoint{1.254980in}{0.150000in}}%
\pgfpathlineto{\pgfqpoint{6.745020in}{0.150000in}}%
\pgfpathlineto{\pgfqpoint{6.745020in}{5.640039in}}%
\pgfpathlineto{\pgfqpoint{1.254980in}{5.640039in}}%
\pgfpathlineto{\pgfqpoint{1.254980in}{0.150000in}}%
\pgfpathclose%
\pgfusepath{fill}%
\end{pgfscope}%
\begin{pgfscope}%
\pgfsetbuttcap%
\pgfsetmiterjoin%
\definecolor{currentfill}{rgb}{0.950000,0.950000,0.950000}%
\pgfsetfillcolor{currentfill}%
\pgfsetfillopacity{0.500000}%
\pgfsetlinewidth{1.003750pt}%
\definecolor{currentstroke}{rgb}{0.950000,0.950000,0.950000}%
\pgfsetstrokecolor{currentstroke}%
\pgfsetstrokeopacity{0.500000}%
\pgfsetdash{}{0pt}%
\pgfpathmoveto{\pgfqpoint{1.669516in}{1.503668in}}%
\pgfpathlineto{\pgfqpoint{3.482506in}{3.023352in}}%
\pgfpathlineto{\pgfqpoint{3.457304in}{5.215008in}}%
\pgfpathlineto{\pgfqpoint{1.557553in}{3.828657in}}%
\pgfusepath{stroke,fill}%
\end{pgfscope}%
\begin{pgfscope}%
\pgfsetbuttcap%
\pgfsetmiterjoin%
\definecolor{currentfill}{rgb}{0.900000,0.900000,0.900000}%
\pgfsetfillcolor{currentfill}%
\pgfsetfillopacity{0.500000}%
\pgfsetlinewidth{1.003750pt}%
\definecolor{currentstroke}{rgb}{0.900000,0.900000,0.900000}%
\pgfsetstrokecolor{currentstroke}%
\pgfsetstrokeopacity{0.500000}%
\pgfsetdash{}{0pt}%
\pgfpathmoveto{\pgfqpoint{3.482506in}{3.023352in}}%
\pgfpathlineto{\pgfqpoint{6.391709in}{2.177762in}}%
\pgfpathlineto{\pgfqpoint{6.495528in}{4.444907in}}%
\pgfpathlineto{\pgfqpoint{3.457304in}{5.215008in}}%
\pgfusepath{stroke,fill}%
\end{pgfscope}%
\begin{pgfscope}%
\pgfsetbuttcap%
\pgfsetmiterjoin%
\definecolor{currentfill}{rgb}{0.925000,0.925000,0.925000}%
\pgfsetfillcolor{currentfill}%
\pgfsetfillopacity{0.500000}%
\pgfsetlinewidth{1.003750pt}%
\definecolor{currentstroke}{rgb}{0.925000,0.925000,0.925000}%
\pgfsetstrokecolor{currentstroke}%
\pgfsetstrokeopacity{0.500000}%
\pgfsetdash{}{0pt}%
\pgfpathmoveto{\pgfqpoint{1.669516in}{1.503668in}}%
\pgfpathlineto{\pgfqpoint{4.753413in}{0.496467in}}%
\pgfpathlineto{\pgfqpoint{6.391709in}{2.177762in}}%
\pgfpathlineto{\pgfqpoint{3.482506in}{3.023352in}}%
\pgfusepath{stroke,fill}%
\end{pgfscope}%
\begin{pgfscope}%
\pgfsetrectcap%
\pgfsetroundjoin%
\pgfsetlinewidth{0.803000pt}%
\definecolor{currentstroke}{rgb}{0.000000,0.000000,0.000000}%
\pgfsetstrokecolor{currentstroke}%
\pgfsetdash{}{0pt}%
\pgfpathmoveto{\pgfqpoint{1.669516in}{1.503668in}}%
\pgfpathlineto{\pgfqpoint{4.753413in}{0.496467in}}%
\pgfusepath{stroke}%
\end{pgfscope}%
\begin{pgfscope}%
\definecolor{textcolor}{rgb}{0.000000,0.000000,0.000000}%
\pgfsetstrokecolor{textcolor}%
\pgfsetfillcolor{textcolor}%
\pgftext[x=2.945156in,y=0.524780in,,]{\color{textcolor}\sffamily\fontsize{10.000000}{12.000000}\selectfont x}%
\end{pgfscope}%
\begin{pgfscope}%
\pgfsetbuttcap%
\pgfsetroundjoin%
\pgfsetlinewidth{0.803000pt}%
\definecolor{currentstroke}{rgb}{0.690196,0.690196,0.690196}%
\pgfsetstrokecolor{currentstroke}%
\pgfsetdash{}{0pt}%
\pgfpathmoveto{\pgfqpoint{1.856293in}{1.442666in}}%
\pgfpathlineto{\pgfqpoint{3.659435in}{2.971926in}}%
\pgfpathlineto{\pgfqpoint{3.641714in}{5.168266in}}%
\pgfusepath{stroke}%
\end{pgfscope}%
\begin{pgfscope}%
\pgfsetbuttcap%
\pgfsetroundjoin%
\pgfsetlinewidth{0.803000pt}%
\definecolor{currentstroke}{rgb}{0.690196,0.690196,0.690196}%
\pgfsetstrokecolor{currentstroke}%
\pgfsetdash{}{0pt}%
\pgfpathmoveto{\pgfqpoint{2.225798in}{1.321986in}}%
\pgfpathlineto{\pgfqpoint{4.009178in}{2.870269in}}%
\pgfpathlineto{\pgfqpoint{4.006383in}{5.075833in}}%
\pgfusepath{stroke}%
\end{pgfscope}%
\begin{pgfscope}%
\pgfsetbuttcap%
\pgfsetroundjoin%
\pgfsetlinewidth{0.803000pt}%
\definecolor{currentstroke}{rgb}{0.690196,0.690196,0.690196}%
\pgfsetstrokecolor{currentstroke}%
\pgfsetdash{}{0pt}%
\pgfpathmoveto{\pgfqpoint{2.600114in}{1.199735in}}%
\pgfpathlineto{\pgfqpoint{4.363097in}{2.767399in}}%
\pgfpathlineto{\pgfqpoint{4.375595in}{4.982249in}}%
\pgfusepath{stroke}%
\end{pgfscope}%
\begin{pgfscope}%
\pgfsetbuttcap%
\pgfsetroundjoin%
\pgfsetlinewidth{0.803000pt}%
\definecolor{currentstroke}{rgb}{0.690196,0.690196,0.690196}%
\pgfsetstrokecolor{currentstroke}%
\pgfsetdash{}{0pt}%
\pgfpathmoveto{\pgfqpoint{2.979334in}{1.075881in}}%
\pgfpathlineto{\pgfqpoint{4.721266in}{2.663293in}}%
\pgfpathlineto{\pgfqpoint{4.749434in}{4.887491in}}%
\pgfusepath{stroke}%
\end{pgfscope}%
\begin{pgfscope}%
\pgfsetbuttcap%
\pgfsetroundjoin%
\pgfsetlinewidth{0.803000pt}%
\definecolor{currentstroke}{rgb}{0.690196,0.690196,0.690196}%
\pgfsetstrokecolor{currentstroke}%
\pgfsetdash{}{0pt}%
\pgfpathmoveto{\pgfqpoint{3.363557in}{0.950394in}}%
\pgfpathlineto{\pgfqpoint{5.083764in}{2.557930in}}%
\pgfpathlineto{\pgfqpoint{5.127989in}{4.791539in}}%
\pgfusepath{stroke}%
\end{pgfscope}%
\begin{pgfscope}%
\pgfsetbuttcap%
\pgfsetroundjoin%
\pgfsetlinewidth{0.803000pt}%
\definecolor{currentstroke}{rgb}{0.690196,0.690196,0.690196}%
\pgfsetstrokecolor{currentstroke}%
\pgfsetdash{}{0pt}%
\pgfpathmoveto{\pgfqpoint{3.752882in}{0.823241in}}%
\pgfpathlineto{\pgfqpoint{5.450668in}{2.451285in}}%
\pgfpathlineto{\pgfqpoint{5.511348in}{4.694368in}}%
\pgfusepath{stroke}%
\end{pgfscope}%
\begin{pgfscope}%
\pgfsetbuttcap%
\pgfsetroundjoin%
\pgfsetlinewidth{0.803000pt}%
\definecolor{currentstroke}{rgb}{0.690196,0.690196,0.690196}%
\pgfsetstrokecolor{currentstroke}%
\pgfsetdash{}{0pt}%
\pgfpathmoveto{\pgfqpoint{4.147410in}{0.694388in}}%
\pgfpathlineto{\pgfqpoint{5.822060in}{2.343336in}}%
\pgfpathlineto{\pgfqpoint{5.899605in}{4.595956in}}%
\pgfusepath{stroke}%
\end{pgfscope}%
\begin{pgfscope}%
\pgfsetbuttcap%
\pgfsetroundjoin%
\pgfsetlinewidth{0.803000pt}%
\definecolor{currentstroke}{rgb}{0.690196,0.690196,0.690196}%
\pgfsetstrokecolor{currentstroke}%
\pgfsetdash{}{0pt}%
\pgfpathmoveto{\pgfqpoint{4.547248in}{0.563801in}}%
\pgfpathlineto{\pgfqpoint{6.198022in}{2.234059in}}%
\pgfpathlineto{\pgfqpoint{6.292853in}{4.496280in}}%
\pgfusepath{stroke}%
\end{pgfscope}%
\begin{pgfscope}%
\pgfsetrectcap%
\pgfsetroundjoin%
\pgfsetlinewidth{0.803000pt}%
\definecolor{currentstroke}{rgb}{0.000000,0.000000,0.000000}%
\pgfsetstrokecolor{currentstroke}%
\pgfsetdash{}{0pt}%
\pgfpathmoveto{\pgfqpoint{1.871995in}{1.455983in}}%
\pgfpathlineto{\pgfqpoint{1.824823in}{1.415976in}}%
\pgfusepath{stroke}%
\end{pgfscope}%
\begin{pgfscope}%
\definecolor{textcolor}{rgb}{0.000000,0.000000,0.000000}%
\pgfsetstrokecolor{textcolor}%
\pgfsetfillcolor{textcolor}%
\pgftext[x=1.751850in,y=1.224727in,,top]{\color{textcolor}\sffamily\fontsize{10.000000}{12.000000}\selectfont \ensuremath{-}1.0}%
\end{pgfscope}%
\begin{pgfscope}%
\pgfsetrectcap%
\pgfsetroundjoin%
\pgfsetlinewidth{0.803000pt}%
\definecolor{currentstroke}{rgb}{0.000000,0.000000,0.000000}%
\pgfsetstrokecolor{currentstroke}%
\pgfsetdash{}{0pt}%
\pgfpathmoveto{\pgfqpoint{2.241336in}{1.335475in}}%
\pgfpathlineto{\pgfqpoint{2.194656in}{1.294949in}}%
\pgfusepath{stroke}%
\end{pgfscope}%
\begin{pgfscope}%
\definecolor{textcolor}{rgb}{0.000000,0.000000,0.000000}%
\pgfsetstrokecolor{textcolor}%
\pgfsetfillcolor{textcolor}%
\pgftext[x=2.121607in,y=1.102330in,,top]{\color{textcolor}\sffamily\fontsize{10.000000}{12.000000}\selectfont \ensuremath{-}0.5}%
\end{pgfscope}%
\begin{pgfscope}%
\pgfsetrectcap%
\pgfsetroundjoin%
\pgfsetlinewidth{0.803000pt}%
\definecolor{currentstroke}{rgb}{0.000000,0.000000,0.000000}%
\pgfsetstrokecolor{currentstroke}%
\pgfsetdash{}{0pt}%
\pgfpathmoveto{\pgfqpoint{2.615482in}{1.213400in}}%
\pgfpathlineto{\pgfqpoint{2.569310in}{1.172344in}}%
\pgfusepath{stroke}%
\end{pgfscope}%
\begin{pgfscope}%
\definecolor{textcolor}{rgb}{0.000000,0.000000,0.000000}%
\pgfsetstrokecolor{textcolor}%
\pgfsetfillcolor{textcolor}%
\pgftext[x=2.496191in,y=0.978335in,,top]{\color{textcolor}\sffamily\fontsize{10.000000}{12.000000}\selectfont 0.0}%
\end{pgfscope}%
\begin{pgfscope}%
\pgfsetrectcap%
\pgfsetroundjoin%
\pgfsetlinewidth{0.803000pt}%
\definecolor{currentstroke}{rgb}{0.000000,0.000000,0.000000}%
\pgfsetstrokecolor{currentstroke}%
\pgfsetdash{}{0pt}%
\pgfpathmoveto{\pgfqpoint{2.994527in}{1.089726in}}%
\pgfpathlineto{\pgfqpoint{2.948882in}{1.048130in}}%
\pgfusepath{stroke}%
\end{pgfscope}%
\begin{pgfscope}%
\definecolor{textcolor}{rgb}{0.000000,0.000000,0.000000}%
\pgfsetstrokecolor{textcolor}%
\pgfsetfillcolor{textcolor}%
\pgftext[x=2.875695in,y=0.852712in,,top]{\color{textcolor}\sffamily\fontsize{10.000000}{12.000000}\selectfont 0.5}%
\end{pgfscope}%
\begin{pgfscope}%
\pgfsetrectcap%
\pgfsetroundjoin%
\pgfsetlinewidth{0.803000pt}%
\definecolor{currentstroke}{rgb}{0.000000,0.000000,0.000000}%
\pgfsetstrokecolor{currentstroke}%
\pgfsetdash{}{0pt}%
\pgfpathmoveto{\pgfqpoint{3.378569in}{0.964422in}}%
\pgfpathlineto{\pgfqpoint{3.333468in}{0.922276in}}%
\pgfusepath{stroke}%
\end{pgfscope}%
\begin{pgfscope}%
\definecolor{textcolor}{rgb}{0.000000,0.000000,0.000000}%
\pgfsetstrokecolor{textcolor}%
\pgfsetfillcolor{textcolor}%
\pgftext[x=3.260218in,y=0.725427in,,top]{\color{textcolor}\sffamily\fontsize{10.000000}{12.000000}\selectfont 1.0}%
\end{pgfscope}%
\begin{pgfscope}%
\pgfsetrectcap%
\pgfsetroundjoin%
\pgfsetlinewidth{0.803000pt}%
\definecolor{currentstroke}{rgb}{0.000000,0.000000,0.000000}%
\pgfsetstrokecolor{currentstroke}%
\pgfsetdash{}{0pt}%
\pgfpathmoveto{\pgfqpoint{3.767706in}{0.837456in}}%
\pgfpathlineto{\pgfqpoint{3.723168in}{0.794747in}}%
\pgfusepath{stroke}%
\end{pgfscope}%
\begin{pgfscope}%
\definecolor{textcolor}{rgb}{0.000000,0.000000,0.000000}%
\pgfsetstrokecolor{textcolor}%
\pgfsetfillcolor{textcolor}%
\pgftext[x=3.649861in,y=0.596448in,,top]{\color{textcolor}\sffamily\fontsize{10.000000}{12.000000}\selectfont 1.5}%
\end{pgfscope}%
\begin{pgfscope}%
\pgfsetrectcap%
\pgfsetroundjoin%
\pgfsetlinewidth{0.803000pt}%
\definecolor{currentstroke}{rgb}{0.000000,0.000000,0.000000}%
\pgfsetstrokecolor{currentstroke}%
\pgfsetdash{}{0pt}%
\pgfpathmoveto{\pgfqpoint{4.162040in}{0.708793in}}%
\pgfpathlineto{\pgfqpoint{4.118084in}{0.665512in}}%
\pgfusepath{stroke}%
\end{pgfscope}%
\begin{pgfscope}%
\definecolor{textcolor}{rgb}{0.000000,0.000000,0.000000}%
\pgfsetstrokecolor{textcolor}%
\pgfsetfillcolor{textcolor}%
\pgftext[x=4.044725in,y=0.465740in,,top]{\color{textcolor}\sffamily\fontsize{10.000000}{12.000000}\selectfont 2.0}%
\end{pgfscope}%
\begin{pgfscope}%
\pgfsetrectcap%
\pgfsetroundjoin%
\pgfsetlinewidth{0.803000pt}%
\definecolor{currentstroke}{rgb}{0.000000,0.000000,0.000000}%
\pgfsetstrokecolor{currentstroke}%
\pgfsetdash{}{0pt}%
\pgfpathmoveto{\pgfqpoint{4.561678in}{0.578401in}}%
\pgfpathlineto{\pgfqpoint{4.518323in}{0.534535in}}%
\pgfusepath{stroke}%
\end{pgfscope}%
\begin{pgfscope}%
\definecolor{textcolor}{rgb}{0.000000,0.000000,0.000000}%
\pgfsetstrokecolor{textcolor}%
\pgfsetfillcolor{textcolor}%
\pgftext[x=4.444916in,y=0.333269in,,top]{\color{textcolor}\sffamily\fontsize{10.000000}{12.000000}\selectfont 2.5}%
\end{pgfscope}%
\begin{pgfscope}%
\pgfsetrectcap%
\pgfsetroundjoin%
\pgfsetlinewidth{0.803000pt}%
\definecolor{currentstroke}{rgb}{0.000000,0.000000,0.000000}%
\pgfsetstrokecolor{currentstroke}%
\pgfsetdash{}{0pt}%
\pgfpathmoveto{\pgfqpoint{6.391709in}{2.177762in}}%
\pgfpathlineto{\pgfqpoint{4.753413in}{0.496467in}}%
\pgfusepath{stroke}%
\end{pgfscope}%
\begin{pgfscope}%
\definecolor{textcolor}{rgb}{0.000000,0.000000,0.000000}%
\pgfsetstrokecolor{textcolor}%
\pgfsetfillcolor{textcolor}%
\pgftext[x=5.983676in,y=0.985873in,,]{\color{textcolor}\sffamily\fontsize{10.000000}{12.000000}\selectfont y}%
\end{pgfscope}%
\begin{pgfscope}%
\pgfsetbuttcap%
\pgfsetroundjoin%
\pgfsetlinewidth{0.803000pt}%
\definecolor{currentstroke}{rgb}{0.690196,0.690196,0.690196}%
\pgfsetstrokecolor{currentstroke}%
\pgfsetdash{}{0pt}%
\pgfpathmoveto{\pgfqpoint{1.688926in}{3.924526in}}%
\pgfpathlineto{\pgfqpoint{1.794447in}{1.608387in}}%
\pgfpathlineto{\pgfqpoint{4.866770in}{0.612800in}}%
\pgfusepath{stroke}%
\end{pgfscope}%
\begin{pgfscope}%
\pgfsetbuttcap%
\pgfsetroundjoin%
\pgfsetlinewidth{0.803000pt}%
\definecolor{currentstroke}{rgb}{0.690196,0.690196,0.690196}%
\pgfsetstrokecolor{currentstroke}%
\pgfsetdash{}{0pt}%
\pgfpathmoveto{\pgfqpoint{1.941873in}{4.109116in}}%
\pgfpathlineto{\pgfqpoint{2.035175in}{1.810170in}}%
\pgfpathlineto{\pgfqpoint{5.085004in}{0.836761in}}%
\pgfusepath{stroke}%
\end{pgfscope}%
\begin{pgfscope}%
\pgfsetbuttcap%
\pgfsetroundjoin%
\pgfsetlinewidth{0.803000pt}%
\definecolor{currentstroke}{rgb}{0.690196,0.690196,0.690196}%
\pgfsetstrokecolor{currentstroke}%
\pgfsetdash{}{0pt}%
\pgfpathmoveto{\pgfqpoint{2.189113in}{4.289540in}}%
\pgfpathlineto{\pgfqpoint{2.270706in}{2.007597in}}%
\pgfpathlineto{\pgfqpoint{5.298278in}{1.055633in}}%
\pgfusepath{stroke}%
\end{pgfscope}%
\begin{pgfscope}%
\pgfsetbuttcap%
\pgfsetroundjoin%
\pgfsetlinewidth{0.803000pt}%
\definecolor{currentstroke}{rgb}{0.690196,0.690196,0.690196}%
\pgfsetstrokecolor{currentstroke}%
\pgfsetdash{}{0pt}%
\pgfpathmoveto{\pgfqpoint{2.430837in}{4.465939in}}%
\pgfpathlineto{\pgfqpoint{2.501208in}{2.200808in}}%
\pgfpathlineto{\pgfqpoint{5.506762in}{1.269588in}}%
\pgfusepath{stroke}%
\end{pgfscope}%
\begin{pgfscope}%
\pgfsetbuttcap%
\pgfsetroundjoin%
\pgfsetlinewidth{0.803000pt}%
\definecolor{currentstroke}{rgb}{0.690196,0.690196,0.690196}%
\pgfsetstrokecolor{currentstroke}%
\pgfsetdash{}{0pt}%
\pgfpathmoveto{\pgfqpoint{2.667228in}{4.638447in}}%
\pgfpathlineto{\pgfqpoint{2.726839in}{2.389937in}}%
\pgfpathlineto{\pgfqpoint{5.710613in}{1.478790in}}%
\pgfusepath{stroke}%
\end{pgfscope}%
\begin{pgfscope}%
\pgfsetbuttcap%
\pgfsetroundjoin%
\pgfsetlinewidth{0.803000pt}%
\definecolor{currentstroke}{rgb}{0.690196,0.690196,0.690196}%
\pgfsetstrokecolor{currentstroke}%
\pgfsetdash{}{0pt}%
\pgfpathmoveto{\pgfqpoint{2.898459in}{4.807189in}}%
\pgfpathlineto{\pgfqpoint{2.947753in}{2.575111in}}%
\pgfpathlineto{\pgfqpoint{5.909986in}{1.683395in}}%
\pgfusepath{stroke}%
\end{pgfscope}%
\begin{pgfscope}%
\pgfsetbuttcap%
\pgfsetroundjoin%
\pgfsetlinewidth{0.803000pt}%
\definecolor{currentstroke}{rgb}{0.690196,0.690196,0.690196}%
\pgfsetstrokecolor{currentstroke}%
\pgfsetdash{}{0pt}%
\pgfpathmoveto{\pgfqpoint{3.124699in}{4.972288in}}%
\pgfpathlineto{\pgfqpoint{3.164096in}{2.756454in}}%
\pgfpathlineto{\pgfqpoint{6.105025in}{1.883554in}}%
\pgfusepath{stroke}%
\end{pgfscope}%
\begin{pgfscope}%
\pgfsetbuttcap%
\pgfsetroundjoin%
\pgfsetlinewidth{0.803000pt}%
\definecolor{currentstroke}{rgb}{0.690196,0.690196,0.690196}%
\pgfsetstrokecolor{currentstroke}%
\pgfsetdash{}{0pt}%
\pgfpathmoveto{\pgfqpoint{3.346107in}{5.133862in}}%
\pgfpathlineto{\pgfqpoint{3.376007in}{2.934082in}}%
\pgfpathlineto{\pgfqpoint{6.295871in}{2.079408in}}%
\pgfusepath{stroke}%
\end{pgfscope}%
\begin{pgfscope}%
\pgfsetrectcap%
\pgfsetroundjoin%
\pgfsetlinewidth{0.803000pt}%
\definecolor{currentstroke}{rgb}{0.000000,0.000000,0.000000}%
\pgfsetstrokecolor{currentstroke}%
\pgfsetdash{}{0pt}%
\pgfpathmoveto{\pgfqpoint{4.840880in}{0.621189in}}%
\pgfpathlineto{\pgfqpoint{4.918618in}{0.595998in}}%
\pgfusepath{stroke}%
\end{pgfscope}%
\begin{pgfscope}%
\definecolor{textcolor}{rgb}{0.000000,0.000000,0.000000}%
\pgfsetstrokecolor{textcolor}%
\pgfsetfillcolor{textcolor}%
\pgftext[x=5.045633in,y=0.426401in,,top]{\color{textcolor}\sffamily\fontsize{10.000000}{12.000000}\selectfont \ensuremath{-}2.5}%
\end{pgfscope}%
\begin{pgfscope}%
\pgfsetrectcap%
\pgfsetroundjoin%
\pgfsetlinewidth{0.803000pt}%
\definecolor{currentstroke}{rgb}{0.000000,0.000000,0.000000}%
\pgfsetstrokecolor{currentstroke}%
\pgfsetdash{}{0pt}%
\pgfpathmoveto{\pgfqpoint{5.059318in}{0.844959in}}%
\pgfpathlineto{\pgfqpoint{5.136441in}{0.820343in}}%
\pgfusepath{stroke}%
\end{pgfscope}%
\begin{pgfscope}%
\definecolor{textcolor}{rgb}{0.000000,0.000000,0.000000}%
\pgfsetstrokecolor{textcolor}%
\pgfsetfillcolor{textcolor}%
\pgftext[x=5.261897in,y=0.652595in,,top]{\color{textcolor}\sffamily\fontsize{10.000000}{12.000000}\selectfont \ensuremath{-}2.0}%
\end{pgfscope}%
\begin{pgfscope}%
\pgfsetrectcap%
\pgfsetroundjoin%
\pgfsetlinewidth{0.803000pt}%
\definecolor{currentstroke}{rgb}{0.000000,0.000000,0.000000}%
\pgfsetstrokecolor{currentstroke}%
\pgfsetdash{}{0pt}%
\pgfpathmoveto{\pgfqpoint{5.272794in}{1.063646in}}%
\pgfpathlineto{\pgfqpoint{5.349311in}{1.039587in}}%
\pgfusepath{stroke}%
\end{pgfscope}%
\begin{pgfscope}%
\definecolor{textcolor}{rgb}{0.000000,0.000000,0.000000}%
\pgfsetstrokecolor{textcolor}%
\pgfsetfillcolor{textcolor}%
\pgftext[x=5.473245in,y=0.873646in,,top]{\color{textcolor}\sffamily\fontsize{10.000000}{12.000000}\selectfont \ensuremath{-}1.5}%
\end{pgfscope}%
\begin{pgfscope}%
\pgfsetrectcap%
\pgfsetroundjoin%
\pgfsetlinewidth{0.803000pt}%
\definecolor{currentstroke}{rgb}{0.000000,0.000000,0.000000}%
\pgfsetstrokecolor{currentstroke}%
\pgfsetdash{}{0pt}%
\pgfpathmoveto{\pgfqpoint{5.481477in}{1.277422in}}%
\pgfpathlineto{\pgfqpoint{5.557394in}{1.253901in}}%
\pgfusepath{stroke}%
\end{pgfscope}%
\begin{pgfscope}%
\definecolor{textcolor}{rgb}{0.000000,0.000000,0.000000}%
\pgfsetstrokecolor{textcolor}%
\pgfsetfillcolor{textcolor}%
\pgftext[x=5.679844in,y=1.089730in,,top]{\color{textcolor}\sffamily\fontsize{10.000000}{12.000000}\selectfont \ensuremath{-}1.0}%
\end{pgfscope}%
\begin{pgfscope}%
\pgfsetrectcap%
\pgfsetroundjoin%
\pgfsetlinewidth{0.803000pt}%
\definecolor{currentstroke}{rgb}{0.000000,0.000000,0.000000}%
\pgfsetstrokecolor{currentstroke}%
\pgfsetdash{}{0pt}%
\pgfpathmoveto{\pgfqpoint{5.685525in}{1.486451in}}%
\pgfpathlineto{\pgfqpoint{5.760851in}{1.463449in}}%
\pgfusepath{stroke}%
\end{pgfscope}%
\begin{pgfscope}%
\definecolor{textcolor}{rgb}{0.000000,0.000000,0.000000}%
\pgfsetstrokecolor{textcolor}%
\pgfsetfillcolor{textcolor}%
\pgftext[x=5.881850in,y=1.301011in,,top]{\color{textcolor}\sffamily\fontsize{10.000000}{12.000000}\selectfont \ensuremath{-}0.5}%
\end{pgfscope}%
\begin{pgfscope}%
\pgfsetrectcap%
\pgfsetroundjoin%
\pgfsetlinewidth{0.803000pt}%
\definecolor{currentstroke}{rgb}{0.000000,0.000000,0.000000}%
\pgfsetstrokecolor{currentstroke}%
\pgfsetdash{}{0pt}%
\pgfpathmoveto{\pgfqpoint{5.885092in}{1.690889in}}%
\pgfpathlineto{\pgfqpoint{5.959833in}{1.668390in}}%
\pgfusepath{stroke}%
\end{pgfscope}%
\begin{pgfscope}%
\definecolor{textcolor}{rgb}{0.000000,0.000000,0.000000}%
\pgfsetstrokecolor{textcolor}%
\pgfsetfillcolor{textcolor}%
\pgftext[x=6.079417in,y=1.507648in,,top]{\color{textcolor}\sffamily\fontsize{10.000000}{12.000000}\selectfont 0.0}%
\end{pgfscope}%
\begin{pgfscope}%
\pgfsetrectcap%
\pgfsetroundjoin%
\pgfsetlinewidth{0.803000pt}%
\definecolor{currentstroke}{rgb}{0.000000,0.000000,0.000000}%
\pgfsetstrokecolor{currentstroke}%
\pgfsetdash{}{0pt}%
\pgfpathmoveto{\pgfqpoint{6.080323in}{1.890885in}}%
\pgfpathlineto{\pgfqpoint{6.154488in}{1.868872in}}%
\pgfusepath{stroke}%
\end{pgfscope}%
\begin{pgfscope}%
\definecolor{textcolor}{rgb}{0.000000,0.000000,0.000000}%
\pgfsetstrokecolor{textcolor}%
\pgfsetfillcolor{textcolor}%
\pgftext[x=6.272688in,y=1.709792in,,top]{\color{textcolor}\sffamily\fontsize{10.000000}{12.000000}\selectfont 0.5}%
\end{pgfscope}%
\begin{pgfscope}%
\pgfsetrectcap%
\pgfsetroundjoin%
\pgfsetlinewidth{0.803000pt}%
\definecolor{currentstroke}{rgb}{0.000000,0.000000,0.000000}%
\pgfsetstrokecolor{currentstroke}%
\pgfsetdash{}{0pt}%
\pgfpathmoveto{\pgfqpoint{6.271359in}{2.086583in}}%
\pgfpathlineto{\pgfqpoint{6.344953in}{2.065041in}}%
\pgfusepath{stroke}%
\end{pgfscope}%
\begin{pgfscope}%
\definecolor{textcolor}{rgb}{0.000000,0.000000,0.000000}%
\pgfsetstrokecolor{textcolor}%
\pgfsetfillcolor{textcolor}%
\pgftext[x=6.461802in,y=1.907589in,,top]{\color{textcolor}\sffamily\fontsize{10.000000}{12.000000}\selectfont 1.0}%
\end{pgfscope}%
\begin{pgfscope}%
\pgfsetrectcap%
\pgfsetroundjoin%
\pgfsetlinewidth{0.803000pt}%
\definecolor{currentstroke}{rgb}{0.000000,0.000000,0.000000}%
\pgfsetstrokecolor{currentstroke}%
\pgfsetdash{}{0pt}%
\pgfpathmoveto{\pgfqpoint{6.391709in}{2.177762in}}%
\pgfpathlineto{\pgfqpoint{6.495528in}{4.444907in}}%
\pgfusepath{stroke}%
\end{pgfscope}%
\begin{pgfscope}%
\definecolor{textcolor}{rgb}{0.000000,0.000000,0.000000}%
\pgfsetstrokecolor{textcolor}%
\pgfsetfillcolor{textcolor}%
\pgftext[x=7.004475in,y=3.361793in,,,rotate=87.378092]{\color{textcolor}\sffamily\fontsize{10.000000}{12.000000}\selectfont f(x,y)}%
\end{pgfscope}%
\begin{pgfscope}%
\pgfsetbuttcap%
\pgfsetroundjoin%
\pgfsetlinewidth{0.803000pt}%
\definecolor{currentstroke}{rgb}{0.690196,0.690196,0.690196}%
\pgfsetstrokecolor{currentstroke}%
\pgfsetdash{}{0pt}%
\pgfpathmoveto{\pgfqpoint{6.395461in}{2.259700in}}%
\pgfpathlineto{\pgfqpoint{3.481594in}{3.102723in}}%
\pgfpathlineto{\pgfqpoint{1.665476in}{1.587560in}}%
\pgfusepath{stroke}%
\end{pgfscope}%
\begin{pgfscope}%
\pgfsetbuttcap%
\pgfsetroundjoin%
\pgfsetlinewidth{0.803000pt}%
\definecolor{currentstroke}{rgb}{0.690196,0.690196,0.690196}%
\pgfsetstrokecolor{currentstroke}%
\pgfsetdash{}{0pt}%
\pgfpathmoveto{\pgfqpoint{6.412058in}{2.622144in}}%
\pgfpathlineto{\pgfqpoint{3.477558in}{3.453668in}}%
\pgfpathlineto{\pgfqpoint{1.647600in}{1.958771in}}%
\pgfusepath{stroke}%
\end{pgfscope}%
\begin{pgfscope}%
\pgfsetbuttcap%
\pgfsetroundjoin%
\pgfsetlinewidth{0.803000pt}%
\definecolor{currentstroke}{rgb}{0.690196,0.690196,0.690196}%
\pgfsetstrokecolor{currentstroke}%
\pgfsetdash{}{0pt}%
\pgfpathmoveto{\pgfqpoint{6.428895in}{2.989809in}}%
\pgfpathlineto{\pgfqpoint{3.473467in}{3.809425in}}%
\pgfpathlineto{\pgfqpoint{1.629457in}{2.335533in}}%
\pgfusepath{stroke}%
\end{pgfscope}%
\begin{pgfscope}%
\pgfsetbuttcap%
\pgfsetroundjoin%
\pgfsetlinewidth{0.803000pt}%
\definecolor{currentstroke}{rgb}{0.690196,0.690196,0.690196}%
\pgfsetstrokecolor{currentstroke}%
\pgfsetdash{}{0pt}%
\pgfpathmoveto{\pgfqpoint{6.445976in}{3.362808in}}%
\pgfpathlineto{\pgfqpoint{3.469320in}{4.170094in}}%
\pgfpathlineto{\pgfqpoint{1.611040in}{2.717971in}}%
\pgfusepath{stroke}%
\end{pgfscope}%
\begin{pgfscope}%
\pgfsetbuttcap%
\pgfsetroundjoin%
\pgfsetlinewidth{0.803000pt}%
\definecolor{currentstroke}{rgb}{0.690196,0.690196,0.690196}%
\pgfsetstrokecolor{currentstroke}%
\pgfsetdash{}{0pt}%
\pgfpathmoveto{\pgfqpoint{6.463306in}{3.741258in}}%
\pgfpathlineto{\pgfqpoint{3.465115in}{4.535779in}}%
\pgfpathlineto{\pgfqpoint{1.592343in}{3.106216in}}%
\pgfusepath{stroke}%
\end{pgfscope}%
\begin{pgfscope}%
\pgfsetbuttcap%
\pgfsetroundjoin%
\pgfsetlinewidth{0.803000pt}%
\definecolor{currentstroke}{rgb}{0.690196,0.690196,0.690196}%
\pgfsetstrokecolor{currentstroke}%
\pgfsetdash{}{0pt}%
\pgfpathmoveto{\pgfqpoint{6.480891in}{4.125280in}}%
\pgfpathlineto{\pgfqpoint{3.460851in}{4.906583in}}%
\pgfpathlineto{\pgfqpoint{1.573361in}{3.500399in}}%
\pgfusepath{stroke}%
\end{pgfscope}%
\begin{pgfscope}%
\pgfsetrectcap%
\pgfsetroundjoin%
\pgfsetlinewidth{0.803000pt}%
\definecolor{currentstroke}{rgb}{0.000000,0.000000,0.000000}%
\pgfsetstrokecolor{currentstroke}%
\pgfsetdash{}{0pt}%
\pgfpathmoveto{\pgfqpoint{6.371004in}{2.266776in}}%
\pgfpathlineto{\pgfqpoint{6.444434in}{2.245531in}}%
\pgfusepath{stroke}%
\end{pgfscope}%
\begin{pgfscope}%
\definecolor{textcolor}{rgb}{0.000000,0.000000,0.000000}%
\pgfsetstrokecolor{textcolor}%
\pgfsetfillcolor{textcolor}%
\pgftext[x=6.649565in,y=2.295809in,,top]{\color{textcolor}\sffamily\fontsize{10.000000}{12.000000}\selectfont 0}%
\end{pgfscope}%
\begin{pgfscope}%
\pgfsetrectcap%
\pgfsetroundjoin%
\pgfsetlinewidth{0.803000pt}%
\definecolor{currentstroke}{rgb}{0.000000,0.000000,0.000000}%
\pgfsetstrokecolor{currentstroke}%
\pgfsetdash{}{0pt}%
\pgfpathmoveto{\pgfqpoint{6.387420in}{2.629126in}}%
\pgfpathlineto{\pgfqpoint{6.461395in}{2.608164in}}%
\pgfusepath{stroke}%
\end{pgfscope}%
\begin{pgfscope}%
\definecolor{textcolor}{rgb}{0.000000,0.000000,0.000000}%
\pgfsetstrokecolor{textcolor}%
\pgfsetfillcolor{textcolor}%
\pgftext[x=6.667942in,y=2.657772in,,top]{\color{textcolor}\sffamily\fontsize{10.000000}{12.000000}\selectfont 10}%
\end{pgfscope}%
\begin{pgfscope}%
\pgfsetrectcap%
\pgfsetroundjoin%
\pgfsetlinewidth{0.803000pt}%
\definecolor{currentstroke}{rgb}{0.000000,0.000000,0.000000}%
\pgfsetstrokecolor{currentstroke}%
\pgfsetdash{}{0pt}%
\pgfpathmoveto{\pgfqpoint{6.404072in}{2.996693in}}%
\pgfpathlineto{\pgfqpoint{6.478600in}{2.976025in}}%
\pgfusepath{stroke}%
\end{pgfscope}%
\begin{pgfscope}%
\definecolor{textcolor}{rgb}{0.000000,0.000000,0.000000}%
\pgfsetstrokecolor{textcolor}%
\pgfsetfillcolor{textcolor}%
\pgftext[x=6.686583in,y=3.024938in,,top]{\color{textcolor}\sffamily\fontsize{10.000000}{12.000000}\selectfont 20}%
\end{pgfscope}%
\begin{pgfscope}%
\pgfsetrectcap%
\pgfsetroundjoin%
\pgfsetlinewidth{0.803000pt}%
\definecolor{currentstroke}{rgb}{0.000000,0.000000,0.000000}%
\pgfsetstrokecolor{currentstroke}%
\pgfsetdash{}{0pt}%
\pgfpathmoveto{\pgfqpoint{6.420966in}{3.369591in}}%
\pgfpathlineto{\pgfqpoint{6.496056in}{3.349226in}}%
\pgfusepath{stroke}%
\end{pgfscope}%
\begin{pgfscope}%
\definecolor{textcolor}{rgb}{0.000000,0.000000,0.000000}%
\pgfsetstrokecolor{textcolor}%
\pgfsetfillcolor{textcolor}%
\pgftext[x=6.705494in,y=3.397420in,,top]{\color{textcolor}\sffamily\fontsize{10.000000}{12.000000}\selectfont 30}%
\end{pgfscope}%
\begin{pgfscope}%
\pgfsetrectcap%
\pgfsetroundjoin%
\pgfsetlinewidth{0.803000pt}%
\definecolor{currentstroke}{rgb}{0.000000,0.000000,0.000000}%
\pgfsetstrokecolor{currentstroke}%
\pgfsetdash{}{0pt}%
\pgfpathmoveto{\pgfqpoint{6.438106in}{3.747936in}}%
\pgfpathlineto{\pgfqpoint{6.513767in}{3.727886in}}%
\pgfusepath{stroke}%
\end{pgfscope}%
\begin{pgfscope}%
\definecolor{textcolor}{rgb}{0.000000,0.000000,0.000000}%
\pgfsetstrokecolor{textcolor}%
\pgfsetfillcolor{textcolor}%
\pgftext[x=6.724680in,y=3.775335in,,top]{\color{textcolor}\sffamily\fontsize{10.000000}{12.000000}\selectfont 40}%
\end{pgfscope}%
\begin{pgfscope}%
\pgfsetrectcap%
\pgfsetroundjoin%
\pgfsetlinewidth{0.803000pt}%
\definecolor{currentstroke}{rgb}{0.000000,0.000000,0.000000}%
\pgfsetstrokecolor{currentstroke}%
\pgfsetdash{}{0pt}%
\pgfpathmoveto{\pgfqpoint{6.455499in}{4.131849in}}%
\pgfpathlineto{\pgfqpoint{6.531738in}{4.112125in}}%
\pgfusepath{stroke}%
\end{pgfscope}%
\begin{pgfscope}%
\definecolor{textcolor}{rgb}{0.000000,0.000000,0.000000}%
\pgfsetstrokecolor{textcolor}%
\pgfsetfillcolor{textcolor}%
\pgftext[x=6.744149in,y=4.158800in,,top]{\color{textcolor}\sffamily\fontsize{10.000000}{12.000000}\selectfont 50}%
\end{pgfscope}%
\begin{pgfscope}%
\pgfpathrectangle{\pgfqpoint{1.254980in}{0.150000in}}{\pgfqpoint{5.490039in}{5.490039in}}%
\pgfusepath{clip}%
\pgfsetrectcap%
\pgfsetroundjoin%
\pgfsetlinewidth{1.505625pt}%
\definecolor{currentstroke}{rgb}{1.000000,0.000000,0.000000}%
\pgfsetstrokecolor{currentstroke}%
\pgfsetdash{}{0pt}%
\pgfpathmoveto{\pgfqpoint{4.488799in}{1.348976in}}%
\pgfpathlineto{\pgfqpoint{5.070138in}{1.636513in}}%
\pgfpathlineto{\pgfqpoint{4.477971in}{1.427694in}}%
\pgfpathlineto{\pgfqpoint{3.236719in}{2.270511in}}%
\pgfpathlineto{\pgfqpoint{3.933625in}{2.351871in}}%
\pgfpathlineto{\pgfqpoint{3.668641in}{2.241280in}}%
\pgfpathlineto{\pgfqpoint{3.824376in}{2.082635in}}%
\pgfpathlineto{\pgfqpoint{3.798216in}{2.063487in}}%
\pgfpathlineto{\pgfqpoint{3.826479in}{2.072829in}}%
\pgfpathlineto{\pgfqpoint{3.805659in}{2.076070in}}%
\pgfpathlineto{\pgfqpoint{3.806914in}{2.073167in}}%
\pgfpathlineto{\pgfqpoint{3.809839in}{2.071740in}}%
\pgfpathlineto{\pgfqpoint{3.808116in}{2.072464in}}%
\pgfpathlineto{\pgfqpoint{3.807911in}{2.072664in}}%
\pgfpathlineto{\pgfqpoint{3.809183in}{2.073021in}}%
\pgfpathlineto{\pgfqpoint{3.808489in}{2.072864in}}%
\pgfusepath{stroke}%
\end{pgfscope}%
\begin{pgfscope}%
\pgfpathrectangle{\pgfqpoint{1.254980in}{0.150000in}}{\pgfqpoint{5.490039in}{5.490039in}}%
\pgfusepath{clip}%
\pgfsetbuttcap%
\pgfsetroundjoin%
\definecolor{currentfill}{rgb}{1.000000,0.000000,0.000000}%
\pgfsetfillcolor{currentfill}%
\pgfsetlinewidth{1.003750pt}%
\definecolor{currentstroke}{rgb}{1.000000,0.000000,0.000000}%
\pgfsetstrokecolor{currentstroke}%
\pgfsetdash{}{0pt}%
\pgfsys@defobject{currentmarker}{\pgfqpoint{-0.041667in}{-0.041667in}}{\pgfqpoint{0.041667in}{0.041667in}}{%
\pgfpathmoveto{\pgfqpoint{0.000000in}{-0.041667in}}%
\pgfpathcurveto{\pgfqpoint{0.011050in}{-0.041667in}}{\pgfqpoint{0.021649in}{-0.037276in}}{\pgfqpoint{0.029463in}{-0.029463in}}%
\pgfpathcurveto{\pgfqpoint{0.037276in}{-0.021649in}}{\pgfqpoint{0.041667in}{-0.011050in}}{\pgfqpoint{0.041667in}{0.000000in}}%
\pgfpathcurveto{\pgfqpoint{0.041667in}{0.011050in}}{\pgfqpoint{0.037276in}{0.021649in}}{\pgfqpoint{0.029463in}{0.029463in}}%
\pgfpathcurveto{\pgfqpoint{0.021649in}{0.037276in}}{\pgfqpoint{0.011050in}{0.041667in}}{\pgfqpoint{0.000000in}{0.041667in}}%
\pgfpathcurveto{\pgfqpoint{-0.011050in}{0.041667in}}{\pgfqpoint{-0.021649in}{0.037276in}}{\pgfqpoint{-0.029463in}{0.029463in}}%
\pgfpathcurveto{\pgfqpoint{-0.037276in}{0.021649in}}{\pgfqpoint{-0.041667in}{0.011050in}}{\pgfqpoint{-0.041667in}{0.000000in}}%
\pgfpathcurveto{\pgfqpoint{-0.041667in}{-0.011050in}}{\pgfqpoint{-0.037276in}{-0.021649in}}{\pgfqpoint{-0.029463in}{-0.029463in}}%
\pgfpathcurveto{\pgfqpoint{-0.021649in}{-0.037276in}}{\pgfqpoint{-0.011050in}{-0.041667in}}{\pgfqpoint{0.000000in}{-0.041667in}}%
\pgfpathlineto{\pgfqpoint{0.000000in}{-0.041667in}}%
\pgfpathclose%
\pgfusepath{stroke,fill}%
}%
\begin{pgfscope}%
\pgfsys@transformshift{4.488799in}{1.348976in}%
\pgfsys@useobject{currentmarker}{}%
\end{pgfscope}%
\begin{pgfscope}%
\pgfsys@transformshift{5.070138in}{1.636513in}%
\pgfsys@useobject{currentmarker}{}%
\end{pgfscope}%
\begin{pgfscope}%
\pgfsys@transformshift{4.477971in}{1.427694in}%
\pgfsys@useobject{currentmarker}{}%
\end{pgfscope}%
\begin{pgfscope}%
\pgfsys@transformshift{3.236719in}{2.270511in}%
\pgfsys@useobject{currentmarker}{}%
\end{pgfscope}%
\begin{pgfscope}%
\pgfsys@transformshift{3.933625in}{2.351871in}%
\pgfsys@useobject{currentmarker}{}%
\end{pgfscope}%
\begin{pgfscope}%
\pgfsys@transformshift{3.668641in}{2.241280in}%
\pgfsys@useobject{currentmarker}{}%
\end{pgfscope}%
\begin{pgfscope}%
\pgfsys@transformshift{3.824376in}{2.082635in}%
\pgfsys@useobject{currentmarker}{}%
\end{pgfscope}%
\begin{pgfscope}%
\pgfsys@transformshift{3.798216in}{2.063487in}%
\pgfsys@useobject{currentmarker}{}%
\end{pgfscope}%
\begin{pgfscope}%
\pgfsys@transformshift{3.826479in}{2.072829in}%
\pgfsys@useobject{currentmarker}{}%
\end{pgfscope}%
\begin{pgfscope}%
\pgfsys@transformshift{3.805659in}{2.076070in}%
\pgfsys@useobject{currentmarker}{}%
\end{pgfscope}%
\begin{pgfscope}%
\pgfsys@transformshift{3.806914in}{2.073167in}%
\pgfsys@useobject{currentmarker}{}%
\end{pgfscope}%
\begin{pgfscope}%
\pgfsys@transformshift{3.809839in}{2.071740in}%
\pgfsys@useobject{currentmarker}{}%
\end{pgfscope}%
\begin{pgfscope}%
\pgfsys@transformshift{3.808116in}{2.072464in}%
\pgfsys@useobject{currentmarker}{}%
\end{pgfscope}%
\begin{pgfscope}%
\pgfsys@transformshift{3.807911in}{2.072664in}%
\pgfsys@useobject{currentmarker}{}%
\end{pgfscope}%
\begin{pgfscope}%
\pgfsys@transformshift{3.809183in}{2.073021in}%
\pgfsys@useobject{currentmarker}{}%
\end{pgfscope}%
\begin{pgfscope}%
\pgfsys@transformshift{3.808489in}{2.072864in}%
\pgfsys@useobject{currentmarker}{}%
\end{pgfscope}%
\end{pgfscope}%
\begin{pgfscope}%
\pgfpathrectangle{\pgfqpoint{1.254980in}{0.150000in}}{\pgfqpoint{5.490039in}{5.490039in}}%
\pgfusepath{clip}%
\pgfsetbuttcap%
\pgfsetroundjoin%
\definecolor{currentfill}{rgb}{0.000000,0.000000,1.000000}%
\pgfsetfillcolor{currentfill}%
\pgfsetlinewidth{1.003750pt}%
\definecolor{currentstroke}{rgb}{0.000000,0.000000,1.000000}%
\pgfsetstrokecolor{currentstroke}%
\pgfsetdash{}{0pt}%
\pgfsys@defobject{currentmarker}{\pgfqpoint{-0.069444in}{-0.069444in}}{\pgfqpoint{0.069444in}{0.069444in}}{%
\pgfpathmoveto{\pgfqpoint{0.000000in}{-0.069444in}}%
\pgfpathcurveto{\pgfqpoint{0.018417in}{-0.069444in}}{\pgfqpoint{0.036082in}{-0.062127in}}{\pgfqpoint{0.049105in}{-0.049105in}}%
\pgfpathcurveto{\pgfqpoint{0.062127in}{-0.036082in}}{\pgfqpoint{0.069444in}{-0.018417in}}{\pgfqpoint{0.069444in}{0.000000in}}%
\pgfpathcurveto{\pgfqpoint{0.069444in}{0.018417in}}{\pgfqpoint{0.062127in}{0.036082in}}{\pgfqpoint{0.049105in}{0.049105in}}%
\pgfpathcurveto{\pgfqpoint{0.036082in}{0.062127in}}{\pgfqpoint{0.018417in}{0.069444in}}{\pgfqpoint{0.000000in}{0.069444in}}%
\pgfpathcurveto{\pgfqpoint{-0.018417in}{0.069444in}}{\pgfqpoint{-0.036082in}{0.062127in}}{\pgfqpoint{-0.049105in}{0.049105in}}%
\pgfpathcurveto{\pgfqpoint{-0.062127in}{0.036082in}}{\pgfqpoint{-0.069444in}{0.018417in}}{\pgfqpoint{-0.069444in}{0.000000in}}%
\pgfpathcurveto{\pgfqpoint{-0.069444in}{-0.018417in}}{\pgfqpoint{-0.062127in}{-0.036082in}}{\pgfqpoint{-0.049105in}{-0.049105in}}%
\pgfpathcurveto{\pgfqpoint{-0.036082in}{-0.062127in}}{\pgfqpoint{-0.018417in}{-0.069444in}}{\pgfqpoint{0.000000in}{-0.069444in}}%
\pgfpathlineto{\pgfqpoint{0.000000in}{-0.069444in}}%
\pgfpathclose%
\pgfusepath{stroke,fill}%
}%
\begin{pgfscope}%
\pgfsys@transformshift{3.808489in}{2.072864in}%
\pgfsys@useobject{currentmarker}{}%
\end{pgfscope}%
\end{pgfscope}%
\begin{pgfscope}%
\pgfpathrectangle{\pgfqpoint{1.254980in}{0.150000in}}{\pgfqpoint{5.490039in}{5.490039in}}%
\pgfusepath{clip}%
\pgfsetbuttcap%
\pgfsetroundjoin%
\definecolor{currentfill}{rgb}{0.278826,0.175490,0.483397}%
\pgfsetfillcolor{currentfill}%
\pgfsetfillopacity{0.700000}%
\pgfsetlinewidth{0.000000pt}%
\definecolor{currentstroke}{rgb}{0.000000,0.000000,0.000000}%
\pgfsetstrokecolor{currentstroke}%
\pgfsetdash{}{0pt}%
\pgfpathmoveto{\pgfqpoint{3.549714in}{3.259336in}}%
\pgfpathlineto{\pgfqpoint{3.562113in}{3.252255in}}%
\pgfpathlineto{\pgfqpoint{3.574515in}{3.245224in}}%
\pgfpathlineto{\pgfqpoint{3.586920in}{3.238242in}}%
\pgfpathlineto{\pgfqpoint{3.599329in}{3.231307in}}%
\pgfpathlineto{\pgfqpoint{3.592008in}{3.218752in}}%
\pgfpathlineto{\pgfqpoint{3.584682in}{3.206388in}}%
\pgfpathlineto{\pgfqpoint{3.577351in}{3.194209in}}%
\pgfpathlineto{\pgfqpoint{3.570015in}{3.182212in}}%
\pgfpathlineto{\pgfqpoint{3.557600in}{3.188991in}}%
\pgfpathlineto{\pgfqpoint{3.545189in}{3.195818in}}%
\pgfpathlineto{\pgfqpoint{3.532780in}{3.202694in}}%
\pgfpathlineto{\pgfqpoint{3.520375in}{3.209619in}}%
\pgfpathlineto{\pgfqpoint{3.527718in}{3.221767in}}%
\pgfpathlineto{\pgfqpoint{3.535056in}{3.234099in}}%
\pgfpathlineto{\pgfqpoint{3.542388in}{3.246620in}}%
\pgfpathlineto{\pgfqpoint{3.549714in}{3.259336in}}%
\pgfpathclose%
\pgfusepath{fill}%
\end{pgfscope}%
\begin{pgfscope}%
\pgfpathrectangle{\pgfqpoint{1.254980in}{0.150000in}}{\pgfqpoint{5.490039in}{5.490039in}}%
\pgfusepath{clip}%
\pgfsetbuttcap%
\pgfsetroundjoin%
\definecolor{currentfill}{rgb}{0.280255,0.165693,0.476498}%
\pgfsetfillcolor{currentfill}%
\pgfsetfillopacity{0.700000}%
\pgfsetlinewidth{0.000000pt}%
\definecolor{currentstroke}{rgb}{0.000000,0.000000,0.000000}%
\pgfsetstrokecolor{currentstroke}%
\pgfsetdash{}{0pt}%
\pgfpathmoveto{\pgfqpoint{3.599329in}{3.231307in}}%
\pgfpathlineto{\pgfqpoint{3.611740in}{3.224421in}}%
\pgfpathlineto{\pgfqpoint{3.624154in}{3.217581in}}%
\pgfpathlineto{\pgfqpoint{3.636572in}{3.210789in}}%
\pgfpathlineto{\pgfqpoint{3.648993in}{3.204043in}}%
\pgfpathlineto{\pgfqpoint{3.641679in}{3.191647in}}%
\pgfpathlineto{\pgfqpoint{3.634360in}{3.179440in}}%
\pgfpathlineto{\pgfqpoint{3.627035in}{3.167415in}}%
\pgfpathlineto{\pgfqpoint{3.619705in}{3.155569in}}%
\pgfpathlineto{\pgfqpoint{3.607277in}{3.162160in}}%
\pgfpathlineto{\pgfqpoint{3.594853in}{3.168797in}}%
\pgfpathlineto{\pgfqpoint{3.582432in}{3.175481in}}%
\pgfpathlineto{\pgfqpoint{3.570015in}{3.182212in}}%
\pgfpathlineto{\pgfqpoint{3.577351in}{3.194209in}}%
\pgfpathlineto{\pgfqpoint{3.584682in}{3.206388in}}%
\pgfpathlineto{\pgfqpoint{3.592008in}{3.218752in}}%
\pgfpathlineto{\pgfqpoint{3.599329in}{3.231307in}}%
\pgfpathclose%
\pgfusepath{fill}%
\end{pgfscope}%
\begin{pgfscope}%
\pgfpathrectangle{\pgfqpoint{1.254980in}{0.150000in}}{\pgfqpoint{5.490039in}{5.490039in}}%
\pgfusepath{clip}%
\pgfsetbuttcap%
\pgfsetroundjoin%
\definecolor{currentfill}{rgb}{0.281412,0.155834,0.469201}%
\pgfsetfillcolor{currentfill}%
\pgfsetfillopacity{0.700000}%
\pgfsetlinewidth{0.000000pt}%
\definecolor{currentstroke}{rgb}{0.000000,0.000000,0.000000}%
\pgfsetstrokecolor{currentstroke}%
\pgfsetdash{}{0pt}%
\pgfpathmoveto{\pgfqpoint{3.520375in}{3.209619in}}%
\pgfpathlineto{\pgfqpoint{3.532780in}{3.202694in}}%
\pgfpathlineto{\pgfqpoint{3.545189in}{3.195818in}}%
\pgfpathlineto{\pgfqpoint{3.557600in}{3.188991in}}%
\pgfpathlineto{\pgfqpoint{3.570015in}{3.182212in}}%
\pgfpathlineto{\pgfqpoint{3.562672in}{3.170392in}}%
\pgfpathlineto{\pgfqpoint{3.555325in}{3.158742in}}%
\pgfpathlineto{\pgfqpoint{3.547971in}{3.147260in}}%
\pgfpathlineto{\pgfqpoint{3.540613in}{3.135940in}}%
\pgfpathlineto{\pgfqpoint{3.528191in}{3.142576in}}%
\pgfpathlineto{\pgfqpoint{3.515773in}{3.149260in}}%
\pgfpathlineto{\pgfqpoint{3.503358in}{3.155993in}}%
\pgfpathlineto{\pgfqpoint{3.490946in}{3.162775in}}%
\pgfpathlineto{\pgfqpoint{3.498312in}{3.174233in}}%
\pgfpathlineto{\pgfqpoint{3.505672in}{3.185857in}}%
\pgfpathlineto{\pgfqpoint{3.513027in}{3.197651in}}%
\pgfpathlineto{\pgfqpoint{3.520375in}{3.209619in}}%
\pgfpathclose%
\pgfusepath{fill}%
\end{pgfscope}%
\begin{pgfscope}%
\pgfpathrectangle{\pgfqpoint{1.254980in}{0.150000in}}{\pgfqpoint{5.490039in}{5.490039in}}%
\pgfusepath{clip}%
\pgfsetbuttcap%
\pgfsetroundjoin%
\definecolor{currentfill}{rgb}{0.281412,0.155834,0.469201}%
\pgfsetfillcolor{currentfill}%
\pgfsetfillopacity{0.700000}%
\pgfsetlinewidth{0.000000pt}%
\definecolor{currentstroke}{rgb}{0.000000,0.000000,0.000000}%
\pgfsetstrokecolor{currentstroke}%
\pgfsetdash{}{0pt}%
\pgfpathmoveto{\pgfqpoint{3.648993in}{3.204043in}}%
\pgfpathlineto{\pgfqpoint{3.661418in}{3.197342in}}%
\pgfpathlineto{\pgfqpoint{3.673845in}{3.190687in}}%
\pgfpathlineto{\pgfqpoint{3.686277in}{3.184077in}}%
\pgfpathlineto{\pgfqpoint{3.698711in}{3.177511in}}%
\pgfpathlineto{\pgfqpoint{3.691403in}{3.165276in}}%
\pgfpathlineto{\pgfqpoint{3.684090in}{3.153226in}}%
\pgfpathlineto{\pgfqpoint{3.676772in}{3.141355in}}%
\pgfpathlineto{\pgfqpoint{3.669449in}{3.129660in}}%
\pgfpathlineto{\pgfqpoint{3.657008in}{3.136070in}}%
\pgfpathlineto{\pgfqpoint{3.644570in}{3.142524in}}%
\pgfpathlineto{\pgfqpoint{3.632136in}{3.149024in}}%
\pgfpathlineto{\pgfqpoint{3.619705in}{3.155569in}}%
\pgfpathlineto{\pgfqpoint{3.627035in}{3.167415in}}%
\pgfpathlineto{\pgfqpoint{3.634360in}{3.179440in}}%
\pgfpathlineto{\pgfqpoint{3.641679in}{3.191647in}}%
\pgfpathlineto{\pgfqpoint{3.648993in}{3.204043in}}%
\pgfpathclose%
\pgfusepath{fill}%
\end{pgfscope}%
\begin{pgfscope}%
\pgfpathrectangle{\pgfqpoint{1.254980in}{0.150000in}}{\pgfqpoint{5.490039in}{5.490039in}}%
\pgfusepath{clip}%
\pgfsetbuttcap%
\pgfsetroundjoin%
\definecolor{currentfill}{rgb}{0.282290,0.145912,0.461510}%
\pgfsetfillcolor{currentfill}%
\pgfsetfillopacity{0.700000}%
\pgfsetlinewidth{0.000000pt}%
\definecolor{currentstroke}{rgb}{0.000000,0.000000,0.000000}%
\pgfsetstrokecolor{currentstroke}%
\pgfsetdash{}{0pt}%
\pgfpathmoveto{\pgfqpoint{3.570015in}{3.182212in}}%
\pgfpathlineto{\pgfqpoint{3.582432in}{3.175481in}}%
\pgfpathlineto{\pgfqpoint{3.594853in}{3.168797in}}%
\pgfpathlineto{\pgfqpoint{3.607277in}{3.162160in}}%
\pgfpathlineto{\pgfqpoint{3.619705in}{3.155569in}}%
\pgfpathlineto{\pgfqpoint{3.612370in}{3.143896in}}%
\pgfpathlineto{\pgfqpoint{3.605029in}{3.132391in}}%
\pgfpathlineto{\pgfqpoint{3.597683in}{3.121051in}}%
\pgfpathlineto{\pgfqpoint{3.590331in}{3.109869in}}%
\pgfpathlineto{\pgfqpoint{3.577897in}{3.116317in}}%
\pgfpathlineto{\pgfqpoint{3.565465in}{3.122811in}}%
\pgfpathlineto{\pgfqpoint{3.553037in}{3.129352in}}%
\pgfpathlineto{\pgfqpoint{3.540613in}{3.135940in}}%
\pgfpathlineto{\pgfqpoint{3.547971in}{3.147260in}}%
\pgfpathlineto{\pgfqpoint{3.555325in}{3.158742in}}%
\pgfpathlineto{\pgfqpoint{3.562672in}{3.170392in}}%
\pgfpathlineto{\pgfqpoint{3.570015in}{3.182212in}}%
\pgfpathclose%
\pgfusepath{fill}%
\end{pgfscope}%
\begin{pgfscope}%
\pgfpathrectangle{\pgfqpoint{1.254980in}{0.150000in}}{\pgfqpoint{5.490039in}{5.490039in}}%
\pgfusepath{clip}%
\pgfsetbuttcap%
\pgfsetroundjoin%
\definecolor{currentfill}{rgb}{0.282290,0.145912,0.461510}%
\pgfsetfillcolor{currentfill}%
\pgfsetfillopacity{0.700000}%
\pgfsetlinewidth{0.000000pt}%
\definecolor{currentstroke}{rgb}{0.000000,0.000000,0.000000}%
\pgfsetstrokecolor{currentstroke}%
\pgfsetdash{}{0pt}%
\pgfpathmoveto{\pgfqpoint{3.698711in}{3.177511in}}%
\pgfpathlineto{\pgfqpoint{3.711149in}{3.170990in}}%
\pgfpathlineto{\pgfqpoint{3.723591in}{3.164512in}}%
\pgfpathlineto{\pgfqpoint{3.736036in}{3.158077in}}%
\pgfpathlineto{\pgfqpoint{3.748484in}{3.151686in}}%
\pgfpathlineto{\pgfqpoint{3.741183in}{3.139610in}}%
\pgfpathlineto{\pgfqpoint{3.733877in}{3.127717in}}%
\pgfpathlineto{\pgfqpoint{3.726566in}{3.116000in}}%
\pgfpathlineto{\pgfqpoint{3.719250in}{3.104456in}}%
\pgfpathlineto{\pgfqpoint{3.706794in}{3.110692in}}%
\pgfpathlineto{\pgfqpoint{3.694343in}{3.116971in}}%
\pgfpathlineto{\pgfqpoint{3.681894in}{3.123293in}}%
\pgfpathlineto{\pgfqpoint{3.669449in}{3.129660in}}%
\pgfpathlineto{\pgfqpoint{3.676772in}{3.141355in}}%
\pgfpathlineto{\pgfqpoint{3.684090in}{3.153226in}}%
\pgfpathlineto{\pgfqpoint{3.691403in}{3.165276in}}%
\pgfpathlineto{\pgfqpoint{3.698711in}{3.177511in}}%
\pgfpathclose%
\pgfusepath{fill}%
\end{pgfscope}%
\begin{pgfscope}%
\pgfpathrectangle{\pgfqpoint{1.254980in}{0.150000in}}{\pgfqpoint{5.490039in}{5.490039in}}%
\pgfusepath{clip}%
\pgfsetbuttcap%
\pgfsetroundjoin%
\definecolor{currentfill}{rgb}{0.282623,0.140926,0.457517}%
\pgfsetfillcolor{currentfill}%
\pgfsetfillopacity{0.700000}%
\pgfsetlinewidth{0.000000pt}%
\definecolor{currentstroke}{rgb}{0.000000,0.000000,0.000000}%
\pgfsetstrokecolor{currentstroke}%
\pgfsetdash{}{0pt}%
\pgfpathmoveto{\pgfqpoint{3.490946in}{3.162775in}}%
\pgfpathlineto{\pgfqpoint{3.503358in}{3.155993in}}%
\pgfpathlineto{\pgfqpoint{3.515773in}{3.149260in}}%
\pgfpathlineto{\pgfqpoint{3.528191in}{3.142576in}}%
\pgfpathlineto{\pgfqpoint{3.540613in}{3.135940in}}%
\pgfpathlineto{\pgfqpoint{3.533248in}{3.124778in}}%
\pgfpathlineto{\pgfqpoint{3.525878in}{3.113770in}}%
\pgfpathlineto{\pgfqpoint{3.518502in}{3.102910in}}%
\pgfpathlineto{\pgfqpoint{3.511121in}{3.092196in}}%
\pgfpathlineto{\pgfqpoint{3.498692in}{3.098701in}}%
\pgfpathlineto{\pgfqpoint{3.486266in}{3.105255in}}%
\pgfpathlineto{\pgfqpoint{3.473844in}{3.111857in}}%
\pgfpathlineto{\pgfqpoint{3.461424in}{3.118508in}}%
\pgfpathlineto{\pgfqpoint{3.468814in}{3.129348in}}%
\pgfpathlineto{\pgfqpoint{3.476197in}{3.140337in}}%
\pgfpathlineto{\pgfqpoint{3.483574in}{3.151477in}}%
\pgfpathlineto{\pgfqpoint{3.490946in}{3.162775in}}%
\pgfpathclose%
\pgfusepath{fill}%
\end{pgfscope}%
\begin{pgfscope}%
\pgfpathrectangle{\pgfqpoint{1.254980in}{0.150000in}}{\pgfqpoint{5.490039in}{5.490039in}}%
\pgfusepath{clip}%
\pgfsetbuttcap%
\pgfsetroundjoin%
\definecolor{currentfill}{rgb}{0.282623,0.140926,0.457517}%
\pgfsetfillcolor{currentfill}%
\pgfsetfillopacity{0.700000}%
\pgfsetlinewidth{0.000000pt}%
\definecolor{currentstroke}{rgb}{0.000000,0.000000,0.000000}%
\pgfsetstrokecolor{currentstroke}%
\pgfsetdash{}{0pt}%
\pgfpathmoveto{\pgfqpoint{3.619705in}{3.155569in}}%
\pgfpathlineto{\pgfqpoint{3.632136in}{3.149024in}}%
\pgfpathlineto{\pgfqpoint{3.644570in}{3.142524in}}%
\pgfpathlineto{\pgfqpoint{3.657008in}{3.136070in}}%
\pgfpathlineto{\pgfqpoint{3.669449in}{3.129660in}}%
\pgfpathlineto{\pgfqpoint{3.662121in}{3.118134in}}%
\pgfpathlineto{\pgfqpoint{3.654788in}{3.106774in}}%
\pgfpathlineto{\pgfqpoint{3.647449in}{3.095575in}}%
\pgfpathlineto{\pgfqpoint{3.640105in}{3.084532in}}%
\pgfpathlineto{\pgfqpoint{3.627656in}{3.090799in}}%
\pgfpathlineto{\pgfqpoint{3.615211in}{3.097111in}}%
\pgfpathlineto{\pgfqpoint{3.602769in}{3.103467in}}%
\pgfpathlineto{\pgfqpoint{3.590331in}{3.109869in}}%
\pgfpathlineto{\pgfqpoint{3.597683in}{3.121051in}}%
\pgfpathlineto{\pgfqpoint{3.605029in}{3.132391in}}%
\pgfpathlineto{\pgfqpoint{3.612370in}{3.143896in}}%
\pgfpathlineto{\pgfqpoint{3.619705in}{3.155569in}}%
\pgfpathclose%
\pgfusepath{fill}%
\end{pgfscope}%
\begin{pgfscope}%
\pgfpathrectangle{\pgfqpoint{1.254980in}{0.150000in}}{\pgfqpoint{5.490039in}{5.490039in}}%
\pgfusepath{clip}%
\pgfsetbuttcap%
\pgfsetroundjoin%
\definecolor{currentfill}{rgb}{0.282623,0.140926,0.457517}%
\pgfsetfillcolor{currentfill}%
\pgfsetfillopacity{0.700000}%
\pgfsetlinewidth{0.000000pt}%
\definecolor{currentstroke}{rgb}{0.000000,0.000000,0.000000}%
\pgfsetstrokecolor{currentstroke}%
\pgfsetdash{}{0pt}%
\pgfpathmoveto{\pgfqpoint{3.748484in}{3.151686in}}%
\pgfpathlineto{\pgfqpoint{3.760936in}{3.145336in}}%
\pgfpathlineto{\pgfqpoint{3.773392in}{3.139029in}}%
\pgfpathlineto{\pgfqpoint{3.785852in}{3.132764in}}%
\pgfpathlineto{\pgfqpoint{3.798315in}{3.126540in}}%
\pgfpathlineto{\pgfqpoint{3.791021in}{3.114624in}}%
\pgfpathlineto{\pgfqpoint{3.783722in}{3.102888in}}%
\pgfpathlineto{\pgfqpoint{3.776418in}{3.091325in}}%
\pgfpathlineto{\pgfqpoint{3.769110in}{3.079931in}}%
\pgfpathlineto{\pgfqpoint{3.756639in}{3.086000in}}%
\pgfpathlineto{\pgfqpoint{3.744172in}{3.092110in}}%
\pgfpathlineto{\pgfqpoint{3.731709in}{3.098262in}}%
\pgfpathlineto{\pgfqpoint{3.719250in}{3.104456in}}%
\pgfpathlineto{\pgfqpoint{3.726566in}{3.116000in}}%
\pgfpathlineto{\pgfqpoint{3.733877in}{3.127717in}}%
\pgfpathlineto{\pgfqpoint{3.741183in}{3.139610in}}%
\pgfpathlineto{\pgfqpoint{3.748484in}{3.151686in}}%
\pgfpathclose%
\pgfusepath{fill}%
\end{pgfscope}%
\begin{pgfscope}%
\pgfpathrectangle{\pgfqpoint{1.254980in}{0.150000in}}{\pgfqpoint{5.490039in}{5.490039in}}%
\pgfusepath{clip}%
\pgfsetbuttcap%
\pgfsetroundjoin%
\definecolor{currentfill}{rgb}{0.282884,0.135920,0.453427}%
\pgfsetfillcolor{currentfill}%
\pgfsetfillopacity{0.700000}%
\pgfsetlinewidth{0.000000pt}%
\definecolor{currentstroke}{rgb}{0.000000,0.000000,0.000000}%
\pgfsetstrokecolor{currentstroke}%
\pgfsetdash{}{0pt}%
\pgfpathmoveto{\pgfqpoint{3.540613in}{3.135940in}}%
\pgfpathlineto{\pgfqpoint{3.553037in}{3.129352in}}%
\pgfpathlineto{\pgfqpoint{3.565465in}{3.122811in}}%
\pgfpathlineto{\pgfqpoint{3.577897in}{3.116317in}}%
\pgfpathlineto{\pgfqpoint{3.590331in}{3.109869in}}%
\pgfpathlineto{\pgfqpoint{3.582974in}{3.098843in}}%
\pgfpathlineto{\pgfqpoint{3.575612in}{3.087967in}}%
\pgfpathlineto{\pgfqpoint{3.568243in}{3.077237in}}%
\pgfpathlineto{\pgfqpoint{3.560870in}{3.066649in}}%
\pgfpathlineto{\pgfqpoint{3.548427in}{3.072966in}}%
\pgfpathlineto{\pgfqpoint{3.535988in}{3.079329in}}%
\pgfpathlineto{\pgfqpoint{3.523553in}{3.085739in}}%
\pgfpathlineto{\pgfqpoint{3.511121in}{3.092196in}}%
\pgfpathlineto{\pgfqpoint{3.518502in}{3.102910in}}%
\pgfpathlineto{\pgfqpoint{3.525878in}{3.113770in}}%
\pgfpathlineto{\pgfqpoint{3.533248in}{3.124778in}}%
\pgfpathlineto{\pgfqpoint{3.540613in}{3.135940in}}%
\pgfpathclose%
\pgfusepath{fill}%
\end{pgfscope}%
\begin{pgfscope}%
\pgfpathrectangle{\pgfqpoint{1.254980in}{0.150000in}}{\pgfqpoint{5.490039in}{5.490039in}}%
\pgfusepath{clip}%
\pgfsetbuttcap%
\pgfsetroundjoin%
\definecolor{currentfill}{rgb}{0.283072,0.130895,0.449241}%
\pgfsetfillcolor{currentfill}%
\pgfsetfillopacity{0.700000}%
\pgfsetlinewidth{0.000000pt}%
\definecolor{currentstroke}{rgb}{0.000000,0.000000,0.000000}%
\pgfsetstrokecolor{currentstroke}%
\pgfsetdash{}{0pt}%
\pgfpathmoveto{\pgfqpoint{3.669449in}{3.129660in}}%
\pgfpathlineto{\pgfqpoint{3.681894in}{3.123293in}}%
\pgfpathlineto{\pgfqpoint{3.694343in}{3.116971in}}%
\pgfpathlineto{\pgfqpoint{3.706794in}{3.110692in}}%
\pgfpathlineto{\pgfqpoint{3.719250in}{3.104456in}}%
\pgfpathlineto{\pgfqpoint{3.711929in}{3.093078in}}%
\pgfpathlineto{\pgfqpoint{3.704603in}{3.081863in}}%
\pgfpathlineto{\pgfqpoint{3.697272in}{3.070805in}}%
\pgfpathlineto{\pgfqpoint{3.689935in}{3.059901in}}%
\pgfpathlineto{\pgfqpoint{3.677472in}{3.065994in}}%
\pgfpathlineto{\pgfqpoint{3.665013in}{3.072130in}}%
\pgfpathlineto{\pgfqpoint{3.652557in}{3.078309in}}%
\pgfpathlineto{\pgfqpoint{3.640105in}{3.084532in}}%
\pgfpathlineto{\pgfqpoint{3.647449in}{3.095575in}}%
\pgfpathlineto{\pgfqpoint{3.654788in}{3.106774in}}%
\pgfpathlineto{\pgfqpoint{3.662121in}{3.118134in}}%
\pgfpathlineto{\pgfqpoint{3.669449in}{3.129660in}}%
\pgfpathclose%
\pgfusepath{fill}%
\end{pgfscope}%
\begin{pgfscope}%
\pgfpathrectangle{\pgfqpoint{1.254980in}{0.150000in}}{\pgfqpoint{5.490039in}{5.490039in}}%
\pgfusepath{clip}%
\pgfsetbuttcap%
\pgfsetroundjoin%
\definecolor{currentfill}{rgb}{0.283072,0.130895,0.449241}%
\pgfsetfillcolor{currentfill}%
\pgfsetfillopacity{0.700000}%
\pgfsetlinewidth{0.000000pt}%
\definecolor{currentstroke}{rgb}{0.000000,0.000000,0.000000}%
\pgfsetstrokecolor{currentstroke}%
\pgfsetdash{}{0pt}%
\pgfpathmoveto{\pgfqpoint{3.798315in}{3.126540in}}%
\pgfpathlineto{\pgfqpoint{3.810782in}{3.120356in}}%
\pgfpathlineto{\pgfqpoint{3.823253in}{3.114214in}}%
\pgfpathlineto{\pgfqpoint{3.835728in}{3.108111in}}%
\pgfpathlineto{\pgfqpoint{3.848206in}{3.102049in}}%
\pgfpathlineto{\pgfqpoint{3.840919in}{3.090293in}}%
\pgfpathlineto{\pgfqpoint{3.833628in}{3.078714in}}%
\pgfpathlineto{\pgfqpoint{3.826331in}{3.067305in}}%
\pgfpathlineto{\pgfqpoint{3.819030in}{3.056063in}}%
\pgfpathlineto{\pgfqpoint{3.806544in}{3.061970in}}%
\pgfpathlineto{\pgfqpoint{3.794062in}{3.067917in}}%
\pgfpathlineto{\pgfqpoint{3.781584in}{3.073904in}}%
\pgfpathlineto{\pgfqpoint{3.769110in}{3.079931in}}%
\pgfpathlineto{\pgfqpoint{3.776418in}{3.091325in}}%
\pgfpathlineto{\pgfqpoint{3.783722in}{3.102888in}}%
\pgfpathlineto{\pgfqpoint{3.791021in}{3.114624in}}%
\pgfpathlineto{\pgfqpoint{3.798315in}{3.126540in}}%
\pgfpathclose%
\pgfusepath{fill}%
\end{pgfscope}%
\begin{pgfscope}%
\pgfpathrectangle{\pgfqpoint{1.254980in}{0.150000in}}{\pgfqpoint{5.490039in}{5.490039in}}%
\pgfusepath{clip}%
\pgfsetbuttcap%
\pgfsetroundjoin%
\definecolor{currentfill}{rgb}{0.283072,0.130895,0.449241}%
\pgfsetfillcolor{currentfill}%
\pgfsetfillopacity{0.700000}%
\pgfsetlinewidth{0.000000pt}%
\definecolor{currentstroke}{rgb}{0.000000,0.000000,0.000000}%
\pgfsetstrokecolor{currentstroke}%
\pgfsetdash{}{0pt}%
\pgfpathmoveto{\pgfqpoint{3.461424in}{3.118508in}}%
\pgfpathlineto{\pgfqpoint{3.473844in}{3.111857in}}%
\pgfpathlineto{\pgfqpoint{3.486266in}{3.105255in}}%
\pgfpathlineto{\pgfqpoint{3.498692in}{3.098701in}}%
\pgfpathlineto{\pgfqpoint{3.511121in}{3.092196in}}%
\pgfpathlineto{\pgfqpoint{3.503733in}{3.081623in}}%
\pgfpathlineto{\pgfqpoint{3.496340in}{3.071187in}}%
\pgfpathlineto{\pgfqpoint{3.488942in}{3.060884in}}%
\pgfpathlineto{\pgfqpoint{3.481537in}{3.050710in}}%
\pgfpathlineto{\pgfqpoint{3.469100in}{3.057096in}}%
\pgfpathlineto{\pgfqpoint{3.456666in}{3.063531in}}%
\pgfpathlineto{\pgfqpoint{3.444236in}{3.070015in}}%
\pgfpathlineto{\pgfqpoint{3.431809in}{3.076547in}}%
\pgfpathlineto{\pgfqpoint{3.439222in}{3.086835in}}%
\pgfpathlineto{\pgfqpoint{3.446628in}{3.097255in}}%
\pgfpathlineto{\pgfqpoint{3.454029in}{3.107811in}}%
\pgfpathlineto{\pgfqpoint{3.461424in}{3.118508in}}%
\pgfpathclose%
\pgfusepath{fill}%
\end{pgfscope}%
\begin{pgfscope}%
\pgfpathrectangle{\pgfqpoint{1.254980in}{0.150000in}}{\pgfqpoint{5.490039in}{5.490039in}}%
\pgfusepath{clip}%
\pgfsetbuttcap%
\pgfsetroundjoin%
\definecolor{currentfill}{rgb}{0.283187,0.125848,0.444960}%
\pgfsetfillcolor{currentfill}%
\pgfsetfillopacity{0.700000}%
\pgfsetlinewidth{0.000000pt}%
\definecolor{currentstroke}{rgb}{0.000000,0.000000,0.000000}%
\pgfsetstrokecolor{currentstroke}%
\pgfsetdash{}{0pt}%
\pgfpathmoveto{\pgfqpoint{3.590331in}{3.109869in}}%
\pgfpathlineto{\pgfqpoint{3.602769in}{3.103467in}}%
\pgfpathlineto{\pgfqpoint{3.615211in}{3.097111in}}%
\pgfpathlineto{\pgfqpoint{3.627656in}{3.090799in}}%
\pgfpathlineto{\pgfqpoint{3.640105in}{3.084532in}}%
\pgfpathlineto{\pgfqpoint{3.632755in}{3.073641in}}%
\pgfpathlineto{\pgfqpoint{3.625400in}{3.062898in}}%
\pgfpathlineto{\pgfqpoint{3.618040in}{3.052297in}}%
\pgfpathlineto{\pgfqpoint{3.610674in}{3.041835in}}%
\pgfpathlineto{\pgfqpoint{3.598218in}{3.047972in}}%
\pgfpathlineto{\pgfqpoint{3.585765in}{3.054152in}}%
\pgfpathlineto{\pgfqpoint{3.573316in}{3.060378in}}%
\pgfpathlineto{\pgfqpoint{3.560870in}{3.066649in}}%
\pgfpathlineto{\pgfqpoint{3.568243in}{3.077237in}}%
\pgfpathlineto{\pgfqpoint{3.575612in}{3.087967in}}%
\pgfpathlineto{\pgfqpoint{3.582974in}{3.098843in}}%
\pgfpathlineto{\pgfqpoint{3.590331in}{3.109869in}}%
\pgfpathclose%
\pgfusepath{fill}%
\end{pgfscope}%
\begin{pgfscope}%
\pgfpathrectangle{\pgfqpoint{1.254980in}{0.150000in}}{\pgfqpoint{5.490039in}{5.490039in}}%
\pgfusepath{clip}%
\pgfsetbuttcap%
\pgfsetroundjoin%
\definecolor{currentfill}{rgb}{0.283187,0.125848,0.444960}%
\pgfsetfillcolor{currentfill}%
\pgfsetfillopacity{0.700000}%
\pgfsetlinewidth{0.000000pt}%
\definecolor{currentstroke}{rgb}{0.000000,0.000000,0.000000}%
\pgfsetstrokecolor{currentstroke}%
\pgfsetdash{}{0pt}%
\pgfpathmoveto{\pgfqpoint{3.719250in}{3.104456in}}%
\pgfpathlineto{\pgfqpoint{3.731709in}{3.098262in}}%
\pgfpathlineto{\pgfqpoint{3.744172in}{3.092110in}}%
\pgfpathlineto{\pgfqpoint{3.756639in}{3.086000in}}%
\pgfpathlineto{\pgfqpoint{3.769110in}{3.079931in}}%
\pgfpathlineto{\pgfqpoint{3.761796in}{3.068702in}}%
\pgfpathlineto{\pgfqpoint{3.754478in}{3.057631in}}%
\pgfpathlineto{\pgfqpoint{3.747154in}{3.046716in}}%
\pgfpathlineto{\pgfqpoint{3.739826in}{3.035950in}}%
\pgfpathlineto{\pgfqpoint{3.727347in}{3.041875in}}%
\pgfpathlineto{\pgfqpoint{3.714873in}{3.047842in}}%
\pgfpathlineto{\pgfqpoint{3.702402in}{3.053851in}}%
\pgfpathlineto{\pgfqpoint{3.689935in}{3.059901in}}%
\pgfpathlineto{\pgfqpoint{3.697272in}{3.070805in}}%
\pgfpathlineto{\pgfqpoint{3.704603in}{3.081863in}}%
\pgfpathlineto{\pgfqpoint{3.711929in}{3.093078in}}%
\pgfpathlineto{\pgfqpoint{3.719250in}{3.104456in}}%
\pgfpathclose%
\pgfusepath{fill}%
\end{pgfscope}%
\begin{pgfscope}%
\pgfpathrectangle{\pgfqpoint{1.254980in}{0.150000in}}{\pgfqpoint{5.490039in}{5.490039in}}%
\pgfusepath{clip}%
\pgfsetbuttcap%
\pgfsetroundjoin%
\definecolor{currentfill}{rgb}{0.283187,0.125848,0.444960}%
\pgfsetfillcolor{currentfill}%
\pgfsetfillopacity{0.700000}%
\pgfsetlinewidth{0.000000pt}%
\definecolor{currentstroke}{rgb}{0.000000,0.000000,0.000000}%
\pgfsetstrokecolor{currentstroke}%
\pgfsetdash{}{0pt}%
\pgfpathmoveto{\pgfqpoint{3.848206in}{3.102049in}}%
\pgfpathlineto{\pgfqpoint{3.860689in}{3.096026in}}%
\pgfpathlineto{\pgfqpoint{3.873175in}{3.090042in}}%
\pgfpathlineto{\pgfqpoint{3.885665in}{3.084097in}}%
\pgfpathlineto{\pgfqpoint{3.898159in}{3.078190in}}%
\pgfpathlineto{\pgfqpoint{3.890880in}{3.066595in}}%
\pgfpathlineto{\pgfqpoint{3.883596in}{3.055173in}}%
\pgfpathlineto{\pgfqpoint{3.876307in}{3.043919in}}%
\pgfpathlineto{\pgfqpoint{3.869014in}{3.032827in}}%
\pgfpathlineto{\pgfqpoint{3.856512in}{3.038578in}}%
\pgfpathlineto{\pgfqpoint{3.844014in}{3.044367in}}%
\pgfpathlineto{\pgfqpoint{3.831520in}{3.050195in}}%
\pgfpathlineto{\pgfqpoint{3.819030in}{3.056063in}}%
\pgfpathlineto{\pgfqpoint{3.826331in}{3.067305in}}%
\pgfpathlineto{\pgfqpoint{3.833628in}{3.078714in}}%
\pgfpathlineto{\pgfqpoint{3.840919in}{3.090293in}}%
\pgfpathlineto{\pgfqpoint{3.848206in}{3.102049in}}%
\pgfpathclose%
\pgfusepath{fill}%
\end{pgfscope}%
\begin{pgfscope}%
\pgfpathrectangle{\pgfqpoint{1.254980in}{0.150000in}}{\pgfqpoint{5.490039in}{5.490039in}}%
\pgfusepath{clip}%
\pgfsetbuttcap%
\pgfsetroundjoin%
\definecolor{currentfill}{rgb}{0.283229,0.120777,0.440584}%
\pgfsetfillcolor{currentfill}%
\pgfsetfillopacity{0.700000}%
\pgfsetlinewidth{0.000000pt}%
\definecolor{currentstroke}{rgb}{0.000000,0.000000,0.000000}%
\pgfsetstrokecolor{currentstroke}%
\pgfsetdash{}{0pt}%
\pgfpathmoveto{\pgfqpoint{3.511121in}{3.092196in}}%
\pgfpathlineto{\pgfqpoint{3.523553in}{3.085739in}}%
\pgfpathlineto{\pgfqpoint{3.535988in}{3.079329in}}%
\pgfpathlineto{\pgfqpoint{3.548427in}{3.072966in}}%
\pgfpathlineto{\pgfqpoint{3.560870in}{3.066649in}}%
\pgfpathlineto{\pgfqpoint{3.553490in}{3.056199in}}%
\pgfpathlineto{\pgfqpoint{3.546106in}{3.045883in}}%
\pgfpathlineto{\pgfqpoint{3.538715in}{3.035697in}}%
\pgfpathlineto{\pgfqpoint{3.531319in}{3.025637in}}%
\pgfpathlineto{\pgfqpoint{3.518868in}{3.031835in}}%
\pgfpathlineto{\pgfqpoint{3.506421in}{3.038080in}}%
\pgfpathlineto{\pgfqpoint{3.493977in}{3.044371in}}%
\pgfpathlineto{\pgfqpoint{3.481537in}{3.050710in}}%
\pgfpathlineto{\pgfqpoint{3.488942in}{3.060884in}}%
\pgfpathlineto{\pgfqpoint{3.496340in}{3.071187in}}%
\pgfpathlineto{\pgfqpoint{3.503733in}{3.081623in}}%
\pgfpathlineto{\pgfqpoint{3.511121in}{3.092196in}}%
\pgfpathclose%
\pgfusepath{fill}%
\end{pgfscope}%
\begin{pgfscope}%
\pgfpathrectangle{\pgfqpoint{1.254980in}{0.150000in}}{\pgfqpoint{5.490039in}{5.490039in}}%
\pgfusepath{clip}%
\pgfsetbuttcap%
\pgfsetroundjoin%
\definecolor{currentfill}{rgb}{0.283229,0.120777,0.440584}%
\pgfsetfillcolor{currentfill}%
\pgfsetfillopacity{0.700000}%
\pgfsetlinewidth{0.000000pt}%
\definecolor{currentstroke}{rgb}{0.000000,0.000000,0.000000}%
\pgfsetstrokecolor{currentstroke}%
\pgfsetdash{}{0pt}%
\pgfpathmoveto{\pgfqpoint{3.640105in}{3.084532in}}%
\pgfpathlineto{\pgfqpoint{3.652557in}{3.078309in}}%
\pgfpathlineto{\pgfqpoint{3.665013in}{3.072130in}}%
\pgfpathlineto{\pgfqpoint{3.677472in}{3.065994in}}%
\pgfpathlineto{\pgfqpoint{3.689935in}{3.059901in}}%
\pgfpathlineto{\pgfqpoint{3.682594in}{3.049146in}}%
\pgfpathlineto{\pgfqpoint{3.675247in}{3.038534in}}%
\pgfpathlineto{\pgfqpoint{3.667895in}{3.028063in}}%
\pgfpathlineto{\pgfqpoint{3.660537in}{3.017728in}}%
\pgfpathlineto{\pgfqpoint{3.648066in}{3.023690in}}%
\pgfpathlineto{\pgfqpoint{3.635598in}{3.029695in}}%
\pgfpathlineto{\pgfqpoint{3.623135in}{3.035743in}}%
\pgfpathlineto{\pgfqpoint{3.610674in}{3.041835in}}%
\pgfpathlineto{\pgfqpoint{3.618040in}{3.052297in}}%
\pgfpathlineto{\pgfqpoint{3.625400in}{3.062898in}}%
\pgfpathlineto{\pgfqpoint{3.632755in}{3.073641in}}%
\pgfpathlineto{\pgfqpoint{3.640105in}{3.084532in}}%
\pgfpathclose%
\pgfusepath{fill}%
\end{pgfscope}%
\begin{pgfscope}%
\pgfpathrectangle{\pgfqpoint{1.254980in}{0.150000in}}{\pgfqpoint{5.490039in}{5.490039in}}%
\pgfusepath{clip}%
\pgfsetbuttcap%
\pgfsetroundjoin%
\definecolor{currentfill}{rgb}{0.283197,0.115680,0.436115}%
\pgfsetfillcolor{currentfill}%
\pgfsetfillopacity{0.700000}%
\pgfsetlinewidth{0.000000pt}%
\definecolor{currentstroke}{rgb}{0.000000,0.000000,0.000000}%
\pgfsetstrokecolor{currentstroke}%
\pgfsetdash{}{0pt}%
\pgfpathmoveto{\pgfqpoint{3.769110in}{3.079931in}}%
\pgfpathlineto{\pgfqpoint{3.781584in}{3.073904in}}%
\pgfpathlineto{\pgfqpoint{3.794062in}{3.067917in}}%
\pgfpathlineto{\pgfqpoint{3.806544in}{3.061970in}}%
\pgfpathlineto{\pgfqpoint{3.819030in}{3.056063in}}%
\pgfpathlineto{\pgfqpoint{3.811725in}{3.044981in}}%
\pgfpathlineto{\pgfqpoint{3.804414in}{3.034055in}}%
\pgfpathlineto{\pgfqpoint{3.797098in}{3.023281in}}%
\pgfpathlineto{\pgfqpoint{3.789778in}{3.012655in}}%
\pgfpathlineto{\pgfqpoint{3.777284in}{3.018418in}}%
\pgfpathlineto{\pgfqpoint{3.764794in}{3.024222in}}%
\pgfpathlineto{\pgfqpoint{3.752308in}{3.030066in}}%
\pgfpathlineto{\pgfqpoint{3.739826in}{3.035950in}}%
\pgfpathlineto{\pgfqpoint{3.747154in}{3.046716in}}%
\pgfpathlineto{\pgfqpoint{3.754478in}{3.057631in}}%
\pgfpathlineto{\pgfqpoint{3.761796in}{3.068702in}}%
\pgfpathlineto{\pgfqpoint{3.769110in}{3.079931in}}%
\pgfpathclose%
\pgfusepath{fill}%
\end{pgfscope}%
\begin{pgfscope}%
\pgfpathrectangle{\pgfqpoint{1.254980in}{0.150000in}}{\pgfqpoint{5.490039in}{5.490039in}}%
\pgfusepath{clip}%
\pgfsetbuttcap%
\pgfsetroundjoin%
\definecolor{currentfill}{rgb}{0.283197,0.115680,0.436115}%
\pgfsetfillcolor{currentfill}%
\pgfsetfillopacity{0.700000}%
\pgfsetlinewidth{0.000000pt}%
\definecolor{currentstroke}{rgb}{0.000000,0.000000,0.000000}%
\pgfsetstrokecolor{currentstroke}%
\pgfsetdash{}{0pt}%
\pgfpathmoveto{\pgfqpoint{3.431809in}{3.076547in}}%
\pgfpathlineto{\pgfqpoint{3.444236in}{3.070015in}}%
\pgfpathlineto{\pgfqpoint{3.456666in}{3.063531in}}%
\pgfpathlineto{\pgfqpoint{3.469100in}{3.057096in}}%
\pgfpathlineto{\pgfqpoint{3.481537in}{3.050710in}}%
\pgfpathlineto{\pgfqpoint{3.474127in}{3.040661in}}%
\pgfpathlineto{\pgfqpoint{3.466710in}{3.030735in}}%
\pgfpathlineto{\pgfqpoint{3.459288in}{3.020927in}}%
\pgfpathlineto{\pgfqpoint{3.451860in}{3.011233in}}%
\pgfpathlineto{\pgfqpoint{3.439415in}{3.017513in}}%
\pgfpathlineto{\pgfqpoint{3.426973in}{3.023842in}}%
\pgfpathlineto{\pgfqpoint{3.414534in}{3.030219in}}%
\pgfpathlineto{\pgfqpoint{3.402098in}{3.036645in}}%
\pgfpathlineto{\pgfqpoint{3.409535in}{3.046440in}}%
\pgfpathlineto{\pgfqpoint{3.416965in}{3.056354in}}%
\pgfpathlineto{\pgfqpoint{3.424390in}{3.066388in}}%
\pgfpathlineto{\pgfqpoint{3.431809in}{3.076547in}}%
\pgfpathclose%
\pgfusepath{fill}%
\end{pgfscope}%
\begin{pgfscope}%
\pgfpathrectangle{\pgfqpoint{1.254980in}{0.150000in}}{\pgfqpoint{5.490039in}{5.490039in}}%
\pgfusepath{clip}%
\pgfsetbuttcap%
\pgfsetroundjoin%
\definecolor{currentfill}{rgb}{0.283229,0.120777,0.440584}%
\pgfsetfillcolor{currentfill}%
\pgfsetfillopacity{0.700000}%
\pgfsetlinewidth{0.000000pt}%
\definecolor{currentstroke}{rgb}{0.000000,0.000000,0.000000}%
\pgfsetstrokecolor{currentstroke}%
\pgfsetdash{}{0pt}%
\pgfpathmoveto{\pgfqpoint{3.898159in}{3.078190in}}%
\pgfpathlineto{\pgfqpoint{3.910658in}{3.072322in}}%
\pgfpathlineto{\pgfqpoint{3.923160in}{3.066492in}}%
\pgfpathlineto{\pgfqpoint{3.935667in}{3.060699in}}%
\pgfpathlineto{\pgfqpoint{3.948177in}{3.054944in}}%
\pgfpathlineto{\pgfqpoint{3.940905in}{3.043509in}}%
\pgfpathlineto{\pgfqpoint{3.933629in}{3.032244in}}%
\pgfpathlineto{\pgfqpoint{3.926349in}{3.021144in}}%
\pgfpathlineto{\pgfqpoint{3.919064in}{3.010203in}}%
\pgfpathlineto{\pgfqpoint{3.906545in}{3.015803in}}%
\pgfpathlineto{\pgfqpoint{3.894031in}{3.021440in}}%
\pgfpathlineto{\pgfqpoint{3.881520in}{3.027114in}}%
\pgfpathlineto{\pgfqpoint{3.869014in}{3.032827in}}%
\pgfpathlineto{\pgfqpoint{3.876307in}{3.043919in}}%
\pgfpathlineto{\pgfqpoint{3.883596in}{3.055173in}}%
\pgfpathlineto{\pgfqpoint{3.890880in}{3.066595in}}%
\pgfpathlineto{\pgfqpoint{3.898159in}{3.078190in}}%
\pgfpathclose%
\pgfusepath{fill}%
\end{pgfscope}%
\begin{pgfscope}%
\pgfpathrectangle{\pgfqpoint{1.254980in}{0.150000in}}{\pgfqpoint{5.490039in}{5.490039in}}%
\pgfusepath{clip}%
\pgfsetbuttcap%
\pgfsetroundjoin%
\definecolor{currentfill}{rgb}{0.283197,0.115680,0.436115}%
\pgfsetfillcolor{currentfill}%
\pgfsetfillopacity{0.700000}%
\pgfsetlinewidth{0.000000pt}%
\definecolor{currentstroke}{rgb}{0.000000,0.000000,0.000000}%
\pgfsetstrokecolor{currentstroke}%
\pgfsetdash{}{0pt}%
\pgfpathmoveto{\pgfqpoint{3.560870in}{3.066649in}}%
\pgfpathlineto{\pgfqpoint{3.573316in}{3.060378in}}%
\pgfpathlineto{\pgfqpoint{3.585765in}{3.054152in}}%
\pgfpathlineto{\pgfqpoint{3.598218in}{3.047972in}}%
\pgfpathlineto{\pgfqpoint{3.610674in}{3.041835in}}%
\pgfpathlineto{\pgfqpoint{3.603303in}{3.031509in}}%
\pgfpathlineto{\pgfqpoint{3.595927in}{3.021313in}}%
\pgfpathlineto{\pgfqpoint{3.588544in}{3.011244in}}%
\pgfpathlineto{\pgfqpoint{3.581157in}{3.001298in}}%
\pgfpathlineto{\pgfqpoint{3.568692in}{3.007315in}}%
\pgfpathlineto{\pgfqpoint{3.556231in}{3.013377in}}%
\pgfpathlineto{\pgfqpoint{3.543773in}{3.019484in}}%
\pgfpathlineto{\pgfqpoint{3.531319in}{3.025637in}}%
\pgfpathlineto{\pgfqpoint{3.538715in}{3.035697in}}%
\pgfpathlineto{\pgfqpoint{3.546106in}{3.045883in}}%
\pgfpathlineto{\pgfqpoint{3.553490in}{3.056199in}}%
\pgfpathlineto{\pgfqpoint{3.560870in}{3.066649in}}%
\pgfpathclose%
\pgfusepath{fill}%
\end{pgfscope}%
\begin{pgfscope}%
\pgfpathrectangle{\pgfqpoint{1.254980in}{0.150000in}}{\pgfqpoint{5.490039in}{5.490039in}}%
\pgfusepath{clip}%
\pgfsetbuttcap%
\pgfsetroundjoin%
\definecolor{currentfill}{rgb}{0.283091,0.110553,0.431554}%
\pgfsetfillcolor{currentfill}%
\pgfsetfillopacity{0.700000}%
\pgfsetlinewidth{0.000000pt}%
\definecolor{currentstroke}{rgb}{0.000000,0.000000,0.000000}%
\pgfsetstrokecolor{currentstroke}%
\pgfsetdash{}{0pt}%
\pgfpathmoveto{\pgfqpoint{3.689935in}{3.059901in}}%
\pgfpathlineto{\pgfqpoint{3.702402in}{3.053851in}}%
\pgfpathlineto{\pgfqpoint{3.714873in}{3.047842in}}%
\pgfpathlineto{\pgfqpoint{3.727347in}{3.041875in}}%
\pgfpathlineto{\pgfqpoint{3.739826in}{3.035950in}}%
\pgfpathlineto{\pgfqpoint{3.732492in}{3.025330in}}%
\pgfpathlineto{\pgfqpoint{3.725153in}{3.014851in}}%
\pgfpathlineto{\pgfqpoint{3.717809in}{3.004510in}}%
\pgfpathlineto{\pgfqpoint{3.710460in}{2.994301in}}%
\pgfpathlineto{\pgfqpoint{3.697974in}{3.000096in}}%
\pgfpathlineto{\pgfqpoint{3.685491in}{3.005931in}}%
\pgfpathlineto{\pgfqpoint{3.673012in}{3.011809in}}%
\pgfpathlineto{\pgfqpoint{3.660537in}{3.017728in}}%
\pgfpathlineto{\pgfqpoint{3.667895in}{3.028063in}}%
\pgfpathlineto{\pgfqpoint{3.675247in}{3.038534in}}%
\pgfpathlineto{\pgfqpoint{3.682594in}{3.049146in}}%
\pgfpathlineto{\pgfqpoint{3.689935in}{3.059901in}}%
\pgfpathclose%
\pgfusepath{fill}%
\end{pgfscope}%
\begin{pgfscope}%
\pgfpathrectangle{\pgfqpoint{1.254980in}{0.150000in}}{\pgfqpoint{5.490039in}{5.490039in}}%
\pgfusepath{clip}%
\pgfsetbuttcap%
\pgfsetroundjoin%
\definecolor{currentfill}{rgb}{0.283091,0.110553,0.431554}%
\pgfsetfillcolor{currentfill}%
\pgfsetfillopacity{0.700000}%
\pgfsetlinewidth{0.000000pt}%
\definecolor{currentstroke}{rgb}{0.000000,0.000000,0.000000}%
\pgfsetstrokecolor{currentstroke}%
\pgfsetdash{}{0pt}%
\pgfpathmoveto{\pgfqpoint{3.819030in}{3.056063in}}%
\pgfpathlineto{\pgfqpoint{3.831520in}{3.050195in}}%
\pgfpathlineto{\pgfqpoint{3.844014in}{3.044367in}}%
\pgfpathlineto{\pgfqpoint{3.856512in}{3.038578in}}%
\pgfpathlineto{\pgfqpoint{3.869014in}{3.032827in}}%
\pgfpathlineto{\pgfqpoint{3.861716in}{3.021893in}}%
\pgfpathlineto{\pgfqpoint{3.854414in}{3.011112in}}%
\pgfpathlineto{\pgfqpoint{3.847107in}{3.000480in}}%
\pgfpathlineto{\pgfqpoint{3.839795in}{2.989992in}}%
\pgfpathlineto{\pgfqpoint{3.827284in}{2.995600in}}%
\pgfpathlineto{\pgfqpoint{3.814778in}{3.001246in}}%
\pgfpathlineto{\pgfqpoint{3.802276in}{3.006930in}}%
\pgfpathlineto{\pgfqpoint{3.789778in}{3.012655in}}%
\pgfpathlineto{\pgfqpoint{3.797098in}{3.023281in}}%
\pgfpathlineto{\pgfqpoint{3.804414in}{3.034055in}}%
\pgfpathlineto{\pgfqpoint{3.811725in}{3.044981in}}%
\pgfpathlineto{\pgfqpoint{3.819030in}{3.056063in}}%
\pgfpathclose%
\pgfusepath{fill}%
\end{pgfscope}%
\begin{pgfscope}%
\pgfpathrectangle{\pgfqpoint{1.254980in}{0.150000in}}{\pgfqpoint{5.490039in}{5.490039in}}%
\pgfusepath{clip}%
\pgfsetbuttcap%
\pgfsetroundjoin%
\definecolor{currentfill}{rgb}{0.283091,0.110553,0.431554}%
\pgfsetfillcolor{currentfill}%
\pgfsetfillopacity{0.700000}%
\pgfsetlinewidth{0.000000pt}%
\definecolor{currentstroke}{rgb}{0.000000,0.000000,0.000000}%
\pgfsetstrokecolor{currentstroke}%
\pgfsetdash{}{0pt}%
\pgfpathmoveto{\pgfqpoint{3.481537in}{3.050710in}}%
\pgfpathlineto{\pgfqpoint{3.493977in}{3.044371in}}%
\pgfpathlineto{\pgfqpoint{3.506421in}{3.038080in}}%
\pgfpathlineto{\pgfqpoint{3.518868in}{3.031835in}}%
\pgfpathlineto{\pgfqpoint{3.531319in}{3.025637in}}%
\pgfpathlineto{\pgfqpoint{3.523917in}{3.015699in}}%
\pgfpathlineto{\pgfqpoint{3.516509in}{3.005881in}}%
\pgfpathlineto{\pgfqpoint{3.509096in}{2.996177in}}%
\pgfpathlineto{\pgfqpoint{3.501677in}{2.986585in}}%
\pgfpathlineto{\pgfqpoint{3.489217in}{2.992677in}}%
\pgfpathlineto{\pgfqpoint{3.476762in}{2.998816in}}%
\pgfpathlineto{\pgfqpoint{3.464309in}{3.005001in}}%
\pgfpathlineto{\pgfqpoint{3.451860in}{3.011233in}}%
\pgfpathlineto{\pgfqpoint{3.459288in}{3.020927in}}%
\pgfpathlineto{\pgfqpoint{3.466710in}{3.030735in}}%
\pgfpathlineto{\pgfqpoint{3.474127in}{3.040661in}}%
\pgfpathlineto{\pgfqpoint{3.481537in}{3.050710in}}%
\pgfpathclose%
\pgfusepath{fill}%
\end{pgfscope}%
\begin{pgfscope}%
\pgfpathrectangle{\pgfqpoint{1.254980in}{0.150000in}}{\pgfqpoint{5.490039in}{5.490039in}}%
\pgfusepath{clip}%
\pgfsetbuttcap%
\pgfsetroundjoin%
\definecolor{currentfill}{rgb}{0.282910,0.105393,0.426902}%
\pgfsetfillcolor{currentfill}%
\pgfsetfillopacity{0.700000}%
\pgfsetlinewidth{0.000000pt}%
\definecolor{currentstroke}{rgb}{0.000000,0.000000,0.000000}%
\pgfsetstrokecolor{currentstroke}%
\pgfsetdash{}{0pt}%
\pgfpathmoveto{\pgfqpoint{3.610674in}{3.041835in}}%
\pgfpathlineto{\pgfqpoint{3.623135in}{3.035743in}}%
\pgfpathlineto{\pgfqpoint{3.635598in}{3.029695in}}%
\pgfpathlineto{\pgfqpoint{3.648066in}{3.023690in}}%
\pgfpathlineto{\pgfqpoint{3.660537in}{3.017728in}}%
\pgfpathlineto{\pgfqpoint{3.653174in}{3.007525in}}%
\pgfpathlineto{\pgfqpoint{3.645806in}{2.997449in}}%
\pgfpathlineto{\pgfqpoint{3.638433in}{2.987498in}}%
\pgfpathlineto{\pgfqpoint{3.631053in}{2.977666in}}%
\pgfpathlineto{\pgfqpoint{3.618574in}{2.983509in}}%
\pgfpathlineto{\pgfqpoint{3.606098in}{2.989395in}}%
\pgfpathlineto{\pgfqpoint{3.593625in}{2.995325in}}%
\pgfpathlineto{\pgfqpoint{3.581157in}{3.001298in}}%
\pgfpathlineto{\pgfqpoint{3.588544in}{3.011244in}}%
\pgfpathlineto{\pgfqpoint{3.595927in}{3.021313in}}%
\pgfpathlineto{\pgfqpoint{3.603303in}{3.031509in}}%
\pgfpathlineto{\pgfqpoint{3.610674in}{3.041835in}}%
\pgfpathclose%
\pgfusepath{fill}%
\end{pgfscope}%
\begin{pgfscope}%
\pgfpathrectangle{\pgfqpoint{1.254980in}{0.150000in}}{\pgfqpoint{5.490039in}{5.490039in}}%
\pgfusepath{clip}%
\pgfsetbuttcap%
\pgfsetroundjoin%
\definecolor{currentfill}{rgb}{0.283091,0.110553,0.431554}%
\pgfsetfillcolor{currentfill}%
\pgfsetfillopacity{0.700000}%
\pgfsetlinewidth{0.000000pt}%
\definecolor{currentstroke}{rgb}{0.000000,0.000000,0.000000}%
\pgfsetstrokecolor{currentstroke}%
\pgfsetdash{}{0pt}%
\pgfpathmoveto{\pgfqpoint{3.948177in}{3.054944in}}%
\pgfpathlineto{\pgfqpoint{3.960692in}{3.049226in}}%
\pgfpathlineto{\pgfqpoint{3.973211in}{3.043544in}}%
\pgfpathlineto{\pgfqpoint{3.985734in}{3.037899in}}%
\pgfpathlineto{\pgfqpoint{3.998261in}{3.032290in}}%
\pgfpathlineto{\pgfqpoint{3.990997in}{3.021015in}}%
\pgfpathlineto{\pgfqpoint{3.983729in}{3.009908in}}%
\pgfpathlineto{\pgfqpoint{3.976457in}{2.998962in}}%
\pgfpathlineto{\pgfqpoint{3.969180in}{2.988172in}}%
\pgfpathlineto{\pgfqpoint{3.956645in}{2.993625in}}%
\pgfpathlineto{\pgfqpoint{3.944113in}{2.999115in}}%
\pgfpathlineto{\pgfqpoint{3.931586in}{3.004641in}}%
\pgfpathlineto{\pgfqpoint{3.919064in}{3.010203in}}%
\pgfpathlineto{\pgfqpoint{3.926349in}{3.021144in}}%
\pgfpathlineto{\pgfqpoint{3.933629in}{3.032244in}}%
\pgfpathlineto{\pgfqpoint{3.940905in}{3.043509in}}%
\pgfpathlineto{\pgfqpoint{3.948177in}{3.054944in}}%
\pgfpathclose%
\pgfusepath{fill}%
\end{pgfscope}%
\begin{pgfscope}%
\pgfpathrectangle{\pgfqpoint{1.254980in}{0.150000in}}{\pgfqpoint{5.490039in}{5.490039in}}%
\pgfusepath{clip}%
\pgfsetbuttcap%
\pgfsetroundjoin%
\definecolor{currentfill}{rgb}{0.282910,0.105393,0.426902}%
\pgfsetfillcolor{currentfill}%
\pgfsetfillopacity{0.700000}%
\pgfsetlinewidth{0.000000pt}%
\definecolor{currentstroke}{rgb}{0.000000,0.000000,0.000000}%
\pgfsetstrokecolor{currentstroke}%
\pgfsetdash{}{0pt}%
\pgfpathmoveto{\pgfqpoint{3.739826in}{3.035950in}}%
\pgfpathlineto{\pgfqpoint{3.752308in}{3.030066in}}%
\pgfpathlineto{\pgfqpoint{3.764794in}{3.024222in}}%
\pgfpathlineto{\pgfqpoint{3.777284in}{3.018418in}}%
\pgfpathlineto{\pgfqpoint{3.789778in}{3.012655in}}%
\pgfpathlineto{\pgfqpoint{3.782453in}{3.002170in}}%
\pgfpathlineto{\pgfqpoint{3.775122in}{2.991824in}}%
\pgfpathlineto{\pgfqpoint{3.767787in}{2.981613in}}%
\pgfpathlineto{\pgfqpoint{3.760447in}{2.971530in}}%
\pgfpathlineto{\pgfqpoint{3.747944in}{2.977163in}}%
\pgfpathlineto{\pgfqpoint{3.735445in}{2.982835in}}%
\pgfpathlineto{\pgfqpoint{3.722951in}{2.988548in}}%
\pgfpathlineto{\pgfqpoint{3.710460in}{2.994301in}}%
\pgfpathlineto{\pgfqpoint{3.717809in}{3.004510in}}%
\pgfpathlineto{\pgfqpoint{3.725153in}{3.014851in}}%
\pgfpathlineto{\pgfqpoint{3.732492in}{3.025330in}}%
\pgfpathlineto{\pgfqpoint{3.739826in}{3.035950in}}%
\pgfpathclose%
\pgfusepath{fill}%
\end{pgfscope}%
\begin{pgfscope}%
\pgfpathrectangle{\pgfqpoint{1.254980in}{0.150000in}}{\pgfqpoint{5.490039in}{5.490039in}}%
\pgfusepath{clip}%
\pgfsetbuttcap%
\pgfsetroundjoin%
\definecolor{currentfill}{rgb}{0.282910,0.105393,0.426902}%
\pgfsetfillcolor{currentfill}%
\pgfsetfillopacity{0.700000}%
\pgfsetlinewidth{0.000000pt}%
\definecolor{currentstroke}{rgb}{0.000000,0.000000,0.000000}%
\pgfsetstrokecolor{currentstroke}%
\pgfsetdash{}{0pt}%
\pgfpathmoveto{\pgfqpoint{3.869014in}{3.032827in}}%
\pgfpathlineto{\pgfqpoint{3.881520in}{3.027114in}}%
\pgfpathlineto{\pgfqpoint{3.894031in}{3.021440in}}%
\pgfpathlineto{\pgfqpoint{3.906545in}{3.015803in}}%
\pgfpathlineto{\pgfqpoint{3.919064in}{3.010203in}}%
\pgfpathlineto{\pgfqpoint{3.911774in}{2.999417in}}%
\pgfpathlineto{\pgfqpoint{3.904480in}{2.988782in}}%
\pgfpathlineto{\pgfqpoint{3.897181in}{2.978292in}}%
\pgfpathlineto{\pgfqpoint{3.889878in}{2.967943in}}%
\pgfpathlineto{\pgfqpoint{3.877351in}{2.973399in}}%
\pgfpathlineto{\pgfqpoint{3.864828in}{2.978892in}}%
\pgfpathlineto{\pgfqpoint{3.852309in}{2.984423in}}%
\pgfpathlineto{\pgfqpoint{3.839795in}{2.989992in}}%
\pgfpathlineto{\pgfqpoint{3.847107in}{3.000480in}}%
\pgfpathlineto{\pgfqpoint{3.854414in}{3.011112in}}%
\pgfpathlineto{\pgfqpoint{3.861716in}{3.021893in}}%
\pgfpathlineto{\pgfqpoint{3.869014in}{3.032827in}}%
\pgfpathclose%
\pgfusepath{fill}%
\end{pgfscope}%
\begin{pgfscope}%
\pgfpathrectangle{\pgfqpoint{1.254980in}{0.150000in}}{\pgfqpoint{5.490039in}{5.490039in}}%
\pgfusepath{clip}%
\pgfsetbuttcap%
\pgfsetroundjoin%
\definecolor{currentfill}{rgb}{0.282910,0.105393,0.426902}%
\pgfsetfillcolor{currentfill}%
\pgfsetfillopacity{0.700000}%
\pgfsetlinewidth{0.000000pt}%
\definecolor{currentstroke}{rgb}{0.000000,0.000000,0.000000}%
\pgfsetstrokecolor{currentstroke}%
\pgfsetdash{}{0pt}%
\pgfpathmoveto{\pgfqpoint{3.402098in}{3.036645in}}%
\pgfpathlineto{\pgfqpoint{3.414534in}{3.030219in}}%
\pgfpathlineto{\pgfqpoint{3.426973in}{3.023842in}}%
\pgfpathlineto{\pgfqpoint{3.439415in}{3.017513in}}%
\pgfpathlineto{\pgfqpoint{3.451860in}{3.011233in}}%
\pgfpathlineto{\pgfqpoint{3.444427in}{3.001651in}}%
\pgfpathlineto{\pgfqpoint{3.436987in}{2.992177in}}%
\pgfpathlineto{\pgfqpoint{3.429541in}{2.982808in}}%
\pgfpathlineto{\pgfqpoint{3.422090in}{2.973541in}}%
\pgfpathlineto{\pgfqpoint{3.409635in}{2.979727in}}%
\pgfpathlineto{\pgfqpoint{3.397184in}{2.985961in}}%
\pgfpathlineto{\pgfqpoint{3.384736in}{2.992244in}}%
\pgfpathlineto{\pgfqpoint{3.372291in}{2.998577in}}%
\pgfpathlineto{\pgfqpoint{3.379752in}{3.007933in}}%
\pgfpathlineto{\pgfqpoint{3.387207in}{3.017395in}}%
\pgfpathlineto{\pgfqpoint{3.394655in}{3.026965in}}%
\pgfpathlineto{\pgfqpoint{3.402098in}{3.036645in}}%
\pgfpathclose%
\pgfusepath{fill}%
\end{pgfscope}%
\begin{pgfscope}%
\pgfpathrectangle{\pgfqpoint{1.254980in}{0.150000in}}{\pgfqpoint{5.490039in}{5.490039in}}%
\pgfusepath{clip}%
\pgfsetbuttcap%
\pgfsetroundjoin%
\definecolor{currentfill}{rgb}{0.282656,0.100196,0.422160}%
\pgfsetfillcolor{currentfill}%
\pgfsetfillopacity{0.700000}%
\pgfsetlinewidth{0.000000pt}%
\definecolor{currentstroke}{rgb}{0.000000,0.000000,0.000000}%
\pgfsetstrokecolor{currentstroke}%
\pgfsetdash{}{0pt}%
\pgfpathmoveto{\pgfqpoint{3.531319in}{3.025637in}}%
\pgfpathlineto{\pgfqpoint{3.543773in}{3.019484in}}%
\pgfpathlineto{\pgfqpoint{3.556231in}{3.013377in}}%
\pgfpathlineto{\pgfqpoint{3.568692in}{3.007315in}}%
\pgfpathlineto{\pgfqpoint{3.581157in}{3.001298in}}%
\pgfpathlineto{\pgfqpoint{3.573763in}{2.991472in}}%
\pgfpathlineto{\pgfqpoint{3.566365in}{2.981761in}}%
\pgfpathlineto{\pgfqpoint{3.558960in}{2.972162in}}%
\pgfpathlineto{\pgfqpoint{3.551550in}{2.962673in}}%
\pgfpathlineto{\pgfqpoint{3.539076in}{2.968584in}}%
\pgfpathlineto{\pgfqpoint{3.526606in}{2.974539in}}%
\pgfpathlineto{\pgfqpoint{3.514140in}{2.980539in}}%
\pgfpathlineto{\pgfqpoint{3.501677in}{2.986585in}}%
\pgfpathlineto{\pgfqpoint{3.509096in}{2.996177in}}%
\pgfpathlineto{\pgfqpoint{3.516509in}{3.005881in}}%
\pgfpathlineto{\pgfqpoint{3.523917in}{3.015699in}}%
\pgfpathlineto{\pgfqpoint{3.531319in}{3.025637in}}%
\pgfpathclose%
\pgfusepath{fill}%
\end{pgfscope}%
\begin{pgfscope}%
\pgfpathrectangle{\pgfqpoint{1.254980in}{0.150000in}}{\pgfqpoint{5.490039in}{5.490039in}}%
\pgfusepath{clip}%
\pgfsetbuttcap%
\pgfsetroundjoin%
\definecolor{currentfill}{rgb}{0.282656,0.100196,0.422160}%
\pgfsetfillcolor{currentfill}%
\pgfsetfillopacity{0.700000}%
\pgfsetlinewidth{0.000000pt}%
\definecolor{currentstroke}{rgb}{0.000000,0.000000,0.000000}%
\pgfsetstrokecolor{currentstroke}%
\pgfsetdash{}{0pt}%
\pgfpathmoveto{\pgfqpoint{3.660537in}{3.017728in}}%
\pgfpathlineto{\pgfqpoint{3.673012in}{3.011809in}}%
\pgfpathlineto{\pgfqpoint{3.685491in}{3.005931in}}%
\pgfpathlineto{\pgfqpoint{3.697974in}{3.000096in}}%
\pgfpathlineto{\pgfqpoint{3.710460in}{2.994301in}}%
\pgfpathlineto{\pgfqpoint{3.703106in}{2.984221in}}%
\pgfpathlineto{\pgfqpoint{3.695747in}{2.974266in}}%
\pgfpathlineto{\pgfqpoint{3.688382in}{2.964432in}}%
\pgfpathlineto{\pgfqpoint{3.681012in}{2.954715in}}%
\pgfpathlineto{\pgfqpoint{3.668516in}{2.960390in}}%
\pgfpathlineto{\pgfqpoint{3.656025in}{2.966107in}}%
\pgfpathlineto{\pgfqpoint{3.643537in}{2.971865in}}%
\pgfpathlineto{\pgfqpoint{3.631053in}{2.977666in}}%
\pgfpathlineto{\pgfqpoint{3.638433in}{2.987498in}}%
\pgfpathlineto{\pgfqpoint{3.645806in}{2.997449in}}%
\pgfpathlineto{\pgfqpoint{3.653174in}{3.007525in}}%
\pgfpathlineto{\pgfqpoint{3.660537in}{3.017728in}}%
\pgfpathclose%
\pgfusepath{fill}%
\end{pgfscope}%
\begin{pgfscope}%
\pgfpathrectangle{\pgfqpoint{1.254980in}{0.150000in}}{\pgfqpoint{5.490039in}{5.490039in}}%
\pgfusepath{clip}%
\pgfsetbuttcap%
\pgfsetroundjoin%
\definecolor{currentfill}{rgb}{0.282910,0.105393,0.426902}%
\pgfsetfillcolor{currentfill}%
\pgfsetfillopacity{0.700000}%
\pgfsetlinewidth{0.000000pt}%
\definecolor{currentstroke}{rgb}{0.000000,0.000000,0.000000}%
\pgfsetstrokecolor{currentstroke}%
\pgfsetdash{}{0pt}%
\pgfpathmoveto{\pgfqpoint{3.998261in}{3.032290in}}%
\pgfpathlineto{\pgfqpoint{4.010793in}{3.026717in}}%
\pgfpathlineto{\pgfqpoint{4.023329in}{3.021179in}}%
\pgfpathlineto{\pgfqpoint{4.035869in}{3.015677in}}%
\pgfpathlineto{\pgfqpoint{4.048414in}{3.010210in}}%
\pgfpathlineto{\pgfqpoint{4.041158in}{2.999096in}}%
\pgfpathlineto{\pgfqpoint{4.033899in}{2.988146in}}%
\pgfpathlineto{\pgfqpoint{4.026635in}{2.977354in}}%
\pgfpathlineto{\pgfqpoint{4.019367in}{2.966716in}}%
\pgfpathlineto{\pgfqpoint{4.006814in}{2.972027in}}%
\pgfpathlineto{\pgfqpoint{3.994265in}{2.977373in}}%
\pgfpathlineto{\pgfqpoint{3.981720in}{2.982755in}}%
\pgfpathlineto{\pgfqpoint{3.969180in}{2.988172in}}%
\pgfpathlineto{\pgfqpoint{3.976457in}{2.998962in}}%
\pgfpathlineto{\pgfqpoint{3.983729in}{3.009908in}}%
\pgfpathlineto{\pgfqpoint{3.990997in}{3.021015in}}%
\pgfpathlineto{\pgfqpoint{3.998261in}{3.032290in}}%
\pgfpathclose%
\pgfusepath{fill}%
\end{pgfscope}%
\begin{pgfscope}%
\pgfpathrectangle{\pgfqpoint{1.254980in}{0.150000in}}{\pgfqpoint{5.490039in}{5.490039in}}%
\pgfusepath{clip}%
\pgfsetbuttcap%
\pgfsetroundjoin%
\definecolor{currentfill}{rgb}{0.282656,0.100196,0.422160}%
\pgfsetfillcolor{currentfill}%
\pgfsetfillopacity{0.700000}%
\pgfsetlinewidth{0.000000pt}%
\definecolor{currentstroke}{rgb}{0.000000,0.000000,0.000000}%
\pgfsetstrokecolor{currentstroke}%
\pgfsetdash{}{0pt}%
\pgfpathmoveto{\pgfqpoint{3.789778in}{3.012655in}}%
\pgfpathlineto{\pgfqpoint{3.802276in}{3.006930in}}%
\pgfpathlineto{\pgfqpoint{3.814778in}{3.001246in}}%
\pgfpathlineto{\pgfqpoint{3.827284in}{2.995600in}}%
\pgfpathlineto{\pgfqpoint{3.839795in}{2.989992in}}%
\pgfpathlineto{\pgfqpoint{3.832478in}{2.979644in}}%
\pgfpathlineto{\pgfqpoint{3.825156in}{2.969431in}}%
\pgfpathlineto{\pgfqpoint{3.817829in}{2.959349in}}%
\pgfpathlineto{\pgfqpoint{3.810498in}{2.949393in}}%
\pgfpathlineto{\pgfqpoint{3.797979in}{2.954869in}}%
\pgfpathlineto{\pgfqpoint{3.785464in}{2.960384in}}%
\pgfpathlineto{\pgfqpoint{3.772953in}{2.965938in}}%
\pgfpathlineto{\pgfqpoint{3.760447in}{2.971530in}}%
\pgfpathlineto{\pgfqpoint{3.767787in}{2.981613in}}%
\pgfpathlineto{\pgfqpoint{3.775122in}{2.991824in}}%
\pgfpathlineto{\pgfqpoint{3.782453in}{3.002170in}}%
\pgfpathlineto{\pgfqpoint{3.789778in}{3.012655in}}%
\pgfpathclose%
\pgfusepath{fill}%
\end{pgfscope}%
\begin{pgfscope}%
\pgfpathrectangle{\pgfqpoint{1.254980in}{0.150000in}}{\pgfqpoint{5.490039in}{5.490039in}}%
\pgfusepath{clip}%
\pgfsetbuttcap%
\pgfsetroundjoin%
\definecolor{currentfill}{rgb}{0.282656,0.100196,0.422160}%
\pgfsetfillcolor{currentfill}%
\pgfsetfillopacity{0.700000}%
\pgfsetlinewidth{0.000000pt}%
\definecolor{currentstroke}{rgb}{0.000000,0.000000,0.000000}%
\pgfsetstrokecolor{currentstroke}%
\pgfsetdash{}{0pt}%
\pgfpathmoveto{\pgfqpoint{3.451860in}{3.011233in}}%
\pgfpathlineto{\pgfqpoint{3.464309in}{3.005001in}}%
\pgfpathlineto{\pgfqpoint{3.476762in}{2.998816in}}%
\pgfpathlineto{\pgfqpoint{3.489217in}{2.992677in}}%
\pgfpathlineto{\pgfqpoint{3.501677in}{2.986585in}}%
\pgfpathlineto{\pgfqpoint{3.494252in}{2.977102in}}%
\pgfpathlineto{\pgfqpoint{3.486821in}{2.967724in}}%
\pgfpathlineto{\pgfqpoint{3.479385in}{2.958447in}}%
\pgfpathlineto{\pgfqpoint{3.471942in}{2.949270in}}%
\pgfpathlineto{\pgfqpoint{3.459474in}{2.955267in}}%
\pgfpathlineto{\pgfqpoint{3.447009in}{2.961311in}}%
\pgfpathlineto{\pgfqpoint{3.434548in}{2.967402in}}%
\pgfpathlineto{\pgfqpoint{3.422090in}{2.973541in}}%
\pgfpathlineto{\pgfqpoint{3.429541in}{2.982808in}}%
\pgfpathlineto{\pgfqpoint{3.436987in}{2.992177in}}%
\pgfpathlineto{\pgfqpoint{3.444427in}{3.001651in}}%
\pgfpathlineto{\pgfqpoint{3.451860in}{3.011233in}}%
\pgfpathclose%
\pgfusepath{fill}%
\end{pgfscope}%
\begin{pgfscope}%
\pgfpathrectangle{\pgfqpoint{1.254980in}{0.150000in}}{\pgfqpoint{5.490039in}{5.490039in}}%
\pgfusepath{clip}%
\pgfsetbuttcap%
\pgfsetroundjoin%
\definecolor{currentfill}{rgb}{0.282656,0.100196,0.422160}%
\pgfsetfillcolor{currentfill}%
\pgfsetfillopacity{0.700000}%
\pgfsetlinewidth{0.000000pt}%
\definecolor{currentstroke}{rgb}{0.000000,0.000000,0.000000}%
\pgfsetstrokecolor{currentstroke}%
\pgfsetdash{}{0pt}%
\pgfpathmoveto{\pgfqpoint{3.919064in}{3.010203in}}%
\pgfpathlineto{\pgfqpoint{3.931586in}{3.004641in}}%
\pgfpathlineto{\pgfqpoint{3.944113in}{2.999115in}}%
\pgfpathlineto{\pgfqpoint{3.956645in}{2.993625in}}%
\pgfpathlineto{\pgfqpoint{3.969180in}{2.988172in}}%
\pgfpathlineto{\pgfqpoint{3.961899in}{2.977535in}}%
\pgfpathlineto{\pgfqpoint{3.954614in}{2.967044in}}%
\pgfpathlineto{\pgfqpoint{3.947324in}{2.956696in}}%
\pgfpathlineto{\pgfqpoint{3.940029in}{2.946486in}}%
\pgfpathlineto{\pgfqpoint{3.927485in}{2.951796in}}%
\pgfpathlineto{\pgfqpoint{3.914945in}{2.957142in}}%
\pgfpathlineto{\pgfqpoint{3.902409in}{2.962524in}}%
\pgfpathlineto{\pgfqpoint{3.889878in}{2.967943in}}%
\pgfpathlineto{\pgfqpoint{3.897181in}{2.978292in}}%
\pgfpathlineto{\pgfqpoint{3.904480in}{2.988782in}}%
\pgfpathlineto{\pgfqpoint{3.911774in}{2.999417in}}%
\pgfpathlineto{\pgfqpoint{3.919064in}{3.010203in}}%
\pgfpathclose%
\pgfusepath{fill}%
\end{pgfscope}%
\begin{pgfscope}%
\pgfpathrectangle{\pgfqpoint{1.254980in}{0.150000in}}{\pgfqpoint{5.490039in}{5.490039in}}%
\pgfusepath{clip}%
\pgfsetbuttcap%
\pgfsetroundjoin%
\definecolor{currentfill}{rgb}{0.282327,0.094955,0.417331}%
\pgfsetfillcolor{currentfill}%
\pgfsetfillopacity{0.700000}%
\pgfsetlinewidth{0.000000pt}%
\definecolor{currentstroke}{rgb}{0.000000,0.000000,0.000000}%
\pgfsetstrokecolor{currentstroke}%
\pgfsetdash{}{0pt}%
\pgfpathmoveto{\pgfqpoint{3.581157in}{3.001298in}}%
\pgfpathlineto{\pgfqpoint{3.593625in}{2.995325in}}%
\pgfpathlineto{\pgfqpoint{3.606098in}{2.989395in}}%
\pgfpathlineto{\pgfqpoint{3.618574in}{2.983509in}}%
\pgfpathlineto{\pgfqpoint{3.631053in}{2.977666in}}%
\pgfpathlineto{\pgfqpoint{3.623669in}{2.967951in}}%
\pgfpathlineto{\pgfqpoint{3.616279in}{2.958348in}}%
\pgfpathlineto{\pgfqpoint{3.608884in}{2.948855in}}%
\pgfpathlineto{\pgfqpoint{3.601483in}{2.939467in}}%
\pgfpathlineto{\pgfqpoint{3.588994in}{2.945203in}}%
\pgfpathlineto{\pgfqpoint{3.576509in}{2.950983in}}%
\pgfpathlineto{\pgfqpoint{3.564027in}{2.956806in}}%
\pgfpathlineto{\pgfqpoint{3.551550in}{2.962673in}}%
\pgfpathlineto{\pgfqpoint{3.558960in}{2.972162in}}%
\pgfpathlineto{\pgfqpoint{3.566365in}{2.981761in}}%
\pgfpathlineto{\pgfqpoint{3.573763in}{2.991472in}}%
\pgfpathlineto{\pgfqpoint{3.581157in}{3.001298in}}%
\pgfpathclose%
\pgfusepath{fill}%
\end{pgfscope}%
\begin{pgfscope}%
\pgfpathrectangle{\pgfqpoint{1.254980in}{0.150000in}}{\pgfqpoint{5.490039in}{5.490039in}}%
\pgfusepath{clip}%
\pgfsetbuttcap%
\pgfsetroundjoin%
\definecolor{currentfill}{rgb}{0.282327,0.094955,0.417331}%
\pgfsetfillcolor{currentfill}%
\pgfsetfillopacity{0.700000}%
\pgfsetlinewidth{0.000000pt}%
\definecolor{currentstroke}{rgb}{0.000000,0.000000,0.000000}%
\pgfsetstrokecolor{currentstroke}%
\pgfsetdash{}{0pt}%
\pgfpathmoveto{\pgfqpoint{3.710460in}{2.994301in}}%
\pgfpathlineto{\pgfqpoint{3.722951in}{2.988548in}}%
\pgfpathlineto{\pgfqpoint{3.735445in}{2.982835in}}%
\pgfpathlineto{\pgfqpoint{3.747944in}{2.977163in}}%
\pgfpathlineto{\pgfqpoint{3.760447in}{2.971530in}}%
\pgfpathlineto{\pgfqpoint{3.753101in}{2.961574in}}%
\pgfpathlineto{\pgfqpoint{3.745750in}{2.951739in}}%
\pgfpathlineto{\pgfqpoint{3.738394in}{2.942022in}}%
\pgfpathlineto{\pgfqpoint{3.731033in}{2.932420in}}%
\pgfpathlineto{\pgfqpoint{3.718522in}{2.937933in}}%
\pgfpathlineto{\pgfqpoint{3.706014in}{2.943486in}}%
\pgfpathlineto{\pgfqpoint{3.693511in}{2.949080in}}%
\pgfpathlineto{\pgfqpoint{3.681012in}{2.954715in}}%
\pgfpathlineto{\pgfqpoint{3.688382in}{2.964432in}}%
\pgfpathlineto{\pgfqpoint{3.695747in}{2.974266in}}%
\pgfpathlineto{\pgfqpoint{3.703106in}{2.984221in}}%
\pgfpathlineto{\pgfqpoint{3.710460in}{2.994301in}}%
\pgfpathclose%
\pgfusepath{fill}%
\end{pgfscope}%
\begin{pgfscope}%
\pgfpathrectangle{\pgfqpoint{1.254980in}{0.150000in}}{\pgfqpoint{5.490039in}{5.490039in}}%
\pgfusepath{clip}%
\pgfsetbuttcap%
\pgfsetroundjoin%
\definecolor{currentfill}{rgb}{0.282656,0.100196,0.422160}%
\pgfsetfillcolor{currentfill}%
\pgfsetfillopacity{0.700000}%
\pgfsetlinewidth{0.000000pt}%
\definecolor{currentstroke}{rgb}{0.000000,0.000000,0.000000}%
\pgfsetstrokecolor{currentstroke}%
\pgfsetdash{}{0pt}%
\pgfpathmoveto{\pgfqpoint{4.048414in}{3.010210in}}%
\pgfpathlineto{\pgfqpoint{4.060963in}{3.004778in}}%
\pgfpathlineto{\pgfqpoint{4.073516in}{2.999380in}}%
\pgfpathlineto{\pgfqpoint{4.086074in}{2.994017in}}%
\pgfpathlineto{\pgfqpoint{4.098636in}{2.988687in}}%
\pgfpathlineto{\pgfqpoint{4.091389in}{2.977734in}}%
\pgfpathlineto{\pgfqpoint{4.084139in}{2.966941in}}%
\pgfpathlineto{\pgfqpoint{4.076884in}{2.956304in}}%
\pgfpathlineto{\pgfqpoint{4.069625in}{2.945817in}}%
\pgfpathlineto{\pgfqpoint{4.057053in}{2.950990in}}%
\pgfpathlineto{\pgfqpoint{4.044487in}{2.956198in}}%
\pgfpathlineto{\pgfqpoint{4.031925in}{2.961439in}}%
\pgfpathlineto{\pgfqpoint{4.019367in}{2.966716in}}%
\pgfpathlineto{\pgfqpoint{4.026635in}{2.977354in}}%
\pgfpathlineto{\pgfqpoint{4.033899in}{2.988146in}}%
\pgfpathlineto{\pgfqpoint{4.041158in}{2.999096in}}%
\pgfpathlineto{\pgfqpoint{4.048414in}{3.010210in}}%
\pgfpathclose%
\pgfusepath{fill}%
\end{pgfscope}%
\begin{pgfscope}%
\pgfpathrectangle{\pgfqpoint{1.254980in}{0.150000in}}{\pgfqpoint{5.490039in}{5.490039in}}%
\pgfusepath{clip}%
\pgfsetbuttcap%
\pgfsetroundjoin%
\definecolor{currentfill}{rgb}{0.282327,0.094955,0.417331}%
\pgfsetfillcolor{currentfill}%
\pgfsetfillopacity{0.700000}%
\pgfsetlinewidth{0.000000pt}%
\definecolor{currentstroke}{rgb}{0.000000,0.000000,0.000000}%
\pgfsetstrokecolor{currentstroke}%
\pgfsetdash{}{0pt}%
\pgfpathmoveto{\pgfqpoint{3.839795in}{2.989992in}}%
\pgfpathlineto{\pgfqpoint{3.852309in}{2.984423in}}%
\pgfpathlineto{\pgfqpoint{3.864828in}{2.978892in}}%
\pgfpathlineto{\pgfqpoint{3.877351in}{2.973399in}}%
\pgfpathlineto{\pgfqpoint{3.889878in}{2.967943in}}%
\pgfpathlineto{\pgfqpoint{3.882570in}{2.957730in}}%
\pgfpathlineto{\pgfqpoint{3.875257in}{2.947650in}}%
\pgfpathlineto{\pgfqpoint{3.867939in}{2.937698in}}%
\pgfpathlineto{\pgfqpoint{3.860616in}{2.927869in}}%
\pgfpathlineto{\pgfqpoint{3.848080in}{2.933194in}}%
\pgfpathlineto{\pgfqpoint{3.835549in}{2.938556in}}%
\pgfpathlineto{\pgfqpoint{3.823021in}{2.943956in}}%
\pgfpathlineto{\pgfqpoint{3.810498in}{2.949393in}}%
\pgfpathlineto{\pgfqpoint{3.817829in}{2.959349in}}%
\pgfpathlineto{\pgfqpoint{3.825156in}{2.969431in}}%
\pgfpathlineto{\pgfqpoint{3.832478in}{2.979644in}}%
\pgfpathlineto{\pgfqpoint{3.839795in}{2.989992in}}%
\pgfpathclose%
\pgfusepath{fill}%
\end{pgfscope}%
\begin{pgfscope}%
\pgfpathrectangle{\pgfqpoint{1.254980in}{0.150000in}}{\pgfqpoint{5.490039in}{5.490039in}}%
\pgfusepath{clip}%
\pgfsetbuttcap%
\pgfsetroundjoin%
\definecolor{currentfill}{rgb}{0.282656,0.100196,0.422160}%
\pgfsetfillcolor{currentfill}%
\pgfsetfillopacity{0.700000}%
\pgfsetlinewidth{0.000000pt}%
\definecolor{currentstroke}{rgb}{0.000000,0.000000,0.000000}%
\pgfsetstrokecolor{currentstroke}%
\pgfsetdash{}{0pt}%
\pgfpathmoveto{\pgfqpoint{3.372291in}{2.998577in}}%
\pgfpathlineto{\pgfqpoint{3.384736in}{2.992244in}}%
\pgfpathlineto{\pgfqpoint{3.397184in}{2.985961in}}%
\pgfpathlineto{\pgfqpoint{3.409635in}{2.979727in}}%
\pgfpathlineto{\pgfqpoint{3.422090in}{2.973541in}}%
\pgfpathlineto{\pgfqpoint{3.414632in}{2.964372in}}%
\pgfpathlineto{\pgfqpoint{3.407169in}{2.955300in}}%
\pgfpathlineto{\pgfqpoint{3.399699in}{2.946320in}}%
\pgfpathlineto{\pgfqpoint{3.392224in}{2.937430in}}%
\pgfpathlineto{\pgfqpoint{3.379760in}{2.943535in}}%
\pgfpathlineto{\pgfqpoint{3.367299in}{2.949687in}}%
\pgfpathlineto{\pgfqpoint{3.354842in}{2.955889in}}%
\pgfpathlineto{\pgfqpoint{3.342388in}{2.962139in}}%
\pgfpathlineto{\pgfqpoint{3.349873in}{2.971106in}}%
\pgfpathlineto{\pgfqpoint{3.357352in}{2.980166in}}%
\pgfpathlineto{\pgfqpoint{3.364825in}{2.989322in}}%
\pgfpathlineto{\pgfqpoint{3.372291in}{2.998577in}}%
\pgfpathclose%
\pgfusepath{fill}%
\end{pgfscope}%
\begin{pgfscope}%
\pgfpathrectangle{\pgfqpoint{1.254980in}{0.150000in}}{\pgfqpoint{5.490039in}{5.490039in}}%
\pgfusepath{clip}%
\pgfsetbuttcap%
\pgfsetroundjoin%
\definecolor{currentfill}{rgb}{0.282327,0.094955,0.417331}%
\pgfsetfillcolor{currentfill}%
\pgfsetfillopacity{0.700000}%
\pgfsetlinewidth{0.000000pt}%
\definecolor{currentstroke}{rgb}{0.000000,0.000000,0.000000}%
\pgfsetstrokecolor{currentstroke}%
\pgfsetdash{}{0pt}%
\pgfpathmoveto{\pgfqpoint{3.501677in}{2.986585in}}%
\pgfpathlineto{\pgfqpoint{3.514140in}{2.980539in}}%
\pgfpathlineto{\pgfqpoint{3.526606in}{2.974539in}}%
\pgfpathlineto{\pgfqpoint{3.539076in}{2.968584in}}%
\pgfpathlineto{\pgfqpoint{3.551550in}{2.962673in}}%
\pgfpathlineto{\pgfqpoint{3.544134in}{2.953288in}}%
\pgfpathlineto{\pgfqpoint{3.536713in}{2.944006in}}%
\pgfpathlineto{\pgfqpoint{3.529286in}{2.934822in}}%
\pgfpathlineto{\pgfqpoint{3.521853in}{2.925734in}}%
\pgfpathlineto{\pgfqpoint{3.509370in}{2.931550in}}%
\pgfpathlineto{\pgfqpoint{3.496890in}{2.937412in}}%
\pgfpathlineto{\pgfqpoint{3.484415in}{2.943318in}}%
\pgfpathlineto{\pgfqpoint{3.471942in}{2.949270in}}%
\pgfpathlineto{\pgfqpoint{3.479385in}{2.958447in}}%
\pgfpathlineto{\pgfqpoint{3.486821in}{2.967724in}}%
\pgfpathlineto{\pgfqpoint{3.494252in}{2.977102in}}%
\pgfpathlineto{\pgfqpoint{3.501677in}{2.986585in}}%
\pgfpathclose%
\pgfusepath{fill}%
\end{pgfscope}%
\begin{pgfscope}%
\pgfpathrectangle{\pgfqpoint{1.254980in}{0.150000in}}{\pgfqpoint{5.490039in}{5.490039in}}%
\pgfusepath{clip}%
\pgfsetbuttcap%
\pgfsetroundjoin%
\definecolor{currentfill}{rgb}{0.282327,0.094955,0.417331}%
\pgfsetfillcolor{currentfill}%
\pgfsetfillopacity{0.700000}%
\pgfsetlinewidth{0.000000pt}%
\definecolor{currentstroke}{rgb}{0.000000,0.000000,0.000000}%
\pgfsetstrokecolor{currentstroke}%
\pgfsetdash{}{0pt}%
\pgfpathmoveto{\pgfqpoint{3.969180in}{2.988172in}}%
\pgfpathlineto{\pgfqpoint{3.981720in}{2.982755in}}%
\pgfpathlineto{\pgfqpoint{3.994265in}{2.977373in}}%
\pgfpathlineto{\pgfqpoint{4.006814in}{2.972027in}}%
\pgfpathlineto{\pgfqpoint{4.019367in}{2.966716in}}%
\pgfpathlineto{\pgfqpoint{4.012095in}{2.956226in}}%
\pgfpathlineto{\pgfqpoint{4.004818in}{2.945881in}}%
\pgfpathlineto{\pgfqpoint{3.997537in}{2.935675in}}%
\pgfpathlineto{\pgfqpoint{3.990252in}{2.925605in}}%
\pgfpathlineto{\pgfqpoint{3.977689in}{2.930772in}}%
\pgfpathlineto{\pgfqpoint{3.965132in}{2.935975in}}%
\pgfpathlineto{\pgfqpoint{3.952578in}{2.941213in}}%
\pgfpathlineto{\pgfqpoint{3.940029in}{2.946486in}}%
\pgfpathlineto{\pgfqpoint{3.947324in}{2.956696in}}%
\pgfpathlineto{\pgfqpoint{3.954614in}{2.967044in}}%
\pgfpathlineto{\pgfqpoint{3.961899in}{2.977535in}}%
\pgfpathlineto{\pgfqpoint{3.969180in}{2.988172in}}%
\pgfpathclose%
\pgfusepath{fill}%
\end{pgfscope}%
\begin{pgfscope}%
\pgfpathrectangle{\pgfqpoint{1.254980in}{0.150000in}}{\pgfqpoint{5.490039in}{5.490039in}}%
\pgfusepath{clip}%
\pgfsetbuttcap%
\pgfsetroundjoin%
\definecolor{currentfill}{rgb}{0.281924,0.089666,0.412415}%
\pgfsetfillcolor{currentfill}%
\pgfsetfillopacity{0.700000}%
\pgfsetlinewidth{0.000000pt}%
\definecolor{currentstroke}{rgb}{0.000000,0.000000,0.000000}%
\pgfsetstrokecolor{currentstroke}%
\pgfsetdash{}{0pt}%
\pgfpathmoveto{\pgfqpoint{3.631053in}{2.977666in}}%
\pgfpathlineto{\pgfqpoint{3.643537in}{2.971865in}}%
\pgfpathlineto{\pgfqpoint{3.656025in}{2.966107in}}%
\pgfpathlineto{\pgfqpoint{3.668516in}{2.960390in}}%
\pgfpathlineto{\pgfqpoint{3.681012in}{2.954715in}}%
\pgfpathlineto{\pgfqpoint{3.673636in}{2.945110in}}%
\pgfpathlineto{\pgfqpoint{3.666255in}{2.935616in}}%
\pgfpathlineto{\pgfqpoint{3.658869in}{2.926228in}}%
\pgfpathlineto{\pgfqpoint{3.651477in}{2.916942in}}%
\pgfpathlineto{\pgfqpoint{3.638973in}{2.922511in}}%
\pgfpathlineto{\pgfqpoint{3.626472in}{2.928121in}}%
\pgfpathlineto{\pgfqpoint{3.613975in}{2.933773in}}%
\pgfpathlineto{\pgfqpoint{3.601483in}{2.939467in}}%
\pgfpathlineto{\pgfqpoint{3.608884in}{2.948855in}}%
\pgfpathlineto{\pgfqpoint{3.616279in}{2.958348in}}%
\pgfpathlineto{\pgfqpoint{3.623669in}{2.967951in}}%
\pgfpathlineto{\pgfqpoint{3.631053in}{2.977666in}}%
\pgfpathclose%
\pgfusepath{fill}%
\end{pgfscope}%
\begin{pgfscope}%
\pgfpathrectangle{\pgfqpoint{1.254980in}{0.150000in}}{\pgfqpoint{5.490039in}{5.490039in}}%
\pgfusepath{clip}%
\pgfsetbuttcap%
\pgfsetroundjoin%
\definecolor{currentfill}{rgb}{0.281924,0.089666,0.412415}%
\pgfsetfillcolor{currentfill}%
\pgfsetfillopacity{0.700000}%
\pgfsetlinewidth{0.000000pt}%
\definecolor{currentstroke}{rgb}{0.000000,0.000000,0.000000}%
\pgfsetstrokecolor{currentstroke}%
\pgfsetdash{}{0pt}%
\pgfpathmoveto{\pgfqpoint{3.760447in}{2.971530in}}%
\pgfpathlineto{\pgfqpoint{3.772953in}{2.965938in}}%
\pgfpathlineto{\pgfqpoint{3.785464in}{2.960384in}}%
\pgfpathlineto{\pgfqpoint{3.797979in}{2.954869in}}%
\pgfpathlineto{\pgfqpoint{3.810498in}{2.949393in}}%
\pgfpathlineto{\pgfqpoint{3.803161in}{2.939561in}}%
\pgfpathlineto{\pgfqpoint{3.795819in}{2.929847in}}%
\pgfpathlineto{\pgfqpoint{3.788473in}{2.920247in}}%
\pgfpathlineto{\pgfqpoint{3.781121in}{2.910759in}}%
\pgfpathlineto{\pgfqpoint{3.768593in}{2.916116in}}%
\pgfpathlineto{\pgfqpoint{3.756069in}{2.921511in}}%
\pgfpathlineto{\pgfqpoint{3.743549in}{2.926946in}}%
\pgfpathlineto{\pgfqpoint{3.731033in}{2.932420in}}%
\pgfpathlineto{\pgfqpoint{3.738394in}{2.942022in}}%
\pgfpathlineto{\pgfqpoint{3.745750in}{2.951739in}}%
\pgfpathlineto{\pgfqpoint{3.753101in}{2.961574in}}%
\pgfpathlineto{\pgfqpoint{3.760447in}{2.971530in}}%
\pgfpathclose%
\pgfusepath{fill}%
\end{pgfscope}%
\begin{pgfscope}%
\pgfpathrectangle{\pgfqpoint{1.254980in}{0.150000in}}{\pgfqpoint{5.490039in}{5.490039in}}%
\pgfusepath{clip}%
\pgfsetbuttcap%
\pgfsetroundjoin%
\definecolor{currentfill}{rgb}{0.282327,0.094955,0.417331}%
\pgfsetfillcolor{currentfill}%
\pgfsetfillopacity{0.700000}%
\pgfsetlinewidth{0.000000pt}%
\definecolor{currentstroke}{rgb}{0.000000,0.000000,0.000000}%
\pgfsetstrokecolor{currentstroke}%
\pgfsetdash{}{0pt}%
\pgfpathmoveto{\pgfqpoint{4.098636in}{2.988687in}}%
\pgfpathlineto{\pgfqpoint{4.111203in}{2.983392in}}%
\pgfpathlineto{\pgfqpoint{4.123775in}{2.978130in}}%
\pgfpathlineto{\pgfqpoint{4.136351in}{2.972902in}}%
\pgfpathlineto{\pgfqpoint{4.148931in}{2.967707in}}%
\pgfpathlineto{\pgfqpoint{4.141693in}{2.956914in}}%
\pgfpathlineto{\pgfqpoint{4.134451in}{2.946278in}}%
\pgfpathlineto{\pgfqpoint{4.127205in}{2.935796in}}%
\pgfpathlineto{\pgfqpoint{4.119956in}{2.925461in}}%
\pgfpathlineto{\pgfqpoint{4.107366in}{2.930500in}}%
\pgfpathlineto{\pgfqpoint{4.094781in}{2.935572in}}%
\pgfpathlineto{\pgfqpoint{4.082200in}{2.940678in}}%
\pgfpathlineto{\pgfqpoint{4.069625in}{2.945817in}}%
\pgfpathlineto{\pgfqpoint{4.076884in}{2.956304in}}%
\pgfpathlineto{\pgfqpoint{4.084139in}{2.966941in}}%
\pgfpathlineto{\pgfqpoint{4.091389in}{2.977734in}}%
\pgfpathlineto{\pgfqpoint{4.098636in}{2.988687in}}%
\pgfpathclose%
\pgfusepath{fill}%
\end{pgfscope}%
\begin{pgfscope}%
\pgfpathrectangle{\pgfqpoint{1.254980in}{0.150000in}}{\pgfqpoint{5.490039in}{5.490039in}}%
\pgfusepath{clip}%
\pgfsetbuttcap%
\pgfsetroundjoin%
\definecolor{currentfill}{rgb}{0.281924,0.089666,0.412415}%
\pgfsetfillcolor{currentfill}%
\pgfsetfillopacity{0.700000}%
\pgfsetlinewidth{0.000000pt}%
\definecolor{currentstroke}{rgb}{0.000000,0.000000,0.000000}%
\pgfsetstrokecolor{currentstroke}%
\pgfsetdash{}{0pt}%
\pgfpathmoveto{\pgfqpoint{3.889878in}{2.967943in}}%
\pgfpathlineto{\pgfqpoint{3.902409in}{2.962524in}}%
\pgfpathlineto{\pgfqpoint{3.914945in}{2.957142in}}%
\pgfpathlineto{\pgfqpoint{3.927485in}{2.951796in}}%
\pgfpathlineto{\pgfqpoint{3.940029in}{2.946486in}}%
\pgfpathlineto{\pgfqpoint{3.932730in}{2.936410in}}%
\pgfpathlineto{\pgfqpoint{3.925426in}{2.926463in}}%
\pgfpathlineto{\pgfqpoint{3.918118in}{2.916640in}}%
\pgfpathlineto{\pgfqpoint{3.910804in}{2.906939in}}%
\pgfpathlineto{\pgfqpoint{3.898251in}{2.912117in}}%
\pgfpathlineto{\pgfqpoint{3.885702in}{2.917331in}}%
\pgfpathlineto{\pgfqpoint{3.873157in}{2.922582in}}%
\pgfpathlineto{\pgfqpoint{3.860616in}{2.927869in}}%
\pgfpathlineto{\pgfqpoint{3.867939in}{2.937698in}}%
\pgfpathlineto{\pgfqpoint{3.875257in}{2.947650in}}%
\pgfpathlineto{\pgfqpoint{3.882570in}{2.957730in}}%
\pgfpathlineto{\pgfqpoint{3.889878in}{2.967943in}}%
\pgfpathclose%
\pgfusepath{fill}%
\end{pgfscope}%
\begin{pgfscope}%
\pgfpathrectangle{\pgfqpoint{1.254980in}{0.150000in}}{\pgfqpoint{5.490039in}{5.490039in}}%
\pgfusepath{clip}%
\pgfsetbuttcap%
\pgfsetroundjoin%
\definecolor{currentfill}{rgb}{0.281924,0.089666,0.412415}%
\pgfsetfillcolor{currentfill}%
\pgfsetfillopacity{0.700000}%
\pgfsetlinewidth{0.000000pt}%
\definecolor{currentstroke}{rgb}{0.000000,0.000000,0.000000}%
\pgfsetstrokecolor{currentstroke}%
\pgfsetdash{}{0pt}%
\pgfpathmoveto{\pgfqpoint{3.422090in}{2.973541in}}%
\pgfpathlineto{\pgfqpoint{3.434548in}{2.967402in}}%
\pgfpathlineto{\pgfqpoint{3.447009in}{2.961311in}}%
\pgfpathlineto{\pgfqpoint{3.459474in}{2.955267in}}%
\pgfpathlineto{\pgfqpoint{3.471942in}{2.949270in}}%
\pgfpathlineto{\pgfqpoint{3.464494in}{2.940188in}}%
\pgfpathlineto{\pgfqpoint{3.457040in}{2.931198in}}%
\pgfpathlineto{\pgfqpoint{3.449581in}{2.922299in}}%
\pgfpathlineto{\pgfqpoint{3.442115in}{2.913487in}}%
\pgfpathlineto{\pgfqpoint{3.429637in}{2.919403in}}%
\pgfpathlineto{\pgfqpoint{3.417162in}{2.925365in}}%
\pgfpathlineto{\pgfqpoint{3.404691in}{2.931374in}}%
\pgfpathlineto{\pgfqpoint{3.392224in}{2.937430in}}%
\pgfpathlineto{\pgfqpoint{3.399699in}{2.946320in}}%
\pgfpathlineto{\pgfqpoint{3.407169in}{2.955300in}}%
\pgfpathlineto{\pgfqpoint{3.414632in}{2.964372in}}%
\pgfpathlineto{\pgfqpoint{3.422090in}{2.973541in}}%
\pgfpathclose%
\pgfusepath{fill}%
\end{pgfscope}%
\begin{pgfscope}%
\pgfpathrectangle{\pgfqpoint{1.254980in}{0.150000in}}{\pgfqpoint{5.490039in}{5.490039in}}%
\pgfusepath{clip}%
\pgfsetbuttcap%
\pgfsetroundjoin%
\definecolor{currentfill}{rgb}{0.281446,0.084320,0.407414}%
\pgfsetfillcolor{currentfill}%
\pgfsetfillopacity{0.700000}%
\pgfsetlinewidth{0.000000pt}%
\definecolor{currentstroke}{rgb}{0.000000,0.000000,0.000000}%
\pgfsetstrokecolor{currentstroke}%
\pgfsetdash{}{0pt}%
\pgfpathmoveto{\pgfqpoint{3.551550in}{2.962673in}}%
\pgfpathlineto{\pgfqpoint{3.564027in}{2.956806in}}%
\pgfpathlineto{\pgfqpoint{3.576509in}{2.950983in}}%
\pgfpathlineto{\pgfqpoint{3.588994in}{2.945203in}}%
\pgfpathlineto{\pgfqpoint{3.601483in}{2.939467in}}%
\pgfpathlineto{\pgfqpoint{3.594076in}{2.930181in}}%
\pgfpathlineto{\pgfqpoint{3.586664in}{2.920995in}}%
\pgfpathlineto{\pgfqpoint{3.579246in}{2.911904in}}%
\pgfpathlineto{\pgfqpoint{3.571823in}{2.902906in}}%
\pgfpathlineto{\pgfqpoint{3.559325in}{2.908548in}}%
\pgfpathlineto{\pgfqpoint{3.546830in}{2.914233in}}%
\pgfpathlineto{\pgfqpoint{3.534339in}{2.919962in}}%
\pgfpathlineto{\pgfqpoint{3.521853in}{2.925734in}}%
\pgfpathlineto{\pgfqpoint{3.529286in}{2.934822in}}%
\pgfpathlineto{\pgfqpoint{3.536713in}{2.944006in}}%
\pgfpathlineto{\pgfqpoint{3.544134in}{2.953288in}}%
\pgfpathlineto{\pgfqpoint{3.551550in}{2.962673in}}%
\pgfpathclose%
\pgfusepath{fill}%
\end{pgfscope}%
\begin{pgfscope}%
\pgfpathrectangle{\pgfqpoint{1.254980in}{0.150000in}}{\pgfqpoint{5.490039in}{5.490039in}}%
\pgfusepath{clip}%
\pgfsetbuttcap%
\pgfsetroundjoin%
\definecolor{currentfill}{rgb}{0.281924,0.089666,0.412415}%
\pgfsetfillcolor{currentfill}%
\pgfsetfillopacity{0.700000}%
\pgfsetlinewidth{0.000000pt}%
\definecolor{currentstroke}{rgb}{0.000000,0.000000,0.000000}%
\pgfsetstrokecolor{currentstroke}%
\pgfsetdash{}{0pt}%
\pgfpathmoveto{\pgfqpoint{4.019367in}{2.966716in}}%
\pgfpathlineto{\pgfqpoint{4.031925in}{2.961439in}}%
\pgfpathlineto{\pgfqpoint{4.044487in}{2.956198in}}%
\pgfpathlineto{\pgfqpoint{4.057053in}{2.950990in}}%
\pgfpathlineto{\pgfqpoint{4.069625in}{2.945817in}}%
\pgfpathlineto{\pgfqpoint{4.062361in}{2.935476in}}%
\pgfpathlineto{\pgfqpoint{4.055094in}{2.925276in}}%
\pgfpathlineto{\pgfqpoint{4.047822in}{2.915213in}}%
\pgfpathlineto{\pgfqpoint{4.040546in}{2.905282in}}%
\pgfpathlineto{\pgfqpoint{4.027965in}{2.910311in}}%
\pgfpathlineto{\pgfqpoint{4.015390in}{2.915374in}}%
\pgfpathlineto{\pgfqpoint{4.002818in}{2.920472in}}%
\pgfpathlineto{\pgfqpoint{3.990252in}{2.925605in}}%
\pgfpathlineto{\pgfqpoint{3.997537in}{2.935675in}}%
\pgfpathlineto{\pgfqpoint{4.004818in}{2.945881in}}%
\pgfpathlineto{\pgfqpoint{4.012095in}{2.956226in}}%
\pgfpathlineto{\pgfqpoint{4.019367in}{2.966716in}}%
\pgfpathclose%
\pgfusepath{fill}%
\end{pgfscope}%
\begin{pgfscope}%
\pgfpathrectangle{\pgfqpoint{1.254980in}{0.150000in}}{\pgfqpoint{5.490039in}{5.490039in}}%
\pgfusepath{clip}%
\pgfsetbuttcap%
\pgfsetroundjoin%
\definecolor{currentfill}{rgb}{0.281446,0.084320,0.407414}%
\pgfsetfillcolor{currentfill}%
\pgfsetfillopacity{0.700000}%
\pgfsetlinewidth{0.000000pt}%
\definecolor{currentstroke}{rgb}{0.000000,0.000000,0.000000}%
\pgfsetstrokecolor{currentstroke}%
\pgfsetdash{}{0pt}%
\pgfpathmoveto{\pgfqpoint{3.681012in}{2.954715in}}%
\pgfpathlineto{\pgfqpoint{3.693511in}{2.949080in}}%
\pgfpathlineto{\pgfqpoint{3.706014in}{2.943486in}}%
\pgfpathlineto{\pgfqpoint{3.718522in}{2.937933in}}%
\pgfpathlineto{\pgfqpoint{3.731033in}{2.932420in}}%
\pgfpathlineto{\pgfqpoint{3.723667in}{2.922927in}}%
\pgfpathlineto{\pgfqpoint{3.716295in}{2.913541in}}%
\pgfpathlineto{\pgfqpoint{3.708918in}{2.904258in}}%
\pgfpathlineto{\pgfqpoint{3.701536in}{2.895074in}}%
\pgfpathlineto{\pgfqpoint{3.689015in}{2.900481in}}%
\pgfpathlineto{\pgfqpoint{3.676498in}{2.905928in}}%
\pgfpathlineto{\pgfqpoint{3.663986in}{2.911414in}}%
\pgfpathlineto{\pgfqpoint{3.651477in}{2.916942in}}%
\pgfpathlineto{\pgfqpoint{3.658869in}{2.926228in}}%
\pgfpathlineto{\pgfqpoint{3.666255in}{2.935616in}}%
\pgfpathlineto{\pgfqpoint{3.673636in}{2.945110in}}%
\pgfpathlineto{\pgfqpoint{3.681012in}{2.954715in}}%
\pgfpathclose%
\pgfusepath{fill}%
\end{pgfscope}%
\begin{pgfscope}%
\pgfpathrectangle{\pgfqpoint{1.254980in}{0.150000in}}{\pgfqpoint{5.490039in}{5.490039in}}%
\pgfusepath{clip}%
\pgfsetbuttcap%
\pgfsetroundjoin%
\definecolor{currentfill}{rgb}{0.281446,0.084320,0.407414}%
\pgfsetfillcolor{currentfill}%
\pgfsetfillopacity{0.700000}%
\pgfsetlinewidth{0.000000pt}%
\definecolor{currentstroke}{rgb}{0.000000,0.000000,0.000000}%
\pgfsetstrokecolor{currentstroke}%
\pgfsetdash{}{0pt}%
\pgfpathmoveto{\pgfqpoint{3.810498in}{2.949393in}}%
\pgfpathlineto{\pgfqpoint{3.823021in}{2.943956in}}%
\pgfpathlineto{\pgfqpoint{3.835549in}{2.938556in}}%
\pgfpathlineto{\pgfqpoint{3.848080in}{2.933194in}}%
\pgfpathlineto{\pgfqpoint{3.860616in}{2.927869in}}%
\pgfpathlineto{\pgfqpoint{3.853289in}{2.918160in}}%
\pgfpathlineto{\pgfqpoint{3.845956in}{2.908567in}}%
\pgfpathlineto{\pgfqpoint{3.838619in}{2.899085in}}%
\pgfpathlineto{\pgfqpoint{3.831276in}{2.889712in}}%
\pgfpathlineto{\pgfqpoint{3.818731in}{2.894917in}}%
\pgfpathlineto{\pgfqpoint{3.806190in}{2.900160in}}%
\pgfpathlineto{\pgfqpoint{3.793653in}{2.905440in}}%
\pgfpathlineto{\pgfqpoint{3.781121in}{2.910759in}}%
\pgfpathlineto{\pgfqpoint{3.788473in}{2.920247in}}%
\pgfpathlineto{\pgfqpoint{3.795819in}{2.929847in}}%
\pgfpathlineto{\pgfqpoint{3.803161in}{2.939561in}}%
\pgfpathlineto{\pgfqpoint{3.810498in}{2.949393in}}%
\pgfpathclose%
\pgfusepath{fill}%
\end{pgfscope}%
\begin{pgfscope}%
\pgfpathrectangle{\pgfqpoint{1.254980in}{0.150000in}}{\pgfqpoint{5.490039in}{5.490039in}}%
\pgfusepath{clip}%
\pgfsetbuttcap%
\pgfsetroundjoin%
\definecolor{currentfill}{rgb}{0.282327,0.094955,0.417331}%
\pgfsetfillcolor{currentfill}%
\pgfsetfillopacity{0.700000}%
\pgfsetlinewidth{0.000000pt}%
\definecolor{currentstroke}{rgb}{0.000000,0.000000,0.000000}%
\pgfsetstrokecolor{currentstroke}%
\pgfsetdash{}{0pt}%
\pgfpathmoveto{\pgfqpoint{4.148931in}{2.967707in}}%
\pgfpathlineto{\pgfqpoint{4.161516in}{2.962544in}}%
\pgfpathlineto{\pgfqpoint{4.174106in}{2.957415in}}%
\pgfpathlineto{\pgfqpoint{4.186701in}{2.952318in}}%
\pgfpathlineto{\pgfqpoint{4.199300in}{2.947253in}}%
\pgfpathlineto{\pgfqpoint{4.192071in}{2.936621in}}%
\pgfpathlineto{\pgfqpoint{4.184839in}{2.926143in}}%
\pgfpathlineto{\pgfqpoint{4.177602in}{2.915815in}}%
\pgfpathlineto{\pgfqpoint{4.170362in}{2.905632in}}%
\pgfpathlineto{\pgfqpoint{4.157753in}{2.910540in}}%
\pgfpathlineto{\pgfqpoint{4.145149in}{2.915481in}}%
\pgfpathlineto{\pgfqpoint{4.132550in}{2.920454in}}%
\pgfpathlineto{\pgfqpoint{4.119956in}{2.925461in}}%
\pgfpathlineto{\pgfqpoint{4.127205in}{2.935796in}}%
\pgfpathlineto{\pgfqpoint{4.134451in}{2.946278in}}%
\pgfpathlineto{\pgfqpoint{4.141693in}{2.956914in}}%
\pgfpathlineto{\pgfqpoint{4.148931in}{2.967707in}}%
\pgfpathclose%
\pgfusepath{fill}%
\end{pgfscope}%
\begin{pgfscope}%
\pgfpathrectangle{\pgfqpoint{1.254980in}{0.150000in}}{\pgfqpoint{5.490039in}{5.490039in}}%
\pgfusepath{clip}%
\pgfsetbuttcap%
\pgfsetroundjoin%
\definecolor{currentfill}{rgb}{0.281924,0.089666,0.412415}%
\pgfsetfillcolor{currentfill}%
\pgfsetfillopacity{0.700000}%
\pgfsetlinewidth{0.000000pt}%
\definecolor{currentstroke}{rgb}{0.000000,0.000000,0.000000}%
\pgfsetstrokecolor{currentstroke}%
\pgfsetdash{}{0pt}%
\pgfpathmoveto{\pgfqpoint{3.342388in}{2.962139in}}%
\pgfpathlineto{\pgfqpoint{3.354842in}{2.955889in}}%
\pgfpathlineto{\pgfqpoint{3.367299in}{2.949687in}}%
\pgfpathlineto{\pgfqpoint{3.379760in}{2.943535in}}%
\pgfpathlineto{\pgfqpoint{3.392224in}{2.937430in}}%
\pgfpathlineto{\pgfqpoint{3.384743in}{2.928628in}}%
\pgfpathlineto{\pgfqpoint{3.377255in}{2.919911in}}%
\pgfpathlineto{\pgfqpoint{3.369762in}{2.911277in}}%
\pgfpathlineto{\pgfqpoint{3.362262in}{2.902722in}}%
\pgfpathlineto{\pgfqpoint{3.349788in}{2.908757in}}%
\pgfpathlineto{\pgfqpoint{3.337318in}{2.914840in}}%
\pgfpathlineto{\pgfqpoint{3.324851in}{2.920971in}}%
\pgfpathlineto{\pgfqpoint{3.312387in}{2.927153in}}%
\pgfpathlineto{\pgfqpoint{3.319896in}{2.935772in}}%
\pgfpathlineto{\pgfqpoint{3.327399in}{2.944475in}}%
\pgfpathlineto{\pgfqpoint{3.334897in}{2.953263in}}%
\pgfpathlineto{\pgfqpoint{3.342388in}{2.962139in}}%
\pgfpathclose%
\pgfusepath{fill}%
\end{pgfscope}%
\begin{pgfscope}%
\pgfpathrectangle{\pgfqpoint{1.254980in}{0.150000in}}{\pgfqpoint{5.490039in}{5.490039in}}%
\pgfusepath{clip}%
\pgfsetbuttcap%
\pgfsetroundjoin%
\definecolor{currentfill}{rgb}{0.281446,0.084320,0.407414}%
\pgfsetfillcolor{currentfill}%
\pgfsetfillopacity{0.700000}%
\pgfsetlinewidth{0.000000pt}%
\definecolor{currentstroke}{rgb}{0.000000,0.000000,0.000000}%
\pgfsetstrokecolor{currentstroke}%
\pgfsetdash{}{0pt}%
\pgfpathmoveto{\pgfqpoint{3.471942in}{2.949270in}}%
\pgfpathlineto{\pgfqpoint{3.484415in}{2.943318in}}%
\pgfpathlineto{\pgfqpoint{3.496890in}{2.937412in}}%
\pgfpathlineto{\pgfqpoint{3.509370in}{2.931550in}}%
\pgfpathlineto{\pgfqpoint{3.521853in}{2.925734in}}%
\pgfpathlineto{\pgfqpoint{3.514414in}{2.916739in}}%
\pgfpathlineto{\pgfqpoint{3.506970in}{2.907833in}}%
\pgfpathlineto{\pgfqpoint{3.499520in}{2.899014in}}%
\pgfpathlineto{\pgfqpoint{3.492064in}{2.890279in}}%
\pgfpathlineto{\pgfqpoint{3.479571in}{2.896014in}}%
\pgfpathlineto{\pgfqpoint{3.467082in}{2.901793in}}%
\pgfpathlineto{\pgfqpoint{3.454597in}{2.907617in}}%
\pgfpathlineto{\pgfqpoint{3.442115in}{2.913487in}}%
\pgfpathlineto{\pgfqpoint{3.449581in}{2.922299in}}%
\pgfpathlineto{\pgfqpoint{3.457040in}{2.931198in}}%
\pgfpathlineto{\pgfqpoint{3.464494in}{2.940188in}}%
\pgfpathlineto{\pgfqpoint{3.471942in}{2.949270in}}%
\pgfpathclose%
\pgfusepath{fill}%
\end{pgfscope}%
\begin{pgfscope}%
\pgfpathrectangle{\pgfqpoint{1.254980in}{0.150000in}}{\pgfqpoint{5.490039in}{5.490039in}}%
\pgfusepath{clip}%
\pgfsetbuttcap%
\pgfsetroundjoin%
\definecolor{currentfill}{rgb}{0.281446,0.084320,0.407414}%
\pgfsetfillcolor{currentfill}%
\pgfsetfillopacity{0.700000}%
\pgfsetlinewidth{0.000000pt}%
\definecolor{currentstroke}{rgb}{0.000000,0.000000,0.000000}%
\pgfsetstrokecolor{currentstroke}%
\pgfsetdash{}{0pt}%
\pgfpathmoveto{\pgfqpoint{3.940029in}{2.946486in}}%
\pgfpathlineto{\pgfqpoint{3.952578in}{2.941213in}}%
\pgfpathlineto{\pgfqpoint{3.965132in}{2.935975in}}%
\pgfpathlineto{\pgfqpoint{3.977689in}{2.930772in}}%
\pgfpathlineto{\pgfqpoint{3.990252in}{2.925605in}}%
\pgfpathlineto{\pgfqpoint{3.982961in}{2.915664in}}%
\pgfpathlineto{\pgfqpoint{3.975667in}{2.905850in}}%
\pgfpathlineto{\pgfqpoint{3.968368in}{2.896158in}}%
\pgfpathlineto{\pgfqpoint{3.961064in}{2.886583in}}%
\pgfpathlineto{\pgfqpoint{3.948492in}{2.891619in}}%
\pgfpathlineto{\pgfqpoint{3.935925in}{2.896690in}}%
\pgfpathlineto{\pgfqpoint{3.923362in}{2.901796in}}%
\pgfpathlineto{\pgfqpoint{3.910804in}{2.906939in}}%
\pgfpathlineto{\pgfqpoint{3.918118in}{2.916640in}}%
\pgfpathlineto{\pgfqpoint{3.925426in}{2.926463in}}%
\pgfpathlineto{\pgfqpoint{3.932730in}{2.936410in}}%
\pgfpathlineto{\pgfqpoint{3.940029in}{2.946486in}}%
\pgfpathclose%
\pgfusepath{fill}%
\end{pgfscope}%
\begin{pgfscope}%
\pgfpathrectangle{\pgfqpoint{1.254980in}{0.150000in}}{\pgfqpoint{5.490039in}{5.490039in}}%
\pgfusepath{clip}%
\pgfsetbuttcap%
\pgfsetroundjoin%
\definecolor{currentfill}{rgb}{0.280894,0.078907,0.402329}%
\pgfsetfillcolor{currentfill}%
\pgfsetfillopacity{0.700000}%
\pgfsetlinewidth{0.000000pt}%
\definecolor{currentstroke}{rgb}{0.000000,0.000000,0.000000}%
\pgfsetstrokecolor{currentstroke}%
\pgfsetdash{}{0pt}%
\pgfpathmoveto{\pgfqpoint{3.601483in}{2.939467in}}%
\pgfpathlineto{\pgfqpoint{3.613975in}{2.933773in}}%
\pgfpathlineto{\pgfqpoint{3.626472in}{2.928121in}}%
\pgfpathlineto{\pgfqpoint{3.638973in}{2.922511in}}%
\pgfpathlineto{\pgfqpoint{3.651477in}{2.916942in}}%
\pgfpathlineto{\pgfqpoint{3.644080in}{2.907756in}}%
\pgfpathlineto{\pgfqpoint{3.636677in}{2.898665in}}%
\pgfpathlineto{\pgfqpoint{3.629269in}{2.889667in}}%
\pgfpathlineto{\pgfqpoint{3.621855in}{2.880759in}}%
\pgfpathlineto{\pgfqpoint{3.609341in}{2.886233in}}%
\pgfpathlineto{\pgfqpoint{3.596831in}{2.891748in}}%
\pgfpathlineto{\pgfqpoint{3.584325in}{2.897306in}}%
\pgfpathlineto{\pgfqpoint{3.571823in}{2.902906in}}%
\pgfpathlineto{\pgfqpoint{3.579246in}{2.911904in}}%
\pgfpathlineto{\pgfqpoint{3.586664in}{2.920995in}}%
\pgfpathlineto{\pgfqpoint{3.594076in}{2.930181in}}%
\pgfpathlineto{\pgfqpoint{3.601483in}{2.939467in}}%
\pgfpathclose%
\pgfusepath{fill}%
\end{pgfscope}%
\begin{pgfscope}%
\pgfpathrectangle{\pgfqpoint{1.254980in}{0.150000in}}{\pgfqpoint{5.490039in}{5.490039in}}%
\pgfusepath{clip}%
\pgfsetbuttcap%
\pgfsetroundjoin%
\definecolor{currentfill}{rgb}{0.280894,0.078907,0.402329}%
\pgfsetfillcolor{currentfill}%
\pgfsetfillopacity{0.700000}%
\pgfsetlinewidth{0.000000pt}%
\definecolor{currentstroke}{rgb}{0.000000,0.000000,0.000000}%
\pgfsetstrokecolor{currentstroke}%
\pgfsetdash{}{0pt}%
\pgfpathmoveto{\pgfqpoint{3.731033in}{2.932420in}}%
\pgfpathlineto{\pgfqpoint{3.743549in}{2.926946in}}%
\pgfpathlineto{\pgfqpoint{3.756069in}{2.921511in}}%
\pgfpathlineto{\pgfqpoint{3.768593in}{2.916116in}}%
\pgfpathlineto{\pgfqpoint{3.781121in}{2.910759in}}%
\pgfpathlineto{\pgfqpoint{3.773764in}{2.901378in}}%
\pgfpathlineto{\pgfqpoint{3.766402in}{2.892100in}}%
\pgfpathlineto{\pgfqpoint{3.759034in}{2.882922in}}%
\pgfpathlineto{\pgfqpoint{3.751661in}{2.873841in}}%
\pgfpathlineto{\pgfqpoint{3.739124in}{2.879091in}}%
\pgfpathlineto{\pgfqpoint{3.726590in}{2.884380in}}%
\pgfpathlineto{\pgfqpoint{3.714061in}{2.889707in}}%
\pgfpathlineto{\pgfqpoint{3.701536in}{2.895074in}}%
\pgfpathlineto{\pgfqpoint{3.708918in}{2.904258in}}%
\pgfpathlineto{\pgfqpoint{3.716295in}{2.913541in}}%
\pgfpathlineto{\pgfqpoint{3.723667in}{2.922927in}}%
\pgfpathlineto{\pgfqpoint{3.731033in}{2.932420in}}%
\pgfpathclose%
\pgfusepath{fill}%
\end{pgfscope}%
\begin{pgfscope}%
\pgfpathrectangle{\pgfqpoint{1.254980in}{0.150000in}}{\pgfqpoint{5.490039in}{5.490039in}}%
\pgfusepath{clip}%
\pgfsetbuttcap%
\pgfsetroundjoin%
\definecolor{currentfill}{rgb}{0.281446,0.084320,0.407414}%
\pgfsetfillcolor{currentfill}%
\pgfsetfillopacity{0.700000}%
\pgfsetlinewidth{0.000000pt}%
\definecolor{currentstroke}{rgb}{0.000000,0.000000,0.000000}%
\pgfsetstrokecolor{currentstroke}%
\pgfsetdash{}{0pt}%
\pgfpathmoveto{\pgfqpoint{4.069625in}{2.945817in}}%
\pgfpathlineto{\pgfqpoint{4.082200in}{2.940678in}}%
\pgfpathlineto{\pgfqpoint{4.094781in}{2.935572in}}%
\pgfpathlineto{\pgfqpoint{4.107366in}{2.930500in}}%
\pgfpathlineto{\pgfqpoint{4.119956in}{2.925461in}}%
\pgfpathlineto{\pgfqpoint{4.112702in}{2.915268in}}%
\pgfpathlineto{\pgfqpoint{4.105444in}{2.905214in}}%
\pgfpathlineto{\pgfqpoint{4.098181in}{2.895293in}}%
\pgfpathlineto{\pgfqpoint{4.090915in}{2.885501in}}%
\pgfpathlineto{\pgfqpoint{4.078316in}{2.890396in}}%
\pgfpathlineto{\pgfqpoint{4.065721in}{2.895324in}}%
\pgfpathlineto{\pgfqpoint{4.053131in}{2.900286in}}%
\pgfpathlineto{\pgfqpoint{4.040546in}{2.905282in}}%
\pgfpathlineto{\pgfqpoint{4.047822in}{2.915213in}}%
\pgfpathlineto{\pgfqpoint{4.055094in}{2.925276in}}%
\pgfpathlineto{\pgfqpoint{4.062361in}{2.935476in}}%
\pgfpathlineto{\pgfqpoint{4.069625in}{2.945817in}}%
\pgfpathclose%
\pgfusepath{fill}%
\end{pgfscope}%
\begin{pgfscope}%
\pgfpathrectangle{\pgfqpoint{1.254980in}{0.150000in}}{\pgfqpoint{5.490039in}{5.490039in}}%
\pgfusepath{clip}%
\pgfsetbuttcap%
\pgfsetroundjoin%
\definecolor{currentfill}{rgb}{0.280894,0.078907,0.402329}%
\pgfsetfillcolor{currentfill}%
\pgfsetfillopacity{0.700000}%
\pgfsetlinewidth{0.000000pt}%
\definecolor{currentstroke}{rgb}{0.000000,0.000000,0.000000}%
\pgfsetstrokecolor{currentstroke}%
\pgfsetdash{}{0pt}%
\pgfpathmoveto{\pgfqpoint{3.860616in}{2.927869in}}%
\pgfpathlineto{\pgfqpoint{3.873157in}{2.922582in}}%
\pgfpathlineto{\pgfqpoint{3.885702in}{2.917331in}}%
\pgfpathlineto{\pgfqpoint{3.898251in}{2.912117in}}%
\pgfpathlineto{\pgfqpoint{3.910804in}{2.906939in}}%
\pgfpathlineto{\pgfqpoint{3.903486in}{2.897353in}}%
\pgfpathlineto{\pgfqpoint{3.896163in}{2.887881in}}%
\pgfpathlineto{\pgfqpoint{3.888835in}{2.878517in}}%
\pgfpathlineto{\pgfqpoint{3.881502in}{2.869258in}}%
\pgfpathlineto{\pgfqpoint{3.868939in}{2.874317in}}%
\pgfpathlineto{\pgfqpoint{3.856381in}{2.879412in}}%
\pgfpathlineto{\pgfqpoint{3.843826in}{2.884543in}}%
\pgfpathlineto{\pgfqpoint{3.831276in}{2.889712in}}%
\pgfpathlineto{\pgfqpoint{3.838619in}{2.899085in}}%
\pgfpathlineto{\pgfqpoint{3.845956in}{2.908567in}}%
\pgfpathlineto{\pgfqpoint{3.853289in}{2.918160in}}%
\pgfpathlineto{\pgfqpoint{3.860616in}{2.927869in}}%
\pgfpathclose%
\pgfusepath{fill}%
\end{pgfscope}%
\begin{pgfscope}%
\pgfpathrectangle{\pgfqpoint{1.254980in}{0.150000in}}{\pgfqpoint{5.490039in}{5.490039in}}%
\pgfusepath{clip}%
\pgfsetbuttcap%
\pgfsetroundjoin%
\definecolor{currentfill}{rgb}{0.281446,0.084320,0.407414}%
\pgfsetfillcolor{currentfill}%
\pgfsetfillopacity{0.700000}%
\pgfsetlinewidth{0.000000pt}%
\definecolor{currentstroke}{rgb}{0.000000,0.000000,0.000000}%
\pgfsetstrokecolor{currentstroke}%
\pgfsetdash{}{0pt}%
\pgfpathmoveto{\pgfqpoint{3.392224in}{2.937430in}}%
\pgfpathlineto{\pgfqpoint{3.404691in}{2.931374in}}%
\pgfpathlineto{\pgfqpoint{3.417162in}{2.925365in}}%
\pgfpathlineto{\pgfqpoint{3.429637in}{2.919403in}}%
\pgfpathlineto{\pgfqpoint{3.442115in}{2.913487in}}%
\pgfpathlineto{\pgfqpoint{3.434643in}{2.904759in}}%
\pgfpathlineto{\pgfqpoint{3.427166in}{2.896113in}}%
\pgfpathlineto{\pgfqpoint{3.419683in}{2.887547in}}%
\pgfpathlineto{\pgfqpoint{3.412193in}{2.879057in}}%
\pgfpathlineto{\pgfqpoint{3.399705in}{2.884903in}}%
\pgfpathlineto{\pgfqpoint{3.387221in}{2.890796in}}%
\pgfpathlineto{\pgfqpoint{3.374740in}{2.896735in}}%
\pgfpathlineto{\pgfqpoint{3.362262in}{2.902722in}}%
\pgfpathlineto{\pgfqpoint{3.369762in}{2.911277in}}%
\pgfpathlineto{\pgfqpoint{3.377255in}{2.919911in}}%
\pgfpathlineto{\pgfqpoint{3.384743in}{2.928628in}}%
\pgfpathlineto{\pgfqpoint{3.392224in}{2.937430in}}%
\pgfpathclose%
\pgfusepath{fill}%
\end{pgfscope}%
\begin{pgfscope}%
\pgfpathrectangle{\pgfqpoint{1.254980in}{0.150000in}}{\pgfqpoint{5.490039in}{5.490039in}}%
\pgfusepath{clip}%
\pgfsetbuttcap%
\pgfsetroundjoin%
\definecolor{currentfill}{rgb}{0.281924,0.089666,0.412415}%
\pgfsetfillcolor{currentfill}%
\pgfsetfillopacity{0.700000}%
\pgfsetlinewidth{0.000000pt}%
\definecolor{currentstroke}{rgb}{0.000000,0.000000,0.000000}%
\pgfsetstrokecolor{currentstroke}%
\pgfsetdash{}{0pt}%
\pgfpathmoveto{\pgfqpoint{4.199300in}{2.947253in}}%
\pgfpathlineto{\pgfqpoint{4.211904in}{2.942220in}}%
\pgfpathlineto{\pgfqpoint{4.224513in}{2.937220in}}%
\pgfpathlineto{\pgfqpoint{4.237126in}{2.932251in}}%
\pgfpathlineto{\pgfqpoint{4.249745in}{2.927313in}}%
\pgfpathlineto{\pgfqpoint{4.242525in}{2.916842in}}%
\pgfpathlineto{\pgfqpoint{4.235302in}{2.906522in}}%
\pgfpathlineto{\pgfqpoint{4.228075in}{2.896349in}}%
\pgfpathlineto{\pgfqpoint{4.220845in}{2.886318in}}%
\pgfpathlineto{\pgfqpoint{4.208217in}{2.891099in}}%
\pgfpathlineto{\pgfqpoint{4.195594in}{2.895911in}}%
\pgfpathlineto{\pgfqpoint{4.182975in}{2.900756in}}%
\pgfpathlineto{\pgfqpoint{4.170362in}{2.905632in}}%
\pgfpathlineto{\pgfqpoint{4.177602in}{2.915815in}}%
\pgfpathlineto{\pgfqpoint{4.184839in}{2.926143in}}%
\pgfpathlineto{\pgfqpoint{4.192071in}{2.936621in}}%
\pgfpathlineto{\pgfqpoint{4.199300in}{2.947253in}}%
\pgfpathclose%
\pgfusepath{fill}%
\end{pgfscope}%
\begin{pgfscope}%
\pgfpathrectangle{\pgfqpoint{1.254980in}{0.150000in}}{\pgfqpoint{5.490039in}{5.490039in}}%
\pgfusepath{clip}%
\pgfsetbuttcap%
\pgfsetroundjoin%
\definecolor{currentfill}{rgb}{0.280894,0.078907,0.402329}%
\pgfsetfillcolor{currentfill}%
\pgfsetfillopacity{0.700000}%
\pgfsetlinewidth{0.000000pt}%
\definecolor{currentstroke}{rgb}{0.000000,0.000000,0.000000}%
\pgfsetstrokecolor{currentstroke}%
\pgfsetdash{}{0pt}%
\pgfpathmoveto{\pgfqpoint{3.521853in}{2.925734in}}%
\pgfpathlineto{\pgfqpoint{3.534339in}{2.919962in}}%
\pgfpathlineto{\pgfqpoint{3.546830in}{2.914233in}}%
\pgfpathlineto{\pgfqpoint{3.559325in}{2.908548in}}%
\pgfpathlineto{\pgfqpoint{3.571823in}{2.902906in}}%
\pgfpathlineto{\pgfqpoint{3.564394in}{2.893997in}}%
\pgfpathlineto{\pgfqpoint{3.556959in}{2.885175in}}%
\pgfpathlineto{\pgfqpoint{3.549519in}{2.876437in}}%
\pgfpathlineto{\pgfqpoint{3.542073in}{2.867780in}}%
\pgfpathlineto{\pgfqpoint{3.529565in}{2.873340in}}%
\pgfpathlineto{\pgfqpoint{3.517061in}{2.878943in}}%
\pgfpathlineto{\pgfqpoint{3.504560in}{2.884589in}}%
\pgfpathlineto{\pgfqpoint{3.492064in}{2.890279in}}%
\pgfpathlineto{\pgfqpoint{3.499520in}{2.899014in}}%
\pgfpathlineto{\pgfqpoint{3.506970in}{2.907833in}}%
\pgfpathlineto{\pgfqpoint{3.514414in}{2.916739in}}%
\pgfpathlineto{\pgfqpoint{3.521853in}{2.925734in}}%
\pgfpathclose%
\pgfusepath{fill}%
\end{pgfscope}%
\begin{pgfscope}%
\pgfpathrectangle{\pgfqpoint{1.254980in}{0.150000in}}{\pgfqpoint{5.490039in}{5.490039in}}%
\pgfusepath{clip}%
\pgfsetbuttcap%
\pgfsetroundjoin%
\definecolor{currentfill}{rgb}{0.280894,0.078907,0.402329}%
\pgfsetfillcolor{currentfill}%
\pgfsetfillopacity{0.700000}%
\pgfsetlinewidth{0.000000pt}%
\definecolor{currentstroke}{rgb}{0.000000,0.000000,0.000000}%
\pgfsetstrokecolor{currentstroke}%
\pgfsetdash{}{0pt}%
\pgfpathmoveto{\pgfqpoint{3.990252in}{2.925605in}}%
\pgfpathlineto{\pgfqpoint{4.002818in}{2.920472in}}%
\pgfpathlineto{\pgfqpoint{4.015390in}{2.915374in}}%
\pgfpathlineto{\pgfqpoint{4.027965in}{2.910311in}}%
\pgfpathlineto{\pgfqpoint{4.040546in}{2.905282in}}%
\pgfpathlineto{\pgfqpoint{4.033265in}{2.895478in}}%
\pgfpathlineto{\pgfqpoint{4.025980in}{2.885797in}}%
\pgfpathlineto{\pgfqpoint{4.018691in}{2.876235in}}%
\pgfpathlineto{\pgfqpoint{4.011397in}{2.866787in}}%
\pgfpathlineto{\pgfqpoint{3.998806in}{2.871685in}}%
\pgfpathlineto{\pgfqpoint{3.986221in}{2.876617in}}%
\pgfpathlineto{\pgfqpoint{3.973640in}{2.881583in}}%
\pgfpathlineto{\pgfqpoint{3.961064in}{2.886583in}}%
\pgfpathlineto{\pgfqpoint{3.968368in}{2.896158in}}%
\pgfpathlineto{\pgfqpoint{3.975667in}{2.905850in}}%
\pgfpathlineto{\pgfqpoint{3.982961in}{2.915664in}}%
\pgfpathlineto{\pgfqpoint{3.990252in}{2.925605in}}%
\pgfpathclose%
\pgfusepath{fill}%
\end{pgfscope}%
\begin{pgfscope}%
\pgfpathrectangle{\pgfqpoint{1.254980in}{0.150000in}}{\pgfqpoint{5.490039in}{5.490039in}}%
\pgfusepath{clip}%
\pgfsetbuttcap%
\pgfsetroundjoin%
\definecolor{currentfill}{rgb}{0.280267,0.073417,0.397163}%
\pgfsetfillcolor{currentfill}%
\pgfsetfillopacity{0.700000}%
\pgfsetlinewidth{0.000000pt}%
\definecolor{currentstroke}{rgb}{0.000000,0.000000,0.000000}%
\pgfsetstrokecolor{currentstroke}%
\pgfsetdash{}{0pt}%
\pgfpathmoveto{\pgfqpoint{3.651477in}{2.916942in}}%
\pgfpathlineto{\pgfqpoint{3.663986in}{2.911414in}}%
\pgfpathlineto{\pgfqpoint{3.676498in}{2.905928in}}%
\pgfpathlineto{\pgfqpoint{3.689015in}{2.900481in}}%
\pgfpathlineto{\pgfqpoint{3.701536in}{2.895074in}}%
\pgfpathlineto{\pgfqpoint{3.694148in}{2.885987in}}%
\pgfpathlineto{\pgfqpoint{3.686755in}{2.876993in}}%
\pgfpathlineto{\pgfqpoint{3.679357in}{2.868088in}}%
\pgfpathlineto{\pgfqpoint{3.671953in}{2.859270in}}%
\pgfpathlineto{\pgfqpoint{3.659422in}{2.864581in}}%
\pgfpathlineto{\pgfqpoint{3.646896in}{2.869933in}}%
\pgfpathlineto{\pgfqpoint{3.634374in}{2.875326in}}%
\pgfpathlineto{\pgfqpoint{3.621855in}{2.880759in}}%
\pgfpathlineto{\pgfqpoint{3.629269in}{2.889667in}}%
\pgfpathlineto{\pgfqpoint{3.636677in}{2.898665in}}%
\pgfpathlineto{\pgfqpoint{3.644080in}{2.907756in}}%
\pgfpathlineto{\pgfqpoint{3.651477in}{2.916942in}}%
\pgfpathclose%
\pgfusepath{fill}%
\end{pgfscope}%
\begin{pgfscope}%
\pgfpathrectangle{\pgfqpoint{1.254980in}{0.150000in}}{\pgfqpoint{5.490039in}{5.490039in}}%
\pgfusepath{clip}%
\pgfsetbuttcap%
\pgfsetroundjoin%
\definecolor{currentfill}{rgb}{0.280267,0.073417,0.397163}%
\pgfsetfillcolor{currentfill}%
\pgfsetfillopacity{0.700000}%
\pgfsetlinewidth{0.000000pt}%
\definecolor{currentstroke}{rgb}{0.000000,0.000000,0.000000}%
\pgfsetstrokecolor{currentstroke}%
\pgfsetdash{}{0pt}%
\pgfpathmoveto{\pgfqpoint{3.781121in}{2.910759in}}%
\pgfpathlineto{\pgfqpoint{3.793653in}{2.905440in}}%
\pgfpathlineto{\pgfqpoint{3.806190in}{2.900160in}}%
\pgfpathlineto{\pgfqpoint{3.818731in}{2.894917in}}%
\pgfpathlineto{\pgfqpoint{3.831276in}{2.889712in}}%
\pgfpathlineto{\pgfqpoint{3.823929in}{2.880442in}}%
\pgfpathlineto{\pgfqpoint{3.816576in}{2.871273in}}%
\pgfpathlineto{\pgfqpoint{3.809219in}{2.862201in}}%
\pgfpathlineto{\pgfqpoint{3.801856in}{2.853222in}}%
\pgfpathlineto{\pgfqpoint{3.789300in}{2.858321in}}%
\pgfpathlineto{\pgfqpoint{3.776750in}{2.863456in}}%
\pgfpathlineto{\pgfqpoint{3.764203in}{2.868630in}}%
\pgfpathlineto{\pgfqpoint{3.751661in}{2.873841in}}%
\pgfpathlineto{\pgfqpoint{3.759034in}{2.882922in}}%
\pgfpathlineto{\pgfqpoint{3.766402in}{2.892100in}}%
\pgfpathlineto{\pgfqpoint{3.773764in}{2.901378in}}%
\pgfpathlineto{\pgfqpoint{3.781121in}{2.910759in}}%
\pgfpathclose%
\pgfusepath{fill}%
\end{pgfscope}%
\begin{pgfscope}%
\pgfpathrectangle{\pgfqpoint{1.254980in}{0.150000in}}{\pgfqpoint{5.490039in}{5.490039in}}%
\pgfusepath{clip}%
\pgfsetbuttcap%
\pgfsetroundjoin%
\definecolor{currentfill}{rgb}{0.280894,0.078907,0.402329}%
\pgfsetfillcolor{currentfill}%
\pgfsetfillopacity{0.700000}%
\pgfsetlinewidth{0.000000pt}%
\definecolor{currentstroke}{rgb}{0.000000,0.000000,0.000000}%
\pgfsetstrokecolor{currentstroke}%
\pgfsetdash{}{0pt}%
\pgfpathmoveto{\pgfqpoint{4.119956in}{2.925461in}}%
\pgfpathlineto{\pgfqpoint{4.132550in}{2.920454in}}%
\pgfpathlineto{\pgfqpoint{4.145149in}{2.915481in}}%
\pgfpathlineto{\pgfqpoint{4.157753in}{2.910540in}}%
\pgfpathlineto{\pgfqpoint{4.170362in}{2.905632in}}%
\pgfpathlineto{\pgfqpoint{4.163117in}{2.895588in}}%
\pgfpathlineto{\pgfqpoint{4.155869in}{2.885680in}}%
\pgfpathlineto{\pgfqpoint{4.148617in}{2.875901in}}%
\pgfpathlineto{\pgfqpoint{4.141360in}{2.866249in}}%
\pgfpathlineto{\pgfqpoint{4.128741in}{2.871013in}}%
\pgfpathlineto{\pgfqpoint{4.116128in}{2.875810in}}%
\pgfpathlineto{\pgfqpoint{4.103519in}{2.880639in}}%
\pgfpathlineto{\pgfqpoint{4.090915in}{2.885501in}}%
\pgfpathlineto{\pgfqpoint{4.098181in}{2.895293in}}%
\pgfpathlineto{\pgfqpoint{4.105444in}{2.905214in}}%
\pgfpathlineto{\pgfqpoint{4.112702in}{2.915268in}}%
\pgfpathlineto{\pgfqpoint{4.119956in}{2.925461in}}%
\pgfpathclose%
\pgfusepath{fill}%
\end{pgfscope}%
\begin{pgfscope}%
\pgfpathrectangle{\pgfqpoint{1.254980in}{0.150000in}}{\pgfqpoint{5.490039in}{5.490039in}}%
\pgfusepath{clip}%
\pgfsetbuttcap%
\pgfsetroundjoin%
\definecolor{currentfill}{rgb}{0.280267,0.073417,0.397163}%
\pgfsetfillcolor{currentfill}%
\pgfsetfillopacity{0.700000}%
\pgfsetlinewidth{0.000000pt}%
\definecolor{currentstroke}{rgb}{0.000000,0.000000,0.000000}%
\pgfsetstrokecolor{currentstroke}%
\pgfsetdash{}{0pt}%
\pgfpathmoveto{\pgfqpoint{3.910804in}{2.906939in}}%
\pgfpathlineto{\pgfqpoint{3.923362in}{2.901796in}}%
\pgfpathlineto{\pgfqpoint{3.935925in}{2.896690in}}%
\pgfpathlineto{\pgfqpoint{3.948492in}{2.891619in}}%
\pgfpathlineto{\pgfqpoint{3.961064in}{2.886583in}}%
\pgfpathlineto{\pgfqpoint{3.953755in}{2.877122in}}%
\pgfpathlineto{\pgfqpoint{3.946442in}{2.867771in}}%
\pgfpathlineto{\pgfqpoint{3.939124in}{2.858525in}}%
\pgfpathlineto{\pgfqpoint{3.931801in}{2.849381in}}%
\pgfpathlineto{\pgfqpoint{3.919219in}{2.854297in}}%
\pgfpathlineto{\pgfqpoint{3.906642in}{2.859249in}}%
\pgfpathlineto{\pgfqpoint{3.894070in}{2.864236in}}%
\pgfpathlineto{\pgfqpoint{3.881502in}{2.869258in}}%
\pgfpathlineto{\pgfqpoint{3.888835in}{2.878517in}}%
\pgfpathlineto{\pgfqpoint{3.896163in}{2.887881in}}%
\pgfpathlineto{\pgfqpoint{3.903486in}{2.897353in}}%
\pgfpathlineto{\pgfqpoint{3.910804in}{2.906939in}}%
\pgfpathclose%
\pgfusepath{fill}%
\end{pgfscope}%
\begin{pgfscope}%
\pgfpathrectangle{\pgfqpoint{1.254980in}{0.150000in}}{\pgfqpoint{5.490039in}{5.490039in}}%
\pgfusepath{clip}%
\pgfsetbuttcap%
\pgfsetroundjoin%
\definecolor{currentfill}{rgb}{0.281446,0.084320,0.407414}%
\pgfsetfillcolor{currentfill}%
\pgfsetfillopacity{0.700000}%
\pgfsetlinewidth{0.000000pt}%
\definecolor{currentstroke}{rgb}{0.000000,0.000000,0.000000}%
\pgfsetstrokecolor{currentstroke}%
\pgfsetdash{}{0pt}%
\pgfpathmoveto{\pgfqpoint{3.312387in}{2.927153in}}%
\pgfpathlineto{\pgfqpoint{3.324851in}{2.920971in}}%
\pgfpathlineto{\pgfqpoint{3.337318in}{2.914840in}}%
\pgfpathlineto{\pgfqpoint{3.349788in}{2.908757in}}%
\pgfpathlineto{\pgfqpoint{3.362262in}{2.902722in}}%
\pgfpathlineto{\pgfqpoint{3.354757in}{2.894245in}}%
\pgfpathlineto{\pgfqpoint{3.347245in}{2.885843in}}%
\pgfpathlineto{\pgfqpoint{3.339728in}{2.877515in}}%
\pgfpathlineto{\pgfqpoint{3.332204in}{2.869257in}}%
\pgfpathlineto{\pgfqpoint{3.319720in}{2.875235in}}%
\pgfpathlineto{\pgfqpoint{3.307239in}{2.881260in}}%
\pgfpathlineto{\pgfqpoint{3.294762in}{2.887335in}}%
\pgfpathlineto{\pgfqpoint{3.282287in}{2.893459in}}%
\pgfpathlineto{\pgfqpoint{3.289821in}{2.901769in}}%
\pgfpathlineto{\pgfqpoint{3.297349in}{2.910153in}}%
\pgfpathlineto{\pgfqpoint{3.304871in}{2.918614in}}%
\pgfpathlineto{\pgfqpoint{3.312387in}{2.927153in}}%
\pgfpathclose%
\pgfusepath{fill}%
\end{pgfscope}%
\begin{pgfscope}%
\pgfpathrectangle{\pgfqpoint{1.254980in}{0.150000in}}{\pgfqpoint{5.490039in}{5.490039in}}%
\pgfusepath{clip}%
\pgfsetbuttcap%
\pgfsetroundjoin%
\definecolor{currentfill}{rgb}{0.280894,0.078907,0.402329}%
\pgfsetfillcolor{currentfill}%
\pgfsetfillopacity{0.700000}%
\pgfsetlinewidth{0.000000pt}%
\definecolor{currentstroke}{rgb}{0.000000,0.000000,0.000000}%
\pgfsetstrokecolor{currentstroke}%
\pgfsetdash{}{0pt}%
\pgfpathmoveto{\pgfqpoint{3.442115in}{2.913487in}}%
\pgfpathlineto{\pgfqpoint{3.454597in}{2.907617in}}%
\pgfpathlineto{\pgfqpoint{3.467082in}{2.901793in}}%
\pgfpathlineto{\pgfqpoint{3.479571in}{2.896014in}}%
\pgfpathlineto{\pgfqpoint{3.492064in}{2.890279in}}%
\pgfpathlineto{\pgfqpoint{3.484602in}{2.881626in}}%
\pgfpathlineto{\pgfqpoint{3.477135in}{2.873051in}}%
\pgfpathlineto{\pgfqpoint{3.469662in}{2.864553in}}%
\pgfpathlineto{\pgfqpoint{3.462183in}{2.856129in}}%
\pgfpathlineto{\pgfqpoint{3.449680in}{2.861793in}}%
\pgfpathlineto{\pgfqpoint{3.437181in}{2.867503in}}%
\pgfpathlineto{\pgfqpoint{3.424685in}{2.873257in}}%
\pgfpathlineto{\pgfqpoint{3.412193in}{2.879057in}}%
\pgfpathlineto{\pgfqpoint{3.419683in}{2.887547in}}%
\pgfpathlineto{\pgfqpoint{3.427166in}{2.896113in}}%
\pgfpathlineto{\pgfqpoint{3.434643in}{2.904759in}}%
\pgfpathlineto{\pgfqpoint{3.442115in}{2.913487in}}%
\pgfpathclose%
\pgfusepath{fill}%
\end{pgfscope}%
\begin{pgfscope}%
\pgfpathrectangle{\pgfqpoint{1.254980in}{0.150000in}}{\pgfqpoint{5.490039in}{5.490039in}}%
\pgfusepath{clip}%
\pgfsetbuttcap%
\pgfsetroundjoin%
\definecolor{currentfill}{rgb}{0.280267,0.073417,0.397163}%
\pgfsetfillcolor{currentfill}%
\pgfsetfillopacity{0.700000}%
\pgfsetlinewidth{0.000000pt}%
\definecolor{currentstroke}{rgb}{0.000000,0.000000,0.000000}%
\pgfsetstrokecolor{currentstroke}%
\pgfsetdash{}{0pt}%
\pgfpathmoveto{\pgfqpoint{3.571823in}{2.902906in}}%
\pgfpathlineto{\pgfqpoint{3.584325in}{2.897306in}}%
\pgfpathlineto{\pgfqpoint{3.596831in}{2.891748in}}%
\pgfpathlineto{\pgfqpoint{3.609341in}{2.886233in}}%
\pgfpathlineto{\pgfqpoint{3.621855in}{2.880759in}}%
\pgfpathlineto{\pgfqpoint{3.614436in}{2.871937in}}%
\pgfpathlineto{\pgfqpoint{3.607012in}{2.863199in}}%
\pgfpathlineto{\pgfqpoint{3.599581in}{2.854541in}}%
\pgfpathlineto{\pgfqpoint{3.592145in}{2.845962in}}%
\pgfpathlineto{\pgfqpoint{3.579621in}{2.851354in}}%
\pgfpathlineto{\pgfqpoint{3.567101in}{2.856787in}}%
\pgfpathlineto{\pgfqpoint{3.554585in}{2.862262in}}%
\pgfpathlineto{\pgfqpoint{3.542073in}{2.867780in}}%
\pgfpathlineto{\pgfqpoint{3.549519in}{2.876437in}}%
\pgfpathlineto{\pgfqpoint{3.556959in}{2.885175in}}%
\pgfpathlineto{\pgfqpoint{3.564394in}{2.893997in}}%
\pgfpathlineto{\pgfqpoint{3.571823in}{2.902906in}}%
\pgfpathclose%
\pgfusepath{fill}%
\end{pgfscope}%
\begin{pgfscope}%
\pgfpathrectangle{\pgfqpoint{1.254980in}{0.150000in}}{\pgfqpoint{5.490039in}{5.490039in}}%
\pgfusepath{clip}%
\pgfsetbuttcap%
\pgfsetroundjoin%
\definecolor{currentfill}{rgb}{0.281446,0.084320,0.407414}%
\pgfsetfillcolor{currentfill}%
\pgfsetfillopacity{0.700000}%
\pgfsetlinewidth{0.000000pt}%
\definecolor{currentstroke}{rgb}{0.000000,0.000000,0.000000}%
\pgfsetstrokecolor{currentstroke}%
\pgfsetdash{}{0pt}%
\pgfpathmoveto{\pgfqpoint{4.249745in}{2.927313in}}%
\pgfpathlineto{\pgfqpoint{4.262368in}{2.922407in}}%
\pgfpathlineto{\pgfqpoint{4.274996in}{2.917532in}}%
\pgfpathlineto{\pgfqpoint{4.287629in}{2.912688in}}%
\pgfpathlineto{\pgfqpoint{4.300266in}{2.907874in}}%
\pgfpathlineto{\pgfqpoint{4.293057in}{2.897564in}}%
\pgfpathlineto{\pgfqpoint{4.285844in}{2.887403in}}%
\pgfpathlineto{\pgfqpoint{4.278627in}{2.877385in}}%
\pgfpathlineto{\pgfqpoint{4.271406in}{2.867505in}}%
\pgfpathlineto{\pgfqpoint{4.258758in}{2.872162in}}%
\pgfpathlineto{\pgfqpoint{4.246116in}{2.876849in}}%
\pgfpathlineto{\pgfqpoint{4.233478in}{2.881568in}}%
\pgfpathlineto{\pgfqpoint{4.220845in}{2.886318in}}%
\pgfpathlineto{\pgfqpoint{4.228075in}{2.896349in}}%
\pgfpathlineto{\pgfqpoint{4.235302in}{2.906522in}}%
\pgfpathlineto{\pgfqpoint{4.242525in}{2.916842in}}%
\pgfpathlineto{\pgfqpoint{4.249745in}{2.927313in}}%
\pgfpathclose%
\pgfusepath{fill}%
\end{pgfscope}%
\begin{pgfscope}%
\pgfpathrectangle{\pgfqpoint{1.254980in}{0.150000in}}{\pgfqpoint{5.490039in}{5.490039in}}%
\pgfusepath{clip}%
\pgfsetbuttcap%
\pgfsetroundjoin%
\definecolor{currentfill}{rgb}{0.280267,0.073417,0.397163}%
\pgfsetfillcolor{currentfill}%
\pgfsetfillopacity{0.700000}%
\pgfsetlinewidth{0.000000pt}%
\definecolor{currentstroke}{rgb}{0.000000,0.000000,0.000000}%
\pgfsetstrokecolor{currentstroke}%
\pgfsetdash{}{0pt}%
\pgfpathmoveto{\pgfqpoint{4.040546in}{2.905282in}}%
\pgfpathlineto{\pgfqpoint{4.053131in}{2.900286in}}%
\pgfpathlineto{\pgfqpoint{4.065721in}{2.895324in}}%
\pgfpathlineto{\pgfqpoint{4.078316in}{2.890396in}}%
\pgfpathlineto{\pgfqpoint{4.090915in}{2.885501in}}%
\pgfpathlineto{\pgfqpoint{4.083644in}{2.875834in}}%
\pgfpathlineto{\pgfqpoint{4.076369in}{2.866286in}}%
\pgfpathlineto{\pgfqpoint{4.069089in}{2.856854in}}%
\pgfpathlineto{\pgfqpoint{4.061805in}{2.847534in}}%
\pgfpathlineto{\pgfqpoint{4.049196in}{2.852297in}}%
\pgfpathlineto{\pgfqpoint{4.036591in}{2.857094in}}%
\pgfpathlineto{\pgfqpoint{4.023991in}{2.861924in}}%
\pgfpathlineto{\pgfqpoint{4.011397in}{2.866787in}}%
\pgfpathlineto{\pgfqpoint{4.018691in}{2.876235in}}%
\pgfpathlineto{\pgfqpoint{4.025980in}{2.885797in}}%
\pgfpathlineto{\pgfqpoint{4.033265in}{2.895478in}}%
\pgfpathlineto{\pgfqpoint{4.040546in}{2.905282in}}%
\pgfpathclose%
\pgfusepath{fill}%
\end{pgfscope}%
\begin{pgfscope}%
\pgfpathrectangle{\pgfqpoint{1.254980in}{0.150000in}}{\pgfqpoint{5.490039in}{5.490039in}}%
\pgfusepath{clip}%
\pgfsetbuttcap%
\pgfsetroundjoin%
\definecolor{currentfill}{rgb}{0.279566,0.067836,0.391917}%
\pgfsetfillcolor{currentfill}%
\pgfsetfillopacity{0.700000}%
\pgfsetlinewidth{0.000000pt}%
\definecolor{currentstroke}{rgb}{0.000000,0.000000,0.000000}%
\pgfsetstrokecolor{currentstroke}%
\pgfsetdash{}{0pt}%
\pgfpathmoveto{\pgfqpoint{3.701536in}{2.895074in}}%
\pgfpathlineto{\pgfqpoint{3.714061in}{2.889707in}}%
\pgfpathlineto{\pgfqpoint{3.726590in}{2.884380in}}%
\pgfpathlineto{\pgfqpoint{3.739124in}{2.879091in}}%
\pgfpathlineto{\pgfqpoint{3.751661in}{2.873841in}}%
\pgfpathlineto{\pgfqpoint{3.744283in}{2.864853in}}%
\pgfpathlineto{\pgfqpoint{3.736900in}{2.855955in}}%
\pgfpathlineto{\pgfqpoint{3.729512in}{2.847144in}}%
\pgfpathlineto{\pgfqpoint{3.722118in}{2.838415in}}%
\pgfpathlineto{\pgfqpoint{3.709570in}{2.843571in}}%
\pgfpathlineto{\pgfqpoint{3.697027in}{2.848764in}}%
\pgfpathlineto{\pgfqpoint{3.684488in}{2.853997in}}%
\pgfpathlineto{\pgfqpoint{3.671953in}{2.859270in}}%
\pgfpathlineto{\pgfqpoint{3.679357in}{2.868088in}}%
\pgfpathlineto{\pgfqpoint{3.686755in}{2.876993in}}%
\pgfpathlineto{\pgfqpoint{3.694148in}{2.885987in}}%
\pgfpathlineto{\pgfqpoint{3.701536in}{2.895074in}}%
\pgfpathclose%
\pgfusepath{fill}%
\end{pgfscope}%
\begin{pgfscope}%
\pgfpathrectangle{\pgfqpoint{1.254980in}{0.150000in}}{\pgfqpoint{5.490039in}{5.490039in}}%
\pgfusepath{clip}%
\pgfsetbuttcap%
\pgfsetroundjoin%
\definecolor{currentfill}{rgb}{0.279566,0.067836,0.391917}%
\pgfsetfillcolor{currentfill}%
\pgfsetfillopacity{0.700000}%
\pgfsetlinewidth{0.000000pt}%
\definecolor{currentstroke}{rgb}{0.000000,0.000000,0.000000}%
\pgfsetstrokecolor{currentstroke}%
\pgfsetdash{}{0pt}%
\pgfpathmoveto{\pgfqpoint{3.831276in}{2.889712in}}%
\pgfpathlineto{\pgfqpoint{3.843826in}{2.884543in}}%
\pgfpathlineto{\pgfqpoint{3.856381in}{2.879412in}}%
\pgfpathlineto{\pgfqpoint{3.868939in}{2.874317in}}%
\pgfpathlineto{\pgfqpoint{3.881502in}{2.869258in}}%
\pgfpathlineto{\pgfqpoint{3.874165in}{2.860101in}}%
\pgfpathlineto{\pgfqpoint{3.866822in}{2.851040in}}%
\pgfpathlineto{\pgfqpoint{3.859474in}{2.842074in}}%
\pgfpathlineto{\pgfqpoint{3.852121in}{2.833198in}}%
\pgfpathlineto{\pgfqpoint{3.839548in}{2.838149in}}%
\pgfpathlineto{\pgfqpoint{3.826979in}{2.843137in}}%
\pgfpathlineto{\pgfqpoint{3.814415in}{2.848161in}}%
\pgfpathlineto{\pgfqpoint{3.801856in}{2.853222in}}%
\pgfpathlineto{\pgfqpoint{3.809219in}{2.862201in}}%
\pgfpathlineto{\pgfqpoint{3.816576in}{2.871273in}}%
\pgfpathlineto{\pgfqpoint{3.823929in}{2.880442in}}%
\pgfpathlineto{\pgfqpoint{3.831276in}{2.889712in}}%
\pgfpathclose%
\pgfusepath{fill}%
\end{pgfscope}%
\begin{pgfscope}%
\pgfpathrectangle{\pgfqpoint{1.254980in}{0.150000in}}{\pgfqpoint{5.490039in}{5.490039in}}%
\pgfusepath{clip}%
\pgfsetbuttcap%
\pgfsetroundjoin%
\definecolor{currentfill}{rgb}{0.280894,0.078907,0.402329}%
\pgfsetfillcolor{currentfill}%
\pgfsetfillopacity{0.700000}%
\pgfsetlinewidth{0.000000pt}%
\definecolor{currentstroke}{rgb}{0.000000,0.000000,0.000000}%
\pgfsetstrokecolor{currentstroke}%
\pgfsetdash{}{0pt}%
\pgfpathmoveto{\pgfqpoint{4.170362in}{2.905632in}}%
\pgfpathlineto{\pgfqpoint{4.182975in}{2.900756in}}%
\pgfpathlineto{\pgfqpoint{4.195594in}{2.895911in}}%
\pgfpathlineto{\pgfqpoint{4.208217in}{2.891099in}}%
\pgfpathlineto{\pgfqpoint{4.220845in}{2.886318in}}%
\pgfpathlineto{\pgfqpoint{4.213610in}{2.876423in}}%
\pgfpathlineto{\pgfqpoint{4.206372in}{2.866660in}}%
\pgfpathlineto{\pgfqpoint{4.199129in}{2.857024in}}%
\pgfpathlineto{\pgfqpoint{4.191883in}{2.847512in}}%
\pgfpathlineto{\pgfqpoint{4.179245in}{2.852149in}}%
\pgfpathlineto{\pgfqpoint{4.166611in}{2.856817in}}%
\pgfpathlineto{\pgfqpoint{4.153983in}{2.861517in}}%
\pgfpathlineto{\pgfqpoint{4.141360in}{2.866249in}}%
\pgfpathlineto{\pgfqpoint{4.148617in}{2.875901in}}%
\pgfpathlineto{\pgfqpoint{4.155869in}{2.885680in}}%
\pgfpathlineto{\pgfqpoint{4.163117in}{2.895588in}}%
\pgfpathlineto{\pgfqpoint{4.170362in}{2.905632in}}%
\pgfpathclose%
\pgfusepath{fill}%
\end{pgfscope}%
\begin{pgfscope}%
\pgfpathrectangle{\pgfqpoint{1.254980in}{0.150000in}}{\pgfqpoint{5.490039in}{5.490039in}}%
\pgfusepath{clip}%
\pgfsetbuttcap%
\pgfsetroundjoin%
\definecolor{currentfill}{rgb}{0.280894,0.078907,0.402329}%
\pgfsetfillcolor{currentfill}%
\pgfsetfillopacity{0.700000}%
\pgfsetlinewidth{0.000000pt}%
\definecolor{currentstroke}{rgb}{0.000000,0.000000,0.000000}%
\pgfsetstrokecolor{currentstroke}%
\pgfsetdash{}{0pt}%
\pgfpathmoveto{\pgfqpoint{3.362262in}{2.902722in}}%
\pgfpathlineto{\pgfqpoint{3.374740in}{2.896735in}}%
\pgfpathlineto{\pgfqpoint{3.387221in}{2.890796in}}%
\pgfpathlineto{\pgfqpoint{3.399705in}{2.884903in}}%
\pgfpathlineto{\pgfqpoint{3.412193in}{2.879057in}}%
\pgfpathlineto{\pgfqpoint{3.404698in}{2.870642in}}%
\pgfpathlineto{\pgfqpoint{3.397197in}{2.862299in}}%
\pgfpathlineto{\pgfqpoint{3.389690in}{2.854027in}}%
\pgfpathlineto{\pgfqpoint{3.382177in}{2.845822in}}%
\pgfpathlineto{\pgfqpoint{3.369678in}{2.851611in}}%
\pgfpathlineto{\pgfqpoint{3.357184in}{2.857446in}}%
\pgfpathlineto{\pgfqpoint{3.344692in}{2.863328in}}%
\pgfpathlineto{\pgfqpoint{3.332204in}{2.869257in}}%
\pgfpathlineto{\pgfqpoint{3.339728in}{2.877515in}}%
\pgfpathlineto{\pgfqpoint{3.347245in}{2.885843in}}%
\pgfpathlineto{\pgfqpoint{3.354757in}{2.894245in}}%
\pgfpathlineto{\pgfqpoint{3.362262in}{2.902722in}}%
\pgfpathclose%
\pgfusepath{fill}%
\end{pgfscope}%
\begin{pgfscope}%
\pgfpathrectangle{\pgfqpoint{1.254980in}{0.150000in}}{\pgfqpoint{5.490039in}{5.490039in}}%
\pgfusepath{clip}%
\pgfsetbuttcap%
\pgfsetroundjoin%
\definecolor{currentfill}{rgb}{0.280267,0.073417,0.397163}%
\pgfsetfillcolor{currentfill}%
\pgfsetfillopacity{0.700000}%
\pgfsetlinewidth{0.000000pt}%
\definecolor{currentstroke}{rgb}{0.000000,0.000000,0.000000}%
\pgfsetstrokecolor{currentstroke}%
\pgfsetdash{}{0pt}%
\pgfpathmoveto{\pgfqpoint{3.492064in}{2.890279in}}%
\pgfpathlineto{\pgfqpoint{3.504560in}{2.884589in}}%
\pgfpathlineto{\pgfqpoint{3.517061in}{2.878943in}}%
\pgfpathlineto{\pgfqpoint{3.529565in}{2.873340in}}%
\pgfpathlineto{\pgfqpoint{3.542073in}{2.867780in}}%
\pgfpathlineto{\pgfqpoint{3.534621in}{2.859201in}}%
\pgfpathlineto{\pgfqpoint{3.527164in}{2.850698in}}%
\pgfpathlineto{\pgfqpoint{3.519701in}{2.842268in}}%
\pgfpathlineto{\pgfqpoint{3.512232in}{2.833909in}}%
\pgfpathlineto{\pgfqpoint{3.499714in}{2.839399in}}%
\pgfpathlineto{\pgfqpoint{3.487200in}{2.844932in}}%
\pgfpathlineto{\pgfqpoint{3.474689in}{2.850508in}}%
\pgfpathlineto{\pgfqpoint{3.462183in}{2.856129in}}%
\pgfpathlineto{\pgfqpoint{3.469662in}{2.864553in}}%
\pgfpathlineto{\pgfqpoint{3.477135in}{2.873051in}}%
\pgfpathlineto{\pgfqpoint{3.484602in}{2.881626in}}%
\pgfpathlineto{\pgfqpoint{3.492064in}{2.890279in}}%
\pgfpathclose%
\pgfusepath{fill}%
\end{pgfscope}%
\begin{pgfscope}%
\pgfpathrectangle{\pgfqpoint{1.254980in}{0.150000in}}{\pgfqpoint{5.490039in}{5.490039in}}%
\pgfusepath{clip}%
\pgfsetbuttcap%
\pgfsetroundjoin%
\definecolor{currentfill}{rgb}{0.279566,0.067836,0.391917}%
\pgfsetfillcolor{currentfill}%
\pgfsetfillopacity{0.700000}%
\pgfsetlinewidth{0.000000pt}%
\definecolor{currentstroke}{rgb}{0.000000,0.000000,0.000000}%
\pgfsetstrokecolor{currentstroke}%
\pgfsetdash{}{0pt}%
\pgfpathmoveto{\pgfqpoint{3.961064in}{2.886583in}}%
\pgfpathlineto{\pgfqpoint{3.973640in}{2.881583in}}%
\pgfpathlineto{\pgfqpoint{3.986221in}{2.876617in}}%
\pgfpathlineto{\pgfqpoint{3.998806in}{2.871685in}}%
\pgfpathlineto{\pgfqpoint{4.011397in}{2.866787in}}%
\pgfpathlineto{\pgfqpoint{4.004098in}{2.857450in}}%
\pgfpathlineto{\pgfqpoint{3.996794in}{2.848220in}}%
\pgfpathlineto{\pgfqpoint{3.989486in}{2.839092in}}%
\pgfpathlineto{\pgfqpoint{3.982173in}{2.830063in}}%
\pgfpathlineto{\pgfqpoint{3.969573in}{2.834841in}}%
\pgfpathlineto{\pgfqpoint{3.956977in}{2.839653in}}%
\pgfpathlineto{\pgfqpoint{3.944387in}{2.844500in}}%
\pgfpathlineto{\pgfqpoint{3.931801in}{2.849381in}}%
\pgfpathlineto{\pgfqpoint{3.939124in}{2.858525in}}%
\pgfpathlineto{\pgfqpoint{3.946442in}{2.867771in}}%
\pgfpathlineto{\pgfqpoint{3.953755in}{2.877122in}}%
\pgfpathlineto{\pgfqpoint{3.961064in}{2.886583in}}%
\pgfpathclose%
\pgfusepath{fill}%
\end{pgfscope}%
\begin{pgfscope}%
\pgfpathrectangle{\pgfqpoint{1.254980in}{0.150000in}}{\pgfqpoint{5.490039in}{5.490039in}}%
\pgfusepath{clip}%
\pgfsetbuttcap%
\pgfsetroundjoin%
\definecolor{currentfill}{rgb}{0.279566,0.067836,0.391917}%
\pgfsetfillcolor{currentfill}%
\pgfsetfillopacity{0.700000}%
\pgfsetlinewidth{0.000000pt}%
\definecolor{currentstroke}{rgb}{0.000000,0.000000,0.000000}%
\pgfsetstrokecolor{currentstroke}%
\pgfsetdash{}{0pt}%
\pgfpathmoveto{\pgfqpoint{3.621855in}{2.880759in}}%
\pgfpathlineto{\pgfqpoint{3.634374in}{2.875326in}}%
\pgfpathlineto{\pgfqpoint{3.646896in}{2.869933in}}%
\pgfpathlineto{\pgfqpoint{3.659422in}{2.864581in}}%
\pgfpathlineto{\pgfqpoint{3.671953in}{2.859270in}}%
\pgfpathlineto{\pgfqpoint{3.664544in}{2.850535in}}%
\pgfpathlineto{\pgfqpoint{3.657129in}{2.841881in}}%
\pgfpathlineto{\pgfqpoint{3.649709in}{2.833304in}}%
\pgfpathlineto{\pgfqpoint{3.642283in}{2.824802in}}%
\pgfpathlineto{\pgfqpoint{3.629742in}{2.830032in}}%
\pgfpathlineto{\pgfqpoint{3.617206in}{2.835301in}}%
\pgfpathlineto{\pgfqpoint{3.604673in}{2.840611in}}%
\pgfpathlineto{\pgfqpoint{3.592145in}{2.845962in}}%
\pgfpathlineto{\pgfqpoint{3.599581in}{2.854541in}}%
\pgfpathlineto{\pgfqpoint{3.607012in}{2.863199in}}%
\pgfpathlineto{\pgfqpoint{3.614436in}{2.871937in}}%
\pgfpathlineto{\pgfqpoint{3.621855in}{2.880759in}}%
\pgfpathclose%
\pgfusepath{fill}%
\end{pgfscope}%
\begin{pgfscope}%
\pgfpathrectangle{\pgfqpoint{1.254980in}{0.150000in}}{\pgfqpoint{5.490039in}{5.490039in}}%
\pgfusepath{clip}%
\pgfsetbuttcap%
\pgfsetroundjoin%
\definecolor{currentfill}{rgb}{0.281446,0.084320,0.407414}%
\pgfsetfillcolor{currentfill}%
\pgfsetfillopacity{0.700000}%
\pgfsetlinewidth{0.000000pt}%
\definecolor{currentstroke}{rgb}{0.000000,0.000000,0.000000}%
\pgfsetstrokecolor{currentstroke}%
\pgfsetdash{}{0pt}%
\pgfpathmoveto{\pgfqpoint{4.300266in}{2.907874in}}%
\pgfpathlineto{\pgfqpoint{4.312909in}{2.903092in}}%
\pgfpathlineto{\pgfqpoint{4.325557in}{2.898339in}}%
\pgfpathlineto{\pgfqpoint{4.338210in}{2.893617in}}%
\pgfpathlineto{\pgfqpoint{4.350867in}{2.888925in}}%
\pgfpathlineto{\pgfqpoint{4.343668in}{2.878777in}}%
\pgfpathlineto{\pgfqpoint{4.336465in}{2.868773in}}%
\pgfpathlineto{\pgfqpoint{4.329258in}{2.858910in}}%
\pgfpathlineto{\pgfqpoint{4.322048in}{2.849183in}}%
\pgfpathlineto{\pgfqpoint{4.309380in}{2.853718in}}%
\pgfpathlineto{\pgfqpoint{4.296717in}{2.858283in}}%
\pgfpathlineto{\pgfqpoint{4.284059in}{2.862879in}}%
\pgfpathlineto{\pgfqpoint{4.271406in}{2.867505in}}%
\pgfpathlineto{\pgfqpoint{4.278627in}{2.877385in}}%
\pgfpathlineto{\pgfqpoint{4.285844in}{2.887403in}}%
\pgfpathlineto{\pgfqpoint{4.293057in}{2.897564in}}%
\pgfpathlineto{\pgfqpoint{4.300266in}{2.907874in}}%
\pgfpathclose%
\pgfusepath{fill}%
\end{pgfscope}%
\begin{pgfscope}%
\pgfpathrectangle{\pgfqpoint{1.254980in}{0.150000in}}{\pgfqpoint{5.490039in}{5.490039in}}%
\pgfusepath{clip}%
\pgfsetbuttcap%
\pgfsetroundjoin%
\definecolor{currentfill}{rgb}{0.279566,0.067836,0.391917}%
\pgfsetfillcolor{currentfill}%
\pgfsetfillopacity{0.700000}%
\pgfsetlinewidth{0.000000pt}%
\definecolor{currentstroke}{rgb}{0.000000,0.000000,0.000000}%
\pgfsetstrokecolor{currentstroke}%
\pgfsetdash{}{0pt}%
\pgfpathmoveto{\pgfqpoint{4.090915in}{2.885501in}}%
\pgfpathlineto{\pgfqpoint{4.103519in}{2.880639in}}%
\pgfpathlineto{\pgfqpoint{4.116128in}{2.875810in}}%
\pgfpathlineto{\pgfqpoint{4.128741in}{2.871013in}}%
\pgfpathlineto{\pgfqpoint{4.141360in}{2.866249in}}%
\pgfpathlineto{\pgfqpoint{4.134099in}{2.856718in}}%
\pgfpathlineto{\pgfqpoint{4.126834in}{2.847304in}}%
\pgfpathlineto{\pgfqpoint{4.119564in}{2.838003in}}%
\pgfpathlineto{\pgfqpoint{4.112290in}{2.828810in}}%
\pgfpathlineto{\pgfqpoint{4.099661in}{2.833443in}}%
\pgfpathlineto{\pgfqpoint{4.087038in}{2.838107in}}%
\pgfpathlineto{\pgfqpoint{4.074419in}{2.842804in}}%
\pgfpathlineto{\pgfqpoint{4.061805in}{2.847534in}}%
\pgfpathlineto{\pgfqpoint{4.069089in}{2.856854in}}%
\pgfpathlineto{\pgfqpoint{4.076369in}{2.866286in}}%
\pgfpathlineto{\pgfqpoint{4.083644in}{2.875834in}}%
\pgfpathlineto{\pgfqpoint{4.090915in}{2.885501in}}%
\pgfpathclose%
\pgfusepath{fill}%
\end{pgfscope}%
\begin{pgfscope}%
\pgfpathrectangle{\pgfqpoint{1.254980in}{0.150000in}}{\pgfqpoint{5.490039in}{5.490039in}}%
\pgfusepath{clip}%
\pgfsetbuttcap%
\pgfsetroundjoin%
\definecolor{currentfill}{rgb}{0.278791,0.062145,0.386592}%
\pgfsetfillcolor{currentfill}%
\pgfsetfillopacity{0.700000}%
\pgfsetlinewidth{0.000000pt}%
\definecolor{currentstroke}{rgb}{0.000000,0.000000,0.000000}%
\pgfsetstrokecolor{currentstroke}%
\pgfsetdash{}{0pt}%
\pgfpathmoveto{\pgfqpoint{3.751661in}{2.873841in}}%
\pgfpathlineto{\pgfqpoint{3.764203in}{2.868630in}}%
\pgfpathlineto{\pgfqpoint{3.776750in}{2.863456in}}%
\pgfpathlineto{\pgfqpoint{3.789300in}{2.858321in}}%
\pgfpathlineto{\pgfqpoint{3.801856in}{2.853222in}}%
\pgfpathlineto{\pgfqpoint{3.794488in}{2.844334in}}%
\pgfpathlineto{\pgfqpoint{3.787114in}{2.835532in}}%
\pgfpathlineto{\pgfqpoint{3.779736in}{2.826814in}}%
\pgfpathlineto{\pgfqpoint{3.772352in}{2.818176in}}%
\pgfpathlineto{\pgfqpoint{3.759787in}{2.823179in}}%
\pgfpathlineto{\pgfqpoint{3.747226in}{2.828220in}}%
\pgfpathlineto{\pgfqpoint{3.734670in}{2.833299in}}%
\pgfpathlineto{\pgfqpoint{3.722118in}{2.838415in}}%
\pgfpathlineto{\pgfqpoint{3.729512in}{2.847144in}}%
\pgfpathlineto{\pgfqpoint{3.736900in}{2.855955in}}%
\pgfpathlineto{\pgfqpoint{3.744283in}{2.864853in}}%
\pgfpathlineto{\pgfqpoint{3.751661in}{2.873841in}}%
\pgfpathclose%
\pgfusepath{fill}%
\end{pgfscope}%
\begin{pgfscope}%
\pgfpathrectangle{\pgfqpoint{1.254980in}{0.150000in}}{\pgfqpoint{5.490039in}{5.490039in}}%
\pgfusepath{clip}%
\pgfsetbuttcap%
\pgfsetroundjoin%
\definecolor{currentfill}{rgb}{0.278791,0.062145,0.386592}%
\pgfsetfillcolor{currentfill}%
\pgfsetfillopacity{0.700000}%
\pgfsetlinewidth{0.000000pt}%
\definecolor{currentstroke}{rgb}{0.000000,0.000000,0.000000}%
\pgfsetstrokecolor{currentstroke}%
\pgfsetdash{}{0pt}%
\pgfpathmoveto{\pgfqpoint{3.881502in}{2.869258in}}%
\pgfpathlineto{\pgfqpoint{3.894070in}{2.864236in}}%
\pgfpathlineto{\pgfqpoint{3.906642in}{2.859249in}}%
\pgfpathlineto{\pgfqpoint{3.919219in}{2.854297in}}%
\pgfpathlineto{\pgfqpoint{3.931801in}{2.849381in}}%
\pgfpathlineto{\pgfqpoint{3.924473in}{2.840335in}}%
\pgfpathlineto{\pgfqpoint{3.917140in}{2.831384in}}%
\pgfpathlineto{\pgfqpoint{3.909802in}{2.822523in}}%
\pgfpathlineto{\pgfqpoint{3.902459in}{2.813750in}}%
\pgfpathlineto{\pgfqpoint{3.889868in}{2.818559in}}%
\pgfpathlineto{\pgfqpoint{3.877281in}{2.823403in}}%
\pgfpathlineto{\pgfqpoint{3.864698in}{2.828282in}}%
\pgfpathlineto{\pgfqpoint{3.852121in}{2.833198in}}%
\pgfpathlineto{\pgfqpoint{3.859474in}{2.842074in}}%
\pgfpathlineto{\pgfqpoint{3.866822in}{2.851040in}}%
\pgfpathlineto{\pgfqpoint{3.874165in}{2.860101in}}%
\pgfpathlineto{\pgfqpoint{3.881502in}{2.869258in}}%
\pgfpathclose%
\pgfusepath{fill}%
\end{pgfscope}%
\begin{pgfscope}%
\pgfpathrectangle{\pgfqpoint{1.254980in}{0.150000in}}{\pgfqpoint{5.490039in}{5.490039in}}%
\pgfusepath{clip}%
\pgfsetbuttcap%
\pgfsetroundjoin%
\definecolor{currentfill}{rgb}{0.280894,0.078907,0.402329}%
\pgfsetfillcolor{currentfill}%
\pgfsetfillopacity{0.700000}%
\pgfsetlinewidth{0.000000pt}%
\definecolor{currentstroke}{rgb}{0.000000,0.000000,0.000000}%
\pgfsetstrokecolor{currentstroke}%
\pgfsetdash{}{0pt}%
\pgfpathmoveto{\pgfqpoint{3.282287in}{2.893459in}}%
\pgfpathlineto{\pgfqpoint{3.294762in}{2.887335in}}%
\pgfpathlineto{\pgfqpoint{3.307239in}{2.881260in}}%
\pgfpathlineto{\pgfqpoint{3.319720in}{2.875235in}}%
\pgfpathlineto{\pgfqpoint{3.332204in}{2.869257in}}%
\pgfpathlineto{\pgfqpoint{3.324675in}{2.861069in}}%
\pgfpathlineto{\pgfqpoint{3.317139in}{2.852948in}}%
\pgfpathlineto{\pgfqpoint{3.309597in}{2.844893in}}%
\pgfpathlineto{\pgfqpoint{3.302049in}{2.836901in}}%
\pgfpathlineto{\pgfqpoint{3.289554in}{2.842833in}}%
\pgfpathlineto{\pgfqpoint{3.277063in}{2.848814in}}%
\pgfpathlineto{\pgfqpoint{3.264574in}{2.854844in}}%
\pgfpathlineto{\pgfqpoint{3.252089in}{2.860923in}}%
\pgfpathlineto{\pgfqpoint{3.259648in}{2.868955in}}%
\pgfpathlineto{\pgfqpoint{3.267201in}{2.877054in}}%
\pgfpathlineto{\pgfqpoint{3.274747in}{2.885221in}}%
\pgfpathlineto{\pgfqpoint{3.282287in}{2.893459in}}%
\pgfpathclose%
\pgfusepath{fill}%
\end{pgfscope}%
\begin{pgfscope}%
\pgfpathrectangle{\pgfqpoint{1.254980in}{0.150000in}}{\pgfqpoint{5.490039in}{5.490039in}}%
\pgfusepath{clip}%
\pgfsetbuttcap%
\pgfsetroundjoin%
\definecolor{currentfill}{rgb}{0.280267,0.073417,0.397163}%
\pgfsetfillcolor{currentfill}%
\pgfsetfillopacity{0.700000}%
\pgfsetlinewidth{0.000000pt}%
\definecolor{currentstroke}{rgb}{0.000000,0.000000,0.000000}%
\pgfsetstrokecolor{currentstroke}%
\pgfsetdash{}{0pt}%
\pgfpathmoveto{\pgfqpoint{4.220845in}{2.886318in}}%
\pgfpathlineto{\pgfqpoint{4.233478in}{2.881568in}}%
\pgfpathlineto{\pgfqpoint{4.246116in}{2.876849in}}%
\pgfpathlineto{\pgfqpoint{4.258758in}{2.872162in}}%
\pgfpathlineto{\pgfqpoint{4.271406in}{2.867505in}}%
\pgfpathlineto{\pgfqpoint{4.264182in}{2.857759in}}%
\pgfpathlineto{\pgfqpoint{4.256954in}{2.848142in}}%
\pgfpathlineto{\pgfqpoint{4.249722in}{2.838650in}}%
\pgfpathlineto{\pgfqpoint{4.242485in}{2.829277in}}%
\pgfpathlineto{\pgfqpoint{4.229827in}{2.833789in}}%
\pgfpathlineto{\pgfqpoint{4.217174in}{2.838332in}}%
\pgfpathlineto{\pgfqpoint{4.204526in}{2.842906in}}%
\pgfpathlineto{\pgfqpoint{4.191883in}{2.847512in}}%
\pgfpathlineto{\pgfqpoint{4.199129in}{2.857024in}}%
\pgfpathlineto{\pgfqpoint{4.206372in}{2.866660in}}%
\pgfpathlineto{\pgfqpoint{4.213610in}{2.876423in}}%
\pgfpathlineto{\pgfqpoint{4.220845in}{2.886318in}}%
\pgfpathclose%
\pgfusepath{fill}%
\end{pgfscope}%
\begin{pgfscope}%
\pgfpathrectangle{\pgfqpoint{1.254980in}{0.150000in}}{\pgfqpoint{5.490039in}{5.490039in}}%
\pgfusepath{clip}%
\pgfsetbuttcap%
\pgfsetroundjoin%
\definecolor{currentfill}{rgb}{0.279566,0.067836,0.391917}%
\pgfsetfillcolor{currentfill}%
\pgfsetfillopacity{0.700000}%
\pgfsetlinewidth{0.000000pt}%
\definecolor{currentstroke}{rgb}{0.000000,0.000000,0.000000}%
\pgfsetstrokecolor{currentstroke}%
\pgfsetdash{}{0pt}%
\pgfpathmoveto{\pgfqpoint{3.412193in}{2.879057in}}%
\pgfpathlineto{\pgfqpoint{3.424685in}{2.873257in}}%
\pgfpathlineto{\pgfqpoint{3.437181in}{2.867503in}}%
\pgfpathlineto{\pgfqpoint{3.449680in}{2.861793in}}%
\pgfpathlineto{\pgfqpoint{3.462183in}{2.856129in}}%
\pgfpathlineto{\pgfqpoint{3.454698in}{2.847776in}}%
\pgfpathlineto{\pgfqpoint{3.447207in}{2.839492in}}%
\pgfpathlineto{\pgfqpoint{3.439710in}{2.831275in}}%
\pgfpathlineto{\pgfqpoint{3.432208in}{2.823124in}}%
\pgfpathlineto{\pgfqpoint{3.419695in}{2.828731in}}%
\pgfpathlineto{\pgfqpoint{3.407185in}{2.834383in}}%
\pgfpathlineto{\pgfqpoint{3.394679in}{2.840080in}}%
\pgfpathlineto{\pgfqpoint{3.382177in}{2.845822in}}%
\pgfpathlineto{\pgfqpoint{3.389690in}{2.854027in}}%
\pgfpathlineto{\pgfqpoint{3.397197in}{2.862299in}}%
\pgfpathlineto{\pgfqpoint{3.404698in}{2.870642in}}%
\pgfpathlineto{\pgfqpoint{3.412193in}{2.879057in}}%
\pgfpathclose%
\pgfusepath{fill}%
\end{pgfscope}%
\begin{pgfscope}%
\pgfpathrectangle{\pgfqpoint{1.254980in}{0.150000in}}{\pgfqpoint{5.490039in}{5.490039in}}%
\pgfusepath{clip}%
\pgfsetbuttcap%
\pgfsetroundjoin%
\definecolor{currentfill}{rgb}{0.278791,0.062145,0.386592}%
\pgfsetfillcolor{currentfill}%
\pgfsetfillopacity{0.700000}%
\pgfsetlinewidth{0.000000pt}%
\definecolor{currentstroke}{rgb}{0.000000,0.000000,0.000000}%
\pgfsetstrokecolor{currentstroke}%
\pgfsetdash{}{0pt}%
\pgfpathmoveto{\pgfqpoint{3.542073in}{2.867780in}}%
\pgfpathlineto{\pgfqpoint{3.554585in}{2.862262in}}%
\pgfpathlineto{\pgfqpoint{3.567101in}{2.856787in}}%
\pgfpathlineto{\pgfqpoint{3.579621in}{2.851354in}}%
\pgfpathlineto{\pgfqpoint{3.592145in}{2.845962in}}%
\pgfpathlineto{\pgfqpoint{3.584704in}{2.837458in}}%
\pgfpathlineto{\pgfqpoint{3.577257in}{2.829026in}}%
\pgfpathlineto{\pgfqpoint{3.569804in}{2.820665in}}%
\pgfpathlineto{\pgfqpoint{3.562346in}{2.812371in}}%
\pgfpathlineto{\pgfqpoint{3.549811in}{2.817693in}}%
\pgfpathlineto{\pgfqpoint{3.537281in}{2.823056in}}%
\pgfpathlineto{\pgfqpoint{3.524755in}{2.828461in}}%
\pgfpathlineto{\pgfqpoint{3.512232in}{2.833909in}}%
\pgfpathlineto{\pgfqpoint{3.519701in}{2.842268in}}%
\pgfpathlineto{\pgfqpoint{3.527164in}{2.850698in}}%
\pgfpathlineto{\pgfqpoint{3.534621in}{2.859201in}}%
\pgfpathlineto{\pgfqpoint{3.542073in}{2.867780in}}%
\pgfpathclose%
\pgfusepath{fill}%
\end{pgfscope}%
\begin{pgfscope}%
\pgfpathrectangle{\pgfqpoint{1.254980in}{0.150000in}}{\pgfqpoint{5.490039in}{5.490039in}}%
\pgfusepath{clip}%
\pgfsetbuttcap%
\pgfsetroundjoin%
\definecolor{currentfill}{rgb}{0.278791,0.062145,0.386592}%
\pgfsetfillcolor{currentfill}%
\pgfsetfillopacity{0.700000}%
\pgfsetlinewidth{0.000000pt}%
\definecolor{currentstroke}{rgb}{0.000000,0.000000,0.000000}%
\pgfsetstrokecolor{currentstroke}%
\pgfsetdash{}{0pt}%
\pgfpathmoveto{\pgfqpoint{4.011397in}{2.866787in}}%
\pgfpathlineto{\pgfqpoint{4.023991in}{2.861924in}}%
\pgfpathlineto{\pgfqpoint{4.036591in}{2.857094in}}%
\pgfpathlineto{\pgfqpoint{4.049196in}{2.852297in}}%
\pgfpathlineto{\pgfqpoint{4.061805in}{2.847534in}}%
\pgfpathlineto{\pgfqpoint{4.054516in}{2.838322in}}%
\pgfpathlineto{\pgfqpoint{4.047222in}{2.829213in}}%
\pgfpathlineto{\pgfqpoint{4.039924in}{2.820203in}}%
\pgfpathlineto{\pgfqpoint{4.032622in}{2.811290in}}%
\pgfpathlineto{\pgfqpoint{4.020002in}{2.815933in}}%
\pgfpathlineto{\pgfqpoint{4.007388in}{2.820609in}}%
\pgfpathlineto{\pgfqpoint{3.994778in}{2.825320in}}%
\pgfpathlineto{\pgfqpoint{3.982173in}{2.830063in}}%
\pgfpathlineto{\pgfqpoint{3.989486in}{2.839092in}}%
\pgfpathlineto{\pgfqpoint{3.996794in}{2.848220in}}%
\pgfpathlineto{\pgfqpoint{4.004098in}{2.857450in}}%
\pgfpathlineto{\pgfqpoint{4.011397in}{2.866787in}}%
\pgfpathclose%
\pgfusepath{fill}%
\end{pgfscope}%
\begin{pgfscope}%
\pgfpathrectangle{\pgfqpoint{1.254980in}{0.150000in}}{\pgfqpoint{5.490039in}{5.490039in}}%
\pgfusepath{clip}%
\pgfsetbuttcap%
\pgfsetroundjoin%
\definecolor{currentfill}{rgb}{0.278791,0.062145,0.386592}%
\pgfsetfillcolor{currentfill}%
\pgfsetfillopacity{0.700000}%
\pgfsetlinewidth{0.000000pt}%
\definecolor{currentstroke}{rgb}{0.000000,0.000000,0.000000}%
\pgfsetstrokecolor{currentstroke}%
\pgfsetdash{}{0pt}%
\pgfpathmoveto{\pgfqpoint{3.671953in}{2.859270in}}%
\pgfpathlineto{\pgfqpoint{3.684488in}{2.853997in}}%
\pgfpathlineto{\pgfqpoint{3.697027in}{2.848764in}}%
\pgfpathlineto{\pgfqpoint{3.709570in}{2.843571in}}%
\pgfpathlineto{\pgfqpoint{3.722118in}{2.838415in}}%
\pgfpathlineto{\pgfqpoint{3.714718in}{2.829768in}}%
\pgfpathlineto{\pgfqpoint{3.707314in}{2.821198in}}%
\pgfpathlineto{\pgfqpoint{3.699904in}{2.812702in}}%
\pgfpathlineto{\pgfqpoint{3.692488in}{2.804279in}}%
\pgfpathlineto{\pgfqpoint{3.679930in}{2.809351in}}%
\pgfpathlineto{\pgfqpoint{3.667377in}{2.814462in}}%
\pgfpathlineto{\pgfqpoint{3.654828in}{2.819613in}}%
\pgfpathlineto{\pgfqpoint{3.642283in}{2.824802in}}%
\pgfpathlineto{\pgfqpoint{3.649709in}{2.833304in}}%
\pgfpathlineto{\pgfqpoint{3.657129in}{2.841881in}}%
\pgfpathlineto{\pgfqpoint{3.664544in}{2.850535in}}%
\pgfpathlineto{\pgfqpoint{3.671953in}{2.859270in}}%
\pgfpathclose%
\pgfusepath{fill}%
\end{pgfscope}%
\begin{pgfscope}%
\pgfpathrectangle{\pgfqpoint{1.254980in}{0.150000in}}{\pgfqpoint{5.490039in}{5.490039in}}%
\pgfusepath{clip}%
\pgfsetbuttcap%
\pgfsetroundjoin%
\definecolor{currentfill}{rgb}{0.280894,0.078907,0.402329}%
\pgfsetfillcolor{currentfill}%
\pgfsetfillopacity{0.700000}%
\pgfsetlinewidth{0.000000pt}%
\definecolor{currentstroke}{rgb}{0.000000,0.000000,0.000000}%
\pgfsetstrokecolor{currentstroke}%
\pgfsetdash{}{0pt}%
\pgfpathmoveto{\pgfqpoint{4.350867in}{2.888925in}}%
\pgfpathlineto{\pgfqpoint{4.363530in}{2.884263in}}%
\pgfpathlineto{\pgfqpoint{4.376198in}{2.879631in}}%
\pgfpathlineto{\pgfqpoint{4.388871in}{2.875028in}}%
\pgfpathlineto{\pgfqpoint{4.401549in}{2.870455in}}%
\pgfpathlineto{\pgfqpoint{4.394360in}{2.860468in}}%
\pgfpathlineto{\pgfqpoint{4.387168in}{2.850623in}}%
\pgfpathlineto{\pgfqpoint{4.379972in}{2.840915in}}%
\pgfpathlineto{\pgfqpoint{4.372772in}{2.831340in}}%
\pgfpathlineto{\pgfqpoint{4.360083in}{2.835756in}}%
\pgfpathlineto{\pgfqpoint{4.347400in}{2.840202in}}%
\pgfpathlineto{\pgfqpoint{4.334721in}{2.844677in}}%
\pgfpathlineto{\pgfqpoint{4.322048in}{2.849183in}}%
\pgfpathlineto{\pgfqpoint{4.329258in}{2.858910in}}%
\pgfpathlineto{\pgfqpoint{4.336465in}{2.868773in}}%
\pgfpathlineto{\pgfqpoint{4.343668in}{2.878777in}}%
\pgfpathlineto{\pgfqpoint{4.350867in}{2.888925in}}%
\pgfpathclose%
\pgfusepath{fill}%
\end{pgfscope}%
\begin{pgfscope}%
\pgfpathrectangle{\pgfqpoint{1.254980in}{0.150000in}}{\pgfqpoint{5.490039in}{5.490039in}}%
\pgfusepath{clip}%
\pgfsetbuttcap%
\pgfsetroundjoin%
\definecolor{currentfill}{rgb}{0.278791,0.062145,0.386592}%
\pgfsetfillcolor{currentfill}%
\pgfsetfillopacity{0.700000}%
\pgfsetlinewidth{0.000000pt}%
\definecolor{currentstroke}{rgb}{0.000000,0.000000,0.000000}%
\pgfsetstrokecolor{currentstroke}%
\pgfsetdash{}{0pt}%
\pgfpathmoveto{\pgfqpoint{3.801856in}{2.853222in}}%
\pgfpathlineto{\pgfqpoint{3.814415in}{2.848161in}}%
\pgfpathlineto{\pgfqpoint{3.826979in}{2.843137in}}%
\pgfpathlineto{\pgfqpoint{3.839548in}{2.838149in}}%
\pgfpathlineto{\pgfqpoint{3.852121in}{2.833198in}}%
\pgfpathlineto{\pgfqpoint{3.844763in}{2.824409in}}%
\pgfpathlineto{\pgfqpoint{3.837400in}{2.815704in}}%
\pgfpathlineto{\pgfqpoint{3.830031in}{2.807079in}}%
\pgfpathlineto{\pgfqpoint{3.822658in}{2.798532in}}%
\pgfpathlineto{\pgfqpoint{3.810074in}{2.803388in}}%
\pgfpathlineto{\pgfqpoint{3.797496in}{2.808280in}}%
\pgfpathlineto{\pgfqpoint{3.784921in}{2.813210in}}%
\pgfpathlineto{\pgfqpoint{3.772352in}{2.818176in}}%
\pgfpathlineto{\pgfqpoint{3.779736in}{2.826814in}}%
\pgfpathlineto{\pgfqpoint{3.787114in}{2.835532in}}%
\pgfpathlineto{\pgfqpoint{3.794488in}{2.844334in}}%
\pgfpathlineto{\pgfqpoint{3.801856in}{2.853222in}}%
\pgfpathclose%
\pgfusepath{fill}%
\end{pgfscope}%
\begin{pgfscope}%
\pgfpathrectangle{\pgfqpoint{1.254980in}{0.150000in}}{\pgfqpoint{5.490039in}{5.490039in}}%
\pgfusepath{clip}%
\pgfsetbuttcap%
\pgfsetroundjoin%
\definecolor{currentfill}{rgb}{0.279566,0.067836,0.391917}%
\pgfsetfillcolor{currentfill}%
\pgfsetfillopacity{0.700000}%
\pgfsetlinewidth{0.000000pt}%
\definecolor{currentstroke}{rgb}{0.000000,0.000000,0.000000}%
\pgfsetstrokecolor{currentstroke}%
\pgfsetdash{}{0pt}%
\pgfpathmoveto{\pgfqpoint{4.141360in}{2.866249in}}%
\pgfpathlineto{\pgfqpoint{4.153983in}{2.861517in}}%
\pgfpathlineto{\pgfqpoint{4.166611in}{2.856817in}}%
\pgfpathlineto{\pgfqpoint{4.179245in}{2.852149in}}%
\pgfpathlineto{\pgfqpoint{4.191883in}{2.847512in}}%
\pgfpathlineto{\pgfqpoint{4.184632in}{2.838118in}}%
\pgfpathlineto{\pgfqpoint{4.177377in}{2.828837in}}%
\pgfpathlineto{\pgfqpoint{4.170118in}{2.819667in}}%
\pgfpathlineto{\pgfqpoint{4.162854in}{2.810602in}}%
\pgfpathlineto{\pgfqpoint{4.150206in}{2.815106in}}%
\pgfpathlineto{\pgfqpoint{4.137562in}{2.819643in}}%
\pgfpathlineto{\pgfqpoint{4.124923in}{2.824210in}}%
\pgfpathlineto{\pgfqpoint{4.112290in}{2.828810in}}%
\pgfpathlineto{\pgfqpoint{4.119564in}{2.838003in}}%
\pgfpathlineto{\pgfqpoint{4.126834in}{2.847304in}}%
\pgfpathlineto{\pgfqpoint{4.134099in}{2.856718in}}%
\pgfpathlineto{\pgfqpoint{4.141360in}{2.866249in}}%
\pgfpathclose%
\pgfusepath{fill}%
\end{pgfscope}%
\begin{pgfscope}%
\pgfpathrectangle{\pgfqpoint{1.254980in}{0.150000in}}{\pgfqpoint{5.490039in}{5.490039in}}%
\pgfusepath{clip}%
\pgfsetbuttcap%
\pgfsetroundjoin%
\definecolor{currentfill}{rgb}{0.280267,0.073417,0.397163}%
\pgfsetfillcolor{currentfill}%
\pgfsetfillopacity{0.700000}%
\pgfsetlinewidth{0.000000pt}%
\definecolor{currentstroke}{rgb}{0.000000,0.000000,0.000000}%
\pgfsetstrokecolor{currentstroke}%
\pgfsetdash{}{0pt}%
\pgfpathmoveto{\pgfqpoint{3.332204in}{2.869257in}}%
\pgfpathlineto{\pgfqpoint{3.344692in}{2.863328in}}%
\pgfpathlineto{\pgfqpoint{3.357184in}{2.857446in}}%
\pgfpathlineto{\pgfqpoint{3.369678in}{2.851611in}}%
\pgfpathlineto{\pgfqpoint{3.382177in}{2.845822in}}%
\pgfpathlineto{\pgfqpoint{3.374658in}{2.837684in}}%
\pgfpathlineto{\pgfqpoint{3.367133in}{2.829609in}}%
\pgfpathlineto{\pgfqpoint{3.359602in}{2.821597in}}%
\pgfpathlineto{\pgfqpoint{3.352065in}{2.813646in}}%
\pgfpathlineto{\pgfqpoint{3.339556in}{2.819390in}}%
\pgfpathlineto{\pgfqpoint{3.327050in}{2.825179in}}%
\pgfpathlineto{\pgfqpoint{3.314548in}{2.831016in}}%
\pgfpathlineto{\pgfqpoint{3.302049in}{2.836901in}}%
\pgfpathlineto{\pgfqpoint{3.309597in}{2.844893in}}%
\pgfpathlineto{\pgfqpoint{3.317139in}{2.852948in}}%
\pgfpathlineto{\pgfqpoint{3.324675in}{2.861069in}}%
\pgfpathlineto{\pgfqpoint{3.332204in}{2.869257in}}%
\pgfpathclose%
\pgfusepath{fill}%
\end{pgfscope}%
\begin{pgfscope}%
\pgfpathrectangle{\pgfqpoint{1.254980in}{0.150000in}}{\pgfqpoint{5.490039in}{5.490039in}}%
\pgfusepath{clip}%
\pgfsetbuttcap%
\pgfsetroundjoin%
\definecolor{currentfill}{rgb}{0.278791,0.062145,0.386592}%
\pgfsetfillcolor{currentfill}%
\pgfsetfillopacity{0.700000}%
\pgfsetlinewidth{0.000000pt}%
\definecolor{currentstroke}{rgb}{0.000000,0.000000,0.000000}%
\pgfsetstrokecolor{currentstroke}%
\pgfsetdash{}{0pt}%
\pgfpathmoveto{\pgfqpoint{3.931801in}{2.849381in}}%
\pgfpathlineto{\pgfqpoint{3.944387in}{2.844500in}}%
\pgfpathlineto{\pgfqpoint{3.956977in}{2.839653in}}%
\pgfpathlineto{\pgfqpoint{3.969573in}{2.834841in}}%
\pgfpathlineto{\pgfqpoint{3.982173in}{2.830063in}}%
\pgfpathlineto{\pgfqpoint{3.974855in}{2.821130in}}%
\pgfpathlineto{\pgfqpoint{3.967532in}{2.812287in}}%
\pgfpathlineto{\pgfqpoint{3.960205in}{2.803532in}}%
\pgfpathlineto{\pgfqpoint{3.952872in}{2.794862in}}%
\pgfpathlineto{\pgfqpoint{3.940262in}{2.799532in}}%
\pgfpathlineto{\pgfqpoint{3.927656in}{2.804237in}}%
\pgfpathlineto{\pgfqpoint{3.915055in}{2.808976in}}%
\pgfpathlineto{\pgfqpoint{3.902459in}{2.813750in}}%
\pgfpathlineto{\pgfqpoint{3.909802in}{2.822523in}}%
\pgfpathlineto{\pgfqpoint{3.917140in}{2.831384in}}%
\pgfpathlineto{\pgfqpoint{3.924473in}{2.840335in}}%
\pgfpathlineto{\pgfqpoint{3.931801in}{2.849381in}}%
\pgfpathclose%
\pgfusepath{fill}%
\end{pgfscope}%
\begin{pgfscope}%
\pgfpathrectangle{\pgfqpoint{1.254980in}{0.150000in}}{\pgfqpoint{5.490039in}{5.490039in}}%
\pgfusepath{clip}%
\pgfsetbuttcap%
\pgfsetroundjoin%
\definecolor{currentfill}{rgb}{0.278791,0.062145,0.386592}%
\pgfsetfillcolor{currentfill}%
\pgfsetfillopacity{0.700000}%
\pgfsetlinewidth{0.000000pt}%
\definecolor{currentstroke}{rgb}{0.000000,0.000000,0.000000}%
\pgfsetstrokecolor{currentstroke}%
\pgfsetdash{}{0pt}%
\pgfpathmoveto{\pgfqpoint{3.462183in}{2.856129in}}%
\pgfpathlineto{\pgfqpoint{3.474689in}{2.850508in}}%
\pgfpathlineto{\pgfqpoint{3.487200in}{2.844932in}}%
\pgfpathlineto{\pgfqpoint{3.499714in}{2.839399in}}%
\pgfpathlineto{\pgfqpoint{3.512232in}{2.833909in}}%
\pgfpathlineto{\pgfqpoint{3.504758in}{2.825618in}}%
\pgfpathlineto{\pgfqpoint{3.497278in}{2.817393in}}%
\pgfpathlineto{\pgfqpoint{3.489792in}{2.809233in}}%
\pgfpathlineto{\pgfqpoint{3.482300in}{2.801134in}}%
\pgfpathlineto{\pgfqpoint{3.469771in}{2.806567in}}%
\pgfpathlineto{\pgfqpoint{3.457246in}{2.812042in}}%
\pgfpathlineto{\pgfqpoint{3.444725in}{2.817561in}}%
\pgfpathlineto{\pgfqpoint{3.432208in}{2.823124in}}%
\pgfpathlineto{\pgfqpoint{3.439710in}{2.831275in}}%
\pgfpathlineto{\pgfqpoint{3.447207in}{2.839492in}}%
\pgfpathlineto{\pgfqpoint{3.454698in}{2.847776in}}%
\pgfpathlineto{\pgfqpoint{3.462183in}{2.856129in}}%
\pgfpathclose%
\pgfusepath{fill}%
\end{pgfscope}%
\begin{pgfscope}%
\pgfpathrectangle{\pgfqpoint{1.254980in}{0.150000in}}{\pgfqpoint{5.490039in}{5.490039in}}%
\pgfusepath{clip}%
\pgfsetbuttcap%
\pgfsetroundjoin%
\definecolor{currentfill}{rgb}{0.280267,0.073417,0.397163}%
\pgfsetfillcolor{currentfill}%
\pgfsetfillopacity{0.700000}%
\pgfsetlinewidth{0.000000pt}%
\definecolor{currentstroke}{rgb}{0.000000,0.000000,0.000000}%
\pgfsetstrokecolor{currentstroke}%
\pgfsetdash{}{0pt}%
\pgfpathmoveto{\pgfqpoint{4.271406in}{2.867505in}}%
\pgfpathlineto{\pgfqpoint{4.284059in}{2.862879in}}%
\pgfpathlineto{\pgfqpoint{4.296717in}{2.858283in}}%
\pgfpathlineto{\pgfqpoint{4.309380in}{2.853718in}}%
\pgfpathlineto{\pgfqpoint{4.322048in}{2.849183in}}%
\pgfpathlineto{\pgfqpoint{4.314834in}{2.839586in}}%
\pgfpathlineto{\pgfqpoint{4.307617in}{2.830115in}}%
\pgfpathlineto{\pgfqpoint{4.300395in}{2.820766in}}%
\pgfpathlineto{\pgfqpoint{4.293169in}{2.811533in}}%
\pgfpathlineto{\pgfqpoint{4.280491in}{2.815924in}}%
\pgfpathlineto{\pgfqpoint{4.267817in}{2.820344in}}%
\pgfpathlineto{\pgfqpoint{4.255149in}{2.824795in}}%
\pgfpathlineto{\pgfqpoint{4.242485in}{2.829277in}}%
\pgfpathlineto{\pgfqpoint{4.249722in}{2.838650in}}%
\pgfpathlineto{\pgfqpoint{4.256954in}{2.848142in}}%
\pgfpathlineto{\pgfqpoint{4.264182in}{2.857759in}}%
\pgfpathlineto{\pgfqpoint{4.271406in}{2.867505in}}%
\pgfpathclose%
\pgfusepath{fill}%
\end{pgfscope}%
\begin{pgfscope}%
\pgfpathrectangle{\pgfqpoint{1.254980in}{0.150000in}}{\pgfqpoint{5.490039in}{5.490039in}}%
\pgfusepath{clip}%
\pgfsetbuttcap%
\pgfsetroundjoin%
\definecolor{currentfill}{rgb}{0.278791,0.062145,0.386592}%
\pgfsetfillcolor{currentfill}%
\pgfsetfillopacity{0.700000}%
\pgfsetlinewidth{0.000000pt}%
\definecolor{currentstroke}{rgb}{0.000000,0.000000,0.000000}%
\pgfsetstrokecolor{currentstroke}%
\pgfsetdash{}{0pt}%
\pgfpathmoveto{\pgfqpoint{3.592145in}{2.845962in}}%
\pgfpathlineto{\pgfqpoint{3.604673in}{2.840611in}}%
\pgfpathlineto{\pgfqpoint{3.617206in}{2.835301in}}%
\pgfpathlineto{\pgfqpoint{3.629742in}{2.830032in}}%
\pgfpathlineto{\pgfqpoint{3.642283in}{2.824802in}}%
\pgfpathlineto{\pgfqpoint{3.634852in}{2.816373in}}%
\pgfpathlineto{\pgfqpoint{3.627415in}{2.808013in}}%
\pgfpathlineto{\pgfqpoint{3.619972in}{2.799720in}}%
\pgfpathlineto{\pgfqpoint{3.612525in}{2.791492in}}%
\pgfpathlineto{\pgfqpoint{3.599973in}{2.796652in}}%
\pgfpathlineto{\pgfqpoint{3.587427in}{2.801851in}}%
\pgfpathlineto{\pgfqpoint{3.574884in}{2.807091in}}%
\pgfpathlineto{\pgfqpoint{3.562346in}{2.812371in}}%
\pgfpathlineto{\pgfqpoint{3.569804in}{2.820665in}}%
\pgfpathlineto{\pgfqpoint{3.577257in}{2.829026in}}%
\pgfpathlineto{\pgfqpoint{3.584704in}{2.837458in}}%
\pgfpathlineto{\pgfqpoint{3.592145in}{2.845962in}}%
\pgfpathclose%
\pgfusepath{fill}%
\end{pgfscope}%
\begin{pgfscope}%
\pgfpathrectangle{\pgfqpoint{1.254980in}{0.150000in}}{\pgfqpoint{5.490039in}{5.490039in}}%
\pgfusepath{clip}%
\pgfsetbuttcap%
\pgfsetroundjoin%
\definecolor{currentfill}{rgb}{0.278791,0.062145,0.386592}%
\pgfsetfillcolor{currentfill}%
\pgfsetfillopacity{0.700000}%
\pgfsetlinewidth{0.000000pt}%
\definecolor{currentstroke}{rgb}{0.000000,0.000000,0.000000}%
\pgfsetstrokecolor{currentstroke}%
\pgfsetdash{}{0pt}%
\pgfpathmoveto{\pgfqpoint{4.061805in}{2.847534in}}%
\pgfpathlineto{\pgfqpoint{4.074419in}{2.842804in}}%
\pgfpathlineto{\pgfqpoint{4.087038in}{2.838107in}}%
\pgfpathlineto{\pgfqpoint{4.099661in}{2.833443in}}%
\pgfpathlineto{\pgfqpoint{4.112290in}{2.828810in}}%
\pgfpathlineto{\pgfqpoint{4.105011in}{2.819722in}}%
\pgfpathlineto{\pgfqpoint{4.097728in}{2.810735in}}%
\pgfpathlineto{\pgfqpoint{4.090440in}{2.801844in}}%
\pgfpathlineto{\pgfqpoint{4.083148in}{2.793045in}}%
\pgfpathlineto{\pgfqpoint{4.070509in}{2.797558in}}%
\pgfpathlineto{\pgfqpoint{4.057875in}{2.802102in}}%
\pgfpathlineto{\pgfqpoint{4.045246in}{2.806679in}}%
\pgfpathlineto{\pgfqpoint{4.032622in}{2.811290in}}%
\pgfpathlineto{\pgfqpoint{4.039924in}{2.820203in}}%
\pgfpathlineto{\pgfqpoint{4.047222in}{2.829213in}}%
\pgfpathlineto{\pgfqpoint{4.054516in}{2.838322in}}%
\pgfpathlineto{\pgfqpoint{4.061805in}{2.847534in}}%
\pgfpathclose%
\pgfusepath{fill}%
\end{pgfscope}%
\begin{pgfscope}%
\pgfpathrectangle{\pgfqpoint{1.254980in}{0.150000in}}{\pgfqpoint{5.490039in}{5.490039in}}%
\pgfusepath{clip}%
\pgfsetbuttcap%
\pgfsetroundjoin%
\definecolor{currentfill}{rgb}{0.277941,0.056324,0.381191}%
\pgfsetfillcolor{currentfill}%
\pgfsetfillopacity{0.700000}%
\pgfsetlinewidth{0.000000pt}%
\definecolor{currentstroke}{rgb}{0.000000,0.000000,0.000000}%
\pgfsetstrokecolor{currentstroke}%
\pgfsetdash{}{0pt}%
\pgfpathmoveto{\pgfqpoint{3.722118in}{2.838415in}}%
\pgfpathlineto{\pgfqpoint{3.734670in}{2.833299in}}%
\pgfpathlineto{\pgfqpoint{3.747226in}{2.828220in}}%
\pgfpathlineto{\pgfqpoint{3.759787in}{2.823179in}}%
\pgfpathlineto{\pgfqpoint{3.772352in}{2.818176in}}%
\pgfpathlineto{\pgfqpoint{3.764963in}{2.809616in}}%
\pgfpathlineto{\pgfqpoint{3.757568in}{2.801130in}}%
\pgfpathlineto{\pgfqpoint{3.750169in}{2.792715in}}%
\pgfpathlineto{\pgfqpoint{3.742763in}{2.784370in}}%
\pgfpathlineto{\pgfqpoint{3.730188in}{2.789290in}}%
\pgfpathlineto{\pgfqpoint{3.717617in}{2.794248in}}%
\pgfpathlineto{\pgfqpoint{3.705050in}{2.799244in}}%
\pgfpathlineto{\pgfqpoint{3.692488in}{2.804279in}}%
\pgfpathlineto{\pgfqpoint{3.699904in}{2.812702in}}%
\pgfpathlineto{\pgfqpoint{3.707314in}{2.821198in}}%
\pgfpathlineto{\pgfqpoint{3.714718in}{2.829768in}}%
\pgfpathlineto{\pgfqpoint{3.722118in}{2.838415in}}%
\pgfpathclose%
\pgfusepath{fill}%
\end{pgfscope}%
\begin{pgfscope}%
\pgfpathrectangle{\pgfqpoint{1.254980in}{0.150000in}}{\pgfqpoint{5.490039in}{5.490039in}}%
\pgfusepath{clip}%
\pgfsetbuttcap%
\pgfsetroundjoin%
\definecolor{currentfill}{rgb}{0.280894,0.078907,0.402329}%
\pgfsetfillcolor{currentfill}%
\pgfsetfillopacity{0.700000}%
\pgfsetlinewidth{0.000000pt}%
\definecolor{currentstroke}{rgb}{0.000000,0.000000,0.000000}%
\pgfsetstrokecolor{currentstroke}%
\pgfsetdash{}{0pt}%
\pgfpathmoveto{\pgfqpoint{4.401549in}{2.870455in}}%
\pgfpathlineto{\pgfqpoint{4.414232in}{2.865911in}}%
\pgfpathlineto{\pgfqpoint{4.426921in}{2.861397in}}%
\pgfpathlineto{\pgfqpoint{4.439614in}{2.856911in}}%
\pgfpathlineto{\pgfqpoint{4.452313in}{2.852454in}}%
\pgfpathlineto{\pgfqpoint{4.445135in}{2.842628in}}%
\pgfpathlineto{\pgfqpoint{4.437953in}{2.832941in}}%
\pgfpathlineto{\pgfqpoint{4.430768in}{2.823389in}}%
\pgfpathlineto{\pgfqpoint{4.423579in}{2.813967in}}%
\pgfpathlineto{\pgfqpoint{4.410870in}{2.818266in}}%
\pgfpathlineto{\pgfqpoint{4.398165in}{2.822595in}}%
\pgfpathlineto{\pgfqpoint{4.385466in}{2.826953in}}%
\pgfpathlineto{\pgfqpoint{4.372772in}{2.831340in}}%
\pgfpathlineto{\pgfqpoint{4.379972in}{2.840915in}}%
\pgfpathlineto{\pgfqpoint{4.387168in}{2.850623in}}%
\pgfpathlineto{\pgfqpoint{4.394360in}{2.860468in}}%
\pgfpathlineto{\pgfqpoint{4.401549in}{2.870455in}}%
\pgfpathclose%
\pgfusepath{fill}%
\end{pgfscope}%
\begin{pgfscope}%
\pgfpathrectangle{\pgfqpoint{1.254980in}{0.150000in}}{\pgfqpoint{5.490039in}{5.490039in}}%
\pgfusepath{clip}%
\pgfsetbuttcap%
\pgfsetroundjoin%
\definecolor{currentfill}{rgb}{0.277941,0.056324,0.381191}%
\pgfsetfillcolor{currentfill}%
\pgfsetfillopacity{0.700000}%
\pgfsetlinewidth{0.000000pt}%
\definecolor{currentstroke}{rgb}{0.000000,0.000000,0.000000}%
\pgfsetstrokecolor{currentstroke}%
\pgfsetdash{}{0pt}%
\pgfpathmoveto{\pgfqpoint{3.852121in}{2.833198in}}%
\pgfpathlineto{\pgfqpoint{3.864698in}{2.828282in}}%
\pgfpathlineto{\pgfqpoint{3.877281in}{2.823403in}}%
\pgfpathlineto{\pgfqpoint{3.889868in}{2.818559in}}%
\pgfpathlineto{\pgfqpoint{3.902459in}{2.813750in}}%
\pgfpathlineto{\pgfqpoint{3.895111in}{2.805061in}}%
\pgfpathlineto{\pgfqpoint{3.887758in}{2.796452in}}%
\pgfpathlineto{\pgfqpoint{3.880400in}{2.787921in}}%
\pgfpathlineto{\pgfqpoint{3.873037in}{2.779464in}}%
\pgfpathlineto{\pgfqpoint{3.860435in}{2.784178in}}%
\pgfpathlineto{\pgfqpoint{3.847838in}{2.788927in}}%
\pgfpathlineto{\pgfqpoint{3.835245in}{2.793711in}}%
\pgfpathlineto{\pgfqpoint{3.822658in}{2.798532in}}%
\pgfpathlineto{\pgfqpoint{3.830031in}{2.807079in}}%
\pgfpathlineto{\pgfqpoint{3.837400in}{2.815704in}}%
\pgfpathlineto{\pgfqpoint{3.844763in}{2.824409in}}%
\pgfpathlineto{\pgfqpoint{3.852121in}{2.833198in}}%
\pgfpathclose%
\pgfusepath{fill}%
\end{pgfscope}%
\begin{pgfscope}%
\pgfpathrectangle{\pgfqpoint{1.254980in}{0.150000in}}{\pgfqpoint{5.490039in}{5.490039in}}%
\pgfusepath{clip}%
\pgfsetbuttcap%
\pgfsetroundjoin%
\definecolor{currentfill}{rgb}{0.278791,0.062145,0.386592}%
\pgfsetfillcolor{currentfill}%
\pgfsetfillopacity{0.700000}%
\pgfsetlinewidth{0.000000pt}%
\definecolor{currentstroke}{rgb}{0.000000,0.000000,0.000000}%
\pgfsetstrokecolor{currentstroke}%
\pgfsetdash{}{0pt}%
\pgfpathmoveto{\pgfqpoint{4.191883in}{2.847512in}}%
\pgfpathlineto{\pgfqpoint{4.204526in}{2.842906in}}%
\pgfpathlineto{\pgfqpoint{4.217174in}{2.838332in}}%
\pgfpathlineto{\pgfqpoint{4.229827in}{2.833789in}}%
\pgfpathlineto{\pgfqpoint{4.242485in}{2.829277in}}%
\pgfpathlineto{\pgfqpoint{4.235245in}{2.820020in}}%
\pgfpathlineto{\pgfqpoint{4.228001in}{2.810873in}}%
\pgfpathlineto{\pgfqpoint{4.220752in}{2.801834in}}%
\pgfpathlineto{\pgfqpoint{4.213499in}{2.792897in}}%
\pgfpathlineto{\pgfqpoint{4.200830in}{2.797276in}}%
\pgfpathlineto{\pgfqpoint{4.188166in}{2.801687in}}%
\pgfpathlineto{\pgfqpoint{4.175508in}{2.806129in}}%
\pgfpathlineto{\pgfqpoint{4.162854in}{2.810602in}}%
\pgfpathlineto{\pgfqpoint{4.170118in}{2.819667in}}%
\pgfpathlineto{\pgfqpoint{4.177377in}{2.828837in}}%
\pgfpathlineto{\pgfqpoint{4.184632in}{2.838118in}}%
\pgfpathlineto{\pgfqpoint{4.191883in}{2.847512in}}%
\pgfpathclose%
\pgfusepath{fill}%
\end{pgfscope}%
\begin{pgfscope}%
\pgfpathrectangle{\pgfqpoint{1.254980in}{0.150000in}}{\pgfqpoint{5.490039in}{5.490039in}}%
\pgfusepath{clip}%
\pgfsetbuttcap%
\pgfsetroundjoin%
\definecolor{currentfill}{rgb}{0.280267,0.073417,0.397163}%
\pgfsetfillcolor{currentfill}%
\pgfsetfillopacity{0.700000}%
\pgfsetlinewidth{0.000000pt}%
\definecolor{currentstroke}{rgb}{0.000000,0.000000,0.000000}%
\pgfsetstrokecolor{currentstroke}%
\pgfsetdash{}{0pt}%
\pgfpathmoveto{\pgfqpoint{3.252089in}{2.860923in}}%
\pgfpathlineto{\pgfqpoint{3.264574in}{2.854844in}}%
\pgfpathlineto{\pgfqpoint{3.277063in}{2.848814in}}%
\pgfpathlineto{\pgfqpoint{3.289554in}{2.842833in}}%
\pgfpathlineto{\pgfqpoint{3.302049in}{2.836901in}}%
\pgfpathlineto{\pgfqpoint{3.294495in}{2.828971in}}%
\pgfpathlineto{\pgfqpoint{3.286935in}{2.821102in}}%
\pgfpathlineto{\pgfqpoint{3.279369in}{2.813291in}}%
\pgfpathlineto{\pgfqpoint{3.271797in}{2.805538in}}%
\pgfpathlineto{\pgfqpoint{3.259290in}{2.811438in}}%
\pgfpathlineto{\pgfqpoint{3.246787in}{2.817386in}}%
\pgfpathlineto{\pgfqpoint{3.234288in}{2.823383in}}%
\pgfpathlineto{\pgfqpoint{3.221792in}{2.829430in}}%
\pgfpathlineto{\pgfqpoint{3.229376in}{2.837210in}}%
\pgfpathlineto{\pgfqpoint{3.236953in}{2.845052in}}%
\pgfpathlineto{\pgfqpoint{3.244524in}{2.852955in}}%
\pgfpathlineto{\pgfqpoint{3.252089in}{2.860923in}}%
\pgfpathclose%
\pgfusepath{fill}%
\end{pgfscope}%
\begin{pgfscope}%
\pgfpathrectangle{\pgfqpoint{1.254980in}{0.150000in}}{\pgfqpoint{5.490039in}{5.490039in}}%
\pgfusepath{clip}%
\pgfsetbuttcap%
\pgfsetroundjoin%
\definecolor{currentfill}{rgb}{0.278791,0.062145,0.386592}%
\pgfsetfillcolor{currentfill}%
\pgfsetfillopacity{0.700000}%
\pgfsetlinewidth{0.000000pt}%
\definecolor{currentstroke}{rgb}{0.000000,0.000000,0.000000}%
\pgfsetstrokecolor{currentstroke}%
\pgfsetdash{}{0pt}%
\pgfpathmoveto{\pgfqpoint{3.382177in}{2.845822in}}%
\pgfpathlineto{\pgfqpoint{3.394679in}{2.840080in}}%
\pgfpathlineto{\pgfqpoint{3.407185in}{2.834383in}}%
\pgfpathlineto{\pgfqpoint{3.419695in}{2.828731in}}%
\pgfpathlineto{\pgfqpoint{3.432208in}{2.823124in}}%
\pgfpathlineto{\pgfqpoint{3.424700in}{2.815035in}}%
\pgfpathlineto{\pgfqpoint{3.417185in}{2.807008in}}%
\pgfpathlineto{\pgfqpoint{3.409665in}{2.799039in}}%
\pgfpathlineto{\pgfqpoint{3.402139in}{2.791129in}}%
\pgfpathlineto{\pgfqpoint{3.389615in}{2.796691in}}%
\pgfpathlineto{\pgfqpoint{3.377095in}{2.802297in}}%
\pgfpathlineto{\pgfqpoint{3.364578in}{2.807949in}}%
\pgfpathlineto{\pgfqpoint{3.352065in}{2.813646in}}%
\pgfpathlineto{\pgfqpoint{3.359602in}{2.821597in}}%
\pgfpathlineto{\pgfqpoint{3.367133in}{2.829609in}}%
\pgfpathlineto{\pgfqpoint{3.374658in}{2.837684in}}%
\pgfpathlineto{\pgfqpoint{3.382177in}{2.845822in}}%
\pgfpathclose%
\pgfusepath{fill}%
\end{pgfscope}%
\begin{pgfscope}%
\pgfpathrectangle{\pgfqpoint{1.254980in}{0.150000in}}{\pgfqpoint{5.490039in}{5.490039in}}%
\pgfusepath{clip}%
\pgfsetbuttcap%
\pgfsetroundjoin%
\definecolor{currentfill}{rgb}{0.277941,0.056324,0.381191}%
\pgfsetfillcolor{currentfill}%
\pgfsetfillopacity{0.700000}%
\pgfsetlinewidth{0.000000pt}%
\definecolor{currentstroke}{rgb}{0.000000,0.000000,0.000000}%
\pgfsetstrokecolor{currentstroke}%
\pgfsetdash{}{0pt}%
\pgfpathmoveto{\pgfqpoint{3.512232in}{2.833909in}}%
\pgfpathlineto{\pgfqpoint{3.524755in}{2.828461in}}%
\pgfpathlineto{\pgfqpoint{3.537281in}{2.823056in}}%
\pgfpathlineto{\pgfqpoint{3.549811in}{2.817693in}}%
\pgfpathlineto{\pgfqpoint{3.562346in}{2.812371in}}%
\pgfpathlineto{\pgfqpoint{3.554882in}{2.804143in}}%
\pgfpathlineto{\pgfqpoint{3.547412in}{2.795978in}}%
\pgfpathlineto{\pgfqpoint{3.539936in}{2.787874in}}%
\pgfpathlineto{\pgfqpoint{3.532455in}{2.779828in}}%
\pgfpathlineto{\pgfqpoint{3.519910in}{2.785092in}}%
\pgfpathlineto{\pgfqpoint{3.507369in}{2.790397in}}%
\pgfpathlineto{\pgfqpoint{3.494833in}{2.795745in}}%
\pgfpathlineto{\pgfqpoint{3.482300in}{2.801134in}}%
\pgfpathlineto{\pgfqpoint{3.489792in}{2.809233in}}%
\pgfpathlineto{\pgfqpoint{3.497278in}{2.817393in}}%
\pgfpathlineto{\pgfqpoint{3.504758in}{2.825618in}}%
\pgfpathlineto{\pgfqpoint{3.512232in}{2.833909in}}%
\pgfpathclose%
\pgfusepath{fill}%
\end{pgfscope}%
\begin{pgfscope}%
\pgfpathrectangle{\pgfqpoint{1.254980in}{0.150000in}}{\pgfqpoint{5.490039in}{5.490039in}}%
\pgfusepath{clip}%
\pgfsetbuttcap%
\pgfsetroundjoin%
\definecolor{currentfill}{rgb}{0.277941,0.056324,0.381191}%
\pgfsetfillcolor{currentfill}%
\pgfsetfillopacity{0.700000}%
\pgfsetlinewidth{0.000000pt}%
\definecolor{currentstroke}{rgb}{0.000000,0.000000,0.000000}%
\pgfsetstrokecolor{currentstroke}%
\pgfsetdash{}{0pt}%
\pgfpathmoveto{\pgfqpoint{3.982173in}{2.830063in}}%
\pgfpathlineto{\pgfqpoint{3.994778in}{2.825320in}}%
\pgfpathlineto{\pgfqpoint{4.007388in}{2.820609in}}%
\pgfpathlineto{\pgfqpoint{4.020002in}{2.815933in}}%
\pgfpathlineto{\pgfqpoint{4.032622in}{2.811290in}}%
\pgfpathlineto{\pgfqpoint{4.025314in}{2.802468in}}%
\pgfpathlineto{\pgfqpoint{4.018002in}{2.793735in}}%
\pgfpathlineto{\pgfqpoint{4.010684in}{2.785086in}}%
\pgfpathlineto{\pgfqpoint{4.003362in}{2.776519in}}%
\pgfpathlineto{\pgfqpoint{3.990733in}{2.781054in}}%
\pgfpathlineto{\pgfqpoint{3.978108in}{2.785623in}}%
\pgfpathlineto{\pgfqpoint{3.965487in}{2.790226in}}%
\pgfpathlineto{\pgfqpoint{3.952872in}{2.794862in}}%
\pgfpathlineto{\pgfqpoint{3.960205in}{2.803532in}}%
\pgfpathlineto{\pgfqpoint{3.967532in}{2.812287in}}%
\pgfpathlineto{\pgfqpoint{3.974855in}{2.821130in}}%
\pgfpathlineto{\pgfqpoint{3.982173in}{2.830063in}}%
\pgfpathclose%
\pgfusepath{fill}%
\end{pgfscope}%
\begin{pgfscope}%
\pgfpathrectangle{\pgfqpoint{1.254980in}{0.150000in}}{\pgfqpoint{5.490039in}{5.490039in}}%
\pgfusepath{clip}%
\pgfsetbuttcap%
\pgfsetroundjoin%
\definecolor{currentfill}{rgb}{0.277941,0.056324,0.381191}%
\pgfsetfillcolor{currentfill}%
\pgfsetfillopacity{0.700000}%
\pgfsetlinewidth{0.000000pt}%
\definecolor{currentstroke}{rgb}{0.000000,0.000000,0.000000}%
\pgfsetstrokecolor{currentstroke}%
\pgfsetdash{}{0pt}%
\pgfpathmoveto{\pgfqpoint{3.642283in}{2.824802in}}%
\pgfpathlineto{\pgfqpoint{3.654828in}{2.819613in}}%
\pgfpathlineto{\pgfqpoint{3.667377in}{2.814462in}}%
\pgfpathlineto{\pgfqpoint{3.679930in}{2.809351in}}%
\pgfpathlineto{\pgfqpoint{3.692488in}{2.804279in}}%
\pgfpathlineto{\pgfqpoint{3.685067in}{2.795924in}}%
\pgfpathlineto{\pgfqpoint{3.677641in}{2.787636in}}%
\pgfpathlineto{\pgfqpoint{3.670209in}{2.779412in}}%
\pgfpathlineto{\pgfqpoint{3.662772in}{2.771250in}}%
\pgfpathlineto{\pgfqpoint{3.650203in}{2.776252in}}%
\pgfpathlineto{\pgfqpoint{3.637639in}{2.781293in}}%
\pgfpathlineto{\pgfqpoint{3.625080in}{2.786373in}}%
\pgfpathlineto{\pgfqpoint{3.612525in}{2.791492in}}%
\pgfpathlineto{\pgfqpoint{3.619972in}{2.799720in}}%
\pgfpathlineto{\pgfqpoint{3.627415in}{2.808013in}}%
\pgfpathlineto{\pgfqpoint{3.634852in}{2.816373in}}%
\pgfpathlineto{\pgfqpoint{3.642283in}{2.824802in}}%
\pgfpathclose%
\pgfusepath{fill}%
\end{pgfscope}%
\begin{pgfscope}%
\pgfpathrectangle{\pgfqpoint{1.254980in}{0.150000in}}{\pgfqpoint{5.490039in}{5.490039in}}%
\pgfusepath{clip}%
\pgfsetbuttcap%
\pgfsetroundjoin%
\definecolor{currentfill}{rgb}{0.279566,0.067836,0.391917}%
\pgfsetfillcolor{currentfill}%
\pgfsetfillopacity{0.700000}%
\pgfsetlinewidth{0.000000pt}%
\definecolor{currentstroke}{rgb}{0.000000,0.000000,0.000000}%
\pgfsetstrokecolor{currentstroke}%
\pgfsetdash{}{0pt}%
\pgfpathmoveto{\pgfqpoint{4.322048in}{2.849183in}}%
\pgfpathlineto{\pgfqpoint{4.334721in}{2.844677in}}%
\pgfpathlineto{\pgfqpoint{4.347400in}{2.840202in}}%
\pgfpathlineto{\pgfqpoint{4.360083in}{2.835756in}}%
\pgfpathlineto{\pgfqpoint{4.372772in}{2.831340in}}%
\pgfpathlineto{\pgfqpoint{4.365569in}{2.821892in}}%
\pgfpathlineto{\pgfqpoint{4.358362in}{2.812568in}}%
\pgfpathlineto{\pgfqpoint{4.351151in}{2.803362in}}%
\pgfpathlineto{\pgfqpoint{4.343937in}{2.794269in}}%
\pgfpathlineto{\pgfqpoint{4.331237in}{2.798541in}}%
\pgfpathlineto{\pgfqpoint{4.318543in}{2.802842in}}%
\pgfpathlineto{\pgfqpoint{4.305853in}{2.807173in}}%
\pgfpathlineto{\pgfqpoint{4.293169in}{2.811533in}}%
\pgfpathlineto{\pgfqpoint{4.300395in}{2.820766in}}%
\pgfpathlineto{\pgfqpoint{4.307617in}{2.830115in}}%
\pgfpathlineto{\pgfqpoint{4.314834in}{2.839586in}}%
\pgfpathlineto{\pgfqpoint{4.322048in}{2.849183in}}%
\pgfpathclose%
\pgfusepath{fill}%
\end{pgfscope}%
\begin{pgfscope}%
\pgfpathrectangle{\pgfqpoint{1.254980in}{0.150000in}}{\pgfqpoint{5.490039in}{5.490039in}}%
\pgfusepath{clip}%
\pgfsetbuttcap%
\pgfsetroundjoin%
\definecolor{currentfill}{rgb}{0.277941,0.056324,0.381191}%
\pgfsetfillcolor{currentfill}%
\pgfsetfillopacity{0.700000}%
\pgfsetlinewidth{0.000000pt}%
\definecolor{currentstroke}{rgb}{0.000000,0.000000,0.000000}%
\pgfsetstrokecolor{currentstroke}%
\pgfsetdash{}{0pt}%
\pgfpathmoveto{\pgfqpoint{4.112290in}{2.828810in}}%
\pgfpathlineto{\pgfqpoint{4.124923in}{2.824210in}}%
\pgfpathlineto{\pgfqpoint{4.137562in}{2.819643in}}%
\pgfpathlineto{\pgfqpoint{4.150206in}{2.815106in}}%
\pgfpathlineto{\pgfqpoint{4.162854in}{2.810602in}}%
\pgfpathlineto{\pgfqpoint{4.155586in}{2.801639in}}%
\pgfpathlineto{\pgfqpoint{4.148313in}{2.792773in}}%
\pgfpathlineto{\pgfqpoint{4.141036in}{2.784000in}}%
\pgfpathlineto{\pgfqpoint{4.133755in}{2.775317in}}%
\pgfpathlineto{\pgfqpoint{4.121095in}{2.779702in}}%
\pgfpathlineto{\pgfqpoint{4.108441in}{2.784118in}}%
\pgfpathlineto{\pgfqpoint{4.095792in}{2.788565in}}%
\pgfpathlineto{\pgfqpoint{4.083148in}{2.793045in}}%
\pgfpathlineto{\pgfqpoint{4.090440in}{2.801844in}}%
\pgfpathlineto{\pgfqpoint{4.097728in}{2.810735in}}%
\pgfpathlineto{\pgfqpoint{4.105011in}{2.819722in}}%
\pgfpathlineto{\pgfqpoint{4.112290in}{2.828810in}}%
\pgfpathclose%
\pgfusepath{fill}%
\end{pgfscope}%
\begin{pgfscope}%
\pgfpathrectangle{\pgfqpoint{1.254980in}{0.150000in}}{\pgfqpoint{5.490039in}{5.490039in}}%
\pgfusepath{clip}%
\pgfsetbuttcap%
\pgfsetroundjoin%
\definecolor{currentfill}{rgb}{0.277018,0.050344,0.375715}%
\pgfsetfillcolor{currentfill}%
\pgfsetfillopacity{0.700000}%
\pgfsetlinewidth{0.000000pt}%
\definecolor{currentstroke}{rgb}{0.000000,0.000000,0.000000}%
\pgfsetstrokecolor{currentstroke}%
\pgfsetdash{}{0pt}%
\pgfpathmoveto{\pgfqpoint{3.772352in}{2.818176in}}%
\pgfpathlineto{\pgfqpoint{3.784921in}{2.813210in}}%
\pgfpathlineto{\pgfqpoint{3.797496in}{2.808280in}}%
\pgfpathlineto{\pgfqpoint{3.810074in}{2.803388in}}%
\pgfpathlineto{\pgfqpoint{3.822658in}{2.798532in}}%
\pgfpathlineto{\pgfqpoint{3.815279in}{2.790059in}}%
\pgfpathlineto{\pgfqpoint{3.807895in}{2.781657in}}%
\pgfpathlineto{\pgfqpoint{3.800505in}{2.773324in}}%
\pgfpathlineto{\pgfqpoint{3.793111in}{2.765057in}}%
\pgfpathlineto{\pgfqpoint{3.780517in}{2.769830in}}%
\pgfpathlineto{\pgfqpoint{3.767928in}{2.774640in}}%
\pgfpathlineto{\pgfqpoint{3.755343in}{2.779486in}}%
\pgfpathlineto{\pgfqpoint{3.742763in}{2.784370in}}%
\pgfpathlineto{\pgfqpoint{3.750169in}{2.792715in}}%
\pgfpathlineto{\pgfqpoint{3.757568in}{2.801130in}}%
\pgfpathlineto{\pgfqpoint{3.764963in}{2.809616in}}%
\pgfpathlineto{\pgfqpoint{3.772352in}{2.818176in}}%
\pgfpathclose%
\pgfusepath{fill}%
\end{pgfscope}%
\begin{pgfscope}%
\pgfpathrectangle{\pgfqpoint{1.254980in}{0.150000in}}{\pgfqpoint{5.490039in}{5.490039in}}%
\pgfusepath{clip}%
\pgfsetbuttcap%
\pgfsetroundjoin%
\definecolor{currentfill}{rgb}{0.280267,0.073417,0.397163}%
\pgfsetfillcolor{currentfill}%
\pgfsetfillopacity{0.700000}%
\pgfsetlinewidth{0.000000pt}%
\definecolor{currentstroke}{rgb}{0.000000,0.000000,0.000000}%
\pgfsetstrokecolor{currentstroke}%
\pgfsetdash{}{0pt}%
\pgfpathmoveto{\pgfqpoint{4.452313in}{2.852454in}}%
\pgfpathlineto{\pgfqpoint{4.465017in}{2.848026in}}%
\pgfpathlineto{\pgfqpoint{4.477727in}{2.843626in}}%
\pgfpathlineto{\pgfqpoint{4.490441in}{2.839255in}}%
\pgfpathlineto{\pgfqpoint{4.503161in}{2.834912in}}%
\pgfpathlineto{\pgfqpoint{4.495994in}{2.825248in}}%
\pgfpathlineto{\pgfqpoint{4.488823in}{2.815720in}}%
\pgfpathlineto{\pgfqpoint{4.481649in}{2.806324in}}%
\pgfpathlineto{\pgfqpoint{4.474472in}{2.797054in}}%
\pgfpathlineto{\pgfqpoint{4.461741in}{2.801239in}}%
\pgfpathlineto{\pgfqpoint{4.449015in}{2.805453in}}%
\pgfpathlineto{\pgfqpoint{4.436295in}{2.809696in}}%
\pgfpathlineto{\pgfqpoint{4.423579in}{2.813967in}}%
\pgfpathlineto{\pgfqpoint{4.430768in}{2.823389in}}%
\pgfpathlineto{\pgfqpoint{4.437953in}{2.832941in}}%
\pgfpathlineto{\pgfqpoint{4.445135in}{2.842628in}}%
\pgfpathlineto{\pgfqpoint{4.452313in}{2.852454in}}%
\pgfpathclose%
\pgfusepath{fill}%
\end{pgfscope}%
\begin{pgfscope}%
\pgfpathrectangle{\pgfqpoint{1.254980in}{0.150000in}}{\pgfqpoint{5.490039in}{5.490039in}}%
\pgfusepath{clip}%
\pgfsetbuttcap%
\pgfsetroundjoin%
\definecolor{currentfill}{rgb}{0.277018,0.050344,0.375715}%
\pgfsetfillcolor{currentfill}%
\pgfsetfillopacity{0.700000}%
\pgfsetlinewidth{0.000000pt}%
\definecolor{currentstroke}{rgb}{0.000000,0.000000,0.000000}%
\pgfsetstrokecolor{currentstroke}%
\pgfsetdash{}{0pt}%
\pgfpathmoveto{\pgfqpoint{3.902459in}{2.813750in}}%
\pgfpathlineto{\pgfqpoint{3.915055in}{2.808976in}}%
\pgfpathlineto{\pgfqpoint{3.927656in}{2.804237in}}%
\pgfpathlineto{\pgfqpoint{3.940262in}{2.799532in}}%
\pgfpathlineto{\pgfqpoint{3.952872in}{2.794862in}}%
\pgfpathlineto{\pgfqpoint{3.945535in}{2.786273in}}%
\pgfpathlineto{\pgfqpoint{3.938192in}{2.777761in}}%
\pgfpathlineto{\pgfqpoint{3.930845in}{2.769324in}}%
\pgfpathlineto{\pgfqpoint{3.923492in}{2.760958in}}%
\pgfpathlineto{\pgfqpoint{3.910871in}{2.765533in}}%
\pgfpathlineto{\pgfqpoint{3.898255in}{2.770142in}}%
\pgfpathlineto{\pgfqpoint{3.885644in}{2.774786in}}%
\pgfpathlineto{\pgfqpoint{3.873037in}{2.779464in}}%
\pgfpathlineto{\pgfqpoint{3.880400in}{2.787921in}}%
\pgfpathlineto{\pgfqpoint{3.887758in}{2.796452in}}%
\pgfpathlineto{\pgfqpoint{3.895111in}{2.805061in}}%
\pgfpathlineto{\pgfqpoint{3.902459in}{2.813750in}}%
\pgfpathclose%
\pgfusepath{fill}%
\end{pgfscope}%
\begin{pgfscope}%
\pgfpathrectangle{\pgfqpoint{1.254980in}{0.150000in}}{\pgfqpoint{5.490039in}{5.490039in}}%
\pgfusepath{clip}%
\pgfsetbuttcap%
\pgfsetroundjoin%
\definecolor{currentfill}{rgb}{0.279566,0.067836,0.391917}%
\pgfsetfillcolor{currentfill}%
\pgfsetfillopacity{0.700000}%
\pgfsetlinewidth{0.000000pt}%
\definecolor{currentstroke}{rgb}{0.000000,0.000000,0.000000}%
\pgfsetstrokecolor{currentstroke}%
\pgfsetdash{}{0pt}%
\pgfpathmoveto{\pgfqpoint{3.302049in}{2.836901in}}%
\pgfpathlineto{\pgfqpoint{3.314548in}{2.831016in}}%
\pgfpathlineto{\pgfqpoint{3.327050in}{2.825179in}}%
\pgfpathlineto{\pgfqpoint{3.339556in}{2.819390in}}%
\pgfpathlineto{\pgfqpoint{3.352065in}{2.813646in}}%
\pgfpathlineto{\pgfqpoint{3.344522in}{2.805754in}}%
\pgfpathlineto{\pgfqpoint{3.336973in}{2.797919in}}%
\pgfpathlineto{\pgfqpoint{3.329418in}{2.790140in}}%
\pgfpathlineto{\pgfqpoint{3.321857in}{2.782415in}}%
\pgfpathlineto{\pgfqpoint{3.309337in}{2.788125in}}%
\pgfpathlineto{\pgfqpoint{3.296820in}{2.793883in}}%
\pgfpathlineto{\pgfqpoint{3.284306in}{2.799687in}}%
\pgfpathlineto{\pgfqpoint{3.271797in}{2.805538in}}%
\pgfpathlineto{\pgfqpoint{3.279369in}{2.813291in}}%
\pgfpathlineto{\pgfqpoint{3.286935in}{2.821102in}}%
\pgfpathlineto{\pgfqpoint{3.294495in}{2.828971in}}%
\pgfpathlineto{\pgfqpoint{3.302049in}{2.836901in}}%
\pgfpathclose%
\pgfusepath{fill}%
\end{pgfscope}%
\begin{pgfscope}%
\pgfpathrectangle{\pgfqpoint{1.254980in}{0.150000in}}{\pgfqpoint{5.490039in}{5.490039in}}%
\pgfusepath{clip}%
\pgfsetbuttcap%
\pgfsetroundjoin%
\definecolor{currentfill}{rgb}{0.278791,0.062145,0.386592}%
\pgfsetfillcolor{currentfill}%
\pgfsetfillopacity{0.700000}%
\pgfsetlinewidth{0.000000pt}%
\definecolor{currentstroke}{rgb}{0.000000,0.000000,0.000000}%
\pgfsetstrokecolor{currentstroke}%
\pgfsetdash{}{0pt}%
\pgfpathmoveto{\pgfqpoint{4.242485in}{2.829277in}}%
\pgfpathlineto{\pgfqpoint{4.255149in}{2.824795in}}%
\pgfpathlineto{\pgfqpoint{4.267817in}{2.820344in}}%
\pgfpathlineto{\pgfqpoint{4.280491in}{2.815924in}}%
\pgfpathlineto{\pgfqpoint{4.293169in}{2.811533in}}%
\pgfpathlineto{\pgfqpoint{4.285940in}{2.802413in}}%
\pgfpathlineto{\pgfqpoint{4.278706in}{2.793401in}}%
\pgfpathlineto{\pgfqpoint{4.271468in}{2.784492in}}%
\pgfpathlineto{\pgfqpoint{4.264226in}{2.775683in}}%
\pgfpathlineto{\pgfqpoint{4.251537in}{2.779941in}}%
\pgfpathlineto{\pgfqpoint{4.238852in}{2.784229in}}%
\pgfpathlineto{\pgfqpoint{4.226173in}{2.788548in}}%
\pgfpathlineto{\pgfqpoint{4.213499in}{2.792897in}}%
\pgfpathlineto{\pgfqpoint{4.220752in}{2.801834in}}%
\pgfpathlineto{\pgfqpoint{4.228001in}{2.810873in}}%
\pgfpathlineto{\pgfqpoint{4.235245in}{2.820020in}}%
\pgfpathlineto{\pgfqpoint{4.242485in}{2.829277in}}%
\pgfpathclose%
\pgfusepath{fill}%
\end{pgfscope}%
\begin{pgfscope}%
\pgfpathrectangle{\pgfqpoint{1.254980in}{0.150000in}}{\pgfqpoint{5.490039in}{5.490039in}}%
\pgfusepath{clip}%
\pgfsetbuttcap%
\pgfsetroundjoin%
\definecolor{currentfill}{rgb}{0.277941,0.056324,0.381191}%
\pgfsetfillcolor{currentfill}%
\pgfsetfillopacity{0.700000}%
\pgfsetlinewidth{0.000000pt}%
\definecolor{currentstroke}{rgb}{0.000000,0.000000,0.000000}%
\pgfsetstrokecolor{currentstroke}%
\pgfsetdash{}{0pt}%
\pgfpathmoveto{\pgfqpoint{3.432208in}{2.823124in}}%
\pgfpathlineto{\pgfqpoint{3.444725in}{2.817561in}}%
\pgfpathlineto{\pgfqpoint{3.457246in}{2.812042in}}%
\pgfpathlineto{\pgfqpoint{3.469771in}{2.806567in}}%
\pgfpathlineto{\pgfqpoint{3.482300in}{2.801134in}}%
\pgfpathlineto{\pgfqpoint{3.474802in}{2.793096in}}%
\pgfpathlineto{\pgfqpoint{3.467299in}{2.785115in}}%
\pgfpathlineto{\pgfqpoint{3.459790in}{2.777191in}}%
\pgfpathlineto{\pgfqpoint{3.452275in}{2.769321in}}%
\pgfpathlineto{\pgfqpoint{3.439735in}{2.774708in}}%
\pgfpathlineto{\pgfqpoint{3.427199in}{2.780138in}}%
\pgfpathlineto{\pgfqpoint{3.414667in}{2.785611in}}%
\pgfpathlineto{\pgfqpoint{3.402139in}{2.791129in}}%
\pgfpathlineto{\pgfqpoint{3.409665in}{2.799039in}}%
\pgfpathlineto{\pgfqpoint{3.417185in}{2.807008in}}%
\pgfpathlineto{\pgfqpoint{3.424700in}{2.815035in}}%
\pgfpathlineto{\pgfqpoint{3.432208in}{2.823124in}}%
\pgfpathclose%
\pgfusepath{fill}%
\end{pgfscope}%
\begin{pgfscope}%
\pgfpathrectangle{\pgfqpoint{1.254980in}{0.150000in}}{\pgfqpoint{5.490039in}{5.490039in}}%
\pgfusepath{clip}%
\pgfsetbuttcap%
\pgfsetroundjoin%
\definecolor{currentfill}{rgb}{0.277018,0.050344,0.375715}%
\pgfsetfillcolor{currentfill}%
\pgfsetfillopacity{0.700000}%
\pgfsetlinewidth{0.000000pt}%
\definecolor{currentstroke}{rgb}{0.000000,0.000000,0.000000}%
\pgfsetstrokecolor{currentstroke}%
\pgfsetdash{}{0pt}%
\pgfpathmoveto{\pgfqpoint{3.562346in}{2.812371in}}%
\pgfpathlineto{\pgfqpoint{3.574884in}{2.807091in}}%
\pgfpathlineto{\pgfqpoint{3.587427in}{2.801851in}}%
\pgfpathlineto{\pgfqpoint{3.599973in}{2.796652in}}%
\pgfpathlineto{\pgfqpoint{3.612525in}{2.791492in}}%
\pgfpathlineto{\pgfqpoint{3.605071in}{2.783327in}}%
\pgfpathlineto{\pgfqpoint{3.597612in}{2.775221in}}%
\pgfpathlineto{\pgfqpoint{3.590147in}{2.767173in}}%
\pgfpathlineto{\pgfqpoint{3.582677in}{2.759181in}}%
\pgfpathlineto{\pgfqpoint{3.570115in}{2.764282in}}%
\pgfpathlineto{\pgfqpoint{3.557558in}{2.769424in}}%
\pgfpathlineto{\pgfqpoint{3.545004in}{2.774606in}}%
\pgfpathlineto{\pgfqpoint{3.532455in}{2.779828in}}%
\pgfpathlineto{\pgfqpoint{3.539936in}{2.787874in}}%
\pgfpathlineto{\pgfqpoint{3.547412in}{2.795978in}}%
\pgfpathlineto{\pgfqpoint{3.554882in}{2.804143in}}%
\pgfpathlineto{\pgfqpoint{3.562346in}{2.812371in}}%
\pgfpathclose%
\pgfusepath{fill}%
\end{pgfscope}%
\begin{pgfscope}%
\pgfpathrectangle{\pgfqpoint{1.254980in}{0.150000in}}{\pgfqpoint{5.490039in}{5.490039in}}%
\pgfusepath{clip}%
\pgfsetbuttcap%
\pgfsetroundjoin%
\definecolor{currentfill}{rgb}{0.277018,0.050344,0.375715}%
\pgfsetfillcolor{currentfill}%
\pgfsetfillopacity{0.700000}%
\pgfsetlinewidth{0.000000pt}%
\definecolor{currentstroke}{rgb}{0.000000,0.000000,0.000000}%
\pgfsetstrokecolor{currentstroke}%
\pgfsetdash{}{0pt}%
\pgfpathmoveto{\pgfqpoint{4.032622in}{2.811290in}}%
\pgfpathlineto{\pgfqpoint{4.045246in}{2.806679in}}%
\pgfpathlineto{\pgfqpoint{4.057875in}{2.802102in}}%
\pgfpathlineto{\pgfqpoint{4.070509in}{2.797558in}}%
\pgfpathlineto{\pgfqpoint{4.083148in}{2.793045in}}%
\pgfpathlineto{\pgfqpoint{4.075851in}{2.784336in}}%
\pgfpathlineto{\pgfqpoint{4.068549in}{2.775712in}}%
\pgfpathlineto{\pgfqpoint{4.061243in}{2.767170in}}%
\pgfpathlineto{\pgfqpoint{4.053931in}{2.758706in}}%
\pgfpathlineto{\pgfqpoint{4.041282in}{2.763110in}}%
\pgfpathlineto{\pgfqpoint{4.028637in}{2.767547in}}%
\pgfpathlineto{\pgfqpoint{4.015997in}{2.772016in}}%
\pgfpathlineto{\pgfqpoint{4.003362in}{2.776519in}}%
\pgfpathlineto{\pgfqpoint{4.010684in}{2.785086in}}%
\pgfpathlineto{\pgfqpoint{4.018002in}{2.793735in}}%
\pgfpathlineto{\pgfqpoint{4.025314in}{2.802468in}}%
\pgfpathlineto{\pgfqpoint{4.032622in}{2.811290in}}%
\pgfpathclose%
\pgfusepath{fill}%
\end{pgfscope}%
\begin{pgfscope}%
\pgfpathrectangle{\pgfqpoint{1.254980in}{0.150000in}}{\pgfqpoint{5.490039in}{5.490039in}}%
\pgfusepath{clip}%
\pgfsetbuttcap%
\pgfsetroundjoin%
\definecolor{currentfill}{rgb}{0.277018,0.050344,0.375715}%
\pgfsetfillcolor{currentfill}%
\pgfsetfillopacity{0.700000}%
\pgfsetlinewidth{0.000000pt}%
\definecolor{currentstroke}{rgb}{0.000000,0.000000,0.000000}%
\pgfsetstrokecolor{currentstroke}%
\pgfsetdash{}{0pt}%
\pgfpathmoveto{\pgfqpoint{3.692488in}{2.804279in}}%
\pgfpathlineto{\pgfqpoint{3.705050in}{2.799244in}}%
\pgfpathlineto{\pgfqpoint{3.717617in}{2.794248in}}%
\pgfpathlineto{\pgfqpoint{3.730188in}{2.789290in}}%
\pgfpathlineto{\pgfqpoint{3.742763in}{2.784370in}}%
\pgfpathlineto{\pgfqpoint{3.735353in}{2.776090in}}%
\pgfpathlineto{\pgfqpoint{3.727937in}{2.767874in}}%
\pgfpathlineto{\pgfqpoint{3.720516in}{2.759719in}}%
\pgfpathlineto{\pgfqpoint{3.713089in}{2.751623in}}%
\pgfpathlineto{\pgfqpoint{3.700503in}{2.756473in}}%
\pgfpathlineto{\pgfqpoint{3.687921in}{2.761361in}}%
\pgfpathlineto{\pgfqpoint{3.675344in}{2.766286in}}%
\pgfpathlineto{\pgfqpoint{3.662772in}{2.771250in}}%
\pgfpathlineto{\pgfqpoint{3.670209in}{2.779412in}}%
\pgfpathlineto{\pgfqpoint{3.677641in}{2.787636in}}%
\pgfpathlineto{\pgfqpoint{3.685067in}{2.795924in}}%
\pgfpathlineto{\pgfqpoint{3.692488in}{2.804279in}}%
\pgfpathclose%
\pgfusepath{fill}%
\end{pgfscope}%
\begin{pgfscope}%
\pgfpathrectangle{\pgfqpoint{1.254980in}{0.150000in}}{\pgfqpoint{5.490039in}{5.490039in}}%
\pgfusepath{clip}%
\pgfsetbuttcap%
\pgfsetroundjoin%
\definecolor{currentfill}{rgb}{0.279566,0.067836,0.391917}%
\pgfsetfillcolor{currentfill}%
\pgfsetfillopacity{0.700000}%
\pgfsetlinewidth{0.000000pt}%
\definecolor{currentstroke}{rgb}{0.000000,0.000000,0.000000}%
\pgfsetstrokecolor{currentstroke}%
\pgfsetdash{}{0pt}%
\pgfpathmoveto{\pgfqpoint{4.372772in}{2.831340in}}%
\pgfpathlineto{\pgfqpoint{4.385466in}{2.826953in}}%
\pgfpathlineto{\pgfqpoint{4.398165in}{2.822595in}}%
\pgfpathlineto{\pgfqpoint{4.410870in}{2.818266in}}%
\pgfpathlineto{\pgfqpoint{4.423579in}{2.813967in}}%
\pgfpathlineto{\pgfqpoint{4.416387in}{2.804669in}}%
\pgfpathlineto{\pgfqpoint{4.409191in}{2.795491in}}%
\pgfpathlineto{\pgfqpoint{4.401992in}{2.786428in}}%
\pgfpathlineto{\pgfqpoint{4.394788in}{2.777476in}}%
\pgfpathlineto{\pgfqpoint{4.382067in}{2.781631in}}%
\pgfpathlineto{\pgfqpoint{4.369352in}{2.785815in}}%
\pgfpathlineto{\pgfqpoint{4.356642in}{2.790027in}}%
\pgfpathlineto{\pgfqpoint{4.343937in}{2.794269in}}%
\pgfpathlineto{\pgfqpoint{4.351151in}{2.803362in}}%
\pgfpathlineto{\pgfqpoint{4.358362in}{2.812568in}}%
\pgfpathlineto{\pgfqpoint{4.365569in}{2.821892in}}%
\pgfpathlineto{\pgfqpoint{4.372772in}{2.831340in}}%
\pgfpathclose%
\pgfusepath{fill}%
\end{pgfscope}%
\begin{pgfscope}%
\pgfpathrectangle{\pgfqpoint{1.254980in}{0.150000in}}{\pgfqpoint{5.490039in}{5.490039in}}%
\pgfusepath{clip}%
\pgfsetbuttcap%
\pgfsetroundjoin%
\definecolor{currentfill}{rgb}{0.277018,0.050344,0.375715}%
\pgfsetfillcolor{currentfill}%
\pgfsetfillopacity{0.700000}%
\pgfsetlinewidth{0.000000pt}%
\definecolor{currentstroke}{rgb}{0.000000,0.000000,0.000000}%
\pgfsetstrokecolor{currentstroke}%
\pgfsetdash{}{0pt}%
\pgfpathmoveto{\pgfqpoint{3.822658in}{2.798532in}}%
\pgfpathlineto{\pgfqpoint{3.835245in}{2.793711in}}%
\pgfpathlineto{\pgfqpoint{3.847838in}{2.788927in}}%
\pgfpathlineto{\pgfqpoint{3.860435in}{2.784178in}}%
\pgfpathlineto{\pgfqpoint{3.873037in}{2.779464in}}%
\pgfpathlineto{\pgfqpoint{3.865669in}{2.771079in}}%
\pgfpathlineto{\pgfqpoint{3.858295in}{2.762762in}}%
\pgfpathlineto{\pgfqpoint{3.850916in}{2.754510in}}%
\pgfpathlineto{\pgfqpoint{3.843532in}{2.746321in}}%
\pgfpathlineto{\pgfqpoint{3.830920in}{2.750952in}}%
\pgfpathlineto{\pgfqpoint{3.818312in}{2.755618in}}%
\pgfpathlineto{\pgfqpoint{3.805709in}{2.760319in}}%
\pgfpathlineto{\pgfqpoint{3.793111in}{2.765057in}}%
\pgfpathlineto{\pgfqpoint{3.800505in}{2.773324in}}%
\pgfpathlineto{\pgfqpoint{3.807895in}{2.781657in}}%
\pgfpathlineto{\pgfqpoint{3.815279in}{2.790059in}}%
\pgfpathlineto{\pgfqpoint{3.822658in}{2.798532in}}%
\pgfpathclose%
\pgfusepath{fill}%
\end{pgfscope}%
\begin{pgfscope}%
\pgfpathrectangle{\pgfqpoint{1.254980in}{0.150000in}}{\pgfqpoint{5.490039in}{5.490039in}}%
\pgfusepath{clip}%
\pgfsetbuttcap%
\pgfsetroundjoin%
\definecolor{currentfill}{rgb}{0.277941,0.056324,0.381191}%
\pgfsetfillcolor{currentfill}%
\pgfsetfillopacity{0.700000}%
\pgfsetlinewidth{0.000000pt}%
\definecolor{currentstroke}{rgb}{0.000000,0.000000,0.000000}%
\pgfsetstrokecolor{currentstroke}%
\pgfsetdash{}{0pt}%
\pgfpathmoveto{\pgfqpoint{4.162854in}{2.810602in}}%
\pgfpathlineto{\pgfqpoint{4.175508in}{2.806129in}}%
\pgfpathlineto{\pgfqpoint{4.188166in}{2.801687in}}%
\pgfpathlineto{\pgfqpoint{4.200830in}{2.797276in}}%
\pgfpathlineto{\pgfqpoint{4.213499in}{2.792897in}}%
\pgfpathlineto{\pgfqpoint{4.206242in}{2.784058in}}%
\pgfpathlineto{\pgfqpoint{4.198980in}{2.775314in}}%
\pgfpathlineto{\pgfqpoint{4.191713in}{2.766660in}}%
\pgfpathlineto{\pgfqpoint{4.184443in}{2.758093in}}%
\pgfpathlineto{\pgfqpoint{4.171763in}{2.762353in}}%
\pgfpathlineto{\pgfqpoint{4.159088in}{2.766643in}}%
\pgfpathlineto{\pgfqpoint{4.146419in}{2.770965in}}%
\pgfpathlineto{\pgfqpoint{4.133755in}{2.775317in}}%
\pgfpathlineto{\pgfqpoint{4.141036in}{2.784000in}}%
\pgfpathlineto{\pgfqpoint{4.148313in}{2.792773in}}%
\pgfpathlineto{\pgfqpoint{4.155586in}{2.801639in}}%
\pgfpathlineto{\pgfqpoint{4.162854in}{2.810602in}}%
\pgfpathclose%
\pgfusepath{fill}%
\end{pgfscope}%
\begin{pgfscope}%
\pgfpathrectangle{\pgfqpoint{1.254980in}{0.150000in}}{\pgfqpoint{5.490039in}{5.490039in}}%
\pgfusepath{clip}%
\pgfsetbuttcap%
\pgfsetroundjoin%
\definecolor{currentfill}{rgb}{0.279566,0.067836,0.391917}%
\pgfsetfillcolor{currentfill}%
\pgfsetfillopacity{0.700000}%
\pgfsetlinewidth{0.000000pt}%
\definecolor{currentstroke}{rgb}{0.000000,0.000000,0.000000}%
\pgfsetstrokecolor{currentstroke}%
\pgfsetdash{}{0pt}%
\pgfpathmoveto{\pgfqpoint{3.221792in}{2.829430in}}%
\pgfpathlineto{\pgfqpoint{3.234288in}{2.823383in}}%
\pgfpathlineto{\pgfqpoint{3.246787in}{2.817386in}}%
\pgfpathlineto{\pgfqpoint{3.259290in}{2.811438in}}%
\pgfpathlineto{\pgfqpoint{3.271797in}{2.805538in}}%
\pgfpathlineto{\pgfqpoint{3.264218in}{2.797842in}}%
\pgfpathlineto{\pgfqpoint{3.256633in}{2.790200in}}%
\pgfpathlineto{\pgfqpoint{3.249042in}{2.782613in}}%
\pgfpathlineto{\pgfqpoint{3.241445in}{2.775078in}}%
\pgfpathlineto{\pgfqpoint{3.228927in}{2.780957in}}%
\pgfpathlineto{\pgfqpoint{3.216413in}{2.786885in}}%
\pgfpathlineto{\pgfqpoint{3.203902in}{2.792862in}}%
\pgfpathlineto{\pgfqpoint{3.191394in}{2.798888in}}%
\pgfpathlineto{\pgfqpoint{3.199003in}{2.806439in}}%
\pgfpathlineto{\pgfqpoint{3.206606in}{2.814045in}}%
\pgfpathlineto{\pgfqpoint{3.214202in}{2.821708in}}%
\pgfpathlineto{\pgfqpoint{3.221792in}{2.829430in}}%
\pgfpathclose%
\pgfusepath{fill}%
\end{pgfscope}%
\begin{pgfscope}%
\pgfpathrectangle{\pgfqpoint{1.254980in}{0.150000in}}{\pgfqpoint{5.490039in}{5.490039in}}%
\pgfusepath{clip}%
\pgfsetbuttcap%
\pgfsetroundjoin%
\definecolor{currentfill}{rgb}{0.278791,0.062145,0.386592}%
\pgfsetfillcolor{currentfill}%
\pgfsetfillopacity{0.700000}%
\pgfsetlinewidth{0.000000pt}%
\definecolor{currentstroke}{rgb}{0.000000,0.000000,0.000000}%
\pgfsetstrokecolor{currentstroke}%
\pgfsetdash{}{0pt}%
\pgfpathmoveto{\pgfqpoint{3.352065in}{2.813646in}}%
\pgfpathlineto{\pgfqpoint{3.364578in}{2.807949in}}%
\pgfpathlineto{\pgfqpoint{3.377095in}{2.802297in}}%
\pgfpathlineto{\pgfqpoint{3.389615in}{2.796691in}}%
\pgfpathlineto{\pgfqpoint{3.402139in}{2.791129in}}%
\pgfpathlineto{\pgfqpoint{3.394607in}{2.783274in}}%
\pgfpathlineto{\pgfqpoint{3.387069in}{2.775473in}}%
\pgfpathlineto{\pgfqpoint{3.379526in}{2.767726in}}%
\pgfpathlineto{\pgfqpoint{3.371976in}{2.760029in}}%
\pgfpathlineto{\pgfqpoint{3.359441in}{2.765558in}}%
\pgfpathlineto{\pgfqpoint{3.346909in}{2.771132in}}%
\pgfpathlineto{\pgfqpoint{3.334381in}{2.776750in}}%
\pgfpathlineto{\pgfqpoint{3.321857in}{2.782415in}}%
\pgfpathlineto{\pgfqpoint{3.329418in}{2.790140in}}%
\pgfpathlineto{\pgfqpoint{3.336973in}{2.797919in}}%
\pgfpathlineto{\pgfqpoint{3.344522in}{2.805754in}}%
\pgfpathlineto{\pgfqpoint{3.352065in}{2.813646in}}%
\pgfpathclose%
\pgfusepath{fill}%
\end{pgfscope}%
\begin{pgfscope}%
\pgfpathrectangle{\pgfqpoint{1.254980in}{0.150000in}}{\pgfqpoint{5.490039in}{5.490039in}}%
\pgfusepath{clip}%
\pgfsetbuttcap%
\pgfsetroundjoin%
\definecolor{currentfill}{rgb}{0.277018,0.050344,0.375715}%
\pgfsetfillcolor{currentfill}%
\pgfsetfillopacity{0.700000}%
\pgfsetlinewidth{0.000000pt}%
\definecolor{currentstroke}{rgb}{0.000000,0.000000,0.000000}%
\pgfsetstrokecolor{currentstroke}%
\pgfsetdash{}{0pt}%
\pgfpathmoveto{\pgfqpoint{3.952872in}{2.794862in}}%
\pgfpathlineto{\pgfqpoint{3.965487in}{2.790226in}}%
\pgfpathlineto{\pgfqpoint{3.978108in}{2.785623in}}%
\pgfpathlineto{\pgfqpoint{3.990733in}{2.781054in}}%
\pgfpathlineto{\pgfqpoint{4.003362in}{2.776519in}}%
\pgfpathlineto{\pgfqpoint{3.996035in}{2.768030in}}%
\pgfpathlineto{\pgfqpoint{3.988704in}{2.759615in}}%
\pgfpathlineto{\pgfqpoint{3.981367in}{2.751272in}}%
\pgfpathlineto{\pgfqpoint{3.974025in}{2.742997in}}%
\pgfpathlineto{\pgfqpoint{3.961384in}{2.747437in}}%
\pgfpathlineto{\pgfqpoint{3.948749in}{2.751910in}}%
\pgfpathlineto{\pgfqpoint{3.936118in}{2.756417in}}%
\pgfpathlineto{\pgfqpoint{3.923492in}{2.760958in}}%
\pgfpathlineto{\pgfqpoint{3.930845in}{2.769324in}}%
\pgfpathlineto{\pgfqpoint{3.938192in}{2.777761in}}%
\pgfpathlineto{\pgfqpoint{3.945535in}{2.786273in}}%
\pgfpathlineto{\pgfqpoint{3.952872in}{2.794862in}}%
\pgfpathclose%
\pgfusepath{fill}%
\end{pgfscope}%
\begin{pgfscope}%
\pgfpathrectangle{\pgfqpoint{1.254980in}{0.150000in}}{\pgfqpoint{5.490039in}{5.490039in}}%
\pgfusepath{clip}%
\pgfsetbuttcap%
\pgfsetroundjoin%
\definecolor{currentfill}{rgb}{0.280267,0.073417,0.397163}%
\pgfsetfillcolor{currentfill}%
\pgfsetfillopacity{0.700000}%
\pgfsetlinewidth{0.000000pt}%
\definecolor{currentstroke}{rgb}{0.000000,0.000000,0.000000}%
\pgfsetstrokecolor{currentstroke}%
\pgfsetdash{}{0pt}%
\pgfpathmoveto{\pgfqpoint{4.503161in}{2.834912in}}%
\pgfpathlineto{\pgfqpoint{4.515887in}{2.830597in}}%
\pgfpathlineto{\pgfqpoint{4.528617in}{2.826311in}}%
\pgfpathlineto{\pgfqpoint{4.541353in}{2.822052in}}%
\pgfpathlineto{\pgfqpoint{4.554095in}{2.817821in}}%
\pgfpathlineto{\pgfqpoint{4.546939in}{2.808319in}}%
\pgfpathlineto{\pgfqpoint{4.539780in}{2.798950in}}%
\pgfpathlineto{\pgfqpoint{4.532617in}{2.789709in}}%
\pgfpathlineto{\pgfqpoint{4.525451in}{2.780592in}}%
\pgfpathlineto{\pgfqpoint{4.512698in}{2.784666in}}%
\pgfpathlineto{\pgfqpoint{4.499951in}{2.788767in}}%
\pgfpathlineto{\pgfqpoint{4.487209in}{2.792896in}}%
\pgfpathlineto{\pgfqpoint{4.474472in}{2.797054in}}%
\pgfpathlineto{\pgfqpoint{4.481649in}{2.806324in}}%
\pgfpathlineto{\pgfqpoint{4.488823in}{2.815720in}}%
\pgfpathlineto{\pgfqpoint{4.495994in}{2.825248in}}%
\pgfpathlineto{\pgfqpoint{4.503161in}{2.834912in}}%
\pgfpathclose%
\pgfusepath{fill}%
\end{pgfscope}%
\begin{pgfscope}%
\pgfpathrectangle{\pgfqpoint{1.254980in}{0.150000in}}{\pgfqpoint{5.490039in}{5.490039in}}%
\pgfusepath{clip}%
\pgfsetbuttcap%
\pgfsetroundjoin%
\definecolor{currentfill}{rgb}{0.277018,0.050344,0.375715}%
\pgfsetfillcolor{currentfill}%
\pgfsetfillopacity{0.700000}%
\pgfsetlinewidth{0.000000pt}%
\definecolor{currentstroke}{rgb}{0.000000,0.000000,0.000000}%
\pgfsetstrokecolor{currentstroke}%
\pgfsetdash{}{0pt}%
\pgfpathmoveto{\pgfqpoint{3.482300in}{2.801134in}}%
\pgfpathlineto{\pgfqpoint{3.494833in}{2.795745in}}%
\pgfpathlineto{\pgfqpoint{3.507369in}{2.790397in}}%
\pgfpathlineto{\pgfqpoint{3.519910in}{2.785092in}}%
\pgfpathlineto{\pgfqpoint{3.532455in}{2.779828in}}%
\pgfpathlineto{\pgfqpoint{3.524969in}{2.771840in}}%
\pgfpathlineto{\pgfqpoint{3.517476in}{2.763906in}}%
\pgfpathlineto{\pgfqpoint{3.509978in}{2.756026in}}%
\pgfpathlineto{\pgfqpoint{3.502474in}{2.748197in}}%
\pgfpathlineto{\pgfqpoint{3.489918in}{2.753415in}}%
\pgfpathlineto{\pgfqpoint{3.477366in}{2.758675in}}%
\pgfpathlineto{\pgfqpoint{3.464818in}{2.763977in}}%
\pgfpathlineto{\pgfqpoint{3.452275in}{2.769321in}}%
\pgfpathlineto{\pgfqpoint{3.459790in}{2.777191in}}%
\pgfpathlineto{\pgfqpoint{3.467299in}{2.785115in}}%
\pgfpathlineto{\pgfqpoint{3.474802in}{2.793096in}}%
\pgfpathlineto{\pgfqpoint{3.482300in}{2.801134in}}%
\pgfpathclose%
\pgfusepath{fill}%
\end{pgfscope}%
\begin{pgfscope}%
\pgfpathrectangle{\pgfqpoint{1.254980in}{0.150000in}}{\pgfqpoint{5.490039in}{5.490039in}}%
\pgfusepath{clip}%
\pgfsetbuttcap%
\pgfsetroundjoin%
\definecolor{currentfill}{rgb}{0.278791,0.062145,0.386592}%
\pgfsetfillcolor{currentfill}%
\pgfsetfillopacity{0.700000}%
\pgfsetlinewidth{0.000000pt}%
\definecolor{currentstroke}{rgb}{0.000000,0.000000,0.000000}%
\pgfsetstrokecolor{currentstroke}%
\pgfsetdash{}{0pt}%
\pgfpathmoveto{\pgfqpoint{4.293169in}{2.811533in}}%
\pgfpathlineto{\pgfqpoint{4.305853in}{2.807173in}}%
\pgfpathlineto{\pgfqpoint{4.318543in}{2.802842in}}%
\pgfpathlineto{\pgfqpoint{4.331237in}{2.798541in}}%
\pgfpathlineto{\pgfqpoint{4.343937in}{2.794269in}}%
\pgfpathlineto{\pgfqpoint{4.336718in}{2.785286in}}%
\pgfpathlineto{\pgfqpoint{4.329495in}{2.776408in}}%
\pgfpathlineto{\pgfqpoint{4.322269in}{2.767631in}}%
\pgfpathlineto{\pgfqpoint{4.315038in}{2.758950in}}%
\pgfpathlineto{\pgfqpoint{4.302327in}{2.763089in}}%
\pgfpathlineto{\pgfqpoint{4.289621in}{2.767257in}}%
\pgfpathlineto{\pgfqpoint{4.276921in}{2.771455in}}%
\pgfpathlineto{\pgfqpoint{4.264226in}{2.775683in}}%
\pgfpathlineto{\pgfqpoint{4.271468in}{2.784492in}}%
\pgfpathlineto{\pgfqpoint{4.278706in}{2.793401in}}%
\pgfpathlineto{\pgfqpoint{4.285940in}{2.802413in}}%
\pgfpathlineto{\pgfqpoint{4.293169in}{2.811533in}}%
\pgfpathclose%
\pgfusepath{fill}%
\end{pgfscope}%
\begin{pgfscope}%
\pgfpathrectangle{\pgfqpoint{1.254980in}{0.150000in}}{\pgfqpoint{5.490039in}{5.490039in}}%
\pgfusepath{clip}%
\pgfsetbuttcap%
\pgfsetroundjoin%
\definecolor{currentfill}{rgb}{0.277018,0.050344,0.375715}%
\pgfsetfillcolor{currentfill}%
\pgfsetfillopacity{0.700000}%
\pgfsetlinewidth{0.000000pt}%
\definecolor{currentstroke}{rgb}{0.000000,0.000000,0.000000}%
\pgfsetstrokecolor{currentstroke}%
\pgfsetdash{}{0pt}%
\pgfpathmoveto{\pgfqpoint{3.612525in}{2.791492in}}%
\pgfpathlineto{\pgfqpoint{3.625080in}{2.786373in}}%
\pgfpathlineto{\pgfqpoint{3.637639in}{2.781293in}}%
\pgfpathlineto{\pgfqpoint{3.650203in}{2.776252in}}%
\pgfpathlineto{\pgfqpoint{3.662772in}{2.771250in}}%
\pgfpathlineto{\pgfqpoint{3.655329in}{2.763147in}}%
\pgfpathlineto{\pgfqpoint{3.647880in}{2.755101in}}%
\pgfpathlineto{\pgfqpoint{3.640426in}{2.747109in}}%
\pgfpathlineto{\pgfqpoint{3.632967in}{2.739171in}}%
\pgfpathlineto{\pgfqpoint{3.620388in}{2.744115in}}%
\pgfpathlineto{\pgfqpoint{3.607813in}{2.749098in}}%
\pgfpathlineto{\pgfqpoint{3.595243in}{2.754120in}}%
\pgfpathlineto{\pgfqpoint{3.582677in}{2.759181in}}%
\pgfpathlineto{\pgfqpoint{3.590147in}{2.767173in}}%
\pgfpathlineto{\pgfqpoint{3.597612in}{2.775221in}}%
\pgfpathlineto{\pgfqpoint{3.605071in}{2.783327in}}%
\pgfpathlineto{\pgfqpoint{3.612525in}{2.791492in}}%
\pgfpathclose%
\pgfusepath{fill}%
\end{pgfscope}%
\begin{pgfscope}%
\pgfpathrectangle{\pgfqpoint{1.254980in}{0.150000in}}{\pgfqpoint{5.490039in}{5.490039in}}%
\pgfusepath{clip}%
\pgfsetbuttcap%
\pgfsetroundjoin%
\definecolor{currentfill}{rgb}{0.277018,0.050344,0.375715}%
\pgfsetfillcolor{currentfill}%
\pgfsetfillopacity{0.700000}%
\pgfsetlinewidth{0.000000pt}%
\definecolor{currentstroke}{rgb}{0.000000,0.000000,0.000000}%
\pgfsetstrokecolor{currentstroke}%
\pgfsetdash{}{0pt}%
\pgfpathmoveto{\pgfqpoint{4.083148in}{2.793045in}}%
\pgfpathlineto{\pgfqpoint{4.095792in}{2.788565in}}%
\pgfpathlineto{\pgfqpoint{4.108441in}{2.784118in}}%
\pgfpathlineto{\pgfqpoint{4.121095in}{2.779702in}}%
\pgfpathlineto{\pgfqpoint{4.133755in}{2.775317in}}%
\pgfpathlineto{\pgfqpoint{4.126468in}{2.766721in}}%
\pgfpathlineto{\pgfqpoint{4.119177in}{2.758206in}}%
\pgfpathlineto{\pgfqpoint{4.111882in}{2.749770in}}%
\pgfpathlineto{\pgfqpoint{4.104581in}{2.741410in}}%
\pgfpathlineto{\pgfqpoint{4.091911in}{2.745686in}}%
\pgfpathlineto{\pgfqpoint{4.079246in}{2.749994in}}%
\pgfpathlineto{\pgfqpoint{4.066586in}{2.754334in}}%
\pgfpathlineto{\pgfqpoint{4.053931in}{2.758706in}}%
\pgfpathlineto{\pgfqpoint{4.061243in}{2.767170in}}%
\pgfpathlineto{\pgfqpoint{4.068549in}{2.775712in}}%
\pgfpathlineto{\pgfqpoint{4.075851in}{2.784336in}}%
\pgfpathlineto{\pgfqpoint{4.083148in}{2.793045in}}%
\pgfpathclose%
\pgfusepath{fill}%
\end{pgfscope}%
\begin{pgfscope}%
\pgfpathrectangle{\pgfqpoint{1.254980in}{0.150000in}}{\pgfqpoint{5.490039in}{5.490039in}}%
\pgfusepath{clip}%
\pgfsetbuttcap%
\pgfsetroundjoin%
\definecolor{currentfill}{rgb}{0.276022,0.044167,0.370164}%
\pgfsetfillcolor{currentfill}%
\pgfsetfillopacity{0.700000}%
\pgfsetlinewidth{0.000000pt}%
\definecolor{currentstroke}{rgb}{0.000000,0.000000,0.000000}%
\pgfsetstrokecolor{currentstroke}%
\pgfsetdash{}{0pt}%
\pgfpathmoveto{\pgfqpoint{3.742763in}{2.784370in}}%
\pgfpathlineto{\pgfqpoint{3.755343in}{2.779486in}}%
\pgfpathlineto{\pgfqpoint{3.767928in}{2.774640in}}%
\pgfpathlineto{\pgfqpoint{3.780517in}{2.769830in}}%
\pgfpathlineto{\pgfqpoint{3.793111in}{2.765057in}}%
\pgfpathlineto{\pgfqpoint{3.785711in}{2.756852in}}%
\pgfpathlineto{\pgfqpoint{3.778306in}{2.748709in}}%
\pgfpathlineto{\pgfqpoint{3.770895in}{2.740623in}}%
\pgfpathlineto{\pgfqpoint{3.763479in}{2.732592in}}%
\pgfpathlineto{\pgfqpoint{3.750875in}{2.737295in}}%
\pgfpathlineto{\pgfqpoint{3.738275in}{2.742034in}}%
\pgfpathlineto{\pgfqpoint{3.725680in}{2.746810in}}%
\pgfpathlineto{\pgfqpoint{3.713089in}{2.751623in}}%
\pgfpathlineto{\pgfqpoint{3.720516in}{2.759719in}}%
\pgfpathlineto{\pgfqpoint{3.727937in}{2.767874in}}%
\pgfpathlineto{\pgfqpoint{3.735353in}{2.776090in}}%
\pgfpathlineto{\pgfqpoint{3.742763in}{2.784370in}}%
\pgfpathclose%
\pgfusepath{fill}%
\end{pgfscope}%
\begin{pgfscope}%
\pgfpathrectangle{\pgfqpoint{1.254980in}{0.150000in}}{\pgfqpoint{5.490039in}{5.490039in}}%
\pgfusepath{clip}%
\pgfsetbuttcap%
\pgfsetroundjoin%
\definecolor{currentfill}{rgb}{0.279566,0.067836,0.391917}%
\pgfsetfillcolor{currentfill}%
\pgfsetfillopacity{0.700000}%
\pgfsetlinewidth{0.000000pt}%
\definecolor{currentstroke}{rgb}{0.000000,0.000000,0.000000}%
\pgfsetstrokecolor{currentstroke}%
\pgfsetdash{}{0pt}%
\pgfpathmoveto{\pgfqpoint{4.423579in}{2.813967in}}%
\pgfpathlineto{\pgfqpoint{4.436295in}{2.809696in}}%
\pgfpathlineto{\pgfqpoint{4.449015in}{2.805453in}}%
\pgfpathlineto{\pgfqpoint{4.461741in}{2.801239in}}%
\pgfpathlineto{\pgfqpoint{4.474472in}{2.797054in}}%
\pgfpathlineto{\pgfqpoint{4.467291in}{2.787905in}}%
\pgfpathlineto{\pgfqpoint{4.460107in}{2.778874in}}%
\pgfpathlineto{\pgfqpoint{4.452919in}{2.769955in}}%
\pgfpathlineto{\pgfqpoint{4.445727in}{2.761144in}}%
\pgfpathlineto{\pgfqpoint{4.432984in}{2.765184in}}%
\pgfpathlineto{\pgfqpoint{4.420247in}{2.769253in}}%
\pgfpathlineto{\pgfqpoint{4.407515in}{2.773350in}}%
\pgfpathlineto{\pgfqpoint{4.394788in}{2.777476in}}%
\pgfpathlineto{\pgfqpoint{4.401992in}{2.786428in}}%
\pgfpathlineto{\pgfqpoint{4.409191in}{2.795491in}}%
\pgfpathlineto{\pgfqpoint{4.416387in}{2.804669in}}%
\pgfpathlineto{\pgfqpoint{4.423579in}{2.813967in}}%
\pgfpathclose%
\pgfusepath{fill}%
\end{pgfscope}%
\begin{pgfscope}%
\pgfpathrectangle{\pgfqpoint{1.254980in}{0.150000in}}{\pgfqpoint{5.490039in}{5.490039in}}%
\pgfusepath{clip}%
\pgfsetbuttcap%
\pgfsetroundjoin%
\definecolor{currentfill}{rgb}{0.276022,0.044167,0.370164}%
\pgfsetfillcolor{currentfill}%
\pgfsetfillopacity{0.700000}%
\pgfsetlinewidth{0.000000pt}%
\definecolor{currentstroke}{rgb}{0.000000,0.000000,0.000000}%
\pgfsetstrokecolor{currentstroke}%
\pgfsetdash{}{0pt}%
\pgfpathmoveto{\pgfqpoint{3.873037in}{2.779464in}}%
\pgfpathlineto{\pgfqpoint{3.885644in}{2.774786in}}%
\pgfpathlineto{\pgfqpoint{3.898255in}{2.770142in}}%
\pgfpathlineto{\pgfqpoint{3.910871in}{2.765533in}}%
\pgfpathlineto{\pgfqpoint{3.923492in}{2.760958in}}%
\pgfpathlineto{\pgfqpoint{3.916134in}{2.752660in}}%
\pgfpathlineto{\pgfqpoint{3.908771in}{2.744428in}}%
\pgfpathlineto{\pgfqpoint{3.901403in}{2.736258in}}%
\pgfpathlineto{\pgfqpoint{3.894030in}{2.728147in}}%
\pgfpathlineto{\pgfqpoint{3.881399in}{2.732639in}}%
\pgfpathlineto{\pgfqpoint{3.868772in}{2.737165in}}%
\pgfpathlineto{\pgfqpoint{3.856150in}{2.741726in}}%
\pgfpathlineto{\pgfqpoint{3.843532in}{2.746321in}}%
\pgfpathlineto{\pgfqpoint{3.850916in}{2.754510in}}%
\pgfpathlineto{\pgfqpoint{3.858295in}{2.762762in}}%
\pgfpathlineto{\pgfqpoint{3.865669in}{2.771079in}}%
\pgfpathlineto{\pgfqpoint{3.873037in}{2.779464in}}%
\pgfpathclose%
\pgfusepath{fill}%
\end{pgfscope}%
\begin{pgfscope}%
\pgfpathrectangle{\pgfqpoint{1.254980in}{0.150000in}}{\pgfqpoint{5.490039in}{5.490039in}}%
\pgfusepath{clip}%
\pgfsetbuttcap%
\pgfsetroundjoin%
\definecolor{currentfill}{rgb}{0.277941,0.056324,0.381191}%
\pgfsetfillcolor{currentfill}%
\pgfsetfillopacity{0.700000}%
\pgfsetlinewidth{0.000000pt}%
\definecolor{currentstroke}{rgb}{0.000000,0.000000,0.000000}%
\pgfsetstrokecolor{currentstroke}%
\pgfsetdash{}{0pt}%
\pgfpathmoveto{\pgfqpoint{4.213499in}{2.792897in}}%
\pgfpathlineto{\pgfqpoint{4.226173in}{2.788548in}}%
\pgfpathlineto{\pgfqpoint{4.238852in}{2.784229in}}%
\pgfpathlineto{\pgfqpoint{4.251537in}{2.779941in}}%
\pgfpathlineto{\pgfqpoint{4.264226in}{2.775683in}}%
\pgfpathlineto{\pgfqpoint{4.256980in}{2.766969in}}%
\pgfpathlineto{\pgfqpoint{4.249729in}{2.758347in}}%
\pgfpathlineto{\pgfqpoint{4.242474in}{2.749812in}}%
\pgfpathlineto{\pgfqpoint{4.235214in}{2.741361in}}%
\pgfpathlineto{\pgfqpoint{4.222513in}{2.745499in}}%
\pgfpathlineto{\pgfqpoint{4.209818in}{2.749666in}}%
\pgfpathlineto{\pgfqpoint{4.197128in}{2.753864in}}%
\pgfpathlineto{\pgfqpoint{4.184443in}{2.758093in}}%
\pgfpathlineto{\pgfqpoint{4.191713in}{2.766660in}}%
\pgfpathlineto{\pgfqpoint{4.198980in}{2.775314in}}%
\pgfpathlineto{\pgfqpoint{4.206242in}{2.784058in}}%
\pgfpathlineto{\pgfqpoint{4.213499in}{2.792897in}}%
\pgfpathclose%
\pgfusepath{fill}%
\end{pgfscope}%
\begin{pgfscope}%
\pgfpathrectangle{\pgfqpoint{1.254980in}{0.150000in}}{\pgfqpoint{5.490039in}{5.490039in}}%
\pgfusepath{clip}%
\pgfsetbuttcap%
\pgfsetroundjoin%
\definecolor{currentfill}{rgb}{0.278791,0.062145,0.386592}%
\pgfsetfillcolor{currentfill}%
\pgfsetfillopacity{0.700000}%
\pgfsetlinewidth{0.000000pt}%
\definecolor{currentstroke}{rgb}{0.000000,0.000000,0.000000}%
\pgfsetstrokecolor{currentstroke}%
\pgfsetdash{}{0pt}%
\pgfpathmoveto{\pgfqpoint{3.271797in}{2.805538in}}%
\pgfpathlineto{\pgfqpoint{3.284306in}{2.799687in}}%
\pgfpathlineto{\pgfqpoint{3.296820in}{2.793883in}}%
\pgfpathlineto{\pgfqpoint{3.309337in}{2.788125in}}%
\pgfpathlineto{\pgfqpoint{3.321857in}{2.782415in}}%
\pgfpathlineto{\pgfqpoint{3.314290in}{2.774744in}}%
\pgfpathlineto{\pgfqpoint{3.306717in}{2.767124in}}%
\pgfpathlineto{\pgfqpoint{3.299137in}{2.759555in}}%
\pgfpathlineto{\pgfqpoint{3.291552in}{2.752037in}}%
\pgfpathlineto{\pgfqpoint{3.279020in}{2.757727in}}%
\pgfpathlineto{\pgfqpoint{3.266492in}{2.763463in}}%
\pgfpathlineto{\pgfqpoint{3.253967in}{2.769247in}}%
\pgfpathlineto{\pgfqpoint{3.241445in}{2.775078in}}%
\pgfpathlineto{\pgfqpoint{3.249042in}{2.782613in}}%
\pgfpathlineto{\pgfqpoint{3.256633in}{2.790200in}}%
\pgfpathlineto{\pgfqpoint{3.264218in}{2.797842in}}%
\pgfpathlineto{\pgfqpoint{3.271797in}{2.805538in}}%
\pgfpathclose%
\pgfusepath{fill}%
\end{pgfscope}%
\begin{pgfscope}%
\pgfpathrectangle{\pgfqpoint{1.254980in}{0.150000in}}{\pgfqpoint{5.490039in}{5.490039in}}%
\pgfusepath{clip}%
\pgfsetbuttcap%
\pgfsetroundjoin%
\definecolor{currentfill}{rgb}{0.277941,0.056324,0.381191}%
\pgfsetfillcolor{currentfill}%
\pgfsetfillopacity{0.700000}%
\pgfsetlinewidth{0.000000pt}%
\definecolor{currentstroke}{rgb}{0.000000,0.000000,0.000000}%
\pgfsetstrokecolor{currentstroke}%
\pgfsetdash{}{0pt}%
\pgfpathmoveto{\pgfqpoint{3.402139in}{2.791129in}}%
\pgfpathlineto{\pgfqpoint{3.414667in}{2.785611in}}%
\pgfpathlineto{\pgfqpoint{3.427199in}{2.780138in}}%
\pgfpathlineto{\pgfqpoint{3.439735in}{2.774708in}}%
\pgfpathlineto{\pgfqpoint{3.452275in}{2.769321in}}%
\pgfpathlineto{\pgfqpoint{3.444754in}{2.761504in}}%
\pgfpathlineto{\pgfqpoint{3.437227in}{2.753738in}}%
\pgfpathlineto{\pgfqpoint{3.429694in}{2.746022in}}%
\pgfpathlineto{\pgfqpoint{3.422156in}{2.738354in}}%
\pgfpathlineto{\pgfqpoint{3.409605in}{2.743708in}}%
\pgfpathlineto{\pgfqpoint{3.397058in}{2.749104in}}%
\pgfpathlineto{\pgfqpoint{3.384515in}{2.754545in}}%
\pgfpathlineto{\pgfqpoint{3.371976in}{2.760029in}}%
\pgfpathlineto{\pgfqpoint{3.379526in}{2.767726in}}%
\pgfpathlineto{\pgfqpoint{3.387069in}{2.775473in}}%
\pgfpathlineto{\pgfqpoint{3.394607in}{2.783274in}}%
\pgfpathlineto{\pgfqpoint{3.402139in}{2.791129in}}%
\pgfpathclose%
\pgfusepath{fill}%
\end{pgfscope}%
\begin{pgfscope}%
\pgfpathrectangle{\pgfqpoint{1.254980in}{0.150000in}}{\pgfqpoint{5.490039in}{5.490039in}}%
\pgfusepath{clip}%
\pgfsetbuttcap%
\pgfsetroundjoin%
\definecolor{currentfill}{rgb}{0.277018,0.050344,0.375715}%
\pgfsetfillcolor{currentfill}%
\pgfsetfillopacity{0.700000}%
\pgfsetlinewidth{0.000000pt}%
\definecolor{currentstroke}{rgb}{0.000000,0.000000,0.000000}%
\pgfsetstrokecolor{currentstroke}%
\pgfsetdash{}{0pt}%
\pgfpathmoveto{\pgfqpoint{3.532455in}{2.779828in}}%
\pgfpathlineto{\pgfqpoint{3.545004in}{2.774606in}}%
\pgfpathlineto{\pgfqpoint{3.557558in}{2.769424in}}%
\pgfpathlineto{\pgfqpoint{3.570115in}{2.764282in}}%
\pgfpathlineto{\pgfqpoint{3.582677in}{2.759181in}}%
\pgfpathlineto{\pgfqpoint{3.575201in}{2.751243in}}%
\pgfpathlineto{\pgfqpoint{3.567719in}{2.743357in}}%
\pgfpathlineto{\pgfqpoint{3.560232in}{2.735520in}}%
\pgfpathlineto{\pgfqpoint{3.552739in}{2.727732in}}%
\pgfpathlineto{\pgfqpoint{3.540166in}{2.732788in}}%
\pgfpathlineto{\pgfqpoint{3.527598in}{2.737883in}}%
\pgfpathlineto{\pgfqpoint{3.515034in}{2.743020in}}%
\pgfpathlineto{\pgfqpoint{3.502474in}{2.748197in}}%
\pgfpathlineto{\pgfqpoint{3.509978in}{2.756026in}}%
\pgfpathlineto{\pgfqpoint{3.517476in}{2.763906in}}%
\pgfpathlineto{\pgfqpoint{3.524969in}{2.771840in}}%
\pgfpathlineto{\pgfqpoint{3.532455in}{2.779828in}}%
\pgfpathclose%
\pgfusepath{fill}%
\end{pgfscope}%
\begin{pgfscope}%
\pgfpathrectangle{\pgfqpoint{1.254980in}{0.150000in}}{\pgfqpoint{5.490039in}{5.490039in}}%
\pgfusepath{clip}%
\pgfsetbuttcap%
\pgfsetroundjoin%
\definecolor{currentfill}{rgb}{0.276022,0.044167,0.370164}%
\pgfsetfillcolor{currentfill}%
\pgfsetfillopacity{0.700000}%
\pgfsetlinewidth{0.000000pt}%
\definecolor{currentstroke}{rgb}{0.000000,0.000000,0.000000}%
\pgfsetstrokecolor{currentstroke}%
\pgfsetdash{}{0pt}%
\pgfpathmoveto{\pgfqpoint{4.003362in}{2.776519in}}%
\pgfpathlineto{\pgfqpoint{4.015997in}{2.772016in}}%
\pgfpathlineto{\pgfqpoint{4.028637in}{2.767547in}}%
\pgfpathlineto{\pgfqpoint{4.041282in}{2.763110in}}%
\pgfpathlineto{\pgfqpoint{4.053931in}{2.758706in}}%
\pgfpathlineto{\pgfqpoint{4.046615in}{2.750317in}}%
\pgfpathlineto{\pgfqpoint{4.039294in}{2.741999in}}%
\pgfpathlineto{\pgfqpoint{4.031968in}{2.733750in}}%
\pgfpathlineto{\pgfqpoint{4.024637in}{2.725566in}}%
\pgfpathlineto{\pgfqpoint{4.011976in}{2.729875in}}%
\pgfpathlineto{\pgfqpoint{3.999321in}{2.734216in}}%
\pgfpathlineto{\pgfqpoint{3.986670in}{2.738590in}}%
\pgfpathlineto{\pgfqpoint{3.974025in}{2.742997in}}%
\pgfpathlineto{\pgfqpoint{3.981367in}{2.751272in}}%
\pgfpathlineto{\pgfqpoint{3.988704in}{2.759615in}}%
\pgfpathlineto{\pgfqpoint{3.996035in}{2.768030in}}%
\pgfpathlineto{\pgfqpoint{4.003362in}{2.776519in}}%
\pgfpathclose%
\pgfusepath{fill}%
\end{pgfscope}%
\begin{pgfscope}%
\pgfpathrectangle{\pgfqpoint{1.254980in}{0.150000in}}{\pgfqpoint{5.490039in}{5.490039in}}%
\pgfusepath{clip}%
\pgfsetbuttcap%
\pgfsetroundjoin%
\definecolor{currentfill}{rgb}{0.280267,0.073417,0.397163}%
\pgfsetfillcolor{currentfill}%
\pgfsetfillopacity{0.700000}%
\pgfsetlinewidth{0.000000pt}%
\definecolor{currentstroke}{rgb}{0.000000,0.000000,0.000000}%
\pgfsetstrokecolor{currentstroke}%
\pgfsetdash{}{0pt}%
\pgfpathmoveto{\pgfqpoint{4.554095in}{2.817821in}}%
\pgfpathlineto{\pgfqpoint{4.566842in}{2.813618in}}%
\pgfpathlineto{\pgfqpoint{4.579594in}{2.809442in}}%
\pgfpathlineto{\pgfqpoint{4.592352in}{2.805294in}}%
\pgfpathlineto{\pgfqpoint{4.605115in}{2.801173in}}%
\pgfpathlineto{\pgfqpoint{4.597971in}{2.791833in}}%
\pgfpathlineto{\pgfqpoint{4.590824in}{2.782623in}}%
\pgfpathlineto{\pgfqpoint{4.583673in}{2.773539in}}%
\pgfpathlineto{\pgfqpoint{4.576519in}{2.764575in}}%
\pgfpathlineto{\pgfqpoint{4.563744in}{2.768538in}}%
\pgfpathlineto{\pgfqpoint{4.550974in}{2.772529in}}%
\pgfpathlineto{\pgfqpoint{4.538210in}{2.776547in}}%
\pgfpathlineto{\pgfqpoint{4.525451in}{2.780592in}}%
\pgfpathlineto{\pgfqpoint{4.532617in}{2.789709in}}%
\pgfpathlineto{\pgfqpoint{4.539780in}{2.798950in}}%
\pgfpathlineto{\pgfqpoint{4.546939in}{2.808319in}}%
\pgfpathlineto{\pgfqpoint{4.554095in}{2.817821in}}%
\pgfpathclose%
\pgfusepath{fill}%
\end{pgfscope}%
\begin{pgfscope}%
\pgfpathrectangle{\pgfqpoint{1.254980in}{0.150000in}}{\pgfqpoint{5.490039in}{5.490039in}}%
\pgfusepath{clip}%
\pgfsetbuttcap%
\pgfsetroundjoin%
\definecolor{currentfill}{rgb}{0.277941,0.056324,0.381191}%
\pgfsetfillcolor{currentfill}%
\pgfsetfillopacity{0.700000}%
\pgfsetlinewidth{0.000000pt}%
\definecolor{currentstroke}{rgb}{0.000000,0.000000,0.000000}%
\pgfsetstrokecolor{currentstroke}%
\pgfsetdash{}{0pt}%
\pgfpathmoveto{\pgfqpoint{4.343937in}{2.794269in}}%
\pgfpathlineto{\pgfqpoint{4.356642in}{2.790027in}}%
\pgfpathlineto{\pgfqpoint{4.369352in}{2.785815in}}%
\pgfpathlineto{\pgfqpoint{4.382067in}{2.781631in}}%
\pgfpathlineto{\pgfqpoint{4.394788in}{2.777476in}}%
\pgfpathlineto{\pgfqpoint{4.387581in}{2.768630in}}%
\pgfpathlineto{\pgfqpoint{4.380370in}{2.759887in}}%
\pgfpathlineto{\pgfqpoint{4.373154in}{2.751241in}}%
\pgfpathlineto{\pgfqpoint{4.365935in}{2.742688in}}%
\pgfpathlineto{\pgfqpoint{4.353202in}{2.746710in}}%
\pgfpathlineto{\pgfqpoint{4.340475in}{2.750761in}}%
\pgfpathlineto{\pgfqpoint{4.327754in}{2.754841in}}%
\pgfpathlineto{\pgfqpoint{4.315038in}{2.758950in}}%
\pgfpathlineto{\pgfqpoint{4.322269in}{2.767631in}}%
\pgfpathlineto{\pgfqpoint{4.329495in}{2.776408in}}%
\pgfpathlineto{\pgfqpoint{4.336718in}{2.785286in}}%
\pgfpathlineto{\pgfqpoint{4.343937in}{2.794269in}}%
\pgfpathclose%
\pgfusepath{fill}%
\end{pgfscope}%
\begin{pgfscope}%
\pgfpathrectangle{\pgfqpoint{1.254980in}{0.150000in}}{\pgfqpoint{5.490039in}{5.490039in}}%
\pgfusepath{clip}%
\pgfsetbuttcap%
\pgfsetroundjoin%
\definecolor{currentfill}{rgb}{0.276022,0.044167,0.370164}%
\pgfsetfillcolor{currentfill}%
\pgfsetfillopacity{0.700000}%
\pgfsetlinewidth{0.000000pt}%
\definecolor{currentstroke}{rgb}{0.000000,0.000000,0.000000}%
\pgfsetstrokecolor{currentstroke}%
\pgfsetdash{}{0pt}%
\pgfpathmoveto{\pgfqpoint{3.662772in}{2.771250in}}%
\pgfpathlineto{\pgfqpoint{3.675344in}{2.766286in}}%
\pgfpathlineto{\pgfqpoint{3.687921in}{2.761361in}}%
\pgfpathlineto{\pgfqpoint{3.700503in}{2.756473in}}%
\pgfpathlineto{\pgfqpoint{3.713089in}{2.751623in}}%
\pgfpathlineto{\pgfqpoint{3.705657in}{2.743583in}}%
\pgfpathlineto{\pgfqpoint{3.698219in}{2.735596in}}%
\pgfpathlineto{\pgfqpoint{3.690776in}{2.727662in}}%
\pgfpathlineto{\pgfqpoint{3.683328in}{2.719777in}}%
\pgfpathlineto{\pgfqpoint{3.670731in}{2.724569in}}%
\pgfpathlineto{\pgfqpoint{3.658138in}{2.729398in}}%
\pgfpathlineto{\pgfqpoint{3.645550in}{2.734265in}}%
\pgfpathlineto{\pgfqpoint{3.632967in}{2.739171in}}%
\pgfpathlineto{\pgfqpoint{3.640426in}{2.747109in}}%
\pgfpathlineto{\pgfqpoint{3.647880in}{2.755101in}}%
\pgfpathlineto{\pgfqpoint{3.655329in}{2.763147in}}%
\pgfpathlineto{\pgfqpoint{3.662772in}{2.771250in}}%
\pgfpathclose%
\pgfusepath{fill}%
\end{pgfscope}%
\begin{pgfscope}%
\pgfpathrectangle{\pgfqpoint{1.254980in}{0.150000in}}{\pgfqpoint{5.490039in}{5.490039in}}%
\pgfusepath{clip}%
\pgfsetbuttcap%
\pgfsetroundjoin%
\definecolor{currentfill}{rgb}{0.277018,0.050344,0.375715}%
\pgfsetfillcolor{currentfill}%
\pgfsetfillopacity{0.700000}%
\pgfsetlinewidth{0.000000pt}%
\definecolor{currentstroke}{rgb}{0.000000,0.000000,0.000000}%
\pgfsetstrokecolor{currentstroke}%
\pgfsetdash{}{0pt}%
\pgfpathmoveto{\pgfqpoint{4.133755in}{2.775317in}}%
\pgfpathlineto{\pgfqpoint{4.146419in}{2.770965in}}%
\pgfpathlineto{\pgfqpoint{4.159088in}{2.766643in}}%
\pgfpathlineto{\pgfqpoint{4.171763in}{2.762353in}}%
\pgfpathlineto{\pgfqpoint{4.184443in}{2.758093in}}%
\pgfpathlineto{\pgfqpoint{4.177167in}{2.749609in}}%
\pgfpathlineto{\pgfqpoint{4.169887in}{2.741204in}}%
\pgfpathlineto{\pgfqpoint{4.162603in}{2.732875in}}%
\pgfpathlineto{\pgfqpoint{4.155313in}{2.724618in}}%
\pgfpathlineto{\pgfqpoint{4.142622in}{2.728769in}}%
\pgfpathlineto{\pgfqpoint{4.129937in}{2.732951in}}%
\pgfpathlineto{\pgfqpoint{4.117256in}{2.737165in}}%
\pgfpathlineto{\pgfqpoint{4.104581in}{2.741410in}}%
\pgfpathlineto{\pgfqpoint{4.111882in}{2.749770in}}%
\pgfpathlineto{\pgfqpoint{4.119177in}{2.758206in}}%
\pgfpathlineto{\pgfqpoint{4.126468in}{2.766721in}}%
\pgfpathlineto{\pgfqpoint{4.133755in}{2.775317in}}%
\pgfpathclose%
\pgfusepath{fill}%
\end{pgfscope}%
\begin{pgfscope}%
\pgfpathrectangle{\pgfqpoint{1.254980in}{0.150000in}}{\pgfqpoint{5.490039in}{5.490039in}}%
\pgfusepath{clip}%
\pgfsetbuttcap%
\pgfsetroundjoin%
\definecolor{currentfill}{rgb}{0.276022,0.044167,0.370164}%
\pgfsetfillcolor{currentfill}%
\pgfsetfillopacity{0.700000}%
\pgfsetlinewidth{0.000000pt}%
\definecolor{currentstroke}{rgb}{0.000000,0.000000,0.000000}%
\pgfsetstrokecolor{currentstroke}%
\pgfsetdash{}{0pt}%
\pgfpathmoveto{\pgfqpoint{3.793111in}{2.765057in}}%
\pgfpathlineto{\pgfqpoint{3.805709in}{2.760319in}}%
\pgfpathlineto{\pgfqpoint{3.818312in}{2.755618in}}%
\pgfpathlineto{\pgfqpoint{3.830920in}{2.750952in}}%
\pgfpathlineto{\pgfqpoint{3.843532in}{2.746321in}}%
\pgfpathlineto{\pgfqpoint{3.836143in}{2.738193in}}%
\pgfpathlineto{\pgfqpoint{3.828749in}{2.730121in}}%
\pgfpathlineto{\pgfqpoint{3.821349in}{2.722104in}}%
\pgfpathlineto{\pgfqpoint{3.813944in}{2.714140in}}%
\pgfpathlineto{\pgfqpoint{3.801321in}{2.718700in}}%
\pgfpathlineto{\pgfqpoint{3.788702in}{2.723295in}}%
\pgfpathlineto{\pgfqpoint{3.776088in}{2.727926in}}%
\pgfpathlineto{\pgfqpoint{3.763479in}{2.732592in}}%
\pgfpathlineto{\pgfqpoint{3.770895in}{2.740623in}}%
\pgfpathlineto{\pgfqpoint{3.778306in}{2.748709in}}%
\pgfpathlineto{\pgfqpoint{3.785711in}{2.756852in}}%
\pgfpathlineto{\pgfqpoint{3.793111in}{2.765057in}}%
\pgfpathclose%
\pgfusepath{fill}%
\end{pgfscope}%
\begin{pgfscope}%
\pgfpathrectangle{\pgfqpoint{1.254980in}{0.150000in}}{\pgfqpoint{5.490039in}{5.490039in}}%
\pgfusepath{clip}%
\pgfsetbuttcap%
\pgfsetroundjoin%
\definecolor{currentfill}{rgb}{0.278791,0.062145,0.386592}%
\pgfsetfillcolor{currentfill}%
\pgfsetfillopacity{0.700000}%
\pgfsetlinewidth{0.000000pt}%
\definecolor{currentstroke}{rgb}{0.000000,0.000000,0.000000}%
\pgfsetstrokecolor{currentstroke}%
\pgfsetdash{}{0pt}%
\pgfpathmoveto{\pgfqpoint{4.474472in}{2.797054in}}%
\pgfpathlineto{\pgfqpoint{4.487209in}{2.792896in}}%
\pgfpathlineto{\pgfqpoint{4.499951in}{2.788767in}}%
\pgfpathlineto{\pgfqpoint{4.512698in}{2.784666in}}%
\pgfpathlineto{\pgfqpoint{4.525451in}{2.780592in}}%
\pgfpathlineto{\pgfqpoint{4.518282in}{2.771594in}}%
\pgfpathlineto{\pgfqpoint{4.511109in}{2.762709in}}%
\pgfpathlineto{\pgfqpoint{4.503933in}{2.753934in}}%
\pgfpathlineto{\pgfqpoint{4.496753in}{2.745264in}}%
\pgfpathlineto{\pgfqpoint{4.483988in}{2.749192in}}%
\pgfpathlineto{\pgfqpoint{4.471229in}{2.753148in}}%
\pgfpathlineto{\pgfqpoint{4.458475in}{2.757132in}}%
\pgfpathlineto{\pgfqpoint{4.445727in}{2.761144in}}%
\pgfpathlineto{\pgfqpoint{4.452919in}{2.769955in}}%
\pgfpathlineto{\pgfqpoint{4.460107in}{2.778874in}}%
\pgfpathlineto{\pgfqpoint{4.467291in}{2.787905in}}%
\pgfpathlineto{\pgfqpoint{4.474472in}{2.797054in}}%
\pgfpathclose%
\pgfusepath{fill}%
\end{pgfscope}%
\begin{pgfscope}%
\pgfpathrectangle{\pgfqpoint{1.254980in}{0.150000in}}{\pgfqpoint{5.490039in}{5.490039in}}%
\pgfusepath{clip}%
\pgfsetbuttcap%
\pgfsetroundjoin%
\definecolor{currentfill}{rgb}{0.276022,0.044167,0.370164}%
\pgfsetfillcolor{currentfill}%
\pgfsetfillopacity{0.700000}%
\pgfsetlinewidth{0.000000pt}%
\definecolor{currentstroke}{rgb}{0.000000,0.000000,0.000000}%
\pgfsetstrokecolor{currentstroke}%
\pgfsetdash{}{0pt}%
\pgfpathmoveto{\pgfqpoint{3.923492in}{2.760958in}}%
\pgfpathlineto{\pgfqpoint{3.936118in}{2.756417in}}%
\pgfpathlineto{\pgfqpoint{3.948749in}{2.751910in}}%
\pgfpathlineto{\pgfqpoint{3.961384in}{2.747437in}}%
\pgfpathlineto{\pgfqpoint{3.974025in}{2.742997in}}%
\pgfpathlineto{\pgfqpoint{3.966678in}{2.734787in}}%
\pgfpathlineto{\pgfqpoint{3.959326in}{2.726639in}}%
\pgfpathlineto{\pgfqpoint{3.951969in}{2.718551in}}%
\pgfpathlineto{\pgfqpoint{3.944606in}{2.710519in}}%
\pgfpathlineto{\pgfqpoint{3.931955in}{2.714876in}}%
\pgfpathlineto{\pgfqpoint{3.919308in}{2.719266in}}%
\pgfpathlineto{\pgfqpoint{3.906667in}{2.723690in}}%
\pgfpathlineto{\pgfqpoint{3.894030in}{2.728147in}}%
\pgfpathlineto{\pgfqpoint{3.901403in}{2.736258in}}%
\pgfpathlineto{\pgfqpoint{3.908771in}{2.744428in}}%
\pgfpathlineto{\pgfqpoint{3.916134in}{2.752660in}}%
\pgfpathlineto{\pgfqpoint{3.923492in}{2.760958in}}%
\pgfpathclose%
\pgfusepath{fill}%
\end{pgfscope}%
\begin{pgfscope}%
\pgfpathrectangle{\pgfqpoint{1.254980in}{0.150000in}}{\pgfqpoint{5.490039in}{5.490039in}}%
\pgfusepath{clip}%
\pgfsetbuttcap%
\pgfsetroundjoin%
\definecolor{currentfill}{rgb}{0.277941,0.056324,0.381191}%
\pgfsetfillcolor{currentfill}%
\pgfsetfillopacity{0.700000}%
\pgfsetlinewidth{0.000000pt}%
\definecolor{currentstroke}{rgb}{0.000000,0.000000,0.000000}%
\pgfsetstrokecolor{currentstroke}%
\pgfsetdash{}{0pt}%
\pgfpathmoveto{\pgfqpoint{3.321857in}{2.782415in}}%
\pgfpathlineto{\pgfqpoint{3.334381in}{2.776750in}}%
\pgfpathlineto{\pgfqpoint{3.346909in}{2.771132in}}%
\pgfpathlineto{\pgfqpoint{3.359441in}{2.765558in}}%
\pgfpathlineto{\pgfqpoint{3.371976in}{2.760029in}}%
\pgfpathlineto{\pgfqpoint{3.364420in}{2.752383in}}%
\pgfpathlineto{\pgfqpoint{3.356858in}{2.744786in}}%
\pgfpathlineto{\pgfqpoint{3.349291in}{2.737236in}}%
\pgfpathlineto{\pgfqpoint{3.341717in}{2.729733in}}%
\pgfpathlineto{\pgfqpoint{3.329170in}{2.735242in}}%
\pgfpathlineto{\pgfqpoint{3.316627in}{2.740794in}}%
\pgfpathlineto{\pgfqpoint{3.304088in}{2.746393in}}%
\pgfpathlineto{\pgfqpoint{3.291552in}{2.752037in}}%
\pgfpathlineto{\pgfqpoint{3.299137in}{2.759555in}}%
\pgfpathlineto{\pgfqpoint{3.306717in}{2.767124in}}%
\pgfpathlineto{\pgfqpoint{3.314290in}{2.774744in}}%
\pgfpathlineto{\pgfqpoint{3.321857in}{2.782415in}}%
\pgfpathclose%
\pgfusepath{fill}%
\end{pgfscope}%
\begin{pgfscope}%
\pgfpathrectangle{\pgfqpoint{1.254980in}{0.150000in}}{\pgfqpoint{5.490039in}{5.490039in}}%
\pgfusepath{clip}%
\pgfsetbuttcap%
\pgfsetroundjoin%
\definecolor{currentfill}{rgb}{0.279566,0.067836,0.391917}%
\pgfsetfillcolor{currentfill}%
\pgfsetfillopacity{0.700000}%
\pgfsetlinewidth{0.000000pt}%
\definecolor{currentstroke}{rgb}{0.000000,0.000000,0.000000}%
\pgfsetstrokecolor{currentstroke}%
\pgfsetdash{}{0pt}%
\pgfpathmoveto{\pgfqpoint{3.191394in}{2.798888in}}%
\pgfpathlineto{\pgfqpoint{3.203902in}{2.792862in}}%
\pgfpathlineto{\pgfqpoint{3.216413in}{2.786885in}}%
\pgfpathlineto{\pgfqpoint{3.228927in}{2.780957in}}%
\pgfpathlineto{\pgfqpoint{3.241445in}{2.775078in}}%
\pgfpathlineto{\pgfqpoint{3.233842in}{2.767596in}}%
\pgfpathlineto{\pgfqpoint{3.226232in}{2.760164in}}%
\pgfpathlineto{\pgfqpoint{3.218617in}{2.752782in}}%
\pgfpathlineto{\pgfqpoint{3.210995in}{2.745450in}}%
\pgfpathlineto{\pgfqpoint{3.198465in}{2.751322in}}%
\pgfpathlineto{\pgfqpoint{3.185938in}{2.757241in}}%
\pgfpathlineto{\pgfqpoint{3.173415in}{2.763210in}}%
\pgfpathlineto{\pgfqpoint{3.160895in}{2.769228in}}%
\pgfpathlineto{\pgfqpoint{3.168530in}{2.776564in}}%
\pgfpathlineto{\pgfqpoint{3.176158in}{2.783952in}}%
\pgfpathlineto{\pgfqpoint{3.183779in}{2.791393in}}%
\pgfpathlineto{\pgfqpoint{3.191394in}{2.798888in}}%
\pgfpathclose%
\pgfusepath{fill}%
\end{pgfscope}%
\begin{pgfscope}%
\pgfpathrectangle{\pgfqpoint{1.254980in}{0.150000in}}{\pgfqpoint{5.490039in}{5.490039in}}%
\pgfusepath{clip}%
\pgfsetbuttcap%
\pgfsetroundjoin%
\definecolor{currentfill}{rgb}{0.277018,0.050344,0.375715}%
\pgfsetfillcolor{currentfill}%
\pgfsetfillopacity{0.700000}%
\pgfsetlinewidth{0.000000pt}%
\definecolor{currentstroke}{rgb}{0.000000,0.000000,0.000000}%
\pgfsetstrokecolor{currentstroke}%
\pgfsetdash{}{0pt}%
\pgfpathmoveto{\pgfqpoint{4.264226in}{2.775683in}}%
\pgfpathlineto{\pgfqpoint{4.276921in}{2.771455in}}%
\pgfpathlineto{\pgfqpoint{4.289621in}{2.767257in}}%
\pgfpathlineto{\pgfqpoint{4.302327in}{2.763089in}}%
\pgfpathlineto{\pgfqpoint{4.315038in}{2.758950in}}%
\pgfpathlineto{\pgfqpoint{4.307802in}{2.750362in}}%
\pgfpathlineto{\pgfqpoint{4.300563in}{2.741861in}}%
\pgfpathlineto{\pgfqpoint{4.293319in}{2.733446in}}%
\pgfpathlineto{\pgfqpoint{4.286071in}{2.725111in}}%
\pgfpathlineto{\pgfqpoint{4.273348in}{2.729129in}}%
\pgfpathlineto{\pgfqpoint{4.260632in}{2.733176in}}%
\pgfpathlineto{\pgfqpoint{4.247920in}{2.737254in}}%
\pgfpathlineto{\pgfqpoint{4.235214in}{2.741361in}}%
\pgfpathlineto{\pgfqpoint{4.242474in}{2.749812in}}%
\pgfpathlineto{\pgfqpoint{4.249729in}{2.758347in}}%
\pgfpathlineto{\pgfqpoint{4.256980in}{2.766969in}}%
\pgfpathlineto{\pgfqpoint{4.264226in}{2.775683in}}%
\pgfpathclose%
\pgfusepath{fill}%
\end{pgfscope}%
\begin{pgfscope}%
\pgfpathrectangle{\pgfqpoint{1.254980in}{0.150000in}}{\pgfqpoint{5.490039in}{5.490039in}}%
\pgfusepath{clip}%
\pgfsetbuttcap%
\pgfsetroundjoin%
\definecolor{currentfill}{rgb}{0.277018,0.050344,0.375715}%
\pgfsetfillcolor{currentfill}%
\pgfsetfillopacity{0.700000}%
\pgfsetlinewidth{0.000000pt}%
\definecolor{currentstroke}{rgb}{0.000000,0.000000,0.000000}%
\pgfsetstrokecolor{currentstroke}%
\pgfsetdash{}{0pt}%
\pgfpathmoveto{\pgfqpoint{3.452275in}{2.769321in}}%
\pgfpathlineto{\pgfqpoint{3.464818in}{2.763977in}}%
\pgfpathlineto{\pgfqpoint{3.477366in}{2.758675in}}%
\pgfpathlineto{\pgfqpoint{3.489918in}{2.753415in}}%
\pgfpathlineto{\pgfqpoint{3.502474in}{2.748197in}}%
\pgfpathlineto{\pgfqpoint{3.494964in}{2.740417in}}%
\pgfpathlineto{\pgfqpoint{3.487448in}{2.732686in}}%
\pgfpathlineto{\pgfqpoint{3.479927in}{2.725002in}}%
\pgfpathlineto{\pgfqpoint{3.472400in}{2.717362in}}%
\pgfpathlineto{\pgfqpoint{3.459833in}{2.722547in}}%
\pgfpathlineto{\pgfqpoint{3.447270in}{2.727774in}}%
\pgfpathlineto{\pgfqpoint{3.434711in}{2.733043in}}%
\pgfpathlineto{\pgfqpoint{3.422156in}{2.738354in}}%
\pgfpathlineto{\pgfqpoint{3.429694in}{2.746022in}}%
\pgfpathlineto{\pgfqpoint{3.437227in}{2.753738in}}%
\pgfpathlineto{\pgfqpoint{3.444754in}{2.761504in}}%
\pgfpathlineto{\pgfqpoint{3.452275in}{2.769321in}}%
\pgfpathclose%
\pgfusepath{fill}%
\end{pgfscope}%
\begin{pgfscope}%
\pgfpathrectangle{\pgfqpoint{1.254980in}{0.150000in}}{\pgfqpoint{5.490039in}{5.490039in}}%
\pgfusepath{clip}%
\pgfsetbuttcap%
\pgfsetroundjoin%
\definecolor{currentfill}{rgb}{0.276022,0.044167,0.370164}%
\pgfsetfillcolor{currentfill}%
\pgfsetfillopacity{0.700000}%
\pgfsetlinewidth{0.000000pt}%
\definecolor{currentstroke}{rgb}{0.000000,0.000000,0.000000}%
\pgfsetstrokecolor{currentstroke}%
\pgfsetdash{}{0pt}%
\pgfpathmoveto{\pgfqpoint{3.582677in}{2.759181in}}%
\pgfpathlineto{\pgfqpoint{3.595243in}{2.754120in}}%
\pgfpathlineto{\pgfqpoint{3.607813in}{2.749098in}}%
\pgfpathlineto{\pgfqpoint{3.620388in}{2.744115in}}%
\pgfpathlineto{\pgfqpoint{3.632967in}{2.739171in}}%
\pgfpathlineto{\pgfqpoint{3.625502in}{2.731283in}}%
\pgfpathlineto{\pgfqpoint{3.618031in}{2.723444in}}%
\pgfpathlineto{\pgfqpoint{3.610555in}{2.715652in}}%
\pgfpathlineto{\pgfqpoint{3.603073in}{2.707905in}}%
\pgfpathlineto{\pgfqpoint{3.590483in}{2.712803in}}%
\pgfpathlineto{\pgfqpoint{3.577897in}{2.717740in}}%
\pgfpathlineto{\pgfqpoint{3.565316in}{2.722717in}}%
\pgfpathlineto{\pgfqpoint{3.552739in}{2.727732in}}%
\pgfpathlineto{\pgfqpoint{3.560232in}{2.735520in}}%
\pgfpathlineto{\pgfqpoint{3.567719in}{2.743357in}}%
\pgfpathlineto{\pgfqpoint{3.575201in}{2.751243in}}%
\pgfpathlineto{\pgfqpoint{3.582677in}{2.759181in}}%
\pgfpathclose%
\pgfusepath{fill}%
\end{pgfscope}%
\begin{pgfscope}%
\pgfpathrectangle{\pgfqpoint{1.254980in}{0.150000in}}{\pgfqpoint{5.490039in}{5.490039in}}%
\pgfusepath{clip}%
\pgfsetbuttcap%
\pgfsetroundjoin%
\definecolor{currentfill}{rgb}{0.276022,0.044167,0.370164}%
\pgfsetfillcolor{currentfill}%
\pgfsetfillopacity{0.700000}%
\pgfsetlinewidth{0.000000pt}%
\definecolor{currentstroke}{rgb}{0.000000,0.000000,0.000000}%
\pgfsetstrokecolor{currentstroke}%
\pgfsetdash{}{0pt}%
\pgfpathmoveto{\pgfqpoint{4.053931in}{2.758706in}}%
\pgfpathlineto{\pgfqpoint{4.066586in}{2.754334in}}%
\pgfpathlineto{\pgfqpoint{4.079246in}{2.749994in}}%
\pgfpathlineto{\pgfqpoint{4.091911in}{2.745686in}}%
\pgfpathlineto{\pgfqpoint{4.104581in}{2.741410in}}%
\pgfpathlineto{\pgfqpoint{4.097276in}{2.733121in}}%
\pgfpathlineto{\pgfqpoint{4.089966in}{2.724901in}}%
\pgfpathlineto{\pgfqpoint{4.082650in}{2.716746in}}%
\pgfpathlineto{\pgfqpoint{4.075331in}{2.708653in}}%
\pgfpathlineto{\pgfqpoint{4.062649in}{2.712833in}}%
\pgfpathlineto{\pgfqpoint{4.049973in}{2.717046in}}%
\pgfpathlineto{\pgfqpoint{4.037303in}{2.721290in}}%
\pgfpathlineto{\pgfqpoint{4.024637in}{2.725566in}}%
\pgfpathlineto{\pgfqpoint{4.031968in}{2.733750in}}%
\pgfpathlineto{\pgfqpoint{4.039294in}{2.741999in}}%
\pgfpathlineto{\pgfqpoint{4.046615in}{2.750317in}}%
\pgfpathlineto{\pgfqpoint{4.053931in}{2.758706in}}%
\pgfpathclose%
\pgfusepath{fill}%
\end{pgfscope}%
\begin{pgfscope}%
\pgfpathrectangle{\pgfqpoint{1.254980in}{0.150000in}}{\pgfqpoint{5.490039in}{5.490039in}}%
\pgfusepath{clip}%
\pgfsetbuttcap%
\pgfsetroundjoin%
\definecolor{currentfill}{rgb}{0.279566,0.067836,0.391917}%
\pgfsetfillcolor{currentfill}%
\pgfsetfillopacity{0.700000}%
\pgfsetlinewidth{0.000000pt}%
\definecolor{currentstroke}{rgb}{0.000000,0.000000,0.000000}%
\pgfsetstrokecolor{currentstroke}%
\pgfsetdash{}{0pt}%
\pgfpathmoveto{\pgfqpoint{4.605115in}{2.801173in}}%
\pgfpathlineto{\pgfqpoint{4.617884in}{2.797080in}}%
\pgfpathlineto{\pgfqpoint{4.630659in}{2.793013in}}%
\pgfpathlineto{\pgfqpoint{4.643439in}{2.788974in}}%
\pgfpathlineto{\pgfqpoint{4.656225in}{2.784961in}}%
\pgfpathlineto{\pgfqpoint{4.649092in}{2.775783in}}%
\pgfpathlineto{\pgfqpoint{4.641957in}{2.766733in}}%
\pgfpathlineto{\pgfqpoint{4.634818in}{2.757805in}}%
\pgfpathlineto{\pgfqpoint{4.627677in}{2.748994in}}%
\pgfpathlineto{\pgfqpoint{4.614879in}{2.752849in}}%
\pgfpathlineto{\pgfqpoint{4.602087in}{2.756730in}}%
\pgfpathlineto{\pgfqpoint{4.589300in}{2.760639in}}%
\pgfpathlineto{\pgfqpoint{4.576519in}{2.764575in}}%
\pgfpathlineto{\pgfqpoint{4.583673in}{2.773539in}}%
\pgfpathlineto{\pgfqpoint{4.590824in}{2.782623in}}%
\pgfpathlineto{\pgfqpoint{4.597971in}{2.791833in}}%
\pgfpathlineto{\pgfqpoint{4.605115in}{2.801173in}}%
\pgfpathclose%
\pgfusepath{fill}%
\end{pgfscope}%
\begin{pgfscope}%
\pgfpathrectangle{\pgfqpoint{1.254980in}{0.150000in}}{\pgfqpoint{5.490039in}{5.490039in}}%
\pgfusepath{clip}%
\pgfsetbuttcap%
\pgfsetroundjoin%
\definecolor{currentfill}{rgb}{0.274952,0.037752,0.364543}%
\pgfsetfillcolor{currentfill}%
\pgfsetfillopacity{0.700000}%
\pgfsetlinewidth{0.000000pt}%
\definecolor{currentstroke}{rgb}{0.000000,0.000000,0.000000}%
\pgfsetstrokecolor{currentstroke}%
\pgfsetdash{}{0pt}%
\pgfpathmoveto{\pgfqpoint{3.713089in}{2.751623in}}%
\pgfpathlineto{\pgfqpoint{3.725680in}{2.746810in}}%
\pgfpathlineto{\pgfqpoint{3.738275in}{2.742034in}}%
\pgfpathlineto{\pgfqpoint{3.750875in}{2.737295in}}%
\pgfpathlineto{\pgfqpoint{3.763479in}{2.732592in}}%
\pgfpathlineto{\pgfqpoint{3.756058in}{2.724615in}}%
\pgfpathlineto{\pgfqpoint{3.748631in}{2.716689in}}%
\pgfpathlineto{\pgfqpoint{3.741199in}{2.708811in}}%
\pgfpathlineto{\pgfqpoint{3.733761in}{2.700980in}}%
\pgfpathlineto{\pgfqpoint{3.721146in}{2.705624in}}%
\pgfpathlineto{\pgfqpoint{3.708535in}{2.710305in}}%
\pgfpathlineto{\pgfqpoint{3.695929in}{2.715022in}}%
\pgfpathlineto{\pgfqpoint{3.683328in}{2.719777in}}%
\pgfpathlineto{\pgfqpoint{3.690776in}{2.727662in}}%
\pgfpathlineto{\pgfqpoint{3.698219in}{2.735596in}}%
\pgfpathlineto{\pgfqpoint{3.705657in}{2.743583in}}%
\pgfpathlineto{\pgfqpoint{3.713089in}{2.751623in}}%
\pgfpathclose%
\pgfusepath{fill}%
\end{pgfscope}%
\begin{pgfscope}%
\pgfpathrectangle{\pgfqpoint{1.254980in}{0.150000in}}{\pgfqpoint{5.490039in}{5.490039in}}%
\pgfusepath{clip}%
\pgfsetbuttcap%
\pgfsetroundjoin%
\definecolor{currentfill}{rgb}{0.277941,0.056324,0.381191}%
\pgfsetfillcolor{currentfill}%
\pgfsetfillopacity{0.700000}%
\pgfsetlinewidth{0.000000pt}%
\definecolor{currentstroke}{rgb}{0.000000,0.000000,0.000000}%
\pgfsetstrokecolor{currentstroke}%
\pgfsetdash{}{0pt}%
\pgfpathmoveto{\pgfqpoint{4.394788in}{2.777476in}}%
\pgfpathlineto{\pgfqpoint{4.407515in}{2.773350in}}%
\pgfpathlineto{\pgfqpoint{4.420247in}{2.769253in}}%
\pgfpathlineto{\pgfqpoint{4.432984in}{2.765184in}}%
\pgfpathlineto{\pgfqpoint{4.445727in}{2.761144in}}%
\pgfpathlineto{\pgfqpoint{4.438531in}{2.752436in}}%
\pgfpathlineto{\pgfqpoint{4.431331in}{2.743827in}}%
\pgfpathlineto{\pgfqpoint{4.424127in}{2.735312in}}%
\pgfpathlineto{\pgfqpoint{4.416919in}{2.726888in}}%
\pgfpathlineto{\pgfqpoint{4.404165in}{2.730795in}}%
\pgfpathlineto{\pgfqpoint{4.391416in}{2.734731in}}%
\pgfpathlineto{\pgfqpoint{4.378673in}{2.738695in}}%
\pgfpathlineto{\pgfqpoint{4.365935in}{2.742688in}}%
\pgfpathlineto{\pgfqpoint{4.373154in}{2.751241in}}%
\pgfpathlineto{\pgfqpoint{4.380370in}{2.759887in}}%
\pgfpathlineto{\pgfqpoint{4.387581in}{2.768630in}}%
\pgfpathlineto{\pgfqpoint{4.394788in}{2.777476in}}%
\pgfpathclose%
\pgfusepath{fill}%
\end{pgfscope}%
\begin{pgfscope}%
\pgfpathrectangle{\pgfqpoint{1.254980in}{0.150000in}}{\pgfqpoint{5.490039in}{5.490039in}}%
\pgfusepath{clip}%
\pgfsetbuttcap%
\pgfsetroundjoin%
\definecolor{currentfill}{rgb}{0.274952,0.037752,0.364543}%
\pgfsetfillcolor{currentfill}%
\pgfsetfillopacity{0.700000}%
\pgfsetlinewidth{0.000000pt}%
\definecolor{currentstroke}{rgb}{0.000000,0.000000,0.000000}%
\pgfsetstrokecolor{currentstroke}%
\pgfsetdash{}{0pt}%
\pgfpathmoveto{\pgfqpoint{3.843532in}{2.746321in}}%
\pgfpathlineto{\pgfqpoint{3.856150in}{2.741726in}}%
\pgfpathlineto{\pgfqpoint{3.868772in}{2.737165in}}%
\pgfpathlineto{\pgfqpoint{3.881399in}{2.732639in}}%
\pgfpathlineto{\pgfqpoint{3.894030in}{2.728147in}}%
\pgfpathlineto{\pgfqpoint{3.886652in}{2.720094in}}%
\pgfpathlineto{\pgfqpoint{3.879268in}{2.712095in}}%
\pgfpathlineto{\pgfqpoint{3.871879in}{2.704148in}}%
\pgfpathlineto{\pgfqpoint{3.864485in}{2.696250in}}%
\pgfpathlineto{\pgfqpoint{3.851843in}{2.700670in}}%
\pgfpathlineto{\pgfqpoint{3.839205in}{2.705126in}}%
\pgfpathlineto{\pgfqpoint{3.826572in}{2.709615in}}%
\pgfpathlineto{\pgfqpoint{3.813944in}{2.714140in}}%
\pgfpathlineto{\pgfqpoint{3.821349in}{2.722104in}}%
\pgfpathlineto{\pgfqpoint{3.828749in}{2.730121in}}%
\pgfpathlineto{\pgfqpoint{3.836143in}{2.738193in}}%
\pgfpathlineto{\pgfqpoint{3.843532in}{2.746321in}}%
\pgfpathclose%
\pgfusepath{fill}%
\end{pgfscope}%
\begin{pgfscope}%
\pgfpathrectangle{\pgfqpoint{1.254980in}{0.150000in}}{\pgfqpoint{5.490039in}{5.490039in}}%
\pgfusepath{clip}%
\pgfsetbuttcap%
\pgfsetroundjoin%
\definecolor{currentfill}{rgb}{0.276022,0.044167,0.370164}%
\pgfsetfillcolor{currentfill}%
\pgfsetfillopacity{0.700000}%
\pgfsetlinewidth{0.000000pt}%
\definecolor{currentstroke}{rgb}{0.000000,0.000000,0.000000}%
\pgfsetstrokecolor{currentstroke}%
\pgfsetdash{}{0pt}%
\pgfpathmoveto{\pgfqpoint{4.184443in}{2.758093in}}%
\pgfpathlineto{\pgfqpoint{4.197128in}{2.753864in}}%
\pgfpathlineto{\pgfqpoint{4.209818in}{2.749666in}}%
\pgfpathlineto{\pgfqpoint{4.222513in}{2.745499in}}%
\pgfpathlineto{\pgfqpoint{4.235214in}{2.741361in}}%
\pgfpathlineto{\pgfqpoint{4.227950in}{2.732990in}}%
\pgfpathlineto{\pgfqpoint{4.220681in}{2.724695in}}%
\pgfpathlineto{\pgfqpoint{4.213408in}{2.716472in}}%
\pgfpathlineto{\pgfqpoint{4.206129in}{2.708319in}}%
\pgfpathlineto{\pgfqpoint{4.193417in}{2.712348in}}%
\pgfpathlineto{\pgfqpoint{4.180711in}{2.716408in}}%
\pgfpathlineto{\pgfqpoint{4.168009in}{2.720497in}}%
\pgfpathlineto{\pgfqpoint{4.155313in}{2.724618in}}%
\pgfpathlineto{\pgfqpoint{4.162603in}{2.732875in}}%
\pgfpathlineto{\pgfqpoint{4.169887in}{2.741204in}}%
\pgfpathlineto{\pgfqpoint{4.177167in}{2.749609in}}%
\pgfpathlineto{\pgfqpoint{4.184443in}{2.758093in}}%
\pgfpathclose%
\pgfusepath{fill}%
\end{pgfscope}%
\begin{pgfscope}%
\pgfpathrectangle{\pgfqpoint{1.254980in}{0.150000in}}{\pgfqpoint{5.490039in}{5.490039in}}%
\pgfusepath{clip}%
\pgfsetbuttcap%
\pgfsetroundjoin%
\definecolor{currentfill}{rgb}{0.277941,0.056324,0.381191}%
\pgfsetfillcolor{currentfill}%
\pgfsetfillopacity{0.700000}%
\pgfsetlinewidth{0.000000pt}%
\definecolor{currentstroke}{rgb}{0.000000,0.000000,0.000000}%
\pgfsetstrokecolor{currentstroke}%
\pgfsetdash{}{0pt}%
\pgfpathmoveto{\pgfqpoint{3.241445in}{2.775078in}}%
\pgfpathlineto{\pgfqpoint{3.253967in}{2.769247in}}%
\pgfpathlineto{\pgfqpoint{3.266492in}{2.763463in}}%
\pgfpathlineto{\pgfqpoint{3.279020in}{2.757727in}}%
\pgfpathlineto{\pgfqpoint{3.291552in}{2.752037in}}%
\pgfpathlineto{\pgfqpoint{3.283960in}{2.744567in}}%
\pgfpathlineto{\pgfqpoint{3.276363in}{2.737145in}}%
\pgfpathlineto{\pgfqpoint{3.268759in}{2.729770in}}%
\pgfpathlineto{\pgfqpoint{3.261149in}{2.722441in}}%
\pgfpathlineto{\pgfqpoint{3.248605in}{2.728123in}}%
\pgfpathlineto{\pgfqpoint{3.236065in}{2.733851in}}%
\pgfpathlineto{\pgfqpoint{3.223528in}{2.739627in}}%
\pgfpathlineto{\pgfqpoint{3.210995in}{2.745450in}}%
\pgfpathlineto{\pgfqpoint{3.218617in}{2.752782in}}%
\pgfpathlineto{\pgfqpoint{3.226232in}{2.760164in}}%
\pgfpathlineto{\pgfqpoint{3.233842in}{2.767596in}}%
\pgfpathlineto{\pgfqpoint{3.241445in}{2.775078in}}%
\pgfpathclose%
\pgfusepath{fill}%
\end{pgfscope}%
\begin{pgfscope}%
\pgfpathrectangle{\pgfqpoint{1.254980in}{0.150000in}}{\pgfqpoint{5.490039in}{5.490039in}}%
\pgfusepath{clip}%
\pgfsetbuttcap%
\pgfsetroundjoin%
\definecolor{currentfill}{rgb}{0.277018,0.050344,0.375715}%
\pgfsetfillcolor{currentfill}%
\pgfsetfillopacity{0.700000}%
\pgfsetlinewidth{0.000000pt}%
\definecolor{currentstroke}{rgb}{0.000000,0.000000,0.000000}%
\pgfsetstrokecolor{currentstroke}%
\pgfsetdash{}{0pt}%
\pgfpathmoveto{\pgfqpoint{3.371976in}{2.760029in}}%
\pgfpathlineto{\pgfqpoint{3.384515in}{2.754545in}}%
\pgfpathlineto{\pgfqpoint{3.397058in}{2.749104in}}%
\pgfpathlineto{\pgfqpoint{3.409605in}{2.743708in}}%
\pgfpathlineto{\pgfqpoint{3.422156in}{2.738354in}}%
\pgfpathlineto{\pgfqpoint{3.414612in}{2.730733in}}%
\pgfpathlineto{\pgfqpoint{3.407062in}{2.723158in}}%
\pgfpathlineto{\pgfqpoint{3.399505in}{2.715628in}}%
\pgfpathlineto{\pgfqpoint{3.391943in}{2.708141in}}%
\pgfpathlineto{\pgfqpoint{3.379381in}{2.713474in}}%
\pgfpathlineto{\pgfqpoint{3.366822in}{2.718850in}}%
\pgfpathlineto{\pgfqpoint{3.354268in}{2.724270in}}%
\pgfpathlineto{\pgfqpoint{3.341717in}{2.729733in}}%
\pgfpathlineto{\pgfqpoint{3.349291in}{2.737236in}}%
\pgfpathlineto{\pgfqpoint{3.356858in}{2.744786in}}%
\pgfpathlineto{\pgfqpoint{3.364420in}{2.752383in}}%
\pgfpathlineto{\pgfqpoint{3.371976in}{2.760029in}}%
\pgfpathclose%
\pgfusepath{fill}%
\end{pgfscope}%
\begin{pgfscope}%
\pgfpathrectangle{\pgfqpoint{1.254980in}{0.150000in}}{\pgfqpoint{5.490039in}{5.490039in}}%
\pgfusepath{clip}%
\pgfsetbuttcap%
\pgfsetroundjoin%
\definecolor{currentfill}{rgb}{0.278791,0.062145,0.386592}%
\pgfsetfillcolor{currentfill}%
\pgfsetfillopacity{0.700000}%
\pgfsetlinewidth{0.000000pt}%
\definecolor{currentstroke}{rgb}{0.000000,0.000000,0.000000}%
\pgfsetstrokecolor{currentstroke}%
\pgfsetdash{}{0pt}%
\pgfpathmoveto{\pgfqpoint{4.525451in}{2.780592in}}%
\pgfpathlineto{\pgfqpoint{4.538210in}{2.776547in}}%
\pgfpathlineto{\pgfqpoint{4.550974in}{2.772529in}}%
\pgfpathlineto{\pgfqpoint{4.563744in}{2.768538in}}%
\pgfpathlineto{\pgfqpoint{4.576519in}{2.764575in}}%
\pgfpathlineto{\pgfqpoint{4.569362in}{2.755727in}}%
\pgfpathlineto{\pgfqpoint{4.562201in}{2.746989in}}%
\pgfpathlineto{\pgfqpoint{4.555037in}{2.738358in}}%
\pgfpathlineto{\pgfqpoint{4.547868in}{2.729829in}}%
\pgfpathlineto{\pgfqpoint{4.535081in}{2.733646in}}%
\pgfpathlineto{\pgfqpoint{4.522299in}{2.737491in}}%
\pgfpathlineto{\pgfqpoint{4.509523in}{2.741364in}}%
\pgfpathlineto{\pgfqpoint{4.496753in}{2.745264in}}%
\pgfpathlineto{\pgfqpoint{4.503933in}{2.753934in}}%
\pgfpathlineto{\pgfqpoint{4.511109in}{2.762709in}}%
\pgfpathlineto{\pgfqpoint{4.518282in}{2.771594in}}%
\pgfpathlineto{\pgfqpoint{4.525451in}{2.780592in}}%
\pgfpathclose%
\pgfusepath{fill}%
\end{pgfscope}%
\begin{pgfscope}%
\pgfpathrectangle{\pgfqpoint{1.254980in}{0.150000in}}{\pgfqpoint{5.490039in}{5.490039in}}%
\pgfusepath{clip}%
\pgfsetbuttcap%
\pgfsetroundjoin%
\definecolor{currentfill}{rgb}{0.274952,0.037752,0.364543}%
\pgfsetfillcolor{currentfill}%
\pgfsetfillopacity{0.700000}%
\pgfsetlinewidth{0.000000pt}%
\definecolor{currentstroke}{rgb}{0.000000,0.000000,0.000000}%
\pgfsetstrokecolor{currentstroke}%
\pgfsetdash{}{0pt}%
\pgfpathmoveto{\pgfqpoint{3.974025in}{2.742997in}}%
\pgfpathlineto{\pgfqpoint{3.986670in}{2.738590in}}%
\pgfpathlineto{\pgfqpoint{3.999321in}{2.734216in}}%
\pgfpathlineto{\pgfqpoint{4.011976in}{2.729875in}}%
\pgfpathlineto{\pgfqpoint{4.024637in}{2.725566in}}%
\pgfpathlineto{\pgfqpoint{4.017301in}{2.717444in}}%
\pgfpathlineto{\pgfqpoint{4.009960in}{2.709382in}}%
\pgfpathlineto{\pgfqpoint{4.002614in}{2.701375in}}%
\pgfpathlineto{\pgfqpoint{3.995263in}{2.693422in}}%
\pgfpathlineto{\pgfqpoint{3.982591in}{2.697648in}}%
\pgfpathlineto{\pgfqpoint{3.969924in}{2.701905in}}%
\pgfpathlineto{\pgfqpoint{3.957263in}{2.706196in}}%
\pgfpathlineto{\pgfqpoint{3.944606in}{2.710519in}}%
\pgfpathlineto{\pgfqpoint{3.951969in}{2.718551in}}%
\pgfpathlineto{\pgfqpoint{3.959326in}{2.726639in}}%
\pgfpathlineto{\pgfqpoint{3.966678in}{2.734787in}}%
\pgfpathlineto{\pgfqpoint{3.974025in}{2.742997in}}%
\pgfpathclose%
\pgfusepath{fill}%
\end{pgfscope}%
\begin{pgfscope}%
\pgfpathrectangle{\pgfqpoint{1.254980in}{0.150000in}}{\pgfqpoint{5.490039in}{5.490039in}}%
\pgfusepath{clip}%
\pgfsetbuttcap%
\pgfsetroundjoin%
\definecolor{currentfill}{rgb}{0.276022,0.044167,0.370164}%
\pgfsetfillcolor{currentfill}%
\pgfsetfillopacity{0.700000}%
\pgfsetlinewidth{0.000000pt}%
\definecolor{currentstroke}{rgb}{0.000000,0.000000,0.000000}%
\pgfsetstrokecolor{currentstroke}%
\pgfsetdash{}{0pt}%
\pgfpathmoveto{\pgfqpoint{3.502474in}{2.748197in}}%
\pgfpathlineto{\pgfqpoint{3.515034in}{2.743020in}}%
\pgfpathlineto{\pgfqpoint{3.527598in}{2.737883in}}%
\pgfpathlineto{\pgfqpoint{3.540166in}{2.732788in}}%
\pgfpathlineto{\pgfqpoint{3.552739in}{2.727732in}}%
\pgfpathlineto{\pgfqpoint{3.545241in}{2.719991in}}%
\pgfpathlineto{\pgfqpoint{3.537736in}{2.712295in}}%
\pgfpathlineto{\pgfqpoint{3.530226in}{2.704642in}}%
\pgfpathlineto{\pgfqpoint{3.522711in}{2.697031in}}%
\pgfpathlineto{\pgfqpoint{3.510127in}{2.702053in}}%
\pgfpathlineto{\pgfqpoint{3.497547in}{2.707116in}}%
\pgfpathlineto{\pgfqpoint{3.484971in}{2.712219in}}%
\pgfpathlineto{\pgfqpoint{3.472400in}{2.717362in}}%
\pgfpathlineto{\pgfqpoint{3.479927in}{2.725002in}}%
\pgfpathlineto{\pgfqpoint{3.487448in}{2.732686in}}%
\pgfpathlineto{\pgfqpoint{3.494964in}{2.740417in}}%
\pgfpathlineto{\pgfqpoint{3.502474in}{2.748197in}}%
\pgfpathclose%
\pgfusepath{fill}%
\end{pgfscope}%
\begin{pgfscope}%
\pgfpathrectangle{\pgfqpoint{1.254980in}{0.150000in}}{\pgfqpoint{5.490039in}{5.490039in}}%
\pgfusepath{clip}%
\pgfsetbuttcap%
\pgfsetroundjoin%
\definecolor{currentfill}{rgb}{0.277018,0.050344,0.375715}%
\pgfsetfillcolor{currentfill}%
\pgfsetfillopacity{0.700000}%
\pgfsetlinewidth{0.000000pt}%
\definecolor{currentstroke}{rgb}{0.000000,0.000000,0.000000}%
\pgfsetstrokecolor{currentstroke}%
\pgfsetdash{}{0pt}%
\pgfpathmoveto{\pgfqpoint{4.315038in}{2.758950in}}%
\pgfpathlineto{\pgfqpoint{4.327754in}{2.754841in}}%
\pgfpathlineto{\pgfqpoint{4.340475in}{2.750761in}}%
\pgfpathlineto{\pgfqpoint{4.353202in}{2.746710in}}%
\pgfpathlineto{\pgfqpoint{4.365935in}{2.742688in}}%
\pgfpathlineto{\pgfqpoint{4.358711in}{2.734225in}}%
\pgfpathlineto{\pgfqpoint{4.351483in}{2.725847in}}%
\pgfpathlineto{\pgfqpoint{4.344251in}{2.717551in}}%
\pgfpathlineto{\pgfqpoint{4.337014in}{2.709332in}}%
\pgfpathlineto{\pgfqpoint{4.324270in}{2.713233in}}%
\pgfpathlineto{\pgfqpoint{4.311531in}{2.717163in}}%
\pgfpathlineto{\pgfqpoint{4.298798in}{2.721122in}}%
\pgfpathlineto{\pgfqpoint{4.286071in}{2.725111in}}%
\pgfpathlineto{\pgfqpoint{4.293319in}{2.733446in}}%
\pgfpathlineto{\pgfqpoint{4.300563in}{2.741861in}}%
\pgfpathlineto{\pgfqpoint{4.307802in}{2.750362in}}%
\pgfpathlineto{\pgfqpoint{4.315038in}{2.758950in}}%
\pgfpathclose%
\pgfusepath{fill}%
\end{pgfscope}%
\begin{pgfscope}%
\pgfpathrectangle{\pgfqpoint{1.254980in}{0.150000in}}{\pgfqpoint{5.490039in}{5.490039in}}%
\pgfusepath{clip}%
\pgfsetbuttcap%
\pgfsetroundjoin%
\definecolor{currentfill}{rgb}{0.274952,0.037752,0.364543}%
\pgfsetfillcolor{currentfill}%
\pgfsetfillopacity{0.700000}%
\pgfsetlinewidth{0.000000pt}%
\definecolor{currentstroke}{rgb}{0.000000,0.000000,0.000000}%
\pgfsetstrokecolor{currentstroke}%
\pgfsetdash{}{0pt}%
\pgfpathmoveto{\pgfqpoint{3.632967in}{2.739171in}}%
\pgfpathlineto{\pgfqpoint{3.645550in}{2.734265in}}%
\pgfpathlineto{\pgfqpoint{3.658138in}{2.729398in}}%
\pgfpathlineto{\pgfqpoint{3.670731in}{2.724569in}}%
\pgfpathlineto{\pgfqpoint{3.683328in}{2.719777in}}%
\pgfpathlineto{\pgfqpoint{3.675874in}{2.711940in}}%
\pgfpathlineto{\pgfqpoint{3.668414in}{2.704148in}}%
\pgfpathlineto{\pgfqpoint{3.660949in}{2.696400in}}%
\pgfpathlineto{\pgfqpoint{3.653478in}{2.688695in}}%
\pgfpathlineto{\pgfqpoint{3.640870in}{2.693441in}}%
\pgfpathlineto{\pgfqpoint{3.628267in}{2.698224in}}%
\pgfpathlineto{\pgfqpoint{3.615668in}{2.703046in}}%
\pgfpathlineto{\pgfqpoint{3.603073in}{2.707905in}}%
\pgfpathlineto{\pgfqpoint{3.610555in}{2.715652in}}%
\pgfpathlineto{\pgfqpoint{3.618031in}{2.723444in}}%
\pgfpathlineto{\pgfqpoint{3.625502in}{2.731283in}}%
\pgfpathlineto{\pgfqpoint{3.632967in}{2.739171in}}%
\pgfpathclose%
\pgfusepath{fill}%
\end{pgfscope}%
\begin{pgfscope}%
\pgfpathrectangle{\pgfqpoint{1.254980in}{0.150000in}}{\pgfqpoint{5.490039in}{5.490039in}}%
\pgfusepath{clip}%
\pgfsetbuttcap%
\pgfsetroundjoin%
\definecolor{currentfill}{rgb}{0.276022,0.044167,0.370164}%
\pgfsetfillcolor{currentfill}%
\pgfsetfillopacity{0.700000}%
\pgfsetlinewidth{0.000000pt}%
\definecolor{currentstroke}{rgb}{0.000000,0.000000,0.000000}%
\pgfsetstrokecolor{currentstroke}%
\pgfsetdash{}{0pt}%
\pgfpathmoveto{\pgfqpoint{4.104581in}{2.741410in}}%
\pgfpathlineto{\pgfqpoint{4.117256in}{2.737165in}}%
\pgfpathlineto{\pgfqpoint{4.129937in}{2.732951in}}%
\pgfpathlineto{\pgfqpoint{4.142622in}{2.728769in}}%
\pgfpathlineto{\pgfqpoint{4.155313in}{2.724618in}}%
\pgfpathlineto{\pgfqpoint{4.148019in}{2.716430in}}%
\pgfpathlineto{\pgfqpoint{4.140720in}{2.708307in}}%
\pgfpathlineto{\pgfqpoint{4.133416in}{2.700246in}}%
\pgfpathlineto{\pgfqpoint{4.126107in}{2.692245in}}%
\pgfpathlineto{\pgfqpoint{4.113405in}{2.696300in}}%
\pgfpathlineto{\pgfqpoint{4.100708in}{2.700387in}}%
\pgfpathlineto{\pgfqpoint{4.088017in}{2.704504in}}%
\pgfpathlineto{\pgfqpoint{4.075331in}{2.708653in}}%
\pgfpathlineto{\pgfqpoint{4.082650in}{2.716746in}}%
\pgfpathlineto{\pgfqpoint{4.089966in}{2.724901in}}%
\pgfpathlineto{\pgfqpoint{4.097276in}{2.733121in}}%
\pgfpathlineto{\pgfqpoint{4.104581in}{2.741410in}}%
\pgfpathclose%
\pgfusepath{fill}%
\end{pgfscope}%
\begin{pgfscope}%
\pgfpathrectangle{\pgfqpoint{1.254980in}{0.150000in}}{\pgfqpoint{5.490039in}{5.490039in}}%
\pgfusepath{clip}%
\pgfsetbuttcap%
\pgfsetroundjoin%
\definecolor{currentfill}{rgb}{0.274952,0.037752,0.364543}%
\pgfsetfillcolor{currentfill}%
\pgfsetfillopacity{0.700000}%
\pgfsetlinewidth{0.000000pt}%
\definecolor{currentstroke}{rgb}{0.000000,0.000000,0.000000}%
\pgfsetstrokecolor{currentstroke}%
\pgfsetdash{}{0pt}%
\pgfpathmoveto{\pgfqpoint{3.763479in}{2.732592in}}%
\pgfpathlineto{\pgfqpoint{3.776088in}{2.727926in}}%
\pgfpathlineto{\pgfqpoint{3.788702in}{2.723295in}}%
\pgfpathlineto{\pgfqpoint{3.801321in}{2.718700in}}%
\pgfpathlineto{\pgfqpoint{3.813944in}{2.714140in}}%
\pgfpathlineto{\pgfqpoint{3.806534in}{2.706226in}}%
\pgfpathlineto{\pgfqpoint{3.799118in}{2.698359in}}%
\pgfpathlineto{\pgfqpoint{3.791697in}{2.690539in}}%
\pgfpathlineto{\pgfqpoint{3.784270in}{2.682761in}}%
\pgfpathlineto{\pgfqpoint{3.771636in}{2.687263in}}%
\pgfpathlineto{\pgfqpoint{3.759006in}{2.691799in}}%
\pgfpathlineto{\pgfqpoint{3.746381in}{2.696371in}}%
\pgfpathlineto{\pgfqpoint{3.733761in}{2.700980in}}%
\pgfpathlineto{\pgfqpoint{3.741199in}{2.708811in}}%
\pgfpathlineto{\pgfqpoint{3.748631in}{2.716689in}}%
\pgfpathlineto{\pgfqpoint{3.756058in}{2.724615in}}%
\pgfpathlineto{\pgfqpoint{3.763479in}{2.732592in}}%
\pgfpathclose%
\pgfusepath{fill}%
\end{pgfscope}%
\begin{pgfscope}%
\pgfpathrectangle{\pgfqpoint{1.254980in}{0.150000in}}{\pgfqpoint{5.490039in}{5.490039in}}%
\pgfusepath{clip}%
\pgfsetbuttcap%
\pgfsetroundjoin%
\definecolor{currentfill}{rgb}{0.279566,0.067836,0.391917}%
\pgfsetfillcolor{currentfill}%
\pgfsetfillopacity{0.700000}%
\pgfsetlinewidth{0.000000pt}%
\definecolor{currentstroke}{rgb}{0.000000,0.000000,0.000000}%
\pgfsetstrokecolor{currentstroke}%
\pgfsetdash{}{0pt}%
\pgfpathmoveto{\pgfqpoint{4.656225in}{2.784961in}}%
\pgfpathlineto{\pgfqpoint{4.669016in}{2.780975in}}%
\pgfpathlineto{\pgfqpoint{4.681813in}{2.777016in}}%
\pgfpathlineto{\pgfqpoint{4.694615in}{2.773084in}}%
\pgfpathlineto{\pgfqpoint{4.707424in}{2.769178in}}%
\pgfpathlineto{\pgfqpoint{4.700304in}{2.760162in}}%
\pgfpathlineto{\pgfqpoint{4.693181in}{2.751271in}}%
\pgfpathlineto{\pgfqpoint{4.686055in}{2.742500in}}%
\pgfpathlineto{\pgfqpoint{4.678926in}{2.733843in}}%
\pgfpathlineto{\pgfqpoint{4.666105in}{2.737590in}}%
\pgfpathlineto{\pgfqpoint{4.653290in}{2.741365in}}%
\pgfpathlineto{\pgfqpoint{4.640480in}{2.745166in}}%
\pgfpathlineto{\pgfqpoint{4.627677in}{2.748994in}}%
\pgfpathlineto{\pgfqpoint{4.634818in}{2.757805in}}%
\pgfpathlineto{\pgfqpoint{4.641957in}{2.766733in}}%
\pgfpathlineto{\pgfqpoint{4.649092in}{2.775783in}}%
\pgfpathlineto{\pgfqpoint{4.656225in}{2.784961in}}%
\pgfpathclose%
\pgfusepath{fill}%
\end{pgfscope}%
\begin{pgfscope}%
\pgfpathrectangle{\pgfqpoint{1.254980in}{0.150000in}}{\pgfqpoint{5.490039in}{5.490039in}}%
\pgfusepath{clip}%
\pgfsetbuttcap%
\pgfsetroundjoin%
\definecolor{currentfill}{rgb}{0.277941,0.056324,0.381191}%
\pgfsetfillcolor{currentfill}%
\pgfsetfillopacity{0.700000}%
\pgfsetlinewidth{0.000000pt}%
\definecolor{currentstroke}{rgb}{0.000000,0.000000,0.000000}%
\pgfsetstrokecolor{currentstroke}%
\pgfsetdash{}{0pt}%
\pgfpathmoveto{\pgfqpoint{4.445727in}{2.761144in}}%
\pgfpathlineto{\pgfqpoint{4.458475in}{2.757132in}}%
\pgfpathlineto{\pgfqpoint{4.471229in}{2.753148in}}%
\pgfpathlineto{\pgfqpoint{4.483988in}{2.749192in}}%
\pgfpathlineto{\pgfqpoint{4.496753in}{2.745264in}}%
\pgfpathlineto{\pgfqpoint{4.489569in}{2.736694in}}%
\pgfpathlineto{\pgfqpoint{4.482381in}{2.728219in}}%
\pgfpathlineto{\pgfqpoint{4.475189in}{2.719837in}}%
\pgfpathlineto{\pgfqpoint{4.467993in}{2.711542in}}%
\pgfpathlineto{\pgfqpoint{4.455216in}{2.715336in}}%
\pgfpathlineto{\pgfqpoint{4.442445in}{2.719159in}}%
\pgfpathlineto{\pgfqpoint{4.429679in}{2.723009in}}%
\pgfpathlineto{\pgfqpoint{4.416919in}{2.726888in}}%
\pgfpathlineto{\pgfqpoint{4.424127in}{2.735312in}}%
\pgfpathlineto{\pgfqpoint{4.431331in}{2.743827in}}%
\pgfpathlineto{\pgfqpoint{4.438531in}{2.752436in}}%
\pgfpathlineto{\pgfqpoint{4.445727in}{2.761144in}}%
\pgfpathclose%
\pgfusepath{fill}%
\end{pgfscope}%
\begin{pgfscope}%
\pgfpathrectangle{\pgfqpoint{1.254980in}{0.150000in}}{\pgfqpoint{5.490039in}{5.490039in}}%
\pgfusepath{clip}%
\pgfsetbuttcap%
\pgfsetroundjoin%
\definecolor{currentfill}{rgb}{0.274952,0.037752,0.364543}%
\pgfsetfillcolor{currentfill}%
\pgfsetfillopacity{0.700000}%
\pgfsetlinewidth{0.000000pt}%
\definecolor{currentstroke}{rgb}{0.000000,0.000000,0.000000}%
\pgfsetstrokecolor{currentstroke}%
\pgfsetdash{}{0pt}%
\pgfpathmoveto{\pgfqpoint{3.894030in}{2.728147in}}%
\pgfpathlineto{\pgfqpoint{3.906667in}{2.723690in}}%
\pgfpathlineto{\pgfqpoint{3.919308in}{2.719266in}}%
\pgfpathlineto{\pgfqpoint{3.931955in}{2.714876in}}%
\pgfpathlineto{\pgfqpoint{3.944606in}{2.710519in}}%
\pgfpathlineto{\pgfqpoint{3.937239in}{2.702541in}}%
\pgfpathlineto{\pgfqpoint{3.929866in}{2.694615in}}%
\pgfpathlineto{\pgfqpoint{3.922489in}{2.686737in}}%
\pgfpathlineto{\pgfqpoint{3.915106in}{2.678906in}}%
\pgfpathlineto{\pgfqpoint{3.902443in}{2.683191in}}%
\pgfpathlineto{\pgfqpoint{3.889786in}{2.687510in}}%
\pgfpathlineto{\pgfqpoint{3.877133in}{2.691863in}}%
\pgfpathlineto{\pgfqpoint{3.864485in}{2.696250in}}%
\pgfpathlineto{\pgfqpoint{3.871879in}{2.704148in}}%
\pgfpathlineto{\pgfqpoint{3.879268in}{2.712095in}}%
\pgfpathlineto{\pgfqpoint{3.886652in}{2.720094in}}%
\pgfpathlineto{\pgfqpoint{3.894030in}{2.728147in}}%
\pgfpathclose%
\pgfusepath{fill}%
\end{pgfscope}%
\begin{pgfscope}%
\pgfpathrectangle{\pgfqpoint{1.254980in}{0.150000in}}{\pgfqpoint{5.490039in}{5.490039in}}%
\pgfusepath{clip}%
\pgfsetbuttcap%
\pgfsetroundjoin%
\definecolor{currentfill}{rgb}{0.276022,0.044167,0.370164}%
\pgfsetfillcolor{currentfill}%
\pgfsetfillopacity{0.700000}%
\pgfsetlinewidth{0.000000pt}%
\definecolor{currentstroke}{rgb}{0.000000,0.000000,0.000000}%
\pgfsetstrokecolor{currentstroke}%
\pgfsetdash{}{0pt}%
\pgfpathmoveto{\pgfqpoint{4.235214in}{2.741361in}}%
\pgfpathlineto{\pgfqpoint{4.247920in}{2.737254in}}%
\pgfpathlineto{\pgfqpoint{4.260632in}{2.733176in}}%
\pgfpathlineto{\pgfqpoint{4.273348in}{2.729129in}}%
\pgfpathlineto{\pgfqpoint{4.286071in}{2.725111in}}%
\pgfpathlineto{\pgfqpoint{4.278818in}{2.716852in}}%
\pgfpathlineto{\pgfqpoint{4.271560in}{2.708667in}}%
\pgfpathlineto{\pgfqpoint{4.264298in}{2.700552in}}%
\pgfpathlineto{\pgfqpoint{4.257032in}{2.692503in}}%
\pgfpathlineto{\pgfqpoint{4.244298in}{2.696412in}}%
\pgfpathlineto{\pgfqpoint{4.231570in}{2.700351in}}%
\pgfpathlineto{\pgfqpoint{4.218847in}{2.704320in}}%
\pgfpathlineto{\pgfqpoint{4.206129in}{2.708319in}}%
\pgfpathlineto{\pgfqpoint{4.213408in}{2.716472in}}%
\pgfpathlineto{\pgfqpoint{4.220681in}{2.724695in}}%
\pgfpathlineto{\pgfqpoint{4.227950in}{2.732990in}}%
\pgfpathlineto{\pgfqpoint{4.235214in}{2.741361in}}%
\pgfpathclose%
\pgfusepath{fill}%
\end{pgfscope}%
\begin{pgfscope}%
\pgfpathrectangle{\pgfqpoint{1.254980in}{0.150000in}}{\pgfqpoint{5.490039in}{5.490039in}}%
\pgfusepath{clip}%
\pgfsetbuttcap%
\pgfsetroundjoin%
\definecolor{currentfill}{rgb}{0.277018,0.050344,0.375715}%
\pgfsetfillcolor{currentfill}%
\pgfsetfillopacity{0.700000}%
\pgfsetlinewidth{0.000000pt}%
\definecolor{currentstroke}{rgb}{0.000000,0.000000,0.000000}%
\pgfsetstrokecolor{currentstroke}%
\pgfsetdash{}{0pt}%
\pgfpathmoveto{\pgfqpoint{3.291552in}{2.752037in}}%
\pgfpathlineto{\pgfqpoint{3.304088in}{2.746393in}}%
\pgfpathlineto{\pgfqpoint{3.316627in}{2.740794in}}%
\pgfpathlineto{\pgfqpoint{3.329170in}{2.735242in}}%
\pgfpathlineto{\pgfqpoint{3.341717in}{2.729733in}}%
\pgfpathlineto{\pgfqpoint{3.334137in}{2.722276in}}%
\pgfpathlineto{\pgfqpoint{3.326552in}{2.714864in}}%
\pgfpathlineto{\pgfqpoint{3.318960in}{2.707496in}}%
\pgfpathlineto{\pgfqpoint{3.311362in}{2.700171in}}%
\pgfpathlineto{\pgfqpoint{3.298803in}{2.705671in}}%
\pgfpathlineto{\pgfqpoint{3.286248in}{2.711215in}}%
\pgfpathlineto{\pgfqpoint{3.273697in}{2.716805in}}%
\pgfpathlineto{\pgfqpoint{3.261149in}{2.722441in}}%
\pgfpathlineto{\pgfqpoint{3.268759in}{2.729770in}}%
\pgfpathlineto{\pgfqpoint{3.276363in}{2.737145in}}%
\pgfpathlineto{\pgfqpoint{3.283960in}{2.744567in}}%
\pgfpathlineto{\pgfqpoint{3.291552in}{2.752037in}}%
\pgfpathclose%
\pgfusepath{fill}%
\end{pgfscope}%
\begin{pgfscope}%
\pgfpathrectangle{\pgfqpoint{1.254980in}{0.150000in}}{\pgfqpoint{5.490039in}{5.490039in}}%
\pgfusepath{clip}%
\pgfsetbuttcap%
\pgfsetroundjoin%
\definecolor{currentfill}{rgb}{0.278791,0.062145,0.386592}%
\pgfsetfillcolor{currentfill}%
\pgfsetfillopacity{0.700000}%
\pgfsetlinewidth{0.000000pt}%
\definecolor{currentstroke}{rgb}{0.000000,0.000000,0.000000}%
\pgfsetstrokecolor{currentstroke}%
\pgfsetdash{}{0pt}%
\pgfpathmoveto{\pgfqpoint{3.160895in}{2.769228in}}%
\pgfpathlineto{\pgfqpoint{3.173415in}{2.763210in}}%
\pgfpathlineto{\pgfqpoint{3.185938in}{2.757241in}}%
\pgfpathlineto{\pgfqpoint{3.198465in}{2.751322in}}%
\pgfpathlineto{\pgfqpoint{3.210995in}{2.745450in}}%
\pgfpathlineto{\pgfqpoint{3.203366in}{2.738167in}}%
\pgfpathlineto{\pgfqpoint{3.195731in}{2.730933in}}%
\pgfpathlineto{\pgfqpoint{3.188090in}{2.723746in}}%
\pgfpathlineto{\pgfqpoint{3.180443in}{2.716606in}}%
\pgfpathlineto{\pgfqpoint{3.167901in}{2.722482in}}%
\pgfpathlineto{\pgfqpoint{3.155362in}{2.728406in}}%
\pgfpathlineto{\pgfqpoint{3.142826in}{2.734380in}}%
\pgfpathlineto{\pgfqpoint{3.130294in}{2.740402in}}%
\pgfpathlineto{\pgfqpoint{3.137954in}{2.747533in}}%
\pgfpathlineto{\pgfqpoint{3.145608in}{2.754713in}}%
\pgfpathlineto{\pgfqpoint{3.153255in}{2.761945in}}%
\pgfpathlineto{\pgfqpoint{3.160895in}{2.769228in}}%
\pgfpathclose%
\pgfusepath{fill}%
\end{pgfscope}%
\begin{pgfscope}%
\pgfpathrectangle{\pgfqpoint{1.254980in}{0.150000in}}{\pgfqpoint{5.490039in}{5.490039in}}%
\pgfusepath{clip}%
\pgfsetbuttcap%
\pgfsetroundjoin%
\definecolor{currentfill}{rgb}{0.276022,0.044167,0.370164}%
\pgfsetfillcolor{currentfill}%
\pgfsetfillopacity{0.700000}%
\pgfsetlinewidth{0.000000pt}%
\definecolor{currentstroke}{rgb}{0.000000,0.000000,0.000000}%
\pgfsetstrokecolor{currentstroke}%
\pgfsetdash{}{0pt}%
\pgfpathmoveto{\pgfqpoint{3.422156in}{2.738354in}}%
\pgfpathlineto{\pgfqpoint{3.434711in}{2.733043in}}%
\pgfpathlineto{\pgfqpoint{3.447270in}{2.727774in}}%
\pgfpathlineto{\pgfqpoint{3.459833in}{2.722547in}}%
\pgfpathlineto{\pgfqpoint{3.472400in}{2.717362in}}%
\pgfpathlineto{\pgfqpoint{3.464867in}{2.709767in}}%
\pgfpathlineto{\pgfqpoint{3.457328in}{2.702215in}}%
\pgfpathlineto{\pgfqpoint{3.449784in}{2.694703in}}%
\pgfpathlineto{\pgfqpoint{3.442234in}{2.687233in}}%
\pgfpathlineto{\pgfqpoint{3.429655in}{2.692397in}}%
\pgfpathlineto{\pgfqpoint{3.417080in}{2.697603in}}%
\pgfpathlineto{\pgfqpoint{3.404510in}{2.702851in}}%
\pgfpathlineto{\pgfqpoint{3.391943in}{2.708141in}}%
\pgfpathlineto{\pgfqpoint{3.399505in}{2.715628in}}%
\pgfpathlineto{\pgfqpoint{3.407062in}{2.723158in}}%
\pgfpathlineto{\pgfqpoint{3.414612in}{2.730733in}}%
\pgfpathlineto{\pgfqpoint{3.422156in}{2.738354in}}%
\pgfpathclose%
\pgfusepath{fill}%
\end{pgfscope}%
\begin{pgfscope}%
\pgfpathrectangle{\pgfqpoint{1.254980in}{0.150000in}}{\pgfqpoint{5.490039in}{5.490039in}}%
\pgfusepath{clip}%
\pgfsetbuttcap%
\pgfsetroundjoin%
\definecolor{currentfill}{rgb}{0.274952,0.037752,0.364543}%
\pgfsetfillcolor{currentfill}%
\pgfsetfillopacity{0.700000}%
\pgfsetlinewidth{0.000000pt}%
\definecolor{currentstroke}{rgb}{0.000000,0.000000,0.000000}%
\pgfsetstrokecolor{currentstroke}%
\pgfsetdash{}{0pt}%
\pgfpathmoveto{\pgfqpoint{3.552739in}{2.727732in}}%
\pgfpathlineto{\pgfqpoint{3.565316in}{2.722717in}}%
\pgfpathlineto{\pgfqpoint{3.577897in}{2.717740in}}%
\pgfpathlineto{\pgfqpoint{3.590483in}{2.712803in}}%
\pgfpathlineto{\pgfqpoint{3.603073in}{2.707905in}}%
\pgfpathlineto{\pgfqpoint{3.595586in}{2.700202in}}%
\pgfpathlineto{\pgfqpoint{3.588093in}{2.692541in}}%
\pgfpathlineto{\pgfqpoint{3.580594in}{2.684920in}}%
\pgfpathlineto{\pgfqpoint{3.573090in}{2.677338in}}%
\pgfpathlineto{\pgfqpoint{3.560488in}{2.682203in}}%
\pgfpathlineto{\pgfqpoint{3.547891in}{2.687106in}}%
\pgfpathlineto{\pgfqpoint{3.535299in}{2.692049in}}%
\pgfpathlineto{\pgfqpoint{3.522711in}{2.697031in}}%
\pgfpathlineto{\pgfqpoint{3.530226in}{2.704642in}}%
\pgfpathlineto{\pgfqpoint{3.537736in}{2.712295in}}%
\pgfpathlineto{\pgfqpoint{3.545241in}{2.719991in}}%
\pgfpathlineto{\pgfqpoint{3.552739in}{2.727732in}}%
\pgfpathclose%
\pgfusepath{fill}%
\end{pgfscope}%
\begin{pgfscope}%
\pgfpathrectangle{\pgfqpoint{1.254980in}{0.150000in}}{\pgfqpoint{5.490039in}{5.490039in}}%
\pgfusepath{clip}%
\pgfsetbuttcap%
\pgfsetroundjoin%
\definecolor{currentfill}{rgb}{0.278791,0.062145,0.386592}%
\pgfsetfillcolor{currentfill}%
\pgfsetfillopacity{0.700000}%
\pgfsetlinewidth{0.000000pt}%
\definecolor{currentstroke}{rgb}{0.000000,0.000000,0.000000}%
\pgfsetstrokecolor{currentstroke}%
\pgfsetdash{}{0pt}%
\pgfpathmoveto{\pgfqpoint{4.576519in}{2.764575in}}%
\pgfpathlineto{\pgfqpoint{4.589300in}{2.760639in}}%
\pgfpathlineto{\pgfqpoint{4.602087in}{2.756730in}}%
\pgfpathlineto{\pgfqpoint{4.614879in}{2.752849in}}%
\pgfpathlineto{\pgfqpoint{4.627677in}{2.748994in}}%
\pgfpathlineto{\pgfqpoint{4.620532in}{2.740296in}}%
\pgfpathlineto{\pgfqpoint{4.613383in}{2.731706in}}%
\pgfpathlineto{\pgfqpoint{4.606231in}{2.723219in}}%
\pgfpathlineto{\pgfqpoint{4.599075in}{2.714831in}}%
\pgfpathlineto{\pgfqpoint{4.586265in}{2.718540in}}%
\pgfpathlineto{\pgfqpoint{4.573460in}{2.722275in}}%
\pgfpathlineto{\pgfqpoint{4.560662in}{2.726038in}}%
\pgfpathlineto{\pgfqpoint{4.547868in}{2.729829in}}%
\pgfpathlineto{\pgfqpoint{4.555037in}{2.738358in}}%
\pgfpathlineto{\pgfqpoint{4.562201in}{2.746989in}}%
\pgfpathlineto{\pgfqpoint{4.569362in}{2.755727in}}%
\pgfpathlineto{\pgfqpoint{4.576519in}{2.764575in}}%
\pgfpathclose%
\pgfusepath{fill}%
\end{pgfscope}%
\begin{pgfscope}%
\pgfpathrectangle{\pgfqpoint{1.254980in}{0.150000in}}{\pgfqpoint{5.490039in}{5.490039in}}%
\pgfusepath{clip}%
\pgfsetbuttcap%
\pgfsetroundjoin%
\definecolor{currentfill}{rgb}{0.274952,0.037752,0.364543}%
\pgfsetfillcolor{currentfill}%
\pgfsetfillopacity{0.700000}%
\pgfsetlinewidth{0.000000pt}%
\definecolor{currentstroke}{rgb}{0.000000,0.000000,0.000000}%
\pgfsetstrokecolor{currentstroke}%
\pgfsetdash{}{0pt}%
\pgfpathmoveto{\pgfqpoint{4.024637in}{2.725566in}}%
\pgfpathlineto{\pgfqpoint{4.037303in}{2.721290in}}%
\pgfpathlineto{\pgfqpoint{4.049973in}{2.717046in}}%
\pgfpathlineto{\pgfqpoint{4.062649in}{2.712833in}}%
\pgfpathlineto{\pgfqpoint{4.075331in}{2.708653in}}%
\pgfpathlineto{\pgfqpoint{4.068006in}{2.700619in}}%
\pgfpathlineto{\pgfqpoint{4.060676in}{2.692642in}}%
\pgfpathlineto{\pgfqpoint{4.053341in}{2.684718in}}%
\pgfpathlineto{\pgfqpoint{4.046001in}{2.676844in}}%
\pgfpathlineto{\pgfqpoint{4.033308in}{2.680941in}}%
\pgfpathlineto{\pgfqpoint{4.020621in}{2.685069in}}%
\pgfpathlineto{\pgfqpoint{4.007939in}{2.689230in}}%
\pgfpathlineto{\pgfqpoint{3.995263in}{2.693422in}}%
\pgfpathlineto{\pgfqpoint{4.002614in}{2.701375in}}%
\pgfpathlineto{\pgfqpoint{4.009960in}{2.709382in}}%
\pgfpathlineto{\pgfqpoint{4.017301in}{2.717444in}}%
\pgfpathlineto{\pgfqpoint{4.024637in}{2.725566in}}%
\pgfpathclose%
\pgfusepath{fill}%
\end{pgfscope}%
\begin{pgfscope}%
\pgfpathrectangle{\pgfqpoint{1.254980in}{0.150000in}}{\pgfqpoint{5.490039in}{5.490039in}}%
\pgfusepath{clip}%
\pgfsetbuttcap%
\pgfsetroundjoin%
\definecolor{currentfill}{rgb}{0.273809,0.031497,0.358853}%
\pgfsetfillcolor{currentfill}%
\pgfsetfillopacity{0.700000}%
\pgfsetlinewidth{0.000000pt}%
\definecolor{currentstroke}{rgb}{0.000000,0.000000,0.000000}%
\pgfsetstrokecolor{currentstroke}%
\pgfsetdash{}{0pt}%
\pgfpathmoveto{\pgfqpoint{3.683328in}{2.719777in}}%
\pgfpathlineto{\pgfqpoint{3.695929in}{2.715022in}}%
\pgfpathlineto{\pgfqpoint{3.708535in}{2.710305in}}%
\pgfpathlineto{\pgfqpoint{3.721146in}{2.705624in}}%
\pgfpathlineto{\pgfqpoint{3.733761in}{2.700980in}}%
\pgfpathlineto{\pgfqpoint{3.726318in}{2.693193in}}%
\pgfpathlineto{\pgfqpoint{3.718870in}{2.685449in}}%
\pgfpathlineto{\pgfqpoint{3.711416in}{2.677746in}}%
\pgfpathlineto{\pgfqpoint{3.703957in}{2.670082in}}%
\pgfpathlineto{\pgfqpoint{3.691330in}{2.674680in}}%
\pgfpathlineto{\pgfqpoint{3.678708in}{2.679315in}}%
\pgfpathlineto{\pgfqpoint{3.666091in}{2.683986in}}%
\pgfpathlineto{\pgfqpoint{3.653478in}{2.688695in}}%
\pgfpathlineto{\pgfqpoint{3.660949in}{2.696400in}}%
\pgfpathlineto{\pgfqpoint{3.668414in}{2.704148in}}%
\pgfpathlineto{\pgfqpoint{3.675874in}{2.711940in}}%
\pgfpathlineto{\pgfqpoint{3.683328in}{2.719777in}}%
\pgfpathclose%
\pgfusepath{fill}%
\end{pgfscope}%
\begin{pgfscope}%
\pgfpathrectangle{\pgfqpoint{1.254980in}{0.150000in}}{\pgfqpoint{5.490039in}{5.490039in}}%
\pgfusepath{clip}%
\pgfsetbuttcap%
\pgfsetroundjoin%
\definecolor{currentfill}{rgb}{0.277018,0.050344,0.375715}%
\pgfsetfillcolor{currentfill}%
\pgfsetfillopacity{0.700000}%
\pgfsetlinewidth{0.000000pt}%
\definecolor{currentstroke}{rgb}{0.000000,0.000000,0.000000}%
\pgfsetstrokecolor{currentstroke}%
\pgfsetdash{}{0pt}%
\pgfpathmoveto{\pgfqpoint{4.365935in}{2.742688in}}%
\pgfpathlineto{\pgfqpoint{4.378673in}{2.738695in}}%
\pgfpathlineto{\pgfqpoint{4.391416in}{2.734731in}}%
\pgfpathlineto{\pgfqpoint{4.404165in}{2.730795in}}%
\pgfpathlineto{\pgfqpoint{4.416919in}{2.726888in}}%
\pgfpathlineto{\pgfqpoint{4.409707in}{2.718551in}}%
\pgfpathlineto{\pgfqpoint{4.402491in}{2.710295in}}%
\pgfpathlineto{\pgfqpoint{4.395270in}{2.702118in}}%
\pgfpathlineto{\pgfqpoint{4.388045in}{2.694016in}}%
\pgfpathlineto{\pgfqpoint{4.375279in}{2.697802in}}%
\pgfpathlineto{\pgfqpoint{4.362518in}{2.701617in}}%
\pgfpathlineto{\pgfqpoint{4.349763in}{2.705460in}}%
\pgfpathlineto{\pgfqpoint{4.337014in}{2.709332in}}%
\pgfpathlineto{\pgfqpoint{4.344251in}{2.717551in}}%
\pgfpathlineto{\pgfqpoint{4.351483in}{2.725847in}}%
\pgfpathlineto{\pgfqpoint{4.358711in}{2.734225in}}%
\pgfpathlineto{\pgfqpoint{4.365935in}{2.742688in}}%
\pgfpathclose%
\pgfusepath{fill}%
\end{pgfscope}%
\begin{pgfscope}%
\pgfpathrectangle{\pgfqpoint{1.254980in}{0.150000in}}{\pgfqpoint{5.490039in}{5.490039in}}%
\pgfusepath{clip}%
\pgfsetbuttcap%
\pgfsetroundjoin%
\definecolor{currentfill}{rgb}{0.274952,0.037752,0.364543}%
\pgfsetfillcolor{currentfill}%
\pgfsetfillopacity{0.700000}%
\pgfsetlinewidth{0.000000pt}%
\definecolor{currentstroke}{rgb}{0.000000,0.000000,0.000000}%
\pgfsetstrokecolor{currentstroke}%
\pgfsetdash{}{0pt}%
\pgfpathmoveto{\pgfqpoint{4.155313in}{2.724618in}}%
\pgfpathlineto{\pgfqpoint{4.168009in}{2.720497in}}%
\pgfpathlineto{\pgfqpoint{4.180711in}{2.716408in}}%
\pgfpathlineto{\pgfqpoint{4.193417in}{2.712348in}}%
\pgfpathlineto{\pgfqpoint{4.206129in}{2.708319in}}%
\pgfpathlineto{\pgfqpoint{4.198847in}{2.700232in}}%
\pgfpathlineto{\pgfqpoint{4.191559in}{2.692207in}}%
\pgfpathlineto{\pgfqpoint{4.184266in}{2.684241in}}%
\pgfpathlineto{\pgfqpoint{4.176969in}{2.676331in}}%
\pgfpathlineto{\pgfqpoint{4.164246in}{2.680264in}}%
\pgfpathlineto{\pgfqpoint{4.151527in}{2.684227in}}%
\pgfpathlineto{\pgfqpoint{4.138815in}{2.688221in}}%
\pgfpathlineto{\pgfqpoint{4.126107in}{2.692245in}}%
\pgfpathlineto{\pgfqpoint{4.133416in}{2.700246in}}%
\pgfpathlineto{\pgfqpoint{4.140720in}{2.708307in}}%
\pgfpathlineto{\pgfqpoint{4.148019in}{2.716430in}}%
\pgfpathlineto{\pgfqpoint{4.155313in}{2.724618in}}%
\pgfpathclose%
\pgfusepath{fill}%
\end{pgfscope}%
\begin{pgfscope}%
\pgfpathrectangle{\pgfqpoint{1.254980in}{0.150000in}}{\pgfqpoint{5.490039in}{5.490039in}}%
\pgfusepath{clip}%
\pgfsetbuttcap%
\pgfsetroundjoin%
\definecolor{currentfill}{rgb}{0.273809,0.031497,0.358853}%
\pgfsetfillcolor{currentfill}%
\pgfsetfillopacity{0.700000}%
\pgfsetlinewidth{0.000000pt}%
\definecolor{currentstroke}{rgb}{0.000000,0.000000,0.000000}%
\pgfsetstrokecolor{currentstroke}%
\pgfsetdash{}{0pt}%
\pgfpathmoveto{\pgfqpoint{3.813944in}{2.714140in}}%
\pgfpathlineto{\pgfqpoint{3.826572in}{2.709615in}}%
\pgfpathlineto{\pgfqpoint{3.839205in}{2.705126in}}%
\pgfpathlineto{\pgfqpoint{3.851843in}{2.700670in}}%
\pgfpathlineto{\pgfqpoint{3.864485in}{2.696250in}}%
\pgfpathlineto{\pgfqpoint{3.857086in}{2.688399in}}%
\pgfpathlineto{\pgfqpoint{3.849681in}{2.680593in}}%
\pgfpathlineto{\pgfqpoint{3.842271in}{2.672829in}}%
\pgfpathlineto{\pgfqpoint{3.834856in}{2.665106in}}%
\pgfpathlineto{\pgfqpoint{3.822202in}{2.669468in}}%
\pgfpathlineto{\pgfqpoint{3.809553in}{2.673864in}}%
\pgfpathlineto{\pgfqpoint{3.796909in}{2.678295in}}%
\pgfpathlineto{\pgfqpoint{3.784270in}{2.682761in}}%
\pgfpathlineto{\pgfqpoint{3.791697in}{2.690539in}}%
\pgfpathlineto{\pgfqpoint{3.799118in}{2.698359in}}%
\pgfpathlineto{\pgfqpoint{3.806534in}{2.706226in}}%
\pgfpathlineto{\pgfqpoint{3.813944in}{2.714140in}}%
\pgfpathclose%
\pgfusepath{fill}%
\end{pgfscope}%
\begin{pgfscope}%
\pgfpathrectangle{\pgfqpoint{1.254980in}{0.150000in}}{\pgfqpoint{5.490039in}{5.490039in}}%
\pgfusepath{clip}%
\pgfsetbuttcap%
\pgfsetroundjoin%
\definecolor{currentfill}{rgb}{0.279566,0.067836,0.391917}%
\pgfsetfillcolor{currentfill}%
\pgfsetfillopacity{0.700000}%
\pgfsetlinewidth{0.000000pt}%
\definecolor{currentstroke}{rgb}{0.000000,0.000000,0.000000}%
\pgfsetstrokecolor{currentstroke}%
\pgfsetdash{}{0pt}%
\pgfpathmoveto{\pgfqpoint{4.707424in}{2.769178in}}%
\pgfpathlineto{\pgfqpoint{4.720238in}{2.765298in}}%
\pgfpathlineto{\pgfqpoint{4.733058in}{2.761445in}}%
\pgfpathlineto{\pgfqpoint{4.745883in}{2.757618in}}%
\pgfpathlineto{\pgfqpoint{4.758715in}{2.753817in}}%
\pgfpathlineto{\pgfqpoint{4.751607in}{2.744964in}}%
\pgfpathlineto{\pgfqpoint{4.744497in}{2.736233in}}%
\pgfpathlineto{\pgfqpoint{4.737384in}{2.727618in}}%
\pgfpathlineto{\pgfqpoint{4.730268in}{2.719115in}}%
\pgfpathlineto{\pgfqpoint{4.717423in}{2.722757in}}%
\pgfpathlineto{\pgfqpoint{4.704585in}{2.726426in}}%
\pgfpathlineto{\pgfqpoint{4.691752in}{2.730121in}}%
\pgfpathlineto{\pgfqpoint{4.678926in}{2.733843in}}%
\pgfpathlineto{\pgfqpoint{4.686055in}{2.742500in}}%
\pgfpathlineto{\pgfqpoint{4.693181in}{2.751271in}}%
\pgfpathlineto{\pgfqpoint{4.700304in}{2.760162in}}%
\pgfpathlineto{\pgfqpoint{4.707424in}{2.769178in}}%
\pgfpathclose%
\pgfusepath{fill}%
\end{pgfscope}%
\begin{pgfscope}%
\pgfpathrectangle{\pgfqpoint{1.254980in}{0.150000in}}{\pgfqpoint{5.490039in}{5.490039in}}%
\pgfusepath{clip}%
\pgfsetbuttcap%
\pgfsetroundjoin%
\definecolor{currentfill}{rgb}{0.277941,0.056324,0.381191}%
\pgfsetfillcolor{currentfill}%
\pgfsetfillopacity{0.700000}%
\pgfsetlinewidth{0.000000pt}%
\definecolor{currentstroke}{rgb}{0.000000,0.000000,0.000000}%
\pgfsetstrokecolor{currentstroke}%
\pgfsetdash{}{0pt}%
\pgfpathmoveto{\pgfqpoint{4.496753in}{2.745264in}}%
\pgfpathlineto{\pgfqpoint{4.509523in}{2.741364in}}%
\pgfpathlineto{\pgfqpoint{4.522299in}{2.737491in}}%
\pgfpathlineto{\pgfqpoint{4.535081in}{2.733646in}}%
\pgfpathlineto{\pgfqpoint{4.547868in}{2.729829in}}%
\pgfpathlineto{\pgfqpoint{4.540697in}{2.721396in}}%
\pgfpathlineto{\pgfqpoint{4.533521in}{2.713057in}}%
\pgfpathlineto{\pgfqpoint{4.526341in}{2.704807in}}%
\pgfpathlineto{\pgfqpoint{4.519157in}{2.696640in}}%
\pgfpathlineto{\pgfqpoint{4.506358in}{2.700324in}}%
\pgfpathlineto{\pgfqpoint{4.493564in}{2.704036in}}%
\pgfpathlineto{\pgfqpoint{4.480776in}{2.707775in}}%
\pgfpathlineto{\pgfqpoint{4.467993in}{2.711542in}}%
\pgfpathlineto{\pgfqpoint{4.475189in}{2.719837in}}%
\pgfpathlineto{\pgfqpoint{4.482381in}{2.728219in}}%
\pgfpathlineto{\pgfqpoint{4.489569in}{2.736694in}}%
\pgfpathlineto{\pgfqpoint{4.496753in}{2.745264in}}%
\pgfpathclose%
\pgfusepath{fill}%
\end{pgfscope}%
\begin{pgfscope}%
\pgfpathrectangle{\pgfqpoint{1.254980in}{0.150000in}}{\pgfqpoint{5.490039in}{5.490039in}}%
\pgfusepath{clip}%
\pgfsetbuttcap%
\pgfsetroundjoin%
\definecolor{currentfill}{rgb}{0.277941,0.056324,0.381191}%
\pgfsetfillcolor{currentfill}%
\pgfsetfillopacity{0.700000}%
\pgfsetlinewidth{0.000000pt}%
\definecolor{currentstroke}{rgb}{0.000000,0.000000,0.000000}%
\pgfsetstrokecolor{currentstroke}%
\pgfsetdash{}{0pt}%
\pgfpathmoveto{\pgfqpoint{3.210995in}{2.745450in}}%
\pgfpathlineto{\pgfqpoint{3.223528in}{2.739627in}}%
\pgfpathlineto{\pgfqpoint{3.236065in}{2.733851in}}%
\pgfpathlineto{\pgfqpoint{3.248605in}{2.728123in}}%
\pgfpathlineto{\pgfqpoint{3.261149in}{2.722441in}}%
\pgfpathlineto{\pgfqpoint{3.253533in}{2.715158in}}%
\pgfpathlineto{\pgfqpoint{3.245910in}{2.707921in}}%
\pgfpathlineto{\pgfqpoint{3.238282in}{2.700728in}}%
\pgfpathlineto{\pgfqpoint{3.230647in}{2.693580in}}%
\pgfpathlineto{\pgfqpoint{3.218091in}{2.699266in}}%
\pgfpathlineto{\pgfqpoint{3.205538in}{2.704999in}}%
\pgfpathlineto{\pgfqpoint{3.192989in}{2.710779in}}%
\pgfpathlineto{\pgfqpoint{3.180443in}{2.716606in}}%
\pgfpathlineto{\pgfqpoint{3.188090in}{2.723746in}}%
\pgfpathlineto{\pgfqpoint{3.195731in}{2.730933in}}%
\pgfpathlineto{\pgfqpoint{3.203366in}{2.738167in}}%
\pgfpathlineto{\pgfqpoint{3.210995in}{2.745450in}}%
\pgfpathclose%
\pgfusepath{fill}%
\end{pgfscope}%
\begin{pgfscope}%
\pgfpathrectangle{\pgfqpoint{1.254980in}{0.150000in}}{\pgfqpoint{5.490039in}{5.490039in}}%
\pgfusepath{clip}%
\pgfsetbuttcap%
\pgfsetroundjoin%
\definecolor{currentfill}{rgb}{0.276022,0.044167,0.370164}%
\pgfsetfillcolor{currentfill}%
\pgfsetfillopacity{0.700000}%
\pgfsetlinewidth{0.000000pt}%
\definecolor{currentstroke}{rgb}{0.000000,0.000000,0.000000}%
\pgfsetstrokecolor{currentstroke}%
\pgfsetdash{}{0pt}%
\pgfpathmoveto{\pgfqpoint{3.341717in}{2.729733in}}%
\pgfpathlineto{\pgfqpoint{3.354268in}{2.724270in}}%
\pgfpathlineto{\pgfqpoint{3.366822in}{2.718850in}}%
\pgfpathlineto{\pgfqpoint{3.379381in}{2.713474in}}%
\pgfpathlineto{\pgfqpoint{3.391943in}{2.708141in}}%
\pgfpathlineto{\pgfqpoint{3.384375in}{2.700697in}}%
\pgfpathlineto{\pgfqpoint{3.376801in}{2.693294in}}%
\pgfpathlineto{\pgfqpoint{3.369222in}{2.685933in}}%
\pgfpathlineto{\pgfqpoint{3.361636in}{2.678612in}}%
\pgfpathlineto{\pgfqpoint{3.349061in}{2.683936in}}%
\pgfpathlineto{\pgfqpoint{3.336491in}{2.689304in}}%
\pgfpathlineto{\pgfqpoint{3.323924in}{2.694715in}}%
\pgfpathlineto{\pgfqpoint{3.311362in}{2.700171in}}%
\pgfpathlineto{\pgfqpoint{3.318960in}{2.707496in}}%
\pgfpathlineto{\pgfqpoint{3.326552in}{2.714864in}}%
\pgfpathlineto{\pgfqpoint{3.334137in}{2.722276in}}%
\pgfpathlineto{\pgfqpoint{3.341717in}{2.729733in}}%
\pgfpathclose%
\pgfusepath{fill}%
\end{pgfscope}%
\begin{pgfscope}%
\pgfpathrectangle{\pgfqpoint{1.254980in}{0.150000in}}{\pgfqpoint{5.490039in}{5.490039in}}%
\pgfusepath{clip}%
\pgfsetbuttcap%
\pgfsetroundjoin%
\definecolor{currentfill}{rgb}{0.273809,0.031497,0.358853}%
\pgfsetfillcolor{currentfill}%
\pgfsetfillopacity{0.700000}%
\pgfsetlinewidth{0.000000pt}%
\definecolor{currentstroke}{rgb}{0.000000,0.000000,0.000000}%
\pgfsetstrokecolor{currentstroke}%
\pgfsetdash{}{0pt}%
\pgfpathmoveto{\pgfqpoint{3.944606in}{2.710519in}}%
\pgfpathlineto{\pgfqpoint{3.957263in}{2.706196in}}%
\pgfpathlineto{\pgfqpoint{3.969924in}{2.701905in}}%
\pgfpathlineto{\pgfqpoint{3.982591in}{2.697648in}}%
\pgfpathlineto{\pgfqpoint{3.995263in}{2.693422in}}%
\pgfpathlineto{\pgfqpoint{3.987906in}{2.685520in}}%
\pgfpathlineto{\pgfqpoint{3.980545in}{2.677667in}}%
\pgfpathlineto{\pgfqpoint{3.973178in}{2.669859in}}%
\pgfpathlineto{\pgfqpoint{3.965807in}{2.662094in}}%
\pgfpathlineto{\pgfqpoint{3.953124in}{2.666248in}}%
\pgfpathlineto{\pgfqpoint{3.940446in}{2.670434in}}%
\pgfpathlineto{\pgfqpoint{3.927773in}{2.674653in}}%
\pgfpathlineto{\pgfqpoint{3.915106in}{2.678906in}}%
\pgfpathlineto{\pgfqpoint{3.922489in}{2.686737in}}%
\pgfpathlineto{\pgfqpoint{3.929866in}{2.694615in}}%
\pgfpathlineto{\pgfqpoint{3.937239in}{2.702541in}}%
\pgfpathlineto{\pgfqpoint{3.944606in}{2.710519in}}%
\pgfpathclose%
\pgfusepath{fill}%
\end{pgfscope}%
\begin{pgfscope}%
\pgfpathrectangle{\pgfqpoint{1.254980in}{0.150000in}}{\pgfqpoint{5.490039in}{5.490039in}}%
\pgfusepath{clip}%
\pgfsetbuttcap%
\pgfsetroundjoin%
\definecolor{currentfill}{rgb}{0.274952,0.037752,0.364543}%
\pgfsetfillcolor{currentfill}%
\pgfsetfillopacity{0.700000}%
\pgfsetlinewidth{0.000000pt}%
\definecolor{currentstroke}{rgb}{0.000000,0.000000,0.000000}%
\pgfsetstrokecolor{currentstroke}%
\pgfsetdash{}{0pt}%
\pgfpathmoveto{\pgfqpoint{3.472400in}{2.717362in}}%
\pgfpathlineto{\pgfqpoint{3.484971in}{2.712219in}}%
\pgfpathlineto{\pgfqpoint{3.497547in}{2.707116in}}%
\pgfpathlineto{\pgfqpoint{3.510127in}{2.702053in}}%
\pgfpathlineto{\pgfqpoint{3.522711in}{2.697031in}}%
\pgfpathlineto{\pgfqpoint{3.515189in}{2.689462in}}%
\pgfpathlineto{\pgfqpoint{3.507662in}{2.681932in}}%
\pgfpathlineto{\pgfqpoint{3.500129in}{2.674440in}}%
\pgfpathlineto{\pgfqpoint{3.492590in}{2.666986in}}%
\pgfpathlineto{\pgfqpoint{3.479995in}{2.671987in}}%
\pgfpathlineto{\pgfqpoint{3.467403in}{2.677028in}}%
\pgfpathlineto{\pgfqpoint{3.454816in}{2.682110in}}%
\pgfpathlineto{\pgfqpoint{3.442234in}{2.687233in}}%
\pgfpathlineto{\pgfqpoint{3.449784in}{2.694703in}}%
\pgfpathlineto{\pgfqpoint{3.457328in}{2.702215in}}%
\pgfpathlineto{\pgfqpoint{3.464867in}{2.709767in}}%
\pgfpathlineto{\pgfqpoint{3.472400in}{2.717362in}}%
\pgfpathclose%
\pgfusepath{fill}%
\end{pgfscope}%
\begin{pgfscope}%
\pgfpathrectangle{\pgfqpoint{1.254980in}{0.150000in}}{\pgfqpoint{5.490039in}{5.490039in}}%
\pgfusepath{clip}%
\pgfsetbuttcap%
\pgfsetroundjoin%
\definecolor{currentfill}{rgb}{0.276022,0.044167,0.370164}%
\pgfsetfillcolor{currentfill}%
\pgfsetfillopacity{0.700000}%
\pgfsetlinewidth{0.000000pt}%
\definecolor{currentstroke}{rgb}{0.000000,0.000000,0.000000}%
\pgfsetstrokecolor{currentstroke}%
\pgfsetdash{}{0pt}%
\pgfpathmoveto{\pgfqpoint{4.286071in}{2.725111in}}%
\pgfpathlineto{\pgfqpoint{4.298798in}{2.721122in}}%
\pgfpathlineto{\pgfqpoint{4.311531in}{2.717163in}}%
\pgfpathlineto{\pgfqpoint{4.324270in}{2.713233in}}%
\pgfpathlineto{\pgfqpoint{4.337014in}{2.709332in}}%
\pgfpathlineto{\pgfqpoint{4.329773in}{2.701187in}}%
\pgfpathlineto{\pgfqpoint{4.322527in}{2.693112in}}%
\pgfpathlineto{\pgfqpoint{4.315276in}{2.685104in}}%
\pgfpathlineto{\pgfqpoint{4.308021in}{2.677159in}}%
\pgfpathlineto{\pgfqpoint{4.295266in}{2.680951in}}%
\pgfpathlineto{\pgfqpoint{4.282516in}{2.684772in}}%
\pgfpathlineto{\pgfqpoint{4.269771in}{2.688623in}}%
\pgfpathlineto{\pgfqpoint{4.257032in}{2.692503in}}%
\pgfpathlineto{\pgfqpoint{4.264298in}{2.700552in}}%
\pgfpathlineto{\pgfqpoint{4.271560in}{2.708667in}}%
\pgfpathlineto{\pgfqpoint{4.278818in}{2.716852in}}%
\pgfpathlineto{\pgfqpoint{4.286071in}{2.725111in}}%
\pgfpathclose%
\pgfusepath{fill}%
\end{pgfscope}%
\begin{pgfscope}%
\pgfpathrectangle{\pgfqpoint{1.254980in}{0.150000in}}{\pgfqpoint{5.490039in}{5.490039in}}%
\pgfusepath{clip}%
\pgfsetbuttcap%
\pgfsetroundjoin%
\definecolor{currentfill}{rgb}{0.273809,0.031497,0.358853}%
\pgfsetfillcolor{currentfill}%
\pgfsetfillopacity{0.700000}%
\pgfsetlinewidth{0.000000pt}%
\definecolor{currentstroke}{rgb}{0.000000,0.000000,0.000000}%
\pgfsetstrokecolor{currentstroke}%
\pgfsetdash{}{0pt}%
\pgfpathmoveto{\pgfqpoint{3.603073in}{2.707905in}}%
\pgfpathlineto{\pgfqpoint{3.615668in}{2.703046in}}%
\pgfpathlineto{\pgfqpoint{3.628267in}{2.698224in}}%
\pgfpathlineto{\pgfqpoint{3.640870in}{2.693441in}}%
\pgfpathlineto{\pgfqpoint{3.653478in}{2.688695in}}%
\pgfpathlineto{\pgfqpoint{3.646002in}{2.681030in}}%
\pgfpathlineto{\pgfqpoint{3.638520in}{2.673404in}}%
\pgfpathlineto{\pgfqpoint{3.631033in}{2.665815in}}%
\pgfpathlineto{\pgfqpoint{3.623540in}{2.658262in}}%
\pgfpathlineto{\pgfqpoint{3.610921in}{2.662975in}}%
\pgfpathlineto{\pgfqpoint{3.598306in}{2.667724in}}%
\pgfpathlineto{\pgfqpoint{3.585696in}{2.672512in}}%
\pgfpathlineto{\pgfqpoint{3.573090in}{2.677338in}}%
\pgfpathlineto{\pgfqpoint{3.580594in}{2.684920in}}%
\pgfpathlineto{\pgfqpoint{3.588093in}{2.692541in}}%
\pgfpathlineto{\pgfqpoint{3.595586in}{2.700202in}}%
\pgfpathlineto{\pgfqpoint{3.603073in}{2.707905in}}%
\pgfpathclose%
\pgfusepath{fill}%
\end{pgfscope}%
\begin{pgfscope}%
\pgfpathrectangle{\pgfqpoint{1.254980in}{0.150000in}}{\pgfqpoint{5.490039in}{5.490039in}}%
\pgfusepath{clip}%
\pgfsetbuttcap%
\pgfsetroundjoin%
\definecolor{currentfill}{rgb}{0.274952,0.037752,0.364543}%
\pgfsetfillcolor{currentfill}%
\pgfsetfillopacity{0.700000}%
\pgfsetlinewidth{0.000000pt}%
\definecolor{currentstroke}{rgb}{0.000000,0.000000,0.000000}%
\pgfsetstrokecolor{currentstroke}%
\pgfsetdash{}{0pt}%
\pgfpathmoveto{\pgfqpoint{4.075331in}{2.708653in}}%
\pgfpathlineto{\pgfqpoint{4.088017in}{2.704504in}}%
\pgfpathlineto{\pgfqpoint{4.100708in}{2.700387in}}%
\pgfpathlineto{\pgfqpoint{4.113405in}{2.696300in}}%
\pgfpathlineto{\pgfqpoint{4.126107in}{2.692245in}}%
\pgfpathlineto{\pgfqpoint{4.118794in}{2.684300in}}%
\pgfpathlineto{\pgfqpoint{4.111475in}{2.676408in}}%
\pgfpathlineto{\pgfqpoint{4.104152in}{2.668566in}}%
\pgfpathlineto{\pgfqpoint{4.096823in}{2.660771in}}%
\pgfpathlineto{\pgfqpoint{4.084109in}{2.664742in}}%
\pgfpathlineto{\pgfqpoint{4.071401in}{2.668745in}}%
\pgfpathlineto{\pgfqpoint{4.058698in}{2.672779in}}%
\pgfpathlineto{\pgfqpoint{4.046001in}{2.676844in}}%
\pgfpathlineto{\pgfqpoint{4.053341in}{2.684718in}}%
\pgfpathlineto{\pgfqpoint{4.060676in}{2.692642in}}%
\pgfpathlineto{\pgfqpoint{4.068006in}{2.700619in}}%
\pgfpathlineto{\pgfqpoint{4.075331in}{2.708653in}}%
\pgfpathclose%
\pgfusepath{fill}%
\end{pgfscope}%
\begin{pgfscope}%
\pgfpathrectangle{\pgfqpoint{1.254980in}{0.150000in}}{\pgfqpoint{5.490039in}{5.490039in}}%
\pgfusepath{clip}%
\pgfsetbuttcap%
\pgfsetroundjoin%
\definecolor{currentfill}{rgb}{0.278791,0.062145,0.386592}%
\pgfsetfillcolor{currentfill}%
\pgfsetfillopacity{0.700000}%
\pgfsetlinewidth{0.000000pt}%
\definecolor{currentstroke}{rgb}{0.000000,0.000000,0.000000}%
\pgfsetstrokecolor{currentstroke}%
\pgfsetdash{}{0pt}%
\pgfpathmoveto{\pgfqpoint{4.627677in}{2.748994in}}%
\pgfpathlineto{\pgfqpoint{4.640480in}{2.745166in}}%
\pgfpathlineto{\pgfqpoint{4.653290in}{2.741365in}}%
\pgfpathlineto{\pgfqpoint{4.666105in}{2.737590in}}%
\pgfpathlineto{\pgfqpoint{4.678926in}{2.733843in}}%
\pgfpathlineto{\pgfqpoint{4.671793in}{2.725295in}}%
\pgfpathlineto{\pgfqpoint{4.664657in}{2.716852in}}%
\pgfpathlineto{\pgfqpoint{4.657518in}{2.708510in}}%
\pgfpathlineto{\pgfqpoint{4.650375in}{2.700263in}}%
\pgfpathlineto{\pgfqpoint{4.637541in}{2.703865in}}%
\pgfpathlineto{\pgfqpoint{4.624713in}{2.707494in}}%
\pgfpathlineto{\pgfqpoint{4.611891in}{2.711149in}}%
\pgfpathlineto{\pgfqpoint{4.599075in}{2.714831in}}%
\pgfpathlineto{\pgfqpoint{4.606231in}{2.723219in}}%
\pgfpathlineto{\pgfqpoint{4.613383in}{2.731706in}}%
\pgfpathlineto{\pgfqpoint{4.620532in}{2.740296in}}%
\pgfpathlineto{\pgfqpoint{4.627677in}{2.748994in}}%
\pgfpathclose%
\pgfusepath{fill}%
\end{pgfscope}%
\begin{pgfscope}%
\pgfpathrectangle{\pgfqpoint{1.254980in}{0.150000in}}{\pgfqpoint{5.490039in}{5.490039in}}%
\pgfusepath{clip}%
\pgfsetbuttcap%
\pgfsetroundjoin%
\definecolor{currentfill}{rgb}{0.273809,0.031497,0.358853}%
\pgfsetfillcolor{currentfill}%
\pgfsetfillopacity{0.700000}%
\pgfsetlinewidth{0.000000pt}%
\definecolor{currentstroke}{rgb}{0.000000,0.000000,0.000000}%
\pgfsetstrokecolor{currentstroke}%
\pgfsetdash{}{0pt}%
\pgfpathmoveto{\pgfqpoint{3.733761in}{2.700980in}}%
\pgfpathlineto{\pgfqpoint{3.746381in}{2.696371in}}%
\pgfpathlineto{\pgfqpoint{3.759006in}{2.691799in}}%
\pgfpathlineto{\pgfqpoint{3.771636in}{2.687263in}}%
\pgfpathlineto{\pgfqpoint{3.784270in}{2.682761in}}%
\pgfpathlineto{\pgfqpoint{3.776838in}{2.675026in}}%
\pgfpathlineto{\pgfqpoint{3.769401in}{2.667330in}}%
\pgfpathlineto{\pgfqpoint{3.761958in}{2.659671in}}%
\pgfpathlineto{\pgfqpoint{3.754510in}{2.652049in}}%
\pgfpathlineto{\pgfqpoint{3.741865in}{2.656504in}}%
\pgfpathlineto{\pgfqpoint{3.729224in}{2.660994in}}%
\pgfpathlineto{\pgfqpoint{3.716588in}{2.665520in}}%
\pgfpathlineto{\pgfqpoint{3.703957in}{2.670082in}}%
\pgfpathlineto{\pgfqpoint{3.711416in}{2.677746in}}%
\pgfpathlineto{\pgfqpoint{3.718870in}{2.685449in}}%
\pgfpathlineto{\pgfqpoint{3.726318in}{2.693193in}}%
\pgfpathlineto{\pgfqpoint{3.733761in}{2.700980in}}%
\pgfpathclose%
\pgfusepath{fill}%
\end{pgfscope}%
\begin{pgfscope}%
\pgfpathrectangle{\pgfqpoint{1.254980in}{0.150000in}}{\pgfqpoint{5.490039in}{5.490039in}}%
\pgfusepath{clip}%
\pgfsetbuttcap%
\pgfsetroundjoin%
\definecolor{currentfill}{rgb}{0.277018,0.050344,0.375715}%
\pgfsetfillcolor{currentfill}%
\pgfsetfillopacity{0.700000}%
\pgfsetlinewidth{0.000000pt}%
\definecolor{currentstroke}{rgb}{0.000000,0.000000,0.000000}%
\pgfsetstrokecolor{currentstroke}%
\pgfsetdash{}{0pt}%
\pgfpathmoveto{\pgfqpoint{4.416919in}{2.726888in}}%
\pgfpathlineto{\pgfqpoint{4.429679in}{2.723009in}}%
\pgfpathlineto{\pgfqpoint{4.442445in}{2.719159in}}%
\pgfpathlineto{\pgfqpoint{4.455216in}{2.715336in}}%
\pgfpathlineto{\pgfqpoint{4.467993in}{2.711542in}}%
\pgfpathlineto{\pgfqpoint{4.460793in}{2.703330in}}%
\pgfpathlineto{\pgfqpoint{4.453589in}{2.695197in}}%
\pgfpathlineto{\pgfqpoint{4.446380in}{2.687140in}}%
\pgfpathlineto{\pgfqpoint{4.439167in}{2.679154in}}%
\pgfpathlineto{\pgfqpoint{4.426378in}{2.682828in}}%
\pgfpathlineto{\pgfqpoint{4.413595in}{2.686529in}}%
\pgfpathlineto{\pgfqpoint{4.400817in}{2.690258in}}%
\pgfpathlineto{\pgfqpoint{4.388045in}{2.694016in}}%
\pgfpathlineto{\pgfqpoint{4.395270in}{2.702118in}}%
\pgfpathlineto{\pgfqpoint{4.402491in}{2.710295in}}%
\pgfpathlineto{\pgfqpoint{4.409707in}{2.718551in}}%
\pgfpathlineto{\pgfqpoint{4.416919in}{2.726888in}}%
\pgfpathclose%
\pgfusepath{fill}%
\end{pgfscope}%
\begin{pgfscope}%
\pgfpathrectangle{\pgfqpoint{1.254980in}{0.150000in}}{\pgfqpoint{5.490039in}{5.490039in}}%
\pgfusepath{clip}%
\pgfsetbuttcap%
\pgfsetroundjoin%
\definecolor{currentfill}{rgb}{0.273809,0.031497,0.358853}%
\pgfsetfillcolor{currentfill}%
\pgfsetfillopacity{0.700000}%
\pgfsetlinewidth{0.000000pt}%
\definecolor{currentstroke}{rgb}{0.000000,0.000000,0.000000}%
\pgfsetstrokecolor{currentstroke}%
\pgfsetdash{}{0pt}%
\pgfpathmoveto{\pgfqpoint{3.864485in}{2.696250in}}%
\pgfpathlineto{\pgfqpoint{3.877133in}{2.691863in}}%
\pgfpathlineto{\pgfqpoint{3.889786in}{2.687510in}}%
\pgfpathlineto{\pgfqpoint{3.902443in}{2.683191in}}%
\pgfpathlineto{\pgfqpoint{3.915106in}{2.678906in}}%
\pgfpathlineto{\pgfqpoint{3.907717in}{2.671118in}}%
\pgfpathlineto{\pgfqpoint{3.900324in}{2.663372in}}%
\pgfpathlineto{\pgfqpoint{3.892925in}{2.655666in}}%
\pgfpathlineto{\pgfqpoint{3.885521in}{2.647997in}}%
\pgfpathlineto{\pgfqpoint{3.872847in}{2.652224in}}%
\pgfpathlineto{\pgfqpoint{3.860179in}{2.656484in}}%
\pgfpathlineto{\pgfqpoint{3.847515in}{2.660778in}}%
\pgfpathlineto{\pgfqpoint{3.834856in}{2.665106in}}%
\pgfpathlineto{\pgfqpoint{3.842271in}{2.672829in}}%
\pgfpathlineto{\pgfqpoint{3.849681in}{2.680593in}}%
\pgfpathlineto{\pgfqpoint{3.857086in}{2.688399in}}%
\pgfpathlineto{\pgfqpoint{3.864485in}{2.696250in}}%
\pgfpathclose%
\pgfusepath{fill}%
\end{pgfscope}%
\begin{pgfscope}%
\pgfpathrectangle{\pgfqpoint{1.254980in}{0.150000in}}{\pgfqpoint{5.490039in}{5.490039in}}%
\pgfusepath{clip}%
\pgfsetbuttcap%
\pgfsetroundjoin%
\definecolor{currentfill}{rgb}{0.274952,0.037752,0.364543}%
\pgfsetfillcolor{currentfill}%
\pgfsetfillopacity{0.700000}%
\pgfsetlinewidth{0.000000pt}%
\definecolor{currentstroke}{rgb}{0.000000,0.000000,0.000000}%
\pgfsetstrokecolor{currentstroke}%
\pgfsetdash{}{0pt}%
\pgfpathmoveto{\pgfqpoint{4.206129in}{2.708319in}}%
\pgfpathlineto{\pgfqpoint{4.218847in}{2.704320in}}%
\pgfpathlineto{\pgfqpoint{4.231570in}{2.700351in}}%
\pgfpathlineto{\pgfqpoint{4.244298in}{2.696412in}}%
\pgfpathlineto{\pgfqpoint{4.257032in}{2.692503in}}%
\pgfpathlineto{\pgfqpoint{4.249760in}{2.684516in}}%
\pgfpathlineto{\pgfqpoint{4.242484in}{2.676589in}}%
\pgfpathlineto{\pgfqpoint{4.235203in}{2.668718in}}%
\pgfpathlineto{\pgfqpoint{4.227918in}{2.660900in}}%
\pgfpathlineto{\pgfqpoint{4.215172in}{2.664713in}}%
\pgfpathlineto{\pgfqpoint{4.202432in}{2.668556in}}%
\pgfpathlineto{\pgfqpoint{4.189698in}{2.672428in}}%
\pgfpathlineto{\pgfqpoint{4.176969in}{2.676331in}}%
\pgfpathlineto{\pgfqpoint{4.184266in}{2.684241in}}%
\pgfpathlineto{\pgfqpoint{4.191559in}{2.692207in}}%
\pgfpathlineto{\pgfqpoint{4.198847in}{2.700232in}}%
\pgfpathlineto{\pgfqpoint{4.206129in}{2.708319in}}%
\pgfpathclose%
\pgfusepath{fill}%
\end{pgfscope}%
\begin{pgfscope}%
\pgfpathrectangle{\pgfqpoint{1.254980in}{0.150000in}}{\pgfqpoint{5.490039in}{5.490039in}}%
\pgfusepath{clip}%
\pgfsetbuttcap%
\pgfsetroundjoin%
\definecolor{currentfill}{rgb}{0.277018,0.050344,0.375715}%
\pgfsetfillcolor{currentfill}%
\pgfsetfillopacity{0.700000}%
\pgfsetlinewidth{0.000000pt}%
\definecolor{currentstroke}{rgb}{0.000000,0.000000,0.000000}%
\pgfsetstrokecolor{currentstroke}%
\pgfsetdash{}{0pt}%
\pgfpathmoveto{\pgfqpoint{3.261149in}{2.722441in}}%
\pgfpathlineto{\pgfqpoint{3.273697in}{2.716805in}}%
\pgfpathlineto{\pgfqpoint{3.286248in}{2.711215in}}%
\pgfpathlineto{\pgfqpoint{3.298803in}{2.705671in}}%
\pgfpathlineto{\pgfqpoint{3.311362in}{2.700171in}}%
\pgfpathlineto{\pgfqpoint{3.303758in}{2.692888in}}%
\pgfpathlineto{\pgfqpoint{3.296148in}{2.685648in}}%
\pgfpathlineto{\pgfqpoint{3.288531in}{2.678450in}}%
\pgfpathlineto{\pgfqpoint{3.280909in}{2.671293in}}%
\pgfpathlineto{\pgfqpoint{3.268338in}{2.676797in}}%
\pgfpathlineto{\pgfqpoint{3.255771in}{2.682346in}}%
\pgfpathlineto{\pgfqpoint{3.243207in}{2.687940in}}%
\pgfpathlineto{\pgfqpoint{3.230647in}{2.693580in}}%
\pgfpathlineto{\pgfqpoint{3.238282in}{2.700728in}}%
\pgfpathlineto{\pgfqpoint{3.245910in}{2.707921in}}%
\pgfpathlineto{\pgfqpoint{3.253533in}{2.715158in}}%
\pgfpathlineto{\pgfqpoint{3.261149in}{2.722441in}}%
\pgfpathclose%
\pgfusepath{fill}%
\end{pgfscope}%
\begin{pgfscope}%
\pgfpathrectangle{\pgfqpoint{1.254980in}{0.150000in}}{\pgfqpoint{5.490039in}{5.490039in}}%
\pgfusepath{clip}%
\pgfsetbuttcap%
\pgfsetroundjoin%
\definecolor{currentfill}{rgb}{0.274952,0.037752,0.364543}%
\pgfsetfillcolor{currentfill}%
\pgfsetfillopacity{0.700000}%
\pgfsetlinewidth{0.000000pt}%
\definecolor{currentstroke}{rgb}{0.000000,0.000000,0.000000}%
\pgfsetstrokecolor{currentstroke}%
\pgfsetdash{}{0pt}%
\pgfpathmoveto{\pgfqpoint{3.391943in}{2.708141in}}%
\pgfpathlineto{\pgfqpoint{3.404510in}{2.702851in}}%
\pgfpathlineto{\pgfqpoint{3.417080in}{2.697603in}}%
\pgfpathlineto{\pgfqpoint{3.429655in}{2.692397in}}%
\pgfpathlineto{\pgfqpoint{3.442234in}{2.687233in}}%
\pgfpathlineto{\pgfqpoint{3.434677in}{2.679802in}}%
\pgfpathlineto{\pgfqpoint{3.427115in}{2.672410in}}%
\pgfpathlineto{\pgfqpoint{3.419547in}{2.665055in}}%
\pgfpathlineto{\pgfqpoint{3.411973in}{2.657738in}}%
\pgfpathlineto{\pgfqpoint{3.399383in}{2.662893in}}%
\pgfpathlineto{\pgfqpoint{3.386796in}{2.668090in}}%
\pgfpathlineto{\pgfqpoint{3.374214in}{2.673330in}}%
\pgfpathlineto{\pgfqpoint{3.361636in}{2.678612in}}%
\pgfpathlineto{\pgfqpoint{3.369222in}{2.685933in}}%
\pgfpathlineto{\pgfqpoint{3.376801in}{2.693294in}}%
\pgfpathlineto{\pgfqpoint{3.384375in}{2.700697in}}%
\pgfpathlineto{\pgfqpoint{3.391943in}{2.708141in}}%
\pgfpathclose%
\pgfusepath{fill}%
\end{pgfscope}%
\begin{pgfscope}%
\pgfpathrectangle{\pgfqpoint{1.254980in}{0.150000in}}{\pgfqpoint{5.490039in}{5.490039in}}%
\pgfusepath{clip}%
\pgfsetbuttcap%
\pgfsetroundjoin%
\definecolor{currentfill}{rgb}{0.278791,0.062145,0.386592}%
\pgfsetfillcolor{currentfill}%
\pgfsetfillopacity{0.700000}%
\pgfsetlinewidth{0.000000pt}%
\definecolor{currentstroke}{rgb}{0.000000,0.000000,0.000000}%
\pgfsetstrokecolor{currentstroke}%
\pgfsetdash{}{0pt}%
\pgfpathmoveto{\pgfqpoint{3.130294in}{2.740402in}}%
\pgfpathlineto{\pgfqpoint{3.142826in}{2.734380in}}%
\pgfpathlineto{\pgfqpoint{3.155362in}{2.728406in}}%
\pgfpathlineto{\pgfqpoint{3.167901in}{2.722482in}}%
\pgfpathlineto{\pgfqpoint{3.180443in}{2.716606in}}%
\pgfpathlineto{\pgfqpoint{3.172789in}{2.709514in}}%
\pgfpathlineto{\pgfqpoint{3.165129in}{2.702470in}}%
\pgfpathlineto{\pgfqpoint{3.157462in}{2.695472in}}%
\pgfpathlineto{\pgfqpoint{3.149789in}{2.688520in}}%
\pgfpathlineto{\pgfqpoint{3.137234in}{2.694413in}}%
\pgfpathlineto{\pgfqpoint{3.124683in}{2.700354in}}%
\pgfpathlineto{\pgfqpoint{3.112134in}{2.706344in}}%
\pgfpathlineto{\pgfqpoint{3.099589in}{2.712384in}}%
\pgfpathlineto{\pgfqpoint{3.107275in}{2.719314in}}%
\pgfpathlineto{\pgfqpoint{3.114955in}{2.726293in}}%
\pgfpathlineto{\pgfqpoint{3.122628in}{2.733323in}}%
\pgfpathlineto{\pgfqpoint{3.130294in}{2.740402in}}%
\pgfpathclose%
\pgfusepath{fill}%
\end{pgfscope}%
\begin{pgfscope}%
\pgfpathrectangle{\pgfqpoint{1.254980in}{0.150000in}}{\pgfqpoint{5.490039in}{5.490039in}}%
\pgfusepath{clip}%
\pgfsetbuttcap%
\pgfsetroundjoin%
\definecolor{currentfill}{rgb}{0.279566,0.067836,0.391917}%
\pgfsetfillcolor{currentfill}%
\pgfsetfillopacity{0.700000}%
\pgfsetlinewidth{0.000000pt}%
\definecolor{currentstroke}{rgb}{0.000000,0.000000,0.000000}%
\pgfsetstrokecolor{currentstroke}%
\pgfsetdash{}{0pt}%
\pgfpathmoveto{\pgfqpoint{4.758715in}{2.753817in}}%
\pgfpathlineto{\pgfqpoint{4.771552in}{2.750042in}}%
\pgfpathlineto{\pgfqpoint{4.784395in}{2.746293in}}%
\pgfpathlineto{\pgfqpoint{4.797244in}{2.742570in}}%
\pgfpathlineto{\pgfqpoint{4.810099in}{2.738873in}}%
\pgfpathlineto{\pgfqpoint{4.803004in}{2.730183in}}%
\pgfpathlineto{\pgfqpoint{4.795907in}{2.721612in}}%
\pgfpathlineto{\pgfqpoint{4.788807in}{2.713154in}}%
\pgfpathlineto{\pgfqpoint{4.781704in}{2.704805in}}%
\pgfpathlineto{\pgfqpoint{4.768836in}{2.708343in}}%
\pgfpathlineto{\pgfqpoint{4.755974in}{2.711908in}}%
\pgfpathlineto{\pgfqpoint{4.743118in}{2.715498in}}%
\pgfpathlineto{\pgfqpoint{4.730268in}{2.719115in}}%
\pgfpathlineto{\pgfqpoint{4.737384in}{2.727618in}}%
\pgfpathlineto{\pgfqpoint{4.744497in}{2.736233in}}%
\pgfpathlineto{\pgfqpoint{4.751607in}{2.744964in}}%
\pgfpathlineto{\pgfqpoint{4.758715in}{2.753817in}}%
\pgfpathclose%
\pgfusepath{fill}%
\end{pgfscope}%
\begin{pgfscope}%
\pgfpathrectangle{\pgfqpoint{1.254980in}{0.150000in}}{\pgfqpoint{5.490039in}{5.490039in}}%
\pgfusepath{clip}%
\pgfsetbuttcap%
\pgfsetroundjoin%
\definecolor{currentfill}{rgb}{0.277941,0.056324,0.381191}%
\pgfsetfillcolor{currentfill}%
\pgfsetfillopacity{0.700000}%
\pgfsetlinewidth{0.000000pt}%
\definecolor{currentstroke}{rgb}{0.000000,0.000000,0.000000}%
\pgfsetstrokecolor{currentstroke}%
\pgfsetdash{}{0pt}%
\pgfpathmoveto{\pgfqpoint{4.547868in}{2.729829in}}%
\pgfpathlineto{\pgfqpoint{4.560662in}{2.726038in}}%
\pgfpathlineto{\pgfqpoint{4.573460in}{2.722275in}}%
\pgfpathlineto{\pgfqpoint{4.586265in}{2.718540in}}%
\pgfpathlineto{\pgfqpoint{4.599075in}{2.714831in}}%
\pgfpathlineto{\pgfqpoint{4.591916in}{2.706537in}}%
\pgfpathlineto{\pgfqpoint{4.584752in}{2.698333in}}%
\pgfpathlineto{\pgfqpoint{4.577585in}{2.690214in}}%
\pgfpathlineto{\pgfqpoint{4.570414in}{2.682177in}}%
\pgfpathlineto{\pgfqpoint{4.557591in}{2.685752in}}%
\pgfpathlineto{\pgfqpoint{4.544774in}{2.689355in}}%
\pgfpathlineto{\pgfqpoint{4.531963in}{2.692984in}}%
\pgfpathlineto{\pgfqpoint{4.519157in}{2.696640in}}%
\pgfpathlineto{\pgfqpoint{4.526341in}{2.704807in}}%
\pgfpathlineto{\pgfqpoint{4.533521in}{2.713057in}}%
\pgfpathlineto{\pgfqpoint{4.540697in}{2.721396in}}%
\pgfpathlineto{\pgfqpoint{4.547868in}{2.729829in}}%
\pgfpathclose%
\pgfusepath{fill}%
\end{pgfscope}%
\begin{pgfscope}%
\pgfpathrectangle{\pgfqpoint{1.254980in}{0.150000in}}{\pgfqpoint{5.490039in}{5.490039in}}%
\pgfusepath{clip}%
\pgfsetbuttcap%
\pgfsetroundjoin%
\definecolor{currentfill}{rgb}{0.273809,0.031497,0.358853}%
\pgfsetfillcolor{currentfill}%
\pgfsetfillopacity{0.700000}%
\pgfsetlinewidth{0.000000pt}%
\definecolor{currentstroke}{rgb}{0.000000,0.000000,0.000000}%
\pgfsetstrokecolor{currentstroke}%
\pgfsetdash{}{0pt}%
\pgfpathmoveto{\pgfqpoint{3.522711in}{2.697031in}}%
\pgfpathlineto{\pgfqpoint{3.535299in}{2.692049in}}%
\pgfpathlineto{\pgfqpoint{3.547891in}{2.687106in}}%
\pgfpathlineto{\pgfqpoint{3.560488in}{2.682203in}}%
\pgfpathlineto{\pgfqpoint{3.573090in}{2.677338in}}%
\pgfpathlineto{\pgfqpoint{3.565580in}{2.669794in}}%
\pgfpathlineto{\pgfqpoint{3.558064in}{2.662287in}}%
\pgfpathlineto{\pgfqpoint{3.550543in}{2.654815in}}%
\pgfpathlineto{\pgfqpoint{3.543016in}{2.647377in}}%
\pgfpathlineto{\pgfqpoint{3.530403in}{2.652221in}}%
\pgfpathlineto{\pgfqpoint{3.517794in}{2.657103in}}%
\pgfpathlineto{\pgfqpoint{3.505190in}{2.662025in}}%
\pgfpathlineto{\pgfqpoint{3.492590in}{2.666986in}}%
\pgfpathlineto{\pgfqpoint{3.500129in}{2.674440in}}%
\pgfpathlineto{\pgfqpoint{3.507662in}{2.681932in}}%
\pgfpathlineto{\pgfqpoint{3.515189in}{2.689462in}}%
\pgfpathlineto{\pgfqpoint{3.522711in}{2.697031in}}%
\pgfpathclose%
\pgfusepath{fill}%
\end{pgfscope}%
\begin{pgfscope}%
\pgfpathrectangle{\pgfqpoint{1.254980in}{0.150000in}}{\pgfqpoint{5.490039in}{5.490039in}}%
\pgfusepath{clip}%
\pgfsetbuttcap%
\pgfsetroundjoin%
\definecolor{currentfill}{rgb}{0.273809,0.031497,0.358853}%
\pgfsetfillcolor{currentfill}%
\pgfsetfillopacity{0.700000}%
\pgfsetlinewidth{0.000000pt}%
\definecolor{currentstroke}{rgb}{0.000000,0.000000,0.000000}%
\pgfsetstrokecolor{currentstroke}%
\pgfsetdash{}{0pt}%
\pgfpathmoveto{\pgfqpoint{3.995263in}{2.693422in}}%
\pgfpathlineto{\pgfqpoint{4.007939in}{2.689230in}}%
\pgfpathlineto{\pgfqpoint{4.020621in}{2.685069in}}%
\pgfpathlineto{\pgfqpoint{4.033308in}{2.680941in}}%
\pgfpathlineto{\pgfqpoint{4.046001in}{2.676844in}}%
\pgfpathlineto{\pgfqpoint{4.038656in}{2.669018in}}%
\pgfpathlineto{\pgfqpoint{4.031306in}{2.661237in}}%
\pgfpathlineto{\pgfqpoint{4.023950in}{2.653499in}}%
\pgfpathlineto{\pgfqpoint{4.016590in}{2.645801in}}%
\pgfpathlineto{\pgfqpoint{4.003886in}{2.649826in}}%
\pgfpathlineto{\pgfqpoint{3.991188in}{2.653883in}}%
\pgfpathlineto{\pgfqpoint{3.978495in}{2.657972in}}%
\pgfpathlineto{\pgfqpoint{3.965807in}{2.662094in}}%
\pgfpathlineto{\pgfqpoint{3.973178in}{2.669859in}}%
\pgfpathlineto{\pgfqpoint{3.980545in}{2.677667in}}%
\pgfpathlineto{\pgfqpoint{3.987906in}{2.685520in}}%
\pgfpathlineto{\pgfqpoint{3.995263in}{2.693422in}}%
\pgfpathclose%
\pgfusepath{fill}%
\end{pgfscope}%
\begin{pgfscope}%
\pgfpathrectangle{\pgfqpoint{1.254980in}{0.150000in}}{\pgfqpoint{5.490039in}{5.490039in}}%
\pgfusepath{clip}%
\pgfsetbuttcap%
\pgfsetroundjoin%
\definecolor{currentfill}{rgb}{0.276022,0.044167,0.370164}%
\pgfsetfillcolor{currentfill}%
\pgfsetfillopacity{0.700000}%
\pgfsetlinewidth{0.000000pt}%
\definecolor{currentstroke}{rgb}{0.000000,0.000000,0.000000}%
\pgfsetstrokecolor{currentstroke}%
\pgfsetdash{}{0pt}%
\pgfpathmoveto{\pgfqpoint{4.337014in}{2.709332in}}%
\pgfpathlineto{\pgfqpoint{4.349763in}{2.705460in}}%
\pgfpathlineto{\pgfqpoint{4.362518in}{2.701617in}}%
\pgfpathlineto{\pgfqpoint{4.375279in}{2.697802in}}%
\pgfpathlineto{\pgfqpoint{4.388045in}{2.694016in}}%
\pgfpathlineto{\pgfqpoint{4.380816in}{2.685985in}}%
\pgfpathlineto{\pgfqpoint{4.373582in}{2.678020in}}%
\pgfpathlineto{\pgfqpoint{4.366344in}{2.670119in}}%
\pgfpathlineto{\pgfqpoint{4.359101in}{2.662278in}}%
\pgfpathlineto{\pgfqpoint{4.346322in}{2.665955in}}%
\pgfpathlineto{\pgfqpoint{4.333550in}{2.669661in}}%
\pgfpathlineto{\pgfqpoint{4.320783in}{2.673396in}}%
\pgfpathlineto{\pgfqpoint{4.308021in}{2.677159in}}%
\pgfpathlineto{\pgfqpoint{4.315276in}{2.685104in}}%
\pgfpathlineto{\pgfqpoint{4.322527in}{2.693112in}}%
\pgfpathlineto{\pgfqpoint{4.329773in}{2.701187in}}%
\pgfpathlineto{\pgfqpoint{4.337014in}{2.709332in}}%
\pgfpathclose%
\pgfusepath{fill}%
\end{pgfscope}%
\begin{pgfscope}%
\pgfpathrectangle{\pgfqpoint{1.254980in}{0.150000in}}{\pgfqpoint{5.490039in}{5.490039in}}%
\pgfusepath{clip}%
\pgfsetbuttcap%
\pgfsetroundjoin%
\definecolor{currentfill}{rgb}{0.273809,0.031497,0.358853}%
\pgfsetfillcolor{currentfill}%
\pgfsetfillopacity{0.700000}%
\pgfsetlinewidth{0.000000pt}%
\definecolor{currentstroke}{rgb}{0.000000,0.000000,0.000000}%
\pgfsetstrokecolor{currentstroke}%
\pgfsetdash{}{0pt}%
\pgfpathmoveto{\pgfqpoint{3.653478in}{2.688695in}}%
\pgfpathlineto{\pgfqpoint{3.666091in}{2.683986in}}%
\pgfpathlineto{\pgfqpoint{3.678708in}{2.679315in}}%
\pgfpathlineto{\pgfqpoint{3.691330in}{2.674680in}}%
\pgfpathlineto{\pgfqpoint{3.703957in}{2.670082in}}%
\pgfpathlineto{\pgfqpoint{3.696492in}{2.662455in}}%
\pgfpathlineto{\pgfqpoint{3.689021in}{2.654865in}}%
\pgfpathlineto{\pgfqpoint{3.681545in}{2.647308in}}%
\pgfpathlineto{\pgfqpoint{3.674064in}{2.639784in}}%
\pgfpathlineto{\pgfqpoint{3.661426in}{2.644349in}}%
\pgfpathlineto{\pgfqpoint{3.648793in}{2.648950in}}%
\pgfpathlineto{\pgfqpoint{3.636164in}{2.653588in}}%
\pgfpathlineto{\pgfqpoint{3.623540in}{2.658262in}}%
\pgfpathlineto{\pgfqpoint{3.631033in}{2.665815in}}%
\pgfpathlineto{\pgfqpoint{3.638520in}{2.673404in}}%
\pgfpathlineto{\pgfqpoint{3.646002in}{2.681030in}}%
\pgfpathlineto{\pgfqpoint{3.653478in}{2.688695in}}%
\pgfpathclose%
\pgfusepath{fill}%
\end{pgfscope}%
\begin{pgfscope}%
\pgfpathrectangle{\pgfqpoint{1.254980in}{0.150000in}}{\pgfqpoint{5.490039in}{5.490039in}}%
\pgfusepath{clip}%
\pgfsetbuttcap%
\pgfsetroundjoin%
\definecolor{currentfill}{rgb}{0.274952,0.037752,0.364543}%
\pgfsetfillcolor{currentfill}%
\pgfsetfillopacity{0.700000}%
\pgfsetlinewidth{0.000000pt}%
\definecolor{currentstroke}{rgb}{0.000000,0.000000,0.000000}%
\pgfsetstrokecolor{currentstroke}%
\pgfsetdash{}{0pt}%
\pgfpathmoveto{\pgfqpoint{4.126107in}{2.692245in}}%
\pgfpathlineto{\pgfqpoint{4.138815in}{2.688221in}}%
\pgfpathlineto{\pgfqpoint{4.151527in}{2.684227in}}%
\pgfpathlineto{\pgfqpoint{4.164246in}{2.680264in}}%
\pgfpathlineto{\pgfqpoint{4.176969in}{2.676331in}}%
\pgfpathlineto{\pgfqpoint{4.169667in}{2.668474in}}%
\pgfpathlineto{\pgfqpoint{4.162360in}{2.660667in}}%
\pgfpathlineto{\pgfqpoint{4.155048in}{2.652908in}}%
\pgfpathlineto{\pgfqpoint{4.147731in}{2.645193in}}%
\pgfpathlineto{\pgfqpoint{4.134996in}{2.649041in}}%
\pgfpathlineto{\pgfqpoint{4.122266in}{2.652921in}}%
\pgfpathlineto{\pgfqpoint{4.109542in}{2.656830in}}%
\pgfpathlineto{\pgfqpoint{4.096823in}{2.660771in}}%
\pgfpathlineto{\pgfqpoint{4.104152in}{2.668566in}}%
\pgfpathlineto{\pgfqpoint{4.111475in}{2.676408in}}%
\pgfpathlineto{\pgfqpoint{4.118794in}{2.684300in}}%
\pgfpathlineto{\pgfqpoint{4.126107in}{2.692245in}}%
\pgfpathclose%
\pgfusepath{fill}%
\end{pgfscope}%
\begin{pgfscope}%
\pgfpathrectangle{\pgfqpoint{1.254980in}{0.150000in}}{\pgfqpoint{5.490039in}{5.490039in}}%
\pgfusepath{clip}%
\pgfsetbuttcap%
\pgfsetroundjoin%
\definecolor{currentfill}{rgb}{0.273809,0.031497,0.358853}%
\pgfsetfillcolor{currentfill}%
\pgfsetfillopacity{0.700000}%
\pgfsetlinewidth{0.000000pt}%
\definecolor{currentstroke}{rgb}{0.000000,0.000000,0.000000}%
\pgfsetstrokecolor{currentstroke}%
\pgfsetdash{}{0pt}%
\pgfpathmoveto{\pgfqpoint{3.784270in}{2.682761in}}%
\pgfpathlineto{\pgfqpoint{3.796909in}{2.678295in}}%
\pgfpathlineto{\pgfqpoint{3.809553in}{2.673864in}}%
\pgfpathlineto{\pgfqpoint{3.822202in}{2.669468in}}%
\pgfpathlineto{\pgfqpoint{3.834856in}{2.665106in}}%
\pgfpathlineto{\pgfqpoint{3.827435in}{2.657421in}}%
\pgfpathlineto{\pgfqpoint{3.820009in}{2.649773in}}%
\pgfpathlineto{\pgfqpoint{3.812578in}{2.642159in}}%
\pgfpathlineto{\pgfqpoint{3.805141in}{2.634578in}}%
\pgfpathlineto{\pgfqpoint{3.792476in}{2.638894in}}%
\pgfpathlineto{\pgfqpoint{3.779816in}{2.643244in}}%
\pgfpathlineto{\pgfqpoint{3.767161in}{2.647629in}}%
\pgfpathlineto{\pgfqpoint{3.754510in}{2.652049in}}%
\pgfpathlineto{\pgfqpoint{3.761958in}{2.659671in}}%
\pgfpathlineto{\pgfqpoint{3.769401in}{2.667330in}}%
\pgfpathlineto{\pgfqpoint{3.776838in}{2.675026in}}%
\pgfpathlineto{\pgfqpoint{3.784270in}{2.682761in}}%
\pgfpathclose%
\pgfusepath{fill}%
\end{pgfscope}%
\begin{pgfscope}%
\pgfpathrectangle{\pgfqpoint{1.254980in}{0.150000in}}{\pgfqpoint{5.490039in}{5.490039in}}%
\pgfusepath{clip}%
\pgfsetbuttcap%
\pgfsetroundjoin%
\definecolor{currentfill}{rgb}{0.278791,0.062145,0.386592}%
\pgfsetfillcolor{currentfill}%
\pgfsetfillopacity{0.700000}%
\pgfsetlinewidth{0.000000pt}%
\definecolor{currentstroke}{rgb}{0.000000,0.000000,0.000000}%
\pgfsetstrokecolor{currentstroke}%
\pgfsetdash{}{0pt}%
\pgfpathmoveto{\pgfqpoint{4.678926in}{2.733843in}}%
\pgfpathlineto{\pgfqpoint{4.691752in}{2.730121in}}%
\pgfpathlineto{\pgfqpoint{4.704585in}{2.726426in}}%
\pgfpathlineto{\pgfqpoint{4.717423in}{2.722757in}}%
\pgfpathlineto{\pgfqpoint{4.730268in}{2.719115in}}%
\pgfpathlineto{\pgfqpoint{4.723148in}{2.710718in}}%
\pgfpathlineto{\pgfqpoint{4.716025in}{2.702423in}}%
\pgfpathlineto{\pgfqpoint{4.708898in}{2.694225in}}%
\pgfpathlineto{\pgfqpoint{4.701768in}{2.686121in}}%
\pgfpathlineto{\pgfqpoint{4.688911in}{2.689617in}}%
\pgfpathlineto{\pgfqpoint{4.676059in}{2.693139in}}%
\pgfpathlineto{\pgfqpoint{4.663214in}{2.696688in}}%
\pgfpathlineto{\pgfqpoint{4.650375in}{2.700263in}}%
\pgfpathlineto{\pgfqpoint{4.657518in}{2.708510in}}%
\pgfpathlineto{\pgfqpoint{4.664657in}{2.716852in}}%
\pgfpathlineto{\pgfqpoint{4.671793in}{2.725295in}}%
\pgfpathlineto{\pgfqpoint{4.678926in}{2.733843in}}%
\pgfpathclose%
\pgfusepath{fill}%
\end{pgfscope}%
\begin{pgfscope}%
\pgfpathrectangle{\pgfqpoint{1.254980in}{0.150000in}}{\pgfqpoint{5.490039in}{5.490039in}}%
\pgfusepath{clip}%
\pgfsetbuttcap%
\pgfsetroundjoin%
\definecolor{currentfill}{rgb}{0.277018,0.050344,0.375715}%
\pgfsetfillcolor{currentfill}%
\pgfsetfillopacity{0.700000}%
\pgfsetlinewidth{0.000000pt}%
\definecolor{currentstroke}{rgb}{0.000000,0.000000,0.000000}%
\pgfsetstrokecolor{currentstroke}%
\pgfsetdash{}{0pt}%
\pgfpathmoveto{\pgfqpoint{4.467993in}{2.711542in}}%
\pgfpathlineto{\pgfqpoint{4.480776in}{2.707775in}}%
\pgfpathlineto{\pgfqpoint{4.493564in}{2.704036in}}%
\pgfpathlineto{\pgfqpoint{4.506358in}{2.700324in}}%
\pgfpathlineto{\pgfqpoint{4.519157in}{2.696640in}}%
\pgfpathlineto{\pgfqpoint{4.511969in}{2.688555in}}%
\pgfpathlineto{\pgfqpoint{4.504777in}{2.680545in}}%
\pgfpathlineto{\pgfqpoint{4.497581in}{2.672608in}}%
\pgfpathlineto{\pgfqpoint{4.490380in}{2.664739in}}%
\pgfpathlineto{\pgfqpoint{4.477568in}{2.668301in}}%
\pgfpathlineto{\pgfqpoint{4.464762in}{2.671891in}}%
\pgfpathlineto{\pgfqpoint{4.451962in}{2.675509in}}%
\pgfpathlineto{\pgfqpoint{4.439167in}{2.679154in}}%
\pgfpathlineto{\pgfqpoint{4.446380in}{2.687140in}}%
\pgfpathlineto{\pgfqpoint{4.453589in}{2.695197in}}%
\pgfpathlineto{\pgfqpoint{4.460793in}{2.703330in}}%
\pgfpathlineto{\pgfqpoint{4.467993in}{2.711542in}}%
\pgfpathclose%
\pgfusepath{fill}%
\end{pgfscope}%
\begin{pgfscope}%
\pgfpathrectangle{\pgfqpoint{1.254980in}{0.150000in}}{\pgfqpoint{5.490039in}{5.490039in}}%
\pgfusepath{clip}%
\pgfsetbuttcap%
\pgfsetroundjoin%
\definecolor{currentfill}{rgb}{0.276022,0.044167,0.370164}%
\pgfsetfillcolor{currentfill}%
\pgfsetfillopacity{0.700000}%
\pgfsetlinewidth{0.000000pt}%
\definecolor{currentstroke}{rgb}{0.000000,0.000000,0.000000}%
\pgfsetstrokecolor{currentstroke}%
\pgfsetdash{}{0pt}%
\pgfpathmoveto{\pgfqpoint{3.311362in}{2.700171in}}%
\pgfpathlineto{\pgfqpoint{3.323924in}{2.694715in}}%
\pgfpathlineto{\pgfqpoint{3.336491in}{2.689304in}}%
\pgfpathlineto{\pgfqpoint{3.349061in}{2.683936in}}%
\pgfpathlineto{\pgfqpoint{3.361636in}{2.678612in}}%
\pgfpathlineto{\pgfqpoint{3.354044in}{2.671330in}}%
\pgfpathlineto{\pgfqpoint{3.346446in}{2.664087in}}%
\pgfpathlineto{\pgfqpoint{3.338842in}{2.656883in}}%
\pgfpathlineto{\pgfqpoint{3.331232in}{2.649717in}}%
\pgfpathlineto{\pgfqpoint{3.318645in}{2.655046in}}%
\pgfpathlineto{\pgfqpoint{3.306063in}{2.660418in}}%
\pgfpathlineto{\pgfqpoint{3.293484in}{2.665833in}}%
\pgfpathlineto{\pgfqpoint{3.280909in}{2.671293in}}%
\pgfpathlineto{\pgfqpoint{3.288531in}{2.678450in}}%
\pgfpathlineto{\pgfqpoint{3.296148in}{2.685648in}}%
\pgfpathlineto{\pgfqpoint{3.303758in}{2.692888in}}%
\pgfpathlineto{\pgfqpoint{3.311362in}{2.700171in}}%
\pgfpathclose%
\pgfusepath{fill}%
\end{pgfscope}%
\begin{pgfscope}%
\pgfpathrectangle{\pgfqpoint{1.254980in}{0.150000in}}{\pgfqpoint{5.490039in}{5.490039in}}%
\pgfusepath{clip}%
\pgfsetbuttcap%
\pgfsetroundjoin%
\definecolor{currentfill}{rgb}{0.273809,0.031497,0.358853}%
\pgfsetfillcolor{currentfill}%
\pgfsetfillopacity{0.700000}%
\pgfsetlinewidth{0.000000pt}%
\definecolor{currentstroke}{rgb}{0.000000,0.000000,0.000000}%
\pgfsetstrokecolor{currentstroke}%
\pgfsetdash{}{0pt}%
\pgfpathmoveto{\pgfqpoint{3.915106in}{2.678906in}}%
\pgfpathlineto{\pgfqpoint{3.927773in}{2.674653in}}%
\pgfpathlineto{\pgfqpoint{3.940446in}{2.670434in}}%
\pgfpathlineto{\pgfqpoint{3.953124in}{2.666248in}}%
\pgfpathlineto{\pgfqpoint{3.965807in}{2.662094in}}%
\pgfpathlineto{\pgfqpoint{3.958430in}{2.654370in}}%
\pgfpathlineto{\pgfqpoint{3.951047in}{2.646684in}}%
\pgfpathlineto{\pgfqpoint{3.943660in}{2.639035in}}%
\pgfpathlineto{\pgfqpoint{3.936267in}{2.631421in}}%
\pgfpathlineto{\pgfqpoint{3.923573in}{2.635516in}}%
\pgfpathlineto{\pgfqpoint{3.910884in}{2.639643in}}%
\pgfpathlineto{\pgfqpoint{3.898200in}{2.643803in}}%
\pgfpathlineto{\pgfqpoint{3.885521in}{2.647997in}}%
\pgfpathlineto{\pgfqpoint{3.892925in}{2.655666in}}%
\pgfpathlineto{\pgfqpoint{3.900324in}{2.663372in}}%
\pgfpathlineto{\pgfqpoint{3.907717in}{2.671118in}}%
\pgfpathlineto{\pgfqpoint{3.915106in}{2.678906in}}%
\pgfpathclose%
\pgfusepath{fill}%
\end{pgfscope}%
\begin{pgfscope}%
\pgfpathrectangle{\pgfqpoint{1.254980in}{0.150000in}}{\pgfqpoint{5.490039in}{5.490039in}}%
\pgfusepath{clip}%
\pgfsetbuttcap%
\pgfsetroundjoin%
\definecolor{currentfill}{rgb}{0.277018,0.050344,0.375715}%
\pgfsetfillcolor{currentfill}%
\pgfsetfillopacity{0.700000}%
\pgfsetlinewidth{0.000000pt}%
\definecolor{currentstroke}{rgb}{0.000000,0.000000,0.000000}%
\pgfsetstrokecolor{currentstroke}%
\pgfsetdash{}{0pt}%
\pgfpathmoveto{\pgfqpoint{3.180443in}{2.716606in}}%
\pgfpathlineto{\pgfqpoint{3.192989in}{2.710779in}}%
\pgfpathlineto{\pgfqpoint{3.205538in}{2.704999in}}%
\pgfpathlineto{\pgfqpoint{3.218091in}{2.699266in}}%
\pgfpathlineto{\pgfqpoint{3.230647in}{2.693580in}}%
\pgfpathlineto{\pgfqpoint{3.223006in}{2.686476in}}%
\pgfpathlineto{\pgfqpoint{3.215358in}{2.679416in}}%
\pgfpathlineto{\pgfqpoint{3.207705in}{2.672399in}}%
\pgfpathlineto{\pgfqpoint{3.200044in}{2.665427in}}%
\pgfpathlineto{\pgfqpoint{3.187475in}{2.671130in}}%
\pgfpathlineto{\pgfqpoint{3.174910in}{2.676879in}}%
\pgfpathlineto{\pgfqpoint{3.162348in}{2.682676in}}%
\pgfpathlineto{\pgfqpoint{3.149789in}{2.688520in}}%
\pgfpathlineto{\pgfqpoint{3.157462in}{2.695472in}}%
\pgfpathlineto{\pgfqpoint{3.165129in}{2.702470in}}%
\pgfpathlineto{\pgfqpoint{3.172789in}{2.709514in}}%
\pgfpathlineto{\pgfqpoint{3.180443in}{2.716606in}}%
\pgfpathclose%
\pgfusepath{fill}%
\end{pgfscope}%
\begin{pgfscope}%
\pgfpathrectangle{\pgfqpoint{1.254980in}{0.150000in}}{\pgfqpoint{5.490039in}{5.490039in}}%
\pgfusepath{clip}%
\pgfsetbuttcap%
\pgfsetroundjoin%
\definecolor{currentfill}{rgb}{0.274952,0.037752,0.364543}%
\pgfsetfillcolor{currentfill}%
\pgfsetfillopacity{0.700000}%
\pgfsetlinewidth{0.000000pt}%
\definecolor{currentstroke}{rgb}{0.000000,0.000000,0.000000}%
\pgfsetstrokecolor{currentstroke}%
\pgfsetdash{}{0pt}%
\pgfpathmoveto{\pgfqpoint{4.257032in}{2.692503in}}%
\pgfpathlineto{\pgfqpoint{4.269771in}{2.688623in}}%
\pgfpathlineto{\pgfqpoint{4.282516in}{2.684772in}}%
\pgfpathlineto{\pgfqpoint{4.295266in}{2.680951in}}%
\pgfpathlineto{\pgfqpoint{4.308021in}{2.677159in}}%
\pgfpathlineto{\pgfqpoint{4.300762in}{2.669273in}}%
\pgfpathlineto{\pgfqpoint{4.293498in}{2.661444in}}%
\pgfpathlineto{\pgfqpoint{4.286228in}{2.653668in}}%
\pgfpathlineto{\pgfqpoint{4.278954in}{2.645942in}}%
\pgfpathlineto{\pgfqpoint{4.266187in}{2.649638in}}%
\pgfpathlineto{\pgfqpoint{4.253425in}{2.653362in}}%
\pgfpathlineto{\pgfqpoint{4.240668in}{2.657116in}}%
\pgfpathlineto{\pgfqpoint{4.227918in}{2.660900in}}%
\pgfpathlineto{\pgfqpoint{4.235203in}{2.668718in}}%
\pgfpathlineto{\pgfqpoint{4.242484in}{2.676589in}}%
\pgfpathlineto{\pgfqpoint{4.249760in}{2.684516in}}%
\pgfpathlineto{\pgfqpoint{4.257032in}{2.692503in}}%
\pgfpathclose%
\pgfusepath{fill}%
\end{pgfscope}%
\begin{pgfscope}%
\pgfpathrectangle{\pgfqpoint{1.254980in}{0.150000in}}{\pgfqpoint{5.490039in}{5.490039in}}%
\pgfusepath{clip}%
\pgfsetbuttcap%
\pgfsetroundjoin%
\definecolor{currentfill}{rgb}{0.274952,0.037752,0.364543}%
\pgfsetfillcolor{currentfill}%
\pgfsetfillopacity{0.700000}%
\pgfsetlinewidth{0.000000pt}%
\definecolor{currentstroke}{rgb}{0.000000,0.000000,0.000000}%
\pgfsetstrokecolor{currentstroke}%
\pgfsetdash{}{0pt}%
\pgfpathmoveto{\pgfqpoint{3.442234in}{2.687233in}}%
\pgfpathlineto{\pgfqpoint{3.454816in}{2.682110in}}%
\pgfpathlineto{\pgfqpoint{3.467403in}{2.677028in}}%
\pgfpathlineto{\pgfqpoint{3.479995in}{2.671987in}}%
\pgfpathlineto{\pgfqpoint{3.492590in}{2.666986in}}%
\pgfpathlineto{\pgfqpoint{3.485046in}{2.659568in}}%
\pgfpathlineto{\pgfqpoint{3.477495in}{2.652186in}}%
\pgfpathlineto{\pgfqpoint{3.469939in}{2.644839in}}%
\pgfpathlineto{\pgfqpoint{3.462377in}{2.637525in}}%
\pgfpathlineto{\pgfqpoint{3.449770in}{2.642517in}}%
\pgfpathlineto{\pgfqpoint{3.437167in}{2.647550in}}%
\pgfpathlineto{\pgfqpoint{3.424568in}{2.652623in}}%
\pgfpathlineto{\pgfqpoint{3.411973in}{2.657738in}}%
\pgfpathlineto{\pgfqpoint{3.419547in}{2.665055in}}%
\pgfpathlineto{\pgfqpoint{3.427115in}{2.672410in}}%
\pgfpathlineto{\pgfqpoint{3.434677in}{2.679802in}}%
\pgfpathlineto{\pgfqpoint{3.442234in}{2.687233in}}%
\pgfpathclose%
\pgfusepath{fill}%
\end{pgfscope}%
\begin{pgfscope}%
\pgfpathrectangle{\pgfqpoint{1.254980in}{0.150000in}}{\pgfqpoint{5.490039in}{5.490039in}}%
\pgfusepath{clip}%
\pgfsetbuttcap%
\pgfsetroundjoin%
\definecolor{currentfill}{rgb}{0.273809,0.031497,0.358853}%
\pgfsetfillcolor{currentfill}%
\pgfsetfillopacity{0.700000}%
\pgfsetlinewidth{0.000000pt}%
\definecolor{currentstroke}{rgb}{0.000000,0.000000,0.000000}%
\pgfsetstrokecolor{currentstroke}%
\pgfsetdash{}{0pt}%
\pgfpathmoveto{\pgfqpoint{3.573090in}{2.677338in}}%
\pgfpathlineto{\pgfqpoint{3.585696in}{2.672512in}}%
\pgfpathlineto{\pgfqpoint{3.598306in}{2.667724in}}%
\pgfpathlineto{\pgfqpoint{3.610921in}{2.662975in}}%
\pgfpathlineto{\pgfqpoint{3.623540in}{2.658262in}}%
\pgfpathlineto{\pgfqpoint{3.616042in}{2.650744in}}%
\pgfpathlineto{\pgfqpoint{3.608537in}{2.643260in}}%
\pgfpathlineto{\pgfqpoint{3.601028in}{2.635808in}}%
\pgfpathlineto{\pgfqpoint{3.593512in}{2.628386in}}%
\pgfpathlineto{\pgfqpoint{3.580881in}{2.633077in}}%
\pgfpathlineto{\pgfqpoint{3.568255in}{2.637806in}}%
\pgfpathlineto{\pgfqpoint{3.555633in}{2.642572in}}%
\pgfpathlineto{\pgfqpoint{3.543016in}{2.647377in}}%
\pgfpathlineto{\pgfqpoint{3.550543in}{2.654815in}}%
\pgfpathlineto{\pgfqpoint{3.558064in}{2.662287in}}%
\pgfpathlineto{\pgfqpoint{3.565580in}{2.669794in}}%
\pgfpathlineto{\pgfqpoint{3.573090in}{2.677338in}}%
\pgfpathclose%
\pgfusepath{fill}%
\end{pgfscope}%
\begin{pgfscope}%
\pgfpathrectangle{\pgfqpoint{1.254980in}{0.150000in}}{\pgfqpoint{5.490039in}{5.490039in}}%
\pgfusepath{clip}%
\pgfsetbuttcap%
\pgfsetroundjoin%
\definecolor{currentfill}{rgb}{0.279566,0.067836,0.391917}%
\pgfsetfillcolor{currentfill}%
\pgfsetfillopacity{0.700000}%
\pgfsetlinewidth{0.000000pt}%
\definecolor{currentstroke}{rgb}{0.000000,0.000000,0.000000}%
\pgfsetstrokecolor{currentstroke}%
\pgfsetdash{}{0pt}%
\pgfpathmoveto{\pgfqpoint{4.810099in}{2.738873in}}%
\pgfpathlineto{\pgfqpoint{4.822959in}{2.735201in}}%
\pgfpathlineto{\pgfqpoint{4.835826in}{2.731556in}}%
\pgfpathlineto{\pgfqpoint{4.848698in}{2.727935in}}%
\pgfpathlineto{\pgfqpoint{4.861577in}{2.724340in}}%
\pgfpathlineto{\pgfqpoint{4.854496in}{2.715814in}}%
\pgfpathlineto{\pgfqpoint{4.847413in}{2.707403in}}%
\pgfpathlineto{\pgfqpoint{4.840326in}{2.699102in}}%
\pgfpathlineto{\pgfqpoint{4.833236in}{2.690907in}}%
\pgfpathlineto{\pgfqpoint{4.820344in}{2.694343in}}%
\pgfpathlineto{\pgfqpoint{4.807458in}{2.697804in}}%
\pgfpathlineto{\pgfqpoint{4.794578in}{2.701292in}}%
\pgfpathlineto{\pgfqpoint{4.781704in}{2.704805in}}%
\pgfpathlineto{\pgfqpoint{4.788807in}{2.713154in}}%
\pgfpathlineto{\pgfqpoint{4.795907in}{2.721612in}}%
\pgfpathlineto{\pgfqpoint{4.803004in}{2.730183in}}%
\pgfpathlineto{\pgfqpoint{4.810099in}{2.738873in}}%
\pgfpathclose%
\pgfusepath{fill}%
\end{pgfscope}%
\begin{pgfscope}%
\pgfpathrectangle{\pgfqpoint{1.254980in}{0.150000in}}{\pgfqpoint{5.490039in}{5.490039in}}%
\pgfusepath{clip}%
\pgfsetbuttcap%
\pgfsetroundjoin%
\definecolor{currentfill}{rgb}{0.277941,0.056324,0.381191}%
\pgfsetfillcolor{currentfill}%
\pgfsetfillopacity{0.700000}%
\pgfsetlinewidth{0.000000pt}%
\definecolor{currentstroke}{rgb}{0.000000,0.000000,0.000000}%
\pgfsetstrokecolor{currentstroke}%
\pgfsetdash{}{0pt}%
\pgfpathmoveto{\pgfqpoint{4.599075in}{2.714831in}}%
\pgfpathlineto{\pgfqpoint{4.611891in}{2.711149in}}%
\pgfpathlineto{\pgfqpoint{4.624713in}{2.707494in}}%
\pgfpathlineto{\pgfqpoint{4.637541in}{2.703865in}}%
\pgfpathlineto{\pgfqpoint{4.650375in}{2.700263in}}%
\pgfpathlineto{\pgfqpoint{4.643228in}{2.692108in}}%
\pgfpathlineto{\pgfqpoint{4.636077in}{2.684039in}}%
\pgfpathlineto{\pgfqpoint{4.628923in}{2.676054in}}%
\pgfpathlineto{\pgfqpoint{4.621764in}{2.668146in}}%
\pgfpathlineto{\pgfqpoint{4.608918in}{2.671614in}}%
\pgfpathlineto{\pgfqpoint{4.596077in}{2.675108in}}%
\pgfpathlineto{\pgfqpoint{4.583242in}{2.678629in}}%
\pgfpathlineto{\pgfqpoint{4.570414in}{2.682177in}}%
\pgfpathlineto{\pgfqpoint{4.577585in}{2.690214in}}%
\pgfpathlineto{\pgfqpoint{4.584752in}{2.698333in}}%
\pgfpathlineto{\pgfqpoint{4.591916in}{2.706537in}}%
\pgfpathlineto{\pgfqpoint{4.599075in}{2.714831in}}%
\pgfpathclose%
\pgfusepath{fill}%
\end{pgfscope}%
\begin{pgfscope}%
\pgfpathrectangle{\pgfqpoint{1.254980in}{0.150000in}}{\pgfqpoint{5.490039in}{5.490039in}}%
\pgfusepath{clip}%
\pgfsetbuttcap%
\pgfsetroundjoin%
\definecolor{currentfill}{rgb}{0.273809,0.031497,0.358853}%
\pgfsetfillcolor{currentfill}%
\pgfsetfillopacity{0.700000}%
\pgfsetlinewidth{0.000000pt}%
\definecolor{currentstroke}{rgb}{0.000000,0.000000,0.000000}%
\pgfsetstrokecolor{currentstroke}%
\pgfsetdash{}{0pt}%
\pgfpathmoveto{\pgfqpoint{4.046001in}{2.676844in}}%
\pgfpathlineto{\pgfqpoint{4.058698in}{2.672779in}}%
\pgfpathlineto{\pgfqpoint{4.071401in}{2.668745in}}%
\pgfpathlineto{\pgfqpoint{4.084109in}{2.664742in}}%
\pgfpathlineto{\pgfqpoint{4.096823in}{2.660771in}}%
\pgfpathlineto{\pgfqpoint{4.089489in}{2.653021in}}%
\pgfpathlineto{\pgfqpoint{4.082151in}{2.645313in}}%
\pgfpathlineto{\pgfqpoint{4.074807in}{2.637645in}}%
\pgfpathlineto{\pgfqpoint{4.067458in}{2.630014in}}%
\pgfpathlineto{\pgfqpoint{4.054733in}{2.633914in}}%
\pgfpathlineto{\pgfqpoint{4.042013in}{2.637845in}}%
\pgfpathlineto{\pgfqpoint{4.029299in}{2.641807in}}%
\pgfpathlineto{\pgfqpoint{4.016590in}{2.645801in}}%
\pgfpathlineto{\pgfqpoint{4.023950in}{2.653499in}}%
\pgfpathlineto{\pgfqpoint{4.031306in}{2.661237in}}%
\pgfpathlineto{\pgfqpoint{4.038656in}{2.669018in}}%
\pgfpathlineto{\pgfqpoint{4.046001in}{2.676844in}}%
\pgfpathclose%
\pgfusepath{fill}%
\end{pgfscope}%
\begin{pgfscope}%
\pgfpathrectangle{\pgfqpoint{1.254980in}{0.150000in}}{\pgfqpoint{5.490039in}{5.490039in}}%
\pgfusepath{clip}%
\pgfsetbuttcap%
\pgfsetroundjoin%
\definecolor{currentfill}{rgb}{0.272594,0.025563,0.353093}%
\pgfsetfillcolor{currentfill}%
\pgfsetfillopacity{0.700000}%
\pgfsetlinewidth{0.000000pt}%
\definecolor{currentstroke}{rgb}{0.000000,0.000000,0.000000}%
\pgfsetstrokecolor{currentstroke}%
\pgfsetdash{}{0pt}%
\pgfpathmoveto{\pgfqpoint{3.703957in}{2.670082in}}%
\pgfpathlineto{\pgfqpoint{3.716588in}{2.665520in}}%
\pgfpathlineto{\pgfqpoint{3.729224in}{2.660994in}}%
\pgfpathlineto{\pgfqpoint{3.741865in}{2.656504in}}%
\pgfpathlineto{\pgfqpoint{3.754510in}{2.652049in}}%
\pgfpathlineto{\pgfqpoint{3.747057in}{2.644461in}}%
\pgfpathlineto{\pgfqpoint{3.739598in}{2.636905in}}%
\pgfpathlineto{\pgfqpoint{3.732133in}{2.629381in}}%
\pgfpathlineto{\pgfqpoint{3.724663in}{2.621887in}}%
\pgfpathlineto{\pgfqpoint{3.712006in}{2.626308in}}%
\pgfpathlineto{\pgfqpoint{3.699354in}{2.630764in}}%
\pgfpathlineto{\pgfqpoint{3.686707in}{2.635256in}}%
\pgfpathlineto{\pgfqpoint{3.674064in}{2.639784in}}%
\pgfpathlineto{\pgfqpoint{3.681545in}{2.647308in}}%
\pgfpathlineto{\pgfqpoint{3.689021in}{2.654865in}}%
\pgfpathlineto{\pgfqpoint{3.696492in}{2.662455in}}%
\pgfpathlineto{\pgfqpoint{3.703957in}{2.670082in}}%
\pgfpathclose%
\pgfusepath{fill}%
\end{pgfscope}%
\begin{pgfscope}%
\pgfpathrectangle{\pgfqpoint{1.254980in}{0.150000in}}{\pgfqpoint{5.490039in}{5.490039in}}%
\pgfusepath{clip}%
\pgfsetbuttcap%
\pgfsetroundjoin%
\definecolor{currentfill}{rgb}{0.276022,0.044167,0.370164}%
\pgfsetfillcolor{currentfill}%
\pgfsetfillopacity{0.700000}%
\pgfsetlinewidth{0.000000pt}%
\definecolor{currentstroke}{rgb}{0.000000,0.000000,0.000000}%
\pgfsetstrokecolor{currentstroke}%
\pgfsetdash{}{0pt}%
\pgfpathmoveto{\pgfqpoint{4.388045in}{2.694016in}}%
\pgfpathlineto{\pgfqpoint{4.400817in}{2.690258in}}%
\pgfpathlineto{\pgfqpoint{4.413595in}{2.686529in}}%
\pgfpathlineto{\pgfqpoint{4.426378in}{2.682828in}}%
\pgfpathlineto{\pgfqpoint{4.439167in}{2.679154in}}%
\pgfpathlineto{\pgfqpoint{4.431950in}{2.671236in}}%
\pgfpathlineto{\pgfqpoint{4.424728in}{2.663382in}}%
\pgfpathlineto{\pgfqpoint{4.417502in}{2.655589in}}%
\pgfpathlineto{\pgfqpoint{4.410271in}{2.647853in}}%
\pgfpathlineto{\pgfqpoint{4.397470in}{2.651417in}}%
\pgfpathlineto{\pgfqpoint{4.384674in}{2.655009in}}%
\pgfpathlineto{\pgfqpoint{4.371885in}{2.658630in}}%
\pgfpathlineto{\pgfqpoint{4.359101in}{2.662278in}}%
\pgfpathlineto{\pgfqpoint{4.366344in}{2.670119in}}%
\pgfpathlineto{\pgfqpoint{4.373582in}{2.678020in}}%
\pgfpathlineto{\pgfqpoint{4.380816in}{2.685985in}}%
\pgfpathlineto{\pgfqpoint{4.388045in}{2.694016in}}%
\pgfpathclose%
\pgfusepath{fill}%
\end{pgfscope}%
\begin{pgfscope}%
\pgfpathrectangle{\pgfqpoint{1.254980in}{0.150000in}}{\pgfqpoint{5.490039in}{5.490039in}}%
\pgfusepath{clip}%
\pgfsetbuttcap%
\pgfsetroundjoin%
\definecolor{currentfill}{rgb}{0.272594,0.025563,0.353093}%
\pgfsetfillcolor{currentfill}%
\pgfsetfillopacity{0.700000}%
\pgfsetlinewidth{0.000000pt}%
\definecolor{currentstroke}{rgb}{0.000000,0.000000,0.000000}%
\pgfsetstrokecolor{currentstroke}%
\pgfsetdash{}{0pt}%
\pgfpathmoveto{\pgfqpoint{3.834856in}{2.665106in}}%
\pgfpathlineto{\pgfqpoint{3.847515in}{2.660778in}}%
\pgfpathlineto{\pgfqpoint{3.860179in}{2.656484in}}%
\pgfpathlineto{\pgfqpoint{3.872847in}{2.652224in}}%
\pgfpathlineto{\pgfqpoint{3.885521in}{2.647997in}}%
\pgfpathlineto{\pgfqpoint{3.878112in}{2.640363in}}%
\pgfpathlineto{\pgfqpoint{3.870697in}{2.632763in}}%
\pgfpathlineto{\pgfqpoint{3.863277in}{2.625194in}}%
\pgfpathlineto{\pgfqpoint{3.855852in}{2.617655in}}%
\pgfpathlineto{\pgfqpoint{3.843167in}{2.621836in}}%
\pgfpathlineto{\pgfqpoint{3.830487in}{2.626049in}}%
\pgfpathlineto{\pgfqpoint{3.817811in}{2.630297in}}%
\pgfpathlineto{\pgfqpoint{3.805141in}{2.634578in}}%
\pgfpathlineto{\pgfqpoint{3.812578in}{2.642159in}}%
\pgfpathlineto{\pgfqpoint{3.820009in}{2.649773in}}%
\pgfpathlineto{\pgfqpoint{3.827435in}{2.657421in}}%
\pgfpathlineto{\pgfqpoint{3.834856in}{2.665106in}}%
\pgfpathclose%
\pgfusepath{fill}%
\end{pgfscope}%
\begin{pgfscope}%
\pgfpathrectangle{\pgfqpoint{1.254980in}{0.150000in}}{\pgfqpoint{5.490039in}{5.490039in}}%
\pgfusepath{clip}%
\pgfsetbuttcap%
\pgfsetroundjoin%
\definecolor{currentfill}{rgb}{0.274952,0.037752,0.364543}%
\pgfsetfillcolor{currentfill}%
\pgfsetfillopacity{0.700000}%
\pgfsetlinewidth{0.000000pt}%
\definecolor{currentstroke}{rgb}{0.000000,0.000000,0.000000}%
\pgfsetstrokecolor{currentstroke}%
\pgfsetdash{}{0pt}%
\pgfpathmoveto{\pgfqpoint{4.176969in}{2.676331in}}%
\pgfpathlineto{\pgfqpoint{4.189698in}{2.672428in}}%
\pgfpathlineto{\pgfqpoint{4.202432in}{2.668556in}}%
\pgfpathlineto{\pgfqpoint{4.215172in}{2.664713in}}%
\pgfpathlineto{\pgfqpoint{4.227918in}{2.660900in}}%
\pgfpathlineto{\pgfqpoint{4.220627in}{2.653132in}}%
\pgfpathlineto{\pgfqpoint{4.213332in}{2.645411in}}%
\pgfpathlineto{\pgfqpoint{4.206031in}{2.637734in}}%
\pgfpathlineto{\pgfqpoint{4.198726in}{2.630098in}}%
\pgfpathlineto{\pgfqpoint{4.185969in}{2.633827in}}%
\pgfpathlineto{\pgfqpoint{4.173217in}{2.637585in}}%
\pgfpathlineto{\pgfqpoint{4.160471in}{2.641374in}}%
\pgfpathlineto{\pgfqpoint{4.147731in}{2.645193in}}%
\pgfpathlineto{\pgfqpoint{4.155048in}{2.652908in}}%
\pgfpathlineto{\pgfqpoint{4.162360in}{2.660667in}}%
\pgfpathlineto{\pgfqpoint{4.169667in}{2.668474in}}%
\pgfpathlineto{\pgfqpoint{4.176969in}{2.676331in}}%
\pgfpathclose%
\pgfusepath{fill}%
\end{pgfscope}%
\begin{pgfscope}%
\pgfpathrectangle{\pgfqpoint{1.254980in}{0.150000in}}{\pgfqpoint{5.490039in}{5.490039in}}%
\pgfusepath{clip}%
\pgfsetbuttcap%
\pgfsetroundjoin%
\definecolor{currentfill}{rgb}{0.278791,0.062145,0.386592}%
\pgfsetfillcolor{currentfill}%
\pgfsetfillopacity{0.700000}%
\pgfsetlinewidth{0.000000pt}%
\definecolor{currentstroke}{rgb}{0.000000,0.000000,0.000000}%
\pgfsetstrokecolor{currentstroke}%
\pgfsetdash{}{0pt}%
\pgfpathmoveto{\pgfqpoint{4.730268in}{2.719115in}}%
\pgfpathlineto{\pgfqpoint{4.743118in}{2.715498in}}%
\pgfpathlineto{\pgfqpoint{4.755974in}{2.711908in}}%
\pgfpathlineto{\pgfqpoint{4.768836in}{2.708343in}}%
\pgfpathlineto{\pgfqpoint{4.781704in}{2.704805in}}%
\pgfpathlineto{\pgfqpoint{4.774597in}{2.696559in}}%
\pgfpathlineto{\pgfqpoint{4.767488in}{2.688412in}}%
\pgfpathlineto{\pgfqpoint{4.760374in}{2.680359in}}%
\pgfpathlineto{\pgfqpoint{4.753257in}{2.672396in}}%
\pgfpathlineto{\pgfqpoint{4.740376in}{2.675788in}}%
\pgfpathlineto{\pgfqpoint{4.727501in}{2.679206in}}%
\pgfpathlineto{\pgfqpoint{4.714631in}{2.682650in}}%
\pgfpathlineto{\pgfqpoint{4.701768in}{2.686121in}}%
\pgfpathlineto{\pgfqpoint{4.708898in}{2.694225in}}%
\pgfpathlineto{\pgfqpoint{4.716025in}{2.702423in}}%
\pgfpathlineto{\pgfqpoint{4.723148in}{2.710718in}}%
\pgfpathlineto{\pgfqpoint{4.730268in}{2.719115in}}%
\pgfpathclose%
\pgfusepath{fill}%
\end{pgfscope}%
\begin{pgfscope}%
\pgfpathrectangle{\pgfqpoint{1.254980in}{0.150000in}}{\pgfqpoint{5.490039in}{5.490039in}}%
\pgfusepath{clip}%
\pgfsetbuttcap%
\pgfsetroundjoin%
\definecolor{currentfill}{rgb}{0.276022,0.044167,0.370164}%
\pgfsetfillcolor{currentfill}%
\pgfsetfillopacity{0.700000}%
\pgfsetlinewidth{0.000000pt}%
\definecolor{currentstroke}{rgb}{0.000000,0.000000,0.000000}%
\pgfsetstrokecolor{currentstroke}%
\pgfsetdash{}{0pt}%
\pgfpathmoveto{\pgfqpoint{3.230647in}{2.693580in}}%
\pgfpathlineto{\pgfqpoint{3.243207in}{2.687940in}}%
\pgfpathlineto{\pgfqpoint{3.255771in}{2.682346in}}%
\pgfpathlineto{\pgfqpoint{3.268338in}{2.676797in}}%
\pgfpathlineto{\pgfqpoint{3.280909in}{2.671293in}}%
\pgfpathlineto{\pgfqpoint{3.273280in}{2.664177in}}%
\pgfpathlineto{\pgfqpoint{3.265646in}{2.657102in}}%
\pgfpathlineto{\pgfqpoint{3.258005in}{2.650067in}}%
\pgfpathlineto{\pgfqpoint{3.250357in}{2.643073in}}%
\pgfpathlineto{\pgfqpoint{3.237774in}{2.648594in}}%
\pgfpathlineto{\pgfqpoint{3.225193in}{2.654159in}}%
\pgfpathlineto{\pgfqpoint{3.212617in}{2.659770in}}%
\pgfpathlineto{\pgfqpoint{3.200044in}{2.665427in}}%
\pgfpathlineto{\pgfqpoint{3.207705in}{2.672399in}}%
\pgfpathlineto{\pgfqpoint{3.215358in}{2.679416in}}%
\pgfpathlineto{\pgfqpoint{3.223006in}{2.686476in}}%
\pgfpathlineto{\pgfqpoint{3.230647in}{2.693580in}}%
\pgfpathclose%
\pgfusepath{fill}%
\end{pgfscope}%
\begin{pgfscope}%
\pgfpathrectangle{\pgfqpoint{1.254980in}{0.150000in}}{\pgfqpoint{5.490039in}{5.490039in}}%
\pgfusepath{clip}%
\pgfsetbuttcap%
\pgfsetroundjoin%
\definecolor{currentfill}{rgb}{0.274952,0.037752,0.364543}%
\pgfsetfillcolor{currentfill}%
\pgfsetfillopacity{0.700000}%
\pgfsetlinewidth{0.000000pt}%
\definecolor{currentstroke}{rgb}{0.000000,0.000000,0.000000}%
\pgfsetstrokecolor{currentstroke}%
\pgfsetdash{}{0pt}%
\pgfpathmoveto{\pgfqpoint{3.361636in}{2.678612in}}%
\pgfpathlineto{\pgfqpoint{3.374214in}{2.673330in}}%
\pgfpathlineto{\pgfqpoint{3.386796in}{2.668090in}}%
\pgfpathlineto{\pgfqpoint{3.399383in}{2.662893in}}%
\pgfpathlineto{\pgfqpoint{3.411973in}{2.657738in}}%
\pgfpathlineto{\pgfqpoint{3.404393in}{2.650457in}}%
\pgfpathlineto{\pgfqpoint{3.396808in}{2.643212in}}%
\pgfpathlineto{\pgfqpoint{3.389216in}{2.636002in}}%
\pgfpathlineto{\pgfqpoint{3.381618in}{2.628828in}}%
\pgfpathlineto{\pgfqpoint{3.369016in}{2.633987in}}%
\pgfpathlineto{\pgfqpoint{3.356417in}{2.639189in}}%
\pgfpathlineto{\pgfqpoint{3.343822in}{2.644432in}}%
\pgfpathlineto{\pgfqpoint{3.331232in}{2.649717in}}%
\pgfpathlineto{\pgfqpoint{3.338842in}{2.656883in}}%
\pgfpathlineto{\pgfqpoint{3.346446in}{2.664087in}}%
\pgfpathlineto{\pgfqpoint{3.354044in}{2.671330in}}%
\pgfpathlineto{\pgfqpoint{3.361636in}{2.678612in}}%
\pgfpathclose%
\pgfusepath{fill}%
\end{pgfscope}%
\begin{pgfscope}%
\pgfpathrectangle{\pgfqpoint{1.254980in}{0.150000in}}{\pgfqpoint{5.490039in}{5.490039in}}%
\pgfusepath{clip}%
\pgfsetbuttcap%
\pgfsetroundjoin%
\definecolor{currentfill}{rgb}{0.277018,0.050344,0.375715}%
\pgfsetfillcolor{currentfill}%
\pgfsetfillopacity{0.700000}%
\pgfsetlinewidth{0.000000pt}%
\definecolor{currentstroke}{rgb}{0.000000,0.000000,0.000000}%
\pgfsetstrokecolor{currentstroke}%
\pgfsetdash{}{0pt}%
\pgfpathmoveto{\pgfqpoint{4.519157in}{2.696640in}}%
\pgfpathlineto{\pgfqpoint{4.531963in}{2.692984in}}%
\pgfpathlineto{\pgfqpoint{4.544774in}{2.689355in}}%
\pgfpathlineto{\pgfqpoint{4.557591in}{2.685752in}}%
\pgfpathlineto{\pgfqpoint{4.570414in}{2.682177in}}%
\pgfpathlineto{\pgfqpoint{4.563238in}{2.674218in}}%
\pgfpathlineto{\pgfqpoint{4.556059in}{2.666331in}}%
\pgfpathlineto{\pgfqpoint{4.548875in}{2.658514in}}%
\pgfpathlineto{\pgfqpoint{4.541687in}{2.650762in}}%
\pgfpathlineto{\pgfqpoint{4.528852in}{2.654216in}}%
\pgfpathlineto{\pgfqpoint{4.516022in}{2.657696in}}%
\pgfpathlineto{\pgfqpoint{4.503198in}{2.661204in}}%
\pgfpathlineto{\pgfqpoint{4.490380in}{2.664739in}}%
\pgfpathlineto{\pgfqpoint{4.497581in}{2.672608in}}%
\pgfpathlineto{\pgfqpoint{4.504777in}{2.680545in}}%
\pgfpathlineto{\pgfqpoint{4.511969in}{2.688555in}}%
\pgfpathlineto{\pgfqpoint{4.519157in}{2.696640in}}%
\pgfpathclose%
\pgfusepath{fill}%
\end{pgfscope}%
\begin{pgfscope}%
\pgfpathrectangle{\pgfqpoint{1.254980in}{0.150000in}}{\pgfqpoint{5.490039in}{5.490039in}}%
\pgfusepath{clip}%
\pgfsetbuttcap%
\pgfsetroundjoin%
\definecolor{currentfill}{rgb}{0.277941,0.056324,0.381191}%
\pgfsetfillcolor{currentfill}%
\pgfsetfillopacity{0.700000}%
\pgfsetlinewidth{0.000000pt}%
\definecolor{currentstroke}{rgb}{0.000000,0.000000,0.000000}%
\pgfsetstrokecolor{currentstroke}%
\pgfsetdash{}{0pt}%
\pgfpathmoveto{\pgfqpoint{3.099589in}{2.712384in}}%
\pgfpathlineto{\pgfqpoint{3.112134in}{2.706344in}}%
\pgfpathlineto{\pgfqpoint{3.124683in}{2.700354in}}%
\pgfpathlineto{\pgfqpoint{3.137234in}{2.694413in}}%
\pgfpathlineto{\pgfqpoint{3.149789in}{2.688520in}}%
\pgfpathlineto{\pgfqpoint{3.142110in}{2.681616in}}%
\pgfpathlineto{\pgfqpoint{3.134424in}{2.674759in}}%
\pgfpathlineto{\pgfqpoint{3.126731in}{2.667950in}}%
\pgfpathlineto{\pgfqpoint{3.119032in}{2.661188in}}%
\pgfpathlineto{\pgfqpoint{3.106463in}{2.667110in}}%
\pgfpathlineto{\pgfqpoint{3.093898in}{2.673080in}}%
\pgfpathlineto{\pgfqpoint{3.081336in}{2.679100in}}%
\pgfpathlineto{\pgfqpoint{3.068778in}{2.685169in}}%
\pgfpathlineto{\pgfqpoint{3.076491in}{2.691897in}}%
\pgfpathlineto{\pgfqpoint{3.084197in}{2.698676in}}%
\pgfpathlineto{\pgfqpoint{3.091896in}{2.705505in}}%
\pgfpathlineto{\pgfqpoint{3.099589in}{2.712384in}}%
\pgfpathclose%
\pgfusepath{fill}%
\end{pgfscope}%
\begin{pgfscope}%
\pgfpathrectangle{\pgfqpoint{1.254980in}{0.150000in}}{\pgfqpoint{5.490039in}{5.490039in}}%
\pgfusepath{clip}%
\pgfsetbuttcap%
\pgfsetroundjoin%
\definecolor{currentfill}{rgb}{0.273809,0.031497,0.358853}%
\pgfsetfillcolor{currentfill}%
\pgfsetfillopacity{0.700000}%
\pgfsetlinewidth{0.000000pt}%
\definecolor{currentstroke}{rgb}{0.000000,0.000000,0.000000}%
\pgfsetstrokecolor{currentstroke}%
\pgfsetdash{}{0pt}%
\pgfpathmoveto{\pgfqpoint{3.492590in}{2.666986in}}%
\pgfpathlineto{\pgfqpoint{3.505190in}{2.662025in}}%
\pgfpathlineto{\pgfqpoint{3.517794in}{2.657103in}}%
\pgfpathlineto{\pgfqpoint{3.530403in}{2.652221in}}%
\pgfpathlineto{\pgfqpoint{3.543016in}{2.647377in}}%
\pgfpathlineto{\pgfqpoint{3.535483in}{2.639973in}}%
\pgfpathlineto{\pgfqpoint{3.527944in}{2.632601in}}%
\pgfpathlineto{\pgfqpoint{3.520400in}{2.625261in}}%
\pgfpathlineto{\pgfqpoint{3.512850in}{2.617952in}}%
\pgfpathlineto{\pgfqpoint{3.500225in}{2.622786in}}%
\pgfpathlineto{\pgfqpoint{3.487605in}{2.627660in}}%
\pgfpathlineto{\pgfqpoint{3.474989in}{2.632573in}}%
\pgfpathlineto{\pgfqpoint{3.462377in}{2.637525in}}%
\pgfpathlineto{\pgfqpoint{3.469939in}{2.644839in}}%
\pgfpathlineto{\pgfqpoint{3.477495in}{2.652186in}}%
\pgfpathlineto{\pgfqpoint{3.485046in}{2.659568in}}%
\pgfpathlineto{\pgfqpoint{3.492590in}{2.666986in}}%
\pgfpathclose%
\pgfusepath{fill}%
\end{pgfscope}%
\begin{pgfscope}%
\pgfpathrectangle{\pgfqpoint{1.254980in}{0.150000in}}{\pgfqpoint{5.490039in}{5.490039in}}%
\pgfusepath{clip}%
\pgfsetbuttcap%
\pgfsetroundjoin%
\definecolor{currentfill}{rgb}{0.272594,0.025563,0.353093}%
\pgfsetfillcolor{currentfill}%
\pgfsetfillopacity{0.700000}%
\pgfsetlinewidth{0.000000pt}%
\definecolor{currentstroke}{rgb}{0.000000,0.000000,0.000000}%
\pgfsetstrokecolor{currentstroke}%
\pgfsetdash{}{0pt}%
\pgfpathmoveto{\pgfqpoint{3.965807in}{2.662094in}}%
\pgfpathlineto{\pgfqpoint{3.978495in}{2.657972in}}%
\pgfpathlineto{\pgfqpoint{3.991188in}{2.653883in}}%
\pgfpathlineto{\pgfqpoint{4.003886in}{2.649826in}}%
\pgfpathlineto{\pgfqpoint{4.016590in}{2.645801in}}%
\pgfpathlineto{\pgfqpoint{4.009224in}{2.638140in}}%
\pgfpathlineto{\pgfqpoint{4.001854in}{2.630516in}}%
\pgfpathlineto{\pgfqpoint{3.994478in}{2.622924in}}%
\pgfpathlineto{\pgfqpoint{3.987097in}{2.615364in}}%
\pgfpathlineto{\pgfqpoint{3.974381in}{2.619330in}}%
\pgfpathlineto{\pgfqpoint{3.961671in}{2.623328in}}%
\pgfpathlineto{\pgfqpoint{3.948967in}{2.627358in}}%
\pgfpathlineto{\pgfqpoint{3.936267in}{2.631421in}}%
\pgfpathlineto{\pgfqpoint{3.943660in}{2.639035in}}%
\pgfpathlineto{\pgfqpoint{3.951047in}{2.646684in}}%
\pgfpathlineto{\pgfqpoint{3.958430in}{2.654370in}}%
\pgfpathlineto{\pgfqpoint{3.965807in}{2.662094in}}%
\pgfpathclose%
\pgfusepath{fill}%
\end{pgfscope}%
\begin{pgfscope}%
\pgfpathrectangle{\pgfqpoint{1.254980in}{0.150000in}}{\pgfqpoint{5.490039in}{5.490039in}}%
\pgfusepath{clip}%
\pgfsetbuttcap%
\pgfsetroundjoin%
\definecolor{currentfill}{rgb}{0.274952,0.037752,0.364543}%
\pgfsetfillcolor{currentfill}%
\pgfsetfillopacity{0.700000}%
\pgfsetlinewidth{0.000000pt}%
\definecolor{currentstroke}{rgb}{0.000000,0.000000,0.000000}%
\pgfsetstrokecolor{currentstroke}%
\pgfsetdash{}{0pt}%
\pgfpathmoveto{\pgfqpoint{4.308021in}{2.677159in}}%
\pgfpathlineto{\pgfqpoint{4.320783in}{2.673396in}}%
\pgfpathlineto{\pgfqpoint{4.333550in}{2.669661in}}%
\pgfpathlineto{\pgfqpoint{4.346322in}{2.665955in}}%
\pgfpathlineto{\pgfqpoint{4.359101in}{2.662278in}}%
\pgfpathlineto{\pgfqpoint{4.351853in}{2.654494in}}%
\pgfpathlineto{\pgfqpoint{4.344601in}{2.646763in}}%
\pgfpathlineto{\pgfqpoint{4.337343in}{2.639082in}}%
\pgfpathlineto{\pgfqpoint{4.330081in}{2.631448in}}%
\pgfpathlineto{\pgfqpoint{4.317291in}{2.635029in}}%
\pgfpathlineto{\pgfqpoint{4.304507in}{2.638638in}}%
\pgfpathlineto{\pgfqpoint{4.291728in}{2.642275in}}%
\pgfpathlineto{\pgfqpoint{4.278954in}{2.645942in}}%
\pgfpathlineto{\pgfqpoint{4.286228in}{2.653668in}}%
\pgfpathlineto{\pgfqpoint{4.293498in}{2.661444in}}%
\pgfpathlineto{\pgfqpoint{4.300762in}{2.669273in}}%
\pgfpathlineto{\pgfqpoint{4.308021in}{2.677159in}}%
\pgfpathclose%
\pgfusepath{fill}%
\end{pgfscope}%
\begin{pgfscope}%
\pgfpathrectangle{\pgfqpoint{1.254980in}{0.150000in}}{\pgfqpoint{5.490039in}{5.490039in}}%
\pgfusepath{clip}%
\pgfsetbuttcap%
\pgfsetroundjoin%
\definecolor{currentfill}{rgb}{0.272594,0.025563,0.353093}%
\pgfsetfillcolor{currentfill}%
\pgfsetfillopacity{0.700000}%
\pgfsetlinewidth{0.000000pt}%
\definecolor{currentstroke}{rgb}{0.000000,0.000000,0.000000}%
\pgfsetstrokecolor{currentstroke}%
\pgfsetdash{}{0pt}%
\pgfpathmoveto{\pgfqpoint{3.623540in}{2.658262in}}%
\pgfpathlineto{\pgfqpoint{3.636164in}{2.653588in}}%
\pgfpathlineto{\pgfqpoint{3.648793in}{2.648950in}}%
\pgfpathlineto{\pgfqpoint{3.661426in}{2.644349in}}%
\pgfpathlineto{\pgfqpoint{3.674064in}{2.639784in}}%
\pgfpathlineto{\pgfqpoint{3.666577in}{2.632292in}}%
\pgfpathlineto{\pgfqpoint{3.659084in}{2.624831in}}%
\pgfpathlineto{\pgfqpoint{3.651586in}{2.617398in}}%
\pgfpathlineto{\pgfqpoint{3.644082in}{2.609994in}}%
\pgfpathlineto{\pgfqpoint{3.631433in}{2.614537in}}%
\pgfpathlineto{\pgfqpoint{3.618788in}{2.619117in}}%
\pgfpathlineto{\pgfqpoint{3.606148in}{2.623733in}}%
\pgfpathlineto{\pgfqpoint{3.593512in}{2.628386in}}%
\pgfpathlineto{\pgfqpoint{3.601028in}{2.635808in}}%
\pgfpathlineto{\pgfqpoint{3.608537in}{2.643260in}}%
\pgfpathlineto{\pgfqpoint{3.616042in}{2.650744in}}%
\pgfpathlineto{\pgfqpoint{3.623540in}{2.658262in}}%
\pgfpathclose%
\pgfusepath{fill}%
\end{pgfscope}%
\begin{pgfscope}%
\pgfpathrectangle{\pgfqpoint{1.254980in}{0.150000in}}{\pgfqpoint{5.490039in}{5.490039in}}%
\pgfusepath{clip}%
\pgfsetbuttcap%
\pgfsetroundjoin%
\definecolor{currentfill}{rgb}{0.279566,0.067836,0.391917}%
\pgfsetfillcolor{currentfill}%
\pgfsetfillopacity{0.700000}%
\pgfsetlinewidth{0.000000pt}%
\definecolor{currentstroke}{rgb}{0.000000,0.000000,0.000000}%
\pgfsetstrokecolor{currentstroke}%
\pgfsetdash{}{0pt}%
\pgfpathmoveto{\pgfqpoint{4.861577in}{2.724340in}}%
\pgfpathlineto{\pgfqpoint{4.874462in}{2.720771in}}%
\pgfpathlineto{\pgfqpoint{4.887352in}{2.717227in}}%
\pgfpathlineto{\pgfqpoint{4.900249in}{2.713708in}}%
\pgfpathlineto{\pgfqpoint{4.913152in}{2.710214in}}%
\pgfpathlineto{\pgfqpoint{4.906084in}{2.701851in}}%
\pgfpathlineto{\pgfqpoint{4.899015in}{2.693600in}}%
\pgfpathlineto{\pgfqpoint{4.891942in}{2.685457in}}%
\pgfpathlineto{\pgfqpoint{4.884866in}{2.677416in}}%
\pgfpathlineto{\pgfqpoint{4.871949in}{2.680751in}}%
\pgfpathlineto{\pgfqpoint{4.859039in}{2.684111in}}%
\pgfpathlineto{\pgfqpoint{4.846134in}{2.687496in}}%
\pgfpathlineto{\pgfqpoint{4.833236in}{2.690907in}}%
\pgfpathlineto{\pgfqpoint{4.840326in}{2.699102in}}%
\pgfpathlineto{\pgfqpoint{4.847413in}{2.707403in}}%
\pgfpathlineto{\pgfqpoint{4.854496in}{2.715814in}}%
\pgfpathlineto{\pgfqpoint{4.861577in}{2.724340in}}%
\pgfpathclose%
\pgfusepath{fill}%
\end{pgfscope}%
\begin{pgfscope}%
\pgfpathrectangle{\pgfqpoint{1.254980in}{0.150000in}}{\pgfqpoint{5.490039in}{5.490039in}}%
\pgfusepath{clip}%
\pgfsetbuttcap%
\pgfsetroundjoin%
\definecolor{currentfill}{rgb}{0.273809,0.031497,0.358853}%
\pgfsetfillcolor{currentfill}%
\pgfsetfillopacity{0.700000}%
\pgfsetlinewidth{0.000000pt}%
\definecolor{currentstroke}{rgb}{0.000000,0.000000,0.000000}%
\pgfsetstrokecolor{currentstroke}%
\pgfsetdash{}{0pt}%
\pgfpathmoveto{\pgfqpoint{4.096823in}{2.660771in}}%
\pgfpathlineto{\pgfqpoint{4.109542in}{2.656830in}}%
\pgfpathlineto{\pgfqpoint{4.122266in}{2.652921in}}%
\pgfpathlineto{\pgfqpoint{4.134996in}{2.649041in}}%
\pgfpathlineto{\pgfqpoint{4.147731in}{2.645193in}}%
\pgfpathlineto{\pgfqpoint{4.140409in}{2.637519in}}%
\pgfpathlineto{\pgfqpoint{4.133081in}{2.629884in}}%
\pgfpathlineto{\pgfqpoint{4.125749in}{2.622286in}}%
\pgfpathlineto{\pgfqpoint{4.118412in}{2.614722in}}%
\pgfpathlineto{\pgfqpoint{4.105665in}{2.618499in}}%
\pgfpathlineto{\pgfqpoint{4.092924in}{2.622307in}}%
\pgfpathlineto{\pgfqpoint{4.080188in}{2.626145in}}%
\pgfpathlineto{\pgfqpoint{4.067458in}{2.630014in}}%
\pgfpathlineto{\pgfqpoint{4.074807in}{2.637645in}}%
\pgfpathlineto{\pgfqpoint{4.082151in}{2.645313in}}%
\pgfpathlineto{\pgfqpoint{4.089489in}{2.653021in}}%
\pgfpathlineto{\pgfqpoint{4.096823in}{2.660771in}}%
\pgfpathclose%
\pgfusepath{fill}%
\end{pgfscope}%
\begin{pgfscope}%
\pgfpathrectangle{\pgfqpoint{1.254980in}{0.150000in}}{\pgfqpoint{5.490039in}{5.490039in}}%
\pgfusepath{clip}%
\pgfsetbuttcap%
\pgfsetroundjoin%
\definecolor{currentfill}{rgb}{0.277941,0.056324,0.381191}%
\pgfsetfillcolor{currentfill}%
\pgfsetfillopacity{0.700000}%
\pgfsetlinewidth{0.000000pt}%
\definecolor{currentstroke}{rgb}{0.000000,0.000000,0.000000}%
\pgfsetstrokecolor{currentstroke}%
\pgfsetdash{}{0pt}%
\pgfpathmoveto{\pgfqpoint{4.650375in}{2.700263in}}%
\pgfpathlineto{\pgfqpoint{4.663214in}{2.696688in}}%
\pgfpathlineto{\pgfqpoint{4.676059in}{2.693139in}}%
\pgfpathlineto{\pgfqpoint{4.688911in}{2.689617in}}%
\pgfpathlineto{\pgfqpoint{4.701768in}{2.686121in}}%
\pgfpathlineto{\pgfqpoint{4.694634in}{2.678104in}}%
\pgfpathlineto{\pgfqpoint{4.687497in}{2.670171in}}%
\pgfpathlineto{\pgfqpoint{4.680355in}{2.662318in}}%
\pgfpathlineto{\pgfqpoint{4.673209in}{2.654540in}}%
\pgfpathlineto{\pgfqpoint{4.660339in}{2.657902in}}%
\pgfpathlineto{\pgfqpoint{4.647475in}{2.661290in}}%
\pgfpathlineto{\pgfqpoint{4.634616in}{2.664705in}}%
\pgfpathlineto{\pgfqpoint{4.621764in}{2.668146in}}%
\pgfpathlineto{\pgfqpoint{4.628923in}{2.676054in}}%
\pgfpathlineto{\pgfqpoint{4.636077in}{2.684039in}}%
\pgfpathlineto{\pgfqpoint{4.643228in}{2.692108in}}%
\pgfpathlineto{\pgfqpoint{4.650375in}{2.700263in}}%
\pgfpathclose%
\pgfusepath{fill}%
\end{pgfscope}%
\begin{pgfscope}%
\pgfpathrectangle{\pgfqpoint{1.254980in}{0.150000in}}{\pgfqpoint{5.490039in}{5.490039in}}%
\pgfusepath{clip}%
\pgfsetbuttcap%
\pgfsetroundjoin%
\definecolor{currentfill}{rgb}{0.272594,0.025563,0.353093}%
\pgfsetfillcolor{currentfill}%
\pgfsetfillopacity{0.700000}%
\pgfsetlinewidth{0.000000pt}%
\definecolor{currentstroke}{rgb}{0.000000,0.000000,0.000000}%
\pgfsetstrokecolor{currentstroke}%
\pgfsetdash{}{0pt}%
\pgfpathmoveto{\pgfqpoint{3.754510in}{2.652049in}}%
\pgfpathlineto{\pgfqpoint{3.767161in}{2.647629in}}%
\pgfpathlineto{\pgfqpoint{3.779816in}{2.643244in}}%
\pgfpathlineto{\pgfqpoint{3.792476in}{2.638894in}}%
\pgfpathlineto{\pgfqpoint{3.805141in}{2.634578in}}%
\pgfpathlineto{\pgfqpoint{3.797699in}{2.627029in}}%
\pgfpathlineto{\pgfqpoint{3.790252in}{2.619509in}}%
\pgfpathlineto{\pgfqpoint{3.782799in}{2.612017in}}%
\pgfpathlineto{\pgfqpoint{3.775340in}{2.604552in}}%
\pgfpathlineto{\pgfqpoint{3.762663in}{2.608834in}}%
\pgfpathlineto{\pgfqpoint{3.749992in}{2.613150in}}%
\pgfpathlineto{\pgfqpoint{3.737325in}{2.617501in}}%
\pgfpathlineto{\pgfqpoint{3.724663in}{2.621887in}}%
\pgfpathlineto{\pgfqpoint{3.732133in}{2.629381in}}%
\pgfpathlineto{\pgfqpoint{3.739598in}{2.636905in}}%
\pgfpathlineto{\pgfqpoint{3.747057in}{2.644461in}}%
\pgfpathlineto{\pgfqpoint{3.754510in}{2.652049in}}%
\pgfpathclose%
\pgfusepath{fill}%
\end{pgfscope}%
\begin{pgfscope}%
\pgfpathrectangle{\pgfqpoint{1.254980in}{0.150000in}}{\pgfqpoint{5.490039in}{5.490039in}}%
\pgfusepath{clip}%
\pgfsetbuttcap%
\pgfsetroundjoin%
\definecolor{currentfill}{rgb}{0.276022,0.044167,0.370164}%
\pgfsetfillcolor{currentfill}%
\pgfsetfillopacity{0.700000}%
\pgfsetlinewidth{0.000000pt}%
\definecolor{currentstroke}{rgb}{0.000000,0.000000,0.000000}%
\pgfsetstrokecolor{currentstroke}%
\pgfsetdash{}{0pt}%
\pgfpathmoveto{\pgfqpoint{4.439167in}{2.679154in}}%
\pgfpathlineto{\pgfqpoint{4.451962in}{2.675509in}}%
\pgfpathlineto{\pgfqpoint{4.464762in}{2.671891in}}%
\pgfpathlineto{\pgfqpoint{4.477568in}{2.668301in}}%
\pgfpathlineto{\pgfqpoint{4.490380in}{2.664739in}}%
\pgfpathlineto{\pgfqpoint{4.483175in}{2.656935in}}%
\pgfpathlineto{\pgfqpoint{4.475966in}{2.649191in}}%
\pgfpathlineto{\pgfqpoint{4.468752in}{2.641506in}}%
\pgfpathlineto{\pgfqpoint{4.461533in}{2.633874in}}%
\pgfpathlineto{\pgfqpoint{4.448709in}{2.637327in}}%
\pgfpathlineto{\pgfqpoint{4.435890in}{2.640808in}}%
\pgfpathlineto{\pgfqpoint{4.423078in}{2.644316in}}%
\pgfpathlineto{\pgfqpoint{4.410271in}{2.647853in}}%
\pgfpathlineto{\pgfqpoint{4.417502in}{2.655589in}}%
\pgfpathlineto{\pgfqpoint{4.424728in}{2.663382in}}%
\pgfpathlineto{\pgfqpoint{4.431950in}{2.671236in}}%
\pgfpathlineto{\pgfqpoint{4.439167in}{2.679154in}}%
\pgfpathclose%
\pgfusepath{fill}%
\end{pgfscope}%
\begin{pgfscope}%
\pgfpathrectangle{\pgfqpoint{1.254980in}{0.150000in}}{\pgfqpoint{5.490039in}{5.490039in}}%
\pgfusepath{clip}%
\pgfsetbuttcap%
\pgfsetroundjoin%
\definecolor{currentfill}{rgb}{0.272594,0.025563,0.353093}%
\pgfsetfillcolor{currentfill}%
\pgfsetfillopacity{0.700000}%
\pgfsetlinewidth{0.000000pt}%
\definecolor{currentstroke}{rgb}{0.000000,0.000000,0.000000}%
\pgfsetstrokecolor{currentstroke}%
\pgfsetdash{}{0pt}%
\pgfpathmoveto{\pgfqpoint{3.885521in}{2.647997in}}%
\pgfpathlineto{\pgfqpoint{3.898200in}{2.643803in}}%
\pgfpathlineto{\pgfqpoint{3.910884in}{2.639643in}}%
\pgfpathlineto{\pgfqpoint{3.923573in}{2.635516in}}%
\pgfpathlineto{\pgfqpoint{3.936267in}{2.631421in}}%
\pgfpathlineto{\pgfqpoint{3.928869in}{2.623838in}}%
\pgfpathlineto{\pgfqpoint{3.921466in}{2.616286in}}%
\pgfpathlineto{\pgfqpoint{3.914058in}{2.608763in}}%
\pgfpathlineto{\pgfqpoint{3.906644in}{2.601266in}}%
\pgfpathlineto{\pgfqpoint{3.893938in}{2.605314in}}%
\pgfpathlineto{\pgfqpoint{3.881238in}{2.609395in}}%
\pgfpathlineto{\pgfqpoint{3.868542in}{2.613508in}}%
\pgfpathlineto{\pgfqpoint{3.855852in}{2.617655in}}%
\pgfpathlineto{\pgfqpoint{3.863277in}{2.625194in}}%
\pgfpathlineto{\pgfqpoint{3.870697in}{2.632763in}}%
\pgfpathlineto{\pgfqpoint{3.878112in}{2.640363in}}%
\pgfpathlineto{\pgfqpoint{3.885521in}{2.647997in}}%
\pgfpathclose%
\pgfusepath{fill}%
\end{pgfscope}%
\begin{pgfscope}%
\pgfpathrectangle{\pgfqpoint{1.254980in}{0.150000in}}{\pgfqpoint{5.490039in}{5.490039in}}%
\pgfusepath{clip}%
\pgfsetbuttcap%
\pgfsetroundjoin%
\definecolor{currentfill}{rgb}{0.274952,0.037752,0.364543}%
\pgfsetfillcolor{currentfill}%
\pgfsetfillopacity{0.700000}%
\pgfsetlinewidth{0.000000pt}%
\definecolor{currentstroke}{rgb}{0.000000,0.000000,0.000000}%
\pgfsetstrokecolor{currentstroke}%
\pgfsetdash{}{0pt}%
\pgfpathmoveto{\pgfqpoint{3.280909in}{2.671293in}}%
\pgfpathlineto{\pgfqpoint{3.293484in}{2.665833in}}%
\pgfpathlineto{\pgfqpoint{3.306063in}{2.660418in}}%
\pgfpathlineto{\pgfqpoint{3.318645in}{2.655046in}}%
\pgfpathlineto{\pgfqpoint{3.331232in}{2.649717in}}%
\pgfpathlineto{\pgfqpoint{3.323616in}{2.642590in}}%
\pgfpathlineto{\pgfqpoint{3.315994in}{2.635500in}}%
\pgfpathlineto{\pgfqpoint{3.308365in}{2.628447in}}%
\pgfpathlineto{\pgfqpoint{3.300731in}{2.621432in}}%
\pgfpathlineto{\pgfqpoint{3.288132in}{2.626777in}}%
\pgfpathlineto{\pgfqpoint{3.275536in}{2.632165in}}%
\pgfpathlineto{\pgfqpoint{3.262945in}{2.637597in}}%
\pgfpathlineto{\pgfqpoint{3.250357in}{2.643073in}}%
\pgfpathlineto{\pgfqpoint{3.258005in}{2.650067in}}%
\pgfpathlineto{\pgfqpoint{3.265646in}{2.657102in}}%
\pgfpathlineto{\pgfqpoint{3.273280in}{2.664177in}}%
\pgfpathlineto{\pgfqpoint{3.280909in}{2.671293in}}%
\pgfpathclose%
\pgfusepath{fill}%
\end{pgfscope}%
\begin{pgfscope}%
\pgfpathrectangle{\pgfqpoint{1.254980in}{0.150000in}}{\pgfqpoint{5.490039in}{5.490039in}}%
\pgfusepath{clip}%
\pgfsetbuttcap%
\pgfsetroundjoin%
\definecolor{currentfill}{rgb}{0.273809,0.031497,0.358853}%
\pgfsetfillcolor{currentfill}%
\pgfsetfillopacity{0.700000}%
\pgfsetlinewidth{0.000000pt}%
\definecolor{currentstroke}{rgb}{0.000000,0.000000,0.000000}%
\pgfsetstrokecolor{currentstroke}%
\pgfsetdash{}{0pt}%
\pgfpathmoveto{\pgfqpoint{4.227918in}{2.660900in}}%
\pgfpathlineto{\pgfqpoint{4.240668in}{2.657116in}}%
\pgfpathlineto{\pgfqpoint{4.253425in}{2.653362in}}%
\pgfpathlineto{\pgfqpoint{4.266187in}{2.649638in}}%
\pgfpathlineto{\pgfqpoint{4.278954in}{2.645942in}}%
\pgfpathlineto{\pgfqpoint{4.271676in}{2.638263in}}%
\pgfpathlineto{\pgfqpoint{4.264392in}{2.630627in}}%
\pgfpathlineto{\pgfqpoint{4.257103in}{2.623033in}}%
\pgfpathlineto{\pgfqpoint{4.249810in}{2.615477in}}%
\pgfpathlineto{\pgfqpoint{4.237030in}{2.619088in}}%
\pgfpathlineto{\pgfqpoint{4.224257in}{2.622729in}}%
\pgfpathlineto{\pgfqpoint{4.211488in}{2.626399in}}%
\pgfpathlineto{\pgfqpoint{4.198726in}{2.630098in}}%
\pgfpathlineto{\pgfqpoint{4.206031in}{2.637734in}}%
\pgfpathlineto{\pgfqpoint{4.213332in}{2.645411in}}%
\pgfpathlineto{\pgfqpoint{4.220627in}{2.653132in}}%
\pgfpathlineto{\pgfqpoint{4.227918in}{2.660900in}}%
\pgfpathclose%
\pgfusepath{fill}%
\end{pgfscope}%
\begin{pgfscope}%
\pgfpathrectangle{\pgfqpoint{1.254980in}{0.150000in}}{\pgfqpoint{5.490039in}{5.490039in}}%
\pgfusepath{clip}%
\pgfsetbuttcap%
\pgfsetroundjoin%
\definecolor{currentfill}{rgb}{0.277018,0.050344,0.375715}%
\pgfsetfillcolor{currentfill}%
\pgfsetfillopacity{0.700000}%
\pgfsetlinewidth{0.000000pt}%
\definecolor{currentstroke}{rgb}{0.000000,0.000000,0.000000}%
\pgfsetstrokecolor{currentstroke}%
\pgfsetdash{}{0pt}%
\pgfpathmoveto{\pgfqpoint{3.149789in}{2.688520in}}%
\pgfpathlineto{\pgfqpoint{3.162348in}{2.682676in}}%
\pgfpathlineto{\pgfqpoint{3.174910in}{2.676879in}}%
\pgfpathlineto{\pgfqpoint{3.187475in}{2.671130in}}%
\pgfpathlineto{\pgfqpoint{3.200044in}{2.665427in}}%
\pgfpathlineto{\pgfqpoint{3.192378in}{2.658498in}}%
\pgfpathlineto{\pgfqpoint{3.184705in}{2.651614in}}%
\pgfpathlineto{\pgfqpoint{3.177026in}{2.644773in}}%
\pgfpathlineto{\pgfqpoint{3.169340in}{2.637977in}}%
\pgfpathlineto{\pgfqpoint{3.156757in}{2.643710in}}%
\pgfpathlineto{\pgfqpoint{3.144179in}{2.649488in}}%
\pgfpathlineto{\pgfqpoint{3.131603in}{2.655314in}}%
\pgfpathlineto{\pgfqpoint{3.119032in}{2.661188in}}%
\pgfpathlineto{\pgfqpoint{3.126731in}{2.667950in}}%
\pgfpathlineto{\pgfqpoint{3.134424in}{2.674759in}}%
\pgfpathlineto{\pgfqpoint{3.142110in}{2.681616in}}%
\pgfpathlineto{\pgfqpoint{3.149789in}{2.688520in}}%
\pgfpathclose%
\pgfusepath{fill}%
\end{pgfscope}%
\begin{pgfscope}%
\pgfpathrectangle{\pgfqpoint{1.254980in}{0.150000in}}{\pgfqpoint{5.490039in}{5.490039in}}%
\pgfusepath{clip}%
\pgfsetbuttcap%
\pgfsetroundjoin%
\definecolor{currentfill}{rgb}{0.273809,0.031497,0.358853}%
\pgfsetfillcolor{currentfill}%
\pgfsetfillopacity{0.700000}%
\pgfsetlinewidth{0.000000pt}%
\definecolor{currentstroke}{rgb}{0.000000,0.000000,0.000000}%
\pgfsetstrokecolor{currentstroke}%
\pgfsetdash{}{0pt}%
\pgfpathmoveto{\pgfqpoint{3.411973in}{2.657738in}}%
\pgfpathlineto{\pgfqpoint{3.424568in}{2.652623in}}%
\pgfpathlineto{\pgfqpoint{3.437167in}{2.647550in}}%
\pgfpathlineto{\pgfqpoint{3.449770in}{2.642517in}}%
\pgfpathlineto{\pgfqpoint{3.462377in}{2.637525in}}%
\pgfpathlineto{\pgfqpoint{3.454810in}{2.630245in}}%
\pgfpathlineto{\pgfqpoint{3.447236in}{2.622998in}}%
\pgfpathlineto{\pgfqpoint{3.439656in}{2.615783in}}%
\pgfpathlineto{\pgfqpoint{3.432071in}{2.608600in}}%
\pgfpathlineto{\pgfqpoint{3.419452in}{2.613596in}}%
\pgfpathlineto{\pgfqpoint{3.406836in}{2.618633in}}%
\pgfpathlineto{\pgfqpoint{3.394225in}{2.623710in}}%
\pgfpathlineto{\pgfqpoint{3.381618in}{2.628828in}}%
\pgfpathlineto{\pgfqpoint{3.389216in}{2.636002in}}%
\pgfpathlineto{\pgfqpoint{3.396808in}{2.643212in}}%
\pgfpathlineto{\pgfqpoint{3.404393in}{2.650457in}}%
\pgfpathlineto{\pgfqpoint{3.411973in}{2.657738in}}%
\pgfpathclose%
\pgfusepath{fill}%
\end{pgfscope}%
\begin{pgfscope}%
\pgfpathrectangle{\pgfqpoint{1.254980in}{0.150000in}}{\pgfqpoint{5.490039in}{5.490039in}}%
\pgfusepath{clip}%
\pgfsetbuttcap%
\pgfsetroundjoin%
\definecolor{currentfill}{rgb}{0.278791,0.062145,0.386592}%
\pgfsetfillcolor{currentfill}%
\pgfsetfillopacity{0.700000}%
\pgfsetlinewidth{0.000000pt}%
\definecolor{currentstroke}{rgb}{0.000000,0.000000,0.000000}%
\pgfsetstrokecolor{currentstroke}%
\pgfsetdash{}{0pt}%
\pgfpathmoveto{\pgfqpoint{4.781704in}{2.704805in}}%
\pgfpathlineto{\pgfqpoint{4.794578in}{2.701292in}}%
\pgfpathlineto{\pgfqpoint{4.807458in}{2.697804in}}%
\pgfpathlineto{\pgfqpoint{4.820344in}{2.694343in}}%
\pgfpathlineto{\pgfqpoint{4.833236in}{2.690907in}}%
\pgfpathlineto{\pgfqpoint{4.826143in}{2.682812in}}%
\pgfpathlineto{\pgfqpoint{4.819047in}{2.674813in}}%
\pgfpathlineto{\pgfqpoint{4.811947in}{2.666906in}}%
\pgfpathlineto{\pgfqpoint{4.804844in}{2.659085in}}%
\pgfpathlineto{\pgfqpoint{4.791938in}{2.662374in}}%
\pgfpathlineto{\pgfqpoint{4.779038in}{2.665689in}}%
\pgfpathlineto{\pgfqpoint{4.766145in}{2.669030in}}%
\pgfpathlineto{\pgfqpoint{4.753257in}{2.672396in}}%
\pgfpathlineto{\pgfqpoint{4.760374in}{2.680359in}}%
\pgfpathlineto{\pgfqpoint{4.767488in}{2.688412in}}%
\pgfpathlineto{\pgfqpoint{4.774597in}{2.696559in}}%
\pgfpathlineto{\pgfqpoint{4.781704in}{2.704805in}}%
\pgfpathclose%
\pgfusepath{fill}%
\end{pgfscope}%
\begin{pgfscope}%
\pgfpathrectangle{\pgfqpoint{1.254980in}{0.150000in}}{\pgfqpoint{5.490039in}{5.490039in}}%
\pgfusepath{clip}%
\pgfsetbuttcap%
\pgfsetroundjoin%
\definecolor{currentfill}{rgb}{0.272594,0.025563,0.353093}%
\pgfsetfillcolor{currentfill}%
\pgfsetfillopacity{0.700000}%
\pgfsetlinewidth{0.000000pt}%
\definecolor{currentstroke}{rgb}{0.000000,0.000000,0.000000}%
\pgfsetstrokecolor{currentstroke}%
\pgfsetdash{}{0pt}%
\pgfpathmoveto{\pgfqpoint{3.543016in}{2.647377in}}%
\pgfpathlineto{\pgfqpoint{3.555633in}{2.642572in}}%
\pgfpathlineto{\pgfqpoint{3.568255in}{2.637806in}}%
\pgfpathlineto{\pgfqpoint{3.580881in}{2.633077in}}%
\pgfpathlineto{\pgfqpoint{3.593512in}{2.628386in}}%
\pgfpathlineto{\pgfqpoint{3.585991in}{2.620996in}}%
\pgfpathlineto{\pgfqpoint{3.578465in}{2.613634in}}%
\pgfpathlineto{\pgfqpoint{3.570932in}{2.606301in}}%
\pgfpathlineto{\pgfqpoint{3.563394in}{2.598996in}}%
\pgfpathlineto{\pgfqpoint{3.550751in}{2.603678in}}%
\pgfpathlineto{\pgfqpoint{3.538113in}{2.608398in}}%
\pgfpathlineto{\pgfqpoint{3.525479in}{2.613156in}}%
\pgfpathlineto{\pgfqpoint{3.512850in}{2.617952in}}%
\pgfpathlineto{\pgfqpoint{3.520400in}{2.625261in}}%
\pgfpathlineto{\pgfqpoint{3.527944in}{2.632601in}}%
\pgfpathlineto{\pgfqpoint{3.535483in}{2.639973in}}%
\pgfpathlineto{\pgfqpoint{3.543016in}{2.647377in}}%
\pgfpathclose%
\pgfusepath{fill}%
\end{pgfscope}%
\begin{pgfscope}%
\pgfpathrectangle{\pgfqpoint{1.254980in}{0.150000in}}{\pgfqpoint{5.490039in}{5.490039in}}%
\pgfusepath{clip}%
\pgfsetbuttcap%
\pgfsetroundjoin%
\definecolor{currentfill}{rgb}{0.277018,0.050344,0.375715}%
\pgfsetfillcolor{currentfill}%
\pgfsetfillopacity{0.700000}%
\pgfsetlinewidth{0.000000pt}%
\definecolor{currentstroke}{rgb}{0.000000,0.000000,0.000000}%
\pgfsetstrokecolor{currentstroke}%
\pgfsetdash{}{0pt}%
\pgfpathmoveto{\pgfqpoint{4.570414in}{2.682177in}}%
\pgfpathlineto{\pgfqpoint{4.583242in}{2.678629in}}%
\pgfpathlineto{\pgfqpoint{4.596077in}{2.675108in}}%
\pgfpathlineto{\pgfqpoint{4.608918in}{2.671614in}}%
\pgfpathlineto{\pgfqpoint{4.621764in}{2.668146in}}%
\pgfpathlineto{\pgfqpoint{4.614601in}{2.660313in}}%
\pgfpathlineto{\pgfqpoint{4.607435in}{2.652549in}}%
\pgfpathlineto{\pgfqpoint{4.600264in}{2.644852in}}%
\pgfpathlineto{\pgfqpoint{4.593088in}{2.637218in}}%
\pgfpathlineto{\pgfqpoint{4.580229in}{2.640564in}}%
\pgfpathlineto{\pgfqpoint{4.567376in}{2.643936in}}%
\pgfpathlineto{\pgfqpoint{4.554528in}{2.647336in}}%
\pgfpathlineto{\pgfqpoint{4.541687in}{2.650762in}}%
\pgfpathlineto{\pgfqpoint{4.548875in}{2.658514in}}%
\pgfpathlineto{\pgfqpoint{4.556059in}{2.666331in}}%
\pgfpathlineto{\pgfqpoint{4.563238in}{2.674218in}}%
\pgfpathlineto{\pgfqpoint{4.570414in}{2.682177in}}%
\pgfpathclose%
\pgfusepath{fill}%
\end{pgfscope}%
\begin{pgfscope}%
\pgfpathrectangle{\pgfqpoint{1.254980in}{0.150000in}}{\pgfqpoint{5.490039in}{5.490039in}}%
\pgfusepath{clip}%
\pgfsetbuttcap%
\pgfsetroundjoin%
\definecolor{currentfill}{rgb}{0.272594,0.025563,0.353093}%
\pgfsetfillcolor{currentfill}%
\pgfsetfillopacity{0.700000}%
\pgfsetlinewidth{0.000000pt}%
\definecolor{currentstroke}{rgb}{0.000000,0.000000,0.000000}%
\pgfsetstrokecolor{currentstroke}%
\pgfsetdash{}{0pt}%
\pgfpathmoveto{\pgfqpoint{4.016590in}{2.645801in}}%
\pgfpathlineto{\pgfqpoint{4.029299in}{2.641807in}}%
\pgfpathlineto{\pgfqpoint{4.042013in}{2.637845in}}%
\pgfpathlineto{\pgfqpoint{4.054733in}{2.633914in}}%
\pgfpathlineto{\pgfqpoint{4.067458in}{2.630014in}}%
\pgfpathlineto{\pgfqpoint{4.060104in}{2.622417in}}%
\pgfpathlineto{\pgfqpoint{4.052745in}{2.614853in}}%
\pgfpathlineto{\pgfqpoint{4.045380in}{2.607320in}}%
\pgfpathlineto{\pgfqpoint{4.038011in}{2.599814in}}%
\pgfpathlineto{\pgfqpoint{4.025274in}{2.603655in}}%
\pgfpathlineto{\pgfqpoint{4.012543in}{2.607526in}}%
\pgfpathlineto{\pgfqpoint{3.999817in}{2.611429in}}%
\pgfpathlineto{\pgfqpoint{3.987097in}{2.615364in}}%
\pgfpathlineto{\pgfqpoint{3.994478in}{2.622924in}}%
\pgfpathlineto{\pgfqpoint{4.001854in}{2.630516in}}%
\pgfpathlineto{\pgfqpoint{4.009224in}{2.638140in}}%
\pgfpathlineto{\pgfqpoint{4.016590in}{2.645801in}}%
\pgfpathclose%
\pgfusepath{fill}%
\end{pgfscope}%
\begin{pgfscope}%
\pgfpathrectangle{\pgfqpoint{1.254980in}{0.150000in}}{\pgfqpoint{5.490039in}{5.490039in}}%
\pgfusepath{clip}%
\pgfsetbuttcap%
\pgfsetroundjoin%
\definecolor{currentfill}{rgb}{0.272594,0.025563,0.353093}%
\pgfsetfillcolor{currentfill}%
\pgfsetfillopacity{0.700000}%
\pgfsetlinewidth{0.000000pt}%
\definecolor{currentstroke}{rgb}{0.000000,0.000000,0.000000}%
\pgfsetstrokecolor{currentstroke}%
\pgfsetdash{}{0pt}%
\pgfpathmoveto{\pgfqpoint{3.674064in}{2.639784in}}%
\pgfpathlineto{\pgfqpoint{3.686707in}{2.635256in}}%
\pgfpathlineto{\pgfqpoint{3.699354in}{2.630764in}}%
\pgfpathlineto{\pgfqpoint{3.712006in}{2.626308in}}%
\pgfpathlineto{\pgfqpoint{3.724663in}{2.621887in}}%
\pgfpathlineto{\pgfqpoint{3.717188in}{2.614421in}}%
\pgfpathlineto{\pgfqpoint{3.709707in}{2.606982in}}%
\pgfpathlineto{\pgfqpoint{3.702220in}{2.599570in}}%
\pgfpathlineto{\pgfqpoint{3.694728in}{2.592182in}}%
\pgfpathlineto{\pgfqpoint{3.682060in}{2.596582in}}%
\pgfpathlineto{\pgfqpoint{3.669396in}{2.601017in}}%
\pgfpathlineto{\pgfqpoint{3.656737in}{2.605487in}}%
\pgfpathlineto{\pgfqpoint{3.644082in}{2.609994in}}%
\pgfpathlineto{\pgfqpoint{3.651586in}{2.617398in}}%
\pgfpathlineto{\pgfqpoint{3.659084in}{2.624831in}}%
\pgfpathlineto{\pgfqpoint{3.666577in}{2.632292in}}%
\pgfpathlineto{\pgfqpoint{3.674064in}{2.639784in}}%
\pgfpathclose%
\pgfusepath{fill}%
\end{pgfscope}%
\begin{pgfscope}%
\pgfpathrectangle{\pgfqpoint{1.254980in}{0.150000in}}{\pgfqpoint{5.490039in}{5.490039in}}%
\pgfusepath{clip}%
\pgfsetbuttcap%
\pgfsetroundjoin%
\definecolor{currentfill}{rgb}{0.274952,0.037752,0.364543}%
\pgfsetfillcolor{currentfill}%
\pgfsetfillopacity{0.700000}%
\pgfsetlinewidth{0.000000pt}%
\definecolor{currentstroke}{rgb}{0.000000,0.000000,0.000000}%
\pgfsetstrokecolor{currentstroke}%
\pgfsetdash{}{0pt}%
\pgfpathmoveto{\pgfqpoint{4.359101in}{2.662278in}}%
\pgfpathlineto{\pgfqpoint{4.371885in}{2.658630in}}%
\pgfpathlineto{\pgfqpoint{4.384674in}{2.655009in}}%
\pgfpathlineto{\pgfqpoint{4.397470in}{2.651417in}}%
\pgfpathlineto{\pgfqpoint{4.410271in}{2.647853in}}%
\pgfpathlineto{\pgfqpoint{4.403035in}{2.640170in}}%
\pgfpathlineto{\pgfqpoint{4.395795in}{2.632537in}}%
\pgfpathlineto{\pgfqpoint{4.388550in}{2.624952in}}%
\pgfpathlineto{\pgfqpoint{4.381300in}{2.617410in}}%
\pgfpathlineto{\pgfqpoint{4.368487in}{2.620877in}}%
\pgfpathlineto{\pgfqpoint{4.355679in}{2.624373in}}%
\pgfpathlineto{\pgfqpoint{4.342877in}{2.627896in}}%
\pgfpathlineto{\pgfqpoint{4.330081in}{2.631448in}}%
\pgfpathlineto{\pgfqpoint{4.337343in}{2.639082in}}%
\pgfpathlineto{\pgfqpoint{4.344601in}{2.646763in}}%
\pgfpathlineto{\pgfqpoint{4.351853in}{2.654494in}}%
\pgfpathlineto{\pgfqpoint{4.359101in}{2.662278in}}%
\pgfpathclose%
\pgfusepath{fill}%
\end{pgfscope}%
\begin{pgfscope}%
\pgfpathrectangle{\pgfqpoint{1.254980in}{0.150000in}}{\pgfqpoint{5.490039in}{5.490039in}}%
\pgfusepath{clip}%
\pgfsetbuttcap%
\pgfsetroundjoin%
\definecolor{currentfill}{rgb}{0.280267,0.073417,0.397163}%
\pgfsetfillcolor{currentfill}%
\pgfsetfillopacity{0.700000}%
\pgfsetlinewidth{0.000000pt}%
\definecolor{currentstroke}{rgb}{0.000000,0.000000,0.000000}%
\pgfsetstrokecolor{currentstroke}%
\pgfsetdash{}{0pt}%
\pgfpathmoveto{\pgfqpoint{4.913152in}{2.710214in}}%
\pgfpathlineto{\pgfqpoint{4.926060in}{2.706745in}}%
\pgfpathlineto{\pgfqpoint{4.938975in}{2.703302in}}%
\pgfpathlineto{\pgfqpoint{4.951896in}{2.699883in}}%
\pgfpathlineto{\pgfqpoint{4.964824in}{2.696490in}}%
\pgfpathlineto{\pgfqpoint{4.957771in}{2.688290in}}%
\pgfpathlineto{\pgfqpoint{4.950715in}{2.680200in}}%
\pgfpathlineto{\pgfqpoint{4.943656in}{2.672215in}}%
\pgfpathlineto{\pgfqpoint{4.936594in}{2.664329in}}%
\pgfpathlineto{\pgfqpoint{4.923653in}{2.667563in}}%
\pgfpathlineto{\pgfqpoint{4.910718in}{2.670823in}}%
\pgfpathlineto{\pgfqpoint{4.897789in}{2.674107in}}%
\pgfpathlineto{\pgfqpoint{4.884866in}{2.677416in}}%
\pgfpathlineto{\pgfqpoint{4.891942in}{2.685457in}}%
\pgfpathlineto{\pgfqpoint{4.899015in}{2.693600in}}%
\pgfpathlineto{\pgfqpoint{4.906084in}{2.701851in}}%
\pgfpathlineto{\pgfqpoint{4.913152in}{2.710214in}}%
\pgfpathclose%
\pgfusepath{fill}%
\end{pgfscope}%
\begin{pgfscope}%
\pgfpathrectangle{\pgfqpoint{1.254980in}{0.150000in}}{\pgfqpoint{5.490039in}{5.490039in}}%
\pgfusepath{clip}%
\pgfsetbuttcap%
\pgfsetroundjoin%
\definecolor{currentfill}{rgb}{0.273809,0.031497,0.358853}%
\pgfsetfillcolor{currentfill}%
\pgfsetfillopacity{0.700000}%
\pgfsetlinewidth{0.000000pt}%
\definecolor{currentstroke}{rgb}{0.000000,0.000000,0.000000}%
\pgfsetstrokecolor{currentstroke}%
\pgfsetdash{}{0pt}%
\pgfpathmoveto{\pgfqpoint{4.147731in}{2.645193in}}%
\pgfpathlineto{\pgfqpoint{4.160471in}{2.641374in}}%
\pgfpathlineto{\pgfqpoint{4.173217in}{2.637585in}}%
\pgfpathlineto{\pgfqpoint{4.185969in}{2.633827in}}%
\pgfpathlineto{\pgfqpoint{4.198726in}{2.630098in}}%
\pgfpathlineto{\pgfqpoint{4.191415in}{2.622501in}}%
\pgfpathlineto{\pgfqpoint{4.184100in}{2.614940in}}%
\pgfpathlineto{\pgfqpoint{4.176779in}{2.607412in}}%
\pgfpathlineto{\pgfqpoint{4.169454in}{2.599915in}}%
\pgfpathlineto{\pgfqpoint{4.156685in}{2.603572in}}%
\pgfpathlineto{\pgfqpoint{4.143922in}{2.607259in}}%
\pgfpathlineto{\pgfqpoint{4.131164in}{2.610975in}}%
\pgfpathlineto{\pgfqpoint{4.118412in}{2.614722in}}%
\pgfpathlineto{\pgfqpoint{4.125749in}{2.622286in}}%
\pgfpathlineto{\pgfqpoint{4.133081in}{2.629884in}}%
\pgfpathlineto{\pgfqpoint{4.140409in}{2.637519in}}%
\pgfpathlineto{\pgfqpoint{4.147731in}{2.645193in}}%
\pgfpathclose%
\pgfusepath{fill}%
\end{pgfscope}%
\begin{pgfscope}%
\pgfpathrectangle{\pgfqpoint{1.254980in}{0.150000in}}{\pgfqpoint{5.490039in}{5.490039in}}%
\pgfusepath{clip}%
\pgfsetbuttcap%
\pgfsetroundjoin%
\definecolor{currentfill}{rgb}{0.272594,0.025563,0.353093}%
\pgfsetfillcolor{currentfill}%
\pgfsetfillopacity{0.700000}%
\pgfsetlinewidth{0.000000pt}%
\definecolor{currentstroke}{rgb}{0.000000,0.000000,0.000000}%
\pgfsetstrokecolor{currentstroke}%
\pgfsetdash{}{0pt}%
\pgfpathmoveto{\pgfqpoint{3.805141in}{2.634578in}}%
\pgfpathlineto{\pgfqpoint{3.817811in}{2.630297in}}%
\pgfpathlineto{\pgfqpoint{3.830487in}{2.626049in}}%
\pgfpathlineto{\pgfqpoint{3.843167in}{2.621836in}}%
\pgfpathlineto{\pgfqpoint{3.855852in}{2.617655in}}%
\pgfpathlineto{\pgfqpoint{3.848421in}{2.610145in}}%
\pgfpathlineto{\pgfqpoint{3.840985in}{2.602661in}}%
\pgfpathlineto{\pgfqpoint{3.833544in}{2.595201in}}%
\pgfpathlineto{\pgfqpoint{3.826097in}{2.587766in}}%
\pgfpathlineto{\pgfqpoint{3.813400in}{2.591912in}}%
\pgfpathlineto{\pgfqpoint{3.800708in}{2.596092in}}%
\pgfpathlineto{\pgfqpoint{3.788022in}{2.600305in}}%
\pgfpathlineto{\pgfqpoint{3.775340in}{2.604552in}}%
\pgfpathlineto{\pgfqpoint{3.782799in}{2.612017in}}%
\pgfpathlineto{\pgfqpoint{3.790252in}{2.619509in}}%
\pgfpathlineto{\pgfqpoint{3.797699in}{2.627029in}}%
\pgfpathlineto{\pgfqpoint{3.805141in}{2.634578in}}%
\pgfpathclose%
\pgfusepath{fill}%
\end{pgfscope}%
\begin{pgfscope}%
\pgfpathrectangle{\pgfqpoint{1.254980in}{0.150000in}}{\pgfqpoint{5.490039in}{5.490039in}}%
\pgfusepath{clip}%
\pgfsetbuttcap%
\pgfsetroundjoin%
\definecolor{currentfill}{rgb}{0.277941,0.056324,0.381191}%
\pgfsetfillcolor{currentfill}%
\pgfsetfillopacity{0.700000}%
\pgfsetlinewidth{0.000000pt}%
\definecolor{currentstroke}{rgb}{0.000000,0.000000,0.000000}%
\pgfsetstrokecolor{currentstroke}%
\pgfsetdash{}{0pt}%
\pgfpathmoveto{\pgfqpoint{4.701768in}{2.686121in}}%
\pgfpathlineto{\pgfqpoint{4.714631in}{2.682650in}}%
\pgfpathlineto{\pgfqpoint{4.727501in}{2.679206in}}%
\pgfpathlineto{\pgfqpoint{4.740376in}{2.675788in}}%
\pgfpathlineto{\pgfqpoint{4.753257in}{2.672396in}}%
\pgfpathlineto{\pgfqpoint{4.746137in}{2.664518in}}%
\pgfpathlineto{\pgfqpoint{4.739012in}{2.656721in}}%
\pgfpathlineto{\pgfqpoint{4.731884in}{2.649001in}}%
\pgfpathlineto{\pgfqpoint{4.724752in}{2.641353in}}%
\pgfpathlineto{\pgfqpoint{4.711857in}{2.644611in}}%
\pgfpathlineto{\pgfqpoint{4.698968in}{2.647894in}}%
\pgfpathlineto{\pgfqpoint{4.686086in}{2.651204in}}%
\pgfpathlineto{\pgfqpoint{4.673209in}{2.654540in}}%
\pgfpathlineto{\pgfqpoint{4.680355in}{2.662318in}}%
\pgfpathlineto{\pgfqpoint{4.687497in}{2.670171in}}%
\pgfpathlineto{\pgfqpoint{4.694634in}{2.678104in}}%
\pgfpathlineto{\pgfqpoint{4.701768in}{2.686121in}}%
\pgfpathclose%
\pgfusepath{fill}%
\end{pgfscope}%
\begin{pgfscope}%
\pgfpathrectangle{\pgfqpoint{1.254980in}{0.150000in}}{\pgfqpoint{5.490039in}{5.490039in}}%
\pgfusepath{clip}%
\pgfsetbuttcap%
\pgfsetroundjoin%
\definecolor{currentfill}{rgb}{0.276022,0.044167,0.370164}%
\pgfsetfillcolor{currentfill}%
\pgfsetfillopacity{0.700000}%
\pgfsetlinewidth{0.000000pt}%
\definecolor{currentstroke}{rgb}{0.000000,0.000000,0.000000}%
\pgfsetstrokecolor{currentstroke}%
\pgfsetdash{}{0pt}%
\pgfpathmoveto{\pgfqpoint{4.490380in}{2.664739in}}%
\pgfpathlineto{\pgfqpoint{4.503198in}{2.661204in}}%
\pgfpathlineto{\pgfqpoint{4.516022in}{2.657696in}}%
\pgfpathlineto{\pgfqpoint{4.528852in}{2.654216in}}%
\pgfpathlineto{\pgfqpoint{4.541687in}{2.650762in}}%
\pgfpathlineto{\pgfqpoint{4.534494in}{2.643072in}}%
\pgfpathlineto{\pgfqpoint{4.527298in}{2.635440in}}%
\pgfpathlineto{\pgfqpoint{4.520096in}{2.627862in}}%
\pgfpathlineto{\pgfqpoint{4.512890in}{2.620336in}}%
\pgfpathlineto{\pgfqpoint{4.500042in}{2.623679in}}%
\pgfpathlineto{\pgfqpoint{4.487200in}{2.627050in}}%
\pgfpathlineto{\pgfqpoint{4.474364in}{2.630449in}}%
\pgfpathlineto{\pgfqpoint{4.461533in}{2.633874in}}%
\pgfpathlineto{\pgfqpoint{4.468752in}{2.641506in}}%
\pgfpathlineto{\pgfqpoint{4.475966in}{2.649191in}}%
\pgfpathlineto{\pgfqpoint{4.483175in}{2.656935in}}%
\pgfpathlineto{\pgfqpoint{4.490380in}{2.664739in}}%
\pgfpathclose%
\pgfusepath{fill}%
\end{pgfscope}%
\begin{pgfscope}%
\pgfpathrectangle{\pgfqpoint{1.254980in}{0.150000in}}{\pgfqpoint{5.490039in}{5.490039in}}%
\pgfusepath{clip}%
\pgfsetbuttcap%
\pgfsetroundjoin%
\definecolor{currentfill}{rgb}{0.276022,0.044167,0.370164}%
\pgfsetfillcolor{currentfill}%
\pgfsetfillopacity{0.700000}%
\pgfsetlinewidth{0.000000pt}%
\definecolor{currentstroke}{rgb}{0.000000,0.000000,0.000000}%
\pgfsetstrokecolor{currentstroke}%
\pgfsetdash{}{0pt}%
\pgfpathmoveto{\pgfqpoint{3.200044in}{2.665427in}}%
\pgfpathlineto{\pgfqpoint{3.212617in}{2.659770in}}%
\pgfpathlineto{\pgfqpoint{3.225193in}{2.654159in}}%
\pgfpathlineto{\pgfqpoint{3.237774in}{2.648594in}}%
\pgfpathlineto{\pgfqpoint{3.250357in}{2.643073in}}%
\pgfpathlineto{\pgfqpoint{3.242704in}{2.636120in}}%
\pgfpathlineto{\pgfqpoint{3.235044in}{2.629208in}}%
\pgfpathlineto{\pgfqpoint{3.227378in}{2.622337in}}%
\pgfpathlineto{\pgfqpoint{3.219705in}{2.615508in}}%
\pgfpathlineto{\pgfqpoint{3.207108in}{2.621057in}}%
\pgfpathlineto{\pgfqpoint{3.194515in}{2.626652in}}%
\pgfpathlineto{\pgfqpoint{3.181926in}{2.632292in}}%
\pgfpathlineto{\pgfqpoint{3.169340in}{2.637977in}}%
\pgfpathlineto{\pgfqpoint{3.177026in}{2.644773in}}%
\pgfpathlineto{\pgfqpoint{3.184705in}{2.651614in}}%
\pgfpathlineto{\pgfqpoint{3.192378in}{2.658498in}}%
\pgfpathlineto{\pgfqpoint{3.200044in}{2.665427in}}%
\pgfpathclose%
\pgfusepath{fill}%
\end{pgfscope}%
\begin{pgfscope}%
\pgfpathrectangle{\pgfqpoint{1.254980in}{0.150000in}}{\pgfqpoint{5.490039in}{5.490039in}}%
\pgfusepath{clip}%
\pgfsetbuttcap%
\pgfsetroundjoin%
\definecolor{currentfill}{rgb}{0.273809,0.031497,0.358853}%
\pgfsetfillcolor{currentfill}%
\pgfsetfillopacity{0.700000}%
\pgfsetlinewidth{0.000000pt}%
\definecolor{currentstroke}{rgb}{0.000000,0.000000,0.000000}%
\pgfsetstrokecolor{currentstroke}%
\pgfsetdash{}{0pt}%
\pgfpathmoveto{\pgfqpoint{3.331232in}{2.649717in}}%
\pgfpathlineto{\pgfqpoint{3.343822in}{2.644432in}}%
\pgfpathlineto{\pgfqpoint{3.356417in}{2.639189in}}%
\pgfpathlineto{\pgfqpoint{3.369016in}{2.633987in}}%
\pgfpathlineto{\pgfqpoint{3.381618in}{2.628828in}}%
\pgfpathlineto{\pgfqpoint{3.374015in}{2.621688in}}%
\pgfpathlineto{\pgfqpoint{3.366405in}{2.614584in}}%
\pgfpathlineto{\pgfqpoint{3.358789in}{2.607513in}}%
\pgfpathlineto{\pgfqpoint{3.351167in}{2.600477in}}%
\pgfpathlineto{\pgfqpoint{3.338552in}{2.605653in}}%
\pgfpathlineto{\pgfqpoint{3.325941in}{2.610870in}}%
\pgfpathlineto{\pgfqpoint{3.313334in}{2.616130in}}%
\pgfpathlineto{\pgfqpoint{3.300731in}{2.621432in}}%
\pgfpathlineto{\pgfqpoint{3.308365in}{2.628447in}}%
\pgfpathlineto{\pgfqpoint{3.315994in}{2.635500in}}%
\pgfpathlineto{\pgfqpoint{3.323616in}{2.642590in}}%
\pgfpathlineto{\pgfqpoint{3.331232in}{2.649717in}}%
\pgfpathclose%
\pgfusepath{fill}%
\end{pgfscope}%
\begin{pgfscope}%
\pgfpathrectangle{\pgfqpoint{1.254980in}{0.150000in}}{\pgfqpoint{5.490039in}{5.490039in}}%
\pgfusepath{clip}%
\pgfsetbuttcap%
\pgfsetroundjoin%
\definecolor{currentfill}{rgb}{0.272594,0.025563,0.353093}%
\pgfsetfillcolor{currentfill}%
\pgfsetfillopacity{0.700000}%
\pgfsetlinewidth{0.000000pt}%
\definecolor{currentstroke}{rgb}{0.000000,0.000000,0.000000}%
\pgfsetstrokecolor{currentstroke}%
\pgfsetdash{}{0pt}%
\pgfpathmoveto{\pgfqpoint{3.936267in}{2.631421in}}%
\pgfpathlineto{\pgfqpoint{3.948967in}{2.627358in}}%
\pgfpathlineto{\pgfqpoint{3.961671in}{2.623328in}}%
\pgfpathlineto{\pgfqpoint{3.974381in}{2.619330in}}%
\pgfpathlineto{\pgfqpoint{3.987097in}{2.615364in}}%
\pgfpathlineto{\pgfqpoint{3.979710in}{2.607833in}}%
\pgfpathlineto{\pgfqpoint{3.972318in}{2.600329in}}%
\pgfpathlineto{\pgfqpoint{3.964921in}{2.592851in}}%
\pgfpathlineto{\pgfqpoint{3.957519in}{2.585396in}}%
\pgfpathlineto{\pgfqpoint{3.944792in}{2.589315in}}%
\pgfpathlineto{\pgfqpoint{3.932071in}{2.593267in}}%
\pgfpathlineto{\pgfqpoint{3.919355in}{2.597250in}}%
\pgfpathlineto{\pgfqpoint{3.906644in}{2.601266in}}%
\pgfpathlineto{\pgfqpoint{3.914058in}{2.608763in}}%
\pgfpathlineto{\pgfqpoint{3.921466in}{2.616286in}}%
\pgfpathlineto{\pgfqpoint{3.928869in}{2.623838in}}%
\pgfpathlineto{\pgfqpoint{3.936267in}{2.631421in}}%
\pgfpathclose%
\pgfusepath{fill}%
\end{pgfscope}%
\begin{pgfscope}%
\pgfpathrectangle{\pgfqpoint{1.254980in}{0.150000in}}{\pgfqpoint{5.490039in}{5.490039in}}%
\pgfusepath{clip}%
\pgfsetbuttcap%
\pgfsetroundjoin%
\definecolor{currentfill}{rgb}{0.277941,0.056324,0.381191}%
\pgfsetfillcolor{currentfill}%
\pgfsetfillopacity{0.700000}%
\pgfsetlinewidth{0.000000pt}%
\definecolor{currentstroke}{rgb}{0.000000,0.000000,0.000000}%
\pgfsetstrokecolor{currentstroke}%
\pgfsetdash{}{0pt}%
\pgfpathmoveto{\pgfqpoint{3.068778in}{2.685169in}}%
\pgfpathlineto{\pgfqpoint{3.081336in}{2.679100in}}%
\pgfpathlineto{\pgfqpoint{3.093898in}{2.673080in}}%
\pgfpathlineto{\pgfqpoint{3.106463in}{2.667110in}}%
\pgfpathlineto{\pgfqpoint{3.119032in}{2.661188in}}%
\pgfpathlineto{\pgfqpoint{3.111326in}{2.654474in}}%
\pgfpathlineto{\pgfqpoint{3.103613in}{2.647809in}}%
\pgfpathlineto{\pgfqpoint{3.095894in}{2.641193in}}%
\pgfpathlineto{\pgfqpoint{3.088168in}{2.634627in}}%
\pgfpathlineto{\pgfqpoint{3.075586in}{2.640590in}}%
\pgfpathlineto{\pgfqpoint{3.063007in}{2.646603in}}%
\pgfpathlineto{\pgfqpoint{3.050431in}{2.652664in}}%
\pgfpathlineto{\pgfqpoint{3.037859in}{2.658775in}}%
\pgfpathlineto{\pgfqpoint{3.045599in}{2.665295in}}%
\pgfpathlineto{\pgfqpoint{3.053332in}{2.671868in}}%
\pgfpathlineto{\pgfqpoint{3.061058in}{2.678492in}}%
\pgfpathlineto{\pgfqpoint{3.068778in}{2.685169in}}%
\pgfpathclose%
\pgfusepath{fill}%
\end{pgfscope}%
\begin{pgfscope}%
\pgfpathrectangle{\pgfqpoint{1.254980in}{0.150000in}}{\pgfqpoint{5.490039in}{5.490039in}}%
\pgfusepath{clip}%
\pgfsetbuttcap%
\pgfsetroundjoin%
\definecolor{currentfill}{rgb}{0.272594,0.025563,0.353093}%
\pgfsetfillcolor{currentfill}%
\pgfsetfillopacity{0.700000}%
\pgfsetlinewidth{0.000000pt}%
\definecolor{currentstroke}{rgb}{0.000000,0.000000,0.000000}%
\pgfsetstrokecolor{currentstroke}%
\pgfsetdash{}{0pt}%
\pgfpathmoveto{\pgfqpoint{3.462377in}{2.637525in}}%
\pgfpathlineto{\pgfqpoint{3.474989in}{2.632573in}}%
\pgfpathlineto{\pgfqpoint{3.487605in}{2.627660in}}%
\pgfpathlineto{\pgfqpoint{3.500225in}{2.622786in}}%
\pgfpathlineto{\pgfqpoint{3.512850in}{2.617952in}}%
\pgfpathlineto{\pgfqpoint{3.505294in}{2.610672in}}%
\pgfpathlineto{\pgfqpoint{3.497733in}{2.603423in}}%
\pgfpathlineto{\pgfqpoint{3.490165in}{2.596203in}}%
\pgfpathlineto{\pgfqpoint{3.482592in}{2.589012in}}%
\pgfpathlineto{\pgfqpoint{3.469955in}{2.593850in}}%
\pgfpathlineto{\pgfqpoint{3.457323in}{2.598728in}}%
\pgfpathlineto{\pgfqpoint{3.444695in}{2.603644in}}%
\pgfpathlineto{\pgfqpoint{3.432071in}{2.608600in}}%
\pgfpathlineto{\pgfqpoint{3.439656in}{2.615783in}}%
\pgfpathlineto{\pgfqpoint{3.447236in}{2.622998in}}%
\pgfpathlineto{\pgfqpoint{3.454810in}{2.630245in}}%
\pgfpathlineto{\pgfqpoint{3.462377in}{2.637525in}}%
\pgfpathclose%
\pgfusepath{fill}%
\end{pgfscope}%
\begin{pgfscope}%
\pgfpathrectangle{\pgfqpoint{1.254980in}{0.150000in}}{\pgfqpoint{5.490039in}{5.490039in}}%
\pgfusepath{clip}%
\pgfsetbuttcap%
\pgfsetroundjoin%
\definecolor{currentfill}{rgb}{0.274952,0.037752,0.364543}%
\pgfsetfillcolor{currentfill}%
\pgfsetfillopacity{0.700000}%
\pgfsetlinewidth{0.000000pt}%
\definecolor{currentstroke}{rgb}{0.000000,0.000000,0.000000}%
\pgfsetstrokecolor{currentstroke}%
\pgfsetdash{}{0pt}%
\pgfpathmoveto{\pgfqpoint{4.278954in}{2.645942in}}%
\pgfpathlineto{\pgfqpoint{4.291728in}{2.642275in}}%
\pgfpathlineto{\pgfqpoint{4.304507in}{2.638638in}}%
\pgfpathlineto{\pgfqpoint{4.317291in}{2.635029in}}%
\pgfpathlineto{\pgfqpoint{4.330081in}{2.631448in}}%
\pgfpathlineto{\pgfqpoint{4.322815in}{2.623858in}}%
\pgfpathlineto{\pgfqpoint{4.315543in}{2.616309in}}%
\pgfpathlineto{\pgfqpoint{4.308266in}{2.608797in}}%
\pgfpathlineto{\pgfqpoint{4.300985in}{2.601321in}}%
\pgfpathlineto{\pgfqpoint{4.288182in}{2.604817in}}%
\pgfpathlineto{\pgfqpoint{4.275386in}{2.608342in}}%
\pgfpathlineto{\pgfqpoint{4.262595in}{2.611895in}}%
\pgfpathlineto{\pgfqpoint{4.249810in}{2.615477in}}%
\pgfpathlineto{\pgfqpoint{4.257103in}{2.623033in}}%
\pgfpathlineto{\pgfqpoint{4.264392in}{2.630627in}}%
\pgfpathlineto{\pgfqpoint{4.271676in}{2.638263in}}%
\pgfpathlineto{\pgfqpoint{4.278954in}{2.645942in}}%
\pgfpathclose%
\pgfusepath{fill}%
\end{pgfscope}%
\begin{pgfscope}%
\pgfpathrectangle{\pgfqpoint{1.254980in}{0.150000in}}{\pgfqpoint{5.490039in}{5.490039in}}%
\pgfusepath{clip}%
\pgfsetbuttcap%
\pgfsetroundjoin%
\definecolor{currentfill}{rgb}{0.272594,0.025563,0.353093}%
\pgfsetfillcolor{currentfill}%
\pgfsetfillopacity{0.700000}%
\pgfsetlinewidth{0.000000pt}%
\definecolor{currentstroke}{rgb}{0.000000,0.000000,0.000000}%
\pgfsetstrokecolor{currentstroke}%
\pgfsetdash{}{0pt}%
\pgfpathmoveto{\pgfqpoint{3.593512in}{2.628386in}}%
\pgfpathlineto{\pgfqpoint{3.606148in}{2.623733in}}%
\pgfpathlineto{\pgfqpoint{3.618788in}{2.619117in}}%
\pgfpathlineto{\pgfqpoint{3.631433in}{2.614537in}}%
\pgfpathlineto{\pgfqpoint{3.644082in}{2.609994in}}%
\pgfpathlineto{\pgfqpoint{3.636573in}{2.602617in}}%
\pgfpathlineto{\pgfqpoint{3.629058in}{2.595266in}}%
\pgfpathlineto{\pgfqpoint{3.621538in}{2.587941in}}%
\pgfpathlineto{\pgfqpoint{3.614011in}{2.580640in}}%
\pgfpathlineto{\pgfqpoint{3.601350in}{2.585174in}}%
\pgfpathlineto{\pgfqpoint{3.588693in}{2.589744in}}%
\pgfpathlineto{\pgfqpoint{3.576041in}{2.594352in}}%
\pgfpathlineto{\pgfqpoint{3.563394in}{2.598996in}}%
\pgfpathlineto{\pgfqpoint{3.570932in}{2.606301in}}%
\pgfpathlineto{\pgfqpoint{3.578465in}{2.613634in}}%
\pgfpathlineto{\pgfqpoint{3.585991in}{2.620996in}}%
\pgfpathlineto{\pgfqpoint{3.593512in}{2.628386in}}%
\pgfpathclose%
\pgfusepath{fill}%
\end{pgfscope}%
\begin{pgfscope}%
\pgfpathrectangle{\pgfqpoint{1.254980in}{0.150000in}}{\pgfqpoint{5.490039in}{5.490039in}}%
\pgfusepath{clip}%
\pgfsetbuttcap%
\pgfsetroundjoin%
\definecolor{currentfill}{rgb}{0.279566,0.067836,0.391917}%
\pgfsetfillcolor{currentfill}%
\pgfsetfillopacity{0.700000}%
\pgfsetlinewidth{0.000000pt}%
\definecolor{currentstroke}{rgb}{0.000000,0.000000,0.000000}%
\pgfsetstrokecolor{currentstroke}%
\pgfsetdash{}{0pt}%
\pgfpathmoveto{\pgfqpoint{4.833236in}{2.690907in}}%
\pgfpathlineto{\pgfqpoint{4.846134in}{2.687496in}}%
\pgfpathlineto{\pgfqpoint{4.859039in}{2.684111in}}%
\pgfpathlineto{\pgfqpoint{4.871949in}{2.680751in}}%
\pgfpathlineto{\pgfqpoint{4.884866in}{2.677416in}}%
\pgfpathlineto{\pgfqpoint{4.877787in}{2.669473in}}%
\pgfpathlineto{\pgfqpoint{4.870704in}{2.661623in}}%
\pgfpathlineto{\pgfqpoint{4.863618in}{2.653861in}}%
\pgfpathlineto{\pgfqpoint{4.856529in}{2.646183in}}%
\pgfpathlineto{\pgfqpoint{4.843598in}{2.649370in}}%
\pgfpathlineto{\pgfqpoint{4.830674in}{2.652583in}}%
\pgfpathlineto{\pgfqpoint{4.817756in}{2.655821in}}%
\pgfpathlineto{\pgfqpoint{4.804844in}{2.659085in}}%
\pgfpathlineto{\pgfqpoint{4.811947in}{2.666906in}}%
\pgfpathlineto{\pgfqpoint{4.819047in}{2.674813in}}%
\pgfpathlineto{\pgfqpoint{4.826143in}{2.682812in}}%
\pgfpathlineto{\pgfqpoint{4.833236in}{2.690907in}}%
\pgfpathclose%
\pgfusepath{fill}%
\end{pgfscope}%
\begin{pgfscope}%
\pgfpathrectangle{\pgfqpoint{1.254980in}{0.150000in}}{\pgfqpoint{5.490039in}{5.490039in}}%
\pgfusepath{clip}%
\pgfsetbuttcap%
\pgfsetroundjoin%
\definecolor{currentfill}{rgb}{0.277018,0.050344,0.375715}%
\pgfsetfillcolor{currentfill}%
\pgfsetfillopacity{0.700000}%
\pgfsetlinewidth{0.000000pt}%
\definecolor{currentstroke}{rgb}{0.000000,0.000000,0.000000}%
\pgfsetstrokecolor{currentstroke}%
\pgfsetdash{}{0pt}%
\pgfpathmoveto{\pgfqpoint{4.621764in}{2.668146in}}%
\pgfpathlineto{\pgfqpoint{4.634616in}{2.664705in}}%
\pgfpathlineto{\pgfqpoint{4.647475in}{2.661290in}}%
\pgfpathlineto{\pgfqpoint{4.660339in}{2.657902in}}%
\pgfpathlineto{\pgfqpoint{4.673209in}{2.654540in}}%
\pgfpathlineto{\pgfqpoint{4.666060in}{2.646833in}}%
\pgfpathlineto{\pgfqpoint{4.658906in}{2.639193in}}%
\pgfpathlineto{\pgfqpoint{4.651748in}{2.631617in}}%
\pgfpathlineto{\pgfqpoint{4.644586in}{2.624100in}}%
\pgfpathlineto{\pgfqpoint{4.631702in}{2.627340in}}%
\pgfpathlineto{\pgfqpoint{4.618825in}{2.630606in}}%
\pgfpathlineto{\pgfqpoint{4.605954in}{2.633899in}}%
\pgfpathlineto{\pgfqpoint{4.593088in}{2.637218in}}%
\pgfpathlineto{\pgfqpoint{4.600264in}{2.644852in}}%
\pgfpathlineto{\pgfqpoint{4.607435in}{2.652549in}}%
\pgfpathlineto{\pgfqpoint{4.614601in}{2.660313in}}%
\pgfpathlineto{\pgfqpoint{4.621764in}{2.668146in}}%
\pgfpathclose%
\pgfusepath{fill}%
\end{pgfscope}%
\begin{pgfscope}%
\pgfpathrectangle{\pgfqpoint{1.254980in}{0.150000in}}{\pgfqpoint{5.490039in}{5.490039in}}%
\pgfusepath{clip}%
\pgfsetbuttcap%
\pgfsetroundjoin%
\definecolor{currentfill}{rgb}{0.272594,0.025563,0.353093}%
\pgfsetfillcolor{currentfill}%
\pgfsetfillopacity{0.700000}%
\pgfsetlinewidth{0.000000pt}%
\definecolor{currentstroke}{rgb}{0.000000,0.000000,0.000000}%
\pgfsetstrokecolor{currentstroke}%
\pgfsetdash{}{0pt}%
\pgfpathmoveto{\pgfqpoint{4.067458in}{2.630014in}}%
\pgfpathlineto{\pgfqpoint{4.080188in}{2.626145in}}%
\pgfpathlineto{\pgfqpoint{4.092924in}{2.622307in}}%
\pgfpathlineto{\pgfqpoint{4.105665in}{2.618499in}}%
\pgfpathlineto{\pgfqpoint{4.118412in}{2.614722in}}%
\pgfpathlineto{\pgfqpoint{4.111069in}{2.607190in}}%
\pgfpathlineto{\pgfqpoint{4.103722in}{2.599687in}}%
\pgfpathlineto{\pgfqpoint{4.096369in}{2.592211in}}%
\pgfpathlineto{\pgfqpoint{4.089011in}{2.584760in}}%
\pgfpathlineto{\pgfqpoint{4.076253in}{2.588478in}}%
\pgfpathlineto{\pgfqpoint{4.063500in}{2.592226in}}%
\pgfpathlineto{\pgfqpoint{4.050752in}{2.596005in}}%
\pgfpathlineto{\pgfqpoint{4.038011in}{2.599814in}}%
\pgfpathlineto{\pgfqpoint{4.045380in}{2.607320in}}%
\pgfpathlineto{\pgfqpoint{4.052745in}{2.614853in}}%
\pgfpathlineto{\pgfqpoint{4.060104in}{2.622417in}}%
\pgfpathlineto{\pgfqpoint{4.067458in}{2.630014in}}%
\pgfpathclose%
\pgfusepath{fill}%
\end{pgfscope}%
\begin{pgfscope}%
\pgfpathrectangle{\pgfqpoint{1.254980in}{0.150000in}}{\pgfqpoint{5.490039in}{5.490039in}}%
\pgfusepath{clip}%
\pgfsetbuttcap%
\pgfsetroundjoin%
\definecolor{currentfill}{rgb}{0.271305,0.019942,0.347269}%
\pgfsetfillcolor{currentfill}%
\pgfsetfillopacity{0.700000}%
\pgfsetlinewidth{0.000000pt}%
\definecolor{currentstroke}{rgb}{0.000000,0.000000,0.000000}%
\pgfsetstrokecolor{currentstroke}%
\pgfsetdash{}{0pt}%
\pgfpathmoveto{\pgfqpoint{3.724663in}{2.621887in}}%
\pgfpathlineto{\pgfqpoint{3.737325in}{2.617501in}}%
\pgfpathlineto{\pgfqpoint{3.749992in}{2.613150in}}%
\pgfpathlineto{\pgfqpoint{3.762663in}{2.608834in}}%
\pgfpathlineto{\pgfqpoint{3.775340in}{2.604552in}}%
\pgfpathlineto{\pgfqpoint{3.767876in}{2.597113in}}%
\pgfpathlineto{\pgfqpoint{3.760407in}{2.589697in}}%
\pgfpathlineto{\pgfqpoint{3.752932in}{2.582305in}}%
\pgfpathlineto{\pgfqpoint{3.745452in}{2.574935in}}%
\pgfpathlineto{\pgfqpoint{3.732763in}{2.579194in}}%
\pgfpathlineto{\pgfqpoint{3.720080in}{2.583489in}}%
\pgfpathlineto{\pgfqpoint{3.707402in}{2.587818in}}%
\pgfpathlineto{\pgfqpoint{3.694728in}{2.592182in}}%
\pgfpathlineto{\pgfqpoint{3.702220in}{2.599570in}}%
\pgfpathlineto{\pgfqpoint{3.709707in}{2.606982in}}%
\pgfpathlineto{\pgfqpoint{3.717188in}{2.614421in}}%
\pgfpathlineto{\pgfqpoint{3.724663in}{2.621887in}}%
\pgfpathclose%
\pgfusepath{fill}%
\end{pgfscope}%
\begin{pgfscope}%
\pgfpathrectangle{\pgfqpoint{1.254980in}{0.150000in}}{\pgfqpoint{5.490039in}{5.490039in}}%
\pgfusepath{clip}%
\pgfsetbuttcap%
\pgfsetroundjoin%
\definecolor{currentfill}{rgb}{0.274952,0.037752,0.364543}%
\pgfsetfillcolor{currentfill}%
\pgfsetfillopacity{0.700000}%
\pgfsetlinewidth{0.000000pt}%
\definecolor{currentstroke}{rgb}{0.000000,0.000000,0.000000}%
\pgfsetstrokecolor{currentstroke}%
\pgfsetdash{}{0pt}%
\pgfpathmoveto{\pgfqpoint{4.410271in}{2.647853in}}%
\pgfpathlineto{\pgfqpoint{4.423078in}{2.644316in}}%
\pgfpathlineto{\pgfqpoint{4.435890in}{2.640808in}}%
\pgfpathlineto{\pgfqpoint{4.448709in}{2.637327in}}%
\pgfpathlineto{\pgfqpoint{4.461533in}{2.633874in}}%
\pgfpathlineto{\pgfqpoint{4.454310in}{2.626293in}}%
\pgfpathlineto{\pgfqpoint{4.447082in}{2.618759in}}%
\pgfpathlineto{\pgfqpoint{4.439850in}{2.611269in}}%
\pgfpathlineto{\pgfqpoint{4.432612in}{2.603820in}}%
\pgfpathlineto{\pgfqpoint{4.419775in}{2.607176in}}%
\pgfpathlineto{\pgfqpoint{4.406944in}{2.610559in}}%
\pgfpathlineto{\pgfqpoint{4.394119in}{2.613971in}}%
\pgfpathlineto{\pgfqpoint{4.381300in}{2.617410in}}%
\pgfpathlineto{\pgfqpoint{4.388550in}{2.624952in}}%
\pgfpathlineto{\pgfqpoint{4.395795in}{2.632537in}}%
\pgfpathlineto{\pgfqpoint{4.403035in}{2.640170in}}%
\pgfpathlineto{\pgfqpoint{4.410271in}{2.647853in}}%
\pgfpathclose%
\pgfusepath{fill}%
\end{pgfscope}%
\begin{pgfscope}%
\pgfpathrectangle{\pgfqpoint{1.254980in}{0.150000in}}{\pgfqpoint{5.490039in}{5.490039in}}%
\pgfusepath{clip}%
\pgfsetbuttcap%
\pgfsetroundjoin%
\definecolor{currentfill}{rgb}{0.271305,0.019942,0.347269}%
\pgfsetfillcolor{currentfill}%
\pgfsetfillopacity{0.700000}%
\pgfsetlinewidth{0.000000pt}%
\definecolor{currentstroke}{rgb}{0.000000,0.000000,0.000000}%
\pgfsetstrokecolor{currentstroke}%
\pgfsetdash{}{0pt}%
\pgfpathmoveto{\pgfqpoint{3.855852in}{2.617655in}}%
\pgfpathlineto{\pgfqpoint{3.868542in}{2.613508in}}%
\pgfpathlineto{\pgfqpoint{3.881238in}{2.609395in}}%
\pgfpathlineto{\pgfqpoint{3.893938in}{2.605314in}}%
\pgfpathlineto{\pgfqpoint{3.906644in}{2.601266in}}%
\pgfpathlineto{\pgfqpoint{3.899225in}{2.593794in}}%
\pgfpathlineto{\pgfqpoint{3.891800in}{2.586346in}}%
\pgfpathlineto{\pgfqpoint{3.884370in}{2.578919in}}%
\pgfpathlineto{\pgfqpoint{3.876935in}{2.571513in}}%
\pgfpathlineto{\pgfqpoint{3.864218in}{2.575527in}}%
\pgfpathlineto{\pgfqpoint{3.851506in}{2.579574in}}%
\pgfpathlineto{\pgfqpoint{3.838799in}{2.583653in}}%
\pgfpathlineto{\pgfqpoint{3.826097in}{2.587766in}}%
\pgfpathlineto{\pgfqpoint{3.833544in}{2.595201in}}%
\pgfpathlineto{\pgfqpoint{3.840985in}{2.602661in}}%
\pgfpathlineto{\pgfqpoint{3.848421in}{2.610145in}}%
\pgfpathlineto{\pgfqpoint{3.855852in}{2.617655in}}%
\pgfpathclose%
\pgfusepath{fill}%
\end{pgfscope}%
\begin{pgfscope}%
\pgfpathrectangle{\pgfqpoint{1.254980in}{0.150000in}}{\pgfqpoint{5.490039in}{5.490039in}}%
\pgfusepath{clip}%
\pgfsetbuttcap%
\pgfsetroundjoin%
\definecolor{currentfill}{rgb}{0.280267,0.073417,0.397163}%
\pgfsetfillcolor{currentfill}%
\pgfsetfillopacity{0.700000}%
\pgfsetlinewidth{0.000000pt}%
\definecolor{currentstroke}{rgb}{0.000000,0.000000,0.000000}%
\pgfsetstrokecolor{currentstroke}%
\pgfsetdash{}{0pt}%
\pgfpathmoveto{\pgfqpoint{4.964824in}{2.696490in}}%
\pgfpathlineto{\pgfqpoint{4.977757in}{2.693121in}}%
\pgfpathlineto{\pgfqpoint{4.990697in}{2.689777in}}%
\pgfpathlineto{\pgfqpoint{5.003643in}{2.686457in}}%
\pgfpathlineto{\pgfqpoint{5.016595in}{2.683163in}}%
\pgfpathlineto{\pgfqpoint{5.009556in}{2.675128in}}%
\pgfpathlineto{\pgfqpoint{5.002515in}{2.667199in}}%
\pgfpathlineto{\pgfqpoint{4.995470in}{2.659371in}}%
\pgfpathlineto{\pgfqpoint{4.988423in}{2.651640in}}%
\pgfpathlineto{\pgfqpoint{4.975456in}{2.654775in}}%
\pgfpathlineto{\pgfqpoint{4.962496in}{2.657935in}}%
\pgfpathlineto{\pgfqpoint{4.949542in}{2.661119in}}%
\pgfpathlineto{\pgfqpoint{4.936594in}{2.664329in}}%
\pgfpathlineto{\pgfqpoint{4.943656in}{2.672215in}}%
\pgfpathlineto{\pgfqpoint{4.950715in}{2.680200in}}%
\pgfpathlineto{\pgfqpoint{4.957771in}{2.688290in}}%
\pgfpathlineto{\pgfqpoint{4.964824in}{2.696490in}}%
\pgfpathclose%
\pgfusepath{fill}%
\end{pgfscope}%
\begin{pgfscope}%
\pgfpathrectangle{\pgfqpoint{1.254980in}{0.150000in}}{\pgfqpoint{5.490039in}{5.490039in}}%
\pgfusepath{clip}%
\pgfsetbuttcap%
\pgfsetroundjoin%
\definecolor{currentfill}{rgb}{0.273809,0.031497,0.358853}%
\pgfsetfillcolor{currentfill}%
\pgfsetfillopacity{0.700000}%
\pgfsetlinewidth{0.000000pt}%
\definecolor{currentstroke}{rgb}{0.000000,0.000000,0.000000}%
\pgfsetstrokecolor{currentstroke}%
\pgfsetdash{}{0pt}%
\pgfpathmoveto{\pgfqpoint{4.198726in}{2.630098in}}%
\pgfpathlineto{\pgfqpoint{4.211488in}{2.626399in}}%
\pgfpathlineto{\pgfqpoint{4.224257in}{2.622729in}}%
\pgfpathlineto{\pgfqpoint{4.237030in}{2.619088in}}%
\pgfpathlineto{\pgfqpoint{4.249810in}{2.615477in}}%
\pgfpathlineto{\pgfqpoint{4.242511in}{2.607957in}}%
\pgfpathlineto{\pgfqpoint{4.235208in}{2.600469in}}%
\pgfpathlineto{\pgfqpoint{4.227899in}{2.593012in}}%
\pgfpathlineto{\pgfqpoint{4.220585in}{2.585582in}}%
\pgfpathlineto{\pgfqpoint{4.207794in}{2.589122in}}%
\pgfpathlineto{\pgfqpoint{4.195008in}{2.592690in}}%
\pgfpathlineto{\pgfqpoint{4.182228in}{2.596288in}}%
\pgfpathlineto{\pgfqpoint{4.169454in}{2.599915in}}%
\pgfpathlineto{\pgfqpoint{4.176779in}{2.607412in}}%
\pgfpathlineto{\pgfqpoint{4.184100in}{2.614940in}}%
\pgfpathlineto{\pgfqpoint{4.191415in}{2.622501in}}%
\pgfpathlineto{\pgfqpoint{4.198726in}{2.630098in}}%
\pgfpathclose%
\pgfusepath{fill}%
\end{pgfscope}%
\begin{pgfscope}%
\pgfpathrectangle{\pgfqpoint{1.254980in}{0.150000in}}{\pgfqpoint{5.490039in}{5.490039in}}%
\pgfusepath{clip}%
\pgfsetbuttcap%
\pgfsetroundjoin%
\definecolor{currentfill}{rgb}{0.274952,0.037752,0.364543}%
\pgfsetfillcolor{currentfill}%
\pgfsetfillopacity{0.700000}%
\pgfsetlinewidth{0.000000pt}%
\definecolor{currentstroke}{rgb}{0.000000,0.000000,0.000000}%
\pgfsetstrokecolor{currentstroke}%
\pgfsetdash{}{0pt}%
\pgfpathmoveto{\pgfqpoint{3.250357in}{2.643073in}}%
\pgfpathlineto{\pgfqpoint{3.262945in}{2.637597in}}%
\pgfpathlineto{\pgfqpoint{3.275536in}{2.632165in}}%
\pgfpathlineto{\pgfqpoint{3.288132in}{2.626777in}}%
\pgfpathlineto{\pgfqpoint{3.300731in}{2.621432in}}%
\pgfpathlineto{\pgfqpoint{3.293090in}{2.614455in}}%
\pgfpathlineto{\pgfqpoint{3.285443in}{2.607515in}}%
\pgfpathlineto{\pgfqpoint{3.277790in}{2.600614in}}%
\pgfpathlineto{\pgfqpoint{3.270131in}{2.593751in}}%
\pgfpathlineto{\pgfqpoint{3.257519in}{2.599124in}}%
\pgfpathlineto{\pgfqpoint{3.244910in}{2.604542in}}%
\pgfpathlineto{\pgfqpoint{3.232306in}{2.610003in}}%
\pgfpathlineto{\pgfqpoint{3.219705in}{2.615508in}}%
\pgfpathlineto{\pgfqpoint{3.227378in}{2.622337in}}%
\pgfpathlineto{\pgfqpoint{3.235044in}{2.629208in}}%
\pgfpathlineto{\pgfqpoint{3.242704in}{2.636120in}}%
\pgfpathlineto{\pgfqpoint{3.250357in}{2.643073in}}%
\pgfpathclose%
\pgfusepath{fill}%
\end{pgfscope}%
\begin{pgfscope}%
\pgfpathrectangle{\pgfqpoint{1.254980in}{0.150000in}}{\pgfqpoint{5.490039in}{5.490039in}}%
\pgfusepath{clip}%
\pgfsetbuttcap%
\pgfsetroundjoin%
\definecolor{currentfill}{rgb}{0.278791,0.062145,0.386592}%
\pgfsetfillcolor{currentfill}%
\pgfsetfillopacity{0.700000}%
\pgfsetlinewidth{0.000000pt}%
\definecolor{currentstroke}{rgb}{0.000000,0.000000,0.000000}%
\pgfsetstrokecolor{currentstroke}%
\pgfsetdash{}{0pt}%
\pgfpathmoveto{\pgfqpoint{4.753257in}{2.672396in}}%
\pgfpathlineto{\pgfqpoint{4.766145in}{2.669030in}}%
\pgfpathlineto{\pgfqpoint{4.779038in}{2.665689in}}%
\pgfpathlineto{\pgfqpoint{4.791938in}{2.662374in}}%
\pgfpathlineto{\pgfqpoint{4.804844in}{2.659085in}}%
\pgfpathlineto{\pgfqpoint{4.797737in}{2.651347in}}%
\pgfpathlineto{\pgfqpoint{4.790626in}{2.643686in}}%
\pgfpathlineto{\pgfqpoint{4.783511in}{2.636098in}}%
\pgfpathlineto{\pgfqpoint{4.776392in}{2.628580in}}%
\pgfpathlineto{\pgfqpoint{4.763473in}{2.631735in}}%
\pgfpathlineto{\pgfqpoint{4.750560in}{2.634915in}}%
\pgfpathlineto{\pgfqpoint{4.737653in}{2.638121in}}%
\pgfpathlineto{\pgfqpoint{4.724752in}{2.641353in}}%
\pgfpathlineto{\pgfqpoint{4.731884in}{2.649001in}}%
\pgfpathlineto{\pgfqpoint{4.739012in}{2.656721in}}%
\pgfpathlineto{\pgfqpoint{4.746137in}{2.664518in}}%
\pgfpathlineto{\pgfqpoint{4.753257in}{2.672396in}}%
\pgfpathclose%
\pgfusepath{fill}%
\end{pgfscope}%
\begin{pgfscope}%
\pgfpathrectangle{\pgfqpoint{1.254980in}{0.150000in}}{\pgfqpoint{5.490039in}{5.490039in}}%
\pgfusepath{clip}%
\pgfsetbuttcap%
\pgfsetroundjoin%
\definecolor{currentfill}{rgb}{0.273809,0.031497,0.358853}%
\pgfsetfillcolor{currentfill}%
\pgfsetfillopacity{0.700000}%
\pgfsetlinewidth{0.000000pt}%
\definecolor{currentstroke}{rgb}{0.000000,0.000000,0.000000}%
\pgfsetstrokecolor{currentstroke}%
\pgfsetdash{}{0pt}%
\pgfpathmoveto{\pgfqpoint{3.381618in}{2.628828in}}%
\pgfpathlineto{\pgfqpoint{3.394225in}{2.623710in}}%
\pgfpathlineto{\pgfqpoint{3.406836in}{2.618633in}}%
\pgfpathlineto{\pgfqpoint{3.419452in}{2.613596in}}%
\pgfpathlineto{\pgfqpoint{3.432071in}{2.608600in}}%
\pgfpathlineto{\pgfqpoint{3.424480in}{2.601449in}}%
\pgfpathlineto{\pgfqpoint{3.416882in}{2.594330in}}%
\pgfpathlineto{\pgfqpoint{3.409279in}{2.587241in}}%
\pgfpathlineto{\pgfqpoint{3.401670in}{2.580184in}}%
\pgfpathlineto{\pgfqpoint{3.389038in}{2.585197in}}%
\pgfpathlineto{\pgfqpoint{3.376410in}{2.590249in}}%
\pgfpathlineto{\pgfqpoint{3.363787in}{2.595343in}}%
\pgfpathlineto{\pgfqpoint{3.351167in}{2.600477in}}%
\pgfpathlineto{\pgfqpoint{3.358789in}{2.607513in}}%
\pgfpathlineto{\pgfqpoint{3.366405in}{2.614584in}}%
\pgfpathlineto{\pgfqpoint{3.374015in}{2.621688in}}%
\pgfpathlineto{\pgfqpoint{3.381618in}{2.628828in}}%
\pgfpathclose%
\pgfusepath{fill}%
\end{pgfscope}%
\begin{pgfscope}%
\pgfpathrectangle{\pgfqpoint{1.254980in}{0.150000in}}{\pgfqpoint{5.490039in}{5.490039in}}%
\pgfusepath{clip}%
\pgfsetbuttcap%
\pgfsetroundjoin%
\definecolor{currentfill}{rgb}{0.277018,0.050344,0.375715}%
\pgfsetfillcolor{currentfill}%
\pgfsetfillopacity{0.700000}%
\pgfsetlinewidth{0.000000pt}%
\definecolor{currentstroke}{rgb}{0.000000,0.000000,0.000000}%
\pgfsetstrokecolor{currentstroke}%
\pgfsetdash{}{0pt}%
\pgfpathmoveto{\pgfqpoint{3.119032in}{2.661188in}}%
\pgfpathlineto{\pgfqpoint{3.131603in}{2.655314in}}%
\pgfpathlineto{\pgfqpoint{3.144179in}{2.649488in}}%
\pgfpathlineto{\pgfqpoint{3.156757in}{2.643710in}}%
\pgfpathlineto{\pgfqpoint{3.169340in}{2.637977in}}%
\pgfpathlineto{\pgfqpoint{3.161647in}{2.631227in}}%
\pgfpathlineto{\pgfqpoint{3.153948in}{2.624521in}}%
\pgfpathlineto{\pgfqpoint{3.146243in}{2.617862in}}%
\pgfpathlineto{\pgfqpoint{3.138531in}{2.611249in}}%
\pgfpathlineto{\pgfqpoint{3.125935in}{2.617023in}}%
\pgfpathlineto{\pgfqpoint{3.113342in}{2.622843in}}%
\pgfpathlineto{\pgfqpoint{3.100753in}{2.628711in}}%
\pgfpathlineto{\pgfqpoint{3.088168in}{2.634627in}}%
\pgfpathlineto{\pgfqpoint{3.095894in}{2.641193in}}%
\pgfpathlineto{\pgfqpoint{3.103613in}{2.647809in}}%
\pgfpathlineto{\pgfqpoint{3.111326in}{2.654474in}}%
\pgfpathlineto{\pgfqpoint{3.119032in}{2.661188in}}%
\pgfpathclose%
\pgfusepath{fill}%
\end{pgfscope}%
\begin{pgfscope}%
\pgfpathrectangle{\pgfqpoint{1.254980in}{0.150000in}}{\pgfqpoint{5.490039in}{5.490039in}}%
\pgfusepath{clip}%
\pgfsetbuttcap%
\pgfsetroundjoin%
\definecolor{currentfill}{rgb}{0.272594,0.025563,0.353093}%
\pgfsetfillcolor{currentfill}%
\pgfsetfillopacity{0.700000}%
\pgfsetlinewidth{0.000000pt}%
\definecolor{currentstroke}{rgb}{0.000000,0.000000,0.000000}%
\pgfsetstrokecolor{currentstroke}%
\pgfsetdash{}{0pt}%
\pgfpathmoveto{\pgfqpoint{3.512850in}{2.617952in}}%
\pgfpathlineto{\pgfqpoint{3.525479in}{2.613156in}}%
\pgfpathlineto{\pgfqpoint{3.538113in}{2.608398in}}%
\pgfpathlineto{\pgfqpoint{3.550751in}{2.603678in}}%
\pgfpathlineto{\pgfqpoint{3.563394in}{2.598996in}}%
\pgfpathlineto{\pgfqpoint{3.555850in}{2.591718in}}%
\pgfpathlineto{\pgfqpoint{3.548301in}{2.584467in}}%
\pgfpathlineto{\pgfqpoint{3.540746in}{2.577242in}}%
\pgfpathlineto{\pgfqpoint{3.533185in}{2.570042in}}%
\pgfpathlineto{\pgfqpoint{3.520530in}{2.574728in}}%
\pgfpathlineto{\pgfqpoint{3.507879in}{2.579451in}}%
\pgfpathlineto{\pgfqpoint{3.495234in}{2.584212in}}%
\pgfpathlineto{\pgfqpoint{3.482592in}{2.589012in}}%
\pgfpathlineto{\pgfqpoint{3.490165in}{2.596203in}}%
\pgfpathlineto{\pgfqpoint{3.497733in}{2.603423in}}%
\pgfpathlineto{\pgfqpoint{3.505294in}{2.610672in}}%
\pgfpathlineto{\pgfqpoint{3.512850in}{2.617952in}}%
\pgfpathclose%
\pgfusepath{fill}%
\end{pgfscope}%
\begin{pgfscope}%
\pgfpathrectangle{\pgfqpoint{1.254980in}{0.150000in}}{\pgfqpoint{5.490039in}{5.490039in}}%
\pgfusepath{clip}%
\pgfsetbuttcap%
\pgfsetroundjoin%
\definecolor{currentfill}{rgb}{0.276022,0.044167,0.370164}%
\pgfsetfillcolor{currentfill}%
\pgfsetfillopacity{0.700000}%
\pgfsetlinewidth{0.000000pt}%
\definecolor{currentstroke}{rgb}{0.000000,0.000000,0.000000}%
\pgfsetstrokecolor{currentstroke}%
\pgfsetdash{}{0pt}%
\pgfpathmoveto{\pgfqpoint{4.541687in}{2.650762in}}%
\pgfpathlineto{\pgfqpoint{4.554528in}{2.647336in}}%
\pgfpathlineto{\pgfqpoint{4.567376in}{2.643936in}}%
\pgfpathlineto{\pgfqpoint{4.580229in}{2.640564in}}%
\pgfpathlineto{\pgfqpoint{4.593088in}{2.637218in}}%
\pgfpathlineto{\pgfqpoint{4.585909in}{2.629642in}}%
\pgfpathlineto{\pgfqpoint{4.578724in}{2.622121in}}%
\pgfpathlineto{\pgfqpoint{4.571536in}{2.614652in}}%
\pgfpathlineto{\pgfqpoint{4.564343in}{2.607230in}}%
\pgfpathlineto{\pgfqpoint{4.551470in}{2.610466in}}%
\pgfpathlineto{\pgfqpoint{4.538604in}{2.613729in}}%
\pgfpathlineto{\pgfqpoint{4.525744in}{2.617019in}}%
\pgfpathlineto{\pgfqpoint{4.512890in}{2.620336in}}%
\pgfpathlineto{\pgfqpoint{4.520096in}{2.627862in}}%
\pgfpathlineto{\pgfqpoint{4.527298in}{2.635440in}}%
\pgfpathlineto{\pgfqpoint{4.534494in}{2.643072in}}%
\pgfpathlineto{\pgfqpoint{4.541687in}{2.650762in}}%
\pgfpathclose%
\pgfusepath{fill}%
\end{pgfscope}%
\begin{pgfscope}%
\pgfpathrectangle{\pgfqpoint{1.254980in}{0.150000in}}{\pgfqpoint{5.490039in}{5.490039in}}%
\pgfusepath{clip}%
\pgfsetbuttcap%
\pgfsetroundjoin%
\definecolor{currentfill}{rgb}{0.272594,0.025563,0.353093}%
\pgfsetfillcolor{currentfill}%
\pgfsetfillopacity{0.700000}%
\pgfsetlinewidth{0.000000pt}%
\definecolor{currentstroke}{rgb}{0.000000,0.000000,0.000000}%
\pgfsetstrokecolor{currentstroke}%
\pgfsetdash{}{0pt}%
\pgfpathmoveto{\pgfqpoint{3.987097in}{2.615364in}}%
\pgfpathlineto{\pgfqpoint{3.999817in}{2.611429in}}%
\pgfpathlineto{\pgfqpoint{4.012543in}{2.607526in}}%
\pgfpathlineto{\pgfqpoint{4.025274in}{2.603655in}}%
\pgfpathlineto{\pgfqpoint{4.038011in}{2.599814in}}%
\pgfpathlineto{\pgfqpoint{4.030636in}{2.592335in}}%
\pgfpathlineto{\pgfqpoint{4.023256in}{2.584879in}}%
\pgfpathlineto{\pgfqpoint{4.015870in}{2.577446in}}%
\pgfpathlineto{\pgfqpoint{4.008480in}{2.570034in}}%
\pgfpathlineto{\pgfqpoint{3.995731in}{2.573828in}}%
\pgfpathlineto{\pgfqpoint{3.982989in}{2.577652in}}%
\pgfpathlineto{\pgfqpoint{3.970251in}{2.581508in}}%
\pgfpathlineto{\pgfqpoint{3.957519in}{2.585396in}}%
\pgfpathlineto{\pgfqpoint{3.964921in}{2.592851in}}%
\pgfpathlineto{\pgfqpoint{3.972318in}{2.600329in}}%
\pgfpathlineto{\pgfqpoint{3.979710in}{2.607833in}}%
\pgfpathlineto{\pgfqpoint{3.987097in}{2.615364in}}%
\pgfpathclose%
\pgfusepath{fill}%
\end{pgfscope}%
\begin{pgfscope}%
\pgfpathrectangle{\pgfqpoint{1.254980in}{0.150000in}}{\pgfqpoint{5.490039in}{5.490039in}}%
\pgfusepath{clip}%
\pgfsetbuttcap%
\pgfsetroundjoin%
\definecolor{currentfill}{rgb}{0.274952,0.037752,0.364543}%
\pgfsetfillcolor{currentfill}%
\pgfsetfillopacity{0.700000}%
\pgfsetlinewidth{0.000000pt}%
\definecolor{currentstroke}{rgb}{0.000000,0.000000,0.000000}%
\pgfsetstrokecolor{currentstroke}%
\pgfsetdash{}{0pt}%
\pgfpathmoveto{\pgfqpoint{4.330081in}{2.631448in}}%
\pgfpathlineto{\pgfqpoint{4.342877in}{2.627896in}}%
\pgfpathlineto{\pgfqpoint{4.355679in}{2.624373in}}%
\pgfpathlineto{\pgfqpoint{4.368487in}{2.620877in}}%
\pgfpathlineto{\pgfqpoint{4.381300in}{2.617410in}}%
\pgfpathlineto{\pgfqpoint{4.374045in}{2.609909in}}%
\pgfpathlineto{\pgfqpoint{4.366786in}{2.602446in}}%
\pgfpathlineto{\pgfqpoint{4.359522in}{2.595018in}}%
\pgfpathlineto{\pgfqpoint{4.352252in}{2.587622in}}%
\pgfpathlineto{\pgfqpoint{4.339427in}{2.591004in}}%
\pgfpathlineto{\pgfqpoint{4.326607in}{2.594415in}}%
\pgfpathlineto{\pgfqpoint{4.313793in}{2.597854in}}%
\pgfpathlineto{\pgfqpoint{4.300985in}{2.601321in}}%
\pgfpathlineto{\pgfqpoint{4.308266in}{2.608797in}}%
\pgfpathlineto{\pgfqpoint{4.315543in}{2.616309in}}%
\pgfpathlineto{\pgfqpoint{4.322815in}{2.623858in}}%
\pgfpathlineto{\pgfqpoint{4.330081in}{2.631448in}}%
\pgfpathclose%
\pgfusepath{fill}%
\end{pgfscope}%
\begin{pgfscope}%
\pgfpathrectangle{\pgfqpoint{1.254980in}{0.150000in}}{\pgfqpoint{5.490039in}{5.490039in}}%
\pgfusepath{clip}%
\pgfsetbuttcap%
\pgfsetroundjoin%
\definecolor{currentfill}{rgb}{0.271305,0.019942,0.347269}%
\pgfsetfillcolor{currentfill}%
\pgfsetfillopacity{0.700000}%
\pgfsetlinewidth{0.000000pt}%
\definecolor{currentstroke}{rgb}{0.000000,0.000000,0.000000}%
\pgfsetstrokecolor{currentstroke}%
\pgfsetdash{}{0pt}%
\pgfpathmoveto{\pgfqpoint{3.644082in}{2.609994in}}%
\pgfpathlineto{\pgfqpoint{3.656737in}{2.605487in}}%
\pgfpathlineto{\pgfqpoint{3.669396in}{2.601017in}}%
\pgfpathlineto{\pgfqpoint{3.682060in}{2.596582in}}%
\pgfpathlineto{\pgfqpoint{3.694728in}{2.592182in}}%
\pgfpathlineto{\pgfqpoint{3.687231in}{2.584819in}}%
\pgfpathlineto{\pgfqpoint{3.679727in}{2.577479in}}%
\pgfpathlineto{\pgfqpoint{3.672219in}{2.570161in}}%
\pgfpathlineto{\pgfqpoint{3.664705in}{2.562864in}}%
\pgfpathlineto{\pgfqpoint{3.652024in}{2.567255in}}%
\pgfpathlineto{\pgfqpoint{3.639348in}{2.571681in}}%
\pgfpathlineto{\pgfqpoint{3.626678in}{2.576142in}}%
\pgfpathlineto{\pgfqpoint{3.614011in}{2.580640in}}%
\pgfpathlineto{\pgfqpoint{3.621538in}{2.587941in}}%
\pgfpathlineto{\pgfqpoint{3.629058in}{2.595266in}}%
\pgfpathlineto{\pgfqpoint{3.636573in}{2.602617in}}%
\pgfpathlineto{\pgfqpoint{3.644082in}{2.609994in}}%
\pgfpathclose%
\pgfusepath{fill}%
\end{pgfscope}%
\begin{pgfscope}%
\pgfpathrectangle{\pgfqpoint{1.254980in}{0.150000in}}{\pgfqpoint{5.490039in}{5.490039in}}%
\pgfusepath{clip}%
\pgfsetbuttcap%
\pgfsetroundjoin%
\definecolor{currentfill}{rgb}{0.279566,0.067836,0.391917}%
\pgfsetfillcolor{currentfill}%
\pgfsetfillopacity{0.700000}%
\pgfsetlinewidth{0.000000pt}%
\definecolor{currentstroke}{rgb}{0.000000,0.000000,0.000000}%
\pgfsetstrokecolor{currentstroke}%
\pgfsetdash{}{0pt}%
\pgfpathmoveto{\pgfqpoint{4.884866in}{2.677416in}}%
\pgfpathlineto{\pgfqpoint{4.897789in}{2.674107in}}%
\pgfpathlineto{\pgfqpoint{4.910718in}{2.670823in}}%
\pgfpathlineto{\pgfqpoint{4.923653in}{2.667563in}}%
\pgfpathlineto{\pgfqpoint{4.936594in}{2.664329in}}%
\pgfpathlineto{\pgfqpoint{4.929529in}{2.656537in}}%
\pgfpathlineto{\pgfqpoint{4.922461in}{2.648836in}}%
\pgfpathlineto{\pgfqpoint{4.915389in}{2.641220in}}%
\pgfpathlineto{\pgfqpoint{4.908314in}{2.633684in}}%
\pgfpathlineto{\pgfqpoint{4.895358in}{2.636771in}}%
\pgfpathlineto{\pgfqpoint{4.882409in}{2.639883in}}%
\pgfpathlineto{\pgfqpoint{4.869466in}{2.643020in}}%
\pgfpathlineto{\pgfqpoint{4.856529in}{2.646183in}}%
\pgfpathlineto{\pgfqpoint{4.863618in}{2.653861in}}%
\pgfpathlineto{\pgfqpoint{4.870704in}{2.661623in}}%
\pgfpathlineto{\pgfqpoint{4.877787in}{2.669473in}}%
\pgfpathlineto{\pgfqpoint{4.884866in}{2.677416in}}%
\pgfpathclose%
\pgfusepath{fill}%
\end{pgfscope}%
\begin{pgfscope}%
\pgfpathrectangle{\pgfqpoint{1.254980in}{0.150000in}}{\pgfqpoint{5.490039in}{5.490039in}}%
\pgfusepath{clip}%
\pgfsetbuttcap%
\pgfsetroundjoin%
\definecolor{currentfill}{rgb}{0.272594,0.025563,0.353093}%
\pgfsetfillcolor{currentfill}%
\pgfsetfillopacity{0.700000}%
\pgfsetlinewidth{0.000000pt}%
\definecolor{currentstroke}{rgb}{0.000000,0.000000,0.000000}%
\pgfsetstrokecolor{currentstroke}%
\pgfsetdash{}{0pt}%
\pgfpathmoveto{\pgfqpoint{4.118412in}{2.614722in}}%
\pgfpathlineto{\pgfqpoint{4.131164in}{2.610975in}}%
\pgfpathlineto{\pgfqpoint{4.143922in}{2.607259in}}%
\pgfpathlineto{\pgfqpoint{4.156685in}{2.603572in}}%
\pgfpathlineto{\pgfqpoint{4.169454in}{2.599915in}}%
\pgfpathlineto{\pgfqpoint{4.162123in}{2.592447in}}%
\pgfpathlineto{\pgfqpoint{4.154787in}{2.585005in}}%
\pgfpathlineto{\pgfqpoint{4.147446in}{2.577587in}}%
\pgfpathlineto{\pgfqpoint{4.140100in}{2.570191in}}%
\pgfpathlineto{\pgfqpoint{4.127320in}{2.573788in}}%
\pgfpathlineto{\pgfqpoint{4.114545in}{2.577416in}}%
\pgfpathlineto{\pgfqpoint{4.101775in}{2.581073in}}%
\pgfpathlineto{\pgfqpoint{4.089011in}{2.584760in}}%
\pgfpathlineto{\pgfqpoint{4.096369in}{2.592211in}}%
\pgfpathlineto{\pgfqpoint{4.103722in}{2.599687in}}%
\pgfpathlineto{\pgfqpoint{4.111069in}{2.607190in}}%
\pgfpathlineto{\pgfqpoint{4.118412in}{2.614722in}}%
\pgfpathclose%
\pgfusepath{fill}%
\end{pgfscope}%
\begin{pgfscope}%
\pgfpathrectangle{\pgfqpoint{1.254980in}{0.150000in}}{\pgfqpoint{5.490039in}{5.490039in}}%
\pgfusepath{clip}%
\pgfsetbuttcap%
\pgfsetroundjoin%
\definecolor{currentfill}{rgb}{0.271305,0.019942,0.347269}%
\pgfsetfillcolor{currentfill}%
\pgfsetfillopacity{0.700000}%
\pgfsetlinewidth{0.000000pt}%
\definecolor{currentstroke}{rgb}{0.000000,0.000000,0.000000}%
\pgfsetstrokecolor{currentstroke}%
\pgfsetdash{}{0pt}%
\pgfpathmoveto{\pgfqpoint{3.775340in}{2.604552in}}%
\pgfpathlineto{\pgfqpoint{3.788022in}{2.600305in}}%
\pgfpathlineto{\pgfqpoint{3.800708in}{2.596092in}}%
\pgfpathlineto{\pgfqpoint{3.813400in}{2.591912in}}%
\pgfpathlineto{\pgfqpoint{3.826097in}{2.587766in}}%
\pgfpathlineto{\pgfqpoint{3.818644in}{2.580353in}}%
\pgfpathlineto{\pgfqpoint{3.811187in}{2.572961in}}%
\pgfpathlineto{\pgfqpoint{3.803724in}{2.565589in}}%
\pgfpathlineto{\pgfqpoint{3.796255in}{2.558235in}}%
\pgfpathlineto{\pgfqpoint{3.783547in}{2.562359in}}%
\pgfpathlineto{\pgfqpoint{3.770843in}{2.566517in}}%
\pgfpathlineto{\pgfqpoint{3.758145in}{2.570709in}}%
\pgfpathlineto{\pgfqpoint{3.745452in}{2.574935in}}%
\pgfpathlineto{\pgfqpoint{3.752932in}{2.582305in}}%
\pgfpathlineto{\pgfqpoint{3.760407in}{2.589697in}}%
\pgfpathlineto{\pgfqpoint{3.767876in}{2.597113in}}%
\pgfpathlineto{\pgfqpoint{3.775340in}{2.604552in}}%
\pgfpathclose%
\pgfusepath{fill}%
\end{pgfscope}%
\begin{pgfscope}%
\pgfpathrectangle{\pgfqpoint{1.254980in}{0.150000in}}{\pgfqpoint{5.490039in}{5.490039in}}%
\pgfusepath{clip}%
\pgfsetbuttcap%
\pgfsetroundjoin%
\definecolor{currentfill}{rgb}{0.277941,0.056324,0.381191}%
\pgfsetfillcolor{currentfill}%
\pgfsetfillopacity{0.700000}%
\pgfsetlinewidth{0.000000pt}%
\definecolor{currentstroke}{rgb}{0.000000,0.000000,0.000000}%
\pgfsetstrokecolor{currentstroke}%
\pgfsetdash{}{0pt}%
\pgfpathmoveto{\pgfqpoint{4.673209in}{2.654540in}}%
\pgfpathlineto{\pgfqpoint{4.686086in}{2.651204in}}%
\pgfpathlineto{\pgfqpoint{4.698968in}{2.647894in}}%
\pgfpathlineto{\pgfqpoint{4.711857in}{2.644611in}}%
\pgfpathlineto{\pgfqpoint{4.724752in}{2.641353in}}%
\pgfpathlineto{\pgfqpoint{4.717616in}{2.633773in}}%
\pgfpathlineto{\pgfqpoint{4.710475in}{2.626257in}}%
\pgfpathlineto{\pgfqpoint{4.703330in}{2.618802in}}%
\pgfpathlineto{\pgfqpoint{4.696181in}{2.611402in}}%
\pgfpathlineto{\pgfqpoint{4.683273in}{2.614538in}}%
\pgfpathlineto{\pgfqpoint{4.670371in}{2.617699in}}%
\pgfpathlineto{\pgfqpoint{4.657476in}{2.620886in}}%
\pgfpathlineto{\pgfqpoint{4.644586in}{2.624100in}}%
\pgfpathlineto{\pgfqpoint{4.651748in}{2.631617in}}%
\pgfpathlineto{\pgfqpoint{4.658906in}{2.639193in}}%
\pgfpathlineto{\pgfqpoint{4.666060in}{2.646833in}}%
\pgfpathlineto{\pgfqpoint{4.673209in}{2.654540in}}%
\pgfpathclose%
\pgfusepath{fill}%
\end{pgfscope}%
\begin{pgfscope}%
\pgfpathrectangle{\pgfqpoint{1.254980in}{0.150000in}}{\pgfqpoint{5.490039in}{5.490039in}}%
\pgfusepath{clip}%
\pgfsetbuttcap%
\pgfsetroundjoin%
\definecolor{currentfill}{rgb}{0.276022,0.044167,0.370164}%
\pgfsetfillcolor{currentfill}%
\pgfsetfillopacity{0.700000}%
\pgfsetlinewidth{0.000000pt}%
\definecolor{currentstroke}{rgb}{0.000000,0.000000,0.000000}%
\pgfsetstrokecolor{currentstroke}%
\pgfsetdash{}{0pt}%
\pgfpathmoveto{\pgfqpoint{4.461533in}{2.633874in}}%
\pgfpathlineto{\pgfqpoint{4.474364in}{2.630449in}}%
\pgfpathlineto{\pgfqpoint{4.487200in}{2.627050in}}%
\pgfpathlineto{\pgfqpoint{4.500042in}{2.623679in}}%
\pgfpathlineto{\pgfqpoint{4.512890in}{2.620336in}}%
\pgfpathlineto{\pgfqpoint{4.505679in}{2.612856in}}%
\pgfpathlineto{\pgfqpoint{4.498464in}{2.605421in}}%
\pgfpathlineto{\pgfqpoint{4.491244in}{2.598027in}}%
\pgfpathlineto{\pgfqpoint{4.484019in}{2.590670in}}%
\pgfpathlineto{\pgfqpoint{4.471158in}{2.593917in}}%
\pgfpathlineto{\pgfqpoint{4.458304in}{2.597190in}}%
\pgfpathlineto{\pgfqpoint{4.445455in}{2.600491in}}%
\pgfpathlineto{\pgfqpoint{4.432612in}{2.603820in}}%
\pgfpathlineto{\pgfqpoint{4.439850in}{2.611269in}}%
\pgfpathlineto{\pgfqpoint{4.447082in}{2.618759in}}%
\pgfpathlineto{\pgfqpoint{4.454310in}{2.626293in}}%
\pgfpathlineto{\pgfqpoint{4.461533in}{2.633874in}}%
\pgfpathclose%
\pgfusepath{fill}%
\end{pgfscope}%
\begin{pgfscope}%
\pgfpathrectangle{\pgfqpoint{1.254980in}{0.150000in}}{\pgfqpoint{5.490039in}{5.490039in}}%
\pgfusepath{clip}%
\pgfsetbuttcap%
\pgfsetroundjoin%
\definecolor{currentfill}{rgb}{0.273809,0.031497,0.358853}%
\pgfsetfillcolor{currentfill}%
\pgfsetfillopacity{0.700000}%
\pgfsetlinewidth{0.000000pt}%
\definecolor{currentstroke}{rgb}{0.000000,0.000000,0.000000}%
\pgfsetstrokecolor{currentstroke}%
\pgfsetdash{}{0pt}%
\pgfpathmoveto{\pgfqpoint{3.300731in}{2.621432in}}%
\pgfpathlineto{\pgfqpoint{3.313334in}{2.616130in}}%
\pgfpathlineto{\pgfqpoint{3.325941in}{2.610870in}}%
\pgfpathlineto{\pgfqpoint{3.338552in}{2.605653in}}%
\pgfpathlineto{\pgfqpoint{3.351167in}{2.600477in}}%
\pgfpathlineto{\pgfqpoint{3.343540in}{2.593476in}}%
\pgfpathlineto{\pgfqpoint{3.335906in}{2.586509in}}%
\pgfpathlineto{\pgfqpoint{3.328265in}{2.579577in}}%
\pgfpathlineto{\pgfqpoint{3.320619in}{2.572680in}}%
\pgfpathlineto{\pgfqpoint{3.307991in}{2.577885in}}%
\pgfpathlineto{\pgfqpoint{3.295367in}{2.583131in}}%
\pgfpathlineto{\pgfqpoint{3.282747in}{2.588420in}}%
\pgfpathlineto{\pgfqpoint{3.270131in}{2.593751in}}%
\pgfpathlineto{\pgfqpoint{3.277790in}{2.600614in}}%
\pgfpathlineto{\pgfqpoint{3.285443in}{2.607515in}}%
\pgfpathlineto{\pgfqpoint{3.293090in}{2.614455in}}%
\pgfpathlineto{\pgfqpoint{3.300731in}{2.621432in}}%
\pgfpathclose%
\pgfusepath{fill}%
\end{pgfscope}%
\begin{pgfscope}%
\pgfpathrectangle{\pgfqpoint{1.254980in}{0.150000in}}{\pgfqpoint{5.490039in}{5.490039in}}%
\pgfusepath{clip}%
\pgfsetbuttcap%
\pgfsetroundjoin%
\definecolor{currentfill}{rgb}{0.276022,0.044167,0.370164}%
\pgfsetfillcolor{currentfill}%
\pgfsetfillopacity{0.700000}%
\pgfsetlinewidth{0.000000pt}%
\definecolor{currentstroke}{rgb}{0.000000,0.000000,0.000000}%
\pgfsetstrokecolor{currentstroke}%
\pgfsetdash{}{0pt}%
\pgfpathmoveto{\pgfqpoint{3.169340in}{2.637977in}}%
\pgfpathlineto{\pgfqpoint{3.181926in}{2.632292in}}%
\pgfpathlineto{\pgfqpoint{3.194515in}{2.626652in}}%
\pgfpathlineto{\pgfqpoint{3.207108in}{2.621057in}}%
\pgfpathlineto{\pgfqpoint{3.219705in}{2.615508in}}%
\pgfpathlineto{\pgfqpoint{3.212026in}{2.608720in}}%
\pgfpathlineto{\pgfqpoint{3.204341in}{2.601974in}}%
\pgfpathlineto{\pgfqpoint{3.196649in}{2.595272in}}%
\pgfpathlineto{\pgfqpoint{3.188950in}{2.588613in}}%
\pgfpathlineto{\pgfqpoint{3.176340in}{2.594204in}}%
\pgfpathlineto{\pgfqpoint{3.163733in}{2.599840in}}%
\pgfpathlineto{\pgfqpoint{3.151130in}{2.605522in}}%
\pgfpathlineto{\pgfqpoint{3.138531in}{2.611249in}}%
\pgfpathlineto{\pgfqpoint{3.146243in}{2.617862in}}%
\pgfpathlineto{\pgfqpoint{3.153948in}{2.624521in}}%
\pgfpathlineto{\pgfqpoint{3.161647in}{2.631227in}}%
\pgfpathlineto{\pgfqpoint{3.169340in}{2.637977in}}%
\pgfpathclose%
\pgfusepath{fill}%
\end{pgfscope}%
\begin{pgfscope}%
\pgfpathrectangle{\pgfqpoint{1.254980in}{0.150000in}}{\pgfqpoint{5.490039in}{5.490039in}}%
\pgfusepath{clip}%
\pgfsetbuttcap%
\pgfsetroundjoin%
\definecolor{currentfill}{rgb}{0.271305,0.019942,0.347269}%
\pgfsetfillcolor{currentfill}%
\pgfsetfillopacity{0.700000}%
\pgfsetlinewidth{0.000000pt}%
\definecolor{currentstroke}{rgb}{0.000000,0.000000,0.000000}%
\pgfsetstrokecolor{currentstroke}%
\pgfsetdash{}{0pt}%
\pgfpathmoveto{\pgfqpoint{3.906644in}{2.601266in}}%
\pgfpathlineto{\pgfqpoint{3.919355in}{2.597250in}}%
\pgfpathlineto{\pgfqpoint{3.932071in}{2.593267in}}%
\pgfpathlineto{\pgfqpoint{3.944792in}{2.589315in}}%
\pgfpathlineto{\pgfqpoint{3.957519in}{2.585396in}}%
\pgfpathlineto{\pgfqpoint{3.950112in}{2.577963in}}%
\pgfpathlineto{\pgfqpoint{3.942699in}{2.570551in}}%
\pgfpathlineto{\pgfqpoint{3.935280in}{2.563157in}}%
\pgfpathlineto{\pgfqpoint{3.927857in}{2.555781in}}%
\pgfpathlineto{\pgfqpoint{3.915118in}{2.559666in}}%
\pgfpathlineto{\pgfqpoint{3.902385in}{2.563583in}}%
\pgfpathlineto{\pgfqpoint{3.889658in}{2.567532in}}%
\pgfpathlineto{\pgfqpoint{3.876935in}{2.571513in}}%
\pgfpathlineto{\pgfqpoint{3.884370in}{2.578919in}}%
\pgfpathlineto{\pgfqpoint{3.891800in}{2.586346in}}%
\pgfpathlineto{\pgfqpoint{3.899225in}{2.593794in}}%
\pgfpathlineto{\pgfqpoint{3.906644in}{2.601266in}}%
\pgfpathclose%
\pgfusepath{fill}%
\end{pgfscope}%
\begin{pgfscope}%
\pgfpathrectangle{\pgfqpoint{1.254980in}{0.150000in}}{\pgfqpoint{5.490039in}{5.490039in}}%
\pgfusepath{clip}%
\pgfsetbuttcap%
\pgfsetroundjoin%
\definecolor{currentfill}{rgb}{0.272594,0.025563,0.353093}%
\pgfsetfillcolor{currentfill}%
\pgfsetfillopacity{0.700000}%
\pgfsetlinewidth{0.000000pt}%
\definecolor{currentstroke}{rgb}{0.000000,0.000000,0.000000}%
\pgfsetstrokecolor{currentstroke}%
\pgfsetdash{}{0pt}%
\pgfpathmoveto{\pgfqpoint{3.432071in}{2.608600in}}%
\pgfpathlineto{\pgfqpoint{3.444695in}{2.603644in}}%
\pgfpathlineto{\pgfqpoint{3.457323in}{2.598728in}}%
\pgfpathlineto{\pgfqpoint{3.469955in}{2.593850in}}%
\pgfpathlineto{\pgfqpoint{3.482592in}{2.589012in}}%
\pgfpathlineto{\pgfqpoint{3.475013in}{2.581849in}}%
\pgfpathlineto{\pgfqpoint{3.467428in}{2.574715in}}%
\pgfpathlineto{\pgfqpoint{3.459838in}{2.567609in}}%
\pgfpathlineto{\pgfqpoint{3.452241in}{2.560532in}}%
\pgfpathlineto{\pgfqpoint{3.439592in}{2.565386in}}%
\pgfpathlineto{\pgfqpoint{3.426947in}{2.570279in}}%
\pgfpathlineto{\pgfqpoint{3.414306in}{2.575212in}}%
\pgfpathlineto{\pgfqpoint{3.401670in}{2.580184in}}%
\pgfpathlineto{\pgfqpoint{3.409279in}{2.587241in}}%
\pgfpathlineto{\pgfqpoint{3.416882in}{2.594330in}}%
\pgfpathlineto{\pgfqpoint{3.424480in}{2.601449in}}%
\pgfpathlineto{\pgfqpoint{3.432071in}{2.608600in}}%
\pgfpathclose%
\pgfusepath{fill}%
\end{pgfscope}%
\begin{pgfscope}%
\pgfpathrectangle{\pgfqpoint{1.254980in}{0.150000in}}{\pgfqpoint{5.490039in}{5.490039in}}%
\pgfusepath{clip}%
\pgfsetbuttcap%
\pgfsetroundjoin%
\definecolor{currentfill}{rgb}{0.273809,0.031497,0.358853}%
\pgfsetfillcolor{currentfill}%
\pgfsetfillopacity{0.700000}%
\pgfsetlinewidth{0.000000pt}%
\definecolor{currentstroke}{rgb}{0.000000,0.000000,0.000000}%
\pgfsetstrokecolor{currentstroke}%
\pgfsetdash{}{0pt}%
\pgfpathmoveto{\pgfqpoint{4.249810in}{2.615477in}}%
\pgfpathlineto{\pgfqpoint{4.262595in}{2.611895in}}%
\pgfpathlineto{\pgfqpoint{4.275386in}{2.608342in}}%
\pgfpathlineto{\pgfqpoint{4.288182in}{2.604817in}}%
\pgfpathlineto{\pgfqpoint{4.300985in}{2.601321in}}%
\pgfpathlineto{\pgfqpoint{4.293698in}{2.593878in}}%
\pgfpathlineto{\pgfqpoint{4.286407in}{2.586464in}}%
\pgfpathlineto{\pgfqpoint{4.279110in}{2.579077in}}%
\pgfpathlineto{\pgfqpoint{4.271809in}{2.571715in}}%
\pgfpathlineto{\pgfqpoint{4.258994in}{2.575139in}}%
\pgfpathlineto{\pgfqpoint{4.246185in}{2.578591in}}%
\pgfpathlineto{\pgfqpoint{4.233383in}{2.582072in}}%
\pgfpathlineto{\pgfqpoint{4.220585in}{2.585582in}}%
\pgfpathlineto{\pgfqpoint{4.227899in}{2.593012in}}%
\pgfpathlineto{\pgfqpoint{4.235208in}{2.600469in}}%
\pgfpathlineto{\pgfqpoint{4.242511in}{2.607957in}}%
\pgfpathlineto{\pgfqpoint{4.249810in}{2.615477in}}%
\pgfpathclose%
\pgfusepath{fill}%
\end{pgfscope}%
\begin{pgfscope}%
\pgfpathrectangle{\pgfqpoint{1.254980in}{0.150000in}}{\pgfqpoint{5.490039in}{5.490039in}}%
\pgfusepath{clip}%
\pgfsetbuttcap%
\pgfsetroundjoin%
\definecolor{currentfill}{rgb}{0.277941,0.056324,0.381191}%
\pgfsetfillcolor{currentfill}%
\pgfsetfillopacity{0.700000}%
\pgfsetlinewidth{0.000000pt}%
\definecolor{currentstroke}{rgb}{0.000000,0.000000,0.000000}%
\pgfsetstrokecolor{currentstroke}%
\pgfsetdash{}{0pt}%
\pgfpathmoveto{\pgfqpoint{3.037859in}{2.658775in}}%
\pgfpathlineto{\pgfqpoint{3.050431in}{2.652664in}}%
\pgfpathlineto{\pgfqpoint{3.063007in}{2.646603in}}%
\pgfpathlineto{\pgfqpoint{3.075586in}{2.640590in}}%
\pgfpathlineto{\pgfqpoint{3.088168in}{2.634627in}}%
\pgfpathlineto{\pgfqpoint{3.080435in}{2.628111in}}%
\pgfpathlineto{\pgfqpoint{3.072695in}{2.621647in}}%
\pgfpathlineto{\pgfqpoint{3.064948in}{2.615235in}}%
\pgfpathlineto{\pgfqpoint{3.057195in}{2.608876in}}%
\pgfpathlineto{\pgfqpoint{3.044598in}{2.614894in}}%
\pgfpathlineto{\pgfqpoint{3.032005in}{2.620961in}}%
\pgfpathlineto{\pgfqpoint{3.019415in}{2.627077in}}%
\pgfpathlineto{\pgfqpoint{3.006828in}{2.633242in}}%
\pgfpathlineto{\pgfqpoint{3.014596in}{2.639542in}}%
\pgfpathlineto{\pgfqpoint{3.022357in}{2.645898in}}%
\pgfpathlineto{\pgfqpoint{3.030111in}{2.652309in}}%
\pgfpathlineto{\pgfqpoint{3.037859in}{2.658775in}}%
\pgfpathclose%
\pgfusepath{fill}%
\end{pgfscope}%
\begin{pgfscope}%
\pgfpathrectangle{\pgfqpoint{1.254980in}{0.150000in}}{\pgfqpoint{5.490039in}{5.490039in}}%
\pgfusepath{clip}%
\pgfsetbuttcap%
\pgfsetroundjoin%
\definecolor{currentfill}{rgb}{0.280267,0.073417,0.397163}%
\pgfsetfillcolor{currentfill}%
\pgfsetfillopacity{0.700000}%
\pgfsetlinewidth{0.000000pt}%
\definecolor{currentstroke}{rgb}{0.000000,0.000000,0.000000}%
\pgfsetstrokecolor{currentstroke}%
\pgfsetdash{}{0pt}%
\pgfpathmoveto{\pgfqpoint{5.016595in}{2.683163in}}%
\pgfpathlineto{\pgfqpoint{5.029553in}{2.679893in}}%
\pgfpathlineto{\pgfqpoint{5.042518in}{2.676647in}}%
\pgfpathlineto{\pgfqpoint{5.055489in}{2.673426in}}%
\pgfpathlineto{\pgfqpoint{5.068466in}{2.670230in}}%
\pgfpathlineto{\pgfqpoint{5.061442in}{2.662359in}}%
\pgfpathlineto{\pgfqpoint{5.054416in}{2.654591in}}%
\pgfpathlineto{\pgfqpoint{5.047386in}{2.646922in}}%
\pgfpathlineto{\pgfqpoint{5.040354in}{2.639346in}}%
\pgfpathlineto{\pgfqpoint{5.027362in}{2.642383in}}%
\pgfpathlineto{\pgfqpoint{5.014376in}{2.645444in}}%
\pgfpathlineto{\pgfqpoint{5.001396in}{2.648530in}}%
\pgfpathlineto{\pgfqpoint{4.988423in}{2.651640in}}%
\pgfpathlineto{\pgfqpoint{4.995470in}{2.659371in}}%
\pgfpathlineto{\pgfqpoint{5.002515in}{2.667199in}}%
\pgfpathlineto{\pgfqpoint{5.009556in}{2.675128in}}%
\pgfpathlineto{\pgfqpoint{5.016595in}{2.683163in}}%
\pgfpathclose%
\pgfusepath{fill}%
\end{pgfscope}%
\begin{pgfscope}%
\pgfpathrectangle{\pgfqpoint{1.254980in}{0.150000in}}{\pgfqpoint{5.490039in}{5.490039in}}%
\pgfusepath{clip}%
\pgfsetbuttcap%
\pgfsetroundjoin%
\definecolor{currentfill}{rgb}{0.271305,0.019942,0.347269}%
\pgfsetfillcolor{currentfill}%
\pgfsetfillopacity{0.700000}%
\pgfsetlinewidth{0.000000pt}%
\definecolor{currentstroke}{rgb}{0.000000,0.000000,0.000000}%
\pgfsetstrokecolor{currentstroke}%
\pgfsetdash{}{0pt}%
\pgfpathmoveto{\pgfqpoint{3.563394in}{2.598996in}}%
\pgfpathlineto{\pgfqpoint{3.576041in}{2.594352in}}%
\pgfpathlineto{\pgfqpoint{3.588693in}{2.589744in}}%
\pgfpathlineto{\pgfqpoint{3.601350in}{2.585174in}}%
\pgfpathlineto{\pgfqpoint{3.614011in}{2.580640in}}%
\pgfpathlineto{\pgfqpoint{3.606480in}{2.573363in}}%
\pgfpathlineto{\pgfqpoint{3.598942in}{2.566110in}}%
\pgfpathlineto{\pgfqpoint{3.591399in}{2.558880in}}%
\pgfpathlineto{\pgfqpoint{3.583850in}{2.551672in}}%
\pgfpathlineto{\pgfqpoint{3.571177in}{2.556210in}}%
\pgfpathlineto{\pgfqpoint{3.558508in}{2.560784in}}%
\pgfpathlineto{\pgfqpoint{3.545844in}{2.565394in}}%
\pgfpathlineto{\pgfqpoint{3.533185in}{2.570042in}}%
\pgfpathlineto{\pgfqpoint{3.540746in}{2.577242in}}%
\pgfpathlineto{\pgfqpoint{3.548301in}{2.584467in}}%
\pgfpathlineto{\pgfqpoint{3.555850in}{2.591718in}}%
\pgfpathlineto{\pgfqpoint{3.563394in}{2.598996in}}%
\pgfpathclose%
\pgfusepath{fill}%
\end{pgfscope}%
\begin{pgfscope}%
\pgfpathrectangle{\pgfqpoint{1.254980in}{0.150000in}}{\pgfqpoint{5.490039in}{5.490039in}}%
\pgfusepath{clip}%
\pgfsetbuttcap%
\pgfsetroundjoin%
\definecolor{currentfill}{rgb}{0.278791,0.062145,0.386592}%
\pgfsetfillcolor{currentfill}%
\pgfsetfillopacity{0.700000}%
\pgfsetlinewidth{0.000000pt}%
\definecolor{currentstroke}{rgb}{0.000000,0.000000,0.000000}%
\pgfsetstrokecolor{currentstroke}%
\pgfsetdash{}{0pt}%
\pgfpathmoveto{\pgfqpoint{4.804844in}{2.659085in}}%
\pgfpathlineto{\pgfqpoint{4.817756in}{2.655821in}}%
\pgfpathlineto{\pgfqpoint{4.830674in}{2.652583in}}%
\pgfpathlineto{\pgfqpoint{4.843598in}{2.649370in}}%
\pgfpathlineto{\pgfqpoint{4.856529in}{2.646183in}}%
\pgfpathlineto{\pgfqpoint{4.849436in}{2.638583in}}%
\pgfpathlineto{\pgfqpoint{4.842339in}{2.631059in}}%
\pgfpathlineto{\pgfqpoint{4.835238in}{2.623605in}}%
\pgfpathlineto{\pgfqpoint{4.828133in}{2.616217in}}%
\pgfpathlineto{\pgfqpoint{4.815188in}{2.619270in}}%
\pgfpathlineto{\pgfqpoint{4.802250in}{2.622348in}}%
\pgfpathlineto{\pgfqpoint{4.789318in}{2.625451in}}%
\pgfpathlineto{\pgfqpoint{4.776392in}{2.628580in}}%
\pgfpathlineto{\pgfqpoint{4.783511in}{2.636098in}}%
\pgfpathlineto{\pgfqpoint{4.790626in}{2.643686in}}%
\pgfpathlineto{\pgfqpoint{4.797737in}{2.651347in}}%
\pgfpathlineto{\pgfqpoint{4.804844in}{2.659085in}}%
\pgfpathclose%
\pgfusepath{fill}%
\end{pgfscope}%
\begin{pgfscope}%
\pgfpathrectangle{\pgfqpoint{1.254980in}{0.150000in}}{\pgfqpoint{5.490039in}{5.490039in}}%
\pgfusepath{clip}%
\pgfsetbuttcap%
\pgfsetroundjoin%
\definecolor{currentfill}{rgb}{0.277018,0.050344,0.375715}%
\pgfsetfillcolor{currentfill}%
\pgfsetfillopacity{0.700000}%
\pgfsetlinewidth{0.000000pt}%
\definecolor{currentstroke}{rgb}{0.000000,0.000000,0.000000}%
\pgfsetstrokecolor{currentstroke}%
\pgfsetdash{}{0pt}%
\pgfpathmoveto{\pgfqpoint{4.593088in}{2.637218in}}%
\pgfpathlineto{\pgfqpoint{4.605954in}{2.633899in}}%
\pgfpathlineto{\pgfqpoint{4.618825in}{2.630606in}}%
\pgfpathlineto{\pgfqpoint{4.631702in}{2.627340in}}%
\pgfpathlineto{\pgfqpoint{4.644586in}{2.624100in}}%
\pgfpathlineto{\pgfqpoint{4.637419in}{2.616639in}}%
\pgfpathlineto{\pgfqpoint{4.630248in}{2.609229in}}%
\pgfpathlineto{\pgfqpoint{4.623073in}{2.601868in}}%
\pgfpathlineto{\pgfqpoint{4.615892in}{2.594552in}}%
\pgfpathlineto{\pgfqpoint{4.602996in}{2.597682in}}%
\pgfpathlineto{\pgfqpoint{4.590105in}{2.600838in}}%
\pgfpathlineto{\pgfqpoint{4.577221in}{2.604021in}}%
\pgfpathlineto{\pgfqpoint{4.564343in}{2.607230in}}%
\pgfpathlineto{\pgfqpoint{4.571536in}{2.614652in}}%
\pgfpathlineto{\pgfqpoint{4.578724in}{2.622121in}}%
\pgfpathlineto{\pgfqpoint{4.585909in}{2.629642in}}%
\pgfpathlineto{\pgfqpoint{4.593088in}{2.637218in}}%
\pgfpathclose%
\pgfusepath{fill}%
\end{pgfscope}%
\begin{pgfscope}%
\pgfpathrectangle{\pgfqpoint{1.254980in}{0.150000in}}{\pgfqpoint{5.490039in}{5.490039in}}%
\pgfusepath{clip}%
\pgfsetbuttcap%
\pgfsetroundjoin%
\definecolor{currentfill}{rgb}{0.272594,0.025563,0.353093}%
\pgfsetfillcolor{currentfill}%
\pgfsetfillopacity{0.700000}%
\pgfsetlinewidth{0.000000pt}%
\definecolor{currentstroke}{rgb}{0.000000,0.000000,0.000000}%
\pgfsetstrokecolor{currentstroke}%
\pgfsetdash{}{0pt}%
\pgfpathmoveto{\pgfqpoint{4.038011in}{2.599814in}}%
\pgfpathlineto{\pgfqpoint{4.050752in}{2.596005in}}%
\pgfpathlineto{\pgfqpoint{4.063500in}{2.592226in}}%
\pgfpathlineto{\pgfqpoint{4.076253in}{2.588478in}}%
\pgfpathlineto{\pgfqpoint{4.089011in}{2.584760in}}%
\pgfpathlineto{\pgfqpoint{4.081648in}{2.577332in}}%
\pgfpathlineto{\pgfqpoint{4.074280in}{2.569926in}}%
\pgfpathlineto{\pgfqpoint{4.066906in}{2.562538in}}%
\pgfpathlineto{\pgfqpoint{4.059527in}{2.555168in}}%
\pgfpathlineto{\pgfqpoint{4.046757in}{2.558839in}}%
\pgfpathlineto{\pgfqpoint{4.033992in}{2.562540in}}%
\pgfpathlineto{\pgfqpoint{4.021233in}{2.566271in}}%
\pgfpathlineto{\pgfqpoint{4.008480in}{2.570034in}}%
\pgfpathlineto{\pgfqpoint{4.015870in}{2.577446in}}%
\pgfpathlineto{\pgfqpoint{4.023256in}{2.584879in}}%
\pgfpathlineto{\pgfqpoint{4.030636in}{2.592335in}}%
\pgfpathlineto{\pgfqpoint{4.038011in}{2.599814in}}%
\pgfpathclose%
\pgfusepath{fill}%
\end{pgfscope}%
\begin{pgfscope}%
\pgfpathrectangle{\pgfqpoint{1.254980in}{0.150000in}}{\pgfqpoint{5.490039in}{5.490039in}}%
\pgfusepath{clip}%
\pgfsetbuttcap%
\pgfsetroundjoin%
\definecolor{currentfill}{rgb}{0.271305,0.019942,0.347269}%
\pgfsetfillcolor{currentfill}%
\pgfsetfillopacity{0.700000}%
\pgfsetlinewidth{0.000000pt}%
\definecolor{currentstroke}{rgb}{0.000000,0.000000,0.000000}%
\pgfsetstrokecolor{currentstroke}%
\pgfsetdash{}{0pt}%
\pgfpathmoveto{\pgfqpoint{3.694728in}{2.592182in}}%
\pgfpathlineto{\pgfqpoint{3.707402in}{2.587818in}}%
\pgfpathlineto{\pgfqpoint{3.720080in}{2.583489in}}%
\pgfpathlineto{\pgfqpoint{3.732763in}{2.579194in}}%
\pgfpathlineto{\pgfqpoint{3.745452in}{2.574935in}}%
\pgfpathlineto{\pgfqpoint{3.737966in}{2.567585in}}%
\pgfpathlineto{\pgfqpoint{3.730475in}{2.560256in}}%
\pgfpathlineto{\pgfqpoint{3.722978in}{2.552945in}}%
\pgfpathlineto{\pgfqpoint{3.715475in}{2.545654in}}%
\pgfpathlineto{\pgfqpoint{3.702775in}{2.549904in}}%
\pgfpathlineto{\pgfqpoint{3.690080in}{2.554189in}}%
\pgfpathlineto{\pgfqpoint{3.677390in}{2.558509in}}%
\pgfpathlineto{\pgfqpoint{3.664705in}{2.562864in}}%
\pgfpathlineto{\pgfqpoint{3.672219in}{2.570161in}}%
\pgfpathlineto{\pgfqpoint{3.679727in}{2.577479in}}%
\pgfpathlineto{\pgfqpoint{3.687231in}{2.584819in}}%
\pgfpathlineto{\pgfqpoint{3.694728in}{2.592182in}}%
\pgfpathclose%
\pgfusepath{fill}%
\end{pgfscope}%
\begin{pgfscope}%
\pgfpathrectangle{\pgfqpoint{1.254980in}{0.150000in}}{\pgfqpoint{5.490039in}{5.490039in}}%
\pgfusepath{clip}%
\pgfsetbuttcap%
\pgfsetroundjoin%
\definecolor{currentfill}{rgb}{0.274952,0.037752,0.364543}%
\pgfsetfillcolor{currentfill}%
\pgfsetfillopacity{0.700000}%
\pgfsetlinewidth{0.000000pt}%
\definecolor{currentstroke}{rgb}{0.000000,0.000000,0.000000}%
\pgfsetstrokecolor{currentstroke}%
\pgfsetdash{}{0pt}%
\pgfpathmoveto{\pgfqpoint{4.381300in}{2.617410in}}%
\pgfpathlineto{\pgfqpoint{4.394119in}{2.613971in}}%
\pgfpathlineto{\pgfqpoint{4.406944in}{2.610559in}}%
\pgfpathlineto{\pgfqpoint{4.419775in}{2.607176in}}%
\pgfpathlineto{\pgfqpoint{4.432612in}{2.603820in}}%
\pgfpathlineto{\pgfqpoint{4.425370in}{2.596408in}}%
\pgfpathlineto{\pgfqpoint{4.418123in}{2.589032in}}%
\pgfpathlineto{\pgfqpoint{4.410871in}{2.581687in}}%
\pgfpathlineto{\pgfqpoint{4.403614in}{2.574371in}}%
\pgfpathlineto{\pgfqpoint{4.390764in}{2.577642in}}%
\pgfpathlineto{\pgfqpoint{4.377921in}{2.580941in}}%
\pgfpathlineto{\pgfqpoint{4.365084in}{2.584267in}}%
\pgfpathlineto{\pgfqpoint{4.352252in}{2.587622in}}%
\pgfpathlineto{\pgfqpoint{4.359522in}{2.595018in}}%
\pgfpathlineto{\pgfqpoint{4.366786in}{2.602446in}}%
\pgfpathlineto{\pgfqpoint{4.374045in}{2.609909in}}%
\pgfpathlineto{\pgfqpoint{4.381300in}{2.617410in}}%
\pgfpathclose%
\pgfusepath{fill}%
\end{pgfscope}%
\begin{pgfscope}%
\pgfpathrectangle{\pgfqpoint{1.254980in}{0.150000in}}{\pgfqpoint{5.490039in}{5.490039in}}%
\pgfusepath{clip}%
\pgfsetbuttcap%
\pgfsetroundjoin%
\definecolor{currentfill}{rgb}{0.271305,0.019942,0.347269}%
\pgfsetfillcolor{currentfill}%
\pgfsetfillopacity{0.700000}%
\pgfsetlinewidth{0.000000pt}%
\definecolor{currentstroke}{rgb}{0.000000,0.000000,0.000000}%
\pgfsetstrokecolor{currentstroke}%
\pgfsetdash{}{0pt}%
\pgfpathmoveto{\pgfqpoint{3.826097in}{2.587766in}}%
\pgfpathlineto{\pgfqpoint{3.838799in}{2.583653in}}%
\pgfpathlineto{\pgfqpoint{3.851506in}{2.579574in}}%
\pgfpathlineto{\pgfqpoint{3.864218in}{2.575527in}}%
\pgfpathlineto{\pgfqpoint{3.876935in}{2.571513in}}%
\pgfpathlineto{\pgfqpoint{3.869494in}{2.564127in}}%
\pgfpathlineto{\pgfqpoint{3.862048in}{2.556758in}}%
\pgfpathlineto{\pgfqpoint{3.854597in}{2.549406in}}%
\pgfpathlineto{\pgfqpoint{3.847140in}{2.542070in}}%
\pgfpathlineto{\pgfqpoint{3.834411in}{2.546062in}}%
\pgfpathlineto{\pgfqpoint{3.821687in}{2.550087in}}%
\pgfpathlineto{\pgfqpoint{3.808968in}{2.554145in}}%
\pgfpathlineto{\pgfqpoint{3.796255in}{2.558235in}}%
\pgfpathlineto{\pgfqpoint{3.803724in}{2.565589in}}%
\pgfpathlineto{\pgfqpoint{3.811187in}{2.572961in}}%
\pgfpathlineto{\pgfqpoint{3.818644in}{2.580353in}}%
\pgfpathlineto{\pgfqpoint{3.826097in}{2.587766in}}%
\pgfpathclose%
\pgfusepath{fill}%
\end{pgfscope}%
\begin{pgfscope}%
\pgfpathrectangle{\pgfqpoint{1.254980in}{0.150000in}}{\pgfqpoint{5.490039in}{5.490039in}}%
\pgfusepath{clip}%
\pgfsetbuttcap%
\pgfsetroundjoin%
\definecolor{currentfill}{rgb}{0.279566,0.067836,0.391917}%
\pgfsetfillcolor{currentfill}%
\pgfsetfillopacity{0.700000}%
\pgfsetlinewidth{0.000000pt}%
\definecolor{currentstroke}{rgb}{0.000000,0.000000,0.000000}%
\pgfsetstrokecolor{currentstroke}%
\pgfsetdash{}{0pt}%
\pgfpathmoveto{\pgfqpoint{4.936594in}{2.664329in}}%
\pgfpathlineto{\pgfqpoint{4.949542in}{2.661119in}}%
\pgfpathlineto{\pgfqpoint{4.962496in}{2.657935in}}%
\pgfpathlineto{\pgfqpoint{4.975456in}{2.654775in}}%
\pgfpathlineto{\pgfqpoint{4.988423in}{2.651640in}}%
\pgfpathlineto{\pgfqpoint{4.981373in}{2.644001in}}%
\pgfpathlineto{\pgfqpoint{4.974319in}{2.636448in}}%
\pgfpathlineto{\pgfqpoint{4.967262in}{2.628978in}}%
\pgfpathlineto{\pgfqpoint{4.960201in}{2.621585in}}%
\pgfpathlineto{\pgfqpoint{4.947219in}{2.624572in}}%
\pgfpathlineto{\pgfqpoint{4.934245in}{2.627585in}}%
\pgfpathlineto{\pgfqpoint{4.921276in}{2.630622in}}%
\pgfpathlineto{\pgfqpoint{4.908314in}{2.633684in}}%
\pgfpathlineto{\pgfqpoint{4.915389in}{2.641220in}}%
\pgfpathlineto{\pgfqpoint{4.922461in}{2.648836in}}%
\pgfpathlineto{\pgfqpoint{4.929529in}{2.656537in}}%
\pgfpathlineto{\pgfqpoint{4.936594in}{2.664329in}}%
\pgfpathclose%
\pgfusepath{fill}%
\end{pgfscope}%
\begin{pgfscope}%
\pgfpathrectangle{\pgfqpoint{1.254980in}{0.150000in}}{\pgfqpoint{5.490039in}{5.490039in}}%
\pgfusepath{clip}%
\pgfsetbuttcap%
\pgfsetroundjoin%
\definecolor{currentfill}{rgb}{0.272594,0.025563,0.353093}%
\pgfsetfillcolor{currentfill}%
\pgfsetfillopacity{0.700000}%
\pgfsetlinewidth{0.000000pt}%
\definecolor{currentstroke}{rgb}{0.000000,0.000000,0.000000}%
\pgfsetstrokecolor{currentstroke}%
\pgfsetdash{}{0pt}%
\pgfpathmoveto{\pgfqpoint{4.169454in}{2.599915in}}%
\pgfpathlineto{\pgfqpoint{4.182228in}{2.596288in}}%
\pgfpathlineto{\pgfqpoint{4.195008in}{2.592690in}}%
\pgfpathlineto{\pgfqpoint{4.207794in}{2.589122in}}%
\pgfpathlineto{\pgfqpoint{4.220585in}{2.585582in}}%
\pgfpathlineto{\pgfqpoint{4.213267in}{2.578178in}}%
\pgfpathlineto{\pgfqpoint{4.205943in}{2.570798in}}%
\pgfpathlineto{\pgfqpoint{4.198614in}{2.563438in}}%
\pgfpathlineto{\pgfqpoint{4.191279in}{2.556097in}}%
\pgfpathlineto{\pgfqpoint{4.178476in}{2.559577in}}%
\pgfpathlineto{\pgfqpoint{4.165678in}{2.563085in}}%
\pgfpathlineto{\pgfqpoint{4.152887in}{2.566623in}}%
\pgfpathlineto{\pgfqpoint{4.140100in}{2.570191in}}%
\pgfpathlineto{\pgfqpoint{4.147446in}{2.577587in}}%
\pgfpathlineto{\pgfqpoint{4.154787in}{2.585005in}}%
\pgfpathlineto{\pgfqpoint{4.162123in}{2.592447in}}%
\pgfpathlineto{\pgfqpoint{4.169454in}{2.599915in}}%
\pgfpathclose%
\pgfusepath{fill}%
\end{pgfscope}%
\begin{pgfscope}%
\pgfpathrectangle{\pgfqpoint{1.254980in}{0.150000in}}{\pgfqpoint{5.490039in}{5.490039in}}%
\pgfusepath{clip}%
\pgfsetbuttcap%
\pgfsetroundjoin%
\definecolor{currentfill}{rgb}{0.277941,0.056324,0.381191}%
\pgfsetfillcolor{currentfill}%
\pgfsetfillopacity{0.700000}%
\pgfsetlinewidth{0.000000pt}%
\definecolor{currentstroke}{rgb}{0.000000,0.000000,0.000000}%
\pgfsetstrokecolor{currentstroke}%
\pgfsetdash{}{0pt}%
\pgfpathmoveto{\pgfqpoint{4.724752in}{2.641353in}}%
\pgfpathlineto{\pgfqpoint{4.737653in}{2.638121in}}%
\pgfpathlineto{\pgfqpoint{4.750560in}{2.634915in}}%
\pgfpathlineto{\pgfqpoint{4.763473in}{2.631735in}}%
\pgfpathlineto{\pgfqpoint{4.776392in}{2.628580in}}%
\pgfpathlineto{\pgfqpoint{4.769270in}{2.621128in}}%
\pgfpathlineto{\pgfqpoint{4.762143in}{2.613736in}}%
\pgfpathlineto{\pgfqpoint{4.755012in}{2.606401in}}%
\pgfpathlineto{\pgfqpoint{4.747876in}{2.599120in}}%
\pgfpathlineto{\pgfqpoint{4.734943in}{2.602152in}}%
\pgfpathlineto{\pgfqpoint{4.722016in}{2.605210in}}%
\pgfpathlineto{\pgfqpoint{4.709096in}{2.608293in}}%
\pgfpathlineto{\pgfqpoint{4.696181in}{2.611402in}}%
\pgfpathlineto{\pgfqpoint{4.703330in}{2.618802in}}%
\pgfpathlineto{\pgfqpoint{4.710475in}{2.626257in}}%
\pgfpathlineto{\pgfqpoint{4.717616in}{2.633773in}}%
\pgfpathlineto{\pgfqpoint{4.724752in}{2.641353in}}%
\pgfpathclose%
\pgfusepath{fill}%
\end{pgfscope}%
\begin{pgfscope}%
\pgfpathrectangle{\pgfqpoint{1.254980in}{0.150000in}}{\pgfqpoint{5.490039in}{5.490039in}}%
\pgfusepath{clip}%
\pgfsetbuttcap%
\pgfsetroundjoin%
\definecolor{currentfill}{rgb}{0.274952,0.037752,0.364543}%
\pgfsetfillcolor{currentfill}%
\pgfsetfillopacity{0.700000}%
\pgfsetlinewidth{0.000000pt}%
\definecolor{currentstroke}{rgb}{0.000000,0.000000,0.000000}%
\pgfsetstrokecolor{currentstroke}%
\pgfsetdash{}{0pt}%
\pgfpathmoveto{\pgfqpoint{3.219705in}{2.615508in}}%
\pgfpathlineto{\pgfqpoint{3.232306in}{2.610003in}}%
\pgfpathlineto{\pgfqpoint{3.244910in}{2.604542in}}%
\pgfpathlineto{\pgfqpoint{3.257519in}{2.599124in}}%
\pgfpathlineto{\pgfqpoint{3.270131in}{2.593751in}}%
\pgfpathlineto{\pgfqpoint{3.262465in}{2.586926in}}%
\pgfpathlineto{\pgfqpoint{3.254793in}{2.580141in}}%
\pgfpathlineto{\pgfqpoint{3.247115in}{2.573395in}}%
\pgfpathlineto{\pgfqpoint{3.239430in}{2.566690in}}%
\pgfpathlineto{\pgfqpoint{3.226804in}{2.572105in}}%
\pgfpathlineto{\pgfqpoint{3.214183in}{2.577564in}}%
\pgfpathlineto{\pgfqpoint{3.201565in}{2.583066in}}%
\pgfpathlineto{\pgfqpoint{3.188950in}{2.588613in}}%
\pgfpathlineto{\pgfqpoint{3.196649in}{2.595272in}}%
\pgfpathlineto{\pgfqpoint{3.204341in}{2.601974in}}%
\pgfpathlineto{\pgfqpoint{3.212026in}{2.608720in}}%
\pgfpathlineto{\pgfqpoint{3.219705in}{2.615508in}}%
\pgfpathclose%
\pgfusepath{fill}%
\end{pgfscope}%
\begin{pgfscope}%
\pgfpathrectangle{\pgfqpoint{1.254980in}{0.150000in}}{\pgfqpoint{5.490039in}{5.490039in}}%
\pgfusepath{clip}%
\pgfsetbuttcap%
\pgfsetroundjoin%
\definecolor{currentfill}{rgb}{0.272594,0.025563,0.353093}%
\pgfsetfillcolor{currentfill}%
\pgfsetfillopacity{0.700000}%
\pgfsetlinewidth{0.000000pt}%
\definecolor{currentstroke}{rgb}{0.000000,0.000000,0.000000}%
\pgfsetstrokecolor{currentstroke}%
\pgfsetdash{}{0pt}%
\pgfpathmoveto{\pgfqpoint{3.351167in}{2.600477in}}%
\pgfpathlineto{\pgfqpoint{3.363787in}{2.595343in}}%
\pgfpathlineto{\pgfqpoint{3.376410in}{2.590249in}}%
\pgfpathlineto{\pgfqpoint{3.389038in}{2.585197in}}%
\pgfpathlineto{\pgfqpoint{3.401670in}{2.580184in}}%
\pgfpathlineto{\pgfqpoint{3.394055in}{2.573159in}}%
\pgfpathlineto{\pgfqpoint{3.386434in}{2.566165in}}%
\pgfpathlineto{\pgfqpoint{3.378806in}{2.559203in}}%
\pgfpathlineto{\pgfqpoint{3.371173in}{2.552273in}}%
\pgfpathlineto{\pgfqpoint{3.358528in}{2.557314in}}%
\pgfpathlineto{\pgfqpoint{3.345888in}{2.562395in}}%
\pgfpathlineto{\pgfqpoint{3.333251in}{2.567517in}}%
\pgfpathlineto{\pgfqpoint{3.320619in}{2.572680in}}%
\pgfpathlineto{\pgfqpoint{3.328265in}{2.579577in}}%
\pgfpathlineto{\pgfqpoint{3.335906in}{2.586509in}}%
\pgfpathlineto{\pgfqpoint{3.343540in}{2.593476in}}%
\pgfpathlineto{\pgfqpoint{3.351167in}{2.600477in}}%
\pgfpathclose%
\pgfusepath{fill}%
\end{pgfscope}%
\begin{pgfscope}%
\pgfpathrectangle{\pgfqpoint{1.254980in}{0.150000in}}{\pgfqpoint{5.490039in}{5.490039in}}%
\pgfusepath{clip}%
\pgfsetbuttcap%
\pgfsetroundjoin%
\definecolor{currentfill}{rgb}{0.277018,0.050344,0.375715}%
\pgfsetfillcolor{currentfill}%
\pgfsetfillopacity{0.700000}%
\pgfsetlinewidth{0.000000pt}%
\definecolor{currentstroke}{rgb}{0.000000,0.000000,0.000000}%
\pgfsetstrokecolor{currentstroke}%
\pgfsetdash{}{0pt}%
\pgfpathmoveto{\pgfqpoint{3.088168in}{2.634627in}}%
\pgfpathlineto{\pgfqpoint{3.100753in}{2.628711in}}%
\pgfpathlineto{\pgfqpoint{3.113342in}{2.622843in}}%
\pgfpathlineto{\pgfqpoint{3.125935in}{2.617023in}}%
\pgfpathlineto{\pgfqpoint{3.138531in}{2.611249in}}%
\pgfpathlineto{\pgfqpoint{3.130812in}{2.604684in}}%
\pgfpathlineto{\pgfqpoint{3.123086in}{2.598167in}}%
\pgfpathlineto{\pgfqpoint{3.115354in}{2.591699in}}%
\pgfpathlineto{\pgfqpoint{3.107615in}{2.585282in}}%
\pgfpathlineto{\pgfqpoint{3.095005in}{2.591110in}}%
\pgfpathlineto{\pgfqpoint{3.082398in}{2.596984in}}%
\pgfpathlineto{\pgfqpoint{3.069795in}{2.602906in}}%
\pgfpathlineto{\pgfqpoint{3.057195in}{2.608876in}}%
\pgfpathlineto{\pgfqpoint{3.064948in}{2.615235in}}%
\pgfpathlineto{\pgfqpoint{3.072695in}{2.621647in}}%
\pgfpathlineto{\pgfqpoint{3.080435in}{2.628111in}}%
\pgfpathlineto{\pgfqpoint{3.088168in}{2.634627in}}%
\pgfpathclose%
\pgfusepath{fill}%
\end{pgfscope}%
\begin{pgfscope}%
\pgfpathrectangle{\pgfqpoint{1.254980in}{0.150000in}}{\pgfqpoint{5.490039in}{5.490039in}}%
\pgfusepath{clip}%
\pgfsetbuttcap%
\pgfsetroundjoin%
\definecolor{currentfill}{rgb}{0.276022,0.044167,0.370164}%
\pgfsetfillcolor{currentfill}%
\pgfsetfillopacity{0.700000}%
\pgfsetlinewidth{0.000000pt}%
\definecolor{currentstroke}{rgb}{0.000000,0.000000,0.000000}%
\pgfsetstrokecolor{currentstroke}%
\pgfsetdash{}{0pt}%
\pgfpathmoveto{\pgfqpoint{4.512890in}{2.620336in}}%
\pgfpathlineto{\pgfqpoint{4.525744in}{2.617019in}}%
\pgfpathlineto{\pgfqpoint{4.538604in}{2.613729in}}%
\pgfpathlineto{\pgfqpoint{4.551470in}{2.610466in}}%
\pgfpathlineto{\pgfqpoint{4.564343in}{2.607230in}}%
\pgfpathlineto{\pgfqpoint{4.557145in}{2.599853in}}%
\pgfpathlineto{\pgfqpoint{4.549942in}{2.592517in}}%
\pgfpathlineto{\pgfqpoint{4.542735in}{2.585219in}}%
\pgfpathlineto{\pgfqpoint{4.535523in}{2.577955in}}%
\pgfpathlineto{\pgfqpoint{4.522638in}{2.581093in}}%
\pgfpathlineto{\pgfqpoint{4.509759in}{2.584259in}}%
\pgfpathlineto{\pgfqpoint{4.496886in}{2.587451in}}%
\pgfpathlineto{\pgfqpoint{4.484019in}{2.590670in}}%
\pgfpathlineto{\pgfqpoint{4.491244in}{2.598027in}}%
\pgfpathlineto{\pgfqpoint{4.498464in}{2.605421in}}%
\pgfpathlineto{\pgfqpoint{4.505679in}{2.612856in}}%
\pgfpathlineto{\pgfqpoint{4.512890in}{2.620336in}}%
\pgfpathclose%
\pgfusepath{fill}%
\end{pgfscope}%
\begin{pgfscope}%
\pgfpathrectangle{\pgfqpoint{1.254980in}{0.150000in}}{\pgfqpoint{5.490039in}{5.490039in}}%
\pgfusepath{clip}%
\pgfsetbuttcap%
\pgfsetroundjoin%
\definecolor{currentfill}{rgb}{0.271305,0.019942,0.347269}%
\pgfsetfillcolor{currentfill}%
\pgfsetfillopacity{0.700000}%
\pgfsetlinewidth{0.000000pt}%
\definecolor{currentstroke}{rgb}{0.000000,0.000000,0.000000}%
\pgfsetstrokecolor{currentstroke}%
\pgfsetdash{}{0pt}%
\pgfpathmoveto{\pgfqpoint{3.482592in}{2.589012in}}%
\pgfpathlineto{\pgfqpoint{3.495234in}{2.584212in}}%
\pgfpathlineto{\pgfqpoint{3.507879in}{2.579451in}}%
\pgfpathlineto{\pgfqpoint{3.520530in}{2.574728in}}%
\pgfpathlineto{\pgfqpoint{3.533185in}{2.570042in}}%
\pgfpathlineto{\pgfqpoint{3.525618in}{2.562869in}}%
\pgfpathlineto{\pgfqpoint{3.518045in}{2.555720in}}%
\pgfpathlineto{\pgfqpoint{3.510467in}{2.548597in}}%
\pgfpathlineto{\pgfqpoint{3.502883in}{2.541498in}}%
\pgfpathlineto{\pgfqpoint{3.490216in}{2.546200in}}%
\pgfpathlineto{\pgfqpoint{3.477553in}{2.550939in}}%
\pgfpathlineto{\pgfqpoint{3.464895in}{2.555716in}}%
\pgfpathlineto{\pgfqpoint{3.452241in}{2.560532in}}%
\pgfpathlineto{\pgfqpoint{3.459838in}{2.567609in}}%
\pgfpathlineto{\pgfqpoint{3.467428in}{2.574715in}}%
\pgfpathlineto{\pgfqpoint{3.475013in}{2.581849in}}%
\pgfpathlineto{\pgfqpoint{3.482592in}{2.589012in}}%
\pgfpathclose%
\pgfusepath{fill}%
\end{pgfscope}%
\begin{pgfscope}%
\pgfpathrectangle{\pgfqpoint{1.254980in}{0.150000in}}{\pgfqpoint{5.490039in}{5.490039in}}%
\pgfusepath{clip}%
\pgfsetbuttcap%
\pgfsetroundjoin%
\definecolor{currentfill}{rgb}{0.271305,0.019942,0.347269}%
\pgfsetfillcolor{currentfill}%
\pgfsetfillopacity{0.700000}%
\pgfsetlinewidth{0.000000pt}%
\definecolor{currentstroke}{rgb}{0.000000,0.000000,0.000000}%
\pgfsetstrokecolor{currentstroke}%
\pgfsetdash{}{0pt}%
\pgfpathmoveto{\pgfqpoint{3.957519in}{2.585396in}}%
\pgfpathlineto{\pgfqpoint{3.970251in}{2.581508in}}%
\pgfpathlineto{\pgfqpoint{3.982989in}{2.577652in}}%
\pgfpathlineto{\pgfqpoint{3.995731in}{2.573828in}}%
\pgfpathlineto{\pgfqpoint{4.008480in}{2.570034in}}%
\pgfpathlineto{\pgfqpoint{4.001084in}{2.562640in}}%
\pgfpathlineto{\pgfqpoint{3.993682in}{2.555264in}}%
\pgfpathlineto{\pgfqpoint{3.986276in}{2.547904in}}%
\pgfpathlineto{\pgfqpoint{3.978864in}{2.540558in}}%
\pgfpathlineto{\pgfqpoint{3.966104in}{2.544317in}}%
\pgfpathlineto{\pgfqpoint{3.953350in}{2.548107in}}%
\pgfpathlineto{\pgfqpoint{3.940601in}{2.551928in}}%
\pgfpathlineto{\pgfqpoint{3.927857in}{2.555781in}}%
\pgfpathlineto{\pgfqpoint{3.935280in}{2.563157in}}%
\pgfpathlineto{\pgfqpoint{3.942699in}{2.570551in}}%
\pgfpathlineto{\pgfqpoint{3.950112in}{2.577963in}}%
\pgfpathlineto{\pgfqpoint{3.957519in}{2.585396in}}%
\pgfpathclose%
\pgfusepath{fill}%
\end{pgfscope}%
\begin{pgfscope}%
\pgfpathrectangle{\pgfqpoint{1.254980in}{0.150000in}}{\pgfqpoint{5.490039in}{5.490039in}}%
\pgfusepath{clip}%
\pgfsetbuttcap%
\pgfsetroundjoin%
\definecolor{currentfill}{rgb}{0.273809,0.031497,0.358853}%
\pgfsetfillcolor{currentfill}%
\pgfsetfillopacity{0.700000}%
\pgfsetlinewidth{0.000000pt}%
\definecolor{currentstroke}{rgb}{0.000000,0.000000,0.000000}%
\pgfsetstrokecolor{currentstroke}%
\pgfsetdash{}{0pt}%
\pgfpathmoveto{\pgfqpoint{4.300985in}{2.601321in}}%
\pgfpathlineto{\pgfqpoint{4.313793in}{2.597854in}}%
\pgfpathlineto{\pgfqpoint{4.326607in}{2.594415in}}%
\pgfpathlineto{\pgfqpoint{4.339427in}{2.591004in}}%
\pgfpathlineto{\pgfqpoint{4.352252in}{2.587622in}}%
\pgfpathlineto{\pgfqpoint{4.344978in}{2.580255in}}%
\pgfpathlineto{\pgfqpoint{4.337699in}{2.572915in}}%
\pgfpathlineto{\pgfqpoint{4.330414in}{2.565599in}}%
\pgfpathlineto{\pgfqpoint{4.323125in}{2.558305in}}%
\pgfpathlineto{\pgfqpoint{4.310287in}{2.561615in}}%
\pgfpathlineto{\pgfqpoint{4.297455in}{2.564954in}}%
\pgfpathlineto{\pgfqpoint{4.284629in}{2.568320in}}%
\pgfpathlineto{\pgfqpoint{4.271809in}{2.571715in}}%
\pgfpathlineto{\pgfqpoint{4.279110in}{2.579077in}}%
\pgfpathlineto{\pgfqpoint{4.286407in}{2.586464in}}%
\pgfpathlineto{\pgfqpoint{4.293698in}{2.593878in}}%
\pgfpathlineto{\pgfqpoint{4.300985in}{2.601321in}}%
\pgfpathclose%
\pgfusepath{fill}%
\end{pgfscope}%
\begin{pgfscope}%
\pgfpathrectangle{\pgfqpoint{1.254980in}{0.150000in}}{\pgfqpoint{5.490039in}{5.490039in}}%
\pgfusepath{clip}%
\pgfsetbuttcap%
\pgfsetroundjoin%
\definecolor{currentfill}{rgb}{0.271305,0.019942,0.347269}%
\pgfsetfillcolor{currentfill}%
\pgfsetfillopacity{0.700000}%
\pgfsetlinewidth{0.000000pt}%
\definecolor{currentstroke}{rgb}{0.000000,0.000000,0.000000}%
\pgfsetstrokecolor{currentstroke}%
\pgfsetdash{}{0pt}%
\pgfpathmoveto{\pgfqpoint{3.614011in}{2.580640in}}%
\pgfpathlineto{\pgfqpoint{3.626678in}{2.576142in}}%
\pgfpathlineto{\pgfqpoint{3.639348in}{2.571681in}}%
\pgfpathlineto{\pgfqpoint{3.652024in}{2.567255in}}%
\pgfpathlineto{\pgfqpoint{3.664705in}{2.562864in}}%
\pgfpathlineto{\pgfqpoint{3.657185in}{2.555589in}}%
\pgfpathlineto{\pgfqpoint{3.649659in}{2.548334in}}%
\pgfpathlineto{\pgfqpoint{3.642128in}{2.541099in}}%
\pgfpathlineto{\pgfqpoint{3.634591in}{2.533884in}}%
\pgfpathlineto{\pgfqpoint{3.621899in}{2.538277in}}%
\pgfpathlineto{\pgfqpoint{3.609211in}{2.542706in}}%
\pgfpathlineto{\pgfqpoint{3.596528in}{2.547171in}}%
\pgfpathlineto{\pgfqpoint{3.583850in}{2.551672in}}%
\pgfpathlineto{\pgfqpoint{3.591399in}{2.558880in}}%
\pgfpathlineto{\pgfqpoint{3.598942in}{2.566110in}}%
\pgfpathlineto{\pgfqpoint{3.606480in}{2.573363in}}%
\pgfpathlineto{\pgfqpoint{3.614011in}{2.580640in}}%
\pgfpathclose%
\pgfusepath{fill}%
\end{pgfscope}%
\begin{pgfscope}%
\pgfpathrectangle{\pgfqpoint{1.254980in}{0.150000in}}{\pgfqpoint{5.490039in}{5.490039in}}%
\pgfusepath{clip}%
\pgfsetbuttcap%
\pgfsetroundjoin%
\definecolor{currentfill}{rgb}{0.280894,0.078907,0.402329}%
\pgfsetfillcolor{currentfill}%
\pgfsetfillopacity{0.700000}%
\pgfsetlinewidth{0.000000pt}%
\definecolor{currentstroke}{rgb}{0.000000,0.000000,0.000000}%
\pgfsetstrokecolor{currentstroke}%
\pgfsetdash{}{0pt}%
\pgfpathmoveto{\pgfqpoint{5.068466in}{2.670230in}}%
\pgfpathlineto{\pgfqpoint{5.081450in}{2.667058in}}%
\pgfpathlineto{\pgfqpoint{5.094440in}{2.663910in}}%
\pgfpathlineto{\pgfqpoint{5.107436in}{2.660787in}}%
\pgfpathlineto{\pgfqpoint{5.120439in}{2.657687in}}%
\pgfpathlineto{\pgfqpoint{5.113431in}{2.649981in}}%
\pgfpathlineto{\pgfqpoint{5.106419in}{2.642375in}}%
\pgfpathlineto{\pgfqpoint{5.099405in}{2.634864in}}%
\pgfpathlineto{\pgfqpoint{5.092388in}{2.627444in}}%
\pgfpathlineto{\pgfqpoint{5.079370in}{2.630383in}}%
\pgfpathlineto{\pgfqpoint{5.066358in}{2.633346in}}%
\pgfpathlineto{\pgfqpoint{5.053353in}{2.636334in}}%
\pgfpathlineto{\pgfqpoint{5.040354in}{2.639346in}}%
\pgfpathlineto{\pgfqpoint{5.047386in}{2.646922in}}%
\pgfpathlineto{\pgfqpoint{5.054416in}{2.654591in}}%
\pgfpathlineto{\pgfqpoint{5.061442in}{2.662359in}}%
\pgfpathlineto{\pgfqpoint{5.068466in}{2.670230in}}%
\pgfpathclose%
\pgfusepath{fill}%
\end{pgfscope}%
\begin{pgfscope}%
\pgfpathrectangle{\pgfqpoint{1.254980in}{0.150000in}}{\pgfqpoint{5.490039in}{5.490039in}}%
\pgfusepath{clip}%
\pgfsetbuttcap%
\pgfsetroundjoin%
\definecolor{currentfill}{rgb}{0.278791,0.062145,0.386592}%
\pgfsetfillcolor{currentfill}%
\pgfsetfillopacity{0.700000}%
\pgfsetlinewidth{0.000000pt}%
\definecolor{currentstroke}{rgb}{0.000000,0.000000,0.000000}%
\pgfsetstrokecolor{currentstroke}%
\pgfsetdash{}{0pt}%
\pgfpathmoveto{\pgfqpoint{4.856529in}{2.646183in}}%
\pgfpathlineto{\pgfqpoint{4.869466in}{2.643020in}}%
\pgfpathlineto{\pgfqpoint{4.882409in}{2.639883in}}%
\pgfpathlineto{\pgfqpoint{4.895358in}{2.636771in}}%
\pgfpathlineto{\pgfqpoint{4.908314in}{2.633684in}}%
\pgfpathlineto{\pgfqpoint{4.901235in}{2.626224in}}%
\pgfpathlineto{\pgfqpoint{4.894152in}{2.618837in}}%
\pgfpathlineto{\pgfqpoint{4.887066in}{2.611516in}}%
\pgfpathlineto{\pgfqpoint{4.879975in}{2.604259in}}%
\pgfpathlineto{\pgfqpoint{4.867005in}{2.607211in}}%
\pgfpathlineto{\pgfqpoint{4.854041in}{2.610188in}}%
\pgfpathlineto{\pgfqpoint{4.841084in}{2.613190in}}%
\pgfpathlineto{\pgfqpoint{4.828133in}{2.616217in}}%
\pgfpathlineto{\pgfqpoint{4.835238in}{2.623605in}}%
\pgfpathlineto{\pgfqpoint{4.842339in}{2.631059in}}%
\pgfpathlineto{\pgfqpoint{4.849436in}{2.638583in}}%
\pgfpathlineto{\pgfqpoint{4.856529in}{2.646183in}}%
\pgfpathclose%
\pgfusepath{fill}%
\end{pgfscope}%
\begin{pgfscope}%
\pgfpathrectangle{\pgfqpoint{1.254980in}{0.150000in}}{\pgfqpoint{5.490039in}{5.490039in}}%
\pgfusepath{clip}%
\pgfsetbuttcap%
\pgfsetroundjoin%
\definecolor{currentfill}{rgb}{0.272594,0.025563,0.353093}%
\pgfsetfillcolor{currentfill}%
\pgfsetfillopacity{0.700000}%
\pgfsetlinewidth{0.000000pt}%
\definecolor{currentstroke}{rgb}{0.000000,0.000000,0.000000}%
\pgfsetstrokecolor{currentstroke}%
\pgfsetdash{}{0pt}%
\pgfpathmoveto{\pgfqpoint{4.089011in}{2.584760in}}%
\pgfpathlineto{\pgfqpoint{4.101775in}{2.581073in}}%
\pgfpathlineto{\pgfqpoint{4.114545in}{2.577416in}}%
\pgfpathlineto{\pgfqpoint{4.127320in}{2.573788in}}%
\pgfpathlineto{\pgfqpoint{4.140100in}{2.570191in}}%
\pgfpathlineto{\pgfqpoint{4.132749in}{2.562815in}}%
\pgfpathlineto{\pgfqpoint{4.125392in}{2.555457in}}%
\pgfpathlineto{\pgfqpoint{4.118030in}{2.548115in}}%
\pgfpathlineto{\pgfqpoint{4.110663in}{2.540788in}}%
\pgfpathlineto{\pgfqpoint{4.097871in}{2.544338in}}%
\pgfpathlineto{\pgfqpoint{4.085084in}{2.547918in}}%
\pgfpathlineto{\pgfqpoint{4.072303in}{2.551528in}}%
\pgfpathlineto{\pgfqpoint{4.059527in}{2.555168in}}%
\pgfpathlineto{\pgfqpoint{4.066906in}{2.562538in}}%
\pgfpathlineto{\pgfqpoint{4.074280in}{2.569926in}}%
\pgfpathlineto{\pgfqpoint{4.081648in}{2.577332in}}%
\pgfpathlineto{\pgfqpoint{4.089011in}{2.584760in}}%
\pgfpathclose%
\pgfusepath{fill}%
\end{pgfscope}%
\begin{pgfscope}%
\pgfpathrectangle{\pgfqpoint{1.254980in}{0.150000in}}{\pgfqpoint{5.490039in}{5.490039in}}%
\pgfusepath{clip}%
\pgfsetbuttcap%
\pgfsetroundjoin%
\definecolor{currentfill}{rgb}{0.271305,0.019942,0.347269}%
\pgfsetfillcolor{currentfill}%
\pgfsetfillopacity{0.700000}%
\pgfsetlinewidth{0.000000pt}%
\definecolor{currentstroke}{rgb}{0.000000,0.000000,0.000000}%
\pgfsetstrokecolor{currentstroke}%
\pgfsetdash{}{0pt}%
\pgfpathmoveto{\pgfqpoint{3.745452in}{2.574935in}}%
\pgfpathlineto{\pgfqpoint{3.758145in}{2.570709in}}%
\pgfpathlineto{\pgfqpoint{3.770843in}{2.566517in}}%
\pgfpathlineto{\pgfqpoint{3.783547in}{2.562359in}}%
\pgfpathlineto{\pgfqpoint{3.796255in}{2.558235in}}%
\pgfpathlineto{\pgfqpoint{3.788781in}{2.550900in}}%
\pgfpathlineto{\pgfqpoint{3.781301in}{2.543581in}}%
\pgfpathlineto{\pgfqpoint{3.773816in}{2.536279in}}%
\pgfpathlineto{\pgfqpoint{3.766326in}{2.528992in}}%
\pgfpathlineto{\pgfqpoint{3.753606in}{2.533107in}}%
\pgfpathlineto{\pgfqpoint{3.740890in}{2.537255in}}%
\pgfpathlineto{\pgfqpoint{3.728180in}{2.541437in}}%
\pgfpathlineto{\pgfqpoint{3.715475in}{2.545654in}}%
\pgfpathlineto{\pgfqpoint{3.722978in}{2.552945in}}%
\pgfpathlineto{\pgfqpoint{3.730475in}{2.560256in}}%
\pgfpathlineto{\pgfqpoint{3.737966in}{2.567585in}}%
\pgfpathlineto{\pgfqpoint{3.745452in}{2.574935in}}%
\pgfpathclose%
\pgfusepath{fill}%
\end{pgfscope}%
\begin{pgfscope}%
\pgfpathrectangle{\pgfqpoint{1.254980in}{0.150000in}}{\pgfqpoint{5.490039in}{5.490039in}}%
\pgfusepath{clip}%
\pgfsetbuttcap%
\pgfsetroundjoin%
\definecolor{currentfill}{rgb}{0.277018,0.050344,0.375715}%
\pgfsetfillcolor{currentfill}%
\pgfsetfillopacity{0.700000}%
\pgfsetlinewidth{0.000000pt}%
\definecolor{currentstroke}{rgb}{0.000000,0.000000,0.000000}%
\pgfsetstrokecolor{currentstroke}%
\pgfsetdash{}{0pt}%
\pgfpathmoveto{\pgfqpoint{4.644586in}{2.624100in}}%
\pgfpathlineto{\pgfqpoint{4.657476in}{2.620886in}}%
\pgfpathlineto{\pgfqpoint{4.670371in}{2.617699in}}%
\pgfpathlineto{\pgfqpoint{4.683273in}{2.614538in}}%
\pgfpathlineto{\pgfqpoint{4.696181in}{2.611402in}}%
\pgfpathlineto{\pgfqpoint{4.689028in}{2.604056in}}%
\pgfpathlineto{\pgfqpoint{4.681870in}{2.596758in}}%
\pgfpathlineto{\pgfqpoint{4.674708in}{2.589505in}}%
\pgfpathlineto{\pgfqpoint{4.667541in}{2.582295in}}%
\pgfpathlineto{\pgfqpoint{4.654619in}{2.585320in}}%
\pgfpathlineto{\pgfqpoint{4.641704in}{2.588371in}}%
\pgfpathlineto{\pgfqpoint{4.628795in}{2.591448in}}%
\pgfpathlineto{\pgfqpoint{4.615892in}{2.594552in}}%
\pgfpathlineto{\pgfqpoint{4.623073in}{2.601868in}}%
\pgfpathlineto{\pgfqpoint{4.630248in}{2.609229in}}%
\pgfpathlineto{\pgfqpoint{4.637419in}{2.616639in}}%
\pgfpathlineto{\pgfqpoint{4.644586in}{2.624100in}}%
\pgfpathclose%
\pgfusepath{fill}%
\end{pgfscope}%
\begin{pgfscope}%
\pgfpathrectangle{\pgfqpoint{1.254980in}{0.150000in}}{\pgfqpoint{5.490039in}{5.490039in}}%
\pgfusepath{clip}%
\pgfsetbuttcap%
\pgfsetroundjoin%
\definecolor{currentfill}{rgb}{0.274952,0.037752,0.364543}%
\pgfsetfillcolor{currentfill}%
\pgfsetfillopacity{0.700000}%
\pgfsetlinewidth{0.000000pt}%
\definecolor{currentstroke}{rgb}{0.000000,0.000000,0.000000}%
\pgfsetstrokecolor{currentstroke}%
\pgfsetdash{}{0pt}%
\pgfpathmoveto{\pgfqpoint{4.432612in}{2.603820in}}%
\pgfpathlineto{\pgfqpoint{4.445455in}{2.600491in}}%
\pgfpathlineto{\pgfqpoint{4.458304in}{2.597190in}}%
\pgfpathlineto{\pgfqpoint{4.471158in}{2.593917in}}%
\pgfpathlineto{\pgfqpoint{4.484019in}{2.590670in}}%
\pgfpathlineto{\pgfqpoint{4.476789in}{2.583349in}}%
\pgfpathlineto{\pgfqpoint{4.469555in}{2.576058in}}%
\pgfpathlineto{\pgfqpoint{4.462315in}{2.568797in}}%
\pgfpathlineto{\pgfqpoint{4.455071in}{2.561562in}}%
\pgfpathlineto{\pgfqpoint{4.442198in}{2.564723in}}%
\pgfpathlineto{\pgfqpoint{4.429330in}{2.567912in}}%
\pgfpathlineto{\pgfqpoint{4.416469in}{2.571128in}}%
\pgfpathlineto{\pgfqpoint{4.403614in}{2.574371in}}%
\pgfpathlineto{\pgfqpoint{4.410871in}{2.581687in}}%
\pgfpathlineto{\pgfqpoint{4.418123in}{2.589032in}}%
\pgfpathlineto{\pgfqpoint{4.425370in}{2.596408in}}%
\pgfpathlineto{\pgfqpoint{4.432612in}{2.603820in}}%
\pgfpathclose%
\pgfusepath{fill}%
\end{pgfscope}%
\begin{pgfscope}%
\pgfpathrectangle{\pgfqpoint{1.254980in}{0.150000in}}{\pgfqpoint{5.490039in}{5.490039in}}%
\pgfusepath{clip}%
\pgfsetbuttcap%
\pgfsetroundjoin%
\definecolor{currentfill}{rgb}{0.271305,0.019942,0.347269}%
\pgfsetfillcolor{currentfill}%
\pgfsetfillopacity{0.700000}%
\pgfsetlinewidth{0.000000pt}%
\definecolor{currentstroke}{rgb}{0.000000,0.000000,0.000000}%
\pgfsetstrokecolor{currentstroke}%
\pgfsetdash{}{0pt}%
\pgfpathmoveto{\pgfqpoint{3.876935in}{2.571513in}}%
\pgfpathlineto{\pgfqpoint{3.889658in}{2.567532in}}%
\pgfpathlineto{\pgfqpoint{3.902385in}{2.563583in}}%
\pgfpathlineto{\pgfqpoint{3.915118in}{2.559666in}}%
\pgfpathlineto{\pgfqpoint{3.927857in}{2.555781in}}%
\pgfpathlineto{\pgfqpoint{3.920428in}{2.548421in}}%
\pgfpathlineto{\pgfqpoint{3.912994in}{2.541076in}}%
\pgfpathlineto{\pgfqpoint{3.905554in}{2.533745in}}%
\pgfpathlineto{\pgfqpoint{3.898109in}{2.526426in}}%
\pgfpathlineto{\pgfqpoint{3.885359in}{2.530289in}}%
\pgfpathlineto{\pgfqpoint{3.872614in}{2.534184in}}%
\pgfpathlineto{\pgfqpoint{3.859874in}{2.538111in}}%
\pgfpathlineto{\pgfqpoint{3.847140in}{2.542070in}}%
\pgfpathlineto{\pgfqpoint{3.854597in}{2.549406in}}%
\pgfpathlineto{\pgfqpoint{3.862048in}{2.556758in}}%
\pgfpathlineto{\pgfqpoint{3.869494in}{2.564127in}}%
\pgfpathlineto{\pgfqpoint{3.876935in}{2.571513in}}%
\pgfpathclose%
\pgfusepath{fill}%
\end{pgfscope}%
\begin{pgfscope}%
\pgfpathrectangle{\pgfqpoint{1.254980in}{0.150000in}}{\pgfqpoint{5.490039in}{5.490039in}}%
\pgfusepath{clip}%
\pgfsetbuttcap%
\pgfsetroundjoin%
\definecolor{currentfill}{rgb}{0.273809,0.031497,0.358853}%
\pgfsetfillcolor{currentfill}%
\pgfsetfillopacity{0.700000}%
\pgfsetlinewidth{0.000000pt}%
\definecolor{currentstroke}{rgb}{0.000000,0.000000,0.000000}%
\pgfsetstrokecolor{currentstroke}%
\pgfsetdash{}{0pt}%
\pgfpathmoveto{\pgfqpoint{3.270131in}{2.593751in}}%
\pgfpathlineto{\pgfqpoint{3.282747in}{2.588420in}}%
\pgfpathlineto{\pgfqpoint{3.295367in}{2.583131in}}%
\pgfpathlineto{\pgfqpoint{3.307991in}{2.577885in}}%
\pgfpathlineto{\pgfqpoint{3.320619in}{2.572680in}}%
\pgfpathlineto{\pgfqpoint{3.312967in}{2.565819in}}%
\pgfpathlineto{\pgfqpoint{3.305308in}{2.558994in}}%
\pgfpathlineto{\pgfqpoint{3.297643in}{2.552206in}}%
\pgfpathlineto{\pgfqpoint{3.289972in}{2.545454in}}%
\pgfpathlineto{\pgfqpoint{3.277330in}{2.550700in}}%
\pgfpathlineto{\pgfqpoint{3.264693in}{2.555988in}}%
\pgfpathlineto{\pgfqpoint{3.252059in}{2.561318in}}%
\pgfpathlineto{\pgfqpoint{3.239430in}{2.566690in}}%
\pgfpathlineto{\pgfqpoint{3.247115in}{2.573395in}}%
\pgfpathlineto{\pgfqpoint{3.254793in}{2.580141in}}%
\pgfpathlineto{\pgfqpoint{3.262465in}{2.586926in}}%
\pgfpathlineto{\pgfqpoint{3.270131in}{2.593751in}}%
\pgfpathclose%
\pgfusepath{fill}%
\end{pgfscope}%
\begin{pgfscope}%
\pgfpathrectangle{\pgfqpoint{1.254980in}{0.150000in}}{\pgfqpoint{5.490039in}{5.490039in}}%
\pgfusepath{clip}%
\pgfsetbuttcap%
\pgfsetroundjoin%
\definecolor{currentfill}{rgb}{0.276022,0.044167,0.370164}%
\pgfsetfillcolor{currentfill}%
\pgfsetfillopacity{0.700000}%
\pgfsetlinewidth{0.000000pt}%
\definecolor{currentstroke}{rgb}{0.000000,0.000000,0.000000}%
\pgfsetstrokecolor{currentstroke}%
\pgfsetdash{}{0pt}%
\pgfpathmoveto{\pgfqpoint{3.138531in}{2.611249in}}%
\pgfpathlineto{\pgfqpoint{3.151130in}{2.605522in}}%
\pgfpathlineto{\pgfqpoint{3.163733in}{2.599840in}}%
\pgfpathlineto{\pgfqpoint{3.176340in}{2.594204in}}%
\pgfpathlineto{\pgfqpoint{3.188950in}{2.588613in}}%
\pgfpathlineto{\pgfqpoint{3.181245in}{2.581998in}}%
\pgfpathlineto{\pgfqpoint{3.173534in}{2.575429in}}%
\pgfpathlineto{\pgfqpoint{3.165816in}{2.568905in}}%
\pgfpathlineto{\pgfqpoint{3.158091in}{2.562429in}}%
\pgfpathlineto{\pgfqpoint{3.145466in}{2.568074in}}%
\pgfpathlineto{\pgfqpoint{3.132845in}{2.573764in}}%
\pgfpathlineto{\pgfqpoint{3.120228in}{2.579500in}}%
\pgfpathlineto{\pgfqpoint{3.107615in}{2.585282in}}%
\pgfpathlineto{\pgfqpoint{3.115354in}{2.591699in}}%
\pgfpathlineto{\pgfqpoint{3.123086in}{2.598167in}}%
\pgfpathlineto{\pgfqpoint{3.130812in}{2.604684in}}%
\pgfpathlineto{\pgfqpoint{3.138531in}{2.611249in}}%
\pgfpathclose%
\pgfusepath{fill}%
\end{pgfscope}%
\begin{pgfscope}%
\pgfpathrectangle{\pgfqpoint{1.254980in}{0.150000in}}{\pgfqpoint{5.490039in}{5.490039in}}%
\pgfusepath{clip}%
\pgfsetbuttcap%
\pgfsetroundjoin%
\definecolor{currentfill}{rgb}{0.272594,0.025563,0.353093}%
\pgfsetfillcolor{currentfill}%
\pgfsetfillopacity{0.700000}%
\pgfsetlinewidth{0.000000pt}%
\definecolor{currentstroke}{rgb}{0.000000,0.000000,0.000000}%
\pgfsetstrokecolor{currentstroke}%
\pgfsetdash{}{0pt}%
\pgfpathmoveto{\pgfqpoint{3.401670in}{2.580184in}}%
\pgfpathlineto{\pgfqpoint{3.414306in}{2.575212in}}%
\pgfpathlineto{\pgfqpoint{3.426947in}{2.570279in}}%
\pgfpathlineto{\pgfqpoint{3.439592in}{2.565386in}}%
\pgfpathlineto{\pgfqpoint{3.452241in}{2.560532in}}%
\pgfpathlineto{\pgfqpoint{3.444639in}{2.553482in}}%
\pgfpathlineto{\pgfqpoint{3.437030in}{2.546461in}}%
\pgfpathlineto{\pgfqpoint{3.429416in}{2.539469in}}%
\pgfpathlineto{\pgfqpoint{3.421795in}{2.532506in}}%
\pgfpathlineto{\pgfqpoint{3.409133in}{2.537389in}}%
\pgfpathlineto{\pgfqpoint{3.396476in}{2.542310in}}%
\pgfpathlineto{\pgfqpoint{3.383822in}{2.547272in}}%
\pgfpathlineto{\pgfqpoint{3.371173in}{2.552273in}}%
\pgfpathlineto{\pgfqpoint{3.378806in}{2.559203in}}%
\pgfpathlineto{\pgfqpoint{3.386434in}{2.566165in}}%
\pgfpathlineto{\pgfqpoint{3.394055in}{2.573159in}}%
\pgfpathlineto{\pgfqpoint{3.401670in}{2.580184in}}%
\pgfpathclose%
\pgfusepath{fill}%
\end{pgfscope}%
\begin{pgfscope}%
\pgfpathrectangle{\pgfqpoint{1.254980in}{0.150000in}}{\pgfqpoint{5.490039in}{5.490039in}}%
\pgfusepath{clip}%
\pgfsetbuttcap%
\pgfsetroundjoin%
\definecolor{currentfill}{rgb}{0.273809,0.031497,0.358853}%
\pgfsetfillcolor{currentfill}%
\pgfsetfillopacity{0.700000}%
\pgfsetlinewidth{0.000000pt}%
\definecolor{currentstroke}{rgb}{0.000000,0.000000,0.000000}%
\pgfsetstrokecolor{currentstroke}%
\pgfsetdash{}{0pt}%
\pgfpathmoveto{\pgfqpoint{4.220585in}{2.585582in}}%
\pgfpathlineto{\pgfqpoint{4.233383in}{2.582072in}}%
\pgfpathlineto{\pgfqpoint{4.246185in}{2.578591in}}%
\pgfpathlineto{\pgfqpoint{4.258994in}{2.575139in}}%
\pgfpathlineto{\pgfqpoint{4.271809in}{2.571715in}}%
\pgfpathlineto{\pgfqpoint{4.264502in}{2.564376in}}%
\pgfpathlineto{\pgfqpoint{4.257190in}{2.557057in}}%
\pgfpathlineto{\pgfqpoint{4.249873in}{2.549755in}}%
\pgfpathlineto{\pgfqpoint{4.242551in}{2.542470in}}%
\pgfpathlineto{\pgfqpoint{4.229724in}{2.545833in}}%
\pgfpathlineto{\pgfqpoint{4.216903in}{2.549226in}}%
\pgfpathlineto{\pgfqpoint{4.204089in}{2.552647in}}%
\pgfpathlineto{\pgfqpoint{4.191279in}{2.556097in}}%
\pgfpathlineto{\pgfqpoint{4.198614in}{2.563438in}}%
\pgfpathlineto{\pgfqpoint{4.205943in}{2.570798in}}%
\pgfpathlineto{\pgfqpoint{4.213267in}{2.578178in}}%
\pgfpathlineto{\pgfqpoint{4.220585in}{2.585582in}}%
\pgfpathclose%
\pgfusepath{fill}%
\end{pgfscope}%
\begin{pgfscope}%
\pgfpathrectangle{\pgfqpoint{1.254980in}{0.150000in}}{\pgfqpoint{5.490039in}{5.490039in}}%
\pgfusepath{clip}%
\pgfsetbuttcap%
\pgfsetroundjoin%
\definecolor{currentfill}{rgb}{0.280267,0.073417,0.397163}%
\pgfsetfillcolor{currentfill}%
\pgfsetfillopacity{0.700000}%
\pgfsetlinewidth{0.000000pt}%
\definecolor{currentstroke}{rgb}{0.000000,0.000000,0.000000}%
\pgfsetstrokecolor{currentstroke}%
\pgfsetdash{}{0pt}%
\pgfpathmoveto{\pgfqpoint{4.988423in}{2.651640in}}%
\pgfpathlineto{\pgfqpoint{5.001396in}{2.648530in}}%
\pgfpathlineto{\pgfqpoint{5.014376in}{2.645444in}}%
\pgfpathlineto{\pgfqpoint{5.027362in}{2.642383in}}%
\pgfpathlineto{\pgfqpoint{5.040354in}{2.639346in}}%
\pgfpathlineto{\pgfqpoint{5.033318in}{2.631859in}}%
\pgfpathlineto{\pgfqpoint{5.026279in}{2.624456in}}%
\pgfpathlineto{\pgfqpoint{5.019237in}{2.617132in}}%
\pgfpathlineto{\pgfqpoint{5.012191in}{2.609882in}}%
\pgfpathlineto{\pgfqpoint{4.999184in}{2.612771in}}%
\pgfpathlineto{\pgfqpoint{4.986183in}{2.615684in}}%
\pgfpathlineto{\pgfqpoint{4.973189in}{2.618622in}}%
\pgfpathlineto{\pgfqpoint{4.960201in}{2.621585in}}%
\pgfpathlineto{\pgfqpoint{4.967262in}{2.628978in}}%
\pgfpathlineto{\pgfqpoint{4.974319in}{2.636448in}}%
\pgfpathlineto{\pgfqpoint{4.981373in}{2.644001in}}%
\pgfpathlineto{\pgfqpoint{4.988423in}{2.651640in}}%
\pgfpathclose%
\pgfusepath{fill}%
\end{pgfscope}%
\begin{pgfscope}%
\pgfpathrectangle{\pgfqpoint{1.254980in}{0.150000in}}{\pgfqpoint{5.490039in}{5.490039in}}%
\pgfusepath{clip}%
\pgfsetbuttcap%
\pgfsetroundjoin%
\definecolor{currentfill}{rgb}{0.277941,0.056324,0.381191}%
\pgfsetfillcolor{currentfill}%
\pgfsetfillopacity{0.700000}%
\pgfsetlinewidth{0.000000pt}%
\definecolor{currentstroke}{rgb}{0.000000,0.000000,0.000000}%
\pgfsetstrokecolor{currentstroke}%
\pgfsetdash{}{0pt}%
\pgfpathmoveto{\pgfqpoint{3.006828in}{2.633242in}}%
\pgfpathlineto{\pgfqpoint{3.019415in}{2.627077in}}%
\pgfpathlineto{\pgfqpoint{3.032005in}{2.620961in}}%
\pgfpathlineto{\pgfqpoint{3.044598in}{2.614894in}}%
\pgfpathlineto{\pgfqpoint{3.057195in}{2.608876in}}%
\pgfpathlineto{\pgfqpoint{3.049434in}{2.602572in}}%
\pgfpathlineto{\pgfqpoint{3.041666in}{2.596323in}}%
\pgfpathlineto{\pgfqpoint{3.033892in}{2.590132in}}%
\pgfpathlineto{\pgfqpoint{3.026109in}{2.583998in}}%
\pgfpathlineto{\pgfqpoint{3.013498in}{2.590083in}}%
\pgfpathlineto{\pgfqpoint{3.000890in}{2.596217in}}%
\pgfpathlineto{\pgfqpoint{2.988285in}{2.602400in}}%
\pgfpathlineto{\pgfqpoint{2.975683in}{2.608633in}}%
\pgfpathlineto{\pgfqpoint{2.983480in}{2.614694in}}%
\pgfpathlineto{\pgfqpoint{2.991270in}{2.620817in}}%
\pgfpathlineto{\pgfqpoint{2.999053in}{2.627001in}}%
\pgfpathlineto{\pgfqpoint{3.006828in}{2.633242in}}%
\pgfpathclose%
\pgfusepath{fill}%
\end{pgfscope}%
\begin{pgfscope}%
\pgfpathrectangle{\pgfqpoint{1.254980in}{0.150000in}}{\pgfqpoint{5.490039in}{5.490039in}}%
\pgfusepath{clip}%
\pgfsetbuttcap%
\pgfsetroundjoin%
\definecolor{currentfill}{rgb}{0.277941,0.056324,0.381191}%
\pgfsetfillcolor{currentfill}%
\pgfsetfillopacity{0.700000}%
\pgfsetlinewidth{0.000000pt}%
\definecolor{currentstroke}{rgb}{0.000000,0.000000,0.000000}%
\pgfsetstrokecolor{currentstroke}%
\pgfsetdash{}{0pt}%
\pgfpathmoveto{\pgfqpoint{4.776392in}{2.628580in}}%
\pgfpathlineto{\pgfqpoint{4.789318in}{2.625451in}}%
\pgfpathlineto{\pgfqpoint{4.802250in}{2.622348in}}%
\pgfpathlineto{\pgfqpoint{4.815188in}{2.619270in}}%
\pgfpathlineto{\pgfqpoint{4.828133in}{2.616217in}}%
\pgfpathlineto{\pgfqpoint{4.821024in}{2.608892in}}%
\pgfpathlineto{\pgfqpoint{4.813911in}{2.601624in}}%
\pgfpathlineto{\pgfqpoint{4.806794in}{2.594411in}}%
\pgfpathlineto{\pgfqpoint{4.799672in}{2.587248in}}%
\pgfpathlineto{\pgfqpoint{4.786714in}{2.590178in}}%
\pgfpathlineto{\pgfqpoint{4.773762in}{2.593133in}}%
\pgfpathlineto{\pgfqpoint{4.760816in}{2.596114in}}%
\pgfpathlineto{\pgfqpoint{4.747876in}{2.599120in}}%
\pgfpathlineto{\pgfqpoint{4.755012in}{2.606401in}}%
\pgfpathlineto{\pgfqpoint{4.762143in}{2.613736in}}%
\pgfpathlineto{\pgfqpoint{4.769270in}{2.621128in}}%
\pgfpathlineto{\pgfqpoint{4.776392in}{2.628580in}}%
\pgfpathclose%
\pgfusepath{fill}%
\end{pgfscope}%
\begin{pgfscope}%
\pgfpathrectangle{\pgfqpoint{1.254980in}{0.150000in}}{\pgfqpoint{5.490039in}{5.490039in}}%
\pgfusepath{clip}%
\pgfsetbuttcap%
\pgfsetroundjoin%
\definecolor{currentfill}{rgb}{0.271305,0.019942,0.347269}%
\pgfsetfillcolor{currentfill}%
\pgfsetfillopacity{0.700000}%
\pgfsetlinewidth{0.000000pt}%
\definecolor{currentstroke}{rgb}{0.000000,0.000000,0.000000}%
\pgfsetstrokecolor{currentstroke}%
\pgfsetdash{}{0pt}%
\pgfpathmoveto{\pgfqpoint{3.533185in}{2.570042in}}%
\pgfpathlineto{\pgfqpoint{3.545844in}{2.565394in}}%
\pgfpathlineto{\pgfqpoint{3.558508in}{2.560784in}}%
\pgfpathlineto{\pgfqpoint{3.571177in}{2.556210in}}%
\pgfpathlineto{\pgfqpoint{3.583850in}{2.551672in}}%
\pgfpathlineto{\pgfqpoint{3.576296in}{2.544487in}}%
\pgfpathlineto{\pgfqpoint{3.568736in}{2.537324in}}%
\pgfpathlineto{\pgfqpoint{3.561170in}{2.530183in}}%
\pgfpathlineto{\pgfqpoint{3.553598in}{2.523064in}}%
\pgfpathlineto{\pgfqpoint{3.540912in}{2.527618in}}%
\pgfpathlineto{\pgfqpoint{3.528231in}{2.532207in}}%
\pgfpathlineto{\pgfqpoint{3.515555in}{2.536834in}}%
\pgfpathlineto{\pgfqpoint{3.502883in}{2.541498in}}%
\pgfpathlineto{\pgfqpoint{3.510467in}{2.548597in}}%
\pgfpathlineto{\pgfqpoint{3.518045in}{2.555720in}}%
\pgfpathlineto{\pgfqpoint{3.525618in}{2.562869in}}%
\pgfpathlineto{\pgfqpoint{3.533185in}{2.570042in}}%
\pgfpathclose%
\pgfusepath{fill}%
\end{pgfscope}%
\begin{pgfscope}%
\pgfpathrectangle{\pgfqpoint{1.254980in}{0.150000in}}{\pgfqpoint{5.490039in}{5.490039in}}%
\pgfusepath{clip}%
\pgfsetbuttcap%
\pgfsetroundjoin%
\definecolor{currentfill}{rgb}{0.276022,0.044167,0.370164}%
\pgfsetfillcolor{currentfill}%
\pgfsetfillopacity{0.700000}%
\pgfsetlinewidth{0.000000pt}%
\definecolor{currentstroke}{rgb}{0.000000,0.000000,0.000000}%
\pgfsetstrokecolor{currentstroke}%
\pgfsetdash{}{0pt}%
\pgfpathmoveto{\pgfqpoint{4.564343in}{2.607230in}}%
\pgfpathlineto{\pgfqpoint{4.577221in}{2.604021in}}%
\pgfpathlineto{\pgfqpoint{4.590105in}{2.600838in}}%
\pgfpathlineto{\pgfqpoint{4.602996in}{2.597682in}}%
\pgfpathlineto{\pgfqpoint{4.615892in}{2.594552in}}%
\pgfpathlineto{\pgfqpoint{4.608707in}{2.587277in}}%
\pgfpathlineto{\pgfqpoint{4.601518in}{2.580040in}}%
\pgfpathlineto{\pgfqpoint{4.594324in}{2.572838in}}%
\pgfpathlineto{\pgfqpoint{4.587124in}{2.565667in}}%
\pgfpathlineto{\pgfqpoint{4.574215in}{2.568699in}}%
\pgfpathlineto{\pgfqpoint{4.561311in}{2.571758in}}%
\pgfpathlineto{\pgfqpoint{4.548414in}{2.574843in}}%
\pgfpathlineto{\pgfqpoint{4.535523in}{2.577955in}}%
\pgfpathlineto{\pgfqpoint{4.542735in}{2.585219in}}%
\pgfpathlineto{\pgfqpoint{4.549942in}{2.592517in}}%
\pgfpathlineto{\pgfqpoint{4.557145in}{2.599853in}}%
\pgfpathlineto{\pgfqpoint{4.564343in}{2.607230in}}%
\pgfpathclose%
\pgfusepath{fill}%
\end{pgfscope}%
\begin{pgfscope}%
\pgfpathrectangle{\pgfqpoint{1.254980in}{0.150000in}}{\pgfqpoint{5.490039in}{5.490039in}}%
\pgfusepath{clip}%
\pgfsetbuttcap%
\pgfsetroundjoin%
\definecolor{currentfill}{rgb}{0.271305,0.019942,0.347269}%
\pgfsetfillcolor{currentfill}%
\pgfsetfillopacity{0.700000}%
\pgfsetlinewidth{0.000000pt}%
\definecolor{currentstroke}{rgb}{0.000000,0.000000,0.000000}%
\pgfsetstrokecolor{currentstroke}%
\pgfsetdash{}{0pt}%
\pgfpathmoveto{\pgfqpoint{4.008480in}{2.570034in}}%
\pgfpathlineto{\pgfqpoint{4.021233in}{2.566271in}}%
\pgfpathlineto{\pgfqpoint{4.033992in}{2.562540in}}%
\pgfpathlineto{\pgfqpoint{4.046757in}{2.558839in}}%
\pgfpathlineto{\pgfqpoint{4.059527in}{2.555168in}}%
\pgfpathlineto{\pgfqpoint{4.052143in}{2.547814in}}%
\pgfpathlineto{\pgfqpoint{4.044754in}{2.540474in}}%
\pgfpathlineto{\pgfqpoint{4.037359in}{2.533147in}}%
\pgfpathlineto{\pgfqpoint{4.029959in}{2.525830in}}%
\pgfpathlineto{\pgfqpoint{4.017177in}{2.529466in}}%
\pgfpathlineto{\pgfqpoint{4.004400in}{2.533133in}}%
\pgfpathlineto{\pgfqpoint{3.991630in}{2.536830in}}%
\pgfpathlineto{\pgfqpoint{3.978864in}{2.540558in}}%
\pgfpathlineto{\pgfqpoint{3.986276in}{2.547904in}}%
\pgfpathlineto{\pgfqpoint{3.993682in}{2.555264in}}%
\pgfpathlineto{\pgfqpoint{4.001084in}{2.562640in}}%
\pgfpathlineto{\pgfqpoint{4.008480in}{2.570034in}}%
\pgfpathclose%
\pgfusepath{fill}%
\end{pgfscope}%
\begin{pgfscope}%
\pgfpathrectangle{\pgfqpoint{1.254980in}{0.150000in}}{\pgfqpoint{5.490039in}{5.490039in}}%
\pgfusepath{clip}%
\pgfsetbuttcap%
\pgfsetroundjoin%
\definecolor{currentfill}{rgb}{0.269944,0.014625,0.341379}%
\pgfsetfillcolor{currentfill}%
\pgfsetfillopacity{0.700000}%
\pgfsetlinewidth{0.000000pt}%
\definecolor{currentstroke}{rgb}{0.000000,0.000000,0.000000}%
\pgfsetstrokecolor{currentstroke}%
\pgfsetdash{}{0pt}%
\pgfpathmoveto{\pgfqpoint{3.664705in}{2.562864in}}%
\pgfpathlineto{\pgfqpoint{3.677390in}{2.558509in}}%
\pgfpathlineto{\pgfqpoint{3.690080in}{2.554189in}}%
\pgfpathlineto{\pgfqpoint{3.702775in}{2.549904in}}%
\pgfpathlineto{\pgfqpoint{3.715475in}{2.545654in}}%
\pgfpathlineto{\pgfqpoint{3.707967in}{2.538380in}}%
\pgfpathlineto{\pgfqpoint{3.700454in}{2.531123in}}%
\pgfpathlineto{\pgfqpoint{3.692935in}{2.523883in}}%
\pgfpathlineto{\pgfqpoint{3.685410in}{2.516660in}}%
\pgfpathlineto{\pgfqpoint{3.672698in}{2.520914in}}%
\pgfpathlineto{\pgfqpoint{3.659991in}{2.525202in}}%
\pgfpathlineto{\pgfqpoint{3.647289in}{2.529525in}}%
\pgfpathlineto{\pgfqpoint{3.634591in}{2.533884in}}%
\pgfpathlineto{\pgfqpoint{3.642128in}{2.541099in}}%
\pgfpathlineto{\pgfqpoint{3.649659in}{2.548334in}}%
\pgfpathlineto{\pgfqpoint{3.657185in}{2.555589in}}%
\pgfpathlineto{\pgfqpoint{3.664705in}{2.562864in}}%
\pgfpathclose%
\pgfusepath{fill}%
\end{pgfscope}%
\begin{pgfscope}%
\pgfpathrectangle{\pgfqpoint{1.254980in}{0.150000in}}{\pgfqpoint{5.490039in}{5.490039in}}%
\pgfusepath{clip}%
\pgfsetbuttcap%
\pgfsetroundjoin%
\definecolor{currentfill}{rgb}{0.273809,0.031497,0.358853}%
\pgfsetfillcolor{currentfill}%
\pgfsetfillopacity{0.700000}%
\pgfsetlinewidth{0.000000pt}%
\definecolor{currentstroke}{rgb}{0.000000,0.000000,0.000000}%
\pgfsetstrokecolor{currentstroke}%
\pgfsetdash{}{0pt}%
\pgfpathmoveto{\pgfqpoint{4.352252in}{2.587622in}}%
\pgfpathlineto{\pgfqpoint{4.365084in}{2.584267in}}%
\pgfpathlineto{\pgfqpoint{4.377921in}{2.580941in}}%
\pgfpathlineto{\pgfqpoint{4.390764in}{2.577642in}}%
\pgfpathlineto{\pgfqpoint{4.403614in}{2.574371in}}%
\pgfpathlineto{\pgfqpoint{4.396352in}{2.567081in}}%
\pgfpathlineto{\pgfqpoint{4.389085in}{2.559816in}}%
\pgfpathlineto{\pgfqpoint{4.381813in}{2.552571in}}%
\pgfpathlineto{\pgfqpoint{4.374535in}{2.545345in}}%
\pgfpathlineto{\pgfqpoint{4.361674in}{2.548543in}}%
\pgfpathlineto{\pgfqpoint{4.348818in}{2.551769in}}%
\pgfpathlineto{\pgfqpoint{4.335968in}{2.555023in}}%
\pgfpathlineto{\pgfqpoint{4.323125in}{2.558305in}}%
\pgfpathlineto{\pgfqpoint{4.330414in}{2.565599in}}%
\pgfpathlineto{\pgfqpoint{4.337699in}{2.572915in}}%
\pgfpathlineto{\pgfqpoint{4.344978in}{2.580255in}}%
\pgfpathlineto{\pgfqpoint{4.352252in}{2.587622in}}%
\pgfpathclose%
\pgfusepath{fill}%
\end{pgfscope}%
\begin{pgfscope}%
\pgfpathrectangle{\pgfqpoint{1.254980in}{0.150000in}}{\pgfqpoint{5.490039in}{5.490039in}}%
\pgfusepath{clip}%
\pgfsetbuttcap%
\pgfsetroundjoin%
\definecolor{currentfill}{rgb}{0.280894,0.078907,0.402329}%
\pgfsetfillcolor{currentfill}%
\pgfsetfillopacity{0.700000}%
\pgfsetlinewidth{0.000000pt}%
\definecolor{currentstroke}{rgb}{0.000000,0.000000,0.000000}%
\pgfsetstrokecolor{currentstroke}%
\pgfsetdash{}{0pt}%
\pgfpathmoveto{\pgfqpoint{5.120439in}{2.657687in}}%
\pgfpathlineto{\pgfqpoint{5.133449in}{2.654612in}}%
\pgfpathlineto{\pgfqpoint{5.146465in}{2.651562in}}%
\pgfpathlineto{\pgfqpoint{5.159487in}{2.648535in}}%
\pgfpathlineto{\pgfqpoint{5.172516in}{2.645532in}}%
\pgfpathlineto{\pgfqpoint{5.165523in}{2.637990in}}%
\pgfpathlineto{\pgfqpoint{5.158527in}{2.630546in}}%
\pgfpathlineto{\pgfqpoint{5.151528in}{2.623194in}}%
\pgfpathlineto{\pgfqpoint{5.144527in}{2.615930in}}%
\pgfpathlineto{\pgfqpoint{5.131482in}{2.618772in}}%
\pgfpathlineto{\pgfqpoint{5.118444in}{2.621639in}}%
\pgfpathlineto{\pgfqpoint{5.105413in}{2.624529in}}%
\pgfpathlineto{\pgfqpoint{5.092388in}{2.627444in}}%
\pgfpathlineto{\pgfqpoint{5.099405in}{2.634864in}}%
\pgfpathlineto{\pgfqpoint{5.106419in}{2.642375in}}%
\pgfpathlineto{\pgfqpoint{5.113431in}{2.649981in}}%
\pgfpathlineto{\pgfqpoint{5.120439in}{2.657687in}}%
\pgfpathclose%
\pgfusepath{fill}%
\end{pgfscope}%
\begin{pgfscope}%
\pgfpathrectangle{\pgfqpoint{1.254980in}{0.150000in}}{\pgfqpoint{5.490039in}{5.490039in}}%
\pgfusepath{clip}%
\pgfsetbuttcap%
\pgfsetroundjoin%
\definecolor{currentfill}{rgb}{0.279566,0.067836,0.391917}%
\pgfsetfillcolor{currentfill}%
\pgfsetfillopacity{0.700000}%
\pgfsetlinewidth{0.000000pt}%
\definecolor{currentstroke}{rgb}{0.000000,0.000000,0.000000}%
\pgfsetstrokecolor{currentstroke}%
\pgfsetdash{}{0pt}%
\pgfpathmoveto{\pgfqpoint{4.908314in}{2.633684in}}%
\pgfpathlineto{\pgfqpoint{4.921276in}{2.630622in}}%
\pgfpathlineto{\pgfqpoint{4.934245in}{2.627585in}}%
\pgfpathlineto{\pgfqpoint{4.947219in}{2.624572in}}%
\pgfpathlineto{\pgfqpoint{4.960201in}{2.621585in}}%
\pgfpathlineto{\pgfqpoint{4.953136in}{2.614265in}}%
\pgfpathlineto{\pgfqpoint{4.946068in}{2.607015in}}%
\pgfpathlineto{\pgfqpoint{4.938996in}{2.599828in}}%
\pgfpathlineto{\pgfqpoint{4.931920in}{2.592702in}}%
\pgfpathlineto{\pgfqpoint{4.918924in}{2.595554in}}%
\pgfpathlineto{\pgfqpoint{4.905934in}{2.598431in}}%
\pgfpathlineto{\pgfqpoint{4.892951in}{2.601333in}}%
\pgfpathlineto{\pgfqpoint{4.879975in}{2.604259in}}%
\pgfpathlineto{\pgfqpoint{4.887066in}{2.611516in}}%
\pgfpathlineto{\pgfqpoint{4.894152in}{2.618837in}}%
\pgfpathlineto{\pgfqpoint{4.901235in}{2.626224in}}%
\pgfpathlineto{\pgfqpoint{4.908314in}{2.633684in}}%
\pgfpathclose%
\pgfusepath{fill}%
\end{pgfscope}%
\begin{pgfscope}%
\pgfpathrectangle{\pgfqpoint{1.254980in}{0.150000in}}{\pgfqpoint{5.490039in}{5.490039in}}%
\pgfusepath{clip}%
\pgfsetbuttcap%
\pgfsetroundjoin%
\definecolor{currentfill}{rgb}{0.269944,0.014625,0.341379}%
\pgfsetfillcolor{currentfill}%
\pgfsetfillopacity{0.700000}%
\pgfsetlinewidth{0.000000pt}%
\definecolor{currentstroke}{rgb}{0.000000,0.000000,0.000000}%
\pgfsetstrokecolor{currentstroke}%
\pgfsetdash{}{0pt}%
\pgfpathmoveto{\pgfqpoint{3.796255in}{2.558235in}}%
\pgfpathlineto{\pgfqpoint{3.808968in}{2.554145in}}%
\pgfpathlineto{\pgfqpoint{3.821687in}{2.550087in}}%
\pgfpathlineto{\pgfqpoint{3.834411in}{2.546062in}}%
\pgfpathlineto{\pgfqpoint{3.847140in}{2.542070in}}%
\pgfpathlineto{\pgfqpoint{3.839678in}{2.534749in}}%
\pgfpathlineto{\pgfqpoint{3.832210in}{2.527441in}}%
\pgfpathlineto{\pgfqpoint{3.824737in}{2.520147in}}%
\pgfpathlineto{\pgfqpoint{3.817258in}{2.512865in}}%
\pgfpathlineto{\pgfqpoint{3.804517in}{2.516847in}}%
\pgfpathlineto{\pgfqpoint{3.791782in}{2.520862in}}%
\pgfpathlineto{\pgfqpoint{3.779051in}{2.524910in}}%
\pgfpathlineto{\pgfqpoint{3.766326in}{2.528992in}}%
\pgfpathlineto{\pgfqpoint{3.773816in}{2.536279in}}%
\pgfpathlineto{\pgfqpoint{3.781301in}{2.543581in}}%
\pgfpathlineto{\pgfqpoint{3.788781in}{2.550900in}}%
\pgfpathlineto{\pgfqpoint{3.796255in}{2.558235in}}%
\pgfpathclose%
\pgfusepath{fill}%
\end{pgfscope}%
\begin{pgfscope}%
\pgfpathrectangle{\pgfqpoint{1.254980in}{0.150000in}}{\pgfqpoint{5.490039in}{5.490039in}}%
\pgfusepath{clip}%
\pgfsetbuttcap%
\pgfsetroundjoin%
\definecolor{currentfill}{rgb}{0.272594,0.025563,0.353093}%
\pgfsetfillcolor{currentfill}%
\pgfsetfillopacity{0.700000}%
\pgfsetlinewidth{0.000000pt}%
\definecolor{currentstroke}{rgb}{0.000000,0.000000,0.000000}%
\pgfsetstrokecolor{currentstroke}%
\pgfsetdash{}{0pt}%
\pgfpathmoveto{\pgfqpoint{4.140100in}{2.570191in}}%
\pgfpathlineto{\pgfqpoint{4.152887in}{2.566623in}}%
\pgfpathlineto{\pgfqpoint{4.165678in}{2.563085in}}%
\pgfpathlineto{\pgfqpoint{4.178476in}{2.559577in}}%
\pgfpathlineto{\pgfqpoint{4.191279in}{2.556097in}}%
\pgfpathlineto{\pgfqpoint{4.183940in}{2.548773in}}%
\pgfpathlineto{\pgfqpoint{4.176595in}{2.541464in}}%
\pgfpathlineto{\pgfqpoint{4.169245in}{2.534168in}}%
\pgfpathlineto{\pgfqpoint{4.161890in}{2.526884in}}%
\pgfpathlineto{\pgfqpoint{4.149075in}{2.530316in}}%
\pgfpathlineto{\pgfqpoint{4.136265in}{2.533777in}}%
\pgfpathlineto{\pgfqpoint{4.123462in}{2.537268in}}%
\pgfpathlineto{\pgfqpoint{4.110663in}{2.540788in}}%
\pgfpathlineto{\pgfqpoint{4.118030in}{2.548115in}}%
\pgfpathlineto{\pgfqpoint{4.125392in}{2.555457in}}%
\pgfpathlineto{\pgfqpoint{4.132749in}{2.562815in}}%
\pgfpathlineto{\pgfqpoint{4.140100in}{2.570191in}}%
\pgfpathclose%
\pgfusepath{fill}%
\end{pgfscope}%
\begin{pgfscope}%
\pgfpathrectangle{\pgfqpoint{1.254980in}{0.150000in}}{\pgfqpoint{5.490039in}{5.490039in}}%
\pgfusepath{clip}%
\pgfsetbuttcap%
\pgfsetroundjoin%
\definecolor{currentfill}{rgb}{0.277018,0.050344,0.375715}%
\pgfsetfillcolor{currentfill}%
\pgfsetfillopacity{0.700000}%
\pgfsetlinewidth{0.000000pt}%
\definecolor{currentstroke}{rgb}{0.000000,0.000000,0.000000}%
\pgfsetstrokecolor{currentstroke}%
\pgfsetdash{}{0pt}%
\pgfpathmoveto{\pgfqpoint{4.696181in}{2.611402in}}%
\pgfpathlineto{\pgfqpoint{4.709096in}{2.608293in}}%
\pgfpathlineto{\pgfqpoint{4.722016in}{2.605210in}}%
\pgfpathlineto{\pgfqpoint{4.734943in}{2.602152in}}%
\pgfpathlineto{\pgfqpoint{4.747876in}{2.599120in}}%
\pgfpathlineto{\pgfqpoint{4.740737in}{2.591888in}}%
\pgfpathlineto{\pgfqpoint{4.733592in}{2.584702in}}%
\pgfpathlineto{\pgfqpoint{4.726443in}{2.577559in}}%
\pgfpathlineto{\pgfqpoint{4.719290in}{2.570454in}}%
\pgfpathlineto{\pgfqpoint{4.706343in}{2.573375in}}%
\pgfpathlineto{\pgfqpoint{4.693403in}{2.576323in}}%
\pgfpathlineto{\pgfqpoint{4.680469in}{2.579296in}}%
\pgfpathlineto{\pgfqpoint{4.667541in}{2.582295in}}%
\pgfpathlineto{\pgfqpoint{4.674708in}{2.589505in}}%
\pgfpathlineto{\pgfqpoint{4.681870in}{2.596758in}}%
\pgfpathlineto{\pgfqpoint{4.689028in}{2.604056in}}%
\pgfpathlineto{\pgfqpoint{4.696181in}{2.611402in}}%
\pgfpathclose%
\pgfusepath{fill}%
\end{pgfscope}%
\begin{pgfscope}%
\pgfpathrectangle{\pgfqpoint{1.254980in}{0.150000in}}{\pgfqpoint{5.490039in}{5.490039in}}%
\pgfusepath{clip}%
\pgfsetbuttcap%
\pgfsetroundjoin%
\definecolor{currentfill}{rgb}{0.274952,0.037752,0.364543}%
\pgfsetfillcolor{currentfill}%
\pgfsetfillopacity{0.700000}%
\pgfsetlinewidth{0.000000pt}%
\definecolor{currentstroke}{rgb}{0.000000,0.000000,0.000000}%
\pgfsetstrokecolor{currentstroke}%
\pgfsetdash{}{0pt}%
\pgfpathmoveto{\pgfqpoint{3.188950in}{2.588613in}}%
\pgfpathlineto{\pgfqpoint{3.201565in}{2.583066in}}%
\pgfpathlineto{\pgfqpoint{3.214183in}{2.577564in}}%
\pgfpathlineto{\pgfqpoint{3.226804in}{2.572105in}}%
\pgfpathlineto{\pgfqpoint{3.239430in}{2.566690in}}%
\pgfpathlineto{\pgfqpoint{3.231739in}{2.560026in}}%
\pgfpathlineto{\pgfqpoint{3.224041in}{2.553404in}}%
\pgfpathlineto{\pgfqpoint{3.216337in}{2.546825in}}%
\pgfpathlineto{\pgfqpoint{3.208626in}{2.540290in}}%
\pgfpathlineto{\pgfqpoint{3.195986in}{2.545759in}}%
\pgfpathlineto{\pgfqpoint{3.183351in}{2.551272in}}%
\pgfpathlineto{\pgfqpoint{3.170719in}{2.556828in}}%
\pgfpathlineto{\pgfqpoint{3.158091in}{2.562429in}}%
\pgfpathlineto{\pgfqpoint{3.165816in}{2.568905in}}%
\pgfpathlineto{\pgfqpoint{3.173534in}{2.575429in}}%
\pgfpathlineto{\pgfqpoint{3.181245in}{2.581998in}}%
\pgfpathlineto{\pgfqpoint{3.188950in}{2.588613in}}%
\pgfpathclose%
\pgfusepath{fill}%
\end{pgfscope}%
\begin{pgfscope}%
\pgfpathrectangle{\pgfqpoint{1.254980in}{0.150000in}}{\pgfqpoint{5.490039in}{5.490039in}}%
\pgfusepath{clip}%
\pgfsetbuttcap%
\pgfsetroundjoin%
\definecolor{currentfill}{rgb}{0.272594,0.025563,0.353093}%
\pgfsetfillcolor{currentfill}%
\pgfsetfillopacity{0.700000}%
\pgfsetlinewidth{0.000000pt}%
\definecolor{currentstroke}{rgb}{0.000000,0.000000,0.000000}%
\pgfsetstrokecolor{currentstroke}%
\pgfsetdash{}{0pt}%
\pgfpathmoveto{\pgfqpoint{3.320619in}{2.572680in}}%
\pgfpathlineto{\pgfqpoint{3.333251in}{2.567517in}}%
\pgfpathlineto{\pgfqpoint{3.345888in}{2.562395in}}%
\pgfpathlineto{\pgfqpoint{3.358528in}{2.557314in}}%
\pgfpathlineto{\pgfqpoint{3.371173in}{2.552273in}}%
\pgfpathlineto{\pgfqpoint{3.363534in}{2.545375in}}%
\pgfpathlineto{\pgfqpoint{3.355888in}{2.538510in}}%
\pgfpathlineto{\pgfqpoint{3.348237in}{2.531679in}}%
\pgfpathlineto{\pgfqpoint{3.340579in}{2.524882in}}%
\pgfpathlineto{\pgfqpoint{3.327921in}{2.529964in}}%
\pgfpathlineto{\pgfqpoint{3.315267in}{2.535087in}}%
\pgfpathlineto{\pgfqpoint{3.302617in}{2.540250in}}%
\pgfpathlineto{\pgfqpoint{3.289972in}{2.545454in}}%
\pgfpathlineto{\pgfqpoint{3.297643in}{2.552206in}}%
\pgfpathlineto{\pgfqpoint{3.305308in}{2.558994in}}%
\pgfpathlineto{\pgfqpoint{3.312967in}{2.565819in}}%
\pgfpathlineto{\pgfqpoint{3.320619in}{2.572680in}}%
\pgfpathclose%
\pgfusepath{fill}%
\end{pgfscope}%
\begin{pgfscope}%
\pgfpathrectangle{\pgfqpoint{1.254980in}{0.150000in}}{\pgfqpoint{5.490039in}{5.490039in}}%
\pgfusepath{clip}%
\pgfsetbuttcap%
\pgfsetroundjoin%
\definecolor{currentfill}{rgb}{0.274952,0.037752,0.364543}%
\pgfsetfillcolor{currentfill}%
\pgfsetfillopacity{0.700000}%
\pgfsetlinewidth{0.000000pt}%
\definecolor{currentstroke}{rgb}{0.000000,0.000000,0.000000}%
\pgfsetstrokecolor{currentstroke}%
\pgfsetdash{}{0pt}%
\pgfpathmoveto{\pgfqpoint{4.484019in}{2.590670in}}%
\pgfpathlineto{\pgfqpoint{4.496886in}{2.587451in}}%
\pgfpathlineto{\pgfqpoint{4.509759in}{2.584259in}}%
\pgfpathlineto{\pgfqpoint{4.522638in}{2.581093in}}%
\pgfpathlineto{\pgfqpoint{4.535523in}{2.577955in}}%
\pgfpathlineto{\pgfqpoint{4.528306in}{2.570723in}}%
\pgfpathlineto{\pgfqpoint{4.521084in}{2.563520in}}%
\pgfpathlineto{\pgfqpoint{4.513857in}{2.556342in}}%
\pgfpathlineto{\pgfqpoint{4.506625in}{2.549187in}}%
\pgfpathlineto{\pgfqpoint{4.493728in}{2.552240in}}%
\pgfpathlineto{\pgfqpoint{4.480836in}{2.555320in}}%
\pgfpathlineto{\pgfqpoint{4.467950in}{2.558427in}}%
\pgfpathlineto{\pgfqpoint{4.455071in}{2.561562in}}%
\pgfpathlineto{\pgfqpoint{4.462315in}{2.568797in}}%
\pgfpathlineto{\pgfqpoint{4.469555in}{2.576058in}}%
\pgfpathlineto{\pgfqpoint{4.476789in}{2.583349in}}%
\pgfpathlineto{\pgfqpoint{4.484019in}{2.590670in}}%
\pgfpathclose%
\pgfusepath{fill}%
\end{pgfscope}%
\begin{pgfscope}%
\pgfpathrectangle{\pgfqpoint{1.254980in}{0.150000in}}{\pgfqpoint{5.490039in}{5.490039in}}%
\pgfusepath{clip}%
\pgfsetbuttcap%
\pgfsetroundjoin%
\definecolor{currentfill}{rgb}{0.271305,0.019942,0.347269}%
\pgfsetfillcolor{currentfill}%
\pgfsetfillopacity{0.700000}%
\pgfsetlinewidth{0.000000pt}%
\definecolor{currentstroke}{rgb}{0.000000,0.000000,0.000000}%
\pgfsetstrokecolor{currentstroke}%
\pgfsetdash{}{0pt}%
\pgfpathmoveto{\pgfqpoint{3.452241in}{2.560532in}}%
\pgfpathlineto{\pgfqpoint{3.464895in}{2.555716in}}%
\pgfpathlineto{\pgfqpoint{3.477553in}{2.550939in}}%
\pgfpathlineto{\pgfqpoint{3.490216in}{2.546200in}}%
\pgfpathlineto{\pgfqpoint{3.502883in}{2.541498in}}%
\pgfpathlineto{\pgfqpoint{3.495293in}{2.534425in}}%
\pgfpathlineto{\pgfqpoint{3.487697in}{2.527377in}}%
\pgfpathlineto{\pgfqpoint{3.480095in}{2.520355in}}%
\pgfpathlineto{\pgfqpoint{3.472488in}{2.513358in}}%
\pgfpathlineto{\pgfqpoint{3.459808in}{2.518088in}}%
\pgfpathlineto{\pgfqpoint{3.447133in}{2.522856in}}%
\pgfpathlineto{\pgfqpoint{3.434462in}{2.527661in}}%
\pgfpathlineto{\pgfqpoint{3.421795in}{2.532506in}}%
\pgfpathlineto{\pgfqpoint{3.429416in}{2.539469in}}%
\pgfpathlineto{\pgfqpoint{3.437030in}{2.546461in}}%
\pgfpathlineto{\pgfqpoint{3.444639in}{2.553482in}}%
\pgfpathlineto{\pgfqpoint{3.452241in}{2.560532in}}%
\pgfpathclose%
\pgfusepath{fill}%
\end{pgfscope}%
\begin{pgfscope}%
\pgfpathrectangle{\pgfqpoint{1.254980in}{0.150000in}}{\pgfqpoint{5.490039in}{5.490039in}}%
\pgfusepath{clip}%
\pgfsetbuttcap%
\pgfsetroundjoin%
\definecolor{currentfill}{rgb}{0.277018,0.050344,0.375715}%
\pgfsetfillcolor{currentfill}%
\pgfsetfillopacity{0.700000}%
\pgfsetlinewidth{0.000000pt}%
\definecolor{currentstroke}{rgb}{0.000000,0.000000,0.000000}%
\pgfsetstrokecolor{currentstroke}%
\pgfsetdash{}{0pt}%
\pgfpathmoveto{\pgfqpoint{3.057195in}{2.608876in}}%
\pgfpathlineto{\pgfqpoint{3.069795in}{2.602906in}}%
\pgfpathlineto{\pgfqpoint{3.082398in}{2.596984in}}%
\pgfpathlineto{\pgfqpoint{3.095005in}{2.591110in}}%
\pgfpathlineto{\pgfqpoint{3.107615in}{2.585282in}}%
\pgfpathlineto{\pgfqpoint{3.099869in}{2.578915in}}%
\pgfpathlineto{\pgfqpoint{3.092116in}{2.572601in}}%
\pgfpathlineto{\pgfqpoint{3.084356in}{2.566341in}}%
\pgfpathlineto{\pgfqpoint{3.076588in}{2.560136in}}%
\pgfpathlineto{\pgfqpoint{3.063964in}{2.566031in}}%
\pgfpathlineto{\pgfqpoint{3.051342in}{2.571972in}}%
\pgfpathlineto{\pgfqpoint{3.038724in}{2.577961in}}%
\pgfpathlineto{\pgfqpoint{3.026109in}{2.583998in}}%
\pgfpathlineto{\pgfqpoint{3.033892in}{2.590132in}}%
\pgfpathlineto{\pgfqpoint{3.041666in}{2.596323in}}%
\pgfpathlineto{\pgfqpoint{3.049434in}{2.602572in}}%
\pgfpathlineto{\pgfqpoint{3.057195in}{2.608876in}}%
\pgfpathclose%
\pgfusepath{fill}%
\end{pgfscope}%
\begin{pgfscope}%
\pgfpathrectangle{\pgfqpoint{1.254980in}{0.150000in}}{\pgfqpoint{5.490039in}{5.490039in}}%
\pgfusepath{clip}%
\pgfsetbuttcap%
\pgfsetroundjoin%
\definecolor{currentfill}{rgb}{0.271305,0.019942,0.347269}%
\pgfsetfillcolor{currentfill}%
\pgfsetfillopacity{0.700000}%
\pgfsetlinewidth{0.000000pt}%
\definecolor{currentstroke}{rgb}{0.000000,0.000000,0.000000}%
\pgfsetstrokecolor{currentstroke}%
\pgfsetdash{}{0pt}%
\pgfpathmoveto{\pgfqpoint{3.927857in}{2.555781in}}%
\pgfpathlineto{\pgfqpoint{3.940601in}{2.551928in}}%
\pgfpathlineto{\pgfqpoint{3.953350in}{2.548107in}}%
\pgfpathlineto{\pgfqpoint{3.966104in}{2.544317in}}%
\pgfpathlineto{\pgfqpoint{3.978864in}{2.540558in}}%
\pgfpathlineto{\pgfqpoint{3.971447in}{2.533224in}}%
\pgfpathlineto{\pgfqpoint{3.964024in}{2.525903in}}%
\pgfpathlineto{\pgfqpoint{3.956597in}{2.518592in}}%
\pgfpathlineto{\pgfqpoint{3.949163in}{2.511291in}}%
\pgfpathlineto{\pgfqpoint{3.936392in}{2.515028in}}%
\pgfpathlineto{\pgfqpoint{3.923625in}{2.518796in}}%
\pgfpathlineto{\pgfqpoint{3.910864in}{2.522595in}}%
\pgfpathlineto{\pgfqpoint{3.898109in}{2.526426in}}%
\pgfpathlineto{\pgfqpoint{3.905554in}{2.533745in}}%
\pgfpathlineto{\pgfqpoint{3.912994in}{2.541076in}}%
\pgfpathlineto{\pgfqpoint{3.920428in}{2.548421in}}%
\pgfpathlineto{\pgfqpoint{3.927857in}{2.555781in}}%
\pgfpathclose%
\pgfusepath{fill}%
\end{pgfscope}%
\begin{pgfscope}%
\pgfpathrectangle{\pgfqpoint{1.254980in}{0.150000in}}{\pgfqpoint{5.490039in}{5.490039in}}%
\pgfusepath{clip}%
\pgfsetbuttcap%
\pgfsetroundjoin%
\definecolor{currentfill}{rgb}{0.273809,0.031497,0.358853}%
\pgfsetfillcolor{currentfill}%
\pgfsetfillopacity{0.700000}%
\pgfsetlinewidth{0.000000pt}%
\definecolor{currentstroke}{rgb}{0.000000,0.000000,0.000000}%
\pgfsetstrokecolor{currentstroke}%
\pgfsetdash{}{0pt}%
\pgfpathmoveto{\pgfqpoint{4.271809in}{2.571715in}}%
\pgfpathlineto{\pgfqpoint{4.284629in}{2.568320in}}%
\pgfpathlineto{\pgfqpoint{4.297455in}{2.564954in}}%
\pgfpathlineto{\pgfqpoint{4.310287in}{2.561615in}}%
\pgfpathlineto{\pgfqpoint{4.323125in}{2.558305in}}%
\pgfpathlineto{\pgfqpoint{4.315830in}{2.551031in}}%
\pgfpathlineto{\pgfqpoint{4.308530in}{2.543773in}}%
\pgfpathlineto{\pgfqpoint{4.301225in}{2.536530in}}%
\pgfpathlineto{\pgfqpoint{4.293915in}{2.529300in}}%
\pgfpathlineto{\pgfqpoint{4.281065in}{2.532550in}}%
\pgfpathlineto{\pgfqpoint{4.268221in}{2.535828in}}%
\pgfpathlineto{\pgfqpoint{4.255383in}{2.539135in}}%
\pgfpathlineto{\pgfqpoint{4.242551in}{2.542470in}}%
\pgfpathlineto{\pgfqpoint{4.249873in}{2.549755in}}%
\pgfpathlineto{\pgfqpoint{4.257190in}{2.557057in}}%
\pgfpathlineto{\pgfqpoint{4.264502in}{2.564376in}}%
\pgfpathlineto{\pgfqpoint{4.271809in}{2.571715in}}%
\pgfpathclose%
\pgfusepath{fill}%
\end{pgfscope}%
\begin{pgfscope}%
\pgfpathrectangle{\pgfqpoint{1.254980in}{0.150000in}}{\pgfqpoint{5.490039in}{5.490039in}}%
\pgfusepath{clip}%
\pgfsetbuttcap%
\pgfsetroundjoin%
\definecolor{currentfill}{rgb}{0.269944,0.014625,0.341379}%
\pgfsetfillcolor{currentfill}%
\pgfsetfillopacity{0.700000}%
\pgfsetlinewidth{0.000000pt}%
\definecolor{currentstroke}{rgb}{0.000000,0.000000,0.000000}%
\pgfsetstrokecolor{currentstroke}%
\pgfsetdash{}{0pt}%
\pgfpathmoveto{\pgfqpoint{3.583850in}{2.551672in}}%
\pgfpathlineto{\pgfqpoint{3.596528in}{2.547171in}}%
\pgfpathlineto{\pgfqpoint{3.609211in}{2.542706in}}%
\pgfpathlineto{\pgfqpoint{3.621899in}{2.538277in}}%
\pgfpathlineto{\pgfqpoint{3.634591in}{2.533884in}}%
\pgfpathlineto{\pgfqpoint{3.627049in}{2.526687in}}%
\pgfpathlineto{\pgfqpoint{3.619501in}{2.519510in}}%
\pgfpathlineto{\pgfqpoint{3.611947in}{2.512352in}}%
\pgfpathlineto{\pgfqpoint{3.604388in}{2.505213in}}%
\pgfpathlineto{\pgfqpoint{3.591683in}{2.509622in}}%
\pgfpathlineto{\pgfqpoint{3.578983in}{2.514067in}}%
\pgfpathlineto{\pgfqpoint{3.566288in}{2.518548in}}%
\pgfpathlineto{\pgfqpoint{3.553598in}{2.523064in}}%
\pgfpathlineto{\pgfqpoint{3.561170in}{2.530183in}}%
\pgfpathlineto{\pgfqpoint{3.568736in}{2.537324in}}%
\pgfpathlineto{\pgfqpoint{3.576296in}{2.544487in}}%
\pgfpathlineto{\pgfqpoint{3.583850in}{2.551672in}}%
\pgfpathclose%
\pgfusepath{fill}%
\end{pgfscope}%
\begin{pgfscope}%
\pgfpathrectangle{\pgfqpoint{1.254980in}{0.150000in}}{\pgfqpoint{5.490039in}{5.490039in}}%
\pgfusepath{clip}%
\pgfsetbuttcap%
\pgfsetroundjoin%
\definecolor{currentfill}{rgb}{0.280267,0.073417,0.397163}%
\pgfsetfillcolor{currentfill}%
\pgfsetfillopacity{0.700000}%
\pgfsetlinewidth{0.000000pt}%
\definecolor{currentstroke}{rgb}{0.000000,0.000000,0.000000}%
\pgfsetstrokecolor{currentstroke}%
\pgfsetdash{}{0pt}%
\pgfpathmoveto{\pgfqpoint{5.040354in}{2.639346in}}%
\pgfpathlineto{\pgfqpoint{5.053353in}{2.636334in}}%
\pgfpathlineto{\pgfqpoint{5.066358in}{2.633346in}}%
\pgfpathlineto{\pgfqpoint{5.079370in}{2.630383in}}%
\pgfpathlineto{\pgfqpoint{5.092388in}{2.627444in}}%
\pgfpathlineto{\pgfqpoint{5.085367in}{2.620109in}}%
\pgfpathlineto{\pgfqpoint{5.078344in}{2.612856in}}%
\pgfpathlineto{\pgfqpoint{5.071316in}{2.605678in}}%
\pgfpathlineto{\pgfqpoint{5.064285in}{2.598572in}}%
\pgfpathlineto{\pgfqpoint{5.051252in}{2.601363in}}%
\pgfpathlineto{\pgfqpoint{5.038225in}{2.604178in}}%
\pgfpathlineto{\pgfqpoint{5.025205in}{2.607018in}}%
\pgfpathlineto{\pgfqpoint{5.012191in}{2.609882in}}%
\pgfpathlineto{\pgfqpoint{5.019237in}{2.617132in}}%
\pgfpathlineto{\pgfqpoint{5.026279in}{2.624456in}}%
\pgfpathlineto{\pgfqpoint{5.033318in}{2.631859in}}%
\pgfpathlineto{\pgfqpoint{5.040354in}{2.639346in}}%
\pgfpathclose%
\pgfusepath{fill}%
\end{pgfscope}%
\begin{pgfscope}%
\pgfpathrectangle{\pgfqpoint{1.254980in}{0.150000in}}{\pgfqpoint{5.490039in}{5.490039in}}%
\pgfusepath{clip}%
\pgfsetbuttcap%
\pgfsetroundjoin%
\definecolor{currentfill}{rgb}{0.278791,0.062145,0.386592}%
\pgfsetfillcolor{currentfill}%
\pgfsetfillopacity{0.700000}%
\pgfsetlinewidth{0.000000pt}%
\definecolor{currentstroke}{rgb}{0.000000,0.000000,0.000000}%
\pgfsetstrokecolor{currentstroke}%
\pgfsetdash{}{0pt}%
\pgfpathmoveto{\pgfqpoint{4.828133in}{2.616217in}}%
\pgfpathlineto{\pgfqpoint{4.841084in}{2.613190in}}%
\pgfpathlineto{\pgfqpoint{4.854041in}{2.610188in}}%
\pgfpathlineto{\pgfqpoint{4.867005in}{2.607211in}}%
\pgfpathlineto{\pgfqpoint{4.879975in}{2.604259in}}%
\pgfpathlineto{\pgfqpoint{4.872880in}{2.597061in}}%
\pgfpathlineto{\pgfqpoint{4.865781in}{2.589919in}}%
\pgfpathlineto{\pgfqpoint{4.858678in}{2.582827in}}%
\pgfpathlineto{\pgfqpoint{4.851571in}{2.575782in}}%
\pgfpathlineto{\pgfqpoint{4.838586in}{2.578611in}}%
\pgfpathlineto{\pgfqpoint{4.825609in}{2.581465in}}%
\pgfpathlineto{\pgfqpoint{4.812637in}{2.584344in}}%
\pgfpathlineto{\pgfqpoint{4.799672in}{2.587248in}}%
\pgfpathlineto{\pgfqpoint{4.806794in}{2.594411in}}%
\pgfpathlineto{\pgfqpoint{4.813911in}{2.601624in}}%
\pgfpathlineto{\pgfqpoint{4.821024in}{2.608892in}}%
\pgfpathlineto{\pgfqpoint{4.828133in}{2.616217in}}%
\pgfpathclose%
\pgfusepath{fill}%
\end{pgfscope}%
\begin{pgfscope}%
\pgfpathrectangle{\pgfqpoint{1.254980in}{0.150000in}}{\pgfqpoint{5.490039in}{5.490039in}}%
\pgfusepath{clip}%
\pgfsetbuttcap%
\pgfsetroundjoin%
\definecolor{currentfill}{rgb}{0.271305,0.019942,0.347269}%
\pgfsetfillcolor{currentfill}%
\pgfsetfillopacity{0.700000}%
\pgfsetlinewidth{0.000000pt}%
\definecolor{currentstroke}{rgb}{0.000000,0.000000,0.000000}%
\pgfsetstrokecolor{currentstroke}%
\pgfsetdash{}{0pt}%
\pgfpathmoveto{\pgfqpoint{4.059527in}{2.555168in}}%
\pgfpathlineto{\pgfqpoint{4.072303in}{2.551528in}}%
\pgfpathlineto{\pgfqpoint{4.085084in}{2.547918in}}%
\pgfpathlineto{\pgfqpoint{4.097871in}{2.544338in}}%
\pgfpathlineto{\pgfqpoint{4.110663in}{2.540788in}}%
\pgfpathlineto{\pgfqpoint{4.103291in}{2.533473in}}%
\pgfpathlineto{\pgfqpoint{4.095914in}{2.526170in}}%
\pgfpathlineto{\pgfqpoint{4.088531in}{2.518875in}}%
\pgfpathlineto{\pgfqpoint{4.081142in}{2.511589in}}%
\pgfpathlineto{\pgfqpoint{4.068338in}{2.515105in}}%
\pgfpathlineto{\pgfqpoint{4.055539in}{2.518650in}}%
\pgfpathlineto{\pgfqpoint{4.042746in}{2.522225in}}%
\pgfpathlineto{\pgfqpoint{4.029959in}{2.525830in}}%
\pgfpathlineto{\pgfqpoint{4.037359in}{2.533147in}}%
\pgfpathlineto{\pgfqpoint{4.044754in}{2.540474in}}%
\pgfpathlineto{\pgfqpoint{4.052143in}{2.547814in}}%
\pgfpathlineto{\pgfqpoint{4.059527in}{2.555168in}}%
\pgfpathclose%
\pgfusepath{fill}%
\end{pgfscope}%
\begin{pgfscope}%
\pgfpathrectangle{\pgfqpoint{1.254980in}{0.150000in}}{\pgfqpoint{5.490039in}{5.490039in}}%
\pgfusepath{clip}%
\pgfsetbuttcap%
\pgfsetroundjoin%
\definecolor{currentfill}{rgb}{0.269944,0.014625,0.341379}%
\pgfsetfillcolor{currentfill}%
\pgfsetfillopacity{0.700000}%
\pgfsetlinewidth{0.000000pt}%
\definecolor{currentstroke}{rgb}{0.000000,0.000000,0.000000}%
\pgfsetstrokecolor{currentstroke}%
\pgfsetdash{}{0pt}%
\pgfpathmoveto{\pgfqpoint{3.715475in}{2.545654in}}%
\pgfpathlineto{\pgfqpoint{3.728180in}{2.541437in}}%
\pgfpathlineto{\pgfqpoint{3.740890in}{2.537255in}}%
\pgfpathlineto{\pgfqpoint{3.753606in}{2.533107in}}%
\pgfpathlineto{\pgfqpoint{3.766326in}{2.528992in}}%
\pgfpathlineto{\pgfqpoint{3.758830in}{2.521719in}}%
\pgfpathlineto{\pgfqpoint{3.751328in}{2.514461in}}%
\pgfpathlineto{\pgfqpoint{3.743821in}{2.507217in}}%
\pgfpathlineto{\pgfqpoint{3.736309in}{2.499986in}}%
\pgfpathlineto{\pgfqpoint{3.723577in}{2.504104in}}%
\pgfpathlineto{\pgfqpoint{3.710849in}{2.508255in}}%
\pgfpathlineto{\pgfqpoint{3.698127in}{2.512440in}}%
\pgfpathlineto{\pgfqpoint{3.685410in}{2.516660in}}%
\pgfpathlineto{\pgfqpoint{3.692935in}{2.523883in}}%
\pgfpathlineto{\pgfqpoint{3.700454in}{2.531123in}}%
\pgfpathlineto{\pgfqpoint{3.707967in}{2.538380in}}%
\pgfpathlineto{\pgfqpoint{3.715475in}{2.545654in}}%
\pgfpathclose%
\pgfusepath{fill}%
\end{pgfscope}%
\begin{pgfscope}%
\pgfpathrectangle{\pgfqpoint{1.254980in}{0.150000in}}{\pgfqpoint{5.490039in}{5.490039in}}%
\pgfusepath{clip}%
\pgfsetbuttcap%
\pgfsetroundjoin%
\definecolor{currentfill}{rgb}{0.277018,0.050344,0.375715}%
\pgfsetfillcolor{currentfill}%
\pgfsetfillopacity{0.700000}%
\pgfsetlinewidth{0.000000pt}%
\definecolor{currentstroke}{rgb}{0.000000,0.000000,0.000000}%
\pgfsetstrokecolor{currentstroke}%
\pgfsetdash{}{0pt}%
\pgfpathmoveto{\pgfqpoint{4.615892in}{2.594552in}}%
\pgfpathlineto{\pgfqpoint{4.628795in}{2.591448in}}%
\pgfpathlineto{\pgfqpoint{4.641704in}{2.588371in}}%
\pgfpathlineto{\pgfqpoint{4.654619in}{2.585320in}}%
\pgfpathlineto{\pgfqpoint{4.667541in}{2.582295in}}%
\pgfpathlineto{\pgfqpoint{4.660369in}{2.575122in}}%
\pgfpathlineto{\pgfqpoint{4.653193in}{2.567985in}}%
\pgfpathlineto{\pgfqpoint{4.646012in}{2.560879in}}%
\pgfpathlineto{\pgfqpoint{4.638826in}{2.553802in}}%
\pgfpathlineto{\pgfqpoint{4.625891in}{2.556729in}}%
\pgfpathlineto{\pgfqpoint{4.612963in}{2.559682in}}%
\pgfpathlineto{\pgfqpoint{4.600040in}{2.562662in}}%
\pgfpathlineto{\pgfqpoint{4.587124in}{2.565667in}}%
\pgfpathlineto{\pgfqpoint{4.594324in}{2.572838in}}%
\pgfpathlineto{\pgfqpoint{4.601518in}{2.580040in}}%
\pgfpathlineto{\pgfqpoint{4.608707in}{2.587277in}}%
\pgfpathlineto{\pgfqpoint{4.615892in}{2.594552in}}%
\pgfpathclose%
\pgfusepath{fill}%
\end{pgfscope}%
\begin{pgfscope}%
\pgfpathrectangle{\pgfqpoint{1.254980in}{0.150000in}}{\pgfqpoint{5.490039in}{5.490039in}}%
\pgfusepath{clip}%
\pgfsetbuttcap%
\pgfsetroundjoin%
\definecolor{currentfill}{rgb}{0.274952,0.037752,0.364543}%
\pgfsetfillcolor{currentfill}%
\pgfsetfillopacity{0.700000}%
\pgfsetlinewidth{0.000000pt}%
\definecolor{currentstroke}{rgb}{0.000000,0.000000,0.000000}%
\pgfsetstrokecolor{currentstroke}%
\pgfsetdash{}{0pt}%
\pgfpathmoveto{\pgfqpoint{4.403614in}{2.574371in}}%
\pgfpathlineto{\pgfqpoint{4.416469in}{2.571128in}}%
\pgfpathlineto{\pgfqpoint{4.429330in}{2.567912in}}%
\pgfpathlineto{\pgfqpoint{4.442198in}{2.564723in}}%
\pgfpathlineto{\pgfqpoint{4.455071in}{2.561562in}}%
\pgfpathlineto{\pgfqpoint{4.447821in}{2.554349in}}%
\pgfpathlineto{\pgfqpoint{4.440567in}{2.547158in}}%
\pgfpathlineto{\pgfqpoint{4.433307in}{2.539984in}}%
\pgfpathlineto{\pgfqpoint{4.426043in}{2.532826in}}%
\pgfpathlineto{\pgfqpoint{4.413157in}{2.535915in}}%
\pgfpathlineto{\pgfqpoint{4.400277in}{2.539031in}}%
\pgfpathlineto{\pgfqpoint{4.387403in}{2.542174in}}%
\pgfpathlineto{\pgfqpoint{4.374535in}{2.545345in}}%
\pgfpathlineto{\pgfqpoint{4.381813in}{2.552571in}}%
\pgfpathlineto{\pgfqpoint{4.389085in}{2.559816in}}%
\pgfpathlineto{\pgfqpoint{4.396352in}{2.567081in}}%
\pgfpathlineto{\pgfqpoint{4.403614in}{2.574371in}}%
\pgfpathclose%
\pgfusepath{fill}%
\end{pgfscope}%
\begin{pgfscope}%
\pgfpathrectangle{\pgfqpoint{1.254980in}{0.150000in}}{\pgfqpoint{5.490039in}{5.490039in}}%
\pgfusepath{clip}%
\pgfsetbuttcap%
\pgfsetroundjoin%
\definecolor{currentfill}{rgb}{0.280894,0.078907,0.402329}%
\pgfsetfillcolor{currentfill}%
\pgfsetfillopacity{0.700000}%
\pgfsetlinewidth{0.000000pt}%
\definecolor{currentstroke}{rgb}{0.000000,0.000000,0.000000}%
\pgfsetstrokecolor{currentstroke}%
\pgfsetdash{}{0pt}%
\pgfpathmoveto{\pgfqpoint{5.172516in}{2.645532in}}%
\pgfpathlineto{\pgfqpoint{5.185552in}{2.642553in}}%
\pgfpathlineto{\pgfqpoint{5.198594in}{2.639599in}}%
\pgfpathlineto{\pgfqpoint{5.211642in}{2.636668in}}%
\pgfpathlineto{\pgfqpoint{5.224698in}{2.633761in}}%
\pgfpathlineto{\pgfqpoint{5.217720in}{2.626384in}}%
\pgfpathlineto{\pgfqpoint{5.210740in}{2.619102in}}%
\pgfpathlineto{\pgfqpoint{5.203758in}{2.611909in}}%
\pgfpathlineto{\pgfqpoint{5.196772in}{2.604801in}}%
\pgfpathlineto{\pgfqpoint{5.183700in}{2.607547in}}%
\pgfpathlineto{\pgfqpoint{5.170636in}{2.610318in}}%
\pgfpathlineto{\pgfqpoint{5.157578in}{2.613112in}}%
\pgfpathlineto{\pgfqpoint{5.144527in}{2.615930in}}%
\pgfpathlineto{\pgfqpoint{5.151528in}{2.623194in}}%
\pgfpathlineto{\pgfqpoint{5.158527in}{2.630546in}}%
\pgfpathlineto{\pgfqpoint{5.165523in}{2.637990in}}%
\pgfpathlineto{\pgfqpoint{5.172516in}{2.645532in}}%
\pgfpathclose%
\pgfusepath{fill}%
\end{pgfscope}%
\begin{pgfscope}%
\pgfpathrectangle{\pgfqpoint{1.254980in}{0.150000in}}{\pgfqpoint{5.490039in}{5.490039in}}%
\pgfusepath{clip}%
\pgfsetbuttcap%
\pgfsetroundjoin%
\definecolor{currentfill}{rgb}{0.269944,0.014625,0.341379}%
\pgfsetfillcolor{currentfill}%
\pgfsetfillopacity{0.700000}%
\pgfsetlinewidth{0.000000pt}%
\definecolor{currentstroke}{rgb}{0.000000,0.000000,0.000000}%
\pgfsetstrokecolor{currentstroke}%
\pgfsetdash{}{0pt}%
\pgfpathmoveto{\pgfqpoint{3.847140in}{2.542070in}}%
\pgfpathlineto{\pgfqpoint{3.859874in}{2.538111in}}%
\pgfpathlineto{\pgfqpoint{3.872614in}{2.534184in}}%
\pgfpathlineto{\pgfqpoint{3.885359in}{2.530289in}}%
\pgfpathlineto{\pgfqpoint{3.898109in}{2.526426in}}%
\pgfpathlineto{\pgfqpoint{3.890658in}{2.519119in}}%
\pgfpathlineto{\pgfqpoint{3.883203in}{2.511823in}}%
\pgfpathlineto{\pgfqpoint{3.875741in}{2.504536in}}%
\pgfpathlineto{\pgfqpoint{3.868275in}{2.497259in}}%
\pgfpathlineto{\pgfqpoint{3.855513in}{2.501112in}}%
\pgfpathlineto{\pgfqpoint{3.842756in}{2.504997in}}%
\pgfpathlineto{\pgfqpoint{3.830004in}{2.508915in}}%
\pgfpathlineto{\pgfqpoint{3.817258in}{2.512865in}}%
\pgfpathlineto{\pgfqpoint{3.824737in}{2.520147in}}%
\pgfpathlineto{\pgfqpoint{3.832210in}{2.527441in}}%
\pgfpathlineto{\pgfqpoint{3.839678in}{2.534749in}}%
\pgfpathlineto{\pgfqpoint{3.847140in}{2.542070in}}%
\pgfpathclose%
\pgfusepath{fill}%
\end{pgfscope}%
\begin{pgfscope}%
\pgfpathrectangle{\pgfqpoint{1.254980in}{0.150000in}}{\pgfqpoint{5.490039in}{5.490039in}}%
\pgfusepath{clip}%
\pgfsetbuttcap%
\pgfsetroundjoin%
\definecolor{currentfill}{rgb}{0.273809,0.031497,0.358853}%
\pgfsetfillcolor{currentfill}%
\pgfsetfillopacity{0.700000}%
\pgfsetlinewidth{0.000000pt}%
\definecolor{currentstroke}{rgb}{0.000000,0.000000,0.000000}%
\pgfsetstrokecolor{currentstroke}%
\pgfsetdash{}{0pt}%
\pgfpathmoveto{\pgfqpoint{3.239430in}{2.566690in}}%
\pgfpathlineto{\pgfqpoint{3.252059in}{2.561318in}}%
\pgfpathlineto{\pgfqpoint{3.264693in}{2.555988in}}%
\pgfpathlineto{\pgfqpoint{3.277330in}{2.550700in}}%
\pgfpathlineto{\pgfqpoint{3.289972in}{2.545454in}}%
\pgfpathlineto{\pgfqpoint{3.282294in}{2.538741in}}%
\pgfpathlineto{\pgfqpoint{3.274610in}{2.532067in}}%
\pgfpathlineto{\pgfqpoint{3.266920in}{2.525432in}}%
\pgfpathlineto{\pgfqpoint{3.259223in}{2.518839in}}%
\pgfpathlineto{\pgfqpoint{3.246568in}{2.524138in}}%
\pgfpathlineto{\pgfqpoint{3.233916in}{2.529480in}}%
\pgfpathlineto{\pgfqpoint{3.221269in}{2.534864in}}%
\pgfpathlineto{\pgfqpoint{3.208626in}{2.540290in}}%
\pgfpathlineto{\pgfqpoint{3.216337in}{2.546825in}}%
\pgfpathlineto{\pgfqpoint{3.224041in}{2.553404in}}%
\pgfpathlineto{\pgfqpoint{3.231739in}{2.560026in}}%
\pgfpathlineto{\pgfqpoint{3.239430in}{2.566690in}}%
\pgfpathclose%
\pgfusepath{fill}%
\end{pgfscope}%
\begin{pgfscope}%
\pgfpathrectangle{\pgfqpoint{1.254980in}{0.150000in}}{\pgfqpoint{5.490039in}{5.490039in}}%
\pgfusepath{clip}%
\pgfsetbuttcap%
\pgfsetroundjoin%
\definecolor{currentfill}{rgb}{0.272594,0.025563,0.353093}%
\pgfsetfillcolor{currentfill}%
\pgfsetfillopacity{0.700000}%
\pgfsetlinewidth{0.000000pt}%
\definecolor{currentstroke}{rgb}{0.000000,0.000000,0.000000}%
\pgfsetstrokecolor{currentstroke}%
\pgfsetdash{}{0pt}%
\pgfpathmoveto{\pgfqpoint{4.191279in}{2.556097in}}%
\pgfpathlineto{\pgfqpoint{4.204089in}{2.552647in}}%
\pgfpathlineto{\pgfqpoint{4.216903in}{2.549226in}}%
\pgfpathlineto{\pgfqpoint{4.229724in}{2.545833in}}%
\pgfpathlineto{\pgfqpoint{4.242551in}{2.542470in}}%
\pgfpathlineto{\pgfqpoint{4.235223in}{2.535198in}}%
\pgfpathlineto{\pgfqpoint{4.227891in}{2.527938in}}%
\pgfpathlineto{\pgfqpoint{4.220553in}{2.520688in}}%
\pgfpathlineto{\pgfqpoint{4.213210in}{2.513446in}}%
\pgfpathlineto{\pgfqpoint{4.200371in}{2.516762in}}%
\pgfpathlineto{\pgfqpoint{4.187538in}{2.520107in}}%
\pgfpathlineto{\pgfqpoint{4.174711in}{2.523481in}}%
\pgfpathlineto{\pgfqpoint{4.161890in}{2.526884in}}%
\pgfpathlineto{\pgfqpoint{4.169245in}{2.534168in}}%
\pgfpathlineto{\pgfqpoint{4.176595in}{2.541464in}}%
\pgfpathlineto{\pgfqpoint{4.183940in}{2.548773in}}%
\pgfpathlineto{\pgfqpoint{4.191279in}{2.556097in}}%
\pgfpathclose%
\pgfusepath{fill}%
\end{pgfscope}%
\begin{pgfscope}%
\pgfpathrectangle{\pgfqpoint{1.254980in}{0.150000in}}{\pgfqpoint{5.490039in}{5.490039in}}%
\pgfusepath{clip}%
\pgfsetbuttcap%
\pgfsetroundjoin%
\definecolor{currentfill}{rgb}{0.279566,0.067836,0.391917}%
\pgfsetfillcolor{currentfill}%
\pgfsetfillopacity{0.700000}%
\pgfsetlinewidth{0.000000pt}%
\definecolor{currentstroke}{rgb}{0.000000,0.000000,0.000000}%
\pgfsetstrokecolor{currentstroke}%
\pgfsetdash{}{0pt}%
\pgfpathmoveto{\pgfqpoint{4.960201in}{2.621585in}}%
\pgfpathlineto{\pgfqpoint{4.973189in}{2.618622in}}%
\pgfpathlineto{\pgfqpoint{4.986183in}{2.615684in}}%
\pgfpathlineto{\pgfqpoint{4.999184in}{2.612771in}}%
\pgfpathlineto{\pgfqpoint{5.012191in}{2.609882in}}%
\pgfpathlineto{\pgfqpoint{5.005141in}{2.602703in}}%
\pgfpathlineto{\pgfqpoint{4.998088in}{2.595589in}}%
\pgfpathlineto{\pgfqpoint{4.991030in}{2.588537in}}%
\pgfpathlineto{\pgfqpoint{4.983969in}{2.581542in}}%
\pgfpathlineto{\pgfqpoint{4.970947in}{2.584295in}}%
\pgfpathlineto{\pgfqpoint{4.957931in}{2.587073in}}%
\pgfpathlineto{\pgfqpoint{4.944922in}{2.589875in}}%
\pgfpathlineto{\pgfqpoint{4.931920in}{2.592702in}}%
\pgfpathlineto{\pgfqpoint{4.938996in}{2.599828in}}%
\pgfpathlineto{\pgfqpoint{4.946068in}{2.607015in}}%
\pgfpathlineto{\pgfqpoint{4.953136in}{2.614265in}}%
\pgfpathlineto{\pgfqpoint{4.960201in}{2.621585in}}%
\pgfpathclose%
\pgfusepath{fill}%
\end{pgfscope}%
\begin{pgfscope}%
\pgfpathrectangle{\pgfqpoint{1.254980in}{0.150000in}}{\pgfqpoint{5.490039in}{5.490039in}}%
\pgfusepath{clip}%
\pgfsetbuttcap%
\pgfsetroundjoin%
\definecolor{currentfill}{rgb}{0.271305,0.019942,0.347269}%
\pgfsetfillcolor{currentfill}%
\pgfsetfillopacity{0.700000}%
\pgfsetlinewidth{0.000000pt}%
\definecolor{currentstroke}{rgb}{0.000000,0.000000,0.000000}%
\pgfsetstrokecolor{currentstroke}%
\pgfsetdash{}{0pt}%
\pgfpathmoveto{\pgfqpoint{3.371173in}{2.552273in}}%
\pgfpathlineto{\pgfqpoint{3.383822in}{2.547272in}}%
\pgfpathlineto{\pgfqpoint{3.396476in}{2.542310in}}%
\pgfpathlineto{\pgfqpoint{3.409133in}{2.537389in}}%
\pgfpathlineto{\pgfqpoint{3.421795in}{2.532506in}}%
\pgfpathlineto{\pgfqpoint{3.414169in}{2.525571in}}%
\pgfpathlineto{\pgfqpoint{3.406536in}{2.518667in}}%
\pgfpathlineto{\pgfqpoint{3.398898in}{2.511793in}}%
\pgfpathlineto{\pgfqpoint{3.391253in}{2.504950in}}%
\pgfpathlineto{\pgfqpoint{3.378578in}{2.509874in}}%
\pgfpathlineto{\pgfqpoint{3.365907in}{2.514837in}}%
\pgfpathlineto{\pgfqpoint{3.353241in}{2.519840in}}%
\pgfpathlineto{\pgfqpoint{3.340579in}{2.524882in}}%
\pgfpathlineto{\pgfqpoint{3.348237in}{2.531679in}}%
\pgfpathlineto{\pgfqpoint{3.355888in}{2.538510in}}%
\pgfpathlineto{\pgfqpoint{3.363534in}{2.545375in}}%
\pgfpathlineto{\pgfqpoint{3.371173in}{2.552273in}}%
\pgfpathclose%
\pgfusepath{fill}%
\end{pgfscope}%
\begin{pgfscope}%
\pgfpathrectangle{\pgfqpoint{1.254980in}{0.150000in}}{\pgfqpoint{5.490039in}{5.490039in}}%
\pgfusepath{clip}%
\pgfsetbuttcap%
\pgfsetroundjoin%
\definecolor{currentfill}{rgb}{0.276022,0.044167,0.370164}%
\pgfsetfillcolor{currentfill}%
\pgfsetfillopacity{0.700000}%
\pgfsetlinewidth{0.000000pt}%
\definecolor{currentstroke}{rgb}{0.000000,0.000000,0.000000}%
\pgfsetstrokecolor{currentstroke}%
\pgfsetdash{}{0pt}%
\pgfpathmoveto{\pgfqpoint{3.107615in}{2.585282in}}%
\pgfpathlineto{\pgfqpoint{3.120228in}{2.579500in}}%
\pgfpathlineto{\pgfqpoint{3.132845in}{2.573764in}}%
\pgfpathlineto{\pgfqpoint{3.145466in}{2.568074in}}%
\pgfpathlineto{\pgfqpoint{3.158091in}{2.562429in}}%
\pgfpathlineto{\pgfqpoint{3.150359in}{2.556000in}}%
\pgfpathlineto{\pgfqpoint{3.142621in}{2.549621in}}%
\pgfpathlineto{\pgfqpoint{3.134875in}{2.543293in}}%
\pgfpathlineto{\pgfqpoint{3.127123in}{2.537017in}}%
\pgfpathlineto{\pgfqpoint{3.114484in}{2.542728in}}%
\pgfpathlineto{\pgfqpoint{3.101849in}{2.548485in}}%
\pgfpathlineto{\pgfqpoint{3.089217in}{2.554288in}}%
\pgfpathlineto{\pgfqpoint{3.076588in}{2.560136in}}%
\pgfpathlineto{\pgfqpoint{3.084356in}{2.566341in}}%
\pgfpathlineto{\pgfqpoint{3.092116in}{2.572601in}}%
\pgfpathlineto{\pgfqpoint{3.099869in}{2.578915in}}%
\pgfpathlineto{\pgfqpoint{3.107615in}{2.585282in}}%
\pgfpathclose%
\pgfusepath{fill}%
\end{pgfscope}%
\begin{pgfscope}%
\pgfpathrectangle{\pgfqpoint{1.254980in}{0.150000in}}{\pgfqpoint{5.490039in}{5.490039in}}%
\pgfusepath{clip}%
\pgfsetbuttcap%
\pgfsetroundjoin%
\definecolor{currentfill}{rgb}{0.277941,0.056324,0.381191}%
\pgfsetfillcolor{currentfill}%
\pgfsetfillopacity{0.700000}%
\pgfsetlinewidth{0.000000pt}%
\definecolor{currentstroke}{rgb}{0.000000,0.000000,0.000000}%
\pgfsetstrokecolor{currentstroke}%
\pgfsetdash{}{0pt}%
\pgfpathmoveto{\pgfqpoint{4.747876in}{2.599120in}}%
\pgfpathlineto{\pgfqpoint{4.760816in}{2.596114in}}%
\pgfpathlineto{\pgfqpoint{4.773762in}{2.593133in}}%
\pgfpathlineto{\pgfqpoint{4.786714in}{2.590178in}}%
\pgfpathlineto{\pgfqpoint{4.799672in}{2.587248in}}%
\pgfpathlineto{\pgfqpoint{4.792546in}{2.580132in}}%
\pgfpathlineto{\pgfqpoint{4.785416in}{2.573058in}}%
\pgfpathlineto{\pgfqpoint{4.778281in}{2.566023in}}%
\pgfpathlineto{\pgfqpoint{4.771141in}{2.559024in}}%
\pgfpathlineto{\pgfqpoint{4.758169in}{2.561843in}}%
\pgfpathlineto{\pgfqpoint{4.745203in}{2.564688in}}%
\pgfpathlineto{\pgfqpoint{4.732243in}{2.567558in}}%
\pgfpathlineto{\pgfqpoint{4.719290in}{2.570454in}}%
\pgfpathlineto{\pgfqpoint{4.726443in}{2.577559in}}%
\pgfpathlineto{\pgfqpoint{4.733592in}{2.584702in}}%
\pgfpathlineto{\pgfqpoint{4.740737in}{2.591888in}}%
\pgfpathlineto{\pgfqpoint{4.747876in}{2.599120in}}%
\pgfpathclose%
\pgfusepath{fill}%
\end{pgfscope}%
\begin{pgfscope}%
\pgfpathrectangle{\pgfqpoint{1.254980in}{0.150000in}}{\pgfqpoint{5.490039in}{5.490039in}}%
\pgfusepath{clip}%
\pgfsetbuttcap%
\pgfsetroundjoin%
\definecolor{currentfill}{rgb}{0.271305,0.019942,0.347269}%
\pgfsetfillcolor{currentfill}%
\pgfsetfillopacity{0.700000}%
\pgfsetlinewidth{0.000000pt}%
\definecolor{currentstroke}{rgb}{0.000000,0.000000,0.000000}%
\pgfsetstrokecolor{currentstroke}%
\pgfsetdash{}{0pt}%
\pgfpathmoveto{\pgfqpoint{3.502883in}{2.541498in}}%
\pgfpathlineto{\pgfqpoint{3.515555in}{2.536834in}}%
\pgfpathlineto{\pgfqpoint{3.528231in}{2.532207in}}%
\pgfpathlineto{\pgfqpoint{3.540912in}{2.527618in}}%
\pgfpathlineto{\pgfqpoint{3.553598in}{2.523064in}}%
\pgfpathlineto{\pgfqpoint{3.546020in}{2.515968in}}%
\pgfpathlineto{\pgfqpoint{3.538437in}{2.508893in}}%
\pgfpathlineto{\pgfqpoint{3.530848in}{2.501840in}}%
\pgfpathlineto{\pgfqpoint{3.523253in}{2.494811in}}%
\pgfpathlineto{\pgfqpoint{3.510555in}{2.499392in}}%
\pgfpathlineto{\pgfqpoint{3.497861in}{2.504010in}}%
\pgfpathlineto{\pgfqpoint{3.485172in}{2.508666in}}%
\pgfpathlineto{\pgfqpoint{3.472488in}{2.513358in}}%
\pgfpathlineto{\pgfqpoint{3.480095in}{2.520355in}}%
\pgfpathlineto{\pgfqpoint{3.487697in}{2.527377in}}%
\pgfpathlineto{\pgfqpoint{3.495293in}{2.534425in}}%
\pgfpathlineto{\pgfqpoint{3.502883in}{2.541498in}}%
\pgfpathclose%
\pgfusepath{fill}%
\end{pgfscope}%
\begin{pgfscope}%
\pgfpathrectangle{\pgfqpoint{1.254980in}{0.150000in}}{\pgfqpoint{5.490039in}{5.490039in}}%
\pgfusepath{clip}%
\pgfsetbuttcap%
\pgfsetroundjoin%
\definecolor{currentfill}{rgb}{0.278791,0.062145,0.386592}%
\pgfsetfillcolor{currentfill}%
\pgfsetfillopacity{0.700000}%
\pgfsetlinewidth{0.000000pt}%
\definecolor{currentstroke}{rgb}{0.000000,0.000000,0.000000}%
\pgfsetstrokecolor{currentstroke}%
\pgfsetdash{}{0pt}%
\pgfpathmoveto{\pgfqpoint{2.975683in}{2.608633in}}%
\pgfpathlineto{\pgfqpoint{2.988285in}{2.602400in}}%
\pgfpathlineto{\pgfqpoint{3.000890in}{2.596217in}}%
\pgfpathlineto{\pgfqpoint{3.013498in}{2.590083in}}%
\pgfpathlineto{\pgfqpoint{3.026109in}{2.583998in}}%
\pgfpathlineto{\pgfqpoint{3.018320in}{2.577924in}}%
\pgfpathlineto{\pgfqpoint{3.010523in}{2.571912in}}%
\pgfpathlineto{\pgfqpoint{3.002719in}{2.565962in}}%
\pgfpathlineto{\pgfqpoint{2.994908in}{2.560076in}}%
\pgfpathlineto{\pgfqpoint{2.982281in}{2.566241in}}%
\pgfpathlineto{\pgfqpoint{2.969657in}{2.572454in}}%
\pgfpathlineto{\pgfqpoint{2.957036in}{2.578717in}}%
\pgfpathlineto{\pgfqpoint{2.944419in}{2.585029in}}%
\pgfpathlineto{\pgfqpoint{2.952246in}{2.590830in}}%
\pgfpathlineto{\pgfqpoint{2.960066in}{2.596699in}}%
\pgfpathlineto{\pgfqpoint{2.967878in}{2.602634in}}%
\pgfpathlineto{\pgfqpoint{2.975683in}{2.608633in}}%
\pgfpathclose%
\pgfusepath{fill}%
\end{pgfscope}%
\begin{pgfscope}%
\pgfpathrectangle{\pgfqpoint{1.254980in}{0.150000in}}{\pgfqpoint{5.490039in}{5.490039in}}%
\pgfusepath{clip}%
\pgfsetbuttcap%
\pgfsetroundjoin%
\definecolor{currentfill}{rgb}{0.276022,0.044167,0.370164}%
\pgfsetfillcolor{currentfill}%
\pgfsetfillopacity{0.700000}%
\pgfsetlinewidth{0.000000pt}%
\definecolor{currentstroke}{rgb}{0.000000,0.000000,0.000000}%
\pgfsetstrokecolor{currentstroke}%
\pgfsetdash{}{0pt}%
\pgfpathmoveto{\pgfqpoint{4.535523in}{2.577955in}}%
\pgfpathlineto{\pgfqpoint{4.548414in}{2.574843in}}%
\pgfpathlineto{\pgfqpoint{4.561311in}{2.571758in}}%
\pgfpathlineto{\pgfqpoint{4.574215in}{2.568699in}}%
\pgfpathlineto{\pgfqpoint{4.587124in}{2.565667in}}%
\pgfpathlineto{\pgfqpoint{4.579920in}{2.558525in}}%
\pgfpathlineto{\pgfqpoint{4.572712in}{2.551409in}}%
\pgfpathlineto{\pgfqpoint{4.565498in}{2.544315in}}%
\pgfpathlineto{\pgfqpoint{4.558279in}{2.537241in}}%
\pgfpathlineto{\pgfqpoint{4.545356in}{2.540188in}}%
\pgfpathlineto{\pgfqpoint{4.532440in}{2.543161in}}%
\pgfpathlineto{\pgfqpoint{4.519529in}{2.546161in}}%
\pgfpathlineto{\pgfqpoint{4.506625in}{2.549187in}}%
\pgfpathlineto{\pgfqpoint{4.513857in}{2.556342in}}%
\pgfpathlineto{\pgfqpoint{4.521084in}{2.563520in}}%
\pgfpathlineto{\pgfqpoint{4.528306in}{2.570723in}}%
\pgfpathlineto{\pgfqpoint{4.535523in}{2.577955in}}%
\pgfpathclose%
\pgfusepath{fill}%
\end{pgfscope}%
\begin{pgfscope}%
\pgfpathrectangle{\pgfqpoint{1.254980in}{0.150000in}}{\pgfqpoint{5.490039in}{5.490039in}}%
\pgfusepath{clip}%
\pgfsetbuttcap%
\pgfsetroundjoin%
\definecolor{currentfill}{rgb}{0.271305,0.019942,0.347269}%
\pgfsetfillcolor{currentfill}%
\pgfsetfillopacity{0.700000}%
\pgfsetlinewidth{0.000000pt}%
\definecolor{currentstroke}{rgb}{0.000000,0.000000,0.000000}%
\pgfsetstrokecolor{currentstroke}%
\pgfsetdash{}{0pt}%
\pgfpathmoveto{\pgfqpoint{3.978864in}{2.540558in}}%
\pgfpathlineto{\pgfqpoint{3.991630in}{2.536830in}}%
\pgfpathlineto{\pgfqpoint{4.004400in}{2.533133in}}%
\pgfpathlineto{\pgfqpoint{4.017177in}{2.529466in}}%
\pgfpathlineto{\pgfqpoint{4.029959in}{2.525830in}}%
\pgfpathlineto{\pgfqpoint{4.022553in}{2.518524in}}%
\pgfpathlineto{\pgfqpoint{4.015143in}{2.511227in}}%
\pgfpathlineto{\pgfqpoint{4.007727in}{2.503937in}}%
\pgfpathlineto{\pgfqpoint{4.000305in}{2.496653in}}%
\pgfpathlineto{\pgfqpoint{3.987511in}{2.500266in}}%
\pgfpathlineto{\pgfqpoint{3.974723in}{2.503910in}}%
\pgfpathlineto{\pgfqpoint{3.961940in}{2.507585in}}%
\pgfpathlineto{\pgfqpoint{3.949163in}{2.511291in}}%
\pgfpathlineto{\pgfqpoint{3.956597in}{2.518592in}}%
\pgfpathlineto{\pgfqpoint{3.964024in}{2.525903in}}%
\pgfpathlineto{\pgfqpoint{3.971447in}{2.533224in}}%
\pgfpathlineto{\pgfqpoint{3.978864in}{2.540558in}}%
\pgfpathclose%
\pgfusepath{fill}%
\end{pgfscope}%
\begin{pgfscope}%
\pgfpathrectangle{\pgfqpoint{1.254980in}{0.150000in}}{\pgfqpoint{5.490039in}{5.490039in}}%
\pgfusepath{clip}%
\pgfsetbuttcap%
\pgfsetroundjoin%
\definecolor{currentfill}{rgb}{0.269944,0.014625,0.341379}%
\pgfsetfillcolor{currentfill}%
\pgfsetfillopacity{0.700000}%
\pgfsetlinewidth{0.000000pt}%
\definecolor{currentstroke}{rgb}{0.000000,0.000000,0.000000}%
\pgfsetstrokecolor{currentstroke}%
\pgfsetdash{}{0pt}%
\pgfpathmoveto{\pgfqpoint{3.634591in}{2.533884in}}%
\pgfpathlineto{\pgfqpoint{3.647289in}{2.529525in}}%
\pgfpathlineto{\pgfqpoint{3.659991in}{2.525202in}}%
\pgfpathlineto{\pgfqpoint{3.672698in}{2.520914in}}%
\pgfpathlineto{\pgfqpoint{3.685410in}{2.516660in}}%
\pgfpathlineto{\pgfqpoint{3.677880in}{2.509453in}}%
\pgfpathlineto{\pgfqpoint{3.670344in}{2.502261in}}%
\pgfpathlineto{\pgfqpoint{3.662803in}{2.495086in}}%
\pgfpathlineto{\pgfqpoint{3.655256in}{2.487926in}}%
\pgfpathlineto{\pgfqpoint{3.642531in}{2.492196in}}%
\pgfpathlineto{\pgfqpoint{3.629812in}{2.496500in}}%
\pgfpathlineto{\pgfqpoint{3.617098in}{2.500839in}}%
\pgfpathlineto{\pgfqpoint{3.604388in}{2.505213in}}%
\pgfpathlineto{\pgfqpoint{3.611947in}{2.512352in}}%
\pgfpathlineto{\pgfqpoint{3.619501in}{2.519510in}}%
\pgfpathlineto{\pgfqpoint{3.627049in}{2.526687in}}%
\pgfpathlineto{\pgfqpoint{3.634591in}{2.533884in}}%
\pgfpathclose%
\pgfusepath{fill}%
\end{pgfscope}%
\begin{pgfscope}%
\pgfpathrectangle{\pgfqpoint{1.254980in}{0.150000in}}{\pgfqpoint{5.490039in}{5.490039in}}%
\pgfusepath{clip}%
\pgfsetbuttcap%
\pgfsetroundjoin%
\definecolor{currentfill}{rgb}{0.273809,0.031497,0.358853}%
\pgfsetfillcolor{currentfill}%
\pgfsetfillopacity{0.700000}%
\pgfsetlinewidth{0.000000pt}%
\definecolor{currentstroke}{rgb}{0.000000,0.000000,0.000000}%
\pgfsetstrokecolor{currentstroke}%
\pgfsetdash{}{0pt}%
\pgfpathmoveto{\pgfqpoint{4.323125in}{2.558305in}}%
\pgfpathlineto{\pgfqpoint{4.335968in}{2.555023in}}%
\pgfpathlineto{\pgfqpoint{4.348818in}{2.551769in}}%
\pgfpathlineto{\pgfqpoint{4.361674in}{2.548543in}}%
\pgfpathlineto{\pgfqpoint{4.374535in}{2.545345in}}%
\pgfpathlineto{\pgfqpoint{4.367253in}{2.538135in}}%
\pgfpathlineto{\pgfqpoint{4.359966in}{2.530939in}}%
\pgfpathlineto{\pgfqpoint{4.352673in}{2.523755in}}%
\pgfpathlineto{\pgfqpoint{4.345375in}{2.516580in}}%
\pgfpathlineto{\pgfqpoint{4.332501in}{2.519718in}}%
\pgfpathlineto{\pgfqpoint{4.319633in}{2.522884in}}%
\pgfpathlineto{\pgfqpoint{4.306771in}{2.526078in}}%
\pgfpathlineto{\pgfqpoint{4.293915in}{2.529300in}}%
\pgfpathlineto{\pgfqpoint{4.301225in}{2.536530in}}%
\pgfpathlineto{\pgfqpoint{4.308530in}{2.543773in}}%
\pgfpathlineto{\pgfqpoint{4.315830in}{2.551031in}}%
\pgfpathlineto{\pgfqpoint{4.323125in}{2.558305in}}%
\pgfpathclose%
\pgfusepath{fill}%
\end{pgfscope}%
\begin{pgfscope}%
\pgfpathrectangle{\pgfqpoint{1.254980in}{0.150000in}}{\pgfqpoint{5.490039in}{5.490039in}}%
\pgfusepath{clip}%
\pgfsetbuttcap%
\pgfsetroundjoin%
\definecolor{currentfill}{rgb}{0.280894,0.078907,0.402329}%
\pgfsetfillcolor{currentfill}%
\pgfsetfillopacity{0.700000}%
\pgfsetlinewidth{0.000000pt}%
\definecolor{currentstroke}{rgb}{0.000000,0.000000,0.000000}%
\pgfsetstrokecolor{currentstroke}%
\pgfsetdash{}{0pt}%
\pgfpathmoveto{\pgfqpoint{5.092388in}{2.627444in}}%
\pgfpathlineto{\pgfqpoint{5.105413in}{2.624529in}}%
\pgfpathlineto{\pgfqpoint{5.118444in}{2.621639in}}%
\pgfpathlineto{\pgfqpoint{5.131482in}{2.618772in}}%
\pgfpathlineto{\pgfqpoint{5.144527in}{2.615930in}}%
\pgfpathlineto{\pgfqpoint{5.137522in}{2.608748in}}%
\pgfpathlineto{\pgfqpoint{5.130513in}{2.601644in}}%
\pgfpathlineto{\pgfqpoint{5.123502in}{2.594613in}}%
\pgfpathlineto{\pgfqpoint{5.116486in}{2.587651in}}%
\pgfpathlineto{\pgfqpoint{5.103426in}{2.590345in}}%
\pgfpathlineto{\pgfqpoint{5.090372in}{2.593063in}}%
\pgfpathlineto{\pgfqpoint{5.077326in}{2.595805in}}%
\pgfpathlineto{\pgfqpoint{5.064285in}{2.598572in}}%
\pgfpathlineto{\pgfqpoint{5.071316in}{2.605678in}}%
\pgfpathlineto{\pgfqpoint{5.078344in}{2.612856in}}%
\pgfpathlineto{\pgfqpoint{5.085367in}{2.620109in}}%
\pgfpathlineto{\pgfqpoint{5.092388in}{2.627444in}}%
\pgfpathclose%
\pgfusepath{fill}%
\end{pgfscope}%
\begin{pgfscope}%
\pgfpathrectangle{\pgfqpoint{1.254980in}{0.150000in}}{\pgfqpoint{5.490039in}{5.490039in}}%
\pgfusepath{clip}%
\pgfsetbuttcap%
\pgfsetroundjoin%
\definecolor{currentfill}{rgb}{0.278791,0.062145,0.386592}%
\pgfsetfillcolor{currentfill}%
\pgfsetfillopacity{0.700000}%
\pgfsetlinewidth{0.000000pt}%
\definecolor{currentstroke}{rgb}{0.000000,0.000000,0.000000}%
\pgfsetstrokecolor{currentstroke}%
\pgfsetdash{}{0pt}%
\pgfpathmoveto{\pgfqpoint{4.879975in}{2.604259in}}%
\pgfpathlineto{\pgfqpoint{4.892951in}{2.601333in}}%
\pgfpathlineto{\pgfqpoint{4.905934in}{2.598431in}}%
\pgfpathlineto{\pgfqpoint{4.918924in}{2.595554in}}%
\pgfpathlineto{\pgfqpoint{4.931920in}{2.592702in}}%
\pgfpathlineto{\pgfqpoint{4.924839in}{2.585632in}}%
\pgfpathlineto{\pgfqpoint{4.917755in}{2.578614in}}%
\pgfpathlineto{\pgfqpoint{4.910666in}{2.571644in}}%
\pgfpathlineto{\pgfqpoint{4.903573in}{2.564718in}}%
\pgfpathlineto{\pgfqpoint{4.890563in}{2.567447in}}%
\pgfpathlineto{\pgfqpoint{4.877559in}{2.570200in}}%
\pgfpathlineto{\pgfqpoint{4.864561in}{2.572979in}}%
\pgfpathlineto{\pgfqpoint{4.851571in}{2.575782in}}%
\pgfpathlineto{\pgfqpoint{4.858678in}{2.582827in}}%
\pgfpathlineto{\pgfqpoint{4.865781in}{2.589919in}}%
\pgfpathlineto{\pgfqpoint{4.872880in}{2.597061in}}%
\pgfpathlineto{\pgfqpoint{4.879975in}{2.604259in}}%
\pgfpathclose%
\pgfusepath{fill}%
\end{pgfscope}%
\begin{pgfscope}%
\pgfpathrectangle{\pgfqpoint{1.254980in}{0.150000in}}{\pgfqpoint{5.490039in}{5.490039in}}%
\pgfusepath{clip}%
\pgfsetbuttcap%
\pgfsetroundjoin%
\definecolor{currentfill}{rgb}{0.269944,0.014625,0.341379}%
\pgfsetfillcolor{currentfill}%
\pgfsetfillopacity{0.700000}%
\pgfsetlinewidth{0.000000pt}%
\definecolor{currentstroke}{rgb}{0.000000,0.000000,0.000000}%
\pgfsetstrokecolor{currentstroke}%
\pgfsetdash{}{0pt}%
\pgfpathmoveto{\pgfqpoint{3.766326in}{2.528992in}}%
\pgfpathlineto{\pgfqpoint{3.779051in}{2.524910in}}%
\pgfpathlineto{\pgfqpoint{3.791782in}{2.520862in}}%
\pgfpathlineto{\pgfqpoint{3.804517in}{2.516847in}}%
\pgfpathlineto{\pgfqpoint{3.817258in}{2.512865in}}%
\pgfpathlineto{\pgfqpoint{3.809774in}{2.505594in}}%
\pgfpathlineto{\pgfqpoint{3.802285in}{2.498334in}}%
\pgfpathlineto{\pgfqpoint{3.794790in}{2.491085in}}%
\pgfpathlineto{\pgfqpoint{3.787289in}{2.483846in}}%
\pgfpathlineto{\pgfqpoint{3.774536in}{2.487832in}}%
\pgfpathlineto{\pgfqpoint{3.761789in}{2.491850in}}%
\pgfpathlineto{\pgfqpoint{3.749046in}{2.495901in}}%
\pgfpathlineto{\pgfqpoint{3.736309in}{2.499986in}}%
\pgfpathlineto{\pgfqpoint{3.743821in}{2.507217in}}%
\pgfpathlineto{\pgfqpoint{3.751328in}{2.514461in}}%
\pgfpathlineto{\pgfqpoint{3.758830in}{2.521719in}}%
\pgfpathlineto{\pgfqpoint{3.766326in}{2.528992in}}%
\pgfpathclose%
\pgfusepath{fill}%
\end{pgfscope}%
\begin{pgfscope}%
\pgfpathrectangle{\pgfqpoint{1.254980in}{0.150000in}}{\pgfqpoint{5.490039in}{5.490039in}}%
\pgfusepath{clip}%
\pgfsetbuttcap%
\pgfsetroundjoin%
\definecolor{currentfill}{rgb}{0.271305,0.019942,0.347269}%
\pgfsetfillcolor{currentfill}%
\pgfsetfillopacity{0.700000}%
\pgfsetlinewidth{0.000000pt}%
\definecolor{currentstroke}{rgb}{0.000000,0.000000,0.000000}%
\pgfsetstrokecolor{currentstroke}%
\pgfsetdash{}{0pt}%
\pgfpathmoveto{\pgfqpoint{4.110663in}{2.540788in}}%
\pgfpathlineto{\pgfqpoint{4.123462in}{2.537268in}}%
\pgfpathlineto{\pgfqpoint{4.136265in}{2.533777in}}%
\pgfpathlineto{\pgfqpoint{4.149075in}{2.530316in}}%
\pgfpathlineto{\pgfqpoint{4.161890in}{2.526884in}}%
\pgfpathlineto{\pgfqpoint{4.154530in}{2.519608in}}%
\pgfpathlineto{\pgfqpoint{4.147164in}{2.512341in}}%
\pgfpathlineto{\pgfqpoint{4.139793in}{2.505081in}}%
\pgfpathlineto{\pgfqpoint{4.132417in}{2.497825in}}%
\pgfpathlineto{\pgfqpoint{4.119590in}{2.501222in}}%
\pgfpathlineto{\pgfqpoint{4.106768in}{2.504648in}}%
\pgfpathlineto{\pgfqpoint{4.093952in}{2.508104in}}%
\pgfpathlineto{\pgfqpoint{4.081142in}{2.511589in}}%
\pgfpathlineto{\pgfqpoint{4.088531in}{2.518875in}}%
\pgfpathlineto{\pgfqpoint{4.095914in}{2.526170in}}%
\pgfpathlineto{\pgfqpoint{4.103291in}{2.533473in}}%
\pgfpathlineto{\pgfqpoint{4.110663in}{2.540788in}}%
\pgfpathclose%
\pgfusepath{fill}%
\end{pgfscope}%
\begin{pgfscope}%
\pgfpathrectangle{\pgfqpoint{1.254980in}{0.150000in}}{\pgfqpoint{5.490039in}{5.490039in}}%
\pgfusepath{clip}%
\pgfsetbuttcap%
\pgfsetroundjoin%
\definecolor{currentfill}{rgb}{0.277018,0.050344,0.375715}%
\pgfsetfillcolor{currentfill}%
\pgfsetfillopacity{0.700000}%
\pgfsetlinewidth{0.000000pt}%
\definecolor{currentstroke}{rgb}{0.000000,0.000000,0.000000}%
\pgfsetstrokecolor{currentstroke}%
\pgfsetdash{}{0pt}%
\pgfpathmoveto{\pgfqpoint{4.667541in}{2.582295in}}%
\pgfpathlineto{\pgfqpoint{4.680469in}{2.579296in}}%
\pgfpathlineto{\pgfqpoint{4.693403in}{2.576323in}}%
\pgfpathlineto{\pgfqpoint{4.706343in}{2.573375in}}%
\pgfpathlineto{\pgfqpoint{4.719290in}{2.570454in}}%
\pgfpathlineto{\pgfqpoint{4.712132in}{2.563384in}}%
\pgfpathlineto{\pgfqpoint{4.704969in}{2.556346in}}%
\pgfpathlineto{\pgfqpoint{4.697801in}{2.549337in}}%
\pgfpathlineto{\pgfqpoint{4.690629in}{2.542354in}}%
\pgfpathlineto{\pgfqpoint{4.677668in}{2.545177in}}%
\pgfpathlineto{\pgfqpoint{4.664715in}{2.548026in}}%
\pgfpathlineto{\pgfqpoint{4.651767in}{2.550901in}}%
\pgfpathlineto{\pgfqpoint{4.638826in}{2.553802in}}%
\pgfpathlineto{\pgfqpoint{4.646012in}{2.560879in}}%
\pgfpathlineto{\pgfqpoint{4.653193in}{2.567985in}}%
\pgfpathlineto{\pgfqpoint{4.660369in}{2.575122in}}%
\pgfpathlineto{\pgfqpoint{4.667541in}{2.582295in}}%
\pgfpathclose%
\pgfusepath{fill}%
\end{pgfscope}%
\begin{pgfscope}%
\pgfpathrectangle{\pgfqpoint{1.254980in}{0.150000in}}{\pgfqpoint{5.490039in}{5.490039in}}%
\pgfusepath{clip}%
\pgfsetbuttcap%
\pgfsetroundjoin%
\definecolor{currentfill}{rgb}{0.272594,0.025563,0.353093}%
\pgfsetfillcolor{currentfill}%
\pgfsetfillopacity{0.700000}%
\pgfsetlinewidth{0.000000pt}%
\definecolor{currentstroke}{rgb}{0.000000,0.000000,0.000000}%
\pgfsetstrokecolor{currentstroke}%
\pgfsetdash{}{0pt}%
\pgfpathmoveto{\pgfqpoint{3.289972in}{2.545454in}}%
\pgfpathlineto{\pgfqpoint{3.302617in}{2.540250in}}%
\pgfpathlineto{\pgfqpoint{3.315267in}{2.535087in}}%
\pgfpathlineto{\pgfqpoint{3.327921in}{2.529964in}}%
\pgfpathlineto{\pgfqpoint{3.340579in}{2.524882in}}%
\pgfpathlineto{\pgfqpoint{3.332914in}{2.518120in}}%
\pgfpathlineto{\pgfqpoint{3.325244in}{2.511393in}}%
\pgfpathlineto{\pgfqpoint{3.317567in}{2.504703in}}%
\pgfpathlineto{\pgfqpoint{3.309884in}{2.498051in}}%
\pgfpathlineto{\pgfqpoint{3.297213in}{2.503187in}}%
\pgfpathlineto{\pgfqpoint{3.284545in}{2.508363in}}%
\pgfpathlineto{\pgfqpoint{3.271882in}{2.513580in}}%
\pgfpathlineto{\pgfqpoint{3.259223in}{2.518839in}}%
\pgfpathlineto{\pgfqpoint{3.266920in}{2.525432in}}%
\pgfpathlineto{\pgfqpoint{3.274610in}{2.532067in}}%
\pgfpathlineto{\pgfqpoint{3.282294in}{2.538741in}}%
\pgfpathlineto{\pgfqpoint{3.289972in}{2.545454in}}%
\pgfpathclose%
\pgfusepath{fill}%
\end{pgfscope}%
\begin{pgfscope}%
\pgfpathrectangle{\pgfqpoint{1.254980in}{0.150000in}}{\pgfqpoint{5.490039in}{5.490039in}}%
\pgfusepath{clip}%
\pgfsetbuttcap%
\pgfsetroundjoin%
\definecolor{currentfill}{rgb}{0.274952,0.037752,0.364543}%
\pgfsetfillcolor{currentfill}%
\pgfsetfillopacity{0.700000}%
\pgfsetlinewidth{0.000000pt}%
\definecolor{currentstroke}{rgb}{0.000000,0.000000,0.000000}%
\pgfsetstrokecolor{currentstroke}%
\pgfsetdash{}{0pt}%
\pgfpathmoveto{\pgfqpoint{4.455071in}{2.561562in}}%
\pgfpathlineto{\pgfqpoint{4.467950in}{2.558427in}}%
\pgfpathlineto{\pgfqpoint{4.480836in}{2.555320in}}%
\pgfpathlineto{\pgfqpoint{4.493728in}{2.552240in}}%
\pgfpathlineto{\pgfqpoint{4.506625in}{2.549187in}}%
\pgfpathlineto{\pgfqpoint{4.499389in}{2.542052in}}%
\pgfpathlineto{\pgfqpoint{4.492147in}{2.534935in}}%
\pgfpathlineto{\pgfqpoint{4.484900in}{2.527833in}}%
\pgfpathlineto{\pgfqpoint{4.477648in}{2.520743in}}%
\pgfpathlineto{\pgfqpoint{4.464737in}{2.523724in}}%
\pgfpathlineto{\pgfqpoint{4.451833in}{2.526731in}}%
\pgfpathlineto{\pgfqpoint{4.438935in}{2.529765in}}%
\pgfpathlineto{\pgfqpoint{4.426043in}{2.532826in}}%
\pgfpathlineto{\pgfqpoint{4.433307in}{2.539984in}}%
\pgfpathlineto{\pgfqpoint{4.440567in}{2.547158in}}%
\pgfpathlineto{\pgfqpoint{4.447821in}{2.554349in}}%
\pgfpathlineto{\pgfqpoint{4.455071in}{2.561562in}}%
\pgfpathclose%
\pgfusepath{fill}%
\end{pgfscope}%
\begin{pgfscope}%
\pgfpathrectangle{\pgfqpoint{1.254980in}{0.150000in}}{\pgfqpoint{5.490039in}{5.490039in}}%
\pgfusepath{clip}%
\pgfsetbuttcap%
\pgfsetroundjoin%
\definecolor{currentfill}{rgb}{0.274952,0.037752,0.364543}%
\pgfsetfillcolor{currentfill}%
\pgfsetfillopacity{0.700000}%
\pgfsetlinewidth{0.000000pt}%
\definecolor{currentstroke}{rgb}{0.000000,0.000000,0.000000}%
\pgfsetstrokecolor{currentstroke}%
\pgfsetdash{}{0pt}%
\pgfpathmoveto{\pgfqpoint{3.158091in}{2.562429in}}%
\pgfpathlineto{\pgfqpoint{3.170719in}{2.556828in}}%
\pgfpathlineto{\pgfqpoint{3.183351in}{2.551272in}}%
\pgfpathlineto{\pgfqpoint{3.195986in}{2.545759in}}%
\pgfpathlineto{\pgfqpoint{3.208626in}{2.540290in}}%
\pgfpathlineto{\pgfqpoint{3.200908in}{2.533800in}}%
\pgfpathlineto{\pgfqpoint{3.193184in}{2.527356in}}%
\pgfpathlineto{\pgfqpoint{3.185453in}{2.520959in}}%
\pgfpathlineto{\pgfqpoint{3.177716in}{2.514611in}}%
\pgfpathlineto{\pgfqpoint{3.165062in}{2.520147in}}%
\pgfpathlineto{\pgfqpoint{3.152412in}{2.525726in}}%
\pgfpathlineto{\pgfqpoint{3.139766in}{2.531349in}}%
\pgfpathlineto{\pgfqpoint{3.127123in}{2.537017in}}%
\pgfpathlineto{\pgfqpoint{3.134875in}{2.543293in}}%
\pgfpathlineto{\pgfqpoint{3.142621in}{2.549621in}}%
\pgfpathlineto{\pgfqpoint{3.150359in}{2.556000in}}%
\pgfpathlineto{\pgfqpoint{3.158091in}{2.562429in}}%
\pgfpathclose%
\pgfusepath{fill}%
\end{pgfscope}%
\begin{pgfscope}%
\pgfpathrectangle{\pgfqpoint{1.254980in}{0.150000in}}{\pgfqpoint{5.490039in}{5.490039in}}%
\pgfusepath{clip}%
\pgfsetbuttcap%
\pgfsetroundjoin%
\definecolor{currentfill}{rgb}{0.269944,0.014625,0.341379}%
\pgfsetfillcolor{currentfill}%
\pgfsetfillopacity{0.700000}%
\pgfsetlinewidth{0.000000pt}%
\definecolor{currentstroke}{rgb}{0.000000,0.000000,0.000000}%
\pgfsetstrokecolor{currentstroke}%
\pgfsetdash{}{0pt}%
\pgfpathmoveto{\pgfqpoint{3.898109in}{2.526426in}}%
\pgfpathlineto{\pgfqpoint{3.910864in}{2.522595in}}%
\pgfpathlineto{\pgfqpoint{3.923625in}{2.518796in}}%
\pgfpathlineto{\pgfqpoint{3.936392in}{2.515028in}}%
\pgfpathlineto{\pgfqpoint{3.949163in}{2.511291in}}%
\pgfpathlineto{\pgfqpoint{3.941725in}{2.503998in}}%
\pgfpathlineto{\pgfqpoint{3.934281in}{2.496713in}}%
\pgfpathlineto{\pgfqpoint{3.926831in}{2.489435in}}%
\pgfpathlineto{\pgfqpoint{3.919376in}{2.482162in}}%
\pgfpathlineto{\pgfqpoint{3.906593in}{2.485889in}}%
\pgfpathlineto{\pgfqpoint{3.893815in}{2.489648in}}%
\pgfpathlineto{\pgfqpoint{3.881042in}{2.493437in}}%
\pgfpathlineto{\pgfqpoint{3.868275in}{2.497259in}}%
\pgfpathlineto{\pgfqpoint{3.875741in}{2.504536in}}%
\pgfpathlineto{\pgfqpoint{3.883203in}{2.511823in}}%
\pgfpathlineto{\pgfqpoint{3.890658in}{2.519119in}}%
\pgfpathlineto{\pgfqpoint{3.898109in}{2.526426in}}%
\pgfpathclose%
\pgfusepath{fill}%
\end{pgfscope}%
\begin{pgfscope}%
\pgfpathrectangle{\pgfqpoint{1.254980in}{0.150000in}}{\pgfqpoint{5.490039in}{5.490039in}}%
\pgfusepath{clip}%
\pgfsetbuttcap%
\pgfsetroundjoin%
\definecolor{currentfill}{rgb}{0.271305,0.019942,0.347269}%
\pgfsetfillcolor{currentfill}%
\pgfsetfillopacity{0.700000}%
\pgfsetlinewidth{0.000000pt}%
\definecolor{currentstroke}{rgb}{0.000000,0.000000,0.000000}%
\pgfsetstrokecolor{currentstroke}%
\pgfsetdash{}{0pt}%
\pgfpathmoveto{\pgfqpoint{3.421795in}{2.532506in}}%
\pgfpathlineto{\pgfqpoint{3.434462in}{2.527661in}}%
\pgfpathlineto{\pgfqpoint{3.447133in}{2.522856in}}%
\pgfpathlineto{\pgfqpoint{3.459808in}{2.518088in}}%
\pgfpathlineto{\pgfqpoint{3.472488in}{2.513358in}}%
\pgfpathlineto{\pgfqpoint{3.464874in}{2.506388in}}%
\pgfpathlineto{\pgfqpoint{3.457255in}{2.499444in}}%
\pgfpathlineto{\pgfqpoint{3.449629in}{2.492527in}}%
\pgfpathlineto{\pgfqpoint{3.441998in}{2.485639in}}%
\pgfpathlineto{\pgfqpoint{3.429305in}{2.490409in}}%
\pgfpathlineto{\pgfqpoint{3.416616in}{2.495218in}}%
\pgfpathlineto{\pgfqpoint{3.403933in}{2.500065in}}%
\pgfpathlineto{\pgfqpoint{3.391253in}{2.504950in}}%
\pgfpathlineto{\pgfqpoint{3.398898in}{2.511793in}}%
\pgfpathlineto{\pgfqpoint{3.406536in}{2.518667in}}%
\pgfpathlineto{\pgfqpoint{3.414169in}{2.525571in}}%
\pgfpathlineto{\pgfqpoint{3.421795in}{2.532506in}}%
\pgfpathclose%
\pgfusepath{fill}%
\end{pgfscope}%
\begin{pgfscope}%
\pgfpathrectangle{\pgfqpoint{1.254980in}{0.150000in}}{\pgfqpoint{5.490039in}{5.490039in}}%
\pgfusepath{clip}%
\pgfsetbuttcap%
\pgfsetroundjoin%
\definecolor{currentfill}{rgb}{0.281446,0.084320,0.407414}%
\pgfsetfillcolor{currentfill}%
\pgfsetfillopacity{0.700000}%
\pgfsetlinewidth{0.000000pt}%
\definecolor{currentstroke}{rgb}{0.000000,0.000000,0.000000}%
\pgfsetstrokecolor{currentstroke}%
\pgfsetdash{}{0pt}%
\pgfpathmoveto{\pgfqpoint{5.224698in}{2.633761in}}%
\pgfpathlineto{\pgfqpoint{5.237759in}{2.630878in}}%
\pgfpathlineto{\pgfqpoint{5.250828in}{2.628018in}}%
\pgfpathlineto{\pgfqpoint{5.263903in}{2.625183in}}%
\pgfpathlineto{\pgfqpoint{5.276985in}{2.622371in}}%
\pgfpathlineto{\pgfqpoint{5.270024in}{2.615160in}}%
\pgfpathlineto{\pgfqpoint{5.263061in}{2.608040in}}%
\pgfpathlineto{\pgfqpoint{5.256094in}{2.601007in}}%
\pgfpathlineto{\pgfqpoint{5.249124in}{2.594055in}}%
\pgfpathlineto{\pgfqpoint{5.236026in}{2.596706in}}%
\pgfpathlineto{\pgfqpoint{5.222935in}{2.599381in}}%
\pgfpathlineto{\pgfqpoint{5.209850in}{2.602079in}}%
\pgfpathlineto{\pgfqpoint{5.196772in}{2.604801in}}%
\pgfpathlineto{\pgfqpoint{5.203758in}{2.611909in}}%
\pgfpathlineto{\pgfqpoint{5.210740in}{2.619102in}}%
\pgfpathlineto{\pgfqpoint{5.217720in}{2.626384in}}%
\pgfpathlineto{\pgfqpoint{5.224698in}{2.633761in}}%
\pgfpathclose%
\pgfusepath{fill}%
\end{pgfscope}%
\begin{pgfscope}%
\pgfpathrectangle{\pgfqpoint{1.254980in}{0.150000in}}{\pgfqpoint{5.490039in}{5.490039in}}%
\pgfusepath{clip}%
\pgfsetbuttcap%
\pgfsetroundjoin%
\definecolor{currentfill}{rgb}{0.277018,0.050344,0.375715}%
\pgfsetfillcolor{currentfill}%
\pgfsetfillopacity{0.700000}%
\pgfsetlinewidth{0.000000pt}%
\definecolor{currentstroke}{rgb}{0.000000,0.000000,0.000000}%
\pgfsetstrokecolor{currentstroke}%
\pgfsetdash{}{0pt}%
\pgfpathmoveto{\pgfqpoint{3.026109in}{2.583998in}}%
\pgfpathlineto{\pgfqpoint{3.038724in}{2.577961in}}%
\pgfpathlineto{\pgfqpoint{3.051342in}{2.571972in}}%
\pgfpathlineto{\pgfqpoint{3.063964in}{2.566031in}}%
\pgfpathlineto{\pgfqpoint{3.076588in}{2.560136in}}%
\pgfpathlineto{\pgfqpoint{3.068814in}{2.553988in}}%
\pgfpathlineto{\pgfqpoint{3.061033in}{2.547897in}}%
\pgfpathlineto{\pgfqpoint{3.053244in}{2.541866in}}%
\pgfpathlineto{\pgfqpoint{3.045448in}{2.535897in}}%
\pgfpathlineto{\pgfqpoint{3.032808in}{2.541871in}}%
\pgfpathlineto{\pgfqpoint{3.020171in}{2.547892in}}%
\pgfpathlineto{\pgfqpoint{3.007538in}{2.553960in}}%
\pgfpathlineto{\pgfqpoint{2.994908in}{2.560076in}}%
\pgfpathlineto{\pgfqpoint{3.002719in}{2.565962in}}%
\pgfpathlineto{\pgfqpoint{3.010523in}{2.571912in}}%
\pgfpathlineto{\pgfqpoint{3.018320in}{2.577924in}}%
\pgfpathlineto{\pgfqpoint{3.026109in}{2.583998in}}%
\pgfpathclose%
\pgfusepath{fill}%
\end{pgfscope}%
\begin{pgfscope}%
\pgfpathrectangle{\pgfqpoint{1.254980in}{0.150000in}}{\pgfqpoint{5.490039in}{5.490039in}}%
\pgfusepath{clip}%
\pgfsetbuttcap%
\pgfsetroundjoin%
\definecolor{currentfill}{rgb}{0.272594,0.025563,0.353093}%
\pgfsetfillcolor{currentfill}%
\pgfsetfillopacity{0.700000}%
\pgfsetlinewidth{0.000000pt}%
\definecolor{currentstroke}{rgb}{0.000000,0.000000,0.000000}%
\pgfsetstrokecolor{currentstroke}%
\pgfsetdash{}{0pt}%
\pgfpathmoveto{\pgfqpoint{4.242551in}{2.542470in}}%
\pgfpathlineto{\pgfqpoint{4.255383in}{2.539135in}}%
\pgfpathlineto{\pgfqpoint{4.268221in}{2.535828in}}%
\pgfpathlineto{\pgfqpoint{4.281065in}{2.532550in}}%
\pgfpathlineto{\pgfqpoint{4.293915in}{2.529300in}}%
\pgfpathlineto{\pgfqpoint{4.286600in}{2.522080in}}%
\pgfpathlineto{\pgfqpoint{4.279279in}{2.514870in}}%
\pgfpathlineto{\pgfqpoint{4.271954in}{2.507666in}}%
\pgfpathlineto{\pgfqpoint{4.264623in}{2.500467in}}%
\pgfpathlineto{\pgfqpoint{4.251760in}{2.503669in}}%
\pgfpathlineto{\pgfqpoint{4.238904in}{2.506900in}}%
\pgfpathlineto{\pgfqpoint{4.226054in}{2.510159in}}%
\pgfpathlineto{\pgfqpoint{4.213210in}{2.513446in}}%
\pgfpathlineto{\pgfqpoint{4.220553in}{2.520688in}}%
\pgfpathlineto{\pgfqpoint{4.227891in}{2.527938in}}%
\pgfpathlineto{\pgfqpoint{4.235223in}{2.535198in}}%
\pgfpathlineto{\pgfqpoint{4.242551in}{2.542470in}}%
\pgfpathclose%
\pgfusepath{fill}%
\end{pgfscope}%
\begin{pgfscope}%
\pgfpathrectangle{\pgfqpoint{1.254980in}{0.150000in}}{\pgfqpoint{5.490039in}{5.490039in}}%
\pgfusepath{clip}%
\pgfsetbuttcap%
\pgfsetroundjoin%
\definecolor{currentfill}{rgb}{0.269944,0.014625,0.341379}%
\pgfsetfillcolor{currentfill}%
\pgfsetfillopacity{0.700000}%
\pgfsetlinewidth{0.000000pt}%
\definecolor{currentstroke}{rgb}{0.000000,0.000000,0.000000}%
\pgfsetstrokecolor{currentstroke}%
\pgfsetdash{}{0pt}%
\pgfpathmoveto{\pgfqpoint{3.553598in}{2.523064in}}%
\pgfpathlineto{\pgfqpoint{3.566288in}{2.518548in}}%
\pgfpathlineto{\pgfqpoint{3.578983in}{2.514067in}}%
\pgfpathlineto{\pgfqpoint{3.591683in}{2.509622in}}%
\pgfpathlineto{\pgfqpoint{3.604388in}{2.505213in}}%
\pgfpathlineto{\pgfqpoint{3.596823in}{2.498092in}}%
\pgfpathlineto{\pgfqpoint{3.589252in}{2.490991in}}%
\pgfpathlineto{\pgfqpoint{3.581676in}{2.483908in}}%
\pgfpathlineto{\pgfqpoint{3.574093in}{2.476846in}}%
\pgfpathlineto{\pgfqpoint{3.561376in}{2.481283in}}%
\pgfpathlineto{\pgfqpoint{3.548664in}{2.485756in}}%
\pgfpathlineto{\pgfqpoint{3.535956in}{2.490265in}}%
\pgfpathlineto{\pgfqpoint{3.523253in}{2.494811in}}%
\pgfpathlineto{\pgfqpoint{3.530848in}{2.501840in}}%
\pgfpathlineto{\pgfqpoint{3.538437in}{2.508893in}}%
\pgfpathlineto{\pgfqpoint{3.546020in}{2.515968in}}%
\pgfpathlineto{\pgfqpoint{3.553598in}{2.523064in}}%
\pgfpathclose%
\pgfusepath{fill}%
\end{pgfscope}%
\begin{pgfscope}%
\pgfpathrectangle{\pgfqpoint{1.254980in}{0.150000in}}{\pgfqpoint{5.490039in}{5.490039in}}%
\pgfusepath{clip}%
\pgfsetbuttcap%
\pgfsetroundjoin%
\definecolor{currentfill}{rgb}{0.280267,0.073417,0.397163}%
\pgfsetfillcolor{currentfill}%
\pgfsetfillopacity{0.700000}%
\pgfsetlinewidth{0.000000pt}%
\definecolor{currentstroke}{rgb}{0.000000,0.000000,0.000000}%
\pgfsetstrokecolor{currentstroke}%
\pgfsetdash{}{0pt}%
\pgfpathmoveto{\pgfqpoint{5.012191in}{2.609882in}}%
\pgfpathlineto{\pgfqpoint{5.025205in}{2.607018in}}%
\pgfpathlineto{\pgfqpoint{5.038225in}{2.604178in}}%
\pgfpathlineto{\pgfqpoint{5.051252in}{2.601363in}}%
\pgfpathlineto{\pgfqpoint{5.064285in}{2.598572in}}%
\pgfpathlineto{\pgfqpoint{5.057251in}{2.591533in}}%
\pgfpathlineto{\pgfqpoint{5.050212in}{2.584557in}}%
\pgfpathlineto{\pgfqpoint{5.043170in}{2.577639in}}%
\pgfpathlineto{\pgfqpoint{5.036124in}{2.570776in}}%
\pgfpathlineto{\pgfqpoint{5.023075in}{2.573430in}}%
\pgfpathlineto{\pgfqpoint{5.010033in}{2.576110in}}%
\pgfpathlineto{\pgfqpoint{4.996998in}{2.578814in}}%
\pgfpathlineto{\pgfqpoint{4.983969in}{2.581542in}}%
\pgfpathlineto{\pgfqpoint{4.991030in}{2.588537in}}%
\pgfpathlineto{\pgfqpoint{4.998088in}{2.595589in}}%
\pgfpathlineto{\pgfqpoint{5.005141in}{2.602703in}}%
\pgfpathlineto{\pgfqpoint{5.012191in}{2.609882in}}%
\pgfpathclose%
\pgfusepath{fill}%
\end{pgfscope}%
\begin{pgfscope}%
\pgfpathrectangle{\pgfqpoint{1.254980in}{0.150000in}}{\pgfqpoint{5.490039in}{5.490039in}}%
\pgfusepath{clip}%
\pgfsetbuttcap%
\pgfsetroundjoin%
\definecolor{currentfill}{rgb}{0.277941,0.056324,0.381191}%
\pgfsetfillcolor{currentfill}%
\pgfsetfillopacity{0.700000}%
\pgfsetlinewidth{0.000000pt}%
\definecolor{currentstroke}{rgb}{0.000000,0.000000,0.000000}%
\pgfsetstrokecolor{currentstroke}%
\pgfsetdash{}{0pt}%
\pgfpathmoveto{\pgfqpoint{4.799672in}{2.587248in}}%
\pgfpathlineto{\pgfqpoint{4.812637in}{2.584344in}}%
\pgfpathlineto{\pgfqpoint{4.825609in}{2.581465in}}%
\pgfpathlineto{\pgfqpoint{4.838586in}{2.578611in}}%
\pgfpathlineto{\pgfqpoint{4.851571in}{2.575782in}}%
\pgfpathlineto{\pgfqpoint{4.844459in}{2.568781in}}%
\pgfpathlineto{\pgfqpoint{4.837342in}{2.561820in}}%
\pgfpathlineto{\pgfqpoint{4.830221in}{2.554894in}}%
\pgfpathlineto{\pgfqpoint{4.823095in}{2.548002in}}%
\pgfpathlineto{\pgfqpoint{4.810097in}{2.550719in}}%
\pgfpathlineto{\pgfqpoint{4.797105in}{2.553462in}}%
\pgfpathlineto{\pgfqpoint{4.784120in}{2.556231in}}%
\pgfpathlineto{\pgfqpoint{4.771141in}{2.559024in}}%
\pgfpathlineto{\pgfqpoint{4.778281in}{2.566023in}}%
\pgfpathlineto{\pgfqpoint{4.785416in}{2.573058in}}%
\pgfpathlineto{\pgfqpoint{4.792546in}{2.580132in}}%
\pgfpathlineto{\pgfqpoint{4.799672in}{2.587248in}}%
\pgfpathclose%
\pgfusepath{fill}%
\end{pgfscope}%
\begin{pgfscope}%
\pgfpathrectangle{\pgfqpoint{1.254980in}{0.150000in}}{\pgfqpoint{5.490039in}{5.490039in}}%
\pgfusepath{clip}%
\pgfsetbuttcap%
\pgfsetroundjoin%
\definecolor{currentfill}{rgb}{0.271305,0.019942,0.347269}%
\pgfsetfillcolor{currentfill}%
\pgfsetfillopacity{0.700000}%
\pgfsetlinewidth{0.000000pt}%
\definecolor{currentstroke}{rgb}{0.000000,0.000000,0.000000}%
\pgfsetstrokecolor{currentstroke}%
\pgfsetdash{}{0pt}%
\pgfpathmoveto{\pgfqpoint{4.029959in}{2.525830in}}%
\pgfpathlineto{\pgfqpoint{4.042746in}{2.522225in}}%
\pgfpathlineto{\pgfqpoint{4.055539in}{2.518650in}}%
\pgfpathlineto{\pgfqpoint{4.068338in}{2.515105in}}%
\pgfpathlineto{\pgfqpoint{4.081142in}{2.511589in}}%
\pgfpathlineto{\pgfqpoint{4.073749in}{2.504310in}}%
\pgfpathlineto{\pgfqpoint{4.066350in}{2.497037in}}%
\pgfpathlineto{\pgfqpoint{4.058946in}{2.489767in}}%
\pgfpathlineto{\pgfqpoint{4.051536in}{2.482501in}}%
\pgfpathlineto{\pgfqpoint{4.038720in}{2.485994in}}%
\pgfpathlineto{\pgfqpoint{4.025909in}{2.489517in}}%
\pgfpathlineto{\pgfqpoint{4.013105in}{2.493070in}}%
\pgfpathlineto{\pgfqpoint{4.000305in}{2.496653in}}%
\pgfpathlineto{\pgfqpoint{4.007727in}{2.503937in}}%
\pgfpathlineto{\pgfqpoint{4.015143in}{2.511227in}}%
\pgfpathlineto{\pgfqpoint{4.022553in}{2.518524in}}%
\pgfpathlineto{\pgfqpoint{4.029959in}{2.525830in}}%
\pgfpathclose%
\pgfusepath{fill}%
\end{pgfscope}%
\begin{pgfscope}%
\pgfpathrectangle{\pgfqpoint{1.254980in}{0.150000in}}{\pgfqpoint{5.490039in}{5.490039in}}%
\pgfusepath{clip}%
\pgfsetbuttcap%
\pgfsetroundjoin%
\definecolor{currentfill}{rgb}{0.269944,0.014625,0.341379}%
\pgfsetfillcolor{currentfill}%
\pgfsetfillopacity{0.700000}%
\pgfsetlinewidth{0.000000pt}%
\definecolor{currentstroke}{rgb}{0.000000,0.000000,0.000000}%
\pgfsetstrokecolor{currentstroke}%
\pgfsetdash{}{0pt}%
\pgfpathmoveto{\pgfqpoint{3.685410in}{2.516660in}}%
\pgfpathlineto{\pgfqpoint{3.698127in}{2.512440in}}%
\pgfpathlineto{\pgfqpoint{3.710849in}{2.508255in}}%
\pgfpathlineto{\pgfqpoint{3.723577in}{2.504104in}}%
\pgfpathlineto{\pgfqpoint{3.736309in}{2.499986in}}%
\pgfpathlineto{\pgfqpoint{3.728791in}{2.492767in}}%
\pgfpathlineto{\pgfqpoint{3.721267in}{2.485562in}}%
\pgfpathlineto{\pgfqpoint{3.713738in}{2.478369in}}%
\pgfpathlineto{\pgfqpoint{3.706203in}{2.471189in}}%
\pgfpathlineto{\pgfqpoint{3.693459in}{2.475323in}}%
\pgfpathlineto{\pgfqpoint{3.680719in}{2.479490in}}%
\pgfpathlineto{\pgfqpoint{3.667985in}{2.483691in}}%
\pgfpathlineto{\pgfqpoint{3.655256in}{2.487926in}}%
\pgfpathlineto{\pgfqpoint{3.662803in}{2.495086in}}%
\pgfpathlineto{\pgfqpoint{3.670344in}{2.502261in}}%
\pgfpathlineto{\pgfqpoint{3.677880in}{2.509453in}}%
\pgfpathlineto{\pgfqpoint{3.685410in}{2.516660in}}%
\pgfpathclose%
\pgfusepath{fill}%
\end{pgfscope}%
\begin{pgfscope}%
\pgfpathrectangle{\pgfqpoint{1.254980in}{0.150000in}}{\pgfqpoint{5.490039in}{5.490039in}}%
\pgfusepath{clip}%
\pgfsetbuttcap%
\pgfsetroundjoin%
\definecolor{currentfill}{rgb}{0.276022,0.044167,0.370164}%
\pgfsetfillcolor{currentfill}%
\pgfsetfillopacity{0.700000}%
\pgfsetlinewidth{0.000000pt}%
\definecolor{currentstroke}{rgb}{0.000000,0.000000,0.000000}%
\pgfsetstrokecolor{currentstroke}%
\pgfsetdash{}{0pt}%
\pgfpathmoveto{\pgfqpoint{4.587124in}{2.565667in}}%
\pgfpathlineto{\pgfqpoint{4.600040in}{2.562662in}}%
\pgfpathlineto{\pgfqpoint{4.612963in}{2.559682in}}%
\pgfpathlineto{\pgfqpoint{4.625891in}{2.556729in}}%
\pgfpathlineto{\pgfqpoint{4.638826in}{2.553802in}}%
\pgfpathlineto{\pgfqpoint{4.631635in}{2.546750in}}%
\pgfpathlineto{\pgfqpoint{4.624439in}{2.539721in}}%
\pgfpathlineto{\pgfqpoint{4.617239in}{2.532711in}}%
\pgfpathlineto{\pgfqpoint{4.610033in}{2.525718in}}%
\pgfpathlineto{\pgfqpoint{4.597085in}{2.528560in}}%
\pgfpathlineto{\pgfqpoint{4.584143in}{2.531427in}}%
\pgfpathlineto{\pgfqpoint{4.571208in}{2.534321in}}%
\pgfpathlineto{\pgfqpoint{4.558279in}{2.537241in}}%
\pgfpathlineto{\pgfqpoint{4.565498in}{2.544315in}}%
\pgfpathlineto{\pgfqpoint{4.572712in}{2.551409in}}%
\pgfpathlineto{\pgfqpoint{4.579920in}{2.558525in}}%
\pgfpathlineto{\pgfqpoint{4.587124in}{2.565667in}}%
\pgfpathclose%
\pgfusepath{fill}%
\end{pgfscope}%
\begin{pgfscope}%
\pgfpathrectangle{\pgfqpoint{1.254980in}{0.150000in}}{\pgfqpoint{5.490039in}{5.490039in}}%
\pgfusepath{clip}%
\pgfsetbuttcap%
\pgfsetroundjoin%
\definecolor{currentfill}{rgb}{0.273809,0.031497,0.358853}%
\pgfsetfillcolor{currentfill}%
\pgfsetfillopacity{0.700000}%
\pgfsetlinewidth{0.000000pt}%
\definecolor{currentstroke}{rgb}{0.000000,0.000000,0.000000}%
\pgfsetstrokecolor{currentstroke}%
\pgfsetdash{}{0pt}%
\pgfpathmoveto{\pgfqpoint{4.374535in}{2.545345in}}%
\pgfpathlineto{\pgfqpoint{4.387403in}{2.542174in}}%
\pgfpathlineto{\pgfqpoint{4.400277in}{2.539031in}}%
\pgfpathlineto{\pgfqpoint{4.413157in}{2.535915in}}%
\pgfpathlineto{\pgfqpoint{4.426043in}{2.532826in}}%
\pgfpathlineto{\pgfqpoint{4.418773in}{2.525681in}}%
\pgfpathlineto{\pgfqpoint{4.411498in}{2.518547in}}%
\pgfpathlineto{\pgfqpoint{4.404218in}{2.511422in}}%
\pgfpathlineto{\pgfqpoint{4.396932in}{2.504303in}}%
\pgfpathlineto{\pgfqpoint{4.384034in}{2.507331in}}%
\pgfpathlineto{\pgfqpoint{4.371141in}{2.510387in}}%
\pgfpathlineto{\pgfqpoint{4.358255in}{2.513470in}}%
\pgfpathlineto{\pgfqpoint{4.345375in}{2.516580in}}%
\pgfpathlineto{\pgfqpoint{4.352673in}{2.523755in}}%
\pgfpathlineto{\pgfqpoint{4.359966in}{2.530939in}}%
\pgfpathlineto{\pgfqpoint{4.367253in}{2.538135in}}%
\pgfpathlineto{\pgfqpoint{4.374535in}{2.545345in}}%
\pgfpathclose%
\pgfusepath{fill}%
\end{pgfscope}%
\begin{pgfscope}%
\pgfpathrectangle{\pgfqpoint{1.254980in}{0.150000in}}{\pgfqpoint{5.490039in}{5.490039in}}%
\pgfusepath{clip}%
\pgfsetbuttcap%
\pgfsetroundjoin%
\definecolor{currentfill}{rgb}{0.269944,0.014625,0.341379}%
\pgfsetfillcolor{currentfill}%
\pgfsetfillopacity{0.700000}%
\pgfsetlinewidth{0.000000pt}%
\definecolor{currentstroke}{rgb}{0.000000,0.000000,0.000000}%
\pgfsetstrokecolor{currentstroke}%
\pgfsetdash{}{0pt}%
\pgfpathmoveto{\pgfqpoint{3.817258in}{2.512865in}}%
\pgfpathlineto{\pgfqpoint{3.830004in}{2.508915in}}%
\pgfpathlineto{\pgfqpoint{3.842756in}{2.504997in}}%
\pgfpathlineto{\pgfqpoint{3.855513in}{2.501112in}}%
\pgfpathlineto{\pgfqpoint{3.868275in}{2.497259in}}%
\pgfpathlineto{\pgfqpoint{3.860802in}{2.489990in}}%
\pgfpathlineto{\pgfqpoint{3.853325in}{2.482729in}}%
\pgfpathlineto{\pgfqpoint{3.845842in}{2.475475in}}%
\pgfpathlineto{\pgfqpoint{3.838353in}{2.468229in}}%
\pgfpathlineto{\pgfqpoint{3.825579in}{2.472085in}}%
\pgfpathlineto{\pgfqpoint{3.812811in}{2.475973in}}%
\pgfpathlineto{\pgfqpoint{3.800047in}{2.479894in}}%
\pgfpathlineto{\pgfqpoint{3.787289in}{2.483846in}}%
\pgfpathlineto{\pgfqpoint{3.794790in}{2.491085in}}%
\pgfpathlineto{\pgfqpoint{3.802285in}{2.498334in}}%
\pgfpathlineto{\pgfqpoint{3.809774in}{2.505594in}}%
\pgfpathlineto{\pgfqpoint{3.817258in}{2.512865in}}%
\pgfpathclose%
\pgfusepath{fill}%
\end{pgfscope}%
\begin{pgfscope}%
\pgfpathrectangle{\pgfqpoint{1.254980in}{0.150000in}}{\pgfqpoint{5.490039in}{5.490039in}}%
\pgfusepath{clip}%
\pgfsetbuttcap%
\pgfsetroundjoin%
\definecolor{currentfill}{rgb}{0.280894,0.078907,0.402329}%
\pgfsetfillcolor{currentfill}%
\pgfsetfillopacity{0.700000}%
\pgfsetlinewidth{0.000000pt}%
\definecolor{currentstroke}{rgb}{0.000000,0.000000,0.000000}%
\pgfsetstrokecolor{currentstroke}%
\pgfsetdash{}{0pt}%
\pgfpathmoveto{\pgfqpoint{5.144527in}{2.615930in}}%
\pgfpathlineto{\pgfqpoint{5.157578in}{2.613112in}}%
\pgfpathlineto{\pgfqpoint{5.170636in}{2.610318in}}%
\pgfpathlineto{\pgfqpoint{5.183700in}{2.607547in}}%
\pgfpathlineto{\pgfqpoint{5.196772in}{2.604801in}}%
\pgfpathlineto{\pgfqpoint{5.189783in}{2.597773in}}%
\pgfpathlineto{\pgfqpoint{5.182790in}{2.590819in}}%
\pgfpathlineto{\pgfqpoint{5.175794in}{2.583935in}}%
\pgfpathlineto{\pgfqpoint{5.168795in}{2.577117in}}%
\pgfpathlineto{\pgfqpoint{5.155707in}{2.579714in}}%
\pgfpathlineto{\pgfqpoint{5.142627in}{2.582336in}}%
\pgfpathlineto{\pgfqpoint{5.129553in}{2.584982in}}%
\pgfpathlineto{\pgfqpoint{5.116486in}{2.587651in}}%
\pgfpathlineto{\pgfqpoint{5.123502in}{2.594613in}}%
\pgfpathlineto{\pgfqpoint{5.130513in}{2.601644in}}%
\pgfpathlineto{\pgfqpoint{5.137522in}{2.608748in}}%
\pgfpathlineto{\pgfqpoint{5.144527in}{2.615930in}}%
\pgfpathclose%
\pgfusepath{fill}%
\end{pgfscope}%
\begin{pgfscope}%
\pgfpathrectangle{\pgfqpoint{1.254980in}{0.150000in}}{\pgfqpoint{5.490039in}{5.490039in}}%
\pgfusepath{clip}%
\pgfsetbuttcap%
\pgfsetroundjoin%
\definecolor{currentfill}{rgb}{0.272594,0.025563,0.353093}%
\pgfsetfillcolor{currentfill}%
\pgfsetfillopacity{0.700000}%
\pgfsetlinewidth{0.000000pt}%
\definecolor{currentstroke}{rgb}{0.000000,0.000000,0.000000}%
\pgfsetstrokecolor{currentstroke}%
\pgfsetdash{}{0pt}%
\pgfpathmoveto{\pgfqpoint{4.161890in}{2.526884in}}%
\pgfpathlineto{\pgfqpoint{4.174711in}{2.523481in}}%
\pgfpathlineto{\pgfqpoint{4.187538in}{2.520107in}}%
\pgfpathlineto{\pgfqpoint{4.200371in}{2.516762in}}%
\pgfpathlineto{\pgfqpoint{4.213210in}{2.513446in}}%
\pgfpathlineto{\pgfqpoint{4.205861in}{2.506211in}}%
\pgfpathlineto{\pgfqpoint{4.198507in}{2.498980in}}%
\pgfpathlineto{\pgfqpoint{4.191148in}{2.491753in}}%
\pgfpathlineto{\pgfqpoint{4.183784in}{2.484528in}}%
\pgfpathlineto{\pgfqpoint{4.170934in}{2.487809in}}%
\pgfpathlineto{\pgfqpoint{4.158089in}{2.491118in}}%
\pgfpathlineto{\pgfqpoint{4.145250in}{2.494457in}}%
\pgfpathlineto{\pgfqpoint{4.132417in}{2.497825in}}%
\pgfpathlineto{\pgfqpoint{4.139793in}{2.505081in}}%
\pgfpathlineto{\pgfqpoint{4.147164in}{2.512341in}}%
\pgfpathlineto{\pgfqpoint{4.154530in}{2.519608in}}%
\pgfpathlineto{\pgfqpoint{4.161890in}{2.526884in}}%
\pgfpathclose%
\pgfusepath{fill}%
\end{pgfscope}%
\begin{pgfscope}%
\pgfpathrectangle{\pgfqpoint{1.254980in}{0.150000in}}{\pgfqpoint{5.490039in}{5.490039in}}%
\pgfusepath{clip}%
\pgfsetbuttcap%
\pgfsetroundjoin%
\definecolor{currentfill}{rgb}{0.279566,0.067836,0.391917}%
\pgfsetfillcolor{currentfill}%
\pgfsetfillopacity{0.700000}%
\pgfsetlinewidth{0.000000pt}%
\definecolor{currentstroke}{rgb}{0.000000,0.000000,0.000000}%
\pgfsetstrokecolor{currentstroke}%
\pgfsetdash{}{0pt}%
\pgfpathmoveto{\pgfqpoint{4.931920in}{2.592702in}}%
\pgfpathlineto{\pgfqpoint{4.944922in}{2.589875in}}%
\pgfpathlineto{\pgfqpoint{4.957931in}{2.587073in}}%
\pgfpathlineto{\pgfqpoint{4.970947in}{2.584295in}}%
\pgfpathlineto{\pgfqpoint{4.983969in}{2.581542in}}%
\pgfpathlineto{\pgfqpoint{4.976903in}{2.574600in}}%
\pgfpathlineto{\pgfqpoint{4.969833in}{2.567707in}}%
\pgfpathlineto{\pgfqpoint{4.962759in}{2.560859in}}%
\pgfpathlineto{\pgfqpoint{4.955681in}{2.554052in}}%
\pgfpathlineto{\pgfqpoint{4.942644in}{2.556681in}}%
\pgfpathlineto{\pgfqpoint{4.929614in}{2.559336in}}%
\pgfpathlineto{\pgfqpoint{4.916590in}{2.562014in}}%
\pgfpathlineto{\pgfqpoint{4.903573in}{2.564718in}}%
\pgfpathlineto{\pgfqpoint{4.910666in}{2.571644in}}%
\pgfpathlineto{\pgfqpoint{4.917755in}{2.578614in}}%
\pgfpathlineto{\pgfqpoint{4.924839in}{2.585632in}}%
\pgfpathlineto{\pgfqpoint{4.931920in}{2.592702in}}%
\pgfpathclose%
\pgfusepath{fill}%
\end{pgfscope}%
\begin{pgfscope}%
\pgfpathrectangle{\pgfqpoint{1.254980in}{0.150000in}}{\pgfqpoint{5.490039in}{5.490039in}}%
\pgfusepath{clip}%
\pgfsetbuttcap%
\pgfsetroundjoin%
\definecolor{currentfill}{rgb}{0.273809,0.031497,0.358853}%
\pgfsetfillcolor{currentfill}%
\pgfsetfillopacity{0.700000}%
\pgfsetlinewidth{0.000000pt}%
\definecolor{currentstroke}{rgb}{0.000000,0.000000,0.000000}%
\pgfsetstrokecolor{currentstroke}%
\pgfsetdash{}{0pt}%
\pgfpathmoveto{\pgfqpoint{3.208626in}{2.540290in}}%
\pgfpathlineto{\pgfqpoint{3.221269in}{2.534864in}}%
\pgfpathlineto{\pgfqpoint{3.233916in}{2.529480in}}%
\pgfpathlineto{\pgfqpoint{3.246568in}{2.524138in}}%
\pgfpathlineto{\pgfqpoint{3.259223in}{2.518839in}}%
\pgfpathlineto{\pgfqpoint{3.251519in}{2.512287in}}%
\pgfpathlineto{\pgfqpoint{3.243809in}{2.505778in}}%
\pgfpathlineto{\pgfqpoint{3.236093in}{2.499313in}}%
\pgfpathlineto{\pgfqpoint{3.228370in}{2.492894in}}%
\pgfpathlineto{\pgfqpoint{3.215700in}{2.498260in}}%
\pgfpathlineto{\pgfqpoint{3.203035in}{2.503668in}}%
\pgfpathlineto{\pgfqpoint{3.190373in}{2.509118in}}%
\pgfpathlineto{\pgfqpoint{3.177716in}{2.514611in}}%
\pgfpathlineto{\pgfqpoint{3.185453in}{2.520959in}}%
\pgfpathlineto{\pgfqpoint{3.193184in}{2.527356in}}%
\pgfpathlineto{\pgfqpoint{3.200908in}{2.533800in}}%
\pgfpathlineto{\pgfqpoint{3.208626in}{2.540290in}}%
\pgfpathclose%
\pgfusepath{fill}%
\end{pgfscope}%
\begin{pgfscope}%
\pgfpathrectangle{\pgfqpoint{1.254980in}{0.150000in}}{\pgfqpoint{5.490039in}{5.490039in}}%
\pgfusepath{clip}%
\pgfsetbuttcap%
\pgfsetroundjoin%
\definecolor{currentfill}{rgb}{0.271305,0.019942,0.347269}%
\pgfsetfillcolor{currentfill}%
\pgfsetfillopacity{0.700000}%
\pgfsetlinewidth{0.000000pt}%
\definecolor{currentstroke}{rgb}{0.000000,0.000000,0.000000}%
\pgfsetstrokecolor{currentstroke}%
\pgfsetdash{}{0pt}%
\pgfpathmoveto{\pgfqpoint{3.340579in}{2.524882in}}%
\pgfpathlineto{\pgfqpoint{3.353241in}{2.519840in}}%
\pgfpathlineto{\pgfqpoint{3.365907in}{2.514837in}}%
\pgfpathlineto{\pgfqpoint{3.378578in}{2.509874in}}%
\pgfpathlineto{\pgfqpoint{3.391253in}{2.504950in}}%
\pgfpathlineto{\pgfqpoint{3.383602in}{2.498139in}}%
\pgfpathlineto{\pgfqpoint{3.375945in}{2.491361in}}%
\pgfpathlineto{\pgfqpoint{3.368282in}{2.484615in}}%
\pgfpathlineto{\pgfqpoint{3.360613in}{2.477905in}}%
\pgfpathlineto{\pgfqpoint{3.347924in}{2.482882in}}%
\pgfpathlineto{\pgfqpoint{3.335240in}{2.487899in}}%
\pgfpathlineto{\pgfqpoint{3.322560in}{2.492955in}}%
\pgfpathlineto{\pgfqpoint{3.309884in}{2.498051in}}%
\pgfpathlineto{\pgfqpoint{3.317567in}{2.504703in}}%
\pgfpathlineto{\pgfqpoint{3.325244in}{2.511393in}}%
\pgfpathlineto{\pgfqpoint{3.332914in}{2.518120in}}%
\pgfpathlineto{\pgfqpoint{3.340579in}{2.524882in}}%
\pgfpathclose%
\pgfusepath{fill}%
\end{pgfscope}%
\begin{pgfscope}%
\pgfpathrectangle{\pgfqpoint{1.254980in}{0.150000in}}{\pgfqpoint{5.490039in}{5.490039in}}%
\pgfusepath{clip}%
\pgfsetbuttcap%
\pgfsetroundjoin%
\definecolor{currentfill}{rgb}{0.277941,0.056324,0.381191}%
\pgfsetfillcolor{currentfill}%
\pgfsetfillopacity{0.700000}%
\pgfsetlinewidth{0.000000pt}%
\definecolor{currentstroke}{rgb}{0.000000,0.000000,0.000000}%
\pgfsetstrokecolor{currentstroke}%
\pgfsetdash{}{0pt}%
\pgfpathmoveto{\pgfqpoint{4.719290in}{2.570454in}}%
\pgfpathlineto{\pgfqpoint{4.732243in}{2.567558in}}%
\pgfpathlineto{\pgfqpoint{4.745203in}{2.564688in}}%
\pgfpathlineto{\pgfqpoint{4.758169in}{2.561843in}}%
\pgfpathlineto{\pgfqpoint{4.771141in}{2.559024in}}%
\pgfpathlineto{\pgfqpoint{4.763996in}{2.552058in}}%
\pgfpathlineto{\pgfqpoint{4.756847in}{2.545120in}}%
\pgfpathlineto{\pgfqpoint{4.749693in}{2.538207in}}%
\pgfpathlineto{\pgfqpoint{4.742534in}{2.531318in}}%
\pgfpathlineto{\pgfqpoint{4.729548in}{2.534038in}}%
\pgfpathlineto{\pgfqpoint{4.716568in}{2.536784in}}%
\pgfpathlineto{\pgfqpoint{4.703595in}{2.539556in}}%
\pgfpathlineto{\pgfqpoint{4.690629in}{2.542354in}}%
\pgfpathlineto{\pgfqpoint{4.697801in}{2.549337in}}%
\pgfpathlineto{\pgfqpoint{4.704969in}{2.556346in}}%
\pgfpathlineto{\pgfqpoint{4.712132in}{2.563384in}}%
\pgfpathlineto{\pgfqpoint{4.719290in}{2.570454in}}%
\pgfpathclose%
\pgfusepath{fill}%
\end{pgfscope}%
\begin{pgfscope}%
\pgfpathrectangle{\pgfqpoint{1.254980in}{0.150000in}}{\pgfqpoint{5.490039in}{5.490039in}}%
\pgfusepath{clip}%
\pgfsetbuttcap%
\pgfsetroundjoin%
\definecolor{currentfill}{rgb}{0.276022,0.044167,0.370164}%
\pgfsetfillcolor{currentfill}%
\pgfsetfillopacity{0.700000}%
\pgfsetlinewidth{0.000000pt}%
\definecolor{currentstroke}{rgb}{0.000000,0.000000,0.000000}%
\pgfsetstrokecolor{currentstroke}%
\pgfsetdash{}{0pt}%
\pgfpathmoveto{\pgfqpoint{3.076588in}{2.560136in}}%
\pgfpathlineto{\pgfqpoint{3.089217in}{2.554288in}}%
\pgfpathlineto{\pgfqpoint{3.101849in}{2.548485in}}%
\pgfpathlineto{\pgfqpoint{3.114484in}{2.542728in}}%
\pgfpathlineto{\pgfqpoint{3.127123in}{2.537017in}}%
\pgfpathlineto{\pgfqpoint{3.119364in}{2.530793in}}%
\pgfpathlineto{\pgfqpoint{3.111597in}{2.524625in}}%
\pgfpathlineto{\pgfqpoint{3.103824in}{2.518513in}}%
\pgfpathlineto{\pgfqpoint{3.096044in}{2.512460in}}%
\pgfpathlineto{\pgfqpoint{3.083389in}{2.518251in}}%
\pgfpathlineto{\pgfqpoint{3.070739in}{2.524087in}}%
\pgfpathlineto{\pgfqpoint{3.058092in}{2.529969in}}%
\pgfpathlineto{\pgfqpoint{3.045448in}{2.535897in}}%
\pgfpathlineto{\pgfqpoint{3.053244in}{2.541866in}}%
\pgfpathlineto{\pgfqpoint{3.061033in}{2.547897in}}%
\pgfpathlineto{\pgfqpoint{3.068814in}{2.553988in}}%
\pgfpathlineto{\pgfqpoint{3.076588in}{2.560136in}}%
\pgfpathclose%
\pgfusepath{fill}%
\end{pgfscope}%
\begin{pgfscope}%
\pgfpathrectangle{\pgfqpoint{1.254980in}{0.150000in}}{\pgfqpoint{5.490039in}{5.490039in}}%
\pgfusepath{clip}%
\pgfsetbuttcap%
\pgfsetroundjoin%
\definecolor{currentfill}{rgb}{0.269944,0.014625,0.341379}%
\pgfsetfillcolor{currentfill}%
\pgfsetfillopacity{0.700000}%
\pgfsetlinewidth{0.000000pt}%
\definecolor{currentstroke}{rgb}{0.000000,0.000000,0.000000}%
\pgfsetstrokecolor{currentstroke}%
\pgfsetdash{}{0pt}%
\pgfpathmoveto{\pgfqpoint{3.472488in}{2.513358in}}%
\pgfpathlineto{\pgfqpoint{3.485172in}{2.508666in}}%
\pgfpathlineto{\pgfqpoint{3.497861in}{2.504010in}}%
\pgfpathlineto{\pgfqpoint{3.510555in}{2.499392in}}%
\pgfpathlineto{\pgfqpoint{3.523253in}{2.494811in}}%
\pgfpathlineto{\pgfqpoint{3.515652in}{2.487804in}}%
\pgfpathlineto{\pgfqpoint{3.508046in}{2.480821in}}%
\pgfpathlineto{\pgfqpoint{3.500433in}{2.473862in}}%
\pgfpathlineto{\pgfqpoint{3.492814in}{2.466927in}}%
\pgfpathlineto{\pgfqpoint{3.480103in}{2.471550in}}%
\pgfpathlineto{\pgfqpoint{3.467397in}{2.476209in}}%
\pgfpathlineto{\pgfqpoint{3.454695in}{2.480905in}}%
\pgfpathlineto{\pgfqpoint{3.441998in}{2.485639in}}%
\pgfpathlineto{\pgfqpoint{3.449629in}{2.492527in}}%
\pgfpathlineto{\pgfqpoint{3.457255in}{2.499444in}}%
\pgfpathlineto{\pgfqpoint{3.464874in}{2.506388in}}%
\pgfpathlineto{\pgfqpoint{3.472488in}{2.513358in}}%
\pgfpathclose%
\pgfusepath{fill}%
\end{pgfscope}%
\begin{pgfscope}%
\pgfpathrectangle{\pgfqpoint{1.254980in}{0.150000in}}{\pgfqpoint{5.490039in}{5.490039in}}%
\pgfusepath{clip}%
\pgfsetbuttcap%
\pgfsetroundjoin%
\definecolor{currentfill}{rgb}{0.269944,0.014625,0.341379}%
\pgfsetfillcolor{currentfill}%
\pgfsetfillopacity{0.700000}%
\pgfsetlinewidth{0.000000pt}%
\definecolor{currentstroke}{rgb}{0.000000,0.000000,0.000000}%
\pgfsetstrokecolor{currentstroke}%
\pgfsetdash{}{0pt}%
\pgfpathmoveto{\pgfqpoint{3.949163in}{2.511291in}}%
\pgfpathlineto{\pgfqpoint{3.961940in}{2.507585in}}%
\pgfpathlineto{\pgfqpoint{3.974723in}{2.503910in}}%
\pgfpathlineto{\pgfqpoint{3.987511in}{2.500266in}}%
\pgfpathlineto{\pgfqpoint{4.000305in}{2.496653in}}%
\pgfpathlineto{\pgfqpoint{3.992878in}{2.489375in}}%
\pgfpathlineto{\pgfqpoint{3.985446in}{2.482101in}}%
\pgfpathlineto{\pgfqpoint{3.978009in}{2.474830in}}%
\pgfpathlineto{\pgfqpoint{3.970566in}{2.467563in}}%
\pgfpathlineto{\pgfqpoint{3.957760in}{2.471167in}}%
\pgfpathlineto{\pgfqpoint{3.944960in}{2.474801in}}%
\pgfpathlineto{\pgfqpoint{3.932165in}{2.478466in}}%
\pgfpathlineto{\pgfqpoint{3.919376in}{2.482162in}}%
\pgfpathlineto{\pgfqpoint{3.926831in}{2.489435in}}%
\pgfpathlineto{\pgfqpoint{3.934281in}{2.496713in}}%
\pgfpathlineto{\pgfqpoint{3.941725in}{2.503998in}}%
\pgfpathlineto{\pgfqpoint{3.949163in}{2.511291in}}%
\pgfpathclose%
\pgfusepath{fill}%
\end{pgfscope}%
\begin{pgfscope}%
\pgfpathrectangle{\pgfqpoint{1.254980in}{0.150000in}}{\pgfqpoint{5.490039in}{5.490039in}}%
\pgfusepath{clip}%
\pgfsetbuttcap%
\pgfsetroundjoin%
\definecolor{currentfill}{rgb}{0.276022,0.044167,0.370164}%
\pgfsetfillcolor{currentfill}%
\pgfsetfillopacity{0.700000}%
\pgfsetlinewidth{0.000000pt}%
\definecolor{currentstroke}{rgb}{0.000000,0.000000,0.000000}%
\pgfsetstrokecolor{currentstroke}%
\pgfsetdash{}{0pt}%
\pgfpathmoveto{\pgfqpoint{4.506625in}{2.549187in}}%
\pgfpathlineto{\pgfqpoint{4.519529in}{2.546161in}}%
\pgfpathlineto{\pgfqpoint{4.532440in}{2.543161in}}%
\pgfpathlineto{\pgfqpoint{4.545356in}{2.540188in}}%
\pgfpathlineto{\pgfqpoint{4.558279in}{2.537241in}}%
\pgfpathlineto{\pgfqpoint{4.551055in}{2.530184in}}%
\pgfpathlineto{\pgfqpoint{4.543826in}{2.523142in}}%
\pgfpathlineto{\pgfqpoint{4.536592in}{2.516111in}}%
\pgfpathlineto{\pgfqpoint{4.529352in}{2.509090in}}%
\pgfpathlineto{\pgfqpoint{4.516417in}{2.511963in}}%
\pgfpathlineto{\pgfqpoint{4.503487in}{2.514863in}}%
\pgfpathlineto{\pgfqpoint{4.490564in}{2.517790in}}%
\pgfpathlineto{\pgfqpoint{4.477648in}{2.520743in}}%
\pgfpathlineto{\pgfqpoint{4.484900in}{2.527833in}}%
\pgfpathlineto{\pgfqpoint{4.492147in}{2.534935in}}%
\pgfpathlineto{\pgfqpoint{4.499389in}{2.542052in}}%
\pgfpathlineto{\pgfqpoint{4.506625in}{2.549187in}}%
\pgfpathclose%
\pgfusepath{fill}%
\end{pgfscope}%
\begin{pgfscope}%
\pgfpathrectangle{\pgfqpoint{1.254980in}{0.150000in}}{\pgfqpoint{5.490039in}{5.490039in}}%
\pgfusepath{clip}%
\pgfsetbuttcap%
\pgfsetroundjoin%
\definecolor{currentfill}{rgb}{0.278791,0.062145,0.386592}%
\pgfsetfillcolor{currentfill}%
\pgfsetfillopacity{0.700000}%
\pgfsetlinewidth{0.000000pt}%
\definecolor{currentstroke}{rgb}{0.000000,0.000000,0.000000}%
\pgfsetstrokecolor{currentstroke}%
\pgfsetdash{}{0pt}%
\pgfpathmoveto{\pgfqpoint{2.944419in}{2.585029in}}%
\pgfpathlineto{\pgfqpoint{2.957036in}{2.578717in}}%
\pgfpathlineto{\pgfqpoint{2.969657in}{2.572454in}}%
\pgfpathlineto{\pgfqpoint{2.982281in}{2.566241in}}%
\pgfpathlineto{\pgfqpoint{2.994908in}{2.560076in}}%
\pgfpathlineto{\pgfqpoint{2.987089in}{2.554257in}}%
\pgfpathlineto{\pgfqpoint{2.979262in}{2.548507in}}%
\pgfpathlineto{\pgfqpoint{2.971427in}{2.542826in}}%
\pgfpathlineto{\pgfqpoint{2.963585in}{2.537217in}}%
\pgfpathlineto{\pgfqpoint{2.950942in}{2.543474in}}%
\pgfpathlineto{\pgfqpoint{2.938302in}{2.549780in}}%
\pgfpathlineto{\pgfqpoint{2.925665in}{2.556135in}}%
\pgfpathlineto{\pgfqpoint{2.913031in}{2.562539in}}%
\pgfpathlineto{\pgfqpoint{2.920890in}{2.568051in}}%
\pgfpathlineto{\pgfqpoint{2.928741in}{2.573638in}}%
\pgfpathlineto{\pgfqpoint{2.936584in}{2.579298in}}%
\pgfpathlineto{\pgfqpoint{2.944419in}{2.585029in}}%
\pgfpathclose%
\pgfusepath{fill}%
\end{pgfscope}%
\begin{pgfscope}%
\pgfpathrectangle{\pgfqpoint{1.254980in}{0.150000in}}{\pgfqpoint{5.490039in}{5.490039in}}%
\pgfusepath{clip}%
\pgfsetbuttcap%
\pgfsetroundjoin%
\definecolor{currentfill}{rgb}{0.269944,0.014625,0.341379}%
\pgfsetfillcolor{currentfill}%
\pgfsetfillopacity{0.700000}%
\pgfsetlinewidth{0.000000pt}%
\definecolor{currentstroke}{rgb}{0.000000,0.000000,0.000000}%
\pgfsetstrokecolor{currentstroke}%
\pgfsetdash{}{0pt}%
\pgfpathmoveto{\pgfqpoint{3.604388in}{2.505213in}}%
\pgfpathlineto{\pgfqpoint{3.617098in}{2.500839in}}%
\pgfpathlineto{\pgfqpoint{3.629812in}{2.496500in}}%
\pgfpathlineto{\pgfqpoint{3.642531in}{2.492196in}}%
\pgfpathlineto{\pgfqpoint{3.655256in}{2.487926in}}%
\pgfpathlineto{\pgfqpoint{3.647703in}{2.480782in}}%
\pgfpathlineto{\pgfqpoint{3.640145in}{2.473654in}}%
\pgfpathlineto{\pgfqpoint{3.632580in}{2.466542in}}%
\pgfpathlineto{\pgfqpoint{3.625011in}{2.459446in}}%
\pgfpathlineto{\pgfqpoint{3.612274in}{2.463744in}}%
\pgfpathlineto{\pgfqpoint{3.599542in}{2.468076in}}%
\pgfpathlineto{\pgfqpoint{3.586815in}{2.472443in}}%
\pgfpathlineto{\pgfqpoint{3.574093in}{2.476846in}}%
\pgfpathlineto{\pgfqpoint{3.581676in}{2.483908in}}%
\pgfpathlineto{\pgfqpoint{3.589252in}{2.490991in}}%
\pgfpathlineto{\pgfqpoint{3.596823in}{2.498092in}}%
\pgfpathlineto{\pgfqpoint{3.604388in}{2.505213in}}%
\pgfpathclose%
\pgfusepath{fill}%
\end{pgfscope}%
\begin{pgfscope}%
\pgfpathrectangle{\pgfqpoint{1.254980in}{0.150000in}}{\pgfqpoint{5.490039in}{5.490039in}}%
\pgfusepath{clip}%
\pgfsetbuttcap%
\pgfsetroundjoin%
\definecolor{currentfill}{rgb}{0.273809,0.031497,0.358853}%
\pgfsetfillcolor{currentfill}%
\pgfsetfillopacity{0.700000}%
\pgfsetlinewidth{0.000000pt}%
\definecolor{currentstroke}{rgb}{0.000000,0.000000,0.000000}%
\pgfsetstrokecolor{currentstroke}%
\pgfsetdash{}{0pt}%
\pgfpathmoveto{\pgfqpoint{4.293915in}{2.529300in}}%
\pgfpathlineto{\pgfqpoint{4.306771in}{2.526078in}}%
\pgfpathlineto{\pgfqpoint{4.319633in}{2.522884in}}%
\pgfpathlineto{\pgfqpoint{4.332501in}{2.519718in}}%
\pgfpathlineto{\pgfqpoint{4.345375in}{2.516580in}}%
\pgfpathlineto{\pgfqpoint{4.338072in}{2.509413in}}%
\pgfpathlineto{\pgfqpoint{4.330764in}{2.502252in}}%
\pgfpathlineto{\pgfqpoint{4.323450in}{2.495094in}}%
\pgfpathlineto{\pgfqpoint{4.316132in}{2.487938in}}%
\pgfpathlineto{\pgfqpoint{4.303245in}{2.491029in}}%
\pgfpathlineto{\pgfqpoint{4.290365in}{2.494147in}}%
\pgfpathlineto{\pgfqpoint{4.277491in}{2.497293in}}%
\pgfpathlineto{\pgfqpoint{4.264623in}{2.500467in}}%
\pgfpathlineto{\pgfqpoint{4.271954in}{2.507666in}}%
\pgfpathlineto{\pgfqpoint{4.279279in}{2.514870in}}%
\pgfpathlineto{\pgfqpoint{4.286600in}{2.522080in}}%
\pgfpathlineto{\pgfqpoint{4.293915in}{2.529300in}}%
\pgfpathclose%
\pgfusepath{fill}%
\end{pgfscope}%
\begin{pgfscope}%
\pgfpathrectangle{\pgfqpoint{1.254980in}{0.150000in}}{\pgfqpoint{5.490039in}{5.490039in}}%
\pgfusepath{clip}%
\pgfsetbuttcap%
\pgfsetroundjoin%
\definecolor{currentfill}{rgb}{0.281924,0.089666,0.412415}%
\pgfsetfillcolor{currentfill}%
\pgfsetfillopacity{0.700000}%
\pgfsetlinewidth{0.000000pt}%
\definecolor{currentstroke}{rgb}{0.000000,0.000000,0.000000}%
\pgfsetstrokecolor{currentstroke}%
\pgfsetdash{}{0pt}%
\pgfpathmoveto{\pgfqpoint{5.276985in}{2.622371in}}%
\pgfpathlineto{\pgfqpoint{5.290074in}{2.619583in}}%
\pgfpathlineto{\pgfqpoint{5.303169in}{2.616819in}}%
\pgfpathlineto{\pgfqpoint{5.316272in}{2.614078in}}%
\pgfpathlineto{\pgfqpoint{5.329381in}{2.611360in}}%
\pgfpathlineto{\pgfqpoint{5.322437in}{2.604315in}}%
\pgfpathlineto{\pgfqpoint{5.315489in}{2.597358in}}%
\pgfpathlineto{\pgfqpoint{5.308540in}{2.590485in}}%
\pgfpathlineto{\pgfqpoint{5.301586in}{2.583690in}}%
\pgfpathlineto{\pgfqpoint{5.288461in}{2.586246in}}%
\pgfpathlineto{\pgfqpoint{5.275342in}{2.588825in}}%
\pgfpathlineto{\pgfqpoint{5.262230in}{2.591428in}}%
\pgfpathlineto{\pgfqpoint{5.249124in}{2.594055in}}%
\pgfpathlineto{\pgfqpoint{5.256094in}{2.601007in}}%
\pgfpathlineto{\pgfqpoint{5.263061in}{2.608040in}}%
\pgfpathlineto{\pgfqpoint{5.270024in}{2.615160in}}%
\pgfpathlineto{\pgfqpoint{5.276985in}{2.622371in}}%
\pgfpathclose%
\pgfusepath{fill}%
\end{pgfscope}%
\begin{pgfscope}%
\pgfpathrectangle{\pgfqpoint{1.254980in}{0.150000in}}{\pgfqpoint{5.490039in}{5.490039in}}%
\pgfusepath{clip}%
\pgfsetbuttcap%
\pgfsetroundjoin%
\definecolor{currentfill}{rgb}{0.280267,0.073417,0.397163}%
\pgfsetfillcolor{currentfill}%
\pgfsetfillopacity{0.700000}%
\pgfsetlinewidth{0.000000pt}%
\definecolor{currentstroke}{rgb}{0.000000,0.000000,0.000000}%
\pgfsetstrokecolor{currentstroke}%
\pgfsetdash{}{0pt}%
\pgfpathmoveto{\pgfqpoint{5.064285in}{2.598572in}}%
\pgfpathlineto{\pgfqpoint{5.077326in}{2.595805in}}%
\pgfpathlineto{\pgfqpoint{5.090372in}{2.593063in}}%
\pgfpathlineto{\pgfqpoint{5.103426in}{2.590345in}}%
\pgfpathlineto{\pgfqpoint{5.116486in}{2.587651in}}%
\pgfpathlineto{\pgfqpoint{5.109467in}{2.580753in}}%
\pgfpathlineto{\pgfqpoint{5.102444in}{2.573915in}}%
\pgfpathlineto{\pgfqpoint{5.095417in}{2.567132in}}%
\pgfpathlineto{\pgfqpoint{5.088386in}{2.560400in}}%
\pgfpathlineto{\pgfqpoint{5.075310in}{2.562957in}}%
\pgfpathlineto{\pgfqpoint{5.062241in}{2.565539in}}%
\pgfpathlineto{\pgfqpoint{5.049179in}{2.568145in}}%
\pgfpathlineto{\pgfqpoint{5.036124in}{2.570776in}}%
\pgfpathlineto{\pgfqpoint{5.043170in}{2.577639in}}%
\pgfpathlineto{\pgfqpoint{5.050212in}{2.584557in}}%
\pgfpathlineto{\pgfqpoint{5.057251in}{2.591533in}}%
\pgfpathlineto{\pgfqpoint{5.064285in}{2.598572in}}%
\pgfpathclose%
\pgfusepath{fill}%
\end{pgfscope}%
\begin{pgfscope}%
\pgfpathrectangle{\pgfqpoint{1.254980in}{0.150000in}}{\pgfqpoint{5.490039in}{5.490039in}}%
\pgfusepath{clip}%
\pgfsetbuttcap%
\pgfsetroundjoin%
\definecolor{currentfill}{rgb}{0.269944,0.014625,0.341379}%
\pgfsetfillcolor{currentfill}%
\pgfsetfillopacity{0.700000}%
\pgfsetlinewidth{0.000000pt}%
\definecolor{currentstroke}{rgb}{0.000000,0.000000,0.000000}%
\pgfsetstrokecolor{currentstroke}%
\pgfsetdash{}{0pt}%
\pgfpathmoveto{\pgfqpoint{3.736309in}{2.499986in}}%
\pgfpathlineto{\pgfqpoint{3.749046in}{2.495901in}}%
\pgfpathlineto{\pgfqpoint{3.761789in}{2.491850in}}%
\pgfpathlineto{\pgfqpoint{3.774536in}{2.487832in}}%
\pgfpathlineto{\pgfqpoint{3.787289in}{2.483846in}}%
\pgfpathlineto{\pgfqpoint{3.779783in}{2.476617in}}%
\pgfpathlineto{\pgfqpoint{3.772272in}{2.469398in}}%
\pgfpathlineto{\pgfqpoint{3.764754in}{2.462188in}}%
\pgfpathlineto{\pgfqpoint{3.757232in}{2.454988in}}%
\pgfpathlineto{\pgfqpoint{3.744467in}{2.458989in}}%
\pgfpathlineto{\pgfqpoint{3.731707in}{2.463023in}}%
\pgfpathlineto{\pgfqpoint{3.718952in}{2.467089in}}%
\pgfpathlineto{\pgfqpoint{3.706203in}{2.471189in}}%
\pgfpathlineto{\pgfqpoint{3.713738in}{2.478369in}}%
\pgfpathlineto{\pgfqpoint{3.721267in}{2.485562in}}%
\pgfpathlineto{\pgfqpoint{3.728791in}{2.492767in}}%
\pgfpathlineto{\pgfqpoint{3.736309in}{2.499986in}}%
\pgfpathclose%
\pgfusepath{fill}%
\end{pgfscope}%
\begin{pgfscope}%
\pgfpathrectangle{\pgfqpoint{1.254980in}{0.150000in}}{\pgfqpoint{5.490039in}{5.490039in}}%
\pgfusepath{clip}%
\pgfsetbuttcap%
\pgfsetroundjoin%
\definecolor{currentfill}{rgb}{0.278791,0.062145,0.386592}%
\pgfsetfillcolor{currentfill}%
\pgfsetfillopacity{0.700000}%
\pgfsetlinewidth{0.000000pt}%
\definecolor{currentstroke}{rgb}{0.000000,0.000000,0.000000}%
\pgfsetstrokecolor{currentstroke}%
\pgfsetdash{}{0pt}%
\pgfpathmoveto{\pgfqpoint{4.851571in}{2.575782in}}%
\pgfpathlineto{\pgfqpoint{4.864561in}{2.572979in}}%
\pgfpathlineto{\pgfqpoint{4.877559in}{2.570200in}}%
\pgfpathlineto{\pgfqpoint{4.890563in}{2.567447in}}%
\pgfpathlineto{\pgfqpoint{4.903573in}{2.564718in}}%
\pgfpathlineto{\pgfqpoint{4.896475in}{2.557833in}}%
\pgfpathlineto{\pgfqpoint{4.889373in}{2.550984in}}%
\pgfpathlineto{\pgfqpoint{4.882266in}{2.544168in}}%
\pgfpathlineto{\pgfqpoint{4.875155in}{2.537382in}}%
\pgfpathlineto{\pgfqpoint{4.862130in}{2.539999in}}%
\pgfpathlineto{\pgfqpoint{4.849112in}{2.542642in}}%
\pgfpathlineto{\pgfqpoint{4.836100in}{2.545309in}}%
\pgfpathlineto{\pgfqpoint{4.823095in}{2.548002in}}%
\pgfpathlineto{\pgfqpoint{4.830221in}{2.554894in}}%
\pgfpathlineto{\pgfqpoint{4.837342in}{2.561820in}}%
\pgfpathlineto{\pgfqpoint{4.844459in}{2.568781in}}%
\pgfpathlineto{\pgfqpoint{4.851571in}{2.575782in}}%
\pgfpathclose%
\pgfusepath{fill}%
\end{pgfscope}%
\begin{pgfscope}%
\pgfpathrectangle{\pgfqpoint{1.254980in}{0.150000in}}{\pgfqpoint{5.490039in}{5.490039in}}%
\pgfusepath{clip}%
\pgfsetbuttcap%
\pgfsetroundjoin%
\definecolor{currentfill}{rgb}{0.271305,0.019942,0.347269}%
\pgfsetfillcolor{currentfill}%
\pgfsetfillopacity{0.700000}%
\pgfsetlinewidth{0.000000pt}%
\definecolor{currentstroke}{rgb}{0.000000,0.000000,0.000000}%
\pgfsetstrokecolor{currentstroke}%
\pgfsetdash{}{0pt}%
\pgfpathmoveto{\pgfqpoint{4.081142in}{2.511589in}}%
\pgfpathlineto{\pgfqpoint{4.093952in}{2.508104in}}%
\pgfpathlineto{\pgfqpoint{4.106768in}{2.504648in}}%
\pgfpathlineto{\pgfqpoint{4.119590in}{2.501222in}}%
\pgfpathlineto{\pgfqpoint{4.132417in}{2.497825in}}%
\pgfpathlineto{\pgfqpoint{4.125035in}{2.490573in}}%
\pgfpathlineto{\pgfqpoint{4.117648in}{2.483323in}}%
\pgfpathlineto{\pgfqpoint{4.110256in}{2.476075in}}%
\pgfpathlineto{\pgfqpoint{4.102859in}{2.468827in}}%
\pgfpathlineto{\pgfqpoint{4.090019in}{2.472201in}}%
\pgfpathlineto{\pgfqpoint{4.077186in}{2.475605in}}%
\pgfpathlineto{\pgfqpoint{4.064358in}{2.479038in}}%
\pgfpathlineto{\pgfqpoint{4.051536in}{2.482501in}}%
\pgfpathlineto{\pgfqpoint{4.058946in}{2.489767in}}%
\pgfpathlineto{\pgfqpoint{4.066350in}{2.497037in}}%
\pgfpathlineto{\pgfqpoint{4.073749in}{2.504310in}}%
\pgfpathlineto{\pgfqpoint{4.081142in}{2.511589in}}%
\pgfpathclose%
\pgfusepath{fill}%
\end{pgfscope}%
\begin{pgfscope}%
\pgfpathrectangle{\pgfqpoint{1.254980in}{0.150000in}}{\pgfqpoint{5.490039in}{5.490039in}}%
\pgfusepath{clip}%
\pgfsetbuttcap%
\pgfsetroundjoin%
\definecolor{currentfill}{rgb}{0.277018,0.050344,0.375715}%
\pgfsetfillcolor{currentfill}%
\pgfsetfillopacity{0.700000}%
\pgfsetlinewidth{0.000000pt}%
\definecolor{currentstroke}{rgb}{0.000000,0.000000,0.000000}%
\pgfsetstrokecolor{currentstroke}%
\pgfsetdash{}{0pt}%
\pgfpathmoveto{\pgfqpoint{4.638826in}{2.553802in}}%
\pgfpathlineto{\pgfqpoint{4.651767in}{2.550901in}}%
\pgfpathlineto{\pgfqpoint{4.664715in}{2.548026in}}%
\pgfpathlineto{\pgfqpoint{4.677668in}{2.545177in}}%
\pgfpathlineto{\pgfqpoint{4.690629in}{2.542354in}}%
\pgfpathlineto{\pgfqpoint{4.683451in}{2.535392in}}%
\pgfpathlineto{\pgfqpoint{4.676269in}{2.528451in}}%
\pgfpathlineto{\pgfqpoint{4.669081in}{2.521525in}}%
\pgfpathlineto{\pgfqpoint{4.661889in}{2.514613in}}%
\pgfpathlineto{\pgfqpoint{4.648915in}{2.517351in}}%
\pgfpathlineto{\pgfqpoint{4.635948in}{2.520114in}}%
\pgfpathlineto{\pgfqpoint{4.622987in}{2.522903in}}%
\pgfpathlineto{\pgfqpoint{4.610033in}{2.525718in}}%
\pgfpathlineto{\pgfqpoint{4.617239in}{2.532711in}}%
\pgfpathlineto{\pgfqpoint{4.624439in}{2.539721in}}%
\pgfpathlineto{\pgfqpoint{4.631635in}{2.546750in}}%
\pgfpathlineto{\pgfqpoint{4.638826in}{2.553802in}}%
\pgfpathclose%
\pgfusepath{fill}%
\end{pgfscope}%
\begin{pgfscope}%
\pgfpathrectangle{\pgfqpoint{1.254980in}{0.150000in}}{\pgfqpoint{5.490039in}{5.490039in}}%
\pgfusepath{clip}%
\pgfsetbuttcap%
\pgfsetroundjoin%
\definecolor{currentfill}{rgb}{0.274952,0.037752,0.364543}%
\pgfsetfillcolor{currentfill}%
\pgfsetfillopacity{0.700000}%
\pgfsetlinewidth{0.000000pt}%
\definecolor{currentstroke}{rgb}{0.000000,0.000000,0.000000}%
\pgfsetstrokecolor{currentstroke}%
\pgfsetdash{}{0pt}%
\pgfpathmoveto{\pgfqpoint{4.426043in}{2.532826in}}%
\pgfpathlineto{\pgfqpoint{4.438935in}{2.529765in}}%
\pgfpathlineto{\pgfqpoint{4.451833in}{2.526731in}}%
\pgfpathlineto{\pgfqpoint{4.464737in}{2.523724in}}%
\pgfpathlineto{\pgfqpoint{4.477648in}{2.520743in}}%
\pgfpathlineto{\pgfqpoint{4.470390in}{2.513664in}}%
\pgfpathlineto{\pgfqpoint{4.463128in}{2.506592in}}%
\pgfpathlineto{\pgfqpoint{4.455860in}{2.499525in}}%
\pgfpathlineto{\pgfqpoint{4.448587in}{2.492462in}}%
\pgfpathlineto{\pgfqpoint{4.435664in}{2.495382in}}%
\pgfpathlineto{\pgfqpoint{4.422747in}{2.498329in}}%
\pgfpathlineto{\pgfqpoint{4.409837in}{2.501302in}}%
\pgfpathlineto{\pgfqpoint{4.396932in}{2.504303in}}%
\pgfpathlineto{\pgfqpoint{4.404218in}{2.511422in}}%
\pgfpathlineto{\pgfqpoint{4.411498in}{2.518547in}}%
\pgfpathlineto{\pgfqpoint{4.418773in}{2.525681in}}%
\pgfpathlineto{\pgfqpoint{4.426043in}{2.532826in}}%
\pgfpathclose%
\pgfusepath{fill}%
\end{pgfscope}%
\begin{pgfscope}%
\pgfpathrectangle{\pgfqpoint{1.254980in}{0.150000in}}{\pgfqpoint{5.490039in}{5.490039in}}%
\pgfusepath{clip}%
\pgfsetbuttcap%
\pgfsetroundjoin%
\definecolor{currentfill}{rgb}{0.269944,0.014625,0.341379}%
\pgfsetfillcolor{currentfill}%
\pgfsetfillopacity{0.700000}%
\pgfsetlinewidth{0.000000pt}%
\definecolor{currentstroke}{rgb}{0.000000,0.000000,0.000000}%
\pgfsetstrokecolor{currentstroke}%
\pgfsetdash{}{0pt}%
\pgfpathmoveto{\pgfqpoint{3.868275in}{2.497259in}}%
\pgfpathlineto{\pgfqpoint{3.881042in}{2.493437in}}%
\pgfpathlineto{\pgfqpoint{3.893815in}{2.489648in}}%
\pgfpathlineto{\pgfqpoint{3.906593in}{2.485889in}}%
\pgfpathlineto{\pgfqpoint{3.919376in}{2.482162in}}%
\pgfpathlineto{\pgfqpoint{3.911916in}{2.474895in}}%
\pgfpathlineto{\pgfqpoint{3.904451in}{2.467633in}}%
\pgfpathlineto{\pgfqpoint{3.896979in}{2.460375in}}%
\pgfpathlineto{\pgfqpoint{3.889503in}{2.453121in}}%
\pgfpathlineto{\pgfqpoint{3.876707in}{2.456851in}}%
\pgfpathlineto{\pgfqpoint{3.863917in}{2.460612in}}%
\pgfpathlineto{\pgfqpoint{3.851133in}{2.464405in}}%
\pgfpathlineto{\pgfqpoint{3.838353in}{2.468229in}}%
\pgfpathlineto{\pgfqpoint{3.845842in}{2.475475in}}%
\pgfpathlineto{\pgfqpoint{3.853325in}{2.482729in}}%
\pgfpathlineto{\pgfqpoint{3.860802in}{2.489990in}}%
\pgfpathlineto{\pgfqpoint{3.868275in}{2.497259in}}%
\pgfpathclose%
\pgfusepath{fill}%
\end{pgfscope}%
\begin{pgfscope}%
\pgfpathrectangle{\pgfqpoint{1.254980in}{0.150000in}}{\pgfqpoint{5.490039in}{5.490039in}}%
\pgfusepath{clip}%
\pgfsetbuttcap%
\pgfsetroundjoin%
\definecolor{currentfill}{rgb}{0.272594,0.025563,0.353093}%
\pgfsetfillcolor{currentfill}%
\pgfsetfillopacity{0.700000}%
\pgfsetlinewidth{0.000000pt}%
\definecolor{currentstroke}{rgb}{0.000000,0.000000,0.000000}%
\pgfsetstrokecolor{currentstroke}%
\pgfsetdash{}{0pt}%
\pgfpathmoveto{\pgfqpoint{3.259223in}{2.518839in}}%
\pgfpathlineto{\pgfqpoint{3.271882in}{2.513580in}}%
\pgfpathlineto{\pgfqpoint{3.284545in}{2.508363in}}%
\pgfpathlineto{\pgfqpoint{3.297213in}{2.503187in}}%
\pgfpathlineto{\pgfqpoint{3.309884in}{2.498051in}}%
\pgfpathlineto{\pgfqpoint{3.302195in}{2.491438in}}%
\pgfpathlineto{\pgfqpoint{3.294499in}{2.484864in}}%
\pgfpathlineto{\pgfqpoint{3.286796in}{2.478331in}}%
\pgfpathlineto{\pgfqpoint{3.279088in}{2.471841in}}%
\pgfpathlineto{\pgfqpoint{3.266402in}{2.477043in}}%
\pgfpathlineto{\pgfqpoint{3.253721in}{2.482286in}}%
\pgfpathlineto{\pgfqpoint{3.241043in}{2.487570in}}%
\pgfpathlineto{\pgfqpoint{3.228370in}{2.492894in}}%
\pgfpathlineto{\pgfqpoint{3.236093in}{2.499313in}}%
\pgfpathlineto{\pgfqpoint{3.243809in}{2.505778in}}%
\pgfpathlineto{\pgfqpoint{3.251519in}{2.512287in}}%
\pgfpathlineto{\pgfqpoint{3.259223in}{2.518839in}}%
\pgfpathclose%
\pgfusepath{fill}%
\end{pgfscope}%
\begin{pgfscope}%
\pgfpathrectangle{\pgfqpoint{1.254980in}{0.150000in}}{\pgfqpoint{5.490039in}{5.490039in}}%
\pgfusepath{clip}%
\pgfsetbuttcap%
\pgfsetroundjoin%
\definecolor{currentfill}{rgb}{0.274952,0.037752,0.364543}%
\pgfsetfillcolor{currentfill}%
\pgfsetfillopacity{0.700000}%
\pgfsetlinewidth{0.000000pt}%
\definecolor{currentstroke}{rgb}{0.000000,0.000000,0.000000}%
\pgfsetstrokecolor{currentstroke}%
\pgfsetdash{}{0pt}%
\pgfpathmoveto{\pgfqpoint{3.127123in}{2.537017in}}%
\pgfpathlineto{\pgfqpoint{3.139766in}{2.531349in}}%
\pgfpathlineto{\pgfqpoint{3.152412in}{2.525726in}}%
\pgfpathlineto{\pgfqpoint{3.165062in}{2.520147in}}%
\pgfpathlineto{\pgfqpoint{3.177716in}{2.514611in}}%
\pgfpathlineto{\pgfqpoint{3.169971in}{2.508314in}}%
\pgfpathlineto{\pgfqpoint{3.162220in}{2.502068in}}%
\pgfpathlineto{\pgfqpoint{3.154462in}{2.495875in}}%
\pgfpathlineto{\pgfqpoint{3.146696in}{2.489737in}}%
\pgfpathlineto{\pgfqpoint{3.134028in}{2.495352in}}%
\pgfpathlineto{\pgfqpoint{3.121363in}{2.501011in}}%
\pgfpathlineto{\pgfqpoint{3.108701in}{2.506713in}}%
\pgfpathlineto{\pgfqpoint{3.096044in}{2.512460in}}%
\pgfpathlineto{\pgfqpoint{3.103824in}{2.518513in}}%
\pgfpathlineto{\pgfqpoint{3.111597in}{2.524625in}}%
\pgfpathlineto{\pgfqpoint{3.119364in}{2.530793in}}%
\pgfpathlineto{\pgfqpoint{3.127123in}{2.537017in}}%
\pgfpathclose%
\pgfusepath{fill}%
\end{pgfscope}%
\begin{pgfscope}%
\pgfpathrectangle{\pgfqpoint{1.254980in}{0.150000in}}{\pgfqpoint{5.490039in}{5.490039in}}%
\pgfusepath{clip}%
\pgfsetbuttcap%
\pgfsetroundjoin%
\definecolor{currentfill}{rgb}{0.271305,0.019942,0.347269}%
\pgfsetfillcolor{currentfill}%
\pgfsetfillopacity{0.700000}%
\pgfsetlinewidth{0.000000pt}%
\definecolor{currentstroke}{rgb}{0.000000,0.000000,0.000000}%
\pgfsetstrokecolor{currentstroke}%
\pgfsetdash{}{0pt}%
\pgfpathmoveto{\pgfqpoint{3.391253in}{2.504950in}}%
\pgfpathlineto{\pgfqpoint{3.403933in}{2.500065in}}%
\pgfpathlineto{\pgfqpoint{3.416616in}{2.495218in}}%
\pgfpathlineto{\pgfqpoint{3.429305in}{2.490409in}}%
\pgfpathlineto{\pgfqpoint{3.441998in}{2.485639in}}%
\pgfpathlineto{\pgfqpoint{3.434360in}{2.478779in}}%
\pgfpathlineto{\pgfqpoint{3.426716in}{2.471948in}}%
\pgfpathlineto{\pgfqpoint{3.419067in}{2.465148in}}%
\pgfpathlineto{\pgfqpoint{3.411411in}{2.458379in}}%
\pgfpathlineto{\pgfqpoint{3.398705in}{2.463203in}}%
\pgfpathlineto{\pgfqpoint{3.386003in}{2.468065in}}%
\pgfpathlineto{\pgfqpoint{3.373306in}{2.472966in}}%
\pgfpathlineto{\pgfqpoint{3.360613in}{2.477905in}}%
\pgfpathlineto{\pgfqpoint{3.368282in}{2.484615in}}%
\pgfpathlineto{\pgfqpoint{3.375945in}{2.491361in}}%
\pgfpathlineto{\pgfqpoint{3.383602in}{2.498139in}}%
\pgfpathlineto{\pgfqpoint{3.391253in}{2.504950in}}%
\pgfpathclose%
\pgfusepath{fill}%
\end{pgfscope}%
\begin{pgfscope}%
\pgfpathrectangle{\pgfqpoint{1.254980in}{0.150000in}}{\pgfqpoint{5.490039in}{5.490039in}}%
\pgfusepath{clip}%
\pgfsetbuttcap%
\pgfsetroundjoin%
\definecolor{currentfill}{rgb}{0.281446,0.084320,0.407414}%
\pgfsetfillcolor{currentfill}%
\pgfsetfillopacity{0.700000}%
\pgfsetlinewidth{0.000000pt}%
\definecolor{currentstroke}{rgb}{0.000000,0.000000,0.000000}%
\pgfsetstrokecolor{currentstroke}%
\pgfsetdash{}{0pt}%
\pgfpathmoveto{\pgfqpoint{5.196772in}{2.604801in}}%
\pgfpathlineto{\pgfqpoint{5.209850in}{2.602079in}}%
\pgfpathlineto{\pgfqpoint{5.222935in}{2.599381in}}%
\pgfpathlineto{\pgfqpoint{5.236026in}{2.596706in}}%
\pgfpathlineto{\pgfqpoint{5.249124in}{2.594055in}}%
\pgfpathlineto{\pgfqpoint{5.242152in}{2.587180in}}%
\pgfpathlineto{\pgfqpoint{5.235175in}{2.580377in}}%
\pgfpathlineto{\pgfqpoint{5.228195in}{2.573640in}}%
\pgfpathlineto{\pgfqpoint{5.221212in}{2.566967in}}%
\pgfpathlineto{\pgfqpoint{5.208097in}{2.569468in}}%
\pgfpathlineto{\pgfqpoint{5.194990in}{2.571994in}}%
\pgfpathlineto{\pgfqpoint{5.181889in}{2.574544in}}%
\pgfpathlineto{\pgfqpoint{5.168795in}{2.577117in}}%
\pgfpathlineto{\pgfqpoint{5.175794in}{2.583935in}}%
\pgfpathlineto{\pgfqpoint{5.182790in}{2.590819in}}%
\pgfpathlineto{\pgfqpoint{5.189783in}{2.597773in}}%
\pgfpathlineto{\pgfqpoint{5.196772in}{2.604801in}}%
\pgfpathclose%
\pgfusepath{fill}%
\end{pgfscope}%
\begin{pgfscope}%
\pgfpathrectangle{\pgfqpoint{1.254980in}{0.150000in}}{\pgfqpoint{5.490039in}{5.490039in}}%
\pgfusepath{clip}%
\pgfsetbuttcap%
\pgfsetroundjoin%
\definecolor{currentfill}{rgb}{0.272594,0.025563,0.353093}%
\pgfsetfillcolor{currentfill}%
\pgfsetfillopacity{0.700000}%
\pgfsetlinewidth{0.000000pt}%
\definecolor{currentstroke}{rgb}{0.000000,0.000000,0.000000}%
\pgfsetstrokecolor{currentstroke}%
\pgfsetdash{}{0pt}%
\pgfpathmoveto{\pgfqpoint{4.213210in}{2.513446in}}%
\pgfpathlineto{\pgfqpoint{4.226054in}{2.510159in}}%
\pgfpathlineto{\pgfqpoint{4.238904in}{2.506900in}}%
\pgfpathlineto{\pgfqpoint{4.251760in}{2.503669in}}%
\pgfpathlineto{\pgfqpoint{4.264623in}{2.500467in}}%
\pgfpathlineto{\pgfqpoint{4.257286in}{2.493271in}}%
\pgfpathlineto{\pgfqpoint{4.249945in}{2.486078in}}%
\pgfpathlineto{\pgfqpoint{4.242598in}{2.478884in}}%
\pgfpathlineto{\pgfqpoint{4.235246in}{2.471689in}}%
\pgfpathlineto{\pgfqpoint{4.222371in}{2.474856in}}%
\pgfpathlineto{\pgfqpoint{4.209503in}{2.478052in}}%
\pgfpathlineto{\pgfqpoint{4.196641in}{2.481275in}}%
\pgfpathlineto{\pgfqpoint{4.183784in}{2.484528in}}%
\pgfpathlineto{\pgfqpoint{4.191148in}{2.491753in}}%
\pgfpathlineto{\pgfqpoint{4.198507in}{2.498980in}}%
\pgfpathlineto{\pgfqpoint{4.205861in}{2.506211in}}%
\pgfpathlineto{\pgfqpoint{4.213210in}{2.513446in}}%
\pgfpathclose%
\pgfusepath{fill}%
\end{pgfscope}%
\begin{pgfscope}%
\pgfpathrectangle{\pgfqpoint{1.254980in}{0.150000in}}{\pgfqpoint{5.490039in}{5.490039in}}%
\pgfusepath{clip}%
\pgfsetbuttcap%
\pgfsetroundjoin%
\definecolor{currentfill}{rgb}{0.269944,0.014625,0.341379}%
\pgfsetfillcolor{currentfill}%
\pgfsetfillopacity{0.700000}%
\pgfsetlinewidth{0.000000pt}%
\definecolor{currentstroke}{rgb}{0.000000,0.000000,0.000000}%
\pgfsetstrokecolor{currentstroke}%
\pgfsetdash{}{0pt}%
\pgfpathmoveto{\pgfqpoint{3.523253in}{2.494811in}}%
\pgfpathlineto{\pgfqpoint{3.535956in}{2.490265in}}%
\pgfpathlineto{\pgfqpoint{3.548664in}{2.485756in}}%
\pgfpathlineto{\pgfqpoint{3.561376in}{2.481283in}}%
\pgfpathlineto{\pgfqpoint{3.574093in}{2.476846in}}%
\pgfpathlineto{\pgfqpoint{3.566505in}{2.469803in}}%
\pgfpathlineto{\pgfqpoint{3.558911in}{2.462780in}}%
\pgfpathlineto{\pgfqpoint{3.551311in}{2.455779in}}%
\pgfpathlineto{\pgfqpoint{3.543706in}{2.448799in}}%
\pgfpathlineto{\pgfqpoint{3.530976in}{2.453277in}}%
\pgfpathlineto{\pgfqpoint{3.518251in}{2.457791in}}%
\pgfpathlineto{\pgfqpoint{3.505530in}{2.462341in}}%
\pgfpathlineto{\pgfqpoint{3.492814in}{2.466927in}}%
\pgfpathlineto{\pgfqpoint{3.500433in}{2.473862in}}%
\pgfpathlineto{\pgfqpoint{3.508046in}{2.480821in}}%
\pgfpathlineto{\pgfqpoint{3.515652in}{2.487804in}}%
\pgfpathlineto{\pgfqpoint{3.523253in}{2.494811in}}%
\pgfpathclose%
\pgfusepath{fill}%
\end{pgfscope}%
\begin{pgfscope}%
\pgfpathrectangle{\pgfqpoint{1.254980in}{0.150000in}}{\pgfqpoint{5.490039in}{5.490039in}}%
\pgfusepath{clip}%
\pgfsetbuttcap%
\pgfsetroundjoin%
\definecolor{currentfill}{rgb}{0.279566,0.067836,0.391917}%
\pgfsetfillcolor{currentfill}%
\pgfsetfillopacity{0.700000}%
\pgfsetlinewidth{0.000000pt}%
\definecolor{currentstroke}{rgb}{0.000000,0.000000,0.000000}%
\pgfsetstrokecolor{currentstroke}%
\pgfsetdash{}{0pt}%
\pgfpathmoveto{\pgfqpoint{4.983969in}{2.581542in}}%
\pgfpathlineto{\pgfqpoint{4.996998in}{2.578814in}}%
\pgfpathlineto{\pgfqpoint{5.010033in}{2.576110in}}%
\pgfpathlineto{\pgfqpoint{5.023075in}{2.573430in}}%
\pgfpathlineto{\pgfqpoint{5.036124in}{2.570776in}}%
\pgfpathlineto{\pgfqpoint{5.029073in}{2.563962in}}%
\pgfpathlineto{\pgfqpoint{5.022018in}{2.557194in}}%
\pgfpathlineto{\pgfqpoint{5.014959in}{2.550469in}}%
\pgfpathlineto{\pgfqpoint{5.007896in}{2.543781in}}%
\pgfpathlineto{\pgfqpoint{4.994832in}{2.546312in}}%
\pgfpathlineto{\pgfqpoint{4.981775in}{2.548867in}}%
\pgfpathlineto{\pgfqpoint{4.968724in}{2.551447in}}%
\pgfpathlineto{\pgfqpoint{4.955681in}{2.554052in}}%
\pgfpathlineto{\pgfqpoint{4.962759in}{2.560859in}}%
\pgfpathlineto{\pgfqpoint{4.969833in}{2.567707in}}%
\pgfpathlineto{\pgfqpoint{4.976903in}{2.574600in}}%
\pgfpathlineto{\pgfqpoint{4.983969in}{2.581542in}}%
\pgfpathclose%
\pgfusepath{fill}%
\end{pgfscope}%
\begin{pgfscope}%
\pgfpathrectangle{\pgfqpoint{1.254980in}{0.150000in}}{\pgfqpoint{5.490039in}{5.490039in}}%
\pgfusepath{clip}%
\pgfsetbuttcap%
\pgfsetroundjoin%
\definecolor{currentfill}{rgb}{0.277941,0.056324,0.381191}%
\pgfsetfillcolor{currentfill}%
\pgfsetfillopacity{0.700000}%
\pgfsetlinewidth{0.000000pt}%
\definecolor{currentstroke}{rgb}{0.000000,0.000000,0.000000}%
\pgfsetstrokecolor{currentstroke}%
\pgfsetdash{}{0pt}%
\pgfpathmoveto{\pgfqpoint{2.994908in}{2.560076in}}%
\pgfpathlineto{\pgfqpoint{3.007538in}{2.553960in}}%
\pgfpathlineto{\pgfqpoint{3.020171in}{2.547892in}}%
\pgfpathlineto{\pgfqpoint{3.032808in}{2.541871in}}%
\pgfpathlineto{\pgfqpoint{3.045448in}{2.535897in}}%
\pgfpathlineto{\pgfqpoint{3.037645in}{2.529990in}}%
\pgfpathlineto{\pgfqpoint{3.029834in}{2.524149in}}%
\pgfpathlineto{\pgfqpoint{3.022016in}{2.518374in}}%
\pgfpathlineto{\pgfqpoint{3.014190in}{2.512669in}}%
\pgfpathlineto{\pgfqpoint{3.001534in}{2.518735in}}%
\pgfpathlineto{\pgfqpoint{2.988881in}{2.524848in}}%
\pgfpathlineto{\pgfqpoint{2.976231in}{2.531009in}}%
\pgfpathlineto{\pgfqpoint{2.963585in}{2.537217in}}%
\pgfpathlineto{\pgfqpoint{2.971427in}{2.542826in}}%
\pgfpathlineto{\pgfqpoint{2.979262in}{2.548507in}}%
\pgfpathlineto{\pgfqpoint{2.987089in}{2.554257in}}%
\pgfpathlineto{\pgfqpoint{2.994908in}{2.560076in}}%
\pgfpathclose%
\pgfusepath{fill}%
\end{pgfscope}%
\begin{pgfscope}%
\pgfpathrectangle{\pgfqpoint{1.254980in}{0.150000in}}{\pgfqpoint{5.490039in}{5.490039in}}%
\pgfusepath{clip}%
\pgfsetbuttcap%
\pgfsetroundjoin%
\definecolor{currentfill}{rgb}{0.277941,0.056324,0.381191}%
\pgfsetfillcolor{currentfill}%
\pgfsetfillopacity{0.700000}%
\pgfsetlinewidth{0.000000pt}%
\definecolor{currentstroke}{rgb}{0.000000,0.000000,0.000000}%
\pgfsetstrokecolor{currentstroke}%
\pgfsetdash{}{0pt}%
\pgfpathmoveto{\pgfqpoint{4.771141in}{2.559024in}}%
\pgfpathlineto{\pgfqpoint{4.784120in}{2.556231in}}%
\pgfpathlineto{\pgfqpoint{4.797105in}{2.553462in}}%
\pgfpathlineto{\pgfqpoint{4.810097in}{2.550719in}}%
\pgfpathlineto{\pgfqpoint{4.823095in}{2.548002in}}%
\pgfpathlineto{\pgfqpoint{4.815965in}{2.541138in}}%
\pgfpathlineto{\pgfqpoint{4.808829in}{2.534300in}}%
\pgfpathlineto{\pgfqpoint{4.801689in}{2.527485in}}%
\pgfpathlineto{\pgfqpoint{4.794544in}{2.520689in}}%
\pgfpathlineto{\pgfqpoint{4.781532in}{2.523308in}}%
\pgfpathlineto{\pgfqpoint{4.768526in}{2.525953in}}%
\pgfpathlineto{\pgfqpoint{4.755527in}{2.528622in}}%
\pgfpathlineto{\pgfqpoint{4.742534in}{2.531318in}}%
\pgfpathlineto{\pgfqpoint{4.749693in}{2.538207in}}%
\pgfpathlineto{\pgfqpoint{4.756847in}{2.545120in}}%
\pgfpathlineto{\pgfqpoint{4.763996in}{2.552058in}}%
\pgfpathlineto{\pgfqpoint{4.771141in}{2.559024in}}%
\pgfpathclose%
\pgfusepath{fill}%
\end{pgfscope}%
\begin{pgfscope}%
\pgfpathrectangle{\pgfqpoint{1.254980in}{0.150000in}}{\pgfqpoint{5.490039in}{5.490039in}}%
\pgfusepath{clip}%
\pgfsetbuttcap%
\pgfsetroundjoin%
\definecolor{currentfill}{rgb}{0.271305,0.019942,0.347269}%
\pgfsetfillcolor{currentfill}%
\pgfsetfillopacity{0.700000}%
\pgfsetlinewidth{0.000000pt}%
\definecolor{currentstroke}{rgb}{0.000000,0.000000,0.000000}%
\pgfsetstrokecolor{currentstroke}%
\pgfsetdash{}{0pt}%
\pgfpathmoveto{\pgfqpoint{4.000305in}{2.496653in}}%
\pgfpathlineto{\pgfqpoint{4.013105in}{2.493070in}}%
\pgfpathlineto{\pgfqpoint{4.025909in}{2.489517in}}%
\pgfpathlineto{\pgfqpoint{4.038720in}{2.485994in}}%
\pgfpathlineto{\pgfqpoint{4.051536in}{2.482501in}}%
\pgfpathlineto{\pgfqpoint{4.044121in}{2.475238in}}%
\pgfpathlineto{\pgfqpoint{4.036701in}{2.467975in}}%
\pgfpathlineto{\pgfqpoint{4.029276in}{2.460713in}}%
\pgfpathlineto{\pgfqpoint{4.021845in}{2.453451in}}%
\pgfpathlineto{\pgfqpoint{4.009016in}{2.456934in}}%
\pgfpathlineto{\pgfqpoint{3.996194in}{2.460447in}}%
\pgfpathlineto{\pgfqpoint{3.983377in}{2.463990in}}%
\pgfpathlineto{\pgfqpoint{3.970566in}{2.467563in}}%
\pgfpathlineto{\pgfqpoint{3.978009in}{2.474830in}}%
\pgfpathlineto{\pgfqpoint{3.985446in}{2.482101in}}%
\pgfpathlineto{\pgfqpoint{3.992878in}{2.489375in}}%
\pgfpathlineto{\pgfqpoint{4.000305in}{2.496653in}}%
\pgfpathclose%
\pgfusepath{fill}%
\end{pgfscope}%
\begin{pgfscope}%
\pgfpathrectangle{\pgfqpoint{1.254980in}{0.150000in}}{\pgfqpoint{5.490039in}{5.490039in}}%
\pgfusepath{clip}%
\pgfsetbuttcap%
\pgfsetroundjoin%
\definecolor{currentfill}{rgb}{0.269944,0.014625,0.341379}%
\pgfsetfillcolor{currentfill}%
\pgfsetfillopacity{0.700000}%
\pgfsetlinewidth{0.000000pt}%
\definecolor{currentstroke}{rgb}{0.000000,0.000000,0.000000}%
\pgfsetstrokecolor{currentstroke}%
\pgfsetdash{}{0pt}%
\pgfpathmoveto{\pgfqpoint{3.655256in}{2.487926in}}%
\pgfpathlineto{\pgfqpoint{3.667985in}{2.483691in}}%
\pgfpathlineto{\pgfqpoint{3.680719in}{2.479490in}}%
\pgfpathlineto{\pgfqpoint{3.693459in}{2.475323in}}%
\pgfpathlineto{\pgfqpoint{3.706203in}{2.471189in}}%
\pgfpathlineto{\pgfqpoint{3.698663in}{2.464022in}}%
\pgfpathlineto{\pgfqpoint{3.691116in}{2.456867in}}%
\pgfpathlineto{\pgfqpoint{3.683565in}{2.449725in}}%
\pgfpathlineto{\pgfqpoint{3.676007in}{2.442597in}}%
\pgfpathlineto{\pgfqpoint{3.663251in}{2.446758in}}%
\pgfpathlineto{\pgfqpoint{3.650499in}{2.450954in}}%
\pgfpathlineto{\pgfqpoint{3.637752in}{2.455183in}}%
\pgfpathlineto{\pgfqpoint{3.625011in}{2.459446in}}%
\pgfpathlineto{\pgfqpoint{3.632580in}{2.466542in}}%
\pgfpathlineto{\pgfqpoint{3.640145in}{2.473654in}}%
\pgfpathlineto{\pgfqpoint{3.647703in}{2.480782in}}%
\pgfpathlineto{\pgfqpoint{3.655256in}{2.487926in}}%
\pgfpathclose%
\pgfusepath{fill}%
\end{pgfscope}%
\begin{pgfscope}%
\pgfpathrectangle{\pgfqpoint{1.254980in}{0.150000in}}{\pgfqpoint{5.490039in}{5.490039in}}%
\pgfusepath{clip}%
\pgfsetbuttcap%
\pgfsetroundjoin%
\definecolor{currentfill}{rgb}{0.276022,0.044167,0.370164}%
\pgfsetfillcolor{currentfill}%
\pgfsetfillopacity{0.700000}%
\pgfsetlinewidth{0.000000pt}%
\definecolor{currentstroke}{rgb}{0.000000,0.000000,0.000000}%
\pgfsetstrokecolor{currentstroke}%
\pgfsetdash{}{0pt}%
\pgfpathmoveto{\pgfqpoint{4.558279in}{2.537241in}}%
\pgfpathlineto{\pgfqpoint{4.571208in}{2.534321in}}%
\pgfpathlineto{\pgfqpoint{4.584143in}{2.531427in}}%
\pgfpathlineto{\pgfqpoint{4.597085in}{2.528560in}}%
\pgfpathlineto{\pgfqpoint{4.610033in}{2.525718in}}%
\pgfpathlineto{\pgfqpoint{4.602822in}{2.518739in}}%
\pgfpathlineto{\pgfqpoint{4.595606in}{2.511772in}}%
\pgfpathlineto{\pgfqpoint{4.588385in}{2.504813in}}%
\pgfpathlineto{\pgfqpoint{4.581158in}{2.497860in}}%
\pgfpathlineto{\pgfqpoint{4.568197in}{2.500628in}}%
\pgfpathlineto{\pgfqpoint{4.555243in}{2.503422in}}%
\pgfpathlineto{\pgfqpoint{4.542294in}{2.506243in}}%
\pgfpathlineto{\pgfqpoint{4.529352in}{2.509090in}}%
\pgfpathlineto{\pgfqpoint{4.536592in}{2.516111in}}%
\pgfpathlineto{\pgfqpoint{4.543826in}{2.523142in}}%
\pgfpathlineto{\pgfqpoint{4.551055in}{2.530184in}}%
\pgfpathlineto{\pgfqpoint{4.558279in}{2.537241in}}%
\pgfpathclose%
\pgfusepath{fill}%
\end{pgfscope}%
\begin{pgfscope}%
\pgfpathrectangle{\pgfqpoint{1.254980in}{0.150000in}}{\pgfqpoint{5.490039in}{5.490039in}}%
\pgfusepath{clip}%
\pgfsetbuttcap%
\pgfsetroundjoin%
\definecolor{currentfill}{rgb}{0.273809,0.031497,0.358853}%
\pgfsetfillcolor{currentfill}%
\pgfsetfillopacity{0.700000}%
\pgfsetlinewidth{0.000000pt}%
\definecolor{currentstroke}{rgb}{0.000000,0.000000,0.000000}%
\pgfsetstrokecolor{currentstroke}%
\pgfsetdash{}{0pt}%
\pgfpathmoveto{\pgfqpoint{4.345375in}{2.516580in}}%
\pgfpathlineto{\pgfqpoint{4.358255in}{2.513470in}}%
\pgfpathlineto{\pgfqpoint{4.371141in}{2.510387in}}%
\pgfpathlineto{\pgfqpoint{4.384034in}{2.507331in}}%
\pgfpathlineto{\pgfqpoint{4.396932in}{2.504303in}}%
\pgfpathlineto{\pgfqpoint{4.389641in}{2.497189in}}%
\pgfpathlineto{\pgfqpoint{4.382345in}{2.490077in}}%
\pgfpathlineto{\pgfqpoint{4.375044in}{2.482966in}}%
\pgfpathlineto{\pgfqpoint{4.367738in}{2.475853in}}%
\pgfpathlineto{\pgfqpoint{4.354827in}{2.478833in}}%
\pgfpathlineto{\pgfqpoint{4.341922in}{2.481841in}}%
\pgfpathlineto{\pgfqpoint{4.329024in}{2.484876in}}%
\pgfpathlineto{\pgfqpoint{4.316132in}{2.487938in}}%
\pgfpathlineto{\pgfqpoint{4.323450in}{2.495094in}}%
\pgfpathlineto{\pgfqpoint{4.330764in}{2.502252in}}%
\pgfpathlineto{\pgfqpoint{4.338072in}{2.509413in}}%
\pgfpathlineto{\pgfqpoint{4.345375in}{2.516580in}}%
\pgfpathclose%
\pgfusepath{fill}%
\end{pgfscope}%
\begin{pgfscope}%
\pgfpathrectangle{\pgfqpoint{1.254980in}{0.150000in}}{\pgfqpoint{5.490039in}{5.490039in}}%
\pgfusepath{clip}%
\pgfsetbuttcap%
\pgfsetroundjoin%
\definecolor{currentfill}{rgb}{0.281924,0.089666,0.412415}%
\pgfsetfillcolor{currentfill}%
\pgfsetfillopacity{0.700000}%
\pgfsetlinewidth{0.000000pt}%
\definecolor{currentstroke}{rgb}{0.000000,0.000000,0.000000}%
\pgfsetstrokecolor{currentstroke}%
\pgfsetdash{}{0pt}%
\pgfpathmoveto{\pgfqpoint{5.329381in}{2.611360in}}%
\pgfpathlineto{\pgfqpoint{5.342497in}{2.608667in}}%
\pgfpathlineto{\pgfqpoint{5.355619in}{2.605997in}}%
\pgfpathlineto{\pgfqpoint{5.368749in}{2.603350in}}%
\pgfpathlineto{\pgfqpoint{5.381885in}{2.600727in}}%
\pgfpathlineto{\pgfqpoint{5.374958in}{2.593847in}}%
\pgfpathlineto{\pgfqpoint{5.368028in}{2.587054in}}%
\pgfpathlineto{\pgfqpoint{5.361095in}{2.580340in}}%
\pgfpathlineto{\pgfqpoint{5.354159in}{2.573702in}}%
\pgfpathlineto{\pgfqpoint{5.341005in}{2.576164in}}%
\pgfpathlineto{\pgfqpoint{5.327859in}{2.578649in}}%
\pgfpathlineto{\pgfqpoint{5.314719in}{2.581157in}}%
\pgfpathlineto{\pgfqpoint{5.301586in}{2.583690in}}%
\pgfpathlineto{\pgfqpoint{5.308540in}{2.590485in}}%
\pgfpathlineto{\pgfqpoint{5.315489in}{2.597358in}}%
\pgfpathlineto{\pgfqpoint{5.322437in}{2.604315in}}%
\pgfpathlineto{\pgfqpoint{5.329381in}{2.611360in}}%
\pgfpathclose%
\pgfusepath{fill}%
\end{pgfscope}%
\begin{pgfscope}%
\pgfpathrectangle{\pgfqpoint{1.254980in}{0.150000in}}{\pgfqpoint{5.490039in}{5.490039in}}%
\pgfusepath{clip}%
\pgfsetbuttcap%
\pgfsetroundjoin%
\definecolor{currentfill}{rgb}{0.269944,0.014625,0.341379}%
\pgfsetfillcolor{currentfill}%
\pgfsetfillopacity{0.700000}%
\pgfsetlinewidth{0.000000pt}%
\definecolor{currentstroke}{rgb}{0.000000,0.000000,0.000000}%
\pgfsetstrokecolor{currentstroke}%
\pgfsetdash{}{0pt}%
\pgfpathmoveto{\pgfqpoint{3.787289in}{2.483846in}}%
\pgfpathlineto{\pgfqpoint{3.800047in}{2.479894in}}%
\pgfpathlineto{\pgfqpoint{3.812811in}{2.475973in}}%
\pgfpathlineto{\pgfqpoint{3.825579in}{2.472085in}}%
\pgfpathlineto{\pgfqpoint{3.838353in}{2.468229in}}%
\pgfpathlineto{\pgfqpoint{3.830859in}{2.460989in}}%
\pgfpathlineto{\pgfqpoint{3.823360in}{2.453756in}}%
\pgfpathlineto{\pgfqpoint{3.815855in}{2.446529in}}%
\pgfpathlineto{\pgfqpoint{3.808344in}{2.439309in}}%
\pgfpathlineto{\pgfqpoint{3.795558in}{2.443180in}}%
\pgfpathlineto{\pgfqpoint{3.782777in}{2.447084in}}%
\pgfpathlineto{\pgfqpoint{3.770002in}{2.451020in}}%
\pgfpathlineto{\pgfqpoint{3.757232in}{2.454988in}}%
\pgfpathlineto{\pgfqpoint{3.764754in}{2.462188in}}%
\pgfpathlineto{\pgfqpoint{3.772272in}{2.469398in}}%
\pgfpathlineto{\pgfqpoint{3.779783in}{2.476617in}}%
\pgfpathlineto{\pgfqpoint{3.787289in}{2.483846in}}%
\pgfpathclose%
\pgfusepath{fill}%
\end{pgfscope}%
\begin{pgfscope}%
\pgfpathrectangle{\pgfqpoint{1.254980in}{0.150000in}}{\pgfqpoint{5.490039in}{5.490039in}}%
\pgfusepath{clip}%
\pgfsetbuttcap%
\pgfsetroundjoin%
\definecolor{currentfill}{rgb}{0.280894,0.078907,0.402329}%
\pgfsetfillcolor{currentfill}%
\pgfsetfillopacity{0.700000}%
\pgfsetlinewidth{0.000000pt}%
\definecolor{currentstroke}{rgb}{0.000000,0.000000,0.000000}%
\pgfsetstrokecolor{currentstroke}%
\pgfsetdash{}{0pt}%
\pgfpathmoveto{\pgfqpoint{5.116486in}{2.587651in}}%
\pgfpathlineto{\pgfqpoint{5.129553in}{2.584982in}}%
\pgfpathlineto{\pgfqpoint{5.142627in}{2.582336in}}%
\pgfpathlineto{\pgfqpoint{5.155707in}{2.579714in}}%
\pgfpathlineto{\pgfqpoint{5.168795in}{2.577117in}}%
\pgfpathlineto{\pgfqpoint{5.161791in}{2.570360in}}%
\pgfpathlineto{\pgfqpoint{5.154784in}{2.563660in}}%
\pgfpathlineto{\pgfqpoint{5.147773in}{2.557012in}}%
\pgfpathlineto{\pgfqpoint{5.140757in}{2.550412in}}%
\pgfpathlineto{\pgfqpoint{5.127654in}{2.552872in}}%
\pgfpathlineto{\pgfqpoint{5.114558in}{2.555357in}}%
\pgfpathlineto{\pgfqpoint{5.101469in}{2.557866in}}%
\pgfpathlineto{\pgfqpoint{5.088386in}{2.560400in}}%
\pgfpathlineto{\pgfqpoint{5.095417in}{2.567132in}}%
\pgfpathlineto{\pgfqpoint{5.102444in}{2.573915in}}%
\pgfpathlineto{\pgfqpoint{5.109467in}{2.580753in}}%
\pgfpathlineto{\pgfqpoint{5.116486in}{2.587651in}}%
\pgfpathclose%
\pgfusepath{fill}%
\end{pgfscope}%
\begin{pgfscope}%
\pgfpathrectangle{\pgfqpoint{1.254980in}{0.150000in}}{\pgfqpoint{5.490039in}{5.490039in}}%
\pgfusepath{clip}%
\pgfsetbuttcap%
\pgfsetroundjoin%
\definecolor{currentfill}{rgb}{0.271305,0.019942,0.347269}%
\pgfsetfillcolor{currentfill}%
\pgfsetfillopacity{0.700000}%
\pgfsetlinewidth{0.000000pt}%
\definecolor{currentstroke}{rgb}{0.000000,0.000000,0.000000}%
\pgfsetstrokecolor{currentstroke}%
\pgfsetdash{}{0pt}%
\pgfpathmoveto{\pgfqpoint{4.132417in}{2.497825in}}%
\pgfpathlineto{\pgfqpoint{4.145250in}{2.494457in}}%
\pgfpathlineto{\pgfqpoint{4.158089in}{2.491118in}}%
\pgfpathlineto{\pgfqpoint{4.170934in}{2.487809in}}%
\pgfpathlineto{\pgfqpoint{4.183784in}{2.484528in}}%
\pgfpathlineto{\pgfqpoint{4.176415in}{2.477303in}}%
\pgfpathlineto{\pgfqpoint{4.169040in}{2.470078in}}%
\pgfpathlineto{\pgfqpoint{4.161659in}{2.462850in}}%
\pgfpathlineto{\pgfqpoint{4.154274in}{2.455620in}}%
\pgfpathlineto{\pgfqpoint{4.141411in}{2.458878in}}%
\pgfpathlineto{\pgfqpoint{4.128554in}{2.462165in}}%
\pgfpathlineto{\pgfqpoint{4.115704in}{2.465482in}}%
\pgfpathlineto{\pgfqpoint{4.102859in}{2.468827in}}%
\pgfpathlineto{\pgfqpoint{4.110256in}{2.476075in}}%
\pgfpathlineto{\pgfqpoint{4.117648in}{2.483323in}}%
\pgfpathlineto{\pgfqpoint{4.125035in}{2.490573in}}%
\pgfpathlineto{\pgfqpoint{4.132417in}{2.497825in}}%
\pgfpathclose%
\pgfusepath{fill}%
\end{pgfscope}%
\begin{pgfscope}%
\pgfpathrectangle{\pgfqpoint{1.254980in}{0.150000in}}{\pgfqpoint{5.490039in}{5.490039in}}%
\pgfusepath{clip}%
\pgfsetbuttcap%
\pgfsetroundjoin%
\definecolor{currentfill}{rgb}{0.279566,0.067836,0.391917}%
\pgfsetfillcolor{currentfill}%
\pgfsetfillopacity{0.700000}%
\pgfsetlinewidth{0.000000pt}%
\definecolor{currentstroke}{rgb}{0.000000,0.000000,0.000000}%
\pgfsetstrokecolor{currentstroke}%
\pgfsetdash{}{0pt}%
\pgfpathmoveto{\pgfqpoint{4.903573in}{2.564718in}}%
\pgfpathlineto{\pgfqpoint{4.916590in}{2.562014in}}%
\pgfpathlineto{\pgfqpoint{4.929614in}{2.559336in}}%
\pgfpathlineto{\pgfqpoint{4.942644in}{2.556681in}}%
\pgfpathlineto{\pgfqpoint{4.955681in}{2.554052in}}%
\pgfpathlineto{\pgfqpoint{4.948598in}{2.547283in}}%
\pgfpathlineto{\pgfqpoint{4.941510in}{2.540547in}}%
\pgfpathlineto{\pgfqpoint{4.934418in}{2.533841in}}%
\pgfpathlineto{\pgfqpoint{4.927321in}{2.527161in}}%
\pgfpathlineto{\pgfqpoint{4.914269in}{2.529679in}}%
\pgfpathlineto{\pgfqpoint{4.901224in}{2.532222in}}%
\pgfpathlineto{\pgfqpoint{4.888186in}{2.534789in}}%
\pgfpathlineto{\pgfqpoint{4.875155in}{2.537382in}}%
\pgfpathlineto{\pgfqpoint{4.882266in}{2.544168in}}%
\pgfpathlineto{\pgfqpoint{4.889373in}{2.550984in}}%
\pgfpathlineto{\pgfqpoint{4.896475in}{2.557833in}}%
\pgfpathlineto{\pgfqpoint{4.903573in}{2.564718in}}%
\pgfpathclose%
\pgfusepath{fill}%
\end{pgfscope}%
\begin{pgfscope}%
\pgfpathrectangle{\pgfqpoint{1.254980in}{0.150000in}}{\pgfqpoint{5.490039in}{5.490039in}}%
\pgfusepath{clip}%
\pgfsetbuttcap%
\pgfsetroundjoin%
\definecolor{currentfill}{rgb}{0.271305,0.019942,0.347269}%
\pgfsetfillcolor{currentfill}%
\pgfsetfillopacity{0.700000}%
\pgfsetlinewidth{0.000000pt}%
\definecolor{currentstroke}{rgb}{0.000000,0.000000,0.000000}%
\pgfsetstrokecolor{currentstroke}%
\pgfsetdash{}{0pt}%
\pgfpathmoveto{\pgfqpoint{3.309884in}{2.498051in}}%
\pgfpathlineto{\pgfqpoint{3.322560in}{2.492955in}}%
\pgfpathlineto{\pgfqpoint{3.335240in}{2.487899in}}%
\pgfpathlineto{\pgfqpoint{3.347924in}{2.482882in}}%
\pgfpathlineto{\pgfqpoint{3.360613in}{2.477905in}}%
\pgfpathlineto{\pgfqpoint{3.352937in}{2.471230in}}%
\pgfpathlineto{\pgfqpoint{3.345255in}{2.464591in}}%
\pgfpathlineto{\pgfqpoint{3.337567in}{2.457991in}}%
\pgfpathlineto{\pgfqpoint{3.329872in}{2.451430in}}%
\pgfpathlineto{\pgfqpoint{3.317169in}{2.456474in}}%
\pgfpathlineto{\pgfqpoint{3.304471in}{2.461557in}}%
\pgfpathlineto{\pgfqpoint{3.291777in}{2.466679in}}%
\pgfpathlineto{\pgfqpoint{3.279088in}{2.471841in}}%
\pgfpathlineto{\pgfqpoint{3.286796in}{2.478331in}}%
\pgfpathlineto{\pgfqpoint{3.294499in}{2.484864in}}%
\pgfpathlineto{\pgfqpoint{3.302195in}{2.491438in}}%
\pgfpathlineto{\pgfqpoint{3.309884in}{2.498051in}}%
\pgfpathclose%
\pgfusepath{fill}%
\end{pgfscope}%
\begin{pgfscope}%
\pgfpathrectangle{\pgfqpoint{1.254980in}{0.150000in}}{\pgfqpoint{5.490039in}{5.490039in}}%
\pgfusepath{clip}%
\pgfsetbuttcap%
\pgfsetroundjoin%
\definecolor{currentfill}{rgb}{0.273809,0.031497,0.358853}%
\pgfsetfillcolor{currentfill}%
\pgfsetfillopacity{0.700000}%
\pgfsetlinewidth{0.000000pt}%
\definecolor{currentstroke}{rgb}{0.000000,0.000000,0.000000}%
\pgfsetstrokecolor{currentstroke}%
\pgfsetdash{}{0pt}%
\pgfpathmoveto{\pgfqpoint{3.177716in}{2.514611in}}%
\pgfpathlineto{\pgfqpoint{3.190373in}{2.509118in}}%
\pgfpathlineto{\pgfqpoint{3.203035in}{2.503668in}}%
\pgfpathlineto{\pgfqpoint{3.215700in}{2.498260in}}%
\pgfpathlineto{\pgfqpoint{3.228370in}{2.492894in}}%
\pgfpathlineto{\pgfqpoint{3.220640in}{2.486522in}}%
\pgfpathlineto{\pgfqpoint{3.212903in}{2.480199in}}%
\pgfpathlineto{\pgfqpoint{3.205160in}{2.473925in}}%
\pgfpathlineto{\pgfqpoint{3.197409in}{2.467704in}}%
\pgfpathlineto{\pgfqpoint{3.184725in}{2.473149in}}%
\pgfpathlineto{\pgfqpoint{3.172045in}{2.478636in}}%
\pgfpathlineto{\pgfqpoint{3.159369in}{2.484165in}}%
\pgfpathlineto{\pgfqpoint{3.146696in}{2.489737in}}%
\pgfpathlineto{\pgfqpoint{3.154462in}{2.495875in}}%
\pgfpathlineto{\pgfqpoint{3.162220in}{2.502068in}}%
\pgfpathlineto{\pgfqpoint{3.169971in}{2.508314in}}%
\pgfpathlineto{\pgfqpoint{3.177716in}{2.514611in}}%
\pgfpathclose%
\pgfusepath{fill}%
\end{pgfscope}%
\begin{pgfscope}%
\pgfpathrectangle{\pgfqpoint{1.254980in}{0.150000in}}{\pgfqpoint{5.490039in}{5.490039in}}%
\pgfusepath{clip}%
\pgfsetbuttcap%
\pgfsetroundjoin%
\definecolor{currentfill}{rgb}{0.277018,0.050344,0.375715}%
\pgfsetfillcolor{currentfill}%
\pgfsetfillopacity{0.700000}%
\pgfsetlinewidth{0.000000pt}%
\definecolor{currentstroke}{rgb}{0.000000,0.000000,0.000000}%
\pgfsetstrokecolor{currentstroke}%
\pgfsetdash{}{0pt}%
\pgfpathmoveto{\pgfqpoint{4.690629in}{2.542354in}}%
\pgfpathlineto{\pgfqpoint{4.703595in}{2.539556in}}%
\pgfpathlineto{\pgfqpoint{4.716568in}{2.536784in}}%
\pgfpathlineto{\pgfqpoint{4.729548in}{2.534038in}}%
\pgfpathlineto{\pgfqpoint{4.742534in}{2.531318in}}%
\pgfpathlineto{\pgfqpoint{4.735370in}{2.524447in}}%
\pgfpathlineto{\pgfqpoint{4.728201in}{2.517593in}}%
\pgfpathlineto{\pgfqpoint{4.721027in}{2.510752in}}%
\pgfpathlineto{\pgfqpoint{4.713848in}{2.503922in}}%
\pgfpathlineto{\pgfqpoint{4.700849in}{2.506556in}}%
\pgfpathlineto{\pgfqpoint{4.687855in}{2.509216in}}%
\pgfpathlineto{\pgfqpoint{4.674869in}{2.511902in}}%
\pgfpathlineto{\pgfqpoint{4.661889in}{2.514613in}}%
\pgfpathlineto{\pgfqpoint{4.669081in}{2.521525in}}%
\pgfpathlineto{\pgfqpoint{4.676269in}{2.528451in}}%
\pgfpathlineto{\pgfqpoint{4.683451in}{2.535392in}}%
\pgfpathlineto{\pgfqpoint{4.690629in}{2.542354in}}%
\pgfpathclose%
\pgfusepath{fill}%
\end{pgfscope}%
\begin{pgfscope}%
\pgfpathrectangle{\pgfqpoint{1.254980in}{0.150000in}}{\pgfqpoint{5.490039in}{5.490039in}}%
\pgfusepath{clip}%
\pgfsetbuttcap%
\pgfsetroundjoin%
\definecolor{currentfill}{rgb}{0.269944,0.014625,0.341379}%
\pgfsetfillcolor{currentfill}%
\pgfsetfillopacity{0.700000}%
\pgfsetlinewidth{0.000000pt}%
\definecolor{currentstroke}{rgb}{0.000000,0.000000,0.000000}%
\pgfsetstrokecolor{currentstroke}%
\pgfsetdash{}{0pt}%
\pgfpathmoveto{\pgfqpoint{3.441998in}{2.485639in}}%
\pgfpathlineto{\pgfqpoint{3.454695in}{2.480905in}}%
\pgfpathlineto{\pgfqpoint{3.467397in}{2.476209in}}%
\pgfpathlineto{\pgfqpoint{3.480103in}{2.471550in}}%
\pgfpathlineto{\pgfqpoint{3.492814in}{2.466927in}}%
\pgfpathlineto{\pgfqpoint{3.485190in}{2.460019in}}%
\pgfpathlineto{\pgfqpoint{3.477559in}{2.453136in}}%
\pgfpathlineto{\pgfqpoint{3.469923in}{2.446281in}}%
\pgfpathlineto{\pgfqpoint{3.462280in}{2.439454in}}%
\pgfpathlineto{\pgfqpoint{3.449556in}{2.444129in}}%
\pgfpathlineto{\pgfqpoint{3.436836in}{2.448842in}}%
\pgfpathlineto{\pgfqpoint{3.424121in}{2.453592in}}%
\pgfpathlineto{\pgfqpoint{3.411411in}{2.458379in}}%
\pgfpathlineto{\pgfqpoint{3.419067in}{2.465148in}}%
\pgfpathlineto{\pgfqpoint{3.426716in}{2.471948in}}%
\pgfpathlineto{\pgfqpoint{3.434360in}{2.478779in}}%
\pgfpathlineto{\pgfqpoint{3.441998in}{2.485639in}}%
\pgfpathclose%
\pgfusepath{fill}%
\end{pgfscope}%
\begin{pgfscope}%
\pgfpathrectangle{\pgfqpoint{1.254980in}{0.150000in}}{\pgfqpoint{5.490039in}{5.490039in}}%
\pgfusepath{clip}%
\pgfsetbuttcap%
\pgfsetroundjoin%
\definecolor{currentfill}{rgb}{0.269944,0.014625,0.341379}%
\pgfsetfillcolor{currentfill}%
\pgfsetfillopacity{0.700000}%
\pgfsetlinewidth{0.000000pt}%
\definecolor{currentstroke}{rgb}{0.000000,0.000000,0.000000}%
\pgfsetstrokecolor{currentstroke}%
\pgfsetdash{}{0pt}%
\pgfpathmoveto{\pgfqpoint{3.919376in}{2.482162in}}%
\pgfpathlineto{\pgfqpoint{3.932165in}{2.478466in}}%
\pgfpathlineto{\pgfqpoint{3.944960in}{2.474801in}}%
\pgfpathlineto{\pgfqpoint{3.957760in}{2.471167in}}%
\pgfpathlineto{\pgfqpoint{3.970566in}{2.467563in}}%
\pgfpathlineto{\pgfqpoint{3.963118in}{2.460298in}}%
\pgfpathlineto{\pgfqpoint{3.955664in}{2.453035in}}%
\pgfpathlineto{\pgfqpoint{3.948205in}{2.445773in}}%
\pgfpathlineto{\pgfqpoint{3.940740in}{2.438511in}}%
\pgfpathlineto{\pgfqpoint{3.927922in}{2.442118in}}%
\pgfpathlineto{\pgfqpoint{3.915110in}{2.445755in}}%
\pgfpathlineto{\pgfqpoint{3.902304in}{2.449422in}}%
\pgfpathlineto{\pgfqpoint{3.889503in}{2.453121in}}%
\pgfpathlineto{\pgfqpoint{3.896979in}{2.460375in}}%
\pgfpathlineto{\pgfqpoint{3.904451in}{2.467633in}}%
\pgfpathlineto{\pgfqpoint{3.911916in}{2.474895in}}%
\pgfpathlineto{\pgfqpoint{3.919376in}{2.482162in}}%
\pgfpathclose%
\pgfusepath{fill}%
\end{pgfscope}%
\begin{pgfscope}%
\pgfpathrectangle{\pgfqpoint{1.254980in}{0.150000in}}{\pgfqpoint{5.490039in}{5.490039in}}%
\pgfusepath{clip}%
\pgfsetbuttcap%
\pgfsetroundjoin%
\definecolor{currentfill}{rgb}{0.276022,0.044167,0.370164}%
\pgfsetfillcolor{currentfill}%
\pgfsetfillopacity{0.700000}%
\pgfsetlinewidth{0.000000pt}%
\definecolor{currentstroke}{rgb}{0.000000,0.000000,0.000000}%
\pgfsetstrokecolor{currentstroke}%
\pgfsetdash{}{0pt}%
\pgfpathmoveto{\pgfqpoint{3.045448in}{2.535897in}}%
\pgfpathlineto{\pgfqpoint{3.058092in}{2.529969in}}%
\pgfpathlineto{\pgfqpoint{3.070739in}{2.524087in}}%
\pgfpathlineto{\pgfqpoint{3.083389in}{2.518251in}}%
\pgfpathlineto{\pgfqpoint{3.096044in}{2.512460in}}%
\pgfpathlineto{\pgfqpoint{3.088256in}{2.506466in}}%
\pgfpathlineto{\pgfqpoint{3.080461in}{2.500534in}}%
\pgfpathlineto{\pgfqpoint{3.072658in}{2.494666in}}%
\pgfpathlineto{\pgfqpoint{3.064848in}{2.488864in}}%
\pgfpathlineto{\pgfqpoint{3.052178in}{2.494747in}}%
\pgfpathlineto{\pgfqpoint{3.039512in}{2.500675in}}%
\pgfpathlineto{\pgfqpoint{3.026849in}{2.506649in}}%
\pgfpathlineto{\pgfqpoint{3.014190in}{2.512669in}}%
\pgfpathlineto{\pgfqpoint{3.022016in}{2.518374in}}%
\pgfpathlineto{\pgfqpoint{3.029834in}{2.524149in}}%
\pgfpathlineto{\pgfqpoint{3.037645in}{2.529990in}}%
\pgfpathlineto{\pgfqpoint{3.045448in}{2.535897in}}%
\pgfpathclose%
\pgfusepath{fill}%
\end{pgfscope}%
\begin{pgfscope}%
\pgfpathrectangle{\pgfqpoint{1.254980in}{0.150000in}}{\pgfqpoint{5.490039in}{5.490039in}}%
\pgfusepath{clip}%
\pgfsetbuttcap%
\pgfsetroundjoin%
\definecolor{currentfill}{rgb}{0.274952,0.037752,0.364543}%
\pgfsetfillcolor{currentfill}%
\pgfsetfillopacity{0.700000}%
\pgfsetlinewidth{0.000000pt}%
\definecolor{currentstroke}{rgb}{0.000000,0.000000,0.000000}%
\pgfsetstrokecolor{currentstroke}%
\pgfsetdash{}{0pt}%
\pgfpathmoveto{\pgfqpoint{4.477648in}{2.520743in}}%
\pgfpathlineto{\pgfqpoint{4.490564in}{2.517790in}}%
\pgfpathlineto{\pgfqpoint{4.503487in}{2.514863in}}%
\pgfpathlineto{\pgfqpoint{4.516417in}{2.511963in}}%
\pgfpathlineto{\pgfqpoint{4.529352in}{2.509090in}}%
\pgfpathlineto{\pgfqpoint{4.522108in}{2.502076in}}%
\pgfpathlineto{\pgfqpoint{4.514858in}{2.495066in}}%
\pgfpathlineto{\pgfqpoint{4.507603in}{2.488059in}}%
\pgfpathlineto{\pgfqpoint{4.500343in}{2.481052in}}%
\pgfpathlineto{\pgfqpoint{4.487395in}{2.483865in}}%
\pgfpathlineto{\pgfqpoint{4.474453in}{2.486704in}}%
\pgfpathlineto{\pgfqpoint{4.461517in}{2.489570in}}%
\pgfpathlineto{\pgfqpoint{4.448587in}{2.492462in}}%
\pgfpathlineto{\pgfqpoint{4.455860in}{2.499525in}}%
\pgfpathlineto{\pgfqpoint{4.463128in}{2.506592in}}%
\pgfpathlineto{\pgfqpoint{4.470390in}{2.513664in}}%
\pgfpathlineto{\pgfqpoint{4.477648in}{2.520743in}}%
\pgfpathclose%
\pgfusepath{fill}%
\end{pgfscope}%
\begin{pgfscope}%
\pgfpathrectangle{\pgfqpoint{1.254980in}{0.150000in}}{\pgfqpoint{5.490039in}{5.490039in}}%
\pgfusepath{clip}%
\pgfsetbuttcap%
\pgfsetroundjoin%
\definecolor{currentfill}{rgb}{0.269944,0.014625,0.341379}%
\pgfsetfillcolor{currentfill}%
\pgfsetfillopacity{0.700000}%
\pgfsetlinewidth{0.000000pt}%
\definecolor{currentstroke}{rgb}{0.000000,0.000000,0.000000}%
\pgfsetstrokecolor{currentstroke}%
\pgfsetdash{}{0pt}%
\pgfpathmoveto{\pgfqpoint{3.574093in}{2.476846in}}%
\pgfpathlineto{\pgfqpoint{3.586815in}{2.472443in}}%
\pgfpathlineto{\pgfqpoint{3.599542in}{2.468076in}}%
\pgfpathlineto{\pgfqpoint{3.612274in}{2.463744in}}%
\pgfpathlineto{\pgfqpoint{3.625011in}{2.459446in}}%
\pgfpathlineto{\pgfqpoint{3.617435in}{2.452367in}}%
\pgfpathlineto{\pgfqpoint{3.609854in}{2.445306in}}%
\pgfpathlineto{\pgfqpoint{3.602267in}{2.438262in}}%
\pgfpathlineto{\pgfqpoint{3.594674in}{2.431236in}}%
\pgfpathlineto{\pgfqpoint{3.581925in}{2.435575in}}%
\pgfpathlineto{\pgfqpoint{3.569180in}{2.439948in}}%
\pgfpathlineto{\pgfqpoint{3.556441in}{2.444356in}}%
\pgfpathlineto{\pgfqpoint{3.543706in}{2.448799in}}%
\pgfpathlineto{\pgfqpoint{3.551311in}{2.455779in}}%
\pgfpathlineto{\pgfqpoint{3.558911in}{2.462780in}}%
\pgfpathlineto{\pgfqpoint{3.566505in}{2.469803in}}%
\pgfpathlineto{\pgfqpoint{3.574093in}{2.476846in}}%
\pgfpathclose%
\pgfusepath{fill}%
\end{pgfscope}%
\begin{pgfscope}%
\pgfpathrectangle{\pgfqpoint{1.254980in}{0.150000in}}{\pgfqpoint{5.490039in}{5.490039in}}%
\pgfusepath{clip}%
\pgfsetbuttcap%
\pgfsetroundjoin%
\definecolor{currentfill}{rgb}{0.272594,0.025563,0.353093}%
\pgfsetfillcolor{currentfill}%
\pgfsetfillopacity{0.700000}%
\pgfsetlinewidth{0.000000pt}%
\definecolor{currentstroke}{rgb}{0.000000,0.000000,0.000000}%
\pgfsetstrokecolor{currentstroke}%
\pgfsetdash{}{0pt}%
\pgfpathmoveto{\pgfqpoint{4.264623in}{2.500467in}}%
\pgfpathlineto{\pgfqpoint{4.277491in}{2.497293in}}%
\pgfpathlineto{\pgfqpoint{4.290365in}{2.494147in}}%
\pgfpathlineto{\pgfqpoint{4.303245in}{2.491029in}}%
\pgfpathlineto{\pgfqpoint{4.316132in}{2.487938in}}%
\pgfpathlineto{\pgfqpoint{4.308807in}{2.480783in}}%
\pgfpathlineto{\pgfqpoint{4.301478in}{2.473626in}}%
\pgfpathlineto{\pgfqpoint{4.294143in}{2.466466in}}%
\pgfpathlineto{\pgfqpoint{4.286803in}{2.459302in}}%
\pgfpathlineto{\pgfqpoint{4.273905in}{2.462357in}}%
\pgfpathlineto{\pgfqpoint{4.261012in}{2.465440in}}%
\pgfpathlineto{\pgfqpoint{4.248126in}{2.468551in}}%
\pgfpathlineto{\pgfqpoint{4.235246in}{2.471689in}}%
\pgfpathlineto{\pgfqpoint{4.242598in}{2.478884in}}%
\pgfpathlineto{\pgfqpoint{4.249945in}{2.486078in}}%
\pgfpathlineto{\pgfqpoint{4.257286in}{2.493271in}}%
\pgfpathlineto{\pgfqpoint{4.264623in}{2.500467in}}%
\pgfpathclose%
\pgfusepath{fill}%
\end{pgfscope}%
\begin{pgfscope}%
\pgfpathrectangle{\pgfqpoint{1.254980in}{0.150000in}}{\pgfqpoint{5.490039in}{5.490039in}}%
\pgfusepath{clip}%
\pgfsetbuttcap%
\pgfsetroundjoin%
\definecolor{currentfill}{rgb}{0.281446,0.084320,0.407414}%
\pgfsetfillcolor{currentfill}%
\pgfsetfillopacity{0.700000}%
\pgfsetlinewidth{0.000000pt}%
\definecolor{currentstroke}{rgb}{0.000000,0.000000,0.000000}%
\pgfsetstrokecolor{currentstroke}%
\pgfsetdash{}{0pt}%
\pgfpathmoveto{\pgfqpoint{5.249124in}{2.594055in}}%
\pgfpathlineto{\pgfqpoint{5.262230in}{2.591428in}}%
\pgfpathlineto{\pgfqpoint{5.275342in}{2.588825in}}%
\pgfpathlineto{\pgfqpoint{5.288461in}{2.586246in}}%
\pgfpathlineto{\pgfqpoint{5.301586in}{2.583690in}}%
\pgfpathlineto{\pgfqpoint{5.294630in}{2.576968in}}%
\pgfpathlineto{\pgfqpoint{5.287670in}{2.570316in}}%
\pgfpathlineto{\pgfqpoint{5.280707in}{2.563727in}}%
\pgfpathlineto{\pgfqpoint{5.273740in}{2.557198in}}%
\pgfpathlineto{\pgfqpoint{5.260598in}{2.559605in}}%
\pgfpathlineto{\pgfqpoint{5.247462in}{2.562035in}}%
\pgfpathlineto{\pgfqpoint{5.234334in}{2.564489in}}%
\pgfpathlineto{\pgfqpoint{5.221212in}{2.566967in}}%
\pgfpathlineto{\pgfqpoint{5.228195in}{2.573640in}}%
\pgfpathlineto{\pgfqpoint{5.235175in}{2.580377in}}%
\pgfpathlineto{\pgfqpoint{5.242152in}{2.587180in}}%
\pgfpathlineto{\pgfqpoint{5.249124in}{2.594055in}}%
\pgfpathclose%
\pgfusepath{fill}%
\end{pgfscope}%
\begin{pgfscope}%
\pgfpathrectangle{\pgfqpoint{1.254980in}{0.150000in}}{\pgfqpoint{5.490039in}{5.490039in}}%
\pgfusepath{clip}%
\pgfsetbuttcap%
\pgfsetroundjoin%
\definecolor{currentfill}{rgb}{0.279566,0.067836,0.391917}%
\pgfsetfillcolor{currentfill}%
\pgfsetfillopacity{0.700000}%
\pgfsetlinewidth{0.000000pt}%
\definecolor{currentstroke}{rgb}{0.000000,0.000000,0.000000}%
\pgfsetstrokecolor{currentstroke}%
\pgfsetdash{}{0pt}%
\pgfpathmoveto{\pgfqpoint{2.913031in}{2.562539in}}%
\pgfpathlineto{\pgfqpoint{2.925665in}{2.556135in}}%
\pgfpathlineto{\pgfqpoint{2.938302in}{2.549780in}}%
\pgfpathlineto{\pgfqpoint{2.950942in}{2.543474in}}%
\pgfpathlineto{\pgfqpoint{2.963585in}{2.537217in}}%
\pgfpathlineto{\pgfqpoint{2.955735in}{2.531683in}}%
\pgfpathlineto{\pgfqpoint{2.947877in}{2.526225in}}%
\pgfpathlineto{\pgfqpoint{2.940010in}{2.520847in}}%
\pgfpathlineto{\pgfqpoint{2.932136in}{2.515549in}}%
\pgfpathlineto{\pgfqpoint{2.919476in}{2.521911in}}%
\pgfpathlineto{\pgfqpoint{2.906819in}{2.528321in}}%
\pgfpathlineto{\pgfqpoint{2.894165in}{2.534781in}}%
\pgfpathlineto{\pgfqpoint{2.881514in}{2.541291in}}%
\pgfpathlineto{\pgfqpoint{2.889405in}{2.546479in}}%
\pgfpathlineto{\pgfqpoint{2.897289in}{2.551751in}}%
\pgfpathlineto{\pgfqpoint{2.905164in}{2.557105in}}%
\pgfpathlineto{\pgfqpoint{2.913031in}{2.562539in}}%
\pgfpathclose%
\pgfusepath{fill}%
\end{pgfscope}%
\begin{pgfscope}%
\pgfpathrectangle{\pgfqpoint{1.254980in}{0.150000in}}{\pgfqpoint{5.490039in}{5.490039in}}%
\pgfusepath{clip}%
\pgfsetbuttcap%
\pgfsetroundjoin%
\definecolor{currentfill}{rgb}{0.280267,0.073417,0.397163}%
\pgfsetfillcolor{currentfill}%
\pgfsetfillopacity{0.700000}%
\pgfsetlinewidth{0.000000pt}%
\definecolor{currentstroke}{rgb}{0.000000,0.000000,0.000000}%
\pgfsetstrokecolor{currentstroke}%
\pgfsetdash{}{0pt}%
\pgfpathmoveto{\pgfqpoint{5.036124in}{2.570776in}}%
\pgfpathlineto{\pgfqpoint{5.049179in}{2.568145in}}%
\pgfpathlineto{\pgfqpoint{5.062241in}{2.565539in}}%
\pgfpathlineto{\pgfqpoint{5.075310in}{2.562957in}}%
\pgfpathlineto{\pgfqpoint{5.088386in}{2.560400in}}%
\pgfpathlineto{\pgfqpoint{5.081351in}{2.553715in}}%
\pgfpathlineto{\pgfqpoint{5.074311in}{2.547073in}}%
\pgfpathlineto{\pgfqpoint{5.067267in}{2.540470in}}%
\pgfpathlineto{\pgfqpoint{5.060219in}{2.533901in}}%
\pgfpathlineto{\pgfqpoint{5.047128in}{2.536335in}}%
\pgfpathlineto{\pgfqpoint{5.034044in}{2.538792in}}%
\pgfpathlineto{\pgfqpoint{5.020966in}{2.541274in}}%
\pgfpathlineto{\pgfqpoint{5.007896in}{2.543781in}}%
\pgfpathlineto{\pgfqpoint{5.014959in}{2.550469in}}%
\pgfpathlineto{\pgfqpoint{5.022018in}{2.557194in}}%
\pgfpathlineto{\pgfqpoint{5.029073in}{2.563962in}}%
\pgfpathlineto{\pgfqpoint{5.036124in}{2.570776in}}%
\pgfpathclose%
\pgfusepath{fill}%
\end{pgfscope}%
\begin{pgfscope}%
\pgfpathrectangle{\pgfqpoint{1.254980in}{0.150000in}}{\pgfqpoint{5.490039in}{5.490039in}}%
\pgfusepath{clip}%
\pgfsetbuttcap%
\pgfsetroundjoin%
\definecolor{currentfill}{rgb}{0.268510,0.009605,0.335427}%
\pgfsetfillcolor{currentfill}%
\pgfsetfillopacity{0.700000}%
\pgfsetlinewidth{0.000000pt}%
\definecolor{currentstroke}{rgb}{0.000000,0.000000,0.000000}%
\pgfsetstrokecolor{currentstroke}%
\pgfsetdash{}{0pt}%
\pgfpathmoveto{\pgfqpoint{3.706203in}{2.471189in}}%
\pgfpathlineto{\pgfqpoint{3.718952in}{2.467089in}}%
\pgfpathlineto{\pgfqpoint{3.731707in}{2.463023in}}%
\pgfpathlineto{\pgfqpoint{3.744467in}{2.458989in}}%
\pgfpathlineto{\pgfqpoint{3.757232in}{2.454988in}}%
\pgfpathlineto{\pgfqpoint{3.749704in}{2.447797in}}%
\pgfpathlineto{\pgfqpoint{3.742170in}{2.440616in}}%
\pgfpathlineto{\pgfqpoint{3.734631in}{2.433444in}}%
\pgfpathlineto{\pgfqpoint{3.727086in}{2.426283in}}%
\pgfpathlineto{\pgfqpoint{3.714308in}{2.430312in}}%
\pgfpathlineto{\pgfqpoint{3.701536in}{2.434374in}}%
\pgfpathlineto{\pgfqpoint{3.688769in}{2.438469in}}%
\pgfpathlineto{\pgfqpoint{3.676007in}{2.442597in}}%
\pgfpathlineto{\pgfqpoint{3.683565in}{2.449725in}}%
\pgfpathlineto{\pgfqpoint{3.691116in}{2.456867in}}%
\pgfpathlineto{\pgfqpoint{3.698663in}{2.464022in}}%
\pgfpathlineto{\pgfqpoint{3.706203in}{2.471189in}}%
\pgfpathclose%
\pgfusepath{fill}%
\end{pgfscope}%
\begin{pgfscope}%
\pgfpathrectangle{\pgfqpoint{1.254980in}{0.150000in}}{\pgfqpoint{5.490039in}{5.490039in}}%
\pgfusepath{clip}%
\pgfsetbuttcap%
\pgfsetroundjoin%
\definecolor{currentfill}{rgb}{0.271305,0.019942,0.347269}%
\pgfsetfillcolor{currentfill}%
\pgfsetfillopacity{0.700000}%
\pgfsetlinewidth{0.000000pt}%
\definecolor{currentstroke}{rgb}{0.000000,0.000000,0.000000}%
\pgfsetstrokecolor{currentstroke}%
\pgfsetdash{}{0pt}%
\pgfpathmoveto{\pgfqpoint{4.051536in}{2.482501in}}%
\pgfpathlineto{\pgfqpoint{4.064358in}{2.479038in}}%
\pgfpathlineto{\pgfqpoint{4.077186in}{2.475605in}}%
\pgfpathlineto{\pgfqpoint{4.090019in}{2.472201in}}%
\pgfpathlineto{\pgfqpoint{4.102859in}{2.468827in}}%
\pgfpathlineto{\pgfqpoint{4.095456in}{2.461578in}}%
\pgfpathlineto{\pgfqpoint{4.088047in}{2.454327in}}%
\pgfpathlineto{\pgfqpoint{4.080634in}{2.447074in}}%
\pgfpathlineto{\pgfqpoint{4.073215in}{2.439817in}}%
\pgfpathlineto{\pgfqpoint{4.060363in}{2.443181in}}%
\pgfpathlineto{\pgfqpoint{4.047518in}{2.446575in}}%
\pgfpathlineto{\pgfqpoint{4.034678in}{2.449998in}}%
\pgfpathlineto{\pgfqpoint{4.021845in}{2.453451in}}%
\pgfpathlineto{\pgfqpoint{4.029276in}{2.460713in}}%
\pgfpathlineto{\pgfqpoint{4.036701in}{2.467975in}}%
\pgfpathlineto{\pgfqpoint{4.044121in}{2.475238in}}%
\pgfpathlineto{\pgfqpoint{4.051536in}{2.482501in}}%
\pgfpathclose%
\pgfusepath{fill}%
\end{pgfscope}%
\begin{pgfscope}%
\pgfpathrectangle{\pgfqpoint{1.254980in}{0.150000in}}{\pgfqpoint{5.490039in}{5.490039in}}%
\pgfusepath{clip}%
\pgfsetbuttcap%
\pgfsetroundjoin%
\definecolor{currentfill}{rgb}{0.278791,0.062145,0.386592}%
\pgfsetfillcolor{currentfill}%
\pgfsetfillopacity{0.700000}%
\pgfsetlinewidth{0.000000pt}%
\definecolor{currentstroke}{rgb}{0.000000,0.000000,0.000000}%
\pgfsetstrokecolor{currentstroke}%
\pgfsetdash{}{0pt}%
\pgfpathmoveto{\pgfqpoint{4.823095in}{2.548002in}}%
\pgfpathlineto{\pgfqpoint{4.836100in}{2.545309in}}%
\pgfpathlineto{\pgfqpoint{4.849112in}{2.542642in}}%
\pgfpathlineto{\pgfqpoint{4.862130in}{2.539999in}}%
\pgfpathlineto{\pgfqpoint{4.875155in}{2.537382in}}%
\pgfpathlineto{\pgfqpoint{4.868038in}{2.530622in}}%
\pgfpathlineto{\pgfqpoint{4.860917in}{2.523884in}}%
\pgfpathlineto{\pgfqpoint{4.853791in}{2.517167in}}%
\pgfpathlineto{\pgfqpoint{4.846660in}{2.510465in}}%
\pgfpathlineto{\pgfqpoint{4.833621in}{2.512983in}}%
\pgfpathlineto{\pgfqpoint{4.820589in}{2.515527in}}%
\pgfpathlineto{\pgfqpoint{4.807563in}{2.518095in}}%
\pgfpathlineto{\pgfqpoint{4.794544in}{2.520689in}}%
\pgfpathlineto{\pgfqpoint{4.801689in}{2.527485in}}%
\pgfpathlineto{\pgfqpoint{4.808829in}{2.534300in}}%
\pgfpathlineto{\pgfqpoint{4.815965in}{2.541138in}}%
\pgfpathlineto{\pgfqpoint{4.823095in}{2.548002in}}%
\pgfpathclose%
\pgfusepath{fill}%
\end{pgfscope}%
\begin{pgfscope}%
\pgfpathrectangle{\pgfqpoint{1.254980in}{0.150000in}}{\pgfqpoint{5.490039in}{5.490039in}}%
\pgfusepath{clip}%
\pgfsetbuttcap%
\pgfsetroundjoin%
\definecolor{currentfill}{rgb}{0.277018,0.050344,0.375715}%
\pgfsetfillcolor{currentfill}%
\pgfsetfillopacity{0.700000}%
\pgfsetlinewidth{0.000000pt}%
\definecolor{currentstroke}{rgb}{0.000000,0.000000,0.000000}%
\pgfsetstrokecolor{currentstroke}%
\pgfsetdash{}{0pt}%
\pgfpathmoveto{\pgfqpoint{4.610033in}{2.525718in}}%
\pgfpathlineto{\pgfqpoint{4.622987in}{2.522903in}}%
\pgfpathlineto{\pgfqpoint{4.635948in}{2.520114in}}%
\pgfpathlineto{\pgfqpoint{4.648915in}{2.517351in}}%
\pgfpathlineto{\pgfqpoint{4.661889in}{2.514613in}}%
\pgfpathlineto{\pgfqpoint{4.654691in}{2.507713in}}%
\pgfpathlineto{\pgfqpoint{4.647488in}{2.500820in}}%
\pgfpathlineto{\pgfqpoint{4.640280in}{2.493933in}}%
\pgfpathlineto{\pgfqpoint{4.633067in}{2.487049in}}%
\pgfpathlineto{\pgfqpoint{4.620080in}{2.489713in}}%
\pgfpathlineto{\pgfqpoint{4.607100in}{2.492403in}}%
\pgfpathlineto{\pgfqpoint{4.594126in}{2.495119in}}%
\pgfpathlineto{\pgfqpoint{4.581158in}{2.497860in}}%
\pgfpathlineto{\pgfqpoint{4.588385in}{2.504813in}}%
\pgfpathlineto{\pgfqpoint{4.595606in}{2.511772in}}%
\pgfpathlineto{\pgfqpoint{4.602822in}{2.518739in}}%
\pgfpathlineto{\pgfqpoint{4.610033in}{2.525718in}}%
\pgfpathclose%
\pgfusepath{fill}%
\end{pgfscope}%
\begin{pgfscope}%
\pgfpathrectangle{\pgfqpoint{1.254980in}{0.150000in}}{\pgfqpoint{5.490039in}{5.490039in}}%
\pgfusepath{clip}%
\pgfsetbuttcap%
\pgfsetroundjoin%
\definecolor{currentfill}{rgb}{0.274952,0.037752,0.364543}%
\pgfsetfillcolor{currentfill}%
\pgfsetfillopacity{0.700000}%
\pgfsetlinewidth{0.000000pt}%
\definecolor{currentstroke}{rgb}{0.000000,0.000000,0.000000}%
\pgfsetstrokecolor{currentstroke}%
\pgfsetdash{}{0pt}%
\pgfpathmoveto{\pgfqpoint{4.396932in}{2.504303in}}%
\pgfpathlineto{\pgfqpoint{4.409837in}{2.501302in}}%
\pgfpathlineto{\pgfqpoint{4.422747in}{2.498329in}}%
\pgfpathlineto{\pgfqpoint{4.435664in}{2.495382in}}%
\pgfpathlineto{\pgfqpoint{4.448587in}{2.492462in}}%
\pgfpathlineto{\pgfqpoint{4.441309in}{2.485401in}}%
\pgfpathlineto{\pgfqpoint{4.434026in}{2.478339in}}%
\pgfpathlineto{\pgfqpoint{4.426737in}{2.471274in}}%
\pgfpathlineto{\pgfqpoint{4.419443in}{2.464205in}}%
\pgfpathlineto{\pgfqpoint{4.406507in}{2.467076in}}%
\pgfpathlineto{\pgfqpoint{4.393578in}{2.469975in}}%
\pgfpathlineto{\pgfqpoint{4.380655in}{2.472900in}}%
\pgfpathlineto{\pgfqpoint{4.367738in}{2.475853in}}%
\pgfpathlineto{\pgfqpoint{4.375044in}{2.482966in}}%
\pgfpathlineto{\pgfqpoint{4.382345in}{2.490077in}}%
\pgfpathlineto{\pgfqpoint{4.389641in}{2.497189in}}%
\pgfpathlineto{\pgfqpoint{4.396932in}{2.504303in}}%
\pgfpathclose%
\pgfusepath{fill}%
\end{pgfscope}%
\begin{pgfscope}%
\pgfpathrectangle{\pgfqpoint{1.254980in}{0.150000in}}{\pgfqpoint{5.490039in}{5.490039in}}%
\pgfusepath{clip}%
\pgfsetbuttcap%
\pgfsetroundjoin%
\definecolor{currentfill}{rgb}{0.269944,0.014625,0.341379}%
\pgfsetfillcolor{currentfill}%
\pgfsetfillopacity{0.700000}%
\pgfsetlinewidth{0.000000pt}%
\definecolor{currentstroke}{rgb}{0.000000,0.000000,0.000000}%
\pgfsetstrokecolor{currentstroke}%
\pgfsetdash{}{0pt}%
\pgfpathmoveto{\pgfqpoint{3.838353in}{2.468229in}}%
\pgfpathlineto{\pgfqpoint{3.851133in}{2.464405in}}%
\pgfpathlineto{\pgfqpoint{3.863917in}{2.460612in}}%
\pgfpathlineto{\pgfqpoint{3.876707in}{2.456851in}}%
\pgfpathlineto{\pgfqpoint{3.889503in}{2.453121in}}%
\pgfpathlineto{\pgfqpoint{3.882021in}{2.445871in}}%
\pgfpathlineto{\pgfqpoint{3.874533in}{2.438624in}}%
\pgfpathlineto{\pgfqpoint{3.867041in}{2.431380in}}%
\pgfpathlineto{\pgfqpoint{3.859542in}{2.424139in}}%
\pgfpathlineto{\pgfqpoint{3.846735in}{2.427885in}}%
\pgfpathlineto{\pgfqpoint{3.833932in}{2.431661in}}%
\pgfpathlineto{\pgfqpoint{3.821136in}{2.435469in}}%
\pgfpathlineto{\pgfqpoint{3.808344in}{2.439309in}}%
\pgfpathlineto{\pgfqpoint{3.815855in}{2.446529in}}%
\pgfpathlineto{\pgfqpoint{3.823360in}{2.453756in}}%
\pgfpathlineto{\pgfqpoint{3.830859in}{2.460989in}}%
\pgfpathlineto{\pgfqpoint{3.838353in}{2.468229in}}%
\pgfpathclose%
\pgfusepath{fill}%
\end{pgfscope}%
\begin{pgfscope}%
\pgfpathrectangle{\pgfqpoint{1.254980in}{0.150000in}}{\pgfqpoint{5.490039in}{5.490039in}}%
\pgfusepath{clip}%
\pgfsetbuttcap%
\pgfsetroundjoin%
\definecolor{currentfill}{rgb}{0.282327,0.094955,0.417331}%
\pgfsetfillcolor{currentfill}%
\pgfsetfillopacity{0.700000}%
\pgfsetlinewidth{0.000000pt}%
\definecolor{currentstroke}{rgb}{0.000000,0.000000,0.000000}%
\pgfsetstrokecolor{currentstroke}%
\pgfsetdash{}{0pt}%
\pgfpathmoveto{\pgfqpoint{5.381885in}{2.600727in}}%
\pgfpathlineto{\pgfqpoint{5.395029in}{2.598127in}}%
\pgfpathlineto{\pgfqpoint{5.408179in}{2.595551in}}%
\pgfpathlineto{\pgfqpoint{5.421336in}{2.592998in}}%
\pgfpathlineto{\pgfqpoint{5.434501in}{2.590468in}}%
\pgfpathlineto{\pgfqpoint{5.427591in}{2.583755in}}%
\pgfpathlineto{\pgfqpoint{5.420678in}{2.577125in}}%
\pgfpathlineto{\pgfqpoint{5.413763in}{2.570572in}}%
\pgfpathlineto{\pgfqpoint{5.406844in}{2.564091in}}%
\pgfpathlineto{\pgfqpoint{5.393662in}{2.566459in}}%
\pgfpathlineto{\pgfqpoint{5.380487in}{2.568850in}}%
\pgfpathlineto{\pgfqpoint{5.367320in}{2.571264in}}%
\pgfpathlineto{\pgfqpoint{5.354159in}{2.573702in}}%
\pgfpathlineto{\pgfqpoint{5.361095in}{2.580340in}}%
\pgfpathlineto{\pgfqpoint{5.368028in}{2.587054in}}%
\pgfpathlineto{\pgfqpoint{5.374958in}{2.593847in}}%
\pgfpathlineto{\pgfqpoint{5.381885in}{2.600727in}}%
\pgfpathclose%
\pgfusepath{fill}%
\end{pgfscope}%
\begin{pgfscope}%
\pgfpathrectangle{\pgfqpoint{1.254980in}{0.150000in}}{\pgfqpoint{5.490039in}{5.490039in}}%
\pgfusepath{clip}%
\pgfsetbuttcap%
\pgfsetroundjoin%
\definecolor{currentfill}{rgb}{0.272594,0.025563,0.353093}%
\pgfsetfillcolor{currentfill}%
\pgfsetfillopacity{0.700000}%
\pgfsetlinewidth{0.000000pt}%
\definecolor{currentstroke}{rgb}{0.000000,0.000000,0.000000}%
\pgfsetstrokecolor{currentstroke}%
\pgfsetdash{}{0pt}%
\pgfpathmoveto{\pgfqpoint{3.228370in}{2.492894in}}%
\pgfpathlineto{\pgfqpoint{3.241043in}{2.487570in}}%
\pgfpathlineto{\pgfqpoint{3.253721in}{2.482286in}}%
\pgfpathlineto{\pgfqpoint{3.266402in}{2.477043in}}%
\pgfpathlineto{\pgfqpoint{3.279088in}{2.471841in}}%
\pgfpathlineto{\pgfqpoint{3.271372in}{2.465395in}}%
\pgfpathlineto{\pgfqpoint{3.263650in}{2.458994in}}%
\pgfpathlineto{\pgfqpoint{3.255921in}{2.452640in}}%
\pgfpathlineto{\pgfqpoint{3.248186in}{2.446335in}}%
\pgfpathlineto{\pgfqpoint{3.235485in}{2.451616in}}%
\pgfpathlineto{\pgfqpoint{3.222789in}{2.456938in}}%
\pgfpathlineto{\pgfqpoint{3.210097in}{2.462300in}}%
\pgfpathlineto{\pgfqpoint{3.197409in}{2.467704in}}%
\pgfpathlineto{\pgfqpoint{3.205160in}{2.473925in}}%
\pgfpathlineto{\pgfqpoint{3.212903in}{2.480199in}}%
\pgfpathlineto{\pgfqpoint{3.220640in}{2.486522in}}%
\pgfpathlineto{\pgfqpoint{3.228370in}{2.492894in}}%
\pgfpathclose%
\pgfusepath{fill}%
\end{pgfscope}%
\begin{pgfscope}%
\pgfpathrectangle{\pgfqpoint{1.254980in}{0.150000in}}{\pgfqpoint{5.490039in}{5.490039in}}%
\pgfusepath{clip}%
\pgfsetbuttcap%
\pgfsetroundjoin%
\definecolor{currentfill}{rgb}{0.271305,0.019942,0.347269}%
\pgfsetfillcolor{currentfill}%
\pgfsetfillopacity{0.700000}%
\pgfsetlinewidth{0.000000pt}%
\definecolor{currentstroke}{rgb}{0.000000,0.000000,0.000000}%
\pgfsetstrokecolor{currentstroke}%
\pgfsetdash{}{0pt}%
\pgfpathmoveto{\pgfqpoint{3.360613in}{2.477905in}}%
\pgfpathlineto{\pgfqpoint{3.373306in}{2.472966in}}%
\pgfpathlineto{\pgfqpoint{3.386003in}{2.468065in}}%
\pgfpathlineto{\pgfqpoint{3.398705in}{2.463203in}}%
\pgfpathlineto{\pgfqpoint{3.411411in}{2.458379in}}%
\pgfpathlineto{\pgfqpoint{3.403749in}{2.451642in}}%
\pgfpathlineto{\pgfqpoint{3.396080in}{2.444939in}}%
\pgfpathlineto{\pgfqpoint{3.388406in}{2.438271in}}%
\pgfpathlineto{\pgfqpoint{3.380725in}{2.431639in}}%
\pgfpathlineto{\pgfqpoint{3.368005in}{2.436529in}}%
\pgfpathlineto{\pgfqpoint{3.355289in}{2.441458in}}%
\pgfpathlineto{\pgfqpoint{3.342578in}{2.446425in}}%
\pgfpathlineto{\pgfqpoint{3.329872in}{2.451430in}}%
\pgfpathlineto{\pgfqpoint{3.337567in}{2.457991in}}%
\pgfpathlineto{\pgfqpoint{3.345255in}{2.464591in}}%
\pgfpathlineto{\pgfqpoint{3.352937in}{2.471230in}}%
\pgfpathlineto{\pgfqpoint{3.360613in}{2.477905in}}%
\pgfpathclose%
\pgfusepath{fill}%
\end{pgfscope}%
\begin{pgfscope}%
\pgfpathrectangle{\pgfqpoint{1.254980in}{0.150000in}}{\pgfqpoint{5.490039in}{5.490039in}}%
\pgfusepath{clip}%
\pgfsetbuttcap%
\pgfsetroundjoin%
\definecolor{currentfill}{rgb}{0.272594,0.025563,0.353093}%
\pgfsetfillcolor{currentfill}%
\pgfsetfillopacity{0.700000}%
\pgfsetlinewidth{0.000000pt}%
\definecolor{currentstroke}{rgb}{0.000000,0.000000,0.000000}%
\pgfsetstrokecolor{currentstroke}%
\pgfsetdash{}{0pt}%
\pgfpathmoveto{\pgfqpoint{4.183784in}{2.484528in}}%
\pgfpathlineto{\pgfqpoint{4.196641in}{2.481275in}}%
\pgfpathlineto{\pgfqpoint{4.209503in}{2.478052in}}%
\pgfpathlineto{\pgfqpoint{4.222371in}{2.474856in}}%
\pgfpathlineto{\pgfqpoint{4.235246in}{2.471689in}}%
\pgfpathlineto{\pgfqpoint{4.227888in}{2.464492in}}%
\pgfpathlineto{\pgfqpoint{4.220525in}{2.457291in}}%
\pgfpathlineto{\pgfqpoint{4.213157in}{2.450085in}}%
\pgfpathlineto{\pgfqpoint{4.205783in}{2.442873in}}%
\pgfpathlineto{\pgfqpoint{4.192897in}{2.446017in}}%
\pgfpathlineto{\pgfqpoint{4.180017in}{2.449189in}}%
\pgfpathlineto{\pgfqpoint{4.167142in}{2.452390in}}%
\pgfpathlineto{\pgfqpoint{4.154274in}{2.455620in}}%
\pgfpathlineto{\pgfqpoint{4.161659in}{2.462850in}}%
\pgfpathlineto{\pgfqpoint{4.169040in}{2.470078in}}%
\pgfpathlineto{\pgfqpoint{4.176415in}{2.477303in}}%
\pgfpathlineto{\pgfqpoint{4.183784in}{2.484528in}}%
\pgfpathclose%
\pgfusepath{fill}%
\end{pgfscope}%
\begin{pgfscope}%
\pgfpathrectangle{\pgfqpoint{1.254980in}{0.150000in}}{\pgfqpoint{5.490039in}{5.490039in}}%
\pgfusepath{clip}%
\pgfsetbuttcap%
\pgfsetroundjoin%
\definecolor{currentfill}{rgb}{0.281446,0.084320,0.407414}%
\pgfsetfillcolor{currentfill}%
\pgfsetfillopacity{0.700000}%
\pgfsetlinewidth{0.000000pt}%
\definecolor{currentstroke}{rgb}{0.000000,0.000000,0.000000}%
\pgfsetstrokecolor{currentstroke}%
\pgfsetdash{}{0pt}%
\pgfpathmoveto{\pgfqpoint{5.168795in}{2.577117in}}%
\pgfpathlineto{\pgfqpoint{5.181889in}{2.574544in}}%
\pgfpathlineto{\pgfqpoint{5.194990in}{2.571994in}}%
\pgfpathlineto{\pgfqpoint{5.208097in}{2.569468in}}%
\pgfpathlineto{\pgfqpoint{5.221212in}{2.566967in}}%
\pgfpathlineto{\pgfqpoint{5.214225in}{2.560351in}}%
\pgfpathlineto{\pgfqpoint{5.207233in}{2.553789in}}%
\pgfpathlineto{\pgfqpoint{5.200238in}{2.547276in}}%
\pgfpathlineto{\pgfqpoint{5.193239in}{2.540809in}}%
\pgfpathlineto{\pgfqpoint{5.180108in}{2.543173in}}%
\pgfpathlineto{\pgfqpoint{5.166984in}{2.545562in}}%
\pgfpathlineto{\pgfqpoint{5.153867in}{2.547975in}}%
\pgfpathlineto{\pgfqpoint{5.140757in}{2.550412in}}%
\pgfpathlineto{\pgfqpoint{5.147773in}{2.557012in}}%
\pgfpathlineto{\pgfqpoint{5.154784in}{2.563660in}}%
\pgfpathlineto{\pgfqpoint{5.161791in}{2.570360in}}%
\pgfpathlineto{\pgfqpoint{5.168795in}{2.577117in}}%
\pgfpathclose%
\pgfusepath{fill}%
\end{pgfscope}%
\begin{pgfscope}%
\pgfpathrectangle{\pgfqpoint{1.254980in}{0.150000in}}{\pgfqpoint{5.490039in}{5.490039in}}%
\pgfusepath{clip}%
\pgfsetbuttcap%
\pgfsetroundjoin%
\definecolor{currentfill}{rgb}{0.274952,0.037752,0.364543}%
\pgfsetfillcolor{currentfill}%
\pgfsetfillopacity{0.700000}%
\pgfsetlinewidth{0.000000pt}%
\definecolor{currentstroke}{rgb}{0.000000,0.000000,0.000000}%
\pgfsetstrokecolor{currentstroke}%
\pgfsetdash{}{0pt}%
\pgfpathmoveto{\pgfqpoint{3.096044in}{2.512460in}}%
\pgfpathlineto{\pgfqpoint{3.108701in}{2.506713in}}%
\pgfpathlineto{\pgfqpoint{3.121363in}{2.501011in}}%
\pgfpathlineto{\pgfqpoint{3.134028in}{2.495352in}}%
\pgfpathlineto{\pgfqpoint{3.146696in}{2.489737in}}%
\pgfpathlineto{\pgfqpoint{3.138924in}{2.483657in}}%
\pgfpathlineto{\pgfqpoint{3.131144in}{2.477635in}}%
\pgfpathlineto{\pgfqpoint{3.123357in}{2.471673in}}%
\pgfpathlineto{\pgfqpoint{3.115563in}{2.465774in}}%
\pgfpathlineto{\pgfqpoint{3.102879in}{2.471481in}}%
\pgfpathlineto{\pgfqpoint{3.090198in}{2.477231in}}%
\pgfpathlineto{\pgfqpoint{3.077522in}{2.483025in}}%
\pgfpathlineto{\pgfqpoint{3.064848in}{2.488864in}}%
\pgfpathlineto{\pgfqpoint{3.072658in}{2.494666in}}%
\pgfpathlineto{\pgfqpoint{3.080461in}{2.500534in}}%
\pgfpathlineto{\pgfqpoint{3.088256in}{2.506466in}}%
\pgfpathlineto{\pgfqpoint{3.096044in}{2.512460in}}%
\pgfpathclose%
\pgfusepath{fill}%
\end{pgfscope}%
\begin{pgfscope}%
\pgfpathrectangle{\pgfqpoint{1.254980in}{0.150000in}}{\pgfqpoint{5.490039in}{5.490039in}}%
\pgfusepath{clip}%
\pgfsetbuttcap%
\pgfsetroundjoin%
\definecolor{currentfill}{rgb}{0.269944,0.014625,0.341379}%
\pgfsetfillcolor{currentfill}%
\pgfsetfillopacity{0.700000}%
\pgfsetlinewidth{0.000000pt}%
\definecolor{currentstroke}{rgb}{0.000000,0.000000,0.000000}%
\pgfsetstrokecolor{currentstroke}%
\pgfsetdash{}{0pt}%
\pgfpathmoveto{\pgfqpoint{3.492814in}{2.466927in}}%
\pgfpathlineto{\pgfqpoint{3.505530in}{2.462341in}}%
\pgfpathlineto{\pgfqpoint{3.518251in}{2.457791in}}%
\pgfpathlineto{\pgfqpoint{3.530976in}{2.453277in}}%
\pgfpathlineto{\pgfqpoint{3.543706in}{2.448799in}}%
\pgfpathlineto{\pgfqpoint{3.536094in}{2.441841in}}%
\pgfpathlineto{\pgfqpoint{3.528477in}{2.434907in}}%
\pgfpathlineto{\pgfqpoint{3.520854in}{2.427997in}}%
\pgfpathlineto{\pgfqpoint{3.513224in}{2.421111in}}%
\pgfpathlineto{\pgfqpoint{3.500481in}{2.425643in}}%
\pgfpathlineto{\pgfqpoint{3.487743in}{2.430211in}}%
\pgfpathlineto{\pgfqpoint{3.475009in}{2.434814in}}%
\pgfpathlineto{\pgfqpoint{3.462280in}{2.439454in}}%
\pgfpathlineto{\pgfqpoint{3.469923in}{2.446281in}}%
\pgfpathlineto{\pgfqpoint{3.477559in}{2.453136in}}%
\pgfpathlineto{\pgfqpoint{3.485190in}{2.460019in}}%
\pgfpathlineto{\pgfqpoint{3.492814in}{2.466927in}}%
\pgfpathclose%
\pgfusepath{fill}%
\end{pgfscope}%
\begin{pgfscope}%
\pgfpathrectangle{\pgfqpoint{1.254980in}{0.150000in}}{\pgfqpoint{5.490039in}{5.490039in}}%
\pgfusepath{clip}%
\pgfsetbuttcap%
\pgfsetroundjoin%
\definecolor{currentfill}{rgb}{0.279566,0.067836,0.391917}%
\pgfsetfillcolor{currentfill}%
\pgfsetfillopacity{0.700000}%
\pgfsetlinewidth{0.000000pt}%
\definecolor{currentstroke}{rgb}{0.000000,0.000000,0.000000}%
\pgfsetstrokecolor{currentstroke}%
\pgfsetdash{}{0pt}%
\pgfpathmoveto{\pgfqpoint{4.955681in}{2.554052in}}%
\pgfpathlineto{\pgfqpoint{4.968724in}{2.551447in}}%
\pgfpathlineto{\pgfqpoint{4.981775in}{2.548867in}}%
\pgfpathlineto{\pgfqpoint{4.994832in}{2.546312in}}%
\pgfpathlineto{\pgfqpoint{5.007896in}{2.543781in}}%
\pgfpathlineto{\pgfqpoint{5.000827in}{2.537128in}}%
\pgfpathlineto{\pgfqpoint{4.993754in}{2.530505in}}%
\pgfpathlineto{\pgfqpoint{4.986677in}{2.523909in}}%
\pgfpathlineto{\pgfqpoint{4.979595in}{2.517337in}}%
\pgfpathlineto{\pgfqpoint{4.966516in}{2.519756in}}%
\pgfpathlineto{\pgfqpoint{4.953444in}{2.522200in}}%
\pgfpathlineto{\pgfqpoint{4.940379in}{2.524668in}}%
\pgfpathlineto{\pgfqpoint{4.927321in}{2.527161in}}%
\pgfpathlineto{\pgfqpoint{4.934418in}{2.533841in}}%
\pgfpathlineto{\pgfqpoint{4.941510in}{2.540547in}}%
\pgfpathlineto{\pgfqpoint{4.948598in}{2.547283in}}%
\pgfpathlineto{\pgfqpoint{4.955681in}{2.554052in}}%
\pgfpathclose%
\pgfusepath{fill}%
\end{pgfscope}%
\begin{pgfscope}%
\pgfpathrectangle{\pgfqpoint{1.254980in}{0.150000in}}{\pgfqpoint{5.490039in}{5.490039in}}%
\pgfusepath{clip}%
\pgfsetbuttcap%
\pgfsetroundjoin%
\definecolor{currentfill}{rgb}{0.277941,0.056324,0.381191}%
\pgfsetfillcolor{currentfill}%
\pgfsetfillopacity{0.700000}%
\pgfsetlinewidth{0.000000pt}%
\definecolor{currentstroke}{rgb}{0.000000,0.000000,0.000000}%
\pgfsetstrokecolor{currentstroke}%
\pgfsetdash{}{0pt}%
\pgfpathmoveto{\pgfqpoint{4.742534in}{2.531318in}}%
\pgfpathlineto{\pgfqpoint{4.755527in}{2.528622in}}%
\pgfpathlineto{\pgfqpoint{4.768526in}{2.525953in}}%
\pgfpathlineto{\pgfqpoint{4.781532in}{2.523308in}}%
\pgfpathlineto{\pgfqpoint{4.794544in}{2.520689in}}%
\pgfpathlineto{\pgfqpoint{4.787394in}{2.513910in}}%
\pgfpathlineto{\pgfqpoint{4.780239in}{2.507143in}}%
\pgfpathlineto{\pgfqpoint{4.773078in}{2.500387in}}%
\pgfpathlineto{\pgfqpoint{4.765913in}{2.493638in}}%
\pgfpathlineto{\pgfqpoint{4.752887in}{2.496171in}}%
\pgfpathlineto{\pgfqpoint{4.739867in}{2.498729in}}%
\pgfpathlineto{\pgfqpoint{4.726855in}{2.501313in}}%
\pgfpathlineto{\pgfqpoint{4.713848in}{2.503922in}}%
\pgfpathlineto{\pgfqpoint{4.721027in}{2.510752in}}%
\pgfpathlineto{\pgfqpoint{4.728201in}{2.517593in}}%
\pgfpathlineto{\pgfqpoint{4.735370in}{2.524447in}}%
\pgfpathlineto{\pgfqpoint{4.742534in}{2.531318in}}%
\pgfpathclose%
\pgfusepath{fill}%
\end{pgfscope}%
\begin{pgfscope}%
\pgfpathrectangle{\pgfqpoint{1.254980in}{0.150000in}}{\pgfqpoint{5.490039in}{5.490039in}}%
\pgfusepath{clip}%
\pgfsetbuttcap%
\pgfsetroundjoin%
\definecolor{currentfill}{rgb}{0.269944,0.014625,0.341379}%
\pgfsetfillcolor{currentfill}%
\pgfsetfillopacity{0.700000}%
\pgfsetlinewidth{0.000000pt}%
\definecolor{currentstroke}{rgb}{0.000000,0.000000,0.000000}%
\pgfsetstrokecolor{currentstroke}%
\pgfsetdash{}{0pt}%
\pgfpathmoveto{\pgfqpoint{3.970566in}{2.467563in}}%
\pgfpathlineto{\pgfqpoint{3.983377in}{2.463990in}}%
\pgfpathlineto{\pgfqpoint{3.996194in}{2.460447in}}%
\pgfpathlineto{\pgfqpoint{4.009016in}{2.456934in}}%
\pgfpathlineto{\pgfqpoint{4.021845in}{2.453451in}}%
\pgfpathlineto{\pgfqpoint{4.014408in}{2.446188in}}%
\pgfpathlineto{\pgfqpoint{4.006966in}{2.438924in}}%
\pgfpathlineto{\pgfqpoint{3.999519in}{2.431658in}}%
\pgfpathlineto{\pgfqpoint{3.992067in}{2.424389in}}%
\pgfpathlineto{\pgfqpoint{3.979227in}{2.427874in}}%
\pgfpathlineto{\pgfqpoint{3.966392in}{2.431390in}}%
\pgfpathlineto{\pgfqpoint{3.953563in}{2.434935in}}%
\pgfpathlineto{\pgfqpoint{3.940740in}{2.438511in}}%
\pgfpathlineto{\pgfqpoint{3.948205in}{2.445773in}}%
\pgfpathlineto{\pgfqpoint{3.955664in}{2.453035in}}%
\pgfpathlineto{\pgfqpoint{3.963118in}{2.460298in}}%
\pgfpathlineto{\pgfqpoint{3.970566in}{2.467563in}}%
\pgfpathclose%
\pgfusepath{fill}%
\end{pgfscope}%
\begin{pgfscope}%
\pgfpathrectangle{\pgfqpoint{1.254980in}{0.150000in}}{\pgfqpoint{5.490039in}{5.490039in}}%
\pgfusepath{clip}%
\pgfsetbuttcap%
\pgfsetroundjoin%
\definecolor{currentfill}{rgb}{0.277941,0.056324,0.381191}%
\pgfsetfillcolor{currentfill}%
\pgfsetfillopacity{0.700000}%
\pgfsetlinewidth{0.000000pt}%
\definecolor{currentstroke}{rgb}{0.000000,0.000000,0.000000}%
\pgfsetstrokecolor{currentstroke}%
\pgfsetdash{}{0pt}%
\pgfpathmoveto{\pgfqpoint{2.963585in}{2.537217in}}%
\pgfpathlineto{\pgfqpoint{2.976231in}{2.531009in}}%
\pgfpathlineto{\pgfqpoint{2.988881in}{2.524848in}}%
\pgfpathlineto{\pgfqpoint{3.001534in}{2.518735in}}%
\pgfpathlineto{\pgfqpoint{3.014190in}{2.512669in}}%
\pgfpathlineto{\pgfqpoint{3.006356in}{2.507035in}}%
\pgfpathlineto{\pgfqpoint{2.998515in}{2.501474in}}%
\pgfpathlineto{\pgfqpoint{2.990665in}{2.495988in}}%
\pgfpathlineto{\pgfqpoint{2.982808in}{2.490581in}}%
\pgfpathlineto{\pgfqpoint{2.970135in}{2.496752in}}%
\pgfpathlineto{\pgfqpoint{2.957465in}{2.502970in}}%
\pgfpathlineto{\pgfqpoint{2.944799in}{2.509236in}}%
\pgfpathlineto{\pgfqpoint{2.932136in}{2.515549in}}%
\pgfpathlineto{\pgfqpoint{2.940010in}{2.520847in}}%
\pgfpathlineto{\pgfqpoint{2.947877in}{2.526225in}}%
\pgfpathlineto{\pgfqpoint{2.955735in}{2.531683in}}%
\pgfpathlineto{\pgfqpoint{2.963585in}{2.537217in}}%
\pgfpathclose%
\pgfusepath{fill}%
\end{pgfscope}%
\begin{pgfscope}%
\pgfpathrectangle{\pgfqpoint{1.254980in}{0.150000in}}{\pgfqpoint{5.490039in}{5.490039in}}%
\pgfusepath{clip}%
\pgfsetbuttcap%
\pgfsetroundjoin%
\definecolor{currentfill}{rgb}{0.268510,0.009605,0.335427}%
\pgfsetfillcolor{currentfill}%
\pgfsetfillopacity{0.700000}%
\pgfsetlinewidth{0.000000pt}%
\definecolor{currentstroke}{rgb}{0.000000,0.000000,0.000000}%
\pgfsetstrokecolor{currentstroke}%
\pgfsetdash{}{0pt}%
\pgfpathmoveto{\pgfqpoint{3.625011in}{2.459446in}}%
\pgfpathlineto{\pgfqpoint{3.637752in}{2.455183in}}%
\pgfpathlineto{\pgfqpoint{3.650499in}{2.450954in}}%
\pgfpathlineto{\pgfqpoint{3.663251in}{2.446758in}}%
\pgfpathlineto{\pgfqpoint{3.676007in}{2.442597in}}%
\pgfpathlineto{\pgfqpoint{3.668444in}{2.435482in}}%
\pgfpathlineto{\pgfqpoint{3.660876in}{2.428381in}}%
\pgfpathlineto{\pgfqpoint{3.653301in}{2.421295in}}%
\pgfpathlineto{\pgfqpoint{3.645721in}{2.414224in}}%
\pgfpathlineto{\pgfqpoint{3.632952in}{2.418427in}}%
\pgfpathlineto{\pgfqpoint{3.620188in}{2.422663in}}%
\pgfpathlineto{\pgfqpoint{3.607428in}{2.426932in}}%
\pgfpathlineto{\pgfqpoint{3.594674in}{2.431236in}}%
\pgfpathlineto{\pgfqpoint{3.602267in}{2.438262in}}%
\pgfpathlineto{\pgfqpoint{3.609854in}{2.445306in}}%
\pgfpathlineto{\pgfqpoint{3.617435in}{2.452367in}}%
\pgfpathlineto{\pgfqpoint{3.625011in}{2.459446in}}%
\pgfpathclose%
\pgfusepath{fill}%
\end{pgfscope}%
\begin{pgfscope}%
\pgfpathrectangle{\pgfqpoint{1.254980in}{0.150000in}}{\pgfqpoint{5.490039in}{5.490039in}}%
\pgfusepath{clip}%
\pgfsetbuttcap%
\pgfsetroundjoin%
\definecolor{currentfill}{rgb}{0.276022,0.044167,0.370164}%
\pgfsetfillcolor{currentfill}%
\pgfsetfillopacity{0.700000}%
\pgfsetlinewidth{0.000000pt}%
\definecolor{currentstroke}{rgb}{0.000000,0.000000,0.000000}%
\pgfsetstrokecolor{currentstroke}%
\pgfsetdash{}{0pt}%
\pgfpathmoveto{\pgfqpoint{4.529352in}{2.509090in}}%
\pgfpathlineto{\pgfqpoint{4.542294in}{2.506243in}}%
\pgfpathlineto{\pgfqpoint{4.555243in}{2.503422in}}%
\pgfpathlineto{\pgfqpoint{4.568197in}{2.500628in}}%
\pgfpathlineto{\pgfqpoint{4.581158in}{2.497860in}}%
\pgfpathlineto{\pgfqpoint{4.573927in}{2.490912in}}%
\pgfpathlineto{\pgfqpoint{4.566690in}{2.483964in}}%
\pgfpathlineto{\pgfqpoint{4.559448in}{2.477017in}}%
\pgfpathlineto{\pgfqpoint{4.552200in}{2.470066in}}%
\pgfpathlineto{\pgfqpoint{4.539226in}{2.472773in}}%
\pgfpathlineto{\pgfqpoint{4.526259in}{2.475506in}}%
\pgfpathlineto{\pgfqpoint{4.513298in}{2.478266in}}%
\pgfpathlineto{\pgfqpoint{4.500343in}{2.481052in}}%
\pgfpathlineto{\pgfqpoint{4.507603in}{2.488059in}}%
\pgfpathlineto{\pgfqpoint{4.514858in}{2.495066in}}%
\pgfpathlineto{\pgfqpoint{4.522108in}{2.502076in}}%
\pgfpathlineto{\pgfqpoint{4.529352in}{2.509090in}}%
\pgfpathclose%
\pgfusepath{fill}%
\end{pgfscope}%
\begin{pgfscope}%
\pgfpathrectangle{\pgfqpoint{1.254980in}{0.150000in}}{\pgfqpoint{5.490039in}{5.490039in}}%
\pgfusepath{clip}%
\pgfsetbuttcap%
\pgfsetroundjoin%
\definecolor{currentfill}{rgb}{0.273809,0.031497,0.358853}%
\pgfsetfillcolor{currentfill}%
\pgfsetfillopacity{0.700000}%
\pgfsetlinewidth{0.000000pt}%
\definecolor{currentstroke}{rgb}{0.000000,0.000000,0.000000}%
\pgfsetstrokecolor{currentstroke}%
\pgfsetdash{}{0pt}%
\pgfpathmoveto{\pgfqpoint{4.316132in}{2.487938in}}%
\pgfpathlineto{\pgfqpoint{4.329024in}{2.484876in}}%
\pgfpathlineto{\pgfqpoint{4.341922in}{2.481841in}}%
\pgfpathlineto{\pgfqpoint{4.354827in}{2.478833in}}%
\pgfpathlineto{\pgfqpoint{4.367738in}{2.475853in}}%
\pgfpathlineto{\pgfqpoint{4.360426in}{2.468738in}}%
\pgfpathlineto{\pgfqpoint{4.353109in}{2.461618in}}%
\pgfpathlineto{\pgfqpoint{4.345786in}{2.454492in}}%
\pgfpathlineto{\pgfqpoint{4.338458in}{2.447359in}}%
\pgfpathlineto{\pgfqpoint{4.325535in}{2.450304in}}%
\pgfpathlineto{\pgfqpoint{4.312618in}{2.453276in}}%
\pgfpathlineto{\pgfqpoint{4.299708in}{2.456275in}}%
\pgfpathlineto{\pgfqpoint{4.286803in}{2.459302in}}%
\pgfpathlineto{\pgfqpoint{4.294143in}{2.466466in}}%
\pgfpathlineto{\pgfqpoint{4.301478in}{2.473626in}}%
\pgfpathlineto{\pgfqpoint{4.308807in}{2.480783in}}%
\pgfpathlineto{\pgfqpoint{4.316132in}{2.487938in}}%
\pgfpathclose%
\pgfusepath{fill}%
\end{pgfscope}%
\begin{pgfscope}%
\pgfpathrectangle{\pgfqpoint{1.254980in}{0.150000in}}{\pgfqpoint{5.490039in}{5.490039in}}%
\pgfusepath{clip}%
\pgfsetbuttcap%
\pgfsetroundjoin%
\definecolor{currentfill}{rgb}{0.281924,0.089666,0.412415}%
\pgfsetfillcolor{currentfill}%
\pgfsetfillopacity{0.700000}%
\pgfsetlinewidth{0.000000pt}%
\definecolor{currentstroke}{rgb}{0.000000,0.000000,0.000000}%
\pgfsetstrokecolor{currentstroke}%
\pgfsetdash{}{0pt}%
\pgfpathmoveto{\pgfqpoint{5.301586in}{2.583690in}}%
\pgfpathlineto{\pgfqpoint{5.314719in}{2.581157in}}%
\pgfpathlineto{\pgfqpoint{5.327859in}{2.578649in}}%
\pgfpathlineto{\pgfqpoint{5.341005in}{2.576164in}}%
\pgfpathlineto{\pgfqpoint{5.354159in}{2.573702in}}%
\pgfpathlineto{\pgfqpoint{5.347220in}{2.567135in}}%
\pgfpathlineto{\pgfqpoint{5.340277in}{2.560633in}}%
\pgfpathlineto{\pgfqpoint{5.333330in}{2.554193in}}%
\pgfpathlineto{\pgfqpoint{5.326380in}{2.547809in}}%
\pgfpathlineto{\pgfqpoint{5.313209in}{2.550121in}}%
\pgfpathlineto{\pgfqpoint{5.300046in}{2.552456in}}%
\pgfpathlineto{\pgfqpoint{5.286889in}{2.554815in}}%
\pgfpathlineto{\pgfqpoint{5.273740in}{2.557198in}}%
\pgfpathlineto{\pgfqpoint{5.280707in}{2.563727in}}%
\pgfpathlineto{\pgfqpoint{5.287670in}{2.570316in}}%
\pgfpathlineto{\pgfqpoint{5.294630in}{2.576968in}}%
\pgfpathlineto{\pgfqpoint{5.301586in}{2.583690in}}%
\pgfpathclose%
\pgfusepath{fill}%
\end{pgfscope}%
\begin{pgfscope}%
\pgfpathrectangle{\pgfqpoint{1.254980in}{0.150000in}}{\pgfqpoint{5.490039in}{5.490039in}}%
\pgfusepath{clip}%
\pgfsetbuttcap%
\pgfsetroundjoin%
\definecolor{currentfill}{rgb}{0.268510,0.009605,0.335427}%
\pgfsetfillcolor{currentfill}%
\pgfsetfillopacity{0.700000}%
\pgfsetlinewidth{0.000000pt}%
\definecolor{currentstroke}{rgb}{0.000000,0.000000,0.000000}%
\pgfsetstrokecolor{currentstroke}%
\pgfsetdash{}{0pt}%
\pgfpathmoveto{\pgfqpoint{3.757232in}{2.454988in}}%
\pgfpathlineto{\pgfqpoint{3.770002in}{2.451020in}}%
\pgfpathlineto{\pgfqpoint{3.782777in}{2.447084in}}%
\pgfpathlineto{\pgfqpoint{3.795558in}{2.443180in}}%
\pgfpathlineto{\pgfqpoint{3.808344in}{2.439309in}}%
\pgfpathlineto{\pgfqpoint{3.800828in}{2.432095in}}%
\pgfpathlineto{\pgfqpoint{3.793307in}{2.424887in}}%
\pgfpathlineto{\pgfqpoint{3.785780in}{2.417686in}}%
\pgfpathlineto{\pgfqpoint{3.778247in}{2.410492in}}%
\pgfpathlineto{\pgfqpoint{3.765449in}{2.414392in}}%
\pgfpathlineto{\pgfqpoint{3.752656in}{2.418323in}}%
\pgfpathlineto{\pgfqpoint{3.739868in}{2.422287in}}%
\pgfpathlineto{\pgfqpoint{3.727086in}{2.426283in}}%
\pgfpathlineto{\pgfqpoint{3.734631in}{2.433444in}}%
\pgfpathlineto{\pgfqpoint{3.742170in}{2.440616in}}%
\pgfpathlineto{\pgfqpoint{3.749704in}{2.447797in}}%
\pgfpathlineto{\pgfqpoint{3.757232in}{2.454988in}}%
\pgfpathclose%
\pgfusepath{fill}%
\end{pgfscope}%
\begin{pgfscope}%
\pgfpathrectangle{\pgfqpoint{1.254980in}{0.150000in}}{\pgfqpoint{5.490039in}{5.490039in}}%
\pgfusepath{clip}%
\pgfsetbuttcap%
\pgfsetroundjoin%
\definecolor{currentfill}{rgb}{0.271305,0.019942,0.347269}%
\pgfsetfillcolor{currentfill}%
\pgfsetfillopacity{0.700000}%
\pgfsetlinewidth{0.000000pt}%
\definecolor{currentstroke}{rgb}{0.000000,0.000000,0.000000}%
\pgfsetstrokecolor{currentstroke}%
\pgfsetdash{}{0pt}%
\pgfpathmoveto{\pgfqpoint{4.102859in}{2.468827in}}%
\pgfpathlineto{\pgfqpoint{4.115704in}{2.465482in}}%
\pgfpathlineto{\pgfqpoint{4.128554in}{2.462165in}}%
\pgfpathlineto{\pgfqpoint{4.141411in}{2.458878in}}%
\pgfpathlineto{\pgfqpoint{4.154274in}{2.455620in}}%
\pgfpathlineto{\pgfqpoint{4.146883in}{2.448386in}}%
\pgfpathlineto{\pgfqpoint{4.139486in}{2.441147in}}%
\pgfpathlineto{\pgfqpoint{4.132085in}{2.433902in}}%
\pgfpathlineto{\pgfqpoint{4.124677in}{2.426650in}}%
\pgfpathlineto{\pgfqpoint{4.111803in}{2.429899in}}%
\pgfpathlineto{\pgfqpoint{4.098934in}{2.433176in}}%
\pgfpathlineto{\pgfqpoint{4.086072in}{2.436482in}}%
\pgfpathlineto{\pgfqpoint{4.073215in}{2.439817in}}%
\pgfpathlineto{\pgfqpoint{4.080634in}{2.447074in}}%
\pgfpathlineto{\pgfqpoint{4.088047in}{2.454327in}}%
\pgfpathlineto{\pgfqpoint{4.095456in}{2.461578in}}%
\pgfpathlineto{\pgfqpoint{4.102859in}{2.468827in}}%
\pgfpathclose%
\pgfusepath{fill}%
\end{pgfscope}%
\begin{pgfscope}%
\pgfpathrectangle{\pgfqpoint{1.254980in}{0.150000in}}{\pgfqpoint{5.490039in}{5.490039in}}%
\pgfusepath{clip}%
\pgfsetbuttcap%
\pgfsetroundjoin%
\definecolor{currentfill}{rgb}{0.280894,0.078907,0.402329}%
\pgfsetfillcolor{currentfill}%
\pgfsetfillopacity{0.700000}%
\pgfsetlinewidth{0.000000pt}%
\definecolor{currentstroke}{rgb}{0.000000,0.000000,0.000000}%
\pgfsetstrokecolor{currentstroke}%
\pgfsetdash{}{0pt}%
\pgfpathmoveto{\pgfqpoint{5.088386in}{2.560400in}}%
\pgfpathlineto{\pgfqpoint{5.101469in}{2.557866in}}%
\pgfpathlineto{\pgfqpoint{5.114558in}{2.555357in}}%
\pgfpathlineto{\pgfqpoint{5.127654in}{2.552872in}}%
\pgfpathlineto{\pgfqpoint{5.140757in}{2.550412in}}%
\pgfpathlineto{\pgfqpoint{5.133738in}{2.543856in}}%
\pgfpathlineto{\pgfqpoint{5.126714in}{2.537339in}}%
\pgfpathlineto{\pgfqpoint{5.119685in}{2.530859in}}%
\pgfpathlineto{\pgfqpoint{5.112652in}{2.524411in}}%
\pgfpathlineto{\pgfqpoint{5.099533in}{2.526747in}}%
\pgfpathlineto{\pgfqpoint{5.086422in}{2.529108in}}%
\pgfpathlineto{\pgfqpoint{5.073317in}{2.531492in}}%
\pgfpathlineto{\pgfqpoint{5.060219in}{2.533901in}}%
\pgfpathlineto{\pgfqpoint{5.067267in}{2.540470in}}%
\pgfpathlineto{\pgfqpoint{5.074311in}{2.547073in}}%
\pgfpathlineto{\pgfqpoint{5.081351in}{2.553715in}}%
\pgfpathlineto{\pgfqpoint{5.088386in}{2.560400in}}%
\pgfpathclose%
\pgfusepath{fill}%
\end{pgfscope}%
\begin{pgfscope}%
\pgfpathrectangle{\pgfqpoint{1.254980in}{0.150000in}}{\pgfqpoint{5.490039in}{5.490039in}}%
\pgfusepath{clip}%
\pgfsetbuttcap%
\pgfsetroundjoin%
\definecolor{currentfill}{rgb}{0.279566,0.067836,0.391917}%
\pgfsetfillcolor{currentfill}%
\pgfsetfillopacity{0.700000}%
\pgfsetlinewidth{0.000000pt}%
\definecolor{currentstroke}{rgb}{0.000000,0.000000,0.000000}%
\pgfsetstrokecolor{currentstroke}%
\pgfsetdash{}{0pt}%
\pgfpathmoveto{\pgfqpoint{4.875155in}{2.537382in}}%
\pgfpathlineto{\pgfqpoint{4.888186in}{2.534789in}}%
\pgfpathlineto{\pgfqpoint{4.901224in}{2.532222in}}%
\pgfpathlineto{\pgfqpoint{4.914269in}{2.529679in}}%
\pgfpathlineto{\pgfqpoint{4.927321in}{2.527161in}}%
\pgfpathlineto{\pgfqpoint{4.920219in}{2.520505in}}%
\pgfpathlineto{\pgfqpoint{4.913112in}{2.513868in}}%
\pgfpathlineto{\pgfqpoint{4.906000in}{2.507248in}}%
\pgfpathlineto{\pgfqpoint{4.898883in}{2.500641in}}%
\pgfpathlineto{\pgfqpoint{4.885817in}{2.503059in}}%
\pgfpathlineto{\pgfqpoint{4.872758in}{2.505503in}}%
\pgfpathlineto{\pgfqpoint{4.859706in}{2.507971in}}%
\pgfpathlineto{\pgfqpoint{4.846660in}{2.510465in}}%
\pgfpathlineto{\pgfqpoint{4.853791in}{2.517167in}}%
\pgfpathlineto{\pgfqpoint{4.860917in}{2.523884in}}%
\pgfpathlineto{\pgfqpoint{4.868038in}{2.530622in}}%
\pgfpathlineto{\pgfqpoint{4.875155in}{2.537382in}}%
\pgfpathclose%
\pgfusepath{fill}%
\end{pgfscope}%
\begin{pgfscope}%
\pgfpathrectangle{\pgfqpoint{1.254980in}{0.150000in}}{\pgfqpoint{5.490039in}{5.490039in}}%
\pgfusepath{clip}%
\pgfsetbuttcap%
\pgfsetroundjoin%
\definecolor{currentfill}{rgb}{0.271305,0.019942,0.347269}%
\pgfsetfillcolor{currentfill}%
\pgfsetfillopacity{0.700000}%
\pgfsetlinewidth{0.000000pt}%
\definecolor{currentstroke}{rgb}{0.000000,0.000000,0.000000}%
\pgfsetstrokecolor{currentstroke}%
\pgfsetdash{}{0pt}%
\pgfpathmoveto{\pgfqpoint{3.279088in}{2.471841in}}%
\pgfpathlineto{\pgfqpoint{3.291777in}{2.466679in}}%
\pgfpathlineto{\pgfqpoint{3.304471in}{2.461557in}}%
\pgfpathlineto{\pgfqpoint{3.317169in}{2.456474in}}%
\pgfpathlineto{\pgfqpoint{3.329872in}{2.451430in}}%
\pgfpathlineto{\pgfqpoint{3.322170in}{2.444909in}}%
\pgfpathlineto{\pgfqpoint{3.314463in}{2.438431in}}%
\pgfpathlineto{\pgfqpoint{3.306748in}{2.431997in}}%
\pgfpathlineto{\pgfqpoint{3.299027in}{2.425608in}}%
\pgfpathlineto{\pgfqpoint{3.286311in}{2.430731in}}%
\pgfpathlineto{\pgfqpoint{3.273598in}{2.435893in}}%
\pgfpathlineto{\pgfqpoint{3.260890in}{2.441094in}}%
\pgfpathlineto{\pgfqpoint{3.248186in}{2.446335in}}%
\pgfpathlineto{\pgfqpoint{3.255921in}{2.452640in}}%
\pgfpathlineto{\pgfqpoint{3.263650in}{2.458994in}}%
\pgfpathlineto{\pgfqpoint{3.271372in}{2.465395in}}%
\pgfpathlineto{\pgfqpoint{3.279088in}{2.471841in}}%
\pgfpathclose%
\pgfusepath{fill}%
\end{pgfscope}%
\begin{pgfscope}%
\pgfpathrectangle{\pgfqpoint{1.254980in}{0.150000in}}{\pgfqpoint{5.490039in}{5.490039in}}%
\pgfusepath{clip}%
\pgfsetbuttcap%
\pgfsetroundjoin%
\definecolor{currentfill}{rgb}{0.277018,0.050344,0.375715}%
\pgfsetfillcolor{currentfill}%
\pgfsetfillopacity{0.700000}%
\pgfsetlinewidth{0.000000pt}%
\definecolor{currentstroke}{rgb}{0.000000,0.000000,0.000000}%
\pgfsetstrokecolor{currentstroke}%
\pgfsetdash{}{0pt}%
\pgfpathmoveto{\pgfqpoint{4.661889in}{2.514613in}}%
\pgfpathlineto{\pgfqpoint{4.674869in}{2.511902in}}%
\pgfpathlineto{\pgfqpoint{4.687855in}{2.509216in}}%
\pgfpathlineto{\pgfqpoint{4.700849in}{2.506556in}}%
\pgfpathlineto{\pgfqpoint{4.713848in}{2.503922in}}%
\pgfpathlineto{\pgfqpoint{4.706664in}{2.497099in}}%
\pgfpathlineto{\pgfqpoint{4.699474in}{2.490282in}}%
\pgfpathlineto{\pgfqpoint{4.692280in}{2.483467in}}%
\pgfpathlineto{\pgfqpoint{4.685080in}{2.476652in}}%
\pgfpathlineto{\pgfqpoint{4.672067in}{2.479213in}}%
\pgfpathlineto{\pgfqpoint{4.659060in}{2.481799in}}%
\pgfpathlineto{\pgfqpoint{4.646060in}{2.484411in}}%
\pgfpathlineto{\pgfqpoint{4.633067in}{2.487049in}}%
\pgfpathlineto{\pgfqpoint{4.640280in}{2.493933in}}%
\pgfpathlineto{\pgfqpoint{4.647488in}{2.500820in}}%
\pgfpathlineto{\pgfqpoint{4.654691in}{2.507713in}}%
\pgfpathlineto{\pgfqpoint{4.661889in}{2.514613in}}%
\pgfpathclose%
\pgfusepath{fill}%
\end{pgfscope}%
\begin{pgfscope}%
\pgfpathrectangle{\pgfqpoint{1.254980in}{0.150000in}}{\pgfqpoint{5.490039in}{5.490039in}}%
\pgfusepath{clip}%
\pgfsetbuttcap%
\pgfsetroundjoin%
\definecolor{currentfill}{rgb}{0.273809,0.031497,0.358853}%
\pgfsetfillcolor{currentfill}%
\pgfsetfillopacity{0.700000}%
\pgfsetlinewidth{0.000000pt}%
\definecolor{currentstroke}{rgb}{0.000000,0.000000,0.000000}%
\pgfsetstrokecolor{currentstroke}%
\pgfsetdash{}{0pt}%
\pgfpathmoveto{\pgfqpoint{3.146696in}{2.489737in}}%
\pgfpathlineto{\pgfqpoint{3.159369in}{2.484165in}}%
\pgfpathlineto{\pgfqpoint{3.172045in}{2.478636in}}%
\pgfpathlineto{\pgfqpoint{3.184725in}{2.473149in}}%
\pgfpathlineto{\pgfqpoint{3.197409in}{2.467704in}}%
\pgfpathlineto{\pgfqpoint{3.189652in}{2.461536in}}%
\pgfpathlineto{\pgfqpoint{3.181888in}{2.455424in}}%
\pgfpathlineto{\pgfqpoint{3.174116in}{2.449369in}}%
\pgfpathlineto{\pgfqpoint{3.166338in}{2.443374in}}%
\pgfpathlineto{\pgfqpoint{3.153638in}{2.448911in}}%
\pgfpathlineto{\pgfqpoint{3.140943in}{2.454489in}}%
\pgfpathlineto{\pgfqpoint{3.128251in}{2.460111in}}%
\pgfpathlineto{\pgfqpoint{3.115563in}{2.465774in}}%
\pgfpathlineto{\pgfqpoint{3.123357in}{2.471673in}}%
\pgfpathlineto{\pgfqpoint{3.131144in}{2.477635in}}%
\pgfpathlineto{\pgfqpoint{3.138924in}{2.483657in}}%
\pgfpathlineto{\pgfqpoint{3.146696in}{2.489737in}}%
\pgfpathclose%
\pgfusepath{fill}%
\end{pgfscope}%
\begin{pgfscope}%
\pgfpathrectangle{\pgfqpoint{1.254980in}{0.150000in}}{\pgfqpoint{5.490039in}{5.490039in}}%
\pgfusepath{clip}%
\pgfsetbuttcap%
\pgfsetroundjoin%
\definecolor{currentfill}{rgb}{0.269944,0.014625,0.341379}%
\pgfsetfillcolor{currentfill}%
\pgfsetfillopacity{0.700000}%
\pgfsetlinewidth{0.000000pt}%
\definecolor{currentstroke}{rgb}{0.000000,0.000000,0.000000}%
\pgfsetstrokecolor{currentstroke}%
\pgfsetdash{}{0pt}%
\pgfpathmoveto{\pgfqpoint{3.411411in}{2.458379in}}%
\pgfpathlineto{\pgfqpoint{3.424121in}{2.453592in}}%
\pgfpathlineto{\pgfqpoint{3.436836in}{2.448842in}}%
\pgfpathlineto{\pgfqpoint{3.449556in}{2.444129in}}%
\pgfpathlineto{\pgfqpoint{3.462280in}{2.439454in}}%
\pgfpathlineto{\pgfqpoint{3.454632in}{2.432656in}}%
\pgfpathlineto{\pgfqpoint{3.446977in}{2.425888in}}%
\pgfpathlineto{\pgfqpoint{3.439316in}{2.419152in}}%
\pgfpathlineto{\pgfqpoint{3.431649in}{2.412449in}}%
\pgfpathlineto{\pgfqpoint{3.418911in}{2.417191in}}%
\pgfpathlineto{\pgfqpoint{3.406178in}{2.421970in}}%
\pgfpathlineto{\pgfqpoint{3.393449in}{2.426786in}}%
\pgfpathlineto{\pgfqpoint{3.380725in}{2.431639in}}%
\pgfpathlineto{\pgfqpoint{3.388406in}{2.438271in}}%
\pgfpathlineto{\pgfqpoint{3.396080in}{2.444939in}}%
\pgfpathlineto{\pgfqpoint{3.403749in}{2.451642in}}%
\pgfpathlineto{\pgfqpoint{3.411411in}{2.458379in}}%
\pgfpathclose%
\pgfusepath{fill}%
\end{pgfscope}%
\begin{pgfscope}%
\pgfpathrectangle{\pgfqpoint{1.254980in}{0.150000in}}{\pgfqpoint{5.490039in}{5.490039in}}%
\pgfusepath{clip}%
\pgfsetbuttcap%
\pgfsetroundjoin%
\definecolor{currentfill}{rgb}{0.269944,0.014625,0.341379}%
\pgfsetfillcolor{currentfill}%
\pgfsetfillopacity{0.700000}%
\pgfsetlinewidth{0.000000pt}%
\definecolor{currentstroke}{rgb}{0.000000,0.000000,0.000000}%
\pgfsetstrokecolor{currentstroke}%
\pgfsetdash{}{0pt}%
\pgfpathmoveto{\pgfqpoint{3.889503in}{2.453121in}}%
\pgfpathlineto{\pgfqpoint{3.902304in}{2.449422in}}%
\pgfpathlineto{\pgfqpoint{3.915110in}{2.445755in}}%
\pgfpathlineto{\pgfqpoint{3.927922in}{2.442118in}}%
\pgfpathlineto{\pgfqpoint{3.940740in}{2.438511in}}%
\pgfpathlineto{\pgfqpoint{3.933270in}{2.431250in}}%
\pgfpathlineto{\pgfqpoint{3.925795in}{2.423990in}}%
\pgfpathlineto{\pgfqpoint{3.918314in}{2.416729in}}%
\pgfpathlineto{\pgfqpoint{3.910828in}{2.409469in}}%
\pgfpathlineto{\pgfqpoint{3.897998in}{2.413090in}}%
\pgfpathlineto{\pgfqpoint{3.885174in}{2.416742in}}%
\pgfpathlineto{\pgfqpoint{3.872355in}{2.420425in}}%
\pgfpathlineto{\pgfqpoint{3.859542in}{2.424139in}}%
\pgfpathlineto{\pgfqpoint{3.867041in}{2.431380in}}%
\pgfpathlineto{\pgfqpoint{3.874533in}{2.438624in}}%
\pgfpathlineto{\pgfqpoint{3.882021in}{2.445871in}}%
\pgfpathlineto{\pgfqpoint{3.889503in}{2.453121in}}%
\pgfpathclose%
\pgfusepath{fill}%
\end{pgfscope}%
\begin{pgfscope}%
\pgfpathrectangle{\pgfqpoint{1.254980in}{0.150000in}}{\pgfqpoint{5.490039in}{5.490039in}}%
\pgfusepath{clip}%
\pgfsetbuttcap%
\pgfsetroundjoin%
\definecolor{currentfill}{rgb}{0.274952,0.037752,0.364543}%
\pgfsetfillcolor{currentfill}%
\pgfsetfillopacity{0.700000}%
\pgfsetlinewidth{0.000000pt}%
\definecolor{currentstroke}{rgb}{0.000000,0.000000,0.000000}%
\pgfsetstrokecolor{currentstroke}%
\pgfsetdash{}{0pt}%
\pgfpathmoveto{\pgfqpoint{4.448587in}{2.492462in}}%
\pgfpathlineto{\pgfqpoint{4.461517in}{2.489570in}}%
\pgfpathlineto{\pgfqpoint{4.474453in}{2.486704in}}%
\pgfpathlineto{\pgfqpoint{4.487395in}{2.483865in}}%
\pgfpathlineto{\pgfqpoint{4.500343in}{2.481052in}}%
\pgfpathlineto{\pgfqpoint{4.493077in}{2.474043in}}%
\pgfpathlineto{\pgfqpoint{4.485806in}{2.467031in}}%
\pgfpathlineto{\pgfqpoint{4.478530in}{2.460013in}}%
\pgfpathlineto{\pgfqpoint{4.471249in}{2.452988in}}%
\pgfpathlineto{\pgfqpoint{4.458288in}{2.455752in}}%
\pgfpathlineto{\pgfqpoint{4.445333in}{2.458543in}}%
\pgfpathlineto{\pgfqpoint{4.432385in}{2.461360in}}%
\pgfpathlineto{\pgfqpoint{4.419443in}{2.464205in}}%
\pgfpathlineto{\pgfqpoint{4.426737in}{2.471274in}}%
\pgfpathlineto{\pgfqpoint{4.434026in}{2.478339in}}%
\pgfpathlineto{\pgfqpoint{4.441309in}{2.485401in}}%
\pgfpathlineto{\pgfqpoint{4.448587in}{2.492462in}}%
\pgfpathclose%
\pgfusepath{fill}%
\end{pgfscope}%
\begin{pgfscope}%
\pgfpathrectangle{\pgfqpoint{1.254980in}{0.150000in}}{\pgfqpoint{5.490039in}{5.490039in}}%
\pgfusepath{clip}%
\pgfsetbuttcap%
\pgfsetroundjoin%
\definecolor{currentfill}{rgb}{0.277018,0.050344,0.375715}%
\pgfsetfillcolor{currentfill}%
\pgfsetfillopacity{0.700000}%
\pgfsetlinewidth{0.000000pt}%
\definecolor{currentstroke}{rgb}{0.000000,0.000000,0.000000}%
\pgfsetstrokecolor{currentstroke}%
\pgfsetdash{}{0pt}%
\pgfpathmoveto{\pgfqpoint{3.014190in}{2.512669in}}%
\pgfpathlineto{\pgfqpoint{3.026849in}{2.506649in}}%
\pgfpathlineto{\pgfqpoint{3.039512in}{2.500675in}}%
\pgfpathlineto{\pgfqpoint{3.052178in}{2.494747in}}%
\pgfpathlineto{\pgfqpoint{3.064848in}{2.488864in}}%
\pgfpathlineto{\pgfqpoint{3.057031in}{2.483130in}}%
\pgfpathlineto{\pgfqpoint{3.049206in}{2.477466in}}%
\pgfpathlineto{\pgfqpoint{3.041373in}{2.471874in}}%
\pgfpathlineto{\pgfqpoint{3.033532in}{2.466357in}}%
\pgfpathlineto{\pgfqpoint{3.020846in}{2.472345in}}%
\pgfpathlineto{\pgfqpoint{3.008163in}{2.478378in}}%
\pgfpathlineto{\pgfqpoint{2.995484in}{2.484457in}}%
\pgfpathlineto{\pgfqpoint{2.982808in}{2.490581in}}%
\pgfpathlineto{\pgfqpoint{2.990665in}{2.495988in}}%
\pgfpathlineto{\pgfqpoint{2.998515in}{2.501474in}}%
\pgfpathlineto{\pgfqpoint{3.006356in}{2.507035in}}%
\pgfpathlineto{\pgfqpoint{3.014190in}{2.512669in}}%
\pgfpathclose%
\pgfusepath{fill}%
\end{pgfscope}%
\begin{pgfscope}%
\pgfpathrectangle{\pgfqpoint{1.254980in}{0.150000in}}{\pgfqpoint{5.490039in}{5.490039in}}%
\pgfusepath{clip}%
\pgfsetbuttcap%
\pgfsetroundjoin%
\definecolor{currentfill}{rgb}{0.268510,0.009605,0.335427}%
\pgfsetfillcolor{currentfill}%
\pgfsetfillopacity{0.700000}%
\pgfsetlinewidth{0.000000pt}%
\definecolor{currentstroke}{rgb}{0.000000,0.000000,0.000000}%
\pgfsetstrokecolor{currentstroke}%
\pgfsetdash{}{0pt}%
\pgfpathmoveto{\pgfqpoint{3.543706in}{2.448799in}}%
\pgfpathlineto{\pgfqpoint{3.556441in}{2.444356in}}%
\pgfpathlineto{\pgfqpoint{3.569180in}{2.439948in}}%
\pgfpathlineto{\pgfqpoint{3.581925in}{2.435575in}}%
\pgfpathlineto{\pgfqpoint{3.594674in}{2.431236in}}%
\pgfpathlineto{\pgfqpoint{3.587076in}{2.424230in}}%
\pgfpathlineto{\pgfqpoint{3.579471in}{2.417244in}}%
\pgfpathlineto{\pgfqpoint{3.571861in}{2.410279in}}%
\pgfpathlineto{\pgfqpoint{3.564245in}{2.403335in}}%
\pgfpathlineto{\pgfqpoint{3.551482in}{2.407727in}}%
\pgfpathlineto{\pgfqpoint{3.538725in}{2.412154in}}%
\pgfpathlineto{\pgfqpoint{3.525972in}{2.416615in}}%
\pgfpathlineto{\pgfqpoint{3.513224in}{2.421111in}}%
\pgfpathlineto{\pgfqpoint{3.520854in}{2.427997in}}%
\pgfpathlineto{\pgfqpoint{3.528477in}{2.434907in}}%
\pgfpathlineto{\pgfqpoint{3.536094in}{2.441841in}}%
\pgfpathlineto{\pgfqpoint{3.543706in}{2.448799in}}%
\pgfpathclose%
\pgfusepath{fill}%
\end{pgfscope}%
\begin{pgfscope}%
\pgfpathrectangle{\pgfqpoint{1.254980in}{0.150000in}}{\pgfqpoint{5.490039in}{5.490039in}}%
\pgfusepath{clip}%
\pgfsetbuttcap%
\pgfsetroundjoin%
\definecolor{currentfill}{rgb}{0.282656,0.100196,0.422160}%
\pgfsetfillcolor{currentfill}%
\pgfsetfillopacity{0.700000}%
\pgfsetlinewidth{0.000000pt}%
\definecolor{currentstroke}{rgb}{0.000000,0.000000,0.000000}%
\pgfsetstrokecolor{currentstroke}%
\pgfsetdash{}{0pt}%
\pgfpathmoveto{\pgfqpoint{5.434501in}{2.590468in}}%
\pgfpathlineto{\pgfqpoint{5.447672in}{2.587962in}}%
\pgfpathlineto{\pgfqpoint{5.460850in}{2.585478in}}%
\pgfpathlineto{\pgfqpoint{5.474036in}{2.583019in}}%
\pgfpathlineto{\pgfqpoint{5.487228in}{2.580582in}}%
\pgfpathlineto{\pgfqpoint{5.480336in}{2.574036in}}%
\pgfpathlineto{\pgfqpoint{5.473441in}{2.567569in}}%
\pgfpathlineto{\pgfqpoint{5.466543in}{2.561177in}}%
\pgfpathlineto{\pgfqpoint{5.459642in}{2.554854in}}%
\pgfpathlineto{\pgfqpoint{5.446432in}{2.557128in}}%
\pgfpathlineto{\pgfqpoint{5.433229in}{2.559426in}}%
\pgfpathlineto{\pgfqpoint{5.420033in}{2.561747in}}%
\pgfpathlineto{\pgfqpoint{5.406844in}{2.564091in}}%
\pgfpathlineto{\pgfqpoint{5.413763in}{2.570572in}}%
\pgfpathlineto{\pgfqpoint{5.420678in}{2.577125in}}%
\pgfpathlineto{\pgfqpoint{5.427591in}{2.583755in}}%
\pgfpathlineto{\pgfqpoint{5.434501in}{2.590468in}}%
\pgfpathclose%
\pgfusepath{fill}%
\end{pgfscope}%
\begin{pgfscope}%
\pgfpathrectangle{\pgfqpoint{1.254980in}{0.150000in}}{\pgfqpoint{5.490039in}{5.490039in}}%
\pgfusepath{clip}%
\pgfsetbuttcap%
\pgfsetroundjoin%
\definecolor{currentfill}{rgb}{0.272594,0.025563,0.353093}%
\pgfsetfillcolor{currentfill}%
\pgfsetfillopacity{0.700000}%
\pgfsetlinewidth{0.000000pt}%
\definecolor{currentstroke}{rgb}{0.000000,0.000000,0.000000}%
\pgfsetstrokecolor{currentstroke}%
\pgfsetdash{}{0pt}%
\pgfpathmoveto{\pgfqpoint{4.235246in}{2.471689in}}%
\pgfpathlineto{\pgfqpoint{4.248126in}{2.468551in}}%
\pgfpathlineto{\pgfqpoint{4.261012in}{2.465440in}}%
\pgfpathlineto{\pgfqpoint{4.273905in}{2.462357in}}%
\pgfpathlineto{\pgfqpoint{4.286803in}{2.459302in}}%
\pgfpathlineto{\pgfqpoint{4.279458in}{2.452133in}}%
\pgfpathlineto{\pgfqpoint{4.272107in}{2.444956in}}%
\pgfpathlineto{\pgfqpoint{4.264751in}{2.437771in}}%
\pgfpathlineto{\pgfqpoint{4.257389in}{2.430577in}}%
\pgfpathlineto{\pgfqpoint{4.244478in}{2.433609in}}%
\pgfpathlineto{\pgfqpoint{4.231574in}{2.436669in}}%
\pgfpathlineto{\pgfqpoint{4.218676in}{2.439757in}}%
\pgfpathlineto{\pgfqpoint{4.205783in}{2.442873in}}%
\pgfpathlineto{\pgfqpoint{4.213157in}{2.450085in}}%
\pgfpathlineto{\pgfqpoint{4.220525in}{2.457291in}}%
\pgfpathlineto{\pgfqpoint{4.227888in}{2.464492in}}%
\pgfpathlineto{\pgfqpoint{4.235246in}{2.471689in}}%
\pgfpathclose%
\pgfusepath{fill}%
\end{pgfscope}%
\begin{pgfscope}%
\pgfpathrectangle{\pgfqpoint{1.254980in}{0.150000in}}{\pgfqpoint{5.490039in}{5.490039in}}%
\pgfusepath{clip}%
\pgfsetbuttcap%
\pgfsetroundjoin%
\definecolor{currentfill}{rgb}{0.281446,0.084320,0.407414}%
\pgfsetfillcolor{currentfill}%
\pgfsetfillopacity{0.700000}%
\pgfsetlinewidth{0.000000pt}%
\definecolor{currentstroke}{rgb}{0.000000,0.000000,0.000000}%
\pgfsetstrokecolor{currentstroke}%
\pgfsetdash{}{0pt}%
\pgfpathmoveto{\pgfqpoint{5.221212in}{2.566967in}}%
\pgfpathlineto{\pgfqpoint{5.234334in}{2.564489in}}%
\pgfpathlineto{\pgfqpoint{5.247462in}{2.562035in}}%
\pgfpathlineto{\pgfqpoint{5.260598in}{2.559605in}}%
\pgfpathlineto{\pgfqpoint{5.273740in}{2.557198in}}%
\pgfpathlineto{\pgfqpoint{5.266769in}{2.550724in}}%
\pgfpathlineto{\pgfqpoint{5.259794in}{2.544301in}}%
\pgfpathlineto{\pgfqpoint{5.252815in}{2.537923in}}%
\pgfpathlineto{\pgfqpoint{5.245832in}{2.531588in}}%
\pgfpathlineto{\pgfqpoint{5.232673in}{2.533858in}}%
\pgfpathlineto{\pgfqpoint{5.219521in}{2.536151in}}%
\pgfpathlineto{\pgfqpoint{5.206377in}{2.538468in}}%
\pgfpathlineto{\pgfqpoint{5.193239in}{2.540809in}}%
\pgfpathlineto{\pgfqpoint{5.200238in}{2.547276in}}%
\pgfpathlineto{\pgfqpoint{5.207233in}{2.553789in}}%
\pgfpathlineto{\pgfqpoint{5.214225in}{2.560351in}}%
\pgfpathlineto{\pgfqpoint{5.221212in}{2.566967in}}%
\pgfpathclose%
\pgfusepath{fill}%
\end{pgfscope}%
\begin{pgfscope}%
\pgfpathrectangle{\pgfqpoint{1.254980in}{0.150000in}}{\pgfqpoint{5.490039in}{5.490039in}}%
\pgfusepath{clip}%
\pgfsetbuttcap%
\pgfsetroundjoin%
\definecolor{currentfill}{rgb}{0.280267,0.073417,0.397163}%
\pgfsetfillcolor{currentfill}%
\pgfsetfillopacity{0.700000}%
\pgfsetlinewidth{0.000000pt}%
\definecolor{currentstroke}{rgb}{0.000000,0.000000,0.000000}%
\pgfsetstrokecolor{currentstroke}%
\pgfsetdash{}{0pt}%
\pgfpathmoveto{\pgfqpoint{5.007896in}{2.543781in}}%
\pgfpathlineto{\pgfqpoint{5.020966in}{2.541274in}}%
\pgfpathlineto{\pgfqpoint{5.034044in}{2.538792in}}%
\pgfpathlineto{\pgfqpoint{5.047128in}{2.536335in}}%
\pgfpathlineto{\pgfqpoint{5.060219in}{2.533901in}}%
\pgfpathlineto{\pgfqpoint{5.053166in}{2.527365in}}%
\pgfpathlineto{\pgfqpoint{5.046108in}{2.520855in}}%
\pgfpathlineto{\pgfqpoint{5.039045in}{2.514370in}}%
\pgfpathlineto{\pgfqpoint{5.031978in}{2.507904in}}%
\pgfpathlineto{\pgfqpoint{5.018872in}{2.510226in}}%
\pgfpathlineto{\pgfqpoint{5.005773in}{2.512572in}}%
\pgfpathlineto{\pgfqpoint{4.992680in}{2.514942in}}%
\pgfpathlineto{\pgfqpoint{4.979595in}{2.517337in}}%
\pgfpathlineto{\pgfqpoint{4.986677in}{2.523909in}}%
\pgfpathlineto{\pgfqpoint{4.993754in}{2.530505in}}%
\pgfpathlineto{\pgfqpoint{5.000827in}{2.537128in}}%
\pgfpathlineto{\pgfqpoint{5.007896in}{2.543781in}}%
\pgfpathclose%
\pgfusepath{fill}%
\end{pgfscope}%
\begin{pgfscope}%
\pgfpathrectangle{\pgfqpoint{1.254980in}{0.150000in}}{\pgfqpoint{5.490039in}{5.490039in}}%
\pgfusepath{clip}%
\pgfsetbuttcap%
\pgfsetroundjoin%
\definecolor{currentfill}{rgb}{0.268510,0.009605,0.335427}%
\pgfsetfillcolor{currentfill}%
\pgfsetfillopacity{0.700000}%
\pgfsetlinewidth{0.000000pt}%
\definecolor{currentstroke}{rgb}{0.000000,0.000000,0.000000}%
\pgfsetstrokecolor{currentstroke}%
\pgfsetdash{}{0pt}%
\pgfpathmoveto{\pgfqpoint{3.676007in}{2.442597in}}%
\pgfpathlineto{\pgfqpoint{3.688769in}{2.438469in}}%
\pgfpathlineto{\pgfqpoint{3.701536in}{2.434374in}}%
\pgfpathlineto{\pgfqpoint{3.714308in}{2.430312in}}%
\pgfpathlineto{\pgfqpoint{3.727086in}{2.426283in}}%
\pgfpathlineto{\pgfqpoint{3.719535in}{2.419132in}}%
\pgfpathlineto{\pgfqpoint{3.711979in}{2.411993in}}%
\pgfpathlineto{\pgfqpoint{3.704417in}{2.404864in}}%
\pgfpathlineto{\pgfqpoint{3.696850in}{2.397748in}}%
\pgfpathlineto{\pgfqpoint{3.684060in}{2.401818in}}%
\pgfpathlineto{\pgfqpoint{3.671275in}{2.405920in}}%
\pgfpathlineto{\pgfqpoint{3.658496in}{2.410056in}}%
\pgfpathlineto{\pgfqpoint{3.645721in}{2.414224in}}%
\pgfpathlineto{\pgfqpoint{3.653301in}{2.421295in}}%
\pgfpathlineto{\pgfqpoint{3.660876in}{2.428381in}}%
\pgfpathlineto{\pgfqpoint{3.668444in}{2.435482in}}%
\pgfpathlineto{\pgfqpoint{3.676007in}{2.442597in}}%
\pgfpathclose%
\pgfusepath{fill}%
\end{pgfscope}%
\begin{pgfscope}%
\pgfpathrectangle{\pgfqpoint{1.254980in}{0.150000in}}{\pgfqpoint{5.490039in}{5.490039in}}%
\pgfusepath{clip}%
\pgfsetbuttcap%
\pgfsetroundjoin%
\definecolor{currentfill}{rgb}{0.269944,0.014625,0.341379}%
\pgfsetfillcolor{currentfill}%
\pgfsetfillopacity{0.700000}%
\pgfsetlinewidth{0.000000pt}%
\definecolor{currentstroke}{rgb}{0.000000,0.000000,0.000000}%
\pgfsetstrokecolor{currentstroke}%
\pgfsetdash{}{0pt}%
\pgfpathmoveto{\pgfqpoint{4.021845in}{2.453451in}}%
\pgfpathlineto{\pgfqpoint{4.034678in}{2.449998in}}%
\pgfpathlineto{\pgfqpoint{4.047518in}{2.446575in}}%
\pgfpathlineto{\pgfqpoint{4.060363in}{2.443181in}}%
\pgfpathlineto{\pgfqpoint{4.073215in}{2.439817in}}%
\pgfpathlineto{\pgfqpoint{4.065790in}{2.432556in}}%
\pgfpathlineto{\pgfqpoint{4.058360in}{2.425291in}}%
\pgfpathlineto{\pgfqpoint{4.050925in}{2.418020in}}%
\pgfpathlineto{\pgfqpoint{4.043484in}{2.410744in}}%
\pgfpathlineto{\pgfqpoint{4.030621in}{2.414111in}}%
\pgfpathlineto{\pgfqpoint{4.017764in}{2.417507in}}%
\pgfpathlineto{\pgfqpoint{4.004912in}{2.420933in}}%
\pgfpathlineto{\pgfqpoint{3.992067in}{2.424389in}}%
\pgfpathlineto{\pgfqpoint{3.999519in}{2.431658in}}%
\pgfpathlineto{\pgfqpoint{4.006966in}{2.438924in}}%
\pgfpathlineto{\pgfqpoint{4.014408in}{2.446188in}}%
\pgfpathlineto{\pgfqpoint{4.021845in}{2.453451in}}%
\pgfpathclose%
\pgfusepath{fill}%
\end{pgfscope}%
\begin{pgfscope}%
\pgfpathrectangle{\pgfqpoint{1.254980in}{0.150000in}}{\pgfqpoint{5.490039in}{5.490039in}}%
\pgfusepath{clip}%
\pgfsetbuttcap%
\pgfsetroundjoin%
\definecolor{currentfill}{rgb}{0.279566,0.067836,0.391917}%
\pgfsetfillcolor{currentfill}%
\pgfsetfillopacity{0.700000}%
\pgfsetlinewidth{0.000000pt}%
\definecolor{currentstroke}{rgb}{0.000000,0.000000,0.000000}%
\pgfsetstrokecolor{currentstroke}%
\pgfsetdash{}{0pt}%
\pgfpathmoveto{\pgfqpoint{2.881514in}{2.541291in}}%
\pgfpathlineto{\pgfqpoint{2.894165in}{2.534781in}}%
\pgfpathlineto{\pgfqpoint{2.906819in}{2.528321in}}%
\pgfpathlineto{\pgfqpoint{2.919476in}{2.521911in}}%
\pgfpathlineto{\pgfqpoint{2.932136in}{2.515549in}}%
\pgfpathlineto{\pgfqpoint{2.924253in}{2.510335in}}%
\pgfpathlineto{\pgfqpoint{2.916362in}{2.505208in}}%
\pgfpathlineto{\pgfqpoint{2.908462in}{2.500169in}}%
\pgfpathlineto{\pgfqpoint{2.900554in}{2.495223in}}%
\pgfpathlineto{\pgfqpoint{2.887876in}{2.501702in}}%
\pgfpathlineto{\pgfqpoint{2.875201in}{2.508231in}}%
\pgfpathlineto{\pgfqpoint{2.862529in}{2.514809in}}%
\pgfpathlineto{\pgfqpoint{2.849860in}{2.521436in}}%
\pgfpathlineto{\pgfqpoint{2.857787in}{2.526260in}}%
\pgfpathlineto{\pgfqpoint{2.865705in}{2.531179in}}%
\pgfpathlineto{\pgfqpoint{2.873613in}{2.536191in}}%
\pgfpathlineto{\pgfqpoint{2.881514in}{2.541291in}}%
\pgfpathclose%
\pgfusepath{fill}%
\end{pgfscope}%
\begin{pgfscope}%
\pgfpathrectangle{\pgfqpoint{1.254980in}{0.150000in}}{\pgfqpoint{5.490039in}{5.490039in}}%
\pgfusepath{clip}%
\pgfsetbuttcap%
\pgfsetroundjoin%
\definecolor{currentfill}{rgb}{0.278791,0.062145,0.386592}%
\pgfsetfillcolor{currentfill}%
\pgfsetfillopacity{0.700000}%
\pgfsetlinewidth{0.000000pt}%
\definecolor{currentstroke}{rgb}{0.000000,0.000000,0.000000}%
\pgfsetstrokecolor{currentstroke}%
\pgfsetdash{}{0pt}%
\pgfpathmoveto{\pgfqpoint{4.794544in}{2.520689in}}%
\pgfpathlineto{\pgfqpoint{4.807563in}{2.518095in}}%
\pgfpathlineto{\pgfqpoint{4.820589in}{2.515527in}}%
\pgfpathlineto{\pgfqpoint{4.833621in}{2.512983in}}%
\pgfpathlineto{\pgfqpoint{4.846660in}{2.510465in}}%
\pgfpathlineto{\pgfqpoint{4.839524in}{2.503776in}}%
\pgfpathlineto{\pgfqpoint{4.832383in}{2.497098in}}%
\pgfpathlineto{\pgfqpoint{4.825236in}{2.490427in}}%
\pgfpathlineto{\pgfqpoint{4.818084in}{2.483760in}}%
\pgfpathlineto{\pgfqpoint{4.805032in}{2.486192in}}%
\pgfpathlineto{\pgfqpoint{4.791985in}{2.488649in}}%
\pgfpathlineto{\pgfqpoint{4.778946in}{2.491131in}}%
\pgfpathlineto{\pgfqpoint{4.765913in}{2.493638in}}%
\pgfpathlineto{\pgfqpoint{4.773078in}{2.500387in}}%
\pgfpathlineto{\pgfqpoint{4.780239in}{2.507143in}}%
\pgfpathlineto{\pgfqpoint{4.787394in}{2.513910in}}%
\pgfpathlineto{\pgfqpoint{4.794544in}{2.520689in}}%
\pgfpathclose%
\pgfusepath{fill}%
\end{pgfscope}%
\begin{pgfscope}%
\pgfpathrectangle{\pgfqpoint{1.254980in}{0.150000in}}{\pgfqpoint{5.490039in}{5.490039in}}%
\pgfusepath{clip}%
\pgfsetbuttcap%
\pgfsetroundjoin%
\definecolor{currentfill}{rgb}{0.276022,0.044167,0.370164}%
\pgfsetfillcolor{currentfill}%
\pgfsetfillopacity{0.700000}%
\pgfsetlinewidth{0.000000pt}%
\definecolor{currentstroke}{rgb}{0.000000,0.000000,0.000000}%
\pgfsetstrokecolor{currentstroke}%
\pgfsetdash{}{0pt}%
\pgfpathmoveto{\pgfqpoint{4.581158in}{2.497860in}}%
\pgfpathlineto{\pgfqpoint{4.594126in}{2.495119in}}%
\pgfpathlineto{\pgfqpoint{4.607100in}{2.492403in}}%
\pgfpathlineto{\pgfqpoint{4.620080in}{2.489713in}}%
\pgfpathlineto{\pgfqpoint{4.633067in}{2.487049in}}%
\pgfpathlineto{\pgfqpoint{4.625848in}{2.480166in}}%
\pgfpathlineto{\pgfqpoint{4.618624in}{2.473282in}}%
\pgfpathlineto{\pgfqpoint{4.611395in}{2.466393in}}%
\pgfpathlineto{\pgfqpoint{4.604161in}{2.459499in}}%
\pgfpathlineto{\pgfqpoint{4.591161in}{2.462102in}}%
\pgfpathlineto{\pgfqpoint{4.578167in}{2.464730in}}%
\pgfpathlineto{\pgfqpoint{4.565181in}{2.467385in}}%
\pgfpathlineto{\pgfqpoint{4.552200in}{2.470066in}}%
\pgfpathlineto{\pgfqpoint{4.559448in}{2.477017in}}%
\pgfpathlineto{\pgfqpoint{4.566690in}{2.483964in}}%
\pgfpathlineto{\pgfqpoint{4.573927in}{2.490912in}}%
\pgfpathlineto{\pgfqpoint{4.581158in}{2.497860in}}%
\pgfpathclose%
\pgfusepath{fill}%
\end{pgfscope}%
\begin{pgfscope}%
\pgfpathrectangle{\pgfqpoint{1.254980in}{0.150000in}}{\pgfqpoint{5.490039in}{5.490039in}}%
\pgfusepath{clip}%
\pgfsetbuttcap%
\pgfsetroundjoin%
\definecolor{currentfill}{rgb}{0.273809,0.031497,0.358853}%
\pgfsetfillcolor{currentfill}%
\pgfsetfillopacity{0.700000}%
\pgfsetlinewidth{0.000000pt}%
\definecolor{currentstroke}{rgb}{0.000000,0.000000,0.000000}%
\pgfsetstrokecolor{currentstroke}%
\pgfsetdash{}{0pt}%
\pgfpathmoveto{\pgfqpoint{4.367738in}{2.475853in}}%
\pgfpathlineto{\pgfqpoint{4.380655in}{2.472900in}}%
\pgfpathlineto{\pgfqpoint{4.393578in}{2.469975in}}%
\pgfpathlineto{\pgfqpoint{4.406507in}{2.467076in}}%
\pgfpathlineto{\pgfqpoint{4.419443in}{2.464205in}}%
\pgfpathlineto{\pgfqpoint{4.412143in}{2.457130in}}%
\pgfpathlineto{\pgfqpoint{4.404839in}{2.450047in}}%
\pgfpathlineto{\pgfqpoint{4.397528in}{2.442956in}}%
\pgfpathlineto{\pgfqpoint{4.390213in}{2.435854in}}%
\pgfpathlineto{\pgfqpoint{4.377265in}{2.438689in}}%
\pgfpathlineto{\pgfqpoint{4.364323in}{2.441552in}}%
\pgfpathlineto{\pgfqpoint{4.351387in}{2.444442in}}%
\pgfpathlineto{\pgfqpoint{4.338458in}{2.447359in}}%
\pgfpathlineto{\pgfqpoint{4.345786in}{2.454492in}}%
\pgfpathlineto{\pgfqpoint{4.353109in}{2.461618in}}%
\pgfpathlineto{\pgfqpoint{4.360426in}{2.468738in}}%
\pgfpathlineto{\pgfqpoint{4.367738in}{2.475853in}}%
\pgfpathclose%
\pgfusepath{fill}%
\end{pgfscope}%
\begin{pgfscope}%
\pgfpathrectangle{\pgfqpoint{1.254980in}{0.150000in}}{\pgfqpoint{5.490039in}{5.490039in}}%
\pgfusepath{clip}%
\pgfsetbuttcap%
\pgfsetroundjoin%
\definecolor{currentfill}{rgb}{0.268510,0.009605,0.335427}%
\pgfsetfillcolor{currentfill}%
\pgfsetfillopacity{0.700000}%
\pgfsetlinewidth{0.000000pt}%
\definecolor{currentstroke}{rgb}{0.000000,0.000000,0.000000}%
\pgfsetstrokecolor{currentstroke}%
\pgfsetdash{}{0pt}%
\pgfpathmoveto{\pgfqpoint{3.808344in}{2.439309in}}%
\pgfpathlineto{\pgfqpoint{3.821136in}{2.435469in}}%
\pgfpathlineto{\pgfqpoint{3.833932in}{2.431661in}}%
\pgfpathlineto{\pgfqpoint{3.846735in}{2.427885in}}%
\pgfpathlineto{\pgfqpoint{3.859542in}{2.424139in}}%
\pgfpathlineto{\pgfqpoint{3.852038in}{2.416902in}}%
\pgfpathlineto{\pgfqpoint{3.844529in}{2.409668in}}%
\pgfpathlineto{\pgfqpoint{3.837014in}{2.402438in}}%
\pgfpathlineto{\pgfqpoint{3.829494in}{2.395211in}}%
\pgfpathlineto{\pgfqpoint{3.816674in}{2.398984in}}%
\pgfpathlineto{\pgfqpoint{3.803860in}{2.402789in}}%
\pgfpathlineto{\pgfqpoint{3.791051in}{2.406625in}}%
\pgfpathlineto{\pgfqpoint{3.778247in}{2.410492in}}%
\pgfpathlineto{\pgfqpoint{3.785780in}{2.417686in}}%
\pgfpathlineto{\pgfqpoint{3.793307in}{2.424887in}}%
\pgfpathlineto{\pgfqpoint{3.800828in}{2.432095in}}%
\pgfpathlineto{\pgfqpoint{3.808344in}{2.439309in}}%
\pgfpathclose%
\pgfusepath{fill}%
\end{pgfscope}%
\begin{pgfscope}%
\pgfpathrectangle{\pgfqpoint{1.254980in}{0.150000in}}{\pgfqpoint{5.490039in}{5.490039in}}%
\pgfusepath{clip}%
\pgfsetbuttcap%
\pgfsetroundjoin%
\definecolor{currentfill}{rgb}{0.282327,0.094955,0.417331}%
\pgfsetfillcolor{currentfill}%
\pgfsetfillopacity{0.700000}%
\pgfsetlinewidth{0.000000pt}%
\definecolor{currentstroke}{rgb}{0.000000,0.000000,0.000000}%
\pgfsetstrokecolor{currentstroke}%
\pgfsetdash{}{0pt}%
\pgfpathmoveto{\pgfqpoint{5.354159in}{2.573702in}}%
\pgfpathlineto{\pgfqpoint{5.367320in}{2.571264in}}%
\pgfpathlineto{\pgfqpoint{5.380487in}{2.568850in}}%
\pgfpathlineto{\pgfqpoint{5.393662in}{2.566459in}}%
\pgfpathlineto{\pgfqpoint{5.406844in}{2.564091in}}%
\pgfpathlineto{\pgfqpoint{5.399922in}{2.557678in}}%
\pgfpathlineto{\pgfqpoint{5.392996in}{2.551328in}}%
\pgfpathlineto{\pgfqpoint{5.386067in}{2.545036in}}%
\pgfpathlineto{\pgfqpoint{5.379134in}{2.538797in}}%
\pgfpathlineto{\pgfqpoint{5.365935in}{2.541015in}}%
\pgfpathlineto{\pgfqpoint{5.352743in}{2.543256in}}%
\pgfpathlineto{\pgfqpoint{5.339558in}{2.545520in}}%
\pgfpathlineto{\pgfqpoint{5.326380in}{2.547809in}}%
\pgfpathlineto{\pgfqpoint{5.333330in}{2.554193in}}%
\pgfpathlineto{\pgfqpoint{5.340277in}{2.560633in}}%
\pgfpathlineto{\pgfqpoint{5.347220in}{2.567135in}}%
\pgfpathlineto{\pgfqpoint{5.354159in}{2.573702in}}%
\pgfpathclose%
\pgfusepath{fill}%
\end{pgfscope}%
\begin{pgfscope}%
\pgfpathrectangle{\pgfqpoint{1.254980in}{0.150000in}}{\pgfqpoint{5.490039in}{5.490039in}}%
\pgfusepath{clip}%
\pgfsetbuttcap%
\pgfsetroundjoin%
\definecolor{currentfill}{rgb}{0.271305,0.019942,0.347269}%
\pgfsetfillcolor{currentfill}%
\pgfsetfillopacity{0.700000}%
\pgfsetlinewidth{0.000000pt}%
\definecolor{currentstroke}{rgb}{0.000000,0.000000,0.000000}%
\pgfsetstrokecolor{currentstroke}%
\pgfsetdash{}{0pt}%
\pgfpathmoveto{\pgfqpoint{4.154274in}{2.455620in}}%
\pgfpathlineto{\pgfqpoint{4.167142in}{2.452390in}}%
\pgfpathlineto{\pgfqpoint{4.180017in}{2.449189in}}%
\pgfpathlineto{\pgfqpoint{4.192897in}{2.446017in}}%
\pgfpathlineto{\pgfqpoint{4.205783in}{2.442873in}}%
\pgfpathlineto{\pgfqpoint{4.198404in}{2.435653in}}%
\pgfpathlineto{\pgfqpoint{4.191020in}{2.428426in}}%
\pgfpathlineto{\pgfqpoint{4.183630in}{2.421190in}}%
\pgfpathlineto{\pgfqpoint{4.176235in}{2.413944in}}%
\pgfpathlineto{\pgfqpoint{4.163337in}{2.417078in}}%
\pgfpathlineto{\pgfqpoint{4.150444in}{2.420240in}}%
\pgfpathlineto{\pgfqpoint{4.137558in}{2.423431in}}%
\pgfpathlineto{\pgfqpoint{4.124677in}{2.426650in}}%
\pgfpathlineto{\pgfqpoint{4.132085in}{2.433902in}}%
\pgfpathlineto{\pgfqpoint{4.139486in}{2.441147in}}%
\pgfpathlineto{\pgfqpoint{4.146883in}{2.448386in}}%
\pgfpathlineto{\pgfqpoint{4.154274in}{2.455620in}}%
\pgfpathclose%
\pgfusepath{fill}%
\end{pgfscope}%
\begin{pgfscope}%
\pgfpathrectangle{\pgfqpoint{1.254980in}{0.150000in}}{\pgfqpoint{5.490039in}{5.490039in}}%
\pgfusepath{clip}%
\pgfsetbuttcap%
\pgfsetroundjoin%
\definecolor{currentfill}{rgb}{0.271305,0.019942,0.347269}%
\pgfsetfillcolor{currentfill}%
\pgfsetfillopacity{0.700000}%
\pgfsetlinewidth{0.000000pt}%
\definecolor{currentstroke}{rgb}{0.000000,0.000000,0.000000}%
\pgfsetstrokecolor{currentstroke}%
\pgfsetdash{}{0pt}%
\pgfpathmoveto{\pgfqpoint{3.329872in}{2.451430in}}%
\pgfpathlineto{\pgfqpoint{3.342578in}{2.446425in}}%
\pgfpathlineto{\pgfqpoint{3.355289in}{2.441458in}}%
\pgfpathlineto{\pgfqpoint{3.368005in}{2.436529in}}%
\pgfpathlineto{\pgfqpoint{3.380725in}{2.431639in}}%
\pgfpathlineto{\pgfqpoint{3.373038in}{2.425044in}}%
\pgfpathlineto{\pgfqpoint{3.365344in}{2.418489in}}%
\pgfpathlineto{\pgfqpoint{3.357644in}{2.411974in}}%
\pgfpathlineto{\pgfqpoint{3.349937in}{2.405502in}}%
\pgfpathlineto{\pgfqpoint{3.337203in}{2.410471in}}%
\pgfpathlineto{\pgfqpoint{3.324474in}{2.415479in}}%
\pgfpathlineto{\pgfqpoint{3.311748in}{2.420524in}}%
\pgfpathlineto{\pgfqpoint{3.299027in}{2.425608in}}%
\pgfpathlineto{\pgfqpoint{3.306748in}{2.431997in}}%
\pgfpathlineto{\pgfqpoint{3.314463in}{2.438431in}}%
\pgfpathlineto{\pgfqpoint{3.322170in}{2.444909in}}%
\pgfpathlineto{\pgfqpoint{3.329872in}{2.451430in}}%
\pgfpathclose%
\pgfusepath{fill}%
\end{pgfscope}%
\begin{pgfscope}%
\pgfpathrectangle{\pgfqpoint{1.254980in}{0.150000in}}{\pgfqpoint{5.490039in}{5.490039in}}%
\pgfusepath{clip}%
\pgfsetbuttcap%
\pgfsetroundjoin%
\definecolor{currentfill}{rgb}{0.272594,0.025563,0.353093}%
\pgfsetfillcolor{currentfill}%
\pgfsetfillopacity{0.700000}%
\pgfsetlinewidth{0.000000pt}%
\definecolor{currentstroke}{rgb}{0.000000,0.000000,0.000000}%
\pgfsetstrokecolor{currentstroke}%
\pgfsetdash{}{0pt}%
\pgfpathmoveto{\pgfqpoint{3.197409in}{2.467704in}}%
\pgfpathlineto{\pgfqpoint{3.210097in}{2.462300in}}%
\pgfpathlineto{\pgfqpoint{3.222789in}{2.456938in}}%
\pgfpathlineto{\pgfqpoint{3.235485in}{2.451616in}}%
\pgfpathlineto{\pgfqpoint{3.248186in}{2.446335in}}%
\pgfpathlineto{\pgfqpoint{3.240443in}{2.440080in}}%
\pgfpathlineto{\pgfqpoint{3.232694in}{2.433878in}}%
\pgfpathlineto{\pgfqpoint{3.224938in}{2.427730in}}%
\pgfpathlineto{\pgfqpoint{3.217174in}{2.421638in}}%
\pgfpathlineto{\pgfqpoint{3.204459in}{2.427011in}}%
\pgfpathlineto{\pgfqpoint{3.191748in}{2.432424in}}%
\pgfpathlineto{\pgfqpoint{3.179041in}{2.437879in}}%
\pgfpathlineto{\pgfqpoint{3.166338in}{2.443374in}}%
\pgfpathlineto{\pgfqpoint{3.174116in}{2.449369in}}%
\pgfpathlineto{\pgfqpoint{3.181888in}{2.455424in}}%
\pgfpathlineto{\pgfqpoint{3.189652in}{2.461536in}}%
\pgfpathlineto{\pgfqpoint{3.197409in}{2.467704in}}%
\pgfpathclose%
\pgfusepath{fill}%
\end{pgfscope}%
\begin{pgfscope}%
\pgfpathrectangle{\pgfqpoint{1.254980in}{0.150000in}}{\pgfqpoint{5.490039in}{5.490039in}}%
\pgfusepath{clip}%
\pgfsetbuttcap%
\pgfsetroundjoin%
\definecolor{currentfill}{rgb}{0.281446,0.084320,0.407414}%
\pgfsetfillcolor{currentfill}%
\pgfsetfillopacity{0.700000}%
\pgfsetlinewidth{0.000000pt}%
\definecolor{currentstroke}{rgb}{0.000000,0.000000,0.000000}%
\pgfsetstrokecolor{currentstroke}%
\pgfsetdash{}{0pt}%
\pgfpathmoveto{\pgfqpoint{5.140757in}{2.550412in}}%
\pgfpathlineto{\pgfqpoint{5.153867in}{2.547975in}}%
\pgfpathlineto{\pgfqpoint{5.166984in}{2.545562in}}%
\pgfpathlineto{\pgfqpoint{5.180108in}{2.543173in}}%
\pgfpathlineto{\pgfqpoint{5.193239in}{2.540809in}}%
\pgfpathlineto{\pgfqpoint{5.186235in}{2.534382in}}%
\pgfpathlineto{\pgfqpoint{5.179227in}{2.527992in}}%
\pgfpathlineto{\pgfqpoint{5.172214in}{2.521635in}}%
\pgfpathlineto{\pgfqpoint{5.165197in}{2.515306in}}%
\pgfpathlineto{\pgfqpoint{5.152050in}{2.517546in}}%
\pgfpathlineto{\pgfqpoint{5.138911in}{2.519810in}}%
\pgfpathlineto{\pgfqpoint{5.125778in}{2.522099in}}%
\pgfpathlineto{\pgfqpoint{5.112652in}{2.524411in}}%
\pgfpathlineto{\pgfqpoint{5.119685in}{2.530859in}}%
\pgfpathlineto{\pgfqpoint{5.126714in}{2.537339in}}%
\pgfpathlineto{\pgfqpoint{5.133738in}{2.543856in}}%
\pgfpathlineto{\pgfqpoint{5.140757in}{2.550412in}}%
\pgfpathclose%
\pgfusepath{fill}%
\end{pgfscope}%
\begin{pgfscope}%
\pgfpathrectangle{\pgfqpoint{1.254980in}{0.150000in}}{\pgfqpoint{5.490039in}{5.490039in}}%
\pgfusepath{clip}%
\pgfsetbuttcap%
\pgfsetroundjoin%
\definecolor{currentfill}{rgb}{0.269944,0.014625,0.341379}%
\pgfsetfillcolor{currentfill}%
\pgfsetfillopacity{0.700000}%
\pgfsetlinewidth{0.000000pt}%
\definecolor{currentstroke}{rgb}{0.000000,0.000000,0.000000}%
\pgfsetstrokecolor{currentstroke}%
\pgfsetdash{}{0pt}%
\pgfpathmoveto{\pgfqpoint{3.462280in}{2.439454in}}%
\pgfpathlineto{\pgfqpoint{3.475009in}{2.434814in}}%
\pgfpathlineto{\pgfqpoint{3.487743in}{2.430211in}}%
\pgfpathlineto{\pgfqpoint{3.500481in}{2.425643in}}%
\pgfpathlineto{\pgfqpoint{3.513224in}{2.421111in}}%
\pgfpathlineto{\pgfqpoint{3.505589in}{2.414252in}}%
\pgfpathlineto{\pgfqpoint{3.497948in}{2.407420in}}%
\pgfpathlineto{\pgfqpoint{3.490300in}{2.400617in}}%
\pgfpathlineto{\pgfqpoint{3.482647in}{2.393843in}}%
\pgfpathlineto{\pgfqpoint{3.469890in}{2.398441in}}%
\pgfpathlineto{\pgfqpoint{3.457139in}{2.403074in}}%
\pgfpathlineto{\pgfqpoint{3.444391in}{2.407744in}}%
\pgfpathlineto{\pgfqpoint{3.431649in}{2.412449in}}%
\pgfpathlineto{\pgfqpoint{3.439316in}{2.419152in}}%
\pgfpathlineto{\pgfqpoint{3.446977in}{2.425888in}}%
\pgfpathlineto{\pgfqpoint{3.454632in}{2.432656in}}%
\pgfpathlineto{\pgfqpoint{3.462280in}{2.439454in}}%
\pgfpathclose%
\pgfusepath{fill}%
\end{pgfscope}%
\begin{pgfscope}%
\pgfpathrectangle{\pgfqpoint{1.254980in}{0.150000in}}{\pgfqpoint{5.490039in}{5.490039in}}%
\pgfusepath{clip}%
\pgfsetbuttcap%
\pgfsetroundjoin%
\definecolor{currentfill}{rgb}{0.276022,0.044167,0.370164}%
\pgfsetfillcolor{currentfill}%
\pgfsetfillopacity{0.700000}%
\pgfsetlinewidth{0.000000pt}%
\definecolor{currentstroke}{rgb}{0.000000,0.000000,0.000000}%
\pgfsetstrokecolor{currentstroke}%
\pgfsetdash{}{0pt}%
\pgfpathmoveto{\pgfqpoint{3.064848in}{2.488864in}}%
\pgfpathlineto{\pgfqpoint{3.077522in}{2.483025in}}%
\pgfpathlineto{\pgfqpoint{3.090198in}{2.477231in}}%
\pgfpathlineto{\pgfqpoint{3.102879in}{2.471481in}}%
\pgfpathlineto{\pgfqpoint{3.115563in}{2.465774in}}%
\pgfpathlineto{\pgfqpoint{3.107762in}{2.459940in}}%
\pgfpathlineto{\pgfqpoint{3.099953in}{2.454173in}}%
\pgfpathlineto{\pgfqpoint{3.092136in}{2.448476in}}%
\pgfpathlineto{\pgfqpoint{3.084312in}{2.442850in}}%
\pgfpathlineto{\pgfqpoint{3.071612in}{2.448661in}}%
\pgfpathlineto{\pgfqpoint{3.058915in}{2.454516in}}%
\pgfpathlineto{\pgfqpoint{3.046222in}{2.460414in}}%
\pgfpathlineto{\pgfqpoint{3.033532in}{2.466357in}}%
\pgfpathlineto{\pgfqpoint{3.041373in}{2.471874in}}%
\pgfpathlineto{\pgfqpoint{3.049206in}{2.477466in}}%
\pgfpathlineto{\pgfqpoint{3.057031in}{2.483130in}}%
\pgfpathlineto{\pgfqpoint{3.064848in}{2.488864in}}%
\pgfpathclose%
\pgfusepath{fill}%
\end{pgfscope}%
\begin{pgfscope}%
\pgfpathrectangle{\pgfqpoint{1.254980in}{0.150000in}}{\pgfqpoint{5.490039in}{5.490039in}}%
\pgfusepath{clip}%
\pgfsetbuttcap%
\pgfsetroundjoin%
\definecolor{currentfill}{rgb}{0.279566,0.067836,0.391917}%
\pgfsetfillcolor{currentfill}%
\pgfsetfillopacity{0.700000}%
\pgfsetlinewidth{0.000000pt}%
\definecolor{currentstroke}{rgb}{0.000000,0.000000,0.000000}%
\pgfsetstrokecolor{currentstroke}%
\pgfsetdash{}{0pt}%
\pgfpathmoveto{\pgfqpoint{4.927321in}{2.527161in}}%
\pgfpathlineto{\pgfqpoint{4.940379in}{2.524668in}}%
\pgfpathlineto{\pgfqpoint{4.953444in}{2.522200in}}%
\pgfpathlineto{\pgfqpoint{4.966516in}{2.519756in}}%
\pgfpathlineto{\pgfqpoint{4.979595in}{2.517337in}}%
\pgfpathlineto{\pgfqpoint{4.972507in}{2.510784in}}%
\pgfpathlineto{\pgfqpoint{4.965415in}{2.504248in}}%
\pgfpathlineto{\pgfqpoint{4.958318in}{2.497726in}}%
\pgfpathlineto{\pgfqpoint{4.951216in}{2.491213in}}%
\pgfpathlineto{\pgfqpoint{4.938122in}{2.493533in}}%
\pgfpathlineto{\pgfqpoint{4.925036in}{2.495877in}}%
\pgfpathlineto{\pgfqpoint{4.911956in}{2.498247in}}%
\pgfpathlineto{\pgfqpoint{4.898883in}{2.500641in}}%
\pgfpathlineto{\pgfqpoint{4.906000in}{2.507248in}}%
\pgfpathlineto{\pgfqpoint{4.913112in}{2.513868in}}%
\pgfpathlineto{\pgfqpoint{4.920219in}{2.520505in}}%
\pgfpathlineto{\pgfqpoint{4.927321in}{2.527161in}}%
\pgfpathclose%
\pgfusepath{fill}%
\end{pgfscope}%
\begin{pgfscope}%
\pgfpathrectangle{\pgfqpoint{1.254980in}{0.150000in}}{\pgfqpoint{5.490039in}{5.490039in}}%
\pgfusepath{clip}%
\pgfsetbuttcap%
\pgfsetroundjoin%
\definecolor{currentfill}{rgb}{0.269944,0.014625,0.341379}%
\pgfsetfillcolor{currentfill}%
\pgfsetfillopacity{0.700000}%
\pgfsetlinewidth{0.000000pt}%
\definecolor{currentstroke}{rgb}{0.000000,0.000000,0.000000}%
\pgfsetstrokecolor{currentstroke}%
\pgfsetdash{}{0pt}%
\pgfpathmoveto{\pgfqpoint{3.940740in}{2.438511in}}%
\pgfpathlineto{\pgfqpoint{3.953563in}{2.434935in}}%
\pgfpathlineto{\pgfqpoint{3.966392in}{2.431390in}}%
\pgfpathlineto{\pgfqpoint{3.979227in}{2.427874in}}%
\pgfpathlineto{\pgfqpoint{3.992067in}{2.424389in}}%
\pgfpathlineto{\pgfqpoint{3.984609in}{2.417117in}}%
\pgfpathlineto{\pgfqpoint{3.977145in}{2.409843in}}%
\pgfpathlineto{\pgfqpoint{3.969676in}{2.402566in}}%
\pgfpathlineto{\pgfqpoint{3.962202in}{2.395285in}}%
\pgfpathlineto{\pgfqpoint{3.949350in}{2.398786in}}%
\pgfpathlineto{\pgfqpoint{3.936504in}{2.402317in}}%
\pgfpathlineto{\pgfqpoint{3.923663in}{2.405877in}}%
\pgfpathlineto{\pgfqpoint{3.910828in}{2.409469in}}%
\pgfpathlineto{\pgfqpoint{3.918314in}{2.416729in}}%
\pgfpathlineto{\pgfqpoint{3.925795in}{2.423990in}}%
\pgfpathlineto{\pgfqpoint{3.933270in}{2.431250in}}%
\pgfpathlineto{\pgfqpoint{3.940740in}{2.438511in}}%
\pgfpathclose%
\pgfusepath{fill}%
\end{pgfscope}%
\begin{pgfscope}%
\pgfpathrectangle{\pgfqpoint{1.254980in}{0.150000in}}{\pgfqpoint{5.490039in}{5.490039in}}%
\pgfusepath{clip}%
\pgfsetbuttcap%
\pgfsetroundjoin%
\definecolor{currentfill}{rgb}{0.277941,0.056324,0.381191}%
\pgfsetfillcolor{currentfill}%
\pgfsetfillopacity{0.700000}%
\pgfsetlinewidth{0.000000pt}%
\definecolor{currentstroke}{rgb}{0.000000,0.000000,0.000000}%
\pgfsetstrokecolor{currentstroke}%
\pgfsetdash{}{0pt}%
\pgfpathmoveto{\pgfqpoint{4.713848in}{2.503922in}}%
\pgfpathlineto{\pgfqpoint{4.726855in}{2.501313in}}%
\pgfpathlineto{\pgfqpoint{4.739867in}{2.498729in}}%
\pgfpathlineto{\pgfqpoint{4.752887in}{2.496171in}}%
\pgfpathlineto{\pgfqpoint{4.765913in}{2.493638in}}%
\pgfpathlineto{\pgfqpoint{4.758742in}{2.486895in}}%
\pgfpathlineto{\pgfqpoint{4.751566in}{2.480153in}}%
\pgfpathlineto{\pgfqpoint{4.744385in}{2.473410in}}%
\pgfpathlineto{\pgfqpoint{4.737199in}{2.466665in}}%
\pgfpathlineto{\pgfqpoint{4.724159in}{2.469123in}}%
\pgfpathlineto{\pgfqpoint{4.711126in}{2.471608in}}%
\pgfpathlineto{\pgfqpoint{4.698099in}{2.474117in}}%
\pgfpathlineto{\pgfqpoint{4.685080in}{2.476652in}}%
\pgfpathlineto{\pgfqpoint{4.692280in}{2.483467in}}%
\pgfpathlineto{\pgfqpoint{4.699474in}{2.490282in}}%
\pgfpathlineto{\pgfqpoint{4.706664in}{2.497099in}}%
\pgfpathlineto{\pgfqpoint{4.713848in}{2.503922in}}%
\pgfpathclose%
\pgfusepath{fill}%
\end{pgfscope}%
\begin{pgfscope}%
\pgfpathrectangle{\pgfqpoint{1.254980in}{0.150000in}}{\pgfqpoint{5.490039in}{5.490039in}}%
\pgfusepath{clip}%
\pgfsetbuttcap%
\pgfsetroundjoin%
\definecolor{currentfill}{rgb}{0.268510,0.009605,0.335427}%
\pgfsetfillcolor{currentfill}%
\pgfsetfillopacity{0.700000}%
\pgfsetlinewidth{0.000000pt}%
\definecolor{currentstroke}{rgb}{0.000000,0.000000,0.000000}%
\pgfsetstrokecolor{currentstroke}%
\pgfsetdash{}{0pt}%
\pgfpathmoveto{\pgfqpoint{3.594674in}{2.431236in}}%
\pgfpathlineto{\pgfqpoint{3.607428in}{2.426932in}}%
\pgfpathlineto{\pgfqpoint{3.620188in}{2.422663in}}%
\pgfpathlineto{\pgfqpoint{3.632952in}{2.418427in}}%
\pgfpathlineto{\pgfqpoint{3.645721in}{2.414224in}}%
\pgfpathlineto{\pgfqpoint{3.638136in}{2.407170in}}%
\pgfpathlineto{\pgfqpoint{3.630544in}{2.400132in}}%
\pgfpathlineto{\pgfqpoint{3.622946in}{2.393112in}}%
\pgfpathlineto{\pgfqpoint{3.615343in}{2.386110in}}%
\pgfpathlineto{\pgfqpoint{3.602561in}{2.390366in}}%
\pgfpathlineto{\pgfqpoint{3.589784in}{2.394655in}}%
\pgfpathlineto{\pgfqpoint{3.577012in}{2.398978in}}%
\pgfpathlineto{\pgfqpoint{3.564245in}{2.403335in}}%
\pgfpathlineto{\pgfqpoint{3.571861in}{2.410279in}}%
\pgfpathlineto{\pgfqpoint{3.579471in}{2.417244in}}%
\pgfpathlineto{\pgfqpoint{3.587076in}{2.424230in}}%
\pgfpathlineto{\pgfqpoint{3.594674in}{2.431236in}}%
\pgfpathclose%
\pgfusepath{fill}%
\end{pgfscope}%
\begin{pgfscope}%
\pgfpathrectangle{\pgfqpoint{1.254980in}{0.150000in}}{\pgfqpoint{5.490039in}{5.490039in}}%
\pgfusepath{clip}%
\pgfsetbuttcap%
\pgfsetroundjoin%
\definecolor{currentfill}{rgb}{0.276022,0.044167,0.370164}%
\pgfsetfillcolor{currentfill}%
\pgfsetfillopacity{0.700000}%
\pgfsetlinewidth{0.000000pt}%
\definecolor{currentstroke}{rgb}{0.000000,0.000000,0.000000}%
\pgfsetstrokecolor{currentstroke}%
\pgfsetdash{}{0pt}%
\pgfpathmoveto{\pgfqpoint{4.500343in}{2.481052in}}%
\pgfpathlineto{\pgfqpoint{4.513298in}{2.478266in}}%
\pgfpathlineto{\pgfqpoint{4.526259in}{2.475506in}}%
\pgfpathlineto{\pgfqpoint{4.539226in}{2.472773in}}%
\pgfpathlineto{\pgfqpoint{4.552200in}{2.470066in}}%
\pgfpathlineto{\pgfqpoint{4.544947in}{2.463110in}}%
\pgfpathlineto{\pgfqpoint{4.537689in}{2.456148in}}%
\pgfpathlineto{\pgfqpoint{4.530426in}{2.449177in}}%
\pgfpathlineto{\pgfqpoint{4.523157in}{2.442195in}}%
\pgfpathlineto{\pgfqpoint{4.510170in}{2.444854in}}%
\pgfpathlineto{\pgfqpoint{4.497190in}{2.447539in}}%
\pgfpathlineto{\pgfqpoint{4.484216in}{2.450250in}}%
\pgfpathlineto{\pgfqpoint{4.471249in}{2.452988in}}%
\pgfpathlineto{\pgfqpoint{4.478530in}{2.460013in}}%
\pgfpathlineto{\pgfqpoint{4.485806in}{2.467031in}}%
\pgfpathlineto{\pgfqpoint{4.493077in}{2.474043in}}%
\pgfpathlineto{\pgfqpoint{4.500343in}{2.481052in}}%
\pgfpathclose%
\pgfusepath{fill}%
\end{pgfscope}%
\begin{pgfscope}%
\pgfpathrectangle{\pgfqpoint{1.254980in}{0.150000in}}{\pgfqpoint{5.490039in}{5.490039in}}%
\pgfusepath{clip}%
\pgfsetbuttcap%
\pgfsetroundjoin%
\definecolor{currentfill}{rgb}{0.278791,0.062145,0.386592}%
\pgfsetfillcolor{currentfill}%
\pgfsetfillopacity{0.700000}%
\pgfsetlinewidth{0.000000pt}%
\definecolor{currentstroke}{rgb}{0.000000,0.000000,0.000000}%
\pgfsetstrokecolor{currentstroke}%
\pgfsetdash{}{0pt}%
\pgfpathmoveto{\pgfqpoint{2.932136in}{2.515549in}}%
\pgfpathlineto{\pgfqpoint{2.944799in}{2.509236in}}%
\pgfpathlineto{\pgfqpoint{2.957465in}{2.502970in}}%
\pgfpathlineto{\pgfqpoint{2.970135in}{2.496752in}}%
\pgfpathlineto{\pgfqpoint{2.982808in}{2.490581in}}%
\pgfpathlineto{\pgfqpoint{2.974942in}{2.485255in}}%
\pgfpathlineto{\pgfqpoint{2.967068in}{2.480011in}}%
\pgfpathlineto{\pgfqpoint{2.959186in}{2.474854in}}%
\pgfpathlineto{\pgfqpoint{2.951296in}{2.469784in}}%
\pgfpathlineto{\pgfqpoint{2.938606in}{2.476073in}}%
\pgfpathlineto{\pgfqpoint{2.925919in}{2.482409in}}%
\pgfpathlineto{\pgfqpoint{2.913235in}{2.488792in}}%
\pgfpathlineto{\pgfqpoint{2.900554in}{2.495223in}}%
\pgfpathlineto{\pgfqpoint{2.908462in}{2.500169in}}%
\pgfpathlineto{\pgfqpoint{2.916362in}{2.505208in}}%
\pgfpathlineto{\pgfqpoint{2.924253in}{2.510335in}}%
\pgfpathlineto{\pgfqpoint{2.932136in}{2.515549in}}%
\pgfpathclose%
\pgfusepath{fill}%
\end{pgfscope}%
\begin{pgfscope}%
\pgfpathrectangle{\pgfqpoint{1.254980in}{0.150000in}}{\pgfqpoint{5.490039in}{5.490039in}}%
\pgfusepath{clip}%
\pgfsetbuttcap%
\pgfsetroundjoin%
\definecolor{currentfill}{rgb}{0.272594,0.025563,0.353093}%
\pgfsetfillcolor{currentfill}%
\pgfsetfillopacity{0.700000}%
\pgfsetlinewidth{0.000000pt}%
\definecolor{currentstroke}{rgb}{0.000000,0.000000,0.000000}%
\pgfsetstrokecolor{currentstroke}%
\pgfsetdash{}{0pt}%
\pgfpathmoveto{\pgfqpoint{4.286803in}{2.459302in}}%
\pgfpathlineto{\pgfqpoint{4.299708in}{2.456275in}}%
\pgfpathlineto{\pgfqpoint{4.312618in}{2.453276in}}%
\pgfpathlineto{\pgfqpoint{4.325535in}{2.450304in}}%
\pgfpathlineto{\pgfqpoint{4.338458in}{2.447359in}}%
\pgfpathlineto{\pgfqpoint{4.331125in}{2.440217in}}%
\pgfpathlineto{\pgfqpoint{4.323786in}{2.433065in}}%
\pgfpathlineto{\pgfqpoint{4.316442in}{2.425902in}}%
\pgfpathlineto{\pgfqpoint{4.309093in}{2.418726in}}%
\pgfpathlineto{\pgfqpoint{4.296158in}{2.421647in}}%
\pgfpathlineto{\pgfqpoint{4.283229in}{2.424596in}}%
\pgfpathlineto{\pgfqpoint{4.270306in}{2.427573in}}%
\pgfpathlineto{\pgfqpoint{4.257389in}{2.430577in}}%
\pgfpathlineto{\pgfqpoint{4.264751in}{2.437771in}}%
\pgfpathlineto{\pgfqpoint{4.272107in}{2.444956in}}%
\pgfpathlineto{\pgfqpoint{4.279458in}{2.452133in}}%
\pgfpathlineto{\pgfqpoint{4.286803in}{2.459302in}}%
\pgfpathclose%
\pgfusepath{fill}%
\end{pgfscope}%
\begin{pgfscope}%
\pgfpathrectangle{\pgfqpoint{1.254980in}{0.150000in}}{\pgfqpoint{5.490039in}{5.490039in}}%
\pgfusepath{clip}%
\pgfsetbuttcap%
\pgfsetroundjoin%
\definecolor{currentfill}{rgb}{0.282910,0.105393,0.426902}%
\pgfsetfillcolor{currentfill}%
\pgfsetfillopacity{0.700000}%
\pgfsetlinewidth{0.000000pt}%
\definecolor{currentstroke}{rgb}{0.000000,0.000000,0.000000}%
\pgfsetstrokecolor{currentstroke}%
\pgfsetdash{}{0pt}%
\pgfpathmoveto{\pgfqpoint{5.487228in}{2.580582in}}%
\pgfpathlineto{\pgfqpoint{5.500428in}{2.578169in}}%
\pgfpathlineto{\pgfqpoint{5.513634in}{2.575779in}}%
\pgfpathlineto{\pgfqpoint{5.526848in}{2.573412in}}%
\pgfpathlineto{\pgfqpoint{5.540069in}{2.571068in}}%
\pgfpathlineto{\pgfqpoint{5.533195in}{2.564689in}}%
\pgfpathlineto{\pgfqpoint{5.526318in}{2.558386in}}%
\pgfpathlineto{\pgfqpoint{5.519439in}{2.552155in}}%
\pgfpathlineto{\pgfqpoint{5.512556in}{2.545990in}}%
\pgfpathlineto{\pgfqpoint{5.499316in}{2.548172in}}%
\pgfpathlineto{\pgfqpoint{5.486084in}{2.550376in}}%
\pgfpathlineto{\pgfqpoint{5.472860in}{2.552603in}}%
\pgfpathlineto{\pgfqpoint{5.459642in}{2.554854in}}%
\pgfpathlineto{\pgfqpoint{5.466543in}{2.561177in}}%
\pgfpathlineto{\pgfqpoint{5.473441in}{2.567569in}}%
\pgfpathlineto{\pgfqpoint{5.480336in}{2.574036in}}%
\pgfpathlineto{\pgfqpoint{5.487228in}{2.580582in}}%
\pgfpathclose%
\pgfusepath{fill}%
\end{pgfscope}%
\begin{pgfscope}%
\pgfpathrectangle{\pgfqpoint{1.254980in}{0.150000in}}{\pgfqpoint{5.490039in}{5.490039in}}%
\pgfusepath{clip}%
\pgfsetbuttcap%
\pgfsetroundjoin%
\definecolor{currentfill}{rgb}{0.268510,0.009605,0.335427}%
\pgfsetfillcolor{currentfill}%
\pgfsetfillopacity{0.700000}%
\pgfsetlinewidth{0.000000pt}%
\definecolor{currentstroke}{rgb}{0.000000,0.000000,0.000000}%
\pgfsetstrokecolor{currentstroke}%
\pgfsetdash{}{0pt}%
\pgfpathmoveto{\pgfqpoint{3.727086in}{2.426283in}}%
\pgfpathlineto{\pgfqpoint{3.739868in}{2.422287in}}%
\pgfpathlineto{\pgfqpoint{3.752656in}{2.418323in}}%
\pgfpathlineto{\pgfqpoint{3.765449in}{2.414392in}}%
\pgfpathlineto{\pgfqpoint{3.778247in}{2.410492in}}%
\pgfpathlineto{\pgfqpoint{3.770709in}{2.403306in}}%
\pgfpathlineto{\pgfqpoint{3.763165in}{2.396127in}}%
\pgfpathlineto{\pgfqpoint{3.755616in}{2.388956in}}%
\pgfpathlineto{\pgfqpoint{3.748061in}{2.381795in}}%
\pgfpathlineto{\pgfqpoint{3.735250in}{2.385735in}}%
\pgfpathlineto{\pgfqpoint{3.722445in}{2.389707in}}%
\pgfpathlineto{\pgfqpoint{3.709644in}{2.393711in}}%
\pgfpathlineto{\pgfqpoint{3.696850in}{2.397748in}}%
\pgfpathlineto{\pgfqpoint{3.704417in}{2.404864in}}%
\pgfpathlineto{\pgfqpoint{3.711979in}{2.411993in}}%
\pgfpathlineto{\pgfqpoint{3.719535in}{2.419132in}}%
\pgfpathlineto{\pgfqpoint{3.727086in}{2.426283in}}%
\pgfpathclose%
\pgfusepath{fill}%
\end{pgfscope}%
\begin{pgfscope}%
\pgfpathrectangle{\pgfqpoint{1.254980in}{0.150000in}}{\pgfqpoint{5.490039in}{5.490039in}}%
\pgfusepath{clip}%
\pgfsetbuttcap%
\pgfsetroundjoin%
\definecolor{currentfill}{rgb}{0.281924,0.089666,0.412415}%
\pgfsetfillcolor{currentfill}%
\pgfsetfillopacity{0.700000}%
\pgfsetlinewidth{0.000000pt}%
\definecolor{currentstroke}{rgb}{0.000000,0.000000,0.000000}%
\pgfsetstrokecolor{currentstroke}%
\pgfsetdash{}{0pt}%
\pgfpathmoveto{\pgfqpoint{5.273740in}{2.557198in}}%
\pgfpathlineto{\pgfqpoint{5.286889in}{2.554815in}}%
\pgfpathlineto{\pgfqpoint{5.300046in}{2.552456in}}%
\pgfpathlineto{\pgfqpoint{5.313209in}{2.550121in}}%
\pgfpathlineto{\pgfqpoint{5.326380in}{2.547809in}}%
\pgfpathlineto{\pgfqpoint{5.319426in}{2.541477in}}%
\pgfpathlineto{\pgfqpoint{5.312468in}{2.535192in}}%
\pgfpathlineto{\pgfqpoint{5.305505in}{2.528951in}}%
\pgfpathlineto{\pgfqpoint{5.298539in}{2.522749in}}%
\pgfpathlineto{\pgfqpoint{5.285352in}{2.524923in}}%
\pgfpathlineto{\pgfqpoint{5.272171in}{2.527121in}}%
\pgfpathlineto{\pgfqpoint{5.258998in}{2.529343in}}%
\pgfpathlineto{\pgfqpoint{5.245832in}{2.531588in}}%
\pgfpathlineto{\pgfqpoint{5.252815in}{2.537923in}}%
\pgfpathlineto{\pgfqpoint{5.259794in}{2.544301in}}%
\pgfpathlineto{\pgfqpoint{5.266769in}{2.550724in}}%
\pgfpathlineto{\pgfqpoint{5.273740in}{2.557198in}}%
\pgfpathclose%
\pgfusepath{fill}%
\end{pgfscope}%
\begin{pgfscope}%
\pgfpathrectangle{\pgfqpoint{1.254980in}{0.150000in}}{\pgfqpoint{5.490039in}{5.490039in}}%
\pgfusepath{clip}%
\pgfsetbuttcap%
\pgfsetroundjoin%
\definecolor{currentfill}{rgb}{0.271305,0.019942,0.347269}%
\pgfsetfillcolor{currentfill}%
\pgfsetfillopacity{0.700000}%
\pgfsetlinewidth{0.000000pt}%
\definecolor{currentstroke}{rgb}{0.000000,0.000000,0.000000}%
\pgfsetstrokecolor{currentstroke}%
\pgfsetdash{}{0pt}%
\pgfpathmoveto{\pgfqpoint{4.073215in}{2.439817in}}%
\pgfpathlineto{\pgfqpoint{4.086072in}{2.436482in}}%
\pgfpathlineto{\pgfqpoint{4.098934in}{2.433176in}}%
\pgfpathlineto{\pgfqpoint{4.111803in}{2.429899in}}%
\pgfpathlineto{\pgfqpoint{4.124677in}{2.426650in}}%
\pgfpathlineto{\pgfqpoint{4.117265in}{2.419392in}}%
\pgfpathlineto{\pgfqpoint{4.109847in}{2.412126in}}%
\pgfpathlineto{\pgfqpoint{4.102424in}{2.404851in}}%
\pgfpathlineto{\pgfqpoint{4.094995in}{2.397568in}}%
\pgfpathlineto{\pgfqpoint{4.082109in}{2.400819in}}%
\pgfpathlineto{\pgfqpoint{4.069228in}{2.404098in}}%
\pgfpathlineto{\pgfqpoint{4.056353in}{2.407406in}}%
\pgfpathlineto{\pgfqpoint{4.043484in}{2.410744in}}%
\pgfpathlineto{\pgfqpoint{4.050925in}{2.418020in}}%
\pgfpathlineto{\pgfqpoint{4.058360in}{2.425291in}}%
\pgfpathlineto{\pgfqpoint{4.065790in}{2.432556in}}%
\pgfpathlineto{\pgfqpoint{4.073215in}{2.439817in}}%
\pgfpathclose%
\pgfusepath{fill}%
\end{pgfscope}%
\begin{pgfscope}%
\pgfpathrectangle{\pgfqpoint{1.254980in}{0.150000in}}{\pgfqpoint{5.490039in}{5.490039in}}%
\pgfusepath{clip}%
\pgfsetbuttcap%
\pgfsetroundjoin%
\definecolor{currentfill}{rgb}{0.280894,0.078907,0.402329}%
\pgfsetfillcolor{currentfill}%
\pgfsetfillopacity{0.700000}%
\pgfsetlinewidth{0.000000pt}%
\definecolor{currentstroke}{rgb}{0.000000,0.000000,0.000000}%
\pgfsetstrokecolor{currentstroke}%
\pgfsetdash{}{0pt}%
\pgfpathmoveto{\pgfqpoint{5.060219in}{2.533901in}}%
\pgfpathlineto{\pgfqpoint{5.073317in}{2.531492in}}%
\pgfpathlineto{\pgfqpoint{5.086422in}{2.529108in}}%
\pgfpathlineto{\pgfqpoint{5.099533in}{2.526747in}}%
\pgfpathlineto{\pgfqpoint{5.112652in}{2.524411in}}%
\pgfpathlineto{\pgfqpoint{5.105615in}{2.517991in}}%
\pgfpathlineto{\pgfqpoint{5.098572in}{2.511595in}}%
\pgfpathlineto{\pgfqpoint{5.091525in}{2.505220in}}%
\pgfpathlineto{\pgfqpoint{5.084473in}{2.498862in}}%
\pgfpathlineto{\pgfqpoint{5.071339in}{2.501086in}}%
\pgfpathlineto{\pgfqpoint{5.058211in}{2.503335in}}%
\pgfpathlineto{\pgfqpoint{5.045091in}{2.505607in}}%
\pgfpathlineto{\pgfqpoint{5.031978in}{2.507904in}}%
\pgfpathlineto{\pgfqpoint{5.039045in}{2.514370in}}%
\pgfpathlineto{\pgfqpoint{5.046108in}{2.520855in}}%
\pgfpathlineto{\pgfqpoint{5.053166in}{2.527365in}}%
\pgfpathlineto{\pgfqpoint{5.060219in}{2.533901in}}%
\pgfpathclose%
\pgfusepath{fill}%
\end{pgfscope}%
\begin{pgfscope}%
\pgfpathrectangle{\pgfqpoint{1.254980in}{0.150000in}}{\pgfqpoint{5.490039in}{5.490039in}}%
\pgfusepath{clip}%
\pgfsetbuttcap%
\pgfsetroundjoin%
\definecolor{currentfill}{rgb}{0.279566,0.067836,0.391917}%
\pgfsetfillcolor{currentfill}%
\pgfsetfillopacity{0.700000}%
\pgfsetlinewidth{0.000000pt}%
\definecolor{currentstroke}{rgb}{0.000000,0.000000,0.000000}%
\pgfsetstrokecolor{currentstroke}%
\pgfsetdash{}{0pt}%
\pgfpathmoveto{\pgfqpoint{4.846660in}{2.510465in}}%
\pgfpathlineto{\pgfqpoint{4.859706in}{2.507971in}}%
\pgfpathlineto{\pgfqpoint{4.872758in}{2.505503in}}%
\pgfpathlineto{\pgfqpoint{4.885817in}{2.503059in}}%
\pgfpathlineto{\pgfqpoint{4.898883in}{2.500641in}}%
\pgfpathlineto{\pgfqpoint{4.891761in}{2.494043in}}%
\pgfpathlineto{\pgfqpoint{4.884634in}{2.487453in}}%
\pgfpathlineto{\pgfqpoint{4.877502in}{2.480867in}}%
\pgfpathlineto{\pgfqpoint{4.870364in}{2.474283in}}%
\pgfpathlineto{\pgfqpoint{4.857284in}{2.476615in}}%
\pgfpathlineto{\pgfqpoint{4.844211in}{2.478971in}}%
\pgfpathlineto{\pgfqpoint{4.831144in}{2.481353in}}%
\pgfpathlineto{\pgfqpoint{4.818084in}{2.483760in}}%
\pgfpathlineto{\pgfqpoint{4.825236in}{2.490427in}}%
\pgfpathlineto{\pgfqpoint{4.832383in}{2.497098in}}%
\pgfpathlineto{\pgfqpoint{4.839524in}{2.503776in}}%
\pgfpathlineto{\pgfqpoint{4.846660in}{2.510465in}}%
\pgfpathclose%
\pgfusepath{fill}%
\end{pgfscope}%
\begin{pgfscope}%
\pgfpathrectangle{\pgfqpoint{1.254980in}{0.150000in}}{\pgfqpoint{5.490039in}{5.490039in}}%
\pgfusepath{clip}%
\pgfsetbuttcap%
\pgfsetroundjoin%
\definecolor{currentfill}{rgb}{0.277018,0.050344,0.375715}%
\pgfsetfillcolor{currentfill}%
\pgfsetfillopacity{0.700000}%
\pgfsetlinewidth{0.000000pt}%
\definecolor{currentstroke}{rgb}{0.000000,0.000000,0.000000}%
\pgfsetstrokecolor{currentstroke}%
\pgfsetdash{}{0pt}%
\pgfpathmoveto{\pgfqpoint{4.633067in}{2.487049in}}%
\pgfpathlineto{\pgfqpoint{4.646060in}{2.484411in}}%
\pgfpathlineto{\pgfqpoint{4.659060in}{2.481799in}}%
\pgfpathlineto{\pgfqpoint{4.672067in}{2.479213in}}%
\pgfpathlineto{\pgfqpoint{4.685080in}{2.476652in}}%
\pgfpathlineto{\pgfqpoint{4.677874in}{2.469835in}}%
\pgfpathlineto{\pgfqpoint{4.670664in}{2.463013in}}%
\pgfpathlineto{\pgfqpoint{4.663448in}{2.456185in}}%
\pgfpathlineto{\pgfqpoint{4.656226in}{2.449347in}}%
\pgfpathlineto{\pgfqpoint{4.643200in}{2.451847in}}%
\pgfpathlineto{\pgfqpoint{4.630180in}{2.454372in}}%
\pgfpathlineto{\pgfqpoint{4.617167in}{2.456923in}}%
\pgfpathlineto{\pgfqpoint{4.604161in}{2.459499in}}%
\pgfpathlineto{\pgfqpoint{4.611395in}{2.466393in}}%
\pgfpathlineto{\pgfqpoint{4.618624in}{2.473282in}}%
\pgfpathlineto{\pgfqpoint{4.625848in}{2.480166in}}%
\pgfpathlineto{\pgfqpoint{4.633067in}{2.487049in}}%
\pgfpathclose%
\pgfusepath{fill}%
\end{pgfscope}%
\begin{pgfscope}%
\pgfpathrectangle{\pgfqpoint{1.254980in}{0.150000in}}{\pgfqpoint{5.490039in}{5.490039in}}%
\pgfusepath{clip}%
\pgfsetbuttcap%
\pgfsetroundjoin%
\definecolor{currentfill}{rgb}{0.268510,0.009605,0.335427}%
\pgfsetfillcolor{currentfill}%
\pgfsetfillopacity{0.700000}%
\pgfsetlinewidth{0.000000pt}%
\definecolor{currentstroke}{rgb}{0.000000,0.000000,0.000000}%
\pgfsetstrokecolor{currentstroke}%
\pgfsetdash{}{0pt}%
\pgfpathmoveto{\pgfqpoint{3.859542in}{2.424139in}}%
\pgfpathlineto{\pgfqpoint{3.872355in}{2.420425in}}%
\pgfpathlineto{\pgfqpoint{3.885174in}{2.416742in}}%
\pgfpathlineto{\pgfqpoint{3.897998in}{2.413090in}}%
\pgfpathlineto{\pgfqpoint{3.910828in}{2.409469in}}%
\pgfpathlineto{\pgfqpoint{3.903336in}{2.402208in}}%
\pgfpathlineto{\pgfqpoint{3.895839in}{2.394948in}}%
\pgfpathlineto{\pgfqpoint{3.888336in}{2.387688in}}%
\pgfpathlineto{\pgfqpoint{3.880828in}{2.380429in}}%
\pgfpathlineto{\pgfqpoint{3.867986in}{2.384078in}}%
\pgfpathlineto{\pgfqpoint{3.855150in}{2.387758in}}%
\pgfpathlineto{\pgfqpoint{3.842319in}{2.391469in}}%
\pgfpathlineto{\pgfqpoint{3.829494in}{2.395211in}}%
\pgfpathlineto{\pgfqpoint{3.837014in}{2.402438in}}%
\pgfpathlineto{\pgfqpoint{3.844529in}{2.409668in}}%
\pgfpathlineto{\pgfqpoint{3.852038in}{2.416902in}}%
\pgfpathlineto{\pgfqpoint{3.859542in}{2.424139in}}%
\pgfpathclose%
\pgfusepath{fill}%
\end{pgfscope}%
\begin{pgfscope}%
\pgfpathrectangle{\pgfqpoint{1.254980in}{0.150000in}}{\pgfqpoint{5.490039in}{5.490039in}}%
\pgfusepath{clip}%
\pgfsetbuttcap%
\pgfsetroundjoin%
\definecolor{currentfill}{rgb}{0.272594,0.025563,0.353093}%
\pgfsetfillcolor{currentfill}%
\pgfsetfillopacity{0.700000}%
\pgfsetlinewidth{0.000000pt}%
\definecolor{currentstroke}{rgb}{0.000000,0.000000,0.000000}%
\pgfsetstrokecolor{currentstroke}%
\pgfsetdash{}{0pt}%
\pgfpathmoveto{\pgfqpoint{3.248186in}{2.446335in}}%
\pgfpathlineto{\pgfqpoint{3.260890in}{2.441094in}}%
\pgfpathlineto{\pgfqpoint{3.273598in}{2.435893in}}%
\pgfpathlineto{\pgfqpoint{3.286311in}{2.430731in}}%
\pgfpathlineto{\pgfqpoint{3.299027in}{2.425608in}}%
\pgfpathlineto{\pgfqpoint{3.291300in}{2.419267in}}%
\pgfpathlineto{\pgfqpoint{3.283565in}{2.412974in}}%
\pgfpathlineto{\pgfqpoint{3.275824in}{2.406733in}}%
\pgfpathlineto{\pgfqpoint{3.268076in}{2.400545in}}%
\pgfpathlineto{\pgfqpoint{3.255344in}{2.405759in}}%
\pgfpathlineto{\pgfqpoint{3.242617in}{2.411013in}}%
\pgfpathlineto{\pgfqpoint{3.229894in}{2.416306in}}%
\pgfpathlineto{\pgfqpoint{3.217174in}{2.421638in}}%
\pgfpathlineto{\pgfqpoint{3.224938in}{2.427730in}}%
\pgfpathlineto{\pgfqpoint{3.232694in}{2.433878in}}%
\pgfpathlineto{\pgfqpoint{3.240443in}{2.440080in}}%
\pgfpathlineto{\pgfqpoint{3.248186in}{2.446335in}}%
\pgfpathclose%
\pgfusepath{fill}%
\end{pgfscope}%
\begin{pgfscope}%
\pgfpathrectangle{\pgfqpoint{1.254980in}{0.150000in}}{\pgfqpoint{5.490039in}{5.490039in}}%
\pgfusepath{clip}%
\pgfsetbuttcap%
\pgfsetroundjoin%
\definecolor{currentfill}{rgb}{0.269944,0.014625,0.341379}%
\pgfsetfillcolor{currentfill}%
\pgfsetfillopacity{0.700000}%
\pgfsetlinewidth{0.000000pt}%
\definecolor{currentstroke}{rgb}{0.000000,0.000000,0.000000}%
\pgfsetstrokecolor{currentstroke}%
\pgfsetdash{}{0pt}%
\pgfpathmoveto{\pgfqpoint{3.380725in}{2.431639in}}%
\pgfpathlineto{\pgfqpoint{3.393449in}{2.426786in}}%
\pgfpathlineto{\pgfqpoint{3.406178in}{2.421970in}}%
\pgfpathlineto{\pgfqpoint{3.418911in}{2.417191in}}%
\pgfpathlineto{\pgfqpoint{3.431649in}{2.412449in}}%
\pgfpathlineto{\pgfqpoint{3.423976in}{2.405781in}}%
\pgfpathlineto{\pgfqpoint{3.416296in}{2.399148in}}%
\pgfpathlineto{\pgfqpoint{3.408610in}{2.392553in}}%
\pgfpathlineto{\pgfqpoint{3.400918in}{2.385998in}}%
\pgfpathlineto{\pgfqpoint{3.388166in}{2.390818in}}%
\pgfpathlineto{\pgfqpoint{3.375418in}{2.395676in}}%
\pgfpathlineto{\pgfqpoint{3.362676in}{2.400570in}}%
\pgfpathlineto{\pgfqpoint{3.349937in}{2.405502in}}%
\pgfpathlineto{\pgfqpoint{3.357644in}{2.411974in}}%
\pgfpathlineto{\pgfqpoint{3.365344in}{2.418489in}}%
\pgfpathlineto{\pgfqpoint{3.373038in}{2.425044in}}%
\pgfpathlineto{\pgfqpoint{3.380725in}{2.431639in}}%
\pgfpathclose%
\pgfusepath{fill}%
\end{pgfscope}%
\begin{pgfscope}%
\pgfpathrectangle{\pgfqpoint{1.254980in}{0.150000in}}{\pgfqpoint{5.490039in}{5.490039in}}%
\pgfusepath{clip}%
\pgfsetbuttcap%
\pgfsetroundjoin%
\definecolor{currentfill}{rgb}{0.274952,0.037752,0.364543}%
\pgfsetfillcolor{currentfill}%
\pgfsetfillopacity{0.700000}%
\pgfsetlinewidth{0.000000pt}%
\definecolor{currentstroke}{rgb}{0.000000,0.000000,0.000000}%
\pgfsetstrokecolor{currentstroke}%
\pgfsetdash{}{0pt}%
\pgfpathmoveto{\pgfqpoint{4.419443in}{2.464205in}}%
\pgfpathlineto{\pgfqpoint{4.432385in}{2.461360in}}%
\pgfpathlineto{\pgfqpoint{4.445333in}{2.458543in}}%
\pgfpathlineto{\pgfqpoint{4.458288in}{2.455752in}}%
\pgfpathlineto{\pgfqpoint{4.471249in}{2.452988in}}%
\pgfpathlineto{\pgfqpoint{4.463962in}{2.445953in}}%
\pgfpathlineto{\pgfqpoint{4.456669in}{2.438908in}}%
\pgfpathlineto{\pgfqpoint{4.449371in}{2.431850in}}%
\pgfpathlineto{\pgfqpoint{4.442068in}{2.424779in}}%
\pgfpathlineto{\pgfqpoint{4.429095in}{2.427508in}}%
\pgfpathlineto{\pgfqpoint{4.416128in}{2.430263in}}%
\pgfpathlineto{\pgfqpoint{4.403167in}{2.433045in}}%
\pgfpathlineto{\pgfqpoint{4.390213in}{2.435854in}}%
\pgfpathlineto{\pgfqpoint{4.397528in}{2.442956in}}%
\pgfpathlineto{\pgfqpoint{4.404839in}{2.450047in}}%
\pgfpathlineto{\pgfqpoint{4.412143in}{2.457130in}}%
\pgfpathlineto{\pgfqpoint{4.419443in}{2.464205in}}%
\pgfpathclose%
\pgfusepath{fill}%
\end{pgfscope}%
\begin{pgfscope}%
\pgfpathrectangle{\pgfqpoint{1.254980in}{0.150000in}}{\pgfqpoint{5.490039in}{5.490039in}}%
\pgfusepath{clip}%
\pgfsetbuttcap%
\pgfsetroundjoin%
\definecolor{currentfill}{rgb}{0.274952,0.037752,0.364543}%
\pgfsetfillcolor{currentfill}%
\pgfsetfillopacity{0.700000}%
\pgfsetlinewidth{0.000000pt}%
\definecolor{currentstroke}{rgb}{0.000000,0.000000,0.000000}%
\pgfsetstrokecolor{currentstroke}%
\pgfsetdash{}{0pt}%
\pgfpathmoveto{\pgfqpoint{3.115563in}{2.465774in}}%
\pgfpathlineto{\pgfqpoint{3.128251in}{2.460111in}}%
\pgfpathlineto{\pgfqpoint{3.140943in}{2.454489in}}%
\pgfpathlineto{\pgfqpoint{3.153638in}{2.448911in}}%
\pgfpathlineto{\pgfqpoint{3.166338in}{2.443374in}}%
\pgfpathlineto{\pgfqpoint{3.158552in}{2.437440in}}%
\pgfpathlineto{\pgfqpoint{3.150759in}{2.431570in}}%
\pgfpathlineto{\pgfqpoint{3.142958in}{2.425767in}}%
\pgfpathlineto{\pgfqpoint{3.135150in}{2.420031in}}%
\pgfpathlineto{\pgfqpoint{3.122435in}{2.425673in}}%
\pgfpathlineto{\pgfqpoint{3.109724in}{2.431356in}}%
\pgfpathlineto{\pgfqpoint{3.097016in}{2.437081in}}%
\pgfpathlineto{\pgfqpoint{3.084312in}{2.442850in}}%
\pgfpathlineto{\pgfqpoint{3.092136in}{2.448476in}}%
\pgfpathlineto{\pgfqpoint{3.099953in}{2.454173in}}%
\pgfpathlineto{\pgfqpoint{3.107762in}{2.459940in}}%
\pgfpathlineto{\pgfqpoint{3.115563in}{2.465774in}}%
\pgfpathclose%
\pgfusepath{fill}%
\end{pgfscope}%
\begin{pgfscope}%
\pgfpathrectangle{\pgfqpoint{1.254980in}{0.150000in}}{\pgfqpoint{5.490039in}{5.490039in}}%
\pgfusepath{clip}%
\pgfsetbuttcap%
\pgfsetroundjoin%
\definecolor{currentfill}{rgb}{0.268510,0.009605,0.335427}%
\pgfsetfillcolor{currentfill}%
\pgfsetfillopacity{0.700000}%
\pgfsetlinewidth{0.000000pt}%
\definecolor{currentstroke}{rgb}{0.000000,0.000000,0.000000}%
\pgfsetstrokecolor{currentstroke}%
\pgfsetdash{}{0pt}%
\pgfpathmoveto{\pgfqpoint{3.513224in}{2.421111in}}%
\pgfpathlineto{\pgfqpoint{3.525972in}{2.416615in}}%
\pgfpathlineto{\pgfqpoint{3.538725in}{2.412154in}}%
\pgfpathlineto{\pgfqpoint{3.551482in}{2.407727in}}%
\pgfpathlineto{\pgfqpoint{3.564245in}{2.403335in}}%
\pgfpathlineto{\pgfqpoint{3.556622in}{2.396415in}}%
\pgfpathlineto{\pgfqpoint{3.548994in}{2.389519in}}%
\pgfpathlineto{\pgfqpoint{3.541360in}{2.382648in}}%
\pgfpathlineto{\pgfqpoint{3.533720in}{2.375803in}}%
\pgfpathlineto{\pgfqpoint{3.520945in}{2.380261in}}%
\pgfpathlineto{\pgfqpoint{3.508174in}{2.384753in}}%
\pgfpathlineto{\pgfqpoint{3.495408in}{2.389281in}}%
\pgfpathlineto{\pgfqpoint{3.482647in}{2.393843in}}%
\pgfpathlineto{\pgfqpoint{3.490300in}{2.400617in}}%
\pgfpathlineto{\pgfqpoint{3.497948in}{2.407420in}}%
\pgfpathlineto{\pgfqpoint{3.505589in}{2.414252in}}%
\pgfpathlineto{\pgfqpoint{3.513224in}{2.421111in}}%
\pgfpathclose%
\pgfusepath{fill}%
\end{pgfscope}%
\begin{pgfscope}%
\pgfpathrectangle{\pgfqpoint{1.254980in}{0.150000in}}{\pgfqpoint{5.490039in}{5.490039in}}%
\pgfusepath{clip}%
\pgfsetbuttcap%
\pgfsetroundjoin%
\definecolor{currentfill}{rgb}{0.272594,0.025563,0.353093}%
\pgfsetfillcolor{currentfill}%
\pgfsetfillopacity{0.700000}%
\pgfsetlinewidth{0.000000pt}%
\definecolor{currentstroke}{rgb}{0.000000,0.000000,0.000000}%
\pgfsetstrokecolor{currentstroke}%
\pgfsetdash{}{0pt}%
\pgfpathmoveto{\pgfqpoint{4.205783in}{2.442873in}}%
\pgfpathlineto{\pgfqpoint{4.218676in}{2.439757in}}%
\pgfpathlineto{\pgfqpoint{4.231574in}{2.436669in}}%
\pgfpathlineto{\pgfqpoint{4.244478in}{2.433609in}}%
\pgfpathlineto{\pgfqpoint{4.257389in}{2.430577in}}%
\pgfpathlineto{\pgfqpoint{4.250022in}{2.423373in}}%
\pgfpathlineto{\pgfqpoint{4.242650in}{2.416157in}}%
\pgfpathlineto{\pgfqpoint{4.235272in}{2.408930in}}%
\pgfpathlineto{\pgfqpoint{4.227889in}{2.401690in}}%
\pgfpathlineto{\pgfqpoint{4.214966in}{2.404711in}}%
\pgfpathlineto{\pgfqpoint{4.202050in}{2.407761in}}%
\pgfpathlineto{\pgfqpoint{4.189139in}{2.410838in}}%
\pgfpathlineto{\pgfqpoint{4.176235in}{2.413944in}}%
\pgfpathlineto{\pgfqpoint{4.183630in}{2.421190in}}%
\pgfpathlineto{\pgfqpoint{4.191020in}{2.428426in}}%
\pgfpathlineto{\pgfqpoint{4.198404in}{2.435653in}}%
\pgfpathlineto{\pgfqpoint{4.205783in}{2.442873in}}%
\pgfpathclose%
\pgfusepath{fill}%
\end{pgfscope}%
\begin{pgfscope}%
\pgfpathrectangle{\pgfqpoint{1.254980in}{0.150000in}}{\pgfqpoint{5.490039in}{5.490039in}}%
\pgfusepath{clip}%
\pgfsetbuttcap%
\pgfsetroundjoin%
\definecolor{currentfill}{rgb}{0.282656,0.100196,0.422160}%
\pgfsetfillcolor{currentfill}%
\pgfsetfillopacity{0.700000}%
\pgfsetlinewidth{0.000000pt}%
\definecolor{currentstroke}{rgb}{0.000000,0.000000,0.000000}%
\pgfsetstrokecolor{currentstroke}%
\pgfsetdash{}{0pt}%
\pgfpathmoveto{\pgfqpoint{5.406844in}{2.564091in}}%
\pgfpathlineto{\pgfqpoint{5.420033in}{2.561747in}}%
\pgfpathlineto{\pgfqpoint{5.433229in}{2.559426in}}%
\pgfpathlineto{\pgfqpoint{5.446432in}{2.557128in}}%
\pgfpathlineto{\pgfqpoint{5.459642in}{2.554854in}}%
\pgfpathlineto{\pgfqpoint{5.452738in}{2.548596in}}%
\pgfpathlineto{\pgfqpoint{5.445830in}{2.542397in}}%
\pgfpathlineto{\pgfqpoint{5.438918in}{2.536254in}}%
\pgfpathlineto{\pgfqpoint{5.432002in}{2.530161in}}%
\pgfpathlineto{\pgfqpoint{5.418774in}{2.532285in}}%
\pgfpathlineto{\pgfqpoint{5.405554in}{2.534432in}}%
\pgfpathlineto{\pgfqpoint{5.392340in}{2.536603in}}%
\pgfpathlineto{\pgfqpoint{5.379134in}{2.538797in}}%
\pgfpathlineto{\pgfqpoint{5.386067in}{2.545036in}}%
\pgfpathlineto{\pgfqpoint{5.392996in}{2.551328in}}%
\pgfpathlineto{\pgfqpoint{5.399922in}{2.557678in}}%
\pgfpathlineto{\pgfqpoint{5.406844in}{2.564091in}}%
\pgfpathclose%
\pgfusepath{fill}%
\end{pgfscope}%
\begin{pgfscope}%
\pgfpathrectangle{\pgfqpoint{1.254980in}{0.150000in}}{\pgfqpoint{5.490039in}{5.490039in}}%
\pgfusepath{clip}%
\pgfsetbuttcap%
\pgfsetroundjoin%
\definecolor{currentfill}{rgb}{0.277941,0.056324,0.381191}%
\pgfsetfillcolor{currentfill}%
\pgfsetfillopacity{0.700000}%
\pgfsetlinewidth{0.000000pt}%
\definecolor{currentstroke}{rgb}{0.000000,0.000000,0.000000}%
\pgfsetstrokecolor{currentstroke}%
\pgfsetdash{}{0pt}%
\pgfpathmoveto{\pgfqpoint{2.982808in}{2.490581in}}%
\pgfpathlineto{\pgfqpoint{2.995484in}{2.484457in}}%
\pgfpathlineto{\pgfqpoint{3.008163in}{2.478378in}}%
\pgfpathlineto{\pgfqpoint{3.020846in}{2.472345in}}%
\pgfpathlineto{\pgfqpoint{3.033532in}{2.466357in}}%
\pgfpathlineto{\pgfqpoint{3.025683in}{2.460918in}}%
\pgfpathlineto{\pgfqpoint{3.017827in}{2.455559in}}%
\pgfpathlineto{\pgfqpoint{3.009962in}{2.450282in}}%
\pgfpathlineto{\pgfqpoint{3.002089in}{2.445091in}}%
\pgfpathlineto{\pgfqpoint{2.989386in}{2.451196in}}%
\pgfpathlineto{\pgfqpoint{2.976686in}{2.457346in}}%
\pgfpathlineto{\pgfqpoint{2.963989in}{2.463542in}}%
\pgfpathlineto{\pgfqpoint{2.951296in}{2.469784in}}%
\pgfpathlineto{\pgfqpoint{2.959186in}{2.474854in}}%
\pgfpathlineto{\pgfqpoint{2.967068in}{2.480011in}}%
\pgfpathlineto{\pgfqpoint{2.974942in}{2.485255in}}%
\pgfpathlineto{\pgfqpoint{2.982808in}{2.490581in}}%
\pgfpathclose%
\pgfusepath{fill}%
\end{pgfscope}%
\begin{pgfscope}%
\pgfpathrectangle{\pgfqpoint{1.254980in}{0.150000in}}{\pgfqpoint{5.490039in}{5.490039in}}%
\pgfusepath{clip}%
\pgfsetbuttcap%
\pgfsetroundjoin%
\definecolor{currentfill}{rgb}{0.281924,0.089666,0.412415}%
\pgfsetfillcolor{currentfill}%
\pgfsetfillopacity{0.700000}%
\pgfsetlinewidth{0.000000pt}%
\definecolor{currentstroke}{rgb}{0.000000,0.000000,0.000000}%
\pgfsetstrokecolor{currentstroke}%
\pgfsetdash{}{0pt}%
\pgfpathmoveto{\pgfqpoint{5.193239in}{2.540809in}}%
\pgfpathlineto{\pgfqpoint{5.206377in}{2.538468in}}%
\pgfpathlineto{\pgfqpoint{5.219521in}{2.536151in}}%
\pgfpathlineto{\pgfqpoint{5.232673in}{2.533858in}}%
\pgfpathlineto{\pgfqpoint{5.245832in}{2.531588in}}%
\pgfpathlineto{\pgfqpoint{5.238845in}{2.525291in}}%
\pgfpathlineto{\pgfqpoint{5.231853in}{2.519028in}}%
\pgfpathlineto{\pgfqpoint{5.224856in}{2.512794in}}%
\pgfpathlineto{\pgfqpoint{5.217855in}{2.506586in}}%
\pgfpathlineto{\pgfqpoint{5.204680in}{2.508730in}}%
\pgfpathlineto{\pgfqpoint{5.191512in}{2.510898in}}%
\pgfpathlineto{\pgfqpoint{5.178351in}{2.513090in}}%
\pgfpathlineto{\pgfqpoint{5.165197in}{2.515306in}}%
\pgfpathlineto{\pgfqpoint{5.172214in}{2.521635in}}%
\pgfpathlineto{\pgfqpoint{5.179227in}{2.527992in}}%
\pgfpathlineto{\pgfqpoint{5.186235in}{2.534382in}}%
\pgfpathlineto{\pgfqpoint{5.193239in}{2.540809in}}%
\pgfpathclose%
\pgfusepath{fill}%
\end{pgfscope}%
\begin{pgfscope}%
\pgfpathrectangle{\pgfqpoint{1.254980in}{0.150000in}}{\pgfqpoint{5.490039in}{5.490039in}}%
\pgfusepath{clip}%
\pgfsetbuttcap%
\pgfsetroundjoin%
\definecolor{currentfill}{rgb}{0.269944,0.014625,0.341379}%
\pgfsetfillcolor{currentfill}%
\pgfsetfillopacity{0.700000}%
\pgfsetlinewidth{0.000000pt}%
\definecolor{currentstroke}{rgb}{0.000000,0.000000,0.000000}%
\pgfsetstrokecolor{currentstroke}%
\pgfsetdash{}{0pt}%
\pgfpathmoveto{\pgfqpoint{3.992067in}{2.424389in}}%
\pgfpathlineto{\pgfqpoint{4.004912in}{2.420933in}}%
\pgfpathlineto{\pgfqpoint{4.017764in}{2.417507in}}%
\pgfpathlineto{\pgfqpoint{4.030621in}{2.414111in}}%
\pgfpathlineto{\pgfqpoint{4.043484in}{2.410744in}}%
\pgfpathlineto{\pgfqpoint{4.036038in}{2.403462in}}%
\pgfpathlineto{\pgfqpoint{4.028587in}{2.396175in}}%
\pgfpathlineto{\pgfqpoint{4.021130in}{2.388880in}}%
\pgfpathlineto{\pgfqpoint{4.013668in}{2.381580in}}%
\pgfpathlineto{\pgfqpoint{4.000793in}{2.384962in}}%
\pgfpathlineto{\pgfqpoint{3.987923in}{2.388374in}}%
\pgfpathlineto{\pgfqpoint{3.975060in}{2.391815in}}%
\pgfpathlineto{\pgfqpoint{3.962202in}{2.395285in}}%
\pgfpathlineto{\pgfqpoint{3.969676in}{2.402566in}}%
\pgfpathlineto{\pgfqpoint{3.977145in}{2.409843in}}%
\pgfpathlineto{\pgfqpoint{3.984609in}{2.417117in}}%
\pgfpathlineto{\pgfqpoint{3.992067in}{2.424389in}}%
\pgfpathclose%
\pgfusepath{fill}%
\end{pgfscope}%
\begin{pgfscope}%
\pgfpathrectangle{\pgfqpoint{1.254980in}{0.150000in}}{\pgfqpoint{5.490039in}{5.490039in}}%
\pgfusepath{clip}%
\pgfsetbuttcap%
\pgfsetroundjoin%
\definecolor{currentfill}{rgb}{0.268510,0.009605,0.335427}%
\pgfsetfillcolor{currentfill}%
\pgfsetfillopacity{0.700000}%
\pgfsetlinewidth{0.000000pt}%
\definecolor{currentstroke}{rgb}{0.000000,0.000000,0.000000}%
\pgfsetstrokecolor{currentstroke}%
\pgfsetdash{}{0pt}%
\pgfpathmoveto{\pgfqpoint{3.645721in}{2.414224in}}%
\pgfpathlineto{\pgfqpoint{3.658496in}{2.410056in}}%
\pgfpathlineto{\pgfqpoint{3.671275in}{2.405920in}}%
\pgfpathlineto{\pgfqpoint{3.684060in}{2.401818in}}%
\pgfpathlineto{\pgfqpoint{3.696850in}{2.397748in}}%
\pgfpathlineto{\pgfqpoint{3.689276in}{2.390645in}}%
\pgfpathlineto{\pgfqpoint{3.681697in}{2.383555in}}%
\pgfpathlineto{\pgfqpoint{3.674113in}{2.376480in}}%
\pgfpathlineto{\pgfqpoint{3.666522in}{2.369421in}}%
\pgfpathlineto{\pgfqpoint{3.653720in}{2.373544in}}%
\pgfpathlineto{\pgfqpoint{3.640923in}{2.377699in}}%
\pgfpathlineto{\pgfqpoint{3.628130in}{2.381888in}}%
\pgfpathlineto{\pgfqpoint{3.615343in}{2.386110in}}%
\pgfpathlineto{\pgfqpoint{3.622946in}{2.393112in}}%
\pgfpathlineto{\pgfqpoint{3.630544in}{2.400132in}}%
\pgfpathlineto{\pgfqpoint{3.638136in}{2.407170in}}%
\pgfpathlineto{\pgfqpoint{3.645721in}{2.414224in}}%
\pgfpathclose%
\pgfusepath{fill}%
\end{pgfscope}%
\begin{pgfscope}%
\pgfpathrectangle{\pgfqpoint{1.254980in}{0.150000in}}{\pgfqpoint{5.490039in}{5.490039in}}%
\pgfusepath{clip}%
\pgfsetbuttcap%
\pgfsetroundjoin%
\definecolor{currentfill}{rgb}{0.280267,0.073417,0.397163}%
\pgfsetfillcolor{currentfill}%
\pgfsetfillopacity{0.700000}%
\pgfsetlinewidth{0.000000pt}%
\definecolor{currentstroke}{rgb}{0.000000,0.000000,0.000000}%
\pgfsetstrokecolor{currentstroke}%
\pgfsetdash{}{0pt}%
\pgfpathmoveto{\pgfqpoint{4.979595in}{2.517337in}}%
\pgfpathlineto{\pgfqpoint{4.992680in}{2.514942in}}%
\pgfpathlineto{\pgfqpoint{5.005773in}{2.512572in}}%
\pgfpathlineto{\pgfqpoint{5.018872in}{2.510226in}}%
\pgfpathlineto{\pgfqpoint{5.031978in}{2.507904in}}%
\pgfpathlineto{\pgfqpoint{5.024906in}{2.501456in}}%
\pgfpathlineto{\pgfqpoint{5.017828in}{2.495021in}}%
\pgfpathlineto{\pgfqpoint{5.010746in}{2.488597in}}%
\pgfpathlineto{\pgfqpoint{5.003658in}{2.482179in}}%
\pgfpathlineto{\pgfqpoint{4.990537in}{2.484401in}}%
\pgfpathlineto{\pgfqpoint{4.977423in}{2.486647in}}%
\pgfpathlineto{\pgfqpoint{4.964316in}{2.488918in}}%
\pgfpathlineto{\pgfqpoint{4.951216in}{2.491213in}}%
\pgfpathlineto{\pgfqpoint{4.958318in}{2.497726in}}%
\pgfpathlineto{\pgfqpoint{4.965415in}{2.504248in}}%
\pgfpathlineto{\pgfqpoint{4.972507in}{2.510784in}}%
\pgfpathlineto{\pgfqpoint{4.979595in}{2.517337in}}%
\pgfpathclose%
\pgfusepath{fill}%
\end{pgfscope}%
\begin{pgfscope}%
\pgfpathrectangle{\pgfqpoint{1.254980in}{0.150000in}}{\pgfqpoint{5.490039in}{5.490039in}}%
\pgfusepath{clip}%
\pgfsetbuttcap%
\pgfsetroundjoin%
\definecolor{currentfill}{rgb}{0.278791,0.062145,0.386592}%
\pgfsetfillcolor{currentfill}%
\pgfsetfillopacity{0.700000}%
\pgfsetlinewidth{0.000000pt}%
\definecolor{currentstroke}{rgb}{0.000000,0.000000,0.000000}%
\pgfsetstrokecolor{currentstroke}%
\pgfsetdash{}{0pt}%
\pgfpathmoveto{\pgfqpoint{4.765913in}{2.493638in}}%
\pgfpathlineto{\pgfqpoint{4.778946in}{2.491131in}}%
\pgfpathlineto{\pgfqpoint{4.791985in}{2.488649in}}%
\pgfpathlineto{\pgfqpoint{4.805032in}{2.486192in}}%
\pgfpathlineto{\pgfqpoint{4.818084in}{2.483760in}}%
\pgfpathlineto{\pgfqpoint{4.810928in}{2.477095in}}%
\pgfpathlineto{\pgfqpoint{4.803765in}{2.470429in}}%
\pgfpathlineto{\pgfqpoint{4.796598in}{2.463759in}}%
\pgfpathlineto{\pgfqpoint{4.789425in}{2.457083in}}%
\pgfpathlineto{\pgfqpoint{4.776358in}{2.459440in}}%
\pgfpathlineto{\pgfqpoint{4.763298in}{2.461823in}}%
\pgfpathlineto{\pgfqpoint{4.750245in}{2.464231in}}%
\pgfpathlineto{\pgfqpoint{4.737199in}{2.466665in}}%
\pgfpathlineto{\pgfqpoint{4.744385in}{2.473410in}}%
\pgfpathlineto{\pgfqpoint{4.751566in}{2.480153in}}%
\pgfpathlineto{\pgfqpoint{4.758742in}{2.486895in}}%
\pgfpathlineto{\pgfqpoint{4.765913in}{2.493638in}}%
\pgfpathclose%
\pgfusepath{fill}%
\end{pgfscope}%
\begin{pgfscope}%
\pgfpathrectangle{\pgfqpoint{1.254980in}{0.150000in}}{\pgfqpoint{5.490039in}{5.490039in}}%
\pgfusepath{clip}%
\pgfsetbuttcap%
\pgfsetroundjoin%
\definecolor{currentfill}{rgb}{0.276022,0.044167,0.370164}%
\pgfsetfillcolor{currentfill}%
\pgfsetfillopacity{0.700000}%
\pgfsetlinewidth{0.000000pt}%
\definecolor{currentstroke}{rgb}{0.000000,0.000000,0.000000}%
\pgfsetstrokecolor{currentstroke}%
\pgfsetdash{}{0pt}%
\pgfpathmoveto{\pgfqpoint{4.552200in}{2.470066in}}%
\pgfpathlineto{\pgfqpoint{4.565181in}{2.467385in}}%
\pgfpathlineto{\pgfqpoint{4.578167in}{2.464730in}}%
\pgfpathlineto{\pgfqpoint{4.591161in}{2.462102in}}%
\pgfpathlineto{\pgfqpoint{4.604161in}{2.459499in}}%
\pgfpathlineto{\pgfqpoint{4.596921in}{2.452597in}}%
\pgfpathlineto{\pgfqpoint{4.589675in}{2.445685in}}%
\pgfpathlineto{\pgfqpoint{4.582425in}{2.438761in}}%
\pgfpathlineto{\pgfqpoint{4.575168in}{2.431823in}}%
\pgfpathlineto{\pgfqpoint{4.562156in}{2.434377in}}%
\pgfpathlineto{\pgfqpoint{4.549149in}{2.436957in}}%
\pgfpathlineto{\pgfqpoint{4.536150in}{2.439563in}}%
\pgfpathlineto{\pgfqpoint{4.523157in}{2.442195in}}%
\pgfpathlineto{\pgfqpoint{4.530426in}{2.449177in}}%
\pgfpathlineto{\pgfqpoint{4.537689in}{2.456148in}}%
\pgfpathlineto{\pgfqpoint{4.544947in}{2.463110in}}%
\pgfpathlineto{\pgfqpoint{4.552200in}{2.470066in}}%
\pgfpathclose%
\pgfusepath{fill}%
\end{pgfscope}%
\begin{pgfscope}%
\pgfpathrectangle{\pgfqpoint{1.254980in}{0.150000in}}{\pgfqpoint{5.490039in}{5.490039in}}%
\pgfusepath{clip}%
\pgfsetbuttcap%
\pgfsetroundjoin%
\definecolor{currentfill}{rgb}{0.280267,0.073417,0.397163}%
\pgfsetfillcolor{currentfill}%
\pgfsetfillopacity{0.700000}%
\pgfsetlinewidth{0.000000pt}%
\definecolor{currentstroke}{rgb}{0.000000,0.000000,0.000000}%
\pgfsetstrokecolor{currentstroke}%
\pgfsetdash{}{0pt}%
\pgfpathmoveto{\pgfqpoint{2.849860in}{2.521436in}}%
\pgfpathlineto{\pgfqpoint{2.862529in}{2.514809in}}%
\pgfpathlineto{\pgfqpoint{2.875201in}{2.508231in}}%
\pgfpathlineto{\pgfqpoint{2.887876in}{2.501702in}}%
\pgfpathlineto{\pgfqpoint{2.900554in}{2.495223in}}%
\pgfpathlineto{\pgfqpoint{2.892637in}{2.490371in}}%
\pgfpathlineto{\pgfqpoint{2.884711in}{2.485616in}}%
\pgfpathlineto{\pgfqpoint{2.876776in}{2.480962in}}%
\pgfpathlineto{\pgfqpoint{2.868832in}{2.476412in}}%
\pgfpathlineto{\pgfqpoint{2.856135in}{2.483022in}}%
\pgfpathlineto{\pgfqpoint{2.843442in}{2.489681in}}%
\pgfpathlineto{\pgfqpoint{2.830751in}{2.496390in}}%
\pgfpathlineto{\pgfqpoint{2.818064in}{2.503148in}}%
\pgfpathlineto{\pgfqpoint{2.826027in}{2.507563in}}%
\pgfpathlineto{\pgfqpoint{2.833981in}{2.512085in}}%
\pgfpathlineto{\pgfqpoint{2.841925in}{2.516710in}}%
\pgfpathlineto{\pgfqpoint{2.849860in}{2.521436in}}%
\pgfpathclose%
\pgfusepath{fill}%
\end{pgfscope}%
\begin{pgfscope}%
\pgfpathrectangle{\pgfqpoint{1.254980in}{0.150000in}}{\pgfqpoint{5.490039in}{5.490039in}}%
\pgfusepath{clip}%
\pgfsetbuttcap%
\pgfsetroundjoin%
\definecolor{currentfill}{rgb}{0.273809,0.031497,0.358853}%
\pgfsetfillcolor{currentfill}%
\pgfsetfillopacity{0.700000}%
\pgfsetlinewidth{0.000000pt}%
\definecolor{currentstroke}{rgb}{0.000000,0.000000,0.000000}%
\pgfsetstrokecolor{currentstroke}%
\pgfsetdash{}{0pt}%
\pgfpathmoveto{\pgfqpoint{4.338458in}{2.447359in}}%
\pgfpathlineto{\pgfqpoint{4.351387in}{2.444442in}}%
\pgfpathlineto{\pgfqpoint{4.364323in}{2.441552in}}%
\pgfpathlineto{\pgfqpoint{4.377265in}{2.438689in}}%
\pgfpathlineto{\pgfqpoint{4.390213in}{2.435854in}}%
\pgfpathlineto{\pgfqpoint{4.382892in}{2.428739in}}%
\pgfpathlineto{\pgfqpoint{4.375565in}{2.421612in}}%
\pgfpathlineto{\pgfqpoint{4.368233in}{2.414470in}}%
\pgfpathlineto{\pgfqpoint{4.360896in}{2.407313in}}%
\pgfpathlineto{\pgfqpoint{4.347936in}{2.410125in}}%
\pgfpathlineto{\pgfqpoint{4.334982in}{2.412965in}}%
\pgfpathlineto{\pgfqpoint{4.322034in}{2.415832in}}%
\pgfpathlineto{\pgfqpoint{4.309093in}{2.418726in}}%
\pgfpathlineto{\pgfqpoint{4.316442in}{2.425902in}}%
\pgfpathlineto{\pgfqpoint{4.323786in}{2.433065in}}%
\pgfpathlineto{\pgfqpoint{4.331125in}{2.440217in}}%
\pgfpathlineto{\pgfqpoint{4.338458in}{2.447359in}}%
\pgfpathclose%
\pgfusepath{fill}%
\end{pgfscope}%
\begin{pgfscope}%
\pgfpathrectangle{\pgfqpoint{1.254980in}{0.150000in}}{\pgfqpoint{5.490039in}{5.490039in}}%
\pgfusepath{clip}%
\pgfsetbuttcap%
\pgfsetroundjoin%
\definecolor{currentfill}{rgb}{0.268510,0.009605,0.335427}%
\pgfsetfillcolor{currentfill}%
\pgfsetfillopacity{0.700000}%
\pgfsetlinewidth{0.000000pt}%
\definecolor{currentstroke}{rgb}{0.000000,0.000000,0.000000}%
\pgfsetstrokecolor{currentstroke}%
\pgfsetdash{}{0pt}%
\pgfpathmoveto{\pgfqpoint{3.778247in}{2.410492in}}%
\pgfpathlineto{\pgfqpoint{3.791051in}{2.406625in}}%
\pgfpathlineto{\pgfqpoint{3.803860in}{2.402789in}}%
\pgfpathlineto{\pgfqpoint{3.816674in}{2.398984in}}%
\pgfpathlineto{\pgfqpoint{3.829494in}{2.395211in}}%
\pgfpathlineto{\pgfqpoint{3.821968in}{2.387989in}}%
\pgfpathlineto{\pgfqpoint{3.814436in}{2.380771in}}%
\pgfpathlineto{\pgfqpoint{3.806899in}{2.373559in}}%
\pgfpathlineto{\pgfqpoint{3.799357in}{2.366352in}}%
\pgfpathlineto{\pgfqpoint{3.786525in}{2.370165in}}%
\pgfpathlineto{\pgfqpoint{3.773698in}{2.374010in}}%
\pgfpathlineto{\pgfqpoint{3.760877in}{2.377887in}}%
\pgfpathlineto{\pgfqpoint{3.748061in}{2.381795in}}%
\pgfpathlineto{\pgfqpoint{3.755616in}{2.388956in}}%
\pgfpathlineto{\pgfqpoint{3.763165in}{2.396127in}}%
\pgfpathlineto{\pgfqpoint{3.770709in}{2.403306in}}%
\pgfpathlineto{\pgfqpoint{3.778247in}{2.410492in}}%
\pgfpathclose%
\pgfusepath{fill}%
\end{pgfscope}%
\begin{pgfscope}%
\pgfpathrectangle{\pgfqpoint{1.254980in}{0.150000in}}{\pgfqpoint{5.490039in}{5.490039in}}%
\pgfusepath{clip}%
\pgfsetbuttcap%
\pgfsetroundjoin%
\definecolor{currentfill}{rgb}{0.282910,0.105393,0.426902}%
\pgfsetfillcolor{currentfill}%
\pgfsetfillopacity{0.700000}%
\pgfsetlinewidth{0.000000pt}%
\definecolor{currentstroke}{rgb}{0.000000,0.000000,0.000000}%
\pgfsetstrokecolor{currentstroke}%
\pgfsetdash{}{0pt}%
\pgfpathmoveto{\pgfqpoint{5.540069in}{2.571068in}}%
\pgfpathlineto{\pgfqpoint{5.553297in}{2.568747in}}%
\pgfpathlineto{\pgfqpoint{5.566533in}{2.566450in}}%
\pgfpathlineto{\pgfqpoint{5.579775in}{2.564175in}}%
\pgfpathlineto{\pgfqpoint{5.593025in}{2.561924in}}%
\pgfpathlineto{\pgfqpoint{5.586170in}{2.555712in}}%
\pgfpathlineto{\pgfqpoint{5.579312in}{2.549574in}}%
\pgfpathlineto{\pgfqpoint{5.572450in}{2.543505in}}%
\pgfpathlineto{\pgfqpoint{5.565586in}{2.537498in}}%
\pgfpathlineto{\pgfqpoint{5.552317in}{2.539587in}}%
\pgfpathlineto{\pgfqpoint{5.539056in}{2.541698in}}%
\pgfpathlineto{\pgfqpoint{5.525802in}{2.543833in}}%
\pgfpathlineto{\pgfqpoint{5.512556in}{2.545990in}}%
\pgfpathlineto{\pgfqpoint{5.519439in}{2.552155in}}%
\pgfpathlineto{\pgfqpoint{5.526318in}{2.558386in}}%
\pgfpathlineto{\pgfqpoint{5.533195in}{2.564689in}}%
\pgfpathlineto{\pgfqpoint{5.540069in}{2.571068in}}%
\pgfpathclose%
\pgfusepath{fill}%
\end{pgfscope}%
\begin{pgfscope}%
\pgfpathrectangle{\pgfqpoint{1.254980in}{0.150000in}}{\pgfqpoint{5.490039in}{5.490039in}}%
\pgfusepath{clip}%
\pgfsetbuttcap%
\pgfsetroundjoin%
\definecolor{currentfill}{rgb}{0.271305,0.019942,0.347269}%
\pgfsetfillcolor{currentfill}%
\pgfsetfillopacity{0.700000}%
\pgfsetlinewidth{0.000000pt}%
\definecolor{currentstroke}{rgb}{0.000000,0.000000,0.000000}%
\pgfsetstrokecolor{currentstroke}%
\pgfsetdash{}{0pt}%
\pgfpathmoveto{\pgfqpoint{4.124677in}{2.426650in}}%
\pgfpathlineto{\pgfqpoint{4.137558in}{2.423431in}}%
\pgfpathlineto{\pgfqpoint{4.150444in}{2.420240in}}%
\pgfpathlineto{\pgfqpoint{4.163337in}{2.417078in}}%
\pgfpathlineto{\pgfqpoint{4.176235in}{2.413944in}}%
\pgfpathlineto{\pgfqpoint{4.168834in}{2.406688in}}%
\pgfpathlineto{\pgfqpoint{4.161428in}{2.399421in}}%
\pgfpathlineto{\pgfqpoint{4.154017in}{2.392142in}}%
\pgfpathlineto{\pgfqpoint{4.146600in}{2.384852in}}%
\pgfpathlineto{\pgfqpoint{4.133690in}{2.387988in}}%
\pgfpathlineto{\pgfqpoint{4.120786in}{2.391153in}}%
\pgfpathlineto{\pgfqpoint{4.107887in}{2.394346in}}%
\pgfpathlineto{\pgfqpoint{4.094995in}{2.397568in}}%
\pgfpathlineto{\pgfqpoint{4.102424in}{2.404851in}}%
\pgfpathlineto{\pgfqpoint{4.109847in}{2.412126in}}%
\pgfpathlineto{\pgfqpoint{4.117265in}{2.419392in}}%
\pgfpathlineto{\pgfqpoint{4.124677in}{2.426650in}}%
\pgfpathclose%
\pgfusepath{fill}%
\end{pgfscope}%
\begin{pgfscope}%
\pgfpathrectangle{\pgfqpoint{1.254980in}{0.150000in}}{\pgfqpoint{5.490039in}{5.490039in}}%
\pgfusepath{clip}%
\pgfsetbuttcap%
\pgfsetroundjoin%
\definecolor{currentfill}{rgb}{0.282327,0.094955,0.417331}%
\pgfsetfillcolor{currentfill}%
\pgfsetfillopacity{0.700000}%
\pgfsetlinewidth{0.000000pt}%
\definecolor{currentstroke}{rgb}{0.000000,0.000000,0.000000}%
\pgfsetstrokecolor{currentstroke}%
\pgfsetdash{}{0pt}%
\pgfpathmoveto{\pgfqpoint{5.326380in}{2.547809in}}%
\pgfpathlineto{\pgfqpoint{5.339558in}{2.545520in}}%
\pgfpathlineto{\pgfqpoint{5.352743in}{2.543256in}}%
\pgfpathlineto{\pgfqpoint{5.365935in}{2.541015in}}%
\pgfpathlineto{\pgfqpoint{5.379134in}{2.538797in}}%
\pgfpathlineto{\pgfqpoint{5.372197in}{2.532607in}}%
\pgfpathlineto{\pgfqpoint{5.365256in}{2.526462in}}%
\pgfpathlineto{\pgfqpoint{5.358310in}{2.520357in}}%
\pgfpathlineto{\pgfqpoint{5.351361in}{2.514288in}}%
\pgfpathlineto{\pgfqpoint{5.338144in}{2.516368in}}%
\pgfpathlineto{\pgfqpoint{5.324935in}{2.518471in}}%
\pgfpathlineto{\pgfqpoint{5.311733in}{2.520598in}}%
\pgfpathlineto{\pgfqpoint{5.298539in}{2.522749in}}%
\pgfpathlineto{\pgfqpoint{5.305505in}{2.528951in}}%
\pgfpathlineto{\pgfqpoint{5.312468in}{2.535192in}}%
\pgfpathlineto{\pgfqpoint{5.319426in}{2.541477in}}%
\pgfpathlineto{\pgfqpoint{5.326380in}{2.547809in}}%
\pgfpathclose%
\pgfusepath{fill}%
\end{pgfscope}%
\begin{pgfscope}%
\pgfpathrectangle{\pgfqpoint{1.254980in}{0.150000in}}{\pgfqpoint{5.490039in}{5.490039in}}%
\pgfusepath{clip}%
\pgfsetbuttcap%
\pgfsetroundjoin%
\definecolor{currentfill}{rgb}{0.271305,0.019942,0.347269}%
\pgfsetfillcolor{currentfill}%
\pgfsetfillopacity{0.700000}%
\pgfsetlinewidth{0.000000pt}%
\definecolor{currentstroke}{rgb}{0.000000,0.000000,0.000000}%
\pgfsetstrokecolor{currentstroke}%
\pgfsetdash{}{0pt}%
\pgfpathmoveto{\pgfqpoint{3.299027in}{2.425608in}}%
\pgfpathlineto{\pgfqpoint{3.311748in}{2.420524in}}%
\pgfpathlineto{\pgfqpoint{3.324474in}{2.415479in}}%
\pgfpathlineto{\pgfqpoint{3.337203in}{2.410471in}}%
\pgfpathlineto{\pgfqpoint{3.349937in}{2.405502in}}%
\pgfpathlineto{\pgfqpoint{3.342224in}{2.399074in}}%
\pgfpathlineto{\pgfqpoint{3.334504in}{2.392692in}}%
\pgfpathlineto{\pgfqpoint{3.326778in}{2.386358in}}%
\pgfpathlineto{\pgfqpoint{3.319045in}{2.380073in}}%
\pgfpathlineto{\pgfqpoint{3.306296in}{2.385134in}}%
\pgfpathlineto{\pgfqpoint{3.293552in}{2.390233in}}%
\pgfpathlineto{\pgfqpoint{3.280812in}{2.395370in}}%
\pgfpathlineto{\pgfqpoint{3.268076in}{2.400545in}}%
\pgfpathlineto{\pgfqpoint{3.275824in}{2.406733in}}%
\pgfpathlineto{\pgfqpoint{3.283565in}{2.412974in}}%
\pgfpathlineto{\pgfqpoint{3.291300in}{2.419267in}}%
\pgfpathlineto{\pgfqpoint{3.299027in}{2.425608in}}%
\pgfpathclose%
\pgfusepath{fill}%
\end{pgfscope}%
\begin{pgfscope}%
\pgfpathrectangle{\pgfqpoint{1.254980in}{0.150000in}}{\pgfqpoint{5.490039in}{5.490039in}}%
\pgfusepath{clip}%
\pgfsetbuttcap%
\pgfsetroundjoin%
\definecolor{currentfill}{rgb}{0.273809,0.031497,0.358853}%
\pgfsetfillcolor{currentfill}%
\pgfsetfillopacity{0.700000}%
\pgfsetlinewidth{0.000000pt}%
\definecolor{currentstroke}{rgb}{0.000000,0.000000,0.000000}%
\pgfsetstrokecolor{currentstroke}%
\pgfsetdash{}{0pt}%
\pgfpathmoveto{\pgfqpoint{3.166338in}{2.443374in}}%
\pgfpathlineto{\pgfqpoint{3.179041in}{2.437879in}}%
\pgfpathlineto{\pgfqpoint{3.191748in}{2.432424in}}%
\pgfpathlineto{\pgfqpoint{3.204459in}{2.427011in}}%
\pgfpathlineto{\pgfqpoint{3.217174in}{2.421638in}}%
\pgfpathlineto{\pgfqpoint{3.209404in}{2.415605in}}%
\pgfpathlineto{\pgfqpoint{3.201627in}{2.409633in}}%
\pgfpathlineto{\pgfqpoint{3.193842in}{2.403723in}}%
\pgfpathlineto{\pgfqpoint{3.186050in}{2.397879in}}%
\pgfpathlineto{\pgfqpoint{3.173319in}{2.403356in}}%
\pgfpathlineto{\pgfqpoint{3.160592in}{2.408873in}}%
\pgfpathlineto{\pgfqpoint{3.147869in}{2.414432in}}%
\pgfpathlineto{\pgfqpoint{3.135150in}{2.420031in}}%
\pgfpathlineto{\pgfqpoint{3.142958in}{2.425767in}}%
\pgfpathlineto{\pgfqpoint{3.150759in}{2.431570in}}%
\pgfpathlineto{\pgfqpoint{3.158552in}{2.437440in}}%
\pgfpathlineto{\pgfqpoint{3.166338in}{2.443374in}}%
\pgfpathclose%
\pgfusepath{fill}%
\end{pgfscope}%
\begin{pgfscope}%
\pgfpathrectangle{\pgfqpoint{1.254980in}{0.150000in}}{\pgfqpoint{5.490039in}{5.490039in}}%
\pgfusepath{clip}%
\pgfsetbuttcap%
\pgfsetroundjoin%
\definecolor{currentfill}{rgb}{0.281446,0.084320,0.407414}%
\pgfsetfillcolor{currentfill}%
\pgfsetfillopacity{0.700000}%
\pgfsetlinewidth{0.000000pt}%
\definecolor{currentstroke}{rgb}{0.000000,0.000000,0.000000}%
\pgfsetstrokecolor{currentstroke}%
\pgfsetdash{}{0pt}%
\pgfpathmoveto{\pgfqpoint{5.112652in}{2.524411in}}%
\pgfpathlineto{\pgfqpoint{5.125778in}{2.522099in}}%
\pgfpathlineto{\pgfqpoint{5.138911in}{2.519810in}}%
\pgfpathlineto{\pgfqpoint{5.152050in}{2.517546in}}%
\pgfpathlineto{\pgfqpoint{5.165197in}{2.515306in}}%
\pgfpathlineto{\pgfqpoint{5.158175in}{2.509003in}}%
\pgfpathlineto{\pgfqpoint{5.151148in}{2.502721in}}%
\pgfpathlineto{\pgfqpoint{5.144117in}{2.496457in}}%
\pgfpathlineto{\pgfqpoint{5.137080in}{2.490208in}}%
\pgfpathlineto{\pgfqpoint{5.123917in}{2.492335in}}%
\pgfpathlineto{\pgfqpoint{5.110762in}{2.494487in}}%
\pgfpathlineto{\pgfqpoint{5.097614in}{2.496662in}}%
\pgfpathlineto{\pgfqpoint{5.084473in}{2.498862in}}%
\pgfpathlineto{\pgfqpoint{5.091525in}{2.505220in}}%
\pgfpathlineto{\pgfqpoint{5.098572in}{2.511595in}}%
\pgfpathlineto{\pgfqpoint{5.105615in}{2.517991in}}%
\pgfpathlineto{\pgfqpoint{5.112652in}{2.524411in}}%
\pgfpathclose%
\pgfusepath{fill}%
\end{pgfscope}%
\begin{pgfscope}%
\pgfpathrectangle{\pgfqpoint{1.254980in}{0.150000in}}{\pgfqpoint{5.490039in}{5.490039in}}%
\pgfusepath{clip}%
\pgfsetbuttcap%
\pgfsetroundjoin%
\definecolor{currentfill}{rgb}{0.269944,0.014625,0.341379}%
\pgfsetfillcolor{currentfill}%
\pgfsetfillopacity{0.700000}%
\pgfsetlinewidth{0.000000pt}%
\definecolor{currentstroke}{rgb}{0.000000,0.000000,0.000000}%
\pgfsetstrokecolor{currentstroke}%
\pgfsetdash{}{0pt}%
\pgfpathmoveto{\pgfqpoint{3.431649in}{2.412449in}}%
\pgfpathlineto{\pgfqpoint{3.444391in}{2.407744in}}%
\pgfpathlineto{\pgfqpoint{3.457139in}{2.403074in}}%
\pgfpathlineto{\pgfqpoint{3.469890in}{2.398441in}}%
\pgfpathlineto{\pgfqpoint{3.482647in}{2.393843in}}%
\pgfpathlineto{\pgfqpoint{3.474987in}{2.387101in}}%
\pgfpathlineto{\pgfqpoint{3.467321in}{2.380391in}}%
\pgfpathlineto{\pgfqpoint{3.459649in}{2.373716in}}%
\pgfpathlineto{\pgfqpoint{3.451971in}{2.367077in}}%
\pgfpathlineto{\pgfqpoint{3.439201in}{2.371753in}}%
\pgfpathlineto{\pgfqpoint{3.426435in}{2.376465in}}%
\pgfpathlineto{\pgfqpoint{3.413674in}{2.381213in}}%
\pgfpathlineto{\pgfqpoint{3.400918in}{2.385998in}}%
\pgfpathlineto{\pgfqpoint{3.408610in}{2.392553in}}%
\pgfpathlineto{\pgfqpoint{3.416296in}{2.399148in}}%
\pgfpathlineto{\pgfqpoint{3.423976in}{2.405781in}}%
\pgfpathlineto{\pgfqpoint{3.431649in}{2.412449in}}%
\pgfpathclose%
\pgfusepath{fill}%
\end{pgfscope}%
\begin{pgfscope}%
\pgfpathrectangle{\pgfqpoint{1.254980in}{0.150000in}}{\pgfqpoint{5.490039in}{5.490039in}}%
\pgfusepath{clip}%
\pgfsetbuttcap%
\pgfsetroundjoin%
\definecolor{currentfill}{rgb}{0.268510,0.009605,0.335427}%
\pgfsetfillcolor{currentfill}%
\pgfsetfillopacity{0.700000}%
\pgfsetlinewidth{0.000000pt}%
\definecolor{currentstroke}{rgb}{0.000000,0.000000,0.000000}%
\pgfsetstrokecolor{currentstroke}%
\pgfsetdash{}{0pt}%
\pgfpathmoveto{\pgfqpoint{3.910828in}{2.409469in}}%
\pgfpathlineto{\pgfqpoint{3.923663in}{2.405877in}}%
\pgfpathlineto{\pgfqpoint{3.936504in}{2.402317in}}%
\pgfpathlineto{\pgfqpoint{3.949350in}{2.398786in}}%
\pgfpathlineto{\pgfqpoint{3.962202in}{2.395285in}}%
\pgfpathlineto{\pgfqpoint{3.954722in}{2.388002in}}%
\pgfpathlineto{\pgfqpoint{3.947237in}{2.380715in}}%
\pgfpathlineto{\pgfqpoint{3.939747in}{2.373426in}}%
\pgfpathlineto{\pgfqpoint{3.932250in}{2.366135in}}%
\pgfpathlineto{\pgfqpoint{3.919386in}{2.369663in}}%
\pgfpathlineto{\pgfqpoint{3.906528in}{2.373221in}}%
\pgfpathlineto{\pgfqpoint{3.893675in}{2.376810in}}%
\pgfpathlineto{\pgfqpoint{3.880828in}{2.380429in}}%
\pgfpathlineto{\pgfqpoint{3.888336in}{2.387688in}}%
\pgfpathlineto{\pgfqpoint{3.895839in}{2.394948in}}%
\pgfpathlineto{\pgfqpoint{3.903336in}{2.402208in}}%
\pgfpathlineto{\pgfqpoint{3.910828in}{2.409469in}}%
\pgfpathclose%
\pgfusepath{fill}%
\end{pgfscope}%
\begin{pgfscope}%
\pgfpathrectangle{\pgfqpoint{1.254980in}{0.150000in}}{\pgfqpoint{5.490039in}{5.490039in}}%
\pgfusepath{clip}%
\pgfsetbuttcap%
\pgfsetroundjoin%
\definecolor{currentfill}{rgb}{0.279566,0.067836,0.391917}%
\pgfsetfillcolor{currentfill}%
\pgfsetfillopacity{0.700000}%
\pgfsetlinewidth{0.000000pt}%
\definecolor{currentstroke}{rgb}{0.000000,0.000000,0.000000}%
\pgfsetstrokecolor{currentstroke}%
\pgfsetdash{}{0pt}%
\pgfpathmoveto{\pgfqpoint{4.898883in}{2.500641in}}%
\pgfpathlineto{\pgfqpoint{4.911956in}{2.498247in}}%
\pgfpathlineto{\pgfqpoint{4.925036in}{2.495877in}}%
\pgfpathlineto{\pgfqpoint{4.938122in}{2.493533in}}%
\pgfpathlineto{\pgfqpoint{4.951216in}{2.491213in}}%
\pgfpathlineto{\pgfqpoint{4.944108in}{2.484707in}}%
\pgfpathlineto{\pgfqpoint{4.936995in}{2.478206in}}%
\pgfpathlineto{\pgfqpoint{4.929877in}{2.471705in}}%
\pgfpathlineto{\pgfqpoint{4.922754in}{2.465203in}}%
\pgfpathlineto{\pgfqpoint{4.909646in}{2.467436in}}%
\pgfpathlineto{\pgfqpoint{4.896545in}{2.469693in}}%
\pgfpathlineto{\pgfqpoint{4.883451in}{2.471976in}}%
\pgfpathlineto{\pgfqpoint{4.870364in}{2.474283in}}%
\pgfpathlineto{\pgfqpoint{4.877502in}{2.480867in}}%
\pgfpathlineto{\pgfqpoint{4.884634in}{2.487453in}}%
\pgfpathlineto{\pgfqpoint{4.891761in}{2.494043in}}%
\pgfpathlineto{\pgfqpoint{4.898883in}{2.500641in}}%
\pgfpathclose%
\pgfusepath{fill}%
\end{pgfscope}%
\begin{pgfscope}%
\pgfpathrectangle{\pgfqpoint{1.254980in}{0.150000in}}{\pgfqpoint{5.490039in}{5.490039in}}%
\pgfusepath{clip}%
\pgfsetbuttcap%
\pgfsetroundjoin%
\definecolor{currentfill}{rgb}{0.277941,0.056324,0.381191}%
\pgfsetfillcolor{currentfill}%
\pgfsetfillopacity{0.700000}%
\pgfsetlinewidth{0.000000pt}%
\definecolor{currentstroke}{rgb}{0.000000,0.000000,0.000000}%
\pgfsetstrokecolor{currentstroke}%
\pgfsetdash{}{0pt}%
\pgfpathmoveto{\pgfqpoint{4.685080in}{2.476652in}}%
\pgfpathlineto{\pgfqpoint{4.698099in}{2.474117in}}%
\pgfpathlineto{\pgfqpoint{4.711126in}{2.471608in}}%
\pgfpathlineto{\pgfqpoint{4.724159in}{2.469123in}}%
\pgfpathlineto{\pgfqpoint{4.737199in}{2.466665in}}%
\pgfpathlineto{\pgfqpoint{4.730007in}{2.459914in}}%
\pgfpathlineto{\pgfqpoint{4.722809in}{2.453155in}}%
\pgfpathlineto{\pgfqpoint{4.715606in}{2.446386in}}%
\pgfpathlineto{\pgfqpoint{4.708398in}{2.439605in}}%
\pgfpathlineto{\pgfqpoint{4.695345in}{2.442003in}}%
\pgfpathlineto{\pgfqpoint{4.682299in}{2.444425in}}%
\pgfpathlineto{\pgfqpoint{4.669259in}{2.446873in}}%
\pgfpathlineto{\pgfqpoint{4.656226in}{2.449347in}}%
\pgfpathlineto{\pgfqpoint{4.663448in}{2.456185in}}%
\pgfpathlineto{\pgfqpoint{4.670664in}{2.463013in}}%
\pgfpathlineto{\pgfqpoint{4.677874in}{2.469835in}}%
\pgfpathlineto{\pgfqpoint{4.685080in}{2.476652in}}%
\pgfpathclose%
\pgfusepath{fill}%
\end{pgfscope}%
\begin{pgfscope}%
\pgfpathrectangle{\pgfqpoint{1.254980in}{0.150000in}}{\pgfqpoint{5.490039in}{5.490039in}}%
\pgfusepath{clip}%
\pgfsetbuttcap%
\pgfsetroundjoin%
\definecolor{currentfill}{rgb}{0.276022,0.044167,0.370164}%
\pgfsetfillcolor{currentfill}%
\pgfsetfillopacity{0.700000}%
\pgfsetlinewidth{0.000000pt}%
\definecolor{currentstroke}{rgb}{0.000000,0.000000,0.000000}%
\pgfsetstrokecolor{currentstroke}%
\pgfsetdash{}{0pt}%
\pgfpathmoveto{\pgfqpoint{3.033532in}{2.466357in}}%
\pgfpathlineto{\pgfqpoint{3.046222in}{2.460414in}}%
\pgfpathlineto{\pgfqpoint{3.058915in}{2.454516in}}%
\pgfpathlineto{\pgfqpoint{3.071612in}{2.448661in}}%
\pgfpathlineto{\pgfqpoint{3.084312in}{2.442850in}}%
\pgfpathlineto{\pgfqpoint{3.076480in}{2.437298in}}%
\pgfpathlineto{\pgfqpoint{3.068640in}{2.431823in}}%
\pgfpathlineto{\pgfqpoint{3.060793in}{2.426427in}}%
\pgfpathlineto{\pgfqpoint{3.052937in}{2.421114in}}%
\pgfpathlineto{\pgfqpoint{3.040220in}{2.427042in}}%
\pgfpathlineto{\pgfqpoint{3.027506in}{2.433014in}}%
\pgfpathlineto{\pgfqpoint{3.014796in}{2.439030in}}%
\pgfpathlineto{\pgfqpoint{3.002089in}{2.445091in}}%
\pgfpathlineto{\pgfqpoint{3.009962in}{2.450282in}}%
\pgfpathlineto{\pgfqpoint{3.017827in}{2.455559in}}%
\pgfpathlineto{\pgfqpoint{3.025683in}{2.460918in}}%
\pgfpathlineto{\pgfqpoint{3.033532in}{2.466357in}}%
\pgfpathclose%
\pgfusepath{fill}%
\end{pgfscope}%
\begin{pgfscope}%
\pgfpathrectangle{\pgfqpoint{1.254980in}{0.150000in}}{\pgfqpoint{5.490039in}{5.490039in}}%
\pgfusepath{clip}%
\pgfsetbuttcap%
\pgfsetroundjoin%
\definecolor{currentfill}{rgb}{0.268510,0.009605,0.335427}%
\pgfsetfillcolor{currentfill}%
\pgfsetfillopacity{0.700000}%
\pgfsetlinewidth{0.000000pt}%
\definecolor{currentstroke}{rgb}{0.000000,0.000000,0.000000}%
\pgfsetstrokecolor{currentstroke}%
\pgfsetdash{}{0pt}%
\pgfpathmoveto{\pgfqpoint{3.564245in}{2.403335in}}%
\pgfpathlineto{\pgfqpoint{3.577012in}{2.398978in}}%
\pgfpathlineto{\pgfqpoint{3.589784in}{2.394655in}}%
\pgfpathlineto{\pgfqpoint{3.602561in}{2.390366in}}%
\pgfpathlineto{\pgfqpoint{3.615343in}{2.386110in}}%
\pgfpathlineto{\pgfqpoint{3.607734in}{2.379128in}}%
\pgfpathlineto{\pgfqpoint{3.600119in}{2.372168in}}%
\pgfpathlineto{\pgfqpoint{3.592498in}{2.365229in}}%
\pgfpathlineto{\pgfqpoint{3.584872in}{2.358314in}}%
\pgfpathlineto{\pgfqpoint{3.572076in}{2.362636in}}%
\pgfpathlineto{\pgfqpoint{3.559286in}{2.366991in}}%
\pgfpathlineto{\pgfqpoint{3.546501in}{2.371380in}}%
\pgfpathlineto{\pgfqpoint{3.533720in}{2.375803in}}%
\pgfpathlineto{\pgfqpoint{3.541360in}{2.382648in}}%
\pgfpathlineto{\pgfqpoint{3.548994in}{2.389519in}}%
\pgfpathlineto{\pgfqpoint{3.556622in}{2.396415in}}%
\pgfpathlineto{\pgfqpoint{3.564245in}{2.403335in}}%
\pgfpathclose%
\pgfusepath{fill}%
\end{pgfscope}%
\begin{pgfscope}%
\pgfpathrectangle{\pgfqpoint{1.254980in}{0.150000in}}{\pgfqpoint{5.490039in}{5.490039in}}%
\pgfusepath{clip}%
\pgfsetbuttcap%
\pgfsetroundjoin%
\definecolor{currentfill}{rgb}{0.274952,0.037752,0.364543}%
\pgfsetfillcolor{currentfill}%
\pgfsetfillopacity{0.700000}%
\pgfsetlinewidth{0.000000pt}%
\definecolor{currentstroke}{rgb}{0.000000,0.000000,0.000000}%
\pgfsetstrokecolor{currentstroke}%
\pgfsetdash{}{0pt}%
\pgfpathmoveto{\pgfqpoint{4.471249in}{2.452988in}}%
\pgfpathlineto{\pgfqpoint{4.484216in}{2.450250in}}%
\pgfpathlineto{\pgfqpoint{4.497190in}{2.447539in}}%
\pgfpathlineto{\pgfqpoint{4.510170in}{2.444854in}}%
\pgfpathlineto{\pgfqpoint{4.523157in}{2.442195in}}%
\pgfpathlineto{\pgfqpoint{4.515882in}{2.435201in}}%
\pgfpathlineto{\pgfqpoint{4.508602in}{2.428194in}}%
\pgfpathlineto{\pgfqpoint{4.501317in}{2.421171in}}%
\pgfpathlineto{\pgfqpoint{4.494026in}{2.414131in}}%
\pgfpathlineto{\pgfqpoint{4.481027in}{2.416753in}}%
\pgfpathlineto{\pgfqpoint{4.468034in}{2.419402in}}%
\pgfpathlineto{\pgfqpoint{4.455048in}{2.422077in}}%
\pgfpathlineto{\pgfqpoint{4.442068in}{2.424779in}}%
\pgfpathlineto{\pgfqpoint{4.449371in}{2.431850in}}%
\pgfpathlineto{\pgfqpoint{4.456669in}{2.438908in}}%
\pgfpathlineto{\pgfqpoint{4.463962in}{2.445953in}}%
\pgfpathlineto{\pgfqpoint{4.471249in}{2.452988in}}%
\pgfpathclose%
\pgfusepath{fill}%
\end{pgfscope}%
\begin{pgfscope}%
\pgfpathrectangle{\pgfqpoint{1.254980in}{0.150000in}}{\pgfqpoint{5.490039in}{5.490039in}}%
\pgfusepath{clip}%
\pgfsetbuttcap%
\pgfsetroundjoin%
\definecolor{currentfill}{rgb}{0.272594,0.025563,0.353093}%
\pgfsetfillcolor{currentfill}%
\pgfsetfillopacity{0.700000}%
\pgfsetlinewidth{0.000000pt}%
\definecolor{currentstroke}{rgb}{0.000000,0.000000,0.000000}%
\pgfsetstrokecolor{currentstroke}%
\pgfsetdash{}{0pt}%
\pgfpathmoveto{\pgfqpoint{4.257389in}{2.430577in}}%
\pgfpathlineto{\pgfqpoint{4.270306in}{2.427573in}}%
\pgfpathlineto{\pgfqpoint{4.283229in}{2.424596in}}%
\pgfpathlineto{\pgfqpoint{4.296158in}{2.421647in}}%
\pgfpathlineto{\pgfqpoint{4.309093in}{2.418726in}}%
\pgfpathlineto{\pgfqpoint{4.301738in}{2.411537in}}%
\pgfpathlineto{\pgfqpoint{4.294378in}{2.404333in}}%
\pgfpathlineto{\pgfqpoint{4.287012in}{2.397115in}}%
\pgfpathlineto{\pgfqpoint{4.279641in}{2.389880in}}%
\pgfpathlineto{\pgfqpoint{4.266693in}{2.392791in}}%
\pgfpathlineto{\pgfqpoint{4.253752in}{2.395730in}}%
\pgfpathlineto{\pgfqpoint{4.240818in}{2.398696in}}%
\pgfpathlineto{\pgfqpoint{4.227889in}{2.401690in}}%
\pgfpathlineto{\pgfqpoint{4.235272in}{2.408930in}}%
\pgfpathlineto{\pgfqpoint{4.242650in}{2.416157in}}%
\pgfpathlineto{\pgfqpoint{4.250022in}{2.423373in}}%
\pgfpathlineto{\pgfqpoint{4.257389in}{2.430577in}}%
\pgfpathclose%
\pgfusepath{fill}%
\end{pgfscope}%
\begin{pgfscope}%
\pgfpathrectangle{\pgfqpoint{1.254980in}{0.150000in}}{\pgfqpoint{5.490039in}{5.490039in}}%
\pgfusepath{clip}%
\pgfsetbuttcap%
\pgfsetroundjoin%
\definecolor{currentfill}{rgb}{0.268510,0.009605,0.335427}%
\pgfsetfillcolor{currentfill}%
\pgfsetfillopacity{0.700000}%
\pgfsetlinewidth{0.000000pt}%
\definecolor{currentstroke}{rgb}{0.000000,0.000000,0.000000}%
\pgfsetstrokecolor{currentstroke}%
\pgfsetdash{}{0pt}%
\pgfpathmoveto{\pgfqpoint{3.696850in}{2.397748in}}%
\pgfpathlineto{\pgfqpoint{3.709644in}{2.393711in}}%
\pgfpathlineto{\pgfqpoint{3.722445in}{2.389707in}}%
\pgfpathlineto{\pgfqpoint{3.735250in}{2.385735in}}%
\pgfpathlineto{\pgfqpoint{3.748061in}{2.381795in}}%
\pgfpathlineto{\pgfqpoint{3.740500in}{2.374643in}}%
\pgfpathlineto{\pgfqpoint{3.732934in}{2.367502in}}%
\pgfpathlineto{\pgfqpoint{3.725362in}{2.360372in}}%
\pgfpathlineto{\pgfqpoint{3.717784in}{2.353255in}}%
\pgfpathlineto{\pgfqpoint{3.704961in}{2.357248in}}%
\pgfpathlineto{\pgfqpoint{3.692143in}{2.361273in}}%
\pgfpathlineto{\pgfqpoint{3.679330in}{2.365331in}}%
\pgfpathlineto{\pgfqpoint{3.666522in}{2.369421in}}%
\pgfpathlineto{\pgfqpoint{3.674113in}{2.376480in}}%
\pgfpathlineto{\pgfqpoint{3.681697in}{2.383555in}}%
\pgfpathlineto{\pgfqpoint{3.689276in}{2.390645in}}%
\pgfpathlineto{\pgfqpoint{3.696850in}{2.397748in}}%
\pgfpathclose%
\pgfusepath{fill}%
\end{pgfscope}%
\begin{pgfscope}%
\pgfpathrectangle{\pgfqpoint{1.254980in}{0.150000in}}{\pgfqpoint{5.490039in}{5.490039in}}%
\pgfusepath{clip}%
\pgfsetbuttcap%
\pgfsetroundjoin%
\definecolor{currentfill}{rgb}{0.282910,0.105393,0.426902}%
\pgfsetfillcolor{currentfill}%
\pgfsetfillopacity{0.700000}%
\pgfsetlinewidth{0.000000pt}%
\definecolor{currentstroke}{rgb}{0.000000,0.000000,0.000000}%
\pgfsetstrokecolor{currentstroke}%
\pgfsetdash{}{0pt}%
\pgfpathmoveto{\pgfqpoint{5.459642in}{2.554854in}}%
\pgfpathlineto{\pgfqpoint{5.472860in}{2.552603in}}%
\pgfpathlineto{\pgfqpoint{5.486084in}{2.550376in}}%
\pgfpathlineto{\pgfqpoint{5.499316in}{2.548172in}}%
\pgfpathlineto{\pgfqpoint{5.512556in}{2.545990in}}%
\pgfpathlineto{\pgfqpoint{5.505669in}{2.539887in}}%
\pgfpathlineto{\pgfqpoint{5.498779in}{2.533841in}}%
\pgfpathlineto{\pgfqpoint{5.491885in}{2.527846in}}%
\pgfpathlineto{\pgfqpoint{5.484988in}{2.521899in}}%
\pgfpathlineto{\pgfqpoint{5.471730in}{2.523930in}}%
\pgfpathlineto{\pgfqpoint{5.458480in}{2.525984in}}%
\pgfpathlineto{\pgfqpoint{5.445238in}{2.528061in}}%
\pgfpathlineto{\pgfqpoint{5.432002in}{2.530161in}}%
\pgfpathlineto{\pgfqpoint{5.438918in}{2.536254in}}%
\pgfpathlineto{\pgfqpoint{5.445830in}{2.542397in}}%
\pgfpathlineto{\pgfqpoint{5.452738in}{2.548596in}}%
\pgfpathlineto{\pgfqpoint{5.459642in}{2.554854in}}%
\pgfpathclose%
\pgfusepath{fill}%
\end{pgfscope}%
\begin{pgfscope}%
\pgfpathrectangle{\pgfqpoint{1.254980in}{0.150000in}}{\pgfqpoint{5.490039in}{5.490039in}}%
\pgfusepath{clip}%
\pgfsetbuttcap%
\pgfsetroundjoin%
\definecolor{currentfill}{rgb}{0.279566,0.067836,0.391917}%
\pgfsetfillcolor{currentfill}%
\pgfsetfillopacity{0.700000}%
\pgfsetlinewidth{0.000000pt}%
\definecolor{currentstroke}{rgb}{0.000000,0.000000,0.000000}%
\pgfsetstrokecolor{currentstroke}%
\pgfsetdash{}{0pt}%
\pgfpathmoveto{\pgfqpoint{2.900554in}{2.495223in}}%
\pgfpathlineto{\pgfqpoint{2.913235in}{2.488792in}}%
\pgfpathlineto{\pgfqpoint{2.925919in}{2.482409in}}%
\pgfpathlineto{\pgfqpoint{2.938606in}{2.476073in}}%
\pgfpathlineto{\pgfqpoint{2.951296in}{2.469784in}}%
\pgfpathlineto{\pgfqpoint{2.943397in}{2.464807in}}%
\pgfpathlineto{\pgfqpoint{2.935489in}{2.459924in}}%
\pgfpathlineto{\pgfqpoint{2.927572in}{2.455138in}}%
\pgfpathlineto{\pgfqpoint{2.919647in}{2.450452in}}%
\pgfpathlineto{\pgfqpoint{2.906939in}{2.456871in}}%
\pgfpathlineto{\pgfqpoint{2.894233in}{2.463337in}}%
\pgfpathlineto{\pgfqpoint{2.881531in}{2.469851in}}%
\pgfpathlineto{\pgfqpoint{2.868832in}{2.476412in}}%
\pgfpathlineto{\pgfqpoint{2.876776in}{2.480962in}}%
\pgfpathlineto{\pgfqpoint{2.884711in}{2.485616in}}%
\pgfpathlineto{\pgfqpoint{2.892637in}{2.490371in}}%
\pgfpathlineto{\pgfqpoint{2.900554in}{2.495223in}}%
\pgfpathclose%
\pgfusepath{fill}%
\end{pgfscope}%
\begin{pgfscope}%
\pgfpathrectangle{\pgfqpoint{1.254980in}{0.150000in}}{\pgfqpoint{5.490039in}{5.490039in}}%
\pgfusepath{clip}%
\pgfsetbuttcap%
\pgfsetroundjoin%
\definecolor{currentfill}{rgb}{0.269944,0.014625,0.341379}%
\pgfsetfillcolor{currentfill}%
\pgfsetfillopacity{0.700000}%
\pgfsetlinewidth{0.000000pt}%
\definecolor{currentstroke}{rgb}{0.000000,0.000000,0.000000}%
\pgfsetstrokecolor{currentstroke}%
\pgfsetdash{}{0pt}%
\pgfpathmoveto{\pgfqpoint{4.043484in}{2.410744in}}%
\pgfpathlineto{\pgfqpoint{4.056353in}{2.407406in}}%
\pgfpathlineto{\pgfqpoint{4.069228in}{2.404098in}}%
\pgfpathlineto{\pgfqpoint{4.082109in}{2.400819in}}%
\pgfpathlineto{\pgfqpoint{4.094995in}{2.397568in}}%
\pgfpathlineto{\pgfqpoint{4.087561in}{2.390276in}}%
\pgfpathlineto{\pgfqpoint{4.080121in}{2.382975in}}%
\pgfpathlineto{\pgfqpoint{4.072677in}{2.375664in}}%
\pgfpathlineto{\pgfqpoint{4.065226in}{2.368344in}}%
\pgfpathlineto{\pgfqpoint{4.052328in}{2.371610in}}%
\pgfpathlineto{\pgfqpoint{4.039435in}{2.374904in}}%
\pgfpathlineto{\pgfqpoint{4.026549in}{2.378228in}}%
\pgfpathlineto{\pgfqpoint{4.013668in}{2.381580in}}%
\pgfpathlineto{\pgfqpoint{4.021130in}{2.388880in}}%
\pgfpathlineto{\pgfqpoint{4.028587in}{2.396175in}}%
\pgfpathlineto{\pgfqpoint{4.036038in}{2.403462in}}%
\pgfpathlineto{\pgfqpoint{4.043484in}{2.410744in}}%
\pgfpathclose%
\pgfusepath{fill}%
\end{pgfscope}%
\begin{pgfscope}%
\pgfpathrectangle{\pgfqpoint{1.254980in}{0.150000in}}{\pgfqpoint{5.490039in}{5.490039in}}%
\pgfusepath{clip}%
\pgfsetbuttcap%
\pgfsetroundjoin%
\definecolor{currentfill}{rgb}{0.281924,0.089666,0.412415}%
\pgfsetfillcolor{currentfill}%
\pgfsetfillopacity{0.700000}%
\pgfsetlinewidth{0.000000pt}%
\definecolor{currentstroke}{rgb}{0.000000,0.000000,0.000000}%
\pgfsetstrokecolor{currentstroke}%
\pgfsetdash{}{0pt}%
\pgfpathmoveto{\pgfqpoint{5.245832in}{2.531588in}}%
\pgfpathlineto{\pgfqpoint{5.258998in}{2.529343in}}%
\pgfpathlineto{\pgfqpoint{5.272171in}{2.527121in}}%
\pgfpathlineto{\pgfqpoint{5.285352in}{2.524923in}}%
\pgfpathlineto{\pgfqpoint{5.298539in}{2.522749in}}%
\pgfpathlineto{\pgfqpoint{5.291568in}{2.516581in}}%
\pgfpathlineto{\pgfqpoint{5.284592in}{2.510445in}}%
\pgfpathlineto{\pgfqpoint{5.277612in}{2.504335in}}%
\pgfpathlineto{\pgfqpoint{5.270627in}{2.498247in}}%
\pgfpathlineto{\pgfqpoint{5.257423in}{2.500296in}}%
\pgfpathlineto{\pgfqpoint{5.244227in}{2.502369in}}%
\pgfpathlineto{\pgfqpoint{5.231037in}{2.504466in}}%
\pgfpathlineto{\pgfqpoint{5.217855in}{2.506586in}}%
\pgfpathlineto{\pgfqpoint{5.224856in}{2.512794in}}%
\pgfpathlineto{\pgfqpoint{5.231853in}{2.519028in}}%
\pgfpathlineto{\pgfqpoint{5.238845in}{2.525291in}}%
\pgfpathlineto{\pgfqpoint{5.245832in}{2.531588in}}%
\pgfpathclose%
\pgfusepath{fill}%
\end{pgfscope}%
\begin{pgfscope}%
\pgfpathrectangle{\pgfqpoint{1.254980in}{0.150000in}}{\pgfqpoint{5.490039in}{5.490039in}}%
\pgfusepath{clip}%
\pgfsetbuttcap%
\pgfsetroundjoin%
\definecolor{currentfill}{rgb}{0.280894,0.078907,0.402329}%
\pgfsetfillcolor{currentfill}%
\pgfsetfillopacity{0.700000}%
\pgfsetlinewidth{0.000000pt}%
\definecolor{currentstroke}{rgb}{0.000000,0.000000,0.000000}%
\pgfsetstrokecolor{currentstroke}%
\pgfsetdash{}{0pt}%
\pgfpathmoveto{\pgfqpoint{5.031978in}{2.507904in}}%
\pgfpathlineto{\pgfqpoint{5.045091in}{2.505607in}}%
\pgfpathlineto{\pgfqpoint{5.058211in}{2.503335in}}%
\pgfpathlineto{\pgfqpoint{5.071339in}{2.501086in}}%
\pgfpathlineto{\pgfqpoint{5.084473in}{2.498862in}}%
\pgfpathlineto{\pgfqpoint{5.077415in}{2.492518in}}%
\pgfpathlineto{\pgfqpoint{5.070353in}{2.486185in}}%
\pgfpathlineto{\pgfqpoint{5.063286in}{2.479859in}}%
\pgfpathlineto{\pgfqpoint{5.056213in}{2.473536in}}%
\pgfpathlineto{\pgfqpoint{5.043064in}{2.475661in}}%
\pgfpathlineto{\pgfqpoint{5.029922in}{2.477809in}}%
\pgfpathlineto{\pgfqpoint{5.016786in}{2.479982in}}%
\pgfpathlineto{\pgfqpoint{5.003658in}{2.482179in}}%
\pgfpathlineto{\pgfqpoint{5.010746in}{2.488597in}}%
\pgfpathlineto{\pgfqpoint{5.017828in}{2.495021in}}%
\pgfpathlineto{\pgfqpoint{5.024906in}{2.501456in}}%
\pgfpathlineto{\pgfqpoint{5.031978in}{2.507904in}}%
\pgfpathclose%
\pgfusepath{fill}%
\end{pgfscope}%
\begin{pgfscope}%
\pgfpathrectangle{\pgfqpoint{1.254980in}{0.150000in}}{\pgfqpoint{5.490039in}{5.490039in}}%
\pgfusepath{clip}%
\pgfsetbuttcap%
\pgfsetroundjoin%
\definecolor{currentfill}{rgb}{0.279566,0.067836,0.391917}%
\pgfsetfillcolor{currentfill}%
\pgfsetfillopacity{0.700000}%
\pgfsetlinewidth{0.000000pt}%
\definecolor{currentstroke}{rgb}{0.000000,0.000000,0.000000}%
\pgfsetstrokecolor{currentstroke}%
\pgfsetdash{}{0pt}%
\pgfpathmoveto{\pgfqpoint{4.818084in}{2.483760in}}%
\pgfpathlineto{\pgfqpoint{4.831144in}{2.481353in}}%
\pgfpathlineto{\pgfqpoint{4.844211in}{2.478971in}}%
\pgfpathlineto{\pgfqpoint{4.857284in}{2.476615in}}%
\pgfpathlineto{\pgfqpoint{4.870364in}{2.474283in}}%
\pgfpathlineto{\pgfqpoint{4.863221in}{2.467696in}}%
\pgfpathlineto{\pgfqpoint{4.856073in}{2.461106in}}%
\pgfpathlineto{\pgfqpoint{4.848919in}{2.454509in}}%
\pgfpathlineto{\pgfqpoint{4.841760in}{2.447902in}}%
\pgfpathlineto{\pgfqpoint{4.828666in}{2.450160in}}%
\pgfpathlineto{\pgfqpoint{4.815579in}{2.452443in}}%
\pgfpathlineto{\pgfqpoint{4.802498in}{2.454750in}}%
\pgfpathlineto{\pgfqpoint{4.789425in}{2.457083in}}%
\pgfpathlineto{\pgfqpoint{4.796598in}{2.463759in}}%
\pgfpathlineto{\pgfqpoint{4.803765in}{2.470429in}}%
\pgfpathlineto{\pgfqpoint{4.810928in}{2.477095in}}%
\pgfpathlineto{\pgfqpoint{4.818084in}{2.483760in}}%
\pgfpathclose%
\pgfusepath{fill}%
\end{pgfscope}%
\begin{pgfscope}%
\pgfpathrectangle{\pgfqpoint{1.254980in}{0.150000in}}{\pgfqpoint{5.490039in}{5.490039in}}%
\pgfusepath{clip}%
\pgfsetbuttcap%
\pgfsetroundjoin%
\definecolor{currentfill}{rgb}{0.268510,0.009605,0.335427}%
\pgfsetfillcolor{currentfill}%
\pgfsetfillopacity{0.700000}%
\pgfsetlinewidth{0.000000pt}%
\definecolor{currentstroke}{rgb}{0.000000,0.000000,0.000000}%
\pgfsetstrokecolor{currentstroke}%
\pgfsetdash{}{0pt}%
\pgfpathmoveto{\pgfqpoint{3.829494in}{2.395211in}}%
\pgfpathlineto{\pgfqpoint{3.842319in}{2.391469in}}%
\pgfpathlineto{\pgfqpoint{3.855150in}{2.387758in}}%
\pgfpathlineto{\pgfqpoint{3.867986in}{2.384078in}}%
\pgfpathlineto{\pgfqpoint{3.880828in}{2.380429in}}%
\pgfpathlineto{\pgfqpoint{3.873314in}{2.373171in}}%
\pgfpathlineto{\pgfqpoint{3.865795in}{2.365914in}}%
\pgfpathlineto{\pgfqpoint{3.858270in}{2.358660in}}%
\pgfpathlineto{\pgfqpoint{3.850740in}{2.351408in}}%
\pgfpathlineto{\pgfqpoint{3.837886in}{2.355098in}}%
\pgfpathlineto{\pgfqpoint{3.825037in}{2.358818in}}%
\pgfpathlineto{\pgfqpoint{3.812194in}{2.362570in}}%
\pgfpathlineto{\pgfqpoint{3.799357in}{2.366352in}}%
\pgfpathlineto{\pgfqpoint{3.806899in}{2.373559in}}%
\pgfpathlineto{\pgfqpoint{3.814436in}{2.380771in}}%
\pgfpathlineto{\pgfqpoint{3.821968in}{2.387989in}}%
\pgfpathlineto{\pgfqpoint{3.829494in}{2.395211in}}%
\pgfpathclose%
\pgfusepath{fill}%
\end{pgfscope}%
\begin{pgfscope}%
\pgfpathrectangle{\pgfqpoint{1.254980in}{0.150000in}}{\pgfqpoint{5.490039in}{5.490039in}}%
\pgfusepath{clip}%
\pgfsetbuttcap%
\pgfsetroundjoin%
\definecolor{currentfill}{rgb}{0.277018,0.050344,0.375715}%
\pgfsetfillcolor{currentfill}%
\pgfsetfillopacity{0.700000}%
\pgfsetlinewidth{0.000000pt}%
\definecolor{currentstroke}{rgb}{0.000000,0.000000,0.000000}%
\pgfsetstrokecolor{currentstroke}%
\pgfsetdash{}{0pt}%
\pgfpathmoveto{\pgfqpoint{4.604161in}{2.459499in}}%
\pgfpathlineto{\pgfqpoint{4.617167in}{2.456923in}}%
\pgfpathlineto{\pgfqpoint{4.630180in}{2.454372in}}%
\pgfpathlineto{\pgfqpoint{4.643200in}{2.451847in}}%
\pgfpathlineto{\pgfqpoint{4.656226in}{2.449347in}}%
\pgfpathlineto{\pgfqpoint{4.648999in}{2.442498in}}%
\pgfpathlineto{\pgfqpoint{4.641767in}{2.435637in}}%
\pgfpathlineto{\pgfqpoint{4.634529in}{2.428760in}}%
\pgfpathlineto{\pgfqpoint{4.627286in}{2.421866in}}%
\pgfpathlineto{\pgfqpoint{4.614246in}{2.424317in}}%
\pgfpathlineto{\pgfqpoint{4.601214in}{2.426793in}}%
\pgfpathlineto{\pgfqpoint{4.588188in}{2.429295in}}%
\pgfpathlineto{\pgfqpoint{4.575168in}{2.431823in}}%
\pgfpathlineto{\pgfqpoint{4.582425in}{2.438761in}}%
\pgfpathlineto{\pgfqpoint{4.589675in}{2.445685in}}%
\pgfpathlineto{\pgfqpoint{4.596921in}{2.452597in}}%
\pgfpathlineto{\pgfqpoint{4.604161in}{2.459499in}}%
\pgfpathclose%
\pgfusepath{fill}%
\end{pgfscope}%
\begin{pgfscope}%
\pgfpathrectangle{\pgfqpoint{1.254980in}{0.150000in}}{\pgfqpoint{5.490039in}{5.490039in}}%
\pgfusepath{clip}%
\pgfsetbuttcap%
\pgfsetroundjoin%
\definecolor{currentfill}{rgb}{0.274952,0.037752,0.364543}%
\pgfsetfillcolor{currentfill}%
\pgfsetfillopacity{0.700000}%
\pgfsetlinewidth{0.000000pt}%
\definecolor{currentstroke}{rgb}{0.000000,0.000000,0.000000}%
\pgfsetstrokecolor{currentstroke}%
\pgfsetdash{}{0pt}%
\pgfpathmoveto{\pgfqpoint{4.390213in}{2.435854in}}%
\pgfpathlineto{\pgfqpoint{4.403167in}{2.433045in}}%
\pgfpathlineto{\pgfqpoint{4.416128in}{2.430263in}}%
\pgfpathlineto{\pgfqpoint{4.429095in}{2.427508in}}%
\pgfpathlineto{\pgfqpoint{4.442068in}{2.424779in}}%
\pgfpathlineto{\pgfqpoint{4.434759in}{2.417693in}}%
\pgfpathlineto{\pgfqpoint{4.427445in}{2.410590in}}%
\pgfpathlineto{\pgfqpoint{4.420126in}{2.403470in}}%
\pgfpathlineto{\pgfqpoint{4.412801in}{2.396331in}}%
\pgfpathlineto{\pgfqpoint{4.399815in}{2.399036in}}%
\pgfpathlineto{\pgfqpoint{4.386836in}{2.401768in}}%
\pgfpathlineto{\pgfqpoint{4.373863in}{2.404527in}}%
\pgfpathlineto{\pgfqpoint{4.360896in}{2.407313in}}%
\pgfpathlineto{\pgfqpoint{4.368233in}{2.414470in}}%
\pgfpathlineto{\pgfqpoint{4.375565in}{2.421612in}}%
\pgfpathlineto{\pgfqpoint{4.382892in}{2.428739in}}%
\pgfpathlineto{\pgfqpoint{4.390213in}{2.435854in}}%
\pgfpathclose%
\pgfusepath{fill}%
\end{pgfscope}%
\begin{pgfscope}%
\pgfpathrectangle{\pgfqpoint{1.254980in}{0.150000in}}{\pgfqpoint{5.490039in}{5.490039in}}%
\pgfusepath{clip}%
\pgfsetbuttcap%
\pgfsetroundjoin%
\definecolor{currentfill}{rgb}{0.269944,0.014625,0.341379}%
\pgfsetfillcolor{currentfill}%
\pgfsetfillopacity{0.700000}%
\pgfsetlinewidth{0.000000pt}%
\definecolor{currentstroke}{rgb}{0.000000,0.000000,0.000000}%
\pgfsetstrokecolor{currentstroke}%
\pgfsetdash{}{0pt}%
\pgfpathmoveto{\pgfqpoint{3.349937in}{2.405502in}}%
\pgfpathlineto{\pgfqpoint{3.362676in}{2.400570in}}%
\pgfpathlineto{\pgfqpoint{3.375418in}{2.395676in}}%
\pgfpathlineto{\pgfqpoint{3.388166in}{2.390818in}}%
\pgfpathlineto{\pgfqpoint{3.400918in}{2.385998in}}%
\pgfpathlineto{\pgfqpoint{3.393219in}{2.379483in}}%
\pgfpathlineto{\pgfqpoint{3.385513in}{2.373011in}}%
\pgfpathlineto{\pgfqpoint{3.377802in}{2.366584in}}%
\pgfpathlineto{\pgfqpoint{3.370083in}{2.360203in}}%
\pgfpathlineto{\pgfqpoint{3.357317in}{2.365115in}}%
\pgfpathlineto{\pgfqpoint{3.344555in}{2.370064in}}%
\pgfpathlineto{\pgfqpoint{3.331798in}{2.375050in}}%
\pgfpathlineto{\pgfqpoint{3.319045in}{2.380073in}}%
\pgfpathlineto{\pgfqpoint{3.326778in}{2.386358in}}%
\pgfpathlineto{\pgfqpoint{3.334504in}{2.392692in}}%
\pgfpathlineto{\pgfqpoint{3.342224in}{2.399074in}}%
\pgfpathlineto{\pgfqpoint{3.349937in}{2.405502in}}%
\pgfpathclose%
\pgfusepath{fill}%
\end{pgfscope}%
\begin{pgfscope}%
\pgfpathrectangle{\pgfqpoint{1.254980in}{0.150000in}}{\pgfqpoint{5.490039in}{5.490039in}}%
\pgfusepath{clip}%
\pgfsetbuttcap%
\pgfsetroundjoin%
\definecolor{currentfill}{rgb}{0.272594,0.025563,0.353093}%
\pgfsetfillcolor{currentfill}%
\pgfsetfillopacity{0.700000}%
\pgfsetlinewidth{0.000000pt}%
\definecolor{currentstroke}{rgb}{0.000000,0.000000,0.000000}%
\pgfsetstrokecolor{currentstroke}%
\pgfsetdash{}{0pt}%
\pgfpathmoveto{\pgfqpoint{3.217174in}{2.421638in}}%
\pgfpathlineto{\pgfqpoint{3.229894in}{2.416306in}}%
\pgfpathlineto{\pgfqpoint{3.242617in}{2.411013in}}%
\pgfpathlineto{\pgfqpoint{3.255344in}{2.405759in}}%
\pgfpathlineto{\pgfqpoint{3.268076in}{2.400545in}}%
\pgfpathlineto{\pgfqpoint{3.260321in}{2.394412in}}%
\pgfpathlineto{\pgfqpoint{3.252559in}{2.388337in}}%
\pgfpathlineto{\pgfqpoint{3.244790in}{2.382322in}}%
\pgfpathlineto{\pgfqpoint{3.237014in}{2.376368in}}%
\pgfpathlineto{\pgfqpoint{3.224267in}{2.381687in}}%
\pgfpathlineto{\pgfqpoint{3.211524in}{2.387045in}}%
\pgfpathlineto{\pgfqpoint{3.198785in}{2.392442in}}%
\pgfpathlineto{\pgfqpoint{3.186050in}{2.397879in}}%
\pgfpathlineto{\pgfqpoint{3.193842in}{2.403723in}}%
\pgfpathlineto{\pgfqpoint{3.201627in}{2.409633in}}%
\pgfpathlineto{\pgfqpoint{3.209404in}{2.415605in}}%
\pgfpathlineto{\pgfqpoint{3.217174in}{2.421638in}}%
\pgfpathclose%
\pgfusepath{fill}%
\end{pgfscope}%
\begin{pgfscope}%
\pgfpathrectangle{\pgfqpoint{1.254980in}{0.150000in}}{\pgfqpoint{5.490039in}{5.490039in}}%
\pgfusepath{clip}%
\pgfsetbuttcap%
\pgfsetroundjoin%
\definecolor{currentfill}{rgb}{0.271305,0.019942,0.347269}%
\pgfsetfillcolor{currentfill}%
\pgfsetfillopacity{0.700000}%
\pgfsetlinewidth{0.000000pt}%
\definecolor{currentstroke}{rgb}{0.000000,0.000000,0.000000}%
\pgfsetstrokecolor{currentstroke}%
\pgfsetdash{}{0pt}%
\pgfpathmoveto{\pgfqpoint{4.176235in}{2.413944in}}%
\pgfpathlineto{\pgfqpoint{4.189139in}{2.410838in}}%
\pgfpathlineto{\pgfqpoint{4.202050in}{2.407761in}}%
\pgfpathlineto{\pgfqpoint{4.214966in}{2.404711in}}%
\pgfpathlineto{\pgfqpoint{4.227889in}{2.401690in}}%
\pgfpathlineto{\pgfqpoint{4.220500in}{2.394436in}}%
\pgfpathlineto{\pgfqpoint{4.213106in}{2.387168in}}%
\pgfpathlineto{\pgfqpoint{4.205707in}{2.379886in}}%
\pgfpathlineto{\pgfqpoint{4.198302in}{2.372589in}}%
\pgfpathlineto{\pgfqpoint{4.185368in}{2.375612in}}%
\pgfpathlineto{\pgfqpoint{4.172439in}{2.378664in}}%
\pgfpathlineto{\pgfqpoint{4.159517in}{2.381744in}}%
\pgfpathlineto{\pgfqpoint{4.146600in}{2.384852in}}%
\pgfpathlineto{\pgfqpoint{4.154017in}{2.392142in}}%
\pgfpathlineto{\pgfqpoint{4.161428in}{2.399421in}}%
\pgfpathlineto{\pgfqpoint{4.168834in}{2.406688in}}%
\pgfpathlineto{\pgfqpoint{4.176235in}{2.413944in}}%
\pgfpathclose%
\pgfusepath{fill}%
\end{pgfscope}%
\begin{pgfscope}%
\pgfpathrectangle{\pgfqpoint{1.254980in}{0.150000in}}{\pgfqpoint{5.490039in}{5.490039in}}%
\pgfusepath{clip}%
\pgfsetbuttcap%
\pgfsetroundjoin%
\definecolor{currentfill}{rgb}{0.268510,0.009605,0.335427}%
\pgfsetfillcolor{currentfill}%
\pgfsetfillopacity{0.700000}%
\pgfsetlinewidth{0.000000pt}%
\definecolor{currentstroke}{rgb}{0.000000,0.000000,0.000000}%
\pgfsetstrokecolor{currentstroke}%
\pgfsetdash{}{0pt}%
\pgfpathmoveto{\pgfqpoint{3.482647in}{2.393843in}}%
\pgfpathlineto{\pgfqpoint{3.495408in}{2.389281in}}%
\pgfpathlineto{\pgfqpoint{3.508174in}{2.384753in}}%
\pgfpathlineto{\pgfqpoint{3.520945in}{2.380261in}}%
\pgfpathlineto{\pgfqpoint{3.533720in}{2.375803in}}%
\pgfpathlineto{\pgfqpoint{3.526074in}{2.368987in}}%
\pgfpathlineto{\pgfqpoint{3.518422in}{2.362200in}}%
\pgfpathlineto{\pgfqpoint{3.510764in}{2.355445in}}%
\pgfpathlineto{\pgfqpoint{3.503099in}{2.348722in}}%
\pgfpathlineto{\pgfqpoint{3.490310in}{2.353259in}}%
\pgfpathlineto{\pgfqpoint{3.477525in}{2.357830in}}%
\pgfpathlineto{\pgfqpoint{3.464746in}{2.362436in}}%
\pgfpathlineto{\pgfqpoint{3.451971in}{2.367077in}}%
\pgfpathlineto{\pgfqpoint{3.459649in}{2.373716in}}%
\pgfpathlineto{\pgfqpoint{3.467321in}{2.380391in}}%
\pgfpathlineto{\pgfqpoint{3.474987in}{2.387101in}}%
\pgfpathlineto{\pgfqpoint{3.482647in}{2.393843in}}%
\pgfpathclose%
\pgfusepath{fill}%
\end{pgfscope}%
\begin{pgfscope}%
\pgfpathrectangle{\pgfqpoint{1.254980in}{0.150000in}}{\pgfqpoint{5.490039in}{5.490039in}}%
\pgfusepath{clip}%
\pgfsetbuttcap%
\pgfsetroundjoin%
\definecolor{currentfill}{rgb}{0.283091,0.110553,0.431554}%
\pgfsetfillcolor{currentfill}%
\pgfsetfillopacity{0.700000}%
\pgfsetlinewidth{0.000000pt}%
\definecolor{currentstroke}{rgb}{0.000000,0.000000,0.000000}%
\pgfsetstrokecolor{currentstroke}%
\pgfsetdash{}{0pt}%
\pgfpathmoveto{\pgfqpoint{5.593025in}{2.561924in}}%
\pgfpathlineto{\pgfqpoint{5.606282in}{2.559696in}}%
\pgfpathlineto{\pgfqpoint{5.619547in}{2.557490in}}%
\pgfpathlineto{\pgfqpoint{5.632818in}{2.555308in}}%
\pgfpathlineto{\pgfqpoint{5.646098in}{2.553149in}}%
\pgfpathlineto{\pgfqpoint{5.639261in}{2.547105in}}%
\pgfpathlineto{\pgfqpoint{5.632422in}{2.541132in}}%
\pgfpathlineto{\pgfqpoint{5.625580in}{2.535224in}}%
\pgfpathlineto{\pgfqpoint{5.618734in}{2.529376in}}%
\pgfpathlineto{\pgfqpoint{5.605436in}{2.531372in}}%
\pgfpathlineto{\pgfqpoint{5.592145in}{2.533391in}}%
\pgfpathlineto{\pgfqpoint{5.578862in}{2.535433in}}%
\pgfpathlineto{\pgfqpoint{5.565586in}{2.537498in}}%
\pgfpathlineto{\pgfqpoint{5.572450in}{2.543505in}}%
\pgfpathlineto{\pgfqpoint{5.579312in}{2.549574in}}%
\pgfpathlineto{\pgfqpoint{5.586170in}{2.555712in}}%
\pgfpathlineto{\pgfqpoint{5.593025in}{2.561924in}}%
\pgfpathclose%
\pgfusepath{fill}%
\end{pgfscope}%
\begin{pgfscope}%
\pgfpathrectangle{\pgfqpoint{1.254980in}{0.150000in}}{\pgfqpoint{5.490039in}{5.490039in}}%
\pgfusepath{clip}%
\pgfsetbuttcap%
\pgfsetroundjoin%
\definecolor{currentfill}{rgb}{0.274952,0.037752,0.364543}%
\pgfsetfillcolor{currentfill}%
\pgfsetfillopacity{0.700000}%
\pgfsetlinewidth{0.000000pt}%
\definecolor{currentstroke}{rgb}{0.000000,0.000000,0.000000}%
\pgfsetstrokecolor{currentstroke}%
\pgfsetdash{}{0pt}%
\pgfpathmoveto{\pgfqpoint{3.084312in}{2.442850in}}%
\pgfpathlineto{\pgfqpoint{3.097016in}{2.437081in}}%
\pgfpathlineto{\pgfqpoint{3.109724in}{2.431356in}}%
\pgfpathlineto{\pgfqpoint{3.122435in}{2.425673in}}%
\pgfpathlineto{\pgfqpoint{3.135150in}{2.420031in}}%
\pgfpathlineto{\pgfqpoint{3.127335in}{2.414367in}}%
\pgfpathlineto{\pgfqpoint{3.119512in}{2.408777in}}%
\pgfpathlineto{\pgfqpoint{3.111681in}{2.403262in}}%
\pgfpathlineto{\pgfqpoint{3.103842in}{2.397827in}}%
\pgfpathlineto{\pgfqpoint{3.091110in}{2.403585in}}%
\pgfpathlineto{\pgfqpoint{3.078382in}{2.409386in}}%
\pgfpathlineto{\pgfqpoint{3.065658in}{2.415228in}}%
\pgfpathlineto{\pgfqpoint{3.052937in}{2.421114in}}%
\pgfpathlineto{\pgfqpoint{3.060793in}{2.426427in}}%
\pgfpathlineto{\pgfqpoint{3.068640in}{2.431823in}}%
\pgfpathlineto{\pgfqpoint{3.076480in}{2.437298in}}%
\pgfpathlineto{\pgfqpoint{3.084312in}{2.442850in}}%
\pgfpathclose%
\pgfusepath{fill}%
\end{pgfscope}%
\begin{pgfscope}%
\pgfpathrectangle{\pgfqpoint{1.254980in}{0.150000in}}{\pgfqpoint{5.490039in}{5.490039in}}%
\pgfusepath{clip}%
\pgfsetbuttcap%
\pgfsetroundjoin%
\definecolor{currentfill}{rgb}{0.282656,0.100196,0.422160}%
\pgfsetfillcolor{currentfill}%
\pgfsetfillopacity{0.700000}%
\pgfsetlinewidth{0.000000pt}%
\definecolor{currentstroke}{rgb}{0.000000,0.000000,0.000000}%
\pgfsetstrokecolor{currentstroke}%
\pgfsetdash{}{0pt}%
\pgfpathmoveto{\pgfqpoint{5.379134in}{2.538797in}}%
\pgfpathlineto{\pgfqpoint{5.392340in}{2.536603in}}%
\pgfpathlineto{\pgfqpoint{5.405554in}{2.534432in}}%
\pgfpathlineto{\pgfqpoint{5.418774in}{2.532285in}}%
\pgfpathlineto{\pgfqpoint{5.432002in}{2.530161in}}%
\pgfpathlineto{\pgfqpoint{5.425083in}{2.524114in}}%
\pgfpathlineto{\pgfqpoint{5.418159in}{2.518108in}}%
\pgfpathlineto{\pgfqpoint{5.411231in}{2.512140in}}%
\pgfpathlineto{\pgfqpoint{5.404299in}{2.506204in}}%
\pgfpathlineto{\pgfqpoint{5.391053in}{2.508190in}}%
\pgfpathlineto{\pgfqpoint{5.377815in}{2.510199in}}%
\pgfpathlineto{\pgfqpoint{5.364584in}{2.512232in}}%
\pgfpathlineto{\pgfqpoint{5.351361in}{2.514288in}}%
\pgfpathlineto{\pgfqpoint{5.358310in}{2.520357in}}%
\pgfpathlineto{\pgfqpoint{5.365256in}{2.526462in}}%
\pgfpathlineto{\pgfqpoint{5.372197in}{2.532607in}}%
\pgfpathlineto{\pgfqpoint{5.379134in}{2.538797in}}%
\pgfpathclose%
\pgfusepath{fill}%
\end{pgfscope}%
\begin{pgfscope}%
\pgfpathrectangle{\pgfqpoint{1.254980in}{0.150000in}}{\pgfqpoint{5.490039in}{5.490039in}}%
\pgfusepath{clip}%
\pgfsetbuttcap%
\pgfsetroundjoin%
\definecolor{currentfill}{rgb}{0.269944,0.014625,0.341379}%
\pgfsetfillcolor{currentfill}%
\pgfsetfillopacity{0.700000}%
\pgfsetlinewidth{0.000000pt}%
\definecolor{currentstroke}{rgb}{0.000000,0.000000,0.000000}%
\pgfsetstrokecolor{currentstroke}%
\pgfsetdash{}{0pt}%
\pgfpathmoveto{\pgfqpoint{3.962202in}{2.395285in}}%
\pgfpathlineto{\pgfqpoint{3.975060in}{2.391815in}}%
\pgfpathlineto{\pgfqpoint{3.987923in}{2.388374in}}%
\pgfpathlineto{\pgfqpoint{4.000793in}{2.384962in}}%
\pgfpathlineto{\pgfqpoint{4.013668in}{2.381580in}}%
\pgfpathlineto{\pgfqpoint{4.006200in}{2.374274in}}%
\pgfpathlineto{\pgfqpoint{3.998727in}{2.366961in}}%
\pgfpathlineto{\pgfqpoint{3.991248in}{2.359643in}}%
\pgfpathlineto{\pgfqpoint{3.983764in}{2.352319in}}%
\pgfpathlineto{\pgfqpoint{3.970877in}{2.355728in}}%
\pgfpathlineto{\pgfqpoint{3.957996in}{2.359167in}}%
\pgfpathlineto{\pgfqpoint{3.945120in}{2.362636in}}%
\pgfpathlineto{\pgfqpoint{3.932250in}{2.366135in}}%
\pgfpathlineto{\pgfqpoint{3.939747in}{2.373426in}}%
\pgfpathlineto{\pgfqpoint{3.947237in}{2.380715in}}%
\pgfpathlineto{\pgfqpoint{3.954722in}{2.388002in}}%
\pgfpathlineto{\pgfqpoint{3.962202in}{2.395285in}}%
\pgfpathclose%
\pgfusepath{fill}%
\end{pgfscope}%
\begin{pgfscope}%
\pgfpathrectangle{\pgfqpoint{1.254980in}{0.150000in}}{\pgfqpoint{5.490039in}{5.490039in}}%
\pgfusepath{clip}%
\pgfsetbuttcap%
\pgfsetroundjoin%
\definecolor{currentfill}{rgb}{0.268510,0.009605,0.335427}%
\pgfsetfillcolor{currentfill}%
\pgfsetfillopacity{0.700000}%
\pgfsetlinewidth{0.000000pt}%
\definecolor{currentstroke}{rgb}{0.000000,0.000000,0.000000}%
\pgfsetstrokecolor{currentstroke}%
\pgfsetdash{}{0pt}%
\pgfpathmoveto{\pgfqpoint{3.615343in}{2.386110in}}%
\pgfpathlineto{\pgfqpoint{3.628130in}{2.381888in}}%
\pgfpathlineto{\pgfqpoint{3.640923in}{2.377699in}}%
\pgfpathlineto{\pgfqpoint{3.653720in}{2.373544in}}%
\pgfpathlineto{\pgfqpoint{3.666522in}{2.369421in}}%
\pgfpathlineto{\pgfqpoint{3.658926in}{2.362378in}}%
\pgfpathlineto{\pgfqpoint{3.651324in}{2.355353in}}%
\pgfpathlineto{\pgfqpoint{3.643717in}{2.348347in}}%
\pgfpathlineto{\pgfqpoint{3.636103in}{2.341362in}}%
\pgfpathlineto{\pgfqpoint{3.623288in}{2.345550in}}%
\pgfpathlineto{\pgfqpoint{3.610477in}{2.349772in}}%
\pgfpathlineto{\pgfqpoint{3.597672in}{2.354026in}}%
\pgfpathlineto{\pgfqpoint{3.584872in}{2.358314in}}%
\pgfpathlineto{\pgfqpoint{3.592498in}{2.365229in}}%
\pgfpathlineto{\pgfqpoint{3.600119in}{2.372168in}}%
\pgfpathlineto{\pgfqpoint{3.607734in}{2.379128in}}%
\pgfpathlineto{\pgfqpoint{3.615343in}{2.386110in}}%
\pgfpathclose%
\pgfusepath{fill}%
\end{pgfscope}%
\begin{pgfscope}%
\pgfpathrectangle{\pgfqpoint{1.254980in}{0.150000in}}{\pgfqpoint{5.490039in}{5.490039in}}%
\pgfusepath{clip}%
\pgfsetbuttcap%
\pgfsetroundjoin%
\definecolor{currentfill}{rgb}{0.281924,0.089666,0.412415}%
\pgfsetfillcolor{currentfill}%
\pgfsetfillopacity{0.700000}%
\pgfsetlinewidth{0.000000pt}%
\definecolor{currentstroke}{rgb}{0.000000,0.000000,0.000000}%
\pgfsetstrokecolor{currentstroke}%
\pgfsetdash{}{0pt}%
\pgfpathmoveto{\pgfqpoint{5.165197in}{2.515306in}}%
\pgfpathlineto{\pgfqpoint{5.178351in}{2.513090in}}%
\pgfpathlineto{\pgfqpoint{5.191512in}{2.510898in}}%
\pgfpathlineto{\pgfqpoint{5.204680in}{2.508730in}}%
\pgfpathlineto{\pgfqpoint{5.217855in}{2.506586in}}%
\pgfpathlineto{\pgfqpoint{5.210849in}{2.500400in}}%
\pgfpathlineto{\pgfqpoint{5.203838in}{2.494232in}}%
\pgfpathlineto{\pgfqpoint{5.196822in}{2.488080in}}%
\pgfpathlineto{\pgfqpoint{5.189801in}{2.481938in}}%
\pgfpathlineto{\pgfqpoint{5.176610in}{2.483969in}}%
\pgfpathlineto{\pgfqpoint{5.163426in}{2.486025in}}%
\pgfpathlineto{\pgfqpoint{5.150249in}{2.488104in}}%
\pgfpathlineto{\pgfqpoint{5.137080in}{2.490208in}}%
\pgfpathlineto{\pgfqpoint{5.144117in}{2.496457in}}%
\pgfpathlineto{\pgfqpoint{5.151148in}{2.502721in}}%
\pgfpathlineto{\pgfqpoint{5.158175in}{2.509003in}}%
\pgfpathlineto{\pgfqpoint{5.165197in}{2.515306in}}%
\pgfpathclose%
\pgfusepath{fill}%
\end{pgfscope}%
\begin{pgfscope}%
\pgfpathrectangle{\pgfqpoint{1.254980in}{0.150000in}}{\pgfqpoint{5.490039in}{5.490039in}}%
\pgfusepath{clip}%
\pgfsetbuttcap%
\pgfsetroundjoin%
\definecolor{currentfill}{rgb}{0.280267,0.073417,0.397163}%
\pgfsetfillcolor{currentfill}%
\pgfsetfillopacity{0.700000}%
\pgfsetlinewidth{0.000000pt}%
\definecolor{currentstroke}{rgb}{0.000000,0.000000,0.000000}%
\pgfsetstrokecolor{currentstroke}%
\pgfsetdash{}{0pt}%
\pgfpathmoveto{\pgfqpoint{4.951216in}{2.491213in}}%
\pgfpathlineto{\pgfqpoint{4.964316in}{2.488918in}}%
\pgfpathlineto{\pgfqpoint{4.977423in}{2.486647in}}%
\pgfpathlineto{\pgfqpoint{4.990537in}{2.484401in}}%
\pgfpathlineto{\pgfqpoint{5.003658in}{2.482179in}}%
\pgfpathlineto{\pgfqpoint{4.996565in}{2.475765in}}%
\pgfpathlineto{\pgfqpoint{4.989467in}{2.469353in}}%
\pgfpathlineto{\pgfqpoint{4.982364in}{2.462938in}}%
\pgfpathlineto{\pgfqpoint{4.975255in}{2.456518in}}%
\pgfpathlineto{\pgfqpoint{4.962119in}{2.458652in}}%
\pgfpathlineto{\pgfqpoint{4.948990in}{2.460811in}}%
\pgfpathlineto{\pgfqpoint{4.935869in}{2.462995in}}%
\pgfpathlineto{\pgfqpoint{4.922754in}{2.465203in}}%
\pgfpathlineto{\pgfqpoint{4.929877in}{2.471705in}}%
\pgfpathlineto{\pgfqpoint{4.936995in}{2.478206in}}%
\pgfpathlineto{\pgfqpoint{4.944108in}{2.484707in}}%
\pgfpathlineto{\pgfqpoint{4.951216in}{2.491213in}}%
\pgfpathclose%
\pgfusepath{fill}%
\end{pgfscope}%
\begin{pgfscope}%
\pgfpathrectangle{\pgfqpoint{1.254980in}{0.150000in}}{\pgfqpoint{5.490039in}{5.490039in}}%
\pgfusepath{clip}%
\pgfsetbuttcap%
\pgfsetroundjoin%
\definecolor{currentfill}{rgb}{0.278791,0.062145,0.386592}%
\pgfsetfillcolor{currentfill}%
\pgfsetfillopacity{0.700000}%
\pgfsetlinewidth{0.000000pt}%
\definecolor{currentstroke}{rgb}{0.000000,0.000000,0.000000}%
\pgfsetstrokecolor{currentstroke}%
\pgfsetdash{}{0pt}%
\pgfpathmoveto{\pgfqpoint{2.951296in}{2.469784in}}%
\pgfpathlineto{\pgfqpoint{2.963989in}{2.463542in}}%
\pgfpathlineto{\pgfqpoint{2.976686in}{2.457346in}}%
\pgfpathlineto{\pgfqpoint{2.989386in}{2.451196in}}%
\pgfpathlineto{\pgfqpoint{3.002089in}{2.445091in}}%
\pgfpathlineto{\pgfqpoint{2.994208in}{2.439988in}}%
\pgfpathlineto{\pgfqpoint{2.986318in}{2.434976in}}%
\pgfpathlineto{\pgfqpoint{2.978420in}{2.430058in}}%
\pgfpathlineto{\pgfqpoint{2.970513in}{2.425237in}}%
\pgfpathlineto{\pgfqpoint{2.957791in}{2.431473in}}%
\pgfpathlineto{\pgfqpoint{2.945073in}{2.437753in}}%
\pgfpathlineto{\pgfqpoint{2.932359in}{2.444080in}}%
\pgfpathlineto{\pgfqpoint{2.919647in}{2.450452in}}%
\pgfpathlineto{\pgfqpoint{2.927572in}{2.455138in}}%
\pgfpathlineto{\pgfqpoint{2.935489in}{2.459924in}}%
\pgfpathlineto{\pgfqpoint{2.943397in}{2.464807in}}%
\pgfpathlineto{\pgfqpoint{2.951296in}{2.469784in}}%
\pgfpathclose%
\pgfusepath{fill}%
\end{pgfscope}%
\begin{pgfscope}%
\pgfpathrectangle{\pgfqpoint{1.254980in}{0.150000in}}{\pgfqpoint{5.490039in}{5.490039in}}%
\pgfusepath{clip}%
\pgfsetbuttcap%
\pgfsetroundjoin%
\definecolor{currentfill}{rgb}{0.278791,0.062145,0.386592}%
\pgfsetfillcolor{currentfill}%
\pgfsetfillopacity{0.700000}%
\pgfsetlinewidth{0.000000pt}%
\definecolor{currentstroke}{rgb}{0.000000,0.000000,0.000000}%
\pgfsetstrokecolor{currentstroke}%
\pgfsetdash{}{0pt}%
\pgfpathmoveto{\pgfqpoint{4.737199in}{2.466665in}}%
\pgfpathlineto{\pgfqpoint{4.750245in}{2.464231in}}%
\pgfpathlineto{\pgfqpoint{4.763298in}{2.461823in}}%
\pgfpathlineto{\pgfqpoint{4.776358in}{2.459440in}}%
\pgfpathlineto{\pgfqpoint{4.789425in}{2.457083in}}%
\pgfpathlineto{\pgfqpoint{4.782246in}{2.450398in}}%
\pgfpathlineto{\pgfqpoint{4.775062in}{2.443703in}}%
\pgfpathlineto{\pgfqpoint{4.767873in}{2.436994in}}%
\pgfpathlineto{\pgfqpoint{4.760678in}{2.430270in}}%
\pgfpathlineto{\pgfqpoint{4.747598in}{2.432566in}}%
\pgfpathlineto{\pgfqpoint{4.734525in}{2.434887in}}%
\pgfpathlineto{\pgfqpoint{4.721458in}{2.437234in}}%
\pgfpathlineto{\pgfqpoint{4.708398in}{2.439605in}}%
\pgfpathlineto{\pgfqpoint{4.715606in}{2.446386in}}%
\pgfpathlineto{\pgfqpoint{4.722809in}{2.453155in}}%
\pgfpathlineto{\pgfqpoint{4.730007in}{2.459914in}}%
\pgfpathlineto{\pgfqpoint{4.737199in}{2.466665in}}%
\pgfpathclose%
\pgfusepath{fill}%
\end{pgfscope}%
\begin{pgfscope}%
\pgfpathrectangle{\pgfqpoint{1.254980in}{0.150000in}}{\pgfqpoint{5.490039in}{5.490039in}}%
\pgfusepath{clip}%
\pgfsetbuttcap%
\pgfsetroundjoin%
\definecolor{currentfill}{rgb}{0.276022,0.044167,0.370164}%
\pgfsetfillcolor{currentfill}%
\pgfsetfillopacity{0.700000}%
\pgfsetlinewidth{0.000000pt}%
\definecolor{currentstroke}{rgb}{0.000000,0.000000,0.000000}%
\pgfsetstrokecolor{currentstroke}%
\pgfsetdash{}{0pt}%
\pgfpathmoveto{\pgfqpoint{4.523157in}{2.442195in}}%
\pgfpathlineto{\pgfqpoint{4.536150in}{2.439563in}}%
\pgfpathlineto{\pgfqpoint{4.549149in}{2.436957in}}%
\pgfpathlineto{\pgfqpoint{4.562156in}{2.434377in}}%
\pgfpathlineto{\pgfqpoint{4.575168in}{2.431823in}}%
\pgfpathlineto{\pgfqpoint{4.567907in}{2.424870in}}%
\pgfpathlineto{\pgfqpoint{4.560639in}{2.417900in}}%
\pgfpathlineto{\pgfqpoint{4.553367in}{2.410911in}}%
\pgfpathlineto{\pgfqpoint{4.546088in}{2.403903in}}%
\pgfpathlineto{\pgfqpoint{4.533063in}{2.406420in}}%
\pgfpathlineto{\pgfqpoint{4.520044in}{2.408964in}}%
\pgfpathlineto{\pgfqpoint{4.507032in}{2.411534in}}%
\pgfpathlineto{\pgfqpoint{4.494026in}{2.414131in}}%
\pgfpathlineto{\pgfqpoint{4.501317in}{2.421171in}}%
\pgfpathlineto{\pgfqpoint{4.508602in}{2.428194in}}%
\pgfpathlineto{\pgfqpoint{4.515882in}{2.435201in}}%
\pgfpathlineto{\pgfqpoint{4.523157in}{2.442195in}}%
\pgfpathclose%
\pgfusepath{fill}%
\end{pgfscope}%
\begin{pgfscope}%
\pgfpathrectangle{\pgfqpoint{1.254980in}{0.150000in}}{\pgfqpoint{5.490039in}{5.490039in}}%
\pgfusepath{clip}%
\pgfsetbuttcap%
\pgfsetroundjoin%
\definecolor{currentfill}{rgb}{0.273809,0.031497,0.358853}%
\pgfsetfillcolor{currentfill}%
\pgfsetfillopacity{0.700000}%
\pgfsetlinewidth{0.000000pt}%
\definecolor{currentstroke}{rgb}{0.000000,0.000000,0.000000}%
\pgfsetstrokecolor{currentstroke}%
\pgfsetdash{}{0pt}%
\pgfpathmoveto{\pgfqpoint{4.309093in}{2.418726in}}%
\pgfpathlineto{\pgfqpoint{4.322034in}{2.415832in}}%
\pgfpathlineto{\pgfqpoint{4.334982in}{2.412965in}}%
\pgfpathlineto{\pgfqpoint{4.347936in}{2.410125in}}%
\pgfpathlineto{\pgfqpoint{4.360896in}{2.407313in}}%
\pgfpathlineto{\pgfqpoint{4.353553in}{2.400139in}}%
\pgfpathlineto{\pgfqpoint{4.346205in}{2.392947in}}%
\pgfpathlineto{\pgfqpoint{4.338852in}{2.385738in}}%
\pgfpathlineto{\pgfqpoint{4.331492in}{2.378509in}}%
\pgfpathlineto{\pgfqpoint{4.318520in}{2.381311in}}%
\pgfpathlineto{\pgfqpoint{4.305554in}{2.384140in}}%
\pgfpathlineto{\pgfqpoint{4.292594in}{2.386996in}}%
\pgfpathlineto{\pgfqpoint{4.279641in}{2.389880in}}%
\pgfpathlineto{\pgfqpoint{4.287012in}{2.397115in}}%
\pgfpathlineto{\pgfqpoint{4.294378in}{2.404333in}}%
\pgfpathlineto{\pgfqpoint{4.301738in}{2.411537in}}%
\pgfpathlineto{\pgfqpoint{4.309093in}{2.418726in}}%
\pgfpathclose%
\pgfusepath{fill}%
\end{pgfscope}%
\begin{pgfscope}%
\pgfpathrectangle{\pgfqpoint{1.254980in}{0.150000in}}{\pgfqpoint{5.490039in}{5.490039in}}%
\pgfusepath{clip}%
\pgfsetbuttcap%
\pgfsetroundjoin%
\definecolor{currentfill}{rgb}{0.268510,0.009605,0.335427}%
\pgfsetfillcolor{currentfill}%
\pgfsetfillopacity{0.700000}%
\pgfsetlinewidth{0.000000pt}%
\definecolor{currentstroke}{rgb}{0.000000,0.000000,0.000000}%
\pgfsetstrokecolor{currentstroke}%
\pgfsetdash{}{0pt}%
\pgfpathmoveto{\pgfqpoint{3.748061in}{2.381795in}}%
\pgfpathlineto{\pgfqpoint{3.760877in}{2.377887in}}%
\pgfpathlineto{\pgfqpoint{3.773698in}{2.374010in}}%
\pgfpathlineto{\pgfqpoint{3.786525in}{2.370165in}}%
\pgfpathlineto{\pgfqpoint{3.799357in}{2.366352in}}%
\pgfpathlineto{\pgfqpoint{3.791809in}{2.359152in}}%
\pgfpathlineto{\pgfqpoint{3.784255in}{2.351959in}}%
\pgfpathlineto{\pgfqpoint{3.776695in}{2.344775in}}%
\pgfpathlineto{\pgfqpoint{3.769130in}{2.337599in}}%
\pgfpathlineto{\pgfqpoint{3.756286in}{2.341466in}}%
\pgfpathlineto{\pgfqpoint{3.743447in}{2.345364in}}%
\pgfpathlineto{\pgfqpoint{3.730613in}{2.349294in}}%
\pgfpathlineto{\pgfqpoint{3.717784in}{2.353255in}}%
\pgfpathlineto{\pgfqpoint{3.725362in}{2.360372in}}%
\pgfpathlineto{\pgfqpoint{3.732934in}{2.367502in}}%
\pgfpathlineto{\pgfqpoint{3.740500in}{2.374643in}}%
\pgfpathlineto{\pgfqpoint{3.748061in}{2.381795in}}%
\pgfpathclose%
\pgfusepath{fill}%
\end{pgfscope}%
\begin{pgfscope}%
\pgfpathrectangle{\pgfqpoint{1.254980in}{0.150000in}}{\pgfqpoint{5.490039in}{5.490039in}}%
\pgfusepath{clip}%
\pgfsetbuttcap%
\pgfsetroundjoin%
\definecolor{currentfill}{rgb}{0.271305,0.019942,0.347269}%
\pgfsetfillcolor{currentfill}%
\pgfsetfillopacity{0.700000}%
\pgfsetlinewidth{0.000000pt}%
\definecolor{currentstroke}{rgb}{0.000000,0.000000,0.000000}%
\pgfsetstrokecolor{currentstroke}%
\pgfsetdash{}{0pt}%
\pgfpathmoveto{\pgfqpoint{4.094995in}{2.397568in}}%
\pgfpathlineto{\pgfqpoint{4.107887in}{2.394346in}}%
\pgfpathlineto{\pgfqpoint{4.120786in}{2.391153in}}%
\pgfpathlineto{\pgfqpoint{4.133690in}{2.387988in}}%
\pgfpathlineto{\pgfqpoint{4.146600in}{2.384852in}}%
\pgfpathlineto{\pgfqpoint{4.139178in}{2.377550in}}%
\pgfpathlineto{\pgfqpoint{4.131751in}{2.370235in}}%
\pgfpathlineto{\pgfqpoint{4.124318in}{2.362908in}}%
\pgfpathlineto{\pgfqpoint{4.116879in}{2.355568in}}%
\pgfpathlineto{\pgfqpoint{4.103957in}{2.358719in}}%
\pgfpathlineto{\pgfqpoint{4.091041in}{2.361899in}}%
\pgfpathlineto{\pgfqpoint{4.078130in}{2.365107in}}%
\pgfpathlineto{\pgfqpoint{4.065226in}{2.368344in}}%
\pgfpathlineto{\pgfqpoint{4.072677in}{2.375664in}}%
\pgfpathlineto{\pgfqpoint{4.080121in}{2.382975in}}%
\pgfpathlineto{\pgfqpoint{4.087561in}{2.390276in}}%
\pgfpathlineto{\pgfqpoint{4.094995in}{2.397568in}}%
\pgfpathclose%
\pgfusepath{fill}%
\end{pgfscope}%
\begin{pgfscope}%
\pgfpathrectangle{\pgfqpoint{1.254980in}{0.150000in}}{\pgfqpoint{5.490039in}{5.490039in}}%
\pgfusepath{clip}%
\pgfsetbuttcap%
\pgfsetroundjoin%
\definecolor{currentfill}{rgb}{0.283091,0.110553,0.431554}%
\pgfsetfillcolor{currentfill}%
\pgfsetfillopacity{0.700000}%
\pgfsetlinewidth{0.000000pt}%
\definecolor{currentstroke}{rgb}{0.000000,0.000000,0.000000}%
\pgfsetstrokecolor{currentstroke}%
\pgfsetdash{}{0pt}%
\pgfpathmoveto{\pgfqpoint{5.512556in}{2.545990in}}%
\pgfpathlineto{\pgfqpoint{5.525802in}{2.543833in}}%
\pgfpathlineto{\pgfqpoint{5.539056in}{2.541698in}}%
\pgfpathlineto{\pgfqpoint{5.552317in}{2.539587in}}%
\pgfpathlineto{\pgfqpoint{5.565586in}{2.537498in}}%
\pgfpathlineto{\pgfqpoint{5.558718in}{2.531550in}}%
\pgfpathlineto{\pgfqpoint{5.551846in}{2.525656in}}%
\pgfpathlineto{\pgfqpoint{5.544970in}{2.519811in}}%
\pgfpathlineto{\pgfqpoint{5.538091in}{2.514011in}}%
\pgfpathlineto{\pgfqpoint{5.524804in}{2.515948in}}%
\pgfpathlineto{\pgfqpoint{5.511524in}{2.517909in}}%
\pgfpathlineto{\pgfqpoint{5.498252in}{2.519892in}}%
\pgfpathlineto{\pgfqpoint{5.484988in}{2.521899in}}%
\pgfpathlineto{\pgfqpoint{5.491885in}{2.527846in}}%
\pgfpathlineto{\pgfqpoint{5.498779in}{2.533841in}}%
\pgfpathlineto{\pgfqpoint{5.505669in}{2.539887in}}%
\pgfpathlineto{\pgfqpoint{5.512556in}{2.545990in}}%
\pgfpathclose%
\pgfusepath{fill}%
\end{pgfscope}%
\begin{pgfscope}%
\pgfpathrectangle{\pgfqpoint{1.254980in}{0.150000in}}{\pgfqpoint{5.490039in}{5.490039in}}%
\pgfusepath{clip}%
\pgfsetbuttcap%
\pgfsetroundjoin%
\definecolor{currentfill}{rgb}{0.281446,0.084320,0.407414}%
\pgfsetfillcolor{currentfill}%
\pgfsetfillopacity{0.700000}%
\pgfsetlinewidth{0.000000pt}%
\definecolor{currentstroke}{rgb}{0.000000,0.000000,0.000000}%
\pgfsetstrokecolor{currentstroke}%
\pgfsetdash{}{0pt}%
\pgfpathmoveto{\pgfqpoint{2.818064in}{2.503148in}}%
\pgfpathlineto{\pgfqpoint{2.830751in}{2.496390in}}%
\pgfpathlineto{\pgfqpoint{2.843442in}{2.489681in}}%
\pgfpathlineto{\pgfqpoint{2.856135in}{2.483022in}}%
\pgfpathlineto{\pgfqpoint{2.868832in}{2.476412in}}%
\pgfpathlineto{\pgfqpoint{2.860878in}{2.471969in}}%
\pgfpathlineto{\pgfqpoint{2.852916in}{2.467636in}}%
\pgfpathlineto{\pgfqpoint{2.844943in}{2.463416in}}%
\pgfpathlineto{\pgfqpoint{2.836961in}{2.459314in}}%
\pgfpathlineto{\pgfqpoint{2.824246in}{2.466067in}}%
\pgfpathlineto{\pgfqpoint{2.811533in}{2.472870in}}%
\pgfpathlineto{\pgfqpoint{2.798822in}{2.479722in}}%
\pgfpathlineto{\pgfqpoint{2.786115in}{2.486624in}}%
\pgfpathlineto{\pgfqpoint{2.794117in}{2.490578in}}%
\pgfpathlineto{\pgfqpoint{2.802109in}{2.494652in}}%
\pgfpathlineto{\pgfqpoint{2.810091in}{2.498844in}}%
\pgfpathlineto{\pgfqpoint{2.818064in}{2.503148in}}%
\pgfpathclose%
\pgfusepath{fill}%
\end{pgfscope}%
\begin{pgfscope}%
\pgfpathrectangle{\pgfqpoint{1.254980in}{0.150000in}}{\pgfqpoint{5.490039in}{5.490039in}}%
\pgfusepath{clip}%
\pgfsetbuttcap%
\pgfsetroundjoin%
\definecolor{currentfill}{rgb}{0.282327,0.094955,0.417331}%
\pgfsetfillcolor{currentfill}%
\pgfsetfillopacity{0.700000}%
\pgfsetlinewidth{0.000000pt}%
\definecolor{currentstroke}{rgb}{0.000000,0.000000,0.000000}%
\pgfsetstrokecolor{currentstroke}%
\pgfsetdash{}{0pt}%
\pgfpathmoveto{\pgfqpoint{5.298539in}{2.522749in}}%
\pgfpathlineto{\pgfqpoint{5.311733in}{2.520598in}}%
\pgfpathlineto{\pgfqpoint{5.324935in}{2.518471in}}%
\pgfpathlineto{\pgfqpoint{5.338144in}{2.516368in}}%
\pgfpathlineto{\pgfqpoint{5.351361in}{2.514288in}}%
\pgfpathlineto{\pgfqpoint{5.344406in}{2.508251in}}%
\pgfpathlineto{\pgfqpoint{5.337448in}{2.502241in}}%
\pgfpathlineto{\pgfqpoint{5.330484in}{2.496255in}}%
\pgfpathlineto{\pgfqpoint{5.323516in}{2.490289in}}%
\pgfpathlineto{\pgfqpoint{5.310283in}{2.492243in}}%
\pgfpathlineto{\pgfqpoint{5.297057in}{2.494221in}}%
\pgfpathlineto{\pgfqpoint{5.283839in}{2.496222in}}%
\pgfpathlineto{\pgfqpoint{5.270627in}{2.498247in}}%
\pgfpathlineto{\pgfqpoint{5.277612in}{2.504335in}}%
\pgfpathlineto{\pgfqpoint{5.284592in}{2.510445in}}%
\pgfpathlineto{\pgfqpoint{5.291568in}{2.516581in}}%
\pgfpathlineto{\pgfqpoint{5.298539in}{2.522749in}}%
\pgfpathclose%
\pgfusepath{fill}%
\end{pgfscope}%
\begin{pgfscope}%
\pgfpathrectangle{\pgfqpoint{1.254980in}{0.150000in}}{\pgfqpoint{5.490039in}{5.490039in}}%
\pgfusepath{clip}%
\pgfsetbuttcap%
\pgfsetroundjoin%
\definecolor{currentfill}{rgb}{0.271305,0.019942,0.347269}%
\pgfsetfillcolor{currentfill}%
\pgfsetfillopacity{0.700000}%
\pgfsetlinewidth{0.000000pt}%
\definecolor{currentstroke}{rgb}{0.000000,0.000000,0.000000}%
\pgfsetstrokecolor{currentstroke}%
\pgfsetdash{}{0pt}%
\pgfpathmoveto{\pgfqpoint{3.268076in}{2.400545in}}%
\pgfpathlineto{\pgfqpoint{3.280812in}{2.395370in}}%
\pgfpathlineto{\pgfqpoint{3.293552in}{2.390233in}}%
\pgfpathlineto{\pgfqpoint{3.306296in}{2.385134in}}%
\pgfpathlineto{\pgfqpoint{3.319045in}{2.380073in}}%
\pgfpathlineto{\pgfqpoint{3.311305in}{2.373841in}}%
\pgfpathlineto{\pgfqpoint{3.303558in}{2.367663in}}%
\pgfpathlineto{\pgfqpoint{3.295804in}{2.361542in}}%
\pgfpathlineto{\pgfqpoint{3.288043in}{2.355480in}}%
\pgfpathlineto{\pgfqpoint{3.275280in}{2.360645in}}%
\pgfpathlineto{\pgfqpoint{3.262520in}{2.365848in}}%
\pgfpathlineto{\pgfqpoint{3.249765in}{2.371089in}}%
\pgfpathlineto{\pgfqpoint{3.237014in}{2.376368in}}%
\pgfpathlineto{\pgfqpoint{3.244790in}{2.382322in}}%
\pgfpathlineto{\pgfqpoint{3.252559in}{2.388337in}}%
\pgfpathlineto{\pgfqpoint{3.260321in}{2.394412in}}%
\pgfpathlineto{\pgfqpoint{3.268076in}{2.400545in}}%
\pgfpathclose%
\pgfusepath{fill}%
\end{pgfscope}%
\begin{pgfscope}%
\pgfpathrectangle{\pgfqpoint{1.254980in}{0.150000in}}{\pgfqpoint{5.490039in}{5.490039in}}%
\pgfusepath{clip}%
\pgfsetbuttcap%
\pgfsetroundjoin%
\definecolor{currentfill}{rgb}{0.268510,0.009605,0.335427}%
\pgfsetfillcolor{currentfill}%
\pgfsetfillopacity{0.700000}%
\pgfsetlinewidth{0.000000pt}%
\definecolor{currentstroke}{rgb}{0.000000,0.000000,0.000000}%
\pgfsetstrokecolor{currentstroke}%
\pgfsetdash{}{0pt}%
\pgfpathmoveto{\pgfqpoint{3.880828in}{2.380429in}}%
\pgfpathlineto{\pgfqpoint{3.893675in}{2.376810in}}%
\pgfpathlineto{\pgfqpoint{3.906528in}{2.373221in}}%
\pgfpathlineto{\pgfqpoint{3.919386in}{2.369663in}}%
\pgfpathlineto{\pgfqpoint{3.932250in}{2.366135in}}%
\pgfpathlineto{\pgfqpoint{3.924749in}{2.358841in}}%
\pgfpathlineto{\pgfqpoint{3.917242in}{2.351545in}}%
\pgfpathlineto{\pgfqpoint{3.909729in}{2.344249in}}%
\pgfpathlineto{\pgfqpoint{3.902211in}{2.336952in}}%
\pgfpathlineto{\pgfqpoint{3.889335in}{2.340521in}}%
\pgfpathlineto{\pgfqpoint{3.876464in}{2.344119in}}%
\pgfpathlineto{\pgfqpoint{3.863599in}{2.347748in}}%
\pgfpathlineto{\pgfqpoint{3.850740in}{2.351408in}}%
\pgfpathlineto{\pgfqpoint{3.858270in}{2.358660in}}%
\pgfpathlineto{\pgfqpoint{3.865795in}{2.365914in}}%
\pgfpathlineto{\pgfqpoint{3.873314in}{2.373171in}}%
\pgfpathlineto{\pgfqpoint{3.880828in}{2.380429in}}%
\pgfpathclose%
\pgfusepath{fill}%
\end{pgfscope}%
\begin{pgfscope}%
\pgfpathrectangle{\pgfqpoint{1.254980in}{0.150000in}}{\pgfqpoint{5.490039in}{5.490039in}}%
\pgfusepath{clip}%
\pgfsetbuttcap%
\pgfsetroundjoin%
\definecolor{currentfill}{rgb}{0.269944,0.014625,0.341379}%
\pgfsetfillcolor{currentfill}%
\pgfsetfillopacity{0.700000}%
\pgfsetlinewidth{0.000000pt}%
\definecolor{currentstroke}{rgb}{0.000000,0.000000,0.000000}%
\pgfsetstrokecolor{currentstroke}%
\pgfsetdash{}{0pt}%
\pgfpathmoveto{\pgfqpoint{3.400918in}{2.385998in}}%
\pgfpathlineto{\pgfqpoint{3.413674in}{2.381213in}}%
\pgfpathlineto{\pgfqpoint{3.426435in}{2.376465in}}%
\pgfpathlineto{\pgfqpoint{3.439201in}{2.371753in}}%
\pgfpathlineto{\pgfqpoint{3.451971in}{2.367077in}}%
\pgfpathlineto{\pgfqpoint{3.444286in}{2.360475in}}%
\pgfpathlineto{\pgfqpoint{3.436595in}{2.353914in}}%
\pgfpathlineto{\pgfqpoint{3.428898in}{2.347394in}}%
\pgfpathlineto{\pgfqpoint{3.421194in}{2.340917in}}%
\pgfpathlineto{\pgfqpoint{3.408409in}{2.345685in}}%
\pgfpathlineto{\pgfqpoint{3.395629in}{2.350488in}}%
\pgfpathlineto{\pgfqpoint{3.382854in}{2.355328in}}%
\pgfpathlineto{\pgfqpoint{3.370083in}{2.360203in}}%
\pgfpathlineto{\pgfqpoint{3.377802in}{2.366584in}}%
\pgfpathlineto{\pgfqpoint{3.385513in}{2.373011in}}%
\pgfpathlineto{\pgfqpoint{3.393219in}{2.379483in}}%
\pgfpathlineto{\pgfqpoint{3.400918in}{2.385998in}}%
\pgfpathclose%
\pgfusepath{fill}%
\end{pgfscope}%
\begin{pgfscope}%
\pgfpathrectangle{\pgfqpoint{1.254980in}{0.150000in}}{\pgfqpoint{5.490039in}{5.490039in}}%
\pgfusepath{clip}%
\pgfsetbuttcap%
\pgfsetroundjoin%
\definecolor{currentfill}{rgb}{0.281446,0.084320,0.407414}%
\pgfsetfillcolor{currentfill}%
\pgfsetfillopacity{0.700000}%
\pgfsetlinewidth{0.000000pt}%
\definecolor{currentstroke}{rgb}{0.000000,0.000000,0.000000}%
\pgfsetstrokecolor{currentstroke}%
\pgfsetdash{}{0pt}%
\pgfpathmoveto{\pgfqpoint{5.084473in}{2.498862in}}%
\pgfpathlineto{\pgfqpoint{5.097614in}{2.496662in}}%
\pgfpathlineto{\pgfqpoint{5.110762in}{2.494487in}}%
\pgfpathlineto{\pgfqpoint{5.123917in}{2.492335in}}%
\pgfpathlineto{\pgfqpoint{5.137080in}{2.490208in}}%
\pgfpathlineto{\pgfqpoint{5.130038in}{2.483968in}}%
\pgfpathlineto{\pgfqpoint{5.122991in}{2.477737in}}%
\pgfpathlineto{\pgfqpoint{5.115939in}{2.471509in}}%
\pgfpathlineto{\pgfqpoint{5.108881in}{2.465282in}}%
\pgfpathlineto{\pgfqpoint{5.095704in}{2.467309in}}%
\pgfpathlineto{\pgfqpoint{5.082533in}{2.469361in}}%
\pgfpathlineto{\pgfqpoint{5.069369in}{2.471436in}}%
\pgfpathlineto{\pgfqpoint{5.056213in}{2.473536in}}%
\pgfpathlineto{\pgfqpoint{5.063286in}{2.479859in}}%
\pgfpathlineto{\pgfqpoint{5.070353in}{2.486185in}}%
\pgfpathlineto{\pgfqpoint{5.077415in}{2.492518in}}%
\pgfpathlineto{\pgfqpoint{5.084473in}{2.498862in}}%
\pgfpathclose%
\pgfusepath{fill}%
\end{pgfscope}%
\begin{pgfscope}%
\pgfpathrectangle{\pgfqpoint{1.254980in}{0.150000in}}{\pgfqpoint{5.490039in}{5.490039in}}%
\pgfusepath{clip}%
\pgfsetbuttcap%
\pgfsetroundjoin%
\definecolor{currentfill}{rgb}{0.273809,0.031497,0.358853}%
\pgfsetfillcolor{currentfill}%
\pgfsetfillopacity{0.700000}%
\pgfsetlinewidth{0.000000pt}%
\definecolor{currentstroke}{rgb}{0.000000,0.000000,0.000000}%
\pgfsetstrokecolor{currentstroke}%
\pgfsetdash{}{0pt}%
\pgfpathmoveto{\pgfqpoint{3.135150in}{2.420031in}}%
\pgfpathlineto{\pgfqpoint{3.147869in}{2.414432in}}%
\pgfpathlineto{\pgfqpoint{3.160592in}{2.408873in}}%
\pgfpathlineto{\pgfqpoint{3.173319in}{2.403356in}}%
\pgfpathlineto{\pgfqpoint{3.186050in}{2.397879in}}%
\pgfpathlineto{\pgfqpoint{3.178251in}{2.392102in}}%
\pgfpathlineto{\pgfqpoint{3.170444in}{2.386396in}}%
\pgfpathlineto{\pgfqpoint{3.162629in}{2.380763in}}%
\pgfpathlineto{\pgfqpoint{3.154807in}{2.375206in}}%
\pgfpathlineto{\pgfqpoint{3.142060in}{2.380800in}}%
\pgfpathlineto{\pgfqpoint{3.129317in}{2.386435in}}%
\pgfpathlineto{\pgfqpoint{3.116578in}{2.392110in}}%
\pgfpathlineto{\pgfqpoint{3.103842in}{2.397827in}}%
\pgfpathlineto{\pgfqpoint{3.111681in}{2.403262in}}%
\pgfpathlineto{\pgfqpoint{3.119512in}{2.408777in}}%
\pgfpathlineto{\pgfqpoint{3.127335in}{2.414367in}}%
\pgfpathlineto{\pgfqpoint{3.135150in}{2.420031in}}%
\pgfpathclose%
\pgfusepath{fill}%
\end{pgfscope}%
\begin{pgfscope}%
\pgfpathrectangle{\pgfqpoint{1.254980in}{0.150000in}}{\pgfqpoint{5.490039in}{5.490039in}}%
\pgfusepath{clip}%
\pgfsetbuttcap%
\pgfsetroundjoin%
\definecolor{currentfill}{rgb}{0.279566,0.067836,0.391917}%
\pgfsetfillcolor{currentfill}%
\pgfsetfillopacity{0.700000}%
\pgfsetlinewidth{0.000000pt}%
\definecolor{currentstroke}{rgb}{0.000000,0.000000,0.000000}%
\pgfsetstrokecolor{currentstroke}%
\pgfsetdash{}{0pt}%
\pgfpathmoveto{\pgfqpoint{4.870364in}{2.474283in}}%
\pgfpathlineto{\pgfqpoint{4.883451in}{2.471976in}}%
\pgfpathlineto{\pgfqpoint{4.896545in}{2.469693in}}%
\pgfpathlineto{\pgfqpoint{4.909646in}{2.467436in}}%
\pgfpathlineto{\pgfqpoint{4.922754in}{2.465203in}}%
\pgfpathlineto{\pgfqpoint{4.915625in}{2.458696in}}%
\pgfpathlineto{\pgfqpoint{4.908491in}{2.452182in}}%
\pgfpathlineto{\pgfqpoint{4.901351in}{2.445657in}}%
\pgfpathlineto{\pgfqpoint{4.894206in}{2.439121in}}%
\pgfpathlineto{\pgfqpoint{4.881084in}{2.441279in}}%
\pgfpathlineto{\pgfqpoint{4.867969in}{2.443462in}}%
\pgfpathlineto{\pgfqpoint{4.854861in}{2.445670in}}%
\pgfpathlineto{\pgfqpoint{4.841760in}{2.447902in}}%
\pgfpathlineto{\pgfqpoint{4.848919in}{2.454509in}}%
\pgfpathlineto{\pgfqpoint{4.856073in}{2.461106in}}%
\pgfpathlineto{\pgfqpoint{4.863221in}{2.467696in}}%
\pgfpathlineto{\pgfqpoint{4.870364in}{2.474283in}}%
\pgfpathclose%
\pgfusepath{fill}%
\end{pgfscope}%
\begin{pgfscope}%
\pgfpathrectangle{\pgfqpoint{1.254980in}{0.150000in}}{\pgfqpoint{5.490039in}{5.490039in}}%
\pgfusepath{clip}%
\pgfsetbuttcap%
\pgfsetroundjoin%
\definecolor{currentfill}{rgb}{0.277941,0.056324,0.381191}%
\pgfsetfillcolor{currentfill}%
\pgfsetfillopacity{0.700000}%
\pgfsetlinewidth{0.000000pt}%
\definecolor{currentstroke}{rgb}{0.000000,0.000000,0.000000}%
\pgfsetstrokecolor{currentstroke}%
\pgfsetdash{}{0pt}%
\pgfpathmoveto{\pgfqpoint{4.656226in}{2.449347in}}%
\pgfpathlineto{\pgfqpoint{4.669259in}{2.446873in}}%
\pgfpathlineto{\pgfqpoint{4.682299in}{2.444425in}}%
\pgfpathlineto{\pgfqpoint{4.695345in}{2.442003in}}%
\pgfpathlineto{\pgfqpoint{4.708398in}{2.439605in}}%
\pgfpathlineto{\pgfqpoint{4.701184in}{2.432810in}}%
\pgfpathlineto{\pgfqpoint{4.693965in}{2.425999in}}%
\pgfpathlineto{\pgfqpoint{4.686740in}{2.419169in}}%
\pgfpathlineto{\pgfqpoint{4.679510in}{2.412320in}}%
\pgfpathlineto{\pgfqpoint{4.666444in}{2.414668in}}%
\pgfpathlineto{\pgfqpoint{4.653384in}{2.417042in}}%
\pgfpathlineto{\pgfqpoint{4.640331in}{2.419441in}}%
\pgfpathlineto{\pgfqpoint{4.627286in}{2.421866in}}%
\pgfpathlineto{\pgfqpoint{4.634529in}{2.428760in}}%
\pgfpathlineto{\pgfqpoint{4.641767in}{2.435637in}}%
\pgfpathlineto{\pgfqpoint{4.648999in}{2.442498in}}%
\pgfpathlineto{\pgfqpoint{4.656226in}{2.449347in}}%
\pgfpathclose%
\pgfusepath{fill}%
\end{pgfscope}%
\begin{pgfscope}%
\pgfpathrectangle{\pgfqpoint{1.254980in}{0.150000in}}{\pgfqpoint{5.490039in}{5.490039in}}%
\pgfusepath{clip}%
\pgfsetbuttcap%
\pgfsetroundjoin%
\definecolor{currentfill}{rgb}{0.268510,0.009605,0.335427}%
\pgfsetfillcolor{currentfill}%
\pgfsetfillopacity{0.700000}%
\pgfsetlinewidth{0.000000pt}%
\definecolor{currentstroke}{rgb}{0.000000,0.000000,0.000000}%
\pgfsetstrokecolor{currentstroke}%
\pgfsetdash{}{0pt}%
\pgfpathmoveto{\pgfqpoint{3.533720in}{2.375803in}}%
\pgfpathlineto{\pgfqpoint{3.546501in}{2.371380in}}%
\pgfpathlineto{\pgfqpoint{3.559286in}{2.366991in}}%
\pgfpathlineto{\pgfqpoint{3.572076in}{2.362636in}}%
\pgfpathlineto{\pgfqpoint{3.584872in}{2.358314in}}%
\pgfpathlineto{\pgfqpoint{3.577239in}{2.351424in}}%
\pgfpathlineto{\pgfqpoint{3.569600in}{2.344561in}}%
\pgfpathlineto{\pgfqpoint{3.561955in}{2.337725in}}%
\pgfpathlineto{\pgfqpoint{3.554305in}{2.330919in}}%
\pgfpathlineto{\pgfqpoint{3.541496in}{2.335319in}}%
\pgfpathlineto{\pgfqpoint{3.528692in}{2.339753in}}%
\pgfpathlineto{\pgfqpoint{3.515893in}{2.344221in}}%
\pgfpathlineto{\pgfqpoint{3.503099in}{2.348722in}}%
\pgfpathlineto{\pgfqpoint{3.510764in}{2.355445in}}%
\pgfpathlineto{\pgfqpoint{3.518422in}{2.362200in}}%
\pgfpathlineto{\pgfqpoint{3.526074in}{2.368987in}}%
\pgfpathlineto{\pgfqpoint{3.533720in}{2.375803in}}%
\pgfpathclose%
\pgfusepath{fill}%
\end{pgfscope}%
\begin{pgfscope}%
\pgfpathrectangle{\pgfqpoint{1.254980in}{0.150000in}}{\pgfqpoint{5.490039in}{5.490039in}}%
\pgfusepath{clip}%
\pgfsetbuttcap%
\pgfsetroundjoin%
\definecolor{currentfill}{rgb}{0.274952,0.037752,0.364543}%
\pgfsetfillcolor{currentfill}%
\pgfsetfillopacity{0.700000}%
\pgfsetlinewidth{0.000000pt}%
\definecolor{currentstroke}{rgb}{0.000000,0.000000,0.000000}%
\pgfsetstrokecolor{currentstroke}%
\pgfsetdash{}{0pt}%
\pgfpathmoveto{\pgfqpoint{4.442068in}{2.424779in}}%
\pgfpathlineto{\pgfqpoint{4.455048in}{2.422077in}}%
\pgfpathlineto{\pgfqpoint{4.468034in}{2.419402in}}%
\pgfpathlineto{\pgfqpoint{4.481027in}{2.416753in}}%
\pgfpathlineto{\pgfqpoint{4.494026in}{2.414131in}}%
\pgfpathlineto{\pgfqpoint{4.486730in}{2.407072in}}%
\pgfpathlineto{\pgfqpoint{4.479428in}{2.399995in}}%
\pgfpathlineto{\pgfqpoint{4.472121in}{2.392896in}}%
\pgfpathlineto{\pgfqpoint{4.464808in}{2.385776in}}%
\pgfpathlineto{\pgfqpoint{4.451796in}{2.388375in}}%
\pgfpathlineto{\pgfqpoint{4.438791in}{2.391001in}}%
\pgfpathlineto{\pgfqpoint{4.425793in}{2.393653in}}%
\pgfpathlineto{\pgfqpoint{4.412801in}{2.396331in}}%
\pgfpathlineto{\pgfqpoint{4.420126in}{2.403470in}}%
\pgfpathlineto{\pgfqpoint{4.427445in}{2.410590in}}%
\pgfpathlineto{\pgfqpoint{4.434759in}{2.417693in}}%
\pgfpathlineto{\pgfqpoint{4.442068in}{2.424779in}}%
\pgfpathclose%
\pgfusepath{fill}%
\end{pgfscope}%
\begin{pgfscope}%
\pgfpathrectangle{\pgfqpoint{1.254980in}{0.150000in}}{\pgfqpoint{5.490039in}{5.490039in}}%
\pgfusepath{clip}%
\pgfsetbuttcap%
\pgfsetroundjoin%
\definecolor{currentfill}{rgb}{0.272594,0.025563,0.353093}%
\pgfsetfillcolor{currentfill}%
\pgfsetfillopacity{0.700000}%
\pgfsetlinewidth{0.000000pt}%
\definecolor{currentstroke}{rgb}{0.000000,0.000000,0.000000}%
\pgfsetstrokecolor{currentstroke}%
\pgfsetdash{}{0pt}%
\pgfpathmoveto{\pgfqpoint{4.227889in}{2.401690in}}%
\pgfpathlineto{\pgfqpoint{4.240818in}{2.398696in}}%
\pgfpathlineto{\pgfqpoint{4.253752in}{2.395730in}}%
\pgfpathlineto{\pgfqpoint{4.266693in}{2.392791in}}%
\pgfpathlineto{\pgfqpoint{4.279641in}{2.389880in}}%
\pgfpathlineto{\pgfqpoint{4.272264in}{2.382629in}}%
\pgfpathlineto{\pgfqpoint{4.264882in}{2.375361in}}%
\pgfpathlineto{\pgfqpoint{4.257495in}{2.368075in}}%
\pgfpathlineto{\pgfqpoint{4.250102in}{2.360770in}}%
\pgfpathlineto{\pgfqpoint{4.237143in}{2.363684in}}%
\pgfpathlineto{\pgfqpoint{4.224190in}{2.366624in}}%
\pgfpathlineto{\pgfqpoint{4.211243in}{2.369593in}}%
\pgfpathlineto{\pgfqpoint{4.198302in}{2.372589in}}%
\pgfpathlineto{\pgfqpoint{4.205707in}{2.379886in}}%
\pgfpathlineto{\pgfqpoint{4.213106in}{2.387168in}}%
\pgfpathlineto{\pgfqpoint{4.220500in}{2.394436in}}%
\pgfpathlineto{\pgfqpoint{4.227889in}{2.401690in}}%
\pgfpathclose%
\pgfusepath{fill}%
\end{pgfscope}%
\begin{pgfscope}%
\pgfpathrectangle{\pgfqpoint{1.254980in}{0.150000in}}{\pgfqpoint{5.490039in}{5.490039in}}%
\pgfusepath{clip}%
\pgfsetbuttcap%
\pgfsetroundjoin%
\definecolor{currentfill}{rgb}{0.277018,0.050344,0.375715}%
\pgfsetfillcolor{currentfill}%
\pgfsetfillopacity{0.700000}%
\pgfsetlinewidth{0.000000pt}%
\definecolor{currentstroke}{rgb}{0.000000,0.000000,0.000000}%
\pgfsetstrokecolor{currentstroke}%
\pgfsetdash{}{0pt}%
\pgfpathmoveto{\pgfqpoint{3.002089in}{2.445091in}}%
\pgfpathlineto{\pgfqpoint{3.014796in}{2.439030in}}%
\pgfpathlineto{\pgfqpoint{3.027506in}{2.433014in}}%
\pgfpathlineto{\pgfqpoint{3.040220in}{2.427042in}}%
\pgfpathlineto{\pgfqpoint{3.052937in}{2.421114in}}%
\pgfpathlineto{\pgfqpoint{3.045073in}{2.415885in}}%
\pgfpathlineto{\pgfqpoint{3.037201in}{2.410745in}}%
\pgfpathlineto{\pgfqpoint{3.029320in}{2.405695in}}%
\pgfpathlineto{\pgfqpoint{3.021431in}{2.400740in}}%
\pgfpathlineto{\pgfqpoint{3.008696in}{2.406799in}}%
\pgfpathlineto{\pgfqpoint{2.995965in}{2.412901in}}%
\pgfpathlineto{\pgfqpoint{2.983237in}{2.419047in}}%
\pgfpathlineto{\pgfqpoint{2.970513in}{2.425237in}}%
\pgfpathlineto{\pgfqpoint{2.978420in}{2.430058in}}%
\pgfpathlineto{\pgfqpoint{2.986318in}{2.434976in}}%
\pgfpathlineto{\pgfqpoint{2.994208in}{2.439988in}}%
\pgfpathlineto{\pgfqpoint{3.002089in}{2.445091in}}%
\pgfpathclose%
\pgfusepath{fill}%
\end{pgfscope}%
\begin{pgfscope}%
\pgfpathrectangle{\pgfqpoint{1.254980in}{0.150000in}}{\pgfqpoint{5.490039in}{5.490039in}}%
\pgfusepath{clip}%
\pgfsetbuttcap%
\pgfsetroundjoin%
\definecolor{currentfill}{rgb}{0.267004,0.004874,0.329415}%
\pgfsetfillcolor{currentfill}%
\pgfsetfillopacity{0.700000}%
\pgfsetlinewidth{0.000000pt}%
\definecolor{currentstroke}{rgb}{0.000000,0.000000,0.000000}%
\pgfsetstrokecolor{currentstroke}%
\pgfsetdash{}{0pt}%
\pgfpathmoveto{\pgfqpoint{3.666522in}{2.369421in}}%
\pgfpathlineto{\pgfqpoint{3.679330in}{2.365331in}}%
\pgfpathlineto{\pgfqpoint{3.692143in}{2.361273in}}%
\pgfpathlineto{\pgfqpoint{3.704961in}{2.357248in}}%
\pgfpathlineto{\pgfqpoint{3.717784in}{2.353255in}}%
\pgfpathlineto{\pgfqpoint{3.710201in}{2.346151in}}%
\pgfpathlineto{\pgfqpoint{3.702612in}{2.339062in}}%
\pgfpathlineto{\pgfqpoint{3.695017in}{2.331989in}}%
\pgfpathlineto{\pgfqpoint{3.687416in}{2.324932in}}%
\pgfpathlineto{\pgfqpoint{3.674580in}{2.328991in}}%
\pgfpathlineto{\pgfqpoint{3.661749in}{2.333083in}}%
\pgfpathlineto{\pgfqpoint{3.648924in}{2.337206in}}%
\pgfpathlineto{\pgfqpoint{3.636103in}{2.341362in}}%
\pgfpathlineto{\pgfqpoint{3.643717in}{2.348347in}}%
\pgfpathlineto{\pgfqpoint{3.651324in}{2.355353in}}%
\pgfpathlineto{\pgfqpoint{3.658926in}{2.362378in}}%
\pgfpathlineto{\pgfqpoint{3.666522in}{2.369421in}}%
\pgfpathclose%
\pgfusepath{fill}%
\end{pgfscope}%
\begin{pgfscope}%
\pgfpathrectangle{\pgfqpoint{1.254980in}{0.150000in}}{\pgfqpoint{5.490039in}{5.490039in}}%
\pgfusepath{clip}%
\pgfsetbuttcap%
\pgfsetroundjoin%
\definecolor{currentfill}{rgb}{0.283197,0.115680,0.436115}%
\pgfsetfillcolor{currentfill}%
\pgfsetfillopacity{0.700000}%
\pgfsetlinewidth{0.000000pt}%
\definecolor{currentstroke}{rgb}{0.000000,0.000000,0.000000}%
\pgfsetstrokecolor{currentstroke}%
\pgfsetdash{}{0pt}%
\pgfpathmoveto{\pgfqpoint{5.646098in}{2.553149in}}%
\pgfpathlineto{\pgfqpoint{5.659384in}{2.551013in}}%
\pgfpathlineto{\pgfqpoint{5.672678in}{2.548899in}}%
\pgfpathlineto{\pgfqpoint{5.685979in}{2.546809in}}%
\pgfpathlineto{\pgfqpoint{5.699288in}{2.544741in}}%
\pgfpathlineto{\pgfqpoint{5.692471in}{2.538866in}}%
\pgfpathlineto{\pgfqpoint{5.685652in}{2.533058in}}%
\pgfpathlineto{\pgfqpoint{5.678828in}{2.527312in}}%
\pgfpathlineto{\pgfqpoint{5.672002in}{2.521624in}}%
\pgfpathlineto{\pgfqpoint{5.658674in}{2.523527in}}%
\pgfpathlineto{\pgfqpoint{5.645353in}{2.525454in}}%
\pgfpathlineto{\pgfqpoint{5.632040in}{2.527404in}}%
\pgfpathlineto{\pgfqpoint{5.618734in}{2.529376in}}%
\pgfpathlineto{\pgfqpoint{5.625580in}{2.535224in}}%
\pgfpathlineto{\pgfqpoint{5.632422in}{2.541132in}}%
\pgfpathlineto{\pgfqpoint{5.639261in}{2.547105in}}%
\pgfpathlineto{\pgfqpoint{5.646098in}{2.553149in}}%
\pgfpathclose%
\pgfusepath{fill}%
\end{pgfscope}%
\begin{pgfscope}%
\pgfpathrectangle{\pgfqpoint{1.254980in}{0.150000in}}{\pgfqpoint{5.490039in}{5.490039in}}%
\pgfusepath{clip}%
\pgfsetbuttcap%
\pgfsetroundjoin%
\definecolor{currentfill}{rgb}{0.269944,0.014625,0.341379}%
\pgfsetfillcolor{currentfill}%
\pgfsetfillopacity{0.700000}%
\pgfsetlinewidth{0.000000pt}%
\definecolor{currentstroke}{rgb}{0.000000,0.000000,0.000000}%
\pgfsetstrokecolor{currentstroke}%
\pgfsetdash{}{0pt}%
\pgfpathmoveto{\pgfqpoint{4.013668in}{2.381580in}}%
\pgfpathlineto{\pgfqpoint{4.026549in}{2.378228in}}%
\pgfpathlineto{\pgfqpoint{4.039435in}{2.374904in}}%
\pgfpathlineto{\pgfqpoint{4.052328in}{2.371610in}}%
\pgfpathlineto{\pgfqpoint{4.065226in}{2.368344in}}%
\pgfpathlineto{\pgfqpoint{4.057770in}{2.361015in}}%
\pgfpathlineto{\pgfqpoint{4.050309in}{2.353676in}}%
\pgfpathlineto{\pgfqpoint{4.042843in}{2.346328in}}%
\pgfpathlineto{\pgfqpoint{4.035371in}{2.338972in}}%
\pgfpathlineto{\pgfqpoint{4.022460in}{2.342265in}}%
\pgfpathlineto{\pgfqpoint{4.009556in}{2.345587in}}%
\pgfpathlineto{\pgfqpoint{3.996657in}{2.348938in}}%
\pgfpathlineto{\pgfqpoint{3.983764in}{2.352319in}}%
\pgfpathlineto{\pgfqpoint{3.991248in}{2.359643in}}%
\pgfpathlineto{\pgfqpoint{3.998727in}{2.366961in}}%
\pgfpathlineto{\pgfqpoint{4.006200in}{2.374274in}}%
\pgfpathlineto{\pgfqpoint{4.013668in}{2.381580in}}%
\pgfpathclose%
\pgfusepath{fill}%
\end{pgfscope}%
\begin{pgfscope}%
\pgfpathrectangle{\pgfqpoint{1.254980in}{0.150000in}}{\pgfqpoint{5.490039in}{5.490039in}}%
\pgfusepath{clip}%
\pgfsetbuttcap%
\pgfsetroundjoin%
\definecolor{currentfill}{rgb}{0.282910,0.105393,0.426902}%
\pgfsetfillcolor{currentfill}%
\pgfsetfillopacity{0.700000}%
\pgfsetlinewidth{0.000000pt}%
\definecolor{currentstroke}{rgb}{0.000000,0.000000,0.000000}%
\pgfsetstrokecolor{currentstroke}%
\pgfsetdash{}{0pt}%
\pgfpathmoveto{\pgfqpoint{5.432002in}{2.530161in}}%
\pgfpathlineto{\pgfqpoint{5.445238in}{2.528061in}}%
\pgfpathlineto{\pgfqpoint{5.458480in}{2.525984in}}%
\pgfpathlineto{\pgfqpoint{5.471730in}{2.523930in}}%
\pgfpathlineto{\pgfqpoint{5.484988in}{2.521899in}}%
\pgfpathlineto{\pgfqpoint{5.478086in}{2.515995in}}%
\pgfpathlineto{\pgfqpoint{5.471180in}{2.510130in}}%
\pgfpathlineto{\pgfqpoint{5.464269in}{2.504298in}}%
\pgfpathlineto{\pgfqpoint{5.457354in}{2.498496in}}%
\pgfpathlineto{\pgfqpoint{5.444079in}{2.500388in}}%
\pgfpathlineto{\pgfqpoint{5.430812in}{2.502304in}}%
\pgfpathlineto{\pgfqpoint{5.417551in}{2.504242in}}%
\pgfpathlineto{\pgfqpoint{5.404299in}{2.506204in}}%
\pgfpathlineto{\pgfqpoint{5.411231in}{2.512140in}}%
\pgfpathlineto{\pgfqpoint{5.418159in}{2.518108in}}%
\pgfpathlineto{\pgfqpoint{5.425083in}{2.524114in}}%
\pgfpathlineto{\pgfqpoint{5.432002in}{2.530161in}}%
\pgfpathclose%
\pgfusepath{fill}%
\end{pgfscope}%
\begin{pgfscope}%
\pgfpathrectangle{\pgfqpoint{1.254980in}{0.150000in}}{\pgfqpoint{5.490039in}{5.490039in}}%
\pgfusepath{clip}%
\pgfsetbuttcap%
\pgfsetroundjoin%
\definecolor{currentfill}{rgb}{0.282327,0.094955,0.417331}%
\pgfsetfillcolor{currentfill}%
\pgfsetfillopacity{0.700000}%
\pgfsetlinewidth{0.000000pt}%
\definecolor{currentstroke}{rgb}{0.000000,0.000000,0.000000}%
\pgfsetstrokecolor{currentstroke}%
\pgfsetdash{}{0pt}%
\pgfpathmoveto{\pgfqpoint{5.217855in}{2.506586in}}%
\pgfpathlineto{\pgfqpoint{5.231037in}{2.504466in}}%
\pgfpathlineto{\pgfqpoint{5.244227in}{2.502369in}}%
\pgfpathlineto{\pgfqpoint{5.257423in}{2.500296in}}%
\pgfpathlineto{\pgfqpoint{5.270627in}{2.498247in}}%
\pgfpathlineto{\pgfqpoint{5.263638in}{2.492179in}}%
\pgfpathlineto{\pgfqpoint{5.256643in}{2.486126in}}%
\pgfpathlineto{\pgfqpoint{5.249643in}{2.480085in}}%
\pgfpathlineto{\pgfqpoint{5.242638in}{2.474051in}}%
\pgfpathlineto{\pgfqpoint{5.229418in}{2.475987in}}%
\pgfpathlineto{\pgfqpoint{5.216205in}{2.477947in}}%
\pgfpathlineto{\pgfqpoint{5.203000in}{2.479930in}}%
\pgfpathlineto{\pgfqpoint{5.189801in}{2.481938in}}%
\pgfpathlineto{\pgfqpoint{5.196822in}{2.488080in}}%
\pgfpathlineto{\pgfqpoint{5.203838in}{2.494232in}}%
\pgfpathlineto{\pgfqpoint{5.210849in}{2.500400in}}%
\pgfpathlineto{\pgfqpoint{5.217855in}{2.506586in}}%
\pgfpathclose%
\pgfusepath{fill}%
\end{pgfscope}%
\begin{pgfscope}%
\pgfpathrectangle{\pgfqpoint{1.254980in}{0.150000in}}{\pgfqpoint{5.490039in}{5.490039in}}%
\pgfusepath{clip}%
\pgfsetbuttcap%
\pgfsetroundjoin%
\definecolor{currentfill}{rgb}{0.280894,0.078907,0.402329}%
\pgfsetfillcolor{currentfill}%
\pgfsetfillopacity{0.700000}%
\pgfsetlinewidth{0.000000pt}%
\definecolor{currentstroke}{rgb}{0.000000,0.000000,0.000000}%
\pgfsetstrokecolor{currentstroke}%
\pgfsetdash{}{0pt}%
\pgfpathmoveto{\pgfqpoint{5.003658in}{2.482179in}}%
\pgfpathlineto{\pgfqpoint{5.016786in}{2.479982in}}%
\pgfpathlineto{\pgfqpoint{5.029922in}{2.477809in}}%
\pgfpathlineto{\pgfqpoint{5.043064in}{2.475661in}}%
\pgfpathlineto{\pgfqpoint{5.056213in}{2.473536in}}%
\pgfpathlineto{\pgfqpoint{5.049135in}{2.467215in}}%
\pgfpathlineto{\pgfqpoint{5.042052in}{2.460891in}}%
\pgfpathlineto{\pgfqpoint{5.034963in}{2.454562in}}%
\pgfpathlineto{\pgfqpoint{5.027869in}{2.448225in}}%
\pgfpathlineto{\pgfqpoint{5.014704in}{2.450261in}}%
\pgfpathlineto{\pgfqpoint{5.001547in}{2.452322in}}%
\pgfpathlineto{\pgfqpoint{4.988398in}{2.454408in}}%
\pgfpathlineto{\pgfqpoint{4.975255in}{2.456518in}}%
\pgfpathlineto{\pgfqpoint{4.982364in}{2.462938in}}%
\pgfpathlineto{\pgfqpoint{4.989467in}{2.469353in}}%
\pgfpathlineto{\pgfqpoint{4.996565in}{2.475765in}}%
\pgfpathlineto{\pgfqpoint{5.003658in}{2.482179in}}%
\pgfpathclose%
\pgfusepath{fill}%
\end{pgfscope}%
\begin{pgfscope}%
\pgfpathrectangle{\pgfqpoint{1.254980in}{0.150000in}}{\pgfqpoint{5.490039in}{5.490039in}}%
\pgfusepath{clip}%
\pgfsetbuttcap%
\pgfsetroundjoin%
\definecolor{currentfill}{rgb}{0.280267,0.073417,0.397163}%
\pgfsetfillcolor{currentfill}%
\pgfsetfillopacity{0.700000}%
\pgfsetlinewidth{0.000000pt}%
\definecolor{currentstroke}{rgb}{0.000000,0.000000,0.000000}%
\pgfsetstrokecolor{currentstroke}%
\pgfsetdash{}{0pt}%
\pgfpathmoveto{\pgfqpoint{2.868832in}{2.476412in}}%
\pgfpathlineto{\pgfqpoint{2.881531in}{2.469851in}}%
\pgfpathlineto{\pgfqpoint{2.894233in}{2.463337in}}%
\pgfpathlineto{\pgfqpoint{2.906939in}{2.456871in}}%
\pgfpathlineto{\pgfqpoint{2.919647in}{2.450452in}}%
\pgfpathlineto{\pgfqpoint{2.911713in}{2.445870in}}%
\pgfpathlineto{\pgfqpoint{2.903769in}{2.441396in}}%
\pgfpathlineto{\pgfqpoint{2.895816in}{2.437031in}}%
\pgfpathlineto{\pgfqpoint{2.887854in}{2.432781in}}%
\pgfpathlineto{\pgfqpoint{2.875126in}{2.439343in}}%
\pgfpathlineto{\pgfqpoint{2.862402in}{2.445952in}}%
\pgfpathlineto{\pgfqpoint{2.849680in}{2.452609in}}%
\pgfpathlineto{\pgfqpoint{2.836961in}{2.459314in}}%
\pgfpathlineto{\pgfqpoint{2.844943in}{2.463416in}}%
\pgfpathlineto{\pgfqpoint{2.852916in}{2.467636in}}%
\pgfpathlineto{\pgfqpoint{2.860878in}{2.471969in}}%
\pgfpathlineto{\pgfqpoint{2.868832in}{2.476412in}}%
\pgfpathclose%
\pgfusepath{fill}%
\end{pgfscope}%
\begin{pgfscope}%
\pgfpathrectangle{\pgfqpoint{1.254980in}{0.150000in}}{\pgfqpoint{5.490039in}{5.490039in}}%
\pgfusepath{clip}%
\pgfsetbuttcap%
\pgfsetroundjoin%
\definecolor{currentfill}{rgb}{0.279566,0.067836,0.391917}%
\pgfsetfillcolor{currentfill}%
\pgfsetfillopacity{0.700000}%
\pgfsetlinewidth{0.000000pt}%
\definecolor{currentstroke}{rgb}{0.000000,0.000000,0.000000}%
\pgfsetstrokecolor{currentstroke}%
\pgfsetdash{}{0pt}%
\pgfpathmoveto{\pgfqpoint{4.789425in}{2.457083in}}%
\pgfpathlineto{\pgfqpoint{4.802498in}{2.454750in}}%
\pgfpathlineto{\pgfqpoint{4.815579in}{2.452443in}}%
\pgfpathlineto{\pgfqpoint{4.828666in}{2.450160in}}%
\pgfpathlineto{\pgfqpoint{4.841760in}{2.447902in}}%
\pgfpathlineto{\pgfqpoint{4.834595in}{2.441284in}}%
\pgfpathlineto{\pgfqpoint{4.827425in}{2.434652in}}%
\pgfpathlineto{\pgfqpoint{4.820249in}{2.428004in}}%
\pgfpathlineto{\pgfqpoint{4.813068in}{2.421337in}}%
\pgfpathlineto{\pgfqpoint{4.799960in}{2.423533in}}%
\pgfpathlineto{\pgfqpoint{4.786859in}{2.425753in}}%
\pgfpathlineto{\pgfqpoint{4.773765in}{2.427999in}}%
\pgfpathlineto{\pgfqpoint{4.760678in}{2.430270in}}%
\pgfpathlineto{\pgfqpoint{4.767873in}{2.436994in}}%
\pgfpathlineto{\pgfqpoint{4.775062in}{2.443703in}}%
\pgfpathlineto{\pgfqpoint{4.782246in}{2.450398in}}%
\pgfpathlineto{\pgfqpoint{4.789425in}{2.457083in}}%
\pgfpathclose%
\pgfusepath{fill}%
\end{pgfscope}%
\begin{pgfscope}%
\pgfpathrectangle{\pgfqpoint{1.254980in}{0.150000in}}{\pgfqpoint{5.490039in}{5.490039in}}%
\pgfusepath{clip}%
\pgfsetbuttcap%
\pgfsetroundjoin%
\definecolor{currentfill}{rgb}{0.268510,0.009605,0.335427}%
\pgfsetfillcolor{currentfill}%
\pgfsetfillopacity{0.700000}%
\pgfsetlinewidth{0.000000pt}%
\definecolor{currentstroke}{rgb}{0.000000,0.000000,0.000000}%
\pgfsetstrokecolor{currentstroke}%
\pgfsetdash{}{0pt}%
\pgfpathmoveto{\pgfqpoint{3.799357in}{2.366352in}}%
\pgfpathlineto{\pgfqpoint{3.812194in}{2.362570in}}%
\pgfpathlineto{\pgfqpoint{3.825037in}{2.358818in}}%
\pgfpathlineto{\pgfqpoint{3.837886in}{2.355098in}}%
\pgfpathlineto{\pgfqpoint{3.850740in}{2.351408in}}%
\pgfpathlineto{\pgfqpoint{3.843204in}{2.344159in}}%
\pgfpathlineto{\pgfqpoint{3.835662in}{2.336915in}}%
\pgfpathlineto{\pgfqpoint{3.828115in}{2.329676in}}%
\pgfpathlineto{\pgfqpoint{3.820563in}{2.322443in}}%
\pgfpathlineto{\pgfqpoint{3.807697in}{2.326186in}}%
\pgfpathlineto{\pgfqpoint{3.794836in}{2.329959in}}%
\pgfpathlineto{\pgfqpoint{3.781980in}{2.333764in}}%
\pgfpathlineto{\pgfqpoint{3.769130in}{2.337599in}}%
\pgfpathlineto{\pgfqpoint{3.776695in}{2.344775in}}%
\pgfpathlineto{\pgfqpoint{3.784255in}{2.351959in}}%
\pgfpathlineto{\pgfqpoint{3.791809in}{2.359152in}}%
\pgfpathlineto{\pgfqpoint{3.799357in}{2.366352in}}%
\pgfpathclose%
\pgfusepath{fill}%
\end{pgfscope}%
\begin{pgfscope}%
\pgfpathrectangle{\pgfqpoint{1.254980in}{0.150000in}}{\pgfqpoint{5.490039in}{5.490039in}}%
\pgfusepath{clip}%
\pgfsetbuttcap%
\pgfsetroundjoin%
\definecolor{currentfill}{rgb}{0.277018,0.050344,0.375715}%
\pgfsetfillcolor{currentfill}%
\pgfsetfillopacity{0.700000}%
\pgfsetlinewidth{0.000000pt}%
\definecolor{currentstroke}{rgb}{0.000000,0.000000,0.000000}%
\pgfsetstrokecolor{currentstroke}%
\pgfsetdash{}{0pt}%
\pgfpathmoveto{\pgfqpoint{4.575168in}{2.431823in}}%
\pgfpathlineto{\pgfqpoint{4.588188in}{2.429295in}}%
\pgfpathlineto{\pgfqpoint{4.601214in}{2.426793in}}%
\pgfpathlineto{\pgfqpoint{4.614246in}{2.424317in}}%
\pgfpathlineto{\pgfqpoint{4.627286in}{2.421866in}}%
\pgfpathlineto{\pgfqpoint{4.620037in}{2.414954in}}%
\pgfpathlineto{\pgfqpoint{4.612782in}{2.408022in}}%
\pgfpathlineto{\pgfqpoint{4.605522in}{2.401068in}}%
\pgfpathlineto{\pgfqpoint{4.598256in}{2.394091in}}%
\pgfpathlineto{\pgfqpoint{4.585204in}{2.396505in}}%
\pgfpathlineto{\pgfqpoint{4.572159in}{2.398945in}}%
\pgfpathlineto{\pgfqpoint{4.559120in}{2.401411in}}%
\pgfpathlineto{\pgfqpoint{4.546088in}{2.403903in}}%
\pgfpathlineto{\pgfqpoint{4.553367in}{2.410911in}}%
\pgfpathlineto{\pgfqpoint{4.560639in}{2.417900in}}%
\pgfpathlineto{\pgfqpoint{4.567907in}{2.424870in}}%
\pgfpathlineto{\pgfqpoint{4.575168in}{2.431823in}}%
\pgfpathclose%
\pgfusepath{fill}%
\end{pgfscope}%
\begin{pgfscope}%
\pgfpathrectangle{\pgfqpoint{1.254980in}{0.150000in}}{\pgfqpoint{5.490039in}{5.490039in}}%
\pgfusepath{clip}%
\pgfsetbuttcap%
\pgfsetroundjoin%
\definecolor{currentfill}{rgb}{0.273809,0.031497,0.358853}%
\pgfsetfillcolor{currentfill}%
\pgfsetfillopacity{0.700000}%
\pgfsetlinewidth{0.000000pt}%
\definecolor{currentstroke}{rgb}{0.000000,0.000000,0.000000}%
\pgfsetstrokecolor{currentstroke}%
\pgfsetdash{}{0pt}%
\pgfpathmoveto{\pgfqpoint{4.360896in}{2.407313in}}%
\pgfpathlineto{\pgfqpoint{4.373863in}{2.404527in}}%
\pgfpathlineto{\pgfqpoint{4.386836in}{2.401768in}}%
\pgfpathlineto{\pgfqpoint{4.399815in}{2.399036in}}%
\pgfpathlineto{\pgfqpoint{4.412801in}{2.396331in}}%
\pgfpathlineto{\pgfqpoint{4.405470in}{2.389173in}}%
\pgfpathlineto{\pgfqpoint{4.398134in}{2.381994in}}%
\pgfpathlineto{\pgfqpoint{4.390792in}{2.374793in}}%
\pgfpathlineto{\pgfqpoint{4.383445in}{2.367570in}}%
\pgfpathlineto{\pgfqpoint{4.370448in}{2.370265in}}%
\pgfpathlineto{\pgfqpoint{4.357456in}{2.372986in}}%
\pgfpathlineto{\pgfqpoint{4.344471in}{2.375734in}}%
\pgfpathlineto{\pgfqpoint{4.331492in}{2.378509in}}%
\pgfpathlineto{\pgfqpoint{4.338852in}{2.385738in}}%
\pgfpathlineto{\pgfqpoint{4.346205in}{2.392947in}}%
\pgfpathlineto{\pgfqpoint{4.353553in}{2.400139in}}%
\pgfpathlineto{\pgfqpoint{4.360896in}{2.407313in}}%
\pgfpathclose%
\pgfusepath{fill}%
\end{pgfscope}%
\begin{pgfscope}%
\pgfpathrectangle{\pgfqpoint{1.254980in}{0.150000in}}{\pgfqpoint{5.490039in}{5.490039in}}%
\pgfusepath{clip}%
\pgfsetbuttcap%
\pgfsetroundjoin%
\definecolor{currentfill}{rgb}{0.271305,0.019942,0.347269}%
\pgfsetfillcolor{currentfill}%
\pgfsetfillopacity{0.700000}%
\pgfsetlinewidth{0.000000pt}%
\definecolor{currentstroke}{rgb}{0.000000,0.000000,0.000000}%
\pgfsetstrokecolor{currentstroke}%
\pgfsetdash{}{0pt}%
\pgfpathmoveto{\pgfqpoint{4.146600in}{2.384852in}}%
\pgfpathlineto{\pgfqpoint{4.159517in}{2.381744in}}%
\pgfpathlineto{\pgfqpoint{4.172439in}{2.378664in}}%
\pgfpathlineto{\pgfqpoint{4.185368in}{2.375612in}}%
\pgfpathlineto{\pgfqpoint{4.198302in}{2.372589in}}%
\pgfpathlineto{\pgfqpoint{4.190892in}{2.365276in}}%
\pgfpathlineto{\pgfqpoint{4.183476in}{2.357948in}}%
\pgfpathlineto{\pgfqpoint{4.176055in}{2.350604in}}%
\pgfpathlineto{\pgfqpoint{4.168629in}{2.343245in}}%
\pgfpathlineto{\pgfqpoint{4.155682in}{2.346284in}}%
\pgfpathlineto{\pgfqpoint{4.142742in}{2.349351in}}%
\pgfpathlineto{\pgfqpoint{4.129807in}{2.352445in}}%
\pgfpathlineto{\pgfqpoint{4.116879in}{2.355568in}}%
\pgfpathlineto{\pgfqpoint{4.124318in}{2.362908in}}%
\pgfpathlineto{\pgfqpoint{4.131751in}{2.370235in}}%
\pgfpathlineto{\pgfqpoint{4.139178in}{2.377550in}}%
\pgfpathlineto{\pgfqpoint{4.146600in}{2.384852in}}%
\pgfpathclose%
\pgfusepath{fill}%
\end{pgfscope}%
\begin{pgfscope}%
\pgfpathrectangle{\pgfqpoint{1.254980in}{0.150000in}}{\pgfqpoint{5.490039in}{5.490039in}}%
\pgfusepath{clip}%
\pgfsetbuttcap%
\pgfsetroundjoin%
\definecolor{currentfill}{rgb}{0.269944,0.014625,0.341379}%
\pgfsetfillcolor{currentfill}%
\pgfsetfillopacity{0.700000}%
\pgfsetlinewidth{0.000000pt}%
\definecolor{currentstroke}{rgb}{0.000000,0.000000,0.000000}%
\pgfsetstrokecolor{currentstroke}%
\pgfsetdash{}{0pt}%
\pgfpathmoveto{\pgfqpoint{3.319045in}{2.380073in}}%
\pgfpathlineto{\pgfqpoint{3.331798in}{2.375050in}}%
\pgfpathlineto{\pgfqpoint{3.344555in}{2.370064in}}%
\pgfpathlineto{\pgfqpoint{3.357317in}{2.365115in}}%
\pgfpathlineto{\pgfqpoint{3.370083in}{2.360203in}}%
\pgfpathlineto{\pgfqpoint{3.362358in}{2.353872in}}%
\pgfpathlineto{\pgfqpoint{3.354626in}{2.347592in}}%
\pgfpathlineto{\pgfqpoint{3.346888in}{2.341365in}}%
\pgfpathlineto{\pgfqpoint{3.339142in}{2.335194in}}%
\pgfpathlineto{\pgfqpoint{3.326361in}{2.340210in}}%
\pgfpathlineto{\pgfqpoint{3.313584in}{2.345263in}}%
\pgfpathlineto{\pgfqpoint{3.300812in}{2.350353in}}%
\pgfpathlineto{\pgfqpoint{3.288043in}{2.355480in}}%
\pgfpathlineto{\pgfqpoint{3.295804in}{2.361542in}}%
\pgfpathlineto{\pgfqpoint{3.303558in}{2.367663in}}%
\pgfpathlineto{\pgfqpoint{3.311305in}{2.373841in}}%
\pgfpathlineto{\pgfqpoint{3.319045in}{2.380073in}}%
\pgfpathclose%
\pgfusepath{fill}%
\end{pgfscope}%
\begin{pgfscope}%
\pgfpathrectangle{\pgfqpoint{1.254980in}{0.150000in}}{\pgfqpoint{5.490039in}{5.490039in}}%
\pgfusepath{clip}%
\pgfsetbuttcap%
\pgfsetroundjoin%
\definecolor{currentfill}{rgb}{0.272594,0.025563,0.353093}%
\pgfsetfillcolor{currentfill}%
\pgfsetfillopacity{0.700000}%
\pgfsetlinewidth{0.000000pt}%
\definecolor{currentstroke}{rgb}{0.000000,0.000000,0.000000}%
\pgfsetstrokecolor{currentstroke}%
\pgfsetdash{}{0pt}%
\pgfpathmoveto{\pgfqpoint{3.186050in}{2.397879in}}%
\pgfpathlineto{\pgfqpoint{3.198785in}{2.392442in}}%
\pgfpathlineto{\pgfqpoint{3.211524in}{2.387045in}}%
\pgfpathlineto{\pgfqpoint{3.224267in}{2.381687in}}%
\pgfpathlineto{\pgfqpoint{3.237014in}{2.376368in}}%
\pgfpathlineto{\pgfqpoint{3.229230in}{2.370480in}}%
\pgfpathlineto{\pgfqpoint{3.221439in}{2.364658in}}%
\pgfpathlineto{\pgfqpoint{3.213641in}{2.358907in}}%
\pgfpathlineto{\pgfqpoint{3.205835in}{2.353228in}}%
\pgfpathlineto{\pgfqpoint{3.193072in}{2.358663in}}%
\pgfpathlineto{\pgfqpoint{3.180313in}{2.364138in}}%
\pgfpathlineto{\pgfqpoint{3.167558in}{2.369652in}}%
\pgfpathlineto{\pgfqpoint{3.154807in}{2.375206in}}%
\pgfpathlineto{\pgfqpoint{3.162629in}{2.380763in}}%
\pgfpathlineto{\pgfqpoint{3.170444in}{2.386396in}}%
\pgfpathlineto{\pgfqpoint{3.178251in}{2.392102in}}%
\pgfpathlineto{\pgfqpoint{3.186050in}{2.397879in}}%
\pgfpathclose%
\pgfusepath{fill}%
\end{pgfscope}%
\begin{pgfscope}%
\pgfpathrectangle{\pgfqpoint{1.254980in}{0.150000in}}{\pgfqpoint{5.490039in}{5.490039in}}%
\pgfusepath{clip}%
\pgfsetbuttcap%
\pgfsetroundjoin%
\definecolor{currentfill}{rgb}{0.268510,0.009605,0.335427}%
\pgfsetfillcolor{currentfill}%
\pgfsetfillopacity{0.700000}%
\pgfsetlinewidth{0.000000pt}%
\definecolor{currentstroke}{rgb}{0.000000,0.000000,0.000000}%
\pgfsetstrokecolor{currentstroke}%
\pgfsetdash{}{0pt}%
\pgfpathmoveto{\pgfqpoint{3.451971in}{2.367077in}}%
\pgfpathlineto{\pgfqpoint{3.464746in}{2.362436in}}%
\pgfpathlineto{\pgfqpoint{3.477525in}{2.357830in}}%
\pgfpathlineto{\pgfqpoint{3.490310in}{2.353259in}}%
\pgfpathlineto{\pgfqpoint{3.503099in}{2.348722in}}%
\pgfpathlineto{\pgfqpoint{3.495428in}{2.342035in}}%
\pgfpathlineto{\pgfqpoint{3.487751in}{2.335383in}}%
\pgfpathlineto{\pgfqpoint{3.480068in}{2.328770in}}%
\pgfpathlineto{\pgfqpoint{3.472378in}{2.322198in}}%
\pgfpathlineto{\pgfqpoint{3.459575in}{2.326825in}}%
\pgfpathlineto{\pgfqpoint{3.446777in}{2.331487in}}%
\pgfpathlineto{\pgfqpoint{3.433983in}{2.336185in}}%
\pgfpathlineto{\pgfqpoint{3.421194in}{2.340917in}}%
\pgfpathlineto{\pgfqpoint{3.428898in}{2.347394in}}%
\pgfpathlineto{\pgfqpoint{3.436595in}{2.353914in}}%
\pgfpathlineto{\pgfqpoint{3.444286in}{2.360475in}}%
\pgfpathlineto{\pgfqpoint{3.451971in}{2.367077in}}%
\pgfpathclose%
\pgfusepath{fill}%
\end{pgfscope}%
\begin{pgfscope}%
\pgfpathrectangle{\pgfqpoint{1.254980in}{0.150000in}}{\pgfqpoint{5.490039in}{5.490039in}}%
\pgfusepath{clip}%
\pgfsetbuttcap%
\pgfsetroundjoin%
\definecolor{currentfill}{rgb}{0.283197,0.115680,0.436115}%
\pgfsetfillcolor{currentfill}%
\pgfsetfillopacity{0.700000}%
\pgfsetlinewidth{0.000000pt}%
\definecolor{currentstroke}{rgb}{0.000000,0.000000,0.000000}%
\pgfsetstrokecolor{currentstroke}%
\pgfsetdash{}{0pt}%
\pgfpathmoveto{\pgfqpoint{5.565586in}{2.537498in}}%
\pgfpathlineto{\pgfqpoint{5.578862in}{2.535433in}}%
\pgfpathlineto{\pgfqpoint{5.592145in}{2.533391in}}%
\pgfpathlineto{\pgfqpoint{5.605436in}{2.531372in}}%
\pgfpathlineto{\pgfqpoint{5.618734in}{2.529376in}}%
\pgfpathlineto{\pgfqpoint{5.611885in}{2.523584in}}%
\pgfpathlineto{\pgfqpoint{5.605032in}{2.517843in}}%
\pgfpathlineto{\pgfqpoint{5.598175in}{2.512148in}}%
\pgfpathlineto{\pgfqpoint{5.591314in}{2.506494in}}%
\pgfpathlineto{\pgfqpoint{5.577997in}{2.508338in}}%
\pgfpathlineto{\pgfqpoint{5.564687in}{2.510206in}}%
\pgfpathlineto{\pgfqpoint{5.551385in}{2.512097in}}%
\pgfpathlineto{\pgfqpoint{5.538091in}{2.514011in}}%
\pgfpathlineto{\pgfqpoint{5.544970in}{2.519811in}}%
\pgfpathlineto{\pgfqpoint{5.551846in}{2.525656in}}%
\pgfpathlineto{\pgfqpoint{5.558718in}{2.531550in}}%
\pgfpathlineto{\pgfqpoint{5.565586in}{2.537498in}}%
\pgfpathclose%
\pgfusepath{fill}%
\end{pgfscope}%
\begin{pgfscope}%
\pgfpathrectangle{\pgfqpoint{1.254980in}{0.150000in}}{\pgfqpoint{5.490039in}{5.490039in}}%
\pgfusepath{clip}%
\pgfsetbuttcap%
\pgfsetroundjoin%
\definecolor{currentfill}{rgb}{0.268510,0.009605,0.335427}%
\pgfsetfillcolor{currentfill}%
\pgfsetfillopacity{0.700000}%
\pgfsetlinewidth{0.000000pt}%
\definecolor{currentstroke}{rgb}{0.000000,0.000000,0.000000}%
\pgfsetstrokecolor{currentstroke}%
\pgfsetdash{}{0pt}%
\pgfpathmoveto{\pgfqpoint{3.932250in}{2.366135in}}%
\pgfpathlineto{\pgfqpoint{3.945120in}{2.362636in}}%
\pgfpathlineto{\pgfqpoint{3.957996in}{2.359167in}}%
\pgfpathlineto{\pgfqpoint{3.970877in}{2.355728in}}%
\pgfpathlineto{\pgfqpoint{3.983764in}{2.352319in}}%
\pgfpathlineto{\pgfqpoint{3.976275in}{2.344989in}}%
\pgfpathlineto{\pgfqpoint{3.968780in}{2.337655in}}%
\pgfpathlineto{\pgfqpoint{3.961279in}{2.330316in}}%
\pgfpathlineto{\pgfqpoint{3.953773in}{2.322974in}}%
\pgfpathlineto{\pgfqpoint{3.940874in}{2.326424in}}%
\pgfpathlineto{\pgfqpoint{3.927981in}{2.329904in}}%
\pgfpathlineto{\pgfqpoint{3.915093in}{2.333413in}}%
\pgfpathlineto{\pgfqpoint{3.902211in}{2.336952in}}%
\pgfpathlineto{\pgfqpoint{3.909729in}{2.344249in}}%
\pgfpathlineto{\pgfqpoint{3.917242in}{2.351545in}}%
\pgfpathlineto{\pgfqpoint{3.924749in}{2.358841in}}%
\pgfpathlineto{\pgfqpoint{3.932250in}{2.366135in}}%
\pgfpathclose%
\pgfusepath{fill}%
\end{pgfscope}%
\begin{pgfscope}%
\pgfpathrectangle{\pgfqpoint{1.254980in}{0.150000in}}{\pgfqpoint{5.490039in}{5.490039in}}%
\pgfusepath{clip}%
\pgfsetbuttcap%
\pgfsetroundjoin%
\definecolor{currentfill}{rgb}{0.276022,0.044167,0.370164}%
\pgfsetfillcolor{currentfill}%
\pgfsetfillopacity{0.700000}%
\pgfsetlinewidth{0.000000pt}%
\definecolor{currentstroke}{rgb}{0.000000,0.000000,0.000000}%
\pgfsetstrokecolor{currentstroke}%
\pgfsetdash{}{0pt}%
\pgfpathmoveto{\pgfqpoint{3.052937in}{2.421114in}}%
\pgfpathlineto{\pgfqpoint{3.065658in}{2.415228in}}%
\pgfpathlineto{\pgfqpoint{3.078382in}{2.409386in}}%
\pgfpathlineto{\pgfqpoint{3.091110in}{2.403585in}}%
\pgfpathlineto{\pgfqpoint{3.103842in}{2.397827in}}%
\pgfpathlineto{\pgfqpoint{3.095995in}{2.392473in}}%
\pgfpathlineto{\pgfqpoint{3.088140in}{2.387204in}}%
\pgfpathlineto{\pgfqpoint{3.080278in}{2.382023in}}%
\pgfpathlineto{\pgfqpoint{3.072406in}{2.376933in}}%
\pgfpathlineto{\pgfqpoint{3.059657in}{2.382821in}}%
\pgfpathlineto{\pgfqpoint{3.046912in}{2.388752in}}%
\pgfpathlineto{\pgfqpoint{3.034170in}{2.394724in}}%
\pgfpathlineto{\pgfqpoint{3.021431in}{2.400740in}}%
\pgfpathlineto{\pgfqpoint{3.029320in}{2.405695in}}%
\pgfpathlineto{\pgfqpoint{3.037201in}{2.410745in}}%
\pgfpathlineto{\pgfqpoint{3.045073in}{2.415885in}}%
\pgfpathlineto{\pgfqpoint{3.052937in}{2.421114in}}%
\pgfpathclose%
\pgfusepath{fill}%
\end{pgfscope}%
\begin{pgfscope}%
\pgfpathrectangle{\pgfqpoint{1.254980in}{0.150000in}}{\pgfqpoint{5.490039in}{5.490039in}}%
\pgfusepath{clip}%
\pgfsetbuttcap%
\pgfsetroundjoin%
\definecolor{currentfill}{rgb}{0.267004,0.004874,0.329415}%
\pgfsetfillcolor{currentfill}%
\pgfsetfillopacity{0.700000}%
\pgfsetlinewidth{0.000000pt}%
\definecolor{currentstroke}{rgb}{0.000000,0.000000,0.000000}%
\pgfsetstrokecolor{currentstroke}%
\pgfsetdash{}{0pt}%
\pgfpathmoveto{\pgfqpoint{3.584872in}{2.358314in}}%
\pgfpathlineto{\pgfqpoint{3.597672in}{2.354026in}}%
\pgfpathlineto{\pgfqpoint{3.610477in}{2.349772in}}%
\pgfpathlineto{\pgfqpoint{3.623288in}{2.345550in}}%
\pgfpathlineto{\pgfqpoint{3.636103in}{2.341362in}}%
\pgfpathlineto{\pgfqpoint{3.628483in}{2.334398in}}%
\pgfpathlineto{\pgfqpoint{3.620858in}{2.327457in}}%
\pgfpathlineto{\pgfqpoint{3.613227in}{2.320542in}}%
\pgfpathlineto{\pgfqpoint{3.605589in}{2.313653in}}%
\pgfpathlineto{\pgfqpoint{3.592761in}{2.317920in}}%
\pgfpathlineto{\pgfqpoint{3.579937in}{2.322220in}}%
\pgfpathlineto{\pgfqpoint{3.567118in}{2.326553in}}%
\pgfpathlineto{\pgfqpoint{3.554305in}{2.330919in}}%
\pgfpathlineto{\pgfqpoint{3.561955in}{2.337725in}}%
\pgfpathlineto{\pgfqpoint{3.569600in}{2.344561in}}%
\pgfpathlineto{\pgfqpoint{3.577239in}{2.351424in}}%
\pgfpathlineto{\pgfqpoint{3.584872in}{2.358314in}}%
\pgfpathclose%
\pgfusepath{fill}%
\end{pgfscope}%
\begin{pgfscope}%
\pgfpathrectangle{\pgfqpoint{1.254980in}{0.150000in}}{\pgfqpoint{5.490039in}{5.490039in}}%
\pgfusepath{clip}%
\pgfsetbuttcap%
\pgfsetroundjoin%
\definecolor{currentfill}{rgb}{0.282656,0.100196,0.422160}%
\pgfsetfillcolor{currentfill}%
\pgfsetfillopacity{0.700000}%
\pgfsetlinewidth{0.000000pt}%
\definecolor{currentstroke}{rgb}{0.000000,0.000000,0.000000}%
\pgfsetstrokecolor{currentstroke}%
\pgfsetdash{}{0pt}%
\pgfpathmoveto{\pgfqpoint{5.351361in}{2.514288in}}%
\pgfpathlineto{\pgfqpoint{5.364584in}{2.512232in}}%
\pgfpathlineto{\pgfqpoint{5.377815in}{2.510199in}}%
\pgfpathlineto{\pgfqpoint{5.391053in}{2.508190in}}%
\pgfpathlineto{\pgfqpoint{5.404299in}{2.506204in}}%
\pgfpathlineto{\pgfqpoint{5.397362in}{2.500298in}}%
\pgfpathlineto{\pgfqpoint{5.390420in}{2.494415in}}%
\pgfpathlineto{\pgfqpoint{5.383473in}{2.488554in}}%
\pgfpathlineto{\pgfqpoint{5.376522in}{2.482709in}}%
\pgfpathlineto{\pgfqpoint{5.363259in}{2.484568in}}%
\pgfpathlineto{\pgfqpoint{5.350004in}{2.486452in}}%
\pgfpathlineto{\pgfqpoint{5.336756in}{2.488358in}}%
\pgfpathlineto{\pgfqpoint{5.323516in}{2.490289in}}%
\pgfpathlineto{\pgfqpoint{5.330484in}{2.496255in}}%
\pgfpathlineto{\pgfqpoint{5.337448in}{2.502241in}}%
\pgfpathlineto{\pgfqpoint{5.344406in}{2.508251in}}%
\pgfpathlineto{\pgfqpoint{5.351361in}{2.514288in}}%
\pgfpathclose%
\pgfusepath{fill}%
\end{pgfscope}%
\begin{pgfscope}%
\pgfpathrectangle{\pgfqpoint{1.254980in}{0.150000in}}{\pgfqpoint{5.490039in}{5.490039in}}%
\pgfusepath{clip}%
\pgfsetbuttcap%
\pgfsetroundjoin%
\definecolor{currentfill}{rgb}{0.281924,0.089666,0.412415}%
\pgfsetfillcolor{currentfill}%
\pgfsetfillopacity{0.700000}%
\pgfsetlinewidth{0.000000pt}%
\definecolor{currentstroke}{rgb}{0.000000,0.000000,0.000000}%
\pgfsetstrokecolor{currentstroke}%
\pgfsetdash{}{0pt}%
\pgfpathmoveto{\pgfqpoint{5.137080in}{2.490208in}}%
\pgfpathlineto{\pgfqpoint{5.150249in}{2.488104in}}%
\pgfpathlineto{\pgfqpoint{5.163426in}{2.486025in}}%
\pgfpathlineto{\pgfqpoint{5.176610in}{2.483969in}}%
\pgfpathlineto{\pgfqpoint{5.189801in}{2.481938in}}%
\pgfpathlineto{\pgfqpoint{5.182775in}{2.475804in}}%
\pgfpathlineto{\pgfqpoint{5.175744in}{2.469674in}}%
\pgfpathlineto{\pgfqpoint{5.168707in}{2.463545in}}%
\pgfpathlineto{\pgfqpoint{5.161665in}{2.457413in}}%
\pgfpathlineto{\pgfqpoint{5.148458in}{2.459344in}}%
\pgfpathlineto{\pgfqpoint{5.135259in}{2.461300in}}%
\pgfpathlineto{\pgfqpoint{5.122066in}{2.463279in}}%
\pgfpathlineto{\pgfqpoint{5.108881in}{2.465282in}}%
\pgfpathlineto{\pgfqpoint{5.115939in}{2.471509in}}%
\pgfpathlineto{\pgfqpoint{5.122991in}{2.477737in}}%
\pgfpathlineto{\pgfqpoint{5.130038in}{2.483968in}}%
\pgfpathlineto{\pgfqpoint{5.137080in}{2.490208in}}%
\pgfpathclose%
\pgfusepath{fill}%
\end{pgfscope}%
\begin{pgfscope}%
\pgfpathrectangle{\pgfqpoint{1.254980in}{0.150000in}}{\pgfqpoint{5.490039in}{5.490039in}}%
\pgfusepath{clip}%
\pgfsetbuttcap%
\pgfsetroundjoin%
\definecolor{currentfill}{rgb}{0.280267,0.073417,0.397163}%
\pgfsetfillcolor{currentfill}%
\pgfsetfillopacity{0.700000}%
\pgfsetlinewidth{0.000000pt}%
\definecolor{currentstroke}{rgb}{0.000000,0.000000,0.000000}%
\pgfsetstrokecolor{currentstroke}%
\pgfsetdash{}{0pt}%
\pgfpathmoveto{\pgfqpoint{4.922754in}{2.465203in}}%
\pgfpathlineto{\pgfqpoint{4.935869in}{2.462995in}}%
\pgfpathlineto{\pgfqpoint{4.948990in}{2.460811in}}%
\pgfpathlineto{\pgfqpoint{4.962119in}{2.458652in}}%
\pgfpathlineto{\pgfqpoint{4.975255in}{2.456518in}}%
\pgfpathlineto{\pgfqpoint{4.968140in}{2.450090in}}%
\pgfpathlineto{\pgfqpoint{4.961020in}{2.443652in}}%
\pgfpathlineto{\pgfqpoint{4.953895in}{2.437201in}}%
\pgfpathlineto{\pgfqpoint{4.946764in}{2.430735in}}%
\pgfpathlineto{\pgfqpoint{4.933614in}{2.432794in}}%
\pgfpathlineto{\pgfqpoint{4.920471in}{2.434879in}}%
\pgfpathlineto{\pgfqpoint{4.907335in}{2.436987in}}%
\pgfpathlineto{\pgfqpoint{4.894206in}{2.439121in}}%
\pgfpathlineto{\pgfqpoint{4.901351in}{2.445657in}}%
\pgfpathlineto{\pgfqpoint{4.908491in}{2.452182in}}%
\pgfpathlineto{\pgfqpoint{4.915625in}{2.458696in}}%
\pgfpathlineto{\pgfqpoint{4.922754in}{2.465203in}}%
\pgfpathclose%
\pgfusepath{fill}%
\end{pgfscope}%
\begin{pgfscope}%
\pgfpathrectangle{\pgfqpoint{1.254980in}{0.150000in}}{\pgfqpoint{5.490039in}{5.490039in}}%
\pgfusepath{clip}%
\pgfsetbuttcap%
\pgfsetroundjoin%
\definecolor{currentfill}{rgb}{0.278791,0.062145,0.386592}%
\pgfsetfillcolor{currentfill}%
\pgfsetfillopacity{0.700000}%
\pgfsetlinewidth{0.000000pt}%
\definecolor{currentstroke}{rgb}{0.000000,0.000000,0.000000}%
\pgfsetstrokecolor{currentstroke}%
\pgfsetdash{}{0pt}%
\pgfpathmoveto{\pgfqpoint{4.708398in}{2.439605in}}%
\pgfpathlineto{\pgfqpoint{4.721458in}{2.437234in}}%
\pgfpathlineto{\pgfqpoint{4.734525in}{2.434887in}}%
\pgfpathlineto{\pgfqpoint{4.747598in}{2.432566in}}%
\pgfpathlineto{\pgfqpoint{4.760678in}{2.430270in}}%
\pgfpathlineto{\pgfqpoint{4.753478in}{2.423529in}}%
\pgfpathlineto{\pgfqpoint{4.746272in}{2.416768in}}%
\pgfpathlineto{\pgfqpoint{4.739060in}{2.409986in}}%
\pgfpathlineto{\pgfqpoint{4.731842in}{2.403181in}}%
\pgfpathlineto{\pgfqpoint{4.718749in}{2.405428in}}%
\pgfpathlineto{\pgfqpoint{4.705662in}{2.407700in}}%
\pgfpathlineto{\pgfqpoint{4.692583in}{2.409997in}}%
\pgfpathlineto{\pgfqpoint{4.679510in}{2.412320in}}%
\pgfpathlineto{\pgfqpoint{4.686740in}{2.419169in}}%
\pgfpathlineto{\pgfqpoint{4.693965in}{2.425999in}}%
\pgfpathlineto{\pgfqpoint{4.701184in}{2.432810in}}%
\pgfpathlineto{\pgfqpoint{4.708398in}{2.439605in}}%
\pgfpathclose%
\pgfusepath{fill}%
\end{pgfscope}%
\begin{pgfscope}%
\pgfpathrectangle{\pgfqpoint{1.254980in}{0.150000in}}{\pgfqpoint{5.490039in}{5.490039in}}%
\pgfusepath{clip}%
\pgfsetbuttcap%
\pgfsetroundjoin%
\definecolor{currentfill}{rgb}{0.276022,0.044167,0.370164}%
\pgfsetfillcolor{currentfill}%
\pgfsetfillopacity{0.700000}%
\pgfsetlinewidth{0.000000pt}%
\definecolor{currentstroke}{rgb}{0.000000,0.000000,0.000000}%
\pgfsetstrokecolor{currentstroke}%
\pgfsetdash{}{0pt}%
\pgfpathmoveto{\pgfqpoint{4.494026in}{2.414131in}}%
\pgfpathlineto{\pgfqpoint{4.507032in}{2.411534in}}%
\pgfpathlineto{\pgfqpoint{4.520044in}{2.408964in}}%
\pgfpathlineto{\pgfqpoint{4.533063in}{2.406420in}}%
\pgfpathlineto{\pgfqpoint{4.546088in}{2.403903in}}%
\pgfpathlineto{\pgfqpoint{4.538805in}{2.396873in}}%
\pgfpathlineto{\pgfqpoint{4.531515in}{2.389820in}}%
\pgfpathlineto{\pgfqpoint{4.524220in}{2.382744in}}%
\pgfpathlineto{\pgfqpoint{4.516920in}{2.375642in}}%
\pgfpathlineto{\pgfqpoint{4.503882in}{2.378136in}}%
\pgfpathlineto{\pgfqpoint{4.490851in}{2.380657in}}%
\pgfpathlineto{\pgfqpoint{4.477826in}{2.383203in}}%
\pgfpathlineto{\pgfqpoint{4.464808in}{2.385776in}}%
\pgfpathlineto{\pgfqpoint{4.472121in}{2.392896in}}%
\pgfpathlineto{\pgfqpoint{4.479428in}{2.399995in}}%
\pgfpathlineto{\pgfqpoint{4.486730in}{2.407072in}}%
\pgfpathlineto{\pgfqpoint{4.494026in}{2.414131in}}%
\pgfpathclose%
\pgfusepath{fill}%
\end{pgfscope}%
\begin{pgfscope}%
\pgfpathrectangle{\pgfqpoint{1.254980in}{0.150000in}}{\pgfqpoint{5.490039in}{5.490039in}}%
\pgfusepath{clip}%
\pgfsetbuttcap%
\pgfsetroundjoin%
\definecolor{currentfill}{rgb}{0.272594,0.025563,0.353093}%
\pgfsetfillcolor{currentfill}%
\pgfsetfillopacity{0.700000}%
\pgfsetlinewidth{0.000000pt}%
\definecolor{currentstroke}{rgb}{0.000000,0.000000,0.000000}%
\pgfsetstrokecolor{currentstroke}%
\pgfsetdash{}{0pt}%
\pgfpathmoveto{\pgfqpoint{4.279641in}{2.389880in}}%
\pgfpathlineto{\pgfqpoint{4.292594in}{2.386996in}}%
\pgfpathlineto{\pgfqpoint{4.305554in}{2.384140in}}%
\pgfpathlineto{\pgfqpoint{4.318520in}{2.381311in}}%
\pgfpathlineto{\pgfqpoint{4.331492in}{2.378509in}}%
\pgfpathlineto{\pgfqpoint{4.324128in}{2.371260in}}%
\pgfpathlineto{\pgfqpoint{4.316758in}{2.363992in}}%
\pgfpathlineto{\pgfqpoint{4.309382in}{2.356702in}}%
\pgfpathlineto{\pgfqpoint{4.302001in}{2.349391in}}%
\pgfpathlineto{\pgfqpoint{4.289017in}{2.352195in}}%
\pgfpathlineto{\pgfqpoint{4.276039in}{2.355026in}}%
\pgfpathlineto{\pgfqpoint{4.263067in}{2.357885in}}%
\pgfpathlineto{\pgfqpoint{4.250102in}{2.360770in}}%
\pgfpathlineto{\pgfqpoint{4.257495in}{2.368075in}}%
\pgfpathlineto{\pgfqpoint{4.264882in}{2.375361in}}%
\pgfpathlineto{\pgfqpoint{4.272264in}{2.382629in}}%
\pgfpathlineto{\pgfqpoint{4.279641in}{2.389880in}}%
\pgfpathclose%
\pgfusepath{fill}%
\end{pgfscope}%
\begin{pgfscope}%
\pgfpathrectangle{\pgfqpoint{1.254980in}{0.150000in}}{\pgfqpoint{5.490039in}{5.490039in}}%
\pgfusepath{clip}%
\pgfsetbuttcap%
\pgfsetroundjoin%
\definecolor{currentfill}{rgb}{0.267004,0.004874,0.329415}%
\pgfsetfillcolor{currentfill}%
\pgfsetfillopacity{0.700000}%
\pgfsetlinewidth{0.000000pt}%
\definecolor{currentstroke}{rgb}{0.000000,0.000000,0.000000}%
\pgfsetstrokecolor{currentstroke}%
\pgfsetdash{}{0pt}%
\pgfpathmoveto{\pgfqpoint{3.717784in}{2.353255in}}%
\pgfpathlineto{\pgfqpoint{3.730613in}{2.349294in}}%
\pgfpathlineto{\pgfqpoint{3.743447in}{2.345364in}}%
\pgfpathlineto{\pgfqpoint{3.756286in}{2.341466in}}%
\pgfpathlineto{\pgfqpoint{3.769130in}{2.337599in}}%
\pgfpathlineto{\pgfqpoint{3.761560in}{2.330435in}}%
\pgfpathlineto{\pgfqpoint{3.753983in}{2.323281in}}%
\pgfpathlineto{\pgfqpoint{3.746401in}{2.316140in}}%
\pgfpathlineto{\pgfqpoint{3.738813in}{2.309014in}}%
\pgfpathlineto{\pgfqpoint{3.725956in}{2.312946in}}%
\pgfpathlineto{\pgfqpoint{3.713104in}{2.316910in}}%
\pgfpathlineto{\pgfqpoint{3.700258in}{2.320905in}}%
\pgfpathlineto{\pgfqpoint{3.687416in}{2.324932in}}%
\pgfpathlineto{\pgfqpoint{3.695017in}{2.331989in}}%
\pgfpathlineto{\pgfqpoint{3.702612in}{2.339062in}}%
\pgfpathlineto{\pgfqpoint{3.710201in}{2.346151in}}%
\pgfpathlineto{\pgfqpoint{3.717784in}{2.353255in}}%
\pgfpathclose%
\pgfusepath{fill}%
\end{pgfscope}%
\begin{pgfscope}%
\pgfpathrectangle{\pgfqpoint{1.254980in}{0.150000in}}{\pgfqpoint{5.490039in}{5.490039in}}%
\pgfusepath{clip}%
\pgfsetbuttcap%
\pgfsetroundjoin%
\definecolor{currentfill}{rgb}{0.279566,0.067836,0.391917}%
\pgfsetfillcolor{currentfill}%
\pgfsetfillopacity{0.700000}%
\pgfsetlinewidth{0.000000pt}%
\definecolor{currentstroke}{rgb}{0.000000,0.000000,0.000000}%
\pgfsetstrokecolor{currentstroke}%
\pgfsetdash{}{0pt}%
\pgfpathmoveto{\pgfqpoint{2.919647in}{2.450452in}}%
\pgfpathlineto{\pgfqpoint{2.932359in}{2.444080in}}%
\pgfpathlineto{\pgfqpoint{2.945073in}{2.437753in}}%
\pgfpathlineto{\pgfqpoint{2.957791in}{2.431473in}}%
\pgfpathlineto{\pgfqpoint{2.970513in}{2.425237in}}%
\pgfpathlineto{\pgfqpoint{2.962597in}{2.420517in}}%
\pgfpathlineto{\pgfqpoint{2.954672in}{2.415901in}}%
\pgfpathlineto{\pgfqpoint{2.946738in}{2.411392in}}%
\pgfpathlineto{\pgfqpoint{2.938795in}{2.406993in}}%
\pgfpathlineto{\pgfqpoint{2.926055in}{2.413372in}}%
\pgfpathlineto{\pgfqpoint{2.913318in}{2.419795in}}%
\pgfpathlineto{\pgfqpoint{2.900584in}{2.426265in}}%
\pgfpathlineto{\pgfqpoint{2.887854in}{2.432781in}}%
\pgfpathlineto{\pgfqpoint{2.895816in}{2.437031in}}%
\pgfpathlineto{\pgfqpoint{2.903769in}{2.441396in}}%
\pgfpathlineto{\pgfqpoint{2.911713in}{2.445870in}}%
\pgfpathlineto{\pgfqpoint{2.919647in}{2.450452in}}%
\pgfpathclose%
\pgfusepath{fill}%
\end{pgfscope}%
\begin{pgfscope}%
\pgfpathrectangle{\pgfqpoint{1.254980in}{0.150000in}}{\pgfqpoint{5.490039in}{5.490039in}}%
\pgfusepath{clip}%
\pgfsetbuttcap%
\pgfsetroundjoin%
\definecolor{currentfill}{rgb}{0.269944,0.014625,0.341379}%
\pgfsetfillcolor{currentfill}%
\pgfsetfillopacity{0.700000}%
\pgfsetlinewidth{0.000000pt}%
\definecolor{currentstroke}{rgb}{0.000000,0.000000,0.000000}%
\pgfsetstrokecolor{currentstroke}%
\pgfsetdash{}{0pt}%
\pgfpathmoveto{\pgfqpoint{4.065226in}{2.368344in}}%
\pgfpathlineto{\pgfqpoint{4.078130in}{2.365107in}}%
\pgfpathlineto{\pgfqpoint{4.091041in}{2.361899in}}%
\pgfpathlineto{\pgfqpoint{4.103957in}{2.358719in}}%
\pgfpathlineto{\pgfqpoint{4.116879in}{2.355568in}}%
\pgfpathlineto{\pgfqpoint{4.109435in}{2.348216in}}%
\pgfpathlineto{\pgfqpoint{4.101986in}{2.340851in}}%
\pgfpathlineto{\pgfqpoint{4.094531in}{2.333474in}}%
\pgfpathlineto{\pgfqpoint{4.087071in}{2.326085in}}%
\pgfpathlineto{\pgfqpoint{4.074137in}{2.329264in}}%
\pgfpathlineto{\pgfqpoint{4.061209in}{2.332471in}}%
\pgfpathlineto{\pgfqpoint{4.048287in}{2.335707in}}%
\pgfpathlineto{\pgfqpoint{4.035371in}{2.338972in}}%
\pgfpathlineto{\pgfqpoint{4.042843in}{2.346328in}}%
\pgfpathlineto{\pgfqpoint{4.050309in}{2.353676in}}%
\pgfpathlineto{\pgfqpoint{4.057770in}{2.361015in}}%
\pgfpathlineto{\pgfqpoint{4.065226in}{2.368344in}}%
\pgfpathclose%
\pgfusepath{fill}%
\end{pgfscope}%
\begin{pgfscope}%
\pgfpathrectangle{\pgfqpoint{1.254980in}{0.150000in}}{\pgfqpoint{5.490039in}{5.490039in}}%
\pgfusepath{clip}%
\pgfsetbuttcap%
\pgfsetroundjoin%
\definecolor{currentfill}{rgb}{0.283229,0.120777,0.440584}%
\pgfsetfillcolor{currentfill}%
\pgfsetfillopacity{0.700000}%
\pgfsetlinewidth{0.000000pt}%
\definecolor{currentstroke}{rgb}{0.000000,0.000000,0.000000}%
\pgfsetstrokecolor{currentstroke}%
\pgfsetdash{}{0pt}%
\pgfpathmoveto{\pgfqpoint{5.699288in}{2.544741in}}%
\pgfpathlineto{\pgfqpoint{5.712605in}{2.542697in}}%
\pgfpathlineto{\pgfqpoint{5.725928in}{2.540675in}}%
\pgfpathlineto{\pgfqpoint{5.739260in}{2.538677in}}%
\pgfpathlineto{\pgfqpoint{5.752598in}{2.536701in}}%
\pgfpathlineto{\pgfqpoint{5.745801in}{2.530994in}}%
\pgfpathlineto{\pgfqpoint{5.739001in}{2.525352in}}%
\pgfpathlineto{\pgfqpoint{5.732198in}{2.519768in}}%
\pgfpathlineto{\pgfqpoint{5.725391in}{2.514240in}}%
\pgfpathlineto{\pgfqpoint{5.712032in}{2.516051in}}%
\pgfpathlineto{\pgfqpoint{5.698681in}{2.517886in}}%
\pgfpathlineto{\pgfqpoint{5.685338in}{2.519743in}}%
\pgfpathlineto{\pgfqpoint{5.672002in}{2.521624in}}%
\pgfpathlineto{\pgfqpoint{5.678828in}{2.527312in}}%
\pgfpathlineto{\pgfqpoint{5.685652in}{2.533058in}}%
\pgfpathlineto{\pgfqpoint{5.692471in}{2.538866in}}%
\pgfpathlineto{\pgfqpoint{5.699288in}{2.544741in}}%
\pgfpathclose%
\pgfusepath{fill}%
\end{pgfscope}%
\begin{pgfscope}%
\pgfpathrectangle{\pgfqpoint{1.254980in}{0.150000in}}{\pgfqpoint{5.490039in}{5.490039in}}%
\pgfusepath{clip}%
\pgfsetbuttcap%
\pgfsetroundjoin%
\definecolor{currentfill}{rgb}{0.283091,0.110553,0.431554}%
\pgfsetfillcolor{currentfill}%
\pgfsetfillopacity{0.700000}%
\pgfsetlinewidth{0.000000pt}%
\definecolor{currentstroke}{rgb}{0.000000,0.000000,0.000000}%
\pgfsetstrokecolor{currentstroke}%
\pgfsetdash{}{0pt}%
\pgfpathmoveto{\pgfqpoint{5.484988in}{2.521899in}}%
\pgfpathlineto{\pgfqpoint{5.498252in}{2.519892in}}%
\pgfpathlineto{\pgfqpoint{5.511524in}{2.517909in}}%
\pgfpathlineto{\pgfqpoint{5.524804in}{2.515948in}}%
\pgfpathlineto{\pgfqpoint{5.538091in}{2.514011in}}%
\pgfpathlineto{\pgfqpoint{5.531207in}{2.508250in}}%
\pgfpathlineto{\pgfqpoint{5.524319in}{2.502525in}}%
\pgfpathlineto{\pgfqpoint{5.517427in}{2.496830in}}%
\pgfpathlineto{\pgfqpoint{5.510530in}{2.491163in}}%
\pgfpathlineto{\pgfqpoint{5.497225in}{2.492961in}}%
\pgfpathlineto{\pgfqpoint{5.483927in}{2.494783in}}%
\pgfpathlineto{\pgfqpoint{5.470637in}{2.496628in}}%
\pgfpathlineto{\pgfqpoint{5.457354in}{2.498496in}}%
\pgfpathlineto{\pgfqpoint{5.464269in}{2.504298in}}%
\pgfpathlineto{\pgfqpoint{5.471180in}{2.510130in}}%
\pgfpathlineto{\pgfqpoint{5.478086in}{2.515995in}}%
\pgfpathlineto{\pgfqpoint{5.484988in}{2.521899in}}%
\pgfpathclose%
\pgfusepath{fill}%
\end{pgfscope}%
\begin{pgfscope}%
\pgfpathrectangle{\pgfqpoint{1.254980in}{0.150000in}}{\pgfqpoint{5.490039in}{5.490039in}}%
\pgfusepath{clip}%
\pgfsetbuttcap%
\pgfsetroundjoin%
\definecolor{currentfill}{rgb}{0.268510,0.009605,0.335427}%
\pgfsetfillcolor{currentfill}%
\pgfsetfillopacity{0.700000}%
\pgfsetlinewidth{0.000000pt}%
\definecolor{currentstroke}{rgb}{0.000000,0.000000,0.000000}%
\pgfsetstrokecolor{currentstroke}%
\pgfsetdash{}{0pt}%
\pgfpathmoveto{\pgfqpoint{3.850740in}{2.351408in}}%
\pgfpathlineto{\pgfqpoint{3.863599in}{2.347748in}}%
\pgfpathlineto{\pgfqpoint{3.876464in}{2.344119in}}%
\pgfpathlineto{\pgfqpoint{3.889335in}{2.340521in}}%
\pgfpathlineto{\pgfqpoint{3.902211in}{2.336952in}}%
\pgfpathlineto{\pgfqpoint{3.894688in}{2.329655in}}%
\pgfpathlineto{\pgfqpoint{3.887159in}{2.322359in}}%
\pgfpathlineto{\pgfqpoint{3.879624in}{2.315066in}}%
\pgfpathlineto{\pgfqpoint{3.872084in}{2.307775in}}%
\pgfpathlineto{\pgfqpoint{3.859195in}{2.311397in}}%
\pgfpathlineto{\pgfqpoint{3.846312in}{2.315048in}}%
\pgfpathlineto{\pgfqpoint{3.833435in}{2.318730in}}%
\pgfpathlineto{\pgfqpoint{3.820563in}{2.322443in}}%
\pgfpathlineto{\pgfqpoint{3.828115in}{2.329676in}}%
\pgfpathlineto{\pgfqpoint{3.835662in}{2.336915in}}%
\pgfpathlineto{\pgfqpoint{3.843204in}{2.344159in}}%
\pgfpathlineto{\pgfqpoint{3.850740in}{2.351408in}}%
\pgfpathclose%
\pgfusepath{fill}%
\end{pgfscope}%
\begin{pgfscope}%
\pgfpathrectangle{\pgfqpoint{1.254980in}{0.150000in}}{\pgfqpoint{5.490039in}{5.490039in}}%
\pgfusepath{clip}%
\pgfsetbuttcap%
\pgfsetroundjoin%
\definecolor{currentfill}{rgb}{0.282656,0.100196,0.422160}%
\pgfsetfillcolor{currentfill}%
\pgfsetfillopacity{0.700000}%
\pgfsetlinewidth{0.000000pt}%
\definecolor{currentstroke}{rgb}{0.000000,0.000000,0.000000}%
\pgfsetstrokecolor{currentstroke}%
\pgfsetdash{}{0pt}%
\pgfpathmoveto{\pgfqpoint{5.270627in}{2.498247in}}%
\pgfpathlineto{\pgfqpoint{5.283839in}{2.496222in}}%
\pgfpathlineto{\pgfqpoint{5.297057in}{2.494221in}}%
\pgfpathlineto{\pgfqpoint{5.310283in}{2.492243in}}%
\pgfpathlineto{\pgfqpoint{5.323516in}{2.490289in}}%
\pgfpathlineto{\pgfqpoint{5.316543in}{2.484338in}}%
\pgfpathlineto{\pgfqpoint{5.309564in}{2.478400in}}%
\pgfpathlineto{\pgfqpoint{5.302581in}{2.472470in}}%
\pgfpathlineto{\pgfqpoint{5.295592in}{2.466546in}}%
\pgfpathlineto{\pgfqpoint{5.282343in}{2.468386in}}%
\pgfpathlineto{\pgfqpoint{5.269100in}{2.470251in}}%
\pgfpathlineto{\pgfqpoint{5.255866in}{2.472139in}}%
\pgfpathlineto{\pgfqpoint{5.242638in}{2.474051in}}%
\pgfpathlineto{\pgfqpoint{5.249643in}{2.480085in}}%
\pgfpathlineto{\pgfqpoint{5.256643in}{2.486126in}}%
\pgfpathlineto{\pgfqpoint{5.263638in}{2.492179in}}%
\pgfpathlineto{\pgfqpoint{5.270627in}{2.498247in}}%
\pgfpathclose%
\pgfusepath{fill}%
\end{pgfscope}%
\begin{pgfscope}%
\pgfpathrectangle{\pgfqpoint{1.254980in}{0.150000in}}{\pgfqpoint{5.490039in}{5.490039in}}%
\pgfusepath{clip}%
\pgfsetbuttcap%
\pgfsetroundjoin%
\definecolor{currentfill}{rgb}{0.269944,0.014625,0.341379}%
\pgfsetfillcolor{currentfill}%
\pgfsetfillopacity{0.700000}%
\pgfsetlinewidth{0.000000pt}%
\definecolor{currentstroke}{rgb}{0.000000,0.000000,0.000000}%
\pgfsetstrokecolor{currentstroke}%
\pgfsetdash{}{0pt}%
\pgfpathmoveto{\pgfqpoint{3.370083in}{2.360203in}}%
\pgfpathlineto{\pgfqpoint{3.382854in}{2.355328in}}%
\pgfpathlineto{\pgfqpoint{3.395629in}{2.350488in}}%
\pgfpathlineto{\pgfqpoint{3.408409in}{2.345685in}}%
\pgfpathlineto{\pgfqpoint{3.421194in}{2.340917in}}%
\pgfpathlineto{\pgfqpoint{3.413483in}{2.334486in}}%
\pgfpathlineto{\pgfqpoint{3.405766in}{2.328104in}}%
\pgfpathlineto{\pgfqpoint{3.398042in}{2.321771in}}%
\pgfpathlineto{\pgfqpoint{3.390312in}{2.315491in}}%
\pgfpathlineto{\pgfqpoint{3.377513in}{2.320363in}}%
\pgfpathlineto{\pgfqpoint{3.364718in}{2.325270in}}%
\pgfpathlineto{\pgfqpoint{3.351928in}{2.330214in}}%
\pgfpathlineto{\pgfqpoint{3.339142in}{2.335194in}}%
\pgfpathlineto{\pgfqpoint{3.346888in}{2.341365in}}%
\pgfpathlineto{\pgfqpoint{3.354626in}{2.347592in}}%
\pgfpathlineto{\pgfqpoint{3.362358in}{2.353872in}}%
\pgfpathlineto{\pgfqpoint{3.370083in}{2.360203in}}%
\pgfpathclose%
\pgfusepath{fill}%
\end{pgfscope}%
\begin{pgfscope}%
\pgfpathrectangle{\pgfqpoint{1.254980in}{0.150000in}}{\pgfqpoint{5.490039in}{5.490039in}}%
\pgfusepath{clip}%
\pgfsetbuttcap%
\pgfsetroundjoin%
\definecolor{currentfill}{rgb}{0.271305,0.019942,0.347269}%
\pgfsetfillcolor{currentfill}%
\pgfsetfillopacity{0.700000}%
\pgfsetlinewidth{0.000000pt}%
\definecolor{currentstroke}{rgb}{0.000000,0.000000,0.000000}%
\pgfsetstrokecolor{currentstroke}%
\pgfsetdash{}{0pt}%
\pgfpathmoveto{\pgfqpoint{3.237014in}{2.376368in}}%
\pgfpathlineto{\pgfqpoint{3.249765in}{2.371089in}}%
\pgfpathlineto{\pgfqpoint{3.262520in}{2.365848in}}%
\pgfpathlineto{\pgfqpoint{3.275280in}{2.360645in}}%
\pgfpathlineto{\pgfqpoint{3.288043in}{2.355480in}}%
\pgfpathlineto{\pgfqpoint{3.280276in}{2.349479in}}%
\pgfpathlineto{\pgfqpoint{3.272501in}{2.343542in}}%
\pgfpathlineto{\pgfqpoint{3.264718in}{2.337672in}}%
\pgfpathlineto{\pgfqpoint{3.256929in}{2.331872in}}%
\pgfpathlineto{\pgfqpoint{3.244149in}{2.337153in}}%
\pgfpathlineto{\pgfqpoint{3.231374in}{2.342473in}}%
\pgfpathlineto{\pgfqpoint{3.218603in}{2.347831in}}%
\pgfpathlineto{\pgfqpoint{3.205835in}{2.353228in}}%
\pgfpathlineto{\pgfqpoint{3.213641in}{2.358907in}}%
\pgfpathlineto{\pgfqpoint{3.221439in}{2.364658in}}%
\pgfpathlineto{\pgfqpoint{3.229230in}{2.370480in}}%
\pgfpathlineto{\pgfqpoint{3.237014in}{2.376368in}}%
\pgfpathclose%
\pgfusepath{fill}%
\end{pgfscope}%
\begin{pgfscope}%
\pgfpathrectangle{\pgfqpoint{1.254980in}{0.150000in}}{\pgfqpoint{5.490039in}{5.490039in}}%
\pgfusepath{clip}%
\pgfsetbuttcap%
\pgfsetroundjoin%
\definecolor{currentfill}{rgb}{0.281446,0.084320,0.407414}%
\pgfsetfillcolor{currentfill}%
\pgfsetfillopacity{0.700000}%
\pgfsetlinewidth{0.000000pt}%
\definecolor{currentstroke}{rgb}{0.000000,0.000000,0.000000}%
\pgfsetstrokecolor{currentstroke}%
\pgfsetdash{}{0pt}%
\pgfpathmoveto{\pgfqpoint{5.056213in}{2.473536in}}%
\pgfpathlineto{\pgfqpoint{5.069369in}{2.471436in}}%
\pgfpathlineto{\pgfqpoint{5.082533in}{2.469361in}}%
\pgfpathlineto{\pgfqpoint{5.095704in}{2.467309in}}%
\pgfpathlineto{\pgfqpoint{5.108881in}{2.465282in}}%
\pgfpathlineto{\pgfqpoint{5.101819in}{2.459053in}}%
\pgfpathlineto{\pgfqpoint{5.094750in}{2.452818in}}%
\pgfpathlineto{\pgfqpoint{5.087676in}{2.446575in}}%
\pgfpathlineto{\pgfqpoint{5.080597in}{2.440321in}}%
\pgfpathlineto{\pgfqpoint{5.067404in}{2.442260in}}%
\pgfpathlineto{\pgfqpoint{5.054218in}{2.444224in}}%
\pgfpathlineto{\pgfqpoint{5.041040in}{2.446212in}}%
\pgfpathlineto{\pgfqpoint{5.027869in}{2.448225in}}%
\pgfpathlineto{\pgfqpoint{5.034963in}{2.454562in}}%
\pgfpathlineto{\pgfqpoint{5.042052in}{2.460891in}}%
\pgfpathlineto{\pgfqpoint{5.049135in}{2.467215in}}%
\pgfpathlineto{\pgfqpoint{5.056213in}{2.473536in}}%
\pgfpathclose%
\pgfusepath{fill}%
\end{pgfscope}%
\begin{pgfscope}%
\pgfpathrectangle{\pgfqpoint{1.254980in}{0.150000in}}{\pgfqpoint{5.490039in}{5.490039in}}%
\pgfusepath{clip}%
\pgfsetbuttcap%
\pgfsetroundjoin%
\definecolor{currentfill}{rgb}{0.279566,0.067836,0.391917}%
\pgfsetfillcolor{currentfill}%
\pgfsetfillopacity{0.700000}%
\pgfsetlinewidth{0.000000pt}%
\definecolor{currentstroke}{rgb}{0.000000,0.000000,0.000000}%
\pgfsetstrokecolor{currentstroke}%
\pgfsetdash{}{0pt}%
\pgfpathmoveto{\pgfqpoint{4.841760in}{2.447902in}}%
\pgfpathlineto{\pgfqpoint{4.854861in}{2.445670in}}%
\pgfpathlineto{\pgfqpoint{4.867969in}{2.443462in}}%
\pgfpathlineto{\pgfqpoint{4.881084in}{2.441279in}}%
\pgfpathlineto{\pgfqpoint{4.894206in}{2.439121in}}%
\pgfpathlineto{\pgfqpoint{4.887055in}{2.432569in}}%
\pgfpathlineto{\pgfqpoint{4.879899in}{2.426001in}}%
\pgfpathlineto{\pgfqpoint{4.872736in}{2.419413in}}%
\pgfpathlineto{\pgfqpoint{4.865569in}{2.412803in}}%
\pgfpathlineto{\pgfqpoint{4.852433in}{2.414900in}}%
\pgfpathlineto{\pgfqpoint{4.839304in}{2.417021in}}%
\pgfpathlineto{\pgfqpoint{4.826183in}{2.419166in}}%
\pgfpathlineto{\pgfqpoint{4.813068in}{2.421337in}}%
\pgfpathlineto{\pgfqpoint{4.820249in}{2.428004in}}%
\pgfpathlineto{\pgfqpoint{4.827425in}{2.434652in}}%
\pgfpathlineto{\pgfqpoint{4.834595in}{2.441284in}}%
\pgfpathlineto{\pgfqpoint{4.841760in}{2.447902in}}%
\pgfpathclose%
\pgfusepath{fill}%
\end{pgfscope}%
\begin{pgfscope}%
\pgfpathrectangle{\pgfqpoint{1.254980in}{0.150000in}}{\pgfqpoint{5.490039in}{5.490039in}}%
\pgfusepath{clip}%
\pgfsetbuttcap%
\pgfsetroundjoin%
\definecolor{currentfill}{rgb}{0.277941,0.056324,0.381191}%
\pgfsetfillcolor{currentfill}%
\pgfsetfillopacity{0.700000}%
\pgfsetlinewidth{0.000000pt}%
\definecolor{currentstroke}{rgb}{0.000000,0.000000,0.000000}%
\pgfsetstrokecolor{currentstroke}%
\pgfsetdash{}{0pt}%
\pgfpathmoveto{\pgfqpoint{4.627286in}{2.421866in}}%
\pgfpathlineto{\pgfqpoint{4.640331in}{2.419441in}}%
\pgfpathlineto{\pgfqpoint{4.653384in}{2.417042in}}%
\pgfpathlineto{\pgfqpoint{4.666444in}{2.414668in}}%
\pgfpathlineto{\pgfqpoint{4.679510in}{2.412320in}}%
\pgfpathlineto{\pgfqpoint{4.672274in}{2.405449in}}%
\pgfpathlineto{\pgfqpoint{4.665032in}{2.398555in}}%
\pgfpathlineto{\pgfqpoint{4.657785in}{2.391635in}}%
\pgfpathlineto{\pgfqpoint{4.650532in}{2.384690in}}%
\pgfpathlineto{\pgfqpoint{4.637453in}{2.387002in}}%
\pgfpathlineto{\pgfqpoint{4.624381in}{2.389339in}}%
\pgfpathlineto{\pgfqpoint{4.611315in}{2.391702in}}%
\pgfpathlineto{\pgfqpoint{4.598256in}{2.394091in}}%
\pgfpathlineto{\pgfqpoint{4.605522in}{2.401068in}}%
\pgfpathlineto{\pgfqpoint{4.612782in}{2.408022in}}%
\pgfpathlineto{\pgfqpoint{4.620037in}{2.414954in}}%
\pgfpathlineto{\pgfqpoint{4.627286in}{2.421866in}}%
\pgfpathclose%
\pgfusepath{fill}%
\end{pgfscope}%
\begin{pgfscope}%
\pgfpathrectangle{\pgfqpoint{1.254980in}{0.150000in}}{\pgfqpoint{5.490039in}{5.490039in}}%
\pgfusepath{clip}%
\pgfsetbuttcap%
\pgfsetroundjoin%
\definecolor{currentfill}{rgb}{0.274952,0.037752,0.364543}%
\pgfsetfillcolor{currentfill}%
\pgfsetfillopacity{0.700000}%
\pgfsetlinewidth{0.000000pt}%
\definecolor{currentstroke}{rgb}{0.000000,0.000000,0.000000}%
\pgfsetstrokecolor{currentstroke}%
\pgfsetdash{}{0pt}%
\pgfpathmoveto{\pgfqpoint{4.412801in}{2.396331in}}%
\pgfpathlineto{\pgfqpoint{4.425793in}{2.393653in}}%
\pgfpathlineto{\pgfqpoint{4.438791in}{2.391001in}}%
\pgfpathlineto{\pgfqpoint{4.451796in}{2.388375in}}%
\pgfpathlineto{\pgfqpoint{4.464808in}{2.385776in}}%
\pgfpathlineto{\pgfqpoint{4.457490in}{2.378633in}}%
\pgfpathlineto{\pgfqpoint{4.450166in}{2.371466in}}%
\pgfpathlineto{\pgfqpoint{4.442836in}{2.364275in}}%
\pgfpathlineto{\pgfqpoint{4.435501in}{2.357058in}}%
\pgfpathlineto{\pgfqpoint{4.422478in}{2.359646in}}%
\pgfpathlineto{\pgfqpoint{4.409460in}{2.362261in}}%
\pgfpathlineto{\pgfqpoint{4.396450in}{2.364902in}}%
\pgfpathlineto{\pgfqpoint{4.383445in}{2.367570in}}%
\pgfpathlineto{\pgfqpoint{4.390792in}{2.374793in}}%
\pgfpathlineto{\pgfqpoint{4.398134in}{2.381994in}}%
\pgfpathlineto{\pgfqpoint{4.405470in}{2.389173in}}%
\pgfpathlineto{\pgfqpoint{4.412801in}{2.396331in}}%
\pgfpathclose%
\pgfusepath{fill}%
\end{pgfscope}%
\begin{pgfscope}%
\pgfpathrectangle{\pgfqpoint{1.254980in}{0.150000in}}{\pgfqpoint{5.490039in}{5.490039in}}%
\pgfusepath{clip}%
\pgfsetbuttcap%
\pgfsetroundjoin%
\definecolor{currentfill}{rgb}{0.281924,0.089666,0.412415}%
\pgfsetfillcolor{currentfill}%
\pgfsetfillopacity{0.700000}%
\pgfsetlinewidth{0.000000pt}%
\definecolor{currentstroke}{rgb}{0.000000,0.000000,0.000000}%
\pgfsetstrokecolor{currentstroke}%
\pgfsetdash{}{0pt}%
\pgfpathmoveto{\pgfqpoint{2.786115in}{2.486624in}}%
\pgfpathlineto{\pgfqpoint{2.798822in}{2.479722in}}%
\pgfpathlineto{\pgfqpoint{2.811533in}{2.472870in}}%
\pgfpathlineto{\pgfqpoint{2.824246in}{2.466067in}}%
\pgfpathlineto{\pgfqpoint{2.836961in}{2.459314in}}%
\pgfpathlineto{\pgfqpoint{2.828969in}{2.455332in}}%
\pgfpathlineto{\pgfqpoint{2.820968in}{2.451475in}}%
\pgfpathlineto{\pgfqpoint{2.812955in}{2.447745in}}%
\pgfpathlineto{\pgfqpoint{2.804933in}{2.444148in}}%
\pgfpathlineto{\pgfqpoint{2.792197in}{2.451057in}}%
\pgfpathlineto{\pgfqpoint{2.779463in}{2.458016in}}%
\pgfpathlineto{\pgfqpoint{2.766732in}{2.465025in}}%
\pgfpathlineto{\pgfqpoint{2.754004in}{2.472084in}}%
\pgfpathlineto{\pgfqpoint{2.762047in}{2.475520in}}%
\pgfpathlineto{\pgfqpoint{2.770080in}{2.479091in}}%
\pgfpathlineto{\pgfqpoint{2.778103in}{2.482794in}}%
\pgfpathlineto{\pgfqpoint{2.786115in}{2.486624in}}%
\pgfpathclose%
\pgfusepath{fill}%
\end{pgfscope}%
\begin{pgfscope}%
\pgfpathrectangle{\pgfqpoint{1.254980in}{0.150000in}}{\pgfqpoint{5.490039in}{5.490039in}}%
\pgfusepath{clip}%
\pgfsetbuttcap%
\pgfsetroundjoin%
\definecolor{currentfill}{rgb}{0.268510,0.009605,0.335427}%
\pgfsetfillcolor{currentfill}%
\pgfsetfillopacity{0.700000}%
\pgfsetlinewidth{0.000000pt}%
\definecolor{currentstroke}{rgb}{0.000000,0.000000,0.000000}%
\pgfsetstrokecolor{currentstroke}%
\pgfsetdash{}{0pt}%
\pgfpathmoveto{\pgfqpoint{3.503099in}{2.348722in}}%
\pgfpathlineto{\pgfqpoint{3.515893in}{2.344221in}}%
\pgfpathlineto{\pgfqpoint{3.528692in}{2.339753in}}%
\pgfpathlineto{\pgfqpoint{3.541496in}{2.335319in}}%
\pgfpathlineto{\pgfqpoint{3.554305in}{2.330919in}}%
\pgfpathlineto{\pgfqpoint{3.546648in}{2.324145in}}%
\pgfpathlineto{\pgfqpoint{3.538984in}{2.317404in}}%
\pgfpathlineto{\pgfqpoint{3.531315in}{2.310698in}}%
\pgfpathlineto{\pgfqpoint{3.523639in}{2.304029in}}%
\pgfpathlineto{\pgfqpoint{3.510817in}{2.308521in}}%
\pgfpathlineto{\pgfqpoint{3.497999in}{2.313046in}}%
\pgfpathlineto{\pgfqpoint{3.485186in}{2.317604in}}%
\pgfpathlineto{\pgfqpoint{3.472378in}{2.322198in}}%
\pgfpathlineto{\pgfqpoint{3.480068in}{2.328770in}}%
\pgfpathlineto{\pgfqpoint{3.487751in}{2.335383in}}%
\pgfpathlineto{\pgfqpoint{3.495428in}{2.342035in}}%
\pgfpathlineto{\pgfqpoint{3.503099in}{2.348722in}}%
\pgfpathclose%
\pgfusepath{fill}%
\end{pgfscope}%
\begin{pgfscope}%
\pgfpathrectangle{\pgfqpoint{1.254980in}{0.150000in}}{\pgfqpoint{5.490039in}{5.490039in}}%
\pgfusepath{clip}%
\pgfsetbuttcap%
\pgfsetroundjoin%
\definecolor{currentfill}{rgb}{0.274952,0.037752,0.364543}%
\pgfsetfillcolor{currentfill}%
\pgfsetfillopacity{0.700000}%
\pgfsetlinewidth{0.000000pt}%
\definecolor{currentstroke}{rgb}{0.000000,0.000000,0.000000}%
\pgfsetstrokecolor{currentstroke}%
\pgfsetdash{}{0pt}%
\pgfpathmoveto{\pgfqpoint{3.103842in}{2.397827in}}%
\pgfpathlineto{\pgfqpoint{3.116578in}{2.392110in}}%
\pgfpathlineto{\pgfqpoint{3.129317in}{2.386435in}}%
\pgfpathlineto{\pgfqpoint{3.142060in}{2.380800in}}%
\pgfpathlineto{\pgfqpoint{3.154807in}{2.375206in}}%
\pgfpathlineto{\pgfqpoint{3.146977in}{2.369727in}}%
\pgfpathlineto{\pgfqpoint{3.139140in}{2.364330in}}%
\pgfpathlineto{\pgfqpoint{3.131294in}{2.359017in}}%
\pgfpathlineto{\pgfqpoint{3.123440in}{2.353793in}}%
\pgfpathlineto{\pgfqpoint{3.110676in}{2.359517in}}%
\pgfpathlineto{\pgfqpoint{3.097916in}{2.365281in}}%
\pgfpathlineto{\pgfqpoint{3.085159in}{2.371086in}}%
\pgfpathlineto{\pgfqpoint{3.072406in}{2.376933in}}%
\pgfpathlineto{\pgfqpoint{3.080278in}{2.382023in}}%
\pgfpathlineto{\pgfqpoint{3.088140in}{2.387204in}}%
\pgfpathlineto{\pgfqpoint{3.095995in}{2.392473in}}%
\pgfpathlineto{\pgfqpoint{3.103842in}{2.397827in}}%
\pgfpathclose%
\pgfusepath{fill}%
\end{pgfscope}%
\begin{pgfscope}%
\pgfpathrectangle{\pgfqpoint{1.254980in}{0.150000in}}{\pgfqpoint{5.490039in}{5.490039in}}%
\pgfusepath{clip}%
\pgfsetbuttcap%
\pgfsetroundjoin%
\definecolor{currentfill}{rgb}{0.271305,0.019942,0.347269}%
\pgfsetfillcolor{currentfill}%
\pgfsetfillopacity{0.700000}%
\pgfsetlinewidth{0.000000pt}%
\definecolor{currentstroke}{rgb}{0.000000,0.000000,0.000000}%
\pgfsetstrokecolor{currentstroke}%
\pgfsetdash{}{0pt}%
\pgfpathmoveto{\pgfqpoint{4.198302in}{2.372589in}}%
\pgfpathlineto{\pgfqpoint{4.211243in}{2.369593in}}%
\pgfpathlineto{\pgfqpoint{4.224190in}{2.366624in}}%
\pgfpathlineto{\pgfqpoint{4.237143in}{2.363684in}}%
\pgfpathlineto{\pgfqpoint{4.250102in}{2.360770in}}%
\pgfpathlineto{\pgfqpoint{4.242704in}{2.353448in}}%
\pgfpathlineto{\pgfqpoint{4.235300in}{2.346107in}}%
\pgfpathlineto{\pgfqpoint{4.227891in}{2.338747in}}%
\pgfpathlineto{\pgfqpoint{4.220476in}{2.331368in}}%
\pgfpathlineto{\pgfqpoint{4.207505in}{2.334296in}}%
\pgfpathlineto{\pgfqpoint{4.194540in}{2.337251in}}%
\pgfpathlineto{\pgfqpoint{4.181581in}{2.340234in}}%
\pgfpathlineto{\pgfqpoint{4.168629in}{2.343245in}}%
\pgfpathlineto{\pgfqpoint{4.176055in}{2.350604in}}%
\pgfpathlineto{\pgfqpoint{4.183476in}{2.357948in}}%
\pgfpathlineto{\pgfqpoint{4.190892in}{2.365276in}}%
\pgfpathlineto{\pgfqpoint{4.198302in}{2.372589in}}%
\pgfpathclose%
\pgfusepath{fill}%
\end{pgfscope}%
\begin{pgfscope}%
\pgfpathrectangle{\pgfqpoint{1.254980in}{0.150000in}}{\pgfqpoint{5.490039in}{5.490039in}}%
\pgfusepath{clip}%
\pgfsetbuttcap%
\pgfsetroundjoin%
\definecolor{currentfill}{rgb}{0.267004,0.004874,0.329415}%
\pgfsetfillcolor{currentfill}%
\pgfsetfillopacity{0.700000}%
\pgfsetlinewidth{0.000000pt}%
\definecolor{currentstroke}{rgb}{0.000000,0.000000,0.000000}%
\pgfsetstrokecolor{currentstroke}%
\pgfsetdash{}{0pt}%
\pgfpathmoveto{\pgfqpoint{3.636103in}{2.341362in}}%
\pgfpathlineto{\pgfqpoint{3.648924in}{2.337206in}}%
\pgfpathlineto{\pgfqpoint{3.661749in}{2.333083in}}%
\pgfpathlineto{\pgfqpoint{3.674580in}{2.328991in}}%
\pgfpathlineto{\pgfqpoint{3.687416in}{2.324932in}}%
\pgfpathlineto{\pgfqpoint{3.679810in}{2.317895in}}%
\pgfpathlineto{\pgfqpoint{3.672198in}{2.310877in}}%
\pgfpathlineto{\pgfqpoint{3.664579in}{2.303882in}}%
\pgfpathlineto{\pgfqpoint{3.656955in}{2.296909in}}%
\pgfpathlineto{\pgfqpoint{3.644106in}{2.301047in}}%
\pgfpathlineto{\pgfqpoint{3.631262in}{2.305216in}}%
\pgfpathlineto{\pgfqpoint{3.618423in}{2.309418in}}%
\pgfpathlineto{\pgfqpoint{3.605589in}{2.313653in}}%
\pgfpathlineto{\pgfqpoint{3.613227in}{2.320542in}}%
\pgfpathlineto{\pgfqpoint{3.620858in}{2.327457in}}%
\pgfpathlineto{\pgfqpoint{3.628483in}{2.334398in}}%
\pgfpathlineto{\pgfqpoint{3.636103in}{2.341362in}}%
\pgfpathclose%
\pgfusepath{fill}%
\end{pgfscope}%
\begin{pgfscope}%
\pgfpathrectangle{\pgfqpoint{1.254980in}{0.150000in}}{\pgfqpoint{5.490039in}{5.490039in}}%
\pgfusepath{clip}%
\pgfsetbuttcap%
\pgfsetroundjoin%
\definecolor{currentfill}{rgb}{0.269944,0.014625,0.341379}%
\pgfsetfillcolor{currentfill}%
\pgfsetfillopacity{0.700000}%
\pgfsetlinewidth{0.000000pt}%
\definecolor{currentstroke}{rgb}{0.000000,0.000000,0.000000}%
\pgfsetstrokecolor{currentstroke}%
\pgfsetdash{}{0pt}%
\pgfpathmoveto{\pgfqpoint{3.983764in}{2.352319in}}%
\pgfpathlineto{\pgfqpoint{3.996657in}{2.348938in}}%
\pgfpathlineto{\pgfqpoint{4.009556in}{2.345587in}}%
\pgfpathlineto{\pgfqpoint{4.022460in}{2.342265in}}%
\pgfpathlineto{\pgfqpoint{4.035371in}{2.338972in}}%
\pgfpathlineto{\pgfqpoint{4.027893in}{2.331607in}}%
\pgfpathlineto{\pgfqpoint{4.020410in}{2.324234in}}%
\pgfpathlineto{\pgfqpoint{4.012922in}{2.316853in}}%
\pgfpathlineto{\pgfqpoint{4.005428in}{2.309466in}}%
\pgfpathlineto{\pgfqpoint{3.992505in}{2.312800in}}%
\pgfpathlineto{\pgfqpoint{3.979589in}{2.316162in}}%
\pgfpathlineto{\pgfqpoint{3.966678in}{2.319554in}}%
\pgfpathlineto{\pgfqpoint{3.953773in}{2.322974in}}%
\pgfpathlineto{\pgfqpoint{3.961279in}{2.330316in}}%
\pgfpathlineto{\pgfqpoint{3.968780in}{2.337655in}}%
\pgfpathlineto{\pgfqpoint{3.976275in}{2.344989in}}%
\pgfpathlineto{\pgfqpoint{3.983764in}{2.352319in}}%
\pgfpathclose%
\pgfusepath{fill}%
\end{pgfscope}%
\begin{pgfscope}%
\pgfpathrectangle{\pgfqpoint{1.254980in}{0.150000in}}{\pgfqpoint{5.490039in}{5.490039in}}%
\pgfusepath{clip}%
\pgfsetbuttcap%
\pgfsetroundjoin%
\definecolor{currentfill}{rgb}{0.277941,0.056324,0.381191}%
\pgfsetfillcolor{currentfill}%
\pgfsetfillopacity{0.700000}%
\pgfsetlinewidth{0.000000pt}%
\definecolor{currentstroke}{rgb}{0.000000,0.000000,0.000000}%
\pgfsetstrokecolor{currentstroke}%
\pgfsetdash{}{0pt}%
\pgfpathmoveto{\pgfqpoint{2.970513in}{2.425237in}}%
\pgfpathlineto{\pgfqpoint{2.983237in}{2.419047in}}%
\pgfpathlineto{\pgfqpoint{2.995965in}{2.412901in}}%
\pgfpathlineto{\pgfqpoint{3.008696in}{2.406799in}}%
\pgfpathlineto{\pgfqpoint{3.021431in}{2.400740in}}%
\pgfpathlineto{\pgfqpoint{3.013534in}{2.395882in}}%
\pgfpathlineto{\pgfqpoint{3.005627in}{2.391124in}}%
\pgfpathlineto{\pgfqpoint{2.997712in}{2.386470in}}%
\pgfpathlineto{\pgfqpoint{2.989788in}{2.381924in}}%
\pgfpathlineto{\pgfqpoint{2.977035in}{2.388125in}}%
\pgfpathlineto{\pgfqpoint{2.964285in}{2.394371in}}%
\pgfpathlineto{\pgfqpoint{2.951538in}{2.400660in}}%
\pgfpathlineto{\pgfqpoint{2.938795in}{2.406993in}}%
\pgfpathlineto{\pgfqpoint{2.946738in}{2.411392in}}%
\pgfpathlineto{\pgfqpoint{2.954672in}{2.415901in}}%
\pgfpathlineto{\pgfqpoint{2.962597in}{2.420517in}}%
\pgfpathlineto{\pgfqpoint{2.970513in}{2.425237in}}%
\pgfpathclose%
\pgfusepath{fill}%
\end{pgfscope}%
\begin{pgfscope}%
\pgfpathrectangle{\pgfqpoint{1.254980in}{0.150000in}}{\pgfqpoint{5.490039in}{5.490039in}}%
\pgfusepath{clip}%
\pgfsetbuttcap%
\pgfsetroundjoin%
\definecolor{currentfill}{rgb}{0.283229,0.120777,0.440584}%
\pgfsetfillcolor{currentfill}%
\pgfsetfillopacity{0.700000}%
\pgfsetlinewidth{0.000000pt}%
\definecolor{currentstroke}{rgb}{0.000000,0.000000,0.000000}%
\pgfsetstrokecolor{currentstroke}%
\pgfsetdash{}{0pt}%
\pgfpathmoveto{\pgfqpoint{5.618734in}{2.529376in}}%
\pgfpathlineto{\pgfqpoint{5.632040in}{2.527404in}}%
\pgfpathlineto{\pgfqpoint{5.645353in}{2.525454in}}%
\pgfpathlineto{\pgfqpoint{5.658674in}{2.523527in}}%
\pgfpathlineto{\pgfqpoint{5.672002in}{2.521624in}}%
\pgfpathlineto{\pgfqpoint{5.665172in}{2.515988in}}%
\pgfpathlineto{\pgfqpoint{5.658338in}{2.510400in}}%
\pgfpathlineto{\pgfqpoint{5.651500in}{2.504855in}}%
\pgfpathlineto{\pgfqpoint{5.644658in}{2.499348in}}%
\pgfpathlineto{\pgfqpoint{5.631310in}{2.501100in}}%
\pgfpathlineto{\pgfqpoint{5.617970in}{2.502875in}}%
\pgfpathlineto{\pgfqpoint{5.604638in}{2.504673in}}%
\pgfpathlineto{\pgfqpoint{5.591314in}{2.506494in}}%
\pgfpathlineto{\pgfqpoint{5.598175in}{2.512148in}}%
\pgfpathlineto{\pgfqpoint{5.605032in}{2.517843in}}%
\pgfpathlineto{\pgfqpoint{5.611885in}{2.523584in}}%
\pgfpathlineto{\pgfqpoint{5.618734in}{2.529376in}}%
\pgfpathclose%
\pgfusepath{fill}%
\end{pgfscope}%
\begin{pgfscope}%
\pgfpathrectangle{\pgfqpoint{1.254980in}{0.150000in}}{\pgfqpoint{5.490039in}{5.490039in}}%
\pgfusepath{clip}%
\pgfsetbuttcap%
\pgfsetroundjoin%
\definecolor{currentfill}{rgb}{0.283091,0.110553,0.431554}%
\pgfsetfillcolor{currentfill}%
\pgfsetfillopacity{0.700000}%
\pgfsetlinewidth{0.000000pt}%
\definecolor{currentstroke}{rgb}{0.000000,0.000000,0.000000}%
\pgfsetstrokecolor{currentstroke}%
\pgfsetdash{}{0pt}%
\pgfpathmoveto{\pgfqpoint{5.404299in}{2.506204in}}%
\pgfpathlineto{\pgfqpoint{5.417551in}{2.504242in}}%
\pgfpathlineto{\pgfqpoint{5.430812in}{2.502304in}}%
\pgfpathlineto{\pgfqpoint{5.444079in}{2.500388in}}%
\pgfpathlineto{\pgfqpoint{5.457354in}{2.498496in}}%
\pgfpathlineto{\pgfqpoint{5.450435in}{2.492720in}}%
\pgfpathlineto{\pgfqpoint{5.443511in}{2.486966in}}%
\pgfpathlineto{\pgfqpoint{5.436581in}{2.481229in}}%
\pgfpathlineto{\pgfqpoint{5.429647in}{2.475505in}}%
\pgfpathlineto{\pgfqpoint{5.416355in}{2.477271in}}%
\pgfpathlineto{\pgfqpoint{5.403070in}{2.479060in}}%
\pgfpathlineto{\pgfqpoint{5.389792in}{2.480873in}}%
\pgfpathlineto{\pgfqpoint{5.376522in}{2.482709in}}%
\pgfpathlineto{\pgfqpoint{5.383473in}{2.488554in}}%
\pgfpathlineto{\pgfqpoint{5.390420in}{2.494415in}}%
\pgfpathlineto{\pgfqpoint{5.397362in}{2.500298in}}%
\pgfpathlineto{\pgfqpoint{5.404299in}{2.506204in}}%
\pgfpathclose%
\pgfusepath{fill}%
\end{pgfscope}%
\begin{pgfscope}%
\pgfpathrectangle{\pgfqpoint{1.254980in}{0.150000in}}{\pgfqpoint{5.490039in}{5.490039in}}%
\pgfusepath{clip}%
\pgfsetbuttcap%
\pgfsetroundjoin%
\definecolor{currentfill}{rgb}{0.282327,0.094955,0.417331}%
\pgfsetfillcolor{currentfill}%
\pgfsetfillopacity{0.700000}%
\pgfsetlinewidth{0.000000pt}%
\definecolor{currentstroke}{rgb}{0.000000,0.000000,0.000000}%
\pgfsetstrokecolor{currentstroke}%
\pgfsetdash{}{0pt}%
\pgfpathmoveto{\pgfqpoint{5.189801in}{2.481938in}}%
\pgfpathlineto{\pgfqpoint{5.203000in}{2.479930in}}%
\pgfpathlineto{\pgfqpoint{5.216205in}{2.477947in}}%
\pgfpathlineto{\pgfqpoint{5.229418in}{2.475987in}}%
\pgfpathlineto{\pgfqpoint{5.242638in}{2.474051in}}%
\pgfpathlineto{\pgfqpoint{5.235628in}{2.468022in}}%
\pgfpathlineto{\pgfqpoint{5.228612in}{2.461995in}}%
\pgfpathlineto{\pgfqpoint{5.221592in}{2.455965in}}%
\pgfpathlineto{\pgfqpoint{5.214565in}{2.449929in}}%
\pgfpathlineto{\pgfqpoint{5.201329in}{2.451764in}}%
\pgfpathlineto{\pgfqpoint{5.188100in}{2.453623in}}%
\pgfpathlineto{\pgfqpoint{5.174879in}{2.455507in}}%
\pgfpathlineto{\pgfqpoint{5.161665in}{2.457413in}}%
\pgfpathlineto{\pgfqpoint{5.168707in}{2.463545in}}%
\pgfpathlineto{\pgfqpoint{5.175744in}{2.469674in}}%
\pgfpathlineto{\pgfqpoint{5.182775in}{2.475804in}}%
\pgfpathlineto{\pgfqpoint{5.189801in}{2.481938in}}%
\pgfpathclose%
\pgfusepath{fill}%
\end{pgfscope}%
\begin{pgfscope}%
\pgfpathrectangle{\pgfqpoint{1.254980in}{0.150000in}}{\pgfqpoint{5.490039in}{5.490039in}}%
\pgfusepath{clip}%
\pgfsetbuttcap%
\pgfsetroundjoin%
\definecolor{currentfill}{rgb}{0.267004,0.004874,0.329415}%
\pgfsetfillcolor{currentfill}%
\pgfsetfillopacity{0.700000}%
\pgfsetlinewidth{0.000000pt}%
\definecolor{currentstroke}{rgb}{0.000000,0.000000,0.000000}%
\pgfsetstrokecolor{currentstroke}%
\pgfsetdash{}{0pt}%
\pgfpathmoveto{\pgfqpoint{3.769130in}{2.337599in}}%
\pgfpathlineto{\pgfqpoint{3.781980in}{2.333764in}}%
\pgfpathlineto{\pgfqpoint{3.794836in}{2.329959in}}%
\pgfpathlineto{\pgfqpoint{3.807697in}{2.326186in}}%
\pgfpathlineto{\pgfqpoint{3.820563in}{2.322443in}}%
\pgfpathlineto{\pgfqpoint{3.813005in}{2.315217in}}%
\pgfpathlineto{\pgfqpoint{3.805441in}{2.307999in}}%
\pgfpathlineto{\pgfqpoint{3.797872in}{2.300791in}}%
\pgfpathlineto{\pgfqpoint{3.790297in}{2.293594in}}%
\pgfpathlineto{\pgfqpoint{3.777418in}{2.297403in}}%
\pgfpathlineto{\pgfqpoint{3.764544in}{2.301242in}}%
\pgfpathlineto{\pgfqpoint{3.751676in}{2.305112in}}%
\pgfpathlineto{\pgfqpoint{3.738813in}{2.309014in}}%
\pgfpathlineto{\pgfqpoint{3.746401in}{2.316140in}}%
\pgfpathlineto{\pgfqpoint{3.753983in}{2.323281in}}%
\pgfpathlineto{\pgfqpoint{3.761560in}{2.330435in}}%
\pgfpathlineto{\pgfqpoint{3.769130in}{2.337599in}}%
\pgfpathclose%
\pgfusepath{fill}%
\end{pgfscope}%
\begin{pgfscope}%
\pgfpathrectangle{\pgfqpoint{1.254980in}{0.150000in}}{\pgfqpoint{5.490039in}{5.490039in}}%
\pgfusepath{clip}%
\pgfsetbuttcap%
\pgfsetroundjoin%
\definecolor{currentfill}{rgb}{0.280894,0.078907,0.402329}%
\pgfsetfillcolor{currentfill}%
\pgfsetfillopacity{0.700000}%
\pgfsetlinewidth{0.000000pt}%
\definecolor{currentstroke}{rgb}{0.000000,0.000000,0.000000}%
\pgfsetstrokecolor{currentstroke}%
\pgfsetdash{}{0pt}%
\pgfpathmoveto{\pgfqpoint{4.975255in}{2.456518in}}%
\pgfpathlineto{\pgfqpoint{4.988398in}{2.454408in}}%
\pgfpathlineto{\pgfqpoint{5.001547in}{2.452322in}}%
\pgfpathlineto{\pgfqpoint{5.014704in}{2.450261in}}%
\pgfpathlineto{\pgfqpoint{5.027869in}{2.448225in}}%
\pgfpathlineto{\pgfqpoint{5.020769in}{2.441877in}}%
\pgfpathlineto{\pgfqpoint{5.013663in}{2.435515in}}%
\pgfpathlineto{\pgfqpoint{5.006552in}{2.429138in}}%
\pgfpathlineto{\pgfqpoint{4.999435in}{2.422741in}}%
\pgfpathlineto{\pgfqpoint{4.986257in}{2.424703in}}%
\pgfpathlineto{\pgfqpoint{4.973085in}{2.426689in}}%
\pgfpathlineto{\pgfqpoint{4.959921in}{2.428700in}}%
\pgfpathlineto{\pgfqpoint{4.946764in}{2.430735in}}%
\pgfpathlineto{\pgfqpoint{4.953895in}{2.437201in}}%
\pgfpathlineto{\pgfqpoint{4.961020in}{2.443652in}}%
\pgfpathlineto{\pgfqpoint{4.968140in}{2.450090in}}%
\pgfpathlineto{\pgfqpoint{4.975255in}{2.456518in}}%
\pgfpathclose%
\pgfusepath{fill}%
\end{pgfscope}%
\begin{pgfscope}%
\pgfpathrectangle{\pgfqpoint{1.254980in}{0.150000in}}{\pgfqpoint{5.490039in}{5.490039in}}%
\pgfusepath{clip}%
\pgfsetbuttcap%
\pgfsetroundjoin%
\definecolor{currentfill}{rgb}{0.279566,0.067836,0.391917}%
\pgfsetfillcolor{currentfill}%
\pgfsetfillopacity{0.700000}%
\pgfsetlinewidth{0.000000pt}%
\definecolor{currentstroke}{rgb}{0.000000,0.000000,0.000000}%
\pgfsetstrokecolor{currentstroke}%
\pgfsetdash{}{0pt}%
\pgfpathmoveto{\pgfqpoint{4.760678in}{2.430270in}}%
\pgfpathlineto{\pgfqpoint{4.773765in}{2.427999in}}%
\pgfpathlineto{\pgfqpoint{4.786859in}{2.425753in}}%
\pgfpathlineto{\pgfqpoint{4.799960in}{2.423533in}}%
\pgfpathlineto{\pgfqpoint{4.813068in}{2.421337in}}%
\pgfpathlineto{\pgfqpoint{4.805881in}{2.414650in}}%
\pgfpathlineto{\pgfqpoint{4.798688in}{2.407940in}}%
\pgfpathlineto{\pgfqpoint{4.791489in}{2.401206in}}%
\pgfpathlineto{\pgfqpoint{4.784285in}{2.394445in}}%
\pgfpathlineto{\pgfqpoint{4.771164in}{2.396591in}}%
\pgfpathlineto{\pgfqpoint{4.758050in}{2.398763in}}%
\pgfpathlineto{\pgfqpoint{4.744943in}{2.400959in}}%
\pgfpathlineto{\pgfqpoint{4.731842in}{2.403181in}}%
\pgfpathlineto{\pgfqpoint{4.739060in}{2.409986in}}%
\pgfpathlineto{\pgfqpoint{4.746272in}{2.416768in}}%
\pgfpathlineto{\pgfqpoint{4.753478in}{2.423529in}}%
\pgfpathlineto{\pgfqpoint{4.760678in}{2.430270in}}%
\pgfpathclose%
\pgfusepath{fill}%
\end{pgfscope}%
\begin{pgfscope}%
\pgfpathrectangle{\pgfqpoint{1.254980in}{0.150000in}}{\pgfqpoint{5.490039in}{5.490039in}}%
\pgfusepath{clip}%
\pgfsetbuttcap%
\pgfsetroundjoin%
\definecolor{currentfill}{rgb}{0.277018,0.050344,0.375715}%
\pgfsetfillcolor{currentfill}%
\pgfsetfillopacity{0.700000}%
\pgfsetlinewidth{0.000000pt}%
\definecolor{currentstroke}{rgb}{0.000000,0.000000,0.000000}%
\pgfsetstrokecolor{currentstroke}%
\pgfsetdash{}{0pt}%
\pgfpathmoveto{\pgfqpoint{4.546088in}{2.403903in}}%
\pgfpathlineto{\pgfqpoint{4.559120in}{2.401411in}}%
\pgfpathlineto{\pgfqpoint{4.572159in}{2.398945in}}%
\pgfpathlineto{\pgfqpoint{4.585204in}{2.396505in}}%
\pgfpathlineto{\pgfqpoint{4.598256in}{2.394091in}}%
\pgfpathlineto{\pgfqpoint{4.590985in}{2.387089in}}%
\pgfpathlineto{\pgfqpoint{4.583708in}{2.380061in}}%
\pgfpathlineto{\pgfqpoint{4.576426in}{2.373007in}}%
\pgfpathlineto{\pgfqpoint{4.569138in}{2.365924in}}%
\pgfpathlineto{\pgfqpoint{4.556073in}{2.368315in}}%
\pgfpathlineto{\pgfqpoint{4.543015in}{2.370731in}}%
\pgfpathlineto{\pgfqpoint{4.529964in}{2.373174in}}%
\pgfpathlineto{\pgfqpoint{4.516920in}{2.375642in}}%
\pgfpathlineto{\pgfqpoint{4.524220in}{2.382744in}}%
\pgfpathlineto{\pgfqpoint{4.531515in}{2.389820in}}%
\pgfpathlineto{\pgfqpoint{4.538805in}{2.396873in}}%
\pgfpathlineto{\pgfqpoint{4.546088in}{2.403903in}}%
\pgfpathclose%
\pgfusepath{fill}%
\end{pgfscope}%
\begin{pgfscope}%
\pgfpathrectangle{\pgfqpoint{1.254980in}{0.150000in}}{\pgfqpoint{5.490039in}{5.490039in}}%
\pgfusepath{clip}%
\pgfsetbuttcap%
\pgfsetroundjoin%
\definecolor{currentfill}{rgb}{0.273809,0.031497,0.358853}%
\pgfsetfillcolor{currentfill}%
\pgfsetfillopacity{0.700000}%
\pgfsetlinewidth{0.000000pt}%
\definecolor{currentstroke}{rgb}{0.000000,0.000000,0.000000}%
\pgfsetstrokecolor{currentstroke}%
\pgfsetdash{}{0pt}%
\pgfpathmoveto{\pgfqpoint{4.331492in}{2.378509in}}%
\pgfpathlineto{\pgfqpoint{4.344471in}{2.375734in}}%
\pgfpathlineto{\pgfqpoint{4.357456in}{2.372986in}}%
\pgfpathlineto{\pgfqpoint{4.370448in}{2.370265in}}%
\pgfpathlineto{\pgfqpoint{4.383445in}{2.367570in}}%
\pgfpathlineto{\pgfqpoint{4.376093in}{2.360324in}}%
\pgfpathlineto{\pgfqpoint{4.368735in}{2.353055in}}%
\pgfpathlineto{\pgfqpoint{4.361371in}{2.345762in}}%
\pgfpathlineto{\pgfqpoint{4.354002in}{2.338444in}}%
\pgfpathlineto{\pgfqpoint{4.340993in}{2.341140in}}%
\pgfpathlineto{\pgfqpoint{4.327989in}{2.343864in}}%
\pgfpathlineto{\pgfqpoint{4.314992in}{2.346614in}}%
\pgfpathlineto{\pgfqpoint{4.302001in}{2.349391in}}%
\pgfpathlineto{\pgfqpoint{4.309382in}{2.356702in}}%
\pgfpathlineto{\pgfqpoint{4.316758in}{2.363992in}}%
\pgfpathlineto{\pgfqpoint{4.324128in}{2.371260in}}%
\pgfpathlineto{\pgfqpoint{4.331492in}{2.378509in}}%
\pgfpathclose%
\pgfusepath{fill}%
\end{pgfscope}%
\begin{pgfscope}%
\pgfpathrectangle{\pgfqpoint{1.254980in}{0.150000in}}{\pgfqpoint{5.490039in}{5.490039in}}%
\pgfusepath{clip}%
\pgfsetbuttcap%
\pgfsetroundjoin%
\definecolor{currentfill}{rgb}{0.271305,0.019942,0.347269}%
\pgfsetfillcolor{currentfill}%
\pgfsetfillopacity{0.700000}%
\pgfsetlinewidth{0.000000pt}%
\definecolor{currentstroke}{rgb}{0.000000,0.000000,0.000000}%
\pgfsetstrokecolor{currentstroke}%
\pgfsetdash{}{0pt}%
\pgfpathmoveto{\pgfqpoint{4.116879in}{2.355568in}}%
\pgfpathlineto{\pgfqpoint{4.129807in}{2.352445in}}%
\pgfpathlineto{\pgfqpoint{4.142742in}{2.349351in}}%
\pgfpathlineto{\pgfqpoint{4.155682in}{2.346284in}}%
\pgfpathlineto{\pgfqpoint{4.168629in}{2.343245in}}%
\pgfpathlineto{\pgfqpoint{4.161197in}{2.335870in}}%
\pgfpathlineto{\pgfqpoint{4.153759in}{2.328479in}}%
\pgfpathlineto{\pgfqpoint{4.146316in}{2.321073in}}%
\pgfpathlineto{\pgfqpoint{4.138868in}{2.313652in}}%
\pgfpathlineto{\pgfqpoint{4.125910in}{2.316718in}}%
\pgfpathlineto{\pgfqpoint{4.112958in}{2.319813in}}%
\pgfpathlineto{\pgfqpoint{4.100011in}{2.322935in}}%
\pgfpathlineto{\pgfqpoint{4.087071in}{2.326085in}}%
\pgfpathlineto{\pgfqpoint{4.094531in}{2.333474in}}%
\pgfpathlineto{\pgfqpoint{4.101986in}{2.340851in}}%
\pgfpathlineto{\pgfqpoint{4.109435in}{2.348216in}}%
\pgfpathlineto{\pgfqpoint{4.116879in}{2.355568in}}%
\pgfpathclose%
\pgfusepath{fill}%
\end{pgfscope}%
\begin{pgfscope}%
\pgfpathrectangle{\pgfqpoint{1.254980in}{0.150000in}}{\pgfqpoint{5.490039in}{5.490039in}}%
\pgfusepath{clip}%
\pgfsetbuttcap%
\pgfsetroundjoin%
\definecolor{currentfill}{rgb}{0.281446,0.084320,0.407414}%
\pgfsetfillcolor{currentfill}%
\pgfsetfillopacity{0.700000}%
\pgfsetlinewidth{0.000000pt}%
\definecolor{currentstroke}{rgb}{0.000000,0.000000,0.000000}%
\pgfsetstrokecolor{currentstroke}%
\pgfsetdash{}{0pt}%
\pgfpathmoveto{\pgfqpoint{2.836961in}{2.459314in}}%
\pgfpathlineto{\pgfqpoint{2.849680in}{2.452609in}}%
\pgfpathlineto{\pgfqpoint{2.862402in}{2.445952in}}%
\pgfpathlineto{\pgfqpoint{2.875126in}{2.439343in}}%
\pgfpathlineto{\pgfqpoint{2.887854in}{2.432781in}}%
\pgfpathlineto{\pgfqpoint{2.879882in}{2.428647in}}%
\pgfpathlineto{\pgfqpoint{2.871900in}{2.424635in}}%
\pgfpathlineto{\pgfqpoint{2.863908in}{2.420748in}}%
\pgfpathlineto{\pgfqpoint{2.855906in}{2.416989in}}%
\pgfpathlineto{\pgfqpoint{2.843159in}{2.423707in}}%
\pgfpathlineto{\pgfqpoint{2.830414in}{2.430473in}}%
\pgfpathlineto{\pgfqpoint{2.817672in}{2.437286in}}%
\pgfpathlineto{\pgfqpoint{2.804933in}{2.444148in}}%
\pgfpathlineto{\pgfqpoint{2.812955in}{2.447745in}}%
\pgfpathlineto{\pgfqpoint{2.820968in}{2.451475in}}%
\pgfpathlineto{\pgfqpoint{2.828969in}{2.455332in}}%
\pgfpathlineto{\pgfqpoint{2.836961in}{2.459314in}}%
\pgfpathclose%
\pgfusepath{fill}%
\end{pgfscope}%
\begin{pgfscope}%
\pgfpathrectangle{\pgfqpoint{1.254980in}{0.150000in}}{\pgfqpoint{5.490039in}{5.490039in}}%
\pgfusepath{clip}%
\pgfsetbuttcap%
\pgfsetroundjoin%
\definecolor{currentfill}{rgb}{0.271305,0.019942,0.347269}%
\pgfsetfillcolor{currentfill}%
\pgfsetfillopacity{0.700000}%
\pgfsetlinewidth{0.000000pt}%
\definecolor{currentstroke}{rgb}{0.000000,0.000000,0.000000}%
\pgfsetstrokecolor{currentstroke}%
\pgfsetdash{}{0pt}%
\pgfpathmoveto{\pgfqpoint{3.288043in}{2.355480in}}%
\pgfpathlineto{\pgfqpoint{3.300812in}{2.350353in}}%
\pgfpathlineto{\pgfqpoint{3.313584in}{2.345263in}}%
\pgfpathlineto{\pgfqpoint{3.326361in}{2.340210in}}%
\pgfpathlineto{\pgfqpoint{3.339142in}{2.335194in}}%
\pgfpathlineto{\pgfqpoint{3.331390in}{2.329081in}}%
\pgfpathlineto{\pgfqpoint{3.323630in}{2.323029in}}%
\pgfpathlineto{\pgfqpoint{3.315864in}{2.317040in}}%
\pgfpathlineto{\pgfqpoint{3.308090in}{2.311118in}}%
\pgfpathlineto{\pgfqpoint{3.295293in}{2.316251in}}%
\pgfpathlineto{\pgfqpoint{3.282501in}{2.321421in}}%
\pgfpathlineto{\pgfqpoint{3.269713in}{2.326628in}}%
\pgfpathlineto{\pgfqpoint{3.256929in}{2.331872in}}%
\pgfpathlineto{\pgfqpoint{3.264718in}{2.337672in}}%
\pgfpathlineto{\pgfqpoint{3.272501in}{2.343542in}}%
\pgfpathlineto{\pgfqpoint{3.280276in}{2.349479in}}%
\pgfpathlineto{\pgfqpoint{3.288043in}{2.355480in}}%
\pgfpathclose%
\pgfusepath{fill}%
\end{pgfscope}%
\begin{pgfscope}%
\pgfpathrectangle{\pgfqpoint{1.254980in}{0.150000in}}{\pgfqpoint{5.490039in}{5.490039in}}%
\pgfusepath{clip}%
\pgfsetbuttcap%
\pgfsetroundjoin%
\definecolor{currentfill}{rgb}{0.268510,0.009605,0.335427}%
\pgfsetfillcolor{currentfill}%
\pgfsetfillopacity{0.700000}%
\pgfsetlinewidth{0.000000pt}%
\definecolor{currentstroke}{rgb}{0.000000,0.000000,0.000000}%
\pgfsetstrokecolor{currentstroke}%
\pgfsetdash{}{0pt}%
\pgfpathmoveto{\pgfqpoint{3.421194in}{2.340917in}}%
\pgfpathlineto{\pgfqpoint{3.433983in}{2.336185in}}%
\pgfpathlineto{\pgfqpoint{3.446777in}{2.331487in}}%
\pgfpathlineto{\pgfqpoint{3.459575in}{2.326825in}}%
\pgfpathlineto{\pgfqpoint{3.472378in}{2.322198in}}%
\pgfpathlineto{\pgfqpoint{3.464682in}{2.315668in}}%
\pgfpathlineto{\pgfqpoint{3.456980in}{2.309183in}}%
\pgfpathlineto{\pgfqpoint{3.449271in}{2.302744in}}%
\pgfpathlineto{\pgfqpoint{3.441555in}{2.296356in}}%
\pgfpathlineto{\pgfqpoint{3.428737in}{2.301087in}}%
\pgfpathlineto{\pgfqpoint{3.415924in}{2.305854in}}%
\pgfpathlineto{\pgfqpoint{3.403116in}{2.310655in}}%
\pgfpathlineto{\pgfqpoint{3.390312in}{2.315491in}}%
\pgfpathlineto{\pgfqpoint{3.398042in}{2.321771in}}%
\pgfpathlineto{\pgfqpoint{3.405766in}{2.328104in}}%
\pgfpathlineto{\pgfqpoint{3.413483in}{2.334486in}}%
\pgfpathlineto{\pgfqpoint{3.421194in}{2.340917in}}%
\pgfpathclose%
\pgfusepath{fill}%
\end{pgfscope}%
\begin{pgfscope}%
\pgfpathrectangle{\pgfqpoint{1.254980in}{0.150000in}}{\pgfqpoint{5.490039in}{5.490039in}}%
\pgfusepath{clip}%
\pgfsetbuttcap%
\pgfsetroundjoin%
\definecolor{currentfill}{rgb}{0.268510,0.009605,0.335427}%
\pgfsetfillcolor{currentfill}%
\pgfsetfillopacity{0.700000}%
\pgfsetlinewidth{0.000000pt}%
\definecolor{currentstroke}{rgb}{0.000000,0.000000,0.000000}%
\pgfsetstrokecolor{currentstroke}%
\pgfsetdash{}{0pt}%
\pgfpathmoveto{\pgfqpoint{3.902211in}{2.336952in}}%
\pgfpathlineto{\pgfqpoint{3.915093in}{2.333413in}}%
\pgfpathlineto{\pgfqpoint{3.927981in}{2.329904in}}%
\pgfpathlineto{\pgfqpoint{3.940874in}{2.326424in}}%
\pgfpathlineto{\pgfqpoint{3.953773in}{2.322974in}}%
\pgfpathlineto{\pgfqpoint{3.946262in}{2.315629in}}%
\pgfpathlineto{\pgfqpoint{3.938745in}{2.308282in}}%
\pgfpathlineto{\pgfqpoint{3.931223in}{2.300934in}}%
\pgfpathlineto{\pgfqpoint{3.923695in}{2.293585in}}%
\pgfpathlineto{\pgfqpoint{3.910784in}{2.297088in}}%
\pgfpathlineto{\pgfqpoint{3.897878in}{2.300621in}}%
\pgfpathlineto{\pgfqpoint{3.884978in}{2.304183in}}%
\pgfpathlineto{\pgfqpoint{3.872084in}{2.307775in}}%
\pgfpathlineto{\pgfqpoint{3.879624in}{2.315066in}}%
\pgfpathlineto{\pgfqpoint{3.887159in}{2.322359in}}%
\pgfpathlineto{\pgfqpoint{3.894688in}{2.329655in}}%
\pgfpathlineto{\pgfqpoint{3.902211in}{2.336952in}}%
\pgfpathclose%
\pgfusepath{fill}%
\end{pgfscope}%
\begin{pgfscope}%
\pgfpathrectangle{\pgfqpoint{1.254980in}{0.150000in}}{\pgfqpoint{5.490039in}{5.490039in}}%
\pgfusepath{clip}%
\pgfsetbuttcap%
\pgfsetroundjoin%
\definecolor{currentfill}{rgb}{0.283187,0.125848,0.444960}%
\pgfsetfillcolor{currentfill}%
\pgfsetfillopacity{0.700000}%
\pgfsetlinewidth{0.000000pt}%
\definecolor{currentstroke}{rgb}{0.000000,0.000000,0.000000}%
\pgfsetstrokecolor{currentstroke}%
\pgfsetdash{}{0pt}%
\pgfpathmoveto{\pgfqpoint{5.752598in}{2.536701in}}%
\pgfpathlineto{\pgfqpoint{5.765945in}{2.534748in}}%
\pgfpathlineto{\pgfqpoint{5.779299in}{2.532818in}}%
\pgfpathlineto{\pgfqpoint{5.792660in}{2.530911in}}%
\pgfpathlineto{\pgfqpoint{5.806030in}{2.529027in}}%
\pgfpathlineto{\pgfqpoint{5.799253in}{2.523489in}}%
\pgfpathlineto{\pgfqpoint{5.792473in}{2.518012in}}%
\pgfpathlineto{\pgfqpoint{5.785690in}{2.512592in}}%
\pgfpathlineto{\pgfqpoint{5.778903in}{2.507223in}}%
\pgfpathlineto{\pgfqpoint{5.765513in}{2.508943in}}%
\pgfpathlineto{\pgfqpoint{5.752131in}{2.510685in}}%
\pgfpathlineto{\pgfqpoint{5.738757in}{2.512451in}}%
\pgfpathlineto{\pgfqpoint{5.725391in}{2.514240in}}%
\pgfpathlineto{\pgfqpoint{5.732198in}{2.519768in}}%
\pgfpathlineto{\pgfqpoint{5.739001in}{2.525352in}}%
\pgfpathlineto{\pgfqpoint{5.745801in}{2.530994in}}%
\pgfpathlineto{\pgfqpoint{5.752598in}{2.536701in}}%
\pgfpathclose%
\pgfusepath{fill}%
\end{pgfscope}%
\begin{pgfscope}%
\pgfpathrectangle{\pgfqpoint{1.254980in}{0.150000in}}{\pgfqpoint{5.490039in}{5.490039in}}%
\pgfusepath{clip}%
\pgfsetbuttcap%
\pgfsetroundjoin%
\definecolor{currentfill}{rgb}{0.273809,0.031497,0.358853}%
\pgfsetfillcolor{currentfill}%
\pgfsetfillopacity{0.700000}%
\pgfsetlinewidth{0.000000pt}%
\definecolor{currentstroke}{rgb}{0.000000,0.000000,0.000000}%
\pgfsetstrokecolor{currentstroke}%
\pgfsetdash{}{0pt}%
\pgfpathmoveto{\pgfqpoint{3.154807in}{2.375206in}}%
\pgfpathlineto{\pgfqpoint{3.167558in}{2.369652in}}%
\pgfpathlineto{\pgfqpoint{3.180313in}{2.364138in}}%
\pgfpathlineto{\pgfqpoint{3.193072in}{2.358663in}}%
\pgfpathlineto{\pgfqpoint{3.205835in}{2.353228in}}%
\pgfpathlineto{\pgfqpoint{3.198022in}{2.347624in}}%
\pgfpathlineto{\pgfqpoint{3.190201in}{2.342099in}}%
\pgfpathlineto{\pgfqpoint{3.182372in}{2.336655in}}%
\pgfpathlineto{\pgfqpoint{3.174536in}{2.331296in}}%
\pgfpathlineto{\pgfqpoint{3.161756in}{2.336861in}}%
\pgfpathlineto{\pgfqpoint{3.148980in}{2.342465in}}%
\pgfpathlineto{\pgfqpoint{3.136208in}{2.348109in}}%
\pgfpathlineto{\pgfqpoint{3.123440in}{2.353793in}}%
\pgfpathlineto{\pgfqpoint{3.131294in}{2.359017in}}%
\pgfpathlineto{\pgfqpoint{3.139140in}{2.364330in}}%
\pgfpathlineto{\pgfqpoint{3.146977in}{2.369727in}}%
\pgfpathlineto{\pgfqpoint{3.154807in}{2.375206in}}%
\pgfpathclose%
\pgfusepath{fill}%
\end{pgfscope}%
\begin{pgfscope}%
\pgfpathrectangle{\pgfqpoint{1.254980in}{0.150000in}}{\pgfqpoint{5.490039in}{5.490039in}}%
\pgfusepath{clip}%
\pgfsetbuttcap%
\pgfsetroundjoin%
\definecolor{currentfill}{rgb}{0.267004,0.004874,0.329415}%
\pgfsetfillcolor{currentfill}%
\pgfsetfillopacity{0.700000}%
\pgfsetlinewidth{0.000000pt}%
\definecolor{currentstroke}{rgb}{0.000000,0.000000,0.000000}%
\pgfsetstrokecolor{currentstroke}%
\pgfsetdash{}{0pt}%
\pgfpathmoveto{\pgfqpoint{3.554305in}{2.330919in}}%
\pgfpathlineto{\pgfqpoint{3.567118in}{2.326553in}}%
\pgfpathlineto{\pgfqpoint{3.579937in}{2.322220in}}%
\pgfpathlineto{\pgfqpoint{3.592761in}{2.317920in}}%
\pgfpathlineto{\pgfqpoint{3.605589in}{2.313653in}}%
\pgfpathlineto{\pgfqpoint{3.597946in}{2.306792in}}%
\pgfpathlineto{\pgfqpoint{3.590296in}{2.299961in}}%
\pgfpathlineto{\pgfqpoint{3.582641in}{2.293163in}}%
\pgfpathlineto{\pgfqpoint{3.574979in}{2.286398in}}%
\pgfpathlineto{\pgfqpoint{3.562137in}{2.290756in}}%
\pgfpathlineto{\pgfqpoint{3.549299in}{2.295148in}}%
\pgfpathlineto{\pgfqpoint{3.536467in}{2.299572in}}%
\pgfpathlineto{\pgfqpoint{3.523639in}{2.304029in}}%
\pgfpathlineto{\pgfqpoint{3.531315in}{2.310698in}}%
\pgfpathlineto{\pgfqpoint{3.538984in}{2.317404in}}%
\pgfpathlineto{\pgfqpoint{3.546648in}{2.324145in}}%
\pgfpathlineto{\pgfqpoint{3.554305in}{2.330919in}}%
\pgfpathclose%
\pgfusepath{fill}%
\end{pgfscope}%
\begin{pgfscope}%
\pgfpathrectangle{\pgfqpoint{1.254980in}{0.150000in}}{\pgfqpoint{5.490039in}{5.490039in}}%
\pgfusepath{clip}%
\pgfsetbuttcap%
\pgfsetroundjoin%
\definecolor{currentfill}{rgb}{0.283197,0.115680,0.436115}%
\pgfsetfillcolor{currentfill}%
\pgfsetfillopacity{0.700000}%
\pgfsetlinewidth{0.000000pt}%
\definecolor{currentstroke}{rgb}{0.000000,0.000000,0.000000}%
\pgfsetstrokecolor{currentstroke}%
\pgfsetdash{}{0pt}%
\pgfpathmoveto{\pgfqpoint{5.538091in}{2.514011in}}%
\pgfpathlineto{\pgfqpoint{5.551385in}{2.512097in}}%
\pgfpathlineto{\pgfqpoint{5.564687in}{2.510206in}}%
\pgfpathlineto{\pgfqpoint{5.577997in}{2.508338in}}%
\pgfpathlineto{\pgfqpoint{5.591314in}{2.506494in}}%
\pgfpathlineto{\pgfqpoint{5.584448in}{2.500877in}}%
\pgfpathlineto{\pgfqpoint{5.577579in}{2.495292in}}%
\pgfpathlineto{\pgfqpoint{5.570705in}{2.489735in}}%
\pgfpathlineto{\pgfqpoint{5.563826in}{2.484202in}}%
\pgfpathlineto{\pgfqpoint{5.550490in}{2.485907in}}%
\pgfpathlineto{\pgfqpoint{5.537163in}{2.487636in}}%
\pgfpathlineto{\pgfqpoint{5.523842in}{2.489388in}}%
\pgfpathlineto{\pgfqpoint{5.510530in}{2.491163in}}%
\pgfpathlineto{\pgfqpoint{5.517427in}{2.496830in}}%
\pgfpathlineto{\pgfqpoint{5.524319in}{2.502525in}}%
\pgfpathlineto{\pgfqpoint{5.531207in}{2.508250in}}%
\pgfpathlineto{\pgfqpoint{5.538091in}{2.514011in}}%
\pgfpathclose%
\pgfusepath{fill}%
\end{pgfscope}%
\begin{pgfscope}%
\pgfpathrectangle{\pgfqpoint{1.254980in}{0.150000in}}{\pgfqpoint{5.490039in}{5.490039in}}%
\pgfusepath{clip}%
\pgfsetbuttcap%
\pgfsetroundjoin%
\definecolor{currentfill}{rgb}{0.282910,0.105393,0.426902}%
\pgfsetfillcolor{currentfill}%
\pgfsetfillopacity{0.700000}%
\pgfsetlinewidth{0.000000pt}%
\definecolor{currentstroke}{rgb}{0.000000,0.000000,0.000000}%
\pgfsetstrokecolor{currentstroke}%
\pgfsetdash{}{0pt}%
\pgfpathmoveto{\pgfqpoint{5.323516in}{2.490289in}}%
\pgfpathlineto{\pgfqpoint{5.336756in}{2.488358in}}%
\pgfpathlineto{\pgfqpoint{5.350004in}{2.486452in}}%
\pgfpathlineto{\pgfqpoint{5.363259in}{2.484568in}}%
\pgfpathlineto{\pgfqpoint{5.376522in}{2.482709in}}%
\pgfpathlineto{\pgfqpoint{5.369566in}{2.476876in}}%
\pgfpathlineto{\pgfqpoint{5.362604in}{2.471053in}}%
\pgfpathlineto{\pgfqpoint{5.355637in}{2.465236in}}%
\pgfpathlineto{\pgfqpoint{5.348665in}{2.459420in}}%
\pgfpathlineto{\pgfqpoint{5.335386in}{2.461166in}}%
\pgfpathlineto{\pgfqpoint{5.322114in}{2.462935in}}%
\pgfpathlineto{\pgfqpoint{5.308849in}{2.464729in}}%
\pgfpathlineto{\pgfqpoint{5.295592in}{2.466546in}}%
\pgfpathlineto{\pgfqpoint{5.302581in}{2.472470in}}%
\pgfpathlineto{\pgfqpoint{5.309564in}{2.478400in}}%
\pgfpathlineto{\pgfqpoint{5.316543in}{2.484338in}}%
\pgfpathlineto{\pgfqpoint{5.323516in}{2.490289in}}%
\pgfpathclose%
\pgfusepath{fill}%
\end{pgfscope}%
\begin{pgfscope}%
\pgfpathrectangle{\pgfqpoint{1.254980in}{0.150000in}}{\pgfqpoint{5.490039in}{5.490039in}}%
\pgfusepath{clip}%
\pgfsetbuttcap%
\pgfsetroundjoin%
\definecolor{currentfill}{rgb}{0.281924,0.089666,0.412415}%
\pgfsetfillcolor{currentfill}%
\pgfsetfillopacity{0.700000}%
\pgfsetlinewidth{0.000000pt}%
\definecolor{currentstroke}{rgb}{0.000000,0.000000,0.000000}%
\pgfsetstrokecolor{currentstroke}%
\pgfsetdash{}{0pt}%
\pgfpathmoveto{\pgfqpoint{5.108881in}{2.465282in}}%
\pgfpathlineto{\pgfqpoint{5.122066in}{2.463279in}}%
\pgfpathlineto{\pgfqpoint{5.135259in}{2.461300in}}%
\pgfpathlineto{\pgfqpoint{5.148458in}{2.459344in}}%
\pgfpathlineto{\pgfqpoint{5.161665in}{2.457413in}}%
\pgfpathlineto{\pgfqpoint{5.154617in}{2.451277in}}%
\pgfpathlineto{\pgfqpoint{5.147564in}{2.445132in}}%
\pgfpathlineto{\pgfqpoint{5.140506in}{2.438975in}}%
\pgfpathlineto{\pgfqpoint{5.133441in}{2.432804in}}%
\pgfpathlineto{\pgfqpoint{5.120219in}{2.434647in}}%
\pgfpathlineto{\pgfqpoint{5.107004in}{2.436514in}}%
\pgfpathlineto{\pgfqpoint{5.093797in}{2.438406in}}%
\pgfpathlineto{\pgfqpoint{5.080597in}{2.440321in}}%
\pgfpathlineto{\pgfqpoint{5.087676in}{2.446575in}}%
\pgfpathlineto{\pgfqpoint{5.094750in}{2.452818in}}%
\pgfpathlineto{\pgfqpoint{5.101819in}{2.459053in}}%
\pgfpathlineto{\pgfqpoint{5.108881in}{2.465282in}}%
\pgfpathclose%
\pgfusepath{fill}%
\end{pgfscope}%
\begin{pgfscope}%
\pgfpathrectangle{\pgfqpoint{1.254980in}{0.150000in}}{\pgfqpoint{5.490039in}{5.490039in}}%
\pgfusepath{clip}%
\pgfsetbuttcap%
\pgfsetroundjoin%
\definecolor{currentfill}{rgb}{0.278791,0.062145,0.386592}%
\pgfsetfillcolor{currentfill}%
\pgfsetfillopacity{0.700000}%
\pgfsetlinewidth{0.000000pt}%
\definecolor{currentstroke}{rgb}{0.000000,0.000000,0.000000}%
\pgfsetstrokecolor{currentstroke}%
\pgfsetdash{}{0pt}%
\pgfpathmoveto{\pgfqpoint{4.679510in}{2.412320in}}%
\pgfpathlineto{\pgfqpoint{4.692583in}{2.409997in}}%
\pgfpathlineto{\pgfqpoint{4.705662in}{2.407700in}}%
\pgfpathlineto{\pgfqpoint{4.718749in}{2.405428in}}%
\pgfpathlineto{\pgfqpoint{4.731842in}{2.403181in}}%
\pgfpathlineto{\pgfqpoint{4.724619in}{2.396351in}}%
\pgfpathlineto{\pgfqpoint{4.717391in}{2.389495in}}%
\pgfpathlineto{\pgfqpoint{4.710156in}{2.382611in}}%
\pgfpathlineto{\pgfqpoint{4.702916in}{2.375697in}}%
\pgfpathlineto{\pgfqpoint{4.689810in}{2.377907in}}%
\pgfpathlineto{\pgfqpoint{4.676710in}{2.380143in}}%
\pgfpathlineto{\pgfqpoint{4.663618in}{2.382404in}}%
\pgfpathlineto{\pgfqpoint{4.650532in}{2.384690in}}%
\pgfpathlineto{\pgfqpoint{4.657785in}{2.391635in}}%
\pgfpathlineto{\pgfqpoint{4.665032in}{2.398555in}}%
\pgfpathlineto{\pgfqpoint{4.672274in}{2.405449in}}%
\pgfpathlineto{\pgfqpoint{4.679510in}{2.412320in}}%
\pgfpathclose%
\pgfusepath{fill}%
\end{pgfscope}%
\begin{pgfscope}%
\pgfpathrectangle{\pgfqpoint{1.254980in}{0.150000in}}{\pgfqpoint{5.490039in}{5.490039in}}%
\pgfusepath{clip}%
\pgfsetbuttcap%
\pgfsetroundjoin%
\definecolor{currentfill}{rgb}{0.277018,0.050344,0.375715}%
\pgfsetfillcolor{currentfill}%
\pgfsetfillopacity{0.700000}%
\pgfsetlinewidth{0.000000pt}%
\definecolor{currentstroke}{rgb}{0.000000,0.000000,0.000000}%
\pgfsetstrokecolor{currentstroke}%
\pgfsetdash{}{0pt}%
\pgfpathmoveto{\pgfqpoint{3.021431in}{2.400740in}}%
\pgfpathlineto{\pgfqpoint{3.034170in}{2.394724in}}%
\pgfpathlineto{\pgfqpoint{3.046912in}{2.388752in}}%
\pgfpathlineto{\pgfqpoint{3.059657in}{2.382821in}}%
\pgfpathlineto{\pgfqpoint{3.072406in}{2.376933in}}%
\pgfpathlineto{\pgfqpoint{3.064527in}{2.371937in}}%
\pgfpathlineto{\pgfqpoint{3.056639in}{2.367038in}}%
\pgfpathlineto{\pgfqpoint{3.048742in}{2.362240in}}%
\pgfpathlineto{\pgfqpoint{3.040836in}{2.357546in}}%
\pgfpathlineto{\pgfqpoint{3.028069in}{2.363577in}}%
\pgfpathlineto{\pgfqpoint{3.015305in}{2.369650in}}%
\pgfpathlineto{\pgfqpoint{3.002545in}{2.375765in}}%
\pgfpathlineto{\pgfqpoint{2.989788in}{2.381924in}}%
\pgfpathlineto{\pgfqpoint{2.997712in}{2.386470in}}%
\pgfpathlineto{\pgfqpoint{3.005627in}{2.391124in}}%
\pgfpathlineto{\pgfqpoint{3.013534in}{2.395882in}}%
\pgfpathlineto{\pgfqpoint{3.021431in}{2.400740in}}%
\pgfpathclose%
\pgfusepath{fill}%
\end{pgfscope}%
\begin{pgfscope}%
\pgfpathrectangle{\pgfqpoint{1.254980in}{0.150000in}}{\pgfqpoint{5.490039in}{5.490039in}}%
\pgfusepath{clip}%
\pgfsetbuttcap%
\pgfsetroundjoin%
\definecolor{currentfill}{rgb}{0.276022,0.044167,0.370164}%
\pgfsetfillcolor{currentfill}%
\pgfsetfillopacity{0.700000}%
\pgfsetlinewidth{0.000000pt}%
\definecolor{currentstroke}{rgb}{0.000000,0.000000,0.000000}%
\pgfsetstrokecolor{currentstroke}%
\pgfsetdash{}{0pt}%
\pgfpathmoveto{\pgfqpoint{4.464808in}{2.385776in}}%
\pgfpathlineto{\pgfqpoint{4.477826in}{2.383203in}}%
\pgfpathlineto{\pgfqpoint{4.490851in}{2.380657in}}%
\pgfpathlineto{\pgfqpoint{4.503882in}{2.378136in}}%
\pgfpathlineto{\pgfqpoint{4.516920in}{2.375642in}}%
\pgfpathlineto{\pgfqpoint{4.509614in}{2.368515in}}%
\pgfpathlineto{\pgfqpoint{4.502302in}{2.361360in}}%
\pgfpathlineto{\pgfqpoint{4.494985in}{2.354178in}}%
\pgfpathlineto{\pgfqpoint{4.487662in}{2.346967in}}%
\pgfpathlineto{\pgfqpoint{4.474612in}{2.349451in}}%
\pgfpathlineto{\pgfqpoint{4.461569in}{2.351960in}}%
\pgfpathlineto{\pgfqpoint{4.448532in}{2.354496in}}%
\pgfpathlineto{\pgfqpoint{4.435501in}{2.357058in}}%
\pgfpathlineto{\pgfqpoint{4.442836in}{2.364275in}}%
\pgfpathlineto{\pgfqpoint{4.450166in}{2.371466in}}%
\pgfpathlineto{\pgfqpoint{4.457490in}{2.378633in}}%
\pgfpathlineto{\pgfqpoint{4.464808in}{2.385776in}}%
\pgfpathclose%
\pgfusepath{fill}%
\end{pgfscope}%
\begin{pgfscope}%
\pgfpathrectangle{\pgfqpoint{1.254980in}{0.150000in}}{\pgfqpoint{5.490039in}{5.490039in}}%
\pgfusepath{clip}%
\pgfsetbuttcap%
\pgfsetroundjoin%
\definecolor{currentfill}{rgb}{0.280267,0.073417,0.397163}%
\pgfsetfillcolor{currentfill}%
\pgfsetfillopacity{0.700000}%
\pgfsetlinewidth{0.000000pt}%
\definecolor{currentstroke}{rgb}{0.000000,0.000000,0.000000}%
\pgfsetstrokecolor{currentstroke}%
\pgfsetdash{}{0pt}%
\pgfpathmoveto{\pgfqpoint{4.894206in}{2.439121in}}%
\pgfpathlineto{\pgfqpoint{4.907335in}{2.436987in}}%
\pgfpathlineto{\pgfqpoint{4.920471in}{2.434879in}}%
\pgfpathlineto{\pgfqpoint{4.933614in}{2.432794in}}%
\pgfpathlineto{\pgfqpoint{4.946764in}{2.430735in}}%
\pgfpathlineto{\pgfqpoint{4.939627in}{2.424250in}}%
\pgfpathlineto{\pgfqpoint{4.932484in}{2.417746in}}%
\pgfpathlineto{\pgfqpoint{4.925336in}{2.411218in}}%
\pgfpathlineto{\pgfqpoint{4.918182in}{2.404666in}}%
\pgfpathlineto{\pgfqpoint{4.905018in}{2.406664in}}%
\pgfpathlineto{\pgfqpoint{4.891861in}{2.408686in}}%
\pgfpathlineto{\pgfqpoint{4.878711in}{2.410732in}}%
\pgfpathlineto{\pgfqpoint{4.865569in}{2.412803in}}%
\pgfpathlineto{\pgfqpoint{4.872736in}{2.419413in}}%
\pgfpathlineto{\pgfqpoint{4.879899in}{2.426001in}}%
\pgfpathlineto{\pgfqpoint{4.887055in}{2.432569in}}%
\pgfpathlineto{\pgfqpoint{4.894206in}{2.439121in}}%
\pgfpathclose%
\pgfusepath{fill}%
\end{pgfscope}%
\begin{pgfscope}%
\pgfpathrectangle{\pgfqpoint{1.254980in}{0.150000in}}{\pgfqpoint{5.490039in}{5.490039in}}%
\pgfusepath{clip}%
\pgfsetbuttcap%
\pgfsetroundjoin%
\definecolor{currentfill}{rgb}{0.272594,0.025563,0.353093}%
\pgfsetfillcolor{currentfill}%
\pgfsetfillopacity{0.700000}%
\pgfsetlinewidth{0.000000pt}%
\definecolor{currentstroke}{rgb}{0.000000,0.000000,0.000000}%
\pgfsetstrokecolor{currentstroke}%
\pgfsetdash{}{0pt}%
\pgfpathmoveto{\pgfqpoint{4.250102in}{2.360770in}}%
\pgfpathlineto{\pgfqpoint{4.263067in}{2.357885in}}%
\pgfpathlineto{\pgfqpoint{4.276039in}{2.355026in}}%
\pgfpathlineto{\pgfqpoint{4.289017in}{2.352195in}}%
\pgfpathlineto{\pgfqpoint{4.302001in}{2.349391in}}%
\pgfpathlineto{\pgfqpoint{4.294615in}{2.342058in}}%
\pgfpathlineto{\pgfqpoint{4.287223in}{2.334704in}}%
\pgfpathlineto{\pgfqpoint{4.279826in}{2.327328in}}%
\pgfpathlineto{\pgfqpoint{4.272423in}{2.319929in}}%
\pgfpathlineto{\pgfqpoint{4.259427in}{2.322748in}}%
\pgfpathlineto{\pgfqpoint{4.246437in}{2.325594in}}%
\pgfpathlineto{\pgfqpoint{4.233453in}{2.328467in}}%
\pgfpathlineto{\pgfqpoint{4.220476in}{2.331368in}}%
\pgfpathlineto{\pgfqpoint{4.227891in}{2.338747in}}%
\pgfpathlineto{\pgfqpoint{4.235300in}{2.346107in}}%
\pgfpathlineto{\pgfqpoint{4.242704in}{2.353448in}}%
\pgfpathlineto{\pgfqpoint{4.250102in}{2.360770in}}%
\pgfpathclose%
\pgfusepath{fill}%
\end{pgfscope}%
\begin{pgfscope}%
\pgfpathrectangle{\pgfqpoint{1.254980in}{0.150000in}}{\pgfqpoint{5.490039in}{5.490039in}}%
\pgfusepath{clip}%
\pgfsetbuttcap%
\pgfsetroundjoin%
\definecolor{currentfill}{rgb}{0.267004,0.004874,0.329415}%
\pgfsetfillcolor{currentfill}%
\pgfsetfillopacity{0.700000}%
\pgfsetlinewidth{0.000000pt}%
\definecolor{currentstroke}{rgb}{0.000000,0.000000,0.000000}%
\pgfsetstrokecolor{currentstroke}%
\pgfsetdash{}{0pt}%
\pgfpathmoveto{\pgfqpoint{3.687416in}{2.324932in}}%
\pgfpathlineto{\pgfqpoint{3.700258in}{2.320905in}}%
\pgfpathlineto{\pgfqpoint{3.713104in}{2.316910in}}%
\pgfpathlineto{\pgfqpoint{3.725956in}{2.312946in}}%
\pgfpathlineto{\pgfqpoint{3.738813in}{2.309014in}}%
\pgfpathlineto{\pgfqpoint{3.731220in}{2.301902in}}%
\pgfpathlineto{\pgfqpoint{3.723621in}{2.294808in}}%
\pgfpathlineto{\pgfqpoint{3.716016in}{2.287732in}}%
\pgfpathlineto{\pgfqpoint{3.708405in}{2.280677in}}%
\pgfpathlineto{\pgfqpoint{3.695534in}{2.284688in}}%
\pgfpathlineto{\pgfqpoint{3.682669in}{2.288730in}}%
\pgfpathlineto{\pgfqpoint{3.669810in}{2.292804in}}%
\pgfpathlineto{\pgfqpoint{3.656955in}{2.296909in}}%
\pgfpathlineto{\pgfqpoint{3.664579in}{2.303882in}}%
\pgfpathlineto{\pgfqpoint{3.672198in}{2.310877in}}%
\pgfpathlineto{\pgfqpoint{3.679810in}{2.317895in}}%
\pgfpathlineto{\pgfqpoint{3.687416in}{2.324932in}}%
\pgfpathclose%
\pgfusepath{fill}%
\end{pgfscope}%
\begin{pgfscope}%
\pgfpathrectangle{\pgfqpoint{1.254980in}{0.150000in}}{\pgfqpoint{5.490039in}{5.490039in}}%
\pgfusepath{clip}%
\pgfsetbuttcap%
\pgfsetroundjoin%
\definecolor{currentfill}{rgb}{0.269944,0.014625,0.341379}%
\pgfsetfillcolor{currentfill}%
\pgfsetfillopacity{0.700000}%
\pgfsetlinewidth{0.000000pt}%
\definecolor{currentstroke}{rgb}{0.000000,0.000000,0.000000}%
\pgfsetstrokecolor{currentstroke}%
\pgfsetdash{}{0pt}%
\pgfpathmoveto{\pgfqpoint{4.035371in}{2.338972in}}%
\pgfpathlineto{\pgfqpoint{4.048287in}{2.335707in}}%
\pgfpathlineto{\pgfqpoint{4.061209in}{2.332471in}}%
\pgfpathlineto{\pgfqpoint{4.074137in}{2.329264in}}%
\pgfpathlineto{\pgfqpoint{4.087071in}{2.326085in}}%
\pgfpathlineto{\pgfqpoint{4.079606in}{2.318685in}}%
\pgfpathlineto{\pgfqpoint{4.072135in}{2.311273in}}%
\pgfpathlineto{\pgfqpoint{4.064658in}{2.303851in}}%
\pgfpathlineto{\pgfqpoint{4.057176in}{2.296419in}}%
\pgfpathlineto{\pgfqpoint{4.044230in}{2.299638in}}%
\pgfpathlineto{\pgfqpoint{4.031290in}{2.302885in}}%
\pgfpathlineto{\pgfqpoint{4.018356in}{2.306161in}}%
\pgfpathlineto{\pgfqpoint{4.005428in}{2.309466in}}%
\pgfpathlineto{\pgfqpoint{4.012922in}{2.316853in}}%
\pgfpathlineto{\pgfqpoint{4.020410in}{2.324234in}}%
\pgfpathlineto{\pgfqpoint{4.027893in}{2.331607in}}%
\pgfpathlineto{\pgfqpoint{4.035371in}{2.338972in}}%
\pgfpathclose%
\pgfusepath{fill}%
\end{pgfscope}%
\begin{pgfscope}%
\pgfpathrectangle{\pgfqpoint{1.254980in}{0.150000in}}{\pgfqpoint{5.490039in}{5.490039in}}%
\pgfusepath{clip}%
\pgfsetbuttcap%
\pgfsetroundjoin%
\definecolor{currentfill}{rgb}{0.280267,0.073417,0.397163}%
\pgfsetfillcolor{currentfill}%
\pgfsetfillopacity{0.700000}%
\pgfsetlinewidth{0.000000pt}%
\definecolor{currentstroke}{rgb}{0.000000,0.000000,0.000000}%
\pgfsetstrokecolor{currentstroke}%
\pgfsetdash{}{0pt}%
\pgfpathmoveto{\pgfqpoint{2.887854in}{2.432781in}}%
\pgfpathlineto{\pgfqpoint{2.900584in}{2.426265in}}%
\pgfpathlineto{\pgfqpoint{2.913318in}{2.419795in}}%
\pgfpathlineto{\pgfqpoint{2.926055in}{2.413372in}}%
\pgfpathlineto{\pgfqpoint{2.938795in}{2.406993in}}%
\pgfpathlineto{\pgfqpoint{2.930842in}{2.402709in}}%
\pgfpathlineto{\pgfqpoint{2.922881in}{2.398542in}}%
\pgfpathlineto{\pgfqpoint{2.914909in}{2.394497in}}%
\pgfpathlineto{\pgfqpoint{2.906927in}{2.390577in}}%
\pgfpathlineto{\pgfqpoint{2.894168in}{2.397112in}}%
\pgfpathlineto{\pgfqpoint{2.881411in}{2.403692in}}%
\pgfpathlineto{\pgfqpoint{2.868657in}{2.410317in}}%
\pgfpathlineto{\pgfqpoint{2.855906in}{2.416989in}}%
\pgfpathlineto{\pgfqpoint{2.863908in}{2.420748in}}%
\pgfpathlineto{\pgfqpoint{2.871900in}{2.424635in}}%
\pgfpathlineto{\pgfqpoint{2.879882in}{2.428647in}}%
\pgfpathlineto{\pgfqpoint{2.887854in}{2.432781in}}%
\pgfpathclose%
\pgfusepath{fill}%
\end{pgfscope}%
\begin{pgfscope}%
\pgfpathrectangle{\pgfqpoint{1.254980in}{0.150000in}}{\pgfqpoint{5.490039in}{5.490039in}}%
\pgfusepath{clip}%
\pgfsetbuttcap%
\pgfsetroundjoin%
\definecolor{currentfill}{rgb}{0.267004,0.004874,0.329415}%
\pgfsetfillcolor{currentfill}%
\pgfsetfillopacity{0.700000}%
\pgfsetlinewidth{0.000000pt}%
\definecolor{currentstroke}{rgb}{0.000000,0.000000,0.000000}%
\pgfsetstrokecolor{currentstroke}%
\pgfsetdash{}{0pt}%
\pgfpathmoveto{\pgfqpoint{3.820563in}{2.322443in}}%
\pgfpathlineto{\pgfqpoint{3.833435in}{2.318730in}}%
\pgfpathlineto{\pgfqpoint{3.846312in}{2.315048in}}%
\pgfpathlineto{\pgfqpoint{3.859195in}{2.311397in}}%
\pgfpathlineto{\pgfqpoint{3.872084in}{2.307775in}}%
\pgfpathlineto{\pgfqpoint{3.864538in}{2.300488in}}%
\pgfpathlineto{\pgfqpoint{3.856987in}{2.293206in}}%
\pgfpathlineto{\pgfqpoint{3.849430in}{2.285931in}}%
\pgfpathlineto{\pgfqpoint{3.841867in}{2.278663in}}%
\pgfpathlineto{\pgfqpoint{3.828966in}{2.282351in}}%
\pgfpathlineto{\pgfqpoint{3.816071in}{2.286068in}}%
\pgfpathlineto{\pgfqpoint{3.803181in}{2.289816in}}%
\pgfpathlineto{\pgfqpoint{3.790297in}{2.293594in}}%
\pgfpathlineto{\pgfqpoint{3.797872in}{2.300791in}}%
\pgfpathlineto{\pgfqpoint{3.805441in}{2.307999in}}%
\pgfpathlineto{\pgfqpoint{3.813005in}{2.315217in}}%
\pgfpathlineto{\pgfqpoint{3.820563in}{2.322443in}}%
\pgfpathclose%
\pgfusepath{fill}%
\end{pgfscope}%
\begin{pgfscope}%
\pgfpathrectangle{\pgfqpoint{1.254980in}{0.150000in}}{\pgfqpoint{5.490039in}{5.490039in}}%
\pgfusepath{clip}%
\pgfsetbuttcap%
\pgfsetroundjoin%
\definecolor{currentfill}{rgb}{0.283187,0.125848,0.444960}%
\pgfsetfillcolor{currentfill}%
\pgfsetfillopacity{0.700000}%
\pgfsetlinewidth{0.000000pt}%
\definecolor{currentstroke}{rgb}{0.000000,0.000000,0.000000}%
\pgfsetstrokecolor{currentstroke}%
\pgfsetdash{}{0pt}%
\pgfpathmoveto{\pgfqpoint{5.672002in}{2.521624in}}%
\pgfpathlineto{\pgfqpoint{5.685338in}{2.519743in}}%
\pgfpathlineto{\pgfqpoint{5.698681in}{2.517886in}}%
\pgfpathlineto{\pgfqpoint{5.712032in}{2.516051in}}%
\pgfpathlineto{\pgfqpoint{5.725391in}{2.514240in}}%
\pgfpathlineto{\pgfqpoint{5.718580in}{2.508760in}}%
\pgfpathlineto{\pgfqpoint{5.711766in}{2.503326in}}%
\pgfpathlineto{\pgfqpoint{5.704947in}{2.497931in}}%
\pgfpathlineto{\pgfqpoint{5.698124in}{2.492572in}}%
\pgfpathlineto{\pgfqpoint{5.684746in}{2.494231in}}%
\pgfpathlineto{\pgfqpoint{5.671376in}{2.495913in}}%
\pgfpathlineto{\pgfqpoint{5.658013in}{2.497619in}}%
\pgfpathlineto{\pgfqpoint{5.644658in}{2.499348in}}%
\pgfpathlineto{\pgfqpoint{5.651500in}{2.504855in}}%
\pgfpathlineto{\pgfqpoint{5.658338in}{2.510400in}}%
\pgfpathlineto{\pgfqpoint{5.665172in}{2.515988in}}%
\pgfpathlineto{\pgfqpoint{5.672002in}{2.521624in}}%
\pgfpathclose%
\pgfusepath{fill}%
\end{pgfscope}%
\begin{pgfscope}%
\pgfpathrectangle{\pgfqpoint{1.254980in}{0.150000in}}{\pgfqpoint{5.490039in}{5.490039in}}%
\pgfusepath{clip}%
\pgfsetbuttcap%
\pgfsetroundjoin%
\definecolor{currentfill}{rgb}{0.283197,0.115680,0.436115}%
\pgfsetfillcolor{currentfill}%
\pgfsetfillopacity{0.700000}%
\pgfsetlinewidth{0.000000pt}%
\definecolor{currentstroke}{rgb}{0.000000,0.000000,0.000000}%
\pgfsetstrokecolor{currentstroke}%
\pgfsetdash{}{0pt}%
\pgfpathmoveto{\pgfqpoint{5.457354in}{2.498496in}}%
\pgfpathlineto{\pgfqpoint{5.470637in}{2.496628in}}%
\pgfpathlineto{\pgfqpoint{5.483927in}{2.494783in}}%
\pgfpathlineto{\pgfqpoint{5.497225in}{2.492961in}}%
\pgfpathlineto{\pgfqpoint{5.510530in}{2.491163in}}%
\pgfpathlineto{\pgfqpoint{5.503628in}{2.485518in}}%
\pgfpathlineto{\pgfqpoint{5.496721in}{2.479891in}}%
\pgfpathlineto{\pgfqpoint{5.489810in}{2.474279in}}%
\pgfpathlineto{\pgfqpoint{5.482893in}{2.468678in}}%
\pgfpathlineto{\pgfqpoint{5.469570in}{2.470349in}}%
\pgfpathlineto{\pgfqpoint{5.456255in}{2.472045in}}%
\pgfpathlineto{\pgfqpoint{5.442947in}{2.473763in}}%
\pgfpathlineto{\pgfqpoint{5.429647in}{2.475505in}}%
\pgfpathlineto{\pgfqpoint{5.436581in}{2.481229in}}%
\pgfpathlineto{\pgfqpoint{5.443511in}{2.486966in}}%
\pgfpathlineto{\pgfqpoint{5.450435in}{2.492720in}}%
\pgfpathlineto{\pgfqpoint{5.457354in}{2.498496in}}%
\pgfpathclose%
\pgfusepath{fill}%
\end{pgfscope}%
\begin{pgfscope}%
\pgfpathrectangle{\pgfqpoint{1.254980in}{0.150000in}}{\pgfqpoint{5.490039in}{5.490039in}}%
\pgfusepath{clip}%
\pgfsetbuttcap%
\pgfsetroundjoin%
\definecolor{currentfill}{rgb}{0.282656,0.100196,0.422160}%
\pgfsetfillcolor{currentfill}%
\pgfsetfillopacity{0.700000}%
\pgfsetlinewidth{0.000000pt}%
\definecolor{currentstroke}{rgb}{0.000000,0.000000,0.000000}%
\pgfsetstrokecolor{currentstroke}%
\pgfsetdash{}{0pt}%
\pgfpathmoveto{\pgfqpoint{5.242638in}{2.474051in}}%
\pgfpathlineto{\pgfqpoint{5.255866in}{2.472139in}}%
\pgfpathlineto{\pgfqpoint{5.269100in}{2.470251in}}%
\pgfpathlineto{\pgfqpoint{5.282343in}{2.468386in}}%
\pgfpathlineto{\pgfqpoint{5.295592in}{2.466546in}}%
\pgfpathlineto{\pgfqpoint{5.288598in}{2.460622in}}%
\pgfpathlineto{\pgfqpoint{5.281599in}{2.454697in}}%
\pgfpathlineto{\pgfqpoint{5.274594in}{2.448766in}}%
\pgfpathlineto{\pgfqpoint{5.267583in}{2.442827in}}%
\pgfpathlineto{\pgfqpoint{5.254318in}{2.444567in}}%
\pgfpathlineto{\pgfqpoint{5.241059in}{2.446331in}}%
\pgfpathlineto{\pgfqpoint{5.227809in}{2.448118in}}%
\pgfpathlineto{\pgfqpoint{5.214565in}{2.449929in}}%
\pgfpathlineto{\pgfqpoint{5.221592in}{2.455965in}}%
\pgfpathlineto{\pgfqpoint{5.228612in}{2.461995in}}%
\pgfpathlineto{\pgfqpoint{5.235628in}{2.468022in}}%
\pgfpathlineto{\pgfqpoint{5.242638in}{2.474051in}}%
\pgfpathclose%
\pgfusepath{fill}%
\end{pgfscope}%
\begin{pgfscope}%
\pgfpathrectangle{\pgfqpoint{1.254980in}{0.150000in}}{\pgfqpoint{5.490039in}{5.490039in}}%
\pgfusepath{clip}%
\pgfsetbuttcap%
\pgfsetroundjoin%
\definecolor{currentfill}{rgb}{0.277941,0.056324,0.381191}%
\pgfsetfillcolor{currentfill}%
\pgfsetfillopacity{0.700000}%
\pgfsetlinewidth{0.000000pt}%
\definecolor{currentstroke}{rgb}{0.000000,0.000000,0.000000}%
\pgfsetstrokecolor{currentstroke}%
\pgfsetdash{}{0pt}%
\pgfpathmoveto{\pgfqpoint{4.598256in}{2.394091in}}%
\pgfpathlineto{\pgfqpoint{4.611315in}{2.391702in}}%
\pgfpathlineto{\pgfqpoint{4.624381in}{2.389339in}}%
\pgfpathlineto{\pgfqpoint{4.637453in}{2.387002in}}%
\pgfpathlineto{\pgfqpoint{4.650532in}{2.384690in}}%
\pgfpathlineto{\pgfqpoint{4.643273in}{2.377717in}}%
\pgfpathlineto{\pgfqpoint{4.636009in}{2.370715in}}%
\pgfpathlineto{\pgfqpoint{4.628739in}{2.363682in}}%
\pgfpathlineto{\pgfqpoint{4.621463in}{2.356619in}}%
\pgfpathlineto{\pgfqpoint{4.608372in}{2.358907in}}%
\pgfpathlineto{\pgfqpoint{4.595287in}{2.361220in}}%
\pgfpathlineto{\pgfqpoint{4.582209in}{2.363559in}}%
\pgfpathlineto{\pgfqpoint{4.569138in}{2.365924in}}%
\pgfpathlineto{\pgfqpoint{4.576426in}{2.373007in}}%
\pgfpathlineto{\pgfqpoint{4.583708in}{2.380061in}}%
\pgfpathlineto{\pgfqpoint{4.590985in}{2.387089in}}%
\pgfpathlineto{\pgfqpoint{4.598256in}{2.394091in}}%
\pgfpathclose%
\pgfusepath{fill}%
\end{pgfscope}%
\begin{pgfscope}%
\pgfpathrectangle{\pgfqpoint{1.254980in}{0.150000in}}{\pgfqpoint{5.490039in}{5.490039in}}%
\pgfusepath{clip}%
\pgfsetbuttcap%
\pgfsetroundjoin%
\definecolor{currentfill}{rgb}{0.274952,0.037752,0.364543}%
\pgfsetfillcolor{currentfill}%
\pgfsetfillopacity{0.700000}%
\pgfsetlinewidth{0.000000pt}%
\definecolor{currentstroke}{rgb}{0.000000,0.000000,0.000000}%
\pgfsetstrokecolor{currentstroke}%
\pgfsetdash{}{0pt}%
\pgfpathmoveto{\pgfqpoint{4.383445in}{2.367570in}}%
\pgfpathlineto{\pgfqpoint{4.396450in}{2.364902in}}%
\pgfpathlineto{\pgfqpoint{4.409460in}{2.362261in}}%
\pgfpathlineto{\pgfqpoint{4.422478in}{2.359646in}}%
\pgfpathlineto{\pgfqpoint{4.435501in}{2.357058in}}%
\pgfpathlineto{\pgfqpoint{4.428161in}{2.349815in}}%
\pgfpathlineto{\pgfqpoint{4.420815in}{2.342545in}}%
\pgfpathlineto{\pgfqpoint{4.413463in}{2.335248in}}%
\pgfpathlineto{\pgfqpoint{4.406106in}{2.327924in}}%
\pgfpathlineto{\pgfqpoint{4.393071in}{2.330514in}}%
\pgfpathlineto{\pgfqpoint{4.380041in}{2.333131in}}%
\pgfpathlineto{\pgfqpoint{4.367019in}{2.335774in}}%
\pgfpathlineto{\pgfqpoint{4.354002in}{2.338444in}}%
\pgfpathlineto{\pgfqpoint{4.361371in}{2.345762in}}%
\pgfpathlineto{\pgfqpoint{4.368735in}{2.353055in}}%
\pgfpathlineto{\pgfqpoint{4.376093in}{2.360324in}}%
\pgfpathlineto{\pgfqpoint{4.383445in}{2.367570in}}%
\pgfpathclose%
\pgfusepath{fill}%
\end{pgfscope}%
\begin{pgfscope}%
\pgfpathrectangle{\pgfqpoint{1.254980in}{0.150000in}}{\pgfqpoint{5.490039in}{5.490039in}}%
\pgfusepath{clip}%
\pgfsetbuttcap%
\pgfsetroundjoin%
\definecolor{currentfill}{rgb}{0.279566,0.067836,0.391917}%
\pgfsetfillcolor{currentfill}%
\pgfsetfillopacity{0.700000}%
\pgfsetlinewidth{0.000000pt}%
\definecolor{currentstroke}{rgb}{0.000000,0.000000,0.000000}%
\pgfsetstrokecolor{currentstroke}%
\pgfsetdash{}{0pt}%
\pgfpathmoveto{\pgfqpoint{4.813068in}{2.421337in}}%
\pgfpathlineto{\pgfqpoint{4.826183in}{2.419166in}}%
\pgfpathlineto{\pgfqpoint{4.839304in}{2.417021in}}%
\pgfpathlineto{\pgfqpoint{4.852433in}{2.414900in}}%
\pgfpathlineto{\pgfqpoint{4.865569in}{2.412803in}}%
\pgfpathlineto{\pgfqpoint{4.858395in}{2.406170in}}%
\pgfpathlineto{\pgfqpoint{4.851216in}{2.399512in}}%
\pgfpathlineto{\pgfqpoint{4.844031in}{2.392825in}}%
\pgfpathlineto{\pgfqpoint{4.836840in}{2.386109in}}%
\pgfpathlineto{\pgfqpoint{4.823691in}{2.388156in}}%
\pgfpathlineto{\pgfqpoint{4.810549in}{2.390227in}}%
\pgfpathlineto{\pgfqpoint{4.797413in}{2.392324in}}%
\pgfpathlineto{\pgfqpoint{4.784285in}{2.394445in}}%
\pgfpathlineto{\pgfqpoint{4.791489in}{2.401206in}}%
\pgfpathlineto{\pgfqpoint{4.798688in}{2.407940in}}%
\pgfpathlineto{\pgfqpoint{4.805881in}{2.414650in}}%
\pgfpathlineto{\pgfqpoint{4.813068in}{2.421337in}}%
\pgfpathclose%
\pgfusepath{fill}%
\end{pgfscope}%
\begin{pgfscope}%
\pgfpathrectangle{\pgfqpoint{1.254980in}{0.150000in}}{\pgfqpoint{5.490039in}{5.490039in}}%
\pgfusepath{clip}%
\pgfsetbuttcap%
\pgfsetroundjoin%
\definecolor{currentfill}{rgb}{0.269944,0.014625,0.341379}%
\pgfsetfillcolor{currentfill}%
\pgfsetfillopacity{0.700000}%
\pgfsetlinewidth{0.000000pt}%
\definecolor{currentstroke}{rgb}{0.000000,0.000000,0.000000}%
\pgfsetstrokecolor{currentstroke}%
\pgfsetdash{}{0pt}%
\pgfpathmoveto{\pgfqpoint{3.339142in}{2.335194in}}%
\pgfpathlineto{\pgfqpoint{3.351928in}{2.330214in}}%
\pgfpathlineto{\pgfqpoint{3.364718in}{2.325270in}}%
\pgfpathlineto{\pgfqpoint{3.377513in}{2.320363in}}%
\pgfpathlineto{\pgfqpoint{3.390312in}{2.315491in}}%
\pgfpathlineto{\pgfqpoint{3.382575in}{2.309266in}}%
\pgfpathlineto{\pgfqpoint{3.374830in}{2.303099in}}%
\pgfpathlineto{\pgfqpoint{3.367079in}{2.296992in}}%
\pgfpathlineto{\pgfqpoint{3.359321in}{2.290949in}}%
\pgfpathlineto{\pgfqpoint{3.346507in}{2.295937in}}%
\pgfpathlineto{\pgfqpoint{3.333697in}{2.300961in}}%
\pgfpathlineto{\pgfqpoint{3.320891in}{2.306022in}}%
\pgfpathlineto{\pgfqpoint{3.308090in}{2.311118in}}%
\pgfpathlineto{\pgfqpoint{3.315864in}{2.317040in}}%
\pgfpathlineto{\pgfqpoint{3.323630in}{2.323029in}}%
\pgfpathlineto{\pgfqpoint{3.331390in}{2.329081in}}%
\pgfpathlineto{\pgfqpoint{3.339142in}{2.335194in}}%
\pgfpathclose%
\pgfusepath{fill}%
\end{pgfscope}%
\begin{pgfscope}%
\pgfpathrectangle{\pgfqpoint{1.254980in}{0.150000in}}{\pgfqpoint{5.490039in}{5.490039in}}%
\pgfusepath{clip}%
\pgfsetbuttcap%
\pgfsetroundjoin%
\definecolor{currentfill}{rgb}{0.281446,0.084320,0.407414}%
\pgfsetfillcolor{currentfill}%
\pgfsetfillopacity{0.700000}%
\pgfsetlinewidth{0.000000pt}%
\definecolor{currentstroke}{rgb}{0.000000,0.000000,0.000000}%
\pgfsetstrokecolor{currentstroke}%
\pgfsetdash{}{0pt}%
\pgfpathmoveto{\pgfqpoint{5.027869in}{2.448225in}}%
\pgfpathlineto{\pgfqpoint{5.041040in}{2.446212in}}%
\pgfpathlineto{\pgfqpoint{5.054218in}{2.444224in}}%
\pgfpathlineto{\pgfqpoint{5.067404in}{2.442260in}}%
\pgfpathlineto{\pgfqpoint{5.080597in}{2.440321in}}%
\pgfpathlineto{\pgfqpoint{5.073512in}{2.434053in}}%
\pgfpathlineto{\pgfqpoint{5.066421in}{2.427768in}}%
\pgfpathlineto{\pgfqpoint{5.059324in}{2.421464in}}%
\pgfpathlineto{\pgfqpoint{5.052222in}{2.415138in}}%
\pgfpathlineto{\pgfqpoint{5.039015in}{2.417003in}}%
\pgfpathlineto{\pgfqpoint{5.025814in}{2.418891in}}%
\pgfpathlineto{\pgfqpoint{5.012621in}{2.420804in}}%
\pgfpathlineto{\pgfqpoint{4.999435in}{2.422741in}}%
\pgfpathlineto{\pgfqpoint{5.006552in}{2.429138in}}%
\pgfpathlineto{\pgfqpoint{5.013663in}{2.435515in}}%
\pgfpathlineto{\pgfqpoint{5.020769in}{2.441877in}}%
\pgfpathlineto{\pgfqpoint{5.027869in}{2.448225in}}%
\pgfpathclose%
\pgfusepath{fill}%
\end{pgfscope}%
\begin{pgfscope}%
\pgfpathrectangle{\pgfqpoint{1.254980in}{0.150000in}}{\pgfqpoint{5.490039in}{5.490039in}}%
\pgfusepath{clip}%
\pgfsetbuttcap%
\pgfsetroundjoin%
\definecolor{currentfill}{rgb}{0.271305,0.019942,0.347269}%
\pgfsetfillcolor{currentfill}%
\pgfsetfillopacity{0.700000}%
\pgfsetlinewidth{0.000000pt}%
\definecolor{currentstroke}{rgb}{0.000000,0.000000,0.000000}%
\pgfsetstrokecolor{currentstroke}%
\pgfsetdash{}{0pt}%
\pgfpathmoveto{\pgfqpoint{4.168629in}{2.343245in}}%
\pgfpathlineto{\pgfqpoint{4.181581in}{2.340234in}}%
\pgfpathlineto{\pgfqpoint{4.194540in}{2.337251in}}%
\pgfpathlineto{\pgfqpoint{4.207505in}{2.334296in}}%
\pgfpathlineto{\pgfqpoint{4.220476in}{2.331368in}}%
\pgfpathlineto{\pgfqpoint{4.213056in}{2.323970in}}%
\pgfpathlineto{\pgfqpoint{4.205630in}{2.316553in}}%
\pgfpathlineto{\pgfqpoint{4.198199in}{2.309118in}}%
\pgfpathlineto{\pgfqpoint{4.190763in}{2.301665in}}%
\pgfpathlineto{\pgfqpoint{4.177780in}{2.304620in}}%
\pgfpathlineto{\pgfqpoint{4.164803in}{2.307603in}}%
\pgfpathlineto{\pgfqpoint{4.151833in}{2.310614in}}%
\pgfpathlineto{\pgfqpoint{4.138868in}{2.313652in}}%
\pgfpathlineto{\pgfqpoint{4.146316in}{2.321073in}}%
\pgfpathlineto{\pgfqpoint{4.153759in}{2.328479in}}%
\pgfpathlineto{\pgfqpoint{4.161197in}{2.335870in}}%
\pgfpathlineto{\pgfqpoint{4.168629in}{2.343245in}}%
\pgfpathclose%
\pgfusepath{fill}%
\end{pgfscope}%
\begin{pgfscope}%
\pgfpathrectangle{\pgfqpoint{1.254980in}{0.150000in}}{\pgfqpoint{5.490039in}{5.490039in}}%
\pgfusepath{clip}%
\pgfsetbuttcap%
\pgfsetroundjoin%
\definecolor{currentfill}{rgb}{0.272594,0.025563,0.353093}%
\pgfsetfillcolor{currentfill}%
\pgfsetfillopacity{0.700000}%
\pgfsetlinewidth{0.000000pt}%
\definecolor{currentstroke}{rgb}{0.000000,0.000000,0.000000}%
\pgfsetstrokecolor{currentstroke}%
\pgfsetdash{}{0pt}%
\pgfpathmoveto{\pgfqpoint{3.205835in}{2.353228in}}%
\pgfpathlineto{\pgfqpoint{3.218603in}{2.347831in}}%
\pgfpathlineto{\pgfqpoint{3.231374in}{2.342473in}}%
\pgfpathlineto{\pgfqpoint{3.244149in}{2.337153in}}%
\pgfpathlineto{\pgfqpoint{3.256929in}{2.331872in}}%
\pgfpathlineto{\pgfqpoint{3.249132in}{2.326143in}}%
\pgfpathlineto{\pgfqpoint{3.241327in}{2.320490in}}%
\pgfpathlineto{\pgfqpoint{3.233515in}{2.314915in}}%
\pgfpathlineto{\pgfqpoint{3.225696in}{2.309421in}}%
\pgfpathlineto{\pgfqpoint{3.212900in}{2.314832in}}%
\pgfpathlineto{\pgfqpoint{3.200108in}{2.320282in}}%
\pgfpathlineto{\pgfqpoint{3.187320in}{2.325769in}}%
\pgfpathlineto{\pgfqpoint{3.174536in}{2.331296in}}%
\pgfpathlineto{\pgfqpoint{3.182372in}{2.336655in}}%
\pgfpathlineto{\pgfqpoint{3.190201in}{2.342099in}}%
\pgfpathlineto{\pgfqpoint{3.198022in}{2.347624in}}%
\pgfpathlineto{\pgfqpoint{3.205835in}{2.353228in}}%
\pgfpathclose%
\pgfusepath{fill}%
\end{pgfscope}%
\begin{pgfscope}%
\pgfpathrectangle{\pgfqpoint{1.254980in}{0.150000in}}{\pgfqpoint{5.490039in}{5.490039in}}%
\pgfusepath{clip}%
\pgfsetbuttcap%
\pgfsetroundjoin%
\definecolor{currentfill}{rgb}{0.268510,0.009605,0.335427}%
\pgfsetfillcolor{currentfill}%
\pgfsetfillopacity{0.700000}%
\pgfsetlinewidth{0.000000pt}%
\definecolor{currentstroke}{rgb}{0.000000,0.000000,0.000000}%
\pgfsetstrokecolor{currentstroke}%
\pgfsetdash{}{0pt}%
\pgfpathmoveto{\pgfqpoint{3.472378in}{2.322198in}}%
\pgfpathlineto{\pgfqpoint{3.485186in}{2.317604in}}%
\pgfpathlineto{\pgfqpoint{3.497999in}{2.313046in}}%
\pgfpathlineto{\pgfqpoint{3.510817in}{2.308521in}}%
\pgfpathlineto{\pgfqpoint{3.523639in}{2.304029in}}%
\pgfpathlineto{\pgfqpoint{3.515958in}{2.297400in}}%
\pgfpathlineto{\pgfqpoint{3.508269in}{2.290813in}}%
\pgfpathlineto{\pgfqpoint{3.500575in}{2.284269in}}%
\pgfpathlineto{\pgfqpoint{3.492874in}{2.277772in}}%
\pgfpathlineto{\pgfqpoint{3.480037in}{2.282367in}}%
\pgfpathlineto{\pgfqpoint{3.467205in}{2.286996in}}%
\pgfpathlineto{\pgfqpoint{3.454377in}{2.291659in}}%
\pgfpathlineto{\pgfqpoint{3.441555in}{2.296356in}}%
\pgfpathlineto{\pgfqpoint{3.449271in}{2.302744in}}%
\pgfpathlineto{\pgfqpoint{3.456980in}{2.309183in}}%
\pgfpathlineto{\pgfqpoint{3.464682in}{2.315668in}}%
\pgfpathlineto{\pgfqpoint{3.472378in}{2.322198in}}%
\pgfpathclose%
\pgfusepath{fill}%
\end{pgfscope}%
\begin{pgfscope}%
\pgfpathrectangle{\pgfqpoint{1.254980in}{0.150000in}}{\pgfqpoint{5.490039in}{5.490039in}}%
\pgfusepath{clip}%
\pgfsetbuttcap%
\pgfsetroundjoin%
\definecolor{currentfill}{rgb}{0.268510,0.009605,0.335427}%
\pgfsetfillcolor{currentfill}%
\pgfsetfillopacity{0.700000}%
\pgfsetlinewidth{0.000000pt}%
\definecolor{currentstroke}{rgb}{0.000000,0.000000,0.000000}%
\pgfsetstrokecolor{currentstroke}%
\pgfsetdash{}{0pt}%
\pgfpathmoveto{\pgfqpoint{3.953773in}{2.322974in}}%
\pgfpathlineto{\pgfqpoint{3.966678in}{2.319554in}}%
\pgfpathlineto{\pgfqpoint{3.979589in}{2.316162in}}%
\pgfpathlineto{\pgfqpoint{3.992505in}{2.312800in}}%
\pgfpathlineto{\pgfqpoint{4.005428in}{2.309466in}}%
\pgfpathlineto{\pgfqpoint{3.997929in}{2.302073in}}%
\pgfpathlineto{\pgfqpoint{3.990424in}{2.294674in}}%
\pgfpathlineto{\pgfqpoint{3.982913in}{2.287271in}}%
\pgfpathlineto{\pgfqpoint{3.975398in}{2.279865in}}%
\pgfpathlineto{\pgfqpoint{3.962463in}{2.283251in}}%
\pgfpathlineto{\pgfqpoint{3.949535in}{2.286667in}}%
\pgfpathlineto{\pgfqpoint{3.936612in}{2.290111in}}%
\pgfpathlineto{\pgfqpoint{3.923695in}{2.293585in}}%
\pgfpathlineto{\pgfqpoint{3.931223in}{2.300934in}}%
\pgfpathlineto{\pgfqpoint{3.938745in}{2.308282in}}%
\pgfpathlineto{\pgfqpoint{3.946262in}{2.315629in}}%
\pgfpathlineto{\pgfqpoint{3.953773in}{2.322974in}}%
\pgfpathclose%
\pgfusepath{fill}%
\end{pgfscope}%
\begin{pgfscope}%
\pgfpathrectangle{\pgfqpoint{1.254980in}{0.150000in}}{\pgfqpoint{5.490039in}{5.490039in}}%
\pgfusepath{clip}%
\pgfsetbuttcap%
\pgfsetroundjoin%
\definecolor{currentfill}{rgb}{0.276022,0.044167,0.370164}%
\pgfsetfillcolor{currentfill}%
\pgfsetfillopacity{0.700000}%
\pgfsetlinewidth{0.000000pt}%
\definecolor{currentstroke}{rgb}{0.000000,0.000000,0.000000}%
\pgfsetstrokecolor{currentstroke}%
\pgfsetdash{}{0pt}%
\pgfpathmoveto{\pgfqpoint{3.072406in}{2.376933in}}%
\pgfpathlineto{\pgfqpoint{3.085159in}{2.371086in}}%
\pgfpathlineto{\pgfqpoint{3.097916in}{2.365281in}}%
\pgfpathlineto{\pgfqpoint{3.110676in}{2.359517in}}%
\pgfpathlineto{\pgfqpoint{3.123440in}{2.353793in}}%
\pgfpathlineto{\pgfqpoint{3.115578in}{2.348658in}}%
\pgfpathlineto{\pgfqpoint{3.107708in}{2.343618in}}%
\pgfpathlineto{\pgfqpoint{3.099829in}{2.338676in}}%
\pgfpathlineto{\pgfqpoint{3.091942in}{2.333834in}}%
\pgfpathlineto{\pgfqpoint{3.079160in}{2.339701in}}%
\pgfpathlineto{\pgfqpoint{3.066382in}{2.345608in}}%
\pgfpathlineto{\pgfqpoint{3.053607in}{2.351556in}}%
\pgfpathlineto{\pgfqpoint{3.040836in}{2.357546in}}%
\pgfpathlineto{\pgfqpoint{3.048742in}{2.362240in}}%
\pgfpathlineto{\pgfqpoint{3.056639in}{2.367038in}}%
\pgfpathlineto{\pgfqpoint{3.064527in}{2.371937in}}%
\pgfpathlineto{\pgfqpoint{3.072406in}{2.376933in}}%
\pgfpathclose%
\pgfusepath{fill}%
\end{pgfscope}%
\begin{pgfscope}%
\pgfpathrectangle{\pgfqpoint{1.254980in}{0.150000in}}{\pgfqpoint{5.490039in}{5.490039in}}%
\pgfusepath{clip}%
\pgfsetbuttcap%
\pgfsetroundjoin%
\definecolor{currentfill}{rgb}{0.267004,0.004874,0.329415}%
\pgfsetfillcolor{currentfill}%
\pgfsetfillopacity{0.700000}%
\pgfsetlinewidth{0.000000pt}%
\definecolor{currentstroke}{rgb}{0.000000,0.000000,0.000000}%
\pgfsetstrokecolor{currentstroke}%
\pgfsetdash{}{0pt}%
\pgfpathmoveto{\pgfqpoint{3.605589in}{2.313653in}}%
\pgfpathlineto{\pgfqpoint{3.618423in}{2.309418in}}%
\pgfpathlineto{\pgfqpoint{3.631262in}{2.305216in}}%
\pgfpathlineto{\pgfqpoint{3.644106in}{2.301047in}}%
\pgfpathlineto{\pgfqpoint{3.656955in}{2.296909in}}%
\pgfpathlineto{\pgfqpoint{3.649325in}{2.289962in}}%
\pgfpathlineto{\pgfqpoint{3.641689in}{2.283042in}}%
\pgfpathlineto{\pgfqpoint{3.634047in}{2.276150in}}%
\pgfpathlineto{\pgfqpoint{3.626399in}{2.269290in}}%
\pgfpathlineto{\pgfqpoint{3.613537in}{2.273519in}}%
\pgfpathlineto{\pgfqpoint{3.600679in}{2.277779in}}%
\pgfpathlineto{\pgfqpoint{3.587827in}{2.282072in}}%
\pgfpathlineto{\pgfqpoint{3.574979in}{2.286398in}}%
\pgfpathlineto{\pgfqpoint{3.582641in}{2.293163in}}%
\pgfpathlineto{\pgfqpoint{3.590296in}{2.299961in}}%
\pgfpathlineto{\pgfqpoint{3.597946in}{2.306792in}}%
\pgfpathlineto{\pgfqpoint{3.605589in}{2.313653in}}%
\pgfpathclose%
\pgfusepath{fill}%
\end{pgfscope}%
\begin{pgfscope}%
\pgfpathrectangle{\pgfqpoint{1.254980in}{0.150000in}}{\pgfqpoint{5.490039in}{5.490039in}}%
\pgfusepath{clip}%
\pgfsetbuttcap%
\pgfsetroundjoin%
\definecolor{currentfill}{rgb}{0.282656,0.100196,0.422160}%
\pgfsetfillcolor{currentfill}%
\pgfsetfillopacity{0.700000}%
\pgfsetlinewidth{0.000000pt}%
\definecolor{currentstroke}{rgb}{0.000000,0.000000,0.000000}%
\pgfsetstrokecolor{currentstroke}%
\pgfsetdash{}{0pt}%
\pgfpathmoveto{\pgfqpoint{2.754004in}{2.472084in}}%
\pgfpathlineto{\pgfqpoint{2.766732in}{2.465025in}}%
\pgfpathlineto{\pgfqpoint{2.779463in}{2.458016in}}%
\pgfpathlineto{\pgfqpoint{2.792197in}{2.451057in}}%
\pgfpathlineto{\pgfqpoint{2.804933in}{2.444148in}}%
\pgfpathlineto{\pgfqpoint{2.796900in}{2.440686in}}%
\pgfpathlineto{\pgfqpoint{2.788856in}{2.437364in}}%
\pgfpathlineto{\pgfqpoint{2.780802in}{2.434186in}}%
\pgfpathlineto{\pgfqpoint{2.772736in}{2.431156in}}%
\pgfpathlineto{\pgfqpoint{2.759978in}{2.438236in}}%
\pgfpathlineto{\pgfqpoint{2.747223in}{2.445364in}}%
\pgfpathlineto{\pgfqpoint{2.734470in}{2.452543in}}%
\pgfpathlineto{\pgfqpoint{2.721720in}{2.459771in}}%
\pgfpathlineto{\pgfqpoint{2.729808in}{2.462626in}}%
\pgfpathlineto{\pgfqpoint{2.737884in}{2.465633in}}%
\pgfpathlineto{\pgfqpoint{2.745950in}{2.468787in}}%
\pgfpathlineto{\pgfqpoint{2.754004in}{2.472084in}}%
\pgfpathclose%
\pgfusepath{fill}%
\end{pgfscope}%
\begin{pgfscope}%
\pgfpathrectangle{\pgfqpoint{1.254980in}{0.150000in}}{\pgfqpoint{5.490039in}{5.490039in}}%
\pgfusepath{clip}%
\pgfsetbuttcap%
\pgfsetroundjoin%
\definecolor{currentfill}{rgb}{0.282884,0.135920,0.453427}%
\pgfsetfillcolor{currentfill}%
\pgfsetfillopacity{0.700000}%
\pgfsetlinewidth{0.000000pt}%
\definecolor{currentstroke}{rgb}{0.000000,0.000000,0.000000}%
\pgfsetstrokecolor{currentstroke}%
\pgfsetdash{}{0pt}%
\pgfpathmoveto{\pgfqpoint{5.806030in}{2.529027in}}%
\pgfpathlineto{\pgfqpoint{5.819407in}{2.527165in}}%
\pgfpathlineto{\pgfqpoint{5.832791in}{2.525327in}}%
\pgfpathlineto{\pgfqpoint{5.846184in}{2.523511in}}%
\pgfpathlineto{\pgfqpoint{5.859584in}{2.521718in}}%
\pgfpathlineto{\pgfqpoint{5.852827in}{2.516349in}}%
\pgfpathlineto{\pgfqpoint{5.846068in}{2.511039in}}%
\pgfpathlineto{\pgfqpoint{5.839305in}{2.505782in}}%
\pgfpathlineto{\pgfqpoint{5.832538in}{2.500574in}}%
\pgfpathlineto{\pgfqpoint{5.819118in}{2.502202in}}%
\pgfpathlineto{\pgfqpoint{5.805705in}{2.503853in}}%
\pgfpathlineto{\pgfqpoint{5.792300in}{2.505526in}}%
\pgfpathlineto{\pgfqpoint{5.778903in}{2.507223in}}%
\pgfpathlineto{\pgfqpoint{5.785690in}{2.512592in}}%
\pgfpathlineto{\pgfqpoint{5.792473in}{2.518012in}}%
\pgfpathlineto{\pgfqpoint{5.799253in}{2.523489in}}%
\pgfpathlineto{\pgfqpoint{5.806030in}{2.529027in}}%
\pgfpathclose%
\pgfusepath{fill}%
\end{pgfscope}%
\begin{pgfscope}%
\pgfpathrectangle{\pgfqpoint{1.254980in}{0.150000in}}{\pgfqpoint{5.490039in}{5.490039in}}%
\pgfusepath{clip}%
\pgfsetbuttcap%
\pgfsetroundjoin%
\definecolor{currentfill}{rgb}{0.283229,0.120777,0.440584}%
\pgfsetfillcolor{currentfill}%
\pgfsetfillopacity{0.700000}%
\pgfsetlinewidth{0.000000pt}%
\definecolor{currentstroke}{rgb}{0.000000,0.000000,0.000000}%
\pgfsetstrokecolor{currentstroke}%
\pgfsetdash{}{0pt}%
\pgfpathmoveto{\pgfqpoint{5.591314in}{2.506494in}}%
\pgfpathlineto{\pgfqpoint{5.604638in}{2.504673in}}%
\pgfpathlineto{\pgfqpoint{5.617970in}{2.502875in}}%
\pgfpathlineto{\pgfqpoint{5.631310in}{2.501100in}}%
\pgfpathlineto{\pgfqpoint{5.644658in}{2.499348in}}%
\pgfpathlineto{\pgfqpoint{5.637811in}{2.493875in}}%
\pgfpathlineto{\pgfqpoint{5.630960in}{2.488431in}}%
\pgfpathlineto{\pgfqpoint{5.624105in}{2.483012in}}%
\pgfpathlineto{\pgfqpoint{5.617245in}{2.477614in}}%
\pgfpathlineto{\pgfqpoint{5.603878in}{2.479226in}}%
\pgfpathlineto{\pgfqpoint{5.590520in}{2.480862in}}%
\pgfpathlineto{\pgfqpoint{5.577169in}{2.482520in}}%
\pgfpathlineto{\pgfqpoint{5.563826in}{2.484202in}}%
\pgfpathlineto{\pgfqpoint{5.570705in}{2.489735in}}%
\pgfpathlineto{\pgfqpoint{5.577579in}{2.495292in}}%
\pgfpathlineto{\pgfqpoint{5.584448in}{2.500877in}}%
\pgfpathlineto{\pgfqpoint{5.591314in}{2.506494in}}%
\pgfpathclose%
\pgfusepath{fill}%
\end{pgfscope}%
\begin{pgfscope}%
\pgfpathrectangle{\pgfqpoint{1.254980in}{0.150000in}}{\pgfqpoint{5.490039in}{5.490039in}}%
\pgfusepath{clip}%
\pgfsetbuttcap%
\pgfsetroundjoin%
\definecolor{currentfill}{rgb}{0.267004,0.004874,0.329415}%
\pgfsetfillcolor{currentfill}%
\pgfsetfillopacity{0.700000}%
\pgfsetlinewidth{0.000000pt}%
\definecolor{currentstroke}{rgb}{0.000000,0.000000,0.000000}%
\pgfsetstrokecolor{currentstroke}%
\pgfsetdash{}{0pt}%
\pgfpathmoveto{\pgfqpoint{3.738813in}{2.309014in}}%
\pgfpathlineto{\pgfqpoint{3.751676in}{2.305112in}}%
\pgfpathlineto{\pgfqpoint{3.764544in}{2.301242in}}%
\pgfpathlineto{\pgfqpoint{3.777418in}{2.297403in}}%
\pgfpathlineto{\pgfqpoint{3.790297in}{2.293594in}}%
\pgfpathlineto{\pgfqpoint{3.782716in}{2.286409in}}%
\pgfpathlineto{\pgfqpoint{3.775129in}{2.279238in}}%
\pgfpathlineto{\pgfqpoint{3.767537in}{2.272082in}}%
\pgfpathlineto{\pgfqpoint{3.759939in}{2.264943in}}%
\pgfpathlineto{\pgfqpoint{3.747047in}{2.268830in}}%
\pgfpathlineto{\pgfqpoint{3.734161in}{2.272748in}}%
\pgfpathlineto{\pgfqpoint{3.721280in}{2.276697in}}%
\pgfpathlineto{\pgfqpoint{3.708405in}{2.280677in}}%
\pgfpathlineto{\pgfqpoint{3.716016in}{2.287732in}}%
\pgfpathlineto{\pgfqpoint{3.723621in}{2.294808in}}%
\pgfpathlineto{\pgfqpoint{3.731220in}{2.301902in}}%
\pgfpathlineto{\pgfqpoint{3.738813in}{2.309014in}}%
\pgfpathclose%
\pgfusepath{fill}%
\end{pgfscope}%
\begin{pgfscope}%
\pgfpathrectangle{\pgfqpoint{1.254980in}{0.150000in}}{\pgfqpoint{5.490039in}{5.490039in}}%
\pgfusepath{clip}%
\pgfsetbuttcap%
\pgfsetroundjoin%
\definecolor{currentfill}{rgb}{0.276022,0.044167,0.370164}%
\pgfsetfillcolor{currentfill}%
\pgfsetfillopacity{0.700000}%
\pgfsetlinewidth{0.000000pt}%
\definecolor{currentstroke}{rgb}{0.000000,0.000000,0.000000}%
\pgfsetstrokecolor{currentstroke}%
\pgfsetdash{}{0pt}%
\pgfpathmoveto{\pgfqpoint{4.516920in}{2.375642in}}%
\pgfpathlineto{\pgfqpoint{4.529964in}{2.373174in}}%
\pgfpathlineto{\pgfqpoint{4.543015in}{2.370731in}}%
\pgfpathlineto{\pgfqpoint{4.556073in}{2.368315in}}%
\pgfpathlineto{\pgfqpoint{4.569138in}{2.365924in}}%
\pgfpathlineto{\pgfqpoint{4.561844in}{2.358813in}}%
\pgfpathlineto{\pgfqpoint{4.554545in}{2.351671in}}%
\pgfpathlineto{\pgfqpoint{4.547240in}{2.344498in}}%
\pgfpathlineto{\pgfqpoint{4.539929in}{2.337293in}}%
\pgfpathlineto{\pgfqpoint{4.526852in}{2.339673in}}%
\pgfpathlineto{\pgfqpoint{4.513782in}{2.342078in}}%
\pgfpathlineto{\pgfqpoint{4.500719in}{2.344510in}}%
\pgfpathlineto{\pgfqpoint{4.487662in}{2.346967in}}%
\pgfpathlineto{\pgfqpoint{4.494985in}{2.354178in}}%
\pgfpathlineto{\pgfqpoint{4.502302in}{2.361360in}}%
\pgfpathlineto{\pgfqpoint{4.509614in}{2.368515in}}%
\pgfpathlineto{\pgfqpoint{4.516920in}{2.375642in}}%
\pgfpathclose%
\pgfusepath{fill}%
\end{pgfscope}%
\begin{pgfscope}%
\pgfpathrectangle{\pgfqpoint{1.254980in}{0.150000in}}{\pgfqpoint{5.490039in}{5.490039in}}%
\pgfusepath{clip}%
\pgfsetbuttcap%
\pgfsetroundjoin%
\definecolor{currentfill}{rgb}{0.283091,0.110553,0.431554}%
\pgfsetfillcolor{currentfill}%
\pgfsetfillopacity{0.700000}%
\pgfsetlinewidth{0.000000pt}%
\definecolor{currentstroke}{rgb}{0.000000,0.000000,0.000000}%
\pgfsetstrokecolor{currentstroke}%
\pgfsetdash{}{0pt}%
\pgfpathmoveto{\pgfqpoint{5.376522in}{2.482709in}}%
\pgfpathlineto{\pgfqpoint{5.389792in}{2.480873in}}%
\pgfpathlineto{\pgfqpoint{5.403070in}{2.479060in}}%
\pgfpathlineto{\pgfqpoint{5.416355in}{2.477271in}}%
\pgfpathlineto{\pgfqpoint{5.429647in}{2.475505in}}%
\pgfpathlineto{\pgfqpoint{5.422708in}{2.469792in}}%
\pgfpathlineto{\pgfqpoint{5.415763in}{2.464084in}}%
\pgfpathlineto{\pgfqpoint{5.408814in}{2.458379in}}%
\pgfpathlineto{\pgfqpoint{5.401858in}{2.452672in}}%
\pgfpathlineto{\pgfqpoint{5.388549in}{2.454324in}}%
\pgfpathlineto{\pgfqpoint{5.375247in}{2.455999in}}%
\pgfpathlineto{\pgfqpoint{5.361952in}{2.457697in}}%
\pgfpathlineto{\pgfqpoint{5.348665in}{2.459420in}}%
\pgfpathlineto{\pgfqpoint{5.355637in}{2.465236in}}%
\pgfpathlineto{\pgfqpoint{5.362604in}{2.471053in}}%
\pgfpathlineto{\pgfqpoint{5.369566in}{2.476876in}}%
\pgfpathlineto{\pgfqpoint{5.376522in}{2.482709in}}%
\pgfpathclose%
\pgfusepath{fill}%
\end{pgfscope}%
\begin{pgfscope}%
\pgfpathrectangle{\pgfqpoint{1.254980in}{0.150000in}}{\pgfqpoint{5.490039in}{5.490039in}}%
\pgfusepath{clip}%
\pgfsetbuttcap%
\pgfsetroundjoin%
\definecolor{currentfill}{rgb}{0.273809,0.031497,0.358853}%
\pgfsetfillcolor{currentfill}%
\pgfsetfillopacity{0.700000}%
\pgfsetlinewidth{0.000000pt}%
\definecolor{currentstroke}{rgb}{0.000000,0.000000,0.000000}%
\pgfsetstrokecolor{currentstroke}%
\pgfsetdash{}{0pt}%
\pgfpathmoveto{\pgfqpoint{4.302001in}{2.349391in}}%
\pgfpathlineto{\pgfqpoint{4.314992in}{2.346614in}}%
\pgfpathlineto{\pgfqpoint{4.327989in}{2.343864in}}%
\pgfpathlineto{\pgfqpoint{4.340993in}{2.341140in}}%
\pgfpathlineto{\pgfqpoint{4.354002in}{2.338444in}}%
\pgfpathlineto{\pgfqpoint{4.346628in}{2.331101in}}%
\pgfpathlineto{\pgfqpoint{4.339248in}{2.323734in}}%
\pgfpathlineto{\pgfqpoint{4.331862in}{2.316341in}}%
\pgfpathlineto{\pgfqpoint{4.324471in}{2.308923in}}%
\pgfpathlineto{\pgfqpoint{4.311450in}{2.311635in}}%
\pgfpathlineto{\pgfqpoint{4.298434in}{2.314372in}}%
\pgfpathlineto{\pgfqpoint{4.285425in}{2.317137in}}%
\pgfpathlineto{\pgfqpoint{4.272423in}{2.319929in}}%
\pgfpathlineto{\pgfqpoint{4.279826in}{2.327328in}}%
\pgfpathlineto{\pgfqpoint{4.287223in}{2.334704in}}%
\pgfpathlineto{\pgfqpoint{4.294615in}{2.342058in}}%
\pgfpathlineto{\pgfqpoint{4.302001in}{2.349391in}}%
\pgfpathclose%
\pgfusepath{fill}%
\end{pgfscope}%
\begin{pgfscope}%
\pgfpathrectangle{\pgfqpoint{1.254980in}{0.150000in}}{\pgfqpoint{5.490039in}{5.490039in}}%
\pgfusepath{clip}%
\pgfsetbuttcap%
\pgfsetroundjoin%
\definecolor{currentfill}{rgb}{0.279566,0.067836,0.391917}%
\pgfsetfillcolor{currentfill}%
\pgfsetfillopacity{0.700000}%
\pgfsetlinewidth{0.000000pt}%
\definecolor{currentstroke}{rgb}{0.000000,0.000000,0.000000}%
\pgfsetstrokecolor{currentstroke}%
\pgfsetdash{}{0pt}%
\pgfpathmoveto{\pgfqpoint{2.938795in}{2.406993in}}%
\pgfpathlineto{\pgfqpoint{2.951538in}{2.400660in}}%
\pgfpathlineto{\pgfqpoint{2.964285in}{2.394371in}}%
\pgfpathlineto{\pgfqpoint{2.977035in}{2.388125in}}%
\pgfpathlineto{\pgfqpoint{2.989788in}{2.381924in}}%
\pgfpathlineto{\pgfqpoint{2.981855in}{2.377488in}}%
\pgfpathlineto{\pgfqpoint{2.973912in}{2.373167in}}%
\pgfpathlineto{\pgfqpoint{2.965960in}{2.368965in}}%
\pgfpathlineto{\pgfqpoint{2.957999in}{2.364884in}}%
\pgfpathlineto{\pgfqpoint{2.945226in}{2.371242in}}%
\pgfpathlineto{\pgfqpoint{2.932457in}{2.377643in}}%
\pgfpathlineto{\pgfqpoint{2.919690in}{2.384088in}}%
\pgfpathlineto{\pgfqpoint{2.906927in}{2.390577in}}%
\pgfpathlineto{\pgfqpoint{2.914909in}{2.394497in}}%
\pgfpathlineto{\pgfqpoint{2.922881in}{2.398542in}}%
\pgfpathlineto{\pgfqpoint{2.930842in}{2.402709in}}%
\pgfpathlineto{\pgfqpoint{2.938795in}{2.406993in}}%
\pgfpathclose%
\pgfusepath{fill}%
\end{pgfscope}%
\begin{pgfscope}%
\pgfpathrectangle{\pgfqpoint{1.254980in}{0.150000in}}{\pgfqpoint{5.490039in}{5.490039in}}%
\pgfusepath{clip}%
\pgfsetbuttcap%
\pgfsetroundjoin%
\definecolor{currentfill}{rgb}{0.278791,0.062145,0.386592}%
\pgfsetfillcolor{currentfill}%
\pgfsetfillopacity{0.700000}%
\pgfsetlinewidth{0.000000pt}%
\definecolor{currentstroke}{rgb}{0.000000,0.000000,0.000000}%
\pgfsetstrokecolor{currentstroke}%
\pgfsetdash{}{0pt}%
\pgfpathmoveto{\pgfqpoint{4.731842in}{2.403181in}}%
\pgfpathlineto{\pgfqpoint{4.744943in}{2.400959in}}%
\pgfpathlineto{\pgfqpoint{4.758050in}{2.398763in}}%
\pgfpathlineto{\pgfqpoint{4.771164in}{2.396591in}}%
\pgfpathlineto{\pgfqpoint{4.784285in}{2.394445in}}%
\pgfpathlineto{\pgfqpoint{4.777075in}{2.387657in}}%
\pgfpathlineto{\pgfqpoint{4.769860in}{2.380839in}}%
\pgfpathlineto{\pgfqpoint{4.762638in}{2.373990in}}%
\pgfpathlineto{\pgfqpoint{4.755411in}{2.367108in}}%
\pgfpathlineto{\pgfqpoint{4.742277in}{2.369217in}}%
\pgfpathlineto{\pgfqpoint{4.729150in}{2.371352in}}%
\pgfpathlineto{\pgfqpoint{4.716029in}{2.373512in}}%
\pgfpathlineto{\pgfqpoint{4.702916in}{2.375697in}}%
\pgfpathlineto{\pgfqpoint{4.710156in}{2.382611in}}%
\pgfpathlineto{\pgfqpoint{4.717391in}{2.389495in}}%
\pgfpathlineto{\pgfqpoint{4.724619in}{2.396351in}}%
\pgfpathlineto{\pgfqpoint{4.731842in}{2.403181in}}%
\pgfpathclose%
\pgfusepath{fill}%
\end{pgfscope}%
\begin{pgfscope}%
\pgfpathrectangle{\pgfqpoint{1.254980in}{0.150000in}}{\pgfqpoint{5.490039in}{5.490039in}}%
\pgfusepath{clip}%
\pgfsetbuttcap%
\pgfsetroundjoin%
\definecolor{currentfill}{rgb}{0.282327,0.094955,0.417331}%
\pgfsetfillcolor{currentfill}%
\pgfsetfillopacity{0.700000}%
\pgfsetlinewidth{0.000000pt}%
\definecolor{currentstroke}{rgb}{0.000000,0.000000,0.000000}%
\pgfsetstrokecolor{currentstroke}%
\pgfsetdash{}{0pt}%
\pgfpathmoveto{\pgfqpoint{5.161665in}{2.457413in}}%
\pgfpathlineto{\pgfqpoint{5.174879in}{2.455507in}}%
\pgfpathlineto{\pgfqpoint{5.188100in}{2.453623in}}%
\pgfpathlineto{\pgfqpoint{5.201329in}{2.451764in}}%
\pgfpathlineto{\pgfqpoint{5.214565in}{2.449929in}}%
\pgfpathlineto{\pgfqpoint{5.207533in}{2.443885in}}%
\pgfpathlineto{\pgfqpoint{5.200495in}{2.437830in}}%
\pgfpathlineto{\pgfqpoint{5.193452in}{2.431760in}}%
\pgfpathlineto{\pgfqpoint{5.186403in}{2.425673in}}%
\pgfpathlineto{\pgfqpoint{5.173151in}{2.427420in}}%
\pgfpathlineto{\pgfqpoint{5.159907in}{2.429190in}}%
\pgfpathlineto{\pgfqpoint{5.146671in}{2.430985in}}%
\pgfpathlineto{\pgfqpoint{5.133441in}{2.432804in}}%
\pgfpathlineto{\pgfqpoint{5.140506in}{2.438975in}}%
\pgfpathlineto{\pgfqpoint{5.147564in}{2.445132in}}%
\pgfpathlineto{\pgfqpoint{5.154617in}{2.451277in}}%
\pgfpathlineto{\pgfqpoint{5.161665in}{2.457413in}}%
\pgfpathclose%
\pgfusepath{fill}%
\end{pgfscope}%
\begin{pgfscope}%
\pgfpathrectangle{\pgfqpoint{1.254980in}{0.150000in}}{\pgfqpoint{5.490039in}{5.490039in}}%
\pgfusepath{clip}%
\pgfsetbuttcap%
\pgfsetroundjoin%
\definecolor{currentfill}{rgb}{0.280894,0.078907,0.402329}%
\pgfsetfillcolor{currentfill}%
\pgfsetfillopacity{0.700000}%
\pgfsetlinewidth{0.000000pt}%
\definecolor{currentstroke}{rgb}{0.000000,0.000000,0.000000}%
\pgfsetstrokecolor{currentstroke}%
\pgfsetdash{}{0pt}%
\pgfpathmoveto{\pgfqpoint{4.946764in}{2.430735in}}%
\pgfpathlineto{\pgfqpoint{4.959921in}{2.428700in}}%
\pgfpathlineto{\pgfqpoint{4.973085in}{2.426689in}}%
\pgfpathlineto{\pgfqpoint{4.986257in}{2.424703in}}%
\pgfpathlineto{\pgfqpoint{4.999435in}{2.422741in}}%
\pgfpathlineto{\pgfqpoint{4.992313in}{2.416324in}}%
\pgfpathlineto{\pgfqpoint{4.985184in}{2.409883in}}%
\pgfpathlineto{\pgfqpoint{4.978050in}{2.403417in}}%
\pgfpathlineto{\pgfqpoint{4.970910in}{2.396923in}}%
\pgfpathlineto{\pgfqpoint{4.957717in}{2.398822in}}%
\pgfpathlineto{\pgfqpoint{4.944532in}{2.400745in}}%
\pgfpathlineto{\pgfqpoint{4.931353in}{2.402694in}}%
\pgfpathlineto{\pgfqpoint{4.918182in}{2.404666in}}%
\pgfpathlineto{\pgfqpoint{4.925336in}{2.411218in}}%
\pgfpathlineto{\pgfqpoint{4.932484in}{2.417746in}}%
\pgfpathlineto{\pgfqpoint{4.939627in}{2.424250in}}%
\pgfpathlineto{\pgfqpoint{4.946764in}{2.430735in}}%
\pgfpathclose%
\pgfusepath{fill}%
\end{pgfscope}%
\begin{pgfscope}%
\pgfpathrectangle{\pgfqpoint{1.254980in}{0.150000in}}{\pgfqpoint{5.490039in}{5.490039in}}%
\pgfusepath{clip}%
\pgfsetbuttcap%
\pgfsetroundjoin%
\definecolor{currentfill}{rgb}{0.269944,0.014625,0.341379}%
\pgfsetfillcolor{currentfill}%
\pgfsetfillopacity{0.700000}%
\pgfsetlinewidth{0.000000pt}%
\definecolor{currentstroke}{rgb}{0.000000,0.000000,0.000000}%
\pgfsetstrokecolor{currentstroke}%
\pgfsetdash{}{0pt}%
\pgfpathmoveto{\pgfqpoint{4.087071in}{2.326085in}}%
\pgfpathlineto{\pgfqpoint{4.100011in}{2.322935in}}%
\pgfpathlineto{\pgfqpoint{4.112958in}{2.319813in}}%
\pgfpathlineto{\pgfqpoint{4.125910in}{2.316718in}}%
\pgfpathlineto{\pgfqpoint{4.138868in}{2.313652in}}%
\pgfpathlineto{\pgfqpoint{4.131414in}{2.306216in}}%
\pgfpathlineto{\pgfqpoint{4.123955in}{2.298766in}}%
\pgfpathlineto{\pgfqpoint{4.116491in}{2.291302in}}%
\pgfpathlineto{\pgfqpoint{4.109021in}{2.283824in}}%
\pgfpathlineto{\pgfqpoint{4.096051in}{2.286931in}}%
\pgfpathlineto{\pgfqpoint{4.083087in}{2.290065in}}%
\pgfpathlineto{\pgfqpoint{4.070128in}{2.293228in}}%
\pgfpathlineto{\pgfqpoint{4.057176in}{2.296419in}}%
\pgfpathlineto{\pgfqpoint{4.064658in}{2.303851in}}%
\pgfpathlineto{\pgfqpoint{4.072135in}{2.311273in}}%
\pgfpathlineto{\pgfqpoint{4.079606in}{2.318685in}}%
\pgfpathlineto{\pgfqpoint{4.087071in}{2.326085in}}%
\pgfpathclose%
\pgfusepath{fill}%
\end{pgfscope}%
\begin{pgfscope}%
\pgfpathrectangle{\pgfqpoint{1.254980in}{0.150000in}}{\pgfqpoint{5.490039in}{5.490039in}}%
\pgfusepath{clip}%
\pgfsetbuttcap%
\pgfsetroundjoin%
\definecolor{currentfill}{rgb}{0.268510,0.009605,0.335427}%
\pgfsetfillcolor{currentfill}%
\pgfsetfillopacity{0.700000}%
\pgfsetlinewidth{0.000000pt}%
\definecolor{currentstroke}{rgb}{0.000000,0.000000,0.000000}%
\pgfsetstrokecolor{currentstroke}%
\pgfsetdash{}{0pt}%
\pgfpathmoveto{\pgfqpoint{3.872084in}{2.307775in}}%
\pgfpathlineto{\pgfqpoint{3.884978in}{2.304183in}}%
\pgfpathlineto{\pgfqpoint{3.897878in}{2.300621in}}%
\pgfpathlineto{\pgfqpoint{3.910784in}{2.297088in}}%
\pgfpathlineto{\pgfqpoint{3.923695in}{2.293585in}}%
\pgfpathlineto{\pgfqpoint{3.916161in}{2.286237in}}%
\pgfpathlineto{\pgfqpoint{3.908622in}{2.278891in}}%
\pgfpathlineto{\pgfqpoint{3.901078in}{2.271549in}}%
\pgfpathlineto{\pgfqpoint{3.893528in}{2.264210in}}%
\pgfpathlineto{\pgfqpoint{3.880604in}{2.267779in}}%
\pgfpathlineto{\pgfqpoint{3.867686in}{2.271377in}}%
\pgfpathlineto{\pgfqpoint{3.854774in}{2.275005in}}%
\pgfpathlineto{\pgfqpoint{3.841867in}{2.278663in}}%
\pgfpathlineto{\pgfqpoint{3.849430in}{2.285931in}}%
\pgfpathlineto{\pgfqpoint{3.856987in}{2.293206in}}%
\pgfpathlineto{\pgfqpoint{3.864538in}{2.300488in}}%
\pgfpathlineto{\pgfqpoint{3.872084in}{2.307775in}}%
\pgfpathclose%
\pgfusepath{fill}%
\end{pgfscope}%
\begin{pgfscope}%
\pgfpathrectangle{\pgfqpoint{1.254980in}{0.150000in}}{\pgfqpoint{5.490039in}{5.490039in}}%
\pgfusepath{clip}%
\pgfsetbuttcap%
\pgfsetroundjoin%
\definecolor{currentfill}{rgb}{0.268510,0.009605,0.335427}%
\pgfsetfillcolor{currentfill}%
\pgfsetfillopacity{0.700000}%
\pgfsetlinewidth{0.000000pt}%
\definecolor{currentstroke}{rgb}{0.000000,0.000000,0.000000}%
\pgfsetstrokecolor{currentstroke}%
\pgfsetdash{}{0pt}%
\pgfpathmoveto{\pgfqpoint{3.390312in}{2.315491in}}%
\pgfpathlineto{\pgfqpoint{3.403116in}{2.310655in}}%
\pgfpathlineto{\pgfqpoint{3.415924in}{2.305854in}}%
\pgfpathlineto{\pgfqpoint{3.428737in}{2.301087in}}%
\pgfpathlineto{\pgfqpoint{3.441555in}{2.296356in}}%
\pgfpathlineto{\pgfqpoint{3.433833in}{2.290019in}}%
\pgfpathlineto{\pgfqpoint{3.426104in}{2.283737in}}%
\pgfpathlineto{\pgfqpoint{3.418368in}{2.277512in}}%
\pgfpathlineto{\pgfqpoint{3.410625in}{2.271346in}}%
\pgfpathlineto{\pgfqpoint{3.397792in}{2.276195in}}%
\pgfpathlineto{\pgfqpoint{3.384964in}{2.281077in}}%
\pgfpathlineto{\pgfqpoint{3.372140in}{2.285995in}}%
\pgfpathlineto{\pgfqpoint{3.359321in}{2.290949in}}%
\pgfpathlineto{\pgfqpoint{3.367079in}{2.296992in}}%
\pgfpathlineto{\pgfqpoint{3.374830in}{2.303099in}}%
\pgfpathlineto{\pgfqpoint{3.382575in}{2.309266in}}%
\pgfpathlineto{\pgfqpoint{3.390312in}{2.315491in}}%
\pgfpathclose%
\pgfusepath{fill}%
\end{pgfscope}%
\begin{pgfscope}%
\pgfpathrectangle{\pgfqpoint{1.254980in}{0.150000in}}{\pgfqpoint{5.490039in}{5.490039in}}%
\pgfusepath{clip}%
\pgfsetbuttcap%
\pgfsetroundjoin%
\definecolor{currentfill}{rgb}{0.271305,0.019942,0.347269}%
\pgfsetfillcolor{currentfill}%
\pgfsetfillopacity{0.700000}%
\pgfsetlinewidth{0.000000pt}%
\definecolor{currentstroke}{rgb}{0.000000,0.000000,0.000000}%
\pgfsetstrokecolor{currentstroke}%
\pgfsetdash{}{0pt}%
\pgfpathmoveto{\pgfqpoint{3.256929in}{2.331872in}}%
\pgfpathlineto{\pgfqpoint{3.269713in}{2.326628in}}%
\pgfpathlineto{\pgfqpoint{3.282501in}{2.321421in}}%
\pgfpathlineto{\pgfqpoint{3.295293in}{2.316251in}}%
\pgfpathlineto{\pgfqpoint{3.308090in}{2.311118in}}%
\pgfpathlineto{\pgfqpoint{3.300309in}{2.305265in}}%
\pgfpathlineto{\pgfqpoint{3.292521in}{2.299483in}}%
\pgfpathlineto{\pgfqpoint{3.284725in}{2.293777in}}%
\pgfpathlineto{\pgfqpoint{3.276922in}{2.288149in}}%
\pgfpathlineto{\pgfqpoint{3.264109in}{2.293411in}}%
\pgfpathlineto{\pgfqpoint{3.251300in}{2.298711in}}%
\pgfpathlineto{\pgfqpoint{3.238496in}{2.304047in}}%
\pgfpathlineto{\pgfqpoint{3.225696in}{2.309421in}}%
\pgfpathlineto{\pgfqpoint{3.233515in}{2.314915in}}%
\pgfpathlineto{\pgfqpoint{3.241327in}{2.320490in}}%
\pgfpathlineto{\pgfqpoint{3.249132in}{2.326143in}}%
\pgfpathlineto{\pgfqpoint{3.256929in}{2.331872in}}%
\pgfpathclose%
\pgfusepath{fill}%
\end{pgfscope}%
\begin{pgfscope}%
\pgfpathrectangle{\pgfqpoint{1.254980in}{0.150000in}}{\pgfqpoint{5.490039in}{5.490039in}}%
\pgfusepath{clip}%
\pgfsetbuttcap%
\pgfsetroundjoin%
\definecolor{currentfill}{rgb}{0.267004,0.004874,0.329415}%
\pgfsetfillcolor{currentfill}%
\pgfsetfillopacity{0.700000}%
\pgfsetlinewidth{0.000000pt}%
\definecolor{currentstroke}{rgb}{0.000000,0.000000,0.000000}%
\pgfsetstrokecolor{currentstroke}%
\pgfsetdash{}{0pt}%
\pgfpathmoveto{\pgfqpoint{3.523639in}{2.304029in}}%
\pgfpathlineto{\pgfqpoint{3.536467in}{2.299572in}}%
\pgfpathlineto{\pgfqpoint{3.549299in}{2.295148in}}%
\pgfpathlineto{\pgfqpoint{3.562137in}{2.290756in}}%
\pgfpathlineto{\pgfqpoint{3.574979in}{2.286398in}}%
\pgfpathlineto{\pgfqpoint{3.567311in}{2.279670in}}%
\pgfpathlineto{\pgfqpoint{3.559637in}{2.272980in}}%
\pgfpathlineto{\pgfqpoint{3.551957in}{2.266331in}}%
\pgfpathlineto{\pgfqpoint{3.544270in}{2.259725in}}%
\pgfpathlineto{\pgfqpoint{3.531413in}{2.264187in}}%
\pgfpathlineto{\pgfqpoint{3.518562in}{2.268682in}}%
\pgfpathlineto{\pgfqpoint{3.505715in}{2.273210in}}%
\pgfpathlineto{\pgfqpoint{3.492874in}{2.277772in}}%
\pgfpathlineto{\pgfqpoint{3.500575in}{2.284269in}}%
\pgfpathlineto{\pgfqpoint{3.508269in}{2.290813in}}%
\pgfpathlineto{\pgfqpoint{3.515958in}{2.297400in}}%
\pgfpathlineto{\pgfqpoint{3.523639in}{2.304029in}}%
\pgfpathclose%
\pgfusepath{fill}%
\end{pgfscope}%
\begin{pgfscope}%
\pgfpathrectangle{\pgfqpoint{1.254980in}{0.150000in}}{\pgfqpoint{5.490039in}{5.490039in}}%
\pgfusepath{clip}%
\pgfsetbuttcap%
\pgfsetroundjoin%
\definecolor{currentfill}{rgb}{0.283072,0.130895,0.449241}%
\pgfsetfillcolor{currentfill}%
\pgfsetfillopacity{0.700000}%
\pgfsetlinewidth{0.000000pt}%
\definecolor{currentstroke}{rgb}{0.000000,0.000000,0.000000}%
\pgfsetstrokecolor{currentstroke}%
\pgfsetdash{}{0pt}%
\pgfpathmoveto{\pgfqpoint{5.725391in}{2.514240in}}%
\pgfpathlineto{\pgfqpoint{5.738757in}{2.512451in}}%
\pgfpathlineto{\pgfqpoint{5.752131in}{2.510685in}}%
\pgfpathlineto{\pgfqpoint{5.765513in}{2.508943in}}%
\pgfpathlineto{\pgfqpoint{5.778903in}{2.507223in}}%
\pgfpathlineto{\pgfqpoint{5.772112in}{2.501901in}}%
\pgfpathlineto{\pgfqpoint{5.765317in}{2.496620in}}%
\pgfpathlineto{\pgfqpoint{5.758518in}{2.491376in}}%
\pgfpathlineto{\pgfqpoint{5.751715in}{2.486164in}}%
\pgfpathlineto{\pgfqpoint{5.738306in}{2.487732in}}%
\pgfpathlineto{\pgfqpoint{5.724904in}{2.489322in}}%
\pgfpathlineto{\pgfqpoint{5.711510in}{2.490935in}}%
\pgfpathlineto{\pgfqpoint{5.698124in}{2.492572in}}%
\pgfpathlineto{\pgfqpoint{5.704947in}{2.497931in}}%
\pgfpathlineto{\pgfqpoint{5.711766in}{2.503326in}}%
\pgfpathlineto{\pgfqpoint{5.718580in}{2.508760in}}%
\pgfpathlineto{\pgfqpoint{5.725391in}{2.514240in}}%
\pgfpathclose%
\pgfusepath{fill}%
\end{pgfscope}%
\begin{pgfscope}%
\pgfpathrectangle{\pgfqpoint{1.254980in}{0.150000in}}{\pgfqpoint{5.490039in}{5.490039in}}%
\pgfusepath{clip}%
\pgfsetbuttcap%
\pgfsetroundjoin%
\definecolor{currentfill}{rgb}{0.281924,0.089666,0.412415}%
\pgfsetfillcolor{currentfill}%
\pgfsetfillopacity{0.700000}%
\pgfsetlinewidth{0.000000pt}%
\definecolor{currentstroke}{rgb}{0.000000,0.000000,0.000000}%
\pgfsetstrokecolor{currentstroke}%
\pgfsetdash{}{0pt}%
\pgfpathmoveto{\pgfqpoint{2.804933in}{2.444148in}}%
\pgfpathlineto{\pgfqpoint{2.817672in}{2.437286in}}%
\pgfpathlineto{\pgfqpoint{2.830414in}{2.430473in}}%
\pgfpathlineto{\pgfqpoint{2.843159in}{2.423707in}}%
\pgfpathlineto{\pgfqpoint{2.855906in}{2.416989in}}%
\pgfpathlineto{\pgfqpoint{2.847894in}{2.413363in}}%
\pgfpathlineto{\pgfqpoint{2.839872in}{2.409873in}}%
\pgfpathlineto{\pgfqpoint{2.831839in}{2.406524in}}%
\pgfpathlineto{\pgfqpoint{2.823795in}{2.403321in}}%
\pgfpathlineto{\pgfqpoint{2.811026in}{2.410208in}}%
\pgfpathlineto{\pgfqpoint{2.798260in}{2.417143in}}%
\pgfpathlineto{\pgfqpoint{2.785497in}{2.424126in}}%
\pgfpathlineto{\pgfqpoint{2.772736in}{2.431156in}}%
\pgfpathlineto{\pgfqpoint{2.780802in}{2.434186in}}%
\pgfpathlineto{\pgfqpoint{2.788856in}{2.437364in}}%
\pgfpathlineto{\pgfqpoint{2.796900in}{2.440686in}}%
\pgfpathlineto{\pgfqpoint{2.804933in}{2.444148in}}%
\pgfpathclose%
\pgfusepath{fill}%
\end{pgfscope}%
\begin{pgfscope}%
\pgfpathrectangle{\pgfqpoint{1.254980in}{0.150000in}}{\pgfqpoint{5.490039in}{5.490039in}}%
\pgfusepath{clip}%
\pgfsetbuttcap%
\pgfsetroundjoin%
\definecolor{currentfill}{rgb}{0.274952,0.037752,0.364543}%
\pgfsetfillcolor{currentfill}%
\pgfsetfillopacity{0.700000}%
\pgfsetlinewidth{0.000000pt}%
\definecolor{currentstroke}{rgb}{0.000000,0.000000,0.000000}%
\pgfsetstrokecolor{currentstroke}%
\pgfsetdash{}{0pt}%
\pgfpathmoveto{\pgfqpoint{3.123440in}{2.353793in}}%
\pgfpathlineto{\pgfqpoint{3.136208in}{2.348109in}}%
\pgfpathlineto{\pgfqpoint{3.148980in}{2.342465in}}%
\pgfpathlineto{\pgfqpoint{3.161756in}{2.336861in}}%
\pgfpathlineto{\pgfqpoint{3.174536in}{2.331296in}}%
\pgfpathlineto{\pgfqpoint{3.166691in}{2.326024in}}%
\pgfpathlineto{\pgfqpoint{3.158839in}{2.320843in}}%
\pgfpathlineto{\pgfqpoint{3.150978in}{2.315756in}}%
\pgfpathlineto{\pgfqpoint{3.143108in}{2.310767in}}%
\pgfpathlineto{\pgfqpoint{3.130311in}{2.316475in}}%
\pgfpathlineto{\pgfqpoint{3.117518in}{2.322222in}}%
\pgfpathlineto{\pgfqpoint{3.104728in}{2.328008in}}%
\pgfpathlineto{\pgfqpoint{3.091942in}{2.333834in}}%
\pgfpathlineto{\pgfqpoint{3.099829in}{2.338676in}}%
\pgfpathlineto{\pgfqpoint{3.107708in}{2.343618in}}%
\pgfpathlineto{\pgfqpoint{3.115578in}{2.348658in}}%
\pgfpathlineto{\pgfqpoint{3.123440in}{2.353793in}}%
\pgfpathclose%
\pgfusepath{fill}%
\end{pgfscope}%
\begin{pgfscope}%
\pgfpathrectangle{\pgfqpoint{1.254980in}{0.150000in}}{\pgfqpoint{5.490039in}{5.490039in}}%
\pgfusepath{clip}%
\pgfsetbuttcap%
\pgfsetroundjoin%
\definecolor{currentfill}{rgb}{0.274952,0.037752,0.364543}%
\pgfsetfillcolor{currentfill}%
\pgfsetfillopacity{0.700000}%
\pgfsetlinewidth{0.000000pt}%
\definecolor{currentstroke}{rgb}{0.000000,0.000000,0.000000}%
\pgfsetstrokecolor{currentstroke}%
\pgfsetdash{}{0pt}%
\pgfpathmoveto{\pgfqpoint{4.435501in}{2.357058in}}%
\pgfpathlineto{\pgfqpoint{4.448532in}{2.354496in}}%
\pgfpathlineto{\pgfqpoint{4.461569in}{2.351960in}}%
\pgfpathlineto{\pgfqpoint{4.474612in}{2.349451in}}%
\pgfpathlineto{\pgfqpoint{4.487662in}{2.346967in}}%
\pgfpathlineto{\pgfqpoint{4.480334in}{2.339727in}}%
\pgfpathlineto{\pgfqpoint{4.473000in}{2.332457in}}%
\pgfpathlineto{\pgfqpoint{4.465660in}{2.325157in}}%
\pgfpathlineto{\pgfqpoint{4.458315in}{2.317826in}}%
\pgfpathlineto{\pgfqpoint{4.445253in}{2.320311in}}%
\pgfpathlineto{\pgfqpoint{4.432198in}{2.322823in}}%
\pgfpathlineto{\pgfqpoint{4.419149in}{2.325360in}}%
\pgfpathlineto{\pgfqpoint{4.406106in}{2.327924in}}%
\pgfpathlineto{\pgfqpoint{4.413463in}{2.335248in}}%
\pgfpathlineto{\pgfqpoint{4.420815in}{2.342545in}}%
\pgfpathlineto{\pgfqpoint{4.428161in}{2.349815in}}%
\pgfpathlineto{\pgfqpoint{4.435501in}{2.357058in}}%
\pgfpathclose%
\pgfusepath{fill}%
\end{pgfscope}%
\begin{pgfscope}%
\pgfpathrectangle{\pgfqpoint{1.254980in}{0.150000in}}{\pgfqpoint{5.490039in}{5.490039in}}%
\pgfusepath{clip}%
\pgfsetbuttcap%
\pgfsetroundjoin%
\definecolor{currentfill}{rgb}{0.283229,0.120777,0.440584}%
\pgfsetfillcolor{currentfill}%
\pgfsetfillopacity{0.700000}%
\pgfsetlinewidth{0.000000pt}%
\definecolor{currentstroke}{rgb}{0.000000,0.000000,0.000000}%
\pgfsetstrokecolor{currentstroke}%
\pgfsetdash{}{0pt}%
\pgfpathmoveto{\pgfqpoint{5.510530in}{2.491163in}}%
\pgfpathlineto{\pgfqpoint{5.523842in}{2.489388in}}%
\pgfpathlineto{\pgfqpoint{5.537163in}{2.487636in}}%
\pgfpathlineto{\pgfqpoint{5.550490in}{2.485907in}}%
\pgfpathlineto{\pgfqpoint{5.563826in}{2.484202in}}%
\pgfpathlineto{\pgfqpoint{5.556942in}{2.478689in}}%
\pgfpathlineto{\pgfqpoint{5.550054in}{2.473191in}}%
\pgfpathlineto{\pgfqpoint{5.543160in}{2.467704in}}%
\pgfpathlineto{\pgfqpoint{5.536261in}{2.462225in}}%
\pgfpathlineto{\pgfqpoint{5.522908in}{2.463803in}}%
\pgfpathlineto{\pgfqpoint{5.509562in}{2.465404in}}%
\pgfpathlineto{\pgfqpoint{5.496224in}{2.467029in}}%
\pgfpathlineto{\pgfqpoint{5.482893in}{2.468678in}}%
\pgfpathlineto{\pgfqpoint{5.489810in}{2.474279in}}%
\pgfpathlineto{\pgfqpoint{5.496721in}{2.479891in}}%
\pgfpathlineto{\pgfqpoint{5.503628in}{2.485518in}}%
\pgfpathlineto{\pgfqpoint{5.510530in}{2.491163in}}%
\pgfpathclose%
\pgfusepath{fill}%
\end{pgfscope}%
\begin{pgfscope}%
\pgfpathrectangle{\pgfqpoint{1.254980in}{0.150000in}}{\pgfqpoint{5.490039in}{5.490039in}}%
\pgfusepath{clip}%
\pgfsetbuttcap%
\pgfsetroundjoin%
\definecolor{currentfill}{rgb}{0.277941,0.056324,0.381191}%
\pgfsetfillcolor{currentfill}%
\pgfsetfillopacity{0.700000}%
\pgfsetlinewidth{0.000000pt}%
\definecolor{currentstroke}{rgb}{0.000000,0.000000,0.000000}%
\pgfsetstrokecolor{currentstroke}%
\pgfsetdash{}{0pt}%
\pgfpathmoveto{\pgfqpoint{4.650532in}{2.384690in}}%
\pgfpathlineto{\pgfqpoint{4.663618in}{2.382404in}}%
\pgfpathlineto{\pgfqpoint{4.676710in}{2.380143in}}%
\pgfpathlineto{\pgfqpoint{4.689810in}{2.377907in}}%
\pgfpathlineto{\pgfqpoint{4.702916in}{2.375697in}}%
\pgfpathlineto{\pgfqpoint{4.695670in}{2.368752in}}%
\pgfpathlineto{\pgfqpoint{4.688419in}{2.361776in}}%
\pgfpathlineto{\pgfqpoint{4.681161in}{2.354766in}}%
\pgfpathlineto{\pgfqpoint{4.673898in}{2.347721in}}%
\pgfpathlineto{\pgfqpoint{4.660779in}{2.349907in}}%
\pgfpathlineto{\pgfqpoint{4.647667in}{2.352119in}}%
\pgfpathlineto{\pgfqpoint{4.634562in}{2.354356in}}%
\pgfpathlineto{\pgfqpoint{4.621463in}{2.356619in}}%
\pgfpathlineto{\pgfqpoint{4.628739in}{2.363682in}}%
\pgfpathlineto{\pgfqpoint{4.636009in}{2.370715in}}%
\pgfpathlineto{\pgfqpoint{4.643273in}{2.377717in}}%
\pgfpathlineto{\pgfqpoint{4.650532in}{2.384690in}}%
\pgfpathclose%
\pgfusepath{fill}%
\end{pgfscope}%
\begin{pgfscope}%
\pgfpathrectangle{\pgfqpoint{1.254980in}{0.150000in}}{\pgfqpoint{5.490039in}{5.490039in}}%
\pgfusepath{clip}%
\pgfsetbuttcap%
\pgfsetroundjoin%
\definecolor{currentfill}{rgb}{0.272594,0.025563,0.353093}%
\pgfsetfillcolor{currentfill}%
\pgfsetfillopacity{0.700000}%
\pgfsetlinewidth{0.000000pt}%
\definecolor{currentstroke}{rgb}{0.000000,0.000000,0.000000}%
\pgfsetstrokecolor{currentstroke}%
\pgfsetdash{}{0pt}%
\pgfpathmoveto{\pgfqpoint{4.220476in}{2.331368in}}%
\pgfpathlineto{\pgfqpoint{4.233453in}{2.328467in}}%
\pgfpathlineto{\pgfqpoint{4.246437in}{2.325594in}}%
\pgfpathlineto{\pgfqpoint{4.259427in}{2.322748in}}%
\pgfpathlineto{\pgfqpoint{4.272423in}{2.319929in}}%
\pgfpathlineto{\pgfqpoint{4.265015in}{2.312509in}}%
\pgfpathlineto{\pgfqpoint{4.257601in}{2.305066in}}%
\pgfpathlineto{\pgfqpoint{4.250182in}{2.297602in}}%
\pgfpathlineto{\pgfqpoint{4.242757in}{2.290116in}}%
\pgfpathlineto{\pgfqpoint{4.229749in}{2.292962in}}%
\pgfpathlineto{\pgfqpoint{4.216747in}{2.295836in}}%
\pgfpathlineto{\pgfqpoint{4.203752in}{2.298737in}}%
\pgfpathlineto{\pgfqpoint{4.190763in}{2.301665in}}%
\pgfpathlineto{\pgfqpoint{4.198199in}{2.309118in}}%
\pgfpathlineto{\pgfqpoint{4.205630in}{2.316553in}}%
\pgfpathlineto{\pgfqpoint{4.213056in}{2.323970in}}%
\pgfpathlineto{\pgfqpoint{4.220476in}{2.331368in}}%
\pgfpathclose%
\pgfusepath{fill}%
\end{pgfscope}%
\begin{pgfscope}%
\pgfpathrectangle{\pgfqpoint{1.254980in}{0.150000in}}{\pgfqpoint{5.490039in}{5.490039in}}%
\pgfusepath{clip}%
\pgfsetbuttcap%
\pgfsetroundjoin%
\definecolor{currentfill}{rgb}{0.280267,0.073417,0.397163}%
\pgfsetfillcolor{currentfill}%
\pgfsetfillopacity{0.700000}%
\pgfsetlinewidth{0.000000pt}%
\definecolor{currentstroke}{rgb}{0.000000,0.000000,0.000000}%
\pgfsetstrokecolor{currentstroke}%
\pgfsetdash{}{0pt}%
\pgfpathmoveto{\pgfqpoint{4.865569in}{2.412803in}}%
\pgfpathlineto{\pgfqpoint{4.878711in}{2.410732in}}%
\pgfpathlineto{\pgfqpoint{4.891861in}{2.408686in}}%
\pgfpathlineto{\pgfqpoint{4.905018in}{2.406664in}}%
\pgfpathlineto{\pgfqpoint{4.918182in}{2.404666in}}%
\pgfpathlineto{\pgfqpoint{4.911022in}{2.398087in}}%
\pgfpathlineto{\pgfqpoint{4.903857in}{2.391480in}}%
\pgfpathlineto{\pgfqpoint{4.896685in}{2.384842in}}%
\pgfpathlineto{\pgfqpoint{4.889508in}{2.378171in}}%
\pgfpathlineto{\pgfqpoint{4.876330in}{2.380118in}}%
\pgfpathlineto{\pgfqpoint{4.863160in}{2.382091in}}%
\pgfpathlineto{\pgfqpoint{4.849996in}{2.384088in}}%
\pgfpathlineto{\pgfqpoint{4.836840in}{2.386109in}}%
\pgfpathlineto{\pgfqpoint{4.844031in}{2.392825in}}%
\pgfpathlineto{\pgfqpoint{4.851216in}{2.399512in}}%
\pgfpathlineto{\pgfqpoint{4.858395in}{2.406170in}}%
\pgfpathlineto{\pgfqpoint{4.865569in}{2.412803in}}%
\pgfpathclose%
\pgfusepath{fill}%
\end{pgfscope}%
\begin{pgfscope}%
\pgfpathrectangle{\pgfqpoint{1.254980in}{0.150000in}}{\pgfqpoint{5.490039in}{5.490039in}}%
\pgfusepath{clip}%
\pgfsetbuttcap%
\pgfsetroundjoin%
\definecolor{currentfill}{rgb}{0.282910,0.105393,0.426902}%
\pgfsetfillcolor{currentfill}%
\pgfsetfillopacity{0.700000}%
\pgfsetlinewidth{0.000000pt}%
\definecolor{currentstroke}{rgb}{0.000000,0.000000,0.000000}%
\pgfsetstrokecolor{currentstroke}%
\pgfsetdash{}{0pt}%
\pgfpathmoveto{\pgfqpoint{5.295592in}{2.466546in}}%
\pgfpathlineto{\pgfqpoint{5.308849in}{2.464729in}}%
\pgfpathlineto{\pgfqpoint{5.322114in}{2.462935in}}%
\pgfpathlineto{\pgfqpoint{5.335386in}{2.461166in}}%
\pgfpathlineto{\pgfqpoint{5.348665in}{2.459420in}}%
\pgfpathlineto{\pgfqpoint{5.341688in}{2.453602in}}%
\pgfpathlineto{\pgfqpoint{5.334705in}{2.447780in}}%
\pgfpathlineto{\pgfqpoint{5.327716in}{2.441949in}}%
\pgfpathlineto{\pgfqpoint{5.320722in}{2.436106in}}%
\pgfpathlineto{\pgfqpoint{5.307426in}{2.437751in}}%
\pgfpathlineto{\pgfqpoint{5.294137in}{2.439419in}}%
\pgfpathlineto{\pgfqpoint{5.280857in}{2.441111in}}%
\pgfpathlineto{\pgfqpoint{5.267583in}{2.442827in}}%
\pgfpathlineto{\pgfqpoint{5.274594in}{2.448766in}}%
\pgfpathlineto{\pgfqpoint{5.281599in}{2.454697in}}%
\pgfpathlineto{\pgfqpoint{5.288598in}{2.460622in}}%
\pgfpathlineto{\pgfqpoint{5.295592in}{2.466546in}}%
\pgfpathclose%
\pgfusepath{fill}%
\end{pgfscope}%
\begin{pgfscope}%
\pgfpathrectangle{\pgfqpoint{1.254980in}{0.150000in}}{\pgfqpoint{5.490039in}{5.490039in}}%
\pgfusepath{clip}%
\pgfsetbuttcap%
\pgfsetroundjoin%
\definecolor{currentfill}{rgb}{0.282327,0.094955,0.417331}%
\pgfsetfillcolor{currentfill}%
\pgfsetfillopacity{0.700000}%
\pgfsetlinewidth{0.000000pt}%
\definecolor{currentstroke}{rgb}{0.000000,0.000000,0.000000}%
\pgfsetstrokecolor{currentstroke}%
\pgfsetdash{}{0pt}%
\pgfpathmoveto{\pgfqpoint{5.080597in}{2.440321in}}%
\pgfpathlineto{\pgfqpoint{5.093797in}{2.438406in}}%
\pgfpathlineto{\pgfqpoint{5.107004in}{2.436514in}}%
\pgfpathlineto{\pgfqpoint{5.120219in}{2.434647in}}%
\pgfpathlineto{\pgfqpoint{5.133441in}{2.432804in}}%
\pgfpathlineto{\pgfqpoint{5.126371in}{2.426616in}}%
\pgfpathlineto{\pgfqpoint{5.119295in}{2.420409in}}%
\pgfpathlineto{\pgfqpoint{5.112213in}{2.414178in}}%
\pgfpathlineto{\pgfqpoint{5.105126in}{2.407923in}}%
\pgfpathlineto{\pgfqpoint{5.091889in}{2.409691in}}%
\pgfpathlineto{\pgfqpoint{5.078659in}{2.411483in}}%
\pgfpathlineto{\pgfqpoint{5.065437in}{2.413298in}}%
\pgfpathlineto{\pgfqpoint{5.052222in}{2.415138in}}%
\pgfpathlineto{\pgfqpoint{5.059324in}{2.421464in}}%
\pgfpathlineto{\pgfqpoint{5.066421in}{2.427768in}}%
\pgfpathlineto{\pgfqpoint{5.073512in}{2.434053in}}%
\pgfpathlineto{\pgfqpoint{5.080597in}{2.440321in}}%
\pgfpathclose%
\pgfusepath{fill}%
\end{pgfscope}%
\begin{pgfscope}%
\pgfpathrectangle{\pgfqpoint{1.254980in}{0.150000in}}{\pgfqpoint{5.490039in}{5.490039in}}%
\pgfusepath{clip}%
\pgfsetbuttcap%
\pgfsetroundjoin%
\definecolor{currentfill}{rgb}{0.267004,0.004874,0.329415}%
\pgfsetfillcolor{currentfill}%
\pgfsetfillopacity{0.700000}%
\pgfsetlinewidth{0.000000pt}%
\definecolor{currentstroke}{rgb}{0.000000,0.000000,0.000000}%
\pgfsetstrokecolor{currentstroke}%
\pgfsetdash{}{0pt}%
\pgfpathmoveto{\pgfqpoint{3.656955in}{2.296909in}}%
\pgfpathlineto{\pgfqpoint{3.669810in}{2.292804in}}%
\pgfpathlineto{\pgfqpoint{3.682669in}{2.288730in}}%
\pgfpathlineto{\pgfqpoint{3.695534in}{2.284688in}}%
\pgfpathlineto{\pgfqpoint{3.708405in}{2.280677in}}%
\pgfpathlineto{\pgfqpoint{3.700788in}{2.273643in}}%
\pgfpathlineto{\pgfqpoint{3.693165in}{2.266633in}}%
\pgfpathlineto{\pgfqpoint{3.685537in}{2.259649in}}%
\pgfpathlineto{\pgfqpoint{3.677902in}{2.252693in}}%
\pgfpathlineto{\pgfqpoint{3.665018in}{2.256795in}}%
\pgfpathlineto{\pgfqpoint{3.652140in}{2.260928in}}%
\pgfpathlineto{\pgfqpoint{3.639267in}{2.265093in}}%
\pgfpathlineto{\pgfqpoint{3.626399in}{2.269290in}}%
\pgfpathlineto{\pgfqpoint{3.634047in}{2.276150in}}%
\pgfpathlineto{\pgfqpoint{3.641689in}{2.283042in}}%
\pgfpathlineto{\pgfqpoint{3.649325in}{2.289962in}}%
\pgfpathlineto{\pgfqpoint{3.656955in}{2.296909in}}%
\pgfpathclose%
\pgfusepath{fill}%
\end{pgfscope}%
\begin{pgfscope}%
\pgfpathrectangle{\pgfqpoint{1.254980in}{0.150000in}}{\pgfqpoint{5.490039in}{5.490039in}}%
\pgfusepath{clip}%
\pgfsetbuttcap%
\pgfsetroundjoin%
\definecolor{currentfill}{rgb}{0.268510,0.009605,0.335427}%
\pgfsetfillcolor{currentfill}%
\pgfsetfillopacity{0.700000}%
\pgfsetlinewidth{0.000000pt}%
\definecolor{currentstroke}{rgb}{0.000000,0.000000,0.000000}%
\pgfsetstrokecolor{currentstroke}%
\pgfsetdash{}{0pt}%
\pgfpathmoveto{\pgfqpoint{4.005428in}{2.309466in}}%
\pgfpathlineto{\pgfqpoint{4.018356in}{2.306161in}}%
\pgfpathlineto{\pgfqpoint{4.031290in}{2.302885in}}%
\pgfpathlineto{\pgfqpoint{4.044230in}{2.299638in}}%
\pgfpathlineto{\pgfqpoint{4.057176in}{2.296419in}}%
\pgfpathlineto{\pgfqpoint{4.049689in}{2.288977in}}%
\pgfpathlineto{\pgfqpoint{4.042196in}{2.281527in}}%
\pgfpathlineto{\pgfqpoint{4.034698in}{2.274070in}}%
\pgfpathlineto{\pgfqpoint{4.027194in}{2.266606in}}%
\pgfpathlineto{\pgfqpoint{4.014236in}{2.269878in}}%
\pgfpathlineto{\pgfqpoint{4.001284in}{2.273178in}}%
\pgfpathlineto{\pgfqpoint{3.988338in}{2.276507in}}%
\pgfpathlineto{\pgfqpoint{3.975398in}{2.279865in}}%
\pgfpathlineto{\pgfqpoint{3.982913in}{2.287271in}}%
\pgfpathlineto{\pgfqpoint{3.990424in}{2.294674in}}%
\pgfpathlineto{\pgfqpoint{3.997929in}{2.302073in}}%
\pgfpathlineto{\pgfqpoint{4.005428in}{2.309466in}}%
\pgfpathclose%
\pgfusepath{fill}%
\end{pgfscope}%
\begin{pgfscope}%
\pgfpathrectangle{\pgfqpoint{1.254980in}{0.150000in}}{\pgfqpoint{5.490039in}{5.490039in}}%
\pgfusepath{clip}%
\pgfsetbuttcap%
\pgfsetroundjoin%
\definecolor{currentfill}{rgb}{0.277941,0.056324,0.381191}%
\pgfsetfillcolor{currentfill}%
\pgfsetfillopacity{0.700000}%
\pgfsetlinewidth{0.000000pt}%
\definecolor{currentstroke}{rgb}{0.000000,0.000000,0.000000}%
\pgfsetstrokecolor{currentstroke}%
\pgfsetdash{}{0pt}%
\pgfpathmoveto{\pgfqpoint{2.989788in}{2.381924in}}%
\pgfpathlineto{\pgfqpoint{3.002545in}{2.375765in}}%
\pgfpathlineto{\pgfqpoint{3.015305in}{2.369650in}}%
\pgfpathlineto{\pgfqpoint{3.028069in}{2.363577in}}%
\pgfpathlineto{\pgfqpoint{3.040836in}{2.357546in}}%
\pgfpathlineto{\pgfqpoint{3.032922in}{2.352959in}}%
\pgfpathlineto{\pgfqpoint{3.024999in}{2.348484in}}%
\pgfpathlineto{\pgfqpoint{3.017066in}{2.344124in}}%
\pgfpathlineto{\pgfqpoint{3.009124in}{2.339883in}}%
\pgfpathlineto{\pgfqpoint{2.996338in}{2.346070in}}%
\pgfpathlineto{\pgfqpoint{2.983555in}{2.352299in}}%
\pgfpathlineto{\pgfqpoint{2.970775in}{2.358570in}}%
\pgfpathlineto{\pgfqpoint{2.957999in}{2.364884in}}%
\pgfpathlineto{\pgfqpoint{2.965960in}{2.368965in}}%
\pgfpathlineto{\pgfqpoint{2.973912in}{2.373167in}}%
\pgfpathlineto{\pgfqpoint{2.981855in}{2.377488in}}%
\pgfpathlineto{\pgfqpoint{2.989788in}{2.381924in}}%
\pgfpathclose%
\pgfusepath{fill}%
\end{pgfscope}%
\begin{pgfscope}%
\pgfpathrectangle{\pgfqpoint{1.254980in}{0.150000in}}{\pgfqpoint{5.490039in}{5.490039in}}%
\pgfusepath{clip}%
\pgfsetbuttcap%
\pgfsetroundjoin%
\definecolor{currentfill}{rgb}{0.267004,0.004874,0.329415}%
\pgfsetfillcolor{currentfill}%
\pgfsetfillopacity{0.700000}%
\pgfsetlinewidth{0.000000pt}%
\definecolor{currentstroke}{rgb}{0.000000,0.000000,0.000000}%
\pgfsetstrokecolor{currentstroke}%
\pgfsetdash{}{0pt}%
\pgfpathmoveto{\pgfqpoint{3.790297in}{2.293594in}}%
\pgfpathlineto{\pgfqpoint{3.803181in}{2.289816in}}%
\pgfpathlineto{\pgfqpoint{3.816071in}{2.286068in}}%
\pgfpathlineto{\pgfqpoint{3.828966in}{2.282351in}}%
\pgfpathlineto{\pgfqpoint{3.841867in}{2.278663in}}%
\pgfpathlineto{\pgfqpoint{3.834299in}{2.271404in}}%
\pgfpathlineto{\pgfqpoint{3.826725in}{2.264156in}}%
\pgfpathlineto{\pgfqpoint{3.819146in}{2.256920in}}%
\pgfpathlineto{\pgfqpoint{3.811561in}{2.249698in}}%
\pgfpathlineto{\pgfqpoint{3.798647in}{2.253464in}}%
\pgfpathlineto{\pgfqpoint{3.785739in}{2.257260in}}%
\pgfpathlineto{\pgfqpoint{3.772836in}{2.261087in}}%
\pgfpathlineto{\pgfqpoint{3.759939in}{2.264943in}}%
\pgfpathlineto{\pgfqpoint{3.767537in}{2.272082in}}%
\pgfpathlineto{\pgfqpoint{3.775129in}{2.279238in}}%
\pgfpathlineto{\pgfqpoint{3.782716in}{2.286409in}}%
\pgfpathlineto{\pgfqpoint{3.790297in}{2.293594in}}%
\pgfpathclose%
\pgfusepath{fill}%
\end{pgfscope}%
\begin{pgfscope}%
\pgfpathrectangle{\pgfqpoint{1.254980in}{0.150000in}}{\pgfqpoint{5.490039in}{5.490039in}}%
\pgfusepath{clip}%
\pgfsetbuttcap%
\pgfsetroundjoin%
\definecolor{currentfill}{rgb}{0.282623,0.140926,0.457517}%
\pgfsetfillcolor{currentfill}%
\pgfsetfillopacity{0.700000}%
\pgfsetlinewidth{0.000000pt}%
\definecolor{currentstroke}{rgb}{0.000000,0.000000,0.000000}%
\pgfsetstrokecolor{currentstroke}%
\pgfsetdash{}{0pt}%
\pgfpathmoveto{\pgfqpoint{5.859584in}{2.521718in}}%
\pgfpathlineto{\pgfqpoint{5.872991in}{2.519948in}}%
\pgfpathlineto{\pgfqpoint{5.886407in}{2.518200in}}%
\pgfpathlineto{\pgfqpoint{5.899830in}{2.516476in}}%
\pgfpathlineto{\pgfqpoint{5.913262in}{2.514774in}}%
\pgfpathlineto{\pgfqpoint{5.906526in}{2.509575in}}%
\pgfpathlineto{\pgfqpoint{5.899788in}{2.504432in}}%
\pgfpathlineto{\pgfqpoint{5.893046in}{2.499338in}}%
\pgfpathlineto{\pgfqpoint{5.886300in}{2.494291in}}%
\pgfpathlineto{\pgfqpoint{5.872848in}{2.495827in}}%
\pgfpathlineto{\pgfqpoint{5.859403in}{2.497386in}}%
\pgfpathlineto{\pgfqpoint{5.845967in}{2.498969in}}%
\pgfpathlineto{\pgfqpoint{5.832538in}{2.500574in}}%
\pgfpathlineto{\pgfqpoint{5.839305in}{2.505782in}}%
\pgfpathlineto{\pgfqpoint{5.846068in}{2.511039in}}%
\pgfpathlineto{\pgfqpoint{5.852827in}{2.516349in}}%
\pgfpathlineto{\pgfqpoint{5.859584in}{2.521718in}}%
\pgfpathclose%
\pgfusepath{fill}%
\end{pgfscope}%
\begin{pgfscope}%
\pgfpathrectangle{\pgfqpoint{1.254980in}{0.150000in}}{\pgfqpoint{5.490039in}{5.490039in}}%
\pgfusepath{clip}%
\pgfsetbuttcap%
\pgfsetroundjoin%
\definecolor{currentfill}{rgb}{0.273809,0.031497,0.358853}%
\pgfsetfillcolor{currentfill}%
\pgfsetfillopacity{0.700000}%
\pgfsetlinewidth{0.000000pt}%
\definecolor{currentstroke}{rgb}{0.000000,0.000000,0.000000}%
\pgfsetstrokecolor{currentstroke}%
\pgfsetdash{}{0pt}%
\pgfpathmoveto{\pgfqpoint{4.354002in}{2.338444in}}%
\pgfpathlineto{\pgfqpoint{4.367019in}{2.335774in}}%
\pgfpathlineto{\pgfqpoint{4.380041in}{2.333131in}}%
\pgfpathlineto{\pgfqpoint{4.393071in}{2.330514in}}%
\pgfpathlineto{\pgfqpoint{4.406106in}{2.327924in}}%
\pgfpathlineto{\pgfqpoint{4.398744in}{2.320572in}}%
\pgfpathlineto{\pgfqpoint{4.391376in}{2.313191in}}%
\pgfpathlineto{\pgfqpoint{4.384002in}{2.305782in}}%
\pgfpathlineto{\pgfqpoint{4.376623in}{2.298345in}}%
\pgfpathlineto{\pgfqpoint{4.363575in}{2.300950in}}%
\pgfpathlineto{\pgfqpoint{4.350534in}{2.303581in}}%
\pgfpathlineto{\pgfqpoint{4.337499in}{2.306239in}}%
\pgfpathlineto{\pgfqpoint{4.324471in}{2.308923in}}%
\pgfpathlineto{\pgfqpoint{4.331862in}{2.316341in}}%
\pgfpathlineto{\pgfqpoint{4.339248in}{2.323734in}}%
\pgfpathlineto{\pgfqpoint{4.346628in}{2.331101in}}%
\pgfpathlineto{\pgfqpoint{4.354002in}{2.338444in}}%
\pgfpathclose%
\pgfusepath{fill}%
\end{pgfscope}%
\begin{pgfscope}%
\pgfpathrectangle{\pgfqpoint{1.254980in}{0.150000in}}{\pgfqpoint{5.490039in}{5.490039in}}%
\pgfusepath{clip}%
\pgfsetbuttcap%
\pgfsetroundjoin%
\definecolor{currentfill}{rgb}{0.283072,0.130895,0.449241}%
\pgfsetfillcolor{currentfill}%
\pgfsetfillopacity{0.700000}%
\pgfsetlinewidth{0.000000pt}%
\definecolor{currentstroke}{rgb}{0.000000,0.000000,0.000000}%
\pgfsetstrokecolor{currentstroke}%
\pgfsetdash{}{0pt}%
\pgfpathmoveto{\pgfqpoint{5.644658in}{2.499348in}}%
\pgfpathlineto{\pgfqpoint{5.658013in}{2.497619in}}%
\pgfpathlineto{\pgfqpoint{5.671376in}{2.495913in}}%
\pgfpathlineto{\pgfqpoint{5.684746in}{2.494231in}}%
\pgfpathlineto{\pgfqpoint{5.698124in}{2.492572in}}%
\pgfpathlineto{\pgfqpoint{5.691297in}{2.487243in}}%
\pgfpathlineto{\pgfqpoint{5.684465in}{2.481940in}}%
\pgfpathlineto{\pgfqpoint{5.677629in}{2.476660in}}%
\pgfpathlineto{\pgfqpoint{5.670787in}{2.471397in}}%
\pgfpathlineto{\pgfqpoint{5.657390in}{2.472916in}}%
\pgfpathlineto{\pgfqpoint{5.644000in}{2.474459in}}%
\pgfpathlineto{\pgfqpoint{5.630619in}{2.476025in}}%
\pgfpathlineto{\pgfqpoint{5.617245in}{2.477614in}}%
\pgfpathlineto{\pgfqpoint{5.624105in}{2.483012in}}%
\pgfpathlineto{\pgfqpoint{5.630960in}{2.488431in}}%
\pgfpathlineto{\pgfqpoint{5.637811in}{2.493875in}}%
\pgfpathlineto{\pgfqpoint{5.644658in}{2.499348in}}%
\pgfpathclose%
\pgfusepath{fill}%
\end{pgfscope}%
\begin{pgfscope}%
\pgfpathrectangle{\pgfqpoint{1.254980in}{0.150000in}}{\pgfqpoint{5.490039in}{5.490039in}}%
\pgfusepath{clip}%
\pgfsetbuttcap%
\pgfsetroundjoin%
\definecolor{currentfill}{rgb}{0.277018,0.050344,0.375715}%
\pgfsetfillcolor{currentfill}%
\pgfsetfillopacity{0.700000}%
\pgfsetlinewidth{0.000000pt}%
\definecolor{currentstroke}{rgb}{0.000000,0.000000,0.000000}%
\pgfsetstrokecolor{currentstroke}%
\pgfsetdash{}{0pt}%
\pgfpathmoveto{\pgfqpoint{4.569138in}{2.365924in}}%
\pgfpathlineto{\pgfqpoint{4.582209in}{2.363559in}}%
\pgfpathlineto{\pgfqpoint{4.595287in}{2.361220in}}%
\pgfpathlineto{\pgfqpoint{4.608372in}{2.358907in}}%
\pgfpathlineto{\pgfqpoint{4.621463in}{2.356619in}}%
\pgfpathlineto{\pgfqpoint{4.614182in}{2.349523in}}%
\pgfpathlineto{\pgfqpoint{4.606895in}{2.342393in}}%
\pgfpathlineto{\pgfqpoint{4.599602in}{2.335230in}}%
\pgfpathlineto{\pgfqpoint{4.592304in}{2.328032in}}%
\pgfpathlineto{\pgfqpoint{4.579200in}{2.330309in}}%
\pgfpathlineto{\pgfqpoint{4.566103in}{2.332611in}}%
\pgfpathlineto{\pgfqpoint{4.553013in}{2.334939in}}%
\pgfpathlineto{\pgfqpoint{4.539929in}{2.337293in}}%
\pgfpathlineto{\pgfqpoint{4.547240in}{2.344498in}}%
\pgfpathlineto{\pgfqpoint{4.554545in}{2.351671in}}%
\pgfpathlineto{\pgfqpoint{4.561844in}{2.358813in}}%
\pgfpathlineto{\pgfqpoint{4.569138in}{2.365924in}}%
\pgfpathclose%
\pgfusepath{fill}%
\end{pgfscope}%
\begin{pgfscope}%
\pgfpathrectangle{\pgfqpoint{1.254980in}{0.150000in}}{\pgfqpoint{5.490039in}{5.490039in}}%
\pgfusepath{clip}%
\pgfsetbuttcap%
\pgfsetroundjoin%
\definecolor{currentfill}{rgb}{0.271305,0.019942,0.347269}%
\pgfsetfillcolor{currentfill}%
\pgfsetfillopacity{0.700000}%
\pgfsetlinewidth{0.000000pt}%
\definecolor{currentstroke}{rgb}{0.000000,0.000000,0.000000}%
\pgfsetstrokecolor{currentstroke}%
\pgfsetdash{}{0pt}%
\pgfpathmoveto{\pgfqpoint{4.138868in}{2.313652in}}%
\pgfpathlineto{\pgfqpoint{4.151833in}{2.310614in}}%
\pgfpathlineto{\pgfqpoint{4.164803in}{2.307603in}}%
\pgfpathlineto{\pgfqpoint{4.177780in}{2.304620in}}%
\pgfpathlineto{\pgfqpoint{4.190763in}{2.301665in}}%
\pgfpathlineto{\pgfqpoint{4.183321in}{2.294193in}}%
\pgfpathlineto{\pgfqpoint{4.175874in}{2.286704in}}%
\pgfpathlineto{\pgfqpoint{4.168421in}{2.279198in}}%
\pgfpathlineto{\pgfqpoint{4.160963in}{2.271676in}}%
\pgfpathlineto{\pgfqpoint{4.147968in}{2.274672in}}%
\pgfpathlineto{\pgfqpoint{4.134979in}{2.277695in}}%
\pgfpathlineto{\pgfqpoint{4.121997in}{2.280746in}}%
\pgfpathlineto{\pgfqpoint{4.109021in}{2.283824in}}%
\pgfpathlineto{\pgfqpoint{4.116491in}{2.291302in}}%
\pgfpathlineto{\pgfqpoint{4.123955in}{2.298766in}}%
\pgfpathlineto{\pgfqpoint{4.131414in}{2.306216in}}%
\pgfpathlineto{\pgfqpoint{4.138868in}{2.313652in}}%
\pgfpathclose%
\pgfusepath{fill}%
\end{pgfscope}%
\begin{pgfscope}%
\pgfpathrectangle{\pgfqpoint{1.254980in}{0.150000in}}{\pgfqpoint{5.490039in}{5.490039in}}%
\pgfusepath{clip}%
\pgfsetbuttcap%
\pgfsetroundjoin%
\definecolor{currentfill}{rgb}{0.279566,0.067836,0.391917}%
\pgfsetfillcolor{currentfill}%
\pgfsetfillopacity{0.700000}%
\pgfsetlinewidth{0.000000pt}%
\definecolor{currentstroke}{rgb}{0.000000,0.000000,0.000000}%
\pgfsetstrokecolor{currentstroke}%
\pgfsetdash{}{0pt}%
\pgfpathmoveto{\pgfqpoint{4.784285in}{2.394445in}}%
\pgfpathlineto{\pgfqpoint{4.797413in}{2.392324in}}%
\pgfpathlineto{\pgfqpoint{4.810549in}{2.390227in}}%
\pgfpathlineto{\pgfqpoint{4.823691in}{2.388156in}}%
\pgfpathlineto{\pgfqpoint{4.836840in}{2.386109in}}%
\pgfpathlineto{\pgfqpoint{4.829643in}{2.379363in}}%
\pgfpathlineto{\pgfqpoint{4.822441in}{2.372583in}}%
\pgfpathlineto{\pgfqpoint{4.815232in}{2.365769in}}%
\pgfpathlineto{\pgfqpoint{4.808018in}{2.358919in}}%
\pgfpathlineto{\pgfqpoint{4.794856in}{2.360929in}}%
\pgfpathlineto{\pgfqpoint{4.781700in}{2.362964in}}%
\pgfpathlineto{\pgfqpoint{4.768552in}{2.365023in}}%
\pgfpathlineto{\pgfqpoint{4.755411in}{2.367108in}}%
\pgfpathlineto{\pgfqpoint{4.762638in}{2.373990in}}%
\pgfpathlineto{\pgfqpoint{4.769860in}{2.380839in}}%
\pgfpathlineto{\pgfqpoint{4.777075in}{2.387657in}}%
\pgfpathlineto{\pgfqpoint{4.784285in}{2.394445in}}%
\pgfpathclose%
\pgfusepath{fill}%
\end{pgfscope}%
\begin{pgfscope}%
\pgfpathrectangle{\pgfqpoint{1.254980in}{0.150000in}}{\pgfqpoint{5.490039in}{5.490039in}}%
\pgfusepath{clip}%
\pgfsetbuttcap%
\pgfsetroundjoin%
\definecolor{currentfill}{rgb}{0.283197,0.115680,0.436115}%
\pgfsetfillcolor{currentfill}%
\pgfsetfillopacity{0.700000}%
\pgfsetlinewidth{0.000000pt}%
\definecolor{currentstroke}{rgb}{0.000000,0.000000,0.000000}%
\pgfsetstrokecolor{currentstroke}%
\pgfsetdash{}{0pt}%
\pgfpathmoveto{\pgfqpoint{5.429647in}{2.475505in}}%
\pgfpathlineto{\pgfqpoint{5.442947in}{2.473763in}}%
\pgfpathlineto{\pgfqpoint{5.456255in}{2.472045in}}%
\pgfpathlineto{\pgfqpoint{5.469570in}{2.470349in}}%
\pgfpathlineto{\pgfqpoint{5.482893in}{2.468678in}}%
\pgfpathlineto{\pgfqpoint{5.475971in}{2.463083in}}%
\pgfpathlineto{\pgfqpoint{5.469044in}{2.457491in}}%
\pgfpathlineto{\pgfqpoint{5.462111in}{2.451898in}}%
\pgfpathlineto{\pgfqpoint{5.455173in}{2.446301in}}%
\pgfpathlineto{\pgfqpoint{5.441833in}{2.447858in}}%
\pgfpathlineto{\pgfqpoint{5.428500in}{2.449439in}}%
\pgfpathlineto{\pgfqpoint{5.415176in}{2.451044in}}%
\pgfpathlineto{\pgfqpoint{5.401858in}{2.452672in}}%
\pgfpathlineto{\pgfqpoint{5.408814in}{2.458379in}}%
\pgfpathlineto{\pgfqpoint{5.415763in}{2.464084in}}%
\pgfpathlineto{\pgfqpoint{5.422708in}{2.469792in}}%
\pgfpathlineto{\pgfqpoint{5.429647in}{2.475505in}}%
\pgfpathclose%
\pgfusepath{fill}%
\end{pgfscope}%
\begin{pgfscope}%
\pgfpathrectangle{\pgfqpoint{1.254980in}{0.150000in}}{\pgfqpoint{5.490039in}{5.490039in}}%
\pgfusepath{clip}%
\pgfsetbuttcap%
\pgfsetroundjoin%
\definecolor{currentfill}{rgb}{0.281924,0.089666,0.412415}%
\pgfsetfillcolor{currentfill}%
\pgfsetfillopacity{0.700000}%
\pgfsetlinewidth{0.000000pt}%
\definecolor{currentstroke}{rgb}{0.000000,0.000000,0.000000}%
\pgfsetstrokecolor{currentstroke}%
\pgfsetdash{}{0pt}%
\pgfpathmoveto{\pgfqpoint{4.999435in}{2.422741in}}%
\pgfpathlineto{\pgfqpoint{5.012621in}{2.420804in}}%
\pgfpathlineto{\pgfqpoint{5.025814in}{2.418891in}}%
\pgfpathlineto{\pgfqpoint{5.039015in}{2.417003in}}%
\pgfpathlineto{\pgfqpoint{5.052222in}{2.415138in}}%
\pgfpathlineto{\pgfqpoint{5.045114in}{2.408788in}}%
\pgfpathlineto{\pgfqpoint{5.038000in}{2.402412in}}%
\pgfpathlineto{\pgfqpoint{5.030880in}{2.396006in}}%
\pgfpathlineto{\pgfqpoint{5.023754in}{2.389570in}}%
\pgfpathlineto{\pgfqpoint{5.010532in}{2.391372in}}%
\pgfpathlineto{\pgfqpoint{4.997317in}{2.393198in}}%
\pgfpathlineto{\pgfqpoint{4.984110in}{2.395048in}}%
\pgfpathlineto{\pgfqpoint{4.970910in}{2.396923in}}%
\pgfpathlineto{\pgfqpoint{4.978050in}{2.403417in}}%
\pgfpathlineto{\pgfqpoint{4.985184in}{2.409883in}}%
\pgfpathlineto{\pgfqpoint{4.992313in}{2.416324in}}%
\pgfpathlineto{\pgfqpoint{4.999435in}{2.422741in}}%
\pgfpathclose%
\pgfusepath{fill}%
\end{pgfscope}%
\begin{pgfscope}%
\pgfpathrectangle{\pgfqpoint{1.254980in}{0.150000in}}{\pgfqpoint{5.490039in}{5.490039in}}%
\pgfusepath{clip}%
\pgfsetbuttcap%
\pgfsetroundjoin%
\definecolor{currentfill}{rgb}{0.282910,0.105393,0.426902}%
\pgfsetfillcolor{currentfill}%
\pgfsetfillopacity{0.700000}%
\pgfsetlinewidth{0.000000pt}%
\definecolor{currentstroke}{rgb}{0.000000,0.000000,0.000000}%
\pgfsetstrokecolor{currentstroke}%
\pgfsetdash{}{0pt}%
\pgfpathmoveto{\pgfqpoint{5.214565in}{2.449929in}}%
\pgfpathlineto{\pgfqpoint{5.227809in}{2.448118in}}%
\pgfpathlineto{\pgfqpoint{5.241059in}{2.446331in}}%
\pgfpathlineto{\pgfqpoint{5.254318in}{2.444567in}}%
\pgfpathlineto{\pgfqpoint{5.267583in}{2.442827in}}%
\pgfpathlineto{\pgfqpoint{5.260567in}{2.436877in}}%
\pgfpathlineto{\pgfqpoint{5.253545in}{2.430911in}}%
\pgfpathlineto{\pgfqpoint{5.246518in}{2.424928in}}%
\pgfpathlineto{\pgfqpoint{5.239484in}{2.418925in}}%
\pgfpathlineto{\pgfqpoint{5.226202in}{2.420576in}}%
\pgfpathlineto{\pgfqpoint{5.212928in}{2.422251in}}%
\pgfpathlineto{\pgfqpoint{5.199662in}{2.423950in}}%
\pgfpathlineto{\pgfqpoint{5.186403in}{2.425673in}}%
\pgfpathlineto{\pgfqpoint{5.193452in}{2.431760in}}%
\pgfpathlineto{\pgfqpoint{5.200495in}{2.437830in}}%
\pgfpathlineto{\pgfqpoint{5.207533in}{2.443885in}}%
\pgfpathlineto{\pgfqpoint{5.214565in}{2.449929in}}%
\pgfpathclose%
\pgfusepath{fill}%
\end{pgfscope}%
\begin{pgfscope}%
\pgfpathrectangle{\pgfqpoint{1.254980in}{0.150000in}}{\pgfqpoint{5.490039in}{5.490039in}}%
\pgfusepath{clip}%
\pgfsetbuttcap%
\pgfsetroundjoin%
\definecolor{currentfill}{rgb}{0.281446,0.084320,0.407414}%
\pgfsetfillcolor{currentfill}%
\pgfsetfillopacity{0.700000}%
\pgfsetlinewidth{0.000000pt}%
\definecolor{currentstroke}{rgb}{0.000000,0.000000,0.000000}%
\pgfsetstrokecolor{currentstroke}%
\pgfsetdash{}{0pt}%
\pgfpathmoveto{\pgfqpoint{2.855906in}{2.416989in}}%
\pgfpathlineto{\pgfqpoint{2.868657in}{2.410317in}}%
\pgfpathlineto{\pgfqpoint{2.881411in}{2.403692in}}%
\pgfpathlineto{\pgfqpoint{2.894168in}{2.397112in}}%
\pgfpathlineto{\pgfqpoint{2.906927in}{2.390577in}}%
\pgfpathlineto{\pgfqpoint{2.898936in}{2.386787in}}%
\pgfpathlineto{\pgfqpoint{2.890934in}{2.383130in}}%
\pgfpathlineto{\pgfqpoint{2.882923in}{2.379611in}}%
\pgfpathlineto{\pgfqpoint{2.874900in}{2.376233in}}%
\pgfpathlineto{\pgfqpoint{2.862120in}{2.382936in}}%
\pgfpathlineto{\pgfqpoint{2.849342in}{2.389685in}}%
\pgfpathlineto{\pgfqpoint{2.836567in}{2.396480in}}%
\pgfpathlineto{\pgfqpoint{2.823795in}{2.403321in}}%
\pgfpathlineto{\pgfqpoint{2.831839in}{2.406524in}}%
\pgfpathlineto{\pgfqpoint{2.839872in}{2.409873in}}%
\pgfpathlineto{\pgfqpoint{2.847894in}{2.413363in}}%
\pgfpathlineto{\pgfqpoint{2.855906in}{2.416989in}}%
\pgfpathclose%
\pgfusepath{fill}%
\end{pgfscope}%
\begin{pgfscope}%
\pgfpathrectangle{\pgfqpoint{1.254980in}{0.150000in}}{\pgfqpoint{5.490039in}{5.490039in}}%
\pgfusepath{clip}%
\pgfsetbuttcap%
\pgfsetroundjoin%
\definecolor{currentfill}{rgb}{0.269944,0.014625,0.341379}%
\pgfsetfillcolor{currentfill}%
\pgfsetfillopacity{0.700000}%
\pgfsetlinewidth{0.000000pt}%
\definecolor{currentstroke}{rgb}{0.000000,0.000000,0.000000}%
\pgfsetstrokecolor{currentstroke}%
\pgfsetdash{}{0pt}%
\pgfpathmoveto{\pgfqpoint{3.308090in}{2.311118in}}%
\pgfpathlineto{\pgfqpoint{3.320891in}{2.306022in}}%
\pgfpathlineto{\pgfqpoint{3.333697in}{2.300961in}}%
\pgfpathlineto{\pgfqpoint{3.346507in}{2.295937in}}%
\pgfpathlineto{\pgfqpoint{3.359321in}{2.290949in}}%
\pgfpathlineto{\pgfqpoint{3.351556in}{2.284971in}}%
\pgfpathlineto{\pgfqpoint{3.343784in}{2.279061in}}%
\pgfpathlineto{\pgfqpoint{3.336004in}{2.273224in}}%
\pgfpathlineto{\pgfqpoint{3.328217in}{2.267461in}}%
\pgfpathlineto{\pgfqpoint{3.315387in}{2.272579in}}%
\pgfpathlineto{\pgfqpoint{3.302561in}{2.277733in}}%
\pgfpathlineto{\pgfqpoint{3.289739in}{2.282923in}}%
\pgfpathlineto{\pgfqpoint{3.276922in}{2.288149in}}%
\pgfpathlineto{\pgfqpoint{3.284725in}{2.293777in}}%
\pgfpathlineto{\pgfqpoint{3.292521in}{2.299483in}}%
\pgfpathlineto{\pgfqpoint{3.300309in}{2.305265in}}%
\pgfpathlineto{\pgfqpoint{3.308090in}{2.311118in}}%
\pgfpathclose%
\pgfusepath{fill}%
\end{pgfscope}%
\begin{pgfscope}%
\pgfpathrectangle{\pgfqpoint{1.254980in}{0.150000in}}{\pgfqpoint{5.490039in}{5.490039in}}%
\pgfusepath{clip}%
\pgfsetbuttcap%
\pgfsetroundjoin%
\definecolor{currentfill}{rgb}{0.268510,0.009605,0.335427}%
\pgfsetfillcolor{currentfill}%
\pgfsetfillopacity{0.700000}%
\pgfsetlinewidth{0.000000pt}%
\definecolor{currentstroke}{rgb}{0.000000,0.000000,0.000000}%
\pgfsetstrokecolor{currentstroke}%
\pgfsetdash{}{0pt}%
\pgfpathmoveto{\pgfqpoint{3.441555in}{2.296356in}}%
\pgfpathlineto{\pgfqpoint{3.454377in}{2.291659in}}%
\pgfpathlineto{\pgfqpoint{3.467205in}{2.286996in}}%
\pgfpathlineto{\pgfqpoint{3.480037in}{2.282367in}}%
\pgfpathlineto{\pgfqpoint{3.492874in}{2.277772in}}%
\pgfpathlineto{\pgfqpoint{3.485166in}{2.271323in}}%
\pgfpathlineto{\pgfqpoint{3.477452in}{2.264926in}}%
\pgfpathlineto{\pgfqpoint{3.469731in}{2.258582in}}%
\pgfpathlineto{\pgfqpoint{3.462003in}{2.252296in}}%
\pgfpathlineto{\pgfqpoint{3.449151in}{2.257007in}}%
\pgfpathlineto{\pgfqpoint{3.436305in}{2.261753in}}%
\pgfpathlineto{\pgfqpoint{3.423462in}{2.266532in}}%
\pgfpathlineto{\pgfqpoint{3.410625in}{2.271346in}}%
\pgfpathlineto{\pgfqpoint{3.418368in}{2.277512in}}%
\pgfpathlineto{\pgfqpoint{3.426104in}{2.283737in}}%
\pgfpathlineto{\pgfqpoint{3.433833in}{2.290019in}}%
\pgfpathlineto{\pgfqpoint{3.441555in}{2.296356in}}%
\pgfpathclose%
\pgfusepath{fill}%
\end{pgfscope}%
\begin{pgfscope}%
\pgfpathrectangle{\pgfqpoint{1.254980in}{0.150000in}}{\pgfqpoint{5.490039in}{5.490039in}}%
\pgfusepath{clip}%
\pgfsetbuttcap%
\pgfsetroundjoin%
\definecolor{currentfill}{rgb}{0.268510,0.009605,0.335427}%
\pgfsetfillcolor{currentfill}%
\pgfsetfillopacity{0.700000}%
\pgfsetlinewidth{0.000000pt}%
\definecolor{currentstroke}{rgb}{0.000000,0.000000,0.000000}%
\pgfsetstrokecolor{currentstroke}%
\pgfsetdash{}{0pt}%
\pgfpathmoveto{\pgfqpoint{3.923695in}{2.293585in}}%
\pgfpathlineto{\pgfqpoint{3.936612in}{2.290111in}}%
\pgfpathlineto{\pgfqpoint{3.949535in}{2.286667in}}%
\pgfpathlineto{\pgfqpoint{3.962463in}{2.283251in}}%
\pgfpathlineto{\pgfqpoint{3.975398in}{2.279865in}}%
\pgfpathlineto{\pgfqpoint{3.967876in}{2.272456in}}%
\pgfpathlineto{\pgfqpoint{3.960350in}{2.265046in}}%
\pgfpathlineto{\pgfqpoint{3.952817in}{2.257636in}}%
\pgfpathlineto{\pgfqpoint{3.945280in}{2.250227in}}%
\pgfpathlineto{\pgfqpoint{3.932333in}{2.253680in}}%
\pgfpathlineto{\pgfqpoint{3.919392in}{2.257161in}}%
\pgfpathlineto{\pgfqpoint{3.906457in}{2.260671in}}%
\pgfpathlineto{\pgfqpoint{3.893528in}{2.264210in}}%
\pgfpathlineto{\pgfqpoint{3.901078in}{2.271549in}}%
\pgfpathlineto{\pgfqpoint{3.908622in}{2.278891in}}%
\pgfpathlineto{\pgfqpoint{3.916161in}{2.286237in}}%
\pgfpathlineto{\pgfqpoint{3.923695in}{2.293585in}}%
\pgfpathclose%
\pgfusepath{fill}%
\end{pgfscope}%
\begin{pgfscope}%
\pgfpathrectangle{\pgfqpoint{1.254980in}{0.150000in}}{\pgfqpoint{5.490039in}{5.490039in}}%
\pgfusepath{clip}%
\pgfsetbuttcap%
\pgfsetroundjoin%
\definecolor{currentfill}{rgb}{0.273809,0.031497,0.358853}%
\pgfsetfillcolor{currentfill}%
\pgfsetfillopacity{0.700000}%
\pgfsetlinewidth{0.000000pt}%
\definecolor{currentstroke}{rgb}{0.000000,0.000000,0.000000}%
\pgfsetstrokecolor{currentstroke}%
\pgfsetdash{}{0pt}%
\pgfpathmoveto{\pgfqpoint{3.174536in}{2.331296in}}%
\pgfpathlineto{\pgfqpoint{3.187320in}{2.325769in}}%
\pgfpathlineto{\pgfqpoint{3.200108in}{2.320282in}}%
\pgfpathlineto{\pgfqpoint{3.212900in}{2.314832in}}%
\pgfpathlineto{\pgfqpoint{3.225696in}{2.309421in}}%
\pgfpathlineto{\pgfqpoint{3.217868in}{2.304011in}}%
\pgfpathlineto{\pgfqpoint{3.210033in}{2.298689in}}%
\pgfpathlineto{\pgfqpoint{3.202189in}{2.293459in}}%
\pgfpathlineto{\pgfqpoint{3.194338in}{2.288322in}}%
\pgfpathlineto{\pgfqpoint{3.181524in}{2.293876in}}%
\pgfpathlineto{\pgfqpoint{3.168715in}{2.299468in}}%
\pgfpathlineto{\pgfqpoint{3.155910in}{2.305098in}}%
\pgfpathlineto{\pgfqpoint{3.143108in}{2.310767in}}%
\pgfpathlineto{\pgfqpoint{3.150978in}{2.315756in}}%
\pgfpathlineto{\pgfqpoint{3.158839in}{2.320843in}}%
\pgfpathlineto{\pgfqpoint{3.166691in}{2.326024in}}%
\pgfpathlineto{\pgfqpoint{3.174536in}{2.331296in}}%
\pgfpathclose%
\pgfusepath{fill}%
\end{pgfscope}%
\begin{pgfscope}%
\pgfpathrectangle{\pgfqpoint{1.254980in}{0.150000in}}{\pgfqpoint{5.490039in}{5.490039in}}%
\pgfusepath{clip}%
\pgfsetbuttcap%
\pgfsetroundjoin%
\definecolor{currentfill}{rgb}{0.267004,0.004874,0.329415}%
\pgfsetfillcolor{currentfill}%
\pgfsetfillopacity{0.700000}%
\pgfsetlinewidth{0.000000pt}%
\definecolor{currentstroke}{rgb}{0.000000,0.000000,0.000000}%
\pgfsetstrokecolor{currentstroke}%
\pgfsetdash{}{0pt}%
\pgfpathmoveto{\pgfqpoint{3.574979in}{2.286398in}}%
\pgfpathlineto{\pgfqpoint{3.587827in}{2.282072in}}%
\pgfpathlineto{\pgfqpoint{3.600679in}{2.277779in}}%
\pgfpathlineto{\pgfqpoint{3.613537in}{2.273519in}}%
\pgfpathlineto{\pgfqpoint{3.626399in}{2.269290in}}%
\pgfpathlineto{\pgfqpoint{3.618745in}{2.262463in}}%
\pgfpathlineto{\pgfqpoint{3.611085in}{2.255670in}}%
\pgfpathlineto{\pgfqpoint{3.603418in}{2.248916in}}%
\pgfpathlineto{\pgfqpoint{3.595746in}{2.242201in}}%
\pgfpathlineto{\pgfqpoint{3.582869in}{2.246534in}}%
\pgfpathlineto{\pgfqpoint{3.569998in}{2.250898in}}%
\pgfpathlineto{\pgfqpoint{3.557131in}{2.255295in}}%
\pgfpathlineto{\pgfqpoint{3.544270in}{2.259725in}}%
\pgfpathlineto{\pgfqpoint{3.551957in}{2.266331in}}%
\pgfpathlineto{\pgfqpoint{3.559637in}{2.272980in}}%
\pgfpathlineto{\pgfqpoint{3.567311in}{2.279670in}}%
\pgfpathlineto{\pgfqpoint{3.574979in}{2.286398in}}%
\pgfpathclose%
\pgfusepath{fill}%
\end{pgfscope}%
\begin{pgfscope}%
\pgfpathrectangle{\pgfqpoint{1.254980in}{0.150000in}}{\pgfqpoint{5.490039in}{5.490039in}}%
\pgfusepath{clip}%
\pgfsetbuttcap%
\pgfsetroundjoin%
\definecolor{currentfill}{rgb}{0.277018,0.050344,0.375715}%
\pgfsetfillcolor{currentfill}%
\pgfsetfillopacity{0.700000}%
\pgfsetlinewidth{0.000000pt}%
\definecolor{currentstroke}{rgb}{0.000000,0.000000,0.000000}%
\pgfsetstrokecolor{currentstroke}%
\pgfsetdash{}{0pt}%
\pgfpathmoveto{\pgfqpoint{3.040836in}{2.357546in}}%
\pgfpathlineto{\pgfqpoint{3.053607in}{2.351556in}}%
\pgfpathlineto{\pgfqpoint{3.066382in}{2.345608in}}%
\pgfpathlineto{\pgfqpoint{3.079160in}{2.339701in}}%
\pgfpathlineto{\pgfqpoint{3.091942in}{2.333834in}}%
\pgfpathlineto{\pgfqpoint{3.084046in}{2.329097in}}%
\pgfpathlineto{\pgfqpoint{3.076142in}{2.324468in}}%
\pgfpathlineto{\pgfqpoint{3.068228in}{2.319951in}}%
\pgfpathlineto{\pgfqpoint{3.060305in}{2.315550in}}%
\pgfpathlineto{\pgfqpoint{3.047505in}{2.321572in}}%
\pgfpathlineto{\pgfqpoint{3.034707in}{2.327635in}}%
\pgfpathlineto{\pgfqpoint{3.021914in}{2.333738in}}%
\pgfpathlineto{\pgfqpoint{3.009124in}{2.339883in}}%
\pgfpathlineto{\pgfqpoint{3.017066in}{2.344124in}}%
\pgfpathlineto{\pgfqpoint{3.024999in}{2.348484in}}%
\pgfpathlineto{\pgfqpoint{3.032922in}{2.352959in}}%
\pgfpathlineto{\pgfqpoint{3.040836in}{2.357546in}}%
\pgfpathclose%
\pgfusepath{fill}%
\end{pgfscope}%
\begin{pgfscope}%
\pgfpathrectangle{\pgfqpoint{1.254980in}{0.150000in}}{\pgfqpoint{5.490039in}{5.490039in}}%
\pgfusepath{clip}%
\pgfsetbuttcap%
\pgfsetroundjoin%
\definecolor{currentfill}{rgb}{0.267004,0.004874,0.329415}%
\pgfsetfillcolor{currentfill}%
\pgfsetfillopacity{0.700000}%
\pgfsetlinewidth{0.000000pt}%
\definecolor{currentstroke}{rgb}{0.000000,0.000000,0.000000}%
\pgfsetstrokecolor{currentstroke}%
\pgfsetdash{}{0pt}%
\pgfpathmoveto{\pgfqpoint{3.708405in}{2.280677in}}%
\pgfpathlineto{\pgfqpoint{3.721280in}{2.276697in}}%
\pgfpathlineto{\pgfqpoint{3.734161in}{2.272748in}}%
\pgfpathlineto{\pgfqpoint{3.747047in}{2.268830in}}%
\pgfpathlineto{\pgfqpoint{3.759939in}{2.264943in}}%
\pgfpathlineto{\pgfqpoint{3.752335in}{2.257823in}}%
\pgfpathlineto{\pgfqpoint{3.744726in}{2.250724in}}%
\pgfpathlineto{\pgfqpoint{3.737110in}{2.243647in}}%
\pgfpathlineto{\pgfqpoint{3.729489in}{2.236595in}}%
\pgfpathlineto{\pgfqpoint{3.716584in}{2.240573in}}%
\pgfpathlineto{\pgfqpoint{3.703685in}{2.244582in}}%
\pgfpathlineto{\pgfqpoint{3.690791in}{2.248622in}}%
\pgfpathlineto{\pgfqpoint{3.677902in}{2.252693in}}%
\pgfpathlineto{\pgfqpoint{3.685537in}{2.259649in}}%
\pgfpathlineto{\pgfqpoint{3.693165in}{2.266633in}}%
\pgfpathlineto{\pgfqpoint{3.700788in}{2.273643in}}%
\pgfpathlineto{\pgfqpoint{3.708405in}{2.280677in}}%
\pgfpathclose%
\pgfusepath{fill}%
\end{pgfscope}%
\begin{pgfscope}%
\pgfpathrectangle{\pgfqpoint{1.254980in}{0.150000in}}{\pgfqpoint{5.490039in}{5.490039in}}%
\pgfusepath{clip}%
\pgfsetbuttcap%
\pgfsetroundjoin%
\definecolor{currentfill}{rgb}{0.282884,0.135920,0.453427}%
\pgfsetfillcolor{currentfill}%
\pgfsetfillopacity{0.700000}%
\pgfsetlinewidth{0.000000pt}%
\definecolor{currentstroke}{rgb}{0.000000,0.000000,0.000000}%
\pgfsetstrokecolor{currentstroke}%
\pgfsetdash{}{0pt}%
\pgfpathmoveto{\pgfqpoint{5.778903in}{2.507223in}}%
\pgfpathlineto{\pgfqpoint{5.792300in}{2.505526in}}%
\pgfpathlineto{\pgfqpoint{5.805705in}{2.503853in}}%
\pgfpathlineto{\pgfqpoint{5.819118in}{2.502202in}}%
\pgfpathlineto{\pgfqpoint{5.832538in}{2.500574in}}%
\pgfpathlineto{\pgfqpoint{5.825768in}{2.495409in}}%
\pgfpathlineto{\pgfqpoint{5.818993in}{2.490282in}}%
\pgfpathlineto{\pgfqpoint{5.812215in}{2.485189in}}%
\pgfpathlineto{\pgfqpoint{5.805432in}{2.480126in}}%
\pgfpathlineto{\pgfqpoint{5.791991in}{2.481601in}}%
\pgfpathlineto{\pgfqpoint{5.778557in}{2.483099in}}%
\pgfpathlineto{\pgfqpoint{5.765132in}{2.484620in}}%
\pgfpathlineto{\pgfqpoint{5.751715in}{2.486164in}}%
\pgfpathlineto{\pgfqpoint{5.758518in}{2.491376in}}%
\pgfpathlineto{\pgfqpoint{5.765317in}{2.496620in}}%
\pgfpathlineto{\pgfqpoint{5.772112in}{2.501901in}}%
\pgfpathlineto{\pgfqpoint{5.778903in}{2.507223in}}%
\pgfpathclose%
\pgfusepath{fill}%
\end{pgfscope}%
\begin{pgfscope}%
\pgfpathrectangle{\pgfqpoint{1.254980in}{0.150000in}}{\pgfqpoint{5.490039in}{5.490039in}}%
\pgfusepath{clip}%
\pgfsetbuttcap%
\pgfsetroundjoin%
\definecolor{currentfill}{rgb}{0.272594,0.025563,0.353093}%
\pgfsetfillcolor{currentfill}%
\pgfsetfillopacity{0.700000}%
\pgfsetlinewidth{0.000000pt}%
\definecolor{currentstroke}{rgb}{0.000000,0.000000,0.000000}%
\pgfsetstrokecolor{currentstroke}%
\pgfsetdash{}{0pt}%
\pgfpathmoveto{\pgfqpoint{4.272423in}{2.319929in}}%
\pgfpathlineto{\pgfqpoint{4.285425in}{2.317137in}}%
\pgfpathlineto{\pgfqpoint{4.298434in}{2.314372in}}%
\pgfpathlineto{\pgfqpoint{4.311450in}{2.311635in}}%
\pgfpathlineto{\pgfqpoint{4.324471in}{2.308923in}}%
\pgfpathlineto{\pgfqpoint{4.317075in}{2.301480in}}%
\pgfpathlineto{\pgfqpoint{4.309673in}{2.294012in}}%
\pgfpathlineto{\pgfqpoint{4.302265in}{2.286519in}}%
\pgfpathlineto{\pgfqpoint{4.294852in}{2.279001in}}%
\pgfpathlineto{\pgfqpoint{4.281819in}{2.281739in}}%
\pgfpathlineto{\pgfqpoint{4.268792in}{2.284505in}}%
\pgfpathlineto{\pgfqpoint{4.255771in}{2.287297in}}%
\pgfpathlineto{\pgfqpoint{4.242757in}{2.290116in}}%
\pgfpathlineto{\pgfqpoint{4.250182in}{2.297602in}}%
\pgfpathlineto{\pgfqpoint{4.257601in}{2.305066in}}%
\pgfpathlineto{\pgfqpoint{4.265015in}{2.312509in}}%
\pgfpathlineto{\pgfqpoint{4.272423in}{2.319929in}}%
\pgfpathclose%
\pgfusepath{fill}%
\end{pgfscope}%
\begin{pgfscope}%
\pgfpathrectangle{\pgfqpoint{1.254980in}{0.150000in}}{\pgfqpoint{5.490039in}{5.490039in}}%
\pgfusepath{clip}%
\pgfsetbuttcap%
\pgfsetroundjoin%
\definecolor{currentfill}{rgb}{0.276022,0.044167,0.370164}%
\pgfsetfillcolor{currentfill}%
\pgfsetfillopacity{0.700000}%
\pgfsetlinewidth{0.000000pt}%
\definecolor{currentstroke}{rgb}{0.000000,0.000000,0.000000}%
\pgfsetstrokecolor{currentstroke}%
\pgfsetdash{}{0pt}%
\pgfpathmoveto{\pgfqpoint{4.487662in}{2.346967in}}%
\pgfpathlineto{\pgfqpoint{4.500719in}{2.344510in}}%
\pgfpathlineto{\pgfqpoint{4.513782in}{2.342078in}}%
\pgfpathlineto{\pgfqpoint{4.526852in}{2.339673in}}%
\pgfpathlineto{\pgfqpoint{4.539929in}{2.337293in}}%
\pgfpathlineto{\pgfqpoint{4.532613in}{2.330056in}}%
\pgfpathlineto{\pgfqpoint{4.525291in}{2.322786in}}%
\pgfpathlineto{\pgfqpoint{4.517963in}{2.315483in}}%
\pgfpathlineto{\pgfqpoint{4.510630in}{2.308145in}}%
\pgfpathlineto{\pgfqpoint{4.497541in}{2.310526in}}%
\pgfpathlineto{\pgfqpoint{4.484459in}{2.312934in}}%
\pgfpathlineto{\pgfqpoint{4.471384in}{2.315367in}}%
\pgfpathlineto{\pgfqpoint{4.458315in}{2.317826in}}%
\pgfpathlineto{\pgfqpoint{4.465660in}{2.325157in}}%
\pgfpathlineto{\pgfqpoint{4.473000in}{2.332457in}}%
\pgfpathlineto{\pgfqpoint{4.480334in}{2.339727in}}%
\pgfpathlineto{\pgfqpoint{4.487662in}{2.346967in}}%
\pgfpathclose%
\pgfusepath{fill}%
\end{pgfscope}%
\begin{pgfscope}%
\pgfpathrectangle{\pgfqpoint{1.254980in}{0.150000in}}{\pgfqpoint{5.490039in}{5.490039in}}%
\pgfusepath{clip}%
\pgfsetbuttcap%
\pgfsetroundjoin%
\definecolor{currentfill}{rgb}{0.278791,0.062145,0.386592}%
\pgfsetfillcolor{currentfill}%
\pgfsetfillopacity{0.700000}%
\pgfsetlinewidth{0.000000pt}%
\definecolor{currentstroke}{rgb}{0.000000,0.000000,0.000000}%
\pgfsetstrokecolor{currentstroke}%
\pgfsetdash{}{0pt}%
\pgfpathmoveto{\pgfqpoint{4.702916in}{2.375697in}}%
\pgfpathlineto{\pgfqpoint{4.716029in}{2.373512in}}%
\pgfpathlineto{\pgfqpoint{4.729150in}{2.371352in}}%
\pgfpathlineto{\pgfqpoint{4.742277in}{2.369217in}}%
\pgfpathlineto{\pgfqpoint{4.755411in}{2.367108in}}%
\pgfpathlineto{\pgfqpoint{4.748178in}{2.360192in}}%
\pgfpathlineto{\pgfqpoint{4.740939in}{2.353241in}}%
\pgfpathlineto{\pgfqpoint{4.733694in}{2.346253in}}%
\pgfpathlineto{\pgfqpoint{4.726444in}{2.339228in}}%
\pgfpathlineto{\pgfqpoint{4.713297in}{2.341313in}}%
\pgfpathlineto{\pgfqpoint{4.700157in}{2.343424in}}%
\pgfpathlineto{\pgfqpoint{4.687024in}{2.345560in}}%
\pgfpathlineto{\pgfqpoint{4.673898in}{2.347721in}}%
\pgfpathlineto{\pgfqpoint{4.681161in}{2.354766in}}%
\pgfpathlineto{\pgfqpoint{4.688419in}{2.361776in}}%
\pgfpathlineto{\pgfqpoint{4.695670in}{2.368752in}}%
\pgfpathlineto{\pgfqpoint{4.702916in}{2.375697in}}%
\pgfpathclose%
\pgfusepath{fill}%
\end{pgfscope}%
\begin{pgfscope}%
\pgfpathrectangle{\pgfqpoint{1.254980in}{0.150000in}}{\pgfqpoint{5.490039in}{5.490039in}}%
\pgfusepath{clip}%
\pgfsetbuttcap%
\pgfsetroundjoin%
\definecolor{currentfill}{rgb}{0.269944,0.014625,0.341379}%
\pgfsetfillcolor{currentfill}%
\pgfsetfillopacity{0.700000}%
\pgfsetlinewidth{0.000000pt}%
\definecolor{currentstroke}{rgb}{0.000000,0.000000,0.000000}%
\pgfsetstrokecolor{currentstroke}%
\pgfsetdash{}{0pt}%
\pgfpathmoveto{\pgfqpoint{4.057176in}{2.296419in}}%
\pgfpathlineto{\pgfqpoint{4.070128in}{2.293228in}}%
\pgfpathlineto{\pgfqpoint{4.083087in}{2.290065in}}%
\pgfpathlineto{\pgfqpoint{4.096051in}{2.286931in}}%
\pgfpathlineto{\pgfqpoint{4.109021in}{2.283824in}}%
\pgfpathlineto{\pgfqpoint{4.101545in}{2.276335in}}%
\pgfpathlineto{\pgfqpoint{4.094064in}{2.268833in}}%
\pgfpathlineto{\pgfqpoint{4.086578in}{2.261321in}}%
\pgfpathlineto{\pgfqpoint{4.079086in}{2.253799in}}%
\pgfpathlineto{\pgfqpoint{4.066104in}{2.256959in}}%
\pgfpathlineto{\pgfqpoint{4.053128in}{2.260146in}}%
\pgfpathlineto{\pgfqpoint{4.040158in}{2.263362in}}%
\pgfpathlineto{\pgfqpoint{4.027194in}{2.266606in}}%
\pgfpathlineto{\pgfqpoint{4.034698in}{2.274070in}}%
\pgfpathlineto{\pgfqpoint{4.042196in}{2.281527in}}%
\pgfpathlineto{\pgfqpoint{4.049689in}{2.288977in}}%
\pgfpathlineto{\pgfqpoint{4.057176in}{2.296419in}}%
\pgfpathclose%
\pgfusepath{fill}%
\end{pgfscope}%
\begin{pgfscope}%
\pgfpathrectangle{\pgfqpoint{1.254980in}{0.150000in}}{\pgfqpoint{5.490039in}{5.490039in}}%
\pgfusepath{clip}%
\pgfsetbuttcap%
\pgfsetroundjoin%
\definecolor{currentfill}{rgb}{0.283187,0.125848,0.444960}%
\pgfsetfillcolor{currentfill}%
\pgfsetfillopacity{0.700000}%
\pgfsetlinewidth{0.000000pt}%
\definecolor{currentstroke}{rgb}{0.000000,0.000000,0.000000}%
\pgfsetstrokecolor{currentstroke}%
\pgfsetdash{}{0pt}%
\pgfpathmoveto{\pgfqpoint{5.563826in}{2.484202in}}%
\pgfpathlineto{\pgfqpoint{5.577169in}{2.482520in}}%
\pgfpathlineto{\pgfqpoint{5.590520in}{2.480862in}}%
\pgfpathlineto{\pgfqpoint{5.603878in}{2.479226in}}%
\pgfpathlineto{\pgfqpoint{5.617245in}{2.477614in}}%
\pgfpathlineto{\pgfqpoint{5.610379in}{2.472232in}}%
\pgfpathlineto{\pgfqpoint{5.603509in}{2.466863in}}%
\pgfpathlineto{\pgfqpoint{5.596634in}{2.461502in}}%
\pgfpathlineto{\pgfqpoint{5.589753in}{2.456145in}}%
\pgfpathlineto{\pgfqpoint{5.576369in}{2.457630in}}%
\pgfpathlineto{\pgfqpoint{5.562992in}{2.459138in}}%
\pgfpathlineto{\pgfqpoint{5.549623in}{2.460670in}}%
\pgfpathlineto{\pgfqpoint{5.536261in}{2.462225in}}%
\pgfpathlineto{\pgfqpoint{5.543160in}{2.467704in}}%
\pgfpathlineto{\pgfqpoint{5.550054in}{2.473191in}}%
\pgfpathlineto{\pgfqpoint{5.556942in}{2.478689in}}%
\pgfpathlineto{\pgfqpoint{5.563826in}{2.484202in}}%
\pgfpathclose%
\pgfusepath{fill}%
\end{pgfscope}%
\begin{pgfscope}%
\pgfpathrectangle{\pgfqpoint{1.254980in}{0.150000in}}{\pgfqpoint{5.490039in}{5.490039in}}%
\pgfusepath{clip}%
\pgfsetbuttcap%
\pgfsetroundjoin%
\definecolor{currentfill}{rgb}{0.281446,0.084320,0.407414}%
\pgfsetfillcolor{currentfill}%
\pgfsetfillopacity{0.700000}%
\pgfsetlinewidth{0.000000pt}%
\definecolor{currentstroke}{rgb}{0.000000,0.000000,0.000000}%
\pgfsetstrokecolor{currentstroke}%
\pgfsetdash{}{0pt}%
\pgfpathmoveto{\pgfqpoint{4.918182in}{2.404666in}}%
\pgfpathlineto{\pgfqpoint{4.931353in}{2.402694in}}%
\pgfpathlineto{\pgfqpoint{4.944532in}{2.400745in}}%
\pgfpathlineto{\pgfqpoint{4.957717in}{2.398822in}}%
\pgfpathlineto{\pgfqpoint{4.970910in}{2.396923in}}%
\pgfpathlineto{\pgfqpoint{4.963764in}{2.390398in}}%
\pgfpathlineto{\pgfqpoint{4.956612in}{2.383842in}}%
\pgfpathlineto{\pgfqpoint{4.949454in}{2.377252in}}%
\pgfpathlineto{\pgfqpoint{4.942291in}{2.370626in}}%
\pgfpathlineto{\pgfqpoint{4.929084in}{2.372476in}}%
\pgfpathlineto{\pgfqpoint{4.915885in}{2.374349in}}%
\pgfpathlineto{\pgfqpoint{4.902693in}{2.376248in}}%
\pgfpathlineto{\pgfqpoint{4.889508in}{2.378171in}}%
\pgfpathlineto{\pgfqpoint{4.896685in}{2.384842in}}%
\pgfpathlineto{\pgfqpoint{4.903857in}{2.391480in}}%
\pgfpathlineto{\pgfqpoint{4.911022in}{2.398087in}}%
\pgfpathlineto{\pgfqpoint{4.918182in}{2.404666in}}%
\pgfpathclose%
\pgfusepath{fill}%
\end{pgfscope}%
\begin{pgfscope}%
\pgfpathrectangle{\pgfqpoint{1.254980in}{0.150000in}}{\pgfqpoint{5.490039in}{5.490039in}}%
\pgfusepath{clip}%
\pgfsetbuttcap%
\pgfsetroundjoin%
\definecolor{currentfill}{rgb}{0.283197,0.115680,0.436115}%
\pgfsetfillcolor{currentfill}%
\pgfsetfillopacity{0.700000}%
\pgfsetlinewidth{0.000000pt}%
\definecolor{currentstroke}{rgb}{0.000000,0.000000,0.000000}%
\pgfsetstrokecolor{currentstroke}%
\pgfsetdash{}{0pt}%
\pgfpathmoveto{\pgfqpoint{5.348665in}{2.459420in}}%
\pgfpathlineto{\pgfqpoint{5.361952in}{2.457697in}}%
\pgfpathlineto{\pgfqpoint{5.375247in}{2.455999in}}%
\pgfpathlineto{\pgfqpoint{5.388549in}{2.454324in}}%
\pgfpathlineto{\pgfqpoint{5.401858in}{2.452672in}}%
\pgfpathlineto{\pgfqpoint{5.394898in}{2.446961in}}%
\pgfpathlineto{\pgfqpoint{5.387931in}{2.441241in}}%
\pgfpathlineto{\pgfqpoint{5.380959in}{2.435510in}}%
\pgfpathlineto{\pgfqpoint{5.373981in}{2.429764in}}%
\pgfpathlineto{\pgfqpoint{5.360655in}{2.431314in}}%
\pgfpathlineto{\pgfqpoint{5.347336in}{2.432888in}}%
\pgfpathlineto{\pgfqpoint{5.334025in}{2.434485in}}%
\pgfpathlineto{\pgfqpoint{5.320722in}{2.436106in}}%
\pgfpathlineto{\pgfqpoint{5.327716in}{2.441949in}}%
\pgfpathlineto{\pgfqpoint{5.334705in}{2.447780in}}%
\pgfpathlineto{\pgfqpoint{5.341688in}{2.453602in}}%
\pgfpathlineto{\pgfqpoint{5.348665in}{2.459420in}}%
\pgfpathclose%
\pgfusepath{fill}%
\end{pgfscope}%
\begin{pgfscope}%
\pgfpathrectangle{\pgfqpoint{1.254980in}{0.150000in}}{\pgfqpoint{5.490039in}{5.490039in}}%
\pgfusepath{clip}%
\pgfsetbuttcap%
\pgfsetroundjoin%
\definecolor{currentfill}{rgb}{0.282656,0.100196,0.422160}%
\pgfsetfillcolor{currentfill}%
\pgfsetfillopacity{0.700000}%
\pgfsetlinewidth{0.000000pt}%
\definecolor{currentstroke}{rgb}{0.000000,0.000000,0.000000}%
\pgfsetstrokecolor{currentstroke}%
\pgfsetdash{}{0pt}%
\pgfpathmoveto{\pgfqpoint{5.133441in}{2.432804in}}%
\pgfpathlineto{\pgfqpoint{5.146671in}{2.430985in}}%
\pgfpathlineto{\pgfqpoint{5.159907in}{2.429190in}}%
\pgfpathlineto{\pgfqpoint{5.173151in}{2.427420in}}%
\pgfpathlineto{\pgfqpoint{5.186403in}{2.425673in}}%
\pgfpathlineto{\pgfqpoint{5.179348in}{2.419565in}}%
\pgfpathlineto{\pgfqpoint{5.172287in}{2.413435in}}%
\pgfpathlineto{\pgfqpoint{5.165220in}{2.407279in}}%
\pgfpathlineto{\pgfqpoint{5.158148in}{2.401095in}}%
\pgfpathlineto{\pgfqpoint{5.144881in}{2.402766in}}%
\pgfpathlineto{\pgfqpoint{5.131622in}{2.404461in}}%
\pgfpathlineto{\pgfqpoint{5.118370in}{2.406180in}}%
\pgfpathlineto{\pgfqpoint{5.105126in}{2.407923in}}%
\pgfpathlineto{\pgfqpoint{5.112213in}{2.414178in}}%
\pgfpathlineto{\pgfqpoint{5.119295in}{2.420409in}}%
\pgfpathlineto{\pgfqpoint{5.126371in}{2.426616in}}%
\pgfpathlineto{\pgfqpoint{5.133441in}{2.432804in}}%
\pgfpathclose%
\pgfusepath{fill}%
\end{pgfscope}%
\begin{pgfscope}%
\pgfpathrectangle{\pgfqpoint{1.254980in}{0.150000in}}{\pgfqpoint{5.490039in}{5.490039in}}%
\pgfusepath{clip}%
\pgfsetbuttcap%
\pgfsetroundjoin%
\definecolor{currentfill}{rgb}{0.283091,0.110553,0.431554}%
\pgfsetfillcolor{currentfill}%
\pgfsetfillopacity{0.700000}%
\pgfsetlinewidth{0.000000pt}%
\definecolor{currentstroke}{rgb}{0.000000,0.000000,0.000000}%
\pgfsetstrokecolor{currentstroke}%
\pgfsetdash{}{0pt}%
\pgfpathmoveto{\pgfqpoint{2.721720in}{2.459771in}}%
\pgfpathlineto{\pgfqpoint{2.734470in}{2.452543in}}%
\pgfpathlineto{\pgfqpoint{2.747223in}{2.445364in}}%
\pgfpathlineto{\pgfqpoint{2.759978in}{2.438236in}}%
\pgfpathlineto{\pgfqpoint{2.772736in}{2.431156in}}%
\pgfpathlineto{\pgfqpoint{2.764659in}{2.428279in}}%
\pgfpathlineto{\pgfqpoint{2.756571in}{2.425559in}}%
\pgfpathlineto{\pgfqpoint{2.748471in}{2.423000in}}%
\pgfpathlineto{\pgfqpoint{2.740359in}{2.420607in}}%
\pgfpathlineto{\pgfqpoint{2.727579in}{2.427869in}}%
\pgfpathlineto{\pgfqpoint{2.714800in}{2.435181in}}%
\pgfpathlineto{\pgfqpoint{2.702025in}{2.442542in}}%
\pgfpathlineto{\pgfqpoint{2.689251in}{2.449953in}}%
\pgfpathlineto{\pgfqpoint{2.697386in}{2.452158in}}%
\pgfpathlineto{\pgfqpoint{2.705510in}{2.454532in}}%
\pgfpathlineto{\pgfqpoint{2.713621in}{2.457072in}}%
\pgfpathlineto{\pgfqpoint{2.721720in}{2.459771in}}%
\pgfpathclose%
\pgfusepath{fill}%
\end{pgfscope}%
\begin{pgfscope}%
\pgfpathrectangle{\pgfqpoint{1.254980in}{0.150000in}}{\pgfqpoint{5.490039in}{5.490039in}}%
\pgfusepath{clip}%
\pgfsetbuttcap%
\pgfsetroundjoin%
\definecolor{currentfill}{rgb}{0.267004,0.004874,0.329415}%
\pgfsetfillcolor{currentfill}%
\pgfsetfillopacity{0.700000}%
\pgfsetlinewidth{0.000000pt}%
\definecolor{currentstroke}{rgb}{0.000000,0.000000,0.000000}%
\pgfsetstrokecolor{currentstroke}%
\pgfsetdash{}{0pt}%
\pgfpathmoveto{\pgfqpoint{3.841867in}{2.278663in}}%
\pgfpathlineto{\pgfqpoint{3.854774in}{2.275005in}}%
\pgfpathlineto{\pgfqpoint{3.867686in}{2.271377in}}%
\pgfpathlineto{\pgfqpoint{3.880604in}{2.267779in}}%
\pgfpathlineto{\pgfqpoint{3.893528in}{2.264210in}}%
\pgfpathlineto{\pgfqpoint{3.885972in}{2.256878in}}%
\pgfpathlineto{\pgfqpoint{3.878411in}{2.249553in}}%
\pgfpathlineto{\pgfqpoint{3.870844in}{2.242237in}}%
\pgfpathlineto{\pgfqpoint{3.863272in}{2.234932in}}%
\pgfpathlineto{\pgfqpoint{3.850335in}{2.238579in}}%
\pgfpathlineto{\pgfqpoint{3.837405in}{2.242256in}}%
\pgfpathlineto{\pgfqpoint{3.824480in}{2.245962in}}%
\pgfpathlineto{\pgfqpoint{3.811561in}{2.249698in}}%
\pgfpathlineto{\pgfqpoint{3.819146in}{2.256920in}}%
\pgfpathlineto{\pgfqpoint{3.826725in}{2.264156in}}%
\pgfpathlineto{\pgfqpoint{3.834299in}{2.271404in}}%
\pgfpathlineto{\pgfqpoint{3.841867in}{2.278663in}}%
\pgfpathclose%
\pgfusepath{fill}%
\end{pgfscope}%
\begin{pgfscope}%
\pgfpathrectangle{\pgfqpoint{1.254980in}{0.150000in}}{\pgfqpoint{5.490039in}{5.490039in}}%
\pgfusepath{clip}%
\pgfsetbuttcap%
\pgfsetroundjoin%
\definecolor{currentfill}{rgb}{0.280267,0.073417,0.397163}%
\pgfsetfillcolor{currentfill}%
\pgfsetfillopacity{0.700000}%
\pgfsetlinewidth{0.000000pt}%
\definecolor{currentstroke}{rgb}{0.000000,0.000000,0.000000}%
\pgfsetstrokecolor{currentstroke}%
\pgfsetdash{}{0pt}%
\pgfpathmoveto{\pgfqpoint{2.906927in}{2.390577in}}%
\pgfpathlineto{\pgfqpoint{2.919690in}{2.384088in}}%
\pgfpathlineto{\pgfqpoint{2.932457in}{2.377643in}}%
\pgfpathlineto{\pgfqpoint{2.945226in}{2.371242in}}%
\pgfpathlineto{\pgfqpoint{2.957999in}{2.364884in}}%
\pgfpathlineto{\pgfqpoint{2.950028in}{2.360930in}}%
\pgfpathlineto{\pgfqpoint{2.942047in}{2.357106in}}%
\pgfpathlineto{\pgfqpoint{2.934056in}{2.353416in}}%
\pgfpathlineto{\pgfqpoint{2.926054in}{2.349864in}}%
\pgfpathlineto{\pgfqpoint{2.913261in}{2.356390in}}%
\pgfpathlineto{\pgfqpoint{2.900471in}{2.362960in}}%
\pgfpathlineto{\pgfqpoint{2.887684in}{2.369574in}}%
\pgfpathlineto{\pgfqpoint{2.874900in}{2.376233in}}%
\pgfpathlineto{\pgfqpoint{2.882923in}{2.379611in}}%
\pgfpathlineto{\pgfqpoint{2.890934in}{2.383130in}}%
\pgfpathlineto{\pgfqpoint{2.898936in}{2.386787in}}%
\pgfpathlineto{\pgfqpoint{2.906927in}{2.390577in}}%
\pgfpathclose%
\pgfusepath{fill}%
\end{pgfscope}%
\begin{pgfscope}%
\pgfpathrectangle{\pgfqpoint{1.254980in}{0.150000in}}{\pgfqpoint{5.490039in}{5.490039in}}%
\pgfusepath{clip}%
\pgfsetbuttcap%
\pgfsetroundjoin%
\definecolor{currentfill}{rgb}{0.269944,0.014625,0.341379}%
\pgfsetfillcolor{currentfill}%
\pgfsetfillopacity{0.700000}%
\pgfsetlinewidth{0.000000pt}%
\definecolor{currentstroke}{rgb}{0.000000,0.000000,0.000000}%
\pgfsetstrokecolor{currentstroke}%
\pgfsetdash{}{0pt}%
\pgfpathmoveto{\pgfqpoint{3.359321in}{2.290949in}}%
\pgfpathlineto{\pgfqpoint{3.372140in}{2.285995in}}%
\pgfpathlineto{\pgfqpoint{3.384964in}{2.281077in}}%
\pgfpathlineto{\pgfqpoint{3.397792in}{2.276195in}}%
\pgfpathlineto{\pgfqpoint{3.410625in}{2.271346in}}%
\pgfpathlineto{\pgfqpoint{3.402875in}{2.265244in}}%
\pgfpathlineto{\pgfqpoint{3.395119in}{2.259206in}}%
\pgfpathlineto{\pgfqpoint{3.387355in}{2.253238in}}%
\pgfpathlineto{\pgfqpoint{3.379584in}{2.247341in}}%
\pgfpathlineto{\pgfqpoint{3.366736in}{2.252319in}}%
\pgfpathlineto{\pgfqpoint{3.353892in}{2.257331in}}%
\pgfpathlineto{\pgfqpoint{3.341052in}{2.262378in}}%
\pgfpathlineto{\pgfqpoint{3.328217in}{2.267461in}}%
\pgfpathlineto{\pgfqpoint{3.336004in}{2.273224in}}%
\pgfpathlineto{\pgfqpoint{3.343784in}{2.279061in}}%
\pgfpathlineto{\pgfqpoint{3.351556in}{2.284971in}}%
\pgfpathlineto{\pgfqpoint{3.359321in}{2.290949in}}%
\pgfpathclose%
\pgfusepath{fill}%
\end{pgfscope}%
\begin{pgfscope}%
\pgfpathrectangle{\pgfqpoint{1.254980in}{0.150000in}}{\pgfqpoint{5.490039in}{5.490039in}}%
\pgfusepath{clip}%
\pgfsetbuttcap%
\pgfsetroundjoin%
\definecolor{currentfill}{rgb}{0.282290,0.145912,0.461510}%
\pgfsetfillcolor{currentfill}%
\pgfsetfillopacity{0.700000}%
\pgfsetlinewidth{0.000000pt}%
\definecolor{currentstroke}{rgb}{0.000000,0.000000,0.000000}%
\pgfsetstrokecolor{currentstroke}%
\pgfsetdash{}{0pt}%
\pgfpathmoveto{\pgfqpoint{5.913262in}{2.514774in}}%
\pgfpathlineto{\pgfqpoint{5.926701in}{2.513095in}}%
\pgfpathlineto{\pgfqpoint{5.940148in}{2.511439in}}%
\pgfpathlineto{\pgfqpoint{5.953602in}{2.509805in}}%
\pgfpathlineto{\pgfqpoint{5.967065in}{2.508194in}}%
\pgfpathlineto{\pgfqpoint{5.960351in}{2.503166in}}%
\pgfpathlineto{\pgfqpoint{5.953634in}{2.498189in}}%
\pgfpathlineto{\pgfqpoint{5.946914in}{2.493260in}}%
\pgfpathlineto{\pgfqpoint{5.940189in}{2.488374in}}%
\pgfpathlineto{\pgfqpoint{5.926705in}{2.489819in}}%
\pgfpathlineto{\pgfqpoint{5.913228in}{2.491287in}}%
\pgfpathlineto{\pgfqpoint{5.899760in}{2.492777in}}%
\pgfpathlineto{\pgfqpoint{5.886300in}{2.494291in}}%
\pgfpathlineto{\pgfqpoint{5.893046in}{2.499338in}}%
\pgfpathlineto{\pgfqpoint{5.899788in}{2.504432in}}%
\pgfpathlineto{\pgfqpoint{5.906526in}{2.509575in}}%
\pgfpathlineto{\pgfqpoint{5.913262in}{2.514774in}}%
\pgfpathclose%
\pgfusepath{fill}%
\end{pgfscope}%
\begin{pgfscope}%
\pgfpathrectangle{\pgfqpoint{1.254980in}{0.150000in}}{\pgfqpoint{5.490039in}{5.490039in}}%
\pgfusepath{clip}%
\pgfsetbuttcap%
\pgfsetroundjoin%
\definecolor{currentfill}{rgb}{0.268510,0.009605,0.335427}%
\pgfsetfillcolor{currentfill}%
\pgfsetfillopacity{0.700000}%
\pgfsetlinewidth{0.000000pt}%
\definecolor{currentstroke}{rgb}{0.000000,0.000000,0.000000}%
\pgfsetstrokecolor{currentstroke}%
\pgfsetdash{}{0pt}%
\pgfpathmoveto{\pgfqpoint{3.492874in}{2.277772in}}%
\pgfpathlineto{\pgfqpoint{3.505715in}{2.273210in}}%
\pgfpathlineto{\pgfqpoint{3.518562in}{2.268682in}}%
\pgfpathlineto{\pgfqpoint{3.531413in}{2.264187in}}%
\pgfpathlineto{\pgfqpoint{3.544270in}{2.259725in}}%
\pgfpathlineto{\pgfqpoint{3.536577in}{2.253164in}}%
\pgfpathlineto{\pgfqpoint{3.528877in}{2.246652in}}%
\pgfpathlineto{\pgfqpoint{3.521171in}{2.240190in}}%
\pgfpathlineto{\pgfqpoint{3.513458in}{2.233782in}}%
\pgfpathlineto{\pgfqpoint{3.500587in}{2.238361in}}%
\pgfpathlineto{\pgfqpoint{3.487721in}{2.242972in}}%
\pgfpathlineto{\pgfqpoint{3.474860in}{2.247617in}}%
\pgfpathlineto{\pgfqpoint{3.462003in}{2.252296in}}%
\pgfpathlineto{\pgfqpoint{3.469731in}{2.258582in}}%
\pgfpathlineto{\pgfqpoint{3.477452in}{2.264926in}}%
\pgfpathlineto{\pgfqpoint{3.485166in}{2.271323in}}%
\pgfpathlineto{\pgfqpoint{3.492874in}{2.277772in}}%
\pgfpathclose%
\pgfusepath{fill}%
\end{pgfscope}%
\begin{pgfscope}%
\pgfpathrectangle{\pgfqpoint{1.254980in}{0.150000in}}{\pgfqpoint{5.490039in}{5.490039in}}%
\pgfusepath{clip}%
\pgfsetbuttcap%
\pgfsetroundjoin%
\definecolor{currentfill}{rgb}{0.272594,0.025563,0.353093}%
\pgfsetfillcolor{currentfill}%
\pgfsetfillopacity{0.700000}%
\pgfsetlinewidth{0.000000pt}%
\definecolor{currentstroke}{rgb}{0.000000,0.000000,0.000000}%
\pgfsetstrokecolor{currentstroke}%
\pgfsetdash{}{0pt}%
\pgfpathmoveto{\pgfqpoint{3.225696in}{2.309421in}}%
\pgfpathlineto{\pgfqpoint{3.238496in}{2.304047in}}%
\pgfpathlineto{\pgfqpoint{3.251300in}{2.298711in}}%
\pgfpathlineto{\pgfqpoint{3.264109in}{2.293411in}}%
\pgfpathlineto{\pgfqpoint{3.276922in}{2.288149in}}%
\pgfpathlineto{\pgfqpoint{3.269111in}{2.282602in}}%
\pgfpathlineto{\pgfqpoint{3.261293in}{2.277139in}}%
\pgfpathlineto{\pgfqpoint{3.253466in}{2.271764in}}%
\pgfpathlineto{\pgfqpoint{3.245632in}{2.266481in}}%
\pgfpathlineto{\pgfqpoint{3.232802in}{2.271886in}}%
\pgfpathlineto{\pgfqpoint{3.219977in}{2.277327in}}%
\pgfpathlineto{\pgfqpoint{3.207155in}{2.282806in}}%
\pgfpathlineto{\pgfqpoint{3.194338in}{2.288322in}}%
\pgfpathlineto{\pgfqpoint{3.202189in}{2.293459in}}%
\pgfpathlineto{\pgfqpoint{3.210033in}{2.298689in}}%
\pgfpathlineto{\pgfqpoint{3.217868in}{2.304011in}}%
\pgfpathlineto{\pgfqpoint{3.225696in}{2.309421in}}%
\pgfpathclose%
\pgfusepath{fill}%
\end{pgfscope}%
\begin{pgfscope}%
\pgfpathrectangle{\pgfqpoint{1.254980in}{0.150000in}}{\pgfqpoint{5.490039in}{5.490039in}}%
\pgfusepath{clip}%
\pgfsetbuttcap%
\pgfsetroundjoin%
\definecolor{currentfill}{rgb}{0.274952,0.037752,0.364543}%
\pgfsetfillcolor{currentfill}%
\pgfsetfillopacity{0.700000}%
\pgfsetlinewidth{0.000000pt}%
\definecolor{currentstroke}{rgb}{0.000000,0.000000,0.000000}%
\pgfsetstrokecolor{currentstroke}%
\pgfsetdash{}{0pt}%
\pgfpathmoveto{\pgfqpoint{4.406106in}{2.327924in}}%
\pgfpathlineto{\pgfqpoint{4.419149in}{2.325360in}}%
\pgfpathlineto{\pgfqpoint{4.432198in}{2.322823in}}%
\pgfpathlineto{\pgfqpoint{4.445253in}{2.320311in}}%
\pgfpathlineto{\pgfqpoint{4.458315in}{2.317826in}}%
\pgfpathlineto{\pgfqpoint{4.450964in}{2.310464in}}%
\pgfpathlineto{\pgfqpoint{4.443608in}{2.303070in}}%
\pgfpathlineto{\pgfqpoint{4.436246in}{2.295645in}}%
\pgfpathlineto{\pgfqpoint{4.428879in}{2.288189in}}%
\pgfpathlineto{\pgfqpoint{4.415805in}{2.290688in}}%
\pgfpathlineto{\pgfqpoint{4.402738in}{2.293214in}}%
\pgfpathlineto{\pgfqpoint{4.389677in}{2.295766in}}%
\pgfpathlineto{\pgfqpoint{4.376623in}{2.298345in}}%
\pgfpathlineto{\pgfqpoint{4.384002in}{2.305782in}}%
\pgfpathlineto{\pgfqpoint{4.391376in}{2.313191in}}%
\pgfpathlineto{\pgfqpoint{4.398744in}{2.320572in}}%
\pgfpathlineto{\pgfqpoint{4.406106in}{2.327924in}}%
\pgfpathclose%
\pgfusepath{fill}%
\end{pgfscope}%
\begin{pgfscope}%
\pgfpathrectangle{\pgfqpoint{1.254980in}{0.150000in}}{\pgfqpoint{5.490039in}{5.490039in}}%
\pgfusepath{clip}%
\pgfsetbuttcap%
\pgfsetroundjoin%
\definecolor{currentfill}{rgb}{0.271305,0.019942,0.347269}%
\pgfsetfillcolor{currentfill}%
\pgfsetfillopacity{0.700000}%
\pgfsetlinewidth{0.000000pt}%
\definecolor{currentstroke}{rgb}{0.000000,0.000000,0.000000}%
\pgfsetstrokecolor{currentstroke}%
\pgfsetdash{}{0pt}%
\pgfpathmoveto{\pgfqpoint{4.190763in}{2.301665in}}%
\pgfpathlineto{\pgfqpoint{4.203752in}{2.298737in}}%
\pgfpathlineto{\pgfqpoint{4.216747in}{2.295836in}}%
\pgfpathlineto{\pgfqpoint{4.229749in}{2.292962in}}%
\pgfpathlineto{\pgfqpoint{4.242757in}{2.290116in}}%
\pgfpathlineto{\pgfqpoint{4.235327in}{2.282609in}}%
\pgfpathlineto{\pgfqpoint{4.227891in}{2.275082in}}%
\pgfpathlineto{\pgfqpoint{4.220450in}{2.267534in}}%
\pgfpathlineto{\pgfqpoint{4.213004in}{2.259967in}}%
\pgfpathlineto{\pgfqpoint{4.199984in}{2.262853in}}%
\pgfpathlineto{\pgfqpoint{4.186971in}{2.265767in}}%
\pgfpathlineto{\pgfqpoint{4.173964in}{2.268708in}}%
\pgfpathlineto{\pgfqpoint{4.160963in}{2.271676in}}%
\pgfpathlineto{\pgfqpoint{4.168421in}{2.279198in}}%
\pgfpathlineto{\pgfqpoint{4.175874in}{2.286704in}}%
\pgfpathlineto{\pgfqpoint{4.183321in}{2.294193in}}%
\pgfpathlineto{\pgfqpoint{4.190763in}{2.301665in}}%
\pgfpathclose%
\pgfusepath{fill}%
\end{pgfscope}%
\begin{pgfscope}%
\pgfpathrectangle{\pgfqpoint{1.254980in}{0.150000in}}{\pgfqpoint{5.490039in}{5.490039in}}%
\pgfusepath{clip}%
\pgfsetbuttcap%
\pgfsetroundjoin%
\definecolor{currentfill}{rgb}{0.277941,0.056324,0.381191}%
\pgfsetfillcolor{currentfill}%
\pgfsetfillopacity{0.700000}%
\pgfsetlinewidth{0.000000pt}%
\definecolor{currentstroke}{rgb}{0.000000,0.000000,0.000000}%
\pgfsetstrokecolor{currentstroke}%
\pgfsetdash{}{0pt}%
\pgfpathmoveto{\pgfqpoint{4.621463in}{2.356619in}}%
\pgfpathlineto{\pgfqpoint{4.634562in}{2.354356in}}%
\pgfpathlineto{\pgfqpoint{4.647667in}{2.352119in}}%
\pgfpathlineto{\pgfqpoint{4.660779in}{2.349907in}}%
\pgfpathlineto{\pgfqpoint{4.673898in}{2.347721in}}%
\pgfpathlineto{\pgfqpoint{4.666629in}{2.340641in}}%
\pgfpathlineto{\pgfqpoint{4.659355in}{2.333525in}}%
\pgfpathlineto{\pgfqpoint{4.652074in}{2.326371in}}%
\pgfpathlineto{\pgfqpoint{4.644788in}{2.319179in}}%
\pgfpathlineto{\pgfqpoint{4.631657in}{2.321354in}}%
\pgfpathlineto{\pgfqpoint{4.618532in}{2.323554in}}%
\pgfpathlineto{\pgfqpoint{4.605415in}{2.325780in}}%
\pgfpathlineto{\pgfqpoint{4.592304in}{2.328032in}}%
\pgfpathlineto{\pgfqpoint{4.599602in}{2.335230in}}%
\pgfpathlineto{\pgfqpoint{4.606895in}{2.342393in}}%
\pgfpathlineto{\pgfqpoint{4.614182in}{2.349523in}}%
\pgfpathlineto{\pgfqpoint{4.621463in}{2.356619in}}%
\pgfpathclose%
\pgfusepath{fill}%
\end{pgfscope}%
\begin{pgfscope}%
\pgfpathrectangle{\pgfqpoint{1.254980in}{0.150000in}}{\pgfqpoint{5.490039in}{5.490039in}}%
\pgfusepath{clip}%
\pgfsetbuttcap%
\pgfsetroundjoin%
\definecolor{currentfill}{rgb}{0.282884,0.135920,0.453427}%
\pgfsetfillcolor{currentfill}%
\pgfsetfillopacity{0.700000}%
\pgfsetlinewidth{0.000000pt}%
\definecolor{currentstroke}{rgb}{0.000000,0.000000,0.000000}%
\pgfsetstrokecolor{currentstroke}%
\pgfsetdash{}{0pt}%
\pgfpathmoveto{\pgfqpoint{5.698124in}{2.492572in}}%
\pgfpathlineto{\pgfqpoint{5.711510in}{2.490935in}}%
\pgfpathlineto{\pgfqpoint{5.724904in}{2.489322in}}%
\pgfpathlineto{\pgfqpoint{5.738306in}{2.487732in}}%
\pgfpathlineto{\pgfqpoint{5.751715in}{2.486164in}}%
\pgfpathlineto{\pgfqpoint{5.744907in}{2.480981in}}%
\pgfpathlineto{\pgfqpoint{5.738095in}{2.475820in}}%
\pgfpathlineto{\pgfqpoint{5.731278in}{2.470678in}}%
\pgfpathlineto{\pgfqpoint{5.724456in}{2.465551in}}%
\pgfpathlineto{\pgfqpoint{5.711027in}{2.466977in}}%
\pgfpathlineto{\pgfqpoint{5.697606in}{2.468427in}}%
\pgfpathlineto{\pgfqpoint{5.684193in}{2.469901in}}%
\pgfpathlineto{\pgfqpoint{5.670787in}{2.471397in}}%
\pgfpathlineto{\pgfqpoint{5.677629in}{2.476660in}}%
\pgfpathlineto{\pgfqpoint{5.684465in}{2.481940in}}%
\pgfpathlineto{\pgfqpoint{5.691297in}{2.487243in}}%
\pgfpathlineto{\pgfqpoint{5.698124in}{2.492572in}}%
\pgfpathclose%
\pgfusepath{fill}%
\end{pgfscope}%
\begin{pgfscope}%
\pgfpathrectangle{\pgfqpoint{1.254980in}{0.150000in}}{\pgfqpoint{5.490039in}{5.490039in}}%
\pgfusepath{clip}%
\pgfsetbuttcap%
\pgfsetroundjoin%
\definecolor{currentfill}{rgb}{0.280267,0.073417,0.397163}%
\pgfsetfillcolor{currentfill}%
\pgfsetfillopacity{0.700000}%
\pgfsetlinewidth{0.000000pt}%
\definecolor{currentstroke}{rgb}{0.000000,0.000000,0.000000}%
\pgfsetstrokecolor{currentstroke}%
\pgfsetdash{}{0pt}%
\pgfpathmoveto{\pgfqpoint{4.836840in}{2.386109in}}%
\pgfpathlineto{\pgfqpoint{4.849996in}{2.384088in}}%
\pgfpathlineto{\pgfqpoint{4.863160in}{2.382091in}}%
\pgfpathlineto{\pgfqpoint{4.876330in}{2.380118in}}%
\pgfpathlineto{\pgfqpoint{4.889508in}{2.378171in}}%
\pgfpathlineto{\pgfqpoint{4.882324in}{2.371466in}}%
\pgfpathlineto{\pgfqpoint{4.875135in}{2.364725in}}%
\pgfpathlineto{\pgfqpoint{4.867940in}{2.357946in}}%
\pgfpathlineto{\pgfqpoint{4.860739in}{2.351128in}}%
\pgfpathlineto{\pgfqpoint{4.847548in}{2.353039in}}%
\pgfpathlineto{\pgfqpoint{4.834364in}{2.354974in}}%
\pgfpathlineto{\pgfqpoint{4.821188in}{2.356934in}}%
\pgfpathlineto{\pgfqpoint{4.808018in}{2.358919in}}%
\pgfpathlineto{\pgfqpoint{4.815232in}{2.365769in}}%
\pgfpathlineto{\pgfqpoint{4.822441in}{2.372583in}}%
\pgfpathlineto{\pgfqpoint{4.829643in}{2.379363in}}%
\pgfpathlineto{\pgfqpoint{4.836840in}{2.386109in}}%
\pgfpathclose%
\pgfusepath{fill}%
\end{pgfscope}%
\begin{pgfscope}%
\pgfpathrectangle{\pgfqpoint{1.254980in}{0.150000in}}{\pgfqpoint{5.490039in}{5.490039in}}%
\pgfusepath{clip}%
\pgfsetbuttcap%
\pgfsetroundjoin%
\definecolor{currentfill}{rgb}{0.268510,0.009605,0.335427}%
\pgfsetfillcolor{currentfill}%
\pgfsetfillopacity{0.700000}%
\pgfsetlinewidth{0.000000pt}%
\definecolor{currentstroke}{rgb}{0.000000,0.000000,0.000000}%
\pgfsetstrokecolor{currentstroke}%
\pgfsetdash{}{0pt}%
\pgfpathmoveto{\pgfqpoint{3.975398in}{2.279865in}}%
\pgfpathlineto{\pgfqpoint{3.988338in}{2.276507in}}%
\pgfpathlineto{\pgfqpoint{4.001284in}{2.273178in}}%
\pgfpathlineto{\pgfqpoint{4.014236in}{2.269878in}}%
\pgfpathlineto{\pgfqpoint{4.027194in}{2.266606in}}%
\pgfpathlineto{\pgfqpoint{4.019685in}{2.259136in}}%
\pgfpathlineto{\pgfqpoint{4.012171in}{2.251662in}}%
\pgfpathlineto{\pgfqpoint{4.004650in}{2.244184in}}%
\pgfpathlineto{\pgfqpoint{3.997125in}{2.236705in}}%
\pgfpathlineto{\pgfqpoint{3.984155in}{2.240043in}}%
\pgfpathlineto{\pgfqpoint{3.971190in}{2.243409in}}%
\pgfpathlineto{\pgfqpoint{3.958232in}{2.246804in}}%
\pgfpathlineto{\pgfqpoint{3.945280in}{2.250227in}}%
\pgfpathlineto{\pgfqpoint{3.952817in}{2.257636in}}%
\pgfpathlineto{\pgfqpoint{3.960350in}{2.265046in}}%
\pgfpathlineto{\pgfqpoint{3.967876in}{2.272456in}}%
\pgfpathlineto{\pgfqpoint{3.975398in}{2.279865in}}%
\pgfpathclose%
\pgfusepath{fill}%
\end{pgfscope}%
\begin{pgfscope}%
\pgfpathrectangle{\pgfqpoint{1.254980in}{0.150000in}}{\pgfqpoint{5.490039in}{5.490039in}}%
\pgfusepath{clip}%
\pgfsetbuttcap%
\pgfsetroundjoin%
\definecolor{currentfill}{rgb}{0.267004,0.004874,0.329415}%
\pgfsetfillcolor{currentfill}%
\pgfsetfillopacity{0.700000}%
\pgfsetlinewidth{0.000000pt}%
\definecolor{currentstroke}{rgb}{0.000000,0.000000,0.000000}%
\pgfsetstrokecolor{currentstroke}%
\pgfsetdash{}{0pt}%
\pgfpathmoveto{\pgfqpoint{3.626399in}{2.269290in}}%
\pgfpathlineto{\pgfqpoint{3.639267in}{2.265093in}}%
\pgfpathlineto{\pgfqpoint{3.652140in}{2.260928in}}%
\pgfpathlineto{\pgfqpoint{3.665018in}{2.256795in}}%
\pgfpathlineto{\pgfqpoint{3.677902in}{2.252693in}}%
\pgfpathlineto{\pgfqpoint{3.670261in}{2.245766in}}%
\pgfpathlineto{\pgfqpoint{3.662615in}{2.238872in}}%
\pgfpathlineto{\pgfqpoint{3.654962in}{2.232012in}}%
\pgfpathlineto{\pgfqpoint{3.647303in}{2.225188in}}%
\pgfpathlineto{\pgfqpoint{3.634406in}{2.229394in}}%
\pgfpathlineto{\pgfqpoint{3.621514in}{2.233631in}}%
\pgfpathlineto{\pgfqpoint{3.608627in}{2.237900in}}%
\pgfpathlineto{\pgfqpoint{3.595746in}{2.242201in}}%
\pgfpathlineto{\pgfqpoint{3.603418in}{2.248916in}}%
\pgfpathlineto{\pgfqpoint{3.611085in}{2.255670in}}%
\pgfpathlineto{\pgfqpoint{3.618745in}{2.262463in}}%
\pgfpathlineto{\pgfqpoint{3.626399in}{2.269290in}}%
\pgfpathclose%
\pgfusepath{fill}%
\end{pgfscope}%
\begin{pgfscope}%
\pgfpathrectangle{\pgfqpoint{1.254980in}{0.150000in}}{\pgfqpoint{5.490039in}{5.490039in}}%
\pgfusepath{clip}%
\pgfsetbuttcap%
\pgfsetroundjoin%
\definecolor{currentfill}{rgb}{0.283229,0.120777,0.440584}%
\pgfsetfillcolor{currentfill}%
\pgfsetfillopacity{0.700000}%
\pgfsetlinewidth{0.000000pt}%
\definecolor{currentstroke}{rgb}{0.000000,0.000000,0.000000}%
\pgfsetstrokecolor{currentstroke}%
\pgfsetdash{}{0pt}%
\pgfpathmoveto{\pgfqpoint{5.482893in}{2.468678in}}%
\pgfpathlineto{\pgfqpoint{5.496224in}{2.467029in}}%
\pgfpathlineto{\pgfqpoint{5.509562in}{2.465404in}}%
\pgfpathlineto{\pgfqpoint{5.522908in}{2.463803in}}%
\pgfpathlineto{\pgfqpoint{5.536261in}{2.462225in}}%
\pgfpathlineto{\pgfqpoint{5.529357in}{2.456749in}}%
\pgfpathlineto{\pgfqpoint{5.522448in}{2.451273in}}%
\pgfpathlineto{\pgfqpoint{5.515532in}{2.445793in}}%
\pgfpathlineto{\pgfqpoint{5.508612in}{2.440306in}}%
\pgfpathlineto{\pgfqpoint{5.495240in}{2.441769in}}%
\pgfpathlineto{\pgfqpoint{5.481877in}{2.443257in}}%
\pgfpathlineto{\pgfqpoint{5.468521in}{2.444767in}}%
\pgfpathlineto{\pgfqpoint{5.455173in}{2.446301in}}%
\pgfpathlineto{\pgfqpoint{5.462111in}{2.451898in}}%
\pgfpathlineto{\pgfqpoint{5.469044in}{2.457491in}}%
\pgfpathlineto{\pgfqpoint{5.475971in}{2.463083in}}%
\pgfpathlineto{\pgfqpoint{5.482893in}{2.468678in}}%
\pgfpathclose%
\pgfusepath{fill}%
\end{pgfscope}%
\begin{pgfscope}%
\pgfpathrectangle{\pgfqpoint{1.254980in}{0.150000in}}{\pgfqpoint{5.490039in}{5.490039in}}%
\pgfusepath{clip}%
\pgfsetbuttcap%
\pgfsetroundjoin%
\definecolor{currentfill}{rgb}{0.282327,0.094955,0.417331}%
\pgfsetfillcolor{currentfill}%
\pgfsetfillopacity{0.700000}%
\pgfsetlinewidth{0.000000pt}%
\definecolor{currentstroke}{rgb}{0.000000,0.000000,0.000000}%
\pgfsetstrokecolor{currentstroke}%
\pgfsetdash{}{0pt}%
\pgfpathmoveto{\pgfqpoint{5.052222in}{2.415138in}}%
\pgfpathlineto{\pgfqpoint{5.065437in}{2.413298in}}%
\pgfpathlineto{\pgfqpoint{5.078659in}{2.411483in}}%
\pgfpathlineto{\pgfqpoint{5.091889in}{2.409691in}}%
\pgfpathlineto{\pgfqpoint{5.105126in}{2.407923in}}%
\pgfpathlineto{\pgfqpoint{5.098032in}{2.401641in}}%
\pgfpathlineto{\pgfqpoint{5.090932in}{2.395329in}}%
\pgfpathlineto{\pgfqpoint{5.083827in}{2.388985in}}%
\pgfpathlineto{\pgfqpoint{5.076715in}{2.382606in}}%
\pgfpathlineto{\pgfqpoint{5.063464in}{2.384311in}}%
\pgfpathlineto{\pgfqpoint{5.050220in}{2.386040in}}%
\pgfpathlineto{\pgfqpoint{5.036983in}{2.387793in}}%
\pgfpathlineto{\pgfqpoint{5.023754in}{2.389570in}}%
\pgfpathlineto{\pgfqpoint{5.030880in}{2.396006in}}%
\pgfpathlineto{\pgfqpoint{5.038000in}{2.402412in}}%
\pgfpathlineto{\pgfqpoint{5.045114in}{2.408788in}}%
\pgfpathlineto{\pgfqpoint{5.052222in}{2.415138in}}%
\pgfpathclose%
\pgfusepath{fill}%
\end{pgfscope}%
\begin{pgfscope}%
\pgfpathrectangle{\pgfqpoint{1.254980in}{0.150000in}}{\pgfqpoint{5.490039in}{5.490039in}}%
\pgfusepath{clip}%
\pgfsetbuttcap%
\pgfsetroundjoin%
\definecolor{currentfill}{rgb}{0.276022,0.044167,0.370164}%
\pgfsetfillcolor{currentfill}%
\pgfsetfillopacity{0.700000}%
\pgfsetlinewidth{0.000000pt}%
\definecolor{currentstroke}{rgb}{0.000000,0.000000,0.000000}%
\pgfsetstrokecolor{currentstroke}%
\pgfsetdash{}{0pt}%
\pgfpathmoveto{\pgfqpoint{3.091942in}{2.333834in}}%
\pgfpathlineto{\pgfqpoint{3.104728in}{2.328008in}}%
\pgfpathlineto{\pgfqpoint{3.117518in}{2.322222in}}%
\pgfpathlineto{\pgfqpoint{3.130311in}{2.316475in}}%
\pgfpathlineto{\pgfqpoint{3.143108in}{2.310767in}}%
\pgfpathlineto{\pgfqpoint{3.135231in}{2.305879in}}%
\pgfpathlineto{\pgfqpoint{3.127345in}{2.301096in}}%
\pgfpathlineto{\pgfqpoint{3.119450in}{2.296422in}}%
\pgfpathlineto{\pgfqpoint{3.111546in}{2.291860in}}%
\pgfpathlineto{\pgfqpoint{3.098730in}{2.297723in}}%
\pgfpathlineto{\pgfqpoint{3.085918in}{2.303626in}}%
\pgfpathlineto{\pgfqpoint{3.073110in}{2.309568in}}%
\pgfpathlineto{\pgfqpoint{3.060305in}{2.315550in}}%
\pgfpathlineto{\pgfqpoint{3.068228in}{2.319951in}}%
\pgfpathlineto{\pgfqpoint{3.076142in}{2.324468in}}%
\pgfpathlineto{\pgfqpoint{3.084046in}{2.329097in}}%
\pgfpathlineto{\pgfqpoint{3.091942in}{2.333834in}}%
\pgfpathclose%
\pgfusepath{fill}%
\end{pgfscope}%
\begin{pgfscope}%
\pgfpathrectangle{\pgfqpoint{1.254980in}{0.150000in}}{\pgfqpoint{5.490039in}{5.490039in}}%
\pgfusepath{clip}%
\pgfsetbuttcap%
\pgfsetroundjoin%
\definecolor{currentfill}{rgb}{0.283091,0.110553,0.431554}%
\pgfsetfillcolor{currentfill}%
\pgfsetfillopacity{0.700000}%
\pgfsetlinewidth{0.000000pt}%
\definecolor{currentstroke}{rgb}{0.000000,0.000000,0.000000}%
\pgfsetstrokecolor{currentstroke}%
\pgfsetdash{}{0pt}%
\pgfpathmoveto{\pgfqpoint{5.267583in}{2.442827in}}%
\pgfpathlineto{\pgfqpoint{5.280857in}{2.441111in}}%
\pgfpathlineto{\pgfqpoint{5.294137in}{2.439419in}}%
\pgfpathlineto{\pgfqpoint{5.307426in}{2.437751in}}%
\pgfpathlineto{\pgfqpoint{5.320722in}{2.436106in}}%
\pgfpathlineto{\pgfqpoint{5.313721in}{2.430249in}}%
\pgfpathlineto{\pgfqpoint{5.306715in}{2.424374in}}%
\pgfpathlineto{\pgfqpoint{5.299703in}{2.418478in}}%
\pgfpathlineto{\pgfqpoint{5.292686in}{2.412558in}}%
\pgfpathlineto{\pgfqpoint{5.279374in}{2.414114in}}%
\pgfpathlineto{\pgfqpoint{5.266070in}{2.415694in}}%
\pgfpathlineto{\pgfqpoint{5.252773in}{2.417297in}}%
\pgfpathlineto{\pgfqpoint{5.239484in}{2.418925in}}%
\pgfpathlineto{\pgfqpoint{5.246518in}{2.424928in}}%
\pgfpathlineto{\pgfqpoint{5.253545in}{2.430911in}}%
\pgfpathlineto{\pgfqpoint{5.260567in}{2.436877in}}%
\pgfpathlineto{\pgfqpoint{5.267583in}{2.442827in}}%
\pgfpathclose%
\pgfusepath{fill}%
\end{pgfscope}%
\begin{pgfscope}%
\pgfpathrectangle{\pgfqpoint{1.254980in}{0.150000in}}{\pgfqpoint{5.490039in}{5.490039in}}%
\pgfusepath{clip}%
\pgfsetbuttcap%
\pgfsetroundjoin%
\definecolor{currentfill}{rgb}{0.282656,0.100196,0.422160}%
\pgfsetfillcolor{currentfill}%
\pgfsetfillopacity{0.700000}%
\pgfsetlinewidth{0.000000pt}%
\definecolor{currentstroke}{rgb}{0.000000,0.000000,0.000000}%
\pgfsetstrokecolor{currentstroke}%
\pgfsetdash{}{0pt}%
\pgfpathmoveto{\pgfqpoint{2.772736in}{2.431156in}}%
\pgfpathlineto{\pgfqpoint{2.785497in}{2.424126in}}%
\pgfpathlineto{\pgfqpoint{2.798260in}{2.417143in}}%
\pgfpathlineto{\pgfqpoint{2.811026in}{2.410208in}}%
\pgfpathlineto{\pgfqpoint{2.823795in}{2.403321in}}%
\pgfpathlineto{\pgfqpoint{2.815741in}{2.400266in}}%
\pgfpathlineto{\pgfqpoint{2.807675in}{2.397365in}}%
\pgfpathlineto{\pgfqpoint{2.799597in}{2.394622in}}%
\pgfpathlineto{\pgfqpoint{2.791509in}{2.392042in}}%
\pgfpathlineto{\pgfqpoint{2.778717in}{2.399112in}}%
\pgfpathlineto{\pgfqpoint{2.765929in}{2.406229in}}%
\pgfpathlineto{\pgfqpoint{2.753143in}{2.413394in}}%
\pgfpathlineto{\pgfqpoint{2.740359in}{2.420607in}}%
\pgfpathlineto{\pgfqpoint{2.748471in}{2.423000in}}%
\pgfpathlineto{\pgfqpoint{2.756571in}{2.425559in}}%
\pgfpathlineto{\pgfqpoint{2.764659in}{2.428279in}}%
\pgfpathlineto{\pgfqpoint{2.772736in}{2.431156in}}%
\pgfpathclose%
\pgfusepath{fill}%
\end{pgfscope}%
\begin{pgfscope}%
\pgfpathrectangle{\pgfqpoint{1.254980in}{0.150000in}}{\pgfqpoint{5.490039in}{5.490039in}}%
\pgfusepath{clip}%
\pgfsetbuttcap%
\pgfsetroundjoin%
\definecolor{currentfill}{rgb}{0.267004,0.004874,0.329415}%
\pgfsetfillcolor{currentfill}%
\pgfsetfillopacity{0.700000}%
\pgfsetlinewidth{0.000000pt}%
\definecolor{currentstroke}{rgb}{0.000000,0.000000,0.000000}%
\pgfsetstrokecolor{currentstroke}%
\pgfsetdash{}{0pt}%
\pgfpathmoveto{\pgfqpoint{3.759939in}{2.264943in}}%
\pgfpathlineto{\pgfqpoint{3.772836in}{2.261087in}}%
\pgfpathlineto{\pgfqpoint{3.785739in}{2.257260in}}%
\pgfpathlineto{\pgfqpoint{3.798647in}{2.253464in}}%
\pgfpathlineto{\pgfqpoint{3.811561in}{2.249698in}}%
\pgfpathlineto{\pgfqpoint{3.803970in}{2.242492in}}%
\pgfpathlineto{\pgfqpoint{3.796373in}{2.235303in}}%
\pgfpathlineto{\pgfqpoint{3.788771in}{2.228133in}}%
\pgfpathlineto{\pgfqpoint{3.781163in}{2.220985in}}%
\pgfpathlineto{\pgfqpoint{3.768236in}{2.224842in}}%
\pgfpathlineto{\pgfqpoint{3.755315in}{2.228729in}}%
\pgfpathlineto{\pgfqpoint{3.742399in}{2.232647in}}%
\pgfpathlineto{\pgfqpoint{3.729489in}{2.236595in}}%
\pgfpathlineto{\pgfqpoint{3.737110in}{2.243647in}}%
\pgfpathlineto{\pgfqpoint{3.744726in}{2.250724in}}%
\pgfpathlineto{\pgfqpoint{3.752335in}{2.257823in}}%
\pgfpathlineto{\pgfqpoint{3.759939in}{2.264943in}}%
\pgfpathclose%
\pgfusepath{fill}%
\end{pgfscope}%
\begin{pgfscope}%
\pgfpathrectangle{\pgfqpoint{1.254980in}{0.150000in}}{\pgfqpoint{5.490039in}{5.490039in}}%
\pgfusepath{clip}%
\pgfsetbuttcap%
\pgfsetroundjoin%
\definecolor{currentfill}{rgb}{0.279566,0.067836,0.391917}%
\pgfsetfillcolor{currentfill}%
\pgfsetfillopacity{0.700000}%
\pgfsetlinewidth{0.000000pt}%
\definecolor{currentstroke}{rgb}{0.000000,0.000000,0.000000}%
\pgfsetstrokecolor{currentstroke}%
\pgfsetdash{}{0pt}%
\pgfpathmoveto{\pgfqpoint{2.957999in}{2.364884in}}%
\pgfpathlineto{\pgfqpoint{2.970775in}{2.358570in}}%
\pgfpathlineto{\pgfqpoint{2.983555in}{2.352299in}}%
\pgfpathlineto{\pgfqpoint{2.996338in}{2.346070in}}%
\pgfpathlineto{\pgfqpoint{3.009124in}{2.339883in}}%
\pgfpathlineto{\pgfqpoint{3.001173in}{2.335765in}}%
\pgfpathlineto{\pgfqpoint{2.993212in}{2.331774in}}%
\pgfpathlineto{\pgfqpoint{2.985241in}{2.327913in}}%
\pgfpathlineto{\pgfqpoint{2.977260in}{2.324188in}}%
\pgfpathlineto{\pgfqpoint{2.964454in}{2.330543in}}%
\pgfpathlineto{\pgfqpoint{2.951650in}{2.336941in}}%
\pgfpathlineto{\pgfqpoint{2.938851in}{2.343381in}}%
\pgfpathlineto{\pgfqpoint{2.926054in}{2.349864in}}%
\pgfpathlineto{\pgfqpoint{2.934056in}{2.353416in}}%
\pgfpathlineto{\pgfqpoint{2.942047in}{2.357106in}}%
\pgfpathlineto{\pgfqpoint{2.950028in}{2.360930in}}%
\pgfpathlineto{\pgfqpoint{2.957999in}{2.364884in}}%
\pgfpathclose%
\pgfusepath{fill}%
\end{pgfscope}%
\begin{pgfscope}%
\pgfpathrectangle{\pgfqpoint{1.254980in}{0.150000in}}{\pgfqpoint{5.490039in}{5.490039in}}%
\pgfusepath{clip}%
\pgfsetbuttcap%
\pgfsetroundjoin%
\definecolor{currentfill}{rgb}{0.273809,0.031497,0.358853}%
\pgfsetfillcolor{currentfill}%
\pgfsetfillopacity{0.700000}%
\pgfsetlinewidth{0.000000pt}%
\definecolor{currentstroke}{rgb}{0.000000,0.000000,0.000000}%
\pgfsetstrokecolor{currentstroke}%
\pgfsetdash{}{0pt}%
\pgfpathmoveto{\pgfqpoint{4.324471in}{2.308923in}}%
\pgfpathlineto{\pgfqpoint{4.337499in}{2.306239in}}%
\pgfpathlineto{\pgfqpoint{4.350534in}{2.303581in}}%
\pgfpathlineto{\pgfqpoint{4.363575in}{2.300950in}}%
\pgfpathlineto{\pgfqpoint{4.376623in}{2.298345in}}%
\pgfpathlineto{\pgfqpoint{4.369238in}{2.290879in}}%
\pgfpathlineto{\pgfqpoint{4.361848in}{2.283385in}}%
\pgfpathlineto{\pgfqpoint{4.354452in}{2.275863in}}%
\pgfpathlineto{\pgfqpoint{4.347051in}{2.268313in}}%
\pgfpathlineto{\pgfqpoint{4.333991in}{2.270945in}}%
\pgfpathlineto{\pgfqpoint{4.320939in}{2.273604in}}%
\pgfpathlineto{\pgfqpoint{4.307892in}{2.276289in}}%
\pgfpathlineto{\pgfqpoint{4.294852in}{2.279001in}}%
\pgfpathlineto{\pgfqpoint{4.302265in}{2.286519in}}%
\pgfpathlineto{\pgfqpoint{4.309673in}{2.294012in}}%
\pgfpathlineto{\pgfqpoint{4.317075in}{2.301480in}}%
\pgfpathlineto{\pgfqpoint{4.324471in}{2.308923in}}%
\pgfpathclose%
\pgfusepath{fill}%
\end{pgfscope}%
\begin{pgfscope}%
\pgfpathrectangle{\pgfqpoint{1.254980in}{0.150000in}}{\pgfqpoint{5.490039in}{5.490039in}}%
\pgfusepath{clip}%
\pgfsetbuttcap%
\pgfsetroundjoin%
\definecolor{currentfill}{rgb}{0.277018,0.050344,0.375715}%
\pgfsetfillcolor{currentfill}%
\pgfsetfillopacity{0.700000}%
\pgfsetlinewidth{0.000000pt}%
\definecolor{currentstroke}{rgb}{0.000000,0.000000,0.000000}%
\pgfsetstrokecolor{currentstroke}%
\pgfsetdash{}{0pt}%
\pgfpathmoveto{\pgfqpoint{4.539929in}{2.337293in}}%
\pgfpathlineto{\pgfqpoint{4.553013in}{2.334939in}}%
\pgfpathlineto{\pgfqpoint{4.566103in}{2.332611in}}%
\pgfpathlineto{\pgfqpoint{4.579200in}{2.330309in}}%
\pgfpathlineto{\pgfqpoint{4.592304in}{2.328032in}}%
\pgfpathlineto{\pgfqpoint{4.585000in}{2.320798in}}%
\pgfpathlineto{\pgfqpoint{4.577690in}{2.313528in}}%
\pgfpathlineto{\pgfqpoint{4.570374in}{2.306221in}}%
\pgfpathlineto{\pgfqpoint{4.563053in}{2.298877in}}%
\pgfpathlineto{\pgfqpoint{4.549937in}{2.301155in}}%
\pgfpathlineto{\pgfqpoint{4.536828in}{2.303460in}}%
\pgfpathlineto{\pgfqpoint{4.523726in}{2.305789in}}%
\pgfpathlineto{\pgfqpoint{4.510630in}{2.308145in}}%
\pgfpathlineto{\pgfqpoint{4.517963in}{2.315483in}}%
\pgfpathlineto{\pgfqpoint{4.525291in}{2.322786in}}%
\pgfpathlineto{\pgfqpoint{4.532613in}{2.330056in}}%
\pgfpathlineto{\pgfqpoint{4.539929in}{2.337293in}}%
\pgfpathclose%
\pgfusepath{fill}%
\end{pgfscope}%
\begin{pgfscope}%
\pgfpathrectangle{\pgfqpoint{1.254980in}{0.150000in}}{\pgfqpoint{5.490039in}{5.490039in}}%
\pgfusepath{clip}%
\pgfsetbuttcap%
\pgfsetroundjoin%
\definecolor{currentfill}{rgb}{0.269944,0.014625,0.341379}%
\pgfsetfillcolor{currentfill}%
\pgfsetfillopacity{0.700000}%
\pgfsetlinewidth{0.000000pt}%
\definecolor{currentstroke}{rgb}{0.000000,0.000000,0.000000}%
\pgfsetstrokecolor{currentstroke}%
\pgfsetdash{}{0pt}%
\pgfpathmoveto{\pgfqpoint{4.109021in}{2.283824in}}%
\pgfpathlineto{\pgfqpoint{4.121997in}{2.280746in}}%
\pgfpathlineto{\pgfqpoint{4.134979in}{2.277695in}}%
\pgfpathlineto{\pgfqpoint{4.147968in}{2.274672in}}%
\pgfpathlineto{\pgfqpoint{4.160963in}{2.271676in}}%
\pgfpathlineto{\pgfqpoint{4.153499in}{2.264138in}}%
\pgfpathlineto{\pgfqpoint{4.146030in}{2.256586in}}%
\pgfpathlineto{\pgfqpoint{4.138556in}{2.249019in}}%
\pgfpathlineto{\pgfqpoint{4.131076in}{2.241439in}}%
\pgfpathlineto{\pgfqpoint{4.118069in}{2.244488in}}%
\pgfpathlineto{\pgfqpoint{4.105069in}{2.247564in}}%
\pgfpathlineto{\pgfqpoint{4.092075in}{2.250668in}}%
\pgfpathlineto{\pgfqpoint{4.079086in}{2.253799in}}%
\pgfpathlineto{\pgfqpoint{4.086578in}{2.261321in}}%
\pgfpathlineto{\pgfqpoint{4.094064in}{2.268833in}}%
\pgfpathlineto{\pgfqpoint{4.101545in}{2.276335in}}%
\pgfpathlineto{\pgfqpoint{4.109021in}{2.283824in}}%
\pgfpathclose%
\pgfusepath{fill}%
\end{pgfscope}%
\begin{pgfscope}%
\pgfpathrectangle{\pgfqpoint{1.254980in}{0.150000in}}{\pgfqpoint{5.490039in}{5.490039in}}%
\pgfusepath{clip}%
\pgfsetbuttcap%
\pgfsetroundjoin%
\definecolor{currentfill}{rgb}{0.282290,0.145912,0.461510}%
\pgfsetfillcolor{currentfill}%
\pgfsetfillopacity{0.700000}%
\pgfsetlinewidth{0.000000pt}%
\definecolor{currentstroke}{rgb}{0.000000,0.000000,0.000000}%
\pgfsetstrokecolor{currentstroke}%
\pgfsetdash{}{0pt}%
\pgfpathmoveto{\pgfqpoint{5.832538in}{2.500574in}}%
\pgfpathlineto{\pgfqpoint{5.845967in}{2.498969in}}%
\pgfpathlineto{\pgfqpoint{5.859403in}{2.497386in}}%
\pgfpathlineto{\pgfqpoint{5.872848in}{2.495827in}}%
\pgfpathlineto{\pgfqpoint{5.886300in}{2.494291in}}%
\pgfpathlineto{\pgfqpoint{5.879550in}{2.489283in}}%
\pgfpathlineto{\pgfqpoint{5.872796in}{2.484311in}}%
\pgfpathlineto{\pgfqpoint{5.866038in}{2.479370in}}%
\pgfpathlineto{\pgfqpoint{5.859275in}{2.474456in}}%
\pgfpathlineto{\pgfqpoint{5.845802in}{2.475839in}}%
\pgfpathlineto{\pgfqpoint{5.832337in}{2.477245in}}%
\pgfpathlineto{\pgfqpoint{5.818880in}{2.478674in}}%
\pgfpathlineto{\pgfqpoint{5.805432in}{2.480126in}}%
\pgfpathlineto{\pgfqpoint{5.812215in}{2.485189in}}%
\pgfpathlineto{\pgfqpoint{5.818993in}{2.490282in}}%
\pgfpathlineto{\pgfqpoint{5.825768in}{2.495409in}}%
\pgfpathlineto{\pgfqpoint{5.832538in}{2.500574in}}%
\pgfpathclose%
\pgfusepath{fill}%
\end{pgfscope}%
\begin{pgfscope}%
\pgfpathrectangle{\pgfqpoint{1.254980in}{0.150000in}}{\pgfqpoint{5.490039in}{5.490039in}}%
\pgfusepath{clip}%
\pgfsetbuttcap%
\pgfsetroundjoin%
\definecolor{currentfill}{rgb}{0.279566,0.067836,0.391917}%
\pgfsetfillcolor{currentfill}%
\pgfsetfillopacity{0.700000}%
\pgfsetlinewidth{0.000000pt}%
\definecolor{currentstroke}{rgb}{0.000000,0.000000,0.000000}%
\pgfsetstrokecolor{currentstroke}%
\pgfsetdash{}{0pt}%
\pgfpathmoveto{\pgfqpoint{4.755411in}{2.367108in}}%
\pgfpathlineto{\pgfqpoint{4.768552in}{2.365023in}}%
\pgfpathlineto{\pgfqpoint{4.781700in}{2.362964in}}%
\pgfpathlineto{\pgfqpoint{4.794856in}{2.360929in}}%
\pgfpathlineto{\pgfqpoint{4.808018in}{2.358919in}}%
\pgfpathlineto{\pgfqpoint{4.800798in}{2.352032in}}%
\pgfpathlineto{\pgfqpoint{4.793572in}{2.345107in}}%
\pgfpathlineto{\pgfqpoint{4.786340in}{2.338142in}}%
\pgfpathlineto{\pgfqpoint{4.779102in}{2.331136in}}%
\pgfpathlineto{\pgfqpoint{4.765927in}{2.333121in}}%
\pgfpathlineto{\pgfqpoint{4.752759in}{2.335132in}}%
\pgfpathlineto{\pgfqpoint{4.739598in}{2.337167in}}%
\pgfpathlineto{\pgfqpoint{4.726444in}{2.339228in}}%
\pgfpathlineto{\pgfqpoint{4.733694in}{2.346253in}}%
\pgfpathlineto{\pgfqpoint{4.740939in}{2.353241in}}%
\pgfpathlineto{\pgfqpoint{4.748178in}{2.360192in}}%
\pgfpathlineto{\pgfqpoint{4.755411in}{2.367108in}}%
\pgfpathclose%
\pgfusepath{fill}%
\end{pgfscope}%
\begin{pgfscope}%
\pgfpathrectangle{\pgfqpoint{1.254980in}{0.150000in}}{\pgfqpoint{5.490039in}{5.490039in}}%
\pgfusepath{clip}%
\pgfsetbuttcap%
\pgfsetroundjoin%
\definecolor{currentfill}{rgb}{0.281924,0.089666,0.412415}%
\pgfsetfillcolor{currentfill}%
\pgfsetfillopacity{0.700000}%
\pgfsetlinewidth{0.000000pt}%
\definecolor{currentstroke}{rgb}{0.000000,0.000000,0.000000}%
\pgfsetstrokecolor{currentstroke}%
\pgfsetdash{}{0pt}%
\pgfpathmoveto{\pgfqpoint{4.970910in}{2.396923in}}%
\pgfpathlineto{\pgfqpoint{4.984110in}{2.395048in}}%
\pgfpathlineto{\pgfqpoint{4.997317in}{2.393198in}}%
\pgfpathlineto{\pgfqpoint{5.010532in}{2.391372in}}%
\pgfpathlineto{\pgfqpoint{5.023754in}{2.389570in}}%
\pgfpathlineto{\pgfqpoint{5.016622in}{2.383101in}}%
\pgfpathlineto{\pgfqpoint{5.009484in}{2.376596in}}%
\pgfpathlineto{\pgfqpoint{5.002340in}{2.370054in}}%
\pgfpathlineto{\pgfqpoint{4.995190in}{2.363474in}}%
\pgfpathlineto{\pgfqpoint{4.981954in}{2.365225in}}%
\pgfpathlineto{\pgfqpoint{4.968725in}{2.367001in}}%
\pgfpathlineto{\pgfqpoint{4.955504in}{2.368801in}}%
\pgfpathlineto{\pgfqpoint{4.942291in}{2.370626in}}%
\pgfpathlineto{\pgfqpoint{4.949454in}{2.377252in}}%
\pgfpathlineto{\pgfqpoint{4.956612in}{2.383842in}}%
\pgfpathlineto{\pgfqpoint{4.963764in}{2.390398in}}%
\pgfpathlineto{\pgfqpoint{4.970910in}{2.396923in}}%
\pgfpathclose%
\pgfusepath{fill}%
\end{pgfscope}%
\begin{pgfscope}%
\pgfpathrectangle{\pgfqpoint{1.254980in}{0.150000in}}{\pgfqpoint{5.490039in}{5.490039in}}%
\pgfusepath{clip}%
\pgfsetbuttcap%
\pgfsetroundjoin%
\definecolor{currentfill}{rgb}{0.283072,0.130895,0.449241}%
\pgfsetfillcolor{currentfill}%
\pgfsetfillopacity{0.700000}%
\pgfsetlinewidth{0.000000pt}%
\definecolor{currentstroke}{rgb}{0.000000,0.000000,0.000000}%
\pgfsetstrokecolor{currentstroke}%
\pgfsetdash{}{0pt}%
\pgfpathmoveto{\pgfqpoint{5.617245in}{2.477614in}}%
\pgfpathlineto{\pgfqpoint{5.630619in}{2.476025in}}%
\pgfpathlineto{\pgfqpoint{5.644000in}{2.474459in}}%
\pgfpathlineto{\pgfqpoint{5.657390in}{2.472916in}}%
\pgfpathlineto{\pgfqpoint{5.670787in}{2.471397in}}%
\pgfpathlineto{\pgfqpoint{5.663941in}{2.466147in}}%
\pgfpathlineto{\pgfqpoint{5.657089in}{2.460907in}}%
\pgfpathlineto{\pgfqpoint{5.650233in}{2.455672in}}%
\pgfpathlineto{\pgfqpoint{5.643370in}{2.450438in}}%
\pgfpathlineto{\pgfqpoint{5.629954in}{2.451830in}}%
\pgfpathlineto{\pgfqpoint{5.616546in}{2.453245in}}%
\pgfpathlineto{\pgfqpoint{5.603146in}{2.454683in}}%
\pgfpathlineto{\pgfqpoint{5.589753in}{2.456145in}}%
\pgfpathlineto{\pgfqpoint{5.596634in}{2.461502in}}%
\pgfpathlineto{\pgfqpoint{5.603509in}{2.466863in}}%
\pgfpathlineto{\pgfqpoint{5.610379in}{2.472232in}}%
\pgfpathlineto{\pgfqpoint{5.617245in}{2.477614in}}%
\pgfpathclose%
\pgfusepath{fill}%
\end{pgfscope}%
\begin{pgfscope}%
\pgfpathrectangle{\pgfqpoint{1.254980in}{0.150000in}}{\pgfqpoint{5.490039in}{5.490039in}}%
\pgfusepath{clip}%
\pgfsetbuttcap%
\pgfsetroundjoin%
\definecolor{currentfill}{rgb}{0.267004,0.004874,0.329415}%
\pgfsetfillcolor{currentfill}%
\pgfsetfillopacity{0.700000}%
\pgfsetlinewidth{0.000000pt}%
\definecolor{currentstroke}{rgb}{0.000000,0.000000,0.000000}%
\pgfsetstrokecolor{currentstroke}%
\pgfsetdash{}{0pt}%
\pgfpathmoveto{\pgfqpoint{3.893528in}{2.264210in}}%
\pgfpathlineto{\pgfqpoint{3.906457in}{2.260671in}}%
\pgfpathlineto{\pgfqpoint{3.919392in}{2.257161in}}%
\pgfpathlineto{\pgfqpoint{3.932333in}{2.253680in}}%
\pgfpathlineto{\pgfqpoint{3.945280in}{2.250227in}}%
\pgfpathlineto{\pgfqpoint{3.937736in}{2.242821in}}%
\pgfpathlineto{\pgfqpoint{3.930187in}{2.235420in}}%
\pgfpathlineto{\pgfqpoint{3.922633in}{2.228024in}}%
\pgfpathlineto{\pgfqpoint{3.915073in}{2.220635in}}%
\pgfpathlineto{\pgfqpoint{3.902114in}{2.224166in}}%
\pgfpathlineto{\pgfqpoint{3.889161in}{2.227726in}}%
\pgfpathlineto{\pgfqpoint{3.876213in}{2.231314in}}%
\pgfpathlineto{\pgfqpoint{3.863272in}{2.234932in}}%
\pgfpathlineto{\pgfqpoint{3.870844in}{2.242237in}}%
\pgfpathlineto{\pgfqpoint{3.878411in}{2.249553in}}%
\pgfpathlineto{\pgfqpoint{3.885972in}{2.256878in}}%
\pgfpathlineto{\pgfqpoint{3.893528in}{2.264210in}}%
\pgfpathclose%
\pgfusepath{fill}%
\end{pgfscope}%
\begin{pgfscope}%
\pgfpathrectangle{\pgfqpoint{1.254980in}{0.150000in}}{\pgfqpoint{5.490039in}{5.490039in}}%
\pgfusepath{clip}%
\pgfsetbuttcap%
\pgfsetroundjoin%
\definecolor{currentfill}{rgb}{0.282910,0.105393,0.426902}%
\pgfsetfillcolor{currentfill}%
\pgfsetfillopacity{0.700000}%
\pgfsetlinewidth{0.000000pt}%
\definecolor{currentstroke}{rgb}{0.000000,0.000000,0.000000}%
\pgfsetstrokecolor{currentstroke}%
\pgfsetdash{}{0pt}%
\pgfpathmoveto{\pgfqpoint{5.186403in}{2.425673in}}%
\pgfpathlineto{\pgfqpoint{5.199662in}{2.423950in}}%
\pgfpathlineto{\pgfqpoint{5.212928in}{2.422251in}}%
\pgfpathlineto{\pgfqpoint{5.226202in}{2.420576in}}%
\pgfpathlineto{\pgfqpoint{5.239484in}{2.418925in}}%
\pgfpathlineto{\pgfqpoint{5.232444in}{2.412898in}}%
\pgfpathlineto{\pgfqpoint{5.225399in}{2.406845in}}%
\pgfpathlineto{\pgfqpoint{5.218347in}{2.400763in}}%
\pgfpathlineto{\pgfqpoint{5.211289in}{2.394650in}}%
\pgfpathlineto{\pgfqpoint{5.197993in}{2.396225in}}%
\pgfpathlineto{\pgfqpoint{5.184703in}{2.397824in}}%
\pgfpathlineto{\pgfqpoint{5.171422in}{2.399447in}}%
\pgfpathlineto{\pgfqpoint{5.158148in}{2.401095in}}%
\pgfpathlineto{\pgfqpoint{5.165220in}{2.407279in}}%
\pgfpathlineto{\pgfqpoint{5.172287in}{2.413435in}}%
\pgfpathlineto{\pgfqpoint{5.179348in}{2.419565in}}%
\pgfpathlineto{\pgfqpoint{5.186403in}{2.425673in}}%
\pgfpathclose%
\pgfusepath{fill}%
\end{pgfscope}%
\begin{pgfscope}%
\pgfpathrectangle{\pgfqpoint{1.254980in}{0.150000in}}{\pgfqpoint{5.490039in}{5.490039in}}%
\pgfusepath{clip}%
\pgfsetbuttcap%
\pgfsetroundjoin%
\definecolor{currentfill}{rgb}{0.283229,0.120777,0.440584}%
\pgfsetfillcolor{currentfill}%
\pgfsetfillopacity{0.700000}%
\pgfsetlinewidth{0.000000pt}%
\definecolor{currentstroke}{rgb}{0.000000,0.000000,0.000000}%
\pgfsetstrokecolor{currentstroke}%
\pgfsetdash{}{0pt}%
\pgfpathmoveto{\pgfqpoint{5.401858in}{2.452672in}}%
\pgfpathlineto{\pgfqpoint{5.415176in}{2.451044in}}%
\pgfpathlineto{\pgfqpoint{5.428500in}{2.449439in}}%
\pgfpathlineto{\pgfqpoint{5.441833in}{2.447858in}}%
\pgfpathlineto{\pgfqpoint{5.455173in}{2.446301in}}%
\pgfpathlineto{\pgfqpoint{5.448229in}{2.440696in}}%
\pgfpathlineto{\pgfqpoint{5.441280in}{2.435080in}}%
\pgfpathlineto{\pgfqpoint{5.434325in}{2.429449in}}%
\pgfpathlineto{\pgfqpoint{5.427363in}{2.423800in}}%
\pgfpathlineto{\pgfqpoint{5.414006in}{2.425256in}}%
\pgfpathlineto{\pgfqpoint{5.400657in}{2.426735in}}%
\pgfpathlineto{\pgfqpoint{5.387315in}{2.428238in}}%
\pgfpathlineto{\pgfqpoint{5.373981in}{2.429764in}}%
\pgfpathlineto{\pgfqpoint{5.380959in}{2.435510in}}%
\pgfpathlineto{\pgfqpoint{5.387931in}{2.441241in}}%
\pgfpathlineto{\pgfqpoint{5.394898in}{2.446961in}}%
\pgfpathlineto{\pgfqpoint{5.401858in}{2.452672in}}%
\pgfpathclose%
\pgfusepath{fill}%
\end{pgfscope}%
\begin{pgfscope}%
\pgfpathrectangle{\pgfqpoint{1.254980in}{0.150000in}}{\pgfqpoint{5.490039in}{5.490039in}}%
\pgfusepath{clip}%
\pgfsetbuttcap%
\pgfsetroundjoin%
\definecolor{currentfill}{rgb}{0.268510,0.009605,0.335427}%
\pgfsetfillcolor{currentfill}%
\pgfsetfillopacity{0.700000}%
\pgfsetlinewidth{0.000000pt}%
\definecolor{currentstroke}{rgb}{0.000000,0.000000,0.000000}%
\pgfsetstrokecolor{currentstroke}%
\pgfsetdash{}{0pt}%
\pgfpathmoveto{\pgfqpoint{3.410625in}{2.271346in}}%
\pgfpathlineto{\pgfqpoint{3.423462in}{2.266532in}}%
\pgfpathlineto{\pgfqpoint{3.436305in}{2.261753in}}%
\pgfpathlineto{\pgfqpoint{3.449151in}{2.257007in}}%
\pgfpathlineto{\pgfqpoint{3.462003in}{2.252296in}}%
\pgfpathlineto{\pgfqpoint{3.454269in}{2.246068in}}%
\pgfpathlineto{\pgfqpoint{3.446527in}{2.239903in}}%
\pgfpathlineto{\pgfqpoint{3.438779in}{2.233803in}}%
\pgfpathlineto{\pgfqpoint{3.431024in}{2.227772in}}%
\pgfpathlineto{\pgfqpoint{3.418157in}{2.232614in}}%
\pgfpathlineto{\pgfqpoint{3.405295in}{2.237489in}}%
\pgfpathlineto{\pgfqpoint{3.392437in}{2.242398in}}%
\pgfpathlineto{\pgfqpoint{3.379584in}{2.247341in}}%
\pgfpathlineto{\pgfqpoint{3.387355in}{2.253238in}}%
\pgfpathlineto{\pgfqpoint{3.395119in}{2.259206in}}%
\pgfpathlineto{\pgfqpoint{3.402875in}{2.265244in}}%
\pgfpathlineto{\pgfqpoint{3.410625in}{2.271346in}}%
\pgfpathclose%
\pgfusepath{fill}%
\end{pgfscope}%
\begin{pgfscope}%
\pgfpathrectangle{\pgfqpoint{1.254980in}{0.150000in}}{\pgfqpoint{5.490039in}{5.490039in}}%
\pgfusepath{clip}%
\pgfsetbuttcap%
\pgfsetroundjoin%
\definecolor{currentfill}{rgb}{0.271305,0.019942,0.347269}%
\pgfsetfillcolor{currentfill}%
\pgfsetfillopacity{0.700000}%
\pgfsetlinewidth{0.000000pt}%
\definecolor{currentstroke}{rgb}{0.000000,0.000000,0.000000}%
\pgfsetstrokecolor{currentstroke}%
\pgfsetdash{}{0pt}%
\pgfpathmoveto{\pgfqpoint{3.276922in}{2.288149in}}%
\pgfpathlineto{\pgfqpoint{3.289739in}{2.282923in}}%
\pgfpathlineto{\pgfqpoint{3.302561in}{2.277733in}}%
\pgfpathlineto{\pgfqpoint{3.315387in}{2.272579in}}%
\pgfpathlineto{\pgfqpoint{3.328217in}{2.267461in}}%
\pgfpathlineto{\pgfqpoint{3.320423in}{2.261776in}}%
\pgfpathlineto{\pgfqpoint{3.312621in}{2.256173in}}%
\pgfpathlineto{\pgfqpoint{3.304812in}{2.250654in}}%
\pgfpathlineto{\pgfqpoint{3.296994in}{2.245224in}}%
\pgfpathlineto{\pgfqpoint{3.284147in}{2.250484in}}%
\pgfpathlineto{\pgfqpoint{3.271305in}{2.255780in}}%
\pgfpathlineto{\pgfqpoint{3.258466in}{2.261112in}}%
\pgfpathlineto{\pgfqpoint{3.245632in}{2.266481in}}%
\pgfpathlineto{\pgfqpoint{3.253466in}{2.271764in}}%
\pgfpathlineto{\pgfqpoint{3.261293in}{2.277139in}}%
\pgfpathlineto{\pgfqpoint{3.269111in}{2.282602in}}%
\pgfpathlineto{\pgfqpoint{3.276922in}{2.288149in}}%
\pgfpathclose%
\pgfusepath{fill}%
\end{pgfscope}%
\begin{pgfscope}%
\pgfpathrectangle{\pgfqpoint{1.254980in}{0.150000in}}{\pgfqpoint{5.490039in}{5.490039in}}%
\pgfusepath{clip}%
\pgfsetbuttcap%
\pgfsetroundjoin%
\definecolor{currentfill}{rgb}{0.267004,0.004874,0.329415}%
\pgfsetfillcolor{currentfill}%
\pgfsetfillopacity{0.700000}%
\pgfsetlinewidth{0.000000pt}%
\definecolor{currentstroke}{rgb}{0.000000,0.000000,0.000000}%
\pgfsetstrokecolor{currentstroke}%
\pgfsetdash{}{0pt}%
\pgfpathmoveto{\pgfqpoint{3.544270in}{2.259725in}}%
\pgfpathlineto{\pgfqpoint{3.557131in}{2.255295in}}%
\pgfpathlineto{\pgfqpoint{3.569998in}{2.250898in}}%
\pgfpathlineto{\pgfqpoint{3.582869in}{2.246534in}}%
\pgfpathlineto{\pgfqpoint{3.595746in}{2.242201in}}%
\pgfpathlineto{\pgfqpoint{3.588067in}{2.235529in}}%
\pgfpathlineto{\pgfqpoint{3.580381in}{2.228901in}}%
\pgfpathlineto{\pgfqpoint{3.572690in}{2.222321in}}%
\pgfpathlineto{\pgfqpoint{3.564991in}{2.215791in}}%
\pgfpathlineto{\pgfqpoint{3.552101in}{2.220241in}}%
\pgfpathlineto{\pgfqpoint{3.539215in}{2.224722in}}%
\pgfpathlineto{\pgfqpoint{3.526334in}{2.229236in}}%
\pgfpathlineto{\pgfqpoint{3.513458in}{2.233782in}}%
\pgfpathlineto{\pgfqpoint{3.521171in}{2.240190in}}%
\pgfpathlineto{\pgfqpoint{3.528877in}{2.246652in}}%
\pgfpathlineto{\pgfqpoint{3.536577in}{2.253164in}}%
\pgfpathlineto{\pgfqpoint{3.544270in}{2.259725in}}%
\pgfpathclose%
\pgfusepath{fill}%
\end{pgfscope}%
\begin{pgfscope}%
\pgfpathrectangle{\pgfqpoint{1.254980in}{0.150000in}}{\pgfqpoint{5.490039in}{5.490039in}}%
\pgfusepath{clip}%
\pgfsetbuttcap%
\pgfsetroundjoin%
\definecolor{currentfill}{rgb}{0.274952,0.037752,0.364543}%
\pgfsetfillcolor{currentfill}%
\pgfsetfillopacity{0.700000}%
\pgfsetlinewidth{0.000000pt}%
\definecolor{currentstroke}{rgb}{0.000000,0.000000,0.000000}%
\pgfsetstrokecolor{currentstroke}%
\pgfsetdash{}{0pt}%
\pgfpathmoveto{\pgfqpoint{3.143108in}{2.310767in}}%
\pgfpathlineto{\pgfqpoint{3.155910in}{2.305098in}}%
\pgfpathlineto{\pgfqpoint{3.168715in}{2.299468in}}%
\pgfpathlineto{\pgfqpoint{3.181524in}{2.293876in}}%
\pgfpathlineto{\pgfqpoint{3.194338in}{2.288322in}}%
\pgfpathlineto{\pgfqpoint{3.186478in}{2.283284in}}%
\pgfpathlineto{\pgfqpoint{3.178610in}{2.278347in}}%
\pgfpathlineto{\pgfqpoint{3.170733in}{2.273516in}}%
\pgfpathlineto{\pgfqpoint{3.162848in}{2.268794in}}%
\pgfpathlineto{\pgfqpoint{3.150016in}{2.274503in}}%
\pgfpathlineto{\pgfqpoint{3.137189in}{2.280251in}}%
\pgfpathlineto{\pgfqpoint{3.124365in}{2.286036in}}%
\pgfpathlineto{\pgfqpoint{3.111546in}{2.291860in}}%
\pgfpathlineto{\pgfqpoint{3.119450in}{2.296422in}}%
\pgfpathlineto{\pgfqpoint{3.127345in}{2.301096in}}%
\pgfpathlineto{\pgfqpoint{3.135231in}{2.305879in}}%
\pgfpathlineto{\pgfqpoint{3.143108in}{2.310767in}}%
\pgfpathclose%
\pgfusepath{fill}%
\end{pgfscope}%
\begin{pgfscope}%
\pgfpathrectangle{\pgfqpoint{1.254980in}{0.150000in}}{\pgfqpoint{5.490039in}{5.490039in}}%
\pgfusepath{clip}%
\pgfsetbuttcap%
\pgfsetroundjoin%
\definecolor{currentfill}{rgb}{0.282327,0.094955,0.417331}%
\pgfsetfillcolor{currentfill}%
\pgfsetfillopacity{0.700000}%
\pgfsetlinewidth{0.000000pt}%
\definecolor{currentstroke}{rgb}{0.000000,0.000000,0.000000}%
\pgfsetstrokecolor{currentstroke}%
\pgfsetdash{}{0pt}%
\pgfpathmoveto{\pgfqpoint{2.823795in}{2.403321in}}%
\pgfpathlineto{\pgfqpoint{2.836567in}{2.396480in}}%
\pgfpathlineto{\pgfqpoint{2.849342in}{2.389685in}}%
\pgfpathlineto{\pgfqpoint{2.862120in}{2.382936in}}%
\pgfpathlineto{\pgfqpoint{2.874900in}{2.376233in}}%
\pgfpathlineto{\pgfqpoint{2.866867in}{2.373001in}}%
\pgfpathlineto{\pgfqpoint{2.858823in}{2.369919in}}%
\pgfpathlineto{\pgfqpoint{2.850768in}{2.366992in}}%
\pgfpathlineto{\pgfqpoint{2.842702in}{2.364225in}}%
\pgfpathlineto{\pgfqpoint{2.829899in}{2.371110in}}%
\pgfpathlineto{\pgfqpoint{2.817100in}{2.378041in}}%
\pgfpathlineto{\pgfqpoint{2.804303in}{2.385018in}}%
\pgfpathlineto{\pgfqpoint{2.791509in}{2.392042in}}%
\pgfpathlineto{\pgfqpoint{2.799597in}{2.394622in}}%
\pgfpathlineto{\pgfqpoint{2.807675in}{2.397365in}}%
\pgfpathlineto{\pgfqpoint{2.815741in}{2.400266in}}%
\pgfpathlineto{\pgfqpoint{2.823795in}{2.403321in}}%
\pgfpathclose%
\pgfusepath{fill}%
\end{pgfscope}%
\begin{pgfscope}%
\pgfpathrectangle{\pgfqpoint{1.254980in}{0.150000in}}{\pgfqpoint{5.490039in}{5.490039in}}%
\pgfusepath{clip}%
\pgfsetbuttcap%
\pgfsetroundjoin%
\definecolor{currentfill}{rgb}{0.272594,0.025563,0.353093}%
\pgfsetfillcolor{currentfill}%
\pgfsetfillopacity{0.700000}%
\pgfsetlinewidth{0.000000pt}%
\definecolor{currentstroke}{rgb}{0.000000,0.000000,0.000000}%
\pgfsetstrokecolor{currentstroke}%
\pgfsetdash{}{0pt}%
\pgfpathmoveto{\pgfqpoint{4.242757in}{2.290116in}}%
\pgfpathlineto{\pgfqpoint{4.255771in}{2.287297in}}%
\pgfpathlineto{\pgfqpoint{4.268792in}{2.284505in}}%
\pgfpathlineto{\pgfqpoint{4.281819in}{2.281739in}}%
\pgfpathlineto{\pgfqpoint{4.294852in}{2.279001in}}%
\pgfpathlineto{\pgfqpoint{4.287434in}{2.271459in}}%
\pgfpathlineto{\pgfqpoint{4.280010in}{2.263893in}}%
\pgfpathlineto{\pgfqpoint{4.272581in}{2.256303in}}%
\pgfpathlineto{\pgfqpoint{4.265146in}{2.248691in}}%
\pgfpathlineto{\pgfqpoint{4.252101in}{2.251470in}}%
\pgfpathlineto{\pgfqpoint{4.239062in}{2.254275in}}%
\pgfpathlineto{\pgfqpoint{4.226030in}{2.257108in}}%
\pgfpathlineto{\pgfqpoint{4.213004in}{2.259967in}}%
\pgfpathlineto{\pgfqpoint{4.220450in}{2.267534in}}%
\pgfpathlineto{\pgfqpoint{4.227891in}{2.275082in}}%
\pgfpathlineto{\pgfqpoint{4.235327in}{2.282609in}}%
\pgfpathlineto{\pgfqpoint{4.242757in}{2.290116in}}%
\pgfpathclose%
\pgfusepath{fill}%
\end{pgfscope}%
\begin{pgfscope}%
\pgfpathrectangle{\pgfqpoint{1.254980in}{0.150000in}}{\pgfqpoint{5.490039in}{5.490039in}}%
\pgfusepath{clip}%
\pgfsetbuttcap%
\pgfsetroundjoin%
\definecolor{currentfill}{rgb}{0.276022,0.044167,0.370164}%
\pgfsetfillcolor{currentfill}%
\pgfsetfillopacity{0.700000}%
\pgfsetlinewidth{0.000000pt}%
\definecolor{currentstroke}{rgb}{0.000000,0.000000,0.000000}%
\pgfsetstrokecolor{currentstroke}%
\pgfsetdash{}{0pt}%
\pgfpathmoveto{\pgfqpoint{4.458315in}{2.317826in}}%
\pgfpathlineto{\pgfqpoint{4.471384in}{2.315367in}}%
\pgfpathlineto{\pgfqpoint{4.484459in}{2.312934in}}%
\pgfpathlineto{\pgfqpoint{4.497541in}{2.310526in}}%
\pgfpathlineto{\pgfqpoint{4.510630in}{2.308145in}}%
\pgfpathlineto{\pgfqpoint{4.503291in}{2.300773in}}%
\pgfpathlineto{\pgfqpoint{4.495947in}{2.293367in}}%
\pgfpathlineto{\pgfqpoint{4.488597in}{2.285926in}}%
\pgfpathlineto{\pgfqpoint{4.481241in}{2.278450in}}%
\pgfpathlineto{\pgfqpoint{4.468141in}{2.280845in}}%
\pgfpathlineto{\pgfqpoint{4.455047in}{2.283267in}}%
\pgfpathlineto{\pgfqpoint{4.441959in}{2.285715in}}%
\pgfpathlineto{\pgfqpoint{4.428879in}{2.288189in}}%
\pgfpathlineto{\pgfqpoint{4.436246in}{2.295645in}}%
\pgfpathlineto{\pgfqpoint{4.443608in}{2.303070in}}%
\pgfpathlineto{\pgfqpoint{4.450964in}{2.310464in}}%
\pgfpathlineto{\pgfqpoint{4.458315in}{2.317826in}}%
\pgfpathclose%
\pgfusepath{fill}%
\end{pgfscope}%
\begin{pgfscope}%
\pgfpathrectangle{\pgfqpoint{1.254980in}{0.150000in}}{\pgfqpoint{5.490039in}{5.490039in}}%
\pgfusepath{clip}%
\pgfsetbuttcap%
\pgfsetroundjoin%
\definecolor{currentfill}{rgb}{0.267004,0.004874,0.329415}%
\pgfsetfillcolor{currentfill}%
\pgfsetfillopacity{0.700000}%
\pgfsetlinewidth{0.000000pt}%
\definecolor{currentstroke}{rgb}{0.000000,0.000000,0.000000}%
\pgfsetstrokecolor{currentstroke}%
\pgfsetdash{}{0pt}%
\pgfpathmoveto{\pgfqpoint{3.677902in}{2.252693in}}%
\pgfpathlineto{\pgfqpoint{3.690791in}{2.248622in}}%
\pgfpathlineto{\pgfqpoint{3.703685in}{2.244582in}}%
\pgfpathlineto{\pgfqpoint{3.716584in}{2.240573in}}%
\pgfpathlineto{\pgfqpoint{3.729489in}{2.236595in}}%
\pgfpathlineto{\pgfqpoint{3.721862in}{2.229569in}}%
\pgfpathlineto{\pgfqpoint{3.714229in}{2.222572in}}%
\pgfpathlineto{\pgfqpoint{3.706590in}{2.215607in}}%
\pgfpathlineto{\pgfqpoint{3.698945in}{2.208674in}}%
\pgfpathlineto{\pgfqpoint{3.686026in}{2.212757in}}%
\pgfpathlineto{\pgfqpoint{3.673113in}{2.216870in}}%
\pgfpathlineto{\pgfqpoint{3.660206in}{2.221013in}}%
\pgfpathlineto{\pgfqpoint{3.647303in}{2.225188in}}%
\pgfpathlineto{\pgfqpoint{3.654962in}{2.232012in}}%
\pgfpathlineto{\pgfqpoint{3.662615in}{2.238872in}}%
\pgfpathlineto{\pgfqpoint{3.670261in}{2.245766in}}%
\pgfpathlineto{\pgfqpoint{3.677902in}{2.252693in}}%
\pgfpathclose%
\pgfusepath{fill}%
\end{pgfscope}%
\begin{pgfscope}%
\pgfpathrectangle{\pgfqpoint{1.254980in}{0.150000in}}{\pgfqpoint{5.490039in}{5.490039in}}%
\pgfusepath{clip}%
\pgfsetbuttcap%
\pgfsetroundjoin%
\definecolor{currentfill}{rgb}{0.281887,0.150881,0.465405}%
\pgfsetfillcolor{currentfill}%
\pgfsetfillopacity{0.700000}%
\pgfsetlinewidth{0.000000pt}%
\definecolor{currentstroke}{rgb}{0.000000,0.000000,0.000000}%
\pgfsetstrokecolor{currentstroke}%
\pgfsetdash{}{0pt}%
\pgfpathmoveto{\pgfqpoint{5.967065in}{2.508194in}}%
\pgfpathlineto{\pgfqpoint{5.980536in}{2.506606in}}%
\pgfpathlineto{\pgfqpoint{5.994015in}{2.505041in}}%
\pgfpathlineto{\pgfqpoint{6.007501in}{2.503499in}}%
\pgfpathlineto{\pgfqpoint{6.020996in}{2.501979in}}%
\pgfpathlineto{\pgfqpoint{6.014304in}{2.497121in}}%
\pgfpathlineto{\pgfqpoint{6.007609in}{2.492312in}}%
\pgfpathlineto{\pgfqpoint{6.000910in}{2.487548in}}%
\pgfpathlineto{\pgfqpoint{5.994207in}{2.482823in}}%
\pgfpathlineto{\pgfqpoint{5.980690in}{2.484176in}}%
\pgfpathlineto{\pgfqpoint{5.967182in}{2.485553in}}%
\pgfpathlineto{\pgfqpoint{5.953681in}{2.486952in}}%
\pgfpathlineto{\pgfqpoint{5.940189in}{2.488374in}}%
\pgfpathlineto{\pgfqpoint{5.946914in}{2.493260in}}%
\pgfpathlineto{\pgfqpoint{5.953634in}{2.498189in}}%
\pgfpathlineto{\pgfqpoint{5.960351in}{2.503166in}}%
\pgfpathlineto{\pgfqpoint{5.967065in}{2.508194in}}%
\pgfpathclose%
\pgfusepath{fill}%
\end{pgfscope}%
\begin{pgfscope}%
\pgfpathrectangle{\pgfqpoint{1.254980in}{0.150000in}}{\pgfqpoint{5.490039in}{5.490039in}}%
\pgfusepath{clip}%
\pgfsetbuttcap%
\pgfsetroundjoin%
\definecolor{currentfill}{rgb}{0.268510,0.009605,0.335427}%
\pgfsetfillcolor{currentfill}%
\pgfsetfillopacity{0.700000}%
\pgfsetlinewidth{0.000000pt}%
\definecolor{currentstroke}{rgb}{0.000000,0.000000,0.000000}%
\pgfsetstrokecolor{currentstroke}%
\pgfsetdash{}{0pt}%
\pgfpathmoveto{\pgfqpoint{4.027194in}{2.266606in}}%
\pgfpathlineto{\pgfqpoint{4.040158in}{2.263362in}}%
\pgfpathlineto{\pgfqpoint{4.053128in}{2.260146in}}%
\pgfpathlineto{\pgfqpoint{4.066104in}{2.256959in}}%
\pgfpathlineto{\pgfqpoint{4.079086in}{2.253799in}}%
\pgfpathlineto{\pgfqpoint{4.071589in}{2.246269in}}%
\pgfpathlineto{\pgfqpoint{4.064086in}{2.238731in}}%
\pgfpathlineto{\pgfqpoint{4.056578in}{2.231186in}}%
\pgfpathlineto{\pgfqpoint{4.049065in}{2.223636in}}%
\pgfpathlineto{\pgfqpoint{4.036071in}{2.226862in}}%
\pgfpathlineto{\pgfqpoint{4.023083in}{2.230115in}}%
\pgfpathlineto{\pgfqpoint{4.010101in}{2.233396in}}%
\pgfpathlineto{\pgfqpoint{3.997125in}{2.236705in}}%
\pgfpathlineto{\pgfqpoint{4.004650in}{2.244184in}}%
\pgfpathlineto{\pgfqpoint{4.012171in}{2.251662in}}%
\pgfpathlineto{\pgfqpoint{4.019685in}{2.259136in}}%
\pgfpathlineto{\pgfqpoint{4.027194in}{2.266606in}}%
\pgfpathclose%
\pgfusepath{fill}%
\end{pgfscope}%
\begin{pgfscope}%
\pgfpathrectangle{\pgfqpoint{1.254980in}{0.150000in}}{\pgfqpoint{5.490039in}{5.490039in}}%
\pgfusepath{clip}%
\pgfsetbuttcap%
\pgfsetroundjoin%
\definecolor{currentfill}{rgb}{0.278791,0.062145,0.386592}%
\pgfsetfillcolor{currentfill}%
\pgfsetfillopacity{0.700000}%
\pgfsetlinewidth{0.000000pt}%
\definecolor{currentstroke}{rgb}{0.000000,0.000000,0.000000}%
\pgfsetstrokecolor{currentstroke}%
\pgfsetdash{}{0pt}%
\pgfpathmoveto{\pgfqpoint{4.673898in}{2.347721in}}%
\pgfpathlineto{\pgfqpoint{4.687024in}{2.345560in}}%
\pgfpathlineto{\pgfqpoint{4.700157in}{2.343424in}}%
\pgfpathlineto{\pgfqpoint{4.713297in}{2.341313in}}%
\pgfpathlineto{\pgfqpoint{4.726444in}{2.339228in}}%
\pgfpathlineto{\pgfqpoint{4.719187in}{2.332164in}}%
\pgfpathlineto{\pgfqpoint{4.711925in}{2.325060in}}%
\pgfpathlineto{\pgfqpoint{4.704657in}{2.317916in}}%
\pgfpathlineto{\pgfqpoint{4.697383in}{2.310731in}}%
\pgfpathlineto{\pgfqpoint{4.684224in}{2.312805in}}%
\pgfpathlineto{\pgfqpoint{4.671072in}{2.314904in}}%
\pgfpathlineto{\pgfqpoint{4.657926in}{2.317029in}}%
\pgfpathlineto{\pgfqpoint{4.644788in}{2.319179in}}%
\pgfpathlineto{\pgfqpoint{4.652074in}{2.326371in}}%
\pgfpathlineto{\pgfqpoint{4.659355in}{2.333525in}}%
\pgfpathlineto{\pgfqpoint{4.666629in}{2.340641in}}%
\pgfpathlineto{\pgfqpoint{4.673898in}{2.347721in}}%
\pgfpathclose%
\pgfusepath{fill}%
\end{pgfscope}%
\begin{pgfscope}%
\pgfpathrectangle{\pgfqpoint{1.254980in}{0.150000in}}{\pgfqpoint{5.490039in}{5.490039in}}%
\pgfusepath{clip}%
\pgfsetbuttcap%
\pgfsetroundjoin%
\definecolor{currentfill}{rgb}{0.281446,0.084320,0.407414}%
\pgfsetfillcolor{currentfill}%
\pgfsetfillopacity{0.700000}%
\pgfsetlinewidth{0.000000pt}%
\definecolor{currentstroke}{rgb}{0.000000,0.000000,0.000000}%
\pgfsetstrokecolor{currentstroke}%
\pgfsetdash{}{0pt}%
\pgfpathmoveto{\pgfqpoint{4.889508in}{2.378171in}}%
\pgfpathlineto{\pgfqpoint{4.902693in}{2.376248in}}%
\pgfpathlineto{\pgfqpoint{4.915885in}{2.374349in}}%
\pgfpathlineto{\pgfqpoint{4.929084in}{2.372476in}}%
\pgfpathlineto{\pgfqpoint{4.942291in}{2.370626in}}%
\pgfpathlineto{\pgfqpoint{4.935121in}{2.363963in}}%
\pgfpathlineto{\pgfqpoint{4.927945in}{2.357261in}}%
\pgfpathlineto{\pgfqpoint{4.920763in}{2.350518in}}%
\pgfpathlineto{\pgfqpoint{4.913575in}{2.343733in}}%
\pgfpathlineto{\pgfqpoint{4.900355in}{2.345545in}}%
\pgfpathlineto{\pgfqpoint{4.887142in}{2.347381in}}%
\pgfpathlineto{\pgfqpoint{4.873937in}{2.349243in}}%
\pgfpathlineto{\pgfqpoint{4.860739in}{2.351128in}}%
\pgfpathlineto{\pgfqpoint{4.867940in}{2.357946in}}%
\pgfpathlineto{\pgfqpoint{4.875135in}{2.364725in}}%
\pgfpathlineto{\pgfqpoint{4.882324in}{2.371466in}}%
\pgfpathlineto{\pgfqpoint{4.889508in}{2.378171in}}%
\pgfpathclose%
\pgfusepath{fill}%
\end{pgfscope}%
\begin{pgfscope}%
\pgfpathrectangle{\pgfqpoint{1.254980in}{0.150000in}}{\pgfqpoint{5.490039in}{5.490039in}}%
\pgfusepath{clip}%
\pgfsetbuttcap%
\pgfsetroundjoin%
\definecolor{currentfill}{rgb}{0.282623,0.140926,0.457517}%
\pgfsetfillcolor{currentfill}%
\pgfsetfillopacity{0.700000}%
\pgfsetlinewidth{0.000000pt}%
\definecolor{currentstroke}{rgb}{0.000000,0.000000,0.000000}%
\pgfsetstrokecolor{currentstroke}%
\pgfsetdash{}{0pt}%
\pgfpathmoveto{\pgfqpoint{5.751715in}{2.486164in}}%
\pgfpathlineto{\pgfqpoint{5.765132in}{2.484620in}}%
\pgfpathlineto{\pgfqpoint{5.778557in}{2.483099in}}%
\pgfpathlineto{\pgfqpoint{5.791991in}{2.481601in}}%
\pgfpathlineto{\pgfqpoint{5.805432in}{2.480126in}}%
\pgfpathlineto{\pgfqpoint{5.798644in}{2.475087in}}%
\pgfpathlineto{\pgfqpoint{5.791851in}{2.470069in}}%
\pgfpathlineto{\pgfqpoint{5.785054in}{2.465066in}}%
\pgfpathlineto{\pgfqpoint{5.778251in}{2.460074in}}%
\pgfpathlineto{\pgfqpoint{5.764790in}{2.461409in}}%
\pgfpathlineto{\pgfqpoint{5.751337in}{2.462766in}}%
\pgfpathlineto{\pgfqpoint{5.737893in}{2.464147in}}%
\pgfpathlineto{\pgfqpoint{5.724456in}{2.465551in}}%
\pgfpathlineto{\pgfqpoint{5.731278in}{2.470678in}}%
\pgfpathlineto{\pgfqpoint{5.738095in}{2.475820in}}%
\pgfpathlineto{\pgfqpoint{5.744907in}{2.480981in}}%
\pgfpathlineto{\pgfqpoint{5.751715in}{2.486164in}}%
\pgfpathclose%
\pgfusepath{fill}%
\end{pgfscope}%
\begin{pgfscope}%
\pgfpathrectangle{\pgfqpoint{1.254980in}{0.150000in}}{\pgfqpoint{5.490039in}{5.490039in}}%
\pgfusepath{clip}%
\pgfsetbuttcap%
\pgfsetroundjoin%
\definecolor{currentfill}{rgb}{0.277941,0.056324,0.381191}%
\pgfsetfillcolor{currentfill}%
\pgfsetfillopacity{0.700000}%
\pgfsetlinewidth{0.000000pt}%
\definecolor{currentstroke}{rgb}{0.000000,0.000000,0.000000}%
\pgfsetstrokecolor{currentstroke}%
\pgfsetdash{}{0pt}%
\pgfpathmoveto{\pgfqpoint{3.009124in}{2.339883in}}%
\pgfpathlineto{\pgfqpoint{3.021914in}{2.333738in}}%
\pgfpathlineto{\pgfqpoint{3.034707in}{2.327635in}}%
\pgfpathlineto{\pgfqpoint{3.047505in}{2.321572in}}%
\pgfpathlineto{\pgfqpoint{3.060305in}{2.315550in}}%
\pgfpathlineto{\pgfqpoint{3.052373in}{2.311268in}}%
\pgfpathlineto{\pgfqpoint{3.044432in}{2.307109in}}%
\pgfpathlineto{\pgfqpoint{3.036481in}{2.303079in}}%
\pgfpathlineto{\pgfqpoint{3.028521in}{2.299180in}}%
\pgfpathlineto{\pgfqpoint{3.015700in}{2.305371in}}%
\pgfpathlineto{\pgfqpoint{3.002884in}{2.311602in}}%
\pgfpathlineto{\pgfqpoint{2.990070in}{2.317874in}}%
\pgfpathlineto{\pgfqpoint{2.977260in}{2.324188in}}%
\pgfpathlineto{\pgfqpoint{2.985241in}{2.327913in}}%
\pgfpathlineto{\pgfqpoint{2.993212in}{2.331774in}}%
\pgfpathlineto{\pgfqpoint{3.001173in}{2.335765in}}%
\pgfpathlineto{\pgfqpoint{3.009124in}{2.339883in}}%
\pgfpathclose%
\pgfusepath{fill}%
\end{pgfscope}%
\begin{pgfscope}%
\pgfpathrectangle{\pgfqpoint{1.254980in}{0.150000in}}{\pgfqpoint{5.490039in}{5.490039in}}%
\pgfusepath{clip}%
\pgfsetbuttcap%
\pgfsetroundjoin%
\definecolor{currentfill}{rgb}{0.282656,0.100196,0.422160}%
\pgfsetfillcolor{currentfill}%
\pgfsetfillopacity{0.700000}%
\pgfsetlinewidth{0.000000pt}%
\definecolor{currentstroke}{rgb}{0.000000,0.000000,0.000000}%
\pgfsetstrokecolor{currentstroke}%
\pgfsetdash{}{0pt}%
\pgfpathmoveto{\pgfqpoint{5.105126in}{2.407923in}}%
\pgfpathlineto{\pgfqpoint{5.118370in}{2.406180in}}%
\pgfpathlineto{\pgfqpoint{5.131622in}{2.404461in}}%
\pgfpathlineto{\pgfqpoint{5.144881in}{2.402766in}}%
\pgfpathlineto{\pgfqpoint{5.158148in}{2.401095in}}%
\pgfpathlineto{\pgfqpoint{5.151069in}{2.394880in}}%
\pgfpathlineto{\pgfqpoint{5.143984in}{2.388632in}}%
\pgfpathlineto{\pgfqpoint{5.136893in}{2.382350in}}%
\pgfpathlineto{\pgfqpoint{5.129795in}{2.376030in}}%
\pgfpathlineto{\pgfqpoint{5.116514in}{2.377638in}}%
\pgfpathlineto{\pgfqpoint{5.103240in}{2.379270in}}%
\pgfpathlineto{\pgfqpoint{5.089974in}{2.380926in}}%
\pgfpathlineto{\pgfqpoint{5.076715in}{2.382606in}}%
\pgfpathlineto{\pgfqpoint{5.083827in}{2.388985in}}%
\pgfpathlineto{\pgfqpoint{5.090932in}{2.395329in}}%
\pgfpathlineto{\pgfqpoint{5.098032in}{2.401641in}}%
\pgfpathlineto{\pgfqpoint{5.105126in}{2.407923in}}%
\pgfpathclose%
\pgfusepath{fill}%
\end{pgfscope}%
\begin{pgfscope}%
\pgfpathrectangle{\pgfqpoint{1.254980in}{0.150000in}}{\pgfqpoint{5.490039in}{5.490039in}}%
\pgfusepath{clip}%
\pgfsetbuttcap%
\pgfsetroundjoin%
\definecolor{currentfill}{rgb}{0.283072,0.130895,0.449241}%
\pgfsetfillcolor{currentfill}%
\pgfsetfillopacity{0.700000}%
\pgfsetlinewidth{0.000000pt}%
\definecolor{currentstroke}{rgb}{0.000000,0.000000,0.000000}%
\pgfsetstrokecolor{currentstroke}%
\pgfsetdash{}{0pt}%
\pgfpathmoveto{\pgfqpoint{5.536261in}{2.462225in}}%
\pgfpathlineto{\pgfqpoint{5.549623in}{2.460670in}}%
\pgfpathlineto{\pgfqpoint{5.562992in}{2.459138in}}%
\pgfpathlineto{\pgfqpoint{5.576369in}{2.457630in}}%
\pgfpathlineto{\pgfqpoint{5.589753in}{2.456145in}}%
\pgfpathlineto{\pgfqpoint{5.582867in}{2.450789in}}%
\pgfpathlineto{\pgfqpoint{5.575975in}{2.445429in}}%
\pgfpathlineto{\pgfqpoint{5.569078in}{2.440063in}}%
\pgfpathlineto{\pgfqpoint{5.562175in}{2.434685in}}%
\pgfpathlineto{\pgfqpoint{5.548772in}{2.436055in}}%
\pgfpathlineto{\pgfqpoint{5.535378in}{2.437449in}}%
\pgfpathlineto{\pgfqpoint{5.521991in}{2.438866in}}%
\pgfpathlineto{\pgfqpoint{5.508612in}{2.440306in}}%
\pgfpathlineto{\pgfqpoint{5.515532in}{2.445793in}}%
\pgfpathlineto{\pgfqpoint{5.522448in}{2.451273in}}%
\pgfpathlineto{\pgfqpoint{5.529357in}{2.456749in}}%
\pgfpathlineto{\pgfqpoint{5.536261in}{2.462225in}}%
\pgfpathclose%
\pgfusepath{fill}%
\end{pgfscope}%
\begin{pgfscope}%
\pgfpathrectangle{\pgfqpoint{1.254980in}{0.150000in}}{\pgfqpoint{5.490039in}{5.490039in}}%
\pgfusepath{clip}%
\pgfsetbuttcap%
\pgfsetroundjoin%
\definecolor{currentfill}{rgb}{0.283197,0.115680,0.436115}%
\pgfsetfillcolor{currentfill}%
\pgfsetfillopacity{0.700000}%
\pgfsetlinewidth{0.000000pt}%
\definecolor{currentstroke}{rgb}{0.000000,0.000000,0.000000}%
\pgfsetstrokecolor{currentstroke}%
\pgfsetdash{}{0pt}%
\pgfpathmoveto{\pgfqpoint{5.320722in}{2.436106in}}%
\pgfpathlineto{\pgfqpoint{5.334025in}{2.434485in}}%
\pgfpathlineto{\pgfqpoint{5.347336in}{2.432888in}}%
\pgfpathlineto{\pgfqpoint{5.360655in}{2.431314in}}%
\pgfpathlineto{\pgfqpoint{5.373981in}{2.429764in}}%
\pgfpathlineto{\pgfqpoint{5.366997in}{2.424001in}}%
\pgfpathlineto{\pgfqpoint{5.360007in}{2.418216in}}%
\pgfpathlineto{\pgfqpoint{5.353011in}{2.412407in}}%
\pgfpathlineto{\pgfqpoint{5.346009in}{2.406572in}}%
\pgfpathlineto{\pgfqpoint{5.332667in}{2.408033in}}%
\pgfpathlineto{\pgfqpoint{5.319332in}{2.409518in}}%
\pgfpathlineto{\pgfqpoint{5.306005in}{2.411026in}}%
\pgfpathlineto{\pgfqpoint{5.292686in}{2.412558in}}%
\pgfpathlineto{\pgfqpoint{5.299703in}{2.418478in}}%
\pgfpathlineto{\pgfqpoint{5.306715in}{2.424374in}}%
\pgfpathlineto{\pgfqpoint{5.313721in}{2.430249in}}%
\pgfpathlineto{\pgfqpoint{5.320722in}{2.436106in}}%
\pgfpathclose%
\pgfusepath{fill}%
\end{pgfscope}%
\begin{pgfscope}%
\pgfpathrectangle{\pgfqpoint{1.254980in}{0.150000in}}{\pgfqpoint{5.490039in}{5.490039in}}%
\pgfusepath{clip}%
\pgfsetbuttcap%
\pgfsetroundjoin%
\definecolor{currentfill}{rgb}{0.267004,0.004874,0.329415}%
\pgfsetfillcolor{currentfill}%
\pgfsetfillopacity{0.700000}%
\pgfsetlinewidth{0.000000pt}%
\definecolor{currentstroke}{rgb}{0.000000,0.000000,0.000000}%
\pgfsetstrokecolor{currentstroke}%
\pgfsetdash{}{0pt}%
\pgfpathmoveto{\pgfqpoint{3.811561in}{2.249698in}}%
\pgfpathlineto{\pgfqpoint{3.824480in}{2.245962in}}%
\pgfpathlineto{\pgfqpoint{3.837405in}{2.242256in}}%
\pgfpathlineto{\pgfqpoint{3.850335in}{2.238579in}}%
\pgfpathlineto{\pgfqpoint{3.863272in}{2.234932in}}%
\pgfpathlineto{\pgfqpoint{3.855693in}{2.227639in}}%
\pgfpathlineto{\pgfqpoint{3.848109in}{2.220361in}}%
\pgfpathlineto{\pgfqpoint{3.840520in}{2.213098in}}%
\pgfpathlineto{\pgfqpoint{3.832925in}{2.205854in}}%
\pgfpathlineto{\pgfqpoint{3.819976in}{2.209592in}}%
\pgfpathlineto{\pgfqpoint{3.807033in}{2.213360in}}%
\pgfpathlineto{\pgfqpoint{3.794095in}{2.217158in}}%
\pgfpathlineto{\pgfqpoint{3.781163in}{2.220985in}}%
\pgfpathlineto{\pgfqpoint{3.788771in}{2.228133in}}%
\pgfpathlineto{\pgfqpoint{3.796373in}{2.235303in}}%
\pgfpathlineto{\pgfqpoint{3.803970in}{2.242492in}}%
\pgfpathlineto{\pgfqpoint{3.811561in}{2.249698in}}%
\pgfpathclose%
\pgfusepath{fill}%
\end{pgfscope}%
\begin{pgfscope}%
\pgfpathrectangle{\pgfqpoint{1.254980in}{0.150000in}}{\pgfqpoint{5.490039in}{5.490039in}}%
\pgfusepath{clip}%
\pgfsetbuttcap%
\pgfsetroundjoin%
\definecolor{currentfill}{rgb}{0.274952,0.037752,0.364543}%
\pgfsetfillcolor{currentfill}%
\pgfsetfillopacity{0.700000}%
\pgfsetlinewidth{0.000000pt}%
\definecolor{currentstroke}{rgb}{0.000000,0.000000,0.000000}%
\pgfsetstrokecolor{currentstroke}%
\pgfsetdash{}{0pt}%
\pgfpathmoveto{\pgfqpoint{4.376623in}{2.298345in}}%
\pgfpathlineto{\pgfqpoint{4.389677in}{2.295766in}}%
\pgfpathlineto{\pgfqpoint{4.402738in}{2.293214in}}%
\pgfpathlineto{\pgfqpoint{4.415805in}{2.290688in}}%
\pgfpathlineto{\pgfqpoint{4.428879in}{2.288189in}}%
\pgfpathlineto{\pgfqpoint{4.421506in}{2.280700in}}%
\pgfpathlineto{\pgfqpoint{4.414127in}{2.273181in}}%
\pgfpathlineto{\pgfqpoint{4.406743in}{2.265629in}}%
\pgfpathlineto{\pgfqpoint{4.399354in}{2.258047in}}%
\pgfpathlineto{\pgfqpoint{4.386268in}{2.260574in}}%
\pgfpathlineto{\pgfqpoint{4.373189in}{2.263128in}}%
\pgfpathlineto{\pgfqpoint{4.360117in}{2.265707in}}%
\pgfpathlineto{\pgfqpoint{4.347051in}{2.268313in}}%
\pgfpathlineto{\pgfqpoint{4.354452in}{2.275863in}}%
\pgfpathlineto{\pgfqpoint{4.361848in}{2.283385in}}%
\pgfpathlineto{\pgfqpoint{4.369238in}{2.290879in}}%
\pgfpathlineto{\pgfqpoint{4.376623in}{2.298345in}}%
\pgfpathclose%
\pgfusepath{fill}%
\end{pgfscope}%
\begin{pgfscope}%
\pgfpathrectangle{\pgfqpoint{1.254980in}{0.150000in}}{\pgfqpoint{5.490039in}{5.490039in}}%
\pgfusepath{clip}%
\pgfsetbuttcap%
\pgfsetroundjoin%
\definecolor{currentfill}{rgb}{0.271305,0.019942,0.347269}%
\pgfsetfillcolor{currentfill}%
\pgfsetfillopacity{0.700000}%
\pgfsetlinewidth{0.000000pt}%
\definecolor{currentstroke}{rgb}{0.000000,0.000000,0.000000}%
\pgfsetstrokecolor{currentstroke}%
\pgfsetdash{}{0pt}%
\pgfpathmoveto{\pgfqpoint{4.160963in}{2.271676in}}%
\pgfpathlineto{\pgfqpoint{4.173964in}{2.268708in}}%
\pgfpathlineto{\pgfqpoint{4.186971in}{2.265767in}}%
\pgfpathlineto{\pgfqpoint{4.199984in}{2.262853in}}%
\pgfpathlineto{\pgfqpoint{4.213004in}{2.259967in}}%
\pgfpathlineto{\pgfqpoint{4.205552in}{2.252381in}}%
\pgfpathlineto{\pgfqpoint{4.198095in}{2.244777in}}%
\pgfpathlineto{\pgfqpoint{4.190632in}{2.237156in}}%
\pgfpathlineto{\pgfqpoint{4.183164in}{2.229518in}}%
\pgfpathlineto{\pgfqpoint{4.170132in}{2.232458in}}%
\pgfpathlineto{\pgfqpoint{4.157107in}{2.235424in}}%
\pgfpathlineto{\pgfqpoint{4.144088in}{2.238418in}}%
\pgfpathlineto{\pgfqpoint{4.131076in}{2.241439in}}%
\pgfpathlineto{\pgfqpoint{4.138556in}{2.249019in}}%
\pgfpathlineto{\pgfqpoint{4.146030in}{2.256586in}}%
\pgfpathlineto{\pgfqpoint{4.153499in}{2.264138in}}%
\pgfpathlineto{\pgfqpoint{4.160963in}{2.271676in}}%
\pgfpathclose%
\pgfusepath{fill}%
\end{pgfscope}%
\begin{pgfscope}%
\pgfpathrectangle{\pgfqpoint{1.254980in}{0.150000in}}{\pgfqpoint{5.490039in}{5.490039in}}%
\pgfusepath{clip}%
\pgfsetbuttcap%
\pgfsetroundjoin%
\definecolor{currentfill}{rgb}{0.277941,0.056324,0.381191}%
\pgfsetfillcolor{currentfill}%
\pgfsetfillopacity{0.700000}%
\pgfsetlinewidth{0.000000pt}%
\definecolor{currentstroke}{rgb}{0.000000,0.000000,0.000000}%
\pgfsetstrokecolor{currentstroke}%
\pgfsetdash{}{0pt}%
\pgfpathmoveto{\pgfqpoint{4.592304in}{2.328032in}}%
\pgfpathlineto{\pgfqpoint{4.605415in}{2.325780in}}%
\pgfpathlineto{\pgfqpoint{4.618532in}{2.323554in}}%
\pgfpathlineto{\pgfqpoint{4.631657in}{2.321354in}}%
\pgfpathlineto{\pgfqpoint{4.644788in}{2.319179in}}%
\pgfpathlineto{\pgfqpoint{4.637496in}{2.311948in}}%
\pgfpathlineto{\pgfqpoint{4.630198in}{2.304678in}}%
\pgfpathlineto{\pgfqpoint{4.622895in}{2.297368in}}%
\pgfpathlineto{\pgfqpoint{4.615586in}{2.290017in}}%
\pgfpathlineto{\pgfqpoint{4.602442in}{2.292194in}}%
\pgfpathlineto{\pgfqpoint{4.589306in}{2.294396in}}%
\pgfpathlineto{\pgfqpoint{4.576176in}{2.296624in}}%
\pgfpathlineto{\pgfqpoint{4.563053in}{2.298877in}}%
\pgfpathlineto{\pgfqpoint{4.570374in}{2.306221in}}%
\pgfpathlineto{\pgfqpoint{4.577690in}{2.313528in}}%
\pgfpathlineto{\pgfqpoint{4.585000in}{2.320798in}}%
\pgfpathlineto{\pgfqpoint{4.592304in}{2.328032in}}%
\pgfpathclose%
\pgfusepath{fill}%
\end{pgfscope}%
\begin{pgfscope}%
\pgfpathrectangle{\pgfqpoint{1.254980in}{0.150000in}}{\pgfqpoint{5.490039in}{5.490039in}}%
\pgfusepath{clip}%
\pgfsetbuttcap%
\pgfsetroundjoin%
\definecolor{currentfill}{rgb}{0.269944,0.014625,0.341379}%
\pgfsetfillcolor{currentfill}%
\pgfsetfillopacity{0.700000}%
\pgfsetlinewidth{0.000000pt}%
\definecolor{currentstroke}{rgb}{0.000000,0.000000,0.000000}%
\pgfsetstrokecolor{currentstroke}%
\pgfsetdash{}{0pt}%
\pgfpathmoveto{\pgfqpoint{3.328217in}{2.267461in}}%
\pgfpathlineto{\pgfqpoint{3.341052in}{2.262378in}}%
\pgfpathlineto{\pgfqpoint{3.353892in}{2.257331in}}%
\pgfpathlineto{\pgfqpoint{3.366736in}{2.252319in}}%
\pgfpathlineto{\pgfqpoint{3.379584in}{2.247341in}}%
\pgfpathlineto{\pgfqpoint{3.371806in}{2.241519in}}%
\pgfpathlineto{\pgfqpoint{3.364020in}{2.235775in}}%
\pgfpathlineto{\pgfqpoint{3.356227in}{2.230112in}}%
\pgfpathlineto{\pgfqpoint{3.348427in}{2.224534in}}%
\pgfpathlineto{\pgfqpoint{3.335562in}{2.229654in}}%
\pgfpathlineto{\pgfqpoint{3.322702in}{2.234809in}}%
\pgfpathlineto{\pgfqpoint{3.309846in}{2.239999in}}%
\pgfpathlineto{\pgfqpoint{3.296994in}{2.245224in}}%
\pgfpathlineto{\pgfqpoint{3.304812in}{2.250654in}}%
\pgfpathlineto{\pgfqpoint{3.312621in}{2.256173in}}%
\pgfpathlineto{\pgfqpoint{3.320423in}{2.261776in}}%
\pgfpathlineto{\pgfqpoint{3.328217in}{2.267461in}}%
\pgfpathclose%
\pgfusepath{fill}%
\end{pgfscope}%
\begin{pgfscope}%
\pgfpathrectangle{\pgfqpoint{1.254980in}{0.150000in}}{\pgfqpoint{5.490039in}{5.490039in}}%
\pgfusepath{clip}%
\pgfsetbuttcap%
\pgfsetroundjoin%
\definecolor{currentfill}{rgb}{0.268510,0.009605,0.335427}%
\pgfsetfillcolor{currentfill}%
\pgfsetfillopacity{0.700000}%
\pgfsetlinewidth{0.000000pt}%
\definecolor{currentstroke}{rgb}{0.000000,0.000000,0.000000}%
\pgfsetstrokecolor{currentstroke}%
\pgfsetdash{}{0pt}%
\pgfpathmoveto{\pgfqpoint{3.462003in}{2.252296in}}%
\pgfpathlineto{\pgfqpoint{3.474860in}{2.247617in}}%
\pgfpathlineto{\pgfqpoint{3.487721in}{2.242972in}}%
\pgfpathlineto{\pgfqpoint{3.500587in}{2.238361in}}%
\pgfpathlineto{\pgfqpoint{3.513458in}{2.233782in}}%
\pgfpathlineto{\pgfqpoint{3.505739in}{2.227430in}}%
\pgfpathlineto{\pgfqpoint{3.498012in}{2.221137in}}%
\pgfpathlineto{\pgfqpoint{3.490280in}{2.214906in}}%
\pgfpathlineto{\pgfqpoint{3.482540in}{2.208741in}}%
\pgfpathlineto{\pgfqpoint{3.469654in}{2.213449in}}%
\pgfpathlineto{\pgfqpoint{3.456772in}{2.218190in}}%
\pgfpathlineto{\pgfqpoint{3.443896in}{2.222965in}}%
\pgfpathlineto{\pgfqpoint{3.431024in}{2.227772in}}%
\pgfpathlineto{\pgfqpoint{3.438779in}{2.233803in}}%
\pgfpathlineto{\pgfqpoint{3.446527in}{2.239903in}}%
\pgfpathlineto{\pgfqpoint{3.454269in}{2.246068in}}%
\pgfpathlineto{\pgfqpoint{3.462003in}{2.252296in}}%
\pgfpathclose%
\pgfusepath{fill}%
\end{pgfscope}%
\begin{pgfscope}%
\pgfpathrectangle{\pgfqpoint{1.254980in}{0.150000in}}{\pgfqpoint{5.490039in}{5.490039in}}%
\pgfusepath{clip}%
\pgfsetbuttcap%
\pgfsetroundjoin%
\definecolor{currentfill}{rgb}{0.283187,0.125848,0.444960}%
\pgfsetfillcolor{currentfill}%
\pgfsetfillopacity{0.700000}%
\pgfsetlinewidth{0.000000pt}%
\definecolor{currentstroke}{rgb}{0.000000,0.000000,0.000000}%
\pgfsetstrokecolor{currentstroke}%
\pgfsetdash{}{0pt}%
\pgfpathmoveto{\pgfqpoint{2.689251in}{2.449953in}}%
\pgfpathlineto{\pgfqpoint{2.702025in}{2.442542in}}%
\pgfpathlineto{\pgfqpoint{2.714800in}{2.435181in}}%
\pgfpathlineto{\pgfqpoint{2.727579in}{2.427869in}}%
\pgfpathlineto{\pgfqpoint{2.740359in}{2.420607in}}%
\pgfpathlineto{\pgfqpoint{2.732235in}{2.418385in}}%
\pgfpathlineto{\pgfqpoint{2.724099in}{2.416339in}}%
\pgfpathlineto{\pgfqpoint{2.715950in}{2.414472in}}%
\pgfpathlineto{\pgfqpoint{2.707789in}{2.412791in}}%
\pgfpathlineto{\pgfqpoint{2.694984in}{2.420249in}}%
\pgfpathlineto{\pgfqpoint{2.682181in}{2.427757in}}%
\pgfpathlineto{\pgfqpoint{2.669381in}{2.435314in}}%
\pgfpathlineto{\pgfqpoint{2.656583in}{2.442921in}}%
\pgfpathlineto{\pgfqpoint{2.664770in}{2.444401in}}%
\pgfpathlineto{\pgfqpoint{2.672943in}{2.446070in}}%
\pgfpathlineto{\pgfqpoint{2.681103in}{2.447922in}}%
\pgfpathlineto{\pgfqpoint{2.689251in}{2.449953in}}%
\pgfpathclose%
\pgfusepath{fill}%
\end{pgfscope}%
\begin{pgfscope}%
\pgfpathrectangle{\pgfqpoint{1.254980in}{0.150000in}}{\pgfqpoint{5.490039in}{5.490039in}}%
\pgfusepath{clip}%
\pgfsetbuttcap%
\pgfsetroundjoin%
\definecolor{currentfill}{rgb}{0.281446,0.084320,0.407414}%
\pgfsetfillcolor{currentfill}%
\pgfsetfillopacity{0.700000}%
\pgfsetlinewidth{0.000000pt}%
\definecolor{currentstroke}{rgb}{0.000000,0.000000,0.000000}%
\pgfsetstrokecolor{currentstroke}%
\pgfsetdash{}{0pt}%
\pgfpathmoveto{\pgfqpoint{2.874900in}{2.376233in}}%
\pgfpathlineto{\pgfqpoint{2.887684in}{2.369574in}}%
\pgfpathlineto{\pgfqpoint{2.900471in}{2.362960in}}%
\pgfpathlineto{\pgfqpoint{2.913261in}{2.356390in}}%
\pgfpathlineto{\pgfqpoint{2.926054in}{2.349864in}}%
\pgfpathlineto{\pgfqpoint{2.918042in}{2.346455in}}%
\pgfpathlineto{\pgfqpoint{2.910020in}{2.343193in}}%
\pgfpathlineto{\pgfqpoint{2.901987in}{2.340082in}}%
\pgfpathlineto{\pgfqpoint{2.893943in}{2.337128in}}%
\pgfpathlineto{\pgfqpoint{2.881128in}{2.343836in}}%
\pgfpathlineto{\pgfqpoint{2.868316in}{2.350588in}}%
\pgfpathlineto{\pgfqpoint{2.855508in}{2.357384in}}%
\pgfpathlineto{\pgfqpoint{2.842702in}{2.364225in}}%
\pgfpathlineto{\pgfqpoint{2.850768in}{2.366992in}}%
\pgfpathlineto{\pgfqpoint{2.858823in}{2.369919in}}%
\pgfpathlineto{\pgfqpoint{2.866867in}{2.373001in}}%
\pgfpathlineto{\pgfqpoint{2.874900in}{2.376233in}}%
\pgfpathclose%
\pgfusepath{fill}%
\end{pgfscope}%
\begin{pgfscope}%
\pgfpathrectangle{\pgfqpoint{1.254980in}{0.150000in}}{\pgfqpoint{5.490039in}{5.490039in}}%
\pgfusepath{clip}%
\pgfsetbuttcap%
\pgfsetroundjoin%
\definecolor{currentfill}{rgb}{0.268510,0.009605,0.335427}%
\pgfsetfillcolor{currentfill}%
\pgfsetfillopacity{0.700000}%
\pgfsetlinewidth{0.000000pt}%
\definecolor{currentstroke}{rgb}{0.000000,0.000000,0.000000}%
\pgfsetstrokecolor{currentstroke}%
\pgfsetdash{}{0pt}%
\pgfpathmoveto{\pgfqpoint{3.945280in}{2.250227in}}%
\pgfpathlineto{\pgfqpoint{3.958232in}{2.246804in}}%
\pgfpathlineto{\pgfqpoint{3.971190in}{2.243409in}}%
\pgfpathlineto{\pgfqpoint{3.984155in}{2.240043in}}%
\pgfpathlineto{\pgfqpoint{3.997125in}{2.236705in}}%
\pgfpathlineto{\pgfqpoint{3.989594in}{2.229226in}}%
\pgfpathlineto{\pgfqpoint{3.982057in}{2.221747in}}%
\pgfpathlineto{\pgfqpoint{3.974515in}{2.214271in}}%
\pgfpathlineto{\pgfqpoint{3.966967in}{2.206799in}}%
\pgfpathlineto{\pgfqpoint{3.953985in}{2.210216in}}%
\pgfpathlineto{\pgfqpoint{3.941008in}{2.213660in}}%
\pgfpathlineto{\pgfqpoint{3.928038in}{2.217133in}}%
\pgfpathlineto{\pgfqpoint{3.915073in}{2.220635in}}%
\pgfpathlineto{\pgfqpoint{3.922633in}{2.228024in}}%
\pgfpathlineto{\pgfqpoint{3.930187in}{2.235420in}}%
\pgfpathlineto{\pgfqpoint{3.937736in}{2.242821in}}%
\pgfpathlineto{\pgfqpoint{3.945280in}{2.250227in}}%
\pgfpathclose%
\pgfusepath{fill}%
\end{pgfscope}%
\begin{pgfscope}%
\pgfpathrectangle{\pgfqpoint{1.254980in}{0.150000in}}{\pgfqpoint{5.490039in}{5.490039in}}%
\pgfusepath{clip}%
\pgfsetbuttcap%
\pgfsetroundjoin%
\definecolor{currentfill}{rgb}{0.280267,0.073417,0.397163}%
\pgfsetfillcolor{currentfill}%
\pgfsetfillopacity{0.700000}%
\pgfsetlinewidth{0.000000pt}%
\definecolor{currentstroke}{rgb}{0.000000,0.000000,0.000000}%
\pgfsetstrokecolor{currentstroke}%
\pgfsetdash{}{0pt}%
\pgfpathmoveto{\pgfqpoint{4.808018in}{2.358919in}}%
\pgfpathlineto{\pgfqpoint{4.821188in}{2.356934in}}%
\pgfpathlineto{\pgfqpoint{4.834364in}{2.354974in}}%
\pgfpathlineto{\pgfqpoint{4.847548in}{2.353039in}}%
\pgfpathlineto{\pgfqpoint{4.860739in}{2.351128in}}%
\pgfpathlineto{\pgfqpoint{4.853532in}{2.344271in}}%
\pgfpathlineto{\pgfqpoint{4.846319in}{2.337371in}}%
\pgfpathlineto{\pgfqpoint{4.839099in}{2.330429in}}%
\pgfpathlineto{\pgfqpoint{4.831874in}{2.323442in}}%
\pgfpathlineto{\pgfqpoint{4.818670in}{2.325328in}}%
\pgfpathlineto{\pgfqpoint{4.805474in}{2.327239in}}%
\pgfpathlineto{\pgfqpoint{4.792284in}{2.329175in}}%
\pgfpathlineto{\pgfqpoint{4.779102in}{2.331136in}}%
\pgfpathlineto{\pgfqpoint{4.786340in}{2.338142in}}%
\pgfpathlineto{\pgfqpoint{4.793572in}{2.345107in}}%
\pgfpathlineto{\pgfqpoint{4.800798in}{2.352032in}}%
\pgfpathlineto{\pgfqpoint{4.808018in}{2.358919in}}%
\pgfpathclose%
\pgfusepath{fill}%
\end{pgfscope}%
\begin{pgfscope}%
\pgfpathrectangle{\pgfqpoint{1.254980in}{0.150000in}}{\pgfqpoint{5.490039in}{5.490039in}}%
\pgfusepath{clip}%
\pgfsetbuttcap%
\pgfsetroundjoin%
\definecolor{currentfill}{rgb}{0.281887,0.150881,0.465405}%
\pgfsetfillcolor{currentfill}%
\pgfsetfillopacity{0.700000}%
\pgfsetlinewidth{0.000000pt}%
\definecolor{currentstroke}{rgb}{0.000000,0.000000,0.000000}%
\pgfsetstrokecolor{currentstroke}%
\pgfsetdash{}{0pt}%
\pgfpathmoveto{\pgfqpoint{5.886300in}{2.494291in}}%
\pgfpathlineto{\pgfqpoint{5.899760in}{2.492777in}}%
\pgfpathlineto{\pgfqpoint{5.913228in}{2.491287in}}%
\pgfpathlineto{\pgfqpoint{5.926705in}{2.489819in}}%
\pgfpathlineto{\pgfqpoint{5.940189in}{2.488374in}}%
\pgfpathlineto{\pgfqpoint{5.933460in}{2.483525in}}%
\pgfpathlineto{\pgfqpoint{5.926727in}{2.478708in}}%
\pgfpathlineto{\pgfqpoint{5.919990in}{2.473919in}}%
\pgfpathlineto{\pgfqpoint{5.913248in}{2.469153in}}%
\pgfpathlineto{\pgfqpoint{5.899743in}{2.470444in}}%
\pgfpathlineto{\pgfqpoint{5.886245in}{2.471758in}}%
\pgfpathlineto{\pgfqpoint{5.872756in}{2.473095in}}%
\pgfpathlineto{\pgfqpoint{5.859275in}{2.474456in}}%
\pgfpathlineto{\pgfqpoint{5.866038in}{2.479370in}}%
\pgfpathlineto{\pgfqpoint{5.872796in}{2.484311in}}%
\pgfpathlineto{\pgfqpoint{5.879550in}{2.489283in}}%
\pgfpathlineto{\pgfqpoint{5.886300in}{2.494291in}}%
\pgfpathclose%
\pgfusepath{fill}%
\end{pgfscope}%
\begin{pgfscope}%
\pgfpathrectangle{\pgfqpoint{1.254980in}{0.150000in}}{\pgfqpoint{5.490039in}{5.490039in}}%
\pgfusepath{clip}%
\pgfsetbuttcap%
\pgfsetroundjoin%
\definecolor{currentfill}{rgb}{0.273809,0.031497,0.358853}%
\pgfsetfillcolor{currentfill}%
\pgfsetfillopacity{0.700000}%
\pgfsetlinewidth{0.000000pt}%
\definecolor{currentstroke}{rgb}{0.000000,0.000000,0.000000}%
\pgfsetstrokecolor{currentstroke}%
\pgfsetdash{}{0pt}%
\pgfpathmoveto{\pgfqpoint{3.194338in}{2.288322in}}%
\pgfpathlineto{\pgfqpoint{3.207155in}{2.282806in}}%
\pgfpathlineto{\pgfqpoint{3.219977in}{2.277327in}}%
\pgfpathlineto{\pgfqpoint{3.232802in}{2.271886in}}%
\pgfpathlineto{\pgfqpoint{3.245632in}{2.266481in}}%
\pgfpathlineto{\pgfqpoint{3.237790in}{2.261292in}}%
\pgfpathlineto{\pgfqpoint{3.229939in}{2.256202in}}%
\pgfpathlineto{\pgfqpoint{3.222081in}{2.251213in}}%
\pgfpathlineto{\pgfqpoint{3.214214in}{2.246331in}}%
\pgfpathlineto{\pgfqpoint{3.201366in}{2.251891in}}%
\pgfpathlineto{\pgfqpoint{3.188523in}{2.257489in}}%
\pgfpathlineto{\pgfqpoint{3.175683in}{2.263123in}}%
\pgfpathlineto{\pgfqpoint{3.162848in}{2.268794in}}%
\pgfpathlineto{\pgfqpoint{3.170733in}{2.273516in}}%
\pgfpathlineto{\pgfqpoint{3.178610in}{2.278347in}}%
\pgfpathlineto{\pgfqpoint{3.186478in}{2.283284in}}%
\pgfpathlineto{\pgfqpoint{3.194338in}{2.288322in}}%
\pgfpathclose%
\pgfusepath{fill}%
\end{pgfscope}%
\begin{pgfscope}%
\pgfpathrectangle{\pgfqpoint{1.254980in}{0.150000in}}{\pgfqpoint{5.490039in}{5.490039in}}%
\pgfusepath{clip}%
\pgfsetbuttcap%
\pgfsetroundjoin%
\definecolor{currentfill}{rgb}{0.267004,0.004874,0.329415}%
\pgfsetfillcolor{currentfill}%
\pgfsetfillopacity{0.700000}%
\pgfsetlinewidth{0.000000pt}%
\definecolor{currentstroke}{rgb}{0.000000,0.000000,0.000000}%
\pgfsetstrokecolor{currentstroke}%
\pgfsetdash{}{0pt}%
\pgfpathmoveto{\pgfqpoint{3.595746in}{2.242201in}}%
\pgfpathlineto{\pgfqpoint{3.608627in}{2.237900in}}%
\pgfpathlineto{\pgfqpoint{3.621514in}{2.233631in}}%
\pgfpathlineto{\pgfqpoint{3.634406in}{2.229394in}}%
\pgfpathlineto{\pgfqpoint{3.647303in}{2.225188in}}%
\pgfpathlineto{\pgfqpoint{3.639638in}{2.218404in}}%
\pgfpathlineto{\pgfqpoint{3.631967in}{2.211661in}}%
\pgfpathlineto{\pgfqpoint{3.624290in}{2.204963in}}%
\pgfpathlineto{\pgfqpoint{3.616606in}{2.198312in}}%
\pgfpathlineto{\pgfqpoint{3.603695in}{2.202635in}}%
\pgfpathlineto{\pgfqpoint{3.590788in}{2.206989in}}%
\pgfpathlineto{\pgfqpoint{3.577887in}{2.211374in}}%
\pgfpathlineto{\pgfqpoint{3.564991in}{2.215791in}}%
\pgfpathlineto{\pgfqpoint{3.572690in}{2.222321in}}%
\pgfpathlineto{\pgfqpoint{3.580381in}{2.228901in}}%
\pgfpathlineto{\pgfqpoint{3.588067in}{2.235529in}}%
\pgfpathlineto{\pgfqpoint{3.595746in}{2.242201in}}%
\pgfpathclose%
\pgfusepath{fill}%
\end{pgfscope}%
\begin{pgfscope}%
\pgfpathrectangle{\pgfqpoint{1.254980in}{0.150000in}}{\pgfqpoint{5.490039in}{5.490039in}}%
\pgfusepath{clip}%
\pgfsetbuttcap%
\pgfsetroundjoin%
\definecolor{currentfill}{rgb}{0.282327,0.094955,0.417331}%
\pgfsetfillcolor{currentfill}%
\pgfsetfillopacity{0.700000}%
\pgfsetlinewidth{0.000000pt}%
\definecolor{currentstroke}{rgb}{0.000000,0.000000,0.000000}%
\pgfsetstrokecolor{currentstroke}%
\pgfsetdash{}{0pt}%
\pgfpathmoveto{\pgfqpoint{5.023754in}{2.389570in}}%
\pgfpathlineto{\pgfqpoint{5.036983in}{2.387793in}}%
\pgfpathlineto{\pgfqpoint{5.050220in}{2.386040in}}%
\pgfpathlineto{\pgfqpoint{5.063464in}{2.384311in}}%
\pgfpathlineto{\pgfqpoint{5.076715in}{2.382606in}}%
\pgfpathlineto{\pgfqpoint{5.069597in}{2.376192in}}%
\pgfpathlineto{\pgfqpoint{5.062473in}{2.369739in}}%
\pgfpathlineto{\pgfqpoint{5.055343in}{2.363246in}}%
\pgfpathlineto{\pgfqpoint{5.048207in}{2.356711in}}%
\pgfpathlineto{\pgfqpoint{5.034941in}{2.358365in}}%
\pgfpathlineto{\pgfqpoint{5.021683in}{2.360044in}}%
\pgfpathlineto{\pgfqpoint{5.008433in}{2.361747in}}%
\pgfpathlineto{\pgfqpoint{4.995190in}{2.363474in}}%
\pgfpathlineto{\pgfqpoint{5.002340in}{2.370054in}}%
\pgfpathlineto{\pgfqpoint{5.009484in}{2.376596in}}%
\pgfpathlineto{\pgfqpoint{5.016622in}{2.383101in}}%
\pgfpathlineto{\pgfqpoint{5.023754in}{2.389570in}}%
\pgfpathclose%
\pgfusepath{fill}%
\end{pgfscope}%
\begin{pgfscope}%
\pgfpathrectangle{\pgfqpoint{1.254980in}{0.150000in}}{\pgfqpoint{5.490039in}{5.490039in}}%
\pgfusepath{clip}%
\pgfsetbuttcap%
\pgfsetroundjoin%
\definecolor{currentfill}{rgb}{0.282623,0.140926,0.457517}%
\pgfsetfillcolor{currentfill}%
\pgfsetfillopacity{0.700000}%
\pgfsetlinewidth{0.000000pt}%
\definecolor{currentstroke}{rgb}{0.000000,0.000000,0.000000}%
\pgfsetstrokecolor{currentstroke}%
\pgfsetdash{}{0pt}%
\pgfpathmoveto{\pgfqpoint{5.670787in}{2.471397in}}%
\pgfpathlineto{\pgfqpoint{5.684193in}{2.469901in}}%
\pgfpathlineto{\pgfqpoint{5.697606in}{2.468427in}}%
\pgfpathlineto{\pgfqpoint{5.711027in}{2.466977in}}%
\pgfpathlineto{\pgfqpoint{5.724456in}{2.465551in}}%
\pgfpathlineto{\pgfqpoint{5.717628in}{2.460433in}}%
\pgfpathlineto{\pgfqpoint{5.710796in}{2.455323in}}%
\pgfpathlineto{\pgfqpoint{5.703958in}{2.450214in}}%
\pgfpathlineto{\pgfqpoint{5.697114in}{2.445103in}}%
\pgfpathlineto{\pgfqpoint{5.683666in}{2.446402in}}%
\pgfpathlineto{\pgfqpoint{5.670226in}{2.447724in}}%
\pgfpathlineto{\pgfqpoint{5.656794in}{2.449069in}}%
\pgfpathlineto{\pgfqpoint{5.643370in}{2.450438in}}%
\pgfpathlineto{\pgfqpoint{5.650233in}{2.455672in}}%
\pgfpathlineto{\pgfqpoint{5.657089in}{2.460907in}}%
\pgfpathlineto{\pgfqpoint{5.663941in}{2.466147in}}%
\pgfpathlineto{\pgfqpoint{5.670787in}{2.471397in}}%
\pgfpathclose%
\pgfusepath{fill}%
\end{pgfscope}%
\begin{pgfscope}%
\pgfpathrectangle{\pgfqpoint{1.254980in}{0.150000in}}{\pgfqpoint{5.490039in}{5.490039in}}%
\pgfusepath{clip}%
\pgfsetbuttcap%
\pgfsetroundjoin%
\definecolor{currentfill}{rgb}{0.283091,0.110553,0.431554}%
\pgfsetfillcolor{currentfill}%
\pgfsetfillopacity{0.700000}%
\pgfsetlinewidth{0.000000pt}%
\definecolor{currentstroke}{rgb}{0.000000,0.000000,0.000000}%
\pgfsetstrokecolor{currentstroke}%
\pgfsetdash{}{0pt}%
\pgfpathmoveto{\pgfqpoint{5.239484in}{2.418925in}}%
\pgfpathlineto{\pgfqpoint{5.252773in}{2.417297in}}%
\pgfpathlineto{\pgfqpoint{5.266070in}{2.415694in}}%
\pgfpathlineto{\pgfqpoint{5.279374in}{2.414114in}}%
\pgfpathlineto{\pgfqpoint{5.292686in}{2.412558in}}%
\pgfpathlineto{\pgfqpoint{5.285662in}{2.406612in}}%
\pgfpathlineto{\pgfqpoint{5.278631in}{2.400637in}}%
\pgfpathlineto{\pgfqpoint{5.271595in}{2.394629in}}%
\pgfpathlineto{\pgfqpoint{5.264553in}{2.388588in}}%
\pgfpathlineto{\pgfqpoint{5.251225in}{2.390068in}}%
\pgfpathlineto{\pgfqpoint{5.237906in}{2.391571in}}%
\pgfpathlineto{\pgfqpoint{5.224594in}{2.393099in}}%
\pgfpathlineto{\pgfqpoint{5.211289in}{2.394650in}}%
\pgfpathlineto{\pgfqpoint{5.218347in}{2.400763in}}%
\pgfpathlineto{\pgfqpoint{5.225399in}{2.406845in}}%
\pgfpathlineto{\pgfqpoint{5.232444in}{2.412898in}}%
\pgfpathlineto{\pgfqpoint{5.239484in}{2.418925in}}%
\pgfpathclose%
\pgfusepath{fill}%
\end{pgfscope}%
\begin{pgfscope}%
\pgfpathrectangle{\pgfqpoint{1.254980in}{0.150000in}}{\pgfqpoint{5.490039in}{5.490039in}}%
\pgfusepath{clip}%
\pgfsetbuttcap%
\pgfsetroundjoin%
\definecolor{currentfill}{rgb}{0.283187,0.125848,0.444960}%
\pgfsetfillcolor{currentfill}%
\pgfsetfillopacity{0.700000}%
\pgfsetlinewidth{0.000000pt}%
\definecolor{currentstroke}{rgb}{0.000000,0.000000,0.000000}%
\pgfsetstrokecolor{currentstroke}%
\pgfsetdash{}{0pt}%
\pgfpathmoveto{\pgfqpoint{5.455173in}{2.446301in}}%
\pgfpathlineto{\pgfqpoint{5.468521in}{2.444767in}}%
\pgfpathlineto{\pgfqpoint{5.481877in}{2.443257in}}%
\pgfpathlineto{\pgfqpoint{5.495240in}{2.441769in}}%
\pgfpathlineto{\pgfqpoint{5.508612in}{2.440306in}}%
\pgfpathlineto{\pgfqpoint{5.501685in}{2.434808in}}%
\pgfpathlineto{\pgfqpoint{5.494753in}{2.429295in}}%
\pgfpathlineto{\pgfqpoint{5.487814in}{2.423765in}}%
\pgfpathlineto{\pgfqpoint{5.480870in}{2.418213in}}%
\pgfpathlineto{\pgfqpoint{5.467481in}{2.419575in}}%
\pgfpathlineto{\pgfqpoint{5.454101in}{2.420960in}}%
\pgfpathlineto{\pgfqpoint{5.440728in}{2.422368in}}%
\pgfpathlineto{\pgfqpoint{5.427363in}{2.423800in}}%
\pgfpathlineto{\pgfqpoint{5.434325in}{2.429449in}}%
\pgfpathlineto{\pgfqpoint{5.441280in}{2.435080in}}%
\pgfpathlineto{\pgfqpoint{5.448229in}{2.440696in}}%
\pgfpathlineto{\pgfqpoint{5.455173in}{2.446301in}}%
\pgfpathclose%
\pgfusepath{fill}%
\end{pgfscope}%
\begin{pgfscope}%
\pgfpathrectangle{\pgfqpoint{1.254980in}{0.150000in}}{\pgfqpoint{5.490039in}{5.490039in}}%
\pgfusepath{clip}%
\pgfsetbuttcap%
\pgfsetroundjoin%
\definecolor{currentfill}{rgb}{0.267004,0.004874,0.329415}%
\pgfsetfillcolor{currentfill}%
\pgfsetfillopacity{0.700000}%
\pgfsetlinewidth{0.000000pt}%
\definecolor{currentstroke}{rgb}{0.000000,0.000000,0.000000}%
\pgfsetstrokecolor{currentstroke}%
\pgfsetdash{}{0pt}%
\pgfpathmoveto{\pgfqpoint{3.729489in}{2.236595in}}%
\pgfpathlineto{\pgfqpoint{3.742399in}{2.232647in}}%
\pgfpathlineto{\pgfqpoint{3.755315in}{2.228729in}}%
\pgfpathlineto{\pgfqpoint{3.768236in}{2.224842in}}%
\pgfpathlineto{\pgfqpoint{3.781163in}{2.220985in}}%
\pgfpathlineto{\pgfqpoint{3.773549in}{2.213860in}}%
\pgfpathlineto{\pgfqpoint{3.765929in}{2.206761in}}%
\pgfpathlineto{\pgfqpoint{3.758303in}{2.199690in}}%
\pgfpathlineto{\pgfqpoint{3.750671in}{2.192649in}}%
\pgfpathlineto{\pgfqpoint{3.737732in}{2.196610in}}%
\pgfpathlineto{\pgfqpoint{3.724797in}{2.200601in}}%
\pgfpathlineto{\pgfqpoint{3.711868in}{2.204623in}}%
\pgfpathlineto{\pgfqpoint{3.698945in}{2.208674in}}%
\pgfpathlineto{\pgfqpoint{3.706590in}{2.215607in}}%
\pgfpathlineto{\pgfqpoint{3.714229in}{2.222572in}}%
\pgfpathlineto{\pgfqpoint{3.721862in}{2.229569in}}%
\pgfpathlineto{\pgfqpoint{3.729489in}{2.236595in}}%
\pgfpathclose%
\pgfusepath{fill}%
\end{pgfscope}%
\begin{pgfscope}%
\pgfpathrectangle{\pgfqpoint{1.254980in}{0.150000in}}{\pgfqpoint{5.490039in}{5.490039in}}%
\pgfusepath{clip}%
\pgfsetbuttcap%
\pgfsetroundjoin%
\definecolor{currentfill}{rgb}{0.277018,0.050344,0.375715}%
\pgfsetfillcolor{currentfill}%
\pgfsetfillopacity{0.700000}%
\pgfsetlinewidth{0.000000pt}%
\definecolor{currentstroke}{rgb}{0.000000,0.000000,0.000000}%
\pgfsetstrokecolor{currentstroke}%
\pgfsetdash{}{0pt}%
\pgfpathmoveto{\pgfqpoint{3.060305in}{2.315550in}}%
\pgfpathlineto{\pgfqpoint{3.073110in}{2.309568in}}%
\pgfpathlineto{\pgfqpoint{3.085918in}{2.303626in}}%
\pgfpathlineto{\pgfqpoint{3.098730in}{2.297723in}}%
\pgfpathlineto{\pgfqpoint{3.111546in}{2.291860in}}%
\pgfpathlineto{\pgfqpoint{3.103633in}{2.287415in}}%
\pgfpathlineto{\pgfqpoint{3.095711in}{2.283090in}}%
\pgfpathlineto{\pgfqpoint{3.087780in}{2.278889in}}%
\pgfpathlineto{\pgfqpoint{3.079839in}{2.274817in}}%
\pgfpathlineto{\pgfqpoint{3.067004in}{2.280848in}}%
\pgfpathlineto{\pgfqpoint{3.054173in}{2.286919in}}%
\pgfpathlineto{\pgfqpoint{3.041345in}{2.293030in}}%
\pgfpathlineto{\pgfqpoint{3.028521in}{2.299180in}}%
\pgfpathlineto{\pgfqpoint{3.036481in}{2.303079in}}%
\pgfpathlineto{\pgfqpoint{3.044432in}{2.307109in}}%
\pgfpathlineto{\pgfqpoint{3.052373in}{2.311268in}}%
\pgfpathlineto{\pgfqpoint{3.060305in}{2.315550in}}%
\pgfpathclose%
\pgfusepath{fill}%
\end{pgfscope}%
\begin{pgfscope}%
\pgfpathrectangle{\pgfqpoint{1.254980in}{0.150000in}}{\pgfqpoint{5.490039in}{5.490039in}}%
\pgfusepath{clip}%
\pgfsetbuttcap%
\pgfsetroundjoin%
\definecolor{currentfill}{rgb}{0.272594,0.025563,0.353093}%
\pgfsetfillcolor{currentfill}%
\pgfsetfillopacity{0.700000}%
\pgfsetlinewidth{0.000000pt}%
\definecolor{currentstroke}{rgb}{0.000000,0.000000,0.000000}%
\pgfsetstrokecolor{currentstroke}%
\pgfsetdash{}{0pt}%
\pgfpathmoveto{\pgfqpoint{4.294852in}{2.279001in}}%
\pgfpathlineto{\pgfqpoint{4.307892in}{2.276289in}}%
\pgfpathlineto{\pgfqpoint{4.320939in}{2.273604in}}%
\pgfpathlineto{\pgfqpoint{4.333991in}{2.270945in}}%
\pgfpathlineto{\pgfqpoint{4.347051in}{2.268313in}}%
\pgfpathlineto{\pgfqpoint{4.339644in}{2.260735in}}%
\pgfpathlineto{\pgfqpoint{4.332232in}{2.253131in}}%
\pgfpathlineto{\pgfqpoint{4.324814in}{2.245500in}}%
\pgfpathlineto{\pgfqpoint{4.317391in}{2.237843in}}%
\pgfpathlineto{\pgfqpoint{4.304320in}{2.240515in}}%
\pgfpathlineto{\pgfqpoint{4.291256in}{2.243214in}}%
\pgfpathlineto{\pgfqpoint{4.278198in}{2.245939in}}%
\pgfpathlineto{\pgfqpoint{4.265146in}{2.248691in}}%
\pgfpathlineto{\pgfqpoint{4.272581in}{2.256303in}}%
\pgfpathlineto{\pgfqpoint{4.280010in}{2.263893in}}%
\pgfpathlineto{\pgfqpoint{4.287434in}{2.271459in}}%
\pgfpathlineto{\pgfqpoint{4.294852in}{2.279001in}}%
\pgfpathclose%
\pgfusepath{fill}%
\end{pgfscope}%
\begin{pgfscope}%
\pgfpathrectangle{\pgfqpoint{1.254980in}{0.150000in}}{\pgfqpoint{5.490039in}{5.490039in}}%
\pgfusepath{clip}%
\pgfsetbuttcap%
\pgfsetroundjoin%
\definecolor{currentfill}{rgb}{0.269944,0.014625,0.341379}%
\pgfsetfillcolor{currentfill}%
\pgfsetfillopacity{0.700000}%
\pgfsetlinewidth{0.000000pt}%
\definecolor{currentstroke}{rgb}{0.000000,0.000000,0.000000}%
\pgfsetstrokecolor{currentstroke}%
\pgfsetdash{}{0pt}%
\pgfpathmoveto{\pgfqpoint{4.079086in}{2.253799in}}%
\pgfpathlineto{\pgfqpoint{4.092075in}{2.250668in}}%
\pgfpathlineto{\pgfqpoint{4.105069in}{2.247564in}}%
\pgfpathlineto{\pgfqpoint{4.118069in}{2.244488in}}%
\pgfpathlineto{\pgfqpoint{4.131076in}{2.241439in}}%
\pgfpathlineto{\pgfqpoint{4.123590in}{2.233848in}}%
\pgfpathlineto{\pgfqpoint{4.116099in}{2.226245in}}%
\pgfpathlineto{\pgfqpoint{4.108603in}{2.218634in}}%
\pgfpathlineto{\pgfqpoint{4.101101in}{2.211013in}}%
\pgfpathlineto{\pgfqpoint{4.088083in}{2.214128in}}%
\pgfpathlineto{\pgfqpoint{4.075071in}{2.217270in}}%
\pgfpathlineto{\pgfqpoint{4.062065in}{2.220439in}}%
\pgfpathlineto{\pgfqpoint{4.049065in}{2.223636in}}%
\pgfpathlineto{\pgfqpoint{4.056578in}{2.231186in}}%
\pgfpathlineto{\pgfqpoint{4.064086in}{2.238731in}}%
\pgfpathlineto{\pgfqpoint{4.071589in}{2.246269in}}%
\pgfpathlineto{\pgfqpoint{4.079086in}{2.253799in}}%
\pgfpathclose%
\pgfusepath{fill}%
\end{pgfscope}%
\begin{pgfscope}%
\pgfpathrectangle{\pgfqpoint{1.254980in}{0.150000in}}{\pgfqpoint{5.490039in}{5.490039in}}%
\pgfusepath{clip}%
\pgfsetbuttcap%
\pgfsetroundjoin%
\definecolor{currentfill}{rgb}{0.277018,0.050344,0.375715}%
\pgfsetfillcolor{currentfill}%
\pgfsetfillopacity{0.700000}%
\pgfsetlinewidth{0.000000pt}%
\definecolor{currentstroke}{rgb}{0.000000,0.000000,0.000000}%
\pgfsetstrokecolor{currentstroke}%
\pgfsetdash{}{0pt}%
\pgfpathmoveto{\pgfqpoint{4.510630in}{2.308145in}}%
\pgfpathlineto{\pgfqpoint{4.523726in}{2.305789in}}%
\pgfpathlineto{\pgfqpoint{4.536828in}{2.303460in}}%
\pgfpathlineto{\pgfqpoint{4.549937in}{2.301155in}}%
\pgfpathlineto{\pgfqpoint{4.563053in}{2.298877in}}%
\pgfpathlineto{\pgfqpoint{4.555726in}{2.291495in}}%
\pgfpathlineto{\pgfqpoint{4.548394in}{2.284076in}}%
\pgfpathlineto{\pgfqpoint{4.541055in}{2.276619in}}%
\pgfpathlineto{\pgfqpoint{4.533711in}{2.269124in}}%
\pgfpathlineto{\pgfqpoint{4.520584in}{2.271417in}}%
\pgfpathlineto{\pgfqpoint{4.507463in}{2.273735in}}%
\pgfpathlineto{\pgfqpoint{4.494349in}{2.276080in}}%
\pgfpathlineto{\pgfqpoint{4.481241in}{2.278450in}}%
\pgfpathlineto{\pgfqpoint{4.488597in}{2.285926in}}%
\pgfpathlineto{\pgfqpoint{4.495947in}{2.293367in}}%
\pgfpathlineto{\pgfqpoint{4.503291in}{2.300773in}}%
\pgfpathlineto{\pgfqpoint{4.510630in}{2.308145in}}%
\pgfpathclose%
\pgfusepath{fill}%
\end{pgfscope}%
\begin{pgfscope}%
\pgfpathrectangle{\pgfqpoint{1.254980in}{0.150000in}}{\pgfqpoint{5.490039in}{5.490039in}}%
\pgfusepath{clip}%
\pgfsetbuttcap%
\pgfsetroundjoin%
\definecolor{currentfill}{rgb}{0.279566,0.067836,0.391917}%
\pgfsetfillcolor{currentfill}%
\pgfsetfillopacity{0.700000}%
\pgfsetlinewidth{0.000000pt}%
\definecolor{currentstroke}{rgb}{0.000000,0.000000,0.000000}%
\pgfsetstrokecolor{currentstroke}%
\pgfsetdash{}{0pt}%
\pgfpathmoveto{\pgfqpoint{4.726444in}{2.339228in}}%
\pgfpathlineto{\pgfqpoint{4.739598in}{2.337167in}}%
\pgfpathlineto{\pgfqpoint{4.752759in}{2.335132in}}%
\pgfpathlineto{\pgfqpoint{4.765927in}{2.333121in}}%
\pgfpathlineto{\pgfqpoint{4.779102in}{2.331136in}}%
\pgfpathlineto{\pgfqpoint{4.771858in}{2.324088in}}%
\pgfpathlineto{\pgfqpoint{4.764609in}{2.316997in}}%
\pgfpathlineto{\pgfqpoint{4.757353in}{2.309863in}}%
\pgfpathlineto{\pgfqpoint{4.750091in}{2.302684in}}%
\pgfpathlineto{\pgfqpoint{4.736904in}{2.304658in}}%
\pgfpathlineto{\pgfqpoint{4.723723in}{2.306657in}}%
\pgfpathlineto{\pgfqpoint{4.710550in}{2.308682in}}%
\pgfpathlineto{\pgfqpoint{4.697383in}{2.310731in}}%
\pgfpathlineto{\pgfqpoint{4.704657in}{2.317916in}}%
\pgfpathlineto{\pgfqpoint{4.711925in}{2.325060in}}%
\pgfpathlineto{\pgfqpoint{4.719187in}{2.332164in}}%
\pgfpathlineto{\pgfqpoint{4.726444in}{2.339228in}}%
\pgfpathclose%
\pgfusepath{fill}%
\end{pgfscope}%
\begin{pgfscope}%
\pgfpathrectangle{\pgfqpoint{1.254980in}{0.150000in}}{\pgfqpoint{5.490039in}{5.490039in}}%
\pgfusepath{clip}%
\pgfsetbuttcap%
\pgfsetroundjoin%
\definecolor{currentfill}{rgb}{0.280868,0.160771,0.472899}%
\pgfsetfillcolor{currentfill}%
\pgfsetfillopacity{0.700000}%
\pgfsetlinewidth{0.000000pt}%
\definecolor{currentstroke}{rgb}{0.000000,0.000000,0.000000}%
\pgfsetstrokecolor{currentstroke}%
\pgfsetdash{}{0pt}%
\pgfpathmoveto{\pgfqpoint{6.020996in}{2.501979in}}%
\pgfpathlineto{\pgfqpoint{6.034499in}{2.500482in}}%
\pgfpathlineto{\pgfqpoint{6.048010in}{2.499008in}}%
\pgfpathlineto{\pgfqpoint{6.061528in}{2.497556in}}%
\pgfpathlineto{\pgfqpoint{6.075056in}{2.496128in}}%
\pgfpathlineto{\pgfqpoint{6.068386in}{2.491441in}}%
\pgfpathlineto{\pgfqpoint{6.061713in}{2.486800in}}%
\pgfpathlineto{\pgfqpoint{6.055036in}{2.482201in}}%
\pgfpathlineto{\pgfqpoint{6.048355in}{2.477638in}}%
\pgfpathlineto{\pgfqpoint{6.034806in}{2.478900in}}%
\pgfpathlineto{\pgfqpoint{6.021264in}{2.480185in}}%
\pgfpathlineto{\pgfqpoint{6.007732in}{2.481492in}}%
\pgfpathlineto{\pgfqpoint{5.994207in}{2.482823in}}%
\pgfpathlineto{\pgfqpoint{6.000910in}{2.487548in}}%
\pgfpathlineto{\pgfqpoint{6.007609in}{2.492312in}}%
\pgfpathlineto{\pgfqpoint{6.014304in}{2.497121in}}%
\pgfpathlineto{\pgfqpoint{6.020996in}{2.501979in}}%
\pgfpathclose%
\pgfusepath{fill}%
\end{pgfscope}%
\begin{pgfscope}%
\pgfpathrectangle{\pgfqpoint{1.254980in}{0.150000in}}{\pgfqpoint{5.490039in}{5.490039in}}%
\pgfusepath{clip}%
\pgfsetbuttcap%
\pgfsetroundjoin%
\definecolor{currentfill}{rgb}{0.267004,0.004874,0.329415}%
\pgfsetfillcolor{currentfill}%
\pgfsetfillopacity{0.700000}%
\pgfsetlinewidth{0.000000pt}%
\definecolor{currentstroke}{rgb}{0.000000,0.000000,0.000000}%
\pgfsetstrokecolor{currentstroke}%
\pgfsetdash{}{0pt}%
\pgfpathmoveto{\pgfqpoint{3.863272in}{2.234932in}}%
\pgfpathlineto{\pgfqpoint{3.876213in}{2.231314in}}%
\pgfpathlineto{\pgfqpoint{3.889161in}{2.227726in}}%
\pgfpathlineto{\pgfqpoint{3.902114in}{2.224166in}}%
\pgfpathlineto{\pgfqpoint{3.915073in}{2.220635in}}%
\pgfpathlineto{\pgfqpoint{3.907507in}{2.213256in}}%
\pgfpathlineto{\pgfqpoint{3.899936in}{2.205888in}}%
\pgfpathlineto{\pgfqpoint{3.892359in}{2.198533in}}%
\pgfpathlineto{\pgfqpoint{3.884777in}{2.191192in}}%
\pgfpathlineto{\pgfqpoint{3.871805in}{2.194814in}}%
\pgfpathlineto{\pgfqpoint{3.858839in}{2.198465in}}%
\pgfpathlineto{\pgfqpoint{3.845879in}{2.202145in}}%
\pgfpathlineto{\pgfqpoint{3.832925in}{2.205854in}}%
\pgfpathlineto{\pgfqpoint{3.840520in}{2.213098in}}%
\pgfpathlineto{\pgfqpoint{3.848109in}{2.220361in}}%
\pgfpathlineto{\pgfqpoint{3.855693in}{2.227639in}}%
\pgfpathlineto{\pgfqpoint{3.863272in}{2.234932in}}%
\pgfpathclose%
\pgfusepath{fill}%
\end{pgfscope}%
\begin{pgfscope}%
\pgfpathrectangle{\pgfqpoint{1.254980in}{0.150000in}}{\pgfqpoint{5.490039in}{5.490039in}}%
\pgfusepath{clip}%
\pgfsetbuttcap%
\pgfsetroundjoin%
\definecolor{currentfill}{rgb}{0.283197,0.115680,0.436115}%
\pgfsetfillcolor{currentfill}%
\pgfsetfillopacity{0.700000}%
\pgfsetlinewidth{0.000000pt}%
\definecolor{currentstroke}{rgb}{0.000000,0.000000,0.000000}%
\pgfsetstrokecolor{currentstroke}%
\pgfsetdash{}{0pt}%
\pgfpathmoveto{\pgfqpoint{2.740359in}{2.420607in}}%
\pgfpathlineto{\pgfqpoint{2.753143in}{2.413394in}}%
\pgfpathlineto{\pgfqpoint{2.765929in}{2.406229in}}%
\pgfpathlineto{\pgfqpoint{2.778717in}{2.399112in}}%
\pgfpathlineto{\pgfqpoint{2.791509in}{2.392042in}}%
\pgfpathlineto{\pgfqpoint{2.783408in}{2.389629in}}%
\pgfpathlineto{\pgfqpoint{2.775295in}{2.387388in}}%
\pgfpathlineto{\pgfqpoint{2.767171in}{2.385324in}}%
\pgfpathlineto{\pgfqpoint{2.759034in}{2.383443in}}%
\pgfpathlineto{\pgfqpoint{2.746218in}{2.390709in}}%
\pgfpathlineto{\pgfqpoint{2.733406in}{2.398022in}}%
\pgfpathlineto{\pgfqpoint{2.720596in}{2.405382in}}%
\pgfpathlineto{\pgfqpoint{2.707789in}{2.412791in}}%
\pgfpathlineto{\pgfqpoint{2.715950in}{2.414472in}}%
\pgfpathlineto{\pgfqpoint{2.724099in}{2.416339in}}%
\pgfpathlineto{\pgfqpoint{2.732235in}{2.418385in}}%
\pgfpathlineto{\pgfqpoint{2.740359in}{2.420607in}}%
\pgfpathclose%
\pgfusepath{fill}%
\end{pgfscope}%
\begin{pgfscope}%
\pgfpathrectangle{\pgfqpoint{1.254980in}{0.150000in}}{\pgfqpoint{5.490039in}{5.490039in}}%
\pgfusepath{clip}%
\pgfsetbuttcap%
\pgfsetroundjoin%
\definecolor{currentfill}{rgb}{0.281924,0.089666,0.412415}%
\pgfsetfillcolor{currentfill}%
\pgfsetfillopacity{0.700000}%
\pgfsetlinewidth{0.000000pt}%
\definecolor{currentstroke}{rgb}{0.000000,0.000000,0.000000}%
\pgfsetstrokecolor{currentstroke}%
\pgfsetdash{}{0pt}%
\pgfpathmoveto{\pgfqpoint{4.942291in}{2.370626in}}%
\pgfpathlineto{\pgfqpoint{4.955504in}{2.368801in}}%
\pgfpathlineto{\pgfqpoint{4.968725in}{2.367001in}}%
\pgfpathlineto{\pgfqpoint{4.981954in}{2.365225in}}%
\pgfpathlineto{\pgfqpoint{4.995190in}{2.363474in}}%
\pgfpathlineto{\pgfqpoint{4.988033in}{2.356853in}}%
\pgfpathlineto{\pgfqpoint{4.980871in}{2.350189in}}%
\pgfpathlineto{\pgfqpoint{4.973703in}{2.343482in}}%
\pgfpathlineto{\pgfqpoint{4.966528in}{2.336729in}}%
\pgfpathlineto{\pgfqpoint{4.953279in}{2.338443in}}%
\pgfpathlineto{\pgfqpoint{4.940037in}{2.340182in}}%
\pgfpathlineto{\pgfqpoint{4.926802in}{2.341945in}}%
\pgfpathlineto{\pgfqpoint{4.913575in}{2.343733in}}%
\pgfpathlineto{\pgfqpoint{4.920763in}{2.350518in}}%
\pgfpathlineto{\pgfqpoint{4.927945in}{2.357261in}}%
\pgfpathlineto{\pgfqpoint{4.935121in}{2.363963in}}%
\pgfpathlineto{\pgfqpoint{4.942291in}{2.370626in}}%
\pgfpathclose%
\pgfusepath{fill}%
\end{pgfscope}%
\begin{pgfscope}%
\pgfpathrectangle{\pgfqpoint{1.254980in}{0.150000in}}{\pgfqpoint{5.490039in}{5.490039in}}%
\pgfusepath{clip}%
\pgfsetbuttcap%
\pgfsetroundjoin%
\definecolor{currentfill}{rgb}{0.281887,0.150881,0.465405}%
\pgfsetfillcolor{currentfill}%
\pgfsetfillopacity{0.700000}%
\pgfsetlinewidth{0.000000pt}%
\definecolor{currentstroke}{rgb}{0.000000,0.000000,0.000000}%
\pgfsetstrokecolor{currentstroke}%
\pgfsetdash{}{0pt}%
\pgfpathmoveto{\pgfqpoint{5.805432in}{2.480126in}}%
\pgfpathlineto{\pgfqpoint{5.818880in}{2.478674in}}%
\pgfpathlineto{\pgfqpoint{5.832337in}{2.477245in}}%
\pgfpathlineto{\pgfqpoint{5.845802in}{2.475839in}}%
\pgfpathlineto{\pgfqpoint{5.859275in}{2.474456in}}%
\pgfpathlineto{\pgfqpoint{5.852508in}{2.469562in}}%
\pgfpathlineto{\pgfqpoint{5.845735in}{2.464686in}}%
\pgfpathlineto{\pgfqpoint{5.838958in}{2.459823in}}%
\pgfpathlineto{\pgfqpoint{5.832175in}{2.454968in}}%
\pgfpathlineto{\pgfqpoint{5.818682in}{2.456210in}}%
\pgfpathlineto{\pgfqpoint{5.805197in}{2.457475in}}%
\pgfpathlineto{\pgfqpoint{5.791720in}{2.458763in}}%
\pgfpathlineto{\pgfqpoint{5.778251in}{2.460074in}}%
\pgfpathlineto{\pgfqpoint{5.785054in}{2.465066in}}%
\pgfpathlineto{\pgfqpoint{5.791851in}{2.470069in}}%
\pgfpathlineto{\pgfqpoint{5.798644in}{2.475087in}}%
\pgfpathlineto{\pgfqpoint{5.805432in}{2.480126in}}%
\pgfpathclose%
\pgfusepath{fill}%
\end{pgfscope}%
\begin{pgfscope}%
\pgfpathrectangle{\pgfqpoint{1.254980in}{0.150000in}}{\pgfqpoint{5.490039in}{5.490039in}}%
\pgfusepath{clip}%
\pgfsetbuttcap%
\pgfsetroundjoin%
\definecolor{currentfill}{rgb}{0.280267,0.073417,0.397163}%
\pgfsetfillcolor{currentfill}%
\pgfsetfillopacity{0.700000}%
\pgfsetlinewidth{0.000000pt}%
\definecolor{currentstroke}{rgb}{0.000000,0.000000,0.000000}%
\pgfsetstrokecolor{currentstroke}%
\pgfsetdash{}{0pt}%
\pgfpathmoveto{\pgfqpoint{2.926054in}{2.349864in}}%
\pgfpathlineto{\pgfqpoint{2.938851in}{2.343381in}}%
\pgfpathlineto{\pgfqpoint{2.951650in}{2.336941in}}%
\pgfpathlineto{\pgfqpoint{2.964454in}{2.330543in}}%
\pgfpathlineto{\pgfqpoint{2.977260in}{2.324188in}}%
\pgfpathlineto{\pgfqpoint{2.969269in}{2.320602in}}%
\pgfpathlineto{\pgfqpoint{2.961268in}{2.317160in}}%
\pgfpathlineto{\pgfqpoint{2.953257in}{2.313866in}}%
\pgfpathlineto{\pgfqpoint{2.945234in}{2.310725in}}%
\pgfpathlineto{\pgfqpoint{2.932407in}{2.317262in}}%
\pgfpathlineto{\pgfqpoint{2.919582in}{2.323842in}}%
\pgfpathlineto{\pgfqpoint{2.906761in}{2.330463in}}%
\pgfpathlineto{\pgfqpoint{2.893943in}{2.337128in}}%
\pgfpathlineto{\pgfqpoint{2.901987in}{2.340082in}}%
\pgfpathlineto{\pgfqpoint{2.910020in}{2.343193in}}%
\pgfpathlineto{\pgfqpoint{2.918042in}{2.346455in}}%
\pgfpathlineto{\pgfqpoint{2.926054in}{2.349864in}}%
\pgfpathclose%
\pgfusepath{fill}%
\end{pgfscope}%
\begin{pgfscope}%
\pgfpathrectangle{\pgfqpoint{1.254980in}{0.150000in}}{\pgfqpoint{5.490039in}{5.490039in}}%
\pgfusepath{clip}%
\pgfsetbuttcap%
\pgfsetroundjoin%
\definecolor{currentfill}{rgb}{0.282910,0.105393,0.426902}%
\pgfsetfillcolor{currentfill}%
\pgfsetfillopacity{0.700000}%
\pgfsetlinewidth{0.000000pt}%
\definecolor{currentstroke}{rgb}{0.000000,0.000000,0.000000}%
\pgfsetstrokecolor{currentstroke}%
\pgfsetdash{}{0pt}%
\pgfpathmoveto{\pgfqpoint{5.158148in}{2.401095in}}%
\pgfpathlineto{\pgfqpoint{5.171422in}{2.399447in}}%
\pgfpathlineto{\pgfqpoint{5.184703in}{2.397824in}}%
\pgfpathlineto{\pgfqpoint{5.197993in}{2.396225in}}%
\pgfpathlineto{\pgfqpoint{5.211289in}{2.394650in}}%
\pgfpathlineto{\pgfqpoint{5.204225in}{2.388503in}}%
\pgfpathlineto{\pgfqpoint{5.197155in}{2.382321in}}%
\pgfpathlineto{\pgfqpoint{5.190079in}{2.376100in}}%
\pgfpathlineto{\pgfqpoint{5.182996in}{2.369838in}}%
\pgfpathlineto{\pgfqpoint{5.169685in}{2.371350in}}%
\pgfpathlineto{\pgfqpoint{5.156381in}{2.372886in}}%
\pgfpathlineto{\pgfqpoint{5.143084in}{2.374446in}}%
\pgfpathlineto{\pgfqpoint{5.129795in}{2.376030in}}%
\pgfpathlineto{\pgfqpoint{5.136893in}{2.382350in}}%
\pgfpathlineto{\pgfqpoint{5.143984in}{2.388632in}}%
\pgfpathlineto{\pgfqpoint{5.151069in}{2.394880in}}%
\pgfpathlineto{\pgfqpoint{5.158148in}{2.401095in}}%
\pgfpathclose%
\pgfusepath{fill}%
\end{pgfscope}%
\begin{pgfscope}%
\pgfpathrectangle{\pgfqpoint{1.254980in}{0.150000in}}{\pgfqpoint{5.490039in}{5.490039in}}%
\pgfusepath{clip}%
\pgfsetbuttcap%
\pgfsetroundjoin%
\definecolor{currentfill}{rgb}{0.282884,0.135920,0.453427}%
\pgfsetfillcolor{currentfill}%
\pgfsetfillopacity{0.700000}%
\pgfsetlinewidth{0.000000pt}%
\definecolor{currentstroke}{rgb}{0.000000,0.000000,0.000000}%
\pgfsetstrokecolor{currentstroke}%
\pgfsetdash{}{0pt}%
\pgfpathmoveto{\pgfqpoint{5.589753in}{2.456145in}}%
\pgfpathlineto{\pgfqpoint{5.603146in}{2.454683in}}%
\pgfpathlineto{\pgfqpoint{5.616546in}{2.453245in}}%
\pgfpathlineto{\pgfqpoint{5.629954in}{2.451830in}}%
\pgfpathlineto{\pgfqpoint{5.643370in}{2.450438in}}%
\pgfpathlineto{\pgfqpoint{5.636503in}{2.445202in}}%
\pgfpathlineto{\pgfqpoint{5.629629in}{2.439959in}}%
\pgfpathlineto{\pgfqpoint{5.622750in}{2.434706in}}%
\pgfpathlineto{\pgfqpoint{5.615865in}{2.429439in}}%
\pgfpathlineto{\pgfqpoint{5.602430in}{2.430716in}}%
\pgfpathlineto{\pgfqpoint{5.589004in}{2.432016in}}%
\pgfpathlineto{\pgfqpoint{5.575585in}{2.433339in}}%
\pgfpathlineto{\pgfqpoint{5.562175in}{2.434685in}}%
\pgfpathlineto{\pgfqpoint{5.569078in}{2.440063in}}%
\pgfpathlineto{\pgfqpoint{5.575975in}{2.445429in}}%
\pgfpathlineto{\pgfqpoint{5.582867in}{2.450789in}}%
\pgfpathlineto{\pgfqpoint{5.589753in}{2.456145in}}%
\pgfpathclose%
\pgfusepath{fill}%
\end{pgfscope}%
\begin{pgfscope}%
\pgfpathrectangle{\pgfqpoint{1.254980in}{0.150000in}}{\pgfqpoint{5.490039in}{5.490039in}}%
\pgfusepath{clip}%
\pgfsetbuttcap%
\pgfsetroundjoin%
\definecolor{currentfill}{rgb}{0.283229,0.120777,0.440584}%
\pgfsetfillcolor{currentfill}%
\pgfsetfillopacity{0.700000}%
\pgfsetlinewidth{0.000000pt}%
\definecolor{currentstroke}{rgb}{0.000000,0.000000,0.000000}%
\pgfsetstrokecolor{currentstroke}%
\pgfsetdash{}{0pt}%
\pgfpathmoveto{\pgfqpoint{5.373981in}{2.429764in}}%
\pgfpathlineto{\pgfqpoint{5.387315in}{2.428238in}}%
\pgfpathlineto{\pgfqpoint{5.400657in}{2.426735in}}%
\pgfpathlineto{\pgfqpoint{5.414006in}{2.425256in}}%
\pgfpathlineto{\pgfqpoint{5.427363in}{2.423800in}}%
\pgfpathlineto{\pgfqpoint{5.420396in}{2.418131in}}%
\pgfpathlineto{\pgfqpoint{5.413422in}{2.412437in}}%
\pgfpathlineto{\pgfqpoint{5.406443in}{2.406716in}}%
\pgfpathlineto{\pgfqpoint{5.399457in}{2.400965in}}%
\pgfpathlineto{\pgfqpoint{5.386083in}{2.402331in}}%
\pgfpathlineto{\pgfqpoint{5.372718in}{2.403721in}}%
\pgfpathlineto{\pgfqpoint{5.359360in}{2.405135in}}%
\pgfpathlineto{\pgfqpoint{5.346009in}{2.406572in}}%
\pgfpathlineto{\pgfqpoint{5.353011in}{2.412407in}}%
\pgfpathlineto{\pgfqpoint{5.360007in}{2.418216in}}%
\pgfpathlineto{\pgfqpoint{5.366997in}{2.424001in}}%
\pgfpathlineto{\pgfqpoint{5.373981in}{2.429764in}}%
\pgfpathclose%
\pgfusepath{fill}%
\end{pgfscope}%
\begin{pgfscope}%
\pgfpathrectangle{\pgfqpoint{1.254980in}{0.150000in}}{\pgfqpoint{5.490039in}{5.490039in}}%
\pgfusepath{clip}%
\pgfsetbuttcap%
\pgfsetroundjoin%
\definecolor{currentfill}{rgb}{0.269944,0.014625,0.341379}%
\pgfsetfillcolor{currentfill}%
\pgfsetfillopacity{0.700000}%
\pgfsetlinewidth{0.000000pt}%
\definecolor{currentstroke}{rgb}{0.000000,0.000000,0.000000}%
\pgfsetstrokecolor{currentstroke}%
\pgfsetdash{}{0pt}%
\pgfpathmoveto{\pgfqpoint{3.379584in}{2.247341in}}%
\pgfpathlineto{\pgfqpoint{3.392437in}{2.242398in}}%
\pgfpathlineto{\pgfqpoint{3.405295in}{2.237489in}}%
\pgfpathlineto{\pgfqpoint{3.418157in}{2.232614in}}%
\pgfpathlineto{\pgfqpoint{3.431024in}{2.227772in}}%
\pgfpathlineto{\pgfqpoint{3.423262in}{2.221813in}}%
\pgfpathlineto{\pgfqpoint{3.415492in}{2.215928in}}%
\pgfpathlineto{\pgfqpoint{3.407716in}{2.210121in}}%
\pgfpathlineto{\pgfqpoint{3.399931in}{2.204396in}}%
\pgfpathlineto{\pgfqpoint{3.387048in}{2.209380in}}%
\pgfpathlineto{\pgfqpoint{3.374170in}{2.214397in}}%
\pgfpathlineto{\pgfqpoint{3.361296in}{2.219448in}}%
\pgfpathlineto{\pgfqpoint{3.348427in}{2.224534in}}%
\pgfpathlineto{\pgfqpoint{3.356227in}{2.230112in}}%
\pgfpathlineto{\pgfqpoint{3.364020in}{2.235775in}}%
\pgfpathlineto{\pgfqpoint{3.371806in}{2.241519in}}%
\pgfpathlineto{\pgfqpoint{3.379584in}{2.247341in}}%
\pgfpathclose%
\pgfusepath{fill}%
\end{pgfscope}%
\begin{pgfscope}%
\pgfpathrectangle{\pgfqpoint{1.254980in}{0.150000in}}{\pgfqpoint{5.490039in}{5.490039in}}%
\pgfusepath{clip}%
\pgfsetbuttcap%
\pgfsetroundjoin%
\definecolor{currentfill}{rgb}{0.267004,0.004874,0.329415}%
\pgfsetfillcolor{currentfill}%
\pgfsetfillopacity{0.700000}%
\pgfsetlinewidth{0.000000pt}%
\definecolor{currentstroke}{rgb}{0.000000,0.000000,0.000000}%
\pgfsetstrokecolor{currentstroke}%
\pgfsetdash{}{0pt}%
\pgfpathmoveto{\pgfqpoint{3.513458in}{2.233782in}}%
\pgfpathlineto{\pgfqpoint{3.526334in}{2.229236in}}%
\pgfpathlineto{\pgfqpoint{3.539215in}{2.224722in}}%
\pgfpathlineto{\pgfqpoint{3.552101in}{2.220241in}}%
\pgfpathlineto{\pgfqpoint{3.564991in}{2.215791in}}%
\pgfpathlineto{\pgfqpoint{3.557287in}{2.209315in}}%
\pgfpathlineto{\pgfqpoint{3.549576in}{2.202894in}}%
\pgfpathlineto{\pgfqpoint{3.541858in}{2.196532in}}%
\pgfpathlineto{\pgfqpoint{3.534133in}{2.190233in}}%
\pgfpathlineto{\pgfqpoint{3.521227in}{2.194811in}}%
\pgfpathlineto{\pgfqpoint{3.508327in}{2.199422in}}%
\pgfpathlineto{\pgfqpoint{3.495431in}{2.204065in}}%
\pgfpathlineto{\pgfqpoint{3.482540in}{2.208741in}}%
\pgfpathlineto{\pgfqpoint{3.490280in}{2.214906in}}%
\pgfpathlineto{\pgfqpoint{3.498012in}{2.221137in}}%
\pgfpathlineto{\pgfqpoint{3.505739in}{2.227430in}}%
\pgfpathlineto{\pgfqpoint{3.513458in}{2.233782in}}%
\pgfpathclose%
\pgfusepath{fill}%
\end{pgfscope}%
\begin{pgfscope}%
\pgfpathrectangle{\pgfqpoint{1.254980in}{0.150000in}}{\pgfqpoint{5.490039in}{5.490039in}}%
\pgfusepath{clip}%
\pgfsetbuttcap%
\pgfsetroundjoin%
\definecolor{currentfill}{rgb}{0.272594,0.025563,0.353093}%
\pgfsetfillcolor{currentfill}%
\pgfsetfillopacity{0.700000}%
\pgfsetlinewidth{0.000000pt}%
\definecolor{currentstroke}{rgb}{0.000000,0.000000,0.000000}%
\pgfsetstrokecolor{currentstroke}%
\pgfsetdash{}{0pt}%
\pgfpathmoveto{\pgfqpoint{3.245632in}{2.266481in}}%
\pgfpathlineto{\pgfqpoint{3.258466in}{2.261112in}}%
\pgfpathlineto{\pgfqpoint{3.271305in}{2.255780in}}%
\pgfpathlineto{\pgfqpoint{3.284147in}{2.250484in}}%
\pgfpathlineto{\pgfqpoint{3.296994in}{2.245224in}}%
\pgfpathlineto{\pgfqpoint{3.289169in}{2.239884in}}%
\pgfpathlineto{\pgfqpoint{3.281336in}{2.234641in}}%
\pgfpathlineto{\pgfqpoint{3.273495in}{2.229496in}}%
\pgfpathlineto{\pgfqpoint{3.265646in}{2.224453in}}%
\pgfpathlineto{\pgfqpoint{3.252782in}{2.229869in}}%
\pgfpathlineto{\pgfqpoint{3.239921in}{2.235320in}}%
\pgfpathlineto{\pgfqpoint{3.227065in}{2.240808in}}%
\pgfpathlineto{\pgfqpoint{3.214214in}{2.246331in}}%
\pgfpathlineto{\pgfqpoint{3.222081in}{2.251213in}}%
\pgfpathlineto{\pgfqpoint{3.229939in}{2.256202in}}%
\pgfpathlineto{\pgfqpoint{3.237790in}{2.261292in}}%
\pgfpathlineto{\pgfqpoint{3.245632in}{2.266481in}}%
\pgfpathclose%
\pgfusepath{fill}%
\end{pgfscope}%
\begin{pgfscope}%
\pgfpathrectangle{\pgfqpoint{1.254980in}{0.150000in}}{\pgfqpoint{5.490039in}{5.490039in}}%
\pgfusepath{clip}%
\pgfsetbuttcap%
\pgfsetroundjoin%
\definecolor{currentfill}{rgb}{0.271305,0.019942,0.347269}%
\pgfsetfillcolor{currentfill}%
\pgfsetfillopacity{0.700000}%
\pgfsetlinewidth{0.000000pt}%
\definecolor{currentstroke}{rgb}{0.000000,0.000000,0.000000}%
\pgfsetstrokecolor{currentstroke}%
\pgfsetdash{}{0pt}%
\pgfpathmoveto{\pgfqpoint{4.213004in}{2.259967in}}%
\pgfpathlineto{\pgfqpoint{4.226030in}{2.257108in}}%
\pgfpathlineto{\pgfqpoint{4.239062in}{2.254275in}}%
\pgfpathlineto{\pgfqpoint{4.252101in}{2.251470in}}%
\pgfpathlineto{\pgfqpoint{4.265146in}{2.248691in}}%
\pgfpathlineto{\pgfqpoint{4.257706in}{2.241057in}}%
\pgfpathlineto{\pgfqpoint{4.250260in}{2.233402in}}%
\pgfpathlineto{\pgfqpoint{4.242809in}{2.225726in}}%
\pgfpathlineto{\pgfqpoint{4.235352in}{2.218031in}}%
\pgfpathlineto{\pgfqpoint{4.222296in}{2.220862in}}%
\pgfpathlineto{\pgfqpoint{4.209245in}{2.223721in}}%
\pgfpathlineto{\pgfqpoint{4.196201in}{2.226606in}}%
\pgfpathlineto{\pgfqpoint{4.183164in}{2.229518in}}%
\pgfpathlineto{\pgfqpoint{4.190632in}{2.237156in}}%
\pgfpathlineto{\pgfqpoint{4.198095in}{2.244777in}}%
\pgfpathlineto{\pgfqpoint{4.205552in}{2.252381in}}%
\pgfpathlineto{\pgfqpoint{4.213004in}{2.259967in}}%
\pgfpathclose%
\pgfusepath{fill}%
\end{pgfscope}%
\begin{pgfscope}%
\pgfpathrectangle{\pgfqpoint{1.254980in}{0.150000in}}{\pgfqpoint{5.490039in}{5.490039in}}%
\pgfusepath{clip}%
\pgfsetbuttcap%
\pgfsetroundjoin%
\definecolor{currentfill}{rgb}{0.274952,0.037752,0.364543}%
\pgfsetfillcolor{currentfill}%
\pgfsetfillopacity{0.700000}%
\pgfsetlinewidth{0.000000pt}%
\definecolor{currentstroke}{rgb}{0.000000,0.000000,0.000000}%
\pgfsetstrokecolor{currentstroke}%
\pgfsetdash{}{0pt}%
\pgfpathmoveto{\pgfqpoint{4.428879in}{2.288189in}}%
\pgfpathlineto{\pgfqpoint{4.441959in}{2.285715in}}%
\pgfpathlineto{\pgfqpoint{4.455047in}{2.283267in}}%
\pgfpathlineto{\pgfqpoint{4.468141in}{2.280845in}}%
\pgfpathlineto{\pgfqpoint{4.481241in}{2.278450in}}%
\pgfpathlineto{\pgfqpoint{4.473880in}{2.270939in}}%
\pgfpathlineto{\pgfqpoint{4.466513in}{2.263393in}}%
\pgfpathlineto{\pgfqpoint{4.459141in}{2.255813in}}%
\pgfpathlineto{\pgfqpoint{4.451763in}{2.248199in}}%
\pgfpathlineto{\pgfqpoint{4.438650in}{2.250622in}}%
\pgfpathlineto{\pgfqpoint{4.425545in}{2.253071in}}%
\pgfpathlineto{\pgfqpoint{4.412446in}{2.255546in}}%
\pgfpathlineto{\pgfqpoint{4.399354in}{2.258047in}}%
\pgfpathlineto{\pgfqpoint{4.406743in}{2.265629in}}%
\pgfpathlineto{\pgfqpoint{4.414127in}{2.273181in}}%
\pgfpathlineto{\pgfqpoint{4.421506in}{2.280700in}}%
\pgfpathlineto{\pgfqpoint{4.428879in}{2.288189in}}%
\pgfpathclose%
\pgfusepath{fill}%
\end{pgfscope}%
\begin{pgfscope}%
\pgfpathrectangle{\pgfqpoint{1.254980in}{0.150000in}}{\pgfqpoint{5.490039in}{5.490039in}}%
\pgfusepath{clip}%
\pgfsetbuttcap%
\pgfsetroundjoin%
\definecolor{currentfill}{rgb}{0.267004,0.004874,0.329415}%
\pgfsetfillcolor{currentfill}%
\pgfsetfillopacity{0.700000}%
\pgfsetlinewidth{0.000000pt}%
\definecolor{currentstroke}{rgb}{0.000000,0.000000,0.000000}%
\pgfsetstrokecolor{currentstroke}%
\pgfsetdash{}{0pt}%
\pgfpathmoveto{\pgfqpoint{3.647303in}{2.225188in}}%
\pgfpathlineto{\pgfqpoint{3.660206in}{2.221013in}}%
\pgfpathlineto{\pgfqpoint{3.673113in}{2.216870in}}%
\pgfpathlineto{\pgfqpoint{3.686026in}{2.212757in}}%
\pgfpathlineto{\pgfqpoint{3.698945in}{2.208674in}}%
\pgfpathlineto{\pgfqpoint{3.691293in}{2.201778in}}%
\pgfpathlineto{\pgfqpoint{3.683636in}{2.194921in}}%
\pgfpathlineto{\pgfqpoint{3.675972in}{2.188104in}}%
\pgfpathlineto{\pgfqpoint{3.668303in}{2.181332in}}%
\pgfpathlineto{\pgfqpoint{3.655371in}{2.185530in}}%
\pgfpathlineto{\pgfqpoint{3.642444in}{2.189760in}}%
\pgfpathlineto{\pgfqpoint{3.629522in}{2.194020in}}%
\pgfpathlineto{\pgfqpoint{3.616606in}{2.198312in}}%
\pgfpathlineto{\pgfqpoint{3.624290in}{2.204963in}}%
\pgfpathlineto{\pgfqpoint{3.631967in}{2.211661in}}%
\pgfpathlineto{\pgfqpoint{3.639638in}{2.218404in}}%
\pgfpathlineto{\pgfqpoint{3.647303in}{2.225188in}}%
\pgfpathclose%
\pgfusepath{fill}%
\end{pgfscope}%
\begin{pgfscope}%
\pgfpathrectangle{\pgfqpoint{1.254980in}{0.150000in}}{\pgfqpoint{5.490039in}{5.490039in}}%
\pgfusepath{clip}%
\pgfsetbuttcap%
\pgfsetroundjoin%
\definecolor{currentfill}{rgb}{0.268510,0.009605,0.335427}%
\pgfsetfillcolor{currentfill}%
\pgfsetfillopacity{0.700000}%
\pgfsetlinewidth{0.000000pt}%
\definecolor{currentstroke}{rgb}{0.000000,0.000000,0.000000}%
\pgfsetstrokecolor{currentstroke}%
\pgfsetdash{}{0pt}%
\pgfpathmoveto{\pgfqpoint{3.997125in}{2.236705in}}%
\pgfpathlineto{\pgfqpoint{4.010101in}{2.233396in}}%
\pgfpathlineto{\pgfqpoint{4.023083in}{2.230115in}}%
\pgfpathlineto{\pgfqpoint{4.036071in}{2.226862in}}%
\pgfpathlineto{\pgfqpoint{4.049065in}{2.223636in}}%
\pgfpathlineto{\pgfqpoint{4.041546in}{2.216083in}}%
\pgfpathlineto{\pgfqpoint{4.034021in}{2.208528in}}%
\pgfpathlineto{\pgfqpoint{4.026491in}{2.200972in}}%
\pgfpathlineto{\pgfqpoint{4.018955in}{2.193417in}}%
\pgfpathlineto{\pgfqpoint{4.005949in}{2.196720in}}%
\pgfpathlineto{\pgfqpoint{3.992949in}{2.200052in}}%
\pgfpathlineto{\pgfqpoint{3.979955in}{2.203412in}}%
\pgfpathlineto{\pgfqpoint{3.966967in}{2.206799in}}%
\pgfpathlineto{\pgfqpoint{3.974515in}{2.214271in}}%
\pgfpathlineto{\pgfqpoint{3.982057in}{2.221747in}}%
\pgfpathlineto{\pgfqpoint{3.989594in}{2.229226in}}%
\pgfpathlineto{\pgfqpoint{3.997125in}{2.236705in}}%
\pgfpathclose%
\pgfusepath{fill}%
\end{pgfscope}%
\begin{pgfscope}%
\pgfpathrectangle{\pgfqpoint{1.254980in}{0.150000in}}{\pgfqpoint{5.490039in}{5.490039in}}%
\pgfusepath{clip}%
\pgfsetbuttcap%
\pgfsetroundjoin%
\definecolor{currentfill}{rgb}{0.278791,0.062145,0.386592}%
\pgfsetfillcolor{currentfill}%
\pgfsetfillopacity{0.700000}%
\pgfsetlinewidth{0.000000pt}%
\definecolor{currentstroke}{rgb}{0.000000,0.000000,0.000000}%
\pgfsetstrokecolor{currentstroke}%
\pgfsetdash{}{0pt}%
\pgfpathmoveto{\pgfqpoint{4.644788in}{2.319179in}}%
\pgfpathlineto{\pgfqpoint{4.657926in}{2.317029in}}%
\pgfpathlineto{\pgfqpoint{4.671072in}{2.314904in}}%
\pgfpathlineto{\pgfqpoint{4.684224in}{2.312805in}}%
\pgfpathlineto{\pgfqpoint{4.697383in}{2.310731in}}%
\pgfpathlineto{\pgfqpoint{4.690104in}{2.303503in}}%
\pgfpathlineto{\pgfqpoint{4.682818in}{2.296233in}}%
\pgfpathlineto{\pgfqpoint{4.675527in}{2.288920in}}%
\pgfpathlineto{\pgfqpoint{4.668229in}{2.281563in}}%
\pgfpathlineto{\pgfqpoint{4.655058in}{2.283639in}}%
\pgfpathlineto{\pgfqpoint{4.641894in}{2.285740in}}%
\pgfpathlineto{\pgfqpoint{4.628736in}{2.287866in}}%
\pgfpathlineto{\pgfqpoint{4.615586in}{2.290017in}}%
\pgfpathlineto{\pgfqpoint{4.622895in}{2.297368in}}%
\pgfpathlineto{\pgfqpoint{4.630198in}{2.304678in}}%
\pgfpathlineto{\pgfqpoint{4.637496in}{2.311948in}}%
\pgfpathlineto{\pgfqpoint{4.644788in}{2.319179in}}%
\pgfpathclose%
\pgfusepath{fill}%
\end{pgfscope}%
\begin{pgfscope}%
\pgfpathrectangle{\pgfqpoint{1.254980in}{0.150000in}}{\pgfqpoint{5.490039in}{5.490039in}}%
\pgfusepath{clip}%
\pgfsetbuttcap%
\pgfsetroundjoin%
\definecolor{currentfill}{rgb}{0.276022,0.044167,0.370164}%
\pgfsetfillcolor{currentfill}%
\pgfsetfillopacity{0.700000}%
\pgfsetlinewidth{0.000000pt}%
\definecolor{currentstroke}{rgb}{0.000000,0.000000,0.000000}%
\pgfsetstrokecolor{currentstroke}%
\pgfsetdash{}{0pt}%
\pgfpathmoveto{\pgfqpoint{3.111546in}{2.291860in}}%
\pgfpathlineto{\pgfqpoint{3.124365in}{2.286036in}}%
\pgfpathlineto{\pgfqpoint{3.137189in}{2.280251in}}%
\pgfpathlineto{\pgfqpoint{3.150016in}{2.274503in}}%
\pgfpathlineto{\pgfqpoint{3.162848in}{2.268794in}}%
\pgfpathlineto{\pgfqpoint{3.154954in}{2.264185in}}%
\pgfpathlineto{\pgfqpoint{3.147051in}{2.259693in}}%
\pgfpathlineto{\pgfqpoint{3.139139in}{2.255323in}}%
\pgfpathlineto{\pgfqpoint{3.131218in}{2.251078in}}%
\pgfpathlineto{\pgfqpoint{3.118367in}{2.256955in}}%
\pgfpathlineto{\pgfqpoint{3.105521in}{2.262871in}}%
\pgfpathlineto{\pgfqpoint{3.092678in}{2.268824in}}%
\pgfpathlineto{\pgfqpoint{3.079839in}{2.274817in}}%
\pgfpathlineto{\pgfqpoint{3.087780in}{2.278889in}}%
\pgfpathlineto{\pgfqpoint{3.095711in}{2.283090in}}%
\pgfpathlineto{\pgfqpoint{3.103633in}{2.287415in}}%
\pgfpathlineto{\pgfqpoint{3.111546in}{2.291860in}}%
\pgfpathclose%
\pgfusepath{fill}%
\end{pgfscope}%
\begin{pgfscope}%
\pgfpathrectangle{\pgfqpoint{1.254980in}{0.150000in}}{\pgfqpoint{5.490039in}{5.490039in}}%
\pgfusepath{clip}%
\pgfsetbuttcap%
\pgfsetroundjoin%
\definecolor{currentfill}{rgb}{0.281446,0.084320,0.407414}%
\pgfsetfillcolor{currentfill}%
\pgfsetfillopacity{0.700000}%
\pgfsetlinewidth{0.000000pt}%
\definecolor{currentstroke}{rgb}{0.000000,0.000000,0.000000}%
\pgfsetstrokecolor{currentstroke}%
\pgfsetdash{}{0pt}%
\pgfpathmoveto{\pgfqpoint{4.860739in}{2.351128in}}%
\pgfpathlineto{\pgfqpoint{4.873937in}{2.349243in}}%
\pgfpathlineto{\pgfqpoint{4.887142in}{2.347381in}}%
\pgfpathlineto{\pgfqpoint{4.900355in}{2.345545in}}%
\pgfpathlineto{\pgfqpoint{4.913575in}{2.343733in}}%
\pgfpathlineto{\pgfqpoint{4.906381in}{2.336904in}}%
\pgfpathlineto{\pgfqpoint{4.899181in}{2.330030in}}%
\pgfpathlineto{\pgfqpoint{4.891974in}{2.323111in}}%
\pgfpathlineto{\pgfqpoint{4.884762in}{2.316144in}}%
\pgfpathlineto{\pgfqpoint{4.871529in}{2.317931in}}%
\pgfpathlineto{\pgfqpoint{4.858304in}{2.319744in}}%
\pgfpathlineto{\pgfqpoint{4.845085in}{2.321581in}}%
\pgfpathlineto{\pgfqpoint{4.831874in}{2.323442in}}%
\pgfpathlineto{\pgfqpoint{4.839099in}{2.330429in}}%
\pgfpathlineto{\pgfqpoint{4.846319in}{2.337371in}}%
\pgfpathlineto{\pgfqpoint{4.853532in}{2.344271in}}%
\pgfpathlineto{\pgfqpoint{4.860739in}{2.351128in}}%
\pgfpathclose%
\pgfusepath{fill}%
\end{pgfscope}%
\begin{pgfscope}%
\pgfpathrectangle{\pgfqpoint{1.254980in}{0.150000in}}{\pgfqpoint{5.490039in}{5.490039in}}%
\pgfusepath{clip}%
\pgfsetbuttcap%
\pgfsetroundjoin%
\definecolor{currentfill}{rgb}{0.281412,0.155834,0.469201}%
\pgfsetfillcolor{currentfill}%
\pgfsetfillopacity{0.700000}%
\pgfsetlinewidth{0.000000pt}%
\definecolor{currentstroke}{rgb}{0.000000,0.000000,0.000000}%
\pgfsetstrokecolor{currentstroke}%
\pgfsetdash{}{0pt}%
\pgfpathmoveto{\pgfqpoint{5.940189in}{2.488374in}}%
\pgfpathlineto{\pgfqpoint{5.953681in}{2.486952in}}%
\pgfpathlineto{\pgfqpoint{5.967182in}{2.485553in}}%
\pgfpathlineto{\pgfqpoint{5.980690in}{2.484176in}}%
\pgfpathlineto{\pgfqpoint{5.994207in}{2.482823in}}%
\pgfpathlineto{\pgfqpoint{5.987500in}{2.478132in}}%
\pgfpathlineto{\pgfqpoint{5.980788in}{2.473471in}}%
\pgfpathlineto{\pgfqpoint{5.974072in}{2.468834in}}%
\pgfpathlineto{\pgfqpoint{5.967351in}{2.464218in}}%
\pgfpathlineto{\pgfqpoint{5.953813in}{2.465417in}}%
\pgfpathlineto{\pgfqpoint{5.940283in}{2.466640in}}%
\pgfpathlineto{\pgfqpoint{5.926761in}{2.467885in}}%
\pgfpathlineto{\pgfqpoint{5.913248in}{2.469153in}}%
\pgfpathlineto{\pgfqpoint{5.919990in}{2.473919in}}%
\pgfpathlineto{\pgfqpoint{5.926727in}{2.478708in}}%
\pgfpathlineto{\pgfqpoint{5.933460in}{2.483525in}}%
\pgfpathlineto{\pgfqpoint{5.940189in}{2.488374in}}%
\pgfpathclose%
\pgfusepath{fill}%
\end{pgfscope}%
\begin{pgfscope}%
\pgfpathrectangle{\pgfqpoint{1.254980in}{0.150000in}}{\pgfqpoint{5.490039in}{5.490039in}}%
\pgfusepath{clip}%
\pgfsetbuttcap%
\pgfsetroundjoin%
\definecolor{currentfill}{rgb}{0.267004,0.004874,0.329415}%
\pgfsetfillcolor{currentfill}%
\pgfsetfillopacity{0.700000}%
\pgfsetlinewidth{0.000000pt}%
\definecolor{currentstroke}{rgb}{0.000000,0.000000,0.000000}%
\pgfsetstrokecolor{currentstroke}%
\pgfsetdash{}{0pt}%
\pgfpathmoveto{\pgfqpoint{3.781163in}{2.220985in}}%
\pgfpathlineto{\pgfqpoint{3.794095in}{2.217158in}}%
\pgfpathlineto{\pgfqpoint{3.807033in}{2.213360in}}%
\pgfpathlineto{\pgfqpoint{3.819976in}{2.209592in}}%
\pgfpathlineto{\pgfqpoint{3.832925in}{2.205854in}}%
\pgfpathlineto{\pgfqpoint{3.825324in}{2.198630in}}%
\pgfpathlineto{\pgfqpoint{3.817717in}{2.191429in}}%
\pgfpathlineto{\pgfqpoint{3.810104in}{2.184252in}}%
\pgfpathlineto{\pgfqpoint{3.802486in}{2.177103in}}%
\pgfpathlineto{\pgfqpoint{3.789524in}{2.180945in}}%
\pgfpathlineto{\pgfqpoint{3.776567in}{2.184817in}}%
\pgfpathlineto{\pgfqpoint{3.763617in}{2.188718in}}%
\pgfpathlineto{\pgfqpoint{3.750671in}{2.192649in}}%
\pgfpathlineto{\pgfqpoint{3.758303in}{2.199690in}}%
\pgfpathlineto{\pgfqpoint{3.765929in}{2.206761in}}%
\pgfpathlineto{\pgfqpoint{3.773549in}{2.213860in}}%
\pgfpathlineto{\pgfqpoint{3.781163in}{2.220985in}}%
\pgfpathclose%
\pgfusepath{fill}%
\end{pgfscope}%
\begin{pgfscope}%
\pgfpathrectangle{\pgfqpoint{1.254980in}{0.150000in}}{\pgfqpoint{5.490039in}{5.490039in}}%
\pgfusepath{clip}%
\pgfsetbuttcap%
\pgfsetroundjoin%
\definecolor{currentfill}{rgb}{0.282656,0.100196,0.422160}%
\pgfsetfillcolor{currentfill}%
\pgfsetfillopacity{0.700000}%
\pgfsetlinewidth{0.000000pt}%
\definecolor{currentstroke}{rgb}{0.000000,0.000000,0.000000}%
\pgfsetstrokecolor{currentstroke}%
\pgfsetdash{}{0pt}%
\pgfpathmoveto{\pgfqpoint{5.076715in}{2.382606in}}%
\pgfpathlineto{\pgfqpoint{5.089974in}{2.380926in}}%
\pgfpathlineto{\pgfqpoint{5.103240in}{2.379270in}}%
\pgfpathlineto{\pgfqpoint{5.116514in}{2.377638in}}%
\pgfpathlineto{\pgfqpoint{5.129795in}{2.376030in}}%
\pgfpathlineto{\pgfqpoint{5.122692in}{2.369670in}}%
\pgfpathlineto{\pgfqpoint{5.115582in}{2.363269in}}%
\pgfpathlineto{\pgfqpoint{5.108466in}{2.356825in}}%
\pgfpathlineto{\pgfqpoint{5.101344in}{2.350335in}}%
\pgfpathlineto{\pgfqpoint{5.088048in}{2.351893in}}%
\pgfpathlineto{\pgfqpoint{5.074760in}{2.353475in}}%
\pgfpathlineto{\pgfqpoint{5.061480in}{2.355081in}}%
\pgfpathlineto{\pgfqpoint{5.048207in}{2.356711in}}%
\pgfpathlineto{\pgfqpoint{5.055343in}{2.363246in}}%
\pgfpathlineto{\pgfqpoint{5.062473in}{2.369739in}}%
\pgfpathlineto{\pgfqpoint{5.069597in}{2.376192in}}%
\pgfpathlineto{\pgfqpoint{5.076715in}{2.382606in}}%
\pgfpathclose%
\pgfusepath{fill}%
\end{pgfscope}%
\begin{pgfscope}%
\pgfpathrectangle{\pgfqpoint{1.254980in}{0.150000in}}{\pgfqpoint{5.490039in}{5.490039in}}%
\pgfusepath{clip}%
\pgfsetbuttcap%
\pgfsetroundjoin%
\definecolor{currentfill}{rgb}{0.282910,0.105393,0.426902}%
\pgfsetfillcolor{currentfill}%
\pgfsetfillopacity{0.700000}%
\pgfsetlinewidth{0.000000pt}%
\definecolor{currentstroke}{rgb}{0.000000,0.000000,0.000000}%
\pgfsetstrokecolor{currentstroke}%
\pgfsetdash{}{0pt}%
\pgfpathmoveto{\pgfqpoint{2.791509in}{2.392042in}}%
\pgfpathlineto{\pgfqpoint{2.804303in}{2.385018in}}%
\pgfpathlineto{\pgfqpoint{2.817100in}{2.378041in}}%
\pgfpathlineto{\pgfqpoint{2.829899in}{2.371110in}}%
\pgfpathlineto{\pgfqpoint{2.842702in}{2.364225in}}%
\pgfpathlineto{\pgfqpoint{2.834624in}{2.361621in}}%
\pgfpathlineto{\pgfqpoint{2.826535in}{2.359187in}}%
\pgfpathlineto{\pgfqpoint{2.818434in}{2.356926in}}%
\pgfpathlineto{\pgfqpoint{2.810321in}{2.354844in}}%
\pgfpathlineto{\pgfqpoint{2.797495in}{2.361925in}}%
\pgfpathlineto{\pgfqpoint{2.784672in}{2.369052in}}%
\pgfpathlineto{\pgfqpoint{2.771851in}{2.376224in}}%
\pgfpathlineto{\pgfqpoint{2.759034in}{2.383443in}}%
\pgfpathlineto{\pgfqpoint{2.767171in}{2.385324in}}%
\pgfpathlineto{\pgfqpoint{2.775295in}{2.387388in}}%
\pgfpathlineto{\pgfqpoint{2.783408in}{2.389629in}}%
\pgfpathlineto{\pgfqpoint{2.791509in}{2.392042in}}%
\pgfpathclose%
\pgfusepath{fill}%
\end{pgfscope}%
\begin{pgfscope}%
\pgfpathrectangle{\pgfqpoint{1.254980in}{0.150000in}}{\pgfqpoint{5.490039in}{5.490039in}}%
\pgfusepath{clip}%
\pgfsetbuttcap%
\pgfsetroundjoin%
\definecolor{currentfill}{rgb}{0.282290,0.145912,0.461510}%
\pgfsetfillcolor{currentfill}%
\pgfsetfillopacity{0.700000}%
\pgfsetlinewidth{0.000000pt}%
\definecolor{currentstroke}{rgb}{0.000000,0.000000,0.000000}%
\pgfsetstrokecolor{currentstroke}%
\pgfsetdash{}{0pt}%
\pgfpathmoveto{\pgfqpoint{5.724456in}{2.465551in}}%
\pgfpathlineto{\pgfqpoint{5.737893in}{2.464147in}}%
\pgfpathlineto{\pgfqpoint{5.751337in}{2.462766in}}%
\pgfpathlineto{\pgfqpoint{5.764790in}{2.461409in}}%
\pgfpathlineto{\pgfqpoint{5.778251in}{2.460074in}}%
\pgfpathlineto{\pgfqpoint{5.771443in}{2.455090in}}%
\pgfpathlineto{\pgfqpoint{5.764630in}{2.450109in}}%
\pgfpathlineto{\pgfqpoint{5.757811in}{2.445127in}}%
\pgfpathlineto{\pgfqpoint{5.750987in}{2.440140in}}%
\pgfpathlineto{\pgfqpoint{5.737507in}{2.441346in}}%
\pgfpathlineto{\pgfqpoint{5.724035in}{2.442575in}}%
\pgfpathlineto{\pgfqpoint{5.710570in}{2.443828in}}%
\pgfpathlineto{\pgfqpoint{5.697114in}{2.445103in}}%
\pgfpathlineto{\pgfqpoint{5.703958in}{2.450214in}}%
\pgfpathlineto{\pgfqpoint{5.710796in}{2.455323in}}%
\pgfpathlineto{\pgfqpoint{5.717628in}{2.460433in}}%
\pgfpathlineto{\pgfqpoint{5.724456in}{2.465551in}}%
\pgfpathclose%
\pgfusepath{fill}%
\end{pgfscope}%
\begin{pgfscope}%
\pgfpathrectangle{\pgfqpoint{1.254980in}{0.150000in}}{\pgfqpoint{5.490039in}{5.490039in}}%
\pgfusepath{clip}%
\pgfsetbuttcap%
\pgfsetroundjoin%
\definecolor{currentfill}{rgb}{0.283197,0.115680,0.436115}%
\pgfsetfillcolor{currentfill}%
\pgfsetfillopacity{0.700000}%
\pgfsetlinewidth{0.000000pt}%
\definecolor{currentstroke}{rgb}{0.000000,0.000000,0.000000}%
\pgfsetstrokecolor{currentstroke}%
\pgfsetdash{}{0pt}%
\pgfpathmoveto{\pgfqpoint{5.292686in}{2.412558in}}%
\pgfpathlineto{\pgfqpoint{5.306005in}{2.411026in}}%
\pgfpathlineto{\pgfqpoint{5.319332in}{2.409518in}}%
\pgfpathlineto{\pgfqpoint{5.332667in}{2.408033in}}%
\pgfpathlineto{\pgfqpoint{5.346009in}{2.406572in}}%
\pgfpathlineto{\pgfqpoint{5.339001in}{2.400707in}}%
\pgfpathlineto{\pgfqpoint{5.331987in}{2.394810in}}%
\pgfpathlineto{\pgfqpoint{5.324966in}{2.388877in}}%
\pgfpathlineto{\pgfqpoint{5.317939in}{2.382907in}}%
\pgfpathlineto{\pgfqpoint{5.304581in}{2.384292in}}%
\pgfpathlineto{\pgfqpoint{5.291230in}{2.385700in}}%
\pgfpathlineto{\pgfqpoint{5.277888in}{2.387132in}}%
\pgfpathlineto{\pgfqpoint{5.264553in}{2.388588in}}%
\pgfpathlineto{\pgfqpoint{5.271595in}{2.394629in}}%
\pgfpathlineto{\pgfqpoint{5.278631in}{2.400637in}}%
\pgfpathlineto{\pgfqpoint{5.285662in}{2.406612in}}%
\pgfpathlineto{\pgfqpoint{5.292686in}{2.412558in}}%
\pgfpathclose%
\pgfusepath{fill}%
\end{pgfscope}%
\begin{pgfscope}%
\pgfpathrectangle{\pgfqpoint{1.254980in}{0.150000in}}{\pgfqpoint{5.490039in}{5.490039in}}%
\pgfusepath{clip}%
\pgfsetbuttcap%
\pgfsetroundjoin%
\definecolor{currentfill}{rgb}{0.282884,0.135920,0.453427}%
\pgfsetfillcolor{currentfill}%
\pgfsetfillopacity{0.700000}%
\pgfsetlinewidth{0.000000pt}%
\definecolor{currentstroke}{rgb}{0.000000,0.000000,0.000000}%
\pgfsetstrokecolor{currentstroke}%
\pgfsetdash{}{0pt}%
\pgfpathmoveto{\pgfqpoint{5.508612in}{2.440306in}}%
\pgfpathlineto{\pgfqpoint{5.521991in}{2.438866in}}%
\pgfpathlineto{\pgfqpoint{5.535378in}{2.437449in}}%
\pgfpathlineto{\pgfqpoint{5.548772in}{2.436055in}}%
\pgfpathlineto{\pgfqpoint{5.562175in}{2.434685in}}%
\pgfpathlineto{\pgfqpoint{5.555266in}{2.429294in}}%
\pgfpathlineto{\pgfqpoint{5.548351in}{2.423886in}}%
\pgfpathlineto{\pgfqpoint{5.541430in}{2.418456in}}%
\pgfpathlineto{\pgfqpoint{5.534502in}{2.413002in}}%
\pgfpathlineto{\pgfqpoint{5.521082in}{2.414270in}}%
\pgfpathlineto{\pgfqpoint{5.507670in}{2.415561in}}%
\pgfpathlineto{\pgfqpoint{5.494266in}{2.416875in}}%
\pgfpathlineto{\pgfqpoint{5.480870in}{2.418213in}}%
\pgfpathlineto{\pgfqpoint{5.487814in}{2.423765in}}%
\pgfpathlineto{\pgfqpoint{5.494753in}{2.429295in}}%
\pgfpathlineto{\pgfqpoint{5.501685in}{2.434808in}}%
\pgfpathlineto{\pgfqpoint{5.508612in}{2.440306in}}%
\pgfpathclose%
\pgfusepath{fill}%
\end{pgfscope}%
\begin{pgfscope}%
\pgfpathrectangle{\pgfqpoint{1.254980in}{0.150000in}}{\pgfqpoint{5.490039in}{5.490039in}}%
\pgfusepath{clip}%
\pgfsetbuttcap%
\pgfsetroundjoin%
\definecolor{currentfill}{rgb}{0.279566,0.067836,0.391917}%
\pgfsetfillcolor{currentfill}%
\pgfsetfillopacity{0.700000}%
\pgfsetlinewidth{0.000000pt}%
\definecolor{currentstroke}{rgb}{0.000000,0.000000,0.000000}%
\pgfsetstrokecolor{currentstroke}%
\pgfsetdash{}{0pt}%
\pgfpathmoveto{\pgfqpoint{2.977260in}{2.324188in}}%
\pgfpathlineto{\pgfqpoint{2.990070in}{2.317874in}}%
\pgfpathlineto{\pgfqpoint{3.002884in}{2.311602in}}%
\pgfpathlineto{\pgfqpoint{3.015700in}{2.305371in}}%
\pgfpathlineto{\pgfqpoint{3.028521in}{2.299180in}}%
\pgfpathlineto{\pgfqpoint{3.020551in}{2.295417in}}%
\pgfpathlineto{\pgfqpoint{3.012570in}{2.291795in}}%
\pgfpathlineto{\pgfqpoint{3.004580in}{2.288318in}}%
\pgfpathlineto{\pgfqpoint{2.996579in}{2.284991in}}%
\pgfpathlineto{\pgfqpoint{2.983738in}{2.291363in}}%
\pgfpathlineto{\pgfqpoint{2.970900in}{2.297776in}}%
\pgfpathlineto{\pgfqpoint{2.958065in}{2.304230in}}%
\pgfpathlineto{\pgfqpoint{2.945234in}{2.310725in}}%
\pgfpathlineto{\pgfqpoint{2.953257in}{2.313866in}}%
\pgfpathlineto{\pgfqpoint{2.961268in}{2.317160in}}%
\pgfpathlineto{\pgfqpoint{2.969269in}{2.320602in}}%
\pgfpathlineto{\pgfqpoint{2.977260in}{2.324188in}}%
\pgfpathclose%
\pgfusepath{fill}%
\end{pgfscope}%
\begin{pgfscope}%
\pgfpathrectangle{\pgfqpoint{1.254980in}{0.150000in}}{\pgfqpoint{5.490039in}{5.490039in}}%
\pgfusepath{clip}%
\pgfsetbuttcap%
\pgfsetroundjoin%
\definecolor{currentfill}{rgb}{0.273809,0.031497,0.358853}%
\pgfsetfillcolor{currentfill}%
\pgfsetfillopacity{0.700000}%
\pgfsetlinewidth{0.000000pt}%
\definecolor{currentstroke}{rgb}{0.000000,0.000000,0.000000}%
\pgfsetstrokecolor{currentstroke}%
\pgfsetdash{}{0pt}%
\pgfpathmoveto{\pgfqpoint{4.347051in}{2.268313in}}%
\pgfpathlineto{\pgfqpoint{4.360117in}{2.265707in}}%
\pgfpathlineto{\pgfqpoint{4.373189in}{2.263128in}}%
\pgfpathlineto{\pgfqpoint{4.386268in}{2.260574in}}%
\pgfpathlineto{\pgfqpoint{4.399354in}{2.258047in}}%
\pgfpathlineto{\pgfqpoint{4.391958in}{2.250434in}}%
\pgfpathlineto{\pgfqpoint{4.384558in}{2.242791in}}%
\pgfpathlineto{\pgfqpoint{4.377152in}{2.235118in}}%
\pgfpathlineto{\pgfqpoint{4.369740in}{2.227416in}}%
\pgfpathlineto{\pgfqpoint{4.356643in}{2.229984in}}%
\pgfpathlineto{\pgfqpoint{4.343552in}{2.232577in}}%
\pgfpathlineto{\pgfqpoint{4.330468in}{2.235197in}}%
\pgfpathlineto{\pgfqpoint{4.317391in}{2.237843in}}%
\pgfpathlineto{\pgfqpoint{4.324814in}{2.245500in}}%
\pgfpathlineto{\pgfqpoint{4.332232in}{2.253131in}}%
\pgfpathlineto{\pgfqpoint{4.339644in}{2.260735in}}%
\pgfpathlineto{\pgfqpoint{4.347051in}{2.268313in}}%
\pgfpathclose%
\pgfusepath{fill}%
\end{pgfscope}%
\begin{pgfscope}%
\pgfpathrectangle{\pgfqpoint{1.254980in}{0.150000in}}{\pgfqpoint{5.490039in}{5.490039in}}%
\pgfusepath{clip}%
\pgfsetbuttcap%
\pgfsetroundjoin%
\definecolor{currentfill}{rgb}{0.269944,0.014625,0.341379}%
\pgfsetfillcolor{currentfill}%
\pgfsetfillopacity{0.700000}%
\pgfsetlinewidth{0.000000pt}%
\definecolor{currentstroke}{rgb}{0.000000,0.000000,0.000000}%
\pgfsetstrokecolor{currentstroke}%
\pgfsetdash{}{0pt}%
\pgfpathmoveto{\pgfqpoint{4.131076in}{2.241439in}}%
\pgfpathlineto{\pgfqpoint{4.144088in}{2.238418in}}%
\pgfpathlineto{\pgfqpoint{4.157107in}{2.235424in}}%
\pgfpathlineto{\pgfqpoint{4.170132in}{2.232458in}}%
\pgfpathlineto{\pgfqpoint{4.183164in}{2.229518in}}%
\pgfpathlineto{\pgfqpoint{4.175690in}{2.221866in}}%
\pgfpathlineto{\pgfqpoint{4.168211in}{2.214200in}}%
\pgfpathlineto{\pgfqpoint{4.160726in}{2.206520in}}%
\pgfpathlineto{\pgfqpoint{4.153236in}{2.198830in}}%
\pgfpathlineto{\pgfqpoint{4.140193in}{2.201835in}}%
\pgfpathlineto{\pgfqpoint{4.127156in}{2.204867in}}%
\pgfpathlineto{\pgfqpoint{4.114126in}{2.207927in}}%
\pgfpathlineto{\pgfqpoint{4.101101in}{2.211013in}}%
\pgfpathlineto{\pgfqpoint{4.108603in}{2.218634in}}%
\pgfpathlineto{\pgfqpoint{4.116099in}{2.226245in}}%
\pgfpathlineto{\pgfqpoint{4.123590in}{2.233848in}}%
\pgfpathlineto{\pgfqpoint{4.131076in}{2.241439in}}%
\pgfpathclose%
\pgfusepath{fill}%
\end{pgfscope}%
\begin{pgfscope}%
\pgfpathrectangle{\pgfqpoint{1.254980in}{0.150000in}}{\pgfqpoint{5.490039in}{5.490039in}}%
\pgfusepath{clip}%
\pgfsetbuttcap%
\pgfsetroundjoin%
\definecolor{currentfill}{rgb}{0.277941,0.056324,0.381191}%
\pgfsetfillcolor{currentfill}%
\pgfsetfillopacity{0.700000}%
\pgfsetlinewidth{0.000000pt}%
\definecolor{currentstroke}{rgb}{0.000000,0.000000,0.000000}%
\pgfsetstrokecolor{currentstroke}%
\pgfsetdash{}{0pt}%
\pgfpathmoveto{\pgfqpoint{4.563053in}{2.298877in}}%
\pgfpathlineto{\pgfqpoint{4.576176in}{2.296624in}}%
\pgfpathlineto{\pgfqpoint{4.589306in}{2.294396in}}%
\pgfpathlineto{\pgfqpoint{4.602442in}{2.292194in}}%
\pgfpathlineto{\pgfqpoint{4.615586in}{2.290017in}}%
\pgfpathlineto{\pgfqpoint{4.608271in}{2.282626in}}%
\pgfpathlineto{\pgfqpoint{4.600950in}{2.275194in}}%
\pgfpathlineto{\pgfqpoint{4.593623in}{2.267721in}}%
\pgfpathlineto{\pgfqpoint{4.586291in}{2.260207in}}%
\pgfpathlineto{\pgfqpoint{4.573136in}{2.262398in}}%
\pgfpathlineto{\pgfqpoint{4.559988in}{2.264614in}}%
\pgfpathlineto{\pgfqpoint{4.546846in}{2.266856in}}%
\pgfpathlineto{\pgfqpoint{4.533711in}{2.269124in}}%
\pgfpathlineto{\pgfqpoint{4.541055in}{2.276619in}}%
\pgfpathlineto{\pgfqpoint{4.548394in}{2.284076in}}%
\pgfpathlineto{\pgfqpoint{4.555726in}{2.291495in}}%
\pgfpathlineto{\pgfqpoint{4.563053in}{2.298877in}}%
\pgfpathclose%
\pgfusepath{fill}%
\end{pgfscope}%
\begin{pgfscope}%
\pgfpathrectangle{\pgfqpoint{1.254980in}{0.150000in}}{\pgfqpoint{5.490039in}{5.490039in}}%
\pgfusepath{clip}%
\pgfsetbuttcap%
\pgfsetroundjoin%
\definecolor{currentfill}{rgb}{0.280255,0.165693,0.476498}%
\pgfsetfillcolor{currentfill}%
\pgfsetfillopacity{0.700000}%
\pgfsetlinewidth{0.000000pt}%
\definecolor{currentstroke}{rgb}{0.000000,0.000000,0.000000}%
\pgfsetstrokecolor{currentstroke}%
\pgfsetdash{}{0pt}%
\pgfpathmoveto{\pgfqpoint{6.075056in}{2.496128in}}%
\pgfpathlineto{\pgfqpoint{6.088591in}{2.494722in}}%
\pgfpathlineto{\pgfqpoint{6.102134in}{2.493338in}}%
\pgfpathlineto{\pgfqpoint{6.115686in}{2.491978in}}%
\pgfpathlineto{\pgfqpoint{6.109033in}{2.487419in}}%
\pgfpathlineto{\pgfqpoint{6.102377in}{2.482905in}}%
\pgfpathlineto{\pgfqpoint{6.095717in}{2.478430in}}%
\pgfpathlineto{\pgfqpoint{6.089053in}{2.473989in}}%
\pgfpathlineto{\pgfqpoint{6.075478in}{2.475182in}}%
\pgfpathlineto{\pgfqpoint{6.061913in}{2.476398in}}%
\pgfpathlineto{\pgfqpoint{6.048355in}{2.477638in}}%
\pgfpathlineto{\pgfqpoint{6.055036in}{2.482201in}}%
\pgfpathlineto{\pgfqpoint{6.061713in}{2.486800in}}%
\pgfpathlineto{\pgfqpoint{6.068386in}{2.491441in}}%
\pgfpathlineto{\pgfqpoint{6.075056in}{2.496128in}}%
\pgfpathclose%
\pgfusepath{fill}%
\end{pgfscope}%
\begin{pgfscope}%
\pgfpathrectangle{\pgfqpoint{1.254980in}{0.150000in}}{\pgfqpoint{5.490039in}{5.490039in}}%
\pgfusepath{clip}%
\pgfsetbuttcap%
\pgfsetroundjoin%
\definecolor{currentfill}{rgb}{0.267004,0.004874,0.329415}%
\pgfsetfillcolor{currentfill}%
\pgfsetfillopacity{0.700000}%
\pgfsetlinewidth{0.000000pt}%
\definecolor{currentstroke}{rgb}{0.000000,0.000000,0.000000}%
\pgfsetstrokecolor{currentstroke}%
\pgfsetdash{}{0pt}%
\pgfpathmoveto{\pgfqpoint{3.915073in}{2.220635in}}%
\pgfpathlineto{\pgfqpoint{3.928038in}{2.217133in}}%
\pgfpathlineto{\pgfqpoint{3.941008in}{2.213660in}}%
\pgfpathlineto{\pgfqpoint{3.953985in}{2.210216in}}%
\pgfpathlineto{\pgfqpoint{3.966967in}{2.206799in}}%
\pgfpathlineto{\pgfqpoint{3.959414in}{2.199334in}}%
\pgfpathlineto{\pgfqpoint{3.951855in}{2.191876in}}%
\pgfpathlineto{\pgfqpoint{3.944290in}{2.184428in}}%
\pgfpathlineto{\pgfqpoint{3.936720in}{2.176992in}}%
\pgfpathlineto{\pgfqpoint{3.923726in}{2.180499in}}%
\pgfpathlineto{\pgfqpoint{3.910737in}{2.184035in}}%
\pgfpathlineto{\pgfqpoint{3.897754in}{2.187599in}}%
\pgfpathlineto{\pgfqpoint{3.884777in}{2.191192in}}%
\pgfpathlineto{\pgfqpoint{3.892359in}{2.198533in}}%
\pgfpathlineto{\pgfqpoint{3.899936in}{2.205888in}}%
\pgfpathlineto{\pgfqpoint{3.907507in}{2.213256in}}%
\pgfpathlineto{\pgfqpoint{3.915073in}{2.220635in}}%
\pgfpathclose%
\pgfusepath{fill}%
\end{pgfscope}%
\begin{pgfscope}%
\pgfpathrectangle{\pgfqpoint{1.254980in}{0.150000in}}{\pgfqpoint{5.490039in}{5.490039in}}%
\pgfusepath{clip}%
\pgfsetbuttcap%
\pgfsetroundjoin%
\definecolor{currentfill}{rgb}{0.280267,0.073417,0.397163}%
\pgfsetfillcolor{currentfill}%
\pgfsetfillopacity{0.700000}%
\pgfsetlinewidth{0.000000pt}%
\definecolor{currentstroke}{rgb}{0.000000,0.000000,0.000000}%
\pgfsetstrokecolor{currentstroke}%
\pgfsetdash{}{0pt}%
\pgfpathmoveto{\pgfqpoint{4.779102in}{2.331136in}}%
\pgfpathlineto{\pgfqpoint{4.792284in}{2.329175in}}%
\pgfpathlineto{\pgfqpoint{4.805474in}{2.327239in}}%
\pgfpathlineto{\pgfqpoint{4.818670in}{2.325328in}}%
\pgfpathlineto{\pgfqpoint{4.831874in}{2.323442in}}%
\pgfpathlineto{\pgfqpoint{4.824643in}{2.316411in}}%
\pgfpathlineto{\pgfqpoint{4.817406in}{2.309333in}}%
\pgfpathlineto{\pgfqpoint{4.810163in}{2.302208in}}%
\pgfpathlineto{\pgfqpoint{4.802914in}{2.295036in}}%
\pgfpathlineto{\pgfqpoint{4.789697in}{2.296911in}}%
\pgfpathlineto{\pgfqpoint{4.776488in}{2.298811in}}%
\pgfpathlineto{\pgfqpoint{4.763286in}{2.300735in}}%
\pgfpathlineto{\pgfqpoint{4.750091in}{2.302684in}}%
\pgfpathlineto{\pgfqpoint{4.757353in}{2.309863in}}%
\pgfpathlineto{\pgfqpoint{4.764609in}{2.316997in}}%
\pgfpathlineto{\pgfqpoint{4.771858in}{2.324088in}}%
\pgfpathlineto{\pgfqpoint{4.779102in}{2.331136in}}%
\pgfpathclose%
\pgfusepath{fill}%
\end{pgfscope}%
\begin{pgfscope}%
\pgfpathrectangle{\pgfqpoint{1.254980in}{0.150000in}}{\pgfqpoint{5.490039in}{5.490039in}}%
\pgfusepath{clip}%
\pgfsetbuttcap%
\pgfsetroundjoin%
\definecolor{currentfill}{rgb}{0.268510,0.009605,0.335427}%
\pgfsetfillcolor{currentfill}%
\pgfsetfillopacity{0.700000}%
\pgfsetlinewidth{0.000000pt}%
\definecolor{currentstroke}{rgb}{0.000000,0.000000,0.000000}%
\pgfsetstrokecolor{currentstroke}%
\pgfsetdash{}{0pt}%
\pgfpathmoveto{\pgfqpoint{3.431024in}{2.227772in}}%
\pgfpathlineto{\pgfqpoint{3.443896in}{2.222965in}}%
\pgfpathlineto{\pgfqpoint{3.456772in}{2.218190in}}%
\pgfpathlineto{\pgfqpoint{3.469654in}{2.213449in}}%
\pgfpathlineto{\pgfqpoint{3.482540in}{2.208741in}}%
\pgfpathlineto{\pgfqpoint{3.474793in}{2.202644in}}%
\pgfpathlineto{\pgfqpoint{3.467039in}{2.196618in}}%
\pgfpathlineto{\pgfqpoint{3.459278in}{2.190668in}}%
\pgfpathlineto{\pgfqpoint{3.451510in}{2.184796in}}%
\pgfpathlineto{\pgfqpoint{3.438609in}{2.189646in}}%
\pgfpathlineto{\pgfqpoint{3.425712in}{2.194530in}}%
\pgfpathlineto{\pgfqpoint{3.412819in}{2.199446in}}%
\pgfpathlineto{\pgfqpoint{3.399931in}{2.204396in}}%
\pgfpathlineto{\pgfqpoint{3.407716in}{2.210121in}}%
\pgfpathlineto{\pgfqpoint{3.415492in}{2.215928in}}%
\pgfpathlineto{\pgfqpoint{3.423262in}{2.221813in}}%
\pgfpathlineto{\pgfqpoint{3.431024in}{2.227772in}}%
\pgfpathclose%
\pgfusepath{fill}%
\end{pgfscope}%
\begin{pgfscope}%
\pgfpathrectangle{\pgfqpoint{1.254980in}{0.150000in}}{\pgfqpoint{5.490039in}{5.490039in}}%
\pgfusepath{clip}%
\pgfsetbuttcap%
\pgfsetroundjoin%
\definecolor{currentfill}{rgb}{0.271305,0.019942,0.347269}%
\pgfsetfillcolor{currentfill}%
\pgfsetfillopacity{0.700000}%
\pgfsetlinewidth{0.000000pt}%
\definecolor{currentstroke}{rgb}{0.000000,0.000000,0.000000}%
\pgfsetstrokecolor{currentstroke}%
\pgfsetdash{}{0pt}%
\pgfpathmoveto{\pgfqpoint{3.296994in}{2.245224in}}%
\pgfpathlineto{\pgfqpoint{3.309846in}{2.239999in}}%
\pgfpathlineto{\pgfqpoint{3.322702in}{2.234809in}}%
\pgfpathlineto{\pgfqpoint{3.335562in}{2.229654in}}%
\pgfpathlineto{\pgfqpoint{3.348427in}{2.224534in}}%
\pgfpathlineto{\pgfqpoint{3.340619in}{2.219045in}}%
\pgfpathlineto{\pgfqpoint{3.332803in}{2.213647in}}%
\pgfpathlineto{\pgfqpoint{3.324979in}{2.208345in}}%
\pgfpathlineto{\pgfqpoint{3.317147in}{2.203143in}}%
\pgfpathlineto{\pgfqpoint{3.304265in}{2.208418in}}%
\pgfpathlineto{\pgfqpoint{3.291388in}{2.213728in}}%
\pgfpathlineto{\pgfqpoint{3.278515in}{2.219073in}}%
\pgfpathlineto{\pgfqpoint{3.265646in}{2.224453in}}%
\pgfpathlineto{\pgfqpoint{3.273495in}{2.229496in}}%
\pgfpathlineto{\pgfqpoint{3.281336in}{2.234641in}}%
\pgfpathlineto{\pgfqpoint{3.289169in}{2.239884in}}%
\pgfpathlineto{\pgfqpoint{3.296994in}{2.245224in}}%
\pgfpathclose%
\pgfusepath{fill}%
\end{pgfscope}%
\begin{pgfscope}%
\pgfpathrectangle{\pgfqpoint{1.254980in}{0.150000in}}{\pgfqpoint{5.490039in}{5.490039in}}%
\pgfusepath{clip}%
\pgfsetbuttcap%
\pgfsetroundjoin%
\definecolor{currentfill}{rgb}{0.267004,0.004874,0.329415}%
\pgfsetfillcolor{currentfill}%
\pgfsetfillopacity{0.700000}%
\pgfsetlinewidth{0.000000pt}%
\definecolor{currentstroke}{rgb}{0.000000,0.000000,0.000000}%
\pgfsetstrokecolor{currentstroke}%
\pgfsetdash{}{0pt}%
\pgfpathmoveto{\pgfqpoint{3.564991in}{2.215791in}}%
\pgfpathlineto{\pgfqpoint{3.577887in}{2.211374in}}%
\pgfpathlineto{\pgfqpoint{3.590788in}{2.206989in}}%
\pgfpathlineto{\pgfqpoint{3.603695in}{2.202635in}}%
\pgfpathlineto{\pgfqpoint{3.616606in}{2.198312in}}%
\pgfpathlineto{\pgfqpoint{3.608916in}{2.191711in}}%
\pgfpathlineto{\pgfqpoint{3.601219in}{2.185162in}}%
\pgfpathlineto{\pgfqpoint{3.593516in}{2.178669in}}%
\pgfpathlineto{\pgfqpoint{3.585806in}{2.172235in}}%
\pgfpathlineto{\pgfqpoint{3.572880in}{2.176687in}}%
\pgfpathlineto{\pgfqpoint{3.559959in}{2.181171in}}%
\pgfpathlineto{\pgfqpoint{3.547044in}{2.185686in}}%
\pgfpathlineto{\pgfqpoint{3.534133in}{2.190233in}}%
\pgfpathlineto{\pgfqpoint{3.541858in}{2.196532in}}%
\pgfpathlineto{\pgfqpoint{3.549576in}{2.202894in}}%
\pgfpathlineto{\pgfqpoint{3.557287in}{2.209315in}}%
\pgfpathlineto{\pgfqpoint{3.564991in}{2.215791in}}%
\pgfpathclose%
\pgfusepath{fill}%
\end{pgfscope}%
\begin{pgfscope}%
\pgfpathrectangle{\pgfqpoint{1.254980in}{0.150000in}}{\pgfqpoint{5.490039in}{5.490039in}}%
\pgfusepath{clip}%
\pgfsetbuttcap%
\pgfsetroundjoin%
\definecolor{currentfill}{rgb}{0.282327,0.094955,0.417331}%
\pgfsetfillcolor{currentfill}%
\pgfsetfillopacity{0.700000}%
\pgfsetlinewidth{0.000000pt}%
\definecolor{currentstroke}{rgb}{0.000000,0.000000,0.000000}%
\pgfsetstrokecolor{currentstroke}%
\pgfsetdash{}{0pt}%
\pgfpathmoveto{\pgfqpoint{4.995190in}{2.363474in}}%
\pgfpathlineto{\pgfqpoint{5.008433in}{2.361747in}}%
\pgfpathlineto{\pgfqpoint{5.021683in}{2.360044in}}%
\pgfpathlineto{\pgfqpoint{5.034941in}{2.358365in}}%
\pgfpathlineto{\pgfqpoint{5.048207in}{2.356711in}}%
\pgfpathlineto{\pgfqpoint{5.041064in}{2.350132in}}%
\pgfpathlineto{\pgfqpoint{5.033916in}{2.343507in}}%
\pgfpathlineto{\pgfqpoint{5.026761in}{2.336836in}}%
\pgfpathlineto{\pgfqpoint{5.019599in}{2.330116in}}%
\pgfpathlineto{\pgfqpoint{5.006320in}{2.331733in}}%
\pgfpathlineto{\pgfqpoint{4.993049in}{2.333374in}}%
\pgfpathlineto{\pgfqpoint{4.979785in}{2.335039in}}%
\pgfpathlineto{\pgfqpoint{4.966528in}{2.336729in}}%
\pgfpathlineto{\pgfqpoint{4.973703in}{2.343482in}}%
\pgfpathlineto{\pgfqpoint{4.980871in}{2.350189in}}%
\pgfpathlineto{\pgfqpoint{4.988033in}{2.356853in}}%
\pgfpathlineto{\pgfqpoint{4.995190in}{2.363474in}}%
\pgfpathclose%
\pgfusepath{fill}%
\end{pgfscope}%
\begin{pgfscope}%
\pgfpathrectangle{\pgfqpoint{1.254980in}{0.150000in}}{\pgfqpoint{5.490039in}{5.490039in}}%
\pgfusepath{clip}%
\pgfsetbuttcap%
\pgfsetroundjoin%
\definecolor{currentfill}{rgb}{0.281412,0.155834,0.469201}%
\pgfsetfillcolor{currentfill}%
\pgfsetfillopacity{0.700000}%
\pgfsetlinewidth{0.000000pt}%
\definecolor{currentstroke}{rgb}{0.000000,0.000000,0.000000}%
\pgfsetstrokecolor{currentstroke}%
\pgfsetdash{}{0pt}%
\pgfpathmoveto{\pgfqpoint{5.859275in}{2.474456in}}%
\pgfpathlineto{\pgfqpoint{5.872756in}{2.473095in}}%
\pgfpathlineto{\pgfqpoint{5.886245in}{2.471758in}}%
\pgfpathlineto{\pgfqpoint{5.899743in}{2.470444in}}%
\pgfpathlineto{\pgfqpoint{5.913248in}{2.469153in}}%
\pgfpathlineto{\pgfqpoint{5.906501in}{2.464406in}}%
\pgfpathlineto{\pgfqpoint{5.899749in}{2.459672in}}%
\pgfpathlineto{\pgfqpoint{5.892992in}{2.454949in}}%
\pgfpathlineto{\pgfqpoint{5.886230in}{2.450230in}}%
\pgfpathlineto{\pgfqpoint{5.872704in}{2.451380in}}%
\pgfpathlineto{\pgfqpoint{5.859186in}{2.452553in}}%
\pgfpathlineto{\pgfqpoint{5.845677in}{2.453749in}}%
\pgfpathlineto{\pgfqpoint{5.832175in}{2.454968in}}%
\pgfpathlineto{\pgfqpoint{5.838958in}{2.459823in}}%
\pgfpathlineto{\pgfqpoint{5.845735in}{2.464686in}}%
\pgfpathlineto{\pgfqpoint{5.852508in}{2.469562in}}%
\pgfpathlineto{\pgfqpoint{5.859275in}{2.474456in}}%
\pgfpathclose%
\pgfusepath{fill}%
\end{pgfscope}%
\begin{pgfscope}%
\pgfpathrectangle{\pgfqpoint{1.254980in}{0.150000in}}{\pgfqpoint{5.490039in}{5.490039in}}%
\pgfusepath{clip}%
\pgfsetbuttcap%
\pgfsetroundjoin%
\definecolor{currentfill}{rgb}{0.283091,0.110553,0.431554}%
\pgfsetfillcolor{currentfill}%
\pgfsetfillopacity{0.700000}%
\pgfsetlinewidth{0.000000pt}%
\definecolor{currentstroke}{rgb}{0.000000,0.000000,0.000000}%
\pgfsetstrokecolor{currentstroke}%
\pgfsetdash{}{0pt}%
\pgfpathmoveto{\pgfqpoint{5.211289in}{2.394650in}}%
\pgfpathlineto{\pgfqpoint{5.224594in}{2.393099in}}%
\pgfpathlineto{\pgfqpoint{5.237906in}{2.391571in}}%
\pgfpathlineto{\pgfqpoint{5.251225in}{2.390068in}}%
\pgfpathlineto{\pgfqpoint{5.264553in}{2.388588in}}%
\pgfpathlineto{\pgfqpoint{5.257504in}{2.382510in}}%
\pgfpathlineto{\pgfqpoint{5.250449in}{2.376392in}}%
\pgfpathlineto{\pgfqpoint{5.243387in}{2.370233in}}%
\pgfpathlineto{\pgfqpoint{5.236319in}{2.364030in}}%
\pgfpathlineto{\pgfqpoint{5.222977in}{2.365446in}}%
\pgfpathlineto{\pgfqpoint{5.209642in}{2.366886in}}%
\pgfpathlineto{\pgfqpoint{5.196315in}{2.368350in}}%
\pgfpathlineto{\pgfqpoint{5.182996in}{2.369838in}}%
\pgfpathlineto{\pgfqpoint{5.190079in}{2.376100in}}%
\pgfpathlineto{\pgfqpoint{5.197155in}{2.382321in}}%
\pgfpathlineto{\pgfqpoint{5.204225in}{2.388503in}}%
\pgfpathlineto{\pgfqpoint{5.211289in}{2.394650in}}%
\pgfpathclose%
\pgfusepath{fill}%
\end{pgfscope}%
\begin{pgfscope}%
\pgfpathrectangle{\pgfqpoint{1.254980in}{0.150000in}}{\pgfqpoint{5.490039in}{5.490039in}}%
\pgfusepath{clip}%
\pgfsetbuttcap%
\pgfsetroundjoin%
\definecolor{currentfill}{rgb}{0.274952,0.037752,0.364543}%
\pgfsetfillcolor{currentfill}%
\pgfsetfillopacity{0.700000}%
\pgfsetlinewidth{0.000000pt}%
\definecolor{currentstroke}{rgb}{0.000000,0.000000,0.000000}%
\pgfsetstrokecolor{currentstroke}%
\pgfsetdash{}{0pt}%
\pgfpathmoveto{\pgfqpoint{3.162848in}{2.268794in}}%
\pgfpathlineto{\pgfqpoint{3.175683in}{2.263123in}}%
\pgfpathlineto{\pgfqpoint{3.188523in}{2.257489in}}%
\pgfpathlineto{\pgfqpoint{3.201366in}{2.251891in}}%
\pgfpathlineto{\pgfqpoint{3.214214in}{2.246331in}}%
\pgfpathlineto{\pgfqpoint{3.206338in}{2.241559in}}%
\pgfpathlineto{\pgfqpoint{3.198454in}{2.236901in}}%
\pgfpathlineto{\pgfqpoint{3.190561in}{2.232360in}}%
\pgfpathlineto{\pgfqpoint{3.182659in}{2.227942in}}%
\pgfpathlineto{\pgfqpoint{3.169793in}{2.233670in}}%
\pgfpathlineto{\pgfqpoint{3.156930in}{2.239435in}}%
\pgfpathlineto{\pgfqpoint{3.144072in}{2.245238in}}%
\pgfpathlineto{\pgfqpoint{3.131218in}{2.251078in}}%
\pgfpathlineto{\pgfqpoint{3.139139in}{2.255323in}}%
\pgfpathlineto{\pgfqpoint{3.147051in}{2.259693in}}%
\pgfpathlineto{\pgfqpoint{3.154954in}{2.264185in}}%
\pgfpathlineto{\pgfqpoint{3.162848in}{2.268794in}}%
\pgfpathclose%
\pgfusepath{fill}%
\end{pgfscope}%
\begin{pgfscope}%
\pgfpathrectangle{\pgfqpoint{1.254980in}{0.150000in}}{\pgfqpoint{5.490039in}{5.490039in}}%
\pgfusepath{clip}%
\pgfsetbuttcap%
\pgfsetroundjoin%
\definecolor{currentfill}{rgb}{0.282290,0.145912,0.461510}%
\pgfsetfillcolor{currentfill}%
\pgfsetfillopacity{0.700000}%
\pgfsetlinewidth{0.000000pt}%
\definecolor{currentstroke}{rgb}{0.000000,0.000000,0.000000}%
\pgfsetstrokecolor{currentstroke}%
\pgfsetdash{}{0pt}%
\pgfpathmoveto{\pgfqpoint{5.643370in}{2.450438in}}%
\pgfpathlineto{\pgfqpoint{5.656794in}{2.449069in}}%
\pgfpathlineto{\pgfqpoint{5.670226in}{2.447724in}}%
\pgfpathlineto{\pgfqpoint{5.683666in}{2.446402in}}%
\pgfpathlineto{\pgfqpoint{5.697114in}{2.445103in}}%
\pgfpathlineto{\pgfqpoint{5.690265in}{2.439987in}}%
\pgfpathlineto{\pgfqpoint{5.683410in}{2.434861in}}%
\pgfpathlineto{\pgfqpoint{5.676549in}{2.429722in}}%
\pgfpathlineto{\pgfqpoint{5.669682in}{2.424566in}}%
\pgfpathlineto{\pgfqpoint{5.656216in}{2.425750in}}%
\pgfpathlineto{\pgfqpoint{5.642757in}{2.426956in}}%
\pgfpathlineto{\pgfqpoint{5.629307in}{2.428186in}}%
\pgfpathlineto{\pgfqpoint{5.615865in}{2.429439in}}%
\pgfpathlineto{\pgfqpoint{5.622750in}{2.434706in}}%
\pgfpathlineto{\pgfqpoint{5.629629in}{2.439959in}}%
\pgfpathlineto{\pgfqpoint{5.636503in}{2.445202in}}%
\pgfpathlineto{\pgfqpoint{5.643370in}{2.450438in}}%
\pgfpathclose%
\pgfusepath{fill}%
\end{pgfscope}%
\begin{pgfscope}%
\pgfpathrectangle{\pgfqpoint{1.254980in}{0.150000in}}{\pgfqpoint{5.490039in}{5.490039in}}%
\pgfusepath{clip}%
\pgfsetbuttcap%
\pgfsetroundjoin%
\definecolor{currentfill}{rgb}{0.283072,0.130895,0.449241}%
\pgfsetfillcolor{currentfill}%
\pgfsetfillopacity{0.700000}%
\pgfsetlinewidth{0.000000pt}%
\definecolor{currentstroke}{rgb}{0.000000,0.000000,0.000000}%
\pgfsetstrokecolor{currentstroke}%
\pgfsetdash{}{0pt}%
\pgfpathmoveto{\pgfqpoint{5.427363in}{2.423800in}}%
\pgfpathlineto{\pgfqpoint{5.440728in}{2.422368in}}%
\pgfpathlineto{\pgfqpoint{5.454101in}{2.420960in}}%
\pgfpathlineto{\pgfqpoint{5.467481in}{2.419575in}}%
\pgfpathlineto{\pgfqpoint{5.480870in}{2.418213in}}%
\pgfpathlineto{\pgfqpoint{5.473919in}{2.412638in}}%
\pgfpathlineto{\pgfqpoint{5.466962in}{2.407035in}}%
\pgfpathlineto{\pgfqpoint{5.459999in}{2.401402in}}%
\pgfpathlineto{\pgfqpoint{5.453030in}{2.395736in}}%
\pgfpathlineto{\pgfqpoint{5.439625in}{2.397008in}}%
\pgfpathlineto{\pgfqpoint{5.426228in}{2.398303in}}%
\pgfpathlineto{\pgfqpoint{5.412838in}{2.399622in}}%
\pgfpathlineto{\pgfqpoint{5.399457in}{2.400965in}}%
\pgfpathlineto{\pgfqpoint{5.406443in}{2.406716in}}%
\pgfpathlineto{\pgfqpoint{5.413422in}{2.412437in}}%
\pgfpathlineto{\pgfqpoint{5.420396in}{2.418131in}}%
\pgfpathlineto{\pgfqpoint{5.427363in}{2.423800in}}%
\pgfpathclose%
\pgfusepath{fill}%
\end{pgfscope}%
\begin{pgfscope}%
\pgfpathrectangle{\pgfqpoint{1.254980in}{0.150000in}}{\pgfqpoint{5.490039in}{5.490039in}}%
\pgfusepath{clip}%
\pgfsetbuttcap%
\pgfsetroundjoin%
\definecolor{currentfill}{rgb}{0.267004,0.004874,0.329415}%
\pgfsetfillcolor{currentfill}%
\pgfsetfillopacity{0.700000}%
\pgfsetlinewidth{0.000000pt}%
\definecolor{currentstroke}{rgb}{0.000000,0.000000,0.000000}%
\pgfsetstrokecolor{currentstroke}%
\pgfsetdash{}{0pt}%
\pgfpathmoveto{\pgfqpoint{3.698945in}{2.208674in}}%
\pgfpathlineto{\pgfqpoint{3.711868in}{2.204623in}}%
\pgfpathlineto{\pgfqpoint{3.724797in}{2.200601in}}%
\pgfpathlineto{\pgfqpoint{3.737732in}{2.196610in}}%
\pgfpathlineto{\pgfqpoint{3.750671in}{2.192649in}}%
\pgfpathlineto{\pgfqpoint{3.743034in}{2.185641in}}%
\pgfpathlineto{\pgfqpoint{3.735390in}{2.178668in}}%
\pgfpathlineto{\pgfqpoint{3.727740in}{2.171734in}}%
\pgfpathlineto{\pgfqpoint{3.720084in}{2.164840in}}%
\pgfpathlineto{\pgfqpoint{3.707131in}{2.168917in}}%
\pgfpathlineto{\pgfqpoint{3.694183in}{2.173025in}}%
\pgfpathlineto{\pgfqpoint{3.681240in}{2.177163in}}%
\pgfpathlineto{\pgfqpoint{3.668303in}{2.181332in}}%
\pgfpathlineto{\pgfqpoint{3.675972in}{2.188104in}}%
\pgfpathlineto{\pgfqpoint{3.683636in}{2.194921in}}%
\pgfpathlineto{\pgfqpoint{3.691293in}{2.201778in}}%
\pgfpathlineto{\pgfqpoint{3.698945in}{2.208674in}}%
\pgfpathclose%
\pgfusepath{fill}%
\end{pgfscope}%
\begin{pgfscope}%
\pgfpathrectangle{\pgfqpoint{1.254980in}{0.150000in}}{\pgfqpoint{5.490039in}{5.490039in}}%
\pgfusepath{clip}%
\pgfsetbuttcap%
\pgfsetroundjoin%
\definecolor{currentfill}{rgb}{0.272594,0.025563,0.353093}%
\pgfsetfillcolor{currentfill}%
\pgfsetfillopacity{0.700000}%
\pgfsetlinewidth{0.000000pt}%
\definecolor{currentstroke}{rgb}{0.000000,0.000000,0.000000}%
\pgfsetstrokecolor{currentstroke}%
\pgfsetdash{}{0pt}%
\pgfpathmoveto{\pgfqpoint{4.265146in}{2.248691in}}%
\pgfpathlineto{\pgfqpoint{4.278198in}{2.245939in}}%
\pgfpathlineto{\pgfqpoint{4.291256in}{2.243214in}}%
\pgfpathlineto{\pgfqpoint{4.304320in}{2.240515in}}%
\pgfpathlineto{\pgfqpoint{4.317391in}{2.237843in}}%
\pgfpathlineto{\pgfqpoint{4.309962in}{2.230160in}}%
\pgfpathlineto{\pgfqpoint{4.302528in}{2.222454in}}%
\pgfpathlineto{\pgfqpoint{4.295088in}{2.214723in}}%
\pgfpathlineto{\pgfqpoint{4.287643in}{2.206971in}}%
\pgfpathlineto{\pgfqpoint{4.274561in}{2.209696in}}%
\pgfpathlineto{\pgfqpoint{4.261485in}{2.212448in}}%
\pgfpathlineto{\pgfqpoint{4.248415in}{2.215226in}}%
\pgfpathlineto{\pgfqpoint{4.235352in}{2.218031in}}%
\pgfpathlineto{\pgfqpoint{4.242809in}{2.225726in}}%
\pgfpathlineto{\pgfqpoint{4.250260in}{2.233402in}}%
\pgfpathlineto{\pgfqpoint{4.257706in}{2.241057in}}%
\pgfpathlineto{\pgfqpoint{4.265146in}{2.248691in}}%
\pgfpathclose%
\pgfusepath{fill}%
\end{pgfscope}%
\begin{pgfscope}%
\pgfpathrectangle{\pgfqpoint{1.254980in}{0.150000in}}{\pgfqpoint{5.490039in}{5.490039in}}%
\pgfusepath{clip}%
\pgfsetbuttcap%
\pgfsetroundjoin%
\definecolor{currentfill}{rgb}{0.282327,0.094955,0.417331}%
\pgfsetfillcolor{currentfill}%
\pgfsetfillopacity{0.700000}%
\pgfsetlinewidth{0.000000pt}%
\definecolor{currentstroke}{rgb}{0.000000,0.000000,0.000000}%
\pgfsetstrokecolor{currentstroke}%
\pgfsetdash{}{0pt}%
\pgfpathmoveto{\pgfqpoint{2.842702in}{2.364225in}}%
\pgfpathlineto{\pgfqpoint{2.855508in}{2.357384in}}%
\pgfpathlineto{\pgfqpoint{2.868316in}{2.350588in}}%
\pgfpathlineto{\pgfqpoint{2.881128in}{2.343836in}}%
\pgfpathlineto{\pgfqpoint{2.893943in}{2.337128in}}%
\pgfpathlineto{\pgfqpoint{2.885888in}{2.334335in}}%
\pgfpathlineto{\pgfqpoint{2.877821in}{2.331707in}}%
\pgfpathlineto{\pgfqpoint{2.869743in}{2.329249in}}%
\pgfpathlineto{\pgfqpoint{2.861654in}{2.326967in}}%
\pgfpathlineto{\pgfqpoint{2.848816in}{2.333870in}}%
\pgfpathlineto{\pgfqpoint{2.835981in}{2.340817in}}%
\pgfpathlineto{\pgfqpoint{2.823150in}{2.347808in}}%
\pgfpathlineto{\pgfqpoint{2.810321in}{2.354844in}}%
\pgfpathlineto{\pgfqpoint{2.818434in}{2.356926in}}%
\pgfpathlineto{\pgfqpoint{2.826535in}{2.359187in}}%
\pgfpathlineto{\pgfqpoint{2.834624in}{2.361621in}}%
\pgfpathlineto{\pgfqpoint{2.842702in}{2.364225in}}%
\pgfpathclose%
\pgfusepath{fill}%
\end{pgfscope}%
\begin{pgfscope}%
\pgfpathrectangle{\pgfqpoint{1.254980in}{0.150000in}}{\pgfqpoint{5.490039in}{5.490039in}}%
\pgfusepath{clip}%
\pgfsetbuttcap%
\pgfsetroundjoin%
\definecolor{currentfill}{rgb}{0.268510,0.009605,0.335427}%
\pgfsetfillcolor{currentfill}%
\pgfsetfillopacity{0.700000}%
\pgfsetlinewidth{0.000000pt}%
\definecolor{currentstroke}{rgb}{0.000000,0.000000,0.000000}%
\pgfsetstrokecolor{currentstroke}%
\pgfsetdash{}{0pt}%
\pgfpathmoveto{\pgfqpoint{4.049065in}{2.223636in}}%
\pgfpathlineto{\pgfqpoint{4.062065in}{2.220439in}}%
\pgfpathlineto{\pgfqpoint{4.075071in}{2.217270in}}%
\pgfpathlineto{\pgfqpoint{4.088083in}{2.214128in}}%
\pgfpathlineto{\pgfqpoint{4.101101in}{2.211013in}}%
\pgfpathlineto{\pgfqpoint{4.093594in}{2.203386in}}%
\pgfpathlineto{\pgfqpoint{4.086082in}{2.195754in}}%
\pgfpathlineto{\pgfqpoint{4.078563in}{2.188118in}}%
\pgfpathlineto{\pgfqpoint{4.071040in}{2.180480in}}%
\pgfpathlineto{\pgfqpoint{4.058010in}{2.183673in}}%
\pgfpathlineto{\pgfqpoint{4.044985in}{2.186893in}}%
\pgfpathlineto{\pgfqpoint{4.031967in}{2.190141in}}%
\pgfpathlineto{\pgfqpoint{4.018955in}{2.193417in}}%
\pgfpathlineto{\pgfqpoint{4.026491in}{2.200972in}}%
\pgfpathlineto{\pgfqpoint{4.034021in}{2.208528in}}%
\pgfpathlineto{\pgfqpoint{4.041546in}{2.216083in}}%
\pgfpathlineto{\pgfqpoint{4.049065in}{2.223636in}}%
\pgfpathclose%
\pgfusepath{fill}%
\end{pgfscope}%
\begin{pgfscope}%
\pgfpathrectangle{\pgfqpoint{1.254980in}{0.150000in}}{\pgfqpoint{5.490039in}{5.490039in}}%
\pgfusepath{clip}%
\pgfsetbuttcap%
\pgfsetroundjoin%
\definecolor{currentfill}{rgb}{0.276022,0.044167,0.370164}%
\pgfsetfillcolor{currentfill}%
\pgfsetfillopacity{0.700000}%
\pgfsetlinewidth{0.000000pt}%
\definecolor{currentstroke}{rgb}{0.000000,0.000000,0.000000}%
\pgfsetstrokecolor{currentstroke}%
\pgfsetdash{}{0pt}%
\pgfpathmoveto{\pgfqpoint{4.481241in}{2.278450in}}%
\pgfpathlineto{\pgfqpoint{4.494349in}{2.276080in}}%
\pgfpathlineto{\pgfqpoint{4.507463in}{2.273735in}}%
\pgfpathlineto{\pgfqpoint{4.520584in}{2.271417in}}%
\pgfpathlineto{\pgfqpoint{4.533711in}{2.269124in}}%
\pgfpathlineto{\pgfqpoint{4.526362in}{2.261590in}}%
\pgfpathlineto{\pgfqpoint{4.519007in}{2.254019in}}%
\pgfpathlineto{\pgfqpoint{4.511646in}{2.246410in}}%
\pgfpathlineto{\pgfqpoint{4.504279in}{2.238764in}}%
\pgfpathlineto{\pgfqpoint{4.491140in}{2.241084in}}%
\pgfpathlineto{\pgfqpoint{4.478007in}{2.243430in}}%
\pgfpathlineto{\pgfqpoint{4.464882in}{2.245802in}}%
\pgfpathlineto{\pgfqpoint{4.451763in}{2.248199in}}%
\pgfpathlineto{\pgfqpoint{4.459141in}{2.255813in}}%
\pgfpathlineto{\pgfqpoint{4.466513in}{2.263393in}}%
\pgfpathlineto{\pgfqpoint{4.473880in}{2.270939in}}%
\pgfpathlineto{\pgfqpoint{4.481241in}{2.278450in}}%
\pgfpathclose%
\pgfusepath{fill}%
\end{pgfscope}%
\begin{pgfscope}%
\pgfpathrectangle{\pgfqpoint{1.254980in}{0.150000in}}{\pgfqpoint{5.490039in}{5.490039in}}%
\pgfusepath{clip}%
\pgfsetbuttcap%
\pgfsetroundjoin%
\definecolor{currentfill}{rgb}{0.282623,0.140926,0.457517}%
\pgfsetfillcolor{currentfill}%
\pgfsetfillopacity{0.700000}%
\pgfsetlinewidth{0.000000pt}%
\definecolor{currentstroke}{rgb}{0.000000,0.000000,0.000000}%
\pgfsetstrokecolor{currentstroke}%
\pgfsetdash{}{0pt}%
\pgfpathmoveto{\pgfqpoint{2.656583in}{2.442921in}}%
\pgfpathlineto{\pgfqpoint{2.669381in}{2.435314in}}%
\pgfpathlineto{\pgfqpoint{2.682181in}{2.427757in}}%
\pgfpathlineto{\pgfqpoint{2.694984in}{2.420249in}}%
\pgfpathlineto{\pgfqpoint{2.707789in}{2.412791in}}%
\pgfpathlineto{\pgfqpoint{2.699614in}{2.411301in}}%
\pgfpathlineto{\pgfqpoint{2.691426in}{2.410006in}}%
\pgfpathlineto{\pgfqpoint{2.683225in}{2.408912in}}%
\pgfpathlineto{\pgfqpoint{2.675010in}{2.408025in}}%
\pgfpathlineto{\pgfqpoint{2.662180in}{2.415692in}}%
\pgfpathlineto{\pgfqpoint{2.649351in}{2.423408in}}%
\pgfpathlineto{\pgfqpoint{2.636525in}{2.431175in}}%
\pgfpathlineto{\pgfqpoint{2.623702in}{2.438992in}}%
\pgfpathlineto{\pgfqpoint{2.631943in}{2.439665in}}%
\pgfpathlineto{\pgfqpoint{2.640170in}{2.440548in}}%
\pgfpathlineto{\pgfqpoint{2.648384in}{2.441635in}}%
\pgfpathlineto{\pgfqpoint{2.656583in}{2.442921in}}%
\pgfpathclose%
\pgfusepath{fill}%
\end{pgfscope}%
\begin{pgfscope}%
\pgfpathrectangle{\pgfqpoint{1.254980in}{0.150000in}}{\pgfqpoint{5.490039in}{5.490039in}}%
\pgfusepath{clip}%
\pgfsetbuttcap%
\pgfsetroundjoin%
\definecolor{currentfill}{rgb}{0.279566,0.067836,0.391917}%
\pgfsetfillcolor{currentfill}%
\pgfsetfillopacity{0.700000}%
\pgfsetlinewidth{0.000000pt}%
\definecolor{currentstroke}{rgb}{0.000000,0.000000,0.000000}%
\pgfsetstrokecolor{currentstroke}%
\pgfsetdash{}{0pt}%
\pgfpathmoveto{\pgfqpoint{4.697383in}{2.310731in}}%
\pgfpathlineto{\pgfqpoint{4.710550in}{2.308682in}}%
\pgfpathlineto{\pgfqpoint{4.723723in}{2.306657in}}%
\pgfpathlineto{\pgfqpoint{4.736904in}{2.304658in}}%
\pgfpathlineto{\pgfqpoint{4.750091in}{2.302684in}}%
\pgfpathlineto{\pgfqpoint{4.742824in}{2.295460in}}%
\pgfpathlineto{\pgfqpoint{4.735550in}{2.288190in}}%
\pgfpathlineto{\pgfqpoint{4.728271in}{2.280874in}}%
\pgfpathlineto{\pgfqpoint{4.720986in}{2.273511in}}%
\pgfpathlineto{\pgfqpoint{4.707786in}{2.275487in}}%
\pgfpathlineto{\pgfqpoint{4.694593in}{2.277487in}}%
\pgfpathlineto{\pgfqpoint{4.681408in}{2.279513in}}%
\pgfpathlineto{\pgfqpoint{4.668229in}{2.281563in}}%
\pgfpathlineto{\pgfqpoint{4.675527in}{2.288920in}}%
\pgfpathlineto{\pgfqpoint{4.682818in}{2.296233in}}%
\pgfpathlineto{\pgfqpoint{4.690104in}{2.303503in}}%
\pgfpathlineto{\pgfqpoint{4.697383in}{2.310731in}}%
\pgfpathclose%
\pgfusepath{fill}%
\end{pgfscope}%
\begin{pgfscope}%
\pgfpathrectangle{\pgfqpoint{1.254980in}{0.150000in}}{\pgfqpoint{5.490039in}{5.490039in}}%
\pgfusepath{clip}%
\pgfsetbuttcap%
\pgfsetroundjoin%
\definecolor{currentfill}{rgb}{0.278791,0.062145,0.386592}%
\pgfsetfillcolor{currentfill}%
\pgfsetfillopacity{0.700000}%
\pgfsetlinewidth{0.000000pt}%
\definecolor{currentstroke}{rgb}{0.000000,0.000000,0.000000}%
\pgfsetstrokecolor{currentstroke}%
\pgfsetdash{}{0pt}%
\pgfpathmoveto{\pgfqpoint{3.028521in}{2.299180in}}%
\pgfpathlineto{\pgfqpoint{3.041345in}{2.293030in}}%
\pgfpathlineto{\pgfqpoint{3.054173in}{2.286919in}}%
\pgfpathlineto{\pgfqpoint{3.067004in}{2.280848in}}%
\pgfpathlineto{\pgfqpoint{3.079839in}{2.274817in}}%
\pgfpathlineto{\pgfqpoint{3.071889in}{2.270878in}}%
\pgfpathlineto{\pgfqpoint{3.063929in}{2.267076in}}%
\pgfpathlineto{\pgfqpoint{3.055959in}{2.263416in}}%
\pgfpathlineto{\pgfqpoint{3.047980in}{2.259902in}}%
\pgfpathlineto{\pgfqpoint{3.035124in}{2.266115in}}%
\pgfpathlineto{\pgfqpoint{3.022272in}{2.272367in}}%
\pgfpathlineto{\pgfqpoint{3.009424in}{2.278659in}}%
\pgfpathlineto{\pgfqpoint{2.996579in}{2.284991in}}%
\pgfpathlineto{\pgfqpoint{3.004580in}{2.288318in}}%
\pgfpathlineto{\pgfqpoint{3.012570in}{2.291795in}}%
\pgfpathlineto{\pgfqpoint{3.020551in}{2.295417in}}%
\pgfpathlineto{\pgfqpoint{3.028521in}{2.299180in}}%
\pgfpathclose%
\pgfusepath{fill}%
\end{pgfscope}%
\begin{pgfscope}%
\pgfpathrectangle{\pgfqpoint{1.254980in}{0.150000in}}{\pgfqpoint{5.490039in}{5.490039in}}%
\pgfusepath{clip}%
\pgfsetbuttcap%
\pgfsetroundjoin%
\definecolor{currentfill}{rgb}{0.267004,0.004874,0.329415}%
\pgfsetfillcolor{currentfill}%
\pgfsetfillopacity{0.700000}%
\pgfsetlinewidth{0.000000pt}%
\definecolor{currentstroke}{rgb}{0.000000,0.000000,0.000000}%
\pgfsetstrokecolor{currentstroke}%
\pgfsetdash{}{0pt}%
\pgfpathmoveto{\pgfqpoint{3.832925in}{2.205854in}}%
\pgfpathlineto{\pgfqpoint{3.845879in}{2.202145in}}%
\pgfpathlineto{\pgfqpoint{3.858839in}{2.198465in}}%
\pgfpathlineto{\pgfqpoint{3.871805in}{2.194814in}}%
\pgfpathlineto{\pgfqpoint{3.884777in}{2.191192in}}%
\pgfpathlineto{\pgfqpoint{3.877188in}{2.183869in}}%
\pgfpathlineto{\pgfqpoint{3.869594in}{2.176566in}}%
\pgfpathlineto{\pgfqpoint{3.861995in}{2.169283in}}%
\pgfpathlineto{\pgfqpoint{3.854389in}{2.162025in}}%
\pgfpathlineto{\pgfqpoint{3.841405in}{2.165751in}}%
\pgfpathlineto{\pgfqpoint{3.828426in}{2.169506in}}%
\pgfpathlineto{\pgfqpoint{3.815453in}{2.173290in}}%
\pgfpathlineto{\pgfqpoint{3.802486in}{2.177103in}}%
\pgfpathlineto{\pgfqpoint{3.810104in}{2.184252in}}%
\pgfpathlineto{\pgfqpoint{3.817717in}{2.191429in}}%
\pgfpathlineto{\pgfqpoint{3.825324in}{2.198630in}}%
\pgfpathlineto{\pgfqpoint{3.832925in}{2.205854in}}%
\pgfpathclose%
\pgfusepath{fill}%
\end{pgfscope}%
\begin{pgfscope}%
\pgfpathrectangle{\pgfqpoint{1.254980in}{0.150000in}}{\pgfqpoint{5.490039in}{5.490039in}}%
\pgfusepath{clip}%
\pgfsetbuttcap%
\pgfsetroundjoin%
\definecolor{currentfill}{rgb}{0.281924,0.089666,0.412415}%
\pgfsetfillcolor{currentfill}%
\pgfsetfillopacity{0.700000}%
\pgfsetlinewidth{0.000000pt}%
\definecolor{currentstroke}{rgb}{0.000000,0.000000,0.000000}%
\pgfsetstrokecolor{currentstroke}%
\pgfsetdash{}{0pt}%
\pgfpathmoveto{\pgfqpoint{4.913575in}{2.343733in}}%
\pgfpathlineto{\pgfqpoint{4.926802in}{2.341945in}}%
\pgfpathlineto{\pgfqpoint{4.940037in}{2.340182in}}%
\pgfpathlineto{\pgfqpoint{4.953279in}{2.338443in}}%
\pgfpathlineto{\pgfqpoint{4.966528in}{2.336729in}}%
\pgfpathlineto{\pgfqpoint{4.959347in}{2.329930in}}%
\pgfpathlineto{\pgfqpoint{4.952160in}{2.323082in}}%
\pgfpathlineto{\pgfqpoint{4.944967in}{2.316186in}}%
\pgfpathlineto{\pgfqpoint{4.937767in}{2.309238in}}%
\pgfpathlineto{\pgfqpoint{4.924505in}{2.310928in}}%
\pgfpathlineto{\pgfqpoint{4.911250in}{2.312642in}}%
\pgfpathlineto{\pgfqpoint{4.898002in}{2.314381in}}%
\pgfpathlineto{\pgfqpoint{4.884762in}{2.316144in}}%
\pgfpathlineto{\pgfqpoint{4.891974in}{2.323111in}}%
\pgfpathlineto{\pgfqpoint{4.899181in}{2.330030in}}%
\pgfpathlineto{\pgfqpoint{4.906381in}{2.336904in}}%
\pgfpathlineto{\pgfqpoint{4.913575in}{2.343733in}}%
\pgfpathclose%
\pgfusepath{fill}%
\end{pgfscope}%
\begin{pgfscope}%
\pgfpathrectangle{\pgfqpoint{1.254980in}{0.150000in}}{\pgfqpoint{5.490039in}{5.490039in}}%
\pgfusepath{clip}%
\pgfsetbuttcap%
\pgfsetroundjoin%
\definecolor{currentfill}{rgb}{0.280255,0.165693,0.476498}%
\pgfsetfillcolor{currentfill}%
\pgfsetfillopacity{0.700000}%
\pgfsetlinewidth{0.000000pt}%
\definecolor{currentstroke}{rgb}{0.000000,0.000000,0.000000}%
\pgfsetstrokecolor{currentstroke}%
\pgfsetdash{}{0pt}%
\pgfpathmoveto{\pgfqpoint{5.994207in}{2.482823in}}%
\pgfpathlineto{\pgfqpoint{6.007732in}{2.481492in}}%
\pgfpathlineto{\pgfqpoint{6.021264in}{2.480185in}}%
\pgfpathlineto{\pgfqpoint{6.034806in}{2.478900in}}%
\pgfpathlineto{\pgfqpoint{6.048355in}{2.477638in}}%
\pgfpathlineto{\pgfqpoint{6.041670in}{2.473106in}}%
\pgfpathlineto{\pgfqpoint{6.034980in}{2.468600in}}%
\pgfpathlineto{\pgfqpoint{6.028285in}{2.464117in}}%
\pgfpathlineto{\pgfqpoint{6.021586in}{2.459650in}}%
\pgfpathlineto{\pgfqpoint{6.008015in}{2.460758in}}%
\pgfpathlineto{\pgfqpoint{5.994452in}{2.461888in}}%
\pgfpathlineto{\pgfqpoint{5.980897in}{2.463042in}}%
\pgfpathlineto{\pgfqpoint{5.967351in}{2.464218in}}%
\pgfpathlineto{\pgfqpoint{5.974072in}{2.468834in}}%
\pgfpathlineto{\pgfqpoint{5.980788in}{2.473471in}}%
\pgfpathlineto{\pgfqpoint{5.987500in}{2.478132in}}%
\pgfpathlineto{\pgfqpoint{5.994207in}{2.482823in}}%
\pgfpathclose%
\pgfusepath{fill}%
\end{pgfscope}%
\begin{pgfscope}%
\pgfpathrectangle{\pgfqpoint{1.254980in}{0.150000in}}{\pgfqpoint{5.490039in}{5.490039in}}%
\pgfusepath{clip}%
\pgfsetbuttcap%
\pgfsetroundjoin%
\definecolor{currentfill}{rgb}{0.282910,0.105393,0.426902}%
\pgfsetfillcolor{currentfill}%
\pgfsetfillopacity{0.700000}%
\pgfsetlinewidth{0.000000pt}%
\definecolor{currentstroke}{rgb}{0.000000,0.000000,0.000000}%
\pgfsetstrokecolor{currentstroke}%
\pgfsetdash{}{0pt}%
\pgfpathmoveto{\pgfqpoint{5.129795in}{2.376030in}}%
\pgfpathlineto{\pgfqpoint{5.143084in}{2.374446in}}%
\pgfpathlineto{\pgfqpoint{5.156381in}{2.372886in}}%
\pgfpathlineto{\pgfqpoint{5.169685in}{2.371350in}}%
\pgfpathlineto{\pgfqpoint{5.182996in}{2.369838in}}%
\pgfpathlineto{\pgfqpoint{5.175907in}{2.363534in}}%
\pgfpathlineto{\pgfqpoint{5.168812in}{2.357185in}}%
\pgfpathlineto{\pgfqpoint{5.161710in}{2.350790in}}%
\pgfpathlineto{\pgfqpoint{5.154602in}{2.344346in}}%
\pgfpathlineto{\pgfqpoint{5.141276in}{2.345807in}}%
\pgfpathlineto{\pgfqpoint{5.127957in}{2.347293in}}%
\pgfpathlineto{\pgfqpoint{5.114647in}{2.348802in}}%
\pgfpathlineto{\pgfqpoint{5.101344in}{2.350335in}}%
\pgfpathlineto{\pgfqpoint{5.108466in}{2.356825in}}%
\pgfpathlineto{\pgfqpoint{5.115582in}{2.363269in}}%
\pgfpathlineto{\pgfqpoint{5.122692in}{2.369670in}}%
\pgfpathlineto{\pgfqpoint{5.129795in}{2.376030in}}%
\pgfpathclose%
\pgfusepath{fill}%
\end{pgfscope}%
\begin{pgfscope}%
\pgfpathrectangle{\pgfqpoint{1.254980in}{0.150000in}}{\pgfqpoint{5.490039in}{5.490039in}}%
\pgfusepath{clip}%
\pgfsetbuttcap%
\pgfsetroundjoin%
\definecolor{currentfill}{rgb}{0.281412,0.155834,0.469201}%
\pgfsetfillcolor{currentfill}%
\pgfsetfillopacity{0.700000}%
\pgfsetlinewidth{0.000000pt}%
\definecolor{currentstroke}{rgb}{0.000000,0.000000,0.000000}%
\pgfsetstrokecolor{currentstroke}%
\pgfsetdash{}{0pt}%
\pgfpathmoveto{\pgfqpoint{5.778251in}{2.460074in}}%
\pgfpathlineto{\pgfqpoint{5.791720in}{2.458763in}}%
\pgfpathlineto{\pgfqpoint{5.805197in}{2.457475in}}%
\pgfpathlineto{\pgfqpoint{5.818682in}{2.456210in}}%
\pgfpathlineto{\pgfqpoint{5.832175in}{2.454968in}}%
\pgfpathlineto{\pgfqpoint{5.825387in}{2.450117in}}%
\pgfpathlineto{\pgfqpoint{5.818594in}{2.445266in}}%
\pgfpathlineto{\pgfqpoint{5.811795in}{2.440411in}}%
\pgfpathlineto{\pgfqpoint{5.804990in}{2.435548in}}%
\pgfpathlineto{\pgfqpoint{5.791477in}{2.436661in}}%
\pgfpathlineto{\pgfqpoint{5.777972in}{2.437797in}}%
\pgfpathlineto{\pgfqpoint{5.764475in}{2.438957in}}%
\pgfpathlineto{\pgfqpoint{5.750987in}{2.440140in}}%
\pgfpathlineto{\pgfqpoint{5.757811in}{2.445127in}}%
\pgfpathlineto{\pgfqpoint{5.764630in}{2.450109in}}%
\pgfpathlineto{\pgfqpoint{5.771443in}{2.455090in}}%
\pgfpathlineto{\pgfqpoint{5.778251in}{2.460074in}}%
\pgfpathclose%
\pgfusepath{fill}%
\end{pgfscope}%
\begin{pgfscope}%
\pgfpathrectangle{\pgfqpoint{1.254980in}{0.150000in}}{\pgfqpoint{5.490039in}{5.490039in}}%
\pgfusepath{clip}%
\pgfsetbuttcap%
\pgfsetroundjoin%
\definecolor{currentfill}{rgb}{0.271305,0.019942,0.347269}%
\pgfsetfillcolor{currentfill}%
\pgfsetfillopacity{0.700000}%
\pgfsetlinewidth{0.000000pt}%
\definecolor{currentstroke}{rgb}{0.000000,0.000000,0.000000}%
\pgfsetstrokecolor{currentstroke}%
\pgfsetdash{}{0pt}%
\pgfpathmoveto{\pgfqpoint{4.183164in}{2.229518in}}%
\pgfpathlineto{\pgfqpoint{4.196201in}{2.226606in}}%
\pgfpathlineto{\pgfqpoint{4.209245in}{2.223721in}}%
\pgfpathlineto{\pgfqpoint{4.222296in}{2.220862in}}%
\pgfpathlineto{\pgfqpoint{4.235352in}{2.218031in}}%
\pgfpathlineto{\pgfqpoint{4.227890in}{2.210317in}}%
\pgfpathlineto{\pgfqpoint{4.220423in}{2.202587in}}%
\pgfpathlineto{\pgfqpoint{4.212950in}{2.194840in}}%
\pgfpathlineto{\pgfqpoint{4.205472in}{2.187079in}}%
\pgfpathlineto{\pgfqpoint{4.192403in}{2.189976in}}%
\pgfpathlineto{\pgfqpoint{4.179341in}{2.192901in}}%
\pgfpathlineto{\pgfqpoint{4.166286in}{2.195852in}}%
\pgfpathlineto{\pgfqpoint{4.153236in}{2.198830in}}%
\pgfpathlineto{\pgfqpoint{4.160726in}{2.206520in}}%
\pgfpathlineto{\pgfqpoint{4.168211in}{2.214200in}}%
\pgfpathlineto{\pgfqpoint{4.175690in}{2.221866in}}%
\pgfpathlineto{\pgfqpoint{4.183164in}{2.229518in}}%
\pgfpathclose%
\pgfusepath{fill}%
\end{pgfscope}%
\begin{pgfscope}%
\pgfpathrectangle{\pgfqpoint{1.254980in}{0.150000in}}{\pgfqpoint{5.490039in}{5.490039in}}%
\pgfusepath{clip}%
\pgfsetbuttcap%
\pgfsetroundjoin%
\definecolor{currentfill}{rgb}{0.283187,0.125848,0.444960}%
\pgfsetfillcolor{currentfill}%
\pgfsetfillopacity{0.700000}%
\pgfsetlinewidth{0.000000pt}%
\definecolor{currentstroke}{rgb}{0.000000,0.000000,0.000000}%
\pgfsetstrokecolor{currentstroke}%
\pgfsetdash{}{0pt}%
\pgfpathmoveto{\pgfqpoint{5.346009in}{2.406572in}}%
\pgfpathlineto{\pgfqpoint{5.359360in}{2.405135in}}%
\pgfpathlineto{\pgfqpoint{5.372718in}{2.403721in}}%
\pgfpathlineto{\pgfqpoint{5.386083in}{2.402331in}}%
\pgfpathlineto{\pgfqpoint{5.399457in}{2.400965in}}%
\pgfpathlineto{\pgfqpoint{5.392465in}{2.395181in}}%
\pgfpathlineto{\pgfqpoint{5.385466in}{2.389362in}}%
\pgfpathlineto{\pgfqpoint{5.378461in}{2.383505in}}%
\pgfpathlineto{\pgfqpoint{5.371450in}{2.377607in}}%
\pgfpathlineto{\pgfqpoint{5.358060in}{2.378896in}}%
\pgfpathlineto{\pgfqpoint{5.344679in}{2.380210in}}%
\pgfpathlineto{\pgfqpoint{5.331305in}{2.381547in}}%
\pgfpathlineto{\pgfqpoint{5.317939in}{2.382907in}}%
\pgfpathlineto{\pgfqpoint{5.324966in}{2.388877in}}%
\pgfpathlineto{\pgfqpoint{5.331987in}{2.394810in}}%
\pgfpathlineto{\pgfqpoint{5.339001in}{2.400707in}}%
\pgfpathlineto{\pgfqpoint{5.346009in}{2.406572in}}%
\pgfpathclose%
\pgfusepath{fill}%
\end{pgfscope}%
\begin{pgfscope}%
\pgfpathrectangle{\pgfqpoint{1.254980in}{0.150000in}}{\pgfqpoint{5.490039in}{5.490039in}}%
\pgfusepath{clip}%
\pgfsetbuttcap%
\pgfsetroundjoin%
\definecolor{currentfill}{rgb}{0.282623,0.140926,0.457517}%
\pgfsetfillcolor{currentfill}%
\pgfsetfillopacity{0.700000}%
\pgfsetlinewidth{0.000000pt}%
\definecolor{currentstroke}{rgb}{0.000000,0.000000,0.000000}%
\pgfsetstrokecolor{currentstroke}%
\pgfsetdash{}{0pt}%
\pgfpathmoveto{\pgfqpoint{5.562175in}{2.434685in}}%
\pgfpathlineto{\pgfqpoint{5.575585in}{2.433339in}}%
\pgfpathlineto{\pgfqpoint{5.589004in}{2.432016in}}%
\pgfpathlineto{\pgfqpoint{5.602430in}{2.430716in}}%
\pgfpathlineto{\pgfqpoint{5.615865in}{2.429439in}}%
\pgfpathlineto{\pgfqpoint{5.608973in}{2.424155in}}%
\pgfpathlineto{\pgfqpoint{5.602076in}{2.418851in}}%
\pgfpathlineto{\pgfqpoint{5.595173in}{2.413522in}}%
\pgfpathlineto{\pgfqpoint{5.588263in}{2.408167in}}%
\pgfpathlineto{\pgfqpoint{5.574811in}{2.409340in}}%
\pgfpathlineto{\pgfqpoint{5.561367in}{2.410538in}}%
\pgfpathlineto{\pgfqpoint{5.547931in}{2.411758in}}%
\pgfpathlineto{\pgfqpoint{5.534502in}{2.413002in}}%
\pgfpathlineto{\pgfqpoint{5.541430in}{2.418456in}}%
\pgfpathlineto{\pgfqpoint{5.548351in}{2.423886in}}%
\pgfpathlineto{\pgfqpoint{5.555266in}{2.429294in}}%
\pgfpathlineto{\pgfqpoint{5.562175in}{2.434685in}}%
\pgfpathclose%
\pgfusepath{fill}%
\end{pgfscope}%
\begin{pgfscope}%
\pgfpathrectangle{\pgfqpoint{1.254980in}{0.150000in}}{\pgfqpoint{5.490039in}{5.490039in}}%
\pgfusepath{clip}%
\pgfsetbuttcap%
\pgfsetroundjoin%
\definecolor{currentfill}{rgb}{0.274952,0.037752,0.364543}%
\pgfsetfillcolor{currentfill}%
\pgfsetfillopacity{0.700000}%
\pgfsetlinewidth{0.000000pt}%
\definecolor{currentstroke}{rgb}{0.000000,0.000000,0.000000}%
\pgfsetstrokecolor{currentstroke}%
\pgfsetdash{}{0pt}%
\pgfpathmoveto{\pgfqpoint{4.399354in}{2.258047in}}%
\pgfpathlineto{\pgfqpoint{4.412446in}{2.255546in}}%
\pgfpathlineto{\pgfqpoint{4.425545in}{2.253071in}}%
\pgfpathlineto{\pgfqpoint{4.438650in}{2.250622in}}%
\pgfpathlineto{\pgfqpoint{4.451763in}{2.248199in}}%
\pgfpathlineto{\pgfqpoint{4.444379in}{2.240551in}}%
\pgfpathlineto{\pgfqpoint{4.436990in}{2.232869in}}%
\pgfpathlineto{\pgfqpoint{4.429595in}{2.225155in}}%
\pgfpathlineto{\pgfqpoint{4.422195in}{2.217408in}}%
\pgfpathlineto{\pgfqpoint{4.409071in}{2.219871in}}%
\pgfpathlineto{\pgfqpoint{4.395954in}{2.222360in}}%
\pgfpathlineto{\pgfqpoint{4.382844in}{2.224875in}}%
\pgfpathlineto{\pgfqpoint{4.369740in}{2.227416in}}%
\pgfpathlineto{\pgfqpoint{4.377152in}{2.235118in}}%
\pgfpathlineto{\pgfqpoint{4.384558in}{2.242791in}}%
\pgfpathlineto{\pgfqpoint{4.391958in}{2.250434in}}%
\pgfpathlineto{\pgfqpoint{4.399354in}{2.258047in}}%
\pgfpathclose%
\pgfusepath{fill}%
\end{pgfscope}%
\begin{pgfscope}%
\pgfpathrectangle{\pgfqpoint{1.254980in}{0.150000in}}{\pgfqpoint{5.490039in}{5.490039in}}%
\pgfusepath{clip}%
\pgfsetbuttcap%
\pgfsetroundjoin%
\definecolor{currentfill}{rgb}{0.268510,0.009605,0.335427}%
\pgfsetfillcolor{currentfill}%
\pgfsetfillopacity{0.700000}%
\pgfsetlinewidth{0.000000pt}%
\definecolor{currentstroke}{rgb}{0.000000,0.000000,0.000000}%
\pgfsetstrokecolor{currentstroke}%
\pgfsetdash{}{0pt}%
\pgfpathmoveto{\pgfqpoint{3.482540in}{2.208741in}}%
\pgfpathlineto{\pgfqpoint{3.495431in}{2.204065in}}%
\pgfpathlineto{\pgfqpoint{3.508327in}{2.199422in}}%
\pgfpathlineto{\pgfqpoint{3.521227in}{2.194811in}}%
\pgfpathlineto{\pgfqpoint{3.534133in}{2.190233in}}%
\pgfpathlineto{\pgfqpoint{3.526402in}{2.183998in}}%
\pgfpathlineto{\pgfqpoint{3.518663in}{2.177832in}}%
\pgfpathlineto{\pgfqpoint{3.510918in}{2.171738in}}%
\pgfpathlineto{\pgfqpoint{3.503166in}{2.165719in}}%
\pgfpathlineto{\pgfqpoint{3.490245in}{2.170440in}}%
\pgfpathlineto{\pgfqpoint{3.477329in}{2.175193in}}%
\pgfpathlineto{\pgfqpoint{3.464417in}{2.179978in}}%
\pgfpathlineto{\pgfqpoint{3.451510in}{2.184796in}}%
\pgfpathlineto{\pgfqpoint{3.459278in}{2.190668in}}%
\pgfpathlineto{\pgfqpoint{3.467039in}{2.196618in}}%
\pgfpathlineto{\pgfqpoint{3.474793in}{2.202644in}}%
\pgfpathlineto{\pgfqpoint{3.482540in}{2.208741in}}%
\pgfpathclose%
\pgfusepath{fill}%
\end{pgfscope}%
\begin{pgfscope}%
\pgfpathrectangle{\pgfqpoint{1.254980in}{0.150000in}}{\pgfqpoint{5.490039in}{5.490039in}}%
\pgfusepath{clip}%
\pgfsetbuttcap%
\pgfsetroundjoin%
\definecolor{currentfill}{rgb}{0.269944,0.014625,0.341379}%
\pgfsetfillcolor{currentfill}%
\pgfsetfillopacity{0.700000}%
\pgfsetlinewidth{0.000000pt}%
\definecolor{currentstroke}{rgb}{0.000000,0.000000,0.000000}%
\pgfsetstrokecolor{currentstroke}%
\pgfsetdash{}{0pt}%
\pgfpathmoveto{\pgfqpoint{3.348427in}{2.224534in}}%
\pgfpathlineto{\pgfqpoint{3.361296in}{2.219448in}}%
\pgfpathlineto{\pgfqpoint{3.374170in}{2.214397in}}%
\pgfpathlineto{\pgfqpoint{3.387048in}{2.209380in}}%
\pgfpathlineto{\pgfqpoint{3.399931in}{2.204396in}}%
\pgfpathlineto{\pgfqpoint{3.392140in}{2.198757in}}%
\pgfpathlineto{\pgfqpoint{3.384341in}{2.193206in}}%
\pgfpathlineto{\pgfqpoint{3.376534in}{2.187747in}}%
\pgfpathlineto{\pgfqpoint{3.368719in}{2.182385in}}%
\pgfpathlineto{\pgfqpoint{3.355820in}{2.187524in}}%
\pgfpathlineto{\pgfqpoint{3.342924in}{2.192696in}}%
\pgfpathlineto{\pgfqpoint{3.330034in}{2.197903in}}%
\pgfpathlineto{\pgfqpoint{3.317147in}{2.203143in}}%
\pgfpathlineto{\pgfqpoint{3.324979in}{2.208345in}}%
\pgfpathlineto{\pgfqpoint{3.332803in}{2.213647in}}%
\pgfpathlineto{\pgfqpoint{3.340619in}{2.219045in}}%
\pgfpathlineto{\pgfqpoint{3.348427in}{2.224534in}}%
\pgfpathclose%
\pgfusepath{fill}%
\end{pgfscope}%
\begin{pgfscope}%
\pgfpathrectangle{\pgfqpoint{1.254980in}{0.150000in}}{\pgfqpoint{5.490039in}{5.490039in}}%
\pgfusepath{clip}%
\pgfsetbuttcap%
\pgfsetroundjoin%
\definecolor{currentfill}{rgb}{0.268510,0.009605,0.335427}%
\pgfsetfillcolor{currentfill}%
\pgfsetfillopacity{0.700000}%
\pgfsetlinewidth{0.000000pt}%
\definecolor{currentstroke}{rgb}{0.000000,0.000000,0.000000}%
\pgfsetstrokecolor{currentstroke}%
\pgfsetdash{}{0pt}%
\pgfpathmoveto{\pgfqpoint{3.966967in}{2.206799in}}%
\pgfpathlineto{\pgfqpoint{3.979955in}{2.203412in}}%
\pgfpathlineto{\pgfqpoint{3.992949in}{2.200052in}}%
\pgfpathlineto{\pgfqpoint{4.005949in}{2.196720in}}%
\pgfpathlineto{\pgfqpoint{4.018955in}{2.193417in}}%
\pgfpathlineto{\pgfqpoint{4.011414in}{2.185865in}}%
\pgfpathlineto{\pgfqpoint{4.003868in}{2.178317in}}%
\pgfpathlineto{\pgfqpoint{3.996315in}{2.170776in}}%
\pgfpathlineto{\pgfqpoint{3.988758in}{2.163244in}}%
\pgfpathlineto{\pgfqpoint{3.975739in}{2.166639in}}%
\pgfpathlineto{\pgfqpoint{3.962727in}{2.170062in}}%
\pgfpathlineto{\pgfqpoint{3.949721in}{2.173513in}}%
\pgfpathlineto{\pgfqpoint{3.936720in}{2.176992in}}%
\pgfpathlineto{\pgfqpoint{3.944290in}{2.184428in}}%
\pgfpathlineto{\pgfqpoint{3.951855in}{2.191876in}}%
\pgfpathlineto{\pgfqpoint{3.959414in}{2.199334in}}%
\pgfpathlineto{\pgfqpoint{3.966967in}{2.206799in}}%
\pgfpathclose%
\pgfusepath{fill}%
\end{pgfscope}%
\begin{pgfscope}%
\pgfpathrectangle{\pgfqpoint{1.254980in}{0.150000in}}{\pgfqpoint{5.490039in}{5.490039in}}%
\pgfusepath{clip}%
\pgfsetbuttcap%
\pgfsetroundjoin%
\definecolor{currentfill}{rgb}{0.278791,0.062145,0.386592}%
\pgfsetfillcolor{currentfill}%
\pgfsetfillopacity{0.700000}%
\pgfsetlinewidth{0.000000pt}%
\definecolor{currentstroke}{rgb}{0.000000,0.000000,0.000000}%
\pgfsetstrokecolor{currentstroke}%
\pgfsetdash{}{0pt}%
\pgfpathmoveto{\pgfqpoint{4.615586in}{2.290017in}}%
\pgfpathlineto{\pgfqpoint{4.628736in}{2.287866in}}%
\pgfpathlineto{\pgfqpoint{4.641894in}{2.285740in}}%
\pgfpathlineto{\pgfqpoint{4.655058in}{2.283639in}}%
\pgfpathlineto{\pgfqpoint{4.668229in}{2.281563in}}%
\pgfpathlineto{\pgfqpoint{4.660926in}{2.274163in}}%
\pgfpathlineto{\pgfqpoint{4.653617in}{2.266718in}}%
\pgfpathlineto{\pgfqpoint{4.646303in}{2.259229in}}%
\pgfpathlineto{\pgfqpoint{4.638982in}{2.251695in}}%
\pgfpathlineto{\pgfqpoint{4.625799in}{2.253785in}}%
\pgfpathlineto{\pgfqpoint{4.612623in}{2.255901in}}%
\pgfpathlineto{\pgfqpoint{4.599454in}{2.258041in}}%
\pgfpathlineto{\pgfqpoint{4.586291in}{2.260207in}}%
\pgfpathlineto{\pgfqpoint{4.593623in}{2.267721in}}%
\pgfpathlineto{\pgfqpoint{4.600950in}{2.275194in}}%
\pgfpathlineto{\pgfqpoint{4.608271in}{2.282626in}}%
\pgfpathlineto{\pgfqpoint{4.615586in}{2.290017in}}%
\pgfpathclose%
\pgfusepath{fill}%
\end{pgfscope}%
\begin{pgfscope}%
\pgfpathrectangle{\pgfqpoint{1.254980in}{0.150000in}}{\pgfqpoint{5.490039in}{5.490039in}}%
\pgfusepath{clip}%
\pgfsetbuttcap%
\pgfsetroundjoin%
\definecolor{currentfill}{rgb}{0.267004,0.004874,0.329415}%
\pgfsetfillcolor{currentfill}%
\pgfsetfillopacity{0.700000}%
\pgfsetlinewidth{0.000000pt}%
\definecolor{currentstroke}{rgb}{0.000000,0.000000,0.000000}%
\pgfsetstrokecolor{currentstroke}%
\pgfsetdash{}{0pt}%
\pgfpathmoveto{\pgfqpoint{3.616606in}{2.198312in}}%
\pgfpathlineto{\pgfqpoint{3.629522in}{2.194020in}}%
\pgfpathlineto{\pgfqpoint{3.642444in}{2.189760in}}%
\pgfpathlineto{\pgfqpoint{3.655371in}{2.185530in}}%
\pgfpathlineto{\pgfqpoint{3.668303in}{2.181332in}}%
\pgfpathlineto{\pgfqpoint{3.660627in}{2.174606in}}%
\pgfpathlineto{\pgfqpoint{3.652944in}{2.167929in}}%
\pgfpathlineto{\pgfqpoint{3.645256in}{2.161305in}}%
\pgfpathlineto{\pgfqpoint{3.637561in}{2.154737in}}%
\pgfpathlineto{\pgfqpoint{3.624614in}{2.159066in}}%
\pgfpathlineto{\pgfqpoint{3.611673in}{2.163425in}}%
\pgfpathlineto{\pgfqpoint{3.598737in}{2.167814in}}%
\pgfpathlineto{\pgfqpoint{3.585806in}{2.172235in}}%
\pgfpathlineto{\pgfqpoint{3.593516in}{2.178669in}}%
\pgfpathlineto{\pgfqpoint{3.601219in}{2.185162in}}%
\pgfpathlineto{\pgfqpoint{3.608916in}{2.191711in}}%
\pgfpathlineto{\pgfqpoint{3.616606in}{2.198312in}}%
\pgfpathclose%
\pgfusepath{fill}%
\end{pgfscope}%
\begin{pgfscope}%
\pgfpathrectangle{\pgfqpoint{1.254980in}{0.150000in}}{\pgfqpoint{5.490039in}{5.490039in}}%
\pgfusepath{clip}%
\pgfsetbuttcap%
\pgfsetroundjoin%
\definecolor{currentfill}{rgb}{0.273809,0.031497,0.358853}%
\pgfsetfillcolor{currentfill}%
\pgfsetfillopacity{0.700000}%
\pgfsetlinewidth{0.000000pt}%
\definecolor{currentstroke}{rgb}{0.000000,0.000000,0.000000}%
\pgfsetstrokecolor{currentstroke}%
\pgfsetdash{}{0pt}%
\pgfpathmoveto{\pgfqpoint{3.214214in}{2.246331in}}%
\pgfpathlineto{\pgfqpoint{3.227065in}{2.240808in}}%
\pgfpathlineto{\pgfqpoint{3.239921in}{2.235320in}}%
\pgfpathlineto{\pgfqpoint{3.252782in}{2.229869in}}%
\pgfpathlineto{\pgfqpoint{3.265646in}{2.224453in}}%
\pgfpathlineto{\pgfqpoint{3.257789in}{2.219518in}}%
\pgfpathlineto{\pgfqpoint{3.249923in}{2.214693in}}%
\pgfpathlineto{\pgfqpoint{3.242048in}{2.209983in}}%
\pgfpathlineto{\pgfqpoint{3.234165in}{2.205391in}}%
\pgfpathlineto{\pgfqpoint{3.221282in}{2.210975in}}%
\pgfpathlineto{\pgfqpoint{3.208404in}{2.216594in}}%
\pgfpathlineto{\pgfqpoint{3.195529in}{2.222250in}}%
\pgfpathlineto{\pgfqpoint{3.182659in}{2.227942in}}%
\pgfpathlineto{\pgfqpoint{3.190561in}{2.232360in}}%
\pgfpathlineto{\pgfqpoint{3.198454in}{2.236901in}}%
\pgfpathlineto{\pgfqpoint{3.206338in}{2.241559in}}%
\pgfpathlineto{\pgfqpoint{3.214214in}{2.246331in}}%
\pgfpathclose%
\pgfusepath{fill}%
\end{pgfscope}%
\begin{pgfscope}%
\pgfpathrectangle{\pgfqpoint{1.254980in}{0.150000in}}{\pgfqpoint{5.490039in}{5.490039in}}%
\pgfusepath{clip}%
\pgfsetbuttcap%
\pgfsetroundjoin%
\definecolor{currentfill}{rgb}{0.281924,0.089666,0.412415}%
\pgfsetfillcolor{currentfill}%
\pgfsetfillopacity{0.700000}%
\pgfsetlinewidth{0.000000pt}%
\definecolor{currentstroke}{rgb}{0.000000,0.000000,0.000000}%
\pgfsetstrokecolor{currentstroke}%
\pgfsetdash{}{0pt}%
\pgfpathmoveto{\pgfqpoint{2.893943in}{2.337128in}}%
\pgfpathlineto{\pgfqpoint{2.906761in}{2.330463in}}%
\pgfpathlineto{\pgfqpoint{2.919582in}{2.323842in}}%
\pgfpathlineto{\pgfqpoint{2.932407in}{2.317262in}}%
\pgfpathlineto{\pgfqpoint{2.945234in}{2.310725in}}%
\pgfpathlineto{\pgfqpoint{2.937201in}{2.307742in}}%
\pgfpathlineto{\pgfqpoint{2.929157in}{2.304920in}}%
\pgfpathlineto{\pgfqpoint{2.921102in}{2.302266in}}%
\pgfpathlineto{\pgfqpoint{2.913035in}{2.299784in}}%
\pgfpathlineto{\pgfqpoint{2.900185in}{2.306516in}}%
\pgfpathlineto{\pgfqpoint{2.887338in}{2.313290in}}%
\pgfpathlineto{\pgfqpoint{2.874494in}{2.320107in}}%
\pgfpathlineto{\pgfqpoint{2.861654in}{2.326967in}}%
\pgfpathlineto{\pgfqpoint{2.869743in}{2.329249in}}%
\pgfpathlineto{\pgfqpoint{2.877821in}{2.331707in}}%
\pgfpathlineto{\pgfqpoint{2.885888in}{2.334335in}}%
\pgfpathlineto{\pgfqpoint{2.893943in}{2.337128in}}%
\pgfpathclose%
\pgfusepath{fill}%
\end{pgfscope}%
\begin{pgfscope}%
\pgfpathrectangle{\pgfqpoint{1.254980in}{0.150000in}}{\pgfqpoint{5.490039in}{5.490039in}}%
\pgfusepath{clip}%
\pgfsetbuttcap%
\pgfsetroundjoin%
\definecolor{currentfill}{rgb}{0.283072,0.130895,0.449241}%
\pgfsetfillcolor{currentfill}%
\pgfsetfillopacity{0.700000}%
\pgfsetlinewidth{0.000000pt}%
\definecolor{currentstroke}{rgb}{0.000000,0.000000,0.000000}%
\pgfsetstrokecolor{currentstroke}%
\pgfsetdash{}{0pt}%
\pgfpathmoveto{\pgfqpoint{2.707789in}{2.412791in}}%
\pgfpathlineto{\pgfqpoint{2.720596in}{2.405382in}}%
\pgfpathlineto{\pgfqpoint{2.733406in}{2.398022in}}%
\pgfpathlineto{\pgfqpoint{2.746218in}{2.390709in}}%
\pgfpathlineto{\pgfqpoint{2.759034in}{2.383443in}}%
\pgfpathlineto{\pgfqpoint{2.750884in}{2.381748in}}%
\pgfpathlineto{\pgfqpoint{2.742721in}{2.380246in}}%
\pgfpathlineto{\pgfqpoint{2.734545in}{2.378941in}}%
\pgfpathlineto{\pgfqpoint{2.726356in}{2.377840in}}%
\pgfpathlineto{\pgfqpoint{2.713516in}{2.385314in}}%
\pgfpathlineto{\pgfqpoint{2.700678in}{2.392836in}}%
\pgfpathlineto{\pgfqpoint{2.687843in}{2.400406in}}%
\pgfpathlineto{\pgfqpoint{2.675010in}{2.408025in}}%
\pgfpathlineto{\pgfqpoint{2.683225in}{2.408912in}}%
\pgfpathlineto{\pgfqpoint{2.691426in}{2.410006in}}%
\pgfpathlineto{\pgfqpoint{2.699614in}{2.411301in}}%
\pgfpathlineto{\pgfqpoint{2.707789in}{2.412791in}}%
\pgfpathclose%
\pgfusepath{fill}%
\end{pgfscope}%
\begin{pgfscope}%
\pgfpathrectangle{\pgfqpoint{1.254980in}{0.150000in}}{\pgfqpoint{5.490039in}{5.490039in}}%
\pgfusepath{clip}%
\pgfsetbuttcap%
\pgfsetroundjoin%
\definecolor{currentfill}{rgb}{0.280894,0.078907,0.402329}%
\pgfsetfillcolor{currentfill}%
\pgfsetfillopacity{0.700000}%
\pgfsetlinewidth{0.000000pt}%
\definecolor{currentstroke}{rgb}{0.000000,0.000000,0.000000}%
\pgfsetstrokecolor{currentstroke}%
\pgfsetdash{}{0pt}%
\pgfpathmoveto{\pgfqpoint{4.831874in}{2.323442in}}%
\pgfpathlineto{\pgfqpoint{4.845085in}{2.321581in}}%
\pgfpathlineto{\pgfqpoint{4.858304in}{2.319744in}}%
\pgfpathlineto{\pgfqpoint{4.871529in}{2.317931in}}%
\pgfpathlineto{\pgfqpoint{4.884762in}{2.316144in}}%
\pgfpathlineto{\pgfqpoint{4.877544in}{2.309129in}}%
\pgfpathlineto{\pgfqpoint{4.870319in}{2.302064in}}%
\pgfpathlineto{\pgfqpoint{4.863088in}{2.294950in}}%
\pgfpathlineto{\pgfqpoint{4.855852in}{2.287784in}}%
\pgfpathlineto{\pgfqpoint{4.842606in}{2.289560in}}%
\pgfpathlineto{\pgfqpoint{4.829368in}{2.291361in}}%
\pgfpathlineto{\pgfqpoint{4.816137in}{2.293186in}}%
\pgfpathlineto{\pgfqpoint{4.802914in}{2.295036in}}%
\pgfpathlineto{\pgfqpoint{4.810163in}{2.302208in}}%
\pgfpathlineto{\pgfqpoint{4.817406in}{2.309333in}}%
\pgfpathlineto{\pgfqpoint{4.824643in}{2.316411in}}%
\pgfpathlineto{\pgfqpoint{4.831874in}{2.323442in}}%
\pgfpathclose%
\pgfusepath{fill}%
\end{pgfscope}%
\begin{pgfscope}%
\pgfpathrectangle{\pgfqpoint{1.254980in}{0.150000in}}{\pgfqpoint{5.490039in}{5.490039in}}%
\pgfusepath{clip}%
\pgfsetbuttcap%
\pgfsetroundjoin%
\definecolor{currentfill}{rgb}{0.267004,0.004874,0.329415}%
\pgfsetfillcolor{currentfill}%
\pgfsetfillopacity{0.700000}%
\pgfsetlinewidth{0.000000pt}%
\definecolor{currentstroke}{rgb}{0.000000,0.000000,0.000000}%
\pgfsetstrokecolor{currentstroke}%
\pgfsetdash{}{0pt}%
\pgfpathmoveto{\pgfqpoint{3.750671in}{2.192649in}}%
\pgfpathlineto{\pgfqpoint{3.763617in}{2.188718in}}%
\pgfpathlineto{\pgfqpoint{3.776567in}{2.184817in}}%
\pgfpathlineto{\pgfqpoint{3.789524in}{2.180945in}}%
\pgfpathlineto{\pgfqpoint{3.802486in}{2.177103in}}%
\pgfpathlineto{\pgfqpoint{3.794861in}{2.169983in}}%
\pgfpathlineto{\pgfqpoint{3.787231in}{2.162895in}}%
\pgfpathlineto{\pgfqpoint{3.779595in}{2.155842in}}%
\pgfpathlineto{\pgfqpoint{3.771952in}{2.148826in}}%
\pgfpathlineto{\pgfqpoint{3.758977in}{2.152785in}}%
\pgfpathlineto{\pgfqpoint{3.746007in}{2.156774in}}%
\pgfpathlineto{\pgfqpoint{3.733043in}{2.160792in}}%
\pgfpathlineto{\pgfqpoint{3.720084in}{2.164840in}}%
\pgfpathlineto{\pgfqpoint{3.727740in}{2.171734in}}%
\pgfpathlineto{\pgfqpoint{3.735390in}{2.178668in}}%
\pgfpathlineto{\pgfqpoint{3.743034in}{2.185641in}}%
\pgfpathlineto{\pgfqpoint{3.750671in}{2.192649in}}%
\pgfpathclose%
\pgfusepath{fill}%
\end{pgfscope}%
\begin{pgfscope}%
\pgfpathrectangle{\pgfqpoint{1.254980in}{0.150000in}}{\pgfqpoint{5.490039in}{5.490039in}}%
\pgfusepath{clip}%
\pgfsetbuttcap%
\pgfsetroundjoin%
\definecolor{currentfill}{rgb}{0.282656,0.100196,0.422160}%
\pgfsetfillcolor{currentfill}%
\pgfsetfillopacity{0.700000}%
\pgfsetlinewidth{0.000000pt}%
\definecolor{currentstroke}{rgb}{0.000000,0.000000,0.000000}%
\pgfsetstrokecolor{currentstroke}%
\pgfsetdash{}{0pt}%
\pgfpathmoveto{\pgfqpoint{5.048207in}{2.356711in}}%
\pgfpathlineto{\pgfqpoint{5.061480in}{2.355081in}}%
\pgfpathlineto{\pgfqpoint{5.074760in}{2.353475in}}%
\pgfpathlineto{\pgfqpoint{5.088048in}{2.351893in}}%
\pgfpathlineto{\pgfqpoint{5.101344in}{2.350335in}}%
\pgfpathlineto{\pgfqpoint{5.094215in}{2.343799in}}%
\pgfpathlineto{\pgfqpoint{5.087080in}{2.337214in}}%
\pgfpathlineto{\pgfqpoint{5.079939in}{2.330578in}}%
\pgfpathlineto{\pgfqpoint{5.072791in}{2.323891in}}%
\pgfpathlineto{\pgfqpoint{5.059482in}{2.325411in}}%
\pgfpathlineto{\pgfqpoint{5.046180in}{2.326955in}}%
\pgfpathlineto{\pgfqpoint{5.032886in}{2.328523in}}%
\pgfpathlineto{\pgfqpoint{5.019599in}{2.330116in}}%
\pgfpathlineto{\pgfqpoint{5.026761in}{2.336836in}}%
\pgfpathlineto{\pgfqpoint{5.033916in}{2.343507in}}%
\pgfpathlineto{\pgfqpoint{5.041064in}{2.350132in}}%
\pgfpathlineto{\pgfqpoint{5.048207in}{2.356711in}}%
\pgfpathclose%
\pgfusepath{fill}%
\end{pgfscope}%
\begin{pgfscope}%
\pgfpathrectangle{\pgfqpoint{1.254980in}{0.150000in}}{\pgfqpoint{5.490039in}{5.490039in}}%
\pgfusepath{clip}%
\pgfsetbuttcap%
\pgfsetroundjoin%
\definecolor{currentfill}{rgb}{0.277018,0.050344,0.375715}%
\pgfsetfillcolor{currentfill}%
\pgfsetfillopacity{0.700000}%
\pgfsetlinewidth{0.000000pt}%
\definecolor{currentstroke}{rgb}{0.000000,0.000000,0.000000}%
\pgfsetstrokecolor{currentstroke}%
\pgfsetdash{}{0pt}%
\pgfpathmoveto{\pgfqpoint{3.079839in}{2.274817in}}%
\pgfpathlineto{\pgfqpoint{3.092678in}{2.268824in}}%
\pgfpathlineto{\pgfqpoint{3.105521in}{2.262871in}}%
\pgfpathlineto{\pgfqpoint{3.118367in}{2.256955in}}%
\pgfpathlineto{\pgfqpoint{3.131218in}{2.251078in}}%
\pgfpathlineto{\pgfqpoint{3.123287in}{2.246962in}}%
\pgfpathlineto{\pgfqpoint{3.115347in}{2.242980in}}%
\pgfpathlineto{\pgfqpoint{3.107398in}{2.239137in}}%
\pgfpathlineto{\pgfqpoint{3.099439in}{2.235437in}}%
\pgfpathlineto{\pgfqpoint{3.086568in}{2.241496in}}%
\pgfpathlineto{\pgfqpoint{3.073702in}{2.247593in}}%
\pgfpathlineto{\pgfqpoint{3.060839in}{2.253728in}}%
\pgfpathlineto{\pgfqpoint{3.047980in}{2.259902in}}%
\pgfpathlineto{\pgfqpoint{3.055959in}{2.263416in}}%
\pgfpathlineto{\pgfqpoint{3.063929in}{2.267076in}}%
\pgfpathlineto{\pgfqpoint{3.071889in}{2.270878in}}%
\pgfpathlineto{\pgfqpoint{3.079839in}{2.274817in}}%
\pgfpathclose%
\pgfusepath{fill}%
\end{pgfscope}%
\begin{pgfscope}%
\pgfpathrectangle{\pgfqpoint{1.254980in}{0.150000in}}{\pgfqpoint{5.490039in}{5.490039in}}%
\pgfusepath{clip}%
\pgfsetbuttcap%
\pgfsetroundjoin%
\definecolor{currentfill}{rgb}{0.273809,0.031497,0.358853}%
\pgfsetfillcolor{currentfill}%
\pgfsetfillopacity{0.700000}%
\pgfsetlinewidth{0.000000pt}%
\definecolor{currentstroke}{rgb}{0.000000,0.000000,0.000000}%
\pgfsetstrokecolor{currentstroke}%
\pgfsetdash{}{0pt}%
\pgfpathmoveto{\pgfqpoint{4.317391in}{2.237843in}}%
\pgfpathlineto{\pgfqpoint{4.330468in}{2.235197in}}%
\pgfpathlineto{\pgfqpoint{4.343552in}{2.232577in}}%
\pgfpathlineto{\pgfqpoint{4.356643in}{2.229984in}}%
\pgfpathlineto{\pgfqpoint{4.369740in}{2.227416in}}%
\pgfpathlineto{\pgfqpoint{4.362323in}{2.219686in}}%
\pgfpathlineto{\pgfqpoint{4.354900in}{2.211928in}}%
\pgfpathlineto{\pgfqpoint{4.347472in}{2.204143in}}%
\pgfpathlineto{\pgfqpoint{4.340038in}{2.196333in}}%
\pgfpathlineto{\pgfqpoint{4.326929in}{2.198953in}}%
\pgfpathlineto{\pgfqpoint{4.313828in}{2.201599in}}%
\pgfpathlineto{\pgfqpoint{4.300732in}{2.204272in}}%
\pgfpathlineto{\pgfqpoint{4.287643in}{2.206971in}}%
\pgfpathlineto{\pgfqpoint{4.295088in}{2.214723in}}%
\pgfpathlineto{\pgfqpoint{4.302528in}{2.222454in}}%
\pgfpathlineto{\pgfqpoint{4.309962in}{2.230160in}}%
\pgfpathlineto{\pgfqpoint{4.317391in}{2.237843in}}%
\pgfpathclose%
\pgfusepath{fill}%
\end{pgfscope}%
\begin{pgfscope}%
\pgfpathrectangle{\pgfqpoint{1.254980in}{0.150000in}}{\pgfqpoint{5.490039in}{5.490039in}}%
\pgfusepath{clip}%
\pgfsetbuttcap%
\pgfsetroundjoin%
\definecolor{currentfill}{rgb}{0.269944,0.014625,0.341379}%
\pgfsetfillcolor{currentfill}%
\pgfsetfillopacity{0.700000}%
\pgfsetlinewidth{0.000000pt}%
\definecolor{currentstroke}{rgb}{0.000000,0.000000,0.000000}%
\pgfsetstrokecolor{currentstroke}%
\pgfsetdash{}{0pt}%
\pgfpathmoveto{\pgfqpoint{4.101101in}{2.211013in}}%
\pgfpathlineto{\pgfqpoint{4.114126in}{2.207927in}}%
\pgfpathlineto{\pgfqpoint{4.127156in}{2.204867in}}%
\pgfpathlineto{\pgfqpoint{4.140193in}{2.201835in}}%
\pgfpathlineto{\pgfqpoint{4.153236in}{2.198830in}}%
\pgfpathlineto{\pgfqpoint{4.145741in}{2.191129in}}%
\pgfpathlineto{\pgfqpoint{4.138240in}{2.183420in}}%
\pgfpathlineto{\pgfqpoint{4.130734in}{2.175703in}}%
\pgfpathlineto{\pgfqpoint{4.123222in}{2.167982in}}%
\pgfpathlineto{\pgfqpoint{4.110167in}{2.171066in}}%
\pgfpathlineto{\pgfqpoint{4.097119in}{2.174176in}}%
\pgfpathlineto{\pgfqpoint{4.084076in}{2.177314in}}%
\pgfpathlineto{\pgfqpoint{4.071040in}{2.180480in}}%
\pgfpathlineto{\pgfqpoint{4.078563in}{2.188118in}}%
\pgfpathlineto{\pgfqpoint{4.086082in}{2.195754in}}%
\pgfpathlineto{\pgfqpoint{4.093594in}{2.203386in}}%
\pgfpathlineto{\pgfqpoint{4.101101in}{2.211013in}}%
\pgfpathclose%
\pgfusepath{fill}%
\end{pgfscope}%
\begin{pgfscope}%
\pgfpathrectangle{\pgfqpoint{1.254980in}{0.150000in}}{\pgfqpoint{5.490039in}{5.490039in}}%
\pgfusepath{clip}%
\pgfsetbuttcap%
\pgfsetroundjoin%
\definecolor{currentfill}{rgb}{0.283229,0.120777,0.440584}%
\pgfsetfillcolor{currentfill}%
\pgfsetfillopacity{0.700000}%
\pgfsetlinewidth{0.000000pt}%
\definecolor{currentstroke}{rgb}{0.000000,0.000000,0.000000}%
\pgfsetstrokecolor{currentstroke}%
\pgfsetdash{}{0pt}%
\pgfpathmoveto{\pgfqpoint{5.264553in}{2.388588in}}%
\pgfpathlineto{\pgfqpoint{5.277888in}{2.387132in}}%
\pgfpathlineto{\pgfqpoint{5.291230in}{2.385700in}}%
\pgfpathlineto{\pgfqpoint{5.304581in}{2.384292in}}%
\pgfpathlineto{\pgfqpoint{5.317939in}{2.382907in}}%
\pgfpathlineto{\pgfqpoint{5.310905in}{2.376897in}}%
\pgfpathlineto{\pgfqpoint{5.303865in}{2.370845in}}%
\pgfpathlineto{\pgfqpoint{5.296819in}{2.364748in}}%
\pgfpathlineto{\pgfqpoint{5.289766in}{2.358604in}}%
\pgfpathlineto{\pgfqpoint{5.276393in}{2.359925in}}%
\pgfpathlineto{\pgfqpoint{5.263027in}{2.361269in}}%
\pgfpathlineto{\pgfqpoint{5.249669in}{2.362638in}}%
\pgfpathlineto{\pgfqpoint{5.236319in}{2.364030in}}%
\pgfpathlineto{\pgfqpoint{5.243387in}{2.370233in}}%
\pgfpathlineto{\pgfqpoint{5.250449in}{2.376392in}}%
\pgfpathlineto{\pgfqpoint{5.257504in}{2.382510in}}%
\pgfpathlineto{\pgfqpoint{5.264553in}{2.388588in}}%
\pgfpathclose%
\pgfusepath{fill}%
\end{pgfscope}%
\begin{pgfscope}%
\pgfpathrectangle{\pgfqpoint{1.254980in}{0.150000in}}{\pgfqpoint{5.490039in}{5.490039in}}%
\pgfusepath{clip}%
\pgfsetbuttcap%
\pgfsetroundjoin%
\definecolor{currentfill}{rgb}{0.280255,0.165693,0.476498}%
\pgfsetfillcolor{currentfill}%
\pgfsetfillopacity{0.700000}%
\pgfsetlinewidth{0.000000pt}%
\definecolor{currentstroke}{rgb}{0.000000,0.000000,0.000000}%
\pgfsetstrokecolor{currentstroke}%
\pgfsetdash{}{0pt}%
\pgfpathmoveto{\pgfqpoint{5.913248in}{2.469153in}}%
\pgfpathlineto{\pgfqpoint{5.926761in}{2.467885in}}%
\pgfpathlineto{\pgfqpoint{5.940283in}{2.466640in}}%
\pgfpathlineto{\pgfqpoint{5.953813in}{2.465417in}}%
\pgfpathlineto{\pgfqpoint{5.967351in}{2.464218in}}%
\pgfpathlineto{\pgfqpoint{5.960625in}{2.459617in}}%
\pgfpathlineto{\pgfqpoint{5.953894in}{2.455027in}}%
\pgfpathlineto{\pgfqpoint{5.947158in}{2.450443in}}%
\pgfpathlineto{\pgfqpoint{5.940416in}{2.445862in}}%
\pgfpathlineto{\pgfqpoint{5.926857in}{2.446920in}}%
\pgfpathlineto{\pgfqpoint{5.913306in}{2.448000in}}%
\pgfpathlineto{\pgfqpoint{5.899764in}{2.449104in}}%
\pgfpathlineto{\pgfqpoint{5.886230in}{2.450230in}}%
\pgfpathlineto{\pgfqpoint{5.892992in}{2.454949in}}%
\pgfpathlineto{\pgfqpoint{5.899749in}{2.459672in}}%
\pgfpathlineto{\pgfqpoint{5.906501in}{2.464406in}}%
\pgfpathlineto{\pgfqpoint{5.913248in}{2.469153in}}%
\pgfpathclose%
\pgfusepath{fill}%
\end{pgfscope}%
\begin{pgfscope}%
\pgfpathrectangle{\pgfqpoint{1.254980in}{0.150000in}}{\pgfqpoint{5.490039in}{5.490039in}}%
\pgfusepath{clip}%
\pgfsetbuttcap%
\pgfsetroundjoin%
\definecolor{currentfill}{rgb}{0.282884,0.135920,0.453427}%
\pgfsetfillcolor{currentfill}%
\pgfsetfillopacity{0.700000}%
\pgfsetlinewidth{0.000000pt}%
\definecolor{currentstroke}{rgb}{0.000000,0.000000,0.000000}%
\pgfsetstrokecolor{currentstroke}%
\pgfsetdash{}{0pt}%
\pgfpathmoveto{\pgfqpoint{5.480870in}{2.418213in}}%
\pgfpathlineto{\pgfqpoint{5.494266in}{2.416875in}}%
\pgfpathlineto{\pgfqpoint{5.507670in}{2.415561in}}%
\pgfpathlineto{\pgfqpoint{5.521082in}{2.414270in}}%
\pgfpathlineto{\pgfqpoint{5.534502in}{2.413002in}}%
\pgfpathlineto{\pgfqpoint{5.527569in}{2.407521in}}%
\pgfpathlineto{\pgfqpoint{5.520629in}{2.402010in}}%
\pgfpathlineto{\pgfqpoint{5.513683in}{2.396465in}}%
\pgfpathlineto{\pgfqpoint{5.506730in}{2.390884in}}%
\pgfpathlineto{\pgfqpoint{5.493293in}{2.392062in}}%
\pgfpathlineto{\pgfqpoint{5.479864in}{2.393263in}}%
\pgfpathlineto{\pgfqpoint{5.466443in}{2.394488in}}%
\pgfpathlineto{\pgfqpoint{5.453030in}{2.395736in}}%
\pgfpathlineto{\pgfqpoint{5.459999in}{2.401402in}}%
\pgfpathlineto{\pgfqpoint{5.466962in}{2.407035in}}%
\pgfpathlineto{\pgfqpoint{5.473919in}{2.412638in}}%
\pgfpathlineto{\pgfqpoint{5.480870in}{2.418213in}}%
\pgfpathclose%
\pgfusepath{fill}%
\end{pgfscope}%
\begin{pgfscope}%
\pgfpathrectangle{\pgfqpoint{1.254980in}{0.150000in}}{\pgfqpoint{5.490039in}{5.490039in}}%
\pgfusepath{clip}%
\pgfsetbuttcap%
\pgfsetroundjoin%
\definecolor{currentfill}{rgb}{0.281887,0.150881,0.465405}%
\pgfsetfillcolor{currentfill}%
\pgfsetfillopacity{0.700000}%
\pgfsetlinewidth{0.000000pt}%
\definecolor{currentstroke}{rgb}{0.000000,0.000000,0.000000}%
\pgfsetstrokecolor{currentstroke}%
\pgfsetdash{}{0pt}%
\pgfpathmoveto{\pgfqpoint{5.697114in}{2.445103in}}%
\pgfpathlineto{\pgfqpoint{5.710570in}{2.443828in}}%
\pgfpathlineto{\pgfqpoint{5.724035in}{2.442575in}}%
\pgfpathlineto{\pgfqpoint{5.737507in}{2.441346in}}%
\pgfpathlineto{\pgfqpoint{5.750987in}{2.440140in}}%
\pgfpathlineto{\pgfqpoint{5.744157in}{2.435144in}}%
\pgfpathlineto{\pgfqpoint{5.737321in}{2.430136in}}%
\pgfpathlineto{\pgfqpoint{5.730478in}{2.425111in}}%
\pgfpathlineto{\pgfqpoint{5.723630in}{2.420066in}}%
\pgfpathlineto{\pgfqpoint{5.710131in}{2.421156in}}%
\pgfpathlineto{\pgfqpoint{5.696640in}{2.422270in}}%
\pgfpathlineto{\pgfqpoint{5.683157in}{2.423406in}}%
\pgfpathlineto{\pgfqpoint{5.669682in}{2.424566in}}%
\pgfpathlineto{\pgfqpoint{5.676549in}{2.429722in}}%
\pgfpathlineto{\pgfqpoint{5.683410in}{2.434861in}}%
\pgfpathlineto{\pgfqpoint{5.690265in}{2.439987in}}%
\pgfpathlineto{\pgfqpoint{5.697114in}{2.445103in}}%
\pgfpathclose%
\pgfusepath{fill}%
\end{pgfscope}%
\begin{pgfscope}%
\pgfpathrectangle{\pgfqpoint{1.254980in}{0.150000in}}{\pgfqpoint{5.490039in}{5.490039in}}%
\pgfusepath{clip}%
\pgfsetbuttcap%
\pgfsetroundjoin%
\definecolor{currentfill}{rgb}{0.277018,0.050344,0.375715}%
\pgfsetfillcolor{currentfill}%
\pgfsetfillopacity{0.700000}%
\pgfsetlinewidth{0.000000pt}%
\definecolor{currentstroke}{rgb}{0.000000,0.000000,0.000000}%
\pgfsetstrokecolor{currentstroke}%
\pgfsetdash{}{0pt}%
\pgfpathmoveto{\pgfqpoint{4.533711in}{2.269124in}}%
\pgfpathlineto{\pgfqpoint{4.546846in}{2.266856in}}%
\pgfpathlineto{\pgfqpoint{4.559988in}{2.264614in}}%
\pgfpathlineto{\pgfqpoint{4.573136in}{2.262398in}}%
\pgfpathlineto{\pgfqpoint{4.586291in}{2.260207in}}%
\pgfpathlineto{\pgfqpoint{4.578953in}{2.252651in}}%
\pgfpathlineto{\pgfqpoint{4.571610in}{2.245054in}}%
\pgfpathlineto{\pgfqpoint{4.564260in}{2.237417in}}%
\pgfpathlineto{\pgfqpoint{4.556905in}{2.229738in}}%
\pgfpathlineto{\pgfqpoint{4.543739in}{2.231957in}}%
\pgfpathlineto{\pgfqpoint{4.530579in}{2.234200in}}%
\pgfpathlineto{\pgfqpoint{4.517425in}{2.236470in}}%
\pgfpathlineto{\pgfqpoint{4.504279in}{2.238764in}}%
\pgfpathlineto{\pgfqpoint{4.511646in}{2.246410in}}%
\pgfpathlineto{\pgfqpoint{4.519007in}{2.254019in}}%
\pgfpathlineto{\pgfqpoint{4.526362in}{2.261590in}}%
\pgfpathlineto{\pgfqpoint{4.533711in}{2.269124in}}%
\pgfpathclose%
\pgfusepath{fill}%
\end{pgfscope}%
\begin{pgfscope}%
\pgfpathrectangle{\pgfqpoint{1.254980in}{0.150000in}}{\pgfqpoint{5.490039in}{5.490039in}}%
\pgfusepath{clip}%
\pgfsetbuttcap%
\pgfsetroundjoin%
\definecolor{currentfill}{rgb}{0.267004,0.004874,0.329415}%
\pgfsetfillcolor{currentfill}%
\pgfsetfillopacity{0.700000}%
\pgfsetlinewidth{0.000000pt}%
\definecolor{currentstroke}{rgb}{0.000000,0.000000,0.000000}%
\pgfsetstrokecolor{currentstroke}%
\pgfsetdash{}{0pt}%
\pgfpathmoveto{\pgfqpoint{3.884777in}{2.191192in}}%
\pgfpathlineto{\pgfqpoint{3.897754in}{2.187599in}}%
\pgfpathlineto{\pgfqpoint{3.910737in}{2.184035in}}%
\pgfpathlineto{\pgfqpoint{3.923726in}{2.180499in}}%
\pgfpathlineto{\pgfqpoint{3.936720in}{2.176992in}}%
\pgfpathlineto{\pgfqpoint{3.929145in}{2.169569in}}%
\pgfpathlineto{\pgfqpoint{3.921563in}{2.162163in}}%
\pgfpathlineto{\pgfqpoint{3.913976in}{2.154775in}}%
\pgfpathlineto{\pgfqpoint{3.906384in}{2.147408in}}%
\pgfpathlineto{\pgfqpoint{3.893376in}{2.151020in}}%
\pgfpathlineto{\pgfqpoint{3.880375in}{2.154660in}}%
\pgfpathlineto{\pgfqpoint{3.867379in}{2.158328in}}%
\pgfpathlineto{\pgfqpoint{3.854389in}{2.162025in}}%
\pgfpathlineto{\pgfqpoint{3.861995in}{2.169283in}}%
\pgfpathlineto{\pgfqpoint{3.869594in}{2.176566in}}%
\pgfpathlineto{\pgfqpoint{3.877188in}{2.183869in}}%
\pgfpathlineto{\pgfqpoint{3.884777in}{2.191192in}}%
\pgfpathclose%
\pgfusepath{fill}%
\end{pgfscope}%
\begin{pgfscope}%
\pgfpathrectangle{\pgfqpoint{1.254980in}{0.150000in}}{\pgfqpoint{5.490039in}{5.490039in}}%
\pgfusepath{clip}%
\pgfsetbuttcap%
\pgfsetroundjoin%
\definecolor{currentfill}{rgb}{0.280267,0.073417,0.397163}%
\pgfsetfillcolor{currentfill}%
\pgfsetfillopacity{0.700000}%
\pgfsetlinewidth{0.000000pt}%
\definecolor{currentstroke}{rgb}{0.000000,0.000000,0.000000}%
\pgfsetstrokecolor{currentstroke}%
\pgfsetdash{}{0pt}%
\pgfpathmoveto{\pgfqpoint{4.750091in}{2.302684in}}%
\pgfpathlineto{\pgfqpoint{4.763286in}{2.300735in}}%
\pgfpathlineto{\pgfqpoint{4.776488in}{2.298811in}}%
\pgfpathlineto{\pgfqpoint{4.789697in}{2.296911in}}%
\pgfpathlineto{\pgfqpoint{4.802914in}{2.295036in}}%
\pgfpathlineto{\pgfqpoint{4.795658in}{2.287816in}}%
\pgfpathlineto{\pgfqpoint{4.788397in}{2.280546in}}%
\pgfpathlineto{\pgfqpoint{4.781130in}{2.273227in}}%
\pgfpathlineto{\pgfqpoint{4.773857in}{2.265858in}}%
\pgfpathlineto{\pgfqpoint{4.760628in}{2.267734in}}%
\pgfpathlineto{\pgfqpoint{4.747407in}{2.269635in}}%
\pgfpathlineto{\pgfqpoint{4.734193in}{2.271560in}}%
\pgfpathlineto{\pgfqpoint{4.720986in}{2.273511in}}%
\pgfpathlineto{\pgfqpoint{4.728271in}{2.280874in}}%
\pgfpathlineto{\pgfqpoint{4.735550in}{2.288190in}}%
\pgfpathlineto{\pgfqpoint{4.742824in}{2.295460in}}%
\pgfpathlineto{\pgfqpoint{4.750091in}{2.302684in}}%
\pgfpathclose%
\pgfusepath{fill}%
\end{pgfscope}%
\begin{pgfscope}%
\pgfpathrectangle{\pgfqpoint{1.254980in}{0.150000in}}{\pgfqpoint{5.490039in}{5.490039in}}%
\pgfusepath{clip}%
\pgfsetbuttcap%
\pgfsetroundjoin%
\definecolor{currentfill}{rgb}{0.283229,0.120777,0.440584}%
\pgfsetfillcolor{currentfill}%
\pgfsetfillopacity{0.700000}%
\pgfsetlinewidth{0.000000pt}%
\definecolor{currentstroke}{rgb}{0.000000,0.000000,0.000000}%
\pgfsetstrokecolor{currentstroke}%
\pgfsetdash{}{0pt}%
\pgfpathmoveto{\pgfqpoint{2.759034in}{2.383443in}}%
\pgfpathlineto{\pgfqpoint{2.771851in}{2.376224in}}%
\pgfpathlineto{\pgfqpoint{2.784672in}{2.369052in}}%
\pgfpathlineto{\pgfqpoint{2.797495in}{2.361925in}}%
\pgfpathlineto{\pgfqpoint{2.810321in}{2.354844in}}%
\pgfpathlineto{\pgfqpoint{2.802195in}{2.352946in}}%
\pgfpathlineto{\pgfqpoint{2.794057in}{2.351237in}}%
\pgfpathlineto{\pgfqpoint{2.785906in}{2.349722in}}%
\pgfpathlineto{\pgfqpoint{2.777742in}{2.348406in}}%
\pgfpathlineto{\pgfqpoint{2.764892in}{2.355695in}}%
\pgfpathlineto{\pgfqpoint{2.752044in}{2.363031in}}%
\pgfpathlineto{\pgfqpoint{2.739198in}{2.370412in}}%
\pgfpathlineto{\pgfqpoint{2.726356in}{2.377840in}}%
\pgfpathlineto{\pgfqpoint{2.734545in}{2.378941in}}%
\pgfpathlineto{\pgfqpoint{2.742721in}{2.380246in}}%
\pgfpathlineto{\pgfqpoint{2.750884in}{2.381748in}}%
\pgfpathlineto{\pgfqpoint{2.759034in}{2.383443in}}%
\pgfpathclose%
\pgfusepath{fill}%
\end{pgfscope}%
\begin{pgfscope}%
\pgfpathrectangle{\pgfqpoint{1.254980in}{0.150000in}}{\pgfqpoint{5.490039in}{5.490039in}}%
\pgfusepath{clip}%
\pgfsetbuttcap%
\pgfsetroundjoin%
\definecolor{currentfill}{rgb}{0.269944,0.014625,0.341379}%
\pgfsetfillcolor{currentfill}%
\pgfsetfillopacity{0.700000}%
\pgfsetlinewidth{0.000000pt}%
\definecolor{currentstroke}{rgb}{0.000000,0.000000,0.000000}%
\pgfsetstrokecolor{currentstroke}%
\pgfsetdash{}{0pt}%
\pgfpathmoveto{\pgfqpoint{3.399931in}{2.204396in}}%
\pgfpathlineto{\pgfqpoint{3.412819in}{2.199446in}}%
\pgfpathlineto{\pgfqpoint{3.425712in}{2.194530in}}%
\pgfpathlineto{\pgfqpoint{3.438609in}{2.189646in}}%
\pgfpathlineto{\pgfqpoint{3.451510in}{2.184796in}}%
\pgfpathlineto{\pgfqpoint{3.443735in}{2.179006in}}%
\pgfpathlineto{\pgfqpoint{3.435953in}{2.173301in}}%
\pgfpathlineto{\pgfqpoint{3.428162in}{2.167686in}}%
\pgfpathlineto{\pgfqpoint{3.420365in}{2.162164in}}%
\pgfpathlineto{\pgfqpoint{3.407446in}{2.167170in}}%
\pgfpathlineto{\pgfqpoint{3.394533in}{2.172208in}}%
\pgfpathlineto{\pgfqpoint{3.381624in}{2.177280in}}%
\pgfpathlineto{\pgfqpoint{3.368719in}{2.182385in}}%
\pgfpathlineto{\pgfqpoint{3.376534in}{2.187747in}}%
\pgfpathlineto{\pgfqpoint{3.384341in}{2.193206in}}%
\pgfpathlineto{\pgfqpoint{3.392140in}{2.198757in}}%
\pgfpathlineto{\pgfqpoint{3.399931in}{2.204396in}}%
\pgfpathclose%
\pgfusepath{fill}%
\end{pgfscope}%
\begin{pgfscope}%
\pgfpathrectangle{\pgfqpoint{1.254980in}{0.150000in}}{\pgfqpoint{5.490039in}{5.490039in}}%
\pgfusepath{clip}%
\pgfsetbuttcap%
\pgfsetroundjoin%
\definecolor{currentfill}{rgb}{0.280894,0.078907,0.402329}%
\pgfsetfillcolor{currentfill}%
\pgfsetfillopacity{0.700000}%
\pgfsetlinewidth{0.000000pt}%
\definecolor{currentstroke}{rgb}{0.000000,0.000000,0.000000}%
\pgfsetstrokecolor{currentstroke}%
\pgfsetdash{}{0pt}%
\pgfpathmoveto{\pgfqpoint{2.945234in}{2.310725in}}%
\pgfpathlineto{\pgfqpoint{2.958065in}{2.304230in}}%
\pgfpathlineto{\pgfqpoint{2.970900in}{2.297776in}}%
\pgfpathlineto{\pgfqpoint{2.983738in}{2.291363in}}%
\pgfpathlineto{\pgfqpoint{2.996579in}{2.284991in}}%
\pgfpathlineto{\pgfqpoint{2.988568in}{2.281817in}}%
\pgfpathlineto{\pgfqpoint{2.980545in}{2.278803in}}%
\pgfpathlineto{\pgfqpoint{2.972512in}{2.275952in}}%
\pgfpathlineto{\pgfqpoint{2.964468in}{2.273270in}}%
\pgfpathlineto{\pgfqpoint{2.951605in}{2.279837in}}%
\pgfpathlineto{\pgfqpoint{2.938745in}{2.286445in}}%
\pgfpathlineto{\pgfqpoint{2.925889in}{2.293094in}}%
\pgfpathlineto{\pgfqpoint{2.913035in}{2.299784in}}%
\pgfpathlineto{\pgfqpoint{2.921102in}{2.302266in}}%
\pgfpathlineto{\pgfqpoint{2.929157in}{2.304920in}}%
\pgfpathlineto{\pgfqpoint{2.937201in}{2.307742in}}%
\pgfpathlineto{\pgfqpoint{2.945234in}{2.310725in}}%
\pgfpathclose%
\pgfusepath{fill}%
\end{pgfscope}%
\begin{pgfscope}%
\pgfpathrectangle{\pgfqpoint{1.254980in}{0.150000in}}{\pgfqpoint{5.490039in}{5.490039in}}%
\pgfusepath{clip}%
\pgfsetbuttcap%
\pgfsetroundjoin%
\definecolor{currentfill}{rgb}{0.279574,0.170599,0.479997}%
\pgfsetfillcolor{currentfill}%
\pgfsetfillopacity{0.700000}%
\pgfsetlinewidth{0.000000pt}%
\definecolor{currentstroke}{rgb}{0.000000,0.000000,0.000000}%
\pgfsetstrokecolor{currentstroke}%
\pgfsetdash{}{0pt}%
\pgfpathmoveto{\pgfqpoint{6.048355in}{2.477638in}}%
\pgfpathlineto{\pgfqpoint{6.061913in}{2.476398in}}%
\pgfpathlineto{\pgfqpoint{6.075478in}{2.475182in}}%
\pgfpathlineto{\pgfqpoint{6.089053in}{2.473989in}}%
\pgfpathlineto{\pgfqpoint{6.082384in}{2.469576in}}%
\pgfpathlineto{\pgfqpoint{6.075711in}{2.465188in}}%
\pgfpathlineto{\pgfqpoint{6.069033in}{2.460819in}}%
\pgfpathlineto{\pgfqpoint{6.062350in}{2.456465in}}%
\pgfpathlineto{\pgfqpoint{6.048754in}{2.457504in}}%
\pgfpathlineto{\pgfqpoint{6.035166in}{2.458566in}}%
\pgfpathlineto{\pgfqpoint{6.021586in}{2.459650in}}%
\pgfpathlineto{\pgfqpoint{6.028285in}{2.464117in}}%
\pgfpathlineto{\pgfqpoint{6.034980in}{2.468600in}}%
\pgfpathlineto{\pgfqpoint{6.041670in}{2.473106in}}%
\pgfpathlineto{\pgfqpoint{6.048355in}{2.477638in}}%
\pgfpathclose%
\pgfusepath{fill}%
\end{pgfscope}%
\begin{pgfscope}%
\pgfpathrectangle{\pgfqpoint{1.254980in}{0.150000in}}{\pgfqpoint{5.490039in}{5.490039in}}%
\pgfusepath{clip}%
\pgfsetbuttcap%
\pgfsetroundjoin%
\definecolor{currentfill}{rgb}{0.282327,0.094955,0.417331}%
\pgfsetfillcolor{currentfill}%
\pgfsetfillopacity{0.700000}%
\pgfsetlinewidth{0.000000pt}%
\definecolor{currentstroke}{rgb}{0.000000,0.000000,0.000000}%
\pgfsetstrokecolor{currentstroke}%
\pgfsetdash{}{0pt}%
\pgfpathmoveto{\pgfqpoint{4.966528in}{2.336729in}}%
\pgfpathlineto{\pgfqpoint{4.979785in}{2.335039in}}%
\pgfpathlineto{\pgfqpoint{4.993049in}{2.333374in}}%
\pgfpathlineto{\pgfqpoint{5.006320in}{2.331733in}}%
\pgfpathlineto{\pgfqpoint{5.019599in}{2.330116in}}%
\pgfpathlineto{\pgfqpoint{5.012432in}{2.323346in}}%
\pgfpathlineto{\pgfqpoint{5.005258in}{2.316525in}}%
\pgfpathlineto{\pgfqpoint{4.998078in}{2.309651in}}%
\pgfpathlineto{\pgfqpoint{4.990891in}{2.302724in}}%
\pgfpathlineto{\pgfqpoint{4.977599in}{2.304316in}}%
\pgfpathlineto{\pgfqpoint{4.964314in}{2.305932in}}%
\pgfpathlineto{\pgfqpoint{4.951037in}{2.307573in}}%
\pgfpathlineto{\pgfqpoint{4.937767in}{2.309238in}}%
\pgfpathlineto{\pgfqpoint{4.944967in}{2.316186in}}%
\pgfpathlineto{\pgfqpoint{4.952160in}{2.323082in}}%
\pgfpathlineto{\pgfqpoint{4.959347in}{2.329930in}}%
\pgfpathlineto{\pgfqpoint{4.966528in}{2.336729in}}%
\pgfpathclose%
\pgfusepath{fill}%
\end{pgfscope}%
\begin{pgfscope}%
\pgfpathrectangle{\pgfqpoint{1.254980in}{0.150000in}}{\pgfqpoint{5.490039in}{5.490039in}}%
\pgfusepath{clip}%
\pgfsetbuttcap%
\pgfsetroundjoin%
\definecolor{currentfill}{rgb}{0.267004,0.004874,0.329415}%
\pgfsetfillcolor{currentfill}%
\pgfsetfillopacity{0.700000}%
\pgfsetlinewidth{0.000000pt}%
\definecolor{currentstroke}{rgb}{0.000000,0.000000,0.000000}%
\pgfsetstrokecolor{currentstroke}%
\pgfsetdash{}{0pt}%
\pgfpathmoveto{\pgfqpoint{3.534133in}{2.190233in}}%
\pgfpathlineto{\pgfqpoint{3.547044in}{2.185686in}}%
\pgfpathlineto{\pgfqpoint{3.559959in}{2.181171in}}%
\pgfpathlineto{\pgfqpoint{3.572880in}{2.176687in}}%
\pgfpathlineto{\pgfqpoint{3.585806in}{2.172235in}}%
\pgfpathlineto{\pgfqpoint{3.578090in}{2.165863in}}%
\pgfpathlineto{\pgfqpoint{3.570367in}{2.159557in}}%
\pgfpathlineto{\pgfqpoint{3.562637in}{2.153319in}}%
\pgfpathlineto{\pgfqpoint{3.554900in}{2.147153in}}%
\pgfpathlineto{\pgfqpoint{3.541959in}{2.151747in}}%
\pgfpathlineto{\pgfqpoint{3.529023in}{2.156373in}}%
\pgfpathlineto{\pgfqpoint{3.516092in}{2.161030in}}%
\pgfpathlineto{\pgfqpoint{3.503166in}{2.165719in}}%
\pgfpathlineto{\pgfqpoint{3.510918in}{2.171738in}}%
\pgfpathlineto{\pgfqpoint{3.518663in}{2.177832in}}%
\pgfpathlineto{\pgfqpoint{3.526402in}{2.183998in}}%
\pgfpathlineto{\pgfqpoint{3.534133in}{2.190233in}}%
\pgfpathclose%
\pgfusepath{fill}%
\end{pgfscope}%
\begin{pgfscope}%
\pgfpathrectangle{\pgfqpoint{1.254980in}{0.150000in}}{\pgfqpoint{5.490039in}{5.490039in}}%
\pgfusepath{clip}%
\pgfsetbuttcap%
\pgfsetroundjoin%
\definecolor{currentfill}{rgb}{0.272594,0.025563,0.353093}%
\pgfsetfillcolor{currentfill}%
\pgfsetfillopacity{0.700000}%
\pgfsetlinewidth{0.000000pt}%
\definecolor{currentstroke}{rgb}{0.000000,0.000000,0.000000}%
\pgfsetstrokecolor{currentstroke}%
\pgfsetdash{}{0pt}%
\pgfpathmoveto{\pgfqpoint{3.265646in}{2.224453in}}%
\pgfpathlineto{\pgfqpoint{3.278515in}{2.219073in}}%
\pgfpathlineto{\pgfqpoint{3.291388in}{2.213728in}}%
\pgfpathlineto{\pgfqpoint{3.304265in}{2.208418in}}%
\pgfpathlineto{\pgfqpoint{3.317147in}{2.203143in}}%
\pgfpathlineto{\pgfqpoint{3.309308in}{2.198044in}}%
\pgfpathlineto{\pgfqpoint{3.301460in}{2.193053in}}%
\pgfpathlineto{\pgfqpoint{3.293603in}{2.188173in}}%
\pgfpathlineto{\pgfqpoint{3.285738in}{2.183409in}}%
\pgfpathlineto{\pgfqpoint{3.272839in}{2.188852in}}%
\pgfpathlineto{\pgfqpoint{3.259943in}{2.194330in}}%
\pgfpathlineto{\pgfqpoint{3.247052in}{2.199843in}}%
\pgfpathlineto{\pgfqpoint{3.234165in}{2.205391in}}%
\pgfpathlineto{\pgfqpoint{3.242048in}{2.209983in}}%
\pgfpathlineto{\pgfqpoint{3.249923in}{2.214693in}}%
\pgfpathlineto{\pgfqpoint{3.257789in}{2.219518in}}%
\pgfpathlineto{\pgfqpoint{3.265646in}{2.224453in}}%
\pgfpathclose%
\pgfusepath{fill}%
\end{pgfscope}%
\begin{pgfscope}%
\pgfpathrectangle{\pgfqpoint{1.254980in}{0.150000in}}{\pgfqpoint{5.490039in}{5.490039in}}%
\pgfusepath{clip}%
\pgfsetbuttcap%
\pgfsetroundjoin%
\definecolor{currentfill}{rgb}{0.271305,0.019942,0.347269}%
\pgfsetfillcolor{currentfill}%
\pgfsetfillopacity{0.700000}%
\pgfsetlinewidth{0.000000pt}%
\definecolor{currentstroke}{rgb}{0.000000,0.000000,0.000000}%
\pgfsetstrokecolor{currentstroke}%
\pgfsetdash{}{0pt}%
\pgfpathmoveto{\pgfqpoint{4.235352in}{2.218031in}}%
\pgfpathlineto{\pgfqpoint{4.248415in}{2.215226in}}%
\pgfpathlineto{\pgfqpoint{4.261485in}{2.212448in}}%
\pgfpathlineto{\pgfqpoint{4.274561in}{2.209696in}}%
\pgfpathlineto{\pgfqpoint{4.287643in}{2.206971in}}%
\pgfpathlineto{\pgfqpoint{4.280193in}{2.199196in}}%
\pgfpathlineto{\pgfqpoint{4.272737in}{2.191401in}}%
\pgfpathlineto{\pgfqpoint{4.265275in}{2.183588in}}%
\pgfpathlineto{\pgfqpoint{4.257808in}{2.175756in}}%
\pgfpathlineto{\pgfqpoint{4.244715in}{2.178547in}}%
\pgfpathlineto{\pgfqpoint{4.231627in}{2.181364in}}%
\pgfpathlineto{\pgfqpoint{4.218546in}{2.184208in}}%
\pgfpathlineto{\pgfqpoint{4.205472in}{2.187079in}}%
\pgfpathlineto{\pgfqpoint{4.212950in}{2.194840in}}%
\pgfpathlineto{\pgfqpoint{4.220423in}{2.202587in}}%
\pgfpathlineto{\pgfqpoint{4.227890in}{2.210317in}}%
\pgfpathlineto{\pgfqpoint{4.235352in}{2.218031in}}%
\pgfpathclose%
\pgfusepath{fill}%
\end{pgfscope}%
\begin{pgfscope}%
\pgfpathrectangle{\pgfqpoint{1.254980in}{0.150000in}}{\pgfqpoint{5.490039in}{5.490039in}}%
\pgfusepath{clip}%
\pgfsetbuttcap%
\pgfsetroundjoin%
\definecolor{currentfill}{rgb}{0.283197,0.115680,0.436115}%
\pgfsetfillcolor{currentfill}%
\pgfsetfillopacity{0.700000}%
\pgfsetlinewidth{0.000000pt}%
\definecolor{currentstroke}{rgb}{0.000000,0.000000,0.000000}%
\pgfsetstrokecolor{currentstroke}%
\pgfsetdash{}{0pt}%
\pgfpathmoveto{\pgfqpoint{5.182996in}{2.369838in}}%
\pgfpathlineto{\pgfqpoint{5.196315in}{2.368350in}}%
\pgfpathlineto{\pgfqpoint{5.209642in}{2.366886in}}%
\pgfpathlineto{\pgfqpoint{5.222977in}{2.365446in}}%
\pgfpathlineto{\pgfqpoint{5.236319in}{2.364030in}}%
\pgfpathlineto{\pgfqpoint{5.229245in}{2.357781in}}%
\pgfpathlineto{\pgfqpoint{5.222164in}{2.351485in}}%
\pgfpathlineto{\pgfqpoint{5.215076in}{2.345139in}}%
\pgfpathlineto{\pgfqpoint{5.207982in}{2.338741in}}%
\pgfpathlineto{\pgfqpoint{5.194626in}{2.340106in}}%
\pgfpathlineto{\pgfqpoint{5.181277in}{2.341496in}}%
\pgfpathlineto{\pgfqpoint{5.167935in}{2.342909in}}%
\pgfpathlineto{\pgfqpoint{5.154602in}{2.344346in}}%
\pgfpathlineto{\pgfqpoint{5.161710in}{2.350790in}}%
\pgfpathlineto{\pgfqpoint{5.168812in}{2.357185in}}%
\pgfpathlineto{\pgfqpoint{5.175907in}{2.363534in}}%
\pgfpathlineto{\pgfqpoint{5.182996in}{2.369838in}}%
\pgfpathclose%
\pgfusepath{fill}%
\end{pgfscope}%
\begin{pgfscope}%
\pgfpathrectangle{\pgfqpoint{1.254980in}{0.150000in}}{\pgfqpoint{5.490039in}{5.490039in}}%
\pgfusepath{clip}%
\pgfsetbuttcap%
\pgfsetroundjoin%
\definecolor{currentfill}{rgb}{0.267004,0.004874,0.329415}%
\pgfsetfillcolor{currentfill}%
\pgfsetfillopacity{0.700000}%
\pgfsetlinewidth{0.000000pt}%
\definecolor{currentstroke}{rgb}{0.000000,0.000000,0.000000}%
\pgfsetstrokecolor{currentstroke}%
\pgfsetdash{}{0pt}%
\pgfpathmoveto{\pgfqpoint{3.668303in}{2.181332in}}%
\pgfpathlineto{\pgfqpoint{3.681240in}{2.177163in}}%
\pgfpathlineto{\pgfqpoint{3.694183in}{2.173025in}}%
\pgfpathlineto{\pgfqpoint{3.707131in}{2.168917in}}%
\pgfpathlineto{\pgfqpoint{3.720084in}{2.164840in}}%
\pgfpathlineto{\pgfqpoint{3.712422in}{2.157989in}}%
\pgfpathlineto{\pgfqpoint{3.704754in}{2.151185in}}%
\pgfpathlineto{\pgfqpoint{3.697080in}{2.144430in}}%
\pgfpathlineto{\pgfqpoint{3.689399in}{2.137727in}}%
\pgfpathlineto{\pgfqpoint{3.676431in}{2.141935in}}%
\pgfpathlineto{\pgfqpoint{3.663469in}{2.146172in}}%
\pgfpathlineto{\pgfqpoint{3.650512in}{2.150439in}}%
\pgfpathlineto{\pgfqpoint{3.637561in}{2.154737in}}%
\pgfpathlineto{\pgfqpoint{3.645256in}{2.161305in}}%
\pgfpathlineto{\pgfqpoint{3.652944in}{2.167929in}}%
\pgfpathlineto{\pgfqpoint{3.660627in}{2.174606in}}%
\pgfpathlineto{\pgfqpoint{3.668303in}{2.181332in}}%
\pgfpathclose%
\pgfusepath{fill}%
\end{pgfscope}%
\begin{pgfscope}%
\pgfpathrectangle{\pgfqpoint{1.254980in}{0.150000in}}{\pgfqpoint{5.490039in}{5.490039in}}%
\pgfusepath{clip}%
\pgfsetbuttcap%
\pgfsetroundjoin%
\definecolor{currentfill}{rgb}{0.268510,0.009605,0.335427}%
\pgfsetfillcolor{currentfill}%
\pgfsetfillopacity{0.700000}%
\pgfsetlinewidth{0.000000pt}%
\definecolor{currentstroke}{rgb}{0.000000,0.000000,0.000000}%
\pgfsetstrokecolor{currentstroke}%
\pgfsetdash{}{0pt}%
\pgfpathmoveto{\pgfqpoint{4.018955in}{2.193417in}}%
\pgfpathlineto{\pgfqpoint{4.031967in}{2.190141in}}%
\pgfpathlineto{\pgfqpoint{4.044985in}{2.186893in}}%
\pgfpathlineto{\pgfqpoint{4.058010in}{2.183673in}}%
\pgfpathlineto{\pgfqpoint{4.071040in}{2.180480in}}%
\pgfpathlineto{\pgfqpoint{4.063511in}{2.172841in}}%
\pgfpathlineto{\pgfqpoint{4.055976in}{2.165204in}}%
\pgfpathlineto{\pgfqpoint{4.048436in}{2.157570in}}%
\pgfpathlineto{\pgfqpoint{4.040891in}{2.149942in}}%
\pgfpathlineto{\pgfqpoint{4.027848in}{2.153226in}}%
\pgfpathlineto{\pgfqpoint{4.014812in}{2.156538in}}%
\pgfpathlineto{\pgfqpoint{4.001782in}{2.159877in}}%
\pgfpathlineto{\pgfqpoint{3.988758in}{2.163244in}}%
\pgfpathlineto{\pgfqpoint{3.996315in}{2.170776in}}%
\pgfpathlineto{\pgfqpoint{4.003868in}{2.178317in}}%
\pgfpathlineto{\pgfqpoint{4.011414in}{2.185865in}}%
\pgfpathlineto{\pgfqpoint{4.018955in}{2.193417in}}%
\pgfpathclose%
\pgfusepath{fill}%
\end{pgfscope}%
\begin{pgfscope}%
\pgfpathrectangle{\pgfqpoint{1.254980in}{0.150000in}}{\pgfqpoint{5.490039in}{5.490039in}}%
\pgfusepath{clip}%
\pgfsetbuttcap%
\pgfsetroundjoin%
\definecolor{currentfill}{rgb}{0.276022,0.044167,0.370164}%
\pgfsetfillcolor{currentfill}%
\pgfsetfillopacity{0.700000}%
\pgfsetlinewidth{0.000000pt}%
\definecolor{currentstroke}{rgb}{0.000000,0.000000,0.000000}%
\pgfsetstrokecolor{currentstroke}%
\pgfsetdash{}{0pt}%
\pgfpathmoveto{\pgfqpoint{4.451763in}{2.248199in}}%
\pgfpathlineto{\pgfqpoint{4.464882in}{2.245802in}}%
\pgfpathlineto{\pgfqpoint{4.478007in}{2.243430in}}%
\pgfpathlineto{\pgfqpoint{4.491140in}{2.241084in}}%
\pgfpathlineto{\pgfqpoint{4.504279in}{2.238764in}}%
\pgfpathlineto{\pgfqpoint{4.496907in}{2.231081in}}%
\pgfpathlineto{\pgfqpoint{4.489529in}{2.223361in}}%
\pgfpathlineto{\pgfqpoint{4.482146in}{2.215604in}}%
\pgfpathlineto{\pgfqpoint{4.474757in}{2.207813in}}%
\pgfpathlineto{\pgfqpoint{4.461606in}{2.210173in}}%
\pgfpathlineto{\pgfqpoint{4.448462in}{2.212559in}}%
\pgfpathlineto{\pgfqpoint{4.435325in}{2.214970in}}%
\pgfpathlineto{\pgfqpoint{4.422195in}{2.217408in}}%
\pgfpathlineto{\pgfqpoint{4.429595in}{2.225155in}}%
\pgfpathlineto{\pgfqpoint{4.436990in}{2.232869in}}%
\pgfpathlineto{\pgfqpoint{4.444379in}{2.240551in}}%
\pgfpathlineto{\pgfqpoint{4.451763in}{2.248199in}}%
\pgfpathclose%
\pgfusepath{fill}%
\end{pgfscope}%
\begin{pgfscope}%
\pgfpathrectangle{\pgfqpoint{1.254980in}{0.150000in}}{\pgfqpoint{5.490039in}{5.490039in}}%
\pgfusepath{clip}%
\pgfsetbuttcap%
\pgfsetroundjoin%
\definecolor{currentfill}{rgb}{0.283072,0.130895,0.449241}%
\pgfsetfillcolor{currentfill}%
\pgfsetfillopacity{0.700000}%
\pgfsetlinewidth{0.000000pt}%
\definecolor{currentstroke}{rgb}{0.000000,0.000000,0.000000}%
\pgfsetstrokecolor{currentstroke}%
\pgfsetdash{}{0pt}%
\pgfpathmoveto{\pgfqpoint{5.399457in}{2.400965in}}%
\pgfpathlineto{\pgfqpoint{5.412838in}{2.399622in}}%
\pgfpathlineto{\pgfqpoint{5.426228in}{2.398303in}}%
\pgfpathlineto{\pgfqpoint{5.439625in}{2.397008in}}%
\pgfpathlineto{\pgfqpoint{5.453030in}{2.395736in}}%
\pgfpathlineto{\pgfqpoint{5.446054in}{2.390034in}}%
\pgfpathlineto{\pgfqpoint{5.439072in}{2.384293in}}%
\pgfpathlineto{\pgfqpoint{5.432083in}{2.378511in}}%
\pgfpathlineto{\pgfqpoint{5.425087in}{2.372685in}}%
\pgfpathlineto{\pgfqpoint{5.411666in}{2.373880in}}%
\pgfpathlineto{\pgfqpoint{5.398253in}{2.375099in}}%
\pgfpathlineto{\pgfqpoint{5.384847in}{2.376341in}}%
\pgfpathlineto{\pgfqpoint{5.371450in}{2.377607in}}%
\pgfpathlineto{\pgfqpoint{5.378461in}{2.383505in}}%
\pgfpathlineto{\pgfqpoint{5.385466in}{2.389362in}}%
\pgfpathlineto{\pgfqpoint{5.392465in}{2.395181in}}%
\pgfpathlineto{\pgfqpoint{5.399457in}{2.400965in}}%
\pgfpathclose%
\pgfusepath{fill}%
\end{pgfscope}%
\begin{pgfscope}%
\pgfpathrectangle{\pgfqpoint{1.254980in}{0.150000in}}{\pgfqpoint{5.490039in}{5.490039in}}%
\pgfusepath{clip}%
\pgfsetbuttcap%
\pgfsetroundjoin%
\definecolor{currentfill}{rgb}{0.280868,0.160771,0.472899}%
\pgfsetfillcolor{currentfill}%
\pgfsetfillopacity{0.700000}%
\pgfsetlinewidth{0.000000pt}%
\definecolor{currentstroke}{rgb}{0.000000,0.000000,0.000000}%
\pgfsetstrokecolor{currentstroke}%
\pgfsetdash{}{0pt}%
\pgfpathmoveto{\pgfqpoint{5.832175in}{2.454968in}}%
\pgfpathlineto{\pgfqpoint{5.845677in}{2.453749in}}%
\pgfpathlineto{\pgfqpoint{5.859186in}{2.452553in}}%
\pgfpathlineto{\pgfqpoint{5.872704in}{2.451380in}}%
\pgfpathlineto{\pgfqpoint{5.886230in}{2.450230in}}%
\pgfpathlineto{\pgfqpoint{5.879462in}{2.445513in}}%
\pgfpathlineto{\pgfqpoint{5.872688in}{2.440793in}}%
\pgfpathlineto{\pgfqpoint{5.865909in}{2.436065in}}%
\pgfpathlineto{\pgfqpoint{5.859124in}{2.431326in}}%
\pgfpathlineto{\pgfqpoint{5.845578in}{2.432347in}}%
\pgfpathlineto{\pgfqpoint{5.832040in}{2.433391in}}%
\pgfpathlineto{\pgfqpoint{5.818511in}{2.434458in}}%
\pgfpathlineto{\pgfqpoint{5.804990in}{2.435548in}}%
\pgfpathlineto{\pgfqpoint{5.811795in}{2.440411in}}%
\pgfpathlineto{\pgfqpoint{5.818594in}{2.445266in}}%
\pgfpathlineto{\pgfqpoint{5.825387in}{2.450117in}}%
\pgfpathlineto{\pgfqpoint{5.832175in}{2.454968in}}%
\pgfpathclose%
\pgfusepath{fill}%
\end{pgfscope}%
\begin{pgfscope}%
\pgfpathrectangle{\pgfqpoint{1.254980in}{0.150000in}}{\pgfqpoint{5.490039in}{5.490039in}}%
\pgfusepath{clip}%
\pgfsetbuttcap%
\pgfsetroundjoin%
\definecolor{currentfill}{rgb}{0.282290,0.145912,0.461510}%
\pgfsetfillcolor{currentfill}%
\pgfsetfillopacity{0.700000}%
\pgfsetlinewidth{0.000000pt}%
\definecolor{currentstroke}{rgb}{0.000000,0.000000,0.000000}%
\pgfsetstrokecolor{currentstroke}%
\pgfsetdash{}{0pt}%
\pgfpathmoveto{\pgfqpoint{5.615865in}{2.429439in}}%
\pgfpathlineto{\pgfqpoint{5.629307in}{2.428186in}}%
\pgfpathlineto{\pgfqpoint{5.642757in}{2.426956in}}%
\pgfpathlineto{\pgfqpoint{5.656216in}{2.425750in}}%
\pgfpathlineto{\pgfqpoint{5.669682in}{2.424566in}}%
\pgfpathlineto{\pgfqpoint{5.662809in}{2.419390in}}%
\pgfpathlineto{\pgfqpoint{5.655930in}{2.414190in}}%
\pgfpathlineto{\pgfqpoint{5.649044in}{2.408963in}}%
\pgfpathlineto{\pgfqpoint{5.642152in}{2.403706in}}%
\pgfpathlineto{\pgfqpoint{5.628668in}{2.404786in}}%
\pgfpathlineto{\pgfqpoint{5.615191in}{2.405889in}}%
\pgfpathlineto{\pgfqpoint{5.601723in}{2.407016in}}%
\pgfpathlineto{\pgfqpoint{5.588263in}{2.408167in}}%
\pgfpathlineto{\pgfqpoint{5.595173in}{2.413522in}}%
\pgfpathlineto{\pgfqpoint{5.602076in}{2.418851in}}%
\pgfpathlineto{\pgfqpoint{5.608973in}{2.424155in}}%
\pgfpathlineto{\pgfqpoint{5.615865in}{2.429439in}}%
\pgfpathclose%
\pgfusepath{fill}%
\end{pgfscope}%
\begin{pgfscope}%
\pgfpathrectangle{\pgfqpoint{1.254980in}{0.150000in}}{\pgfqpoint{5.490039in}{5.490039in}}%
\pgfusepath{clip}%
\pgfsetbuttcap%
\pgfsetroundjoin%
\definecolor{currentfill}{rgb}{0.276022,0.044167,0.370164}%
\pgfsetfillcolor{currentfill}%
\pgfsetfillopacity{0.700000}%
\pgfsetlinewidth{0.000000pt}%
\definecolor{currentstroke}{rgb}{0.000000,0.000000,0.000000}%
\pgfsetstrokecolor{currentstroke}%
\pgfsetdash{}{0pt}%
\pgfpathmoveto{\pgfqpoint{3.131218in}{2.251078in}}%
\pgfpathlineto{\pgfqpoint{3.144072in}{2.245238in}}%
\pgfpathlineto{\pgfqpoint{3.156930in}{2.239435in}}%
\pgfpathlineto{\pgfqpoint{3.169793in}{2.233670in}}%
\pgfpathlineto{\pgfqpoint{3.182659in}{2.227942in}}%
\pgfpathlineto{\pgfqpoint{3.174748in}{2.223650in}}%
\pgfpathlineto{\pgfqpoint{3.166828in}{2.219489in}}%
\pgfpathlineto{\pgfqpoint{3.158898in}{2.215463in}}%
\pgfpathlineto{\pgfqpoint{3.150959in}{2.211576in}}%
\pgfpathlineto{\pgfqpoint{3.138073in}{2.217486in}}%
\pgfpathlineto{\pgfqpoint{3.125191in}{2.223432in}}%
\pgfpathlineto{\pgfqpoint{3.112313in}{2.229416in}}%
\pgfpathlineto{\pgfqpoint{3.099439in}{2.235437in}}%
\pgfpathlineto{\pgfqpoint{3.107398in}{2.239137in}}%
\pgfpathlineto{\pgfqpoint{3.115347in}{2.242980in}}%
\pgfpathlineto{\pgfqpoint{3.123287in}{2.246962in}}%
\pgfpathlineto{\pgfqpoint{3.131218in}{2.251078in}}%
\pgfpathclose%
\pgfusepath{fill}%
\end{pgfscope}%
\begin{pgfscope}%
\pgfpathrectangle{\pgfqpoint{1.254980in}{0.150000in}}{\pgfqpoint{5.490039in}{5.490039in}}%
\pgfusepath{clip}%
\pgfsetbuttcap%
\pgfsetroundjoin%
\definecolor{currentfill}{rgb}{0.279566,0.067836,0.391917}%
\pgfsetfillcolor{currentfill}%
\pgfsetfillopacity{0.700000}%
\pgfsetlinewidth{0.000000pt}%
\definecolor{currentstroke}{rgb}{0.000000,0.000000,0.000000}%
\pgfsetstrokecolor{currentstroke}%
\pgfsetdash{}{0pt}%
\pgfpathmoveto{\pgfqpoint{4.668229in}{2.281563in}}%
\pgfpathlineto{\pgfqpoint{4.681408in}{2.279513in}}%
\pgfpathlineto{\pgfqpoint{4.694593in}{2.277487in}}%
\pgfpathlineto{\pgfqpoint{4.707786in}{2.275487in}}%
\pgfpathlineto{\pgfqpoint{4.720986in}{2.273511in}}%
\pgfpathlineto{\pgfqpoint{4.713695in}{2.266101in}}%
\pgfpathlineto{\pgfqpoint{4.706398in}{2.258644in}}%
\pgfpathlineto{\pgfqpoint{4.699095in}{2.251139in}}%
\pgfpathlineto{\pgfqpoint{4.691786in}{2.243586in}}%
\pgfpathlineto{\pgfqpoint{4.678574in}{2.245576in}}%
\pgfpathlineto{\pgfqpoint{4.665370in}{2.247591in}}%
\pgfpathlineto{\pgfqpoint{4.652173in}{2.249630in}}%
\pgfpathlineto{\pgfqpoint{4.638982in}{2.251695in}}%
\pgfpathlineto{\pgfqpoint{4.646303in}{2.259229in}}%
\pgfpathlineto{\pgfqpoint{4.653617in}{2.266718in}}%
\pgfpathlineto{\pgfqpoint{4.660926in}{2.274163in}}%
\pgfpathlineto{\pgfqpoint{4.668229in}{2.281563in}}%
\pgfpathclose%
\pgfusepath{fill}%
\end{pgfscope}%
\begin{pgfscope}%
\pgfpathrectangle{\pgfqpoint{1.254980in}{0.150000in}}{\pgfqpoint{5.490039in}{5.490039in}}%
\pgfusepath{clip}%
\pgfsetbuttcap%
\pgfsetroundjoin%
\definecolor{currentfill}{rgb}{0.267004,0.004874,0.329415}%
\pgfsetfillcolor{currentfill}%
\pgfsetfillopacity{0.700000}%
\pgfsetlinewidth{0.000000pt}%
\definecolor{currentstroke}{rgb}{0.000000,0.000000,0.000000}%
\pgfsetstrokecolor{currentstroke}%
\pgfsetdash{}{0pt}%
\pgfpathmoveto{\pgfqpoint{3.802486in}{2.177103in}}%
\pgfpathlineto{\pgfqpoint{3.815453in}{2.173290in}}%
\pgfpathlineto{\pgfqpoint{3.828426in}{2.169506in}}%
\pgfpathlineto{\pgfqpoint{3.841405in}{2.165751in}}%
\pgfpathlineto{\pgfqpoint{3.854389in}{2.162025in}}%
\pgfpathlineto{\pgfqpoint{3.846778in}{2.154793in}}%
\pgfpathlineto{\pgfqpoint{3.839161in}{2.147590in}}%
\pgfpathlineto{\pgfqpoint{3.831538in}{2.140419in}}%
\pgfpathlineto{\pgfqpoint{3.823909in}{2.133282in}}%
\pgfpathlineto{\pgfqpoint{3.810911in}{2.137124in}}%
\pgfpathlineto{\pgfqpoint{3.797919in}{2.140996in}}%
\pgfpathlineto{\pgfqpoint{3.784933in}{2.144896in}}%
\pgfpathlineto{\pgfqpoint{3.771952in}{2.148826in}}%
\pgfpathlineto{\pgfqpoint{3.779595in}{2.155842in}}%
\pgfpathlineto{\pgfqpoint{3.787231in}{2.162895in}}%
\pgfpathlineto{\pgfqpoint{3.794861in}{2.169983in}}%
\pgfpathlineto{\pgfqpoint{3.802486in}{2.177103in}}%
\pgfpathclose%
\pgfusepath{fill}%
\end{pgfscope}%
\begin{pgfscope}%
\pgfpathrectangle{\pgfqpoint{1.254980in}{0.150000in}}{\pgfqpoint{5.490039in}{5.490039in}}%
\pgfusepath{clip}%
\pgfsetbuttcap%
\pgfsetroundjoin%
\definecolor{currentfill}{rgb}{0.281924,0.089666,0.412415}%
\pgfsetfillcolor{currentfill}%
\pgfsetfillopacity{0.700000}%
\pgfsetlinewidth{0.000000pt}%
\definecolor{currentstroke}{rgb}{0.000000,0.000000,0.000000}%
\pgfsetstrokecolor{currentstroke}%
\pgfsetdash{}{0pt}%
\pgfpathmoveto{\pgfqpoint{4.884762in}{2.316144in}}%
\pgfpathlineto{\pgfqpoint{4.898002in}{2.314381in}}%
\pgfpathlineto{\pgfqpoint{4.911250in}{2.312642in}}%
\pgfpathlineto{\pgfqpoint{4.924505in}{2.310928in}}%
\pgfpathlineto{\pgfqpoint{4.937767in}{2.309238in}}%
\pgfpathlineto{\pgfqpoint{4.930562in}{2.302240in}}%
\pgfpathlineto{\pgfqpoint{4.923350in}{2.295189in}}%
\pgfpathlineto{\pgfqpoint{4.916132in}{2.288084in}}%
\pgfpathlineto{\pgfqpoint{4.908907in}{2.280926in}}%
\pgfpathlineto{\pgfqpoint{4.895632in}{2.282603in}}%
\pgfpathlineto{\pgfqpoint{4.882365in}{2.284306in}}%
\pgfpathlineto{\pgfqpoint{4.869105in}{2.286033in}}%
\pgfpathlineto{\pgfqpoint{4.855852in}{2.287784in}}%
\pgfpathlineto{\pgfqpoint{4.863088in}{2.294950in}}%
\pgfpathlineto{\pgfqpoint{4.870319in}{2.302064in}}%
\pgfpathlineto{\pgfqpoint{4.877544in}{2.309129in}}%
\pgfpathlineto{\pgfqpoint{4.884762in}{2.316144in}}%
\pgfpathclose%
\pgfusepath{fill}%
\end{pgfscope}%
\begin{pgfscope}%
\pgfpathrectangle{\pgfqpoint{1.254980in}{0.150000in}}{\pgfqpoint{5.490039in}{5.490039in}}%
\pgfusepath{clip}%
\pgfsetbuttcap%
\pgfsetroundjoin%
\definecolor{currentfill}{rgb}{0.283091,0.110553,0.431554}%
\pgfsetfillcolor{currentfill}%
\pgfsetfillopacity{0.700000}%
\pgfsetlinewidth{0.000000pt}%
\definecolor{currentstroke}{rgb}{0.000000,0.000000,0.000000}%
\pgfsetstrokecolor{currentstroke}%
\pgfsetdash{}{0pt}%
\pgfpathmoveto{\pgfqpoint{2.810321in}{2.354844in}}%
\pgfpathlineto{\pgfqpoint{2.823150in}{2.347808in}}%
\pgfpathlineto{\pgfqpoint{2.835981in}{2.340817in}}%
\pgfpathlineto{\pgfqpoint{2.848816in}{2.333870in}}%
\pgfpathlineto{\pgfqpoint{2.861654in}{2.326967in}}%
\pgfpathlineto{\pgfqpoint{2.853552in}{2.324865in}}%
\pgfpathlineto{\pgfqpoint{2.845438in}{2.322949in}}%
\pgfpathlineto{\pgfqpoint{2.837311in}{2.321224in}}%
\pgfpathlineto{\pgfqpoint{2.829172in}{2.319695in}}%
\pgfpathlineto{\pgfqpoint{2.816311in}{2.326806in}}%
\pgfpathlineto{\pgfqpoint{2.803452in}{2.333962in}}%
\pgfpathlineto{\pgfqpoint{2.790596in}{2.341161in}}%
\pgfpathlineto{\pgfqpoint{2.777742in}{2.348406in}}%
\pgfpathlineto{\pgfqpoint{2.785906in}{2.349722in}}%
\pgfpathlineto{\pgfqpoint{2.794057in}{2.351237in}}%
\pgfpathlineto{\pgfqpoint{2.802195in}{2.352946in}}%
\pgfpathlineto{\pgfqpoint{2.810321in}{2.354844in}}%
\pgfpathclose%
\pgfusepath{fill}%
\end{pgfscope}%
\begin{pgfscope}%
\pgfpathrectangle{\pgfqpoint{1.254980in}{0.150000in}}{\pgfqpoint{5.490039in}{5.490039in}}%
\pgfusepath{clip}%
\pgfsetbuttcap%
\pgfsetroundjoin%
\definecolor{currentfill}{rgb}{0.269944,0.014625,0.341379}%
\pgfsetfillcolor{currentfill}%
\pgfsetfillopacity{0.700000}%
\pgfsetlinewidth{0.000000pt}%
\definecolor{currentstroke}{rgb}{0.000000,0.000000,0.000000}%
\pgfsetstrokecolor{currentstroke}%
\pgfsetdash{}{0pt}%
\pgfpathmoveto{\pgfqpoint{4.153236in}{2.198830in}}%
\pgfpathlineto{\pgfqpoint{4.166286in}{2.195852in}}%
\pgfpathlineto{\pgfqpoint{4.179341in}{2.192901in}}%
\pgfpathlineto{\pgfqpoint{4.192403in}{2.189976in}}%
\pgfpathlineto{\pgfqpoint{4.205472in}{2.187079in}}%
\pgfpathlineto{\pgfqpoint{4.197988in}{2.179305in}}%
\pgfpathlineto{\pgfqpoint{4.190499in}{2.171518in}}%
\pgfpathlineto{\pgfqpoint{4.183004in}{2.163722in}}%
\pgfpathlineto{\pgfqpoint{4.175504in}{2.155917in}}%
\pgfpathlineto{\pgfqpoint{4.162424in}{2.158893in}}%
\pgfpathlineto{\pgfqpoint{4.149350in}{2.161896in}}%
\pgfpathlineto{\pgfqpoint{4.136283in}{2.164925in}}%
\pgfpathlineto{\pgfqpoint{4.123222in}{2.167982in}}%
\pgfpathlineto{\pgfqpoint{4.130734in}{2.175703in}}%
\pgfpathlineto{\pgfqpoint{4.138240in}{2.183420in}}%
\pgfpathlineto{\pgfqpoint{4.145741in}{2.191129in}}%
\pgfpathlineto{\pgfqpoint{4.153236in}{2.198830in}}%
\pgfpathclose%
\pgfusepath{fill}%
\end{pgfscope}%
\begin{pgfscope}%
\pgfpathrectangle{\pgfqpoint{1.254980in}{0.150000in}}{\pgfqpoint{5.490039in}{5.490039in}}%
\pgfusepath{clip}%
\pgfsetbuttcap%
\pgfsetroundjoin%
\definecolor{currentfill}{rgb}{0.283091,0.110553,0.431554}%
\pgfsetfillcolor{currentfill}%
\pgfsetfillopacity{0.700000}%
\pgfsetlinewidth{0.000000pt}%
\definecolor{currentstroke}{rgb}{0.000000,0.000000,0.000000}%
\pgfsetstrokecolor{currentstroke}%
\pgfsetdash{}{0pt}%
\pgfpathmoveto{\pgfqpoint{5.101344in}{2.350335in}}%
\pgfpathlineto{\pgfqpoint{5.114647in}{2.348802in}}%
\pgfpathlineto{\pgfqpoint{5.127957in}{2.347293in}}%
\pgfpathlineto{\pgfqpoint{5.141276in}{2.345807in}}%
\pgfpathlineto{\pgfqpoint{5.154602in}{2.344346in}}%
\pgfpathlineto{\pgfqpoint{5.147487in}{2.337852in}}%
\pgfpathlineto{\pgfqpoint{5.140366in}{2.331306in}}%
\pgfpathlineto{\pgfqpoint{5.133238in}{2.324707in}}%
\pgfpathlineto{\pgfqpoint{5.126104in}{2.318053in}}%
\pgfpathlineto{\pgfqpoint{5.112764in}{2.319476in}}%
\pgfpathlineto{\pgfqpoint{5.099432in}{2.320924in}}%
\pgfpathlineto{\pgfqpoint{5.086108in}{2.322395in}}%
\pgfpathlineto{\pgfqpoint{5.072791in}{2.323891in}}%
\pgfpathlineto{\pgfqpoint{5.079939in}{2.330578in}}%
\pgfpathlineto{\pgfqpoint{5.087080in}{2.337214in}}%
\pgfpathlineto{\pgfqpoint{5.094215in}{2.343799in}}%
\pgfpathlineto{\pgfqpoint{5.101344in}{2.350335in}}%
\pgfpathclose%
\pgfusepath{fill}%
\end{pgfscope}%
\begin{pgfscope}%
\pgfpathrectangle{\pgfqpoint{1.254980in}{0.150000in}}{\pgfqpoint{5.490039in}{5.490039in}}%
\pgfusepath{clip}%
\pgfsetbuttcap%
\pgfsetroundjoin%
\definecolor{currentfill}{rgb}{0.273809,0.031497,0.358853}%
\pgfsetfillcolor{currentfill}%
\pgfsetfillopacity{0.700000}%
\pgfsetlinewidth{0.000000pt}%
\definecolor{currentstroke}{rgb}{0.000000,0.000000,0.000000}%
\pgfsetstrokecolor{currentstroke}%
\pgfsetdash{}{0pt}%
\pgfpathmoveto{\pgfqpoint{4.369740in}{2.227416in}}%
\pgfpathlineto{\pgfqpoint{4.382844in}{2.224875in}}%
\pgfpathlineto{\pgfqpoint{4.395954in}{2.222360in}}%
\pgfpathlineto{\pgfqpoint{4.409071in}{2.219871in}}%
\pgfpathlineto{\pgfqpoint{4.422195in}{2.217408in}}%
\pgfpathlineto{\pgfqpoint{4.414789in}{2.209629in}}%
\pgfpathlineto{\pgfqpoint{4.407378in}{2.201820in}}%
\pgfpathlineto{\pgfqpoint{4.399961in}{2.193981in}}%
\pgfpathlineto{\pgfqpoint{4.392538in}{2.186112in}}%
\pgfpathlineto{\pgfqpoint{4.379403in}{2.188629in}}%
\pgfpathlineto{\pgfqpoint{4.366275in}{2.191171in}}%
\pgfpathlineto{\pgfqpoint{4.353153in}{2.193739in}}%
\pgfpathlineto{\pgfqpoint{4.340038in}{2.196333in}}%
\pgfpathlineto{\pgfqpoint{4.347472in}{2.204143in}}%
\pgfpathlineto{\pgfqpoint{4.354900in}{2.211928in}}%
\pgfpathlineto{\pgfqpoint{4.362323in}{2.219686in}}%
\pgfpathlineto{\pgfqpoint{4.369740in}{2.227416in}}%
\pgfpathclose%
\pgfusepath{fill}%
\end{pgfscope}%
\begin{pgfscope}%
\pgfpathrectangle{\pgfqpoint{1.254980in}{0.150000in}}{\pgfqpoint{5.490039in}{5.490039in}}%
\pgfusepath{clip}%
\pgfsetbuttcap%
\pgfsetroundjoin%
\definecolor{currentfill}{rgb}{0.280267,0.073417,0.397163}%
\pgfsetfillcolor{currentfill}%
\pgfsetfillopacity{0.700000}%
\pgfsetlinewidth{0.000000pt}%
\definecolor{currentstroke}{rgb}{0.000000,0.000000,0.000000}%
\pgfsetstrokecolor{currentstroke}%
\pgfsetdash{}{0pt}%
\pgfpathmoveto{\pgfqpoint{2.996579in}{2.284991in}}%
\pgfpathlineto{\pgfqpoint{3.009424in}{2.278659in}}%
\pgfpathlineto{\pgfqpoint{3.022272in}{2.272367in}}%
\pgfpathlineto{\pgfqpoint{3.035124in}{2.266115in}}%
\pgfpathlineto{\pgfqpoint{3.047980in}{2.259902in}}%
\pgfpathlineto{\pgfqpoint{3.039989in}{2.256539in}}%
\pgfpathlineto{\pgfqpoint{3.031989in}{2.253331in}}%
\pgfpathlineto{\pgfqpoint{3.023978in}{2.250285in}}%
\pgfpathlineto{\pgfqpoint{3.015956in}{2.247403in}}%
\pgfpathlineto{\pgfqpoint{3.003079in}{2.253811in}}%
\pgfpathlineto{\pgfqpoint{2.990205in}{2.260257in}}%
\pgfpathlineto{\pgfqpoint{2.977335in}{2.266744in}}%
\pgfpathlineto{\pgfqpoint{2.964468in}{2.273270in}}%
\pgfpathlineto{\pgfqpoint{2.972512in}{2.275952in}}%
\pgfpathlineto{\pgfqpoint{2.980545in}{2.278803in}}%
\pgfpathlineto{\pgfqpoint{2.988568in}{2.281817in}}%
\pgfpathlineto{\pgfqpoint{2.996579in}{2.284991in}}%
\pgfpathclose%
\pgfusepath{fill}%
\end{pgfscope}%
\begin{pgfscope}%
\pgfpathrectangle{\pgfqpoint{1.254980in}{0.150000in}}{\pgfqpoint{5.490039in}{5.490039in}}%
\pgfusepath{clip}%
\pgfsetbuttcap%
\pgfsetroundjoin%
\definecolor{currentfill}{rgb}{0.267004,0.004874,0.329415}%
\pgfsetfillcolor{currentfill}%
\pgfsetfillopacity{0.700000}%
\pgfsetlinewidth{0.000000pt}%
\definecolor{currentstroke}{rgb}{0.000000,0.000000,0.000000}%
\pgfsetstrokecolor{currentstroke}%
\pgfsetdash{}{0pt}%
\pgfpathmoveto{\pgfqpoint{3.936720in}{2.176992in}}%
\pgfpathlineto{\pgfqpoint{3.949721in}{2.173513in}}%
\pgfpathlineto{\pgfqpoint{3.962727in}{2.170062in}}%
\pgfpathlineto{\pgfqpoint{3.975739in}{2.166639in}}%
\pgfpathlineto{\pgfqpoint{3.988758in}{2.163244in}}%
\pgfpathlineto{\pgfqpoint{3.981194in}{2.155722in}}%
\pgfpathlineto{\pgfqpoint{3.973626in}{2.148214in}}%
\pgfpathlineto{\pgfqpoint{3.966051in}{2.140720in}}%
\pgfpathlineto{\pgfqpoint{3.958471in}{2.133245in}}%
\pgfpathlineto{\pgfqpoint{3.945440in}{2.136743in}}%
\pgfpathlineto{\pgfqpoint{3.932416in}{2.140270in}}%
\pgfpathlineto{\pgfqpoint{3.919397in}{2.143825in}}%
\pgfpathlineto{\pgfqpoint{3.906384in}{2.147408in}}%
\pgfpathlineto{\pgfqpoint{3.913976in}{2.154775in}}%
\pgfpathlineto{\pgfqpoint{3.921563in}{2.162163in}}%
\pgfpathlineto{\pgfqpoint{3.929145in}{2.169569in}}%
\pgfpathlineto{\pgfqpoint{3.936720in}{2.176992in}}%
\pgfpathclose%
\pgfusepath{fill}%
\end{pgfscope}%
\begin{pgfscope}%
\pgfpathrectangle{\pgfqpoint{1.254980in}{0.150000in}}{\pgfqpoint{5.490039in}{5.490039in}}%
\pgfusepath{clip}%
\pgfsetbuttcap%
\pgfsetroundjoin%
\definecolor{currentfill}{rgb}{0.283187,0.125848,0.444960}%
\pgfsetfillcolor{currentfill}%
\pgfsetfillopacity{0.700000}%
\pgfsetlinewidth{0.000000pt}%
\definecolor{currentstroke}{rgb}{0.000000,0.000000,0.000000}%
\pgfsetstrokecolor{currentstroke}%
\pgfsetdash{}{0pt}%
\pgfpathmoveto{\pgfqpoint{5.317939in}{2.382907in}}%
\pgfpathlineto{\pgfqpoint{5.331305in}{2.381547in}}%
\pgfpathlineto{\pgfqpoint{5.344679in}{2.380210in}}%
\pgfpathlineto{\pgfqpoint{5.358060in}{2.378896in}}%
\pgfpathlineto{\pgfqpoint{5.371450in}{2.377607in}}%
\pgfpathlineto{\pgfqpoint{5.364432in}{2.371666in}}%
\pgfpathlineto{\pgfqpoint{5.357407in}{2.365679in}}%
\pgfpathlineto{\pgfqpoint{5.350376in}{2.359644in}}%
\pgfpathlineto{\pgfqpoint{5.343338in}{2.353559in}}%
\pgfpathlineto{\pgfqpoint{5.329933in}{2.354785in}}%
\pgfpathlineto{\pgfqpoint{5.316536in}{2.356034in}}%
\pgfpathlineto{\pgfqpoint{5.303147in}{2.357307in}}%
\pgfpathlineto{\pgfqpoint{5.289766in}{2.358604in}}%
\pgfpathlineto{\pgfqpoint{5.296819in}{2.364748in}}%
\pgfpathlineto{\pgfqpoint{5.303865in}{2.370845in}}%
\pgfpathlineto{\pgfqpoint{5.310905in}{2.376897in}}%
\pgfpathlineto{\pgfqpoint{5.317939in}{2.382907in}}%
\pgfpathclose%
\pgfusepath{fill}%
\end{pgfscope}%
\begin{pgfscope}%
\pgfpathrectangle{\pgfqpoint{1.254980in}{0.150000in}}{\pgfqpoint{5.490039in}{5.490039in}}%
\pgfusepath{clip}%
\pgfsetbuttcap%
\pgfsetroundjoin%
\definecolor{currentfill}{rgb}{0.277941,0.056324,0.381191}%
\pgfsetfillcolor{currentfill}%
\pgfsetfillopacity{0.700000}%
\pgfsetlinewidth{0.000000pt}%
\definecolor{currentstroke}{rgb}{0.000000,0.000000,0.000000}%
\pgfsetstrokecolor{currentstroke}%
\pgfsetdash{}{0pt}%
\pgfpathmoveto{\pgfqpoint{4.586291in}{2.260207in}}%
\pgfpathlineto{\pgfqpoint{4.599454in}{2.258041in}}%
\pgfpathlineto{\pgfqpoint{4.612623in}{2.255901in}}%
\pgfpathlineto{\pgfqpoint{4.625799in}{2.253785in}}%
\pgfpathlineto{\pgfqpoint{4.638982in}{2.251695in}}%
\pgfpathlineto{\pgfqpoint{4.631656in}{2.244117in}}%
\pgfpathlineto{\pgfqpoint{4.624324in}{2.236495in}}%
\pgfpathlineto{\pgfqpoint{4.616986in}{2.228829in}}%
\pgfpathlineto{\pgfqpoint{4.609643in}{2.221118in}}%
\pgfpathlineto{\pgfqpoint{4.596448in}{2.223235in}}%
\pgfpathlineto{\pgfqpoint{4.583260in}{2.225378in}}%
\pgfpathlineto{\pgfqpoint{4.570079in}{2.227545in}}%
\pgfpathlineto{\pgfqpoint{4.556905in}{2.229738in}}%
\pgfpathlineto{\pgfqpoint{4.564260in}{2.237417in}}%
\pgfpathlineto{\pgfqpoint{4.571610in}{2.245054in}}%
\pgfpathlineto{\pgfqpoint{4.578953in}{2.252651in}}%
\pgfpathlineto{\pgfqpoint{4.586291in}{2.260207in}}%
\pgfpathclose%
\pgfusepath{fill}%
\end{pgfscope}%
\begin{pgfscope}%
\pgfpathrectangle{\pgfqpoint{1.254980in}{0.150000in}}{\pgfqpoint{5.490039in}{5.490039in}}%
\pgfusepath{clip}%
\pgfsetbuttcap%
\pgfsetroundjoin%
\definecolor{currentfill}{rgb}{0.281412,0.155834,0.469201}%
\pgfsetfillcolor{currentfill}%
\pgfsetfillopacity{0.700000}%
\pgfsetlinewidth{0.000000pt}%
\definecolor{currentstroke}{rgb}{0.000000,0.000000,0.000000}%
\pgfsetstrokecolor{currentstroke}%
\pgfsetdash{}{0pt}%
\pgfpathmoveto{\pgfqpoint{2.623702in}{2.438992in}}%
\pgfpathlineto{\pgfqpoint{2.636525in}{2.431175in}}%
\pgfpathlineto{\pgfqpoint{2.649351in}{2.423408in}}%
\pgfpathlineto{\pgfqpoint{2.662180in}{2.415692in}}%
\pgfpathlineto{\pgfqpoint{2.675010in}{2.408025in}}%
\pgfpathlineto{\pgfqpoint{2.666781in}{2.407349in}}%
\pgfpathlineto{\pgfqpoint{2.658537in}{2.406890in}}%
\pgfpathlineto{\pgfqpoint{2.650279in}{2.406654in}}%
\pgfpathlineto{\pgfqpoint{2.642007in}{2.406648in}}%
\pgfpathlineto{\pgfqpoint{2.629149in}{2.414538in}}%
\pgfpathlineto{\pgfqpoint{2.616294in}{2.422477in}}%
\pgfpathlineto{\pgfqpoint{2.603440in}{2.430467in}}%
\pgfpathlineto{\pgfqpoint{2.590589in}{2.438507in}}%
\pgfpathlineto{\pgfqpoint{2.598890in}{2.438286in}}%
\pgfpathlineto{\pgfqpoint{2.607175in}{2.438297in}}%
\pgfpathlineto{\pgfqpoint{2.615446in}{2.438534in}}%
\pgfpathlineto{\pgfqpoint{2.623702in}{2.438992in}}%
\pgfpathclose%
\pgfusepath{fill}%
\end{pgfscope}%
\begin{pgfscope}%
\pgfpathrectangle{\pgfqpoint{1.254980in}{0.150000in}}{\pgfqpoint{5.490039in}{5.490039in}}%
\pgfusepath{clip}%
\pgfsetbuttcap%
\pgfsetroundjoin%
\definecolor{currentfill}{rgb}{0.279574,0.170599,0.479997}%
\pgfsetfillcolor{currentfill}%
\pgfsetfillopacity{0.700000}%
\pgfsetlinewidth{0.000000pt}%
\definecolor{currentstroke}{rgb}{0.000000,0.000000,0.000000}%
\pgfsetstrokecolor{currentstroke}%
\pgfsetdash{}{0pt}%
\pgfpathmoveto{\pgfqpoint{5.967351in}{2.464218in}}%
\pgfpathlineto{\pgfqpoint{5.980897in}{2.463042in}}%
\pgfpathlineto{\pgfqpoint{5.994452in}{2.461888in}}%
\pgfpathlineto{\pgfqpoint{6.008015in}{2.460758in}}%
\pgfpathlineto{\pgfqpoint{6.021586in}{2.459650in}}%
\pgfpathlineto{\pgfqpoint{6.014882in}{2.455196in}}%
\pgfpathlineto{\pgfqpoint{6.008172in}{2.450750in}}%
\pgfpathlineto{\pgfqpoint{6.001457in}{2.446307in}}%
\pgfpathlineto{\pgfqpoint{5.994736in}{2.441863in}}%
\pgfpathlineto{\pgfqpoint{5.981143in}{2.442828in}}%
\pgfpathlineto{\pgfqpoint{5.967559in}{2.443816in}}%
\pgfpathlineto{\pgfqpoint{5.953984in}{2.444828in}}%
\pgfpathlineto{\pgfqpoint{5.940416in}{2.445862in}}%
\pgfpathlineto{\pgfqpoint{5.947158in}{2.450443in}}%
\pgfpathlineto{\pgfqpoint{5.953894in}{2.455027in}}%
\pgfpathlineto{\pgfqpoint{5.960625in}{2.459617in}}%
\pgfpathlineto{\pgfqpoint{5.967351in}{2.464218in}}%
\pgfpathclose%
\pgfusepath{fill}%
\end{pgfscope}%
\begin{pgfscope}%
\pgfpathrectangle{\pgfqpoint{1.254980in}{0.150000in}}{\pgfqpoint{5.490039in}{5.490039in}}%
\pgfusepath{clip}%
\pgfsetbuttcap%
\pgfsetroundjoin%
\definecolor{currentfill}{rgb}{0.268510,0.009605,0.335427}%
\pgfsetfillcolor{currentfill}%
\pgfsetfillopacity{0.700000}%
\pgfsetlinewidth{0.000000pt}%
\definecolor{currentstroke}{rgb}{0.000000,0.000000,0.000000}%
\pgfsetstrokecolor{currentstroke}%
\pgfsetdash{}{0pt}%
\pgfpathmoveto{\pgfqpoint{3.451510in}{2.184796in}}%
\pgfpathlineto{\pgfqpoint{3.464417in}{2.179978in}}%
\pgfpathlineto{\pgfqpoint{3.477329in}{2.175193in}}%
\pgfpathlineto{\pgfqpoint{3.490245in}{2.170440in}}%
\pgfpathlineto{\pgfqpoint{3.503166in}{2.165719in}}%
\pgfpathlineto{\pgfqpoint{3.495407in}{2.159779in}}%
\pgfpathlineto{\pgfqpoint{3.487640in}{2.153921in}}%
\pgfpathlineto{\pgfqpoint{3.479866in}{2.148149in}}%
\pgfpathlineto{\pgfqpoint{3.472085in}{2.142467in}}%
\pgfpathlineto{\pgfqpoint{3.459148in}{2.147343in}}%
\pgfpathlineto{\pgfqpoint{3.446215in}{2.152251in}}%
\pgfpathlineto{\pgfqpoint{3.433288in}{2.157191in}}%
\pgfpathlineto{\pgfqpoint{3.420365in}{2.162164in}}%
\pgfpathlineto{\pgfqpoint{3.428162in}{2.167686in}}%
\pgfpathlineto{\pgfqpoint{3.435953in}{2.173301in}}%
\pgfpathlineto{\pgfqpoint{3.443735in}{2.179006in}}%
\pgfpathlineto{\pgfqpoint{3.451510in}{2.184796in}}%
\pgfpathclose%
\pgfusepath{fill}%
\end{pgfscope}%
\begin{pgfscope}%
\pgfpathrectangle{\pgfqpoint{1.254980in}{0.150000in}}{\pgfqpoint{5.490039in}{5.490039in}}%
\pgfusepath{clip}%
\pgfsetbuttcap%
\pgfsetroundjoin%
\definecolor{currentfill}{rgb}{0.282290,0.145912,0.461510}%
\pgfsetfillcolor{currentfill}%
\pgfsetfillopacity{0.700000}%
\pgfsetlinewidth{0.000000pt}%
\definecolor{currentstroke}{rgb}{0.000000,0.000000,0.000000}%
\pgfsetstrokecolor{currentstroke}%
\pgfsetdash{}{0pt}%
\pgfpathmoveto{\pgfqpoint{5.534502in}{2.413002in}}%
\pgfpathlineto{\pgfqpoint{5.547931in}{2.411758in}}%
\pgfpathlineto{\pgfqpoint{5.561367in}{2.410538in}}%
\pgfpathlineto{\pgfqpoint{5.574811in}{2.409340in}}%
\pgfpathlineto{\pgfqpoint{5.588263in}{2.408167in}}%
\pgfpathlineto{\pgfqpoint{5.581346in}{2.402781in}}%
\pgfpathlineto{\pgfqpoint{5.574424in}{2.397361in}}%
\pgfpathlineto{\pgfqpoint{5.567494in}{2.391905in}}%
\pgfpathlineto{\pgfqpoint{5.560558in}{2.386409in}}%
\pgfpathlineto{\pgfqpoint{5.547089in}{2.387493in}}%
\pgfpathlineto{\pgfqpoint{5.533628in}{2.388600in}}%
\pgfpathlineto{\pgfqpoint{5.520175in}{2.389730in}}%
\pgfpathlineto{\pgfqpoint{5.506730in}{2.390884in}}%
\pgfpathlineto{\pgfqpoint{5.513683in}{2.396465in}}%
\pgfpathlineto{\pgfqpoint{5.520629in}{2.402010in}}%
\pgfpathlineto{\pgfqpoint{5.527569in}{2.407521in}}%
\pgfpathlineto{\pgfqpoint{5.534502in}{2.413002in}}%
\pgfpathclose%
\pgfusepath{fill}%
\end{pgfscope}%
\begin{pgfscope}%
\pgfpathrectangle{\pgfqpoint{1.254980in}{0.150000in}}{\pgfqpoint{5.490039in}{5.490039in}}%
\pgfusepath{clip}%
\pgfsetbuttcap%
\pgfsetroundjoin%
\definecolor{currentfill}{rgb}{0.280868,0.160771,0.472899}%
\pgfsetfillcolor{currentfill}%
\pgfsetfillopacity{0.700000}%
\pgfsetlinewidth{0.000000pt}%
\definecolor{currentstroke}{rgb}{0.000000,0.000000,0.000000}%
\pgfsetstrokecolor{currentstroke}%
\pgfsetdash{}{0pt}%
\pgfpathmoveto{\pgfqpoint{5.750987in}{2.440140in}}%
\pgfpathlineto{\pgfqpoint{5.764475in}{2.438957in}}%
\pgfpathlineto{\pgfqpoint{5.777972in}{2.437797in}}%
\pgfpathlineto{\pgfqpoint{5.791477in}{2.436661in}}%
\pgfpathlineto{\pgfqpoint{5.804990in}{2.435548in}}%
\pgfpathlineto{\pgfqpoint{5.798179in}{2.430673in}}%
\pgfpathlineto{\pgfqpoint{5.791362in}{2.425782in}}%
\pgfpathlineto{\pgfqpoint{5.784538in}{2.420872in}}%
\pgfpathlineto{\pgfqpoint{5.777709in}{2.415939in}}%
\pgfpathlineto{\pgfqpoint{5.764177in}{2.416936in}}%
\pgfpathlineto{\pgfqpoint{5.750653in}{2.417956in}}%
\pgfpathlineto{\pgfqpoint{5.737137in}{2.418999in}}%
\pgfpathlineto{\pgfqpoint{5.723630in}{2.420066in}}%
\pgfpathlineto{\pgfqpoint{5.730478in}{2.425111in}}%
\pgfpathlineto{\pgfqpoint{5.737321in}{2.430136in}}%
\pgfpathlineto{\pgfqpoint{5.744157in}{2.435144in}}%
\pgfpathlineto{\pgfqpoint{5.750987in}{2.440140in}}%
\pgfpathclose%
\pgfusepath{fill}%
\end{pgfscope}%
\begin{pgfscope}%
\pgfpathrectangle{\pgfqpoint{1.254980in}{0.150000in}}{\pgfqpoint{5.490039in}{5.490039in}}%
\pgfusepath{clip}%
\pgfsetbuttcap%
\pgfsetroundjoin%
\definecolor{currentfill}{rgb}{0.271305,0.019942,0.347269}%
\pgfsetfillcolor{currentfill}%
\pgfsetfillopacity{0.700000}%
\pgfsetlinewidth{0.000000pt}%
\definecolor{currentstroke}{rgb}{0.000000,0.000000,0.000000}%
\pgfsetstrokecolor{currentstroke}%
\pgfsetdash{}{0pt}%
\pgfpathmoveto{\pgfqpoint{3.317147in}{2.203143in}}%
\pgfpathlineto{\pgfqpoint{3.330034in}{2.197903in}}%
\pgfpathlineto{\pgfqpoint{3.342924in}{2.192696in}}%
\pgfpathlineto{\pgfqpoint{3.355820in}{2.187524in}}%
\pgfpathlineto{\pgfqpoint{3.368719in}{2.182385in}}%
\pgfpathlineto{\pgfqpoint{3.360897in}{2.177123in}}%
\pgfpathlineto{\pgfqpoint{3.353067in}{2.171965in}}%
\pgfpathlineto{\pgfqpoint{3.345228in}{2.166916in}}%
\pgfpathlineto{\pgfqpoint{3.337381in}{2.161979in}}%
\pgfpathlineto{\pgfqpoint{3.324464in}{2.167285in}}%
\pgfpathlineto{\pgfqpoint{3.311551in}{2.172626in}}%
\pgfpathlineto{\pgfqpoint{3.298643in}{2.178000in}}%
\pgfpathlineto{\pgfqpoint{3.285738in}{2.183409in}}%
\pgfpathlineto{\pgfqpoint{3.293603in}{2.188173in}}%
\pgfpathlineto{\pgfqpoint{3.301460in}{2.193053in}}%
\pgfpathlineto{\pgfqpoint{3.309308in}{2.198044in}}%
\pgfpathlineto{\pgfqpoint{3.317147in}{2.203143in}}%
\pgfpathclose%
\pgfusepath{fill}%
\end{pgfscope}%
\begin{pgfscope}%
\pgfpathrectangle{\pgfqpoint{1.254980in}{0.150000in}}{\pgfqpoint{5.490039in}{5.490039in}}%
\pgfusepath{clip}%
\pgfsetbuttcap%
\pgfsetroundjoin%
\definecolor{currentfill}{rgb}{0.267004,0.004874,0.329415}%
\pgfsetfillcolor{currentfill}%
\pgfsetfillopacity{0.700000}%
\pgfsetlinewidth{0.000000pt}%
\definecolor{currentstroke}{rgb}{0.000000,0.000000,0.000000}%
\pgfsetstrokecolor{currentstroke}%
\pgfsetdash{}{0pt}%
\pgfpathmoveto{\pgfqpoint{3.585806in}{2.172235in}}%
\pgfpathlineto{\pgfqpoint{3.598737in}{2.167814in}}%
\pgfpathlineto{\pgfqpoint{3.611673in}{2.163425in}}%
\pgfpathlineto{\pgfqpoint{3.624614in}{2.159066in}}%
\pgfpathlineto{\pgfqpoint{3.637561in}{2.154737in}}%
\pgfpathlineto{\pgfqpoint{3.629859in}{2.148228in}}%
\pgfpathlineto{\pgfqpoint{3.622151in}{2.141781in}}%
\pgfpathlineto{\pgfqpoint{3.614436in}{2.135399in}}%
\pgfpathlineto{\pgfqpoint{3.606714in}{2.129085in}}%
\pgfpathlineto{\pgfqpoint{3.593753in}{2.133556in}}%
\pgfpathlineto{\pgfqpoint{3.580797in}{2.138057in}}%
\pgfpathlineto{\pgfqpoint{3.567846in}{2.142589in}}%
\pgfpathlineto{\pgfqpoint{3.554900in}{2.147153in}}%
\pgfpathlineto{\pgfqpoint{3.562637in}{2.153319in}}%
\pgfpathlineto{\pgfqpoint{3.570367in}{2.159557in}}%
\pgfpathlineto{\pgfqpoint{3.578090in}{2.165863in}}%
\pgfpathlineto{\pgfqpoint{3.585806in}{2.172235in}}%
\pgfpathclose%
\pgfusepath{fill}%
\end{pgfscope}%
\begin{pgfscope}%
\pgfpathrectangle{\pgfqpoint{1.254980in}{0.150000in}}{\pgfqpoint{5.490039in}{5.490039in}}%
\pgfusepath{clip}%
\pgfsetbuttcap%
\pgfsetroundjoin%
\definecolor{currentfill}{rgb}{0.280894,0.078907,0.402329}%
\pgfsetfillcolor{currentfill}%
\pgfsetfillopacity{0.700000}%
\pgfsetlinewidth{0.000000pt}%
\definecolor{currentstroke}{rgb}{0.000000,0.000000,0.000000}%
\pgfsetstrokecolor{currentstroke}%
\pgfsetdash{}{0pt}%
\pgfpathmoveto{\pgfqpoint{4.802914in}{2.295036in}}%
\pgfpathlineto{\pgfqpoint{4.816137in}{2.293186in}}%
\pgfpathlineto{\pgfqpoint{4.829368in}{2.291361in}}%
\pgfpathlineto{\pgfqpoint{4.842606in}{2.289560in}}%
\pgfpathlineto{\pgfqpoint{4.855852in}{2.287784in}}%
\pgfpathlineto{\pgfqpoint{4.848609in}{2.280567in}}%
\pgfpathlineto{\pgfqpoint{4.841360in}{2.273298in}}%
\pgfpathlineto{\pgfqpoint{4.834105in}{2.265976in}}%
\pgfpathlineto{\pgfqpoint{4.826844in}{2.258601in}}%
\pgfpathlineto{\pgfqpoint{4.813586in}{2.260378in}}%
\pgfpathlineto{\pgfqpoint{4.800336in}{2.262180in}}%
\pgfpathlineto{\pgfqpoint{4.787093in}{2.264006in}}%
\pgfpathlineto{\pgfqpoint{4.773857in}{2.265858in}}%
\pgfpathlineto{\pgfqpoint{4.781130in}{2.273227in}}%
\pgfpathlineto{\pgfqpoint{4.788397in}{2.280546in}}%
\pgfpathlineto{\pgfqpoint{4.795658in}{2.287816in}}%
\pgfpathlineto{\pgfqpoint{4.802914in}{2.295036in}}%
\pgfpathclose%
\pgfusepath{fill}%
\end{pgfscope}%
\begin{pgfscope}%
\pgfpathrectangle{\pgfqpoint{1.254980in}{0.150000in}}{\pgfqpoint{5.490039in}{5.490039in}}%
\pgfusepath{clip}%
\pgfsetbuttcap%
\pgfsetroundjoin%
\definecolor{currentfill}{rgb}{0.267004,0.004874,0.329415}%
\pgfsetfillcolor{currentfill}%
\pgfsetfillopacity{0.700000}%
\pgfsetlinewidth{0.000000pt}%
\definecolor{currentstroke}{rgb}{0.000000,0.000000,0.000000}%
\pgfsetstrokecolor{currentstroke}%
\pgfsetdash{}{0pt}%
\pgfpathmoveto{\pgfqpoint{3.720084in}{2.164840in}}%
\pgfpathlineto{\pgfqpoint{3.733043in}{2.160792in}}%
\pgfpathlineto{\pgfqpoint{3.746007in}{2.156774in}}%
\pgfpathlineto{\pgfqpoint{3.758977in}{2.152785in}}%
\pgfpathlineto{\pgfqpoint{3.771952in}{2.148826in}}%
\pgfpathlineto{\pgfqpoint{3.764304in}{2.141851in}}%
\pgfpathlineto{\pgfqpoint{3.756650in}{2.134919in}}%
\pgfpathlineto{\pgfqpoint{3.748989in}{2.128033in}}%
\pgfpathlineto{\pgfqpoint{3.741322in}{2.121196in}}%
\pgfpathlineto{\pgfqpoint{3.728333in}{2.125285in}}%
\pgfpathlineto{\pgfqpoint{3.715350in}{2.129403in}}%
\pgfpathlineto{\pgfqpoint{3.702372in}{2.133550in}}%
\pgfpathlineto{\pgfqpoint{3.689399in}{2.137727in}}%
\pgfpathlineto{\pgfqpoint{3.697080in}{2.144430in}}%
\pgfpathlineto{\pgfqpoint{3.704754in}{2.151185in}}%
\pgfpathlineto{\pgfqpoint{3.712422in}{2.157989in}}%
\pgfpathlineto{\pgfqpoint{3.720084in}{2.164840in}}%
\pgfpathclose%
\pgfusepath{fill}%
\end{pgfscope}%
\begin{pgfscope}%
\pgfpathrectangle{\pgfqpoint{1.254980in}{0.150000in}}{\pgfqpoint{5.490039in}{5.490039in}}%
\pgfusepath{clip}%
\pgfsetbuttcap%
\pgfsetroundjoin%
\definecolor{currentfill}{rgb}{0.274952,0.037752,0.364543}%
\pgfsetfillcolor{currentfill}%
\pgfsetfillopacity{0.700000}%
\pgfsetlinewidth{0.000000pt}%
\definecolor{currentstroke}{rgb}{0.000000,0.000000,0.000000}%
\pgfsetstrokecolor{currentstroke}%
\pgfsetdash{}{0pt}%
\pgfpathmoveto{\pgfqpoint{3.182659in}{2.227942in}}%
\pgfpathlineto{\pgfqpoint{3.195529in}{2.222250in}}%
\pgfpathlineto{\pgfqpoint{3.208404in}{2.216594in}}%
\pgfpathlineto{\pgfqpoint{3.221282in}{2.210975in}}%
\pgfpathlineto{\pgfqpoint{3.234165in}{2.205391in}}%
\pgfpathlineto{\pgfqpoint{3.226273in}{2.200923in}}%
\pgfpathlineto{\pgfqpoint{3.218372in}{2.196582in}}%
\pgfpathlineto{\pgfqpoint{3.210462in}{2.192374in}}%
\pgfpathlineto{\pgfqpoint{3.202543in}{2.188301in}}%
\pgfpathlineto{\pgfqpoint{3.189641in}{2.194066in}}%
\pgfpathlineto{\pgfqpoint{3.176743in}{2.199867in}}%
\pgfpathlineto{\pgfqpoint{3.163849in}{2.205703in}}%
\pgfpathlineto{\pgfqpoint{3.150959in}{2.211576in}}%
\pgfpathlineto{\pgfqpoint{3.158898in}{2.215463in}}%
\pgfpathlineto{\pgfqpoint{3.166828in}{2.219489in}}%
\pgfpathlineto{\pgfqpoint{3.174748in}{2.223650in}}%
\pgfpathlineto{\pgfqpoint{3.182659in}{2.227942in}}%
\pgfpathclose%
\pgfusepath{fill}%
\end{pgfscope}%
\begin{pgfscope}%
\pgfpathrectangle{\pgfqpoint{1.254980in}{0.150000in}}{\pgfqpoint{5.490039in}{5.490039in}}%
\pgfusepath{clip}%
\pgfsetbuttcap%
\pgfsetroundjoin%
\definecolor{currentfill}{rgb}{0.282656,0.100196,0.422160}%
\pgfsetfillcolor{currentfill}%
\pgfsetfillopacity{0.700000}%
\pgfsetlinewidth{0.000000pt}%
\definecolor{currentstroke}{rgb}{0.000000,0.000000,0.000000}%
\pgfsetstrokecolor{currentstroke}%
\pgfsetdash{}{0pt}%
\pgfpathmoveto{\pgfqpoint{5.019599in}{2.330116in}}%
\pgfpathlineto{\pgfqpoint{5.032886in}{2.328523in}}%
\pgfpathlineto{\pgfqpoint{5.046180in}{2.326955in}}%
\pgfpathlineto{\pgfqpoint{5.059482in}{2.325411in}}%
\pgfpathlineto{\pgfqpoint{5.072791in}{2.323891in}}%
\pgfpathlineto{\pgfqpoint{5.065637in}{2.317151in}}%
\pgfpathlineto{\pgfqpoint{5.058476in}{2.310356in}}%
\pgfpathlineto{\pgfqpoint{5.051309in}{2.303505in}}%
\pgfpathlineto{\pgfqpoint{5.044135in}{2.296598in}}%
\pgfpathlineto{\pgfqpoint{5.030813in}{2.298093in}}%
\pgfpathlineto{\pgfqpoint{5.017498in}{2.299612in}}%
\pgfpathlineto{\pgfqpoint{5.004191in}{2.301156in}}%
\pgfpathlineto{\pgfqpoint{4.990891in}{2.302724in}}%
\pgfpathlineto{\pgfqpoint{4.998078in}{2.309651in}}%
\pgfpathlineto{\pgfqpoint{5.005258in}{2.316525in}}%
\pgfpathlineto{\pgfqpoint{5.012432in}{2.323346in}}%
\pgfpathlineto{\pgfqpoint{5.019599in}{2.330116in}}%
\pgfpathclose%
\pgfusepath{fill}%
\end{pgfscope}%
\begin{pgfscope}%
\pgfpathrectangle{\pgfqpoint{1.254980in}{0.150000in}}{\pgfqpoint{5.490039in}{5.490039in}}%
\pgfusepath{clip}%
\pgfsetbuttcap%
\pgfsetroundjoin%
\definecolor{currentfill}{rgb}{0.272594,0.025563,0.353093}%
\pgfsetfillcolor{currentfill}%
\pgfsetfillopacity{0.700000}%
\pgfsetlinewidth{0.000000pt}%
\definecolor{currentstroke}{rgb}{0.000000,0.000000,0.000000}%
\pgfsetstrokecolor{currentstroke}%
\pgfsetdash{}{0pt}%
\pgfpathmoveto{\pgfqpoint{4.287643in}{2.206971in}}%
\pgfpathlineto{\pgfqpoint{4.300732in}{2.204272in}}%
\pgfpathlineto{\pgfqpoint{4.313828in}{2.201599in}}%
\pgfpathlineto{\pgfqpoint{4.326929in}{2.198953in}}%
\pgfpathlineto{\pgfqpoint{4.340038in}{2.196333in}}%
\pgfpathlineto{\pgfqpoint{4.332599in}{2.188497in}}%
\pgfpathlineto{\pgfqpoint{4.325154in}{2.180639in}}%
\pgfpathlineto{\pgfqpoint{4.317704in}{2.172757in}}%
\pgfpathlineto{\pgfqpoint{4.310249in}{2.164855in}}%
\pgfpathlineto{\pgfqpoint{4.297129in}{2.167541in}}%
\pgfpathlineto{\pgfqpoint{4.284016in}{2.170253in}}%
\pgfpathlineto{\pgfqpoint{4.270909in}{2.172991in}}%
\pgfpathlineto{\pgfqpoint{4.257808in}{2.175756in}}%
\pgfpathlineto{\pgfqpoint{4.265275in}{2.183588in}}%
\pgfpathlineto{\pgfqpoint{4.272737in}{2.191401in}}%
\pgfpathlineto{\pgfqpoint{4.280193in}{2.199196in}}%
\pgfpathlineto{\pgfqpoint{4.287643in}{2.206971in}}%
\pgfpathclose%
\pgfusepath{fill}%
\end{pgfscope}%
\begin{pgfscope}%
\pgfpathrectangle{\pgfqpoint{1.254980in}{0.150000in}}{\pgfqpoint{5.490039in}{5.490039in}}%
\pgfusepath{clip}%
\pgfsetbuttcap%
\pgfsetroundjoin%
\definecolor{currentfill}{rgb}{0.268510,0.009605,0.335427}%
\pgfsetfillcolor{currentfill}%
\pgfsetfillopacity{0.700000}%
\pgfsetlinewidth{0.000000pt}%
\definecolor{currentstroke}{rgb}{0.000000,0.000000,0.000000}%
\pgfsetstrokecolor{currentstroke}%
\pgfsetdash{}{0pt}%
\pgfpathmoveto{\pgfqpoint{4.071040in}{2.180480in}}%
\pgfpathlineto{\pgfqpoint{4.084076in}{2.177314in}}%
\pgfpathlineto{\pgfqpoint{4.097119in}{2.174176in}}%
\pgfpathlineto{\pgfqpoint{4.110167in}{2.171066in}}%
\pgfpathlineto{\pgfqpoint{4.123222in}{2.167982in}}%
\pgfpathlineto{\pgfqpoint{4.115705in}{2.160257in}}%
\pgfpathlineto{\pgfqpoint{4.108182in}{2.152530in}}%
\pgfpathlineto{\pgfqpoint{4.100654in}{2.144803in}}%
\pgfpathlineto{\pgfqpoint{4.093120in}{2.137079in}}%
\pgfpathlineto{\pgfqpoint{4.080054in}{2.140254in}}%
\pgfpathlineto{\pgfqpoint{4.066993in}{2.143456in}}%
\pgfpathlineto{\pgfqpoint{4.053939in}{2.146685in}}%
\pgfpathlineto{\pgfqpoint{4.040891in}{2.149942in}}%
\pgfpathlineto{\pgfqpoint{4.048436in}{2.157570in}}%
\pgfpathlineto{\pgfqpoint{4.055976in}{2.165204in}}%
\pgfpathlineto{\pgfqpoint{4.063511in}{2.172841in}}%
\pgfpathlineto{\pgfqpoint{4.071040in}{2.180480in}}%
\pgfpathclose%
\pgfusepath{fill}%
\end{pgfscope}%
\begin{pgfscope}%
\pgfpathrectangle{\pgfqpoint{1.254980in}{0.150000in}}{\pgfqpoint{5.490039in}{5.490039in}}%
\pgfusepath{clip}%
\pgfsetbuttcap%
\pgfsetroundjoin%
\definecolor{currentfill}{rgb}{0.277018,0.050344,0.375715}%
\pgfsetfillcolor{currentfill}%
\pgfsetfillopacity{0.700000}%
\pgfsetlinewidth{0.000000pt}%
\definecolor{currentstroke}{rgb}{0.000000,0.000000,0.000000}%
\pgfsetstrokecolor{currentstroke}%
\pgfsetdash{}{0pt}%
\pgfpathmoveto{\pgfqpoint{4.504279in}{2.238764in}}%
\pgfpathlineto{\pgfqpoint{4.517425in}{2.236470in}}%
\pgfpathlineto{\pgfqpoint{4.530579in}{2.234200in}}%
\pgfpathlineto{\pgfqpoint{4.543739in}{2.231957in}}%
\pgfpathlineto{\pgfqpoint{4.556905in}{2.229738in}}%
\pgfpathlineto{\pgfqpoint{4.549545in}{2.222020in}}%
\pgfpathlineto{\pgfqpoint{4.542178in}{2.214261in}}%
\pgfpathlineto{\pgfqpoint{4.534806in}{2.206463in}}%
\pgfpathlineto{\pgfqpoint{4.527429in}{2.198627in}}%
\pgfpathlineto{\pgfqpoint{4.514250in}{2.200885in}}%
\pgfpathlineto{\pgfqpoint{4.501079in}{2.203169in}}%
\pgfpathlineto{\pgfqpoint{4.487915in}{2.205478in}}%
\pgfpathlineto{\pgfqpoint{4.474757in}{2.207813in}}%
\pgfpathlineto{\pgfqpoint{4.482146in}{2.215604in}}%
\pgfpathlineto{\pgfqpoint{4.489529in}{2.223361in}}%
\pgfpathlineto{\pgfqpoint{4.496907in}{2.231081in}}%
\pgfpathlineto{\pgfqpoint{4.504279in}{2.238764in}}%
\pgfpathclose%
\pgfusepath{fill}%
\end{pgfscope}%
\begin{pgfscope}%
\pgfpathrectangle{\pgfqpoint{1.254980in}{0.150000in}}{\pgfqpoint{5.490039in}{5.490039in}}%
\pgfusepath{clip}%
\pgfsetbuttcap%
\pgfsetroundjoin%
\definecolor{currentfill}{rgb}{0.282656,0.100196,0.422160}%
\pgfsetfillcolor{currentfill}%
\pgfsetfillopacity{0.700000}%
\pgfsetlinewidth{0.000000pt}%
\definecolor{currentstroke}{rgb}{0.000000,0.000000,0.000000}%
\pgfsetstrokecolor{currentstroke}%
\pgfsetdash{}{0pt}%
\pgfpathmoveto{\pgfqpoint{2.861654in}{2.326967in}}%
\pgfpathlineto{\pgfqpoint{2.874494in}{2.320107in}}%
\pgfpathlineto{\pgfqpoint{2.887338in}{2.313290in}}%
\pgfpathlineto{\pgfqpoint{2.900185in}{2.306516in}}%
\pgfpathlineto{\pgfqpoint{2.913035in}{2.299784in}}%
\pgfpathlineto{\pgfqpoint{2.904957in}{2.297479in}}%
\pgfpathlineto{\pgfqpoint{2.896867in}{2.295356in}}%
\pgfpathlineto{\pgfqpoint{2.888764in}{2.293421in}}%
\pgfpathlineto{\pgfqpoint{2.880650in}{2.291679in}}%
\pgfpathlineto{\pgfqpoint{2.867776in}{2.298619in}}%
\pgfpathlineto{\pgfqpoint{2.854905in}{2.305601in}}%
\pgfpathlineto{\pgfqpoint{2.842037in}{2.312626in}}%
\pgfpathlineto{\pgfqpoint{2.829172in}{2.319695in}}%
\pgfpathlineto{\pgfqpoint{2.837311in}{2.321224in}}%
\pgfpathlineto{\pgfqpoint{2.845438in}{2.322949in}}%
\pgfpathlineto{\pgfqpoint{2.853552in}{2.324865in}}%
\pgfpathlineto{\pgfqpoint{2.861654in}{2.326967in}}%
\pgfpathclose%
\pgfusepath{fill}%
\end{pgfscope}%
\begin{pgfscope}%
\pgfpathrectangle{\pgfqpoint{1.254980in}{0.150000in}}{\pgfqpoint{5.490039in}{5.490039in}}%
\pgfusepath{clip}%
\pgfsetbuttcap%
\pgfsetroundjoin%
\definecolor{currentfill}{rgb}{0.283229,0.120777,0.440584}%
\pgfsetfillcolor{currentfill}%
\pgfsetfillopacity{0.700000}%
\pgfsetlinewidth{0.000000pt}%
\definecolor{currentstroke}{rgb}{0.000000,0.000000,0.000000}%
\pgfsetstrokecolor{currentstroke}%
\pgfsetdash{}{0pt}%
\pgfpathmoveto{\pgfqpoint{5.236319in}{2.364030in}}%
\pgfpathlineto{\pgfqpoint{5.249669in}{2.362638in}}%
\pgfpathlineto{\pgfqpoint{5.263027in}{2.361269in}}%
\pgfpathlineto{\pgfqpoint{5.276393in}{2.359925in}}%
\pgfpathlineto{\pgfqpoint{5.289766in}{2.358604in}}%
\pgfpathlineto{\pgfqpoint{5.282706in}{2.352411in}}%
\pgfpathlineto{\pgfqpoint{5.275640in}{2.346168in}}%
\pgfpathlineto{\pgfqpoint{5.268567in}{2.339871in}}%
\pgfpathlineto{\pgfqpoint{5.261487in}{2.333519in}}%
\pgfpathlineto{\pgfqpoint{5.248099in}{2.334789in}}%
\pgfpathlineto{\pgfqpoint{5.234719in}{2.336082in}}%
\pgfpathlineto{\pgfqpoint{5.221347in}{2.337400in}}%
\pgfpathlineto{\pgfqpoint{5.207982in}{2.338741in}}%
\pgfpathlineto{\pgfqpoint{5.215076in}{2.345139in}}%
\pgfpathlineto{\pgfqpoint{5.222164in}{2.351485in}}%
\pgfpathlineto{\pgfqpoint{5.229245in}{2.357781in}}%
\pgfpathlineto{\pgfqpoint{5.236319in}{2.364030in}}%
\pgfpathclose%
\pgfusepath{fill}%
\end{pgfscope}%
\begin{pgfscope}%
\pgfpathrectangle{\pgfqpoint{1.254980in}{0.150000in}}{\pgfqpoint{5.490039in}{5.490039in}}%
\pgfusepath{clip}%
\pgfsetbuttcap%
\pgfsetroundjoin%
\definecolor{currentfill}{rgb}{0.267004,0.004874,0.329415}%
\pgfsetfillcolor{currentfill}%
\pgfsetfillopacity{0.700000}%
\pgfsetlinewidth{0.000000pt}%
\definecolor{currentstroke}{rgb}{0.000000,0.000000,0.000000}%
\pgfsetstrokecolor{currentstroke}%
\pgfsetdash{}{0pt}%
\pgfpathmoveto{\pgfqpoint{3.854389in}{2.162025in}}%
\pgfpathlineto{\pgfqpoint{3.867379in}{2.158328in}}%
\pgfpathlineto{\pgfqpoint{3.880375in}{2.154660in}}%
\pgfpathlineto{\pgfqpoint{3.893376in}{2.151020in}}%
\pgfpathlineto{\pgfqpoint{3.906384in}{2.147408in}}%
\pgfpathlineto{\pgfqpoint{3.898785in}{2.140065in}}%
\pgfpathlineto{\pgfqpoint{3.891181in}{2.132746in}}%
\pgfpathlineto{\pgfqpoint{3.883571in}{2.125457in}}%
\pgfpathlineto{\pgfqpoint{3.875955in}{2.118198in}}%
\pgfpathlineto{\pgfqpoint{3.862935in}{2.121926in}}%
\pgfpathlineto{\pgfqpoint{3.849921in}{2.125683in}}%
\pgfpathlineto{\pgfqpoint{3.836912in}{2.129468in}}%
\pgfpathlineto{\pgfqpoint{3.823909in}{2.133282in}}%
\pgfpathlineto{\pgfqpoint{3.831538in}{2.140419in}}%
\pgfpathlineto{\pgfqpoint{3.839161in}{2.147590in}}%
\pgfpathlineto{\pgfqpoint{3.846778in}{2.154793in}}%
\pgfpathlineto{\pgfqpoint{3.854389in}{2.162025in}}%
\pgfpathclose%
\pgfusepath{fill}%
\end{pgfscope}%
\begin{pgfscope}%
\pgfpathrectangle{\pgfqpoint{1.254980in}{0.150000in}}{\pgfqpoint{5.490039in}{5.490039in}}%
\pgfusepath{clip}%
\pgfsetbuttcap%
\pgfsetroundjoin%
\definecolor{currentfill}{rgb}{0.282623,0.140926,0.457517}%
\pgfsetfillcolor{currentfill}%
\pgfsetfillopacity{0.700000}%
\pgfsetlinewidth{0.000000pt}%
\definecolor{currentstroke}{rgb}{0.000000,0.000000,0.000000}%
\pgfsetstrokecolor{currentstroke}%
\pgfsetdash{}{0pt}%
\pgfpathmoveto{\pgfqpoint{5.453030in}{2.395736in}}%
\pgfpathlineto{\pgfqpoint{5.466443in}{2.394488in}}%
\pgfpathlineto{\pgfqpoint{5.479864in}{2.393263in}}%
\pgfpathlineto{\pgfqpoint{5.493293in}{2.392062in}}%
\pgfpathlineto{\pgfqpoint{5.506730in}{2.390884in}}%
\pgfpathlineto{\pgfqpoint{5.499770in}{2.385264in}}%
\pgfpathlineto{\pgfqpoint{5.492804in}{2.379602in}}%
\pgfpathlineto{\pgfqpoint{5.485832in}{2.373896in}}%
\pgfpathlineto{\pgfqpoint{5.478852in}{2.368142in}}%
\pgfpathlineto{\pgfqpoint{5.465399in}{2.369243in}}%
\pgfpathlineto{\pgfqpoint{5.451953in}{2.370367in}}%
\pgfpathlineto{\pgfqpoint{5.438516in}{2.371514in}}%
\pgfpathlineto{\pgfqpoint{5.425087in}{2.372685in}}%
\pgfpathlineto{\pgfqpoint{5.432083in}{2.378511in}}%
\pgfpathlineto{\pgfqpoint{5.439072in}{2.384293in}}%
\pgfpathlineto{\pgfqpoint{5.446054in}{2.390034in}}%
\pgfpathlineto{\pgfqpoint{5.453030in}{2.395736in}}%
\pgfpathclose%
\pgfusepath{fill}%
\end{pgfscope}%
\begin{pgfscope}%
\pgfpathrectangle{\pgfqpoint{1.254980in}{0.150000in}}{\pgfqpoint{5.490039in}{5.490039in}}%
\pgfusepath{clip}%
\pgfsetbuttcap%
\pgfsetroundjoin%
\definecolor{currentfill}{rgb}{0.282290,0.145912,0.461510}%
\pgfsetfillcolor{currentfill}%
\pgfsetfillopacity{0.700000}%
\pgfsetlinewidth{0.000000pt}%
\definecolor{currentstroke}{rgb}{0.000000,0.000000,0.000000}%
\pgfsetstrokecolor{currentstroke}%
\pgfsetdash{}{0pt}%
\pgfpathmoveto{\pgfqpoint{2.675010in}{2.408025in}}%
\pgfpathlineto{\pgfqpoint{2.687843in}{2.400406in}}%
\pgfpathlineto{\pgfqpoint{2.700678in}{2.392836in}}%
\pgfpathlineto{\pgfqpoint{2.713516in}{2.385314in}}%
\pgfpathlineto{\pgfqpoint{2.726356in}{2.377840in}}%
\pgfpathlineto{\pgfqpoint{2.718153in}{2.376946in}}%
\pgfpathlineto{\pgfqpoint{2.709936in}{2.376267in}}%
\pgfpathlineto{\pgfqpoint{2.701705in}{2.375807in}}%
\pgfpathlineto{\pgfqpoint{2.693460in}{2.375572in}}%
\pgfpathlineto{\pgfqpoint{2.680593in}{2.383269in}}%
\pgfpathlineto{\pgfqpoint{2.667728in}{2.391014in}}%
\pgfpathlineto{\pgfqpoint{2.654866in}{2.398806in}}%
\pgfpathlineto{\pgfqpoint{2.642007in}{2.406648in}}%
\pgfpathlineto{\pgfqpoint{2.650279in}{2.406654in}}%
\pgfpathlineto{\pgfqpoint{2.658537in}{2.406890in}}%
\pgfpathlineto{\pgfqpoint{2.666781in}{2.407349in}}%
\pgfpathlineto{\pgfqpoint{2.675010in}{2.408025in}}%
\pgfpathclose%
\pgfusepath{fill}%
\end{pgfscope}%
\begin{pgfscope}%
\pgfpathrectangle{\pgfqpoint{1.254980in}{0.150000in}}{\pgfqpoint{5.490039in}{5.490039in}}%
\pgfusepath{clip}%
\pgfsetbuttcap%
\pgfsetroundjoin%
\definecolor{currentfill}{rgb}{0.279574,0.170599,0.479997}%
\pgfsetfillcolor{currentfill}%
\pgfsetfillopacity{0.700000}%
\pgfsetlinewidth{0.000000pt}%
\definecolor{currentstroke}{rgb}{0.000000,0.000000,0.000000}%
\pgfsetstrokecolor{currentstroke}%
\pgfsetdash{}{0pt}%
\pgfpathmoveto{\pgfqpoint{5.886230in}{2.450230in}}%
\pgfpathlineto{\pgfqpoint{5.899764in}{2.449104in}}%
\pgfpathlineto{\pgfqpoint{5.913306in}{2.448000in}}%
\pgfpathlineto{\pgfqpoint{5.926857in}{2.446920in}}%
\pgfpathlineto{\pgfqpoint{5.940416in}{2.445862in}}%
\pgfpathlineto{\pgfqpoint{5.933669in}{2.441279in}}%
\pgfpathlineto{\pgfqpoint{5.926916in}{2.436689in}}%
\pgfpathlineto{\pgfqpoint{5.920156in}{2.432089in}}%
\pgfpathlineto{\pgfqpoint{5.913391in}{2.427475in}}%
\pgfpathlineto{\pgfqpoint{5.899812in}{2.428403in}}%
\pgfpathlineto{\pgfqpoint{5.886241in}{2.429355in}}%
\pgfpathlineto{\pgfqpoint{5.872678in}{2.430329in}}%
\pgfpathlineto{\pgfqpoint{5.859124in}{2.431326in}}%
\pgfpathlineto{\pgfqpoint{5.865909in}{2.436065in}}%
\pgfpathlineto{\pgfqpoint{5.872688in}{2.440793in}}%
\pgfpathlineto{\pgfqpoint{5.879462in}{2.445513in}}%
\pgfpathlineto{\pgfqpoint{5.886230in}{2.450230in}}%
\pgfpathclose%
\pgfusepath{fill}%
\end{pgfscope}%
\begin{pgfscope}%
\pgfpathrectangle{\pgfqpoint{1.254980in}{0.150000in}}{\pgfqpoint{5.490039in}{5.490039in}}%
\pgfusepath{clip}%
\pgfsetbuttcap%
\pgfsetroundjoin%
\definecolor{currentfill}{rgb}{0.280267,0.073417,0.397163}%
\pgfsetfillcolor{currentfill}%
\pgfsetfillopacity{0.700000}%
\pgfsetlinewidth{0.000000pt}%
\definecolor{currentstroke}{rgb}{0.000000,0.000000,0.000000}%
\pgfsetstrokecolor{currentstroke}%
\pgfsetdash{}{0pt}%
\pgfpathmoveto{\pgfqpoint{4.720986in}{2.273511in}}%
\pgfpathlineto{\pgfqpoint{4.734193in}{2.271560in}}%
\pgfpathlineto{\pgfqpoint{4.747407in}{2.269635in}}%
\pgfpathlineto{\pgfqpoint{4.760628in}{2.267734in}}%
\pgfpathlineto{\pgfqpoint{4.773857in}{2.265858in}}%
\pgfpathlineto{\pgfqpoint{4.766578in}{2.258438in}}%
\pgfpathlineto{\pgfqpoint{4.759292in}{2.250968in}}%
\pgfpathlineto{\pgfqpoint{4.752001in}{2.243447in}}%
\pgfpathlineto{\pgfqpoint{4.744704in}{2.235876in}}%
\pgfpathlineto{\pgfqpoint{4.731464in}{2.237766in}}%
\pgfpathlineto{\pgfqpoint{4.718231in}{2.239681in}}%
\pgfpathlineto{\pgfqpoint{4.705005in}{2.241621in}}%
\pgfpathlineto{\pgfqpoint{4.691786in}{2.243586in}}%
\pgfpathlineto{\pgfqpoint{4.699095in}{2.251139in}}%
\pgfpathlineto{\pgfqpoint{4.706398in}{2.258644in}}%
\pgfpathlineto{\pgfqpoint{4.713695in}{2.266101in}}%
\pgfpathlineto{\pgfqpoint{4.720986in}{2.273511in}}%
\pgfpathclose%
\pgfusepath{fill}%
\end{pgfscope}%
\begin{pgfscope}%
\pgfpathrectangle{\pgfqpoint{1.254980in}{0.150000in}}{\pgfqpoint{5.490039in}{5.490039in}}%
\pgfusepath{clip}%
\pgfsetbuttcap%
\pgfsetroundjoin%
\definecolor{currentfill}{rgb}{0.281412,0.155834,0.469201}%
\pgfsetfillcolor{currentfill}%
\pgfsetfillopacity{0.700000}%
\pgfsetlinewidth{0.000000pt}%
\definecolor{currentstroke}{rgb}{0.000000,0.000000,0.000000}%
\pgfsetstrokecolor{currentstroke}%
\pgfsetdash{}{0pt}%
\pgfpathmoveto{\pgfqpoint{5.669682in}{2.424566in}}%
\pgfpathlineto{\pgfqpoint{5.683157in}{2.423406in}}%
\pgfpathlineto{\pgfqpoint{5.696640in}{2.422270in}}%
\pgfpathlineto{\pgfqpoint{5.710131in}{2.421156in}}%
\pgfpathlineto{\pgfqpoint{5.723630in}{2.420066in}}%
\pgfpathlineto{\pgfqpoint{5.716775in}{2.414998in}}%
\pgfpathlineto{\pgfqpoint{5.709914in}{2.409903in}}%
\pgfpathlineto{\pgfqpoint{5.703047in}{2.404778in}}%
\pgfpathlineto{\pgfqpoint{5.696172in}{2.399619in}}%
\pgfpathlineto{\pgfqpoint{5.682655in}{2.400606in}}%
\pgfpathlineto{\pgfqpoint{5.669146in}{2.401616in}}%
\pgfpathlineto{\pgfqpoint{5.655645in}{2.402649in}}%
\pgfpathlineto{\pgfqpoint{5.642152in}{2.403706in}}%
\pgfpathlineto{\pgfqpoint{5.649044in}{2.408963in}}%
\pgfpathlineto{\pgfqpoint{5.655930in}{2.414190in}}%
\pgfpathlineto{\pgfqpoint{5.662809in}{2.419390in}}%
\pgfpathlineto{\pgfqpoint{5.669682in}{2.424566in}}%
\pgfpathclose%
\pgfusepath{fill}%
\end{pgfscope}%
\begin{pgfscope}%
\pgfpathrectangle{\pgfqpoint{1.254980in}{0.150000in}}{\pgfqpoint{5.490039in}{5.490039in}}%
\pgfusepath{clip}%
\pgfsetbuttcap%
\pgfsetroundjoin%
\definecolor{currentfill}{rgb}{0.278791,0.062145,0.386592}%
\pgfsetfillcolor{currentfill}%
\pgfsetfillopacity{0.700000}%
\pgfsetlinewidth{0.000000pt}%
\definecolor{currentstroke}{rgb}{0.000000,0.000000,0.000000}%
\pgfsetstrokecolor{currentstroke}%
\pgfsetdash{}{0pt}%
\pgfpathmoveto{\pgfqpoint{3.047980in}{2.259902in}}%
\pgfpathlineto{\pgfqpoint{3.060839in}{2.253728in}}%
\pgfpathlineto{\pgfqpoint{3.073702in}{2.247593in}}%
\pgfpathlineto{\pgfqpoint{3.086568in}{2.241496in}}%
\pgfpathlineto{\pgfqpoint{3.099439in}{2.235437in}}%
\pgfpathlineto{\pgfqpoint{3.091469in}{2.231884in}}%
\pgfpathlineto{\pgfqpoint{3.083490in}{2.228484in}}%
\pgfpathlineto{\pgfqpoint{3.075500in}{2.225241in}}%
\pgfpathlineto{\pgfqpoint{3.067500in}{2.222161in}}%
\pgfpathlineto{\pgfqpoint{3.054608in}{2.228414in}}%
\pgfpathlineto{\pgfqpoint{3.041721in}{2.234705in}}%
\pgfpathlineto{\pgfqpoint{3.028836in}{2.241035in}}%
\pgfpathlineto{\pgfqpoint{3.015956in}{2.247403in}}%
\pgfpathlineto{\pgfqpoint{3.023978in}{2.250285in}}%
\pgfpathlineto{\pgfqpoint{3.031989in}{2.253331in}}%
\pgfpathlineto{\pgfqpoint{3.039989in}{2.256539in}}%
\pgfpathlineto{\pgfqpoint{3.047980in}{2.259902in}}%
\pgfpathclose%
\pgfusepath{fill}%
\end{pgfscope}%
\begin{pgfscope}%
\pgfpathrectangle{\pgfqpoint{1.254980in}{0.150000in}}{\pgfqpoint{5.490039in}{5.490039in}}%
\pgfusepath{clip}%
\pgfsetbuttcap%
\pgfsetroundjoin%
\definecolor{currentfill}{rgb}{0.282327,0.094955,0.417331}%
\pgfsetfillcolor{currentfill}%
\pgfsetfillopacity{0.700000}%
\pgfsetlinewidth{0.000000pt}%
\definecolor{currentstroke}{rgb}{0.000000,0.000000,0.000000}%
\pgfsetstrokecolor{currentstroke}%
\pgfsetdash{}{0pt}%
\pgfpathmoveto{\pgfqpoint{4.937767in}{2.309238in}}%
\pgfpathlineto{\pgfqpoint{4.951037in}{2.307573in}}%
\pgfpathlineto{\pgfqpoint{4.964314in}{2.305932in}}%
\pgfpathlineto{\pgfqpoint{4.977599in}{2.304316in}}%
\pgfpathlineto{\pgfqpoint{4.990891in}{2.302724in}}%
\pgfpathlineto{\pgfqpoint{4.983698in}{2.295742in}}%
\pgfpathlineto{\pgfqpoint{4.976499in}{2.288704in}}%
\pgfpathlineto{\pgfqpoint{4.969294in}{2.281610in}}%
\pgfpathlineto{\pgfqpoint{4.962082in}{2.274458in}}%
\pgfpathlineto{\pgfqpoint{4.948777in}{2.276038in}}%
\pgfpathlineto{\pgfqpoint{4.935480in}{2.277643in}}%
\pgfpathlineto{\pgfqpoint{4.922190in}{2.279272in}}%
\pgfpathlineto{\pgfqpoint{4.908907in}{2.280926in}}%
\pgfpathlineto{\pgfqpoint{4.916132in}{2.288084in}}%
\pgfpathlineto{\pgfqpoint{4.923350in}{2.295189in}}%
\pgfpathlineto{\pgfqpoint{4.930562in}{2.302240in}}%
\pgfpathlineto{\pgfqpoint{4.937767in}{2.309238in}}%
\pgfpathclose%
\pgfusepath{fill}%
\end{pgfscope}%
\begin{pgfscope}%
\pgfpathrectangle{\pgfqpoint{1.254980in}{0.150000in}}{\pgfqpoint{5.490039in}{5.490039in}}%
\pgfusepath{clip}%
\pgfsetbuttcap%
\pgfsetroundjoin%
\definecolor{currentfill}{rgb}{0.271305,0.019942,0.347269}%
\pgfsetfillcolor{currentfill}%
\pgfsetfillopacity{0.700000}%
\pgfsetlinewidth{0.000000pt}%
\definecolor{currentstroke}{rgb}{0.000000,0.000000,0.000000}%
\pgfsetstrokecolor{currentstroke}%
\pgfsetdash{}{0pt}%
\pgfpathmoveto{\pgfqpoint{4.205472in}{2.187079in}}%
\pgfpathlineto{\pgfqpoint{4.218546in}{2.184208in}}%
\pgfpathlineto{\pgfqpoint{4.231627in}{2.181364in}}%
\pgfpathlineto{\pgfqpoint{4.244715in}{2.178547in}}%
\pgfpathlineto{\pgfqpoint{4.257808in}{2.175756in}}%
\pgfpathlineto{\pgfqpoint{4.250336in}{2.167908in}}%
\pgfpathlineto{\pgfqpoint{4.242858in}{2.160045in}}%
\pgfpathlineto{\pgfqpoint{4.235375in}{2.152168in}}%
\pgfpathlineto{\pgfqpoint{4.227887in}{2.144280in}}%
\pgfpathlineto{\pgfqpoint{4.214781in}{2.147149in}}%
\pgfpathlineto{\pgfqpoint{4.201682in}{2.150045in}}%
\pgfpathlineto{\pgfqpoint{4.188590in}{2.152968in}}%
\pgfpathlineto{\pgfqpoint{4.175504in}{2.155917in}}%
\pgfpathlineto{\pgfqpoint{4.183004in}{2.163722in}}%
\pgfpathlineto{\pgfqpoint{4.190499in}{2.171518in}}%
\pgfpathlineto{\pgfqpoint{4.197988in}{2.179305in}}%
\pgfpathlineto{\pgfqpoint{4.205472in}{2.187079in}}%
\pgfpathclose%
\pgfusepath{fill}%
\end{pgfscope}%
\begin{pgfscope}%
\pgfpathrectangle{\pgfqpoint{1.254980in}{0.150000in}}{\pgfqpoint{5.490039in}{5.490039in}}%
\pgfusepath{clip}%
\pgfsetbuttcap%
\pgfsetroundjoin%
\definecolor{currentfill}{rgb}{0.268510,0.009605,0.335427}%
\pgfsetfillcolor{currentfill}%
\pgfsetfillopacity{0.700000}%
\pgfsetlinewidth{0.000000pt}%
\definecolor{currentstroke}{rgb}{0.000000,0.000000,0.000000}%
\pgfsetstrokecolor{currentstroke}%
\pgfsetdash{}{0pt}%
\pgfpathmoveto{\pgfqpoint{3.503166in}{2.165719in}}%
\pgfpathlineto{\pgfqpoint{3.516092in}{2.161030in}}%
\pgfpathlineto{\pgfqpoint{3.529023in}{2.156373in}}%
\pgfpathlineto{\pgfqpoint{3.541959in}{2.151747in}}%
\pgfpathlineto{\pgfqpoint{3.554900in}{2.147153in}}%
\pgfpathlineto{\pgfqpoint{3.547156in}{2.141062in}}%
\pgfpathlineto{\pgfqpoint{3.539406in}{2.135051in}}%
\pgfpathlineto{\pgfqpoint{3.531648in}{2.129122in}}%
\pgfpathlineto{\pgfqpoint{3.523883in}{2.123280in}}%
\pgfpathlineto{\pgfqpoint{3.510926in}{2.128030in}}%
\pgfpathlineto{\pgfqpoint{3.497974in}{2.132810in}}%
\pgfpathlineto{\pgfqpoint{3.485027in}{2.137623in}}%
\pgfpathlineto{\pgfqpoint{3.472085in}{2.142467in}}%
\pgfpathlineto{\pgfqpoint{3.479866in}{2.148149in}}%
\pgfpathlineto{\pgfqpoint{3.487640in}{2.153921in}}%
\pgfpathlineto{\pgfqpoint{3.495407in}{2.159779in}}%
\pgfpathlineto{\pgfqpoint{3.503166in}{2.165719in}}%
\pgfpathclose%
\pgfusepath{fill}%
\end{pgfscope}%
\begin{pgfscope}%
\pgfpathrectangle{\pgfqpoint{1.254980in}{0.150000in}}{\pgfqpoint{5.490039in}{5.490039in}}%
\pgfusepath{clip}%
\pgfsetbuttcap%
\pgfsetroundjoin%
\definecolor{currentfill}{rgb}{0.274952,0.037752,0.364543}%
\pgfsetfillcolor{currentfill}%
\pgfsetfillopacity{0.700000}%
\pgfsetlinewidth{0.000000pt}%
\definecolor{currentstroke}{rgb}{0.000000,0.000000,0.000000}%
\pgfsetstrokecolor{currentstroke}%
\pgfsetdash{}{0pt}%
\pgfpathmoveto{\pgfqpoint{4.422195in}{2.217408in}}%
\pgfpathlineto{\pgfqpoint{4.435325in}{2.214970in}}%
\pgfpathlineto{\pgfqpoint{4.448462in}{2.212559in}}%
\pgfpathlineto{\pgfqpoint{4.461606in}{2.210173in}}%
\pgfpathlineto{\pgfqpoint{4.474757in}{2.207813in}}%
\pgfpathlineto{\pgfqpoint{4.467363in}{2.199986in}}%
\pgfpathlineto{\pgfqpoint{4.459963in}{2.192125in}}%
\pgfpathlineto{\pgfqpoint{4.452557in}{2.184231in}}%
\pgfpathlineto{\pgfqpoint{4.445146in}{2.176305in}}%
\pgfpathlineto{\pgfqpoint{4.431984in}{2.178719in}}%
\pgfpathlineto{\pgfqpoint{4.418829in}{2.181158in}}%
\pgfpathlineto{\pgfqpoint{4.405680in}{2.183622in}}%
\pgfpathlineto{\pgfqpoint{4.392538in}{2.186112in}}%
\pgfpathlineto{\pgfqpoint{4.399961in}{2.193981in}}%
\pgfpathlineto{\pgfqpoint{4.407378in}{2.201820in}}%
\pgfpathlineto{\pgfqpoint{4.414789in}{2.209629in}}%
\pgfpathlineto{\pgfqpoint{4.422195in}{2.217408in}}%
\pgfpathclose%
\pgfusepath{fill}%
\end{pgfscope}%
\begin{pgfscope}%
\pgfpathrectangle{\pgfqpoint{1.254980in}{0.150000in}}{\pgfqpoint{5.490039in}{5.490039in}}%
\pgfusepath{clip}%
\pgfsetbuttcap%
\pgfsetroundjoin%
\definecolor{currentfill}{rgb}{0.271305,0.019942,0.347269}%
\pgfsetfillcolor{currentfill}%
\pgfsetfillopacity{0.700000}%
\pgfsetlinewidth{0.000000pt}%
\definecolor{currentstroke}{rgb}{0.000000,0.000000,0.000000}%
\pgfsetstrokecolor{currentstroke}%
\pgfsetdash{}{0pt}%
\pgfpathmoveto{\pgfqpoint{3.368719in}{2.182385in}}%
\pgfpathlineto{\pgfqpoint{3.381624in}{2.177280in}}%
\pgfpathlineto{\pgfqpoint{3.394533in}{2.172208in}}%
\pgfpathlineto{\pgfqpoint{3.407446in}{2.167170in}}%
\pgfpathlineto{\pgfqpoint{3.420365in}{2.162164in}}%
\pgfpathlineto{\pgfqpoint{3.412559in}{2.156739in}}%
\pgfpathlineto{\pgfqpoint{3.404746in}{2.151415in}}%
\pgfpathlineto{\pgfqpoint{3.396925in}{2.146196in}}%
\pgfpathlineto{\pgfqpoint{3.389096in}{2.141086in}}%
\pgfpathlineto{\pgfqpoint{3.376161in}{2.146259in}}%
\pgfpathlineto{\pgfqpoint{3.363230in}{2.151466in}}%
\pgfpathlineto{\pgfqpoint{3.350303in}{2.156706in}}%
\pgfpathlineto{\pgfqpoint{3.337381in}{2.161979in}}%
\pgfpathlineto{\pgfqpoint{3.345228in}{2.166916in}}%
\pgfpathlineto{\pgfqpoint{3.353067in}{2.171965in}}%
\pgfpathlineto{\pgfqpoint{3.360897in}{2.177123in}}%
\pgfpathlineto{\pgfqpoint{3.368719in}{2.182385in}}%
\pgfpathclose%
\pgfusepath{fill}%
\end{pgfscope}%
\begin{pgfscope}%
\pgfpathrectangle{\pgfqpoint{1.254980in}{0.150000in}}{\pgfqpoint{5.490039in}{5.490039in}}%
\pgfusepath{clip}%
\pgfsetbuttcap%
\pgfsetroundjoin%
\definecolor{currentfill}{rgb}{0.268510,0.009605,0.335427}%
\pgfsetfillcolor{currentfill}%
\pgfsetfillopacity{0.700000}%
\pgfsetlinewidth{0.000000pt}%
\definecolor{currentstroke}{rgb}{0.000000,0.000000,0.000000}%
\pgfsetstrokecolor{currentstroke}%
\pgfsetdash{}{0pt}%
\pgfpathmoveto{\pgfqpoint{3.988758in}{2.163244in}}%
\pgfpathlineto{\pgfqpoint{4.001782in}{2.159877in}}%
\pgfpathlineto{\pgfqpoint{4.014812in}{2.156538in}}%
\pgfpathlineto{\pgfqpoint{4.027848in}{2.153226in}}%
\pgfpathlineto{\pgfqpoint{4.040891in}{2.149942in}}%
\pgfpathlineto{\pgfqpoint{4.033339in}{2.142321in}}%
\pgfpathlineto{\pgfqpoint{4.025783in}{2.134710in}}%
\pgfpathlineto{\pgfqpoint{4.018221in}{2.127111in}}%
\pgfpathlineto{\pgfqpoint{4.010653in}{2.119526in}}%
\pgfpathlineto{\pgfqpoint{3.997598in}{2.122914in}}%
\pgfpathlineto{\pgfqpoint{3.984550in}{2.126330in}}%
\pgfpathlineto{\pgfqpoint{3.971507in}{2.129773in}}%
\pgfpathlineto{\pgfqpoint{3.958471in}{2.133245in}}%
\pgfpathlineto{\pgfqpoint{3.966051in}{2.140720in}}%
\pgfpathlineto{\pgfqpoint{3.973626in}{2.148214in}}%
\pgfpathlineto{\pgfqpoint{3.981194in}{2.155722in}}%
\pgfpathlineto{\pgfqpoint{3.988758in}{2.163244in}}%
\pgfpathclose%
\pgfusepath{fill}%
\end{pgfscope}%
\begin{pgfscope}%
\pgfpathrectangle{\pgfqpoint{1.254980in}{0.150000in}}{\pgfqpoint{5.490039in}{5.490039in}}%
\pgfusepath{clip}%
\pgfsetbuttcap%
\pgfsetroundjoin%
\definecolor{currentfill}{rgb}{0.283197,0.115680,0.436115}%
\pgfsetfillcolor{currentfill}%
\pgfsetfillopacity{0.700000}%
\pgfsetlinewidth{0.000000pt}%
\definecolor{currentstroke}{rgb}{0.000000,0.000000,0.000000}%
\pgfsetstrokecolor{currentstroke}%
\pgfsetdash{}{0pt}%
\pgfpathmoveto{\pgfqpoint{5.154602in}{2.344346in}}%
\pgfpathlineto{\pgfqpoint{5.167935in}{2.342909in}}%
\pgfpathlineto{\pgfqpoint{5.181277in}{2.341496in}}%
\pgfpathlineto{\pgfqpoint{5.194626in}{2.340106in}}%
\pgfpathlineto{\pgfqpoint{5.207982in}{2.338741in}}%
\pgfpathlineto{\pgfqpoint{5.200882in}{2.332290in}}%
\pgfpathlineto{\pgfqpoint{5.193775in}{2.325784in}}%
\pgfpathlineto{\pgfqpoint{5.186661in}{2.319221in}}%
\pgfpathlineto{\pgfqpoint{5.179540in}{2.312600in}}%
\pgfpathlineto{\pgfqpoint{5.166170in}{2.313927in}}%
\pgfpathlineto{\pgfqpoint{5.152807in}{2.315278in}}%
\pgfpathlineto{\pgfqpoint{5.139451in}{2.316653in}}%
\pgfpathlineto{\pgfqpoint{5.126104in}{2.318053in}}%
\pgfpathlineto{\pgfqpoint{5.133238in}{2.324707in}}%
\pgfpathlineto{\pgfqpoint{5.140366in}{2.331306in}}%
\pgfpathlineto{\pgfqpoint{5.147487in}{2.337852in}}%
\pgfpathlineto{\pgfqpoint{5.154602in}{2.344346in}}%
\pgfpathclose%
\pgfusepath{fill}%
\end{pgfscope}%
\begin{pgfscope}%
\pgfpathrectangle{\pgfqpoint{1.254980in}{0.150000in}}{\pgfqpoint{5.490039in}{5.490039in}}%
\pgfusepath{clip}%
\pgfsetbuttcap%
\pgfsetroundjoin%
\definecolor{currentfill}{rgb}{0.267004,0.004874,0.329415}%
\pgfsetfillcolor{currentfill}%
\pgfsetfillopacity{0.700000}%
\pgfsetlinewidth{0.000000pt}%
\definecolor{currentstroke}{rgb}{0.000000,0.000000,0.000000}%
\pgfsetstrokecolor{currentstroke}%
\pgfsetdash{}{0pt}%
\pgfpathmoveto{\pgfqpoint{3.637561in}{2.154737in}}%
\pgfpathlineto{\pgfqpoint{3.650512in}{2.150439in}}%
\pgfpathlineto{\pgfqpoint{3.663469in}{2.146172in}}%
\pgfpathlineto{\pgfqpoint{3.676431in}{2.141935in}}%
\pgfpathlineto{\pgfqpoint{3.689399in}{2.137727in}}%
\pgfpathlineto{\pgfqpoint{3.681712in}{2.131081in}}%
\pgfpathlineto{\pgfqpoint{3.674018in}{2.124493in}}%
\pgfpathlineto{\pgfqpoint{3.666318in}{2.117967in}}%
\pgfpathlineto{\pgfqpoint{3.658611in}{2.111506in}}%
\pgfpathlineto{\pgfqpoint{3.645629in}{2.115856in}}%
\pgfpathlineto{\pgfqpoint{3.632652in}{2.120236in}}%
\pgfpathlineto{\pgfqpoint{3.619681in}{2.124645in}}%
\pgfpathlineto{\pgfqpoint{3.606714in}{2.129085in}}%
\pgfpathlineto{\pgfqpoint{3.614436in}{2.135399in}}%
\pgfpathlineto{\pgfqpoint{3.622151in}{2.141781in}}%
\pgfpathlineto{\pgfqpoint{3.629859in}{2.148228in}}%
\pgfpathlineto{\pgfqpoint{3.637561in}{2.154737in}}%
\pgfpathclose%
\pgfusepath{fill}%
\end{pgfscope}%
\begin{pgfscope}%
\pgfpathrectangle{\pgfqpoint{1.254980in}{0.150000in}}{\pgfqpoint{5.490039in}{5.490039in}}%
\pgfusepath{clip}%
\pgfsetbuttcap%
\pgfsetroundjoin%
\definecolor{currentfill}{rgb}{0.278826,0.175490,0.483397}%
\pgfsetfillcolor{currentfill}%
\pgfsetfillopacity{0.700000}%
\pgfsetlinewidth{0.000000pt}%
\definecolor{currentstroke}{rgb}{0.000000,0.000000,0.000000}%
\pgfsetstrokecolor{currentstroke}%
\pgfsetdash{}{0pt}%
\pgfpathmoveto{\pgfqpoint{6.021586in}{2.459650in}}%
\pgfpathlineto{\pgfqpoint{6.035166in}{2.458566in}}%
\pgfpathlineto{\pgfqpoint{6.048754in}{2.457504in}}%
\pgfpathlineto{\pgfqpoint{6.062350in}{2.456465in}}%
\pgfpathlineto{\pgfqpoint{6.055662in}{2.452122in}}%
\pgfpathlineto{\pgfqpoint{6.048968in}{2.447783in}}%
\pgfpathlineto{\pgfqpoint{6.042269in}{2.443446in}}%
\pgfpathlineto{\pgfqpoint{6.035564in}{2.439106in}}%
\pgfpathlineto{\pgfqpoint{6.021946in}{2.440002in}}%
\pgfpathlineto{\pgfqpoint{6.008337in}{2.440921in}}%
\pgfpathlineto{\pgfqpoint{5.994736in}{2.441863in}}%
\pgfpathlineto{\pgfqpoint{6.001457in}{2.446307in}}%
\pgfpathlineto{\pgfqpoint{6.008172in}{2.450750in}}%
\pgfpathlineto{\pgfqpoint{6.014882in}{2.455196in}}%
\pgfpathlineto{\pgfqpoint{6.021586in}{2.459650in}}%
\pgfpathclose%
\pgfusepath{fill}%
\end{pgfscope}%
\begin{pgfscope}%
\pgfpathrectangle{\pgfqpoint{1.254980in}{0.150000in}}{\pgfqpoint{5.490039in}{5.490039in}}%
\pgfusepath{clip}%
\pgfsetbuttcap%
\pgfsetroundjoin%
\definecolor{currentfill}{rgb}{0.278791,0.062145,0.386592}%
\pgfsetfillcolor{currentfill}%
\pgfsetfillopacity{0.700000}%
\pgfsetlinewidth{0.000000pt}%
\definecolor{currentstroke}{rgb}{0.000000,0.000000,0.000000}%
\pgfsetstrokecolor{currentstroke}%
\pgfsetdash{}{0pt}%
\pgfpathmoveto{\pgfqpoint{4.638982in}{2.251695in}}%
\pgfpathlineto{\pgfqpoint{4.652173in}{2.249630in}}%
\pgfpathlineto{\pgfqpoint{4.665370in}{2.247591in}}%
\pgfpathlineto{\pgfqpoint{4.678574in}{2.245576in}}%
\pgfpathlineto{\pgfqpoint{4.691786in}{2.243586in}}%
\pgfpathlineto{\pgfqpoint{4.684471in}{2.235986in}}%
\pgfpathlineto{\pgfqpoint{4.677151in}{2.228338in}}%
\pgfpathlineto{\pgfqpoint{4.669825in}{2.220643in}}%
\pgfpathlineto{\pgfqpoint{4.662492in}{2.212900in}}%
\pgfpathlineto{\pgfqpoint{4.649269in}{2.214917in}}%
\pgfpathlineto{\pgfqpoint{4.636053in}{2.216959in}}%
\pgfpathlineto{\pgfqpoint{4.622844in}{2.219026in}}%
\pgfpathlineto{\pgfqpoint{4.609643in}{2.221118in}}%
\pgfpathlineto{\pgfqpoint{4.616986in}{2.228829in}}%
\pgfpathlineto{\pgfqpoint{4.624324in}{2.236495in}}%
\pgfpathlineto{\pgfqpoint{4.631656in}{2.244117in}}%
\pgfpathlineto{\pgfqpoint{4.638982in}{2.251695in}}%
\pgfpathclose%
\pgfusepath{fill}%
\end{pgfscope}%
\begin{pgfscope}%
\pgfpathrectangle{\pgfqpoint{1.254980in}{0.150000in}}{\pgfqpoint{5.490039in}{5.490039in}}%
\pgfusepath{clip}%
\pgfsetbuttcap%
\pgfsetroundjoin%
\definecolor{currentfill}{rgb}{0.273809,0.031497,0.358853}%
\pgfsetfillcolor{currentfill}%
\pgfsetfillopacity{0.700000}%
\pgfsetlinewidth{0.000000pt}%
\definecolor{currentstroke}{rgb}{0.000000,0.000000,0.000000}%
\pgfsetstrokecolor{currentstroke}%
\pgfsetdash{}{0pt}%
\pgfpathmoveto{\pgfqpoint{3.234165in}{2.205391in}}%
\pgfpathlineto{\pgfqpoint{3.247052in}{2.199843in}}%
\pgfpathlineto{\pgfqpoint{3.259943in}{2.194330in}}%
\pgfpathlineto{\pgfqpoint{3.272839in}{2.188852in}}%
\pgfpathlineto{\pgfqpoint{3.285738in}{2.183409in}}%
\pgfpathlineto{\pgfqpoint{3.277865in}{2.178764in}}%
\pgfpathlineto{\pgfqpoint{3.269983in}{2.174244in}}%
\pgfpathlineto{\pgfqpoint{3.262092in}{2.169853in}}%
\pgfpathlineto{\pgfqpoint{3.254192in}{2.165595in}}%
\pgfpathlineto{\pgfqpoint{3.241273in}{2.171219in}}%
\pgfpathlineto{\pgfqpoint{3.228359in}{2.176878in}}%
\pgfpathlineto{\pgfqpoint{3.215449in}{2.182572in}}%
\pgfpathlineto{\pgfqpoint{3.202543in}{2.188301in}}%
\pgfpathlineto{\pgfqpoint{3.210462in}{2.192374in}}%
\pgfpathlineto{\pgfqpoint{3.218372in}{2.196582in}}%
\pgfpathlineto{\pgfqpoint{3.226273in}{2.200923in}}%
\pgfpathlineto{\pgfqpoint{3.234165in}{2.205391in}}%
\pgfpathclose%
\pgfusepath{fill}%
\end{pgfscope}%
\begin{pgfscope}%
\pgfpathrectangle{\pgfqpoint{1.254980in}{0.150000in}}{\pgfqpoint{5.490039in}{5.490039in}}%
\pgfusepath{clip}%
\pgfsetbuttcap%
\pgfsetroundjoin%
\definecolor{currentfill}{rgb}{0.282884,0.135920,0.453427}%
\pgfsetfillcolor{currentfill}%
\pgfsetfillopacity{0.700000}%
\pgfsetlinewidth{0.000000pt}%
\definecolor{currentstroke}{rgb}{0.000000,0.000000,0.000000}%
\pgfsetstrokecolor{currentstroke}%
\pgfsetdash{}{0pt}%
\pgfpathmoveto{\pgfqpoint{5.371450in}{2.377607in}}%
\pgfpathlineto{\pgfqpoint{5.384847in}{2.376341in}}%
\pgfpathlineto{\pgfqpoint{5.398253in}{2.375099in}}%
\pgfpathlineto{\pgfqpoint{5.411666in}{2.373880in}}%
\pgfpathlineto{\pgfqpoint{5.425087in}{2.372685in}}%
\pgfpathlineto{\pgfqpoint{5.418085in}{2.366813in}}%
\pgfpathlineto{\pgfqpoint{5.411076in}{2.360892in}}%
\pgfpathlineto{\pgfqpoint{5.404060in}{2.354920in}}%
\pgfpathlineto{\pgfqpoint{5.397037in}{2.348895in}}%
\pgfpathlineto{\pgfqpoint{5.383600in}{2.350025in}}%
\pgfpathlineto{\pgfqpoint{5.370172in}{2.351180in}}%
\pgfpathlineto{\pgfqpoint{5.356751in}{2.352358in}}%
\pgfpathlineto{\pgfqpoint{5.343338in}{2.353559in}}%
\pgfpathlineto{\pgfqpoint{5.350376in}{2.359644in}}%
\pgfpathlineto{\pgfqpoint{5.357407in}{2.365679in}}%
\pgfpathlineto{\pgfqpoint{5.364432in}{2.371666in}}%
\pgfpathlineto{\pgfqpoint{5.371450in}{2.377607in}}%
\pgfpathclose%
\pgfusepath{fill}%
\end{pgfscope}%
\begin{pgfscope}%
\pgfpathrectangle{\pgfqpoint{1.254980in}{0.150000in}}{\pgfqpoint{5.490039in}{5.490039in}}%
\pgfusepath{clip}%
\pgfsetbuttcap%
\pgfsetroundjoin%
\definecolor{currentfill}{rgb}{0.281924,0.089666,0.412415}%
\pgfsetfillcolor{currentfill}%
\pgfsetfillopacity{0.700000}%
\pgfsetlinewidth{0.000000pt}%
\definecolor{currentstroke}{rgb}{0.000000,0.000000,0.000000}%
\pgfsetstrokecolor{currentstroke}%
\pgfsetdash{}{0pt}%
\pgfpathmoveto{\pgfqpoint{2.913035in}{2.299784in}}%
\pgfpathlineto{\pgfqpoint{2.925889in}{2.293094in}}%
\pgfpathlineto{\pgfqpoint{2.938745in}{2.286445in}}%
\pgfpathlineto{\pgfqpoint{2.951605in}{2.279837in}}%
\pgfpathlineto{\pgfqpoint{2.964468in}{2.273270in}}%
\pgfpathlineto{\pgfqpoint{2.956413in}{2.270762in}}%
\pgfpathlineto{\pgfqpoint{2.948346in}{2.268433in}}%
\pgfpathlineto{\pgfqpoint{2.940267in}{2.266288in}}%
\pgfpathlineto{\pgfqpoint{2.932176in}{2.264333in}}%
\pgfpathlineto{\pgfqpoint{2.919290in}{2.271108in}}%
\pgfpathlineto{\pgfqpoint{2.906407in}{2.277923in}}%
\pgfpathlineto{\pgfqpoint{2.893526in}{2.284780in}}%
\pgfpathlineto{\pgfqpoint{2.880650in}{2.291679in}}%
\pgfpathlineto{\pgfqpoint{2.888764in}{2.293421in}}%
\pgfpathlineto{\pgfqpoint{2.896867in}{2.295356in}}%
\pgfpathlineto{\pgfqpoint{2.904957in}{2.297479in}}%
\pgfpathlineto{\pgfqpoint{2.913035in}{2.299784in}}%
\pgfpathclose%
\pgfusepath{fill}%
\end{pgfscope}%
\begin{pgfscope}%
\pgfpathrectangle{\pgfqpoint{1.254980in}{0.150000in}}{\pgfqpoint{5.490039in}{5.490039in}}%
\pgfusepath{clip}%
\pgfsetbuttcap%
\pgfsetroundjoin%
\definecolor{currentfill}{rgb}{0.281887,0.150881,0.465405}%
\pgfsetfillcolor{currentfill}%
\pgfsetfillopacity{0.700000}%
\pgfsetlinewidth{0.000000pt}%
\definecolor{currentstroke}{rgb}{0.000000,0.000000,0.000000}%
\pgfsetstrokecolor{currentstroke}%
\pgfsetdash{}{0pt}%
\pgfpathmoveto{\pgfqpoint{5.588263in}{2.408167in}}%
\pgfpathlineto{\pgfqpoint{5.601723in}{2.407016in}}%
\pgfpathlineto{\pgfqpoint{5.615191in}{2.405889in}}%
\pgfpathlineto{\pgfqpoint{5.628668in}{2.404786in}}%
\pgfpathlineto{\pgfqpoint{5.642152in}{2.403706in}}%
\pgfpathlineto{\pgfqpoint{5.635253in}{2.398415in}}%
\pgfpathlineto{\pgfqpoint{5.628348in}{2.393087in}}%
\pgfpathlineto{\pgfqpoint{5.621436in}{2.387720in}}%
\pgfpathlineto{\pgfqpoint{5.614517in}{2.382309in}}%
\pgfpathlineto{\pgfqpoint{5.601015in}{2.383299in}}%
\pgfpathlineto{\pgfqpoint{5.587521in}{2.384312in}}%
\pgfpathlineto{\pgfqpoint{5.574036in}{2.385349in}}%
\pgfpathlineto{\pgfqpoint{5.560558in}{2.386409in}}%
\pgfpathlineto{\pgfqpoint{5.567494in}{2.391905in}}%
\pgfpathlineto{\pgfqpoint{5.574424in}{2.397361in}}%
\pgfpathlineto{\pgfqpoint{5.581346in}{2.402781in}}%
\pgfpathlineto{\pgfqpoint{5.588263in}{2.408167in}}%
\pgfpathclose%
\pgfusepath{fill}%
\end{pgfscope}%
\begin{pgfscope}%
\pgfpathrectangle{\pgfqpoint{1.254980in}{0.150000in}}{\pgfqpoint{5.490039in}{5.490039in}}%
\pgfusepath{clip}%
\pgfsetbuttcap%
\pgfsetroundjoin%
\definecolor{currentfill}{rgb}{0.267004,0.004874,0.329415}%
\pgfsetfillcolor{currentfill}%
\pgfsetfillopacity{0.700000}%
\pgfsetlinewidth{0.000000pt}%
\definecolor{currentstroke}{rgb}{0.000000,0.000000,0.000000}%
\pgfsetstrokecolor{currentstroke}%
\pgfsetdash{}{0pt}%
\pgfpathmoveto{\pgfqpoint{3.771952in}{2.148826in}}%
\pgfpathlineto{\pgfqpoint{3.784933in}{2.144896in}}%
\pgfpathlineto{\pgfqpoint{3.797919in}{2.140996in}}%
\pgfpathlineto{\pgfqpoint{3.810911in}{2.137124in}}%
\pgfpathlineto{\pgfqpoint{3.823909in}{2.133282in}}%
\pgfpathlineto{\pgfqpoint{3.816274in}{2.126182in}}%
\pgfpathlineto{\pgfqpoint{3.808633in}{2.119122in}}%
\pgfpathlineto{\pgfqpoint{3.800986in}{2.112104in}}%
\pgfpathlineto{\pgfqpoint{3.793333in}{2.105133in}}%
\pgfpathlineto{\pgfqpoint{3.780322in}{2.109106in}}%
\pgfpathlineto{\pgfqpoint{3.767317in}{2.113107in}}%
\pgfpathlineto{\pgfqpoint{3.754317in}{2.117137in}}%
\pgfpathlineto{\pgfqpoint{3.741322in}{2.121196in}}%
\pgfpathlineto{\pgfqpoint{3.748989in}{2.128033in}}%
\pgfpathlineto{\pgfqpoint{3.756650in}{2.134919in}}%
\pgfpathlineto{\pgfqpoint{3.764304in}{2.141851in}}%
\pgfpathlineto{\pgfqpoint{3.771952in}{2.148826in}}%
\pgfpathclose%
\pgfusepath{fill}%
\end{pgfscope}%
\begin{pgfscope}%
\pgfpathrectangle{\pgfqpoint{1.254980in}{0.150000in}}{\pgfqpoint{5.490039in}{5.490039in}}%
\pgfusepath{clip}%
\pgfsetbuttcap%
\pgfsetroundjoin%
\definecolor{currentfill}{rgb}{0.280255,0.165693,0.476498}%
\pgfsetfillcolor{currentfill}%
\pgfsetfillopacity{0.700000}%
\pgfsetlinewidth{0.000000pt}%
\definecolor{currentstroke}{rgb}{0.000000,0.000000,0.000000}%
\pgfsetstrokecolor{currentstroke}%
\pgfsetdash{}{0pt}%
\pgfpathmoveto{\pgfqpoint{5.804990in}{2.435548in}}%
\pgfpathlineto{\pgfqpoint{5.818511in}{2.434458in}}%
\pgfpathlineto{\pgfqpoint{5.832040in}{2.433391in}}%
\pgfpathlineto{\pgfqpoint{5.845578in}{2.432347in}}%
\pgfpathlineto{\pgfqpoint{5.859124in}{2.431326in}}%
\pgfpathlineto{\pgfqpoint{5.852332in}{2.426572in}}%
\pgfpathlineto{\pgfqpoint{5.845535in}{2.421800in}}%
\pgfpathlineto{\pgfqpoint{5.838731in}{2.417005in}}%
\pgfpathlineto{\pgfqpoint{5.831920in}{2.412183in}}%
\pgfpathlineto{\pgfqpoint{5.818355in}{2.413087in}}%
\pgfpathlineto{\pgfqpoint{5.804798in}{2.414015in}}%
\pgfpathlineto{\pgfqpoint{5.791249in}{2.414965in}}%
\pgfpathlineto{\pgfqpoint{5.777709in}{2.415939in}}%
\pgfpathlineto{\pgfqpoint{5.784538in}{2.420872in}}%
\pgfpathlineto{\pgfqpoint{5.791362in}{2.425782in}}%
\pgfpathlineto{\pgfqpoint{5.798179in}{2.430673in}}%
\pgfpathlineto{\pgfqpoint{5.804990in}{2.435548in}}%
\pgfpathclose%
\pgfusepath{fill}%
\end{pgfscope}%
\begin{pgfscope}%
\pgfpathrectangle{\pgfqpoint{1.254980in}{0.150000in}}{\pgfqpoint{5.490039in}{5.490039in}}%
\pgfusepath{clip}%
\pgfsetbuttcap%
\pgfsetroundjoin%
\definecolor{currentfill}{rgb}{0.282884,0.135920,0.453427}%
\pgfsetfillcolor{currentfill}%
\pgfsetfillopacity{0.700000}%
\pgfsetlinewidth{0.000000pt}%
\definecolor{currentstroke}{rgb}{0.000000,0.000000,0.000000}%
\pgfsetstrokecolor{currentstroke}%
\pgfsetdash{}{0pt}%
\pgfpathmoveto{\pgfqpoint{2.726356in}{2.377840in}}%
\pgfpathlineto{\pgfqpoint{2.739198in}{2.370412in}}%
\pgfpathlineto{\pgfqpoint{2.752044in}{2.363031in}}%
\pgfpathlineto{\pgfqpoint{2.764892in}{2.355695in}}%
\pgfpathlineto{\pgfqpoint{2.777742in}{2.348406in}}%
\pgfpathlineto{\pgfqpoint{2.769565in}{2.347295in}}%
\pgfpathlineto{\pgfqpoint{2.761374in}{2.346395in}}%
\pgfpathlineto{\pgfqpoint{2.753170in}{2.345711in}}%
\pgfpathlineto{\pgfqpoint{2.744951in}{2.345249in}}%
\pgfpathlineto{\pgfqpoint{2.732074in}{2.352761in}}%
\pgfpathlineto{\pgfqpoint{2.719200in}{2.360318in}}%
\pgfpathlineto{\pgfqpoint{2.706329in}{2.367922in}}%
\pgfpathlineto{\pgfqpoint{2.693460in}{2.375572in}}%
\pgfpathlineto{\pgfqpoint{2.701705in}{2.375807in}}%
\pgfpathlineto{\pgfqpoint{2.709936in}{2.376267in}}%
\pgfpathlineto{\pgfqpoint{2.718153in}{2.376946in}}%
\pgfpathlineto{\pgfqpoint{2.726356in}{2.377840in}}%
\pgfpathclose%
\pgfusepath{fill}%
\end{pgfscope}%
\begin{pgfscope}%
\pgfpathrectangle{\pgfqpoint{1.254980in}{0.150000in}}{\pgfqpoint{5.490039in}{5.490039in}}%
\pgfusepath{clip}%
\pgfsetbuttcap%
\pgfsetroundjoin%
\definecolor{currentfill}{rgb}{0.281446,0.084320,0.407414}%
\pgfsetfillcolor{currentfill}%
\pgfsetfillopacity{0.700000}%
\pgfsetlinewidth{0.000000pt}%
\definecolor{currentstroke}{rgb}{0.000000,0.000000,0.000000}%
\pgfsetstrokecolor{currentstroke}%
\pgfsetdash{}{0pt}%
\pgfpathmoveto{\pgfqpoint{4.855852in}{2.287784in}}%
\pgfpathlineto{\pgfqpoint{4.869105in}{2.286033in}}%
\pgfpathlineto{\pgfqpoint{4.882365in}{2.284306in}}%
\pgfpathlineto{\pgfqpoint{4.895632in}{2.282603in}}%
\pgfpathlineto{\pgfqpoint{4.908907in}{2.280926in}}%
\pgfpathlineto{\pgfqpoint{4.901677in}{2.273712in}}%
\pgfpathlineto{\pgfqpoint{4.894440in}{2.266443in}}%
\pgfpathlineto{\pgfqpoint{4.887197in}{2.259119in}}%
\pgfpathlineto{\pgfqpoint{4.879948in}{2.251737in}}%
\pgfpathlineto{\pgfqpoint{4.866661in}{2.253416in}}%
\pgfpathlineto{\pgfqpoint{4.853381in}{2.255120in}}%
\pgfpathlineto{\pgfqpoint{4.840109in}{2.256848in}}%
\pgfpathlineto{\pgfqpoint{4.826844in}{2.258601in}}%
\pgfpathlineto{\pgfqpoint{4.834105in}{2.265976in}}%
\pgfpathlineto{\pgfqpoint{4.841360in}{2.273298in}}%
\pgfpathlineto{\pgfqpoint{4.848609in}{2.280567in}}%
\pgfpathlineto{\pgfqpoint{4.855852in}{2.287784in}}%
\pgfpathclose%
\pgfusepath{fill}%
\end{pgfscope}%
\begin{pgfscope}%
\pgfpathrectangle{\pgfqpoint{1.254980in}{0.150000in}}{\pgfqpoint{5.490039in}{5.490039in}}%
\pgfusepath{clip}%
\pgfsetbuttcap%
\pgfsetroundjoin%
\definecolor{currentfill}{rgb}{0.269944,0.014625,0.341379}%
\pgfsetfillcolor{currentfill}%
\pgfsetfillopacity{0.700000}%
\pgfsetlinewidth{0.000000pt}%
\definecolor{currentstroke}{rgb}{0.000000,0.000000,0.000000}%
\pgfsetstrokecolor{currentstroke}%
\pgfsetdash{}{0pt}%
\pgfpathmoveto{\pgfqpoint{4.123222in}{2.167982in}}%
\pgfpathlineto{\pgfqpoint{4.136283in}{2.164925in}}%
\pgfpathlineto{\pgfqpoint{4.149350in}{2.161896in}}%
\pgfpathlineto{\pgfqpoint{4.162424in}{2.158893in}}%
\pgfpathlineto{\pgfqpoint{4.175504in}{2.155917in}}%
\pgfpathlineto{\pgfqpoint{4.167998in}{2.148106in}}%
\pgfpathlineto{\pgfqpoint{4.160487in}{2.140289in}}%
\pgfpathlineto{\pgfqpoint{4.152971in}{2.132469in}}%
\pgfpathlineto{\pgfqpoint{4.145449in}{2.124649in}}%
\pgfpathlineto{\pgfqpoint{4.132358in}{2.127716in}}%
\pgfpathlineto{\pgfqpoint{4.119272in}{2.130810in}}%
\pgfpathlineto{\pgfqpoint{4.106193in}{2.133931in}}%
\pgfpathlineto{\pgfqpoint{4.093120in}{2.137079in}}%
\pgfpathlineto{\pgfqpoint{4.100654in}{2.144803in}}%
\pgfpathlineto{\pgfqpoint{4.108182in}{2.152530in}}%
\pgfpathlineto{\pgfqpoint{4.115705in}{2.160257in}}%
\pgfpathlineto{\pgfqpoint{4.123222in}{2.167982in}}%
\pgfpathclose%
\pgfusepath{fill}%
\end{pgfscope}%
\begin{pgfscope}%
\pgfpathrectangle{\pgfqpoint{1.254980in}{0.150000in}}{\pgfqpoint{5.490039in}{5.490039in}}%
\pgfusepath{clip}%
\pgfsetbuttcap%
\pgfsetroundjoin%
\definecolor{currentfill}{rgb}{0.273809,0.031497,0.358853}%
\pgfsetfillcolor{currentfill}%
\pgfsetfillopacity{0.700000}%
\pgfsetlinewidth{0.000000pt}%
\definecolor{currentstroke}{rgb}{0.000000,0.000000,0.000000}%
\pgfsetstrokecolor{currentstroke}%
\pgfsetdash{}{0pt}%
\pgfpathmoveto{\pgfqpoint{4.340038in}{2.196333in}}%
\pgfpathlineto{\pgfqpoint{4.353153in}{2.193739in}}%
\pgfpathlineto{\pgfqpoint{4.366275in}{2.191171in}}%
\pgfpathlineto{\pgfqpoint{4.379403in}{2.188629in}}%
\pgfpathlineto{\pgfqpoint{4.392538in}{2.186112in}}%
\pgfpathlineto{\pgfqpoint{4.385110in}{2.178216in}}%
\pgfpathlineto{\pgfqpoint{4.377677in}{2.170293in}}%
\pgfpathlineto{\pgfqpoint{4.370238in}{2.162344in}}%
\pgfpathlineto{\pgfqpoint{4.362794in}{2.154372in}}%
\pgfpathlineto{\pgfqpoint{4.349648in}{2.156954in}}%
\pgfpathlineto{\pgfqpoint{4.336508in}{2.159561in}}%
\pgfpathlineto{\pgfqpoint{4.323375in}{2.162195in}}%
\pgfpathlineto{\pgfqpoint{4.310249in}{2.164855in}}%
\pgfpathlineto{\pgfqpoint{4.317704in}{2.172757in}}%
\pgfpathlineto{\pgfqpoint{4.325154in}{2.180639in}}%
\pgfpathlineto{\pgfqpoint{4.332599in}{2.188497in}}%
\pgfpathlineto{\pgfqpoint{4.340038in}{2.196333in}}%
\pgfpathclose%
\pgfusepath{fill}%
\end{pgfscope}%
\begin{pgfscope}%
\pgfpathrectangle{\pgfqpoint{1.254980in}{0.150000in}}{\pgfqpoint{5.490039in}{5.490039in}}%
\pgfusepath{clip}%
\pgfsetbuttcap%
\pgfsetroundjoin%
\definecolor{currentfill}{rgb}{0.277941,0.056324,0.381191}%
\pgfsetfillcolor{currentfill}%
\pgfsetfillopacity{0.700000}%
\pgfsetlinewidth{0.000000pt}%
\definecolor{currentstroke}{rgb}{0.000000,0.000000,0.000000}%
\pgfsetstrokecolor{currentstroke}%
\pgfsetdash{}{0pt}%
\pgfpathmoveto{\pgfqpoint{3.099439in}{2.235437in}}%
\pgfpathlineto{\pgfqpoint{3.112313in}{2.229416in}}%
\pgfpathlineto{\pgfqpoint{3.125191in}{2.223432in}}%
\pgfpathlineto{\pgfqpoint{3.138073in}{2.217486in}}%
\pgfpathlineto{\pgfqpoint{3.150959in}{2.211576in}}%
\pgfpathlineto{\pgfqpoint{3.143010in}{2.207834in}}%
\pgfpathlineto{\pgfqpoint{3.135051in}{2.204242in}}%
\pgfpathlineto{\pgfqpoint{3.127083in}{2.200803in}}%
\pgfpathlineto{\pgfqpoint{3.119104in}{2.197523in}}%
\pgfpathlineto{\pgfqpoint{3.106197in}{2.203627in}}%
\pgfpathlineto{\pgfqpoint{3.093294in}{2.209767in}}%
\pgfpathlineto{\pgfqpoint{3.080395in}{2.215945in}}%
\pgfpathlineto{\pgfqpoint{3.067500in}{2.222161in}}%
\pgfpathlineto{\pgfqpoint{3.075500in}{2.225241in}}%
\pgfpathlineto{\pgfqpoint{3.083490in}{2.228484in}}%
\pgfpathlineto{\pgfqpoint{3.091469in}{2.231884in}}%
\pgfpathlineto{\pgfqpoint{3.099439in}{2.235437in}}%
\pgfpathclose%
\pgfusepath{fill}%
\end{pgfscope}%
\begin{pgfscope}%
\pgfpathrectangle{\pgfqpoint{1.254980in}{0.150000in}}{\pgfqpoint{5.490039in}{5.490039in}}%
\pgfusepath{clip}%
\pgfsetbuttcap%
\pgfsetroundjoin%
\definecolor{currentfill}{rgb}{0.283091,0.110553,0.431554}%
\pgfsetfillcolor{currentfill}%
\pgfsetfillopacity{0.700000}%
\pgfsetlinewidth{0.000000pt}%
\definecolor{currentstroke}{rgb}{0.000000,0.000000,0.000000}%
\pgfsetstrokecolor{currentstroke}%
\pgfsetdash{}{0pt}%
\pgfpathmoveto{\pgfqpoint{5.072791in}{2.323891in}}%
\pgfpathlineto{\pgfqpoint{5.086108in}{2.322395in}}%
\pgfpathlineto{\pgfqpoint{5.099432in}{2.320924in}}%
\pgfpathlineto{\pgfqpoint{5.112764in}{2.319476in}}%
\pgfpathlineto{\pgfqpoint{5.126104in}{2.318053in}}%
\pgfpathlineto{\pgfqpoint{5.118963in}{2.311342in}}%
\pgfpathlineto{\pgfqpoint{5.111816in}{2.304574in}}%
\pgfpathlineto{\pgfqpoint{5.104662in}{2.297747in}}%
\pgfpathlineto{\pgfqpoint{5.097502in}{2.290859in}}%
\pgfpathlineto{\pgfqpoint{5.084149in}{2.292258in}}%
\pgfpathlineto{\pgfqpoint{5.070803in}{2.293680in}}%
\pgfpathlineto{\pgfqpoint{5.057466in}{2.295127in}}%
\pgfpathlineto{\pgfqpoint{5.044135in}{2.296598in}}%
\pgfpathlineto{\pgfqpoint{5.051309in}{2.303505in}}%
\pgfpathlineto{\pgfqpoint{5.058476in}{2.310356in}}%
\pgfpathlineto{\pgfqpoint{5.065637in}{2.317151in}}%
\pgfpathlineto{\pgfqpoint{5.072791in}{2.323891in}}%
\pgfpathclose%
\pgfusepath{fill}%
\end{pgfscope}%
\begin{pgfscope}%
\pgfpathrectangle{\pgfqpoint{1.254980in}{0.150000in}}{\pgfqpoint{5.490039in}{5.490039in}}%
\pgfusepath{clip}%
\pgfsetbuttcap%
\pgfsetroundjoin%
\definecolor{currentfill}{rgb}{0.267004,0.004874,0.329415}%
\pgfsetfillcolor{currentfill}%
\pgfsetfillopacity{0.700000}%
\pgfsetlinewidth{0.000000pt}%
\definecolor{currentstroke}{rgb}{0.000000,0.000000,0.000000}%
\pgfsetstrokecolor{currentstroke}%
\pgfsetdash{}{0pt}%
\pgfpathmoveto{\pgfqpoint{3.906384in}{2.147408in}}%
\pgfpathlineto{\pgfqpoint{3.919397in}{2.143825in}}%
\pgfpathlineto{\pgfqpoint{3.932416in}{2.140270in}}%
\pgfpathlineto{\pgfqpoint{3.945440in}{2.136743in}}%
\pgfpathlineto{\pgfqpoint{3.958471in}{2.133245in}}%
\pgfpathlineto{\pgfqpoint{3.950885in}{2.125789in}}%
\pgfpathlineto{\pgfqpoint{3.943294in}{2.118355in}}%
\pgfpathlineto{\pgfqpoint{3.935696in}{2.110947in}}%
\pgfpathlineto{\pgfqpoint{3.928094in}{2.103567in}}%
\pgfpathlineto{\pgfqpoint{3.915050in}{2.107182in}}%
\pgfpathlineto{\pgfqpoint{3.902013in}{2.110826in}}%
\pgfpathlineto{\pgfqpoint{3.888981in}{2.114498in}}%
\pgfpathlineto{\pgfqpoint{3.875955in}{2.118198in}}%
\pgfpathlineto{\pgfqpoint{3.883571in}{2.125457in}}%
\pgfpathlineto{\pgfqpoint{3.891181in}{2.132746in}}%
\pgfpathlineto{\pgfqpoint{3.898785in}{2.140065in}}%
\pgfpathlineto{\pgfqpoint{3.906384in}{2.147408in}}%
\pgfpathclose%
\pgfusepath{fill}%
\end{pgfscope}%
\begin{pgfscope}%
\pgfpathrectangle{\pgfqpoint{1.254980in}{0.150000in}}{\pgfqpoint{5.490039in}{5.490039in}}%
\pgfusepath{clip}%
\pgfsetbuttcap%
\pgfsetroundjoin%
\definecolor{currentfill}{rgb}{0.277941,0.056324,0.381191}%
\pgfsetfillcolor{currentfill}%
\pgfsetfillopacity{0.700000}%
\pgfsetlinewidth{0.000000pt}%
\definecolor{currentstroke}{rgb}{0.000000,0.000000,0.000000}%
\pgfsetstrokecolor{currentstroke}%
\pgfsetdash{}{0pt}%
\pgfpathmoveto{\pgfqpoint{4.556905in}{2.229738in}}%
\pgfpathlineto{\pgfqpoint{4.570079in}{2.227545in}}%
\pgfpathlineto{\pgfqpoint{4.583260in}{2.225378in}}%
\pgfpathlineto{\pgfqpoint{4.596448in}{2.223235in}}%
\pgfpathlineto{\pgfqpoint{4.609643in}{2.221118in}}%
\pgfpathlineto{\pgfqpoint{4.602293in}{2.213364in}}%
\pgfpathlineto{\pgfqpoint{4.594938in}{2.205567in}}%
\pgfpathlineto{\pgfqpoint{4.587578in}{2.197728in}}%
\pgfpathlineto{\pgfqpoint{4.580211in}{2.189846in}}%
\pgfpathlineto{\pgfqpoint{4.567005in}{2.192003in}}%
\pgfpathlineto{\pgfqpoint{4.553806in}{2.194186in}}%
\pgfpathlineto{\pgfqpoint{4.540614in}{2.196394in}}%
\pgfpathlineto{\pgfqpoint{4.527429in}{2.198627in}}%
\pgfpathlineto{\pgfqpoint{4.534806in}{2.206463in}}%
\pgfpathlineto{\pgfqpoint{4.542178in}{2.214261in}}%
\pgfpathlineto{\pgfqpoint{4.549545in}{2.222020in}}%
\pgfpathlineto{\pgfqpoint{4.556905in}{2.229738in}}%
\pgfpathclose%
\pgfusepath{fill}%
\end{pgfscope}%
\begin{pgfscope}%
\pgfpathrectangle{\pgfqpoint{1.254980in}{0.150000in}}{\pgfqpoint{5.490039in}{5.490039in}}%
\pgfusepath{clip}%
\pgfsetbuttcap%
\pgfsetroundjoin%
\definecolor{currentfill}{rgb}{0.283072,0.130895,0.449241}%
\pgfsetfillcolor{currentfill}%
\pgfsetfillopacity{0.700000}%
\pgfsetlinewidth{0.000000pt}%
\definecolor{currentstroke}{rgb}{0.000000,0.000000,0.000000}%
\pgfsetstrokecolor{currentstroke}%
\pgfsetdash{}{0pt}%
\pgfpathmoveto{\pgfqpoint{5.289766in}{2.358604in}}%
\pgfpathlineto{\pgfqpoint{5.303147in}{2.357307in}}%
\pgfpathlineto{\pgfqpoint{5.316536in}{2.356034in}}%
\pgfpathlineto{\pgfqpoint{5.329933in}{2.354785in}}%
\pgfpathlineto{\pgfqpoint{5.343338in}{2.353559in}}%
\pgfpathlineto{\pgfqpoint{5.336293in}{2.347423in}}%
\pgfpathlineto{\pgfqpoint{5.329242in}{2.341231in}}%
\pgfpathlineto{\pgfqpoint{5.322184in}{2.334984in}}%
\pgfpathlineto{\pgfqpoint{5.315119in}{2.328679in}}%
\pgfpathlineto{\pgfqpoint{5.301699in}{2.329853in}}%
\pgfpathlineto{\pgfqpoint{5.288287in}{2.331051in}}%
\pgfpathlineto{\pgfqpoint{5.274883in}{2.332273in}}%
\pgfpathlineto{\pgfqpoint{5.261487in}{2.333519in}}%
\pgfpathlineto{\pgfqpoint{5.268567in}{2.339871in}}%
\pgfpathlineto{\pgfqpoint{5.275640in}{2.346168in}}%
\pgfpathlineto{\pgfqpoint{5.282706in}{2.352411in}}%
\pgfpathlineto{\pgfqpoint{5.289766in}{2.358604in}}%
\pgfpathclose%
\pgfusepath{fill}%
\end{pgfscope}%
\begin{pgfscope}%
\pgfpathrectangle{\pgfqpoint{1.254980in}{0.150000in}}{\pgfqpoint{5.490039in}{5.490039in}}%
\pgfusepath{clip}%
\pgfsetbuttcap%
\pgfsetroundjoin%
\definecolor{currentfill}{rgb}{0.280894,0.078907,0.402329}%
\pgfsetfillcolor{currentfill}%
\pgfsetfillopacity{0.700000}%
\pgfsetlinewidth{0.000000pt}%
\definecolor{currentstroke}{rgb}{0.000000,0.000000,0.000000}%
\pgfsetstrokecolor{currentstroke}%
\pgfsetdash{}{0pt}%
\pgfpathmoveto{\pgfqpoint{4.773857in}{2.265858in}}%
\pgfpathlineto{\pgfqpoint{4.787093in}{2.264006in}}%
\pgfpathlineto{\pgfqpoint{4.800336in}{2.262180in}}%
\pgfpathlineto{\pgfqpoint{4.813586in}{2.260378in}}%
\pgfpathlineto{\pgfqpoint{4.826844in}{2.258601in}}%
\pgfpathlineto{\pgfqpoint{4.819576in}{2.251172in}}%
\pgfpathlineto{\pgfqpoint{4.812303in}{2.243689in}}%
\pgfpathlineto{\pgfqpoint{4.805024in}{2.236153in}}%
\pgfpathlineto{\pgfqpoint{4.797738in}{2.228562in}}%
\pgfpathlineto{\pgfqpoint{4.784469in}{2.230353in}}%
\pgfpathlineto{\pgfqpoint{4.771207in}{2.232169in}}%
\pgfpathlineto{\pgfqpoint{4.757952in}{2.234010in}}%
\pgfpathlineto{\pgfqpoint{4.744704in}{2.235876in}}%
\pgfpathlineto{\pgfqpoint{4.752001in}{2.243447in}}%
\pgfpathlineto{\pgfqpoint{4.759292in}{2.250968in}}%
\pgfpathlineto{\pgfqpoint{4.766578in}{2.258438in}}%
\pgfpathlineto{\pgfqpoint{4.773857in}{2.265858in}}%
\pgfpathclose%
\pgfusepath{fill}%
\end{pgfscope}%
\begin{pgfscope}%
\pgfpathrectangle{\pgfqpoint{1.254980in}{0.150000in}}{\pgfqpoint{5.490039in}{5.490039in}}%
\pgfusepath{clip}%
\pgfsetbuttcap%
\pgfsetroundjoin%
\definecolor{currentfill}{rgb}{0.282290,0.145912,0.461510}%
\pgfsetfillcolor{currentfill}%
\pgfsetfillopacity{0.700000}%
\pgfsetlinewidth{0.000000pt}%
\definecolor{currentstroke}{rgb}{0.000000,0.000000,0.000000}%
\pgfsetstrokecolor{currentstroke}%
\pgfsetdash{}{0pt}%
\pgfpathmoveto{\pgfqpoint{5.506730in}{2.390884in}}%
\pgfpathlineto{\pgfqpoint{5.520175in}{2.389730in}}%
\pgfpathlineto{\pgfqpoint{5.533628in}{2.388600in}}%
\pgfpathlineto{\pgfqpoint{5.547089in}{2.387493in}}%
\pgfpathlineto{\pgfqpoint{5.560558in}{2.386409in}}%
\pgfpathlineto{\pgfqpoint{5.553616in}{2.380871in}}%
\pgfpathlineto{\pgfqpoint{5.546666in}{2.375288in}}%
\pgfpathlineto{\pgfqpoint{5.539710in}{2.369657in}}%
\pgfpathlineto{\pgfqpoint{5.532746in}{2.363976in}}%
\pgfpathlineto{\pgfqpoint{5.519261in}{2.364983in}}%
\pgfpathlineto{\pgfqpoint{5.505783in}{2.366012in}}%
\pgfpathlineto{\pgfqpoint{5.492313in}{2.367065in}}%
\pgfpathlineto{\pgfqpoint{5.478852in}{2.368142in}}%
\pgfpathlineto{\pgfqpoint{5.485832in}{2.373896in}}%
\pgfpathlineto{\pgfqpoint{5.492804in}{2.379602in}}%
\pgfpathlineto{\pgfqpoint{5.499770in}{2.385264in}}%
\pgfpathlineto{\pgfqpoint{5.506730in}{2.390884in}}%
\pgfpathclose%
\pgfusepath{fill}%
\end{pgfscope}%
\begin{pgfscope}%
\pgfpathrectangle{\pgfqpoint{1.254980in}{0.150000in}}{\pgfqpoint{5.490039in}{5.490039in}}%
\pgfusepath{clip}%
\pgfsetbuttcap%
\pgfsetroundjoin%
\definecolor{currentfill}{rgb}{0.269944,0.014625,0.341379}%
\pgfsetfillcolor{currentfill}%
\pgfsetfillopacity{0.700000}%
\pgfsetlinewidth{0.000000pt}%
\definecolor{currentstroke}{rgb}{0.000000,0.000000,0.000000}%
\pgfsetstrokecolor{currentstroke}%
\pgfsetdash{}{0pt}%
\pgfpathmoveto{\pgfqpoint{3.420365in}{2.162164in}}%
\pgfpathlineto{\pgfqpoint{3.433288in}{2.157191in}}%
\pgfpathlineto{\pgfqpoint{3.446215in}{2.152251in}}%
\pgfpathlineto{\pgfqpoint{3.459148in}{2.147343in}}%
\pgfpathlineto{\pgfqpoint{3.472085in}{2.142467in}}%
\pgfpathlineto{\pgfqpoint{3.464297in}{2.136878in}}%
\pgfpathlineto{\pgfqpoint{3.456500in}{2.131388in}}%
\pgfpathlineto{\pgfqpoint{3.448696in}{2.125999in}}%
\pgfpathlineto{\pgfqpoint{3.440885in}{2.120716in}}%
\pgfpathlineto{\pgfqpoint{3.427930in}{2.125760in}}%
\pgfpathlineto{\pgfqpoint{3.414981in}{2.130836in}}%
\pgfpathlineto{\pgfqpoint{3.402036in}{2.135945in}}%
\pgfpathlineto{\pgfqpoint{3.389096in}{2.141086in}}%
\pgfpathlineto{\pgfqpoint{3.396925in}{2.146196in}}%
\pgfpathlineto{\pgfqpoint{3.404746in}{2.151415in}}%
\pgfpathlineto{\pgfqpoint{3.412559in}{2.156739in}}%
\pgfpathlineto{\pgfqpoint{3.420365in}{2.162164in}}%
\pgfpathclose%
\pgfusepath{fill}%
\end{pgfscope}%
\begin{pgfscope}%
\pgfpathrectangle{\pgfqpoint{1.254980in}{0.150000in}}{\pgfqpoint{5.490039in}{5.490039in}}%
\pgfusepath{clip}%
\pgfsetbuttcap%
\pgfsetroundjoin%
\definecolor{currentfill}{rgb}{0.267004,0.004874,0.329415}%
\pgfsetfillcolor{currentfill}%
\pgfsetfillopacity{0.700000}%
\pgfsetlinewidth{0.000000pt}%
\definecolor{currentstroke}{rgb}{0.000000,0.000000,0.000000}%
\pgfsetstrokecolor{currentstroke}%
\pgfsetdash{}{0pt}%
\pgfpathmoveto{\pgfqpoint{3.554900in}{2.147153in}}%
\pgfpathlineto{\pgfqpoint{3.567846in}{2.142589in}}%
\pgfpathlineto{\pgfqpoint{3.580797in}{2.138057in}}%
\pgfpathlineto{\pgfqpoint{3.593753in}{2.133556in}}%
\pgfpathlineto{\pgfqpoint{3.606714in}{2.129085in}}%
\pgfpathlineto{\pgfqpoint{3.598986in}{2.122845in}}%
\pgfpathlineto{\pgfqpoint{3.591251in}{2.116680in}}%
\pgfpathlineto{\pgfqpoint{3.583509in}{2.110594in}}%
\pgfpathlineto{\pgfqpoint{3.575760in}{2.104593in}}%
\pgfpathlineto{\pgfqpoint{3.562783in}{2.109218in}}%
\pgfpathlineto{\pgfqpoint{3.549811in}{2.113875in}}%
\pgfpathlineto{\pgfqpoint{3.536845in}{2.118562in}}%
\pgfpathlineto{\pgfqpoint{3.523883in}{2.123280in}}%
\pgfpathlineto{\pgfqpoint{3.531648in}{2.129122in}}%
\pgfpathlineto{\pgfqpoint{3.539406in}{2.135051in}}%
\pgfpathlineto{\pgfqpoint{3.547156in}{2.141062in}}%
\pgfpathlineto{\pgfqpoint{3.554900in}{2.147153in}}%
\pgfpathclose%
\pgfusepath{fill}%
\end{pgfscope}%
\begin{pgfscope}%
\pgfpathrectangle{\pgfqpoint{1.254980in}{0.150000in}}{\pgfqpoint{5.490039in}{5.490039in}}%
\pgfusepath{clip}%
\pgfsetbuttcap%
\pgfsetroundjoin%
\definecolor{currentfill}{rgb}{0.278826,0.175490,0.483397}%
\pgfsetfillcolor{currentfill}%
\pgfsetfillopacity{0.700000}%
\pgfsetlinewidth{0.000000pt}%
\definecolor{currentstroke}{rgb}{0.000000,0.000000,0.000000}%
\pgfsetstrokecolor{currentstroke}%
\pgfsetdash{}{0pt}%
\pgfpathmoveto{\pgfqpoint{5.940416in}{2.445862in}}%
\pgfpathlineto{\pgfqpoint{5.953984in}{2.444828in}}%
\pgfpathlineto{\pgfqpoint{5.967559in}{2.443816in}}%
\pgfpathlineto{\pgfqpoint{5.981143in}{2.442828in}}%
\pgfpathlineto{\pgfqpoint{5.994736in}{2.441863in}}%
\pgfpathlineto{\pgfqpoint{5.988009in}{2.437414in}}%
\pgfpathlineto{\pgfqpoint{5.981277in}{2.432956in}}%
\pgfpathlineto{\pgfqpoint{5.974538in}{2.428484in}}%
\pgfpathlineto{\pgfqpoint{5.967794in}{2.423995in}}%
\pgfpathlineto{\pgfqpoint{5.954180in}{2.424831in}}%
\pgfpathlineto{\pgfqpoint{5.940575in}{2.425689in}}%
\pgfpathlineto{\pgfqpoint{5.926979in}{2.426571in}}%
\pgfpathlineto{\pgfqpoint{5.913391in}{2.427475in}}%
\pgfpathlineto{\pgfqpoint{5.920156in}{2.432089in}}%
\pgfpathlineto{\pgfqpoint{5.926916in}{2.436689in}}%
\pgfpathlineto{\pgfqpoint{5.933669in}{2.441279in}}%
\pgfpathlineto{\pgfqpoint{5.940416in}{2.445862in}}%
\pgfpathclose%
\pgfusepath{fill}%
\end{pgfscope}%
\begin{pgfscope}%
\pgfpathrectangle{\pgfqpoint{1.254980in}{0.150000in}}{\pgfqpoint{5.490039in}{5.490039in}}%
\pgfusepath{clip}%
\pgfsetbuttcap%
\pgfsetroundjoin%
\definecolor{currentfill}{rgb}{0.280255,0.165693,0.476498}%
\pgfsetfillcolor{currentfill}%
\pgfsetfillopacity{0.700000}%
\pgfsetlinewidth{0.000000pt}%
\definecolor{currentstroke}{rgb}{0.000000,0.000000,0.000000}%
\pgfsetstrokecolor{currentstroke}%
\pgfsetdash{}{0pt}%
\pgfpathmoveto{\pgfqpoint{5.723630in}{2.420066in}}%
\pgfpathlineto{\pgfqpoint{5.737137in}{2.418999in}}%
\pgfpathlineto{\pgfqpoint{5.750653in}{2.417956in}}%
\pgfpathlineto{\pgfqpoint{5.764177in}{2.416936in}}%
\pgfpathlineto{\pgfqpoint{5.777709in}{2.415939in}}%
\pgfpathlineto{\pgfqpoint{5.770873in}{2.410979in}}%
\pgfpathlineto{\pgfqpoint{5.764030in}{2.405989in}}%
\pgfpathlineto{\pgfqpoint{5.757181in}{2.400966in}}%
\pgfpathlineto{\pgfqpoint{5.750325in}{2.395906in}}%
\pgfpathlineto{\pgfqpoint{5.736774in}{2.396799in}}%
\pgfpathlineto{\pgfqpoint{5.723232in}{2.397716in}}%
\pgfpathlineto{\pgfqpoint{5.709698in}{2.398656in}}%
\pgfpathlineto{\pgfqpoint{5.696172in}{2.399619in}}%
\pgfpathlineto{\pgfqpoint{5.703047in}{2.404778in}}%
\pgfpathlineto{\pgfqpoint{5.709914in}{2.409903in}}%
\pgfpathlineto{\pgfqpoint{5.716775in}{2.414998in}}%
\pgfpathlineto{\pgfqpoint{5.723630in}{2.420066in}}%
\pgfpathclose%
\pgfusepath{fill}%
\end{pgfscope}%
\begin{pgfscope}%
\pgfpathrectangle{\pgfqpoint{1.254980in}{0.150000in}}{\pgfqpoint{5.490039in}{5.490039in}}%
\pgfusepath{clip}%
\pgfsetbuttcap%
\pgfsetroundjoin%
\definecolor{currentfill}{rgb}{0.281446,0.084320,0.407414}%
\pgfsetfillcolor{currentfill}%
\pgfsetfillopacity{0.700000}%
\pgfsetlinewidth{0.000000pt}%
\definecolor{currentstroke}{rgb}{0.000000,0.000000,0.000000}%
\pgfsetstrokecolor{currentstroke}%
\pgfsetdash{}{0pt}%
\pgfpathmoveto{\pgfqpoint{2.964468in}{2.273270in}}%
\pgfpathlineto{\pgfqpoint{2.977335in}{2.266744in}}%
\pgfpathlineto{\pgfqpoint{2.990205in}{2.260257in}}%
\pgfpathlineto{\pgfqpoint{3.003079in}{2.253811in}}%
\pgfpathlineto{\pgfqpoint{3.015956in}{2.247403in}}%
\pgfpathlineto{\pgfqpoint{3.007923in}{2.244692in}}%
\pgfpathlineto{\pgfqpoint{2.999879in}{2.242157in}}%
\pgfpathlineto{\pgfqpoint{2.991823in}{2.239803in}}%
\pgfpathlineto{\pgfqpoint{2.983756in}{2.237634in}}%
\pgfpathlineto{\pgfqpoint{2.970856in}{2.244250in}}%
\pgfpathlineto{\pgfqpoint{2.957959in}{2.250904in}}%
\pgfpathlineto{\pgfqpoint{2.945066in}{2.257599in}}%
\pgfpathlineto{\pgfqpoint{2.932176in}{2.264333in}}%
\pgfpathlineto{\pgfqpoint{2.940267in}{2.266288in}}%
\pgfpathlineto{\pgfqpoint{2.948346in}{2.268433in}}%
\pgfpathlineto{\pgfqpoint{2.956413in}{2.270762in}}%
\pgfpathlineto{\pgfqpoint{2.964468in}{2.273270in}}%
\pgfpathclose%
\pgfusepath{fill}%
\end{pgfscope}%
\begin{pgfscope}%
\pgfpathrectangle{\pgfqpoint{1.254980in}{0.150000in}}{\pgfqpoint{5.490039in}{5.490039in}}%
\pgfusepath{clip}%
\pgfsetbuttcap%
\pgfsetroundjoin%
\definecolor{currentfill}{rgb}{0.283187,0.125848,0.444960}%
\pgfsetfillcolor{currentfill}%
\pgfsetfillopacity{0.700000}%
\pgfsetlinewidth{0.000000pt}%
\definecolor{currentstroke}{rgb}{0.000000,0.000000,0.000000}%
\pgfsetstrokecolor{currentstroke}%
\pgfsetdash{}{0pt}%
\pgfpathmoveto{\pgfqpoint{2.777742in}{2.348406in}}%
\pgfpathlineto{\pgfqpoint{2.790596in}{2.341161in}}%
\pgfpathlineto{\pgfqpoint{2.803452in}{2.333962in}}%
\pgfpathlineto{\pgfqpoint{2.816311in}{2.326806in}}%
\pgfpathlineto{\pgfqpoint{2.829172in}{2.319695in}}%
\pgfpathlineto{\pgfqpoint{2.821020in}{2.318367in}}%
\pgfpathlineto{\pgfqpoint{2.812855in}{2.317247in}}%
\pgfpathlineto{\pgfqpoint{2.804677in}{2.316339in}}%
\pgfpathlineto{\pgfqpoint{2.796485in}{2.315650in}}%
\pgfpathlineto{\pgfqpoint{2.783597in}{2.322984in}}%
\pgfpathlineto{\pgfqpoint{2.770712in}{2.330361in}}%
\pgfpathlineto{\pgfqpoint{2.757830in}{2.337783in}}%
\pgfpathlineto{\pgfqpoint{2.744951in}{2.345249in}}%
\pgfpathlineto{\pgfqpoint{2.753170in}{2.345711in}}%
\pgfpathlineto{\pgfqpoint{2.761374in}{2.346395in}}%
\pgfpathlineto{\pgfqpoint{2.769565in}{2.347295in}}%
\pgfpathlineto{\pgfqpoint{2.777742in}{2.348406in}}%
\pgfpathclose%
\pgfusepath{fill}%
\end{pgfscope}%
\begin{pgfscope}%
\pgfpathrectangle{\pgfqpoint{1.254980in}{0.150000in}}{\pgfqpoint{5.490039in}{5.490039in}}%
\pgfusepath{clip}%
\pgfsetbuttcap%
\pgfsetroundjoin%
\definecolor{currentfill}{rgb}{0.272594,0.025563,0.353093}%
\pgfsetfillcolor{currentfill}%
\pgfsetfillopacity{0.700000}%
\pgfsetlinewidth{0.000000pt}%
\definecolor{currentstroke}{rgb}{0.000000,0.000000,0.000000}%
\pgfsetstrokecolor{currentstroke}%
\pgfsetdash{}{0pt}%
\pgfpathmoveto{\pgfqpoint{4.257808in}{2.175756in}}%
\pgfpathlineto{\pgfqpoint{4.270909in}{2.172991in}}%
\pgfpathlineto{\pgfqpoint{4.284016in}{2.170253in}}%
\pgfpathlineto{\pgfqpoint{4.297129in}{2.167541in}}%
\pgfpathlineto{\pgfqpoint{4.310249in}{2.164855in}}%
\pgfpathlineto{\pgfqpoint{4.302788in}{2.156933in}}%
\pgfpathlineto{\pgfqpoint{4.295321in}{2.148993in}}%
\pgfpathlineto{\pgfqpoint{4.287850in}{2.141036in}}%
\pgfpathlineto{\pgfqpoint{4.280372in}{2.133064in}}%
\pgfpathlineto{\pgfqpoint{4.267241in}{2.135829in}}%
\pgfpathlineto{\pgfqpoint{4.254117in}{2.138620in}}%
\pgfpathlineto{\pgfqpoint{4.240998in}{2.141437in}}%
\pgfpathlineto{\pgfqpoint{4.227887in}{2.144280in}}%
\pgfpathlineto{\pgfqpoint{4.235375in}{2.152168in}}%
\pgfpathlineto{\pgfqpoint{4.242858in}{2.160045in}}%
\pgfpathlineto{\pgfqpoint{4.250336in}{2.167908in}}%
\pgfpathlineto{\pgfqpoint{4.257808in}{2.175756in}}%
\pgfpathclose%
\pgfusepath{fill}%
\end{pgfscope}%
\begin{pgfscope}%
\pgfpathrectangle{\pgfqpoint{1.254980in}{0.150000in}}{\pgfqpoint{5.490039in}{5.490039in}}%
\pgfusepath{clip}%
\pgfsetbuttcap%
\pgfsetroundjoin%
\definecolor{currentfill}{rgb}{0.272594,0.025563,0.353093}%
\pgfsetfillcolor{currentfill}%
\pgfsetfillopacity{0.700000}%
\pgfsetlinewidth{0.000000pt}%
\definecolor{currentstroke}{rgb}{0.000000,0.000000,0.000000}%
\pgfsetstrokecolor{currentstroke}%
\pgfsetdash{}{0pt}%
\pgfpathmoveto{\pgfqpoint{3.285738in}{2.183409in}}%
\pgfpathlineto{\pgfqpoint{3.298643in}{2.178000in}}%
\pgfpathlineto{\pgfqpoint{3.311551in}{2.172626in}}%
\pgfpathlineto{\pgfqpoint{3.324464in}{2.167285in}}%
\pgfpathlineto{\pgfqpoint{3.337381in}{2.161979in}}%
\pgfpathlineto{\pgfqpoint{3.329526in}{2.157158in}}%
\pgfpathlineto{\pgfqpoint{3.321663in}{2.152459in}}%
\pgfpathlineto{\pgfqpoint{3.313791in}{2.147884in}}%
\pgfpathlineto{\pgfqpoint{3.305910in}{2.143440in}}%
\pgfpathlineto{\pgfqpoint{3.292974in}{2.148928in}}%
\pgfpathlineto{\pgfqpoint{3.280042in}{2.154449in}}%
\pgfpathlineto{\pgfqpoint{3.267115in}{2.160005in}}%
\pgfpathlineto{\pgfqpoint{3.254192in}{2.165595in}}%
\pgfpathlineto{\pgfqpoint{3.262092in}{2.169853in}}%
\pgfpathlineto{\pgfqpoint{3.269983in}{2.174244in}}%
\pgfpathlineto{\pgfqpoint{3.277865in}{2.178764in}}%
\pgfpathlineto{\pgfqpoint{3.285738in}{2.183409in}}%
\pgfpathclose%
\pgfusepath{fill}%
\end{pgfscope}%
\begin{pgfscope}%
\pgfpathrectangle{\pgfqpoint{1.254980in}{0.150000in}}{\pgfqpoint{5.490039in}{5.490039in}}%
\pgfusepath{clip}%
\pgfsetbuttcap%
\pgfsetroundjoin%
\definecolor{currentfill}{rgb}{0.267004,0.004874,0.329415}%
\pgfsetfillcolor{currentfill}%
\pgfsetfillopacity{0.700000}%
\pgfsetlinewidth{0.000000pt}%
\definecolor{currentstroke}{rgb}{0.000000,0.000000,0.000000}%
\pgfsetstrokecolor{currentstroke}%
\pgfsetdash{}{0pt}%
\pgfpathmoveto{\pgfqpoint{3.689399in}{2.137727in}}%
\pgfpathlineto{\pgfqpoint{3.702372in}{2.133550in}}%
\pgfpathlineto{\pgfqpoint{3.715350in}{2.129403in}}%
\pgfpathlineto{\pgfqpoint{3.728333in}{2.125285in}}%
\pgfpathlineto{\pgfqpoint{3.741322in}{2.121196in}}%
\pgfpathlineto{\pgfqpoint{3.733649in}{2.114412in}}%
\pgfpathlineto{\pgfqpoint{3.725970in}{2.107683in}}%
\pgfpathlineto{\pgfqpoint{3.718285in}{2.101013in}}%
\pgfpathlineto{\pgfqpoint{3.710592in}{2.094406in}}%
\pgfpathlineto{\pgfqpoint{3.697589in}{2.098637in}}%
\pgfpathlineto{\pgfqpoint{3.684591in}{2.102897in}}%
\pgfpathlineto{\pgfqpoint{3.671599in}{2.107187in}}%
\pgfpathlineto{\pgfqpoint{3.658611in}{2.111506in}}%
\pgfpathlineto{\pgfqpoint{3.666318in}{2.117967in}}%
\pgfpathlineto{\pgfqpoint{3.674018in}{2.124493in}}%
\pgfpathlineto{\pgfqpoint{3.681712in}{2.131081in}}%
\pgfpathlineto{\pgfqpoint{3.689399in}{2.137727in}}%
\pgfpathclose%
\pgfusepath{fill}%
\end{pgfscope}%
\begin{pgfscope}%
\pgfpathrectangle{\pgfqpoint{1.254980in}{0.150000in}}{\pgfqpoint{5.490039in}{5.490039in}}%
\pgfusepath{clip}%
\pgfsetbuttcap%
\pgfsetroundjoin%
\definecolor{currentfill}{rgb}{0.268510,0.009605,0.335427}%
\pgfsetfillcolor{currentfill}%
\pgfsetfillopacity{0.700000}%
\pgfsetlinewidth{0.000000pt}%
\definecolor{currentstroke}{rgb}{0.000000,0.000000,0.000000}%
\pgfsetstrokecolor{currentstroke}%
\pgfsetdash{}{0pt}%
\pgfpathmoveto{\pgfqpoint{4.040891in}{2.149942in}}%
\pgfpathlineto{\pgfqpoint{4.053939in}{2.146685in}}%
\pgfpathlineto{\pgfqpoint{4.066993in}{2.143456in}}%
\pgfpathlineto{\pgfqpoint{4.080054in}{2.140254in}}%
\pgfpathlineto{\pgfqpoint{4.093120in}{2.137079in}}%
\pgfpathlineto{\pgfqpoint{4.085581in}{2.129359in}}%
\pgfpathlineto{\pgfqpoint{4.078037in}{2.121645in}}%
\pgfpathlineto{\pgfqpoint{4.070487in}{2.113940in}}%
\pgfpathlineto{\pgfqpoint{4.062931in}{2.106247in}}%
\pgfpathlineto{\pgfqpoint{4.049852in}{2.109526in}}%
\pgfpathlineto{\pgfqpoint{4.036780in}{2.112832in}}%
\pgfpathlineto{\pgfqpoint{4.023713in}{2.116165in}}%
\pgfpathlineto{\pgfqpoint{4.010653in}{2.119526in}}%
\pgfpathlineto{\pgfqpoint{4.018221in}{2.127111in}}%
\pgfpathlineto{\pgfqpoint{4.025783in}{2.134710in}}%
\pgfpathlineto{\pgfqpoint{4.033339in}{2.142321in}}%
\pgfpathlineto{\pgfqpoint{4.040891in}{2.149942in}}%
\pgfpathclose%
\pgfusepath{fill}%
\end{pgfscope}%
\begin{pgfscope}%
\pgfpathrectangle{\pgfqpoint{1.254980in}{0.150000in}}{\pgfqpoint{5.490039in}{5.490039in}}%
\pgfusepath{clip}%
\pgfsetbuttcap%
\pgfsetroundjoin%
\definecolor{currentfill}{rgb}{0.282656,0.100196,0.422160}%
\pgfsetfillcolor{currentfill}%
\pgfsetfillopacity{0.700000}%
\pgfsetlinewidth{0.000000pt}%
\definecolor{currentstroke}{rgb}{0.000000,0.000000,0.000000}%
\pgfsetstrokecolor{currentstroke}%
\pgfsetdash{}{0pt}%
\pgfpathmoveto{\pgfqpoint{4.990891in}{2.302724in}}%
\pgfpathlineto{\pgfqpoint{5.004191in}{2.301156in}}%
\pgfpathlineto{\pgfqpoint{5.017498in}{2.299612in}}%
\pgfpathlineto{\pgfqpoint{5.030813in}{2.298093in}}%
\pgfpathlineto{\pgfqpoint{5.044135in}{2.296598in}}%
\pgfpathlineto{\pgfqpoint{5.036956in}{2.289633in}}%
\pgfpathlineto{\pgfqpoint{5.029769in}{2.282609in}}%
\pgfpathlineto{\pgfqpoint{5.022576in}{2.275525in}}%
\pgfpathlineto{\pgfqpoint{5.015377in}{2.268380in}}%
\pgfpathlineto{\pgfqpoint{5.002042in}{2.269863in}}%
\pgfpathlineto{\pgfqpoint{4.988714in}{2.271371in}}%
\pgfpathlineto{\pgfqpoint{4.975394in}{2.272902in}}%
\pgfpathlineto{\pgfqpoint{4.962082in}{2.274458in}}%
\pgfpathlineto{\pgfqpoint{4.969294in}{2.281610in}}%
\pgfpathlineto{\pgfqpoint{4.976499in}{2.288704in}}%
\pgfpathlineto{\pgfqpoint{4.983698in}{2.295742in}}%
\pgfpathlineto{\pgfqpoint{4.990891in}{2.302724in}}%
\pgfpathclose%
\pgfusepath{fill}%
\end{pgfscope}%
\begin{pgfscope}%
\pgfpathrectangle{\pgfqpoint{1.254980in}{0.150000in}}{\pgfqpoint{5.490039in}{5.490039in}}%
\pgfusepath{clip}%
\pgfsetbuttcap%
\pgfsetroundjoin%
\definecolor{currentfill}{rgb}{0.276022,0.044167,0.370164}%
\pgfsetfillcolor{currentfill}%
\pgfsetfillopacity{0.700000}%
\pgfsetlinewidth{0.000000pt}%
\definecolor{currentstroke}{rgb}{0.000000,0.000000,0.000000}%
\pgfsetstrokecolor{currentstroke}%
\pgfsetdash{}{0pt}%
\pgfpathmoveto{\pgfqpoint{4.474757in}{2.207813in}}%
\pgfpathlineto{\pgfqpoint{4.487915in}{2.205478in}}%
\pgfpathlineto{\pgfqpoint{4.501079in}{2.203169in}}%
\pgfpathlineto{\pgfqpoint{4.514250in}{2.200885in}}%
\pgfpathlineto{\pgfqpoint{4.527429in}{2.198627in}}%
\pgfpathlineto{\pgfqpoint{4.520045in}{2.190752in}}%
\pgfpathlineto{\pgfqpoint{4.512657in}{2.182840in}}%
\pgfpathlineto{\pgfqpoint{4.505262in}{2.174891in}}%
\pgfpathlineto{\pgfqpoint{4.497862in}{2.166908in}}%
\pgfpathlineto{\pgfqpoint{4.484673in}{2.169219in}}%
\pgfpathlineto{\pgfqpoint{4.471490in}{2.171556in}}%
\pgfpathlineto{\pgfqpoint{4.458315in}{2.173918in}}%
\pgfpathlineto{\pgfqpoint{4.445146in}{2.176305in}}%
\pgfpathlineto{\pgfqpoint{4.452557in}{2.184231in}}%
\pgfpathlineto{\pgfqpoint{4.459963in}{2.192125in}}%
\pgfpathlineto{\pgfqpoint{4.467363in}{2.199986in}}%
\pgfpathlineto{\pgfqpoint{4.474757in}{2.207813in}}%
\pgfpathclose%
\pgfusepath{fill}%
\end{pgfscope}%
\begin{pgfscope}%
\pgfpathrectangle{\pgfqpoint{1.254980in}{0.150000in}}{\pgfqpoint{5.490039in}{5.490039in}}%
\pgfusepath{clip}%
\pgfsetbuttcap%
\pgfsetroundjoin%
\definecolor{currentfill}{rgb}{0.283229,0.120777,0.440584}%
\pgfsetfillcolor{currentfill}%
\pgfsetfillopacity{0.700000}%
\pgfsetlinewidth{0.000000pt}%
\definecolor{currentstroke}{rgb}{0.000000,0.000000,0.000000}%
\pgfsetstrokecolor{currentstroke}%
\pgfsetdash{}{0pt}%
\pgfpathmoveto{\pgfqpoint{5.207982in}{2.338741in}}%
\pgfpathlineto{\pgfqpoint{5.221347in}{2.337400in}}%
\pgfpathlineto{\pgfqpoint{5.234719in}{2.336082in}}%
\pgfpathlineto{\pgfqpoint{5.248099in}{2.334789in}}%
\pgfpathlineto{\pgfqpoint{5.261487in}{2.333519in}}%
\pgfpathlineto{\pgfqpoint{5.254401in}{2.327111in}}%
\pgfpathlineto{\pgfqpoint{5.247308in}{2.320644in}}%
\pgfpathlineto{\pgfqpoint{5.240208in}{2.314118in}}%
\pgfpathlineto{\pgfqpoint{5.233102in}{2.307530in}}%
\pgfpathlineto{\pgfqpoint{5.219699in}{2.308761in}}%
\pgfpathlineto{\pgfqpoint{5.206305in}{2.310017in}}%
\pgfpathlineto{\pgfqpoint{5.192919in}{2.311296in}}%
\pgfpathlineto{\pgfqpoint{5.179540in}{2.312600in}}%
\pgfpathlineto{\pgfqpoint{5.186661in}{2.319221in}}%
\pgfpathlineto{\pgfqpoint{5.193775in}{2.325784in}}%
\pgfpathlineto{\pgfqpoint{5.200882in}{2.332290in}}%
\pgfpathlineto{\pgfqpoint{5.207982in}{2.338741in}}%
\pgfpathclose%
\pgfusepath{fill}%
\end{pgfscope}%
\begin{pgfscope}%
\pgfpathrectangle{\pgfqpoint{1.254980in}{0.150000in}}{\pgfqpoint{5.490039in}{5.490039in}}%
\pgfusepath{clip}%
\pgfsetbuttcap%
\pgfsetroundjoin%
\definecolor{currentfill}{rgb}{0.267004,0.004874,0.329415}%
\pgfsetfillcolor{currentfill}%
\pgfsetfillopacity{0.700000}%
\pgfsetlinewidth{0.000000pt}%
\definecolor{currentstroke}{rgb}{0.000000,0.000000,0.000000}%
\pgfsetstrokecolor{currentstroke}%
\pgfsetdash{}{0pt}%
\pgfpathmoveto{\pgfqpoint{3.823909in}{2.133282in}}%
\pgfpathlineto{\pgfqpoint{3.836912in}{2.129468in}}%
\pgfpathlineto{\pgfqpoint{3.849921in}{2.125683in}}%
\pgfpathlineto{\pgfqpoint{3.862935in}{2.121926in}}%
\pgfpathlineto{\pgfqpoint{3.875955in}{2.118198in}}%
\pgfpathlineto{\pgfqpoint{3.868334in}{2.110973in}}%
\pgfpathlineto{\pgfqpoint{3.860706in}{2.103785in}}%
\pgfpathlineto{\pgfqpoint{3.853073in}{2.096637in}}%
\pgfpathlineto{\pgfqpoint{3.845433in}{2.089531in}}%
\pgfpathlineto{\pgfqpoint{3.832400in}{2.093389in}}%
\pgfpathlineto{\pgfqpoint{3.819372in}{2.097275in}}%
\pgfpathlineto{\pgfqpoint{3.806350in}{2.101190in}}%
\pgfpathlineto{\pgfqpoint{3.793333in}{2.105133in}}%
\pgfpathlineto{\pgfqpoint{3.800986in}{2.112104in}}%
\pgfpathlineto{\pgfqpoint{3.808633in}{2.119122in}}%
\pgfpathlineto{\pgfqpoint{3.816274in}{2.126182in}}%
\pgfpathlineto{\pgfqpoint{3.823909in}{2.133282in}}%
\pgfpathclose%
\pgfusepath{fill}%
\end{pgfscope}%
\begin{pgfscope}%
\pgfpathrectangle{\pgfqpoint{1.254980in}{0.150000in}}{\pgfqpoint{5.490039in}{5.490039in}}%
\pgfusepath{clip}%
\pgfsetbuttcap%
\pgfsetroundjoin%
\definecolor{currentfill}{rgb}{0.277018,0.050344,0.375715}%
\pgfsetfillcolor{currentfill}%
\pgfsetfillopacity{0.700000}%
\pgfsetlinewidth{0.000000pt}%
\definecolor{currentstroke}{rgb}{0.000000,0.000000,0.000000}%
\pgfsetstrokecolor{currentstroke}%
\pgfsetdash{}{0pt}%
\pgfpathmoveto{\pgfqpoint{3.150959in}{2.211576in}}%
\pgfpathlineto{\pgfqpoint{3.163849in}{2.205703in}}%
\pgfpathlineto{\pgfqpoint{3.176743in}{2.199867in}}%
\pgfpathlineto{\pgfqpoint{3.189641in}{2.194066in}}%
\pgfpathlineto{\pgfqpoint{3.202543in}{2.188301in}}%
\pgfpathlineto{\pgfqpoint{3.194614in}{2.184370in}}%
\pgfpathlineto{\pgfqpoint{3.186675in}{2.180585in}}%
\pgfpathlineto{\pgfqpoint{3.178727in}{2.176950in}}%
\pgfpathlineto{\pgfqpoint{3.170769in}{2.173471in}}%
\pgfpathlineto{\pgfqpoint{3.157847in}{2.179430in}}%
\pgfpathlineto{\pgfqpoint{3.144929in}{2.185425in}}%
\pgfpathlineto{\pgfqpoint{3.132014in}{2.191456in}}%
\pgfpathlineto{\pgfqpoint{3.119104in}{2.197523in}}%
\pgfpathlineto{\pgfqpoint{3.127083in}{2.200803in}}%
\pgfpathlineto{\pgfqpoint{3.135051in}{2.204242in}}%
\pgfpathlineto{\pgfqpoint{3.143010in}{2.207834in}}%
\pgfpathlineto{\pgfqpoint{3.150959in}{2.211576in}}%
\pgfpathclose%
\pgfusepath{fill}%
\end{pgfscope}%
\begin{pgfscope}%
\pgfpathrectangle{\pgfqpoint{1.254980in}{0.150000in}}{\pgfqpoint{5.490039in}{5.490039in}}%
\pgfusepath{clip}%
\pgfsetbuttcap%
\pgfsetroundjoin%
\definecolor{currentfill}{rgb}{0.279566,0.067836,0.391917}%
\pgfsetfillcolor{currentfill}%
\pgfsetfillopacity{0.700000}%
\pgfsetlinewidth{0.000000pt}%
\definecolor{currentstroke}{rgb}{0.000000,0.000000,0.000000}%
\pgfsetstrokecolor{currentstroke}%
\pgfsetdash{}{0pt}%
\pgfpathmoveto{\pgfqpoint{4.691786in}{2.243586in}}%
\pgfpathlineto{\pgfqpoint{4.705005in}{2.241621in}}%
\pgfpathlineto{\pgfqpoint{4.718231in}{2.239681in}}%
\pgfpathlineto{\pgfqpoint{4.731464in}{2.237766in}}%
\pgfpathlineto{\pgfqpoint{4.744704in}{2.235876in}}%
\pgfpathlineto{\pgfqpoint{4.737401in}{2.228253in}}%
\pgfpathlineto{\pgfqpoint{4.730092in}{2.220580in}}%
\pgfpathlineto{\pgfqpoint{4.722777in}{2.212856in}}%
\pgfpathlineto{\pgfqpoint{4.715456in}{2.205082in}}%
\pgfpathlineto{\pgfqpoint{4.702205in}{2.206999in}}%
\pgfpathlineto{\pgfqpoint{4.688960in}{2.208941in}}%
\pgfpathlineto{\pgfqpoint{4.675723in}{2.210908in}}%
\pgfpathlineto{\pgfqpoint{4.662492in}{2.212900in}}%
\pgfpathlineto{\pgfqpoint{4.669825in}{2.220643in}}%
\pgfpathlineto{\pgfqpoint{4.677151in}{2.228338in}}%
\pgfpathlineto{\pgfqpoint{4.684471in}{2.235986in}}%
\pgfpathlineto{\pgfqpoint{4.691786in}{2.243586in}}%
\pgfpathclose%
\pgfusepath{fill}%
\end{pgfscope}%
\begin{pgfscope}%
\pgfpathrectangle{\pgfqpoint{1.254980in}{0.150000in}}{\pgfqpoint{5.490039in}{5.490039in}}%
\pgfusepath{clip}%
\pgfsetbuttcap%
\pgfsetroundjoin%
\definecolor{currentfill}{rgb}{0.278826,0.175490,0.483397}%
\pgfsetfillcolor{currentfill}%
\pgfsetfillopacity{0.700000}%
\pgfsetlinewidth{0.000000pt}%
\definecolor{currentstroke}{rgb}{0.000000,0.000000,0.000000}%
\pgfsetstrokecolor{currentstroke}%
\pgfsetdash{}{0pt}%
\pgfpathmoveto{\pgfqpoint{2.590589in}{2.438507in}}%
\pgfpathlineto{\pgfqpoint{2.603440in}{2.430467in}}%
\pgfpathlineto{\pgfqpoint{2.616294in}{2.422477in}}%
\pgfpathlineto{\pgfqpoint{2.629149in}{2.414538in}}%
\pgfpathlineto{\pgfqpoint{2.642007in}{2.406648in}}%
\pgfpathlineto{\pgfqpoint{2.633719in}{2.406875in}}%
\pgfpathlineto{\pgfqpoint{2.625415in}{2.407344in}}%
\pgfpathlineto{\pgfqpoint{2.617096in}{2.408059in}}%
\pgfpathlineto{\pgfqpoint{2.608761in}{2.409027in}}%
\pgfpathlineto{\pgfqpoint{2.595874in}{2.417154in}}%
\pgfpathlineto{\pgfqpoint{2.582990in}{2.425330in}}%
\pgfpathlineto{\pgfqpoint{2.570108in}{2.433557in}}%
\pgfpathlineto{\pgfqpoint{2.557227in}{2.441834in}}%
\pgfpathlineto{\pgfqpoint{2.565592in}{2.440624in}}%
\pgfpathlineto{\pgfqpoint{2.573940in}{2.439670in}}%
\pgfpathlineto{\pgfqpoint{2.582273in}{2.438966in}}%
\pgfpathlineto{\pgfqpoint{2.590589in}{2.438507in}}%
\pgfpathclose%
\pgfusepath{fill}%
\end{pgfscope}%
\begin{pgfscope}%
\pgfpathrectangle{\pgfqpoint{1.254980in}{0.150000in}}{\pgfqpoint{5.490039in}{5.490039in}}%
\pgfusepath{clip}%
\pgfsetbuttcap%
\pgfsetroundjoin%
\definecolor{currentfill}{rgb}{0.282623,0.140926,0.457517}%
\pgfsetfillcolor{currentfill}%
\pgfsetfillopacity{0.700000}%
\pgfsetlinewidth{0.000000pt}%
\definecolor{currentstroke}{rgb}{0.000000,0.000000,0.000000}%
\pgfsetstrokecolor{currentstroke}%
\pgfsetdash{}{0pt}%
\pgfpathmoveto{\pgfqpoint{5.425087in}{2.372685in}}%
\pgfpathlineto{\pgfqpoint{5.438516in}{2.371514in}}%
\pgfpathlineto{\pgfqpoint{5.451953in}{2.370367in}}%
\pgfpathlineto{\pgfqpoint{5.465399in}{2.369243in}}%
\pgfpathlineto{\pgfqpoint{5.478852in}{2.368142in}}%
\pgfpathlineto{\pgfqpoint{5.471866in}{2.362339in}}%
\pgfpathlineto{\pgfqpoint{5.464872in}{2.356484in}}%
\pgfpathlineto{\pgfqpoint{5.457872in}{2.350575in}}%
\pgfpathlineto{\pgfqpoint{5.450865in}{2.344609in}}%
\pgfpathlineto{\pgfqpoint{5.437396in}{2.345645in}}%
\pgfpathlineto{\pgfqpoint{5.423935in}{2.346704in}}%
\pgfpathlineto{\pgfqpoint{5.410482in}{2.347788in}}%
\pgfpathlineto{\pgfqpoint{5.397037in}{2.348895in}}%
\pgfpathlineto{\pgfqpoint{5.404060in}{2.354920in}}%
\pgfpathlineto{\pgfqpoint{5.411076in}{2.360892in}}%
\pgfpathlineto{\pgfqpoint{5.418085in}{2.366813in}}%
\pgfpathlineto{\pgfqpoint{5.425087in}{2.372685in}}%
\pgfpathclose%
\pgfusepath{fill}%
\end{pgfscope}%
\begin{pgfscope}%
\pgfpathrectangle{\pgfqpoint{1.254980in}{0.150000in}}{\pgfqpoint{5.490039in}{5.490039in}}%
\pgfusepath{clip}%
\pgfsetbuttcap%
\pgfsetroundjoin%
\definecolor{currentfill}{rgb}{0.269944,0.014625,0.341379}%
\pgfsetfillcolor{currentfill}%
\pgfsetfillopacity{0.700000}%
\pgfsetlinewidth{0.000000pt}%
\definecolor{currentstroke}{rgb}{0.000000,0.000000,0.000000}%
\pgfsetstrokecolor{currentstroke}%
\pgfsetdash{}{0pt}%
\pgfpathmoveto{\pgfqpoint{4.175504in}{2.155917in}}%
\pgfpathlineto{\pgfqpoint{4.188590in}{2.152968in}}%
\pgfpathlineto{\pgfqpoint{4.201682in}{2.150045in}}%
\pgfpathlineto{\pgfqpoint{4.214781in}{2.147149in}}%
\pgfpathlineto{\pgfqpoint{4.227887in}{2.144280in}}%
\pgfpathlineto{\pgfqpoint{4.220393in}{2.136382in}}%
\pgfpathlineto{\pgfqpoint{4.212893in}{2.128475in}}%
\pgfpathlineto{\pgfqpoint{4.205388in}{2.120563in}}%
\pgfpathlineto{\pgfqpoint{4.197878in}{2.112646in}}%
\pgfpathlineto{\pgfqpoint{4.184761in}{2.115607in}}%
\pgfpathlineto{\pgfqpoint{4.171651in}{2.118594in}}%
\pgfpathlineto{\pgfqpoint{4.158547in}{2.121608in}}%
\pgfpathlineto{\pgfqpoint{4.145449in}{2.124649in}}%
\pgfpathlineto{\pgfqpoint{4.152971in}{2.132469in}}%
\pgfpathlineto{\pgfqpoint{4.160487in}{2.140289in}}%
\pgfpathlineto{\pgfqpoint{4.167998in}{2.148106in}}%
\pgfpathlineto{\pgfqpoint{4.175504in}{2.155917in}}%
\pgfpathclose%
\pgfusepath{fill}%
\end{pgfscope}%
\begin{pgfscope}%
\pgfpathrectangle{\pgfqpoint{1.254980in}{0.150000in}}{\pgfqpoint{5.490039in}{5.490039in}}%
\pgfusepath{clip}%
\pgfsetbuttcap%
\pgfsetroundjoin%
\definecolor{currentfill}{rgb}{0.280868,0.160771,0.472899}%
\pgfsetfillcolor{currentfill}%
\pgfsetfillopacity{0.700000}%
\pgfsetlinewidth{0.000000pt}%
\definecolor{currentstroke}{rgb}{0.000000,0.000000,0.000000}%
\pgfsetstrokecolor{currentstroke}%
\pgfsetdash{}{0pt}%
\pgfpathmoveto{\pgfqpoint{5.642152in}{2.403706in}}%
\pgfpathlineto{\pgfqpoint{5.655645in}{2.402649in}}%
\pgfpathlineto{\pgfqpoint{5.669146in}{2.401616in}}%
\pgfpathlineto{\pgfqpoint{5.682655in}{2.400606in}}%
\pgfpathlineto{\pgfqpoint{5.696172in}{2.399619in}}%
\pgfpathlineto{\pgfqpoint{5.689291in}{2.394423in}}%
\pgfpathlineto{\pgfqpoint{5.682404in}{2.389188in}}%
\pgfpathlineto{\pgfqpoint{5.675509in}{2.383909in}}%
\pgfpathlineto{\pgfqpoint{5.668607in}{2.378585in}}%
\pgfpathlineto{\pgfqpoint{5.655072in}{2.379481in}}%
\pgfpathlineto{\pgfqpoint{5.641546in}{2.380401in}}%
\pgfpathlineto{\pgfqpoint{5.628027in}{2.381343in}}%
\pgfpathlineto{\pgfqpoint{5.614517in}{2.382309in}}%
\pgfpathlineto{\pgfqpoint{5.621436in}{2.387720in}}%
\pgfpathlineto{\pgfqpoint{5.628348in}{2.393087in}}%
\pgfpathlineto{\pgfqpoint{5.635253in}{2.398415in}}%
\pgfpathlineto{\pgfqpoint{5.642152in}{2.403706in}}%
\pgfpathclose%
\pgfusepath{fill}%
\end{pgfscope}%
\begin{pgfscope}%
\pgfpathrectangle{\pgfqpoint{1.254980in}{0.150000in}}{\pgfqpoint{5.490039in}{5.490039in}}%
\pgfusepath{clip}%
\pgfsetbuttcap%
\pgfsetroundjoin%
\definecolor{currentfill}{rgb}{0.282327,0.094955,0.417331}%
\pgfsetfillcolor{currentfill}%
\pgfsetfillopacity{0.700000}%
\pgfsetlinewidth{0.000000pt}%
\definecolor{currentstroke}{rgb}{0.000000,0.000000,0.000000}%
\pgfsetstrokecolor{currentstroke}%
\pgfsetdash{}{0pt}%
\pgfpathmoveto{\pgfqpoint{4.908907in}{2.280926in}}%
\pgfpathlineto{\pgfqpoint{4.922190in}{2.279272in}}%
\pgfpathlineto{\pgfqpoint{4.935480in}{2.277643in}}%
\pgfpathlineto{\pgfqpoint{4.948777in}{2.276038in}}%
\pgfpathlineto{\pgfqpoint{4.962082in}{2.274458in}}%
\pgfpathlineto{\pgfqpoint{4.954864in}{2.267249in}}%
\pgfpathlineto{\pgfqpoint{4.947639in}{2.259980in}}%
\pgfpathlineto{\pgfqpoint{4.940409in}{2.252653in}}%
\pgfpathlineto{\pgfqpoint{4.933172in}{2.245266in}}%
\pgfpathlineto{\pgfqpoint{4.919854in}{2.246847in}}%
\pgfpathlineto{\pgfqpoint{4.906545in}{2.248453in}}%
\pgfpathlineto{\pgfqpoint{4.893243in}{2.250083in}}%
\pgfpathlineto{\pgfqpoint{4.879948in}{2.251737in}}%
\pgfpathlineto{\pgfqpoint{4.887197in}{2.259119in}}%
\pgfpathlineto{\pgfqpoint{4.894440in}{2.266443in}}%
\pgfpathlineto{\pgfqpoint{4.901677in}{2.273712in}}%
\pgfpathlineto{\pgfqpoint{4.908907in}{2.280926in}}%
\pgfpathclose%
\pgfusepath{fill}%
\end{pgfscope}%
\begin{pgfscope}%
\pgfpathrectangle{\pgfqpoint{1.254980in}{0.150000in}}{\pgfqpoint{5.490039in}{5.490039in}}%
\pgfusepath{clip}%
\pgfsetbuttcap%
\pgfsetroundjoin%
\definecolor{currentfill}{rgb}{0.278826,0.175490,0.483397}%
\pgfsetfillcolor{currentfill}%
\pgfsetfillopacity{0.700000}%
\pgfsetlinewidth{0.000000pt}%
\definecolor{currentstroke}{rgb}{0.000000,0.000000,0.000000}%
\pgfsetstrokecolor{currentstroke}%
\pgfsetdash{}{0pt}%
\pgfpathmoveto{\pgfqpoint{5.859124in}{2.431326in}}%
\pgfpathlineto{\pgfqpoint{5.872678in}{2.430329in}}%
\pgfpathlineto{\pgfqpoint{5.886241in}{2.429355in}}%
\pgfpathlineto{\pgfqpoint{5.899812in}{2.428403in}}%
\pgfpathlineto{\pgfqpoint{5.913391in}{2.427475in}}%
\pgfpathlineto{\pgfqpoint{5.906620in}{2.422843in}}%
\pgfpathlineto{\pgfqpoint{5.899842in}{2.418189in}}%
\pgfpathlineto{\pgfqpoint{5.893058in}{2.413509in}}%
\pgfpathlineto{\pgfqpoint{5.886267in}{2.408800in}}%
\pgfpathlineto{\pgfqpoint{5.872667in}{2.409611in}}%
\pgfpathlineto{\pgfqpoint{5.859077in}{2.410445in}}%
\pgfpathlineto{\pgfqpoint{5.845494in}{2.411303in}}%
\pgfpathlineto{\pgfqpoint{5.831920in}{2.412183in}}%
\pgfpathlineto{\pgfqpoint{5.838731in}{2.417005in}}%
\pgfpathlineto{\pgfqpoint{5.845535in}{2.421800in}}%
\pgfpathlineto{\pgfqpoint{5.852332in}{2.426572in}}%
\pgfpathlineto{\pgfqpoint{5.859124in}{2.431326in}}%
\pgfpathclose%
\pgfusepath{fill}%
\end{pgfscope}%
\begin{pgfscope}%
\pgfpathrectangle{\pgfqpoint{1.254980in}{0.150000in}}{\pgfqpoint{5.490039in}{5.490039in}}%
\pgfusepath{clip}%
\pgfsetbuttcap%
\pgfsetroundjoin%
\definecolor{currentfill}{rgb}{0.274952,0.037752,0.364543}%
\pgfsetfillcolor{currentfill}%
\pgfsetfillopacity{0.700000}%
\pgfsetlinewidth{0.000000pt}%
\definecolor{currentstroke}{rgb}{0.000000,0.000000,0.000000}%
\pgfsetstrokecolor{currentstroke}%
\pgfsetdash{}{0pt}%
\pgfpathmoveto{\pgfqpoint{4.392538in}{2.186112in}}%
\pgfpathlineto{\pgfqpoint{4.405680in}{2.183622in}}%
\pgfpathlineto{\pgfqpoint{4.418829in}{2.181158in}}%
\pgfpathlineto{\pgfqpoint{4.431984in}{2.178719in}}%
\pgfpathlineto{\pgfqpoint{4.445146in}{2.176305in}}%
\pgfpathlineto{\pgfqpoint{4.437729in}{2.168348in}}%
\pgfpathlineto{\pgfqpoint{4.430307in}{2.160361in}}%
\pgfpathlineto{\pgfqpoint{4.422879in}{2.152345in}}%
\pgfpathlineto{\pgfqpoint{4.415446in}{2.144301in}}%
\pgfpathlineto{\pgfqpoint{4.402273in}{2.146780in}}%
\pgfpathlineto{\pgfqpoint{4.389107in}{2.149285in}}%
\pgfpathlineto{\pgfqpoint{4.375947in}{2.151815in}}%
\pgfpathlineto{\pgfqpoint{4.362794in}{2.154372in}}%
\pgfpathlineto{\pgfqpoint{4.370238in}{2.162344in}}%
\pgfpathlineto{\pgfqpoint{4.377677in}{2.170293in}}%
\pgfpathlineto{\pgfqpoint{4.385110in}{2.178216in}}%
\pgfpathlineto{\pgfqpoint{4.392538in}{2.186112in}}%
\pgfpathclose%
\pgfusepath{fill}%
\end{pgfscope}%
\begin{pgfscope}%
\pgfpathrectangle{\pgfqpoint{1.254980in}{0.150000in}}{\pgfqpoint{5.490039in}{5.490039in}}%
\pgfusepath{clip}%
\pgfsetbuttcap%
\pgfsetroundjoin%
\definecolor{currentfill}{rgb}{0.267004,0.004874,0.329415}%
\pgfsetfillcolor{currentfill}%
\pgfsetfillopacity{0.700000}%
\pgfsetlinewidth{0.000000pt}%
\definecolor{currentstroke}{rgb}{0.000000,0.000000,0.000000}%
\pgfsetstrokecolor{currentstroke}%
\pgfsetdash{}{0pt}%
\pgfpathmoveto{\pgfqpoint{3.958471in}{2.133245in}}%
\pgfpathlineto{\pgfqpoint{3.971507in}{2.129773in}}%
\pgfpathlineto{\pgfqpoint{3.984550in}{2.126330in}}%
\pgfpathlineto{\pgfqpoint{3.997598in}{2.122914in}}%
\pgfpathlineto{\pgfqpoint{4.010653in}{2.119526in}}%
\pgfpathlineto{\pgfqpoint{4.003079in}{2.111958in}}%
\pgfpathlineto{\pgfqpoint{3.995500in}{2.104410in}}%
\pgfpathlineto{\pgfqpoint{3.987916in}{2.096883in}}%
\pgfpathlineto{\pgfqpoint{3.980326in}{2.089381in}}%
\pgfpathlineto{\pgfqpoint{3.967259in}{2.092886in}}%
\pgfpathlineto{\pgfqpoint{3.954198in}{2.096419in}}%
\pgfpathlineto{\pgfqpoint{3.941143in}{2.099979in}}%
\pgfpathlineto{\pgfqpoint{3.928094in}{2.103567in}}%
\pgfpathlineto{\pgfqpoint{3.935696in}{2.110947in}}%
\pgfpathlineto{\pgfqpoint{3.943294in}{2.118355in}}%
\pgfpathlineto{\pgfqpoint{3.950885in}{2.125789in}}%
\pgfpathlineto{\pgfqpoint{3.958471in}{2.133245in}}%
\pgfpathclose%
\pgfusepath{fill}%
\end{pgfscope}%
\begin{pgfscope}%
\pgfpathrectangle{\pgfqpoint{1.254980in}{0.150000in}}{\pgfqpoint{5.490039in}{5.490039in}}%
\pgfusepath{clip}%
\pgfsetbuttcap%
\pgfsetroundjoin%
\definecolor{currentfill}{rgb}{0.283197,0.115680,0.436115}%
\pgfsetfillcolor{currentfill}%
\pgfsetfillopacity{0.700000}%
\pgfsetlinewidth{0.000000pt}%
\definecolor{currentstroke}{rgb}{0.000000,0.000000,0.000000}%
\pgfsetstrokecolor{currentstroke}%
\pgfsetdash{}{0pt}%
\pgfpathmoveto{\pgfqpoint{2.829172in}{2.319695in}}%
\pgfpathlineto{\pgfqpoint{2.842037in}{2.312626in}}%
\pgfpathlineto{\pgfqpoint{2.854905in}{2.305601in}}%
\pgfpathlineto{\pgfqpoint{2.867776in}{2.298619in}}%
\pgfpathlineto{\pgfqpoint{2.880650in}{2.291679in}}%
\pgfpathlineto{\pgfqpoint{2.872522in}{2.290134in}}%
\pgfpathlineto{\pgfqpoint{2.864382in}{2.288794in}}%
\pgfpathlineto{\pgfqpoint{2.856229in}{2.287663in}}%
\pgfpathlineto{\pgfqpoint{2.848063in}{2.286748in}}%
\pgfpathlineto{\pgfqpoint{2.835164in}{2.293909in}}%
\pgfpathlineto{\pgfqpoint{2.822268in}{2.301113in}}%
\pgfpathlineto{\pgfqpoint{2.809375in}{2.308360in}}%
\pgfpathlineto{\pgfqpoint{2.796485in}{2.315650in}}%
\pgfpathlineto{\pgfqpoint{2.804677in}{2.316339in}}%
\pgfpathlineto{\pgfqpoint{2.812855in}{2.317247in}}%
\pgfpathlineto{\pgfqpoint{2.821020in}{2.318367in}}%
\pgfpathlineto{\pgfqpoint{2.829172in}{2.319695in}}%
\pgfpathclose%
\pgfusepath{fill}%
\end{pgfscope}%
\begin{pgfscope}%
\pgfpathrectangle{\pgfqpoint{1.254980in}{0.150000in}}{\pgfqpoint{5.490039in}{5.490039in}}%
\pgfusepath{clip}%
\pgfsetbuttcap%
\pgfsetroundjoin%
\definecolor{currentfill}{rgb}{0.280894,0.078907,0.402329}%
\pgfsetfillcolor{currentfill}%
\pgfsetfillopacity{0.700000}%
\pgfsetlinewidth{0.000000pt}%
\definecolor{currentstroke}{rgb}{0.000000,0.000000,0.000000}%
\pgfsetstrokecolor{currentstroke}%
\pgfsetdash{}{0pt}%
\pgfpathmoveto{\pgfqpoint{3.015956in}{2.247403in}}%
\pgfpathlineto{\pgfqpoint{3.028836in}{2.241035in}}%
\pgfpathlineto{\pgfqpoint{3.041721in}{2.234705in}}%
\pgfpathlineto{\pgfqpoint{3.054608in}{2.228414in}}%
\pgfpathlineto{\pgfqpoint{3.067500in}{2.222161in}}%
\pgfpathlineto{\pgfqpoint{3.059489in}{2.219247in}}%
\pgfpathlineto{\pgfqpoint{3.051467in}{2.216506in}}%
\pgfpathlineto{\pgfqpoint{3.043434in}{2.213942in}}%
\pgfpathlineto{\pgfqpoint{3.035390in}{2.211561in}}%
\pgfpathlineto{\pgfqpoint{3.022476in}{2.218022in}}%
\pgfpathlineto{\pgfqpoint{3.009566in}{2.224521in}}%
\pgfpathlineto{\pgfqpoint{2.996659in}{2.231058in}}%
\pgfpathlineto{\pgfqpoint{2.983756in}{2.237634in}}%
\pgfpathlineto{\pgfqpoint{2.991823in}{2.239803in}}%
\pgfpathlineto{\pgfqpoint{2.999879in}{2.242157in}}%
\pgfpathlineto{\pgfqpoint{3.007923in}{2.244692in}}%
\pgfpathlineto{\pgfqpoint{3.015956in}{2.247403in}}%
\pgfpathclose%
\pgfusepath{fill}%
\end{pgfscope}%
\begin{pgfscope}%
\pgfpathrectangle{\pgfqpoint{1.254980in}{0.150000in}}{\pgfqpoint{5.490039in}{5.490039in}}%
\pgfusepath{clip}%
\pgfsetbuttcap%
\pgfsetroundjoin%
\definecolor{currentfill}{rgb}{0.278791,0.062145,0.386592}%
\pgfsetfillcolor{currentfill}%
\pgfsetfillopacity{0.700000}%
\pgfsetlinewidth{0.000000pt}%
\definecolor{currentstroke}{rgb}{0.000000,0.000000,0.000000}%
\pgfsetstrokecolor{currentstroke}%
\pgfsetdash{}{0pt}%
\pgfpathmoveto{\pgfqpoint{4.609643in}{2.221118in}}%
\pgfpathlineto{\pgfqpoint{4.622844in}{2.219026in}}%
\pgfpathlineto{\pgfqpoint{4.636053in}{2.216959in}}%
\pgfpathlineto{\pgfqpoint{4.649269in}{2.214917in}}%
\pgfpathlineto{\pgfqpoint{4.662492in}{2.212900in}}%
\pgfpathlineto{\pgfqpoint{4.655154in}{2.205111in}}%
\pgfpathlineto{\pgfqpoint{4.647811in}{2.197276in}}%
\pgfpathlineto{\pgfqpoint{4.640461in}{2.189394in}}%
\pgfpathlineto{\pgfqpoint{4.633106in}{2.181468in}}%
\pgfpathlineto{\pgfqpoint{4.619872in}{2.183525in}}%
\pgfpathlineto{\pgfqpoint{4.606644in}{2.185607in}}%
\pgfpathlineto{\pgfqpoint{4.593424in}{2.187714in}}%
\pgfpathlineto{\pgfqpoint{4.580211in}{2.189846in}}%
\pgfpathlineto{\pgfqpoint{4.587578in}{2.197728in}}%
\pgfpathlineto{\pgfqpoint{4.594938in}{2.205567in}}%
\pgfpathlineto{\pgfqpoint{4.602293in}{2.213364in}}%
\pgfpathlineto{\pgfqpoint{4.609643in}{2.221118in}}%
\pgfpathclose%
\pgfusepath{fill}%
\end{pgfscope}%
\begin{pgfscope}%
\pgfpathrectangle{\pgfqpoint{1.254980in}{0.150000in}}{\pgfqpoint{5.490039in}{5.490039in}}%
\pgfusepath{clip}%
\pgfsetbuttcap%
\pgfsetroundjoin%
\definecolor{currentfill}{rgb}{0.283197,0.115680,0.436115}%
\pgfsetfillcolor{currentfill}%
\pgfsetfillopacity{0.700000}%
\pgfsetlinewidth{0.000000pt}%
\definecolor{currentstroke}{rgb}{0.000000,0.000000,0.000000}%
\pgfsetstrokecolor{currentstroke}%
\pgfsetdash{}{0pt}%
\pgfpathmoveto{\pgfqpoint{5.126104in}{2.318053in}}%
\pgfpathlineto{\pgfqpoint{5.139451in}{2.316653in}}%
\pgfpathlineto{\pgfqpoint{5.152807in}{2.315278in}}%
\pgfpathlineto{\pgfqpoint{5.166170in}{2.313927in}}%
\pgfpathlineto{\pgfqpoint{5.179540in}{2.312600in}}%
\pgfpathlineto{\pgfqpoint{5.172413in}{2.305919in}}%
\pgfpathlineto{\pgfqpoint{5.165279in}{2.299177in}}%
\pgfpathlineto{\pgfqpoint{5.158139in}{2.292374in}}%
\pgfpathlineto{\pgfqpoint{5.150992in}{2.285506in}}%
\pgfpathlineto{\pgfqpoint{5.137608in}{2.286809in}}%
\pgfpathlineto{\pgfqpoint{5.124231in}{2.288135in}}%
\pgfpathlineto{\pgfqpoint{5.110863in}{2.289485in}}%
\pgfpathlineto{\pgfqpoint{5.097502in}{2.290859in}}%
\pgfpathlineto{\pgfqpoint{5.104662in}{2.297747in}}%
\pgfpathlineto{\pgfqpoint{5.111816in}{2.304574in}}%
\pgfpathlineto{\pgfqpoint{5.118963in}{2.311342in}}%
\pgfpathlineto{\pgfqpoint{5.126104in}{2.318053in}}%
\pgfpathclose%
\pgfusepath{fill}%
\end{pgfscope}%
\begin{pgfscope}%
\pgfpathrectangle{\pgfqpoint{1.254980in}{0.150000in}}{\pgfqpoint{5.490039in}{5.490039in}}%
\pgfusepath{clip}%
\pgfsetbuttcap%
\pgfsetroundjoin%
\definecolor{currentfill}{rgb}{0.268510,0.009605,0.335427}%
\pgfsetfillcolor{currentfill}%
\pgfsetfillopacity{0.700000}%
\pgfsetlinewidth{0.000000pt}%
\definecolor{currentstroke}{rgb}{0.000000,0.000000,0.000000}%
\pgfsetstrokecolor{currentstroke}%
\pgfsetdash{}{0pt}%
\pgfpathmoveto{\pgfqpoint{3.472085in}{2.142467in}}%
\pgfpathlineto{\pgfqpoint{3.485027in}{2.137623in}}%
\pgfpathlineto{\pgfqpoint{3.497974in}{2.132810in}}%
\pgfpathlineto{\pgfqpoint{3.510926in}{2.128030in}}%
\pgfpathlineto{\pgfqpoint{3.523883in}{2.123280in}}%
\pgfpathlineto{\pgfqpoint{3.516111in}{2.117529in}}%
\pgfpathlineto{\pgfqpoint{3.508331in}{2.111872in}}%
\pgfpathlineto{\pgfqpoint{3.500544in}{2.106313in}}%
\pgfpathlineto{\pgfqpoint{3.492749in}{2.100858in}}%
\pgfpathlineto{\pgfqpoint{3.479776in}{2.105775in}}%
\pgfpathlineto{\pgfqpoint{3.466807in}{2.110724in}}%
\pgfpathlineto{\pgfqpoint{3.453844in}{2.115704in}}%
\pgfpathlineto{\pgfqpoint{3.440885in}{2.120716in}}%
\pgfpathlineto{\pgfqpoint{3.448696in}{2.125999in}}%
\pgfpathlineto{\pgfqpoint{3.456500in}{2.131388in}}%
\pgfpathlineto{\pgfqpoint{3.464297in}{2.136878in}}%
\pgfpathlineto{\pgfqpoint{3.472085in}{2.142467in}}%
\pgfpathclose%
\pgfusepath{fill}%
\end{pgfscope}%
\begin{pgfscope}%
\pgfpathrectangle{\pgfqpoint{1.254980in}{0.150000in}}{\pgfqpoint{5.490039in}{5.490039in}}%
\pgfusepath{clip}%
\pgfsetbuttcap%
\pgfsetroundjoin%
\definecolor{currentfill}{rgb}{0.267004,0.004874,0.329415}%
\pgfsetfillcolor{currentfill}%
\pgfsetfillopacity{0.700000}%
\pgfsetlinewidth{0.000000pt}%
\definecolor{currentstroke}{rgb}{0.000000,0.000000,0.000000}%
\pgfsetstrokecolor{currentstroke}%
\pgfsetdash{}{0pt}%
\pgfpathmoveto{\pgfqpoint{3.606714in}{2.129085in}}%
\pgfpathlineto{\pgfqpoint{3.619681in}{2.124645in}}%
\pgfpathlineto{\pgfqpoint{3.632652in}{2.120236in}}%
\pgfpathlineto{\pgfqpoint{3.645629in}{2.115856in}}%
\pgfpathlineto{\pgfqpoint{3.658611in}{2.111506in}}%
\pgfpathlineto{\pgfqpoint{3.650898in}{2.105115in}}%
\pgfpathlineto{\pgfqpoint{3.643178in}{2.098797in}}%
\pgfpathlineto{\pgfqpoint{3.635451in}{2.092555in}}%
\pgfpathlineto{\pgfqpoint{3.627718in}{2.086393in}}%
\pgfpathlineto{\pgfqpoint{3.614721in}{2.090898in}}%
\pgfpathlineto{\pgfqpoint{3.601728in}{2.095432in}}%
\pgfpathlineto{\pgfqpoint{3.588742in}{2.099997in}}%
\pgfpathlineto{\pgfqpoint{3.575760in}{2.104593in}}%
\pgfpathlineto{\pgfqpoint{3.583509in}{2.110594in}}%
\pgfpathlineto{\pgfqpoint{3.591251in}{2.116680in}}%
\pgfpathlineto{\pgfqpoint{3.598986in}{2.122845in}}%
\pgfpathlineto{\pgfqpoint{3.606714in}{2.129085in}}%
\pgfpathclose%
\pgfusepath{fill}%
\end{pgfscope}%
\begin{pgfscope}%
\pgfpathrectangle{\pgfqpoint{1.254980in}{0.150000in}}{\pgfqpoint{5.490039in}{5.490039in}}%
\pgfusepath{clip}%
\pgfsetbuttcap%
\pgfsetroundjoin%
\definecolor{currentfill}{rgb}{0.272594,0.025563,0.353093}%
\pgfsetfillcolor{currentfill}%
\pgfsetfillopacity{0.700000}%
\pgfsetlinewidth{0.000000pt}%
\definecolor{currentstroke}{rgb}{0.000000,0.000000,0.000000}%
\pgfsetstrokecolor{currentstroke}%
\pgfsetdash{}{0pt}%
\pgfpathmoveto{\pgfqpoint{3.337381in}{2.161979in}}%
\pgfpathlineto{\pgfqpoint{3.350303in}{2.156706in}}%
\pgfpathlineto{\pgfqpoint{3.363230in}{2.151466in}}%
\pgfpathlineto{\pgfqpoint{3.376161in}{2.146259in}}%
\pgfpathlineto{\pgfqpoint{3.389096in}{2.141086in}}%
\pgfpathlineto{\pgfqpoint{3.381259in}{2.136089in}}%
\pgfpathlineto{\pgfqpoint{3.373413in}{2.131210in}}%
\pgfpathlineto{\pgfqpoint{3.365560in}{2.126454in}}%
\pgfpathlineto{\pgfqpoint{3.357697in}{2.121824in}}%
\pgfpathlineto{\pgfqpoint{3.344744in}{2.127178in}}%
\pgfpathlineto{\pgfqpoint{3.331795in}{2.132566in}}%
\pgfpathlineto{\pgfqpoint{3.318850in}{2.137986in}}%
\pgfpathlineto{\pgfqpoint{3.305910in}{2.143440in}}%
\pgfpathlineto{\pgfqpoint{3.313791in}{2.147884in}}%
\pgfpathlineto{\pgfqpoint{3.321663in}{2.152459in}}%
\pgfpathlineto{\pgfqpoint{3.329526in}{2.157158in}}%
\pgfpathlineto{\pgfqpoint{3.337381in}{2.161979in}}%
\pgfpathclose%
\pgfusepath{fill}%
\end{pgfscope}%
\begin{pgfscope}%
\pgfpathrectangle{\pgfqpoint{1.254980in}{0.150000in}}{\pgfqpoint{5.490039in}{5.490039in}}%
\pgfusepath{clip}%
\pgfsetbuttcap%
\pgfsetroundjoin%
\definecolor{currentfill}{rgb}{0.280255,0.165693,0.476498}%
\pgfsetfillcolor{currentfill}%
\pgfsetfillopacity{0.700000}%
\pgfsetlinewidth{0.000000pt}%
\definecolor{currentstroke}{rgb}{0.000000,0.000000,0.000000}%
\pgfsetstrokecolor{currentstroke}%
\pgfsetdash{}{0pt}%
\pgfpathmoveto{\pgfqpoint{2.642007in}{2.406648in}}%
\pgfpathlineto{\pgfqpoint{2.654866in}{2.398806in}}%
\pgfpathlineto{\pgfqpoint{2.667728in}{2.391014in}}%
\pgfpathlineto{\pgfqpoint{2.680593in}{2.383269in}}%
\pgfpathlineto{\pgfqpoint{2.693460in}{2.375572in}}%
\pgfpathlineto{\pgfqpoint{2.685199in}{2.375569in}}%
\pgfpathlineto{\pgfqpoint{2.676924in}{2.375803in}}%
\pgfpathlineto{\pgfqpoint{2.668634in}{2.376280in}}%
\pgfpathlineto{\pgfqpoint{2.660327in}{2.377006in}}%
\pgfpathlineto{\pgfqpoint{2.647432in}{2.384940in}}%
\pgfpathlineto{\pgfqpoint{2.634540in}{2.392921in}}%
\pgfpathlineto{\pgfqpoint{2.621649in}{2.400950in}}%
\pgfpathlineto{\pgfqpoint{2.608761in}{2.409027in}}%
\pgfpathlineto{\pgfqpoint{2.617096in}{2.408059in}}%
\pgfpathlineto{\pgfqpoint{2.625415in}{2.407344in}}%
\pgfpathlineto{\pgfqpoint{2.633719in}{2.406875in}}%
\pgfpathlineto{\pgfqpoint{2.642007in}{2.406648in}}%
\pgfpathclose%
\pgfusepath{fill}%
\end{pgfscope}%
\begin{pgfscope}%
\pgfpathrectangle{\pgfqpoint{1.254980in}{0.150000in}}{\pgfqpoint{5.490039in}{5.490039in}}%
\pgfusepath{clip}%
\pgfsetbuttcap%
\pgfsetroundjoin%
\definecolor{currentfill}{rgb}{0.282884,0.135920,0.453427}%
\pgfsetfillcolor{currentfill}%
\pgfsetfillopacity{0.700000}%
\pgfsetlinewidth{0.000000pt}%
\definecolor{currentstroke}{rgb}{0.000000,0.000000,0.000000}%
\pgfsetstrokecolor{currentstroke}%
\pgfsetdash{}{0pt}%
\pgfpathmoveto{\pgfqpoint{5.343338in}{2.353559in}}%
\pgfpathlineto{\pgfqpoint{5.356751in}{2.352358in}}%
\pgfpathlineto{\pgfqpoint{5.370172in}{2.351180in}}%
\pgfpathlineto{\pgfqpoint{5.383600in}{2.350025in}}%
\pgfpathlineto{\pgfqpoint{5.397037in}{2.348895in}}%
\pgfpathlineto{\pgfqpoint{5.390008in}{2.342814in}}%
\pgfpathlineto{\pgfqpoint{5.382971in}{2.336676in}}%
\pgfpathlineto{\pgfqpoint{5.375928in}{2.330478in}}%
\pgfpathlineto{\pgfqpoint{5.368877in}{2.324219in}}%
\pgfpathlineto{\pgfqpoint{5.355425in}{2.325299in}}%
\pgfpathlineto{\pgfqpoint{5.341982in}{2.326401in}}%
\pgfpathlineto{\pgfqpoint{5.328546in}{2.327528in}}%
\pgfpathlineto{\pgfqpoint{5.315119in}{2.328679in}}%
\pgfpathlineto{\pgfqpoint{5.322184in}{2.334984in}}%
\pgfpathlineto{\pgfqpoint{5.329242in}{2.341231in}}%
\pgfpathlineto{\pgfqpoint{5.336293in}{2.347423in}}%
\pgfpathlineto{\pgfqpoint{5.343338in}{2.353559in}}%
\pgfpathclose%
\pgfusepath{fill}%
\end{pgfscope}%
\begin{pgfscope}%
\pgfpathrectangle{\pgfqpoint{1.254980in}{0.150000in}}{\pgfqpoint{5.490039in}{5.490039in}}%
\pgfusepath{clip}%
\pgfsetbuttcap%
\pgfsetroundjoin%
\definecolor{currentfill}{rgb}{0.277134,0.185228,0.489898}%
\pgfsetfillcolor{currentfill}%
\pgfsetfillopacity{0.700000}%
\pgfsetlinewidth{0.000000pt}%
\definecolor{currentstroke}{rgb}{0.000000,0.000000,0.000000}%
\pgfsetstrokecolor{currentstroke}%
\pgfsetdash{}{0pt}%
\pgfpathmoveto{\pgfqpoint{5.994736in}{2.441863in}}%
\pgfpathlineto{\pgfqpoint{6.008337in}{2.440921in}}%
\pgfpathlineto{\pgfqpoint{6.021946in}{2.440002in}}%
\pgfpathlineto{\pgfqpoint{6.035564in}{2.439106in}}%
\pgfpathlineto{\pgfqpoint{6.028854in}{2.434758in}}%
\pgfpathlineto{\pgfqpoint{6.022137in}{2.430398in}}%
\pgfpathlineto{\pgfqpoint{6.015414in}{2.426023in}}%
\pgfpathlineto{\pgfqpoint{6.008685in}{2.421628in}}%
\pgfpathlineto{\pgfqpoint{5.995046in}{2.422394in}}%
\pgfpathlineto{\pgfqpoint{5.981415in}{2.423183in}}%
\pgfpathlineto{\pgfqpoint{5.967794in}{2.423995in}}%
\pgfpathlineto{\pgfqpoint{5.974538in}{2.428484in}}%
\pgfpathlineto{\pgfqpoint{5.981277in}{2.432956in}}%
\pgfpathlineto{\pgfqpoint{5.988009in}{2.437414in}}%
\pgfpathlineto{\pgfqpoint{5.994736in}{2.441863in}}%
\pgfpathclose%
\pgfusepath{fill}%
\end{pgfscope}%
\begin{pgfscope}%
\pgfpathrectangle{\pgfqpoint{1.254980in}{0.150000in}}{\pgfqpoint{5.490039in}{5.490039in}}%
\pgfusepath{clip}%
\pgfsetbuttcap%
\pgfsetroundjoin%
\definecolor{currentfill}{rgb}{0.267004,0.004874,0.329415}%
\pgfsetfillcolor{currentfill}%
\pgfsetfillopacity{0.700000}%
\pgfsetlinewidth{0.000000pt}%
\definecolor{currentstroke}{rgb}{0.000000,0.000000,0.000000}%
\pgfsetstrokecolor{currentstroke}%
\pgfsetdash{}{0pt}%
\pgfpathmoveto{\pgfqpoint{3.741322in}{2.121196in}}%
\pgfpathlineto{\pgfqpoint{3.754317in}{2.117137in}}%
\pgfpathlineto{\pgfqpoint{3.767317in}{2.113107in}}%
\pgfpathlineto{\pgfqpoint{3.780322in}{2.109106in}}%
\pgfpathlineto{\pgfqpoint{3.793333in}{2.105133in}}%
\pgfpathlineto{\pgfqpoint{3.785674in}{2.098211in}}%
\pgfpathlineto{\pgfqpoint{3.778009in}{2.091342in}}%
\pgfpathlineto{\pgfqpoint{3.770338in}{2.084528in}}%
\pgfpathlineto{\pgfqpoint{3.762660in}{2.077774in}}%
\pgfpathlineto{\pgfqpoint{3.749635in}{2.081888in}}%
\pgfpathlineto{\pgfqpoint{3.736615in}{2.086032in}}%
\pgfpathlineto{\pgfqpoint{3.723601in}{2.090204in}}%
\pgfpathlineto{\pgfqpoint{3.710592in}{2.094406in}}%
\pgfpathlineto{\pgfqpoint{3.718285in}{2.101013in}}%
\pgfpathlineto{\pgfqpoint{3.725970in}{2.107683in}}%
\pgfpathlineto{\pgfqpoint{3.733649in}{2.114412in}}%
\pgfpathlineto{\pgfqpoint{3.741322in}{2.121196in}}%
\pgfpathclose%
\pgfusepath{fill}%
\end{pgfscope}%
\begin{pgfscope}%
\pgfpathrectangle{\pgfqpoint{1.254980in}{0.150000in}}{\pgfqpoint{5.490039in}{5.490039in}}%
\pgfusepath{clip}%
\pgfsetbuttcap%
\pgfsetroundjoin%
\definecolor{currentfill}{rgb}{0.281446,0.084320,0.407414}%
\pgfsetfillcolor{currentfill}%
\pgfsetfillopacity{0.700000}%
\pgfsetlinewidth{0.000000pt}%
\definecolor{currentstroke}{rgb}{0.000000,0.000000,0.000000}%
\pgfsetstrokecolor{currentstroke}%
\pgfsetdash{}{0pt}%
\pgfpathmoveto{\pgfqpoint{4.826844in}{2.258601in}}%
\pgfpathlineto{\pgfqpoint{4.840109in}{2.256848in}}%
\pgfpathlineto{\pgfqpoint{4.853381in}{2.255120in}}%
\pgfpathlineto{\pgfqpoint{4.866661in}{2.253416in}}%
\pgfpathlineto{\pgfqpoint{4.879948in}{2.251737in}}%
\pgfpathlineto{\pgfqpoint{4.872693in}{2.244299in}}%
\pgfpathlineto{\pgfqpoint{4.865431in}{2.236804in}}%
\pgfpathlineto{\pgfqpoint{4.858164in}{2.229252in}}%
\pgfpathlineto{\pgfqpoint{4.850890in}{2.221642in}}%
\pgfpathlineto{\pgfqpoint{4.837591in}{2.223335in}}%
\pgfpathlineto{\pgfqpoint{4.824299in}{2.225053in}}%
\pgfpathlineto{\pgfqpoint{4.811015in}{2.226795in}}%
\pgfpathlineto{\pgfqpoint{4.797738in}{2.228562in}}%
\pgfpathlineto{\pgfqpoint{4.805024in}{2.236153in}}%
\pgfpathlineto{\pgfqpoint{4.812303in}{2.243689in}}%
\pgfpathlineto{\pgfqpoint{4.819576in}{2.251172in}}%
\pgfpathlineto{\pgfqpoint{4.826844in}{2.258601in}}%
\pgfpathclose%
\pgfusepath{fill}%
\end{pgfscope}%
\begin{pgfscope}%
\pgfpathrectangle{\pgfqpoint{1.254980in}{0.150000in}}{\pgfqpoint{5.490039in}{5.490039in}}%
\pgfusepath{clip}%
\pgfsetbuttcap%
\pgfsetroundjoin%
\definecolor{currentfill}{rgb}{0.268510,0.009605,0.335427}%
\pgfsetfillcolor{currentfill}%
\pgfsetfillopacity{0.700000}%
\pgfsetlinewidth{0.000000pt}%
\definecolor{currentstroke}{rgb}{0.000000,0.000000,0.000000}%
\pgfsetstrokecolor{currentstroke}%
\pgfsetdash{}{0pt}%
\pgfpathmoveto{\pgfqpoint{4.093120in}{2.137079in}}%
\pgfpathlineto{\pgfqpoint{4.106193in}{2.133931in}}%
\pgfpathlineto{\pgfqpoint{4.119272in}{2.130810in}}%
\pgfpathlineto{\pgfqpoint{4.132358in}{2.127716in}}%
\pgfpathlineto{\pgfqpoint{4.145449in}{2.124649in}}%
\pgfpathlineto{\pgfqpoint{4.137922in}{2.116829in}}%
\pgfpathlineto{\pgfqpoint{4.130389in}{2.109013in}}%
\pgfpathlineto{\pgfqpoint{4.122851in}{2.101202in}}%
\pgfpathlineto{\pgfqpoint{4.115307in}{2.093400in}}%
\pgfpathlineto{\pgfqpoint{4.102204in}{2.096571in}}%
\pgfpathlineto{\pgfqpoint{4.089107in}{2.099770in}}%
\pgfpathlineto{\pgfqpoint{4.076016in}{2.102995in}}%
\pgfpathlineto{\pgfqpoint{4.062931in}{2.106247in}}%
\pgfpathlineto{\pgfqpoint{4.070487in}{2.113940in}}%
\pgfpathlineto{\pgfqpoint{4.078037in}{2.121645in}}%
\pgfpathlineto{\pgfqpoint{4.085581in}{2.129359in}}%
\pgfpathlineto{\pgfqpoint{4.093120in}{2.137079in}}%
\pgfpathclose%
\pgfusepath{fill}%
\end{pgfscope}%
\begin{pgfscope}%
\pgfpathrectangle{\pgfqpoint{1.254980in}{0.150000in}}{\pgfqpoint{5.490039in}{5.490039in}}%
\pgfusepath{clip}%
\pgfsetbuttcap%
\pgfsetroundjoin%
\definecolor{currentfill}{rgb}{0.272594,0.025563,0.353093}%
\pgfsetfillcolor{currentfill}%
\pgfsetfillopacity{0.700000}%
\pgfsetlinewidth{0.000000pt}%
\definecolor{currentstroke}{rgb}{0.000000,0.000000,0.000000}%
\pgfsetstrokecolor{currentstroke}%
\pgfsetdash{}{0pt}%
\pgfpathmoveto{\pgfqpoint{4.310249in}{2.164855in}}%
\pgfpathlineto{\pgfqpoint{4.323375in}{2.162195in}}%
\pgfpathlineto{\pgfqpoint{4.336508in}{2.159561in}}%
\pgfpathlineto{\pgfqpoint{4.349648in}{2.156954in}}%
\pgfpathlineto{\pgfqpoint{4.362794in}{2.154372in}}%
\pgfpathlineto{\pgfqpoint{4.355344in}{2.146376in}}%
\pgfpathlineto{\pgfqpoint{4.347889in}{2.138359in}}%
\pgfpathlineto{\pgfqpoint{4.340429in}{2.130322in}}%
\pgfpathlineto{\pgfqpoint{4.332963in}{2.122266in}}%
\pgfpathlineto{\pgfqpoint{4.319805in}{2.124927in}}%
\pgfpathlineto{\pgfqpoint{4.306654in}{2.127614in}}%
\pgfpathlineto{\pgfqpoint{4.293510in}{2.130326in}}%
\pgfpathlineto{\pgfqpoint{4.280372in}{2.133064in}}%
\pgfpathlineto{\pgfqpoint{4.287850in}{2.141036in}}%
\pgfpathlineto{\pgfqpoint{4.295321in}{2.148993in}}%
\pgfpathlineto{\pgfqpoint{4.302788in}{2.156933in}}%
\pgfpathlineto{\pgfqpoint{4.310249in}{2.164855in}}%
\pgfpathclose%
\pgfusepath{fill}%
\end{pgfscope}%
\begin{pgfscope}%
\pgfpathrectangle{\pgfqpoint{1.254980in}{0.150000in}}{\pgfqpoint{5.490039in}{5.490039in}}%
\pgfusepath{clip}%
\pgfsetbuttcap%
\pgfsetroundjoin%
\definecolor{currentfill}{rgb}{0.276022,0.044167,0.370164}%
\pgfsetfillcolor{currentfill}%
\pgfsetfillopacity{0.700000}%
\pgfsetlinewidth{0.000000pt}%
\definecolor{currentstroke}{rgb}{0.000000,0.000000,0.000000}%
\pgfsetstrokecolor{currentstroke}%
\pgfsetdash{}{0pt}%
\pgfpathmoveto{\pgfqpoint{3.202543in}{2.188301in}}%
\pgfpathlineto{\pgfqpoint{3.215449in}{2.182572in}}%
\pgfpathlineto{\pgfqpoint{3.228359in}{2.176878in}}%
\pgfpathlineto{\pgfqpoint{3.241273in}{2.171219in}}%
\pgfpathlineto{\pgfqpoint{3.254192in}{2.165595in}}%
\pgfpathlineto{\pgfqpoint{3.246283in}{2.161474in}}%
\pgfpathlineto{\pgfqpoint{3.238365in}{2.157496in}}%
\pgfpathlineto{\pgfqpoint{3.230437in}{2.153666in}}%
\pgfpathlineto{\pgfqpoint{3.222499in}{2.149988in}}%
\pgfpathlineto{\pgfqpoint{3.209561in}{2.155806in}}%
\pgfpathlineto{\pgfqpoint{3.196626in}{2.161659in}}%
\pgfpathlineto{\pgfqpoint{3.183696in}{2.167548in}}%
\pgfpathlineto{\pgfqpoint{3.170769in}{2.173471in}}%
\pgfpathlineto{\pgfqpoint{3.178727in}{2.176950in}}%
\pgfpathlineto{\pgfqpoint{3.186675in}{2.180585in}}%
\pgfpathlineto{\pgfqpoint{3.194614in}{2.184370in}}%
\pgfpathlineto{\pgfqpoint{3.202543in}{2.188301in}}%
\pgfpathclose%
\pgfusepath{fill}%
\end{pgfscope}%
\begin{pgfscope}%
\pgfpathrectangle{\pgfqpoint{1.254980in}{0.150000in}}{\pgfqpoint{5.490039in}{5.490039in}}%
\pgfusepath{clip}%
\pgfsetbuttcap%
\pgfsetroundjoin%
\definecolor{currentfill}{rgb}{0.281412,0.155834,0.469201}%
\pgfsetfillcolor{currentfill}%
\pgfsetfillopacity{0.700000}%
\pgfsetlinewidth{0.000000pt}%
\definecolor{currentstroke}{rgb}{0.000000,0.000000,0.000000}%
\pgfsetstrokecolor{currentstroke}%
\pgfsetdash{}{0pt}%
\pgfpathmoveto{\pgfqpoint{5.560558in}{2.386409in}}%
\pgfpathlineto{\pgfqpoint{5.574036in}{2.385349in}}%
\pgfpathlineto{\pgfqpoint{5.587521in}{2.384312in}}%
\pgfpathlineto{\pgfqpoint{5.601015in}{2.383299in}}%
\pgfpathlineto{\pgfqpoint{5.614517in}{2.382309in}}%
\pgfpathlineto{\pgfqpoint{5.607591in}{2.376854in}}%
\pgfpathlineto{\pgfqpoint{5.600658in}{2.371350in}}%
\pgfpathlineto{\pgfqpoint{5.593719in}{2.365796in}}%
\pgfpathlineto{\pgfqpoint{5.586772in}{2.360188in}}%
\pgfpathlineto{\pgfqpoint{5.573253in}{2.361100in}}%
\pgfpathlineto{\pgfqpoint{5.559743in}{2.362035in}}%
\pgfpathlineto{\pgfqpoint{5.546240in}{2.362994in}}%
\pgfpathlineto{\pgfqpoint{5.532746in}{2.363976in}}%
\pgfpathlineto{\pgfqpoint{5.539710in}{2.369657in}}%
\pgfpathlineto{\pgfqpoint{5.546666in}{2.375288in}}%
\pgfpathlineto{\pgfqpoint{5.553616in}{2.380871in}}%
\pgfpathlineto{\pgfqpoint{5.560558in}{2.386409in}}%
\pgfpathclose%
\pgfusepath{fill}%
\end{pgfscope}%
\begin{pgfscope}%
\pgfpathrectangle{\pgfqpoint{1.254980in}{0.150000in}}{\pgfqpoint{5.490039in}{5.490039in}}%
\pgfusepath{clip}%
\pgfsetbuttcap%
\pgfsetroundjoin%
\definecolor{currentfill}{rgb}{0.279574,0.170599,0.479997}%
\pgfsetfillcolor{currentfill}%
\pgfsetfillopacity{0.700000}%
\pgfsetlinewidth{0.000000pt}%
\definecolor{currentstroke}{rgb}{0.000000,0.000000,0.000000}%
\pgfsetstrokecolor{currentstroke}%
\pgfsetdash{}{0pt}%
\pgfpathmoveto{\pgfqpoint{5.777709in}{2.415939in}}%
\pgfpathlineto{\pgfqpoint{5.791249in}{2.414965in}}%
\pgfpathlineto{\pgfqpoint{5.804798in}{2.414015in}}%
\pgfpathlineto{\pgfqpoint{5.818355in}{2.413087in}}%
\pgfpathlineto{\pgfqpoint{5.831920in}{2.412183in}}%
\pgfpathlineto{\pgfqpoint{5.825103in}{2.407332in}}%
\pgfpathlineto{\pgfqpoint{5.818280in}{2.402448in}}%
\pgfpathlineto{\pgfqpoint{5.811449in}{2.397527in}}%
\pgfpathlineto{\pgfqpoint{5.804612in}{2.392566in}}%
\pgfpathlineto{\pgfqpoint{5.791027in}{2.393366in}}%
\pgfpathlineto{\pgfqpoint{5.777451in}{2.394189in}}%
\pgfpathlineto{\pgfqpoint{5.763884in}{2.395036in}}%
\pgfpathlineto{\pgfqpoint{5.750325in}{2.395906in}}%
\pgfpathlineto{\pgfqpoint{5.757181in}{2.400966in}}%
\pgfpathlineto{\pgfqpoint{5.764030in}{2.405989in}}%
\pgfpathlineto{\pgfqpoint{5.770873in}{2.410979in}}%
\pgfpathlineto{\pgfqpoint{5.777709in}{2.415939in}}%
\pgfpathclose%
\pgfusepath{fill}%
\end{pgfscope}%
\begin{pgfscope}%
\pgfpathrectangle{\pgfqpoint{1.254980in}{0.150000in}}{\pgfqpoint{5.490039in}{5.490039in}}%
\pgfusepath{clip}%
\pgfsetbuttcap%
\pgfsetroundjoin%
\definecolor{currentfill}{rgb}{0.277018,0.050344,0.375715}%
\pgfsetfillcolor{currentfill}%
\pgfsetfillopacity{0.700000}%
\pgfsetlinewidth{0.000000pt}%
\definecolor{currentstroke}{rgb}{0.000000,0.000000,0.000000}%
\pgfsetstrokecolor{currentstroke}%
\pgfsetdash{}{0pt}%
\pgfpathmoveto{\pgfqpoint{4.527429in}{2.198627in}}%
\pgfpathlineto{\pgfqpoint{4.540614in}{2.196394in}}%
\pgfpathlineto{\pgfqpoint{4.553806in}{2.194186in}}%
\pgfpathlineto{\pgfqpoint{4.567005in}{2.192003in}}%
\pgfpathlineto{\pgfqpoint{4.580211in}{2.189846in}}%
\pgfpathlineto{\pgfqpoint{4.572839in}{2.181923in}}%
\pgfpathlineto{\pgfqpoint{4.565461in}{2.173960in}}%
\pgfpathlineto{\pgfqpoint{4.558078in}{2.165957in}}%
\pgfpathlineto{\pgfqpoint{4.550689in}{2.157915in}}%
\pgfpathlineto{\pgfqpoint{4.537472in}{2.160125in}}%
\pgfpathlineto{\pgfqpoint{4.524262in}{2.162361in}}%
\pgfpathlineto{\pgfqpoint{4.511058in}{2.164622in}}%
\pgfpathlineto{\pgfqpoint{4.497862in}{2.166908in}}%
\pgfpathlineto{\pgfqpoint{4.505262in}{2.174891in}}%
\pgfpathlineto{\pgfqpoint{4.512657in}{2.182840in}}%
\pgfpathlineto{\pgfqpoint{4.520045in}{2.190752in}}%
\pgfpathlineto{\pgfqpoint{4.527429in}{2.198627in}}%
\pgfpathclose%
\pgfusepath{fill}%
\end{pgfscope}%
\begin{pgfscope}%
\pgfpathrectangle{\pgfqpoint{1.254980in}{0.150000in}}{\pgfqpoint{5.490039in}{5.490039in}}%
\pgfusepath{clip}%
\pgfsetbuttcap%
\pgfsetroundjoin%
\definecolor{currentfill}{rgb}{0.283091,0.110553,0.431554}%
\pgfsetfillcolor{currentfill}%
\pgfsetfillopacity{0.700000}%
\pgfsetlinewidth{0.000000pt}%
\definecolor{currentstroke}{rgb}{0.000000,0.000000,0.000000}%
\pgfsetstrokecolor{currentstroke}%
\pgfsetdash{}{0pt}%
\pgfpathmoveto{\pgfqpoint{5.044135in}{2.296598in}}%
\pgfpathlineto{\pgfqpoint{5.057466in}{2.295127in}}%
\pgfpathlineto{\pgfqpoint{5.070803in}{2.293680in}}%
\pgfpathlineto{\pgfqpoint{5.084149in}{2.292258in}}%
\pgfpathlineto{\pgfqpoint{5.097502in}{2.290859in}}%
\pgfpathlineto{\pgfqpoint{5.090335in}{2.283911in}}%
\pgfpathlineto{\pgfqpoint{5.083161in}{2.276901in}}%
\pgfpathlineto{\pgfqpoint{5.075981in}{2.269827in}}%
\pgfpathlineto{\pgfqpoint{5.068795in}{2.262690in}}%
\pgfpathlineto{\pgfqpoint{5.055429in}{2.264076in}}%
\pgfpathlineto{\pgfqpoint{5.042071in}{2.265487in}}%
\pgfpathlineto{\pgfqpoint{5.028720in}{2.266921in}}%
\pgfpathlineto{\pgfqpoint{5.015377in}{2.268380in}}%
\pgfpathlineto{\pgfqpoint{5.022576in}{2.275525in}}%
\pgfpathlineto{\pgfqpoint{5.029769in}{2.282609in}}%
\pgfpathlineto{\pgfqpoint{5.036956in}{2.289633in}}%
\pgfpathlineto{\pgfqpoint{5.044135in}{2.296598in}}%
\pgfpathclose%
\pgfusepath{fill}%
\end{pgfscope}%
\begin{pgfscope}%
\pgfpathrectangle{\pgfqpoint{1.254980in}{0.150000in}}{\pgfqpoint{5.490039in}{5.490039in}}%
\pgfusepath{clip}%
\pgfsetbuttcap%
\pgfsetroundjoin%
\definecolor{currentfill}{rgb}{0.267004,0.004874,0.329415}%
\pgfsetfillcolor{currentfill}%
\pgfsetfillopacity{0.700000}%
\pgfsetlinewidth{0.000000pt}%
\definecolor{currentstroke}{rgb}{0.000000,0.000000,0.000000}%
\pgfsetstrokecolor{currentstroke}%
\pgfsetdash{}{0pt}%
\pgfpathmoveto{\pgfqpoint{3.875955in}{2.118198in}}%
\pgfpathlineto{\pgfqpoint{3.888981in}{2.114498in}}%
\pgfpathlineto{\pgfqpoint{3.902013in}{2.110826in}}%
\pgfpathlineto{\pgfqpoint{3.915050in}{2.107182in}}%
\pgfpathlineto{\pgfqpoint{3.928094in}{2.103567in}}%
\pgfpathlineto{\pgfqpoint{3.920485in}{2.096217in}}%
\pgfpathlineto{\pgfqpoint{3.912870in}{2.088901in}}%
\pgfpathlineto{\pgfqpoint{3.905250in}{2.081621in}}%
\pgfpathlineto{\pgfqpoint{3.897624in}{2.074381in}}%
\pgfpathlineto{\pgfqpoint{3.884568in}{2.078127in}}%
\pgfpathlineto{\pgfqpoint{3.871517in}{2.081900in}}%
\pgfpathlineto{\pgfqpoint{3.858472in}{2.085702in}}%
\pgfpathlineto{\pgfqpoint{3.845433in}{2.089531in}}%
\pgfpathlineto{\pgfqpoint{3.853073in}{2.096637in}}%
\pgfpathlineto{\pgfqpoint{3.860706in}{2.103785in}}%
\pgfpathlineto{\pgfqpoint{3.868334in}{2.110973in}}%
\pgfpathlineto{\pgfqpoint{3.875955in}{2.118198in}}%
\pgfpathclose%
\pgfusepath{fill}%
\end{pgfscope}%
\begin{pgfscope}%
\pgfpathrectangle{\pgfqpoint{1.254980in}{0.150000in}}{\pgfqpoint{5.490039in}{5.490039in}}%
\pgfusepath{clip}%
\pgfsetbuttcap%
\pgfsetroundjoin%
\definecolor{currentfill}{rgb}{0.282910,0.105393,0.426902}%
\pgfsetfillcolor{currentfill}%
\pgfsetfillopacity{0.700000}%
\pgfsetlinewidth{0.000000pt}%
\definecolor{currentstroke}{rgb}{0.000000,0.000000,0.000000}%
\pgfsetstrokecolor{currentstroke}%
\pgfsetdash{}{0pt}%
\pgfpathmoveto{\pgfqpoint{2.880650in}{2.291679in}}%
\pgfpathlineto{\pgfqpoint{2.893526in}{2.284780in}}%
\pgfpathlineto{\pgfqpoint{2.906407in}{2.277923in}}%
\pgfpathlineto{\pgfqpoint{2.919290in}{2.271108in}}%
\pgfpathlineto{\pgfqpoint{2.932176in}{2.264333in}}%
\pgfpathlineto{\pgfqpoint{2.924073in}{2.262572in}}%
\pgfpathlineto{\pgfqpoint{2.915958in}{2.261012in}}%
\pgfpathlineto{\pgfqpoint{2.907830in}{2.259658in}}%
\pgfpathlineto{\pgfqpoint{2.899689in}{2.258516in}}%
\pgfpathlineto{\pgfqpoint{2.886778in}{2.265513in}}%
\pgfpathlineto{\pgfqpoint{2.873870in}{2.272550in}}%
\pgfpathlineto{\pgfqpoint{2.860965in}{2.279628in}}%
\pgfpathlineto{\pgfqpoint{2.848063in}{2.286748in}}%
\pgfpathlineto{\pgfqpoint{2.856229in}{2.287663in}}%
\pgfpathlineto{\pgfqpoint{2.864382in}{2.288794in}}%
\pgfpathlineto{\pgfqpoint{2.872522in}{2.290134in}}%
\pgfpathlineto{\pgfqpoint{2.880650in}{2.291679in}}%
\pgfpathclose%
\pgfusepath{fill}%
\end{pgfscope}%
\begin{pgfscope}%
\pgfpathrectangle{\pgfqpoint{1.254980in}{0.150000in}}{\pgfqpoint{5.490039in}{5.490039in}}%
\pgfusepath{clip}%
\pgfsetbuttcap%
\pgfsetroundjoin%
\definecolor{currentfill}{rgb}{0.280267,0.073417,0.397163}%
\pgfsetfillcolor{currentfill}%
\pgfsetfillopacity{0.700000}%
\pgfsetlinewidth{0.000000pt}%
\definecolor{currentstroke}{rgb}{0.000000,0.000000,0.000000}%
\pgfsetstrokecolor{currentstroke}%
\pgfsetdash{}{0pt}%
\pgfpathmoveto{\pgfqpoint{4.744704in}{2.235876in}}%
\pgfpathlineto{\pgfqpoint{4.757952in}{2.234010in}}%
\pgfpathlineto{\pgfqpoint{4.771207in}{2.232169in}}%
\pgfpathlineto{\pgfqpoint{4.784469in}{2.230353in}}%
\pgfpathlineto{\pgfqpoint{4.797738in}{2.228562in}}%
\pgfpathlineto{\pgfqpoint{4.790447in}{2.220917in}}%
\pgfpathlineto{\pgfqpoint{4.783149in}{2.213218in}}%
\pgfpathlineto{\pgfqpoint{4.775846in}{2.205466in}}%
\pgfpathlineto{\pgfqpoint{4.768536in}{2.197660in}}%
\pgfpathlineto{\pgfqpoint{4.755255in}{2.199478in}}%
\pgfpathlineto{\pgfqpoint{4.741982in}{2.201321in}}%
\pgfpathlineto{\pgfqpoint{4.728715in}{2.203189in}}%
\pgfpathlineto{\pgfqpoint{4.715456in}{2.205082in}}%
\pgfpathlineto{\pgfqpoint{4.722777in}{2.212856in}}%
\pgfpathlineto{\pgfqpoint{4.730092in}{2.220580in}}%
\pgfpathlineto{\pgfqpoint{4.737401in}{2.228253in}}%
\pgfpathlineto{\pgfqpoint{4.744704in}{2.235876in}}%
\pgfpathclose%
\pgfusepath{fill}%
\end{pgfscope}%
\begin{pgfscope}%
\pgfpathrectangle{\pgfqpoint{1.254980in}{0.150000in}}{\pgfqpoint{5.490039in}{5.490039in}}%
\pgfusepath{clip}%
\pgfsetbuttcap%
\pgfsetroundjoin%
\definecolor{currentfill}{rgb}{0.283072,0.130895,0.449241}%
\pgfsetfillcolor{currentfill}%
\pgfsetfillopacity{0.700000}%
\pgfsetlinewidth{0.000000pt}%
\definecolor{currentstroke}{rgb}{0.000000,0.000000,0.000000}%
\pgfsetstrokecolor{currentstroke}%
\pgfsetdash{}{0pt}%
\pgfpathmoveto{\pgfqpoint{5.261487in}{2.333519in}}%
\pgfpathlineto{\pgfqpoint{5.274883in}{2.332273in}}%
\pgfpathlineto{\pgfqpoint{5.288287in}{2.331051in}}%
\pgfpathlineto{\pgfqpoint{5.301699in}{2.329853in}}%
\pgfpathlineto{\pgfqpoint{5.315119in}{2.328679in}}%
\pgfpathlineto{\pgfqpoint{5.308047in}{2.322314in}}%
\pgfpathlineto{\pgfqpoint{5.300968in}{2.315887in}}%
\pgfpathlineto{\pgfqpoint{5.293882in}{2.309397in}}%
\pgfpathlineto{\pgfqpoint{5.286789in}{2.302843in}}%
\pgfpathlineto{\pgfqpoint{5.273355in}{2.303979in}}%
\pgfpathlineto{\pgfqpoint{5.259930in}{2.305139in}}%
\pgfpathlineto{\pgfqpoint{5.246512in}{2.306322in}}%
\pgfpathlineto{\pgfqpoint{5.233102in}{2.307530in}}%
\pgfpathlineto{\pgfqpoint{5.240208in}{2.314118in}}%
\pgfpathlineto{\pgfqpoint{5.247308in}{2.320644in}}%
\pgfpathlineto{\pgfqpoint{5.254401in}{2.327111in}}%
\pgfpathlineto{\pgfqpoint{5.261487in}{2.333519in}}%
\pgfpathclose%
\pgfusepath{fill}%
\end{pgfscope}%
\begin{pgfscope}%
\pgfpathrectangle{\pgfqpoint{1.254980in}{0.150000in}}{\pgfqpoint{5.490039in}{5.490039in}}%
\pgfusepath{clip}%
\pgfsetbuttcap%
\pgfsetroundjoin%
\definecolor{currentfill}{rgb}{0.279566,0.067836,0.391917}%
\pgfsetfillcolor{currentfill}%
\pgfsetfillopacity{0.700000}%
\pgfsetlinewidth{0.000000pt}%
\definecolor{currentstroke}{rgb}{0.000000,0.000000,0.000000}%
\pgfsetstrokecolor{currentstroke}%
\pgfsetdash{}{0pt}%
\pgfpathmoveto{\pgfqpoint{3.067500in}{2.222161in}}%
\pgfpathlineto{\pgfqpoint{3.080395in}{2.215945in}}%
\pgfpathlineto{\pgfqpoint{3.093294in}{2.209767in}}%
\pgfpathlineto{\pgfqpoint{3.106197in}{2.203627in}}%
\pgfpathlineto{\pgfqpoint{3.119104in}{2.197523in}}%
\pgfpathlineto{\pgfqpoint{3.111114in}{2.194407in}}%
\pgfpathlineto{\pgfqpoint{3.103115in}{2.191460in}}%
\pgfpathlineto{\pgfqpoint{3.095104in}{2.188687in}}%
\pgfpathlineto{\pgfqpoint{3.087082in}{2.186094in}}%
\pgfpathlineto{\pgfqpoint{3.074154in}{2.192405in}}%
\pgfpathlineto{\pgfqpoint{3.061229in}{2.198753in}}%
\pgfpathlineto{\pgfqpoint{3.048308in}{2.205138in}}%
\pgfpathlineto{\pgfqpoint{3.035390in}{2.211561in}}%
\pgfpathlineto{\pgfqpoint{3.043434in}{2.213942in}}%
\pgfpathlineto{\pgfqpoint{3.051467in}{2.216506in}}%
\pgfpathlineto{\pgfqpoint{3.059489in}{2.219247in}}%
\pgfpathlineto{\pgfqpoint{3.067500in}{2.222161in}}%
\pgfpathclose%
\pgfusepath{fill}%
\end{pgfscope}%
\begin{pgfscope}%
\pgfpathrectangle{\pgfqpoint{1.254980in}{0.150000in}}{\pgfqpoint{5.490039in}{5.490039in}}%
\pgfusepath{clip}%
\pgfsetbuttcap%
\pgfsetroundjoin%
\definecolor{currentfill}{rgb}{0.281887,0.150881,0.465405}%
\pgfsetfillcolor{currentfill}%
\pgfsetfillopacity{0.700000}%
\pgfsetlinewidth{0.000000pt}%
\definecolor{currentstroke}{rgb}{0.000000,0.000000,0.000000}%
\pgfsetstrokecolor{currentstroke}%
\pgfsetdash{}{0pt}%
\pgfpathmoveto{\pgfqpoint{2.693460in}{2.375572in}}%
\pgfpathlineto{\pgfqpoint{2.706329in}{2.367922in}}%
\pgfpathlineto{\pgfqpoint{2.719200in}{2.360318in}}%
\pgfpathlineto{\pgfqpoint{2.732074in}{2.352761in}}%
\pgfpathlineto{\pgfqpoint{2.744951in}{2.345249in}}%
\pgfpathlineto{\pgfqpoint{2.736718in}{2.345015in}}%
\pgfpathlineto{\pgfqpoint{2.728471in}{2.345015in}}%
\pgfpathlineto{\pgfqpoint{2.720208in}{2.345254in}}%
\pgfpathlineto{\pgfqpoint{2.711931in}{2.345740in}}%
\pgfpathlineto{\pgfqpoint{2.699026in}{2.353488in}}%
\pgfpathlineto{\pgfqpoint{2.686124in}{2.361281in}}%
\pgfpathlineto{\pgfqpoint{2.673225in}{2.369120in}}%
\pgfpathlineto{\pgfqpoint{2.660327in}{2.377006in}}%
\pgfpathlineto{\pgfqpoint{2.668634in}{2.376280in}}%
\pgfpathlineto{\pgfqpoint{2.676924in}{2.375803in}}%
\pgfpathlineto{\pgfqpoint{2.685199in}{2.375569in}}%
\pgfpathlineto{\pgfqpoint{2.693460in}{2.375572in}}%
\pgfpathclose%
\pgfusepath{fill}%
\end{pgfscope}%
\begin{pgfscope}%
\pgfpathrectangle{\pgfqpoint{1.254980in}{0.150000in}}{\pgfqpoint{5.490039in}{5.490039in}}%
\pgfusepath{clip}%
\pgfsetbuttcap%
\pgfsetroundjoin%
\definecolor{currentfill}{rgb}{0.271305,0.019942,0.347269}%
\pgfsetfillcolor{currentfill}%
\pgfsetfillopacity{0.700000}%
\pgfsetlinewidth{0.000000pt}%
\definecolor{currentstroke}{rgb}{0.000000,0.000000,0.000000}%
\pgfsetstrokecolor{currentstroke}%
\pgfsetdash{}{0pt}%
\pgfpathmoveto{\pgfqpoint{4.227887in}{2.144280in}}%
\pgfpathlineto{\pgfqpoint{4.240998in}{2.141437in}}%
\pgfpathlineto{\pgfqpoint{4.254117in}{2.138620in}}%
\pgfpathlineto{\pgfqpoint{4.267241in}{2.135829in}}%
\pgfpathlineto{\pgfqpoint{4.280372in}{2.133064in}}%
\pgfpathlineto{\pgfqpoint{4.272890in}{2.125080in}}%
\pgfpathlineto{\pgfqpoint{4.265402in}{2.117083in}}%
\pgfpathlineto{\pgfqpoint{4.257908in}{2.109078in}}%
\pgfpathlineto{\pgfqpoint{4.250410in}{2.101065in}}%
\pgfpathlineto{\pgfqpoint{4.237267in}{2.103921in}}%
\pgfpathlineto{\pgfqpoint{4.224131in}{2.106803in}}%
\pgfpathlineto{\pgfqpoint{4.211001in}{2.109711in}}%
\pgfpathlineto{\pgfqpoint{4.197878in}{2.112646in}}%
\pgfpathlineto{\pgfqpoint{4.205388in}{2.120563in}}%
\pgfpathlineto{\pgfqpoint{4.212893in}{2.128475in}}%
\pgfpathlineto{\pgfqpoint{4.220393in}{2.136382in}}%
\pgfpathlineto{\pgfqpoint{4.227887in}{2.144280in}}%
\pgfpathclose%
\pgfusepath{fill}%
\end{pgfscope}%
\begin{pgfscope}%
\pgfpathrectangle{\pgfqpoint{1.254980in}{0.150000in}}{\pgfqpoint{5.490039in}{5.490039in}}%
\pgfusepath{clip}%
\pgfsetbuttcap%
\pgfsetroundjoin%
\definecolor{currentfill}{rgb}{0.281887,0.150881,0.465405}%
\pgfsetfillcolor{currentfill}%
\pgfsetfillopacity{0.700000}%
\pgfsetlinewidth{0.000000pt}%
\definecolor{currentstroke}{rgb}{0.000000,0.000000,0.000000}%
\pgfsetstrokecolor{currentstroke}%
\pgfsetdash{}{0pt}%
\pgfpathmoveto{\pgfqpoint{5.478852in}{2.368142in}}%
\pgfpathlineto{\pgfqpoint{5.492313in}{2.367065in}}%
\pgfpathlineto{\pgfqpoint{5.505783in}{2.366012in}}%
\pgfpathlineto{\pgfqpoint{5.519261in}{2.364983in}}%
\pgfpathlineto{\pgfqpoint{5.532746in}{2.363976in}}%
\pgfpathlineto{\pgfqpoint{5.525776in}{2.358243in}}%
\pgfpathlineto{\pgfqpoint{5.518799in}{2.352454in}}%
\pgfpathlineto{\pgfqpoint{5.511814in}{2.346608in}}%
\pgfpathlineto{\pgfqpoint{5.504822in}{2.340702in}}%
\pgfpathlineto{\pgfqpoint{5.491321in}{2.341643in}}%
\pgfpathlineto{\pgfqpoint{5.477827in}{2.342608in}}%
\pgfpathlineto{\pgfqpoint{5.464342in}{2.343597in}}%
\pgfpathlineto{\pgfqpoint{5.450865in}{2.344609in}}%
\pgfpathlineto{\pgfqpoint{5.457872in}{2.350575in}}%
\pgfpathlineto{\pgfqpoint{5.464872in}{2.356484in}}%
\pgfpathlineto{\pgfqpoint{5.471866in}{2.362339in}}%
\pgfpathlineto{\pgfqpoint{5.478852in}{2.368142in}}%
\pgfpathclose%
\pgfusepath{fill}%
\end{pgfscope}%
\begin{pgfscope}%
\pgfpathrectangle{\pgfqpoint{1.254980in}{0.150000in}}{\pgfqpoint{5.490039in}{5.490039in}}%
\pgfusepath{clip}%
\pgfsetbuttcap%
\pgfsetroundjoin%
\definecolor{currentfill}{rgb}{0.267004,0.004874,0.329415}%
\pgfsetfillcolor{currentfill}%
\pgfsetfillopacity{0.700000}%
\pgfsetlinewidth{0.000000pt}%
\definecolor{currentstroke}{rgb}{0.000000,0.000000,0.000000}%
\pgfsetstrokecolor{currentstroke}%
\pgfsetdash{}{0pt}%
\pgfpathmoveto{\pgfqpoint{4.010653in}{2.119526in}}%
\pgfpathlineto{\pgfqpoint{4.023713in}{2.116165in}}%
\pgfpathlineto{\pgfqpoint{4.036780in}{2.112832in}}%
\pgfpathlineto{\pgfqpoint{4.049852in}{2.109526in}}%
\pgfpathlineto{\pgfqpoint{4.062931in}{2.106247in}}%
\pgfpathlineto{\pgfqpoint{4.055370in}{2.098567in}}%
\pgfpathlineto{\pgfqpoint{4.047803in}{2.090903in}}%
\pgfpathlineto{\pgfqpoint{4.040231in}{2.083257in}}%
\pgfpathlineto{\pgfqpoint{4.032653in}{2.075633in}}%
\pgfpathlineto{\pgfqpoint{4.019562in}{2.079030in}}%
\pgfpathlineto{\pgfqpoint{4.006477in}{2.082453in}}%
\pgfpathlineto{\pgfqpoint{3.993398in}{2.085903in}}%
\pgfpathlineto{\pgfqpoint{3.980326in}{2.089381in}}%
\pgfpathlineto{\pgfqpoint{3.987916in}{2.096883in}}%
\pgfpathlineto{\pgfqpoint{3.995500in}{2.104410in}}%
\pgfpathlineto{\pgfqpoint{4.003079in}{2.111958in}}%
\pgfpathlineto{\pgfqpoint{4.010653in}{2.119526in}}%
\pgfpathclose%
\pgfusepath{fill}%
\end{pgfscope}%
\begin{pgfscope}%
\pgfpathrectangle{\pgfqpoint{1.254980in}{0.150000in}}{\pgfqpoint{5.490039in}{5.490039in}}%
\pgfusepath{clip}%
\pgfsetbuttcap%
\pgfsetroundjoin%
\definecolor{currentfill}{rgb}{0.268510,0.009605,0.335427}%
\pgfsetfillcolor{currentfill}%
\pgfsetfillopacity{0.700000}%
\pgfsetlinewidth{0.000000pt}%
\definecolor{currentstroke}{rgb}{0.000000,0.000000,0.000000}%
\pgfsetstrokecolor{currentstroke}%
\pgfsetdash{}{0pt}%
\pgfpathmoveto{\pgfqpoint{3.523883in}{2.123280in}}%
\pgfpathlineto{\pgfqpoint{3.536845in}{2.118562in}}%
\pgfpathlineto{\pgfqpoint{3.549811in}{2.113875in}}%
\pgfpathlineto{\pgfqpoint{3.562783in}{2.109218in}}%
\pgfpathlineto{\pgfqpoint{3.575760in}{2.104593in}}%
\pgfpathlineto{\pgfqpoint{3.568004in}{2.098678in}}%
\pgfpathlineto{\pgfqpoint{3.560240in}{2.092855in}}%
\pgfpathlineto{\pgfqpoint{3.552469in}{2.087127in}}%
\pgfpathlineto{\pgfqpoint{3.544691in}{2.081498in}}%
\pgfpathlineto{\pgfqpoint{3.531698in}{2.086292in}}%
\pgfpathlineto{\pgfqpoint{3.518710in}{2.091116in}}%
\pgfpathlineto{\pgfqpoint{3.505727in}{2.095972in}}%
\pgfpathlineto{\pgfqpoint{3.492749in}{2.100858in}}%
\pgfpathlineto{\pgfqpoint{3.500544in}{2.106313in}}%
\pgfpathlineto{\pgfqpoint{3.508331in}{2.111872in}}%
\pgfpathlineto{\pgfqpoint{3.516111in}{2.117529in}}%
\pgfpathlineto{\pgfqpoint{3.523883in}{2.123280in}}%
\pgfpathclose%
\pgfusepath{fill}%
\end{pgfscope}%
\begin{pgfscope}%
\pgfpathrectangle{\pgfqpoint{1.254980in}{0.150000in}}{\pgfqpoint{5.490039in}{5.490039in}}%
\pgfusepath{clip}%
\pgfsetbuttcap%
\pgfsetroundjoin%
\definecolor{currentfill}{rgb}{0.274952,0.037752,0.364543}%
\pgfsetfillcolor{currentfill}%
\pgfsetfillopacity{0.700000}%
\pgfsetlinewidth{0.000000pt}%
\definecolor{currentstroke}{rgb}{0.000000,0.000000,0.000000}%
\pgfsetstrokecolor{currentstroke}%
\pgfsetdash{}{0pt}%
\pgfpathmoveto{\pgfqpoint{4.445146in}{2.176305in}}%
\pgfpathlineto{\pgfqpoint{4.458315in}{2.173918in}}%
\pgfpathlineto{\pgfqpoint{4.471490in}{2.171556in}}%
\pgfpathlineto{\pgfqpoint{4.484673in}{2.169219in}}%
\pgfpathlineto{\pgfqpoint{4.497862in}{2.166908in}}%
\pgfpathlineto{\pgfqpoint{4.490457in}{2.158889in}}%
\pgfpathlineto{\pgfqpoint{4.483045in}{2.150838in}}%
\pgfpathlineto{\pgfqpoint{4.475629in}{2.142754in}}%
\pgfpathlineto{\pgfqpoint{4.468207in}{2.134640in}}%
\pgfpathlineto{\pgfqpoint{4.455006in}{2.137017in}}%
\pgfpathlineto{\pgfqpoint{4.441813in}{2.139420in}}%
\pgfpathlineto{\pgfqpoint{4.428626in}{2.141848in}}%
\pgfpathlineto{\pgfqpoint{4.415446in}{2.144301in}}%
\pgfpathlineto{\pgfqpoint{4.422879in}{2.152345in}}%
\pgfpathlineto{\pgfqpoint{4.430307in}{2.160361in}}%
\pgfpathlineto{\pgfqpoint{4.437729in}{2.168348in}}%
\pgfpathlineto{\pgfqpoint{4.445146in}{2.176305in}}%
\pgfpathclose%
\pgfusepath{fill}%
\end{pgfscope}%
\begin{pgfscope}%
\pgfpathrectangle{\pgfqpoint{1.254980in}{0.150000in}}{\pgfqpoint{5.490039in}{5.490039in}}%
\pgfusepath{clip}%
\pgfsetbuttcap%
\pgfsetroundjoin%
\definecolor{currentfill}{rgb}{0.282656,0.100196,0.422160}%
\pgfsetfillcolor{currentfill}%
\pgfsetfillopacity{0.700000}%
\pgfsetlinewidth{0.000000pt}%
\definecolor{currentstroke}{rgb}{0.000000,0.000000,0.000000}%
\pgfsetstrokecolor{currentstroke}%
\pgfsetdash{}{0pt}%
\pgfpathmoveto{\pgfqpoint{4.962082in}{2.274458in}}%
\pgfpathlineto{\pgfqpoint{4.975394in}{2.272902in}}%
\pgfpathlineto{\pgfqpoint{4.988714in}{2.271371in}}%
\pgfpathlineto{\pgfqpoint{5.002042in}{2.269863in}}%
\pgfpathlineto{\pgfqpoint{5.015377in}{2.268380in}}%
\pgfpathlineto{\pgfqpoint{5.008171in}{2.261174in}}%
\pgfpathlineto{\pgfqpoint{5.000959in}{2.253907in}}%
\pgfpathlineto{\pgfqpoint{4.993741in}{2.246576in}}%
\pgfpathlineto{\pgfqpoint{4.986516in}{2.239183in}}%
\pgfpathlineto{\pgfqpoint{4.973168in}{2.240668in}}%
\pgfpathlineto{\pgfqpoint{4.959829in}{2.242176in}}%
\pgfpathlineto{\pgfqpoint{4.946496in}{2.243709in}}%
\pgfpathlineto{\pgfqpoint{4.933172in}{2.245266in}}%
\pgfpathlineto{\pgfqpoint{4.940409in}{2.252653in}}%
\pgfpathlineto{\pgfqpoint{4.947639in}{2.259980in}}%
\pgfpathlineto{\pgfqpoint{4.954864in}{2.267249in}}%
\pgfpathlineto{\pgfqpoint{4.962082in}{2.274458in}}%
\pgfpathclose%
\pgfusepath{fill}%
\end{pgfscope}%
\begin{pgfscope}%
\pgfpathrectangle{\pgfqpoint{1.254980in}{0.150000in}}{\pgfqpoint{5.490039in}{5.490039in}}%
\pgfusepath{clip}%
\pgfsetbuttcap%
\pgfsetroundjoin%
\definecolor{currentfill}{rgb}{0.271305,0.019942,0.347269}%
\pgfsetfillcolor{currentfill}%
\pgfsetfillopacity{0.700000}%
\pgfsetlinewidth{0.000000pt}%
\definecolor{currentstroke}{rgb}{0.000000,0.000000,0.000000}%
\pgfsetstrokecolor{currentstroke}%
\pgfsetdash{}{0pt}%
\pgfpathmoveto{\pgfqpoint{3.389096in}{2.141086in}}%
\pgfpathlineto{\pgfqpoint{3.402036in}{2.135945in}}%
\pgfpathlineto{\pgfqpoint{3.414981in}{2.130836in}}%
\pgfpathlineto{\pgfqpoint{3.427930in}{2.125760in}}%
\pgfpathlineto{\pgfqpoint{3.440885in}{2.120716in}}%
\pgfpathlineto{\pgfqpoint{3.433065in}{2.115544in}}%
\pgfpathlineto{\pgfqpoint{3.425237in}{2.110486in}}%
\pgfpathlineto{\pgfqpoint{3.417402in}{2.105546in}}%
\pgfpathlineto{\pgfqpoint{3.409557in}{2.100731in}}%
\pgfpathlineto{\pgfqpoint{3.396586in}{2.105955in}}%
\pgfpathlineto{\pgfqpoint{3.383618in}{2.111212in}}%
\pgfpathlineto{\pgfqpoint{3.370656in}{2.116502in}}%
\pgfpathlineto{\pgfqpoint{3.357697in}{2.121824in}}%
\pgfpathlineto{\pgfqpoint{3.365560in}{2.126454in}}%
\pgfpathlineto{\pgfqpoint{3.373413in}{2.131210in}}%
\pgfpathlineto{\pgfqpoint{3.381259in}{2.136089in}}%
\pgfpathlineto{\pgfqpoint{3.389096in}{2.141086in}}%
\pgfpathclose%
\pgfusepath{fill}%
\end{pgfscope}%
\begin{pgfscope}%
\pgfpathrectangle{\pgfqpoint{1.254980in}{0.150000in}}{\pgfqpoint{5.490039in}{5.490039in}}%
\pgfusepath{clip}%
\pgfsetbuttcap%
\pgfsetroundjoin%
\definecolor{currentfill}{rgb}{0.267004,0.004874,0.329415}%
\pgfsetfillcolor{currentfill}%
\pgfsetfillopacity{0.700000}%
\pgfsetlinewidth{0.000000pt}%
\definecolor{currentstroke}{rgb}{0.000000,0.000000,0.000000}%
\pgfsetstrokecolor{currentstroke}%
\pgfsetdash{}{0pt}%
\pgfpathmoveto{\pgfqpoint{3.658611in}{2.111506in}}%
\pgfpathlineto{\pgfqpoint{3.671599in}{2.107187in}}%
\pgfpathlineto{\pgfqpoint{3.684591in}{2.102897in}}%
\pgfpathlineto{\pgfqpoint{3.697589in}{2.098637in}}%
\pgfpathlineto{\pgfqpoint{3.710592in}{2.094406in}}%
\pgfpathlineto{\pgfqpoint{3.702894in}{2.087864in}}%
\pgfpathlineto{\pgfqpoint{3.695189in}{2.081392in}}%
\pgfpathlineto{\pgfqpoint{3.687477in}{2.074993in}}%
\pgfpathlineto{\pgfqpoint{3.679759in}{2.068672in}}%
\pgfpathlineto{\pgfqpoint{3.666741in}{2.073058in}}%
\pgfpathlineto{\pgfqpoint{3.653728in}{2.077473in}}%
\pgfpathlineto{\pgfqpoint{3.640720in}{2.081918in}}%
\pgfpathlineto{\pgfqpoint{3.627718in}{2.086393in}}%
\pgfpathlineto{\pgfqpoint{3.635451in}{2.092555in}}%
\pgfpathlineto{\pgfqpoint{3.643178in}{2.098797in}}%
\pgfpathlineto{\pgfqpoint{3.650898in}{2.105115in}}%
\pgfpathlineto{\pgfqpoint{3.658611in}{2.111506in}}%
\pgfpathclose%
\pgfusepath{fill}%
\end{pgfscope}%
\begin{pgfscope}%
\pgfpathrectangle{\pgfqpoint{1.254980in}{0.150000in}}{\pgfqpoint{5.490039in}{5.490039in}}%
\pgfusepath{clip}%
\pgfsetbuttcap%
\pgfsetroundjoin%
\definecolor{currentfill}{rgb}{0.280255,0.165693,0.476498}%
\pgfsetfillcolor{currentfill}%
\pgfsetfillopacity{0.700000}%
\pgfsetlinewidth{0.000000pt}%
\definecolor{currentstroke}{rgb}{0.000000,0.000000,0.000000}%
\pgfsetstrokecolor{currentstroke}%
\pgfsetdash{}{0pt}%
\pgfpathmoveto{\pgfqpoint{5.696172in}{2.399619in}}%
\pgfpathlineto{\pgfqpoint{5.709698in}{2.398656in}}%
\pgfpathlineto{\pgfqpoint{5.723232in}{2.397716in}}%
\pgfpathlineto{\pgfqpoint{5.736774in}{2.396799in}}%
\pgfpathlineto{\pgfqpoint{5.750325in}{2.395906in}}%
\pgfpathlineto{\pgfqpoint{5.743462in}{2.390806in}}%
\pgfpathlineto{\pgfqpoint{5.736592in}{2.385663in}}%
\pgfpathlineto{\pgfqpoint{5.729715in}{2.380474in}}%
\pgfpathlineto{\pgfqpoint{5.722831in}{2.375236in}}%
\pgfpathlineto{\pgfqpoint{5.709263in}{2.376038in}}%
\pgfpathlineto{\pgfqpoint{5.695703in}{2.376864in}}%
\pgfpathlineto{\pgfqpoint{5.682151in}{2.377713in}}%
\pgfpathlineto{\pgfqpoint{5.668607in}{2.378585in}}%
\pgfpathlineto{\pgfqpoint{5.675509in}{2.383909in}}%
\pgfpathlineto{\pgfqpoint{5.682404in}{2.389188in}}%
\pgfpathlineto{\pgfqpoint{5.689291in}{2.394423in}}%
\pgfpathlineto{\pgfqpoint{5.696172in}{2.399619in}}%
\pgfpathclose%
\pgfusepath{fill}%
\end{pgfscope}%
\begin{pgfscope}%
\pgfpathrectangle{\pgfqpoint{1.254980in}{0.150000in}}{\pgfqpoint{5.490039in}{5.490039in}}%
\pgfusepath{clip}%
\pgfsetbuttcap%
\pgfsetroundjoin%
\definecolor{currentfill}{rgb}{0.277134,0.185228,0.489898}%
\pgfsetfillcolor{currentfill}%
\pgfsetfillopacity{0.700000}%
\pgfsetlinewidth{0.000000pt}%
\definecolor{currentstroke}{rgb}{0.000000,0.000000,0.000000}%
\pgfsetstrokecolor{currentstroke}%
\pgfsetdash{}{0pt}%
\pgfpathmoveto{\pgfqpoint{5.913391in}{2.427475in}}%
\pgfpathlineto{\pgfqpoint{5.926979in}{2.426571in}}%
\pgfpathlineto{\pgfqpoint{5.940575in}{2.425689in}}%
\pgfpathlineto{\pgfqpoint{5.954180in}{2.424831in}}%
\pgfpathlineto{\pgfqpoint{5.967794in}{2.423995in}}%
\pgfpathlineto{\pgfqpoint{5.961042in}{2.419485in}}%
\pgfpathlineto{\pgfqpoint{5.954285in}{2.414950in}}%
\pgfpathlineto{\pgfqpoint{5.947520in}{2.410386in}}%
\pgfpathlineto{\pgfqpoint{5.940749in}{2.405789in}}%
\pgfpathlineto{\pgfqpoint{5.927115in}{2.406507in}}%
\pgfpathlineto{\pgfqpoint{5.913491in}{2.407248in}}%
\pgfpathlineto{\pgfqpoint{5.899874in}{2.408013in}}%
\pgfpathlineto{\pgfqpoint{5.886267in}{2.408800in}}%
\pgfpathlineto{\pgfqpoint{5.893058in}{2.413509in}}%
\pgfpathlineto{\pgfqpoint{5.899842in}{2.418189in}}%
\pgfpathlineto{\pgfqpoint{5.906620in}{2.422843in}}%
\pgfpathlineto{\pgfqpoint{5.913391in}{2.427475in}}%
\pgfpathclose%
\pgfusepath{fill}%
\end{pgfscope}%
\begin{pgfscope}%
\pgfpathrectangle{\pgfqpoint{1.254980in}{0.150000in}}{\pgfqpoint{5.490039in}{5.490039in}}%
\pgfusepath{clip}%
\pgfsetbuttcap%
\pgfsetroundjoin%
\definecolor{currentfill}{rgb}{0.279566,0.067836,0.391917}%
\pgfsetfillcolor{currentfill}%
\pgfsetfillopacity{0.700000}%
\pgfsetlinewidth{0.000000pt}%
\definecolor{currentstroke}{rgb}{0.000000,0.000000,0.000000}%
\pgfsetstrokecolor{currentstroke}%
\pgfsetdash{}{0pt}%
\pgfpathmoveto{\pgfqpoint{4.662492in}{2.212900in}}%
\pgfpathlineto{\pgfqpoint{4.675723in}{2.210908in}}%
\pgfpathlineto{\pgfqpoint{4.688960in}{2.208941in}}%
\pgfpathlineto{\pgfqpoint{4.702205in}{2.206999in}}%
\pgfpathlineto{\pgfqpoint{4.715456in}{2.205082in}}%
\pgfpathlineto{\pgfqpoint{4.708130in}{2.197257in}}%
\pgfpathlineto{\pgfqpoint{4.700797in}{2.189383in}}%
\pgfpathlineto{\pgfqpoint{4.693459in}{2.181461in}}%
\pgfpathlineto{\pgfqpoint{4.686115in}{2.173489in}}%
\pgfpathlineto{\pgfqpoint{4.672852in}{2.175447in}}%
\pgfpathlineto{\pgfqpoint{4.659596in}{2.177429in}}%
\pgfpathlineto{\pgfqpoint{4.646347in}{2.179436in}}%
\pgfpathlineto{\pgfqpoint{4.633106in}{2.181468in}}%
\pgfpathlineto{\pgfqpoint{4.640461in}{2.189394in}}%
\pgfpathlineto{\pgfqpoint{4.647811in}{2.197276in}}%
\pgfpathlineto{\pgfqpoint{4.655154in}{2.205111in}}%
\pgfpathlineto{\pgfqpoint{4.662492in}{2.212900in}}%
\pgfpathclose%
\pgfusepath{fill}%
\end{pgfscope}%
\begin{pgfscope}%
\pgfpathrectangle{\pgfqpoint{1.254980in}{0.150000in}}{\pgfqpoint{5.490039in}{5.490039in}}%
\pgfusepath{clip}%
\pgfsetbuttcap%
\pgfsetroundjoin%
\definecolor{currentfill}{rgb}{0.274952,0.037752,0.364543}%
\pgfsetfillcolor{currentfill}%
\pgfsetfillopacity{0.700000}%
\pgfsetlinewidth{0.000000pt}%
\definecolor{currentstroke}{rgb}{0.000000,0.000000,0.000000}%
\pgfsetstrokecolor{currentstroke}%
\pgfsetdash{}{0pt}%
\pgfpathmoveto{\pgfqpoint{3.254192in}{2.165595in}}%
\pgfpathlineto{\pgfqpoint{3.267115in}{2.160005in}}%
\pgfpathlineto{\pgfqpoint{3.280042in}{2.154449in}}%
\pgfpathlineto{\pgfqpoint{3.292974in}{2.148928in}}%
\pgfpathlineto{\pgfqpoint{3.305910in}{2.143440in}}%
\pgfpathlineto{\pgfqpoint{3.298020in}{2.139131in}}%
\pgfpathlineto{\pgfqpoint{3.290121in}{2.134961in}}%
\pgfpathlineto{\pgfqpoint{3.282213in}{2.130935in}}%
\pgfpathlineto{\pgfqpoint{3.274295in}{2.127058in}}%
\pgfpathlineto{\pgfqpoint{3.261340in}{2.132739in}}%
\pgfpathlineto{\pgfqpoint{3.248389in}{2.138455in}}%
\pgfpathlineto{\pgfqpoint{3.235442in}{2.144204in}}%
\pgfpathlineto{\pgfqpoint{3.222499in}{2.149988in}}%
\pgfpathlineto{\pgfqpoint{3.230437in}{2.153666in}}%
\pgfpathlineto{\pgfqpoint{3.238365in}{2.157496in}}%
\pgfpathlineto{\pgfqpoint{3.246283in}{2.161474in}}%
\pgfpathlineto{\pgfqpoint{3.254192in}{2.165595in}}%
\pgfpathclose%
\pgfusepath{fill}%
\end{pgfscope}%
\begin{pgfscope}%
\pgfpathrectangle{\pgfqpoint{1.254980in}{0.150000in}}{\pgfqpoint{5.490039in}{5.490039in}}%
\pgfusepath{clip}%
\pgfsetbuttcap%
\pgfsetroundjoin%
\definecolor{currentfill}{rgb}{0.267004,0.004874,0.329415}%
\pgfsetfillcolor{currentfill}%
\pgfsetfillopacity{0.700000}%
\pgfsetlinewidth{0.000000pt}%
\definecolor{currentstroke}{rgb}{0.000000,0.000000,0.000000}%
\pgfsetstrokecolor{currentstroke}%
\pgfsetdash{}{0pt}%
\pgfpathmoveto{\pgfqpoint{3.793333in}{2.105133in}}%
\pgfpathlineto{\pgfqpoint{3.806350in}{2.101190in}}%
\pgfpathlineto{\pgfqpoint{3.819372in}{2.097275in}}%
\pgfpathlineto{\pgfqpoint{3.832400in}{2.093389in}}%
\pgfpathlineto{\pgfqpoint{3.845433in}{2.089531in}}%
\pgfpathlineto{\pgfqpoint{3.837788in}{2.082472in}}%
\pgfpathlineto{\pgfqpoint{3.830136in}{2.075461in}}%
\pgfpathlineto{\pgfqpoint{3.822479in}{2.068504in}}%
\pgfpathlineto{\pgfqpoint{3.814815in}{2.061602in}}%
\pgfpathlineto{\pgfqpoint{3.801768in}{2.065602in}}%
\pgfpathlineto{\pgfqpoint{3.788726in}{2.069631in}}%
\pgfpathlineto{\pgfqpoint{3.775690in}{2.073688in}}%
\pgfpathlineto{\pgfqpoint{3.762660in}{2.077774in}}%
\pgfpathlineto{\pgfqpoint{3.770338in}{2.084528in}}%
\pgfpathlineto{\pgfqpoint{3.778009in}{2.091342in}}%
\pgfpathlineto{\pgfqpoint{3.785674in}{2.098211in}}%
\pgfpathlineto{\pgfqpoint{3.793333in}{2.105133in}}%
\pgfpathclose%
\pgfusepath{fill}%
\end{pgfscope}%
\begin{pgfscope}%
\pgfpathrectangle{\pgfqpoint{1.254980in}{0.150000in}}{\pgfqpoint{5.490039in}{5.490039in}}%
\pgfusepath{clip}%
\pgfsetbuttcap%
\pgfsetroundjoin%
\definecolor{currentfill}{rgb}{0.283229,0.120777,0.440584}%
\pgfsetfillcolor{currentfill}%
\pgfsetfillopacity{0.700000}%
\pgfsetlinewidth{0.000000pt}%
\definecolor{currentstroke}{rgb}{0.000000,0.000000,0.000000}%
\pgfsetstrokecolor{currentstroke}%
\pgfsetdash{}{0pt}%
\pgfpathmoveto{\pgfqpoint{5.179540in}{2.312600in}}%
\pgfpathlineto{\pgfqpoint{5.192919in}{2.311296in}}%
\pgfpathlineto{\pgfqpoint{5.206305in}{2.310017in}}%
\pgfpathlineto{\pgfqpoint{5.219699in}{2.308761in}}%
\pgfpathlineto{\pgfqpoint{5.233102in}{2.307530in}}%
\pgfpathlineto{\pgfqpoint{5.225988in}{2.300879in}}%
\pgfpathlineto{\pgfqpoint{5.218868in}{2.294165in}}%
\pgfpathlineto{\pgfqpoint{5.211741in}{2.287384in}}%
\pgfpathlineto{\pgfqpoint{5.204607in}{2.280538in}}%
\pgfpathlineto{\pgfqpoint{5.191191in}{2.281744in}}%
\pgfpathlineto{\pgfqpoint{5.177784in}{2.282974in}}%
\pgfpathlineto{\pgfqpoint{5.164384in}{2.284228in}}%
\pgfpathlineto{\pgfqpoint{5.150992in}{2.285506in}}%
\pgfpathlineto{\pgfqpoint{5.158139in}{2.292374in}}%
\pgfpathlineto{\pgfqpoint{5.165279in}{2.299177in}}%
\pgfpathlineto{\pgfqpoint{5.172413in}{2.305919in}}%
\pgfpathlineto{\pgfqpoint{5.179540in}{2.312600in}}%
\pgfpathclose%
\pgfusepath{fill}%
\end{pgfscope}%
\begin{pgfscope}%
\pgfpathrectangle{\pgfqpoint{1.254980in}{0.150000in}}{\pgfqpoint{5.490039in}{5.490039in}}%
\pgfusepath{clip}%
\pgfsetbuttcap%
\pgfsetroundjoin%
\definecolor{currentfill}{rgb}{0.282656,0.100196,0.422160}%
\pgfsetfillcolor{currentfill}%
\pgfsetfillopacity{0.700000}%
\pgfsetlinewidth{0.000000pt}%
\definecolor{currentstroke}{rgb}{0.000000,0.000000,0.000000}%
\pgfsetstrokecolor{currentstroke}%
\pgfsetdash{}{0pt}%
\pgfpathmoveto{\pgfqpoint{2.932176in}{2.264333in}}%
\pgfpathlineto{\pgfqpoint{2.945066in}{2.257599in}}%
\pgfpathlineto{\pgfqpoint{2.957959in}{2.250904in}}%
\pgfpathlineto{\pgfqpoint{2.970856in}{2.244250in}}%
\pgfpathlineto{\pgfqpoint{2.983756in}{2.237634in}}%
\pgfpathlineto{\pgfqpoint{2.975677in}{2.235658in}}%
\pgfpathlineto{\pgfqpoint{2.967586in}{2.233878in}}%
\pgfpathlineto{\pgfqpoint{2.959482in}{2.232302in}}%
\pgfpathlineto{\pgfqpoint{2.951366in}{2.230933in}}%
\pgfpathlineto{\pgfqpoint{2.938442in}{2.237769in}}%
\pgfpathlineto{\pgfqpoint{2.925521in}{2.244645in}}%
\pgfpathlineto{\pgfqpoint{2.912604in}{2.251561in}}%
\pgfpathlineto{\pgfqpoint{2.899689in}{2.258516in}}%
\pgfpathlineto{\pgfqpoint{2.907830in}{2.259658in}}%
\pgfpathlineto{\pgfqpoint{2.915958in}{2.261012in}}%
\pgfpathlineto{\pgfqpoint{2.924073in}{2.262572in}}%
\pgfpathlineto{\pgfqpoint{2.932176in}{2.264333in}}%
\pgfpathclose%
\pgfusepath{fill}%
\end{pgfscope}%
\begin{pgfscope}%
\pgfpathrectangle{\pgfqpoint{1.254980in}{0.150000in}}{\pgfqpoint{5.490039in}{5.490039in}}%
\pgfusepath{clip}%
\pgfsetbuttcap%
\pgfsetroundjoin%
\definecolor{currentfill}{rgb}{0.269944,0.014625,0.341379}%
\pgfsetfillcolor{currentfill}%
\pgfsetfillopacity{0.700000}%
\pgfsetlinewidth{0.000000pt}%
\definecolor{currentstroke}{rgb}{0.000000,0.000000,0.000000}%
\pgfsetstrokecolor{currentstroke}%
\pgfsetdash{}{0pt}%
\pgfpathmoveto{\pgfqpoint{4.145449in}{2.124649in}}%
\pgfpathlineto{\pgfqpoint{4.158547in}{2.121608in}}%
\pgfpathlineto{\pgfqpoint{4.171651in}{2.118594in}}%
\pgfpathlineto{\pgfqpoint{4.184761in}{2.115607in}}%
\pgfpathlineto{\pgfqpoint{4.197878in}{2.112646in}}%
\pgfpathlineto{\pgfqpoint{4.190363in}{2.104727in}}%
\pgfpathlineto{\pgfqpoint{4.182841in}{2.096808in}}%
\pgfpathlineto{\pgfqpoint{4.175315in}{2.088891in}}%
\pgfpathlineto{\pgfqpoint{4.167783in}{2.080980in}}%
\pgfpathlineto{\pgfqpoint{4.154655in}{2.084045in}}%
\pgfpathlineto{\pgfqpoint{4.141533in}{2.087137in}}%
\pgfpathlineto{\pgfqpoint{4.128417in}{2.090255in}}%
\pgfpathlineto{\pgfqpoint{4.115307in}{2.093400in}}%
\pgfpathlineto{\pgfqpoint{4.122851in}{2.101202in}}%
\pgfpathlineto{\pgfqpoint{4.130389in}{2.109013in}}%
\pgfpathlineto{\pgfqpoint{4.137922in}{2.116829in}}%
\pgfpathlineto{\pgfqpoint{4.145449in}{2.124649in}}%
\pgfpathclose%
\pgfusepath{fill}%
\end{pgfscope}%
\begin{pgfscope}%
\pgfpathrectangle{\pgfqpoint{1.254980in}{0.150000in}}{\pgfqpoint{5.490039in}{5.490039in}}%
\pgfusepath{clip}%
\pgfsetbuttcap%
\pgfsetroundjoin%
\definecolor{currentfill}{rgb}{0.281924,0.089666,0.412415}%
\pgfsetfillcolor{currentfill}%
\pgfsetfillopacity{0.700000}%
\pgfsetlinewidth{0.000000pt}%
\definecolor{currentstroke}{rgb}{0.000000,0.000000,0.000000}%
\pgfsetstrokecolor{currentstroke}%
\pgfsetdash{}{0pt}%
\pgfpathmoveto{\pgfqpoint{4.879948in}{2.251737in}}%
\pgfpathlineto{\pgfqpoint{4.893243in}{2.250083in}}%
\pgfpathlineto{\pgfqpoint{4.906545in}{2.248453in}}%
\pgfpathlineto{\pgfqpoint{4.919854in}{2.246847in}}%
\pgfpathlineto{\pgfqpoint{4.933172in}{2.245266in}}%
\pgfpathlineto{\pgfqpoint{4.925928in}{2.237818in}}%
\pgfpathlineto{\pgfqpoint{4.918679in}{2.230311in}}%
\pgfpathlineto{\pgfqpoint{4.911423in}{2.222743in}}%
\pgfpathlineto{\pgfqpoint{4.904161in}{2.215114in}}%
\pgfpathlineto{\pgfqpoint{4.890832in}{2.216710in}}%
\pgfpathlineto{\pgfqpoint{4.877510in}{2.218329in}}%
\pgfpathlineto{\pgfqpoint{4.864196in}{2.219974in}}%
\pgfpathlineto{\pgfqpoint{4.850890in}{2.221642in}}%
\pgfpathlineto{\pgfqpoint{4.858164in}{2.229252in}}%
\pgfpathlineto{\pgfqpoint{4.865431in}{2.236804in}}%
\pgfpathlineto{\pgfqpoint{4.872693in}{2.244299in}}%
\pgfpathlineto{\pgfqpoint{4.879948in}{2.251737in}}%
\pgfpathclose%
\pgfusepath{fill}%
\end{pgfscope}%
\begin{pgfscope}%
\pgfpathrectangle{\pgfqpoint{1.254980in}{0.150000in}}{\pgfqpoint{5.490039in}{5.490039in}}%
\pgfusepath{clip}%
\pgfsetbuttcap%
\pgfsetroundjoin%
\definecolor{currentfill}{rgb}{0.273809,0.031497,0.358853}%
\pgfsetfillcolor{currentfill}%
\pgfsetfillopacity{0.700000}%
\pgfsetlinewidth{0.000000pt}%
\definecolor{currentstroke}{rgb}{0.000000,0.000000,0.000000}%
\pgfsetstrokecolor{currentstroke}%
\pgfsetdash{}{0pt}%
\pgfpathmoveto{\pgfqpoint{4.362794in}{2.154372in}}%
\pgfpathlineto{\pgfqpoint{4.375947in}{2.151815in}}%
\pgfpathlineto{\pgfqpoint{4.389107in}{2.149285in}}%
\pgfpathlineto{\pgfqpoint{4.402273in}{2.146780in}}%
\pgfpathlineto{\pgfqpoint{4.415446in}{2.144301in}}%
\pgfpathlineto{\pgfqpoint{4.408008in}{2.136232in}}%
\pgfpathlineto{\pgfqpoint{4.400564in}{2.128137in}}%
\pgfpathlineto{\pgfqpoint{4.393114in}{2.120020in}}%
\pgfpathlineto{\pgfqpoint{4.385659in}{2.111881in}}%
\pgfpathlineto{\pgfqpoint{4.372475in}{2.114439in}}%
\pgfpathlineto{\pgfqpoint{4.359298in}{2.117022in}}%
\pgfpathlineto{\pgfqpoint{4.346127in}{2.119632in}}%
\pgfpathlineto{\pgfqpoint{4.332963in}{2.122266in}}%
\pgfpathlineto{\pgfqpoint{4.340429in}{2.130322in}}%
\pgfpathlineto{\pgfqpoint{4.347889in}{2.138359in}}%
\pgfpathlineto{\pgfqpoint{4.355344in}{2.146376in}}%
\pgfpathlineto{\pgfqpoint{4.362794in}{2.154372in}}%
\pgfpathclose%
\pgfusepath{fill}%
\end{pgfscope}%
\begin{pgfscope}%
\pgfpathrectangle{\pgfqpoint{1.254980in}{0.150000in}}{\pgfqpoint{5.490039in}{5.490039in}}%
\pgfusepath{clip}%
\pgfsetbuttcap%
\pgfsetroundjoin%
\definecolor{currentfill}{rgb}{0.282290,0.145912,0.461510}%
\pgfsetfillcolor{currentfill}%
\pgfsetfillopacity{0.700000}%
\pgfsetlinewidth{0.000000pt}%
\definecolor{currentstroke}{rgb}{0.000000,0.000000,0.000000}%
\pgfsetstrokecolor{currentstroke}%
\pgfsetdash{}{0pt}%
\pgfpathmoveto{\pgfqpoint{5.397037in}{2.348895in}}%
\pgfpathlineto{\pgfqpoint{5.410482in}{2.347788in}}%
\pgfpathlineto{\pgfqpoint{5.423935in}{2.346704in}}%
\pgfpathlineto{\pgfqpoint{5.437396in}{2.345645in}}%
\pgfpathlineto{\pgfqpoint{5.450865in}{2.344609in}}%
\pgfpathlineto{\pgfqpoint{5.443850in}{2.338585in}}%
\pgfpathlineto{\pgfqpoint{5.436829in}{2.332500in}}%
\pgfpathlineto{\pgfqpoint{5.429801in}{2.326352in}}%
\pgfpathlineto{\pgfqpoint{5.422765in}{2.320140in}}%
\pgfpathlineto{\pgfqpoint{5.409281in}{2.321124in}}%
\pgfpathlineto{\pgfqpoint{5.395805in}{2.322132in}}%
\pgfpathlineto{\pgfqpoint{5.382337in}{2.323164in}}%
\pgfpathlineto{\pgfqpoint{5.368877in}{2.324219in}}%
\pgfpathlineto{\pgfqpoint{5.375928in}{2.330478in}}%
\pgfpathlineto{\pgfqpoint{5.382971in}{2.336676in}}%
\pgfpathlineto{\pgfqpoint{5.390008in}{2.342814in}}%
\pgfpathlineto{\pgfqpoint{5.397037in}{2.348895in}}%
\pgfpathclose%
\pgfusepath{fill}%
\end{pgfscope}%
\begin{pgfscope}%
\pgfpathrectangle{\pgfqpoint{1.254980in}{0.150000in}}{\pgfqpoint{5.490039in}{5.490039in}}%
\pgfusepath{clip}%
\pgfsetbuttcap%
\pgfsetroundjoin%
\definecolor{currentfill}{rgb}{0.282623,0.140926,0.457517}%
\pgfsetfillcolor{currentfill}%
\pgfsetfillopacity{0.700000}%
\pgfsetlinewidth{0.000000pt}%
\definecolor{currentstroke}{rgb}{0.000000,0.000000,0.000000}%
\pgfsetstrokecolor{currentstroke}%
\pgfsetdash{}{0pt}%
\pgfpathmoveto{\pgfqpoint{2.744951in}{2.345249in}}%
\pgfpathlineto{\pgfqpoint{2.757830in}{2.337783in}}%
\pgfpathlineto{\pgfqpoint{2.770712in}{2.330361in}}%
\pgfpathlineto{\pgfqpoint{2.783597in}{2.322984in}}%
\pgfpathlineto{\pgfqpoint{2.796485in}{2.315650in}}%
\pgfpathlineto{\pgfqpoint{2.788278in}{2.315186in}}%
\pgfpathlineto{\pgfqpoint{2.780058in}{2.314951in}}%
\pgfpathlineto{\pgfqpoint{2.771823in}{2.314954in}}%
\pgfpathlineto{\pgfqpoint{2.763574in}{2.315199in}}%
\pgfpathlineto{\pgfqpoint{2.750659in}{2.322768in}}%
\pgfpathlineto{\pgfqpoint{2.737747in}{2.330380in}}%
\pgfpathlineto{\pgfqpoint{2.724838in}{2.338038in}}%
\pgfpathlineto{\pgfqpoint{2.711931in}{2.345740in}}%
\pgfpathlineto{\pgfqpoint{2.720208in}{2.345254in}}%
\pgfpathlineto{\pgfqpoint{2.728471in}{2.345015in}}%
\pgfpathlineto{\pgfqpoint{2.736718in}{2.345015in}}%
\pgfpathlineto{\pgfqpoint{2.744951in}{2.345249in}}%
\pgfpathclose%
\pgfusepath{fill}%
\end{pgfscope}%
\begin{pgfscope}%
\pgfpathrectangle{\pgfqpoint{1.254980in}{0.150000in}}{\pgfqpoint{5.490039in}{5.490039in}}%
\pgfusepath{clip}%
\pgfsetbuttcap%
\pgfsetroundjoin%
\definecolor{currentfill}{rgb}{0.267004,0.004874,0.329415}%
\pgfsetfillcolor{currentfill}%
\pgfsetfillopacity{0.700000}%
\pgfsetlinewidth{0.000000pt}%
\definecolor{currentstroke}{rgb}{0.000000,0.000000,0.000000}%
\pgfsetstrokecolor{currentstroke}%
\pgfsetdash{}{0pt}%
\pgfpathmoveto{\pgfqpoint{3.928094in}{2.103567in}}%
\pgfpathlineto{\pgfqpoint{3.941143in}{2.099979in}}%
\pgfpathlineto{\pgfqpoint{3.954198in}{2.096419in}}%
\pgfpathlineto{\pgfqpoint{3.967259in}{2.092886in}}%
\pgfpathlineto{\pgfqpoint{3.980326in}{2.089381in}}%
\pgfpathlineto{\pgfqpoint{3.972730in}{2.081906in}}%
\pgfpathlineto{\pgfqpoint{3.965128in}{2.074462in}}%
\pgfpathlineto{\pgfqpoint{3.957521in}{2.067051in}}%
\pgfpathlineto{\pgfqpoint{3.949908in}{2.059677in}}%
\pgfpathlineto{\pgfqpoint{3.936828in}{2.063311in}}%
\pgfpathlineto{\pgfqpoint{3.923754in}{2.066974in}}%
\pgfpathlineto{\pgfqpoint{3.910686in}{2.070664in}}%
\pgfpathlineto{\pgfqpoint{3.897624in}{2.074381in}}%
\pgfpathlineto{\pgfqpoint{3.905250in}{2.081621in}}%
\pgfpathlineto{\pgfqpoint{3.912870in}{2.088901in}}%
\pgfpathlineto{\pgfqpoint{3.920485in}{2.096217in}}%
\pgfpathlineto{\pgfqpoint{3.928094in}{2.103567in}}%
\pgfpathclose%
\pgfusepath{fill}%
\end{pgfscope}%
\begin{pgfscope}%
\pgfpathrectangle{\pgfqpoint{1.254980in}{0.150000in}}{\pgfqpoint{5.490039in}{5.490039in}}%
\pgfusepath{clip}%
\pgfsetbuttcap%
\pgfsetroundjoin%
\definecolor{currentfill}{rgb}{0.278791,0.062145,0.386592}%
\pgfsetfillcolor{currentfill}%
\pgfsetfillopacity{0.700000}%
\pgfsetlinewidth{0.000000pt}%
\definecolor{currentstroke}{rgb}{0.000000,0.000000,0.000000}%
\pgfsetstrokecolor{currentstroke}%
\pgfsetdash{}{0pt}%
\pgfpathmoveto{\pgfqpoint{3.119104in}{2.197523in}}%
\pgfpathlineto{\pgfqpoint{3.132014in}{2.191456in}}%
\pgfpathlineto{\pgfqpoint{3.144929in}{2.185425in}}%
\pgfpathlineto{\pgfqpoint{3.157847in}{2.179430in}}%
\pgfpathlineto{\pgfqpoint{3.170769in}{2.173471in}}%
\pgfpathlineto{\pgfqpoint{3.162801in}{2.170153in}}%
\pgfpathlineto{\pgfqpoint{3.154823in}{2.167000in}}%
\pgfpathlineto{\pgfqpoint{3.146834in}{2.164018in}}%
\pgfpathlineto{\pgfqpoint{3.138835in}{2.161212in}}%
\pgfpathlineto{\pgfqpoint{3.125891in}{2.167379in}}%
\pgfpathlineto{\pgfqpoint{3.112951in}{2.173581in}}%
\pgfpathlineto{\pgfqpoint{3.100015in}{2.179819in}}%
\pgfpathlineto{\pgfqpoint{3.087082in}{2.186094in}}%
\pgfpathlineto{\pgfqpoint{3.095104in}{2.188687in}}%
\pgfpathlineto{\pgfqpoint{3.103115in}{2.191460in}}%
\pgfpathlineto{\pgfqpoint{3.111114in}{2.194407in}}%
\pgfpathlineto{\pgfqpoint{3.119104in}{2.197523in}}%
\pgfpathclose%
\pgfusepath{fill}%
\end{pgfscope}%
\begin{pgfscope}%
\pgfpathrectangle{\pgfqpoint{1.254980in}{0.150000in}}{\pgfqpoint{5.490039in}{5.490039in}}%
\pgfusepath{clip}%
\pgfsetbuttcap%
\pgfsetroundjoin%
\definecolor{currentfill}{rgb}{0.280255,0.165693,0.476498}%
\pgfsetfillcolor{currentfill}%
\pgfsetfillopacity{0.700000}%
\pgfsetlinewidth{0.000000pt}%
\definecolor{currentstroke}{rgb}{0.000000,0.000000,0.000000}%
\pgfsetstrokecolor{currentstroke}%
\pgfsetdash{}{0pt}%
\pgfpathmoveto{\pgfqpoint{5.614517in}{2.382309in}}%
\pgfpathlineto{\pgfqpoint{5.628027in}{2.381343in}}%
\pgfpathlineto{\pgfqpoint{5.641546in}{2.380401in}}%
\pgfpathlineto{\pgfqpoint{5.655072in}{2.379481in}}%
\pgfpathlineto{\pgfqpoint{5.668607in}{2.378585in}}%
\pgfpathlineto{\pgfqpoint{5.661699in}{2.373213in}}%
\pgfpathlineto{\pgfqpoint{5.654783in}{2.367788in}}%
\pgfpathlineto{\pgfqpoint{5.647860in}{2.362310in}}%
\pgfpathlineto{\pgfqpoint{5.640930in}{2.356775in}}%
\pgfpathlineto{\pgfqpoint{5.627378in}{2.357593in}}%
\pgfpathlineto{\pgfqpoint{5.613834in}{2.358434in}}%
\pgfpathlineto{\pgfqpoint{5.600299in}{2.359299in}}%
\pgfpathlineto{\pgfqpoint{5.586772in}{2.360188in}}%
\pgfpathlineto{\pgfqpoint{5.593719in}{2.365796in}}%
\pgfpathlineto{\pgfqpoint{5.600658in}{2.371350in}}%
\pgfpathlineto{\pgfqpoint{5.607591in}{2.376854in}}%
\pgfpathlineto{\pgfqpoint{5.614517in}{2.382309in}}%
\pgfpathclose%
\pgfusepath{fill}%
\end{pgfscope}%
\begin{pgfscope}%
\pgfpathrectangle{\pgfqpoint{1.254980in}{0.150000in}}{\pgfqpoint{5.490039in}{5.490039in}}%
\pgfusepath{clip}%
\pgfsetbuttcap%
\pgfsetroundjoin%
\definecolor{currentfill}{rgb}{0.277941,0.056324,0.381191}%
\pgfsetfillcolor{currentfill}%
\pgfsetfillopacity{0.700000}%
\pgfsetlinewidth{0.000000pt}%
\definecolor{currentstroke}{rgb}{0.000000,0.000000,0.000000}%
\pgfsetstrokecolor{currentstroke}%
\pgfsetdash{}{0pt}%
\pgfpathmoveto{\pgfqpoint{4.580211in}{2.189846in}}%
\pgfpathlineto{\pgfqpoint{4.593424in}{2.187714in}}%
\pgfpathlineto{\pgfqpoint{4.606644in}{2.185607in}}%
\pgfpathlineto{\pgfqpoint{4.619872in}{2.183525in}}%
\pgfpathlineto{\pgfqpoint{4.633106in}{2.181468in}}%
\pgfpathlineto{\pgfqpoint{4.625745in}{2.173497in}}%
\pgfpathlineto{\pgfqpoint{4.618378in}{2.165482in}}%
\pgfpathlineto{\pgfqpoint{4.611006in}{2.157425in}}%
\pgfpathlineto{\pgfqpoint{4.603628in}{2.149325in}}%
\pgfpathlineto{\pgfqpoint{4.590382in}{2.151435in}}%
\pgfpathlineto{\pgfqpoint{4.577144in}{2.153570in}}%
\pgfpathlineto{\pgfqpoint{4.563913in}{2.155730in}}%
\pgfpathlineto{\pgfqpoint{4.550689in}{2.157915in}}%
\pgfpathlineto{\pgfqpoint{4.558078in}{2.165957in}}%
\pgfpathlineto{\pgfqpoint{4.565461in}{2.173960in}}%
\pgfpathlineto{\pgfqpoint{4.572839in}{2.181923in}}%
\pgfpathlineto{\pgfqpoint{4.580211in}{2.189846in}}%
\pgfpathclose%
\pgfusepath{fill}%
\end{pgfscope}%
\begin{pgfscope}%
\pgfpathrectangle{\pgfqpoint{1.254980in}{0.150000in}}{\pgfqpoint{5.490039in}{5.490039in}}%
\pgfusepath{clip}%
\pgfsetbuttcap%
\pgfsetroundjoin%
\definecolor{currentfill}{rgb}{0.278012,0.180367,0.486697}%
\pgfsetfillcolor{currentfill}%
\pgfsetfillopacity{0.700000}%
\pgfsetlinewidth{0.000000pt}%
\definecolor{currentstroke}{rgb}{0.000000,0.000000,0.000000}%
\pgfsetstrokecolor{currentstroke}%
\pgfsetdash{}{0pt}%
\pgfpathmoveto{\pgfqpoint{5.831920in}{2.412183in}}%
\pgfpathlineto{\pgfqpoint{5.845494in}{2.411303in}}%
\pgfpathlineto{\pgfqpoint{5.859077in}{2.410445in}}%
\pgfpathlineto{\pgfqpoint{5.872667in}{2.409611in}}%
\pgfpathlineto{\pgfqpoint{5.886267in}{2.408800in}}%
\pgfpathlineto{\pgfqpoint{5.879469in}{2.404058in}}%
\pgfpathlineto{\pgfqpoint{5.872664in}{2.399280in}}%
\pgfpathlineto{\pgfqpoint{5.865852in}{2.394462in}}%
\pgfpathlineto{\pgfqpoint{5.859034in}{2.389600in}}%
\pgfpathlineto{\pgfqpoint{5.845415in}{2.390307in}}%
\pgfpathlineto{\pgfqpoint{5.831806in}{2.391037in}}%
\pgfpathlineto{\pgfqpoint{5.818204in}{2.391790in}}%
\pgfpathlineto{\pgfqpoint{5.804612in}{2.392566in}}%
\pgfpathlineto{\pgfqpoint{5.811449in}{2.397527in}}%
\pgfpathlineto{\pgfqpoint{5.818280in}{2.402448in}}%
\pgfpathlineto{\pgfqpoint{5.825103in}{2.407332in}}%
\pgfpathlineto{\pgfqpoint{5.831920in}{2.412183in}}%
\pgfpathclose%
\pgfusepath{fill}%
\end{pgfscope}%
\begin{pgfscope}%
\pgfpathrectangle{\pgfqpoint{1.254980in}{0.150000in}}{\pgfqpoint{5.490039in}{5.490039in}}%
\pgfusepath{clip}%
\pgfsetbuttcap%
\pgfsetroundjoin%
\definecolor{currentfill}{rgb}{0.283197,0.115680,0.436115}%
\pgfsetfillcolor{currentfill}%
\pgfsetfillopacity{0.700000}%
\pgfsetlinewidth{0.000000pt}%
\definecolor{currentstroke}{rgb}{0.000000,0.000000,0.000000}%
\pgfsetstrokecolor{currentstroke}%
\pgfsetdash{}{0pt}%
\pgfpathmoveto{\pgfqpoint{5.097502in}{2.290859in}}%
\pgfpathlineto{\pgfqpoint{5.110863in}{2.289485in}}%
\pgfpathlineto{\pgfqpoint{5.124231in}{2.288135in}}%
\pgfpathlineto{\pgfqpoint{5.137608in}{2.286809in}}%
\pgfpathlineto{\pgfqpoint{5.150992in}{2.285506in}}%
\pgfpathlineto{\pgfqpoint{5.143838in}{2.278575in}}%
\pgfpathlineto{\pgfqpoint{5.136677in}{2.271578in}}%
\pgfpathlineto{\pgfqpoint{5.129510in}{2.264515in}}%
\pgfpathlineto{\pgfqpoint{5.122336in}{2.257385in}}%
\pgfpathlineto{\pgfqpoint{5.108939in}{2.258675in}}%
\pgfpathlineto{\pgfqpoint{5.095550in}{2.259989in}}%
\pgfpathlineto{\pgfqpoint{5.082168in}{2.261327in}}%
\pgfpathlineto{\pgfqpoint{5.068795in}{2.262690in}}%
\pgfpathlineto{\pgfqpoint{5.075981in}{2.269827in}}%
\pgfpathlineto{\pgfqpoint{5.083161in}{2.276901in}}%
\pgfpathlineto{\pgfqpoint{5.090335in}{2.283911in}}%
\pgfpathlineto{\pgfqpoint{5.097502in}{2.290859in}}%
\pgfpathclose%
\pgfusepath{fill}%
\end{pgfscope}%
\begin{pgfscope}%
\pgfpathrectangle{\pgfqpoint{1.254980in}{0.150000in}}{\pgfqpoint{5.490039in}{5.490039in}}%
\pgfusepath{clip}%
\pgfsetbuttcap%
\pgfsetroundjoin%
\definecolor{currentfill}{rgb}{0.268510,0.009605,0.335427}%
\pgfsetfillcolor{currentfill}%
\pgfsetfillopacity{0.700000}%
\pgfsetlinewidth{0.000000pt}%
\definecolor{currentstroke}{rgb}{0.000000,0.000000,0.000000}%
\pgfsetstrokecolor{currentstroke}%
\pgfsetdash{}{0pt}%
\pgfpathmoveto{\pgfqpoint{3.575760in}{2.104593in}}%
\pgfpathlineto{\pgfqpoint{3.588742in}{2.099997in}}%
\pgfpathlineto{\pgfqpoint{3.601728in}{2.095432in}}%
\pgfpathlineto{\pgfqpoint{3.614721in}{2.090898in}}%
\pgfpathlineto{\pgfqpoint{3.627718in}{2.086393in}}%
\pgfpathlineto{\pgfqpoint{3.619977in}{2.080315in}}%
\pgfpathlineto{\pgfqpoint{3.612230in}{2.074326in}}%
\pgfpathlineto{\pgfqpoint{3.604475in}{2.068428in}}%
\pgfpathlineto{\pgfqpoint{3.596713in}{2.062627in}}%
\pgfpathlineto{\pgfqpoint{3.583700in}{2.067299in}}%
\pgfpathlineto{\pgfqpoint{3.570692in}{2.072002in}}%
\pgfpathlineto{\pgfqpoint{3.557689in}{2.076735in}}%
\pgfpathlineto{\pgfqpoint{3.544691in}{2.081498in}}%
\pgfpathlineto{\pgfqpoint{3.552469in}{2.087127in}}%
\pgfpathlineto{\pgfqpoint{3.560240in}{2.092855in}}%
\pgfpathlineto{\pgfqpoint{3.568004in}{2.098678in}}%
\pgfpathlineto{\pgfqpoint{3.575760in}{2.104593in}}%
\pgfpathclose%
\pgfusepath{fill}%
\end{pgfscope}%
\begin{pgfscope}%
\pgfpathrectangle{\pgfqpoint{1.254980in}{0.150000in}}{\pgfqpoint{5.490039in}{5.490039in}}%
\pgfusepath{clip}%
\pgfsetbuttcap%
\pgfsetroundjoin%
\definecolor{currentfill}{rgb}{0.269944,0.014625,0.341379}%
\pgfsetfillcolor{currentfill}%
\pgfsetfillopacity{0.700000}%
\pgfsetlinewidth{0.000000pt}%
\definecolor{currentstroke}{rgb}{0.000000,0.000000,0.000000}%
\pgfsetstrokecolor{currentstroke}%
\pgfsetdash{}{0pt}%
\pgfpathmoveto{\pgfqpoint{3.440885in}{2.120716in}}%
\pgfpathlineto{\pgfqpoint{3.453844in}{2.115704in}}%
\pgfpathlineto{\pgfqpoint{3.466807in}{2.110724in}}%
\pgfpathlineto{\pgfqpoint{3.479776in}{2.105775in}}%
\pgfpathlineto{\pgfqpoint{3.492749in}{2.100858in}}%
\pgfpathlineto{\pgfqpoint{3.484947in}{2.095509in}}%
\pgfpathlineto{\pgfqpoint{3.477136in}{2.090272in}}%
\pgfpathlineto{\pgfqpoint{3.469318in}{2.085150in}}%
\pgfpathlineto{\pgfqpoint{3.461492in}{2.080148in}}%
\pgfpathlineto{\pgfqpoint{3.448501in}{2.085247in}}%
\pgfpathlineto{\pgfqpoint{3.435515in}{2.090376in}}%
\pgfpathlineto{\pgfqpoint{3.422534in}{2.095538in}}%
\pgfpathlineto{\pgfqpoint{3.409557in}{2.100731in}}%
\pgfpathlineto{\pgfqpoint{3.417402in}{2.105546in}}%
\pgfpathlineto{\pgfqpoint{3.425237in}{2.110486in}}%
\pgfpathlineto{\pgfqpoint{3.433065in}{2.115544in}}%
\pgfpathlineto{\pgfqpoint{3.440885in}{2.120716in}}%
\pgfpathclose%
\pgfusepath{fill}%
\end{pgfscope}%
\begin{pgfscope}%
\pgfpathrectangle{\pgfqpoint{1.254980in}{0.150000in}}{\pgfqpoint{5.490039in}{5.490039in}}%
\pgfusepath{clip}%
\pgfsetbuttcap%
\pgfsetroundjoin%
\definecolor{currentfill}{rgb}{0.281446,0.084320,0.407414}%
\pgfsetfillcolor{currentfill}%
\pgfsetfillopacity{0.700000}%
\pgfsetlinewidth{0.000000pt}%
\definecolor{currentstroke}{rgb}{0.000000,0.000000,0.000000}%
\pgfsetstrokecolor{currentstroke}%
\pgfsetdash{}{0pt}%
\pgfpathmoveto{\pgfqpoint{4.797738in}{2.228562in}}%
\pgfpathlineto{\pgfqpoint{4.811015in}{2.226795in}}%
\pgfpathlineto{\pgfqpoint{4.824299in}{2.225053in}}%
\pgfpathlineto{\pgfqpoint{4.837591in}{2.223335in}}%
\pgfpathlineto{\pgfqpoint{4.850890in}{2.221642in}}%
\pgfpathlineto{\pgfqpoint{4.843610in}{2.213975in}}%
\pgfpathlineto{\pgfqpoint{4.836324in}{2.206251in}}%
\pgfpathlineto{\pgfqpoint{4.829032in}{2.198470in}}%
\pgfpathlineto{\pgfqpoint{4.821734in}{2.190632in}}%
\pgfpathlineto{\pgfqpoint{4.808423in}{2.192352in}}%
\pgfpathlineto{\pgfqpoint{4.795120in}{2.194096in}}%
\pgfpathlineto{\pgfqpoint{4.781825in}{2.195866in}}%
\pgfpathlineto{\pgfqpoint{4.768536in}{2.197660in}}%
\pgfpathlineto{\pgfqpoint{4.775846in}{2.205466in}}%
\pgfpathlineto{\pgfqpoint{4.783149in}{2.213218in}}%
\pgfpathlineto{\pgfqpoint{4.790447in}{2.220917in}}%
\pgfpathlineto{\pgfqpoint{4.797738in}{2.228562in}}%
\pgfpathclose%
\pgfusepath{fill}%
\end{pgfscope}%
\begin{pgfscope}%
\pgfpathrectangle{\pgfqpoint{1.254980in}{0.150000in}}{\pgfqpoint{5.490039in}{5.490039in}}%
\pgfusepath{clip}%
\pgfsetbuttcap%
\pgfsetroundjoin%
\definecolor{currentfill}{rgb}{0.272594,0.025563,0.353093}%
\pgfsetfillcolor{currentfill}%
\pgfsetfillopacity{0.700000}%
\pgfsetlinewidth{0.000000pt}%
\definecolor{currentstroke}{rgb}{0.000000,0.000000,0.000000}%
\pgfsetstrokecolor{currentstroke}%
\pgfsetdash{}{0pt}%
\pgfpathmoveto{\pgfqpoint{4.280372in}{2.133064in}}%
\pgfpathlineto{\pgfqpoint{4.293510in}{2.130326in}}%
\pgfpathlineto{\pgfqpoint{4.306654in}{2.127614in}}%
\pgfpathlineto{\pgfqpoint{4.319805in}{2.124927in}}%
\pgfpathlineto{\pgfqpoint{4.332963in}{2.122266in}}%
\pgfpathlineto{\pgfqpoint{4.325491in}{2.114195in}}%
\pgfpathlineto{\pgfqpoint{4.318014in}{2.106109in}}%
\pgfpathlineto{\pgfqpoint{4.310532in}{2.098010in}}%
\pgfpathlineto{\pgfqpoint{4.303045in}{2.089900in}}%
\pgfpathlineto{\pgfqpoint{4.289876in}{2.092653in}}%
\pgfpathlineto{\pgfqpoint{4.276714in}{2.095431in}}%
\pgfpathlineto{\pgfqpoint{4.263559in}{2.098235in}}%
\pgfpathlineto{\pgfqpoint{4.250410in}{2.101065in}}%
\pgfpathlineto{\pgfqpoint{4.257908in}{2.109078in}}%
\pgfpathlineto{\pgfqpoint{4.265402in}{2.117083in}}%
\pgfpathlineto{\pgfqpoint{4.272890in}{2.125080in}}%
\pgfpathlineto{\pgfqpoint{4.280372in}{2.133064in}}%
\pgfpathclose%
\pgfusepath{fill}%
\end{pgfscope}%
\begin{pgfscope}%
\pgfpathrectangle{\pgfqpoint{1.254980in}{0.150000in}}{\pgfqpoint{5.490039in}{5.490039in}}%
\pgfusepath{clip}%
\pgfsetbuttcap%
\pgfsetroundjoin%
\definecolor{currentfill}{rgb}{0.268510,0.009605,0.335427}%
\pgfsetfillcolor{currentfill}%
\pgfsetfillopacity{0.700000}%
\pgfsetlinewidth{0.000000pt}%
\definecolor{currentstroke}{rgb}{0.000000,0.000000,0.000000}%
\pgfsetstrokecolor{currentstroke}%
\pgfsetdash{}{0pt}%
\pgfpathmoveto{\pgfqpoint{4.062931in}{2.106247in}}%
\pgfpathlineto{\pgfqpoint{4.076016in}{2.102995in}}%
\pgfpathlineto{\pgfqpoint{4.089107in}{2.099770in}}%
\pgfpathlineto{\pgfqpoint{4.102204in}{2.096571in}}%
\pgfpathlineto{\pgfqpoint{4.115307in}{2.093400in}}%
\pgfpathlineto{\pgfqpoint{4.107758in}{2.085607in}}%
\pgfpathlineto{\pgfqpoint{4.100203in}{2.077828in}}%
\pgfpathlineto{\pgfqpoint{4.092643in}{2.070064in}}%
\pgfpathlineto{\pgfqpoint{4.085078in}{2.062318in}}%
\pgfpathlineto{\pgfqpoint{4.071962in}{2.065607in}}%
\pgfpathlineto{\pgfqpoint{4.058853in}{2.068922in}}%
\pgfpathlineto{\pgfqpoint{4.045750in}{2.072264in}}%
\pgfpathlineto{\pgfqpoint{4.032653in}{2.075633in}}%
\pgfpathlineto{\pgfqpoint{4.040231in}{2.083257in}}%
\pgfpathlineto{\pgfqpoint{4.047803in}{2.090903in}}%
\pgfpathlineto{\pgfqpoint{4.055370in}{2.098567in}}%
\pgfpathlineto{\pgfqpoint{4.062931in}{2.106247in}}%
\pgfpathclose%
\pgfusepath{fill}%
\end{pgfscope}%
\begin{pgfscope}%
\pgfpathrectangle{\pgfqpoint{1.254980in}{0.150000in}}{\pgfqpoint{5.490039in}{5.490039in}}%
\pgfusepath{clip}%
\pgfsetbuttcap%
\pgfsetroundjoin%
\definecolor{currentfill}{rgb}{0.282884,0.135920,0.453427}%
\pgfsetfillcolor{currentfill}%
\pgfsetfillopacity{0.700000}%
\pgfsetlinewidth{0.000000pt}%
\definecolor{currentstroke}{rgb}{0.000000,0.000000,0.000000}%
\pgfsetstrokecolor{currentstroke}%
\pgfsetdash{}{0pt}%
\pgfpathmoveto{\pgfqpoint{5.315119in}{2.328679in}}%
\pgfpathlineto{\pgfqpoint{5.328546in}{2.327528in}}%
\pgfpathlineto{\pgfqpoint{5.341982in}{2.326401in}}%
\pgfpathlineto{\pgfqpoint{5.355425in}{2.325299in}}%
\pgfpathlineto{\pgfqpoint{5.368877in}{2.324219in}}%
\pgfpathlineto{\pgfqpoint{5.361820in}{2.317897in}}%
\pgfpathlineto{\pgfqpoint{5.354755in}{2.311511in}}%
\pgfpathlineto{\pgfqpoint{5.347684in}{2.305058in}}%
\pgfpathlineto{\pgfqpoint{5.340605in}{2.298537in}}%
\pgfpathlineto{\pgfqpoint{5.327139in}{2.299578in}}%
\pgfpathlineto{\pgfqpoint{5.313681in}{2.300642in}}%
\pgfpathlineto{\pgfqpoint{5.300231in}{2.301731in}}%
\pgfpathlineto{\pgfqpoint{5.286789in}{2.302843in}}%
\pgfpathlineto{\pgfqpoint{5.293882in}{2.309397in}}%
\pgfpathlineto{\pgfqpoint{5.300968in}{2.315887in}}%
\pgfpathlineto{\pgfqpoint{5.308047in}{2.322314in}}%
\pgfpathlineto{\pgfqpoint{5.315119in}{2.328679in}}%
\pgfpathclose%
\pgfusepath{fill}%
\end{pgfscope}%
\begin{pgfscope}%
\pgfpathrectangle{\pgfqpoint{1.254980in}{0.150000in}}{\pgfqpoint{5.490039in}{5.490039in}}%
\pgfusepath{clip}%
\pgfsetbuttcap%
\pgfsetroundjoin%
\definecolor{currentfill}{rgb}{0.267004,0.004874,0.329415}%
\pgfsetfillcolor{currentfill}%
\pgfsetfillopacity{0.700000}%
\pgfsetlinewidth{0.000000pt}%
\definecolor{currentstroke}{rgb}{0.000000,0.000000,0.000000}%
\pgfsetstrokecolor{currentstroke}%
\pgfsetdash{}{0pt}%
\pgfpathmoveto{\pgfqpoint{3.710592in}{2.094406in}}%
\pgfpathlineto{\pgfqpoint{3.723601in}{2.090204in}}%
\pgfpathlineto{\pgfqpoint{3.736615in}{2.086032in}}%
\pgfpathlineto{\pgfqpoint{3.749635in}{2.081888in}}%
\pgfpathlineto{\pgfqpoint{3.762660in}{2.077774in}}%
\pgfpathlineto{\pgfqpoint{3.754976in}{2.071082in}}%
\pgfpathlineto{\pgfqpoint{3.747285in}{2.064456in}}%
\pgfpathlineto{\pgfqpoint{3.739588in}{2.057901in}}%
\pgfpathlineto{\pgfqpoint{3.731885in}{2.051419in}}%
\pgfpathlineto{\pgfqpoint{3.718845in}{2.055688in}}%
\pgfpathlineto{\pgfqpoint{3.705811in}{2.059987in}}%
\pgfpathlineto{\pgfqpoint{3.692782in}{2.064315in}}%
\pgfpathlineto{\pgfqpoint{3.679759in}{2.068672in}}%
\pgfpathlineto{\pgfqpoint{3.687477in}{2.074993in}}%
\pgfpathlineto{\pgfqpoint{3.695189in}{2.081392in}}%
\pgfpathlineto{\pgfqpoint{3.702894in}{2.087864in}}%
\pgfpathlineto{\pgfqpoint{3.710592in}{2.094406in}}%
\pgfpathclose%
\pgfusepath{fill}%
\end{pgfscope}%
\begin{pgfscope}%
\pgfpathrectangle{\pgfqpoint{1.254980in}{0.150000in}}{\pgfqpoint{5.490039in}{5.490039in}}%
\pgfusepath{clip}%
\pgfsetbuttcap%
\pgfsetroundjoin%
\definecolor{currentfill}{rgb}{0.275191,0.194905,0.496005}%
\pgfsetfillcolor{currentfill}%
\pgfsetfillopacity{0.700000}%
\pgfsetlinewidth{0.000000pt}%
\definecolor{currentstroke}{rgb}{0.000000,0.000000,0.000000}%
\pgfsetstrokecolor{currentstroke}%
\pgfsetdash{}{0pt}%
\pgfpathmoveto{\pgfqpoint{2.557227in}{2.441834in}}%
\pgfpathlineto{\pgfqpoint{2.570108in}{2.433557in}}%
\pgfpathlineto{\pgfqpoint{2.582990in}{2.425330in}}%
\pgfpathlineto{\pgfqpoint{2.595874in}{2.417154in}}%
\pgfpathlineto{\pgfqpoint{2.608761in}{2.409027in}}%
\pgfpathlineto{\pgfqpoint{2.600409in}{2.410255in}}%
\pgfpathlineto{\pgfqpoint{2.592041in}{2.411748in}}%
\pgfpathlineto{\pgfqpoint{2.583655in}{2.413513in}}%
\pgfpathlineto{\pgfqpoint{2.575252in}{2.415557in}}%
\pgfpathlineto{\pgfqpoint{2.562335in}{2.423934in}}%
\pgfpathlineto{\pgfqpoint{2.549420in}{2.432361in}}%
\pgfpathlineto{\pgfqpoint{2.536507in}{2.440838in}}%
\pgfpathlineto{\pgfqpoint{2.523596in}{2.449366in}}%
\pgfpathlineto{\pgfqpoint{2.532030in}{2.447066in}}%
\pgfpathlineto{\pgfqpoint{2.540446in}{2.445048in}}%
\pgfpathlineto{\pgfqpoint{2.548845in}{2.443306in}}%
\pgfpathlineto{\pgfqpoint{2.557227in}{2.441834in}}%
\pgfpathclose%
\pgfusepath{fill}%
\end{pgfscope}%
\begin{pgfscope}%
\pgfpathrectangle{\pgfqpoint{1.254980in}{0.150000in}}{\pgfqpoint{5.490039in}{5.490039in}}%
\pgfusepath{clip}%
\pgfsetbuttcap%
\pgfsetroundjoin%
\definecolor{currentfill}{rgb}{0.273809,0.031497,0.358853}%
\pgfsetfillcolor{currentfill}%
\pgfsetfillopacity{0.700000}%
\pgfsetlinewidth{0.000000pt}%
\definecolor{currentstroke}{rgb}{0.000000,0.000000,0.000000}%
\pgfsetstrokecolor{currentstroke}%
\pgfsetdash{}{0pt}%
\pgfpathmoveto{\pgfqpoint{3.305910in}{2.143440in}}%
\pgfpathlineto{\pgfqpoint{3.318850in}{2.137986in}}%
\pgfpathlineto{\pgfqpoint{3.331795in}{2.132566in}}%
\pgfpathlineto{\pgfqpoint{3.344744in}{2.127178in}}%
\pgfpathlineto{\pgfqpoint{3.357697in}{2.121824in}}%
\pgfpathlineto{\pgfqpoint{3.349826in}{2.117325in}}%
\pgfpathlineto{\pgfqpoint{3.341947in}{2.112963in}}%
\pgfpathlineto{\pgfqpoint{3.334058in}{2.108741in}}%
\pgfpathlineto{\pgfqpoint{3.326160in}{2.104665in}}%
\pgfpathlineto{\pgfqpoint{3.313187in}{2.110214in}}%
\pgfpathlineto{\pgfqpoint{3.300219in}{2.115795in}}%
\pgfpathlineto{\pgfqpoint{3.287255in}{2.121410in}}%
\pgfpathlineto{\pgfqpoint{3.274295in}{2.127058in}}%
\pgfpathlineto{\pgfqpoint{3.282213in}{2.130935in}}%
\pgfpathlineto{\pgfqpoint{3.290121in}{2.134961in}}%
\pgfpathlineto{\pgfqpoint{3.298020in}{2.139131in}}%
\pgfpathlineto{\pgfqpoint{3.305910in}{2.143440in}}%
\pgfpathclose%
\pgfusepath{fill}%
\end{pgfscope}%
\begin{pgfscope}%
\pgfpathrectangle{\pgfqpoint{1.254980in}{0.150000in}}{\pgfqpoint{5.490039in}{5.490039in}}%
\pgfusepath{clip}%
\pgfsetbuttcap%
\pgfsetroundjoin%
\definecolor{currentfill}{rgb}{0.276022,0.044167,0.370164}%
\pgfsetfillcolor{currentfill}%
\pgfsetfillopacity{0.700000}%
\pgfsetlinewidth{0.000000pt}%
\definecolor{currentstroke}{rgb}{0.000000,0.000000,0.000000}%
\pgfsetstrokecolor{currentstroke}%
\pgfsetdash{}{0pt}%
\pgfpathmoveto{\pgfqpoint{4.497862in}{2.166908in}}%
\pgfpathlineto{\pgfqpoint{4.511058in}{2.164622in}}%
\pgfpathlineto{\pgfqpoint{4.524262in}{2.162361in}}%
\pgfpathlineto{\pgfqpoint{4.537472in}{2.160125in}}%
\pgfpathlineto{\pgfqpoint{4.550689in}{2.157915in}}%
\pgfpathlineto{\pgfqpoint{4.543294in}{2.149836in}}%
\pgfpathlineto{\pgfqpoint{4.535894in}{2.141720in}}%
\pgfpathlineto{\pgfqpoint{4.528488in}{2.133569in}}%
\pgfpathlineto{\pgfqpoint{4.521077in}{2.125384in}}%
\pgfpathlineto{\pgfqpoint{4.507849in}{2.127660in}}%
\pgfpathlineto{\pgfqpoint{4.494628in}{2.129962in}}%
\pgfpathlineto{\pgfqpoint{4.481414in}{2.132288in}}%
\pgfpathlineto{\pgfqpoint{4.468207in}{2.134640in}}%
\pgfpathlineto{\pgfqpoint{4.475629in}{2.142754in}}%
\pgfpathlineto{\pgfqpoint{4.483045in}{2.150838in}}%
\pgfpathlineto{\pgfqpoint{4.490457in}{2.158889in}}%
\pgfpathlineto{\pgfqpoint{4.497862in}{2.166908in}}%
\pgfpathclose%
\pgfusepath{fill}%
\end{pgfscope}%
\begin{pgfscope}%
\pgfpathrectangle{\pgfqpoint{1.254980in}{0.150000in}}{\pgfqpoint{5.490039in}{5.490039in}}%
\pgfusepath{clip}%
\pgfsetbuttcap%
\pgfsetroundjoin%
\definecolor{currentfill}{rgb}{0.281924,0.089666,0.412415}%
\pgfsetfillcolor{currentfill}%
\pgfsetfillopacity{0.700000}%
\pgfsetlinewidth{0.000000pt}%
\definecolor{currentstroke}{rgb}{0.000000,0.000000,0.000000}%
\pgfsetstrokecolor{currentstroke}%
\pgfsetdash{}{0pt}%
\pgfpathmoveto{\pgfqpoint{2.983756in}{2.237634in}}%
\pgfpathlineto{\pgfqpoint{2.996659in}{2.231058in}}%
\pgfpathlineto{\pgfqpoint{3.009566in}{2.224521in}}%
\pgfpathlineto{\pgfqpoint{3.022476in}{2.218022in}}%
\pgfpathlineto{\pgfqpoint{3.035390in}{2.211561in}}%
\pgfpathlineto{\pgfqpoint{3.027335in}{2.209369in}}%
\pgfpathlineto{\pgfqpoint{3.019267in}{2.207370in}}%
\pgfpathlineto{\pgfqpoint{3.011188in}{2.205571in}}%
\pgfpathlineto{\pgfqpoint{3.003096in}{2.203976in}}%
\pgfpathlineto{\pgfqpoint{2.990159in}{2.210658in}}%
\pgfpathlineto{\pgfqpoint{2.977224in}{2.217378in}}%
\pgfpathlineto{\pgfqpoint{2.964294in}{2.224136in}}%
\pgfpathlineto{\pgfqpoint{2.951366in}{2.230933in}}%
\pgfpathlineto{\pgfqpoint{2.959482in}{2.232302in}}%
\pgfpathlineto{\pgfqpoint{2.967586in}{2.233878in}}%
\pgfpathlineto{\pgfqpoint{2.975677in}{2.235658in}}%
\pgfpathlineto{\pgfqpoint{2.983756in}{2.237634in}}%
\pgfpathclose%
\pgfusepath{fill}%
\end{pgfscope}%
\begin{pgfscope}%
\pgfpathrectangle{\pgfqpoint{1.254980in}{0.150000in}}{\pgfqpoint{5.490039in}{5.490039in}}%
\pgfusepath{clip}%
\pgfsetbuttcap%
\pgfsetroundjoin%
\definecolor{currentfill}{rgb}{0.276194,0.190074,0.493001}%
\pgfsetfillcolor{currentfill}%
\pgfsetfillopacity{0.700000}%
\pgfsetlinewidth{0.000000pt}%
\definecolor{currentstroke}{rgb}{0.000000,0.000000,0.000000}%
\pgfsetstrokecolor{currentstroke}%
\pgfsetdash{}{0pt}%
\pgfpathmoveto{\pgfqpoint{5.967794in}{2.423995in}}%
\pgfpathlineto{\pgfqpoint{5.981415in}{2.423183in}}%
\pgfpathlineto{\pgfqpoint{5.995046in}{2.422394in}}%
\pgfpathlineto{\pgfqpoint{6.008685in}{2.421628in}}%
\pgfpathlineto{\pgfqpoint{6.001949in}{2.417210in}}%
\pgfpathlineto{\pgfqpoint{5.995206in}{2.412764in}}%
\pgfpathlineto{\pgfqpoint{5.988457in}{2.408287in}}%
\pgfpathlineto{\pgfqpoint{5.981701in}{2.403775in}}%
\pgfpathlineto{\pgfqpoint{5.968042in}{2.404423in}}%
\pgfpathlineto{\pgfqpoint{5.954391in}{2.405094in}}%
\pgfpathlineto{\pgfqpoint{5.940749in}{2.405789in}}%
\pgfpathlineto{\pgfqpoint{5.947520in}{2.410386in}}%
\pgfpathlineto{\pgfqpoint{5.954285in}{2.414950in}}%
\pgfpathlineto{\pgfqpoint{5.961042in}{2.419485in}}%
\pgfpathlineto{\pgfqpoint{5.967794in}{2.423995in}}%
\pgfpathclose%
\pgfusepath{fill}%
\end{pgfscope}%
\begin{pgfscope}%
\pgfpathrectangle{\pgfqpoint{1.254980in}{0.150000in}}{\pgfqpoint{5.490039in}{5.490039in}}%
\pgfusepath{clip}%
\pgfsetbuttcap%
\pgfsetroundjoin%
\definecolor{currentfill}{rgb}{0.280868,0.160771,0.472899}%
\pgfsetfillcolor{currentfill}%
\pgfsetfillopacity{0.700000}%
\pgfsetlinewidth{0.000000pt}%
\definecolor{currentstroke}{rgb}{0.000000,0.000000,0.000000}%
\pgfsetstrokecolor{currentstroke}%
\pgfsetdash{}{0pt}%
\pgfpathmoveto{\pgfqpoint{5.532746in}{2.363976in}}%
\pgfpathlineto{\pgfqpoint{5.546240in}{2.362994in}}%
\pgfpathlineto{\pgfqpoint{5.559743in}{2.362035in}}%
\pgfpathlineto{\pgfqpoint{5.573253in}{2.361100in}}%
\pgfpathlineto{\pgfqpoint{5.586772in}{2.360188in}}%
\pgfpathlineto{\pgfqpoint{5.579818in}{2.354524in}}%
\pgfpathlineto{\pgfqpoint{5.572856in}{2.348801in}}%
\pgfpathlineto{\pgfqpoint{5.565888in}{2.343018in}}%
\pgfpathlineto{\pgfqpoint{5.558912in}{2.337172in}}%
\pgfpathlineto{\pgfqpoint{5.545377in}{2.338019in}}%
\pgfpathlineto{\pgfqpoint{5.531851in}{2.338890in}}%
\pgfpathlineto{\pgfqpoint{5.518332in}{2.339784in}}%
\pgfpathlineto{\pgfqpoint{5.504822in}{2.340702in}}%
\pgfpathlineto{\pgfqpoint{5.511814in}{2.346608in}}%
\pgfpathlineto{\pgfqpoint{5.518799in}{2.352454in}}%
\pgfpathlineto{\pgfqpoint{5.525776in}{2.358243in}}%
\pgfpathlineto{\pgfqpoint{5.532746in}{2.363976in}}%
\pgfpathclose%
\pgfusepath{fill}%
\end{pgfscope}%
\begin{pgfscope}%
\pgfpathrectangle{\pgfqpoint{1.254980in}{0.150000in}}{\pgfqpoint{5.490039in}{5.490039in}}%
\pgfusepath{clip}%
\pgfsetbuttcap%
\pgfsetroundjoin%
\definecolor{currentfill}{rgb}{0.283072,0.130895,0.449241}%
\pgfsetfillcolor{currentfill}%
\pgfsetfillopacity{0.700000}%
\pgfsetlinewidth{0.000000pt}%
\definecolor{currentstroke}{rgb}{0.000000,0.000000,0.000000}%
\pgfsetstrokecolor{currentstroke}%
\pgfsetdash{}{0pt}%
\pgfpathmoveto{\pgfqpoint{2.796485in}{2.315650in}}%
\pgfpathlineto{\pgfqpoint{2.809375in}{2.308360in}}%
\pgfpathlineto{\pgfqpoint{2.822268in}{2.301113in}}%
\pgfpathlineto{\pgfqpoint{2.835164in}{2.293909in}}%
\pgfpathlineto{\pgfqpoint{2.848063in}{2.286748in}}%
\pgfpathlineto{\pgfqpoint{2.839883in}{2.286053in}}%
\pgfpathlineto{\pgfqpoint{2.831690in}{2.285585in}}%
\pgfpathlineto{\pgfqpoint{2.823482in}{2.285351in}}%
\pgfpathlineto{\pgfqpoint{2.815260in}{2.285355in}}%
\pgfpathlineto{\pgfqpoint{2.802334in}{2.292752in}}%
\pgfpathlineto{\pgfqpoint{2.789411in}{2.300191in}}%
\pgfpathlineto{\pgfqpoint{2.776491in}{2.307673in}}%
\pgfpathlineto{\pgfqpoint{2.763574in}{2.315199in}}%
\pgfpathlineto{\pgfqpoint{2.771823in}{2.314954in}}%
\pgfpathlineto{\pgfqpoint{2.780058in}{2.314951in}}%
\pgfpathlineto{\pgfqpoint{2.788278in}{2.315186in}}%
\pgfpathlineto{\pgfqpoint{2.796485in}{2.315650in}}%
\pgfpathclose%
\pgfusepath{fill}%
\end{pgfscope}%
\begin{pgfscope}%
\pgfpathrectangle{\pgfqpoint{1.254980in}{0.150000in}}{\pgfqpoint{5.490039in}{5.490039in}}%
\pgfusepath{clip}%
\pgfsetbuttcap%
\pgfsetroundjoin%
\definecolor{currentfill}{rgb}{0.282910,0.105393,0.426902}%
\pgfsetfillcolor{currentfill}%
\pgfsetfillopacity{0.700000}%
\pgfsetlinewidth{0.000000pt}%
\definecolor{currentstroke}{rgb}{0.000000,0.000000,0.000000}%
\pgfsetstrokecolor{currentstroke}%
\pgfsetdash{}{0pt}%
\pgfpathmoveto{\pgfqpoint{5.015377in}{2.268380in}}%
\pgfpathlineto{\pgfqpoint{5.028720in}{2.266921in}}%
\pgfpathlineto{\pgfqpoint{5.042071in}{2.265487in}}%
\pgfpathlineto{\pgfqpoint{5.055429in}{2.264076in}}%
\pgfpathlineto{\pgfqpoint{5.068795in}{2.262690in}}%
\pgfpathlineto{\pgfqpoint{5.061602in}{2.255488in}}%
\pgfpathlineto{\pgfqpoint{5.054402in}{2.248221in}}%
\pgfpathlineto{\pgfqpoint{5.047196in}{2.240888in}}%
\pgfpathlineto{\pgfqpoint{5.039983in}{2.233489in}}%
\pgfpathlineto{\pgfqpoint{5.026604in}{2.234876in}}%
\pgfpathlineto{\pgfqpoint{5.013234in}{2.236288in}}%
\pgfpathlineto{\pgfqpoint{4.999871in}{2.237724in}}%
\pgfpathlineto{\pgfqpoint{4.986516in}{2.239183in}}%
\pgfpathlineto{\pgfqpoint{4.993741in}{2.246576in}}%
\pgfpathlineto{\pgfqpoint{5.000959in}{2.253907in}}%
\pgfpathlineto{\pgfqpoint{5.008171in}{2.261174in}}%
\pgfpathlineto{\pgfqpoint{5.015377in}{2.268380in}}%
\pgfpathclose%
\pgfusepath{fill}%
\end{pgfscope}%
\begin{pgfscope}%
\pgfpathrectangle{\pgfqpoint{1.254980in}{0.150000in}}{\pgfqpoint{5.490039in}{5.490039in}}%
\pgfusepath{clip}%
\pgfsetbuttcap%
\pgfsetroundjoin%
\definecolor{currentfill}{rgb}{0.267004,0.004874,0.329415}%
\pgfsetfillcolor{currentfill}%
\pgfsetfillopacity{0.700000}%
\pgfsetlinewidth{0.000000pt}%
\definecolor{currentstroke}{rgb}{0.000000,0.000000,0.000000}%
\pgfsetstrokecolor{currentstroke}%
\pgfsetdash{}{0pt}%
\pgfpathmoveto{\pgfqpoint{3.845433in}{2.089531in}}%
\pgfpathlineto{\pgfqpoint{3.858472in}{2.085702in}}%
\pgfpathlineto{\pgfqpoint{3.871517in}{2.081900in}}%
\pgfpathlineto{\pgfqpoint{3.884568in}{2.078127in}}%
\pgfpathlineto{\pgfqpoint{3.897624in}{2.074381in}}%
\pgfpathlineto{\pgfqpoint{3.889992in}{2.067184in}}%
\pgfpathlineto{\pgfqpoint{3.882354in}{2.060033in}}%
\pgfpathlineto{\pgfqpoint{3.874710in}{2.052931in}}%
\pgfpathlineto{\pgfqpoint{3.867060in}{2.045882in}}%
\pgfpathlineto{\pgfqpoint{3.853990in}{2.049770in}}%
\pgfpathlineto{\pgfqpoint{3.840926in}{2.053686in}}%
\pgfpathlineto{\pgfqpoint{3.827868in}{2.057630in}}%
\pgfpathlineto{\pgfqpoint{3.814815in}{2.061602in}}%
\pgfpathlineto{\pgfqpoint{3.822479in}{2.068504in}}%
\pgfpathlineto{\pgfqpoint{3.830136in}{2.075461in}}%
\pgfpathlineto{\pgfqpoint{3.837788in}{2.082472in}}%
\pgfpathlineto{\pgfqpoint{3.845433in}{2.089531in}}%
\pgfpathclose%
\pgfusepath{fill}%
\end{pgfscope}%
\begin{pgfscope}%
\pgfpathrectangle{\pgfqpoint{1.254980in}{0.150000in}}{\pgfqpoint{5.490039in}{5.490039in}}%
\pgfusepath{clip}%
\pgfsetbuttcap%
\pgfsetroundjoin%
\definecolor{currentfill}{rgb}{0.278826,0.175490,0.483397}%
\pgfsetfillcolor{currentfill}%
\pgfsetfillopacity{0.700000}%
\pgfsetlinewidth{0.000000pt}%
\definecolor{currentstroke}{rgb}{0.000000,0.000000,0.000000}%
\pgfsetstrokecolor{currentstroke}%
\pgfsetdash{}{0pt}%
\pgfpathmoveto{\pgfqpoint{5.750325in}{2.395906in}}%
\pgfpathlineto{\pgfqpoint{5.763884in}{2.395036in}}%
\pgfpathlineto{\pgfqpoint{5.777451in}{2.394189in}}%
\pgfpathlineto{\pgfqpoint{5.791027in}{2.393366in}}%
\pgfpathlineto{\pgfqpoint{5.804612in}{2.392566in}}%
\pgfpathlineto{\pgfqpoint{5.797767in}{2.387563in}}%
\pgfpathlineto{\pgfqpoint{5.790915in}{2.382513in}}%
\pgfpathlineto{\pgfqpoint{5.784056in}{2.377413in}}%
\pgfpathlineto{\pgfqpoint{5.777190in}{2.372262in}}%
\pgfpathlineto{\pgfqpoint{5.763588in}{2.372970in}}%
\pgfpathlineto{\pgfqpoint{5.749994in}{2.373702in}}%
\pgfpathlineto{\pgfqpoint{5.736408in}{2.374457in}}%
\pgfpathlineto{\pgfqpoint{5.722831in}{2.375236in}}%
\pgfpathlineto{\pgfqpoint{5.729715in}{2.380474in}}%
\pgfpathlineto{\pgfqpoint{5.736592in}{2.385663in}}%
\pgfpathlineto{\pgfqpoint{5.743462in}{2.390806in}}%
\pgfpathlineto{\pgfqpoint{5.750325in}{2.395906in}}%
\pgfpathclose%
\pgfusepath{fill}%
\end{pgfscope}%
\begin{pgfscope}%
\pgfpathrectangle{\pgfqpoint{1.254980in}{0.150000in}}{\pgfqpoint{5.490039in}{5.490039in}}%
\pgfusepath{clip}%
\pgfsetbuttcap%
\pgfsetroundjoin%
\definecolor{currentfill}{rgb}{0.280267,0.073417,0.397163}%
\pgfsetfillcolor{currentfill}%
\pgfsetfillopacity{0.700000}%
\pgfsetlinewidth{0.000000pt}%
\definecolor{currentstroke}{rgb}{0.000000,0.000000,0.000000}%
\pgfsetstrokecolor{currentstroke}%
\pgfsetdash{}{0pt}%
\pgfpathmoveto{\pgfqpoint{4.715456in}{2.205082in}}%
\pgfpathlineto{\pgfqpoint{4.728715in}{2.203189in}}%
\pgfpathlineto{\pgfqpoint{4.741982in}{2.201321in}}%
\pgfpathlineto{\pgfqpoint{4.755255in}{2.199478in}}%
\pgfpathlineto{\pgfqpoint{4.768536in}{2.197660in}}%
\pgfpathlineto{\pgfqpoint{4.761221in}{2.189800in}}%
\pgfpathlineto{\pgfqpoint{4.753900in}{2.181888in}}%
\pgfpathlineto{\pgfqpoint{4.746572in}{2.173923in}}%
\pgfpathlineto{\pgfqpoint{4.739239in}{2.165907in}}%
\pgfpathlineto{\pgfqpoint{4.725947in}{2.167765in}}%
\pgfpathlineto{\pgfqpoint{4.712662in}{2.169649in}}%
\pgfpathlineto{\pgfqpoint{4.699385in}{2.171557in}}%
\pgfpathlineto{\pgfqpoint{4.686115in}{2.173489in}}%
\pgfpathlineto{\pgfqpoint{4.693459in}{2.181461in}}%
\pgfpathlineto{\pgfqpoint{4.700797in}{2.189383in}}%
\pgfpathlineto{\pgfqpoint{4.708130in}{2.197257in}}%
\pgfpathlineto{\pgfqpoint{4.715456in}{2.205082in}}%
\pgfpathclose%
\pgfusepath{fill}%
\end{pgfscope}%
\begin{pgfscope}%
\pgfpathrectangle{\pgfqpoint{1.254980in}{0.150000in}}{\pgfqpoint{5.490039in}{5.490039in}}%
\pgfusepath{clip}%
\pgfsetbuttcap%
\pgfsetroundjoin%
\definecolor{currentfill}{rgb}{0.277941,0.056324,0.381191}%
\pgfsetfillcolor{currentfill}%
\pgfsetfillopacity{0.700000}%
\pgfsetlinewidth{0.000000pt}%
\definecolor{currentstroke}{rgb}{0.000000,0.000000,0.000000}%
\pgfsetstrokecolor{currentstroke}%
\pgfsetdash{}{0pt}%
\pgfpathmoveto{\pgfqpoint{3.170769in}{2.173471in}}%
\pgfpathlineto{\pgfqpoint{3.183696in}{2.167548in}}%
\pgfpathlineto{\pgfqpoint{3.196626in}{2.161659in}}%
\pgfpathlineto{\pgfqpoint{3.209561in}{2.155806in}}%
\pgfpathlineto{\pgfqpoint{3.222499in}{2.149988in}}%
\pgfpathlineto{\pgfqpoint{3.214552in}{2.146467in}}%
\pgfpathlineto{\pgfqpoint{3.206595in}{2.143109in}}%
\pgfpathlineto{\pgfqpoint{3.198627in}{2.139918in}}%
\pgfpathlineto{\pgfqpoint{3.190649in}{2.136900in}}%
\pgfpathlineto{\pgfqpoint{3.177690in}{2.142926in}}%
\pgfpathlineto{\pgfqpoint{3.164734in}{2.148986in}}%
\pgfpathlineto{\pgfqpoint{3.151782in}{2.155082in}}%
\pgfpathlineto{\pgfqpoint{3.138835in}{2.161212in}}%
\pgfpathlineto{\pgfqpoint{3.146834in}{2.164018in}}%
\pgfpathlineto{\pgfqpoint{3.154823in}{2.167000in}}%
\pgfpathlineto{\pgfqpoint{3.162801in}{2.170153in}}%
\pgfpathlineto{\pgfqpoint{3.170769in}{2.173471in}}%
\pgfpathclose%
\pgfusepath{fill}%
\end{pgfscope}%
\begin{pgfscope}%
\pgfpathrectangle{\pgfqpoint{1.254980in}{0.150000in}}{\pgfqpoint{5.490039in}{5.490039in}}%
\pgfusepath{clip}%
\pgfsetbuttcap%
\pgfsetroundjoin%
\definecolor{currentfill}{rgb}{0.283072,0.130895,0.449241}%
\pgfsetfillcolor{currentfill}%
\pgfsetfillopacity{0.700000}%
\pgfsetlinewidth{0.000000pt}%
\definecolor{currentstroke}{rgb}{0.000000,0.000000,0.000000}%
\pgfsetstrokecolor{currentstroke}%
\pgfsetdash{}{0pt}%
\pgfpathmoveto{\pgfqpoint{5.233102in}{2.307530in}}%
\pgfpathlineto{\pgfqpoint{5.246512in}{2.306322in}}%
\pgfpathlineto{\pgfqpoint{5.259930in}{2.305139in}}%
\pgfpathlineto{\pgfqpoint{5.273355in}{2.303979in}}%
\pgfpathlineto{\pgfqpoint{5.286789in}{2.302843in}}%
\pgfpathlineto{\pgfqpoint{5.279690in}{2.296222in}}%
\pgfpathlineto{\pgfqpoint{5.272583in}{2.289535in}}%
\pgfpathlineto{\pgfqpoint{5.265469in}{2.282778in}}%
\pgfpathlineto{\pgfqpoint{5.258349in}{2.275952in}}%
\pgfpathlineto{\pgfqpoint{5.244901in}{2.277063in}}%
\pgfpathlineto{\pgfqpoint{5.231462in}{2.278197in}}%
\pgfpathlineto{\pgfqpoint{5.218030in}{2.279355in}}%
\pgfpathlineto{\pgfqpoint{5.204607in}{2.280538in}}%
\pgfpathlineto{\pgfqpoint{5.211741in}{2.287384in}}%
\pgfpathlineto{\pgfqpoint{5.218868in}{2.294165in}}%
\pgfpathlineto{\pgfqpoint{5.225988in}{2.300879in}}%
\pgfpathlineto{\pgfqpoint{5.233102in}{2.307530in}}%
\pgfpathclose%
\pgfusepath{fill}%
\end{pgfscope}%
\begin{pgfscope}%
\pgfpathrectangle{\pgfqpoint{1.254980in}{0.150000in}}{\pgfqpoint{5.490039in}{5.490039in}}%
\pgfusepath{clip}%
\pgfsetbuttcap%
\pgfsetroundjoin%
\definecolor{currentfill}{rgb}{0.269944,0.014625,0.341379}%
\pgfsetfillcolor{currentfill}%
\pgfsetfillopacity{0.700000}%
\pgfsetlinewidth{0.000000pt}%
\definecolor{currentstroke}{rgb}{0.000000,0.000000,0.000000}%
\pgfsetstrokecolor{currentstroke}%
\pgfsetdash{}{0pt}%
\pgfpathmoveto{\pgfqpoint{4.197878in}{2.112646in}}%
\pgfpathlineto{\pgfqpoint{4.211001in}{2.109711in}}%
\pgfpathlineto{\pgfqpoint{4.224131in}{2.106803in}}%
\pgfpathlineto{\pgfqpoint{4.237267in}{2.103921in}}%
\pgfpathlineto{\pgfqpoint{4.250410in}{2.101065in}}%
\pgfpathlineto{\pgfqpoint{4.242905in}{2.093046in}}%
\pgfpathlineto{\pgfqpoint{4.235396in}{2.085024in}}%
\pgfpathlineto{\pgfqpoint{4.227881in}{2.077002in}}%
\pgfpathlineto{\pgfqpoint{4.220360in}{2.068981in}}%
\pgfpathlineto{\pgfqpoint{4.207206in}{2.071941in}}%
\pgfpathlineto{\pgfqpoint{4.194059in}{2.074928in}}%
\pgfpathlineto{\pgfqpoint{4.180918in}{2.077941in}}%
\pgfpathlineto{\pgfqpoint{4.167783in}{2.080980in}}%
\pgfpathlineto{\pgfqpoint{4.175315in}{2.088891in}}%
\pgfpathlineto{\pgfqpoint{4.182841in}{2.096808in}}%
\pgfpathlineto{\pgfqpoint{4.190363in}{2.104727in}}%
\pgfpathlineto{\pgfqpoint{4.197878in}{2.112646in}}%
\pgfpathclose%
\pgfusepath{fill}%
\end{pgfscope}%
\begin{pgfscope}%
\pgfpathrectangle{\pgfqpoint{1.254980in}{0.150000in}}{\pgfqpoint{5.490039in}{5.490039in}}%
\pgfusepath{clip}%
\pgfsetbuttcap%
\pgfsetroundjoin%
\definecolor{currentfill}{rgb}{0.274952,0.037752,0.364543}%
\pgfsetfillcolor{currentfill}%
\pgfsetfillopacity{0.700000}%
\pgfsetlinewidth{0.000000pt}%
\definecolor{currentstroke}{rgb}{0.000000,0.000000,0.000000}%
\pgfsetstrokecolor{currentstroke}%
\pgfsetdash{}{0pt}%
\pgfpathmoveto{\pgfqpoint{4.415446in}{2.144301in}}%
\pgfpathlineto{\pgfqpoint{4.428626in}{2.141848in}}%
\pgfpathlineto{\pgfqpoint{4.441813in}{2.139420in}}%
\pgfpathlineto{\pgfqpoint{4.455006in}{2.137017in}}%
\pgfpathlineto{\pgfqpoint{4.468207in}{2.134640in}}%
\pgfpathlineto{\pgfqpoint{4.460779in}{2.126496in}}%
\pgfpathlineto{\pgfqpoint{4.453346in}{2.118325in}}%
\pgfpathlineto{\pgfqpoint{4.445907in}{2.110127in}}%
\pgfpathlineto{\pgfqpoint{4.438463in}{2.101905in}}%
\pgfpathlineto{\pgfqpoint{4.425252in}{2.104361in}}%
\pgfpathlineto{\pgfqpoint{4.412048in}{2.106842in}}%
\pgfpathlineto{\pgfqpoint{4.398850in}{2.109349in}}%
\pgfpathlineto{\pgfqpoint{4.385659in}{2.111881in}}%
\pgfpathlineto{\pgfqpoint{4.393114in}{2.120020in}}%
\pgfpathlineto{\pgfqpoint{4.400564in}{2.128137in}}%
\pgfpathlineto{\pgfqpoint{4.408008in}{2.136232in}}%
\pgfpathlineto{\pgfqpoint{4.415446in}{2.144301in}}%
\pgfpathclose%
\pgfusepath{fill}%
\end{pgfscope}%
\begin{pgfscope}%
\pgfpathrectangle{\pgfqpoint{1.254980in}{0.150000in}}{\pgfqpoint{5.490039in}{5.490039in}}%
\pgfusepath{clip}%
\pgfsetbuttcap%
\pgfsetroundjoin%
\definecolor{currentfill}{rgb}{0.267004,0.004874,0.329415}%
\pgfsetfillcolor{currentfill}%
\pgfsetfillopacity{0.700000}%
\pgfsetlinewidth{0.000000pt}%
\definecolor{currentstroke}{rgb}{0.000000,0.000000,0.000000}%
\pgfsetstrokecolor{currentstroke}%
\pgfsetdash{}{0pt}%
\pgfpathmoveto{\pgfqpoint{3.980326in}{2.089381in}}%
\pgfpathlineto{\pgfqpoint{3.993398in}{2.085903in}}%
\pgfpathlineto{\pgfqpoint{4.006477in}{2.082453in}}%
\pgfpathlineto{\pgfqpoint{4.019562in}{2.079030in}}%
\pgfpathlineto{\pgfqpoint{4.032653in}{2.075633in}}%
\pgfpathlineto{\pgfqpoint{4.025070in}{2.068034in}}%
\pgfpathlineto{\pgfqpoint{4.017480in}{2.060461in}}%
\pgfpathlineto{\pgfqpoint{4.009886in}{2.052919in}}%
\pgfpathlineto{\pgfqpoint{4.002285in}{2.045410in}}%
\pgfpathlineto{\pgfqpoint{3.989182in}{2.048936in}}%
\pgfpathlineto{\pgfqpoint{3.976085in}{2.052489in}}%
\pgfpathlineto{\pgfqpoint{3.962993in}{2.056069in}}%
\pgfpathlineto{\pgfqpoint{3.949908in}{2.059677in}}%
\pgfpathlineto{\pgfqpoint{3.957521in}{2.067051in}}%
\pgfpathlineto{\pgfqpoint{3.965128in}{2.074462in}}%
\pgfpathlineto{\pgfqpoint{3.972730in}{2.081906in}}%
\pgfpathlineto{\pgfqpoint{3.980326in}{2.089381in}}%
\pgfpathclose%
\pgfusepath{fill}%
\end{pgfscope}%
\begin{pgfscope}%
\pgfpathrectangle{\pgfqpoint{1.254980in}{0.150000in}}{\pgfqpoint{5.490039in}{5.490039in}}%
\pgfusepath{clip}%
\pgfsetbuttcap%
\pgfsetroundjoin%
\definecolor{currentfill}{rgb}{0.277134,0.185228,0.489898}%
\pgfsetfillcolor{currentfill}%
\pgfsetfillopacity{0.700000}%
\pgfsetlinewidth{0.000000pt}%
\definecolor{currentstroke}{rgb}{0.000000,0.000000,0.000000}%
\pgfsetstrokecolor{currentstroke}%
\pgfsetdash{}{0pt}%
\pgfpathmoveto{\pgfqpoint{2.608761in}{2.409027in}}%
\pgfpathlineto{\pgfqpoint{2.621649in}{2.400950in}}%
\pgfpathlineto{\pgfqpoint{2.634540in}{2.392921in}}%
\pgfpathlineto{\pgfqpoint{2.647432in}{2.384940in}}%
\pgfpathlineto{\pgfqpoint{2.660327in}{2.377006in}}%
\pgfpathlineto{\pgfqpoint{2.652005in}{2.377989in}}%
\pgfpathlineto{\pgfqpoint{2.643667in}{2.379233in}}%
\pgfpathlineto{\pgfqpoint{2.635312in}{2.380747in}}%
\pgfpathlineto{\pgfqpoint{2.626940in}{2.382535in}}%
\pgfpathlineto{\pgfqpoint{2.614015in}{2.390719in}}%
\pgfpathlineto{\pgfqpoint{2.601092in}{2.398950in}}%
\pgfpathlineto{\pgfqpoint{2.588171in}{2.407229in}}%
\pgfpathlineto{\pgfqpoint{2.575252in}{2.415557in}}%
\pgfpathlineto{\pgfqpoint{2.583655in}{2.413513in}}%
\pgfpathlineto{\pgfqpoint{2.592041in}{2.411748in}}%
\pgfpathlineto{\pgfqpoint{2.600409in}{2.410255in}}%
\pgfpathlineto{\pgfqpoint{2.608761in}{2.409027in}}%
\pgfpathclose%
\pgfusepath{fill}%
\end{pgfscope}%
\begin{pgfscope}%
\pgfpathrectangle{\pgfqpoint{1.254980in}{0.150000in}}{\pgfqpoint{5.490039in}{5.490039in}}%
\pgfusepath{clip}%
\pgfsetbuttcap%
\pgfsetroundjoin%
\definecolor{currentfill}{rgb}{0.282656,0.100196,0.422160}%
\pgfsetfillcolor{currentfill}%
\pgfsetfillopacity{0.700000}%
\pgfsetlinewidth{0.000000pt}%
\definecolor{currentstroke}{rgb}{0.000000,0.000000,0.000000}%
\pgfsetstrokecolor{currentstroke}%
\pgfsetdash{}{0pt}%
\pgfpathmoveto{\pgfqpoint{4.933172in}{2.245266in}}%
\pgfpathlineto{\pgfqpoint{4.946496in}{2.243709in}}%
\pgfpathlineto{\pgfqpoint{4.959829in}{2.242176in}}%
\pgfpathlineto{\pgfqpoint{4.973168in}{2.240668in}}%
\pgfpathlineto{\pgfqpoint{4.986516in}{2.239183in}}%
\pgfpathlineto{\pgfqpoint{4.979285in}{2.231727in}}%
\pgfpathlineto{\pgfqpoint{4.972047in}{2.224207in}}%
\pgfpathlineto{\pgfqpoint{4.964803in}{2.216624in}}%
\pgfpathlineto{\pgfqpoint{4.957552in}{2.208976in}}%
\pgfpathlineto{\pgfqpoint{4.944193in}{2.210474in}}%
\pgfpathlineto{\pgfqpoint{4.930841in}{2.211997in}}%
\pgfpathlineto{\pgfqpoint{4.917497in}{2.213543in}}%
\pgfpathlineto{\pgfqpoint{4.904161in}{2.215114in}}%
\pgfpathlineto{\pgfqpoint{4.911423in}{2.222743in}}%
\pgfpathlineto{\pgfqpoint{4.918679in}{2.230311in}}%
\pgfpathlineto{\pgfqpoint{4.925928in}{2.237818in}}%
\pgfpathlineto{\pgfqpoint{4.933172in}{2.245266in}}%
\pgfpathclose%
\pgfusepath{fill}%
\end{pgfscope}%
\begin{pgfscope}%
\pgfpathrectangle{\pgfqpoint{1.254980in}{0.150000in}}{\pgfqpoint{5.490039in}{5.490039in}}%
\pgfusepath{clip}%
\pgfsetbuttcap%
\pgfsetroundjoin%
\definecolor{currentfill}{rgb}{0.281887,0.150881,0.465405}%
\pgfsetfillcolor{currentfill}%
\pgfsetfillopacity{0.700000}%
\pgfsetlinewidth{0.000000pt}%
\definecolor{currentstroke}{rgb}{0.000000,0.000000,0.000000}%
\pgfsetstrokecolor{currentstroke}%
\pgfsetdash{}{0pt}%
\pgfpathmoveto{\pgfqpoint{5.450865in}{2.344609in}}%
\pgfpathlineto{\pgfqpoint{5.464342in}{2.343597in}}%
\pgfpathlineto{\pgfqpoint{5.477827in}{2.342608in}}%
\pgfpathlineto{\pgfqpoint{5.491321in}{2.341643in}}%
\pgfpathlineto{\pgfqpoint{5.504822in}{2.340702in}}%
\pgfpathlineto{\pgfqpoint{5.497824in}{2.334734in}}%
\pgfpathlineto{\pgfqpoint{5.490818in}{2.328702in}}%
\pgfpathlineto{\pgfqpoint{5.483804in}{2.322605in}}%
\pgfpathlineto{\pgfqpoint{5.476784in}{2.316440in}}%
\pgfpathlineto{\pgfqpoint{5.463267in}{2.317329in}}%
\pgfpathlineto{\pgfqpoint{5.449758in}{2.318242in}}%
\pgfpathlineto{\pgfqpoint{5.436257in}{2.319179in}}%
\pgfpathlineto{\pgfqpoint{5.422765in}{2.320140in}}%
\pgfpathlineto{\pgfqpoint{5.429801in}{2.326352in}}%
\pgfpathlineto{\pgfqpoint{5.436829in}{2.332500in}}%
\pgfpathlineto{\pgfqpoint{5.443850in}{2.338585in}}%
\pgfpathlineto{\pgfqpoint{5.450865in}{2.344609in}}%
\pgfpathclose%
\pgfusepath{fill}%
\end{pgfscope}%
\begin{pgfscope}%
\pgfpathrectangle{\pgfqpoint{1.254980in}{0.150000in}}{\pgfqpoint{5.490039in}{5.490039in}}%
\pgfusepath{clip}%
\pgfsetbuttcap%
\pgfsetroundjoin%
\definecolor{currentfill}{rgb}{0.269944,0.014625,0.341379}%
\pgfsetfillcolor{currentfill}%
\pgfsetfillopacity{0.700000}%
\pgfsetlinewidth{0.000000pt}%
\definecolor{currentstroke}{rgb}{0.000000,0.000000,0.000000}%
\pgfsetstrokecolor{currentstroke}%
\pgfsetdash{}{0pt}%
\pgfpathmoveto{\pgfqpoint{3.492749in}{2.100858in}}%
\pgfpathlineto{\pgfqpoint{3.505727in}{2.095972in}}%
\pgfpathlineto{\pgfqpoint{3.518710in}{2.091116in}}%
\pgfpathlineto{\pgfqpoint{3.531698in}{2.086292in}}%
\pgfpathlineto{\pgfqpoint{3.544691in}{2.081498in}}%
\pgfpathlineto{\pgfqpoint{3.536906in}{2.075973in}}%
\pgfpathlineto{\pgfqpoint{3.529112in}{2.070557in}}%
\pgfpathlineto{\pgfqpoint{3.521312in}{2.065252in}}%
\pgfpathlineto{\pgfqpoint{3.513503in}{2.060065in}}%
\pgfpathlineto{\pgfqpoint{3.500493in}{2.065040in}}%
\pgfpathlineto{\pgfqpoint{3.487488in}{2.070045in}}%
\pgfpathlineto{\pgfqpoint{3.474487in}{2.075081in}}%
\pgfpathlineto{\pgfqpoint{3.461492in}{2.080148in}}%
\pgfpathlineto{\pgfqpoint{3.469318in}{2.085150in}}%
\pgfpathlineto{\pgfqpoint{3.477136in}{2.090272in}}%
\pgfpathlineto{\pgfqpoint{3.484947in}{2.095509in}}%
\pgfpathlineto{\pgfqpoint{3.492749in}{2.100858in}}%
\pgfpathclose%
\pgfusepath{fill}%
\end{pgfscope}%
\begin{pgfscope}%
\pgfpathrectangle{\pgfqpoint{1.254980in}{0.150000in}}{\pgfqpoint{5.490039in}{5.490039in}}%
\pgfusepath{clip}%
\pgfsetbuttcap%
\pgfsetroundjoin%
\definecolor{currentfill}{rgb}{0.278791,0.062145,0.386592}%
\pgfsetfillcolor{currentfill}%
\pgfsetfillopacity{0.700000}%
\pgfsetlinewidth{0.000000pt}%
\definecolor{currentstroke}{rgb}{0.000000,0.000000,0.000000}%
\pgfsetstrokecolor{currentstroke}%
\pgfsetdash{}{0pt}%
\pgfpathmoveto{\pgfqpoint{4.633106in}{2.181468in}}%
\pgfpathlineto{\pgfqpoint{4.646347in}{2.179436in}}%
\pgfpathlineto{\pgfqpoint{4.659596in}{2.177429in}}%
\pgfpathlineto{\pgfqpoint{4.672852in}{2.175447in}}%
\pgfpathlineto{\pgfqpoint{4.686115in}{2.173489in}}%
\pgfpathlineto{\pgfqpoint{4.678765in}{2.165470in}}%
\pgfpathlineto{\pgfqpoint{4.671409in}{2.157404in}}%
\pgfpathlineto{\pgfqpoint{4.664047in}{2.149292in}}%
\pgfpathlineto{\pgfqpoint{4.656680in}{2.141134in}}%
\pgfpathlineto{\pgfqpoint{4.643406in}{2.143145in}}%
\pgfpathlineto{\pgfqpoint{4.630140in}{2.145180in}}%
\pgfpathlineto{\pgfqpoint{4.616880in}{2.147240in}}%
\pgfpathlineto{\pgfqpoint{4.603628in}{2.149325in}}%
\pgfpathlineto{\pgfqpoint{4.611006in}{2.157425in}}%
\pgfpathlineto{\pgfqpoint{4.618378in}{2.165482in}}%
\pgfpathlineto{\pgfqpoint{4.625745in}{2.173497in}}%
\pgfpathlineto{\pgfqpoint{4.633106in}{2.181468in}}%
\pgfpathclose%
\pgfusepath{fill}%
\end{pgfscope}%
\begin{pgfscope}%
\pgfpathrectangle{\pgfqpoint{1.254980in}{0.150000in}}{\pgfqpoint{5.490039in}{5.490039in}}%
\pgfusepath{clip}%
\pgfsetbuttcap%
\pgfsetroundjoin%
\definecolor{currentfill}{rgb}{0.267004,0.004874,0.329415}%
\pgfsetfillcolor{currentfill}%
\pgfsetfillopacity{0.700000}%
\pgfsetlinewidth{0.000000pt}%
\definecolor{currentstroke}{rgb}{0.000000,0.000000,0.000000}%
\pgfsetstrokecolor{currentstroke}%
\pgfsetdash{}{0pt}%
\pgfpathmoveto{\pgfqpoint{3.627718in}{2.086393in}}%
\pgfpathlineto{\pgfqpoint{3.640720in}{2.081918in}}%
\pgfpathlineto{\pgfqpoint{3.653728in}{2.077473in}}%
\pgfpathlineto{\pgfqpoint{3.666741in}{2.073058in}}%
\pgfpathlineto{\pgfqpoint{3.679759in}{2.068672in}}%
\pgfpathlineto{\pgfqpoint{3.672034in}{2.062431in}}%
\pgfpathlineto{\pgfqpoint{3.664302in}{2.056275in}}%
\pgfpathlineto{\pgfqpoint{3.656563in}{2.050207in}}%
\pgfpathlineto{\pgfqpoint{3.648817in}{2.044233in}}%
\pgfpathlineto{\pgfqpoint{3.635783in}{2.048787in}}%
\pgfpathlineto{\pgfqpoint{3.622755in}{2.053371in}}%
\pgfpathlineto{\pgfqpoint{3.609731in}{2.057984in}}%
\pgfpathlineto{\pgfqpoint{3.596713in}{2.062627in}}%
\pgfpathlineto{\pgfqpoint{3.604475in}{2.068428in}}%
\pgfpathlineto{\pgfqpoint{3.612230in}{2.074326in}}%
\pgfpathlineto{\pgfqpoint{3.619977in}{2.080315in}}%
\pgfpathlineto{\pgfqpoint{3.627718in}{2.086393in}}%
\pgfpathclose%
\pgfusepath{fill}%
\end{pgfscope}%
\begin{pgfscope}%
\pgfpathrectangle{\pgfqpoint{1.254980in}{0.150000in}}{\pgfqpoint{5.490039in}{5.490039in}}%
\pgfusepath{clip}%
\pgfsetbuttcap%
\pgfsetroundjoin%
\definecolor{currentfill}{rgb}{0.279574,0.170599,0.479997}%
\pgfsetfillcolor{currentfill}%
\pgfsetfillopacity{0.700000}%
\pgfsetlinewidth{0.000000pt}%
\definecolor{currentstroke}{rgb}{0.000000,0.000000,0.000000}%
\pgfsetstrokecolor{currentstroke}%
\pgfsetdash{}{0pt}%
\pgfpathmoveto{\pgfqpoint{5.668607in}{2.378585in}}%
\pgfpathlineto{\pgfqpoint{5.682151in}{2.377713in}}%
\pgfpathlineto{\pgfqpoint{5.695703in}{2.376864in}}%
\pgfpathlineto{\pgfqpoint{5.709263in}{2.376038in}}%
\pgfpathlineto{\pgfqpoint{5.722831in}{2.375236in}}%
\pgfpathlineto{\pgfqpoint{5.715940in}{2.369946in}}%
\pgfpathlineto{\pgfqpoint{5.709041in}{2.364602in}}%
\pgfpathlineto{\pgfqpoint{5.702136in}{2.359200in}}%
\pgfpathlineto{\pgfqpoint{5.695222in}{2.353739in}}%
\pgfpathlineto{\pgfqpoint{5.681636in}{2.354463in}}%
\pgfpathlineto{\pgfqpoint{5.668059in}{2.355210in}}%
\pgfpathlineto{\pgfqpoint{5.654490in}{2.355981in}}%
\pgfpathlineto{\pgfqpoint{5.640930in}{2.356775in}}%
\pgfpathlineto{\pgfqpoint{5.647860in}{2.362310in}}%
\pgfpathlineto{\pgfqpoint{5.654783in}{2.367788in}}%
\pgfpathlineto{\pgfqpoint{5.661699in}{2.373213in}}%
\pgfpathlineto{\pgfqpoint{5.668607in}{2.378585in}}%
\pgfpathclose%
\pgfusepath{fill}%
\end{pgfscope}%
\begin{pgfscope}%
\pgfpathrectangle{\pgfqpoint{1.254980in}{0.150000in}}{\pgfqpoint{5.490039in}{5.490039in}}%
\pgfusepath{clip}%
\pgfsetbuttcap%
\pgfsetroundjoin%
\definecolor{currentfill}{rgb}{0.281446,0.084320,0.407414}%
\pgfsetfillcolor{currentfill}%
\pgfsetfillopacity{0.700000}%
\pgfsetlinewidth{0.000000pt}%
\definecolor{currentstroke}{rgb}{0.000000,0.000000,0.000000}%
\pgfsetstrokecolor{currentstroke}%
\pgfsetdash{}{0pt}%
\pgfpathmoveto{\pgfqpoint{3.035390in}{2.211561in}}%
\pgfpathlineto{\pgfqpoint{3.048308in}{2.205138in}}%
\pgfpathlineto{\pgfqpoint{3.061229in}{2.198753in}}%
\pgfpathlineto{\pgfqpoint{3.074154in}{2.192405in}}%
\pgfpathlineto{\pgfqpoint{3.087082in}{2.186094in}}%
\pgfpathlineto{\pgfqpoint{3.079050in}{2.183685in}}%
\pgfpathlineto{\pgfqpoint{3.071005in}{2.181467in}}%
\pgfpathlineto{\pgfqpoint{3.062950in}{2.179445in}}%
\pgfpathlineto{\pgfqpoint{3.054882in}{2.177625in}}%
\pgfpathlineto{\pgfqpoint{3.041930in}{2.184157in}}%
\pgfpathlineto{\pgfqpoint{3.028982in}{2.190726in}}%
\pgfpathlineto{\pgfqpoint{3.016037in}{2.197332in}}%
\pgfpathlineto{\pgfqpoint{3.003096in}{2.203976in}}%
\pgfpathlineto{\pgfqpoint{3.011188in}{2.205571in}}%
\pgfpathlineto{\pgfqpoint{3.019267in}{2.207370in}}%
\pgfpathlineto{\pgfqpoint{3.027335in}{2.209369in}}%
\pgfpathlineto{\pgfqpoint{3.035390in}{2.211561in}}%
\pgfpathclose%
\pgfusepath{fill}%
\end{pgfscope}%
\begin{pgfscope}%
\pgfpathrectangle{\pgfqpoint{1.254980in}{0.150000in}}{\pgfqpoint{5.490039in}{5.490039in}}%
\pgfusepath{clip}%
\pgfsetbuttcap%
\pgfsetroundjoin%
\definecolor{currentfill}{rgb}{0.283187,0.125848,0.444960}%
\pgfsetfillcolor{currentfill}%
\pgfsetfillopacity{0.700000}%
\pgfsetlinewidth{0.000000pt}%
\definecolor{currentstroke}{rgb}{0.000000,0.000000,0.000000}%
\pgfsetstrokecolor{currentstroke}%
\pgfsetdash{}{0pt}%
\pgfpathmoveto{\pgfqpoint{2.848063in}{2.286748in}}%
\pgfpathlineto{\pgfqpoint{2.860965in}{2.279628in}}%
\pgfpathlineto{\pgfqpoint{2.873870in}{2.272550in}}%
\pgfpathlineto{\pgfqpoint{2.886778in}{2.265513in}}%
\pgfpathlineto{\pgfqpoint{2.899689in}{2.258516in}}%
\pgfpathlineto{\pgfqpoint{2.891535in}{2.257592in}}%
\pgfpathlineto{\pgfqpoint{2.883368in}{2.256891in}}%
\pgfpathlineto{\pgfqpoint{2.875187in}{2.256420in}}%
\pgfpathlineto{\pgfqpoint{2.866991in}{2.256184in}}%
\pgfpathlineto{\pgfqpoint{2.854054in}{2.263415in}}%
\pgfpathlineto{\pgfqpoint{2.841120in}{2.270687in}}%
\pgfpathlineto{\pgfqpoint{2.828188in}{2.278000in}}%
\pgfpathlineto{\pgfqpoint{2.815260in}{2.285355in}}%
\pgfpathlineto{\pgfqpoint{2.823482in}{2.285351in}}%
\pgfpathlineto{\pgfqpoint{2.831690in}{2.285585in}}%
\pgfpathlineto{\pgfqpoint{2.839883in}{2.286053in}}%
\pgfpathlineto{\pgfqpoint{2.848063in}{2.286748in}}%
\pgfpathclose%
\pgfusepath{fill}%
\end{pgfscope}%
\begin{pgfscope}%
\pgfpathrectangle{\pgfqpoint{1.254980in}{0.150000in}}{\pgfqpoint{5.490039in}{5.490039in}}%
\pgfusepath{clip}%
\pgfsetbuttcap%
\pgfsetroundjoin%
\definecolor{currentfill}{rgb}{0.272594,0.025563,0.353093}%
\pgfsetfillcolor{currentfill}%
\pgfsetfillopacity{0.700000}%
\pgfsetlinewidth{0.000000pt}%
\definecolor{currentstroke}{rgb}{0.000000,0.000000,0.000000}%
\pgfsetstrokecolor{currentstroke}%
\pgfsetdash{}{0pt}%
\pgfpathmoveto{\pgfqpoint{3.357697in}{2.121824in}}%
\pgfpathlineto{\pgfqpoint{3.370656in}{2.116502in}}%
\pgfpathlineto{\pgfqpoint{3.383618in}{2.111212in}}%
\pgfpathlineto{\pgfqpoint{3.396586in}{2.105955in}}%
\pgfpathlineto{\pgfqpoint{3.409557in}{2.100731in}}%
\pgfpathlineto{\pgfqpoint{3.401705in}{2.096043in}}%
\pgfpathlineto{\pgfqpoint{3.393844in}{2.091488in}}%
\pgfpathlineto{\pgfqpoint{3.385974in}{2.087071in}}%
\pgfpathlineto{\pgfqpoint{3.378096in}{2.082796in}}%
\pgfpathlineto{\pgfqpoint{3.365105in}{2.088215in}}%
\pgfpathlineto{\pgfqpoint{3.352119in}{2.093666in}}%
\pgfpathlineto{\pgfqpoint{3.339137in}{2.099149in}}%
\pgfpathlineto{\pgfqpoint{3.326160in}{2.104665in}}%
\pgfpathlineto{\pgfqpoint{3.334058in}{2.108741in}}%
\pgfpathlineto{\pgfqpoint{3.341947in}{2.112963in}}%
\pgfpathlineto{\pgfqpoint{3.349826in}{2.117325in}}%
\pgfpathlineto{\pgfqpoint{3.357697in}{2.121824in}}%
\pgfpathclose%
\pgfusepath{fill}%
\end{pgfscope}%
\begin{pgfscope}%
\pgfpathrectangle{\pgfqpoint{1.254980in}{0.150000in}}{\pgfqpoint{5.490039in}{5.490039in}}%
\pgfusepath{clip}%
\pgfsetbuttcap%
\pgfsetroundjoin%
\definecolor{currentfill}{rgb}{0.283229,0.120777,0.440584}%
\pgfsetfillcolor{currentfill}%
\pgfsetfillopacity{0.700000}%
\pgfsetlinewidth{0.000000pt}%
\definecolor{currentstroke}{rgb}{0.000000,0.000000,0.000000}%
\pgfsetstrokecolor{currentstroke}%
\pgfsetdash{}{0pt}%
\pgfpathmoveto{\pgfqpoint{5.150992in}{2.285506in}}%
\pgfpathlineto{\pgfqpoint{5.164384in}{2.284228in}}%
\pgfpathlineto{\pgfqpoint{5.177784in}{2.282974in}}%
\pgfpathlineto{\pgfqpoint{5.191191in}{2.281744in}}%
\pgfpathlineto{\pgfqpoint{5.204607in}{2.280538in}}%
\pgfpathlineto{\pgfqpoint{5.197466in}{2.273623in}}%
\pgfpathlineto{\pgfqpoint{5.190319in}{2.266641in}}%
\pgfpathlineto{\pgfqpoint{5.183164in}{2.259588in}}%
\pgfpathlineto{\pgfqpoint{5.176003in}{2.252465in}}%
\pgfpathlineto{\pgfqpoint{5.162574in}{2.253659in}}%
\pgfpathlineto{\pgfqpoint{5.149154in}{2.254877in}}%
\pgfpathlineto{\pgfqpoint{5.135741in}{2.256119in}}%
\pgfpathlineto{\pgfqpoint{5.122336in}{2.257385in}}%
\pgfpathlineto{\pgfqpoint{5.129510in}{2.264515in}}%
\pgfpathlineto{\pgfqpoint{5.136677in}{2.271578in}}%
\pgfpathlineto{\pgfqpoint{5.143838in}{2.278575in}}%
\pgfpathlineto{\pgfqpoint{5.150992in}{2.285506in}}%
\pgfpathclose%
\pgfusepath{fill}%
\end{pgfscope}%
\begin{pgfscope}%
\pgfpathrectangle{\pgfqpoint{1.254980in}{0.150000in}}{\pgfqpoint{5.490039in}{5.490039in}}%
\pgfusepath{clip}%
\pgfsetbuttcap%
\pgfsetroundjoin%
\definecolor{currentfill}{rgb}{0.276194,0.190074,0.493001}%
\pgfsetfillcolor{currentfill}%
\pgfsetfillopacity{0.700000}%
\pgfsetlinewidth{0.000000pt}%
\definecolor{currentstroke}{rgb}{0.000000,0.000000,0.000000}%
\pgfsetstrokecolor{currentstroke}%
\pgfsetdash{}{0pt}%
\pgfpathmoveto{\pgfqpoint{5.886267in}{2.408800in}}%
\pgfpathlineto{\pgfqpoint{5.899874in}{2.408013in}}%
\pgfpathlineto{\pgfqpoint{5.913491in}{2.407248in}}%
\pgfpathlineto{\pgfqpoint{5.927115in}{2.406507in}}%
\pgfpathlineto{\pgfqpoint{5.940749in}{2.405789in}}%
\pgfpathlineto{\pgfqpoint{5.933971in}{2.401156in}}%
\pgfpathlineto{\pgfqpoint{5.927185in}{2.396484in}}%
\pgfpathlineto{\pgfqpoint{5.920393in}{2.391769in}}%
\pgfpathlineto{\pgfqpoint{5.913593in}{2.387008in}}%
\pgfpathlineto{\pgfqpoint{5.899940in}{2.387621in}}%
\pgfpathlineto{\pgfqpoint{5.886296in}{2.388257in}}%
\pgfpathlineto{\pgfqpoint{5.872661in}{2.388917in}}%
\pgfpathlineto{\pgfqpoint{5.859034in}{2.389600in}}%
\pgfpathlineto{\pgfqpoint{5.865852in}{2.394462in}}%
\pgfpathlineto{\pgfqpoint{5.872664in}{2.399280in}}%
\pgfpathlineto{\pgfqpoint{5.879469in}{2.404058in}}%
\pgfpathlineto{\pgfqpoint{5.886267in}{2.408800in}}%
\pgfpathclose%
\pgfusepath{fill}%
\end{pgfscope}%
\begin{pgfscope}%
\pgfpathrectangle{\pgfqpoint{1.254980in}{0.150000in}}{\pgfqpoint{5.490039in}{5.490039in}}%
\pgfusepath{clip}%
\pgfsetbuttcap%
\pgfsetroundjoin%
\definecolor{currentfill}{rgb}{0.267004,0.004874,0.329415}%
\pgfsetfillcolor{currentfill}%
\pgfsetfillopacity{0.700000}%
\pgfsetlinewidth{0.000000pt}%
\definecolor{currentstroke}{rgb}{0.000000,0.000000,0.000000}%
\pgfsetstrokecolor{currentstroke}%
\pgfsetdash{}{0pt}%
\pgfpathmoveto{\pgfqpoint{3.762660in}{2.077774in}}%
\pgfpathlineto{\pgfqpoint{3.775690in}{2.073688in}}%
\pgfpathlineto{\pgfqpoint{3.788726in}{2.069631in}}%
\pgfpathlineto{\pgfqpoint{3.801768in}{2.065602in}}%
\pgfpathlineto{\pgfqpoint{3.814815in}{2.061602in}}%
\pgfpathlineto{\pgfqpoint{3.807145in}{2.054759in}}%
\pgfpathlineto{\pgfqpoint{3.799469in}{2.047980in}}%
\pgfpathlineto{\pgfqpoint{3.791786in}{2.041268in}}%
\pgfpathlineto{\pgfqpoint{3.784098in}{2.034626in}}%
\pgfpathlineto{\pgfqpoint{3.771036in}{2.038781in}}%
\pgfpathlineto{\pgfqpoint{3.757980in}{2.042965in}}%
\pgfpathlineto{\pgfqpoint{3.744930in}{2.047178in}}%
\pgfpathlineto{\pgfqpoint{3.731885in}{2.051419in}}%
\pgfpathlineto{\pgfqpoint{3.739588in}{2.057901in}}%
\pgfpathlineto{\pgfqpoint{3.747285in}{2.064456in}}%
\pgfpathlineto{\pgfqpoint{3.754976in}{2.071082in}}%
\pgfpathlineto{\pgfqpoint{3.762660in}{2.077774in}}%
\pgfpathclose%
\pgfusepath{fill}%
\end{pgfscope}%
\begin{pgfscope}%
\pgfpathrectangle{\pgfqpoint{1.254980in}{0.150000in}}{\pgfqpoint{5.490039in}{5.490039in}}%
\pgfusepath{clip}%
\pgfsetbuttcap%
\pgfsetroundjoin%
\definecolor{currentfill}{rgb}{0.268510,0.009605,0.335427}%
\pgfsetfillcolor{currentfill}%
\pgfsetfillopacity{0.700000}%
\pgfsetlinewidth{0.000000pt}%
\definecolor{currentstroke}{rgb}{0.000000,0.000000,0.000000}%
\pgfsetstrokecolor{currentstroke}%
\pgfsetdash{}{0pt}%
\pgfpathmoveto{\pgfqpoint{4.115307in}{2.093400in}}%
\pgfpathlineto{\pgfqpoint{4.128417in}{2.090255in}}%
\pgfpathlineto{\pgfqpoint{4.141533in}{2.087137in}}%
\pgfpathlineto{\pgfqpoint{4.154655in}{2.084045in}}%
\pgfpathlineto{\pgfqpoint{4.167783in}{2.080980in}}%
\pgfpathlineto{\pgfqpoint{4.160246in}{2.073075in}}%
\pgfpathlineto{\pgfqpoint{4.152703in}{2.065180in}}%
\pgfpathlineto{\pgfqpoint{4.145155in}{2.057297in}}%
\pgfpathlineto{\pgfqpoint{4.137601in}{2.049430in}}%
\pgfpathlineto{\pgfqpoint{4.124461in}{2.052612in}}%
\pgfpathlineto{\pgfqpoint{4.111327in}{2.055821in}}%
\pgfpathlineto{\pgfqpoint{4.098199in}{2.059056in}}%
\pgfpathlineto{\pgfqpoint{4.085078in}{2.062318in}}%
\pgfpathlineto{\pgfqpoint{4.092643in}{2.070064in}}%
\pgfpathlineto{\pgfqpoint{4.100203in}{2.077828in}}%
\pgfpathlineto{\pgfqpoint{4.107758in}{2.085607in}}%
\pgfpathlineto{\pgfqpoint{4.115307in}{2.093400in}}%
\pgfpathclose%
\pgfusepath{fill}%
\end{pgfscope}%
\begin{pgfscope}%
\pgfpathrectangle{\pgfqpoint{1.254980in}{0.150000in}}{\pgfqpoint{5.490039in}{5.490039in}}%
\pgfusepath{clip}%
\pgfsetbuttcap%
\pgfsetroundjoin%
\definecolor{currentfill}{rgb}{0.272594,0.025563,0.353093}%
\pgfsetfillcolor{currentfill}%
\pgfsetfillopacity{0.700000}%
\pgfsetlinewidth{0.000000pt}%
\definecolor{currentstroke}{rgb}{0.000000,0.000000,0.000000}%
\pgfsetstrokecolor{currentstroke}%
\pgfsetdash{}{0pt}%
\pgfpathmoveto{\pgfqpoint{4.332963in}{2.122266in}}%
\pgfpathlineto{\pgfqpoint{4.346127in}{2.119632in}}%
\pgfpathlineto{\pgfqpoint{4.359298in}{2.117022in}}%
\pgfpathlineto{\pgfqpoint{4.372475in}{2.114439in}}%
\pgfpathlineto{\pgfqpoint{4.385659in}{2.111881in}}%
\pgfpathlineto{\pgfqpoint{4.378199in}{2.103723in}}%
\pgfpathlineto{\pgfqpoint{4.370733in}{2.095547in}}%
\pgfpathlineto{\pgfqpoint{4.363262in}{2.087355in}}%
\pgfpathlineto{\pgfqpoint{4.355785in}{2.079149in}}%
\pgfpathlineto{\pgfqpoint{4.342590in}{2.081798in}}%
\pgfpathlineto{\pgfqpoint{4.329402in}{2.084473in}}%
\pgfpathlineto{\pgfqpoint{4.316220in}{2.087174in}}%
\pgfpathlineto{\pgfqpoint{4.303045in}{2.089900in}}%
\pgfpathlineto{\pgfqpoint{4.310532in}{2.098010in}}%
\pgfpathlineto{\pgfqpoint{4.318014in}{2.106109in}}%
\pgfpathlineto{\pgfqpoint{4.325491in}{2.114195in}}%
\pgfpathlineto{\pgfqpoint{4.332963in}{2.122266in}}%
\pgfpathclose%
\pgfusepath{fill}%
\end{pgfscope}%
\begin{pgfscope}%
\pgfpathrectangle{\pgfqpoint{1.254980in}{0.150000in}}{\pgfqpoint{5.490039in}{5.490039in}}%
\pgfusepath{clip}%
\pgfsetbuttcap%
\pgfsetroundjoin%
\definecolor{currentfill}{rgb}{0.281924,0.089666,0.412415}%
\pgfsetfillcolor{currentfill}%
\pgfsetfillopacity{0.700000}%
\pgfsetlinewidth{0.000000pt}%
\definecolor{currentstroke}{rgb}{0.000000,0.000000,0.000000}%
\pgfsetstrokecolor{currentstroke}%
\pgfsetdash{}{0pt}%
\pgfpathmoveto{\pgfqpoint{4.850890in}{2.221642in}}%
\pgfpathlineto{\pgfqpoint{4.864196in}{2.219974in}}%
\pgfpathlineto{\pgfqpoint{4.877510in}{2.218329in}}%
\pgfpathlineto{\pgfqpoint{4.890832in}{2.216710in}}%
\pgfpathlineto{\pgfqpoint{4.904161in}{2.215114in}}%
\pgfpathlineto{\pgfqpoint{4.896892in}{2.207425in}}%
\pgfpathlineto{\pgfqpoint{4.889618in}{2.199676in}}%
\pgfpathlineto{\pgfqpoint{4.882337in}{2.191866in}}%
\pgfpathlineto{\pgfqpoint{4.875050in}{2.183996in}}%
\pgfpathlineto{\pgfqpoint{4.861710in}{2.185618in}}%
\pgfpathlineto{\pgfqpoint{4.848377in}{2.187265in}}%
\pgfpathlineto{\pgfqpoint{4.835052in}{2.188936in}}%
\pgfpathlineto{\pgfqpoint{4.821734in}{2.190632in}}%
\pgfpathlineto{\pgfqpoint{4.829032in}{2.198470in}}%
\pgfpathlineto{\pgfqpoint{4.836324in}{2.206251in}}%
\pgfpathlineto{\pgfqpoint{4.843610in}{2.213975in}}%
\pgfpathlineto{\pgfqpoint{4.850890in}{2.221642in}}%
\pgfpathclose%
\pgfusepath{fill}%
\end{pgfscope}%
\begin{pgfscope}%
\pgfpathrectangle{\pgfqpoint{1.254980in}{0.150000in}}{\pgfqpoint{5.490039in}{5.490039in}}%
\pgfusepath{clip}%
\pgfsetbuttcap%
\pgfsetroundjoin%
\definecolor{currentfill}{rgb}{0.277018,0.050344,0.375715}%
\pgfsetfillcolor{currentfill}%
\pgfsetfillopacity{0.700000}%
\pgfsetlinewidth{0.000000pt}%
\definecolor{currentstroke}{rgb}{0.000000,0.000000,0.000000}%
\pgfsetstrokecolor{currentstroke}%
\pgfsetdash{}{0pt}%
\pgfpathmoveto{\pgfqpoint{3.222499in}{2.149988in}}%
\pgfpathlineto{\pgfqpoint{3.235442in}{2.144204in}}%
\pgfpathlineto{\pgfqpoint{3.248389in}{2.138455in}}%
\pgfpathlineto{\pgfqpoint{3.261340in}{2.132739in}}%
\pgfpathlineto{\pgfqpoint{3.274295in}{2.127058in}}%
\pgfpathlineto{\pgfqpoint{3.266368in}{2.123335in}}%
\pgfpathlineto{\pgfqpoint{3.258432in}{2.119771in}}%
\pgfpathlineto{\pgfqpoint{3.250485in}{2.116371in}}%
\pgfpathlineto{\pgfqpoint{3.242529in}{2.113141in}}%
\pgfpathlineto{\pgfqpoint{3.229553in}{2.119030in}}%
\pgfpathlineto{\pgfqpoint{3.216581in}{2.124953in}}%
\pgfpathlineto{\pgfqpoint{3.203613in}{2.130909in}}%
\pgfpathlineto{\pgfqpoint{3.190649in}{2.136900in}}%
\pgfpathlineto{\pgfqpoint{3.198627in}{2.139918in}}%
\pgfpathlineto{\pgfqpoint{3.206595in}{2.143109in}}%
\pgfpathlineto{\pgfqpoint{3.214552in}{2.146467in}}%
\pgfpathlineto{\pgfqpoint{3.222499in}{2.149988in}}%
\pgfpathclose%
\pgfusepath{fill}%
\end{pgfscope}%
\begin{pgfscope}%
\pgfpathrectangle{\pgfqpoint{1.254980in}{0.150000in}}{\pgfqpoint{5.490039in}{5.490039in}}%
\pgfusepath{clip}%
\pgfsetbuttcap%
\pgfsetroundjoin%
\definecolor{currentfill}{rgb}{0.282290,0.145912,0.461510}%
\pgfsetfillcolor{currentfill}%
\pgfsetfillopacity{0.700000}%
\pgfsetlinewidth{0.000000pt}%
\definecolor{currentstroke}{rgb}{0.000000,0.000000,0.000000}%
\pgfsetstrokecolor{currentstroke}%
\pgfsetdash{}{0pt}%
\pgfpathmoveto{\pgfqpoint{5.368877in}{2.324219in}}%
\pgfpathlineto{\pgfqpoint{5.382337in}{2.323164in}}%
\pgfpathlineto{\pgfqpoint{5.395805in}{2.322132in}}%
\pgfpathlineto{\pgfqpoint{5.409281in}{2.321124in}}%
\pgfpathlineto{\pgfqpoint{5.422765in}{2.320140in}}%
\pgfpathlineto{\pgfqpoint{5.415722in}{2.313861in}}%
\pgfpathlineto{\pgfqpoint{5.408672in}{2.307515in}}%
\pgfpathlineto{\pgfqpoint{5.401615in}{2.301099in}}%
\pgfpathlineto{\pgfqpoint{5.394551in}{2.294612in}}%
\pgfpathlineto{\pgfqpoint{5.381052in}{2.295558in}}%
\pgfpathlineto{\pgfqpoint{5.367561in}{2.296527in}}%
\pgfpathlineto{\pgfqpoint{5.354079in}{2.297520in}}%
\pgfpathlineto{\pgfqpoint{5.340605in}{2.298537in}}%
\pgfpathlineto{\pgfqpoint{5.347684in}{2.305058in}}%
\pgfpathlineto{\pgfqpoint{5.354755in}{2.311511in}}%
\pgfpathlineto{\pgfqpoint{5.361820in}{2.317897in}}%
\pgfpathlineto{\pgfqpoint{5.368877in}{2.324219in}}%
\pgfpathclose%
\pgfusepath{fill}%
\end{pgfscope}%
\begin{pgfscope}%
\pgfpathrectangle{\pgfqpoint{1.254980in}{0.150000in}}{\pgfqpoint{5.490039in}{5.490039in}}%
\pgfusepath{clip}%
\pgfsetbuttcap%
\pgfsetroundjoin%
\definecolor{currentfill}{rgb}{0.278826,0.175490,0.483397}%
\pgfsetfillcolor{currentfill}%
\pgfsetfillopacity{0.700000}%
\pgfsetlinewidth{0.000000pt}%
\definecolor{currentstroke}{rgb}{0.000000,0.000000,0.000000}%
\pgfsetstrokecolor{currentstroke}%
\pgfsetdash{}{0pt}%
\pgfpathmoveto{\pgfqpoint{2.660327in}{2.377006in}}%
\pgfpathlineto{\pgfqpoint{2.673225in}{2.369120in}}%
\pgfpathlineto{\pgfqpoint{2.686124in}{2.361281in}}%
\pgfpathlineto{\pgfqpoint{2.699026in}{2.353488in}}%
\pgfpathlineto{\pgfqpoint{2.711931in}{2.345740in}}%
\pgfpathlineto{\pgfqpoint{2.703637in}{2.346478in}}%
\pgfpathlineto{\pgfqpoint{2.695329in}{2.347474in}}%
\pgfpathlineto{\pgfqpoint{2.687003in}{2.348736in}}%
\pgfpathlineto{\pgfqpoint{2.678662in}{2.350270in}}%
\pgfpathlineto{\pgfqpoint{2.665728in}{2.358267in}}%
\pgfpathlineto{\pgfqpoint{2.652796in}{2.366310in}}%
\pgfpathlineto{\pgfqpoint{2.639867in}{2.374399in}}%
\pgfpathlineto{\pgfqpoint{2.626940in}{2.382535in}}%
\pgfpathlineto{\pgfqpoint{2.635312in}{2.380747in}}%
\pgfpathlineto{\pgfqpoint{2.643667in}{2.379233in}}%
\pgfpathlineto{\pgfqpoint{2.652005in}{2.377989in}}%
\pgfpathlineto{\pgfqpoint{2.660327in}{2.377006in}}%
\pgfpathclose%
\pgfusepath{fill}%
\end{pgfscope}%
\begin{pgfscope}%
\pgfpathrectangle{\pgfqpoint{1.254980in}{0.150000in}}{\pgfqpoint{5.490039in}{5.490039in}}%
\pgfusepath{clip}%
\pgfsetbuttcap%
\pgfsetroundjoin%
\definecolor{currentfill}{rgb}{0.267004,0.004874,0.329415}%
\pgfsetfillcolor{currentfill}%
\pgfsetfillopacity{0.700000}%
\pgfsetlinewidth{0.000000pt}%
\definecolor{currentstroke}{rgb}{0.000000,0.000000,0.000000}%
\pgfsetstrokecolor{currentstroke}%
\pgfsetdash{}{0pt}%
\pgfpathmoveto{\pgfqpoint{3.897624in}{2.074381in}}%
\pgfpathlineto{\pgfqpoint{3.910686in}{2.070664in}}%
\pgfpathlineto{\pgfqpoint{3.923754in}{2.066974in}}%
\pgfpathlineto{\pgfqpoint{3.936828in}{2.063311in}}%
\pgfpathlineto{\pgfqpoint{3.949908in}{2.059677in}}%
\pgfpathlineto{\pgfqpoint{3.942289in}{2.052342in}}%
\pgfpathlineto{\pgfqpoint{3.934664in}{2.045049in}}%
\pgfpathlineto{\pgfqpoint{3.927033in}{2.037803in}}%
\pgfpathlineto{\pgfqpoint{3.919397in}{2.030607in}}%
\pgfpathlineto{\pgfqpoint{3.906304in}{2.034384in}}%
\pgfpathlineto{\pgfqpoint{3.893217in}{2.038189in}}%
\pgfpathlineto{\pgfqpoint{3.880136in}{2.042022in}}%
\pgfpathlineto{\pgfqpoint{3.867060in}{2.045882in}}%
\pgfpathlineto{\pgfqpoint{3.874710in}{2.052931in}}%
\pgfpathlineto{\pgfqpoint{3.882354in}{2.060033in}}%
\pgfpathlineto{\pgfqpoint{3.889992in}{2.067184in}}%
\pgfpathlineto{\pgfqpoint{3.897624in}{2.074381in}}%
\pgfpathclose%
\pgfusepath{fill}%
\end{pgfscope}%
\begin{pgfscope}%
\pgfpathrectangle{\pgfqpoint{1.254980in}{0.150000in}}{\pgfqpoint{5.490039in}{5.490039in}}%
\pgfusepath{clip}%
\pgfsetbuttcap%
\pgfsetroundjoin%
\definecolor{currentfill}{rgb}{0.277018,0.050344,0.375715}%
\pgfsetfillcolor{currentfill}%
\pgfsetfillopacity{0.700000}%
\pgfsetlinewidth{0.000000pt}%
\definecolor{currentstroke}{rgb}{0.000000,0.000000,0.000000}%
\pgfsetstrokecolor{currentstroke}%
\pgfsetdash{}{0pt}%
\pgfpathmoveto{\pgfqpoint{4.550689in}{2.157915in}}%
\pgfpathlineto{\pgfqpoint{4.563913in}{2.155730in}}%
\pgfpathlineto{\pgfqpoint{4.577144in}{2.153570in}}%
\pgfpathlineto{\pgfqpoint{4.590382in}{2.151435in}}%
\pgfpathlineto{\pgfqpoint{4.603628in}{2.149325in}}%
\pgfpathlineto{\pgfqpoint{4.596244in}{2.141185in}}%
\pgfpathlineto{\pgfqpoint{4.588855in}{2.133005in}}%
\pgfpathlineto{\pgfqpoint{4.581460in}{2.124786in}}%
\pgfpathlineto{\pgfqpoint{4.574059in}{2.116530in}}%
\pgfpathlineto{\pgfqpoint{4.560803in}{2.118706in}}%
\pgfpathlineto{\pgfqpoint{4.547554in}{2.120907in}}%
\pgfpathlineto{\pgfqpoint{4.534312in}{2.123133in}}%
\pgfpathlineto{\pgfqpoint{4.521077in}{2.125384in}}%
\pgfpathlineto{\pgfqpoint{4.528488in}{2.133569in}}%
\pgfpathlineto{\pgfqpoint{4.535894in}{2.141720in}}%
\pgfpathlineto{\pgfqpoint{4.543294in}{2.149836in}}%
\pgfpathlineto{\pgfqpoint{4.550689in}{2.157915in}}%
\pgfpathclose%
\pgfusepath{fill}%
\end{pgfscope}%
\begin{pgfscope}%
\pgfpathrectangle{\pgfqpoint{1.254980in}{0.150000in}}{\pgfqpoint{5.490039in}{5.490039in}}%
\pgfusepath{clip}%
\pgfsetbuttcap%
\pgfsetroundjoin%
\definecolor{currentfill}{rgb}{0.283197,0.115680,0.436115}%
\pgfsetfillcolor{currentfill}%
\pgfsetfillopacity{0.700000}%
\pgfsetlinewidth{0.000000pt}%
\definecolor{currentstroke}{rgb}{0.000000,0.000000,0.000000}%
\pgfsetstrokecolor{currentstroke}%
\pgfsetdash{}{0pt}%
\pgfpathmoveto{\pgfqpoint{5.068795in}{2.262690in}}%
\pgfpathlineto{\pgfqpoint{5.082168in}{2.261327in}}%
\pgfpathlineto{\pgfqpoint{5.095550in}{2.259989in}}%
\pgfpathlineto{\pgfqpoint{5.108939in}{2.258675in}}%
\pgfpathlineto{\pgfqpoint{5.122336in}{2.257385in}}%
\pgfpathlineto{\pgfqpoint{5.115155in}{2.250187in}}%
\pgfpathlineto{\pgfqpoint{5.107968in}{2.242921in}}%
\pgfpathlineto{\pgfqpoint{5.100774in}{2.235586in}}%
\pgfpathlineto{\pgfqpoint{5.093573in}{2.228181in}}%
\pgfpathlineto{\pgfqpoint{5.080164in}{2.229472in}}%
\pgfpathlineto{\pgfqpoint{5.066762in}{2.230787in}}%
\pgfpathlineto{\pgfqpoint{5.053369in}{2.232126in}}%
\pgfpathlineto{\pgfqpoint{5.039983in}{2.233489in}}%
\pgfpathlineto{\pgfqpoint{5.047196in}{2.240888in}}%
\pgfpathlineto{\pgfqpoint{5.054402in}{2.248221in}}%
\pgfpathlineto{\pgfqpoint{5.061602in}{2.255488in}}%
\pgfpathlineto{\pgfqpoint{5.068795in}{2.262690in}}%
\pgfpathclose%
\pgfusepath{fill}%
\end{pgfscope}%
\begin{pgfscope}%
\pgfpathrectangle{\pgfqpoint{1.254980in}{0.150000in}}{\pgfqpoint{5.490039in}{5.490039in}}%
\pgfusepath{clip}%
\pgfsetbuttcap%
\pgfsetroundjoin%
\definecolor{currentfill}{rgb}{0.280255,0.165693,0.476498}%
\pgfsetfillcolor{currentfill}%
\pgfsetfillopacity{0.700000}%
\pgfsetlinewidth{0.000000pt}%
\definecolor{currentstroke}{rgb}{0.000000,0.000000,0.000000}%
\pgfsetstrokecolor{currentstroke}%
\pgfsetdash{}{0pt}%
\pgfpathmoveto{\pgfqpoint{5.586772in}{2.360188in}}%
\pgfpathlineto{\pgfqpoint{5.600299in}{2.359299in}}%
\pgfpathlineto{\pgfqpoint{5.613834in}{2.358434in}}%
\pgfpathlineto{\pgfqpoint{5.627378in}{2.357593in}}%
\pgfpathlineto{\pgfqpoint{5.640930in}{2.356775in}}%
\pgfpathlineto{\pgfqpoint{5.633992in}{2.351181in}}%
\pgfpathlineto{\pgfqpoint{5.627047in}{2.345526in}}%
\pgfpathlineto{\pgfqpoint{5.620095in}{2.339806in}}%
\pgfpathlineto{\pgfqpoint{5.613135in}{2.334020in}}%
\pgfpathlineto{\pgfqpoint{5.599567in}{2.334773in}}%
\pgfpathlineto{\pgfqpoint{5.586007in}{2.335549in}}%
\pgfpathlineto{\pgfqpoint{5.572455in}{2.336349in}}%
\pgfpathlineto{\pgfqpoint{5.558912in}{2.337172in}}%
\pgfpathlineto{\pgfqpoint{5.565888in}{2.343018in}}%
\pgfpathlineto{\pgfqpoint{5.572856in}{2.348801in}}%
\pgfpathlineto{\pgfqpoint{5.579818in}{2.354524in}}%
\pgfpathlineto{\pgfqpoint{5.586772in}{2.360188in}}%
\pgfpathclose%
\pgfusepath{fill}%
\end{pgfscope}%
\begin{pgfscope}%
\pgfpathrectangle{\pgfqpoint{1.254980in}{0.150000in}}{\pgfqpoint{5.490039in}{5.490039in}}%
\pgfusepath{clip}%
\pgfsetbuttcap%
\pgfsetroundjoin%
\definecolor{currentfill}{rgb}{0.280894,0.078907,0.402329}%
\pgfsetfillcolor{currentfill}%
\pgfsetfillopacity{0.700000}%
\pgfsetlinewidth{0.000000pt}%
\definecolor{currentstroke}{rgb}{0.000000,0.000000,0.000000}%
\pgfsetstrokecolor{currentstroke}%
\pgfsetdash{}{0pt}%
\pgfpathmoveto{\pgfqpoint{4.768536in}{2.197660in}}%
\pgfpathlineto{\pgfqpoint{4.781825in}{2.195866in}}%
\pgfpathlineto{\pgfqpoint{4.795120in}{2.194096in}}%
\pgfpathlineto{\pgfqpoint{4.808423in}{2.192352in}}%
\pgfpathlineto{\pgfqpoint{4.821734in}{2.190632in}}%
\pgfpathlineto{\pgfqpoint{4.814430in}{2.182737in}}%
\pgfpathlineto{\pgfqpoint{4.807119in}{2.174786in}}%
\pgfpathlineto{\pgfqpoint{4.799803in}{2.166780in}}%
\pgfpathlineto{\pgfqpoint{4.792481in}{2.158719in}}%
\pgfpathlineto{\pgfqpoint{4.779159in}{2.160479in}}%
\pgfpathlineto{\pgfqpoint{4.765845in}{2.162264in}}%
\pgfpathlineto{\pgfqpoint{4.752539in}{2.164073in}}%
\pgfpathlineto{\pgfqpoint{4.739239in}{2.165907in}}%
\pgfpathlineto{\pgfqpoint{4.746572in}{2.173923in}}%
\pgfpathlineto{\pgfqpoint{4.753900in}{2.181888in}}%
\pgfpathlineto{\pgfqpoint{4.761221in}{2.189800in}}%
\pgfpathlineto{\pgfqpoint{4.768536in}{2.197660in}}%
\pgfpathclose%
\pgfusepath{fill}%
\end{pgfscope}%
\begin{pgfscope}%
\pgfpathrectangle{\pgfqpoint{1.254980in}{0.150000in}}{\pgfqpoint{5.490039in}{5.490039in}}%
\pgfusepath{clip}%
\pgfsetbuttcap%
\pgfsetroundjoin%
\definecolor{currentfill}{rgb}{0.277134,0.185228,0.489898}%
\pgfsetfillcolor{currentfill}%
\pgfsetfillopacity{0.700000}%
\pgfsetlinewidth{0.000000pt}%
\definecolor{currentstroke}{rgb}{0.000000,0.000000,0.000000}%
\pgfsetstrokecolor{currentstroke}%
\pgfsetdash{}{0pt}%
\pgfpathmoveto{\pgfqpoint{5.804612in}{2.392566in}}%
\pgfpathlineto{\pgfqpoint{5.818204in}{2.391790in}}%
\pgfpathlineto{\pgfqpoint{5.831806in}{2.391037in}}%
\pgfpathlineto{\pgfqpoint{5.845415in}{2.390307in}}%
\pgfpathlineto{\pgfqpoint{5.859034in}{2.389600in}}%
\pgfpathlineto{\pgfqpoint{5.852208in}{2.384693in}}%
\pgfpathlineto{\pgfqpoint{5.845374in}{2.379736in}}%
\pgfpathlineto{\pgfqpoint{5.838534in}{2.374727in}}%
\pgfpathlineto{\pgfqpoint{5.831686in}{2.369662in}}%
\pgfpathlineto{\pgfqpoint{5.818049in}{2.370277in}}%
\pgfpathlineto{\pgfqpoint{5.804421in}{2.370915in}}%
\pgfpathlineto{\pgfqpoint{5.790801in}{2.371577in}}%
\pgfpathlineto{\pgfqpoint{5.777190in}{2.372262in}}%
\pgfpathlineto{\pgfqpoint{5.784056in}{2.377413in}}%
\pgfpathlineto{\pgfqpoint{5.790915in}{2.382513in}}%
\pgfpathlineto{\pgfqpoint{5.797767in}{2.387563in}}%
\pgfpathlineto{\pgfqpoint{5.804612in}{2.392566in}}%
\pgfpathclose%
\pgfusepath{fill}%
\end{pgfscope}%
\begin{pgfscope}%
\pgfpathrectangle{\pgfqpoint{1.254980in}{0.150000in}}{\pgfqpoint{5.490039in}{5.490039in}}%
\pgfusepath{clip}%
\pgfsetbuttcap%
\pgfsetroundjoin%
\definecolor{currentfill}{rgb}{0.271305,0.019942,0.347269}%
\pgfsetfillcolor{currentfill}%
\pgfsetfillopacity{0.700000}%
\pgfsetlinewidth{0.000000pt}%
\definecolor{currentstroke}{rgb}{0.000000,0.000000,0.000000}%
\pgfsetstrokecolor{currentstroke}%
\pgfsetdash{}{0pt}%
\pgfpathmoveto{\pgfqpoint{4.250410in}{2.101065in}}%
\pgfpathlineto{\pgfqpoint{4.263559in}{2.098235in}}%
\pgfpathlineto{\pgfqpoint{4.276714in}{2.095431in}}%
\pgfpathlineto{\pgfqpoint{4.289876in}{2.092653in}}%
\pgfpathlineto{\pgfqpoint{4.303045in}{2.089900in}}%
\pgfpathlineto{\pgfqpoint{4.295552in}{2.081782in}}%
\pgfpathlineto{\pgfqpoint{4.288053in}{2.073658in}}%
\pgfpathlineto{\pgfqpoint{4.280550in}{2.065529in}}%
\pgfpathlineto{\pgfqpoint{4.273041in}{2.057399in}}%
\pgfpathlineto{\pgfqpoint{4.259861in}{2.060256in}}%
\pgfpathlineto{\pgfqpoint{4.246687in}{2.063138in}}%
\pgfpathlineto{\pgfqpoint{4.233521in}{2.066047in}}%
\pgfpathlineto{\pgfqpoint{4.220360in}{2.068981in}}%
\pgfpathlineto{\pgfqpoint{4.227881in}{2.077002in}}%
\pgfpathlineto{\pgfqpoint{4.235396in}{2.085024in}}%
\pgfpathlineto{\pgfqpoint{4.242905in}{2.093046in}}%
\pgfpathlineto{\pgfqpoint{4.250410in}{2.101065in}}%
\pgfpathclose%
\pgfusepath{fill}%
\end{pgfscope}%
\begin{pgfscope}%
\pgfpathrectangle{\pgfqpoint{1.254980in}{0.150000in}}{\pgfqpoint{5.490039in}{5.490039in}}%
\pgfusepath{clip}%
\pgfsetbuttcap%
\pgfsetroundjoin%
\definecolor{currentfill}{rgb}{0.283197,0.115680,0.436115}%
\pgfsetfillcolor{currentfill}%
\pgfsetfillopacity{0.700000}%
\pgfsetlinewidth{0.000000pt}%
\definecolor{currentstroke}{rgb}{0.000000,0.000000,0.000000}%
\pgfsetstrokecolor{currentstroke}%
\pgfsetdash{}{0pt}%
\pgfpathmoveto{\pgfqpoint{2.899689in}{2.258516in}}%
\pgfpathlineto{\pgfqpoint{2.912604in}{2.251561in}}%
\pgfpathlineto{\pgfqpoint{2.925521in}{2.244645in}}%
\pgfpathlineto{\pgfqpoint{2.938442in}{2.237769in}}%
\pgfpathlineto{\pgfqpoint{2.951366in}{2.230933in}}%
\pgfpathlineto{\pgfqpoint{2.943237in}{2.229779in}}%
\pgfpathlineto{\pgfqpoint{2.935096in}{2.228845in}}%
\pgfpathlineto{\pgfqpoint{2.926940in}{2.228138in}}%
\pgfpathlineto{\pgfqpoint{2.918772in}{2.227662in}}%
\pgfpathlineto{\pgfqpoint{2.905822in}{2.234733in}}%
\pgfpathlineto{\pgfqpoint{2.892875in}{2.241844in}}%
\pgfpathlineto{\pgfqpoint{2.879932in}{2.248994in}}%
\pgfpathlineto{\pgfqpoint{2.866991in}{2.256184in}}%
\pgfpathlineto{\pgfqpoint{2.875187in}{2.256420in}}%
\pgfpathlineto{\pgfqpoint{2.883368in}{2.256891in}}%
\pgfpathlineto{\pgfqpoint{2.891535in}{2.257592in}}%
\pgfpathlineto{\pgfqpoint{2.899689in}{2.258516in}}%
\pgfpathclose%
\pgfusepath{fill}%
\end{pgfscope}%
\begin{pgfscope}%
\pgfpathrectangle{\pgfqpoint{1.254980in}{0.150000in}}{\pgfqpoint{5.490039in}{5.490039in}}%
\pgfusepath{clip}%
\pgfsetbuttcap%
\pgfsetroundjoin%
\definecolor{currentfill}{rgb}{0.267004,0.004874,0.329415}%
\pgfsetfillcolor{currentfill}%
\pgfsetfillopacity{0.700000}%
\pgfsetlinewidth{0.000000pt}%
\definecolor{currentstroke}{rgb}{0.000000,0.000000,0.000000}%
\pgfsetstrokecolor{currentstroke}%
\pgfsetdash{}{0pt}%
\pgfpathmoveto{\pgfqpoint{4.032653in}{2.075633in}}%
\pgfpathlineto{\pgfqpoint{4.045750in}{2.072264in}}%
\pgfpathlineto{\pgfqpoint{4.058853in}{2.068922in}}%
\pgfpathlineto{\pgfqpoint{4.071962in}{2.065607in}}%
\pgfpathlineto{\pgfqpoint{4.085078in}{2.062318in}}%
\pgfpathlineto{\pgfqpoint{4.077506in}{2.054594in}}%
\pgfpathlineto{\pgfqpoint{4.069930in}{2.046893in}}%
\pgfpathlineto{\pgfqpoint{4.062347in}{2.039219in}}%
\pgfpathlineto{\pgfqpoint{4.054759in}{2.031575in}}%
\pgfpathlineto{\pgfqpoint{4.041632in}{2.034994in}}%
\pgfpathlineto{\pgfqpoint{4.028510in}{2.038439in}}%
\pgfpathlineto{\pgfqpoint{4.015395in}{2.041911in}}%
\pgfpathlineto{\pgfqpoint{4.002285in}{2.045410in}}%
\pgfpathlineto{\pgfqpoint{4.009886in}{2.052919in}}%
\pgfpathlineto{\pgfqpoint{4.017480in}{2.060461in}}%
\pgfpathlineto{\pgfqpoint{4.025070in}{2.068034in}}%
\pgfpathlineto{\pgfqpoint{4.032653in}{2.075633in}}%
\pgfpathclose%
\pgfusepath{fill}%
\end{pgfscope}%
\begin{pgfscope}%
\pgfpathrectangle{\pgfqpoint{1.254980in}{0.150000in}}{\pgfqpoint{5.490039in}{5.490039in}}%
\pgfusepath{clip}%
\pgfsetbuttcap%
\pgfsetroundjoin%
\definecolor{currentfill}{rgb}{0.280267,0.073417,0.397163}%
\pgfsetfillcolor{currentfill}%
\pgfsetfillopacity{0.700000}%
\pgfsetlinewidth{0.000000pt}%
\definecolor{currentstroke}{rgb}{0.000000,0.000000,0.000000}%
\pgfsetstrokecolor{currentstroke}%
\pgfsetdash{}{0pt}%
\pgfpathmoveto{\pgfqpoint{3.087082in}{2.186094in}}%
\pgfpathlineto{\pgfqpoint{3.100015in}{2.179819in}}%
\pgfpathlineto{\pgfqpoint{3.112951in}{2.173581in}}%
\pgfpathlineto{\pgfqpoint{3.125891in}{2.167379in}}%
\pgfpathlineto{\pgfqpoint{3.138835in}{2.161212in}}%
\pgfpathlineto{\pgfqpoint{3.130824in}{2.158588in}}%
\pgfpathlineto{\pgfqpoint{3.122803in}{2.156152in}}%
\pgfpathlineto{\pgfqpoint{3.114770in}{2.153907in}}%
\pgfpathlineto{\pgfqpoint{3.106726in}{2.151862in}}%
\pgfpathlineto{\pgfqpoint{3.093760in}{2.158249in}}%
\pgfpathlineto{\pgfqpoint{3.080797in}{2.164671in}}%
\pgfpathlineto{\pgfqpoint{3.067838in}{2.171130in}}%
\pgfpathlineto{\pgfqpoint{3.054882in}{2.177625in}}%
\pgfpathlineto{\pgfqpoint{3.062950in}{2.179445in}}%
\pgfpathlineto{\pgfqpoint{3.071005in}{2.181467in}}%
\pgfpathlineto{\pgfqpoint{3.079050in}{2.183685in}}%
\pgfpathlineto{\pgfqpoint{3.087082in}{2.186094in}}%
\pgfpathclose%
\pgfusepath{fill}%
\end{pgfscope}%
\begin{pgfscope}%
\pgfpathrectangle{\pgfqpoint{1.254980in}{0.150000in}}{\pgfqpoint{5.490039in}{5.490039in}}%
\pgfusepath{clip}%
\pgfsetbuttcap%
\pgfsetroundjoin%
\definecolor{currentfill}{rgb}{0.268510,0.009605,0.335427}%
\pgfsetfillcolor{currentfill}%
\pgfsetfillopacity{0.700000}%
\pgfsetlinewidth{0.000000pt}%
\definecolor{currentstroke}{rgb}{0.000000,0.000000,0.000000}%
\pgfsetstrokecolor{currentstroke}%
\pgfsetdash{}{0pt}%
\pgfpathmoveto{\pgfqpoint{3.544691in}{2.081498in}}%
\pgfpathlineto{\pgfqpoint{3.557689in}{2.076735in}}%
\pgfpathlineto{\pgfqpoint{3.570692in}{2.072002in}}%
\pgfpathlineto{\pgfqpoint{3.583700in}{2.067299in}}%
\pgfpathlineto{\pgfqpoint{3.596713in}{2.062627in}}%
\pgfpathlineto{\pgfqpoint{3.588944in}{2.056926in}}%
\pgfpathlineto{\pgfqpoint{3.581168in}{2.051330in}}%
\pgfpathlineto{\pgfqpoint{3.573384in}{2.045843in}}%
\pgfpathlineto{\pgfqpoint{3.565592in}{2.040470in}}%
\pgfpathlineto{\pgfqpoint{3.552562in}{2.045324in}}%
\pgfpathlineto{\pgfqpoint{3.539537in}{2.050207in}}%
\pgfpathlineto{\pgfqpoint{3.526518in}{2.055121in}}%
\pgfpathlineto{\pgfqpoint{3.513503in}{2.060065in}}%
\pgfpathlineto{\pgfqpoint{3.521312in}{2.065252in}}%
\pgfpathlineto{\pgfqpoint{3.529112in}{2.070557in}}%
\pgfpathlineto{\pgfqpoint{3.536906in}{2.075973in}}%
\pgfpathlineto{\pgfqpoint{3.544691in}{2.081498in}}%
\pgfpathclose%
\pgfusepath{fill}%
\end{pgfscope}%
\begin{pgfscope}%
\pgfpathrectangle{\pgfqpoint{1.254980in}{0.150000in}}{\pgfqpoint{5.490039in}{5.490039in}}%
\pgfusepath{clip}%
\pgfsetbuttcap%
\pgfsetroundjoin%
\definecolor{currentfill}{rgb}{0.276022,0.044167,0.370164}%
\pgfsetfillcolor{currentfill}%
\pgfsetfillopacity{0.700000}%
\pgfsetlinewidth{0.000000pt}%
\definecolor{currentstroke}{rgb}{0.000000,0.000000,0.000000}%
\pgfsetstrokecolor{currentstroke}%
\pgfsetdash{}{0pt}%
\pgfpathmoveto{\pgfqpoint{4.468207in}{2.134640in}}%
\pgfpathlineto{\pgfqpoint{4.481414in}{2.132288in}}%
\pgfpathlineto{\pgfqpoint{4.494628in}{2.129962in}}%
\pgfpathlineto{\pgfqpoint{4.507849in}{2.127660in}}%
\pgfpathlineto{\pgfqpoint{4.521077in}{2.125384in}}%
\pgfpathlineto{\pgfqpoint{4.513660in}{2.117167in}}%
\pgfpathlineto{\pgfqpoint{4.506238in}{2.108918in}}%
\pgfpathlineto{\pgfqpoint{4.498810in}{2.100640in}}%
\pgfpathlineto{\pgfqpoint{4.491377in}{2.092334in}}%
\pgfpathlineto{\pgfqpoint{4.478138in}{2.094689in}}%
\pgfpathlineto{\pgfqpoint{4.464906in}{2.097069in}}%
\pgfpathlineto{\pgfqpoint{4.451681in}{2.099474in}}%
\pgfpathlineto{\pgfqpoint{4.438463in}{2.101905in}}%
\pgfpathlineto{\pgfqpoint{4.445907in}{2.110127in}}%
\pgfpathlineto{\pgfqpoint{4.453346in}{2.118325in}}%
\pgfpathlineto{\pgfqpoint{4.460779in}{2.126496in}}%
\pgfpathlineto{\pgfqpoint{4.468207in}{2.134640in}}%
\pgfpathclose%
\pgfusepath{fill}%
\end{pgfscope}%
\begin{pgfscope}%
\pgfpathrectangle{\pgfqpoint{1.254980in}{0.150000in}}{\pgfqpoint{5.490039in}{5.490039in}}%
\pgfusepath{clip}%
\pgfsetbuttcap%
\pgfsetroundjoin%
\definecolor{currentfill}{rgb}{0.282623,0.140926,0.457517}%
\pgfsetfillcolor{currentfill}%
\pgfsetfillopacity{0.700000}%
\pgfsetlinewidth{0.000000pt}%
\definecolor{currentstroke}{rgb}{0.000000,0.000000,0.000000}%
\pgfsetstrokecolor{currentstroke}%
\pgfsetdash{}{0pt}%
\pgfpathmoveto{\pgfqpoint{5.286789in}{2.302843in}}%
\pgfpathlineto{\pgfqpoint{5.300231in}{2.301731in}}%
\pgfpathlineto{\pgfqpoint{5.313681in}{2.300642in}}%
\pgfpathlineto{\pgfqpoint{5.327139in}{2.299578in}}%
\pgfpathlineto{\pgfqpoint{5.340605in}{2.298537in}}%
\pgfpathlineto{\pgfqpoint{5.333519in}{2.291947in}}%
\pgfpathlineto{\pgfqpoint{5.326426in}{2.285287in}}%
\pgfpathlineto{\pgfqpoint{5.319327in}{2.278554in}}%
\pgfpathlineto{\pgfqpoint{5.312219in}{2.271748in}}%
\pgfpathlineto{\pgfqpoint{5.298740in}{2.272764in}}%
\pgfpathlineto{\pgfqpoint{5.285268in}{2.273803in}}%
\pgfpathlineto{\pgfqpoint{5.271804in}{2.274865in}}%
\pgfpathlineto{\pgfqpoint{5.258349in}{2.275952in}}%
\pgfpathlineto{\pgfqpoint{5.265469in}{2.282778in}}%
\pgfpathlineto{\pgfqpoint{5.272583in}{2.289535in}}%
\pgfpathlineto{\pgfqpoint{5.279690in}{2.296222in}}%
\pgfpathlineto{\pgfqpoint{5.286789in}{2.302843in}}%
\pgfpathclose%
\pgfusepath{fill}%
\end{pgfscope}%
\begin{pgfscope}%
\pgfpathrectangle{\pgfqpoint{1.254980in}{0.150000in}}{\pgfqpoint{5.490039in}{5.490039in}}%
\pgfusepath{clip}%
\pgfsetbuttcap%
\pgfsetroundjoin%
\definecolor{currentfill}{rgb}{0.267004,0.004874,0.329415}%
\pgfsetfillcolor{currentfill}%
\pgfsetfillopacity{0.700000}%
\pgfsetlinewidth{0.000000pt}%
\definecolor{currentstroke}{rgb}{0.000000,0.000000,0.000000}%
\pgfsetstrokecolor{currentstroke}%
\pgfsetdash{}{0pt}%
\pgfpathmoveto{\pgfqpoint{3.679759in}{2.068672in}}%
\pgfpathlineto{\pgfqpoint{3.692782in}{2.064315in}}%
\pgfpathlineto{\pgfqpoint{3.705811in}{2.059987in}}%
\pgfpathlineto{\pgfqpoint{3.718845in}{2.055688in}}%
\pgfpathlineto{\pgfqpoint{3.731885in}{2.051419in}}%
\pgfpathlineto{\pgfqpoint{3.724175in}{2.045014in}}%
\pgfpathlineto{\pgfqpoint{3.716458in}{2.038692in}}%
\pgfpathlineto{\pgfqpoint{3.708735in}{2.032455in}}%
\pgfpathlineto{\pgfqpoint{3.701004in}{2.026308in}}%
\pgfpathlineto{\pgfqpoint{3.687949in}{2.030746in}}%
\pgfpathlineto{\pgfqpoint{3.674900in}{2.035212in}}%
\pgfpathlineto{\pgfqpoint{3.661856in}{2.039708in}}%
\pgfpathlineto{\pgfqpoint{3.648817in}{2.044233in}}%
\pgfpathlineto{\pgfqpoint{3.656563in}{2.050207in}}%
\pgfpathlineto{\pgfqpoint{3.664302in}{2.056275in}}%
\pgfpathlineto{\pgfqpoint{3.672034in}{2.062431in}}%
\pgfpathlineto{\pgfqpoint{3.679759in}{2.068672in}}%
\pgfpathclose%
\pgfusepath{fill}%
\end{pgfscope}%
\begin{pgfscope}%
\pgfpathrectangle{\pgfqpoint{1.254980in}{0.150000in}}{\pgfqpoint{5.490039in}{5.490039in}}%
\pgfusepath{clip}%
\pgfsetbuttcap%
\pgfsetroundjoin%
\definecolor{currentfill}{rgb}{0.271305,0.019942,0.347269}%
\pgfsetfillcolor{currentfill}%
\pgfsetfillopacity{0.700000}%
\pgfsetlinewidth{0.000000pt}%
\definecolor{currentstroke}{rgb}{0.000000,0.000000,0.000000}%
\pgfsetstrokecolor{currentstroke}%
\pgfsetdash{}{0pt}%
\pgfpathmoveto{\pgfqpoint{3.409557in}{2.100731in}}%
\pgfpathlineto{\pgfqpoint{3.422534in}{2.095538in}}%
\pgfpathlineto{\pgfqpoint{3.435515in}{2.090376in}}%
\pgfpathlineto{\pgfqpoint{3.448501in}{2.085247in}}%
\pgfpathlineto{\pgfqpoint{3.461492in}{2.080148in}}%
\pgfpathlineto{\pgfqpoint{3.453658in}{2.075272in}}%
\pgfpathlineto{\pgfqpoint{3.445815in}{2.070525in}}%
\pgfpathlineto{\pgfqpoint{3.437964in}{2.065912in}}%
\pgfpathlineto{\pgfqpoint{3.430104in}{2.061438in}}%
\pgfpathlineto{\pgfqpoint{3.417095in}{2.066731in}}%
\pgfpathlineto{\pgfqpoint{3.404091in}{2.072054in}}%
\pgfpathlineto{\pgfqpoint{3.391091in}{2.077409in}}%
\pgfpathlineto{\pgfqpoint{3.378096in}{2.082796in}}%
\pgfpathlineto{\pgfqpoint{3.385974in}{2.087071in}}%
\pgfpathlineto{\pgfqpoint{3.393844in}{2.091488in}}%
\pgfpathlineto{\pgfqpoint{3.401705in}{2.096043in}}%
\pgfpathlineto{\pgfqpoint{3.409557in}{2.100731in}}%
\pgfpathclose%
\pgfusepath{fill}%
\end{pgfscope}%
\begin{pgfscope}%
\pgfpathrectangle{\pgfqpoint{1.254980in}{0.150000in}}{\pgfqpoint{5.490039in}{5.490039in}}%
\pgfusepath{clip}%
\pgfsetbuttcap%
\pgfsetroundjoin%
\definecolor{currentfill}{rgb}{0.282910,0.105393,0.426902}%
\pgfsetfillcolor{currentfill}%
\pgfsetfillopacity{0.700000}%
\pgfsetlinewidth{0.000000pt}%
\definecolor{currentstroke}{rgb}{0.000000,0.000000,0.000000}%
\pgfsetstrokecolor{currentstroke}%
\pgfsetdash{}{0pt}%
\pgfpathmoveto{\pgfqpoint{4.986516in}{2.239183in}}%
\pgfpathlineto{\pgfqpoint{4.999871in}{2.237724in}}%
\pgfpathlineto{\pgfqpoint{5.013234in}{2.236288in}}%
\pgfpathlineto{\pgfqpoint{5.026604in}{2.234876in}}%
\pgfpathlineto{\pgfqpoint{5.039983in}{2.233489in}}%
\pgfpathlineto{\pgfqpoint{5.032763in}{2.226024in}}%
\pgfpathlineto{\pgfqpoint{5.025538in}{2.218492in}}%
\pgfpathlineto{\pgfqpoint{5.018305in}{2.210892in}}%
\pgfpathlineto{\pgfqpoint{5.011066in}{2.203226in}}%
\pgfpathlineto{\pgfqpoint{4.997676in}{2.204627in}}%
\pgfpathlineto{\pgfqpoint{4.984294in}{2.206053in}}%
\pgfpathlineto{\pgfqpoint{4.970919in}{2.207502in}}%
\pgfpathlineto{\pgfqpoint{4.957552in}{2.208976in}}%
\pgfpathlineto{\pgfqpoint{4.964803in}{2.216624in}}%
\pgfpathlineto{\pgfqpoint{4.972047in}{2.224207in}}%
\pgfpathlineto{\pgfqpoint{4.979285in}{2.231727in}}%
\pgfpathlineto{\pgfqpoint{4.986516in}{2.239183in}}%
\pgfpathclose%
\pgfusepath{fill}%
\end{pgfscope}%
\begin{pgfscope}%
\pgfpathrectangle{\pgfqpoint{1.254980in}{0.150000in}}{\pgfqpoint{5.490039in}{5.490039in}}%
\pgfusepath{clip}%
\pgfsetbuttcap%
\pgfsetroundjoin%
\definecolor{currentfill}{rgb}{0.280868,0.160771,0.472899}%
\pgfsetfillcolor{currentfill}%
\pgfsetfillopacity{0.700000}%
\pgfsetlinewidth{0.000000pt}%
\definecolor{currentstroke}{rgb}{0.000000,0.000000,0.000000}%
\pgfsetstrokecolor{currentstroke}%
\pgfsetdash{}{0pt}%
\pgfpathmoveto{\pgfqpoint{2.711931in}{2.345740in}}%
\pgfpathlineto{\pgfqpoint{2.724838in}{2.338038in}}%
\pgfpathlineto{\pgfqpoint{2.737747in}{2.330380in}}%
\pgfpathlineto{\pgfqpoint{2.750659in}{2.322768in}}%
\pgfpathlineto{\pgfqpoint{2.763574in}{2.315199in}}%
\pgfpathlineto{\pgfqpoint{2.755309in}{2.315692in}}%
\pgfpathlineto{\pgfqpoint{2.747029in}{2.316441in}}%
\pgfpathlineto{\pgfqpoint{2.738733in}{2.317452in}}%
\pgfpathlineto{\pgfqpoint{2.730422in}{2.318731in}}%
\pgfpathlineto{\pgfqpoint{2.717478in}{2.326549in}}%
\pgfpathlineto{\pgfqpoint{2.704537in}{2.334411in}}%
\pgfpathlineto{\pgfqpoint{2.691598in}{2.342318in}}%
\pgfpathlineto{\pgfqpoint{2.678662in}{2.350270in}}%
\pgfpathlineto{\pgfqpoint{2.687003in}{2.348736in}}%
\pgfpathlineto{\pgfqpoint{2.695329in}{2.347474in}}%
\pgfpathlineto{\pgfqpoint{2.703637in}{2.346478in}}%
\pgfpathlineto{\pgfqpoint{2.711931in}{2.345740in}}%
\pgfpathclose%
\pgfusepath{fill}%
\end{pgfscope}%
\begin{pgfscope}%
\pgfpathrectangle{\pgfqpoint{1.254980in}{0.150000in}}{\pgfqpoint{5.490039in}{5.490039in}}%
\pgfusepath{clip}%
\pgfsetbuttcap%
\pgfsetroundjoin%
\definecolor{currentfill}{rgb}{0.267004,0.004874,0.329415}%
\pgfsetfillcolor{currentfill}%
\pgfsetfillopacity{0.700000}%
\pgfsetlinewidth{0.000000pt}%
\definecolor{currentstroke}{rgb}{0.000000,0.000000,0.000000}%
\pgfsetstrokecolor{currentstroke}%
\pgfsetdash{}{0pt}%
\pgfpathmoveto{\pgfqpoint{3.814815in}{2.061602in}}%
\pgfpathlineto{\pgfqpoint{3.827868in}{2.057630in}}%
\pgfpathlineto{\pgfqpoint{3.840926in}{2.053686in}}%
\pgfpathlineto{\pgfqpoint{3.853990in}{2.049770in}}%
\pgfpathlineto{\pgfqpoint{3.867060in}{2.045882in}}%
\pgfpathlineto{\pgfqpoint{3.859404in}{2.038889in}}%
\pgfpathlineto{\pgfqpoint{3.851742in}{2.031956in}}%
\pgfpathlineto{\pgfqpoint{3.844073in}{2.025087in}}%
\pgfpathlineto{\pgfqpoint{3.836399in}{2.018284in}}%
\pgfpathlineto{\pgfqpoint{3.823315in}{2.022328in}}%
\pgfpathlineto{\pgfqpoint{3.810237in}{2.026399in}}%
\pgfpathlineto{\pgfqpoint{3.797164in}{2.030498in}}%
\pgfpathlineto{\pgfqpoint{3.784098in}{2.034626in}}%
\pgfpathlineto{\pgfqpoint{3.791786in}{2.041268in}}%
\pgfpathlineto{\pgfqpoint{3.799469in}{2.047980in}}%
\pgfpathlineto{\pgfqpoint{3.807145in}{2.054759in}}%
\pgfpathlineto{\pgfqpoint{3.814815in}{2.061602in}}%
\pgfpathclose%
\pgfusepath{fill}%
\end{pgfscope}%
\begin{pgfscope}%
\pgfpathrectangle{\pgfqpoint{1.254980in}{0.150000in}}{\pgfqpoint{5.490039in}{5.490039in}}%
\pgfusepath{clip}%
\pgfsetbuttcap%
\pgfsetroundjoin%
\definecolor{currentfill}{rgb}{0.279566,0.067836,0.391917}%
\pgfsetfillcolor{currentfill}%
\pgfsetfillopacity{0.700000}%
\pgfsetlinewidth{0.000000pt}%
\definecolor{currentstroke}{rgb}{0.000000,0.000000,0.000000}%
\pgfsetstrokecolor{currentstroke}%
\pgfsetdash{}{0pt}%
\pgfpathmoveto{\pgfqpoint{4.686115in}{2.173489in}}%
\pgfpathlineto{\pgfqpoint{4.699385in}{2.171557in}}%
\pgfpathlineto{\pgfqpoint{4.712662in}{2.169649in}}%
\pgfpathlineto{\pgfqpoint{4.725947in}{2.167765in}}%
\pgfpathlineto{\pgfqpoint{4.739239in}{2.165907in}}%
\pgfpathlineto{\pgfqpoint{4.731900in}{2.157840in}}%
\pgfpathlineto{\pgfqpoint{4.724555in}{2.149722in}}%
\pgfpathlineto{\pgfqpoint{4.717205in}{2.141555in}}%
\pgfpathlineto{\pgfqpoint{4.709848in}{2.133340in}}%
\pgfpathlineto{\pgfqpoint{4.696545in}{2.135252in}}%
\pgfpathlineto{\pgfqpoint{4.683250in}{2.137188in}}%
\pgfpathlineto{\pgfqpoint{4.669961in}{2.139149in}}%
\pgfpathlineto{\pgfqpoint{4.656680in}{2.141134in}}%
\pgfpathlineto{\pgfqpoint{4.664047in}{2.149292in}}%
\pgfpathlineto{\pgfqpoint{4.671409in}{2.157404in}}%
\pgfpathlineto{\pgfqpoint{4.678765in}{2.165470in}}%
\pgfpathlineto{\pgfqpoint{4.686115in}{2.173489in}}%
\pgfpathclose%
\pgfusepath{fill}%
\end{pgfscope}%
\begin{pgfscope}%
\pgfpathrectangle{\pgfqpoint{1.254980in}{0.150000in}}{\pgfqpoint{5.490039in}{5.490039in}}%
\pgfusepath{clip}%
\pgfsetbuttcap%
\pgfsetroundjoin%
\definecolor{currentfill}{rgb}{0.280868,0.160771,0.472899}%
\pgfsetfillcolor{currentfill}%
\pgfsetfillopacity{0.700000}%
\pgfsetlinewidth{0.000000pt}%
\definecolor{currentstroke}{rgb}{0.000000,0.000000,0.000000}%
\pgfsetstrokecolor{currentstroke}%
\pgfsetdash{}{0pt}%
\pgfpathmoveto{\pgfqpoint{5.504822in}{2.340702in}}%
\pgfpathlineto{\pgfqpoint{5.518332in}{2.339784in}}%
\pgfpathlineto{\pgfqpoint{5.531851in}{2.338890in}}%
\pgfpathlineto{\pgfqpoint{5.545377in}{2.338019in}}%
\pgfpathlineto{\pgfqpoint{5.558912in}{2.337172in}}%
\pgfpathlineto{\pgfqpoint{5.551929in}{2.331261in}}%
\pgfpathlineto{\pgfqpoint{5.544938in}{2.325283in}}%
\pgfpathlineto{\pgfqpoint{5.537940in}{2.319236in}}%
\pgfpathlineto{\pgfqpoint{5.530935in}{2.313118in}}%
\pgfpathlineto{\pgfqpoint{5.517384in}{2.313913in}}%
\pgfpathlineto{\pgfqpoint{5.503842in}{2.314732in}}%
\pgfpathlineto{\pgfqpoint{5.490309in}{2.315574in}}%
\pgfpathlineto{\pgfqpoint{5.476784in}{2.316440in}}%
\pgfpathlineto{\pgfqpoint{5.483804in}{2.322605in}}%
\pgfpathlineto{\pgfqpoint{5.490818in}{2.328702in}}%
\pgfpathlineto{\pgfqpoint{5.497824in}{2.334734in}}%
\pgfpathlineto{\pgfqpoint{5.504822in}{2.340702in}}%
\pgfpathclose%
\pgfusepath{fill}%
\end{pgfscope}%
\begin{pgfscope}%
\pgfpathrectangle{\pgfqpoint{1.254980in}{0.150000in}}{\pgfqpoint{5.490039in}{5.490039in}}%
\pgfusepath{clip}%
\pgfsetbuttcap%
\pgfsetroundjoin%
\definecolor{currentfill}{rgb}{0.276022,0.044167,0.370164}%
\pgfsetfillcolor{currentfill}%
\pgfsetfillopacity{0.700000}%
\pgfsetlinewidth{0.000000pt}%
\definecolor{currentstroke}{rgb}{0.000000,0.000000,0.000000}%
\pgfsetstrokecolor{currentstroke}%
\pgfsetdash{}{0pt}%
\pgfpathmoveto{\pgfqpoint{3.274295in}{2.127058in}}%
\pgfpathlineto{\pgfqpoint{3.287255in}{2.121410in}}%
\pgfpathlineto{\pgfqpoint{3.300219in}{2.115795in}}%
\pgfpathlineto{\pgfqpoint{3.313187in}{2.110214in}}%
\pgfpathlineto{\pgfqpoint{3.326160in}{2.104665in}}%
\pgfpathlineto{\pgfqpoint{3.318253in}{2.100740in}}%
\pgfpathlineto{\pgfqpoint{3.310337in}{2.096971in}}%
\pgfpathlineto{\pgfqpoint{3.302411in}{2.093362in}}%
\pgfpathlineto{\pgfqpoint{3.294475in}{2.089921in}}%
\pgfpathlineto{\pgfqpoint{3.281482in}{2.095676in}}%
\pgfpathlineto{\pgfqpoint{3.268493in}{2.101465in}}%
\pgfpathlineto{\pgfqpoint{3.255509in}{2.107286in}}%
\pgfpathlineto{\pgfqpoint{3.242529in}{2.113141in}}%
\pgfpathlineto{\pgfqpoint{3.250485in}{2.116371in}}%
\pgfpathlineto{\pgfqpoint{3.258432in}{2.119771in}}%
\pgfpathlineto{\pgfqpoint{3.266368in}{2.123335in}}%
\pgfpathlineto{\pgfqpoint{3.274295in}{2.127058in}}%
\pgfpathclose%
\pgfusepath{fill}%
\end{pgfscope}%
\begin{pgfscope}%
\pgfpathrectangle{\pgfqpoint{1.254980in}{0.150000in}}{\pgfqpoint{5.490039in}{5.490039in}}%
\pgfusepath{clip}%
\pgfsetbuttcap%
\pgfsetroundjoin%
\definecolor{currentfill}{rgb}{0.275191,0.194905,0.496005}%
\pgfsetfillcolor{currentfill}%
\pgfsetfillopacity{0.700000}%
\pgfsetlinewidth{0.000000pt}%
\definecolor{currentstroke}{rgb}{0.000000,0.000000,0.000000}%
\pgfsetstrokecolor{currentstroke}%
\pgfsetdash{}{0pt}%
\pgfpathmoveto{\pgfqpoint{5.940749in}{2.405789in}}%
\pgfpathlineto{\pgfqpoint{5.954391in}{2.405094in}}%
\pgfpathlineto{\pgfqpoint{5.968042in}{2.404423in}}%
\pgfpathlineto{\pgfqpoint{5.981701in}{2.403775in}}%
\pgfpathlineto{\pgfqpoint{5.974938in}{2.399224in}}%
\pgfpathlineto{\pgfqpoint{5.968167in}{2.394632in}}%
\pgfpathlineto{\pgfqpoint{5.961389in}{2.389995in}}%
\pgfpathlineto{\pgfqpoint{5.954604in}{2.385308in}}%
\pgfpathlineto{\pgfqpoint{5.940925in}{2.385852in}}%
\pgfpathlineto{\pgfqpoint{5.927255in}{2.386418in}}%
\pgfpathlineto{\pgfqpoint{5.913593in}{2.387008in}}%
\pgfpathlineto{\pgfqpoint{5.920393in}{2.391769in}}%
\pgfpathlineto{\pgfqpoint{5.927185in}{2.396484in}}%
\pgfpathlineto{\pgfqpoint{5.933971in}{2.401156in}}%
\pgfpathlineto{\pgfqpoint{5.940749in}{2.405789in}}%
\pgfpathclose%
\pgfusepath{fill}%
\end{pgfscope}%
\begin{pgfscope}%
\pgfpathrectangle{\pgfqpoint{1.254980in}{0.150000in}}{\pgfqpoint{5.490039in}{5.490039in}}%
\pgfusepath{clip}%
\pgfsetbuttcap%
\pgfsetroundjoin%
\definecolor{currentfill}{rgb}{0.269944,0.014625,0.341379}%
\pgfsetfillcolor{currentfill}%
\pgfsetfillopacity{0.700000}%
\pgfsetlinewidth{0.000000pt}%
\definecolor{currentstroke}{rgb}{0.000000,0.000000,0.000000}%
\pgfsetstrokecolor{currentstroke}%
\pgfsetdash{}{0pt}%
\pgfpathmoveto{\pgfqpoint{4.167783in}{2.080980in}}%
\pgfpathlineto{\pgfqpoint{4.180918in}{2.077941in}}%
\pgfpathlineto{\pgfqpoint{4.194059in}{2.074928in}}%
\pgfpathlineto{\pgfqpoint{4.207206in}{2.071941in}}%
\pgfpathlineto{\pgfqpoint{4.220360in}{2.068981in}}%
\pgfpathlineto{\pgfqpoint{4.212834in}{2.060964in}}%
\pgfpathlineto{\pgfqpoint{4.205303in}{2.052954in}}%
\pgfpathlineto{\pgfqpoint{4.197767in}{2.044952in}}%
\pgfpathlineto{\pgfqpoint{4.190225in}{2.036962in}}%
\pgfpathlineto{\pgfqpoint{4.177059in}{2.040040in}}%
\pgfpathlineto{\pgfqpoint{4.163900in}{2.043144in}}%
\pgfpathlineto{\pgfqpoint{4.150747in}{2.046274in}}%
\pgfpathlineto{\pgfqpoint{4.137601in}{2.049430in}}%
\pgfpathlineto{\pgfqpoint{4.145155in}{2.057297in}}%
\pgfpathlineto{\pgfqpoint{4.152703in}{2.065180in}}%
\pgfpathlineto{\pgfqpoint{4.160246in}{2.073075in}}%
\pgfpathlineto{\pgfqpoint{4.167783in}{2.080980in}}%
\pgfpathclose%
\pgfusepath{fill}%
\end{pgfscope}%
\begin{pgfscope}%
\pgfpathrectangle{\pgfqpoint{1.254980in}{0.150000in}}{\pgfqpoint{5.490039in}{5.490039in}}%
\pgfusepath{clip}%
\pgfsetbuttcap%
\pgfsetroundjoin%
\definecolor{currentfill}{rgb}{0.278012,0.180367,0.486697}%
\pgfsetfillcolor{currentfill}%
\pgfsetfillopacity{0.700000}%
\pgfsetlinewidth{0.000000pt}%
\definecolor{currentstroke}{rgb}{0.000000,0.000000,0.000000}%
\pgfsetstrokecolor{currentstroke}%
\pgfsetdash{}{0pt}%
\pgfpathmoveto{\pgfqpoint{5.722831in}{2.375236in}}%
\pgfpathlineto{\pgfqpoint{5.736408in}{2.374457in}}%
\pgfpathlineto{\pgfqpoint{5.749994in}{2.373702in}}%
\pgfpathlineto{\pgfqpoint{5.763588in}{2.372970in}}%
\pgfpathlineto{\pgfqpoint{5.777190in}{2.372262in}}%
\pgfpathlineto{\pgfqpoint{5.770316in}{2.367055in}}%
\pgfpathlineto{\pgfqpoint{5.763435in}{2.361791in}}%
\pgfpathlineto{\pgfqpoint{5.756547in}{2.356466in}}%
\pgfpathlineto{\pgfqpoint{5.749650in}{2.351079in}}%
\pgfpathlineto{\pgfqpoint{5.736031in}{2.351708in}}%
\pgfpathlineto{\pgfqpoint{5.722419in}{2.352362in}}%
\pgfpathlineto{\pgfqpoint{5.708816in}{2.353039in}}%
\pgfpathlineto{\pgfqpoint{5.695222in}{2.353739in}}%
\pgfpathlineto{\pgfqpoint{5.702136in}{2.359200in}}%
\pgfpathlineto{\pgfqpoint{5.709041in}{2.364602in}}%
\pgfpathlineto{\pgfqpoint{5.715940in}{2.369946in}}%
\pgfpathlineto{\pgfqpoint{5.722831in}{2.375236in}}%
\pgfpathclose%
\pgfusepath{fill}%
\end{pgfscope}%
\begin{pgfscope}%
\pgfpathrectangle{\pgfqpoint{1.254980in}{0.150000in}}{\pgfqpoint{5.490039in}{5.490039in}}%
\pgfusepath{clip}%
\pgfsetbuttcap%
\pgfsetroundjoin%
\definecolor{currentfill}{rgb}{0.273809,0.031497,0.358853}%
\pgfsetfillcolor{currentfill}%
\pgfsetfillopacity{0.700000}%
\pgfsetlinewidth{0.000000pt}%
\definecolor{currentstroke}{rgb}{0.000000,0.000000,0.000000}%
\pgfsetstrokecolor{currentstroke}%
\pgfsetdash{}{0pt}%
\pgfpathmoveto{\pgfqpoint{4.385659in}{2.111881in}}%
\pgfpathlineto{\pgfqpoint{4.398850in}{2.109349in}}%
\pgfpathlineto{\pgfqpoint{4.412048in}{2.106842in}}%
\pgfpathlineto{\pgfqpoint{4.425252in}{2.104361in}}%
\pgfpathlineto{\pgfqpoint{4.438463in}{2.101905in}}%
\pgfpathlineto{\pgfqpoint{4.431014in}{2.093660in}}%
\pgfpathlineto{\pgfqpoint{4.423559in}{2.085394in}}%
\pgfpathlineto{\pgfqpoint{4.416099in}{2.077108in}}%
\pgfpathlineto{\pgfqpoint{4.408633in}{2.068806in}}%
\pgfpathlineto{\pgfqpoint{4.395411in}{2.071354in}}%
\pgfpathlineto{\pgfqpoint{4.382196in}{2.073927in}}%
\pgfpathlineto{\pgfqpoint{4.368987in}{2.076525in}}%
\pgfpathlineto{\pgfqpoint{4.355785in}{2.079149in}}%
\pgfpathlineto{\pgfqpoint{4.363262in}{2.087355in}}%
\pgfpathlineto{\pgfqpoint{4.370733in}{2.095547in}}%
\pgfpathlineto{\pgfqpoint{4.378199in}{2.103723in}}%
\pgfpathlineto{\pgfqpoint{4.385659in}{2.111881in}}%
\pgfpathclose%
\pgfusepath{fill}%
\end{pgfscope}%
\begin{pgfscope}%
\pgfpathrectangle{\pgfqpoint{1.254980in}{0.150000in}}{\pgfqpoint{5.490039in}{5.490039in}}%
\pgfusepath{clip}%
\pgfsetbuttcap%
\pgfsetroundjoin%
\definecolor{currentfill}{rgb}{0.283072,0.130895,0.449241}%
\pgfsetfillcolor{currentfill}%
\pgfsetfillopacity{0.700000}%
\pgfsetlinewidth{0.000000pt}%
\definecolor{currentstroke}{rgb}{0.000000,0.000000,0.000000}%
\pgfsetstrokecolor{currentstroke}%
\pgfsetdash{}{0pt}%
\pgfpathmoveto{\pgfqpoint{5.204607in}{2.280538in}}%
\pgfpathlineto{\pgfqpoint{5.218030in}{2.279355in}}%
\pgfpathlineto{\pgfqpoint{5.231462in}{2.278197in}}%
\pgfpathlineto{\pgfqpoint{5.244901in}{2.277063in}}%
\pgfpathlineto{\pgfqpoint{5.258349in}{2.275952in}}%
\pgfpathlineto{\pgfqpoint{5.251221in}{2.269055in}}%
\pgfpathlineto{\pgfqpoint{5.244087in}{2.262086in}}%
\pgfpathlineto{\pgfqpoint{5.236945in}{2.255044in}}%
\pgfpathlineto{\pgfqpoint{5.229797in}{2.247929in}}%
\pgfpathlineto{\pgfqpoint{5.216337in}{2.249027in}}%
\pgfpathlineto{\pgfqpoint{5.202884in}{2.250149in}}%
\pgfpathlineto{\pgfqpoint{5.189440in}{2.251295in}}%
\pgfpathlineto{\pgfqpoint{5.176003in}{2.252465in}}%
\pgfpathlineto{\pgfqpoint{5.183164in}{2.259588in}}%
\pgfpathlineto{\pgfqpoint{5.190319in}{2.266641in}}%
\pgfpathlineto{\pgfqpoint{5.197466in}{2.273623in}}%
\pgfpathlineto{\pgfqpoint{5.204607in}{2.280538in}}%
\pgfpathclose%
\pgfusepath{fill}%
\end{pgfscope}%
\begin{pgfscope}%
\pgfpathrectangle{\pgfqpoint{1.254980in}{0.150000in}}{\pgfqpoint{5.490039in}{5.490039in}}%
\pgfusepath{clip}%
\pgfsetbuttcap%
\pgfsetroundjoin%
\definecolor{currentfill}{rgb}{0.267004,0.004874,0.329415}%
\pgfsetfillcolor{currentfill}%
\pgfsetfillopacity{0.700000}%
\pgfsetlinewidth{0.000000pt}%
\definecolor{currentstroke}{rgb}{0.000000,0.000000,0.000000}%
\pgfsetstrokecolor{currentstroke}%
\pgfsetdash{}{0pt}%
\pgfpathmoveto{\pgfqpoint{3.949908in}{2.059677in}}%
\pgfpathlineto{\pgfqpoint{3.962993in}{2.056069in}}%
\pgfpathlineto{\pgfqpoint{3.976085in}{2.052489in}}%
\pgfpathlineto{\pgfqpoint{3.989182in}{2.048936in}}%
\pgfpathlineto{\pgfqpoint{4.002285in}{2.045410in}}%
\pgfpathlineto{\pgfqpoint{3.994679in}{2.037937in}}%
\pgfpathlineto{\pgfqpoint{3.987067in}{2.030504in}}%
\pgfpathlineto{\pgfqpoint{3.979450in}{2.023114in}}%
\pgfpathlineto{\pgfqpoint{3.971826in}{2.015770in}}%
\pgfpathlineto{\pgfqpoint{3.958710in}{2.019438in}}%
\pgfpathlineto{\pgfqpoint{3.945600in}{2.023134in}}%
\pgfpathlineto{\pgfqpoint{3.932495in}{2.026857in}}%
\pgfpathlineto{\pgfqpoint{3.919397in}{2.030607in}}%
\pgfpathlineto{\pgfqpoint{3.927033in}{2.037803in}}%
\pgfpathlineto{\pgfqpoint{3.934664in}{2.045049in}}%
\pgfpathlineto{\pgfqpoint{3.942289in}{2.052342in}}%
\pgfpathlineto{\pgfqpoint{3.949908in}{2.059677in}}%
\pgfpathclose%
\pgfusepath{fill}%
\end{pgfscope}%
\begin{pgfscope}%
\pgfpathrectangle{\pgfqpoint{1.254980in}{0.150000in}}{\pgfqpoint{5.490039in}{5.490039in}}%
\pgfusepath{clip}%
\pgfsetbuttcap%
\pgfsetroundjoin%
\definecolor{currentfill}{rgb}{0.282327,0.094955,0.417331}%
\pgfsetfillcolor{currentfill}%
\pgfsetfillopacity{0.700000}%
\pgfsetlinewidth{0.000000pt}%
\definecolor{currentstroke}{rgb}{0.000000,0.000000,0.000000}%
\pgfsetstrokecolor{currentstroke}%
\pgfsetdash{}{0pt}%
\pgfpathmoveto{\pgfqpoint{4.904161in}{2.215114in}}%
\pgfpathlineto{\pgfqpoint{4.917497in}{2.213543in}}%
\pgfpathlineto{\pgfqpoint{4.930841in}{2.211997in}}%
\pgfpathlineto{\pgfqpoint{4.944193in}{2.210474in}}%
\pgfpathlineto{\pgfqpoint{4.957552in}{2.208976in}}%
\pgfpathlineto{\pgfqpoint{4.950296in}{2.201265in}}%
\pgfpathlineto{\pgfqpoint{4.943032in}{2.193490in}}%
\pgfpathlineto{\pgfqpoint{4.935763in}{2.185651in}}%
\pgfpathlineto{\pgfqpoint{4.928487in}{2.177749in}}%
\pgfpathlineto{\pgfqpoint{4.915117in}{2.179275in}}%
\pgfpathlineto{\pgfqpoint{4.901754in}{2.180824in}}%
\pgfpathlineto{\pgfqpoint{4.888398in}{2.182398in}}%
\pgfpathlineto{\pgfqpoint{4.875050in}{2.183996in}}%
\pgfpathlineto{\pgfqpoint{4.882337in}{2.191866in}}%
\pgfpathlineto{\pgfqpoint{4.889618in}{2.199676in}}%
\pgfpathlineto{\pgfqpoint{4.896892in}{2.207425in}}%
\pgfpathlineto{\pgfqpoint{4.904161in}{2.215114in}}%
\pgfpathclose%
\pgfusepath{fill}%
\end{pgfscope}%
\begin{pgfscope}%
\pgfpathrectangle{\pgfqpoint{1.254980in}{0.150000in}}{\pgfqpoint{5.490039in}{5.490039in}}%
\pgfusepath{clip}%
\pgfsetbuttcap%
\pgfsetroundjoin%
\definecolor{currentfill}{rgb}{0.282910,0.105393,0.426902}%
\pgfsetfillcolor{currentfill}%
\pgfsetfillopacity{0.700000}%
\pgfsetlinewidth{0.000000pt}%
\definecolor{currentstroke}{rgb}{0.000000,0.000000,0.000000}%
\pgfsetstrokecolor{currentstroke}%
\pgfsetdash{}{0pt}%
\pgfpathmoveto{\pgfqpoint{2.951366in}{2.230933in}}%
\pgfpathlineto{\pgfqpoint{2.964294in}{2.224136in}}%
\pgfpathlineto{\pgfqpoint{2.977224in}{2.217378in}}%
\pgfpathlineto{\pgfqpoint{2.990159in}{2.210658in}}%
\pgfpathlineto{\pgfqpoint{3.003096in}{2.203976in}}%
\pgfpathlineto{\pgfqpoint{2.994992in}{2.202593in}}%
\pgfpathlineto{\pgfqpoint{2.986876in}{2.201426in}}%
\pgfpathlineto{\pgfqpoint{2.978746in}{2.200483in}}%
\pgfpathlineto{\pgfqpoint{2.970604in}{2.199768in}}%
\pgfpathlineto{\pgfqpoint{2.957641in}{2.206684in}}%
\pgfpathlineto{\pgfqpoint{2.944681in}{2.213638in}}%
\pgfpathlineto{\pgfqpoint{2.931725in}{2.220631in}}%
\pgfpathlineto{\pgfqpoint{2.918772in}{2.227662in}}%
\pgfpathlineto{\pgfqpoint{2.926940in}{2.228138in}}%
\pgfpathlineto{\pgfqpoint{2.935096in}{2.228845in}}%
\pgfpathlineto{\pgfqpoint{2.943237in}{2.229779in}}%
\pgfpathlineto{\pgfqpoint{2.951366in}{2.230933in}}%
\pgfpathclose%
\pgfusepath{fill}%
\end{pgfscope}%
\begin{pgfscope}%
\pgfpathrectangle{\pgfqpoint{1.254980in}{0.150000in}}{\pgfqpoint{5.490039in}{5.490039in}}%
\pgfusepath{clip}%
\pgfsetbuttcap%
\pgfsetroundjoin%
\definecolor{currentfill}{rgb}{0.277941,0.056324,0.381191}%
\pgfsetfillcolor{currentfill}%
\pgfsetfillopacity{0.700000}%
\pgfsetlinewidth{0.000000pt}%
\definecolor{currentstroke}{rgb}{0.000000,0.000000,0.000000}%
\pgfsetstrokecolor{currentstroke}%
\pgfsetdash{}{0pt}%
\pgfpathmoveto{\pgfqpoint{4.603628in}{2.149325in}}%
\pgfpathlineto{\pgfqpoint{4.616880in}{2.147240in}}%
\pgfpathlineto{\pgfqpoint{4.630140in}{2.145180in}}%
\pgfpathlineto{\pgfqpoint{4.643406in}{2.143145in}}%
\pgfpathlineto{\pgfqpoint{4.656680in}{2.141134in}}%
\pgfpathlineto{\pgfqpoint{4.649307in}{2.132933in}}%
\pgfpathlineto{\pgfqpoint{4.641929in}{2.124689in}}%
\pgfpathlineto{\pgfqpoint{4.634544in}{2.116402in}}%
\pgfpathlineto{\pgfqpoint{4.627154in}{2.108076in}}%
\pgfpathlineto{\pgfqpoint{4.613870in}{2.110152in}}%
\pgfpathlineto{\pgfqpoint{4.600593in}{2.112253in}}%
\pgfpathlineto{\pgfqpoint{4.587322in}{2.114380in}}%
\pgfpathlineto{\pgfqpoint{4.574059in}{2.116530in}}%
\pgfpathlineto{\pgfqpoint{4.581460in}{2.124786in}}%
\pgfpathlineto{\pgfqpoint{4.588855in}{2.133005in}}%
\pgfpathlineto{\pgfqpoint{4.596244in}{2.141185in}}%
\pgfpathlineto{\pgfqpoint{4.603628in}{2.149325in}}%
\pgfpathclose%
\pgfusepath{fill}%
\end{pgfscope}%
\begin{pgfscope}%
\pgfpathrectangle{\pgfqpoint{1.254980in}{0.150000in}}{\pgfqpoint{5.490039in}{5.490039in}}%
\pgfusepath{clip}%
\pgfsetbuttcap%
\pgfsetroundjoin%
\definecolor{currentfill}{rgb}{0.279566,0.067836,0.391917}%
\pgfsetfillcolor{currentfill}%
\pgfsetfillopacity{0.700000}%
\pgfsetlinewidth{0.000000pt}%
\definecolor{currentstroke}{rgb}{0.000000,0.000000,0.000000}%
\pgfsetstrokecolor{currentstroke}%
\pgfsetdash{}{0pt}%
\pgfpathmoveto{\pgfqpoint{3.138835in}{2.161212in}}%
\pgfpathlineto{\pgfqpoint{3.151782in}{2.155082in}}%
\pgfpathlineto{\pgfqpoint{3.164734in}{2.148986in}}%
\pgfpathlineto{\pgfqpoint{3.177690in}{2.142926in}}%
\pgfpathlineto{\pgfqpoint{3.190649in}{2.136900in}}%
\pgfpathlineto{\pgfqpoint{3.182661in}{2.134061in}}%
\pgfpathlineto{\pgfqpoint{3.174662in}{2.131405in}}%
\pgfpathlineto{\pgfqpoint{3.166652in}{2.128939in}}%
\pgfpathlineto{\pgfqpoint{3.158631in}{2.126668in}}%
\pgfpathlineto{\pgfqpoint{3.145649in}{2.132914in}}%
\pgfpathlineto{\pgfqpoint{3.132671in}{2.139195in}}%
\pgfpathlineto{\pgfqpoint{3.119697in}{2.145510in}}%
\pgfpathlineto{\pgfqpoint{3.106726in}{2.151862in}}%
\pgfpathlineto{\pgfqpoint{3.114770in}{2.153907in}}%
\pgfpathlineto{\pgfqpoint{3.122803in}{2.156152in}}%
\pgfpathlineto{\pgfqpoint{3.130824in}{2.158588in}}%
\pgfpathlineto{\pgfqpoint{3.138835in}{2.161212in}}%
\pgfpathclose%
\pgfusepath{fill}%
\end{pgfscope}%
\begin{pgfscope}%
\pgfpathrectangle{\pgfqpoint{1.254980in}{0.150000in}}{\pgfqpoint{5.490039in}{5.490039in}}%
\pgfusepath{clip}%
\pgfsetbuttcap%
\pgfsetroundjoin%
\definecolor{currentfill}{rgb}{0.281412,0.155834,0.469201}%
\pgfsetfillcolor{currentfill}%
\pgfsetfillopacity{0.700000}%
\pgfsetlinewidth{0.000000pt}%
\definecolor{currentstroke}{rgb}{0.000000,0.000000,0.000000}%
\pgfsetstrokecolor{currentstroke}%
\pgfsetdash{}{0pt}%
\pgfpathmoveto{\pgfqpoint{5.422765in}{2.320140in}}%
\pgfpathlineto{\pgfqpoint{5.436257in}{2.319179in}}%
\pgfpathlineto{\pgfqpoint{5.449758in}{2.318242in}}%
\pgfpathlineto{\pgfqpoint{5.463267in}{2.317329in}}%
\pgfpathlineto{\pgfqpoint{5.476784in}{2.316440in}}%
\pgfpathlineto{\pgfqpoint{5.469756in}{2.310205in}}%
\pgfpathlineto{\pgfqpoint{5.462720in}{2.303899in}}%
\pgfpathlineto{\pgfqpoint{5.455677in}{2.297520in}}%
\pgfpathlineto{\pgfqpoint{5.448627in}{2.291067in}}%
\pgfpathlineto{\pgfqpoint{5.435096in}{2.291918in}}%
\pgfpathlineto{\pgfqpoint{5.421572in}{2.292792in}}%
\pgfpathlineto{\pgfqpoint{5.408057in}{2.293690in}}%
\pgfpathlineto{\pgfqpoint{5.394551in}{2.294612in}}%
\pgfpathlineto{\pgfqpoint{5.401615in}{2.301099in}}%
\pgfpathlineto{\pgfqpoint{5.408672in}{2.307515in}}%
\pgfpathlineto{\pgfqpoint{5.415722in}{2.313861in}}%
\pgfpathlineto{\pgfqpoint{5.422765in}{2.320140in}}%
\pgfpathclose%
\pgfusepath{fill}%
\end{pgfscope}%
\begin{pgfscope}%
\pgfpathrectangle{\pgfqpoint{1.254980in}{0.150000in}}{\pgfqpoint{5.490039in}{5.490039in}}%
\pgfusepath{clip}%
\pgfsetbuttcap%
\pgfsetroundjoin%
\definecolor{currentfill}{rgb}{0.269308,0.218818,0.509577}%
\pgfsetfillcolor{currentfill}%
\pgfsetfillopacity{0.700000}%
\pgfsetlinewidth{0.000000pt}%
\definecolor{currentstroke}{rgb}{0.000000,0.000000,0.000000}%
\pgfsetstrokecolor{currentstroke}%
\pgfsetdash{}{0pt}%
\pgfpathmoveto{\pgfqpoint{2.523596in}{2.449366in}}%
\pgfpathlineto{\pgfqpoint{2.536507in}{2.440838in}}%
\pgfpathlineto{\pgfqpoint{2.549420in}{2.432361in}}%
\pgfpathlineto{\pgfqpoint{2.562335in}{2.423934in}}%
\pgfpathlineto{\pgfqpoint{2.575252in}{2.415557in}}%
\pgfpathlineto{\pgfqpoint{2.566832in}{2.417886in}}%
\pgfpathlineto{\pgfqpoint{2.558393in}{2.420508in}}%
\pgfpathlineto{\pgfqpoint{2.549936in}{2.423430in}}%
\pgfpathlineto{\pgfqpoint{2.541460in}{2.426657in}}%
\pgfpathlineto{\pgfqpoint{2.528510in}{2.435299in}}%
\pgfpathlineto{\pgfqpoint{2.515562in}{2.443991in}}%
\pgfpathlineto{\pgfqpoint{2.502616in}{2.452733in}}%
\pgfpathlineto{\pgfqpoint{2.489672in}{2.461526in}}%
\pgfpathlineto{\pgfqpoint{2.498182in}{2.458028in}}%
\pgfpathlineto{\pgfqpoint{2.506672in}{2.454840in}}%
\pgfpathlineto{\pgfqpoint{2.515143in}{2.451955in}}%
\pgfpathlineto{\pgfqpoint{2.523596in}{2.449366in}}%
\pgfpathclose%
\pgfusepath{fill}%
\end{pgfscope}%
\begin{pgfscope}%
\pgfpathrectangle{\pgfqpoint{1.254980in}{0.150000in}}{\pgfqpoint{5.490039in}{5.490039in}}%
\pgfusepath{clip}%
\pgfsetbuttcap%
\pgfsetroundjoin%
\definecolor{currentfill}{rgb}{0.281887,0.150881,0.465405}%
\pgfsetfillcolor{currentfill}%
\pgfsetfillopacity{0.700000}%
\pgfsetlinewidth{0.000000pt}%
\definecolor{currentstroke}{rgb}{0.000000,0.000000,0.000000}%
\pgfsetstrokecolor{currentstroke}%
\pgfsetdash{}{0pt}%
\pgfpathmoveto{\pgfqpoint{2.763574in}{2.315199in}}%
\pgfpathlineto{\pgfqpoint{2.776491in}{2.307673in}}%
\pgfpathlineto{\pgfqpoint{2.789411in}{2.300191in}}%
\pgfpathlineto{\pgfqpoint{2.802334in}{2.292752in}}%
\pgfpathlineto{\pgfqpoint{2.815260in}{2.285355in}}%
\pgfpathlineto{\pgfqpoint{2.807023in}{2.285605in}}%
\pgfpathlineto{\pgfqpoint{2.798771in}{2.286107in}}%
\pgfpathlineto{\pgfqpoint{2.790504in}{2.286867in}}%
\pgfpathlineto{\pgfqpoint{2.782222in}{2.287892in}}%
\pgfpathlineto{\pgfqpoint{2.769268in}{2.295537in}}%
\pgfpathlineto{\pgfqpoint{2.756316in}{2.303226in}}%
\pgfpathlineto{\pgfqpoint{2.743368in}{2.310957in}}%
\pgfpathlineto{\pgfqpoint{2.730422in}{2.318731in}}%
\pgfpathlineto{\pgfqpoint{2.738733in}{2.317452in}}%
\pgfpathlineto{\pgfqpoint{2.747029in}{2.316441in}}%
\pgfpathlineto{\pgfqpoint{2.755309in}{2.315692in}}%
\pgfpathlineto{\pgfqpoint{2.763574in}{2.315199in}}%
\pgfpathclose%
\pgfusepath{fill}%
\end{pgfscope}%
\begin{pgfscope}%
\pgfpathrectangle{\pgfqpoint{1.254980in}{0.150000in}}{\pgfqpoint{5.490039in}{5.490039in}}%
\pgfusepath{clip}%
\pgfsetbuttcap%
\pgfsetroundjoin%
\definecolor{currentfill}{rgb}{0.268510,0.009605,0.335427}%
\pgfsetfillcolor{currentfill}%
\pgfsetfillopacity{0.700000}%
\pgfsetlinewidth{0.000000pt}%
\definecolor{currentstroke}{rgb}{0.000000,0.000000,0.000000}%
\pgfsetstrokecolor{currentstroke}%
\pgfsetdash{}{0pt}%
\pgfpathmoveto{\pgfqpoint{3.596713in}{2.062627in}}%
\pgfpathlineto{\pgfqpoint{3.609731in}{2.057984in}}%
\pgfpathlineto{\pgfqpoint{3.622755in}{2.053371in}}%
\pgfpathlineto{\pgfqpoint{3.635783in}{2.048787in}}%
\pgfpathlineto{\pgfqpoint{3.648817in}{2.044233in}}%
\pgfpathlineto{\pgfqpoint{3.641064in}{2.038356in}}%
\pgfpathlineto{\pgfqpoint{3.633304in}{2.032581in}}%
\pgfpathlineto{\pgfqpoint{3.625536in}{2.026911in}}%
\pgfpathlineto{\pgfqpoint{3.617761in}{2.021352in}}%
\pgfpathlineto{\pgfqpoint{3.604711in}{2.026088in}}%
\pgfpathlineto{\pgfqpoint{3.591666in}{2.030852in}}%
\pgfpathlineto{\pgfqpoint{3.578627in}{2.035646in}}%
\pgfpathlineto{\pgfqpoint{3.565592in}{2.040470in}}%
\pgfpathlineto{\pgfqpoint{3.573384in}{2.045843in}}%
\pgfpathlineto{\pgfqpoint{3.581168in}{2.051330in}}%
\pgfpathlineto{\pgfqpoint{3.588944in}{2.056926in}}%
\pgfpathlineto{\pgfqpoint{3.596713in}{2.062627in}}%
\pgfpathclose%
\pgfusepath{fill}%
\end{pgfscope}%
\begin{pgfscope}%
\pgfpathrectangle{\pgfqpoint{1.254980in}{0.150000in}}{\pgfqpoint{5.490039in}{5.490039in}}%
\pgfusepath{clip}%
\pgfsetbuttcap%
\pgfsetroundjoin%
\definecolor{currentfill}{rgb}{0.283229,0.120777,0.440584}%
\pgfsetfillcolor{currentfill}%
\pgfsetfillopacity{0.700000}%
\pgfsetlinewidth{0.000000pt}%
\definecolor{currentstroke}{rgb}{0.000000,0.000000,0.000000}%
\pgfsetstrokecolor{currentstroke}%
\pgfsetdash{}{0pt}%
\pgfpathmoveto{\pgfqpoint{5.122336in}{2.257385in}}%
\pgfpathlineto{\pgfqpoint{5.135741in}{2.256119in}}%
\pgfpathlineto{\pgfqpoint{5.149154in}{2.254877in}}%
\pgfpathlineto{\pgfqpoint{5.162574in}{2.253659in}}%
\pgfpathlineto{\pgfqpoint{5.176003in}{2.252465in}}%
\pgfpathlineto{\pgfqpoint{5.168835in}{2.245272in}}%
\pgfpathlineto{\pgfqpoint{5.161660in}{2.238006in}}%
\pgfpathlineto{\pgfqpoint{5.154478in}{2.230669in}}%
\pgfpathlineto{\pgfqpoint{5.147290in}{2.223258in}}%
\pgfpathlineto{\pgfqpoint{5.133849in}{2.224453in}}%
\pgfpathlineto{\pgfqpoint{5.120416in}{2.225672in}}%
\pgfpathlineto{\pgfqpoint{5.106991in}{2.226915in}}%
\pgfpathlineto{\pgfqpoint{5.093573in}{2.228181in}}%
\pgfpathlineto{\pgfqpoint{5.100774in}{2.235586in}}%
\pgfpathlineto{\pgfqpoint{5.107968in}{2.242921in}}%
\pgfpathlineto{\pgfqpoint{5.115155in}{2.250187in}}%
\pgfpathlineto{\pgfqpoint{5.122336in}{2.257385in}}%
\pgfpathclose%
\pgfusepath{fill}%
\end{pgfscope}%
\begin{pgfscope}%
\pgfpathrectangle{\pgfqpoint{1.254980in}{0.150000in}}{\pgfqpoint{5.490039in}{5.490039in}}%
\pgfusepath{clip}%
\pgfsetbuttcap%
\pgfsetroundjoin%
\definecolor{currentfill}{rgb}{0.271305,0.019942,0.347269}%
\pgfsetfillcolor{currentfill}%
\pgfsetfillopacity{0.700000}%
\pgfsetlinewidth{0.000000pt}%
\definecolor{currentstroke}{rgb}{0.000000,0.000000,0.000000}%
\pgfsetstrokecolor{currentstroke}%
\pgfsetdash{}{0pt}%
\pgfpathmoveto{\pgfqpoint{3.461492in}{2.080148in}}%
\pgfpathlineto{\pgfqpoint{3.474487in}{2.075081in}}%
\pgfpathlineto{\pgfqpoint{3.487488in}{2.070045in}}%
\pgfpathlineto{\pgfqpoint{3.500493in}{2.065040in}}%
\pgfpathlineto{\pgfqpoint{3.513503in}{2.060065in}}%
\pgfpathlineto{\pgfqpoint{3.505686in}{2.055000in}}%
\pgfpathlineto{\pgfqpoint{3.497861in}{2.050060in}}%
\pgfpathlineto{\pgfqpoint{3.490029in}{2.045252in}}%
\pgfpathlineto{\pgfqpoint{3.482187in}{2.040580in}}%
\pgfpathlineto{\pgfqpoint{3.469159in}{2.045748in}}%
\pgfpathlineto{\pgfqpoint{3.456136in}{2.050947in}}%
\pgfpathlineto{\pgfqpoint{3.443118in}{2.056177in}}%
\pgfpathlineto{\pgfqpoint{3.430104in}{2.061438in}}%
\pgfpathlineto{\pgfqpoint{3.437964in}{2.065912in}}%
\pgfpathlineto{\pgfqpoint{3.445815in}{2.070525in}}%
\pgfpathlineto{\pgfqpoint{3.453658in}{2.075272in}}%
\pgfpathlineto{\pgfqpoint{3.461492in}{2.080148in}}%
\pgfpathclose%
\pgfusepath{fill}%
\end{pgfscope}%
\begin{pgfscope}%
\pgfpathrectangle{\pgfqpoint{1.254980in}{0.150000in}}{\pgfqpoint{5.490039in}{5.490039in}}%
\pgfusepath{clip}%
\pgfsetbuttcap%
\pgfsetroundjoin%
\definecolor{currentfill}{rgb}{0.272594,0.025563,0.353093}%
\pgfsetfillcolor{currentfill}%
\pgfsetfillopacity{0.700000}%
\pgfsetlinewidth{0.000000pt}%
\definecolor{currentstroke}{rgb}{0.000000,0.000000,0.000000}%
\pgfsetstrokecolor{currentstroke}%
\pgfsetdash{}{0pt}%
\pgfpathmoveto{\pgfqpoint{4.303045in}{2.089900in}}%
\pgfpathlineto{\pgfqpoint{4.316220in}{2.087174in}}%
\pgfpathlineto{\pgfqpoint{4.329402in}{2.084473in}}%
\pgfpathlineto{\pgfqpoint{4.342590in}{2.081798in}}%
\pgfpathlineto{\pgfqpoint{4.355785in}{2.079149in}}%
\pgfpathlineto{\pgfqpoint{4.348303in}{2.070931in}}%
\pgfpathlineto{\pgfqpoint{4.340816in}{2.062704in}}%
\pgfpathlineto{\pgfqpoint{4.333323in}{2.054469in}}%
\pgfpathlineto{\pgfqpoint{4.325825in}{2.046230in}}%
\pgfpathlineto{\pgfqpoint{4.312619in}{2.048984in}}%
\pgfpathlineto{\pgfqpoint{4.299420in}{2.051763in}}%
\pgfpathlineto{\pgfqpoint{4.286227in}{2.054568in}}%
\pgfpathlineto{\pgfqpoint{4.273041in}{2.057399in}}%
\pgfpathlineto{\pgfqpoint{4.280550in}{2.065529in}}%
\pgfpathlineto{\pgfqpoint{4.288053in}{2.073658in}}%
\pgfpathlineto{\pgfqpoint{4.295552in}{2.081782in}}%
\pgfpathlineto{\pgfqpoint{4.303045in}{2.089900in}}%
\pgfpathclose%
\pgfusepath{fill}%
\end{pgfscope}%
\begin{pgfscope}%
\pgfpathrectangle{\pgfqpoint{1.254980in}{0.150000in}}{\pgfqpoint{5.490039in}{5.490039in}}%
\pgfusepath{clip}%
\pgfsetbuttcap%
\pgfsetroundjoin%
\definecolor{currentfill}{rgb}{0.268510,0.009605,0.335427}%
\pgfsetfillcolor{currentfill}%
\pgfsetfillopacity{0.700000}%
\pgfsetlinewidth{0.000000pt}%
\definecolor{currentstroke}{rgb}{0.000000,0.000000,0.000000}%
\pgfsetstrokecolor{currentstroke}%
\pgfsetdash{}{0pt}%
\pgfpathmoveto{\pgfqpoint{4.085078in}{2.062318in}}%
\pgfpathlineto{\pgfqpoint{4.098199in}{2.059056in}}%
\pgfpathlineto{\pgfqpoint{4.111327in}{2.055821in}}%
\pgfpathlineto{\pgfqpoint{4.124461in}{2.052612in}}%
\pgfpathlineto{\pgfqpoint{4.137601in}{2.049430in}}%
\pgfpathlineto{\pgfqpoint{4.130042in}{2.041580in}}%
\pgfpathlineto{\pgfqpoint{4.122477in}{2.033751in}}%
\pgfpathlineto{\pgfqpoint{4.114907in}{2.025945in}}%
\pgfpathlineto{\pgfqpoint{4.107331in}{2.018167in}}%
\pgfpathlineto{\pgfqpoint{4.094179in}{2.021479in}}%
\pgfpathlineto{\pgfqpoint{4.081033in}{2.024818in}}%
\pgfpathlineto{\pgfqpoint{4.067893in}{2.028183in}}%
\pgfpathlineto{\pgfqpoint{4.054759in}{2.031575in}}%
\pgfpathlineto{\pgfqpoint{4.062347in}{2.039219in}}%
\pgfpathlineto{\pgfqpoint{4.069930in}{2.046893in}}%
\pgfpathlineto{\pgfqpoint{4.077506in}{2.054594in}}%
\pgfpathlineto{\pgfqpoint{4.085078in}{2.062318in}}%
\pgfpathclose%
\pgfusepath{fill}%
\end{pgfscope}%
\begin{pgfscope}%
\pgfpathrectangle{\pgfqpoint{1.254980in}{0.150000in}}{\pgfqpoint{5.490039in}{5.490039in}}%
\pgfusepath{clip}%
\pgfsetbuttcap%
\pgfsetroundjoin%
\definecolor{currentfill}{rgb}{0.278826,0.175490,0.483397}%
\pgfsetfillcolor{currentfill}%
\pgfsetfillopacity{0.700000}%
\pgfsetlinewidth{0.000000pt}%
\definecolor{currentstroke}{rgb}{0.000000,0.000000,0.000000}%
\pgfsetstrokecolor{currentstroke}%
\pgfsetdash{}{0pt}%
\pgfpathmoveto{\pgfqpoint{5.640930in}{2.356775in}}%
\pgfpathlineto{\pgfqpoint{5.654490in}{2.355981in}}%
\pgfpathlineto{\pgfqpoint{5.668059in}{2.355210in}}%
\pgfpathlineto{\pgfqpoint{5.681636in}{2.354463in}}%
\pgfpathlineto{\pgfqpoint{5.695222in}{2.353739in}}%
\pgfpathlineto{\pgfqpoint{5.688301in}{2.348215in}}%
\pgfpathlineto{\pgfqpoint{5.681373in}{2.342626in}}%
\pgfpathlineto{\pgfqpoint{5.674437in}{2.336971in}}%
\pgfpathlineto{\pgfqpoint{5.667493in}{2.331246in}}%
\pgfpathlineto{\pgfqpoint{5.653891in}{2.331904in}}%
\pgfpathlineto{\pgfqpoint{5.640297in}{2.332586in}}%
\pgfpathlineto{\pgfqpoint{5.626712in}{2.333291in}}%
\pgfpathlineto{\pgfqpoint{5.613135in}{2.334020in}}%
\pgfpathlineto{\pgfqpoint{5.620095in}{2.339806in}}%
\pgfpathlineto{\pgfqpoint{5.627047in}{2.345526in}}%
\pgfpathlineto{\pgfqpoint{5.633992in}{2.351181in}}%
\pgfpathlineto{\pgfqpoint{5.640930in}{2.356775in}}%
\pgfpathclose%
\pgfusepath{fill}%
\end{pgfscope}%
\begin{pgfscope}%
\pgfpathrectangle{\pgfqpoint{1.254980in}{0.150000in}}{\pgfqpoint{5.490039in}{5.490039in}}%
\pgfusepath{clip}%
\pgfsetbuttcap%
\pgfsetroundjoin%
\definecolor{currentfill}{rgb}{0.267004,0.004874,0.329415}%
\pgfsetfillcolor{currentfill}%
\pgfsetfillopacity{0.700000}%
\pgfsetlinewidth{0.000000pt}%
\definecolor{currentstroke}{rgb}{0.000000,0.000000,0.000000}%
\pgfsetstrokecolor{currentstroke}%
\pgfsetdash{}{0pt}%
\pgfpathmoveto{\pgfqpoint{3.731885in}{2.051419in}}%
\pgfpathlineto{\pgfqpoint{3.744930in}{2.047178in}}%
\pgfpathlineto{\pgfqpoint{3.757980in}{2.042965in}}%
\pgfpathlineto{\pgfqpoint{3.771036in}{2.038781in}}%
\pgfpathlineto{\pgfqpoint{3.784098in}{2.034626in}}%
\pgfpathlineto{\pgfqpoint{3.776402in}{2.028058in}}%
\pgfpathlineto{\pgfqpoint{3.768700in}{2.021569in}}%
\pgfpathlineto{\pgfqpoint{3.760992in}{2.015162in}}%
\pgfpathlineto{\pgfqpoint{3.753277in}{2.008842in}}%
\pgfpathlineto{\pgfqpoint{3.740201in}{2.013166in}}%
\pgfpathlineto{\pgfqpoint{3.727130in}{2.017518in}}%
\pgfpathlineto{\pgfqpoint{3.714064in}{2.021899in}}%
\pgfpathlineto{\pgfqpoint{3.701004in}{2.026308in}}%
\pgfpathlineto{\pgfqpoint{3.708735in}{2.032455in}}%
\pgfpathlineto{\pgfqpoint{3.716458in}{2.038692in}}%
\pgfpathlineto{\pgfqpoint{3.724175in}{2.045014in}}%
\pgfpathlineto{\pgfqpoint{3.731885in}{2.051419in}}%
\pgfpathclose%
\pgfusepath{fill}%
\end{pgfscope}%
\begin{pgfscope}%
\pgfpathrectangle{\pgfqpoint{1.254980in}{0.150000in}}{\pgfqpoint{5.490039in}{5.490039in}}%
\pgfusepath{clip}%
\pgfsetbuttcap%
\pgfsetroundjoin%
\definecolor{currentfill}{rgb}{0.281446,0.084320,0.407414}%
\pgfsetfillcolor{currentfill}%
\pgfsetfillopacity{0.700000}%
\pgfsetlinewidth{0.000000pt}%
\definecolor{currentstroke}{rgb}{0.000000,0.000000,0.000000}%
\pgfsetstrokecolor{currentstroke}%
\pgfsetdash{}{0pt}%
\pgfpathmoveto{\pgfqpoint{4.821734in}{2.190632in}}%
\pgfpathlineto{\pgfqpoint{4.835052in}{2.188936in}}%
\pgfpathlineto{\pgfqpoint{4.848377in}{2.187265in}}%
\pgfpathlineto{\pgfqpoint{4.861710in}{2.185618in}}%
\pgfpathlineto{\pgfqpoint{4.875050in}{2.183996in}}%
\pgfpathlineto{\pgfqpoint{4.867757in}{2.176066in}}%
\pgfpathlineto{\pgfqpoint{4.860458in}{2.168077in}}%
\pgfpathlineto{\pgfqpoint{4.853153in}{2.160029in}}%
\pgfpathlineto{\pgfqpoint{4.845841in}{2.151923in}}%
\pgfpathlineto{\pgfqpoint{4.832490in}{2.153585in}}%
\pgfpathlineto{\pgfqpoint{4.819146in}{2.155272in}}%
\pgfpathlineto{\pgfqpoint{4.805810in}{2.156983in}}%
\pgfpathlineto{\pgfqpoint{4.792481in}{2.158719in}}%
\pgfpathlineto{\pgfqpoint{4.799803in}{2.166780in}}%
\pgfpathlineto{\pgfqpoint{4.807119in}{2.174786in}}%
\pgfpathlineto{\pgfqpoint{4.814430in}{2.182737in}}%
\pgfpathlineto{\pgfqpoint{4.821734in}{2.190632in}}%
\pgfpathclose%
\pgfusepath{fill}%
\end{pgfscope}%
\begin{pgfscope}%
\pgfpathrectangle{\pgfqpoint{1.254980in}{0.150000in}}{\pgfqpoint{5.490039in}{5.490039in}}%
\pgfusepath{clip}%
\pgfsetbuttcap%
\pgfsetroundjoin%
\definecolor{currentfill}{rgb}{0.277018,0.050344,0.375715}%
\pgfsetfillcolor{currentfill}%
\pgfsetfillopacity{0.700000}%
\pgfsetlinewidth{0.000000pt}%
\definecolor{currentstroke}{rgb}{0.000000,0.000000,0.000000}%
\pgfsetstrokecolor{currentstroke}%
\pgfsetdash{}{0pt}%
\pgfpathmoveto{\pgfqpoint{4.521077in}{2.125384in}}%
\pgfpathlineto{\pgfqpoint{4.534312in}{2.123133in}}%
\pgfpathlineto{\pgfqpoint{4.547554in}{2.120907in}}%
\pgfpathlineto{\pgfqpoint{4.560803in}{2.118706in}}%
\pgfpathlineto{\pgfqpoint{4.574059in}{2.116530in}}%
\pgfpathlineto{\pgfqpoint{4.566653in}{2.108239in}}%
\pgfpathlineto{\pgfqpoint{4.559241in}{2.099913in}}%
\pgfpathlineto{\pgfqpoint{4.551824in}{2.091554in}}%
\pgfpathlineto{\pgfqpoint{4.544402in}{2.083165in}}%
\pgfpathlineto{\pgfqpoint{4.531135in}{2.085420in}}%
\pgfpathlineto{\pgfqpoint{4.517875in}{2.087699in}}%
\pgfpathlineto{\pgfqpoint{4.504623in}{2.090004in}}%
\pgfpathlineto{\pgfqpoint{4.491377in}{2.092334in}}%
\pgfpathlineto{\pgfqpoint{4.498810in}{2.100640in}}%
\pgfpathlineto{\pgfqpoint{4.506238in}{2.108918in}}%
\pgfpathlineto{\pgfqpoint{4.513660in}{2.117167in}}%
\pgfpathlineto{\pgfqpoint{4.521077in}{2.125384in}}%
\pgfpathclose%
\pgfusepath{fill}%
\end{pgfscope}%
\begin{pgfscope}%
\pgfpathrectangle{\pgfqpoint{1.254980in}{0.150000in}}{\pgfqpoint{5.490039in}{5.490039in}}%
\pgfusepath{clip}%
\pgfsetbuttcap%
\pgfsetroundjoin%
\definecolor{currentfill}{rgb}{0.275191,0.194905,0.496005}%
\pgfsetfillcolor{currentfill}%
\pgfsetfillopacity{0.700000}%
\pgfsetlinewidth{0.000000pt}%
\definecolor{currentstroke}{rgb}{0.000000,0.000000,0.000000}%
\pgfsetstrokecolor{currentstroke}%
\pgfsetdash{}{0pt}%
\pgfpathmoveto{\pgfqpoint{5.859034in}{2.389600in}}%
\pgfpathlineto{\pgfqpoint{5.872661in}{2.388917in}}%
\pgfpathlineto{\pgfqpoint{5.886296in}{2.388257in}}%
\pgfpathlineto{\pgfqpoint{5.899940in}{2.387621in}}%
\pgfpathlineto{\pgfqpoint{5.913593in}{2.387008in}}%
\pgfpathlineto{\pgfqpoint{5.906786in}{2.382197in}}%
\pgfpathlineto{\pgfqpoint{5.899971in}{2.377334in}}%
\pgfpathlineto{\pgfqpoint{5.893149in}{2.372415in}}%
\pgfpathlineto{\pgfqpoint{5.886319in}{2.367437in}}%
\pgfpathlineto{\pgfqpoint{5.872648in}{2.367958in}}%
\pgfpathlineto{\pgfqpoint{5.858985in}{2.368503in}}%
\pgfpathlineto{\pgfqpoint{5.845331in}{2.369071in}}%
\pgfpathlineto{\pgfqpoint{5.831686in}{2.369662in}}%
\pgfpathlineto{\pgfqpoint{5.838534in}{2.374727in}}%
\pgfpathlineto{\pgfqpoint{5.845374in}{2.379736in}}%
\pgfpathlineto{\pgfqpoint{5.852208in}{2.384693in}}%
\pgfpathlineto{\pgfqpoint{5.859034in}{2.389600in}}%
\pgfpathclose%
\pgfusepath{fill}%
\end{pgfscope}%
\begin{pgfscope}%
\pgfpathrectangle{\pgfqpoint{1.254980in}{0.150000in}}{\pgfqpoint{5.490039in}{5.490039in}}%
\pgfusepath{clip}%
\pgfsetbuttcap%
\pgfsetroundjoin%
\definecolor{currentfill}{rgb}{0.274952,0.037752,0.364543}%
\pgfsetfillcolor{currentfill}%
\pgfsetfillopacity{0.700000}%
\pgfsetlinewidth{0.000000pt}%
\definecolor{currentstroke}{rgb}{0.000000,0.000000,0.000000}%
\pgfsetstrokecolor{currentstroke}%
\pgfsetdash{}{0pt}%
\pgfpathmoveto{\pgfqpoint{3.326160in}{2.104665in}}%
\pgfpathlineto{\pgfqpoint{3.339137in}{2.099149in}}%
\pgfpathlineto{\pgfqpoint{3.352119in}{2.093666in}}%
\pgfpathlineto{\pgfqpoint{3.365105in}{2.088215in}}%
\pgfpathlineto{\pgfqpoint{3.378096in}{2.082796in}}%
\pgfpathlineto{\pgfqpoint{3.370208in}{2.078669in}}%
\pgfpathlineto{\pgfqpoint{3.362312in}{2.074694in}}%
\pgfpathlineto{\pgfqpoint{3.354406in}{2.070877in}}%
\pgfpathlineto{\pgfqpoint{3.346491in}{2.067224in}}%
\pgfpathlineto{\pgfqpoint{3.333480in}{2.072849in}}%
\pgfpathlineto{\pgfqpoint{3.320474in}{2.078507in}}%
\pgfpathlineto{\pgfqpoint{3.307472in}{2.084198in}}%
\pgfpathlineto{\pgfqpoint{3.294475in}{2.089921in}}%
\pgfpathlineto{\pgfqpoint{3.302411in}{2.093362in}}%
\pgfpathlineto{\pgfqpoint{3.310337in}{2.096971in}}%
\pgfpathlineto{\pgfqpoint{3.318253in}{2.100740in}}%
\pgfpathlineto{\pgfqpoint{3.326160in}{2.104665in}}%
\pgfpathclose%
\pgfusepath{fill}%
\end{pgfscope}%
\begin{pgfscope}%
\pgfpathrectangle{\pgfqpoint{1.254980in}{0.150000in}}{\pgfqpoint{5.490039in}{5.490039in}}%
\pgfusepath{clip}%
\pgfsetbuttcap%
\pgfsetroundjoin%
\definecolor{currentfill}{rgb}{0.267004,0.004874,0.329415}%
\pgfsetfillcolor{currentfill}%
\pgfsetfillopacity{0.700000}%
\pgfsetlinewidth{0.000000pt}%
\definecolor{currentstroke}{rgb}{0.000000,0.000000,0.000000}%
\pgfsetstrokecolor{currentstroke}%
\pgfsetdash{}{0pt}%
\pgfpathmoveto{\pgfqpoint{3.867060in}{2.045882in}}%
\pgfpathlineto{\pgfqpoint{3.880136in}{2.042022in}}%
\pgfpathlineto{\pgfqpoint{3.893217in}{2.038189in}}%
\pgfpathlineto{\pgfqpoint{3.906304in}{2.034384in}}%
\pgfpathlineto{\pgfqpoint{3.919397in}{2.030607in}}%
\pgfpathlineto{\pgfqpoint{3.911754in}{2.023463in}}%
\pgfpathlineto{\pgfqpoint{3.904105in}{2.016377in}}%
\pgfpathlineto{\pgfqpoint{3.896451in}{2.009350in}}%
\pgfpathlineto{\pgfqpoint{3.888790in}{2.002388in}}%
\pgfpathlineto{\pgfqpoint{3.875684in}{2.006321in}}%
\pgfpathlineto{\pgfqpoint{3.862583in}{2.010281in}}%
\pgfpathlineto{\pgfqpoint{3.849488in}{2.014269in}}%
\pgfpathlineto{\pgfqpoint{3.836399in}{2.018284in}}%
\pgfpathlineto{\pgfqpoint{3.844073in}{2.025087in}}%
\pgfpathlineto{\pgfqpoint{3.851742in}{2.031956in}}%
\pgfpathlineto{\pgfqpoint{3.859404in}{2.038889in}}%
\pgfpathlineto{\pgfqpoint{3.867060in}{2.045882in}}%
\pgfpathclose%
\pgfusepath{fill}%
\end{pgfscope}%
\begin{pgfscope}%
\pgfpathrectangle{\pgfqpoint{1.254980in}{0.150000in}}{\pgfqpoint{5.490039in}{5.490039in}}%
\pgfusepath{clip}%
\pgfsetbuttcap%
\pgfsetroundjoin%
\definecolor{currentfill}{rgb}{0.282290,0.145912,0.461510}%
\pgfsetfillcolor{currentfill}%
\pgfsetfillopacity{0.700000}%
\pgfsetlinewidth{0.000000pt}%
\definecolor{currentstroke}{rgb}{0.000000,0.000000,0.000000}%
\pgfsetstrokecolor{currentstroke}%
\pgfsetdash{}{0pt}%
\pgfpathmoveto{\pgfqpoint{5.340605in}{2.298537in}}%
\pgfpathlineto{\pgfqpoint{5.354079in}{2.297520in}}%
\pgfpathlineto{\pgfqpoint{5.367561in}{2.296527in}}%
\pgfpathlineto{\pgfqpoint{5.381052in}{2.295558in}}%
\pgfpathlineto{\pgfqpoint{5.394551in}{2.294612in}}%
\pgfpathlineto{\pgfqpoint{5.387479in}{2.288053in}}%
\pgfpathlineto{\pgfqpoint{5.380400in}{2.281419in}}%
\pgfpathlineto{\pgfqpoint{5.373314in}{2.274711in}}%
\pgfpathlineto{\pgfqpoint{5.366220in}{2.267926in}}%
\pgfpathlineto{\pgfqpoint{5.352708in}{2.268846in}}%
\pgfpathlineto{\pgfqpoint{5.339203in}{2.269790in}}%
\pgfpathlineto{\pgfqpoint{5.325707in}{2.270757in}}%
\pgfpathlineto{\pgfqpoint{5.312219in}{2.271748in}}%
\pgfpathlineto{\pgfqpoint{5.319327in}{2.278554in}}%
\pgfpathlineto{\pgfqpoint{5.326426in}{2.285287in}}%
\pgfpathlineto{\pgfqpoint{5.333519in}{2.291947in}}%
\pgfpathlineto{\pgfqpoint{5.340605in}{2.298537in}}%
\pgfpathclose%
\pgfusepath{fill}%
\end{pgfscope}%
\begin{pgfscope}%
\pgfpathrectangle{\pgfqpoint{1.254980in}{0.150000in}}{\pgfqpoint{5.490039in}{5.490039in}}%
\pgfusepath{clip}%
\pgfsetbuttcap%
\pgfsetroundjoin%
\definecolor{currentfill}{rgb}{0.282656,0.100196,0.422160}%
\pgfsetfillcolor{currentfill}%
\pgfsetfillopacity{0.700000}%
\pgfsetlinewidth{0.000000pt}%
\definecolor{currentstroke}{rgb}{0.000000,0.000000,0.000000}%
\pgfsetstrokecolor{currentstroke}%
\pgfsetdash{}{0pt}%
\pgfpathmoveto{\pgfqpoint{3.003096in}{2.203976in}}%
\pgfpathlineto{\pgfqpoint{3.016037in}{2.197332in}}%
\pgfpathlineto{\pgfqpoint{3.028982in}{2.190726in}}%
\pgfpathlineto{\pgfqpoint{3.041930in}{2.184157in}}%
\pgfpathlineto{\pgfqpoint{3.054882in}{2.177625in}}%
\pgfpathlineto{\pgfqpoint{3.046802in}{2.176013in}}%
\pgfpathlineto{\pgfqpoint{3.038711in}{2.174614in}}%
\pgfpathlineto{\pgfqpoint{3.030606in}{2.173434in}}%
\pgfpathlineto{\pgfqpoint{3.022489in}{2.172480in}}%
\pgfpathlineto{\pgfqpoint{3.009512in}{2.179246in}}%
\pgfpathlineto{\pgfqpoint{2.996539in}{2.186049in}}%
\pgfpathlineto{\pgfqpoint{2.983570in}{2.192890in}}%
\pgfpathlineto{\pgfqpoint{2.970604in}{2.199768in}}%
\pgfpathlineto{\pgfqpoint{2.978746in}{2.200483in}}%
\pgfpathlineto{\pgfqpoint{2.986876in}{2.201426in}}%
\pgfpathlineto{\pgfqpoint{2.994992in}{2.202593in}}%
\pgfpathlineto{\pgfqpoint{3.003096in}{2.203976in}}%
\pgfpathclose%
\pgfusepath{fill}%
\end{pgfscope}%
\begin{pgfscope}%
\pgfpathrectangle{\pgfqpoint{1.254980in}{0.150000in}}{\pgfqpoint{5.490039in}{5.490039in}}%
\pgfusepath{clip}%
\pgfsetbuttcap%
\pgfsetroundjoin%
\definecolor{currentfill}{rgb}{0.283091,0.110553,0.431554}%
\pgfsetfillcolor{currentfill}%
\pgfsetfillopacity{0.700000}%
\pgfsetlinewidth{0.000000pt}%
\definecolor{currentstroke}{rgb}{0.000000,0.000000,0.000000}%
\pgfsetstrokecolor{currentstroke}%
\pgfsetdash{}{0pt}%
\pgfpathmoveto{\pgfqpoint{5.039983in}{2.233489in}}%
\pgfpathlineto{\pgfqpoint{5.053369in}{2.232126in}}%
\pgfpathlineto{\pgfqpoint{5.066762in}{2.230787in}}%
\pgfpathlineto{\pgfqpoint{5.080164in}{2.229472in}}%
\pgfpathlineto{\pgfqpoint{5.093573in}{2.228181in}}%
\pgfpathlineto{\pgfqpoint{5.086366in}{2.220707in}}%
\pgfpathlineto{\pgfqpoint{5.079152in}{2.213162in}}%
\pgfpathlineto{\pgfqpoint{5.071932in}{2.205548in}}%
\pgfpathlineto{\pgfqpoint{5.064705in}{2.197863in}}%
\pgfpathlineto{\pgfqpoint{5.051283in}{2.199168in}}%
\pgfpathlineto{\pgfqpoint{5.037870in}{2.200496in}}%
\pgfpathlineto{\pgfqpoint{5.024464in}{2.201849in}}%
\pgfpathlineto{\pgfqpoint{5.011066in}{2.203226in}}%
\pgfpathlineto{\pgfqpoint{5.018305in}{2.210892in}}%
\pgfpathlineto{\pgfqpoint{5.025538in}{2.218492in}}%
\pgfpathlineto{\pgfqpoint{5.032763in}{2.226024in}}%
\pgfpathlineto{\pgfqpoint{5.039983in}{2.233489in}}%
\pgfpathclose%
\pgfusepath{fill}%
\end{pgfscope}%
\begin{pgfscope}%
\pgfpathrectangle{\pgfqpoint{1.254980in}{0.150000in}}{\pgfqpoint{5.490039in}{5.490039in}}%
\pgfusepath{clip}%
\pgfsetbuttcap%
\pgfsetroundjoin%
\definecolor{currentfill}{rgb}{0.271828,0.209303,0.504434}%
\pgfsetfillcolor{currentfill}%
\pgfsetfillopacity{0.700000}%
\pgfsetlinewidth{0.000000pt}%
\definecolor{currentstroke}{rgb}{0.000000,0.000000,0.000000}%
\pgfsetstrokecolor{currentstroke}%
\pgfsetdash{}{0pt}%
\pgfpathmoveto{\pgfqpoint{2.575252in}{2.415557in}}%
\pgfpathlineto{\pgfqpoint{2.588171in}{2.407229in}}%
\pgfpathlineto{\pgfqpoint{2.601092in}{2.398950in}}%
\pgfpathlineto{\pgfqpoint{2.614015in}{2.390719in}}%
\pgfpathlineto{\pgfqpoint{2.626940in}{2.382535in}}%
\pgfpathlineto{\pgfqpoint{2.618551in}{2.384606in}}%
\pgfpathlineto{\pgfqpoint{2.610144in}{2.386965in}}%
\pgfpathlineto{\pgfqpoint{2.601719in}{2.389620in}}%
\pgfpathlineto{\pgfqpoint{2.593276in}{2.392579in}}%
\pgfpathlineto{\pgfqpoint{2.580319in}{2.401026in}}%
\pgfpathlineto{\pgfqpoint{2.567364in}{2.409521in}}%
\pgfpathlineto{\pgfqpoint{2.554411in}{2.418065in}}%
\pgfpathlineto{\pgfqpoint{2.541460in}{2.426657in}}%
\pgfpathlineto{\pgfqpoint{2.549936in}{2.423430in}}%
\pgfpathlineto{\pgfqpoint{2.558393in}{2.420508in}}%
\pgfpathlineto{\pgfqpoint{2.566832in}{2.417886in}}%
\pgfpathlineto{\pgfqpoint{2.575252in}{2.415557in}}%
\pgfpathclose%
\pgfusepath{fill}%
\end{pgfscope}%
\begin{pgfscope}%
\pgfpathrectangle{\pgfqpoint{1.254980in}{0.150000in}}{\pgfqpoint{5.490039in}{5.490039in}}%
\pgfusepath{clip}%
\pgfsetbuttcap%
\pgfsetroundjoin%
\definecolor{currentfill}{rgb}{0.280267,0.073417,0.397163}%
\pgfsetfillcolor{currentfill}%
\pgfsetfillopacity{0.700000}%
\pgfsetlinewidth{0.000000pt}%
\definecolor{currentstroke}{rgb}{0.000000,0.000000,0.000000}%
\pgfsetstrokecolor{currentstroke}%
\pgfsetdash{}{0pt}%
\pgfpathmoveto{\pgfqpoint{4.739239in}{2.165907in}}%
\pgfpathlineto{\pgfqpoint{4.752539in}{2.164073in}}%
\pgfpathlineto{\pgfqpoint{4.765845in}{2.162264in}}%
\pgfpathlineto{\pgfqpoint{4.779159in}{2.160479in}}%
\pgfpathlineto{\pgfqpoint{4.792481in}{2.158719in}}%
\pgfpathlineto{\pgfqpoint{4.785153in}{2.150603in}}%
\pgfpathlineto{\pgfqpoint{4.777819in}{2.142434in}}%
\pgfpathlineto{\pgfqpoint{4.770479in}{2.134213in}}%
\pgfpathlineto{\pgfqpoint{4.763133in}{2.125940in}}%
\pgfpathlineto{\pgfqpoint{4.749801in}{2.127753in}}%
\pgfpathlineto{\pgfqpoint{4.736476in}{2.129591in}}%
\pgfpathlineto{\pgfqpoint{4.723158in}{2.131453in}}%
\pgfpathlineto{\pgfqpoint{4.709848in}{2.133340in}}%
\pgfpathlineto{\pgfqpoint{4.717205in}{2.141555in}}%
\pgfpathlineto{\pgfqpoint{4.724555in}{2.149722in}}%
\pgfpathlineto{\pgfqpoint{4.731900in}{2.157840in}}%
\pgfpathlineto{\pgfqpoint{4.739239in}{2.165907in}}%
\pgfpathclose%
\pgfusepath{fill}%
\end{pgfscope}%
\begin{pgfscope}%
\pgfpathrectangle{\pgfqpoint{1.254980in}{0.150000in}}{\pgfqpoint{5.490039in}{5.490039in}}%
\pgfusepath{clip}%
\pgfsetbuttcap%
\pgfsetroundjoin%
\definecolor{currentfill}{rgb}{0.269944,0.014625,0.341379}%
\pgfsetfillcolor{currentfill}%
\pgfsetfillopacity{0.700000}%
\pgfsetlinewidth{0.000000pt}%
\definecolor{currentstroke}{rgb}{0.000000,0.000000,0.000000}%
\pgfsetstrokecolor{currentstroke}%
\pgfsetdash{}{0pt}%
\pgfpathmoveto{\pgfqpoint{4.220360in}{2.068981in}}%
\pgfpathlineto{\pgfqpoint{4.233521in}{2.066047in}}%
\pgfpathlineto{\pgfqpoint{4.246687in}{2.063138in}}%
\pgfpathlineto{\pgfqpoint{4.259861in}{2.060256in}}%
\pgfpathlineto{\pgfqpoint{4.273041in}{2.057399in}}%
\pgfpathlineto{\pgfqpoint{4.265526in}{2.049270in}}%
\pgfpathlineto{\pgfqpoint{4.258006in}{2.041144in}}%
\pgfpathlineto{\pgfqpoint{4.250481in}{2.033023in}}%
\pgfpathlineto{\pgfqpoint{4.242950in}{2.024911in}}%
\pgfpathlineto{\pgfqpoint{4.229759in}{2.027885in}}%
\pgfpathlineto{\pgfqpoint{4.216575in}{2.030885in}}%
\pgfpathlineto{\pgfqpoint{4.203396in}{2.033911in}}%
\pgfpathlineto{\pgfqpoint{4.190225in}{2.036962in}}%
\pgfpathlineto{\pgfqpoint{4.197767in}{2.044952in}}%
\pgfpathlineto{\pgfqpoint{4.205303in}{2.052954in}}%
\pgfpathlineto{\pgfqpoint{4.212834in}{2.060964in}}%
\pgfpathlineto{\pgfqpoint{4.220360in}{2.068981in}}%
\pgfpathclose%
\pgfusepath{fill}%
\end{pgfscope}%
\begin{pgfscope}%
\pgfpathrectangle{\pgfqpoint{1.254980in}{0.150000in}}{\pgfqpoint{5.490039in}{5.490039in}}%
\pgfusepath{clip}%
\pgfsetbuttcap%
\pgfsetroundjoin%
\definecolor{currentfill}{rgb}{0.282623,0.140926,0.457517}%
\pgfsetfillcolor{currentfill}%
\pgfsetfillopacity{0.700000}%
\pgfsetlinewidth{0.000000pt}%
\definecolor{currentstroke}{rgb}{0.000000,0.000000,0.000000}%
\pgfsetstrokecolor{currentstroke}%
\pgfsetdash{}{0pt}%
\pgfpathmoveto{\pgfqpoint{2.815260in}{2.285355in}}%
\pgfpathlineto{\pgfqpoint{2.828188in}{2.278000in}}%
\pgfpathlineto{\pgfqpoint{2.841120in}{2.270687in}}%
\pgfpathlineto{\pgfqpoint{2.854054in}{2.263415in}}%
\pgfpathlineto{\pgfqpoint{2.866991in}{2.256184in}}%
\pgfpathlineto{\pgfqpoint{2.858782in}{2.256191in}}%
\pgfpathlineto{\pgfqpoint{2.850558in}{2.256446in}}%
\pgfpathlineto{\pgfqpoint{2.842320in}{2.256955in}}%
\pgfpathlineto{\pgfqpoint{2.834066in}{2.257726in}}%
\pgfpathlineto{\pgfqpoint{2.821101in}{2.265206in}}%
\pgfpathlineto{\pgfqpoint{2.808138in}{2.272726in}}%
\pgfpathlineto{\pgfqpoint{2.795179in}{2.280288in}}%
\pgfpathlineto{\pgfqpoint{2.782222in}{2.287892in}}%
\pgfpathlineto{\pgfqpoint{2.790504in}{2.286867in}}%
\pgfpathlineto{\pgfqpoint{2.798771in}{2.286107in}}%
\pgfpathlineto{\pgfqpoint{2.807023in}{2.285605in}}%
\pgfpathlineto{\pgfqpoint{2.815260in}{2.285355in}}%
\pgfpathclose%
\pgfusepath{fill}%
\end{pgfscope}%
\begin{pgfscope}%
\pgfpathrectangle{\pgfqpoint{1.254980in}{0.150000in}}{\pgfqpoint{5.490039in}{5.490039in}}%
\pgfusepath{clip}%
\pgfsetbuttcap%
\pgfsetroundjoin%
\definecolor{currentfill}{rgb}{0.279574,0.170599,0.479997}%
\pgfsetfillcolor{currentfill}%
\pgfsetfillopacity{0.700000}%
\pgfsetlinewidth{0.000000pt}%
\definecolor{currentstroke}{rgb}{0.000000,0.000000,0.000000}%
\pgfsetstrokecolor{currentstroke}%
\pgfsetdash{}{0pt}%
\pgfpathmoveto{\pgfqpoint{5.558912in}{2.337172in}}%
\pgfpathlineto{\pgfqpoint{5.572455in}{2.336349in}}%
\pgfpathlineto{\pgfqpoint{5.586007in}{2.335549in}}%
\pgfpathlineto{\pgfqpoint{5.599567in}{2.334773in}}%
\pgfpathlineto{\pgfqpoint{5.613135in}{2.334020in}}%
\pgfpathlineto{\pgfqpoint{5.606168in}{2.328166in}}%
\pgfpathlineto{\pgfqpoint{5.599193in}{2.322242in}}%
\pgfpathlineto{\pgfqpoint{5.592210in}{2.316246in}}%
\pgfpathlineto{\pgfqpoint{5.585220in}{2.310175in}}%
\pgfpathlineto{\pgfqpoint{5.571636in}{2.310875in}}%
\pgfpathlineto{\pgfqpoint{5.558060in}{2.311599in}}%
\pgfpathlineto{\pgfqpoint{5.544493in}{2.312347in}}%
\pgfpathlineto{\pgfqpoint{5.530935in}{2.313118in}}%
\pgfpathlineto{\pgfqpoint{5.537940in}{2.319236in}}%
\pgfpathlineto{\pgfqpoint{5.544938in}{2.325283in}}%
\pgfpathlineto{\pgfqpoint{5.551929in}{2.331261in}}%
\pgfpathlineto{\pgfqpoint{5.558912in}{2.337172in}}%
\pgfpathclose%
\pgfusepath{fill}%
\end{pgfscope}%
\begin{pgfscope}%
\pgfpathrectangle{\pgfqpoint{1.254980in}{0.150000in}}{\pgfqpoint{5.490039in}{5.490039in}}%
\pgfusepath{clip}%
\pgfsetbuttcap%
\pgfsetroundjoin%
\definecolor{currentfill}{rgb}{0.278791,0.062145,0.386592}%
\pgfsetfillcolor{currentfill}%
\pgfsetfillopacity{0.700000}%
\pgfsetlinewidth{0.000000pt}%
\definecolor{currentstroke}{rgb}{0.000000,0.000000,0.000000}%
\pgfsetstrokecolor{currentstroke}%
\pgfsetdash{}{0pt}%
\pgfpathmoveto{\pgfqpoint{3.190649in}{2.136900in}}%
\pgfpathlineto{\pgfqpoint{3.203613in}{2.130909in}}%
\pgfpathlineto{\pgfqpoint{3.216581in}{2.124953in}}%
\pgfpathlineto{\pgfqpoint{3.229553in}{2.119030in}}%
\pgfpathlineto{\pgfqpoint{3.242529in}{2.113141in}}%
\pgfpathlineto{\pgfqpoint{3.234562in}{2.110087in}}%
\pgfpathlineto{\pgfqpoint{3.226585in}{2.107212in}}%
\pgfpathlineto{\pgfqpoint{3.218597in}{2.104524in}}%
\pgfpathlineto{\pgfqpoint{3.210599in}{2.102027in}}%
\pgfpathlineto{\pgfqpoint{3.197601in}{2.108136in}}%
\pgfpathlineto{\pgfqpoint{3.184607in}{2.114279in}}%
\pgfpathlineto{\pgfqpoint{3.171617in}{2.120456in}}%
\pgfpathlineto{\pgfqpoint{3.158631in}{2.126668in}}%
\pgfpathlineto{\pgfqpoint{3.166652in}{2.128939in}}%
\pgfpathlineto{\pgfqpoint{3.174662in}{2.131405in}}%
\pgfpathlineto{\pgfqpoint{3.182661in}{2.134061in}}%
\pgfpathlineto{\pgfqpoint{3.190649in}{2.136900in}}%
\pgfpathclose%
\pgfusepath{fill}%
\end{pgfscope}%
\begin{pgfscope}%
\pgfpathrectangle{\pgfqpoint{1.254980in}{0.150000in}}{\pgfqpoint{5.490039in}{5.490039in}}%
\pgfusepath{clip}%
\pgfsetbuttcap%
\pgfsetroundjoin%
\definecolor{currentfill}{rgb}{0.267004,0.004874,0.329415}%
\pgfsetfillcolor{currentfill}%
\pgfsetfillopacity{0.700000}%
\pgfsetlinewidth{0.000000pt}%
\definecolor{currentstroke}{rgb}{0.000000,0.000000,0.000000}%
\pgfsetstrokecolor{currentstroke}%
\pgfsetdash{}{0pt}%
\pgfpathmoveto{\pgfqpoint{4.002285in}{2.045410in}}%
\pgfpathlineto{\pgfqpoint{4.015395in}{2.041911in}}%
\pgfpathlineto{\pgfqpoint{4.028510in}{2.038439in}}%
\pgfpathlineto{\pgfqpoint{4.041632in}{2.034994in}}%
\pgfpathlineto{\pgfqpoint{4.054759in}{2.031575in}}%
\pgfpathlineto{\pgfqpoint{4.047166in}{2.023965in}}%
\pgfpathlineto{\pgfqpoint{4.039566in}{2.016390in}}%
\pgfpathlineto{\pgfqpoint{4.031962in}{2.008856in}}%
\pgfpathlineto{\pgfqpoint{4.024351in}{2.001364in}}%
\pgfpathlineto{\pgfqpoint{4.011211in}{2.004925in}}%
\pgfpathlineto{\pgfqpoint{3.998077in}{2.008513in}}%
\pgfpathlineto{\pgfqpoint{3.984948in}{2.012128in}}%
\pgfpathlineto{\pgfqpoint{3.971826in}{2.015770in}}%
\pgfpathlineto{\pgfqpoint{3.979450in}{2.023114in}}%
\pgfpathlineto{\pgfqpoint{3.987067in}{2.030504in}}%
\pgfpathlineto{\pgfqpoint{3.994679in}{2.037937in}}%
\pgfpathlineto{\pgfqpoint{4.002285in}{2.045410in}}%
\pgfpathclose%
\pgfusepath{fill}%
\end{pgfscope}%
\begin{pgfscope}%
\pgfpathrectangle{\pgfqpoint{1.254980in}{0.150000in}}{\pgfqpoint{5.490039in}{5.490039in}}%
\pgfusepath{clip}%
\pgfsetbuttcap%
\pgfsetroundjoin%
\definecolor{currentfill}{rgb}{0.274952,0.037752,0.364543}%
\pgfsetfillcolor{currentfill}%
\pgfsetfillopacity{0.700000}%
\pgfsetlinewidth{0.000000pt}%
\definecolor{currentstroke}{rgb}{0.000000,0.000000,0.000000}%
\pgfsetstrokecolor{currentstroke}%
\pgfsetdash{}{0pt}%
\pgfpathmoveto{\pgfqpoint{4.438463in}{2.101905in}}%
\pgfpathlineto{\pgfqpoint{4.451681in}{2.099474in}}%
\pgfpathlineto{\pgfqpoint{4.464906in}{2.097069in}}%
\pgfpathlineto{\pgfqpoint{4.478138in}{2.094689in}}%
\pgfpathlineto{\pgfqpoint{4.491377in}{2.092334in}}%
\pgfpathlineto{\pgfqpoint{4.483938in}{2.084002in}}%
\pgfpathlineto{\pgfqpoint{4.476494in}{2.075646in}}%
\pgfpathlineto{\pgfqpoint{4.469044in}{2.067267in}}%
\pgfpathlineto{\pgfqpoint{4.461589in}{2.058868in}}%
\pgfpathlineto{\pgfqpoint{4.448340in}{2.061315in}}%
\pgfpathlineto{\pgfqpoint{4.435098in}{2.063786in}}%
\pgfpathlineto{\pgfqpoint{4.421862in}{2.066284in}}%
\pgfpathlineto{\pgfqpoint{4.408633in}{2.068806in}}%
\pgfpathlineto{\pgfqpoint{4.416099in}{2.077108in}}%
\pgfpathlineto{\pgfqpoint{4.423559in}{2.085394in}}%
\pgfpathlineto{\pgfqpoint{4.431014in}{2.093660in}}%
\pgfpathlineto{\pgfqpoint{4.438463in}{2.101905in}}%
\pgfpathclose%
\pgfusepath{fill}%
\end{pgfscope}%
\begin{pgfscope}%
\pgfpathrectangle{\pgfqpoint{1.254980in}{0.150000in}}{\pgfqpoint{5.490039in}{5.490039in}}%
\pgfusepath{clip}%
\pgfsetbuttcap%
\pgfsetroundjoin%
\definecolor{currentfill}{rgb}{0.276194,0.190074,0.493001}%
\pgfsetfillcolor{currentfill}%
\pgfsetfillopacity{0.700000}%
\pgfsetlinewidth{0.000000pt}%
\definecolor{currentstroke}{rgb}{0.000000,0.000000,0.000000}%
\pgfsetstrokecolor{currentstroke}%
\pgfsetdash{}{0pt}%
\pgfpathmoveto{\pgfqpoint{5.777190in}{2.372262in}}%
\pgfpathlineto{\pgfqpoint{5.790801in}{2.371577in}}%
\pgfpathlineto{\pgfqpoint{5.804421in}{2.370915in}}%
\pgfpathlineto{\pgfqpoint{5.818049in}{2.370277in}}%
\pgfpathlineto{\pgfqpoint{5.831686in}{2.369662in}}%
\pgfpathlineto{\pgfqpoint{5.824830in}{2.364539in}}%
\pgfpathlineto{\pgfqpoint{5.817966in}{2.359356in}}%
\pgfpathlineto{\pgfqpoint{5.811095in}{2.354108in}}%
\pgfpathlineto{\pgfqpoint{5.804216in}{2.348794in}}%
\pgfpathlineto{\pgfqpoint{5.790562in}{2.349330in}}%
\pgfpathlineto{\pgfqpoint{5.776916in}{2.349889in}}%
\pgfpathlineto{\pgfqpoint{5.763279in}{2.350472in}}%
\pgfpathlineto{\pgfqpoint{5.749650in}{2.351079in}}%
\pgfpathlineto{\pgfqpoint{5.756547in}{2.356466in}}%
\pgfpathlineto{\pgfqpoint{5.763435in}{2.361791in}}%
\pgfpathlineto{\pgfqpoint{5.770316in}{2.367055in}}%
\pgfpathlineto{\pgfqpoint{5.777190in}{2.372262in}}%
\pgfpathclose%
\pgfusepath{fill}%
\end{pgfscope}%
\begin{pgfscope}%
\pgfpathrectangle{\pgfqpoint{1.254980in}{0.150000in}}{\pgfqpoint{5.490039in}{5.490039in}}%
\pgfusepath{clip}%
\pgfsetbuttcap%
\pgfsetroundjoin%
\definecolor{currentfill}{rgb}{0.282623,0.140926,0.457517}%
\pgfsetfillcolor{currentfill}%
\pgfsetfillopacity{0.700000}%
\pgfsetlinewidth{0.000000pt}%
\definecolor{currentstroke}{rgb}{0.000000,0.000000,0.000000}%
\pgfsetstrokecolor{currentstroke}%
\pgfsetdash{}{0pt}%
\pgfpathmoveto{\pgfqpoint{5.258349in}{2.275952in}}%
\pgfpathlineto{\pgfqpoint{5.271804in}{2.274865in}}%
\pgfpathlineto{\pgfqpoint{5.285268in}{2.273803in}}%
\pgfpathlineto{\pgfqpoint{5.298740in}{2.272764in}}%
\pgfpathlineto{\pgfqpoint{5.312219in}{2.271748in}}%
\pgfpathlineto{\pgfqpoint{5.305105in}{2.264869in}}%
\pgfpathlineto{\pgfqpoint{5.297984in}{2.257914in}}%
\pgfpathlineto{\pgfqpoint{5.290855in}{2.250883in}}%
\pgfpathlineto{\pgfqpoint{5.283720in}{2.243775in}}%
\pgfpathlineto{\pgfqpoint{5.270227in}{2.244778in}}%
\pgfpathlineto{\pgfqpoint{5.256742in}{2.245805in}}%
\pgfpathlineto{\pgfqpoint{5.243266in}{2.246855in}}%
\pgfpathlineto{\pgfqpoint{5.229797in}{2.247929in}}%
\pgfpathlineto{\pgfqpoint{5.236945in}{2.255044in}}%
\pgfpathlineto{\pgfqpoint{5.244087in}{2.262086in}}%
\pgfpathlineto{\pgfqpoint{5.251221in}{2.269055in}}%
\pgfpathlineto{\pgfqpoint{5.258349in}{2.275952in}}%
\pgfpathclose%
\pgfusepath{fill}%
\end{pgfscope}%
\begin{pgfscope}%
\pgfpathrectangle{\pgfqpoint{1.254980in}{0.150000in}}{\pgfqpoint{5.490039in}{5.490039in}}%
\pgfusepath{clip}%
\pgfsetbuttcap%
\pgfsetroundjoin%
\definecolor{currentfill}{rgb}{0.282910,0.105393,0.426902}%
\pgfsetfillcolor{currentfill}%
\pgfsetfillopacity{0.700000}%
\pgfsetlinewidth{0.000000pt}%
\definecolor{currentstroke}{rgb}{0.000000,0.000000,0.000000}%
\pgfsetstrokecolor{currentstroke}%
\pgfsetdash{}{0pt}%
\pgfpathmoveto{\pgfqpoint{4.957552in}{2.208976in}}%
\pgfpathlineto{\pgfqpoint{4.970919in}{2.207502in}}%
\pgfpathlineto{\pgfqpoint{4.984294in}{2.206053in}}%
\pgfpathlineto{\pgfqpoint{4.997676in}{2.204627in}}%
\pgfpathlineto{\pgfqpoint{5.011066in}{2.203226in}}%
\pgfpathlineto{\pgfqpoint{5.003821in}{2.195493in}}%
\pgfpathlineto{\pgfqpoint{4.996569in}{2.187693in}}%
\pgfpathlineto{\pgfqpoint{4.989311in}{2.179825in}}%
\pgfpathlineto{\pgfqpoint{4.982047in}{2.171892in}}%
\pgfpathlineto{\pgfqpoint{4.968646in}{2.173320in}}%
\pgfpathlineto{\pgfqpoint{4.955252in}{2.174772in}}%
\pgfpathlineto{\pgfqpoint{4.941866in}{2.176249in}}%
\pgfpathlineto{\pgfqpoint{4.928487in}{2.177749in}}%
\pgfpathlineto{\pgfqpoint{4.935763in}{2.185651in}}%
\pgfpathlineto{\pgfqpoint{4.943032in}{2.193490in}}%
\pgfpathlineto{\pgfqpoint{4.950296in}{2.201265in}}%
\pgfpathlineto{\pgfqpoint{4.957552in}{2.208976in}}%
\pgfpathclose%
\pgfusepath{fill}%
\end{pgfscope}%
\begin{pgfscope}%
\pgfpathrectangle{\pgfqpoint{1.254980in}{0.150000in}}{\pgfqpoint{5.490039in}{5.490039in}}%
\pgfusepath{clip}%
\pgfsetbuttcap%
\pgfsetroundjoin%
\definecolor{currentfill}{rgb}{0.267004,0.004874,0.329415}%
\pgfsetfillcolor{currentfill}%
\pgfsetfillopacity{0.700000}%
\pgfsetlinewidth{0.000000pt}%
\definecolor{currentstroke}{rgb}{0.000000,0.000000,0.000000}%
\pgfsetstrokecolor{currentstroke}%
\pgfsetdash{}{0pt}%
\pgfpathmoveto{\pgfqpoint{3.648817in}{2.044233in}}%
\pgfpathlineto{\pgfqpoint{3.661856in}{2.039708in}}%
\pgfpathlineto{\pgfqpoint{3.674900in}{2.035212in}}%
\pgfpathlineto{\pgfqpoint{3.687949in}{2.030746in}}%
\pgfpathlineto{\pgfqpoint{3.701004in}{2.026308in}}%
\pgfpathlineto{\pgfqpoint{3.693267in}{2.020255in}}%
\pgfpathlineto{\pgfqpoint{3.685523in}{2.014300in}}%
\pgfpathlineto{\pgfqpoint{3.677771in}{2.008448in}}%
\pgfpathlineto{\pgfqpoint{3.670013in}{2.002703in}}%
\pgfpathlineto{\pgfqpoint{3.656942in}{2.007322in}}%
\pgfpathlineto{\pgfqpoint{3.643877in}{2.011970in}}%
\pgfpathlineto{\pgfqpoint{3.630816in}{2.016647in}}%
\pgfpathlineto{\pgfqpoint{3.617761in}{2.021352in}}%
\pgfpathlineto{\pgfqpoint{3.625536in}{2.026911in}}%
\pgfpathlineto{\pgfqpoint{3.633304in}{2.032581in}}%
\pgfpathlineto{\pgfqpoint{3.641064in}{2.038356in}}%
\pgfpathlineto{\pgfqpoint{3.648817in}{2.044233in}}%
\pgfpathclose%
\pgfusepath{fill}%
\end{pgfscope}%
\begin{pgfscope}%
\pgfpathrectangle{\pgfqpoint{1.254980in}{0.150000in}}{\pgfqpoint{5.490039in}{5.490039in}}%
\pgfusepath{clip}%
\pgfsetbuttcap%
\pgfsetroundjoin%
\definecolor{currentfill}{rgb}{0.269944,0.014625,0.341379}%
\pgfsetfillcolor{currentfill}%
\pgfsetfillopacity{0.700000}%
\pgfsetlinewidth{0.000000pt}%
\definecolor{currentstroke}{rgb}{0.000000,0.000000,0.000000}%
\pgfsetstrokecolor{currentstroke}%
\pgfsetdash{}{0pt}%
\pgfpathmoveto{\pgfqpoint{3.513503in}{2.060065in}}%
\pgfpathlineto{\pgfqpoint{3.526518in}{2.055121in}}%
\pgfpathlineto{\pgfqpoint{3.539537in}{2.050207in}}%
\pgfpathlineto{\pgfqpoint{3.552562in}{2.045324in}}%
\pgfpathlineto{\pgfqpoint{3.565592in}{2.040470in}}%
\pgfpathlineto{\pgfqpoint{3.557793in}{2.035215in}}%
\pgfpathlineto{\pgfqpoint{3.549985in}{2.030084in}}%
\pgfpathlineto{\pgfqpoint{3.542170in}{2.025080in}}%
\pgfpathlineto{\pgfqpoint{3.534347in}{2.020209in}}%
\pgfpathlineto{\pgfqpoint{3.521300in}{2.025256in}}%
\pgfpathlineto{\pgfqpoint{3.508258in}{2.030334in}}%
\pgfpathlineto{\pgfqpoint{3.495220in}{2.035441in}}%
\pgfpathlineto{\pgfqpoint{3.482187in}{2.040580in}}%
\pgfpathlineto{\pgfqpoint{3.490029in}{2.045252in}}%
\pgfpathlineto{\pgfqpoint{3.497861in}{2.050060in}}%
\pgfpathlineto{\pgfqpoint{3.505686in}{2.055000in}}%
\pgfpathlineto{\pgfqpoint{3.513503in}{2.060065in}}%
\pgfpathclose%
\pgfusepath{fill}%
\end{pgfscope}%
\begin{pgfscope}%
\pgfpathrectangle{\pgfqpoint{1.254980in}{0.150000in}}{\pgfqpoint{5.490039in}{5.490039in}}%
\pgfusepath{clip}%
\pgfsetbuttcap%
\pgfsetroundjoin%
\definecolor{currentfill}{rgb}{0.278791,0.062145,0.386592}%
\pgfsetfillcolor{currentfill}%
\pgfsetfillopacity{0.700000}%
\pgfsetlinewidth{0.000000pt}%
\definecolor{currentstroke}{rgb}{0.000000,0.000000,0.000000}%
\pgfsetstrokecolor{currentstroke}%
\pgfsetdash{}{0pt}%
\pgfpathmoveto{\pgfqpoint{4.656680in}{2.141134in}}%
\pgfpathlineto{\pgfqpoint{4.669961in}{2.139149in}}%
\pgfpathlineto{\pgfqpoint{4.683250in}{2.137188in}}%
\pgfpathlineto{\pgfqpoint{4.696545in}{2.135252in}}%
\pgfpathlineto{\pgfqpoint{4.709848in}{2.133340in}}%
\pgfpathlineto{\pgfqpoint{4.702486in}{2.125078in}}%
\pgfpathlineto{\pgfqpoint{4.695118in}{2.116769in}}%
\pgfpathlineto{\pgfqpoint{4.687744in}{2.108415in}}%
\pgfpathlineto{\pgfqpoint{4.680365in}{2.100017in}}%
\pgfpathlineto{\pgfqpoint{4.667051in}{2.101995in}}%
\pgfpathlineto{\pgfqpoint{4.653745in}{2.103997in}}%
\pgfpathlineto{\pgfqpoint{4.640446in}{2.106024in}}%
\pgfpathlineto{\pgfqpoint{4.627154in}{2.108076in}}%
\pgfpathlineto{\pgfqpoint{4.634544in}{2.116402in}}%
\pgfpathlineto{\pgfqpoint{4.641929in}{2.124689in}}%
\pgfpathlineto{\pgfqpoint{4.649307in}{2.132933in}}%
\pgfpathlineto{\pgfqpoint{4.656680in}{2.141134in}}%
\pgfpathclose%
\pgfusepath{fill}%
\end{pgfscope}%
\begin{pgfscope}%
\pgfpathrectangle{\pgfqpoint{1.254980in}{0.150000in}}{\pgfqpoint{5.490039in}{5.490039in}}%
\pgfusepath{clip}%
\pgfsetbuttcap%
\pgfsetroundjoin%
\definecolor{currentfill}{rgb}{0.267004,0.004874,0.329415}%
\pgfsetfillcolor{currentfill}%
\pgfsetfillopacity{0.700000}%
\pgfsetlinewidth{0.000000pt}%
\definecolor{currentstroke}{rgb}{0.000000,0.000000,0.000000}%
\pgfsetstrokecolor{currentstroke}%
\pgfsetdash{}{0pt}%
\pgfpathmoveto{\pgfqpoint{3.784098in}{2.034626in}}%
\pgfpathlineto{\pgfqpoint{3.797164in}{2.030498in}}%
\pgfpathlineto{\pgfqpoint{3.810237in}{2.026399in}}%
\pgfpathlineto{\pgfqpoint{3.823315in}{2.022328in}}%
\pgfpathlineto{\pgfqpoint{3.836399in}{2.018284in}}%
\pgfpathlineto{\pgfqpoint{3.828718in}{2.011554in}}%
\pgfpathlineto{\pgfqpoint{3.821030in}{2.004898in}}%
\pgfpathlineto{\pgfqpoint{3.813337in}{1.998321in}}%
\pgfpathlineto{\pgfqpoint{3.805636in}{1.991828in}}%
\pgfpathlineto{\pgfqpoint{3.792538in}{1.996040in}}%
\pgfpathlineto{\pgfqpoint{3.779446in}{2.000279in}}%
\pgfpathlineto{\pgfqpoint{3.766358in}{2.004547in}}%
\pgfpathlineto{\pgfqpoint{3.753277in}{2.008842in}}%
\pgfpathlineto{\pgfqpoint{3.760992in}{2.015162in}}%
\pgfpathlineto{\pgfqpoint{3.768700in}{2.021569in}}%
\pgfpathlineto{\pgfqpoint{3.776402in}{2.028058in}}%
\pgfpathlineto{\pgfqpoint{3.784098in}{2.034626in}}%
\pgfpathclose%
\pgfusepath{fill}%
\end{pgfscope}%
\begin{pgfscope}%
\pgfpathrectangle{\pgfqpoint{1.254980in}{0.150000in}}{\pgfqpoint{5.490039in}{5.490039in}}%
\pgfusepath{clip}%
\pgfsetbuttcap%
\pgfsetroundjoin%
\definecolor{currentfill}{rgb}{0.275191,0.194905,0.496005}%
\pgfsetfillcolor{currentfill}%
\pgfsetfillopacity{0.700000}%
\pgfsetlinewidth{0.000000pt}%
\definecolor{currentstroke}{rgb}{0.000000,0.000000,0.000000}%
\pgfsetstrokecolor{currentstroke}%
\pgfsetdash{}{0pt}%
\pgfpathmoveto{\pgfqpoint{2.626940in}{2.382535in}}%
\pgfpathlineto{\pgfqpoint{2.639867in}{2.374399in}}%
\pgfpathlineto{\pgfqpoint{2.652796in}{2.366310in}}%
\pgfpathlineto{\pgfqpoint{2.665728in}{2.358267in}}%
\pgfpathlineto{\pgfqpoint{2.678662in}{2.350270in}}%
\pgfpathlineto{\pgfqpoint{2.670304in}{2.352082in}}%
\pgfpathlineto{\pgfqpoint{2.661928in}{2.354179in}}%
\pgfpathlineto{\pgfqpoint{2.653535in}{2.356569in}}%
\pgfpathlineto{\pgfqpoint{2.645124in}{2.359258in}}%
\pgfpathlineto{\pgfqpoint{2.632159in}{2.367519in}}%
\pgfpathlineto{\pgfqpoint{2.619196in}{2.375825in}}%
\pgfpathlineto{\pgfqpoint{2.606235in}{2.384178in}}%
\pgfpathlineto{\pgfqpoint{2.593276in}{2.392579in}}%
\pgfpathlineto{\pgfqpoint{2.601719in}{2.389620in}}%
\pgfpathlineto{\pgfqpoint{2.610144in}{2.386965in}}%
\pgfpathlineto{\pgfqpoint{2.618551in}{2.384606in}}%
\pgfpathlineto{\pgfqpoint{2.626940in}{2.382535in}}%
\pgfpathclose%
\pgfusepath{fill}%
\end{pgfscope}%
\begin{pgfscope}%
\pgfpathrectangle{\pgfqpoint{1.254980in}{0.150000in}}{\pgfqpoint{5.490039in}{5.490039in}}%
\pgfusepath{clip}%
\pgfsetbuttcap%
\pgfsetroundjoin%
\definecolor{currentfill}{rgb}{0.273809,0.031497,0.358853}%
\pgfsetfillcolor{currentfill}%
\pgfsetfillopacity{0.700000}%
\pgfsetlinewidth{0.000000pt}%
\definecolor{currentstroke}{rgb}{0.000000,0.000000,0.000000}%
\pgfsetstrokecolor{currentstroke}%
\pgfsetdash{}{0pt}%
\pgfpathmoveto{\pgfqpoint{3.378096in}{2.082796in}}%
\pgfpathlineto{\pgfqpoint{3.391091in}{2.077409in}}%
\pgfpathlineto{\pgfqpoint{3.404091in}{2.072054in}}%
\pgfpathlineto{\pgfqpoint{3.417095in}{2.066731in}}%
\pgfpathlineto{\pgfqpoint{3.430104in}{2.061438in}}%
\pgfpathlineto{\pgfqpoint{3.422236in}{2.057109in}}%
\pgfpathlineto{\pgfqpoint{3.414359in}{2.052929in}}%
\pgfpathlineto{\pgfqpoint{3.406472in}{2.048904in}}%
\pgfpathlineto{\pgfqpoint{3.398577in}{2.045038in}}%
\pgfpathlineto{\pgfqpoint{3.385549in}{2.050537in}}%
\pgfpathlineto{\pgfqpoint{3.372525in}{2.056068in}}%
\pgfpathlineto{\pgfqpoint{3.359505in}{2.061630in}}%
\pgfpathlineto{\pgfqpoint{3.346491in}{2.067224in}}%
\pgfpathlineto{\pgfqpoint{3.354406in}{2.070877in}}%
\pgfpathlineto{\pgfqpoint{3.362312in}{2.074694in}}%
\pgfpathlineto{\pgfqpoint{3.370208in}{2.078669in}}%
\pgfpathlineto{\pgfqpoint{3.378096in}{2.082796in}}%
\pgfpathclose%
\pgfusepath{fill}%
\end{pgfscope}%
\begin{pgfscope}%
\pgfpathrectangle{\pgfqpoint{1.254980in}{0.150000in}}{\pgfqpoint{5.490039in}{5.490039in}}%
\pgfusepath{clip}%
\pgfsetbuttcap%
\pgfsetroundjoin%
\definecolor{currentfill}{rgb}{0.268510,0.009605,0.335427}%
\pgfsetfillcolor{currentfill}%
\pgfsetfillopacity{0.700000}%
\pgfsetlinewidth{0.000000pt}%
\definecolor{currentstroke}{rgb}{0.000000,0.000000,0.000000}%
\pgfsetstrokecolor{currentstroke}%
\pgfsetdash{}{0pt}%
\pgfpathmoveto{\pgfqpoint{4.137601in}{2.049430in}}%
\pgfpathlineto{\pgfqpoint{4.150747in}{2.046274in}}%
\pgfpathlineto{\pgfqpoint{4.163900in}{2.043144in}}%
\pgfpathlineto{\pgfqpoint{4.177059in}{2.040040in}}%
\pgfpathlineto{\pgfqpoint{4.190225in}{2.036962in}}%
\pgfpathlineto{\pgfqpoint{4.182677in}{2.028987in}}%
\pgfpathlineto{\pgfqpoint{4.175124in}{2.021030in}}%
\pgfpathlineto{\pgfqpoint{4.167566in}{2.013093in}}%
\pgfpathlineto{\pgfqpoint{4.160002in}{2.005179in}}%
\pgfpathlineto{\pgfqpoint{4.146825in}{2.008387in}}%
\pgfpathlineto{\pgfqpoint{4.133654in}{2.011621in}}%
\pgfpathlineto{\pgfqpoint{4.120489in}{2.014881in}}%
\pgfpathlineto{\pgfqpoint{4.107331in}{2.018167in}}%
\pgfpathlineto{\pgfqpoint{4.114907in}{2.025945in}}%
\pgfpathlineto{\pgfqpoint{4.122477in}{2.033751in}}%
\pgfpathlineto{\pgfqpoint{4.130042in}{2.041580in}}%
\pgfpathlineto{\pgfqpoint{4.137601in}{2.049430in}}%
\pgfpathclose%
\pgfusepath{fill}%
\end{pgfscope}%
\begin{pgfscope}%
\pgfpathrectangle{\pgfqpoint{1.254980in}{0.150000in}}{\pgfqpoint{5.490039in}{5.490039in}}%
\pgfusepath{clip}%
\pgfsetbuttcap%
\pgfsetroundjoin%
\definecolor{currentfill}{rgb}{0.280868,0.160771,0.472899}%
\pgfsetfillcolor{currentfill}%
\pgfsetfillopacity{0.700000}%
\pgfsetlinewidth{0.000000pt}%
\definecolor{currentstroke}{rgb}{0.000000,0.000000,0.000000}%
\pgfsetstrokecolor{currentstroke}%
\pgfsetdash{}{0pt}%
\pgfpathmoveto{\pgfqpoint{5.476784in}{2.316440in}}%
\pgfpathlineto{\pgfqpoint{5.490309in}{2.315574in}}%
\pgfpathlineto{\pgfqpoint{5.503842in}{2.314732in}}%
\pgfpathlineto{\pgfqpoint{5.517384in}{2.313913in}}%
\pgfpathlineto{\pgfqpoint{5.530935in}{2.313118in}}%
\pgfpathlineto{\pgfqpoint{5.523922in}{2.306927in}}%
\pgfpathlineto{\pgfqpoint{5.516901in}{2.300662in}}%
\pgfpathlineto{\pgfqpoint{5.509873in}{2.294321in}}%
\pgfpathlineto{\pgfqpoint{5.502837in}{2.287902in}}%
\pgfpathlineto{\pgfqpoint{5.489272in}{2.288658in}}%
\pgfpathlineto{\pgfqpoint{5.475715in}{2.289437in}}%
\pgfpathlineto{\pgfqpoint{5.462167in}{2.290240in}}%
\pgfpathlineto{\pgfqpoint{5.448627in}{2.291067in}}%
\pgfpathlineto{\pgfqpoint{5.455677in}{2.297520in}}%
\pgfpathlineto{\pgfqpoint{5.462720in}{2.303899in}}%
\pgfpathlineto{\pgfqpoint{5.469756in}{2.310205in}}%
\pgfpathlineto{\pgfqpoint{5.476784in}{2.316440in}}%
\pgfpathclose%
\pgfusepath{fill}%
\end{pgfscope}%
\begin{pgfscope}%
\pgfpathrectangle{\pgfqpoint{1.254980in}{0.150000in}}{\pgfqpoint{5.490039in}{5.490039in}}%
\pgfusepath{clip}%
\pgfsetbuttcap%
\pgfsetroundjoin%
\definecolor{currentfill}{rgb}{0.272594,0.025563,0.353093}%
\pgfsetfillcolor{currentfill}%
\pgfsetfillopacity{0.700000}%
\pgfsetlinewidth{0.000000pt}%
\definecolor{currentstroke}{rgb}{0.000000,0.000000,0.000000}%
\pgfsetstrokecolor{currentstroke}%
\pgfsetdash{}{0pt}%
\pgfpathmoveto{\pgfqpoint{4.355785in}{2.079149in}}%
\pgfpathlineto{\pgfqpoint{4.368987in}{2.076525in}}%
\pgfpathlineto{\pgfqpoint{4.382196in}{2.073927in}}%
\pgfpathlineto{\pgfqpoint{4.395411in}{2.071354in}}%
\pgfpathlineto{\pgfqpoint{4.408633in}{2.068806in}}%
\pgfpathlineto{\pgfqpoint{4.401162in}{2.060488in}}%
\pgfpathlineto{\pgfqpoint{4.393686in}{2.052158in}}%
\pgfpathlineto{\pgfqpoint{4.386204in}{2.043817in}}%
\pgfpathlineto{\pgfqpoint{4.378717in}{2.035468in}}%
\pgfpathlineto{\pgfqpoint{4.365484in}{2.038121in}}%
\pgfpathlineto{\pgfqpoint{4.352258in}{2.040798in}}%
\pgfpathlineto{\pgfqpoint{4.339038in}{2.043501in}}%
\pgfpathlineto{\pgfqpoint{4.325825in}{2.046230in}}%
\pgfpathlineto{\pgfqpoint{4.333323in}{2.054469in}}%
\pgfpathlineto{\pgfqpoint{4.340816in}{2.062704in}}%
\pgfpathlineto{\pgfqpoint{4.348303in}{2.070931in}}%
\pgfpathlineto{\pgfqpoint{4.355785in}{2.079149in}}%
\pgfpathclose%
\pgfusepath{fill}%
\end{pgfscope}%
\begin{pgfscope}%
\pgfpathrectangle{\pgfqpoint{1.254980in}{0.150000in}}{\pgfqpoint{5.490039in}{5.490039in}}%
\pgfusepath{clip}%
\pgfsetbuttcap%
\pgfsetroundjoin%
\definecolor{currentfill}{rgb}{0.281924,0.089666,0.412415}%
\pgfsetfillcolor{currentfill}%
\pgfsetfillopacity{0.700000}%
\pgfsetlinewidth{0.000000pt}%
\definecolor{currentstroke}{rgb}{0.000000,0.000000,0.000000}%
\pgfsetstrokecolor{currentstroke}%
\pgfsetdash{}{0pt}%
\pgfpathmoveto{\pgfqpoint{3.054882in}{2.177625in}}%
\pgfpathlineto{\pgfqpoint{3.067838in}{2.171130in}}%
\pgfpathlineto{\pgfqpoint{3.080797in}{2.164671in}}%
\pgfpathlineto{\pgfqpoint{3.093760in}{2.158249in}}%
\pgfpathlineto{\pgfqpoint{3.106726in}{2.151862in}}%
\pgfpathlineto{\pgfqpoint{3.098671in}{2.150020in}}%
\pgfpathlineto{\pgfqpoint{3.090603in}{2.148389in}}%
\pgfpathlineto{\pgfqpoint{3.082523in}{2.146973in}}%
\pgfpathlineto{\pgfqpoint{3.074431in}{2.145780in}}%
\pgfpathlineto{\pgfqpoint{3.061440in}{2.152401in}}%
\pgfpathlineto{\pgfqpoint{3.048453in}{2.159058in}}%
\pgfpathlineto{\pgfqpoint{3.035469in}{2.165751in}}%
\pgfpathlineto{\pgfqpoint{3.022489in}{2.172480in}}%
\pgfpathlineto{\pgfqpoint{3.030606in}{2.173434in}}%
\pgfpathlineto{\pgfqpoint{3.038711in}{2.174614in}}%
\pgfpathlineto{\pgfqpoint{3.046802in}{2.176013in}}%
\pgfpathlineto{\pgfqpoint{3.054882in}{2.177625in}}%
\pgfpathclose%
\pgfusepath{fill}%
\end{pgfscope}%
\begin{pgfscope}%
\pgfpathrectangle{\pgfqpoint{1.254980in}{0.150000in}}{\pgfqpoint{5.490039in}{5.490039in}}%
\pgfusepath{clip}%
\pgfsetbuttcap%
\pgfsetroundjoin%
\definecolor{currentfill}{rgb}{0.283072,0.130895,0.449241}%
\pgfsetfillcolor{currentfill}%
\pgfsetfillopacity{0.700000}%
\pgfsetlinewidth{0.000000pt}%
\definecolor{currentstroke}{rgb}{0.000000,0.000000,0.000000}%
\pgfsetstrokecolor{currentstroke}%
\pgfsetdash{}{0pt}%
\pgfpathmoveto{\pgfqpoint{5.176003in}{2.252465in}}%
\pgfpathlineto{\pgfqpoint{5.189440in}{2.251295in}}%
\pgfpathlineto{\pgfqpoint{5.202884in}{2.250149in}}%
\pgfpathlineto{\pgfqpoint{5.216337in}{2.249027in}}%
\pgfpathlineto{\pgfqpoint{5.229797in}{2.247929in}}%
\pgfpathlineto{\pgfqpoint{5.222642in}{2.240740in}}%
\pgfpathlineto{\pgfqpoint{5.215479in}{2.233475in}}%
\pgfpathlineto{\pgfqpoint{5.208310in}{2.226135in}}%
\pgfpathlineto{\pgfqpoint{5.201134in}{2.218719in}}%
\pgfpathlineto{\pgfqpoint{5.187661in}{2.219818in}}%
\pgfpathlineto{\pgfqpoint{5.174196in}{2.220941in}}%
\pgfpathlineto{\pgfqpoint{5.160739in}{2.222088in}}%
\pgfpathlineto{\pgfqpoint{5.147290in}{2.223258in}}%
\pgfpathlineto{\pgfqpoint{5.154478in}{2.230669in}}%
\pgfpathlineto{\pgfqpoint{5.161660in}{2.238006in}}%
\pgfpathlineto{\pgfqpoint{5.168835in}{2.245272in}}%
\pgfpathlineto{\pgfqpoint{5.176003in}{2.252465in}}%
\pgfpathclose%
\pgfusepath{fill}%
\end{pgfscope}%
\begin{pgfscope}%
\pgfpathrectangle{\pgfqpoint{1.254980in}{0.150000in}}{\pgfqpoint{5.490039in}{5.490039in}}%
\pgfusepath{clip}%
\pgfsetbuttcap%
\pgfsetroundjoin%
\definecolor{currentfill}{rgb}{0.283072,0.130895,0.449241}%
\pgfsetfillcolor{currentfill}%
\pgfsetfillopacity{0.700000}%
\pgfsetlinewidth{0.000000pt}%
\definecolor{currentstroke}{rgb}{0.000000,0.000000,0.000000}%
\pgfsetstrokecolor{currentstroke}%
\pgfsetdash{}{0pt}%
\pgfpathmoveto{\pgfqpoint{2.866991in}{2.256184in}}%
\pgfpathlineto{\pgfqpoint{2.879932in}{2.248994in}}%
\pgfpathlineto{\pgfqpoint{2.892875in}{2.241844in}}%
\pgfpathlineto{\pgfqpoint{2.905822in}{2.234733in}}%
\pgfpathlineto{\pgfqpoint{2.918772in}{2.227662in}}%
\pgfpathlineto{\pgfqpoint{2.910589in}{2.227426in}}%
\pgfpathlineto{\pgfqpoint{2.902393in}{2.227434in}}%
\pgfpathlineto{\pgfqpoint{2.894182in}{2.227694in}}%
\pgfpathlineto{\pgfqpoint{2.885956in}{2.228211in}}%
\pgfpathlineto{\pgfqpoint{2.872979in}{2.235530in}}%
\pgfpathlineto{\pgfqpoint{2.860005in}{2.242889in}}%
\pgfpathlineto{\pgfqpoint{2.847034in}{2.250287in}}%
\pgfpathlineto{\pgfqpoint{2.834066in}{2.257726in}}%
\pgfpathlineto{\pgfqpoint{2.842320in}{2.256955in}}%
\pgfpathlineto{\pgfqpoint{2.850558in}{2.256446in}}%
\pgfpathlineto{\pgfqpoint{2.858782in}{2.256191in}}%
\pgfpathlineto{\pgfqpoint{2.866991in}{2.256184in}}%
\pgfpathclose%
\pgfusepath{fill}%
\end{pgfscope}%
\begin{pgfscope}%
\pgfpathrectangle{\pgfqpoint{1.254980in}{0.150000in}}{\pgfqpoint{5.490039in}{5.490039in}}%
\pgfusepath{clip}%
\pgfsetbuttcap%
\pgfsetroundjoin%
\definecolor{currentfill}{rgb}{0.282327,0.094955,0.417331}%
\pgfsetfillcolor{currentfill}%
\pgfsetfillopacity{0.700000}%
\pgfsetlinewidth{0.000000pt}%
\definecolor{currentstroke}{rgb}{0.000000,0.000000,0.000000}%
\pgfsetstrokecolor{currentstroke}%
\pgfsetdash{}{0pt}%
\pgfpathmoveto{\pgfqpoint{4.875050in}{2.183996in}}%
\pgfpathlineto{\pgfqpoint{4.888398in}{2.182398in}}%
\pgfpathlineto{\pgfqpoint{4.901754in}{2.180824in}}%
\pgfpathlineto{\pgfqpoint{4.915117in}{2.179275in}}%
\pgfpathlineto{\pgfqpoint{4.928487in}{2.177749in}}%
\pgfpathlineto{\pgfqpoint{4.921206in}{2.169784in}}%
\pgfpathlineto{\pgfqpoint{4.913917in}{2.161757in}}%
\pgfpathlineto{\pgfqpoint{4.906623in}{2.153667in}}%
\pgfpathlineto{\pgfqpoint{4.899323in}{2.145516in}}%
\pgfpathlineto{\pgfqpoint{4.885941in}{2.147081in}}%
\pgfpathlineto{\pgfqpoint{4.872567in}{2.148671in}}%
\pgfpathlineto{\pgfqpoint{4.859200in}{2.150285in}}%
\pgfpathlineto{\pgfqpoint{4.845841in}{2.151923in}}%
\pgfpathlineto{\pgfqpoint{4.853153in}{2.160029in}}%
\pgfpathlineto{\pgfqpoint{4.860458in}{2.168077in}}%
\pgfpathlineto{\pgfqpoint{4.867757in}{2.176066in}}%
\pgfpathlineto{\pgfqpoint{4.875050in}{2.183996in}}%
\pgfpathclose%
\pgfusepath{fill}%
\end{pgfscope}%
\begin{pgfscope}%
\pgfpathrectangle{\pgfqpoint{1.254980in}{0.150000in}}{\pgfqpoint{5.490039in}{5.490039in}}%
\pgfusepath{clip}%
\pgfsetbuttcap%
\pgfsetroundjoin%
\definecolor{currentfill}{rgb}{0.267004,0.004874,0.329415}%
\pgfsetfillcolor{currentfill}%
\pgfsetfillopacity{0.700000}%
\pgfsetlinewidth{0.000000pt}%
\definecolor{currentstroke}{rgb}{0.000000,0.000000,0.000000}%
\pgfsetstrokecolor{currentstroke}%
\pgfsetdash{}{0pt}%
\pgfpathmoveto{\pgfqpoint{3.919397in}{2.030607in}}%
\pgfpathlineto{\pgfqpoint{3.932495in}{2.026857in}}%
\pgfpathlineto{\pgfqpoint{3.945600in}{2.023134in}}%
\pgfpathlineto{\pgfqpoint{3.958710in}{2.019438in}}%
\pgfpathlineto{\pgfqpoint{3.971826in}{2.015770in}}%
\pgfpathlineto{\pgfqpoint{3.964197in}{2.008475in}}%
\pgfpathlineto{\pgfqpoint{3.956562in}{2.001235in}}%
\pgfpathlineto{\pgfqpoint{3.948921in}{1.994051in}}%
\pgfpathlineto{\pgfqpoint{3.941274in}{1.986928in}}%
\pgfpathlineto{\pgfqpoint{3.928144in}{1.990753in}}%
\pgfpathlineto{\pgfqpoint{3.915020in}{1.994604in}}%
\pgfpathlineto{\pgfqpoint{3.901902in}{1.998482in}}%
\pgfpathlineto{\pgfqpoint{3.888790in}{2.002388in}}%
\pgfpathlineto{\pgfqpoint{3.896451in}{2.009350in}}%
\pgfpathlineto{\pgfqpoint{3.904105in}{2.016377in}}%
\pgfpathlineto{\pgfqpoint{3.911754in}{2.023463in}}%
\pgfpathlineto{\pgfqpoint{3.919397in}{2.030607in}}%
\pgfpathclose%
\pgfusepath{fill}%
\end{pgfscope}%
\begin{pgfscope}%
\pgfpathrectangle{\pgfqpoint{1.254980in}{0.150000in}}{\pgfqpoint{5.490039in}{5.490039in}}%
\pgfusepath{clip}%
\pgfsetbuttcap%
\pgfsetroundjoin%
\definecolor{currentfill}{rgb}{0.274128,0.199721,0.498911}%
\pgfsetfillcolor{currentfill}%
\pgfsetfillopacity{0.700000}%
\pgfsetlinewidth{0.000000pt}%
\definecolor{currentstroke}{rgb}{0.000000,0.000000,0.000000}%
\pgfsetstrokecolor{currentstroke}%
\pgfsetdash{}{0pt}%
\pgfpathmoveto{\pgfqpoint{5.913593in}{2.387008in}}%
\pgfpathlineto{\pgfqpoint{5.927255in}{2.386418in}}%
\pgfpathlineto{\pgfqpoint{5.940925in}{2.385852in}}%
\pgfpathlineto{\pgfqpoint{5.954604in}{2.385308in}}%
\pgfpathlineto{\pgfqpoint{5.947811in}{2.380570in}}%
\pgfpathlineto{\pgfqpoint{5.941011in}{2.375777in}}%
\pgfpathlineto{\pgfqpoint{5.934203in}{2.370926in}}%
\pgfpathlineto{\pgfqpoint{5.927387in}{2.366014in}}%
\pgfpathlineto{\pgfqpoint{5.913689in}{2.366465in}}%
\pgfpathlineto{\pgfqpoint{5.900000in}{2.366940in}}%
\pgfpathlineto{\pgfqpoint{5.886319in}{2.367437in}}%
\pgfpathlineto{\pgfqpoint{5.893149in}{2.372415in}}%
\pgfpathlineto{\pgfqpoint{5.899971in}{2.377334in}}%
\pgfpathlineto{\pgfqpoint{5.906786in}{2.382197in}}%
\pgfpathlineto{\pgfqpoint{5.913593in}{2.387008in}}%
\pgfpathclose%
\pgfusepath{fill}%
\end{pgfscope}%
\begin{pgfscope}%
\pgfpathrectangle{\pgfqpoint{1.254980in}{0.150000in}}{\pgfqpoint{5.490039in}{5.490039in}}%
\pgfusepath{clip}%
\pgfsetbuttcap%
\pgfsetroundjoin%
\definecolor{currentfill}{rgb}{0.277134,0.185228,0.489898}%
\pgfsetfillcolor{currentfill}%
\pgfsetfillopacity{0.700000}%
\pgfsetlinewidth{0.000000pt}%
\definecolor{currentstroke}{rgb}{0.000000,0.000000,0.000000}%
\pgfsetstrokecolor{currentstroke}%
\pgfsetdash{}{0pt}%
\pgfpathmoveto{\pgfqpoint{5.695222in}{2.353739in}}%
\pgfpathlineto{\pgfqpoint{5.708816in}{2.353039in}}%
\pgfpathlineto{\pgfqpoint{5.722419in}{2.352362in}}%
\pgfpathlineto{\pgfqpoint{5.736031in}{2.351708in}}%
\pgfpathlineto{\pgfqpoint{5.749650in}{2.351079in}}%
\pgfpathlineto{\pgfqpoint{5.742746in}{2.345625in}}%
\pgfpathlineto{\pgfqpoint{5.735835in}{2.340104in}}%
\pgfpathlineto{\pgfqpoint{5.728915in}{2.334512in}}%
\pgfpathlineto{\pgfqpoint{5.721988in}{2.328848in}}%
\pgfpathlineto{\pgfqpoint{5.708351in}{2.329412in}}%
\pgfpathlineto{\pgfqpoint{5.694723in}{2.329999in}}%
\pgfpathlineto{\pgfqpoint{5.681104in}{2.330611in}}%
\pgfpathlineto{\pgfqpoint{5.667493in}{2.331246in}}%
\pgfpathlineto{\pgfqpoint{5.674437in}{2.336971in}}%
\pgfpathlineto{\pgfqpoint{5.681373in}{2.342626in}}%
\pgfpathlineto{\pgfqpoint{5.688301in}{2.348215in}}%
\pgfpathlineto{\pgfqpoint{5.695222in}{2.353739in}}%
\pgfpathclose%
\pgfusepath{fill}%
\end{pgfscope}%
\begin{pgfscope}%
\pgfpathrectangle{\pgfqpoint{1.254980in}{0.150000in}}{\pgfqpoint{5.490039in}{5.490039in}}%
\pgfusepath{clip}%
\pgfsetbuttcap%
\pgfsetroundjoin%
\definecolor{currentfill}{rgb}{0.277941,0.056324,0.381191}%
\pgfsetfillcolor{currentfill}%
\pgfsetfillopacity{0.700000}%
\pgfsetlinewidth{0.000000pt}%
\definecolor{currentstroke}{rgb}{0.000000,0.000000,0.000000}%
\pgfsetstrokecolor{currentstroke}%
\pgfsetdash{}{0pt}%
\pgfpathmoveto{\pgfqpoint{4.574059in}{2.116530in}}%
\pgfpathlineto{\pgfqpoint{4.587322in}{2.114380in}}%
\pgfpathlineto{\pgfqpoint{4.600593in}{2.112253in}}%
\pgfpathlineto{\pgfqpoint{4.613870in}{2.110152in}}%
\pgfpathlineto{\pgfqpoint{4.627154in}{2.108076in}}%
\pgfpathlineto{\pgfqpoint{4.619759in}{2.099710in}}%
\pgfpathlineto{\pgfqpoint{4.612358in}{2.091307in}}%
\pgfpathlineto{\pgfqpoint{4.604951in}{2.082868in}}%
\pgfpathlineto{\pgfqpoint{4.597539in}{2.074394in}}%
\pgfpathlineto{\pgfqpoint{4.584244in}{2.076550in}}%
\pgfpathlineto{\pgfqpoint{4.570956in}{2.078730in}}%
\pgfpathlineto{\pgfqpoint{4.557675in}{2.080935in}}%
\pgfpathlineto{\pgfqpoint{4.544402in}{2.083165in}}%
\pgfpathlineto{\pgfqpoint{4.551824in}{2.091554in}}%
\pgfpathlineto{\pgfqpoint{4.559241in}{2.099913in}}%
\pgfpathlineto{\pgfqpoint{4.566653in}{2.108239in}}%
\pgfpathlineto{\pgfqpoint{4.574059in}{2.116530in}}%
\pgfpathclose%
\pgfusepath{fill}%
\end{pgfscope}%
\begin{pgfscope}%
\pgfpathrectangle{\pgfqpoint{1.254980in}{0.150000in}}{\pgfqpoint{5.490039in}{5.490039in}}%
\pgfusepath{clip}%
\pgfsetbuttcap%
\pgfsetroundjoin%
\definecolor{currentfill}{rgb}{0.277941,0.056324,0.381191}%
\pgfsetfillcolor{currentfill}%
\pgfsetfillopacity{0.700000}%
\pgfsetlinewidth{0.000000pt}%
\definecolor{currentstroke}{rgb}{0.000000,0.000000,0.000000}%
\pgfsetstrokecolor{currentstroke}%
\pgfsetdash{}{0pt}%
\pgfpathmoveto{\pgfqpoint{3.242529in}{2.113141in}}%
\pgfpathlineto{\pgfqpoint{3.255509in}{2.107286in}}%
\pgfpathlineto{\pgfqpoint{3.268493in}{2.101465in}}%
\pgfpathlineto{\pgfqpoint{3.281482in}{2.095676in}}%
\pgfpathlineto{\pgfqpoint{3.294475in}{2.089921in}}%
\pgfpathlineto{\pgfqpoint{3.286530in}{2.086650in}}%
\pgfpathlineto{\pgfqpoint{3.278574in}{2.083557in}}%
\pgfpathlineto{\pgfqpoint{3.270608in}{2.080647in}}%
\pgfpathlineto{\pgfqpoint{3.262632in}{2.077925in}}%
\pgfpathlineto{\pgfqpoint{3.249617in}{2.083901in}}%
\pgfpathlineto{\pgfqpoint{3.236607in}{2.089910in}}%
\pgfpathlineto{\pgfqpoint{3.223601in}{2.095952in}}%
\pgfpathlineto{\pgfqpoint{3.210599in}{2.102027in}}%
\pgfpathlineto{\pgfqpoint{3.218597in}{2.104524in}}%
\pgfpathlineto{\pgfqpoint{3.226585in}{2.107212in}}%
\pgfpathlineto{\pgfqpoint{3.234562in}{2.110087in}}%
\pgfpathlineto{\pgfqpoint{3.242529in}{2.113141in}}%
\pgfpathclose%
\pgfusepath{fill}%
\end{pgfscope}%
\begin{pgfscope}%
\pgfpathrectangle{\pgfqpoint{1.254980in}{0.150000in}}{\pgfqpoint{5.490039in}{5.490039in}}%
\pgfusepath{clip}%
\pgfsetbuttcap%
\pgfsetroundjoin%
\definecolor{currentfill}{rgb}{0.281412,0.155834,0.469201}%
\pgfsetfillcolor{currentfill}%
\pgfsetfillopacity{0.700000}%
\pgfsetlinewidth{0.000000pt}%
\definecolor{currentstroke}{rgb}{0.000000,0.000000,0.000000}%
\pgfsetstrokecolor{currentstroke}%
\pgfsetdash{}{0pt}%
\pgfpathmoveto{\pgfqpoint{5.394551in}{2.294612in}}%
\pgfpathlineto{\pgfqpoint{5.408057in}{2.293690in}}%
\pgfpathlineto{\pgfqpoint{5.421572in}{2.292792in}}%
\pgfpathlineto{\pgfqpoint{5.435096in}{2.291918in}}%
\pgfpathlineto{\pgfqpoint{5.448627in}{2.291067in}}%
\pgfpathlineto{\pgfqpoint{5.441570in}{2.284538in}}%
\pgfpathlineto{\pgfqpoint{5.434505in}{2.277933in}}%
\pgfpathlineto{\pgfqpoint{5.427432in}{2.271248in}}%
\pgfpathlineto{\pgfqpoint{5.420352in}{2.264484in}}%
\pgfpathlineto{\pgfqpoint{5.406807in}{2.265309in}}%
\pgfpathlineto{\pgfqpoint{5.393270in}{2.266158in}}%
\pgfpathlineto{\pgfqpoint{5.379741in}{2.267030in}}%
\pgfpathlineto{\pgfqpoint{5.366220in}{2.267926in}}%
\pgfpathlineto{\pgfqpoint{5.373314in}{2.274711in}}%
\pgfpathlineto{\pgfqpoint{5.380400in}{2.281419in}}%
\pgfpathlineto{\pgfqpoint{5.387479in}{2.288053in}}%
\pgfpathlineto{\pgfqpoint{5.394551in}{2.294612in}}%
\pgfpathclose%
\pgfusepath{fill}%
\end{pgfscope}%
\begin{pgfscope}%
\pgfpathrectangle{\pgfqpoint{1.254980in}{0.150000in}}{\pgfqpoint{5.490039in}{5.490039in}}%
\pgfusepath{clip}%
\pgfsetbuttcap%
\pgfsetroundjoin%
\definecolor{currentfill}{rgb}{0.271305,0.019942,0.347269}%
\pgfsetfillcolor{currentfill}%
\pgfsetfillopacity{0.700000}%
\pgfsetlinewidth{0.000000pt}%
\definecolor{currentstroke}{rgb}{0.000000,0.000000,0.000000}%
\pgfsetstrokecolor{currentstroke}%
\pgfsetdash{}{0pt}%
\pgfpathmoveto{\pgfqpoint{4.273041in}{2.057399in}}%
\pgfpathlineto{\pgfqpoint{4.286227in}{2.054568in}}%
\pgfpathlineto{\pgfqpoint{4.299420in}{2.051763in}}%
\pgfpathlineto{\pgfqpoint{4.312619in}{2.048984in}}%
\pgfpathlineto{\pgfqpoint{4.325825in}{2.046230in}}%
\pgfpathlineto{\pgfqpoint{4.318322in}{2.037988in}}%
\pgfpathlineto{\pgfqpoint{4.310813in}{2.029746in}}%
\pgfpathlineto{\pgfqpoint{4.303299in}{2.021506in}}%
\pgfpathlineto{\pgfqpoint{4.295780in}{2.013272in}}%
\pgfpathlineto{\pgfqpoint{4.282563in}{2.016144in}}%
\pgfpathlineto{\pgfqpoint{4.269352in}{2.019041in}}%
\pgfpathlineto{\pgfqpoint{4.256148in}{2.021963in}}%
\pgfpathlineto{\pgfqpoint{4.242950in}{2.024911in}}%
\pgfpathlineto{\pgfqpoint{4.250481in}{2.033023in}}%
\pgfpathlineto{\pgfqpoint{4.258006in}{2.041144in}}%
\pgfpathlineto{\pgfqpoint{4.265526in}{2.049270in}}%
\pgfpathlineto{\pgfqpoint{4.273041in}{2.057399in}}%
\pgfpathclose%
\pgfusepath{fill}%
\end{pgfscope}%
\begin{pgfscope}%
\pgfpathrectangle{\pgfqpoint{1.254980in}{0.150000in}}{\pgfqpoint{5.490039in}{5.490039in}}%
\pgfusepath{clip}%
\pgfsetbuttcap%
\pgfsetroundjoin%
\definecolor{currentfill}{rgb}{0.283229,0.120777,0.440584}%
\pgfsetfillcolor{currentfill}%
\pgfsetfillopacity{0.700000}%
\pgfsetlinewidth{0.000000pt}%
\definecolor{currentstroke}{rgb}{0.000000,0.000000,0.000000}%
\pgfsetstrokecolor{currentstroke}%
\pgfsetdash{}{0pt}%
\pgfpathmoveto{\pgfqpoint{5.093573in}{2.228181in}}%
\pgfpathlineto{\pgfqpoint{5.106991in}{2.226915in}}%
\pgfpathlineto{\pgfqpoint{5.120416in}{2.225672in}}%
\pgfpathlineto{\pgfqpoint{5.133849in}{2.224453in}}%
\pgfpathlineto{\pgfqpoint{5.147290in}{2.223258in}}%
\pgfpathlineto{\pgfqpoint{5.140095in}{2.215775in}}%
\pgfpathlineto{\pgfqpoint{5.132893in}{2.208219in}}%
\pgfpathlineto{\pgfqpoint{5.125684in}{2.200588in}}%
\pgfpathlineto{\pgfqpoint{5.118468in}{2.192885in}}%
\pgfpathlineto{\pgfqpoint{5.105016in}{2.194093in}}%
\pgfpathlineto{\pgfqpoint{5.091571in}{2.195326in}}%
\pgfpathlineto{\pgfqpoint{5.078134in}{2.196582in}}%
\pgfpathlineto{\pgfqpoint{5.064705in}{2.197863in}}%
\pgfpathlineto{\pgfqpoint{5.071932in}{2.205548in}}%
\pgfpathlineto{\pgfqpoint{5.079152in}{2.213162in}}%
\pgfpathlineto{\pgfqpoint{5.086366in}{2.220707in}}%
\pgfpathlineto{\pgfqpoint{5.093573in}{2.228181in}}%
\pgfpathclose%
\pgfusepath{fill}%
\end{pgfscope}%
\begin{pgfscope}%
\pgfpathrectangle{\pgfqpoint{1.254980in}{0.150000in}}{\pgfqpoint{5.490039in}{5.490039in}}%
\pgfusepath{clip}%
\pgfsetbuttcap%
\pgfsetroundjoin%
\definecolor{currentfill}{rgb}{0.267004,0.004874,0.329415}%
\pgfsetfillcolor{currentfill}%
\pgfsetfillopacity{0.700000}%
\pgfsetlinewidth{0.000000pt}%
\definecolor{currentstroke}{rgb}{0.000000,0.000000,0.000000}%
\pgfsetstrokecolor{currentstroke}%
\pgfsetdash{}{0pt}%
\pgfpathmoveto{\pgfqpoint{4.054759in}{2.031575in}}%
\pgfpathlineto{\pgfqpoint{4.067893in}{2.028183in}}%
\pgfpathlineto{\pgfqpoint{4.081033in}{2.024818in}}%
\pgfpathlineto{\pgfqpoint{4.094179in}{2.021479in}}%
\pgfpathlineto{\pgfqpoint{4.107331in}{2.018167in}}%
\pgfpathlineto{\pgfqpoint{4.099750in}{2.010418in}}%
\pgfpathlineto{\pgfqpoint{4.092163in}{2.002702in}}%
\pgfpathlineto{\pgfqpoint{4.084570in}{1.995023in}}%
\pgfpathlineto{\pgfqpoint{4.076972in}{1.987384in}}%
\pgfpathlineto{\pgfqpoint{4.063808in}{1.990839in}}%
\pgfpathlineto{\pgfqpoint{4.050649in}{1.994321in}}%
\pgfpathlineto{\pgfqpoint{4.037497in}{1.997829in}}%
\pgfpathlineto{\pgfqpoint{4.024351in}{2.001364in}}%
\pgfpathlineto{\pgfqpoint{4.031962in}{2.008856in}}%
\pgfpathlineto{\pgfqpoint{4.039566in}{2.016390in}}%
\pgfpathlineto{\pgfqpoint{4.047166in}{2.023965in}}%
\pgfpathlineto{\pgfqpoint{4.054759in}{2.031575in}}%
\pgfpathclose%
\pgfusepath{fill}%
\end{pgfscope}%
\begin{pgfscope}%
\pgfpathrectangle{\pgfqpoint{1.254980in}{0.150000in}}{\pgfqpoint{5.490039in}{5.490039in}}%
\pgfusepath{clip}%
\pgfsetbuttcap%
\pgfsetroundjoin%
\definecolor{currentfill}{rgb}{0.277134,0.185228,0.489898}%
\pgfsetfillcolor{currentfill}%
\pgfsetfillopacity{0.700000}%
\pgfsetlinewidth{0.000000pt}%
\definecolor{currentstroke}{rgb}{0.000000,0.000000,0.000000}%
\pgfsetstrokecolor{currentstroke}%
\pgfsetdash{}{0pt}%
\pgfpathmoveto{\pgfqpoint{2.678662in}{2.350270in}}%
\pgfpathlineto{\pgfqpoint{2.691598in}{2.342318in}}%
\pgfpathlineto{\pgfqpoint{2.704537in}{2.334411in}}%
\pgfpathlineto{\pgfqpoint{2.717478in}{2.326549in}}%
\pgfpathlineto{\pgfqpoint{2.730422in}{2.318731in}}%
\pgfpathlineto{\pgfqpoint{2.722093in}{2.320285in}}%
\pgfpathlineto{\pgfqpoint{2.713749in}{2.322121in}}%
\pgfpathlineto{\pgfqpoint{2.705387in}{2.324245in}}%
\pgfpathlineto{\pgfqpoint{2.697008in}{2.326666in}}%
\pgfpathlineto{\pgfqpoint{2.684034in}{2.334747in}}%
\pgfpathlineto{\pgfqpoint{2.671062in}{2.342872in}}%
\pgfpathlineto{\pgfqpoint{2.658092in}{2.351043in}}%
\pgfpathlineto{\pgfqpoint{2.645124in}{2.359258in}}%
\pgfpathlineto{\pgfqpoint{2.653535in}{2.356569in}}%
\pgfpathlineto{\pgfqpoint{2.661928in}{2.354179in}}%
\pgfpathlineto{\pgfqpoint{2.670304in}{2.352082in}}%
\pgfpathlineto{\pgfqpoint{2.678662in}{2.350270in}}%
\pgfpathclose%
\pgfusepath{fill}%
\end{pgfscope}%
\begin{pgfscope}%
\pgfpathrectangle{\pgfqpoint{1.254980in}{0.150000in}}{\pgfqpoint{5.490039in}{5.490039in}}%
\pgfusepath{clip}%
\pgfsetbuttcap%
\pgfsetroundjoin%
\definecolor{currentfill}{rgb}{0.281446,0.084320,0.407414}%
\pgfsetfillcolor{currentfill}%
\pgfsetfillopacity{0.700000}%
\pgfsetlinewidth{0.000000pt}%
\definecolor{currentstroke}{rgb}{0.000000,0.000000,0.000000}%
\pgfsetstrokecolor{currentstroke}%
\pgfsetdash{}{0pt}%
\pgfpathmoveto{\pgfqpoint{4.792481in}{2.158719in}}%
\pgfpathlineto{\pgfqpoint{4.805810in}{2.156983in}}%
\pgfpathlineto{\pgfqpoint{4.819146in}{2.155272in}}%
\pgfpathlineto{\pgfqpoint{4.832490in}{2.153585in}}%
\pgfpathlineto{\pgfqpoint{4.845841in}{2.151923in}}%
\pgfpathlineto{\pgfqpoint{4.838524in}{2.143759in}}%
\pgfpathlineto{\pgfqpoint{4.831201in}{2.135539in}}%
\pgfpathlineto{\pgfqpoint{4.823871in}{2.127262in}}%
\pgfpathlineto{\pgfqpoint{4.816536in}{2.118931in}}%
\pgfpathlineto{\pgfqpoint{4.803174in}{2.120647in}}%
\pgfpathlineto{\pgfqpoint{4.789820in}{2.122387in}}%
\pgfpathlineto{\pgfqpoint{4.776472in}{2.124151in}}%
\pgfpathlineto{\pgfqpoint{4.763133in}{2.125940in}}%
\pgfpathlineto{\pgfqpoint{4.770479in}{2.134213in}}%
\pgfpathlineto{\pgfqpoint{4.777819in}{2.142434in}}%
\pgfpathlineto{\pgfqpoint{4.785153in}{2.150603in}}%
\pgfpathlineto{\pgfqpoint{4.792481in}{2.158719in}}%
\pgfpathclose%
\pgfusepath{fill}%
\end{pgfscope}%
\begin{pgfscope}%
\pgfpathrectangle{\pgfqpoint{1.254980in}{0.150000in}}{\pgfqpoint{5.490039in}{5.490039in}}%
\pgfusepath{clip}%
\pgfsetbuttcap%
\pgfsetroundjoin%
\definecolor{currentfill}{rgb}{0.276022,0.044167,0.370164}%
\pgfsetfillcolor{currentfill}%
\pgfsetfillopacity{0.700000}%
\pgfsetlinewidth{0.000000pt}%
\definecolor{currentstroke}{rgb}{0.000000,0.000000,0.000000}%
\pgfsetstrokecolor{currentstroke}%
\pgfsetdash{}{0pt}%
\pgfpathmoveto{\pgfqpoint{4.491377in}{2.092334in}}%
\pgfpathlineto{\pgfqpoint{4.504623in}{2.090004in}}%
\pgfpathlineto{\pgfqpoint{4.517875in}{2.087699in}}%
\pgfpathlineto{\pgfqpoint{4.531135in}{2.085420in}}%
\pgfpathlineto{\pgfqpoint{4.544402in}{2.083165in}}%
\pgfpathlineto{\pgfqpoint{4.536973in}{2.074746in}}%
\pgfpathlineto{\pgfqpoint{4.529540in}{2.066299in}}%
\pgfpathlineto{\pgfqpoint{4.522101in}{2.057827in}}%
\pgfpathlineto{\pgfqpoint{4.514656in}{2.049331in}}%
\pgfpathlineto{\pgfqpoint{4.501379in}{2.051678in}}%
\pgfpathlineto{\pgfqpoint{4.488109in}{2.054050in}}%
\pgfpathlineto{\pgfqpoint{4.474846in}{2.056446in}}%
\pgfpathlineto{\pgfqpoint{4.461589in}{2.058868in}}%
\pgfpathlineto{\pgfqpoint{4.469044in}{2.067267in}}%
\pgfpathlineto{\pgfqpoint{4.476494in}{2.075646in}}%
\pgfpathlineto{\pgfqpoint{4.483938in}{2.084002in}}%
\pgfpathlineto{\pgfqpoint{4.491377in}{2.092334in}}%
\pgfpathclose%
\pgfusepath{fill}%
\end{pgfscope}%
\begin{pgfscope}%
\pgfpathrectangle{\pgfqpoint{1.254980in}{0.150000in}}{\pgfqpoint{5.490039in}{5.490039in}}%
\pgfusepath{clip}%
\pgfsetbuttcap%
\pgfsetroundjoin%
\definecolor{currentfill}{rgb}{0.269944,0.014625,0.341379}%
\pgfsetfillcolor{currentfill}%
\pgfsetfillopacity{0.700000}%
\pgfsetlinewidth{0.000000pt}%
\definecolor{currentstroke}{rgb}{0.000000,0.000000,0.000000}%
\pgfsetstrokecolor{currentstroke}%
\pgfsetdash{}{0pt}%
\pgfpathmoveto{\pgfqpoint{3.565592in}{2.040470in}}%
\pgfpathlineto{\pgfqpoint{3.578627in}{2.035646in}}%
\pgfpathlineto{\pgfqpoint{3.591666in}{2.030852in}}%
\pgfpathlineto{\pgfqpoint{3.604711in}{2.026088in}}%
\pgfpathlineto{\pgfqpoint{3.617761in}{2.021352in}}%
\pgfpathlineto{\pgfqpoint{3.609979in}{2.015909in}}%
\pgfpathlineto{\pgfqpoint{3.602189in}{2.010585in}}%
\pgfpathlineto{\pgfqpoint{3.594391in}{2.005385in}}%
\pgfpathlineto{\pgfqpoint{3.586586in}{2.000315in}}%
\pgfpathlineto{\pgfqpoint{3.573519in}{2.005244in}}%
\pgfpathlineto{\pgfqpoint{3.560457in}{2.010203in}}%
\pgfpathlineto{\pgfqpoint{3.547399in}{2.015191in}}%
\pgfpathlineto{\pgfqpoint{3.534347in}{2.020209in}}%
\pgfpathlineto{\pgfqpoint{3.542170in}{2.025080in}}%
\pgfpathlineto{\pgfqpoint{3.549985in}{2.030084in}}%
\pgfpathlineto{\pgfqpoint{3.557793in}{2.035215in}}%
\pgfpathlineto{\pgfqpoint{3.565592in}{2.040470in}}%
\pgfpathclose%
\pgfusepath{fill}%
\end{pgfscope}%
\begin{pgfscope}%
\pgfpathrectangle{\pgfqpoint{1.254980in}{0.150000in}}{\pgfqpoint{5.490039in}{5.490039in}}%
\pgfusepath{clip}%
\pgfsetbuttcap%
\pgfsetroundjoin%
\definecolor{currentfill}{rgb}{0.267004,0.004874,0.329415}%
\pgfsetfillcolor{currentfill}%
\pgfsetfillopacity{0.700000}%
\pgfsetlinewidth{0.000000pt}%
\definecolor{currentstroke}{rgb}{0.000000,0.000000,0.000000}%
\pgfsetstrokecolor{currentstroke}%
\pgfsetdash{}{0pt}%
\pgfpathmoveto{\pgfqpoint{3.701004in}{2.026308in}}%
\pgfpathlineto{\pgfqpoint{3.714064in}{2.021899in}}%
\pgfpathlineto{\pgfqpoint{3.727130in}{2.017518in}}%
\pgfpathlineto{\pgfqpoint{3.740201in}{2.013166in}}%
\pgfpathlineto{\pgfqpoint{3.753277in}{2.008842in}}%
\pgfpathlineto{\pgfqpoint{3.745555in}{2.002613in}}%
\pgfpathlineto{\pgfqpoint{3.737826in}{1.996478in}}%
\pgfpathlineto{\pgfqpoint{3.730091in}{1.990444in}}%
\pgfpathlineto{\pgfqpoint{3.722348in}{1.984513in}}%
\pgfpathlineto{\pgfqpoint{3.709256in}{1.989018in}}%
\pgfpathlineto{\pgfqpoint{3.696170in}{1.993551in}}%
\pgfpathlineto{\pgfqpoint{3.683089in}{1.998113in}}%
\pgfpathlineto{\pgfqpoint{3.670013in}{2.002703in}}%
\pgfpathlineto{\pgfqpoint{3.677771in}{2.008448in}}%
\pgfpathlineto{\pgfqpoint{3.685523in}{2.014300in}}%
\pgfpathlineto{\pgfqpoint{3.693267in}{2.020255in}}%
\pgfpathlineto{\pgfqpoint{3.701004in}{2.026308in}}%
\pgfpathclose%
\pgfusepath{fill}%
\end{pgfscope}%
\begin{pgfscope}%
\pgfpathrectangle{\pgfqpoint{1.254980in}{0.150000in}}{\pgfqpoint{5.490039in}{5.490039in}}%
\pgfusepath{clip}%
\pgfsetbuttcap%
\pgfsetroundjoin%
\definecolor{currentfill}{rgb}{0.278012,0.180367,0.486697}%
\pgfsetfillcolor{currentfill}%
\pgfsetfillopacity{0.700000}%
\pgfsetlinewidth{0.000000pt}%
\definecolor{currentstroke}{rgb}{0.000000,0.000000,0.000000}%
\pgfsetstrokecolor{currentstroke}%
\pgfsetdash{}{0pt}%
\pgfpathmoveto{\pgfqpoint{5.613135in}{2.334020in}}%
\pgfpathlineto{\pgfqpoint{5.626712in}{2.333291in}}%
\pgfpathlineto{\pgfqpoint{5.640297in}{2.332586in}}%
\pgfpathlineto{\pgfqpoint{5.653891in}{2.331904in}}%
\pgfpathlineto{\pgfqpoint{5.667493in}{2.331246in}}%
\pgfpathlineto{\pgfqpoint{5.660542in}{2.325449in}}%
\pgfpathlineto{\pgfqpoint{5.653582in}{2.319579in}}%
\pgfpathlineto{\pgfqpoint{5.646615in}{2.313633in}}%
\pgfpathlineto{\pgfqpoint{5.639640in}{2.307610in}}%
\pgfpathlineto{\pgfqpoint{5.626022in}{2.308216in}}%
\pgfpathlineto{\pgfqpoint{5.612413in}{2.308845in}}%
\pgfpathlineto{\pgfqpoint{5.598812in}{2.309498in}}%
\pgfpathlineto{\pgfqpoint{5.585220in}{2.310175in}}%
\pgfpathlineto{\pgfqpoint{5.592210in}{2.316246in}}%
\pgfpathlineto{\pgfqpoint{5.599193in}{2.322242in}}%
\pgfpathlineto{\pgfqpoint{5.606168in}{2.328166in}}%
\pgfpathlineto{\pgfqpoint{5.613135in}{2.334020in}}%
\pgfpathclose%
\pgfusepath{fill}%
\end{pgfscope}%
\begin{pgfscope}%
\pgfpathrectangle{\pgfqpoint{1.254980in}{0.150000in}}{\pgfqpoint{5.490039in}{5.490039in}}%
\pgfusepath{clip}%
\pgfsetbuttcap%
\pgfsetroundjoin%
\definecolor{currentfill}{rgb}{0.272594,0.025563,0.353093}%
\pgfsetfillcolor{currentfill}%
\pgfsetfillopacity{0.700000}%
\pgfsetlinewidth{0.000000pt}%
\definecolor{currentstroke}{rgb}{0.000000,0.000000,0.000000}%
\pgfsetstrokecolor{currentstroke}%
\pgfsetdash{}{0pt}%
\pgfpathmoveto{\pgfqpoint{3.430104in}{2.061438in}}%
\pgfpathlineto{\pgfqpoint{3.443118in}{2.056177in}}%
\pgfpathlineto{\pgfqpoint{3.456136in}{2.050947in}}%
\pgfpathlineto{\pgfqpoint{3.469159in}{2.045748in}}%
\pgfpathlineto{\pgfqpoint{3.482187in}{2.040580in}}%
\pgfpathlineto{\pgfqpoint{3.474338in}{2.036048in}}%
\pgfpathlineto{\pgfqpoint{3.466479in}{2.031663in}}%
\pgfpathlineto{\pgfqpoint{3.458612in}{2.027429in}}%
\pgfpathlineto{\pgfqpoint{3.450737in}{2.023351in}}%
\pgfpathlineto{\pgfqpoint{3.437690in}{2.028727in}}%
\pgfpathlineto{\pgfqpoint{3.424648in}{2.034133in}}%
\pgfpathlineto{\pgfqpoint{3.411610in}{2.039570in}}%
\pgfpathlineto{\pgfqpoint{3.398577in}{2.045038in}}%
\pgfpathlineto{\pgfqpoint{3.406472in}{2.048904in}}%
\pgfpathlineto{\pgfqpoint{3.414359in}{2.052929in}}%
\pgfpathlineto{\pgfqpoint{3.422236in}{2.057109in}}%
\pgfpathlineto{\pgfqpoint{3.430104in}{2.061438in}}%
\pgfpathclose%
\pgfusepath{fill}%
\end{pgfscope}%
\begin{pgfscope}%
\pgfpathrectangle{\pgfqpoint{1.254980in}{0.150000in}}{\pgfqpoint{5.490039in}{5.490039in}}%
\pgfusepath{clip}%
\pgfsetbuttcap%
\pgfsetroundjoin%
\definecolor{currentfill}{rgb}{0.281446,0.084320,0.407414}%
\pgfsetfillcolor{currentfill}%
\pgfsetfillopacity{0.700000}%
\pgfsetlinewidth{0.000000pt}%
\definecolor{currentstroke}{rgb}{0.000000,0.000000,0.000000}%
\pgfsetstrokecolor{currentstroke}%
\pgfsetdash{}{0pt}%
\pgfpathmoveto{\pgfqpoint{3.106726in}{2.151862in}}%
\pgfpathlineto{\pgfqpoint{3.119697in}{2.145510in}}%
\pgfpathlineto{\pgfqpoint{3.132671in}{2.139195in}}%
\pgfpathlineto{\pgfqpoint{3.145649in}{2.132914in}}%
\pgfpathlineto{\pgfqpoint{3.158631in}{2.126668in}}%
\pgfpathlineto{\pgfqpoint{3.150599in}{2.124597in}}%
\pgfpathlineto{\pgfqpoint{3.142555in}{2.122734in}}%
\pgfpathlineto{\pgfqpoint{3.134499in}{2.121083in}}%
\pgfpathlineto{\pgfqpoint{3.126431in}{2.119651in}}%
\pgfpathlineto{\pgfqpoint{3.113426in}{2.126131in}}%
\pgfpathlineto{\pgfqpoint{3.100424in}{2.132645in}}%
\pgfpathlineto{\pgfqpoint{3.087425in}{2.139195in}}%
\pgfpathlineto{\pgfqpoint{3.074431in}{2.145780in}}%
\pgfpathlineto{\pgfqpoint{3.082523in}{2.146973in}}%
\pgfpathlineto{\pgfqpoint{3.090603in}{2.148389in}}%
\pgfpathlineto{\pgfqpoint{3.098671in}{2.150020in}}%
\pgfpathlineto{\pgfqpoint{3.106726in}{2.151862in}}%
\pgfpathclose%
\pgfusepath{fill}%
\end{pgfscope}%
\begin{pgfscope}%
\pgfpathrectangle{\pgfqpoint{1.254980in}{0.150000in}}{\pgfqpoint{5.490039in}{5.490039in}}%
\pgfusepath{clip}%
\pgfsetbuttcap%
\pgfsetroundjoin%
\definecolor{currentfill}{rgb}{0.283187,0.125848,0.444960}%
\pgfsetfillcolor{currentfill}%
\pgfsetfillopacity{0.700000}%
\pgfsetlinewidth{0.000000pt}%
\definecolor{currentstroke}{rgb}{0.000000,0.000000,0.000000}%
\pgfsetstrokecolor{currentstroke}%
\pgfsetdash{}{0pt}%
\pgfpathmoveto{\pgfqpoint{2.918772in}{2.227662in}}%
\pgfpathlineto{\pgfqpoint{2.931725in}{2.220631in}}%
\pgfpathlineto{\pgfqpoint{2.944681in}{2.213638in}}%
\pgfpathlineto{\pgfqpoint{2.957641in}{2.206684in}}%
\pgfpathlineto{\pgfqpoint{2.970604in}{2.199768in}}%
\pgfpathlineto{\pgfqpoint{2.962447in}{2.199288in}}%
\pgfpathlineto{\pgfqpoint{2.954278in}{2.199050in}}%
\pgfpathlineto{\pgfqpoint{2.946094in}{2.199060in}}%
\pgfpathlineto{\pgfqpoint{2.937896in}{2.199324in}}%
\pgfpathlineto{\pgfqpoint{2.924906in}{2.206488in}}%
\pgfpathlineto{\pgfqpoint{2.911920in}{2.213690in}}%
\pgfpathlineto{\pgfqpoint{2.898937in}{2.220931in}}%
\pgfpathlineto{\pgfqpoint{2.885956in}{2.228211in}}%
\pgfpathlineto{\pgfqpoint{2.894182in}{2.227694in}}%
\pgfpathlineto{\pgfqpoint{2.902393in}{2.227434in}}%
\pgfpathlineto{\pgfqpoint{2.910589in}{2.227426in}}%
\pgfpathlineto{\pgfqpoint{2.918772in}{2.227662in}}%
\pgfpathclose%
\pgfusepath{fill}%
\end{pgfscope}%
\begin{pgfscope}%
\pgfpathrectangle{\pgfqpoint{1.254980in}{0.150000in}}{\pgfqpoint{5.490039in}{5.490039in}}%
\pgfusepath{clip}%
\pgfsetbuttcap%
\pgfsetroundjoin%
\definecolor{currentfill}{rgb}{0.267004,0.004874,0.329415}%
\pgfsetfillcolor{currentfill}%
\pgfsetfillopacity{0.700000}%
\pgfsetlinewidth{0.000000pt}%
\definecolor{currentstroke}{rgb}{0.000000,0.000000,0.000000}%
\pgfsetstrokecolor{currentstroke}%
\pgfsetdash{}{0pt}%
\pgfpathmoveto{\pgfqpoint{3.836399in}{2.018284in}}%
\pgfpathlineto{\pgfqpoint{3.849488in}{2.014269in}}%
\pgfpathlineto{\pgfqpoint{3.862583in}{2.010281in}}%
\pgfpathlineto{\pgfqpoint{3.875684in}{2.006321in}}%
\pgfpathlineto{\pgfqpoint{3.888790in}{2.002388in}}%
\pgfpathlineto{\pgfqpoint{3.881123in}{1.995493in}}%
\pgfpathlineto{\pgfqpoint{3.873450in}{1.988671in}}%
\pgfpathlineto{\pgfqpoint{3.865771in}{1.981924in}}%
\pgfpathlineto{\pgfqpoint{3.858085in}{1.975258in}}%
\pgfpathlineto{\pgfqpoint{3.844965in}{1.979359in}}%
\pgfpathlineto{\pgfqpoint{3.831850in}{1.983488in}}%
\pgfpathlineto{\pgfqpoint{3.818740in}{1.987644in}}%
\pgfpathlineto{\pgfqpoint{3.805636in}{1.991828in}}%
\pgfpathlineto{\pgfqpoint{3.813337in}{1.998321in}}%
\pgfpathlineto{\pgfqpoint{3.821030in}{2.004898in}}%
\pgfpathlineto{\pgfqpoint{3.828718in}{2.011554in}}%
\pgfpathlineto{\pgfqpoint{3.836399in}{2.018284in}}%
\pgfpathclose%
\pgfusepath{fill}%
\end{pgfscope}%
\begin{pgfscope}%
\pgfpathrectangle{\pgfqpoint{1.254980in}{0.150000in}}{\pgfqpoint{5.490039in}{5.490039in}}%
\pgfusepath{clip}%
\pgfsetbuttcap%
\pgfsetroundjoin%
\definecolor{currentfill}{rgb}{0.282290,0.145912,0.461510}%
\pgfsetfillcolor{currentfill}%
\pgfsetfillopacity{0.700000}%
\pgfsetlinewidth{0.000000pt}%
\definecolor{currentstroke}{rgb}{0.000000,0.000000,0.000000}%
\pgfsetstrokecolor{currentstroke}%
\pgfsetdash{}{0pt}%
\pgfpathmoveto{\pgfqpoint{5.312219in}{2.271748in}}%
\pgfpathlineto{\pgfqpoint{5.325707in}{2.270757in}}%
\pgfpathlineto{\pgfqpoint{5.339203in}{2.269790in}}%
\pgfpathlineto{\pgfqpoint{5.352708in}{2.268846in}}%
\pgfpathlineto{\pgfqpoint{5.366220in}{2.267926in}}%
\pgfpathlineto{\pgfqpoint{5.359119in}{2.261064in}}%
\pgfpathlineto{\pgfqpoint{5.352011in}{2.254123in}}%
\pgfpathlineto{\pgfqpoint{5.344896in}{2.247103in}}%
\pgfpathlineto{\pgfqpoint{5.337773in}{2.240003in}}%
\pgfpathlineto{\pgfqpoint{5.324247in}{2.240911in}}%
\pgfpathlineto{\pgfqpoint{5.310730in}{2.241842in}}%
\pgfpathlineto{\pgfqpoint{5.297221in}{2.242797in}}%
\pgfpathlineto{\pgfqpoint{5.283720in}{2.243775in}}%
\pgfpathlineto{\pgfqpoint{5.290855in}{2.250883in}}%
\pgfpathlineto{\pgfqpoint{5.297984in}{2.257914in}}%
\pgfpathlineto{\pgfqpoint{5.305105in}{2.264869in}}%
\pgfpathlineto{\pgfqpoint{5.312219in}{2.271748in}}%
\pgfpathclose%
\pgfusepath{fill}%
\end{pgfscope}%
\begin{pgfscope}%
\pgfpathrectangle{\pgfqpoint{1.254980in}{0.150000in}}{\pgfqpoint{5.490039in}{5.490039in}}%
\pgfusepath{clip}%
\pgfsetbuttcap%
\pgfsetroundjoin%
\definecolor{currentfill}{rgb}{0.274128,0.199721,0.498911}%
\pgfsetfillcolor{currentfill}%
\pgfsetfillopacity{0.700000}%
\pgfsetlinewidth{0.000000pt}%
\definecolor{currentstroke}{rgb}{0.000000,0.000000,0.000000}%
\pgfsetstrokecolor{currentstroke}%
\pgfsetdash{}{0pt}%
\pgfpathmoveto{\pgfqpoint{5.831686in}{2.369662in}}%
\pgfpathlineto{\pgfqpoint{5.845331in}{2.369071in}}%
\pgfpathlineto{\pgfqpoint{5.858985in}{2.368503in}}%
\pgfpathlineto{\pgfqpoint{5.872648in}{2.367958in}}%
\pgfpathlineto{\pgfqpoint{5.886319in}{2.367437in}}%
\pgfpathlineto{\pgfqpoint{5.879482in}{2.362398in}}%
\pgfpathlineto{\pgfqpoint{5.872636in}{2.357295in}}%
\pgfpathlineto{\pgfqpoint{5.865783in}{2.352125in}}%
\pgfpathlineto{\pgfqpoint{5.858921in}{2.346885in}}%
\pgfpathlineto{\pgfqpoint{5.845232in}{2.347327in}}%
\pgfpathlineto{\pgfqpoint{5.831551in}{2.347793in}}%
\pgfpathlineto{\pgfqpoint{5.817879in}{2.348282in}}%
\pgfpathlineto{\pgfqpoint{5.804216in}{2.348794in}}%
\pgfpathlineto{\pgfqpoint{5.811095in}{2.354108in}}%
\pgfpathlineto{\pgfqpoint{5.817966in}{2.359356in}}%
\pgfpathlineto{\pgfqpoint{5.824830in}{2.364539in}}%
\pgfpathlineto{\pgfqpoint{5.831686in}{2.369662in}}%
\pgfpathclose%
\pgfusepath{fill}%
\end{pgfscope}%
\begin{pgfscope}%
\pgfpathrectangle{\pgfqpoint{1.254980in}{0.150000in}}{\pgfqpoint{5.490039in}{5.490039in}}%
\pgfusepath{clip}%
\pgfsetbuttcap%
\pgfsetroundjoin%
\definecolor{currentfill}{rgb}{0.283091,0.110553,0.431554}%
\pgfsetfillcolor{currentfill}%
\pgfsetfillopacity{0.700000}%
\pgfsetlinewidth{0.000000pt}%
\definecolor{currentstroke}{rgb}{0.000000,0.000000,0.000000}%
\pgfsetstrokecolor{currentstroke}%
\pgfsetdash{}{0pt}%
\pgfpathmoveto{\pgfqpoint{5.011066in}{2.203226in}}%
\pgfpathlineto{\pgfqpoint{5.024464in}{2.201849in}}%
\pgfpathlineto{\pgfqpoint{5.037870in}{2.200496in}}%
\pgfpathlineto{\pgfqpoint{5.051283in}{2.199168in}}%
\pgfpathlineto{\pgfqpoint{5.064705in}{2.197863in}}%
\pgfpathlineto{\pgfqpoint{5.057471in}{2.190107in}}%
\pgfpathlineto{\pgfqpoint{5.050231in}{2.182282in}}%
\pgfpathlineto{\pgfqpoint{5.042984in}{2.174386in}}%
\pgfpathlineto{\pgfqpoint{5.035730in}{2.166420in}}%
\pgfpathlineto{\pgfqpoint{5.022298in}{2.167752in}}%
\pgfpathlineto{\pgfqpoint{5.008873in}{2.169108in}}%
\pgfpathlineto{\pgfqpoint{4.995456in}{2.170488in}}%
\pgfpathlineto{\pgfqpoint{4.982047in}{2.171892in}}%
\pgfpathlineto{\pgfqpoint{4.989311in}{2.179825in}}%
\pgfpathlineto{\pgfqpoint{4.996569in}{2.187693in}}%
\pgfpathlineto{\pgfqpoint{5.003821in}{2.195493in}}%
\pgfpathlineto{\pgfqpoint{5.011066in}{2.203226in}}%
\pgfpathclose%
\pgfusepath{fill}%
\end{pgfscope}%
\begin{pgfscope}%
\pgfpathrectangle{\pgfqpoint{1.254980in}{0.150000in}}{\pgfqpoint{5.490039in}{5.490039in}}%
\pgfusepath{clip}%
\pgfsetbuttcap%
\pgfsetroundjoin%
\definecolor{currentfill}{rgb}{0.280267,0.073417,0.397163}%
\pgfsetfillcolor{currentfill}%
\pgfsetfillopacity{0.700000}%
\pgfsetlinewidth{0.000000pt}%
\definecolor{currentstroke}{rgb}{0.000000,0.000000,0.000000}%
\pgfsetstrokecolor{currentstroke}%
\pgfsetdash{}{0pt}%
\pgfpathmoveto{\pgfqpoint{4.709848in}{2.133340in}}%
\pgfpathlineto{\pgfqpoint{4.723158in}{2.131453in}}%
\pgfpathlineto{\pgfqpoint{4.736476in}{2.129591in}}%
\pgfpathlineto{\pgfqpoint{4.749801in}{2.127753in}}%
\pgfpathlineto{\pgfqpoint{4.763133in}{2.125940in}}%
\pgfpathlineto{\pgfqpoint{4.755781in}{2.117616in}}%
\pgfpathlineto{\pgfqpoint{4.748424in}{2.109243in}}%
\pgfpathlineto{\pgfqpoint{4.741060in}{2.100821in}}%
\pgfpathlineto{\pgfqpoint{4.733691in}{2.092353in}}%
\pgfpathlineto{\pgfqpoint{4.720349in}{2.094232in}}%
\pgfpathlineto{\pgfqpoint{4.707013in}{2.096136in}}%
\pgfpathlineto{\pgfqpoint{4.693685in}{2.098064in}}%
\pgfpathlineto{\pgfqpoint{4.680365in}{2.100017in}}%
\pgfpathlineto{\pgfqpoint{4.687744in}{2.108415in}}%
\pgfpathlineto{\pgfqpoint{4.695118in}{2.116769in}}%
\pgfpathlineto{\pgfqpoint{4.702486in}{2.125078in}}%
\pgfpathlineto{\pgfqpoint{4.709848in}{2.133340in}}%
\pgfpathclose%
\pgfusepath{fill}%
\end{pgfscope}%
\begin{pgfscope}%
\pgfpathrectangle{\pgfqpoint{1.254980in}{0.150000in}}{\pgfqpoint{5.490039in}{5.490039in}}%
\pgfusepath{clip}%
\pgfsetbuttcap%
\pgfsetroundjoin%
\definecolor{currentfill}{rgb}{0.269944,0.014625,0.341379}%
\pgfsetfillcolor{currentfill}%
\pgfsetfillopacity{0.700000}%
\pgfsetlinewidth{0.000000pt}%
\definecolor{currentstroke}{rgb}{0.000000,0.000000,0.000000}%
\pgfsetstrokecolor{currentstroke}%
\pgfsetdash{}{0pt}%
\pgfpathmoveto{\pgfqpoint{4.190225in}{2.036962in}}%
\pgfpathlineto{\pgfqpoint{4.203396in}{2.033911in}}%
\pgfpathlineto{\pgfqpoint{4.216575in}{2.030885in}}%
\pgfpathlineto{\pgfqpoint{4.229759in}{2.027885in}}%
\pgfpathlineto{\pgfqpoint{4.242950in}{2.024911in}}%
\pgfpathlineto{\pgfqpoint{4.235414in}{2.016811in}}%
\pgfpathlineto{\pgfqpoint{4.227873in}{2.008725in}}%
\pgfpathlineto{\pgfqpoint{4.220326in}{2.000656in}}%
\pgfpathlineto{\pgfqpoint{4.212774in}{1.992607in}}%
\pgfpathlineto{\pgfqpoint{4.199572in}{1.995712in}}%
\pgfpathlineto{\pgfqpoint{4.186375in}{1.998842in}}%
\pgfpathlineto{\pgfqpoint{4.173185in}{2.001997in}}%
\pgfpathlineto{\pgfqpoint{4.160002in}{2.005179in}}%
\pgfpathlineto{\pgfqpoint{4.167566in}{2.013093in}}%
\pgfpathlineto{\pgfqpoint{4.175124in}{2.021030in}}%
\pgfpathlineto{\pgfqpoint{4.182677in}{2.028987in}}%
\pgfpathlineto{\pgfqpoint{4.190225in}{2.036962in}}%
\pgfpathclose%
\pgfusepath{fill}%
\end{pgfscope}%
\begin{pgfscope}%
\pgfpathrectangle{\pgfqpoint{1.254980in}{0.150000in}}{\pgfqpoint{5.490039in}{5.490039in}}%
\pgfusepath{clip}%
\pgfsetbuttcap%
\pgfsetroundjoin%
\definecolor{currentfill}{rgb}{0.273809,0.031497,0.358853}%
\pgfsetfillcolor{currentfill}%
\pgfsetfillopacity{0.700000}%
\pgfsetlinewidth{0.000000pt}%
\definecolor{currentstroke}{rgb}{0.000000,0.000000,0.000000}%
\pgfsetstrokecolor{currentstroke}%
\pgfsetdash{}{0pt}%
\pgfpathmoveto{\pgfqpoint{4.408633in}{2.068806in}}%
\pgfpathlineto{\pgfqpoint{4.421862in}{2.066284in}}%
\pgfpathlineto{\pgfqpoint{4.435098in}{2.063786in}}%
\pgfpathlineto{\pgfqpoint{4.448340in}{2.061315in}}%
\pgfpathlineto{\pgfqpoint{4.461589in}{2.058868in}}%
\pgfpathlineto{\pgfqpoint{4.454129in}{2.050451in}}%
\pgfpathlineto{\pgfqpoint{4.446663in}{2.042017in}}%
\pgfpathlineto{\pgfqpoint{4.439192in}{2.033570in}}%
\pgfpathlineto{\pgfqpoint{4.431716in}{2.025112in}}%
\pgfpathlineto{\pgfqpoint{4.418456in}{2.027663in}}%
\pgfpathlineto{\pgfqpoint{4.405203in}{2.030240in}}%
\pgfpathlineto{\pgfqpoint{4.391956in}{2.032841in}}%
\pgfpathlineto{\pgfqpoint{4.378717in}{2.035468in}}%
\pgfpathlineto{\pgfqpoint{4.386204in}{2.043817in}}%
\pgfpathlineto{\pgfqpoint{4.393686in}{2.052158in}}%
\pgfpathlineto{\pgfqpoint{4.401162in}{2.060488in}}%
\pgfpathlineto{\pgfqpoint{4.408633in}{2.068806in}}%
\pgfpathclose%
\pgfusepath{fill}%
\end{pgfscope}%
\begin{pgfscope}%
\pgfpathrectangle{\pgfqpoint{1.254980in}{0.150000in}}{\pgfqpoint{5.490039in}{5.490039in}}%
\pgfusepath{clip}%
\pgfsetbuttcap%
\pgfsetroundjoin%
\definecolor{currentfill}{rgb}{0.277018,0.050344,0.375715}%
\pgfsetfillcolor{currentfill}%
\pgfsetfillopacity{0.700000}%
\pgfsetlinewidth{0.000000pt}%
\definecolor{currentstroke}{rgb}{0.000000,0.000000,0.000000}%
\pgfsetstrokecolor{currentstroke}%
\pgfsetdash{}{0pt}%
\pgfpathmoveto{\pgfqpoint{3.294475in}{2.089921in}}%
\pgfpathlineto{\pgfqpoint{3.307472in}{2.084198in}}%
\pgfpathlineto{\pgfqpoint{3.320474in}{2.078507in}}%
\pgfpathlineto{\pgfqpoint{3.333480in}{2.072849in}}%
\pgfpathlineto{\pgfqpoint{3.346491in}{2.067224in}}%
\pgfpathlineto{\pgfqpoint{3.338566in}{2.063738in}}%
\pgfpathlineto{\pgfqpoint{3.330631in}{2.060427in}}%
\pgfpathlineto{\pgfqpoint{3.322686in}{2.057295in}}%
\pgfpathlineto{\pgfqpoint{3.314732in}{2.054348in}}%
\pgfpathlineto{\pgfqpoint{3.301700in}{2.060194in}}%
\pgfpathlineto{\pgfqpoint{3.288673in}{2.066072in}}%
\pgfpathlineto{\pgfqpoint{3.275650in}{2.071982in}}%
\pgfpathlineto{\pgfqpoint{3.262632in}{2.077925in}}%
\pgfpathlineto{\pgfqpoint{3.270608in}{2.080647in}}%
\pgfpathlineto{\pgfqpoint{3.278574in}{2.083557in}}%
\pgfpathlineto{\pgfqpoint{3.286530in}{2.086650in}}%
\pgfpathlineto{\pgfqpoint{3.294475in}{2.089921in}}%
\pgfpathclose%
\pgfusepath{fill}%
\end{pgfscope}%
\begin{pgfscope}%
\pgfpathrectangle{\pgfqpoint{1.254980in}{0.150000in}}{\pgfqpoint{5.490039in}{5.490039in}}%
\pgfusepath{clip}%
\pgfsetbuttcap%
\pgfsetroundjoin%
\definecolor{currentfill}{rgb}{0.267004,0.004874,0.329415}%
\pgfsetfillcolor{currentfill}%
\pgfsetfillopacity{0.700000}%
\pgfsetlinewidth{0.000000pt}%
\definecolor{currentstroke}{rgb}{0.000000,0.000000,0.000000}%
\pgfsetstrokecolor{currentstroke}%
\pgfsetdash{}{0pt}%
\pgfpathmoveto{\pgfqpoint{3.971826in}{2.015770in}}%
\pgfpathlineto{\pgfqpoint{3.984948in}{2.012128in}}%
\pgfpathlineto{\pgfqpoint{3.998077in}{2.008513in}}%
\pgfpathlineto{\pgfqpoint{4.011211in}{2.004925in}}%
\pgfpathlineto{\pgfqpoint{4.024351in}{2.001364in}}%
\pgfpathlineto{\pgfqpoint{4.016735in}{1.993919in}}%
\pgfpathlineto{\pgfqpoint{4.009112in}{1.986524in}}%
\pgfpathlineto{\pgfqpoint{4.001484in}{1.979183in}}%
\pgfpathlineto{\pgfqpoint{3.993851in}{1.971900in}}%
\pgfpathlineto{\pgfqpoint{3.980698in}{1.975617in}}%
\pgfpathlineto{\pgfqpoint{3.967550in}{1.979361in}}%
\pgfpathlineto{\pgfqpoint{3.954409in}{1.983131in}}%
\pgfpathlineto{\pgfqpoint{3.941274in}{1.986928in}}%
\pgfpathlineto{\pgfqpoint{3.948921in}{1.994051in}}%
\pgfpathlineto{\pgfqpoint{3.956562in}{2.001235in}}%
\pgfpathlineto{\pgfqpoint{3.964197in}{2.008475in}}%
\pgfpathlineto{\pgfqpoint{3.971826in}{2.015770in}}%
\pgfpathclose%
\pgfusepath{fill}%
\end{pgfscope}%
\begin{pgfscope}%
\pgfpathrectangle{\pgfqpoint{1.254980in}{0.150000in}}{\pgfqpoint{5.490039in}{5.490039in}}%
\pgfusepath{clip}%
\pgfsetbuttcap%
\pgfsetroundjoin%
\definecolor{currentfill}{rgb}{0.278826,0.175490,0.483397}%
\pgfsetfillcolor{currentfill}%
\pgfsetfillopacity{0.700000}%
\pgfsetlinewidth{0.000000pt}%
\definecolor{currentstroke}{rgb}{0.000000,0.000000,0.000000}%
\pgfsetstrokecolor{currentstroke}%
\pgfsetdash{}{0pt}%
\pgfpathmoveto{\pgfqpoint{2.730422in}{2.318731in}}%
\pgfpathlineto{\pgfqpoint{2.743368in}{2.310957in}}%
\pgfpathlineto{\pgfqpoint{2.756316in}{2.303226in}}%
\pgfpathlineto{\pgfqpoint{2.769268in}{2.295537in}}%
\pgfpathlineto{\pgfqpoint{2.782222in}{2.287892in}}%
\pgfpathlineto{\pgfqpoint{2.773924in}{2.289188in}}%
\pgfpathlineto{\pgfqpoint{2.765609in}{2.290763in}}%
\pgfpathlineto{\pgfqpoint{2.757278in}{2.292622in}}%
\pgfpathlineto{\pgfqpoint{2.748930in}{2.294774in}}%
\pgfpathlineto{\pgfqpoint{2.735946in}{2.302683in}}%
\pgfpathlineto{\pgfqpoint{2.722964in}{2.310634in}}%
\pgfpathlineto{\pgfqpoint{2.709985in}{2.318628in}}%
\pgfpathlineto{\pgfqpoint{2.697008in}{2.326666in}}%
\pgfpathlineto{\pgfqpoint{2.705387in}{2.324245in}}%
\pgfpathlineto{\pgfqpoint{2.713749in}{2.322121in}}%
\pgfpathlineto{\pgfqpoint{2.722093in}{2.320285in}}%
\pgfpathlineto{\pgfqpoint{2.730422in}{2.318731in}}%
\pgfpathclose%
\pgfusepath{fill}%
\end{pgfscope}%
\begin{pgfscope}%
\pgfpathrectangle{\pgfqpoint{1.254980in}{0.150000in}}{\pgfqpoint{5.490039in}{5.490039in}}%
\pgfusepath{clip}%
\pgfsetbuttcap%
\pgfsetroundjoin%
\definecolor{currentfill}{rgb}{0.279574,0.170599,0.479997}%
\pgfsetfillcolor{currentfill}%
\pgfsetfillopacity{0.700000}%
\pgfsetlinewidth{0.000000pt}%
\definecolor{currentstroke}{rgb}{0.000000,0.000000,0.000000}%
\pgfsetstrokecolor{currentstroke}%
\pgfsetdash{}{0pt}%
\pgfpathmoveto{\pgfqpoint{5.530935in}{2.313118in}}%
\pgfpathlineto{\pgfqpoint{5.544493in}{2.312347in}}%
\pgfpathlineto{\pgfqpoint{5.558060in}{2.311599in}}%
\pgfpathlineto{\pgfqpoint{5.571636in}{2.310875in}}%
\pgfpathlineto{\pgfqpoint{5.585220in}{2.310175in}}%
\pgfpathlineto{\pgfqpoint{5.578222in}{2.304028in}}%
\pgfpathlineto{\pgfqpoint{5.571216in}{2.297804in}}%
\pgfpathlineto{\pgfqpoint{5.564203in}{2.291500in}}%
\pgfpathlineto{\pgfqpoint{5.557181in}{2.285115in}}%
\pgfpathlineto{\pgfqpoint{5.543583in}{2.285776in}}%
\pgfpathlineto{\pgfqpoint{5.529992in}{2.286461in}}%
\pgfpathlineto{\pgfqpoint{5.516410in}{2.287170in}}%
\pgfpathlineto{\pgfqpoint{5.502837in}{2.287902in}}%
\pgfpathlineto{\pgfqpoint{5.509873in}{2.294321in}}%
\pgfpathlineto{\pgfqpoint{5.516901in}{2.300662in}}%
\pgfpathlineto{\pgfqpoint{5.523922in}{2.306927in}}%
\pgfpathlineto{\pgfqpoint{5.530935in}{2.313118in}}%
\pgfpathclose%
\pgfusepath{fill}%
\end{pgfscope}%
\begin{pgfscope}%
\pgfpathrectangle{\pgfqpoint{1.254980in}{0.150000in}}{\pgfqpoint{5.490039in}{5.490039in}}%
\pgfusepath{clip}%
\pgfsetbuttcap%
\pgfsetroundjoin%
\definecolor{currentfill}{rgb}{0.282884,0.135920,0.453427}%
\pgfsetfillcolor{currentfill}%
\pgfsetfillopacity{0.700000}%
\pgfsetlinewidth{0.000000pt}%
\definecolor{currentstroke}{rgb}{0.000000,0.000000,0.000000}%
\pgfsetstrokecolor{currentstroke}%
\pgfsetdash{}{0pt}%
\pgfpathmoveto{\pgfqpoint{5.229797in}{2.247929in}}%
\pgfpathlineto{\pgfqpoint{5.243266in}{2.246855in}}%
\pgfpathlineto{\pgfqpoint{5.256742in}{2.245805in}}%
\pgfpathlineto{\pgfqpoint{5.270227in}{2.244778in}}%
\pgfpathlineto{\pgfqpoint{5.283720in}{2.243775in}}%
\pgfpathlineto{\pgfqpoint{5.276577in}{2.236590in}}%
\pgfpathlineto{\pgfqpoint{5.269427in}{2.229327in}}%
\pgfpathlineto{\pgfqpoint{5.262270in}{2.221985in}}%
\pgfpathlineto{\pgfqpoint{5.255106in}{2.214563in}}%
\pgfpathlineto{\pgfqpoint{5.241601in}{2.215566in}}%
\pgfpathlineto{\pgfqpoint{5.228104in}{2.216594in}}%
\pgfpathlineto{\pgfqpoint{5.214615in}{2.217645in}}%
\pgfpathlineto{\pgfqpoint{5.201134in}{2.218719in}}%
\pgfpathlineto{\pgfqpoint{5.208310in}{2.226135in}}%
\pgfpathlineto{\pgfqpoint{5.215479in}{2.233475in}}%
\pgfpathlineto{\pgfqpoint{5.222642in}{2.240740in}}%
\pgfpathlineto{\pgfqpoint{5.229797in}{2.247929in}}%
\pgfpathclose%
\pgfusepath{fill}%
\end{pgfscope}%
\begin{pgfscope}%
\pgfpathrectangle{\pgfqpoint{1.254980in}{0.150000in}}{\pgfqpoint{5.490039in}{5.490039in}}%
\pgfusepath{clip}%
\pgfsetbuttcap%
\pgfsetroundjoin%
\definecolor{currentfill}{rgb}{0.282656,0.100196,0.422160}%
\pgfsetfillcolor{currentfill}%
\pgfsetfillopacity{0.700000}%
\pgfsetlinewidth{0.000000pt}%
\definecolor{currentstroke}{rgb}{0.000000,0.000000,0.000000}%
\pgfsetstrokecolor{currentstroke}%
\pgfsetdash{}{0pt}%
\pgfpathmoveto{\pgfqpoint{4.928487in}{2.177749in}}%
\pgfpathlineto{\pgfqpoint{4.941866in}{2.176249in}}%
\pgfpathlineto{\pgfqpoint{4.955252in}{2.174772in}}%
\pgfpathlineto{\pgfqpoint{4.968646in}{2.173320in}}%
\pgfpathlineto{\pgfqpoint{4.982047in}{2.171892in}}%
\pgfpathlineto{\pgfqpoint{4.974776in}{2.163891in}}%
\pgfpathlineto{\pgfqpoint{4.967499in}{2.155826in}}%
\pgfpathlineto{\pgfqpoint{4.960216in}{2.147694in}}%
\pgfpathlineto{\pgfqpoint{4.952926in}{2.139498in}}%
\pgfpathlineto{\pgfqpoint{4.939513in}{2.140966in}}%
\pgfpathlineto{\pgfqpoint{4.926109in}{2.142459in}}%
\pgfpathlineto{\pgfqpoint{4.912712in}{2.143975in}}%
\pgfpathlineto{\pgfqpoint{4.899323in}{2.145516in}}%
\pgfpathlineto{\pgfqpoint{4.906623in}{2.153667in}}%
\pgfpathlineto{\pgfqpoint{4.913917in}{2.161757in}}%
\pgfpathlineto{\pgfqpoint{4.921206in}{2.169784in}}%
\pgfpathlineto{\pgfqpoint{4.928487in}{2.177749in}}%
\pgfpathclose%
\pgfusepath{fill}%
\end{pgfscope}%
\begin{pgfscope}%
\pgfpathrectangle{\pgfqpoint{1.254980in}{0.150000in}}{\pgfqpoint{5.490039in}{5.490039in}}%
\pgfusepath{clip}%
\pgfsetbuttcap%
\pgfsetroundjoin%
\definecolor{currentfill}{rgb}{0.260571,0.246922,0.522828}%
\pgfsetfillcolor{currentfill}%
\pgfsetfillopacity{0.700000}%
\pgfsetlinewidth{0.000000pt}%
\definecolor{currentstroke}{rgb}{0.000000,0.000000,0.000000}%
\pgfsetstrokecolor{currentstroke}%
\pgfsetdash{}{0pt}%
\pgfpathmoveto{\pgfqpoint{2.489672in}{2.461526in}}%
\pgfpathlineto{\pgfqpoint{2.502616in}{2.452733in}}%
\pgfpathlineto{\pgfqpoint{2.515562in}{2.443991in}}%
\pgfpathlineto{\pgfqpoint{2.528510in}{2.435299in}}%
\pgfpathlineto{\pgfqpoint{2.541460in}{2.426657in}}%
\pgfpathlineto{\pgfqpoint{2.532964in}{2.430199in}}%
\pgfpathlineto{\pgfqpoint{2.524449in}{2.434061in}}%
\pgfpathlineto{\pgfqpoint{2.515915in}{2.438252in}}%
\pgfpathlineto{\pgfqpoint{2.507359in}{2.442778in}}%
\pgfpathlineto{\pgfqpoint{2.494375in}{2.451699in}}%
\pgfpathlineto{\pgfqpoint{2.481392in}{2.460670in}}%
\pgfpathlineto{\pgfqpoint{2.468411in}{2.469691in}}%
\pgfpathlineto{\pgfqpoint{2.455432in}{2.478764in}}%
\pgfpathlineto{\pgfqpoint{2.464023in}{2.473953in}}%
\pgfpathlineto{\pgfqpoint{2.472593in}{2.469481in}}%
\pgfpathlineto{\pgfqpoint{2.481143in}{2.465341in}}%
\pgfpathlineto{\pgfqpoint{2.489672in}{2.461526in}}%
\pgfpathclose%
\pgfusepath{fill}%
\end{pgfscope}%
\begin{pgfscope}%
\pgfpathrectangle{\pgfqpoint{1.254980in}{0.150000in}}{\pgfqpoint{5.490039in}{5.490039in}}%
\pgfusepath{clip}%
\pgfsetbuttcap%
\pgfsetroundjoin%
\definecolor{currentfill}{rgb}{0.278791,0.062145,0.386592}%
\pgfsetfillcolor{currentfill}%
\pgfsetfillopacity{0.700000}%
\pgfsetlinewidth{0.000000pt}%
\definecolor{currentstroke}{rgb}{0.000000,0.000000,0.000000}%
\pgfsetstrokecolor{currentstroke}%
\pgfsetdash{}{0pt}%
\pgfpathmoveto{\pgfqpoint{4.627154in}{2.108076in}}%
\pgfpathlineto{\pgfqpoint{4.640446in}{2.106024in}}%
\pgfpathlineto{\pgfqpoint{4.653745in}{2.103997in}}%
\pgfpathlineto{\pgfqpoint{4.667051in}{2.101995in}}%
\pgfpathlineto{\pgfqpoint{4.680365in}{2.100017in}}%
\pgfpathlineto{\pgfqpoint{4.672980in}{2.091578in}}%
\pgfpathlineto{\pgfqpoint{4.665589in}{2.083097in}}%
\pgfpathlineto{\pgfqpoint{4.658192in}{2.074577in}}%
\pgfpathlineto{\pgfqpoint{4.650791in}{2.066020in}}%
\pgfpathlineto{\pgfqpoint{4.637467in}{2.068077in}}%
\pgfpathlineto{\pgfqpoint{4.624150in}{2.070158in}}%
\pgfpathlineto{\pgfqpoint{4.610841in}{2.072264in}}%
\pgfpathlineto{\pgfqpoint{4.597539in}{2.074394in}}%
\pgfpathlineto{\pgfqpoint{4.604951in}{2.082868in}}%
\pgfpathlineto{\pgfqpoint{4.612358in}{2.091307in}}%
\pgfpathlineto{\pgfqpoint{4.619759in}{2.099710in}}%
\pgfpathlineto{\pgfqpoint{4.627154in}{2.108076in}}%
\pgfpathclose%
\pgfusepath{fill}%
\end{pgfscope}%
\begin{pgfscope}%
\pgfpathrectangle{\pgfqpoint{1.254980in}{0.150000in}}{\pgfqpoint{5.490039in}{5.490039in}}%
\pgfusepath{clip}%
\pgfsetbuttcap%
\pgfsetroundjoin%
\definecolor{currentfill}{rgb}{0.275191,0.194905,0.496005}%
\pgfsetfillcolor{currentfill}%
\pgfsetfillopacity{0.700000}%
\pgfsetlinewidth{0.000000pt}%
\definecolor{currentstroke}{rgb}{0.000000,0.000000,0.000000}%
\pgfsetstrokecolor{currentstroke}%
\pgfsetdash{}{0pt}%
\pgfpathmoveto{\pgfqpoint{5.749650in}{2.351079in}}%
\pgfpathlineto{\pgfqpoint{5.763279in}{2.350472in}}%
\pgfpathlineto{\pgfqpoint{5.776916in}{2.349889in}}%
\pgfpathlineto{\pgfqpoint{5.790562in}{2.349330in}}%
\pgfpathlineto{\pgfqpoint{5.804216in}{2.348794in}}%
\pgfpathlineto{\pgfqpoint{5.797329in}{2.343411in}}%
\pgfpathlineto{\pgfqpoint{5.790435in}{2.337957in}}%
\pgfpathlineto{\pgfqpoint{5.783532in}{2.332430in}}%
\pgfpathlineto{\pgfqpoint{5.776621in}{2.326827in}}%
\pgfpathlineto{\pgfqpoint{5.762950in}{2.327296in}}%
\pgfpathlineto{\pgfqpoint{5.749287in}{2.327790in}}%
\pgfpathlineto{\pgfqpoint{5.735633in}{2.328307in}}%
\pgfpathlineto{\pgfqpoint{5.721988in}{2.328848in}}%
\pgfpathlineto{\pgfqpoint{5.728915in}{2.334512in}}%
\pgfpathlineto{\pgfqpoint{5.735835in}{2.340104in}}%
\pgfpathlineto{\pgfqpoint{5.742746in}{2.345625in}}%
\pgfpathlineto{\pgfqpoint{5.749650in}{2.351079in}}%
\pgfpathclose%
\pgfusepath{fill}%
\end{pgfscope}%
\begin{pgfscope}%
\pgfpathrectangle{\pgfqpoint{1.254980in}{0.150000in}}{\pgfqpoint{5.490039in}{5.490039in}}%
\pgfusepath{clip}%
\pgfsetbuttcap%
\pgfsetroundjoin%
\definecolor{currentfill}{rgb}{0.283197,0.115680,0.436115}%
\pgfsetfillcolor{currentfill}%
\pgfsetfillopacity{0.700000}%
\pgfsetlinewidth{0.000000pt}%
\definecolor{currentstroke}{rgb}{0.000000,0.000000,0.000000}%
\pgfsetstrokecolor{currentstroke}%
\pgfsetdash{}{0pt}%
\pgfpathmoveto{\pgfqpoint{2.970604in}{2.199768in}}%
\pgfpathlineto{\pgfqpoint{2.983570in}{2.192890in}}%
\pgfpathlineto{\pgfqpoint{2.996539in}{2.186049in}}%
\pgfpathlineto{\pgfqpoint{3.009512in}{2.179246in}}%
\pgfpathlineto{\pgfqpoint{3.022489in}{2.172480in}}%
\pgfpathlineto{\pgfqpoint{3.014359in}{2.171758in}}%
\pgfpathlineto{\pgfqpoint{3.006215in}{2.171274in}}%
\pgfpathlineto{\pgfqpoint{2.998058in}{2.171034in}}%
\pgfpathlineto{\pgfqpoint{2.989888in}{2.171045in}}%
\pgfpathlineto{\pgfqpoint{2.976885in}{2.178059in}}%
\pgfpathlineto{\pgfqpoint{2.963885in}{2.185110in}}%
\pgfpathlineto{\pgfqpoint{2.950889in}{2.192198in}}%
\pgfpathlineto{\pgfqpoint{2.937896in}{2.199324in}}%
\pgfpathlineto{\pgfqpoint{2.946094in}{2.199060in}}%
\pgfpathlineto{\pgfqpoint{2.954278in}{2.199050in}}%
\pgfpathlineto{\pgfqpoint{2.962447in}{2.199288in}}%
\pgfpathlineto{\pgfqpoint{2.970604in}{2.199768in}}%
\pgfpathclose%
\pgfusepath{fill}%
\end{pgfscope}%
\begin{pgfscope}%
\pgfpathrectangle{\pgfqpoint{1.254980in}{0.150000in}}{\pgfqpoint{5.490039in}{5.490039in}}%
\pgfusepath{clip}%
\pgfsetbuttcap%
\pgfsetroundjoin%
\definecolor{currentfill}{rgb}{0.272594,0.025563,0.353093}%
\pgfsetfillcolor{currentfill}%
\pgfsetfillopacity{0.700000}%
\pgfsetlinewidth{0.000000pt}%
\definecolor{currentstroke}{rgb}{0.000000,0.000000,0.000000}%
\pgfsetstrokecolor{currentstroke}%
\pgfsetdash{}{0pt}%
\pgfpathmoveto{\pgfqpoint{4.325825in}{2.046230in}}%
\pgfpathlineto{\pgfqpoint{4.339038in}{2.043501in}}%
\pgfpathlineto{\pgfqpoint{4.352258in}{2.040798in}}%
\pgfpathlineto{\pgfqpoint{4.365484in}{2.038121in}}%
\pgfpathlineto{\pgfqpoint{4.378717in}{2.035468in}}%
\pgfpathlineto{\pgfqpoint{4.371224in}{2.027114in}}%
\pgfpathlineto{\pgfqpoint{4.363726in}{2.018756in}}%
\pgfpathlineto{\pgfqpoint{4.356223in}{2.010397in}}%
\pgfpathlineto{\pgfqpoint{4.348715in}{2.002041in}}%
\pgfpathlineto{\pgfqpoint{4.335471in}{2.004811in}}%
\pgfpathlineto{\pgfqpoint{4.322234in}{2.007606in}}%
\pgfpathlineto{\pgfqpoint{4.309004in}{2.010426in}}%
\pgfpathlineto{\pgfqpoint{4.295780in}{2.013272in}}%
\pgfpathlineto{\pgfqpoint{4.303299in}{2.021506in}}%
\pgfpathlineto{\pgfqpoint{4.310813in}{2.029746in}}%
\pgfpathlineto{\pgfqpoint{4.318322in}{2.037988in}}%
\pgfpathlineto{\pgfqpoint{4.325825in}{2.046230in}}%
\pgfpathclose%
\pgfusepath{fill}%
\end{pgfscope}%
\begin{pgfscope}%
\pgfpathrectangle{\pgfqpoint{1.254980in}{0.150000in}}{\pgfqpoint{5.490039in}{5.490039in}}%
\pgfusepath{clip}%
\pgfsetbuttcap%
\pgfsetroundjoin%
\definecolor{currentfill}{rgb}{0.268510,0.009605,0.335427}%
\pgfsetfillcolor{currentfill}%
\pgfsetfillopacity{0.700000}%
\pgfsetlinewidth{0.000000pt}%
\definecolor{currentstroke}{rgb}{0.000000,0.000000,0.000000}%
\pgfsetstrokecolor{currentstroke}%
\pgfsetdash{}{0pt}%
\pgfpathmoveto{\pgfqpoint{4.107331in}{2.018167in}}%
\pgfpathlineto{\pgfqpoint{4.120489in}{2.014881in}}%
\pgfpathlineto{\pgfqpoint{4.133654in}{2.011621in}}%
\pgfpathlineto{\pgfqpoint{4.146825in}{2.008387in}}%
\pgfpathlineto{\pgfqpoint{4.160002in}{2.005179in}}%
\pgfpathlineto{\pgfqpoint{4.152433in}{1.997292in}}%
\pgfpathlineto{\pgfqpoint{4.144858in}{1.989435in}}%
\pgfpathlineto{\pgfqpoint{4.137278in}{1.981612in}}%
\pgfpathlineto{\pgfqpoint{4.129692in}{1.973825in}}%
\pgfpathlineto{\pgfqpoint{4.116503in}{1.977175in}}%
\pgfpathlineto{\pgfqpoint{4.103320in}{1.980552in}}%
\pgfpathlineto{\pgfqpoint{4.090143in}{1.983955in}}%
\pgfpathlineto{\pgfqpoint{4.076972in}{1.987384in}}%
\pgfpathlineto{\pgfqpoint{4.084570in}{1.995023in}}%
\pgfpathlineto{\pgfqpoint{4.092163in}{2.002702in}}%
\pgfpathlineto{\pgfqpoint{4.099750in}{2.010418in}}%
\pgfpathlineto{\pgfqpoint{4.107331in}{2.018167in}}%
\pgfpathclose%
\pgfusepath{fill}%
\end{pgfscope}%
\begin{pgfscope}%
\pgfpathrectangle{\pgfqpoint{1.254980in}{0.150000in}}{\pgfqpoint{5.490039in}{5.490039in}}%
\pgfusepath{clip}%
\pgfsetbuttcap%
\pgfsetroundjoin%
\definecolor{currentfill}{rgb}{0.280267,0.073417,0.397163}%
\pgfsetfillcolor{currentfill}%
\pgfsetfillopacity{0.700000}%
\pgfsetlinewidth{0.000000pt}%
\definecolor{currentstroke}{rgb}{0.000000,0.000000,0.000000}%
\pgfsetstrokecolor{currentstroke}%
\pgfsetdash{}{0pt}%
\pgfpathmoveto{\pgfqpoint{3.158631in}{2.126668in}}%
\pgfpathlineto{\pgfqpoint{3.171617in}{2.120456in}}%
\pgfpathlineto{\pgfqpoint{3.184607in}{2.114279in}}%
\pgfpathlineto{\pgfqpoint{3.197601in}{2.108136in}}%
\pgfpathlineto{\pgfqpoint{3.210599in}{2.102027in}}%
\pgfpathlineto{\pgfqpoint{3.202589in}{2.099728in}}%
\pgfpathlineto{\pgfqpoint{3.194569in}{2.097633in}}%
\pgfpathlineto{\pgfqpoint{3.186536in}{2.095746in}}%
\pgfpathlineto{\pgfqpoint{3.178493in}{2.094076in}}%
\pgfpathlineto{\pgfqpoint{3.165472in}{2.100418in}}%
\pgfpathlineto{\pgfqpoint{3.152454in}{2.106795in}}%
\pgfpathlineto{\pgfqpoint{3.139441in}{2.113206in}}%
\pgfpathlineto{\pgfqpoint{3.126431in}{2.119651in}}%
\pgfpathlineto{\pgfqpoint{3.134499in}{2.121083in}}%
\pgfpathlineto{\pgfqpoint{3.142555in}{2.122734in}}%
\pgfpathlineto{\pgfqpoint{3.150599in}{2.124597in}}%
\pgfpathlineto{\pgfqpoint{3.158631in}{2.126668in}}%
\pgfpathclose%
\pgfusepath{fill}%
\end{pgfscope}%
\begin{pgfscope}%
\pgfpathrectangle{\pgfqpoint{1.254980in}{0.150000in}}{\pgfqpoint{5.490039in}{5.490039in}}%
\pgfusepath{clip}%
\pgfsetbuttcap%
\pgfsetroundjoin%
\definecolor{currentfill}{rgb}{0.268510,0.009605,0.335427}%
\pgfsetfillcolor{currentfill}%
\pgfsetfillopacity{0.700000}%
\pgfsetlinewidth{0.000000pt}%
\definecolor{currentstroke}{rgb}{0.000000,0.000000,0.000000}%
\pgfsetstrokecolor{currentstroke}%
\pgfsetdash{}{0pt}%
\pgfpathmoveto{\pgfqpoint{3.617761in}{2.021352in}}%
\pgfpathlineto{\pgfqpoint{3.630816in}{2.016647in}}%
\pgfpathlineto{\pgfqpoint{3.643877in}{2.011970in}}%
\pgfpathlineto{\pgfqpoint{3.656942in}{2.007322in}}%
\pgfpathlineto{\pgfqpoint{3.670013in}{2.002703in}}%
\pgfpathlineto{\pgfqpoint{3.662247in}{1.997070in}}%
\pgfpathlineto{\pgfqpoint{3.654474in}{1.991554in}}%
\pgfpathlineto{\pgfqpoint{3.646693in}{1.986159in}}%
\pgfpathlineto{\pgfqpoint{3.638905in}{1.980890in}}%
\pgfpathlineto{\pgfqpoint{3.625817in}{1.985703in}}%
\pgfpathlineto{\pgfqpoint{3.612735in}{1.990545in}}%
\pgfpathlineto{\pgfqpoint{3.599658in}{1.995415in}}%
\pgfpathlineto{\pgfqpoint{3.586586in}{2.000315in}}%
\pgfpathlineto{\pgfqpoint{3.594391in}{2.005385in}}%
\pgfpathlineto{\pgfqpoint{3.602189in}{2.010585in}}%
\pgfpathlineto{\pgfqpoint{3.609979in}{2.015909in}}%
\pgfpathlineto{\pgfqpoint{3.617761in}{2.021352in}}%
\pgfpathclose%
\pgfusepath{fill}%
\end{pgfscope}%
\begin{pgfscope}%
\pgfpathrectangle{\pgfqpoint{1.254980in}{0.150000in}}{\pgfqpoint{5.490039in}{5.490039in}}%
\pgfusepath{clip}%
\pgfsetbuttcap%
\pgfsetroundjoin%
\definecolor{currentfill}{rgb}{0.267004,0.004874,0.329415}%
\pgfsetfillcolor{currentfill}%
\pgfsetfillopacity{0.700000}%
\pgfsetlinewidth{0.000000pt}%
\definecolor{currentstroke}{rgb}{0.000000,0.000000,0.000000}%
\pgfsetstrokecolor{currentstroke}%
\pgfsetdash{}{0pt}%
\pgfpathmoveto{\pgfqpoint{3.753277in}{2.008842in}}%
\pgfpathlineto{\pgfqpoint{3.766358in}{2.004547in}}%
\pgfpathlineto{\pgfqpoint{3.779446in}{2.000279in}}%
\pgfpathlineto{\pgfqpoint{3.792538in}{1.996040in}}%
\pgfpathlineto{\pgfqpoint{3.805636in}{1.991828in}}%
\pgfpathlineto{\pgfqpoint{3.797930in}{1.985422in}}%
\pgfpathlineto{\pgfqpoint{3.790216in}{1.979109in}}%
\pgfpathlineto{\pgfqpoint{3.782496in}{1.972891in}}%
\pgfpathlineto{\pgfqpoint{3.774769in}{1.966774in}}%
\pgfpathlineto{\pgfqpoint{3.761656in}{1.971167in}}%
\pgfpathlineto{\pgfqpoint{3.748548in}{1.975588in}}%
\pgfpathlineto{\pgfqpoint{3.735445in}{1.980036in}}%
\pgfpathlineto{\pgfqpoint{3.722348in}{1.984513in}}%
\pgfpathlineto{\pgfqpoint{3.730091in}{1.990444in}}%
\pgfpathlineto{\pgfqpoint{3.737826in}{1.996478in}}%
\pgfpathlineto{\pgfqpoint{3.745555in}{2.002613in}}%
\pgfpathlineto{\pgfqpoint{3.753277in}{2.008842in}}%
\pgfpathclose%
\pgfusepath{fill}%
\end{pgfscope}%
\begin{pgfscope}%
\pgfpathrectangle{\pgfqpoint{1.254980in}{0.150000in}}{\pgfqpoint{5.490039in}{5.490039in}}%
\pgfusepath{clip}%
\pgfsetbuttcap%
\pgfsetroundjoin%
\definecolor{currentfill}{rgb}{0.271305,0.019942,0.347269}%
\pgfsetfillcolor{currentfill}%
\pgfsetfillopacity{0.700000}%
\pgfsetlinewidth{0.000000pt}%
\definecolor{currentstroke}{rgb}{0.000000,0.000000,0.000000}%
\pgfsetstrokecolor{currentstroke}%
\pgfsetdash{}{0pt}%
\pgfpathmoveto{\pgfqpoint{3.482187in}{2.040580in}}%
\pgfpathlineto{\pgfqpoint{3.495220in}{2.035441in}}%
\pgfpathlineto{\pgfqpoint{3.508258in}{2.030334in}}%
\pgfpathlineto{\pgfqpoint{3.521300in}{2.025256in}}%
\pgfpathlineto{\pgfqpoint{3.534347in}{2.020209in}}%
\pgfpathlineto{\pgfqpoint{3.526516in}{2.015475in}}%
\pgfpathlineto{\pgfqpoint{3.518676in}{2.010885in}}%
\pgfpathlineto{\pgfqpoint{3.510828in}{2.006442in}}%
\pgfpathlineto{\pgfqpoint{3.502971in}{2.002152in}}%
\pgfpathlineto{\pgfqpoint{3.489905in}{2.007407in}}%
\pgfpathlineto{\pgfqpoint{3.476844in}{2.012691in}}%
\pgfpathlineto{\pgfqpoint{3.463788in}{2.018006in}}%
\pgfpathlineto{\pgfqpoint{3.450737in}{2.023351in}}%
\pgfpathlineto{\pgfqpoint{3.458612in}{2.027429in}}%
\pgfpathlineto{\pgfqpoint{3.466479in}{2.031663in}}%
\pgfpathlineto{\pgfqpoint{3.474338in}{2.036048in}}%
\pgfpathlineto{\pgfqpoint{3.482187in}{2.040580in}}%
\pgfpathclose%
\pgfusepath{fill}%
\end{pgfscope}%
\begin{pgfscope}%
\pgfpathrectangle{\pgfqpoint{1.254980in}{0.150000in}}{\pgfqpoint{5.490039in}{5.490039in}}%
\pgfusepath{clip}%
\pgfsetbuttcap%
\pgfsetroundjoin%
\definecolor{currentfill}{rgb}{0.280255,0.165693,0.476498}%
\pgfsetfillcolor{currentfill}%
\pgfsetfillopacity{0.700000}%
\pgfsetlinewidth{0.000000pt}%
\definecolor{currentstroke}{rgb}{0.000000,0.000000,0.000000}%
\pgfsetstrokecolor{currentstroke}%
\pgfsetdash{}{0pt}%
\pgfpathmoveto{\pgfqpoint{5.448627in}{2.291067in}}%
\pgfpathlineto{\pgfqpoint{5.462167in}{2.290240in}}%
\pgfpathlineto{\pgfqpoint{5.475715in}{2.289437in}}%
\pgfpathlineto{\pgfqpoint{5.489272in}{2.288658in}}%
\pgfpathlineto{\pgfqpoint{5.502837in}{2.287902in}}%
\pgfpathlineto{\pgfqpoint{5.495794in}{2.281404in}}%
\pgfpathlineto{\pgfqpoint{5.488743in}{2.274825in}}%
\pgfpathlineto{\pgfqpoint{5.481684in}{2.268165in}}%
\pgfpathlineto{\pgfqpoint{5.474618in}{2.261423in}}%
\pgfpathlineto{\pgfqpoint{5.461039in}{2.262152in}}%
\pgfpathlineto{\pgfqpoint{5.447468in}{2.262906in}}%
\pgfpathlineto{\pgfqpoint{5.433906in}{2.263683in}}%
\pgfpathlineto{\pgfqpoint{5.420352in}{2.264484in}}%
\pgfpathlineto{\pgfqpoint{5.427432in}{2.271248in}}%
\pgfpathlineto{\pgfqpoint{5.434505in}{2.277933in}}%
\pgfpathlineto{\pgfqpoint{5.441570in}{2.284538in}}%
\pgfpathlineto{\pgfqpoint{5.448627in}{2.291067in}}%
\pgfpathclose%
\pgfusepath{fill}%
\end{pgfscope}%
\begin{pgfscope}%
\pgfpathrectangle{\pgfqpoint{1.254980in}{0.150000in}}{\pgfqpoint{5.490039in}{5.490039in}}%
\pgfusepath{clip}%
\pgfsetbuttcap%
\pgfsetroundjoin%
\definecolor{currentfill}{rgb}{0.281924,0.089666,0.412415}%
\pgfsetfillcolor{currentfill}%
\pgfsetfillopacity{0.700000}%
\pgfsetlinewidth{0.000000pt}%
\definecolor{currentstroke}{rgb}{0.000000,0.000000,0.000000}%
\pgfsetstrokecolor{currentstroke}%
\pgfsetdash{}{0pt}%
\pgfpathmoveto{\pgfqpoint{4.845841in}{2.151923in}}%
\pgfpathlineto{\pgfqpoint{4.859200in}{2.150285in}}%
\pgfpathlineto{\pgfqpoint{4.872567in}{2.148671in}}%
\pgfpathlineto{\pgfqpoint{4.885941in}{2.147081in}}%
\pgfpathlineto{\pgfqpoint{4.899323in}{2.145516in}}%
\pgfpathlineto{\pgfqpoint{4.892016in}{2.137305in}}%
\pgfpathlineto{\pgfqpoint{4.884703in}{2.129033in}}%
\pgfpathlineto{\pgfqpoint{4.877384in}{2.120702in}}%
\pgfpathlineto{\pgfqpoint{4.870060in}{2.112313in}}%
\pgfpathlineto{\pgfqpoint{4.856667in}{2.113931in}}%
\pgfpathlineto{\pgfqpoint{4.843283in}{2.115573in}}%
\pgfpathlineto{\pgfqpoint{4.829906in}{2.117240in}}%
\pgfpathlineto{\pgfqpoint{4.816536in}{2.118931in}}%
\pgfpathlineto{\pgfqpoint{4.823871in}{2.127262in}}%
\pgfpathlineto{\pgfqpoint{4.831201in}{2.135539in}}%
\pgfpathlineto{\pgfqpoint{4.838524in}{2.143759in}}%
\pgfpathlineto{\pgfqpoint{4.845841in}{2.151923in}}%
\pgfpathclose%
\pgfusepath{fill}%
\end{pgfscope}%
\begin{pgfscope}%
\pgfpathrectangle{\pgfqpoint{1.254980in}{0.150000in}}{\pgfqpoint{5.490039in}{5.490039in}}%
\pgfusepath{clip}%
\pgfsetbuttcap%
\pgfsetroundjoin%
\definecolor{currentfill}{rgb}{0.283187,0.125848,0.444960}%
\pgfsetfillcolor{currentfill}%
\pgfsetfillopacity{0.700000}%
\pgfsetlinewidth{0.000000pt}%
\definecolor{currentstroke}{rgb}{0.000000,0.000000,0.000000}%
\pgfsetstrokecolor{currentstroke}%
\pgfsetdash{}{0pt}%
\pgfpathmoveto{\pgfqpoint{5.147290in}{2.223258in}}%
\pgfpathlineto{\pgfqpoint{5.160739in}{2.222088in}}%
\pgfpathlineto{\pgfqpoint{5.174196in}{2.220941in}}%
\pgfpathlineto{\pgfqpoint{5.187661in}{2.219818in}}%
\pgfpathlineto{\pgfqpoint{5.201134in}{2.218719in}}%
\pgfpathlineto{\pgfqpoint{5.193950in}{2.211227in}}%
\pgfpathlineto{\pgfqpoint{5.186760in}{2.203659in}}%
\pgfpathlineto{\pgfqpoint{5.179563in}{2.196013in}}%
\pgfpathlineto{\pgfqpoint{5.172359in}{2.188291in}}%
\pgfpathlineto{\pgfqpoint{5.158875in}{2.189403in}}%
\pgfpathlineto{\pgfqpoint{5.145398in}{2.190540in}}%
\pgfpathlineto{\pgfqpoint{5.131929in}{2.191700in}}%
\pgfpathlineto{\pgfqpoint{5.118468in}{2.192885in}}%
\pgfpathlineto{\pgfqpoint{5.125684in}{2.200588in}}%
\pgfpathlineto{\pgfqpoint{5.132893in}{2.208219in}}%
\pgfpathlineto{\pgfqpoint{5.140095in}{2.215775in}}%
\pgfpathlineto{\pgfqpoint{5.147290in}{2.223258in}}%
\pgfpathclose%
\pgfusepath{fill}%
\end{pgfscope}%
\begin{pgfscope}%
\pgfpathrectangle{\pgfqpoint{1.254980in}{0.150000in}}{\pgfqpoint{5.490039in}{5.490039in}}%
\pgfusepath{clip}%
\pgfsetbuttcap%
\pgfsetroundjoin%
\definecolor{currentfill}{rgb}{0.277018,0.050344,0.375715}%
\pgfsetfillcolor{currentfill}%
\pgfsetfillopacity{0.700000}%
\pgfsetlinewidth{0.000000pt}%
\definecolor{currentstroke}{rgb}{0.000000,0.000000,0.000000}%
\pgfsetstrokecolor{currentstroke}%
\pgfsetdash{}{0pt}%
\pgfpathmoveto{\pgfqpoint{4.544402in}{2.083165in}}%
\pgfpathlineto{\pgfqpoint{4.557675in}{2.080935in}}%
\pgfpathlineto{\pgfqpoint{4.570956in}{2.078730in}}%
\pgfpathlineto{\pgfqpoint{4.584244in}{2.076550in}}%
\pgfpathlineto{\pgfqpoint{4.597539in}{2.074394in}}%
\pgfpathlineto{\pgfqpoint{4.590121in}{2.065888in}}%
\pgfpathlineto{\pgfqpoint{4.582698in}{2.057351in}}%
\pgfpathlineto{\pgfqpoint{4.575269in}{2.048786in}}%
\pgfpathlineto{\pgfqpoint{4.567835in}{2.040193in}}%
\pgfpathlineto{\pgfqpoint{4.554530in}{2.042441in}}%
\pgfpathlineto{\pgfqpoint{4.541232in}{2.044713in}}%
\pgfpathlineto{\pgfqpoint{4.527940in}{2.047010in}}%
\pgfpathlineto{\pgfqpoint{4.514656in}{2.049331in}}%
\pgfpathlineto{\pgfqpoint{4.522101in}{2.057827in}}%
\pgfpathlineto{\pgfqpoint{4.529540in}{2.066299in}}%
\pgfpathlineto{\pgfqpoint{4.536973in}{2.074746in}}%
\pgfpathlineto{\pgfqpoint{4.544402in}{2.083165in}}%
\pgfpathclose%
\pgfusepath{fill}%
\end{pgfscope}%
\begin{pgfscope}%
\pgfpathrectangle{\pgfqpoint{1.254980in}{0.150000in}}{\pgfqpoint{5.490039in}{5.490039in}}%
\pgfusepath{clip}%
\pgfsetbuttcap%
\pgfsetroundjoin%
\definecolor{currentfill}{rgb}{0.280255,0.165693,0.476498}%
\pgfsetfillcolor{currentfill}%
\pgfsetfillopacity{0.700000}%
\pgfsetlinewidth{0.000000pt}%
\definecolor{currentstroke}{rgb}{0.000000,0.000000,0.000000}%
\pgfsetstrokecolor{currentstroke}%
\pgfsetdash{}{0pt}%
\pgfpathmoveto{\pgfqpoint{2.782222in}{2.287892in}}%
\pgfpathlineto{\pgfqpoint{2.795179in}{2.280288in}}%
\pgfpathlineto{\pgfqpoint{2.808138in}{2.272726in}}%
\pgfpathlineto{\pgfqpoint{2.821101in}{2.265206in}}%
\pgfpathlineto{\pgfqpoint{2.834066in}{2.257726in}}%
\pgfpathlineto{\pgfqpoint{2.825797in}{2.258765in}}%
\pgfpathlineto{\pgfqpoint{2.817512in}{2.260079in}}%
\pgfpathlineto{\pgfqpoint{2.809211in}{2.261675in}}%
\pgfpathlineto{\pgfqpoint{2.800894in}{2.263559in}}%
\pgfpathlineto{\pgfqpoint{2.787899in}{2.271301in}}%
\pgfpathlineto{\pgfqpoint{2.774907in}{2.279084in}}%
\pgfpathlineto{\pgfqpoint{2.761917in}{2.286908in}}%
\pgfpathlineto{\pgfqpoint{2.748930in}{2.294774in}}%
\pgfpathlineto{\pgfqpoint{2.757278in}{2.292622in}}%
\pgfpathlineto{\pgfqpoint{2.765609in}{2.290763in}}%
\pgfpathlineto{\pgfqpoint{2.773924in}{2.289188in}}%
\pgfpathlineto{\pgfqpoint{2.782222in}{2.287892in}}%
\pgfpathclose%
\pgfusepath{fill}%
\end{pgfscope}%
\begin{pgfscope}%
\pgfpathrectangle{\pgfqpoint{1.254980in}{0.150000in}}{\pgfqpoint{5.490039in}{5.490039in}}%
\pgfusepath{clip}%
\pgfsetbuttcap%
\pgfsetroundjoin%
\definecolor{currentfill}{rgb}{0.267004,0.004874,0.329415}%
\pgfsetfillcolor{currentfill}%
\pgfsetfillopacity{0.700000}%
\pgfsetlinewidth{0.000000pt}%
\definecolor{currentstroke}{rgb}{0.000000,0.000000,0.000000}%
\pgfsetstrokecolor{currentstroke}%
\pgfsetdash{}{0pt}%
\pgfpathmoveto{\pgfqpoint{3.888790in}{2.002388in}}%
\pgfpathlineto{\pgfqpoint{3.901902in}{1.998482in}}%
\pgfpathlineto{\pgfqpoint{3.915020in}{1.994604in}}%
\pgfpathlineto{\pgfqpoint{3.928144in}{1.990753in}}%
\pgfpathlineto{\pgfqpoint{3.941274in}{1.986928in}}%
\pgfpathlineto{\pgfqpoint{3.933620in}{1.979870in}}%
\pgfpathlineto{\pgfqpoint{3.925961in}{1.972881in}}%
\pgfpathlineto{\pgfqpoint{3.918296in}{1.965965in}}%
\pgfpathlineto{\pgfqpoint{3.910624in}{1.959125in}}%
\pgfpathlineto{\pgfqpoint{3.897481in}{1.963118in}}%
\pgfpathlineto{\pgfqpoint{3.884343in}{1.967137in}}%
\pgfpathlineto{\pgfqpoint{3.871211in}{1.971184in}}%
\pgfpathlineto{\pgfqpoint{3.858085in}{1.975258in}}%
\pgfpathlineto{\pgfqpoint{3.865771in}{1.981924in}}%
\pgfpathlineto{\pgfqpoint{3.873450in}{1.988671in}}%
\pgfpathlineto{\pgfqpoint{3.881123in}{1.995493in}}%
\pgfpathlineto{\pgfqpoint{3.888790in}{2.002388in}}%
\pgfpathclose%
\pgfusepath{fill}%
\end{pgfscope}%
\begin{pgfscope}%
\pgfpathrectangle{\pgfqpoint{1.254980in}{0.150000in}}{\pgfqpoint{5.490039in}{5.490039in}}%
\pgfusepath{clip}%
\pgfsetbuttcap%
\pgfsetroundjoin%
\definecolor{currentfill}{rgb}{0.265145,0.232956,0.516599}%
\pgfsetfillcolor{currentfill}%
\pgfsetfillopacity{0.700000}%
\pgfsetlinewidth{0.000000pt}%
\definecolor{currentstroke}{rgb}{0.000000,0.000000,0.000000}%
\pgfsetstrokecolor{currentstroke}%
\pgfsetdash{}{0pt}%
\pgfpathmoveto{\pgfqpoint{2.541460in}{2.426657in}}%
\pgfpathlineto{\pgfqpoint{2.554411in}{2.418065in}}%
\pgfpathlineto{\pgfqpoint{2.567364in}{2.409521in}}%
\pgfpathlineto{\pgfqpoint{2.580319in}{2.401026in}}%
\pgfpathlineto{\pgfqpoint{2.593276in}{2.392579in}}%
\pgfpathlineto{\pgfqpoint{2.584814in}{2.395847in}}%
\pgfpathlineto{\pgfqpoint{2.576333in}{2.399432in}}%
\pgfpathlineto{\pgfqpoint{2.567833in}{2.403342in}}%
\pgfpathlineto{\pgfqpoint{2.559312in}{2.407584in}}%
\pgfpathlineto{\pgfqpoint{2.546321in}{2.416310in}}%
\pgfpathlineto{\pgfqpoint{2.533332in}{2.425084in}}%
\pgfpathlineto{\pgfqpoint{2.520345in}{2.433907in}}%
\pgfpathlineto{\pgfqpoint{2.507359in}{2.442778in}}%
\pgfpathlineto{\pgfqpoint{2.515915in}{2.438252in}}%
\pgfpathlineto{\pgfqpoint{2.524449in}{2.434061in}}%
\pgfpathlineto{\pgfqpoint{2.532964in}{2.430199in}}%
\pgfpathlineto{\pgfqpoint{2.541460in}{2.426657in}}%
\pgfpathclose%
\pgfusepath{fill}%
\end{pgfscope}%
\begin{pgfscope}%
\pgfpathrectangle{\pgfqpoint{1.254980in}{0.150000in}}{\pgfqpoint{5.490039in}{5.490039in}}%
\pgfusepath{clip}%
\pgfsetbuttcap%
\pgfsetroundjoin%
\definecolor{currentfill}{rgb}{0.276022,0.044167,0.370164}%
\pgfsetfillcolor{currentfill}%
\pgfsetfillopacity{0.700000}%
\pgfsetlinewidth{0.000000pt}%
\definecolor{currentstroke}{rgb}{0.000000,0.000000,0.000000}%
\pgfsetstrokecolor{currentstroke}%
\pgfsetdash{}{0pt}%
\pgfpathmoveto{\pgfqpoint{3.346491in}{2.067224in}}%
\pgfpathlineto{\pgfqpoint{3.359505in}{2.061630in}}%
\pgfpathlineto{\pgfqpoint{3.372525in}{2.056068in}}%
\pgfpathlineto{\pgfqpoint{3.385549in}{2.050537in}}%
\pgfpathlineto{\pgfqpoint{3.398577in}{2.045038in}}%
\pgfpathlineto{\pgfqpoint{3.390672in}{2.041337in}}%
\pgfpathlineto{\pgfqpoint{3.382758in}{2.037807in}}%
\pgfpathlineto{\pgfqpoint{3.374834in}{2.034453in}}%
\pgfpathlineto{\pgfqpoint{3.366901in}{2.031281in}}%
\pgfpathlineto{\pgfqpoint{3.353852in}{2.037001in}}%
\pgfpathlineto{\pgfqpoint{3.340807in}{2.042751in}}%
\pgfpathlineto{\pgfqpoint{3.327767in}{2.048534in}}%
\pgfpathlineto{\pgfqpoint{3.314732in}{2.054348in}}%
\pgfpathlineto{\pgfqpoint{3.322686in}{2.057295in}}%
\pgfpathlineto{\pgfqpoint{3.330631in}{2.060427in}}%
\pgfpathlineto{\pgfqpoint{3.338566in}{2.063738in}}%
\pgfpathlineto{\pgfqpoint{3.346491in}{2.067224in}}%
\pgfpathclose%
\pgfusepath{fill}%
\end{pgfscope}%
\begin{pgfscope}%
\pgfpathrectangle{\pgfqpoint{1.254980in}{0.150000in}}{\pgfqpoint{5.490039in}{5.490039in}}%
\pgfusepath{clip}%
\pgfsetbuttcap%
\pgfsetroundjoin%
\definecolor{currentfill}{rgb}{0.277134,0.185228,0.489898}%
\pgfsetfillcolor{currentfill}%
\pgfsetfillopacity{0.700000}%
\pgfsetlinewidth{0.000000pt}%
\definecolor{currentstroke}{rgb}{0.000000,0.000000,0.000000}%
\pgfsetstrokecolor{currentstroke}%
\pgfsetdash{}{0pt}%
\pgfpathmoveto{\pgfqpoint{5.667493in}{2.331246in}}%
\pgfpathlineto{\pgfqpoint{5.681104in}{2.330611in}}%
\pgfpathlineto{\pgfqpoint{5.694723in}{2.329999in}}%
\pgfpathlineto{\pgfqpoint{5.708351in}{2.329412in}}%
\pgfpathlineto{\pgfqpoint{5.721988in}{2.328848in}}%
\pgfpathlineto{\pgfqpoint{5.715053in}{2.323108in}}%
\pgfpathlineto{\pgfqpoint{5.708109in}{2.317293in}}%
\pgfpathlineto{\pgfqpoint{5.701158in}{2.311398in}}%
\pgfpathlineto{\pgfqpoint{5.694199in}{2.305422in}}%
\pgfpathlineto{\pgfqpoint{5.680546in}{2.305934in}}%
\pgfpathlineto{\pgfqpoint{5.666902in}{2.306469in}}%
\pgfpathlineto{\pgfqpoint{5.653267in}{2.307027in}}%
\pgfpathlineto{\pgfqpoint{5.639640in}{2.307610in}}%
\pgfpathlineto{\pgfqpoint{5.646615in}{2.313633in}}%
\pgfpathlineto{\pgfqpoint{5.653582in}{2.319579in}}%
\pgfpathlineto{\pgfqpoint{5.660542in}{2.325449in}}%
\pgfpathlineto{\pgfqpoint{5.667493in}{2.331246in}}%
\pgfpathclose%
\pgfusepath{fill}%
\end{pgfscope}%
\begin{pgfscope}%
\pgfpathrectangle{\pgfqpoint{1.254980in}{0.150000in}}{\pgfqpoint{5.490039in}{5.490039in}}%
\pgfusepath{clip}%
\pgfsetbuttcap%
\pgfsetroundjoin%
\definecolor{currentfill}{rgb}{0.273006,0.204520,0.501721}%
\pgfsetfillcolor{currentfill}%
\pgfsetfillopacity{0.700000}%
\pgfsetlinewidth{0.000000pt}%
\definecolor{currentstroke}{rgb}{0.000000,0.000000,0.000000}%
\pgfsetstrokecolor{currentstroke}%
\pgfsetdash{}{0pt}%
\pgfpathmoveto{\pgfqpoint{5.886319in}{2.367437in}}%
\pgfpathlineto{\pgfqpoint{5.900000in}{2.366940in}}%
\pgfpathlineto{\pgfqpoint{5.913689in}{2.366465in}}%
\pgfpathlineto{\pgfqpoint{5.927387in}{2.366014in}}%
\pgfpathlineto{\pgfqpoint{5.920563in}{2.361039in}}%
\pgfpathlineto{\pgfqpoint{5.913731in}{2.355996in}}%
\pgfpathlineto{\pgfqpoint{5.906891in}{2.350884in}}%
\pgfpathlineto{\pgfqpoint{5.900043in}{2.345701in}}%
\pgfpathlineto{\pgfqpoint{5.886327in}{2.346072in}}%
\pgfpathlineto{\pgfqpoint{5.872620in}{2.346467in}}%
\pgfpathlineto{\pgfqpoint{5.858921in}{2.346885in}}%
\pgfpathlineto{\pgfqpoint{5.865783in}{2.352125in}}%
\pgfpathlineto{\pgfqpoint{5.872636in}{2.357295in}}%
\pgfpathlineto{\pgfqpoint{5.879482in}{2.362398in}}%
\pgfpathlineto{\pgfqpoint{5.886319in}{2.367437in}}%
\pgfpathclose%
\pgfusepath{fill}%
\end{pgfscope}%
\begin{pgfscope}%
\pgfpathrectangle{\pgfqpoint{1.254980in}{0.150000in}}{\pgfqpoint{5.490039in}{5.490039in}}%
\pgfusepath{clip}%
\pgfsetbuttcap%
\pgfsetroundjoin%
\definecolor{currentfill}{rgb}{0.269944,0.014625,0.341379}%
\pgfsetfillcolor{currentfill}%
\pgfsetfillopacity{0.700000}%
\pgfsetlinewidth{0.000000pt}%
\definecolor{currentstroke}{rgb}{0.000000,0.000000,0.000000}%
\pgfsetstrokecolor{currentstroke}%
\pgfsetdash{}{0pt}%
\pgfpathmoveto{\pgfqpoint{4.242950in}{2.024911in}}%
\pgfpathlineto{\pgfqpoint{4.256148in}{2.021963in}}%
\pgfpathlineto{\pgfqpoint{4.269352in}{2.019041in}}%
\pgfpathlineto{\pgfqpoint{4.282563in}{2.016144in}}%
\pgfpathlineto{\pgfqpoint{4.295780in}{2.013272in}}%
\pgfpathlineto{\pgfqpoint{4.288255in}{2.005046in}}%
\pgfpathlineto{\pgfqpoint{4.280725in}{1.996832in}}%
\pgfpathlineto{\pgfqpoint{4.273190in}{1.988631in}}%
\pgfpathlineto{\pgfqpoint{4.265649in}{1.980447in}}%
\pgfpathlineto{\pgfqpoint{4.252421in}{1.983449in}}%
\pgfpathlineto{\pgfqpoint{4.239199in}{1.986476in}}%
\pgfpathlineto{\pgfqpoint{4.225983in}{1.989529in}}%
\pgfpathlineto{\pgfqpoint{4.212774in}{1.992607in}}%
\pgfpathlineto{\pgfqpoint{4.220326in}{2.000656in}}%
\pgfpathlineto{\pgfqpoint{4.227873in}{2.008725in}}%
\pgfpathlineto{\pgfqpoint{4.235414in}{2.016811in}}%
\pgfpathlineto{\pgfqpoint{4.242950in}{2.024911in}}%
\pgfpathclose%
\pgfusepath{fill}%
\end{pgfscope}%
\begin{pgfscope}%
\pgfpathrectangle{\pgfqpoint{1.254980in}{0.150000in}}{\pgfqpoint{5.490039in}{5.490039in}}%
\pgfusepath{clip}%
\pgfsetbuttcap%
\pgfsetroundjoin%
\definecolor{currentfill}{rgb}{0.280894,0.078907,0.402329}%
\pgfsetfillcolor{currentfill}%
\pgfsetfillopacity{0.700000}%
\pgfsetlinewidth{0.000000pt}%
\definecolor{currentstroke}{rgb}{0.000000,0.000000,0.000000}%
\pgfsetstrokecolor{currentstroke}%
\pgfsetdash{}{0pt}%
\pgfpathmoveto{\pgfqpoint{4.763133in}{2.125940in}}%
\pgfpathlineto{\pgfqpoint{4.776472in}{2.124151in}}%
\pgfpathlineto{\pgfqpoint{4.789820in}{2.122387in}}%
\pgfpathlineto{\pgfqpoint{4.803174in}{2.120647in}}%
\pgfpathlineto{\pgfqpoint{4.816536in}{2.118931in}}%
\pgfpathlineto{\pgfqpoint{4.809195in}{2.110546in}}%
\pgfpathlineto{\pgfqpoint{4.801848in}{2.102109in}}%
\pgfpathlineto{\pgfqpoint{4.794495in}{2.093619in}}%
\pgfpathlineto{\pgfqpoint{4.787136in}{2.085080in}}%
\pgfpathlineto{\pgfqpoint{4.773764in}{2.086862in}}%
\pgfpathlineto{\pgfqpoint{4.760399in}{2.088668in}}%
\pgfpathlineto{\pgfqpoint{4.747041in}{2.090498in}}%
\pgfpathlineto{\pgfqpoint{4.733691in}{2.092353in}}%
\pgfpathlineto{\pgfqpoint{4.741060in}{2.100821in}}%
\pgfpathlineto{\pgfqpoint{4.748424in}{2.109243in}}%
\pgfpathlineto{\pgfqpoint{4.755781in}{2.117616in}}%
\pgfpathlineto{\pgfqpoint{4.763133in}{2.125940in}}%
\pgfpathclose%
\pgfusepath{fill}%
\end{pgfscope}%
\begin{pgfscope}%
\pgfpathrectangle{\pgfqpoint{1.254980in}{0.150000in}}{\pgfqpoint{5.490039in}{5.490039in}}%
\pgfusepath{clip}%
\pgfsetbuttcap%
\pgfsetroundjoin%
\definecolor{currentfill}{rgb}{0.281412,0.155834,0.469201}%
\pgfsetfillcolor{currentfill}%
\pgfsetfillopacity{0.700000}%
\pgfsetlinewidth{0.000000pt}%
\definecolor{currentstroke}{rgb}{0.000000,0.000000,0.000000}%
\pgfsetstrokecolor{currentstroke}%
\pgfsetdash{}{0pt}%
\pgfpathmoveto{\pgfqpoint{5.366220in}{2.267926in}}%
\pgfpathlineto{\pgfqpoint{5.379741in}{2.267030in}}%
\pgfpathlineto{\pgfqpoint{5.393270in}{2.266158in}}%
\pgfpathlineto{\pgfqpoint{5.406807in}{2.265309in}}%
\pgfpathlineto{\pgfqpoint{5.420352in}{2.264484in}}%
\pgfpathlineto{\pgfqpoint{5.413265in}{2.257640in}}%
\pgfpathlineto{\pgfqpoint{5.406170in}{2.250713in}}%
\pgfpathlineto{\pgfqpoint{5.399068in}{2.243705in}}%
\pgfpathlineto{\pgfqpoint{5.391958in}{2.236613in}}%
\pgfpathlineto{\pgfqpoint{5.378399in}{2.237425in}}%
\pgfpathlineto{\pgfqpoint{5.364849in}{2.238260in}}%
\pgfpathlineto{\pgfqpoint{5.351307in}{2.239120in}}%
\pgfpathlineto{\pgfqpoint{5.337773in}{2.240003in}}%
\pgfpathlineto{\pgfqpoint{5.344896in}{2.247103in}}%
\pgfpathlineto{\pgfqpoint{5.352011in}{2.254123in}}%
\pgfpathlineto{\pgfqpoint{5.359119in}{2.261064in}}%
\pgfpathlineto{\pgfqpoint{5.366220in}{2.267926in}}%
\pgfpathclose%
\pgfusepath{fill}%
\end{pgfscope}%
\begin{pgfscope}%
\pgfpathrectangle{\pgfqpoint{1.254980in}{0.150000in}}{\pgfqpoint{5.490039in}{5.490039in}}%
\pgfusepath{clip}%
\pgfsetbuttcap%
\pgfsetroundjoin%
\definecolor{currentfill}{rgb}{0.283197,0.115680,0.436115}%
\pgfsetfillcolor{currentfill}%
\pgfsetfillopacity{0.700000}%
\pgfsetlinewidth{0.000000pt}%
\definecolor{currentstroke}{rgb}{0.000000,0.000000,0.000000}%
\pgfsetstrokecolor{currentstroke}%
\pgfsetdash{}{0pt}%
\pgfpathmoveto{\pgfqpoint{5.064705in}{2.197863in}}%
\pgfpathlineto{\pgfqpoint{5.078134in}{2.196582in}}%
\pgfpathlineto{\pgfqpoint{5.091571in}{2.195326in}}%
\pgfpathlineto{\pgfqpoint{5.105016in}{2.194093in}}%
\pgfpathlineto{\pgfqpoint{5.118468in}{2.192885in}}%
\pgfpathlineto{\pgfqpoint{5.111246in}{2.185107in}}%
\pgfpathlineto{\pgfqpoint{5.104017in}{2.177256in}}%
\pgfpathlineto{\pgfqpoint{5.096782in}{2.169332in}}%
\pgfpathlineto{\pgfqpoint{5.089540in}{2.161334in}}%
\pgfpathlineto{\pgfqpoint{5.076075in}{2.162570in}}%
\pgfpathlineto{\pgfqpoint{5.062619in}{2.163829in}}%
\pgfpathlineto{\pgfqpoint{5.049171in}{2.165113in}}%
\pgfpathlineto{\pgfqpoint{5.035730in}{2.166420in}}%
\pgfpathlineto{\pgfqpoint{5.042984in}{2.174386in}}%
\pgfpathlineto{\pgfqpoint{5.050231in}{2.182282in}}%
\pgfpathlineto{\pgfqpoint{5.057471in}{2.190107in}}%
\pgfpathlineto{\pgfqpoint{5.064705in}{2.197863in}}%
\pgfpathclose%
\pgfusepath{fill}%
\end{pgfscope}%
\begin{pgfscope}%
\pgfpathrectangle{\pgfqpoint{1.254980in}{0.150000in}}{\pgfqpoint{5.490039in}{5.490039in}}%
\pgfusepath{clip}%
\pgfsetbuttcap%
\pgfsetroundjoin%
\definecolor{currentfill}{rgb}{0.267004,0.004874,0.329415}%
\pgfsetfillcolor{currentfill}%
\pgfsetfillopacity{0.700000}%
\pgfsetlinewidth{0.000000pt}%
\definecolor{currentstroke}{rgb}{0.000000,0.000000,0.000000}%
\pgfsetstrokecolor{currentstroke}%
\pgfsetdash{}{0pt}%
\pgfpathmoveto{\pgfqpoint{4.024351in}{2.001364in}}%
\pgfpathlineto{\pgfqpoint{4.037497in}{1.997829in}}%
\pgfpathlineto{\pgfqpoint{4.050649in}{1.994321in}}%
\pgfpathlineto{\pgfqpoint{4.063808in}{1.990839in}}%
\pgfpathlineto{\pgfqpoint{4.076972in}{1.987384in}}%
\pgfpathlineto{\pgfqpoint{4.069369in}{1.979788in}}%
\pgfpathlineto{\pgfqpoint{4.061759in}{1.972239in}}%
\pgfpathlineto{\pgfqpoint{4.054144in}{1.964741in}}%
\pgfpathlineto{\pgfqpoint{4.046523in}{1.957297in}}%
\pgfpathlineto{\pgfqpoint{4.033346in}{1.960909in}}%
\pgfpathlineto{\pgfqpoint{4.020175in}{1.964546in}}%
\pgfpathlineto{\pgfqpoint{4.007010in}{1.968210in}}%
\pgfpathlineto{\pgfqpoint{3.993851in}{1.971900in}}%
\pgfpathlineto{\pgfqpoint{4.001484in}{1.979183in}}%
\pgfpathlineto{\pgfqpoint{4.009112in}{1.986524in}}%
\pgfpathlineto{\pgfqpoint{4.016735in}{1.993919in}}%
\pgfpathlineto{\pgfqpoint{4.024351in}{2.001364in}}%
\pgfpathclose%
\pgfusepath{fill}%
\end{pgfscope}%
\begin{pgfscope}%
\pgfpathrectangle{\pgfqpoint{1.254980in}{0.150000in}}{\pgfqpoint{5.490039in}{5.490039in}}%
\pgfusepath{clip}%
\pgfsetbuttcap%
\pgfsetroundjoin%
\definecolor{currentfill}{rgb}{0.274952,0.037752,0.364543}%
\pgfsetfillcolor{currentfill}%
\pgfsetfillopacity{0.700000}%
\pgfsetlinewidth{0.000000pt}%
\definecolor{currentstroke}{rgb}{0.000000,0.000000,0.000000}%
\pgfsetstrokecolor{currentstroke}%
\pgfsetdash{}{0pt}%
\pgfpathmoveto{\pgfqpoint{4.461589in}{2.058868in}}%
\pgfpathlineto{\pgfqpoint{4.474846in}{2.056446in}}%
\pgfpathlineto{\pgfqpoint{4.488109in}{2.054050in}}%
\pgfpathlineto{\pgfqpoint{4.501379in}{2.051678in}}%
\pgfpathlineto{\pgfqpoint{4.514656in}{2.049331in}}%
\pgfpathlineto{\pgfqpoint{4.507206in}{2.040814in}}%
\pgfpathlineto{\pgfqpoint{4.499751in}{2.032278in}}%
\pgfpathlineto{\pgfqpoint{4.492291in}{2.023724in}}%
\pgfpathlineto{\pgfqpoint{4.484825in}{2.015156in}}%
\pgfpathlineto{\pgfqpoint{4.471537in}{2.017607in}}%
\pgfpathlineto{\pgfqpoint{4.458256in}{2.020084in}}%
\pgfpathlineto{\pgfqpoint{4.444983in}{2.022585in}}%
\pgfpathlineto{\pgfqpoint{4.431716in}{2.025112in}}%
\pgfpathlineto{\pgfqpoint{4.439192in}{2.033570in}}%
\pgfpathlineto{\pgfqpoint{4.446663in}{2.042017in}}%
\pgfpathlineto{\pgfqpoint{4.454129in}{2.050451in}}%
\pgfpathlineto{\pgfqpoint{4.461589in}{2.058868in}}%
\pgfpathclose%
\pgfusepath{fill}%
\end{pgfscope}%
\begin{pgfscope}%
\pgfpathrectangle{\pgfqpoint{1.254980in}{0.150000in}}{\pgfqpoint{5.490039in}{5.490039in}}%
\pgfusepath{clip}%
\pgfsetbuttcap%
\pgfsetroundjoin%
\definecolor{currentfill}{rgb}{0.282910,0.105393,0.426902}%
\pgfsetfillcolor{currentfill}%
\pgfsetfillopacity{0.700000}%
\pgfsetlinewidth{0.000000pt}%
\definecolor{currentstroke}{rgb}{0.000000,0.000000,0.000000}%
\pgfsetstrokecolor{currentstroke}%
\pgfsetdash{}{0pt}%
\pgfpathmoveto{\pgfqpoint{3.022489in}{2.172480in}}%
\pgfpathlineto{\pgfqpoint{3.035469in}{2.165751in}}%
\pgfpathlineto{\pgfqpoint{3.048453in}{2.159058in}}%
\pgfpathlineto{\pgfqpoint{3.061440in}{2.152401in}}%
\pgfpathlineto{\pgfqpoint{3.074431in}{2.145780in}}%
\pgfpathlineto{\pgfqpoint{3.066326in}{2.144816in}}%
\pgfpathlineto{\pgfqpoint{3.058208in}{2.144086in}}%
\pgfpathlineto{\pgfqpoint{3.050078in}{2.143597in}}%
\pgfpathlineto{\pgfqpoint{3.041933in}{2.143355in}}%
\pgfpathlineto{\pgfqpoint{3.028917in}{2.150224in}}%
\pgfpathlineto{\pgfqpoint{3.015904in}{2.157128in}}%
\pgfpathlineto{\pgfqpoint{3.002894in}{2.164068in}}%
\pgfpathlineto{\pgfqpoint{2.989888in}{2.171045in}}%
\pgfpathlineto{\pgfqpoint{2.998058in}{2.171034in}}%
\pgfpathlineto{\pgfqpoint{3.006215in}{2.171274in}}%
\pgfpathlineto{\pgfqpoint{3.014359in}{2.171758in}}%
\pgfpathlineto{\pgfqpoint{3.022489in}{2.172480in}}%
\pgfpathclose%
\pgfusepath{fill}%
\end{pgfscope}%
\begin{pgfscope}%
\pgfpathrectangle{\pgfqpoint{1.254980in}{0.150000in}}{\pgfqpoint{5.490039in}{5.490039in}}%
\pgfusepath{clip}%
\pgfsetbuttcap%
\pgfsetroundjoin%
\definecolor{currentfill}{rgb}{0.279566,0.067836,0.391917}%
\pgfsetfillcolor{currentfill}%
\pgfsetfillopacity{0.700000}%
\pgfsetlinewidth{0.000000pt}%
\definecolor{currentstroke}{rgb}{0.000000,0.000000,0.000000}%
\pgfsetstrokecolor{currentstroke}%
\pgfsetdash{}{0pt}%
\pgfpathmoveto{\pgfqpoint{3.210599in}{2.102027in}}%
\pgfpathlineto{\pgfqpoint{3.223601in}{2.095952in}}%
\pgfpathlineto{\pgfqpoint{3.236607in}{2.089910in}}%
\pgfpathlineto{\pgfqpoint{3.249617in}{2.083901in}}%
\pgfpathlineto{\pgfqpoint{3.262632in}{2.077925in}}%
\pgfpathlineto{\pgfqpoint{3.254644in}{2.075398in}}%
\pgfpathlineto{\pgfqpoint{3.246647in}{2.073070in}}%
\pgfpathlineto{\pgfqpoint{3.238638in}{2.070949in}}%
\pgfpathlineto{\pgfqpoint{3.230617in}{2.069039in}}%
\pgfpathlineto{\pgfqpoint{3.217580in}{2.075249in}}%
\pgfpathlineto{\pgfqpoint{3.204547in}{2.081491in}}%
\pgfpathlineto{\pgfqpoint{3.191518in}{2.087767in}}%
\pgfpathlineto{\pgfqpoint{3.178493in}{2.094076in}}%
\pgfpathlineto{\pgfqpoint{3.186536in}{2.095746in}}%
\pgfpathlineto{\pgfqpoint{3.194569in}{2.097633in}}%
\pgfpathlineto{\pgfqpoint{3.202589in}{2.099728in}}%
\pgfpathlineto{\pgfqpoint{3.210599in}{2.102027in}}%
\pgfpathclose%
\pgfusepath{fill}%
\end{pgfscope}%
\begin{pgfscope}%
\pgfpathrectangle{\pgfqpoint{1.254980in}{0.150000in}}{\pgfqpoint{5.490039in}{5.490039in}}%
\pgfusepath{clip}%
\pgfsetbuttcap%
\pgfsetroundjoin%
\definecolor{currentfill}{rgb}{0.267968,0.223549,0.512008}%
\pgfsetfillcolor{currentfill}%
\pgfsetfillopacity{0.700000}%
\pgfsetlinewidth{0.000000pt}%
\definecolor{currentstroke}{rgb}{0.000000,0.000000,0.000000}%
\pgfsetstrokecolor{currentstroke}%
\pgfsetdash{}{0pt}%
\pgfpathmoveto{\pgfqpoint{2.593276in}{2.392579in}}%
\pgfpathlineto{\pgfqpoint{2.606235in}{2.384178in}}%
\pgfpathlineto{\pgfqpoint{2.619196in}{2.375825in}}%
\pgfpathlineto{\pgfqpoint{2.632159in}{2.367519in}}%
\pgfpathlineto{\pgfqpoint{2.645124in}{2.359258in}}%
\pgfpathlineto{\pgfqpoint{2.636695in}{2.362253in}}%
\pgfpathlineto{\pgfqpoint{2.628248in}{2.365562in}}%
\pgfpathlineto{\pgfqpoint{2.619781in}{2.369192in}}%
\pgfpathlineto{\pgfqpoint{2.611295in}{2.373151in}}%
\pgfpathlineto{\pgfqpoint{2.598297in}{2.381690in}}%
\pgfpathlineto{\pgfqpoint{2.585300in}{2.390275in}}%
\pgfpathlineto{\pgfqpoint{2.572305in}{2.398906in}}%
\pgfpathlineto{\pgfqpoint{2.559312in}{2.407584in}}%
\pgfpathlineto{\pgfqpoint{2.567833in}{2.403342in}}%
\pgfpathlineto{\pgfqpoint{2.576333in}{2.399432in}}%
\pgfpathlineto{\pgfqpoint{2.584814in}{2.395847in}}%
\pgfpathlineto{\pgfqpoint{2.593276in}{2.392579in}}%
\pgfpathclose%
\pgfusepath{fill}%
\end{pgfscope}%
\begin{pgfscope}%
\pgfpathrectangle{\pgfqpoint{1.254980in}{0.150000in}}{\pgfqpoint{5.490039in}{5.490039in}}%
\pgfusepath{clip}%
\pgfsetbuttcap%
\pgfsetroundjoin%
\definecolor{currentfill}{rgb}{0.281412,0.155834,0.469201}%
\pgfsetfillcolor{currentfill}%
\pgfsetfillopacity{0.700000}%
\pgfsetlinewidth{0.000000pt}%
\definecolor{currentstroke}{rgb}{0.000000,0.000000,0.000000}%
\pgfsetstrokecolor{currentstroke}%
\pgfsetdash{}{0pt}%
\pgfpathmoveto{\pgfqpoint{2.834066in}{2.257726in}}%
\pgfpathlineto{\pgfqpoint{2.847034in}{2.250287in}}%
\pgfpathlineto{\pgfqpoint{2.860005in}{2.242889in}}%
\pgfpathlineto{\pgfqpoint{2.872979in}{2.235530in}}%
\pgfpathlineto{\pgfqpoint{2.885956in}{2.228211in}}%
\pgfpathlineto{\pgfqpoint{2.877716in}{2.228993in}}%
\pgfpathlineto{\pgfqpoint{2.869460in}{2.230047in}}%
\pgfpathlineto{\pgfqpoint{2.861189in}{2.231378in}}%
\pgfpathlineto{\pgfqpoint{2.852902in}{2.232995in}}%
\pgfpathlineto{\pgfqpoint{2.839895in}{2.240576in}}%
\pgfpathlineto{\pgfqpoint{2.826892in}{2.248197in}}%
\pgfpathlineto{\pgfqpoint{2.813891in}{2.255858in}}%
\pgfpathlineto{\pgfqpoint{2.800894in}{2.263559in}}%
\pgfpathlineto{\pgfqpoint{2.809211in}{2.261675in}}%
\pgfpathlineto{\pgfqpoint{2.817512in}{2.260079in}}%
\pgfpathlineto{\pgfqpoint{2.825797in}{2.258765in}}%
\pgfpathlineto{\pgfqpoint{2.834066in}{2.257726in}}%
\pgfpathclose%
\pgfusepath{fill}%
\end{pgfscope}%
\begin{pgfscope}%
\pgfpathrectangle{\pgfqpoint{1.254980in}{0.150000in}}{\pgfqpoint{5.490039in}{5.490039in}}%
\pgfusepath{clip}%
\pgfsetbuttcap%
\pgfsetroundjoin%
\definecolor{currentfill}{rgb}{0.278012,0.180367,0.486697}%
\pgfsetfillcolor{currentfill}%
\pgfsetfillopacity{0.700000}%
\pgfsetlinewidth{0.000000pt}%
\definecolor{currentstroke}{rgb}{0.000000,0.000000,0.000000}%
\pgfsetstrokecolor{currentstroke}%
\pgfsetdash{}{0pt}%
\pgfpathmoveto{\pgfqpoint{5.585220in}{2.310175in}}%
\pgfpathlineto{\pgfqpoint{5.598812in}{2.309498in}}%
\pgfpathlineto{\pgfqpoint{5.612413in}{2.308845in}}%
\pgfpathlineto{\pgfqpoint{5.626022in}{2.308216in}}%
\pgfpathlineto{\pgfqpoint{5.639640in}{2.307610in}}%
\pgfpathlineto{\pgfqpoint{5.632658in}{2.301507in}}%
\pgfpathlineto{\pgfqpoint{5.625667in}{2.295324in}}%
\pgfpathlineto{\pgfqpoint{5.618669in}{2.289058in}}%
\pgfpathlineto{\pgfqpoint{5.611662in}{2.282708in}}%
\pgfpathlineto{\pgfqpoint{5.598029in}{2.283274in}}%
\pgfpathlineto{\pgfqpoint{5.584405in}{2.283864in}}%
\pgfpathlineto{\pgfqpoint{5.570789in}{2.284478in}}%
\pgfpathlineto{\pgfqpoint{5.557181in}{2.285115in}}%
\pgfpathlineto{\pgfqpoint{5.564203in}{2.291500in}}%
\pgfpathlineto{\pgfqpoint{5.571216in}{2.297804in}}%
\pgfpathlineto{\pgfqpoint{5.578222in}{2.304028in}}%
\pgfpathlineto{\pgfqpoint{5.585220in}{2.310175in}}%
\pgfpathclose%
\pgfusepath{fill}%
\end{pgfscope}%
\begin{pgfscope}%
\pgfpathrectangle{\pgfqpoint{1.254980in}{0.150000in}}{\pgfqpoint{5.490039in}{5.490039in}}%
\pgfusepath{clip}%
\pgfsetbuttcap%
\pgfsetroundjoin%
\definecolor{currentfill}{rgb}{0.268510,0.009605,0.335427}%
\pgfsetfillcolor{currentfill}%
\pgfsetfillopacity{0.700000}%
\pgfsetlinewidth{0.000000pt}%
\definecolor{currentstroke}{rgb}{0.000000,0.000000,0.000000}%
\pgfsetstrokecolor{currentstroke}%
\pgfsetdash{}{0pt}%
\pgfpathmoveto{\pgfqpoint{3.670013in}{2.002703in}}%
\pgfpathlineto{\pgfqpoint{3.683089in}{1.998113in}}%
\pgfpathlineto{\pgfqpoint{3.696170in}{1.993551in}}%
\pgfpathlineto{\pgfqpoint{3.709256in}{1.989018in}}%
\pgfpathlineto{\pgfqpoint{3.722348in}{1.984513in}}%
\pgfpathlineto{\pgfqpoint{3.714599in}{1.978691in}}%
\pgfpathlineto{\pgfqpoint{3.706842in}{1.972982in}}%
\pgfpathlineto{\pgfqpoint{3.699078in}{1.967392in}}%
\pgfpathlineto{\pgfqpoint{3.691306in}{1.961924in}}%
\pgfpathlineto{\pgfqpoint{3.678198in}{1.966623in}}%
\pgfpathlineto{\pgfqpoint{3.665095in}{1.971350in}}%
\pgfpathlineto{\pgfqpoint{3.651997in}{1.976106in}}%
\pgfpathlineto{\pgfqpoint{3.638905in}{1.980890in}}%
\pgfpathlineto{\pgfqpoint{3.646693in}{1.986159in}}%
\pgfpathlineto{\pgfqpoint{3.654474in}{1.991554in}}%
\pgfpathlineto{\pgfqpoint{3.662247in}{1.997070in}}%
\pgfpathlineto{\pgfqpoint{3.670013in}{2.002703in}}%
\pgfpathclose%
\pgfusepath{fill}%
\end{pgfscope}%
\begin{pgfscope}%
\pgfpathrectangle{\pgfqpoint{1.254980in}{0.150000in}}{\pgfqpoint{5.490039in}{5.490039in}}%
\pgfusepath{clip}%
\pgfsetbuttcap%
\pgfsetroundjoin%
\definecolor{currentfill}{rgb}{0.279566,0.067836,0.391917}%
\pgfsetfillcolor{currentfill}%
\pgfsetfillopacity{0.700000}%
\pgfsetlinewidth{0.000000pt}%
\definecolor{currentstroke}{rgb}{0.000000,0.000000,0.000000}%
\pgfsetstrokecolor{currentstroke}%
\pgfsetdash{}{0pt}%
\pgfpathmoveto{\pgfqpoint{4.680365in}{2.100017in}}%
\pgfpathlineto{\pgfqpoint{4.693685in}{2.098064in}}%
\pgfpathlineto{\pgfqpoint{4.707013in}{2.096136in}}%
\pgfpathlineto{\pgfqpoint{4.720349in}{2.094232in}}%
\pgfpathlineto{\pgfqpoint{4.733691in}{2.092353in}}%
\pgfpathlineto{\pgfqpoint{4.726316in}{2.083839in}}%
\pgfpathlineto{\pgfqpoint{4.718936in}{2.075281in}}%
\pgfpathlineto{\pgfqpoint{4.711550in}{2.066680in}}%
\pgfpathlineto{\pgfqpoint{4.704158in}{2.058039in}}%
\pgfpathlineto{\pgfqpoint{4.690805in}{2.059997in}}%
\pgfpathlineto{\pgfqpoint{4.677460in}{2.061980in}}%
\pgfpathlineto{\pgfqpoint{4.664121in}{2.063988in}}%
\pgfpathlineto{\pgfqpoint{4.650791in}{2.066020in}}%
\pgfpathlineto{\pgfqpoint{4.658192in}{2.074577in}}%
\pgfpathlineto{\pgfqpoint{4.665589in}{2.083097in}}%
\pgfpathlineto{\pgfqpoint{4.672980in}{2.091578in}}%
\pgfpathlineto{\pgfqpoint{4.680365in}{2.100017in}}%
\pgfpathclose%
\pgfusepath{fill}%
\end{pgfscope}%
\begin{pgfscope}%
\pgfpathrectangle{\pgfqpoint{1.254980in}{0.150000in}}{\pgfqpoint{5.490039in}{5.490039in}}%
\pgfusepath{clip}%
\pgfsetbuttcap%
\pgfsetroundjoin%
\definecolor{currentfill}{rgb}{0.282910,0.105393,0.426902}%
\pgfsetfillcolor{currentfill}%
\pgfsetfillopacity{0.700000}%
\pgfsetlinewidth{0.000000pt}%
\definecolor{currentstroke}{rgb}{0.000000,0.000000,0.000000}%
\pgfsetstrokecolor{currentstroke}%
\pgfsetdash{}{0pt}%
\pgfpathmoveto{\pgfqpoint{4.982047in}{2.171892in}}%
\pgfpathlineto{\pgfqpoint{4.995456in}{2.170488in}}%
\pgfpathlineto{\pgfqpoint{5.008873in}{2.169108in}}%
\pgfpathlineto{\pgfqpoint{5.022298in}{2.167752in}}%
\pgfpathlineto{\pgfqpoint{5.035730in}{2.166420in}}%
\pgfpathlineto{\pgfqpoint{5.028471in}{2.158385in}}%
\pgfpathlineto{\pgfqpoint{5.021204in}{2.150281in}}%
\pgfpathlineto{\pgfqpoint{5.013932in}{2.142108in}}%
\pgfpathlineto{\pgfqpoint{5.006653in}{2.133867in}}%
\pgfpathlineto{\pgfqpoint{4.993209in}{2.135239in}}%
\pgfpathlineto{\pgfqpoint{4.979774in}{2.136634in}}%
\pgfpathlineto{\pgfqpoint{4.966346in}{2.138054in}}%
\pgfpathlineto{\pgfqpoint{4.952926in}{2.139498in}}%
\pgfpathlineto{\pgfqpoint{4.960216in}{2.147694in}}%
\pgfpathlineto{\pgfqpoint{4.967499in}{2.155826in}}%
\pgfpathlineto{\pgfqpoint{4.974776in}{2.163891in}}%
\pgfpathlineto{\pgfqpoint{4.982047in}{2.171892in}}%
\pgfpathclose%
\pgfusepath{fill}%
\end{pgfscope}%
\begin{pgfscope}%
\pgfpathrectangle{\pgfqpoint{1.254980in}{0.150000in}}{\pgfqpoint{5.490039in}{5.490039in}}%
\pgfusepath{clip}%
\pgfsetbuttcap%
\pgfsetroundjoin%
\definecolor{currentfill}{rgb}{0.271305,0.019942,0.347269}%
\pgfsetfillcolor{currentfill}%
\pgfsetfillopacity{0.700000}%
\pgfsetlinewidth{0.000000pt}%
\definecolor{currentstroke}{rgb}{0.000000,0.000000,0.000000}%
\pgfsetstrokecolor{currentstroke}%
\pgfsetdash{}{0pt}%
\pgfpathmoveto{\pgfqpoint{3.534347in}{2.020209in}}%
\pgfpathlineto{\pgfqpoint{3.547399in}{2.015191in}}%
\pgfpathlineto{\pgfqpoint{3.560457in}{2.010203in}}%
\pgfpathlineto{\pgfqpoint{3.573519in}{2.005244in}}%
\pgfpathlineto{\pgfqpoint{3.586586in}{2.000315in}}%
\pgfpathlineto{\pgfqpoint{3.578772in}{1.995380in}}%
\pgfpathlineto{\pgfqpoint{3.570951in}{1.990584in}}%
\pgfpathlineto{\pgfqpoint{3.563121in}{1.985933in}}%
\pgfpathlineto{\pgfqpoint{3.555283in}{1.981431in}}%
\pgfpathlineto{\pgfqpoint{3.542198in}{1.986567in}}%
\pgfpathlineto{\pgfqpoint{3.529117in}{1.991733in}}%
\pgfpathlineto{\pgfqpoint{3.516042in}{1.996928in}}%
\pgfpathlineto{\pgfqpoint{3.502971in}{2.002152in}}%
\pgfpathlineto{\pgfqpoint{3.510828in}{2.006442in}}%
\pgfpathlineto{\pgfqpoint{3.518676in}{2.010885in}}%
\pgfpathlineto{\pgfqpoint{3.526516in}{2.015475in}}%
\pgfpathlineto{\pgfqpoint{3.534347in}{2.020209in}}%
\pgfpathclose%
\pgfusepath{fill}%
\end{pgfscope}%
\begin{pgfscope}%
\pgfpathrectangle{\pgfqpoint{1.254980in}{0.150000in}}{\pgfqpoint{5.490039in}{5.490039in}}%
\pgfusepath{clip}%
\pgfsetbuttcap%
\pgfsetroundjoin%
\definecolor{currentfill}{rgb}{0.282290,0.145912,0.461510}%
\pgfsetfillcolor{currentfill}%
\pgfsetfillopacity{0.700000}%
\pgfsetlinewidth{0.000000pt}%
\definecolor{currentstroke}{rgb}{0.000000,0.000000,0.000000}%
\pgfsetstrokecolor{currentstroke}%
\pgfsetdash{}{0pt}%
\pgfpathmoveto{\pgfqpoint{5.283720in}{2.243775in}}%
\pgfpathlineto{\pgfqpoint{5.297221in}{2.242797in}}%
\pgfpathlineto{\pgfqpoint{5.310730in}{2.241842in}}%
\pgfpathlineto{\pgfqpoint{5.324247in}{2.240911in}}%
\pgfpathlineto{\pgfqpoint{5.337773in}{2.240003in}}%
\pgfpathlineto{\pgfqpoint{5.330643in}{2.232823in}}%
\pgfpathlineto{\pgfqpoint{5.323506in}{2.225560in}}%
\pgfpathlineto{\pgfqpoint{5.316361in}{2.218216in}}%
\pgfpathlineto{\pgfqpoint{5.309209in}{2.210789in}}%
\pgfpathlineto{\pgfqpoint{5.295671in}{2.211697in}}%
\pgfpathlineto{\pgfqpoint{5.282141in}{2.212629in}}%
\pgfpathlineto{\pgfqpoint{5.268620in}{2.213584in}}%
\pgfpathlineto{\pgfqpoint{5.255106in}{2.214563in}}%
\pgfpathlineto{\pgfqpoint{5.262270in}{2.221985in}}%
\pgfpathlineto{\pgfqpoint{5.269427in}{2.229327in}}%
\pgfpathlineto{\pgfqpoint{5.276577in}{2.236590in}}%
\pgfpathlineto{\pgfqpoint{5.283720in}{2.243775in}}%
\pgfpathclose%
\pgfusepath{fill}%
\end{pgfscope}%
\begin{pgfscope}%
\pgfpathrectangle{\pgfqpoint{1.254980in}{0.150000in}}{\pgfqpoint{5.490039in}{5.490039in}}%
\pgfusepath{clip}%
\pgfsetbuttcap%
\pgfsetroundjoin%
\definecolor{currentfill}{rgb}{0.267004,0.004874,0.329415}%
\pgfsetfillcolor{currentfill}%
\pgfsetfillopacity{0.700000}%
\pgfsetlinewidth{0.000000pt}%
\definecolor{currentstroke}{rgb}{0.000000,0.000000,0.000000}%
\pgfsetstrokecolor{currentstroke}%
\pgfsetdash{}{0pt}%
\pgfpathmoveto{\pgfqpoint{3.805636in}{1.991828in}}%
\pgfpathlineto{\pgfqpoint{3.818740in}{1.987644in}}%
\pgfpathlineto{\pgfqpoint{3.831850in}{1.983488in}}%
\pgfpathlineto{\pgfqpoint{3.844965in}{1.979359in}}%
\pgfpathlineto{\pgfqpoint{3.858085in}{1.975258in}}%
\pgfpathlineto{\pgfqpoint{3.850393in}{1.968676in}}%
\pgfpathlineto{\pgfqpoint{3.842695in}{1.962183in}}%
\pgfpathlineto{\pgfqpoint{3.834989in}{1.955782in}}%
\pgfpathlineto{\pgfqpoint{3.827278in}{1.949480in}}%
\pgfpathlineto{\pgfqpoint{3.814142in}{1.953762in}}%
\pgfpathlineto{\pgfqpoint{3.801012in}{1.958072in}}%
\pgfpathlineto{\pgfqpoint{3.787888in}{1.962409in}}%
\pgfpathlineto{\pgfqpoint{3.774769in}{1.966774in}}%
\pgfpathlineto{\pgfqpoint{3.782496in}{1.972891in}}%
\pgfpathlineto{\pgfqpoint{3.790216in}{1.979109in}}%
\pgfpathlineto{\pgfqpoint{3.797930in}{1.985422in}}%
\pgfpathlineto{\pgfqpoint{3.805636in}{1.991828in}}%
\pgfpathclose%
\pgfusepath{fill}%
\end{pgfscope}%
\begin{pgfscope}%
\pgfpathrectangle{\pgfqpoint{1.254980in}{0.150000in}}{\pgfqpoint{5.490039in}{5.490039in}}%
\pgfusepath{clip}%
\pgfsetbuttcap%
\pgfsetroundjoin%
\definecolor{currentfill}{rgb}{0.268510,0.009605,0.335427}%
\pgfsetfillcolor{currentfill}%
\pgfsetfillopacity{0.700000}%
\pgfsetlinewidth{0.000000pt}%
\definecolor{currentstroke}{rgb}{0.000000,0.000000,0.000000}%
\pgfsetstrokecolor{currentstroke}%
\pgfsetdash{}{0pt}%
\pgfpathmoveto{\pgfqpoint{4.160002in}{2.005179in}}%
\pgfpathlineto{\pgfqpoint{4.173185in}{2.001997in}}%
\pgfpathlineto{\pgfqpoint{4.186375in}{1.998842in}}%
\pgfpathlineto{\pgfqpoint{4.199572in}{1.995712in}}%
\pgfpathlineto{\pgfqpoint{4.212774in}{1.992607in}}%
\pgfpathlineto{\pgfqpoint{4.205217in}{1.984582in}}%
\pgfpathlineto{\pgfqpoint{4.197654in}{1.976584in}}%
\pgfpathlineto{\pgfqpoint{4.190085in}{1.968615in}}%
\pgfpathlineto{\pgfqpoint{4.182511in}{1.960681in}}%
\pgfpathlineto{\pgfqpoint{4.169297in}{1.963928in}}%
\pgfpathlineto{\pgfqpoint{4.156089in}{1.967201in}}%
\pgfpathlineto{\pgfqpoint{4.142887in}{1.970500in}}%
\pgfpathlineto{\pgfqpoint{4.129692in}{1.973825in}}%
\pgfpathlineto{\pgfqpoint{4.137278in}{1.981612in}}%
\pgfpathlineto{\pgfqpoint{4.144858in}{1.989435in}}%
\pgfpathlineto{\pgfqpoint{4.152433in}{1.997292in}}%
\pgfpathlineto{\pgfqpoint{4.160002in}{2.005179in}}%
\pgfpathclose%
\pgfusepath{fill}%
\end{pgfscope}%
\begin{pgfscope}%
\pgfpathrectangle{\pgfqpoint{1.254980in}{0.150000in}}{\pgfqpoint{5.490039in}{5.490039in}}%
\pgfusepath{clip}%
\pgfsetbuttcap%
\pgfsetroundjoin%
\definecolor{currentfill}{rgb}{0.272594,0.025563,0.353093}%
\pgfsetfillcolor{currentfill}%
\pgfsetfillopacity{0.700000}%
\pgfsetlinewidth{0.000000pt}%
\definecolor{currentstroke}{rgb}{0.000000,0.000000,0.000000}%
\pgfsetstrokecolor{currentstroke}%
\pgfsetdash{}{0pt}%
\pgfpathmoveto{\pgfqpoint{4.378717in}{2.035468in}}%
\pgfpathlineto{\pgfqpoint{4.391956in}{2.032841in}}%
\pgfpathlineto{\pgfqpoint{4.405203in}{2.030240in}}%
\pgfpathlineto{\pgfqpoint{4.418456in}{2.027663in}}%
\pgfpathlineto{\pgfqpoint{4.431716in}{2.025112in}}%
\pgfpathlineto{\pgfqpoint{4.424234in}{2.016644in}}%
\pgfpathlineto{\pgfqpoint{4.416747in}{2.008170in}}%
\pgfpathlineto{\pgfqpoint{4.409255in}{1.999693in}}%
\pgfpathlineto{\pgfqpoint{4.401757in}{1.991214in}}%
\pgfpathlineto{\pgfqpoint{4.388486in}{1.993883in}}%
\pgfpathlineto{\pgfqpoint{4.375223in}{1.996577in}}%
\pgfpathlineto{\pgfqpoint{4.361965in}{1.999296in}}%
\pgfpathlineto{\pgfqpoint{4.348715in}{2.002041in}}%
\pgfpathlineto{\pgfqpoint{4.356223in}{2.010397in}}%
\pgfpathlineto{\pgfqpoint{4.363726in}{2.018756in}}%
\pgfpathlineto{\pgfqpoint{4.371224in}{2.027114in}}%
\pgfpathlineto{\pgfqpoint{4.378717in}{2.035468in}}%
\pgfpathclose%
\pgfusepath{fill}%
\end{pgfscope}%
\begin{pgfscope}%
\pgfpathrectangle{\pgfqpoint{1.254980in}{0.150000in}}{\pgfqpoint{5.490039in}{5.490039in}}%
\pgfusepath{clip}%
\pgfsetbuttcap%
\pgfsetroundjoin%
\definecolor{currentfill}{rgb}{0.273006,0.204520,0.501721}%
\pgfsetfillcolor{currentfill}%
\pgfsetfillopacity{0.700000}%
\pgfsetlinewidth{0.000000pt}%
\definecolor{currentstroke}{rgb}{0.000000,0.000000,0.000000}%
\pgfsetstrokecolor{currentstroke}%
\pgfsetdash{}{0pt}%
\pgfpathmoveto{\pgfqpoint{5.804216in}{2.348794in}}%
\pgfpathlineto{\pgfqpoint{5.817879in}{2.348282in}}%
\pgfpathlineto{\pgfqpoint{5.831551in}{2.347793in}}%
\pgfpathlineto{\pgfqpoint{5.845232in}{2.347327in}}%
\pgfpathlineto{\pgfqpoint{5.858921in}{2.346885in}}%
\pgfpathlineto{\pgfqpoint{5.852052in}{2.341574in}}%
\pgfpathlineto{\pgfqpoint{5.845174in}{2.336188in}}%
\pgfpathlineto{\pgfqpoint{5.838288in}{2.330725in}}%
\pgfpathlineto{\pgfqpoint{5.831394in}{2.325182in}}%
\pgfpathlineto{\pgfqpoint{5.817688in}{2.325558in}}%
\pgfpathlineto{\pgfqpoint{5.803990in}{2.325957in}}%
\pgfpathlineto{\pgfqpoint{5.790301in}{2.326380in}}%
\pgfpathlineto{\pgfqpoint{5.776621in}{2.326827in}}%
\pgfpathlineto{\pgfqpoint{5.783532in}{2.332430in}}%
\pgfpathlineto{\pgfqpoint{5.790435in}{2.337957in}}%
\pgfpathlineto{\pgfqpoint{5.797329in}{2.343411in}}%
\pgfpathlineto{\pgfqpoint{5.804216in}{2.348794in}}%
\pgfpathclose%
\pgfusepath{fill}%
\end{pgfscope}%
\begin{pgfscope}%
\pgfpathrectangle{\pgfqpoint{1.254980in}{0.150000in}}{\pgfqpoint{5.490039in}{5.490039in}}%
\pgfusepath{clip}%
\pgfsetbuttcap%
\pgfsetroundjoin%
\definecolor{currentfill}{rgb}{0.274952,0.037752,0.364543}%
\pgfsetfillcolor{currentfill}%
\pgfsetfillopacity{0.700000}%
\pgfsetlinewidth{0.000000pt}%
\definecolor{currentstroke}{rgb}{0.000000,0.000000,0.000000}%
\pgfsetstrokecolor{currentstroke}%
\pgfsetdash{}{0pt}%
\pgfpathmoveto{\pgfqpoint{3.398577in}{2.045038in}}%
\pgfpathlineto{\pgfqpoint{3.411610in}{2.039570in}}%
\pgfpathlineto{\pgfqpoint{3.424648in}{2.034133in}}%
\pgfpathlineto{\pgfqpoint{3.437690in}{2.028727in}}%
\pgfpathlineto{\pgfqpoint{3.450737in}{2.023351in}}%
\pgfpathlineto{\pgfqpoint{3.442852in}{2.019435in}}%
\pgfpathlineto{\pgfqpoint{3.434958in}{2.015687in}}%
\pgfpathlineto{\pgfqpoint{3.427054in}{2.012111in}}%
\pgfpathlineto{\pgfqpoint{3.419141in}{2.008714in}}%
\pgfpathlineto{\pgfqpoint{3.406074in}{2.014310in}}%
\pgfpathlineto{\pgfqpoint{3.393012in}{2.019936in}}%
\pgfpathlineto{\pgfqpoint{3.379954in}{2.025593in}}%
\pgfpathlineto{\pgfqpoint{3.366901in}{2.031281in}}%
\pgfpathlineto{\pgfqpoint{3.374834in}{2.034453in}}%
\pgfpathlineto{\pgfqpoint{3.382758in}{2.037807in}}%
\pgfpathlineto{\pgfqpoint{3.390672in}{2.041337in}}%
\pgfpathlineto{\pgfqpoint{3.398577in}{2.045038in}}%
\pgfpathclose%
\pgfusepath{fill}%
\end{pgfscope}%
\begin{pgfscope}%
\pgfpathrectangle{\pgfqpoint{1.254980in}{0.150000in}}{\pgfqpoint{5.490039in}{5.490039in}}%
\pgfusepath{clip}%
\pgfsetbuttcap%
\pgfsetroundjoin%
\definecolor{currentfill}{rgb}{0.267004,0.004874,0.329415}%
\pgfsetfillcolor{currentfill}%
\pgfsetfillopacity{0.700000}%
\pgfsetlinewidth{0.000000pt}%
\definecolor{currentstroke}{rgb}{0.000000,0.000000,0.000000}%
\pgfsetstrokecolor{currentstroke}%
\pgfsetdash{}{0pt}%
\pgfpathmoveto{\pgfqpoint{3.941274in}{1.986928in}}%
\pgfpathlineto{\pgfqpoint{3.954409in}{1.983131in}}%
\pgfpathlineto{\pgfqpoint{3.967550in}{1.979361in}}%
\pgfpathlineto{\pgfqpoint{3.980698in}{1.975617in}}%
\pgfpathlineto{\pgfqpoint{3.993851in}{1.971900in}}%
\pgfpathlineto{\pgfqpoint{3.986211in}{1.964679in}}%
\pgfpathlineto{\pgfqpoint{3.978565in}{1.957523in}}%
\pgfpathlineto{\pgfqpoint{3.970914in}{1.950436in}}%
\pgfpathlineto{\pgfqpoint{3.963256in}{1.943423in}}%
\pgfpathlineto{\pgfqpoint{3.950089in}{1.947308in}}%
\pgfpathlineto{\pgfqpoint{3.936929in}{1.951221in}}%
\pgfpathlineto{\pgfqpoint{3.923774in}{1.955159in}}%
\pgfpathlineto{\pgfqpoint{3.910624in}{1.959125in}}%
\pgfpathlineto{\pgfqpoint{3.918296in}{1.965965in}}%
\pgfpathlineto{\pgfqpoint{3.925961in}{1.972881in}}%
\pgfpathlineto{\pgfqpoint{3.933620in}{1.979870in}}%
\pgfpathlineto{\pgfqpoint{3.941274in}{1.986928in}}%
\pgfpathclose%
\pgfusepath{fill}%
\end{pgfscope}%
\begin{pgfscope}%
\pgfpathrectangle{\pgfqpoint{1.254980in}{0.150000in}}{\pgfqpoint{5.490039in}{5.490039in}}%
\pgfusepath{clip}%
\pgfsetbuttcap%
\pgfsetroundjoin%
\definecolor{currentfill}{rgb}{0.277941,0.056324,0.381191}%
\pgfsetfillcolor{currentfill}%
\pgfsetfillopacity{0.700000}%
\pgfsetlinewidth{0.000000pt}%
\definecolor{currentstroke}{rgb}{0.000000,0.000000,0.000000}%
\pgfsetstrokecolor{currentstroke}%
\pgfsetdash{}{0pt}%
\pgfpathmoveto{\pgfqpoint{4.597539in}{2.074394in}}%
\pgfpathlineto{\pgfqpoint{4.610841in}{2.072264in}}%
\pgfpathlineto{\pgfqpoint{4.624150in}{2.070158in}}%
\pgfpathlineto{\pgfqpoint{4.637467in}{2.068077in}}%
\pgfpathlineto{\pgfqpoint{4.650791in}{2.066020in}}%
\pgfpathlineto{\pgfqpoint{4.643383in}{2.057427in}}%
\pgfpathlineto{\pgfqpoint{4.635970in}{2.048799in}}%
\pgfpathlineto{\pgfqpoint{4.628552in}{2.040140in}}%
\pgfpathlineto{\pgfqpoint{4.621128in}{2.031451in}}%
\pgfpathlineto{\pgfqpoint{4.607794in}{2.033599in}}%
\pgfpathlineto{\pgfqpoint{4.594467in}{2.035773in}}%
\pgfpathlineto{\pgfqpoint{4.581147in}{2.037971in}}%
\pgfpathlineto{\pgfqpoint{4.567835in}{2.040193in}}%
\pgfpathlineto{\pgfqpoint{4.575269in}{2.048786in}}%
\pgfpathlineto{\pgfqpoint{4.582698in}{2.057351in}}%
\pgfpathlineto{\pgfqpoint{4.590121in}{2.065888in}}%
\pgfpathlineto{\pgfqpoint{4.597539in}{2.074394in}}%
\pgfpathclose%
\pgfusepath{fill}%
\end{pgfscope}%
\begin{pgfscope}%
\pgfpathrectangle{\pgfqpoint{1.254980in}{0.150000in}}{\pgfqpoint{5.490039in}{5.490039in}}%
\pgfusepath{clip}%
\pgfsetbuttcap%
\pgfsetroundjoin%
\definecolor{currentfill}{rgb}{0.282327,0.094955,0.417331}%
\pgfsetfillcolor{currentfill}%
\pgfsetfillopacity{0.700000}%
\pgfsetlinewidth{0.000000pt}%
\definecolor{currentstroke}{rgb}{0.000000,0.000000,0.000000}%
\pgfsetstrokecolor{currentstroke}%
\pgfsetdash{}{0pt}%
\pgfpathmoveto{\pgfqpoint{4.899323in}{2.145516in}}%
\pgfpathlineto{\pgfqpoint{4.912712in}{2.143975in}}%
\pgfpathlineto{\pgfqpoint{4.926109in}{2.142459in}}%
\pgfpathlineto{\pgfqpoint{4.939513in}{2.140966in}}%
\pgfpathlineto{\pgfqpoint{4.952926in}{2.139498in}}%
\pgfpathlineto{\pgfqpoint{4.945630in}{2.131238in}}%
\pgfpathlineto{\pgfqpoint{4.938328in}{2.122915in}}%
\pgfpathlineto{\pgfqpoint{4.931019in}{2.114529in}}%
\pgfpathlineto{\pgfqpoint{4.923705in}{2.106082in}}%
\pgfpathlineto{\pgfqpoint{4.910282in}{2.107603in}}%
\pgfpathlineto{\pgfqpoint{4.896867in}{2.109149in}}%
\pgfpathlineto{\pgfqpoint{4.883459in}{2.110719in}}%
\pgfpathlineto{\pgfqpoint{4.870060in}{2.112313in}}%
\pgfpathlineto{\pgfqpoint{4.877384in}{2.120702in}}%
\pgfpathlineto{\pgfqpoint{4.884703in}{2.129033in}}%
\pgfpathlineto{\pgfqpoint{4.892016in}{2.137305in}}%
\pgfpathlineto{\pgfqpoint{4.899323in}{2.145516in}}%
\pgfpathclose%
\pgfusepath{fill}%
\end{pgfscope}%
\begin{pgfscope}%
\pgfpathrectangle{\pgfqpoint{1.254980in}{0.150000in}}{\pgfqpoint{5.490039in}{5.490039in}}%
\pgfusepath{clip}%
\pgfsetbuttcap%
\pgfsetroundjoin%
\definecolor{currentfill}{rgb}{0.279574,0.170599,0.479997}%
\pgfsetfillcolor{currentfill}%
\pgfsetfillopacity{0.700000}%
\pgfsetlinewidth{0.000000pt}%
\definecolor{currentstroke}{rgb}{0.000000,0.000000,0.000000}%
\pgfsetstrokecolor{currentstroke}%
\pgfsetdash{}{0pt}%
\pgfpathmoveto{\pgfqpoint{5.502837in}{2.287902in}}%
\pgfpathlineto{\pgfqpoint{5.516410in}{2.287170in}}%
\pgfpathlineto{\pgfqpoint{5.529992in}{2.286461in}}%
\pgfpathlineto{\pgfqpoint{5.543583in}{2.285776in}}%
\pgfpathlineto{\pgfqpoint{5.557181in}{2.285115in}}%
\pgfpathlineto{\pgfqpoint{5.550152in}{2.278648in}}%
\pgfpathlineto{\pgfqpoint{5.543116in}{2.272098in}}%
\pgfpathlineto{\pgfqpoint{5.536071in}{2.265462in}}%
\pgfpathlineto{\pgfqpoint{5.529019in}{2.258740in}}%
\pgfpathlineto{\pgfqpoint{5.515406in}{2.259375in}}%
\pgfpathlineto{\pgfqpoint{5.501801in}{2.260034in}}%
\pgfpathlineto{\pgfqpoint{5.488205in}{2.260716in}}%
\pgfpathlineto{\pgfqpoint{5.474618in}{2.261423in}}%
\pgfpathlineto{\pgfqpoint{5.481684in}{2.268165in}}%
\pgfpathlineto{\pgfqpoint{5.488743in}{2.274825in}}%
\pgfpathlineto{\pgfqpoint{5.495794in}{2.281404in}}%
\pgfpathlineto{\pgfqpoint{5.502837in}{2.287902in}}%
\pgfpathclose%
\pgfusepath{fill}%
\end{pgfscope}%
\begin{pgfscope}%
\pgfpathrectangle{\pgfqpoint{1.254980in}{0.150000in}}{\pgfqpoint{5.490039in}{5.490039in}}%
\pgfusepath{clip}%
\pgfsetbuttcap%
\pgfsetroundjoin%
\definecolor{currentfill}{rgb}{0.282884,0.135920,0.453427}%
\pgfsetfillcolor{currentfill}%
\pgfsetfillopacity{0.700000}%
\pgfsetlinewidth{0.000000pt}%
\definecolor{currentstroke}{rgb}{0.000000,0.000000,0.000000}%
\pgfsetstrokecolor{currentstroke}%
\pgfsetdash{}{0pt}%
\pgfpathmoveto{\pgfqpoint{5.201134in}{2.218719in}}%
\pgfpathlineto{\pgfqpoint{5.214615in}{2.217645in}}%
\pgfpathlineto{\pgfqpoint{5.228104in}{2.216594in}}%
\pgfpathlineto{\pgfqpoint{5.241601in}{2.215566in}}%
\pgfpathlineto{\pgfqpoint{5.255106in}{2.214563in}}%
\pgfpathlineto{\pgfqpoint{5.247935in}{2.207062in}}%
\pgfpathlineto{\pgfqpoint{5.240757in}{2.199482in}}%
\pgfpathlineto{\pgfqpoint{5.233572in}{2.191821in}}%
\pgfpathlineto{\pgfqpoint{5.226379in}{2.184079in}}%
\pgfpathlineto{\pgfqpoint{5.212862in}{2.185096in}}%
\pgfpathlineto{\pgfqpoint{5.199353in}{2.186137in}}%
\pgfpathlineto{\pgfqpoint{5.185852in}{2.187202in}}%
\pgfpathlineto{\pgfqpoint{5.172359in}{2.188291in}}%
\pgfpathlineto{\pgfqpoint{5.179563in}{2.196013in}}%
\pgfpathlineto{\pgfqpoint{5.186760in}{2.203659in}}%
\pgfpathlineto{\pgfqpoint{5.193950in}{2.211227in}}%
\pgfpathlineto{\pgfqpoint{5.201134in}{2.218719in}}%
\pgfpathclose%
\pgfusepath{fill}%
\end{pgfscope}%
\begin{pgfscope}%
\pgfpathrectangle{\pgfqpoint{1.254980in}{0.150000in}}{\pgfqpoint{5.490039in}{5.490039in}}%
\pgfusepath{clip}%
\pgfsetbuttcap%
\pgfsetroundjoin%
\definecolor{currentfill}{rgb}{0.282656,0.100196,0.422160}%
\pgfsetfillcolor{currentfill}%
\pgfsetfillopacity{0.700000}%
\pgfsetlinewidth{0.000000pt}%
\definecolor{currentstroke}{rgb}{0.000000,0.000000,0.000000}%
\pgfsetstrokecolor{currentstroke}%
\pgfsetdash{}{0pt}%
\pgfpathmoveto{\pgfqpoint{3.074431in}{2.145780in}}%
\pgfpathlineto{\pgfqpoint{3.087425in}{2.139195in}}%
\pgfpathlineto{\pgfqpoint{3.100424in}{2.132645in}}%
\pgfpathlineto{\pgfqpoint{3.113426in}{2.126131in}}%
\pgfpathlineto{\pgfqpoint{3.126431in}{2.119651in}}%
\pgfpathlineto{\pgfqpoint{3.118351in}{2.118444in}}%
\pgfpathlineto{\pgfqpoint{3.110259in}{2.117468in}}%
\pgfpathlineto{\pgfqpoint{3.102154in}{2.116730in}}%
\pgfpathlineto{\pgfqpoint{3.094036in}{2.116237in}}%
\pgfpathlineto{\pgfqpoint{3.081005in}{2.122964in}}%
\pgfpathlineto{\pgfqpoint{3.067977in}{2.129726in}}%
\pgfpathlineto{\pgfqpoint{3.054954in}{2.136523in}}%
\pgfpathlineto{\pgfqpoint{3.041933in}{2.143355in}}%
\pgfpathlineto{\pgfqpoint{3.050078in}{2.143597in}}%
\pgfpathlineto{\pgfqpoint{3.058208in}{2.144086in}}%
\pgfpathlineto{\pgfqpoint{3.066326in}{2.144816in}}%
\pgfpathlineto{\pgfqpoint{3.074431in}{2.145780in}}%
\pgfpathclose%
\pgfusepath{fill}%
\end{pgfscope}%
\begin{pgfscope}%
\pgfpathrectangle{\pgfqpoint{1.254980in}{0.150000in}}{\pgfqpoint{5.490039in}{5.490039in}}%
\pgfusepath{clip}%
\pgfsetbuttcap%
\pgfsetroundjoin%
\definecolor{currentfill}{rgb}{0.271828,0.209303,0.504434}%
\pgfsetfillcolor{currentfill}%
\pgfsetfillopacity{0.700000}%
\pgfsetlinewidth{0.000000pt}%
\definecolor{currentstroke}{rgb}{0.000000,0.000000,0.000000}%
\pgfsetstrokecolor{currentstroke}%
\pgfsetdash{}{0pt}%
\pgfpathmoveto{\pgfqpoint{2.645124in}{2.359258in}}%
\pgfpathlineto{\pgfqpoint{2.658092in}{2.351043in}}%
\pgfpathlineto{\pgfqpoint{2.671062in}{2.342872in}}%
\pgfpathlineto{\pgfqpoint{2.684034in}{2.334747in}}%
\pgfpathlineto{\pgfqpoint{2.697008in}{2.326666in}}%
\pgfpathlineto{\pgfqpoint{2.688611in}{2.329389in}}%
\pgfpathlineto{\pgfqpoint{2.680196in}{2.332422in}}%
\pgfpathlineto{\pgfqpoint{2.671763in}{2.335773in}}%
\pgfpathlineto{\pgfqpoint{2.663311in}{2.339448in}}%
\pgfpathlineto{\pgfqpoint{2.650304in}{2.347807in}}%
\pgfpathlineto{\pgfqpoint{2.637299in}{2.356210in}}%
\pgfpathlineto{\pgfqpoint{2.624296in}{2.364658in}}%
\pgfpathlineto{\pgfqpoint{2.611295in}{2.373151in}}%
\pgfpathlineto{\pgfqpoint{2.619781in}{2.369192in}}%
\pgfpathlineto{\pgfqpoint{2.628248in}{2.365562in}}%
\pgfpathlineto{\pgfqpoint{2.636695in}{2.362253in}}%
\pgfpathlineto{\pgfqpoint{2.645124in}{2.359258in}}%
\pgfpathclose%
\pgfusepath{fill}%
\end{pgfscope}%
\begin{pgfscope}%
\pgfpathrectangle{\pgfqpoint{1.254980in}{0.150000in}}{\pgfqpoint{5.490039in}{5.490039in}}%
\pgfusepath{clip}%
\pgfsetbuttcap%
\pgfsetroundjoin%
\definecolor{currentfill}{rgb}{0.271305,0.019942,0.347269}%
\pgfsetfillcolor{currentfill}%
\pgfsetfillopacity{0.700000}%
\pgfsetlinewidth{0.000000pt}%
\definecolor{currentstroke}{rgb}{0.000000,0.000000,0.000000}%
\pgfsetstrokecolor{currentstroke}%
\pgfsetdash{}{0pt}%
\pgfpathmoveto{\pgfqpoint{4.295780in}{2.013272in}}%
\pgfpathlineto{\pgfqpoint{4.309004in}{2.010426in}}%
\pgfpathlineto{\pgfqpoint{4.322234in}{2.007606in}}%
\pgfpathlineto{\pgfqpoint{4.335471in}{2.004811in}}%
\pgfpathlineto{\pgfqpoint{4.348715in}{2.002041in}}%
\pgfpathlineto{\pgfqpoint{4.341201in}{1.993690in}}%
\pgfpathlineto{\pgfqpoint{4.333682in}{1.985346in}}%
\pgfpathlineto{\pgfqpoint{4.326158in}{1.977013in}}%
\pgfpathlineto{\pgfqpoint{4.318628in}{1.968694in}}%
\pgfpathlineto{\pgfqpoint{4.305374in}{1.971594in}}%
\pgfpathlineto{\pgfqpoint{4.292126in}{1.974520in}}%
\pgfpathlineto{\pgfqpoint{4.278884in}{1.977471in}}%
\pgfpathlineto{\pgfqpoint{4.265649in}{1.980447in}}%
\pgfpathlineto{\pgfqpoint{4.273190in}{1.988631in}}%
\pgfpathlineto{\pgfqpoint{4.280725in}{1.996832in}}%
\pgfpathlineto{\pgfqpoint{4.288255in}{2.005046in}}%
\pgfpathlineto{\pgfqpoint{4.295780in}{2.013272in}}%
\pgfpathclose%
\pgfusepath{fill}%
\end{pgfscope}%
\begin{pgfscope}%
\pgfpathrectangle{\pgfqpoint{1.254980in}{0.150000in}}{\pgfqpoint{5.490039in}{5.490039in}}%
\pgfusepath{clip}%
\pgfsetbuttcap%
\pgfsetroundjoin%
\definecolor{currentfill}{rgb}{0.282290,0.145912,0.461510}%
\pgfsetfillcolor{currentfill}%
\pgfsetfillopacity{0.700000}%
\pgfsetlinewidth{0.000000pt}%
\definecolor{currentstroke}{rgb}{0.000000,0.000000,0.000000}%
\pgfsetstrokecolor{currentstroke}%
\pgfsetdash{}{0pt}%
\pgfpathmoveto{\pgfqpoint{2.885956in}{2.228211in}}%
\pgfpathlineto{\pgfqpoint{2.898937in}{2.220931in}}%
\pgfpathlineto{\pgfqpoint{2.911920in}{2.213690in}}%
\pgfpathlineto{\pgfqpoint{2.924906in}{2.206488in}}%
\pgfpathlineto{\pgfqpoint{2.937896in}{2.199324in}}%
\pgfpathlineto{\pgfqpoint{2.929684in}{2.199850in}}%
\pgfpathlineto{\pgfqpoint{2.921457in}{2.200643in}}%
\pgfpathlineto{\pgfqpoint{2.913214in}{2.201711in}}%
\pgfpathlineto{\pgfqpoint{2.904956in}{2.203061in}}%
\pgfpathlineto{\pgfqpoint{2.891938in}{2.210486in}}%
\pgfpathlineto{\pgfqpoint{2.878923in}{2.217951in}}%
\pgfpathlineto{\pgfqpoint{2.865911in}{2.225453in}}%
\pgfpathlineto{\pgfqpoint{2.852902in}{2.232995in}}%
\pgfpathlineto{\pgfqpoint{2.861189in}{2.231378in}}%
\pgfpathlineto{\pgfqpoint{2.869460in}{2.230047in}}%
\pgfpathlineto{\pgfqpoint{2.877716in}{2.228993in}}%
\pgfpathlineto{\pgfqpoint{2.885956in}{2.228211in}}%
\pgfpathclose%
\pgfusepath{fill}%
\end{pgfscope}%
\begin{pgfscope}%
\pgfpathrectangle{\pgfqpoint{1.254980in}{0.150000in}}{\pgfqpoint{5.490039in}{5.490039in}}%
\pgfusepath{clip}%
\pgfsetbuttcap%
\pgfsetroundjoin%
\definecolor{currentfill}{rgb}{0.278791,0.062145,0.386592}%
\pgfsetfillcolor{currentfill}%
\pgfsetfillopacity{0.700000}%
\pgfsetlinewidth{0.000000pt}%
\definecolor{currentstroke}{rgb}{0.000000,0.000000,0.000000}%
\pgfsetstrokecolor{currentstroke}%
\pgfsetdash{}{0pt}%
\pgfpathmoveto{\pgfqpoint{3.262632in}{2.077925in}}%
\pgfpathlineto{\pgfqpoint{3.275650in}{2.071982in}}%
\pgfpathlineto{\pgfqpoint{3.288673in}{2.066072in}}%
\pgfpathlineto{\pgfqpoint{3.301700in}{2.060194in}}%
\pgfpathlineto{\pgfqpoint{3.314732in}{2.054348in}}%
\pgfpathlineto{\pgfqpoint{3.306766in}{2.051591in}}%
\pgfpathlineto{\pgfqpoint{3.298791in}{2.049032in}}%
\pgfpathlineto{\pgfqpoint{3.290805in}{2.046675in}}%
\pgfpathlineto{\pgfqpoint{3.282807in}{2.044528in}}%
\pgfpathlineto{\pgfqpoint{3.269754in}{2.050607in}}%
\pgfpathlineto{\pgfqpoint{3.256704in}{2.056719in}}%
\pgfpathlineto{\pgfqpoint{3.243659in}{2.062863in}}%
\pgfpathlineto{\pgfqpoint{3.230617in}{2.069039in}}%
\pgfpathlineto{\pgfqpoint{3.238638in}{2.070949in}}%
\pgfpathlineto{\pgfqpoint{3.246647in}{2.073070in}}%
\pgfpathlineto{\pgfqpoint{3.254644in}{2.075398in}}%
\pgfpathlineto{\pgfqpoint{3.262632in}{2.077925in}}%
\pgfpathclose%
\pgfusepath{fill}%
\end{pgfscope}%
\begin{pgfscope}%
\pgfpathrectangle{\pgfqpoint{1.254980in}{0.150000in}}{\pgfqpoint{5.490039in}{5.490039in}}%
\pgfusepath{clip}%
\pgfsetbuttcap%
\pgfsetroundjoin%
\definecolor{currentfill}{rgb}{0.267004,0.004874,0.329415}%
\pgfsetfillcolor{currentfill}%
\pgfsetfillopacity{0.700000}%
\pgfsetlinewidth{0.000000pt}%
\definecolor{currentstroke}{rgb}{0.000000,0.000000,0.000000}%
\pgfsetstrokecolor{currentstroke}%
\pgfsetdash{}{0pt}%
\pgfpathmoveto{\pgfqpoint{4.076972in}{1.987384in}}%
\pgfpathlineto{\pgfqpoint{4.090143in}{1.983955in}}%
\pgfpathlineto{\pgfqpoint{4.103320in}{1.980552in}}%
\pgfpathlineto{\pgfqpoint{4.116503in}{1.977175in}}%
\pgfpathlineto{\pgfqpoint{4.129692in}{1.973825in}}%
\pgfpathlineto{\pgfqpoint{4.122100in}{1.966078in}}%
\pgfpathlineto{\pgfqpoint{4.114504in}{1.958375in}}%
\pgfpathlineto{\pgfqpoint{4.106901in}{1.950719in}}%
\pgfpathlineto{\pgfqpoint{4.099293in}{1.943115in}}%
\pgfpathlineto{\pgfqpoint{4.086091in}{1.946621in}}%
\pgfpathlineto{\pgfqpoint{4.072896in}{1.950154in}}%
\pgfpathlineto{\pgfqpoint{4.059706in}{1.953713in}}%
\pgfpathlineto{\pgfqpoint{4.046523in}{1.957297in}}%
\pgfpathlineto{\pgfqpoint{4.054144in}{1.964741in}}%
\pgfpathlineto{\pgfqpoint{4.061759in}{1.972239in}}%
\pgfpathlineto{\pgfqpoint{4.069369in}{1.979788in}}%
\pgfpathlineto{\pgfqpoint{4.076972in}{1.987384in}}%
\pgfpathclose%
\pgfusepath{fill}%
\end{pgfscope}%
\begin{pgfscope}%
\pgfpathrectangle{\pgfqpoint{1.254980in}{0.150000in}}{\pgfqpoint{5.490039in}{5.490039in}}%
\pgfusepath{clip}%
\pgfsetbuttcap%
\pgfsetroundjoin%
\definecolor{currentfill}{rgb}{0.275191,0.194905,0.496005}%
\pgfsetfillcolor{currentfill}%
\pgfsetfillopacity{0.700000}%
\pgfsetlinewidth{0.000000pt}%
\definecolor{currentstroke}{rgb}{0.000000,0.000000,0.000000}%
\pgfsetstrokecolor{currentstroke}%
\pgfsetdash{}{0pt}%
\pgfpathmoveto{\pgfqpoint{5.721988in}{2.328848in}}%
\pgfpathlineto{\pgfqpoint{5.735633in}{2.328307in}}%
\pgfpathlineto{\pgfqpoint{5.749287in}{2.327790in}}%
\pgfpathlineto{\pgfqpoint{5.762950in}{2.327296in}}%
\pgfpathlineto{\pgfqpoint{5.776621in}{2.326827in}}%
\pgfpathlineto{\pgfqpoint{5.769702in}{2.321145in}}%
\pgfpathlineto{\pgfqpoint{5.762775in}{2.315384in}}%
\pgfpathlineto{\pgfqpoint{5.755840in}{2.309540in}}%
\pgfpathlineto{\pgfqpoint{5.748896in}{2.303613in}}%
\pgfpathlineto{\pgfqpoint{5.735209in}{2.304030in}}%
\pgfpathlineto{\pgfqpoint{5.721530in}{2.304470in}}%
\pgfpathlineto{\pgfqpoint{5.707860in}{2.304934in}}%
\pgfpathlineto{\pgfqpoint{5.694199in}{2.305422in}}%
\pgfpathlineto{\pgfqpoint{5.701158in}{2.311398in}}%
\pgfpathlineto{\pgfqpoint{5.708109in}{2.317293in}}%
\pgfpathlineto{\pgfqpoint{5.715053in}{2.323108in}}%
\pgfpathlineto{\pgfqpoint{5.721988in}{2.328848in}}%
\pgfpathclose%
\pgfusepath{fill}%
\end{pgfscope}%
\begin{pgfscope}%
\pgfpathrectangle{\pgfqpoint{1.254980in}{0.150000in}}{\pgfqpoint{5.490039in}{5.490039in}}%
\pgfusepath{clip}%
\pgfsetbuttcap%
\pgfsetroundjoin%
\definecolor{currentfill}{rgb}{0.281446,0.084320,0.407414}%
\pgfsetfillcolor{currentfill}%
\pgfsetfillopacity{0.700000}%
\pgfsetlinewidth{0.000000pt}%
\definecolor{currentstroke}{rgb}{0.000000,0.000000,0.000000}%
\pgfsetstrokecolor{currentstroke}%
\pgfsetdash{}{0pt}%
\pgfpathmoveto{\pgfqpoint{4.816536in}{2.118931in}}%
\pgfpathlineto{\pgfqpoint{4.829906in}{2.117240in}}%
\pgfpathlineto{\pgfqpoint{4.843283in}{2.115573in}}%
\pgfpathlineto{\pgfqpoint{4.856667in}{2.113931in}}%
\pgfpathlineto{\pgfqpoint{4.870060in}{2.112313in}}%
\pgfpathlineto{\pgfqpoint{4.862729in}{2.103866in}}%
\pgfpathlineto{\pgfqpoint{4.855392in}{2.095364in}}%
\pgfpathlineto{\pgfqpoint{4.848049in}{2.086807in}}%
\pgfpathlineto{\pgfqpoint{4.840700in}{2.078197in}}%
\pgfpathlineto{\pgfqpoint{4.827298in}{2.079881in}}%
\pgfpathlineto{\pgfqpoint{4.813903in}{2.081590in}}%
\pgfpathlineto{\pgfqpoint{4.800516in}{2.083323in}}%
\pgfpathlineto{\pgfqpoint{4.787136in}{2.085080in}}%
\pgfpathlineto{\pgfqpoint{4.794495in}{2.093619in}}%
\pgfpathlineto{\pgfqpoint{4.801848in}{2.102109in}}%
\pgfpathlineto{\pgfqpoint{4.809195in}{2.110546in}}%
\pgfpathlineto{\pgfqpoint{4.816536in}{2.118931in}}%
\pgfpathclose%
\pgfusepath{fill}%
\end{pgfscope}%
\begin{pgfscope}%
\pgfpathrectangle{\pgfqpoint{1.254980in}{0.150000in}}{\pgfqpoint{5.490039in}{5.490039in}}%
\pgfusepath{clip}%
\pgfsetbuttcap%
\pgfsetroundjoin%
\definecolor{currentfill}{rgb}{0.276022,0.044167,0.370164}%
\pgfsetfillcolor{currentfill}%
\pgfsetfillopacity{0.700000}%
\pgfsetlinewidth{0.000000pt}%
\definecolor{currentstroke}{rgb}{0.000000,0.000000,0.000000}%
\pgfsetstrokecolor{currentstroke}%
\pgfsetdash{}{0pt}%
\pgfpathmoveto{\pgfqpoint{4.514656in}{2.049331in}}%
\pgfpathlineto{\pgfqpoint{4.527940in}{2.047010in}}%
\pgfpathlineto{\pgfqpoint{4.541232in}{2.044713in}}%
\pgfpathlineto{\pgfqpoint{4.554530in}{2.042441in}}%
\pgfpathlineto{\pgfqpoint{4.567835in}{2.040193in}}%
\pgfpathlineto{\pgfqpoint{4.560396in}{2.031576in}}%
\pgfpathlineto{\pgfqpoint{4.552951in}{2.022936in}}%
\pgfpathlineto{\pgfqpoint{4.545500in}{2.014276in}}%
\pgfpathlineto{\pgfqpoint{4.538045in}{2.005598in}}%
\pgfpathlineto{\pgfqpoint{4.524729in}{2.007951in}}%
\pgfpathlineto{\pgfqpoint{4.511421in}{2.010328in}}%
\pgfpathlineto{\pgfqpoint{4.498119in}{2.012729in}}%
\pgfpathlineto{\pgfqpoint{4.484825in}{2.015156in}}%
\pgfpathlineto{\pgfqpoint{4.492291in}{2.023724in}}%
\pgfpathlineto{\pgfqpoint{4.499751in}{2.032278in}}%
\pgfpathlineto{\pgfqpoint{4.507206in}{2.040814in}}%
\pgfpathlineto{\pgfqpoint{4.514656in}{2.049331in}}%
\pgfpathclose%
\pgfusepath{fill}%
\end{pgfscope}%
\begin{pgfscope}%
\pgfpathrectangle{\pgfqpoint{1.254980in}{0.150000in}}{\pgfqpoint{5.490039in}{5.490039in}}%
\pgfusepath{clip}%
\pgfsetbuttcap%
\pgfsetroundjoin%
\definecolor{currentfill}{rgb}{0.283187,0.125848,0.444960}%
\pgfsetfillcolor{currentfill}%
\pgfsetfillopacity{0.700000}%
\pgfsetlinewidth{0.000000pt}%
\definecolor{currentstroke}{rgb}{0.000000,0.000000,0.000000}%
\pgfsetstrokecolor{currentstroke}%
\pgfsetdash{}{0pt}%
\pgfpathmoveto{\pgfqpoint{5.118468in}{2.192885in}}%
\pgfpathlineto{\pgfqpoint{5.131929in}{2.191700in}}%
\pgfpathlineto{\pgfqpoint{5.145398in}{2.190540in}}%
\pgfpathlineto{\pgfqpoint{5.158875in}{2.189403in}}%
\pgfpathlineto{\pgfqpoint{5.172359in}{2.188291in}}%
\pgfpathlineto{\pgfqpoint{5.165149in}{2.180491in}}%
\pgfpathlineto{\pgfqpoint{5.157931in}{2.172615in}}%
\pgfpathlineto{\pgfqpoint{5.150707in}{2.164662in}}%
\pgfpathlineto{\pgfqpoint{5.143476in}{2.156633in}}%
\pgfpathlineto{\pgfqpoint{5.129980in}{2.157772in}}%
\pgfpathlineto{\pgfqpoint{5.116492in}{2.158936in}}%
\pgfpathlineto{\pgfqpoint{5.103012in}{2.160123in}}%
\pgfpathlineto{\pgfqpoint{5.089540in}{2.161334in}}%
\pgfpathlineto{\pgfqpoint{5.096782in}{2.169332in}}%
\pgfpathlineto{\pgfqpoint{5.104017in}{2.177256in}}%
\pgfpathlineto{\pgfqpoint{5.111246in}{2.185107in}}%
\pgfpathlineto{\pgfqpoint{5.118468in}{2.192885in}}%
\pgfpathclose%
\pgfusepath{fill}%
\end{pgfscope}%
\begin{pgfscope}%
\pgfpathrectangle{\pgfqpoint{1.254980in}{0.150000in}}{\pgfqpoint{5.490039in}{5.490039in}}%
\pgfusepath{clip}%
\pgfsetbuttcap%
\pgfsetroundjoin%
\definecolor{currentfill}{rgb}{0.280255,0.165693,0.476498}%
\pgfsetfillcolor{currentfill}%
\pgfsetfillopacity{0.700000}%
\pgfsetlinewidth{0.000000pt}%
\definecolor{currentstroke}{rgb}{0.000000,0.000000,0.000000}%
\pgfsetstrokecolor{currentstroke}%
\pgfsetdash{}{0pt}%
\pgfpathmoveto{\pgfqpoint{5.420352in}{2.264484in}}%
\pgfpathlineto{\pgfqpoint{5.433906in}{2.263683in}}%
\pgfpathlineto{\pgfqpoint{5.447468in}{2.262906in}}%
\pgfpathlineto{\pgfqpoint{5.461039in}{2.262152in}}%
\pgfpathlineto{\pgfqpoint{5.474618in}{2.261423in}}%
\pgfpathlineto{\pgfqpoint{5.467544in}{2.254596in}}%
\pgfpathlineto{\pgfqpoint{5.460463in}{2.247684in}}%
\pgfpathlineto{\pgfqpoint{5.453374in}{2.240686in}}%
\pgfpathlineto{\pgfqpoint{5.446277in}{2.233602in}}%
\pgfpathlineto{\pgfqpoint{5.432685in}{2.234319in}}%
\pgfpathlineto{\pgfqpoint{5.419101in}{2.235060in}}%
\pgfpathlineto{\pgfqpoint{5.405525in}{2.235824in}}%
\pgfpathlineto{\pgfqpoint{5.391958in}{2.236613in}}%
\pgfpathlineto{\pgfqpoint{5.399068in}{2.243705in}}%
\pgfpathlineto{\pgfqpoint{5.406170in}{2.250713in}}%
\pgfpathlineto{\pgfqpoint{5.413265in}{2.257640in}}%
\pgfpathlineto{\pgfqpoint{5.420352in}{2.264484in}}%
\pgfpathclose%
\pgfusepath{fill}%
\end{pgfscope}%
\begin{pgfscope}%
\pgfpathrectangle{\pgfqpoint{1.254980in}{0.150000in}}{\pgfqpoint{5.490039in}{5.490039in}}%
\pgfusepath{clip}%
\pgfsetbuttcap%
\pgfsetroundjoin%
\definecolor{currentfill}{rgb}{0.268510,0.009605,0.335427}%
\pgfsetfillcolor{currentfill}%
\pgfsetfillopacity{0.700000}%
\pgfsetlinewidth{0.000000pt}%
\definecolor{currentstroke}{rgb}{0.000000,0.000000,0.000000}%
\pgfsetstrokecolor{currentstroke}%
\pgfsetdash{}{0pt}%
\pgfpathmoveto{\pgfqpoint{3.722348in}{1.984513in}}%
\pgfpathlineto{\pgfqpoint{3.735445in}{1.980036in}}%
\pgfpathlineto{\pgfqpoint{3.748548in}{1.975588in}}%
\pgfpathlineto{\pgfqpoint{3.761656in}{1.971167in}}%
\pgfpathlineto{\pgfqpoint{3.774769in}{1.966774in}}%
\pgfpathlineto{\pgfqpoint{3.767035in}{1.960763in}}%
\pgfpathlineto{\pgfqpoint{3.759295in}{1.954862in}}%
\pgfpathlineto{\pgfqpoint{3.751547in}{1.949076in}}%
\pgfpathlineto{\pgfqpoint{3.743792in}{1.943409in}}%
\pgfpathlineto{\pgfqpoint{3.730662in}{1.947996in}}%
\pgfpathlineto{\pgfqpoint{3.717538in}{1.952611in}}%
\pgfpathlineto{\pgfqpoint{3.704420in}{1.957253in}}%
\pgfpathlineto{\pgfqpoint{3.691306in}{1.961924in}}%
\pgfpathlineto{\pgfqpoint{3.699078in}{1.967392in}}%
\pgfpathlineto{\pgfqpoint{3.706842in}{1.972982in}}%
\pgfpathlineto{\pgfqpoint{3.714599in}{1.978691in}}%
\pgfpathlineto{\pgfqpoint{3.722348in}{1.984513in}}%
\pgfpathclose%
\pgfusepath{fill}%
\end{pgfscope}%
\begin{pgfscope}%
\pgfpathrectangle{\pgfqpoint{1.254980in}{0.150000in}}{\pgfqpoint{5.490039in}{5.490039in}}%
\pgfusepath{clip}%
\pgfsetbuttcap%
\pgfsetroundjoin%
\definecolor{currentfill}{rgb}{0.269944,0.014625,0.341379}%
\pgfsetfillcolor{currentfill}%
\pgfsetfillopacity{0.700000}%
\pgfsetlinewidth{0.000000pt}%
\definecolor{currentstroke}{rgb}{0.000000,0.000000,0.000000}%
\pgfsetstrokecolor{currentstroke}%
\pgfsetdash{}{0pt}%
\pgfpathmoveto{\pgfqpoint{3.586586in}{2.000315in}}%
\pgfpathlineto{\pgfqpoint{3.599658in}{1.995415in}}%
\pgfpathlineto{\pgfqpoint{3.612735in}{1.990545in}}%
\pgfpathlineto{\pgfqpoint{3.625817in}{1.985703in}}%
\pgfpathlineto{\pgfqpoint{3.638905in}{1.980890in}}%
\pgfpathlineto{\pgfqpoint{3.631109in}{1.975752in}}%
\pgfpathlineto{\pgfqpoint{3.623305in}{1.970751in}}%
\pgfpathlineto{\pgfqpoint{3.615493in}{1.965891in}}%
\pgfpathlineto{\pgfqpoint{3.607673in}{1.961178in}}%
\pgfpathlineto{\pgfqpoint{3.594568in}{1.966198in}}%
\pgfpathlineto{\pgfqpoint{3.581468in}{1.971247in}}%
\pgfpathlineto{\pgfqpoint{3.568373in}{1.976324in}}%
\pgfpathlineto{\pgfqpoint{3.555283in}{1.981431in}}%
\pgfpathlineto{\pgfqpoint{3.563121in}{1.985933in}}%
\pgfpathlineto{\pgfqpoint{3.570951in}{1.990584in}}%
\pgfpathlineto{\pgfqpoint{3.578772in}{1.995380in}}%
\pgfpathlineto{\pgfqpoint{3.586586in}{2.000315in}}%
\pgfpathclose%
\pgfusepath{fill}%
\end{pgfscope}%
\begin{pgfscope}%
\pgfpathrectangle{\pgfqpoint{1.254980in}{0.150000in}}{\pgfqpoint{5.490039in}{5.490039in}}%
\pgfusepath{clip}%
\pgfsetbuttcap%
\pgfsetroundjoin%
\definecolor{currentfill}{rgb}{0.267004,0.004874,0.329415}%
\pgfsetfillcolor{currentfill}%
\pgfsetfillopacity{0.700000}%
\pgfsetlinewidth{0.000000pt}%
\definecolor{currentstroke}{rgb}{0.000000,0.000000,0.000000}%
\pgfsetstrokecolor{currentstroke}%
\pgfsetdash{}{0pt}%
\pgfpathmoveto{\pgfqpoint{3.858085in}{1.975258in}}%
\pgfpathlineto{\pgfqpoint{3.871211in}{1.971184in}}%
\pgfpathlineto{\pgfqpoint{3.884343in}{1.967137in}}%
\pgfpathlineto{\pgfqpoint{3.897481in}{1.963118in}}%
\pgfpathlineto{\pgfqpoint{3.910624in}{1.959125in}}%
\pgfpathlineto{\pgfqpoint{3.902947in}{1.952367in}}%
\pgfpathlineto{\pgfqpoint{3.895263in}{1.945693in}}%
\pgfpathlineto{\pgfqpoint{3.887572in}{1.939110in}}%
\pgfpathlineto{\pgfqpoint{3.879875in}{1.932621in}}%
\pgfpathlineto{\pgfqpoint{3.866718in}{1.936795in}}%
\pgfpathlineto{\pgfqpoint{3.853565in}{1.940996in}}%
\pgfpathlineto{\pgfqpoint{3.840419in}{1.945224in}}%
\pgfpathlineto{\pgfqpoint{3.827278in}{1.949480in}}%
\pgfpathlineto{\pgfqpoint{3.834989in}{1.955782in}}%
\pgfpathlineto{\pgfqpoint{3.842695in}{1.962183in}}%
\pgfpathlineto{\pgfqpoint{3.850393in}{1.968676in}}%
\pgfpathlineto{\pgfqpoint{3.858085in}{1.975258in}}%
\pgfpathclose%
\pgfusepath{fill}%
\end{pgfscope}%
\begin{pgfscope}%
\pgfpathrectangle{\pgfqpoint{1.254980in}{0.150000in}}{\pgfqpoint{5.490039in}{5.490039in}}%
\pgfusepath{clip}%
\pgfsetbuttcap%
\pgfsetroundjoin%
\definecolor{currentfill}{rgb}{0.273809,0.031497,0.358853}%
\pgfsetfillcolor{currentfill}%
\pgfsetfillopacity{0.700000}%
\pgfsetlinewidth{0.000000pt}%
\definecolor{currentstroke}{rgb}{0.000000,0.000000,0.000000}%
\pgfsetstrokecolor{currentstroke}%
\pgfsetdash{}{0pt}%
\pgfpathmoveto{\pgfqpoint{3.450737in}{2.023351in}}%
\pgfpathlineto{\pgfqpoint{3.463788in}{2.018006in}}%
\pgfpathlineto{\pgfqpoint{3.476844in}{2.012691in}}%
\pgfpathlineto{\pgfqpoint{3.489905in}{2.007407in}}%
\pgfpathlineto{\pgfqpoint{3.502971in}{2.002152in}}%
\pgfpathlineto{\pgfqpoint{3.495106in}{1.998021in}}%
\pgfpathlineto{\pgfqpoint{3.487231in}{1.994055in}}%
\pgfpathlineto{\pgfqpoint{3.479348in}{1.990257in}}%
\pgfpathlineto{\pgfqpoint{3.471455in}{1.986635in}}%
\pgfpathlineto{\pgfqpoint{3.458370in}{1.992110in}}%
\pgfpathlineto{\pgfqpoint{3.445289in}{1.997614in}}%
\pgfpathlineto{\pgfqpoint{3.432213in}{2.003149in}}%
\pgfpathlineto{\pgfqpoint{3.419141in}{2.008714in}}%
\pgfpathlineto{\pgfqpoint{3.427054in}{2.012111in}}%
\pgfpathlineto{\pgfqpoint{3.434958in}{2.015687in}}%
\pgfpathlineto{\pgfqpoint{3.442852in}{2.019435in}}%
\pgfpathlineto{\pgfqpoint{3.450737in}{2.023351in}}%
\pgfpathclose%
\pgfusepath{fill}%
\end{pgfscope}%
\begin{pgfscope}%
\pgfpathrectangle{\pgfqpoint{1.254980in}{0.150000in}}{\pgfqpoint{5.490039in}{5.490039in}}%
\pgfusepath{clip}%
\pgfsetbuttcap%
\pgfsetroundjoin%
\definecolor{currentfill}{rgb}{0.280267,0.073417,0.397163}%
\pgfsetfillcolor{currentfill}%
\pgfsetfillopacity{0.700000}%
\pgfsetlinewidth{0.000000pt}%
\definecolor{currentstroke}{rgb}{0.000000,0.000000,0.000000}%
\pgfsetstrokecolor{currentstroke}%
\pgfsetdash{}{0pt}%
\pgfpathmoveto{\pgfqpoint{4.733691in}{2.092353in}}%
\pgfpathlineto{\pgfqpoint{4.747041in}{2.090498in}}%
\pgfpathlineto{\pgfqpoint{4.760399in}{2.088668in}}%
\pgfpathlineto{\pgfqpoint{4.773764in}{2.086862in}}%
\pgfpathlineto{\pgfqpoint{4.787136in}{2.085080in}}%
\pgfpathlineto{\pgfqpoint{4.779771in}{2.076492in}}%
\pgfpathlineto{\pgfqpoint{4.772401in}{2.067856in}}%
\pgfpathlineto{\pgfqpoint{4.765025in}{2.059175in}}%
\pgfpathlineto{\pgfqpoint{4.757643in}{2.050449in}}%
\pgfpathlineto{\pgfqpoint{4.744261in}{2.052310in}}%
\pgfpathlineto{\pgfqpoint{4.730886in}{2.054195in}}%
\pgfpathlineto{\pgfqpoint{4.717518in}{2.056105in}}%
\pgfpathlineto{\pgfqpoint{4.704158in}{2.058039in}}%
\pgfpathlineto{\pgfqpoint{4.711550in}{2.066680in}}%
\pgfpathlineto{\pgfqpoint{4.718936in}{2.075281in}}%
\pgfpathlineto{\pgfqpoint{4.726316in}{2.083839in}}%
\pgfpathlineto{\pgfqpoint{4.733691in}{2.092353in}}%
\pgfpathclose%
\pgfusepath{fill}%
\end{pgfscope}%
\begin{pgfscope}%
\pgfpathrectangle{\pgfqpoint{1.254980in}{0.150000in}}{\pgfqpoint{5.490039in}{5.490039in}}%
\pgfusepath{clip}%
\pgfsetbuttcap%
\pgfsetroundjoin%
\definecolor{currentfill}{rgb}{0.269944,0.014625,0.341379}%
\pgfsetfillcolor{currentfill}%
\pgfsetfillopacity{0.700000}%
\pgfsetlinewidth{0.000000pt}%
\definecolor{currentstroke}{rgb}{0.000000,0.000000,0.000000}%
\pgfsetstrokecolor{currentstroke}%
\pgfsetdash{}{0pt}%
\pgfpathmoveto{\pgfqpoint{4.212774in}{1.992607in}}%
\pgfpathlineto{\pgfqpoint{4.225983in}{1.989529in}}%
\pgfpathlineto{\pgfqpoint{4.239199in}{1.986476in}}%
\pgfpathlineto{\pgfqpoint{4.252421in}{1.983449in}}%
\pgfpathlineto{\pgfqpoint{4.265649in}{1.980447in}}%
\pgfpathlineto{\pgfqpoint{4.258103in}{1.972284in}}%
\pgfpathlineto{\pgfqpoint{4.250552in}{1.964144in}}%
\pgfpathlineto{\pgfqpoint{4.242995in}{1.956030in}}%
\pgfpathlineto{\pgfqpoint{4.235433in}{1.947947in}}%
\pgfpathlineto{\pgfqpoint{4.222193in}{1.951092in}}%
\pgfpathlineto{\pgfqpoint{4.208959in}{1.954263in}}%
\pgfpathlineto{\pgfqpoint{4.195732in}{1.957459in}}%
\pgfpathlineto{\pgfqpoint{4.182511in}{1.960681in}}%
\pgfpathlineto{\pgfqpoint{4.190085in}{1.968615in}}%
\pgfpathlineto{\pgfqpoint{4.197654in}{1.976584in}}%
\pgfpathlineto{\pgfqpoint{4.205217in}{1.984582in}}%
\pgfpathlineto{\pgfqpoint{4.212774in}{1.992607in}}%
\pgfpathclose%
\pgfusepath{fill}%
\end{pgfscope}%
\begin{pgfscope}%
\pgfpathrectangle{\pgfqpoint{1.254980in}{0.150000in}}{\pgfqpoint{5.490039in}{5.490039in}}%
\pgfusepath{clip}%
\pgfsetbuttcap%
\pgfsetroundjoin%
\definecolor{currentfill}{rgb}{0.274128,0.199721,0.498911}%
\pgfsetfillcolor{currentfill}%
\pgfsetfillopacity{0.700000}%
\pgfsetlinewidth{0.000000pt}%
\definecolor{currentstroke}{rgb}{0.000000,0.000000,0.000000}%
\pgfsetstrokecolor{currentstroke}%
\pgfsetdash{}{0pt}%
\pgfpathmoveto{\pgfqpoint{2.697008in}{2.326666in}}%
\pgfpathlineto{\pgfqpoint{2.709985in}{2.318628in}}%
\pgfpathlineto{\pgfqpoint{2.722964in}{2.310634in}}%
\pgfpathlineto{\pgfqpoint{2.735946in}{2.302683in}}%
\pgfpathlineto{\pgfqpoint{2.748930in}{2.294774in}}%
\pgfpathlineto{\pgfqpoint{2.740565in}{2.297226in}}%
\pgfpathlineto{\pgfqpoint{2.732183in}{2.299984in}}%
\pgfpathlineto{\pgfqpoint{2.723782in}{2.303056in}}%
\pgfpathlineto{\pgfqpoint{2.715363in}{2.306449in}}%
\pgfpathlineto{\pgfqpoint{2.702347in}{2.314634in}}%
\pgfpathlineto{\pgfqpoint{2.689333in}{2.322862in}}%
\pgfpathlineto{\pgfqpoint{2.676321in}{2.331134in}}%
\pgfpathlineto{\pgfqpoint{2.663311in}{2.339448in}}%
\pgfpathlineto{\pgfqpoint{2.671763in}{2.335773in}}%
\pgfpathlineto{\pgfqpoint{2.680196in}{2.332422in}}%
\pgfpathlineto{\pgfqpoint{2.688611in}{2.329389in}}%
\pgfpathlineto{\pgfqpoint{2.697008in}{2.326666in}}%
\pgfpathclose%
\pgfusepath{fill}%
\end{pgfscope}%
\begin{pgfscope}%
\pgfpathrectangle{\pgfqpoint{1.254980in}{0.150000in}}{\pgfqpoint{5.490039in}{5.490039in}}%
\pgfusepath{clip}%
\pgfsetbuttcap%
\pgfsetroundjoin%
\definecolor{currentfill}{rgb}{0.283197,0.115680,0.436115}%
\pgfsetfillcolor{currentfill}%
\pgfsetfillopacity{0.700000}%
\pgfsetlinewidth{0.000000pt}%
\definecolor{currentstroke}{rgb}{0.000000,0.000000,0.000000}%
\pgfsetstrokecolor{currentstroke}%
\pgfsetdash{}{0pt}%
\pgfpathmoveto{\pgfqpoint{5.035730in}{2.166420in}}%
\pgfpathlineto{\pgfqpoint{5.049171in}{2.165113in}}%
\pgfpathlineto{\pgfqpoint{5.062619in}{2.163829in}}%
\pgfpathlineto{\pgfqpoint{5.076075in}{2.162570in}}%
\pgfpathlineto{\pgfqpoint{5.089540in}{2.161334in}}%
\pgfpathlineto{\pgfqpoint{5.082291in}{2.153264in}}%
\pgfpathlineto{\pgfqpoint{5.075035in}{2.145121in}}%
\pgfpathlineto{\pgfqpoint{5.067774in}{2.136907in}}%
\pgfpathlineto{\pgfqpoint{5.060505in}{2.128621in}}%
\pgfpathlineto{\pgfqpoint{5.047030in}{2.129896in}}%
\pgfpathlineto{\pgfqpoint{5.033563in}{2.131196in}}%
\pgfpathlineto{\pgfqpoint{5.020104in}{2.132519in}}%
\pgfpathlineto{\pgfqpoint{5.006653in}{2.133867in}}%
\pgfpathlineto{\pgfqpoint{5.013932in}{2.142108in}}%
\pgfpathlineto{\pgfqpoint{5.021204in}{2.150281in}}%
\pgfpathlineto{\pgfqpoint{5.028471in}{2.158385in}}%
\pgfpathlineto{\pgfqpoint{5.035730in}{2.166420in}}%
\pgfpathclose%
\pgfusepath{fill}%
\end{pgfscope}%
\begin{pgfscope}%
\pgfpathrectangle{\pgfqpoint{1.254980in}{0.150000in}}{\pgfqpoint{5.490039in}{5.490039in}}%
\pgfusepath{clip}%
\pgfsetbuttcap%
\pgfsetroundjoin%
\definecolor{currentfill}{rgb}{0.281924,0.089666,0.412415}%
\pgfsetfillcolor{currentfill}%
\pgfsetfillopacity{0.700000}%
\pgfsetlinewidth{0.000000pt}%
\definecolor{currentstroke}{rgb}{0.000000,0.000000,0.000000}%
\pgfsetstrokecolor{currentstroke}%
\pgfsetdash{}{0pt}%
\pgfpathmoveto{\pgfqpoint{3.126431in}{2.119651in}}%
\pgfpathlineto{\pgfqpoint{3.139441in}{2.113206in}}%
\pgfpathlineto{\pgfqpoint{3.152454in}{2.106795in}}%
\pgfpathlineto{\pgfqpoint{3.165472in}{2.100418in}}%
\pgfpathlineto{\pgfqpoint{3.178493in}{2.094076in}}%
\pgfpathlineto{\pgfqpoint{3.170437in}{2.092627in}}%
\pgfpathlineto{\pgfqpoint{3.162369in}{2.091405in}}%
\pgfpathlineto{\pgfqpoint{3.154289in}{2.090418in}}%
\pgfpathlineto{\pgfqpoint{3.146197in}{2.089672in}}%
\pgfpathlineto{\pgfqpoint{3.133151in}{2.096262in}}%
\pgfpathlineto{\pgfqpoint{3.120109in}{2.102886in}}%
\pgfpathlineto{\pgfqpoint{3.107070in}{2.109544in}}%
\pgfpathlineto{\pgfqpoint{3.094036in}{2.116237in}}%
\pgfpathlineto{\pgfqpoint{3.102154in}{2.116730in}}%
\pgfpathlineto{\pgfqpoint{3.110259in}{2.117468in}}%
\pgfpathlineto{\pgfqpoint{3.118351in}{2.118444in}}%
\pgfpathlineto{\pgfqpoint{3.126431in}{2.119651in}}%
\pgfpathclose%
\pgfusepath{fill}%
\end{pgfscope}%
\begin{pgfscope}%
\pgfpathrectangle{\pgfqpoint{1.254980in}{0.150000in}}{\pgfqpoint{5.490039in}{5.490039in}}%
\pgfusepath{clip}%
\pgfsetbuttcap%
\pgfsetroundjoin%
\definecolor{currentfill}{rgb}{0.276194,0.190074,0.493001}%
\pgfsetfillcolor{currentfill}%
\pgfsetfillopacity{0.700000}%
\pgfsetlinewidth{0.000000pt}%
\definecolor{currentstroke}{rgb}{0.000000,0.000000,0.000000}%
\pgfsetstrokecolor{currentstroke}%
\pgfsetdash{}{0pt}%
\pgfpathmoveto{\pgfqpoint{5.639640in}{2.307610in}}%
\pgfpathlineto{\pgfqpoint{5.653267in}{2.307027in}}%
\pgfpathlineto{\pgfqpoint{5.666902in}{2.306469in}}%
\pgfpathlineto{\pgfqpoint{5.680546in}{2.305934in}}%
\pgfpathlineto{\pgfqpoint{5.694199in}{2.305422in}}%
\pgfpathlineto{\pgfqpoint{5.687231in}{2.299364in}}%
\pgfpathlineto{\pgfqpoint{5.680256in}{2.293222in}}%
\pgfpathlineto{\pgfqpoint{5.673272in}{2.286994in}}%
\pgfpathlineto{\pgfqpoint{5.666281in}{2.280678in}}%
\pgfpathlineto{\pgfqpoint{5.652613in}{2.281150in}}%
\pgfpathlineto{\pgfqpoint{5.638954in}{2.281646in}}%
\pgfpathlineto{\pgfqpoint{5.625304in}{2.282165in}}%
\pgfpathlineto{\pgfqpoint{5.611662in}{2.282708in}}%
\pgfpathlineto{\pgfqpoint{5.618669in}{2.289058in}}%
\pgfpathlineto{\pgfqpoint{5.625667in}{2.295324in}}%
\pgfpathlineto{\pgfqpoint{5.632658in}{2.301507in}}%
\pgfpathlineto{\pgfqpoint{5.639640in}{2.307610in}}%
\pgfpathclose%
\pgfusepath{fill}%
\end{pgfscope}%
\begin{pgfscope}%
\pgfpathrectangle{\pgfqpoint{1.254980in}{0.150000in}}{\pgfqpoint{5.490039in}{5.490039in}}%
\pgfusepath{clip}%
\pgfsetbuttcap%
\pgfsetroundjoin%
\definecolor{currentfill}{rgb}{0.273809,0.031497,0.358853}%
\pgfsetfillcolor{currentfill}%
\pgfsetfillopacity{0.700000}%
\pgfsetlinewidth{0.000000pt}%
\definecolor{currentstroke}{rgb}{0.000000,0.000000,0.000000}%
\pgfsetstrokecolor{currentstroke}%
\pgfsetdash{}{0pt}%
\pgfpathmoveto{\pgfqpoint{4.431716in}{2.025112in}}%
\pgfpathlineto{\pgfqpoint{4.444983in}{2.022585in}}%
\pgfpathlineto{\pgfqpoint{4.458256in}{2.020084in}}%
\pgfpathlineto{\pgfqpoint{4.471537in}{2.017607in}}%
\pgfpathlineto{\pgfqpoint{4.484825in}{2.015156in}}%
\pgfpathlineto{\pgfqpoint{4.477353in}{2.006576in}}%
\pgfpathlineto{\pgfqpoint{4.469877in}{1.997986in}}%
\pgfpathlineto{\pgfqpoint{4.462395in}{1.989388in}}%
\pgfpathlineto{\pgfqpoint{4.454908in}{1.980787in}}%
\pgfpathlineto{\pgfqpoint{4.441610in}{1.983356in}}%
\pgfpathlineto{\pgfqpoint{4.428319in}{1.985950in}}%
\pgfpathlineto{\pgfqpoint{4.415035in}{1.988570in}}%
\pgfpathlineto{\pgfqpoint{4.401757in}{1.991214in}}%
\pgfpathlineto{\pgfqpoint{4.409255in}{1.999693in}}%
\pgfpathlineto{\pgfqpoint{4.416747in}{2.008170in}}%
\pgfpathlineto{\pgfqpoint{4.424234in}{2.016644in}}%
\pgfpathlineto{\pgfqpoint{4.431716in}{2.025112in}}%
\pgfpathclose%
\pgfusepath{fill}%
\end{pgfscope}%
\begin{pgfscope}%
\pgfpathrectangle{\pgfqpoint{1.254980in}{0.150000in}}{\pgfqpoint{5.490039in}{5.490039in}}%
\pgfusepath{clip}%
\pgfsetbuttcap%
\pgfsetroundjoin%
\definecolor{currentfill}{rgb}{0.281412,0.155834,0.469201}%
\pgfsetfillcolor{currentfill}%
\pgfsetfillopacity{0.700000}%
\pgfsetlinewidth{0.000000pt}%
\definecolor{currentstroke}{rgb}{0.000000,0.000000,0.000000}%
\pgfsetstrokecolor{currentstroke}%
\pgfsetdash{}{0pt}%
\pgfpathmoveto{\pgfqpoint{5.337773in}{2.240003in}}%
\pgfpathlineto{\pgfqpoint{5.351307in}{2.239120in}}%
\pgfpathlineto{\pgfqpoint{5.364849in}{2.238260in}}%
\pgfpathlineto{\pgfqpoint{5.378399in}{2.237425in}}%
\pgfpathlineto{\pgfqpoint{5.391958in}{2.236613in}}%
\pgfpathlineto{\pgfqpoint{5.384841in}{2.229436in}}%
\pgfpathlineto{\pgfqpoint{5.377716in}{2.222175in}}%
\pgfpathlineto{\pgfqpoint{5.370584in}{2.214829in}}%
\pgfpathlineto{\pgfqpoint{5.363445in}{2.207397in}}%
\pgfpathlineto{\pgfqpoint{5.349873in}{2.208209in}}%
\pgfpathlineto{\pgfqpoint{5.336310in}{2.209045in}}%
\pgfpathlineto{\pgfqpoint{5.322756in}{2.209905in}}%
\pgfpathlineto{\pgfqpoint{5.309209in}{2.210789in}}%
\pgfpathlineto{\pgfqpoint{5.316361in}{2.218216in}}%
\pgfpathlineto{\pgfqpoint{5.323506in}{2.225560in}}%
\pgfpathlineto{\pgfqpoint{5.330643in}{2.232823in}}%
\pgfpathlineto{\pgfqpoint{5.337773in}{2.240003in}}%
\pgfpathclose%
\pgfusepath{fill}%
\end{pgfscope}%
\begin{pgfscope}%
\pgfpathrectangle{\pgfqpoint{1.254980in}{0.150000in}}{\pgfqpoint{5.490039in}{5.490039in}}%
\pgfusepath{clip}%
\pgfsetbuttcap%
\pgfsetroundjoin%
\definecolor{currentfill}{rgb}{0.267004,0.004874,0.329415}%
\pgfsetfillcolor{currentfill}%
\pgfsetfillopacity{0.700000}%
\pgfsetlinewidth{0.000000pt}%
\definecolor{currentstroke}{rgb}{0.000000,0.000000,0.000000}%
\pgfsetstrokecolor{currentstroke}%
\pgfsetdash{}{0pt}%
\pgfpathmoveto{\pgfqpoint{3.993851in}{1.971900in}}%
\pgfpathlineto{\pgfqpoint{4.007010in}{1.968210in}}%
\pgfpathlineto{\pgfqpoint{4.020175in}{1.964546in}}%
\pgfpathlineto{\pgfqpoint{4.033346in}{1.960909in}}%
\pgfpathlineto{\pgfqpoint{4.046523in}{1.957297in}}%
\pgfpathlineto{\pgfqpoint{4.038897in}{1.949912in}}%
\pgfpathlineto{\pgfqpoint{4.031264in}{1.942589in}}%
\pgfpathlineto{\pgfqpoint{4.023626in}{1.935332in}}%
\pgfpathlineto{\pgfqpoint{4.015982in}{1.928146in}}%
\pgfpathlineto{\pgfqpoint{4.002792in}{1.931925in}}%
\pgfpathlineto{\pgfqpoint{3.989607in}{1.935732in}}%
\pgfpathlineto{\pgfqpoint{3.976429in}{1.939564in}}%
\pgfpathlineto{\pgfqpoint{3.963256in}{1.943423in}}%
\pgfpathlineto{\pgfqpoint{3.970914in}{1.950436in}}%
\pgfpathlineto{\pgfqpoint{3.978565in}{1.957523in}}%
\pgfpathlineto{\pgfqpoint{3.986211in}{1.964679in}}%
\pgfpathlineto{\pgfqpoint{3.993851in}{1.971900in}}%
\pgfpathclose%
\pgfusepath{fill}%
\end{pgfscope}%
\begin{pgfscope}%
\pgfpathrectangle{\pgfqpoint{1.254980in}{0.150000in}}{\pgfqpoint{5.490039in}{5.490039in}}%
\pgfusepath{clip}%
\pgfsetbuttcap%
\pgfsetroundjoin%
\definecolor{currentfill}{rgb}{0.282884,0.135920,0.453427}%
\pgfsetfillcolor{currentfill}%
\pgfsetfillopacity{0.700000}%
\pgfsetlinewidth{0.000000pt}%
\definecolor{currentstroke}{rgb}{0.000000,0.000000,0.000000}%
\pgfsetstrokecolor{currentstroke}%
\pgfsetdash{}{0pt}%
\pgfpathmoveto{\pgfqpoint{2.937896in}{2.199324in}}%
\pgfpathlineto{\pgfqpoint{2.950889in}{2.192198in}}%
\pgfpathlineto{\pgfqpoint{2.963885in}{2.185110in}}%
\pgfpathlineto{\pgfqpoint{2.976885in}{2.178059in}}%
\pgfpathlineto{\pgfqpoint{2.989888in}{2.171045in}}%
\pgfpathlineto{\pgfqpoint{2.981703in}{2.171314in}}%
\pgfpathlineto{\pgfqpoint{2.973504in}{2.171848in}}%
\pgfpathlineto{\pgfqpoint{2.965290in}{2.172653in}}%
\pgfpathlineto{\pgfqpoint{2.957061in}{2.173736in}}%
\pgfpathlineto{\pgfqpoint{2.944030in}{2.181011in}}%
\pgfpathlineto{\pgfqpoint{2.931002in}{2.188323in}}%
\pgfpathlineto{\pgfqpoint{2.917978in}{2.195673in}}%
\pgfpathlineto{\pgfqpoint{2.904956in}{2.203061in}}%
\pgfpathlineto{\pgfqpoint{2.913214in}{2.201711in}}%
\pgfpathlineto{\pgfqpoint{2.921457in}{2.200643in}}%
\pgfpathlineto{\pgfqpoint{2.929684in}{2.199850in}}%
\pgfpathlineto{\pgfqpoint{2.937896in}{2.199324in}}%
\pgfpathclose%
\pgfusepath{fill}%
\end{pgfscope}%
\begin{pgfscope}%
\pgfpathrectangle{\pgfqpoint{1.254980in}{0.150000in}}{\pgfqpoint{5.490039in}{5.490039in}}%
\pgfusepath{clip}%
\pgfsetbuttcap%
\pgfsetroundjoin%
\definecolor{currentfill}{rgb}{0.271828,0.209303,0.504434}%
\pgfsetfillcolor{currentfill}%
\pgfsetfillopacity{0.700000}%
\pgfsetlinewidth{0.000000pt}%
\definecolor{currentstroke}{rgb}{0.000000,0.000000,0.000000}%
\pgfsetstrokecolor{currentstroke}%
\pgfsetdash{}{0pt}%
\pgfpathmoveto{\pgfqpoint{5.858921in}{2.346885in}}%
\pgfpathlineto{\pgfqpoint{5.872620in}{2.346467in}}%
\pgfpathlineto{\pgfqpoint{5.886327in}{2.346072in}}%
\pgfpathlineto{\pgfqpoint{5.900043in}{2.345701in}}%
\pgfpathlineto{\pgfqpoint{5.893186in}{2.340442in}}%
\pgfpathlineto{\pgfqpoint{5.886321in}{2.335107in}}%
\pgfpathlineto{\pgfqpoint{5.879448in}{2.329693in}}%
\pgfpathlineto{\pgfqpoint{5.872567in}{2.324197in}}%
\pgfpathlineto{\pgfqpoint{5.858834in}{2.324502in}}%
\pgfpathlineto{\pgfqpoint{5.845110in}{2.324830in}}%
\pgfpathlineto{\pgfqpoint{5.831394in}{2.325182in}}%
\pgfpathlineto{\pgfqpoint{5.838288in}{2.330725in}}%
\pgfpathlineto{\pgfqpoint{5.845174in}{2.336188in}}%
\pgfpathlineto{\pgfqpoint{5.852052in}{2.341574in}}%
\pgfpathlineto{\pgfqpoint{5.858921in}{2.346885in}}%
\pgfpathclose%
\pgfusepath{fill}%
\end{pgfscope}%
\begin{pgfscope}%
\pgfpathrectangle{\pgfqpoint{1.254980in}{0.150000in}}{\pgfqpoint{5.490039in}{5.490039in}}%
\pgfusepath{clip}%
\pgfsetbuttcap%
\pgfsetroundjoin%
\definecolor{currentfill}{rgb}{0.277941,0.056324,0.381191}%
\pgfsetfillcolor{currentfill}%
\pgfsetfillopacity{0.700000}%
\pgfsetlinewidth{0.000000pt}%
\definecolor{currentstroke}{rgb}{0.000000,0.000000,0.000000}%
\pgfsetstrokecolor{currentstroke}%
\pgfsetdash{}{0pt}%
\pgfpathmoveto{\pgfqpoint{3.314732in}{2.054348in}}%
\pgfpathlineto{\pgfqpoint{3.327767in}{2.048534in}}%
\pgfpathlineto{\pgfqpoint{3.340807in}{2.042751in}}%
\pgfpathlineto{\pgfqpoint{3.353852in}{2.037001in}}%
\pgfpathlineto{\pgfqpoint{3.366901in}{2.031281in}}%
\pgfpathlineto{\pgfqpoint{3.358957in}{2.028297in}}%
\pgfpathlineto{\pgfqpoint{3.351003in}{2.025506in}}%
\pgfpathlineto{\pgfqpoint{3.343039in}{2.022914in}}%
\pgfpathlineto{\pgfqpoint{3.335065in}{2.020528in}}%
\pgfpathlineto{\pgfqpoint{3.321994in}{2.026480in}}%
\pgfpathlineto{\pgfqpoint{3.308927in}{2.032464in}}%
\pgfpathlineto{\pgfqpoint{3.295865in}{2.038480in}}%
\pgfpathlineto{\pgfqpoint{3.282807in}{2.044528in}}%
\pgfpathlineto{\pgfqpoint{3.290805in}{2.046675in}}%
\pgfpathlineto{\pgfqpoint{3.298791in}{2.049032in}}%
\pgfpathlineto{\pgfqpoint{3.306766in}{2.051591in}}%
\pgfpathlineto{\pgfqpoint{3.314732in}{2.054348in}}%
\pgfpathclose%
\pgfusepath{fill}%
\end{pgfscope}%
\begin{pgfscope}%
\pgfpathrectangle{\pgfqpoint{1.254980in}{0.150000in}}{\pgfqpoint{5.490039in}{5.490039in}}%
\pgfusepath{clip}%
\pgfsetbuttcap%
\pgfsetroundjoin%
\definecolor{currentfill}{rgb}{0.250425,0.274290,0.533103}%
\pgfsetfillcolor{currentfill}%
\pgfsetfillopacity{0.700000}%
\pgfsetlinewidth{0.000000pt}%
\definecolor{currentstroke}{rgb}{0.000000,0.000000,0.000000}%
\pgfsetstrokecolor{currentstroke}%
\pgfsetdash{}{0pt}%
\pgfpathmoveto{\pgfqpoint{2.455432in}{2.478764in}}%
\pgfpathlineto{\pgfqpoint{2.468411in}{2.469691in}}%
\pgfpathlineto{\pgfqpoint{2.481392in}{2.460670in}}%
\pgfpathlineto{\pgfqpoint{2.494375in}{2.451699in}}%
\pgfpathlineto{\pgfqpoint{2.507359in}{2.442778in}}%
\pgfpathlineto{\pgfqpoint{2.498783in}{2.447648in}}%
\pgfpathlineto{\pgfqpoint{2.490185in}{2.452869in}}%
\pgfpathlineto{\pgfqpoint{2.481566in}{2.458450in}}%
\pgfpathlineto{\pgfqpoint{2.472925in}{2.464397in}}%
\pgfpathlineto{\pgfqpoint{2.459904in}{2.473612in}}%
\pgfpathlineto{\pgfqpoint{2.446884in}{2.482877in}}%
\pgfpathlineto{\pgfqpoint{2.433866in}{2.492193in}}%
\pgfpathlineto{\pgfqpoint{2.420849in}{2.501560in}}%
\pgfpathlineto{\pgfqpoint{2.429528in}{2.495312in}}%
\pgfpathlineto{\pgfqpoint{2.438185in}{2.489436in}}%
\pgfpathlineto{\pgfqpoint{2.446819in}{2.483922in}}%
\pgfpathlineto{\pgfqpoint{2.455432in}{2.478764in}}%
\pgfpathclose%
\pgfusepath{fill}%
\end{pgfscope}%
\begin{pgfscope}%
\pgfpathrectangle{\pgfqpoint{1.254980in}{0.150000in}}{\pgfqpoint{5.490039in}{5.490039in}}%
\pgfusepath{clip}%
\pgfsetbuttcap%
\pgfsetroundjoin%
\definecolor{currentfill}{rgb}{0.278791,0.062145,0.386592}%
\pgfsetfillcolor{currentfill}%
\pgfsetfillopacity{0.700000}%
\pgfsetlinewidth{0.000000pt}%
\definecolor{currentstroke}{rgb}{0.000000,0.000000,0.000000}%
\pgfsetstrokecolor{currentstroke}%
\pgfsetdash{}{0pt}%
\pgfpathmoveto{\pgfqpoint{4.650791in}{2.066020in}}%
\pgfpathlineto{\pgfqpoint{4.664121in}{2.063988in}}%
\pgfpathlineto{\pgfqpoint{4.677460in}{2.061980in}}%
\pgfpathlineto{\pgfqpoint{4.690805in}{2.059997in}}%
\pgfpathlineto{\pgfqpoint{4.704158in}{2.058039in}}%
\pgfpathlineto{\pgfqpoint{4.696761in}{2.049359in}}%
\pgfpathlineto{\pgfqpoint{4.689358in}{2.040641in}}%
\pgfpathlineto{\pgfqpoint{4.681949in}{2.031888in}}%
\pgfpathlineto{\pgfqpoint{4.674535in}{2.023101in}}%
\pgfpathlineto{\pgfqpoint{4.661172in}{2.025152in}}%
\pgfpathlineto{\pgfqpoint{4.647817in}{2.027227in}}%
\pgfpathlineto{\pgfqpoint{4.634469in}{2.029327in}}%
\pgfpathlineto{\pgfqpoint{4.621128in}{2.031451in}}%
\pgfpathlineto{\pgfqpoint{4.628552in}{2.040140in}}%
\pgfpathlineto{\pgfqpoint{4.635970in}{2.048799in}}%
\pgfpathlineto{\pgfqpoint{4.643383in}{2.057427in}}%
\pgfpathlineto{\pgfqpoint{4.650791in}{2.066020in}}%
\pgfpathclose%
\pgfusepath{fill}%
\end{pgfscope}%
\begin{pgfscope}%
\pgfpathrectangle{\pgfqpoint{1.254980in}{0.150000in}}{\pgfqpoint{5.490039in}{5.490039in}}%
\pgfusepath{clip}%
\pgfsetbuttcap%
\pgfsetroundjoin%
\definecolor{currentfill}{rgb}{0.282910,0.105393,0.426902}%
\pgfsetfillcolor{currentfill}%
\pgfsetfillopacity{0.700000}%
\pgfsetlinewidth{0.000000pt}%
\definecolor{currentstroke}{rgb}{0.000000,0.000000,0.000000}%
\pgfsetstrokecolor{currentstroke}%
\pgfsetdash{}{0pt}%
\pgfpathmoveto{\pgfqpoint{4.952926in}{2.139498in}}%
\pgfpathlineto{\pgfqpoint{4.966346in}{2.138054in}}%
\pgfpathlineto{\pgfqpoint{4.979774in}{2.136634in}}%
\pgfpathlineto{\pgfqpoint{4.993209in}{2.135239in}}%
\pgfpathlineto{\pgfqpoint{5.006653in}{2.133867in}}%
\pgfpathlineto{\pgfqpoint{4.999367in}{2.125559in}}%
\pgfpathlineto{\pgfqpoint{4.992076in}{2.117184in}}%
\pgfpathlineto{\pgfqpoint{4.984778in}{2.108743in}}%
\pgfpathlineto{\pgfqpoint{4.977474in}{2.100238in}}%
\pgfpathlineto{\pgfqpoint{4.964020in}{2.101663in}}%
\pgfpathlineto{\pgfqpoint{4.950574in}{2.103112in}}%
\pgfpathlineto{\pgfqpoint{4.937135in}{2.104585in}}%
\pgfpathlineto{\pgfqpoint{4.923705in}{2.106082in}}%
\pgfpathlineto{\pgfqpoint{4.931019in}{2.114529in}}%
\pgfpathlineto{\pgfqpoint{4.938328in}{2.122915in}}%
\pgfpathlineto{\pgfqpoint{4.945630in}{2.131238in}}%
\pgfpathlineto{\pgfqpoint{4.952926in}{2.139498in}}%
\pgfpathclose%
\pgfusepath{fill}%
\end{pgfscope}%
\begin{pgfscope}%
\pgfpathrectangle{\pgfqpoint{1.254980in}{0.150000in}}{\pgfqpoint{5.490039in}{5.490039in}}%
\pgfusepath{clip}%
\pgfsetbuttcap%
\pgfsetroundjoin%
\definecolor{currentfill}{rgb}{0.282290,0.145912,0.461510}%
\pgfsetfillcolor{currentfill}%
\pgfsetfillopacity{0.700000}%
\pgfsetlinewidth{0.000000pt}%
\definecolor{currentstroke}{rgb}{0.000000,0.000000,0.000000}%
\pgfsetstrokecolor{currentstroke}%
\pgfsetdash{}{0pt}%
\pgfpathmoveto{\pgfqpoint{5.255106in}{2.214563in}}%
\pgfpathlineto{\pgfqpoint{5.268620in}{2.213584in}}%
\pgfpathlineto{\pgfqpoint{5.282141in}{2.212629in}}%
\pgfpathlineto{\pgfqpoint{5.295671in}{2.211697in}}%
\pgfpathlineto{\pgfqpoint{5.309209in}{2.210789in}}%
\pgfpathlineto{\pgfqpoint{5.302050in}{2.203280in}}%
\pgfpathlineto{\pgfqpoint{5.294884in}{2.195687in}}%
\pgfpathlineto{\pgfqpoint{5.287711in}{2.188010in}}%
\pgfpathlineto{\pgfqpoint{5.280530in}{2.180251in}}%
\pgfpathlineto{\pgfqpoint{5.266980in}{2.181172in}}%
\pgfpathlineto{\pgfqpoint{5.253438in}{2.182117in}}%
\pgfpathlineto{\pgfqpoint{5.239905in}{2.183086in}}%
\pgfpathlineto{\pgfqpoint{5.226379in}{2.184079in}}%
\pgfpathlineto{\pgfqpoint{5.233572in}{2.191821in}}%
\pgfpathlineto{\pgfqpoint{5.240757in}{2.199482in}}%
\pgfpathlineto{\pgfqpoint{5.247935in}{2.207062in}}%
\pgfpathlineto{\pgfqpoint{5.255106in}{2.214563in}}%
\pgfpathclose%
\pgfusepath{fill}%
\end{pgfscope}%
\begin{pgfscope}%
\pgfpathrectangle{\pgfqpoint{1.254980in}{0.150000in}}{\pgfqpoint{5.490039in}{5.490039in}}%
\pgfusepath{clip}%
\pgfsetbuttcap%
\pgfsetroundjoin%
\definecolor{currentfill}{rgb}{0.278012,0.180367,0.486697}%
\pgfsetfillcolor{currentfill}%
\pgfsetfillopacity{0.700000}%
\pgfsetlinewidth{0.000000pt}%
\definecolor{currentstroke}{rgb}{0.000000,0.000000,0.000000}%
\pgfsetstrokecolor{currentstroke}%
\pgfsetdash{}{0pt}%
\pgfpathmoveto{\pgfqpoint{5.557181in}{2.285115in}}%
\pgfpathlineto{\pgfqpoint{5.570789in}{2.284478in}}%
\pgfpathlineto{\pgfqpoint{5.584405in}{2.283864in}}%
\pgfpathlineto{\pgfqpoint{5.598029in}{2.283274in}}%
\pgfpathlineto{\pgfqpoint{5.611662in}{2.282708in}}%
\pgfpathlineto{\pgfqpoint{5.604648in}{2.276272in}}%
\pgfpathlineto{\pgfqpoint{5.597625in}{2.269749in}}%
\pgfpathlineto{\pgfqpoint{5.590595in}{2.263138in}}%
\pgfpathlineto{\pgfqpoint{5.583556in}{2.256438in}}%
\pgfpathlineto{\pgfqpoint{5.569909in}{2.256978in}}%
\pgfpathlineto{\pgfqpoint{5.556270in}{2.257542in}}%
\pgfpathlineto{\pgfqpoint{5.542640in}{2.258129in}}%
\pgfpathlineto{\pgfqpoint{5.529019in}{2.258740in}}%
\pgfpathlineto{\pgfqpoint{5.536071in}{2.265462in}}%
\pgfpathlineto{\pgfqpoint{5.543116in}{2.272098in}}%
\pgfpathlineto{\pgfqpoint{5.550152in}{2.278648in}}%
\pgfpathlineto{\pgfqpoint{5.557181in}{2.285115in}}%
\pgfpathclose%
\pgfusepath{fill}%
\end{pgfscope}%
\begin{pgfscope}%
\pgfpathrectangle{\pgfqpoint{1.254980in}{0.150000in}}{\pgfqpoint{5.490039in}{5.490039in}}%
\pgfusepath{clip}%
\pgfsetbuttcap%
\pgfsetroundjoin%
\definecolor{currentfill}{rgb}{0.271305,0.019942,0.347269}%
\pgfsetfillcolor{currentfill}%
\pgfsetfillopacity{0.700000}%
\pgfsetlinewidth{0.000000pt}%
\definecolor{currentstroke}{rgb}{0.000000,0.000000,0.000000}%
\pgfsetstrokecolor{currentstroke}%
\pgfsetdash{}{0pt}%
\pgfpathmoveto{\pgfqpoint{4.348715in}{2.002041in}}%
\pgfpathlineto{\pgfqpoint{4.361965in}{1.999296in}}%
\pgfpathlineto{\pgfqpoint{4.375223in}{1.996577in}}%
\pgfpathlineto{\pgfqpoint{4.388486in}{1.993883in}}%
\pgfpathlineto{\pgfqpoint{4.401757in}{1.991214in}}%
\pgfpathlineto{\pgfqpoint{4.394254in}{1.982737in}}%
\pgfpathlineto{\pgfqpoint{4.386746in}{1.974264in}}%
\pgfpathlineto{\pgfqpoint{4.379233in}{1.965799in}}%
\pgfpathlineto{\pgfqpoint{4.371714in}{1.957345in}}%
\pgfpathlineto{\pgfqpoint{4.358433in}{1.960144in}}%
\pgfpathlineto{\pgfqpoint{4.345158in}{1.962969in}}%
\pgfpathlineto{\pgfqpoint{4.331890in}{1.965819in}}%
\pgfpathlineto{\pgfqpoint{4.318628in}{1.968694in}}%
\pgfpathlineto{\pgfqpoint{4.326158in}{1.977013in}}%
\pgfpathlineto{\pgfqpoint{4.333682in}{1.985346in}}%
\pgfpathlineto{\pgfqpoint{4.341201in}{1.993690in}}%
\pgfpathlineto{\pgfqpoint{4.348715in}{2.002041in}}%
\pgfpathclose%
\pgfusepath{fill}%
\end{pgfscope}%
\begin{pgfscope}%
\pgfpathrectangle{\pgfqpoint{1.254980in}{0.150000in}}{\pgfqpoint{5.490039in}{5.490039in}}%
\pgfusepath{clip}%
\pgfsetbuttcap%
\pgfsetroundjoin%
\definecolor{currentfill}{rgb}{0.268510,0.009605,0.335427}%
\pgfsetfillcolor{currentfill}%
\pgfsetfillopacity{0.700000}%
\pgfsetlinewidth{0.000000pt}%
\definecolor{currentstroke}{rgb}{0.000000,0.000000,0.000000}%
\pgfsetstrokecolor{currentstroke}%
\pgfsetdash{}{0pt}%
\pgfpathmoveto{\pgfqpoint{4.129692in}{1.973825in}}%
\pgfpathlineto{\pgfqpoint{4.142887in}{1.970500in}}%
\pgfpathlineto{\pgfqpoint{4.156089in}{1.967201in}}%
\pgfpathlineto{\pgfqpoint{4.169297in}{1.963928in}}%
\pgfpathlineto{\pgfqpoint{4.182511in}{1.960681in}}%
\pgfpathlineto{\pgfqpoint{4.174932in}{1.952783in}}%
\pgfpathlineto{\pgfqpoint{4.167347in}{1.944925in}}%
\pgfpathlineto{\pgfqpoint{4.159757in}{1.937112in}}%
\pgfpathlineto{\pgfqpoint{4.152161in}{1.929347in}}%
\pgfpathlineto{\pgfqpoint{4.138935in}{1.932750in}}%
\pgfpathlineto{\pgfqpoint{4.125715in}{1.936179in}}%
\pgfpathlineto{\pgfqpoint{4.112501in}{1.939634in}}%
\pgfpathlineto{\pgfqpoint{4.099293in}{1.943115in}}%
\pgfpathlineto{\pgfqpoint{4.106901in}{1.950719in}}%
\pgfpathlineto{\pgfqpoint{4.114504in}{1.958375in}}%
\pgfpathlineto{\pgfqpoint{4.122100in}{1.966078in}}%
\pgfpathlineto{\pgfqpoint{4.129692in}{1.973825in}}%
\pgfpathclose%
\pgfusepath{fill}%
\end{pgfscope}%
\begin{pgfscope}%
\pgfpathrectangle{\pgfqpoint{1.254980in}{0.150000in}}{\pgfqpoint{5.490039in}{5.490039in}}%
\pgfusepath{clip}%
\pgfsetbuttcap%
\pgfsetroundjoin%
\definecolor{currentfill}{rgb}{0.276194,0.190074,0.493001}%
\pgfsetfillcolor{currentfill}%
\pgfsetfillopacity{0.700000}%
\pgfsetlinewidth{0.000000pt}%
\definecolor{currentstroke}{rgb}{0.000000,0.000000,0.000000}%
\pgfsetstrokecolor{currentstroke}%
\pgfsetdash{}{0pt}%
\pgfpathmoveto{\pgfqpoint{2.748930in}{2.294774in}}%
\pgfpathlineto{\pgfqpoint{2.761917in}{2.286908in}}%
\pgfpathlineto{\pgfqpoint{2.774907in}{2.279084in}}%
\pgfpathlineto{\pgfqpoint{2.787899in}{2.271301in}}%
\pgfpathlineto{\pgfqpoint{2.800894in}{2.263559in}}%
\pgfpathlineto{\pgfqpoint{2.792560in}{2.265739in}}%
\pgfpathlineto{\pgfqpoint{2.784209in}{2.268222in}}%
\pgfpathlineto{\pgfqpoint{2.775840in}{2.271015in}}%
\pgfpathlineto{\pgfqpoint{2.767454in}{2.274127in}}%
\pgfpathlineto{\pgfqpoint{2.754428in}{2.282145in}}%
\pgfpathlineto{\pgfqpoint{2.741404in}{2.290205in}}%
\pgfpathlineto{\pgfqpoint{2.728382in}{2.298306in}}%
\pgfpathlineto{\pgfqpoint{2.715363in}{2.306449in}}%
\pgfpathlineto{\pgfqpoint{2.723782in}{2.303056in}}%
\pgfpathlineto{\pgfqpoint{2.732183in}{2.299984in}}%
\pgfpathlineto{\pgfqpoint{2.740565in}{2.297226in}}%
\pgfpathlineto{\pgfqpoint{2.748930in}{2.294774in}}%
\pgfpathclose%
\pgfusepath{fill}%
\end{pgfscope}%
\begin{pgfscope}%
\pgfpathrectangle{\pgfqpoint{1.254980in}{0.150000in}}{\pgfqpoint{5.490039in}{5.490039in}}%
\pgfusepath{clip}%
\pgfsetbuttcap%
\pgfsetroundjoin%
\definecolor{currentfill}{rgb}{0.269944,0.014625,0.341379}%
\pgfsetfillcolor{currentfill}%
\pgfsetfillopacity{0.700000}%
\pgfsetlinewidth{0.000000pt}%
\definecolor{currentstroke}{rgb}{0.000000,0.000000,0.000000}%
\pgfsetstrokecolor{currentstroke}%
\pgfsetdash{}{0pt}%
\pgfpathmoveto{\pgfqpoint{3.638905in}{1.980890in}}%
\pgfpathlineto{\pgfqpoint{3.651997in}{1.976106in}}%
\pgfpathlineto{\pgfqpoint{3.665095in}{1.971350in}}%
\pgfpathlineto{\pgfqpoint{3.678198in}{1.966623in}}%
\pgfpathlineto{\pgfqpoint{3.691306in}{1.961924in}}%
\pgfpathlineto{\pgfqpoint{3.683527in}{1.956584in}}%
\pgfpathlineto{\pgfqpoint{3.675741in}{1.951377in}}%
\pgfpathlineto{\pgfqpoint{3.667947in}{1.946309in}}%
\pgfpathlineto{\pgfqpoint{3.660144in}{1.941384in}}%
\pgfpathlineto{\pgfqpoint{3.647019in}{1.946290in}}%
\pgfpathlineto{\pgfqpoint{3.633899in}{1.951224in}}%
\pgfpathlineto{\pgfqpoint{3.620783in}{1.956187in}}%
\pgfpathlineto{\pgfqpoint{3.607673in}{1.961178in}}%
\pgfpathlineto{\pgfqpoint{3.615493in}{1.965891in}}%
\pgfpathlineto{\pgfqpoint{3.623305in}{1.970751in}}%
\pgfpathlineto{\pgfqpoint{3.631109in}{1.975752in}}%
\pgfpathlineto{\pgfqpoint{3.638905in}{1.980890in}}%
\pgfpathclose%
\pgfusepath{fill}%
\end{pgfscope}%
\begin{pgfscope}%
\pgfpathrectangle{\pgfqpoint{1.254980in}{0.150000in}}{\pgfqpoint{5.490039in}{5.490039in}}%
\pgfusepath{clip}%
\pgfsetbuttcap%
\pgfsetroundjoin%
\definecolor{currentfill}{rgb}{0.267004,0.004874,0.329415}%
\pgfsetfillcolor{currentfill}%
\pgfsetfillopacity{0.700000}%
\pgfsetlinewidth{0.000000pt}%
\definecolor{currentstroke}{rgb}{0.000000,0.000000,0.000000}%
\pgfsetstrokecolor{currentstroke}%
\pgfsetdash{}{0pt}%
\pgfpathmoveto{\pgfqpoint{3.774769in}{1.966774in}}%
\pgfpathlineto{\pgfqpoint{3.787888in}{1.962409in}}%
\pgfpathlineto{\pgfqpoint{3.801012in}{1.958072in}}%
\pgfpathlineto{\pgfqpoint{3.814142in}{1.953762in}}%
\pgfpathlineto{\pgfqpoint{3.827278in}{1.949480in}}%
\pgfpathlineto{\pgfqpoint{3.819559in}{1.943279in}}%
\pgfpathlineto{\pgfqpoint{3.811834in}{1.937185in}}%
\pgfpathlineto{\pgfqpoint{3.804102in}{1.931203in}}%
\pgfpathlineto{\pgfqpoint{3.796363in}{1.925337in}}%
\pgfpathlineto{\pgfqpoint{3.783212in}{1.929814in}}%
\pgfpathlineto{\pgfqpoint{3.770067in}{1.934318in}}%
\pgfpathlineto{\pgfqpoint{3.756926in}{1.938850in}}%
\pgfpathlineto{\pgfqpoint{3.743792in}{1.943409in}}%
\pgfpathlineto{\pgfqpoint{3.751547in}{1.949076in}}%
\pgfpathlineto{\pgfqpoint{3.759295in}{1.954862in}}%
\pgfpathlineto{\pgfqpoint{3.767035in}{1.960763in}}%
\pgfpathlineto{\pgfqpoint{3.774769in}{1.966774in}}%
\pgfpathclose%
\pgfusepath{fill}%
\end{pgfscope}%
\begin{pgfscope}%
\pgfpathrectangle{\pgfqpoint{1.254980in}{0.150000in}}{\pgfqpoint{5.490039in}{5.490039in}}%
\pgfusepath{clip}%
\pgfsetbuttcap%
\pgfsetroundjoin%
\definecolor{currentfill}{rgb}{0.277018,0.050344,0.375715}%
\pgfsetfillcolor{currentfill}%
\pgfsetfillopacity{0.700000}%
\pgfsetlinewidth{0.000000pt}%
\definecolor{currentstroke}{rgb}{0.000000,0.000000,0.000000}%
\pgfsetstrokecolor{currentstroke}%
\pgfsetdash{}{0pt}%
\pgfpathmoveto{\pgfqpoint{4.567835in}{2.040193in}}%
\pgfpathlineto{\pgfqpoint{4.581147in}{2.037971in}}%
\pgfpathlineto{\pgfqpoint{4.594467in}{2.035773in}}%
\pgfpathlineto{\pgfqpoint{4.607794in}{2.033599in}}%
\pgfpathlineto{\pgfqpoint{4.621128in}{2.031451in}}%
\pgfpathlineto{\pgfqpoint{4.613698in}{2.022733in}}%
\pgfpathlineto{\pgfqpoint{4.606263in}{2.013990in}}%
\pgfpathlineto{\pgfqpoint{4.598823in}{2.005224in}}%
\pgfpathlineto{\pgfqpoint{4.591377in}{1.996436in}}%
\pgfpathlineto{\pgfqpoint{4.578034in}{1.998689in}}%
\pgfpathlineto{\pgfqpoint{4.564697in}{2.000968in}}%
\pgfpathlineto{\pgfqpoint{4.551367in}{2.003271in}}%
\pgfpathlineto{\pgfqpoint{4.538045in}{2.005598in}}%
\pgfpathlineto{\pgfqpoint{4.545500in}{2.014276in}}%
\pgfpathlineto{\pgfqpoint{4.552951in}{2.022936in}}%
\pgfpathlineto{\pgfqpoint{4.560396in}{2.031576in}}%
\pgfpathlineto{\pgfqpoint{4.567835in}{2.040193in}}%
\pgfpathclose%
\pgfusepath{fill}%
\end{pgfscope}%
\begin{pgfscope}%
\pgfpathrectangle{\pgfqpoint{1.254980in}{0.150000in}}{\pgfqpoint{5.490039in}{5.490039in}}%
\pgfusepath{clip}%
\pgfsetbuttcap%
\pgfsetroundjoin%
\definecolor{currentfill}{rgb}{0.281924,0.089666,0.412415}%
\pgfsetfillcolor{currentfill}%
\pgfsetfillopacity{0.700000}%
\pgfsetlinewidth{0.000000pt}%
\definecolor{currentstroke}{rgb}{0.000000,0.000000,0.000000}%
\pgfsetstrokecolor{currentstroke}%
\pgfsetdash{}{0pt}%
\pgfpathmoveto{\pgfqpoint{4.870060in}{2.112313in}}%
\pgfpathlineto{\pgfqpoint{4.883459in}{2.110719in}}%
\pgfpathlineto{\pgfqpoint{4.896867in}{2.109149in}}%
\pgfpathlineto{\pgfqpoint{4.910282in}{2.107603in}}%
\pgfpathlineto{\pgfqpoint{4.923705in}{2.106082in}}%
\pgfpathlineto{\pgfqpoint{4.916384in}{2.097575in}}%
\pgfpathlineto{\pgfqpoint{4.909058in}{2.089008in}}%
\pgfpathlineto{\pgfqpoint{4.901725in}{2.080383in}}%
\pgfpathlineto{\pgfqpoint{4.894386in}{2.071701in}}%
\pgfpathlineto{\pgfqpoint{4.880953in}{2.073289in}}%
\pgfpathlineto{\pgfqpoint{4.867528in}{2.074901in}}%
\pgfpathlineto{\pgfqpoint{4.854111in}{2.076536in}}%
\pgfpathlineto{\pgfqpoint{4.840700in}{2.078197in}}%
\pgfpathlineto{\pgfqpoint{4.848049in}{2.086807in}}%
\pgfpathlineto{\pgfqpoint{4.855392in}{2.095364in}}%
\pgfpathlineto{\pgfqpoint{4.862729in}{2.103866in}}%
\pgfpathlineto{\pgfqpoint{4.870060in}{2.112313in}}%
\pgfpathclose%
\pgfusepath{fill}%
\end{pgfscope}%
\begin{pgfscope}%
\pgfpathrectangle{\pgfqpoint{1.254980in}{0.150000in}}{\pgfqpoint{5.490039in}{5.490039in}}%
\pgfusepath{clip}%
\pgfsetbuttcap%
\pgfsetroundjoin%
\definecolor{currentfill}{rgb}{0.272594,0.025563,0.353093}%
\pgfsetfillcolor{currentfill}%
\pgfsetfillopacity{0.700000}%
\pgfsetlinewidth{0.000000pt}%
\definecolor{currentstroke}{rgb}{0.000000,0.000000,0.000000}%
\pgfsetstrokecolor{currentstroke}%
\pgfsetdash{}{0pt}%
\pgfpathmoveto{\pgfqpoint{3.502971in}{2.002152in}}%
\pgfpathlineto{\pgfqpoint{3.516042in}{1.996928in}}%
\pgfpathlineto{\pgfqpoint{3.529117in}{1.991733in}}%
\pgfpathlineto{\pgfqpoint{3.542198in}{1.986567in}}%
\pgfpathlineto{\pgfqpoint{3.555283in}{1.981431in}}%
\pgfpathlineto{\pgfqpoint{3.547436in}{1.977085in}}%
\pgfpathlineto{\pgfqpoint{3.539581in}{1.972900in}}%
\pgfpathlineto{\pgfqpoint{3.531717in}{1.968881in}}%
\pgfpathlineto{\pgfqpoint{3.523844in}{1.965033in}}%
\pgfpathlineto{\pgfqpoint{3.510740in}{1.970390in}}%
\pgfpathlineto{\pgfqpoint{3.497640in}{1.975775in}}%
\pgfpathlineto{\pgfqpoint{3.484545in}{1.981190in}}%
\pgfpathlineto{\pgfqpoint{3.471455in}{1.986635in}}%
\pgfpathlineto{\pgfqpoint{3.479348in}{1.990257in}}%
\pgfpathlineto{\pgfqpoint{3.487231in}{1.994055in}}%
\pgfpathlineto{\pgfqpoint{3.495106in}{1.998021in}}%
\pgfpathlineto{\pgfqpoint{3.502971in}{2.002152in}}%
\pgfpathclose%
\pgfusepath{fill}%
\end{pgfscope}%
\begin{pgfscope}%
\pgfpathrectangle{\pgfqpoint{1.254980in}{0.150000in}}{\pgfqpoint{5.490039in}{5.490039in}}%
\pgfusepath{clip}%
\pgfsetbuttcap%
\pgfsetroundjoin%
\definecolor{currentfill}{rgb}{0.281446,0.084320,0.407414}%
\pgfsetfillcolor{currentfill}%
\pgfsetfillopacity{0.700000}%
\pgfsetlinewidth{0.000000pt}%
\definecolor{currentstroke}{rgb}{0.000000,0.000000,0.000000}%
\pgfsetstrokecolor{currentstroke}%
\pgfsetdash{}{0pt}%
\pgfpathmoveto{\pgfqpoint{3.178493in}{2.094076in}}%
\pgfpathlineto{\pgfqpoint{3.191518in}{2.087767in}}%
\pgfpathlineto{\pgfqpoint{3.204547in}{2.081491in}}%
\pgfpathlineto{\pgfqpoint{3.217580in}{2.075249in}}%
\pgfpathlineto{\pgfqpoint{3.230617in}{2.069039in}}%
\pgfpathlineto{\pgfqpoint{3.222586in}{2.067348in}}%
\pgfpathlineto{\pgfqpoint{3.214542in}{2.065882in}}%
\pgfpathlineto{\pgfqpoint{3.206487in}{2.064646in}}%
\pgfpathlineto{\pgfqpoint{3.198419in}{2.063648in}}%
\pgfpathlineto{\pgfqpoint{3.185358in}{2.070104in}}%
\pgfpathlineto{\pgfqpoint{3.172300in}{2.076594in}}%
\pgfpathlineto{\pgfqpoint{3.159247in}{2.083116in}}%
\pgfpathlineto{\pgfqpoint{3.146197in}{2.089672in}}%
\pgfpathlineto{\pgfqpoint{3.154289in}{2.090418in}}%
\pgfpathlineto{\pgfqpoint{3.162369in}{2.091405in}}%
\pgfpathlineto{\pgfqpoint{3.170437in}{2.092627in}}%
\pgfpathlineto{\pgfqpoint{3.178493in}{2.094076in}}%
\pgfpathclose%
\pgfusepath{fill}%
\end{pgfscope}%
\begin{pgfscope}%
\pgfpathrectangle{\pgfqpoint{1.254980in}{0.150000in}}{\pgfqpoint{5.490039in}{5.490039in}}%
\pgfusepath{clip}%
\pgfsetbuttcap%
\pgfsetroundjoin%
\definecolor{currentfill}{rgb}{0.267004,0.004874,0.329415}%
\pgfsetfillcolor{currentfill}%
\pgfsetfillopacity{0.700000}%
\pgfsetlinewidth{0.000000pt}%
\definecolor{currentstroke}{rgb}{0.000000,0.000000,0.000000}%
\pgfsetstrokecolor{currentstroke}%
\pgfsetdash{}{0pt}%
\pgfpathmoveto{\pgfqpoint{3.910624in}{1.959125in}}%
\pgfpathlineto{\pgfqpoint{3.923774in}{1.955159in}}%
\pgfpathlineto{\pgfqpoint{3.936929in}{1.951221in}}%
\pgfpathlineto{\pgfqpoint{3.950089in}{1.947308in}}%
\pgfpathlineto{\pgfqpoint{3.963256in}{1.943423in}}%
\pgfpathlineto{\pgfqpoint{3.955592in}{1.936488in}}%
\pgfpathlineto{\pgfqpoint{3.947923in}{1.929635in}}%
\pgfpathlineto{\pgfqpoint{3.940246in}{1.922869in}}%
\pgfpathlineto{\pgfqpoint{3.932564in}{1.916194in}}%
\pgfpathlineto{\pgfqpoint{3.919383in}{1.920260in}}%
\pgfpathlineto{\pgfqpoint{3.906208in}{1.924354in}}%
\pgfpathlineto{\pgfqpoint{3.893039in}{1.928474in}}%
\pgfpathlineto{\pgfqpoint{3.879875in}{1.932621in}}%
\pgfpathlineto{\pgfqpoint{3.887572in}{1.939110in}}%
\pgfpathlineto{\pgfqpoint{3.895263in}{1.945693in}}%
\pgfpathlineto{\pgfqpoint{3.902947in}{1.952367in}}%
\pgfpathlineto{\pgfqpoint{3.910624in}{1.959125in}}%
\pgfpathclose%
\pgfusepath{fill}%
\end{pgfscope}%
\begin{pgfscope}%
\pgfpathrectangle{\pgfqpoint{1.254980in}{0.150000in}}{\pgfqpoint{5.490039in}{5.490039in}}%
\pgfusepath{clip}%
\pgfsetbuttcap%
\pgfsetroundjoin%
\definecolor{currentfill}{rgb}{0.273006,0.204520,0.501721}%
\pgfsetfillcolor{currentfill}%
\pgfsetfillopacity{0.700000}%
\pgfsetlinewidth{0.000000pt}%
\definecolor{currentstroke}{rgb}{0.000000,0.000000,0.000000}%
\pgfsetstrokecolor{currentstroke}%
\pgfsetdash{}{0pt}%
\pgfpathmoveto{\pgfqpoint{5.776621in}{2.326827in}}%
\pgfpathlineto{\pgfqpoint{5.790301in}{2.326380in}}%
\pgfpathlineto{\pgfqpoint{5.803990in}{2.325957in}}%
\pgfpathlineto{\pgfqpoint{5.817688in}{2.325558in}}%
\pgfpathlineto{\pgfqpoint{5.831394in}{2.325182in}}%
\pgfpathlineto{\pgfqpoint{5.824492in}{2.319559in}}%
\pgfpathlineto{\pgfqpoint{5.817581in}{2.313852in}}%
\pgfpathlineto{\pgfqpoint{5.810662in}{2.308060in}}%
\pgfpathlineto{\pgfqpoint{5.803734in}{2.302181in}}%
\pgfpathlineto{\pgfqpoint{5.790011in}{2.302503in}}%
\pgfpathlineto{\pgfqpoint{5.776297in}{2.302849in}}%
\pgfpathlineto{\pgfqpoint{5.762592in}{2.303219in}}%
\pgfpathlineto{\pgfqpoint{5.748896in}{2.303613in}}%
\pgfpathlineto{\pgfqpoint{5.755840in}{2.309540in}}%
\pgfpathlineto{\pgfqpoint{5.762775in}{2.315384in}}%
\pgfpathlineto{\pgfqpoint{5.769702in}{2.321145in}}%
\pgfpathlineto{\pgfqpoint{5.776621in}{2.326827in}}%
\pgfpathclose%
\pgfusepath{fill}%
\end{pgfscope}%
\begin{pgfscope}%
\pgfpathrectangle{\pgfqpoint{1.254980in}{0.150000in}}{\pgfqpoint{5.490039in}{5.490039in}}%
\pgfusepath{clip}%
\pgfsetbuttcap%
\pgfsetroundjoin%
\definecolor{currentfill}{rgb}{0.283187,0.125848,0.444960}%
\pgfsetfillcolor{currentfill}%
\pgfsetfillopacity{0.700000}%
\pgfsetlinewidth{0.000000pt}%
\definecolor{currentstroke}{rgb}{0.000000,0.000000,0.000000}%
\pgfsetstrokecolor{currentstroke}%
\pgfsetdash{}{0pt}%
\pgfpathmoveto{\pgfqpoint{2.989888in}{2.171045in}}%
\pgfpathlineto{\pgfqpoint{3.002894in}{2.164068in}}%
\pgfpathlineto{\pgfqpoint{3.015904in}{2.157128in}}%
\pgfpathlineto{\pgfqpoint{3.028917in}{2.150224in}}%
\pgfpathlineto{\pgfqpoint{3.041933in}{2.143355in}}%
\pgfpathlineto{\pgfqpoint{3.033775in}{2.143368in}}%
\pgfpathlineto{\pgfqpoint{3.025604in}{2.143642in}}%
\pgfpathlineto{\pgfqpoint{3.017418in}{2.144184in}}%
\pgfpathlineto{\pgfqpoint{3.009217in}{2.145000in}}%
\pgfpathlineto{\pgfqpoint{2.996173in}{2.152130in}}%
\pgfpathlineto{\pgfqpoint{2.983132in}{2.159295in}}%
\pgfpathlineto{\pgfqpoint{2.970095in}{2.166497in}}%
\pgfpathlineto{\pgfqpoint{2.957061in}{2.173736in}}%
\pgfpathlineto{\pgfqpoint{2.965290in}{2.172653in}}%
\pgfpathlineto{\pgfqpoint{2.973504in}{2.171848in}}%
\pgfpathlineto{\pgfqpoint{2.981703in}{2.171314in}}%
\pgfpathlineto{\pgfqpoint{2.989888in}{2.171045in}}%
\pgfpathclose%
\pgfusepath{fill}%
\end{pgfscope}%
\begin{pgfscope}%
\pgfpathrectangle{\pgfqpoint{1.254980in}{0.150000in}}{\pgfqpoint{5.490039in}{5.490039in}}%
\pgfusepath{clip}%
\pgfsetbuttcap%
\pgfsetroundjoin%
\definecolor{currentfill}{rgb}{0.253935,0.265254,0.529983}%
\pgfsetfillcolor{currentfill}%
\pgfsetfillopacity{0.700000}%
\pgfsetlinewidth{0.000000pt}%
\definecolor{currentstroke}{rgb}{0.000000,0.000000,0.000000}%
\pgfsetstrokecolor{currentstroke}%
\pgfsetdash{}{0pt}%
\pgfpathmoveto{\pgfqpoint{2.507359in}{2.442778in}}%
\pgfpathlineto{\pgfqpoint{2.520345in}{2.433907in}}%
\pgfpathlineto{\pgfqpoint{2.533332in}{2.425084in}}%
\pgfpathlineto{\pgfqpoint{2.546321in}{2.416310in}}%
\pgfpathlineto{\pgfqpoint{2.559312in}{2.407584in}}%
\pgfpathlineto{\pgfqpoint{2.550772in}{2.412166in}}%
\pgfpathlineto{\pgfqpoint{2.542210in}{2.417096in}}%
\pgfpathlineto{\pgfqpoint{2.533628in}{2.422382in}}%
\pgfpathlineto{\pgfqpoint{2.525024in}{2.428030in}}%
\pgfpathlineto{\pgfqpoint{2.511997in}{2.437049in}}%
\pgfpathlineto{\pgfqpoint{2.498971in}{2.446117in}}%
\pgfpathlineto{\pgfqpoint{2.485947in}{2.455232in}}%
\pgfpathlineto{\pgfqpoint{2.472925in}{2.464397in}}%
\pgfpathlineto{\pgfqpoint{2.481566in}{2.458450in}}%
\pgfpathlineto{\pgfqpoint{2.490185in}{2.452869in}}%
\pgfpathlineto{\pgfqpoint{2.498783in}{2.447648in}}%
\pgfpathlineto{\pgfqpoint{2.507359in}{2.442778in}}%
\pgfpathclose%
\pgfusepath{fill}%
\end{pgfscope}%
\begin{pgfscope}%
\pgfpathrectangle{\pgfqpoint{1.254980in}{0.150000in}}{\pgfqpoint{5.490039in}{5.490039in}}%
\pgfusepath{clip}%
\pgfsetbuttcap%
\pgfsetroundjoin%
\definecolor{currentfill}{rgb}{0.282884,0.135920,0.453427}%
\pgfsetfillcolor{currentfill}%
\pgfsetfillopacity{0.700000}%
\pgfsetlinewidth{0.000000pt}%
\definecolor{currentstroke}{rgb}{0.000000,0.000000,0.000000}%
\pgfsetstrokecolor{currentstroke}%
\pgfsetdash{}{0pt}%
\pgfpathmoveto{\pgfqpoint{5.172359in}{2.188291in}}%
\pgfpathlineto{\pgfqpoint{5.185852in}{2.187202in}}%
\pgfpathlineto{\pgfqpoint{5.199353in}{2.186137in}}%
\pgfpathlineto{\pgfqpoint{5.212862in}{2.185096in}}%
\pgfpathlineto{\pgfqpoint{5.226379in}{2.184079in}}%
\pgfpathlineto{\pgfqpoint{5.219180in}{2.176258in}}%
\pgfpathlineto{\pgfqpoint{5.211974in}{2.168357in}}%
\pgfpathlineto{\pgfqpoint{5.204761in}{2.160375in}}%
\pgfpathlineto{\pgfqpoint{5.197541in}{2.152314in}}%
\pgfpathlineto{\pgfqpoint{5.184012in}{2.153358in}}%
\pgfpathlineto{\pgfqpoint{5.170492in}{2.154426in}}%
\pgfpathlineto{\pgfqpoint{5.156980in}{2.155517in}}%
\pgfpathlineto{\pgfqpoint{5.143476in}{2.156633in}}%
\pgfpathlineto{\pgfqpoint{5.150707in}{2.164662in}}%
\pgfpathlineto{\pgfqpoint{5.157931in}{2.172615in}}%
\pgfpathlineto{\pgfqpoint{5.165149in}{2.180491in}}%
\pgfpathlineto{\pgfqpoint{5.172359in}{2.188291in}}%
\pgfpathclose%
\pgfusepath{fill}%
\end{pgfscope}%
\begin{pgfscope}%
\pgfpathrectangle{\pgfqpoint{1.254980in}{0.150000in}}{\pgfqpoint{5.490039in}{5.490039in}}%
\pgfusepath{clip}%
\pgfsetbuttcap%
\pgfsetroundjoin%
\definecolor{currentfill}{rgb}{0.279574,0.170599,0.479997}%
\pgfsetfillcolor{currentfill}%
\pgfsetfillopacity{0.700000}%
\pgfsetlinewidth{0.000000pt}%
\definecolor{currentstroke}{rgb}{0.000000,0.000000,0.000000}%
\pgfsetstrokecolor{currentstroke}%
\pgfsetdash{}{0pt}%
\pgfpathmoveto{\pgfqpoint{5.474618in}{2.261423in}}%
\pgfpathlineto{\pgfqpoint{5.488205in}{2.260716in}}%
\pgfpathlineto{\pgfqpoint{5.501801in}{2.260034in}}%
\pgfpathlineto{\pgfqpoint{5.515406in}{2.259375in}}%
\pgfpathlineto{\pgfqpoint{5.529019in}{2.258740in}}%
\pgfpathlineto{\pgfqpoint{5.521958in}{2.251931in}}%
\pgfpathlineto{\pgfqpoint{5.514890in}{2.245034in}}%
\pgfpathlineto{\pgfqpoint{5.507815in}{2.238048in}}%
\pgfpathlineto{\pgfqpoint{5.500731in}{2.230972in}}%
\pgfpathlineto{\pgfqpoint{5.487105in}{2.231594in}}%
\pgfpathlineto{\pgfqpoint{5.473487in}{2.232239in}}%
\pgfpathlineto{\pgfqpoint{5.459878in}{2.232909in}}%
\pgfpathlineto{\pgfqpoint{5.446277in}{2.233602in}}%
\pgfpathlineto{\pgfqpoint{5.453374in}{2.240686in}}%
\pgfpathlineto{\pgfqpoint{5.460463in}{2.247684in}}%
\pgfpathlineto{\pgfqpoint{5.467544in}{2.254596in}}%
\pgfpathlineto{\pgfqpoint{5.474618in}{2.261423in}}%
\pgfpathclose%
\pgfusepath{fill}%
\end{pgfscope}%
\begin{pgfscope}%
\pgfpathrectangle{\pgfqpoint{1.254980in}{0.150000in}}{\pgfqpoint{5.490039in}{5.490039in}}%
\pgfusepath{clip}%
\pgfsetbuttcap%
\pgfsetroundjoin%
\definecolor{currentfill}{rgb}{0.269944,0.014625,0.341379}%
\pgfsetfillcolor{currentfill}%
\pgfsetfillopacity{0.700000}%
\pgfsetlinewidth{0.000000pt}%
\definecolor{currentstroke}{rgb}{0.000000,0.000000,0.000000}%
\pgfsetstrokecolor{currentstroke}%
\pgfsetdash{}{0pt}%
\pgfpathmoveto{\pgfqpoint{4.265649in}{1.980447in}}%
\pgfpathlineto{\pgfqpoint{4.278884in}{1.977471in}}%
\pgfpathlineto{\pgfqpoint{4.292126in}{1.974520in}}%
\pgfpathlineto{\pgfqpoint{4.305374in}{1.971594in}}%
\pgfpathlineto{\pgfqpoint{4.318628in}{1.968694in}}%
\pgfpathlineto{\pgfqpoint{4.311094in}{1.960392in}}%
\pgfpathlineto{\pgfqpoint{4.303553in}{1.952110in}}%
\pgfpathlineto{\pgfqpoint{4.296008in}{1.943852in}}%
\pgfpathlineto{\pgfqpoint{4.288457in}{1.935621in}}%
\pgfpathlineto{\pgfqpoint{4.275191in}{1.938665in}}%
\pgfpathlineto{\pgfqpoint{4.261932in}{1.941734in}}%
\pgfpathlineto{\pgfqpoint{4.248679in}{1.944828in}}%
\pgfpathlineto{\pgfqpoint{4.235433in}{1.947947in}}%
\pgfpathlineto{\pgfqpoint{4.242995in}{1.956030in}}%
\pgfpathlineto{\pgfqpoint{4.250552in}{1.964144in}}%
\pgfpathlineto{\pgfqpoint{4.258103in}{1.972284in}}%
\pgfpathlineto{\pgfqpoint{4.265649in}{1.980447in}}%
\pgfpathclose%
\pgfusepath{fill}%
\end{pgfscope}%
\begin{pgfscope}%
\pgfpathrectangle{\pgfqpoint{1.254980in}{0.150000in}}{\pgfqpoint{5.490039in}{5.490039in}}%
\pgfusepath{clip}%
\pgfsetbuttcap%
\pgfsetroundjoin%
\definecolor{currentfill}{rgb}{0.277018,0.050344,0.375715}%
\pgfsetfillcolor{currentfill}%
\pgfsetfillopacity{0.700000}%
\pgfsetlinewidth{0.000000pt}%
\definecolor{currentstroke}{rgb}{0.000000,0.000000,0.000000}%
\pgfsetstrokecolor{currentstroke}%
\pgfsetdash{}{0pt}%
\pgfpathmoveto{\pgfqpoint{3.366901in}{2.031281in}}%
\pgfpathlineto{\pgfqpoint{3.379954in}{2.025593in}}%
\pgfpathlineto{\pgfqpoint{3.393012in}{2.019936in}}%
\pgfpathlineto{\pgfqpoint{3.406074in}{2.014310in}}%
\pgfpathlineto{\pgfqpoint{3.419141in}{2.008714in}}%
\pgfpathlineto{\pgfqpoint{3.411219in}{2.005501in}}%
\pgfpathlineto{\pgfqpoint{3.403286in}{2.002478in}}%
\pgfpathlineto{\pgfqpoint{3.395344in}{1.999652in}}%
\pgfpathlineto{\pgfqpoint{3.387391in}{1.997027in}}%
\pgfpathlineto{\pgfqpoint{3.374303in}{2.002856in}}%
\pgfpathlineto{\pgfqpoint{3.361219in}{2.008716in}}%
\pgfpathlineto{\pgfqpoint{3.348140in}{2.014606in}}%
\pgfpathlineto{\pgfqpoint{3.335065in}{2.020528in}}%
\pgfpathlineto{\pgfqpoint{3.343039in}{2.022914in}}%
\pgfpathlineto{\pgfqpoint{3.351003in}{2.025506in}}%
\pgfpathlineto{\pgfqpoint{3.358957in}{2.028297in}}%
\pgfpathlineto{\pgfqpoint{3.366901in}{2.031281in}}%
\pgfpathclose%
\pgfusepath{fill}%
\end{pgfscope}%
\begin{pgfscope}%
\pgfpathrectangle{\pgfqpoint{1.254980in}{0.150000in}}{\pgfqpoint{5.490039in}{5.490039in}}%
\pgfusepath{clip}%
\pgfsetbuttcap%
\pgfsetroundjoin%
\definecolor{currentfill}{rgb}{0.280894,0.078907,0.402329}%
\pgfsetfillcolor{currentfill}%
\pgfsetfillopacity{0.700000}%
\pgfsetlinewidth{0.000000pt}%
\definecolor{currentstroke}{rgb}{0.000000,0.000000,0.000000}%
\pgfsetstrokecolor{currentstroke}%
\pgfsetdash{}{0pt}%
\pgfpathmoveto{\pgfqpoint{4.787136in}{2.085080in}}%
\pgfpathlineto{\pgfqpoint{4.800516in}{2.083323in}}%
\pgfpathlineto{\pgfqpoint{4.813903in}{2.081590in}}%
\pgfpathlineto{\pgfqpoint{4.827298in}{2.079881in}}%
\pgfpathlineto{\pgfqpoint{4.840700in}{2.078197in}}%
\pgfpathlineto{\pgfqpoint{4.833346in}{2.069534in}}%
\pgfpathlineto{\pgfqpoint{4.825986in}{2.060821in}}%
\pgfpathlineto{\pgfqpoint{4.818619in}{2.052059in}}%
\pgfpathlineto{\pgfqpoint{4.811247in}{2.043249in}}%
\pgfpathlineto{\pgfqpoint{4.797835in}{2.045013in}}%
\pgfpathlineto{\pgfqpoint{4.784430in}{2.046801in}}%
\pgfpathlineto{\pgfqpoint{4.771033in}{2.048613in}}%
\pgfpathlineto{\pgfqpoint{4.757643in}{2.050449in}}%
\pgfpathlineto{\pgfqpoint{4.765025in}{2.059175in}}%
\pgfpathlineto{\pgfqpoint{4.772401in}{2.067856in}}%
\pgfpathlineto{\pgfqpoint{4.779771in}{2.076492in}}%
\pgfpathlineto{\pgfqpoint{4.787136in}{2.085080in}}%
\pgfpathclose%
\pgfusepath{fill}%
\end{pgfscope}%
\begin{pgfscope}%
\pgfpathrectangle{\pgfqpoint{1.254980in}{0.150000in}}{\pgfqpoint{5.490039in}{5.490039in}}%
\pgfusepath{clip}%
\pgfsetbuttcap%
\pgfsetroundjoin%
\definecolor{currentfill}{rgb}{0.267004,0.004874,0.329415}%
\pgfsetfillcolor{currentfill}%
\pgfsetfillopacity{0.700000}%
\pgfsetlinewidth{0.000000pt}%
\definecolor{currentstroke}{rgb}{0.000000,0.000000,0.000000}%
\pgfsetstrokecolor{currentstroke}%
\pgfsetdash{}{0pt}%
\pgfpathmoveto{\pgfqpoint{4.046523in}{1.957297in}}%
\pgfpathlineto{\pgfqpoint{4.059706in}{1.953713in}}%
\pgfpathlineto{\pgfqpoint{4.072896in}{1.950154in}}%
\pgfpathlineto{\pgfqpoint{4.086091in}{1.946621in}}%
\pgfpathlineto{\pgfqpoint{4.099293in}{1.943115in}}%
\pgfpathlineto{\pgfqpoint{4.091679in}{1.935565in}}%
\pgfpathlineto{\pgfqpoint{4.084060in}{1.928075in}}%
\pgfpathlineto{\pgfqpoint{4.076434in}{1.920648in}}%
\pgfpathlineto{\pgfqpoint{4.068803in}{1.913288in}}%
\pgfpathlineto{\pgfqpoint{4.055589in}{1.916963in}}%
\pgfpathlineto{\pgfqpoint{4.042381in}{1.920665in}}%
\pgfpathlineto{\pgfqpoint{4.029178in}{1.924392in}}%
\pgfpathlineto{\pgfqpoint{4.015982in}{1.928146in}}%
\pgfpathlineto{\pgfqpoint{4.023626in}{1.935332in}}%
\pgfpathlineto{\pgfqpoint{4.031264in}{1.942589in}}%
\pgfpathlineto{\pgfqpoint{4.038897in}{1.949912in}}%
\pgfpathlineto{\pgfqpoint{4.046523in}{1.957297in}}%
\pgfpathclose%
\pgfusepath{fill}%
\end{pgfscope}%
\begin{pgfscope}%
\pgfpathrectangle{\pgfqpoint{1.254980in}{0.150000in}}{\pgfqpoint{5.490039in}{5.490039in}}%
\pgfusepath{clip}%
\pgfsetbuttcap%
\pgfsetroundjoin%
\definecolor{currentfill}{rgb}{0.274952,0.037752,0.364543}%
\pgfsetfillcolor{currentfill}%
\pgfsetfillopacity{0.700000}%
\pgfsetlinewidth{0.000000pt}%
\definecolor{currentstroke}{rgb}{0.000000,0.000000,0.000000}%
\pgfsetstrokecolor{currentstroke}%
\pgfsetdash{}{0pt}%
\pgfpathmoveto{\pgfqpoint{4.484825in}{2.015156in}}%
\pgfpathlineto{\pgfqpoint{4.498119in}{2.012729in}}%
\pgfpathlineto{\pgfqpoint{4.511421in}{2.010328in}}%
\pgfpathlineto{\pgfqpoint{4.524729in}{2.007951in}}%
\pgfpathlineto{\pgfqpoint{4.538045in}{2.005598in}}%
\pgfpathlineto{\pgfqpoint{4.530584in}{1.996905in}}%
\pgfpathlineto{\pgfqpoint{4.523117in}{1.988199in}}%
\pgfpathlineto{\pgfqpoint{4.515646in}{1.979482in}}%
\pgfpathlineto{\pgfqpoint{4.508169in}{1.970758in}}%
\pgfpathlineto{\pgfqpoint{4.494843in}{1.973228in}}%
\pgfpathlineto{\pgfqpoint{4.481525in}{1.975723in}}%
\pgfpathlineto{\pgfqpoint{4.468213in}{1.978243in}}%
\pgfpathlineto{\pgfqpoint{4.454908in}{1.980787in}}%
\pgfpathlineto{\pgfqpoint{4.462395in}{1.989388in}}%
\pgfpathlineto{\pgfqpoint{4.469877in}{1.997986in}}%
\pgfpathlineto{\pgfqpoint{4.477353in}{2.006576in}}%
\pgfpathlineto{\pgfqpoint{4.484825in}{2.015156in}}%
\pgfpathclose%
\pgfusepath{fill}%
\end{pgfscope}%
\begin{pgfscope}%
\pgfpathrectangle{\pgfqpoint{1.254980in}{0.150000in}}{\pgfqpoint{5.490039in}{5.490039in}}%
\pgfusepath{clip}%
\pgfsetbuttcap%
\pgfsetroundjoin%
\definecolor{currentfill}{rgb}{0.278012,0.180367,0.486697}%
\pgfsetfillcolor{currentfill}%
\pgfsetfillopacity{0.700000}%
\pgfsetlinewidth{0.000000pt}%
\definecolor{currentstroke}{rgb}{0.000000,0.000000,0.000000}%
\pgfsetstrokecolor{currentstroke}%
\pgfsetdash{}{0pt}%
\pgfpathmoveto{\pgfqpoint{2.800894in}{2.263559in}}%
\pgfpathlineto{\pgfqpoint{2.813891in}{2.255858in}}%
\pgfpathlineto{\pgfqpoint{2.826892in}{2.248197in}}%
\pgfpathlineto{\pgfqpoint{2.839895in}{2.240576in}}%
\pgfpathlineto{\pgfqpoint{2.852902in}{2.232995in}}%
\pgfpathlineto{\pgfqpoint{2.844598in}{2.234904in}}%
\pgfpathlineto{\pgfqpoint{2.836278in}{2.237113in}}%
\pgfpathlineto{\pgfqpoint{2.827941in}{2.239628in}}%
\pgfpathlineto{\pgfqpoint{2.819587in}{2.242458in}}%
\pgfpathlineto{\pgfqpoint{2.806550in}{2.250315in}}%
\pgfpathlineto{\pgfqpoint{2.793515in}{2.258212in}}%
\pgfpathlineto{\pgfqpoint{2.780483in}{2.266149in}}%
\pgfpathlineto{\pgfqpoint{2.767454in}{2.274127in}}%
\pgfpathlineto{\pgfqpoint{2.775840in}{2.271015in}}%
\pgfpathlineto{\pgfqpoint{2.784209in}{2.268222in}}%
\pgfpathlineto{\pgfqpoint{2.792560in}{2.265739in}}%
\pgfpathlineto{\pgfqpoint{2.800894in}{2.263559in}}%
\pgfpathclose%
\pgfusepath{fill}%
\end{pgfscope}%
\begin{pgfscope}%
\pgfpathrectangle{\pgfqpoint{1.254980in}{0.150000in}}{\pgfqpoint{5.490039in}{5.490039in}}%
\pgfusepath{clip}%
\pgfsetbuttcap%
\pgfsetroundjoin%
\definecolor{currentfill}{rgb}{0.283229,0.120777,0.440584}%
\pgfsetfillcolor{currentfill}%
\pgfsetfillopacity{0.700000}%
\pgfsetlinewidth{0.000000pt}%
\definecolor{currentstroke}{rgb}{0.000000,0.000000,0.000000}%
\pgfsetstrokecolor{currentstroke}%
\pgfsetdash{}{0pt}%
\pgfpathmoveto{\pgfqpoint{5.089540in}{2.161334in}}%
\pgfpathlineto{\pgfqpoint{5.103012in}{2.160123in}}%
\pgfpathlineto{\pgfqpoint{5.116492in}{2.158936in}}%
\pgfpathlineto{\pgfqpoint{5.129980in}{2.157772in}}%
\pgfpathlineto{\pgfqpoint{5.143476in}{2.156633in}}%
\pgfpathlineto{\pgfqpoint{5.136238in}{2.148527in}}%
\pgfpathlineto{\pgfqpoint{5.128993in}{2.140346in}}%
\pgfpathlineto{\pgfqpoint{5.121742in}{2.132090in}}%
\pgfpathlineto{\pgfqpoint{5.114485in}{2.123759in}}%
\pgfpathlineto{\pgfqpoint{5.100978in}{2.124938in}}%
\pgfpathlineto{\pgfqpoint{5.087479in}{2.126142in}}%
\pgfpathlineto{\pgfqpoint{5.073988in}{2.127369in}}%
\pgfpathlineto{\pgfqpoint{5.060505in}{2.128621in}}%
\pgfpathlineto{\pgfqpoint{5.067774in}{2.136907in}}%
\pgfpathlineto{\pgfqpoint{5.075035in}{2.145121in}}%
\pgfpathlineto{\pgfqpoint{5.082291in}{2.153264in}}%
\pgfpathlineto{\pgfqpoint{5.089540in}{2.161334in}}%
\pgfpathclose%
\pgfusepath{fill}%
\end{pgfscope}%
\begin{pgfscope}%
\pgfpathrectangle{\pgfqpoint{1.254980in}{0.150000in}}{\pgfqpoint{5.490039in}{5.490039in}}%
\pgfusepath{clip}%
\pgfsetbuttcap%
\pgfsetroundjoin%
\definecolor{currentfill}{rgb}{0.274128,0.199721,0.498911}%
\pgfsetfillcolor{currentfill}%
\pgfsetfillopacity{0.700000}%
\pgfsetlinewidth{0.000000pt}%
\definecolor{currentstroke}{rgb}{0.000000,0.000000,0.000000}%
\pgfsetstrokecolor{currentstroke}%
\pgfsetdash{}{0pt}%
\pgfpathmoveto{\pgfqpoint{5.694199in}{2.305422in}}%
\pgfpathlineto{\pgfqpoint{5.707860in}{2.304934in}}%
\pgfpathlineto{\pgfqpoint{5.721530in}{2.304470in}}%
\pgfpathlineto{\pgfqpoint{5.735209in}{2.304030in}}%
\pgfpathlineto{\pgfqpoint{5.748896in}{2.303613in}}%
\pgfpathlineto{\pgfqpoint{5.741944in}{2.297599in}}%
\pgfpathlineto{\pgfqpoint{5.734984in}{2.291498in}}%
\pgfpathlineto{\pgfqpoint{5.728016in}{2.285308in}}%
\pgfpathlineto{\pgfqpoint{5.721039in}{2.279028in}}%
\pgfpathlineto{\pgfqpoint{5.707336in}{2.279405in}}%
\pgfpathlineto{\pgfqpoint{5.693642in}{2.279806in}}%
\pgfpathlineto{\pgfqpoint{5.679957in}{2.280230in}}%
\pgfpathlineto{\pgfqpoint{5.666281in}{2.280678in}}%
\pgfpathlineto{\pgfqpoint{5.673272in}{2.286994in}}%
\pgfpathlineto{\pgfqpoint{5.680256in}{2.293222in}}%
\pgfpathlineto{\pgfqpoint{5.687231in}{2.299364in}}%
\pgfpathlineto{\pgfqpoint{5.694199in}{2.305422in}}%
\pgfpathclose%
\pgfusepath{fill}%
\end{pgfscope}%
\begin{pgfscope}%
\pgfpathrectangle{\pgfqpoint{1.254980in}{0.150000in}}{\pgfqpoint{5.490039in}{5.490039in}}%
\pgfusepath{clip}%
\pgfsetbuttcap%
\pgfsetroundjoin%
\definecolor{currentfill}{rgb}{0.280868,0.160771,0.472899}%
\pgfsetfillcolor{currentfill}%
\pgfsetfillopacity{0.700000}%
\pgfsetlinewidth{0.000000pt}%
\definecolor{currentstroke}{rgb}{0.000000,0.000000,0.000000}%
\pgfsetstrokecolor{currentstroke}%
\pgfsetdash{}{0pt}%
\pgfpathmoveto{\pgfqpoint{5.391958in}{2.236613in}}%
\pgfpathlineto{\pgfqpoint{5.405525in}{2.235824in}}%
\pgfpathlineto{\pgfqpoint{5.419101in}{2.235060in}}%
\pgfpathlineto{\pgfqpoint{5.432685in}{2.234319in}}%
\pgfpathlineto{\pgfqpoint{5.446277in}{2.233602in}}%
\pgfpathlineto{\pgfqpoint{5.439172in}{2.226430in}}%
\pgfpathlineto{\pgfqpoint{5.432060in}{2.219171in}}%
\pgfpathlineto{\pgfqpoint{5.424941in}{2.211822in}}%
\pgfpathlineto{\pgfqpoint{5.417814in}{2.204385in}}%
\pgfpathlineto{\pgfqpoint{5.404209in}{2.205102in}}%
\pgfpathlineto{\pgfqpoint{5.390612in}{2.205843in}}%
\pgfpathlineto{\pgfqpoint{5.377024in}{2.206608in}}%
\pgfpathlineto{\pgfqpoint{5.363445in}{2.207397in}}%
\pgfpathlineto{\pgfqpoint{5.370584in}{2.214829in}}%
\pgfpathlineto{\pgfqpoint{5.377716in}{2.222175in}}%
\pgfpathlineto{\pgfqpoint{5.384841in}{2.229436in}}%
\pgfpathlineto{\pgfqpoint{5.391958in}{2.236613in}}%
\pgfpathclose%
\pgfusepath{fill}%
\end{pgfscope}%
\begin{pgfscope}%
\pgfpathrectangle{\pgfqpoint{1.254980in}{0.150000in}}{\pgfqpoint{5.490039in}{5.490039in}}%
\pgfusepath{clip}%
\pgfsetbuttcap%
\pgfsetroundjoin%
\definecolor{currentfill}{rgb}{0.258965,0.251537,0.524736}%
\pgfsetfillcolor{currentfill}%
\pgfsetfillopacity{0.700000}%
\pgfsetlinewidth{0.000000pt}%
\definecolor{currentstroke}{rgb}{0.000000,0.000000,0.000000}%
\pgfsetstrokecolor{currentstroke}%
\pgfsetdash{}{0pt}%
\pgfpathmoveto{\pgfqpoint{2.559312in}{2.407584in}}%
\pgfpathlineto{\pgfqpoint{2.572305in}{2.398906in}}%
\pgfpathlineto{\pgfqpoint{2.585300in}{2.390275in}}%
\pgfpathlineto{\pgfqpoint{2.598297in}{2.381690in}}%
\pgfpathlineto{\pgfqpoint{2.611295in}{2.373151in}}%
\pgfpathlineto{\pgfqpoint{2.602790in}{2.377446in}}%
\pgfpathlineto{\pgfqpoint{2.594264in}{2.382085in}}%
\pgfpathlineto{\pgfqpoint{2.585718in}{2.387076in}}%
\pgfpathlineto{\pgfqpoint{2.577150in}{2.392426in}}%
\pgfpathlineto{\pgfqpoint{2.564116in}{2.401257in}}%
\pgfpathlineto{\pgfqpoint{2.551084in}{2.410135in}}%
\pgfpathlineto{\pgfqpoint{2.538053in}{2.419059in}}%
\pgfpathlineto{\pgfqpoint{2.525024in}{2.428030in}}%
\pgfpathlineto{\pgfqpoint{2.533628in}{2.422382in}}%
\pgfpathlineto{\pgfqpoint{2.542210in}{2.417096in}}%
\pgfpathlineto{\pgfqpoint{2.550772in}{2.412166in}}%
\pgfpathlineto{\pgfqpoint{2.559312in}{2.407584in}}%
\pgfpathclose%
\pgfusepath{fill}%
\end{pgfscope}%
\begin{pgfscope}%
\pgfpathrectangle{\pgfqpoint{1.254980in}{0.150000in}}{\pgfqpoint{5.490039in}{5.490039in}}%
\pgfusepath{clip}%
\pgfsetbuttcap%
\pgfsetroundjoin%
\definecolor{currentfill}{rgb}{0.279566,0.067836,0.391917}%
\pgfsetfillcolor{currentfill}%
\pgfsetfillopacity{0.700000}%
\pgfsetlinewidth{0.000000pt}%
\definecolor{currentstroke}{rgb}{0.000000,0.000000,0.000000}%
\pgfsetstrokecolor{currentstroke}%
\pgfsetdash{}{0pt}%
\pgfpathmoveto{\pgfqpoint{4.704158in}{2.058039in}}%
\pgfpathlineto{\pgfqpoint{4.717518in}{2.056105in}}%
\pgfpathlineto{\pgfqpoint{4.730886in}{2.054195in}}%
\pgfpathlineto{\pgfqpoint{4.744261in}{2.052310in}}%
\pgfpathlineto{\pgfqpoint{4.757643in}{2.050449in}}%
\pgfpathlineto{\pgfqpoint{4.750256in}{2.041682in}}%
\pgfpathlineto{\pgfqpoint{4.742863in}{2.032873in}}%
\pgfpathlineto{\pgfqpoint{4.735464in}{2.024026in}}%
\pgfpathlineto{\pgfqpoint{4.728060in}{2.015143in}}%
\pgfpathlineto{\pgfqpoint{4.714668in}{2.017096in}}%
\pgfpathlineto{\pgfqpoint{4.701283in}{2.019074in}}%
\pgfpathlineto{\pgfqpoint{4.687905in}{2.021075in}}%
\pgfpathlineto{\pgfqpoint{4.674535in}{2.023101in}}%
\pgfpathlineto{\pgfqpoint{4.681949in}{2.031888in}}%
\pgfpathlineto{\pgfqpoint{4.689358in}{2.040641in}}%
\pgfpathlineto{\pgfqpoint{4.696761in}{2.049359in}}%
\pgfpathlineto{\pgfqpoint{4.704158in}{2.058039in}}%
\pgfpathclose%
\pgfusepath{fill}%
\end{pgfscope}%
\begin{pgfscope}%
\pgfpathrectangle{\pgfqpoint{1.254980in}{0.150000in}}{\pgfqpoint{5.490039in}{5.490039in}}%
\pgfusepath{clip}%
\pgfsetbuttcap%
\pgfsetroundjoin%
\definecolor{currentfill}{rgb}{0.268510,0.009605,0.335427}%
\pgfsetfillcolor{currentfill}%
\pgfsetfillopacity{0.700000}%
\pgfsetlinewidth{0.000000pt}%
\definecolor{currentstroke}{rgb}{0.000000,0.000000,0.000000}%
\pgfsetstrokecolor{currentstroke}%
\pgfsetdash{}{0pt}%
\pgfpathmoveto{\pgfqpoint{3.691306in}{1.961924in}}%
\pgfpathlineto{\pgfqpoint{3.704420in}{1.957253in}}%
\pgfpathlineto{\pgfqpoint{3.717538in}{1.952611in}}%
\pgfpathlineto{\pgfqpoint{3.730662in}{1.947996in}}%
\pgfpathlineto{\pgfqpoint{3.743792in}{1.943409in}}%
\pgfpathlineto{\pgfqpoint{3.736029in}{1.937867in}}%
\pgfpathlineto{\pgfqpoint{3.728260in}{1.932455in}}%
\pgfpathlineto{\pgfqpoint{3.720483in}{1.927177in}}%
\pgfpathlineto{\pgfqpoint{3.712698in}{1.922040in}}%
\pgfpathlineto{\pgfqpoint{3.699552in}{1.926834in}}%
\pgfpathlineto{\pgfqpoint{3.686411in}{1.931656in}}%
\pgfpathlineto{\pgfqpoint{3.673275in}{1.936506in}}%
\pgfpathlineto{\pgfqpoint{3.660144in}{1.941384in}}%
\pgfpathlineto{\pgfqpoint{3.667947in}{1.946309in}}%
\pgfpathlineto{\pgfqpoint{3.675741in}{1.951377in}}%
\pgfpathlineto{\pgfqpoint{3.683527in}{1.956584in}}%
\pgfpathlineto{\pgfqpoint{3.691306in}{1.961924in}}%
\pgfpathclose%
\pgfusepath{fill}%
\end{pgfscope}%
\begin{pgfscope}%
\pgfpathrectangle{\pgfqpoint{1.254980in}{0.150000in}}{\pgfqpoint{5.490039in}{5.490039in}}%
\pgfusepath{clip}%
\pgfsetbuttcap%
\pgfsetroundjoin%
\definecolor{currentfill}{rgb}{0.283229,0.120777,0.440584}%
\pgfsetfillcolor{currentfill}%
\pgfsetfillopacity{0.700000}%
\pgfsetlinewidth{0.000000pt}%
\definecolor{currentstroke}{rgb}{0.000000,0.000000,0.000000}%
\pgfsetstrokecolor{currentstroke}%
\pgfsetdash{}{0pt}%
\pgfpathmoveto{\pgfqpoint{3.041933in}{2.143355in}}%
\pgfpathlineto{\pgfqpoint{3.054954in}{2.136523in}}%
\pgfpathlineto{\pgfqpoint{3.067977in}{2.129726in}}%
\pgfpathlineto{\pgfqpoint{3.081005in}{2.122964in}}%
\pgfpathlineto{\pgfqpoint{3.094036in}{2.116237in}}%
\pgfpathlineto{\pgfqpoint{3.085904in}{2.115994in}}%
\pgfpathlineto{\pgfqpoint{3.077759in}{2.116008in}}%
\pgfpathlineto{\pgfqpoint{3.069601in}{2.116287in}}%
\pgfpathlineto{\pgfqpoint{3.061428in}{2.116837in}}%
\pgfpathlineto{\pgfqpoint{3.048370in}{2.123825in}}%
\pgfpathlineto{\pgfqpoint{3.035316in}{2.130848in}}%
\pgfpathlineto{\pgfqpoint{3.022265in}{2.137907in}}%
\pgfpathlineto{\pgfqpoint{3.009217in}{2.145000in}}%
\pgfpathlineto{\pgfqpoint{3.017418in}{2.144184in}}%
\pgfpathlineto{\pgfqpoint{3.025604in}{2.143642in}}%
\pgfpathlineto{\pgfqpoint{3.033775in}{2.143368in}}%
\pgfpathlineto{\pgfqpoint{3.041933in}{2.143355in}}%
\pgfpathclose%
\pgfusepath{fill}%
\end{pgfscope}%
\begin{pgfscope}%
\pgfpathrectangle{\pgfqpoint{1.254980in}{0.150000in}}{\pgfqpoint{5.490039in}{5.490039in}}%
\pgfusepath{clip}%
\pgfsetbuttcap%
\pgfsetroundjoin%
\definecolor{currentfill}{rgb}{0.268510,0.009605,0.335427}%
\pgfsetfillcolor{currentfill}%
\pgfsetfillopacity{0.700000}%
\pgfsetlinewidth{0.000000pt}%
\definecolor{currentstroke}{rgb}{0.000000,0.000000,0.000000}%
\pgfsetstrokecolor{currentstroke}%
\pgfsetdash{}{0pt}%
\pgfpathmoveto{\pgfqpoint{4.182511in}{1.960681in}}%
\pgfpathlineto{\pgfqpoint{4.195732in}{1.957459in}}%
\pgfpathlineto{\pgfqpoint{4.208959in}{1.954263in}}%
\pgfpathlineto{\pgfqpoint{4.222193in}{1.951092in}}%
\pgfpathlineto{\pgfqpoint{4.235433in}{1.947947in}}%
\pgfpathlineto{\pgfqpoint{4.227865in}{1.939898in}}%
\pgfpathlineto{\pgfqpoint{4.220292in}{1.931886in}}%
\pgfpathlineto{\pgfqpoint{4.212714in}{1.923915in}}%
\pgfpathlineto{\pgfqpoint{4.205130in}{1.915989in}}%
\pgfpathlineto{\pgfqpoint{4.191878in}{1.919290in}}%
\pgfpathlineto{\pgfqpoint{4.178633in}{1.922617in}}%
\pgfpathlineto{\pgfqpoint{4.165394in}{1.925969in}}%
\pgfpathlineto{\pgfqpoint{4.152161in}{1.929347in}}%
\pgfpathlineto{\pgfqpoint{4.159757in}{1.937112in}}%
\pgfpathlineto{\pgfqpoint{4.167347in}{1.944925in}}%
\pgfpathlineto{\pgfqpoint{4.174932in}{1.952783in}}%
\pgfpathlineto{\pgfqpoint{4.182511in}{1.960681in}}%
\pgfpathclose%
\pgfusepath{fill}%
\end{pgfscope}%
\begin{pgfscope}%
\pgfpathrectangle{\pgfqpoint{1.254980in}{0.150000in}}{\pgfqpoint{5.490039in}{5.490039in}}%
\pgfusepath{clip}%
\pgfsetbuttcap%
\pgfsetroundjoin%
\definecolor{currentfill}{rgb}{0.280894,0.078907,0.402329}%
\pgfsetfillcolor{currentfill}%
\pgfsetfillopacity{0.700000}%
\pgfsetlinewidth{0.000000pt}%
\definecolor{currentstroke}{rgb}{0.000000,0.000000,0.000000}%
\pgfsetstrokecolor{currentstroke}%
\pgfsetdash{}{0pt}%
\pgfpathmoveto{\pgfqpoint{3.230617in}{2.069039in}}%
\pgfpathlineto{\pgfqpoint{3.243659in}{2.062863in}}%
\pgfpathlineto{\pgfqpoint{3.256704in}{2.056719in}}%
\pgfpathlineto{\pgfqpoint{3.269754in}{2.050607in}}%
\pgfpathlineto{\pgfqpoint{3.282807in}{2.044528in}}%
\pgfpathlineto{\pgfqpoint{3.274799in}{2.042595in}}%
\pgfpathlineto{\pgfqpoint{3.266779in}{2.040883in}}%
\pgfpathlineto{\pgfqpoint{3.258748in}{2.039399in}}%
\pgfpathlineto{\pgfqpoint{3.250705in}{2.038149in}}%
\pgfpathlineto{\pgfqpoint{3.237627in}{2.044475in}}%
\pgfpathlineto{\pgfqpoint{3.224554in}{2.050833in}}%
\pgfpathlineto{\pgfqpoint{3.211485in}{2.057224in}}%
\pgfpathlineto{\pgfqpoint{3.198419in}{2.063648in}}%
\pgfpathlineto{\pgfqpoint{3.206487in}{2.064646in}}%
\pgfpathlineto{\pgfqpoint{3.214542in}{2.065882in}}%
\pgfpathlineto{\pgfqpoint{3.222586in}{2.067348in}}%
\pgfpathlineto{\pgfqpoint{3.230617in}{2.069039in}}%
\pgfpathclose%
\pgfusepath{fill}%
\end{pgfscope}%
\begin{pgfscope}%
\pgfpathrectangle{\pgfqpoint{1.254980in}{0.150000in}}{\pgfqpoint{5.490039in}{5.490039in}}%
\pgfusepath{clip}%
\pgfsetbuttcap%
\pgfsetroundjoin%
\definecolor{currentfill}{rgb}{0.283091,0.110553,0.431554}%
\pgfsetfillcolor{currentfill}%
\pgfsetfillopacity{0.700000}%
\pgfsetlinewidth{0.000000pt}%
\definecolor{currentstroke}{rgb}{0.000000,0.000000,0.000000}%
\pgfsetstrokecolor{currentstroke}%
\pgfsetdash{}{0pt}%
\pgfpathmoveto{\pgfqpoint{5.006653in}{2.133867in}}%
\pgfpathlineto{\pgfqpoint{5.020104in}{2.132519in}}%
\pgfpathlineto{\pgfqpoint{5.033563in}{2.131196in}}%
\pgfpathlineto{\pgfqpoint{5.047030in}{2.129896in}}%
\pgfpathlineto{\pgfqpoint{5.060505in}{2.128621in}}%
\pgfpathlineto{\pgfqpoint{5.053230in}{2.120264in}}%
\pgfpathlineto{\pgfqpoint{5.045949in}{2.111838in}}%
\pgfpathlineto{\pgfqpoint{5.038661in}{2.103342in}}%
\pgfpathlineto{\pgfqpoint{5.031367in}{2.094779in}}%
\pgfpathlineto{\pgfqpoint{5.017882in}{2.096108in}}%
\pgfpathlineto{\pgfqpoint{5.004405in}{2.097460in}}%
\pgfpathlineto{\pgfqpoint{4.990935in}{2.098837in}}%
\pgfpathlineto{\pgfqpoint{4.977474in}{2.100238in}}%
\pgfpathlineto{\pgfqpoint{4.984778in}{2.108743in}}%
\pgfpathlineto{\pgfqpoint{4.992076in}{2.117184in}}%
\pgfpathlineto{\pgfqpoint{4.999367in}{2.125559in}}%
\pgfpathlineto{\pgfqpoint{5.006653in}{2.133867in}}%
\pgfpathclose%
\pgfusepath{fill}%
\end{pgfscope}%
\begin{pgfscope}%
\pgfpathrectangle{\pgfqpoint{1.254980in}{0.150000in}}{\pgfqpoint{5.490039in}{5.490039in}}%
\pgfusepath{clip}%
\pgfsetbuttcap%
\pgfsetroundjoin%
\definecolor{currentfill}{rgb}{0.272594,0.025563,0.353093}%
\pgfsetfillcolor{currentfill}%
\pgfsetfillopacity{0.700000}%
\pgfsetlinewidth{0.000000pt}%
\definecolor{currentstroke}{rgb}{0.000000,0.000000,0.000000}%
\pgfsetstrokecolor{currentstroke}%
\pgfsetdash{}{0pt}%
\pgfpathmoveto{\pgfqpoint{4.401757in}{1.991214in}}%
\pgfpathlineto{\pgfqpoint{4.415035in}{1.988570in}}%
\pgfpathlineto{\pgfqpoint{4.428319in}{1.985950in}}%
\pgfpathlineto{\pgfqpoint{4.441610in}{1.983356in}}%
\pgfpathlineto{\pgfqpoint{4.454908in}{1.980787in}}%
\pgfpathlineto{\pgfqpoint{4.447416in}{1.972184in}}%
\pgfpathlineto{\pgfqpoint{4.439918in}{1.963582in}}%
\pgfpathlineto{\pgfqpoint{4.432415in}{1.954985in}}%
\pgfpathlineto{\pgfqpoint{4.424907in}{1.946395in}}%
\pgfpathlineto{\pgfqpoint{4.411599in}{1.949095in}}%
\pgfpathlineto{\pgfqpoint{4.398297in}{1.951820in}}%
\pgfpathlineto{\pgfqpoint{4.385002in}{1.954570in}}%
\pgfpathlineto{\pgfqpoint{4.371714in}{1.957345in}}%
\pgfpathlineto{\pgfqpoint{4.379233in}{1.965799in}}%
\pgfpathlineto{\pgfqpoint{4.386746in}{1.974264in}}%
\pgfpathlineto{\pgfqpoint{4.394254in}{1.982737in}}%
\pgfpathlineto{\pgfqpoint{4.401757in}{1.991214in}}%
\pgfpathclose%
\pgfusepath{fill}%
\end{pgfscope}%
\begin{pgfscope}%
\pgfpathrectangle{\pgfqpoint{1.254980in}{0.150000in}}{\pgfqpoint{5.490039in}{5.490039in}}%
\pgfusepath{clip}%
\pgfsetbuttcap%
\pgfsetroundjoin%
\definecolor{currentfill}{rgb}{0.267004,0.004874,0.329415}%
\pgfsetfillcolor{currentfill}%
\pgfsetfillopacity{0.700000}%
\pgfsetlinewidth{0.000000pt}%
\definecolor{currentstroke}{rgb}{0.000000,0.000000,0.000000}%
\pgfsetstrokecolor{currentstroke}%
\pgfsetdash{}{0pt}%
\pgfpathmoveto{\pgfqpoint{3.827278in}{1.949480in}}%
\pgfpathlineto{\pgfqpoint{3.840419in}{1.945224in}}%
\pgfpathlineto{\pgfqpoint{3.853565in}{1.940996in}}%
\pgfpathlineto{\pgfqpoint{3.866718in}{1.936795in}}%
\pgfpathlineto{\pgfqpoint{3.879875in}{1.932621in}}%
\pgfpathlineto{\pgfqpoint{3.872172in}{1.926231in}}%
\pgfpathlineto{\pgfqpoint{3.864462in}{1.919945in}}%
\pgfpathlineto{\pgfqpoint{3.856746in}{1.913767in}}%
\pgfpathlineto{\pgfqpoint{3.849022in}{1.907702in}}%
\pgfpathlineto{\pgfqpoint{3.835849in}{1.912070in}}%
\pgfpathlineto{\pgfqpoint{3.822682in}{1.916466in}}%
\pgfpathlineto{\pgfqpoint{3.809520in}{1.920888in}}%
\pgfpathlineto{\pgfqpoint{3.796363in}{1.925337in}}%
\pgfpathlineto{\pgfqpoint{3.804102in}{1.931203in}}%
\pgfpathlineto{\pgfqpoint{3.811834in}{1.937185in}}%
\pgfpathlineto{\pgfqpoint{3.819559in}{1.943279in}}%
\pgfpathlineto{\pgfqpoint{3.827278in}{1.949480in}}%
\pgfpathclose%
\pgfusepath{fill}%
\end{pgfscope}%
\begin{pgfscope}%
\pgfpathrectangle{\pgfqpoint{1.254980in}{0.150000in}}{\pgfqpoint{5.490039in}{5.490039in}}%
\pgfusepath{clip}%
\pgfsetbuttcap%
\pgfsetroundjoin%
\definecolor{currentfill}{rgb}{0.272594,0.025563,0.353093}%
\pgfsetfillcolor{currentfill}%
\pgfsetfillopacity{0.700000}%
\pgfsetlinewidth{0.000000pt}%
\definecolor{currentstroke}{rgb}{0.000000,0.000000,0.000000}%
\pgfsetstrokecolor{currentstroke}%
\pgfsetdash{}{0pt}%
\pgfpathmoveto{\pgfqpoint{3.555283in}{1.981431in}}%
\pgfpathlineto{\pgfqpoint{3.568373in}{1.976324in}}%
\pgfpathlineto{\pgfqpoint{3.581468in}{1.971247in}}%
\pgfpathlineto{\pgfqpoint{3.594568in}{1.966198in}}%
\pgfpathlineto{\pgfqpoint{3.607673in}{1.961178in}}%
\pgfpathlineto{\pgfqpoint{3.599845in}{1.956617in}}%
\pgfpathlineto{\pgfqpoint{3.592009in}{1.952213in}}%
\pgfpathlineto{\pgfqpoint{3.584164in}{1.947972in}}%
\pgfpathlineto{\pgfqpoint{3.576311in}{1.943900in}}%
\pgfpathlineto{\pgfqpoint{3.563187in}{1.949140in}}%
\pgfpathlineto{\pgfqpoint{3.550068in}{1.954409in}}%
\pgfpathlineto{\pgfqpoint{3.536954in}{1.959707in}}%
\pgfpathlineto{\pgfqpoint{3.523844in}{1.965033in}}%
\pgfpathlineto{\pgfqpoint{3.531717in}{1.968881in}}%
\pgfpathlineto{\pgfqpoint{3.539581in}{1.972900in}}%
\pgfpathlineto{\pgfqpoint{3.547436in}{1.977085in}}%
\pgfpathlineto{\pgfqpoint{3.555283in}{1.981431in}}%
\pgfpathclose%
\pgfusepath{fill}%
\end{pgfscope}%
\begin{pgfscope}%
\pgfpathrectangle{\pgfqpoint{1.254980in}{0.150000in}}{\pgfqpoint{5.490039in}{5.490039in}}%
\pgfusepath{clip}%
\pgfsetbuttcap%
\pgfsetroundjoin%
\definecolor{currentfill}{rgb}{0.281887,0.150881,0.465405}%
\pgfsetfillcolor{currentfill}%
\pgfsetfillopacity{0.700000}%
\pgfsetlinewidth{0.000000pt}%
\definecolor{currentstroke}{rgb}{0.000000,0.000000,0.000000}%
\pgfsetstrokecolor{currentstroke}%
\pgfsetdash{}{0pt}%
\pgfpathmoveto{\pgfqpoint{5.309209in}{2.210789in}}%
\pgfpathlineto{\pgfqpoint{5.322756in}{2.209905in}}%
\pgfpathlineto{\pgfqpoint{5.336310in}{2.209045in}}%
\pgfpathlineto{\pgfqpoint{5.349873in}{2.208209in}}%
\pgfpathlineto{\pgfqpoint{5.363445in}{2.207397in}}%
\pgfpathlineto{\pgfqpoint{5.356298in}{2.199878in}}%
\pgfpathlineto{\pgfqpoint{5.349143in}{2.192273in}}%
\pgfpathlineto{\pgfqpoint{5.341982in}{2.184582in}}%
\pgfpathlineto{\pgfqpoint{5.334813in}{2.176804in}}%
\pgfpathlineto{\pgfqpoint{5.321230in}{2.177630in}}%
\pgfpathlineto{\pgfqpoint{5.307655in}{2.178479in}}%
\pgfpathlineto{\pgfqpoint{5.294088in}{2.179353in}}%
\pgfpathlineto{\pgfqpoint{5.280530in}{2.180251in}}%
\pgfpathlineto{\pgfqpoint{5.287711in}{2.188010in}}%
\pgfpathlineto{\pgfqpoint{5.294884in}{2.195687in}}%
\pgfpathlineto{\pgfqpoint{5.302050in}{2.203280in}}%
\pgfpathlineto{\pgfqpoint{5.309209in}{2.210789in}}%
\pgfpathclose%
\pgfusepath{fill}%
\end{pgfscope}%
\begin{pgfscope}%
\pgfpathrectangle{\pgfqpoint{1.254980in}{0.150000in}}{\pgfqpoint{5.490039in}{5.490039in}}%
\pgfusepath{clip}%
\pgfsetbuttcap%
\pgfsetroundjoin%
\definecolor{currentfill}{rgb}{0.276194,0.190074,0.493001}%
\pgfsetfillcolor{currentfill}%
\pgfsetfillopacity{0.700000}%
\pgfsetlinewidth{0.000000pt}%
\definecolor{currentstroke}{rgb}{0.000000,0.000000,0.000000}%
\pgfsetstrokecolor{currentstroke}%
\pgfsetdash{}{0pt}%
\pgfpathmoveto{\pgfqpoint{5.611662in}{2.282708in}}%
\pgfpathlineto{\pgfqpoint{5.625304in}{2.282165in}}%
\pgfpathlineto{\pgfqpoint{5.638954in}{2.281646in}}%
\pgfpathlineto{\pgfqpoint{5.652613in}{2.281150in}}%
\pgfpathlineto{\pgfqpoint{5.666281in}{2.280678in}}%
\pgfpathlineto{\pgfqpoint{5.659281in}{2.274274in}}%
\pgfpathlineto{\pgfqpoint{5.652273in}{2.267779in}}%
\pgfpathlineto{\pgfqpoint{5.645256in}{2.261193in}}%
\pgfpathlineto{\pgfqpoint{5.638232in}{2.254514in}}%
\pgfpathlineto{\pgfqpoint{5.624550in}{2.254959in}}%
\pgfpathlineto{\pgfqpoint{5.610877in}{2.255428in}}%
\pgfpathlineto{\pgfqpoint{5.597212in}{2.255921in}}%
\pgfpathlineto{\pgfqpoint{5.583556in}{2.256438in}}%
\pgfpathlineto{\pgfqpoint{5.590595in}{2.263138in}}%
\pgfpathlineto{\pgfqpoint{5.597625in}{2.269749in}}%
\pgfpathlineto{\pgfqpoint{5.604648in}{2.276272in}}%
\pgfpathlineto{\pgfqpoint{5.611662in}{2.282708in}}%
\pgfpathclose%
\pgfusepath{fill}%
\end{pgfscope}%
\begin{pgfscope}%
\pgfpathrectangle{\pgfqpoint{1.254980in}{0.150000in}}{\pgfqpoint{5.490039in}{5.490039in}}%
\pgfusepath{clip}%
\pgfsetbuttcap%
\pgfsetroundjoin%
\definecolor{currentfill}{rgb}{0.267004,0.004874,0.329415}%
\pgfsetfillcolor{currentfill}%
\pgfsetfillopacity{0.700000}%
\pgfsetlinewidth{0.000000pt}%
\definecolor{currentstroke}{rgb}{0.000000,0.000000,0.000000}%
\pgfsetstrokecolor{currentstroke}%
\pgfsetdash{}{0pt}%
\pgfpathmoveto{\pgfqpoint{3.963256in}{1.943423in}}%
\pgfpathlineto{\pgfqpoint{3.976429in}{1.939564in}}%
\pgfpathlineto{\pgfqpoint{3.989607in}{1.935732in}}%
\pgfpathlineto{\pgfqpoint{4.002792in}{1.931925in}}%
\pgfpathlineto{\pgfqpoint{4.015982in}{1.928146in}}%
\pgfpathlineto{\pgfqpoint{4.008332in}{1.921034in}}%
\pgfpathlineto{\pgfqpoint{4.000676in}{1.914001in}}%
\pgfpathlineto{\pgfqpoint{3.993014in}{1.907052in}}%
\pgfpathlineto{\pgfqpoint{3.985345in}{1.900190in}}%
\pgfpathlineto{\pgfqpoint{3.972141in}{1.904152in}}%
\pgfpathlineto{\pgfqpoint{3.958943in}{1.908139in}}%
\pgfpathlineto{\pgfqpoint{3.945751in}{1.912153in}}%
\pgfpathlineto{\pgfqpoint{3.932564in}{1.916194in}}%
\pgfpathlineto{\pgfqpoint{3.940246in}{1.922869in}}%
\pgfpathlineto{\pgfqpoint{3.947923in}{1.929635in}}%
\pgfpathlineto{\pgfqpoint{3.955592in}{1.936488in}}%
\pgfpathlineto{\pgfqpoint{3.963256in}{1.943423in}}%
\pgfpathclose%
\pgfusepath{fill}%
\end{pgfscope}%
\begin{pgfscope}%
\pgfpathrectangle{\pgfqpoint{1.254980in}{0.150000in}}{\pgfqpoint{5.490039in}{5.490039in}}%
\pgfusepath{clip}%
\pgfsetbuttcap%
\pgfsetroundjoin%
\definecolor{currentfill}{rgb}{0.277941,0.056324,0.381191}%
\pgfsetfillcolor{currentfill}%
\pgfsetfillopacity{0.700000}%
\pgfsetlinewidth{0.000000pt}%
\definecolor{currentstroke}{rgb}{0.000000,0.000000,0.000000}%
\pgfsetstrokecolor{currentstroke}%
\pgfsetdash{}{0pt}%
\pgfpathmoveto{\pgfqpoint{4.621128in}{2.031451in}}%
\pgfpathlineto{\pgfqpoint{4.634469in}{2.029327in}}%
\pgfpathlineto{\pgfqpoint{4.647817in}{2.027227in}}%
\pgfpathlineto{\pgfqpoint{4.661172in}{2.025152in}}%
\pgfpathlineto{\pgfqpoint{4.674535in}{2.023101in}}%
\pgfpathlineto{\pgfqpoint{4.667116in}{2.014284in}}%
\pgfpathlineto{\pgfqpoint{4.659691in}{2.005437in}}%
\pgfpathlineto{\pgfqpoint{4.652261in}{1.996564in}}%
\pgfpathlineto{\pgfqpoint{4.644825in}{1.987666in}}%
\pgfpathlineto{\pgfqpoint{4.631452in}{1.989822in}}%
\pgfpathlineto{\pgfqpoint{4.618087in}{1.992002in}}%
\pgfpathlineto{\pgfqpoint{4.604729in}{1.994207in}}%
\pgfpathlineto{\pgfqpoint{4.591377in}{1.996436in}}%
\pgfpathlineto{\pgfqpoint{4.598823in}{2.005224in}}%
\pgfpathlineto{\pgfqpoint{4.606263in}{2.013990in}}%
\pgfpathlineto{\pgfqpoint{4.613698in}{2.022733in}}%
\pgfpathlineto{\pgfqpoint{4.621128in}{2.031451in}}%
\pgfpathclose%
\pgfusepath{fill}%
\end{pgfscope}%
\begin{pgfscope}%
\pgfpathrectangle{\pgfqpoint{1.254980in}{0.150000in}}{\pgfqpoint{5.490039in}{5.490039in}}%
\pgfusepath{clip}%
\pgfsetbuttcap%
\pgfsetroundjoin%
\definecolor{currentfill}{rgb}{0.279574,0.170599,0.479997}%
\pgfsetfillcolor{currentfill}%
\pgfsetfillopacity{0.700000}%
\pgfsetlinewidth{0.000000pt}%
\definecolor{currentstroke}{rgb}{0.000000,0.000000,0.000000}%
\pgfsetstrokecolor{currentstroke}%
\pgfsetdash{}{0pt}%
\pgfpathmoveto{\pgfqpoint{2.852902in}{2.232995in}}%
\pgfpathlineto{\pgfqpoint{2.865911in}{2.225453in}}%
\pgfpathlineto{\pgfqpoint{2.878923in}{2.217951in}}%
\pgfpathlineto{\pgfqpoint{2.891938in}{2.210486in}}%
\pgfpathlineto{\pgfqpoint{2.904956in}{2.203061in}}%
\pgfpathlineto{\pgfqpoint{2.896683in}{2.204699in}}%
\pgfpathlineto{\pgfqpoint{2.888393in}{2.206634in}}%
\pgfpathlineto{\pgfqpoint{2.880087in}{2.208872in}}%
\pgfpathlineto{\pgfqpoint{2.871765in}{2.211420in}}%
\pgfpathlineto{\pgfqpoint{2.858716in}{2.219121in}}%
\pgfpathlineto{\pgfqpoint{2.845670in}{2.226861in}}%
\pgfpathlineto{\pgfqpoint{2.832627in}{2.234640in}}%
\pgfpathlineto{\pgfqpoint{2.819587in}{2.242458in}}%
\pgfpathlineto{\pgfqpoint{2.827941in}{2.239628in}}%
\pgfpathlineto{\pgfqpoint{2.836278in}{2.237113in}}%
\pgfpathlineto{\pgfqpoint{2.844598in}{2.234904in}}%
\pgfpathlineto{\pgfqpoint{2.852902in}{2.232995in}}%
\pgfpathclose%
\pgfusepath{fill}%
\end{pgfscope}%
\begin{pgfscope}%
\pgfpathrectangle{\pgfqpoint{1.254980in}{0.150000in}}{\pgfqpoint{5.490039in}{5.490039in}}%
\pgfusepath{clip}%
\pgfsetbuttcap%
\pgfsetroundjoin%
\definecolor{currentfill}{rgb}{0.276022,0.044167,0.370164}%
\pgfsetfillcolor{currentfill}%
\pgfsetfillopacity{0.700000}%
\pgfsetlinewidth{0.000000pt}%
\definecolor{currentstroke}{rgb}{0.000000,0.000000,0.000000}%
\pgfsetstrokecolor{currentstroke}%
\pgfsetdash{}{0pt}%
\pgfpathmoveto{\pgfqpoint{3.419141in}{2.008714in}}%
\pgfpathlineto{\pgfqpoint{3.432213in}{2.003149in}}%
\pgfpathlineto{\pgfqpoint{3.445289in}{1.997614in}}%
\pgfpathlineto{\pgfqpoint{3.458370in}{1.992110in}}%
\pgfpathlineto{\pgfqpoint{3.471455in}{1.986635in}}%
\pgfpathlineto{\pgfqpoint{3.463553in}{1.983194in}}%
\pgfpathlineto{\pgfqpoint{3.455642in}{1.979939in}}%
\pgfpathlineto{\pgfqpoint{3.447721in}{1.976878in}}%
\pgfpathlineto{\pgfqpoint{3.439789in}{1.974015in}}%
\pgfpathlineto{\pgfqpoint{3.426683in}{1.979723in}}%
\pgfpathlineto{\pgfqpoint{3.413581in}{1.985461in}}%
\pgfpathlineto{\pgfqpoint{3.400484in}{1.991229in}}%
\pgfpathlineto{\pgfqpoint{3.387391in}{1.997027in}}%
\pgfpathlineto{\pgfqpoint{3.395344in}{1.999652in}}%
\pgfpathlineto{\pgfqpoint{3.403286in}{2.002478in}}%
\pgfpathlineto{\pgfqpoint{3.411219in}{2.005501in}}%
\pgfpathlineto{\pgfqpoint{3.419141in}{2.008714in}}%
\pgfpathclose%
\pgfusepath{fill}%
\end{pgfscope}%
\begin{pgfscope}%
\pgfpathrectangle{\pgfqpoint{1.254980in}{0.150000in}}{\pgfqpoint{5.490039in}{5.490039in}}%
\pgfusepath{clip}%
\pgfsetbuttcap%
\pgfsetroundjoin%
\definecolor{currentfill}{rgb}{0.282656,0.100196,0.422160}%
\pgfsetfillcolor{currentfill}%
\pgfsetfillopacity{0.700000}%
\pgfsetlinewidth{0.000000pt}%
\definecolor{currentstroke}{rgb}{0.000000,0.000000,0.000000}%
\pgfsetstrokecolor{currentstroke}%
\pgfsetdash{}{0pt}%
\pgfpathmoveto{\pgfqpoint{4.923705in}{2.106082in}}%
\pgfpathlineto{\pgfqpoint{4.937135in}{2.104585in}}%
\pgfpathlineto{\pgfqpoint{4.950574in}{2.103112in}}%
\pgfpathlineto{\pgfqpoint{4.964020in}{2.101663in}}%
\pgfpathlineto{\pgfqpoint{4.977474in}{2.100238in}}%
\pgfpathlineto{\pgfqpoint{4.970163in}{2.091669in}}%
\pgfpathlineto{\pgfqpoint{4.962847in}{2.083038in}}%
\pgfpathlineto{\pgfqpoint{4.955524in}{2.074345in}}%
\pgfpathlineto{\pgfqpoint{4.948195in}{2.065592in}}%
\pgfpathlineto{\pgfqpoint{4.934731in}{2.067083in}}%
\pgfpathlineto{\pgfqpoint{4.921275in}{2.068598in}}%
\pgfpathlineto{\pgfqpoint{4.907827in}{2.070138in}}%
\pgfpathlineto{\pgfqpoint{4.894386in}{2.071701in}}%
\pgfpathlineto{\pgfqpoint{4.901725in}{2.080383in}}%
\pgfpathlineto{\pgfqpoint{4.909058in}{2.089008in}}%
\pgfpathlineto{\pgfqpoint{4.916384in}{2.097575in}}%
\pgfpathlineto{\pgfqpoint{4.923705in}{2.106082in}}%
\pgfpathclose%
\pgfusepath{fill}%
\end{pgfscope}%
\begin{pgfscope}%
\pgfpathrectangle{\pgfqpoint{1.254980in}{0.150000in}}{\pgfqpoint{5.490039in}{5.490039in}}%
\pgfusepath{clip}%
\pgfsetbuttcap%
\pgfsetroundjoin%
\definecolor{currentfill}{rgb}{0.270595,0.214069,0.507052}%
\pgfsetfillcolor{currentfill}%
\pgfsetfillopacity{0.700000}%
\pgfsetlinewidth{0.000000pt}%
\definecolor{currentstroke}{rgb}{0.000000,0.000000,0.000000}%
\pgfsetstrokecolor{currentstroke}%
\pgfsetdash{}{0pt}%
\pgfpathmoveto{\pgfqpoint{5.831394in}{2.325182in}}%
\pgfpathlineto{\pgfqpoint{5.845110in}{2.324830in}}%
\pgfpathlineto{\pgfqpoint{5.858834in}{2.324502in}}%
\pgfpathlineto{\pgfqpoint{5.872567in}{2.324197in}}%
\pgfpathlineto{\pgfqpoint{5.865677in}{2.318617in}}%
\pgfpathlineto{\pgfqpoint{5.858779in}{2.312951in}}%
\pgfpathlineto{\pgfqpoint{5.851872in}{2.307198in}}%
\pgfpathlineto{\pgfqpoint{5.844956in}{2.301355in}}%
\pgfpathlineto{\pgfqpoint{5.831206in}{2.301607in}}%
\pgfpathlineto{\pgfqpoint{5.817466in}{2.301882in}}%
\pgfpathlineto{\pgfqpoint{5.803734in}{2.302181in}}%
\pgfpathlineto{\pgfqpoint{5.810662in}{2.308060in}}%
\pgfpathlineto{\pgfqpoint{5.817581in}{2.313852in}}%
\pgfpathlineto{\pgfqpoint{5.824492in}{2.319559in}}%
\pgfpathlineto{\pgfqpoint{5.831394in}{2.325182in}}%
\pgfpathclose%
\pgfusepath{fill}%
\end{pgfscope}%
\begin{pgfscope}%
\pgfpathrectangle{\pgfqpoint{1.254980in}{0.150000in}}{\pgfqpoint{5.490039in}{5.490039in}}%
\pgfusepath{clip}%
\pgfsetbuttcap%
\pgfsetroundjoin%
\definecolor{currentfill}{rgb}{0.263663,0.237631,0.518762}%
\pgfsetfillcolor{currentfill}%
\pgfsetfillopacity{0.700000}%
\pgfsetlinewidth{0.000000pt}%
\definecolor{currentstroke}{rgb}{0.000000,0.000000,0.000000}%
\pgfsetstrokecolor{currentstroke}%
\pgfsetdash{}{0pt}%
\pgfpathmoveto{\pgfqpoint{2.611295in}{2.373151in}}%
\pgfpathlineto{\pgfqpoint{2.624296in}{2.364658in}}%
\pgfpathlineto{\pgfqpoint{2.637299in}{2.356210in}}%
\pgfpathlineto{\pgfqpoint{2.650304in}{2.347807in}}%
\pgfpathlineto{\pgfqpoint{2.663311in}{2.339448in}}%
\pgfpathlineto{\pgfqpoint{2.654840in}{2.343457in}}%
\pgfpathlineto{\pgfqpoint{2.646349in}{2.347805in}}%
\pgfpathlineto{\pgfqpoint{2.637838in}{2.352502in}}%
\pgfpathlineto{\pgfqpoint{2.629307in}{2.357555in}}%
\pgfpathlineto{\pgfqpoint{2.616265in}{2.366206in}}%
\pgfpathlineto{\pgfqpoint{2.603225in}{2.374901in}}%
\pgfpathlineto{\pgfqpoint{2.590187in}{2.383641in}}%
\pgfpathlineto{\pgfqpoint{2.577150in}{2.392426in}}%
\pgfpathlineto{\pgfqpoint{2.585718in}{2.387076in}}%
\pgfpathlineto{\pgfqpoint{2.594264in}{2.382085in}}%
\pgfpathlineto{\pgfqpoint{2.602790in}{2.377446in}}%
\pgfpathlineto{\pgfqpoint{2.611295in}{2.373151in}}%
\pgfpathclose%
\pgfusepath{fill}%
\end{pgfscope}%
\begin{pgfscope}%
\pgfpathrectangle{\pgfqpoint{1.254980in}{0.150000in}}{\pgfqpoint{5.490039in}{5.490039in}}%
\pgfusepath{clip}%
\pgfsetbuttcap%
\pgfsetroundjoin%
\definecolor{currentfill}{rgb}{0.271305,0.019942,0.347269}%
\pgfsetfillcolor{currentfill}%
\pgfsetfillopacity{0.700000}%
\pgfsetlinewidth{0.000000pt}%
\definecolor{currentstroke}{rgb}{0.000000,0.000000,0.000000}%
\pgfsetstrokecolor{currentstroke}%
\pgfsetdash{}{0pt}%
\pgfpathmoveto{\pgfqpoint{4.318628in}{1.968694in}}%
\pgfpathlineto{\pgfqpoint{4.331890in}{1.965819in}}%
\pgfpathlineto{\pgfqpoint{4.345158in}{1.962969in}}%
\pgfpathlineto{\pgfqpoint{4.358433in}{1.960144in}}%
\pgfpathlineto{\pgfqpoint{4.371714in}{1.957345in}}%
\pgfpathlineto{\pgfqpoint{4.364190in}{1.948904in}}%
\pgfpathlineto{\pgfqpoint{4.356661in}{1.940480in}}%
\pgfpathlineto{\pgfqpoint{4.349126in}{1.932077in}}%
\pgfpathlineto{\pgfqpoint{4.341587in}{1.923698in}}%
\pgfpathlineto{\pgfqpoint{4.328294in}{1.926641in}}%
\pgfpathlineto{\pgfqpoint{4.315009in}{1.929609in}}%
\pgfpathlineto{\pgfqpoint{4.301730in}{1.932602in}}%
\pgfpathlineto{\pgfqpoint{4.288457in}{1.935621in}}%
\pgfpathlineto{\pgfqpoint{4.296008in}{1.943852in}}%
\pgfpathlineto{\pgfqpoint{4.303553in}{1.952110in}}%
\pgfpathlineto{\pgfqpoint{4.311094in}{1.960392in}}%
\pgfpathlineto{\pgfqpoint{4.318628in}{1.968694in}}%
\pgfpathclose%
\pgfusepath{fill}%
\end{pgfscope}%
\begin{pgfscope}%
\pgfpathrectangle{\pgfqpoint{1.254980in}{0.150000in}}{\pgfqpoint{5.490039in}{5.490039in}}%
\pgfusepath{clip}%
\pgfsetbuttcap%
\pgfsetroundjoin%
\definecolor{currentfill}{rgb}{0.282623,0.140926,0.457517}%
\pgfsetfillcolor{currentfill}%
\pgfsetfillopacity{0.700000}%
\pgfsetlinewidth{0.000000pt}%
\definecolor{currentstroke}{rgb}{0.000000,0.000000,0.000000}%
\pgfsetstrokecolor{currentstroke}%
\pgfsetdash{}{0pt}%
\pgfpathmoveto{\pgfqpoint{5.226379in}{2.184079in}}%
\pgfpathlineto{\pgfqpoint{5.239905in}{2.183086in}}%
\pgfpathlineto{\pgfqpoint{5.253438in}{2.182117in}}%
\pgfpathlineto{\pgfqpoint{5.266980in}{2.181172in}}%
\pgfpathlineto{\pgfqpoint{5.280530in}{2.180251in}}%
\pgfpathlineto{\pgfqpoint{5.273342in}{2.172408in}}%
\pgfpathlineto{\pgfqpoint{5.266147in}{2.164481in}}%
\pgfpathlineto{\pgfqpoint{5.258946in}{2.156471in}}%
\pgfpathlineto{\pgfqpoint{5.251737in}{2.148378in}}%
\pgfpathlineto{\pgfqpoint{5.238175in}{2.149326in}}%
\pgfpathlineto{\pgfqpoint{5.224622in}{2.150298in}}%
\pgfpathlineto{\pgfqpoint{5.211078in}{2.151294in}}%
\pgfpathlineto{\pgfqpoint{5.197541in}{2.152314in}}%
\pgfpathlineto{\pgfqpoint{5.204761in}{2.160375in}}%
\pgfpathlineto{\pgfqpoint{5.211974in}{2.168357in}}%
\pgfpathlineto{\pgfqpoint{5.219180in}{2.176258in}}%
\pgfpathlineto{\pgfqpoint{5.226379in}{2.184079in}}%
\pgfpathclose%
\pgfusepath{fill}%
\end{pgfscope}%
\begin{pgfscope}%
\pgfpathrectangle{\pgfqpoint{1.254980in}{0.150000in}}{\pgfqpoint{5.490039in}{5.490039in}}%
\pgfusepath{clip}%
\pgfsetbuttcap%
\pgfsetroundjoin%
\definecolor{currentfill}{rgb}{0.267004,0.004874,0.329415}%
\pgfsetfillcolor{currentfill}%
\pgfsetfillopacity{0.700000}%
\pgfsetlinewidth{0.000000pt}%
\definecolor{currentstroke}{rgb}{0.000000,0.000000,0.000000}%
\pgfsetstrokecolor{currentstroke}%
\pgfsetdash{}{0pt}%
\pgfpathmoveto{\pgfqpoint{4.099293in}{1.943115in}}%
\pgfpathlineto{\pgfqpoint{4.112501in}{1.939634in}}%
\pgfpathlineto{\pgfqpoint{4.125715in}{1.936179in}}%
\pgfpathlineto{\pgfqpoint{4.138935in}{1.932750in}}%
\pgfpathlineto{\pgfqpoint{4.152161in}{1.929347in}}%
\pgfpathlineto{\pgfqpoint{4.144560in}{1.921633in}}%
\pgfpathlineto{\pgfqpoint{4.136953in}{1.913976in}}%
\pgfpathlineto{\pgfqpoint{4.129340in}{1.906378in}}%
\pgfpathlineto{\pgfqpoint{4.121722in}{1.898844in}}%
\pgfpathlineto{\pgfqpoint{4.108483in}{1.902417in}}%
\pgfpathlineto{\pgfqpoint{4.095251in}{1.906015in}}%
\pgfpathlineto{\pgfqpoint{4.082024in}{1.909638in}}%
\pgfpathlineto{\pgfqpoint{4.068803in}{1.913288in}}%
\pgfpathlineto{\pgfqpoint{4.076434in}{1.920648in}}%
\pgfpathlineto{\pgfqpoint{4.084060in}{1.928075in}}%
\pgfpathlineto{\pgfqpoint{4.091679in}{1.935565in}}%
\pgfpathlineto{\pgfqpoint{4.099293in}{1.943115in}}%
\pgfpathclose%
\pgfusepath{fill}%
\end{pgfscope}%
\begin{pgfscope}%
\pgfpathrectangle{\pgfqpoint{1.254980in}{0.150000in}}{\pgfqpoint{5.490039in}{5.490039in}}%
\pgfusepath{clip}%
\pgfsetbuttcap%
\pgfsetroundjoin%
\definecolor{currentfill}{rgb}{0.278012,0.180367,0.486697}%
\pgfsetfillcolor{currentfill}%
\pgfsetfillopacity{0.700000}%
\pgfsetlinewidth{0.000000pt}%
\definecolor{currentstroke}{rgb}{0.000000,0.000000,0.000000}%
\pgfsetstrokecolor{currentstroke}%
\pgfsetdash{}{0pt}%
\pgfpathmoveto{\pgfqpoint{5.529019in}{2.258740in}}%
\pgfpathlineto{\pgfqpoint{5.542640in}{2.258129in}}%
\pgfpathlineto{\pgfqpoint{5.556270in}{2.257542in}}%
\pgfpathlineto{\pgfqpoint{5.569909in}{2.256978in}}%
\pgfpathlineto{\pgfqpoint{5.583556in}{2.256438in}}%
\pgfpathlineto{\pgfqpoint{5.576510in}{2.249647in}}%
\pgfpathlineto{\pgfqpoint{5.569455in}{2.242764in}}%
\pgfpathlineto{\pgfqpoint{5.562393in}{2.235789in}}%
\pgfpathlineto{\pgfqpoint{5.555322in}{2.228721in}}%
\pgfpathlineto{\pgfqpoint{5.541661in}{2.229248in}}%
\pgfpathlineto{\pgfqpoint{5.528009in}{2.229799in}}%
\pgfpathlineto{\pgfqpoint{5.514366in}{2.230374in}}%
\pgfpathlineto{\pgfqpoint{5.500731in}{2.230972in}}%
\pgfpathlineto{\pgfqpoint{5.507815in}{2.238048in}}%
\pgfpathlineto{\pgfqpoint{5.514890in}{2.245034in}}%
\pgfpathlineto{\pgfqpoint{5.521958in}{2.251931in}}%
\pgfpathlineto{\pgfqpoint{5.529019in}{2.258740in}}%
\pgfpathclose%
\pgfusepath{fill}%
\end{pgfscope}%
\begin{pgfscope}%
\pgfpathrectangle{\pgfqpoint{1.254980in}{0.150000in}}{\pgfqpoint{5.490039in}{5.490039in}}%
\pgfusepath{clip}%
\pgfsetbuttcap%
\pgfsetroundjoin%
\definecolor{currentfill}{rgb}{0.276022,0.044167,0.370164}%
\pgfsetfillcolor{currentfill}%
\pgfsetfillopacity{0.700000}%
\pgfsetlinewidth{0.000000pt}%
\definecolor{currentstroke}{rgb}{0.000000,0.000000,0.000000}%
\pgfsetstrokecolor{currentstroke}%
\pgfsetdash{}{0pt}%
\pgfpathmoveto{\pgfqpoint{4.538045in}{2.005598in}}%
\pgfpathlineto{\pgfqpoint{4.551367in}{2.003271in}}%
\pgfpathlineto{\pgfqpoint{4.564697in}{2.000968in}}%
\pgfpathlineto{\pgfqpoint{4.578034in}{1.998689in}}%
\pgfpathlineto{\pgfqpoint{4.591377in}{1.996436in}}%
\pgfpathlineto{\pgfqpoint{4.583927in}{1.987629in}}%
\pgfpathlineto{\pgfqpoint{4.576470in}{1.978807in}}%
\pgfpathlineto{\pgfqpoint{4.569009in}{1.969970in}}%
\pgfpathlineto{\pgfqpoint{4.561542in}{1.961123in}}%
\pgfpathlineto{\pgfqpoint{4.548188in}{1.963495in}}%
\pgfpathlineto{\pgfqpoint{4.534841in}{1.965891in}}%
\pgfpathlineto{\pgfqpoint{4.521502in}{1.968312in}}%
\pgfpathlineto{\pgfqpoint{4.508169in}{1.970758in}}%
\pgfpathlineto{\pgfqpoint{4.515646in}{1.979482in}}%
\pgfpathlineto{\pgfqpoint{4.523117in}{1.988199in}}%
\pgfpathlineto{\pgfqpoint{4.530584in}{1.996905in}}%
\pgfpathlineto{\pgfqpoint{4.538045in}{2.005598in}}%
\pgfpathclose%
\pgfusepath{fill}%
\end{pgfscope}%
\begin{pgfscope}%
\pgfpathrectangle{\pgfqpoint{1.254980in}{0.150000in}}{\pgfqpoint{5.490039in}{5.490039in}}%
\pgfusepath{clip}%
\pgfsetbuttcap%
\pgfsetroundjoin%
\definecolor{currentfill}{rgb}{0.283091,0.110553,0.431554}%
\pgfsetfillcolor{currentfill}%
\pgfsetfillopacity{0.700000}%
\pgfsetlinewidth{0.000000pt}%
\definecolor{currentstroke}{rgb}{0.000000,0.000000,0.000000}%
\pgfsetstrokecolor{currentstroke}%
\pgfsetdash{}{0pt}%
\pgfpathmoveto{\pgfqpoint{3.094036in}{2.116237in}}%
\pgfpathlineto{\pgfqpoint{3.107070in}{2.109544in}}%
\pgfpathlineto{\pgfqpoint{3.120109in}{2.102886in}}%
\pgfpathlineto{\pgfqpoint{3.133151in}{2.096262in}}%
\pgfpathlineto{\pgfqpoint{3.146197in}{2.089672in}}%
\pgfpathlineto{\pgfqpoint{3.138091in}{2.089174in}}%
\pgfpathlineto{\pgfqpoint{3.129973in}{2.088929in}}%
\pgfpathlineto{\pgfqpoint{3.121841in}{2.088946in}}%
\pgfpathlineto{\pgfqpoint{3.113696in}{2.089230in}}%
\pgfpathlineto{\pgfqpoint{3.100623in}{2.096081in}}%
\pgfpathlineto{\pgfqpoint{3.087555in}{2.102965in}}%
\pgfpathlineto{\pgfqpoint{3.074490in}{2.109884in}}%
\pgfpathlineto{\pgfqpoint{3.061428in}{2.116837in}}%
\pgfpathlineto{\pgfqpoint{3.069601in}{2.116287in}}%
\pgfpathlineto{\pgfqpoint{3.077759in}{2.116008in}}%
\pgfpathlineto{\pgfqpoint{3.085904in}{2.115994in}}%
\pgfpathlineto{\pgfqpoint{3.094036in}{2.116237in}}%
\pgfpathclose%
\pgfusepath{fill}%
\end{pgfscope}%
\begin{pgfscope}%
\pgfpathrectangle{\pgfqpoint{1.254980in}{0.150000in}}{\pgfqpoint{5.490039in}{5.490039in}}%
\pgfusepath{clip}%
\pgfsetbuttcap%
\pgfsetroundjoin%
\definecolor{currentfill}{rgb}{0.281446,0.084320,0.407414}%
\pgfsetfillcolor{currentfill}%
\pgfsetfillopacity{0.700000}%
\pgfsetlinewidth{0.000000pt}%
\definecolor{currentstroke}{rgb}{0.000000,0.000000,0.000000}%
\pgfsetstrokecolor{currentstroke}%
\pgfsetdash{}{0pt}%
\pgfpathmoveto{\pgfqpoint{4.840700in}{2.078197in}}%
\pgfpathlineto{\pgfqpoint{4.854111in}{2.076536in}}%
\pgfpathlineto{\pgfqpoint{4.867528in}{2.074901in}}%
\pgfpathlineto{\pgfqpoint{4.880953in}{2.073289in}}%
\pgfpathlineto{\pgfqpoint{4.894386in}{2.071701in}}%
\pgfpathlineto{\pgfqpoint{4.887042in}{2.062964in}}%
\pgfpathlineto{\pgfqpoint{4.879691in}{2.054173in}}%
\pgfpathlineto{\pgfqpoint{4.872335in}{2.045330in}}%
\pgfpathlineto{\pgfqpoint{4.864973in}{2.036436in}}%
\pgfpathlineto{\pgfqpoint{4.851530in}{2.038103in}}%
\pgfpathlineto{\pgfqpoint{4.838095in}{2.039794in}}%
\pgfpathlineto{\pgfqpoint{4.824667in}{2.041510in}}%
\pgfpathlineto{\pgfqpoint{4.811247in}{2.043249in}}%
\pgfpathlineto{\pgfqpoint{4.818619in}{2.052059in}}%
\pgfpathlineto{\pgfqpoint{4.825986in}{2.060821in}}%
\pgfpathlineto{\pgfqpoint{4.833346in}{2.069534in}}%
\pgfpathlineto{\pgfqpoint{4.840700in}{2.078197in}}%
\pgfpathclose%
\pgfusepath{fill}%
\end{pgfscope}%
\begin{pgfscope}%
\pgfpathrectangle{\pgfqpoint{1.254980in}{0.150000in}}{\pgfqpoint{5.490039in}{5.490039in}}%
\pgfusepath{clip}%
\pgfsetbuttcap%
\pgfsetroundjoin%
\definecolor{currentfill}{rgb}{0.280267,0.073417,0.397163}%
\pgfsetfillcolor{currentfill}%
\pgfsetfillopacity{0.700000}%
\pgfsetlinewidth{0.000000pt}%
\definecolor{currentstroke}{rgb}{0.000000,0.000000,0.000000}%
\pgfsetstrokecolor{currentstroke}%
\pgfsetdash{}{0pt}%
\pgfpathmoveto{\pgfqpoint{3.282807in}{2.044528in}}%
\pgfpathlineto{\pgfqpoint{3.295865in}{2.038480in}}%
\pgfpathlineto{\pgfqpoint{3.308927in}{2.032464in}}%
\pgfpathlineto{\pgfqpoint{3.321994in}{2.026480in}}%
\pgfpathlineto{\pgfqpoint{3.335065in}{2.020528in}}%
\pgfpathlineto{\pgfqpoint{3.327079in}{2.018353in}}%
\pgfpathlineto{\pgfqpoint{3.319083in}{2.016396in}}%
\pgfpathlineto{\pgfqpoint{3.311075in}{2.014663in}}%
\pgfpathlineto{\pgfqpoint{3.303056in}{2.013161in}}%
\pgfpathlineto{\pgfqpoint{3.289962in}{2.019361in}}%
\pgfpathlineto{\pgfqpoint{3.276872in}{2.025592in}}%
\pgfpathlineto{\pgfqpoint{3.263786in}{2.031854in}}%
\pgfpathlineto{\pgfqpoint{3.250705in}{2.038149in}}%
\pgfpathlineto{\pgfqpoint{3.258748in}{2.039399in}}%
\pgfpathlineto{\pgfqpoint{3.266779in}{2.040883in}}%
\pgfpathlineto{\pgfqpoint{3.274799in}{2.042595in}}%
\pgfpathlineto{\pgfqpoint{3.282807in}{2.044528in}}%
\pgfpathclose%
\pgfusepath{fill}%
\end{pgfscope}%
\begin{pgfscope}%
\pgfpathrectangle{\pgfqpoint{1.254980in}{0.150000in}}{\pgfqpoint{5.490039in}{5.490039in}}%
\pgfusepath{clip}%
\pgfsetbuttcap%
\pgfsetroundjoin%
\definecolor{currentfill}{rgb}{0.283072,0.130895,0.449241}%
\pgfsetfillcolor{currentfill}%
\pgfsetfillopacity{0.700000}%
\pgfsetlinewidth{0.000000pt}%
\definecolor{currentstroke}{rgb}{0.000000,0.000000,0.000000}%
\pgfsetstrokecolor{currentstroke}%
\pgfsetdash{}{0pt}%
\pgfpathmoveto{\pgfqpoint{5.143476in}{2.156633in}}%
\pgfpathlineto{\pgfqpoint{5.156980in}{2.155517in}}%
\pgfpathlineto{\pgfqpoint{5.170492in}{2.154426in}}%
\pgfpathlineto{\pgfqpoint{5.184012in}{2.153358in}}%
\pgfpathlineto{\pgfqpoint{5.197541in}{2.152314in}}%
\pgfpathlineto{\pgfqpoint{5.190314in}{2.144173in}}%
\pgfpathlineto{\pgfqpoint{5.183080in}{2.135954in}}%
\pgfpathlineto{\pgfqpoint{5.175840in}{2.127656in}}%
\pgfpathlineto{\pgfqpoint{5.168593in}{2.119280in}}%
\pgfpathlineto{\pgfqpoint{5.155053in}{2.120364in}}%
\pgfpathlineto{\pgfqpoint{5.141522in}{2.121471in}}%
\pgfpathlineto{\pgfqpoint{5.127999in}{2.122603in}}%
\pgfpathlineto{\pgfqpoint{5.114485in}{2.123759in}}%
\pgfpathlineto{\pgfqpoint{5.121742in}{2.132090in}}%
\pgfpathlineto{\pgfqpoint{5.128993in}{2.140346in}}%
\pgfpathlineto{\pgfqpoint{5.136238in}{2.148527in}}%
\pgfpathlineto{\pgfqpoint{5.143476in}{2.156633in}}%
\pgfpathclose%
\pgfusepath{fill}%
\end{pgfscope}%
\begin{pgfscope}%
\pgfpathrectangle{\pgfqpoint{1.254980in}{0.150000in}}{\pgfqpoint{5.490039in}{5.490039in}}%
\pgfusepath{clip}%
\pgfsetbuttcap%
\pgfsetroundjoin%
\definecolor{currentfill}{rgb}{0.268510,0.009605,0.335427}%
\pgfsetfillcolor{currentfill}%
\pgfsetfillopacity{0.700000}%
\pgfsetlinewidth{0.000000pt}%
\definecolor{currentstroke}{rgb}{0.000000,0.000000,0.000000}%
\pgfsetstrokecolor{currentstroke}%
\pgfsetdash{}{0pt}%
\pgfpathmoveto{\pgfqpoint{3.743792in}{1.943409in}}%
\pgfpathlineto{\pgfqpoint{3.756926in}{1.938850in}}%
\pgfpathlineto{\pgfqpoint{3.770067in}{1.934318in}}%
\pgfpathlineto{\pgfqpoint{3.783212in}{1.929814in}}%
\pgfpathlineto{\pgfqpoint{3.796363in}{1.925337in}}%
\pgfpathlineto{\pgfqpoint{3.788617in}{1.919593in}}%
\pgfpathlineto{\pgfqpoint{3.780864in}{1.913975in}}%
\pgfpathlineto{\pgfqpoint{3.773104in}{1.908489in}}%
\pgfpathlineto{\pgfqpoint{3.765336in}{1.903140in}}%
\pgfpathlineto{\pgfqpoint{3.752168in}{1.907824in}}%
\pgfpathlineto{\pgfqpoint{3.739006in}{1.912535in}}%
\pgfpathlineto{\pgfqpoint{3.725850in}{1.917274in}}%
\pgfpathlineto{\pgfqpoint{3.712698in}{1.922040in}}%
\pgfpathlineto{\pgfqpoint{3.720483in}{1.927177in}}%
\pgfpathlineto{\pgfqpoint{3.728260in}{1.932455in}}%
\pgfpathlineto{\pgfqpoint{3.736029in}{1.937867in}}%
\pgfpathlineto{\pgfqpoint{3.743792in}{1.943409in}}%
\pgfpathclose%
\pgfusepath{fill}%
\end{pgfscope}%
\begin{pgfscope}%
\pgfpathrectangle{\pgfqpoint{1.254980in}{0.150000in}}{\pgfqpoint{5.490039in}{5.490039in}}%
\pgfusepath{clip}%
\pgfsetbuttcap%
\pgfsetroundjoin%
\definecolor{currentfill}{rgb}{0.271305,0.019942,0.347269}%
\pgfsetfillcolor{currentfill}%
\pgfsetfillopacity{0.700000}%
\pgfsetlinewidth{0.000000pt}%
\definecolor{currentstroke}{rgb}{0.000000,0.000000,0.000000}%
\pgfsetstrokecolor{currentstroke}%
\pgfsetdash{}{0pt}%
\pgfpathmoveto{\pgfqpoint{3.607673in}{1.961178in}}%
\pgfpathlineto{\pgfqpoint{3.620783in}{1.956187in}}%
\pgfpathlineto{\pgfqpoint{3.633899in}{1.951224in}}%
\pgfpathlineto{\pgfqpoint{3.647019in}{1.946290in}}%
\pgfpathlineto{\pgfqpoint{3.660144in}{1.941384in}}%
\pgfpathlineto{\pgfqpoint{3.652334in}{1.936607in}}%
\pgfpathlineto{\pgfqpoint{3.644516in}{1.931985in}}%
\pgfpathlineto{\pgfqpoint{3.636690in}{1.927522in}}%
\pgfpathlineto{\pgfqpoint{3.628856in}{1.923225in}}%
\pgfpathlineto{\pgfqpoint{3.615712in}{1.928351in}}%
\pgfpathlineto{\pgfqpoint{3.602573in}{1.933506in}}%
\pgfpathlineto{\pgfqpoint{3.589439in}{1.938688in}}%
\pgfpathlineto{\pgfqpoint{3.576311in}{1.943900in}}%
\pgfpathlineto{\pgfqpoint{3.584164in}{1.947972in}}%
\pgfpathlineto{\pgfqpoint{3.592009in}{1.952213in}}%
\pgfpathlineto{\pgfqpoint{3.599845in}{1.956617in}}%
\pgfpathlineto{\pgfqpoint{3.607673in}{1.961178in}}%
\pgfpathclose%
\pgfusepath{fill}%
\end{pgfscope}%
\begin{pgfscope}%
\pgfpathrectangle{\pgfqpoint{1.254980in}{0.150000in}}{\pgfqpoint{5.490039in}{5.490039in}}%
\pgfusepath{clip}%
\pgfsetbuttcap%
\pgfsetroundjoin%
\definecolor{currentfill}{rgb}{0.280868,0.160771,0.472899}%
\pgfsetfillcolor{currentfill}%
\pgfsetfillopacity{0.700000}%
\pgfsetlinewidth{0.000000pt}%
\definecolor{currentstroke}{rgb}{0.000000,0.000000,0.000000}%
\pgfsetstrokecolor{currentstroke}%
\pgfsetdash{}{0pt}%
\pgfpathmoveto{\pgfqpoint{2.904956in}{2.203061in}}%
\pgfpathlineto{\pgfqpoint{2.917978in}{2.195673in}}%
\pgfpathlineto{\pgfqpoint{2.931002in}{2.188323in}}%
\pgfpathlineto{\pgfqpoint{2.944030in}{2.181011in}}%
\pgfpathlineto{\pgfqpoint{2.957061in}{2.173736in}}%
\pgfpathlineto{\pgfqpoint{2.948817in}{2.175104in}}%
\pgfpathlineto{\pgfqpoint{2.940557in}{2.176765in}}%
\pgfpathlineto{\pgfqpoint{2.932281in}{2.178725in}}%
\pgfpathlineto{\pgfqpoint{2.923989in}{2.180993in}}%
\pgfpathlineto{\pgfqpoint{2.910929in}{2.188543in}}%
\pgfpathlineto{\pgfqpoint{2.897871in}{2.196131in}}%
\pgfpathlineto{\pgfqpoint{2.884816in}{2.203757in}}%
\pgfpathlineto{\pgfqpoint{2.871765in}{2.211420in}}%
\pgfpathlineto{\pgfqpoint{2.880087in}{2.208872in}}%
\pgfpathlineto{\pgfqpoint{2.888393in}{2.206634in}}%
\pgfpathlineto{\pgfqpoint{2.896683in}{2.204699in}}%
\pgfpathlineto{\pgfqpoint{2.904956in}{2.203061in}}%
\pgfpathclose%
\pgfusepath{fill}%
\end{pgfscope}%
\begin{pgfscope}%
\pgfpathrectangle{\pgfqpoint{1.254980in}{0.150000in}}{\pgfqpoint{5.490039in}{5.490039in}}%
\pgfusepath{clip}%
\pgfsetbuttcap%
\pgfsetroundjoin%
\definecolor{currentfill}{rgb}{0.266580,0.228262,0.514349}%
\pgfsetfillcolor{currentfill}%
\pgfsetfillopacity{0.700000}%
\pgfsetlinewidth{0.000000pt}%
\definecolor{currentstroke}{rgb}{0.000000,0.000000,0.000000}%
\pgfsetstrokecolor{currentstroke}%
\pgfsetdash{}{0pt}%
\pgfpathmoveto{\pgfqpoint{2.663311in}{2.339448in}}%
\pgfpathlineto{\pgfqpoint{2.676321in}{2.331134in}}%
\pgfpathlineto{\pgfqpoint{2.689333in}{2.322862in}}%
\pgfpathlineto{\pgfqpoint{2.702347in}{2.314634in}}%
\pgfpathlineto{\pgfqpoint{2.715363in}{2.306449in}}%
\pgfpathlineto{\pgfqpoint{2.706926in}{2.310171in}}%
\pgfpathlineto{\pgfqpoint{2.698469in}{2.314230in}}%
\pgfpathlineto{\pgfqpoint{2.689993in}{2.318633in}}%
\pgfpathlineto{\pgfqpoint{2.681498in}{2.323389in}}%
\pgfpathlineto{\pgfqpoint{2.668447in}{2.331866in}}%
\pgfpathlineto{\pgfqpoint{2.655398in}{2.340386in}}%
\pgfpathlineto{\pgfqpoint{2.642352in}{2.348949in}}%
\pgfpathlineto{\pgfqpoint{2.629307in}{2.357555in}}%
\pgfpathlineto{\pgfqpoint{2.637838in}{2.352502in}}%
\pgfpathlineto{\pgfqpoint{2.646349in}{2.347805in}}%
\pgfpathlineto{\pgfqpoint{2.654840in}{2.343457in}}%
\pgfpathlineto{\pgfqpoint{2.663311in}{2.339448in}}%
\pgfpathclose%
\pgfusepath{fill}%
\end{pgfscope}%
\begin{pgfscope}%
\pgfpathrectangle{\pgfqpoint{1.254980in}{0.150000in}}{\pgfqpoint{5.490039in}{5.490039in}}%
\pgfusepath{clip}%
\pgfsetbuttcap%
\pgfsetroundjoin%
\definecolor{currentfill}{rgb}{0.267004,0.004874,0.329415}%
\pgfsetfillcolor{currentfill}%
\pgfsetfillopacity{0.700000}%
\pgfsetlinewidth{0.000000pt}%
\definecolor{currentstroke}{rgb}{0.000000,0.000000,0.000000}%
\pgfsetstrokecolor{currentstroke}%
\pgfsetdash{}{0pt}%
\pgfpathmoveto{\pgfqpoint{3.879875in}{1.932621in}}%
\pgfpathlineto{\pgfqpoint{3.893039in}{1.928474in}}%
\pgfpathlineto{\pgfqpoint{3.906208in}{1.924354in}}%
\pgfpathlineto{\pgfqpoint{3.919383in}{1.920260in}}%
\pgfpathlineto{\pgfqpoint{3.932564in}{1.916194in}}%
\pgfpathlineto{\pgfqpoint{3.924875in}{1.909614in}}%
\pgfpathlineto{\pgfqpoint{3.917180in}{1.903135in}}%
\pgfpathlineto{\pgfqpoint{3.909479in}{1.896761in}}%
\pgfpathlineto{\pgfqpoint{3.901771in}{1.890497in}}%
\pgfpathlineto{\pgfqpoint{3.888575in}{1.894758in}}%
\pgfpathlineto{\pgfqpoint{3.875385in}{1.899046in}}%
\pgfpathlineto{\pgfqpoint{3.862201in}{1.903361in}}%
\pgfpathlineto{\pgfqpoint{3.849022in}{1.907702in}}%
\pgfpathlineto{\pgfqpoint{3.856746in}{1.913767in}}%
\pgfpathlineto{\pgfqpoint{3.864462in}{1.919945in}}%
\pgfpathlineto{\pgfqpoint{3.872172in}{1.926231in}}%
\pgfpathlineto{\pgfqpoint{3.879875in}{1.932621in}}%
\pgfpathclose%
\pgfusepath{fill}%
\end{pgfscope}%
\begin{pgfscope}%
\pgfpathrectangle{\pgfqpoint{1.254980in}{0.150000in}}{\pgfqpoint{5.490039in}{5.490039in}}%
\pgfusepath{clip}%
\pgfsetbuttcap%
\pgfsetroundjoin%
\definecolor{currentfill}{rgb}{0.271828,0.209303,0.504434}%
\pgfsetfillcolor{currentfill}%
\pgfsetfillopacity{0.700000}%
\pgfsetlinewidth{0.000000pt}%
\definecolor{currentstroke}{rgb}{0.000000,0.000000,0.000000}%
\pgfsetstrokecolor{currentstroke}%
\pgfsetdash{}{0pt}%
\pgfpathmoveto{\pgfqpoint{5.748896in}{2.303613in}}%
\pgfpathlineto{\pgfqpoint{5.762592in}{2.303219in}}%
\pgfpathlineto{\pgfqpoint{5.776297in}{2.302849in}}%
\pgfpathlineto{\pgfqpoint{5.790011in}{2.302503in}}%
\pgfpathlineto{\pgfqpoint{5.803734in}{2.302181in}}%
\pgfpathlineto{\pgfqpoint{5.796798in}{2.296212in}}%
\pgfpathlineto{\pgfqpoint{5.789853in}{2.290153in}}%
\pgfpathlineto{\pgfqpoint{5.782900in}{2.284001in}}%
\pgfpathlineto{\pgfqpoint{5.775938in}{2.277756in}}%
\pgfpathlineto{\pgfqpoint{5.762200in}{2.278038in}}%
\pgfpathlineto{\pgfqpoint{5.748471in}{2.278344in}}%
\pgfpathlineto{\pgfqpoint{5.734750in}{2.278674in}}%
\pgfpathlineto{\pgfqpoint{5.721039in}{2.279028in}}%
\pgfpathlineto{\pgfqpoint{5.728016in}{2.285308in}}%
\pgfpathlineto{\pgfqpoint{5.734984in}{2.291498in}}%
\pgfpathlineto{\pgfqpoint{5.741944in}{2.297599in}}%
\pgfpathlineto{\pgfqpoint{5.748896in}{2.303613in}}%
\pgfpathclose%
\pgfusepath{fill}%
\end{pgfscope}%
\begin{pgfscope}%
\pgfpathrectangle{\pgfqpoint{1.254980in}{0.150000in}}{\pgfqpoint{5.490039in}{5.490039in}}%
\pgfusepath{clip}%
\pgfsetbuttcap%
\pgfsetroundjoin%
\definecolor{currentfill}{rgb}{0.279574,0.170599,0.479997}%
\pgfsetfillcolor{currentfill}%
\pgfsetfillopacity{0.700000}%
\pgfsetlinewidth{0.000000pt}%
\definecolor{currentstroke}{rgb}{0.000000,0.000000,0.000000}%
\pgfsetstrokecolor{currentstroke}%
\pgfsetdash{}{0pt}%
\pgfpathmoveto{\pgfqpoint{5.446277in}{2.233602in}}%
\pgfpathlineto{\pgfqpoint{5.459878in}{2.232909in}}%
\pgfpathlineto{\pgfqpoint{5.473487in}{2.232239in}}%
\pgfpathlineto{\pgfqpoint{5.487105in}{2.231594in}}%
\pgfpathlineto{\pgfqpoint{5.500731in}{2.230972in}}%
\pgfpathlineto{\pgfqpoint{5.493640in}{2.223805in}}%
\pgfpathlineto{\pgfqpoint{5.486540in}{2.216546in}}%
\pgfpathlineto{\pgfqpoint{5.479433in}{2.209196in}}%
\pgfpathlineto{\pgfqpoint{5.472318in}{2.201753in}}%
\pgfpathlineto{\pgfqpoint{5.458679in}{2.202376in}}%
\pgfpathlineto{\pgfqpoint{5.445049in}{2.203021in}}%
\pgfpathlineto{\pgfqpoint{5.431427in}{2.203691in}}%
\pgfpathlineto{\pgfqpoint{5.417814in}{2.204385in}}%
\pgfpathlineto{\pgfqpoint{5.424941in}{2.211822in}}%
\pgfpathlineto{\pgfqpoint{5.432060in}{2.219171in}}%
\pgfpathlineto{\pgfqpoint{5.439172in}{2.226430in}}%
\pgfpathlineto{\pgfqpoint{5.446277in}{2.233602in}}%
\pgfpathclose%
\pgfusepath{fill}%
\end{pgfscope}%
\begin{pgfscope}%
\pgfpathrectangle{\pgfqpoint{1.254980in}{0.150000in}}{\pgfqpoint{5.490039in}{5.490039in}}%
\pgfusepath{clip}%
\pgfsetbuttcap%
\pgfsetroundjoin%
\definecolor{currentfill}{rgb}{0.268510,0.009605,0.335427}%
\pgfsetfillcolor{currentfill}%
\pgfsetfillopacity{0.700000}%
\pgfsetlinewidth{0.000000pt}%
\definecolor{currentstroke}{rgb}{0.000000,0.000000,0.000000}%
\pgfsetstrokecolor{currentstroke}%
\pgfsetdash{}{0pt}%
\pgfpathmoveto{\pgfqpoint{4.235433in}{1.947947in}}%
\pgfpathlineto{\pgfqpoint{4.248679in}{1.944828in}}%
\pgfpathlineto{\pgfqpoint{4.261932in}{1.941734in}}%
\pgfpathlineto{\pgfqpoint{4.275191in}{1.938665in}}%
\pgfpathlineto{\pgfqpoint{4.288457in}{1.935621in}}%
\pgfpathlineto{\pgfqpoint{4.280901in}{1.927420in}}%
\pgfpathlineto{\pgfqpoint{4.273340in}{1.919254in}}%
\pgfpathlineto{\pgfqpoint{4.265773in}{1.911125in}}%
\pgfpathlineto{\pgfqpoint{4.258201in}{1.903038in}}%
\pgfpathlineto{\pgfqpoint{4.244924in}{1.906238in}}%
\pgfpathlineto{\pgfqpoint{4.231653in}{1.909463in}}%
\pgfpathlineto{\pgfqpoint{4.218388in}{1.912713in}}%
\pgfpathlineto{\pgfqpoint{4.205130in}{1.915989in}}%
\pgfpathlineto{\pgfqpoint{4.212714in}{1.923915in}}%
\pgfpathlineto{\pgfqpoint{4.220292in}{1.931886in}}%
\pgfpathlineto{\pgfqpoint{4.227865in}{1.939898in}}%
\pgfpathlineto{\pgfqpoint{4.235433in}{1.947947in}}%
\pgfpathclose%
\pgfusepath{fill}%
\end{pgfscope}%
\begin{pgfscope}%
\pgfpathrectangle{\pgfqpoint{1.254980in}{0.150000in}}{\pgfqpoint{5.490039in}{5.490039in}}%
\pgfusepath{clip}%
\pgfsetbuttcap%
\pgfsetroundjoin%
\definecolor{currentfill}{rgb}{0.280267,0.073417,0.397163}%
\pgfsetfillcolor{currentfill}%
\pgfsetfillopacity{0.700000}%
\pgfsetlinewidth{0.000000pt}%
\definecolor{currentstroke}{rgb}{0.000000,0.000000,0.000000}%
\pgfsetstrokecolor{currentstroke}%
\pgfsetdash{}{0pt}%
\pgfpathmoveto{\pgfqpoint{4.757643in}{2.050449in}}%
\pgfpathlineto{\pgfqpoint{4.771033in}{2.048613in}}%
\pgfpathlineto{\pgfqpoint{4.784430in}{2.046801in}}%
\pgfpathlineto{\pgfqpoint{4.797835in}{2.045013in}}%
\pgfpathlineto{\pgfqpoint{4.811247in}{2.043249in}}%
\pgfpathlineto{\pgfqpoint{4.803870in}{2.034394in}}%
\pgfpathlineto{\pgfqpoint{4.796487in}{2.025495in}}%
\pgfpathlineto{\pgfqpoint{4.789098in}{2.016554in}}%
\pgfpathlineto{\pgfqpoint{4.781703in}{2.007573in}}%
\pgfpathlineto{\pgfqpoint{4.768281in}{2.009429in}}%
\pgfpathlineto{\pgfqpoint{4.754866in}{2.011310in}}%
\pgfpathlineto{\pgfqpoint{4.741459in}{2.013214in}}%
\pgfpathlineto{\pgfqpoint{4.728060in}{2.015143in}}%
\pgfpathlineto{\pgfqpoint{4.735464in}{2.024026in}}%
\pgfpathlineto{\pgfqpoint{4.742863in}{2.032873in}}%
\pgfpathlineto{\pgfqpoint{4.750256in}{2.041682in}}%
\pgfpathlineto{\pgfqpoint{4.757643in}{2.050449in}}%
\pgfpathclose%
\pgfusepath{fill}%
\end{pgfscope}%
\begin{pgfscope}%
\pgfpathrectangle{\pgfqpoint{1.254980in}{0.150000in}}{\pgfqpoint{5.490039in}{5.490039in}}%
\pgfusepath{clip}%
\pgfsetbuttcap%
\pgfsetroundjoin%
\definecolor{currentfill}{rgb}{0.273809,0.031497,0.358853}%
\pgfsetfillcolor{currentfill}%
\pgfsetfillopacity{0.700000}%
\pgfsetlinewidth{0.000000pt}%
\definecolor{currentstroke}{rgb}{0.000000,0.000000,0.000000}%
\pgfsetstrokecolor{currentstroke}%
\pgfsetdash{}{0pt}%
\pgfpathmoveto{\pgfqpoint{4.454908in}{1.980787in}}%
\pgfpathlineto{\pgfqpoint{4.468213in}{1.978243in}}%
\pgfpathlineto{\pgfqpoint{4.481525in}{1.975723in}}%
\pgfpathlineto{\pgfqpoint{4.494843in}{1.973228in}}%
\pgfpathlineto{\pgfqpoint{4.508169in}{1.970758in}}%
\pgfpathlineto{\pgfqpoint{4.500687in}{1.962029in}}%
\pgfpathlineto{\pgfqpoint{4.493200in}{1.953298in}}%
\pgfpathlineto{\pgfqpoint{4.485707in}{1.944568in}}%
\pgfpathlineto{\pgfqpoint{4.478209in}{1.935843in}}%
\pgfpathlineto{\pgfqpoint{4.464873in}{1.938444in}}%
\pgfpathlineto{\pgfqpoint{4.451544in}{1.941069in}}%
\pgfpathlineto{\pgfqpoint{4.438222in}{1.943720in}}%
\pgfpathlineto{\pgfqpoint{4.424907in}{1.946395in}}%
\pgfpathlineto{\pgfqpoint{4.432415in}{1.954985in}}%
\pgfpathlineto{\pgfqpoint{4.439918in}{1.963582in}}%
\pgfpathlineto{\pgfqpoint{4.447416in}{1.972184in}}%
\pgfpathlineto{\pgfqpoint{4.454908in}{1.980787in}}%
\pgfpathclose%
\pgfusepath{fill}%
\end{pgfscope}%
\begin{pgfscope}%
\pgfpathrectangle{\pgfqpoint{1.254980in}{0.150000in}}{\pgfqpoint{5.490039in}{5.490039in}}%
\pgfusepath{clip}%
\pgfsetbuttcap%
\pgfsetroundjoin%
\definecolor{currentfill}{rgb}{0.274952,0.037752,0.364543}%
\pgfsetfillcolor{currentfill}%
\pgfsetfillopacity{0.700000}%
\pgfsetlinewidth{0.000000pt}%
\definecolor{currentstroke}{rgb}{0.000000,0.000000,0.000000}%
\pgfsetstrokecolor{currentstroke}%
\pgfsetdash{}{0pt}%
\pgfpathmoveto{\pgfqpoint{3.471455in}{1.986635in}}%
\pgfpathlineto{\pgfqpoint{3.484545in}{1.981190in}}%
\pgfpathlineto{\pgfqpoint{3.497640in}{1.975775in}}%
\pgfpathlineto{\pgfqpoint{3.510740in}{1.970390in}}%
\pgfpathlineto{\pgfqpoint{3.523844in}{1.965033in}}%
\pgfpathlineto{\pgfqpoint{3.515963in}{1.961364in}}%
\pgfpathlineto{\pgfqpoint{3.508071in}{1.957878in}}%
\pgfpathlineto{\pgfqpoint{3.500171in}{1.954581in}}%
\pgfpathlineto{\pgfqpoint{3.492261in}{1.951480in}}%
\pgfpathlineto{\pgfqpoint{3.479136in}{1.957069in}}%
\pgfpathlineto{\pgfqpoint{3.466016in}{1.962688in}}%
\pgfpathlineto{\pgfqpoint{3.452900in}{1.968337in}}%
\pgfpathlineto{\pgfqpoint{3.439789in}{1.974015in}}%
\pgfpathlineto{\pgfqpoint{3.447721in}{1.976878in}}%
\pgfpathlineto{\pgfqpoint{3.455642in}{1.979939in}}%
\pgfpathlineto{\pgfqpoint{3.463553in}{1.983194in}}%
\pgfpathlineto{\pgfqpoint{3.471455in}{1.986635in}}%
\pgfpathclose%
\pgfusepath{fill}%
\end{pgfscope}%
\begin{pgfscope}%
\pgfpathrectangle{\pgfqpoint{1.254980in}{0.150000in}}{\pgfqpoint{5.490039in}{5.490039in}}%
\pgfusepath{clip}%
\pgfsetbuttcap%
\pgfsetroundjoin%
\definecolor{currentfill}{rgb}{0.283229,0.120777,0.440584}%
\pgfsetfillcolor{currentfill}%
\pgfsetfillopacity{0.700000}%
\pgfsetlinewidth{0.000000pt}%
\definecolor{currentstroke}{rgb}{0.000000,0.000000,0.000000}%
\pgfsetstrokecolor{currentstroke}%
\pgfsetdash{}{0pt}%
\pgfpathmoveto{\pgfqpoint{5.060505in}{2.128621in}}%
\pgfpathlineto{\pgfqpoint{5.073988in}{2.127369in}}%
\pgfpathlineto{\pgfqpoint{5.087479in}{2.126142in}}%
\pgfpathlineto{\pgfqpoint{5.100978in}{2.124938in}}%
\pgfpathlineto{\pgfqpoint{5.114485in}{2.123759in}}%
\pgfpathlineto{\pgfqpoint{5.107220in}{2.115354in}}%
\pgfpathlineto{\pgfqpoint{5.099949in}{2.106876in}}%
\pgfpathlineto{\pgfqpoint{5.092672in}{2.098326in}}%
\pgfpathlineto{\pgfqpoint{5.085388in}{2.089704in}}%
\pgfpathlineto{\pgfqpoint{5.071871in}{2.090937in}}%
\pgfpathlineto{\pgfqpoint{5.058362in}{2.092194in}}%
\pgfpathlineto{\pgfqpoint{5.044861in}{2.093474in}}%
\pgfpathlineto{\pgfqpoint{5.031367in}{2.094779in}}%
\pgfpathlineto{\pgfqpoint{5.038661in}{2.103342in}}%
\pgfpathlineto{\pgfqpoint{5.045949in}{2.111838in}}%
\pgfpathlineto{\pgfqpoint{5.053230in}{2.120264in}}%
\pgfpathlineto{\pgfqpoint{5.060505in}{2.128621in}}%
\pgfpathclose%
\pgfusepath{fill}%
\end{pgfscope}%
\begin{pgfscope}%
\pgfpathrectangle{\pgfqpoint{1.254980in}{0.150000in}}{\pgfqpoint{5.490039in}{5.490039in}}%
\pgfusepath{clip}%
\pgfsetbuttcap%
\pgfsetroundjoin%
\definecolor{currentfill}{rgb}{0.267004,0.004874,0.329415}%
\pgfsetfillcolor{currentfill}%
\pgfsetfillopacity{0.700000}%
\pgfsetlinewidth{0.000000pt}%
\definecolor{currentstroke}{rgb}{0.000000,0.000000,0.000000}%
\pgfsetstrokecolor{currentstroke}%
\pgfsetdash{}{0pt}%
\pgfpathmoveto{\pgfqpoint{4.015982in}{1.928146in}}%
\pgfpathlineto{\pgfqpoint{4.029178in}{1.924392in}}%
\pgfpathlineto{\pgfqpoint{4.042381in}{1.920665in}}%
\pgfpathlineto{\pgfqpoint{4.055589in}{1.916963in}}%
\pgfpathlineto{\pgfqpoint{4.068803in}{1.913288in}}%
\pgfpathlineto{\pgfqpoint{4.061167in}{1.905999in}}%
\pgfpathlineto{\pgfqpoint{4.053524in}{1.898786in}}%
\pgfpathlineto{\pgfqpoint{4.045875in}{1.891654in}}%
\pgfpathlineto{\pgfqpoint{4.038221in}{1.884606in}}%
\pgfpathlineto{\pgfqpoint{4.024993in}{1.888463in}}%
\pgfpathlineto{\pgfqpoint{4.011771in}{1.892346in}}%
\pgfpathlineto{\pgfqpoint{3.998555in}{1.896255in}}%
\pgfpathlineto{\pgfqpoint{3.985345in}{1.900190in}}%
\pgfpathlineto{\pgfqpoint{3.993014in}{1.907052in}}%
\pgfpathlineto{\pgfqpoint{4.000676in}{1.914001in}}%
\pgfpathlineto{\pgfqpoint{4.008332in}{1.921034in}}%
\pgfpathlineto{\pgfqpoint{4.015982in}{1.928146in}}%
\pgfpathclose%
\pgfusepath{fill}%
\end{pgfscope}%
\begin{pgfscope}%
\pgfpathrectangle{\pgfqpoint{1.254980in}{0.150000in}}{\pgfqpoint{5.490039in}{5.490039in}}%
\pgfusepath{clip}%
\pgfsetbuttcap%
\pgfsetroundjoin%
\definecolor{currentfill}{rgb}{0.280868,0.160771,0.472899}%
\pgfsetfillcolor{currentfill}%
\pgfsetfillopacity{0.700000}%
\pgfsetlinewidth{0.000000pt}%
\definecolor{currentstroke}{rgb}{0.000000,0.000000,0.000000}%
\pgfsetstrokecolor{currentstroke}%
\pgfsetdash{}{0pt}%
\pgfpathmoveto{\pgfqpoint{5.363445in}{2.207397in}}%
\pgfpathlineto{\pgfqpoint{5.377024in}{2.206608in}}%
\pgfpathlineto{\pgfqpoint{5.390612in}{2.205843in}}%
\pgfpathlineto{\pgfqpoint{5.404209in}{2.205102in}}%
\pgfpathlineto{\pgfqpoint{5.417814in}{2.204385in}}%
\pgfpathlineto{\pgfqpoint{5.410679in}{2.196858in}}%
\pgfpathlineto{\pgfqpoint{5.403537in}{2.189241in}}%
\pgfpathlineto{\pgfqpoint{5.396387in}{2.181534in}}%
\pgfpathlineto{\pgfqpoint{5.389230in}{2.173737in}}%
\pgfpathlineto{\pgfqpoint{5.375613in}{2.174468in}}%
\pgfpathlineto{\pgfqpoint{5.362004in}{2.175223in}}%
\pgfpathlineto{\pgfqpoint{5.348404in}{2.176001in}}%
\pgfpathlineto{\pgfqpoint{5.334813in}{2.176804in}}%
\pgfpathlineto{\pgfqpoint{5.341982in}{2.184582in}}%
\pgfpathlineto{\pgfqpoint{5.349143in}{2.192273in}}%
\pgfpathlineto{\pgfqpoint{5.356298in}{2.199878in}}%
\pgfpathlineto{\pgfqpoint{5.363445in}{2.207397in}}%
\pgfpathclose%
\pgfusepath{fill}%
\end{pgfscope}%
\begin{pgfscope}%
\pgfpathrectangle{\pgfqpoint{1.254980in}{0.150000in}}{\pgfqpoint{5.490039in}{5.490039in}}%
\pgfusepath{clip}%
\pgfsetbuttcap%
\pgfsetroundjoin%
\definecolor{currentfill}{rgb}{0.278791,0.062145,0.386592}%
\pgfsetfillcolor{currentfill}%
\pgfsetfillopacity{0.700000}%
\pgfsetlinewidth{0.000000pt}%
\definecolor{currentstroke}{rgb}{0.000000,0.000000,0.000000}%
\pgfsetstrokecolor{currentstroke}%
\pgfsetdash{}{0pt}%
\pgfpathmoveto{\pgfqpoint{4.674535in}{2.023101in}}%
\pgfpathlineto{\pgfqpoint{4.687905in}{2.021075in}}%
\pgfpathlineto{\pgfqpoint{4.701283in}{2.019074in}}%
\pgfpathlineto{\pgfqpoint{4.714668in}{2.017096in}}%
\pgfpathlineto{\pgfqpoint{4.728060in}{2.015143in}}%
\pgfpathlineto{\pgfqpoint{4.720650in}{2.006225in}}%
\pgfpathlineto{\pgfqpoint{4.713235in}{1.997275in}}%
\pgfpathlineto{\pgfqpoint{4.705814in}{1.988294in}}%
\pgfpathlineto{\pgfqpoint{4.698388in}{1.979287in}}%
\pgfpathlineto{\pgfqpoint{4.684986in}{1.981345in}}%
\pgfpathlineto{\pgfqpoint{4.671592in}{1.983428in}}%
\pgfpathlineto{\pgfqpoint{4.658205in}{1.985535in}}%
\pgfpathlineto{\pgfqpoint{4.644825in}{1.987666in}}%
\pgfpathlineto{\pgfqpoint{4.652261in}{1.996564in}}%
\pgfpathlineto{\pgfqpoint{4.659691in}{2.005437in}}%
\pgfpathlineto{\pgfqpoint{4.667116in}{2.014284in}}%
\pgfpathlineto{\pgfqpoint{4.674535in}{2.023101in}}%
\pgfpathclose%
\pgfusepath{fill}%
\end{pgfscope}%
\begin{pgfscope}%
\pgfpathrectangle{\pgfqpoint{1.254980in}{0.150000in}}{\pgfqpoint{5.490039in}{5.490039in}}%
\pgfusepath{clip}%
\pgfsetbuttcap%
\pgfsetroundjoin%
\definecolor{currentfill}{rgb}{0.282910,0.105393,0.426902}%
\pgfsetfillcolor{currentfill}%
\pgfsetfillopacity{0.700000}%
\pgfsetlinewidth{0.000000pt}%
\definecolor{currentstroke}{rgb}{0.000000,0.000000,0.000000}%
\pgfsetstrokecolor{currentstroke}%
\pgfsetdash{}{0pt}%
\pgfpathmoveto{\pgfqpoint{3.146197in}{2.089672in}}%
\pgfpathlineto{\pgfqpoint{3.159247in}{2.083116in}}%
\pgfpathlineto{\pgfqpoint{3.172300in}{2.076594in}}%
\pgfpathlineto{\pgfqpoint{3.185358in}{2.070104in}}%
\pgfpathlineto{\pgfqpoint{3.198419in}{2.063648in}}%
\pgfpathlineto{\pgfqpoint{3.190339in}{2.062894in}}%
\pgfpathlineto{\pgfqpoint{3.182247in}{2.062390in}}%
\pgfpathlineto{\pgfqpoint{3.174141in}{2.062144in}}%
\pgfpathlineto{\pgfqpoint{3.166022in}{2.062163in}}%
\pgfpathlineto{\pgfqpoint{3.152935in}{2.068880in}}%
\pgfpathlineto{\pgfqpoint{3.139851in}{2.075630in}}%
\pgfpathlineto{\pgfqpoint{3.126772in}{2.082413in}}%
\pgfpathlineto{\pgfqpoint{3.113696in}{2.089230in}}%
\pgfpathlineto{\pgfqpoint{3.121841in}{2.088946in}}%
\pgfpathlineto{\pgfqpoint{3.129973in}{2.088929in}}%
\pgfpathlineto{\pgfqpoint{3.138091in}{2.089174in}}%
\pgfpathlineto{\pgfqpoint{3.146197in}{2.089672in}}%
\pgfpathclose%
\pgfusepath{fill}%
\end{pgfscope}%
\begin{pgfscope}%
\pgfpathrectangle{\pgfqpoint{1.254980in}{0.150000in}}{\pgfqpoint{5.490039in}{5.490039in}}%
\pgfusepath{clip}%
\pgfsetbuttcap%
\pgfsetroundjoin%
\definecolor{currentfill}{rgb}{0.274128,0.199721,0.498911}%
\pgfsetfillcolor{currentfill}%
\pgfsetfillopacity{0.700000}%
\pgfsetlinewidth{0.000000pt}%
\definecolor{currentstroke}{rgb}{0.000000,0.000000,0.000000}%
\pgfsetstrokecolor{currentstroke}%
\pgfsetdash{}{0pt}%
\pgfpathmoveto{\pgfqpoint{5.666281in}{2.280678in}}%
\pgfpathlineto{\pgfqpoint{5.679957in}{2.280230in}}%
\pgfpathlineto{\pgfqpoint{5.693642in}{2.279806in}}%
\pgfpathlineto{\pgfqpoint{5.707336in}{2.279405in}}%
\pgfpathlineto{\pgfqpoint{5.721039in}{2.279028in}}%
\pgfpathlineto{\pgfqpoint{5.714054in}{2.272655in}}%
\pgfpathlineto{\pgfqpoint{5.707060in}{2.266188in}}%
\pgfpathlineto{\pgfqpoint{5.700058in}{2.259627in}}%
\pgfpathlineto{\pgfqpoint{5.693048in}{2.252970in}}%
\pgfpathlineto{\pgfqpoint{5.679331in}{2.253320in}}%
\pgfpathlineto{\pgfqpoint{5.665622in}{2.253695in}}%
\pgfpathlineto{\pgfqpoint{5.651923in}{2.254092in}}%
\pgfpathlineto{\pgfqpoint{5.638232in}{2.254514in}}%
\pgfpathlineto{\pgfqpoint{5.645256in}{2.261193in}}%
\pgfpathlineto{\pgfqpoint{5.652273in}{2.267779in}}%
\pgfpathlineto{\pgfqpoint{5.659281in}{2.274274in}}%
\pgfpathlineto{\pgfqpoint{5.666281in}{2.280678in}}%
\pgfpathclose%
\pgfusepath{fill}%
\end{pgfscope}%
\begin{pgfscope}%
\pgfpathrectangle{\pgfqpoint{1.254980in}{0.150000in}}{\pgfqpoint{5.490039in}{5.490039in}}%
\pgfusepath{clip}%
\pgfsetbuttcap%
\pgfsetroundjoin%
\definecolor{currentfill}{rgb}{0.235526,0.309527,0.542944}%
\pgfsetfillcolor{currentfill}%
\pgfsetfillopacity{0.700000}%
\pgfsetlinewidth{0.000000pt}%
\definecolor{currentstroke}{rgb}{0.000000,0.000000,0.000000}%
\pgfsetstrokecolor{currentstroke}%
\pgfsetdash{}{0pt}%
\pgfpathmoveto{\pgfqpoint{2.420849in}{2.501560in}}%
\pgfpathlineto{\pgfqpoint{2.433866in}{2.492193in}}%
\pgfpathlineto{\pgfqpoint{2.446884in}{2.482877in}}%
\pgfpathlineto{\pgfqpoint{2.459904in}{2.473612in}}%
\pgfpathlineto{\pgfqpoint{2.472925in}{2.464397in}}%
\pgfpathlineto{\pgfqpoint{2.464261in}{2.470720in}}%
\pgfpathlineto{\pgfqpoint{2.455573in}{2.477427in}}%
\pgfpathlineto{\pgfqpoint{2.446863in}{2.484526in}}%
\pgfpathlineto{\pgfqpoint{2.438128in}{2.492025in}}%
\pgfpathlineto{\pgfqpoint{2.425067in}{2.501549in}}%
\pgfpathlineto{\pgfqpoint{2.412008in}{2.511123in}}%
\pgfpathlineto{\pgfqpoint{2.398950in}{2.520748in}}%
\pgfpathlineto{\pgfqpoint{2.385893in}{2.530424in}}%
\pgfpathlineto{\pgfqpoint{2.394669in}{2.522610in}}%
\pgfpathlineto{\pgfqpoint{2.403420in}{2.515200in}}%
\pgfpathlineto{\pgfqpoint{2.412146in}{2.508186in}}%
\pgfpathlineto{\pgfqpoint{2.420849in}{2.501560in}}%
\pgfpathclose%
\pgfusepath{fill}%
\end{pgfscope}%
\begin{pgfscope}%
\pgfpathrectangle{\pgfqpoint{1.254980in}{0.150000in}}{\pgfqpoint{5.490039in}{5.490039in}}%
\pgfusepath{clip}%
\pgfsetbuttcap%
\pgfsetroundjoin%
\definecolor{currentfill}{rgb}{0.282910,0.105393,0.426902}%
\pgfsetfillcolor{currentfill}%
\pgfsetfillopacity{0.700000}%
\pgfsetlinewidth{0.000000pt}%
\definecolor{currentstroke}{rgb}{0.000000,0.000000,0.000000}%
\pgfsetstrokecolor{currentstroke}%
\pgfsetdash{}{0pt}%
\pgfpathmoveto{\pgfqpoint{4.977474in}{2.100238in}}%
\pgfpathlineto{\pgfqpoint{4.990935in}{2.098837in}}%
\pgfpathlineto{\pgfqpoint{5.004405in}{2.097460in}}%
\pgfpathlineto{\pgfqpoint{5.017882in}{2.096108in}}%
\pgfpathlineto{\pgfqpoint{5.031367in}{2.094779in}}%
\pgfpathlineto{\pgfqpoint{5.024067in}{2.086149in}}%
\pgfpathlineto{\pgfqpoint{5.016760in}{2.077453in}}%
\pgfpathlineto{\pgfqpoint{5.009448in}{2.068692in}}%
\pgfpathlineto{\pgfqpoint{5.002129in}{2.059868in}}%
\pgfpathlineto{\pgfqpoint{4.988634in}{2.061263in}}%
\pgfpathlineto{\pgfqpoint{4.975146in}{2.062682in}}%
\pgfpathlineto{\pgfqpoint{4.961667in}{2.064125in}}%
\pgfpathlineto{\pgfqpoint{4.948195in}{2.065592in}}%
\pgfpathlineto{\pgfqpoint{4.955524in}{2.074345in}}%
\pgfpathlineto{\pgfqpoint{4.962847in}{2.083038in}}%
\pgfpathlineto{\pgfqpoint{4.970163in}{2.091669in}}%
\pgfpathlineto{\pgfqpoint{4.977474in}{2.100238in}}%
\pgfpathclose%
\pgfusepath{fill}%
\end{pgfscope}%
\begin{pgfscope}%
\pgfpathrectangle{\pgfqpoint{1.254980in}{0.150000in}}{\pgfqpoint{5.490039in}{5.490039in}}%
\pgfusepath{clip}%
\pgfsetbuttcap%
\pgfsetroundjoin%
\definecolor{currentfill}{rgb}{0.279566,0.067836,0.391917}%
\pgfsetfillcolor{currentfill}%
\pgfsetfillopacity{0.700000}%
\pgfsetlinewidth{0.000000pt}%
\definecolor{currentstroke}{rgb}{0.000000,0.000000,0.000000}%
\pgfsetstrokecolor{currentstroke}%
\pgfsetdash{}{0pt}%
\pgfpathmoveto{\pgfqpoint{3.335065in}{2.020528in}}%
\pgfpathlineto{\pgfqpoint{3.348140in}{2.014606in}}%
\pgfpathlineto{\pgfqpoint{3.361219in}{2.008716in}}%
\pgfpathlineto{\pgfqpoint{3.374303in}{2.002856in}}%
\pgfpathlineto{\pgfqpoint{3.387391in}{1.997027in}}%
\pgfpathlineto{\pgfqpoint{3.379428in}{1.994611in}}%
\pgfpathlineto{\pgfqpoint{3.371454in}{1.992409in}}%
\pgfpathlineto{\pgfqpoint{3.363470in}{1.990428in}}%
\pgfpathlineto{\pgfqpoint{3.355474in}{1.988674in}}%
\pgfpathlineto{\pgfqpoint{3.342363in}{1.994750in}}%
\pgfpathlineto{\pgfqpoint{3.329257in}{2.000856in}}%
\pgfpathlineto{\pgfqpoint{3.316154in}{2.006993in}}%
\pgfpathlineto{\pgfqpoint{3.303056in}{2.013161in}}%
\pgfpathlineto{\pgfqpoint{3.311075in}{2.014663in}}%
\pgfpathlineto{\pgfqpoint{3.319083in}{2.016396in}}%
\pgfpathlineto{\pgfqpoint{3.327079in}{2.018353in}}%
\pgfpathlineto{\pgfqpoint{3.335065in}{2.020528in}}%
\pgfpathclose%
\pgfusepath{fill}%
\end{pgfscope}%
\begin{pgfscope}%
\pgfpathrectangle{\pgfqpoint{1.254980in}{0.150000in}}{\pgfqpoint{5.490039in}{5.490039in}}%
\pgfusepath{clip}%
\pgfsetbuttcap%
\pgfsetroundjoin%
\definecolor{currentfill}{rgb}{0.271305,0.019942,0.347269}%
\pgfsetfillcolor{currentfill}%
\pgfsetfillopacity{0.700000}%
\pgfsetlinewidth{0.000000pt}%
\definecolor{currentstroke}{rgb}{0.000000,0.000000,0.000000}%
\pgfsetstrokecolor{currentstroke}%
\pgfsetdash{}{0pt}%
\pgfpathmoveto{\pgfqpoint{4.371714in}{1.957345in}}%
\pgfpathlineto{\pgfqpoint{4.385002in}{1.954570in}}%
\pgfpathlineto{\pgfqpoint{4.398297in}{1.951820in}}%
\pgfpathlineto{\pgfqpoint{4.411599in}{1.949095in}}%
\pgfpathlineto{\pgfqpoint{4.424907in}{1.946395in}}%
\pgfpathlineto{\pgfqpoint{4.417394in}{1.937816in}}%
\pgfpathlineto{\pgfqpoint{4.409875in}{1.929250in}}%
\pgfpathlineto{\pgfqpoint{4.402352in}{1.920702in}}%
\pgfpathlineto{\pgfqpoint{4.394823in}{1.912174in}}%
\pgfpathlineto{\pgfqpoint{4.381504in}{1.915017in}}%
\pgfpathlineto{\pgfqpoint{4.368191in}{1.917886in}}%
\pgfpathlineto{\pgfqpoint{4.354886in}{1.920779in}}%
\pgfpathlineto{\pgfqpoint{4.341587in}{1.923698in}}%
\pgfpathlineto{\pgfqpoint{4.349126in}{1.932077in}}%
\pgfpathlineto{\pgfqpoint{4.356661in}{1.940480in}}%
\pgfpathlineto{\pgfqpoint{4.364190in}{1.948904in}}%
\pgfpathlineto{\pgfqpoint{4.371714in}{1.957345in}}%
\pgfpathclose%
\pgfusepath{fill}%
\end{pgfscope}%
\begin{pgfscope}%
\pgfpathrectangle{\pgfqpoint{1.254980in}{0.150000in}}{\pgfqpoint{5.490039in}{5.490039in}}%
\pgfusepath{clip}%
\pgfsetbuttcap%
\pgfsetroundjoin%
\definecolor{currentfill}{rgb}{0.270595,0.214069,0.507052}%
\pgfsetfillcolor{currentfill}%
\pgfsetfillopacity{0.700000}%
\pgfsetlinewidth{0.000000pt}%
\definecolor{currentstroke}{rgb}{0.000000,0.000000,0.000000}%
\pgfsetstrokecolor{currentstroke}%
\pgfsetdash{}{0pt}%
\pgfpathmoveto{\pgfqpoint{2.715363in}{2.306449in}}%
\pgfpathlineto{\pgfqpoint{2.728382in}{2.298306in}}%
\pgfpathlineto{\pgfqpoint{2.741404in}{2.290205in}}%
\pgfpathlineto{\pgfqpoint{2.754428in}{2.282145in}}%
\pgfpathlineto{\pgfqpoint{2.767454in}{2.274127in}}%
\pgfpathlineto{\pgfqpoint{2.759050in}{2.277563in}}%
\pgfpathlineto{\pgfqpoint{2.750627in}{2.281333in}}%
\pgfpathlineto{\pgfqpoint{2.742185in}{2.285443in}}%
\pgfpathlineto{\pgfqpoint{2.733724in}{2.289903in}}%
\pgfpathlineto{\pgfqpoint{2.720664in}{2.298212in}}%
\pgfpathlineto{\pgfqpoint{2.707606in}{2.306563in}}%
\pgfpathlineto{\pgfqpoint{2.694551in}{2.314955in}}%
\pgfpathlineto{\pgfqpoint{2.681498in}{2.323389in}}%
\pgfpathlineto{\pgfqpoint{2.689993in}{2.318633in}}%
\pgfpathlineto{\pgfqpoint{2.698469in}{2.314230in}}%
\pgfpathlineto{\pgfqpoint{2.706926in}{2.310171in}}%
\pgfpathlineto{\pgfqpoint{2.715363in}{2.306449in}}%
\pgfpathclose%
\pgfusepath{fill}%
\end{pgfscope}%
\begin{pgfscope}%
\pgfpathrectangle{\pgfqpoint{1.254980in}{0.150000in}}{\pgfqpoint{5.490039in}{5.490039in}}%
\pgfusepath{clip}%
\pgfsetbuttcap%
\pgfsetroundjoin%
\definecolor{currentfill}{rgb}{0.268510,0.009605,0.335427}%
\pgfsetfillcolor{currentfill}%
\pgfsetfillopacity{0.700000}%
\pgfsetlinewidth{0.000000pt}%
\definecolor{currentstroke}{rgb}{0.000000,0.000000,0.000000}%
\pgfsetstrokecolor{currentstroke}%
\pgfsetdash{}{0pt}%
\pgfpathmoveto{\pgfqpoint{4.152161in}{1.929347in}}%
\pgfpathlineto{\pgfqpoint{4.165394in}{1.925969in}}%
\pgfpathlineto{\pgfqpoint{4.178633in}{1.922617in}}%
\pgfpathlineto{\pgfqpoint{4.191878in}{1.919290in}}%
\pgfpathlineto{\pgfqpoint{4.205130in}{1.915989in}}%
\pgfpathlineto{\pgfqpoint{4.197541in}{1.908111in}}%
\pgfpathlineto{\pgfqpoint{4.189946in}{1.900287in}}%
\pgfpathlineto{\pgfqpoint{4.182346in}{1.892518in}}%
\pgfpathlineto{\pgfqpoint{4.174740in}{1.884811in}}%
\pgfpathlineto{\pgfqpoint{4.161476in}{1.888281in}}%
\pgfpathlineto{\pgfqpoint{4.148219in}{1.891776in}}%
\pgfpathlineto{\pgfqpoint{4.134967in}{1.895298in}}%
\pgfpathlineto{\pgfqpoint{4.121722in}{1.898844in}}%
\pgfpathlineto{\pgfqpoint{4.129340in}{1.906378in}}%
\pgfpathlineto{\pgfqpoint{4.136953in}{1.913976in}}%
\pgfpathlineto{\pgfqpoint{4.144560in}{1.921633in}}%
\pgfpathlineto{\pgfqpoint{4.152161in}{1.929347in}}%
\pgfpathclose%
\pgfusepath{fill}%
\end{pgfscope}%
\begin{pgfscope}%
\pgfpathrectangle{\pgfqpoint{1.254980in}{0.150000in}}{\pgfqpoint{5.490039in}{5.490039in}}%
\pgfusepath{clip}%
\pgfsetbuttcap%
\pgfsetroundjoin%
\definecolor{currentfill}{rgb}{0.281887,0.150881,0.465405}%
\pgfsetfillcolor{currentfill}%
\pgfsetfillopacity{0.700000}%
\pgfsetlinewidth{0.000000pt}%
\definecolor{currentstroke}{rgb}{0.000000,0.000000,0.000000}%
\pgfsetstrokecolor{currentstroke}%
\pgfsetdash{}{0pt}%
\pgfpathmoveto{\pgfqpoint{2.957061in}{2.173736in}}%
\pgfpathlineto{\pgfqpoint{2.970095in}{2.166497in}}%
\pgfpathlineto{\pgfqpoint{2.983132in}{2.159295in}}%
\pgfpathlineto{\pgfqpoint{2.996173in}{2.152130in}}%
\pgfpathlineto{\pgfqpoint{3.009217in}{2.145000in}}%
\pgfpathlineto{\pgfqpoint{3.001002in}{2.146099in}}%
\pgfpathlineto{\pgfqpoint{2.992771in}{2.147486in}}%
\pgfpathlineto{\pgfqpoint{2.984526in}{2.149169in}}%
\pgfpathlineto{\pgfqpoint{2.976264in}{2.151157in}}%
\pgfpathlineto{\pgfqpoint{2.963191in}{2.158561in}}%
\pgfpathlineto{\pgfqpoint{2.950120in}{2.166002in}}%
\pgfpathlineto{\pgfqpoint{2.937053in}{2.173479in}}%
\pgfpathlineto{\pgfqpoint{2.923989in}{2.180993in}}%
\pgfpathlineto{\pgfqpoint{2.932281in}{2.178725in}}%
\pgfpathlineto{\pgfqpoint{2.940557in}{2.176765in}}%
\pgfpathlineto{\pgfqpoint{2.948817in}{2.175104in}}%
\pgfpathlineto{\pgfqpoint{2.957061in}{2.173736in}}%
\pgfpathclose%
\pgfusepath{fill}%
\end{pgfscope}%
\begin{pgfscope}%
\pgfpathrectangle{\pgfqpoint{1.254980in}{0.150000in}}{\pgfqpoint{5.490039in}{5.490039in}}%
\pgfusepath{clip}%
\pgfsetbuttcap%
\pgfsetroundjoin%
\definecolor{currentfill}{rgb}{0.281887,0.150881,0.465405}%
\pgfsetfillcolor{currentfill}%
\pgfsetfillopacity{0.700000}%
\pgfsetlinewidth{0.000000pt}%
\definecolor{currentstroke}{rgb}{0.000000,0.000000,0.000000}%
\pgfsetstrokecolor{currentstroke}%
\pgfsetdash{}{0pt}%
\pgfpathmoveto{\pgfqpoint{5.280530in}{2.180251in}}%
\pgfpathlineto{\pgfqpoint{5.294088in}{2.179353in}}%
\pgfpathlineto{\pgfqpoint{5.307655in}{2.178479in}}%
\pgfpathlineto{\pgfqpoint{5.321230in}{2.177630in}}%
\pgfpathlineto{\pgfqpoint{5.334813in}{2.176804in}}%
\pgfpathlineto{\pgfqpoint{5.327637in}{2.168938in}}%
\pgfpathlineto{\pgfqpoint{5.320453in}{2.160987in}}%
\pgfpathlineto{\pgfqpoint{5.313262in}{2.152948in}}%
\pgfpathlineto{\pgfqpoint{5.306064in}{2.144823in}}%
\pgfpathlineto{\pgfqpoint{5.292470in}{2.145676in}}%
\pgfpathlineto{\pgfqpoint{5.278884in}{2.146553in}}%
\pgfpathlineto{\pgfqpoint{5.265306in}{2.147453in}}%
\pgfpathlineto{\pgfqpoint{5.251737in}{2.148378in}}%
\pgfpathlineto{\pgfqpoint{5.258946in}{2.156471in}}%
\pgfpathlineto{\pgfqpoint{5.266147in}{2.164481in}}%
\pgfpathlineto{\pgfqpoint{5.273342in}{2.172408in}}%
\pgfpathlineto{\pgfqpoint{5.280530in}{2.180251in}}%
\pgfpathclose%
\pgfusepath{fill}%
\end{pgfscope}%
\begin{pgfscope}%
\pgfpathrectangle{\pgfqpoint{1.254980in}{0.150000in}}{\pgfqpoint{5.490039in}{5.490039in}}%
\pgfusepath{clip}%
\pgfsetbuttcap%
\pgfsetroundjoin%
\definecolor{currentfill}{rgb}{0.268510,0.009605,0.335427}%
\pgfsetfillcolor{currentfill}%
\pgfsetfillopacity{0.700000}%
\pgfsetlinewidth{0.000000pt}%
\definecolor{currentstroke}{rgb}{0.000000,0.000000,0.000000}%
\pgfsetstrokecolor{currentstroke}%
\pgfsetdash{}{0pt}%
\pgfpathmoveto{\pgfqpoint{3.796363in}{1.925337in}}%
\pgfpathlineto{\pgfqpoint{3.809520in}{1.920888in}}%
\pgfpathlineto{\pgfqpoint{3.822682in}{1.916466in}}%
\pgfpathlineto{\pgfqpoint{3.835849in}{1.912070in}}%
\pgfpathlineto{\pgfqpoint{3.849022in}{1.907702in}}%
\pgfpathlineto{\pgfqpoint{3.841292in}{1.901756in}}%
\pgfpathlineto{\pgfqpoint{3.833555in}{1.895932in}}%
\pgfpathlineto{\pgfqpoint{3.825811in}{1.890237in}}%
\pgfpathlineto{\pgfqpoint{3.818060in}{1.884676in}}%
\pgfpathlineto{\pgfqpoint{3.804871in}{1.889251in}}%
\pgfpathlineto{\pgfqpoint{3.791687in}{1.893854in}}%
\pgfpathlineto{\pgfqpoint{3.778509in}{1.898483in}}%
\pgfpathlineto{\pgfqpoint{3.765336in}{1.903140in}}%
\pgfpathlineto{\pgfqpoint{3.773104in}{1.908489in}}%
\pgfpathlineto{\pgfqpoint{3.780864in}{1.913975in}}%
\pgfpathlineto{\pgfqpoint{3.788617in}{1.919593in}}%
\pgfpathlineto{\pgfqpoint{3.796363in}{1.925337in}}%
\pgfpathclose%
\pgfusepath{fill}%
\end{pgfscope}%
\begin{pgfscope}%
\pgfpathrectangle{\pgfqpoint{1.254980in}{0.150000in}}{\pgfqpoint{5.490039in}{5.490039in}}%
\pgfusepath{clip}%
\pgfsetbuttcap%
\pgfsetroundjoin%
\definecolor{currentfill}{rgb}{0.277018,0.050344,0.375715}%
\pgfsetfillcolor{currentfill}%
\pgfsetfillopacity{0.700000}%
\pgfsetlinewidth{0.000000pt}%
\definecolor{currentstroke}{rgb}{0.000000,0.000000,0.000000}%
\pgfsetstrokecolor{currentstroke}%
\pgfsetdash{}{0pt}%
\pgfpathmoveto{\pgfqpoint{4.591377in}{1.996436in}}%
\pgfpathlineto{\pgfqpoint{4.604729in}{1.994207in}}%
\pgfpathlineto{\pgfqpoint{4.618087in}{1.992002in}}%
\pgfpathlineto{\pgfqpoint{4.631452in}{1.989822in}}%
\pgfpathlineto{\pgfqpoint{4.644825in}{1.987666in}}%
\pgfpathlineto{\pgfqpoint{4.637384in}{1.978746in}}%
\pgfpathlineto{\pgfqpoint{4.629937in}{1.969807in}}%
\pgfpathlineto{\pgfqpoint{4.622486in}{1.960851in}}%
\pgfpathlineto{\pgfqpoint{4.615029in}{1.951881in}}%
\pgfpathlineto{\pgfqpoint{4.601646in}{1.954155in}}%
\pgfpathlineto{\pgfqpoint{4.588271in}{1.956453in}}%
\pgfpathlineto{\pgfqpoint{4.574903in}{1.958776in}}%
\pgfpathlineto{\pgfqpoint{4.561542in}{1.961123in}}%
\pgfpathlineto{\pgfqpoint{4.569009in}{1.969970in}}%
\pgfpathlineto{\pgfqpoint{4.576470in}{1.978807in}}%
\pgfpathlineto{\pgfqpoint{4.583927in}{1.987629in}}%
\pgfpathlineto{\pgfqpoint{4.591377in}{1.996436in}}%
\pgfpathclose%
\pgfusepath{fill}%
\end{pgfscope}%
\begin{pgfscope}%
\pgfpathrectangle{\pgfqpoint{1.254980in}{0.150000in}}{\pgfqpoint{5.490039in}{5.490039in}}%
\pgfusepath{clip}%
\pgfsetbuttcap%
\pgfsetroundjoin%
\definecolor{currentfill}{rgb}{0.271305,0.019942,0.347269}%
\pgfsetfillcolor{currentfill}%
\pgfsetfillopacity{0.700000}%
\pgfsetlinewidth{0.000000pt}%
\definecolor{currentstroke}{rgb}{0.000000,0.000000,0.000000}%
\pgfsetstrokecolor{currentstroke}%
\pgfsetdash{}{0pt}%
\pgfpathmoveto{\pgfqpoint{3.660144in}{1.941384in}}%
\pgfpathlineto{\pgfqpoint{3.673275in}{1.936506in}}%
\pgfpathlineto{\pgfqpoint{3.686411in}{1.931656in}}%
\pgfpathlineto{\pgfqpoint{3.699552in}{1.926834in}}%
\pgfpathlineto{\pgfqpoint{3.712698in}{1.922040in}}%
\pgfpathlineto{\pgfqpoint{3.704906in}{1.917049in}}%
\pgfpathlineto{\pgfqpoint{3.697106in}{1.912208in}}%
\pgfpathlineto{\pgfqpoint{3.689297in}{1.907523in}}%
\pgfpathlineto{\pgfqpoint{3.681481in}{1.903001in}}%
\pgfpathlineto{\pgfqpoint{3.668317in}{1.908015in}}%
\pgfpathlineto{\pgfqpoint{3.655158in}{1.913057in}}%
\pgfpathlineto{\pgfqpoint{3.642004in}{1.918127in}}%
\pgfpathlineto{\pgfqpoint{3.628856in}{1.923225in}}%
\pgfpathlineto{\pgfqpoint{3.636690in}{1.927522in}}%
\pgfpathlineto{\pgfqpoint{3.644516in}{1.931985in}}%
\pgfpathlineto{\pgfqpoint{3.652334in}{1.936607in}}%
\pgfpathlineto{\pgfqpoint{3.660144in}{1.941384in}}%
\pgfpathclose%
\pgfusepath{fill}%
\end{pgfscope}%
\begin{pgfscope}%
\pgfpathrectangle{\pgfqpoint{1.254980in}{0.150000in}}{\pgfqpoint{5.490039in}{5.490039in}}%
\pgfusepath{clip}%
\pgfsetbuttcap%
\pgfsetroundjoin%
\definecolor{currentfill}{rgb}{0.276194,0.190074,0.493001}%
\pgfsetfillcolor{currentfill}%
\pgfsetfillopacity{0.700000}%
\pgfsetlinewidth{0.000000pt}%
\definecolor{currentstroke}{rgb}{0.000000,0.000000,0.000000}%
\pgfsetstrokecolor{currentstroke}%
\pgfsetdash{}{0pt}%
\pgfpathmoveto{\pgfqpoint{5.583556in}{2.256438in}}%
\pgfpathlineto{\pgfqpoint{5.597212in}{2.255921in}}%
\pgfpathlineto{\pgfqpoint{5.610877in}{2.255428in}}%
\pgfpathlineto{\pgfqpoint{5.624550in}{2.254959in}}%
\pgfpathlineto{\pgfqpoint{5.638232in}{2.254514in}}%
\pgfpathlineto{\pgfqpoint{5.631199in}{2.247741in}}%
\pgfpathlineto{\pgfqpoint{5.624158in}{2.240874in}}%
\pgfpathlineto{\pgfqpoint{5.617109in}{2.233910in}}%
\pgfpathlineto{\pgfqpoint{5.610052in}{2.226850in}}%
\pgfpathlineto{\pgfqpoint{5.596356in}{2.227283in}}%
\pgfpathlineto{\pgfqpoint{5.582670in}{2.227738in}}%
\pgfpathlineto{\pgfqpoint{5.568992in}{2.228218in}}%
\pgfpathlineto{\pgfqpoint{5.555322in}{2.228721in}}%
\pgfpathlineto{\pgfqpoint{5.562393in}{2.235789in}}%
\pgfpathlineto{\pgfqpoint{5.569455in}{2.242764in}}%
\pgfpathlineto{\pgfqpoint{5.576510in}{2.249647in}}%
\pgfpathlineto{\pgfqpoint{5.583556in}{2.256438in}}%
\pgfpathclose%
\pgfusepath{fill}%
\end{pgfscope}%
\begin{pgfscope}%
\pgfpathrectangle{\pgfqpoint{1.254980in}{0.150000in}}{\pgfqpoint{5.490039in}{5.490039in}}%
\pgfusepath{clip}%
\pgfsetbuttcap%
\pgfsetroundjoin%
\definecolor{currentfill}{rgb}{0.282327,0.094955,0.417331}%
\pgfsetfillcolor{currentfill}%
\pgfsetfillopacity{0.700000}%
\pgfsetlinewidth{0.000000pt}%
\definecolor{currentstroke}{rgb}{0.000000,0.000000,0.000000}%
\pgfsetstrokecolor{currentstroke}%
\pgfsetdash{}{0pt}%
\pgfpathmoveto{\pgfqpoint{4.894386in}{2.071701in}}%
\pgfpathlineto{\pgfqpoint{4.907827in}{2.070138in}}%
\pgfpathlineto{\pgfqpoint{4.921275in}{2.068598in}}%
\pgfpathlineto{\pgfqpoint{4.934731in}{2.067083in}}%
\pgfpathlineto{\pgfqpoint{4.948195in}{2.065592in}}%
\pgfpathlineto{\pgfqpoint{4.940860in}{2.056780in}}%
\pgfpathlineto{\pgfqpoint{4.933520in}{2.047912in}}%
\pgfpathlineto{\pgfqpoint{4.926173in}{2.038988in}}%
\pgfpathlineto{\pgfqpoint{4.918820in}{2.030009in}}%
\pgfpathlineto{\pgfqpoint{4.905347in}{2.031580in}}%
\pgfpathlineto{\pgfqpoint{4.891881in}{2.033175in}}%
\pgfpathlineto{\pgfqpoint{4.878423in}{2.034793in}}%
\pgfpathlineto{\pgfqpoint{4.864973in}{2.036436in}}%
\pgfpathlineto{\pgfqpoint{4.872335in}{2.045330in}}%
\pgfpathlineto{\pgfqpoint{4.879691in}{2.054173in}}%
\pgfpathlineto{\pgfqpoint{4.887042in}{2.062964in}}%
\pgfpathlineto{\pgfqpoint{4.894386in}{2.071701in}}%
\pgfpathclose%
\pgfusepath{fill}%
\end{pgfscope}%
\begin{pgfscope}%
\pgfpathrectangle{\pgfqpoint{1.254980in}{0.150000in}}{\pgfqpoint{5.490039in}{5.490039in}}%
\pgfusepath{clip}%
\pgfsetbuttcap%
\pgfsetroundjoin%
\definecolor{currentfill}{rgb}{0.267004,0.004874,0.329415}%
\pgfsetfillcolor{currentfill}%
\pgfsetfillopacity{0.700000}%
\pgfsetlinewidth{0.000000pt}%
\definecolor{currentstroke}{rgb}{0.000000,0.000000,0.000000}%
\pgfsetstrokecolor{currentstroke}%
\pgfsetdash{}{0pt}%
\pgfpathmoveto{\pgfqpoint{3.932564in}{1.916194in}}%
\pgfpathlineto{\pgfqpoint{3.945751in}{1.912153in}}%
\pgfpathlineto{\pgfqpoint{3.958943in}{1.908139in}}%
\pgfpathlineto{\pgfqpoint{3.972141in}{1.904152in}}%
\pgfpathlineto{\pgfqpoint{3.985345in}{1.900190in}}%
\pgfpathlineto{\pgfqpoint{3.977671in}{1.893421in}}%
\pgfpathlineto{\pgfqpoint{3.969990in}{1.886749in}}%
\pgfpathlineto{\pgfqpoint{3.962304in}{1.880179in}}%
\pgfpathlineto{\pgfqpoint{3.954610in}{1.873716in}}%
\pgfpathlineto{\pgfqpoint{3.941392in}{1.877872in}}%
\pgfpathlineto{\pgfqpoint{3.928179in}{1.882054in}}%
\pgfpathlineto{\pgfqpoint{3.914972in}{1.886262in}}%
\pgfpathlineto{\pgfqpoint{3.901771in}{1.890497in}}%
\pgfpathlineto{\pgfqpoint{3.909479in}{1.896761in}}%
\pgfpathlineto{\pgfqpoint{3.917180in}{1.903135in}}%
\pgfpathlineto{\pgfqpoint{3.924875in}{1.909614in}}%
\pgfpathlineto{\pgfqpoint{3.932564in}{1.916194in}}%
\pgfpathclose%
\pgfusepath{fill}%
\end{pgfscope}%
\begin{pgfscope}%
\pgfpathrectangle{\pgfqpoint{1.254980in}{0.150000in}}{\pgfqpoint{5.490039in}{5.490039in}}%
\pgfusepath{clip}%
\pgfsetbuttcap%
\pgfsetroundjoin%
\definecolor{currentfill}{rgb}{0.273809,0.031497,0.358853}%
\pgfsetfillcolor{currentfill}%
\pgfsetfillopacity{0.700000}%
\pgfsetlinewidth{0.000000pt}%
\definecolor{currentstroke}{rgb}{0.000000,0.000000,0.000000}%
\pgfsetstrokecolor{currentstroke}%
\pgfsetdash{}{0pt}%
\pgfpathmoveto{\pgfqpoint{3.523844in}{1.965033in}}%
\pgfpathlineto{\pgfqpoint{3.536954in}{1.959707in}}%
\pgfpathlineto{\pgfqpoint{3.550068in}{1.954409in}}%
\pgfpathlineto{\pgfqpoint{3.563187in}{1.949140in}}%
\pgfpathlineto{\pgfqpoint{3.576311in}{1.943900in}}%
\pgfpathlineto{\pgfqpoint{3.568448in}{1.940002in}}%
\pgfpathlineto{\pgfqpoint{3.560577in}{1.936284in}}%
\pgfpathlineto{\pgfqpoint{3.552697in}{1.932753in}}%
\pgfpathlineto{\pgfqpoint{3.544807in}{1.929413in}}%
\pgfpathlineto{\pgfqpoint{3.531664in}{1.934887in}}%
\pgfpathlineto{\pgfqpoint{3.518525in}{1.940389in}}%
\pgfpathlineto{\pgfqpoint{3.505390in}{1.945920in}}%
\pgfpathlineto{\pgfqpoint{3.492261in}{1.951480in}}%
\pgfpathlineto{\pgfqpoint{3.500171in}{1.954581in}}%
\pgfpathlineto{\pgfqpoint{3.508071in}{1.957878in}}%
\pgfpathlineto{\pgfqpoint{3.515963in}{1.961364in}}%
\pgfpathlineto{\pgfqpoint{3.523844in}{1.965033in}}%
\pgfpathclose%
\pgfusepath{fill}%
\end{pgfscope}%
\begin{pgfscope}%
\pgfpathrectangle{\pgfqpoint{1.254980in}{0.150000in}}{\pgfqpoint{5.490039in}{5.490039in}}%
\pgfusepath{clip}%
\pgfsetbuttcap%
\pgfsetroundjoin%
\definecolor{currentfill}{rgb}{0.241237,0.296485,0.539709}%
\pgfsetfillcolor{currentfill}%
\pgfsetfillopacity{0.700000}%
\pgfsetlinewidth{0.000000pt}%
\definecolor{currentstroke}{rgb}{0.000000,0.000000,0.000000}%
\pgfsetstrokecolor{currentstroke}%
\pgfsetdash{}{0pt}%
\pgfpathmoveto{\pgfqpoint{2.472925in}{2.464397in}}%
\pgfpathlineto{\pgfqpoint{2.485947in}{2.455232in}}%
\pgfpathlineto{\pgfqpoint{2.498971in}{2.446117in}}%
\pgfpathlineto{\pgfqpoint{2.511997in}{2.437049in}}%
\pgfpathlineto{\pgfqpoint{2.525024in}{2.428030in}}%
\pgfpathlineto{\pgfqpoint{2.516398in}{2.434051in}}%
\pgfpathlineto{\pgfqpoint{2.507749in}{2.440451in}}%
\pgfpathlineto{\pgfqpoint{2.499078in}{2.447240in}}%
\pgfpathlineto{\pgfqpoint{2.490383in}{2.454425in}}%
\pgfpathlineto{\pgfqpoint{2.477317in}{2.463752in}}%
\pgfpathlineto{\pgfqpoint{2.464252in}{2.473128in}}%
\pgfpathlineto{\pgfqpoint{2.451189in}{2.482552in}}%
\pgfpathlineto{\pgfqpoint{2.438128in}{2.492025in}}%
\pgfpathlineto{\pgfqpoint{2.446863in}{2.484526in}}%
\pgfpathlineto{\pgfqpoint{2.455573in}{2.477427in}}%
\pgfpathlineto{\pgfqpoint{2.464261in}{2.470720in}}%
\pgfpathlineto{\pgfqpoint{2.472925in}{2.464397in}}%
\pgfpathclose%
\pgfusepath{fill}%
\end{pgfscope}%
\begin{pgfscope}%
\pgfpathrectangle{\pgfqpoint{1.254980in}{0.150000in}}{\pgfqpoint{5.490039in}{5.490039in}}%
\pgfusepath{clip}%
\pgfsetbuttcap%
\pgfsetroundjoin%
\definecolor{currentfill}{rgb}{0.269944,0.014625,0.341379}%
\pgfsetfillcolor{currentfill}%
\pgfsetfillopacity{0.700000}%
\pgfsetlinewidth{0.000000pt}%
\definecolor{currentstroke}{rgb}{0.000000,0.000000,0.000000}%
\pgfsetstrokecolor{currentstroke}%
\pgfsetdash{}{0pt}%
\pgfpathmoveto{\pgfqpoint{4.288457in}{1.935621in}}%
\pgfpathlineto{\pgfqpoint{4.301730in}{1.932602in}}%
\pgfpathlineto{\pgfqpoint{4.315009in}{1.929609in}}%
\pgfpathlineto{\pgfqpoint{4.328294in}{1.926641in}}%
\pgfpathlineto{\pgfqpoint{4.341587in}{1.923698in}}%
\pgfpathlineto{\pgfqpoint{4.334042in}{1.915345in}}%
\pgfpathlineto{\pgfqpoint{4.326492in}{1.907024in}}%
\pgfpathlineto{\pgfqpoint{4.318936in}{1.898737in}}%
\pgfpathlineto{\pgfqpoint{4.311376in}{1.890489in}}%
\pgfpathlineto{\pgfqpoint{4.298072in}{1.893588in}}%
\pgfpathlineto{\pgfqpoint{4.284775in}{1.896713in}}%
\pgfpathlineto{\pgfqpoint{4.271485in}{1.899863in}}%
\pgfpathlineto{\pgfqpoint{4.258201in}{1.903038in}}%
\pgfpathlineto{\pgfqpoint{4.265773in}{1.911125in}}%
\pgfpathlineto{\pgfqpoint{4.273340in}{1.919254in}}%
\pgfpathlineto{\pgfqpoint{4.280901in}{1.927420in}}%
\pgfpathlineto{\pgfqpoint{4.288457in}{1.935621in}}%
\pgfpathclose%
\pgfusepath{fill}%
\end{pgfscope}%
\begin{pgfscope}%
\pgfpathrectangle{\pgfqpoint{1.254980in}{0.150000in}}{\pgfqpoint{5.490039in}{5.490039in}}%
\pgfusepath{clip}%
\pgfsetbuttcap%
\pgfsetroundjoin%
\definecolor{currentfill}{rgb}{0.282623,0.140926,0.457517}%
\pgfsetfillcolor{currentfill}%
\pgfsetfillopacity{0.700000}%
\pgfsetlinewidth{0.000000pt}%
\definecolor{currentstroke}{rgb}{0.000000,0.000000,0.000000}%
\pgfsetstrokecolor{currentstroke}%
\pgfsetdash{}{0pt}%
\pgfpathmoveto{\pgfqpoint{5.197541in}{2.152314in}}%
\pgfpathlineto{\pgfqpoint{5.211078in}{2.151294in}}%
\pgfpathlineto{\pgfqpoint{5.224622in}{2.150298in}}%
\pgfpathlineto{\pgfqpoint{5.238175in}{2.149326in}}%
\pgfpathlineto{\pgfqpoint{5.251737in}{2.148378in}}%
\pgfpathlineto{\pgfqpoint{5.244521in}{2.140202in}}%
\pgfpathlineto{\pgfqpoint{5.237298in}{2.131944in}}%
\pgfpathlineto{\pgfqpoint{5.230068in}{2.123604in}}%
\pgfpathlineto{\pgfqpoint{5.222831in}{2.115183in}}%
\pgfpathlineto{\pgfqpoint{5.209259in}{2.116172in}}%
\pgfpathlineto{\pgfqpoint{5.195695in}{2.117184in}}%
\pgfpathlineto{\pgfqpoint{5.182140in}{2.118220in}}%
\pgfpathlineto{\pgfqpoint{5.168593in}{2.119280in}}%
\pgfpathlineto{\pgfqpoint{5.175840in}{2.127656in}}%
\pgfpathlineto{\pgfqpoint{5.183080in}{2.135954in}}%
\pgfpathlineto{\pgfqpoint{5.190314in}{2.144173in}}%
\pgfpathlineto{\pgfqpoint{5.197541in}{2.152314in}}%
\pgfpathclose%
\pgfusepath{fill}%
\end{pgfscope}%
\begin{pgfscope}%
\pgfpathrectangle{\pgfqpoint{1.254980in}{0.150000in}}{\pgfqpoint{5.490039in}{5.490039in}}%
\pgfusepath{clip}%
\pgfsetbuttcap%
\pgfsetroundjoin%
\definecolor{currentfill}{rgb}{0.270595,0.214069,0.507052}%
\pgfsetfillcolor{currentfill}%
\pgfsetfillopacity{0.700000}%
\pgfsetlinewidth{0.000000pt}%
\definecolor{currentstroke}{rgb}{0.000000,0.000000,0.000000}%
\pgfsetstrokecolor{currentstroke}%
\pgfsetdash{}{0pt}%
\pgfpathmoveto{\pgfqpoint{5.803734in}{2.302181in}}%
\pgfpathlineto{\pgfqpoint{5.817466in}{2.301882in}}%
\pgfpathlineto{\pgfqpoint{5.831206in}{2.301607in}}%
\pgfpathlineto{\pgfqpoint{5.844956in}{2.301355in}}%
\pgfpathlineto{\pgfqpoint{5.838032in}{2.295420in}}%
\pgfpathlineto{\pgfqpoint{5.831099in}{2.289392in}}%
\pgfpathlineto{\pgfqpoint{5.824157in}{2.283269in}}%
\pgfpathlineto{\pgfqpoint{5.817207in}{2.277050in}}%
\pgfpathlineto{\pgfqpoint{5.803442in}{2.277262in}}%
\pgfpathlineto{\pgfqpoint{5.789685in}{2.277497in}}%
\pgfpathlineto{\pgfqpoint{5.775938in}{2.277756in}}%
\pgfpathlineto{\pgfqpoint{5.782900in}{2.284001in}}%
\pgfpathlineto{\pgfqpoint{5.789853in}{2.290153in}}%
\pgfpathlineto{\pgfqpoint{5.796798in}{2.296212in}}%
\pgfpathlineto{\pgfqpoint{5.803734in}{2.302181in}}%
\pgfpathclose%
\pgfusepath{fill}%
\end{pgfscope}%
\begin{pgfscope}%
\pgfpathrectangle{\pgfqpoint{1.254980in}{0.150000in}}{\pgfqpoint{5.490039in}{5.490039in}}%
\pgfusepath{clip}%
\pgfsetbuttcap%
\pgfsetroundjoin%
\definecolor{currentfill}{rgb}{0.273006,0.204520,0.501721}%
\pgfsetfillcolor{currentfill}%
\pgfsetfillopacity{0.700000}%
\pgfsetlinewidth{0.000000pt}%
\definecolor{currentstroke}{rgb}{0.000000,0.000000,0.000000}%
\pgfsetstrokecolor{currentstroke}%
\pgfsetdash{}{0pt}%
\pgfpathmoveto{\pgfqpoint{2.767454in}{2.274127in}}%
\pgfpathlineto{\pgfqpoint{2.780483in}{2.266149in}}%
\pgfpathlineto{\pgfqpoint{2.793515in}{2.258212in}}%
\pgfpathlineto{\pgfqpoint{2.806550in}{2.250315in}}%
\pgfpathlineto{\pgfqpoint{2.819587in}{2.242458in}}%
\pgfpathlineto{\pgfqpoint{2.811215in}{2.245609in}}%
\pgfpathlineto{\pgfqpoint{2.802826in}{2.249090in}}%
\pgfpathlineto{\pgfqpoint{2.794418in}{2.252908in}}%
\pgfpathlineto{\pgfqpoint{2.785991in}{2.257071in}}%
\pgfpathlineto{\pgfqpoint{2.772920in}{2.265219in}}%
\pgfpathlineto{\pgfqpoint{2.759853in}{2.273406in}}%
\pgfpathlineto{\pgfqpoint{2.746787in}{2.281634in}}%
\pgfpathlineto{\pgfqpoint{2.733724in}{2.289903in}}%
\pgfpathlineto{\pgfqpoint{2.742185in}{2.285443in}}%
\pgfpathlineto{\pgfqpoint{2.750627in}{2.281333in}}%
\pgfpathlineto{\pgfqpoint{2.759050in}{2.277563in}}%
\pgfpathlineto{\pgfqpoint{2.767454in}{2.274127in}}%
\pgfpathclose%
\pgfusepath{fill}%
\end{pgfscope}%
\begin{pgfscope}%
\pgfpathrectangle{\pgfqpoint{1.254980in}{0.150000in}}{\pgfqpoint{5.490039in}{5.490039in}}%
\pgfusepath{clip}%
\pgfsetbuttcap%
\pgfsetroundjoin%
\definecolor{currentfill}{rgb}{0.282327,0.094955,0.417331}%
\pgfsetfillcolor{currentfill}%
\pgfsetfillopacity{0.700000}%
\pgfsetlinewidth{0.000000pt}%
\definecolor{currentstroke}{rgb}{0.000000,0.000000,0.000000}%
\pgfsetstrokecolor{currentstroke}%
\pgfsetdash{}{0pt}%
\pgfpathmoveto{\pgfqpoint{3.198419in}{2.063648in}}%
\pgfpathlineto{\pgfqpoint{3.211485in}{2.057224in}}%
\pgfpathlineto{\pgfqpoint{3.224554in}{2.050833in}}%
\pgfpathlineto{\pgfqpoint{3.237627in}{2.044475in}}%
\pgfpathlineto{\pgfqpoint{3.250705in}{2.038149in}}%
\pgfpathlineto{\pgfqpoint{3.242650in}{2.037139in}}%
\pgfpathlineto{\pgfqpoint{3.234583in}{2.036377in}}%
\pgfpathlineto{\pgfqpoint{3.226503in}{2.035869in}}%
\pgfpathlineto{\pgfqpoint{3.218410in}{2.035621in}}%
\pgfpathlineto{\pgfqpoint{3.205307in}{2.042208in}}%
\pgfpathlineto{\pgfqpoint{3.192208in}{2.048827in}}%
\pgfpathlineto{\pgfqpoint{3.179113in}{2.055479in}}%
\pgfpathlineto{\pgfqpoint{3.166022in}{2.062163in}}%
\pgfpathlineto{\pgfqpoint{3.174141in}{2.062144in}}%
\pgfpathlineto{\pgfqpoint{3.182247in}{2.062390in}}%
\pgfpathlineto{\pgfqpoint{3.190339in}{2.062894in}}%
\pgfpathlineto{\pgfqpoint{3.198419in}{2.063648in}}%
\pgfpathclose%
\pgfusepath{fill}%
\end{pgfscope}%
\begin{pgfscope}%
\pgfpathrectangle{\pgfqpoint{1.254980in}{0.150000in}}{\pgfqpoint{5.490039in}{5.490039in}}%
\pgfusepath{clip}%
\pgfsetbuttcap%
\pgfsetroundjoin%
\definecolor{currentfill}{rgb}{0.267004,0.004874,0.329415}%
\pgfsetfillcolor{currentfill}%
\pgfsetfillopacity{0.700000}%
\pgfsetlinewidth{0.000000pt}%
\definecolor{currentstroke}{rgb}{0.000000,0.000000,0.000000}%
\pgfsetstrokecolor{currentstroke}%
\pgfsetdash{}{0pt}%
\pgfpathmoveto{\pgfqpoint{4.068803in}{1.913288in}}%
\pgfpathlineto{\pgfqpoint{4.082024in}{1.909638in}}%
\pgfpathlineto{\pgfqpoint{4.095251in}{1.906015in}}%
\pgfpathlineto{\pgfqpoint{4.108483in}{1.902417in}}%
\pgfpathlineto{\pgfqpoint{4.121722in}{1.898844in}}%
\pgfpathlineto{\pgfqpoint{4.114098in}{1.891379in}}%
\pgfpathlineto{\pgfqpoint{4.106469in}{1.883986in}}%
\pgfpathlineto{\pgfqpoint{4.098834in}{1.876670in}}%
\pgfpathlineto{\pgfqpoint{4.091193in}{1.869435in}}%
\pgfpathlineto{\pgfqpoint{4.077941in}{1.873189in}}%
\pgfpathlineto{\pgfqpoint{4.064695in}{1.876969in}}%
\pgfpathlineto{\pgfqpoint{4.051455in}{1.880774in}}%
\pgfpathlineto{\pgfqpoint{4.038221in}{1.884606in}}%
\pgfpathlineto{\pgfqpoint{4.045875in}{1.891654in}}%
\pgfpathlineto{\pgfqpoint{4.053524in}{1.898786in}}%
\pgfpathlineto{\pgfqpoint{4.061167in}{1.905999in}}%
\pgfpathlineto{\pgfqpoint{4.068803in}{1.913288in}}%
\pgfpathclose%
\pgfusepath{fill}%
\end{pgfscope}%
\begin{pgfscope}%
\pgfpathrectangle{\pgfqpoint{1.254980in}{0.150000in}}{\pgfqpoint{5.490039in}{5.490039in}}%
\pgfusepath{clip}%
\pgfsetbuttcap%
\pgfsetroundjoin%
\definecolor{currentfill}{rgb}{0.274952,0.037752,0.364543}%
\pgfsetfillcolor{currentfill}%
\pgfsetfillopacity{0.700000}%
\pgfsetlinewidth{0.000000pt}%
\definecolor{currentstroke}{rgb}{0.000000,0.000000,0.000000}%
\pgfsetstrokecolor{currentstroke}%
\pgfsetdash{}{0pt}%
\pgfpathmoveto{\pgfqpoint{4.508169in}{1.970758in}}%
\pgfpathlineto{\pgfqpoint{4.521502in}{1.968312in}}%
\pgfpathlineto{\pgfqpoint{4.534841in}{1.965891in}}%
\pgfpathlineto{\pgfqpoint{4.548188in}{1.963495in}}%
\pgfpathlineto{\pgfqpoint{4.561542in}{1.961123in}}%
\pgfpathlineto{\pgfqpoint{4.554070in}{1.952268in}}%
\pgfpathlineto{\pgfqpoint{4.546593in}{1.943408in}}%
\pgfpathlineto{\pgfqpoint{4.539110in}{1.934546in}}%
\pgfpathlineto{\pgfqpoint{4.531623in}{1.925684in}}%
\pgfpathlineto{\pgfqpoint{4.518259in}{1.928187in}}%
\pgfpathlineto{\pgfqpoint{4.504902in}{1.930714in}}%
\pgfpathlineto{\pgfqpoint{4.491552in}{1.933266in}}%
\pgfpathlineto{\pgfqpoint{4.478209in}{1.935843in}}%
\pgfpathlineto{\pgfqpoint{4.485707in}{1.944568in}}%
\pgfpathlineto{\pgfqpoint{4.493200in}{1.953298in}}%
\pgfpathlineto{\pgfqpoint{4.500687in}{1.962029in}}%
\pgfpathlineto{\pgfqpoint{4.508169in}{1.970758in}}%
\pgfpathclose%
\pgfusepath{fill}%
\end{pgfscope}%
\begin{pgfscope}%
\pgfpathrectangle{\pgfqpoint{1.254980in}{0.150000in}}{\pgfqpoint{5.490039in}{5.490039in}}%
\pgfusepath{clip}%
\pgfsetbuttcap%
\pgfsetroundjoin%
\definecolor{currentfill}{rgb}{0.280894,0.078907,0.402329}%
\pgfsetfillcolor{currentfill}%
\pgfsetfillopacity{0.700000}%
\pgfsetlinewidth{0.000000pt}%
\definecolor{currentstroke}{rgb}{0.000000,0.000000,0.000000}%
\pgfsetstrokecolor{currentstroke}%
\pgfsetdash{}{0pt}%
\pgfpathmoveto{\pgfqpoint{4.811247in}{2.043249in}}%
\pgfpathlineto{\pgfqpoint{4.824667in}{2.041510in}}%
\pgfpathlineto{\pgfqpoint{4.838095in}{2.039794in}}%
\pgfpathlineto{\pgfqpoint{4.851530in}{2.038103in}}%
\pgfpathlineto{\pgfqpoint{4.864973in}{2.036436in}}%
\pgfpathlineto{\pgfqpoint{4.857605in}{2.027494in}}%
\pgfpathlineto{\pgfqpoint{4.850231in}{2.018504in}}%
\pgfpathlineto{\pgfqpoint{4.842851in}{2.009469in}}%
\pgfpathlineto{\pgfqpoint{4.835466in}{2.000391in}}%
\pgfpathlineto{\pgfqpoint{4.822014in}{2.002150in}}%
\pgfpathlineto{\pgfqpoint{4.808569in}{2.003934in}}%
\pgfpathlineto{\pgfqpoint{4.795132in}{2.005742in}}%
\pgfpathlineto{\pgfqpoint{4.781703in}{2.007573in}}%
\pgfpathlineto{\pgfqpoint{4.789098in}{2.016554in}}%
\pgfpathlineto{\pgfqpoint{4.796487in}{2.025495in}}%
\pgfpathlineto{\pgfqpoint{4.803870in}{2.034394in}}%
\pgfpathlineto{\pgfqpoint{4.811247in}{2.043249in}}%
\pgfpathclose%
\pgfusepath{fill}%
\end{pgfscope}%
\begin{pgfscope}%
\pgfpathrectangle{\pgfqpoint{1.254980in}{0.150000in}}{\pgfqpoint{5.490039in}{5.490039in}}%
\pgfusepath{clip}%
\pgfsetbuttcap%
\pgfsetroundjoin%
\definecolor{currentfill}{rgb}{0.278012,0.180367,0.486697}%
\pgfsetfillcolor{currentfill}%
\pgfsetfillopacity{0.700000}%
\pgfsetlinewidth{0.000000pt}%
\definecolor{currentstroke}{rgb}{0.000000,0.000000,0.000000}%
\pgfsetstrokecolor{currentstroke}%
\pgfsetdash{}{0pt}%
\pgfpathmoveto{\pgfqpoint{5.500731in}{2.230972in}}%
\pgfpathlineto{\pgfqpoint{5.514366in}{2.230374in}}%
\pgfpathlineto{\pgfqpoint{5.528009in}{2.229799in}}%
\pgfpathlineto{\pgfqpoint{5.541661in}{2.229248in}}%
\pgfpathlineto{\pgfqpoint{5.555322in}{2.228721in}}%
\pgfpathlineto{\pgfqpoint{5.548244in}{2.221559in}}%
\pgfpathlineto{\pgfqpoint{5.541157in}{2.214302in}}%
\pgfpathlineto{\pgfqpoint{5.534063in}{2.206950in}}%
\pgfpathlineto{\pgfqpoint{5.526960in}{2.199502in}}%
\pgfpathlineto{\pgfqpoint{5.513287in}{2.200029in}}%
\pgfpathlineto{\pgfqpoint{5.499622in}{2.200580in}}%
\pgfpathlineto{\pgfqpoint{5.485966in}{2.201155in}}%
\pgfpathlineto{\pgfqpoint{5.472318in}{2.201753in}}%
\pgfpathlineto{\pgfqpoint{5.479433in}{2.209196in}}%
\pgfpathlineto{\pgfqpoint{5.486540in}{2.216546in}}%
\pgfpathlineto{\pgfqpoint{5.493640in}{2.223805in}}%
\pgfpathlineto{\pgfqpoint{5.500731in}{2.230972in}}%
\pgfpathclose%
\pgfusepath{fill}%
\end{pgfscope}%
\begin{pgfscope}%
\pgfpathrectangle{\pgfqpoint{1.254980in}{0.150000in}}{\pgfqpoint{5.490039in}{5.490039in}}%
\pgfusepath{clip}%
\pgfsetbuttcap%
\pgfsetroundjoin%
\definecolor{currentfill}{rgb}{0.282623,0.140926,0.457517}%
\pgfsetfillcolor{currentfill}%
\pgfsetfillopacity{0.700000}%
\pgfsetlinewidth{0.000000pt}%
\definecolor{currentstroke}{rgb}{0.000000,0.000000,0.000000}%
\pgfsetstrokecolor{currentstroke}%
\pgfsetdash{}{0pt}%
\pgfpathmoveto{\pgfqpoint{3.009217in}{2.145000in}}%
\pgfpathlineto{\pgfqpoint{3.022265in}{2.137907in}}%
\pgfpathlineto{\pgfqpoint{3.035316in}{2.130848in}}%
\pgfpathlineto{\pgfqpoint{3.048370in}{2.123825in}}%
\pgfpathlineto{\pgfqpoint{3.061428in}{2.116837in}}%
\pgfpathlineto{\pgfqpoint{3.053241in}{2.117666in}}%
\pgfpathlineto{\pgfqpoint{3.045039in}{2.118780in}}%
\pgfpathlineto{\pgfqpoint{3.036823in}{2.120187in}}%
\pgfpathlineto{\pgfqpoint{3.028591in}{2.121894in}}%
\pgfpathlineto{\pgfqpoint{3.015504in}{2.129157in}}%
\pgfpathlineto{\pgfqpoint{3.002421in}{2.136455in}}%
\pgfpathlineto{\pgfqpoint{2.989341in}{2.143788in}}%
\pgfpathlineto{\pgfqpoint{2.976264in}{2.151157in}}%
\pgfpathlineto{\pgfqpoint{2.984526in}{2.149169in}}%
\pgfpathlineto{\pgfqpoint{2.992771in}{2.147486in}}%
\pgfpathlineto{\pgfqpoint{3.001002in}{2.146099in}}%
\pgfpathlineto{\pgfqpoint{3.009217in}{2.145000in}}%
\pgfpathclose%
\pgfusepath{fill}%
\end{pgfscope}%
\begin{pgfscope}%
\pgfpathrectangle{\pgfqpoint{1.254980in}{0.150000in}}{\pgfqpoint{5.490039in}{5.490039in}}%
\pgfusepath{clip}%
\pgfsetbuttcap%
\pgfsetroundjoin%
\definecolor{currentfill}{rgb}{0.277941,0.056324,0.381191}%
\pgfsetfillcolor{currentfill}%
\pgfsetfillopacity{0.700000}%
\pgfsetlinewidth{0.000000pt}%
\definecolor{currentstroke}{rgb}{0.000000,0.000000,0.000000}%
\pgfsetstrokecolor{currentstroke}%
\pgfsetdash{}{0pt}%
\pgfpathmoveto{\pgfqpoint{3.387391in}{1.997027in}}%
\pgfpathlineto{\pgfqpoint{3.400484in}{1.991229in}}%
\pgfpathlineto{\pgfqpoint{3.413581in}{1.985461in}}%
\pgfpathlineto{\pgfqpoint{3.426683in}{1.979723in}}%
\pgfpathlineto{\pgfqpoint{3.439789in}{1.974015in}}%
\pgfpathlineto{\pgfqpoint{3.431848in}{1.971357in}}%
\pgfpathlineto{\pgfqpoint{3.423897in}{1.968910in}}%
\pgfpathlineto{\pgfqpoint{3.415935in}{1.966680in}}%
\pgfpathlineto{\pgfqpoint{3.407962in}{1.964675in}}%
\pgfpathlineto{\pgfqpoint{3.394834in}{1.970630in}}%
\pgfpathlineto{\pgfqpoint{3.381709in}{1.976614in}}%
\pgfpathlineto{\pgfqpoint{3.368590in}{1.982629in}}%
\pgfpathlineto{\pgfqpoint{3.355474in}{1.988674in}}%
\pgfpathlineto{\pgfqpoint{3.363470in}{1.990428in}}%
\pgfpathlineto{\pgfqpoint{3.371454in}{1.992409in}}%
\pgfpathlineto{\pgfqpoint{3.379428in}{1.994611in}}%
\pgfpathlineto{\pgfqpoint{3.387391in}{1.997027in}}%
\pgfpathclose%
\pgfusepath{fill}%
\end{pgfscope}%
\begin{pgfscope}%
\pgfpathrectangle{\pgfqpoint{1.254980in}{0.150000in}}{\pgfqpoint{5.490039in}{5.490039in}}%
\pgfusepath{clip}%
\pgfsetbuttcap%
\pgfsetroundjoin%
\definecolor{currentfill}{rgb}{0.283187,0.125848,0.444960}%
\pgfsetfillcolor{currentfill}%
\pgfsetfillopacity{0.700000}%
\pgfsetlinewidth{0.000000pt}%
\definecolor{currentstroke}{rgb}{0.000000,0.000000,0.000000}%
\pgfsetstrokecolor{currentstroke}%
\pgfsetdash{}{0pt}%
\pgfpathmoveto{\pgfqpoint{5.114485in}{2.123759in}}%
\pgfpathlineto{\pgfqpoint{5.127999in}{2.122603in}}%
\pgfpathlineto{\pgfqpoint{5.141522in}{2.121471in}}%
\pgfpathlineto{\pgfqpoint{5.155053in}{2.120364in}}%
\pgfpathlineto{\pgfqpoint{5.168593in}{2.119280in}}%
\pgfpathlineto{\pgfqpoint{5.161339in}{2.110827in}}%
\pgfpathlineto{\pgfqpoint{5.154078in}{2.102297in}}%
\pgfpathlineto{\pgfqpoint{5.146811in}{2.093692in}}%
\pgfpathlineto{\pgfqpoint{5.139537in}{2.085012in}}%
\pgfpathlineto{\pgfqpoint{5.125987in}{2.086149in}}%
\pgfpathlineto{\pgfqpoint{5.112446in}{2.087310in}}%
\pgfpathlineto{\pgfqpoint{5.098913in}{2.088495in}}%
\pgfpathlineto{\pgfqpoint{5.085388in}{2.089704in}}%
\pgfpathlineto{\pgfqpoint{5.092672in}{2.098326in}}%
\pgfpathlineto{\pgfqpoint{5.099949in}{2.106876in}}%
\pgfpathlineto{\pgfqpoint{5.107220in}{2.115354in}}%
\pgfpathlineto{\pgfqpoint{5.114485in}{2.123759in}}%
\pgfpathclose%
\pgfusepath{fill}%
\end{pgfscope}%
\begin{pgfscope}%
\pgfpathrectangle{\pgfqpoint{1.254980in}{0.150000in}}{\pgfqpoint{5.490039in}{5.490039in}}%
\pgfusepath{clip}%
\pgfsetbuttcap%
\pgfsetroundjoin%
\definecolor{currentfill}{rgb}{0.279566,0.067836,0.391917}%
\pgfsetfillcolor{currentfill}%
\pgfsetfillopacity{0.700000}%
\pgfsetlinewidth{0.000000pt}%
\definecolor{currentstroke}{rgb}{0.000000,0.000000,0.000000}%
\pgfsetstrokecolor{currentstroke}%
\pgfsetdash{}{0pt}%
\pgfpathmoveto{\pgfqpoint{4.728060in}{2.015143in}}%
\pgfpathlineto{\pgfqpoint{4.741459in}{2.013214in}}%
\pgfpathlineto{\pgfqpoint{4.754866in}{2.011310in}}%
\pgfpathlineto{\pgfqpoint{4.768281in}{2.009429in}}%
\pgfpathlineto{\pgfqpoint{4.781703in}{2.007573in}}%
\pgfpathlineto{\pgfqpoint{4.774303in}{1.998555in}}%
\pgfpathlineto{\pgfqpoint{4.766897in}{1.989501in}}%
\pgfpathlineto{\pgfqpoint{4.759486in}{1.980414in}}%
\pgfpathlineto{\pgfqpoint{4.752069in}{1.971295in}}%
\pgfpathlineto{\pgfqpoint{4.738638in}{1.973257in}}%
\pgfpathlineto{\pgfqpoint{4.725214in}{1.975243in}}%
\pgfpathlineto{\pgfqpoint{4.711797in}{1.977252in}}%
\pgfpathlineto{\pgfqpoint{4.698388in}{1.979287in}}%
\pgfpathlineto{\pgfqpoint{4.705814in}{1.988294in}}%
\pgfpathlineto{\pgfqpoint{4.713235in}{1.997275in}}%
\pgfpathlineto{\pgfqpoint{4.720650in}{2.006225in}}%
\pgfpathlineto{\pgfqpoint{4.728060in}{2.015143in}}%
\pgfpathclose%
\pgfusepath{fill}%
\end{pgfscope}%
\begin{pgfscope}%
\pgfpathrectangle{\pgfqpoint{1.254980in}{0.150000in}}{\pgfqpoint{5.490039in}{5.490039in}}%
\pgfusepath{clip}%
\pgfsetbuttcap%
\pgfsetroundjoin%
\definecolor{currentfill}{rgb}{0.246811,0.283237,0.535941}%
\pgfsetfillcolor{currentfill}%
\pgfsetfillopacity{0.700000}%
\pgfsetlinewidth{0.000000pt}%
\definecolor{currentstroke}{rgb}{0.000000,0.000000,0.000000}%
\pgfsetstrokecolor{currentstroke}%
\pgfsetdash{}{0pt}%
\pgfpathmoveto{\pgfqpoint{2.525024in}{2.428030in}}%
\pgfpathlineto{\pgfqpoint{2.538053in}{2.419059in}}%
\pgfpathlineto{\pgfqpoint{2.551084in}{2.410135in}}%
\pgfpathlineto{\pgfqpoint{2.564116in}{2.401257in}}%
\pgfpathlineto{\pgfqpoint{2.577150in}{2.392426in}}%
\pgfpathlineto{\pgfqpoint{2.568562in}{2.398145in}}%
\pgfpathlineto{\pgfqpoint{2.559951in}{2.404239in}}%
\pgfpathlineto{\pgfqpoint{2.551318in}{2.410718in}}%
\pgfpathlineto{\pgfqpoint{2.542662in}{2.417590in}}%
\pgfpathlineto{\pgfqpoint{2.529590in}{2.426729in}}%
\pgfpathlineto{\pgfqpoint{2.516519in}{2.435914in}}%
\pgfpathlineto{\pgfqpoint{2.503450in}{2.445146in}}%
\pgfpathlineto{\pgfqpoint{2.490383in}{2.454425in}}%
\pgfpathlineto{\pgfqpoint{2.499078in}{2.447240in}}%
\pgfpathlineto{\pgfqpoint{2.507749in}{2.440451in}}%
\pgfpathlineto{\pgfqpoint{2.516398in}{2.434051in}}%
\pgfpathlineto{\pgfqpoint{2.525024in}{2.428030in}}%
\pgfpathclose%
\pgfusepath{fill}%
\end{pgfscope}%
\begin{pgfscope}%
\pgfpathrectangle{\pgfqpoint{1.254980in}{0.150000in}}{\pgfqpoint{5.490039in}{5.490039in}}%
\pgfusepath{clip}%
\pgfsetbuttcap%
\pgfsetroundjoin%
\definecolor{currentfill}{rgb}{0.268510,0.009605,0.335427}%
\pgfsetfillcolor{currentfill}%
\pgfsetfillopacity{0.700000}%
\pgfsetlinewidth{0.000000pt}%
\definecolor{currentstroke}{rgb}{0.000000,0.000000,0.000000}%
\pgfsetstrokecolor{currentstroke}%
\pgfsetdash{}{0pt}%
\pgfpathmoveto{\pgfqpoint{4.205130in}{1.915989in}}%
\pgfpathlineto{\pgfqpoint{4.218388in}{1.912713in}}%
\pgfpathlineto{\pgfqpoint{4.231653in}{1.909463in}}%
\pgfpathlineto{\pgfqpoint{4.244924in}{1.906238in}}%
\pgfpathlineto{\pgfqpoint{4.258201in}{1.903038in}}%
\pgfpathlineto{\pgfqpoint{4.250624in}{1.894996in}}%
\pgfpathlineto{\pgfqpoint{4.243041in}{1.887003in}}%
\pgfpathlineto{\pgfqpoint{4.235453in}{1.879064in}}%
\pgfpathlineto{\pgfqpoint{4.227859in}{1.871182in}}%
\pgfpathlineto{\pgfqpoint{4.214570in}{1.874552in}}%
\pgfpathlineto{\pgfqpoint{4.201287in}{1.877946in}}%
\pgfpathlineto{\pgfqpoint{4.188010in}{1.881366in}}%
\pgfpathlineto{\pgfqpoint{4.174740in}{1.884811in}}%
\pgfpathlineto{\pgfqpoint{4.182346in}{1.892518in}}%
\pgfpathlineto{\pgfqpoint{4.189946in}{1.900287in}}%
\pgfpathlineto{\pgfqpoint{4.197541in}{1.908111in}}%
\pgfpathlineto{\pgfqpoint{4.205130in}{1.915989in}}%
\pgfpathclose%
\pgfusepath{fill}%
\end{pgfscope}%
\begin{pgfscope}%
\pgfpathrectangle{\pgfqpoint{1.254980in}{0.150000in}}{\pgfqpoint{5.490039in}{5.490039in}}%
\pgfusepath{clip}%
\pgfsetbuttcap%
\pgfsetroundjoin%
\definecolor{currentfill}{rgb}{0.279574,0.170599,0.479997}%
\pgfsetfillcolor{currentfill}%
\pgfsetfillopacity{0.700000}%
\pgfsetlinewidth{0.000000pt}%
\definecolor{currentstroke}{rgb}{0.000000,0.000000,0.000000}%
\pgfsetstrokecolor{currentstroke}%
\pgfsetdash{}{0pt}%
\pgfpathmoveto{\pgfqpoint{5.417814in}{2.204385in}}%
\pgfpathlineto{\pgfqpoint{5.431427in}{2.203691in}}%
\pgfpathlineto{\pgfqpoint{5.445049in}{2.203021in}}%
\pgfpathlineto{\pgfqpoint{5.458679in}{2.202376in}}%
\pgfpathlineto{\pgfqpoint{5.472318in}{2.201753in}}%
\pgfpathlineto{\pgfqpoint{5.465196in}{2.194218in}}%
\pgfpathlineto{\pgfqpoint{5.458066in}{2.186589in}}%
\pgfpathlineto{\pgfqpoint{5.450928in}{2.178867in}}%
\pgfpathlineto{\pgfqpoint{5.443782in}{2.171052in}}%
\pgfpathlineto{\pgfqpoint{5.430131in}{2.171688in}}%
\pgfpathlineto{\pgfqpoint{5.416489in}{2.172347in}}%
\pgfpathlineto{\pgfqpoint{5.402855in}{2.173030in}}%
\pgfpathlineto{\pgfqpoint{5.389230in}{2.173737in}}%
\pgfpathlineto{\pgfqpoint{5.396387in}{2.181534in}}%
\pgfpathlineto{\pgfqpoint{5.403537in}{2.189241in}}%
\pgfpathlineto{\pgfqpoint{5.410679in}{2.196858in}}%
\pgfpathlineto{\pgfqpoint{5.417814in}{2.204385in}}%
\pgfpathclose%
\pgfusepath{fill}%
\end{pgfscope}%
\begin{pgfscope}%
\pgfpathrectangle{\pgfqpoint{1.254980in}{0.150000in}}{\pgfqpoint{5.490039in}{5.490039in}}%
\pgfusepath{clip}%
\pgfsetbuttcap%
\pgfsetroundjoin%
\definecolor{currentfill}{rgb}{0.272594,0.025563,0.353093}%
\pgfsetfillcolor{currentfill}%
\pgfsetfillopacity{0.700000}%
\pgfsetlinewidth{0.000000pt}%
\definecolor{currentstroke}{rgb}{0.000000,0.000000,0.000000}%
\pgfsetstrokecolor{currentstroke}%
\pgfsetdash{}{0pt}%
\pgfpathmoveto{\pgfqpoint{4.424907in}{1.946395in}}%
\pgfpathlineto{\pgfqpoint{4.438222in}{1.943720in}}%
\pgfpathlineto{\pgfqpoint{4.451544in}{1.941069in}}%
\pgfpathlineto{\pgfqpoint{4.464873in}{1.938444in}}%
\pgfpathlineto{\pgfqpoint{4.478209in}{1.935843in}}%
\pgfpathlineto{\pgfqpoint{4.470706in}{1.927124in}}%
\pgfpathlineto{\pgfqpoint{4.463198in}{1.918416in}}%
\pgfpathlineto{\pgfqpoint{4.455685in}{1.909722in}}%
\pgfpathlineto{\pgfqpoint{4.448167in}{1.901046in}}%
\pgfpathlineto{\pgfqpoint{4.434821in}{1.903791in}}%
\pgfpathlineto{\pgfqpoint{4.421481in}{1.906560in}}%
\pgfpathlineto{\pgfqpoint{4.408149in}{1.909355in}}%
\pgfpathlineto{\pgfqpoint{4.394823in}{1.912174in}}%
\pgfpathlineto{\pgfqpoint{4.402352in}{1.920702in}}%
\pgfpathlineto{\pgfqpoint{4.409875in}{1.929250in}}%
\pgfpathlineto{\pgfqpoint{4.417394in}{1.937816in}}%
\pgfpathlineto{\pgfqpoint{4.424907in}{1.946395in}}%
\pgfpathclose%
\pgfusepath{fill}%
\end{pgfscope}%
\begin{pgfscope}%
\pgfpathrectangle{\pgfqpoint{1.254980in}{0.150000in}}{\pgfqpoint{5.490039in}{5.490039in}}%
\pgfusepath{clip}%
\pgfsetbuttcap%
\pgfsetroundjoin%
\definecolor{currentfill}{rgb}{0.269944,0.014625,0.341379}%
\pgfsetfillcolor{currentfill}%
\pgfsetfillopacity{0.700000}%
\pgfsetlinewidth{0.000000pt}%
\definecolor{currentstroke}{rgb}{0.000000,0.000000,0.000000}%
\pgfsetstrokecolor{currentstroke}%
\pgfsetdash{}{0pt}%
\pgfpathmoveto{\pgfqpoint{3.712698in}{1.922040in}}%
\pgfpathlineto{\pgfqpoint{3.725850in}{1.917274in}}%
\pgfpathlineto{\pgfqpoint{3.739006in}{1.912535in}}%
\pgfpathlineto{\pgfqpoint{3.752168in}{1.907824in}}%
\pgfpathlineto{\pgfqpoint{3.765336in}{1.903140in}}%
\pgfpathlineto{\pgfqpoint{3.757561in}{1.897933in}}%
\pgfpathlineto{\pgfqpoint{3.749778in}{1.892874in}}%
\pgfpathlineto{\pgfqpoint{3.741988in}{1.887967in}}%
\pgfpathlineto{\pgfqpoint{3.734190in}{1.883220in}}%
\pgfpathlineto{\pgfqpoint{3.721005in}{1.888124in}}%
\pgfpathlineto{\pgfqpoint{3.707825in}{1.893056in}}%
\pgfpathlineto{\pgfqpoint{3.694651in}{1.898015in}}%
\pgfpathlineto{\pgfqpoint{3.681481in}{1.903001in}}%
\pgfpathlineto{\pgfqpoint{3.689297in}{1.907523in}}%
\pgfpathlineto{\pgfqpoint{3.697106in}{1.912208in}}%
\pgfpathlineto{\pgfqpoint{3.704906in}{1.917049in}}%
\pgfpathlineto{\pgfqpoint{3.712698in}{1.922040in}}%
\pgfpathclose%
\pgfusepath{fill}%
\end{pgfscope}%
\begin{pgfscope}%
\pgfpathrectangle{\pgfqpoint{1.254980in}{0.150000in}}{\pgfqpoint{5.490039in}{5.490039in}}%
\pgfusepath{clip}%
\pgfsetbuttcap%
\pgfsetroundjoin%
\definecolor{currentfill}{rgb}{0.268510,0.009605,0.335427}%
\pgfsetfillcolor{currentfill}%
\pgfsetfillopacity{0.700000}%
\pgfsetlinewidth{0.000000pt}%
\definecolor{currentstroke}{rgb}{0.000000,0.000000,0.000000}%
\pgfsetstrokecolor{currentstroke}%
\pgfsetdash{}{0pt}%
\pgfpathmoveto{\pgfqpoint{3.849022in}{1.907702in}}%
\pgfpathlineto{\pgfqpoint{3.862201in}{1.903361in}}%
\pgfpathlineto{\pgfqpoint{3.875385in}{1.899046in}}%
\pgfpathlineto{\pgfqpoint{3.888575in}{1.894758in}}%
\pgfpathlineto{\pgfqpoint{3.901771in}{1.890497in}}%
\pgfpathlineto{\pgfqpoint{3.894056in}{1.884348in}}%
\pgfpathlineto{\pgfqpoint{3.886335in}{1.878319in}}%
\pgfpathlineto{\pgfqpoint{3.878606in}{1.872415in}}%
\pgfpathlineto{\pgfqpoint{3.870871in}{1.866641in}}%
\pgfpathlineto{\pgfqpoint{3.857660in}{1.871110in}}%
\pgfpathlineto{\pgfqpoint{3.844454in}{1.875605in}}%
\pgfpathlineto{\pgfqpoint{3.831254in}{1.880127in}}%
\pgfpathlineto{\pgfqpoint{3.818060in}{1.884676in}}%
\pgfpathlineto{\pgfqpoint{3.825811in}{1.890237in}}%
\pgfpathlineto{\pgfqpoint{3.833555in}{1.895932in}}%
\pgfpathlineto{\pgfqpoint{3.841292in}{1.901756in}}%
\pgfpathlineto{\pgfqpoint{3.849022in}{1.907702in}}%
\pgfpathclose%
\pgfusepath{fill}%
\end{pgfscope}%
\begin{pgfscope}%
\pgfpathrectangle{\pgfqpoint{1.254980in}{0.150000in}}{\pgfqpoint{5.490039in}{5.490039in}}%
\pgfusepath{clip}%
\pgfsetbuttcap%
\pgfsetroundjoin%
\definecolor{currentfill}{rgb}{0.271828,0.209303,0.504434}%
\pgfsetfillcolor{currentfill}%
\pgfsetfillopacity{0.700000}%
\pgfsetlinewidth{0.000000pt}%
\definecolor{currentstroke}{rgb}{0.000000,0.000000,0.000000}%
\pgfsetstrokecolor{currentstroke}%
\pgfsetdash{}{0pt}%
\pgfpathmoveto{\pgfqpoint{5.721039in}{2.279028in}}%
\pgfpathlineto{\pgfqpoint{5.734750in}{2.278674in}}%
\pgfpathlineto{\pgfqpoint{5.748471in}{2.278344in}}%
\pgfpathlineto{\pgfqpoint{5.762200in}{2.278038in}}%
\pgfpathlineto{\pgfqpoint{5.775938in}{2.277756in}}%
\pgfpathlineto{\pgfqpoint{5.768968in}{2.271414in}}%
\pgfpathlineto{\pgfqpoint{5.761989in}{2.264976in}}%
\pgfpathlineto{\pgfqpoint{5.755002in}{2.258440in}}%
\pgfpathlineto{\pgfqpoint{5.748005in}{2.251805in}}%
\pgfpathlineto{\pgfqpoint{5.734253in}{2.252060in}}%
\pgfpathlineto{\pgfqpoint{5.720509in}{2.252340in}}%
\pgfpathlineto{\pgfqpoint{5.706774in}{2.252643in}}%
\pgfpathlineto{\pgfqpoint{5.693048in}{2.252970in}}%
\pgfpathlineto{\pgfqpoint{5.700058in}{2.259627in}}%
\pgfpathlineto{\pgfqpoint{5.707060in}{2.266188in}}%
\pgfpathlineto{\pgfqpoint{5.714054in}{2.272655in}}%
\pgfpathlineto{\pgfqpoint{5.721039in}{2.279028in}}%
\pgfpathclose%
\pgfusepath{fill}%
\end{pgfscope}%
\begin{pgfscope}%
\pgfpathrectangle{\pgfqpoint{1.254980in}{0.150000in}}{\pgfqpoint{5.490039in}{5.490039in}}%
\pgfusepath{clip}%
\pgfsetbuttcap%
\pgfsetroundjoin%
\definecolor{currentfill}{rgb}{0.283197,0.115680,0.436115}%
\pgfsetfillcolor{currentfill}%
\pgfsetfillopacity{0.700000}%
\pgfsetlinewidth{0.000000pt}%
\definecolor{currentstroke}{rgb}{0.000000,0.000000,0.000000}%
\pgfsetstrokecolor{currentstroke}%
\pgfsetdash{}{0pt}%
\pgfpathmoveto{\pgfqpoint{5.031367in}{2.094779in}}%
\pgfpathlineto{\pgfqpoint{5.044861in}{2.093474in}}%
\pgfpathlineto{\pgfqpoint{5.058362in}{2.092194in}}%
\pgfpathlineto{\pgfqpoint{5.071871in}{2.090937in}}%
\pgfpathlineto{\pgfqpoint{5.085388in}{2.089704in}}%
\pgfpathlineto{\pgfqpoint{5.078098in}{2.081012in}}%
\pgfpathlineto{\pgfqpoint{5.070801in}{2.072251in}}%
\pgfpathlineto{\pgfqpoint{5.063498in}{2.063422in}}%
\pgfpathlineto{\pgfqpoint{5.056188in}{2.054527in}}%
\pgfpathlineto{\pgfqpoint{5.042662in}{2.055826in}}%
\pgfpathlineto{\pgfqpoint{5.029143in}{2.057149in}}%
\pgfpathlineto{\pgfqpoint{5.015632in}{2.058496in}}%
\pgfpathlineto{\pgfqpoint{5.002129in}{2.059868in}}%
\pgfpathlineto{\pgfqpoint{5.009448in}{2.068692in}}%
\pgfpathlineto{\pgfqpoint{5.016760in}{2.077453in}}%
\pgfpathlineto{\pgfqpoint{5.024067in}{2.086149in}}%
\pgfpathlineto{\pgfqpoint{5.031367in}{2.094779in}}%
\pgfpathclose%
\pgfusepath{fill}%
\end{pgfscope}%
\begin{pgfscope}%
\pgfpathrectangle{\pgfqpoint{1.254980in}{0.150000in}}{\pgfqpoint{5.490039in}{5.490039in}}%
\pgfusepath{clip}%
\pgfsetbuttcap%
\pgfsetroundjoin%
\definecolor{currentfill}{rgb}{0.275191,0.194905,0.496005}%
\pgfsetfillcolor{currentfill}%
\pgfsetfillopacity{0.700000}%
\pgfsetlinewidth{0.000000pt}%
\definecolor{currentstroke}{rgb}{0.000000,0.000000,0.000000}%
\pgfsetstrokecolor{currentstroke}%
\pgfsetdash{}{0pt}%
\pgfpathmoveto{\pgfqpoint{2.819587in}{2.242458in}}%
\pgfpathlineto{\pgfqpoint{2.832627in}{2.234640in}}%
\pgfpathlineto{\pgfqpoint{2.845670in}{2.226861in}}%
\pgfpathlineto{\pgfqpoint{2.858716in}{2.219121in}}%
\pgfpathlineto{\pgfqpoint{2.871765in}{2.211420in}}%
\pgfpathlineto{\pgfqpoint{2.863425in}{2.214287in}}%
\pgfpathlineto{\pgfqpoint{2.855068in}{2.217479in}}%
\pgfpathlineto{\pgfqpoint{2.846693in}{2.221005in}}%
\pgfpathlineto{\pgfqpoint{2.838299in}{2.224873in}}%
\pgfpathlineto{\pgfqpoint{2.825218in}{2.232864in}}%
\pgfpathlineto{\pgfqpoint{2.812140in}{2.240894in}}%
\pgfpathlineto{\pgfqpoint{2.799064in}{2.248963in}}%
\pgfpathlineto{\pgfqpoint{2.785991in}{2.257071in}}%
\pgfpathlineto{\pgfqpoint{2.794418in}{2.252908in}}%
\pgfpathlineto{\pgfqpoint{2.802826in}{2.249090in}}%
\pgfpathlineto{\pgfqpoint{2.811215in}{2.245609in}}%
\pgfpathlineto{\pgfqpoint{2.819587in}{2.242458in}}%
\pgfpathclose%
\pgfusepath{fill}%
\end{pgfscope}%
\begin{pgfscope}%
\pgfpathrectangle{\pgfqpoint{1.254980in}{0.150000in}}{\pgfqpoint{5.490039in}{5.490039in}}%
\pgfusepath{clip}%
\pgfsetbuttcap%
\pgfsetroundjoin%
\definecolor{currentfill}{rgb}{0.273809,0.031497,0.358853}%
\pgfsetfillcolor{currentfill}%
\pgfsetfillopacity{0.700000}%
\pgfsetlinewidth{0.000000pt}%
\definecolor{currentstroke}{rgb}{0.000000,0.000000,0.000000}%
\pgfsetstrokecolor{currentstroke}%
\pgfsetdash{}{0pt}%
\pgfpathmoveto{\pgfqpoint{3.576311in}{1.943900in}}%
\pgfpathlineto{\pgfqpoint{3.589439in}{1.938688in}}%
\pgfpathlineto{\pgfqpoint{3.602573in}{1.933506in}}%
\pgfpathlineto{\pgfqpoint{3.615712in}{1.928351in}}%
\pgfpathlineto{\pgfqpoint{3.628856in}{1.923225in}}%
\pgfpathlineto{\pgfqpoint{3.621013in}{1.919099in}}%
\pgfpathlineto{\pgfqpoint{3.613161in}{1.915150in}}%
\pgfpathlineto{\pgfqpoint{3.605300in}{1.911383in}}%
\pgfpathlineto{\pgfqpoint{3.597431in}{1.907806in}}%
\pgfpathlineto{\pgfqpoint{3.584268in}{1.913165in}}%
\pgfpathlineto{\pgfqpoint{3.571109in}{1.918553in}}%
\pgfpathlineto{\pgfqpoint{3.557956in}{1.923969in}}%
\pgfpathlineto{\pgfqpoint{3.544807in}{1.929413in}}%
\pgfpathlineto{\pgfqpoint{3.552697in}{1.932753in}}%
\pgfpathlineto{\pgfqpoint{3.560577in}{1.936284in}}%
\pgfpathlineto{\pgfqpoint{3.568448in}{1.940002in}}%
\pgfpathlineto{\pgfqpoint{3.576311in}{1.943900in}}%
\pgfpathclose%
\pgfusepath{fill}%
\end{pgfscope}%
\begin{pgfscope}%
\pgfpathrectangle{\pgfqpoint{1.254980in}{0.150000in}}{\pgfqpoint{5.490039in}{5.490039in}}%
\pgfusepath{clip}%
\pgfsetbuttcap%
\pgfsetroundjoin%
\definecolor{currentfill}{rgb}{0.267004,0.004874,0.329415}%
\pgfsetfillcolor{currentfill}%
\pgfsetfillopacity{0.700000}%
\pgfsetlinewidth{0.000000pt}%
\definecolor{currentstroke}{rgb}{0.000000,0.000000,0.000000}%
\pgfsetstrokecolor{currentstroke}%
\pgfsetdash{}{0pt}%
\pgfpathmoveto{\pgfqpoint{3.985345in}{1.900190in}}%
\pgfpathlineto{\pgfqpoint{3.998555in}{1.896255in}}%
\pgfpathlineto{\pgfqpoint{4.011771in}{1.892346in}}%
\pgfpathlineto{\pgfqpoint{4.024993in}{1.888463in}}%
\pgfpathlineto{\pgfqpoint{4.038221in}{1.884606in}}%
\pgfpathlineto{\pgfqpoint{4.030560in}{1.877647in}}%
\pgfpathlineto{\pgfqpoint{4.022894in}{1.870782in}}%
\pgfpathlineto{\pgfqpoint{4.015221in}{1.864016in}}%
\pgfpathlineto{\pgfqpoint{4.007543in}{1.857353in}}%
\pgfpathlineto{\pgfqpoint{3.994301in}{1.861405in}}%
\pgfpathlineto{\pgfqpoint{3.981065in}{1.865482in}}%
\pgfpathlineto{\pgfqpoint{3.967835in}{1.869586in}}%
\pgfpathlineto{\pgfqpoint{3.954610in}{1.873716in}}%
\pgfpathlineto{\pgfqpoint{3.962304in}{1.880179in}}%
\pgfpathlineto{\pgfqpoint{3.969990in}{1.886749in}}%
\pgfpathlineto{\pgfqpoint{3.977671in}{1.893421in}}%
\pgfpathlineto{\pgfqpoint{3.985345in}{1.900190in}}%
\pgfpathclose%
\pgfusepath{fill}%
\end{pgfscope}%
\begin{pgfscope}%
\pgfpathrectangle{\pgfqpoint{1.254980in}{0.150000in}}{\pgfqpoint{5.490039in}{5.490039in}}%
\pgfusepath{clip}%
\pgfsetbuttcap%
\pgfsetroundjoin%
\definecolor{currentfill}{rgb}{0.277941,0.056324,0.381191}%
\pgfsetfillcolor{currentfill}%
\pgfsetfillopacity{0.700000}%
\pgfsetlinewidth{0.000000pt}%
\definecolor{currentstroke}{rgb}{0.000000,0.000000,0.000000}%
\pgfsetstrokecolor{currentstroke}%
\pgfsetdash{}{0pt}%
\pgfpathmoveto{\pgfqpoint{4.644825in}{1.987666in}}%
\pgfpathlineto{\pgfqpoint{4.658205in}{1.985535in}}%
\pgfpathlineto{\pgfqpoint{4.671592in}{1.983428in}}%
\pgfpathlineto{\pgfqpoint{4.684986in}{1.981345in}}%
\pgfpathlineto{\pgfqpoint{4.698388in}{1.979287in}}%
\pgfpathlineto{\pgfqpoint{4.690957in}{1.970253in}}%
\pgfpathlineto{\pgfqpoint{4.683520in}{1.961197in}}%
\pgfpathlineto{\pgfqpoint{4.676078in}{1.952122in}}%
\pgfpathlineto{\pgfqpoint{4.668630in}{1.943028in}}%
\pgfpathlineto{\pgfqpoint{4.655219in}{1.945205in}}%
\pgfpathlineto{\pgfqpoint{4.641815in}{1.947406in}}%
\pgfpathlineto{\pgfqpoint{4.628418in}{1.949631in}}%
\pgfpathlineto{\pgfqpoint{4.615029in}{1.951881in}}%
\pgfpathlineto{\pgfqpoint{4.622486in}{1.960851in}}%
\pgfpathlineto{\pgfqpoint{4.629937in}{1.969807in}}%
\pgfpathlineto{\pgfqpoint{4.637384in}{1.978746in}}%
\pgfpathlineto{\pgfqpoint{4.644825in}{1.987666in}}%
\pgfpathclose%
\pgfusepath{fill}%
\end{pgfscope}%
\begin{pgfscope}%
\pgfpathrectangle{\pgfqpoint{1.254980in}{0.150000in}}{\pgfqpoint{5.490039in}{5.490039in}}%
\pgfusepath{clip}%
\pgfsetbuttcap%
\pgfsetroundjoin%
\definecolor{currentfill}{rgb}{0.281924,0.089666,0.412415}%
\pgfsetfillcolor{currentfill}%
\pgfsetfillopacity{0.700000}%
\pgfsetlinewidth{0.000000pt}%
\definecolor{currentstroke}{rgb}{0.000000,0.000000,0.000000}%
\pgfsetstrokecolor{currentstroke}%
\pgfsetdash{}{0pt}%
\pgfpathmoveto{\pgfqpoint{3.250705in}{2.038149in}}%
\pgfpathlineto{\pgfqpoint{3.263786in}{2.031854in}}%
\pgfpathlineto{\pgfqpoint{3.276872in}{2.025592in}}%
\pgfpathlineto{\pgfqpoint{3.289962in}{2.019361in}}%
\pgfpathlineto{\pgfqpoint{3.303056in}{2.013161in}}%
\pgfpathlineto{\pgfqpoint{3.295025in}{2.011897in}}%
\pgfpathlineto{\pgfqpoint{3.286983in}{2.010876in}}%
\pgfpathlineto{\pgfqpoint{3.278928in}{2.010106in}}%
\pgfpathlineto{\pgfqpoint{3.270861in}{2.009593in}}%
\pgfpathlineto{\pgfqpoint{3.257742in}{2.016053in}}%
\pgfpathlineto{\pgfqpoint{3.244628in}{2.022544in}}%
\pgfpathlineto{\pgfqpoint{3.231517in}{2.029067in}}%
\pgfpathlineto{\pgfqpoint{3.218410in}{2.035621in}}%
\pgfpathlineto{\pgfqpoint{3.226503in}{2.035869in}}%
\pgfpathlineto{\pgfqpoint{3.234583in}{2.036377in}}%
\pgfpathlineto{\pgfqpoint{3.242650in}{2.037139in}}%
\pgfpathlineto{\pgfqpoint{3.250705in}{2.038149in}}%
\pgfpathclose%
\pgfusepath{fill}%
\end{pgfscope}%
\begin{pgfscope}%
\pgfpathrectangle{\pgfqpoint{1.254980in}{0.150000in}}{\pgfqpoint{5.490039in}{5.490039in}}%
\pgfusepath{clip}%
\pgfsetbuttcap%
\pgfsetroundjoin%
\definecolor{currentfill}{rgb}{0.280868,0.160771,0.472899}%
\pgfsetfillcolor{currentfill}%
\pgfsetfillopacity{0.700000}%
\pgfsetlinewidth{0.000000pt}%
\definecolor{currentstroke}{rgb}{0.000000,0.000000,0.000000}%
\pgfsetstrokecolor{currentstroke}%
\pgfsetdash{}{0pt}%
\pgfpathmoveto{\pgfqpoint{5.334813in}{2.176804in}}%
\pgfpathlineto{\pgfqpoint{5.348404in}{2.176001in}}%
\pgfpathlineto{\pgfqpoint{5.362004in}{2.175223in}}%
\pgfpathlineto{\pgfqpoint{5.375613in}{2.174468in}}%
\pgfpathlineto{\pgfqpoint{5.389230in}{2.173737in}}%
\pgfpathlineto{\pgfqpoint{5.382065in}{2.165850in}}%
\pgfpathlineto{\pgfqpoint{5.374893in}{2.157873in}}%
\pgfpathlineto{\pgfqpoint{5.367713in}{2.149807in}}%
\pgfpathlineto{\pgfqpoint{5.360526in}{2.141650in}}%
\pgfpathlineto{\pgfqpoint{5.346898in}{2.142408in}}%
\pgfpathlineto{\pgfqpoint{5.333278in}{2.143189in}}%
\pgfpathlineto{\pgfqpoint{5.319667in}{2.143994in}}%
\pgfpathlineto{\pgfqpoint{5.306064in}{2.144823in}}%
\pgfpathlineto{\pgfqpoint{5.313262in}{2.152948in}}%
\pgfpathlineto{\pgfqpoint{5.320453in}{2.160987in}}%
\pgfpathlineto{\pgfqpoint{5.327637in}{2.168938in}}%
\pgfpathlineto{\pgfqpoint{5.334813in}{2.176804in}}%
\pgfpathclose%
\pgfusepath{fill}%
\end{pgfscope}%
\begin{pgfscope}%
\pgfpathrectangle{\pgfqpoint{1.254980in}{0.150000in}}{\pgfqpoint{5.490039in}{5.490039in}}%
\pgfusepath{clip}%
\pgfsetbuttcap%
\pgfsetroundjoin%
\definecolor{currentfill}{rgb}{0.283072,0.130895,0.449241}%
\pgfsetfillcolor{currentfill}%
\pgfsetfillopacity{0.700000}%
\pgfsetlinewidth{0.000000pt}%
\definecolor{currentstroke}{rgb}{0.000000,0.000000,0.000000}%
\pgfsetstrokecolor{currentstroke}%
\pgfsetdash{}{0pt}%
\pgfpathmoveto{\pgfqpoint{3.061428in}{2.116837in}}%
\pgfpathlineto{\pgfqpoint{3.074490in}{2.109884in}}%
\pgfpathlineto{\pgfqpoint{3.087555in}{2.102965in}}%
\pgfpathlineto{\pgfqpoint{3.100623in}{2.096081in}}%
\pgfpathlineto{\pgfqpoint{3.113696in}{2.089230in}}%
\pgfpathlineto{\pgfqpoint{3.105536in}{2.089789in}}%
\pgfpathlineto{\pgfqpoint{3.097363in}{2.090630in}}%
\pgfpathlineto{\pgfqpoint{3.089175in}{2.091761in}}%
\pgfpathlineto{\pgfqpoint{3.080972in}{2.093188in}}%
\pgfpathlineto{\pgfqpoint{3.067871in}{2.100313in}}%
\pgfpathlineto{\pgfqpoint{3.054774in}{2.107472in}}%
\pgfpathlineto{\pgfqpoint{3.041681in}{2.114666in}}%
\pgfpathlineto{\pgfqpoint{3.028591in}{2.121894in}}%
\pgfpathlineto{\pgfqpoint{3.036823in}{2.120187in}}%
\pgfpathlineto{\pgfqpoint{3.045039in}{2.118780in}}%
\pgfpathlineto{\pgfqpoint{3.053241in}{2.117666in}}%
\pgfpathlineto{\pgfqpoint{3.061428in}{2.116837in}}%
\pgfpathclose%
\pgfusepath{fill}%
\end{pgfscope}%
\begin{pgfscope}%
\pgfpathrectangle{\pgfqpoint{1.254980in}{0.150000in}}{\pgfqpoint{5.490039in}{5.490039in}}%
\pgfusepath{clip}%
\pgfsetbuttcap%
\pgfsetroundjoin%
\definecolor{currentfill}{rgb}{0.282656,0.100196,0.422160}%
\pgfsetfillcolor{currentfill}%
\pgfsetfillopacity{0.700000}%
\pgfsetlinewidth{0.000000pt}%
\definecolor{currentstroke}{rgb}{0.000000,0.000000,0.000000}%
\pgfsetstrokecolor{currentstroke}%
\pgfsetdash{}{0pt}%
\pgfpathmoveto{\pgfqpoint{4.948195in}{2.065592in}}%
\pgfpathlineto{\pgfqpoint{4.961667in}{2.064125in}}%
\pgfpathlineto{\pgfqpoint{4.975146in}{2.062682in}}%
\pgfpathlineto{\pgfqpoint{4.988634in}{2.061263in}}%
\pgfpathlineto{\pgfqpoint{5.002129in}{2.059868in}}%
\pgfpathlineto{\pgfqpoint{4.994804in}{2.050982in}}%
\pgfpathlineto{\pgfqpoint{4.987472in}{2.042035in}}%
\pgfpathlineto{\pgfqpoint{4.980135in}{2.033030in}}%
\pgfpathlineto{\pgfqpoint{4.972792in}{2.023967in}}%
\pgfpathlineto{\pgfqpoint{4.959287in}{2.025442in}}%
\pgfpathlineto{\pgfqpoint{4.945790in}{2.026940in}}%
\pgfpathlineto{\pgfqpoint{4.932301in}{2.028463in}}%
\pgfpathlineto{\pgfqpoint{4.918820in}{2.030009in}}%
\pgfpathlineto{\pgfqpoint{4.926173in}{2.038988in}}%
\pgfpathlineto{\pgfqpoint{4.933520in}{2.047912in}}%
\pgfpathlineto{\pgfqpoint{4.940860in}{2.056780in}}%
\pgfpathlineto{\pgfqpoint{4.948195in}{2.065592in}}%
\pgfpathclose%
\pgfusepath{fill}%
\end{pgfscope}%
\begin{pgfscope}%
\pgfpathrectangle{\pgfqpoint{1.254980in}{0.150000in}}{\pgfqpoint{5.490039in}{5.490039in}}%
\pgfusepath{clip}%
\pgfsetbuttcap%
\pgfsetroundjoin%
\definecolor{currentfill}{rgb}{0.252194,0.269783,0.531579}%
\pgfsetfillcolor{currentfill}%
\pgfsetfillopacity{0.700000}%
\pgfsetlinewidth{0.000000pt}%
\definecolor{currentstroke}{rgb}{0.000000,0.000000,0.000000}%
\pgfsetstrokecolor{currentstroke}%
\pgfsetdash{}{0pt}%
\pgfpathmoveto{\pgfqpoint{2.577150in}{2.392426in}}%
\pgfpathlineto{\pgfqpoint{2.590187in}{2.383641in}}%
\pgfpathlineto{\pgfqpoint{2.603225in}{2.374901in}}%
\pgfpathlineto{\pgfqpoint{2.616265in}{2.366206in}}%
\pgfpathlineto{\pgfqpoint{2.629307in}{2.357555in}}%
\pgfpathlineto{\pgfqpoint{2.620755in}{2.362972in}}%
\pgfpathlineto{\pgfqpoint{2.612182in}{2.368762in}}%
\pgfpathlineto{\pgfqpoint{2.603587in}{2.374932in}}%
\pgfpathlineto{\pgfqpoint{2.594970in}{2.381491in}}%
\pgfpathlineto{\pgfqpoint{2.581890in}{2.390449in}}%
\pgfpathlineto{\pgfqpoint{2.568812in}{2.399451in}}%
\pgfpathlineto{\pgfqpoint{2.555736in}{2.408498in}}%
\pgfpathlineto{\pgfqpoint{2.542662in}{2.417590in}}%
\pgfpathlineto{\pgfqpoint{2.551318in}{2.410718in}}%
\pgfpathlineto{\pgfqpoint{2.559951in}{2.404239in}}%
\pgfpathlineto{\pgfqpoint{2.568562in}{2.398145in}}%
\pgfpathlineto{\pgfqpoint{2.577150in}{2.392426in}}%
\pgfpathclose%
\pgfusepath{fill}%
\end{pgfscope}%
\begin{pgfscope}%
\pgfpathrectangle{\pgfqpoint{1.254980in}{0.150000in}}{\pgfqpoint{5.490039in}{5.490039in}}%
\pgfusepath{clip}%
\pgfsetbuttcap%
\pgfsetroundjoin%
\definecolor{currentfill}{rgb}{0.271305,0.019942,0.347269}%
\pgfsetfillcolor{currentfill}%
\pgfsetfillopacity{0.700000}%
\pgfsetlinewidth{0.000000pt}%
\definecolor{currentstroke}{rgb}{0.000000,0.000000,0.000000}%
\pgfsetstrokecolor{currentstroke}%
\pgfsetdash{}{0pt}%
\pgfpathmoveto{\pgfqpoint{4.341587in}{1.923698in}}%
\pgfpathlineto{\pgfqpoint{4.354886in}{1.920779in}}%
\pgfpathlineto{\pgfqpoint{4.368191in}{1.917886in}}%
\pgfpathlineto{\pgfqpoint{4.381504in}{1.915017in}}%
\pgfpathlineto{\pgfqpoint{4.394823in}{1.912174in}}%
\pgfpathlineto{\pgfqpoint{4.387289in}{1.903670in}}%
\pgfpathlineto{\pgfqpoint{4.379749in}{1.895193in}}%
\pgfpathlineto{\pgfqpoint{4.372205in}{1.886748in}}%
\pgfpathlineto{\pgfqpoint{4.364655in}{1.878338in}}%
\pgfpathlineto{\pgfqpoint{4.351325in}{1.881339in}}%
\pgfpathlineto{\pgfqpoint{4.338002in}{1.884364in}}%
\pgfpathlineto{\pgfqpoint{4.324686in}{1.887414in}}%
\pgfpathlineto{\pgfqpoint{4.311376in}{1.890489in}}%
\pgfpathlineto{\pgfqpoint{4.318936in}{1.898737in}}%
\pgfpathlineto{\pgfqpoint{4.326492in}{1.907024in}}%
\pgfpathlineto{\pgfqpoint{4.334042in}{1.915345in}}%
\pgfpathlineto{\pgfqpoint{4.341587in}{1.923698in}}%
\pgfpathclose%
\pgfusepath{fill}%
\end{pgfscope}%
\begin{pgfscope}%
\pgfpathrectangle{\pgfqpoint{1.254980in}{0.150000in}}{\pgfqpoint{5.490039in}{5.490039in}}%
\pgfusepath{clip}%
\pgfsetbuttcap%
\pgfsetroundjoin%
\definecolor{currentfill}{rgb}{0.274128,0.199721,0.498911}%
\pgfsetfillcolor{currentfill}%
\pgfsetfillopacity{0.700000}%
\pgfsetlinewidth{0.000000pt}%
\definecolor{currentstroke}{rgb}{0.000000,0.000000,0.000000}%
\pgfsetstrokecolor{currentstroke}%
\pgfsetdash{}{0pt}%
\pgfpathmoveto{\pgfqpoint{5.638232in}{2.254514in}}%
\pgfpathlineto{\pgfqpoint{5.651923in}{2.254092in}}%
\pgfpathlineto{\pgfqpoint{5.665622in}{2.253695in}}%
\pgfpathlineto{\pgfqpoint{5.679331in}{2.253320in}}%
\pgfpathlineto{\pgfqpoint{5.693048in}{2.252970in}}%
\pgfpathlineto{\pgfqpoint{5.686029in}{2.246215in}}%
\pgfpathlineto{\pgfqpoint{5.679002in}{2.239363in}}%
\pgfpathlineto{\pgfqpoint{5.671966in}{2.232411in}}%
\pgfpathlineto{\pgfqpoint{5.664922in}{2.225359in}}%
\pgfpathlineto{\pgfqpoint{5.651191in}{2.225697in}}%
\pgfpathlineto{\pgfqpoint{5.637469in}{2.226057in}}%
\pgfpathlineto{\pgfqpoint{5.623756in}{2.226442in}}%
\pgfpathlineto{\pgfqpoint{5.610052in}{2.226850in}}%
\pgfpathlineto{\pgfqpoint{5.617109in}{2.233910in}}%
\pgfpathlineto{\pgfqpoint{5.624158in}{2.240874in}}%
\pgfpathlineto{\pgfqpoint{5.631199in}{2.247741in}}%
\pgfpathlineto{\pgfqpoint{5.638232in}{2.254514in}}%
\pgfpathclose%
\pgfusepath{fill}%
\end{pgfscope}%
\begin{pgfscope}%
\pgfpathrectangle{\pgfqpoint{1.254980in}{0.150000in}}{\pgfqpoint{5.490039in}{5.490039in}}%
\pgfusepath{clip}%
\pgfsetbuttcap%
\pgfsetroundjoin%
\definecolor{currentfill}{rgb}{0.277018,0.050344,0.375715}%
\pgfsetfillcolor{currentfill}%
\pgfsetfillopacity{0.700000}%
\pgfsetlinewidth{0.000000pt}%
\definecolor{currentstroke}{rgb}{0.000000,0.000000,0.000000}%
\pgfsetstrokecolor{currentstroke}%
\pgfsetdash{}{0pt}%
\pgfpathmoveto{\pgfqpoint{3.439789in}{1.974015in}}%
\pgfpathlineto{\pgfqpoint{3.452900in}{1.968337in}}%
\pgfpathlineto{\pgfqpoint{3.466016in}{1.962688in}}%
\pgfpathlineto{\pgfqpoint{3.479136in}{1.957069in}}%
\pgfpathlineto{\pgfqpoint{3.492261in}{1.951480in}}%
\pgfpathlineto{\pgfqpoint{3.484341in}{1.948580in}}%
\pgfpathlineto{\pgfqpoint{3.476411in}{1.945889in}}%
\pgfpathlineto{\pgfqpoint{3.468471in}{1.943411in}}%
\pgfpathlineto{\pgfqpoint{3.460521in}{1.941154in}}%
\pgfpathlineto{\pgfqpoint{3.447375in}{1.946990in}}%
\pgfpathlineto{\pgfqpoint{3.434233in}{1.952855in}}%
\pgfpathlineto{\pgfqpoint{3.421095in}{1.958750in}}%
\pgfpathlineto{\pgfqpoint{3.407962in}{1.964675in}}%
\pgfpathlineto{\pgfqpoint{3.415935in}{1.966680in}}%
\pgfpathlineto{\pgfqpoint{3.423897in}{1.968910in}}%
\pgfpathlineto{\pgfqpoint{3.431848in}{1.971357in}}%
\pgfpathlineto{\pgfqpoint{3.439789in}{1.974015in}}%
\pgfpathclose%
\pgfusepath{fill}%
\end{pgfscope}%
\begin{pgfscope}%
\pgfpathrectangle{\pgfqpoint{1.254980in}{0.150000in}}{\pgfqpoint{5.490039in}{5.490039in}}%
\pgfusepath{clip}%
\pgfsetbuttcap%
\pgfsetroundjoin%
\definecolor{currentfill}{rgb}{0.267004,0.004874,0.329415}%
\pgfsetfillcolor{currentfill}%
\pgfsetfillopacity{0.700000}%
\pgfsetlinewidth{0.000000pt}%
\definecolor{currentstroke}{rgb}{0.000000,0.000000,0.000000}%
\pgfsetstrokecolor{currentstroke}%
\pgfsetdash{}{0pt}%
\pgfpathmoveto{\pgfqpoint{4.121722in}{1.898844in}}%
\pgfpathlineto{\pgfqpoint{4.134967in}{1.895298in}}%
\pgfpathlineto{\pgfqpoint{4.148219in}{1.891776in}}%
\pgfpathlineto{\pgfqpoint{4.161476in}{1.888281in}}%
\pgfpathlineto{\pgfqpoint{4.174740in}{1.884811in}}%
\pgfpathlineto{\pgfqpoint{4.167129in}{1.877168in}}%
\pgfpathlineto{\pgfqpoint{4.159512in}{1.869595in}}%
\pgfpathlineto{\pgfqpoint{4.151890in}{1.862095in}}%
\pgfpathlineto{\pgfqpoint{4.144262in}{1.854674in}}%
\pgfpathlineto{\pgfqpoint{4.130985in}{1.858326in}}%
\pgfpathlineto{\pgfqpoint{4.117715in}{1.862003in}}%
\pgfpathlineto{\pgfqpoint{4.104451in}{1.865706in}}%
\pgfpathlineto{\pgfqpoint{4.091193in}{1.869435in}}%
\pgfpathlineto{\pgfqpoint{4.098834in}{1.876670in}}%
\pgfpathlineto{\pgfqpoint{4.106469in}{1.883986in}}%
\pgfpathlineto{\pgfqpoint{4.114098in}{1.891379in}}%
\pgfpathlineto{\pgfqpoint{4.121722in}{1.898844in}}%
\pgfpathclose%
\pgfusepath{fill}%
\end{pgfscope}%
\begin{pgfscope}%
\pgfpathrectangle{\pgfqpoint{1.254980in}{0.150000in}}{\pgfqpoint{5.490039in}{5.490039in}}%
\pgfusepath{clip}%
\pgfsetbuttcap%
\pgfsetroundjoin%
\definecolor{currentfill}{rgb}{0.276022,0.044167,0.370164}%
\pgfsetfillcolor{currentfill}%
\pgfsetfillopacity{0.700000}%
\pgfsetlinewidth{0.000000pt}%
\definecolor{currentstroke}{rgb}{0.000000,0.000000,0.000000}%
\pgfsetstrokecolor{currentstroke}%
\pgfsetdash{}{0pt}%
\pgfpathmoveto{\pgfqpoint{4.561542in}{1.961123in}}%
\pgfpathlineto{\pgfqpoint{4.574903in}{1.958776in}}%
\pgfpathlineto{\pgfqpoint{4.588271in}{1.956453in}}%
\pgfpathlineto{\pgfqpoint{4.601646in}{1.954155in}}%
\pgfpathlineto{\pgfqpoint{4.615029in}{1.951881in}}%
\pgfpathlineto{\pgfqpoint{4.607566in}{1.942900in}}%
\pgfpathlineto{\pgfqpoint{4.600099in}{1.933910in}}%
\pgfpathlineto{\pgfqpoint{4.592626in}{1.924915in}}%
\pgfpathlineto{\pgfqpoint{4.585149in}{1.915917in}}%
\pgfpathlineto{\pgfqpoint{4.571756in}{1.918322in}}%
\pgfpathlineto{\pgfqpoint{4.558371in}{1.920752in}}%
\pgfpathlineto{\pgfqpoint{4.544994in}{1.923206in}}%
\pgfpathlineto{\pgfqpoint{4.531623in}{1.925684in}}%
\pgfpathlineto{\pgfqpoint{4.539110in}{1.934546in}}%
\pgfpathlineto{\pgfqpoint{4.546593in}{1.943408in}}%
\pgfpathlineto{\pgfqpoint{4.554070in}{1.952268in}}%
\pgfpathlineto{\pgfqpoint{4.561542in}{1.961123in}}%
\pgfpathclose%
\pgfusepath{fill}%
\end{pgfscope}%
\begin{pgfscope}%
\pgfpathrectangle{\pgfqpoint{1.254980in}{0.150000in}}{\pgfqpoint{5.490039in}{5.490039in}}%
\pgfusepath{clip}%
\pgfsetbuttcap%
\pgfsetroundjoin%
\definecolor{currentfill}{rgb}{0.282290,0.145912,0.461510}%
\pgfsetfillcolor{currentfill}%
\pgfsetfillopacity{0.700000}%
\pgfsetlinewidth{0.000000pt}%
\definecolor{currentstroke}{rgb}{0.000000,0.000000,0.000000}%
\pgfsetstrokecolor{currentstroke}%
\pgfsetdash{}{0pt}%
\pgfpathmoveto{\pgfqpoint{5.251737in}{2.148378in}}%
\pgfpathlineto{\pgfqpoint{5.265306in}{2.147453in}}%
\pgfpathlineto{\pgfqpoint{5.278884in}{2.146553in}}%
\pgfpathlineto{\pgfqpoint{5.292470in}{2.145676in}}%
\pgfpathlineto{\pgfqpoint{5.306064in}{2.144823in}}%
\pgfpathlineto{\pgfqpoint{5.298859in}{2.136612in}}%
\pgfpathlineto{\pgfqpoint{5.291647in}{2.128316in}}%
\pgfpathlineto{\pgfqpoint{5.284428in}{2.119934in}}%
\pgfpathlineto{\pgfqpoint{5.277201in}{2.111468in}}%
\pgfpathlineto{\pgfqpoint{5.263596in}{2.112361in}}%
\pgfpathlineto{\pgfqpoint{5.250000in}{2.113278in}}%
\pgfpathlineto{\pgfqpoint{5.236411in}{2.114219in}}%
\pgfpathlineto{\pgfqpoint{5.222831in}{2.115183in}}%
\pgfpathlineto{\pgfqpoint{5.230068in}{2.123604in}}%
\pgfpathlineto{\pgfqpoint{5.237298in}{2.131944in}}%
\pgfpathlineto{\pgfqpoint{5.244521in}{2.140202in}}%
\pgfpathlineto{\pgfqpoint{5.251737in}{2.148378in}}%
\pgfpathclose%
\pgfusepath{fill}%
\end{pgfscope}%
\begin{pgfscope}%
\pgfpathrectangle{\pgfqpoint{1.254980in}{0.150000in}}{\pgfqpoint{5.490039in}{5.490039in}}%
\pgfusepath{clip}%
\pgfsetbuttcap%
\pgfsetroundjoin%
\definecolor{currentfill}{rgb}{0.281924,0.089666,0.412415}%
\pgfsetfillcolor{currentfill}%
\pgfsetfillopacity{0.700000}%
\pgfsetlinewidth{0.000000pt}%
\definecolor{currentstroke}{rgb}{0.000000,0.000000,0.000000}%
\pgfsetstrokecolor{currentstroke}%
\pgfsetdash{}{0pt}%
\pgfpathmoveto{\pgfqpoint{4.864973in}{2.036436in}}%
\pgfpathlineto{\pgfqpoint{4.878423in}{2.034793in}}%
\pgfpathlineto{\pgfqpoint{4.891881in}{2.033175in}}%
\pgfpathlineto{\pgfqpoint{4.905347in}{2.031580in}}%
\pgfpathlineto{\pgfqpoint{4.918820in}{2.030009in}}%
\pgfpathlineto{\pgfqpoint{4.911462in}{2.020979in}}%
\pgfpathlineto{\pgfqpoint{4.904097in}{2.011898in}}%
\pgfpathlineto{\pgfqpoint{4.896727in}{2.002769in}}%
\pgfpathlineto{\pgfqpoint{4.889351in}{1.993594in}}%
\pgfpathlineto{\pgfqpoint{4.875868in}{1.995257in}}%
\pgfpathlineto{\pgfqpoint{4.862393in}{1.996944in}}%
\pgfpathlineto{\pgfqpoint{4.848926in}{1.998656in}}%
\pgfpathlineto{\pgfqpoint{4.835466in}{2.000391in}}%
\pgfpathlineto{\pgfqpoint{4.842851in}{2.009469in}}%
\pgfpathlineto{\pgfqpoint{4.850231in}{2.018504in}}%
\pgfpathlineto{\pgfqpoint{4.857605in}{2.027494in}}%
\pgfpathlineto{\pgfqpoint{4.864973in}{2.036436in}}%
\pgfpathclose%
\pgfusepath{fill}%
\end{pgfscope}%
\begin{pgfscope}%
\pgfpathrectangle{\pgfqpoint{1.254980in}{0.150000in}}{\pgfqpoint{5.490039in}{5.490039in}}%
\pgfusepath{clip}%
\pgfsetbuttcap%
\pgfsetroundjoin%
\definecolor{currentfill}{rgb}{0.277134,0.185228,0.489898}%
\pgfsetfillcolor{currentfill}%
\pgfsetfillopacity{0.700000}%
\pgfsetlinewidth{0.000000pt}%
\definecolor{currentstroke}{rgb}{0.000000,0.000000,0.000000}%
\pgfsetstrokecolor{currentstroke}%
\pgfsetdash{}{0pt}%
\pgfpathmoveto{\pgfqpoint{2.871765in}{2.211420in}}%
\pgfpathlineto{\pgfqpoint{2.884816in}{2.203757in}}%
\pgfpathlineto{\pgfqpoint{2.897871in}{2.196131in}}%
\pgfpathlineto{\pgfqpoint{2.910929in}{2.188543in}}%
\pgfpathlineto{\pgfqpoint{2.923989in}{2.180993in}}%
\pgfpathlineto{\pgfqpoint{2.915681in}{2.183575in}}%
\pgfpathlineto{\pgfqpoint{2.907356in}{2.186479in}}%
\pgfpathlineto{\pgfqpoint{2.899013in}{2.189714in}}%
\pgfpathlineto{\pgfqpoint{2.890653in}{2.193287in}}%
\pgfpathlineto{\pgfqpoint{2.877560in}{2.201127in}}%
\pgfpathlineto{\pgfqpoint{2.864470in}{2.209004in}}%
\pgfpathlineto{\pgfqpoint{2.851384in}{2.216920in}}%
\pgfpathlineto{\pgfqpoint{2.838299in}{2.224873in}}%
\pgfpathlineto{\pgfqpoint{2.846693in}{2.221005in}}%
\pgfpathlineto{\pgfqpoint{2.855068in}{2.217479in}}%
\pgfpathlineto{\pgfqpoint{2.863425in}{2.214287in}}%
\pgfpathlineto{\pgfqpoint{2.871765in}{2.211420in}}%
\pgfpathclose%
\pgfusepath{fill}%
\end{pgfscope}%
\begin{pgfscope}%
\pgfpathrectangle{\pgfqpoint{1.254980in}{0.150000in}}{\pgfqpoint{5.490039in}{5.490039in}}%
\pgfusepath{clip}%
\pgfsetbuttcap%
\pgfsetroundjoin%
\definecolor{currentfill}{rgb}{0.276194,0.190074,0.493001}%
\pgfsetfillcolor{currentfill}%
\pgfsetfillopacity{0.700000}%
\pgfsetlinewidth{0.000000pt}%
\definecolor{currentstroke}{rgb}{0.000000,0.000000,0.000000}%
\pgfsetstrokecolor{currentstroke}%
\pgfsetdash{}{0pt}%
\pgfpathmoveto{\pgfqpoint{5.555322in}{2.228721in}}%
\pgfpathlineto{\pgfqpoint{5.568992in}{2.228218in}}%
\pgfpathlineto{\pgfqpoint{5.582670in}{2.227738in}}%
\pgfpathlineto{\pgfqpoint{5.596356in}{2.227283in}}%
\pgfpathlineto{\pgfqpoint{5.610052in}{2.226850in}}%
\pgfpathlineto{\pgfqpoint{5.602987in}{2.219693in}}%
\pgfpathlineto{\pgfqpoint{5.595913in}{2.212438in}}%
\pgfpathlineto{\pgfqpoint{5.588831in}{2.205084in}}%
\pgfpathlineto{\pgfqpoint{5.581741in}{2.197631in}}%
\pgfpathlineto{\pgfqpoint{5.568033in}{2.198063in}}%
\pgfpathlineto{\pgfqpoint{5.554333in}{2.198519in}}%
\pgfpathlineto{\pgfqpoint{5.540642in}{2.198999in}}%
\pgfpathlineto{\pgfqpoint{5.526960in}{2.199502in}}%
\pgfpathlineto{\pgfqpoint{5.534063in}{2.206950in}}%
\pgfpathlineto{\pgfqpoint{5.541157in}{2.214302in}}%
\pgfpathlineto{\pgfqpoint{5.548244in}{2.221559in}}%
\pgfpathlineto{\pgfqpoint{5.555322in}{2.228721in}}%
\pgfpathclose%
\pgfusepath{fill}%
\end{pgfscope}%
\begin{pgfscope}%
\pgfpathrectangle{\pgfqpoint{1.254980in}{0.150000in}}{\pgfqpoint{5.490039in}{5.490039in}}%
\pgfusepath{clip}%
\pgfsetbuttcap%
\pgfsetroundjoin%
\definecolor{currentfill}{rgb}{0.269944,0.014625,0.341379}%
\pgfsetfillcolor{currentfill}%
\pgfsetfillopacity{0.700000}%
\pgfsetlinewidth{0.000000pt}%
\definecolor{currentstroke}{rgb}{0.000000,0.000000,0.000000}%
\pgfsetstrokecolor{currentstroke}%
\pgfsetdash{}{0pt}%
\pgfpathmoveto{\pgfqpoint{3.765336in}{1.903140in}}%
\pgfpathlineto{\pgfqpoint{3.778509in}{1.898483in}}%
\pgfpathlineto{\pgfqpoint{3.791687in}{1.893854in}}%
\pgfpathlineto{\pgfqpoint{3.804871in}{1.889251in}}%
\pgfpathlineto{\pgfqpoint{3.818060in}{1.884676in}}%
\pgfpathlineto{\pgfqpoint{3.810301in}{1.879254in}}%
\pgfpathlineto{\pgfqpoint{3.802536in}{1.873975in}}%
\pgfpathlineto{\pgfqpoint{3.794763in}{1.868847in}}%
\pgfpathlineto{\pgfqpoint{3.786982in}{1.863875in}}%
\pgfpathlineto{\pgfqpoint{3.773776in}{1.868671in}}%
\pgfpathlineto{\pgfqpoint{3.760575in}{1.873493in}}%
\pgfpathlineto{\pgfqpoint{3.747380in}{1.878343in}}%
\pgfpathlineto{\pgfqpoint{3.734190in}{1.883220in}}%
\pgfpathlineto{\pgfqpoint{3.741988in}{1.887967in}}%
\pgfpathlineto{\pgfqpoint{3.749778in}{1.892874in}}%
\pgfpathlineto{\pgfqpoint{3.757561in}{1.897933in}}%
\pgfpathlineto{\pgfqpoint{3.765336in}{1.903140in}}%
\pgfpathclose%
\pgfusepath{fill}%
\end{pgfscope}%
\begin{pgfscope}%
\pgfpathrectangle{\pgfqpoint{1.254980in}{0.150000in}}{\pgfqpoint{5.490039in}{5.490039in}}%
\pgfusepath{clip}%
\pgfsetbuttcap%
\pgfsetroundjoin%
\definecolor{currentfill}{rgb}{0.282884,0.135920,0.453427}%
\pgfsetfillcolor{currentfill}%
\pgfsetfillopacity{0.700000}%
\pgfsetlinewidth{0.000000pt}%
\definecolor{currentstroke}{rgb}{0.000000,0.000000,0.000000}%
\pgfsetstrokecolor{currentstroke}%
\pgfsetdash{}{0pt}%
\pgfpathmoveto{\pgfqpoint{5.168593in}{2.119280in}}%
\pgfpathlineto{\pgfqpoint{5.182140in}{2.118220in}}%
\pgfpathlineto{\pgfqpoint{5.195695in}{2.117184in}}%
\pgfpathlineto{\pgfqpoint{5.209259in}{2.116172in}}%
\pgfpathlineto{\pgfqpoint{5.222831in}{2.115183in}}%
\pgfpathlineto{\pgfqpoint{5.215587in}{2.106682in}}%
\pgfpathlineto{\pgfqpoint{5.208337in}{2.098100in}}%
\pgfpathlineto{\pgfqpoint{5.201080in}{2.089440in}}%
\pgfpathlineto{\pgfqpoint{5.193816in}{2.080702in}}%
\pgfpathlineto{\pgfqpoint{5.180234in}{2.081744in}}%
\pgfpathlineto{\pgfqpoint{5.166660in}{2.082809in}}%
\pgfpathlineto{\pgfqpoint{5.153094in}{2.083899in}}%
\pgfpathlineto{\pgfqpoint{5.139537in}{2.085012in}}%
\pgfpathlineto{\pgfqpoint{5.146811in}{2.093692in}}%
\pgfpathlineto{\pgfqpoint{5.154078in}{2.102297in}}%
\pgfpathlineto{\pgfqpoint{5.161339in}{2.110827in}}%
\pgfpathlineto{\pgfqpoint{5.168593in}{2.119280in}}%
\pgfpathclose%
\pgfusepath{fill}%
\end{pgfscope}%
\begin{pgfscope}%
\pgfpathrectangle{\pgfqpoint{1.254980in}{0.150000in}}{\pgfqpoint{5.490039in}{5.490039in}}%
\pgfusepath{clip}%
\pgfsetbuttcap%
\pgfsetroundjoin%
\definecolor{currentfill}{rgb}{0.268510,0.009605,0.335427}%
\pgfsetfillcolor{currentfill}%
\pgfsetfillopacity{0.700000}%
\pgfsetlinewidth{0.000000pt}%
\definecolor{currentstroke}{rgb}{0.000000,0.000000,0.000000}%
\pgfsetstrokecolor{currentstroke}%
\pgfsetdash{}{0pt}%
\pgfpathmoveto{\pgfqpoint{4.258201in}{1.903038in}}%
\pgfpathlineto{\pgfqpoint{4.271485in}{1.899863in}}%
\pgfpathlineto{\pgfqpoint{4.284775in}{1.896713in}}%
\pgfpathlineto{\pgfqpoint{4.298072in}{1.893588in}}%
\pgfpathlineto{\pgfqpoint{4.311376in}{1.890489in}}%
\pgfpathlineto{\pgfqpoint{4.303810in}{1.882282in}}%
\pgfpathlineto{\pgfqpoint{4.296238in}{1.874122in}}%
\pgfpathlineto{\pgfqpoint{4.288662in}{1.866012in}}%
\pgfpathlineto{\pgfqpoint{4.281080in}{1.857956in}}%
\pgfpathlineto{\pgfqpoint{4.267765in}{1.861225in}}%
\pgfpathlineto{\pgfqpoint{4.254457in}{1.864519in}}%
\pgfpathlineto{\pgfqpoint{4.241155in}{1.867838in}}%
\pgfpathlineto{\pgfqpoint{4.227859in}{1.871182in}}%
\pgfpathlineto{\pgfqpoint{4.235453in}{1.879064in}}%
\pgfpathlineto{\pgfqpoint{4.243041in}{1.887003in}}%
\pgfpathlineto{\pgfqpoint{4.250624in}{1.894996in}}%
\pgfpathlineto{\pgfqpoint{4.258201in}{1.903038in}}%
\pgfpathclose%
\pgfusepath{fill}%
\end{pgfscope}%
\begin{pgfscope}%
\pgfpathrectangle{\pgfqpoint{1.254980in}{0.150000in}}{\pgfqpoint{5.490039in}{5.490039in}}%
\pgfusepath{clip}%
\pgfsetbuttcap%
\pgfsetroundjoin%
\definecolor{currentfill}{rgb}{0.257322,0.256130,0.526563}%
\pgfsetfillcolor{currentfill}%
\pgfsetfillopacity{0.700000}%
\pgfsetlinewidth{0.000000pt}%
\definecolor{currentstroke}{rgb}{0.000000,0.000000,0.000000}%
\pgfsetstrokecolor{currentstroke}%
\pgfsetdash{}{0pt}%
\pgfpathmoveto{\pgfqpoint{2.629307in}{2.357555in}}%
\pgfpathlineto{\pgfqpoint{2.642352in}{2.348949in}}%
\pgfpathlineto{\pgfqpoint{2.655398in}{2.340386in}}%
\pgfpathlineto{\pgfqpoint{2.668447in}{2.331866in}}%
\pgfpathlineto{\pgfqpoint{2.681498in}{2.323389in}}%
\pgfpathlineto{\pgfqpoint{2.672982in}{2.328505in}}%
\pgfpathlineto{\pgfqpoint{2.664445in}{2.333990in}}%
\pgfpathlineto{\pgfqpoint{2.655887in}{2.339852in}}%
\pgfpathlineto{\pgfqpoint{2.647308in}{2.346100in}}%
\pgfpathlineto{\pgfqpoint{2.634220in}{2.354883in}}%
\pgfpathlineto{\pgfqpoint{2.621135in}{2.363709in}}%
\pgfpathlineto{\pgfqpoint{2.608051in}{2.372578in}}%
\pgfpathlineto{\pgfqpoint{2.594970in}{2.381491in}}%
\pgfpathlineto{\pgfqpoint{2.603587in}{2.374932in}}%
\pgfpathlineto{\pgfqpoint{2.612182in}{2.368762in}}%
\pgfpathlineto{\pgfqpoint{2.620755in}{2.362972in}}%
\pgfpathlineto{\pgfqpoint{2.629307in}{2.357555in}}%
\pgfpathclose%
\pgfusepath{fill}%
\end{pgfscope}%
\begin{pgfscope}%
\pgfpathrectangle{\pgfqpoint{1.254980in}{0.150000in}}{\pgfqpoint{5.490039in}{5.490039in}}%
\pgfusepath{clip}%
\pgfsetbuttcap%
\pgfsetroundjoin%
\definecolor{currentfill}{rgb}{0.267004,0.004874,0.329415}%
\pgfsetfillcolor{currentfill}%
\pgfsetfillopacity{0.700000}%
\pgfsetlinewidth{0.000000pt}%
\definecolor{currentstroke}{rgb}{0.000000,0.000000,0.000000}%
\pgfsetstrokecolor{currentstroke}%
\pgfsetdash{}{0pt}%
\pgfpathmoveto{\pgfqpoint{3.901771in}{1.890497in}}%
\pgfpathlineto{\pgfqpoint{3.914972in}{1.886262in}}%
\pgfpathlineto{\pgfqpoint{3.928179in}{1.882054in}}%
\pgfpathlineto{\pgfqpoint{3.941392in}{1.877872in}}%
\pgfpathlineto{\pgfqpoint{3.954610in}{1.873716in}}%
\pgfpathlineto{\pgfqpoint{3.946911in}{1.867364in}}%
\pgfpathlineto{\pgfqpoint{3.939204in}{1.861129in}}%
\pgfpathlineto{\pgfqpoint{3.931492in}{1.855016in}}%
\pgfpathlineto{\pgfqpoint{3.923772in}{1.849030in}}%
\pgfpathlineto{\pgfqpoint{3.910539in}{1.853394in}}%
\pgfpathlineto{\pgfqpoint{3.897311in}{1.857783in}}%
\pgfpathlineto{\pgfqpoint{3.884088in}{1.862199in}}%
\pgfpathlineto{\pgfqpoint{3.870871in}{1.866641in}}%
\pgfpathlineto{\pgfqpoint{3.878606in}{1.872415in}}%
\pgfpathlineto{\pgfqpoint{3.886335in}{1.878319in}}%
\pgfpathlineto{\pgfqpoint{3.894056in}{1.884348in}}%
\pgfpathlineto{\pgfqpoint{3.901771in}{1.890497in}}%
\pgfpathclose%
\pgfusepath{fill}%
\end{pgfscope}%
\begin{pgfscope}%
\pgfpathrectangle{\pgfqpoint{1.254980in}{0.150000in}}{\pgfqpoint{5.490039in}{5.490039in}}%
\pgfusepath{clip}%
\pgfsetbuttcap%
\pgfsetroundjoin%
\definecolor{currentfill}{rgb}{0.281446,0.084320,0.407414}%
\pgfsetfillcolor{currentfill}%
\pgfsetfillopacity{0.700000}%
\pgfsetlinewidth{0.000000pt}%
\definecolor{currentstroke}{rgb}{0.000000,0.000000,0.000000}%
\pgfsetstrokecolor{currentstroke}%
\pgfsetdash{}{0pt}%
\pgfpathmoveto{\pgfqpoint{3.303056in}{2.013161in}}%
\pgfpathlineto{\pgfqpoint{3.316154in}{2.006993in}}%
\pgfpathlineto{\pgfqpoint{3.329257in}{2.000856in}}%
\pgfpathlineto{\pgfqpoint{3.342363in}{1.994750in}}%
\pgfpathlineto{\pgfqpoint{3.355474in}{1.988674in}}%
\pgfpathlineto{\pgfqpoint{3.347468in}{1.987154in}}%
\pgfpathlineto{\pgfqpoint{3.339449in}{1.985875in}}%
\pgfpathlineto{\pgfqpoint{3.331419in}{1.984843in}}%
\pgfpathlineto{\pgfqpoint{3.323377in}{1.984065in}}%
\pgfpathlineto{\pgfqpoint{3.310242in}{1.990400in}}%
\pgfpathlineto{\pgfqpoint{3.297111in}{1.996767in}}%
\pgfpathlineto{\pgfqpoint{3.283984in}{2.003164in}}%
\pgfpathlineto{\pgfqpoint{3.270861in}{2.009593in}}%
\pgfpathlineto{\pgfqpoint{3.278928in}{2.010106in}}%
\pgfpathlineto{\pgfqpoint{3.286983in}{2.010876in}}%
\pgfpathlineto{\pgfqpoint{3.295025in}{2.011897in}}%
\pgfpathlineto{\pgfqpoint{3.303056in}{2.013161in}}%
\pgfpathclose%
\pgfusepath{fill}%
\end{pgfscope}%
\begin{pgfscope}%
\pgfpathrectangle{\pgfqpoint{1.254980in}{0.150000in}}{\pgfqpoint{5.490039in}{5.490039in}}%
\pgfusepath{clip}%
\pgfsetbuttcap%
\pgfsetroundjoin%
\definecolor{currentfill}{rgb}{0.280267,0.073417,0.397163}%
\pgfsetfillcolor{currentfill}%
\pgfsetfillopacity{0.700000}%
\pgfsetlinewidth{0.000000pt}%
\definecolor{currentstroke}{rgb}{0.000000,0.000000,0.000000}%
\pgfsetstrokecolor{currentstroke}%
\pgfsetdash{}{0pt}%
\pgfpathmoveto{\pgfqpoint{4.781703in}{2.007573in}}%
\pgfpathlineto{\pgfqpoint{4.795132in}{2.005742in}}%
\pgfpathlineto{\pgfqpoint{4.808569in}{2.003934in}}%
\pgfpathlineto{\pgfqpoint{4.822014in}{2.002150in}}%
\pgfpathlineto{\pgfqpoint{4.835466in}{2.000391in}}%
\pgfpathlineto{\pgfqpoint{4.828075in}{1.991272in}}%
\pgfpathlineto{\pgfqpoint{4.820679in}{1.982114in}}%
\pgfpathlineto{\pgfqpoint{4.813277in}{1.972919in}}%
\pgfpathlineto{\pgfqpoint{4.805870in}{1.963691in}}%
\pgfpathlineto{\pgfqpoint{4.792408in}{1.965556in}}%
\pgfpathlineto{\pgfqpoint{4.778955in}{1.967445in}}%
\pgfpathlineto{\pgfqpoint{4.765508in}{1.969358in}}%
\pgfpathlineto{\pgfqpoint{4.752069in}{1.971295in}}%
\pgfpathlineto{\pgfqpoint{4.759486in}{1.980414in}}%
\pgfpathlineto{\pgfqpoint{4.766897in}{1.989501in}}%
\pgfpathlineto{\pgfqpoint{4.774303in}{1.998555in}}%
\pgfpathlineto{\pgfqpoint{4.781703in}{2.007573in}}%
\pgfpathclose%
\pgfusepath{fill}%
\end{pgfscope}%
\begin{pgfscope}%
\pgfpathrectangle{\pgfqpoint{1.254980in}{0.150000in}}{\pgfqpoint{5.490039in}{5.490039in}}%
\pgfusepath{clip}%
\pgfsetbuttcap%
\pgfsetroundjoin%
\definecolor{currentfill}{rgb}{0.272594,0.025563,0.353093}%
\pgfsetfillcolor{currentfill}%
\pgfsetfillopacity{0.700000}%
\pgfsetlinewidth{0.000000pt}%
\definecolor{currentstroke}{rgb}{0.000000,0.000000,0.000000}%
\pgfsetstrokecolor{currentstroke}%
\pgfsetdash{}{0pt}%
\pgfpathmoveto{\pgfqpoint{3.628856in}{1.923225in}}%
\pgfpathlineto{\pgfqpoint{3.642004in}{1.918127in}}%
\pgfpathlineto{\pgfqpoint{3.655158in}{1.913057in}}%
\pgfpathlineto{\pgfqpoint{3.668317in}{1.908015in}}%
\pgfpathlineto{\pgfqpoint{3.681481in}{1.903001in}}%
\pgfpathlineto{\pgfqpoint{3.673657in}{1.898647in}}%
\pgfpathlineto{\pgfqpoint{3.665824in}{1.894466in}}%
\pgfpathlineto{\pgfqpoint{3.657983in}{1.890464in}}%
\pgfpathlineto{\pgfqpoint{3.650133in}{1.886648in}}%
\pgfpathlineto{\pgfqpoint{3.636950in}{1.891896in}}%
\pgfpathlineto{\pgfqpoint{3.623772in}{1.897171in}}%
\pgfpathlineto{\pgfqpoint{3.610599in}{1.902474in}}%
\pgfpathlineto{\pgfqpoint{3.597431in}{1.907806in}}%
\pgfpathlineto{\pgfqpoint{3.605300in}{1.911383in}}%
\pgfpathlineto{\pgfqpoint{3.613161in}{1.915150in}}%
\pgfpathlineto{\pgfqpoint{3.621013in}{1.919099in}}%
\pgfpathlineto{\pgfqpoint{3.628856in}{1.923225in}}%
\pgfpathclose%
\pgfusepath{fill}%
\end{pgfscope}%
\begin{pgfscope}%
\pgfpathrectangle{\pgfqpoint{1.254980in}{0.150000in}}{\pgfqpoint{5.490039in}{5.490039in}}%
\pgfusepath{clip}%
\pgfsetbuttcap%
\pgfsetroundjoin%
\definecolor{currentfill}{rgb}{0.283187,0.125848,0.444960}%
\pgfsetfillcolor{currentfill}%
\pgfsetfillopacity{0.700000}%
\pgfsetlinewidth{0.000000pt}%
\definecolor{currentstroke}{rgb}{0.000000,0.000000,0.000000}%
\pgfsetstrokecolor{currentstroke}%
\pgfsetdash{}{0pt}%
\pgfpathmoveto{\pgfqpoint{3.113696in}{2.089230in}}%
\pgfpathlineto{\pgfqpoint{3.126772in}{2.082413in}}%
\pgfpathlineto{\pgfqpoint{3.139851in}{2.075630in}}%
\pgfpathlineto{\pgfqpoint{3.152935in}{2.068880in}}%
\pgfpathlineto{\pgfqpoint{3.166022in}{2.062163in}}%
\pgfpathlineto{\pgfqpoint{3.157890in}{2.062453in}}%
\pgfpathlineto{\pgfqpoint{3.149744in}{2.063021in}}%
\pgfpathlineto{\pgfqpoint{3.141584in}{2.063875in}}%
\pgfpathlineto{\pgfqpoint{3.133410in}{2.065022in}}%
\pgfpathlineto{\pgfqpoint{3.120295in}{2.072014in}}%
\pgfpathlineto{\pgfqpoint{3.107184in}{2.079038in}}%
\pgfpathlineto{\pgfqpoint{3.094076in}{2.086096in}}%
\pgfpathlineto{\pgfqpoint{3.080972in}{2.093188in}}%
\pgfpathlineto{\pgfqpoint{3.089175in}{2.091761in}}%
\pgfpathlineto{\pgfqpoint{3.097363in}{2.090630in}}%
\pgfpathlineto{\pgfqpoint{3.105536in}{2.089789in}}%
\pgfpathlineto{\pgfqpoint{3.113696in}{2.089230in}}%
\pgfpathclose%
\pgfusepath{fill}%
\end{pgfscope}%
\begin{pgfscope}%
\pgfpathrectangle{\pgfqpoint{1.254980in}{0.150000in}}{\pgfqpoint{5.490039in}{5.490039in}}%
\pgfusepath{clip}%
\pgfsetbuttcap%
\pgfsetroundjoin%
\definecolor{currentfill}{rgb}{0.273809,0.031497,0.358853}%
\pgfsetfillcolor{currentfill}%
\pgfsetfillopacity{0.700000}%
\pgfsetlinewidth{0.000000pt}%
\definecolor{currentstroke}{rgb}{0.000000,0.000000,0.000000}%
\pgfsetstrokecolor{currentstroke}%
\pgfsetdash{}{0pt}%
\pgfpathmoveto{\pgfqpoint{4.478209in}{1.935843in}}%
\pgfpathlineto{\pgfqpoint{4.491552in}{1.933266in}}%
\pgfpathlineto{\pgfqpoint{4.504902in}{1.930714in}}%
\pgfpathlineto{\pgfqpoint{4.518259in}{1.928187in}}%
\pgfpathlineto{\pgfqpoint{4.531623in}{1.925684in}}%
\pgfpathlineto{\pgfqpoint{4.524130in}{1.916827in}}%
\pgfpathlineto{\pgfqpoint{4.516632in}{1.907977in}}%
\pgfpathlineto{\pgfqpoint{4.509129in}{1.899137in}}%
\pgfpathlineto{\pgfqpoint{4.501621in}{1.890312in}}%
\pgfpathlineto{\pgfqpoint{4.488247in}{1.892958in}}%
\pgfpathlineto{\pgfqpoint{4.474880in}{1.895630in}}%
\pgfpathlineto{\pgfqpoint{4.461520in}{1.898325in}}%
\pgfpathlineto{\pgfqpoint{4.448167in}{1.901046in}}%
\pgfpathlineto{\pgfqpoint{4.455685in}{1.909722in}}%
\pgfpathlineto{\pgfqpoint{4.463198in}{1.918416in}}%
\pgfpathlineto{\pgfqpoint{4.470706in}{1.927124in}}%
\pgfpathlineto{\pgfqpoint{4.478209in}{1.935843in}}%
\pgfpathclose%
\pgfusepath{fill}%
\end{pgfscope}%
\begin{pgfscope}%
\pgfpathrectangle{\pgfqpoint{1.254980in}{0.150000in}}{\pgfqpoint{5.490039in}{5.490039in}}%
\pgfusepath{clip}%
\pgfsetbuttcap%
\pgfsetroundjoin%
\definecolor{currentfill}{rgb}{0.269308,0.218818,0.509577}%
\pgfsetfillcolor{currentfill}%
\pgfsetfillopacity{0.700000}%
\pgfsetlinewidth{0.000000pt}%
\definecolor{currentstroke}{rgb}{0.000000,0.000000,0.000000}%
\pgfsetstrokecolor{currentstroke}%
\pgfsetdash{}{0pt}%
\pgfpathmoveto{\pgfqpoint{5.775938in}{2.277756in}}%
\pgfpathlineto{\pgfqpoint{5.789685in}{2.277497in}}%
\pgfpathlineto{\pgfqpoint{5.803442in}{2.277262in}}%
\pgfpathlineto{\pgfqpoint{5.817207in}{2.277050in}}%
\pgfpathlineto{\pgfqpoint{5.810248in}{2.270733in}}%
\pgfpathlineto{\pgfqpoint{5.803280in}{2.264316in}}%
\pgfpathlineto{\pgfqpoint{5.796303in}{2.257799in}}%
\pgfpathlineto{\pgfqpoint{5.789318in}{2.251180in}}%
\pgfpathlineto{\pgfqpoint{5.775538in}{2.251364in}}%
\pgfpathlineto{\pgfqpoint{5.761767in}{2.251573in}}%
\pgfpathlineto{\pgfqpoint{5.748005in}{2.251805in}}%
\pgfpathlineto{\pgfqpoint{5.755002in}{2.258440in}}%
\pgfpathlineto{\pgfqpoint{5.761989in}{2.264976in}}%
\pgfpathlineto{\pgfqpoint{5.768968in}{2.271414in}}%
\pgfpathlineto{\pgfqpoint{5.775938in}{2.277756in}}%
\pgfpathclose%
\pgfusepath{fill}%
\end{pgfscope}%
\begin{pgfscope}%
\pgfpathrectangle{\pgfqpoint{1.254980in}{0.150000in}}{\pgfqpoint{5.490039in}{5.490039in}}%
\pgfusepath{clip}%
\pgfsetbuttcap%
\pgfsetroundjoin%
\definecolor{currentfill}{rgb}{0.267004,0.004874,0.329415}%
\pgfsetfillcolor{currentfill}%
\pgfsetfillopacity{0.700000}%
\pgfsetlinewidth{0.000000pt}%
\definecolor{currentstroke}{rgb}{0.000000,0.000000,0.000000}%
\pgfsetstrokecolor{currentstroke}%
\pgfsetdash{}{0pt}%
\pgfpathmoveto{\pgfqpoint{4.038221in}{1.884606in}}%
\pgfpathlineto{\pgfqpoint{4.051455in}{1.880774in}}%
\pgfpathlineto{\pgfqpoint{4.064695in}{1.876969in}}%
\pgfpathlineto{\pgfqpoint{4.077941in}{1.873189in}}%
\pgfpathlineto{\pgfqpoint{4.091193in}{1.869435in}}%
\pgfpathlineto{\pgfqpoint{4.083546in}{1.862286in}}%
\pgfpathlineto{\pgfqpoint{4.075893in}{1.855228in}}%
\pgfpathlineto{\pgfqpoint{4.068234in}{1.848266in}}%
\pgfpathlineto{\pgfqpoint{4.060569in}{1.841403in}}%
\pgfpathlineto{\pgfqpoint{4.047304in}{1.845352in}}%
\pgfpathlineto{\pgfqpoint{4.034044in}{1.849327in}}%
\pgfpathlineto{\pgfqpoint{4.020790in}{1.853327in}}%
\pgfpathlineto{\pgfqpoint{4.007543in}{1.857353in}}%
\pgfpathlineto{\pgfqpoint{4.015221in}{1.864016in}}%
\pgfpathlineto{\pgfqpoint{4.022894in}{1.870782in}}%
\pgfpathlineto{\pgfqpoint{4.030560in}{1.877647in}}%
\pgfpathlineto{\pgfqpoint{4.038221in}{1.884606in}}%
\pgfpathclose%
\pgfusepath{fill}%
\end{pgfscope}%
\begin{pgfscope}%
\pgfpathrectangle{\pgfqpoint{1.254980in}{0.150000in}}{\pgfqpoint{5.490039in}{5.490039in}}%
\pgfusepath{clip}%
\pgfsetbuttcap%
\pgfsetroundjoin%
\definecolor{currentfill}{rgb}{0.278012,0.180367,0.486697}%
\pgfsetfillcolor{currentfill}%
\pgfsetfillopacity{0.700000}%
\pgfsetlinewidth{0.000000pt}%
\definecolor{currentstroke}{rgb}{0.000000,0.000000,0.000000}%
\pgfsetstrokecolor{currentstroke}%
\pgfsetdash{}{0pt}%
\pgfpathmoveto{\pgfqpoint{5.472318in}{2.201753in}}%
\pgfpathlineto{\pgfqpoint{5.485966in}{2.201155in}}%
\pgfpathlineto{\pgfqpoint{5.499622in}{2.200580in}}%
\pgfpathlineto{\pgfqpoint{5.513287in}{2.200029in}}%
\pgfpathlineto{\pgfqpoint{5.526960in}{2.199502in}}%
\pgfpathlineto{\pgfqpoint{5.519850in}{2.191958in}}%
\pgfpathlineto{\pgfqpoint{5.512732in}{2.184318in}}%
\pgfpathlineto{\pgfqpoint{5.505606in}{2.176581in}}%
\pgfpathlineto{\pgfqpoint{5.498472in}{2.168747in}}%
\pgfpathlineto{\pgfqpoint{5.484786in}{2.169288in}}%
\pgfpathlineto{\pgfqpoint{5.471110in}{2.169852in}}%
\pgfpathlineto{\pgfqpoint{5.457441in}{2.170440in}}%
\pgfpathlineto{\pgfqpoint{5.443782in}{2.171052in}}%
\pgfpathlineto{\pgfqpoint{5.450928in}{2.178867in}}%
\pgfpathlineto{\pgfqpoint{5.458066in}{2.186589in}}%
\pgfpathlineto{\pgfqpoint{5.465196in}{2.194218in}}%
\pgfpathlineto{\pgfqpoint{5.472318in}{2.201753in}}%
\pgfpathclose%
\pgfusepath{fill}%
\end{pgfscope}%
\begin{pgfscope}%
\pgfpathrectangle{\pgfqpoint{1.254980in}{0.150000in}}{\pgfqpoint{5.490039in}{5.490039in}}%
\pgfusepath{clip}%
\pgfsetbuttcap%
\pgfsetroundjoin%
\definecolor{currentfill}{rgb}{0.283229,0.120777,0.440584}%
\pgfsetfillcolor{currentfill}%
\pgfsetfillopacity{0.700000}%
\pgfsetlinewidth{0.000000pt}%
\definecolor{currentstroke}{rgb}{0.000000,0.000000,0.000000}%
\pgfsetstrokecolor{currentstroke}%
\pgfsetdash{}{0pt}%
\pgfpathmoveto{\pgfqpoint{5.085388in}{2.089704in}}%
\pgfpathlineto{\pgfqpoint{5.098913in}{2.088495in}}%
\pgfpathlineto{\pgfqpoint{5.112446in}{2.087310in}}%
\pgfpathlineto{\pgfqpoint{5.125987in}{2.086149in}}%
\pgfpathlineto{\pgfqpoint{5.139537in}{2.085012in}}%
\pgfpathlineto{\pgfqpoint{5.132256in}{2.076259in}}%
\pgfpathlineto{\pgfqpoint{5.124969in}{2.067433in}}%
\pgfpathlineto{\pgfqpoint{5.117676in}{2.058536in}}%
\pgfpathlineto{\pgfqpoint{5.110376in}{2.049569in}}%
\pgfpathlineto{\pgfqpoint{5.096817in}{2.050773in}}%
\pgfpathlineto{\pgfqpoint{5.083266in}{2.052000in}}%
\pgfpathlineto{\pgfqpoint{5.069723in}{2.053252in}}%
\pgfpathlineto{\pgfqpoint{5.056188in}{2.054527in}}%
\pgfpathlineto{\pgfqpoint{5.063498in}{2.063422in}}%
\pgfpathlineto{\pgfqpoint{5.070801in}{2.072251in}}%
\pgfpathlineto{\pgfqpoint{5.078098in}{2.081012in}}%
\pgfpathlineto{\pgfqpoint{5.085388in}{2.089704in}}%
\pgfpathclose%
\pgfusepath{fill}%
\end{pgfscope}%
\begin{pgfscope}%
\pgfpathrectangle{\pgfqpoint{1.254980in}{0.150000in}}{\pgfqpoint{5.490039in}{5.490039in}}%
\pgfusepath{clip}%
\pgfsetbuttcap%
\pgfsetroundjoin%
\definecolor{currentfill}{rgb}{0.277018,0.050344,0.375715}%
\pgfsetfillcolor{currentfill}%
\pgfsetfillopacity{0.700000}%
\pgfsetlinewidth{0.000000pt}%
\definecolor{currentstroke}{rgb}{0.000000,0.000000,0.000000}%
\pgfsetstrokecolor{currentstroke}%
\pgfsetdash{}{0pt}%
\pgfpathmoveto{\pgfqpoint{3.492261in}{1.951480in}}%
\pgfpathlineto{\pgfqpoint{3.505390in}{1.945920in}}%
\pgfpathlineto{\pgfqpoint{3.518525in}{1.940389in}}%
\pgfpathlineto{\pgfqpoint{3.531664in}{1.934887in}}%
\pgfpathlineto{\pgfqpoint{3.544807in}{1.929413in}}%
\pgfpathlineto{\pgfqpoint{3.536908in}{1.926272in}}%
\pgfpathlineto{\pgfqpoint{3.529000in}{1.923336in}}%
\pgfpathlineto{\pgfqpoint{3.521082in}{1.920610in}}%
\pgfpathlineto{\pgfqpoint{3.513154in}{1.918101in}}%
\pgfpathlineto{\pgfqpoint{3.499989in}{1.923821in}}%
\pgfpathlineto{\pgfqpoint{3.486828in}{1.929570in}}%
\pgfpathlineto{\pgfqpoint{3.473672in}{1.935347in}}%
\pgfpathlineto{\pgfqpoint{3.460521in}{1.941154in}}%
\pgfpathlineto{\pgfqpoint{3.468471in}{1.943411in}}%
\pgfpathlineto{\pgfqpoint{3.476411in}{1.945889in}}%
\pgfpathlineto{\pgfqpoint{3.484341in}{1.948580in}}%
\pgfpathlineto{\pgfqpoint{3.492261in}{1.951480in}}%
\pgfpathclose%
\pgfusepath{fill}%
\end{pgfscope}%
\begin{pgfscope}%
\pgfpathrectangle{\pgfqpoint{1.254980in}{0.150000in}}{\pgfqpoint{5.490039in}{5.490039in}}%
\pgfusepath{clip}%
\pgfsetbuttcap%
\pgfsetroundjoin%
\definecolor{currentfill}{rgb}{0.278791,0.062145,0.386592}%
\pgfsetfillcolor{currentfill}%
\pgfsetfillopacity{0.700000}%
\pgfsetlinewidth{0.000000pt}%
\definecolor{currentstroke}{rgb}{0.000000,0.000000,0.000000}%
\pgfsetstrokecolor{currentstroke}%
\pgfsetdash{}{0pt}%
\pgfpathmoveto{\pgfqpoint{4.698388in}{1.979287in}}%
\pgfpathlineto{\pgfqpoint{4.711797in}{1.977252in}}%
\pgfpathlineto{\pgfqpoint{4.725214in}{1.975243in}}%
\pgfpathlineto{\pgfqpoint{4.738638in}{1.973257in}}%
\pgfpathlineto{\pgfqpoint{4.752069in}{1.971295in}}%
\pgfpathlineto{\pgfqpoint{4.744647in}{1.962149in}}%
\pgfpathlineto{\pgfqpoint{4.737220in}{1.952976in}}%
\pgfpathlineto{\pgfqpoint{4.729787in}{1.943780in}}%
\pgfpathlineto{\pgfqpoint{4.722349in}{1.934563in}}%
\pgfpathlineto{\pgfqpoint{4.708908in}{1.936643in}}%
\pgfpathlineto{\pgfqpoint{4.695475in}{1.938748in}}%
\pgfpathlineto{\pgfqpoint{4.682049in}{1.940876in}}%
\pgfpathlineto{\pgfqpoint{4.668630in}{1.943028in}}%
\pgfpathlineto{\pgfqpoint{4.676078in}{1.952122in}}%
\pgfpathlineto{\pgfqpoint{4.683520in}{1.961197in}}%
\pgfpathlineto{\pgfqpoint{4.690957in}{1.970253in}}%
\pgfpathlineto{\pgfqpoint{4.698388in}{1.979287in}}%
\pgfpathclose%
\pgfusepath{fill}%
\end{pgfscope}%
\begin{pgfscope}%
\pgfpathrectangle{\pgfqpoint{1.254980in}{0.150000in}}{\pgfqpoint{5.490039in}{5.490039in}}%
\pgfusepath{clip}%
\pgfsetbuttcap%
\pgfsetroundjoin%
\definecolor{currentfill}{rgb}{0.278826,0.175490,0.483397}%
\pgfsetfillcolor{currentfill}%
\pgfsetfillopacity{0.700000}%
\pgfsetlinewidth{0.000000pt}%
\definecolor{currentstroke}{rgb}{0.000000,0.000000,0.000000}%
\pgfsetstrokecolor{currentstroke}%
\pgfsetdash{}{0pt}%
\pgfpathmoveto{\pgfqpoint{2.923989in}{2.180993in}}%
\pgfpathlineto{\pgfqpoint{2.937053in}{2.173479in}}%
\pgfpathlineto{\pgfqpoint{2.950120in}{2.166002in}}%
\pgfpathlineto{\pgfqpoint{2.963191in}{2.158561in}}%
\pgfpathlineto{\pgfqpoint{2.976264in}{2.151157in}}%
\pgfpathlineto{\pgfqpoint{2.967986in}{2.153455in}}%
\pgfpathlineto{\pgfqpoint{2.959692in}{2.156072in}}%
\pgfpathlineto{\pgfqpoint{2.951382in}{2.159015in}}%
\pgfpathlineto{\pgfqpoint{2.943054in}{2.162293in}}%
\pgfpathlineto{\pgfqpoint{2.929949in}{2.169987in}}%
\pgfpathlineto{\pgfqpoint{2.916847in}{2.177717in}}%
\pgfpathlineto{\pgfqpoint{2.903749in}{2.185483in}}%
\pgfpathlineto{\pgfqpoint{2.890653in}{2.193287in}}%
\pgfpathlineto{\pgfqpoint{2.899013in}{2.189714in}}%
\pgfpathlineto{\pgfqpoint{2.907356in}{2.186479in}}%
\pgfpathlineto{\pgfqpoint{2.915681in}{2.183575in}}%
\pgfpathlineto{\pgfqpoint{2.923989in}{2.180993in}}%
\pgfpathclose%
\pgfusepath{fill}%
\end{pgfscope}%
\begin{pgfscope}%
\pgfpathrectangle{\pgfqpoint{1.254980in}{0.150000in}}{\pgfqpoint{5.490039in}{5.490039in}}%
\pgfusepath{clip}%
\pgfsetbuttcap%
\pgfsetroundjoin%
\definecolor{currentfill}{rgb}{0.271305,0.019942,0.347269}%
\pgfsetfillcolor{currentfill}%
\pgfsetfillopacity{0.700000}%
\pgfsetlinewidth{0.000000pt}%
\definecolor{currentstroke}{rgb}{0.000000,0.000000,0.000000}%
\pgfsetstrokecolor{currentstroke}%
\pgfsetdash{}{0pt}%
\pgfpathmoveto{\pgfqpoint{4.394823in}{1.912174in}}%
\pgfpathlineto{\pgfqpoint{4.408149in}{1.909355in}}%
\pgfpathlineto{\pgfqpoint{4.421481in}{1.906560in}}%
\pgfpathlineto{\pgfqpoint{4.434821in}{1.903791in}}%
\pgfpathlineto{\pgfqpoint{4.448167in}{1.901046in}}%
\pgfpathlineto{\pgfqpoint{4.440643in}{1.892390in}}%
\pgfpathlineto{\pgfqpoint{4.433115in}{1.883758in}}%
\pgfpathlineto{\pgfqpoint{4.425581in}{1.875155in}}%
\pgfpathlineto{\pgfqpoint{4.418042in}{1.866584in}}%
\pgfpathlineto{\pgfqpoint{4.404685in}{1.869485in}}%
\pgfpathlineto{\pgfqpoint{4.391335in}{1.872412in}}%
\pgfpathlineto{\pgfqpoint{4.377992in}{1.875363in}}%
\pgfpathlineto{\pgfqpoint{4.364655in}{1.878338in}}%
\pgfpathlineto{\pgfqpoint{4.372205in}{1.886748in}}%
\pgfpathlineto{\pgfqpoint{4.379749in}{1.895193in}}%
\pgfpathlineto{\pgfqpoint{4.387289in}{1.903670in}}%
\pgfpathlineto{\pgfqpoint{4.394823in}{1.912174in}}%
\pgfpathclose%
\pgfusepath{fill}%
\end{pgfscope}%
\begin{pgfscope}%
\pgfpathrectangle{\pgfqpoint{1.254980in}{0.150000in}}{\pgfqpoint{5.490039in}{5.490039in}}%
\pgfusepath{clip}%
\pgfsetbuttcap%
\pgfsetroundjoin%
\definecolor{currentfill}{rgb}{0.268510,0.009605,0.335427}%
\pgfsetfillcolor{currentfill}%
\pgfsetfillopacity{0.700000}%
\pgfsetlinewidth{0.000000pt}%
\definecolor{currentstroke}{rgb}{0.000000,0.000000,0.000000}%
\pgfsetstrokecolor{currentstroke}%
\pgfsetdash{}{0pt}%
\pgfpathmoveto{\pgfqpoint{4.174740in}{1.884811in}}%
\pgfpathlineto{\pgfqpoint{4.188010in}{1.881366in}}%
\pgfpathlineto{\pgfqpoint{4.201287in}{1.877946in}}%
\pgfpathlineto{\pgfqpoint{4.214570in}{1.874552in}}%
\pgfpathlineto{\pgfqpoint{4.227859in}{1.871182in}}%
\pgfpathlineto{\pgfqpoint{4.220260in}{1.863363in}}%
\pgfpathlineto{\pgfqpoint{4.212656in}{1.855609in}}%
\pgfpathlineto{\pgfqpoint{4.205046in}{1.847925in}}%
\pgfpathlineto{\pgfqpoint{4.197430in}{1.840317in}}%
\pgfpathlineto{\pgfqpoint{4.184129in}{1.843868in}}%
\pgfpathlineto{\pgfqpoint{4.170833in}{1.847445in}}%
\pgfpathlineto{\pgfqpoint{4.157544in}{1.851047in}}%
\pgfpathlineto{\pgfqpoint{4.144262in}{1.854674in}}%
\pgfpathlineto{\pgfqpoint{4.151890in}{1.862095in}}%
\pgfpathlineto{\pgfqpoint{4.159512in}{1.869595in}}%
\pgfpathlineto{\pgfqpoint{4.167129in}{1.877168in}}%
\pgfpathlineto{\pgfqpoint{4.174740in}{1.884811in}}%
\pgfpathclose%
\pgfusepath{fill}%
\end{pgfscope}%
\begin{pgfscope}%
\pgfpathrectangle{\pgfqpoint{1.254980in}{0.150000in}}{\pgfqpoint{5.490039in}{5.490039in}}%
\pgfusepath{clip}%
\pgfsetbuttcap%
\pgfsetroundjoin%
\definecolor{currentfill}{rgb}{0.280255,0.165693,0.476498}%
\pgfsetfillcolor{currentfill}%
\pgfsetfillopacity{0.700000}%
\pgfsetlinewidth{0.000000pt}%
\definecolor{currentstroke}{rgb}{0.000000,0.000000,0.000000}%
\pgfsetstrokecolor{currentstroke}%
\pgfsetdash{}{0pt}%
\pgfpathmoveto{\pgfqpoint{5.389230in}{2.173737in}}%
\pgfpathlineto{\pgfqpoint{5.402855in}{2.173030in}}%
\pgfpathlineto{\pgfqpoint{5.416489in}{2.172347in}}%
\pgfpathlineto{\pgfqpoint{5.430131in}{2.171688in}}%
\pgfpathlineto{\pgfqpoint{5.443782in}{2.171052in}}%
\pgfpathlineto{\pgfqpoint{5.436629in}{2.163143in}}%
\pgfpathlineto{\pgfqpoint{5.429468in}{2.155141in}}%
\pgfpathlineto{\pgfqpoint{5.422299in}{2.147046in}}%
\pgfpathlineto{\pgfqpoint{5.415123in}{2.138857in}}%
\pgfpathlineto{\pgfqpoint{5.401461in}{2.139520in}}%
\pgfpathlineto{\pgfqpoint{5.387808in}{2.140206in}}%
\pgfpathlineto{\pgfqpoint{5.374163in}{2.140916in}}%
\pgfpathlineto{\pgfqpoint{5.360526in}{2.141650in}}%
\pgfpathlineto{\pgfqpoint{5.367713in}{2.149807in}}%
\pgfpathlineto{\pgfqpoint{5.374893in}{2.157873in}}%
\pgfpathlineto{\pgfqpoint{5.382065in}{2.165850in}}%
\pgfpathlineto{\pgfqpoint{5.389230in}{2.173737in}}%
\pgfpathclose%
\pgfusepath{fill}%
\end{pgfscope}%
\begin{pgfscope}%
\pgfpathrectangle{\pgfqpoint{1.254980in}{0.150000in}}{\pgfqpoint{5.490039in}{5.490039in}}%
\pgfusepath{clip}%
\pgfsetbuttcap%
\pgfsetroundjoin%
\definecolor{currentfill}{rgb}{0.260571,0.246922,0.522828}%
\pgfsetfillcolor{currentfill}%
\pgfsetfillopacity{0.700000}%
\pgfsetlinewidth{0.000000pt}%
\definecolor{currentstroke}{rgb}{0.000000,0.000000,0.000000}%
\pgfsetstrokecolor{currentstroke}%
\pgfsetdash{}{0pt}%
\pgfpathmoveto{\pgfqpoint{2.681498in}{2.323389in}}%
\pgfpathlineto{\pgfqpoint{2.694551in}{2.314955in}}%
\pgfpathlineto{\pgfqpoint{2.707606in}{2.306563in}}%
\pgfpathlineto{\pgfqpoint{2.720664in}{2.298212in}}%
\pgfpathlineto{\pgfqpoint{2.733724in}{2.289903in}}%
\pgfpathlineto{\pgfqpoint{2.725244in}{2.294719in}}%
\pgfpathlineto{\pgfqpoint{2.716743in}{2.299900in}}%
\pgfpathlineto{\pgfqpoint{2.708222in}{2.305454in}}%
\pgfpathlineto{\pgfqpoint{2.699680in}{2.311390in}}%
\pgfpathlineto{\pgfqpoint{2.686583in}{2.320005in}}%
\pgfpathlineto{\pgfqpoint{2.673489in}{2.328661in}}%
\pgfpathlineto{\pgfqpoint{2.660397in}{2.337360in}}%
\pgfpathlineto{\pgfqpoint{2.647308in}{2.346100in}}%
\pgfpathlineto{\pgfqpoint{2.655887in}{2.339852in}}%
\pgfpathlineto{\pgfqpoint{2.664445in}{2.333990in}}%
\pgfpathlineto{\pgfqpoint{2.672982in}{2.328505in}}%
\pgfpathlineto{\pgfqpoint{2.681498in}{2.323389in}}%
\pgfpathclose%
\pgfusepath{fill}%
\end{pgfscope}%
\begin{pgfscope}%
\pgfpathrectangle{\pgfqpoint{1.254980in}{0.150000in}}{\pgfqpoint{5.490039in}{5.490039in}}%
\pgfusepath{clip}%
\pgfsetbuttcap%
\pgfsetroundjoin%
\definecolor{currentfill}{rgb}{0.283091,0.110553,0.431554}%
\pgfsetfillcolor{currentfill}%
\pgfsetfillopacity{0.700000}%
\pgfsetlinewidth{0.000000pt}%
\definecolor{currentstroke}{rgb}{0.000000,0.000000,0.000000}%
\pgfsetstrokecolor{currentstroke}%
\pgfsetdash{}{0pt}%
\pgfpathmoveto{\pgfqpoint{5.002129in}{2.059868in}}%
\pgfpathlineto{\pgfqpoint{5.015632in}{2.058496in}}%
\pgfpathlineto{\pgfqpoint{5.029143in}{2.057149in}}%
\pgfpathlineto{\pgfqpoint{5.042662in}{2.055826in}}%
\pgfpathlineto{\pgfqpoint{5.056188in}{2.054527in}}%
\pgfpathlineto{\pgfqpoint{5.048873in}{2.045566in}}%
\pgfpathlineto{\pgfqpoint{5.041551in}{2.036542in}}%
\pgfpathlineto{\pgfqpoint{5.034223in}{2.027455in}}%
\pgfpathlineto{\pgfqpoint{5.026889in}{2.018308in}}%
\pgfpathlineto{\pgfqpoint{5.013353in}{2.019687in}}%
\pgfpathlineto{\pgfqpoint{4.999825in}{2.021090in}}%
\pgfpathlineto{\pgfqpoint{4.986304in}{2.022516in}}%
\pgfpathlineto{\pgfqpoint{4.972792in}{2.023967in}}%
\pgfpathlineto{\pgfqpoint{4.980135in}{2.033030in}}%
\pgfpathlineto{\pgfqpoint{4.987472in}{2.042035in}}%
\pgfpathlineto{\pgfqpoint{4.994804in}{2.050982in}}%
\pgfpathlineto{\pgfqpoint{5.002129in}{2.059868in}}%
\pgfpathclose%
\pgfusepath{fill}%
\end{pgfscope}%
\begin{pgfscope}%
\pgfpathrectangle{\pgfqpoint{1.254980in}{0.150000in}}{\pgfqpoint{5.490039in}{5.490039in}}%
\pgfusepath{clip}%
\pgfsetbuttcap%
\pgfsetroundjoin%
\definecolor{currentfill}{rgb}{0.271828,0.209303,0.504434}%
\pgfsetfillcolor{currentfill}%
\pgfsetfillopacity{0.700000}%
\pgfsetlinewidth{0.000000pt}%
\definecolor{currentstroke}{rgb}{0.000000,0.000000,0.000000}%
\pgfsetstrokecolor{currentstroke}%
\pgfsetdash{}{0pt}%
\pgfpathmoveto{\pgfqpoint{5.693048in}{2.252970in}}%
\pgfpathlineto{\pgfqpoint{5.706774in}{2.252643in}}%
\pgfpathlineto{\pgfqpoint{5.720509in}{2.252340in}}%
\pgfpathlineto{\pgfqpoint{5.734253in}{2.252060in}}%
\pgfpathlineto{\pgfqpoint{5.748005in}{2.251805in}}%
\pgfpathlineto{\pgfqpoint{5.741001in}{2.245069in}}%
\pgfpathlineto{\pgfqpoint{5.733987in}{2.238231in}}%
\pgfpathlineto{\pgfqpoint{5.726965in}{2.231291in}}%
\pgfpathlineto{\pgfqpoint{5.719935in}{2.224248in}}%
\pgfpathlineto{\pgfqpoint{5.706168in}{2.224490in}}%
\pgfpathlineto{\pgfqpoint{5.692411in}{2.224756in}}%
\pgfpathlineto{\pgfqpoint{5.678662in}{2.225046in}}%
\pgfpathlineto{\pgfqpoint{5.664922in}{2.225359in}}%
\pgfpathlineto{\pgfqpoint{5.671966in}{2.232411in}}%
\pgfpathlineto{\pgfqpoint{5.679002in}{2.239363in}}%
\pgfpathlineto{\pgfqpoint{5.686029in}{2.246215in}}%
\pgfpathlineto{\pgfqpoint{5.693048in}{2.252970in}}%
\pgfpathclose%
\pgfusepath{fill}%
\end{pgfscope}%
\begin{pgfscope}%
\pgfpathrectangle{\pgfqpoint{1.254980in}{0.150000in}}{\pgfqpoint{5.490039in}{5.490039in}}%
\pgfusepath{clip}%
\pgfsetbuttcap%
\pgfsetroundjoin%
\definecolor{currentfill}{rgb}{0.277018,0.050344,0.375715}%
\pgfsetfillcolor{currentfill}%
\pgfsetfillopacity{0.700000}%
\pgfsetlinewidth{0.000000pt}%
\definecolor{currentstroke}{rgb}{0.000000,0.000000,0.000000}%
\pgfsetstrokecolor{currentstroke}%
\pgfsetdash{}{0pt}%
\pgfpathmoveto{\pgfqpoint{4.615029in}{1.951881in}}%
\pgfpathlineto{\pgfqpoint{4.628418in}{1.949631in}}%
\pgfpathlineto{\pgfqpoint{4.641815in}{1.947406in}}%
\pgfpathlineto{\pgfqpoint{4.655219in}{1.945205in}}%
\pgfpathlineto{\pgfqpoint{4.668630in}{1.943028in}}%
\pgfpathlineto{\pgfqpoint{4.661178in}{1.933920in}}%
\pgfpathlineto{\pgfqpoint{4.653720in}{1.924801in}}%
\pgfpathlineto{\pgfqpoint{4.646257in}{1.915673in}}%
\pgfpathlineto{\pgfqpoint{4.638789in}{1.906539in}}%
\pgfpathlineto{\pgfqpoint{4.625368in}{1.908847in}}%
\pgfpathlineto{\pgfqpoint{4.611954in}{1.911180in}}%
\pgfpathlineto{\pgfqpoint{4.598548in}{1.913536in}}%
\pgfpathlineto{\pgfqpoint{4.585149in}{1.915917in}}%
\pgfpathlineto{\pgfqpoint{4.592626in}{1.924915in}}%
\pgfpathlineto{\pgfqpoint{4.600099in}{1.933910in}}%
\pgfpathlineto{\pgfqpoint{4.607566in}{1.942900in}}%
\pgfpathlineto{\pgfqpoint{4.615029in}{1.951881in}}%
\pgfpathclose%
\pgfusepath{fill}%
\end{pgfscope}%
\begin{pgfscope}%
\pgfpathrectangle{\pgfqpoint{1.254980in}{0.150000in}}{\pgfqpoint{5.490039in}{5.490039in}}%
\pgfusepath{clip}%
\pgfsetbuttcap%
\pgfsetroundjoin%
\definecolor{currentfill}{rgb}{0.283197,0.115680,0.436115}%
\pgfsetfillcolor{currentfill}%
\pgfsetfillopacity{0.700000}%
\pgfsetlinewidth{0.000000pt}%
\definecolor{currentstroke}{rgb}{0.000000,0.000000,0.000000}%
\pgfsetstrokecolor{currentstroke}%
\pgfsetdash{}{0pt}%
\pgfpathmoveto{\pgfqpoint{3.166022in}{2.062163in}}%
\pgfpathlineto{\pgfqpoint{3.179113in}{2.055479in}}%
\pgfpathlineto{\pgfqpoint{3.192208in}{2.048827in}}%
\pgfpathlineto{\pgfqpoint{3.205307in}{2.042208in}}%
\pgfpathlineto{\pgfqpoint{3.218410in}{2.035621in}}%
\pgfpathlineto{\pgfqpoint{3.210304in}{2.035642in}}%
\pgfpathlineto{\pgfqpoint{3.202186in}{2.035938in}}%
\pgfpathlineto{\pgfqpoint{3.194053in}{2.036516in}}%
\pgfpathlineto{\pgfqpoint{3.185907in}{2.037384in}}%
\pgfpathlineto{\pgfqpoint{3.172777in}{2.044245in}}%
\pgfpathlineto{\pgfqpoint{3.159651in}{2.051138in}}%
\pgfpathlineto{\pgfqpoint{3.146529in}{2.058064in}}%
\pgfpathlineto{\pgfqpoint{3.133410in}{2.065022in}}%
\pgfpathlineto{\pgfqpoint{3.141584in}{2.063875in}}%
\pgfpathlineto{\pgfqpoint{3.149744in}{2.063021in}}%
\pgfpathlineto{\pgfqpoint{3.157890in}{2.062453in}}%
\pgfpathlineto{\pgfqpoint{3.166022in}{2.062163in}}%
\pgfpathclose%
\pgfusepath{fill}%
\end{pgfscope}%
\begin{pgfscope}%
\pgfpathrectangle{\pgfqpoint{1.254980in}{0.150000in}}{\pgfqpoint{5.490039in}{5.490039in}}%
\pgfusepath{clip}%
\pgfsetbuttcap%
\pgfsetroundjoin%
\definecolor{currentfill}{rgb}{0.220057,0.343307,0.549413}%
\pgfsetfillcolor{currentfill}%
\pgfsetfillopacity{0.700000}%
\pgfsetlinewidth{0.000000pt}%
\definecolor{currentstroke}{rgb}{0.000000,0.000000,0.000000}%
\pgfsetstrokecolor{currentstroke}%
\pgfsetdash{}{0pt}%
\pgfpathmoveto{\pgfqpoint{2.385893in}{2.530424in}}%
\pgfpathlineto{\pgfqpoint{2.398950in}{2.520748in}}%
\pgfpathlineto{\pgfqpoint{2.412008in}{2.511123in}}%
\pgfpathlineto{\pgfqpoint{2.425067in}{2.501549in}}%
\pgfpathlineto{\pgfqpoint{2.438128in}{2.492025in}}%
\pgfpathlineto{\pgfqpoint{2.429368in}{2.499934in}}%
\pgfpathlineto{\pgfqpoint{2.420584in}{2.508261in}}%
\pgfpathlineto{\pgfqpoint{2.411774in}{2.517015in}}%
\pgfpathlineto{\pgfqpoint{2.402938in}{2.526204in}}%
\pgfpathlineto{\pgfqpoint{2.389835in}{2.536052in}}%
\pgfpathlineto{\pgfqpoint{2.376734in}{2.545950in}}%
\pgfpathlineto{\pgfqpoint{2.363634in}{2.555900in}}%
\pgfpathlineto{\pgfqpoint{2.350534in}{2.565901in}}%
\pgfpathlineto{\pgfqpoint{2.359414in}{2.556381in}}%
\pgfpathlineto{\pgfqpoint{2.368266in}{2.547301in}}%
\pgfpathlineto{\pgfqpoint{2.377093in}{2.538651in}}%
\pgfpathlineto{\pgfqpoint{2.385893in}{2.530424in}}%
\pgfpathclose%
\pgfusepath{fill}%
\end{pgfscope}%
\begin{pgfscope}%
\pgfpathrectangle{\pgfqpoint{1.254980in}{0.150000in}}{\pgfqpoint{5.490039in}{5.490039in}}%
\pgfusepath{clip}%
\pgfsetbuttcap%
\pgfsetroundjoin%
\definecolor{currentfill}{rgb}{0.280267,0.073417,0.397163}%
\pgfsetfillcolor{currentfill}%
\pgfsetfillopacity{0.700000}%
\pgfsetlinewidth{0.000000pt}%
\definecolor{currentstroke}{rgb}{0.000000,0.000000,0.000000}%
\pgfsetstrokecolor{currentstroke}%
\pgfsetdash{}{0pt}%
\pgfpathmoveto{\pgfqpoint{3.355474in}{1.988674in}}%
\pgfpathlineto{\pgfqpoint{3.368590in}{1.982629in}}%
\pgfpathlineto{\pgfqpoint{3.381709in}{1.976614in}}%
\pgfpathlineto{\pgfqpoint{3.394834in}{1.970630in}}%
\pgfpathlineto{\pgfqpoint{3.407962in}{1.964675in}}%
\pgfpathlineto{\pgfqpoint{3.399979in}{1.962900in}}%
\pgfpathlineto{\pgfqpoint{3.391984in}{1.961362in}}%
\pgfpathlineto{\pgfqpoint{3.383978in}{1.960068in}}%
\pgfpathlineto{\pgfqpoint{3.375961in}{1.959025in}}%
\pgfpathlineto{\pgfqpoint{3.362809in}{1.965240in}}%
\pgfpathlineto{\pgfqpoint{3.349661in}{1.971484in}}%
\pgfpathlineto{\pgfqpoint{3.336517in}{1.977759in}}%
\pgfpathlineto{\pgfqpoint{3.323377in}{1.984065in}}%
\pgfpathlineto{\pgfqpoint{3.331419in}{1.984843in}}%
\pgfpathlineto{\pgfqpoint{3.339449in}{1.985875in}}%
\pgfpathlineto{\pgfqpoint{3.347468in}{1.987154in}}%
\pgfpathlineto{\pgfqpoint{3.355474in}{1.988674in}}%
\pgfpathclose%
\pgfusepath{fill}%
\end{pgfscope}%
\begin{pgfscope}%
\pgfpathrectangle{\pgfqpoint{1.254980in}{0.150000in}}{\pgfqpoint{5.490039in}{5.490039in}}%
\pgfusepath{clip}%
\pgfsetbuttcap%
\pgfsetroundjoin%
\definecolor{currentfill}{rgb}{0.268510,0.009605,0.335427}%
\pgfsetfillcolor{currentfill}%
\pgfsetfillopacity{0.700000}%
\pgfsetlinewidth{0.000000pt}%
\definecolor{currentstroke}{rgb}{0.000000,0.000000,0.000000}%
\pgfsetstrokecolor{currentstroke}%
\pgfsetdash{}{0pt}%
\pgfpathmoveto{\pgfqpoint{3.818060in}{1.884676in}}%
\pgfpathlineto{\pgfqpoint{3.831254in}{1.880127in}}%
\pgfpathlineto{\pgfqpoint{3.844454in}{1.875605in}}%
\pgfpathlineto{\pgfqpoint{3.857660in}{1.871110in}}%
\pgfpathlineto{\pgfqpoint{3.870871in}{1.866641in}}%
\pgfpathlineto{\pgfqpoint{3.863129in}{1.861003in}}%
\pgfpathlineto{\pgfqpoint{3.855380in}{1.855507in}}%
\pgfpathlineto{\pgfqpoint{3.847624in}{1.850157in}}%
\pgfpathlineto{\pgfqpoint{3.839860in}{1.844959in}}%
\pgfpathlineto{\pgfqpoint{3.826633in}{1.849648in}}%
\pgfpathlineto{\pgfqpoint{3.813410in}{1.854363in}}%
\pgfpathlineto{\pgfqpoint{3.800193in}{1.859106in}}%
\pgfpathlineto{\pgfqpoint{3.786982in}{1.863875in}}%
\pgfpathlineto{\pgfqpoint{3.794763in}{1.868847in}}%
\pgfpathlineto{\pgfqpoint{3.802536in}{1.873975in}}%
\pgfpathlineto{\pgfqpoint{3.810301in}{1.879254in}}%
\pgfpathlineto{\pgfqpoint{3.818060in}{1.884676in}}%
\pgfpathclose%
\pgfusepath{fill}%
\end{pgfscope}%
\begin{pgfscope}%
\pgfpathrectangle{\pgfqpoint{1.254980in}{0.150000in}}{\pgfqpoint{5.490039in}{5.490039in}}%
\pgfusepath{clip}%
\pgfsetbuttcap%
\pgfsetroundjoin%
\definecolor{currentfill}{rgb}{0.281412,0.155834,0.469201}%
\pgfsetfillcolor{currentfill}%
\pgfsetfillopacity{0.700000}%
\pgfsetlinewidth{0.000000pt}%
\definecolor{currentstroke}{rgb}{0.000000,0.000000,0.000000}%
\pgfsetstrokecolor{currentstroke}%
\pgfsetdash{}{0pt}%
\pgfpathmoveto{\pgfqpoint{5.306064in}{2.144823in}}%
\pgfpathlineto{\pgfqpoint{5.319667in}{2.143994in}}%
\pgfpathlineto{\pgfqpoint{5.333278in}{2.143189in}}%
\pgfpathlineto{\pgfqpoint{5.346898in}{2.142408in}}%
\pgfpathlineto{\pgfqpoint{5.360526in}{2.141650in}}%
\pgfpathlineto{\pgfqpoint{5.353332in}{2.133404in}}%
\pgfpathlineto{\pgfqpoint{5.346130in}{2.125069in}}%
\pgfpathlineto{\pgfqpoint{5.338921in}{2.116646in}}%
\pgfpathlineto{\pgfqpoint{5.331705in}{2.108134in}}%
\pgfpathlineto{\pgfqpoint{5.318067in}{2.108932in}}%
\pgfpathlineto{\pgfqpoint{5.304436in}{2.109754in}}%
\pgfpathlineto{\pgfqpoint{5.290815in}{2.110599in}}%
\pgfpathlineto{\pgfqpoint{5.277201in}{2.111468in}}%
\pgfpathlineto{\pgfqpoint{5.284428in}{2.119934in}}%
\pgfpathlineto{\pgfqpoint{5.291647in}{2.128316in}}%
\pgfpathlineto{\pgfqpoint{5.298859in}{2.136612in}}%
\pgfpathlineto{\pgfqpoint{5.306064in}{2.144823in}}%
\pgfpathclose%
\pgfusepath{fill}%
\end{pgfscope}%
\begin{pgfscope}%
\pgfpathrectangle{\pgfqpoint{1.254980in}{0.150000in}}{\pgfqpoint{5.490039in}{5.490039in}}%
\pgfusepath{clip}%
\pgfsetbuttcap%
\pgfsetroundjoin%
\definecolor{currentfill}{rgb}{0.282327,0.094955,0.417331}%
\pgfsetfillcolor{currentfill}%
\pgfsetfillopacity{0.700000}%
\pgfsetlinewidth{0.000000pt}%
\definecolor{currentstroke}{rgb}{0.000000,0.000000,0.000000}%
\pgfsetstrokecolor{currentstroke}%
\pgfsetdash{}{0pt}%
\pgfpathmoveto{\pgfqpoint{4.918820in}{2.030009in}}%
\pgfpathlineto{\pgfqpoint{4.932301in}{2.028463in}}%
\pgfpathlineto{\pgfqpoint{4.945790in}{2.026940in}}%
\pgfpathlineto{\pgfqpoint{4.959287in}{2.025442in}}%
\pgfpathlineto{\pgfqpoint{4.972792in}{2.023967in}}%
\pgfpathlineto{\pgfqpoint{4.965443in}{2.014849in}}%
\pgfpathlineto{\pgfqpoint{4.958087in}{2.005677in}}%
\pgfpathlineto{\pgfqpoint{4.950726in}{1.996454in}}%
\pgfpathlineto{\pgfqpoint{4.943360in}{1.987181in}}%
\pgfpathlineto{\pgfqpoint{4.929846in}{1.988748in}}%
\pgfpathlineto{\pgfqpoint{4.916340in}{1.990339in}}%
\pgfpathlineto{\pgfqpoint{4.902842in}{1.991955in}}%
\pgfpathlineto{\pgfqpoint{4.889351in}{1.993594in}}%
\pgfpathlineto{\pgfqpoint{4.896727in}{2.002769in}}%
\pgfpathlineto{\pgfqpoint{4.904097in}{2.011898in}}%
\pgfpathlineto{\pgfqpoint{4.911462in}{2.020979in}}%
\pgfpathlineto{\pgfqpoint{4.918820in}{2.030009in}}%
\pgfpathclose%
\pgfusepath{fill}%
\end{pgfscope}%
\begin{pgfscope}%
\pgfpathrectangle{\pgfqpoint{1.254980in}{0.150000in}}{\pgfqpoint{5.490039in}{5.490039in}}%
\pgfusepath{clip}%
\pgfsetbuttcap%
\pgfsetroundjoin%
\definecolor{currentfill}{rgb}{0.271305,0.019942,0.347269}%
\pgfsetfillcolor{currentfill}%
\pgfsetfillopacity{0.700000}%
\pgfsetlinewidth{0.000000pt}%
\definecolor{currentstroke}{rgb}{0.000000,0.000000,0.000000}%
\pgfsetstrokecolor{currentstroke}%
\pgfsetdash{}{0pt}%
\pgfpathmoveto{\pgfqpoint{3.681481in}{1.903001in}}%
\pgfpathlineto{\pgfqpoint{3.694651in}{1.898015in}}%
\pgfpathlineto{\pgfqpoint{3.707825in}{1.893056in}}%
\pgfpathlineto{\pgfqpoint{3.721005in}{1.888124in}}%
\pgfpathlineto{\pgfqpoint{3.734190in}{1.883220in}}%
\pgfpathlineto{\pgfqpoint{3.726383in}{1.878637in}}%
\pgfpathlineto{\pgfqpoint{3.718569in}{1.874224in}}%
\pgfpathlineto{\pgfqpoint{3.710747in}{1.869988in}}%
\pgfpathlineto{\pgfqpoint{3.702916in}{1.865934in}}%
\pgfpathlineto{\pgfqpoint{3.689713in}{1.871071in}}%
\pgfpathlineto{\pgfqpoint{3.676515in}{1.876236in}}%
\pgfpathlineto{\pgfqpoint{3.663322in}{1.881428in}}%
\pgfpathlineto{\pgfqpoint{3.650133in}{1.886648in}}%
\pgfpathlineto{\pgfqpoint{3.657983in}{1.890464in}}%
\pgfpathlineto{\pgfqpoint{3.665824in}{1.894466in}}%
\pgfpathlineto{\pgfqpoint{3.673657in}{1.898647in}}%
\pgfpathlineto{\pgfqpoint{3.681481in}{1.903001in}}%
\pgfpathclose%
\pgfusepath{fill}%
\end{pgfscope}%
\begin{pgfscope}%
\pgfpathrectangle{\pgfqpoint{1.254980in}{0.150000in}}{\pgfqpoint{5.490039in}{5.490039in}}%
\pgfusepath{clip}%
\pgfsetbuttcap%
\pgfsetroundjoin%
\definecolor{currentfill}{rgb}{0.267004,0.004874,0.329415}%
\pgfsetfillcolor{currentfill}%
\pgfsetfillopacity{0.700000}%
\pgfsetlinewidth{0.000000pt}%
\definecolor{currentstroke}{rgb}{0.000000,0.000000,0.000000}%
\pgfsetstrokecolor{currentstroke}%
\pgfsetdash{}{0pt}%
\pgfpathmoveto{\pgfqpoint{3.954610in}{1.873716in}}%
\pgfpathlineto{\pgfqpoint{3.967835in}{1.869586in}}%
\pgfpathlineto{\pgfqpoint{3.981065in}{1.865482in}}%
\pgfpathlineto{\pgfqpoint{3.994301in}{1.861405in}}%
\pgfpathlineto{\pgfqpoint{4.007543in}{1.857353in}}%
\pgfpathlineto{\pgfqpoint{3.999858in}{1.850798in}}%
\pgfpathlineto{\pgfqpoint{3.992166in}{1.844358in}}%
\pgfpathlineto{\pgfqpoint{3.984468in}{1.838035in}}%
\pgfpathlineto{\pgfqpoint{3.976764in}{1.831837in}}%
\pgfpathlineto{\pgfqpoint{3.963508in}{1.836097in}}%
\pgfpathlineto{\pgfqpoint{3.950257in}{1.840382in}}%
\pgfpathlineto{\pgfqpoint{3.937012in}{1.844693in}}%
\pgfpathlineto{\pgfqpoint{3.923772in}{1.849030in}}%
\pgfpathlineto{\pgfqpoint{3.931492in}{1.855016in}}%
\pgfpathlineto{\pgfqpoint{3.939204in}{1.861129in}}%
\pgfpathlineto{\pgfqpoint{3.946911in}{1.867364in}}%
\pgfpathlineto{\pgfqpoint{3.954610in}{1.873716in}}%
\pgfpathclose%
\pgfusepath{fill}%
\end{pgfscope}%
\begin{pgfscope}%
\pgfpathrectangle{\pgfqpoint{1.254980in}{0.150000in}}{\pgfqpoint{5.490039in}{5.490039in}}%
\pgfusepath{clip}%
\pgfsetbuttcap%
\pgfsetroundjoin%
\definecolor{currentfill}{rgb}{0.269944,0.014625,0.341379}%
\pgfsetfillcolor{currentfill}%
\pgfsetfillopacity{0.700000}%
\pgfsetlinewidth{0.000000pt}%
\definecolor{currentstroke}{rgb}{0.000000,0.000000,0.000000}%
\pgfsetstrokecolor{currentstroke}%
\pgfsetdash{}{0pt}%
\pgfpathmoveto{\pgfqpoint{4.311376in}{1.890489in}}%
\pgfpathlineto{\pgfqpoint{4.324686in}{1.887414in}}%
\pgfpathlineto{\pgfqpoint{4.338002in}{1.884364in}}%
\pgfpathlineto{\pgfqpoint{4.351325in}{1.881339in}}%
\pgfpathlineto{\pgfqpoint{4.364655in}{1.878338in}}%
\pgfpathlineto{\pgfqpoint{4.357101in}{1.869967in}}%
\pgfpathlineto{\pgfqpoint{4.349541in}{1.861639in}}%
\pgfpathlineto{\pgfqpoint{4.341975in}{1.853358in}}%
\pgfpathlineto{\pgfqpoint{4.334405in}{1.845128in}}%
\pgfpathlineto{\pgfqpoint{4.321064in}{1.848298in}}%
\pgfpathlineto{\pgfqpoint{4.307729in}{1.851492in}}%
\pgfpathlineto{\pgfqpoint{4.294402in}{1.854712in}}%
\pgfpathlineto{\pgfqpoint{4.281080in}{1.857956in}}%
\pgfpathlineto{\pgfqpoint{4.288662in}{1.866012in}}%
\pgfpathlineto{\pgfqpoint{4.296238in}{1.874122in}}%
\pgfpathlineto{\pgfqpoint{4.303810in}{1.882282in}}%
\pgfpathlineto{\pgfqpoint{4.311376in}{1.890489in}}%
\pgfpathclose%
\pgfusepath{fill}%
\end{pgfscope}%
\begin{pgfscope}%
\pgfpathrectangle{\pgfqpoint{1.254980in}{0.150000in}}{\pgfqpoint{5.490039in}{5.490039in}}%
\pgfusepath{clip}%
\pgfsetbuttcap%
\pgfsetroundjoin%
\definecolor{currentfill}{rgb}{0.280255,0.165693,0.476498}%
\pgfsetfillcolor{currentfill}%
\pgfsetfillopacity{0.700000}%
\pgfsetlinewidth{0.000000pt}%
\definecolor{currentstroke}{rgb}{0.000000,0.000000,0.000000}%
\pgfsetstrokecolor{currentstroke}%
\pgfsetdash{}{0pt}%
\pgfpathmoveto{\pgfqpoint{2.976264in}{2.151157in}}%
\pgfpathlineto{\pgfqpoint{2.989341in}{2.143788in}}%
\pgfpathlineto{\pgfqpoint{3.002421in}{2.136455in}}%
\pgfpathlineto{\pgfqpoint{3.015504in}{2.129157in}}%
\pgfpathlineto{\pgfqpoint{3.028591in}{2.121894in}}%
\pgfpathlineto{\pgfqpoint{3.020343in}{2.123908in}}%
\pgfpathlineto{\pgfqpoint{3.012080in}{2.126238in}}%
\pgfpathlineto{\pgfqpoint{3.003800in}{2.128890in}}%
\pgfpathlineto{\pgfqpoint{2.995504in}{2.131874in}}%
\pgfpathlineto{\pgfqpoint{2.982387in}{2.139426in}}%
\pgfpathlineto{\pgfqpoint{2.969273in}{2.147013in}}%
\pgfpathlineto{\pgfqpoint{2.956162in}{2.154635in}}%
\pgfpathlineto{\pgfqpoint{2.943054in}{2.162293in}}%
\pgfpathlineto{\pgfqpoint{2.951382in}{2.159015in}}%
\pgfpathlineto{\pgfqpoint{2.959692in}{2.156072in}}%
\pgfpathlineto{\pgfqpoint{2.967986in}{2.153455in}}%
\pgfpathlineto{\pgfqpoint{2.976264in}{2.151157in}}%
\pgfpathclose%
\pgfusepath{fill}%
\end{pgfscope}%
\begin{pgfscope}%
\pgfpathrectangle{\pgfqpoint{1.254980in}{0.150000in}}{\pgfqpoint{5.490039in}{5.490039in}}%
\pgfusepath{clip}%
\pgfsetbuttcap%
\pgfsetroundjoin%
\definecolor{currentfill}{rgb}{0.274128,0.199721,0.498911}%
\pgfsetfillcolor{currentfill}%
\pgfsetfillopacity{0.700000}%
\pgfsetlinewidth{0.000000pt}%
\definecolor{currentstroke}{rgb}{0.000000,0.000000,0.000000}%
\pgfsetstrokecolor{currentstroke}%
\pgfsetdash{}{0pt}%
\pgfpathmoveto{\pgfqpoint{5.610052in}{2.226850in}}%
\pgfpathlineto{\pgfqpoint{5.623756in}{2.226442in}}%
\pgfpathlineto{\pgfqpoint{5.637469in}{2.226057in}}%
\pgfpathlineto{\pgfqpoint{5.651191in}{2.225697in}}%
\pgfpathlineto{\pgfqpoint{5.664922in}{2.225359in}}%
\pgfpathlineto{\pgfqpoint{5.657870in}{2.218207in}}%
\pgfpathlineto{\pgfqpoint{5.650809in}{2.210953in}}%
\pgfpathlineto{\pgfqpoint{5.643740in}{2.203598in}}%
\pgfpathlineto{\pgfqpoint{5.636662in}{2.196140in}}%
\pgfpathlineto{\pgfqpoint{5.622919in}{2.196477in}}%
\pgfpathlineto{\pgfqpoint{5.609184in}{2.196838in}}%
\pgfpathlineto{\pgfqpoint{5.595458in}{2.197223in}}%
\pgfpathlineto{\pgfqpoint{5.581741in}{2.197631in}}%
\pgfpathlineto{\pgfqpoint{5.588831in}{2.205084in}}%
\pgfpathlineto{\pgfqpoint{5.595913in}{2.212438in}}%
\pgfpathlineto{\pgfqpoint{5.602987in}{2.219693in}}%
\pgfpathlineto{\pgfqpoint{5.610052in}{2.226850in}}%
\pgfpathclose%
\pgfusepath{fill}%
\end{pgfscope}%
\begin{pgfscope}%
\pgfpathrectangle{\pgfqpoint{1.254980in}{0.150000in}}{\pgfqpoint{5.490039in}{5.490039in}}%
\pgfusepath{clip}%
\pgfsetbuttcap%
\pgfsetroundjoin%
\definecolor{currentfill}{rgb}{0.265145,0.232956,0.516599}%
\pgfsetfillcolor{currentfill}%
\pgfsetfillopacity{0.700000}%
\pgfsetlinewidth{0.000000pt}%
\definecolor{currentstroke}{rgb}{0.000000,0.000000,0.000000}%
\pgfsetstrokecolor{currentstroke}%
\pgfsetdash{}{0pt}%
\pgfpathmoveto{\pgfqpoint{2.733724in}{2.289903in}}%
\pgfpathlineto{\pgfqpoint{2.746787in}{2.281634in}}%
\pgfpathlineto{\pgfqpoint{2.759853in}{2.273406in}}%
\pgfpathlineto{\pgfqpoint{2.772920in}{2.265219in}}%
\pgfpathlineto{\pgfqpoint{2.785991in}{2.257071in}}%
\pgfpathlineto{\pgfqpoint{2.777545in}{2.261588in}}%
\pgfpathlineto{\pgfqpoint{2.769080in}{2.266465in}}%
\pgfpathlineto{\pgfqpoint{2.760594in}{2.271713in}}%
\pgfpathlineto{\pgfqpoint{2.752089in}{2.277338in}}%
\pgfpathlineto{\pgfqpoint{2.738983in}{2.285791in}}%
\pgfpathlineto{\pgfqpoint{2.725879in}{2.294283in}}%
\pgfpathlineto{\pgfqpoint{2.712778in}{2.302816in}}%
\pgfpathlineto{\pgfqpoint{2.699680in}{2.311390in}}%
\pgfpathlineto{\pgfqpoint{2.708222in}{2.305454in}}%
\pgfpathlineto{\pgfqpoint{2.716743in}{2.299900in}}%
\pgfpathlineto{\pgfqpoint{2.725244in}{2.294719in}}%
\pgfpathlineto{\pgfqpoint{2.733724in}{2.289903in}}%
\pgfpathclose%
\pgfusepath{fill}%
\end{pgfscope}%
\begin{pgfscope}%
\pgfpathrectangle{\pgfqpoint{1.254980in}{0.150000in}}{\pgfqpoint{5.490039in}{5.490039in}}%
\pgfusepath{clip}%
\pgfsetbuttcap%
\pgfsetroundjoin%
\definecolor{currentfill}{rgb}{0.274952,0.037752,0.364543}%
\pgfsetfillcolor{currentfill}%
\pgfsetfillopacity{0.700000}%
\pgfsetlinewidth{0.000000pt}%
\definecolor{currentstroke}{rgb}{0.000000,0.000000,0.000000}%
\pgfsetstrokecolor{currentstroke}%
\pgfsetdash{}{0pt}%
\pgfpathmoveto{\pgfqpoint{4.531623in}{1.925684in}}%
\pgfpathlineto{\pgfqpoint{4.544994in}{1.923206in}}%
\pgfpathlineto{\pgfqpoint{4.558371in}{1.920752in}}%
\pgfpathlineto{\pgfqpoint{4.571756in}{1.918322in}}%
\pgfpathlineto{\pgfqpoint{4.585149in}{1.915917in}}%
\pgfpathlineto{\pgfqpoint{4.577666in}{1.906920in}}%
\pgfpathlineto{\pgfqpoint{4.570178in}{1.897928in}}%
\pgfpathlineto{\pgfqpoint{4.562685in}{1.888942in}}%
\pgfpathlineto{\pgfqpoint{4.555186in}{1.879968in}}%
\pgfpathlineto{\pgfqpoint{4.541784in}{1.882517in}}%
\pgfpathlineto{\pgfqpoint{4.528390in}{1.885091in}}%
\pgfpathlineto{\pgfqpoint{4.515002in}{1.887689in}}%
\pgfpathlineto{\pgfqpoint{4.501621in}{1.890312in}}%
\pgfpathlineto{\pgfqpoint{4.509129in}{1.899137in}}%
\pgfpathlineto{\pgfqpoint{4.516632in}{1.907977in}}%
\pgfpathlineto{\pgfqpoint{4.524130in}{1.916827in}}%
\pgfpathlineto{\pgfqpoint{4.531623in}{1.925684in}}%
\pgfpathclose%
\pgfusepath{fill}%
\end{pgfscope}%
\begin{pgfscope}%
\pgfpathrectangle{\pgfqpoint{1.254980in}{0.150000in}}{\pgfqpoint{5.490039in}{5.490039in}}%
\pgfusepath{clip}%
\pgfsetbuttcap%
\pgfsetroundjoin%
\definecolor{currentfill}{rgb}{0.276022,0.044167,0.370164}%
\pgfsetfillcolor{currentfill}%
\pgfsetfillopacity{0.700000}%
\pgfsetlinewidth{0.000000pt}%
\definecolor{currentstroke}{rgb}{0.000000,0.000000,0.000000}%
\pgfsetstrokecolor{currentstroke}%
\pgfsetdash{}{0pt}%
\pgfpathmoveto{\pgfqpoint{3.544807in}{1.929413in}}%
\pgfpathlineto{\pgfqpoint{3.557956in}{1.923969in}}%
\pgfpathlineto{\pgfqpoint{3.571109in}{1.918553in}}%
\pgfpathlineto{\pgfqpoint{3.584268in}{1.913165in}}%
\pgfpathlineto{\pgfqpoint{3.597431in}{1.907806in}}%
\pgfpathlineto{\pgfqpoint{3.589552in}{1.904423in}}%
\pgfpathlineto{\pgfqpoint{3.581665in}{1.901241in}}%
\pgfpathlineto{\pgfqpoint{3.573768in}{1.898267in}}%
\pgfpathlineto{\pgfqpoint{3.565861in}{1.895507in}}%
\pgfpathlineto{\pgfqpoint{3.552677in}{1.901113in}}%
\pgfpathlineto{\pgfqpoint{3.539498in}{1.906747in}}%
\pgfpathlineto{\pgfqpoint{3.526323in}{1.912410in}}%
\pgfpathlineto{\pgfqpoint{3.513154in}{1.918101in}}%
\pgfpathlineto{\pgfqpoint{3.521082in}{1.920610in}}%
\pgfpathlineto{\pgfqpoint{3.529000in}{1.923336in}}%
\pgfpathlineto{\pgfqpoint{3.536908in}{1.926272in}}%
\pgfpathlineto{\pgfqpoint{3.544807in}{1.929413in}}%
\pgfpathclose%
\pgfusepath{fill}%
\end{pgfscope}%
\begin{pgfscope}%
\pgfpathrectangle{\pgfqpoint{1.254980in}{0.150000in}}{\pgfqpoint{5.490039in}{5.490039in}}%
\pgfusepath{clip}%
\pgfsetbuttcap%
\pgfsetroundjoin%
\definecolor{currentfill}{rgb}{0.282623,0.140926,0.457517}%
\pgfsetfillcolor{currentfill}%
\pgfsetfillopacity{0.700000}%
\pgfsetlinewidth{0.000000pt}%
\definecolor{currentstroke}{rgb}{0.000000,0.000000,0.000000}%
\pgfsetstrokecolor{currentstroke}%
\pgfsetdash{}{0pt}%
\pgfpathmoveto{\pgfqpoint{5.222831in}{2.115183in}}%
\pgfpathlineto{\pgfqpoint{5.236411in}{2.114219in}}%
\pgfpathlineto{\pgfqpoint{5.250000in}{2.113278in}}%
\pgfpathlineto{\pgfqpoint{5.263596in}{2.112361in}}%
\pgfpathlineto{\pgfqpoint{5.277201in}{2.111468in}}%
\pgfpathlineto{\pgfqpoint{5.269968in}{2.102918in}}%
\pgfpathlineto{\pgfqpoint{5.262728in}{2.094285in}}%
\pgfpathlineto{\pgfqpoint{5.255480in}{2.085570in}}%
\pgfpathlineto{\pgfqpoint{5.248226in}{2.076774in}}%
\pgfpathlineto{\pgfqpoint{5.234611in}{2.077720in}}%
\pgfpathlineto{\pgfqpoint{5.221004in}{2.078690in}}%
\pgfpathlineto{\pgfqpoint{5.207406in}{2.079684in}}%
\pgfpathlineto{\pgfqpoint{5.193816in}{2.080702in}}%
\pgfpathlineto{\pgfqpoint{5.201080in}{2.089440in}}%
\pgfpathlineto{\pgfqpoint{5.208337in}{2.098100in}}%
\pgfpathlineto{\pgfqpoint{5.215587in}{2.106682in}}%
\pgfpathlineto{\pgfqpoint{5.222831in}{2.115183in}}%
\pgfpathclose%
\pgfusepath{fill}%
\end{pgfscope}%
\begin{pgfscope}%
\pgfpathrectangle{\pgfqpoint{1.254980in}{0.150000in}}{\pgfqpoint{5.490039in}{5.490039in}}%
\pgfusepath{clip}%
\pgfsetbuttcap%
\pgfsetroundjoin%
\definecolor{currentfill}{rgb}{0.267004,0.004874,0.329415}%
\pgfsetfillcolor{currentfill}%
\pgfsetfillopacity{0.700000}%
\pgfsetlinewidth{0.000000pt}%
\definecolor{currentstroke}{rgb}{0.000000,0.000000,0.000000}%
\pgfsetstrokecolor{currentstroke}%
\pgfsetdash{}{0pt}%
\pgfpathmoveto{\pgfqpoint{4.091193in}{1.869435in}}%
\pgfpathlineto{\pgfqpoint{4.104451in}{1.865706in}}%
\pgfpathlineto{\pgfqpoint{4.117715in}{1.862003in}}%
\pgfpathlineto{\pgfqpoint{4.130985in}{1.858326in}}%
\pgfpathlineto{\pgfqpoint{4.144262in}{1.854674in}}%
\pgfpathlineto{\pgfqpoint{4.136628in}{1.847335in}}%
\pgfpathlineto{\pgfqpoint{4.128988in}{1.840083in}}%
\pgfpathlineto{\pgfqpoint{4.121343in}{1.832924in}}%
\pgfpathlineto{\pgfqpoint{4.113692in}{1.825862in}}%
\pgfpathlineto{\pgfqpoint{4.100402in}{1.829709in}}%
\pgfpathlineto{\pgfqpoint{4.087119in}{1.833582in}}%
\pgfpathlineto{\pgfqpoint{4.073841in}{1.837480in}}%
\pgfpathlineto{\pgfqpoint{4.060569in}{1.841403in}}%
\pgfpathlineto{\pgfqpoint{4.068234in}{1.848266in}}%
\pgfpathlineto{\pgfqpoint{4.075893in}{1.855228in}}%
\pgfpathlineto{\pgfqpoint{4.083546in}{1.862286in}}%
\pgfpathlineto{\pgfqpoint{4.091193in}{1.869435in}}%
\pgfpathclose%
\pgfusepath{fill}%
\end{pgfscope}%
\begin{pgfscope}%
\pgfpathrectangle{\pgfqpoint{1.254980in}{0.150000in}}{\pgfqpoint{5.490039in}{5.490039in}}%
\pgfusepath{clip}%
\pgfsetbuttcap%
\pgfsetroundjoin%
\definecolor{currentfill}{rgb}{0.280894,0.078907,0.402329}%
\pgfsetfillcolor{currentfill}%
\pgfsetfillopacity{0.700000}%
\pgfsetlinewidth{0.000000pt}%
\definecolor{currentstroke}{rgb}{0.000000,0.000000,0.000000}%
\pgfsetstrokecolor{currentstroke}%
\pgfsetdash{}{0pt}%
\pgfpathmoveto{\pgfqpoint{4.835466in}{2.000391in}}%
\pgfpathlineto{\pgfqpoint{4.848926in}{1.998656in}}%
\pgfpathlineto{\pgfqpoint{4.862393in}{1.996944in}}%
\pgfpathlineto{\pgfqpoint{4.875868in}{1.995257in}}%
\pgfpathlineto{\pgfqpoint{4.889351in}{1.993594in}}%
\pgfpathlineto{\pgfqpoint{4.881970in}{1.984374in}}%
\pgfpathlineto{\pgfqpoint{4.874582in}{1.975112in}}%
\pgfpathlineto{\pgfqpoint{4.867190in}{1.965810in}}%
\pgfpathlineto{\pgfqpoint{4.859791in}{1.956471in}}%
\pgfpathlineto{\pgfqpoint{4.846299in}{1.958240in}}%
\pgfpathlineto{\pgfqpoint{4.832815in}{1.960033in}}%
\pgfpathlineto{\pgfqpoint{4.819339in}{1.961850in}}%
\pgfpathlineto{\pgfqpoint{4.805870in}{1.963691in}}%
\pgfpathlineto{\pgfqpoint{4.813277in}{1.972919in}}%
\pgfpathlineto{\pgfqpoint{4.820679in}{1.982114in}}%
\pgfpathlineto{\pgfqpoint{4.828075in}{1.991272in}}%
\pgfpathlineto{\pgfqpoint{4.835466in}{2.000391in}}%
\pgfpathclose%
\pgfusepath{fill}%
\end{pgfscope}%
\begin{pgfscope}%
\pgfpathrectangle{\pgfqpoint{1.254980in}{0.150000in}}{\pgfqpoint{5.490039in}{5.490039in}}%
\pgfusepath{clip}%
\pgfsetbuttcap%
\pgfsetroundjoin%
\definecolor{currentfill}{rgb}{0.225863,0.330805,0.547314}%
\pgfsetfillcolor{currentfill}%
\pgfsetfillopacity{0.700000}%
\pgfsetlinewidth{0.000000pt}%
\definecolor{currentstroke}{rgb}{0.000000,0.000000,0.000000}%
\pgfsetstrokecolor{currentstroke}%
\pgfsetdash{}{0pt}%
\pgfpathmoveto{\pgfqpoint{2.438128in}{2.492025in}}%
\pgfpathlineto{\pgfqpoint{2.451189in}{2.482552in}}%
\pgfpathlineto{\pgfqpoint{2.464252in}{2.473128in}}%
\pgfpathlineto{\pgfqpoint{2.477317in}{2.463752in}}%
\pgfpathlineto{\pgfqpoint{2.490383in}{2.454425in}}%
\pgfpathlineto{\pgfqpoint{2.481664in}{2.462016in}}%
\pgfpathlineto{\pgfqpoint{2.472921in}{2.470021in}}%
\pgfpathlineto{\pgfqpoint{2.464153in}{2.478449in}}%
\pgfpathlineto{\pgfqpoint{2.455359in}{2.487309in}}%
\pgfpathlineto{\pgfqpoint{2.442252in}{2.496959in}}%
\pgfpathlineto{\pgfqpoint{2.429146in}{2.506658in}}%
\pgfpathlineto{\pgfqpoint{2.416041in}{2.516407in}}%
\pgfpathlineto{\pgfqpoint{2.402938in}{2.526204in}}%
\pgfpathlineto{\pgfqpoint{2.411774in}{2.517015in}}%
\pgfpathlineto{\pgfqpoint{2.420584in}{2.508261in}}%
\pgfpathlineto{\pgfqpoint{2.429368in}{2.499934in}}%
\pgfpathlineto{\pgfqpoint{2.438128in}{2.492025in}}%
\pgfpathclose%
\pgfusepath{fill}%
\end{pgfscope}%
\begin{pgfscope}%
\pgfpathrectangle{\pgfqpoint{1.254980in}{0.150000in}}{\pgfqpoint{5.490039in}{5.490039in}}%
\pgfusepath{clip}%
\pgfsetbuttcap%
\pgfsetroundjoin%
\definecolor{currentfill}{rgb}{0.276194,0.190074,0.493001}%
\pgfsetfillcolor{currentfill}%
\pgfsetfillopacity{0.700000}%
\pgfsetlinewidth{0.000000pt}%
\definecolor{currentstroke}{rgb}{0.000000,0.000000,0.000000}%
\pgfsetstrokecolor{currentstroke}%
\pgfsetdash{}{0pt}%
\pgfpathmoveto{\pgfqpoint{5.526960in}{2.199502in}}%
\pgfpathlineto{\pgfqpoint{5.540642in}{2.198999in}}%
\pgfpathlineto{\pgfqpoint{5.554333in}{2.198519in}}%
\pgfpathlineto{\pgfqpoint{5.568033in}{2.198063in}}%
\pgfpathlineto{\pgfqpoint{5.581741in}{2.197631in}}%
\pgfpathlineto{\pgfqpoint{5.574643in}{2.190079in}}%
\pgfpathlineto{\pgfqpoint{5.567537in}{2.182426in}}%
\pgfpathlineto{\pgfqpoint{5.560422in}{2.174674in}}%
\pgfpathlineto{\pgfqpoint{5.553300in}{2.166822in}}%
\pgfpathlineto{\pgfqpoint{5.539580in}{2.167268in}}%
\pgfpathlineto{\pgfqpoint{5.525868in}{2.167737in}}%
\pgfpathlineto{\pgfqpoint{5.512166in}{2.168230in}}%
\pgfpathlineto{\pgfqpoint{5.498472in}{2.168747in}}%
\pgfpathlineto{\pgfqpoint{5.505606in}{2.176581in}}%
\pgfpathlineto{\pgfqpoint{5.512732in}{2.184318in}}%
\pgfpathlineto{\pgfqpoint{5.519850in}{2.191958in}}%
\pgfpathlineto{\pgfqpoint{5.526960in}{2.199502in}}%
\pgfpathclose%
\pgfusepath{fill}%
\end{pgfscope}%
\begin{pgfscope}%
\pgfpathrectangle{\pgfqpoint{1.254980in}{0.150000in}}{\pgfqpoint{5.490039in}{5.490039in}}%
\pgfusepath{clip}%
\pgfsetbuttcap%
\pgfsetroundjoin%
\definecolor{currentfill}{rgb}{0.283072,0.130895,0.449241}%
\pgfsetfillcolor{currentfill}%
\pgfsetfillopacity{0.700000}%
\pgfsetlinewidth{0.000000pt}%
\definecolor{currentstroke}{rgb}{0.000000,0.000000,0.000000}%
\pgfsetstrokecolor{currentstroke}%
\pgfsetdash{}{0pt}%
\pgfpathmoveto{\pgfqpoint{5.139537in}{2.085012in}}%
\pgfpathlineto{\pgfqpoint{5.153094in}{2.083899in}}%
\pgfpathlineto{\pgfqpoint{5.166660in}{2.082809in}}%
\pgfpathlineto{\pgfqpoint{5.180234in}{2.081744in}}%
\pgfpathlineto{\pgfqpoint{5.193816in}{2.080702in}}%
\pgfpathlineto{\pgfqpoint{5.186545in}{2.071887in}}%
\pgfpathlineto{\pgfqpoint{5.179268in}{2.062997in}}%
\pgfpathlineto{\pgfqpoint{5.171984in}{2.054031in}}%
\pgfpathlineto{\pgfqpoint{5.164693in}{2.044993in}}%
\pgfpathlineto{\pgfqpoint{5.151102in}{2.046101in}}%
\pgfpathlineto{\pgfqpoint{5.137518in}{2.047233in}}%
\pgfpathlineto{\pgfqpoint{5.123943in}{2.048389in}}%
\pgfpathlineto{\pgfqpoint{5.110376in}{2.049569in}}%
\pgfpathlineto{\pgfqpoint{5.117676in}{2.058536in}}%
\pgfpathlineto{\pgfqpoint{5.124969in}{2.067433in}}%
\pgfpathlineto{\pgfqpoint{5.132256in}{2.076259in}}%
\pgfpathlineto{\pgfqpoint{5.139537in}{2.085012in}}%
\pgfpathclose%
\pgfusepath{fill}%
\end{pgfscope}%
\begin{pgfscope}%
\pgfpathrectangle{\pgfqpoint{1.254980in}{0.150000in}}{\pgfqpoint{5.490039in}{5.490039in}}%
\pgfusepath{clip}%
\pgfsetbuttcap%
\pgfsetroundjoin%
\definecolor{currentfill}{rgb}{0.283091,0.110553,0.431554}%
\pgfsetfillcolor{currentfill}%
\pgfsetfillopacity{0.700000}%
\pgfsetlinewidth{0.000000pt}%
\definecolor{currentstroke}{rgb}{0.000000,0.000000,0.000000}%
\pgfsetstrokecolor{currentstroke}%
\pgfsetdash{}{0pt}%
\pgfpathmoveto{\pgfqpoint{3.218410in}{2.035621in}}%
\pgfpathlineto{\pgfqpoint{3.231517in}{2.029067in}}%
\pgfpathlineto{\pgfqpoint{3.244628in}{2.022544in}}%
\pgfpathlineto{\pgfqpoint{3.257742in}{2.016053in}}%
\pgfpathlineto{\pgfqpoint{3.270861in}{2.009593in}}%
\pgfpathlineto{\pgfqpoint{3.262782in}{2.009345in}}%
\pgfpathlineto{\pgfqpoint{3.254689in}{2.009368in}}%
\pgfpathlineto{\pgfqpoint{3.246584in}{2.009671in}}%
\pgfpathlineto{\pgfqpoint{3.238465in}{2.010259in}}%
\pgfpathlineto{\pgfqpoint{3.225319in}{2.016993in}}%
\pgfpathlineto{\pgfqpoint{3.212178in}{2.023758in}}%
\pgfpathlineto{\pgfqpoint{3.199041in}{2.030555in}}%
\pgfpathlineto{\pgfqpoint{3.185907in}{2.037384in}}%
\pgfpathlineto{\pgfqpoint{3.194053in}{2.036516in}}%
\pgfpathlineto{\pgfqpoint{3.202186in}{2.035938in}}%
\pgfpathlineto{\pgfqpoint{3.210304in}{2.035642in}}%
\pgfpathlineto{\pgfqpoint{3.218410in}{2.035621in}}%
\pgfpathclose%
\pgfusepath{fill}%
\end{pgfscope}%
\begin{pgfscope}%
\pgfpathrectangle{\pgfqpoint{1.254980in}{0.150000in}}{\pgfqpoint{5.490039in}{5.490039in}}%
\pgfusepath{clip}%
\pgfsetbuttcap%
\pgfsetroundjoin%
\definecolor{currentfill}{rgb}{0.279566,0.067836,0.391917}%
\pgfsetfillcolor{currentfill}%
\pgfsetfillopacity{0.700000}%
\pgfsetlinewidth{0.000000pt}%
\definecolor{currentstroke}{rgb}{0.000000,0.000000,0.000000}%
\pgfsetstrokecolor{currentstroke}%
\pgfsetdash{}{0pt}%
\pgfpathmoveto{\pgfqpoint{4.752069in}{1.971295in}}%
\pgfpathlineto{\pgfqpoint{4.765508in}{1.969358in}}%
\pgfpathlineto{\pgfqpoint{4.778955in}{1.967445in}}%
\pgfpathlineto{\pgfqpoint{4.792408in}{1.965556in}}%
\pgfpathlineto{\pgfqpoint{4.805870in}{1.963691in}}%
\pgfpathlineto{\pgfqpoint{4.798457in}{1.954430in}}%
\pgfpathlineto{\pgfqpoint{4.791039in}{1.945141in}}%
\pgfpathlineto{\pgfqpoint{4.783615in}{1.935824in}}%
\pgfpathlineto{\pgfqpoint{4.776186in}{1.926484in}}%
\pgfpathlineto{\pgfqpoint{4.762716in}{1.928468in}}%
\pgfpathlineto{\pgfqpoint{4.749253in}{1.930476in}}%
\pgfpathlineto{\pgfqpoint{4.735797in}{1.932508in}}%
\pgfpathlineto{\pgfqpoint{4.722349in}{1.934563in}}%
\pgfpathlineto{\pgfqpoint{4.729787in}{1.943780in}}%
\pgfpathlineto{\pgfqpoint{4.737220in}{1.952976in}}%
\pgfpathlineto{\pgfqpoint{4.744647in}{1.962149in}}%
\pgfpathlineto{\pgfqpoint{4.752069in}{1.971295in}}%
\pgfpathclose%
\pgfusepath{fill}%
\end{pgfscope}%
\begin{pgfscope}%
\pgfpathrectangle{\pgfqpoint{1.254980in}{0.150000in}}{\pgfqpoint{5.490039in}{5.490039in}}%
\pgfusepath{clip}%
\pgfsetbuttcap%
\pgfsetroundjoin%
\definecolor{currentfill}{rgb}{0.268510,0.009605,0.335427}%
\pgfsetfillcolor{currentfill}%
\pgfsetfillopacity{0.700000}%
\pgfsetlinewidth{0.000000pt}%
\definecolor{currentstroke}{rgb}{0.000000,0.000000,0.000000}%
\pgfsetstrokecolor{currentstroke}%
\pgfsetdash{}{0pt}%
\pgfpathmoveto{\pgfqpoint{4.227859in}{1.871182in}}%
\pgfpathlineto{\pgfqpoint{4.241155in}{1.867838in}}%
\pgfpathlineto{\pgfqpoint{4.254457in}{1.864519in}}%
\pgfpathlineto{\pgfqpoint{4.267765in}{1.861225in}}%
\pgfpathlineto{\pgfqpoint{4.281080in}{1.857956in}}%
\pgfpathlineto{\pgfqpoint{4.273493in}{1.849959in}}%
\pgfpathlineto{\pgfqpoint{4.265900in}{1.842024in}}%
\pgfpathlineto{\pgfqpoint{4.258303in}{1.834157in}}%
\pgfpathlineto{\pgfqpoint{4.250699in}{1.826362in}}%
\pgfpathlineto{\pgfqpoint{4.237373in}{1.829813in}}%
\pgfpathlineto{\pgfqpoint{4.224052in}{1.833289in}}%
\pgfpathlineto{\pgfqpoint{4.210738in}{1.836791in}}%
\pgfpathlineto{\pgfqpoint{4.197430in}{1.840317in}}%
\pgfpathlineto{\pgfqpoint{4.205046in}{1.847925in}}%
\pgfpathlineto{\pgfqpoint{4.212656in}{1.855609in}}%
\pgfpathlineto{\pgfqpoint{4.220260in}{1.863363in}}%
\pgfpathlineto{\pgfqpoint{4.227859in}{1.871182in}}%
\pgfpathclose%
\pgfusepath{fill}%
\end{pgfscope}%
\begin{pgfscope}%
\pgfpathrectangle{\pgfqpoint{1.254980in}{0.150000in}}{\pgfqpoint{5.490039in}{5.490039in}}%
\pgfusepath{clip}%
\pgfsetbuttcap%
\pgfsetroundjoin%
\definecolor{currentfill}{rgb}{0.272594,0.025563,0.353093}%
\pgfsetfillcolor{currentfill}%
\pgfsetfillopacity{0.700000}%
\pgfsetlinewidth{0.000000pt}%
\definecolor{currentstroke}{rgb}{0.000000,0.000000,0.000000}%
\pgfsetstrokecolor{currentstroke}%
\pgfsetdash{}{0pt}%
\pgfpathmoveto{\pgfqpoint{4.448167in}{1.901046in}}%
\pgfpathlineto{\pgfqpoint{4.461520in}{1.898325in}}%
\pgfpathlineto{\pgfqpoint{4.474880in}{1.895630in}}%
\pgfpathlineto{\pgfqpoint{4.488247in}{1.892958in}}%
\pgfpathlineto{\pgfqpoint{4.501621in}{1.890312in}}%
\pgfpathlineto{\pgfqpoint{4.494108in}{1.881504in}}%
\pgfpathlineto{\pgfqpoint{4.486589in}{1.872717in}}%
\pgfpathlineto{\pgfqpoint{4.479066in}{1.863955in}}%
\pgfpathlineto{\pgfqpoint{4.471538in}{1.855222in}}%
\pgfpathlineto{\pgfqpoint{4.458154in}{1.858025in}}%
\pgfpathlineto{\pgfqpoint{4.444776in}{1.860854in}}%
\pgfpathlineto{\pgfqpoint{4.431406in}{1.863706in}}%
\pgfpathlineto{\pgfqpoint{4.418042in}{1.866584in}}%
\pgfpathlineto{\pgfqpoint{4.425581in}{1.875155in}}%
\pgfpathlineto{\pgfqpoint{4.433115in}{1.883758in}}%
\pgfpathlineto{\pgfqpoint{4.440643in}{1.892390in}}%
\pgfpathlineto{\pgfqpoint{4.448167in}{1.901046in}}%
\pgfpathclose%
\pgfusepath{fill}%
\end{pgfscope}%
\begin{pgfscope}%
\pgfpathrectangle{\pgfqpoint{1.254980in}{0.150000in}}{\pgfqpoint{5.490039in}{5.490039in}}%
\pgfusepath{clip}%
\pgfsetbuttcap%
\pgfsetroundjoin%
\definecolor{currentfill}{rgb}{0.279566,0.067836,0.391917}%
\pgfsetfillcolor{currentfill}%
\pgfsetfillopacity{0.700000}%
\pgfsetlinewidth{0.000000pt}%
\definecolor{currentstroke}{rgb}{0.000000,0.000000,0.000000}%
\pgfsetstrokecolor{currentstroke}%
\pgfsetdash{}{0pt}%
\pgfpathmoveto{\pgfqpoint{3.407962in}{1.964675in}}%
\pgfpathlineto{\pgfqpoint{3.421095in}{1.958750in}}%
\pgfpathlineto{\pgfqpoint{3.434233in}{1.952855in}}%
\pgfpathlineto{\pgfqpoint{3.447375in}{1.946990in}}%
\pgfpathlineto{\pgfqpoint{3.460521in}{1.941154in}}%
\pgfpathlineto{\pgfqpoint{3.452561in}{1.939124in}}%
\pgfpathlineto{\pgfqpoint{3.444589in}{1.937328in}}%
\pgfpathlineto{\pgfqpoint{3.436607in}{1.935772in}}%
\pgfpathlineto{\pgfqpoint{3.428614in}{1.934464in}}%
\pgfpathlineto{\pgfqpoint{3.415444in}{1.940560in}}%
\pgfpathlineto{\pgfqpoint{3.402279in}{1.946685in}}%
\pgfpathlineto{\pgfqpoint{3.389118in}{1.952840in}}%
\pgfpathlineto{\pgfqpoint{3.375961in}{1.959025in}}%
\pgfpathlineto{\pgfqpoint{3.383978in}{1.960068in}}%
\pgfpathlineto{\pgfqpoint{3.391984in}{1.961362in}}%
\pgfpathlineto{\pgfqpoint{3.399979in}{1.962900in}}%
\pgfpathlineto{\pgfqpoint{3.407962in}{1.964675in}}%
\pgfpathclose%
\pgfusepath{fill}%
\end{pgfscope}%
\begin{pgfscope}%
\pgfpathrectangle{\pgfqpoint{1.254980in}{0.150000in}}{\pgfqpoint{5.490039in}{5.490039in}}%
\pgfusepath{clip}%
\pgfsetbuttcap%
\pgfsetroundjoin%
\definecolor{currentfill}{rgb}{0.267968,0.223549,0.512008}%
\pgfsetfillcolor{currentfill}%
\pgfsetfillopacity{0.700000}%
\pgfsetlinewidth{0.000000pt}%
\definecolor{currentstroke}{rgb}{0.000000,0.000000,0.000000}%
\pgfsetstrokecolor{currentstroke}%
\pgfsetdash{}{0pt}%
\pgfpathmoveto{\pgfqpoint{2.785991in}{2.257071in}}%
\pgfpathlineto{\pgfqpoint{2.799064in}{2.248963in}}%
\pgfpathlineto{\pgfqpoint{2.812140in}{2.240894in}}%
\pgfpathlineto{\pgfqpoint{2.825218in}{2.232864in}}%
\pgfpathlineto{\pgfqpoint{2.838299in}{2.224873in}}%
\pgfpathlineto{\pgfqpoint{2.829888in}{2.229090in}}%
\pgfpathlineto{\pgfqpoint{2.821457in}{2.233665in}}%
\pgfpathlineto{\pgfqpoint{2.813007in}{2.238606in}}%
\pgfpathlineto{\pgfqpoint{2.804537in}{2.243921in}}%
\pgfpathlineto{\pgfqpoint{2.791421in}{2.252217in}}%
\pgfpathlineto{\pgfqpoint{2.778308in}{2.260551in}}%
\pgfpathlineto{\pgfqpoint{2.765197in}{2.268925in}}%
\pgfpathlineto{\pgfqpoint{2.752089in}{2.277338in}}%
\pgfpathlineto{\pgfqpoint{2.760594in}{2.271713in}}%
\pgfpathlineto{\pgfqpoint{2.769080in}{2.266465in}}%
\pgfpathlineto{\pgfqpoint{2.777545in}{2.261588in}}%
\pgfpathlineto{\pgfqpoint{2.785991in}{2.257071in}}%
\pgfpathclose%
\pgfusepath{fill}%
\end{pgfscope}%
\begin{pgfscope}%
\pgfpathrectangle{\pgfqpoint{1.254980in}{0.150000in}}{\pgfqpoint{5.490039in}{5.490039in}}%
\pgfusepath{clip}%
\pgfsetbuttcap%
\pgfsetroundjoin%
\definecolor{currentfill}{rgb}{0.281412,0.155834,0.469201}%
\pgfsetfillcolor{currentfill}%
\pgfsetfillopacity{0.700000}%
\pgfsetlinewidth{0.000000pt}%
\definecolor{currentstroke}{rgb}{0.000000,0.000000,0.000000}%
\pgfsetstrokecolor{currentstroke}%
\pgfsetdash{}{0pt}%
\pgfpathmoveto{\pgfqpoint{3.028591in}{2.121894in}}%
\pgfpathlineto{\pgfqpoint{3.041681in}{2.114666in}}%
\pgfpathlineto{\pgfqpoint{3.054774in}{2.107472in}}%
\pgfpathlineto{\pgfqpoint{3.067871in}{2.100313in}}%
\pgfpathlineto{\pgfqpoint{3.080972in}{2.093188in}}%
\pgfpathlineto{\pgfqpoint{3.072754in}{2.094919in}}%
\pgfpathlineto{\pgfqpoint{3.064521in}{2.096961in}}%
\pgfpathlineto{\pgfqpoint{3.056272in}{2.099323in}}%
\pgfpathlineto{\pgfqpoint{3.048007in}{2.102013in}}%
\pgfpathlineto{\pgfqpoint{3.034876in}{2.109426in}}%
\pgfpathlineto{\pgfqpoint{3.021749in}{2.116874in}}%
\pgfpathlineto{\pgfqpoint{3.008625in}{2.124357in}}%
\pgfpathlineto{\pgfqpoint{2.995504in}{2.131874in}}%
\pgfpathlineto{\pgfqpoint{3.003800in}{2.128890in}}%
\pgfpathlineto{\pgfqpoint{3.012080in}{2.126238in}}%
\pgfpathlineto{\pgfqpoint{3.020343in}{2.123908in}}%
\pgfpathlineto{\pgfqpoint{3.028591in}{2.121894in}}%
\pgfpathclose%
\pgfusepath{fill}%
\end{pgfscope}%
\begin{pgfscope}%
\pgfpathrectangle{\pgfqpoint{1.254980in}{0.150000in}}{\pgfqpoint{5.490039in}{5.490039in}}%
\pgfusepath{clip}%
\pgfsetbuttcap%
\pgfsetroundjoin%
\definecolor{currentfill}{rgb}{0.231674,0.318106,0.544834}%
\pgfsetfillcolor{currentfill}%
\pgfsetfillopacity{0.700000}%
\pgfsetlinewidth{0.000000pt}%
\definecolor{currentstroke}{rgb}{0.000000,0.000000,0.000000}%
\pgfsetstrokecolor{currentstroke}%
\pgfsetdash{}{0pt}%
\pgfpathmoveto{\pgfqpoint{2.490383in}{2.454425in}}%
\pgfpathlineto{\pgfqpoint{2.503450in}{2.445146in}}%
\pgfpathlineto{\pgfqpoint{2.516519in}{2.435914in}}%
\pgfpathlineto{\pgfqpoint{2.529590in}{2.426729in}}%
\pgfpathlineto{\pgfqpoint{2.542662in}{2.417590in}}%
\pgfpathlineto{\pgfqpoint{2.533983in}{2.424864in}}%
\pgfpathlineto{\pgfqpoint{2.525281in}{2.432548in}}%
\pgfpathlineto{\pgfqpoint{2.516554in}{2.440651in}}%
\pgfpathlineto{\pgfqpoint{2.507802in}{2.449182in}}%
\pgfpathlineto{\pgfqpoint{2.494689in}{2.458644in}}%
\pgfpathlineto{\pgfqpoint{2.481577in}{2.468151in}}%
\pgfpathlineto{\pgfqpoint{2.468468in}{2.477706in}}%
\pgfpathlineto{\pgfqpoint{2.455359in}{2.487309in}}%
\pgfpathlineto{\pgfqpoint{2.464153in}{2.478449in}}%
\pgfpathlineto{\pgfqpoint{2.472921in}{2.470021in}}%
\pgfpathlineto{\pgfqpoint{2.481664in}{2.462016in}}%
\pgfpathlineto{\pgfqpoint{2.490383in}{2.454425in}}%
\pgfpathclose%
\pgfusepath{fill}%
\end{pgfscope}%
\begin{pgfscope}%
\pgfpathrectangle{\pgfqpoint{1.254980in}{0.150000in}}{\pgfqpoint{5.490039in}{5.490039in}}%
\pgfusepath{clip}%
\pgfsetbuttcap%
\pgfsetroundjoin%
\definecolor{currentfill}{rgb}{0.268510,0.009605,0.335427}%
\pgfsetfillcolor{currentfill}%
\pgfsetfillopacity{0.700000}%
\pgfsetlinewidth{0.000000pt}%
\definecolor{currentstroke}{rgb}{0.000000,0.000000,0.000000}%
\pgfsetstrokecolor{currentstroke}%
\pgfsetdash{}{0pt}%
\pgfpathmoveto{\pgfqpoint{3.870871in}{1.866641in}}%
\pgfpathlineto{\pgfqpoint{3.884088in}{1.862199in}}%
\pgfpathlineto{\pgfqpoint{3.897311in}{1.857783in}}%
\pgfpathlineto{\pgfqpoint{3.910539in}{1.853394in}}%
\pgfpathlineto{\pgfqpoint{3.923772in}{1.849030in}}%
\pgfpathlineto{\pgfqpoint{3.916046in}{1.843177in}}%
\pgfpathlineto{\pgfqpoint{3.908313in}{1.837461in}}%
\pgfpathlineto{\pgfqpoint{3.900573in}{1.831889in}}%
\pgfpathlineto{\pgfqpoint{3.892826in}{1.826466in}}%
\pgfpathlineto{\pgfqpoint{3.879577in}{1.831050in}}%
\pgfpathlineto{\pgfqpoint{3.866332in}{1.835660in}}%
\pgfpathlineto{\pgfqpoint{3.853094in}{1.840296in}}%
\pgfpathlineto{\pgfqpoint{3.839860in}{1.844959in}}%
\pgfpathlineto{\pgfqpoint{3.847624in}{1.850157in}}%
\pgfpathlineto{\pgfqpoint{3.855380in}{1.855507in}}%
\pgfpathlineto{\pgfqpoint{3.863129in}{1.861003in}}%
\pgfpathlineto{\pgfqpoint{3.870871in}{1.866641in}}%
\pgfpathclose%
\pgfusepath{fill}%
\end{pgfscope}%
\begin{pgfscope}%
\pgfpathrectangle{\pgfqpoint{1.254980in}{0.150000in}}{\pgfqpoint{5.490039in}{5.490039in}}%
\pgfusepath{clip}%
\pgfsetbuttcap%
\pgfsetroundjoin%
\definecolor{currentfill}{rgb}{0.283197,0.115680,0.436115}%
\pgfsetfillcolor{currentfill}%
\pgfsetfillopacity{0.700000}%
\pgfsetlinewidth{0.000000pt}%
\definecolor{currentstroke}{rgb}{0.000000,0.000000,0.000000}%
\pgfsetstrokecolor{currentstroke}%
\pgfsetdash{}{0pt}%
\pgfpathmoveto{\pgfqpoint{5.056188in}{2.054527in}}%
\pgfpathlineto{\pgfqpoint{5.069723in}{2.053252in}}%
\pgfpathlineto{\pgfqpoint{5.083266in}{2.052000in}}%
\pgfpathlineto{\pgfqpoint{5.096817in}{2.050773in}}%
\pgfpathlineto{\pgfqpoint{5.110376in}{2.049569in}}%
\pgfpathlineto{\pgfqpoint{5.103070in}{2.040533in}}%
\pgfpathlineto{\pgfqpoint{5.095757in}{2.031431in}}%
\pgfpathlineto{\pgfqpoint{5.088439in}{2.022263in}}%
\pgfpathlineto{\pgfqpoint{5.081114in}{2.013031in}}%
\pgfpathlineto{\pgfqpoint{5.067546in}{2.014315in}}%
\pgfpathlineto{\pgfqpoint{5.053985in}{2.015622in}}%
\pgfpathlineto{\pgfqpoint{5.040433in}{2.016953in}}%
\pgfpathlineto{\pgfqpoint{5.026889in}{2.018308in}}%
\pgfpathlineto{\pgfqpoint{5.034223in}{2.027455in}}%
\pgfpathlineto{\pgfqpoint{5.041551in}{2.036542in}}%
\pgfpathlineto{\pgfqpoint{5.048873in}{2.045566in}}%
\pgfpathlineto{\pgfqpoint{5.056188in}{2.054527in}}%
\pgfpathclose%
\pgfusepath{fill}%
\end{pgfscope}%
\begin{pgfscope}%
\pgfpathrectangle{\pgfqpoint{1.254980in}{0.150000in}}{\pgfqpoint{5.490039in}{5.490039in}}%
\pgfusepath{clip}%
\pgfsetbuttcap%
\pgfsetroundjoin%
\definecolor{currentfill}{rgb}{0.271305,0.019942,0.347269}%
\pgfsetfillcolor{currentfill}%
\pgfsetfillopacity{0.700000}%
\pgfsetlinewidth{0.000000pt}%
\definecolor{currentstroke}{rgb}{0.000000,0.000000,0.000000}%
\pgfsetstrokecolor{currentstroke}%
\pgfsetdash{}{0pt}%
\pgfpathmoveto{\pgfqpoint{3.734190in}{1.883220in}}%
\pgfpathlineto{\pgfqpoint{3.747380in}{1.878343in}}%
\pgfpathlineto{\pgfqpoint{3.760575in}{1.873493in}}%
\pgfpathlineto{\pgfqpoint{3.773776in}{1.868671in}}%
\pgfpathlineto{\pgfqpoint{3.786982in}{1.863875in}}%
\pgfpathlineto{\pgfqpoint{3.779194in}{1.859063in}}%
\pgfpathlineto{\pgfqpoint{3.771398in}{1.854419in}}%
\pgfpathlineto{\pgfqpoint{3.763594in}{1.849947in}}%
\pgfpathlineto{\pgfqpoint{3.755782in}{1.845655in}}%
\pgfpathlineto{\pgfqpoint{3.742558in}{1.850684in}}%
\pgfpathlineto{\pgfqpoint{3.729339in}{1.855740in}}%
\pgfpathlineto{\pgfqpoint{3.716125in}{1.860824in}}%
\pgfpathlineto{\pgfqpoint{3.702916in}{1.865934in}}%
\pgfpathlineto{\pgfqpoint{3.710747in}{1.869988in}}%
\pgfpathlineto{\pgfqpoint{3.718569in}{1.874224in}}%
\pgfpathlineto{\pgfqpoint{3.726383in}{1.878637in}}%
\pgfpathlineto{\pgfqpoint{3.734190in}{1.883220in}}%
\pgfpathclose%
\pgfusepath{fill}%
\end{pgfscope}%
\begin{pgfscope}%
\pgfpathrectangle{\pgfqpoint{1.254980in}{0.150000in}}{\pgfqpoint{5.490039in}{5.490039in}}%
\pgfusepath{clip}%
\pgfsetbuttcap%
\pgfsetroundjoin%
\definecolor{currentfill}{rgb}{0.278826,0.175490,0.483397}%
\pgfsetfillcolor{currentfill}%
\pgfsetfillopacity{0.700000}%
\pgfsetlinewidth{0.000000pt}%
\definecolor{currentstroke}{rgb}{0.000000,0.000000,0.000000}%
\pgfsetstrokecolor{currentstroke}%
\pgfsetdash{}{0pt}%
\pgfpathmoveto{\pgfqpoint{5.443782in}{2.171052in}}%
\pgfpathlineto{\pgfqpoint{5.457441in}{2.170440in}}%
\pgfpathlineto{\pgfqpoint{5.471110in}{2.169852in}}%
\pgfpathlineto{\pgfqpoint{5.484786in}{2.169288in}}%
\pgfpathlineto{\pgfqpoint{5.498472in}{2.168747in}}%
\pgfpathlineto{\pgfqpoint{5.491330in}{2.160816in}}%
\pgfpathlineto{\pgfqpoint{5.484180in}{2.152789in}}%
\pgfpathlineto{\pgfqpoint{5.477023in}{2.144665in}}%
\pgfpathlineto{\pgfqpoint{5.469858in}{2.136445in}}%
\pgfpathlineto{\pgfqpoint{5.456161in}{2.137012in}}%
\pgfpathlineto{\pgfqpoint{5.442473in}{2.137604in}}%
\pgfpathlineto{\pgfqpoint{5.428794in}{2.138218in}}%
\pgfpathlineto{\pgfqpoint{5.415123in}{2.138857in}}%
\pgfpathlineto{\pgfqpoint{5.422299in}{2.147046in}}%
\pgfpathlineto{\pgfqpoint{5.429468in}{2.155141in}}%
\pgfpathlineto{\pgfqpoint{5.436629in}{2.163143in}}%
\pgfpathlineto{\pgfqpoint{5.443782in}{2.171052in}}%
\pgfpathclose%
\pgfusepath{fill}%
\end{pgfscope}%
\begin{pgfscope}%
\pgfpathrectangle{\pgfqpoint{1.254980in}{0.150000in}}{\pgfqpoint{5.490039in}{5.490039in}}%
\pgfusepath{clip}%
\pgfsetbuttcap%
\pgfsetroundjoin%
\definecolor{currentfill}{rgb}{0.269308,0.218818,0.509577}%
\pgfsetfillcolor{currentfill}%
\pgfsetfillopacity{0.700000}%
\pgfsetlinewidth{0.000000pt}%
\definecolor{currentstroke}{rgb}{0.000000,0.000000,0.000000}%
\pgfsetstrokecolor{currentstroke}%
\pgfsetdash{}{0pt}%
\pgfpathmoveto{\pgfqpoint{5.748005in}{2.251805in}}%
\pgfpathlineto{\pgfqpoint{5.761767in}{2.251573in}}%
\pgfpathlineto{\pgfqpoint{5.775538in}{2.251364in}}%
\pgfpathlineto{\pgfqpoint{5.789318in}{2.251180in}}%
\pgfpathlineto{\pgfqpoint{5.782323in}{2.244458in}}%
\pgfpathlineto{\pgfqpoint{5.775320in}{2.237631in}}%
\pgfpathlineto{\pgfqpoint{5.768309in}{2.230700in}}%
\pgfpathlineto{\pgfqpoint{5.761288in}{2.223664in}}%
\pgfpathlineto{\pgfqpoint{5.747495in}{2.223835in}}%
\pgfpathlineto{\pgfqpoint{5.733710in}{2.224029in}}%
\pgfpathlineto{\pgfqpoint{5.719935in}{2.224248in}}%
\pgfpathlineto{\pgfqpoint{5.726965in}{2.231291in}}%
\pgfpathlineto{\pgfqpoint{5.733987in}{2.238231in}}%
\pgfpathlineto{\pgfqpoint{5.741001in}{2.245069in}}%
\pgfpathlineto{\pgfqpoint{5.748005in}{2.251805in}}%
\pgfpathclose%
\pgfusepath{fill}%
\end{pgfscope}%
\begin{pgfscope}%
\pgfpathrectangle{\pgfqpoint{1.254980in}{0.150000in}}{\pgfqpoint{5.490039in}{5.490039in}}%
\pgfusepath{clip}%
\pgfsetbuttcap%
\pgfsetroundjoin%
\definecolor{currentfill}{rgb}{0.277941,0.056324,0.381191}%
\pgfsetfillcolor{currentfill}%
\pgfsetfillopacity{0.700000}%
\pgfsetlinewidth{0.000000pt}%
\definecolor{currentstroke}{rgb}{0.000000,0.000000,0.000000}%
\pgfsetstrokecolor{currentstroke}%
\pgfsetdash{}{0pt}%
\pgfpathmoveto{\pgfqpoint{4.668630in}{1.943028in}}%
\pgfpathlineto{\pgfqpoint{4.682049in}{1.940876in}}%
\pgfpathlineto{\pgfqpoint{4.695475in}{1.938748in}}%
\pgfpathlineto{\pgfqpoint{4.708908in}{1.936643in}}%
\pgfpathlineto{\pgfqpoint{4.722349in}{1.934563in}}%
\pgfpathlineto{\pgfqpoint{4.714906in}{1.925329in}}%
\pgfpathlineto{\pgfqpoint{4.707457in}{1.916080in}}%
\pgfpathlineto{\pgfqpoint{4.700003in}{1.906819in}}%
\pgfpathlineto{\pgfqpoint{4.692545in}{1.897548in}}%
\pgfpathlineto{\pgfqpoint{4.679095in}{1.899760in}}%
\pgfpathlineto{\pgfqpoint{4.665652in}{1.901996in}}%
\pgfpathlineto{\pgfqpoint{4.652217in}{1.904255in}}%
\pgfpathlineto{\pgfqpoint{4.638789in}{1.906539in}}%
\pgfpathlineto{\pgfqpoint{4.646257in}{1.915673in}}%
\pgfpathlineto{\pgfqpoint{4.653720in}{1.924801in}}%
\pgfpathlineto{\pgfqpoint{4.661178in}{1.933920in}}%
\pgfpathlineto{\pgfqpoint{4.668630in}{1.943028in}}%
\pgfpathclose%
\pgfusepath{fill}%
\end{pgfscope}%
\begin{pgfscope}%
\pgfpathrectangle{\pgfqpoint{1.254980in}{0.150000in}}{\pgfqpoint{5.490039in}{5.490039in}}%
\pgfusepath{clip}%
\pgfsetbuttcap%
\pgfsetroundjoin%
\definecolor{currentfill}{rgb}{0.267004,0.004874,0.329415}%
\pgfsetfillcolor{currentfill}%
\pgfsetfillopacity{0.700000}%
\pgfsetlinewidth{0.000000pt}%
\definecolor{currentstroke}{rgb}{0.000000,0.000000,0.000000}%
\pgfsetstrokecolor{currentstroke}%
\pgfsetdash{}{0pt}%
\pgfpathmoveto{\pgfqpoint{4.007543in}{1.857353in}}%
\pgfpathlineto{\pgfqpoint{4.020790in}{1.853327in}}%
\pgfpathlineto{\pgfqpoint{4.034044in}{1.849327in}}%
\pgfpathlineto{\pgfqpoint{4.047304in}{1.845352in}}%
\pgfpathlineto{\pgfqpoint{4.060569in}{1.841403in}}%
\pgfpathlineto{\pgfqpoint{4.052898in}{1.834646in}}%
\pgfpathlineto{\pgfqpoint{4.045222in}{1.827999in}}%
\pgfpathlineto{\pgfqpoint{4.037538in}{1.821468in}}%
\pgfpathlineto{\pgfqpoint{4.029849in}{1.815057in}}%
\pgfpathlineto{\pgfqpoint{4.016569in}{1.819213in}}%
\pgfpathlineto{\pgfqpoint{4.003295in}{1.823396in}}%
\pgfpathlineto{\pgfqpoint{3.990027in}{1.827604in}}%
\pgfpathlineto{\pgfqpoint{3.976764in}{1.831837in}}%
\pgfpathlineto{\pgfqpoint{3.984468in}{1.838035in}}%
\pgfpathlineto{\pgfqpoint{3.992166in}{1.844358in}}%
\pgfpathlineto{\pgfqpoint{3.999858in}{1.850798in}}%
\pgfpathlineto{\pgfqpoint{4.007543in}{1.857353in}}%
\pgfpathclose%
\pgfusepath{fill}%
\end{pgfscope}%
\begin{pgfscope}%
\pgfpathrectangle{\pgfqpoint{1.254980in}{0.150000in}}{\pgfqpoint{5.490039in}{5.490039in}}%
\pgfusepath{clip}%
\pgfsetbuttcap%
\pgfsetroundjoin%
\definecolor{currentfill}{rgb}{0.274952,0.037752,0.364543}%
\pgfsetfillcolor{currentfill}%
\pgfsetfillopacity{0.700000}%
\pgfsetlinewidth{0.000000pt}%
\definecolor{currentstroke}{rgb}{0.000000,0.000000,0.000000}%
\pgfsetstrokecolor{currentstroke}%
\pgfsetdash{}{0pt}%
\pgfpathmoveto{\pgfqpoint{3.597431in}{1.907806in}}%
\pgfpathlineto{\pgfqpoint{3.610599in}{1.902474in}}%
\pgfpathlineto{\pgfqpoint{3.623772in}{1.897171in}}%
\pgfpathlineto{\pgfqpoint{3.636950in}{1.891896in}}%
\pgfpathlineto{\pgfqpoint{3.650133in}{1.886648in}}%
\pgfpathlineto{\pgfqpoint{3.642275in}{1.883024in}}%
\pgfpathlineto{\pgfqpoint{3.634407in}{1.879598in}}%
\pgfpathlineto{\pgfqpoint{3.626531in}{1.876375in}}%
\pgfpathlineto{\pgfqpoint{3.618645in}{1.873364in}}%
\pgfpathlineto{\pgfqpoint{3.605442in}{1.878858in}}%
\pgfpathlineto{\pgfqpoint{3.592243in}{1.884379in}}%
\pgfpathlineto{\pgfqpoint{3.579050in}{1.889929in}}%
\pgfpathlineto{\pgfqpoint{3.565861in}{1.895507in}}%
\pgfpathlineto{\pgfqpoint{3.573768in}{1.898267in}}%
\pgfpathlineto{\pgfqpoint{3.581665in}{1.901241in}}%
\pgfpathlineto{\pgfqpoint{3.589552in}{1.904423in}}%
\pgfpathlineto{\pgfqpoint{3.597431in}{1.907806in}}%
\pgfpathclose%
\pgfusepath{fill}%
\end{pgfscope}%
\begin{pgfscope}%
\pgfpathrectangle{\pgfqpoint{1.254980in}{0.150000in}}{\pgfqpoint{5.490039in}{5.490039in}}%
\pgfusepath{clip}%
\pgfsetbuttcap%
\pgfsetroundjoin%
\definecolor{currentfill}{rgb}{0.269944,0.014625,0.341379}%
\pgfsetfillcolor{currentfill}%
\pgfsetfillopacity{0.700000}%
\pgfsetlinewidth{0.000000pt}%
\definecolor{currentstroke}{rgb}{0.000000,0.000000,0.000000}%
\pgfsetstrokecolor{currentstroke}%
\pgfsetdash{}{0pt}%
\pgfpathmoveto{\pgfqpoint{4.364655in}{1.878338in}}%
\pgfpathlineto{\pgfqpoint{4.377992in}{1.875363in}}%
\pgfpathlineto{\pgfqpoint{4.391335in}{1.872412in}}%
\pgfpathlineto{\pgfqpoint{4.404685in}{1.869485in}}%
\pgfpathlineto{\pgfqpoint{4.418042in}{1.866584in}}%
\pgfpathlineto{\pgfqpoint{4.410498in}{1.858048in}}%
\pgfpathlineto{\pgfqpoint{4.402949in}{1.849552in}}%
\pgfpathlineto{\pgfqpoint{4.395395in}{1.841099in}}%
\pgfpathlineto{\pgfqpoint{4.387836in}{1.832694in}}%
\pgfpathlineto{\pgfqpoint{4.374468in}{1.835766in}}%
\pgfpathlineto{\pgfqpoint{4.361107in}{1.838862in}}%
\pgfpathlineto{\pgfqpoint{4.347753in}{1.841982in}}%
\pgfpathlineto{\pgfqpoint{4.334405in}{1.845128in}}%
\pgfpathlineto{\pgfqpoint{4.341975in}{1.853358in}}%
\pgfpathlineto{\pgfqpoint{4.349541in}{1.861639in}}%
\pgfpathlineto{\pgfqpoint{4.357101in}{1.869967in}}%
\pgfpathlineto{\pgfqpoint{4.364655in}{1.878338in}}%
\pgfpathclose%
\pgfusepath{fill}%
\end{pgfscope}%
\begin{pgfscope}%
\pgfpathrectangle{\pgfqpoint{1.254980in}{0.150000in}}{\pgfqpoint{5.490039in}{5.490039in}}%
\pgfusepath{clip}%
\pgfsetbuttcap%
\pgfsetroundjoin%
\definecolor{currentfill}{rgb}{0.282656,0.100196,0.422160}%
\pgfsetfillcolor{currentfill}%
\pgfsetfillopacity{0.700000}%
\pgfsetlinewidth{0.000000pt}%
\definecolor{currentstroke}{rgb}{0.000000,0.000000,0.000000}%
\pgfsetstrokecolor{currentstroke}%
\pgfsetdash{}{0pt}%
\pgfpathmoveto{\pgfqpoint{4.972792in}{2.023967in}}%
\pgfpathlineto{\pgfqpoint{4.986304in}{2.022516in}}%
\pgfpathlineto{\pgfqpoint{4.999825in}{2.021090in}}%
\pgfpathlineto{\pgfqpoint{5.013353in}{2.019687in}}%
\pgfpathlineto{\pgfqpoint{5.026889in}{2.018308in}}%
\pgfpathlineto{\pgfqpoint{5.019549in}{2.009102in}}%
\pgfpathlineto{\pgfqpoint{5.012203in}{1.999839in}}%
\pgfpathlineto{\pgfqpoint{5.004851in}{1.990521in}}%
\pgfpathlineto{\pgfqpoint{4.997493in}{1.981151in}}%
\pgfpathlineto{\pgfqpoint{4.983948in}{1.982622in}}%
\pgfpathlineto{\pgfqpoint{4.970411in}{1.984118in}}%
\pgfpathlineto{\pgfqpoint{4.956881in}{1.985637in}}%
\pgfpathlineto{\pgfqpoint{4.943360in}{1.987181in}}%
\pgfpathlineto{\pgfqpoint{4.950726in}{1.996454in}}%
\pgfpathlineto{\pgfqpoint{4.958087in}{2.005677in}}%
\pgfpathlineto{\pgfqpoint{4.965443in}{2.014849in}}%
\pgfpathlineto{\pgfqpoint{4.972792in}{2.023967in}}%
\pgfpathclose%
\pgfusepath{fill}%
\end{pgfscope}%
\begin{pgfscope}%
\pgfpathrectangle{\pgfqpoint{1.254980in}{0.150000in}}{\pgfqpoint{5.490039in}{5.490039in}}%
\pgfusepath{clip}%
\pgfsetbuttcap%
\pgfsetroundjoin%
\definecolor{currentfill}{rgb}{0.280255,0.165693,0.476498}%
\pgfsetfillcolor{currentfill}%
\pgfsetfillopacity{0.700000}%
\pgfsetlinewidth{0.000000pt}%
\definecolor{currentstroke}{rgb}{0.000000,0.000000,0.000000}%
\pgfsetstrokecolor{currentstroke}%
\pgfsetdash{}{0pt}%
\pgfpathmoveto{\pgfqpoint{5.360526in}{2.141650in}}%
\pgfpathlineto{\pgfqpoint{5.374163in}{2.140916in}}%
\pgfpathlineto{\pgfqpoint{5.387808in}{2.140206in}}%
\pgfpathlineto{\pgfqpoint{5.401461in}{2.139520in}}%
\pgfpathlineto{\pgfqpoint{5.415123in}{2.138857in}}%
\pgfpathlineto{\pgfqpoint{5.407940in}{2.130576in}}%
\pgfpathlineto{\pgfqpoint{5.400749in}{2.122203in}}%
\pgfpathlineto{\pgfqpoint{5.393550in}{2.113738in}}%
\pgfpathlineto{\pgfqpoint{5.386345in}{2.105181in}}%
\pgfpathlineto{\pgfqpoint{5.372672in}{2.105884in}}%
\pgfpathlineto{\pgfqpoint{5.359008in}{2.106610in}}%
\pgfpathlineto{\pgfqpoint{5.345352in}{2.107360in}}%
\pgfpathlineto{\pgfqpoint{5.331705in}{2.108134in}}%
\pgfpathlineto{\pgfqpoint{5.338921in}{2.116646in}}%
\pgfpathlineto{\pgfqpoint{5.346130in}{2.125069in}}%
\pgfpathlineto{\pgfqpoint{5.353332in}{2.133404in}}%
\pgfpathlineto{\pgfqpoint{5.360526in}{2.141650in}}%
\pgfpathclose%
\pgfusepath{fill}%
\end{pgfscope}%
\begin{pgfscope}%
\pgfpathrectangle{\pgfqpoint{1.254980in}{0.150000in}}{\pgfqpoint{5.490039in}{5.490039in}}%
\pgfusepath{clip}%
\pgfsetbuttcap%
\pgfsetroundjoin%
\definecolor{currentfill}{rgb}{0.267004,0.004874,0.329415}%
\pgfsetfillcolor{currentfill}%
\pgfsetfillopacity{0.700000}%
\pgfsetlinewidth{0.000000pt}%
\definecolor{currentstroke}{rgb}{0.000000,0.000000,0.000000}%
\pgfsetstrokecolor{currentstroke}%
\pgfsetdash{}{0pt}%
\pgfpathmoveto{\pgfqpoint{4.144262in}{1.854674in}}%
\pgfpathlineto{\pgfqpoint{4.157544in}{1.851047in}}%
\pgfpathlineto{\pgfqpoint{4.170833in}{1.847445in}}%
\pgfpathlineto{\pgfqpoint{4.184129in}{1.843868in}}%
\pgfpathlineto{\pgfqpoint{4.197430in}{1.840317in}}%
\pgfpathlineto{\pgfqpoint{4.189809in}{1.832788in}}%
\pgfpathlineto{\pgfqpoint{4.182183in}{1.825343in}}%
\pgfpathlineto{\pgfqpoint{4.174550in}{1.817988in}}%
\pgfpathlineto{\pgfqpoint{4.166912in}{1.810726in}}%
\pgfpathlineto{\pgfqpoint{4.153598in}{1.814472in}}%
\pgfpathlineto{\pgfqpoint{4.140290in}{1.818244in}}%
\pgfpathlineto{\pgfqpoint{4.126988in}{1.822040in}}%
\pgfpathlineto{\pgfqpoint{4.113692in}{1.825862in}}%
\pgfpathlineto{\pgfqpoint{4.121343in}{1.832924in}}%
\pgfpathlineto{\pgfqpoint{4.128988in}{1.840083in}}%
\pgfpathlineto{\pgfqpoint{4.136628in}{1.847335in}}%
\pgfpathlineto{\pgfqpoint{4.144262in}{1.854674in}}%
\pgfpathclose%
\pgfusepath{fill}%
\end{pgfscope}%
\begin{pgfscope}%
\pgfpathrectangle{\pgfqpoint{1.254980in}{0.150000in}}{\pgfqpoint{5.490039in}{5.490039in}}%
\pgfusepath{clip}%
\pgfsetbuttcap%
\pgfsetroundjoin%
\definecolor{currentfill}{rgb}{0.276022,0.044167,0.370164}%
\pgfsetfillcolor{currentfill}%
\pgfsetfillopacity{0.700000}%
\pgfsetlinewidth{0.000000pt}%
\definecolor{currentstroke}{rgb}{0.000000,0.000000,0.000000}%
\pgfsetstrokecolor{currentstroke}%
\pgfsetdash{}{0pt}%
\pgfpathmoveto{\pgfqpoint{4.585149in}{1.915917in}}%
\pgfpathlineto{\pgfqpoint{4.598548in}{1.913536in}}%
\pgfpathlineto{\pgfqpoint{4.611954in}{1.911180in}}%
\pgfpathlineto{\pgfqpoint{4.625368in}{1.908847in}}%
\pgfpathlineto{\pgfqpoint{4.638789in}{1.906539in}}%
\pgfpathlineto{\pgfqpoint{4.631315in}{1.897403in}}%
\pgfpathlineto{\pgfqpoint{4.623837in}{1.888268in}}%
\pgfpathlineto{\pgfqpoint{4.616353in}{1.879137in}}%
\pgfpathlineto{\pgfqpoint{4.608865in}{1.870013in}}%
\pgfpathlineto{\pgfqpoint{4.595435in}{1.872465in}}%
\pgfpathlineto{\pgfqpoint{4.582011in}{1.874942in}}%
\pgfpathlineto{\pgfqpoint{4.568595in}{1.877443in}}%
\pgfpathlineto{\pgfqpoint{4.555186in}{1.879968in}}%
\pgfpathlineto{\pgfqpoint{4.562685in}{1.888942in}}%
\pgfpathlineto{\pgfqpoint{4.570178in}{1.897928in}}%
\pgfpathlineto{\pgfqpoint{4.577666in}{1.906920in}}%
\pgfpathlineto{\pgfqpoint{4.585149in}{1.915917in}}%
\pgfpathclose%
\pgfusepath{fill}%
\end{pgfscope}%
\begin{pgfscope}%
\pgfpathrectangle{\pgfqpoint{1.254980in}{0.150000in}}{\pgfqpoint{5.490039in}{5.490039in}}%
\pgfusepath{clip}%
\pgfsetbuttcap%
\pgfsetroundjoin%
\definecolor{currentfill}{rgb}{0.282656,0.100196,0.422160}%
\pgfsetfillcolor{currentfill}%
\pgfsetfillopacity{0.700000}%
\pgfsetlinewidth{0.000000pt}%
\definecolor{currentstroke}{rgb}{0.000000,0.000000,0.000000}%
\pgfsetstrokecolor{currentstroke}%
\pgfsetdash{}{0pt}%
\pgfpathmoveto{\pgfqpoint{3.270861in}{2.009593in}}%
\pgfpathlineto{\pgfqpoint{3.283984in}{2.003164in}}%
\pgfpathlineto{\pgfqpoint{3.297111in}{1.996767in}}%
\pgfpathlineto{\pgfqpoint{3.310242in}{1.990400in}}%
\pgfpathlineto{\pgfqpoint{3.323377in}{1.984065in}}%
\pgfpathlineto{\pgfqpoint{3.315323in}{1.983548in}}%
\pgfpathlineto{\pgfqpoint{3.307257in}{1.983299in}}%
\pgfpathlineto{\pgfqpoint{3.299178in}{1.983326in}}%
\pgfpathlineto{\pgfqpoint{3.291086in}{1.983635in}}%
\pgfpathlineto{\pgfqpoint{3.277924in}{1.990245in}}%
\pgfpathlineto{\pgfqpoint{3.264767in}{1.996886in}}%
\pgfpathlineto{\pgfqpoint{3.251614in}{2.003557in}}%
\pgfpathlineto{\pgfqpoint{3.238465in}{2.010259in}}%
\pgfpathlineto{\pgfqpoint{3.246584in}{2.009671in}}%
\pgfpathlineto{\pgfqpoint{3.254689in}{2.009368in}}%
\pgfpathlineto{\pgfqpoint{3.262782in}{2.009345in}}%
\pgfpathlineto{\pgfqpoint{3.270861in}{2.009593in}}%
\pgfpathclose%
\pgfusepath{fill}%
\end{pgfscope}%
\begin{pgfscope}%
\pgfpathrectangle{\pgfqpoint{1.254980in}{0.150000in}}{\pgfqpoint{5.490039in}{5.490039in}}%
\pgfusepath{clip}%
\pgfsetbuttcap%
\pgfsetroundjoin%
\definecolor{currentfill}{rgb}{0.237441,0.305202,0.541921}%
\pgfsetfillcolor{currentfill}%
\pgfsetfillopacity{0.700000}%
\pgfsetlinewidth{0.000000pt}%
\definecolor{currentstroke}{rgb}{0.000000,0.000000,0.000000}%
\pgfsetstrokecolor{currentstroke}%
\pgfsetdash{}{0pt}%
\pgfpathmoveto{\pgfqpoint{2.542662in}{2.417590in}}%
\pgfpathlineto{\pgfqpoint{2.555736in}{2.408498in}}%
\pgfpathlineto{\pgfqpoint{2.568812in}{2.399451in}}%
\pgfpathlineto{\pgfqpoint{2.581890in}{2.390449in}}%
\pgfpathlineto{\pgfqpoint{2.594970in}{2.381491in}}%
\pgfpathlineto{\pgfqpoint{2.586330in}{2.388448in}}%
\pgfpathlineto{\pgfqpoint{2.577667in}{2.395812in}}%
\pgfpathlineto{\pgfqpoint{2.568980in}{2.403591in}}%
\pgfpathlineto{\pgfqpoint{2.560270in}{2.411795in}}%
\pgfpathlineto{\pgfqpoint{2.547150in}{2.421074in}}%
\pgfpathlineto{\pgfqpoint{2.534032in}{2.430398in}}%
\pgfpathlineto{\pgfqpoint{2.520916in}{2.439767in}}%
\pgfpathlineto{\pgfqpoint{2.507802in}{2.449182in}}%
\pgfpathlineto{\pgfqpoint{2.516554in}{2.440651in}}%
\pgfpathlineto{\pgfqpoint{2.525281in}{2.432548in}}%
\pgfpathlineto{\pgfqpoint{2.533983in}{2.424864in}}%
\pgfpathlineto{\pgfqpoint{2.542662in}{2.417590in}}%
\pgfpathclose%
\pgfusepath{fill}%
\end{pgfscope}%
\begin{pgfscope}%
\pgfpathrectangle{\pgfqpoint{1.254980in}{0.150000in}}{\pgfqpoint{5.490039in}{5.490039in}}%
\pgfusepath{clip}%
\pgfsetbuttcap%
\pgfsetroundjoin%
\definecolor{currentfill}{rgb}{0.270595,0.214069,0.507052}%
\pgfsetfillcolor{currentfill}%
\pgfsetfillopacity{0.700000}%
\pgfsetlinewidth{0.000000pt}%
\definecolor{currentstroke}{rgb}{0.000000,0.000000,0.000000}%
\pgfsetstrokecolor{currentstroke}%
\pgfsetdash{}{0pt}%
\pgfpathmoveto{\pgfqpoint{2.838299in}{2.224873in}}%
\pgfpathlineto{\pgfqpoint{2.851384in}{2.216920in}}%
\pgfpathlineto{\pgfqpoint{2.864470in}{2.209004in}}%
\pgfpathlineto{\pgfqpoint{2.877560in}{2.201127in}}%
\pgfpathlineto{\pgfqpoint{2.890653in}{2.193287in}}%
\pgfpathlineto{\pgfqpoint{2.882275in}{2.197205in}}%
\pgfpathlineto{\pgfqpoint{2.873878in}{2.201477in}}%
\pgfpathlineto{\pgfqpoint{2.865463in}{2.206112in}}%
\pgfpathlineto{\pgfqpoint{2.857028in}{2.211117in}}%
\pgfpathlineto{\pgfqpoint{2.843901in}{2.219262in}}%
\pgfpathlineto{\pgfqpoint{2.830777in}{2.227443in}}%
\pgfpathlineto{\pgfqpoint{2.817656in}{2.235663in}}%
\pgfpathlineto{\pgfqpoint{2.804537in}{2.243921in}}%
\pgfpathlineto{\pgfqpoint{2.813007in}{2.238606in}}%
\pgfpathlineto{\pgfqpoint{2.821457in}{2.233665in}}%
\pgfpathlineto{\pgfqpoint{2.829888in}{2.229090in}}%
\pgfpathlineto{\pgfqpoint{2.838299in}{2.224873in}}%
\pgfpathclose%
\pgfusepath{fill}%
\end{pgfscope}%
\begin{pgfscope}%
\pgfpathrectangle{\pgfqpoint{1.254980in}{0.150000in}}{\pgfqpoint{5.490039in}{5.490039in}}%
\pgfusepath{clip}%
\pgfsetbuttcap%
\pgfsetroundjoin%
\definecolor{currentfill}{rgb}{0.281924,0.089666,0.412415}%
\pgfsetfillcolor{currentfill}%
\pgfsetfillopacity{0.700000}%
\pgfsetlinewidth{0.000000pt}%
\definecolor{currentstroke}{rgb}{0.000000,0.000000,0.000000}%
\pgfsetstrokecolor{currentstroke}%
\pgfsetdash{}{0pt}%
\pgfpathmoveto{\pgfqpoint{4.889351in}{1.993594in}}%
\pgfpathlineto{\pgfqpoint{4.902842in}{1.991955in}}%
\pgfpathlineto{\pgfqpoint{4.916340in}{1.990339in}}%
\pgfpathlineto{\pgfqpoint{4.929846in}{1.988748in}}%
\pgfpathlineto{\pgfqpoint{4.943360in}{1.987181in}}%
\pgfpathlineto{\pgfqpoint{4.935987in}{1.977860in}}%
\pgfpathlineto{\pgfqpoint{4.928609in}{1.968494in}}%
\pgfpathlineto{\pgfqpoint{4.921225in}{1.959085in}}%
\pgfpathlineto{\pgfqpoint{4.913835in}{1.949635in}}%
\pgfpathlineto{\pgfqpoint{4.900313in}{1.951308in}}%
\pgfpathlineto{\pgfqpoint{4.886798in}{1.953005in}}%
\pgfpathlineto{\pgfqpoint{4.873291in}{1.954726in}}%
\pgfpathlineto{\pgfqpoint{4.859791in}{1.956471in}}%
\pgfpathlineto{\pgfqpoint{4.867190in}{1.965810in}}%
\pgfpathlineto{\pgfqpoint{4.874582in}{1.975112in}}%
\pgfpathlineto{\pgfqpoint{4.881970in}{1.984374in}}%
\pgfpathlineto{\pgfqpoint{4.889351in}{1.993594in}}%
\pgfpathclose%
\pgfusepath{fill}%
\end{pgfscope}%
\begin{pgfscope}%
\pgfpathrectangle{\pgfqpoint{1.254980in}{0.150000in}}{\pgfqpoint{5.490039in}{5.490039in}}%
\pgfusepath{clip}%
\pgfsetbuttcap%
\pgfsetroundjoin%
\definecolor{currentfill}{rgb}{0.271828,0.209303,0.504434}%
\pgfsetfillcolor{currentfill}%
\pgfsetfillopacity{0.700000}%
\pgfsetlinewidth{0.000000pt}%
\definecolor{currentstroke}{rgb}{0.000000,0.000000,0.000000}%
\pgfsetstrokecolor{currentstroke}%
\pgfsetdash{}{0pt}%
\pgfpathmoveto{\pgfqpoint{5.664922in}{2.225359in}}%
\pgfpathlineto{\pgfqpoint{5.678662in}{2.225046in}}%
\pgfpathlineto{\pgfqpoint{5.692411in}{2.224756in}}%
\pgfpathlineto{\pgfqpoint{5.706168in}{2.224490in}}%
\pgfpathlineto{\pgfqpoint{5.719935in}{2.224248in}}%
\pgfpathlineto{\pgfqpoint{5.712895in}{2.217101in}}%
\pgfpathlineto{\pgfqpoint{5.705847in}{2.209849in}}%
\pgfpathlineto{\pgfqpoint{5.698791in}{2.202492in}}%
\pgfpathlineto{\pgfqpoint{5.691726in}{2.195029in}}%
\pgfpathlineto{\pgfqpoint{5.677947in}{2.195271in}}%
\pgfpathlineto{\pgfqpoint{5.664176in}{2.195537in}}%
\pgfpathlineto{\pgfqpoint{5.650415in}{2.195827in}}%
\pgfpathlineto{\pgfqpoint{5.636662in}{2.196140in}}%
\pgfpathlineto{\pgfqpoint{5.643740in}{2.203598in}}%
\pgfpathlineto{\pgfqpoint{5.650809in}{2.210953in}}%
\pgfpathlineto{\pgfqpoint{5.657870in}{2.218207in}}%
\pgfpathlineto{\pgfqpoint{5.664922in}{2.225359in}}%
\pgfpathclose%
\pgfusepath{fill}%
\end{pgfscope}%
\begin{pgfscope}%
\pgfpathrectangle{\pgfqpoint{1.254980in}{0.150000in}}{\pgfqpoint{5.490039in}{5.490039in}}%
\pgfusepath{clip}%
\pgfsetbuttcap%
\pgfsetroundjoin%
\definecolor{currentfill}{rgb}{0.278791,0.062145,0.386592}%
\pgfsetfillcolor{currentfill}%
\pgfsetfillopacity{0.700000}%
\pgfsetlinewidth{0.000000pt}%
\definecolor{currentstroke}{rgb}{0.000000,0.000000,0.000000}%
\pgfsetstrokecolor{currentstroke}%
\pgfsetdash{}{0pt}%
\pgfpathmoveto{\pgfqpoint{3.460521in}{1.941154in}}%
\pgfpathlineto{\pgfqpoint{3.473672in}{1.935347in}}%
\pgfpathlineto{\pgfqpoint{3.486828in}{1.929570in}}%
\pgfpathlineto{\pgfqpoint{3.499989in}{1.923821in}}%
\pgfpathlineto{\pgfqpoint{3.513154in}{1.918101in}}%
\pgfpathlineto{\pgfqpoint{3.505215in}{1.915816in}}%
\pgfpathlineto{\pgfqpoint{3.497267in}{1.913762in}}%
\pgfpathlineto{\pgfqpoint{3.489307in}{1.911944in}}%
\pgfpathlineto{\pgfqpoint{3.481338in}{1.910371in}}%
\pgfpathlineto{\pgfqpoint{3.468150in}{1.916351in}}%
\pgfpathlineto{\pgfqpoint{3.454967in}{1.922360in}}%
\pgfpathlineto{\pgfqpoint{3.441788in}{1.928397in}}%
\pgfpathlineto{\pgfqpoint{3.428614in}{1.934464in}}%
\pgfpathlineto{\pgfqpoint{3.436607in}{1.935772in}}%
\pgfpathlineto{\pgfqpoint{3.444589in}{1.937328in}}%
\pgfpathlineto{\pgfqpoint{3.452561in}{1.939124in}}%
\pgfpathlineto{\pgfqpoint{3.460521in}{1.941154in}}%
\pgfpathclose%
\pgfusepath{fill}%
\end{pgfscope}%
\begin{pgfscope}%
\pgfpathrectangle{\pgfqpoint{1.254980in}{0.150000in}}{\pgfqpoint{5.490039in}{5.490039in}}%
\pgfusepath{clip}%
\pgfsetbuttcap%
\pgfsetroundjoin%
\definecolor{currentfill}{rgb}{0.281887,0.150881,0.465405}%
\pgfsetfillcolor{currentfill}%
\pgfsetfillopacity{0.700000}%
\pgfsetlinewidth{0.000000pt}%
\definecolor{currentstroke}{rgb}{0.000000,0.000000,0.000000}%
\pgfsetstrokecolor{currentstroke}%
\pgfsetdash{}{0pt}%
\pgfpathmoveto{\pgfqpoint{5.277201in}{2.111468in}}%
\pgfpathlineto{\pgfqpoint{5.290815in}{2.110599in}}%
\pgfpathlineto{\pgfqpoint{5.304436in}{2.109754in}}%
\pgfpathlineto{\pgfqpoint{5.318067in}{2.108932in}}%
\pgfpathlineto{\pgfqpoint{5.331705in}{2.108134in}}%
\pgfpathlineto{\pgfqpoint{5.324482in}{2.099536in}}%
\pgfpathlineto{\pgfqpoint{5.317252in}{2.090851in}}%
\pgfpathlineto{\pgfqpoint{5.310014in}{2.082081in}}%
\pgfpathlineto{\pgfqpoint{5.302770in}{2.073226in}}%
\pgfpathlineto{\pgfqpoint{5.289121in}{2.074077in}}%
\pgfpathlineto{\pgfqpoint{5.275481in}{2.074952in}}%
\pgfpathlineto{\pgfqpoint{5.261849in}{2.075851in}}%
\pgfpathlineto{\pgfqpoint{5.248226in}{2.076774in}}%
\pgfpathlineto{\pgfqpoint{5.255480in}{2.085570in}}%
\pgfpathlineto{\pgfqpoint{5.262728in}{2.094285in}}%
\pgfpathlineto{\pgfqpoint{5.269968in}{2.102918in}}%
\pgfpathlineto{\pgfqpoint{5.277201in}{2.111468in}}%
\pgfpathclose%
\pgfusepath{fill}%
\end{pgfscope}%
\begin{pgfscope}%
\pgfpathrectangle{\pgfqpoint{1.254980in}{0.150000in}}{\pgfqpoint{5.490039in}{5.490039in}}%
\pgfusepath{clip}%
\pgfsetbuttcap%
\pgfsetroundjoin%
\definecolor{currentfill}{rgb}{0.282290,0.145912,0.461510}%
\pgfsetfillcolor{currentfill}%
\pgfsetfillopacity{0.700000}%
\pgfsetlinewidth{0.000000pt}%
\definecolor{currentstroke}{rgb}{0.000000,0.000000,0.000000}%
\pgfsetstrokecolor{currentstroke}%
\pgfsetdash{}{0pt}%
\pgfpathmoveto{\pgfqpoint{3.080972in}{2.093188in}}%
\pgfpathlineto{\pgfqpoint{3.094076in}{2.086096in}}%
\pgfpathlineto{\pgfqpoint{3.107184in}{2.079038in}}%
\pgfpathlineto{\pgfqpoint{3.120295in}{2.072014in}}%
\pgfpathlineto{\pgfqpoint{3.133410in}{2.065022in}}%
\pgfpathlineto{\pgfqpoint{3.125221in}{2.066470in}}%
\pgfpathlineto{\pgfqpoint{3.117017in}{2.068226in}}%
\pgfpathlineto{\pgfqpoint{3.108798in}{2.070298in}}%
\pgfpathlineto{\pgfqpoint{3.100564in}{2.072694in}}%
\pgfpathlineto{\pgfqpoint{3.087419in}{2.079973in}}%
\pgfpathlineto{\pgfqpoint{3.074278in}{2.087286in}}%
\pgfpathlineto{\pgfqpoint{3.061141in}{2.094633in}}%
\pgfpathlineto{\pgfqpoint{3.048007in}{2.102013in}}%
\pgfpathlineto{\pgfqpoint{3.056272in}{2.099323in}}%
\pgfpathlineto{\pgfqpoint{3.064521in}{2.096961in}}%
\pgfpathlineto{\pgfqpoint{3.072754in}{2.094919in}}%
\pgfpathlineto{\pgfqpoint{3.080972in}{2.093188in}}%
\pgfpathclose%
\pgfusepath{fill}%
\end{pgfscope}%
\begin{pgfscope}%
\pgfpathrectangle{\pgfqpoint{1.254980in}{0.150000in}}{\pgfqpoint{5.490039in}{5.490039in}}%
\pgfusepath{clip}%
\pgfsetbuttcap%
\pgfsetroundjoin%
\definecolor{currentfill}{rgb}{0.268510,0.009605,0.335427}%
\pgfsetfillcolor{currentfill}%
\pgfsetfillopacity{0.700000}%
\pgfsetlinewidth{0.000000pt}%
\definecolor{currentstroke}{rgb}{0.000000,0.000000,0.000000}%
\pgfsetstrokecolor{currentstroke}%
\pgfsetdash{}{0pt}%
\pgfpathmoveto{\pgfqpoint{4.281080in}{1.857956in}}%
\pgfpathlineto{\pgfqpoint{4.294402in}{1.854712in}}%
\pgfpathlineto{\pgfqpoint{4.307729in}{1.851492in}}%
\pgfpathlineto{\pgfqpoint{4.321064in}{1.848298in}}%
\pgfpathlineto{\pgfqpoint{4.334405in}{1.845128in}}%
\pgfpathlineto{\pgfqpoint{4.326830in}{1.836953in}}%
\pgfpathlineto{\pgfqpoint{4.319249in}{1.828838in}}%
\pgfpathlineto{\pgfqpoint{4.311663in}{1.820786in}}%
\pgfpathlineto{\pgfqpoint{4.304071in}{1.812803in}}%
\pgfpathlineto{\pgfqpoint{4.290719in}{1.816156in}}%
\pgfpathlineto{\pgfqpoint{4.277372in}{1.819533in}}%
\pgfpathlineto{\pgfqpoint{4.264033in}{1.822935in}}%
\pgfpathlineto{\pgfqpoint{4.250699in}{1.826362in}}%
\pgfpathlineto{\pgfqpoint{4.258303in}{1.834157in}}%
\pgfpathlineto{\pgfqpoint{4.265900in}{1.842024in}}%
\pgfpathlineto{\pgfqpoint{4.273493in}{1.849959in}}%
\pgfpathlineto{\pgfqpoint{4.281080in}{1.857956in}}%
\pgfpathclose%
\pgfusepath{fill}%
\end{pgfscope}%
\begin{pgfscope}%
\pgfpathrectangle{\pgfqpoint{1.254980in}{0.150000in}}{\pgfqpoint{5.490039in}{5.490039in}}%
\pgfusepath{clip}%
\pgfsetbuttcap%
\pgfsetroundjoin%
\definecolor{currentfill}{rgb}{0.273809,0.031497,0.358853}%
\pgfsetfillcolor{currentfill}%
\pgfsetfillopacity{0.700000}%
\pgfsetlinewidth{0.000000pt}%
\definecolor{currentstroke}{rgb}{0.000000,0.000000,0.000000}%
\pgfsetstrokecolor{currentstroke}%
\pgfsetdash{}{0pt}%
\pgfpathmoveto{\pgfqpoint{4.501621in}{1.890312in}}%
\pgfpathlineto{\pgfqpoint{4.515002in}{1.887689in}}%
\pgfpathlineto{\pgfqpoint{4.528390in}{1.885091in}}%
\pgfpathlineto{\pgfqpoint{4.541784in}{1.882517in}}%
\pgfpathlineto{\pgfqpoint{4.555186in}{1.879968in}}%
\pgfpathlineto{\pgfqpoint{4.547683in}{1.871008in}}%
\pgfpathlineto{\pgfqpoint{4.540175in}{1.862066in}}%
\pgfpathlineto{\pgfqpoint{4.532662in}{1.853145in}}%
\pgfpathlineto{\pgfqpoint{4.525143in}{1.844250in}}%
\pgfpathlineto{\pgfqpoint{4.511731in}{1.846956in}}%
\pgfpathlineto{\pgfqpoint{4.498327in}{1.849687in}}%
\pgfpathlineto{\pgfqpoint{4.484929in}{1.852442in}}%
\pgfpathlineto{\pgfqpoint{4.471538in}{1.855222in}}%
\pgfpathlineto{\pgfqpoint{4.479066in}{1.863955in}}%
\pgfpathlineto{\pgfqpoint{4.486589in}{1.872717in}}%
\pgfpathlineto{\pgfqpoint{4.494108in}{1.881504in}}%
\pgfpathlineto{\pgfqpoint{4.501621in}{1.890312in}}%
\pgfpathclose%
\pgfusepath{fill}%
\end{pgfscope}%
\begin{pgfscope}%
\pgfpathrectangle{\pgfqpoint{1.254980in}{0.150000in}}{\pgfqpoint{5.490039in}{5.490039in}}%
\pgfusepath{clip}%
\pgfsetbuttcap%
\pgfsetroundjoin%
\definecolor{currentfill}{rgb}{0.280267,0.073417,0.397163}%
\pgfsetfillcolor{currentfill}%
\pgfsetfillopacity{0.700000}%
\pgfsetlinewidth{0.000000pt}%
\definecolor{currentstroke}{rgb}{0.000000,0.000000,0.000000}%
\pgfsetstrokecolor{currentstroke}%
\pgfsetdash{}{0pt}%
\pgfpathmoveto{\pgfqpoint{4.805870in}{1.963691in}}%
\pgfpathlineto{\pgfqpoint{4.819339in}{1.961850in}}%
\pgfpathlineto{\pgfqpoint{4.832815in}{1.960033in}}%
\pgfpathlineto{\pgfqpoint{4.846299in}{1.958240in}}%
\pgfpathlineto{\pgfqpoint{4.859791in}{1.956471in}}%
\pgfpathlineto{\pgfqpoint{4.852387in}{1.947097in}}%
\pgfpathlineto{\pgfqpoint{4.844978in}{1.937690in}}%
\pgfpathlineto{\pgfqpoint{4.837563in}{1.928253in}}%
\pgfpathlineto{\pgfqpoint{4.830143in}{1.918790in}}%
\pgfpathlineto{\pgfqpoint{4.816643in}{1.920677in}}%
\pgfpathlineto{\pgfqpoint{4.803150in}{1.922589in}}%
\pgfpathlineto{\pgfqpoint{4.789664in}{1.924525in}}%
\pgfpathlineto{\pgfqpoint{4.776186in}{1.926484in}}%
\pgfpathlineto{\pgfqpoint{4.783615in}{1.935824in}}%
\pgfpathlineto{\pgfqpoint{4.791039in}{1.945141in}}%
\pgfpathlineto{\pgfqpoint{4.798457in}{1.954430in}}%
\pgfpathlineto{\pgfqpoint{4.805870in}{1.963691in}}%
\pgfpathclose%
\pgfusepath{fill}%
\end{pgfscope}%
\begin{pgfscope}%
\pgfpathrectangle{\pgfqpoint{1.254980in}{0.150000in}}{\pgfqpoint{5.490039in}{5.490039in}}%
\pgfusepath{clip}%
\pgfsetbuttcap%
\pgfsetroundjoin%
\definecolor{currentfill}{rgb}{0.269944,0.014625,0.341379}%
\pgfsetfillcolor{currentfill}%
\pgfsetfillopacity{0.700000}%
\pgfsetlinewidth{0.000000pt}%
\definecolor{currentstroke}{rgb}{0.000000,0.000000,0.000000}%
\pgfsetstrokecolor{currentstroke}%
\pgfsetdash{}{0pt}%
\pgfpathmoveto{\pgfqpoint{3.786982in}{1.863875in}}%
\pgfpathlineto{\pgfqpoint{3.800193in}{1.859106in}}%
\pgfpathlineto{\pgfqpoint{3.813410in}{1.854363in}}%
\pgfpathlineto{\pgfqpoint{3.826633in}{1.849648in}}%
\pgfpathlineto{\pgfqpoint{3.839860in}{1.844959in}}%
\pgfpathlineto{\pgfqpoint{3.832089in}{1.839919in}}%
\pgfpathlineto{\pgfqpoint{3.824311in}{1.835042in}}%
\pgfpathlineto{\pgfqpoint{3.816525in}{1.830336in}}%
\pgfpathlineto{\pgfqpoint{3.808731in}{1.825805in}}%
\pgfpathlineto{\pgfqpoint{3.795486in}{1.830728in}}%
\pgfpathlineto{\pgfqpoint{3.782246in}{1.835677in}}%
\pgfpathlineto{\pgfqpoint{3.769011in}{1.840652in}}%
\pgfpathlineto{\pgfqpoint{3.755782in}{1.845655in}}%
\pgfpathlineto{\pgfqpoint{3.763594in}{1.849947in}}%
\pgfpathlineto{\pgfqpoint{3.771398in}{1.854419in}}%
\pgfpathlineto{\pgfqpoint{3.779194in}{1.859063in}}%
\pgfpathlineto{\pgfqpoint{3.786982in}{1.863875in}}%
\pgfpathclose%
\pgfusepath{fill}%
\end{pgfscope}%
\begin{pgfscope}%
\pgfpathrectangle{\pgfqpoint{1.254980in}{0.150000in}}{\pgfqpoint{5.490039in}{5.490039in}}%
\pgfusepath{clip}%
\pgfsetbuttcap%
\pgfsetroundjoin%
\definecolor{currentfill}{rgb}{0.282884,0.135920,0.453427}%
\pgfsetfillcolor{currentfill}%
\pgfsetfillopacity{0.700000}%
\pgfsetlinewidth{0.000000pt}%
\definecolor{currentstroke}{rgb}{0.000000,0.000000,0.000000}%
\pgfsetstrokecolor{currentstroke}%
\pgfsetdash{}{0pt}%
\pgfpathmoveto{\pgfqpoint{5.193816in}{2.080702in}}%
\pgfpathlineto{\pgfqpoint{5.207406in}{2.079684in}}%
\pgfpathlineto{\pgfqpoint{5.221004in}{2.078690in}}%
\pgfpathlineto{\pgfqpoint{5.234611in}{2.077720in}}%
\pgfpathlineto{\pgfqpoint{5.248226in}{2.076774in}}%
\pgfpathlineto{\pgfqpoint{5.240965in}{2.067897in}}%
\pgfpathlineto{\pgfqpoint{5.233697in}{2.058942in}}%
\pgfpathlineto{\pgfqpoint{5.226423in}{2.049908in}}%
\pgfpathlineto{\pgfqpoint{5.219142in}{2.040798in}}%
\pgfpathlineto{\pgfqpoint{5.205517in}{2.041811in}}%
\pgfpathlineto{\pgfqpoint{5.191901in}{2.042848in}}%
\pgfpathlineto{\pgfqpoint{5.178293in}{2.043908in}}%
\pgfpathlineto{\pgfqpoint{5.164693in}{2.044993in}}%
\pgfpathlineto{\pgfqpoint{5.171984in}{2.054031in}}%
\pgfpathlineto{\pgfqpoint{5.179268in}{2.062997in}}%
\pgfpathlineto{\pgfqpoint{5.186545in}{2.071887in}}%
\pgfpathlineto{\pgfqpoint{5.193816in}{2.080702in}}%
\pgfpathclose%
\pgfusepath{fill}%
\end{pgfscope}%
\begin{pgfscope}%
\pgfpathrectangle{\pgfqpoint{1.254980in}{0.150000in}}{\pgfqpoint{5.490039in}{5.490039in}}%
\pgfusepath{clip}%
\pgfsetbuttcap%
\pgfsetroundjoin%
\definecolor{currentfill}{rgb}{0.268510,0.009605,0.335427}%
\pgfsetfillcolor{currentfill}%
\pgfsetfillopacity{0.700000}%
\pgfsetlinewidth{0.000000pt}%
\definecolor{currentstroke}{rgb}{0.000000,0.000000,0.000000}%
\pgfsetstrokecolor{currentstroke}%
\pgfsetdash{}{0pt}%
\pgfpathmoveto{\pgfqpoint{3.923772in}{1.849030in}}%
\pgfpathlineto{\pgfqpoint{3.937012in}{1.844693in}}%
\pgfpathlineto{\pgfqpoint{3.950257in}{1.840382in}}%
\pgfpathlineto{\pgfqpoint{3.963508in}{1.836097in}}%
\pgfpathlineto{\pgfqpoint{3.976764in}{1.831837in}}%
\pgfpathlineto{\pgfqpoint{3.969054in}{1.825768in}}%
\pgfpathlineto{\pgfqpoint{3.961336in}{1.819833in}}%
\pgfpathlineto{\pgfqpoint{3.953612in}{1.814039in}}%
\pgfpathlineto{\pgfqpoint{3.945882in}{1.808390in}}%
\pgfpathlineto{\pgfqpoint{3.932609in}{1.812870in}}%
\pgfpathlineto{\pgfqpoint{3.919343in}{1.817376in}}%
\pgfpathlineto{\pgfqpoint{3.906082in}{1.821908in}}%
\pgfpathlineto{\pgfqpoint{3.892826in}{1.826466in}}%
\pgfpathlineto{\pgfqpoint{3.900573in}{1.831889in}}%
\pgfpathlineto{\pgfqpoint{3.908313in}{1.837461in}}%
\pgfpathlineto{\pgfqpoint{3.916046in}{1.843177in}}%
\pgfpathlineto{\pgfqpoint{3.923772in}{1.849030in}}%
\pgfpathclose%
\pgfusepath{fill}%
\end{pgfscope}%
\begin{pgfscope}%
\pgfpathrectangle{\pgfqpoint{1.254980in}{0.150000in}}{\pgfqpoint{5.490039in}{5.490039in}}%
\pgfusepath{clip}%
\pgfsetbuttcap%
\pgfsetroundjoin%
\definecolor{currentfill}{rgb}{0.274128,0.199721,0.498911}%
\pgfsetfillcolor{currentfill}%
\pgfsetfillopacity{0.700000}%
\pgfsetlinewidth{0.000000pt}%
\definecolor{currentstroke}{rgb}{0.000000,0.000000,0.000000}%
\pgfsetstrokecolor{currentstroke}%
\pgfsetdash{}{0pt}%
\pgfpathmoveto{\pgfqpoint{5.581741in}{2.197631in}}%
\pgfpathlineto{\pgfqpoint{5.595458in}{2.197223in}}%
\pgfpathlineto{\pgfqpoint{5.609184in}{2.196838in}}%
\pgfpathlineto{\pgfqpoint{5.622919in}{2.196477in}}%
\pgfpathlineto{\pgfqpoint{5.636662in}{2.196140in}}%
\pgfpathlineto{\pgfqpoint{5.629577in}{2.188579in}}%
\pgfpathlineto{\pgfqpoint{5.622482in}{2.180915in}}%
\pgfpathlineto{\pgfqpoint{5.615380in}{2.173148in}}%
\pgfpathlineto{\pgfqpoint{5.608269in}{2.165278in}}%
\pgfpathlineto{\pgfqpoint{5.594514in}{2.165628in}}%
\pgfpathlineto{\pgfqpoint{5.580767in}{2.166002in}}%
\pgfpathlineto{\pgfqpoint{5.567029in}{2.166400in}}%
\pgfpathlineto{\pgfqpoint{5.553300in}{2.166822in}}%
\pgfpathlineto{\pgfqpoint{5.560422in}{2.174674in}}%
\pgfpathlineto{\pgfqpoint{5.567537in}{2.182426in}}%
\pgfpathlineto{\pgfqpoint{5.574643in}{2.190079in}}%
\pgfpathlineto{\pgfqpoint{5.581741in}{2.197631in}}%
\pgfpathclose%
\pgfusepath{fill}%
\end{pgfscope}%
\begin{pgfscope}%
\pgfpathrectangle{\pgfqpoint{1.254980in}{0.150000in}}{\pgfqpoint{5.490039in}{5.490039in}}%
\pgfusepath{clip}%
\pgfsetbuttcap%
\pgfsetroundjoin%
\definecolor{currentfill}{rgb}{0.243113,0.292092,0.538516}%
\pgfsetfillcolor{currentfill}%
\pgfsetfillopacity{0.700000}%
\pgfsetlinewidth{0.000000pt}%
\definecolor{currentstroke}{rgb}{0.000000,0.000000,0.000000}%
\pgfsetstrokecolor{currentstroke}%
\pgfsetdash{}{0pt}%
\pgfpathmoveto{\pgfqpoint{2.594970in}{2.381491in}}%
\pgfpathlineto{\pgfqpoint{2.608051in}{2.372578in}}%
\pgfpathlineto{\pgfqpoint{2.621135in}{2.363709in}}%
\pgfpathlineto{\pgfqpoint{2.634220in}{2.354883in}}%
\pgfpathlineto{\pgfqpoint{2.647308in}{2.346100in}}%
\pgfpathlineto{\pgfqpoint{2.638706in}{2.352741in}}%
\pgfpathlineto{\pgfqpoint{2.630083in}{2.359786in}}%
\pgfpathlineto{\pgfqpoint{2.621436in}{2.367241in}}%
\pgfpathlineto{\pgfqpoint{2.612766in}{2.375117in}}%
\pgfpathlineto{\pgfqpoint{2.599639in}{2.384222in}}%
\pgfpathlineto{\pgfqpoint{2.586514in}{2.393369in}}%
\pgfpathlineto{\pgfqpoint{2.573391in}{2.402560in}}%
\pgfpathlineto{\pgfqpoint{2.560270in}{2.411795in}}%
\pgfpathlineto{\pgfqpoint{2.568980in}{2.403591in}}%
\pgfpathlineto{\pgfqpoint{2.577667in}{2.395812in}}%
\pgfpathlineto{\pgfqpoint{2.586330in}{2.388448in}}%
\pgfpathlineto{\pgfqpoint{2.594970in}{2.381491in}}%
\pgfpathclose%
\pgfusepath{fill}%
\end{pgfscope}%
\begin{pgfscope}%
\pgfpathrectangle{\pgfqpoint{1.254980in}{0.150000in}}{\pgfqpoint{5.490039in}{5.490039in}}%
\pgfusepath{clip}%
\pgfsetbuttcap%
\pgfsetroundjoin%
\definecolor{currentfill}{rgb}{0.273809,0.031497,0.358853}%
\pgfsetfillcolor{currentfill}%
\pgfsetfillopacity{0.700000}%
\pgfsetlinewidth{0.000000pt}%
\definecolor{currentstroke}{rgb}{0.000000,0.000000,0.000000}%
\pgfsetstrokecolor{currentstroke}%
\pgfsetdash{}{0pt}%
\pgfpathmoveto{\pgfqpoint{3.650133in}{1.886648in}}%
\pgfpathlineto{\pgfqpoint{3.663322in}{1.881428in}}%
\pgfpathlineto{\pgfqpoint{3.676515in}{1.876236in}}%
\pgfpathlineto{\pgfqpoint{3.689713in}{1.871071in}}%
\pgfpathlineto{\pgfqpoint{3.702916in}{1.865934in}}%
\pgfpathlineto{\pgfqpoint{3.695077in}{1.862068in}}%
\pgfpathlineto{\pgfqpoint{3.687230in}{1.858397in}}%
\pgfpathlineto{\pgfqpoint{3.679373in}{1.854926in}}%
\pgfpathlineto{\pgfqpoint{3.671508in}{1.851663in}}%
\pgfpathlineto{\pgfqpoint{3.658285in}{1.857047in}}%
\pgfpathlineto{\pgfqpoint{3.645066in}{1.862458in}}%
\pgfpathlineto{\pgfqpoint{3.631853in}{1.867897in}}%
\pgfpathlineto{\pgfqpoint{3.618645in}{1.873364in}}%
\pgfpathlineto{\pgfqpoint{3.626531in}{1.876375in}}%
\pgfpathlineto{\pgfqpoint{3.634407in}{1.879598in}}%
\pgfpathlineto{\pgfqpoint{3.642275in}{1.883024in}}%
\pgfpathlineto{\pgfqpoint{3.650133in}{1.886648in}}%
\pgfpathclose%
\pgfusepath{fill}%
\end{pgfscope}%
\begin{pgfscope}%
\pgfpathrectangle{\pgfqpoint{1.254980in}{0.150000in}}{\pgfqpoint{5.490039in}{5.490039in}}%
\pgfusepath{clip}%
\pgfsetbuttcap%
\pgfsetroundjoin%
\definecolor{currentfill}{rgb}{0.267004,0.004874,0.329415}%
\pgfsetfillcolor{currentfill}%
\pgfsetfillopacity{0.700000}%
\pgfsetlinewidth{0.000000pt}%
\definecolor{currentstroke}{rgb}{0.000000,0.000000,0.000000}%
\pgfsetstrokecolor{currentstroke}%
\pgfsetdash{}{0pt}%
\pgfpathmoveto{\pgfqpoint{4.060569in}{1.841403in}}%
\pgfpathlineto{\pgfqpoint{4.073841in}{1.837480in}}%
\pgfpathlineto{\pgfqpoint{4.087119in}{1.833582in}}%
\pgfpathlineto{\pgfqpoint{4.100402in}{1.829709in}}%
\pgfpathlineto{\pgfqpoint{4.113692in}{1.825862in}}%
\pgfpathlineto{\pgfqpoint{4.106035in}{1.818902in}}%
\pgfpathlineto{\pgfqpoint{4.098372in}{1.812049in}}%
\pgfpathlineto{\pgfqpoint{4.090703in}{1.805308in}}%
\pgfpathlineto{\pgfqpoint{4.083028in}{1.798685in}}%
\pgfpathlineto{\pgfqpoint{4.069725in}{1.802740in}}%
\pgfpathlineto{\pgfqpoint{4.056427in}{1.806820in}}%
\pgfpathlineto{\pgfqpoint{4.043135in}{1.810926in}}%
\pgfpathlineto{\pgfqpoint{4.029849in}{1.815057in}}%
\pgfpathlineto{\pgfqpoint{4.037538in}{1.821468in}}%
\pgfpathlineto{\pgfqpoint{4.045222in}{1.827999in}}%
\pgfpathlineto{\pgfqpoint{4.052898in}{1.834646in}}%
\pgfpathlineto{\pgfqpoint{4.060569in}{1.841403in}}%
\pgfpathclose%
\pgfusepath{fill}%
\end{pgfscope}%
\begin{pgfscope}%
\pgfpathrectangle{\pgfqpoint{1.254980in}{0.150000in}}{\pgfqpoint{5.490039in}{5.490039in}}%
\pgfusepath{clip}%
\pgfsetbuttcap%
\pgfsetroundjoin%
\definecolor{currentfill}{rgb}{0.283187,0.125848,0.444960}%
\pgfsetfillcolor{currentfill}%
\pgfsetfillopacity{0.700000}%
\pgfsetlinewidth{0.000000pt}%
\definecolor{currentstroke}{rgb}{0.000000,0.000000,0.000000}%
\pgfsetstrokecolor{currentstroke}%
\pgfsetdash{}{0pt}%
\pgfpathmoveto{\pgfqpoint{5.110376in}{2.049569in}}%
\pgfpathlineto{\pgfqpoint{5.123943in}{2.048389in}}%
\pgfpathlineto{\pgfqpoint{5.137518in}{2.047233in}}%
\pgfpathlineto{\pgfqpoint{5.151102in}{2.046101in}}%
\pgfpathlineto{\pgfqpoint{5.164693in}{2.044993in}}%
\pgfpathlineto{\pgfqpoint{5.157396in}{2.035883in}}%
\pgfpathlineto{\pgfqpoint{5.150093in}{2.026702in}}%
\pgfpathlineto{\pgfqpoint{5.142784in}{2.017452in}}%
\pgfpathlineto{\pgfqpoint{5.135468in}{2.008136in}}%
\pgfpathlineto{\pgfqpoint{5.121867in}{2.009324in}}%
\pgfpathlineto{\pgfqpoint{5.108274in}{2.010536in}}%
\pgfpathlineto{\pgfqpoint{5.094690in}{2.011772in}}%
\pgfpathlineto{\pgfqpoint{5.081114in}{2.013031in}}%
\pgfpathlineto{\pgfqpoint{5.088439in}{2.022263in}}%
\pgfpathlineto{\pgfqpoint{5.095757in}{2.031431in}}%
\pgfpathlineto{\pgfqpoint{5.103070in}{2.040533in}}%
\pgfpathlineto{\pgfqpoint{5.110376in}{2.049569in}}%
\pgfpathclose%
\pgfusepath{fill}%
\end{pgfscope}%
\begin{pgfscope}%
\pgfpathrectangle{\pgfqpoint{1.254980in}{0.150000in}}{\pgfqpoint{5.490039in}{5.490039in}}%
\pgfusepath{clip}%
\pgfsetbuttcap%
\pgfsetroundjoin%
\definecolor{currentfill}{rgb}{0.273006,0.204520,0.501721}%
\pgfsetfillcolor{currentfill}%
\pgfsetfillopacity{0.700000}%
\pgfsetlinewidth{0.000000pt}%
\definecolor{currentstroke}{rgb}{0.000000,0.000000,0.000000}%
\pgfsetstrokecolor{currentstroke}%
\pgfsetdash{}{0pt}%
\pgfpathmoveto{\pgfqpoint{2.890653in}{2.193287in}}%
\pgfpathlineto{\pgfqpoint{2.903749in}{2.185483in}}%
\pgfpathlineto{\pgfqpoint{2.916847in}{2.177717in}}%
\pgfpathlineto{\pgfqpoint{2.929949in}{2.169987in}}%
\pgfpathlineto{\pgfqpoint{2.943054in}{2.162293in}}%
\pgfpathlineto{\pgfqpoint{2.934708in}{2.165913in}}%
\pgfpathlineto{\pgfqpoint{2.926345in}{2.169883in}}%
\pgfpathlineto{\pgfqpoint{2.917964in}{2.174212in}}%
\pgfpathlineto{\pgfqpoint{2.909564in}{2.178908in}}%
\pgfpathlineto{\pgfqpoint{2.896426in}{2.186906in}}%
\pgfpathlineto{\pgfqpoint{2.883290in}{2.194940in}}%
\pgfpathlineto{\pgfqpoint{2.870158in}{2.203010in}}%
\pgfpathlineto{\pgfqpoint{2.857028in}{2.211117in}}%
\pgfpathlineto{\pgfqpoint{2.865463in}{2.206112in}}%
\pgfpathlineto{\pgfqpoint{2.873878in}{2.201477in}}%
\pgfpathlineto{\pgfqpoint{2.882275in}{2.197205in}}%
\pgfpathlineto{\pgfqpoint{2.890653in}{2.193287in}}%
\pgfpathclose%
\pgfusepath{fill}%
\end{pgfscope}%
\begin{pgfscope}%
\pgfpathrectangle{\pgfqpoint{1.254980in}{0.150000in}}{\pgfqpoint{5.490039in}{5.490039in}}%
\pgfusepath{clip}%
\pgfsetbuttcap%
\pgfsetroundjoin%
\definecolor{currentfill}{rgb}{0.278791,0.062145,0.386592}%
\pgfsetfillcolor{currentfill}%
\pgfsetfillopacity{0.700000}%
\pgfsetlinewidth{0.000000pt}%
\definecolor{currentstroke}{rgb}{0.000000,0.000000,0.000000}%
\pgfsetstrokecolor{currentstroke}%
\pgfsetdash{}{0pt}%
\pgfpathmoveto{\pgfqpoint{4.722349in}{1.934563in}}%
\pgfpathlineto{\pgfqpoint{4.735797in}{1.932508in}}%
\pgfpathlineto{\pgfqpoint{4.749253in}{1.930476in}}%
\pgfpathlineto{\pgfqpoint{4.762716in}{1.928468in}}%
\pgfpathlineto{\pgfqpoint{4.776186in}{1.926484in}}%
\pgfpathlineto{\pgfqpoint{4.768752in}{1.917123in}}%
\pgfpathlineto{\pgfqpoint{4.761312in}{1.907744in}}%
\pgfpathlineto{\pgfqpoint{4.753868in}{1.898349in}}%
\pgfpathlineto{\pgfqpoint{4.746418in}{1.888943in}}%
\pgfpathlineto{\pgfqpoint{4.732938in}{1.891058in}}%
\pgfpathlineto{\pgfqpoint{4.719466in}{1.893198in}}%
\pgfpathlineto{\pgfqpoint{4.706002in}{1.895361in}}%
\pgfpathlineto{\pgfqpoint{4.692545in}{1.897548in}}%
\pgfpathlineto{\pgfqpoint{4.700003in}{1.906819in}}%
\pgfpathlineto{\pgfqpoint{4.707457in}{1.916080in}}%
\pgfpathlineto{\pgfqpoint{4.714906in}{1.925329in}}%
\pgfpathlineto{\pgfqpoint{4.722349in}{1.934563in}}%
\pgfpathclose%
\pgfusepath{fill}%
\end{pgfscope}%
\begin{pgfscope}%
\pgfpathrectangle{\pgfqpoint{1.254980in}{0.150000in}}{\pgfqpoint{5.490039in}{5.490039in}}%
\pgfusepath{clip}%
\pgfsetbuttcap%
\pgfsetroundjoin%
\definecolor{currentfill}{rgb}{0.282327,0.094955,0.417331}%
\pgfsetfillcolor{currentfill}%
\pgfsetfillopacity{0.700000}%
\pgfsetlinewidth{0.000000pt}%
\definecolor{currentstroke}{rgb}{0.000000,0.000000,0.000000}%
\pgfsetstrokecolor{currentstroke}%
\pgfsetdash{}{0pt}%
\pgfpathmoveto{\pgfqpoint{3.323377in}{1.984065in}}%
\pgfpathlineto{\pgfqpoint{3.336517in}{1.977759in}}%
\pgfpathlineto{\pgfqpoint{3.349661in}{1.971484in}}%
\pgfpathlineto{\pgfqpoint{3.362809in}{1.965240in}}%
\pgfpathlineto{\pgfqpoint{3.375961in}{1.959025in}}%
\pgfpathlineto{\pgfqpoint{3.367932in}{1.958240in}}%
\pgfpathlineto{\pgfqpoint{3.359891in}{1.957719in}}%
\pgfpathlineto{\pgfqpoint{3.351837in}{1.957470in}}%
\pgfpathlineto{\pgfqpoint{3.343772in}{1.957501in}}%
\pgfpathlineto{\pgfqpoint{3.330594in}{1.963989in}}%
\pgfpathlineto{\pgfqpoint{3.317420in}{1.970508in}}%
\pgfpathlineto{\pgfqpoint{3.304251in}{1.977056in}}%
\pgfpathlineto{\pgfqpoint{3.291086in}{1.983635in}}%
\pgfpathlineto{\pgfqpoint{3.299178in}{1.983326in}}%
\pgfpathlineto{\pgfqpoint{3.307257in}{1.983299in}}%
\pgfpathlineto{\pgfqpoint{3.315323in}{1.983548in}}%
\pgfpathlineto{\pgfqpoint{3.323377in}{1.984065in}}%
\pgfpathclose%
\pgfusepath{fill}%
\end{pgfscope}%
\begin{pgfscope}%
\pgfpathrectangle{\pgfqpoint{1.254980in}{0.150000in}}{\pgfqpoint{5.490039in}{5.490039in}}%
\pgfusepath{clip}%
\pgfsetbuttcap%
\pgfsetroundjoin%
\definecolor{currentfill}{rgb}{0.277134,0.185228,0.489898}%
\pgfsetfillcolor{currentfill}%
\pgfsetfillopacity{0.700000}%
\pgfsetlinewidth{0.000000pt}%
\definecolor{currentstroke}{rgb}{0.000000,0.000000,0.000000}%
\pgfsetstrokecolor{currentstroke}%
\pgfsetdash{}{0pt}%
\pgfpathmoveto{\pgfqpoint{5.498472in}{2.168747in}}%
\pgfpathlineto{\pgfqpoint{5.512166in}{2.168230in}}%
\pgfpathlineto{\pgfqpoint{5.525868in}{2.167737in}}%
\pgfpathlineto{\pgfqpoint{5.539580in}{2.167268in}}%
\pgfpathlineto{\pgfqpoint{5.553300in}{2.166822in}}%
\pgfpathlineto{\pgfqpoint{5.546170in}{2.158870in}}%
\pgfpathlineto{\pgfqpoint{5.539031in}{2.150817in}}%
\pgfpathlineto{\pgfqpoint{5.531885in}{2.142665in}}%
\pgfpathlineto{\pgfqpoint{5.524731in}{2.134413in}}%
\pgfpathlineto{\pgfqpoint{5.511000in}{2.134885in}}%
\pgfpathlineto{\pgfqpoint{5.497277in}{2.135382in}}%
\pgfpathlineto{\pgfqpoint{5.483563in}{2.135901in}}%
\pgfpathlineto{\pgfqpoint{5.469858in}{2.136445in}}%
\pgfpathlineto{\pgfqpoint{5.477023in}{2.144665in}}%
\pgfpathlineto{\pgfqpoint{5.484180in}{2.152789in}}%
\pgfpathlineto{\pgfqpoint{5.491330in}{2.160816in}}%
\pgfpathlineto{\pgfqpoint{5.498472in}{2.168747in}}%
\pgfpathclose%
\pgfusepath{fill}%
\end{pgfscope}%
\begin{pgfscope}%
\pgfpathrectangle{\pgfqpoint{1.254980in}{0.150000in}}{\pgfqpoint{5.490039in}{5.490039in}}%
\pgfusepath{clip}%
\pgfsetbuttcap%
\pgfsetroundjoin%
\definecolor{currentfill}{rgb}{0.271305,0.019942,0.347269}%
\pgfsetfillcolor{currentfill}%
\pgfsetfillopacity{0.700000}%
\pgfsetlinewidth{0.000000pt}%
\definecolor{currentstroke}{rgb}{0.000000,0.000000,0.000000}%
\pgfsetstrokecolor{currentstroke}%
\pgfsetdash{}{0pt}%
\pgfpathmoveto{\pgfqpoint{4.418042in}{1.866584in}}%
\pgfpathlineto{\pgfqpoint{4.431406in}{1.863706in}}%
\pgfpathlineto{\pgfqpoint{4.444776in}{1.860854in}}%
\pgfpathlineto{\pgfqpoint{4.458154in}{1.858025in}}%
\pgfpathlineto{\pgfqpoint{4.471538in}{1.855222in}}%
\pgfpathlineto{\pgfqpoint{4.464004in}{1.846521in}}%
\pgfpathlineto{\pgfqpoint{4.456466in}{1.837856in}}%
\pgfpathlineto{\pgfqpoint{4.448922in}{1.829232in}}%
\pgfpathlineto{\pgfqpoint{4.441374in}{1.820653in}}%
\pgfpathlineto{\pgfqpoint{4.427979in}{1.823627in}}%
\pgfpathlineto{\pgfqpoint{4.414591in}{1.826625in}}%
\pgfpathlineto{\pgfqpoint{4.401210in}{1.829647in}}%
\pgfpathlineto{\pgfqpoint{4.387836in}{1.832694in}}%
\pgfpathlineto{\pgfqpoint{4.395395in}{1.841099in}}%
\pgfpathlineto{\pgfqpoint{4.402949in}{1.849552in}}%
\pgfpathlineto{\pgfqpoint{4.410498in}{1.858048in}}%
\pgfpathlineto{\pgfqpoint{4.418042in}{1.866584in}}%
\pgfpathclose%
\pgfusepath{fill}%
\end{pgfscope}%
\begin{pgfscope}%
\pgfpathrectangle{\pgfqpoint{1.254980in}{0.150000in}}{\pgfqpoint{5.490039in}{5.490039in}}%
\pgfusepath{clip}%
\pgfsetbuttcap%
\pgfsetroundjoin%
\definecolor{currentfill}{rgb}{0.282623,0.140926,0.457517}%
\pgfsetfillcolor{currentfill}%
\pgfsetfillopacity{0.700000}%
\pgfsetlinewidth{0.000000pt}%
\definecolor{currentstroke}{rgb}{0.000000,0.000000,0.000000}%
\pgfsetstrokecolor{currentstroke}%
\pgfsetdash{}{0pt}%
\pgfpathmoveto{\pgfqpoint{3.133410in}{2.065022in}}%
\pgfpathlineto{\pgfqpoint{3.146529in}{2.058064in}}%
\pgfpathlineto{\pgfqpoint{3.159651in}{2.051138in}}%
\pgfpathlineto{\pgfqpoint{3.172777in}{2.044245in}}%
\pgfpathlineto{\pgfqpoint{3.185907in}{2.037384in}}%
\pgfpathlineto{\pgfqpoint{3.177746in}{2.038549in}}%
\pgfpathlineto{\pgfqpoint{3.169572in}{2.040019in}}%
\pgfpathlineto{\pgfqpoint{3.161382in}{2.041800in}}%
\pgfpathlineto{\pgfqpoint{3.153178in}{2.043902in}}%
\pgfpathlineto{\pgfqpoint{3.140019in}{2.051051in}}%
\pgfpathlineto{\pgfqpoint{3.126864in}{2.058233in}}%
\pgfpathlineto{\pgfqpoint{3.113712in}{2.065447in}}%
\pgfpathlineto{\pgfqpoint{3.100564in}{2.072694in}}%
\pgfpathlineto{\pgfqpoint{3.108798in}{2.070298in}}%
\pgfpathlineto{\pgfqpoint{3.117017in}{2.068226in}}%
\pgfpathlineto{\pgfqpoint{3.125221in}{2.066470in}}%
\pgfpathlineto{\pgfqpoint{3.133410in}{2.065022in}}%
\pgfpathclose%
\pgfusepath{fill}%
\end{pgfscope}%
\begin{pgfscope}%
\pgfpathrectangle{\pgfqpoint{1.254980in}{0.150000in}}{\pgfqpoint{5.490039in}{5.490039in}}%
\pgfusepath{clip}%
\pgfsetbuttcap%
\pgfsetroundjoin%
\definecolor{currentfill}{rgb}{0.268510,0.009605,0.335427}%
\pgfsetfillcolor{currentfill}%
\pgfsetfillopacity{0.700000}%
\pgfsetlinewidth{0.000000pt}%
\definecolor{currentstroke}{rgb}{0.000000,0.000000,0.000000}%
\pgfsetstrokecolor{currentstroke}%
\pgfsetdash{}{0pt}%
\pgfpathmoveto{\pgfqpoint{4.197430in}{1.840317in}}%
\pgfpathlineto{\pgfqpoint{4.210738in}{1.836791in}}%
\pgfpathlineto{\pgfqpoint{4.224052in}{1.833289in}}%
\pgfpathlineto{\pgfqpoint{4.237373in}{1.829813in}}%
\pgfpathlineto{\pgfqpoint{4.250699in}{1.826362in}}%
\pgfpathlineto{\pgfqpoint{4.243091in}{1.818642in}}%
\pgfpathlineto{\pgfqpoint{4.235477in}{1.811004in}}%
\pgfpathlineto{\pgfqpoint{4.227857in}{1.803451in}}%
\pgfpathlineto{\pgfqpoint{4.220232in}{1.795989in}}%
\pgfpathlineto{\pgfqpoint{4.206893in}{1.799636in}}%
\pgfpathlineto{\pgfqpoint{4.193560in}{1.803308in}}%
\pgfpathlineto{\pgfqpoint{4.180233in}{1.807004in}}%
\pgfpathlineto{\pgfqpoint{4.166912in}{1.810726in}}%
\pgfpathlineto{\pgfqpoint{4.174550in}{1.817988in}}%
\pgfpathlineto{\pgfqpoint{4.182183in}{1.825343in}}%
\pgfpathlineto{\pgfqpoint{4.189809in}{1.832788in}}%
\pgfpathlineto{\pgfqpoint{4.197430in}{1.840317in}}%
\pgfpathclose%
\pgfusepath{fill}%
\end{pgfscope}%
\begin{pgfscope}%
\pgfpathrectangle{\pgfqpoint{1.254980in}{0.150000in}}{\pgfqpoint{5.490039in}{5.490039in}}%
\pgfusepath{clip}%
\pgfsetbuttcap%
\pgfsetroundjoin%
\definecolor{currentfill}{rgb}{0.277941,0.056324,0.381191}%
\pgfsetfillcolor{currentfill}%
\pgfsetfillopacity{0.700000}%
\pgfsetlinewidth{0.000000pt}%
\definecolor{currentstroke}{rgb}{0.000000,0.000000,0.000000}%
\pgfsetstrokecolor{currentstroke}%
\pgfsetdash{}{0pt}%
\pgfpathmoveto{\pgfqpoint{3.513154in}{1.918101in}}%
\pgfpathlineto{\pgfqpoint{3.526323in}{1.912410in}}%
\pgfpathlineto{\pgfqpoint{3.539498in}{1.906747in}}%
\pgfpathlineto{\pgfqpoint{3.552677in}{1.901113in}}%
\pgfpathlineto{\pgfqpoint{3.565861in}{1.895507in}}%
\pgfpathlineto{\pgfqpoint{3.557944in}{1.892967in}}%
\pgfpathlineto{\pgfqpoint{3.550018in}{1.890655in}}%
\pgfpathlineto{\pgfqpoint{3.542081in}{1.888576in}}%
\pgfpathlineto{\pgfqpoint{3.534134in}{1.886737in}}%
\pgfpathlineto{\pgfqpoint{3.520928in}{1.892603in}}%
\pgfpathlineto{\pgfqpoint{3.507727in}{1.898497in}}%
\pgfpathlineto{\pgfqpoint{3.494530in}{1.904420in}}%
\pgfpathlineto{\pgfqpoint{3.481338in}{1.910371in}}%
\pgfpathlineto{\pgfqpoint{3.489307in}{1.911944in}}%
\pgfpathlineto{\pgfqpoint{3.497267in}{1.913762in}}%
\pgfpathlineto{\pgfqpoint{3.505215in}{1.915816in}}%
\pgfpathlineto{\pgfqpoint{3.513154in}{1.918101in}}%
\pgfpathclose%
\pgfusepath{fill}%
\end{pgfscope}%
\begin{pgfscope}%
\pgfpathrectangle{\pgfqpoint{1.254980in}{0.150000in}}{\pgfqpoint{5.490039in}{5.490039in}}%
\pgfusepath{clip}%
\pgfsetbuttcap%
\pgfsetroundjoin%
\definecolor{currentfill}{rgb}{0.283091,0.110553,0.431554}%
\pgfsetfillcolor{currentfill}%
\pgfsetfillopacity{0.700000}%
\pgfsetlinewidth{0.000000pt}%
\definecolor{currentstroke}{rgb}{0.000000,0.000000,0.000000}%
\pgfsetstrokecolor{currentstroke}%
\pgfsetdash{}{0pt}%
\pgfpathmoveto{\pgfqpoint{5.026889in}{2.018308in}}%
\pgfpathlineto{\pgfqpoint{5.040433in}{2.016953in}}%
\pgfpathlineto{\pgfqpoint{5.053985in}{2.015622in}}%
\pgfpathlineto{\pgfqpoint{5.067546in}{2.014315in}}%
\pgfpathlineto{\pgfqpoint{5.081114in}{2.013031in}}%
\pgfpathlineto{\pgfqpoint{5.073783in}{2.003737in}}%
\pgfpathlineto{\pgfqpoint{5.066446in}{1.994383in}}%
\pgfpathlineto{\pgfqpoint{5.059102in}{1.984971in}}%
\pgfpathlineto{\pgfqpoint{5.051753in}{1.975502in}}%
\pgfpathlineto{\pgfqpoint{5.038176in}{1.976878in}}%
\pgfpathlineto{\pgfqpoint{5.024607in}{1.978279in}}%
\pgfpathlineto{\pgfqpoint{5.011046in}{1.979703in}}%
\pgfpathlineto{\pgfqpoint{4.997493in}{1.981151in}}%
\pgfpathlineto{\pgfqpoint{5.004851in}{1.990521in}}%
\pgfpathlineto{\pgfqpoint{5.012203in}{1.999839in}}%
\pgfpathlineto{\pgfqpoint{5.019549in}{2.009102in}}%
\pgfpathlineto{\pgfqpoint{5.026889in}{2.018308in}}%
\pgfpathclose%
\pgfusepath{fill}%
\end{pgfscope}%
\begin{pgfscope}%
\pgfpathrectangle{\pgfqpoint{1.254980in}{0.150000in}}{\pgfqpoint{5.490039in}{5.490039in}}%
\pgfusepath{clip}%
\pgfsetbuttcap%
\pgfsetroundjoin%
\definecolor{currentfill}{rgb}{0.248629,0.278775,0.534556}%
\pgfsetfillcolor{currentfill}%
\pgfsetfillopacity{0.700000}%
\pgfsetlinewidth{0.000000pt}%
\definecolor{currentstroke}{rgb}{0.000000,0.000000,0.000000}%
\pgfsetstrokecolor{currentstroke}%
\pgfsetdash{}{0pt}%
\pgfpathmoveto{\pgfqpoint{2.647308in}{2.346100in}}%
\pgfpathlineto{\pgfqpoint{2.660397in}{2.337360in}}%
\pgfpathlineto{\pgfqpoint{2.673489in}{2.328661in}}%
\pgfpathlineto{\pgfqpoint{2.686583in}{2.320005in}}%
\pgfpathlineto{\pgfqpoint{2.699680in}{2.311390in}}%
\pgfpathlineto{\pgfqpoint{2.691116in}{2.317717in}}%
\pgfpathlineto{\pgfqpoint{2.682531in}{2.324442in}}%
\pgfpathlineto{\pgfqpoint{2.673923in}{2.331575in}}%
\pgfpathlineto{\pgfqpoint{2.665293in}{2.339124in}}%
\pgfpathlineto{\pgfqpoint{2.652158in}{2.348060in}}%
\pgfpathlineto{\pgfqpoint{2.639025in}{2.357037in}}%
\pgfpathlineto{\pgfqpoint{2.625894in}{2.366056in}}%
\pgfpathlineto{\pgfqpoint{2.612766in}{2.375117in}}%
\pgfpathlineto{\pgfqpoint{2.621436in}{2.367241in}}%
\pgfpathlineto{\pgfqpoint{2.630083in}{2.359786in}}%
\pgfpathlineto{\pgfqpoint{2.638706in}{2.352741in}}%
\pgfpathlineto{\pgfqpoint{2.647308in}{2.346100in}}%
\pgfpathclose%
\pgfusepath{fill}%
\end{pgfscope}%
\begin{pgfscope}%
\pgfpathrectangle{\pgfqpoint{1.254980in}{0.150000in}}{\pgfqpoint{5.490039in}{5.490039in}}%
\pgfusepath{clip}%
\pgfsetbuttcap%
\pgfsetroundjoin%
\definecolor{currentfill}{rgb}{0.277018,0.050344,0.375715}%
\pgfsetfillcolor{currentfill}%
\pgfsetfillopacity{0.700000}%
\pgfsetlinewidth{0.000000pt}%
\definecolor{currentstroke}{rgb}{0.000000,0.000000,0.000000}%
\pgfsetstrokecolor{currentstroke}%
\pgfsetdash{}{0pt}%
\pgfpathmoveto{\pgfqpoint{4.638789in}{1.906539in}}%
\pgfpathlineto{\pgfqpoint{4.652217in}{1.904255in}}%
\pgfpathlineto{\pgfqpoint{4.665652in}{1.901996in}}%
\pgfpathlineto{\pgfqpoint{4.679095in}{1.899760in}}%
\pgfpathlineto{\pgfqpoint{4.692545in}{1.897548in}}%
\pgfpathlineto{\pgfqpoint{4.685081in}{1.888273in}}%
\pgfpathlineto{\pgfqpoint{4.677611in}{1.878994in}}%
\pgfpathlineto{\pgfqpoint{4.670137in}{1.869717in}}%
\pgfpathlineto{\pgfqpoint{4.662658in}{1.860444in}}%
\pgfpathlineto{\pgfqpoint{4.649199in}{1.862800in}}%
\pgfpathlineto{\pgfqpoint{4.635747in}{1.865180in}}%
\pgfpathlineto{\pgfqpoint{4.622302in}{1.867585in}}%
\pgfpathlineto{\pgfqpoint{4.608865in}{1.870013in}}%
\pgfpathlineto{\pgfqpoint{4.616353in}{1.879137in}}%
\pgfpathlineto{\pgfqpoint{4.623837in}{1.888268in}}%
\pgfpathlineto{\pgfqpoint{4.631315in}{1.897403in}}%
\pgfpathlineto{\pgfqpoint{4.638789in}{1.906539in}}%
\pgfpathclose%
\pgfusepath{fill}%
\end{pgfscope}%
\begin{pgfscope}%
\pgfpathrectangle{\pgfqpoint{1.254980in}{0.150000in}}{\pgfqpoint{5.490039in}{5.490039in}}%
\pgfusepath{clip}%
\pgfsetbuttcap%
\pgfsetroundjoin%
\definecolor{currentfill}{rgb}{0.278826,0.175490,0.483397}%
\pgfsetfillcolor{currentfill}%
\pgfsetfillopacity{0.700000}%
\pgfsetlinewidth{0.000000pt}%
\definecolor{currentstroke}{rgb}{0.000000,0.000000,0.000000}%
\pgfsetstrokecolor{currentstroke}%
\pgfsetdash{}{0pt}%
\pgfpathmoveto{\pgfqpoint{5.415123in}{2.138857in}}%
\pgfpathlineto{\pgfqpoint{5.428794in}{2.138218in}}%
\pgfpathlineto{\pgfqpoint{5.442473in}{2.137604in}}%
\pgfpathlineto{\pgfqpoint{5.456161in}{2.137012in}}%
\pgfpathlineto{\pgfqpoint{5.469858in}{2.136445in}}%
\pgfpathlineto{\pgfqpoint{5.462685in}{2.128129in}}%
\pgfpathlineto{\pgfqpoint{5.455505in}{2.119717in}}%
\pgfpathlineto{\pgfqpoint{5.448317in}{2.111210in}}%
\pgfpathlineto{\pgfqpoint{5.441121in}{2.102608in}}%
\pgfpathlineto{\pgfqpoint{5.427414in}{2.103216in}}%
\pgfpathlineto{\pgfqpoint{5.413716in}{2.103847in}}%
\pgfpathlineto{\pgfqpoint{5.400026in}{2.104502in}}%
\pgfpathlineto{\pgfqpoint{5.386345in}{2.105181in}}%
\pgfpathlineto{\pgfqpoint{5.393550in}{2.113738in}}%
\pgfpathlineto{\pgfqpoint{5.400749in}{2.122203in}}%
\pgfpathlineto{\pgfqpoint{5.407940in}{2.130576in}}%
\pgfpathlineto{\pgfqpoint{5.415123in}{2.138857in}}%
\pgfpathclose%
\pgfusepath{fill}%
\end{pgfscope}%
\begin{pgfscope}%
\pgfpathrectangle{\pgfqpoint{1.254980in}{0.150000in}}{\pgfqpoint{5.490039in}{5.490039in}}%
\pgfusepath{clip}%
\pgfsetbuttcap%
\pgfsetroundjoin%
\definecolor{currentfill}{rgb}{0.269308,0.218818,0.509577}%
\pgfsetfillcolor{currentfill}%
\pgfsetfillopacity{0.700000}%
\pgfsetlinewidth{0.000000pt}%
\definecolor{currentstroke}{rgb}{0.000000,0.000000,0.000000}%
\pgfsetstrokecolor{currentstroke}%
\pgfsetdash{}{0pt}%
\pgfpathmoveto{\pgfqpoint{5.719935in}{2.224248in}}%
\pgfpathlineto{\pgfqpoint{5.733710in}{2.224029in}}%
\pgfpathlineto{\pgfqpoint{5.747495in}{2.223835in}}%
\pgfpathlineto{\pgfqpoint{5.761288in}{2.223664in}}%
\pgfpathlineto{\pgfqpoint{5.754259in}{2.216520in}}%
\pgfpathlineto{\pgfqpoint{5.747221in}{2.209270in}}%
\pgfpathlineto{\pgfqpoint{5.740174in}{2.201912in}}%
\pgfpathlineto{\pgfqpoint{5.733118in}{2.194445in}}%
\pgfpathlineto{\pgfqpoint{5.719312in}{2.194616in}}%
\pgfpathlineto{\pgfqpoint{5.705514in}{2.194811in}}%
\pgfpathlineto{\pgfqpoint{5.691726in}{2.195029in}}%
\pgfpathlineto{\pgfqpoint{5.698791in}{2.202492in}}%
\pgfpathlineto{\pgfqpoint{5.705847in}{2.209849in}}%
\pgfpathlineto{\pgfqpoint{5.712895in}{2.217101in}}%
\pgfpathlineto{\pgfqpoint{5.719935in}{2.224248in}}%
\pgfpathclose%
\pgfusepath{fill}%
\end{pgfscope}%
\begin{pgfscope}%
\pgfpathrectangle{\pgfqpoint{1.254980in}{0.150000in}}{\pgfqpoint{5.490039in}{5.490039in}}%
\pgfusepath{clip}%
\pgfsetbuttcap%
\pgfsetroundjoin%
\definecolor{currentfill}{rgb}{0.282327,0.094955,0.417331}%
\pgfsetfillcolor{currentfill}%
\pgfsetfillopacity{0.700000}%
\pgfsetlinewidth{0.000000pt}%
\definecolor{currentstroke}{rgb}{0.000000,0.000000,0.000000}%
\pgfsetstrokecolor{currentstroke}%
\pgfsetdash{}{0pt}%
\pgfpathmoveto{\pgfqpoint{4.943360in}{1.987181in}}%
\pgfpathlineto{\pgfqpoint{4.956881in}{1.985637in}}%
\pgfpathlineto{\pgfqpoint{4.970411in}{1.984118in}}%
\pgfpathlineto{\pgfqpoint{4.983948in}{1.982622in}}%
\pgfpathlineto{\pgfqpoint{4.997493in}{1.981151in}}%
\pgfpathlineto{\pgfqpoint{4.990129in}{1.971729in}}%
\pgfpathlineto{\pgfqpoint{4.982760in}{1.962258in}}%
\pgfpathlineto{\pgfqpoint{4.975385in}{1.952742in}}%
\pgfpathlineto{\pgfqpoint{4.968004in}{1.943181in}}%
\pgfpathlineto{\pgfqpoint{4.954450in}{1.944758in}}%
\pgfpathlineto{\pgfqpoint{4.940904in}{1.946360in}}%
\pgfpathlineto{\pgfqpoint{4.927366in}{1.947985in}}%
\pgfpathlineto{\pgfqpoint{4.913835in}{1.949635in}}%
\pgfpathlineto{\pgfqpoint{4.921225in}{1.959085in}}%
\pgfpathlineto{\pgfqpoint{4.928609in}{1.968494in}}%
\pgfpathlineto{\pgfqpoint{4.935987in}{1.977860in}}%
\pgfpathlineto{\pgfqpoint{4.943360in}{1.987181in}}%
\pgfpathclose%
\pgfusepath{fill}%
\end{pgfscope}%
\begin{pgfscope}%
\pgfpathrectangle{\pgfqpoint{1.254980in}{0.150000in}}{\pgfqpoint{5.490039in}{5.490039in}}%
\pgfusepath{clip}%
\pgfsetbuttcap%
\pgfsetroundjoin%
\definecolor{currentfill}{rgb}{0.269944,0.014625,0.341379}%
\pgfsetfillcolor{currentfill}%
\pgfsetfillopacity{0.700000}%
\pgfsetlinewidth{0.000000pt}%
\definecolor{currentstroke}{rgb}{0.000000,0.000000,0.000000}%
\pgfsetstrokecolor{currentstroke}%
\pgfsetdash{}{0pt}%
\pgfpathmoveto{\pgfqpoint{3.839860in}{1.844959in}}%
\pgfpathlineto{\pgfqpoint{3.853094in}{1.840296in}}%
\pgfpathlineto{\pgfqpoint{3.866332in}{1.835660in}}%
\pgfpathlineto{\pgfqpoint{3.879577in}{1.831050in}}%
\pgfpathlineto{\pgfqpoint{3.892826in}{1.826466in}}%
\pgfpathlineto{\pgfqpoint{3.885072in}{1.821197in}}%
\pgfpathlineto{\pgfqpoint{3.877311in}{1.816089in}}%
\pgfpathlineto{\pgfqpoint{3.869542in}{1.811147in}}%
\pgfpathlineto{\pgfqpoint{3.861766in}{1.806378in}}%
\pgfpathlineto{\pgfqpoint{3.848499in}{1.811195in}}%
\pgfpathlineto{\pgfqpoint{3.835238in}{1.816039in}}%
\pgfpathlineto{\pgfqpoint{3.821982in}{1.820909in}}%
\pgfpathlineto{\pgfqpoint{3.808731in}{1.825805in}}%
\pgfpathlineto{\pgfqpoint{3.816525in}{1.830336in}}%
\pgfpathlineto{\pgfqpoint{3.824311in}{1.835042in}}%
\pgfpathlineto{\pgfqpoint{3.832089in}{1.839919in}}%
\pgfpathlineto{\pgfqpoint{3.839860in}{1.844959in}}%
\pgfpathclose%
\pgfusepath{fill}%
\end{pgfscope}%
\begin{pgfscope}%
\pgfpathrectangle{\pgfqpoint{1.254980in}{0.150000in}}{\pgfqpoint{5.490039in}{5.490039in}}%
\pgfusepath{clip}%
\pgfsetbuttcap%
\pgfsetroundjoin%
\definecolor{currentfill}{rgb}{0.276194,0.190074,0.493001}%
\pgfsetfillcolor{currentfill}%
\pgfsetfillopacity{0.700000}%
\pgfsetlinewidth{0.000000pt}%
\definecolor{currentstroke}{rgb}{0.000000,0.000000,0.000000}%
\pgfsetstrokecolor{currentstroke}%
\pgfsetdash{}{0pt}%
\pgfpathmoveto{\pgfqpoint{2.943054in}{2.162293in}}%
\pgfpathlineto{\pgfqpoint{2.956162in}{2.154635in}}%
\pgfpathlineto{\pgfqpoint{2.969273in}{2.147013in}}%
\pgfpathlineto{\pgfqpoint{2.982387in}{2.139426in}}%
\pgfpathlineto{\pgfqpoint{2.995504in}{2.131874in}}%
\pgfpathlineto{\pgfqpoint{2.987191in}{2.135196in}}%
\pgfpathlineto{\pgfqpoint{2.978861in}{2.138864in}}%
\pgfpathlineto{\pgfqpoint{2.970513in}{2.142888in}}%
\pgfpathlineto{\pgfqpoint{2.962147in}{2.147275in}}%
\pgfpathlineto{\pgfqpoint{2.948997in}{2.155130in}}%
\pgfpathlineto{\pgfqpoint{2.935849in}{2.163021in}}%
\pgfpathlineto{\pgfqpoint{2.922705in}{2.170947in}}%
\pgfpathlineto{\pgfqpoint{2.909564in}{2.178908in}}%
\pgfpathlineto{\pgfqpoint{2.917964in}{2.174212in}}%
\pgfpathlineto{\pgfqpoint{2.926345in}{2.169883in}}%
\pgfpathlineto{\pgfqpoint{2.934708in}{2.165913in}}%
\pgfpathlineto{\pgfqpoint{2.943054in}{2.162293in}}%
\pgfpathclose%
\pgfusepath{fill}%
\end{pgfscope}%
\begin{pgfscope}%
\pgfpathrectangle{\pgfqpoint{1.254980in}{0.150000in}}{\pgfqpoint{5.490039in}{5.490039in}}%
\pgfusepath{clip}%
\pgfsetbuttcap%
\pgfsetroundjoin%
\definecolor{currentfill}{rgb}{0.269944,0.014625,0.341379}%
\pgfsetfillcolor{currentfill}%
\pgfsetfillopacity{0.700000}%
\pgfsetlinewidth{0.000000pt}%
\definecolor{currentstroke}{rgb}{0.000000,0.000000,0.000000}%
\pgfsetstrokecolor{currentstroke}%
\pgfsetdash{}{0pt}%
\pgfpathmoveto{\pgfqpoint{4.334405in}{1.845128in}}%
\pgfpathlineto{\pgfqpoint{4.347753in}{1.841982in}}%
\pgfpathlineto{\pgfqpoint{4.361107in}{1.838862in}}%
\pgfpathlineto{\pgfqpoint{4.374468in}{1.835766in}}%
\pgfpathlineto{\pgfqpoint{4.387836in}{1.832694in}}%
\pgfpathlineto{\pgfqpoint{4.380271in}{1.824342in}}%
\pgfpathlineto{\pgfqpoint{4.372702in}{1.816045in}}%
\pgfpathlineto{\pgfqpoint{4.365127in}{1.807810in}}%
\pgfpathlineto{\pgfqpoint{4.357547in}{1.799640in}}%
\pgfpathlineto{\pgfqpoint{4.344169in}{1.802894in}}%
\pgfpathlineto{\pgfqpoint{4.330796in}{1.806172in}}%
\pgfpathlineto{\pgfqpoint{4.317431in}{1.809476in}}%
\pgfpathlineto{\pgfqpoint{4.304071in}{1.812803in}}%
\pgfpathlineto{\pgfqpoint{4.311663in}{1.820786in}}%
\pgfpathlineto{\pgfqpoint{4.319249in}{1.828838in}}%
\pgfpathlineto{\pgfqpoint{4.326830in}{1.836953in}}%
\pgfpathlineto{\pgfqpoint{4.334405in}{1.845128in}}%
\pgfpathclose%
\pgfusepath{fill}%
\end{pgfscope}%
\begin{pgfscope}%
\pgfpathrectangle{\pgfqpoint{1.254980in}{0.150000in}}{\pgfqpoint{5.490039in}{5.490039in}}%
\pgfusepath{clip}%
\pgfsetbuttcap%
\pgfsetroundjoin%
\definecolor{currentfill}{rgb}{0.268510,0.009605,0.335427}%
\pgfsetfillcolor{currentfill}%
\pgfsetfillopacity{0.700000}%
\pgfsetlinewidth{0.000000pt}%
\definecolor{currentstroke}{rgb}{0.000000,0.000000,0.000000}%
\pgfsetstrokecolor{currentstroke}%
\pgfsetdash{}{0pt}%
\pgfpathmoveto{\pgfqpoint{3.976764in}{1.831837in}}%
\pgfpathlineto{\pgfqpoint{3.990027in}{1.827604in}}%
\pgfpathlineto{\pgfqpoint{4.003295in}{1.823396in}}%
\pgfpathlineto{\pgfqpoint{4.016569in}{1.819213in}}%
\pgfpathlineto{\pgfqpoint{4.029849in}{1.815057in}}%
\pgfpathlineto{\pgfqpoint{4.022153in}{1.808772in}}%
\pgfpathlineto{\pgfqpoint{4.014451in}{1.802618in}}%
\pgfpathlineto{\pgfqpoint{4.006743in}{1.796601in}}%
\pgfpathlineto{\pgfqpoint{3.999028in}{1.790727in}}%
\pgfpathlineto{\pgfqpoint{3.985733in}{1.795105in}}%
\pgfpathlineto{\pgfqpoint{3.972443in}{1.799507in}}%
\pgfpathlineto{\pgfqpoint{3.959160in}{1.803936in}}%
\pgfpathlineto{\pgfqpoint{3.945882in}{1.808390in}}%
\pgfpathlineto{\pgfqpoint{3.953612in}{1.814039in}}%
\pgfpathlineto{\pgfqpoint{3.961336in}{1.819833in}}%
\pgfpathlineto{\pgfqpoint{3.969054in}{1.825768in}}%
\pgfpathlineto{\pgfqpoint{3.976764in}{1.831837in}}%
\pgfpathclose%
\pgfusepath{fill}%
\end{pgfscope}%
\begin{pgfscope}%
\pgfpathrectangle{\pgfqpoint{1.254980in}{0.150000in}}{\pgfqpoint{5.490039in}{5.490039in}}%
\pgfusepath{clip}%
\pgfsetbuttcap%
\pgfsetroundjoin%
\definecolor{currentfill}{rgb}{0.280868,0.160771,0.472899}%
\pgfsetfillcolor{currentfill}%
\pgfsetfillopacity{0.700000}%
\pgfsetlinewidth{0.000000pt}%
\definecolor{currentstroke}{rgb}{0.000000,0.000000,0.000000}%
\pgfsetstrokecolor{currentstroke}%
\pgfsetdash{}{0pt}%
\pgfpathmoveto{\pgfqpoint{5.331705in}{2.108134in}}%
\pgfpathlineto{\pgfqpoint{5.345352in}{2.107360in}}%
\pgfpathlineto{\pgfqpoint{5.359008in}{2.106610in}}%
\pgfpathlineto{\pgfqpoint{5.372672in}{2.105884in}}%
\pgfpathlineto{\pgfqpoint{5.386345in}{2.105181in}}%
\pgfpathlineto{\pgfqpoint{5.379132in}{2.096534in}}%
\pgfpathlineto{\pgfqpoint{5.371911in}{2.087798in}}%
\pgfpathlineto{\pgfqpoint{5.364684in}{2.078973in}}%
\pgfpathlineto{\pgfqpoint{5.357449in}{2.070059in}}%
\pgfpathlineto{\pgfqpoint{5.343766in}{2.070815in}}%
\pgfpathlineto{\pgfqpoint{5.330092in}{2.071595in}}%
\pgfpathlineto{\pgfqpoint{5.316427in}{2.072399in}}%
\pgfpathlineto{\pgfqpoint{5.302770in}{2.073226in}}%
\pgfpathlineto{\pgfqpoint{5.310014in}{2.082081in}}%
\pgfpathlineto{\pgfqpoint{5.317252in}{2.090851in}}%
\pgfpathlineto{\pgfqpoint{5.324482in}{2.099536in}}%
\pgfpathlineto{\pgfqpoint{5.331705in}{2.108134in}}%
\pgfpathclose%
\pgfusepath{fill}%
\end{pgfscope}%
\begin{pgfscope}%
\pgfpathrectangle{\pgfqpoint{1.254980in}{0.150000in}}{\pgfqpoint{5.490039in}{5.490039in}}%
\pgfusepath{clip}%
\pgfsetbuttcap%
\pgfsetroundjoin%
\definecolor{currentfill}{rgb}{0.273809,0.031497,0.358853}%
\pgfsetfillcolor{currentfill}%
\pgfsetfillopacity{0.700000}%
\pgfsetlinewidth{0.000000pt}%
\definecolor{currentstroke}{rgb}{0.000000,0.000000,0.000000}%
\pgfsetstrokecolor{currentstroke}%
\pgfsetdash{}{0pt}%
\pgfpathmoveto{\pgfqpoint{3.702916in}{1.865934in}}%
\pgfpathlineto{\pgfqpoint{3.716125in}{1.860824in}}%
\pgfpathlineto{\pgfqpoint{3.729339in}{1.855740in}}%
\pgfpathlineto{\pgfqpoint{3.742558in}{1.850684in}}%
\pgfpathlineto{\pgfqpoint{3.755782in}{1.845655in}}%
\pgfpathlineto{\pgfqpoint{3.747962in}{1.841547in}}%
\pgfpathlineto{\pgfqpoint{3.740133in}{1.837631in}}%
\pgfpathlineto{\pgfqpoint{3.732296in}{1.833912in}}%
\pgfpathlineto{\pgfqpoint{3.724451in}{1.830398in}}%
\pgfpathlineto{\pgfqpoint{3.711207in}{1.835673in}}%
\pgfpathlineto{\pgfqpoint{3.697969in}{1.840976in}}%
\pgfpathlineto{\pgfqpoint{3.684736in}{1.846306in}}%
\pgfpathlineto{\pgfqpoint{3.671508in}{1.851663in}}%
\pgfpathlineto{\pgfqpoint{3.679373in}{1.854926in}}%
\pgfpathlineto{\pgfqpoint{3.687230in}{1.858397in}}%
\pgfpathlineto{\pgfqpoint{3.695077in}{1.862068in}}%
\pgfpathlineto{\pgfqpoint{3.702916in}{1.865934in}}%
\pgfpathclose%
\pgfusepath{fill}%
\end{pgfscope}%
\begin{pgfscope}%
\pgfpathrectangle{\pgfqpoint{1.254980in}{0.150000in}}{\pgfqpoint{5.490039in}{5.490039in}}%
\pgfusepath{clip}%
\pgfsetbuttcap%
\pgfsetroundjoin%
\definecolor{currentfill}{rgb}{0.274952,0.037752,0.364543}%
\pgfsetfillcolor{currentfill}%
\pgfsetfillopacity{0.700000}%
\pgfsetlinewidth{0.000000pt}%
\definecolor{currentstroke}{rgb}{0.000000,0.000000,0.000000}%
\pgfsetstrokecolor{currentstroke}%
\pgfsetdash{}{0pt}%
\pgfpathmoveto{\pgfqpoint{4.555186in}{1.879968in}}%
\pgfpathlineto{\pgfqpoint{4.568595in}{1.877443in}}%
\pgfpathlineto{\pgfqpoint{4.582011in}{1.874942in}}%
\pgfpathlineto{\pgfqpoint{4.595435in}{1.872465in}}%
\pgfpathlineto{\pgfqpoint{4.608865in}{1.870013in}}%
\pgfpathlineto{\pgfqpoint{4.601371in}{1.860900in}}%
\pgfpathlineto{\pgfqpoint{4.593873in}{1.851802in}}%
\pgfpathlineto{\pgfqpoint{4.586369in}{1.842723in}}%
\pgfpathlineto{\pgfqpoint{4.578861in}{1.833665in}}%
\pgfpathlineto{\pgfqpoint{4.565421in}{1.836275in}}%
\pgfpathlineto{\pgfqpoint{4.551988in}{1.838909in}}%
\pgfpathlineto{\pgfqpoint{4.538562in}{1.841567in}}%
\pgfpathlineto{\pgfqpoint{4.525143in}{1.844250in}}%
\pgfpathlineto{\pgfqpoint{4.532662in}{1.853145in}}%
\pgfpathlineto{\pgfqpoint{4.540175in}{1.862066in}}%
\pgfpathlineto{\pgfqpoint{4.547683in}{1.871008in}}%
\pgfpathlineto{\pgfqpoint{4.555186in}{1.879968in}}%
\pgfpathclose%
\pgfusepath{fill}%
\end{pgfscope}%
\begin{pgfscope}%
\pgfpathrectangle{\pgfqpoint{1.254980in}{0.150000in}}{\pgfqpoint{5.490039in}{5.490039in}}%
\pgfusepath{clip}%
\pgfsetbuttcap%
\pgfsetroundjoin%
\definecolor{currentfill}{rgb}{0.281924,0.089666,0.412415}%
\pgfsetfillcolor{currentfill}%
\pgfsetfillopacity{0.700000}%
\pgfsetlinewidth{0.000000pt}%
\definecolor{currentstroke}{rgb}{0.000000,0.000000,0.000000}%
\pgfsetstrokecolor{currentstroke}%
\pgfsetdash{}{0pt}%
\pgfpathmoveto{\pgfqpoint{3.375961in}{1.959025in}}%
\pgfpathlineto{\pgfqpoint{3.389118in}{1.952840in}}%
\pgfpathlineto{\pgfqpoint{3.402279in}{1.946685in}}%
\pgfpathlineto{\pgfqpoint{3.415444in}{1.940560in}}%
\pgfpathlineto{\pgfqpoint{3.428614in}{1.934464in}}%
\pgfpathlineto{\pgfqpoint{3.420609in}{1.933410in}}%
\pgfpathlineto{\pgfqpoint{3.412593in}{1.932617in}}%
\pgfpathlineto{\pgfqpoint{3.404565in}{1.932093in}}%
\pgfpathlineto{\pgfqpoint{3.396525in}{1.931845in}}%
\pgfpathlineto{\pgfqpoint{3.383330in}{1.938215in}}%
\pgfpathlineto{\pgfqpoint{3.370140in}{1.944614in}}%
\pgfpathlineto{\pgfqpoint{3.356954in}{1.951042in}}%
\pgfpathlineto{\pgfqpoint{3.343772in}{1.957501in}}%
\pgfpathlineto{\pgfqpoint{3.351837in}{1.957470in}}%
\pgfpathlineto{\pgfqpoint{3.359891in}{1.957719in}}%
\pgfpathlineto{\pgfqpoint{3.367932in}{1.958240in}}%
\pgfpathlineto{\pgfqpoint{3.375961in}{1.959025in}}%
\pgfpathclose%
\pgfusepath{fill}%
\end{pgfscope}%
\begin{pgfscope}%
\pgfpathrectangle{\pgfqpoint{1.254980in}{0.150000in}}{\pgfqpoint{5.490039in}{5.490039in}}%
\pgfusepath{clip}%
\pgfsetbuttcap%
\pgfsetroundjoin%
\definecolor{currentfill}{rgb}{0.280894,0.078907,0.402329}%
\pgfsetfillcolor{currentfill}%
\pgfsetfillopacity{0.700000}%
\pgfsetlinewidth{0.000000pt}%
\definecolor{currentstroke}{rgb}{0.000000,0.000000,0.000000}%
\pgfsetstrokecolor{currentstroke}%
\pgfsetdash{}{0pt}%
\pgfpathmoveto{\pgfqpoint{4.859791in}{1.956471in}}%
\pgfpathlineto{\pgfqpoint{4.873291in}{1.954726in}}%
\pgfpathlineto{\pgfqpoint{4.886798in}{1.953005in}}%
\pgfpathlineto{\pgfqpoint{4.900313in}{1.951308in}}%
\pgfpathlineto{\pgfqpoint{4.913835in}{1.949635in}}%
\pgfpathlineto{\pgfqpoint{4.906440in}{1.940146in}}%
\pgfpathlineto{\pgfqpoint{4.899040in}{1.930622in}}%
\pgfpathlineto{\pgfqpoint{4.891634in}{1.921065in}}%
\pgfpathlineto{\pgfqpoint{4.884222in}{1.911478in}}%
\pgfpathlineto{\pgfqpoint{4.870691in}{1.913270in}}%
\pgfpathlineto{\pgfqpoint{4.857167in}{1.915086in}}%
\pgfpathlineto{\pgfqpoint{4.843652in}{1.916926in}}%
\pgfpathlineto{\pgfqpoint{4.830143in}{1.918790in}}%
\pgfpathlineto{\pgfqpoint{4.837563in}{1.928253in}}%
\pgfpathlineto{\pgfqpoint{4.844978in}{1.937690in}}%
\pgfpathlineto{\pgfqpoint{4.852387in}{1.947097in}}%
\pgfpathlineto{\pgfqpoint{4.859791in}{1.956471in}}%
\pgfpathclose%
\pgfusepath{fill}%
\end{pgfscope}%
\begin{pgfscope}%
\pgfpathrectangle{\pgfqpoint{1.254980in}{0.150000in}}{\pgfqpoint{5.490039in}{5.490039in}}%
\pgfusepath{clip}%
\pgfsetbuttcap%
\pgfsetroundjoin%
\definecolor{currentfill}{rgb}{0.203063,0.379716,0.553925}%
\pgfsetfillcolor{currentfill}%
\pgfsetfillopacity{0.700000}%
\pgfsetlinewidth{0.000000pt}%
\definecolor{currentstroke}{rgb}{0.000000,0.000000,0.000000}%
\pgfsetstrokecolor{currentstroke}%
\pgfsetdash{}{0pt}%
\pgfpathmoveto{\pgfqpoint{2.350534in}{2.565901in}}%
\pgfpathlineto{\pgfqpoint{2.363634in}{2.555900in}}%
\pgfpathlineto{\pgfqpoint{2.376734in}{2.545950in}}%
\pgfpathlineto{\pgfqpoint{2.389835in}{2.536052in}}%
\pgfpathlineto{\pgfqpoint{2.402938in}{2.526204in}}%
\pgfpathlineto{\pgfqpoint{2.394075in}{2.535839in}}%
\pgfpathlineto{\pgfqpoint{2.385185in}{2.545929in}}%
\pgfpathlineto{\pgfqpoint{2.376267in}{2.556483in}}%
\pgfpathlineto{\pgfqpoint{2.367321in}{2.567510in}}%
\pgfpathlineto{\pgfqpoint{2.354173in}{2.577698in}}%
\pgfpathlineto{\pgfqpoint{2.341027in}{2.587937in}}%
\pgfpathlineto{\pgfqpoint{2.327882in}{2.598227in}}%
\pgfpathlineto{\pgfqpoint{2.314737in}{2.608569in}}%
\pgfpathlineto{\pgfqpoint{2.323729in}{2.597195in}}%
\pgfpathlineto{\pgfqpoint{2.332693in}{2.586299in}}%
\pgfpathlineto{\pgfqpoint{2.341627in}{2.575871in}}%
\pgfpathlineto{\pgfqpoint{2.350534in}{2.565901in}}%
\pgfpathclose%
\pgfusepath{fill}%
\end{pgfscope}%
\begin{pgfscope}%
\pgfpathrectangle{\pgfqpoint{1.254980in}{0.150000in}}{\pgfqpoint{5.490039in}{5.490039in}}%
\pgfusepath{clip}%
\pgfsetbuttcap%
\pgfsetroundjoin%
\definecolor{currentfill}{rgb}{0.267004,0.004874,0.329415}%
\pgfsetfillcolor{currentfill}%
\pgfsetfillopacity{0.700000}%
\pgfsetlinewidth{0.000000pt}%
\definecolor{currentstroke}{rgb}{0.000000,0.000000,0.000000}%
\pgfsetstrokecolor{currentstroke}%
\pgfsetdash{}{0pt}%
\pgfpathmoveto{\pgfqpoint{4.113692in}{1.825862in}}%
\pgfpathlineto{\pgfqpoint{4.126988in}{1.822040in}}%
\pgfpathlineto{\pgfqpoint{4.140290in}{1.818244in}}%
\pgfpathlineto{\pgfqpoint{4.153598in}{1.814472in}}%
\pgfpathlineto{\pgfqpoint{4.166912in}{1.810726in}}%
\pgfpathlineto{\pgfqpoint{4.159269in}{1.803562in}}%
\pgfpathlineto{\pgfqpoint{4.151620in}{1.796503in}}%
\pgfpathlineto{\pgfqpoint{4.143965in}{1.789552in}}%
\pgfpathlineto{\pgfqpoint{4.136304in}{1.782716in}}%
\pgfpathlineto{\pgfqpoint{4.122976in}{1.786670in}}%
\pgfpathlineto{\pgfqpoint{4.109654in}{1.790650in}}%
\pgfpathlineto{\pgfqpoint{4.096338in}{1.794655in}}%
\pgfpathlineto{\pgfqpoint{4.083028in}{1.798685in}}%
\pgfpathlineto{\pgfqpoint{4.090703in}{1.805308in}}%
\pgfpathlineto{\pgfqpoint{4.098372in}{1.812049in}}%
\pgfpathlineto{\pgfqpoint{4.106035in}{1.818902in}}%
\pgfpathlineto{\pgfqpoint{4.113692in}{1.825862in}}%
\pgfpathclose%
\pgfusepath{fill}%
\end{pgfscope}%
\begin{pgfscope}%
\pgfpathrectangle{\pgfqpoint{1.254980in}{0.150000in}}{\pgfqpoint{5.490039in}{5.490039in}}%
\pgfusepath{clip}%
\pgfsetbuttcap%
\pgfsetroundjoin%
\definecolor{currentfill}{rgb}{0.283072,0.130895,0.449241}%
\pgfsetfillcolor{currentfill}%
\pgfsetfillopacity{0.700000}%
\pgfsetlinewidth{0.000000pt}%
\definecolor{currentstroke}{rgb}{0.000000,0.000000,0.000000}%
\pgfsetstrokecolor{currentstroke}%
\pgfsetdash{}{0pt}%
\pgfpathmoveto{\pgfqpoint{3.185907in}{2.037384in}}%
\pgfpathlineto{\pgfqpoint{3.199041in}{2.030555in}}%
\pgfpathlineto{\pgfqpoint{3.212178in}{2.023758in}}%
\pgfpathlineto{\pgfqpoint{3.225319in}{2.016993in}}%
\pgfpathlineto{\pgfqpoint{3.238465in}{2.010259in}}%
\pgfpathlineto{\pgfqpoint{3.230332in}{2.011142in}}%
\pgfpathlineto{\pgfqpoint{3.222186in}{2.012325in}}%
\pgfpathlineto{\pgfqpoint{3.214025in}{2.013817in}}%
\pgfpathlineto{\pgfqpoint{3.205850in}{2.015626in}}%
\pgfpathlineto{\pgfqpoint{3.192676in}{2.022647in}}%
\pgfpathlineto{\pgfqpoint{3.179506in}{2.029700in}}%
\pgfpathlineto{\pgfqpoint{3.166340in}{2.036785in}}%
\pgfpathlineto{\pgfqpoint{3.153178in}{2.043902in}}%
\pgfpathlineto{\pgfqpoint{3.161382in}{2.041800in}}%
\pgfpathlineto{\pgfqpoint{3.169572in}{2.040019in}}%
\pgfpathlineto{\pgfqpoint{3.177746in}{2.038549in}}%
\pgfpathlineto{\pgfqpoint{3.185907in}{2.037384in}}%
\pgfpathclose%
\pgfusepath{fill}%
\end{pgfscope}%
\begin{pgfscope}%
\pgfpathrectangle{\pgfqpoint{1.254980in}{0.150000in}}{\pgfqpoint{5.490039in}{5.490039in}}%
\pgfusepath{clip}%
\pgfsetbuttcap%
\pgfsetroundjoin%
\definecolor{currentfill}{rgb}{0.253935,0.265254,0.529983}%
\pgfsetfillcolor{currentfill}%
\pgfsetfillopacity{0.700000}%
\pgfsetlinewidth{0.000000pt}%
\definecolor{currentstroke}{rgb}{0.000000,0.000000,0.000000}%
\pgfsetstrokecolor{currentstroke}%
\pgfsetdash{}{0pt}%
\pgfpathmoveto{\pgfqpoint{2.699680in}{2.311390in}}%
\pgfpathlineto{\pgfqpoint{2.712778in}{2.302816in}}%
\pgfpathlineto{\pgfqpoint{2.725879in}{2.294283in}}%
\pgfpathlineto{\pgfqpoint{2.738983in}{2.285791in}}%
\pgfpathlineto{\pgfqpoint{2.752089in}{2.277338in}}%
\pgfpathlineto{\pgfqpoint{2.743562in}{2.283350in}}%
\pgfpathlineto{\pgfqpoint{2.735015in}{2.289757in}}%
\pgfpathlineto{\pgfqpoint{2.726446in}{2.296568in}}%
\pgfpathlineto{\pgfqpoint{2.717854in}{2.303791in}}%
\pgfpathlineto{\pgfqpoint{2.704711in}{2.312564in}}%
\pgfpathlineto{\pgfqpoint{2.691569in}{2.321377in}}%
\pgfpathlineto{\pgfqpoint{2.678430in}{2.330230in}}%
\pgfpathlineto{\pgfqpoint{2.665293in}{2.339124in}}%
\pgfpathlineto{\pgfqpoint{2.673923in}{2.331575in}}%
\pgfpathlineto{\pgfqpoint{2.682531in}{2.324442in}}%
\pgfpathlineto{\pgfqpoint{2.691116in}{2.317717in}}%
\pgfpathlineto{\pgfqpoint{2.699680in}{2.311390in}}%
\pgfpathclose%
\pgfusepath{fill}%
\end{pgfscope}%
\begin{pgfscope}%
\pgfpathrectangle{\pgfqpoint{1.254980in}{0.150000in}}{\pgfqpoint{5.490039in}{5.490039in}}%
\pgfusepath{clip}%
\pgfsetbuttcap%
\pgfsetroundjoin%
\definecolor{currentfill}{rgb}{0.282290,0.145912,0.461510}%
\pgfsetfillcolor{currentfill}%
\pgfsetfillopacity{0.700000}%
\pgfsetlinewidth{0.000000pt}%
\definecolor{currentstroke}{rgb}{0.000000,0.000000,0.000000}%
\pgfsetstrokecolor{currentstroke}%
\pgfsetdash{}{0pt}%
\pgfpathmoveto{\pgfqpoint{5.248226in}{2.076774in}}%
\pgfpathlineto{\pgfqpoint{5.261849in}{2.075851in}}%
\pgfpathlineto{\pgfqpoint{5.275481in}{2.074952in}}%
\pgfpathlineto{\pgfqpoint{5.289121in}{2.074077in}}%
\pgfpathlineto{\pgfqpoint{5.302770in}{2.073226in}}%
\pgfpathlineto{\pgfqpoint{5.295519in}{2.064288in}}%
\pgfpathlineto{\pgfqpoint{5.288260in}{2.055267in}}%
\pgfpathlineto{\pgfqpoint{5.280995in}{2.046165in}}%
\pgfpathlineto{\pgfqpoint{5.273723in}{2.036983in}}%
\pgfpathlineto{\pgfqpoint{5.260065in}{2.037901in}}%
\pgfpathlineto{\pgfqpoint{5.246416in}{2.038843in}}%
\pgfpathlineto{\pgfqpoint{5.232775in}{2.039809in}}%
\pgfpathlineto{\pgfqpoint{5.219142in}{2.040798in}}%
\pgfpathlineto{\pgfqpoint{5.226423in}{2.049908in}}%
\pgfpathlineto{\pgfqpoint{5.233697in}{2.058942in}}%
\pgfpathlineto{\pgfqpoint{5.240965in}{2.067897in}}%
\pgfpathlineto{\pgfqpoint{5.248226in}{2.076774in}}%
\pgfpathclose%
\pgfusepath{fill}%
\end{pgfscope}%
\begin{pgfscope}%
\pgfpathrectangle{\pgfqpoint{1.254980in}{0.150000in}}{\pgfqpoint{5.490039in}{5.490039in}}%
\pgfusepath{clip}%
\pgfsetbuttcap%
\pgfsetroundjoin%
\definecolor{currentfill}{rgb}{0.271828,0.209303,0.504434}%
\pgfsetfillcolor{currentfill}%
\pgfsetfillopacity{0.700000}%
\pgfsetlinewidth{0.000000pt}%
\definecolor{currentstroke}{rgb}{0.000000,0.000000,0.000000}%
\pgfsetstrokecolor{currentstroke}%
\pgfsetdash{}{0pt}%
\pgfpathmoveto{\pgfqpoint{5.636662in}{2.196140in}}%
\pgfpathlineto{\pgfqpoint{5.650415in}{2.195827in}}%
\pgfpathlineto{\pgfqpoint{5.664176in}{2.195537in}}%
\pgfpathlineto{\pgfqpoint{5.677947in}{2.195271in}}%
\pgfpathlineto{\pgfqpoint{5.691726in}{2.195029in}}%
\pgfpathlineto{\pgfqpoint{5.684653in}{2.187460in}}%
\pgfpathlineto{\pgfqpoint{5.677570in}{2.179784in}}%
\pgfpathlineto{\pgfqpoint{5.670480in}{2.172002in}}%
\pgfpathlineto{\pgfqpoint{5.663381in}{2.164113in}}%
\pgfpathlineto{\pgfqpoint{5.649590in}{2.164369in}}%
\pgfpathlineto{\pgfqpoint{5.635807in}{2.164648in}}%
\pgfpathlineto{\pgfqpoint{5.622034in}{2.164951in}}%
\pgfpathlineto{\pgfqpoint{5.608269in}{2.165278in}}%
\pgfpathlineto{\pgfqpoint{5.615380in}{2.173148in}}%
\pgfpathlineto{\pgfqpoint{5.622482in}{2.180915in}}%
\pgfpathlineto{\pgfqpoint{5.629577in}{2.188579in}}%
\pgfpathlineto{\pgfqpoint{5.636662in}{2.196140in}}%
\pgfpathclose%
\pgfusepath{fill}%
\end{pgfscope}%
\begin{pgfscope}%
\pgfpathrectangle{\pgfqpoint{1.254980in}{0.150000in}}{\pgfqpoint{5.490039in}{5.490039in}}%
\pgfusepath{clip}%
\pgfsetbuttcap%
\pgfsetroundjoin%
\definecolor{currentfill}{rgb}{0.277018,0.050344,0.375715}%
\pgfsetfillcolor{currentfill}%
\pgfsetfillopacity{0.700000}%
\pgfsetlinewidth{0.000000pt}%
\definecolor{currentstroke}{rgb}{0.000000,0.000000,0.000000}%
\pgfsetstrokecolor{currentstroke}%
\pgfsetdash{}{0pt}%
\pgfpathmoveto{\pgfqpoint{3.565861in}{1.895507in}}%
\pgfpathlineto{\pgfqpoint{3.579050in}{1.889929in}}%
\pgfpathlineto{\pgfqpoint{3.592243in}{1.884379in}}%
\pgfpathlineto{\pgfqpoint{3.605442in}{1.878858in}}%
\pgfpathlineto{\pgfqpoint{3.618645in}{1.873364in}}%
\pgfpathlineto{\pgfqpoint{3.610750in}{1.870569in}}%
\pgfpathlineto{\pgfqpoint{3.602845in}{1.867998in}}%
\pgfpathlineto{\pgfqpoint{3.594930in}{1.865657in}}%
\pgfpathlineto{\pgfqpoint{3.587006in}{1.863554in}}%
\pgfpathlineto{\pgfqpoint{3.573781in}{1.869308in}}%
\pgfpathlineto{\pgfqpoint{3.560561in}{1.875090in}}%
\pgfpathlineto{\pgfqpoint{3.547345in}{1.880900in}}%
\pgfpathlineto{\pgfqpoint{3.534134in}{1.886737in}}%
\pgfpathlineto{\pgfqpoint{3.542081in}{1.888576in}}%
\pgfpathlineto{\pgfqpoint{3.550018in}{1.890655in}}%
\pgfpathlineto{\pgfqpoint{3.557944in}{1.892967in}}%
\pgfpathlineto{\pgfqpoint{3.565861in}{1.895507in}}%
\pgfpathclose%
\pgfusepath{fill}%
\end{pgfscope}%
\begin{pgfscope}%
\pgfpathrectangle{\pgfqpoint{1.254980in}{0.150000in}}{\pgfqpoint{5.490039in}{5.490039in}}%
\pgfusepath{clip}%
\pgfsetbuttcap%
\pgfsetroundjoin%
\definecolor{currentfill}{rgb}{0.279566,0.067836,0.391917}%
\pgfsetfillcolor{currentfill}%
\pgfsetfillopacity{0.700000}%
\pgfsetlinewidth{0.000000pt}%
\definecolor{currentstroke}{rgb}{0.000000,0.000000,0.000000}%
\pgfsetstrokecolor{currentstroke}%
\pgfsetdash{}{0pt}%
\pgfpathmoveto{\pgfqpoint{4.776186in}{1.926484in}}%
\pgfpathlineto{\pgfqpoint{4.789664in}{1.924525in}}%
\pgfpathlineto{\pgfqpoint{4.803150in}{1.922589in}}%
\pgfpathlineto{\pgfqpoint{4.816643in}{1.920677in}}%
\pgfpathlineto{\pgfqpoint{4.830143in}{1.918790in}}%
\pgfpathlineto{\pgfqpoint{4.822718in}{1.909302in}}%
\pgfpathlineto{\pgfqpoint{4.815287in}{1.899792in}}%
\pgfpathlineto{\pgfqpoint{4.807851in}{1.890264in}}%
\pgfpathlineto{\pgfqpoint{4.800410in}{1.880721in}}%
\pgfpathlineto{\pgfqpoint{4.786901in}{1.882740in}}%
\pgfpathlineto{\pgfqpoint{4.773399in}{1.884784in}}%
\pgfpathlineto{\pgfqpoint{4.759905in}{1.886851in}}%
\pgfpathlineto{\pgfqpoint{4.746418in}{1.888943in}}%
\pgfpathlineto{\pgfqpoint{4.753868in}{1.898349in}}%
\pgfpathlineto{\pgfqpoint{4.761312in}{1.907744in}}%
\pgfpathlineto{\pgfqpoint{4.768752in}{1.917123in}}%
\pgfpathlineto{\pgfqpoint{4.776186in}{1.926484in}}%
\pgfpathclose%
\pgfusepath{fill}%
\end{pgfscope}%
\begin{pgfscope}%
\pgfpathrectangle{\pgfqpoint{1.254980in}{0.150000in}}{\pgfqpoint{5.490039in}{5.490039in}}%
\pgfusepath{clip}%
\pgfsetbuttcap%
\pgfsetroundjoin%
\definecolor{currentfill}{rgb}{0.272594,0.025563,0.353093}%
\pgfsetfillcolor{currentfill}%
\pgfsetfillopacity{0.700000}%
\pgfsetlinewidth{0.000000pt}%
\definecolor{currentstroke}{rgb}{0.000000,0.000000,0.000000}%
\pgfsetstrokecolor{currentstroke}%
\pgfsetdash{}{0pt}%
\pgfpathmoveto{\pgfqpoint{4.471538in}{1.855222in}}%
\pgfpathlineto{\pgfqpoint{4.484929in}{1.852442in}}%
\pgfpathlineto{\pgfqpoint{4.498327in}{1.849687in}}%
\pgfpathlineto{\pgfqpoint{4.511731in}{1.846956in}}%
\pgfpathlineto{\pgfqpoint{4.525143in}{1.844250in}}%
\pgfpathlineto{\pgfqpoint{4.517620in}{1.835384in}}%
\pgfpathlineto{\pgfqpoint{4.510092in}{1.826551in}}%
\pgfpathlineto{\pgfqpoint{4.502559in}{1.817755in}}%
\pgfpathlineto{\pgfqpoint{4.495020in}{1.809001in}}%
\pgfpathlineto{\pgfqpoint{4.481598in}{1.811877in}}%
\pgfpathlineto{\pgfqpoint{4.468183in}{1.814778in}}%
\pgfpathlineto{\pgfqpoint{4.454775in}{1.817703in}}%
\pgfpathlineto{\pgfqpoint{4.441374in}{1.820653in}}%
\pgfpathlineto{\pgfqpoint{4.448922in}{1.829232in}}%
\pgfpathlineto{\pgfqpoint{4.456466in}{1.837856in}}%
\pgfpathlineto{\pgfqpoint{4.464004in}{1.846521in}}%
\pgfpathlineto{\pgfqpoint{4.471538in}{1.855222in}}%
\pgfpathclose%
\pgfusepath{fill}%
\end{pgfscope}%
\begin{pgfscope}%
\pgfpathrectangle{\pgfqpoint{1.254980in}{0.150000in}}{\pgfqpoint{5.490039in}{5.490039in}}%
\pgfusepath{clip}%
\pgfsetbuttcap%
\pgfsetroundjoin%
\definecolor{currentfill}{rgb}{0.283072,0.130895,0.449241}%
\pgfsetfillcolor{currentfill}%
\pgfsetfillopacity{0.700000}%
\pgfsetlinewidth{0.000000pt}%
\definecolor{currentstroke}{rgb}{0.000000,0.000000,0.000000}%
\pgfsetstrokecolor{currentstroke}%
\pgfsetdash{}{0pt}%
\pgfpathmoveto{\pgfqpoint{5.164693in}{2.044993in}}%
\pgfpathlineto{\pgfqpoint{5.178293in}{2.043908in}}%
\pgfpathlineto{\pgfqpoint{5.191901in}{2.042848in}}%
\pgfpathlineto{\pgfqpoint{5.205517in}{2.041811in}}%
\pgfpathlineto{\pgfqpoint{5.219142in}{2.040798in}}%
\pgfpathlineto{\pgfqpoint{5.211854in}{2.031613in}}%
\pgfpathlineto{\pgfqpoint{5.204560in}{2.022354in}}%
\pgfpathlineto{\pgfqpoint{5.197259in}{2.013023in}}%
\pgfpathlineto{\pgfqpoint{5.189952in}{2.003621in}}%
\pgfpathlineto{\pgfqpoint{5.176319in}{2.004714in}}%
\pgfpathlineto{\pgfqpoint{5.162693in}{2.005831in}}%
\pgfpathlineto{\pgfqpoint{5.149076in}{2.006971in}}%
\pgfpathlineto{\pgfqpoint{5.135468in}{2.008136in}}%
\pgfpathlineto{\pgfqpoint{5.142784in}{2.017452in}}%
\pgfpathlineto{\pgfqpoint{5.150093in}{2.026702in}}%
\pgfpathlineto{\pgfqpoint{5.157396in}{2.035883in}}%
\pgfpathlineto{\pgfqpoint{5.164693in}{2.044993in}}%
\pgfpathclose%
\pgfusepath{fill}%
\end{pgfscope}%
\begin{pgfscope}%
\pgfpathrectangle{\pgfqpoint{1.254980in}{0.150000in}}{\pgfqpoint{5.490039in}{5.490039in}}%
\pgfusepath{clip}%
\pgfsetbuttcap%
\pgfsetroundjoin%
\definecolor{currentfill}{rgb}{0.268510,0.009605,0.335427}%
\pgfsetfillcolor{currentfill}%
\pgfsetfillopacity{0.700000}%
\pgfsetlinewidth{0.000000pt}%
\definecolor{currentstroke}{rgb}{0.000000,0.000000,0.000000}%
\pgfsetstrokecolor{currentstroke}%
\pgfsetdash{}{0pt}%
\pgfpathmoveto{\pgfqpoint{4.250699in}{1.826362in}}%
\pgfpathlineto{\pgfqpoint{4.264033in}{1.822935in}}%
\pgfpathlineto{\pgfqpoint{4.277372in}{1.819533in}}%
\pgfpathlineto{\pgfqpoint{4.290719in}{1.816156in}}%
\pgfpathlineto{\pgfqpoint{4.304071in}{1.812803in}}%
\pgfpathlineto{\pgfqpoint{4.296475in}{1.804894in}}%
\pgfpathlineto{\pgfqpoint{4.288873in}{1.797062in}}%
\pgfpathlineto{\pgfqpoint{4.281266in}{1.789312in}}%
\pgfpathlineto{\pgfqpoint{4.273653in}{1.781650in}}%
\pgfpathlineto{\pgfqpoint{4.260288in}{1.785198in}}%
\pgfpathlineto{\pgfqpoint{4.246930in}{1.788770in}}%
\pgfpathlineto{\pgfqpoint{4.233578in}{1.792367in}}%
\pgfpathlineto{\pgfqpoint{4.220232in}{1.795989in}}%
\pgfpathlineto{\pgfqpoint{4.227857in}{1.803451in}}%
\pgfpathlineto{\pgfqpoint{4.235477in}{1.811004in}}%
\pgfpathlineto{\pgfqpoint{4.243091in}{1.818642in}}%
\pgfpathlineto{\pgfqpoint{4.250699in}{1.826362in}}%
\pgfpathclose%
\pgfusepath{fill}%
\end{pgfscope}%
\begin{pgfscope}%
\pgfpathrectangle{\pgfqpoint{1.254980in}{0.150000in}}{\pgfqpoint{5.490039in}{5.490039in}}%
\pgfusepath{clip}%
\pgfsetbuttcap%
\pgfsetroundjoin%
\definecolor{currentfill}{rgb}{0.278012,0.180367,0.486697}%
\pgfsetfillcolor{currentfill}%
\pgfsetfillopacity{0.700000}%
\pgfsetlinewidth{0.000000pt}%
\definecolor{currentstroke}{rgb}{0.000000,0.000000,0.000000}%
\pgfsetstrokecolor{currentstroke}%
\pgfsetdash{}{0pt}%
\pgfpathmoveto{\pgfqpoint{2.995504in}{2.131874in}}%
\pgfpathlineto{\pgfqpoint{3.008625in}{2.124357in}}%
\pgfpathlineto{\pgfqpoint{3.021749in}{2.116874in}}%
\pgfpathlineto{\pgfqpoint{3.034876in}{2.109426in}}%
\pgfpathlineto{\pgfqpoint{3.048007in}{2.102013in}}%
\pgfpathlineto{\pgfqpoint{3.039726in}{2.105037in}}%
\pgfpathlineto{\pgfqpoint{3.031427in}{2.108404in}}%
\pgfpathlineto{\pgfqpoint{3.023112in}{2.112123in}}%
\pgfpathlineto{\pgfqpoint{3.014780in}{2.116202in}}%
\pgfpathlineto{\pgfqpoint{3.001617in}{2.123918in}}%
\pgfpathlineto{\pgfqpoint{2.988457in}{2.131670in}}%
\pgfpathlineto{\pgfqpoint{2.975300in}{2.139455in}}%
\pgfpathlineto{\pgfqpoint{2.962147in}{2.147275in}}%
\pgfpathlineto{\pgfqpoint{2.970513in}{2.142888in}}%
\pgfpathlineto{\pgfqpoint{2.978861in}{2.138864in}}%
\pgfpathlineto{\pgfqpoint{2.987191in}{2.135196in}}%
\pgfpathlineto{\pgfqpoint{2.995504in}{2.131874in}}%
\pgfpathclose%
\pgfusepath{fill}%
\end{pgfscope}%
\begin{pgfscope}%
\pgfpathrectangle{\pgfqpoint{1.254980in}{0.150000in}}{\pgfqpoint{5.490039in}{5.490039in}}%
\pgfusepath{clip}%
\pgfsetbuttcap%
\pgfsetroundjoin%
\definecolor{currentfill}{rgb}{0.210503,0.363727,0.552206}%
\pgfsetfillcolor{currentfill}%
\pgfsetfillopacity{0.700000}%
\pgfsetlinewidth{0.000000pt}%
\definecolor{currentstroke}{rgb}{0.000000,0.000000,0.000000}%
\pgfsetstrokecolor{currentstroke}%
\pgfsetdash{}{0pt}%
\pgfpathmoveto{\pgfqpoint{2.402938in}{2.526204in}}%
\pgfpathlineto{\pgfqpoint{2.416041in}{2.516407in}}%
\pgfpathlineto{\pgfqpoint{2.429146in}{2.506658in}}%
\pgfpathlineto{\pgfqpoint{2.442252in}{2.496959in}}%
\pgfpathlineto{\pgfqpoint{2.455359in}{2.487309in}}%
\pgfpathlineto{\pgfqpoint{2.446539in}{2.496610in}}%
\pgfpathlineto{\pgfqpoint{2.437693in}{2.506362in}}%
\pgfpathlineto{\pgfqpoint{2.428820in}{2.516575in}}%
\pgfpathlineto{\pgfqpoint{2.419919in}{2.527257in}}%
\pgfpathlineto{\pgfqpoint{2.406768in}{2.537246in}}%
\pgfpathlineto{\pgfqpoint{2.393618in}{2.547285in}}%
\pgfpathlineto{\pgfqpoint{2.380469in}{2.557373in}}%
\pgfpathlineto{\pgfqpoint{2.367321in}{2.567510in}}%
\pgfpathlineto{\pgfqpoint{2.376267in}{2.556483in}}%
\pgfpathlineto{\pgfqpoint{2.385185in}{2.545929in}}%
\pgfpathlineto{\pgfqpoint{2.394075in}{2.535839in}}%
\pgfpathlineto{\pgfqpoint{2.402938in}{2.526204in}}%
\pgfpathclose%
\pgfusepath{fill}%
\end{pgfscope}%
\begin{pgfscope}%
\pgfpathrectangle{\pgfqpoint{1.254980in}{0.150000in}}{\pgfqpoint{5.490039in}{5.490039in}}%
\pgfusepath{clip}%
\pgfsetbuttcap%
\pgfsetroundjoin%
\definecolor{currentfill}{rgb}{0.275191,0.194905,0.496005}%
\pgfsetfillcolor{currentfill}%
\pgfsetfillopacity{0.700000}%
\pgfsetlinewidth{0.000000pt}%
\definecolor{currentstroke}{rgb}{0.000000,0.000000,0.000000}%
\pgfsetstrokecolor{currentstroke}%
\pgfsetdash{}{0pt}%
\pgfpathmoveto{\pgfqpoint{5.553300in}{2.166822in}}%
\pgfpathlineto{\pgfqpoint{5.567029in}{2.166400in}}%
\pgfpathlineto{\pgfqpoint{5.580767in}{2.166002in}}%
\pgfpathlineto{\pgfqpoint{5.594514in}{2.165628in}}%
\pgfpathlineto{\pgfqpoint{5.608269in}{2.165278in}}%
\pgfpathlineto{\pgfqpoint{5.601150in}{2.157304in}}%
\pgfpathlineto{\pgfqpoint{5.594023in}{2.149226in}}%
\pgfpathlineto{\pgfqpoint{5.586888in}{2.141045in}}%
\pgfpathlineto{\pgfqpoint{5.579745in}{2.132761in}}%
\pgfpathlineto{\pgfqpoint{5.565978in}{2.133139in}}%
\pgfpathlineto{\pgfqpoint{5.552220in}{2.133540in}}%
\pgfpathlineto{\pgfqpoint{5.538471in}{2.133965in}}%
\pgfpathlineto{\pgfqpoint{5.524731in}{2.134413in}}%
\pgfpathlineto{\pgfqpoint{5.531885in}{2.142665in}}%
\pgfpathlineto{\pgfqpoint{5.539031in}{2.150817in}}%
\pgfpathlineto{\pgfqpoint{5.546170in}{2.158870in}}%
\pgfpathlineto{\pgfqpoint{5.553300in}{2.166822in}}%
\pgfpathclose%
\pgfusepath{fill}%
\end{pgfscope}%
\begin{pgfscope}%
\pgfpathrectangle{\pgfqpoint{1.254980in}{0.150000in}}{\pgfqpoint{5.490039in}{5.490039in}}%
\pgfusepath{clip}%
\pgfsetbuttcap%
\pgfsetroundjoin%
\definecolor{currentfill}{rgb}{0.283197,0.115680,0.436115}%
\pgfsetfillcolor{currentfill}%
\pgfsetfillopacity{0.700000}%
\pgfsetlinewidth{0.000000pt}%
\definecolor{currentstroke}{rgb}{0.000000,0.000000,0.000000}%
\pgfsetstrokecolor{currentstroke}%
\pgfsetdash{}{0pt}%
\pgfpathmoveto{\pgfqpoint{5.081114in}{2.013031in}}%
\pgfpathlineto{\pgfqpoint{5.094690in}{2.011772in}}%
\pgfpathlineto{\pgfqpoint{5.108274in}{2.010536in}}%
\pgfpathlineto{\pgfqpoint{5.121867in}{2.009324in}}%
\pgfpathlineto{\pgfqpoint{5.135468in}{2.008136in}}%
\pgfpathlineto{\pgfqpoint{5.128145in}{1.998754in}}%
\pgfpathlineto{\pgfqpoint{5.120817in}{1.989308in}}%
\pgfpathlineto{\pgfqpoint{5.113482in}{1.979801in}}%
\pgfpathlineto{\pgfqpoint{5.106142in}{1.970234in}}%
\pgfpathlineto{\pgfqpoint{5.092533in}{1.971516in}}%
\pgfpathlineto{\pgfqpoint{5.078931in}{1.972821in}}%
\pgfpathlineto{\pgfqpoint{5.065338in}{1.974149in}}%
\pgfpathlineto{\pgfqpoint{5.051753in}{1.975502in}}%
\pgfpathlineto{\pgfqpoint{5.059102in}{1.984971in}}%
\pgfpathlineto{\pgfqpoint{5.066446in}{1.994383in}}%
\pgfpathlineto{\pgfqpoint{5.073783in}{2.003737in}}%
\pgfpathlineto{\pgfqpoint{5.081114in}{2.013031in}}%
\pgfpathclose%
\pgfusepath{fill}%
\end{pgfscope}%
\begin{pgfscope}%
\pgfpathrectangle{\pgfqpoint{1.254980in}{0.150000in}}{\pgfqpoint{5.490039in}{5.490039in}}%
\pgfusepath{clip}%
\pgfsetbuttcap%
\pgfsetroundjoin%
\definecolor{currentfill}{rgb}{0.277941,0.056324,0.381191}%
\pgfsetfillcolor{currentfill}%
\pgfsetfillopacity{0.700000}%
\pgfsetlinewidth{0.000000pt}%
\definecolor{currentstroke}{rgb}{0.000000,0.000000,0.000000}%
\pgfsetstrokecolor{currentstroke}%
\pgfsetdash{}{0pt}%
\pgfpathmoveto{\pgfqpoint{4.692545in}{1.897548in}}%
\pgfpathlineto{\pgfqpoint{4.706002in}{1.895361in}}%
\pgfpathlineto{\pgfqpoint{4.719466in}{1.893198in}}%
\pgfpathlineto{\pgfqpoint{4.732938in}{1.891058in}}%
\pgfpathlineto{\pgfqpoint{4.746418in}{1.888943in}}%
\pgfpathlineto{\pgfqpoint{4.738963in}{1.879527in}}%
\pgfpathlineto{\pgfqpoint{4.731503in}{1.870106in}}%
\pgfpathlineto{\pgfqpoint{4.724038in}{1.860682in}}%
\pgfpathlineto{\pgfqpoint{4.716567in}{1.851260in}}%
\pgfpathlineto{\pgfqpoint{4.703079in}{1.853520in}}%
\pgfpathlineto{\pgfqpoint{4.689598in}{1.855804in}}%
\pgfpathlineto{\pgfqpoint{4.676124in}{1.858112in}}%
\pgfpathlineto{\pgfqpoint{4.662658in}{1.860444in}}%
\pgfpathlineto{\pgfqpoint{4.670137in}{1.869717in}}%
\pgfpathlineto{\pgfqpoint{4.677611in}{1.878994in}}%
\pgfpathlineto{\pgfqpoint{4.685081in}{1.888273in}}%
\pgfpathlineto{\pgfqpoint{4.692545in}{1.897548in}}%
\pgfpathclose%
\pgfusepath{fill}%
\end{pgfscope}%
\begin{pgfscope}%
\pgfpathrectangle{\pgfqpoint{1.254980in}{0.150000in}}{\pgfqpoint{5.490039in}{5.490039in}}%
\pgfusepath{clip}%
\pgfsetbuttcap%
\pgfsetroundjoin%
\definecolor{currentfill}{rgb}{0.257322,0.256130,0.526563}%
\pgfsetfillcolor{currentfill}%
\pgfsetfillopacity{0.700000}%
\pgfsetlinewidth{0.000000pt}%
\definecolor{currentstroke}{rgb}{0.000000,0.000000,0.000000}%
\pgfsetstrokecolor{currentstroke}%
\pgfsetdash{}{0pt}%
\pgfpathmoveto{\pgfqpoint{2.752089in}{2.277338in}}%
\pgfpathlineto{\pgfqpoint{2.765197in}{2.268925in}}%
\pgfpathlineto{\pgfqpoint{2.778308in}{2.260551in}}%
\pgfpathlineto{\pgfqpoint{2.791421in}{2.252217in}}%
\pgfpathlineto{\pgfqpoint{2.804537in}{2.243921in}}%
\pgfpathlineto{\pgfqpoint{2.796047in}{2.249618in}}%
\pgfpathlineto{\pgfqpoint{2.787537in}{2.255708in}}%
\pgfpathlineto{\pgfqpoint{2.779006in}{2.262197in}}%
\pgfpathlineto{\pgfqpoint{2.770453in}{2.269095in}}%
\pgfpathlineto{\pgfqpoint{2.757299in}{2.277711in}}%
\pgfpathlineto{\pgfqpoint{2.744149in}{2.286365in}}%
\pgfpathlineto{\pgfqpoint{2.731000in}{2.295059in}}%
\pgfpathlineto{\pgfqpoint{2.717854in}{2.303791in}}%
\pgfpathlineto{\pgfqpoint{2.726446in}{2.296568in}}%
\pgfpathlineto{\pgfqpoint{2.735015in}{2.289757in}}%
\pgfpathlineto{\pgfqpoint{2.743562in}{2.283350in}}%
\pgfpathlineto{\pgfqpoint{2.752089in}{2.277338in}}%
\pgfpathclose%
\pgfusepath{fill}%
\end{pgfscope}%
\begin{pgfscope}%
\pgfpathrectangle{\pgfqpoint{1.254980in}{0.150000in}}{\pgfqpoint{5.490039in}{5.490039in}}%
\pgfusepath{clip}%
\pgfsetbuttcap%
\pgfsetroundjoin%
\definecolor{currentfill}{rgb}{0.269944,0.014625,0.341379}%
\pgfsetfillcolor{currentfill}%
\pgfsetfillopacity{0.700000}%
\pgfsetlinewidth{0.000000pt}%
\definecolor{currentstroke}{rgb}{0.000000,0.000000,0.000000}%
\pgfsetstrokecolor{currentstroke}%
\pgfsetdash{}{0pt}%
\pgfpathmoveto{\pgfqpoint{3.892826in}{1.826466in}}%
\pgfpathlineto{\pgfqpoint{3.906082in}{1.821908in}}%
\pgfpathlineto{\pgfqpoint{3.919343in}{1.817376in}}%
\pgfpathlineto{\pgfqpoint{3.932609in}{1.812870in}}%
\pgfpathlineto{\pgfqpoint{3.945882in}{1.808390in}}%
\pgfpathlineto{\pgfqpoint{3.938144in}{1.802893in}}%
\pgfpathlineto{\pgfqpoint{3.930399in}{1.797553in}}%
\pgfpathlineto{\pgfqpoint{3.922648in}{1.792376in}}%
\pgfpathlineto{\pgfqpoint{3.914889in}{1.787368in}}%
\pgfpathlineto{\pgfqpoint{3.901600in}{1.792081in}}%
\pgfpathlineto{\pgfqpoint{3.888317in}{1.796821in}}%
\pgfpathlineto{\pgfqpoint{3.875039in}{1.801586in}}%
\pgfpathlineto{\pgfqpoint{3.861766in}{1.806378in}}%
\pgfpathlineto{\pgfqpoint{3.869542in}{1.811147in}}%
\pgfpathlineto{\pgfqpoint{3.877311in}{1.816089in}}%
\pgfpathlineto{\pgfqpoint{3.885072in}{1.821197in}}%
\pgfpathlineto{\pgfqpoint{3.892826in}{1.826466in}}%
\pgfpathclose%
\pgfusepath{fill}%
\end{pgfscope}%
\begin{pgfscope}%
\pgfpathrectangle{\pgfqpoint{1.254980in}{0.150000in}}{\pgfqpoint{5.490039in}{5.490039in}}%
\pgfusepath{clip}%
\pgfsetbuttcap%
\pgfsetroundjoin%
\definecolor{currentfill}{rgb}{0.280894,0.078907,0.402329}%
\pgfsetfillcolor{currentfill}%
\pgfsetfillopacity{0.700000}%
\pgfsetlinewidth{0.000000pt}%
\definecolor{currentstroke}{rgb}{0.000000,0.000000,0.000000}%
\pgfsetstrokecolor{currentstroke}%
\pgfsetdash{}{0pt}%
\pgfpathmoveto{\pgfqpoint{3.428614in}{1.934464in}}%
\pgfpathlineto{\pgfqpoint{3.441788in}{1.928397in}}%
\pgfpathlineto{\pgfqpoint{3.454967in}{1.922360in}}%
\pgfpathlineto{\pgfqpoint{3.468150in}{1.916351in}}%
\pgfpathlineto{\pgfqpoint{3.481338in}{1.910371in}}%
\pgfpathlineto{\pgfqpoint{3.473357in}{1.909049in}}%
\pgfpathlineto{\pgfqpoint{3.465365in}{1.907984in}}%
\pgfpathlineto{\pgfqpoint{3.457361in}{1.907185in}}%
\pgfpathlineto{\pgfqpoint{3.449346in}{1.906658in}}%
\pgfpathlineto{\pgfqpoint{3.436134in}{1.912911in}}%
\pgfpathlineto{\pgfqpoint{3.422927in}{1.919193in}}%
\pgfpathlineto{\pgfqpoint{3.409724in}{1.925505in}}%
\pgfpathlineto{\pgfqpoint{3.396525in}{1.931845in}}%
\pgfpathlineto{\pgfqpoint{3.404565in}{1.932093in}}%
\pgfpathlineto{\pgfqpoint{3.412593in}{1.932617in}}%
\pgfpathlineto{\pgfqpoint{3.420609in}{1.933410in}}%
\pgfpathlineto{\pgfqpoint{3.428614in}{1.934464in}}%
\pgfpathclose%
\pgfusepath{fill}%
\end{pgfscope}%
\begin{pgfscope}%
\pgfpathrectangle{\pgfqpoint{1.254980in}{0.150000in}}{\pgfqpoint{5.490039in}{5.490039in}}%
\pgfusepath{clip}%
\pgfsetbuttcap%
\pgfsetroundjoin%
\definecolor{currentfill}{rgb}{0.283229,0.120777,0.440584}%
\pgfsetfillcolor{currentfill}%
\pgfsetfillopacity{0.700000}%
\pgfsetlinewidth{0.000000pt}%
\definecolor{currentstroke}{rgb}{0.000000,0.000000,0.000000}%
\pgfsetstrokecolor{currentstroke}%
\pgfsetdash{}{0pt}%
\pgfpathmoveto{\pgfqpoint{3.238465in}{2.010259in}}%
\pgfpathlineto{\pgfqpoint{3.251614in}{2.003557in}}%
\pgfpathlineto{\pgfqpoint{3.264767in}{1.996886in}}%
\pgfpathlineto{\pgfqpoint{3.277924in}{1.990245in}}%
\pgfpathlineto{\pgfqpoint{3.291086in}{1.983635in}}%
\pgfpathlineto{\pgfqpoint{3.282980in}{1.984235in}}%
\pgfpathlineto{\pgfqpoint{3.274862in}{1.985132in}}%
\pgfpathlineto{\pgfqpoint{3.266729in}{1.986335in}}%
\pgfpathlineto{\pgfqpoint{3.258583in}{1.987850in}}%
\pgfpathlineto{\pgfqpoint{3.245394in}{1.994748in}}%
\pgfpathlineto{\pgfqpoint{3.232209in}{2.001676in}}%
\pgfpathlineto{\pgfqpoint{3.219027in}{2.008635in}}%
\pgfpathlineto{\pgfqpoint{3.205850in}{2.015626in}}%
\pgfpathlineto{\pgfqpoint{3.214025in}{2.013817in}}%
\pgfpathlineto{\pgfqpoint{3.222186in}{2.012325in}}%
\pgfpathlineto{\pgfqpoint{3.230332in}{2.011142in}}%
\pgfpathlineto{\pgfqpoint{3.238465in}{2.010259in}}%
\pgfpathclose%
\pgfusepath{fill}%
\end{pgfscope}%
\begin{pgfscope}%
\pgfpathrectangle{\pgfqpoint{1.254980in}{0.150000in}}{\pgfqpoint{5.490039in}{5.490039in}}%
\pgfusepath{clip}%
\pgfsetbuttcap%
\pgfsetroundjoin%
\definecolor{currentfill}{rgb}{0.268510,0.009605,0.335427}%
\pgfsetfillcolor{currentfill}%
\pgfsetfillopacity{0.700000}%
\pgfsetlinewidth{0.000000pt}%
\definecolor{currentstroke}{rgb}{0.000000,0.000000,0.000000}%
\pgfsetstrokecolor{currentstroke}%
\pgfsetdash{}{0pt}%
\pgfpathmoveto{\pgfqpoint{4.029849in}{1.815057in}}%
\pgfpathlineto{\pgfqpoint{4.043135in}{1.810926in}}%
\pgfpathlineto{\pgfqpoint{4.056427in}{1.806820in}}%
\pgfpathlineto{\pgfqpoint{4.069725in}{1.802740in}}%
\pgfpathlineto{\pgfqpoint{4.083028in}{1.798685in}}%
\pgfpathlineto{\pgfqpoint{4.075347in}{1.792184in}}%
\pgfpathlineto{\pgfqpoint{4.067660in}{1.785811in}}%
\pgfpathlineto{\pgfqpoint{4.059967in}{1.779572in}}%
\pgfpathlineto{\pgfqpoint{4.052267in}{1.773472in}}%
\pgfpathlineto{\pgfqpoint{4.038948in}{1.777747in}}%
\pgfpathlineto{\pgfqpoint{4.025636in}{1.782049in}}%
\pgfpathlineto{\pgfqpoint{4.012329in}{1.786375in}}%
\pgfpathlineto{\pgfqpoint{3.999028in}{1.790727in}}%
\pgfpathlineto{\pgfqpoint{4.006743in}{1.796601in}}%
\pgfpathlineto{\pgfqpoint{4.014451in}{1.802618in}}%
\pgfpathlineto{\pgfqpoint{4.022153in}{1.808772in}}%
\pgfpathlineto{\pgfqpoint{4.029849in}{1.815057in}}%
\pgfpathclose%
\pgfusepath{fill}%
\end{pgfscope}%
\begin{pgfscope}%
\pgfpathrectangle{\pgfqpoint{1.254980in}{0.150000in}}{\pgfqpoint{5.490039in}{5.490039in}}%
\pgfusepath{clip}%
\pgfsetbuttcap%
\pgfsetroundjoin%
\definecolor{currentfill}{rgb}{0.272594,0.025563,0.353093}%
\pgfsetfillcolor{currentfill}%
\pgfsetfillopacity{0.700000}%
\pgfsetlinewidth{0.000000pt}%
\definecolor{currentstroke}{rgb}{0.000000,0.000000,0.000000}%
\pgfsetstrokecolor{currentstroke}%
\pgfsetdash{}{0pt}%
\pgfpathmoveto{\pgfqpoint{3.755782in}{1.845655in}}%
\pgfpathlineto{\pgfqpoint{3.769011in}{1.840652in}}%
\pgfpathlineto{\pgfqpoint{3.782246in}{1.835677in}}%
\pgfpathlineto{\pgfqpoint{3.795486in}{1.830728in}}%
\pgfpathlineto{\pgfqpoint{3.808731in}{1.825805in}}%
\pgfpathlineto{\pgfqpoint{3.800929in}{1.821456in}}%
\pgfpathlineto{\pgfqpoint{3.793120in}{1.817294in}}%
\pgfpathlineto{\pgfqpoint{3.785302in}{1.813327in}}%
\pgfpathlineto{\pgfqpoint{3.777476in}{1.809561in}}%
\pgfpathlineto{\pgfqpoint{3.764212in}{1.814730in}}%
\pgfpathlineto{\pgfqpoint{3.750953in}{1.819926in}}%
\pgfpathlineto{\pgfqpoint{3.737699in}{1.825148in}}%
\pgfpathlineto{\pgfqpoint{3.724451in}{1.830398in}}%
\pgfpathlineto{\pgfqpoint{3.732296in}{1.833912in}}%
\pgfpathlineto{\pgfqpoint{3.740133in}{1.837631in}}%
\pgfpathlineto{\pgfqpoint{3.747962in}{1.841547in}}%
\pgfpathlineto{\pgfqpoint{3.755782in}{1.845655in}}%
\pgfpathclose%
\pgfusepath{fill}%
\end{pgfscope}%
\begin{pgfscope}%
\pgfpathrectangle{\pgfqpoint{1.254980in}{0.150000in}}{\pgfqpoint{5.490039in}{5.490039in}}%
\pgfusepath{clip}%
\pgfsetbuttcap%
\pgfsetroundjoin%
\definecolor{currentfill}{rgb}{0.277134,0.185228,0.489898}%
\pgfsetfillcolor{currentfill}%
\pgfsetfillopacity{0.700000}%
\pgfsetlinewidth{0.000000pt}%
\definecolor{currentstroke}{rgb}{0.000000,0.000000,0.000000}%
\pgfsetstrokecolor{currentstroke}%
\pgfsetdash{}{0pt}%
\pgfpathmoveto{\pgfqpoint{5.469858in}{2.136445in}}%
\pgfpathlineto{\pgfqpoint{5.483563in}{2.135901in}}%
\pgfpathlineto{\pgfqpoint{5.497277in}{2.135382in}}%
\pgfpathlineto{\pgfqpoint{5.511000in}{2.134885in}}%
\pgfpathlineto{\pgfqpoint{5.524731in}{2.134413in}}%
\pgfpathlineto{\pgfqpoint{5.517569in}{2.126062in}}%
\pgfpathlineto{\pgfqpoint{5.510399in}{2.117611in}}%
\pgfpathlineto{\pgfqpoint{5.503221in}{2.109063in}}%
\pgfpathlineto{\pgfqpoint{5.496036in}{2.100416in}}%
\pgfpathlineto{\pgfqpoint{5.482294in}{2.100928in}}%
\pgfpathlineto{\pgfqpoint{5.468561in}{2.101465in}}%
\pgfpathlineto{\pgfqpoint{5.454837in}{2.102025in}}%
\pgfpathlineto{\pgfqpoint{5.441121in}{2.102608in}}%
\pgfpathlineto{\pgfqpoint{5.448317in}{2.111210in}}%
\pgfpathlineto{\pgfqpoint{5.455505in}{2.119717in}}%
\pgfpathlineto{\pgfqpoint{5.462685in}{2.128129in}}%
\pgfpathlineto{\pgfqpoint{5.469858in}{2.136445in}}%
\pgfpathclose%
\pgfusepath{fill}%
\end{pgfscope}%
\begin{pgfscope}%
\pgfpathrectangle{\pgfqpoint{1.254980in}{0.150000in}}{\pgfqpoint{5.490039in}{5.490039in}}%
\pgfusepath{clip}%
\pgfsetbuttcap%
\pgfsetroundjoin%
\definecolor{currentfill}{rgb}{0.269944,0.014625,0.341379}%
\pgfsetfillcolor{currentfill}%
\pgfsetfillopacity{0.700000}%
\pgfsetlinewidth{0.000000pt}%
\definecolor{currentstroke}{rgb}{0.000000,0.000000,0.000000}%
\pgfsetstrokecolor{currentstroke}%
\pgfsetdash{}{0pt}%
\pgfpathmoveto{\pgfqpoint{4.387836in}{1.832694in}}%
\pgfpathlineto{\pgfqpoint{4.401210in}{1.829647in}}%
\pgfpathlineto{\pgfqpoint{4.414591in}{1.826625in}}%
\pgfpathlineto{\pgfqpoint{4.427979in}{1.823627in}}%
\pgfpathlineto{\pgfqpoint{4.441374in}{1.820653in}}%
\pgfpathlineto{\pgfqpoint{4.433820in}{1.812122in}}%
\pgfpathlineto{\pgfqpoint{4.426261in}{1.803645in}}%
\pgfpathlineto{\pgfqpoint{4.418698in}{1.795225in}}%
\pgfpathlineto{\pgfqpoint{4.411129in}{1.786867in}}%
\pgfpathlineto{\pgfqpoint{4.397724in}{1.790023in}}%
\pgfpathlineto{\pgfqpoint{4.384325in}{1.793204in}}%
\pgfpathlineto{\pgfqpoint{4.370933in}{1.796410in}}%
\pgfpathlineto{\pgfqpoint{4.357547in}{1.799640in}}%
\pgfpathlineto{\pgfqpoint{4.365127in}{1.807810in}}%
\pgfpathlineto{\pgfqpoint{4.372702in}{1.816045in}}%
\pgfpathlineto{\pgfqpoint{4.380271in}{1.824342in}}%
\pgfpathlineto{\pgfqpoint{4.387836in}{1.832694in}}%
\pgfpathclose%
\pgfusepath{fill}%
\end{pgfscope}%
\begin{pgfscope}%
\pgfpathrectangle{\pgfqpoint{1.254980in}{0.150000in}}{\pgfqpoint{5.490039in}{5.490039in}}%
\pgfusepath{clip}%
\pgfsetbuttcap%
\pgfsetroundjoin%
\definecolor{currentfill}{rgb}{0.282910,0.105393,0.426902}%
\pgfsetfillcolor{currentfill}%
\pgfsetfillopacity{0.700000}%
\pgfsetlinewidth{0.000000pt}%
\definecolor{currentstroke}{rgb}{0.000000,0.000000,0.000000}%
\pgfsetstrokecolor{currentstroke}%
\pgfsetdash{}{0pt}%
\pgfpathmoveto{\pgfqpoint{4.997493in}{1.981151in}}%
\pgfpathlineto{\pgfqpoint{5.011046in}{1.979703in}}%
\pgfpathlineto{\pgfqpoint{5.024607in}{1.978279in}}%
\pgfpathlineto{\pgfqpoint{5.038176in}{1.976878in}}%
\pgfpathlineto{\pgfqpoint{5.051753in}{1.975502in}}%
\pgfpathlineto{\pgfqpoint{5.044398in}{1.965979in}}%
\pgfpathlineto{\pgfqpoint{5.037037in}{1.956404in}}%
\pgfpathlineto{\pgfqpoint{5.029670in}{1.946780in}}%
\pgfpathlineto{\pgfqpoint{5.022298in}{1.937108in}}%
\pgfpathlineto{\pgfqpoint{5.008712in}{1.938590in}}%
\pgfpathlineto{\pgfqpoint{4.995135in}{1.940097in}}%
\pgfpathlineto{\pgfqpoint{4.981565in}{1.941627in}}%
\pgfpathlineto{\pgfqpoint{4.968004in}{1.943181in}}%
\pgfpathlineto{\pgfqpoint{4.975385in}{1.952742in}}%
\pgfpathlineto{\pgfqpoint{4.982760in}{1.962258in}}%
\pgfpathlineto{\pgfqpoint{4.990129in}{1.971729in}}%
\pgfpathlineto{\pgfqpoint{4.997493in}{1.981151in}}%
\pgfpathclose%
\pgfusepath{fill}%
\end{pgfscope}%
\begin{pgfscope}%
\pgfpathrectangle{\pgfqpoint{1.254980in}{0.150000in}}{\pgfqpoint{5.490039in}{5.490039in}}%
\pgfusepath{clip}%
\pgfsetbuttcap%
\pgfsetroundjoin%
\definecolor{currentfill}{rgb}{0.216210,0.351535,0.550627}%
\pgfsetfillcolor{currentfill}%
\pgfsetfillopacity{0.700000}%
\pgfsetlinewidth{0.000000pt}%
\definecolor{currentstroke}{rgb}{0.000000,0.000000,0.000000}%
\pgfsetstrokecolor{currentstroke}%
\pgfsetdash{}{0pt}%
\pgfpathmoveto{\pgfqpoint{2.455359in}{2.487309in}}%
\pgfpathlineto{\pgfqpoint{2.468468in}{2.477706in}}%
\pgfpathlineto{\pgfqpoint{2.481577in}{2.468151in}}%
\pgfpathlineto{\pgfqpoint{2.494689in}{2.458644in}}%
\pgfpathlineto{\pgfqpoint{2.507802in}{2.449182in}}%
\pgfpathlineto{\pgfqpoint{2.499025in}{2.458151in}}%
\pgfpathlineto{\pgfqpoint{2.490222in}{2.467567in}}%
\pgfpathlineto{\pgfqpoint{2.481393in}{2.477438in}}%
\pgfpathlineto{\pgfqpoint{2.472537in}{2.487776in}}%
\pgfpathlineto{\pgfqpoint{2.459381in}{2.497575in}}%
\pgfpathlineto{\pgfqpoint{2.446226in}{2.507421in}}%
\pgfpathlineto{\pgfqpoint{2.433072in}{2.517315in}}%
\pgfpathlineto{\pgfqpoint{2.419919in}{2.527257in}}%
\pgfpathlineto{\pgfqpoint{2.428820in}{2.516575in}}%
\pgfpathlineto{\pgfqpoint{2.437693in}{2.506362in}}%
\pgfpathlineto{\pgfqpoint{2.446539in}{2.496610in}}%
\pgfpathlineto{\pgfqpoint{2.455359in}{2.487309in}}%
\pgfpathclose%
\pgfusepath{fill}%
\end{pgfscope}%
\begin{pgfscope}%
\pgfpathrectangle{\pgfqpoint{1.254980in}{0.150000in}}{\pgfqpoint{5.490039in}{5.490039in}}%
\pgfusepath{clip}%
\pgfsetbuttcap%
\pgfsetroundjoin%
\definecolor{currentfill}{rgb}{0.268510,0.009605,0.335427}%
\pgfsetfillcolor{currentfill}%
\pgfsetfillopacity{0.700000}%
\pgfsetlinewidth{0.000000pt}%
\definecolor{currentstroke}{rgb}{0.000000,0.000000,0.000000}%
\pgfsetstrokecolor{currentstroke}%
\pgfsetdash{}{0pt}%
\pgfpathmoveto{\pgfqpoint{4.166912in}{1.810726in}}%
\pgfpathlineto{\pgfqpoint{4.180233in}{1.807004in}}%
\pgfpathlineto{\pgfqpoint{4.193560in}{1.803308in}}%
\pgfpathlineto{\pgfqpoint{4.206893in}{1.799636in}}%
\pgfpathlineto{\pgfqpoint{4.220232in}{1.795989in}}%
\pgfpathlineto{\pgfqpoint{4.212602in}{1.788623in}}%
\pgfpathlineto{\pgfqpoint{4.204966in}{1.781357in}}%
\pgfpathlineto{\pgfqpoint{4.197324in}{1.774196in}}%
\pgfpathlineto{\pgfqpoint{4.189676in}{1.767147in}}%
\pgfpathlineto{\pgfqpoint{4.176324in}{1.771002in}}%
\pgfpathlineto{\pgfqpoint{4.162978in}{1.774882in}}%
\pgfpathlineto{\pgfqpoint{4.149638in}{1.778786in}}%
\pgfpathlineto{\pgfqpoint{4.136304in}{1.782716in}}%
\pgfpathlineto{\pgfqpoint{4.143965in}{1.789552in}}%
\pgfpathlineto{\pgfqpoint{4.151620in}{1.796503in}}%
\pgfpathlineto{\pgfqpoint{4.159269in}{1.803562in}}%
\pgfpathlineto{\pgfqpoint{4.166912in}{1.810726in}}%
\pgfpathclose%
\pgfusepath{fill}%
\end{pgfscope}%
\begin{pgfscope}%
\pgfpathrectangle{\pgfqpoint{1.254980in}{0.150000in}}{\pgfqpoint{5.490039in}{5.490039in}}%
\pgfusepath{clip}%
\pgfsetbuttcap%
\pgfsetroundjoin%
\definecolor{currentfill}{rgb}{0.274952,0.037752,0.364543}%
\pgfsetfillcolor{currentfill}%
\pgfsetfillopacity{0.700000}%
\pgfsetlinewidth{0.000000pt}%
\definecolor{currentstroke}{rgb}{0.000000,0.000000,0.000000}%
\pgfsetstrokecolor{currentstroke}%
\pgfsetdash{}{0pt}%
\pgfpathmoveto{\pgfqpoint{4.608865in}{1.870013in}}%
\pgfpathlineto{\pgfqpoint{4.622302in}{1.867585in}}%
\pgfpathlineto{\pgfqpoint{4.635747in}{1.865180in}}%
\pgfpathlineto{\pgfqpoint{4.649199in}{1.862800in}}%
\pgfpathlineto{\pgfqpoint{4.662658in}{1.860444in}}%
\pgfpathlineto{\pgfqpoint{4.655174in}{1.851179in}}%
\pgfpathlineto{\pgfqpoint{4.647685in}{1.841925in}}%
\pgfpathlineto{\pgfqpoint{4.640191in}{1.832686in}}%
\pgfpathlineto{\pgfqpoint{4.632692in}{1.823467in}}%
\pgfpathlineto{\pgfqpoint{4.619223in}{1.825980in}}%
\pgfpathlineto{\pgfqpoint{4.605762in}{1.828518in}}%
\pgfpathlineto{\pgfqpoint{4.592308in}{1.831080in}}%
\pgfpathlineto{\pgfqpoint{4.578861in}{1.833665in}}%
\pgfpathlineto{\pgfqpoint{4.586369in}{1.842723in}}%
\pgfpathlineto{\pgfqpoint{4.593873in}{1.851802in}}%
\pgfpathlineto{\pgfqpoint{4.601371in}{1.860900in}}%
\pgfpathlineto{\pgfqpoint{4.608865in}{1.870013in}}%
\pgfpathclose%
\pgfusepath{fill}%
\end{pgfscope}%
\begin{pgfscope}%
\pgfpathrectangle{\pgfqpoint{1.254980in}{0.150000in}}{\pgfqpoint{5.490039in}{5.490039in}}%
\pgfusepath{clip}%
\pgfsetbuttcap%
\pgfsetroundjoin%
\definecolor{currentfill}{rgb}{0.276022,0.044167,0.370164}%
\pgfsetfillcolor{currentfill}%
\pgfsetfillopacity{0.700000}%
\pgfsetlinewidth{0.000000pt}%
\definecolor{currentstroke}{rgb}{0.000000,0.000000,0.000000}%
\pgfsetstrokecolor{currentstroke}%
\pgfsetdash{}{0pt}%
\pgfpathmoveto{\pgfqpoint{3.618645in}{1.873364in}}%
\pgfpathlineto{\pgfqpoint{3.631853in}{1.867897in}}%
\pgfpathlineto{\pgfqpoint{3.645066in}{1.862458in}}%
\pgfpathlineto{\pgfqpoint{3.658285in}{1.857047in}}%
\pgfpathlineto{\pgfqpoint{3.671508in}{1.851663in}}%
\pgfpathlineto{\pgfqpoint{3.663633in}{1.848613in}}%
\pgfpathlineto{\pgfqpoint{3.655750in}{1.845784in}}%
\pgfpathlineto{\pgfqpoint{3.647857in}{1.843182in}}%
\pgfpathlineto{\pgfqpoint{3.639954in}{1.840814in}}%
\pgfpathlineto{\pgfqpoint{3.626710in}{1.846458in}}%
\pgfpathlineto{\pgfqpoint{3.613470in}{1.852129in}}%
\pgfpathlineto{\pgfqpoint{3.600236in}{1.857828in}}%
\pgfpathlineto{\pgfqpoint{3.587006in}{1.863554in}}%
\pgfpathlineto{\pgfqpoint{3.594930in}{1.865657in}}%
\pgfpathlineto{\pgfqpoint{3.602845in}{1.867998in}}%
\pgfpathlineto{\pgfqpoint{3.610750in}{1.870569in}}%
\pgfpathlineto{\pgfqpoint{3.618645in}{1.873364in}}%
\pgfpathclose%
\pgfusepath{fill}%
\end{pgfscope}%
\begin{pgfscope}%
\pgfpathrectangle{\pgfqpoint{1.254980in}{0.150000in}}{\pgfqpoint{5.490039in}{5.490039in}}%
\pgfusepath{clip}%
\pgfsetbuttcap%
\pgfsetroundjoin%
\definecolor{currentfill}{rgb}{0.279574,0.170599,0.479997}%
\pgfsetfillcolor{currentfill}%
\pgfsetfillopacity{0.700000}%
\pgfsetlinewidth{0.000000pt}%
\definecolor{currentstroke}{rgb}{0.000000,0.000000,0.000000}%
\pgfsetstrokecolor{currentstroke}%
\pgfsetdash{}{0pt}%
\pgfpathmoveto{\pgfqpoint{5.386345in}{2.105181in}}%
\pgfpathlineto{\pgfqpoint{5.400026in}{2.104502in}}%
\pgfpathlineto{\pgfqpoint{5.413716in}{2.103847in}}%
\pgfpathlineto{\pgfqpoint{5.427414in}{2.103216in}}%
\pgfpathlineto{\pgfqpoint{5.441121in}{2.102608in}}%
\pgfpathlineto{\pgfqpoint{5.433918in}{2.093913in}}%
\pgfpathlineto{\pgfqpoint{5.426708in}{2.085125in}}%
\pgfpathlineto{\pgfqpoint{5.419490in}{2.076244in}}%
\pgfpathlineto{\pgfqpoint{5.412264in}{2.067272in}}%
\pgfpathlineto{\pgfqpoint{5.398548in}{2.067933in}}%
\pgfpathlineto{\pgfqpoint{5.384839in}{2.068618in}}%
\pgfpathlineto{\pgfqpoint{5.371140in}{2.069327in}}%
\pgfpathlineto{\pgfqpoint{5.357449in}{2.070059in}}%
\pgfpathlineto{\pgfqpoint{5.364684in}{2.078973in}}%
\pgfpathlineto{\pgfqpoint{5.371911in}{2.087798in}}%
\pgfpathlineto{\pgfqpoint{5.379132in}{2.096534in}}%
\pgfpathlineto{\pgfqpoint{5.386345in}{2.105181in}}%
\pgfpathclose%
\pgfusepath{fill}%
\end{pgfscope}%
\begin{pgfscope}%
\pgfpathrectangle{\pgfqpoint{1.254980in}{0.150000in}}{\pgfqpoint{5.490039in}{5.490039in}}%
\pgfusepath{clip}%
\pgfsetbuttcap%
\pgfsetroundjoin%
\definecolor{currentfill}{rgb}{0.278826,0.175490,0.483397}%
\pgfsetfillcolor{currentfill}%
\pgfsetfillopacity{0.700000}%
\pgfsetlinewidth{0.000000pt}%
\definecolor{currentstroke}{rgb}{0.000000,0.000000,0.000000}%
\pgfsetstrokecolor{currentstroke}%
\pgfsetdash{}{0pt}%
\pgfpathmoveto{\pgfqpoint{3.048007in}{2.102013in}}%
\pgfpathlineto{\pgfqpoint{3.061141in}{2.094633in}}%
\pgfpathlineto{\pgfqpoint{3.074278in}{2.087286in}}%
\pgfpathlineto{\pgfqpoint{3.087419in}{2.079973in}}%
\pgfpathlineto{\pgfqpoint{3.100564in}{2.072694in}}%
\pgfpathlineto{\pgfqpoint{3.092314in}{2.075421in}}%
\pgfpathlineto{\pgfqpoint{3.084047in}{2.078487in}}%
\pgfpathlineto{\pgfqpoint{3.075764in}{2.081901in}}%
\pgfpathlineto{\pgfqpoint{3.067464in}{2.085672in}}%
\pgfpathlineto{\pgfqpoint{3.054288in}{2.093254in}}%
\pgfpathlineto{\pgfqpoint{3.041115in}{2.100870in}}%
\pgfpathlineto{\pgfqpoint{3.027946in}{2.108519in}}%
\pgfpathlineto{\pgfqpoint{3.014780in}{2.116202in}}%
\pgfpathlineto{\pgfqpoint{3.023112in}{2.112123in}}%
\pgfpathlineto{\pgfqpoint{3.031427in}{2.108404in}}%
\pgfpathlineto{\pgfqpoint{3.039726in}{2.105037in}}%
\pgfpathlineto{\pgfqpoint{3.048007in}{2.102013in}}%
\pgfpathclose%
\pgfusepath{fill}%
\end{pgfscope}%
\begin{pgfscope}%
\pgfpathrectangle{\pgfqpoint{1.254980in}{0.150000in}}{\pgfqpoint{5.490039in}{5.490039in}}%
\pgfusepath{clip}%
\pgfsetbuttcap%
\pgfsetroundjoin%
\definecolor{currentfill}{rgb}{0.281924,0.089666,0.412415}%
\pgfsetfillcolor{currentfill}%
\pgfsetfillopacity{0.700000}%
\pgfsetlinewidth{0.000000pt}%
\definecolor{currentstroke}{rgb}{0.000000,0.000000,0.000000}%
\pgfsetstrokecolor{currentstroke}%
\pgfsetdash{}{0pt}%
\pgfpathmoveto{\pgfqpoint{4.913835in}{1.949635in}}%
\pgfpathlineto{\pgfqpoint{4.927366in}{1.947985in}}%
\pgfpathlineto{\pgfqpoint{4.940904in}{1.946360in}}%
\pgfpathlineto{\pgfqpoint{4.954450in}{1.944758in}}%
\pgfpathlineto{\pgfqpoint{4.968004in}{1.943181in}}%
\pgfpathlineto{\pgfqpoint{4.960617in}{1.933578in}}%
\pgfpathlineto{\pgfqpoint{4.953225in}{1.923937in}}%
\pgfpathlineto{\pgfqpoint{4.945827in}{1.914259in}}%
\pgfpathlineto{\pgfqpoint{4.938424in}{1.904547in}}%
\pgfpathlineto{\pgfqpoint{4.924862in}{1.906244in}}%
\pgfpathlineto{\pgfqpoint{4.911308in}{1.907965in}}%
\pgfpathlineto{\pgfqpoint{4.897761in}{1.909709in}}%
\pgfpathlineto{\pgfqpoint{4.884222in}{1.911478in}}%
\pgfpathlineto{\pgfqpoint{4.891634in}{1.921065in}}%
\pgfpathlineto{\pgfqpoint{4.899040in}{1.930622in}}%
\pgfpathlineto{\pgfqpoint{4.906440in}{1.940146in}}%
\pgfpathlineto{\pgfqpoint{4.913835in}{1.949635in}}%
\pgfpathclose%
\pgfusepath{fill}%
\end{pgfscope}%
\begin{pgfscope}%
\pgfpathrectangle{\pgfqpoint{1.254980in}{0.150000in}}{\pgfqpoint{5.490039in}{5.490039in}}%
\pgfusepath{clip}%
\pgfsetbuttcap%
\pgfsetroundjoin%
\definecolor{currentfill}{rgb}{0.260571,0.246922,0.522828}%
\pgfsetfillcolor{currentfill}%
\pgfsetfillopacity{0.700000}%
\pgfsetlinewidth{0.000000pt}%
\definecolor{currentstroke}{rgb}{0.000000,0.000000,0.000000}%
\pgfsetstrokecolor{currentstroke}%
\pgfsetdash{}{0pt}%
\pgfpathmoveto{\pgfqpoint{2.804537in}{2.243921in}}%
\pgfpathlineto{\pgfqpoint{2.817656in}{2.235663in}}%
\pgfpathlineto{\pgfqpoint{2.830777in}{2.227443in}}%
\pgfpathlineto{\pgfqpoint{2.843901in}{2.219262in}}%
\pgfpathlineto{\pgfqpoint{2.857028in}{2.211117in}}%
\pgfpathlineto{\pgfqpoint{2.848574in}{2.216502in}}%
\pgfpathlineto{\pgfqpoint{2.840100in}{2.222273in}}%
\pgfpathlineto{\pgfqpoint{2.831606in}{2.228442in}}%
\pgfpathlineto{\pgfqpoint{2.823091in}{2.235016in}}%
\pgfpathlineto{\pgfqpoint{2.809927in}{2.243479in}}%
\pgfpathlineto{\pgfqpoint{2.796766in}{2.251980in}}%
\pgfpathlineto{\pgfqpoint{2.783608in}{2.260519in}}%
\pgfpathlineto{\pgfqpoint{2.770453in}{2.269095in}}%
\pgfpathlineto{\pgfqpoint{2.779006in}{2.262197in}}%
\pgfpathlineto{\pgfqpoint{2.787537in}{2.255708in}}%
\pgfpathlineto{\pgfqpoint{2.796047in}{2.249618in}}%
\pgfpathlineto{\pgfqpoint{2.804537in}{2.243921in}}%
\pgfpathclose%
\pgfusepath{fill}%
\end{pgfscope}%
\begin{pgfscope}%
\pgfpathrectangle{\pgfqpoint{1.254980in}{0.150000in}}{\pgfqpoint{5.490039in}{5.490039in}}%
\pgfusepath{clip}%
\pgfsetbuttcap%
\pgfsetroundjoin%
\definecolor{currentfill}{rgb}{0.269308,0.218818,0.509577}%
\pgfsetfillcolor{currentfill}%
\pgfsetfillopacity{0.700000}%
\pgfsetlinewidth{0.000000pt}%
\definecolor{currentstroke}{rgb}{0.000000,0.000000,0.000000}%
\pgfsetstrokecolor{currentstroke}%
\pgfsetdash{}{0pt}%
\pgfpathmoveto{\pgfqpoint{5.691726in}{2.195029in}}%
\pgfpathlineto{\pgfqpoint{5.705514in}{2.194811in}}%
\pgfpathlineto{\pgfqpoint{5.719312in}{2.194616in}}%
\pgfpathlineto{\pgfqpoint{5.733118in}{2.194445in}}%
\pgfpathlineto{\pgfqpoint{5.726054in}{2.186870in}}%
\pgfpathlineto{\pgfqpoint{5.718981in}{2.179186in}}%
\pgfpathlineto{\pgfqpoint{5.711899in}{2.171392in}}%
\pgfpathlineto{\pgfqpoint{5.704809in}{2.163490in}}%
\pgfpathlineto{\pgfqpoint{5.690991in}{2.163674in}}%
\pgfpathlineto{\pgfqpoint{5.677181in}{2.163882in}}%
\pgfpathlineto{\pgfqpoint{5.663381in}{2.164113in}}%
\pgfpathlineto{\pgfqpoint{5.670480in}{2.172002in}}%
\pgfpathlineto{\pgfqpoint{5.677570in}{2.179784in}}%
\pgfpathlineto{\pgfqpoint{5.684653in}{2.187460in}}%
\pgfpathlineto{\pgfqpoint{5.691726in}{2.195029in}}%
\pgfpathclose%
\pgfusepath{fill}%
\end{pgfscope}%
\begin{pgfscope}%
\pgfpathrectangle{\pgfqpoint{1.254980in}{0.150000in}}{\pgfqpoint{5.490039in}{5.490039in}}%
\pgfusepath{clip}%
\pgfsetbuttcap%
\pgfsetroundjoin%
\definecolor{currentfill}{rgb}{0.281412,0.155834,0.469201}%
\pgfsetfillcolor{currentfill}%
\pgfsetfillopacity{0.700000}%
\pgfsetlinewidth{0.000000pt}%
\definecolor{currentstroke}{rgb}{0.000000,0.000000,0.000000}%
\pgfsetstrokecolor{currentstroke}%
\pgfsetdash{}{0pt}%
\pgfpathmoveto{\pgfqpoint{5.302770in}{2.073226in}}%
\pgfpathlineto{\pgfqpoint{5.316427in}{2.072399in}}%
\pgfpathlineto{\pgfqpoint{5.330092in}{2.071595in}}%
\pgfpathlineto{\pgfqpoint{5.343766in}{2.070815in}}%
\pgfpathlineto{\pgfqpoint{5.357449in}{2.070059in}}%
\pgfpathlineto{\pgfqpoint{5.350207in}{2.061059in}}%
\pgfpathlineto{\pgfqpoint{5.342958in}{2.051973in}}%
\pgfpathlineto{\pgfqpoint{5.335702in}{2.042803in}}%
\pgfpathlineto{\pgfqpoint{5.328439in}{2.033549in}}%
\pgfpathlineto{\pgfqpoint{5.314748in}{2.034372in}}%
\pgfpathlineto{\pgfqpoint{5.301064in}{2.035219in}}%
\pgfpathlineto{\pgfqpoint{5.287390in}{2.036089in}}%
\pgfpathlineto{\pgfqpoint{5.273723in}{2.036983in}}%
\pgfpathlineto{\pgfqpoint{5.280995in}{2.046165in}}%
\pgfpathlineto{\pgfqpoint{5.288260in}{2.055267in}}%
\pgfpathlineto{\pgfqpoint{5.295519in}{2.064288in}}%
\pgfpathlineto{\pgfqpoint{5.302770in}{2.073226in}}%
\pgfpathclose%
\pgfusepath{fill}%
\end{pgfscope}%
\begin{pgfscope}%
\pgfpathrectangle{\pgfqpoint{1.254980in}{0.150000in}}{\pgfqpoint{5.490039in}{5.490039in}}%
\pgfusepath{clip}%
\pgfsetbuttcap%
\pgfsetroundjoin%
\definecolor{currentfill}{rgb}{0.268510,0.009605,0.335427}%
\pgfsetfillcolor{currentfill}%
\pgfsetfillopacity{0.700000}%
\pgfsetlinewidth{0.000000pt}%
\definecolor{currentstroke}{rgb}{0.000000,0.000000,0.000000}%
\pgfsetstrokecolor{currentstroke}%
\pgfsetdash{}{0pt}%
\pgfpathmoveto{\pgfqpoint{4.304071in}{1.812803in}}%
\pgfpathlineto{\pgfqpoint{4.317431in}{1.809476in}}%
\pgfpathlineto{\pgfqpoint{4.330796in}{1.806172in}}%
\pgfpathlineto{\pgfqpoint{4.344169in}{1.802894in}}%
\pgfpathlineto{\pgfqpoint{4.357547in}{1.799640in}}%
\pgfpathlineto{\pgfqpoint{4.349962in}{1.791539in}}%
\pgfpathlineto{\pgfqpoint{4.342372in}{1.783513in}}%
\pgfpathlineto{\pgfqpoint{4.334777in}{1.775566in}}%
\pgfpathlineto{\pgfqpoint{4.327176in}{1.767704in}}%
\pgfpathlineto{\pgfqpoint{4.313786in}{1.771153in}}%
\pgfpathlineto{\pgfqpoint{4.300402in}{1.774628in}}%
\pgfpathlineto{\pgfqpoint{4.287024in}{1.778126in}}%
\pgfpathlineto{\pgfqpoint{4.273653in}{1.781650in}}%
\pgfpathlineto{\pgfqpoint{4.281266in}{1.789312in}}%
\pgfpathlineto{\pgfqpoint{4.288873in}{1.797062in}}%
\pgfpathlineto{\pgfqpoint{4.296475in}{1.804894in}}%
\pgfpathlineto{\pgfqpoint{4.304071in}{1.812803in}}%
\pgfpathclose%
\pgfusepath{fill}%
\end{pgfscope}%
\begin{pgfscope}%
\pgfpathrectangle{\pgfqpoint{1.254980in}{0.150000in}}{\pgfqpoint{5.490039in}{5.490039in}}%
\pgfusepath{clip}%
\pgfsetbuttcap%
\pgfsetroundjoin%
\definecolor{currentfill}{rgb}{0.221989,0.339161,0.548752}%
\pgfsetfillcolor{currentfill}%
\pgfsetfillopacity{0.700000}%
\pgfsetlinewidth{0.000000pt}%
\definecolor{currentstroke}{rgb}{0.000000,0.000000,0.000000}%
\pgfsetstrokecolor{currentstroke}%
\pgfsetdash{}{0pt}%
\pgfpathmoveto{\pgfqpoint{2.507802in}{2.449182in}}%
\pgfpathlineto{\pgfqpoint{2.520916in}{2.439767in}}%
\pgfpathlineto{\pgfqpoint{2.534032in}{2.430398in}}%
\pgfpathlineto{\pgfqpoint{2.547150in}{2.421074in}}%
\pgfpathlineto{\pgfqpoint{2.560270in}{2.411795in}}%
\pgfpathlineto{\pgfqpoint{2.551534in}{2.420432in}}%
\pgfpathlineto{\pgfqpoint{2.542774in}{2.429511in}}%
\pgfpathlineto{\pgfqpoint{2.533988in}{2.439043in}}%
\pgfpathlineto{\pgfqpoint{2.525176in}{2.449037in}}%
\pgfpathlineto{\pgfqpoint{2.512014in}{2.458653in}}%
\pgfpathlineto{\pgfqpoint{2.498854in}{2.468315in}}%
\pgfpathlineto{\pgfqpoint{2.485694in}{2.478022in}}%
\pgfpathlineto{\pgfqpoint{2.472537in}{2.487776in}}%
\pgfpathlineto{\pgfqpoint{2.481393in}{2.477438in}}%
\pgfpathlineto{\pgfqpoint{2.490222in}{2.467567in}}%
\pgfpathlineto{\pgfqpoint{2.499025in}{2.458151in}}%
\pgfpathlineto{\pgfqpoint{2.507802in}{2.449182in}}%
\pgfpathclose%
\pgfusepath{fill}%
\end{pgfscope}%
\begin{pgfscope}%
\pgfpathrectangle{\pgfqpoint{1.254980in}{0.150000in}}{\pgfqpoint{5.490039in}{5.490039in}}%
\pgfusepath{clip}%
\pgfsetbuttcap%
\pgfsetroundjoin%
\definecolor{currentfill}{rgb}{0.272594,0.025563,0.353093}%
\pgfsetfillcolor{currentfill}%
\pgfsetfillopacity{0.700000}%
\pgfsetlinewidth{0.000000pt}%
\definecolor{currentstroke}{rgb}{0.000000,0.000000,0.000000}%
\pgfsetstrokecolor{currentstroke}%
\pgfsetdash{}{0pt}%
\pgfpathmoveto{\pgfqpoint{4.525143in}{1.844250in}}%
\pgfpathlineto{\pgfqpoint{4.538562in}{1.841567in}}%
\pgfpathlineto{\pgfqpoint{4.551988in}{1.838909in}}%
\pgfpathlineto{\pgfqpoint{4.565421in}{1.836275in}}%
\pgfpathlineto{\pgfqpoint{4.578861in}{1.833665in}}%
\pgfpathlineto{\pgfqpoint{4.571347in}{1.824634in}}%
\pgfpathlineto{\pgfqpoint{4.563829in}{1.815632in}}%
\pgfpathlineto{\pgfqpoint{4.556306in}{1.806665in}}%
\pgfpathlineto{\pgfqpoint{4.548778in}{1.797735in}}%
\pgfpathlineto{\pgfqpoint{4.535328in}{1.800515in}}%
\pgfpathlineto{\pgfqpoint{4.521885in}{1.803320in}}%
\pgfpathlineto{\pgfqpoint{4.508449in}{1.806148in}}%
\pgfpathlineto{\pgfqpoint{4.495020in}{1.809001in}}%
\pgfpathlineto{\pgfqpoint{4.502559in}{1.817755in}}%
\pgfpathlineto{\pgfqpoint{4.510092in}{1.826551in}}%
\pgfpathlineto{\pgfqpoint{4.517620in}{1.835384in}}%
\pgfpathlineto{\pgfqpoint{4.525143in}{1.844250in}}%
\pgfpathclose%
\pgfusepath{fill}%
\end{pgfscope}%
\begin{pgfscope}%
\pgfpathrectangle{\pgfqpoint{1.254980in}{0.150000in}}{\pgfqpoint{5.490039in}{5.490039in}}%
\pgfusepath{clip}%
\pgfsetbuttcap%
\pgfsetroundjoin%
\definecolor{currentfill}{rgb}{0.280267,0.073417,0.397163}%
\pgfsetfillcolor{currentfill}%
\pgfsetfillopacity{0.700000}%
\pgfsetlinewidth{0.000000pt}%
\definecolor{currentstroke}{rgb}{0.000000,0.000000,0.000000}%
\pgfsetstrokecolor{currentstroke}%
\pgfsetdash{}{0pt}%
\pgfpathmoveto{\pgfqpoint{4.830143in}{1.918790in}}%
\pgfpathlineto{\pgfqpoint{4.843652in}{1.916926in}}%
\pgfpathlineto{\pgfqpoint{4.857167in}{1.915086in}}%
\pgfpathlineto{\pgfqpoint{4.870691in}{1.913270in}}%
\pgfpathlineto{\pgfqpoint{4.884222in}{1.911478in}}%
\pgfpathlineto{\pgfqpoint{4.876805in}{1.901862in}}%
\pgfpathlineto{\pgfqpoint{4.869383in}{1.892223in}}%
\pgfpathlineto{\pgfqpoint{4.861956in}{1.882561in}}%
\pgfpathlineto{\pgfqpoint{4.854523in}{1.872881in}}%
\pgfpathlineto{\pgfqpoint{4.840984in}{1.874805in}}%
\pgfpathlineto{\pgfqpoint{4.827452in}{1.876753in}}%
\pgfpathlineto{\pgfqpoint{4.813927in}{1.878725in}}%
\pgfpathlineto{\pgfqpoint{4.800410in}{1.880721in}}%
\pgfpathlineto{\pgfqpoint{4.807851in}{1.890264in}}%
\pgfpathlineto{\pgfqpoint{4.815287in}{1.899792in}}%
\pgfpathlineto{\pgfqpoint{4.822718in}{1.909302in}}%
\pgfpathlineto{\pgfqpoint{4.830143in}{1.918790in}}%
\pgfpathclose%
\pgfusepath{fill}%
\end{pgfscope}%
\begin{pgfscope}%
\pgfpathrectangle{\pgfqpoint{1.254980in}{0.150000in}}{\pgfqpoint{5.490039in}{5.490039in}}%
\pgfusepath{clip}%
\pgfsetbuttcap%
\pgfsetroundjoin%
\definecolor{currentfill}{rgb}{0.283197,0.115680,0.436115}%
\pgfsetfillcolor{currentfill}%
\pgfsetfillopacity{0.700000}%
\pgfsetlinewidth{0.000000pt}%
\definecolor{currentstroke}{rgb}{0.000000,0.000000,0.000000}%
\pgfsetstrokecolor{currentstroke}%
\pgfsetdash{}{0pt}%
\pgfpathmoveto{\pgfqpoint{3.291086in}{1.983635in}}%
\pgfpathlineto{\pgfqpoint{3.304251in}{1.977056in}}%
\pgfpathlineto{\pgfqpoint{3.317420in}{1.970508in}}%
\pgfpathlineto{\pgfqpoint{3.330594in}{1.963989in}}%
\pgfpathlineto{\pgfqpoint{3.343772in}{1.957501in}}%
\pgfpathlineto{\pgfqpoint{3.335693in}{1.957818in}}%
\pgfpathlineto{\pgfqpoint{3.327602in}{1.958429in}}%
\pgfpathlineto{\pgfqpoint{3.319497in}{1.959343in}}%
\pgfpathlineto{\pgfqpoint{3.311378in}{1.960565in}}%
\pgfpathlineto{\pgfqpoint{3.298173in}{1.967341in}}%
\pgfpathlineto{\pgfqpoint{3.284973in}{1.974147in}}%
\pgfpathlineto{\pgfqpoint{3.271776in}{1.980984in}}%
\pgfpathlineto{\pgfqpoint{3.258583in}{1.987850in}}%
\pgfpathlineto{\pgfqpoint{3.266729in}{1.986335in}}%
\pgfpathlineto{\pgfqpoint{3.274862in}{1.985132in}}%
\pgfpathlineto{\pgfqpoint{3.282980in}{1.984235in}}%
\pgfpathlineto{\pgfqpoint{3.291086in}{1.983635in}}%
\pgfpathclose%
\pgfusepath{fill}%
\end{pgfscope}%
\begin{pgfscope}%
\pgfpathrectangle{\pgfqpoint{1.254980in}{0.150000in}}{\pgfqpoint{5.490039in}{5.490039in}}%
\pgfusepath{clip}%
\pgfsetbuttcap%
\pgfsetroundjoin%
\definecolor{currentfill}{rgb}{0.280267,0.073417,0.397163}%
\pgfsetfillcolor{currentfill}%
\pgfsetfillopacity{0.700000}%
\pgfsetlinewidth{0.000000pt}%
\definecolor{currentstroke}{rgb}{0.000000,0.000000,0.000000}%
\pgfsetstrokecolor{currentstroke}%
\pgfsetdash{}{0pt}%
\pgfpathmoveto{\pgfqpoint{3.481338in}{1.910371in}}%
\pgfpathlineto{\pgfqpoint{3.494530in}{1.904420in}}%
\pgfpathlineto{\pgfqpoint{3.507727in}{1.898497in}}%
\pgfpathlineto{\pgfqpoint{3.520928in}{1.892603in}}%
\pgfpathlineto{\pgfqpoint{3.534134in}{1.886737in}}%
\pgfpathlineto{\pgfqpoint{3.526177in}{1.885146in}}%
\pgfpathlineto{\pgfqpoint{3.518209in}{1.883810in}}%
\pgfpathlineto{\pgfqpoint{3.510229in}{1.882736in}}%
\pgfpathlineto{\pgfqpoint{3.502239in}{1.881930in}}%
\pgfpathlineto{\pgfqpoint{3.489009in}{1.888069in}}%
\pgfpathlineto{\pgfqpoint{3.475784in}{1.894237in}}%
\pgfpathlineto{\pgfqpoint{3.462563in}{1.900433in}}%
\pgfpathlineto{\pgfqpoint{3.449346in}{1.906658in}}%
\pgfpathlineto{\pgfqpoint{3.457361in}{1.907185in}}%
\pgfpathlineto{\pgfqpoint{3.465365in}{1.907984in}}%
\pgfpathlineto{\pgfqpoint{3.473357in}{1.909049in}}%
\pgfpathlineto{\pgfqpoint{3.481338in}{1.910371in}}%
\pgfpathclose%
\pgfusepath{fill}%
\end{pgfscope}%
\begin{pgfscope}%
\pgfpathrectangle{\pgfqpoint{1.254980in}{0.150000in}}{\pgfqpoint{5.490039in}{5.490039in}}%
\pgfusepath{clip}%
\pgfsetbuttcap%
\pgfsetroundjoin%
\definecolor{currentfill}{rgb}{0.282623,0.140926,0.457517}%
\pgfsetfillcolor{currentfill}%
\pgfsetfillopacity{0.700000}%
\pgfsetlinewidth{0.000000pt}%
\definecolor{currentstroke}{rgb}{0.000000,0.000000,0.000000}%
\pgfsetstrokecolor{currentstroke}%
\pgfsetdash{}{0pt}%
\pgfpathmoveto{\pgfqpoint{5.219142in}{2.040798in}}%
\pgfpathlineto{\pgfqpoint{5.232775in}{2.039809in}}%
\pgfpathlineto{\pgfqpoint{5.246416in}{2.038843in}}%
\pgfpathlineto{\pgfqpoint{5.260065in}{2.037901in}}%
\pgfpathlineto{\pgfqpoint{5.273723in}{2.036983in}}%
\pgfpathlineto{\pgfqpoint{5.266445in}{2.027723in}}%
\pgfpathlineto{\pgfqpoint{5.259159in}{2.018386in}}%
\pgfpathlineto{\pgfqpoint{5.251867in}{2.008973in}}%
\pgfpathlineto{\pgfqpoint{5.244569in}{1.999486in}}%
\pgfpathlineto{\pgfqpoint{5.230902in}{2.000485in}}%
\pgfpathlineto{\pgfqpoint{5.217244in}{2.001506in}}%
\pgfpathlineto{\pgfqpoint{5.203594in}{2.002552in}}%
\pgfpathlineto{\pgfqpoint{5.189952in}{2.003621in}}%
\pgfpathlineto{\pgfqpoint{5.197259in}{2.013023in}}%
\pgfpathlineto{\pgfqpoint{5.204560in}{2.022354in}}%
\pgfpathlineto{\pgfqpoint{5.211854in}{2.031613in}}%
\pgfpathlineto{\pgfqpoint{5.219142in}{2.040798in}}%
\pgfpathclose%
\pgfusepath{fill}%
\end{pgfscope}%
\begin{pgfscope}%
\pgfpathrectangle{\pgfqpoint{1.254980in}{0.150000in}}{\pgfqpoint{5.490039in}{5.490039in}}%
\pgfusepath{clip}%
\pgfsetbuttcap%
\pgfsetroundjoin%
\definecolor{currentfill}{rgb}{0.269944,0.014625,0.341379}%
\pgfsetfillcolor{currentfill}%
\pgfsetfillopacity{0.700000}%
\pgfsetlinewidth{0.000000pt}%
\definecolor{currentstroke}{rgb}{0.000000,0.000000,0.000000}%
\pgfsetstrokecolor{currentstroke}%
\pgfsetdash{}{0pt}%
\pgfpathmoveto{\pgfqpoint{3.945882in}{1.808390in}}%
\pgfpathlineto{\pgfqpoint{3.959160in}{1.803936in}}%
\pgfpathlineto{\pgfqpoint{3.972443in}{1.799507in}}%
\pgfpathlineto{\pgfqpoint{3.985733in}{1.795105in}}%
\pgfpathlineto{\pgfqpoint{3.999028in}{1.790727in}}%
\pgfpathlineto{\pgfqpoint{3.991306in}{1.785001in}}%
\pgfpathlineto{\pgfqpoint{3.983578in}{1.779429in}}%
\pgfpathlineto{\pgfqpoint{3.975843in}{1.774016in}}%
\pgfpathlineto{\pgfqpoint{3.968101in}{1.768770in}}%
\pgfpathlineto{\pgfqpoint{3.954789in}{1.773381in}}%
\pgfpathlineto{\pgfqpoint{3.941483in}{1.778018in}}%
\pgfpathlineto{\pgfqpoint{3.928183in}{1.782680in}}%
\pgfpathlineto{\pgfqpoint{3.914889in}{1.787368in}}%
\pgfpathlineto{\pgfqpoint{3.922648in}{1.792376in}}%
\pgfpathlineto{\pgfqpoint{3.930399in}{1.797553in}}%
\pgfpathlineto{\pgfqpoint{3.938144in}{1.802893in}}%
\pgfpathlineto{\pgfqpoint{3.945882in}{1.808390in}}%
\pgfpathclose%
\pgfusepath{fill}%
\end{pgfscope}%
\begin{pgfscope}%
\pgfpathrectangle{\pgfqpoint{1.254980in}{0.150000in}}{\pgfqpoint{5.490039in}{5.490039in}}%
\pgfusepath{clip}%
\pgfsetbuttcap%
\pgfsetroundjoin%
\definecolor{currentfill}{rgb}{0.272594,0.025563,0.353093}%
\pgfsetfillcolor{currentfill}%
\pgfsetfillopacity{0.700000}%
\pgfsetlinewidth{0.000000pt}%
\definecolor{currentstroke}{rgb}{0.000000,0.000000,0.000000}%
\pgfsetstrokecolor{currentstroke}%
\pgfsetdash{}{0pt}%
\pgfpathmoveto{\pgfqpoint{3.808731in}{1.825805in}}%
\pgfpathlineto{\pgfqpoint{3.821982in}{1.820909in}}%
\pgfpathlineto{\pgfqpoint{3.835238in}{1.816039in}}%
\pgfpathlineto{\pgfqpoint{3.848499in}{1.811195in}}%
\pgfpathlineto{\pgfqpoint{3.861766in}{1.806378in}}%
\pgfpathlineto{\pgfqpoint{3.853983in}{1.801787in}}%
\pgfpathlineto{\pgfqpoint{3.846191in}{1.797381in}}%
\pgfpathlineto{\pgfqpoint{3.838392in}{1.793165in}}%
\pgfpathlineto{\pgfqpoint{3.830585in}{1.789147in}}%
\pgfpathlineto{\pgfqpoint{3.817300in}{1.794211in}}%
\pgfpathlineto{\pgfqpoint{3.804020in}{1.799301in}}%
\pgfpathlineto{\pgfqpoint{3.790745in}{1.804418in}}%
\pgfpathlineto{\pgfqpoint{3.777476in}{1.809561in}}%
\pgfpathlineto{\pgfqpoint{3.785302in}{1.813327in}}%
\pgfpathlineto{\pgfqpoint{3.793120in}{1.817294in}}%
\pgfpathlineto{\pgfqpoint{3.800929in}{1.821456in}}%
\pgfpathlineto{\pgfqpoint{3.808731in}{1.825805in}}%
\pgfpathclose%
\pgfusepath{fill}%
\end{pgfscope}%
\begin{pgfscope}%
\pgfpathrectangle{\pgfqpoint{1.254980in}{0.150000in}}{\pgfqpoint{5.490039in}{5.490039in}}%
\pgfusepath{clip}%
\pgfsetbuttcap%
\pgfsetroundjoin%
\definecolor{currentfill}{rgb}{0.268510,0.009605,0.335427}%
\pgfsetfillcolor{currentfill}%
\pgfsetfillopacity{0.700000}%
\pgfsetlinewidth{0.000000pt}%
\definecolor{currentstroke}{rgb}{0.000000,0.000000,0.000000}%
\pgfsetstrokecolor{currentstroke}%
\pgfsetdash{}{0pt}%
\pgfpathmoveto{\pgfqpoint{4.083028in}{1.798685in}}%
\pgfpathlineto{\pgfqpoint{4.096338in}{1.794655in}}%
\pgfpathlineto{\pgfqpoint{4.109654in}{1.790650in}}%
\pgfpathlineto{\pgfqpoint{4.122976in}{1.786670in}}%
\pgfpathlineto{\pgfqpoint{4.136304in}{1.782716in}}%
\pgfpathlineto{\pgfqpoint{4.128637in}{1.775999in}}%
\pgfpathlineto{\pgfqpoint{4.120964in}{1.769407in}}%
\pgfpathlineto{\pgfqpoint{4.113285in}{1.762945in}}%
\pgfpathlineto{\pgfqpoint{4.105600in}{1.756619in}}%
\pgfpathlineto{\pgfqpoint{4.092258in}{1.760795in}}%
\pgfpathlineto{\pgfqpoint{4.078922in}{1.764995in}}%
\pgfpathlineto{\pgfqpoint{4.065591in}{1.769221in}}%
\pgfpathlineto{\pgfqpoint{4.052267in}{1.773472in}}%
\pgfpathlineto{\pgfqpoint{4.059967in}{1.779572in}}%
\pgfpathlineto{\pgfqpoint{4.067660in}{1.785811in}}%
\pgfpathlineto{\pgfqpoint{4.075347in}{1.792184in}}%
\pgfpathlineto{\pgfqpoint{4.083028in}{1.798685in}}%
\pgfpathclose%
\pgfusepath{fill}%
\end{pgfscope}%
\begin{pgfscope}%
\pgfpathrectangle{\pgfqpoint{1.254980in}{0.150000in}}{\pgfqpoint{5.490039in}{5.490039in}}%
\pgfusepath{clip}%
\pgfsetbuttcap%
\pgfsetroundjoin%
\definecolor{currentfill}{rgb}{0.278791,0.062145,0.386592}%
\pgfsetfillcolor{currentfill}%
\pgfsetfillopacity{0.700000}%
\pgfsetlinewidth{0.000000pt}%
\definecolor{currentstroke}{rgb}{0.000000,0.000000,0.000000}%
\pgfsetstrokecolor{currentstroke}%
\pgfsetdash{}{0pt}%
\pgfpathmoveto{\pgfqpoint{4.746418in}{1.888943in}}%
\pgfpathlineto{\pgfqpoint{4.759905in}{1.886851in}}%
\pgfpathlineto{\pgfqpoint{4.773399in}{1.884784in}}%
\pgfpathlineto{\pgfqpoint{4.786901in}{1.882740in}}%
\pgfpathlineto{\pgfqpoint{4.800410in}{1.880721in}}%
\pgfpathlineto{\pgfqpoint{4.792964in}{1.871165in}}%
\pgfpathlineto{\pgfqpoint{4.785513in}{1.861601in}}%
\pgfpathlineto{\pgfqpoint{4.778056in}{1.852030in}}%
\pgfpathlineto{\pgfqpoint{4.770595in}{1.842458in}}%
\pgfpathlineto{\pgfqpoint{4.757077in}{1.844623in}}%
\pgfpathlineto{\pgfqpoint{4.743566in}{1.846811in}}%
\pgfpathlineto{\pgfqpoint{4.730063in}{1.849023in}}%
\pgfpathlineto{\pgfqpoint{4.716567in}{1.851260in}}%
\pgfpathlineto{\pgfqpoint{4.724038in}{1.860682in}}%
\pgfpathlineto{\pgfqpoint{4.731503in}{1.870106in}}%
\pgfpathlineto{\pgfqpoint{4.738963in}{1.879527in}}%
\pgfpathlineto{\pgfqpoint{4.746418in}{1.888943in}}%
\pgfpathclose%
\pgfusepath{fill}%
\end{pgfscope}%
\begin{pgfscope}%
\pgfpathrectangle{\pgfqpoint{1.254980in}{0.150000in}}{\pgfqpoint{5.490039in}{5.490039in}}%
\pgfusepath{clip}%
\pgfsetbuttcap%
\pgfsetroundjoin%
\definecolor{currentfill}{rgb}{0.273006,0.204520,0.501721}%
\pgfsetfillcolor{currentfill}%
\pgfsetfillopacity{0.700000}%
\pgfsetlinewidth{0.000000pt}%
\definecolor{currentstroke}{rgb}{0.000000,0.000000,0.000000}%
\pgfsetstrokecolor{currentstroke}%
\pgfsetdash{}{0pt}%
\pgfpathmoveto{\pgfqpoint{5.608269in}{2.165278in}}%
\pgfpathlineto{\pgfqpoint{5.622034in}{2.164951in}}%
\pgfpathlineto{\pgfqpoint{5.635807in}{2.164648in}}%
\pgfpathlineto{\pgfqpoint{5.649590in}{2.164369in}}%
\pgfpathlineto{\pgfqpoint{5.663381in}{2.164113in}}%
\pgfpathlineto{\pgfqpoint{5.656274in}{2.156118in}}%
\pgfpathlineto{\pgfqpoint{5.649158in}{2.148015in}}%
\pgfpathlineto{\pgfqpoint{5.642034in}{2.139806in}}%
\pgfpathlineto{\pgfqpoint{5.634901in}{2.131490in}}%
\pgfpathlineto{\pgfqpoint{5.621099in}{2.131772in}}%
\pgfpathlineto{\pgfqpoint{5.607305in}{2.132078in}}%
\pgfpathlineto{\pgfqpoint{5.593521in}{2.132408in}}%
\pgfpathlineto{\pgfqpoint{5.579745in}{2.132761in}}%
\pgfpathlineto{\pgfqpoint{5.586888in}{2.141045in}}%
\pgfpathlineto{\pgfqpoint{5.594023in}{2.149226in}}%
\pgfpathlineto{\pgfqpoint{5.601150in}{2.157304in}}%
\pgfpathlineto{\pgfqpoint{5.608269in}{2.165278in}}%
\pgfpathclose%
\pgfusepath{fill}%
\end{pgfscope}%
\begin{pgfscope}%
\pgfpathrectangle{\pgfqpoint{1.254980in}{0.150000in}}{\pgfqpoint{5.490039in}{5.490039in}}%
\pgfusepath{clip}%
\pgfsetbuttcap%
\pgfsetroundjoin%
\definecolor{currentfill}{rgb}{0.280255,0.165693,0.476498}%
\pgfsetfillcolor{currentfill}%
\pgfsetfillopacity{0.700000}%
\pgfsetlinewidth{0.000000pt}%
\definecolor{currentstroke}{rgb}{0.000000,0.000000,0.000000}%
\pgfsetstrokecolor{currentstroke}%
\pgfsetdash{}{0pt}%
\pgfpathmoveto{\pgfqpoint{3.100564in}{2.072694in}}%
\pgfpathlineto{\pgfqpoint{3.113712in}{2.065447in}}%
\pgfpathlineto{\pgfqpoint{3.126864in}{2.058233in}}%
\pgfpathlineto{\pgfqpoint{3.140019in}{2.051051in}}%
\pgfpathlineto{\pgfqpoint{3.153178in}{2.043902in}}%
\pgfpathlineto{\pgfqpoint{3.144958in}{2.046332in}}%
\pgfpathlineto{\pgfqpoint{3.136722in}{2.049098in}}%
\pgfpathlineto{\pgfqpoint{3.128471in}{2.052208in}}%
\pgfpathlineto{\pgfqpoint{3.120203in}{2.055671in}}%
\pgfpathlineto{\pgfqpoint{3.107013in}{2.063122in}}%
\pgfpathlineto{\pgfqpoint{3.093826in}{2.070606in}}%
\pgfpathlineto{\pgfqpoint{3.080644in}{2.078122in}}%
\pgfpathlineto{\pgfqpoint{3.067464in}{2.085672in}}%
\pgfpathlineto{\pgfqpoint{3.075764in}{2.081901in}}%
\pgfpathlineto{\pgfqpoint{3.084047in}{2.078487in}}%
\pgfpathlineto{\pgfqpoint{3.092314in}{2.075421in}}%
\pgfpathlineto{\pgfqpoint{3.100564in}{2.072694in}}%
\pgfpathclose%
\pgfusepath{fill}%
\end{pgfscope}%
\begin{pgfscope}%
\pgfpathrectangle{\pgfqpoint{1.254980in}{0.150000in}}{\pgfqpoint{5.490039in}{5.490039in}}%
\pgfusepath{clip}%
\pgfsetbuttcap%
\pgfsetroundjoin%
\definecolor{currentfill}{rgb}{0.283187,0.125848,0.444960}%
\pgfsetfillcolor{currentfill}%
\pgfsetfillopacity{0.700000}%
\pgfsetlinewidth{0.000000pt}%
\definecolor{currentstroke}{rgb}{0.000000,0.000000,0.000000}%
\pgfsetstrokecolor{currentstroke}%
\pgfsetdash{}{0pt}%
\pgfpathmoveto{\pgfqpoint{5.135468in}{2.008136in}}%
\pgfpathlineto{\pgfqpoint{5.149076in}{2.006971in}}%
\pgfpathlineto{\pgfqpoint{5.162693in}{2.005831in}}%
\pgfpathlineto{\pgfqpoint{5.176319in}{2.004714in}}%
\pgfpathlineto{\pgfqpoint{5.189952in}{2.003621in}}%
\pgfpathlineto{\pgfqpoint{5.182638in}{1.994151in}}%
\pgfpathlineto{\pgfqpoint{5.175319in}{1.984614in}}%
\pgfpathlineto{\pgfqpoint{5.167993in}{1.975012in}}%
\pgfpathlineto{\pgfqpoint{5.160660in}{1.965347in}}%
\pgfpathlineto{\pgfqpoint{5.147018in}{1.966533in}}%
\pgfpathlineto{\pgfqpoint{5.133385in}{1.967743in}}%
\pgfpathlineto{\pgfqpoint{5.119759in}{1.968977in}}%
\pgfpathlineto{\pgfqpoint{5.106142in}{1.970234in}}%
\pgfpathlineto{\pgfqpoint{5.113482in}{1.979801in}}%
\pgfpathlineto{\pgfqpoint{5.120817in}{1.989308in}}%
\pgfpathlineto{\pgfqpoint{5.128145in}{1.998754in}}%
\pgfpathlineto{\pgfqpoint{5.135468in}{2.008136in}}%
\pgfpathclose%
\pgfusepath{fill}%
\end{pgfscope}%
\begin{pgfscope}%
\pgfpathrectangle{\pgfqpoint{1.254980in}{0.150000in}}{\pgfqpoint{5.490039in}{5.490039in}}%
\pgfusepath{clip}%
\pgfsetbuttcap%
\pgfsetroundjoin%
\definecolor{currentfill}{rgb}{0.265145,0.232956,0.516599}%
\pgfsetfillcolor{currentfill}%
\pgfsetfillopacity{0.700000}%
\pgfsetlinewidth{0.000000pt}%
\definecolor{currentstroke}{rgb}{0.000000,0.000000,0.000000}%
\pgfsetstrokecolor{currentstroke}%
\pgfsetdash{}{0pt}%
\pgfpathmoveto{\pgfqpoint{2.857028in}{2.211117in}}%
\pgfpathlineto{\pgfqpoint{2.870158in}{2.203010in}}%
\pgfpathlineto{\pgfqpoint{2.883290in}{2.194940in}}%
\pgfpathlineto{\pgfqpoint{2.896426in}{2.186906in}}%
\pgfpathlineto{\pgfqpoint{2.909564in}{2.178908in}}%
\pgfpathlineto{\pgfqpoint{2.901145in}{2.183979in}}%
\pgfpathlineto{\pgfqpoint{2.892707in}{2.189435in}}%
\pgfpathlineto{\pgfqpoint{2.884249in}{2.195283in}}%
\pgfpathlineto{\pgfqpoint{2.875771in}{2.201532in}}%
\pgfpathlineto{\pgfqpoint{2.862597in}{2.209848in}}%
\pgfpathlineto{\pgfqpoint{2.849425in}{2.218200in}}%
\pgfpathlineto{\pgfqpoint{2.836256in}{2.226589in}}%
\pgfpathlineto{\pgfqpoint{2.823091in}{2.235016in}}%
\pgfpathlineto{\pgfqpoint{2.831606in}{2.228442in}}%
\pgfpathlineto{\pgfqpoint{2.840100in}{2.222273in}}%
\pgfpathlineto{\pgfqpoint{2.848574in}{2.216502in}}%
\pgfpathlineto{\pgfqpoint{2.857028in}{2.211117in}}%
\pgfpathclose%
\pgfusepath{fill}%
\end{pgfscope}%
\begin{pgfscope}%
\pgfpathrectangle{\pgfqpoint{1.254980in}{0.150000in}}{\pgfqpoint{5.490039in}{5.490039in}}%
\pgfusepath{clip}%
\pgfsetbuttcap%
\pgfsetroundjoin%
\definecolor{currentfill}{rgb}{0.271305,0.019942,0.347269}%
\pgfsetfillcolor{currentfill}%
\pgfsetfillopacity{0.700000}%
\pgfsetlinewidth{0.000000pt}%
\definecolor{currentstroke}{rgb}{0.000000,0.000000,0.000000}%
\pgfsetstrokecolor{currentstroke}%
\pgfsetdash{}{0pt}%
\pgfpathmoveto{\pgfqpoint{4.441374in}{1.820653in}}%
\pgfpathlineto{\pgfqpoint{4.454775in}{1.817703in}}%
\pgfpathlineto{\pgfqpoint{4.468183in}{1.814778in}}%
\pgfpathlineto{\pgfqpoint{4.481598in}{1.811877in}}%
\pgfpathlineto{\pgfqpoint{4.495020in}{1.809001in}}%
\pgfpathlineto{\pgfqpoint{4.487477in}{1.800292in}}%
\pgfpathlineto{\pgfqpoint{4.479929in}{1.791633in}}%
\pgfpathlineto{\pgfqpoint{4.472376in}{1.783028in}}%
\pgfpathlineto{\pgfqpoint{4.464818in}{1.774482in}}%
\pgfpathlineto{\pgfqpoint{4.451386in}{1.777542in}}%
\pgfpathlineto{\pgfqpoint{4.437960in}{1.780626in}}%
\pgfpathlineto{\pgfqpoint{4.424541in}{1.783734in}}%
\pgfpathlineto{\pgfqpoint{4.411129in}{1.786867in}}%
\pgfpathlineto{\pgfqpoint{4.418698in}{1.795225in}}%
\pgfpathlineto{\pgfqpoint{4.426261in}{1.803645in}}%
\pgfpathlineto{\pgfqpoint{4.433820in}{1.812122in}}%
\pgfpathlineto{\pgfqpoint{4.441374in}{1.820653in}}%
\pgfpathclose%
\pgfusepath{fill}%
\end{pgfscope}%
\begin{pgfscope}%
\pgfpathrectangle{\pgfqpoint{1.254980in}{0.150000in}}{\pgfqpoint{5.490039in}{5.490039in}}%
\pgfusepath{clip}%
\pgfsetbuttcap%
\pgfsetroundjoin%
\definecolor{currentfill}{rgb}{0.276022,0.044167,0.370164}%
\pgfsetfillcolor{currentfill}%
\pgfsetfillopacity{0.700000}%
\pgfsetlinewidth{0.000000pt}%
\definecolor{currentstroke}{rgb}{0.000000,0.000000,0.000000}%
\pgfsetstrokecolor{currentstroke}%
\pgfsetdash{}{0pt}%
\pgfpathmoveto{\pgfqpoint{3.671508in}{1.851663in}}%
\pgfpathlineto{\pgfqpoint{3.684736in}{1.846306in}}%
\pgfpathlineto{\pgfqpoint{3.697969in}{1.840976in}}%
\pgfpathlineto{\pgfqpoint{3.711207in}{1.835673in}}%
\pgfpathlineto{\pgfqpoint{3.724451in}{1.830398in}}%
\pgfpathlineto{\pgfqpoint{3.716597in}{1.827093in}}%
\pgfpathlineto{\pgfqpoint{3.708734in}{1.824006in}}%
\pgfpathlineto{\pgfqpoint{3.700862in}{1.821142in}}%
\pgfpathlineto{\pgfqpoint{3.692980in}{1.818509in}}%
\pgfpathlineto{\pgfqpoint{3.679716in}{1.824045in}}%
\pgfpathlineto{\pgfqpoint{3.666457in}{1.829607in}}%
\pgfpathlineto{\pgfqpoint{3.653203in}{1.835197in}}%
\pgfpathlineto{\pgfqpoint{3.639954in}{1.840814in}}%
\pgfpathlineto{\pgfqpoint{3.647857in}{1.843182in}}%
\pgfpathlineto{\pgfqpoint{3.655750in}{1.845784in}}%
\pgfpathlineto{\pgfqpoint{3.663633in}{1.848613in}}%
\pgfpathlineto{\pgfqpoint{3.671508in}{1.851663in}}%
\pgfpathclose%
\pgfusepath{fill}%
\end{pgfscope}%
\begin{pgfscope}%
\pgfpathrectangle{\pgfqpoint{1.254980in}{0.150000in}}{\pgfqpoint{5.490039in}{5.490039in}}%
\pgfusepath{clip}%
\pgfsetbuttcap%
\pgfsetroundjoin%
\definecolor{currentfill}{rgb}{0.227802,0.326594,0.546532}%
\pgfsetfillcolor{currentfill}%
\pgfsetfillopacity{0.700000}%
\pgfsetlinewidth{0.000000pt}%
\definecolor{currentstroke}{rgb}{0.000000,0.000000,0.000000}%
\pgfsetstrokecolor{currentstroke}%
\pgfsetdash{}{0pt}%
\pgfpathmoveto{\pgfqpoint{2.560270in}{2.411795in}}%
\pgfpathlineto{\pgfqpoint{2.573391in}{2.402560in}}%
\pgfpathlineto{\pgfqpoint{2.586514in}{2.393369in}}%
\pgfpathlineto{\pgfqpoint{2.599639in}{2.384222in}}%
\pgfpathlineto{\pgfqpoint{2.612766in}{2.375117in}}%
\pgfpathlineto{\pgfqpoint{2.604072in}{2.383423in}}%
\pgfpathlineto{\pgfqpoint{2.595353in}{2.392168in}}%
\pgfpathlineto{\pgfqpoint{2.586610in}{2.401361in}}%
\pgfpathlineto{\pgfqpoint{2.577841in}{2.411011in}}%
\pgfpathlineto{\pgfqpoint{2.564672in}{2.420452in}}%
\pgfpathlineto{\pgfqpoint{2.551505in}{2.429936in}}%
\pgfpathlineto{\pgfqpoint{2.538340in}{2.439464in}}%
\pgfpathlineto{\pgfqpoint{2.525176in}{2.449037in}}%
\pgfpathlineto{\pgfqpoint{2.533988in}{2.439043in}}%
\pgfpathlineto{\pgfqpoint{2.542774in}{2.429511in}}%
\pgfpathlineto{\pgfqpoint{2.551534in}{2.420432in}}%
\pgfpathlineto{\pgfqpoint{2.560270in}{2.411795in}}%
\pgfpathclose%
\pgfusepath{fill}%
\end{pgfscope}%
\begin{pgfscope}%
\pgfpathrectangle{\pgfqpoint{1.254980in}{0.150000in}}{\pgfqpoint{5.490039in}{5.490039in}}%
\pgfusepath{clip}%
\pgfsetbuttcap%
\pgfsetroundjoin%
\definecolor{currentfill}{rgb}{0.268510,0.009605,0.335427}%
\pgfsetfillcolor{currentfill}%
\pgfsetfillopacity{0.700000}%
\pgfsetlinewidth{0.000000pt}%
\definecolor{currentstroke}{rgb}{0.000000,0.000000,0.000000}%
\pgfsetstrokecolor{currentstroke}%
\pgfsetdash{}{0pt}%
\pgfpathmoveto{\pgfqpoint{4.220232in}{1.795989in}}%
\pgfpathlineto{\pgfqpoint{4.233578in}{1.792367in}}%
\pgfpathlineto{\pgfqpoint{4.246930in}{1.788770in}}%
\pgfpathlineto{\pgfqpoint{4.260288in}{1.785198in}}%
\pgfpathlineto{\pgfqpoint{4.273653in}{1.781650in}}%
\pgfpathlineto{\pgfqpoint{4.266035in}{1.774080in}}%
\pgfpathlineto{\pgfqpoint{4.258412in}{1.766607in}}%
\pgfpathlineto{\pgfqpoint{4.250783in}{1.759237in}}%
\pgfpathlineto{\pgfqpoint{4.243149in}{1.751974in}}%
\pgfpathlineto{\pgfqpoint{4.229771in}{1.755730in}}%
\pgfpathlineto{\pgfqpoint{4.216400in}{1.759511in}}%
\pgfpathlineto{\pgfqpoint{4.203035in}{1.763317in}}%
\pgfpathlineto{\pgfqpoint{4.189676in}{1.767147in}}%
\pgfpathlineto{\pgfqpoint{4.197324in}{1.774196in}}%
\pgfpathlineto{\pgfqpoint{4.204966in}{1.781357in}}%
\pgfpathlineto{\pgfqpoint{4.212602in}{1.788623in}}%
\pgfpathlineto{\pgfqpoint{4.220232in}{1.795989in}}%
\pgfpathclose%
\pgfusepath{fill}%
\end{pgfscope}%
\begin{pgfscope}%
\pgfpathrectangle{\pgfqpoint{1.254980in}{0.150000in}}{\pgfqpoint{5.490039in}{5.490039in}}%
\pgfusepath{clip}%
\pgfsetbuttcap%
\pgfsetroundjoin%
\definecolor{currentfill}{rgb}{0.276194,0.190074,0.493001}%
\pgfsetfillcolor{currentfill}%
\pgfsetfillopacity{0.700000}%
\pgfsetlinewidth{0.000000pt}%
\definecolor{currentstroke}{rgb}{0.000000,0.000000,0.000000}%
\pgfsetstrokecolor{currentstroke}%
\pgfsetdash{}{0pt}%
\pgfpathmoveto{\pgfqpoint{5.524731in}{2.134413in}}%
\pgfpathlineto{\pgfqpoint{5.538471in}{2.133965in}}%
\pgfpathlineto{\pgfqpoint{5.552220in}{2.133540in}}%
\pgfpathlineto{\pgfqpoint{5.565978in}{2.133139in}}%
\pgfpathlineto{\pgfqpoint{5.579745in}{2.132761in}}%
\pgfpathlineto{\pgfqpoint{5.572594in}{2.124375in}}%
\pgfpathlineto{\pgfqpoint{5.565434in}{2.115886in}}%
\pgfpathlineto{\pgfqpoint{5.558267in}{2.107295in}}%
\pgfpathlineto{\pgfqpoint{5.551092in}{2.098604in}}%
\pgfpathlineto{\pgfqpoint{5.537315in}{2.099021in}}%
\pgfpathlineto{\pgfqpoint{5.523546in}{2.099462in}}%
\pgfpathlineto{\pgfqpoint{5.509787in}{2.099927in}}%
\pgfpathlineto{\pgfqpoint{5.496036in}{2.100416in}}%
\pgfpathlineto{\pgfqpoint{5.503221in}{2.109063in}}%
\pgfpathlineto{\pgfqpoint{5.510399in}{2.117611in}}%
\pgfpathlineto{\pgfqpoint{5.517569in}{2.126062in}}%
\pgfpathlineto{\pgfqpoint{5.524731in}{2.134413in}}%
\pgfpathclose%
\pgfusepath{fill}%
\end{pgfscope}%
\begin{pgfscope}%
\pgfpathrectangle{\pgfqpoint{1.254980in}{0.150000in}}{\pgfqpoint{5.490039in}{5.490039in}}%
\pgfusepath{clip}%
\pgfsetbuttcap%
\pgfsetroundjoin%
\definecolor{currentfill}{rgb}{0.283091,0.110553,0.431554}%
\pgfsetfillcolor{currentfill}%
\pgfsetfillopacity{0.700000}%
\pgfsetlinewidth{0.000000pt}%
\definecolor{currentstroke}{rgb}{0.000000,0.000000,0.000000}%
\pgfsetstrokecolor{currentstroke}%
\pgfsetdash{}{0pt}%
\pgfpathmoveto{\pgfqpoint{5.051753in}{1.975502in}}%
\pgfpathlineto{\pgfqpoint{5.065338in}{1.974149in}}%
\pgfpathlineto{\pgfqpoint{5.078931in}{1.972821in}}%
\pgfpathlineto{\pgfqpoint{5.092533in}{1.971516in}}%
\pgfpathlineto{\pgfqpoint{5.106142in}{1.970234in}}%
\pgfpathlineto{\pgfqpoint{5.098795in}{1.960610in}}%
\pgfpathlineto{\pgfqpoint{5.091443in}{1.950931in}}%
\pgfpathlineto{\pgfqpoint{5.084084in}{1.941198in}}%
\pgfpathlineto{\pgfqpoint{5.076720in}{1.931415in}}%
\pgfpathlineto{\pgfqpoint{5.063102in}{1.932803in}}%
\pgfpathlineto{\pgfqpoint{5.049493in}{1.934214in}}%
\pgfpathlineto{\pgfqpoint{5.035891in}{1.935649in}}%
\pgfpathlineto{\pgfqpoint{5.022298in}{1.937108in}}%
\pgfpathlineto{\pgfqpoint{5.029670in}{1.946780in}}%
\pgfpathlineto{\pgfqpoint{5.037037in}{1.956404in}}%
\pgfpathlineto{\pgfqpoint{5.044398in}{1.965979in}}%
\pgfpathlineto{\pgfqpoint{5.051753in}{1.975502in}}%
\pgfpathclose%
\pgfusepath{fill}%
\end{pgfscope}%
\begin{pgfscope}%
\pgfpathrectangle{\pgfqpoint{1.254980in}{0.150000in}}{\pgfqpoint{5.490039in}{5.490039in}}%
\pgfusepath{clip}%
\pgfsetbuttcap%
\pgfsetroundjoin%
\definecolor{currentfill}{rgb}{0.276022,0.044167,0.370164}%
\pgfsetfillcolor{currentfill}%
\pgfsetfillopacity{0.700000}%
\pgfsetlinewidth{0.000000pt}%
\definecolor{currentstroke}{rgb}{0.000000,0.000000,0.000000}%
\pgfsetstrokecolor{currentstroke}%
\pgfsetdash{}{0pt}%
\pgfpathmoveto{\pgfqpoint{4.662658in}{1.860444in}}%
\pgfpathlineto{\pgfqpoint{4.676124in}{1.858112in}}%
\pgfpathlineto{\pgfqpoint{4.689598in}{1.855804in}}%
\pgfpathlineto{\pgfqpoint{4.703079in}{1.853520in}}%
\pgfpathlineto{\pgfqpoint{4.716567in}{1.851260in}}%
\pgfpathlineto{\pgfqpoint{4.709092in}{1.841842in}}%
\pgfpathlineto{\pgfqpoint{4.701612in}{1.832432in}}%
\pgfpathlineto{\pgfqpoint{4.694127in}{1.823034in}}%
\pgfpathlineto{\pgfqpoint{4.686637in}{1.813651in}}%
\pgfpathlineto{\pgfqpoint{4.673140in}{1.816069in}}%
\pgfpathlineto{\pgfqpoint{4.659650in}{1.818511in}}%
\pgfpathlineto{\pgfqpoint{4.646167in}{1.820977in}}%
\pgfpathlineto{\pgfqpoint{4.632692in}{1.823467in}}%
\pgfpathlineto{\pgfqpoint{4.640191in}{1.832686in}}%
\pgfpathlineto{\pgfqpoint{4.647685in}{1.841925in}}%
\pgfpathlineto{\pgfqpoint{4.655174in}{1.851179in}}%
\pgfpathlineto{\pgfqpoint{4.662658in}{1.860444in}}%
\pgfpathclose%
\pgfusepath{fill}%
\end{pgfscope}%
\begin{pgfscope}%
\pgfpathrectangle{\pgfqpoint{1.254980in}{0.150000in}}{\pgfqpoint{5.490039in}{5.490039in}}%
\pgfusepath{clip}%
\pgfsetbuttcap%
\pgfsetroundjoin%
\definecolor{currentfill}{rgb}{0.282910,0.105393,0.426902}%
\pgfsetfillcolor{currentfill}%
\pgfsetfillopacity{0.700000}%
\pgfsetlinewidth{0.000000pt}%
\definecolor{currentstroke}{rgb}{0.000000,0.000000,0.000000}%
\pgfsetstrokecolor{currentstroke}%
\pgfsetdash{}{0pt}%
\pgfpathmoveto{\pgfqpoint{3.343772in}{1.957501in}}%
\pgfpathlineto{\pgfqpoint{3.356954in}{1.951042in}}%
\pgfpathlineto{\pgfqpoint{3.370140in}{1.944614in}}%
\pgfpathlineto{\pgfqpoint{3.383330in}{1.938215in}}%
\pgfpathlineto{\pgfqpoint{3.396525in}{1.931845in}}%
\pgfpathlineto{\pgfqpoint{3.388472in}{1.931880in}}%
\pgfpathlineto{\pgfqpoint{3.380407in}{1.932206in}}%
\pgfpathlineto{\pgfqpoint{3.372330in}{1.932830in}}%
\pgfpathlineto{\pgfqpoint{3.364239in}{1.933760in}}%
\pgfpathlineto{\pgfqpoint{3.351017in}{1.940417in}}%
\pgfpathlineto{\pgfqpoint{3.337800in}{1.947103in}}%
\pgfpathlineto{\pgfqpoint{3.324587in}{1.953819in}}%
\pgfpathlineto{\pgfqpoint{3.311378in}{1.960565in}}%
\pgfpathlineto{\pgfqpoint{3.319497in}{1.959343in}}%
\pgfpathlineto{\pgfqpoint{3.327602in}{1.958429in}}%
\pgfpathlineto{\pgfqpoint{3.335693in}{1.957818in}}%
\pgfpathlineto{\pgfqpoint{3.343772in}{1.957501in}}%
\pgfpathclose%
\pgfusepath{fill}%
\end{pgfscope}%
\begin{pgfscope}%
\pgfpathrectangle{\pgfqpoint{1.254980in}{0.150000in}}{\pgfqpoint{5.490039in}{5.490039in}}%
\pgfusepath{clip}%
\pgfsetbuttcap%
\pgfsetroundjoin%
\definecolor{currentfill}{rgb}{0.278012,0.180367,0.486697}%
\pgfsetfillcolor{currentfill}%
\pgfsetfillopacity{0.700000}%
\pgfsetlinewidth{0.000000pt}%
\definecolor{currentstroke}{rgb}{0.000000,0.000000,0.000000}%
\pgfsetstrokecolor{currentstroke}%
\pgfsetdash{}{0pt}%
\pgfpathmoveto{\pgfqpoint{5.441121in}{2.102608in}}%
\pgfpathlineto{\pgfqpoint{5.454837in}{2.102025in}}%
\pgfpathlineto{\pgfqpoint{5.468561in}{2.101465in}}%
\pgfpathlineto{\pgfqpoint{5.482294in}{2.100928in}}%
\pgfpathlineto{\pgfqpoint{5.496036in}{2.100416in}}%
\pgfpathlineto{\pgfqpoint{5.488843in}{2.091672in}}%
\pgfpathlineto{\pgfqpoint{5.481642in}{2.082832in}}%
\pgfpathlineto{\pgfqpoint{5.474434in}{2.073896in}}%
\pgfpathlineto{\pgfqpoint{5.467219in}{2.064865in}}%
\pgfpathlineto{\pgfqpoint{5.453467in}{2.065432in}}%
\pgfpathlineto{\pgfqpoint{5.439724in}{2.066021in}}%
\pgfpathlineto{\pgfqpoint{5.425990in}{2.066635in}}%
\pgfpathlineto{\pgfqpoint{5.412264in}{2.067272in}}%
\pgfpathlineto{\pgfqpoint{5.419490in}{2.076244in}}%
\pgfpathlineto{\pgfqpoint{5.426708in}{2.085125in}}%
\pgfpathlineto{\pgfqpoint{5.433918in}{2.093913in}}%
\pgfpathlineto{\pgfqpoint{5.441121in}{2.102608in}}%
\pgfpathclose%
\pgfusepath{fill}%
\end{pgfscope}%
\begin{pgfscope}%
\pgfpathrectangle{\pgfqpoint{1.254980in}{0.150000in}}{\pgfqpoint{5.490039in}{5.490039in}}%
\pgfusepath{clip}%
\pgfsetbuttcap%
\pgfsetroundjoin%
\definecolor{currentfill}{rgb}{0.279566,0.067836,0.391917}%
\pgfsetfillcolor{currentfill}%
\pgfsetfillopacity{0.700000}%
\pgfsetlinewidth{0.000000pt}%
\definecolor{currentstroke}{rgb}{0.000000,0.000000,0.000000}%
\pgfsetstrokecolor{currentstroke}%
\pgfsetdash{}{0pt}%
\pgfpathmoveto{\pgfqpoint{3.534134in}{1.886737in}}%
\pgfpathlineto{\pgfqpoint{3.547345in}{1.880900in}}%
\pgfpathlineto{\pgfqpoint{3.560561in}{1.875090in}}%
\pgfpathlineto{\pgfqpoint{3.573781in}{1.869308in}}%
\pgfpathlineto{\pgfqpoint{3.587006in}{1.863554in}}%
\pgfpathlineto{\pgfqpoint{3.579071in}{1.861695in}}%
\pgfpathlineto{\pgfqpoint{3.571126in}{1.860087in}}%
\pgfpathlineto{\pgfqpoint{3.563170in}{1.858737in}}%
\pgfpathlineto{\pgfqpoint{3.555204in}{1.857653in}}%
\pgfpathlineto{\pgfqpoint{3.541956in}{1.863681in}}%
\pgfpathlineto{\pgfqpoint{3.528712in}{1.869736in}}%
\pgfpathlineto{\pgfqpoint{3.515473in}{1.875819in}}%
\pgfpathlineto{\pgfqpoint{3.502239in}{1.881930in}}%
\pgfpathlineto{\pgfqpoint{3.510229in}{1.882736in}}%
\pgfpathlineto{\pgfqpoint{3.518209in}{1.883810in}}%
\pgfpathlineto{\pgfqpoint{3.526177in}{1.885146in}}%
\pgfpathlineto{\pgfqpoint{3.534134in}{1.886737in}}%
\pgfpathclose%
\pgfusepath{fill}%
\end{pgfscope}%
\begin{pgfscope}%
\pgfpathrectangle{\pgfqpoint{1.254980in}{0.150000in}}{\pgfqpoint{5.490039in}{5.490039in}}%
\pgfusepath{clip}%
\pgfsetbuttcap%
\pgfsetroundjoin%
\definecolor{currentfill}{rgb}{0.282327,0.094955,0.417331}%
\pgfsetfillcolor{currentfill}%
\pgfsetfillopacity{0.700000}%
\pgfsetlinewidth{0.000000pt}%
\definecolor{currentstroke}{rgb}{0.000000,0.000000,0.000000}%
\pgfsetstrokecolor{currentstroke}%
\pgfsetdash{}{0pt}%
\pgfpathmoveto{\pgfqpoint{4.968004in}{1.943181in}}%
\pgfpathlineto{\pgfqpoint{4.981565in}{1.941627in}}%
\pgfpathlineto{\pgfqpoint{4.995135in}{1.940097in}}%
\pgfpathlineto{\pgfqpoint{5.008712in}{1.938590in}}%
\pgfpathlineto{\pgfqpoint{5.022298in}{1.937108in}}%
\pgfpathlineto{\pgfqpoint{5.014920in}{1.927391in}}%
\pgfpathlineto{\pgfqpoint{5.007536in}{1.917632in}}%
\pgfpathlineto{\pgfqpoint{5.000147in}{1.907833in}}%
\pgfpathlineto{\pgfqpoint{4.992752in}{1.897997in}}%
\pgfpathlineto{\pgfqpoint{4.979158in}{1.899599in}}%
\pgfpathlineto{\pgfqpoint{4.965572in}{1.901225in}}%
\pgfpathlineto{\pgfqpoint{4.951994in}{1.902874in}}%
\pgfpathlineto{\pgfqpoint{4.938424in}{1.904547in}}%
\pgfpathlineto{\pgfqpoint{4.945827in}{1.914259in}}%
\pgfpathlineto{\pgfqpoint{4.953225in}{1.923937in}}%
\pgfpathlineto{\pgfqpoint{4.960617in}{1.933578in}}%
\pgfpathlineto{\pgfqpoint{4.968004in}{1.943181in}}%
\pgfpathclose%
\pgfusepath{fill}%
\end{pgfscope}%
\begin{pgfscope}%
\pgfpathrectangle{\pgfqpoint{1.254980in}{0.150000in}}{\pgfqpoint{5.490039in}{5.490039in}}%
\pgfusepath{clip}%
\pgfsetbuttcap%
\pgfsetroundjoin%
\definecolor{currentfill}{rgb}{0.269944,0.014625,0.341379}%
\pgfsetfillcolor{currentfill}%
\pgfsetfillopacity{0.700000}%
\pgfsetlinewidth{0.000000pt}%
\definecolor{currentstroke}{rgb}{0.000000,0.000000,0.000000}%
\pgfsetstrokecolor{currentstroke}%
\pgfsetdash{}{0pt}%
\pgfpathmoveto{\pgfqpoint{4.357547in}{1.799640in}}%
\pgfpathlineto{\pgfqpoint{4.370933in}{1.796410in}}%
\pgfpathlineto{\pgfqpoint{4.384325in}{1.793204in}}%
\pgfpathlineto{\pgfqpoint{4.397724in}{1.790023in}}%
\pgfpathlineto{\pgfqpoint{4.411129in}{1.786867in}}%
\pgfpathlineto{\pgfqpoint{4.403555in}{1.778575in}}%
\pgfpathlineto{\pgfqpoint{4.395977in}{1.770355in}}%
\pgfpathlineto{\pgfqpoint{4.388393in}{1.762211in}}%
\pgfpathlineto{\pgfqpoint{4.380804in}{1.754148in}}%
\pgfpathlineto{\pgfqpoint{4.367387in}{1.757500in}}%
\pgfpathlineto{\pgfqpoint{4.353977in}{1.760877in}}%
\pgfpathlineto{\pgfqpoint{4.340574in}{1.764278in}}%
\pgfpathlineto{\pgfqpoint{4.327176in}{1.767704in}}%
\pgfpathlineto{\pgfqpoint{4.334777in}{1.775566in}}%
\pgfpathlineto{\pgfqpoint{4.342372in}{1.783513in}}%
\pgfpathlineto{\pgfqpoint{4.349962in}{1.791539in}}%
\pgfpathlineto{\pgfqpoint{4.357547in}{1.799640in}}%
\pgfpathclose%
\pgfusepath{fill}%
\end{pgfscope}%
\begin{pgfscope}%
\pgfpathrectangle{\pgfqpoint{1.254980in}{0.150000in}}{\pgfqpoint{5.490039in}{5.490039in}}%
\pgfusepath{clip}%
\pgfsetbuttcap%
\pgfsetroundjoin%
\definecolor{currentfill}{rgb}{0.233603,0.313828,0.543914}%
\pgfsetfillcolor{currentfill}%
\pgfsetfillopacity{0.700000}%
\pgfsetlinewidth{0.000000pt}%
\definecolor{currentstroke}{rgb}{0.000000,0.000000,0.000000}%
\pgfsetstrokecolor{currentstroke}%
\pgfsetdash{}{0pt}%
\pgfpathmoveto{\pgfqpoint{2.612766in}{2.375117in}}%
\pgfpathlineto{\pgfqpoint{2.625894in}{2.366056in}}%
\pgfpathlineto{\pgfqpoint{2.639025in}{2.357037in}}%
\pgfpathlineto{\pgfqpoint{2.652158in}{2.348060in}}%
\pgfpathlineto{\pgfqpoint{2.665293in}{2.339124in}}%
\pgfpathlineto{\pgfqpoint{2.656639in}{2.347100in}}%
\pgfpathlineto{\pgfqpoint{2.647962in}{2.355510in}}%
\pgfpathlineto{\pgfqpoint{2.639260in}{2.364365in}}%
\pgfpathlineto{\pgfqpoint{2.630534in}{2.373673in}}%
\pgfpathlineto{\pgfqpoint{2.617358in}{2.382945in}}%
\pgfpathlineto{\pgfqpoint{2.604184in}{2.392258in}}%
\pgfpathlineto{\pgfqpoint{2.591011in}{2.401613in}}%
\pgfpathlineto{\pgfqpoint{2.577841in}{2.411011in}}%
\pgfpathlineto{\pgfqpoint{2.586610in}{2.401361in}}%
\pgfpathlineto{\pgfqpoint{2.595353in}{2.392168in}}%
\pgfpathlineto{\pgfqpoint{2.604072in}{2.383423in}}%
\pgfpathlineto{\pgfqpoint{2.612766in}{2.375117in}}%
\pgfpathclose%
\pgfusepath{fill}%
\end{pgfscope}%
\begin{pgfscope}%
\pgfpathrectangle{\pgfqpoint{1.254980in}{0.150000in}}{\pgfqpoint{5.490039in}{5.490039in}}%
\pgfusepath{clip}%
\pgfsetbuttcap%
\pgfsetroundjoin%
\definecolor{currentfill}{rgb}{0.267968,0.223549,0.512008}%
\pgfsetfillcolor{currentfill}%
\pgfsetfillopacity{0.700000}%
\pgfsetlinewidth{0.000000pt}%
\definecolor{currentstroke}{rgb}{0.000000,0.000000,0.000000}%
\pgfsetstrokecolor{currentstroke}%
\pgfsetdash{}{0pt}%
\pgfpathmoveto{\pgfqpoint{2.909564in}{2.178908in}}%
\pgfpathlineto{\pgfqpoint{2.922705in}{2.170947in}}%
\pgfpathlineto{\pgfqpoint{2.935849in}{2.163021in}}%
\pgfpathlineto{\pgfqpoint{2.948997in}{2.155130in}}%
\pgfpathlineto{\pgfqpoint{2.962147in}{2.147275in}}%
\pgfpathlineto{\pgfqpoint{2.953763in}{2.152034in}}%
\pgfpathlineto{\pgfqpoint{2.945360in}{2.157173in}}%
\pgfpathlineto{\pgfqpoint{2.936937in}{2.162701in}}%
\pgfpathlineto{\pgfqpoint{2.928495in}{2.168626in}}%
\pgfpathlineto{\pgfqpoint{2.915310in}{2.176800in}}%
\pgfpathlineto{\pgfqpoint{2.902127in}{2.185008in}}%
\pgfpathlineto{\pgfqpoint{2.888948in}{2.193252in}}%
\pgfpathlineto{\pgfqpoint{2.875771in}{2.201532in}}%
\pgfpathlineto{\pgfqpoint{2.884249in}{2.195283in}}%
\pgfpathlineto{\pgfqpoint{2.892707in}{2.189435in}}%
\pgfpathlineto{\pgfqpoint{2.901145in}{2.183979in}}%
\pgfpathlineto{\pgfqpoint{2.909564in}{2.178908in}}%
\pgfpathclose%
\pgfusepath{fill}%
\end{pgfscope}%
\begin{pgfscope}%
\pgfpathrectangle{\pgfqpoint{1.254980in}{0.150000in}}{\pgfqpoint{5.490039in}{5.490039in}}%
\pgfusepath{clip}%
\pgfsetbuttcap%
\pgfsetroundjoin%
\definecolor{currentfill}{rgb}{0.281412,0.155834,0.469201}%
\pgfsetfillcolor{currentfill}%
\pgfsetfillopacity{0.700000}%
\pgfsetlinewidth{0.000000pt}%
\definecolor{currentstroke}{rgb}{0.000000,0.000000,0.000000}%
\pgfsetstrokecolor{currentstroke}%
\pgfsetdash{}{0pt}%
\pgfpathmoveto{\pgfqpoint{3.153178in}{2.043902in}}%
\pgfpathlineto{\pgfqpoint{3.166340in}{2.036785in}}%
\pgfpathlineto{\pgfqpoint{3.179506in}{2.029700in}}%
\pgfpathlineto{\pgfqpoint{3.192676in}{2.022647in}}%
\pgfpathlineto{\pgfqpoint{3.205850in}{2.015626in}}%
\pgfpathlineto{\pgfqpoint{3.197660in}{2.017759in}}%
\pgfpathlineto{\pgfqpoint{3.189455in}{2.020224in}}%
\pgfpathlineto{\pgfqpoint{3.181234in}{2.023030in}}%
\pgfpathlineto{\pgfqpoint{3.172997in}{2.026185in}}%
\pgfpathlineto{\pgfqpoint{3.159793in}{2.033509in}}%
\pgfpathlineto{\pgfqpoint{3.146593in}{2.040864in}}%
\pgfpathlineto{\pgfqpoint{3.133396in}{2.048251in}}%
\pgfpathlineto{\pgfqpoint{3.120203in}{2.055671in}}%
\pgfpathlineto{\pgfqpoint{3.128471in}{2.052208in}}%
\pgfpathlineto{\pgfqpoint{3.136722in}{2.049098in}}%
\pgfpathlineto{\pgfqpoint{3.144958in}{2.046332in}}%
\pgfpathlineto{\pgfqpoint{3.153178in}{2.043902in}}%
\pgfpathclose%
\pgfusepath{fill}%
\end{pgfscope}%
\begin{pgfscope}%
\pgfpathrectangle{\pgfqpoint{1.254980in}{0.150000in}}{\pgfqpoint{5.490039in}{5.490039in}}%
\pgfusepath{clip}%
\pgfsetbuttcap%
\pgfsetroundjoin%
\definecolor{currentfill}{rgb}{0.273809,0.031497,0.358853}%
\pgfsetfillcolor{currentfill}%
\pgfsetfillopacity{0.700000}%
\pgfsetlinewidth{0.000000pt}%
\definecolor{currentstroke}{rgb}{0.000000,0.000000,0.000000}%
\pgfsetstrokecolor{currentstroke}%
\pgfsetdash{}{0pt}%
\pgfpathmoveto{\pgfqpoint{4.578861in}{1.833665in}}%
\pgfpathlineto{\pgfqpoint{4.592308in}{1.831080in}}%
\pgfpathlineto{\pgfqpoint{4.605762in}{1.828518in}}%
\pgfpathlineto{\pgfqpoint{4.619223in}{1.825980in}}%
\pgfpathlineto{\pgfqpoint{4.632692in}{1.823467in}}%
\pgfpathlineto{\pgfqpoint{4.625188in}{1.814269in}}%
\pgfpathlineto{\pgfqpoint{4.617679in}{1.805099in}}%
\pgfpathlineto{\pgfqpoint{4.610165in}{1.795959in}}%
\pgfpathlineto{\pgfqpoint{4.602647in}{1.786854in}}%
\pgfpathlineto{\pgfqpoint{4.589169in}{1.789539in}}%
\pgfpathlineto{\pgfqpoint{4.575698in}{1.792247in}}%
\pgfpathlineto{\pgfqpoint{4.562234in}{1.794979in}}%
\pgfpathlineto{\pgfqpoint{4.548778in}{1.797735in}}%
\pgfpathlineto{\pgfqpoint{4.556306in}{1.806665in}}%
\pgfpathlineto{\pgfqpoint{4.563829in}{1.815632in}}%
\pgfpathlineto{\pgfqpoint{4.571347in}{1.824634in}}%
\pgfpathlineto{\pgfqpoint{4.578861in}{1.833665in}}%
\pgfpathclose%
\pgfusepath{fill}%
\end{pgfscope}%
\begin{pgfscope}%
\pgfpathrectangle{\pgfqpoint{1.254980in}{0.150000in}}{\pgfqpoint{5.490039in}{5.490039in}}%
\pgfusepath{clip}%
\pgfsetbuttcap%
\pgfsetroundjoin%
\definecolor{currentfill}{rgb}{0.268510,0.009605,0.335427}%
\pgfsetfillcolor{currentfill}%
\pgfsetfillopacity{0.700000}%
\pgfsetlinewidth{0.000000pt}%
\definecolor{currentstroke}{rgb}{0.000000,0.000000,0.000000}%
\pgfsetstrokecolor{currentstroke}%
\pgfsetdash{}{0pt}%
\pgfpathmoveto{\pgfqpoint{3.999028in}{1.790727in}}%
\pgfpathlineto{\pgfqpoint{4.012329in}{1.786375in}}%
\pgfpathlineto{\pgfqpoint{4.025636in}{1.782049in}}%
\pgfpathlineto{\pgfqpoint{4.038948in}{1.777747in}}%
\pgfpathlineto{\pgfqpoint{4.052267in}{1.773472in}}%
\pgfpathlineto{\pgfqpoint{4.044561in}{1.767516in}}%
\pgfpathlineto{\pgfqpoint{4.036848in}{1.761712in}}%
\pgfpathlineto{\pgfqpoint{4.029129in}{1.756064in}}%
\pgfpathlineto{\pgfqpoint{4.021403in}{1.750578in}}%
\pgfpathlineto{\pgfqpoint{4.008069in}{1.755088in}}%
\pgfpathlineto{\pgfqpoint{3.994740in}{1.759623in}}%
\pgfpathlineto{\pgfqpoint{3.981418in}{1.764184in}}%
\pgfpathlineto{\pgfqpoint{3.968101in}{1.768770in}}%
\pgfpathlineto{\pgfqpoint{3.975843in}{1.774016in}}%
\pgfpathlineto{\pgfqpoint{3.983578in}{1.779429in}}%
\pgfpathlineto{\pgfqpoint{3.991306in}{1.785001in}}%
\pgfpathlineto{\pgfqpoint{3.999028in}{1.790727in}}%
\pgfpathclose%
\pgfusepath{fill}%
\end{pgfscope}%
\begin{pgfscope}%
\pgfpathrectangle{\pgfqpoint{1.254980in}{0.150000in}}{\pgfqpoint{5.490039in}{5.490039in}}%
\pgfusepath{clip}%
\pgfsetbuttcap%
\pgfsetroundjoin%
\definecolor{currentfill}{rgb}{0.271305,0.019942,0.347269}%
\pgfsetfillcolor{currentfill}%
\pgfsetfillopacity{0.700000}%
\pgfsetlinewidth{0.000000pt}%
\definecolor{currentstroke}{rgb}{0.000000,0.000000,0.000000}%
\pgfsetstrokecolor{currentstroke}%
\pgfsetdash{}{0pt}%
\pgfpathmoveto{\pgfqpoint{3.861766in}{1.806378in}}%
\pgfpathlineto{\pgfqpoint{3.875039in}{1.801586in}}%
\pgfpathlineto{\pgfqpoint{3.888317in}{1.796821in}}%
\pgfpathlineto{\pgfqpoint{3.901600in}{1.792081in}}%
\pgfpathlineto{\pgfqpoint{3.914889in}{1.787368in}}%
\pgfpathlineto{\pgfqpoint{3.907123in}{1.782535in}}%
\pgfpathlineto{\pgfqpoint{3.899349in}{1.777883in}}%
\pgfpathlineto{\pgfqpoint{3.891568in}{1.773420in}}%
\pgfpathlineto{\pgfqpoint{3.883779in}{1.769150in}}%
\pgfpathlineto{\pgfqpoint{3.870473in}{1.774110in}}%
\pgfpathlineto{\pgfqpoint{3.857171in}{1.779097in}}%
\pgfpathlineto{\pgfqpoint{3.843876in}{1.784109in}}%
\pgfpathlineto{\pgfqpoint{3.830585in}{1.789147in}}%
\pgfpathlineto{\pgfqpoint{3.838392in}{1.793165in}}%
\pgfpathlineto{\pgfqpoint{3.846191in}{1.797381in}}%
\pgfpathlineto{\pgfqpoint{3.853983in}{1.801787in}}%
\pgfpathlineto{\pgfqpoint{3.861766in}{1.806378in}}%
\pgfpathclose%
\pgfusepath{fill}%
\end{pgfscope}%
\begin{pgfscope}%
\pgfpathrectangle{\pgfqpoint{1.254980in}{0.150000in}}{\pgfqpoint{5.490039in}{5.490039in}}%
\pgfusepath{clip}%
\pgfsetbuttcap%
\pgfsetroundjoin%
\definecolor{currentfill}{rgb}{0.280255,0.165693,0.476498}%
\pgfsetfillcolor{currentfill}%
\pgfsetfillopacity{0.700000}%
\pgfsetlinewidth{0.000000pt}%
\definecolor{currentstroke}{rgb}{0.000000,0.000000,0.000000}%
\pgfsetstrokecolor{currentstroke}%
\pgfsetdash{}{0pt}%
\pgfpathmoveto{\pgfqpoint{5.357449in}{2.070059in}}%
\pgfpathlineto{\pgfqpoint{5.371140in}{2.069327in}}%
\pgfpathlineto{\pgfqpoint{5.384839in}{2.068618in}}%
\pgfpathlineto{\pgfqpoint{5.398548in}{2.067933in}}%
\pgfpathlineto{\pgfqpoint{5.412264in}{2.067272in}}%
\pgfpathlineto{\pgfqpoint{5.405032in}{2.058210in}}%
\pgfpathlineto{\pgfqpoint{5.397792in}{2.049059in}}%
\pgfpathlineto{\pgfqpoint{5.390546in}{2.039820in}}%
\pgfpathlineto{\pgfqpoint{5.383292in}{2.030495in}}%
\pgfpathlineto{\pgfqpoint{5.369566in}{2.031223in}}%
\pgfpathlineto{\pgfqpoint{5.355848in}{2.031974in}}%
\pgfpathlineto{\pgfqpoint{5.342140in}{2.032750in}}%
\pgfpathlineto{\pgfqpoint{5.328439in}{2.033549in}}%
\pgfpathlineto{\pgfqpoint{5.335702in}{2.042803in}}%
\pgfpathlineto{\pgfqpoint{5.342958in}{2.051973in}}%
\pgfpathlineto{\pgfqpoint{5.350207in}{2.061059in}}%
\pgfpathlineto{\pgfqpoint{5.357449in}{2.070059in}}%
\pgfpathclose%
\pgfusepath{fill}%
\end{pgfscope}%
\begin{pgfscope}%
\pgfpathrectangle{\pgfqpoint{1.254980in}{0.150000in}}{\pgfqpoint{5.490039in}{5.490039in}}%
\pgfusepath{clip}%
\pgfsetbuttcap%
\pgfsetroundjoin%
\definecolor{currentfill}{rgb}{0.280894,0.078907,0.402329}%
\pgfsetfillcolor{currentfill}%
\pgfsetfillopacity{0.700000}%
\pgfsetlinewidth{0.000000pt}%
\definecolor{currentstroke}{rgb}{0.000000,0.000000,0.000000}%
\pgfsetstrokecolor{currentstroke}%
\pgfsetdash{}{0pt}%
\pgfpathmoveto{\pgfqpoint{4.884222in}{1.911478in}}%
\pgfpathlineto{\pgfqpoint{4.897761in}{1.909709in}}%
\pgfpathlineto{\pgfqpoint{4.911308in}{1.907965in}}%
\pgfpathlineto{\pgfqpoint{4.924862in}{1.906244in}}%
\pgfpathlineto{\pgfqpoint{4.938424in}{1.904547in}}%
\pgfpathlineto{\pgfqpoint{4.931016in}{1.894805in}}%
\pgfpathlineto{\pgfqpoint{4.923602in}{1.885034in}}%
\pgfpathlineto{\pgfqpoint{4.916183in}{1.875239in}}%
\pgfpathlineto{\pgfqpoint{4.908759in}{1.865421in}}%
\pgfpathlineto{\pgfqpoint{4.895188in}{1.867250in}}%
\pgfpathlineto{\pgfqpoint{4.881626in}{1.869103in}}%
\pgfpathlineto{\pgfqpoint{4.868071in}{1.870980in}}%
\pgfpathlineto{\pgfqpoint{4.854523in}{1.872881in}}%
\pgfpathlineto{\pgfqpoint{4.861956in}{1.882561in}}%
\pgfpathlineto{\pgfqpoint{4.869383in}{1.892223in}}%
\pgfpathlineto{\pgfqpoint{4.876805in}{1.901862in}}%
\pgfpathlineto{\pgfqpoint{4.884222in}{1.911478in}}%
\pgfpathclose%
\pgfusepath{fill}%
\end{pgfscope}%
\begin{pgfscope}%
\pgfpathrectangle{\pgfqpoint{1.254980in}{0.150000in}}{\pgfqpoint{5.490039in}{5.490039in}}%
\pgfusepath{clip}%
\pgfsetbuttcap%
\pgfsetroundjoin%
\definecolor{currentfill}{rgb}{0.268510,0.009605,0.335427}%
\pgfsetfillcolor{currentfill}%
\pgfsetfillopacity{0.700000}%
\pgfsetlinewidth{0.000000pt}%
\definecolor{currentstroke}{rgb}{0.000000,0.000000,0.000000}%
\pgfsetstrokecolor{currentstroke}%
\pgfsetdash{}{0pt}%
\pgfpathmoveto{\pgfqpoint{4.136304in}{1.782716in}}%
\pgfpathlineto{\pgfqpoint{4.149638in}{1.778786in}}%
\pgfpathlineto{\pgfqpoint{4.162978in}{1.774882in}}%
\pgfpathlineto{\pgfqpoint{4.176324in}{1.771002in}}%
\pgfpathlineto{\pgfqpoint{4.189676in}{1.767147in}}%
\pgfpathlineto{\pgfqpoint{4.182023in}{1.760214in}}%
\pgfpathlineto{\pgfqpoint{4.174364in}{1.753402in}}%
\pgfpathlineto{\pgfqpoint{4.166700in}{1.746717in}}%
\pgfpathlineto{\pgfqpoint{4.159029in}{1.740165in}}%
\pgfpathlineto{\pgfqpoint{4.145663in}{1.744242in}}%
\pgfpathlineto{\pgfqpoint{4.132303in}{1.748343in}}%
\pgfpathlineto{\pgfqpoint{4.118948in}{1.752468in}}%
\pgfpathlineto{\pgfqpoint{4.105600in}{1.756619in}}%
\pgfpathlineto{\pgfqpoint{4.113285in}{1.762945in}}%
\pgfpathlineto{\pgfqpoint{4.120964in}{1.769407in}}%
\pgfpathlineto{\pgfqpoint{4.128637in}{1.775999in}}%
\pgfpathlineto{\pgfqpoint{4.136304in}{1.782716in}}%
\pgfpathclose%
\pgfusepath{fill}%
\end{pgfscope}%
\begin{pgfscope}%
\pgfpathrectangle{\pgfqpoint{1.254980in}{0.150000in}}{\pgfqpoint{5.490039in}{5.490039in}}%
\pgfusepath{clip}%
\pgfsetbuttcap%
\pgfsetroundjoin%
\definecolor{currentfill}{rgb}{0.274952,0.037752,0.364543}%
\pgfsetfillcolor{currentfill}%
\pgfsetfillopacity{0.700000}%
\pgfsetlinewidth{0.000000pt}%
\definecolor{currentstroke}{rgb}{0.000000,0.000000,0.000000}%
\pgfsetstrokecolor{currentstroke}%
\pgfsetdash{}{0pt}%
\pgfpathmoveto{\pgfqpoint{3.724451in}{1.830398in}}%
\pgfpathlineto{\pgfqpoint{3.737699in}{1.825148in}}%
\pgfpathlineto{\pgfqpoint{3.750953in}{1.819926in}}%
\pgfpathlineto{\pgfqpoint{3.764212in}{1.814730in}}%
\pgfpathlineto{\pgfqpoint{3.777476in}{1.809561in}}%
\pgfpathlineto{\pgfqpoint{3.769642in}{1.806002in}}%
\pgfpathlineto{\pgfqpoint{3.761799in}{1.802656in}}%
\pgfpathlineto{\pgfqpoint{3.753947in}{1.799531in}}%
\pgfpathlineto{\pgfqpoint{3.746086in}{1.796632in}}%
\pgfpathlineto{\pgfqpoint{3.732802in}{1.802062in}}%
\pgfpathlineto{\pgfqpoint{3.719523in}{1.807517in}}%
\pgfpathlineto{\pgfqpoint{3.706249in}{1.813000in}}%
\pgfpathlineto{\pgfqpoint{3.692980in}{1.818509in}}%
\pgfpathlineto{\pgfqpoint{3.700862in}{1.821142in}}%
\pgfpathlineto{\pgfqpoint{3.708734in}{1.824006in}}%
\pgfpathlineto{\pgfqpoint{3.716597in}{1.827093in}}%
\pgfpathlineto{\pgfqpoint{3.724451in}{1.830398in}}%
\pgfpathclose%
\pgfusepath{fill}%
\end{pgfscope}%
\begin{pgfscope}%
\pgfpathrectangle{\pgfqpoint{1.254980in}{0.150000in}}{\pgfqpoint{5.490039in}{5.490039in}}%
\pgfusepath{clip}%
\pgfsetbuttcap%
\pgfsetroundjoin%
\definecolor{currentfill}{rgb}{0.281887,0.150881,0.465405}%
\pgfsetfillcolor{currentfill}%
\pgfsetfillopacity{0.700000}%
\pgfsetlinewidth{0.000000pt}%
\definecolor{currentstroke}{rgb}{0.000000,0.000000,0.000000}%
\pgfsetstrokecolor{currentstroke}%
\pgfsetdash{}{0pt}%
\pgfpathmoveto{\pgfqpoint{5.273723in}{2.036983in}}%
\pgfpathlineto{\pgfqpoint{5.287390in}{2.036089in}}%
\pgfpathlineto{\pgfqpoint{5.301064in}{2.035219in}}%
\pgfpathlineto{\pgfqpoint{5.314748in}{2.034372in}}%
\pgfpathlineto{\pgfqpoint{5.328439in}{2.033549in}}%
\pgfpathlineto{\pgfqpoint{5.321170in}{2.024214in}}%
\pgfpathlineto{\pgfqpoint{5.313893in}{2.014798in}}%
\pgfpathlineto{\pgfqpoint{5.306610in}{2.005303in}}%
\pgfpathlineto{\pgfqpoint{5.299320in}{1.995732in}}%
\pgfpathlineto{\pgfqpoint{5.285619in}{1.996635in}}%
\pgfpathlineto{\pgfqpoint{5.271927in}{1.997562in}}%
\pgfpathlineto{\pgfqpoint{5.258244in}{1.998512in}}%
\pgfpathlineto{\pgfqpoint{5.244569in}{1.999486in}}%
\pgfpathlineto{\pgfqpoint{5.251867in}{2.008973in}}%
\pgfpathlineto{\pgfqpoint{5.259159in}{2.018386in}}%
\pgfpathlineto{\pgfqpoint{5.266445in}{2.027723in}}%
\pgfpathlineto{\pgfqpoint{5.273723in}{2.036983in}}%
\pgfpathclose%
\pgfusepath{fill}%
\end{pgfscope}%
\begin{pgfscope}%
\pgfpathrectangle{\pgfqpoint{1.254980in}{0.150000in}}{\pgfqpoint{5.490039in}{5.490039in}}%
\pgfusepath{clip}%
\pgfsetbuttcap%
\pgfsetroundjoin%
\definecolor{currentfill}{rgb}{0.270595,0.214069,0.507052}%
\pgfsetfillcolor{currentfill}%
\pgfsetfillopacity{0.700000}%
\pgfsetlinewidth{0.000000pt}%
\definecolor{currentstroke}{rgb}{0.000000,0.000000,0.000000}%
\pgfsetstrokecolor{currentstroke}%
\pgfsetdash{}{0pt}%
\pgfpathmoveto{\pgfqpoint{5.663381in}{2.164113in}}%
\pgfpathlineto{\pgfqpoint{5.677181in}{2.163882in}}%
\pgfpathlineto{\pgfqpoint{5.690991in}{2.163674in}}%
\pgfpathlineto{\pgfqpoint{5.704809in}{2.163490in}}%
\pgfpathlineto{\pgfqpoint{5.697710in}{2.155478in}}%
\pgfpathlineto{\pgfqpoint{5.690603in}{2.147356in}}%
\pgfpathlineto{\pgfqpoint{5.683487in}{2.139126in}}%
\pgfpathlineto{\pgfqpoint{5.676363in}{2.130786in}}%
\pgfpathlineto{\pgfqpoint{5.662533in}{2.130997in}}%
\pgfpathlineto{\pgfqpoint{5.648713in}{2.131232in}}%
\pgfpathlineto{\pgfqpoint{5.634901in}{2.131490in}}%
\pgfpathlineto{\pgfqpoint{5.642034in}{2.139806in}}%
\pgfpathlineto{\pgfqpoint{5.649158in}{2.148015in}}%
\pgfpathlineto{\pgfqpoint{5.656274in}{2.156118in}}%
\pgfpathlineto{\pgfqpoint{5.663381in}{2.164113in}}%
\pgfpathclose%
\pgfusepath{fill}%
\end{pgfscope}%
\begin{pgfscope}%
\pgfpathrectangle{\pgfqpoint{1.254980in}{0.150000in}}{\pgfqpoint{5.490039in}{5.490039in}}%
\pgfusepath{clip}%
\pgfsetbuttcap%
\pgfsetroundjoin%
\definecolor{currentfill}{rgb}{0.279566,0.067836,0.391917}%
\pgfsetfillcolor{currentfill}%
\pgfsetfillopacity{0.700000}%
\pgfsetlinewidth{0.000000pt}%
\definecolor{currentstroke}{rgb}{0.000000,0.000000,0.000000}%
\pgfsetstrokecolor{currentstroke}%
\pgfsetdash{}{0pt}%
\pgfpathmoveto{\pgfqpoint{4.800410in}{1.880721in}}%
\pgfpathlineto{\pgfqpoint{4.813927in}{1.878725in}}%
\pgfpathlineto{\pgfqpoint{4.827452in}{1.876753in}}%
\pgfpathlineto{\pgfqpoint{4.840984in}{1.874805in}}%
\pgfpathlineto{\pgfqpoint{4.854523in}{1.872881in}}%
\pgfpathlineto{\pgfqpoint{4.847086in}{1.863185in}}%
\pgfpathlineto{\pgfqpoint{4.839643in}{1.853477in}}%
\pgfpathlineto{\pgfqpoint{4.832195in}{1.843760in}}%
\pgfpathlineto{\pgfqpoint{4.824742in}{1.834037in}}%
\pgfpathlineto{\pgfqpoint{4.811194in}{1.836107in}}%
\pgfpathlineto{\pgfqpoint{4.797653in}{1.838200in}}%
\pgfpathlineto{\pgfqpoint{4.784120in}{1.840317in}}%
\pgfpathlineto{\pgfqpoint{4.770595in}{1.842458in}}%
\pgfpathlineto{\pgfqpoint{4.778056in}{1.852030in}}%
\pgfpathlineto{\pgfqpoint{4.785513in}{1.861601in}}%
\pgfpathlineto{\pgfqpoint{4.792964in}{1.871165in}}%
\pgfpathlineto{\pgfqpoint{4.800410in}{1.880721in}}%
\pgfpathclose%
\pgfusepath{fill}%
\end{pgfscope}%
\begin{pgfscope}%
\pgfpathrectangle{\pgfqpoint{1.254980in}{0.150000in}}{\pgfqpoint{5.490039in}{5.490039in}}%
\pgfusepath{clip}%
\pgfsetbuttcap%
\pgfsetroundjoin%
\definecolor{currentfill}{rgb}{0.272594,0.025563,0.353093}%
\pgfsetfillcolor{currentfill}%
\pgfsetfillopacity{0.700000}%
\pgfsetlinewidth{0.000000pt}%
\definecolor{currentstroke}{rgb}{0.000000,0.000000,0.000000}%
\pgfsetstrokecolor{currentstroke}%
\pgfsetdash{}{0pt}%
\pgfpathmoveto{\pgfqpoint{4.495020in}{1.809001in}}%
\pgfpathlineto{\pgfqpoint{4.508449in}{1.806148in}}%
\pgfpathlineto{\pgfqpoint{4.521885in}{1.803320in}}%
\pgfpathlineto{\pgfqpoint{4.535328in}{1.800515in}}%
\pgfpathlineto{\pgfqpoint{4.548778in}{1.797735in}}%
\pgfpathlineto{\pgfqpoint{4.541245in}{1.788848in}}%
\pgfpathlineto{\pgfqpoint{4.533707in}{1.780008in}}%
\pgfpathlineto{\pgfqpoint{4.526164in}{1.771218in}}%
\pgfpathlineto{\pgfqpoint{4.518617in}{1.762484in}}%
\pgfpathlineto{\pgfqpoint{4.505157in}{1.765448in}}%
\pgfpathlineto{\pgfqpoint{4.491704in}{1.768435in}}%
\pgfpathlineto{\pgfqpoint{4.478258in}{1.771447in}}%
\pgfpathlineto{\pgfqpoint{4.464818in}{1.774482in}}%
\pgfpathlineto{\pgfqpoint{4.472376in}{1.783028in}}%
\pgfpathlineto{\pgfqpoint{4.479929in}{1.791633in}}%
\pgfpathlineto{\pgfqpoint{4.487477in}{1.800292in}}%
\pgfpathlineto{\pgfqpoint{4.495020in}{1.809001in}}%
\pgfpathclose%
\pgfusepath{fill}%
\end{pgfscope}%
\begin{pgfscope}%
\pgfpathrectangle{\pgfqpoint{1.254980in}{0.150000in}}{\pgfqpoint{5.490039in}{5.490039in}}%
\pgfusepath{clip}%
\pgfsetbuttcap%
\pgfsetroundjoin%
\definecolor{currentfill}{rgb}{0.239346,0.300855,0.540844}%
\pgfsetfillcolor{currentfill}%
\pgfsetfillopacity{0.700000}%
\pgfsetlinewidth{0.000000pt}%
\definecolor{currentstroke}{rgb}{0.000000,0.000000,0.000000}%
\pgfsetstrokecolor{currentstroke}%
\pgfsetdash{}{0pt}%
\pgfpathmoveto{\pgfqpoint{2.665293in}{2.339124in}}%
\pgfpathlineto{\pgfqpoint{2.678430in}{2.330230in}}%
\pgfpathlineto{\pgfqpoint{2.691569in}{2.321377in}}%
\pgfpathlineto{\pgfqpoint{2.704711in}{2.312564in}}%
\pgfpathlineto{\pgfqpoint{2.717854in}{2.303791in}}%
\pgfpathlineto{\pgfqpoint{2.709240in}{2.311437in}}%
\pgfpathlineto{\pgfqpoint{2.700603in}{2.319514in}}%
\pgfpathlineto{\pgfqpoint{2.691943in}{2.328031in}}%
\pgfpathlineto{\pgfqpoint{2.683258in}{2.336998in}}%
\pgfpathlineto{\pgfqpoint{2.670074in}{2.346106in}}%
\pgfpathlineto{\pgfqpoint{2.656892in}{2.355254in}}%
\pgfpathlineto{\pgfqpoint{2.643712in}{2.364443in}}%
\pgfpathlineto{\pgfqpoint{2.630534in}{2.373673in}}%
\pgfpathlineto{\pgfqpoint{2.639260in}{2.364365in}}%
\pgfpathlineto{\pgfqpoint{2.647962in}{2.355510in}}%
\pgfpathlineto{\pgfqpoint{2.656639in}{2.347100in}}%
\pgfpathlineto{\pgfqpoint{2.665293in}{2.339124in}}%
\pgfpathclose%
\pgfusepath{fill}%
\end{pgfscope}%
\begin{pgfscope}%
\pgfpathrectangle{\pgfqpoint{1.254980in}{0.150000in}}{\pgfqpoint{5.490039in}{5.490039in}}%
\pgfusepath{clip}%
\pgfsetbuttcap%
\pgfsetroundjoin%
\definecolor{currentfill}{rgb}{0.282656,0.100196,0.422160}%
\pgfsetfillcolor{currentfill}%
\pgfsetfillopacity{0.700000}%
\pgfsetlinewidth{0.000000pt}%
\definecolor{currentstroke}{rgb}{0.000000,0.000000,0.000000}%
\pgfsetstrokecolor{currentstroke}%
\pgfsetdash{}{0pt}%
\pgfpathmoveto{\pgfqpoint{3.396525in}{1.931845in}}%
\pgfpathlineto{\pgfqpoint{3.409724in}{1.925505in}}%
\pgfpathlineto{\pgfqpoint{3.422927in}{1.919193in}}%
\pgfpathlineto{\pgfqpoint{3.436134in}{1.912911in}}%
\pgfpathlineto{\pgfqpoint{3.449346in}{1.906658in}}%
\pgfpathlineto{\pgfqpoint{3.441320in}{1.906411in}}%
\pgfpathlineto{\pgfqpoint{3.433281in}{1.906451in}}%
\pgfpathlineto{\pgfqpoint{3.425229in}{1.906786in}}%
\pgfpathlineto{\pgfqpoint{3.417166in}{1.907423in}}%
\pgfpathlineto{\pgfqpoint{3.403927in}{1.913964in}}%
\pgfpathlineto{\pgfqpoint{3.390694in}{1.920533in}}%
\pgfpathlineto{\pgfqpoint{3.377464in}{1.927132in}}%
\pgfpathlineto{\pgfqpoint{3.364239in}{1.933760in}}%
\pgfpathlineto{\pgfqpoint{3.372330in}{1.932830in}}%
\pgfpathlineto{\pgfqpoint{3.380407in}{1.932206in}}%
\pgfpathlineto{\pgfqpoint{3.388472in}{1.931880in}}%
\pgfpathlineto{\pgfqpoint{3.396525in}{1.931845in}}%
\pgfpathclose%
\pgfusepath{fill}%
\end{pgfscope}%
\begin{pgfscope}%
\pgfpathrectangle{\pgfqpoint{1.254980in}{0.150000in}}{\pgfqpoint{5.490039in}{5.490039in}}%
\pgfusepath{clip}%
\pgfsetbuttcap%
\pgfsetroundjoin%
\definecolor{currentfill}{rgb}{0.268510,0.009605,0.335427}%
\pgfsetfillcolor{currentfill}%
\pgfsetfillopacity{0.700000}%
\pgfsetlinewidth{0.000000pt}%
\definecolor{currentstroke}{rgb}{0.000000,0.000000,0.000000}%
\pgfsetstrokecolor{currentstroke}%
\pgfsetdash{}{0pt}%
\pgfpathmoveto{\pgfqpoint{4.273653in}{1.781650in}}%
\pgfpathlineto{\pgfqpoint{4.287024in}{1.778126in}}%
\pgfpathlineto{\pgfqpoint{4.300402in}{1.774628in}}%
\pgfpathlineto{\pgfqpoint{4.313786in}{1.771153in}}%
\pgfpathlineto{\pgfqpoint{4.327176in}{1.767704in}}%
\pgfpathlineto{\pgfqpoint{4.319571in}{1.759930in}}%
\pgfpathlineto{\pgfqpoint{4.311960in}{1.752250in}}%
\pgfpathlineto{\pgfqpoint{4.304343in}{1.744670in}}%
\pgfpathlineto{\pgfqpoint{4.296722in}{1.737194in}}%
\pgfpathlineto{\pgfqpoint{4.283319in}{1.740852in}}%
\pgfpathlineto{\pgfqpoint{4.269923in}{1.744535in}}%
\pgfpathlineto{\pgfqpoint{4.256532in}{1.748242in}}%
\pgfpathlineto{\pgfqpoint{4.243149in}{1.751974in}}%
\pgfpathlineto{\pgfqpoint{4.250783in}{1.759237in}}%
\pgfpathlineto{\pgfqpoint{4.258412in}{1.766607in}}%
\pgfpathlineto{\pgfqpoint{4.266035in}{1.774080in}}%
\pgfpathlineto{\pgfqpoint{4.273653in}{1.781650in}}%
\pgfpathclose%
\pgfusepath{fill}%
\end{pgfscope}%
\begin{pgfscope}%
\pgfpathrectangle{\pgfqpoint{1.254980in}{0.150000in}}{\pgfqpoint{5.490039in}{5.490039in}}%
\pgfusepath{clip}%
\pgfsetbuttcap%
\pgfsetroundjoin%
\definecolor{currentfill}{rgb}{0.282884,0.135920,0.453427}%
\pgfsetfillcolor{currentfill}%
\pgfsetfillopacity{0.700000}%
\pgfsetlinewidth{0.000000pt}%
\definecolor{currentstroke}{rgb}{0.000000,0.000000,0.000000}%
\pgfsetstrokecolor{currentstroke}%
\pgfsetdash{}{0pt}%
\pgfpathmoveto{\pgfqpoint{5.189952in}{2.003621in}}%
\pgfpathlineto{\pgfqpoint{5.203594in}{2.002552in}}%
\pgfpathlineto{\pgfqpoint{5.217244in}{2.001506in}}%
\pgfpathlineto{\pgfqpoint{5.230902in}{2.000485in}}%
\pgfpathlineto{\pgfqpoint{5.244569in}{1.999486in}}%
\pgfpathlineto{\pgfqpoint{5.237264in}{1.989928in}}%
\pgfpathlineto{\pgfqpoint{5.229952in}{1.980299in}}%
\pgfpathlineto{\pgfqpoint{5.222635in}{1.970602in}}%
\pgfpathlineto{\pgfqpoint{5.215311in}{1.960839in}}%
\pgfpathlineto{\pgfqpoint{5.201636in}{1.961931in}}%
\pgfpathlineto{\pgfqpoint{5.187969in}{1.963046in}}%
\pgfpathlineto{\pgfqpoint{5.174310in}{1.964184in}}%
\pgfpathlineto{\pgfqpoint{5.160660in}{1.965347in}}%
\pgfpathlineto{\pgfqpoint{5.167993in}{1.975012in}}%
\pgfpathlineto{\pgfqpoint{5.175319in}{1.984614in}}%
\pgfpathlineto{\pgfqpoint{5.182638in}{1.994151in}}%
\pgfpathlineto{\pgfqpoint{5.189952in}{2.003621in}}%
\pgfpathclose%
\pgfusepath{fill}%
\end{pgfscope}%
\begin{pgfscope}%
\pgfpathrectangle{\pgfqpoint{1.254980in}{0.150000in}}{\pgfqpoint{5.490039in}{5.490039in}}%
\pgfusepath{clip}%
\pgfsetbuttcap%
\pgfsetroundjoin%
\definecolor{currentfill}{rgb}{0.270595,0.214069,0.507052}%
\pgfsetfillcolor{currentfill}%
\pgfsetfillopacity{0.700000}%
\pgfsetlinewidth{0.000000pt}%
\definecolor{currentstroke}{rgb}{0.000000,0.000000,0.000000}%
\pgfsetstrokecolor{currentstroke}%
\pgfsetdash{}{0pt}%
\pgfpathmoveto{\pgfqpoint{2.962147in}{2.147275in}}%
\pgfpathlineto{\pgfqpoint{2.975300in}{2.139455in}}%
\pgfpathlineto{\pgfqpoint{2.988457in}{2.131670in}}%
\pgfpathlineto{\pgfqpoint{3.001617in}{2.123918in}}%
\pgfpathlineto{\pgfqpoint{3.014780in}{2.116202in}}%
\pgfpathlineto{\pgfqpoint{3.006429in}{2.120648in}}%
\pgfpathlineto{\pgfqpoint{2.998061in}{2.125471in}}%
\pgfpathlineto{\pgfqpoint{2.989673in}{2.130680in}}%
\pgfpathlineto{\pgfqpoint{2.981267in}{2.136282in}}%
\pgfpathlineto{\pgfqpoint{2.968070in}{2.144316in}}%
\pgfpathlineto{\pgfqpoint{2.954875in}{2.152385in}}%
\pgfpathlineto{\pgfqpoint{2.941684in}{2.160488in}}%
\pgfpathlineto{\pgfqpoint{2.928495in}{2.168626in}}%
\pgfpathlineto{\pgfqpoint{2.936937in}{2.162701in}}%
\pgfpathlineto{\pgfqpoint{2.945360in}{2.157173in}}%
\pgfpathlineto{\pgfqpoint{2.953763in}{2.152034in}}%
\pgfpathlineto{\pgfqpoint{2.962147in}{2.147275in}}%
\pgfpathclose%
\pgfusepath{fill}%
\end{pgfscope}%
\begin{pgfscope}%
\pgfpathrectangle{\pgfqpoint{1.254980in}{0.150000in}}{\pgfqpoint{5.490039in}{5.490039in}}%
\pgfusepath{clip}%
\pgfsetbuttcap%
\pgfsetroundjoin%
\definecolor{currentfill}{rgb}{0.278791,0.062145,0.386592}%
\pgfsetfillcolor{currentfill}%
\pgfsetfillopacity{0.700000}%
\pgfsetlinewidth{0.000000pt}%
\definecolor{currentstroke}{rgb}{0.000000,0.000000,0.000000}%
\pgfsetstrokecolor{currentstroke}%
\pgfsetdash{}{0pt}%
\pgfpathmoveto{\pgfqpoint{3.587006in}{1.863554in}}%
\pgfpathlineto{\pgfqpoint{3.600236in}{1.857828in}}%
\pgfpathlineto{\pgfqpoint{3.613470in}{1.852129in}}%
\pgfpathlineto{\pgfqpoint{3.626710in}{1.846458in}}%
\pgfpathlineto{\pgfqpoint{3.639954in}{1.840814in}}%
\pgfpathlineto{\pgfqpoint{3.632041in}{1.838686in}}%
\pgfpathlineto{\pgfqpoint{3.624119in}{1.836807in}}%
\pgfpathlineto{\pgfqpoint{3.616186in}{1.835182in}}%
\pgfpathlineto{\pgfqpoint{3.608243in}{1.833819in}}%
\pgfpathlineto{\pgfqpoint{3.594976in}{1.839737in}}%
\pgfpathlineto{\pgfqpoint{3.581714in}{1.845681in}}%
\pgfpathlineto{\pgfqpoint{3.568457in}{1.851654in}}%
\pgfpathlineto{\pgfqpoint{3.555204in}{1.857653in}}%
\pgfpathlineto{\pgfqpoint{3.563170in}{1.858737in}}%
\pgfpathlineto{\pgfqpoint{3.571126in}{1.860087in}}%
\pgfpathlineto{\pgfqpoint{3.579071in}{1.861695in}}%
\pgfpathlineto{\pgfqpoint{3.587006in}{1.863554in}}%
\pgfpathclose%
\pgfusepath{fill}%
\end{pgfscope}%
\begin{pgfscope}%
\pgfpathrectangle{\pgfqpoint{1.254980in}{0.150000in}}{\pgfqpoint{5.490039in}{5.490039in}}%
\pgfusepath{clip}%
\pgfsetbuttcap%
\pgfsetroundjoin%
\definecolor{currentfill}{rgb}{0.282290,0.145912,0.461510}%
\pgfsetfillcolor{currentfill}%
\pgfsetfillopacity{0.700000}%
\pgfsetlinewidth{0.000000pt}%
\definecolor{currentstroke}{rgb}{0.000000,0.000000,0.000000}%
\pgfsetstrokecolor{currentstroke}%
\pgfsetdash{}{0pt}%
\pgfpathmoveto{\pgfqpoint{3.205850in}{2.015626in}}%
\pgfpathlineto{\pgfqpoint{3.219027in}{2.008635in}}%
\pgfpathlineto{\pgfqpoint{3.232209in}{2.001676in}}%
\pgfpathlineto{\pgfqpoint{3.245394in}{1.994748in}}%
\pgfpathlineto{\pgfqpoint{3.258583in}{1.987850in}}%
\pgfpathlineto{\pgfqpoint{3.250422in}{1.989687in}}%
\pgfpathlineto{\pgfqpoint{3.242246in}{1.991852in}}%
\pgfpathlineto{\pgfqpoint{3.234056in}{1.994355in}}%
\pgfpathlineto{\pgfqpoint{3.225850in}{1.997202in}}%
\pgfpathlineto{\pgfqpoint{3.212632in}{2.004401in}}%
\pgfpathlineto{\pgfqpoint{3.199417in}{2.011632in}}%
\pgfpathlineto{\pgfqpoint{3.186205in}{2.018893in}}%
\pgfpathlineto{\pgfqpoint{3.172997in}{2.026185in}}%
\pgfpathlineto{\pgfqpoint{3.181234in}{2.023030in}}%
\pgfpathlineto{\pgfqpoint{3.189455in}{2.020224in}}%
\pgfpathlineto{\pgfqpoint{3.197660in}{2.017759in}}%
\pgfpathlineto{\pgfqpoint{3.205850in}{2.015626in}}%
\pgfpathclose%
\pgfusepath{fill}%
\end{pgfscope}%
\begin{pgfscope}%
\pgfpathrectangle{\pgfqpoint{1.254980in}{0.150000in}}{\pgfqpoint{5.490039in}{5.490039in}}%
\pgfusepath{clip}%
\pgfsetbuttcap%
\pgfsetroundjoin%
\definecolor{currentfill}{rgb}{0.185556,0.418570,0.556753}%
\pgfsetfillcolor{currentfill}%
\pgfsetfillopacity{0.700000}%
\pgfsetlinewidth{0.000000pt}%
\definecolor{currentstroke}{rgb}{0.000000,0.000000,0.000000}%
\pgfsetstrokecolor{currentstroke}%
\pgfsetdash{}{0pt}%
\pgfpathmoveto{\pgfqpoint{2.314737in}{2.608569in}}%
\pgfpathlineto{\pgfqpoint{2.327882in}{2.598227in}}%
\pgfpathlineto{\pgfqpoint{2.341027in}{2.587937in}}%
\pgfpathlineto{\pgfqpoint{2.354173in}{2.577698in}}%
\pgfpathlineto{\pgfqpoint{2.367321in}{2.567510in}}%
\pgfpathlineto{\pgfqpoint{2.358346in}{2.579021in}}%
\pgfpathlineto{\pgfqpoint{2.349341in}{2.591026in}}%
\pgfpathlineto{\pgfqpoint{2.340306in}{2.603534in}}%
\pgfpathlineto{\pgfqpoint{2.331240in}{2.616556in}}%
\pgfpathlineto{\pgfqpoint{2.318045in}{2.627100in}}%
\pgfpathlineto{\pgfqpoint{2.304851in}{2.637696in}}%
\pgfpathlineto{\pgfqpoint{2.291657in}{2.648343in}}%
\pgfpathlineto{\pgfqpoint{2.278464in}{2.659042in}}%
\pgfpathlineto{\pgfqpoint{2.287579in}{2.645657in}}%
\pgfpathlineto{\pgfqpoint{2.296662in}{2.632790in}}%
\pgfpathlineto{\pgfqpoint{2.305715in}{2.620431in}}%
\pgfpathlineto{\pgfqpoint{2.314737in}{2.608569in}}%
\pgfpathclose%
\pgfusepath{fill}%
\end{pgfscope}%
\begin{pgfscope}%
\pgfpathrectangle{\pgfqpoint{1.254980in}{0.150000in}}{\pgfqpoint{5.490039in}{5.490039in}}%
\pgfusepath{clip}%
\pgfsetbuttcap%
\pgfsetroundjoin%
\definecolor{currentfill}{rgb}{0.277018,0.050344,0.375715}%
\pgfsetfillcolor{currentfill}%
\pgfsetfillopacity{0.700000}%
\pgfsetlinewidth{0.000000pt}%
\definecolor{currentstroke}{rgb}{0.000000,0.000000,0.000000}%
\pgfsetstrokecolor{currentstroke}%
\pgfsetdash{}{0pt}%
\pgfpathmoveto{\pgfqpoint{4.716567in}{1.851260in}}%
\pgfpathlineto{\pgfqpoint{4.730063in}{1.849023in}}%
\pgfpathlineto{\pgfqpoint{4.743566in}{1.846811in}}%
\pgfpathlineto{\pgfqpoint{4.757077in}{1.844623in}}%
\pgfpathlineto{\pgfqpoint{4.770595in}{1.842458in}}%
\pgfpathlineto{\pgfqpoint{4.763129in}{1.832887in}}%
\pgfpathlineto{\pgfqpoint{4.755657in}{1.823321in}}%
\pgfpathlineto{\pgfqpoint{4.748181in}{1.813763in}}%
\pgfpathlineto{\pgfqpoint{4.740700in}{1.804218in}}%
\pgfpathlineto{\pgfqpoint{4.727173in}{1.806540in}}%
\pgfpathlineto{\pgfqpoint{4.713654in}{1.808887in}}%
\pgfpathlineto{\pgfqpoint{4.700142in}{1.811257in}}%
\pgfpathlineto{\pgfqpoint{4.686637in}{1.813651in}}%
\pgfpathlineto{\pgfqpoint{4.694127in}{1.823034in}}%
\pgfpathlineto{\pgfqpoint{4.701612in}{1.832432in}}%
\pgfpathlineto{\pgfqpoint{4.709092in}{1.841842in}}%
\pgfpathlineto{\pgfqpoint{4.716567in}{1.851260in}}%
\pgfpathclose%
\pgfusepath{fill}%
\end{pgfscope}%
\begin{pgfscope}%
\pgfpathrectangle{\pgfqpoint{1.254980in}{0.150000in}}{\pgfqpoint{5.490039in}{5.490039in}}%
\pgfusepath{clip}%
\pgfsetbuttcap%
\pgfsetroundjoin%
\definecolor{currentfill}{rgb}{0.283229,0.120777,0.440584}%
\pgfsetfillcolor{currentfill}%
\pgfsetfillopacity{0.700000}%
\pgfsetlinewidth{0.000000pt}%
\definecolor{currentstroke}{rgb}{0.000000,0.000000,0.000000}%
\pgfsetstrokecolor{currentstroke}%
\pgfsetdash{}{0pt}%
\pgfpathmoveto{\pgfqpoint{5.106142in}{1.970234in}}%
\pgfpathlineto{\pgfqpoint{5.119759in}{1.968977in}}%
\pgfpathlineto{\pgfqpoint{5.133385in}{1.967743in}}%
\pgfpathlineto{\pgfqpoint{5.147018in}{1.966533in}}%
\pgfpathlineto{\pgfqpoint{5.160660in}{1.965347in}}%
\pgfpathlineto{\pgfqpoint{5.153322in}{1.955621in}}%
\pgfpathlineto{\pgfqpoint{5.145978in}{1.945837in}}%
\pgfpathlineto{\pgfqpoint{5.138627in}{1.935996in}}%
\pgfpathlineto{\pgfqpoint{5.131271in}{1.926102in}}%
\pgfpathlineto{\pgfqpoint{5.117621in}{1.927395in}}%
\pgfpathlineto{\pgfqpoint{5.103979in}{1.928711in}}%
\pgfpathlineto{\pgfqpoint{5.090345in}{1.930051in}}%
\pgfpathlineto{\pgfqpoint{5.076720in}{1.931415in}}%
\pgfpathlineto{\pgfqpoint{5.084084in}{1.941198in}}%
\pgfpathlineto{\pgfqpoint{5.091443in}{1.950931in}}%
\pgfpathlineto{\pgfqpoint{5.098795in}{1.960610in}}%
\pgfpathlineto{\pgfqpoint{5.106142in}{1.970234in}}%
\pgfpathclose%
\pgfusepath{fill}%
\end{pgfscope}%
\begin{pgfscope}%
\pgfpathrectangle{\pgfqpoint{1.254980in}{0.150000in}}{\pgfqpoint{5.490039in}{5.490039in}}%
\pgfusepath{clip}%
\pgfsetbuttcap%
\pgfsetroundjoin%
\definecolor{currentfill}{rgb}{0.274128,0.199721,0.498911}%
\pgfsetfillcolor{currentfill}%
\pgfsetfillopacity{0.700000}%
\pgfsetlinewidth{0.000000pt}%
\definecolor{currentstroke}{rgb}{0.000000,0.000000,0.000000}%
\pgfsetstrokecolor{currentstroke}%
\pgfsetdash{}{0pt}%
\pgfpathmoveto{\pgfqpoint{5.579745in}{2.132761in}}%
\pgfpathlineto{\pgfqpoint{5.593521in}{2.132408in}}%
\pgfpathlineto{\pgfqpoint{5.607305in}{2.132078in}}%
\pgfpathlineto{\pgfqpoint{5.621099in}{2.131772in}}%
\pgfpathlineto{\pgfqpoint{5.634901in}{2.131490in}}%
\pgfpathlineto{\pgfqpoint{5.627760in}{2.123069in}}%
\pgfpathlineto{\pgfqpoint{5.620612in}{2.114541in}}%
\pgfpathlineto{\pgfqpoint{5.613455in}{2.105909in}}%
\pgfpathlineto{\pgfqpoint{5.606289in}{2.097171in}}%
\pgfpathlineto{\pgfqpoint{5.592477in}{2.097494in}}%
\pgfpathlineto{\pgfqpoint{5.578673in}{2.097840in}}%
\pgfpathlineto{\pgfqpoint{5.564878in}{2.098210in}}%
\pgfpathlineto{\pgfqpoint{5.551092in}{2.098604in}}%
\pgfpathlineto{\pgfqpoint{5.558267in}{2.107295in}}%
\pgfpathlineto{\pgfqpoint{5.565434in}{2.115886in}}%
\pgfpathlineto{\pgfqpoint{5.572594in}{2.124375in}}%
\pgfpathlineto{\pgfqpoint{5.579745in}{2.132761in}}%
\pgfpathclose%
\pgfusepath{fill}%
\end{pgfscope}%
\begin{pgfscope}%
\pgfpathrectangle{\pgfqpoint{1.254980in}{0.150000in}}{\pgfqpoint{5.490039in}{5.490039in}}%
\pgfusepath{clip}%
\pgfsetbuttcap%
\pgfsetroundjoin%
\definecolor{currentfill}{rgb}{0.269944,0.014625,0.341379}%
\pgfsetfillcolor{currentfill}%
\pgfsetfillopacity{0.700000}%
\pgfsetlinewidth{0.000000pt}%
\definecolor{currentstroke}{rgb}{0.000000,0.000000,0.000000}%
\pgfsetstrokecolor{currentstroke}%
\pgfsetdash{}{0pt}%
\pgfpathmoveto{\pgfqpoint{4.411129in}{1.786867in}}%
\pgfpathlineto{\pgfqpoint{4.424541in}{1.783734in}}%
\pgfpathlineto{\pgfqpoint{4.437960in}{1.780626in}}%
\pgfpathlineto{\pgfqpoint{4.451386in}{1.777542in}}%
\pgfpathlineto{\pgfqpoint{4.464818in}{1.774482in}}%
\pgfpathlineto{\pgfqpoint{4.457256in}{1.766000in}}%
\pgfpathlineto{\pgfqpoint{4.449688in}{1.757585in}}%
\pgfpathlineto{\pgfqpoint{4.442115in}{1.749244in}}%
\pgfpathlineto{\pgfqpoint{4.434537in}{1.740980in}}%
\pgfpathlineto{\pgfqpoint{4.421094in}{1.744235in}}%
\pgfpathlineto{\pgfqpoint{4.407657in}{1.747515in}}%
\pgfpathlineto{\pgfqpoint{4.394227in}{1.750819in}}%
\pgfpathlineto{\pgfqpoint{4.380804in}{1.754148in}}%
\pgfpathlineto{\pgfqpoint{4.388393in}{1.762211in}}%
\pgfpathlineto{\pgfqpoint{4.395977in}{1.770355in}}%
\pgfpathlineto{\pgfqpoint{4.403555in}{1.778575in}}%
\pgfpathlineto{\pgfqpoint{4.411129in}{1.786867in}}%
\pgfpathclose%
\pgfusepath{fill}%
\end{pgfscope}%
\begin{pgfscope}%
\pgfpathrectangle{\pgfqpoint{1.254980in}{0.150000in}}{\pgfqpoint{5.490039in}{5.490039in}}%
\pgfusepath{clip}%
\pgfsetbuttcap%
\pgfsetroundjoin%
\definecolor{currentfill}{rgb}{0.271305,0.019942,0.347269}%
\pgfsetfillcolor{currentfill}%
\pgfsetfillopacity{0.700000}%
\pgfsetlinewidth{0.000000pt}%
\definecolor{currentstroke}{rgb}{0.000000,0.000000,0.000000}%
\pgfsetstrokecolor{currentstroke}%
\pgfsetdash{}{0pt}%
\pgfpathmoveto{\pgfqpoint{3.914889in}{1.787368in}}%
\pgfpathlineto{\pgfqpoint{3.928183in}{1.782680in}}%
\pgfpathlineto{\pgfqpoint{3.941483in}{1.778018in}}%
\pgfpathlineto{\pgfqpoint{3.954789in}{1.773381in}}%
\pgfpathlineto{\pgfqpoint{3.968101in}{1.768770in}}%
\pgfpathlineto{\pgfqpoint{3.960351in}{1.763695in}}%
\pgfpathlineto{\pgfqpoint{3.952595in}{1.758798in}}%
\pgfpathlineto{\pgfqpoint{3.944832in}{1.754086in}}%
\pgfpathlineto{\pgfqpoint{3.937061in}{1.749564in}}%
\pgfpathlineto{\pgfqpoint{3.923732in}{1.754422in}}%
\pgfpathlineto{\pgfqpoint{3.910409in}{1.759306in}}%
\pgfpathlineto{\pgfqpoint{3.897092in}{1.764215in}}%
\pgfpathlineto{\pgfqpoint{3.883779in}{1.769150in}}%
\pgfpathlineto{\pgfqpoint{3.891568in}{1.773420in}}%
\pgfpathlineto{\pgfqpoint{3.899349in}{1.777883in}}%
\pgfpathlineto{\pgfqpoint{3.907123in}{1.782535in}}%
\pgfpathlineto{\pgfqpoint{3.914889in}{1.787368in}}%
\pgfpathclose%
\pgfusepath{fill}%
\end{pgfscope}%
\begin{pgfscope}%
\pgfpathrectangle{\pgfqpoint{1.254980in}{0.150000in}}{\pgfqpoint{5.490039in}{5.490039in}}%
\pgfusepath{clip}%
\pgfsetbuttcap%
\pgfsetroundjoin%
\definecolor{currentfill}{rgb}{0.268510,0.009605,0.335427}%
\pgfsetfillcolor{currentfill}%
\pgfsetfillopacity{0.700000}%
\pgfsetlinewidth{0.000000pt}%
\definecolor{currentstroke}{rgb}{0.000000,0.000000,0.000000}%
\pgfsetstrokecolor{currentstroke}%
\pgfsetdash{}{0pt}%
\pgfpathmoveto{\pgfqpoint{4.052267in}{1.773472in}}%
\pgfpathlineto{\pgfqpoint{4.065591in}{1.769221in}}%
\pgfpathlineto{\pgfqpoint{4.078922in}{1.764995in}}%
\pgfpathlineto{\pgfqpoint{4.092258in}{1.760795in}}%
\pgfpathlineto{\pgfqpoint{4.105600in}{1.756619in}}%
\pgfpathlineto{\pgfqpoint{4.097909in}{1.750435in}}%
\pgfpathlineto{\pgfqpoint{4.090212in}{1.744398in}}%
\pgfpathlineto{\pgfqpoint{4.082508in}{1.738514in}}%
\pgfpathlineto{\pgfqpoint{4.074798in}{1.732790in}}%
\pgfpathlineto{\pgfqpoint{4.061440in}{1.737199in}}%
\pgfpathlineto{\pgfqpoint{4.048089in}{1.741634in}}%
\pgfpathlineto{\pgfqpoint{4.034743in}{1.746094in}}%
\pgfpathlineto{\pgfqpoint{4.021403in}{1.750578in}}%
\pgfpathlineto{\pgfqpoint{4.029129in}{1.756064in}}%
\pgfpathlineto{\pgfqpoint{4.036848in}{1.761712in}}%
\pgfpathlineto{\pgfqpoint{4.044561in}{1.767516in}}%
\pgfpathlineto{\pgfqpoint{4.052267in}{1.773472in}}%
\pgfpathclose%
\pgfusepath{fill}%
\end{pgfscope}%
\begin{pgfscope}%
\pgfpathrectangle{\pgfqpoint{1.254980in}{0.150000in}}{\pgfqpoint{5.490039in}{5.490039in}}%
\pgfusepath{clip}%
\pgfsetbuttcap%
\pgfsetroundjoin%
\definecolor{currentfill}{rgb}{0.282910,0.105393,0.426902}%
\pgfsetfillcolor{currentfill}%
\pgfsetfillopacity{0.700000}%
\pgfsetlinewidth{0.000000pt}%
\definecolor{currentstroke}{rgb}{0.000000,0.000000,0.000000}%
\pgfsetstrokecolor{currentstroke}%
\pgfsetdash{}{0pt}%
\pgfpathmoveto{\pgfqpoint{5.022298in}{1.937108in}}%
\pgfpathlineto{\pgfqpoint{5.035891in}{1.935649in}}%
\pgfpathlineto{\pgfqpoint{5.049493in}{1.934214in}}%
\pgfpathlineto{\pgfqpoint{5.063102in}{1.932803in}}%
\pgfpathlineto{\pgfqpoint{5.076720in}{1.931415in}}%
\pgfpathlineto{\pgfqpoint{5.069350in}{1.921584in}}%
\pgfpathlineto{\pgfqpoint{5.061974in}{1.911707in}}%
\pgfpathlineto{\pgfqpoint{5.054593in}{1.901787in}}%
\pgfpathlineto{\pgfqpoint{5.047206in}{1.891827in}}%
\pgfpathlineto{\pgfqpoint{5.033580in}{1.893334in}}%
\pgfpathlineto{\pgfqpoint{5.019963in}{1.894864in}}%
\pgfpathlineto{\pgfqpoint{5.006353in}{1.896419in}}%
\pgfpathlineto{\pgfqpoint{4.992752in}{1.897997in}}%
\pgfpathlineto{\pgfqpoint{5.000147in}{1.907833in}}%
\pgfpathlineto{\pgfqpoint{5.007536in}{1.917632in}}%
\pgfpathlineto{\pgfqpoint{5.014920in}{1.927391in}}%
\pgfpathlineto{\pgfqpoint{5.022298in}{1.937108in}}%
\pgfpathclose%
\pgfusepath{fill}%
\end{pgfscope}%
\begin{pgfscope}%
\pgfpathrectangle{\pgfqpoint{1.254980in}{0.150000in}}{\pgfqpoint{5.490039in}{5.490039in}}%
\pgfusepath{clip}%
\pgfsetbuttcap%
\pgfsetroundjoin%
\definecolor{currentfill}{rgb}{0.243113,0.292092,0.538516}%
\pgfsetfillcolor{currentfill}%
\pgfsetfillopacity{0.700000}%
\pgfsetlinewidth{0.000000pt}%
\definecolor{currentstroke}{rgb}{0.000000,0.000000,0.000000}%
\pgfsetstrokecolor{currentstroke}%
\pgfsetdash{}{0pt}%
\pgfpathmoveto{\pgfqpoint{2.717854in}{2.303791in}}%
\pgfpathlineto{\pgfqpoint{2.731000in}{2.295059in}}%
\pgfpathlineto{\pgfqpoint{2.744149in}{2.286365in}}%
\pgfpathlineto{\pgfqpoint{2.757299in}{2.277711in}}%
\pgfpathlineto{\pgfqpoint{2.770453in}{2.269095in}}%
\pgfpathlineto{\pgfqpoint{2.761878in}{2.276412in}}%
\pgfpathlineto{\pgfqpoint{2.753280in}{2.284156in}}%
\pgfpathlineto{\pgfqpoint{2.744660in}{2.292336in}}%
\pgfpathlineto{\pgfqpoint{2.736017in}{2.300962in}}%
\pgfpathlineto{\pgfqpoint{2.722824in}{2.309912in}}%
\pgfpathlineto{\pgfqpoint{2.709633in}{2.318901in}}%
\pgfpathlineto{\pgfqpoint{2.696445in}{2.327930in}}%
\pgfpathlineto{\pgfqpoint{2.683258in}{2.336998in}}%
\pgfpathlineto{\pgfqpoint{2.691943in}{2.328031in}}%
\pgfpathlineto{\pgfqpoint{2.700603in}{2.319514in}}%
\pgfpathlineto{\pgfqpoint{2.709240in}{2.311437in}}%
\pgfpathlineto{\pgfqpoint{2.717854in}{2.303791in}}%
\pgfpathclose%
\pgfusepath{fill}%
\end{pgfscope}%
\begin{pgfscope}%
\pgfpathrectangle{\pgfqpoint{1.254980in}{0.150000in}}{\pgfqpoint{5.490039in}{5.490039in}}%
\pgfusepath{clip}%
\pgfsetbuttcap%
\pgfsetroundjoin%
\definecolor{currentfill}{rgb}{0.276194,0.190074,0.493001}%
\pgfsetfillcolor{currentfill}%
\pgfsetfillopacity{0.700000}%
\pgfsetlinewidth{0.000000pt}%
\definecolor{currentstroke}{rgb}{0.000000,0.000000,0.000000}%
\pgfsetstrokecolor{currentstroke}%
\pgfsetdash{}{0pt}%
\pgfpathmoveto{\pgfqpoint{5.496036in}{2.100416in}}%
\pgfpathlineto{\pgfqpoint{5.509787in}{2.099927in}}%
\pgfpathlineto{\pgfqpoint{5.523546in}{2.099462in}}%
\pgfpathlineto{\pgfqpoint{5.537315in}{2.099021in}}%
\pgfpathlineto{\pgfqpoint{5.551092in}{2.098604in}}%
\pgfpathlineto{\pgfqpoint{5.543909in}{2.089811in}}%
\pgfpathlineto{\pgfqpoint{5.536718in}{2.080919in}}%
\pgfpathlineto{\pgfqpoint{5.529519in}{2.071928in}}%
\pgfpathlineto{\pgfqpoint{5.522313in}{2.062839in}}%
\pgfpathlineto{\pgfqpoint{5.508526in}{2.063310in}}%
\pgfpathlineto{\pgfqpoint{5.494748in}{2.063805in}}%
\pgfpathlineto{\pgfqpoint{5.480979in}{2.064323in}}%
\pgfpathlineto{\pgfqpoint{5.467219in}{2.064865in}}%
\pgfpathlineto{\pgfqpoint{5.474434in}{2.073896in}}%
\pgfpathlineto{\pgfqpoint{5.481642in}{2.082832in}}%
\pgfpathlineto{\pgfqpoint{5.488843in}{2.091672in}}%
\pgfpathlineto{\pgfqpoint{5.496036in}{2.100416in}}%
\pgfpathclose%
\pgfusepath{fill}%
\end{pgfscope}%
\begin{pgfscope}%
\pgfpathrectangle{\pgfqpoint{1.254980in}{0.150000in}}{\pgfqpoint{5.490039in}{5.490039in}}%
\pgfusepath{clip}%
\pgfsetbuttcap%
\pgfsetroundjoin%
\definecolor{currentfill}{rgb}{0.274952,0.037752,0.364543}%
\pgfsetfillcolor{currentfill}%
\pgfsetfillopacity{0.700000}%
\pgfsetlinewidth{0.000000pt}%
\definecolor{currentstroke}{rgb}{0.000000,0.000000,0.000000}%
\pgfsetstrokecolor{currentstroke}%
\pgfsetdash{}{0pt}%
\pgfpathmoveto{\pgfqpoint{4.632692in}{1.823467in}}%
\pgfpathlineto{\pgfqpoint{4.646167in}{1.820977in}}%
\pgfpathlineto{\pgfqpoint{4.659650in}{1.818511in}}%
\pgfpathlineto{\pgfqpoint{4.673140in}{1.816069in}}%
\pgfpathlineto{\pgfqpoint{4.686637in}{1.813651in}}%
\pgfpathlineto{\pgfqpoint{4.679143in}{1.804288in}}%
\pgfpathlineto{\pgfqpoint{4.671643in}{1.794949in}}%
\pgfpathlineto{\pgfqpoint{4.664139in}{1.785637in}}%
\pgfpathlineto{\pgfqpoint{4.656630in}{1.776356in}}%
\pgfpathlineto{\pgfqpoint{4.643123in}{1.778945in}}%
\pgfpathlineto{\pgfqpoint{4.629624in}{1.781558in}}%
\pgfpathlineto{\pgfqpoint{4.616132in}{1.784194in}}%
\pgfpathlineto{\pgfqpoint{4.602647in}{1.786854in}}%
\pgfpathlineto{\pgfqpoint{4.610165in}{1.795959in}}%
\pgfpathlineto{\pgfqpoint{4.617679in}{1.805099in}}%
\pgfpathlineto{\pgfqpoint{4.625188in}{1.814269in}}%
\pgfpathlineto{\pgfqpoint{4.632692in}{1.823467in}}%
\pgfpathclose%
\pgfusepath{fill}%
\end{pgfscope}%
\begin{pgfscope}%
\pgfpathrectangle{\pgfqpoint{1.254980in}{0.150000in}}{\pgfqpoint{5.490039in}{5.490039in}}%
\pgfusepath{clip}%
\pgfsetbuttcap%
\pgfsetroundjoin%
\definecolor{currentfill}{rgb}{0.273809,0.031497,0.358853}%
\pgfsetfillcolor{currentfill}%
\pgfsetfillopacity{0.700000}%
\pgfsetlinewidth{0.000000pt}%
\definecolor{currentstroke}{rgb}{0.000000,0.000000,0.000000}%
\pgfsetstrokecolor{currentstroke}%
\pgfsetdash{}{0pt}%
\pgfpathmoveto{\pgfqpoint{3.777476in}{1.809561in}}%
\pgfpathlineto{\pgfqpoint{3.790745in}{1.804418in}}%
\pgfpathlineto{\pgfqpoint{3.804020in}{1.799301in}}%
\pgfpathlineto{\pgfqpoint{3.817300in}{1.794211in}}%
\pgfpathlineto{\pgfqpoint{3.830585in}{1.789147in}}%
\pgfpathlineto{\pgfqpoint{3.822770in}{1.785333in}}%
\pgfpathlineto{\pgfqpoint{3.814946in}{1.781728in}}%
\pgfpathlineto{\pgfqpoint{3.807115in}{1.778342in}}%
\pgfpathlineto{\pgfqpoint{3.799274in}{1.775178in}}%
\pgfpathlineto{\pgfqpoint{3.785970in}{1.780503in}}%
\pgfpathlineto{\pgfqpoint{3.772670in}{1.785853in}}%
\pgfpathlineto{\pgfqpoint{3.759376in}{1.791229in}}%
\pgfpathlineto{\pgfqpoint{3.746086in}{1.796632in}}%
\pgfpathlineto{\pgfqpoint{3.753947in}{1.799531in}}%
\pgfpathlineto{\pgfqpoint{3.761799in}{1.802656in}}%
\pgfpathlineto{\pgfqpoint{3.769642in}{1.806002in}}%
\pgfpathlineto{\pgfqpoint{3.777476in}{1.809561in}}%
\pgfpathclose%
\pgfusepath{fill}%
\end{pgfscope}%
\begin{pgfscope}%
\pgfpathrectangle{\pgfqpoint{1.254980in}{0.150000in}}{\pgfqpoint{5.490039in}{5.490039in}}%
\pgfusepath{clip}%
\pgfsetbuttcap%
\pgfsetroundjoin%
\definecolor{currentfill}{rgb}{0.192357,0.403199,0.555836}%
\pgfsetfillcolor{currentfill}%
\pgfsetfillopacity{0.700000}%
\pgfsetlinewidth{0.000000pt}%
\definecolor{currentstroke}{rgb}{0.000000,0.000000,0.000000}%
\pgfsetstrokecolor{currentstroke}%
\pgfsetdash{}{0pt}%
\pgfpathmoveto{\pgfqpoint{2.367321in}{2.567510in}}%
\pgfpathlineto{\pgfqpoint{2.380469in}{2.557373in}}%
\pgfpathlineto{\pgfqpoint{2.393618in}{2.547285in}}%
\pgfpathlineto{\pgfqpoint{2.406768in}{2.537246in}}%
\pgfpathlineto{\pgfqpoint{2.419919in}{2.527257in}}%
\pgfpathlineto{\pgfqpoint{2.410991in}{2.538418in}}%
\pgfpathlineto{\pgfqpoint{2.402033in}{2.550069in}}%
\pgfpathlineto{\pgfqpoint{2.393046in}{2.562219in}}%
\pgfpathlineto{\pgfqpoint{2.384029in}{2.574879in}}%
\pgfpathlineto{\pgfqpoint{2.370830in}{2.585224in}}%
\pgfpathlineto{\pgfqpoint{2.357633in}{2.595618in}}%
\pgfpathlineto{\pgfqpoint{2.344436in}{2.606062in}}%
\pgfpathlineto{\pgfqpoint{2.331240in}{2.616556in}}%
\pgfpathlineto{\pgfqpoint{2.340306in}{2.603534in}}%
\pgfpathlineto{\pgfqpoint{2.349341in}{2.591026in}}%
\pgfpathlineto{\pgfqpoint{2.358346in}{2.579021in}}%
\pgfpathlineto{\pgfqpoint{2.367321in}{2.567510in}}%
\pgfpathclose%
\pgfusepath{fill}%
\end{pgfscope}%
\begin{pgfscope}%
\pgfpathrectangle{\pgfqpoint{1.254980in}{0.150000in}}{\pgfqpoint{5.490039in}{5.490039in}}%
\pgfusepath{clip}%
\pgfsetbuttcap%
\pgfsetroundjoin%
\definecolor{currentfill}{rgb}{0.268510,0.009605,0.335427}%
\pgfsetfillcolor{currentfill}%
\pgfsetfillopacity{0.700000}%
\pgfsetlinewidth{0.000000pt}%
\definecolor{currentstroke}{rgb}{0.000000,0.000000,0.000000}%
\pgfsetstrokecolor{currentstroke}%
\pgfsetdash{}{0pt}%
\pgfpathmoveto{\pgfqpoint{4.189676in}{1.767147in}}%
\pgfpathlineto{\pgfqpoint{4.203035in}{1.763317in}}%
\pgfpathlineto{\pgfqpoint{4.216400in}{1.759511in}}%
\pgfpathlineto{\pgfqpoint{4.229771in}{1.755730in}}%
\pgfpathlineto{\pgfqpoint{4.243149in}{1.751974in}}%
\pgfpathlineto{\pgfqpoint{4.235509in}{1.744824in}}%
\pgfpathlineto{\pgfqpoint{4.227863in}{1.737793in}}%
\pgfpathlineto{\pgfqpoint{4.220212in}{1.730885in}}%
\pgfpathlineto{\pgfqpoint{4.212556in}{1.724107in}}%
\pgfpathlineto{\pgfqpoint{4.199165in}{1.728085in}}%
\pgfpathlineto{\pgfqpoint{4.185780in}{1.732087in}}%
\pgfpathlineto{\pgfqpoint{4.172402in}{1.736114in}}%
\pgfpathlineto{\pgfqpoint{4.159029in}{1.740165in}}%
\pgfpathlineto{\pgfqpoint{4.166700in}{1.746717in}}%
\pgfpathlineto{\pgfqpoint{4.174364in}{1.753402in}}%
\pgfpathlineto{\pgfqpoint{4.182023in}{1.760214in}}%
\pgfpathlineto{\pgfqpoint{4.189676in}{1.767147in}}%
\pgfpathclose%
\pgfusepath{fill}%
\end{pgfscope}%
\begin{pgfscope}%
\pgfpathrectangle{\pgfqpoint{1.254980in}{0.150000in}}{\pgfqpoint{5.490039in}{5.490039in}}%
\pgfusepath{clip}%
\pgfsetbuttcap%
\pgfsetroundjoin%
\definecolor{currentfill}{rgb}{0.273006,0.204520,0.501721}%
\pgfsetfillcolor{currentfill}%
\pgfsetfillopacity{0.700000}%
\pgfsetlinewidth{0.000000pt}%
\definecolor{currentstroke}{rgb}{0.000000,0.000000,0.000000}%
\pgfsetstrokecolor{currentstroke}%
\pgfsetdash{}{0pt}%
\pgfpathmoveto{\pgfqpoint{3.014780in}{2.116202in}}%
\pgfpathlineto{\pgfqpoint{3.027946in}{2.108519in}}%
\pgfpathlineto{\pgfqpoint{3.041115in}{2.100870in}}%
\pgfpathlineto{\pgfqpoint{3.054288in}{2.093254in}}%
\pgfpathlineto{\pgfqpoint{3.067464in}{2.085672in}}%
\pgfpathlineto{\pgfqpoint{3.059147in}{2.089806in}}%
\pgfpathlineto{\pgfqpoint{3.050812in}{2.094314in}}%
\pgfpathlineto{\pgfqpoint{3.042459in}{2.099203in}}%
\pgfpathlineto{\pgfqpoint{3.034088in}{2.104483in}}%
\pgfpathlineto{\pgfqpoint{3.020878in}{2.112382in}}%
\pgfpathlineto{\pgfqpoint{3.007671in}{2.120315in}}%
\pgfpathlineto{\pgfqpoint{2.994468in}{2.128282in}}%
\pgfpathlineto{\pgfqpoint{2.981267in}{2.136282in}}%
\pgfpathlineto{\pgfqpoint{2.989673in}{2.130680in}}%
\pgfpathlineto{\pgfqpoint{2.998061in}{2.125471in}}%
\pgfpathlineto{\pgfqpoint{3.006429in}{2.120648in}}%
\pgfpathlineto{\pgfqpoint{3.014780in}{2.116202in}}%
\pgfpathclose%
\pgfusepath{fill}%
\end{pgfscope}%
\begin{pgfscope}%
\pgfpathrectangle{\pgfqpoint{1.254980in}{0.150000in}}{\pgfqpoint{5.490039in}{5.490039in}}%
\pgfusepath{clip}%
\pgfsetbuttcap%
\pgfsetroundjoin%
\definecolor{currentfill}{rgb}{0.282327,0.094955,0.417331}%
\pgfsetfillcolor{currentfill}%
\pgfsetfillopacity{0.700000}%
\pgfsetlinewidth{0.000000pt}%
\definecolor{currentstroke}{rgb}{0.000000,0.000000,0.000000}%
\pgfsetstrokecolor{currentstroke}%
\pgfsetdash{}{0pt}%
\pgfpathmoveto{\pgfqpoint{3.449346in}{1.906658in}}%
\pgfpathlineto{\pgfqpoint{3.462563in}{1.900433in}}%
\pgfpathlineto{\pgfqpoint{3.475784in}{1.894237in}}%
\pgfpathlineto{\pgfqpoint{3.489009in}{1.888069in}}%
\pgfpathlineto{\pgfqpoint{3.502239in}{1.881930in}}%
\pgfpathlineto{\pgfqpoint{3.494237in}{1.881401in}}%
\pgfpathlineto{\pgfqpoint{3.486224in}{1.881156in}}%
\pgfpathlineto{\pgfqpoint{3.478198in}{1.881201in}}%
\pgfpathlineto{\pgfqpoint{3.470161in}{1.881547in}}%
\pgfpathlineto{\pgfqpoint{3.456905in}{1.887973in}}%
\pgfpathlineto{\pgfqpoint{3.443654in}{1.894428in}}%
\pgfpathlineto{\pgfqpoint{3.430408in}{1.900911in}}%
\pgfpathlineto{\pgfqpoint{3.417166in}{1.907423in}}%
\pgfpathlineto{\pgfqpoint{3.425229in}{1.906786in}}%
\pgfpathlineto{\pgfqpoint{3.433281in}{1.906451in}}%
\pgfpathlineto{\pgfqpoint{3.441320in}{1.906411in}}%
\pgfpathlineto{\pgfqpoint{3.449346in}{1.906658in}}%
\pgfpathclose%
\pgfusepath{fill}%
\end{pgfscope}%
\begin{pgfscope}%
\pgfpathrectangle{\pgfqpoint{1.254980in}{0.150000in}}{\pgfqpoint{5.490039in}{5.490039in}}%
\pgfusepath{clip}%
\pgfsetbuttcap%
\pgfsetroundjoin%
\definecolor{currentfill}{rgb}{0.281924,0.089666,0.412415}%
\pgfsetfillcolor{currentfill}%
\pgfsetfillopacity{0.700000}%
\pgfsetlinewidth{0.000000pt}%
\definecolor{currentstroke}{rgb}{0.000000,0.000000,0.000000}%
\pgfsetstrokecolor{currentstroke}%
\pgfsetdash{}{0pt}%
\pgfpathmoveto{\pgfqpoint{4.938424in}{1.904547in}}%
\pgfpathlineto{\pgfqpoint{4.951994in}{1.902874in}}%
\pgfpathlineto{\pgfqpoint{4.965572in}{1.901225in}}%
\pgfpathlineto{\pgfqpoint{4.979158in}{1.899599in}}%
\pgfpathlineto{\pgfqpoint{4.992752in}{1.897997in}}%
\pgfpathlineto{\pgfqpoint{4.985351in}{1.888127in}}%
\pgfpathlineto{\pgfqpoint{4.977946in}{1.878226in}}%
\pgfpathlineto{\pgfqpoint{4.970534in}{1.868296in}}%
\pgfpathlineto{\pgfqpoint{4.963118in}{1.858341in}}%
\pgfpathlineto{\pgfqpoint{4.949517in}{1.860076in}}%
\pgfpathlineto{\pgfqpoint{4.935923in}{1.861834in}}%
\pgfpathlineto{\pgfqpoint{4.922337in}{1.863616in}}%
\pgfpathlineto{\pgfqpoint{4.908759in}{1.865421in}}%
\pgfpathlineto{\pgfqpoint{4.916183in}{1.875239in}}%
\pgfpathlineto{\pgfqpoint{4.923602in}{1.885034in}}%
\pgfpathlineto{\pgfqpoint{4.931016in}{1.894805in}}%
\pgfpathlineto{\pgfqpoint{4.938424in}{1.904547in}}%
\pgfpathclose%
\pgfusepath{fill}%
\end{pgfscope}%
\begin{pgfscope}%
\pgfpathrectangle{\pgfqpoint{1.254980in}{0.150000in}}{\pgfqpoint{5.490039in}{5.490039in}}%
\pgfusepath{clip}%
\pgfsetbuttcap%
\pgfsetroundjoin%
\definecolor{currentfill}{rgb}{0.279574,0.170599,0.479997}%
\pgfsetfillcolor{currentfill}%
\pgfsetfillopacity{0.700000}%
\pgfsetlinewidth{0.000000pt}%
\definecolor{currentstroke}{rgb}{0.000000,0.000000,0.000000}%
\pgfsetstrokecolor{currentstroke}%
\pgfsetdash{}{0pt}%
\pgfpathmoveto{\pgfqpoint{5.412264in}{2.067272in}}%
\pgfpathlineto{\pgfqpoint{5.425990in}{2.066635in}}%
\pgfpathlineto{\pgfqpoint{5.439724in}{2.066021in}}%
\pgfpathlineto{\pgfqpoint{5.453467in}{2.065432in}}%
\pgfpathlineto{\pgfqpoint{5.467219in}{2.064865in}}%
\pgfpathlineto{\pgfqpoint{5.459995in}{2.055741in}}%
\pgfpathlineto{\pgfqpoint{5.452765in}{2.046525in}}%
\pgfpathlineto{\pgfqpoint{5.445527in}{2.037217in}}%
\pgfpathlineto{\pgfqpoint{5.438282in}{2.027820in}}%
\pgfpathlineto{\pgfqpoint{5.424521in}{2.028453in}}%
\pgfpathlineto{\pgfqpoint{5.410769in}{2.029110in}}%
\pgfpathlineto{\pgfqpoint{5.397026in}{2.029790in}}%
\pgfpathlineto{\pgfqpoint{5.383292in}{2.030495in}}%
\pgfpathlineto{\pgfqpoint{5.390546in}{2.039820in}}%
\pgfpathlineto{\pgfqpoint{5.397792in}{2.049059in}}%
\pgfpathlineto{\pgfqpoint{5.405032in}{2.058210in}}%
\pgfpathlineto{\pgfqpoint{5.412264in}{2.067272in}}%
\pgfpathclose%
\pgfusepath{fill}%
\end{pgfscope}%
\begin{pgfscope}%
\pgfpathrectangle{\pgfqpoint{1.254980in}{0.150000in}}{\pgfqpoint{5.490039in}{5.490039in}}%
\pgfusepath{clip}%
\pgfsetbuttcap%
\pgfsetroundjoin%
\definecolor{currentfill}{rgb}{0.282623,0.140926,0.457517}%
\pgfsetfillcolor{currentfill}%
\pgfsetfillopacity{0.700000}%
\pgfsetlinewidth{0.000000pt}%
\definecolor{currentstroke}{rgb}{0.000000,0.000000,0.000000}%
\pgfsetstrokecolor{currentstroke}%
\pgfsetdash{}{0pt}%
\pgfpathmoveto{\pgfqpoint{3.258583in}{1.987850in}}%
\pgfpathlineto{\pgfqpoint{3.271776in}{1.980984in}}%
\pgfpathlineto{\pgfqpoint{3.284973in}{1.974147in}}%
\pgfpathlineto{\pgfqpoint{3.298173in}{1.967341in}}%
\pgfpathlineto{\pgfqpoint{3.311378in}{1.960565in}}%
\pgfpathlineto{\pgfqpoint{3.303246in}{1.962106in}}%
\pgfpathlineto{\pgfqpoint{3.295100in}{1.963971in}}%
\pgfpathlineto{\pgfqpoint{3.286939in}{1.966170in}}%
\pgfpathlineto{\pgfqpoint{3.278764in}{1.968710in}}%
\pgfpathlineto{\pgfqpoint{3.265530in}{1.975788in}}%
\pgfpathlineto{\pgfqpoint{3.252299in}{1.982896in}}%
\pgfpathlineto{\pgfqpoint{3.239073in}{1.990034in}}%
\pgfpathlineto{\pgfqpoint{3.225850in}{1.997202in}}%
\pgfpathlineto{\pgfqpoint{3.234056in}{1.994355in}}%
\pgfpathlineto{\pgfqpoint{3.242246in}{1.991852in}}%
\pgfpathlineto{\pgfqpoint{3.250422in}{1.989687in}}%
\pgfpathlineto{\pgfqpoint{3.258583in}{1.987850in}}%
\pgfpathclose%
\pgfusepath{fill}%
\end{pgfscope}%
\begin{pgfscope}%
\pgfpathrectangle{\pgfqpoint{1.254980in}{0.150000in}}{\pgfqpoint{5.490039in}{5.490039in}}%
\pgfusepath{clip}%
\pgfsetbuttcap%
\pgfsetroundjoin%
\definecolor{currentfill}{rgb}{0.277941,0.056324,0.381191}%
\pgfsetfillcolor{currentfill}%
\pgfsetfillopacity{0.700000}%
\pgfsetlinewidth{0.000000pt}%
\definecolor{currentstroke}{rgb}{0.000000,0.000000,0.000000}%
\pgfsetstrokecolor{currentstroke}%
\pgfsetdash{}{0pt}%
\pgfpathmoveto{\pgfqpoint{3.639954in}{1.840814in}}%
\pgfpathlineto{\pgfqpoint{3.653203in}{1.835197in}}%
\pgfpathlineto{\pgfqpoint{3.666457in}{1.829607in}}%
\pgfpathlineto{\pgfqpoint{3.679716in}{1.824045in}}%
\pgfpathlineto{\pgfqpoint{3.692980in}{1.818509in}}%
\pgfpathlineto{\pgfqpoint{3.685089in}{1.816113in}}%
\pgfpathlineto{\pgfqpoint{3.677189in}{1.813962in}}%
\pgfpathlineto{\pgfqpoint{3.669279in}{1.812062in}}%
\pgfpathlineto{\pgfqpoint{3.661359in}{1.810421in}}%
\pgfpathlineto{\pgfqpoint{3.648073in}{1.816230in}}%
\pgfpathlineto{\pgfqpoint{3.634791in}{1.822066in}}%
\pgfpathlineto{\pgfqpoint{3.621515in}{1.827929in}}%
\pgfpathlineto{\pgfqpoint{3.608243in}{1.833819in}}%
\pgfpathlineto{\pgfqpoint{3.616186in}{1.835182in}}%
\pgfpathlineto{\pgfqpoint{3.624119in}{1.836807in}}%
\pgfpathlineto{\pgfqpoint{3.632041in}{1.838686in}}%
\pgfpathlineto{\pgfqpoint{3.639954in}{1.840814in}}%
\pgfpathclose%
\pgfusepath{fill}%
\end{pgfscope}%
\begin{pgfscope}%
\pgfpathrectangle{\pgfqpoint{1.254980in}{0.150000in}}{\pgfqpoint{5.490039in}{5.490039in}}%
\pgfusepath{clip}%
\pgfsetbuttcap%
\pgfsetroundjoin%
\definecolor{currentfill}{rgb}{0.268510,0.009605,0.335427}%
\pgfsetfillcolor{currentfill}%
\pgfsetfillopacity{0.700000}%
\pgfsetlinewidth{0.000000pt}%
\definecolor{currentstroke}{rgb}{0.000000,0.000000,0.000000}%
\pgfsetstrokecolor{currentstroke}%
\pgfsetdash{}{0pt}%
\pgfpathmoveto{\pgfqpoint{4.327176in}{1.767704in}}%
\pgfpathlineto{\pgfqpoint{4.340574in}{1.764278in}}%
\pgfpathlineto{\pgfqpoint{4.353977in}{1.760877in}}%
\pgfpathlineto{\pgfqpoint{4.367387in}{1.757500in}}%
\pgfpathlineto{\pgfqpoint{4.380804in}{1.754148in}}%
\pgfpathlineto{\pgfqpoint{4.373210in}{1.746170in}}%
\pgfpathlineto{\pgfqpoint{4.365611in}{1.738284in}}%
\pgfpathlineto{\pgfqpoint{4.358007in}{1.730493in}}%
\pgfpathlineto{\pgfqpoint{4.350397in}{1.722803in}}%
\pgfpathlineto{\pgfqpoint{4.336969in}{1.726364in}}%
\pgfpathlineto{\pgfqpoint{4.323547in}{1.729950in}}%
\pgfpathlineto{\pgfqpoint{4.310131in}{1.733560in}}%
\pgfpathlineto{\pgfqpoint{4.296722in}{1.737194in}}%
\pgfpathlineto{\pgfqpoint{4.304343in}{1.744670in}}%
\pgfpathlineto{\pgfqpoint{4.311960in}{1.752250in}}%
\pgfpathlineto{\pgfqpoint{4.319571in}{1.759930in}}%
\pgfpathlineto{\pgfqpoint{4.327176in}{1.767704in}}%
\pgfpathclose%
\pgfusepath{fill}%
\end{pgfscope}%
\begin{pgfscope}%
\pgfpathrectangle{\pgfqpoint{1.254980in}{0.150000in}}{\pgfqpoint{5.490039in}{5.490039in}}%
\pgfusepath{clip}%
\pgfsetbuttcap%
\pgfsetroundjoin%
\definecolor{currentfill}{rgb}{0.272594,0.025563,0.353093}%
\pgfsetfillcolor{currentfill}%
\pgfsetfillopacity{0.700000}%
\pgfsetlinewidth{0.000000pt}%
\definecolor{currentstroke}{rgb}{0.000000,0.000000,0.000000}%
\pgfsetstrokecolor{currentstroke}%
\pgfsetdash{}{0pt}%
\pgfpathmoveto{\pgfqpoint{4.548778in}{1.797735in}}%
\pgfpathlineto{\pgfqpoint{4.562234in}{1.794979in}}%
\pgfpathlineto{\pgfqpoint{4.575698in}{1.792247in}}%
\pgfpathlineto{\pgfqpoint{4.589169in}{1.789539in}}%
\pgfpathlineto{\pgfqpoint{4.602647in}{1.786854in}}%
\pgfpathlineto{\pgfqpoint{4.595124in}{1.777789in}}%
\pgfpathlineto{\pgfqpoint{4.587596in}{1.768766in}}%
\pgfpathlineto{\pgfqpoint{4.580063in}{1.759792in}}%
\pgfpathlineto{\pgfqpoint{4.572525in}{1.750869in}}%
\pgfpathlineto{\pgfqpoint{4.559038in}{1.753737in}}%
\pgfpathlineto{\pgfqpoint{4.545557in}{1.756629in}}%
\pgfpathlineto{\pgfqpoint{4.532083in}{1.759545in}}%
\pgfpathlineto{\pgfqpoint{4.518617in}{1.762484in}}%
\pgfpathlineto{\pgfqpoint{4.526164in}{1.771218in}}%
\pgfpathlineto{\pgfqpoint{4.533707in}{1.780008in}}%
\pgfpathlineto{\pgfqpoint{4.541245in}{1.788848in}}%
\pgfpathlineto{\pgfqpoint{4.548778in}{1.797735in}}%
\pgfpathclose%
\pgfusepath{fill}%
\end{pgfscope}%
\begin{pgfscope}%
\pgfpathrectangle{\pgfqpoint{1.254980in}{0.150000in}}{\pgfqpoint{5.490039in}{5.490039in}}%
\pgfusepath{clip}%
\pgfsetbuttcap%
\pgfsetroundjoin%
\definecolor{currentfill}{rgb}{0.281412,0.155834,0.469201}%
\pgfsetfillcolor{currentfill}%
\pgfsetfillopacity{0.700000}%
\pgfsetlinewidth{0.000000pt}%
\definecolor{currentstroke}{rgb}{0.000000,0.000000,0.000000}%
\pgfsetstrokecolor{currentstroke}%
\pgfsetdash{}{0pt}%
\pgfpathmoveto{\pgfqpoint{5.328439in}{2.033549in}}%
\pgfpathlineto{\pgfqpoint{5.342140in}{2.032750in}}%
\pgfpathlineto{\pgfqpoint{5.355848in}{2.031974in}}%
\pgfpathlineto{\pgfqpoint{5.369566in}{2.031223in}}%
\pgfpathlineto{\pgfqpoint{5.383292in}{2.030495in}}%
\pgfpathlineto{\pgfqpoint{5.376031in}{2.021084in}}%
\pgfpathlineto{\pgfqpoint{5.368763in}{2.011590in}}%
\pgfpathlineto{\pgfqpoint{5.361488in}{2.002013in}}%
\pgfpathlineto{\pgfqpoint{5.354206in}{1.992356in}}%
\pgfpathlineto{\pgfqpoint{5.340472in}{1.993164in}}%
\pgfpathlineto{\pgfqpoint{5.326746in}{1.993996in}}%
\pgfpathlineto{\pgfqpoint{5.313029in}{1.994852in}}%
\pgfpathlineto{\pgfqpoint{5.299320in}{1.995732in}}%
\pgfpathlineto{\pgfqpoint{5.306610in}{2.005303in}}%
\pgfpathlineto{\pgfqpoint{5.313893in}{2.014798in}}%
\pgfpathlineto{\pgfqpoint{5.321170in}{2.024214in}}%
\pgfpathlineto{\pgfqpoint{5.328439in}{2.033549in}}%
\pgfpathclose%
\pgfusepath{fill}%
\end{pgfscope}%
\begin{pgfscope}%
\pgfpathrectangle{\pgfqpoint{1.254980in}{0.150000in}}{\pgfqpoint{5.490039in}{5.490039in}}%
\pgfusepath{clip}%
\pgfsetbuttcap%
\pgfsetroundjoin%
\definecolor{currentfill}{rgb}{0.280267,0.073417,0.397163}%
\pgfsetfillcolor{currentfill}%
\pgfsetfillopacity{0.700000}%
\pgfsetlinewidth{0.000000pt}%
\definecolor{currentstroke}{rgb}{0.000000,0.000000,0.000000}%
\pgfsetstrokecolor{currentstroke}%
\pgfsetdash{}{0pt}%
\pgfpathmoveto{\pgfqpoint{4.854523in}{1.872881in}}%
\pgfpathlineto{\pgfqpoint{4.868071in}{1.870980in}}%
\pgfpathlineto{\pgfqpoint{4.881626in}{1.869103in}}%
\pgfpathlineto{\pgfqpoint{4.895188in}{1.867250in}}%
\pgfpathlineto{\pgfqpoint{4.908759in}{1.865421in}}%
\pgfpathlineto{\pgfqpoint{4.901329in}{1.855585in}}%
\pgfpathlineto{\pgfqpoint{4.893895in}{1.845733in}}%
\pgfpathlineto{\pgfqpoint{4.886455in}{1.835869in}}%
\pgfpathlineto{\pgfqpoint{4.879010in}{1.825997in}}%
\pgfpathlineto{\pgfqpoint{4.865432in}{1.827971in}}%
\pgfpathlineto{\pgfqpoint{4.851861in}{1.829970in}}%
\pgfpathlineto{\pgfqpoint{4.838297in}{1.831992in}}%
\pgfpathlineto{\pgfqpoint{4.824742in}{1.834037in}}%
\pgfpathlineto{\pgfqpoint{4.832195in}{1.843760in}}%
\pgfpathlineto{\pgfqpoint{4.839643in}{1.853477in}}%
\pgfpathlineto{\pgfqpoint{4.847086in}{1.863185in}}%
\pgfpathlineto{\pgfqpoint{4.854523in}{1.872881in}}%
\pgfpathclose%
\pgfusepath{fill}%
\end{pgfscope}%
\begin{pgfscope}%
\pgfpathrectangle{\pgfqpoint{1.254980in}{0.150000in}}{\pgfqpoint{5.490039in}{5.490039in}}%
\pgfusepath{clip}%
\pgfsetbuttcap%
\pgfsetroundjoin%
\definecolor{currentfill}{rgb}{0.248629,0.278775,0.534556}%
\pgfsetfillcolor{currentfill}%
\pgfsetfillopacity{0.700000}%
\pgfsetlinewidth{0.000000pt}%
\definecolor{currentstroke}{rgb}{0.000000,0.000000,0.000000}%
\pgfsetstrokecolor{currentstroke}%
\pgfsetdash{}{0pt}%
\pgfpathmoveto{\pgfqpoint{2.770453in}{2.269095in}}%
\pgfpathlineto{\pgfqpoint{2.783608in}{2.260519in}}%
\pgfpathlineto{\pgfqpoint{2.796766in}{2.251980in}}%
\pgfpathlineto{\pgfqpoint{2.809927in}{2.243479in}}%
\pgfpathlineto{\pgfqpoint{2.823091in}{2.235016in}}%
\pgfpathlineto{\pgfqpoint{2.814554in}{2.242004in}}%
\pgfpathlineto{\pgfqpoint{2.805996in}{2.249415in}}%
\pgfpathlineto{\pgfqpoint{2.797415in}{2.257259in}}%
\pgfpathlineto{\pgfqpoint{2.788812in}{2.265545in}}%
\pgfpathlineto{\pgfqpoint{2.775610in}{2.274343in}}%
\pgfpathlineto{\pgfqpoint{2.762410in}{2.283178in}}%
\pgfpathlineto{\pgfqpoint{2.749212in}{2.292051in}}%
\pgfpathlineto{\pgfqpoint{2.736017in}{2.300962in}}%
\pgfpathlineto{\pgfqpoint{2.744660in}{2.292336in}}%
\pgfpathlineto{\pgfqpoint{2.753280in}{2.284156in}}%
\pgfpathlineto{\pgfqpoint{2.761878in}{2.276412in}}%
\pgfpathlineto{\pgfqpoint{2.770453in}{2.269095in}}%
\pgfpathclose%
\pgfusepath{fill}%
\end{pgfscope}%
\begin{pgfscope}%
\pgfpathrectangle{\pgfqpoint{1.254980in}{0.150000in}}{\pgfqpoint{5.490039in}{5.490039in}}%
\pgfusepath{clip}%
\pgfsetbuttcap%
\pgfsetroundjoin%
\definecolor{currentfill}{rgb}{0.197636,0.391528,0.554969}%
\pgfsetfillcolor{currentfill}%
\pgfsetfillopacity{0.700000}%
\pgfsetlinewidth{0.000000pt}%
\definecolor{currentstroke}{rgb}{0.000000,0.000000,0.000000}%
\pgfsetstrokecolor{currentstroke}%
\pgfsetdash{}{0pt}%
\pgfpathmoveto{\pgfqpoint{2.419919in}{2.527257in}}%
\pgfpathlineto{\pgfqpoint{2.433072in}{2.517315in}}%
\pgfpathlineto{\pgfqpoint{2.446226in}{2.507421in}}%
\pgfpathlineto{\pgfqpoint{2.459381in}{2.497575in}}%
\pgfpathlineto{\pgfqpoint{2.472537in}{2.487776in}}%
\pgfpathlineto{\pgfqpoint{2.463653in}{2.498588in}}%
\pgfpathlineto{\pgfqpoint{2.454742in}{2.509886in}}%
\pgfpathlineto{\pgfqpoint{2.445802in}{2.521680in}}%
\pgfpathlineto{\pgfqpoint{2.436832in}{2.533978in}}%
\pgfpathlineto{\pgfqpoint{2.423630in}{2.544132in}}%
\pgfpathlineto{\pgfqpoint{2.410428in}{2.554333in}}%
\pgfpathlineto{\pgfqpoint{2.397228in}{2.564582in}}%
\pgfpathlineto{\pgfqpoint{2.384029in}{2.574879in}}%
\pgfpathlineto{\pgfqpoint{2.393046in}{2.562219in}}%
\pgfpathlineto{\pgfqpoint{2.402033in}{2.550069in}}%
\pgfpathlineto{\pgfqpoint{2.410991in}{2.538418in}}%
\pgfpathlineto{\pgfqpoint{2.419919in}{2.527257in}}%
\pgfpathclose%
\pgfusepath{fill}%
\end{pgfscope}%
\begin{pgfscope}%
\pgfpathrectangle{\pgfqpoint{1.254980in}{0.150000in}}{\pgfqpoint{5.490039in}{5.490039in}}%
\pgfusepath{clip}%
\pgfsetbuttcap%
\pgfsetroundjoin%
\definecolor{currentfill}{rgb}{0.282623,0.140926,0.457517}%
\pgfsetfillcolor{currentfill}%
\pgfsetfillopacity{0.700000}%
\pgfsetlinewidth{0.000000pt}%
\definecolor{currentstroke}{rgb}{0.000000,0.000000,0.000000}%
\pgfsetstrokecolor{currentstroke}%
\pgfsetdash{}{0pt}%
\pgfpathmoveto{\pgfqpoint{5.244569in}{1.999486in}}%
\pgfpathlineto{\pgfqpoint{5.258244in}{1.998512in}}%
\pgfpathlineto{\pgfqpoint{5.271927in}{1.997562in}}%
\pgfpathlineto{\pgfqpoint{5.285619in}{1.996635in}}%
\pgfpathlineto{\pgfqpoint{5.299320in}{1.995732in}}%
\pgfpathlineto{\pgfqpoint{5.292023in}{1.986085in}}%
\pgfpathlineto{\pgfqpoint{5.284720in}{1.976364in}}%
\pgfpathlineto{\pgfqpoint{5.277410in}{1.966572in}}%
\pgfpathlineto{\pgfqpoint{5.270094in}{1.956710in}}%
\pgfpathlineto{\pgfqpoint{5.256386in}{1.957707in}}%
\pgfpathlineto{\pgfqpoint{5.242686in}{1.958727in}}%
\pgfpathlineto{\pgfqpoint{5.228994in}{1.959771in}}%
\pgfpathlineto{\pgfqpoint{5.215311in}{1.960839in}}%
\pgfpathlineto{\pgfqpoint{5.222635in}{1.970602in}}%
\pgfpathlineto{\pgfqpoint{5.229952in}{1.980299in}}%
\pgfpathlineto{\pgfqpoint{5.237264in}{1.989928in}}%
\pgfpathlineto{\pgfqpoint{5.244569in}{1.999486in}}%
\pgfpathclose%
\pgfusepath{fill}%
\end{pgfscope}%
\begin{pgfscope}%
\pgfpathrectangle{\pgfqpoint{1.254980in}{0.150000in}}{\pgfqpoint{5.490039in}{5.490039in}}%
\pgfusepath{clip}%
\pgfsetbuttcap%
\pgfsetroundjoin%
\definecolor{currentfill}{rgb}{0.277941,0.056324,0.381191}%
\pgfsetfillcolor{currentfill}%
\pgfsetfillopacity{0.700000}%
\pgfsetlinewidth{0.000000pt}%
\definecolor{currentstroke}{rgb}{0.000000,0.000000,0.000000}%
\pgfsetstrokecolor{currentstroke}%
\pgfsetdash{}{0pt}%
\pgfpathmoveto{\pgfqpoint{4.770595in}{1.842458in}}%
\pgfpathlineto{\pgfqpoint{4.784120in}{1.840317in}}%
\pgfpathlineto{\pgfqpoint{4.797653in}{1.838200in}}%
\pgfpathlineto{\pgfqpoint{4.811194in}{1.836107in}}%
\pgfpathlineto{\pgfqpoint{4.824742in}{1.834037in}}%
\pgfpathlineto{\pgfqpoint{4.817284in}{1.824313in}}%
\pgfpathlineto{\pgfqpoint{4.809821in}{1.814590in}}%
\pgfpathlineto{\pgfqpoint{4.802353in}{1.804873in}}%
\pgfpathlineto{\pgfqpoint{4.794881in}{1.795165in}}%
\pgfpathlineto{\pgfqpoint{4.781325in}{1.797392in}}%
\pgfpathlineto{\pgfqpoint{4.767776in}{1.799644in}}%
\pgfpathlineto{\pgfqpoint{4.754234in}{1.801919in}}%
\pgfpathlineto{\pgfqpoint{4.740700in}{1.804218in}}%
\pgfpathlineto{\pgfqpoint{4.748181in}{1.813763in}}%
\pgfpathlineto{\pgfqpoint{4.755657in}{1.823321in}}%
\pgfpathlineto{\pgfqpoint{4.763129in}{1.832887in}}%
\pgfpathlineto{\pgfqpoint{4.770595in}{1.842458in}}%
\pgfpathclose%
\pgfusepath{fill}%
\end{pgfscope}%
\begin{pgfscope}%
\pgfpathrectangle{\pgfqpoint{1.254980in}{0.150000in}}{\pgfqpoint{5.490039in}{5.490039in}}%
\pgfusepath{clip}%
\pgfsetbuttcap%
\pgfsetroundjoin%
\definecolor{currentfill}{rgb}{0.269944,0.014625,0.341379}%
\pgfsetfillcolor{currentfill}%
\pgfsetfillopacity{0.700000}%
\pgfsetlinewidth{0.000000pt}%
\definecolor{currentstroke}{rgb}{0.000000,0.000000,0.000000}%
\pgfsetstrokecolor{currentstroke}%
\pgfsetdash{}{0pt}%
\pgfpathmoveto{\pgfqpoint{3.968101in}{1.768770in}}%
\pgfpathlineto{\pgfqpoint{3.981418in}{1.764184in}}%
\pgfpathlineto{\pgfqpoint{3.994740in}{1.759623in}}%
\pgfpathlineto{\pgfqpoint{4.008069in}{1.755088in}}%
\pgfpathlineto{\pgfqpoint{4.021403in}{1.750578in}}%
\pgfpathlineto{\pgfqpoint{4.013670in}{1.745262in}}%
\pgfpathlineto{\pgfqpoint{4.005931in}{1.740119in}}%
\pgfpathlineto{\pgfqpoint{3.998185in}{1.735158in}}%
\pgfpathlineto{\pgfqpoint{3.990431in}{1.730385in}}%
\pgfpathlineto{\pgfqpoint{3.977080in}{1.735142in}}%
\pgfpathlineto{\pgfqpoint{3.963735in}{1.739924in}}%
\pgfpathlineto{\pgfqpoint{3.950395in}{1.744731in}}%
\pgfpathlineto{\pgfqpoint{3.937061in}{1.749564in}}%
\pgfpathlineto{\pgfqpoint{3.944832in}{1.754086in}}%
\pgfpathlineto{\pgfqpoint{3.952595in}{1.758798in}}%
\pgfpathlineto{\pgfqpoint{3.960351in}{1.763695in}}%
\pgfpathlineto{\pgfqpoint{3.968101in}{1.768770in}}%
\pgfpathclose%
\pgfusepath{fill}%
\end{pgfscope}%
\begin{pgfscope}%
\pgfpathrectangle{\pgfqpoint{1.254980in}{0.150000in}}{\pgfqpoint{5.490039in}{5.490039in}}%
\pgfusepath{clip}%
\pgfsetbuttcap%
\pgfsetroundjoin%
\definecolor{currentfill}{rgb}{0.275191,0.194905,0.496005}%
\pgfsetfillcolor{currentfill}%
\pgfsetfillopacity{0.700000}%
\pgfsetlinewidth{0.000000pt}%
\definecolor{currentstroke}{rgb}{0.000000,0.000000,0.000000}%
\pgfsetstrokecolor{currentstroke}%
\pgfsetdash{}{0pt}%
\pgfpathmoveto{\pgfqpoint{3.067464in}{2.085672in}}%
\pgfpathlineto{\pgfqpoint{3.080644in}{2.078122in}}%
\pgfpathlineto{\pgfqpoint{3.093826in}{2.070606in}}%
\pgfpathlineto{\pgfqpoint{3.107013in}{2.063122in}}%
\pgfpathlineto{\pgfqpoint{3.120203in}{2.055671in}}%
\pgfpathlineto{\pgfqpoint{3.111918in}{2.059494in}}%
\pgfpathlineto{\pgfqpoint{3.103617in}{2.063686in}}%
\pgfpathlineto{\pgfqpoint{3.095298in}{2.068257in}}%
\pgfpathlineto{\pgfqpoint{3.086961in}{2.073214in}}%
\pgfpathlineto{\pgfqpoint{3.073738in}{2.080982in}}%
\pgfpathlineto{\pgfqpoint{3.060518in}{2.088783in}}%
\pgfpathlineto{\pgfqpoint{3.047301in}{2.096616in}}%
\pgfpathlineto{\pgfqpoint{3.034088in}{2.104483in}}%
\pgfpathlineto{\pgfqpoint{3.042459in}{2.099203in}}%
\pgfpathlineto{\pgfqpoint{3.050812in}{2.094314in}}%
\pgfpathlineto{\pgfqpoint{3.059147in}{2.089806in}}%
\pgfpathlineto{\pgfqpoint{3.067464in}{2.085672in}}%
\pgfpathclose%
\pgfusepath{fill}%
\end{pgfscope}%
\begin{pgfscope}%
\pgfpathrectangle{\pgfqpoint{1.254980in}{0.150000in}}{\pgfqpoint{5.490039in}{5.490039in}}%
\pgfusepath{clip}%
\pgfsetbuttcap%
\pgfsetroundjoin%
\definecolor{currentfill}{rgb}{0.268510,0.009605,0.335427}%
\pgfsetfillcolor{currentfill}%
\pgfsetfillopacity{0.700000}%
\pgfsetlinewidth{0.000000pt}%
\definecolor{currentstroke}{rgb}{0.000000,0.000000,0.000000}%
\pgfsetstrokecolor{currentstroke}%
\pgfsetdash{}{0pt}%
\pgfpathmoveto{\pgfqpoint{4.105600in}{1.756619in}}%
\pgfpathlineto{\pgfqpoint{4.118948in}{1.752468in}}%
\pgfpathlineto{\pgfqpoint{4.132303in}{1.748343in}}%
\pgfpathlineto{\pgfqpoint{4.145663in}{1.744242in}}%
\pgfpathlineto{\pgfqpoint{4.159029in}{1.740165in}}%
\pgfpathlineto{\pgfqpoint{4.151353in}{1.733752in}}%
\pgfpathlineto{\pgfqpoint{4.143670in}{1.727482in}}%
\pgfpathlineto{\pgfqpoint{4.135981in}{1.721363in}}%
\pgfpathlineto{\pgfqpoint{4.128287in}{1.715400in}}%
\pgfpathlineto{\pgfqpoint{4.114906in}{1.719710in}}%
\pgfpathlineto{\pgfqpoint{4.101530in}{1.724045in}}%
\pgfpathlineto{\pgfqpoint{4.088161in}{1.728405in}}%
\pgfpathlineto{\pgfqpoint{4.074798in}{1.732790in}}%
\pgfpathlineto{\pgfqpoint{4.082508in}{1.738514in}}%
\pgfpathlineto{\pgfqpoint{4.090212in}{1.744398in}}%
\pgfpathlineto{\pgfqpoint{4.097909in}{1.750435in}}%
\pgfpathlineto{\pgfqpoint{4.105600in}{1.756619in}}%
\pgfpathclose%
\pgfusepath{fill}%
\end{pgfscope}%
\begin{pgfscope}%
\pgfpathrectangle{\pgfqpoint{1.254980in}{0.150000in}}{\pgfqpoint{5.490039in}{5.490039in}}%
\pgfusepath{clip}%
\pgfsetbuttcap%
\pgfsetroundjoin%
\definecolor{currentfill}{rgb}{0.271828,0.209303,0.504434}%
\pgfsetfillcolor{currentfill}%
\pgfsetfillopacity{0.700000}%
\pgfsetlinewidth{0.000000pt}%
\definecolor{currentstroke}{rgb}{0.000000,0.000000,0.000000}%
\pgfsetstrokecolor{currentstroke}%
\pgfsetdash{}{0pt}%
\pgfpathmoveto{\pgfqpoint{5.634901in}{2.131490in}}%
\pgfpathlineto{\pgfqpoint{5.648713in}{2.131232in}}%
\pgfpathlineto{\pgfqpoint{5.662533in}{2.130997in}}%
\pgfpathlineto{\pgfqpoint{5.676363in}{2.130786in}}%
\pgfpathlineto{\pgfqpoint{5.669230in}{2.122338in}}%
\pgfpathlineto{\pgfqpoint{5.662089in}{2.113782in}}%
\pgfpathlineto{\pgfqpoint{5.654940in}{2.105118in}}%
\pgfpathlineto{\pgfqpoint{5.647782in}{2.096347in}}%
\pgfpathlineto{\pgfqpoint{5.633942in}{2.096598in}}%
\pgfpathlineto{\pgfqpoint{5.620111in}{2.096873in}}%
\pgfpathlineto{\pgfqpoint{5.606289in}{2.097171in}}%
\pgfpathlineto{\pgfqpoint{5.613455in}{2.105909in}}%
\pgfpathlineto{\pgfqpoint{5.620612in}{2.114541in}}%
\pgfpathlineto{\pgfqpoint{5.627760in}{2.123069in}}%
\pgfpathlineto{\pgfqpoint{5.634901in}{2.131490in}}%
\pgfpathclose%
\pgfusepath{fill}%
\end{pgfscope}%
\begin{pgfscope}%
\pgfpathrectangle{\pgfqpoint{1.254980in}{0.150000in}}{\pgfqpoint{5.490039in}{5.490039in}}%
\pgfusepath{clip}%
\pgfsetbuttcap%
\pgfsetroundjoin%
\definecolor{currentfill}{rgb}{0.283187,0.125848,0.444960}%
\pgfsetfillcolor{currentfill}%
\pgfsetfillopacity{0.700000}%
\pgfsetlinewidth{0.000000pt}%
\definecolor{currentstroke}{rgb}{0.000000,0.000000,0.000000}%
\pgfsetstrokecolor{currentstroke}%
\pgfsetdash{}{0pt}%
\pgfpathmoveto{\pgfqpoint{5.160660in}{1.965347in}}%
\pgfpathlineto{\pgfqpoint{5.174310in}{1.964184in}}%
\pgfpathlineto{\pgfqpoint{5.187969in}{1.963046in}}%
\pgfpathlineto{\pgfqpoint{5.201636in}{1.961931in}}%
\pgfpathlineto{\pgfqpoint{5.215311in}{1.960839in}}%
\pgfpathlineto{\pgfqpoint{5.207980in}{1.951012in}}%
\pgfpathlineto{\pgfqpoint{5.200644in}{1.941123in}}%
\pgfpathlineto{\pgfqpoint{5.193301in}{1.931174in}}%
\pgfpathlineto{\pgfqpoint{5.185953in}{1.921168in}}%
\pgfpathlineto{\pgfqpoint{5.172270in}{1.922366in}}%
\pgfpathlineto{\pgfqpoint{5.158595in}{1.923587in}}%
\pgfpathlineto{\pgfqpoint{5.144929in}{1.924833in}}%
\pgfpathlineto{\pgfqpoint{5.131271in}{1.926102in}}%
\pgfpathlineto{\pgfqpoint{5.138627in}{1.935996in}}%
\pgfpathlineto{\pgfqpoint{5.145978in}{1.945837in}}%
\pgfpathlineto{\pgfqpoint{5.153322in}{1.955621in}}%
\pgfpathlineto{\pgfqpoint{5.160660in}{1.965347in}}%
\pgfpathclose%
\pgfusepath{fill}%
\end{pgfscope}%
\begin{pgfscope}%
\pgfpathrectangle{\pgfqpoint{1.254980in}{0.150000in}}{\pgfqpoint{5.490039in}{5.490039in}}%
\pgfusepath{clip}%
\pgfsetbuttcap%
\pgfsetroundjoin%
\definecolor{currentfill}{rgb}{0.273809,0.031497,0.358853}%
\pgfsetfillcolor{currentfill}%
\pgfsetfillopacity{0.700000}%
\pgfsetlinewidth{0.000000pt}%
\definecolor{currentstroke}{rgb}{0.000000,0.000000,0.000000}%
\pgfsetstrokecolor{currentstroke}%
\pgfsetdash{}{0pt}%
\pgfpathmoveto{\pgfqpoint{3.830585in}{1.789147in}}%
\pgfpathlineto{\pgfqpoint{3.843876in}{1.784109in}}%
\pgfpathlineto{\pgfqpoint{3.857171in}{1.779097in}}%
\pgfpathlineto{\pgfqpoint{3.870473in}{1.774110in}}%
\pgfpathlineto{\pgfqpoint{3.883779in}{1.769150in}}%
\pgfpathlineto{\pgfqpoint{3.875983in}{1.765080in}}%
\pgfpathlineto{\pgfqpoint{3.868179in}{1.761218in}}%
\pgfpathlineto{\pgfqpoint{3.860366in}{1.757569in}}%
\pgfpathlineto{\pgfqpoint{3.852546in}{1.754141in}}%
\pgfpathlineto{\pgfqpoint{3.839220in}{1.759362in}}%
\pgfpathlineto{\pgfqpoint{3.825899in}{1.764608in}}%
\pgfpathlineto{\pgfqpoint{3.812584in}{1.769880in}}%
\pgfpathlineto{\pgfqpoint{3.799274in}{1.775178in}}%
\pgfpathlineto{\pgfqpoint{3.807115in}{1.778342in}}%
\pgfpathlineto{\pgfqpoint{3.814946in}{1.781728in}}%
\pgfpathlineto{\pgfqpoint{3.822770in}{1.785333in}}%
\pgfpathlineto{\pgfqpoint{3.830585in}{1.789147in}}%
\pgfpathclose%
\pgfusepath{fill}%
\end{pgfscope}%
\begin{pgfscope}%
\pgfpathrectangle{\pgfqpoint{1.254980in}{0.150000in}}{\pgfqpoint{5.490039in}{5.490039in}}%
\pgfusepath{clip}%
\pgfsetbuttcap%
\pgfsetroundjoin%
\definecolor{currentfill}{rgb}{0.271305,0.019942,0.347269}%
\pgfsetfillcolor{currentfill}%
\pgfsetfillopacity{0.700000}%
\pgfsetlinewidth{0.000000pt}%
\definecolor{currentstroke}{rgb}{0.000000,0.000000,0.000000}%
\pgfsetstrokecolor{currentstroke}%
\pgfsetdash{}{0pt}%
\pgfpathmoveto{\pgfqpoint{4.464818in}{1.774482in}}%
\pgfpathlineto{\pgfqpoint{4.478258in}{1.771447in}}%
\pgfpathlineto{\pgfqpoint{4.491704in}{1.768435in}}%
\pgfpathlineto{\pgfqpoint{4.505157in}{1.765448in}}%
\pgfpathlineto{\pgfqpoint{4.518617in}{1.762484in}}%
\pgfpathlineto{\pgfqpoint{4.511064in}{1.753810in}}%
\pgfpathlineto{\pgfqpoint{4.503507in}{1.745201in}}%
\pgfpathlineto{\pgfqpoint{4.495945in}{1.736662in}}%
\pgfpathlineto{\pgfqpoint{4.488378in}{1.728196in}}%
\pgfpathlineto{\pgfqpoint{4.474908in}{1.731356in}}%
\pgfpathlineto{\pgfqpoint{4.461444in}{1.734540in}}%
\pgfpathlineto{\pgfqpoint{4.447988in}{1.737748in}}%
\pgfpathlineto{\pgfqpoint{4.434537in}{1.740980in}}%
\pgfpathlineto{\pgfqpoint{4.442115in}{1.749244in}}%
\pgfpathlineto{\pgfqpoint{4.449688in}{1.757585in}}%
\pgfpathlineto{\pgfqpoint{4.457256in}{1.766000in}}%
\pgfpathlineto{\pgfqpoint{4.464818in}{1.774482in}}%
\pgfpathclose%
\pgfusepath{fill}%
\end{pgfscope}%
\begin{pgfscope}%
\pgfpathrectangle{\pgfqpoint{1.254980in}{0.150000in}}{\pgfqpoint{5.490039in}{5.490039in}}%
\pgfusepath{clip}%
\pgfsetbuttcap%
\pgfsetroundjoin%
\definecolor{currentfill}{rgb}{0.281924,0.089666,0.412415}%
\pgfsetfillcolor{currentfill}%
\pgfsetfillopacity{0.700000}%
\pgfsetlinewidth{0.000000pt}%
\definecolor{currentstroke}{rgb}{0.000000,0.000000,0.000000}%
\pgfsetstrokecolor{currentstroke}%
\pgfsetdash{}{0pt}%
\pgfpathmoveto{\pgfqpoint{3.502239in}{1.881930in}}%
\pgfpathlineto{\pgfqpoint{3.515473in}{1.875819in}}%
\pgfpathlineto{\pgfqpoint{3.528712in}{1.869736in}}%
\pgfpathlineto{\pgfqpoint{3.541956in}{1.863681in}}%
\pgfpathlineto{\pgfqpoint{3.555204in}{1.857653in}}%
\pgfpathlineto{\pgfqpoint{3.547227in}{1.856842in}}%
\pgfpathlineto{\pgfqpoint{3.539238in}{1.856311in}}%
\pgfpathlineto{\pgfqpoint{3.531238in}{1.856068in}}%
\pgfpathlineto{\pgfqpoint{3.523226in}{1.856121in}}%
\pgfpathlineto{\pgfqpoint{3.509953in}{1.862436in}}%
\pgfpathlineto{\pgfqpoint{3.496685in}{1.868778in}}%
\pgfpathlineto{\pgfqpoint{3.483420in}{1.875148in}}%
\pgfpathlineto{\pgfqpoint{3.470161in}{1.881547in}}%
\pgfpathlineto{\pgfqpoint{3.478198in}{1.881201in}}%
\pgfpathlineto{\pgfqpoint{3.486224in}{1.881156in}}%
\pgfpathlineto{\pgfqpoint{3.494237in}{1.881401in}}%
\pgfpathlineto{\pgfqpoint{3.502239in}{1.881930in}}%
\pgfpathclose%
\pgfusepath{fill}%
\end{pgfscope}%
\begin{pgfscope}%
\pgfpathrectangle{\pgfqpoint{1.254980in}{0.150000in}}{\pgfqpoint{5.490039in}{5.490039in}}%
\pgfusepath{clip}%
\pgfsetbuttcap%
\pgfsetroundjoin%
\definecolor{currentfill}{rgb}{0.204903,0.375746,0.553533}%
\pgfsetfillcolor{currentfill}%
\pgfsetfillopacity{0.700000}%
\pgfsetlinewidth{0.000000pt}%
\definecolor{currentstroke}{rgb}{0.000000,0.000000,0.000000}%
\pgfsetstrokecolor{currentstroke}%
\pgfsetdash{}{0pt}%
\pgfpathmoveto{\pgfqpoint{2.472537in}{2.487776in}}%
\pgfpathlineto{\pgfqpoint{2.485694in}{2.478022in}}%
\pgfpathlineto{\pgfqpoint{2.498854in}{2.468315in}}%
\pgfpathlineto{\pgfqpoint{2.512014in}{2.458653in}}%
\pgfpathlineto{\pgfqpoint{2.525176in}{2.449037in}}%
\pgfpathlineto{\pgfqpoint{2.516337in}{2.459502in}}%
\pgfpathlineto{\pgfqpoint{2.507471in}{2.470448in}}%
\pgfpathlineto{\pgfqpoint{2.498578in}{2.481885in}}%
\pgfpathlineto{\pgfqpoint{2.489655in}{2.493824in}}%
\pgfpathlineto{\pgfqpoint{2.476448in}{2.503794in}}%
\pgfpathlineto{\pgfqpoint{2.463241in}{2.513810in}}%
\pgfpathlineto{\pgfqpoint{2.450036in}{2.523871in}}%
\pgfpathlineto{\pgfqpoint{2.436832in}{2.533978in}}%
\pgfpathlineto{\pgfqpoint{2.445802in}{2.521680in}}%
\pgfpathlineto{\pgfqpoint{2.454742in}{2.509886in}}%
\pgfpathlineto{\pgfqpoint{2.463653in}{2.498588in}}%
\pgfpathlineto{\pgfqpoint{2.472537in}{2.487776in}}%
\pgfpathclose%
\pgfusepath{fill}%
\end{pgfscope}%
\begin{pgfscope}%
\pgfpathrectangle{\pgfqpoint{1.254980in}{0.150000in}}{\pgfqpoint{5.490039in}{5.490039in}}%
\pgfusepath{clip}%
\pgfsetbuttcap%
\pgfsetroundjoin%
\definecolor{currentfill}{rgb}{0.283072,0.130895,0.449241}%
\pgfsetfillcolor{currentfill}%
\pgfsetfillopacity{0.700000}%
\pgfsetlinewidth{0.000000pt}%
\definecolor{currentstroke}{rgb}{0.000000,0.000000,0.000000}%
\pgfsetstrokecolor{currentstroke}%
\pgfsetdash{}{0pt}%
\pgfpathmoveto{\pgfqpoint{3.311378in}{1.960565in}}%
\pgfpathlineto{\pgfqpoint{3.324587in}{1.953819in}}%
\pgfpathlineto{\pgfqpoint{3.337800in}{1.947103in}}%
\pgfpathlineto{\pgfqpoint{3.351017in}{1.940417in}}%
\pgfpathlineto{\pgfqpoint{3.364239in}{1.933760in}}%
\pgfpathlineto{\pgfqpoint{3.356135in}{1.935004in}}%
\pgfpathlineto{\pgfqpoint{3.348017in}{1.936569in}}%
\pgfpathlineto{\pgfqpoint{3.339885in}{1.938465in}}%
\pgfpathlineto{\pgfqpoint{3.331739in}{1.940699in}}%
\pgfpathlineto{\pgfqpoint{3.318489in}{1.947657in}}%
\pgfpathlineto{\pgfqpoint{3.305244in}{1.954645in}}%
\pgfpathlineto{\pgfqpoint{3.292002in}{1.961663in}}%
\pgfpathlineto{\pgfqpoint{3.278764in}{1.968710in}}%
\pgfpathlineto{\pgfqpoint{3.286939in}{1.966170in}}%
\pgfpathlineto{\pgfqpoint{3.295100in}{1.963971in}}%
\pgfpathlineto{\pgfqpoint{3.303246in}{1.962106in}}%
\pgfpathlineto{\pgfqpoint{3.311378in}{1.960565in}}%
\pgfpathclose%
\pgfusepath{fill}%
\end{pgfscope}%
\begin{pgfscope}%
\pgfpathrectangle{\pgfqpoint{1.254980in}{0.150000in}}{\pgfqpoint{5.490039in}{5.490039in}}%
\pgfusepath{clip}%
\pgfsetbuttcap%
\pgfsetroundjoin%
\definecolor{currentfill}{rgb}{0.268510,0.009605,0.335427}%
\pgfsetfillcolor{currentfill}%
\pgfsetfillopacity{0.700000}%
\pgfsetlinewidth{0.000000pt}%
\definecolor{currentstroke}{rgb}{0.000000,0.000000,0.000000}%
\pgfsetstrokecolor{currentstroke}%
\pgfsetdash{}{0pt}%
\pgfpathmoveto{\pgfqpoint{4.243149in}{1.751974in}}%
\pgfpathlineto{\pgfqpoint{4.256532in}{1.748242in}}%
\pgfpathlineto{\pgfqpoint{4.269923in}{1.744535in}}%
\pgfpathlineto{\pgfqpoint{4.283319in}{1.740852in}}%
\pgfpathlineto{\pgfqpoint{4.296722in}{1.737194in}}%
\pgfpathlineto{\pgfqpoint{4.289095in}{1.729827in}}%
\pgfpathlineto{\pgfqpoint{4.281463in}{1.722576in}}%
\pgfpathlineto{\pgfqpoint{4.273825in}{1.715446in}}%
\pgfpathlineto{\pgfqpoint{4.266181in}{1.708441in}}%
\pgfpathlineto{\pgfqpoint{4.252766in}{1.712321in}}%
\pgfpathlineto{\pgfqpoint{4.239356in}{1.716225in}}%
\pgfpathlineto{\pgfqpoint{4.225953in}{1.720154in}}%
\pgfpathlineto{\pgfqpoint{4.212556in}{1.724107in}}%
\pgfpathlineto{\pgfqpoint{4.220212in}{1.730885in}}%
\pgfpathlineto{\pgfqpoint{4.227863in}{1.737793in}}%
\pgfpathlineto{\pgfqpoint{4.235509in}{1.744824in}}%
\pgfpathlineto{\pgfqpoint{4.243149in}{1.751974in}}%
\pgfpathclose%
\pgfusepath{fill}%
\end{pgfscope}%
\begin{pgfscope}%
\pgfpathrectangle{\pgfqpoint{1.254980in}{0.150000in}}{\pgfqpoint{5.490039in}{5.490039in}}%
\pgfusepath{clip}%
\pgfsetbuttcap%
\pgfsetroundjoin%
\definecolor{currentfill}{rgb}{0.252194,0.269783,0.531579}%
\pgfsetfillcolor{currentfill}%
\pgfsetfillopacity{0.700000}%
\pgfsetlinewidth{0.000000pt}%
\definecolor{currentstroke}{rgb}{0.000000,0.000000,0.000000}%
\pgfsetstrokecolor{currentstroke}%
\pgfsetdash{}{0pt}%
\pgfpathmoveto{\pgfqpoint{2.823091in}{2.235016in}}%
\pgfpathlineto{\pgfqpoint{2.836256in}{2.226589in}}%
\pgfpathlineto{\pgfqpoint{2.849425in}{2.218200in}}%
\pgfpathlineto{\pgfqpoint{2.862597in}{2.209848in}}%
\pgfpathlineto{\pgfqpoint{2.875771in}{2.201532in}}%
\pgfpathlineto{\pgfqpoint{2.867272in}{2.208192in}}%
\pgfpathlineto{\pgfqpoint{2.858752in}{2.215271in}}%
\pgfpathlineto{\pgfqpoint{2.850210in}{2.222780in}}%
\pgfpathlineto{\pgfqpoint{2.841647in}{2.230727in}}%
\pgfpathlineto{\pgfqpoint{2.828435in}{2.239376in}}%
\pgfpathlineto{\pgfqpoint{2.815225in}{2.248062in}}%
\pgfpathlineto{\pgfqpoint{2.802017in}{2.256785in}}%
\pgfpathlineto{\pgfqpoint{2.788812in}{2.265545in}}%
\pgfpathlineto{\pgfqpoint{2.797415in}{2.257259in}}%
\pgfpathlineto{\pgfqpoint{2.805996in}{2.249415in}}%
\pgfpathlineto{\pgfqpoint{2.814554in}{2.242004in}}%
\pgfpathlineto{\pgfqpoint{2.823091in}{2.235016in}}%
\pgfpathclose%
\pgfusepath{fill}%
\end{pgfscope}%
\begin{pgfscope}%
\pgfpathrectangle{\pgfqpoint{1.254980in}{0.150000in}}{\pgfqpoint{5.490039in}{5.490039in}}%
\pgfusepath{clip}%
\pgfsetbuttcap%
\pgfsetroundjoin%
\definecolor{currentfill}{rgb}{0.283091,0.110553,0.431554}%
\pgfsetfillcolor{currentfill}%
\pgfsetfillopacity{0.700000}%
\pgfsetlinewidth{0.000000pt}%
\definecolor{currentstroke}{rgb}{0.000000,0.000000,0.000000}%
\pgfsetstrokecolor{currentstroke}%
\pgfsetdash{}{0pt}%
\pgfpathmoveto{\pgfqpoint{5.076720in}{1.931415in}}%
\pgfpathlineto{\pgfqpoint{5.090345in}{1.930051in}}%
\pgfpathlineto{\pgfqpoint{5.103979in}{1.928711in}}%
\pgfpathlineto{\pgfqpoint{5.117621in}{1.927395in}}%
\pgfpathlineto{\pgfqpoint{5.131271in}{1.926102in}}%
\pgfpathlineto{\pgfqpoint{5.123909in}{1.916156in}}%
\pgfpathlineto{\pgfqpoint{5.116541in}{1.906161in}}%
\pgfpathlineto{\pgfqpoint{5.109167in}{1.896120in}}%
\pgfpathlineto{\pgfqpoint{5.101788in}{1.886035in}}%
\pgfpathlineto{\pgfqpoint{5.088130in}{1.887447in}}%
\pgfpathlineto{\pgfqpoint{5.074481in}{1.888883in}}%
\pgfpathlineto{\pgfqpoint{5.060839in}{1.890343in}}%
\pgfpathlineto{\pgfqpoint{5.047206in}{1.891827in}}%
\pgfpathlineto{\pgfqpoint{5.054593in}{1.901787in}}%
\pgfpathlineto{\pgfqpoint{5.061974in}{1.911707in}}%
\pgfpathlineto{\pgfqpoint{5.069350in}{1.921584in}}%
\pgfpathlineto{\pgfqpoint{5.076720in}{1.931415in}}%
\pgfpathclose%
\pgfusepath{fill}%
\end{pgfscope}%
\begin{pgfscope}%
\pgfpathrectangle{\pgfqpoint{1.254980in}{0.150000in}}{\pgfqpoint{5.490039in}{5.490039in}}%
\pgfusepath{clip}%
\pgfsetbuttcap%
\pgfsetroundjoin%
\definecolor{currentfill}{rgb}{0.276022,0.044167,0.370164}%
\pgfsetfillcolor{currentfill}%
\pgfsetfillopacity{0.700000}%
\pgfsetlinewidth{0.000000pt}%
\definecolor{currentstroke}{rgb}{0.000000,0.000000,0.000000}%
\pgfsetstrokecolor{currentstroke}%
\pgfsetdash{}{0pt}%
\pgfpathmoveto{\pgfqpoint{4.686637in}{1.813651in}}%
\pgfpathlineto{\pgfqpoint{4.700142in}{1.811257in}}%
\pgfpathlineto{\pgfqpoint{4.713654in}{1.808887in}}%
\pgfpathlineto{\pgfqpoint{4.727173in}{1.806540in}}%
\pgfpathlineto{\pgfqpoint{4.740700in}{1.804218in}}%
\pgfpathlineto{\pgfqpoint{4.733214in}{1.794689in}}%
\pgfpathlineto{\pgfqpoint{4.725724in}{1.785180in}}%
\pgfpathlineto{\pgfqpoint{4.718228in}{1.775695in}}%
\pgfpathlineto{\pgfqpoint{4.710728in}{1.766239in}}%
\pgfpathlineto{\pgfqpoint{4.697193in}{1.768733in}}%
\pgfpathlineto{\pgfqpoint{4.683664in}{1.771250in}}%
\pgfpathlineto{\pgfqpoint{4.670143in}{1.773791in}}%
\pgfpathlineto{\pgfqpoint{4.656630in}{1.776356in}}%
\pgfpathlineto{\pgfqpoint{4.664139in}{1.785637in}}%
\pgfpathlineto{\pgfqpoint{4.671643in}{1.794949in}}%
\pgfpathlineto{\pgfqpoint{4.679143in}{1.804288in}}%
\pgfpathlineto{\pgfqpoint{4.686637in}{1.813651in}}%
\pgfpathclose%
\pgfusepath{fill}%
\end{pgfscope}%
\begin{pgfscope}%
\pgfpathrectangle{\pgfqpoint{1.254980in}{0.150000in}}{\pgfqpoint{5.490039in}{5.490039in}}%
\pgfusepath{clip}%
\pgfsetbuttcap%
\pgfsetroundjoin%
\definecolor{currentfill}{rgb}{0.277018,0.050344,0.375715}%
\pgfsetfillcolor{currentfill}%
\pgfsetfillopacity{0.700000}%
\pgfsetlinewidth{0.000000pt}%
\definecolor{currentstroke}{rgb}{0.000000,0.000000,0.000000}%
\pgfsetstrokecolor{currentstroke}%
\pgfsetdash{}{0pt}%
\pgfpathmoveto{\pgfqpoint{3.692980in}{1.818509in}}%
\pgfpathlineto{\pgfqpoint{3.706249in}{1.813000in}}%
\pgfpathlineto{\pgfqpoint{3.719523in}{1.807517in}}%
\pgfpathlineto{\pgfqpoint{3.732802in}{1.802062in}}%
\pgfpathlineto{\pgfqpoint{3.746086in}{1.796632in}}%
\pgfpathlineto{\pgfqpoint{3.738217in}{1.793968in}}%
\pgfpathlineto{\pgfqpoint{3.730338in}{1.791545in}}%
\pgfpathlineto{\pgfqpoint{3.722450in}{1.789370in}}%
\pgfpathlineto{\pgfqpoint{3.714552in}{1.787451in}}%
\pgfpathlineto{\pgfqpoint{3.701246in}{1.793154in}}%
\pgfpathlineto{\pgfqpoint{3.687946in}{1.798883in}}%
\pgfpathlineto{\pgfqpoint{3.674650in}{1.804638in}}%
\pgfpathlineto{\pgfqpoint{3.661359in}{1.810421in}}%
\pgfpathlineto{\pgfqpoint{3.669279in}{1.812062in}}%
\pgfpathlineto{\pgfqpoint{3.677189in}{1.813962in}}%
\pgfpathlineto{\pgfqpoint{3.685089in}{1.816113in}}%
\pgfpathlineto{\pgfqpoint{3.692980in}{1.818509in}}%
\pgfpathclose%
\pgfusepath{fill}%
\end{pgfscope}%
\begin{pgfscope}%
\pgfpathrectangle{\pgfqpoint{1.254980in}{0.150000in}}{\pgfqpoint{5.490039in}{5.490039in}}%
\pgfusepath{clip}%
\pgfsetbuttcap%
\pgfsetroundjoin%
\definecolor{currentfill}{rgb}{0.274128,0.199721,0.498911}%
\pgfsetfillcolor{currentfill}%
\pgfsetfillopacity{0.700000}%
\pgfsetlinewidth{0.000000pt}%
\definecolor{currentstroke}{rgb}{0.000000,0.000000,0.000000}%
\pgfsetstrokecolor{currentstroke}%
\pgfsetdash{}{0pt}%
\pgfpathmoveto{\pgfqpoint{5.551092in}{2.098604in}}%
\pgfpathlineto{\pgfqpoint{5.564878in}{2.098210in}}%
\pgfpathlineto{\pgfqpoint{5.578673in}{2.097840in}}%
\pgfpathlineto{\pgfqpoint{5.592477in}{2.097494in}}%
\pgfpathlineto{\pgfqpoint{5.606289in}{2.097171in}}%
\pgfpathlineto{\pgfqpoint{5.599116in}{2.088330in}}%
\pgfpathlineto{\pgfqpoint{5.591935in}{2.079386in}}%
\pgfpathlineto{\pgfqpoint{5.584746in}{2.070340in}}%
\pgfpathlineto{\pgfqpoint{5.577549in}{2.061192in}}%
\pgfpathlineto{\pgfqpoint{5.563726in}{2.061568in}}%
\pgfpathlineto{\pgfqpoint{5.549913in}{2.061968in}}%
\pgfpathlineto{\pgfqpoint{5.536108in}{2.062391in}}%
\pgfpathlineto{\pgfqpoint{5.522313in}{2.062839in}}%
\pgfpathlineto{\pgfqpoint{5.529519in}{2.071928in}}%
\pgfpathlineto{\pgfqpoint{5.536718in}{2.080919in}}%
\pgfpathlineto{\pgfqpoint{5.543909in}{2.089811in}}%
\pgfpathlineto{\pgfqpoint{5.551092in}{2.098604in}}%
\pgfpathclose%
\pgfusepath{fill}%
\end{pgfscope}%
\begin{pgfscope}%
\pgfpathrectangle{\pgfqpoint{1.254980in}{0.150000in}}{\pgfqpoint{5.490039in}{5.490039in}}%
\pgfusepath{clip}%
\pgfsetbuttcap%
\pgfsetroundjoin%
\definecolor{currentfill}{rgb}{0.282327,0.094955,0.417331}%
\pgfsetfillcolor{currentfill}%
\pgfsetfillopacity{0.700000}%
\pgfsetlinewidth{0.000000pt}%
\definecolor{currentstroke}{rgb}{0.000000,0.000000,0.000000}%
\pgfsetstrokecolor{currentstroke}%
\pgfsetdash{}{0pt}%
\pgfpathmoveto{\pgfqpoint{4.992752in}{1.897997in}}%
\pgfpathlineto{\pgfqpoint{5.006353in}{1.896419in}}%
\pgfpathlineto{\pgfqpoint{5.019963in}{1.894864in}}%
\pgfpathlineto{\pgfqpoint{5.033580in}{1.893334in}}%
\pgfpathlineto{\pgfqpoint{5.047206in}{1.891827in}}%
\pgfpathlineto{\pgfqpoint{5.039813in}{1.881829in}}%
\pgfpathlineto{\pgfqpoint{5.032415in}{1.871797in}}%
\pgfpathlineto{\pgfqpoint{5.025012in}{1.861733in}}%
\pgfpathlineto{\pgfqpoint{5.017603in}{1.851640in}}%
\pgfpathlineto{\pgfqpoint{5.003970in}{1.853280in}}%
\pgfpathlineto{\pgfqpoint{4.990345in}{1.854944in}}%
\pgfpathlineto{\pgfqpoint{4.976728in}{1.856631in}}%
\pgfpathlineto{\pgfqpoint{4.963118in}{1.858341in}}%
\pgfpathlineto{\pgfqpoint{4.970534in}{1.868296in}}%
\pgfpathlineto{\pgfqpoint{4.977946in}{1.878226in}}%
\pgfpathlineto{\pgfqpoint{4.985351in}{1.888127in}}%
\pgfpathlineto{\pgfqpoint{4.992752in}{1.897997in}}%
\pgfpathclose%
\pgfusepath{fill}%
\end{pgfscope}%
\begin{pgfscope}%
\pgfpathrectangle{\pgfqpoint{1.254980in}{0.150000in}}{\pgfqpoint{5.490039in}{5.490039in}}%
\pgfusepath{clip}%
\pgfsetbuttcap%
\pgfsetroundjoin%
\definecolor{currentfill}{rgb}{0.210503,0.363727,0.552206}%
\pgfsetfillcolor{currentfill}%
\pgfsetfillopacity{0.700000}%
\pgfsetlinewidth{0.000000pt}%
\definecolor{currentstroke}{rgb}{0.000000,0.000000,0.000000}%
\pgfsetstrokecolor{currentstroke}%
\pgfsetdash{}{0pt}%
\pgfpathmoveto{\pgfqpoint{2.525176in}{2.449037in}}%
\pgfpathlineto{\pgfqpoint{2.538340in}{2.439464in}}%
\pgfpathlineto{\pgfqpoint{2.551505in}{2.429936in}}%
\pgfpathlineto{\pgfqpoint{2.564672in}{2.420452in}}%
\pgfpathlineto{\pgfqpoint{2.577841in}{2.411011in}}%
\pgfpathlineto{\pgfqpoint{2.569046in}{2.421129in}}%
\pgfpathlineto{\pgfqpoint{2.560225in}{2.431724in}}%
\pgfpathlineto{\pgfqpoint{2.551376in}{2.442807in}}%
\pgfpathlineto{\pgfqpoint{2.542500in}{2.454386in}}%
\pgfpathlineto{\pgfqpoint{2.529287in}{2.464180in}}%
\pgfpathlineto{\pgfqpoint{2.516075in}{2.474017in}}%
\pgfpathlineto{\pgfqpoint{2.502864in}{2.483898in}}%
\pgfpathlineto{\pgfqpoint{2.489655in}{2.493824in}}%
\pgfpathlineto{\pgfqpoint{2.498578in}{2.481885in}}%
\pgfpathlineto{\pgfqpoint{2.507471in}{2.470448in}}%
\pgfpathlineto{\pgfqpoint{2.516337in}{2.459502in}}%
\pgfpathlineto{\pgfqpoint{2.525176in}{2.449037in}}%
\pgfpathclose%
\pgfusepath{fill}%
\end{pgfscope}%
\begin{pgfscope}%
\pgfpathrectangle{\pgfqpoint{1.254980in}{0.150000in}}{\pgfqpoint{5.490039in}{5.490039in}}%
\pgfusepath{clip}%
\pgfsetbuttcap%
\pgfsetroundjoin%
\definecolor{currentfill}{rgb}{0.269944,0.014625,0.341379}%
\pgfsetfillcolor{currentfill}%
\pgfsetfillopacity{0.700000}%
\pgfsetlinewidth{0.000000pt}%
\definecolor{currentstroke}{rgb}{0.000000,0.000000,0.000000}%
\pgfsetstrokecolor{currentstroke}%
\pgfsetdash{}{0pt}%
\pgfpathmoveto{\pgfqpoint{4.380804in}{1.754148in}}%
\pgfpathlineto{\pgfqpoint{4.394227in}{1.750819in}}%
\pgfpathlineto{\pgfqpoint{4.407657in}{1.747515in}}%
\pgfpathlineto{\pgfqpoint{4.421094in}{1.744235in}}%
\pgfpathlineto{\pgfqpoint{4.434537in}{1.740980in}}%
\pgfpathlineto{\pgfqpoint{4.426955in}{1.732798in}}%
\pgfpathlineto{\pgfqpoint{4.419367in}{1.724704in}}%
\pgfpathlineto{\pgfqpoint{4.411775in}{1.716703in}}%
\pgfpathlineto{\pgfqpoint{4.404177in}{1.708799in}}%
\pgfpathlineto{\pgfqpoint{4.390722in}{1.712264in}}%
\pgfpathlineto{\pgfqpoint{4.377274in}{1.715753in}}%
\pgfpathlineto{\pgfqpoint{4.363833in}{1.719266in}}%
\pgfpathlineto{\pgfqpoint{4.350397in}{1.722803in}}%
\pgfpathlineto{\pgfqpoint{4.358007in}{1.730493in}}%
\pgfpathlineto{\pgfqpoint{4.365611in}{1.738284in}}%
\pgfpathlineto{\pgfqpoint{4.373210in}{1.746170in}}%
\pgfpathlineto{\pgfqpoint{4.380804in}{1.754148in}}%
\pgfpathclose%
\pgfusepath{fill}%
\end{pgfscope}%
\begin{pgfscope}%
\pgfpathrectangle{\pgfqpoint{1.254980in}{0.150000in}}{\pgfqpoint{5.490039in}{5.490039in}}%
\pgfusepath{clip}%
\pgfsetbuttcap%
\pgfsetroundjoin%
\definecolor{currentfill}{rgb}{0.277134,0.185228,0.489898}%
\pgfsetfillcolor{currentfill}%
\pgfsetfillopacity{0.700000}%
\pgfsetlinewidth{0.000000pt}%
\definecolor{currentstroke}{rgb}{0.000000,0.000000,0.000000}%
\pgfsetstrokecolor{currentstroke}%
\pgfsetdash{}{0pt}%
\pgfpathmoveto{\pgfqpoint{3.120203in}{2.055671in}}%
\pgfpathlineto{\pgfqpoint{3.133396in}{2.048251in}}%
\pgfpathlineto{\pgfqpoint{3.146593in}{2.040864in}}%
\pgfpathlineto{\pgfqpoint{3.159793in}{2.033509in}}%
\pgfpathlineto{\pgfqpoint{3.172997in}{2.026185in}}%
\pgfpathlineto{\pgfqpoint{3.164745in}{2.029697in}}%
\pgfpathlineto{\pgfqpoint{3.156476in}{2.033575in}}%
\pgfpathlineto{\pgfqpoint{3.148190in}{2.037827in}}%
\pgfpathlineto{\pgfqpoint{3.139887in}{2.042462in}}%
\pgfpathlineto{\pgfqpoint{3.126650in}{2.050102in}}%
\pgfpathlineto{\pgfqpoint{3.113417in}{2.057774in}}%
\pgfpathlineto{\pgfqpoint{3.100187in}{2.065478in}}%
\pgfpathlineto{\pgfqpoint{3.086961in}{2.073214in}}%
\pgfpathlineto{\pgfqpoint{3.095298in}{2.068257in}}%
\pgfpathlineto{\pgfqpoint{3.103617in}{2.063686in}}%
\pgfpathlineto{\pgfqpoint{3.111918in}{2.059494in}}%
\pgfpathlineto{\pgfqpoint{3.120203in}{2.055671in}}%
\pgfpathclose%
\pgfusepath{fill}%
\end{pgfscope}%
\begin{pgfscope}%
\pgfpathrectangle{\pgfqpoint{1.254980in}{0.150000in}}{\pgfqpoint{5.490039in}{5.490039in}}%
\pgfusepath{clip}%
\pgfsetbuttcap%
\pgfsetroundjoin%
\definecolor{currentfill}{rgb}{0.278012,0.180367,0.486697}%
\pgfsetfillcolor{currentfill}%
\pgfsetfillopacity{0.700000}%
\pgfsetlinewidth{0.000000pt}%
\definecolor{currentstroke}{rgb}{0.000000,0.000000,0.000000}%
\pgfsetstrokecolor{currentstroke}%
\pgfsetdash{}{0pt}%
\pgfpathmoveto{\pgfqpoint{5.467219in}{2.064865in}}%
\pgfpathlineto{\pgfqpoint{5.480979in}{2.064323in}}%
\pgfpathlineto{\pgfqpoint{5.494748in}{2.063805in}}%
\pgfpathlineto{\pgfqpoint{5.508526in}{2.063310in}}%
\pgfpathlineto{\pgfqpoint{5.522313in}{2.062839in}}%
\pgfpathlineto{\pgfqpoint{5.515099in}{2.053653in}}%
\pgfpathlineto{\pgfqpoint{5.507877in}{2.044371in}}%
\pgfpathlineto{\pgfqpoint{5.500648in}{2.034994in}}%
\pgfpathlineto{\pgfqpoint{5.493412in}{2.025525in}}%
\pgfpathlineto{\pgfqpoint{5.479616in}{2.026063in}}%
\pgfpathlineto{\pgfqpoint{5.465829in}{2.026625in}}%
\pgfpathlineto{\pgfqpoint{5.452051in}{2.027210in}}%
\pgfpathlineto{\pgfqpoint{5.438282in}{2.027820in}}%
\pgfpathlineto{\pgfqpoint{5.445527in}{2.037217in}}%
\pgfpathlineto{\pgfqpoint{5.452765in}{2.046525in}}%
\pgfpathlineto{\pgfqpoint{5.459995in}{2.055741in}}%
\pgfpathlineto{\pgfqpoint{5.467219in}{2.064865in}}%
\pgfpathclose%
\pgfusepath{fill}%
\end{pgfscope}%
\begin{pgfscope}%
\pgfpathrectangle{\pgfqpoint{1.254980in}{0.150000in}}{\pgfqpoint{5.490039in}{5.490039in}}%
\pgfusepath{clip}%
\pgfsetbuttcap%
\pgfsetroundjoin%
\definecolor{currentfill}{rgb}{0.273809,0.031497,0.358853}%
\pgfsetfillcolor{currentfill}%
\pgfsetfillopacity{0.700000}%
\pgfsetlinewidth{0.000000pt}%
\definecolor{currentstroke}{rgb}{0.000000,0.000000,0.000000}%
\pgfsetstrokecolor{currentstroke}%
\pgfsetdash{}{0pt}%
\pgfpathmoveto{\pgfqpoint{4.602647in}{1.786854in}}%
\pgfpathlineto{\pgfqpoint{4.616132in}{1.784194in}}%
\pgfpathlineto{\pgfqpoint{4.629624in}{1.781558in}}%
\pgfpathlineto{\pgfqpoint{4.643123in}{1.778945in}}%
\pgfpathlineto{\pgfqpoint{4.656630in}{1.776356in}}%
\pgfpathlineto{\pgfqpoint{4.649116in}{1.767112in}}%
\pgfpathlineto{\pgfqpoint{4.641597in}{1.757907in}}%
\pgfpathlineto{\pgfqpoint{4.634074in}{1.748747in}}%
\pgfpathlineto{\pgfqpoint{4.626546in}{1.739637in}}%
\pgfpathlineto{\pgfqpoint{4.613031in}{1.742409in}}%
\pgfpathlineto{\pgfqpoint{4.599522in}{1.745205in}}%
\pgfpathlineto{\pgfqpoint{4.586020in}{1.748026in}}%
\pgfpathlineto{\pgfqpoint{4.572525in}{1.750869in}}%
\pgfpathlineto{\pgfqpoint{4.580063in}{1.759792in}}%
\pgfpathlineto{\pgfqpoint{4.587596in}{1.768766in}}%
\pgfpathlineto{\pgfqpoint{4.595124in}{1.777789in}}%
\pgfpathlineto{\pgfqpoint{4.602647in}{1.786854in}}%
\pgfpathclose%
\pgfusepath{fill}%
\end{pgfscope}%
\begin{pgfscope}%
\pgfpathrectangle{\pgfqpoint{1.254980in}{0.150000in}}{\pgfqpoint{5.490039in}{5.490039in}}%
\pgfusepath{clip}%
\pgfsetbuttcap%
\pgfsetroundjoin%
\definecolor{currentfill}{rgb}{0.280894,0.078907,0.402329}%
\pgfsetfillcolor{currentfill}%
\pgfsetfillopacity{0.700000}%
\pgfsetlinewidth{0.000000pt}%
\definecolor{currentstroke}{rgb}{0.000000,0.000000,0.000000}%
\pgfsetstrokecolor{currentstroke}%
\pgfsetdash{}{0pt}%
\pgfpathmoveto{\pgfqpoint{4.908759in}{1.865421in}}%
\pgfpathlineto{\pgfqpoint{4.922337in}{1.863616in}}%
\pgfpathlineto{\pgfqpoint{4.935923in}{1.861834in}}%
\pgfpathlineto{\pgfqpoint{4.949517in}{1.860076in}}%
\pgfpathlineto{\pgfqpoint{4.963118in}{1.858341in}}%
\pgfpathlineto{\pgfqpoint{4.955697in}{1.848365in}}%
\pgfpathlineto{\pgfqpoint{4.948270in}{1.838369in}}%
\pgfpathlineto{\pgfqpoint{4.940838in}{1.828358in}}%
\pgfpathlineto{\pgfqpoint{4.933401in}{1.818335in}}%
\pgfpathlineto{\pgfqpoint{4.919792in}{1.820215in}}%
\pgfpathlineto{\pgfqpoint{4.906190in}{1.822118in}}%
\pgfpathlineto{\pgfqpoint{4.892596in}{1.824046in}}%
\pgfpathlineto{\pgfqpoint{4.879010in}{1.825997in}}%
\pgfpathlineto{\pgfqpoint{4.886455in}{1.835869in}}%
\pgfpathlineto{\pgfqpoint{4.893895in}{1.845733in}}%
\pgfpathlineto{\pgfqpoint{4.901329in}{1.855585in}}%
\pgfpathlineto{\pgfqpoint{4.908759in}{1.865421in}}%
\pgfpathclose%
\pgfusepath{fill}%
\end{pgfscope}%
\begin{pgfscope}%
\pgfpathrectangle{\pgfqpoint{1.254980in}{0.150000in}}{\pgfqpoint{5.490039in}{5.490039in}}%
\pgfusepath{clip}%
\pgfsetbuttcap%
\pgfsetroundjoin%
\definecolor{currentfill}{rgb}{0.257322,0.256130,0.526563}%
\pgfsetfillcolor{currentfill}%
\pgfsetfillopacity{0.700000}%
\pgfsetlinewidth{0.000000pt}%
\definecolor{currentstroke}{rgb}{0.000000,0.000000,0.000000}%
\pgfsetstrokecolor{currentstroke}%
\pgfsetdash{}{0pt}%
\pgfpathmoveto{\pgfqpoint{2.875771in}{2.201532in}}%
\pgfpathlineto{\pgfqpoint{2.888948in}{2.193252in}}%
\pgfpathlineto{\pgfqpoint{2.902127in}{2.185008in}}%
\pgfpathlineto{\pgfqpoint{2.915310in}{2.176800in}}%
\pgfpathlineto{\pgfqpoint{2.928495in}{2.168626in}}%
\pgfpathlineto{\pgfqpoint{2.920034in}{2.174959in}}%
\pgfpathlineto{\pgfqpoint{2.911551in}{2.181707in}}%
\pgfpathlineto{\pgfqpoint{2.903048in}{2.188880in}}%
\pgfpathlineto{\pgfqpoint{2.894524in}{2.196488in}}%
\pgfpathlineto{\pgfqpoint{2.881301in}{2.204994in}}%
\pgfpathlineto{\pgfqpoint{2.868080in}{2.213536in}}%
\pgfpathlineto{\pgfqpoint{2.854862in}{2.222113in}}%
\pgfpathlineto{\pgfqpoint{2.841647in}{2.230727in}}%
\pgfpathlineto{\pgfqpoint{2.850210in}{2.222780in}}%
\pgfpathlineto{\pgfqpoint{2.858752in}{2.215271in}}%
\pgfpathlineto{\pgfqpoint{2.867272in}{2.208192in}}%
\pgfpathlineto{\pgfqpoint{2.875771in}{2.201532in}}%
\pgfpathclose%
\pgfusepath{fill}%
\end{pgfscope}%
\begin{pgfscope}%
\pgfpathrectangle{\pgfqpoint{1.254980in}{0.150000in}}{\pgfqpoint{5.490039in}{5.490039in}}%
\pgfusepath{clip}%
\pgfsetbuttcap%
\pgfsetroundjoin%
\definecolor{currentfill}{rgb}{0.280255,0.165693,0.476498}%
\pgfsetfillcolor{currentfill}%
\pgfsetfillopacity{0.700000}%
\pgfsetlinewidth{0.000000pt}%
\definecolor{currentstroke}{rgb}{0.000000,0.000000,0.000000}%
\pgfsetstrokecolor{currentstroke}%
\pgfsetdash{}{0pt}%
\pgfpathmoveto{\pgfqpoint{5.383292in}{2.030495in}}%
\pgfpathlineto{\pgfqpoint{5.397026in}{2.029790in}}%
\pgfpathlineto{\pgfqpoint{5.410769in}{2.029110in}}%
\pgfpathlineto{\pgfqpoint{5.424521in}{2.028453in}}%
\pgfpathlineto{\pgfqpoint{5.438282in}{2.027820in}}%
\pgfpathlineto{\pgfqpoint{5.431030in}{2.018334in}}%
\pgfpathlineto{\pgfqpoint{5.423770in}{2.008761in}}%
\pgfpathlineto{\pgfqpoint{5.416504in}{1.999102in}}%
\pgfpathlineto{\pgfqpoint{5.409230in}{1.989360in}}%
\pgfpathlineto{\pgfqpoint{5.395461in}{1.990073in}}%
\pgfpathlineto{\pgfqpoint{5.381701in}{1.990810in}}%
\pgfpathlineto{\pgfqpoint{5.367949in}{1.991571in}}%
\pgfpathlineto{\pgfqpoint{5.354206in}{1.992356in}}%
\pgfpathlineto{\pgfqpoint{5.361488in}{2.002013in}}%
\pgfpathlineto{\pgfqpoint{5.368763in}{2.011590in}}%
\pgfpathlineto{\pgfqpoint{5.376031in}{2.021084in}}%
\pgfpathlineto{\pgfqpoint{5.383292in}{2.030495in}}%
\pgfpathclose%
\pgfusepath{fill}%
\end{pgfscope}%
\begin{pgfscope}%
\pgfpathrectangle{\pgfqpoint{1.254980in}{0.150000in}}{\pgfqpoint{5.490039in}{5.490039in}}%
\pgfusepath{clip}%
\pgfsetbuttcap%
\pgfsetroundjoin%
\definecolor{currentfill}{rgb}{0.269944,0.014625,0.341379}%
\pgfsetfillcolor{currentfill}%
\pgfsetfillopacity{0.700000}%
\pgfsetlinewidth{0.000000pt}%
\definecolor{currentstroke}{rgb}{0.000000,0.000000,0.000000}%
\pgfsetstrokecolor{currentstroke}%
\pgfsetdash{}{0pt}%
\pgfpathmoveto{\pgfqpoint{4.021403in}{1.750578in}}%
\pgfpathlineto{\pgfqpoint{4.034743in}{1.746094in}}%
\pgfpathlineto{\pgfqpoint{4.048089in}{1.741634in}}%
\pgfpathlineto{\pgfqpoint{4.061440in}{1.737199in}}%
\pgfpathlineto{\pgfqpoint{4.074798in}{1.732790in}}%
\pgfpathlineto{\pgfqpoint{4.067081in}{1.727231in}}%
\pgfpathlineto{\pgfqpoint{4.059358in}{1.721843in}}%
\pgfpathlineto{\pgfqpoint{4.051628in}{1.716633in}}%
\pgfpathlineto{\pgfqpoint{4.043892in}{1.711608in}}%
\pgfpathlineto{\pgfqpoint{4.030518in}{1.716264in}}%
\pgfpathlineto{\pgfqpoint{4.017150in}{1.720946in}}%
\pgfpathlineto{\pgfqpoint{4.003788in}{1.725653in}}%
\pgfpathlineto{\pgfqpoint{3.990431in}{1.730385in}}%
\pgfpathlineto{\pgfqpoint{3.998185in}{1.735158in}}%
\pgfpathlineto{\pgfqpoint{4.005931in}{1.740119in}}%
\pgfpathlineto{\pgfqpoint{4.013670in}{1.745262in}}%
\pgfpathlineto{\pgfqpoint{4.021403in}{1.750578in}}%
\pgfpathclose%
\pgfusepath{fill}%
\end{pgfscope}%
\begin{pgfscope}%
\pgfpathrectangle{\pgfqpoint{1.254980in}{0.150000in}}{\pgfqpoint{5.490039in}{5.490039in}}%
\pgfusepath{clip}%
\pgfsetbuttcap%
\pgfsetroundjoin%
\definecolor{currentfill}{rgb}{0.281446,0.084320,0.407414}%
\pgfsetfillcolor{currentfill}%
\pgfsetfillopacity{0.700000}%
\pgfsetlinewidth{0.000000pt}%
\definecolor{currentstroke}{rgb}{0.000000,0.000000,0.000000}%
\pgfsetstrokecolor{currentstroke}%
\pgfsetdash{}{0pt}%
\pgfpathmoveto{\pgfqpoint{3.555204in}{1.857653in}}%
\pgfpathlineto{\pgfqpoint{3.568457in}{1.851654in}}%
\pgfpathlineto{\pgfqpoint{3.581714in}{1.845681in}}%
\pgfpathlineto{\pgfqpoint{3.594976in}{1.839737in}}%
\pgfpathlineto{\pgfqpoint{3.608243in}{1.833819in}}%
\pgfpathlineto{\pgfqpoint{3.600290in}{1.832726in}}%
\pgfpathlineto{\pgfqpoint{3.592326in}{1.831910in}}%
\pgfpathlineto{\pgfqpoint{3.584350in}{1.831378in}}%
\pgfpathlineto{\pgfqpoint{3.576364in}{1.831139in}}%
\pgfpathlineto{\pgfqpoint{3.563072in}{1.837343in}}%
\pgfpathlineto{\pgfqpoint{3.549786in}{1.843575in}}%
\pgfpathlineto{\pgfqpoint{3.536504in}{1.849834in}}%
\pgfpathlineto{\pgfqpoint{3.523226in}{1.856121in}}%
\pgfpathlineto{\pgfqpoint{3.531238in}{1.856068in}}%
\pgfpathlineto{\pgfqpoint{3.539238in}{1.856311in}}%
\pgfpathlineto{\pgfqpoint{3.547227in}{1.856842in}}%
\pgfpathlineto{\pgfqpoint{3.555204in}{1.857653in}}%
\pgfpathclose%
\pgfusepath{fill}%
\end{pgfscope}%
\begin{pgfscope}%
\pgfpathrectangle{\pgfqpoint{1.254980in}{0.150000in}}{\pgfqpoint{5.490039in}{5.490039in}}%
\pgfusepath{clip}%
\pgfsetbuttcap%
\pgfsetroundjoin%
\definecolor{currentfill}{rgb}{0.283187,0.125848,0.444960}%
\pgfsetfillcolor{currentfill}%
\pgfsetfillopacity{0.700000}%
\pgfsetlinewidth{0.000000pt}%
\definecolor{currentstroke}{rgb}{0.000000,0.000000,0.000000}%
\pgfsetstrokecolor{currentstroke}%
\pgfsetdash{}{0pt}%
\pgfpathmoveto{\pgfqpoint{3.364239in}{1.933760in}}%
\pgfpathlineto{\pgfqpoint{3.377464in}{1.927132in}}%
\pgfpathlineto{\pgfqpoint{3.390694in}{1.920533in}}%
\pgfpathlineto{\pgfqpoint{3.403927in}{1.913964in}}%
\pgfpathlineto{\pgfqpoint{3.417166in}{1.907423in}}%
\pgfpathlineto{\pgfqpoint{3.409089in}{1.908371in}}%
\pgfpathlineto{\pgfqpoint{3.400999in}{1.909637in}}%
\pgfpathlineto{\pgfqpoint{3.392896in}{1.911230in}}%
\pgfpathlineto{\pgfqpoint{3.384779in}{1.913157in}}%
\pgfpathlineto{\pgfqpoint{3.371513in}{1.919999in}}%
\pgfpathlineto{\pgfqpoint{3.358251in}{1.926870in}}%
\pgfpathlineto{\pgfqpoint{3.344993in}{1.933770in}}%
\pgfpathlineto{\pgfqpoint{3.331739in}{1.940699in}}%
\pgfpathlineto{\pgfqpoint{3.339885in}{1.938465in}}%
\pgfpathlineto{\pgfqpoint{3.348017in}{1.936569in}}%
\pgfpathlineto{\pgfqpoint{3.356135in}{1.935004in}}%
\pgfpathlineto{\pgfqpoint{3.364239in}{1.933760in}}%
\pgfpathclose%
\pgfusepath{fill}%
\end{pgfscope}%
\begin{pgfscope}%
\pgfpathrectangle{\pgfqpoint{1.254980in}{0.150000in}}{\pgfqpoint{5.490039in}{5.490039in}}%
\pgfusepath{clip}%
\pgfsetbuttcap%
\pgfsetroundjoin%
\definecolor{currentfill}{rgb}{0.272594,0.025563,0.353093}%
\pgfsetfillcolor{currentfill}%
\pgfsetfillopacity{0.700000}%
\pgfsetlinewidth{0.000000pt}%
\definecolor{currentstroke}{rgb}{0.000000,0.000000,0.000000}%
\pgfsetstrokecolor{currentstroke}%
\pgfsetdash{}{0pt}%
\pgfpathmoveto{\pgfqpoint{3.883779in}{1.769150in}}%
\pgfpathlineto{\pgfqpoint{3.897092in}{1.764215in}}%
\pgfpathlineto{\pgfqpoint{3.910409in}{1.759306in}}%
\pgfpathlineto{\pgfqpoint{3.923732in}{1.754422in}}%
\pgfpathlineto{\pgfqpoint{3.937061in}{1.749564in}}%
\pgfpathlineto{\pgfqpoint{3.929283in}{1.745239in}}%
\pgfpathlineto{\pgfqpoint{3.921497in}{1.741118in}}%
\pgfpathlineto{\pgfqpoint{3.913703in}{1.737208in}}%
\pgfpathlineto{\pgfqpoint{3.905902in}{1.733515in}}%
\pgfpathlineto{\pgfqpoint{3.892555in}{1.738633in}}%
\pgfpathlineto{\pgfqpoint{3.879213in}{1.743777in}}%
\pgfpathlineto{\pgfqpoint{3.865877in}{1.748946in}}%
\pgfpathlineto{\pgfqpoint{3.852546in}{1.754141in}}%
\pgfpathlineto{\pgfqpoint{3.860366in}{1.757569in}}%
\pgfpathlineto{\pgfqpoint{3.868179in}{1.761218in}}%
\pgfpathlineto{\pgfqpoint{3.875983in}{1.765080in}}%
\pgfpathlineto{\pgfqpoint{3.883779in}{1.769150in}}%
\pgfpathclose%
\pgfusepath{fill}%
\end{pgfscope}%
\begin{pgfscope}%
\pgfpathrectangle{\pgfqpoint{1.254980in}{0.150000in}}{\pgfqpoint{5.490039in}{5.490039in}}%
\pgfusepath{clip}%
\pgfsetbuttcap%
\pgfsetroundjoin%
\definecolor{currentfill}{rgb}{0.268510,0.009605,0.335427}%
\pgfsetfillcolor{currentfill}%
\pgfsetfillopacity{0.700000}%
\pgfsetlinewidth{0.000000pt}%
\definecolor{currentstroke}{rgb}{0.000000,0.000000,0.000000}%
\pgfsetstrokecolor{currentstroke}%
\pgfsetdash{}{0pt}%
\pgfpathmoveto{\pgfqpoint{4.159029in}{1.740165in}}%
\pgfpathlineto{\pgfqpoint{4.172402in}{1.736114in}}%
\pgfpathlineto{\pgfqpoint{4.185780in}{1.732087in}}%
\pgfpathlineto{\pgfqpoint{4.199165in}{1.728085in}}%
\pgfpathlineto{\pgfqpoint{4.212556in}{1.724107in}}%
\pgfpathlineto{\pgfqpoint{4.204893in}{1.717464in}}%
\pgfpathlineto{\pgfqpoint{4.197225in}{1.710962in}}%
\pgfpathlineto{\pgfqpoint{4.189551in}{1.704607in}}%
\pgfpathlineto{\pgfqpoint{4.181871in}{1.698404in}}%
\pgfpathlineto{\pgfqpoint{4.168466in}{1.702616in}}%
\pgfpathlineto{\pgfqpoint{4.155067in}{1.706853in}}%
\pgfpathlineto{\pgfqpoint{4.141674in}{1.711114in}}%
\pgfpathlineto{\pgfqpoint{4.128287in}{1.715400in}}%
\pgfpathlineto{\pgfqpoint{4.135981in}{1.721363in}}%
\pgfpathlineto{\pgfqpoint{4.143670in}{1.727482in}}%
\pgfpathlineto{\pgfqpoint{4.151353in}{1.733752in}}%
\pgfpathlineto{\pgfqpoint{4.159029in}{1.740165in}}%
\pgfpathclose%
\pgfusepath{fill}%
\end{pgfscope}%
\begin{pgfscope}%
\pgfpathrectangle{\pgfqpoint{1.254980in}{0.150000in}}{\pgfqpoint{5.490039in}{5.490039in}}%
\pgfusepath{clip}%
\pgfsetbuttcap%
\pgfsetroundjoin%
\definecolor{currentfill}{rgb}{0.281887,0.150881,0.465405}%
\pgfsetfillcolor{currentfill}%
\pgfsetfillopacity{0.700000}%
\pgfsetlinewidth{0.000000pt}%
\definecolor{currentstroke}{rgb}{0.000000,0.000000,0.000000}%
\pgfsetstrokecolor{currentstroke}%
\pgfsetdash{}{0pt}%
\pgfpathmoveto{\pgfqpoint{5.299320in}{1.995732in}}%
\pgfpathlineto{\pgfqpoint{5.313029in}{1.994852in}}%
\pgfpathlineto{\pgfqpoint{5.326746in}{1.993996in}}%
\pgfpathlineto{\pgfqpoint{5.340472in}{1.993164in}}%
\pgfpathlineto{\pgfqpoint{5.354206in}{1.992356in}}%
\pgfpathlineto{\pgfqpoint{5.346918in}{1.982620in}}%
\pgfpathlineto{\pgfqpoint{5.339623in}{1.972808in}}%
\pgfpathlineto{\pgfqpoint{5.332321in}{1.962921in}}%
\pgfpathlineto{\pgfqpoint{5.325013in}{1.952960in}}%
\pgfpathlineto{\pgfqpoint{5.311271in}{1.953862in}}%
\pgfpathlineto{\pgfqpoint{5.297537in}{1.954788in}}%
\pgfpathlineto{\pgfqpoint{5.283811in}{1.955737in}}%
\pgfpathlineto{\pgfqpoint{5.270094in}{1.956710in}}%
\pgfpathlineto{\pgfqpoint{5.277410in}{1.966572in}}%
\pgfpathlineto{\pgfqpoint{5.284720in}{1.976364in}}%
\pgfpathlineto{\pgfqpoint{5.292023in}{1.986085in}}%
\pgfpathlineto{\pgfqpoint{5.299320in}{1.995732in}}%
\pgfpathclose%
\pgfusepath{fill}%
\end{pgfscope}%
\begin{pgfscope}%
\pgfpathrectangle{\pgfqpoint{1.254980in}{0.150000in}}{\pgfqpoint{5.490039in}{5.490039in}}%
\pgfusepath{clip}%
\pgfsetbuttcap%
\pgfsetroundjoin%
\definecolor{currentfill}{rgb}{0.278791,0.062145,0.386592}%
\pgfsetfillcolor{currentfill}%
\pgfsetfillopacity{0.700000}%
\pgfsetlinewidth{0.000000pt}%
\definecolor{currentstroke}{rgb}{0.000000,0.000000,0.000000}%
\pgfsetstrokecolor{currentstroke}%
\pgfsetdash{}{0pt}%
\pgfpathmoveto{\pgfqpoint{4.824742in}{1.834037in}}%
\pgfpathlineto{\pgfqpoint{4.838297in}{1.831992in}}%
\pgfpathlineto{\pgfqpoint{4.851861in}{1.829970in}}%
\pgfpathlineto{\pgfqpoint{4.865432in}{1.827971in}}%
\pgfpathlineto{\pgfqpoint{4.879010in}{1.825997in}}%
\pgfpathlineto{\pgfqpoint{4.871560in}{1.816119in}}%
\pgfpathlineto{\pgfqpoint{4.864106in}{1.806239in}}%
\pgfpathlineto{\pgfqpoint{4.856646in}{1.796362in}}%
\pgfpathlineto{\pgfqpoint{4.849182in}{1.786490in}}%
\pgfpathlineto{\pgfqpoint{4.835595in}{1.788623in}}%
\pgfpathlineto{\pgfqpoint{4.822016in}{1.790780in}}%
\pgfpathlineto{\pgfqpoint{4.808445in}{1.792961in}}%
\pgfpathlineto{\pgfqpoint{4.794881in}{1.795165in}}%
\pgfpathlineto{\pgfqpoint{4.802353in}{1.804873in}}%
\pgfpathlineto{\pgfqpoint{4.809821in}{1.814590in}}%
\pgfpathlineto{\pgfqpoint{4.817284in}{1.824313in}}%
\pgfpathlineto{\pgfqpoint{4.824742in}{1.834037in}}%
\pgfpathclose%
\pgfusepath{fill}%
\end{pgfscope}%
\begin{pgfscope}%
\pgfpathrectangle{\pgfqpoint{1.254980in}{0.150000in}}{\pgfqpoint{5.490039in}{5.490039in}}%
\pgfusepath{clip}%
\pgfsetbuttcap%
\pgfsetroundjoin%
\definecolor{currentfill}{rgb}{0.216210,0.351535,0.550627}%
\pgfsetfillcolor{currentfill}%
\pgfsetfillopacity{0.700000}%
\pgfsetlinewidth{0.000000pt}%
\definecolor{currentstroke}{rgb}{0.000000,0.000000,0.000000}%
\pgfsetstrokecolor{currentstroke}%
\pgfsetdash{}{0pt}%
\pgfpathmoveto{\pgfqpoint{2.577841in}{2.411011in}}%
\pgfpathlineto{\pgfqpoint{2.591011in}{2.401613in}}%
\pgfpathlineto{\pgfqpoint{2.604184in}{2.392258in}}%
\pgfpathlineto{\pgfqpoint{2.617358in}{2.382945in}}%
\pgfpathlineto{\pgfqpoint{2.630534in}{2.373673in}}%
\pgfpathlineto{\pgfqpoint{2.621782in}{2.383445in}}%
\pgfpathlineto{\pgfqpoint{2.613005in}{2.393690in}}%
\pgfpathlineto{\pgfqpoint{2.604201in}{2.404418in}}%
\pgfpathlineto{\pgfqpoint{2.595371in}{2.415640in}}%
\pgfpathlineto{\pgfqpoint{2.582151in}{2.425263in}}%
\pgfpathlineto{\pgfqpoint{2.568932in}{2.434928in}}%
\pgfpathlineto{\pgfqpoint{2.555716in}{2.444636in}}%
\pgfpathlineto{\pgfqpoint{2.542500in}{2.454386in}}%
\pgfpathlineto{\pgfqpoint{2.551376in}{2.442807in}}%
\pgfpathlineto{\pgfqpoint{2.560225in}{2.431724in}}%
\pgfpathlineto{\pgfqpoint{2.569046in}{2.421129in}}%
\pgfpathlineto{\pgfqpoint{2.577841in}{2.411011in}}%
\pgfpathclose%
\pgfusepath{fill}%
\end{pgfscope}%
\begin{pgfscope}%
\pgfpathrectangle{\pgfqpoint{1.254980in}{0.150000in}}{\pgfqpoint{5.490039in}{5.490039in}}%
\pgfusepath{clip}%
\pgfsetbuttcap%
\pgfsetroundjoin%
\definecolor{currentfill}{rgb}{0.271305,0.019942,0.347269}%
\pgfsetfillcolor{currentfill}%
\pgfsetfillopacity{0.700000}%
\pgfsetlinewidth{0.000000pt}%
\definecolor{currentstroke}{rgb}{0.000000,0.000000,0.000000}%
\pgfsetstrokecolor{currentstroke}%
\pgfsetdash{}{0pt}%
\pgfpathmoveto{\pgfqpoint{4.518617in}{1.762484in}}%
\pgfpathlineto{\pgfqpoint{4.532083in}{1.759545in}}%
\pgfpathlineto{\pgfqpoint{4.545557in}{1.756629in}}%
\pgfpathlineto{\pgfqpoint{4.559038in}{1.753737in}}%
\pgfpathlineto{\pgfqpoint{4.572525in}{1.750869in}}%
\pgfpathlineto{\pgfqpoint{4.564983in}{1.742004in}}%
\pgfpathlineto{\pgfqpoint{4.557436in}{1.733200in}}%
\pgfpathlineto{\pgfqpoint{4.549884in}{1.724463in}}%
\pgfpathlineto{\pgfqpoint{4.542328in}{1.715796in}}%
\pgfpathlineto{\pgfqpoint{4.528830in}{1.718860in}}%
\pgfpathlineto{\pgfqpoint{4.515339in}{1.721948in}}%
\pgfpathlineto{\pgfqpoint{4.501855in}{1.725060in}}%
\pgfpathlineto{\pgfqpoint{4.488378in}{1.728196in}}%
\pgfpathlineto{\pgfqpoint{4.495945in}{1.736662in}}%
\pgfpathlineto{\pgfqpoint{4.503507in}{1.745201in}}%
\pgfpathlineto{\pgfqpoint{4.511064in}{1.753810in}}%
\pgfpathlineto{\pgfqpoint{4.518617in}{1.762484in}}%
\pgfpathclose%
\pgfusepath{fill}%
\end{pgfscope}%
\begin{pgfscope}%
\pgfpathrectangle{\pgfqpoint{1.254980in}{0.150000in}}{\pgfqpoint{5.490039in}{5.490039in}}%
\pgfusepath{clip}%
\pgfsetbuttcap%
\pgfsetroundjoin%
\definecolor{currentfill}{rgb}{0.277018,0.050344,0.375715}%
\pgfsetfillcolor{currentfill}%
\pgfsetfillopacity{0.700000}%
\pgfsetlinewidth{0.000000pt}%
\definecolor{currentstroke}{rgb}{0.000000,0.000000,0.000000}%
\pgfsetstrokecolor{currentstroke}%
\pgfsetdash{}{0pt}%
\pgfpathmoveto{\pgfqpoint{3.746086in}{1.796632in}}%
\pgfpathlineto{\pgfqpoint{3.759376in}{1.791229in}}%
\pgfpathlineto{\pgfqpoint{3.772670in}{1.785853in}}%
\pgfpathlineto{\pgfqpoint{3.785970in}{1.780503in}}%
\pgfpathlineto{\pgfqpoint{3.799274in}{1.775178in}}%
\pgfpathlineto{\pgfqpoint{3.791425in}{1.772246in}}%
\pgfpathlineto{\pgfqpoint{3.783567in}{1.769551in}}%
\pgfpathlineto{\pgfqpoint{3.775701in}{1.767101in}}%
\pgfpathlineto{\pgfqpoint{3.767825in}{1.764903in}}%
\pgfpathlineto{\pgfqpoint{3.754499in}{1.770501in}}%
\pgfpathlineto{\pgfqpoint{3.741178in}{1.776125in}}%
\pgfpathlineto{\pgfqpoint{3.727863in}{1.781775in}}%
\pgfpathlineto{\pgfqpoint{3.714552in}{1.787451in}}%
\pgfpathlineto{\pgfqpoint{3.722450in}{1.789370in}}%
\pgfpathlineto{\pgfqpoint{3.730338in}{1.791545in}}%
\pgfpathlineto{\pgfqpoint{3.738217in}{1.793968in}}%
\pgfpathlineto{\pgfqpoint{3.746086in}{1.796632in}}%
\pgfpathclose%
\pgfusepath{fill}%
\end{pgfscope}%
\begin{pgfscope}%
\pgfpathrectangle{\pgfqpoint{1.254980in}{0.150000in}}{\pgfqpoint{5.490039in}{5.490039in}}%
\pgfusepath{clip}%
\pgfsetbuttcap%
\pgfsetroundjoin%
\definecolor{currentfill}{rgb}{0.268510,0.009605,0.335427}%
\pgfsetfillcolor{currentfill}%
\pgfsetfillopacity{0.700000}%
\pgfsetlinewidth{0.000000pt}%
\definecolor{currentstroke}{rgb}{0.000000,0.000000,0.000000}%
\pgfsetstrokecolor{currentstroke}%
\pgfsetdash{}{0pt}%
\pgfpathmoveto{\pgfqpoint{4.296722in}{1.737194in}}%
\pgfpathlineto{\pgfqpoint{4.310131in}{1.733560in}}%
\pgfpathlineto{\pgfqpoint{4.323547in}{1.729950in}}%
\pgfpathlineto{\pgfqpoint{4.336969in}{1.726364in}}%
\pgfpathlineto{\pgfqpoint{4.350397in}{1.722803in}}%
\pgfpathlineto{\pgfqpoint{4.342783in}{1.715220in}}%
\pgfpathlineto{\pgfqpoint{4.335163in}{1.707749in}}%
\pgfpathlineto{\pgfqpoint{4.327538in}{1.700395in}}%
\pgfpathlineto{\pgfqpoint{4.319908in}{1.693164in}}%
\pgfpathlineto{\pgfqpoint{4.306467in}{1.696947in}}%
\pgfpathlineto{\pgfqpoint{4.293032in}{1.700754in}}%
\pgfpathlineto{\pgfqpoint{4.279604in}{1.704585in}}%
\pgfpathlineto{\pgfqpoint{4.266181in}{1.708441in}}%
\pgfpathlineto{\pgfqpoint{4.273825in}{1.715446in}}%
\pgfpathlineto{\pgfqpoint{4.281463in}{1.722576in}}%
\pgfpathlineto{\pgfqpoint{4.289095in}{1.729827in}}%
\pgfpathlineto{\pgfqpoint{4.296722in}{1.737194in}}%
\pgfpathclose%
\pgfusepath{fill}%
\end{pgfscope}%
\begin{pgfscope}%
\pgfpathrectangle{\pgfqpoint{1.254980in}{0.150000in}}{\pgfqpoint{5.490039in}{5.490039in}}%
\pgfusepath{clip}%
\pgfsetbuttcap%
\pgfsetroundjoin%
\definecolor{currentfill}{rgb}{0.282884,0.135920,0.453427}%
\pgfsetfillcolor{currentfill}%
\pgfsetfillopacity{0.700000}%
\pgfsetlinewidth{0.000000pt}%
\definecolor{currentstroke}{rgb}{0.000000,0.000000,0.000000}%
\pgfsetstrokecolor{currentstroke}%
\pgfsetdash{}{0pt}%
\pgfpathmoveto{\pgfqpoint{5.215311in}{1.960839in}}%
\pgfpathlineto{\pgfqpoint{5.228994in}{1.959771in}}%
\pgfpathlineto{\pgfqpoint{5.242686in}{1.958727in}}%
\pgfpathlineto{\pgfqpoint{5.256386in}{1.957707in}}%
\pgfpathlineto{\pgfqpoint{5.270094in}{1.956710in}}%
\pgfpathlineto{\pgfqpoint{5.262772in}{1.946781in}}%
\pgfpathlineto{\pgfqpoint{5.255443in}{1.936787in}}%
\pgfpathlineto{\pgfqpoint{5.248108in}{1.926730in}}%
\pgfpathlineto{\pgfqpoint{5.240767in}{1.916612in}}%
\pgfpathlineto{\pgfqpoint{5.227051in}{1.917715in}}%
\pgfpathlineto{\pgfqpoint{5.213343in}{1.918842in}}%
\pgfpathlineto{\pgfqpoint{5.199644in}{1.919993in}}%
\pgfpathlineto{\pgfqpoint{5.185953in}{1.921168in}}%
\pgfpathlineto{\pgfqpoint{5.193301in}{1.931174in}}%
\pgfpathlineto{\pgfqpoint{5.200644in}{1.941123in}}%
\pgfpathlineto{\pgfqpoint{5.207980in}{1.951012in}}%
\pgfpathlineto{\pgfqpoint{5.215311in}{1.960839in}}%
\pgfpathclose%
\pgfusepath{fill}%
\end{pgfscope}%
\begin{pgfscope}%
\pgfpathrectangle{\pgfqpoint{1.254980in}{0.150000in}}{\pgfqpoint{5.490039in}{5.490039in}}%
\pgfusepath{clip}%
\pgfsetbuttcap%
\pgfsetroundjoin%
\definecolor{currentfill}{rgb}{0.278826,0.175490,0.483397}%
\pgfsetfillcolor{currentfill}%
\pgfsetfillopacity{0.700000}%
\pgfsetlinewidth{0.000000pt}%
\definecolor{currentstroke}{rgb}{0.000000,0.000000,0.000000}%
\pgfsetstrokecolor{currentstroke}%
\pgfsetdash{}{0pt}%
\pgfpathmoveto{\pgfqpoint{3.172997in}{2.026185in}}%
\pgfpathlineto{\pgfqpoint{3.186205in}{2.018893in}}%
\pgfpathlineto{\pgfqpoint{3.199417in}{2.011632in}}%
\pgfpathlineto{\pgfqpoint{3.212632in}{2.004401in}}%
\pgfpathlineto{\pgfqpoint{3.225850in}{1.997202in}}%
\pgfpathlineto{\pgfqpoint{3.217629in}{2.000403in}}%
\pgfpathlineto{\pgfqpoint{3.209392in}{2.003967in}}%
\pgfpathlineto{\pgfqpoint{3.201139in}{2.007900in}}%
\pgfpathlineto{\pgfqpoint{3.192869in}{2.012214in}}%
\pgfpathlineto{\pgfqpoint{3.179618in}{2.019729in}}%
\pgfpathlineto{\pgfqpoint{3.166371in}{2.027275in}}%
\pgfpathlineto{\pgfqpoint{3.153127in}{2.034853in}}%
\pgfpathlineto{\pgfqpoint{3.139887in}{2.042462in}}%
\pgfpathlineto{\pgfqpoint{3.148190in}{2.037827in}}%
\pgfpathlineto{\pgfqpoint{3.156476in}{2.033575in}}%
\pgfpathlineto{\pgfqpoint{3.164745in}{2.029697in}}%
\pgfpathlineto{\pgfqpoint{3.172997in}{2.026185in}}%
\pgfpathclose%
\pgfusepath{fill}%
\end{pgfscope}%
\begin{pgfscope}%
\pgfpathrectangle{\pgfqpoint{1.254980in}{0.150000in}}{\pgfqpoint{5.490039in}{5.490039in}}%
\pgfusepath{clip}%
\pgfsetbuttcap%
\pgfsetroundjoin%
\definecolor{currentfill}{rgb}{0.277018,0.050344,0.375715}%
\pgfsetfillcolor{currentfill}%
\pgfsetfillopacity{0.700000}%
\pgfsetlinewidth{0.000000pt}%
\definecolor{currentstroke}{rgb}{0.000000,0.000000,0.000000}%
\pgfsetstrokecolor{currentstroke}%
\pgfsetdash{}{0pt}%
\pgfpathmoveto{\pgfqpoint{4.740700in}{1.804218in}}%
\pgfpathlineto{\pgfqpoint{4.754234in}{1.801919in}}%
\pgfpathlineto{\pgfqpoint{4.767776in}{1.799644in}}%
\pgfpathlineto{\pgfqpoint{4.781325in}{1.797392in}}%
\pgfpathlineto{\pgfqpoint{4.794881in}{1.795165in}}%
\pgfpathlineto{\pgfqpoint{4.787404in}{1.785469in}}%
\pgfpathlineto{\pgfqpoint{4.779922in}{1.775791in}}%
\pgfpathlineto{\pgfqpoint{4.772435in}{1.766133in}}%
\pgfpathlineto{\pgfqpoint{4.764943in}{1.756501in}}%
\pgfpathlineto{\pgfqpoint{4.751378in}{1.758900in}}%
\pgfpathlineto{\pgfqpoint{4.737821in}{1.761323in}}%
\pgfpathlineto{\pgfqpoint{4.724271in}{1.763769in}}%
\pgfpathlineto{\pgfqpoint{4.710728in}{1.766239in}}%
\pgfpathlineto{\pgfqpoint{4.718228in}{1.775695in}}%
\pgfpathlineto{\pgfqpoint{4.725724in}{1.785180in}}%
\pgfpathlineto{\pgfqpoint{4.733214in}{1.794689in}}%
\pgfpathlineto{\pgfqpoint{4.740700in}{1.804218in}}%
\pgfpathclose%
\pgfusepath{fill}%
\end{pgfscope}%
\begin{pgfscope}%
\pgfpathrectangle{\pgfqpoint{1.254980in}{0.150000in}}{\pgfqpoint{5.490039in}{5.490039in}}%
\pgfusepath{clip}%
\pgfsetbuttcap%
\pgfsetroundjoin%
\definecolor{currentfill}{rgb}{0.260571,0.246922,0.522828}%
\pgfsetfillcolor{currentfill}%
\pgfsetfillopacity{0.700000}%
\pgfsetlinewidth{0.000000pt}%
\definecolor{currentstroke}{rgb}{0.000000,0.000000,0.000000}%
\pgfsetstrokecolor{currentstroke}%
\pgfsetdash{}{0pt}%
\pgfpathmoveto{\pgfqpoint{2.928495in}{2.168626in}}%
\pgfpathlineto{\pgfqpoint{2.941684in}{2.160488in}}%
\pgfpathlineto{\pgfqpoint{2.954875in}{2.152385in}}%
\pgfpathlineto{\pgfqpoint{2.968070in}{2.144316in}}%
\pgfpathlineto{\pgfqpoint{2.981267in}{2.136282in}}%
\pgfpathlineto{\pgfqpoint{2.972842in}{2.142287in}}%
\pgfpathlineto{\pgfqpoint{2.964396in}{2.148705in}}%
\pgfpathlineto{\pgfqpoint{2.955931in}{2.155544in}}%
\pgfpathlineto{\pgfqpoint{2.947445in}{2.162813in}}%
\pgfpathlineto{\pgfqpoint{2.934210in}{2.171180in}}%
\pgfpathlineto{\pgfqpoint{2.920979in}{2.179581in}}%
\pgfpathlineto{\pgfqpoint{2.907750in}{2.188017in}}%
\pgfpathlineto{\pgfqpoint{2.894524in}{2.196488in}}%
\pgfpathlineto{\pgfqpoint{2.903048in}{2.188880in}}%
\pgfpathlineto{\pgfqpoint{2.911551in}{2.181707in}}%
\pgfpathlineto{\pgfqpoint{2.920034in}{2.174959in}}%
\pgfpathlineto{\pgfqpoint{2.928495in}{2.168626in}}%
\pgfpathclose%
\pgfusepath{fill}%
\end{pgfscope}%
\begin{pgfscope}%
\pgfpathrectangle{\pgfqpoint{1.254980in}{0.150000in}}{\pgfqpoint{5.490039in}{5.490039in}}%
\pgfusepath{clip}%
\pgfsetbuttcap%
\pgfsetroundjoin%
\definecolor{currentfill}{rgb}{0.283229,0.120777,0.440584}%
\pgfsetfillcolor{currentfill}%
\pgfsetfillopacity{0.700000}%
\pgfsetlinewidth{0.000000pt}%
\definecolor{currentstroke}{rgb}{0.000000,0.000000,0.000000}%
\pgfsetstrokecolor{currentstroke}%
\pgfsetdash{}{0pt}%
\pgfpathmoveto{\pgfqpoint{5.131271in}{1.926102in}}%
\pgfpathlineto{\pgfqpoint{5.144929in}{1.924833in}}%
\pgfpathlineto{\pgfqpoint{5.158595in}{1.923587in}}%
\pgfpathlineto{\pgfqpoint{5.172270in}{1.922366in}}%
\pgfpathlineto{\pgfqpoint{5.185953in}{1.921168in}}%
\pgfpathlineto{\pgfqpoint{5.178598in}{1.911107in}}%
\pgfpathlineto{\pgfqpoint{5.171238in}{1.900993in}}%
\pgfpathlineto{\pgfqpoint{5.163872in}{1.890831in}}%
\pgfpathlineto{\pgfqpoint{5.156500in}{1.880621in}}%
\pgfpathlineto{\pgfqpoint{5.142810in}{1.881939in}}%
\pgfpathlineto{\pgfqpoint{5.129128in}{1.883281in}}%
\pgfpathlineto{\pgfqpoint{5.115454in}{1.884646in}}%
\pgfpathlineto{\pgfqpoint{5.101788in}{1.886035in}}%
\pgfpathlineto{\pgfqpoint{5.109167in}{1.896120in}}%
\pgfpathlineto{\pgfqpoint{5.116541in}{1.906161in}}%
\pgfpathlineto{\pgfqpoint{5.123909in}{1.916156in}}%
\pgfpathlineto{\pgfqpoint{5.131271in}{1.926102in}}%
\pgfpathclose%
\pgfusepath{fill}%
\end{pgfscope}%
\begin{pgfscope}%
\pgfpathrectangle{\pgfqpoint{1.254980in}{0.150000in}}{\pgfqpoint{5.490039in}{5.490039in}}%
\pgfusepath{clip}%
\pgfsetbuttcap%
\pgfsetroundjoin%
\definecolor{currentfill}{rgb}{0.273006,0.204520,0.501721}%
\pgfsetfillcolor{currentfill}%
\pgfsetfillopacity{0.700000}%
\pgfsetlinewidth{0.000000pt}%
\definecolor{currentstroke}{rgb}{0.000000,0.000000,0.000000}%
\pgfsetstrokecolor{currentstroke}%
\pgfsetdash{}{0pt}%
\pgfpathmoveto{\pgfqpoint{5.606289in}{2.097171in}}%
\pgfpathlineto{\pgfqpoint{5.620111in}{2.096873in}}%
\pgfpathlineto{\pgfqpoint{5.633942in}{2.096598in}}%
\pgfpathlineto{\pgfqpoint{5.647782in}{2.096347in}}%
\pgfpathlineto{\pgfqpoint{5.640616in}{2.087469in}}%
\pgfpathlineto{\pgfqpoint{5.633442in}{2.078486in}}%
\pgfpathlineto{\pgfqpoint{5.626260in}{2.069398in}}%
\pgfpathlineto{\pgfqpoint{5.619070in}{2.060206in}}%
\pgfpathlineto{\pgfqpoint{5.605221in}{2.060511in}}%
\pgfpathlineto{\pgfqpoint{5.591380in}{2.060839in}}%
\pgfpathlineto{\pgfqpoint{5.577549in}{2.061192in}}%
\pgfpathlineto{\pgfqpoint{5.584746in}{2.070340in}}%
\pgfpathlineto{\pgfqpoint{5.591935in}{2.079386in}}%
\pgfpathlineto{\pgfqpoint{5.599116in}{2.088330in}}%
\pgfpathlineto{\pgfqpoint{5.606289in}{2.097171in}}%
\pgfpathclose%
\pgfusepath{fill}%
\end{pgfscope}%
\begin{pgfscope}%
\pgfpathrectangle{\pgfqpoint{1.254980in}{0.150000in}}{\pgfqpoint{5.490039in}{5.490039in}}%
\pgfusepath{clip}%
\pgfsetbuttcap%
\pgfsetroundjoin%
\definecolor{currentfill}{rgb}{0.221989,0.339161,0.548752}%
\pgfsetfillcolor{currentfill}%
\pgfsetfillopacity{0.700000}%
\pgfsetlinewidth{0.000000pt}%
\definecolor{currentstroke}{rgb}{0.000000,0.000000,0.000000}%
\pgfsetstrokecolor{currentstroke}%
\pgfsetdash{}{0pt}%
\pgfpathmoveto{\pgfqpoint{2.630534in}{2.373673in}}%
\pgfpathlineto{\pgfqpoint{2.643712in}{2.364443in}}%
\pgfpathlineto{\pgfqpoint{2.656892in}{2.355254in}}%
\pgfpathlineto{\pgfqpoint{2.670074in}{2.346106in}}%
\pgfpathlineto{\pgfqpoint{2.683258in}{2.336998in}}%
\pgfpathlineto{\pgfqpoint{2.674549in}{2.346424in}}%
\pgfpathlineto{\pgfqpoint{2.665815in}{2.356320in}}%
\pgfpathlineto{\pgfqpoint{2.657056in}{2.366695in}}%
\pgfpathlineto{\pgfqpoint{2.648270in}{2.377559in}}%
\pgfpathlineto{\pgfqpoint{2.635042in}{2.387018in}}%
\pgfpathlineto{\pgfqpoint{2.621817in}{2.396518in}}%
\pgfpathlineto{\pgfqpoint{2.608593in}{2.406058in}}%
\pgfpathlineto{\pgfqpoint{2.595371in}{2.415640in}}%
\pgfpathlineto{\pgfqpoint{2.604201in}{2.404418in}}%
\pgfpathlineto{\pgfqpoint{2.613005in}{2.393690in}}%
\pgfpathlineto{\pgfqpoint{2.621782in}{2.383445in}}%
\pgfpathlineto{\pgfqpoint{2.630534in}{2.373673in}}%
\pgfpathclose%
\pgfusepath{fill}%
\end{pgfscope}%
\begin{pgfscope}%
\pgfpathrectangle{\pgfqpoint{1.254980in}{0.150000in}}{\pgfqpoint{5.490039in}{5.490039in}}%
\pgfusepath{clip}%
\pgfsetbuttcap%
\pgfsetroundjoin%
\definecolor{currentfill}{rgb}{0.269944,0.014625,0.341379}%
\pgfsetfillcolor{currentfill}%
\pgfsetfillopacity{0.700000}%
\pgfsetlinewidth{0.000000pt}%
\definecolor{currentstroke}{rgb}{0.000000,0.000000,0.000000}%
\pgfsetstrokecolor{currentstroke}%
\pgfsetdash{}{0pt}%
\pgfpathmoveto{\pgfqpoint{4.434537in}{1.740980in}}%
\pgfpathlineto{\pgfqpoint{4.447988in}{1.737748in}}%
\pgfpathlineto{\pgfqpoint{4.461444in}{1.734540in}}%
\pgfpathlineto{\pgfqpoint{4.474908in}{1.731356in}}%
\pgfpathlineto{\pgfqpoint{4.488378in}{1.728196in}}%
\pgfpathlineto{\pgfqpoint{4.480807in}{1.719810in}}%
\pgfpathlineto{\pgfqpoint{4.473230in}{1.711509in}}%
\pgfpathlineto{\pgfqpoint{4.465649in}{1.703297in}}%
\pgfpathlineto{\pgfqpoint{4.458063in}{1.695179in}}%
\pgfpathlineto{\pgfqpoint{4.444581in}{1.698548in}}%
\pgfpathlineto{\pgfqpoint{4.431107in}{1.701941in}}%
\pgfpathlineto{\pgfqpoint{4.417639in}{1.705358in}}%
\pgfpathlineto{\pgfqpoint{4.404177in}{1.708799in}}%
\pgfpathlineto{\pgfqpoint{4.411775in}{1.716703in}}%
\pgfpathlineto{\pgfqpoint{4.419367in}{1.724704in}}%
\pgfpathlineto{\pgfqpoint{4.426955in}{1.732798in}}%
\pgfpathlineto{\pgfqpoint{4.434537in}{1.740980in}}%
\pgfpathclose%
\pgfusepath{fill}%
\end{pgfscope}%
\begin{pgfscope}%
\pgfpathrectangle{\pgfqpoint{1.254980in}{0.150000in}}{\pgfqpoint{5.490039in}{5.490039in}}%
\pgfusepath{clip}%
\pgfsetbuttcap%
\pgfsetroundjoin%
\definecolor{currentfill}{rgb}{0.283197,0.115680,0.436115}%
\pgfsetfillcolor{currentfill}%
\pgfsetfillopacity{0.700000}%
\pgfsetlinewidth{0.000000pt}%
\definecolor{currentstroke}{rgb}{0.000000,0.000000,0.000000}%
\pgfsetstrokecolor{currentstroke}%
\pgfsetdash{}{0pt}%
\pgfpathmoveto{\pgfqpoint{3.417166in}{1.907423in}}%
\pgfpathlineto{\pgfqpoint{3.430408in}{1.900911in}}%
\pgfpathlineto{\pgfqpoint{3.443654in}{1.894428in}}%
\pgfpathlineto{\pgfqpoint{3.456905in}{1.887973in}}%
\pgfpathlineto{\pgfqpoint{3.470161in}{1.881547in}}%
\pgfpathlineto{\pgfqpoint{3.462111in}{1.882199in}}%
\pgfpathlineto{\pgfqpoint{3.454048in}{1.883165in}}%
\pgfpathlineto{\pgfqpoint{3.445973in}{1.884455in}}%
\pgfpathlineto{\pgfqpoint{3.437884in}{1.886076in}}%
\pgfpathlineto{\pgfqpoint{3.424602in}{1.892803in}}%
\pgfpathlineto{\pgfqpoint{3.411323in}{1.899559in}}%
\pgfpathlineto{\pgfqpoint{3.398049in}{1.906344in}}%
\pgfpathlineto{\pgfqpoint{3.384779in}{1.913157in}}%
\pgfpathlineto{\pgfqpoint{3.392896in}{1.911230in}}%
\pgfpathlineto{\pgfqpoint{3.400999in}{1.909637in}}%
\pgfpathlineto{\pgfqpoint{3.409089in}{1.908371in}}%
\pgfpathlineto{\pgfqpoint{3.417166in}{1.907423in}}%
\pgfpathclose%
\pgfusepath{fill}%
\end{pgfscope}%
\begin{pgfscope}%
\pgfpathrectangle{\pgfqpoint{1.254980in}{0.150000in}}{\pgfqpoint{5.490039in}{5.490039in}}%
\pgfusepath{clip}%
\pgfsetbuttcap%
\pgfsetroundjoin%
\definecolor{currentfill}{rgb}{0.282656,0.100196,0.422160}%
\pgfsetfillcolor{currentfill}%
\pgfsetfillopacity{0.700000}%
\pgfsetlinewidth{0.000000pt}%
\definecolor{currentstroke}{rgb}{0.000000,0.000000,0.000000}%
\pgfsetstrokecolor{currentstroke}%
\pgfsetdash{}{0pt}%
\pgfpathmoveto{\pgfqpoint{5.047206in}{1.891827in}}%
\pgfpathlineto{\pgfqpoint{5.060839in}{1.890343in}}%
\pgfpathlineto{\pgfqpoint{5.074481in}{1.888883in}}%
\pgfpathlineto{\pgfqpoint{5.088130in}{1.887447in}}%
\pgfpathlineto{\pgfqpoint{5.101788in}{1.886035in}}%
\pgfpathlineto{\pgfqpoint{5.094403in}{1.875909in}}%
\pgfpathlineto{\pgfqpoint{5.087013in}{1.865746in}}%
\pgfpathlineto{\pgfqpoint{5.079617in}{1.855547in}}%
\pgfpathlineto{\pgfqpoint{5.072215in}{1.845317in}}%
\pgfpathlineto{\pgfqpoint{5.058550in}{1.846862in}}%
\pgfpathlineto{\pgfqpoint{5.044893in}{1.848432in}}%
\pgfpathlineto{\pgfqpoint{5.031244in}{1.850024in}}%
\pgfpathlineto{\pgfqpoint{5.017603in}{1.851640in}}%
\pgfpathlineto{\pgfqpoint{5.025012in}{1.861733in}}%
\pgfpathlineto{\pgfqpoint{5.032415in}{1.871797in}}%
\pgfpathlineto{\pgfqpoint{5.039813in}{1.881829in}}%
\pgfpathlineto{\pgfqpoint{5.047206in}{1.891827in}}%
\pgfpathclose%
\pgfusepath{fill}%
\end{pgfscope}%
\begin{pgfscope}%
\pgfpathrectangle{\pgfqpoint{1.254980in}{0.150000in}}{\pgfqpoint{5.490039in}{5.490039in}}%
\pgfusepath{clip}%
\pgfsetbuttcap%
\pgfsetroundjoin%
\definecolor{currentfill}{rgb}{0.280894,0.078907,0.402329}%
\pgfsetfillcolor{currentfill}%
\pgfsetfillopacity{0.700000}%
\pgfsetlinewidth{0.000000pt}%
\definecolor{currentstroke}{rgb}{0.000000,0.000000,0.000000}%
\pgfsetstrokecolor{currentstroke}%
\pgfsetdash{}{0pt}%
\pgfpathmoveto{\pgfqpoint{3.608243in}{1.833819in}}%
\pgfpathlineto{\pgfqpoint{3.621515in}{1.827929in}}%
\pgfpathlineto{\pgfqpoint{3.634791in}{1.822066in}}%
\pgfpathlineto{\pgfqpoint{3.648073in}{1.816230in}}%
\pgfpathlineto{\pgfqpoint{3.661359in}{1.810421in}}%
\pgfpathlineto{\pgfqpoint{3.653429in}{1.809046in}}%
\pgfpathlineto{\pgfqpoint{3.645488in}{1.807944in}}%
\pgfpathlineto{\pgfqpoint{3.637537in}{1.807124in}}%
\pgfpathlineto{\pgfqpoint{3.629575in}{1.806592in}}%
\pgfpathlineto{\pgfqpoint{3.616265in}{1.812689in}}%
\pgfpathlineto{\pgfqpoint{3.602960in}{1.818812in}}%
\pgfpathlineto{\pgfqpoint{3.589659in}{1.824962in}}%
\pgfpathlineto{\pgfqpoint{3.576364in}{1.831139in}}%
\pgfpathlineto{\pgfqpoint{3.584350in}{1.831378in}}%
\pgfpathlineto{\pgfqpoint{3.592326in}{1.831910in}}%
\pgfpathlineto{\pgfqpoint{3.600290in}{1.832726in}}%
\pgfpathlineto{\pgfqpoint{3.608243in}{1.833819in}}%
\pgfpathclose%
\pgfusepath{fill}%
\end{pgfscope}%
\begin{pgfscope}%
\pgfpathrectangle{\pgfqpoint{1.254980in}{0.150000in}}{\pgfqpoint{5.490039in}{5.490039in}}%
\pgfusepath{clip}%
\pgfsetbuttcap%
\pgfsetroundjoin%
\definecolor{currentfill}{rgb}{0.274952,0.037752,0.364543}%
\pgfsetfillcolor{currentfill}%
\pgfsetfillopacity{0.700000}%
\pgfsetlinewidth{0.000000pt}%
\definecolor{currentstroke}{rgb}{0.000000,0.000000,0.000000}%
\pgfsetstrokecolor{currentstroke}%
\pgfsetdash{}{0pt}%
\pgfpathmoveto{\pgfqpoint{4.656630in}{1.776356in}}%
\pgfpathlineto{\pgfqpoint{4.670143in}{1.773791in}}%
\pgfpathlineto{\pgfqpoint{4.683664in}{1.771250in}}%
\pgfpathlineto{\pgfqpoint{4.697193in}{1.768733in}}%
\pgfpathlineto{\pgfqpoint{4.710728in}{1.766239in}}%
\pgfpathlineto{\pgfqpoint{4.703223in}{1.756815in}}%
\pgfpathlineto{\pgfqpoint{4.695714in}{1.747429in}}%
\pgfpathlineto{\pgfqpoint{4.688200in}{1.738083in}}%
\pgfpathlineto{\pgfqpoint{4.680681in}{1.728784in}}%
\pgfpathlineto{\pgfqpoint{4.667137in}{1.731461in}}%
\pgfpathlineto{\pgfqpoint{4.653599in}{1.734163in}}%
\pgfpathlineto{\pgfqpoint{4.640069in}{1.736888in}}%
\pgfpathlineto{\pgfqpoint{4.626546in}{1.739637in}}%
\pgfpathlineto{\pgfqpoint{4.634074in}{1.748747in}}%
\pgfpathlineto{\pgfqpoint{4.641597in}{1.757907in}}%
\pgfpathlineto{\pgfqpoint{4.649116in}{1.767112in}}%
\pgfpathlineto{\pgfqpoint{4.656630in}{1.776356in}}%
\pgfpathclose%
\pgfusepath{fill}%
\end{pgfscope}%
\begin{pgfscope}%
\pgfpathrectangle{\pgfqpoint{1.254980in}{0.150000in}}{\pgfqpoint{5.490039in}{5.490039in}}%
\pgfusepath{clip}%
\pgfsetbuttcap%
\pgfsetroundjoin%
\definecolor{currentfill}{rgb}{0.269944,0.014625,0.341379}%
\pgfsetfillcolor{currentfill}%
\pgfsetfillopacity{0.700000}%
\pgfsetlinewidth{0.000000pt}%
\definecolor{currentstroke}{rgb}{0.000000,0.000000,0.000000}%
\pgfsetstrokecolor{currentstroke}%
\pgfsetdash{}{0pt}%
\pgfpathmoveto{\pgfqpoint{4.074798in}{1.732790in}}%
\pgfpathlineto{\pgfqpoint{4.088161in}{1.728405in}}%
\pgfpathlineto{\pgfqpoint{4.101530in}{1.724045in}}%
\pgfpathlineto{\pgfqpoint{4.114906in}{1.719710in}}%
\pgfpathlineto{\pgfqpoint{4.128287in}{1.715400in}}%
\pgfpathlineto{\pgfqpoint{4.120586in}{1.709598in}}%
\pgfpathlineto{\pgfqpoint{4.112878in}{1.703965in}}%
\pgfpathlineto{\pgfqpoint{4.105165in}{1.698506in}}%
\pgfpathlineto{\pgfqpoint{4.097445in}{1.693228in}}%
\pgfpathlineto{\pgfqpoint{4.084048in}{1.697786in}}%
\pgfpathlineto{\pgfqpoint{4.070657in}{1.702368in}}%
\pgfpathlineto{\pgfqpoint{4.057271in}{1.706976in}}%
\pgfpathlineto{\pgfqpoint{4.043892in}{1.711608in}}%
\pgfpathlineto{\pgfqpoint{4.051628in}{1.716633in}}%
\pgfpathlineto{\pgfqpoint{4.059358in}{1.721843in}}%
\pgfpathlineto{\pgfqpoint{4.067081in}{1.727231in}}%
\pgfpathlineto{\pgfqpoint{4.074798in}{1.732790in}}%
\pgfpathclose%
\pgfusepath{fill}%
\end{pgfscope}%
\begin{pgfscope}%
\pgfpathrectangle{\pgfqpoint{1.254980in}{0.150000in}}{\pgfqpoint{5.490039in}{5.490039in}}%
\pgfusepath{clip}%
\pgfsetbuttcap%
\pgfsetroundjoin%
\definecolor{currentfill}{rgb}{0.272594,0.025563,0.353093}%
\pgfsetfillcolor{currentfill}%
\pgfsetfillopacity{0.700000}%
\pgfsetlinewidth{0.000000pt}%
\definecolor{currentstroke}{rgb}{0.000000,0.000000,0.000000}%
\pgfsetstrokecolor{currentstroke}%
\pgfsetdash{}{0pt}%
\pgfpathmoveto{\pgfqpoint{3.937061in}{1.749564in}}%
\pgfpathlineto{\pgfqpoint{3.950395in}{1.744731in}}%
\pgfpathlineto{\pgfqpoint{3.963735in}{1.739924in}}%
\pgfpathlineto{\pgfqpoint{3.977080in}{1.735142in}}%
\pgfpathlineto{\pgfqpoint{3.990431in}{1.730385in}}%
\pgfpathlineto{\pgfqpoint{3.982671in}{1.725805in}}%
\pgfpathlineto{\pgfqpoint{3.974903in}{1.721425in}}%
\pgfpathlineto{\pgfqpoint{3.967128in}{1.717253in}}%
\pgfpathlineto{\pgfqpoint{3.959345in}{1.713295in}}%
\pgfpathlineto{\pgfqpoint{3.945976in}{1.718312in}}%
\pgfpathlineto{\pgfqpoint{3.932612in}{1.723354in}}%
\pgfpathlineto{\pgfqpoint{3.919254in}{1.728422in}}%
\pgfpathlineto{\pgfqpoint{3.905902in}{1.733515in}}%
\pgfpathlineto{\pgfqpoint{3.913703in}{1.737208in}}%
\pgfpathlineto{\pgfqpoint{3.921497in}{1.741118in}}%
\pgfpathlineto{\pgfqpoint{3.929283in}{1.745239in}}%
\pgfpathlineto{\pgfqpoint{3.937061in}{1.749564in}}%
\pgfpathclose%
\pgfusepath{fill}%
\end{pgfscope}%
\begin{pgfscope}%
\pgfpathrectangle{\pgfqpoint{1.254980in}{0.150000in}}{\pgfqpoint{5.490039in}{5.490039in}}%
\pgfusepath{clip}%
\pgfsetbuttcap%
\pgfsetroundjoin%
\definecolor{currentfill}{rgb}{0.276194,0.190074,0.493001}%
\pgfsetfillcolor{currentfill}%
\pgfsetfillopacity{0.700000}%
\pgfsetlinewidth{0.000000pt}%
\definecolor{currentstroke}{rgb}{0.000000,0.000000,0.000000}%
\pgfsetstrokecolor{currentstroke}%
\pgfsetdash{}{0pt}%
\pgfpathmoveto{\pgfqpoint{5.522313in}{2.062839in}}%
\pgfpathlineto{\pgfqpoint{5.536108in}{2.062391in}}%
\pgfpathlineto{\pgfqpoint{5.549913in}{2.061968in}}%
\pgfpathlineto{\pgfqpoint{5.563726in}{2.061568in}}%
\pgfpathlineto{\pgfqpoint{5.577549in}{2.061192in}}%
\pgfpathlineto{\pgfqpoint{5.570344in}{2.051944in}}%
\pgfpathlineto{\pgfqpoint{5.563131in}{2.042596in}}%
\pgfpathlineto{\pgfqpoint{5.555911in}{2.033151in}}%
\pgfpathlineto{\pgfqpoint{5.548683in}{2.023609in}}%
\pgfpathlineto{\pgfqpoint{5.534852in}{2.024052in}}%
\pgfpathlineto{\pgfqpoint{5.521030in}{2.024519in}}%
\pgfpathlineto{\pgfqpoint{5.507216in}{2.025010in}}%
\pgfpathlineto{\pgfqpoint{5.493412in}{2.025525in}}%
\pgfpathlineto{\pgfqpoint{5.500648in}{2.034994in}}%
\pgfpathlineto{\pgfqpoint{5.507877in}{2.044371in}}%
\pgfpathlineto{\pgfqpoint{5.515099in}{2.053653in}}%
\pgfpathlineto{\pgfqpoint{5.522313in}{2.062839in}}%
\pgfpathclose%
\pgfusepath{fill}%
\end{pgfscope}%
\begin{pgfscope}%
\pgfpathrectangle{\pgfqpoint{1.254980in}{0.150000in}}{\pgfqpoint{5.490039in}{5.490039in}}%
\pgfusepath{clip}%
\pgfsetbuttcap%
\pgfsetroundjoin%
\definecolor{currentfill}{rgb}{0.268510,0.009605,0.335427}%
\pgfsetfillcolor{currentfill}%
\pgfsetfillopacity{0.700000}%
\pgfsetlinewidth{0.000000pt}%
\definecolor{currentstroke}{rgb}{0.000000,0.000000,0.000000}%
\pgfsetstrokecolor{currentstroke}%
\pgfsetdash{}{0pt}%
\pgfpathmoveto{\pgfqpoint{4.212556in}{1.724107in}}%
\pgfpathlineto{\pgfqpoint{4.225953in}{1.720154in}}%
\pgfpathlineto{\pgfqpoint{4.239356in}{1.716225in}}%
\pgfpathlineto{\pgfqpoint{4.252766in}{1.712321in}}%
\pgfpathlineto{\pgfqpoint{4.266181in}{1.708441in}}%
\pgfpathlineto{\pgfqpoint{4.258533in}{1.701568in}}%
\pgfpathlineto{\pgfqpoint{4.250878in}{1.694833in}}%
\pgfpathlineto{\pgfqpoint{4.243218in}{1.688242in}}%
\pgfpathlineto{\pgfqpoint{4.235553in}{1.681800in}}%
\pgfpathlineto{\pgfqpoint{4.222123in}{1.685914in}}%
\pgfpathlineto{\pgfqpoint{4.208700in}{1.690053in}}%
\pgfpathlineto{\pgfqpoint{4.195282in}{1.694216in}}%
\pgfpathlineto{\pgfqpoint{4.181871in}{1.698404in}}%
\pgfpathlineto{\pgfqpoint{4.189551in}{1.704607in}}%
\pgfpathlineto{\pgfqpoint{4.197225in}{1.710962in}}%
\pgfpathlineto{\pgfqpoint{4.204893in}{1.717464in}}%
\pgfpathlineto{\pgfqpoint{4.212556in}{1.724107in}}%
\pgfpathclose%
\pgfusepath{fill}%
\end{pgfscope}%
\begin{pgfscope}%
\pgfpathrectangle{\pgfqpoint{1.254980in}{0.150000in}}{\pgfqpoint{5.490039in}{5.490039in}}%
\pgfusepath{clip}%
\pgfsetbuttcap%
\pgfsetroundjoin%
\definecolor{currentfill}{rgb}{0.281446,0.084320,0.407414}%
\pgfsetfillcolor{currentfill}%
\pgfsetfillopacity{0.700000}%
\pgfsetlinewidth{0.000000pt}%
\definecolor{currentstroke}{rgb}{0.000000,0.000000,0.000000}%
\pgfsetstrokecolor{currentstroke}%
\pgfsetdash{}{0pt}%
\pgfpathmoveto{\pgfqpoint{4.963118in}{1.858341in}}%
\pgfpathlineto{\pgfqpoint{4.976728in}{1.856631in}}%
\pgfpathlineto{\pgfqpoint{4.990345in}{1.854944in}}%
\pgfpathlineto{\pgfqpoint{5.003970in}{1.853280in}}%
\pgfpathlineto{\pgfqpoint{5.017603in}{1.851640in}}%
\pgfpathlineto{\pgfqpoint{5.010189in}{1.841523in}}%
\pgfpathlineto{\pgfqpoint{5.002770in}{1.831383in}}%
\pgfpathlineto{\pgfqpoint{4.995346in}{1.821224in}}%
\pgfpathlineto{\pgfqpoint{4.987917in}{1.811051in}}%
\pgfpathlineto{\pgfqpoint{4.974276in}{1.812836in}}%
\pgfpathlineto{\pgfqpoint{4.960643in}{1.814645in}}%
\pgfpathlineto{\pgfqpoint{4.947018in}{1.816478in}}%
\pgfpathlineto{\pgfqpoint{4.933401in}{1.818335in}}%
\pgfpathlineto{\pgfqpoint{4.940838in}{1.828358in}}%
\pgfpathlineto{\pgfqpoint{4.948270in}{1.838369in}}%
\pgfpathlineto{\pgfqpoint{4.955697in}{1.848365in}}%
\pgfpathlineto{\pgfqpoint{4.963118in}{1.858341in}}%
\pgfpathclose%
\pgfusepath{fill}%
\end{pgfscope}%
\begin{pgfscope}%
\pgfpathrectangle{\pgfqpoint{1.254980in}{0.150000in}}{\pgfqpoint{5.490039in}{5.490039in}}%
\pgfusepath{clip}%
\pgfsetbuttcap%
\pgfsetroundjoin%
\definecolor{currentfill}{rgb}{0.280255,0.165693,0.476498}%
\pgfsetfillcolor{currentfill}%
\pgfsetfillopacity{0.700000}%
\pgfsetlinewidth{0.000000pt}%
\definecolor{currentstroke}{rgb}{0.000000,0.000000,0.000000}%
\pgfsetstrokecolor{currentstroke}%
\pgfsetdash{}{0pt}%
\pgfpathmoveto{\pgfqpoint{3.225850in}{1.997202in}}%
\pgfpathlineto{\pgfqpoint{3.239073in}{1.990034in}}%
\pgfpathlineto{\pgfqpoint{3.252299in}{1.982896in}}%
\pgfpathlineto{\pgfqpoint{3.265530in}{1.975788in}}%
\pgfpathlineto{\pgfqpoint{3.278764in}{1.968710in}}%
\pgfpathlineto{\pgfqpoint{3.270573in}{1.971601in}}%
\pgfpathlineto{\pgfqpoint{3.262367in}{1.974850in}}%
\pgfpathlineto{\pgfqpoint{3.254146in}{1.978466in}}%
\pgfpathlineto{\pgfqpoint{3.245908in}{1.982457in}}%
\pgfpathlineto{\pgfqpoint{3.232643in}{1.989851in}}%
\pgfpathlineto{\pgfqpoint{3.219381in}{1.997275in}}%
\pgfpathlineto{\pgfqpoint{3.206123in}{2.004729in}}%
\pgfpathlineto{\pgfqpoint{3.192869in}{2.012214in}}%
\pgfpathlineto{\pgfqpoint{3.201139in}{2.007900in}}%
\pgfpathlineto{\pgfqpoint{3.209392in}{2.003967in}}%
\pgfpathlineto{\pgfqpoint{3.217629in}{2.000403in}}%
\pgfpathlineto{\pgfqpoint{3.225850in}{1.997202in}}%
\pgfpathclose%
\pgfusepath{fill}%
\end{pgfscope}%
\begin{pgfscope}%
\pgfpathrectangle{\pgfqpoint{1.254980in}{0.150000in}}{\pgfqpoint{5.490039in}{5.490039in}}%
\pgfusepath{clip}%
\pgfsetbuttcap%
\pgfsetroundjoin%
\definecolor{currentfill}{rgb}{0.263663,0.237631,0.518762}%
\pgfsetfillcolor{currentfill}%
\pgfsetfillopacity{0.700000}%
\pgfsetlinewidth{0.000000pt}%
\definecolor{currentstroke}{rgb}{0.000000,0.000000,0.000000}%
\pgfsetstrokecolor{currentstroke}%
\pgfsetdash{}{0pt}%
\pgfpathmoveto{\pgfqpoint{2.981267in}{2.136282in}}%
\pgfpathlineto{\pgfqpoint{2.994468in}{2.128282in}}%
\pgfpathlineto{\pgfqpoint{3.007671in}{2.120315in}}%
\pgfpathlineto{\pgfqpoint{3.020878in}{2.112382in}}%
\pgfpathlineto{\pgfqpoint{3.034088in}{2.104483in}}%
\pgfpathlineto{\pgfqpoint{3.025698in}{2.110161in}}%
\pgfpathlineto{\pgfqpoint{3.017289in}{2.116249in}}%
\pgfpathlineto{\pgfqpoint{3.008861in}{2.122753in}}%
\pgfpathlineto{\pgfqpoint{3.000412in}{2.129684in}}%
\pgfpathlineto{\pgfqpoint{2.987166in}{2.137916in}}%
\pgfpathlineto{\pgfqpoint{2.973922in}{2.146181in}}%
\pgfpathlineto{\pgfqpoint{2.960682in}{2.154480in}}%
\pgfpathlineto{\pgfqpoint{2.947445in}{2.162813in}}%
\pgfpathlineto{\pgfqpoint{2.955931in}{2.155544in}}%
\pgfpathlineto{\pgfqpoint{2.964396in}{2.148705in}}%
\pgfpathlineto{\pgfqpoint{2.972842in}{2.142287in}}%
\pgfpathlineto{\pgfqpoint{2.981267in}{2.136282in}}%
\pgfpathclose%
\pgfusepath{fill}%
\end{pgfscope}%
\begin{pgfscope}%
\pgfpathrectangle{\pgfqpoint{1.254980in}{0.150000in}}{\pgfqpoint{5.490039in}{5.490039in}}%
\pgfusepath{clip}%
\pgfsetbuttcap%
\pgfsetroundjoin%
\definecolor{currentfill}{rgb}{0.276022,0.044167,0.370164}%
\pgfsetfillcolor{currentfill}%
\pgfsetfillopacity{0.700000}%
\pgfsetlinewidth{0.000000pt}%
\definecolor{currentstroke}{rgb}{0.000000,0.000000,0.000000}%
\pgfsetstrokecolor{currentstroke}%
\pgfsetdash{}{0pt}%
\pgfpathmoveto{\pgfqpoint{3.799274in}{1.775178in}}%
\pgfpathlineto{\pgfqpoint{3.812584in}{1.769880in}}%
\pgfpathlineto{\pgfqpoint{3.825899in}{1.764608in}}%
\pgfpathlineto{\pgfqpoint{3.839220in}{1.759362in}}%
\pgfpathlineto{\pgfqpoint{3.852546in}{1.754141in}}%
\pgfpathlineto{\pgfqpoint{3.844717in}{1.750940in}}%
\pgfpathlineto{\pgfqpoint{3.836879in}{1.747974in}}%
\pgfpathlineto{\pgfqpoint{3.829033in}{1.745249in}}%
\pgfpathlineto{\pgfqpoint{3.821178in}{1.742772in}}%
\pgfpathlineto{\pgfqpoint{3.807832in}{1.748266in}}%
\pgfpathlineto{\pgfqpoint{3.794491in}{1.753786in}}%
\pgfpathlineto{\pgfqpoint{3.781155in}{1.759332in}}%
\pgfpathlineto{\pgfqpoint{3.767825in}{1.764903in}}%
\pgfpathlineto{\pgfqpoint{3.775701in}{1.767101in}}%
\pgfpathlineto{\pgfqpoint{3.783567in}{1.769551in}}%
\pgfpathlineto{\pgfqpoint{3.791425in}{1.772246in}}%
\pgfpathlineto{\pgfqpoint{3.799274in}{1.775178in}}%
\pgfpathclose%
\pgfusepath{fill}%
\end{pgfscope}%
\begin{pgfscope}%
\pgfpathrectangle{\pgfqpoint{1.254980in}{0.150000in}}{\pgfqpoint{5.490039in}{5.490039in}}%
\pgfusepath{clip}%
\pgfsetbuttcap%
\pgfsetroundjoin%
\definecolor{currentfill}{rgb}{0.227802,0.326594,0.546532}%
\pgfsetfillcolor{currentfill}%
\pgfsetfillopacity{0.700000}%
\pgfsetlinewidth{0.000000pt}%
\definecolor{currentstroke}{rgb}{0.000000,0.000000,0.000000}%
\pgfsetstrokecolor{currentstroke}%
\pgfsetdash{}{0pt}%
\pgfpathmoveto{\pgfqpoint{2.683258in}{2.336998in}}%
\pgfpathlineto{\pgfqpoint{2.696445in}{2.327930in}}%
\pgfpathlineto{\pgfqpoint{2.709633in}{2.318901in}}%
\pgfpathlineto{\pgfqpoint{2.722824in}{2.309912in}}%
\pgfpathlineto{\pgfqpoint{2.736017in}{2.300962in}}%
\pgfpathlineto{\pgfqpoint{2.727350in}{2.310044in}}%
\pgfpathlineto{\pgfqpoint{2.718658in}{2.319591in}}%
\pgfpathlineto{\pgfqpoint{2.709942in}{2.329613in}}%
\pgfpathlineto{\pgfqpoint{2.701200in}{2.340121in}}%
\pgfpathlineto{\pgfqpoint{2.687964in}{2.349421in}}%
\pgfpathlineto{\pgfqpoint{2.674731in}{2.358761in}}%
\pgfpathlineto{\pgfqpoint{2.661499in}{2.368140in}}%
\pgfpathlineto{\pgfqpoint{2.648270in}{2.377559in}}%
\pgfpathlineto{\pgfqpoint{2.657056in}{2.366695in}}%
\pgfpathlineto{\pgfqpoint{2.665815in}{2.356320in}}%
\pgfpathlineto{\pgfqpoint{2.674549in}{2.346424in}}%
\pgfpathlineto{\pgfqpoint{2.683258in}{2.336998in}}%
\pgfpathclose%
\pgfusepath{fill}%
\end{pgfscope}%
\begin{pgfscope}%
\pgfpathrectangle{\pgfqpoint{1.254980in}{0.150000in}}{\pgfqpoint{5.490039in}{5.490039in}}%
\pgfusepath{clip}%
\pgfsetbuttcap%
\pgfsetroundjoin%
\definecolor{currentfill}{rgb}{0.278826,0.175490,0.483397}%
\pgfsetfillcolor{currentfill}%
\pgfsetfillopacity{0.700000}%
\pgfsetlinewidth{0.000000pt}%
\definecolor{currentstroke}{rgb}{0.000000,0.000000,0.000000}%
\pgfsetstrokecolor{currentstroke}%
\pgfsetdash{}{0pt}%
\pgfpathmoveto{\pgfqpoint{5.438282in}{2.027820in}}%
\pgfpathlineto{\pgfqpoint{5.452051in}{2.027210in}}%
\pgfpathlineto{\pgfqpoint{5.465829in}{2.026625in}}%
\pgfpathlineto{\pgfqpoint{5.479616in}{2.026063in}}%
\pgfpathlineto{\pgfqpoint{5.493412in}{2.025525in}}%
\pgfpathlineto{\pgfqpoint{5.486168in}{2.015963in}}%
\pgfpathlineto{\pgfqpoint{5.478917in}{2.006311in}}%
\pgfpathlineto{\pgfqpoint{5.471659in}{1.996570in}}%
\pgfpathlineto{\pgfqpoint{5.464394in}{1.986742in}}%
\pgfpathlineto{\pgfqpoint{5.450590in}{1.987361in}}%
\pgfpathlineto{\pgfqpoint{5.436794in}{1.988004in}}%
\pgfpathlineto{\pgfqpoint{5.423008in}{1.988670in}}%
\pgfpathlineto{\pgfqpoint{5.409230in}{1.989360in}}%
\pgfpathlineto{\pgfqpoint{5.416504in}{1.999102in}}%
\pgfpathlineto{\pgfqpoint{5.423770in}{2.008761in}}%
\pgfpathlineto{\pgfqpoint{5.431030in}{2.018334in}}%
\pgfpathlineto{\pgfqpoint{5.438282in}{2.027820in}}%
\pgfpathclose%
\pgfusepath{fill}%
\end{pgfscope}%
\begin{pgfscope}%
\pgfpathrectangle{\pgfqpoint{1.254980in}{0.150000in}}{\pgfqpoint{5.490039in}{5.490039in}}%
\pgfusepath{clip}%
\pgfsetbuttcap%
\pgfsetroundjoin%
\definecolor{currentfill}{rgb}{0.168126,0.459988,0.558082}%
\pgfsetfillcolor{currentfill}%
\pgfsetfillopacity{0.700000}%
\pgfsetlinewidth{0.000000pt}%
\definecolor{currentstroke}{rgb}{0.000000,0.000000,0.000000}%
\pgfsetstrokecolor{currentstroke}%
\pgfsetdash{}{0pt}%
\pgfpathmoveto{\pgfqpoint{2.278464in}{2.659042in}}%
\pgfpathlineto{\pgfqpoint{2.291657in}{2.648343in}}%
\pgfpathlineto{\pgfqpoint{2.304851in}{2.637696in}}%
\pgfpathlineto{\pgfqpoint{2.318045in}{2.627100in}}%
\pgfpathlineto{\pgfqpoint{2.331240in}{2.616556in}}%
\pgfpathlineto{\pgfqpoint{2.322143in}{2.630102in}}%
\pgfpathlineto{\pgfqpoint{2.313014in}{2.644183in}}%
\pgfpathlineto{\pgfqpoint{2.303852in}{2.658809in}}%
\pgfpathlineto{\pgfqpoint{2.294657in}{2.673992in}}%
\pgfpathlineto{\pgfqpoint{2.281411in}{2.684910in}}%
\pgfpathlineto{\pgfqpoint{2.268165in}{2.695879in}}%
\pgfpathlineto{\pgfqpoint{2.254920in}{2.706900in}}%
\pgfpathlineto{\pgfqpoint{2.241675in}{2.717974in}}%
\pgfpathlineto{\pgfqpoint{2.250923in}{2.702411in}}%
\pgfpathlineto{\pgfqpoint{2.260136in}{2.687409in}}%
\pgfpathlineto{\pgfqpoint{2.269317in}{2.672956in}}%
\pgfpathlineto{\pgfqpoint{2.278464in}{2.659042in}}%
\pgfpathclose%
\pgfusepath{fill}%
\end{pgfscope}%
\begin{pgfscope}%
\pgfpathrectangle{\pgfqpoint{1.254980in}{0.150000in}}{\pgfqpoint{5.490039in}{5.490039in}}%
\pgfusepath{clip}%
\pgfsetbuttcap%
\pgfsetroundjoin%
\definecolor{currentfill}{rgb}{0.272594,0.025563,0.353093}%
\pgfsetfillcolor{currentfill}%
\pgfsetfillopacity{0.700000}%
\pgfsetlinewidth{0.000000pt}%
\definecolor{currentstroke}{rgb}{0.000000,0.000000,0.000000}%
\pgfsetstrokecolor{currentstroke}%
\pgfsetdash{}{0pt}%
\pgfpathmoveto{\pgfqpoint{4.572525in}{1.750869in}}%
\pgfpathlineto{\pgfqpoint{4.586020in}{1.748026in}}%
\pgfpathlineto{\pgfqpoint{4.599522in}{1.745205in}}%
\pgfpathlineto{\pgfqpoint{4.613031in}{1.742409in}}%
\pgfpathlineto{\pgfqpoint{4.626546in}{1.739637in}}%
\pgfpathlineto{\pgfqpoint{4.619014in}{1.730579in}}%
\pgfpathlineto{\pgfqpoint{4.611477in}{1.721580in}}%
\pgfpathlineto{\pgfqpoint{4.603935in}{1.712644in}}%
\pgfpathlineto{\pgfqpoint{4.596389in}{1.703776in}}%
\pgfpathlineto{\pgfqpoint{4.582863in}{1.706746in}}%
\pgfpathlineto{\pgfqpoint{4.569344in}{1.709739in}}%
\pgfpathlineto{\pgfqpoint{4.555833in}{1.712756in}}%
\pgfpathlineto{\pgfqpoint{4.542328in}{1.715796in}}%
\pgfpathlineto{\pgfqpoint{4.549884in}{1.724463in}}%
\pgfpathlineto{\pgfqpoint{4.557436in}{1.733200in}}%
\pgfpathlineto{\pgfqpoint{4.564983in}{1.742004in}}%
\pgfpathlineto{\pgfqpoint{4.572525in}{1.750869in}}%
\pgfpathclose%
\pgfusepath{fill}%
\end{pgfscope}%
\begin{pgfscope}%
\pgfpathrectangle{\pgfqpoint{1.254980in}{0.150000in}}{\pgfqpoint{5.490039in}{5.490039in}}%
\pgfusepath{clip}%
\pgfsetbuttcap%
\pgfsetroundjoin%
\definecolor{currentfill}{rgb}{0.280267,0.073417,0.397163}%
\pgfsetfillcolor{currentfill}%
\pgfsetfillopacity{0.700000}%
\pgfsetlinewidth{0.000000pt}%
\definecolor{currentstroke}{rgb}{0.000000,0.000000,0.000000}%
\pgfsetstrokecolor{currentstroke}%
\pgfsetdash{}{0pt}%
\pgfpathmoveto{\pgfqpoint{4.879010in}{1.825997in}}%
\pgfpathlineto{\pgfqpoint{4.892596in}{1.824046in}}%
\pgfpathlineto{\pgfqpoint{4.906190in}{1.822118in}}%
\pgfpathlineto{\pgfqpoint{4.919792in}{1.820215in}}%
\pgfpathlineto{\pgfqpoint{4.933401in}{1.818335in}}%
\pgfpathlineto{\pgfqpoint{4.925959in}{1.808303in}}%
\pgfpathlineto{\pgfqpoint{4.918512in}{1.798267in}}%
\pgfpathlineto{\pgfqpoint{4.911061in}{1.788229in}}%
\pgfpathlineto{\pgfqpoint{4.903604in}{1.778194in}}%
\pgfpathlineto{\pgfqpoint{4.889987in}{1.780233in}}%
\pgfpathlineto{\pgfqpoint{4.876378in}{1.782295in}}%
\pgfpathlineto{\pgfqpoint{4.862776in}{1.784381in}}%
\pgfpathlineto{\pgfqpoint{4.849182in}{1.786490in}}%
\pgfpathlineto{\pgfqpoint{4.856646in}{1.796362in}}%
\pgfpathlineto{\pgfqpoint{4.864106in}{1.806239in}}%
\pgfpathlineto{\pgfqpoint{4.871560in}{1.816119in}}%
\pgfpathlineto{\pgfqpoint{4.879010in}{1.825997in}}%
\pgfpathclose%
\pgfusepath{fill}%
\end{pgfscope}%
\begin{pgfscope}%
\pgfpathrectangle{\pgfqpoint{1.254980in}{0.150000in}}{\pgfqpoint{5.490039in}{5.490039in}}%
\pgfusepath{clip}%
\pgfsetbuttcap%
\pgfsetroundjoin%
\definecolor{currentfill}{rgb}{0.268510,0.009605,0.335427}%
\pgfsetfillcolor{currentfill}%
\pgfsetfillopacity{0.700000}%
\pgfsetlinewidth{0.000000pt}%
\definecolor{currentstroke}{rgb}{0.000000,0.000000,0.000000}%
\pgfsetstrokecolor{currentstroke}%
\pgfsetdash{}{0pt}%
\pgfpathmoveto{\pgfqpoint{4.350397in}{1.722803in}}%
\pgfpathlineto{\pgfqpoint{4.363833in}{1.719266in}}%
\pgfpathlineto{\pgfqpoint{4.377274in}{1.715753in}}%
\pgfpathlineto{\pgfqpoint{4.390722in}{1.712264in}}%
\pgfpathlineto{\pgfqpoint{4.404177in}{1.708799in}}%
\pgfpathlineto{\pgfqpoint{4.396575in}{1.700999in}}%
\pgfpathlineto{\pgfqpoint{4.388967in}{1.693307in}}%
\pgfpathlineto{\pgfqpoint{4.381354in}{1.685730in}}%
\pgfpathlineto{\pgfqpoint{4.373737in}{1.678272in}}%
\pgfpathlineto{\pgfqpoint{4.360270in}{1.681959in}}%
\pgfpathlineto{\pgfqpoint{4.346809in}{1.685670in}}%
\pgfpathlineto{\pgfqpoint{4.333356in}{1.689405in}}%
\pgfpathlineto{\pgfqpoint{4.319908in}{1.693164in}}%
\pgfpathlineto{\pgfqpoint{4.327538in}{1.700395in}}%
\pgfpathlineto{\pgfqpoint{4.335163in}{1.707749in}}%
\pgfpathlineto{\pgfqpoint{4.342783in}{1.715220in}}%
\pgfpathlineto{\pgfqpoint{4.350397in}{1.722803in}}%
\pgfpathclose%
\pgfusepath{fill}%
\end{pgfscope}%
\begin{pgfscope}%
\pgfpathrectangle{\pgfqpoint{1.254980in}{0.150000in}}{\pgfqpoint{5.490039in}{5.490039in}}%
\pgfusepath{clip}%
\pgfsetbuttcap%
\pgfsetroundjoin%
\definecolor{currentfill}{rgb}{0.280868,0.160771,0.472899}%
\pgfsetfillcolor{currentfill}%
\pgfsetfillopacity{0.700000}%
\pgfsetlinewidth{0.000000pt}%
\definecolor{currentstroke}{rgb}{0.000000,0.000000,0.000000}%
\pgfsetstrokecolor{currentstroke}%
\pgfsetdash{}{0pt}%
\pgfpathmoveto{\pgfqpoint{5.354206in}{1.992356in}}%
\pgfpathlineto{\pgfqpoint{5.367949in}{1.991571in}}%
\pgfpathlineto{\pgfqpoint{5.381701in}{1.990810in}}%
\pgfpathlineto{\pgfqpoint{5.395461in}{1.990073in}}%
\pgfpathlineto{\pgfqpoint{5.409230in}{1.989360in}}%
\pgfpathlineto{\pgfqpoint{5.401950in}{1.979535in}}%
\pgfpathlineto{\pgfqpoint{5.394663in}{1.969631in}}%
\pgfpathlineto{\pgfqpoint{5.387369in}{1.959648in}}%
\pgfpathlineto{\pgfqpoint{5.380068in}{1.949589in}}%
\pgfpathlineto{\pgfqpoint{5.366292in}{1.950396in}}%
\pgfpathlineto{\pgfqpoint{5.352523in}{1.951227in}}%
\pgfpathlineto{\pgfqpoint{5.338764in}{1.952082in}}%
\pgfpathlineto{\pgfqpoint{5.325013in}{1.952960in}}%
\pgfpathlineto{\pgfqpoint{5.332321in}{1.962921in}}%
\pgfpathlineto{\pgfqpoint{5.339623in}{1.972808in}}%
\pgfpathlineto{\pgfqpoint{5.346918in}{1.982620in}}%
\pgfpathlineto{\pgfqpoint{5.354206in}{1.992356in}}%
\pgfpathclose%
\pgfusepath{fill}%
\end{pgfscope}%
\begin{pgfscope}%
\pgfpathrectangle{\pgfqpoint{1.254980in}{0.150000in}}{\pgfqpoint{5.490039in}{5.490039in}}%
\pgfusepath{clip}%
\pgfsetbuttcap%
\pgfsetroundjoin%
\definecolor{currentfill}{rgb}{0.283091,0.110553,0.431554}%
\pgfsetfillcolor{currentfill}%
\pgfsetfillopacity{0.700000}%
\pgfsetlinewidth{0.000000pt}%
\definecolor{currentstroke}{rgb}{0.000000,0.000000,0.000000}%
\pgfsetstrokecolor{currentstroke}%
\pgfsetdash{}{0pt}%
\pgfpathmoveto{\pgfqpoint{3.470161in}{1.881547in}}%
\pgfpathlineto{\pgfqpoint{3.483420in}{1.875148in}}%
\pgfpathlineto{\pgfqpoint{3.496685in}{1.868778in}}%
\pgfpathlineto{\pgfqpoint{3.509953in}{1.862436in}}%
\pgfpathlineto{\pgfqpoint{3.523226in}{1.856121in}}%
\pgfpathlineto{\pgfqpoint{3.515203in}{1.856477in}}%
\pgfpathlineto{\pgfqpoint{3.507167in}{1.857145in}}%
\pgfpathlineto{\pgfqpoint{3.499119in}{1.858132in}}%
\pgfpathlineto{\pgfqpoint{3.491058in}{1.859446in}}%
\pgfpathlineto{\pgfqpoint{3.477758in}{1.866062in}}%
\pgfpathlineto{\pgfqpoint{3.464462in}{1.872705in}}%
\pgfpathlineto{\pgfqpoint{3.451171in}{1.879376in}}%
\pgfpathlineto{\pgfqpoint{3.437884in}{1.886076in}}%
\pgfpathlineto{\pgfqpoint{3.445973in}{1.884455in}}%
\pgfpathlineto{\pgfqpoint{3.454048in}{1.883165in}}%
\pgfpathlineto{\pgfqpoint{3.462111in}{1.882199in}}%
\pgfpathlineto{\pgfqpoint{3.470161in}{1.881547in}}%
\pgfpathclose%
\pgfusepath{fill}%
\end{pgfscope}%
\begin{pgfscope}%
\pgfpathrectangle{\pgfqpoint{1.254980in}{0.150000in}}{\pgfqpoint{5.490039in}{5.490039in}}%
\pgfusepath{clip}%
\pgfsetbuttcap%
\pgfsetroundjoin%
\definecolor{currentfill}{rgb}{0.282623,0.140926,0.457517}%
\pgfsetfillcolor{currentfill}%
\pgfsetfillopacity{0.700000}%
\pgfsetlinewidth{0.000000pt}%
\definecolor{currentstroke}{rgb}{0.000000,0.000000,0.000000}%
\pgfsetstrokecolor{currentstroke}%
\pgfsetdash{}{0pt}%
\pgfpathmoveto{\pgfqpoint{5.270094in}{1.956710in}}%
\pgfpathlineto{\pgfqpoint{5.283811in}{1.955737in}}%
\pgfpathlineto{\pgfqpoint{5.297537in}{1.954788in}}%
\pgfpathlineto{\pgfqpoint{5.311271in}{1.953862in}}%
\pgfpathlineto{\pgfqpoint{5.325013in}{1.952960in}}%
\pgfpathlineto{\pgfqpoint{5.317698in}{1.942929in}}%
\pgfpathlineto{\pgfqpoint{5.310377in}{1.932830in}}%
\pgfpathlineto{\pgfqpoint{5.303050in}{1.922664in}}%
\pgfpathlineto{\pgfqpoint{5.295716in}{1.912434in}}%
\pgfpathlineto{\pgfqpoint{5.281966in}{1.913443in}}%
\pgfpathlineto{\pgfqpoint{5.268225in}{1.914475in}}%
\pgfpathlineto{\pgfqpoint{5.254492in}{1.915532in}}%
\pgfpathlineto{\pgfqpoint{5.240767in}{1.916612in}}%
\pgfpathlineto{\pgfqpoint{5.248108in}{1.926730in}}%
\pgfpathlineto{\pgfqpoint{5.255443in}{1.936787in}}%
\pgfpathlineto{\pgfqpoint{5.262772in}{1.946781in}}%
\pgfpathlineto{\pgfqpoint{5.270094in}{1.956710in}}%
\pgfpathclose%
\pgfusepath{fill}%
\end{pgfscope}%
\begin{pgfscope}%
\pgfpathrectangle{\pgfqpoint{1.254980in}{0.150000in}}{\pgfqpoint{5.490039in}{5.490039in}}%
\pgfusepath{clip}%
\pgfsetbuttcap%
\pgfsetroundjoin%
\definecolor{currentfill}{rgb}{0.280267,0.073417,0.397163}%
\pgfsetfillcolor{currentfill}%
\pgfsetfillopacity{0.700000}%
\pgfsetlinewidth{0.000000pt}%
\definecolor{currentstroke}{rgb}{0.000000,0.000000,0.000000}%
\pgfsetstrokecolor{currentstroke}%
\pgfsetdash{}{0pt}%
\pgfpathmoveto{\pgfqpoint{3.661359in}{1.810421in}}%
\pgfpathlineto{\pgfqpoint{3.674650in}{1.804638in}}%
\pgfpathlineto{\pgfqpoint{3.687946in}{1.798883in}}%
\pgfpathlineto{\pgfqpoint{3.701246in}{1.793154in}}%
\pgfpathlineto{\pgfqpoint{3.714552in}{1.787451in}}%
\pgfpathlineto{\pgfqpoint{3.706644in}{1.785794in}}%
\pgfpathlineto{\pgfqpoint{3.698727in}{1.784407in}}%
\pgfpathlineto{\pgfqpoint{3.690799in}{1.783298in}}%
\pgfpathlineto{\pgfqpoint{3.682861in}{1.782475in}}%
\pgfpathlineto{\pgfqpoint{3.669533in}{1.788464in}}%
\pgfpathlineto{\pgfqpoint{3.656209in}{1.794480in}}%
\pgfpathlineto{\pgfqpoint{3.642889in}{1.800523in}}%
\pgfpathlineto{\pgfqpoint{3.629575in}{1.806592in}}%
\pgfpathlineto{\pgfqpoint{3.637537in}{1.807124in}}%
\pgfpathlineto{\pgfqpoint{3.645488in}{1.807944in}}%
\pgfpathlineto{\pgfqpoint{3.653429in}{1.809046in}}%
\pgfpathlineto{\pgfqpoint{3.661359in}{1.810421in}}%
\pgfpathclose%
\pgfusepath{fill}%
\end{pgfscope}%
\begin{pgfscope}%
\pgfpathrectangle{\pgfqpoint{1.254980in}{0.150000in}}{\pgfqpoint{5.490039in}{5.490039in}}%
\pgfusepath{clip}%
\pgfsetbuttcap%
\pgfsetroundjoin%
\definecolor{currentfill}{rgb}{0.277941,0.056324,0.381191}%
\pgfsetfillcolor{currentfill}%
\pgfsetfillopacity{0.700000}%
\pgfsetlinewidth{0.000000pt}%
\definecolor{currentstroke}{rgb}{0.000000,0.000000,0.000000}%
\pgfsetstrokecolor{currentstroke}%
\pgfsetdash{}{0pt}%
\pgfpathmoveto{\pgfqpoint{4.794881in}{1.795165in}}%
\pgfpathlineto{\pgfqpoint{4.808445in}{1.792961in}}%
\pgfpathlineto{\pgfqpoint{4.822016in}{1.790780in}}%
\pgfpathlineto{\pgfqpoint{4.835595in}{1.788623in}}%
\pgfpathlineto{\pgfqpoint{4.849182in}{1.786490in}}%
\pgfpathlineto{\pgfqpoint{4.841713in}{1.776628in}}%
\pgfpathlineto{\pgfqpoint{4.834239in}{1.766780in}}%
\pgfpathlineto{\pgfqpoint{4.826760in}{1.756950in}}%
\pgfpathlineto{\pgfqpoint{4.819277in}{1.747141in}}%
\pgfpathlineto{\pgfqpoint{4.805682in}{1.749446in}}%
\pgfpathlineto{\pgfqpoint{4.792095in}{1.751774in}}%
\pgfpathlineto{\pgfqpoint{4.778516in}{1.754126in}}%
\pgfpathlineto{\pgfqpoint{4.764943in}{1.756501in}}%
\pgfpathlineto{\pgfqpoint{4.772435in}{1.766133in}}%
\pgfpathlineto{\pgfqpoint{4.779922in}{1.775791in}}%
\pgfpathlineto{\pgfqpoint{4.787404in}{1.785469in}}%
\pgfpathlineto{\pgfqpoint{4.794881in}{1.795165in}}%
\pgfpathclose%
\pgfusepath{fill}%
\end{pgfscope}%
\begin{pgfscope}%
\pgfpathrectangle{\pgfqpoint{1.254980in}{0.150000in}}{\pgfqpoint{5.490039in}{5.490039in}}%
\pgfusepath{clip}%
\pgfsetbuttcap%
\pgfsetroundjoin%
\definecolor{currentfill}{rgb}{0.174274,0.445044,0.557792}%
\pgfsetfillcolor{currentfill}%
\pgfsetfillopacity{0.700000}%
\pgfsetlinewidth{0.000000pt}%
\definecolor{currentstroke}{rgb}{0.000000,0.000000,0.000000}%
\pgfsetstrokecolor{currentstroke}%
\pgfsetdash{}{0pt}%
\pgfpathmoveto{\pgfqpoint{2.331240in}{2.616556in}}%
\pgfpathlineto{\pgfqpoint{2.344436in}{2.606062in}}%
\pgfpathlineto{\pgfqpoint{2.357633in}{2.595618in}}%
\pgfpathlineto{\pgfqpoint{2.370830in}{2.585224in}}%
\pgfpathlineto{\pgfqpoint{2.384029in}{2.574879in}}%
\pgfpathlineto{\pgfqpoint{2.374981in}{2.588058in}}%
\pgfpathlineto{\pgfqpoint{2.365902in}{2.601769in}}%
\pgfpathlineto{\pgfqpoint{2.356791in}{2.616020in}}%
\pgfpathlineto{\pgfqpoint{2.347648in}{2.630824in}}%
\pgfpathlineto{\pgfqpoint{2.334399in}{2.641541in}}%
\pgfpathlineto{\pgfqpoint{2.321151in}{2.652308in}}%
\pgfpathlineto{\pgfqpoint{2.307904in}{2.663125in}}%
\pgfpathlineto{\pgfqpoint{2.294657in}{2.673992in}}%
\pgfpathlineto{\pgfqpoint{2.303852in}{2.658809in}}%
\pgfpathlineto{\pgfqpoint{2.313014in}{2.644183in}}%
\pgfpathlineto{\pgfqpoint{2.322143in}{2.630102in}}%
\pgfpathlineto{\pgfqpoint{2.331240in}{2.616556in}}%
\pgfpathclose%
\pgfusepath{fill}%
\end{pgfscope}%
\begin{pgfscope}%
\pgfpathrectangle{\pgfqpoint{1.254980in}{0.150000in}}{\pgfqpoint{5.490039in}{5.490039in}}%
\pgfusepath{clip}%
\pgfsetbuttcap%
\pgfsetroundjoin%
\definecolor{currentfill}{rgb}{0.231674,0.318106,0.544834}%
\pgfsetfillcolor{currentfill}%
\pgfsetfillopacity{0.700000}%
\pgfsetlinewidth{0.000000pt}%
\definecolor{currentstroke}{rgb}{0.000000,0.000000,0.000000}%
\pgfsetstrokecolor{currentstroke}%
\pgfsetdash{}{0pt}%
\pgfpathmoveto{\pgfqpoint{2.736017in}{2.300962in}}%
\pgfpathlineto{\pgfqpoint{2.749212in}{2.292051in}}%
\pgfpathlineto{\pgfqpoint{2.762410in}{2.283178in}}%
\pgfpathlineto{\pgfqpoint{2.775610in}{2.274343in}}%
\pgfpathlineto{\pgfqpoint{2.788812in}{2.265545in}}%
\pgfpathlineto{\pgfqpoint{2.780186in}{2.274283in}}%
\pgfpathlineto{\pgfqpoint{2.771536in}{2.283482in}}%
\pgfpathlineto{\pgfqpoint{2.762862in}{2.293152in}}%
\pgfpathlineto{\pgfqpoint{2.754164in}{2.303304in}}%
\pgfpathlineto{\pgfqpoint{2.740919in}{2.312451in}}%
\pgfpathlineto{\pgfqpoint{2.727677in}{2.321636in}}%
\pgfpathlineto{\pgfqpoint{2.714437in}{2.330859in}}%
\pgfpathlineto{\pgfqpoint{2.701200in}{2.340121in}}%
\pgfpathlineto{\pgfqpoint{2.709942in}{2.329613in}}%
\pgfpathlineto{\pgfqpoint{2.718658in}{2.319591in}}%
\pgfpathlineto{\pgfqpoint{2.727350in}{2.310044in}}%
\pgfpathlineto{\pgfqpoint{2.736017in}{2.300962in}}%
\pgfpathclose%
\pgfusepath{fill}%
\end{pgfscope}%
\begin{pgfscope}%
\pgfpathrectangle{\pgfqpoint{1.254980in}{0.150000in}}{\pgfqpoint{5.490039in}{5.490039in}}%
\pgfusepath{clip}%
\pgfsetbuttcap%
\pgfsetroundjoin%
\definecolor{currentfill}{rgb}{0.267968,0.223549,0.512008}%
\pgfsetfillcolor{currentfill}%
\pgfsetfillopacity{0.700000}%
\pgfsetlinewidth{0.000000pt}%
\definecolor{currentstroke}{rgb}{0.000000,0.000000,0.000000}%
\pgfsetstrokecolor{currentstroke}%
\pgfsetdash{}{0pt}%
\pgfpathmoveto{\pgfqpoint{3.034088in}{2.104483in}}%
\pgfpathlineto{\pgfqpoint{3.047301in}{2.096616in}}%
\pgfpathlineto{\pgfqpoint{3.060518in}{2.088783in}}%
\pgfpathlineto{\pgfqpoint{3.073738in}{2.080982in}}%
\pgfpathlineto{\pgfqpoint{3.086961in}{2.073214in}}%
\pgfpathlineto{\pgfqpoint{3.078606in}{2.078566in}}%
\pgfpathlineto{\pgfqpoint{3.070232in}{2.084324in}}%
\pgfpathlineto{\pgfqpoint{3.061840in}{2.090494in}}%
\pgfpathlineto{\pgfqpoint{3.053428in}{2.097088in}}%
\pgfpathlineto{\pgfqpoint{3.040170in}{2.105188in}}%
\pgfpathlineto{\pgfqpoint{3.026914in}{2.113321in}}%
\pgfpathlineto{\pgfqpoint{3.013662in}{2.121486in}}%
\pgfpathlineto{\pgfqpoint{3.000412in}{2.129684in}}%
\pgfpathlineto{\pgfqpoint{3.008861in}{2.122753in}}%
\pgfpathlineto{\pgfqpoint{3.017289in}{2.116249in}}%
\pgfpathlineto{\pgfqpoint{3.025698in}{2.110161in}}%
\pgfpathlineto{\pgfqpoint{3.034088in}{2.104483in}}%
\pgfpathclose%
\pgfusepath{fill}%
\end{pgfscope}%
\begin{pgfscope}%
\pgfpathrectangle{\pgfqpoint{1.254980in}{0.150000in}}{\pgfqpoint{5.490039in}{5.490039in}}%
\pgfusepath{clip}%
\pgfsetbuttcap%
\pgfsetroundjoin%
\definecolor{currentfill}{rgb}{0.281412,0.155834,0.469201}%
\pgfsetfillcolor{currentfill}%
\pgfsetfillopacity{0.700000}%
\pgfsetlinewidth{0.000000pt}%
\definecolor{currentstroke}{rgb}{0.000000,0.000000,0.000000}%
\pgfsetstrokecolor{currentstroke}%
\pgfsetdash{}{0pt}%
\pgfpathmoveto{\pgfqpoint{3.278764in}{1.968710in}}%
\pgfpathlineto{\pgfqpoint{3.292002in}{1.961663in}}%
\pgfpathlineto{\pgfqpoint{3.305244in}{1.954645in}}%
\pgfpathlineto{\pgfqpoint{3.318489in}{1.947657in}}%
\pgfpathlineto{\pgfqpoint{3.331739in}{1.940699in}}%
\pgfpathlineto{\pgfqpoint{3.323579in}{1.943279in}}%
\pgfpathlineto{\pgfqpoint{3.315404in}{1.946214in}}%
\pgfpathlineto{\pgfqpoint{3.307213in}{1.949512in}}%
\pgfpathlineto{\pgfqpoint{3.299007in}{1.953183in}}%
\pgfpathlineto{\pgfqpoint{3.285727in}{1.960457in}}%
\pgfpathlineto{\pgfqpoint{3.272450in}{1.967761in}}%
\pgfpathlineto{\pgfqpoint{3.259177in}{1.975094in}}%
\pgfpathlineto{\pgfqpoint{3.245908in}{1.982457in}}%
\pgfpathlineto{\pgfqpoint{3.254146in}{1.978466in}}%
\pgfpathlineto{\pgfqpoint{3.262367in}{1.974850in}}%
\pgfpathlineto{\pgfqpoint{3.270573in}{1.971601in}}%
\pgfpathlineto{\pgfqpoint{3.278764in}{1.968710in}}%
\pgfpathclose%
\pgfusepath{fill}%
\end{pgfscope}%
\begin{pgfscope}%
\pgfpathrectangle{\pgfqpoint{1.254980in}{0.150000in}}{\pgfqpoint{5.490039in}{5.490039in}}%
\pgfusepath{clip}%
\pgfsetbuttcap%
\pgfsetroundjoin%
\definecolor{currentfill}{rgb}{0.271305,0.019942,0.347269}%
\pgfsetfillcolor{currentfill}%
\pgfsetfillopacity{0.700000}%
\pgfsetlinewidth{0.000000pt}%
\definecolor{currentstroke}{rgb}{0.000000,0.000000,0.000000}%
\pgfsetstrokecolor{currentstroke}%
\pgfsetdash{}{0pt}%
\pgfpathmoveto{\pgfqpoint{4.488378in}{1.728196in}}%
\pgfpathlineto{\pgfqpoint{4.501855in}{1.725060in}}%
\pgfpathlineto{\pgfqpoint{4.515339in}{1.721948in}}%
\pgfpathlineto{\pgfqpoint{4.528830in}{1.718860in}}%
\pgfpathlineto{\pgfqpoint{4.542328in}{1.715796in}}%
\pgfpathlineto{\pgfqpoint{4.534767in}{1.707206in}}%
\pgfpathlineto{\pgfqpoint{4.527201in}{1.698696in}}%
\pgfpathlineto{\pgfqpoint{4.519631in}{1.690273in}}%
\pgfpathlineto{\pgfqpoint{4.512056in}{1.681942in}}%
\pgfpathlineto{\pgfqpoint{4.498547in}{1.685215in}}%
\pgfpathlineto{\pgfqpoint{4.485046in}{1.688513in}}%
\pgfpathlineto{\pgfqpoint{4.471551in}{1.691834in}}%
\pgfpathlineto{\pgfqpoint{4.458063in}{1.695179in}}%
\pgfpathlineto{\pgfqpoint{4.465649in}{1.703297in}}%
\pgfpathlineto{\pgfqpoint{4.473230in}{1.711509in}}%
\pgfpathlineto{\pgfqpoint{4.480807in}{1.719810in}}%
\pgfpathlineto{\pgfqpoint{4.488378in}{1.728196in}}%
\pgfpathclose%
\pgfusepath{fill}%
\end{pgfscope}%
\begin{pgfscope}%
\pgfpathrectangle{\pgfqpoint{1.254980in}{0.150000in}}{\pgfqpoint{5.490039in}{5.490039in}}%
\pgfusepath{clip}%
\pgfsetbuttcap%
\pgfsetroundjoin%
\definecolor{currentfill}{rgb}{0.283187,0.125848,0.444960}%
\pgfsetfillcolor{currentfill}%
\pgfsetfillopacity{0.700000}%
\pgfsetlinewidth{0.000000pt}%
\definecolor{currentstroke}{rgb}{0.000000,0.000000,0.000000}%
\pgfsetstrokecolor{currentstroke}%
\pgfsetdash{}{0pt}%
\pgfpathmoveto{\pgfqpoint{5.185953in}{1.921168in}}%
\pgfpathlineto{\pgfqpoint{5.199644in}{1.919993in}}%
\pgfpathlineto{\pgfqpoint{5.213343in}{1.918842in}}%
\pgfpathlineto{\pgfqpoint{5.227051in}{1.917715in}}%
\pgfpathlineto{\pgfqpoint{5.240767in}{1.916612in}}%
\pgfpathlineto{\pgfqpoint{5.233420in}{1.906436in}}%
\pgfpathlineto{\pgfqpoint{5.226067in}{1.896204in}}%
\pgfpathlineto{\pgfqpoint{5.218709in}{1.885920in}}%
\pgfpathlineto{\pgfqpoint{5.211344in}{1.875585in}}%
\pgfpathlineto{\pgfqpoint{5.197621in}{1.876808in}}%
\pgfpathlineto{\pgfqpoint{5.183905in}{1.878056in}}%
\pgfpathlineto{\pgfqpoint{5.170199in}{1.879327in}}%
\pgfpathlineto{\pgfqpoint{5.156500in}{1.880621in}}%
\pgfpathlineto{\pgfqpoint{5.163872in}{1.890831in}}%
\pgfpathlineto{\pgfqpoint{5.171238in}{1.900993in}}%
\pgfpathlineto{\pgfqpoint{5.178598in}{1.911107in}}%
\pgfpathlineto{\pgfqpoint{5.185953in}{1.921168in}}%
\pgfpathclose%
\pgfusepath{fill}%
\end{pgfscope}%
\begin{pgfscope}%
\pgfpathrectangle{\pgfqpoint{1.254980in}{0.150000in}}{\pgfqpoint{5.490039in}{5.490039in}}%
\pgfusepath{clip}%
\pgfsetbuttcap%
\pgfsetroundjoin%
\definecolor{currentfill}{rgb}{0.269944,0.014625,0.341379}%
\pgfsetfillcolor{currentfill}%
\pgfsetfillopacity{0.700000}%
\pgfsetlinewidth{0.000000pt}%
\definecolor{currentstroke}{rgb}{0.000000,0.000000,0.000000}%
\pgfsetstrokecolor{currentstroke}%
\pgfsetdash{}{0pt}%
\pgfpathmoveto{\pgfqpoint{4.128287in}{1.715400in}}%
\pgfpathlineto{\pgfqpoint{4.141674in}{1.711114in}}%
\pgfpathlineto{\pgfqpoint{4.155067in}{1.706853in}}%
\pgfpathlineto{\pgfqpoint{4.168466in}{1.702616in}}%
\pgfpathlineto{\pgfqpoint{4.181871in}{1.698404in}}%
\pgfpathlineto{\pgfqpoint{4.174185in}{1.692360in}}%
\pgfpathlineto{\pgfqpoint{4.166493in}{1.686481in}}%
\pgfpathlineto{\pgfqpoint{4.158795in}{1.680773in}}%
\pgfpathlineto{\pgfqpoint{4.151091in}{1.675243in}}%
\pgfpathlineto{\pgfqpoint{4.137670in}{1.679703in}}%
\pgfpathlineto{\pgfqpoint{4.124256in}{1.684187in}}%
\pgfpathlineto{\pgfqpoint{4.110847in}{1.688695in}}%
\pgfpathlineto{\pgfqpoint{4.097445in}{1.693228in}}%
\pgfpathlineto{\pgfqpoint{4.105165in}{1.698506in}}%
\pgfpathlineto{\pgfqpoint{4.112878in}{1.703965in}}%
\pgfpathlineto{\pgfqpoint{4.120586in}{1.709598in}}%
\pgfpathlineto{\pgfqpoint{4.128287in}{1.715400in}}%
\pgfpathclose%
\pgfusepath{fill}%
\end{pgfscope}%
\begin{pgfscope}%
\pgfpathrectangle{\pgfqpoint{1.254980in}{0.150000in}}{\pgfqpoint{5.490039in}{5.490039in}}%
\pgfusepath{clip}%
\pgfsetbuttcap%
\pgfsetroundjoin%
\definecolor{currentfill}{rgb}{0.271305,0.019942,0.347269}%
\pgfsetfillcolor{currentfill}%
\pgfsetfillopacity{0.700000}%
\pgfsetlinewidth{0.000000pt}%
\definecolor{currentstroke}{rgb}{0.000000,0.000000,0.000000}%
\pgfsetstrokecolor{currentstroke}%
\pgfsetdash{}{0pt}%
\pgfpathmoveto{\pgfqpoint{3.990431in}{1.730385in}}%
\pgfpathlineto{\pgfqpoint{4.003788in}{1.725653in}}%
\pgfpathlineto{\pgfqpoint{4.017150in}{1.720946in}}%
\pgfpathlineto{\pgfqpoint{4.030518in}{1.716264in}}%
\pgfpathlineto{\pgfqpoint{4.043892in}{1.711608in}}%
\pgfpathlineto{\pgfqpoint{4.036148in}{1.706772in}}%
\pgfpathlineto{\pgfqpoint{4.028398in}{1.702134in}}%
\pgfpathlineto{\pgfqpoint{4.020641in}{1.697700in}}%
\pgfpathlineto{\pgfqpoint{4.012876in}{1.693476in}}%
\pgfpathlineto{\pgfqpoint{3.999485in}{1.698393in}}%
\pgfpathlineto{\pgfqpoint{3.986099in}{1.703335in}}%
\pgfpathlineto{\pgfqpoint{3.972719in}{1.708303in}}%
\pgfpathlineto{\pgfqpoint{3.959345in}{1.713295in}}%
\pgfpathlineto{\pgfqpoint{3.967128in}{1.717253in}}%
\pgfpathlineto{\pgfqpoint{3.974903in}{1.721425in}}%
\pgfpathlineto{\pgfqpoint{3.982671in}{1.725805in}}%
\pgfpathlineto{\pgfqpoint{3.990431in}{1.730385in}}%
\pgfpathclose%
\pgfusepath{fill}%
\end{pgfscope}%
\begin{pgfscope}%
\pgfpathrectangle{\pgfqpoint{1.254980in}{0.150000in}}{\pgfqpoint{5.490039in}{5.490039in}}%
\pgfusepath{clip}%
\pgfsetbuttcap%
\pgfsetroundjoin%
\definecolor{currentfill}{rgb}{0.276022,0.044167,0.370164}%
\pgfsetfillcolor{currentfill}%
\pgfsetfillopacity{0.700000}%
\pgfsetlinewidth{0.000000pt}%
\definecolor{currentstroke}{rgb}{0.000000,0.000000,0.000000}%
\pgfsetstrokecolor{currentstroke}%
\pgfsetdash{}{0pt}%
\pgfpathmoveto{\pgfqpoint{4.710728in}{1.766239in}}%
\pgfpathlineto{\pgfqpoint{4.724271in}{1.763769in}}%
\pgfpathlineto{\pgfqpoint{4.737821in}{1.761323in}}%
\pgfpathlineto{\pgfqpoint{4.751378in}{1.758900in}}%
\pgfpathlineto{\pgfqpoint{4.764943in}{1.756501in}}%
\pgfpathlineto{\pgfqpoint{4.757447in}{1.746898in}}%
\pgfpathlineto{\pgfqpoint{4.749947in}{1.737329in}}%
\pgfpathlineto{\pgfqpoint{4.742441in}{1.727798in}}%
\pgfpathlineto{\pgfqpoint{4.734932in}{1.718309in}}%
\pgfpathlineto{\pgfqpoint{4.721358in}{1.720892in}}%
\pgfpathlineto{\pgfqpoint{4.707792in}{1.723499in}}%
\pgfpathlineto{\pgfqpoint{4.694233in}{1.726129in}}%
\pgfpathlineto{\pgfqpoint{4.680681in}{1.728784in}}%
\pgfpathlineto{\pgfqpoint{4.688200in}{1.738083in}}%
\pgfpathlineto{\pgfqpoint{4.695714in}{1.747429in}}%
\pgfpathlineto{\pgfqpoint{4.703223in}{1.756815in}}%
\pgfpathlineto{\pgfqpoint{4.710728in}{1.766239in}}%
\pgfpathclose%
\pgfusepath{fill}%
\end{pgfscope}%
\begin{pgfscope}%
\pgfpathrectangle{\pgfqpoint{1.254980in}{0.150000in}}{\pgfqpoint{5.490039in}{5.490039in}}%
\pgfusepath{clip}%
\pgfsetbuttcap%
\pgfsetroundjoin%
\definecolor{currentfill}{rgb}{0.283091,0.110553,0.431554}%
\pgfsetfillcolor{currentfill}%
\pgfsetfillopacity{0.700000}%
\pgfsetlinewidth{0.000000pt}%
\definecolor{currentstroke}{rgb}{0.000000,0.000000,0.000000}%
\pgfsetstrokecolor{currentstroke}%
\pgfsetdash{}{0pt}%
\pgfpathmoveto{\pgfqpoint{5.101788in}{1.886035in}}%
\pgfpathlineto{\pgfqpoint{5.115454in}{1.884646in}}%
\pgfpathlineto{\pgfqpoint{5.129128in}{1.883281in}}%
\pgfpathlineto{\pgfqpoint{5.142810in}{1.881939in}}%
\pgfpathlineto{\pgfqpoint{5.156500in}{1.880621in}}%
\pgfpathlineto{\pgfqpoint{5.149122in}{1.870367in}}%
\pgfpathlineto{\pgfqpoint{5.141739in}{1.860072in}}%
\pgfpathlineto{\pgfqpoint{5.134351in}{1.849739in}}%
\pgfpathlineto{\pgfqpoint{5.126957in}{1.839371in}}%
\pgfpathlineto{\pgfqpoint{5.113259in}{1.840822in}}%
\pgfpathlineto{\pgfqpoint{5.099570in}{1.842297in}}%
\pgfpathlineto{\pgfqpoint{5.085889in}{1.843795in}}%
\pgfpathlineto{\pgfqpoint{5.072215in}{1.845317in}}%
\pgfpathlineto{\pgfqpoint{5.079617in}{1.855547in}}%
\pgfpathlineto{\pgfqpoint{5.087013in}{1.865746in}}%
\pgfpathlineto{\pgfqpoint{5.094403in}{1.875909in}}%
\pgfpathlineto{\pgfqpoint{5.101788in}{1.886035in}}%
\pgfpathclose%
\pgfusepath{fill}%
\end{pgfscope}%
\begin{pgfscope}%
\pgfpathrectangle{\pgfqpoint{1.254980in}{0.150000in}}{\pgfqpoint{5.490039in}{5.490039in}}%
\pgfusepath{clip}%
\pgfsetbuttcap%
\pgfsetroundjoin%
\definecolor{currentfill}{rgb}{0.268510,0.009605,0.335427}%
\pgfsetfillcolor{currentfill}%
\pgfsetfillopacity{0.700000}%
\pgfsetlinewidth{0.000000pt}%
\definecolor{currentstroke}{rgb}{0.000000,0.000000,0.000000}%
\pgfsetstrokecolor{currentstroke}%
\pgfsetdash{}{0pt}%
\pgfpathmoveto{\pgfqpoint{4.266181in}{1.708441in}}%
\pgfpathlineto{\pgfqpoint{4.279604in}{1.704585in}}%
\pgfpathlineto{\pgfqpoint{4.293032in}{1.700754in}}%
\pgfpathlineto{\pgfqpoint{4.306467in}{1.696947in}}%
\pgfpathlineto{\pgfqpoint{4.319908in}{1.693164in}}%
\pgfpathlineto{\pgfqpoint{4.312272in}{1.686061in}}%
\pgfpathlineto{\pgfqpoint{4.304631in}{1.679093in}}%
\pgfpathlineto{\pgfqpoint{4.296985in}{1.672265in}}%
\pgfpathlineto{\pgfqpoint{4.289333in}{1.665583in}}%
\pgfpathlineto{\pgfqpoint{4.275879in}{1.669601in}}%
\pgfpathlineto{\pgfqpoint{4.262431in}{1.673643in}}%
\pgfpathlineto{\pgfqpoint{4.248989in}{1.677709in}}%
\pgfpathlineto{\pgfqpoint{4.235553in}{1.681800in}}%
\pgfpathlineto{\pgfqpoint{4.243218in}{1.688242in}}%
\pgfpathlineto{\pgfqpoint{4.250878in}{1.694833in}}%
\pgfpathlineto{\pgfqpoint{4.258533in}{1.701568in}}%
\pgfpathlineto{\pgfqpoint{4.266181in}{1.708441in}}%
\pgfpathclose%
\pgfusepath{fill}%
\end{pgfscope}%
\begin{pgfscope}%
\pgfpathrectangle{\pgfqpoint{1.254980in}{0.150000in}}{\pgfqpoint{5.490039in}{5.490039in}}%
\pgfusepath{clip}%
\pgfsetbuttcap%
\pgfsetroundjoin%
\definecolor{currentfill}{rgb}{0.274952,0.037752,0.364543}%
\pgfsetfillcolor{currentfill}%
\pgfsetfillopacity{0.700000}%
\pgfsetlinewidth{0.000000pt}%
\definecolor{currentstroke}{rgb}{0.000000,0.000000,0.000000}%
\pgfsetstrokecolor{currentstroke}%
\pgfsetdash{}{0pt}%
\pgfpathmoveto{\pgfqpoint{3.852546in}{1.754141in}}%
\pgfpathlineto{\pgfqpoint{3.865877in}{1.748946in}}%
\pgfpathlineto{\pgfqpoint{3.879213in}{1.743777in}}%
\pgfpathlineto{\pgfqpoint{3.892555in}{1.738633in}}%
\pgfpathlineto{\pgfqpoint{3.905902in}{1.733515in}}%
\pgfpathlineto{\pgfqpoint{3.898092in}{1.730045in}}%
\pgfpathlineto{\pgfqpoint{3.890275in}{1.726807in}}%
\pgfpathlineto{\pgfqpoint{3.882449in}{1.723807in}}%
\pgfpathlineto{\pgfqpoint{3.874615in}{1.721052in}}%
\pgfpathlineto{\pgfqpoint{3.861248in}{1.726444in}}%
\pgfpathlineto{\pgfqpoint{3.847886in}{1.731861in}}%
\pgfpathlineto{\pgfqpoint{3.834530in}{1.737304in}}%
\pgfpathlineto{\pgfqpoint{3.821178in}{1.742772in}}%
\pgfpathlineto{\pgfqpoint{3.829033in}{1.745249in}}%
\pgfpathlineto{\pgfqpoint{3.836879in}{1.747974in}}%
\pgfpathlineto{\pgfqpoint{3.844717in}{1.750940in}}%
\pgfpathlineto{\pgfqpoint{3.852546in}{1.754141in}}%
\pgfpathclose%
\pgfusepath{fill}%
\end{pgfscope}%
\begin{pgfscope}%
\pgfpathrectangle{\pgfqpoint{1.254980in}{0.150000in}}{\pgfqpoint{5.490039in}{5.490039in}}%
\pgfusepath{clip}%
\pgfsetbuttcap%
\pgfsetroundjoin%
\definecolor{currentfill}{rgb}{0.180629,0.429975,0.557282}%
\pgfsetfillcolor{currentfill}%
\pgfsetfillopacity{0.700000}%
\pgfsetlinewidth{0.000000pt}%
\definecolor{currentstroke}{rgb}{0.000000,0.000000,0.000000}%
\pgfsetstrokecolor{currentstroke}%
\pgfsetdash{}{0pt}%
\pgfpathmoveto{\pgfqpoint{2.384029in}{2.574879in}}%
\pgfpathlineto{\pgfqpoint{2.397228in}{2.564582in}}%
\pgfpathlineto{\pgfqpoint{2.410428in}{2.554333in}}%
\pgfpathlineto{\pgfqpoint{2.423630in}{2.544132in}}%
\pgfpathlineto{\pgfqpoint{2.436832in}{2.533978in}}%
\pgfpathlineto{\pgfqpoint{2.427833in}{2.546793in}}%
\pgfpathlineto{\pgfqpoint{2.418804in}{2.560134in}}%
\pgfpathlineto{\pgfqpoint{2.409743in}{2.574012in}}%
\pgfpathlineto{\pgfqpoint{2.400651in}{2.588437in}}%
\pgfpathlineto{\pgfqpoint{2.387399in}{2.598962in}}%
\pgfpathlineto{\pgfqpoint{2.374148in}{2.609535in}}%
\pgfpathlineto{\pgfqpoint{2.360897in}{2.620155in}}%
\pgfpathlineto{\pgfqpoint{2.347648in}{2.630824in}}%
\pgfpathlineto{\pgfqpoint{2.356791in}{2.616020in}}%
\pgfpathlineto{\pgfqpoint{2.365902in}{2.601769in}}%
\pgfpathlineto{\pgfqpoint{2.374981in}{2.588058in}}%
\pgfpathlineto{\pgfqpoint{2.384029in}{2.574879in}}%
\pgfpathclose%
\pgfusepath{fill}%
\end{pgfscope}%
\begin{pgfscope}%
\pgfpathrectangle{\pgfqpoint{1.254980in}{0.150000in}}{\pgfqpoint{5.490039in}{5.490039in}}%
\pgfusepath{clip}%
\pgfsetbuttcap%
\pgfsetroundjoin%
\definecolor{currentfill}{rgb}{0.274128,0.199721,0.498911}%
\pgfsetfillcolor{currentfill}%
\pgfsetfillopacity{0.700000}%
\pgfsetlinewidth{0.000000pt}%
\definecolor{currentstroke}{rgb}{0.000000,0.000000,0.000000}%
\pgfsetstrokecolor{currentstroke}%
\pgfsetdash{}{0pt}%
\pgfpathmoveto{\pgfqpoint{5.577549in}{2.061192in}}%
\pgfpathlineto{\pgfqpoint{5.591380in}{2.060839in}}%
\pgfpathlineto{\pgfqpoint{5.605221in}{2.060511in}}%
\pgfpathlineto{\pgfqpoint{5.619070in}{2.060206in}}%
\pgfpathlineto{\pgfqpoint{5.611872in}{2.050911in}}%
\pgfpathlineto{\pgfqpoint{5.604666in}{2.041515in}}%
\pgfpathlineto{\pgfqpoint{5.597452in}{2.032018in}}%
\pgfpathlineto{\pgfqpoint{5.590231in}{2.022422in}}%
\pgfpathlineto{\pgfqpoint{5.576373in}{2.022794in}}%
\pgfpathlineto{\pgfqpoint{5.562524in}{2.023190in}}%
\pgfpathlineto{\pgfqpoint{5.548683in}{2.023609in}}%
\pgfpathlineto{\pgfqpoint{5.555911in}{2.033151in}}%
\pgfpathlineto{\pgfqpoint{5.563131in}{2.042596in}}%
\pgfpathlineto{\pgfqpoint{5.570344in}{2.051944in}}%
\pgfpathlineto{\pgfqpoint{5.577549in}{2.061192in}}%
\pgfpathclose%
\pgfusepath{fill}%
\end{pgfscope}%
\begin{pgfscope}%
\pgfpathrectangle{\pgfqpoint{1.254980in}{0.150000in}}{\pgfqpoint{5.490039in}{5.490039in}}%
\pgfusepath{clip}%
\pgfsetbuttcap%
\pgfsetroundjoin%
\definecolor{currentfill}{rgb}{0.282327,0.094955,0.417331}%
\pgfsetfillcolor{currentfill}%
\pgfsetfillopacity{0.700000}%
\pgfsetlinewidth{0.000000pt}%
\definecolor{currentstroke}{rgb}{0.000000,0.000000,0.000000}%
\pgfsetstrokecolor{currentstroke}%
\pgfsetdash{}{0pt}%
\pgfpathmoveto{\pgfqpoint{5.017603in}{1.851640in}}%
\pgfpathlineto{\pgfqpoint{5.031244in}{1.850024in}}%
\pgfpathlineto{\pgfqpoint{5.044893in}{1.848432in}}%
\pgfpathlineto{\pgfqpoint{5.058550in}{1.846862in}}%
\pgfpathlineto{\pgfqpoint{5.072215in}{1.845317in}}%
\pgfpathlineto{\pgfqpoint{5.064809in}{1.835058in}}%
\pgfpathlineto{\pgfqpoint{5.057397in}{1.824774in}}%
\pgfpathlineto{\pgfqpoint{5.049980in}{1.814468in}}%
\pgfpathlineto{\pgfqpoint{5.042558in}{1.804143in}}%
\pgfpathlineto{\pgfqpoint{5.028886in}{1.805835in}}%
\pgfpathlineto{\pgfqpoint{5.015222in}{1.807550in}}%
\pgfpathlineto{\pgfqpoint{5.001565in}{1.809288in}}%
\pgfpathlineto{\pgfqpoint{4.987917in}{1.811051in}}%
\pgfpathlineto{\pgfqpoint{4.995346in}{1.821224in}}%
\pgfpathlineto{\pgfqpoint{5.002770in}{1.831383in}}%
\pgfpathlineto{\pgfqpoint{5.010189in}{1.841523in}}%
\pgfpathlineto{\pgfqpoint{5.017603in}{1.851640in}}%
\pgfpathclose%
\pgfusepath{fill}%
\end{pgfscope}%
\begin{pgfscope}%
\pgfpathrectangle{\pgfqpoint{1.254980in}{0.150000in}}{\pgfqpoint{5.490039in}{5.490039in}}%
\pgfusepath{clip}%
\pgfsetbuttcap%
\pgfsetroundjoin%
\definecolor{currentfill}{rgb}{0.282910,0.105393,0.426902}%
\pgfsetfillcolor{currentfill}%
\pgfsetfillopacity{0.700000}%
\pgfsetlinewidth{0.000000pt}%
\definecolor{currentstroke}{rgb}{0.000000,0.000000,0.000000}%
\pgfsetstrokecolor{currentstroke}%
\pgfsetdash{}{0pt}%
\pgfpathmoveto{\pgfqpoint{3.523226in}{1.856121in}}%
\pgfpathlineto{\pgfqpoint{3.536504in}{1.849834in}}%
\pgfpathlineto{\pgfqpoint{3.549786in}{1.843575in}}%
\pgfpathlineto{\pgfqpoint{3.563072in}{1.837343in}}%
\pgfpathlineto{\pgfqpoint{3.576364in}{1.831139in}}%
\pgfpathlineto{\pgfqpoint{3.568366in}{1.831200in}}%
\pgfpathlineto{\pgfqpoint{3.560356in}{1.831568in}}%
\pgfpathlineto{\pgfqpoint{3.552334in}{1.832252in}}%
\pgfpathlineto{\pgfqpoint{3.544300in}{1.833260in}}%
\pgfpathlineto{\pgfqpoint{3.530983in}{1.839766in}}%
\pgfpathlineto{\pgfqpoint{3.517670in}{1.846298in}}%
\pgfpathlineto{\pgfqpoint{3.504362in}{1.852858in}}%
\pgfpathlineto{\pgfqpoint{3.491058in}{1.859446in}}%
\pgfpathlineto{\pgfqpoint{3.499119in}{1.858132in}}%
\pgfpathlineto{\pgfqpoint{3.507167in}{1.857145in}}%
\pgfpathlineto{\pgfqpoint{3.515203in}{1.856477in}}%
\pgfpathlineto{\pgfqpoint{3.523226in}{1.856121in}}%
\pgfpathclose%
\pgfusepath{fill}%
\end{pgfscope}%
\begin{pgfscope}%
\pgfpathrectangle{\pgfqpoint{1.254980in}{0.150000in}}{\pgfqpoint{5.490039in}{5.490039in}}%
\pgfusepath{clip}%
\pgfsetbuttcap%
\pgfsetroundjoin%
\definecolor{currentfill}{rgb}{0.237441,0.305202,0.541921}%
\pgfsetfillcolor{currentfill}%
\pgfsetfillopacity{0.700000}%
\pgfsetlinewidth{0.000000pt}%
\definecolor{currentstroke}{rgb}{0.000000,0.000000,0.000000}%
\pgfsetstrokecolor{currentstroke}%
\pgfsetdash{}{0pt}%
\pgfpathmoveto{\pgfqpoint{2.788812in}{2.265545in}}%
\pgfpathlineto{\pgfqpoint{2.802017in}{2.256785in}}%
\pgfpathlineto{\pgfqpoint{2.815225in}{2.248062in}}%
\pgfpathlineto{\pgfqpoint{2.828435in}{2.239376in}}%
\pgfpathlineto{\pgfqpoint{2.841647in}{2.230727in}}%
\pgfpathlineto{\pgfqpoint{2.833061in}{2.239121in}}%
\pgfpathlineto{\pgfqpoint{2.824452in}{2.247973in}}%
\pgfpathlineto{\pgfqpoint{2.815820in}{2.257292in}}%
\pgfpathlineto{\pgfqpoint{2.807164in}{2.267088in}}%
\pgfpathlineto{\pgfqpoint{2.793910in}{2.276087in}}%
\pgfpathlineto{\pgfqpoint{2.780659in}{2.285122in}}%
\pgfpathlineto{\pgfqpoint{2.767410in}{2.294194in}}%
\pgfpathlineto{\pgfqpoint{2.754164in}{2.303304in}}%
\pgfpathlineto{\pgfqpoint{2.762862in}{2.293152in}}%
\pgfpathlineto{\pgfqpoint{2.771536in}{2.283482in}}%
\pgfpathlineto{\pgfqpoint{2.780186in}{2.274283in}}%
\pgfpathlineto{\pgfqpoint{2.788812in}{2.265545in}}%
\pgfpathclose%
\pgfusepath{fill}%
\end{pgfscope}%
\begin{pgfscope}%
\pgfpathrectangle{\pgfqpoint{1.254980in}{0.150000in}}{\pgfqpoint{5.490039in}{5.490039in}}%
\pgfusepath{clip}%
\pgfsetbuttcap%
\pgfsetroundjoin%
\definecolor{currentfill}{rgb}{0.269944,0.014625,0.341379}%
\pgfsetfillcolor{currentfill}%
\pgfsetfillopacity{0.700000}%
\pgfsetlinewidth{0.000000pt}%
\definecolor{currentstroke}{rgb}{0.000000,0.000000,0.000000}%
\pgfsetstrokecolor{currentstroke}%
\pgfsetdash{}{0pt}%
\pgfpathmoveto{\pgfqpoint{4.404177in}{1.708799in}}%
\pgfpathlineto{\pgfqpoint{4.417639in}{1.705358in}}%
\pgfpathlineto{\pgfqpoint{4.431107in}{1.701941in}}%
\pgfpathlineto{\pgfqpoint{4.444581in}{1.698548in}}%
\pgfpathlineto{\pgfqpoint{4.458063in}{1.695179in}}%
\pgfpathlineto{\pgfqpoint{4.450472in}{1.687162in}}%
\pgfpathlineto{\pgfqpoint{4.442876in}{1.679250in}}%
\pgfpathlineto{\pgfqpoint{4.435275in}{1.671449in}}%
\pgfpathlineto{\pgfqpoint{4.427670in}{1.663764in}}%
\pgfpathlineto{\pgfqpoint{4.414177in}{1.667355in}}%
\pgfpathlineto{\pgfqpoint{4.400690in}{1.670970in}}%
\pgfpathlineto{\pgfqpoint{4.387210in}{1.674609in}}%
\pgfpathlineto{\pgfqpoint{4.373737in}{1.678272in}}%
\pgfpathlineto{\pgfqpoint{4.381354in}{1.685730in}}%
\pgfpathlineto{\pgfqpoint{4.388967in}{1.693307in}}%
\pgfpathlineto{\pgfqpoint{4.396575in}{1.700999in}}%
\pgfpathlineto{\pgfqpoint{4.404177in}{1.708799in}}%
\pgfpathclose%
\pgfusepath{fill}%
\end{pgfscope}%
\begin{pgfscope}%
\pgfpathrectangle{\pgfqpoint{1.254980in}{0.150000in}}{\pgfqpoint{5.490039in}{5.490039in}}%
\pgfusepath{clip}%
\pgfsetbuttcap%
\pgfsetroundjoin%
\definecolor{currentfill}{rgb}{0.273809,0.031497,0.358853}%
\pgfsetfillcolor{currentfill}%
\pgfsetfillopacity{0.700000}%
\pgfsetlinewidth{0.000000pt}%
\definecolor{currentstroke}{rgb}{0.000000,0.000000,0.000000}%
\pgfsetstrokecolor{currentstroke}%
\pgfsetdash{}{0pt}%
\pgfpathmoveto{\pgfqpoint{4.626546in}{1.739637in}}%
\pgfpathlineto{\pgfqpoint{4.640069in}{1.736888in}}%
\pgfpathlineto{\pgfqpoint{4.653599in}{1.734163in}}%
\pgfpathlineto{\pgfqpoint{4.667137in}{1.731461in}}%
\pgfpathlineto{\pgfqpoint{4.680681in}{1.728784in}}%
\pgfpathlineto{\pgfqpoint{4.673158in}{1.719534in}}%
\pgfpathlineto{\pgfqpoint{4.665630in}{1.710340in}}%
\pgfpathlineto{\pgfqpoint{4.658098in}{1.701206in}}%
\pgfpathlineto{\pgfqpoint{4.650561in}{1.692136in}}%
\pgfpathlineto{\pgfqpoint{4.637008in}{1.695011in}}%
\pgfpathlineto{\pgfqpoint{4.623461in}{1.697909in}}%
\pgfpathlineto{\pgfqpoint{4.609921in}{1.700831in}}%
\pgfpathlineto{\pgfqpoint{4.596389in}{1.703776in}}%
\pgfpathlineto{\pgfqpoint{4.603935in}{1.712644in}}%
\pgfpathlineto{\pgfqpoint{4.611477in}{1.721580in}}%
\pgfpathlineto{\pgfqpoint{4.619014in}{1.730579in}}%
\pgfpathlineto{\pgfqpoint{4.626546in}{1.739637in}}%
\pgfpathclose%
\pgfusepath{fill}%
\end{pgfscope}%
\begin{pgfscope}%
\pgfpathrectangle{\pgfqpoint{1.254980in}{0.150000in}}{\pgfqpoint{5.490039in}{5.490039in}}%
\pgfusepath{clip}%
\pgfsetbuttcap%
\pgfsetroundjoin%
\definecolor{currentfill}{rgb}{0.270595,0.214069,0.507052}%
\pgfsetfillcolor{currentfill}%
\pgfsetfillopacity{0.700000}%
\pgfsetlinewidth{0.000000pt}%
\definecolor{currentstroke}{rgb}{0.000000,0.000000,0.000000}%
\pgfsetstrokecolor{currentstroke}%
\pgfsetdash{}{0pt}%
\pgfpathmoveto{\pgfqpoint{3.086961in}{2.073214in}}%
\pgfpathlineto{\pgfqpoint{3.100187in}{2.065478in}}%
\pgfpathlineto{\pgfqpoint{3.113417in}{2.057774in}}%
\pgfpathlineto{\pgfqpoint{3.126650in}{2.050102in}}%
\pgfpathlineto{\pgfqpoint{3.139887in}{2.042462in}}%
\pgfpathlineto{\pgfqpoint{3.131566in}{2.047488in}}%
\pgfpathlineto{\pgfqpoint{3.123228in}{2.052916in}}%
\pgfpathlineto{\pgfqpoint{3.114871in}{2.058754in}}%
\pgfpathlineto{\pgfqpoint{3.106496in}{2.065010in}}%
\pgfpathlineto{\pgfqpoint{3.093224in}{2.072982in}}%
\pgfpathlineto{\pgfqpoint{3.079955in}{2.080985in}}%
\pgfpathlineto{\pgfqpoint{3.066690in}{2.089021in}}%
\pgfpathlineto{\pgfqpoint{3.053428in}{2.097088in}}%
\pgfpathlineto{\pgfqpoint{3.061840in}{2.090494in}}%
\pgfpathlineto{\pgfqpoint{3.070232in}{2.084324in}}%
\pgfpathlineto{\pgfqpoint{3.078606in}{2.078566in}}%
\pgfpathlineto{\pgfqpoint{3.086961in}{2.073214in}}%
\pgfpathclose%
\pgfusepath{fill}%
\end{pgfscope}%
\begin{pgfscope}%
\pgfpathrectangle{\pgfqpoint{1.254980in}{0.150000in}}{\pgfqpoint{5.490039in}{5.490039in}}%
\pgfusepath{clip}%
\pgfsetbuttcap%
\pgfsetroundjoin%
\definecolor{currentfill}{rgb}{0.279566,0.067836,0.391917}%
\pgfsetfillcolor{currentfill}%
\pgfsetfillopacity{0.700000}%
\pgfsetlinewidth{0.000000pt}%
\definecolor{currentstroke}{rgb}{0.000000,0.000000,0.000000}%
\pgfsetstrokecolor{currentstroke}%
\pgfsetdash{}{0pt}%
\pgfpathmoveto{\pgfqpoint{3.714552in}{1.787451in}}%
\pgfpathlineto{\pgfqpoint{3.727863in}{1.781775in}}%
\pgfpathlineto{\pgfqpoint{3.741178in}{1.776125in}}%
\pgfpathlineto{\pgfqpoint{3.754499in}{1.770501in}}%
\pgfpathlineto{\pgfqpoint{3.767825in}{1.764903in}}%
\pgfpathlineto{\pgfqpoint{3.759939in}{1.762965in}}%
\pgfpathlineto{\pgfqpoint{3.752044in}{1.761293in}}%
\pgfpathlineto{\pgfqpoint{3.744140in}{1.759895in}}%
\pgfpathlineto{\pgfqpoint{3.736225in}{1.758779in}}%
\pgfpathlineto{\pgfqpoint{3.722877in}{1.764664in}}%
\pgfpathlineto{\pgfqpoint{3.709534in}{1.770575in}}%
\pgfpathlineto{\pgfqpoint{3.696195in}{1.776512in}}%
\pgfpathlineto{\pgfqpoint{3.682861in}{1.782475in}}%
\pgfpathlineto{\pgfqpoint{3.690799in}{1.783298in}}%
\pgfpathlineto{\pgfqpoint{3.698727in}{1.784407in}}%
\pgfpathlineto{\pgfqpoint{3.706644in}{1.785794in}}%
\pgfpathlineto{\pgfqpoint{3.714552in}{1.787451in}}%
\pgfpathclose%
\pgfusepath{fill}%
\end{pgfscope}%
\begin{pgfscope}%
\pgfpathrectangle{\pgfqpoint{1.254980in}{0.150000in}}{\pgfqpoint{5.490039in}{5.490039in}}%
\pgfusepath{clip}%
\pgfsetbuttcap%
\pgfsetroundjoin%
\definecolor{currentfill}{rgb}{0.280894,0.078907,0.402329}%
\pgfsetfillcolor{currentfill}%
\pgfsetfillopacity{0.700000}%
\pgfsetlinewidth{0.000000pt}%
\definecolor{currentstroke}{rgb}{0.000000,0.000000,0.000000}%
\pgfsetstrokecolor{currentstroke}%
\pgfsetdash{}{0pt}%
\pgfpathmoveto{\pgfqpoint{4.933401in}{1.818335in}}%
\pgfpathlineto{\pgfqpoint{4.947018in}{1.816478in}}%
\pgfpathlineto{\pgfqpoint{4.960643in}{1.814645in}}%
\pgfpathlineto{\pgfqpoint{4.974276in}{1.812836in}}%
\pgfpathlineto{\pgfqpoint{4.987917in}{1.811051in}}%
\pgfpathlineto{\pgfqpoint{4.980482in}{1.800865in}}%
\pgfpathlineto{\pgfqpoint{4.973043in}{1.790671in}}%
\pgfpathlineto{\pgfqpoint{4.965599in}{1.780473in}}%
\pgfpathlineto{\pgfqpoint{4.958150in}{1.770274in}}%
\pgfpathlineto{\pgfqpoint{4.944502in}{1.772219in}}%
\pgfpathlineto{\pgfqpoint{4.930862in}{1.774187in}}%
\pgfpathlineto{\pgfqpoint{4.917229in}{1.776179in}}%
\pgfpathlineto{\pgfqpoint{4.903604in}{1.778194in}}%
\pgfpathlineto{\pgfqpoint{4.911061in}{1.788229in}}%
\pgfpathlineto{\pgfqpoint{4.918512in}{1.798267in}}%
\pgfpathlineto{\pgfqpoint{4.925959in}{1.808303in}}%
\pgfpathlineto{\pgfqpoint{4.933401in}{1.818335in}}%
\pgfpathclose%
\pgfusepath{fill}%
\end{pgfscope}%
\begin{pgfscope}%
\pgfpathrectangle{\pgfqpoint{1.254980in}{0.150000in}}{\pgfqpoint{5.490039in}{5.490039in}}%
\pgfusepath{clip}%
\pgfsetbuttcap%
\pgfsetroundjoin%
\definecolor{currentfill}{rgb}{0.277134,0.185228,0.489898}%
\pgfsetfillcolor{currentfill}%
\pgfsetfillopacity{0.700000}%
\pgfsetlinewidth{0.000000pt}%
\definecolor{currentstroke}{rgb}{0.000000,0.000000,0.000000}%
\pgfsetstrokecolor{currentstroke}%
\pgfsetdash{}{0pt}%
\pgfpathmoveto{\pgfqpoint{5.493412in}{2.025525in}}%
\pgfpathlineto{\pgfqpoint{5.507216in}{2.025010in}}%
\pgfpathlineto{\pgfqpoint{5.521030in}{2.024519in}}%
\pgfpathlineto{\pgfqpoint{5.534852in}{2.024052in}}%
\pgfpathlineto{\pgfqpoint{5.548683in}{2.023609in}}%
\pgfpathlineto{\pgfqpoint{5.541448in}{2.013972in}}%
\pgfpathlineto{\pgfqpoint{5.534205in}{2.004241in}}%
\pgfpathlineto{\pgfqpoint{5.526955in}{1.994418in}}%
\pgfpathlineto{\pgfqpoint{5.519698in}{1.984504in}}%
\pgfpathlineto{\pgfqpoint{5.505858in}{1.985028in}}%
\pgfpathlineto{\pgfqpoint{5.492028in}{1.985576in}}%
\pgfpathlineto{\pgfqpoint{5.478206in}{1.986147in}}%
\pgfpathlineto{\pgfqpoint{5.464394in}{1.986742in}}%
\pgfpathlineto{\pgfqpoint{5.471659in}{1.996570in}}%
\pgfpathlineto{\pgfqpoint{5.478917in}{2.006311in}}%
\pgfpathlineto{\pgfqpoint{5.486168in}{2.015963in}}%
\pgfpathlineto{\pgfqpoint{5.493412in}{2.025525in}}%
\pgfpathclose%
\pgfusepath{fill}%
\end{pgfscope}%
\begin{pgfscope}%
\pgfpathrectangle{\pgfqpoint{1.254980in}{0.150000in}}{\pgfqpoint{5.490039in}{5.490039in}}%
\pgfusepath{clip}%
\pgfsetbuttcap%
\pgfsetroundjoin%
\definecolor{currentfill}{rgb}{0.281887,0.150881,0.465405}%
\pgfsetfillcolor{currentfill}%
\pgfsetfillopacity{0.700000}%
\pgfsetlinewidth{0.000000pt}%
\definecolor{currentstroke}{rgb}{0.000000,0.000000,0.000000}%
\pgfsetstrokecolor{currentstroke}%
\pgfsetdash{}{0pt}%
\pgfpathmoveto{\pgfqpoint{3.331739in}{1.940699in}}%
\pgfpathlineto{\pgfqpoint{3.344993in}{1.933770in}}%
\pgfpathlineto{\pgfqpoint{3.358251in}{1.926870in}}%
\pgfpathlineto{\pgfqpoint{3.371513in}{1.919999in}}%
\pgfpathlineto{\pgfqpoint{3.384779in}{1.913157in}}%
\pgfpathlineto{\pgfqpoint{3.376648in}{1.915427in}}%
\pgfpathlineto{\pgfqpoint{3.368503in}{1.918048in}}%
\pgfpathlineto{\pgfqpoint{3.360343in}{1.921029in}}%
\pgfpathlineto{\pgfqpoint{3.352168in}{1.924379in}}%
\pgfpathlineto{\pgfqpoint{3.338872in}{1.931536in}}%
\pgfpathlineto{\pgfqpoint{3.325580in}{1.938722in}}%
\pgfpathlineto{\pgfqpoint{3.312292in}{1.945938in}}%
\pgfpathlineto{\pgfqpoint{3.299007in}{1.953183in}}%
\pgfpathlineto{\pgfqpoint{3.307213in}{1.949512in}}%
\pgfpathlineto{\pgfqpoint{3.315404in}{1.946214in}}%
\pgfpathlineto{\pgfqpoint{3.323579in}{1.943279in}}%
\pgfpathlineto{\pgfqpoint{3.331739in}{1.940699in}}%
\pgfpathclose%
\pgfusepath{fill}%
\end{pgfscope}%
\begin{pgfscope}%
\pgfpathrectangle{\pgfqpoint{1.254980in}{0.150000in}}{\pgfqpoint{5.490039in}{5.490039in}}%
\pgfusepath{clip}%
\pgfsetbuttcap%
\pgfsetroundjoin%
\definecolor{currentfill}{rgb}{0.185556,0.418570,0.556753}%
\pgfsetfillcolor{currentfill}%
\pgfsetfillopacity{0.700000}%
\pgfsetlinewidth{0.000000pt}%
\definecolor{currentstroke}{rgb}{0.000000,0.000000,0.000000}%
\pgfsetstrokecolor{currentstroke}%
\pgfsetdash{}{0pt}%
\pgfpathmoveto{\pgfqpoint{2.436832in}{2.533978in}}%
\pgfpathlineto{\pgfqpoint{2.450036in}{2.523871in}}%
\pgfpathlineto{\pgfqpoint{2.463241in}{2.513810in}}%
\pgfpathlineto{\pgfqpoint{2.476448in}{2.503794in}}%
\pgfpathlineto{\pgfqpoint{2.489655in}{2.493824in}}%
\pgfpathlineto{\pgfqpoint{2.480704in}{2.506274in}}%
\pgfpathlineto{\pgfqpoint{2.471723in}{2.519247in}}%
\pgfpathlineto{\pgfqpoint{2.462712in}{2.532752in}}%
\pgfpathlineto{\pgfqpoint{2.453670in}{2.546801in}}%
\pgfpathlineto{\pgfqpoint{2.440414in}{2.557141in}}%
\pgfpathlineto{\pgfqpoint{2.427158in}{2.567527in}}%
\pgfpathlineto{\pgfqpoint{2.413904in}{2.577959in}}%
\pgfpathlineto{\pgfqpoint{2.400651in}{2.588437in}}%
\pgfpathlineto{\pgfqpoint{2.409743in}{2.574012in}}%
\pgfpathlineto{\pgfqpoint{2.418804in}{2.560134in}}%
\pgfpathlineto{\pgfqpoint{2.427833in}{2.546793in}}%
\pgfpathlineto{\pgfqpoint{2.436832in}{2.533978in}}%
\pgfpathclose%
\pgfusepath{fill}%
\end{pgfscope}%
\begin{pgfscope}%
\pgfpathrectangle{\pgfqpoint{1.254980in}{0.150000in}}{\pgfqpoint{5.490039in}{5.490039in}}%
\pgfusepath{clip}%
\pgfsetbuttcap%
\pgfsetroundjoin%
\definecolor{currentfill}{rgb}{0.279574,0.170599,0.479997}%
\pgfsetfillcolor{currentfill}%
\pgfsetfillopacity{0.700000}%
\pgfsetlinewidth{0.000000pt}%
\definecolor{currentstroke}{rgb}{0.000000,0.000000,0.000000}%
\pgfsetstrokecolor{currentstroke}%
\pgfsetdash{}{0pt}%
\pgfpathmoveto{\pgfqpoint{5.409230in}{1.989360in}}%
\pgfpathlineto{\pgfqpoint{5.423008in}{1.988670in}}%
\pgfpathlineto{\pgfqpoint{5.436794in}{1.988004in}}%
\pgfpathlineto{\pgfqpoint{5.450590in}{1.987361in}}%
\pgfpathlineto{\pgfqpoint{5.464394in}{1.986742in}}%
\pgfpathlineto{\pgfqpoint{5.457121in}{1.976829in}}%
\pgfpathlineto{\pgfqpoint{5.449842in}{1.966832in}}%
\pgfpathlineto{\pgfqpoint{5.442555in}{1.956754in}}%
\pgfpathlineto{\pgfqpoint{5.435262in}{1.946596in}}%
\pgfpathlineto{\pgfqpoint{5.421451in}{1.947309in}}%
\pgfpathlineto{\pgfqpoint{5.407648in}{1.948045in}}%
\pgfpathlineto{\pgfqpoint{5.393854in}{1.948805in}}%
\pgfpathlineto{\pgfqpoint{5.380068in}{1.949589in}}%
\pgfpathlineto{\pgfqpoint{5.387369in}{1.959648in}}%
\pgfpathlineto{\pgfqpoint{5.394663in}{1.969631in}}%
\pgfpathlineto{\pgfqpoint{5.401950in}{1.979535in}}%
\pgfpathlineto{\pgfqpoint{5.409230in}{1.989360in}}%
\pgfpathclose%
\pgfusepath{fill}%
\end{pgfscope}%
\begin{pgfscope}%
\pgfpathrectangle{\pgfqpoint{1.254980in}{0.150000in}}{\pgfqpoint{5.490039in}{5.490039in}}%
\pgfusepath{clip}%
\pgfsetbuttcap%
\pgfsetroundjoin%
\definecolor{currentfill}{rgb}{0.278791,0.062145,0.386592}%
\pgfsetfillcolor{currentfill}%
\pgfsetfillopacity{0.700000}%
\pgfsetlinewidth{0.000000pt}%
\definecolor{currentstroke}{rgb}{0.000000,0.000000,0.000000}%
\pgfsetstrokecolor{currentstroke}%
\pgfsetdash{}{0pt}%
\pgfpathmoveto{\pgfqpoint{4.849182in}{1.786490in}}%
\pgfpathlineto{\pgfqpoint{4.862776in}{1.784381in}}%
\pgfpathlineto{\pgfqpoint{4.876378in}{1.782295in}}%
\pgfpathlineto{\pgfqpoint{4.889987in}{1.780233in}}%
\pgfpathlineto{\pgfqpoint{4.903604in}{1.778194in}}%
\pgfpathlineto{\pgfqpoint{4.896143in}{1.768165in}}%
\pgfpathlineto{\pgfqpoint{4.888677in}{1.758147in}}%
\pgfpathlineto{\pgfqpoint{4.881206in}{1.748143in}}%
\pgfpathlineto{\pgfqpoint{4.873731in}{1.738158in}}%
\pgfpathlineto{\pgfqpoint{4.860106in}{1.740368in}}%
\pgfpathlineto{\pgfqpoint{4.846489in}{1.742602in}}%
\pgfpathlineto{\pgfqpoint{4.832879in}{1.744860in}}%
\pgfpathlineto{\pgfqpoint{4.819277in}{1.747141in}}%
\pgfpathlineto{\pgfqpoint{4.826760in}{1.756950in}}%
\pgfpathlineto{\pgfqpoint{4.834239in}{1.766780in}}%
\pgfpathlineto{\pgfqpoint{4.841713in}{1.776628in}}%
\pgfpathlineto{\pgfqpoint{4.849182in}{1.786490in}}%
\pgfpathclose%
\pgfusepath{fill}%
\end{pgfscope}%
\begin{pgfscope}%
\pgfpathrectangle{\pgfqpoint{1.254980in}{0.150000in}}{\pgfqpoint{5.490039in}{5.490039in}}%
\pgfusepath{clip}%
\pgfsetbuttcap%
\pgfsetroundjoin%
\definecolor{currentfill}{rgb}{0.271305,0.019942,0.347269}%
\pgfsetfillcolor{currentfill}%
\pgfsetfillopacity{0.700000}%
\pgfsetlinewidth{0.000000pt}%
\definecolor{currentstroke}{rgb}{0.000000,0.000000,0.000000}%
\pgfsetstrokecolor{currentstroke}%
\pgfsetdash{}{0pt}%
\pgfpathmoveto{\pgfqpoint{4.043892in}{1.711608in}}%
\pgfpathlineto{\pgfqpoint{4.057271in}{1.706976in}}%
\pgfpathlineto{\pgfqpoint{4.070657in}{1.702368in}}%
\pgfpathlineto{\pgfqpoint{4.084048in}{1.697786in}}%
\pgfpathlineto{\pgfqpoint{4.097445in}{1.693228in}}%
\pgfpathlineto{\pgfqpoint{4.089718in}{1.688138in}}%
\pgfpathlineto{\pgfqpoint{4.081984in}{1.683241in}}%
\pgfpathlineto{\pgfqpoint{4.074244in}{1.678544in}}%
\pgfpathlineto{\pgfqpoint{4.066497in}{1.674055in}}%
\pgfpathlineto{\pgfqpoint{4.053083in}{1.678873in}}%
\pgfpathlineto{\pgfqpoint{4.039675in}{1.683716in}}%
\pgfpathlineto{\pgfqpoint{4.026273in}{1.688584in}}%
\pgfpathlineto{\pgfqpoint{4.012876in}{1.693476in}}%
\pgfpathlineto{\pgfqpoint{4.020641in}{1.697700in}}%
\pgfpathlineto{\pgfqpoint{4.028398in}{1.702134in}}%
\pgfpathlineto{\pgfqpoint{4.036148in}{1.706772in}}%
\pgfpathlineto{\pgfqpoint{4.043892in}{1.711608in}}%
\pgfpathclose%
\pgfusepath{fill}%
\end{pgfscope}%
\begin{pgfscope}%
\pgfpathrectangle{\pgfqpoint{1.254980in}{0.150000in}}{\pgfqpoint{5.490039in}{5.490039in}}%
\pgfusepath{clip}%
\pgfsetbuttcap%
\pgfsetroundjoin%
\definecolor{currentfill}{rgb}{0.269944,0.014625,0.341379}%
\pgfsetfillcolor{currentfill}%
\pgfsetfillopacity{0.700000}%
\pgfsetlinewidth{0.000000pt}%
\definecolor{currentstroke}{rgb}{0.000000,0.000000,0.000000}%
\pgfsetstrokecolor{currentstroke}%
\pgfsetdash{}{0pt}%
\pgfpathmoveto{\pgfqpoint{4.181871in}{1.698404in}}%
\pgfpathlineto{\pgfqpoint{4.195282in}{1.694216in}}%
\pgfpathlineto{\pgfqpoint{4.208700in}{1.690053in}}%
\pgfpathlineto{\pgfqpoint{4.222123in}{1.685914in}}%
\pgfpathlineto{\pgfqpoint{4.235553in}{1.681800in}}%
\pgfpathlineto{\pgfqpoint{4.227881in}{1.675513in}}%
\pgfpathlineto{\pgfqpoint{4.220204in}{1.669388in}}%
\pgfpathlineto{\pgfqpoint{4.212521in}{1.663431in}}%
\pgfpathlineto{\pgfqpoint{4.204832in}{1.657649in}}%
\pgfpathlineto{\pgfqpoint{4.191388in}{1.662011in}}%
\pgfpathlineto{\pgfqpoint{4.177950in}{1.666397in}}%
\pgfpathlineto{\pgfqpoint{4.164517in}{1.670808in}}%
\pgfpathlineto{\pgfqpoint{4.151091in}{1.675243in}}%
\pgfpathlineto{\pgfqpoint{4.158795in}{1.680773in}}%
\pgfpathlineto{\pgfqpoint{4.166493in}{1.686481in}}%
\pgfpathlineto{\pgfqpoint{4.174185in}{1.692360in}}%
\pgfpathlineto{\pgfqpoint{4.181871in}{1.698404in}}%
\pgfpathclose%
\pgfusepath{fill}%
\end{pgfscope}%
\begin{pgfscope}%
\pgfpathrectangle{\pgfqpoint{1.254980in}{0.150000in}}{\pgfqpoint{5.490039in}{5.490039in}}%
\pgfusepath{clip}%
\pgfsetbuttcap%
\pgfsetroundjoin%
\definecolor{currentfill}{rgb}{0.281887,0.150881,0.465405}%
\pgfsetfillcolor{currentfill}%
\pgfsetfillopacity{0.700000}%
\pgfsetlinewidth{0.000000pt}%
\definecolor{currentstroke}{rgb}{0.000000,0.000000,0.000000}%
\pgfsetstrokecolor{currentstroke}%
\pgfsetdash{}{0pt}%
\pgfpathmoveto{\pgfqpoint{5.325013in}{1.952960in}}%
\pgfpathlineto{\pgfqpoint{5.338764in}{1.952082in}}%
\pgfpathlineto{\pgfqpoint{5.352523in}{1.951227in}}%
\pgfpathlineto{\pgfqpoint{5.366292in}{1.950396in}}%
\pgfpathlineto{\pgfqpoint{5.380068in}{1.949589in}}%
\pgfpathlineto{\pgfqpoint{5.372761in}{1.939456in}}%
\pgfpathlineto{\pgfqpoint{5.365448in}{1.929250in}}%
\pgfpathlineto{\pgfqpoint{5.358127in}{1.918976in}}%
\pgfpathlineto{\pgfqpoint{5.350801in}{1.908634in}}%
\pgfpathlineto{\pgfqpoint{5.337017in}{1.909548in}}%
\pgfpathlineto{\pgfqpoint{5.323242in}{1.910487in}}%
\pgfpathlineto{\pgfqpoint{5.309475in}{1.911448in}}%
\pgfpathlineto{\pgfqpoint{5.295716in}{1.912434in}}%
\pgfpathlineto{\pgfqpoint{5.303050in}{1.922664in}}%
\pgfpathlineto{\pgfqpoint{5.310377in}{1.932830in}}%
\pgfpathlineto{\pgfqpoint{5.317698in}{1.942929in}}%
\pgfpathlineto{\pgfqpoint{5.325013in}{1.952960in}}%
\pgfpathclose%
\pgfusepath{fill}%
\end{pgfscope}%
\begin{pgfscope}%
\pgfpathrectangle{\pgfqpoint{1.254980in}{0.150000in}}{\pgfqpoint{5.490039in}{5.490039in}}%
\pgfusepath{clip}%
\pgfsetbuttcap%
\pgfsetroundjoin%
\definecolor{currentfill}{rgb}{0.271305,0.019942,0.347269}%
\pgfsetfillcolor{currentfill}%
\pgfsetfillopacity{0.700000}%
\pgfsetlinewidth{0.000000pt}%
\definecolor{currentstroke}{rgb}{0.000000,0.000000,0.000000}%
\pgfsetstrokecolor{currentstroke}%
\pgfsetdash{}{0pt}%
\pgfpathmoveto{\pgfqpoint{4.542328in}{1.715796in}}%
\pgfpathlineto{\pgfqpoint{4.555833in}{1.712756in}}%
\pgfpathlineto{\pgfqpoint{4.569344in}{1.709739in}}%
\pgfpathlineto{\pgfqpoint{4.582863in}{1.706746in}}%
\pgfpathlineto{\pgfqpoint{4.596389in}{1.703776in}}%
\pgfpathlineto{\pgfqpoint{4.588838in}{1.694981in}}%
\pgfpathlineto{\pgfqpoint{4.581282in}{1.686264in}}%
\pgfpathlineto{\pgfqpoint{4.573722in}{1.677630in}}%
\pgfpathlineto{\pgfqpoint{4.566157in}{1.669083in}}%
\pgfpathlineto{\pgfqpoint{4.552622in}{1.672262in}}%
\pgfpathlineto{\pgfqpoint{4.539093in}{1.675465in}}%
\pgfpathlineto{\pgfqpoint{4.525571in}{1.678691in}}%
\pgfpathlineto{\pgfqpoint{4.512056in}{1.681942in}}%
\pgfpathlineto{\pgfqpoint{4.519631in}{1.690273in}}%
\pgfpathlineto{\pgfqpoint{4.527201in}{1.698696in}}%
\pgfpathlineto{\pgfqpoint{4.534767in}{1.707206in}}%
\pgfpathlineto{\pgfqpoint{4.542328in}{1.715796in}}%
\pgfpathclose%
\pgfusepath{fill}%
\end{pgfscope}%
\begin{pgfscope}%
\pgfpathrectangle{\pgfqpoint{1.254980in}{0.150000in}}{\pgfqpoint{5.490039in}{5.490039in}}%
\pgfusepath{clip}%
\pgfsetbuttcap%
\pgfsetroundjoin%
\definecolor{currentfill}{rgb}{0.243113,0.292092,0.538516}%
\pgfsetfillcolor{currentfill}%
\pgfsetfillopacity{0.700000}%
\pgfsetlinewidth{0.000000pt}%
\definecolor{currentstroke}{rgb}{0.000000,0.000000,0.000000}%
\pgfsetstrokecolor{currentstroke}%
\pgfsetdash{}{0pt}%
\pgfpathmoveto{\pgfqpoint{2.841647in}{2.230727in}}%
\pgfpathlineto{\pgfqpoint{2.854862in}{2.222113in}}%
\pgfpathlineto{\pgfqpoint{2.868080in}{2.213536in}}%
\pgfpathlineto{\pgfqpoint{2.881301in}{2.204994in}}%
\pgfpathlineto{\pgfqpoint{2.894524in}{2.196488in}}%
\pgfpathlineto{\pgfqpoint{2.885978in}{2.204540in}}%
\pgfpathlineto{\pgfqpoint{2.877409in}{2.213045in}}%
\pgfpathlineto{\pgfqpoint{2.868818in}{2.222013in}}%
\pgfpathlineto{\pgfqpoint{2.860204in}{2.231455in}}%
\pgfpathlineto{\pgfqpoint{2.846940in}{2.240309in}}%
\pgfpathlineto{\pgfqpoint{2.833679in}{2.249200in}}%
\pgfpathlineto{\pgfqpoint{2.820420in}{2.258126in}}%
\pgfpathlineto{\pgfqpoint{2.807164in}{2.267088in}}%
\pgfpathlineto{\pgfqpoint{2.815820in}{2.257292in}}%
\pgfpathlineto{\pgfqpoint{2.824452in}{2.247973in}}%
\pgfpathlineto{\pgfqpoint{2.833061in}{2.239121in}}%
\pgfpathlineto{\pgfqpoint{2.841647in}{2.230727in}}%
\pgfpathclose%
\pgfusepath{fill}%
\end{pgfscope}%
\begin{pgfscope}%
\pgfpathrectangle{\pgfqpoint{1.254980in}{0.150000in}}{\pgfqpoint{5.490039in}{5.490039in}}%
\pgfusepath{clip}%
\pgfsetbuttcap%
\pgfsetroundjoin%
\definecolor{currentfill}{rgb}{0.274952,0.037752,0.364543}%
\pgfsetfillcolor{currentfill}%
\pgfsetfillopacity{0.700000}%
\pgfsetlinewidth{0.000000pt}%
\definecolor{currentstroke}{rgb}{0.000000,0.000000,0.000000}%
\pgfsetstrokecolor{currentstroke}%
\pgfsetdash{}{0pt}%
\pgfpathmoveto{\pgfqpoint{3.905902in}{1.733515in}}%
\pgfpathlineto{\pgfqpoint{3.919254in}{1.728422in}}%
\pgfpathlineto{\pgfqpoint{3.932612in}{1.723354in}}%
\pgfpathlineto{\pgfqpoint{3.945976in}{1.718312in}}%
\pgfpathlineto{\pgfqpoint{3.959345in}{1.713295in}}%
\pgfpathlineto{\pgfqpoint{3.951554in}{1.709557in}}%
\pgfpathlineto{\pgfqpoint{3.943756in}{1.706047in}}%
\pgfpathlineto{\pgfqpoint{3.935950in}{1.702771in}}%
\pgfpathlineto{\pgfqpoint{3.928136in}{1.699738in}}%
\pgfpathlineto{\pgfqpoint{3.914748in}{1.705029in}}%
\pgfpathlineto{\pgfqpoint{3.901365in}{1.710344in}}%
\pgfpathlineto{\pgfqpoint{3.887987in}{1.715686in}}%
\pgfpathlineto{\pgfqpoint{3.874615in}{1.721052in}}%
\pgfpathlineto{\pgfqpoint{3.882449in}{1.723807in}}%
\pgfpathlineto{\pgfqpoint{3.890275in}{1.726807in}}%
\pgfpathlineto{\pgfqpoint{3.898092in}{1.730045in}}%
\pgfpathlineto{\pgfqpoint{3.905902in}{1.733515in}}%
\pgfpathclose%
\pgfusepath{fill}%
\end{pgfscope}%
\begin{pgfscope}%
\pgfpathrectangle{\pgfqpoint{1.254980in}{0.150000in}}{\pgfqpoint{5.490039in}{5.490039in}}%
\pgfusepath{clip}%
\pgfsetbuttcap%
\pgfsetroundjoin%
\definecolor{currentfill}{rgb}{0.190631,0.407061,0.556089}%
\pgfsetfillcolor{currentfill}%
\pgfsetfillopacity{0.700000}%
\pgfsetlinewidth{0.000000pt}%
\definecolor{currentstroke}{rgb}{0.000000,0.000000,0.000000}%
\pgfsetstrokecolor{currentstroke}%
\pgfsetdash{}{0pt}%
\pgfpathmoveto{\pgfqpoint{2.489655in}{2.493824in}}%
\pgfpathlineto{\pgfqpoint{2.502864in}{2.483898in}}%
\pgfpathlineto{\pgfqpoint{2.516075in}{2.474017in}}%
\pgfpathlineto{\pgfqpoint{2.529287in}{2.464180in}}%
\pgfpathlineto{\pgfqpoint{2.542500in}{2.454386in}}%
\pgfpathlineto{\pgfqpoint{2.533596in}{2.466474in}}%
\pgfpathlineto{\pgfqpoint{2.524663in}{2.479079in}}%
\pgfpathlineto{\pgfqpoint{2.515701in}{2.492212in}}%
\pgfpathlineto{\pgfqpoint{2.506708in}{2.505885in}}%
\pgfpathlineto{\pgfqpoint{2.493447in}{2.516048in}}%
\pgfpathlineto{\pgfqpoint{2.480186in}{2.526254in}}%
\pgfpathlineto{\pgfqpoint{2.466928in}{2.536505in}}%
\pgfpathlineto{\pgfqpoint{2.453670in}{2.546801in}}%
\pgfpathlineto{\pgfqpoint{2.462712in}{2.532752in}}%
\pgfpathlineto{\pgfqpoint{2.471723in}{2.519247in}}%
\pgfpathlineto{\pgfqpoint{2.480704in}{2.506274in}}%
\pgfpathlineto{\pgfqpoint{2.489655in}{2.493824in}}%
\pgfpathclose%
\pgfusepath{fill}%
\end{pgfscope}%
\begin{pgfscope}%
\pgfpathrectangle{\pgfqpoint{1.254980in}{0.150000in}}{\pgfqpoint{5.490039in}{5.490039in}}%
\pgfusepath{clip}%
\pgfsetbuttcap%
\pgfsetroundjoin%
\definecolor{currentfill}{rgb}{0.268510,0.009605,0.335427}%
\pgfsetfillcolor{currentfill}%
\pgfsetfillopacity{0.700000}%
\pgfsetlinewidth{0.000000pt}%
\definecolor{currentstroke}{rgb}{0.000000,0.000000,0.000000}%
\pgfsetstrokecolor{currentstroke}%
\pgfsetdash{}{0pt}%
\pgfpathmoveto{\pgfqpoint{4.319908in}{1.693164in}}%
\pgfpathlineto{\pgfqpoint{4.333356in}{1.689405in}}%
\pgfpathlineto{\pgfqpoint{4.346809in}{1.685670in}}%
\pgfpathlineto{\pgfqpoint{4.360270in}{1.681959in}}%
\pgfpathlineto{\pgfqpoint{4.373737in}{1.678272in}}%
\pgfpathlineto{\pgfqpoint{4.366114in}{1.670940in}}%
\pgfpathlineto{\pgfqpoint{4.358486in}{1.663738in}}%
\pgfpathlineto{\pgfqpoint{4.350853in}{1.656674in}}%
\pgfpathlineto{\pgfqpoint{4.343214in}{1.649752in}}%
\pgfpathlineto{\pgfqpoint{4.329735in}{1.653674in}}%
\pgfpathlineto{\pgfqpoint{4.316261in}{1.657620in}}%
\pgfpathlineto{\pgfqpoint{4.302794in}{1.661590in}}%
\pgfpathlineto{\pgfqpoint{4.289333in}{1.665583in}}%
\pgfpathlineto{\pgfqpoint{4.296985in}{1.672265in}}%
\pgfpathlineto{\pgfqpoint{4.304631in}{1.679093in}}%
\pgfpathlineto{\pgfqpoint{4.312272in}{1.686061in}}%
\pgfpathlineto{\pgfqpoint{4.319908in}{1.693164in}}%
\pgfpathclose%
\pgfusepath{fill}%
\end{pgfscope}%
\begin{pgfscope}%
\pgfpathrectangle{\pgfqpoint{1.254980in}{0.150000in}}{\pgfqpoint{5.490039in}{5.490039in}}%
\pgfusepath{clip}%
\pgfsetbuttcap%
\pgfsetroundjoin%
\definecolor{currentfill}{rgb}{0.282884,0.135920,0.453427}%
\pgfsetfillcolor{currentfill}%
\pgfsetfillopacity{0.700000}%
\pgfsetlinewidth{0.000000pt}%
\definecolor{currentstroke}{rgb}{0.000000,0.000000,0.000000}%
\pgfsetstrokecolor{currentstroke}%
\pgfsetdash{}{0pt}%
\pgfpathmoveto{\pgfqpoint{5.240767in}{1.916612in}}%
\pgfpathlineto{\pgfqpoint{5.254492in}{1.915532in}}%
\pgfpathlineto{\pgfqpoint{5.268225in}{1.914475in}}%
\pgfpathlineto{\pgfqpoint{5.281966in}{1.913443in}}%
\pgfpathlineto{\pgfqpoint{5.295716in}{1.912434in}}%
\pgfpathlineto{\pgfqpoint{5.288376in}{1.902142in}}%
\pgfpathlineto{\pgfqpoint{5.281031in}{1.891792in}}%
\pgfpathlineto{\pgfqpoint{5.273679in}{1.881386in}}%
\pgfpathlineto{\pgfqpoint{5.266321in}{1.870926in}}%
\pgfpathlineto{\pgfqpoint{5.252564in}{1.872055in}}%
\pgfpathlineto{\pgfqpoint{5.238816in}{1.873208in}}%
\pgfpathlineto{\pgfqpoint{5.225076in}{1.874385in}}%
\pgfpathlineto{\pgfqpoint{5.211344in}{1.875585in}}%
\pgfpathlineto{\pgfqpoint{5.218709in}{1.885920in}}%
\pgfpathlineto{\pgfqpoint{5.226067in}{1.896204in}}%
\pgfpathlineto{\pgfqpoint{5.233420in}{1.906436in}}%
\pgfpathlineto{\pgfqpoint{5.240767in}{1.916612in}}%
\pgfpathclose%
\pgfusepath{fill}%
\end{pgfscope}%
\begin{pgfscope}%
\pgfpathrectangle{\pgfqpoint{1.254980in}{0.150000in}}{\pgfqpoint{5.490039in}{5.490039in}}%
\pgfusepath{clip}%
\pgfsetbuttcap%
\pgfsetroundjoin%
\definecolor{currentfill}{rgb}{0.273006,0.204520,0.501721}%
\pgfsetfillcolor{currentfill}%
\pgfsetfillopacity{0.700000}%
\pgfsetlinewidth{0.000000pt}%
\definecolor{currentstroke}{rgb}{0.000000,0.000000,0.000000}%
\pgfsetstrokecolor{currentstroke}%
\pgfsetdash{}{0pt}%
\pgfpathmoveto{\pgfqpoint{3.139887in}{2.042462in}}%
\pgfpathlineto{\pgfqpoint{3.153127in}{2.034853in}}%
\pgfpathlineto{\pgfqpoint{3.166371in}{2.027275in}}%
\pgfpathlineto{\pgfqpoint{3.179618in}{2.019729in}}%
\pgfpathlineto{\pgfqpoint{3.192869in}{2.012214in}}%
\pgfpathlineto{\pgfqpoint{3.184582in}{2.016915in}}%
\pgfpathlineto{\pgfqpoint{3.176278in}{2.022013in}}%
\pgfpathlineto{\pgfqpoint{3.167956in}{2.027518in}}%
\pgfpathlineto{\pgfqpoint{3.159616in}{2.033438in}}%
\pgfpathlineto{\pgfqpoint{3.146331in}{2.041284in}}%
\pgfpathlineto{\pgfqpoint{3.133049in}{2.049162in}}%
\pgfpathlineto{\pgfqpoint{3.119771in}{2.057070in}}%
\pgfpathlineto{\pgfqpoint{3.106496in}{2.065010in}}%
\pgfpathlineto{\pgfqpoint{3.114871in}{2.058754in}}%
\pgfpathlineto{\pgfqpoint{3.123228in}{2.052916in}}%
\pgfpathlineto{\pgfqpoint{3.131566in}{2.047488in}}%
\pgfpathlineto{\pgfqpoint{3.139887in}{2.042462in}}%
\pgfpathclose%
\pgfusepath{fill}%
\end{pgfscope}%
\begin{pgfscope}%
\pgfpathrectangle{\pgfqpoint{1.254980in}{0.150000in}}{\pgfqpoint{5.490039in}{5.490039in}}%
\pgfusepath{clip}%
\pgfsetbuttcap%
\pgfsetroundjoin%
\definecolor{currentfill}{rgb}{0.282327,0.094955,0.417331}%
\pgfsetfillcolor{currentfill}%
\pgfsetfillopacity{0.700000}%
\pgfsetlinewidth{0.000000pt}%
\definecolor{currentstroke}{rgb}{0.000000,0.000000,0.000000}%
\pgfsetstrokecolor{currentstroke}%
\pgfsetdash{}{0pt}%
\pgfpathmoveto{\pgfqpoint{3.576364in}{1.831139in}}%
\pgfpathlineto{\pgfqpoint{3.589659in}{1.824962in}}%
\pgfpathlineto{\pgfqpoint{3.602960in}{1.818812in}}%
\pgfpathlineto{\pgfqpoint{3.616265in}{1.812689in}}%
\pgfpathlineto{\pgfqpoint{3.629575in}{1.806592in}}%
\pgfpathlineto{\pgfqpoint{3.621602in}{1.806357in}}%
\pgfpathlineto{\pgfqpoint{3.613618in}{1.806427in}}%
\pgfpathlineto{\pgfqpoint{3.605622in}{1.806808in}}%
\pgfpathlineto{\pgfqpoint{3.597615in}{1.807511in}}%
\pgfpathlineto{\pgfqpoint{3.584279in}{1.813908in}}%
\pgfpathlineto{\pgfqpoint{3.570949in}{1.820331in}}%
\pgfpathlineto{\pgfqpoint{3.557622in}{1.826782in}}%
\pgfpathlineto{\pgfqpoint{3.544300in}{1.833260in}}%
\pgfpathlineto{\pgfqpoint{3.552334in}{1.832252in}}%
\pgfpathlineto{\pgfqpoint{3.560356in}{1.831568in}}%
\pgfpathlineto{\pgfqpoint{3.568366in}{1.831200in}}%
\pgfpathlineto{\pgfqpoint{3.576364in}{1.831139in}}%
\pgfpathclose%
\pgfusepath{fill}%
\end{pgfscope}%
\begin{pgfscope}%
\pgfpathrectangle{\pgfqpoint{1.254980in}{0.150000in}}{\pgfqpoint{5.490039in}{5.490039in}}%
\pgfusepath{clip}%
\pgfsetbuttcap%
\pgfsetroundjoin%
\definecolor{currentfill}{rgb}{0.277018,0.050344,0.375715}%
\pgfsetfillcolor{currentfill}%
\pgfsetfillopacity{0.700000}%
\pgfsetlinewidth{0.000000pt}%
\definecolor{currentstroke}{rgb}{0.000000,0.000000,0.000000}%
\pgfsetstrokecolor{currentstroke}%
\pgfsetdash{}{0pt}%
\pgfpathmoveto{\pgfqpoint{4.764943in}{1.756501in}}%
\pgfpathlineto{\pgfqpoint{4.778516in}{1.754126in}}%
\pgfpathlineto{\pgfqpoint{4.792095in}{1.751774in}}%
\pgfpathlineto{\pgfqpoint{4.805682in}{1.749446in}}%
\pgfpathlineto{\pgfqpoint{4.819277in}{1.747141in}}%
\pgfpathlineto{\pgfqpoint{4.811789in}{1.737359in}}%
\pgfpathlineto{\pgfqpoint{4.804297in}{1.727606in}}%
\pgfpathlineto{\pgfqpoint{4.796800in}{1.717889in}}%
\pgfpathlineto{\pgfqpoint{4.789299in}{1.708211in}}%
\pgfpathlineto{\pgfqpoint{4.775696in}{1.710700in}}%
\pgfpathlineto{\pgfqpoint{4.762101in}{1.713213in}}%
\pgfpathlineto{\pgfqpoint{4.748512in}{1.715749in}}%
\pgfpathlineto{\pgfqpoint{4.734932in}{1.718309in}}%
\pgfpathlineto{\pgfqpoint{4.742441in}{1.727798in}}%
\pgfpathlineto{\pgfqpoint{4.749947in}{1.737329in}}%
\pgfpathlineto{\pgfqpoint{4.757447in}{1.746898in}}%
\pgfpathlineto{\pgfqpoint{4.764943in}{1.756501in}}%
\pgfpathclose%
\pgfusepath{fill}%
\end{pgfscope}%
\begin{pgfscope}%
\pgfpathrectangle{\pgfqpoint{1.254980in}{0.150000in}}{\pgfqpoint{5.490039in}{5.490039in}}%
\pgfusepath{clip}%
\pgfsetbuttcap%
\pgfsetroundjoin%
\definecolor{currentfill}{rgb}{0.283197,0.115680,0.436115}%
\pgfsetfillcolor{currentfill}%
\pgfsetfillopacity{0.700000}%
\pgfsetlinewidth{0.000000pt}%
\definecolor{currentstroke}{rgb}{0.000000,0.000000,0.000000}%
\pgfsetstrokecolor{currentstroke}%
\pgfsetdash{}{0pt}%
\pgfpathmoveto{\pgfqpoint{5.156500in}{1.880621in}}%
\pgfpathlineto{\pgfqpoint{5.170199in}{1.879327in}}%
\pgfpathlineto{\pgfqpoint{5.183905in}{1.878056in}}%
\pgfpathlineto{\pgfqpoint{5.197621in}{1.876808in}}%
\pgfpathlineto{\pgfqpoint{5.211344in}{1.875585in}}%
\pgfpathlineto{\pgfqpoint{5.203974in}{1.865203in}}%
\pgfpathlineto{\pgfqpoint{5.196597in}{1.854776in}}%
\pgfpathlineto{\pgfqpoint{5.189216in}{1.844308in}}%
\pgfpathlineto{\pgfqpoint{5.181829in}{1.833802in}}%
\pgfpathlineto{\pgfqpoint{5.168098in}{1.835159in}}%
\pgfpathlineto{\pgfqpoint{5.154376in}{1.836539in}}%
\pgfpathlineto{\pgfqpoint{5.140662in}{1.837943in}}%
\pgfpathlineto{\pgfqpoint{5.126957in}{1.839371in}}%
\pgfpathlineto{\pgfqpoint{5.134351in}{1.849739in}}%
\pgfpathlineto{\pgfqpoint{5.141739in}{1.860072in}}%
\pgfpathlineto{\pgfqpoint{5.149122in}{1.870367in}}%
\pgfpathlineto{\pgfqpoint{5.156500in}{1.880621in}}%
\pgfpathclose%
\pgfusepath{fill}%
\end{pgfscope}%
\begin{pgfscope}%
\pgfpathrectangle{\pgfqpoint{1.254980in}{0.150000in}}{\pgfqpoint{5.490039in}{5.490039in}}%
\pgfusepath{clip}%
\pgfsetbuttcap%
\pgfsetroundjoin%
\definecolor{currentfill}{rgb}{0.282623,0.140926,0.457517}%
\pgfsetfillcolor{currentfill}%
\pgfsetfillopacity{0.700000}%
\pgfsetlinewidth{0.000000pt}%
\definecolor{currentstroke}{rgb}{0.000000,0.000000,0.000000}%
\pgfsetstrokecolor{currentstroke}%
\pgfsetdash{}{0pt}%
\pgfpathmoveto{\pgfqpoint{3.384779in}{1.913157in}}%
\pgfpathlineto{\pgfqpoint{3.398049in}{1.906344in}}%
\pgfpathlineto{\pgfqpoint{3.411323in}{1.899559in}}%
\pgfpathlineto{\pgfqpoint{3.424602in}{1.892803in}}%
\pgfpathlineto{\pgfqpoint{3.437884in}{1.886076in}}%
\pgfpathlineto{\pgfqpoint{3.429782in}{1.888036in}}%
\pgfpathlineto{\pgfqpoint{3.421666in}{1.890343in}}%
\pgfpathlineto{\pgfqpoint{3.413537in}{1.893007in}}%
\pgfpathlineto{\pgfqpoint{3.405392in}{1.896036in}}%
\pgfpathlineto{\pgfqpoint{3.392080in}{1.903079in}}%
\pgfpathlineto{\pgfqpoint{3.378772in}{1.910150in}}%
\pgfpathlineto{\pgfqpoint{3.365468in}{1.917250in}}%
\pgfpathlineto{\pgfqpoint{3.352168in}{1.924379in}}%
\pgfpathlineto{\pgfqpoint{3.360343in}{1.921029in}}%
\pgfpathlineto{\pgfqpoint{3.368503in}{1.918048in}}%
\pgfpathlineto{\pgfqpoint{3.376648in}{1.915427in}}%
\pgfpathlineto{\pgfqpoint{3.384779in}{1.913157in}}%
\pgfpathclose%
\pgfusepath{fill}%
\end{pgfscope}%
\begin{pgfscope}%
\pgfpathrectangle{\pgfqpoint{1.254980in}{0.150000in}}{\pgfqpoint{5.490039in}{5.490039in}}%
\pgfusepath{clip}%
\pgfsetbuttcap%
\pgfsetroundjoin%
\definecolor{currentfill}{rgb}{0.278791,0.062145,0.386592}%
\pgfsetfillcolor{currentfill}%
\pgfsetfillopacity{0.700000}%
\pgfsetlinewidth{0.000000pt}%
\definecolor{currentstroke}{rgb}{0.000000,0.000000,0.000000}%
\pgfsetstrokecolor{currentstroke}%
\pgfsetdash{}{0pt}%
\pgfpathmoveto{\pgfqpoint{3.767825in}{1.764903in}}%
\pgfpathlineto{\pgfqpoint{3.781155in}{1.759332in}}%
\pgfpathlineto{\pgfqpoint{3.794491in}{1.753786in}}%
\pgfpathlineto{\pgfqpoint{3.807832in}{1.748266in}}%
\pgfpathlineto{\pgfqpoint{3.821178in}{1.742772in}}%
\pgfpathlineto{\pgfqpoint{3.813315in}{1.740552in}}%
\pgfpathlineto{\pgfqpoint{3.805442in}{1.738595in}}%
\pgfpathlineto{\pgfqpoint{3.797560in}{1.736908in}}%
\pgfpathlineto{\pgfqpoint{3.789668in}{1.735501in}}%
\pgfpathlineto{\pgfqpoint{3.776300in}{1.741282in}}%
\pgfpathlineto{\pgfqpoint{3.762937in}{1.747088in}}%
\pgfpathlineto{\pgfqpoint{3.749578in}{1.752921in}}%
\pgfpathlineto{\pgfqpoint{3.736225in}{1.758779in}}%
\pgfpathlineto{\pgfqpoint{3.744140in}{1.759895in}}%
\pgfpathlineto{\pgfqpoint{3.752044in}{1.761293in}}%
\pgfpathlineto{\pgfqpoint{3.759939in}{1.762965in}}%
\pgfpathlineto{\pgfqpoint{3.767825in}{1.764903in}}%
\pgfpathclose%
\pgfusepath{fill}%
\end{pgfscope}%
\begin{pgfscope}%
\pgfpathrectangle{\pgfqpoint{1.254980in}{0.150000in}}{\pgfqpoint{5.490039in}{5.490039in}}%
\pgfusepath{clip}%
\pgfsetbuttcap%
\pgfsetroundjoin%
\definecolor{currentfill}{rgb}{0.269944,0.014625,0.341379}%
\pgfsetfillcolor{currentfill}%
\pgfsetfillopacity{0.700000}%
\pgfsetlinewidth{0.000000pt}%
\definecolor{currentstroke}{rgb}{0.000000,0.000000,0.000000}%
\pgfsetstrokecolor{currentstroke}%
\pgfsetdash{}{0pt}%
\pgfpathmoveto{\pgfqpoint{4.458063in}{1.695179in}}%
\pgfpathlineto{\pgfqpoint{4.471551in}{1.691834in}}%
\pgfpathlineto{\pgfqpoint{4.485046in}{1.688513in}}%
\pgfpathlineto{\pgfqpoint{4.498547in}{1.685215in}}%
\pgfpathlineto{\pgfqpoint{4.512056in}{1.681942in}}%
\pgfpathlineto{\pgfqpoint{4.504476in}{1.673707in}}%
\pgfpathlineto{\pgfqpoint{4.496891in}{1.665574in}}%
\pgfpathlineto{\pgfqpoint{4.489302in}{1.657549in}}%
\pgfpathlineto{\pgfqpoint{4.481708in}{1.649636in}}%
\pgfpathlineto{\pgfqpoint{4.468188in}{1.653133in}}%
\pgfpathlineto{\pgfqpoint{4.454675in}{1.656653in}}%
\pgfpathlineto{\pgfqpoint{4.441169in}{1.660196in}}%
\pgfpathlineto{\pgfqpoint{4.427670in}{1.663764in}}%
\pgfpathlineto{\pgfqpoint{4.435275in}{1.671449in}}%
\pgfpathlineto{\pgfqpoint{4.442876in}{1.679250in}}%
\pgfpathlineto{\pgfqpoint{4.450472in}{1.687162in}}%
\pgfpathlineto{\pgfqpoint{4.458063in}{1.695179in}}%
\pgfpathclose%
\pgfusepath{fill}%
\end{pgfscope}%
\begin{pgfscope}%
\pgfpathrectangle{\pgfqpoint{1.254980in}{0.150000in}}{\pgfqpoint{5.490039in}{5.490039in}}%
\pgfusepath{clip}%
\pgfsetbuttcap%
\pgfsetroundjoin%
\definecolor{currentfill}{rgb}{0.282656,0.100196,0.422160}%
\pgfsetfillcolor{currentfill}%
\pgfsetfillopacity{0.700000}%
\pgfsetlinewidth{0.000000pt}%
\definecolor{currentstroke}{rgb}{0.000000,0.000000,0.000000}%
\pgfsetstrokecolor{currentstroke}%
\pgfsetdash{}{0pt}%
\pgfpathmoveto{\pgfqpoint{5.072215in}{1.845317in}}%
\pgfpathlineto{\pgfqpoint{5.085889in}{1.843795in}}%
\pgfpathlineto{\pgfqpoint{5.099570in}{1.842297in}}%
\pgfpathlineto{\pgfqpoint{5.113259in}{1.840822in}}%
\pgfpathlineto{\pgfqpoint{5.126957in}{1.839371in}}%
\pgfpathlineto{\pgfqpoint{5.119557in}{1.828971in}}%
\pgfpathlineto{\pgfqpoint{5.112153in}{1.818542in}}%
\pgfpathlineto{\pgfqpoint{5.104743in}{1.808088in}}%
\pgfpathlineto{\pgfqpoint{5.097328in}{1.797612in}}%
\pgfpathlineto{\pgfqpoint{5.083623in}{1.799210in}}%
\pgfpathlineto{\pgfqpoint{5.069927in}{1.800831in}}%
\pgfpathlineto{\pgfqpoint{5.056239in}{1.802475in}}%
\pgfpathlineto{\pgfqpoint{5.042558in}{1.804143in}}%
\pgfpathlineto{\pgfqpoint{5.049980in}{1.814468in}}%
\pgfpathlineto{\pgfqpoint{5.057397in}{1.824774in}}%
\pgfpathlineto{\pgfqpoint{5.064809in}{1.835058in}}%
\pgfpathlineto{\pgfqpoint{5.072215in}{1.845317in}}%
\pgfpathclose%
\pgfusepath{fill}%
\end{pgfscope}%
\begin{pgfscope}%
\pgfpathrectangle{\pgfqpoint{1.254980in}{0.150000in}}{\pgfqpoint{5.490039in}{5.490039in}}%
\pgfusepath{clip}%
\pgfsetbuttcap%
\pgfsetroundjoin%
\definecolor{currentfill}{rgb}{0.197636,0.391528,0.554969}%
\pgfsetfillcolor{currentfill}%
\pgfsetfillopacity{0.700000}%
\pgfsetlinewidth{0.000000pt}%
\definecolor{currentstroke}{rgb}{0.000000,0.000000,0.000000}%
\pgfsetstrokecolor{currentstroke}%
\pgfsetdash{}{0pt}%
\pgfpathmoveto{\pgfqpoint{2.542500in}{2.454386in}}%
\pgfpathlineto{\pgfqpoint{2.555716in}{2.444636in}}%
\pgfpathlineto{\pgfqpoint{2.568932in}{2.434928in}}%
\pgfpathlineto{\pgfqpoint{2.582151in}{2.425263in}}%
\pgfpathlineto{\pgfqpoint{2.595371in}{2.415640in}}%
\pgfpathlineto{\pgfqpoint{2.586513in}{2.427365in}}%
\pgfpathlineto{\pgfqpoint{2.577627in}{2.439603in}}%
\pgfpathlineto{\pgfqpoint{2.568712in}{2.452366in}}%
\pgfpathlineto{\pgfqpoint{2.559768in}{2.465664in}}%
\pgfpathlineto{\pgfqpoint{2.546501in}{2.475656in}}%
\pgfpathlineto{\pgfqpoint{2.533235in}{2.485689in}}%
\pgfpathlineto{\pgfqpoint{2.519971in}{2.495766in}}%
\pgfpathlineto{\pgfqpoint{2.506708in}{2.505885in}}%
\pgfpathlineto{\pgfqpoint{2.515701in}{2.492212in}}%
\pgfpathlineto{\pgfqpoint{2.524663in}{2.479079in}}%
\pgfpathlineto{\pgfqpoint{2.533596in}{2.466474in}}%
\pgfpathlineto{\pgfqpoint{2.542500in}{2.454386in}}%
\pgfpathclose%
\pgfusepath{fill}%
\end{pgfscope}%
\begin{pgfscope}%
\pgfpathrectangle{\pgfqpoint{1.254980in}{0.150000in}}{\pgfqpoint{5.490039in}{5.490039in}}%
\pgfusepath{clip}%
\pgfsetbuttcap%
\pgfsetroundjoin%
\definecolor{currentfill}{rgb}{0.246811,0.283237,0.535941}%
\pgfsetfillcolor{currentfill}%
\pgfsetfillopacity{0.700000}%
\pgfsetlinewidth{0.000000pt}%
\definecolor{currentstroke}{rgb}{0.000000,0.000000,0.000000}%
\pgfsetstrokecolor{currentstroke}%
\pgfsetdash{}{0pt}%
\pgfpathmoveto{\pgfqpoint{2.894524in}{2.196488in}}%
\pgfpathlineto{\pgfqpoint{2.907750in}{2.188017in}}%
\pgfpathlineto{\pgfqpoint{2.920979in}{2.179581in}}%
\pgfpathlineto{\pgfqpoint{2.934210in}{2.171180in}}%
\pgfpathlineto{\pgfqpoint{2.947445in}{2.162813in}}%
\pgfpathlineto{\pgfqpoint{2.938937in}{2.170522in}}%
\pgfpathlineto{\pgfqpoint{2.930409in}{2.178681in}}%
\pgfpathlineto{\pgfqpoint{2.921858in}{2.187299in}}%
\pgfpathlineto{\pgfqpoint{2.913285in}{2.196387in}}%
\pgfpathlineto{\pgfqpoint{2.900010in}{2.205102in}}%
\pgfpathlineto{\pgfqpoint{2.886739in}{2.213851in}}%
\pgfpathlineto{\pgfqpoint{2.873470in}{2.222635in}}%
\pgfpathlineto{\pgfqpoint{2.860204in}{2.231455in}}%
\pgfpathlineto{\pgfqpoint{2.868818in}{2.222013in}}%
\pgfpathlineto{\pgfqpoint{2.877409in}{2.213045in}}%
\pgfpathlineto{\pgfqpoint{2.885978in}{2.204540in}}%
\pgfpathlineto{\pgfqpoint{2.894524in}{2.196488in}}%
\pgfpathclose%
\pgfusepath{fill}%
\end{pgfscope}%
\begin{pgfscope}%
\pgfpathrectangle{\pgfqpoint{1.254980in}{0.150000in}}{\pgfqpoint{5.490039in}{5.490039in}}%
\pgfusepath{clip}%
\pgfsetbuttcap%
\pgfsetroundjoin%
\definecolor{currentfill}{rgb}{0.273809,0.031497,0.358853}%
\pgfsetfillcolor{currentfill}%
\pgfsetfillopacity{0.700000}%
\pgfsetlinewidth{0.000000pt}%
\definecolor{currentstroke}{rgb}{0.000000,0.000000,0.000000}%
\pgfsetstrokecolor{currentstroke}%
\pgfsetdash{}{0pt}%
\pgfpathmoveto{\pgfqpoint{4.680681in}{1.728784in}}%
\pgfpathlineto{\pgfqpoint{4.694233in}{1.726129in}}%
\pgfpathlineto{\pgfqpoint{4.707792in}{1.723499in}}%
\pgfpathlineto{\pgfqpoint{4.721358in}{1.720892in}}%
\pgfpathlineto{\pgfqpoint{4.734932in}{1.718309in}}%
\pgfpathlineto{\pgfqpoint{4.727417in}{1.708867in}}%
\pgfpathlineto{\pgfqpoint{4.719899in}{1.699478in}}%
\pgfpathlineto{\pgfqpoint{4.712376in}{1.690144in}}%
\pgfpathlineto{\pgfqpoint{4.704848in}{1.680872in}}%
\pgfpathlineto{\pgfqpoint{4.691266in}{1.683653in}}%
\pgfpathlineto{\pgfqpoint{4.677690in}{1.686457in}}%
\pgfpathlineto{\pgfqpoint{4.664122in}{1.689285in}}%
\pgfpathlineto{\pgfqpoint{4.650561in}{1.692136in}}%
\pgfpathlineto{\pgfqpoint{4.658098in}{1.701206in}}%
\pgfpathlineto{\pgfqpoint{4.665630in}{1.710340in}}%
\pgfpathlineto{\pgfqpoint{4.673158in}{1.719534in}}%
\pgfpathlineto{\pgfqpoint{4.680681in}{1.728784in}}%
\pgfpathclose%
\pgfusepath{fill}%
\end{pgfscope}%
\begin{pgfscope}%
\pgfpathrectangle{\pgfqpoint{1.254980in}{0.150000in}}{\pgfqpoint{5.490039in}{5.490039in}}%
\pgfusepath{clip}%
\pgfsetbuttcap%
\pgfsetroundjoin%
\definecolor{currentfill}{rgb}{0.275191,0.194905,0.496005}%
\pgfsetfillcolor{currentfill}%
\pgfsetfillopacity{0.700000}%
\pgfsetlinewidth{0.000000pt}%
\definecolor{currentstroke}{rgb}{0.000000,0.000000,0.000000}%
\pgfsetstrokecolor{currentstroke}%
\pgfsetdash{}{0pt}%
\pgfpathmoveto{\pgfqpoint{5.548683in}{2.023609in}}%
\pgfpathlineto{\pgfqpoint{5.562524in}{2.023190in}}%
\pgfpathlineto{\pgfqpoint{5.576373in}{2.022794in}}%
\pgfpathlineto{\pgfqpoint{5.590231in}{2.022422in}}%
\pgfpathlineto{\pgfqpoint{5.583002in}{2.012728in}}%
\pgfpathlineto{\pgfqpoint{5.575765in}{2.002938in}}%
\pgfpathlineto{\pgfqpoint{5.568521in}{1.993053in}}%
\pgfpathlineto{\pgfqpoint{5.561269in}{1.983075in}}%
\pgfpathlineto{\pgfqpoint{5.547403in}{1.983527in}}%
\pgfpathlineto{\pgfqpoint{5.533546in}{1.984004in}}%
\pgfpathlineto{\pgfqpoint{5.519698in}{1.984504in}}%
\pgfpathlineto{\pgfqpoint{5.526955in}{1.994418in}}%
\pgfpathlineto{\pgfqpoint{5.534205in}{2.004241in}}%
\pgfpathlineto{\pgfqpoint{5.541448in}{2.013972in}}%
\pgfpathlineto{\pgfqpoint{5.548683in}{2.023609in}}%
\pgfpathclose%
\pgfusepath{fill}%
\end{pgfscope}%
\begin{pgfscope}%
\pgfpathrectangle{\pgfqpoint{1.254980in}{0.150000in}}{\pgfqpoint{5.490039in}{5.490039in}}%
\pgfusepath{clip}%
\pgfsetbuttcap%
\pgfsetroundjoin%
\definecolor{currentfill}{rgb}{0.281446,0.084320,0.407414}%
\pgfsetfillcolor{currentfill}%
\pgfsetfillopacity{0.700000}%
\pgfsetlinewidth{0.000000pt}%
\definecolor{currentstroke}{rgb}{0.000000,0.000000,0.000000}%
\pgfsetstrokecolor{currentstroke}%
\pgfsetdash{}{0pt}%
\pgfpathmoveto{\pgfqpoint{4.987917in}{1.811051in}}%
\pgfpathlineto{\pgfqpoint{5.001565in}{1.809288in}}%
\pgfpathlineto{\pgfqpoint{5.015222in}{1.807550in}}%
\pgfpathlineto{\pgfqpoint{5.028886in}{1.805835in}}%
\pgfpathlineto{\pgfqpoint{5.042558in}{1.804143in}}%
\pgfpathlineto{\pgfqpoint{5.035131in}{1.793803in}}%
\pgfpathlineto{\pgfqpoint{5.027699in}{1.783452in}}%
\pgfpathlineto{\pgfqpoint{5.020262in}{1.773093in}}%
\pgfpathlineto{\pgfqpoint{5.012820in}{1.762730in}}%
\pgfpathlineto{\pgfqpoint{4.999141in}{1.764581in}}%
\pgfpathlineto{\pgfqpoint{4.985469in}{1.766455in}}%
\pgfpathlineto{\pgfqpoint{4.971806in}{1.768353in}}%
\pgfpathlineto{\pgfqpoint{4.958150in}{1.770274in}}%
\pgfpathlineto{\pgfqpoint{4.965599in}{1.780473in}}%
\pgfpathlineto{\pgfqpoint{4.973043in}{1.790671in}}%
\pgfpathlineto{\pgfqpoint{4.980482in}{1.800865in}}%
\pgfpathlineto{\pgfqpoint{4.987917in}{1.811051in}}%
\pgfpathclose%
\pgfusepath{fill}%
\end{pgfscope}%
\begin{pgfscope}%
\pgfpathrectangle{\pgfqpoint{1.254980in}{0.150000in}}{\pgfqpoint{5.490039in}{5.490039in}}%
\pgfusepath{clip}%
\pgfsetbuttcap%
\pgfsetroundjoin%
\definecolor{currentfill}{rgb}{0.271305,0.019942,0.347269}%
\pgfsetfillcolor{currentfill}%
\pgfsetfillopacity{0.700000}%
\pgfsetlinewidth{0.000000pt}%
\definecolor{currentstroke}{rgb}{0.000000,0.000000,0.000000}%
\pgfsetstrokecolor{currentstroke}%
\pgfsetdash{}{0pt}%
\pgfpathmoveto{\pgfqpoint{4.097445in}{1.693228in}}%
\pgfpathlineto{\pgfqpoint{4.110847in}{1.688695in}}%
\pgfpathlineto{\pgfqpoint{4.124256in}{1.684187in}}%
\pgfpathlineto{\pgfqpoint{4.137670in}{1.679703in}}%
\pgfpathlineto{\pgfqpoint{4.151091in}{1.675243in}}%
\pgfpathlineto{\pgfqpoint{4.143380in}{1.669897in}}%
\pgfpathlineto{\pgfqpoint{4.135663in}{1.664741in}}%
\pgfpathlineto{\pgfqpoint{4.127940in}{1.659782in}}%
\pgfpathlineto{\pgfqpoint{4.120210in}{1.655028in}}%
\pgfpathlineto{\pgfqpoint{4.106773in}{1.659748in}}%
\pgfpathlineto{\pgfqpoint{4.093342in}{1.664493in}}%
\pgfpathlineto{\pgfqpoint{4.079917in}{1.669262in}}%
\pgfpathlineto{\pgfqpoint{4.066497in}{1.674055in}}%
\pgfpathlineto{\pgfqpoint{4.074244in}{1.678544in}}%
\pgfpathlineto{\pgfqpoint{4.081984in}{1.683241in}}%
\pgfpathlineto{\pgfqpoint{4.089718in}{1.688138in}}%
\pgfpathlineto{\pgfqpoint{4.097445in}{1.693228in}}%
\pgfpathclose%
\pgfusepath{fill}%
\end{pgfscope}%
\begin{pgfscope}%
\pgfpathrectangle{\pgfqpoint{1.254980in}{0.150000in}}{\pgfqpoint{5.490039in}{5.490039in}}%
\pgfusepath{clip}%
\pgfsetbuttcap%
\pgfsetroundjoin%
\definecolor{currentfill}{rgb}{0.275191,0.194905,0.496005}%
\pgfsetfillcolor{currentfill}%
\pgfsetfillopacity{0.700000}%
\pgfsetlinewidth{0.000000pt}%
\definecolor{currentstroke}{rgb}{0.000000,0.000000,0.000000}%
\pgfsetstrokecolor{currentstroke}%
\pgfsetdash{}{0pt}%
\pgfpathmoveto{\pgfqpoint{3.192869in}{2.012214in}}%
\pgfpathlineto{\pgfqpoint{3.206123in}{2.004729in}}%
\pgfpathlineto{\pgfqpoint{3.219381in}{1.997275in}}%
\pgfpathlineto{\pgfqpoint{3.232643in}{1.989851in}}%
\pgfpathlineto{\pgfqpoint{3.245908in}{1.982457in}}%
\pgfpathlineto{\pgfqpoint{3.237654in}{1.986834in}}%
\pgfpathlineto{\pgfqpoint{3.229384in}{1.991603in}}%
\pgfpathlineto{\pgfqpoint{3.221096in}{1.996775in}}%
\pgfpathlineto{\pgfqpoint{3.212791in}{2.002359in}}%
\pgfpathlineto{\pgfqpoint{3.199492in}{2.010083in}}%
\pgfpathlineto{\pgfqpoint{3.186196in}{2.017838in}}%
\pgfpathlineto{\pgfqpoint{3.172904in}{2.025622in}}%
\pgfpathlineto{\pgfqpoint{3.159616in}{2.033438in}}%
\pgfpathlineto{\pgfqpoint{3.167956in}{2.027518in}}%
\pgfpathlineto{\pgfqpoint{3.176278in}{2.022013in}}%
\pgfpathlineto{\pgfqpoint{3.184582in}{2.016915in}}%
\pgfpathlineto{\pgfqpoint{3.192869in}{2.012214in}}%
\pgfpathclose%
\pgfusepath{fill}%
\end{pgfscope}%
\begin{pgfscope}%
\pgfpathrectangle{\pgfqpoint{1.254980in}{0.150000in}}{\pgfqpoint{5.490039in}{5.490039in}}%
\pgfusepath{clip}%
\pgfsetbuttcap%
\pgfsetroundjoin%
\definecolor{currentfill}{rgb}{0.269944,0.014625,0.341379}%
\pgfsetfillcolor{currentfill}%
\pgfsetfillopacity{0.700000}%
\pgfsetlinewidth{0.000000pt}%
\definecolor{currentstroke}{rgb}{0.000000,0.000000,0.000000}%
\pgfsetstrokecolor{currentstroke}%
\pgfsetdash{}{0pt}%
\pgfpathmoveto{\pgfqpoint{4.235553in}{1.681800in}}%
\pgfpathlineto{\pgfqpoint{4.248989in}{1.677709in}}%
\pgfpathlineto{\pgfqpoint{4.262431in}{1.673643in}}%
\pgfpathlineto{\pgfqpoint{4.275879in}{1.669601in}}%
\pgfpathlineto{\pgfqpoint{4.289333in}{1.665583in}}%
\pgfpathlineto{\pgfqpoint{4.281676in}{1.659054in}}%
\pgfpathlineto{\pgfqpoint{4.274013in}{1.652683in}}%
\pgfpathlineto{\pgfqpoint{4.266345in}{1.646477in}}%
\pgfpathlineto{\pgfqpoint{4.258671in}{1.640441in}}%
\pgfpathlineto{\pgfqpoint{4.245202in}{1.644707in}}%
\pgfpathlineto{\pgfqpoint{4.231739in}{1.648997in}}%
\pgfpathlineto{\pgfqpoint{4.218283in}{1.653310in}}%
\pgfpathlineto{\pgfqpoint{4.204832in}{1.657649in}}%
\pgfpathlineto{\pgfqpoint{4.212521in}{1.663431in}}%
\pgfpathlineto{\pgfqpoint{4.220204in}{1.669388in}}%
\pgfpathlineto{\pgfqpoint{4.227881in}{1.675513in}}%
\pgfpathlineto{\pgfqpoint{4.235553in}{1.681800in}}%
\pgfpathclose%
\pgfusepath{fill}%
\end{pgfscope}%
\begin{pgfscope}%
\pgfpathrectangle{\pgfqpoint{1.254980in}{0.150000in}}{\pgfqpoint{5.490039in}{5.490039in}}%
\pgfusepath{clip}%
\pgfsetbuttcap%
\pgfsetroundjoin%
\definecolor{currentfill}{rgb}{0.273809,0.031497,0.358853}%
\pgfsetfillcolor{currentfill}%
\pgfsetfillopacity{0.700000}%
\pgfsetlinewidth{0.000000pt}%
\definecolor{currentstroke}{rgb}{0.000000,0.000000,0.000000}%
\pgfsetstrokecolor{currentstroke}%
\pgfsetdash{}{0pt}%
\pgfpathmoveto{\pgfqpoint{3.959345in}{1.713295in}}%
\pgfpathlineto{\pgfqpoint{3.972719in}{1.708303in}}%
\pgfpathlineto{\pgfqpoint{3.986099in}{1.703335in}}%
\pgfpathlineto{\pgfqpoint{3.999485in}{1.698393in}}%
\pgfpathlineto{\pgfqpoint{4.012876in}{1.693476in}}%
\pgfpathlineto{\pgfqpoint{4.005104in}{1.689470in}}%
\pgfpathlineto{\pgfqpoint{3.997325in}{1.685688in}}%
\pgfpathlineto{\pgfqpoint{3.989538in}{1.682137in}}%
\pgfpathlineto{\pgfqpoint{3.981743in}{1.678825in}}%
\pgfpathlineto{\pgfqpoint{3.968333in}{1.684016in}}%
\pgfpathlineto{\pgfqpoint{3.954929in}{1.689232in}}%
\pgfpathlineto{\pgfqpoint{3.941530in}{1.694472in}}%
\pgfpathlineto{\pgfqpoint{3.928136in}{1.699738in}}%
\pgfpathlineto{\pgfqpoint{3.935950in}{1.702771in}}%
\pgfpathlineto{\pgfqpoint{3.943756in}{1.706047in}}%
\pgfpathlineto{\pgfqpoint{3.951554in}{1.709557in}}%
\pgfpathlineto{\pgfqpoint{3.959345in}{1.713295in}}%
\pgfpathclose%
\pgfusepath{fill}%
\end{pgfscope}%
\begin{pgfscope}%
\pgfpathrectangle{\pgfqpoint{1.254980in}{0.150000in}}{\pgfqpoint{5.490039in}{5.490039in}}%
\pgfusepath{clip}%
\pgfsetbuttcap%
\pgfsetroundjoin%
\definecolor{currentfill}{rgb}{0.281924,0.089666,0.412415}%
\pgfsetfillcolor{currentfill}%
\pgfsetfillopacity{0.700000}%
\pgfsetlinewidth{0.000000pt}%
\definecolor{currentstroke}{rgb}{0.000000,0.000000,0.000000}%
\pgfsetstrokecolor{currentstroke}%
\pgfsetdash{}{0pt}%
\pgfpathmoveto{\pgfqpoint{3.629575in}{1.806592in}}%
\pgfpathlineto{\pgfqpoint{3.642889in}{1.800523in}}%
\pgfpathlineto{\pgfqpoint{3.656209in}{1.794480in}}%
\pgfpathlineto{\pgfqpoint{3.669533in}{1.788464in}}%
\pgfpathlineto{\pgfqpoint{3.682861in}{1.782475in}}%
\pgfpathlineto{\pgfqpoint{3.674913in}{1.781944in}}%
\pgfpathlineto{\pgfqpoint{3.666954in}{1.781715in}}%
\pgfpathlineto{\pgfqpoint{3.658983in}{1.781794in}}%
\pgfpathlineto{\pgfqpoint{3.651002in}{1.782190in}}%
\pgfpathlineto{\pgfqpoint{3.637648in}{1.788480in}}%
\pgfpathlineto{\pgfqpoint{3.624299in}{1.794797in}}%
\pgfpathlineto{\pgfqpoint{3.610954in}{1.801141in}}%
\pgfpathlineto{\pgfqpoint{3.597615in}{1.807511in}}%
\pgfpathlineto{\pgfqpoint{3.605622in}{1.806808in}}%
\pgfpathlineto{\pgfqpoint{3.613618in}{1.806427in}}%
\pgfpathlineto{\pgfqpoint{3.621602in}{1.806357in}}%
\pgfpathlineto{\pgfqpoint{3.629575in}{1.806592in}}%
\pgfpathclose%
\pgfusepath{fill}%
\end{pgfscope}%
\begin{pgfscope}%
\pgfpathrectangle{\pgfqpoint{1.254980in}{0.150000in}}{\pgfqpoint{5.490039in}{5.490039in}}%
\pgfusepath{clip}%
\pgfsetbuttcap%
\pgfsetroundjoin%
\definecolor{currentfill}{rgb}{0.279566,0.067836,0.391917}%
\pgfsetfillcolor{currentfill}%
\pgfsetfillopacity{0.700000}%
\pgfsetlinewidth{0.000000pt}%
\definecolor{currentstroke}{rgb}{0.000000,0.000000,0.000000}%
\pgfsetstrokecolor{currentstroke}%
\pgfsetdash{}{0pt}%
\pgfpathmoveto{\pgfqpoint{4.903604in}{1.778194in}}%
\pgfpathlineto{\pgfqpoint{4.917229in}{1.776179in}}%
\pgfpathlineto{\pgfqpoint{4.930862in}{1.774187in}}%
\pgfpathlineto{\pgfqpoint{4.944502in}{1.772219in}}%
\pgfpathlineto{\pgfqpoint{4.958150in}{1.770274in}}%
\pgfpathlineto{\pgfqpoint{4.950696in}{1.760078in}}%
\pgfpathlineto{\pgfqpoint{4.943238in}{1.749890in}}%
\pgfpathlineto{\pgfqpoint{4.935775in}{1.739712in}}%
\pgfpathlineto{\pgfqpoint{4.928307in}{1.729550in}}%
\pgfpathlineto{\pgfqpoint{4.914652in}{1.731667in}}%
\pgfpathlineto{\pgfqpoint{4.901004in}{1.733807in}}%
\pgfpathlineto{\pgfqpoint{4.887364in}{1.735971in}}%
\pgfpathlineto{\pgfqpoint{4.873731in}{1.738158in}}%
\pgfpathlineto{\pgfqpoint{4.881206in}{1.748143in}}%
\pgfpathlineto{\pgfqpoint{4.888677in}{1.758147in}}%
\pgfpathlineto{\pgfqpoint{4.896143in}{1.768165in}}%
\pgfpathlineto{\pgfqpoint{4.903604in}{1.778194in}}%
\pgfpathclose%
\pgfusepath{fill}%
\end{pgfscope}%
\begin{pgfscope}%
\pgfpathrectangle{\pgfqpoint{1.254980in}{0.150000in}}{\pgfqpoint{5.490039in}{5.490039in}}%
\pgfusepath{clip}%
\pgfsetbuttcap%
\pgfsetroundjoin%
\definecolor{currentfill}{rgb}{0.272594,0.025563,0.353093}%
\pgfsetfillcolor{currentfill}%
\pgfsetfillopacity{0.700000}%
\pgfsetlinewidth{0.000000pt}%
\definecolor{currentstroke}{rgb}{0.000000,0.000000,0.000000}%
\pgfsetstrokecolor{currentstroke}%
\pgfsetdash{}{0pt}%
\pgfpathmoveto{\pgfqpoint{4.596389in}{1.703776in}}%
\pgfpathlineto{\pgfqpoint{4.609921in}{1.700831in}}%
\pgfpathlineto{\pgfqpoint{4.623461in}{1.697909in}}%
\pgfpathlineto{\pgfqpoint{4.637008in}{1.695011in}}%
\pgfpathlineto{\pgfqpoint{4.650561in}{1.692136in}}%
\pgfpathlineto{\pgfqpoint{4.643020in}{1.683136in}}%
\pgfpathlineto{\pgfqpoint{4.635474in}{1.674210in}}%
\pgfpathlineto{\pgfqpoint{4.627924in}{1.665364in}}%
\pgfpathlineto{\pgfqpoint{4.620370in}{1.656603in}}%
\pgfpathlineto{\pgfqpoint{4.606806in}{1.659688in}}%
\pgfpathlineto{\pgfqpoint{4.593250in}{1.662796in}}%
\pgfpathlineto{\pgfqpoint{4.579700in}{1.665928in}}%
\pgfpathlineto{\pgfqpoint{4.566157in}{1.669083in}}%
\pgfpathlineto{\pgfqpoint{4.573722in}{1.677630in}}%
\pgfpathlineto{\pgfqpoint{4.581282in}{1.686264in}}%
\pgfpathlineto{\pgfqpoint{4.588838in}{1.694981in}}%
\pgfpathlineto{\pgfqpoint{4.596389in}{1.703776in}}%
\pgfpathclose%
\pgfusepath{fill}%
\end{pgfscope}%
\begin{pgfscope}%
\pgfpathrectangle{\pgfqpoint{1.254980in}{0.150000in}}{\pgfqpoint{5.490039in}{5.490039in}}%
\pgfusepath{clip}%
\pgfsetbuttcap%
\pgfsetroundjoin%
\definecolor{currentfill}{rgb}{0.203063,0.379716,0.553925}%
\pgfsetfillcolor{currentfill}%
\pgfsetfillopacity{0.700000}%
\pgfsetlinewidth{0.000000pt}%
\definecolor{currentstroke}{rgb}{0.000000,0.000000,0.000000}%
\pgfsetstrokecolor{currentstroke}%
\pgfsetdash{}{0pt}%
\pgfpathmoveto{\pgfqpoint{2.595371in}{2.415640in}}%
\pgfpathlineto{\pgfqpoint{2.608593in}{2.406058in}}%
\pgfpathlineto{\pgfqpoint{2.621817in}{2.396518in}}%
\pgfpathlineto{\pgfqpoint{2.635042in}{2.387018in}}%
\pgfpathlineto{\pgfqpoint{2.648270in}{2.377559in}}%
\pgfpathlineto{\pgfqpoint{2.639457in}{2.388922in}}%
\pgfpathlineto{\pgfqpoint{2.630617in}{2.400795in}}%
\pgfpathlineto{\pgfqpoint{2.621750in}{2.413188in}}%
\pgfpathlineto{\pgfqpoint{2.612854in}{2.426112in}}%
\pgfpathlineto{\pgfqpoint{2.599580in}{2.435939in}}%
\pgfpathlineto{\pgfqpoint{2.586308in}{2.445806in}}%
\pgfpathlineto{\pgfqpoint{2.573037in}{2.455714in}}%
\pgfpathlineto{\pgfqpoint{2.559768in}{2.465664in}}%
\pgfpathlineto{\pgfqpoint{2.568712in}{2.452366in}}%
\pgfpathlineto{\pgfqpoint{2.577627in}{2.439603in}}%
\pgfpathlineto{\pgfqpoint{2.586513in}{2.427365in}}%
\pgfpathlineto{\pgfqpoint{2.595371in}{2.415640in}}%
\pgfpathclose%
\pgfusepath{fill}%
\end{pgfscope}%
\begin{pgfscope}%
\pgfpathrectangle{\pgfqpoint{1.254980in}{0.150000in}}{\pgfqpoint{5.490039in}{5.490039in}}%
\pgfusepath{clip}%
\pgfsetbuttcap%
\pgfsetroundjoin%
\definecolor{currentfill}{rgb}{0.278826,0.175490,0.483397}%
\pgfsetfillcolor{currentfill}%
\pgfsetfillopacity{0.700000}%
\pgfsetlinewidth{0.000000pt}%
\definecolor{currentstroke}{rgb}{0.000000,0.000000,0.000000}%
\pgfsetstrokecolor{currentstroke}%
\pgfsetdash{}{0pt}%
\pgfpathmoveto{\pgfqpoint{5.464394in}{1.986742in}}%
\pgfpathlineto{\pgfqpoint{5.478206in}{1.986147in}}%
\pgfpathlineto{\pgfqpoint{5.492028in}{1.985576in}}%
\pgfpathlineto{\pgfqpoint{5.505858in}{1.985028in}}%
\pgfpathlineto{\pgfqpoint{5.519698in}{1.984504in}}%
\pgfpathlineto{\pgfqpoint{5.512433in}{1.974502in}}%
\pgfpathlineto{\pgfqpoint{5.505161in}{1.964413in}}%
\pgfpathlineto{\pgfqpoint{5.497882in}{1.954238in}}%
\pgfpathlineto{\pgfqpoint{5.490597in}{1.943981in}}%
\pgfpathlineto{\pgfqpoint{5.476750in}{1.944599in}}%
\pgfpathlineto{\pgfqpoint{5.462912in}{1.945241in}}%
\pgfpathlineto{\pgfqpoint{5.449083in}{1.945907in}}%
\pgfpathlineto{\pgfqpoint{5.435262in}{1.946596in}}%
\pgfpathlineto{\pgfqpoint{5.442555in}{1.956754in}}%
\pgfpathlineto{\pgfqpoint{5.449842in}{1.966832in}}%
\pgfpathlineto{\pgfqpoint{5.457121in}{1.976829in}}%
\pgfpathlineto{\pgfqpoint{5.464394in}{1.986742in}}%
\pgfpathclose%
\pgfusepath{fill}%
\end{pgfscope}%
\begin{pgfscope}%
\pgfpathrectangle{\pgfqpoint{1.254980in}{0.150000in}}{\pgfqpoint{5.490039in}{5.490039in}}%
\pgfusepath{clip}%
\pgfsetbuttcap%
\pgfsetroundjoin%
\definecolor{currentfill}{rgb}{0.282884,0.135920,0.453427}%
\pgfsetfillcolor{currentfill}%
\pgfsetfillopacity{0.700000}%
\pgfsetlinewidth{0.000000pt}%
\definecolor{currentstroke}{rgb}{0.000000,0.000000,0.000000}%
\pgfsetstrokecolor{currentstroke}%
\pgfsetdash{}{0pt}%
\pgfpathmoveto{\pgfqpoint{3.437884in}{1.886076in}}%
\pgfpathlineto{\pgfqpoint{3.451171in}{1.879376in}}%
\pgfpathlineto{\pgfqpoint{3.464462in}{1.872705in}}%
\pgfpathlineto{\pgfqpoint{3.477758in}{1.866062in}}%
\pgfpathlineto{\pgfqpoint{3.491058in}{1.859446in}}%
\pgfpathlineto{\pgfqpoint{3.482984in}{1.861096in}}%
\pgfpathlineto{\pgfqpoint{3.474897in}{1.863091in}}%
\pgfpathlineto{\pgfqpoint{3.466796in}{1.865438in}}%
\pgfpathlineto{\pgfqpoint{3.458682in}{1.868146in}}%
\pgfpathlineto{\pgfqpoint{3.445353in}{1.875076in}}%
\pgfpathlineto{\pgfqpoint{3.432029in}{1.882035in}}%
\pgfpathlineto{\pgfqpoint{3.418708in}{1.889021in}}%
\pgfpathlineto{\pgfqpoint{3.405392in}{1.896036in}}%
\pgfpathlineto{\pgfqpoint{3.413537in}{1.893007in}}%
\pgfpathlineto{\pgfqpoint{3.421666in}{1.890343in}}%
\pgfpathlineto{\pgfqpoint{3.429782in}{1.888036in}}%
\pgfpathlineto{\pgfqpoint{3.437884in}{1.886076in}}%
\pgfpathclose%
\pgfusepath{fill}%
\end{pgfscope}%
\begin{pgfscope}%
\pgfpathrectangle{\pgfqpoint{1.254980in}{0.150000in}}{\pgfqpoint{5.490039in}{5.490039in}}%
\pgfusepath{clip}%
\pgfsetbuttcap%
\pgfsetroundjoin%
\definecolor{currentfill}{rgb}{0.269944,0.014625,0.341379}%
\pgfsetfillcolor{currentfill}%
\pgfsetfillopacity{0.700000}%
\pgfsetlinewidth{0.000000pt}%
\definecolor{currentstroke}{rgb}{0.000000,0.000000,0.000000}%
\pgfsetstrokecolor{currentstroke}%
\pgfsetdash{}{0pt}%
\pgfpathmoveto{\pgfqpoint{4.373737in}{1.678272in}}%
\pgfpathlineto{\pgfqpoint{4.387210in}{1.674609in}}%
\pgfpathlineto{\pgfqpoint{4.400690in}{1.670970in}}%
\pgfpathlineto{\pgfqpoint{4.414177in}{1.667355in}}%
\pgfpathlineto{\pgfqpoint{4.427670in}{1.663764in}}%
\pgfpathlineto{\pgfqpoint{4.420059in}{1.656201in}}%
\pgfpathlineto{\pgfqpoint{4.412443in}{1.648766in}}%
\pgfpathlineto{\pgfqpoint{4.404823in}{1.641465in}}%
\pgfpathlineto{\pgfqpoint{4.397197in}{1.634304in}}%
\pgfpathlineto{\pgfqpoint{4.383692in}{1.638130in}}%
\pgfpathlineto{\pgfqpoint{4.370193in}{1.641980in}}%
\pgfpathlineto{\pgfqpoint{4.356700in}{1.645854in}}%
\pgfpathlineto{\pgfqpoint{4.343214in}{1.649752in}}%
\pgfpathlineto{\pgfqpoint{4.350853in}{1.656674in}}%
\pgfpathlineto{\pgfqpoint{4.358486in}{1.663738in}}%
\pgfpathlineto{\pgfqpoint{4.366114in}{1.670940in}}%
\pgfpathlineto{\pgfqpoint{4.373737in}{1.678272in}}%
\pgfpathclose%
\pgfusepath{fill}%
\end{pgfscope}%
\begin{pgfscope}%
\pgfpathrectangle{\pgfqpoint{1.254980in}{0.150000in}}{\pgfqpoint{5.490039in}{5.490039in}}%
\pgfusepath{clip}%
\pgfsetbuttcap%
\pgfsetroundjoin%
\definecolor{currentfill}{rgb}{0.252194,0.269783,0.531579}%
\pgfsetfillcolor{currentfill}%
\pgfsetfillopacity{0.700000}%
\pgfsetlinewidth{0.000000pt}%
\definecolor{currentstroke}{rgb}{0.000000,0.000000,0.000000}%
\pgfsetstrokecolor{currentstroke}%
\pgfsetdash{}{0pt}%
\pgfpathmoveto{\pgfqpoint{2.947445in}{2.162813in}}%
\pgfpathlineto{\pgfqpoint{2.960682in}{2.154480in}}%
\pgfpathlineto{\pgfqpoint{2.973922in}{2.146181in}}%
\pgfpathlineto{\pgfqpoint{2.987166in}{2.137916in}}%
\pgfpathlineto{\pgfqpoint{3.000412in}{2.129684in}}%
\pgfpathlineto{\pgfqpoint{2.991943in}{2.137052in}}%
\pgfpathlineto{\pgfqpoint{2.983453in}{2.144865in}}%
\pgfpathlineto{\pgfqpoint{2.974942in}{2.153134in}}%
\pgfpathlineto{\pgfqpoint{2.966409in}{2.161868in}}%
\pgfpathlineto{\pgfqpoint{2.953124in}{2.170447in}}%
\pgfpathlineto{\pgfqpoint{2.939841in}{2.179060in}}%
\pgfpathlineto{\pgfqpoint{2.926561in}{2.187706in}}%
\pgfpathlineto{\pgfqpoint{2.913285in}{2.196387in}}%
\pgfpathlineto{\pgfqpoint{2.921858in}{2.187299in}}%
\pgfpathlineto{\pgfqpoint{2.930409in}{2.178681in}}%
\pgfpathlineto{\pgfqpoint{2.938937in}{2.170522in}}%
\pgfpathlineto{\pgfqpoint{2.947445in}{2.162813in}}%
\pgfpathclose%
\pgfusepath{fill}%
\end{pgfscope}%
\begin{pgfscope}%
\pgfpathrectangle{\pgfqpoint{1.254980in}{0.150000in}}{\pgfqpoint{5.490039in}{5.490039in}}%
\pgfusepath{clip}%
\pgfsetbuttcap%
\pgfsetroundjoin%
\definecolor{currentfill}{rgb}{0.280868,0.160771,0.472899}%
\pgfsetfillcolor{currentfill}%
\pgfsetfillopacity{0.700000}%
\pgfsetlinewidth{0.000000pt}%
\definecolor{currentstroke}{rgb}{0.000000,0.000000,0.000000}%
\pgfsetstrokecolor{currentstroke}%
\pgfsetdash{}{0pt}%
\pgfpathmoveto{\pgfqpoint{5.380068in}{1.949589in}}%
\pgfpathlineto{\pgfqpoint{5.393854in}{1.948805in}}%
\pgfpathlineto{\pgfqpoint{5.407648in}{1.948045in}}%
\pgfpathlineto{\pgfqpoint{5.421451in}{1.947309in}}%
\pgfpathlineto{\pgfqpoint{5.435262in}{1.946596in}}%
\pgfpathlineto{\pgfqpoint{5.427963in}{1.936360in}}%
\pgfpathlineto{\pgfqpoint{5.420656in}{1.926049in}}%
\pgfpathlineto{\pgfqpoint{5.413343in}{1.915666in}}%
\pgfpathlineto{\pgfqpoint{5.406024in}{1.905211in}}%
\pgfpathlineto{\pgfqpoint{5.392205in}{1.906031in}}%
\pgfpathlineto{\pgfqpoint{5.378395in}{1.906875in}}%
\pgfpathlineto{\pgfqpoint{5.364594in}{1.907743in}}%
\pgfpathlineto{\pgfqpoint{5.350801in}{1.908634in}}%
\pgfpathlineto{\pgfqpoint{5.358127in}{1.918976in}}%
\pgfpathlineto{\pgfqpoint{5.365448in}{1.929250in}}%
\pgfpathlineto{\pgfqpoint{5.372761in}{1.939456in}}%
\pgfpathlineto{\pgfqpoint{5.380068in}{1.949589in}}%
\pgfpathclose%
\pgfusepath{fill}%
\end{pgfscope}%
\begin{pgfscope}%
\pgfpathrectangle{\pgfqpoint{1.254980in}{0.150000in}}{\pgfqpoint{5.490039in}{5.490039in}}%
\pgfusepath{clip}%
\pgfsetbuttcap%
\pgfsetroundjoin%
\definecolor{currentfill}{rgb}{0.277941,0.056324,0.381191}%
\pgfsetfillcolor{currentfill}%
\pgfsetfillopacity{0.700000}%
\pgfsetlinewidth{0.000000pt}%
\definecolor{currentstroke}{rgb}{0.000000,0.000000,0.000000}%
\pgfsetstrokecolor{currentstroke}%
\pgfsetdash{}{0pt}%
\pgfpathmoveto{\pgfqpoint{3.821178in}{1.742772in}}%
\pgfpathlineto{\pgfqpoint{3.834530in}{1.737304in}}%
\pgfpathlineto{\pgfqpoint{3.847886in}{1.731861in}}%
\pgfpathlineto{\pgfqpoint{3.861248in}{1.726444in}}%
\pgfpathlineto{\pgfqpoint{3.874615in}{1.721052in}}%
\pgfpathlineto{\pgfqpoint{3.866772in}{1.718550in}}%
\pgfpathlineto{\pgfqpoint{3.858921in}{1.716307in}}%
\pgfpathlineto{\pgfqpoint{3.851060in}{1.714332in}}%
\pgfpathlineto{\pgfqpoint{3.843191in}{1.712633in}}%
\pgfpathlineto{\pgfqpoint{3.829802in}{1.718311in}}%
\pgfpathlineto{\pgfqpoint{3.816419in}{1.724016in}}%
\pgfpathlineto{\pgfqpoint{3.803041in}{1.729745in}}%
\pgfpathlineto{\pgfqpoint{3.789668in}{1.735501in}}%
\pgfpathlineto{\pgfqpoint{3.797560in}{1.736908in}}%
\pgfpathlineto{\pgfqpoint{3.805442in}{1.738595in}}%
\pgfpathlineto{\pgfqpoint{3.813315in}{1.740552in}}%
\pgfpathlineto{\pgfqpoint{3.821178in}{1.742772in}}%
\pgfpathclose%
\pgfusepath{fill}%
\end{pgfscope}%
\begin{pgfscope}%
\pgfpathrectangle{\pgfqpoint{1.254980in}{0.150000in}}{\pgfqpoint{5.490039in}{5.490039in}}%
\pgfusepath{clip}%
\pgfsetbuttcap%
\pgfsetroundjoin%
\definecolor{currentfill}{rgb}{0.277941,0.056324,0.381191}%
\pgfsetfillcolor{currentfill}%
\pgfsetfillopacity{0.700000}%
\pgfsetlinewidth{0.000000pt}%
\definecolor{currentstroke}{rgb}{0.000000,0.000000,0.000000}%
\pgfsetstrokecolor{currentstroke}%
\pgfsetdash{}{0pt}%
\pgfpathmoveto{\pgfqpoint{4.819277in}{1.747141in}}%
\pgfpathlineto{\pgfqpoint{4.832879in}{1.744860in}}%
\pgfpathlineto{\pgfqpoint{4.846489in}{1.742602in}}%
\pgfpathlineto{\pgfqpoint{4.860106in}{1.740368in}}%
\pgfpathlineto{\pgfqpoint{4.873731in}{1.738158in}}%
\pgfpathlineto{\pgfqpoint{4.866251in}{1.728195in}}%
\pgfpathlineto{\pgfqpoint{4.858767in}{1.718260in}}%
\pgfpathlineto{\pgfqpoint{4.851278in}{1.708357in}}%
\pgfpathlineto{\pgfqpoint{4.843785in}{1.698489in}}%
\pgfpathlineto{\pgfqpoint{4.830152in}{1.700884in}}%
\pgfpathlineto{\pgfqpoint{4.816527in}{1.703303in}}%
\pgfpathlineto{\pgfqpoint{4.802909in}{1.705745in}}%
\pgfpathlineto{\pgfqpoint{4.789299in}{1.708211in}}%
\pgfpathlineto{\pgfqpoint{4.796800in}{1.717889in}}%
\pgfpathlineto{\pgfqpoint{4.804297in}{1.727606in}}%
\pgfpathlineto{\pgfqpoint{4.811789in}{1.737359in}}%
\pgfpathlineto{\pgfqpoint{4.819277in}{1.747141in}}%
\pgfpathclose%
\pgfusepath{fill}%
\end{pgfscope}%
\begin{pgfscope}%
\pgfpathrectangle{\pgfqpoint{1.254980in}{0.150000in}}{\pgfqpoint{5.490039in}{5.490039in}}%
\pgfusepath{clip}%
\pgfsetbuttcap%
\pgfsetroundjoin%
\definecolor{currentfill}{rgb}{0.282623,0.140926,0.457517}%
\pgfsetfillcolor{currentfill}%
\pgfsetfillopacity{0.700000}%
\pgfsetlinewidth{0.000000pt}%
\definecolor{currentstroke}{rgb}{0.000000,0.000000,0.000000}%
\pgfsetstrokecolor{currentstroke}%
\pgfsetdash{}{0pt}%
\pgfpathmoveto{\pgfqpoint{5.295716in}{1.912434in}}%
\pgfpathlineto{\pgfqpoint{5.309475in}{1.911448in}}%
\pgfpathlineto{\pgfqpoint{5.323242in}{1.910487in}}%
\pgfpathlineto{\pgfqpoint{5.337017in}{1.909548in}}%
\pgfpathlineto{\pgfqpoint{5.350801in}{1.908634in}}%
\pgfpathlineto{\pgfqpoint{5.343468in}{1.898227in}}%
\pgfpathlineto{\pgfqpoint{5.336129in}{1.887758in}}%
\pgfpathlineto{\pgfqpoint{5.328784in}{1.877229in}}%
\pgfpathlineto{\pgfqpoint{5.321433in}{1.866644in}}%
\pgfpathlineto{\pgfqpoint{5.307643in}{1.867679in}}%
\pgfpathlineto{\pgfqpoint{5.293860in}{1.868738in}}%
\pgfpathlineto{\pgfqpoint{5.280087in}{1.869820in}}%
\pgfpathlineto{\pgfqpoint{5.266321in}{1.870926in}}%
\pgfpathlineto{\pgfqpoint{5.273679in}{1.881386in}}%
\pgfpathlineto{\pgfqpoint{5.281031in}{1.891792in}}%
\pgfpathlineto{\pgfqpoint{5.288376in}{1.902142in}}%
\pgfpathlineto{\pgfqpoint{5.295716in}{1.912434in}}%
\pgfpathclose%
\pgfusepath{fill}%
\end{pgfscope}%
\begin{pgfscope}%
\pgfpathrectangle{\pgfqpoint{1.254980in}{0.150000in}}{\pgfqpoint{5.490039in}{5.490039in}}%
\pgfusepath{clip}%
\pgfsetbuttcap%
\pgfsetroundjoin%
\definecolor{currentfill}{rgb}{0.277134,0.185228,0.489898}%
\pgfsetfillcolor{currentfill}%
\pgfsetfillopacity{0.700000}%
\pgfsetlinewidth{0.000000pt}%
\definecolor{currentstroke}{rgb}{0.000000,0.000000,0.000000}%
\pgfsetstrokecolor{currentstroke}%
\pgfsetdash{}{0pt}%
\pgfpathmoveto{\pgfqpoint{3.245908in}{1.982457in}}%
\pgfpathlineto{\pgfqpoint{3.259177in}{1.975094in}}%
\pgfpathlineto{\pgfqpoint{3.272450in}{1.967761in}}%
\pgfpathlineto{\pgfqpoint{3.285727in}{1.960457in}}%
\pgfpathlineto{\pgfqpoint{3.299007in}{1.953183in}}%
\pgfpathlineto{\pgfqpoint{3.290786in}{1.957234in}}%
\pgfpathlineto{\pgfqpoint{3.282548in}{1.961675in}}%
\pgfpathlineto{\pgfqpoint{3.274294in}{1.966515in}}%
\pgfpathlineto{\pgfqpoint{3.266023in}{1.971763in}}%
\pgfpathlineto{\pgfqpoint{3.252709in}{1.979367in}}%
\pgfpathlineto{\pgfqpoint{3.239400in}{1.987001in}}%
\pgfpathlineto{\pgfqpoint{3.226093in}{1.994665in}}%
\pgfpathlineto{\pgfqpoint{3.212791in}{2.002359in}}%
\pgfpathlineto{\pgfqpoint{3.221096in}{1.996775in}}%
\pgfpathlineto{\pgfqpoint{3.229384in}{1.991603in}}%
\pgfpathlineto{\pgfqpoint{3.237654in}{1.986834in}}%
\pgfpathlineto{\pgfqpoint{3.245908in}{1.982457in}}%
\pgfpathclose%
\pgfusepath{fill}%
\end{pgfscope}%
\begin{pgfscope}%
\pgfpathrectangle{\pgfqpoint{1.254980in}{0.150000in}}{\pgfqpoint{5.490039in}{5.490039in}}%
\pgfusepath{clip}%
\pgfsetbuttcap%
\pgfsetroundjoin%
\definecolor{currentfill}{rgb}{0.208623,0.367752,0.552675}%
\pgfsetfillcolor{currentfill}%
\pgfsetfillopacity{0.700000}%
\pgfsetlinewidth{0.000000pt}%
\definecolor{currentstroke}{rgb}{0.000000,0.000000,0.000000}%
\pgfsetstrokecolor{currentstroke}%
\pgfsetdash{}{0pt}%
\pgfpathmoveto{\pgfqpoint{2.648270in}{2.377559in}}%
\pgfpathlineto{\pgfqpoint{2.661499in}{2.368140in}}%
\pgfpathlineto{\pgfqpoint{2.674731in}{2.358761in}}%
\pgfpathlineto{\pgfqpoint{2.687964in}{2.349421in}}%
\pgfpathlineto{\pgfqpoint{2.701200in}{2.340121in}}%
\pgfpathlineto{\pgfqpoint{2.692432in}{2.351123in}}%
\pgfpathlineto{\pgfqpoint{2.683638in}{2.362631in}}%
\pgfpathlineto{\pgfqpoint{2.674817in}{2.374656in}}%
\pgfpathlineto{\pgfqpoint{2.665968in}{2.387207in}}%
\pgfpathlineto{\pgfqpoint{2.652686in}{2.396874in}}%
\pgfpathlineto{\pgfqpoint{2.639407in}{2.406580in}}%
\pgfpathlineto{\pgfqpoint{2.626130in}{2.416326in}}%
\pgfpathlineto{\pgfqpoint{2.612854in}{2.426112in}}%
\pgfpathlineto{\pgfqpoint{2.621750in}{2.413188in}}%
\pgfpathlineto{\pgfqpoint{2.630617in}{2.400795in}}%
\pgfpathlineto{\pgfqpoint{2.639457in}{2.388922in}}%
\pgfpathlineto{\pgfqpoint{2.648270in}{2.377559in}}%
\pgfpathclose%
\pgfusepath{fill}%
\end{pgfscope}%
\begin{pgfscope}%
\pgfpathrectangle{\pgfqpoint{1.254980in}{0.150000in}}{\pgfqpoint{5.490039in}{5.490039in}}%
\pgfusepath{clip}%
\pgfsetbuttcap%
\pgfsetroundjoin%
\definecolor{currentfill}{rgb}{0.283187,0.125848,0.444960}%
\pgfsetfillcolor{currentfill}%
\pgfsetfillopacity{0.700000}%
\pgfsetlinewidth{0.000000pt}%
\definecolor{currentstroke}{rgb}{0.000000,0.000000,0.000000}%
\pgfsetstrokecolor{currentstroke}%
\pgfsetdash{}{0pt}%
\pgfpathmoveto{\pgfqpoint{5.211344in}{1.875585in}}%
\pgfpathlineto{\pgfqpoint{5.225076in}{1.874385in}}%
\pgfpathlineto{\pgfqpoint{5.238816in}{1.873208in}}%
\pgfpathlineto{\pgfqpoint{5.252564in}{1.872055in}}%
\pgfpathlineto{\pgfqpoint{5.266321in}{1.870926in}}%
\pgfpathlineto{\pgfqpoint{5.258958in}{1.860415in}}%
\pgfpathlineto{\pgfqpoint{5.251588in}{1.849857in}}%
\pgfpathlineto{\pgfqpoint{5.244213in}{1.839253in}}%
\pgfpathlineto{\pgfqpoint{5.236833in}{1.828608in}}%
\pgfpathlineto{\pgfqpoint{5.223069in}{1.829871in}}%
\pgfpathlineto{\pgfqpoint{5.209314in}{1.831158in}}%
\pgfpathlineto{\pgfqpoint{5.195567in}{1.832468in}}%
\pgfpathlineto{\pgfqpoint{5.181829in}{1.833802in}}%
\pgfpathlineto{\pgfqpoint{5.189216in}{1.844308in}}%
\pgfpathlineto{\pgfqpoint{5.196597in}{1.854776in}}%
\pgfpathlineto{\pgfqpoint{5.203974in}{1.865203in}}%
\pgfpathlineto{\pgfqpoint{5.211344in}{1.875585in}}%
\pgfpathclose%
\pgfusepath{fill}%
\end{pgfscope}%
\begin{pgfscope}%
\pgfpathrectangle{\pgfqpoint{1.254980in}{0.150000in}}{\pgfqpoint{5.490039in}{5.490039in}}%
\pgfusepath{clip}%
\pgfsetbuttcap%
\pgfsetroundjoin%
\definecolor{currentfill}{rgb}{0.271305,0.019942,0.347269}%
\pgfsetfillcolor{currentfill}%
\pgfsetfillopacity{0.700000}%
\pgfsetlinewidth{0.000000pt}%
\definecolor{currentstroke}{rgb}{0.000000,0.000000,0.000000}%
\pgfsetstrokecolor{currentstroke}%
\pgfsetdash{}{0pt}%
\pgfpathmoveto{\pgfqpoint{4.512056in}{1.681942in}}%
\pgfpathlineto{\pgfqpoint{4.525571in}{1.678691in}}%
\pgfpathlineto{\pgfqpoint{4.539093in}{1.675465in}}%
\pgfpathlineto{\pgfqpoint{4.552622in}{1.672262in}}%
\pgfpathlineto{\pgfqpoint{4.566157in}{1.669083in}}%
\pgfpathlineto{\pgfqpoint{4.558588in}{1.660631in}}%
\pgfpathlineto{\pgfqpoint{4.551014in}{1.652277in}}%
\pgfpathlineto{\pgfqpoint{4.543436in}{1.644028in}}%
\pgfpathlineto{\pgfqpoint{4.535853in}{1.635888in}}%
\pgfpathlineto{\pgfqpoint{4.522307in}{1.639290in}}%
\pgfpathlineto{\pgfqpoint{4.508767in}{1.642715in}}%
\pgfpathlineto{\pgfqpoint{4.495234in}{1.646164in}}%
\pgfpathlineto{\pgfqpoint{4.481708in}{1.649636in}}%
\pgfpathlineto{\pgfqpoint{4.489302in}{1.657549in}}%
\pgfpathlineto{\pgfqpoint{4.496891in}{1.665574in}}%
\pgfpathlineto{\pgfqpoint{4.504476in}{1.673707in}}%
\pgfpathlineto{\pgfqpoint{4.512056in}{1.681942in}}%
\pgfpathclose%
\pgfusepath{fill}%
\end{pgfscope}%
\begin{pgfscope}%
\pgfpathrectangle{\pgfqpoint{1.254980in}{0.150000in}}{\pgfqpoint{5.490039in}{5.490039in}}%
\pgfusepath{clip}%
\pgfsetbuttcap%
\pgfsetroundjoin%
\definecolor{currentfill}{rgb}{0.274952,0.037752,0.364543}%
\pgfsetfillcolor{currentfill}%
\pgfsetfillopacity{0.700000}%
\pgfsetlinewidth{0.000000pt}%
\definecolor{currentstroke}{rgb}{0.000000,0.000000,0.000000}%
\pgfsetstrokecolor{currentstroke}%
\pgfsetdash{}{0pt}%
\pgfpathmoveto{\pgfqpoint{4.734932in}{1.718309in}}%
\pgfpathlineto{\pgfqpoint{4.748512in}{1.715749in}}%
\pgfpathlineto{\pgfqpoint{4.762101in}{1.713213in}}%
\pgfpathlineto{\pgfqpoint{4.775696in}{1.710700in}}%
\pgfpathlineto{\pgfqpoint{4.789299in}{1.708211in}}%
\pgfpathlineto{\pgfqpoint{4.781793in}{1.698577in}}%
\pgfpathlineto{\pgfqpoint{4.774283in}{1.688991in}}%
\pgfpathlineto{\pgfqpoint{4.766769in}{1.679459in}}%
\pgfpathlineto{\pgfqpoint{4.759250in}{1.669985in}}%
\pgfpathlineto{\pgfqpoint{4.745639in}{1.672672in}}%
\pgfpathlineto{\pgfqpoint{4.732035in}{1.675382in}}%
\pgfpathlineto{\pgfqpoint{4.718438in}{1.678115in}}%
\pgfpathlineto{\pgfqpoint{4.704848in}{1.680872in}}%
\pgfpathlineto{\pgfqpoint{4.712376in}{1.690144in}}%
\pgfpathlineto{\pgfqpoint{4.719899in}{1.699478in}}%
\pgfpathlineto{\pgfqpoint{4.727417in}{1.708867in}}%
\pgfpathlineto{\pgfqpoint{4.734932in}{1.718309in}}%
\pgfpathclose%
\pgfusepath{fill}%
\end{pgfscope}%
\begin{pgfscope}%
\pgfpathrectangle{\pgfqpoint{1.254980in}{0.150000in}}{\pgfqpoint{5.490039in}{5.490039in}}%
\pgfusepath{clip}%
\pgfsetbuttcap%
\pgfsetroundjoin%
\definecolor{currentfill}{rgb}{0.283091,0.110553,0.431554}%
\pgfsetfillcolor{currentfill}%
\pgfsetfillopacity{0.700000}%
\pgfsetlinewidth{0.000000pt}%
\definecolor{currentstroke}{rgb}{0.000000,0.000000,0.000000}%
\pgfsetstrokecolor{currentstroke}%
\pgfsetdash{}{0pt}%
\pgfpathmoveto{\pgfqpoint{5.126957in}{1.839371in}}%
\pgfpathlineto{\pgfqpoint{5.140662in}{1.837943in}}%
\pgfpathlineto{\pgfqpoint{5.154376in}{1.836539in}}%
\pgfpathlineto{\pgfqpoint{5.168098in}{1.835159in}}%
\pgfpathlineto{\pgfqpoint{5.181829in}{1.833802in}}%
\pgfpathlineto{\pgfqpoint{5.174436in}{1.823260in}}%
\pgfpathlineto{\pgfqpoint{5.167038in}{1.812686in}}%
\pgfpathlineto{\pgfqpoint{5.159635in}{1.802084in}}%
\pgfpathlineto{\pgfqpoint{5.152227in}{1.791457in}}%
\pgfpathlineto{\pgfqpoint{5.138490in}{1.792960in}}%
\pgfpathlineto{\pgfqpoint{5.124761in}{1.794487in}}%
\pgfpathlineto{\pgfqpoint{5.111040in}{1.796038in}}%
\pgfpathlineto{\pgfqpoint{5.097328in}{1.797612in}}%
\pgfpathlineto{\pgfqpoint{5.104743in}{1.808088in}}%
\pgfpathlineto{\pgfqpoint{5.112153in}{1.818542in}}%
\pgfpathlineto{\pgfqpoint{5.119557in}{1.828971in}}%
\pgfpathlineto{\pgfqpoint{5.126957in}{1.839371in}}%
\pgfpathclose%
\pgfusepath{fill}%
\end{pgfscope}%
\begin{pgfscope}%
\pgfpathrectangle{\pgfqpoint{1.254980in}{0.150000in}}{\pgfqpoint{5.490039in}{5.490039in}}%
\pgfusepath{clip}%
\pgfsetbuttcap%
\pgfsetroundjoin%
\definecolor{currentfill}{rgb}{0.271305,0.019942,0.347269}%
\pgfsetfillcolor{currentfill}%
\pgfsetfillopacity{0.700000}%
\pgfsetlinewidth{0.000000pt}%
\definecolor{currentstroke}{rgb}{0.000000,0.000000,0.000000}%
\pgfsetstrokecolor{currentstroke}%
\pgfsetdash{}{0pt}%
\pgfpathmoveto{\pgfqpoint{4.151091in}{1.675243in}}%
\pgfpathlineto{\pgfqpoint{4.164517in}{1.670808in}}%
\pgfpathlineto{\pgfqpoint{4.177950in}{1.666397in}}%
\pgfpathlineto{\pgfqpoint{4.191388in}{1.662011in}}%
\pgfpathlineto{\pgfqpoint{4.204832in}{1.657649in}}%
\pgfpathlineto{\pgfqpoint{4.197137in}{1.652046in}}%
\pgfpathlineto{\pgfqpoint{4.189436in}{1.646632in}}%
\pgfpathlineto{\pgfqpoint{4.181729in}{1.641411in}}%
\pgfpathlineto{\pgfqpoint{4.174016in}{1.636390in}}%
\pgfpathlineto{\pgfqpoint{4.160555in}{1.641013in}}%
\pgfpathlineto{\pgfqpoint{4.147101in}{1.645661in}}%
\pgfpathlineto{\pgfqpoint{4.133653in}{1.650332in}}%
\pgfpathlineto{\pgfqpoint{4.120210in}{1.655028in}}%
\pgfpathlineto{\pgfqpoint{4.127940in}{1.659782in}}%
\pgfpathlineto{\pgfqpoint{4.135663in}{1.664741in}}%
\pgfpathlineto{\pgfqpoint{4.143380in}{1.669897in}}%
\pgfpathlineto{\pgfqpoint{4.151091in}{1.675243in}}%
\pgfpathclose%
\pgfusepath{fill}%
\end{pgfscope}%
\begin{pgfscope}%
\pgfpathrectangle{\pgfqpoint{1.254980in}{0.150000in}}{\pgfqpoint{5.490039in}{5.490039in}}%
\pgfusepath{clip}%
\pgfsetbuttcap%
\pgfsetroundjoin%
\definecolor{currentfill}{rgb}{0.255645,0.260703,0.528312}%
\pgfsetfillcolor{currentfill}%
\pgfsetfillopacity{0.700000}%
\pgfsetlinewidth{0.000000pt}%
\definecolor{currentstroke}{rgb}{0.000000,0.000000,0.000000}%
\pgfsetstrokecolor{currentstroke}%
\pgfsetdash{}{0pt}%
\pgfpathmoveto{\pgfqpoint{3.000412in}{2.129684in}}%
\pgfpathlineto{\pgfqpoint{3.013662in}{2.121486in}}%
\pgfpathlineto{\pgfqpoint{3.026914in}{2.113321in}}%
\pgfpathlineto{\pgfqpoint{3.040170in}{2.105188in}}%
\pgfpathlineto{\pgfqpoint{3.053428in}{2.097088in}}%
\pgfpathlineto{\pgfqpoint{3.044997in}{2.104114in}}%
\pgfpathlineto{\pgfqpoint{3.036545in}{2.111583in}}%
\pgfpathlineto{\pgfqpoint{3.028073in}{2.119503in}}%
\pgfpathlineto{\pgfqpoint{3.019580in}{2.127884in}}%
\pgfpathlineto{\pgfqpoint{3.006283in}{2.136331in}}%
\pgfpathlineto{\pgfqpoint{2.992988in}{2.144810in}}%
\pgfpathlineto{\pgfqpoint{2.979697in}{2.153323in}}%
\pgfpathlineto{\pgfqpoint{2.966409in}{2.161868in}}%
\pgfpathlineto{\pgfqpoint{2.974942in}{2.153134in}}%
\pgfpathlineto{\pgfqpoint{2.983453in}{2.144865in}}%
\pgfpathlineto{\pgfqpoint{2.991943in}{2.137052in}}%
\pgfpathlineto{\pgfqpoint{3.000412in}{2.129684in}}%
\pgfpathclose%
\pgfusepath{fill}%
\end{pgfscope}%
\begin{pgfscope}%
\pgfpathrectangle{\pgfqpoint{1.254980in}{0.150000in}}{\pgfqpoint{5.490039in}{5.490039in}}%
\pgfusepath{clip}%
\pgfsetbuttcap%
\pgfsetroundjoin%
\definecolor{currentfill}{rgb}{0.281446,0.084320,0.407414}%
\pgfsetfillcolor{currentfill}%
\pgfsetfillopacity{0.700000}%
\pgfsetlinewidth{0.000000pt}%
\definecolor{currentstroke}{rgb}{0.000000,0.000000,0.000000}%
\pgfsetstrokecolor{currentstroke}%
\pgfsetdash{}{0pt}%
\pgfpathmoveto{\pgfqpoint{3.682861in}{1.782475in}}%
\pgfpathlineto{\pgfqpoint{3.696195in}{1.776512in}}%
\pgfpathlineto{\pgfqpoint{3.709534in}{1.770575in}}%
\pgfpathlineto{\pgfqpoint{3.722877in}{1.764664in}}%
\pgfpathlineto{\pgfqpoint{3.736225in}{1.758779in}}%
\pgfpathlineto{\pgfqpoint{3.728301in}{1.757953in}}%
\pgfpathlineto{\pgfqpoint{3.720366in}{1.757425in}}%
\pgfpathlineto{\pgfqpoint{3.712420in}{1.757202in}}%
\pgfpathlineto{\pgfqpoint{3.704464in}{1.757292in}}%
\pgfpathlineto{\pgfqpoint{3.691091in}{1.763477in}}%
\pgfpathlineto{\pgfqpoint{3.677723in}{1.769689in}}%
\pgfpathlineto{\pgfqpoint{3.664360in}{1.775926in}}%
\pgfpathlineto{\pgfqpoint{3.651002in}{1.782190in}}%
\pgfpathlineto{\pgfqpoint{3.658983in}{1.781794in}}%
\pgfpathlineto{\pgfqpoint{3.666954in}{1.781715in}}%
\pgfpathlineto{\pgfqpoint{3.674913in}{1.781944in}}%
\pgfpathlineto{\pgfqpoint{3.682861in}{1.782475in}}%
\pgfpathclose%
\pgfusepath{fill}%
\end{pgfscope}%
\begin{pgfscope}%
\pgfpathrectangle{\pgfqpoint{1.254980in}{0.150000in}}{\pgfqpoint{5.490039in}{5.490039in}}%
\pgfusepath{clip}%
\pgfsetbuttcap%
\pgfsetroundjoin%
\definecolor{currentfill}{rgb}{0.273809,0.031497,0.358853}%
\pgfsetfillcolor{currentfill}%
\pgfsetfillopacity{0.700000}%
\pgfsetlinewidth{0.000000pt}%
\definecolor{currentstroke}{rgb}{0.000000,0.000000,0.000000}%
\pgfsetstrokecolor{currentstroke}%
\pgfsetdash{}{0pt}%
\pgfpathmoveto{\pgfqpoint{4.012876in}{1.693476in}}%
\pgfpathlineto{\pgfqpoint{4.026273in}{1.688584in}}%
\pgfpathlineto{\pgfqpoint{4.039675in}{1.683716in}}%
\pgfpathlineto{\pgfqpoint{4.053083in}{1.678873in}}%
\pgfpathlineto{\pgfqpoint{4.066497in}{1.674055in}}%
\pgfpathlineto{\pgfqpoint{4.058743in}{1.669780in}}%
\pgfpathlineto{\pgfqpoint{4.050982in}{1.665726in}}%
\pgfpathlineto{\pgfqpoint{4.043214in}{1.661900in}}%
\pgfpathlineto{\pgfqpoint{4.035438in}{1.658309in}}%
\pgfpathlineto{\pgfqpoint{4.022006in}{1.663401in}}%
\pgfpathlineto{\pgfqpoint{4.008580in}{1.668518in}}%
\pgfpathlineto{\pgfqpoint{3.995159in}{1.673659in}}%
\pgfpathlineto{\pgfqpoint{3.981743in}{1.678825in}}%
\pgfpathlineto{\pgfqpoint{3.989538in}{1.682137in}}%
\pgfpathlineto{\pgfqpoint{3.997325in}{1.685688in}}%
\pgfpathlineto{\pgfqpoint{4.005104in}{1.689470in}}%
\pgfpathlineto{\pgfqpoint{4.012876in}{1.693476in}}%
\pgfpathclose%
\pgfusepath{fill}%
\end{pgfscope}%
\begin{pgfscope}%
\pgfpathrectangle{\pgfqpoint{1.254980in}{0.150000in}}{\pgfqpoint{5.490039in}{5.490039in}}%
\pgfusepath{clip}%
\pgfsetbuttcap%
\pgfsetroundjoin%
\definecolor{currentfill}{rgb}{0.283187,0.125848,0.444960}%
\pgfsetfillcolor{currentfill}%
\pgfsetfillopacity{0.700000}%
\pgfsetlinewidth{0.000000pt}%
\definecolor{currentstroke}{rgb}{0.000000,0.000000,0.000000}%
\pgfsetstrokecolor{currentstroke}%
\pgfsetdash{}{0pt}%
\pgfpathmoveto{\pgfqpoint{3.491058in}{1.859446in}}%
\pgfpathlineto{\pgfqpoint{3.504362in}{1.852858in}}%
\pgfpathlineto{\pgfqpoint{3.517670in}{1.846298in}}%
\pgfpathlineto{\pgfqpoint{3.530983in}{1.839766in}}%
\pgfpathlineto{\pgfqpoint{3.544300in}{1.833260in}}%
\pgfpathlineto{\pgfqpoint{3.536254in}{1.834601in}}%
\pgfpathlineto{\pgfqpoint{3.528195in}{1.836282in}}%
\pgfpathlineto{\pgfqpoint{3.520123in}{1.838312in}}%
\pgfpathlineto{\pgfqpoint{3.512038in}{1.840700in}}%
\pgfpathlineto{\pgfqpoint{3.498693in}{1.847520in}}%
\pgfpathlineto{\pgfqpoint{3.485351in}{1.854368in}}%
\pgfpathlineto{\pgfqpoint{3.472014in}{1.861243in}}%
\pgfpathlineto{\pgfqpoint{3.458682in}{1.868146in}}%
\pgfpathlineto{\pgfqpoint{3.466796in}{1.865438in}}%
\pgfpathlineto{\pgfqpoint{3.474897in}{1.863091in}}%
\pgfpathlineto{\pgfqpoint{3.482984in}{1.861096in}}%
\pgfpathlineto{\pgfqpoint{3.491058in}{1.859446in}}%
\pgfpathclose%
\pgfusepath{fill}%
\end{pgfscope}%
\begin{pgfscope}%
\pgfpathrectangle{\pgfqpoint{1.254980in}{0.150000in}}{\pgfqpoint{5.490039in}{5.490039in}}%
\pgfusepath{clip}%
\pgfsetbuttcap%
\pgfsetroundjoin%
\definecolor{currentfill}{rgb}{0.269944,0.014625,0.341379}%
\pgfsetfillcolor{currentfill}%
\pgfsetfillopacity{0.700000}%
\pgfsetlinewidth{0.000000pt}%
\definecolor{currentstroke}{rgb}{0.000000,0.000000,0.000000}%
\pgfsetstrokecolor{currentstroke}%
\pgfsetdash{}{0pt}%
\pgfpathmoveto{\pgfqpoint{4.289333in}{1.665583in}}%
\pgfpathlineto{\pgfqpoint{4.302794in}{1.661590in}}%
\pgfpathlineto{\pgfqpoint{4.316261in}{1.657620in}}%
\pgfpathlineto{\pgfqpoint{4.329735in}{1.653674in}}%
\pgfpathlineto{\pgfqpoint{4.343214in}{1.649752in}}%
\pgfpathlineto{\pgfqpoint{4.335571in}{1.642980in}}%
\pgfpathlineto{\pgfqpoint{4.327922in}{1.636363in}}%
\pgfpathlineto{\pgfqpoint{4.320267in}{1.629907in}}%
\pgfpathlineto{\pgfqpoint{4.312608in}{1.623618in}}%
\pgfpathlineto{\pgfqpoint{4.299114in}{1.627788in}}%
\pgfpathlineto{\pgfqpoint{4.285627in}{1.631982in}}%
\pgfpathlineto{\pgfqpoint{4.272146in}{1.636200in}}%
\pgfpathlineto{\pgfqpoint{4.258671in}{1.640441in}}%
\pgfpathlineto{\pgfqpoint{4.266345in}{1.646477in}}%
\pgfpathlineto{\pgfqpoint{4.274013in}{1.652683in}}%
\pgfpathlineto{\pgfqpoint{4.281676in}{1.659054in}}%
\pgfpathlineto{\pgfqpoint{4.289333in}{1.665583in}}%
\pgfpathclose%
\pgfusepath{fill}%
\end{pgfscope}%
\begin{pgfscope}%
\pgfpathrectangle{\pgfqpoint{1.254980in}{0.150000in}}{\pgfqpoint{5.490039in}{5.490039in}}%
\pgfusepath{clip}%
\pgfsetbuttcap%
\pgfsetroundjoin%
\definecolor{currentfill}{rgb}{0.281924,0.089666,0.412415}%
\pgfsetfillcolor{currentfill}%
\pgfsetfillopacity{0.700000}%
\pgfsetlinewidth{0.000000pt}%
\definecolor{currentstroke}{rgb}{0.000000,0.000000,0.000000}%
\pgfsetstrokecolor{currentstroke}%
\pgfsetdash{}{0pt}%
\pgfpathmoveto{\pgfqpoint{5.042558in}{1.804143in}}%
\pgfpathlineto{\pgfqpoint{5.056239in}{1.802475in}}%
\pgfpathlineto{\pgfqpoint{5.069927in}{1.800831in}}%
\pgfpathlineto{\pgfqpoint{5.083623in}{1.799210in}}%
\pgfpathlineto{\pgfqpoint{5.097328in}{1.797612in}}%
\pgfpathlineto{\pgfqpoint{5.089908in}{1.787118in}}%
\pgfpathlineto{\pgfqpoint{5.082482in}{1.776609in}}%
\pgfpathlineto{\pgfqpoint{5.075052in}{1.766089in}}%
\pgfpathlineto{\pgfqpoint{5.067617in}{1.755561in}}%
\pgfpathlineto{\pgfqpoint{5.053906in}{1.757318in}}%
\pgfpathlineto{\pgfqpoint{5.040203in}{1.759099in}}%
\pgfpathlineto{\pgfqpoint{5.026508in}{1.760903in}}%
\pgfpathlineto{\pgfqpoint{5.012820in}{1.762730in}}%
\pgfpathlineto{\pgfqpoint{5.020262in}{1.773093in}}%
\pgfpathlineto{\pgfqpoint{5.027699in}{1.783452in}}%
\pgfpathlineto{\pgfqpoint{5.035131in}{1.793803in}}%
\pgfpathlineto{\pgfqpoint{5.042558in}{1.804143in}}%
\pgfpathclose%
\pgfusepath{fill}%
\end{pgfscope}%
\begin{pgfscope}%
\pgfpathrectangle{\pgfqpoint{1.254980in}{0.150000in}}{\pgfqpoint{5.490039in}{5.490039in}}%
\pgfusepath{clip}%
\pgfsetbuttcap%
\pgfsetroundjoin%
\definecolor{currentfill}{rgb}{0.150476,0.504369,0.557430}%
\pgfsetfillcolor{currentfill}%
\pgfsetfillopacity{0.700000}%
\pgfsetlinewidth{0.000000pt}%
\definecolor{currentstroke}{rgb}{0.000000,0.000000,0.000000}%
\pgfsetstrokecolor{currentstroke}%
\pgfsetdash{}{0pt}%
\pgfpathmoveto{\pgfqpoint{2.241675in}{2.717974in}}%
\pgfpathlineto{\pgfqpoint{2.254920in}{2.706900in}}%
\pgfpathlineto{\pgfqpoint{2.268165in}{2.695879in}}%
\pgfpathlineto{\pgfqpoint{2.281411in}{2.684910in}}%
\pgfpathlineto{\pgfqpoint{2.294657in}{2.673992in}}%
\pgfpathlineto{\pgfqpoint{2.285428in}{2.689742in}}%
\pgfpathlineto{\pgfqpoint{2.276163in}{2.706071in}}%
\pgfpathlineto{\pgfqpoint{2.266864in}{2.722989in}}%
\pgfpathlineto{\pgfqpoint{2.257528in}{2.740510in}}%
\pgfpathlineto{\pgfqpoint{2.244227in}{2.751819in}}%
\pgfpathlineto{\pgfqpoint{2.230926in}{2.763179in}}%
\pgfpathlineto{\pgfqpoint{2.217626in}{2.774592in}}%
\pgfpathlineto{\pgfqpoint{2.204325in}{2.786058in}}%
\pgfpathlineto{\pgfqpoint{2.213718in}{2.768139in}}%
\pgfpathlineto{\pgfqpoint{2.223073in}{2.750827in}}%
\pgfpathlineto{\pgfqpoint{2.232392in}{2.734109in}}%
\pgfpathlineto{\pgfqpoint{2.241675in}{2.717974in}}%
\pgfpathclose%
\pgfusepath{fill}%
\end{pgfscope}%
\begin{pgfscope}%
\pgfpathrectangle{\pgfqpoint{1.254980in}{0.150000in}}{\pgfqpoint{5.490039in}{5.490039in}}%
\pgfusepath{clip}%
\pgfsetbuttcap%
\pgfsetroundjoin%
\definecolor{currentfill}{rgb}{0.214298,0.355619,0.551184}%
\pgfsetfillcolor{currentfill}%
\pgfsetfillopacity{0.700000}%
\pgfsetlinewidth{0.000000pt}%
\definecolor{currentstroke}{rgb}{0.000000,0.000000,0.000000}%
\pgfsetstrokecolor{currentstroke}%
\pgfsetdash{}{0pt}%
\pgfpathmoveto{\pgfqpoint{2.701200in}{2.340121in}}%
\pgfpathlineto{\pgfqpoint{2.714437in}{2.330859in}}%
\pgfpathlineto{\pgfqpoint{2.727677in}{2.321636in}}%
\pgfpathlineto{\pgfqpoint{2.740919in}{2.312451in}}%
\pgfpathlineto{\pgfqpoint{2.754164in}{2.303304in}}%
\pgfpathlineto{\pgfqpoint{2.745440in}{2.313946in}}%
\pgfpathlineto{\pgfqpoint{2.736690in}{2.325091in}}%
\pgfpathlineto{\pgfqpoint{2.727915in}{2.336747in}}%
\pgfpathlineto{\pgfqpoint{2.719112in}{2.348926in}}%
\pgfpathlineto{\pgfqpoint{2.705823in}{2.358438in}}%
\pgfpathlineto{\pgfqpoint{2.692536in}{2.367989in}}%
\pgfpathlineto{\pgfqpoint{2.679251in}{2.377579in}}%
\pgfpathlineto{\pgfqpoint{2.665968in}{2.387207in}}%
\pgfpathlineto{\pgfqpoint{2.674817in}{2.374656in}}%
\pgfpathlineto{\pgfqpoint{2.683638in}{2.362631in}}%
\pgfpathlineto{\pgfqpoint{2.692432in}{2.351123in}}%
\pgfpathlineto{\pgfqpoint{2.701200in}{2.340121in}}%
\pgfpathclose%
\pgfusepath{fill}%
\end{pgfscope}%
\begin{pgfscope}%
\pgfpathrectangle{\pgfqpoint{1.254980in}{0.150000in}}{\pgfqpoint{5.490039in}{5.490039in}}%
\pgfusepath{clip}%
\pgfsetbuttcap%
\pgfsetroundjoin%
\definecolor{currentfill}{rgb}{0.277018,0.050344,0.375715}%
\pgfsetfillcolor{currentfill}%
\pgfsetfillopacity{0.700000}%
\pgfsetlinewidth{0.000000pt}%
\definecolor{currentstroke}{rgb}{0.000000,0.000000,0.000000}%
\pgfsetstrokecolor{currentstroke}%
\pgfsetdash{}{0pt}%
\pgfpathmoveto{\pgfqpoint{3.874615in}{1.721052in}}%
\pgfpathlineto{\pgfqpoint{3.887987in}{1.715686in}}%
\pgfpathlineto{\pgfqpoint{3.901365in}{1.710344in}}%
\pgfpathlineto{\pgfqpoint{3.914748in}{1.705029in}}%
\pgfpathlineto{\pgfqpoint{3.928136in}{1.699738in}}%
\pgfpathlineto{\pgfqpoint{3.920314in}{1.696954in}}%
\pgfpathlineto{\pgfqpoint{3.912483in}{1.694426in}}%
\pgfpathlineto{\pgfqpoint{3.904644in}{1.692162in}}%
\pgfpathlineto{\pgfqpoint{3.896796in}{1.690170in}}%
\pgfpathlineto{\pgfqpoint{3.883387in}{1.695748in}}%
\pgfpathlineto{\pgfqpoint{3.869983in}{1.701351in}}%
\pgfpathlineto{\pgfqpoint{3.856584in}{1.706979in}}%
\pgfpathlineto{\pgfqpoint{3.843191in}{1.712633in}}%
\pgfpathlineto{\pgfqpoint{3.851060in}{1.714332in}}%
\pgfpathlineto{\pgfqpoint{3.858921in}{1.716307in}}%
\pgfpathlineto{\pgfqpoint{3.866772in}{1.718550in}}%
\pgfpathlineto{\pgfqpoint{3.874615in}{1.721052in}}%
\pgfpathclose%
\pgfusepath{fill}%
\end{pgfscope}%
\begin{pgfscope}%
\pgfpathrectangle{\pgfqpoint{1.254980in}{0.150000in}}{\pgfqpoint{5.490039in}{5.490039in}}%
\pgfusepath{clip}%
\pgfsetbuttcap%
\pgfsetroundjoin%
\definecolor{currentfill}{rgb}{0.272594,0.025563,0.353093}%
\pgfsetfillcolor{currentfill}%
\pgfsetfillopacity{0.700000}%
\pgfsetlinewidth{0.000000pt}%
\definecolor{currentstroke}{rgb}{0.000000,0.000000,0.000000}%
\pgfsetstrokecolor{currentstroke}%
\pgfsetdash{}{0pt}%
\pgfpathmoveto{\pgfqpoint{4.650561in}{1.692136in}}%
\pgfpathlineto{\pgfqpoint{4.664122in}{1.689285in}}%
\pgfpathlineto{\pgfqpoint{4.677690in}{1.686457in}}%
\pgfpathlineto{\pgfqpoint{4.691266in}{1.683653in}}%
\pgfpathlineto{\pgfqpoint{4.704848in}{1.680872in}}%
\pgfpathlineto{\pgfqpoint{4.697316in}{1.671667in}}%
\pgfpathlineto{\pgfqpoint{4.689780in}{1.662533in}}%
\pgfpathlineto{\pgfqpoint{4.682239in}{1.653475in}}%
\pgfpathlineto{\pgfqpoint{4.674694in}{1.644499in}}%
\pgfpathlineto{\pgfqpoint{4.661103in}{1.647490in}}%
\pgfpathlineto{\pgfqpoint{4.647518in}{1.650504in}}%
\pgfpathlineto{\pgfqpoint{4.633940in}{1.653542in}}%
\pgfpathlineto{\pgfqpoint{4.620370in}{1.656603in}}%
\pgfpathlineto{\pgfqpoint{4.627924in}{1.665364in}}%
\pgfpathlineto{\pgfqpoint{4.635474in}{1.674210in}}%
\pgfpathlineto{\pgfqpoint{4.643020in}{1.683136in}}%
\pgfpathlineto{\pgfqpoint{4.650561in}{1.692136in}}%
\pgfpathclose%
\pgfusepath{fill}%
\end{pgfscope}%
\begin{pgfscope}%
\pgfpathrectangle{\pgfqpoint{1.254980in}{0.150000in}}{\pgfqpoint{5.490039in}{5.490039in}}%
\pgfusepath{clip}%
\pgfsetbuttcap%
\pgfsetroundjoin%
\definecolor{currentfill}{rgb}{0.269944,0.014625,0.341379}%
\pgfsetfillcolor{currentfill}%
\pgfsetfillopacity{0.700000}%
\pgfsetlinewidth{0.000000pt}%
\definecolor{currentstroke}{rgb}{0.000000,0.000000,0.000000}%
\pgfsetstrokecolor{currentstroke}%
\pgfsetdash{}{0pt}%
\pgfpathmoveto{\pgfqpoint{4.427670in}{1.663764in}}%
\pgfpathlineto{\pgfqpoint{4.441169in}{1.660196in}}%
\pgfpathlineto{\pgfqpoint{4.454675in}{1.656653in}}%
\pgfpathlineto{\pgfqpoint{4.468188in}{1.653133in}}%
\pgfpathlineto{\pgfqpoint{4.481708in}{1.649636in}}%
\pgfpathlineto{\pgfqpoint{4.474109in}{1.641843in}}%
\pgfpathlineto{\pgfqpoint{4.466505in}{1.634175in}}%
\pgfpathlineto{\pgfqpoint{4.458897in}{1.626637in}}%
\pgfpathlineto{\pgfqpoint{4.451284in}{1.619235in}}%
\pgfpathlineto{\pgfqpoint{4.437753in}{1.622967in}}%
\pgfpathlineto{\pgfqpoint{4.424228in}{1.626722in}}%
\pgfpathlineto{\pgfqpoint{4.410709in}{1.630501in}}%
\pgfpathlineto{\pgfqpoint{4.397197in}{1.634304in}}%
\pgfpathlineto{\pgfqpoint{4.404823in}{1.641465in}}%
\pgfpathlineto{\pgfqpoint{4.412443in}{1.648766in}}%
\pgfpathlineto{\pgfqpoint{4.420059in}{1.656201in}}%
\pgfpathlineto{\pgfqpoint{4.427670in}{1.663764in}}%
\pgfpathclose%
\pgfusepath{fill}%
\end{pgfscope}%
\begin{pgfscope}%
\pgfpathrectangle{\pgfqpoint{1.254980in}{0.150000in}}{\pgfqpoint{5.490039in}{5.490039in}}%
\pgfusepath{clip}%
\pgfsetbuttcap%
\pgfsetroundjoin%
\definecolor{currentfill}{rgb}{0.277134,0.185228,0.489898}%
\pgfsetfillcolor{currentfill}%
\pgfsetfillopacity{0.700000}%
\pgfsetlinewidth{0.000000pt}%
\definecolor{currentstroke}{rgb}{0.000000,0.000000,0.000000}%
\pgfsetstrokecolor{currentstroke}%
\pgfsetdash{}{0pt}%
\pgfpathmoveto{\pgfqpoint{5.519698in}{1.984504in}}%
\pgfpathlineto{\pgfqpoint{5.533546in}{1.984004in}}%
\pgfpathlineto{\pgfqpoint{5.547403in}{1.983527in}}%
\pgfpathlineto{\pgfqpoint{5.561269in}{1.983075in}}%
\pgfpathlineto{\pgfqpoint{5.554010in}{1.973005in}}%
\pgfpathlineto{\pgfqpoint{5.546744in}{1.962847in}}%
\pgfpathlineto{\pgfqpoint{5.539471in}{1.952600in}}%
\pgfpathlineto{\pgfqpoint{5.532190in}{1.942269in}}%
\pgfpathlineto{\pgfqpoint{5.518317in}{1.942816in}}%
\pgfpathlineto{\pgfqpoint{5.504452in}{1.943387in}}%
\pgfpathlineto{\pgfqpoint{5.490597in}{1.943981in}}%
\pgfpathlineto{\pgfqpoint{5.497882in}{1.954238in}}%
\pgfpathlineto{\pgfqpoint{5.505161in}{1.964413in}}%
\pgfpathlineto{\pgfqpoint{5.512433in}{1.974502in}}%
\pgfpathlineto{\pgfqpoint{5.519698in}{1.984504in}}%
\pgfpathclose%
\pgfusepath{fill}%
\end{pgfscope}%
\begin{pgfscope}%
\pgfpathrectangle{\pgfqpoint{1.254980in}{0.150000in}}{\pgfqpoint{5.490039in}{5.490039in}}%
\pgfusepath{clip}%
\pgfsetbuttcap%
\pgfsetroundjoin%
\definecolor{currentfill}{rgb}{0.280267,0.073417,0.397163}%
\pgfsetfillcolor{currentfill}%
\pgfsetfillopacity{0.700000}%
\pgfsetlinewidth{0.000000pt}%
\definecolor{currentstroke}{rgb}{0.000000,0.000000,0.000000}%
\pgfsetstrokecolor{currentstroke}%
\pgfsetdash{}{0pt}%
\pgfpathmoveto{\pgfqpoint{4.958150in}{1.770274in}}%
\pgfpathlineto{\pgfqpoint{4.971806in}{1.768353in}}%
\pgfpathlineto{\pgfqpoint{4.985469in}{1.766455in}}%
\pgfpathlineto{\pgfqpoint{4.999141in}{1.764581in}}%
\pgfpathlineto{\pgfqpoint{5.012820in}{1.762730in}}%
\pgfpathlineto{\pgfqpoint{5.005374in}{1.752367in}}%
\pgfpathlineto{\pgfqpoint{4.997923in}{1.742008in}}%
\pgfpathlineto{\pgfqpoint{4.990467in}{1.731657in}}%
\pgfpathlineto{\pgfqpoint{4.983006in}{1.721317in}}%
\pgfpathlineto{\pgfqpoint{4.969320in}{1.723341in}}%
\pgfpathlineto{\pgfqpoint{4.955641in}{1.725387in}}%
\pgfpathlineto{\pgfqpoint{4.941970in}{1.727457in}}%
\pgfpathlineto{\pgfqpoint{4.928307in}{1.729550in}}%
\pgfpathlineto{\pgfqpoint{4.935775in}{1.739712in}}%
\pgfpathlineto{\pgfqpoint{4.943238in}{1.749890in}}%
\pgfpathlineto{\pgfqpoint{4.950696in}{1.760078in}}%
\pgfpathlineto{\pgfqpoint{4.958150in}{1.770274in}}%
\pgfpathclose%
\pgfusepath{fill}%
\end{pgfscope}%
\begin{pgfscope}%
\pgfpathrectangle{\pgfqpoint{1.254980in}{0.150000in}}{\pgfqpoint{5.490039in}{5.490039in}}%
\pgfusepath{clip}%
\pgfsetbuttcap%
\pgfsetroundjoin%
\definecolor{currentfill}{rgb}{0.278012,0.180367,0.486697}%
\pgfsetfillcolor{currentfill}%
\pgfsetfillopacity{0.700000}%
\pgfsetlinewidth{0.000000pt}%
\definecolor{currentstroke}{rgb}{0.000000,0.000000,0.000000}%
\pgfsetstrokecolor{currentstroke}%
\pgfsetdash{}{0pt}%
\pgfpathmoveto{\pgfqpoint{3.299007in}{1.953183in}}%
\pgfpathlineto{\pgfqpoint{3.312292in}{1.945938in}}%
\pgfpathlineto{\pgfqpoint{3.325580in}{1.938722in}}%
\pgfpathlineto{\pgfqpoint{3.338872in}{1.931536in}}%
\pgfpathlineto{\pgfqpoint{3.352168in}{1.924379in}}%
\pgfpathlineto{\pgfqpoint{3.343978in}{1.928105in}}%
\pgfpathlineto{\pgfqpoint{3.335773in}{1.932218in}}%
\pgfpathlineto{\pgfqpoint{3.327551in}{1.936726in}}%
\pgfpathlineto{\pgfqpoint{3.319314in}{1.941638in}}%
\pgfpathlineto{\pgfqpoint{3.305986in}{1.949126in}}%
\pgfpathlineto{\pgfqpoint{3.292661in}{1.956642in}}%
\pgfpathlineto{\pgfqpoint{3.279340in}{1.964188in}}%
\pgfpathlineto{\pgfqpoint{3.266023in}{1.971763in}}%
\pgfpathlineto{\pgfqpoint{3.274294in}{1.966515in}}%
\pgfpathlineto{\pgfqpoint{3.282548in}{1.961675in}}%
\pgfpathlineto{\pgfqpoint{3.290786in}{1.957234in}}%
\pgfpathlineto{\pgfqpoint{3.299007in}{1.953183in}}%
\pgfpathclose%
\pgfusepath{fill}%
\end{pgfscope}%
\begin{pgfscope}%
\pgfpathrectangle{\pgfqpoint{1.254980in}{0.150000in}}{\pgfqpoint{5.490039in}{5.490039in}}%
\pgfusepath{clip}%
\pgfsetbuttcap%
\pgfsetroundjoin%
\definecolor{currentfill}{rgb}{0.156270,0.489624,0.557936}%
\pgfsetfillcolor{currentfill}%
\pgfsetfillopacity{0.700000}%
\pgfsetlinewidth{0.000000pt}%
\definecolor{currentstroke}{rgb}{0.000000,0.000000,0.000000}%
\pgfsetstrokecolor{currentstroke}%
\pgfsetdash{}{0pt}%
\pgfpathmoveto{\pgfqpoint{2.294657in}{2.673992in}}%
\pgfpathlineto{\pgfqpoint{2.307904in}{2.663125in}}%
\pgfpathlineto{\pgfqpoint{2.321151in}{2.652308in}}%
\pgfpathlineto{\pgfqpoint{2.334399in}{2.641541in}}%
\pgfpathlineto{\pgfqpoint{2.347648in}{2.630824in}}%
\pgfpathlineto{\pgfqpoint{2.338471in}{2.646190in}}%
\pgfpathlineto{\pgfqpoint{2.329261in}{2.662131in}}%
\pgfpathlineto{\pgfqpoint{2.320015in}{2.678658in}}%
\pgfpathlineto{\pgfqpoint{2.310735in}{2.695781in}}%
\pgfpathlineto{\pgfqpoint{2.297432in}{2.706889in}}%
\pgfpathlineto{\pgfqpoint{2.284130in}{2.718045in}}%
\pgfpathlineto{\pgfqpoint{2.270829in}{2.729252in}}%
\pgfpathlineto{\pgfqpoint{2.257528in}{2.740510in}}%
\pgfpathlineto{\pgfqpoint{2.266864in}{2.722989in}}%
\pgfpathlineto{\pgfqpoint{2.276163in}{2.706071in}}%
\pgfpathlineto{\pgfqpoint{2.285428in}{2.689742in}}%
\pgfpathlineto{\pgfqpoint{2.294657in}{2.673992in}}%
\pgfpathclose%
\pgfusepath{fill}%
\end{pgfscope}%
\begin{pgfscope}%
\pgfpathrectangle{\pgfqpoint{1.254980in}{0.150000in}}{\pgfqpoint{5.490039in}{5.490039in}}%
\pgfusepath{clip}%
\pgfsetbuttcap%
\pgfsetroundjoin%
\definecolor{currentfill}{rgb}{0.258965,0.251537,0.524736}%
\pgfsetfillcolor{currentfill}%
\pgfsetfillopacity{0.700000}%
\pgfsetlinewidth{0.000000pt}%
\definecolor{currentstroke}{rgb}{0.000000,0.000000,0.000000}%
\pgfsetstrokecolor{currentstroke}%
\pgfsetdash{}{0pt}%
\pgfpathmoveto{\pgfqpoint{3.053428in}{2.097088in}}%
\pgfpathlineto{\pgfqpoint{3.066690in}{2.089021in}}%
\pgfpathlineto{\pgfqpoint{3.079955in}{2.080985in}}%
\pgfpathlineto{\pgfqpoint{3.093224in}{2.072982in}}%
\pgfpathlineto{\pgfqpoint{3.106496in}{2.065010in}}%
\pgfpathlineto{\pgfqpoint{3.098101in}{2.071696in}}%
\pgfpathlineto{\pgfqpoint{3.089687in}{2.078819in}}%
\pgfpathlineto{\pgfqpoint{3.081253in}{2.086391in}}%
\pgfpathlineto{\pgfqpoint{3.072798in}{2.094420in}}%
\pgfpathlineto{\pgfqpoint{3.059489in}{2.102738in}}%
\pgfpathlineto{\pgfqpoint{3.046183in}{2.111088in}}%
\pgfpathlineto{\pgfqpoint{3.032880in}{2.119470in}}%
\pgfpathlineto{\pgfqpoint{3.019580in}{2.127884in}}%
\pgfpathlineto{\pgfqpoint{3.028073in}{2.119503in}}%
\pgfpathlineto{\pgfqpoint{3.036545in}{2.111583in}}%
\pgfpathlineto{\pgfqpoint{3.044997in}{2.104114in}}%
\pgfpathlineto{\pgfqpoint{3.053428in}{2.097088in}}%
\pgfpathclose%
\pgfusepath{fill}%
\end{pgfscope}%
\begin{pgfscope}%
\pgfpathrectangle{\pgfqpoint{1.254980in}{0.150000in}}{\pgfqpoint{5.490039in}{5.490039in}}%
\pgfusepath{clip}%
\pgfsetbuttcap%
\pgfsetroundjoin%
\definecolor{currentfill}{rgb}{0.278791,0.062145,0.386592}%
\pgfsetfillcolor{currentfill}%
\pgfsetfillopacity{0.700000}%
\pgfsetlinewidth{0.000000pt}%
\definecolor{currentstroke}{rgb}{0.000000,0.000000,0.000000}%
\pgfsetstrokecolor{currentstroke}%
\pgfsetdash{}{0pt}%
\pgfpathmoveto{\pgfqpoint{4.873731in}{1.738158in}}%
\pgfpathlineto{\pgfqpoint{4.887364in}{1.735971in}}%
\pgfpathlineto{\pgfqpoint{4.901004in}{1.733807in}}%
\pgfpathlineto{\pgfqpoint{4.914652in}{1.731667in}}%
\pgfpathlineto{\pgfqpoint{4.928307in}{1.729550in}}%
\pgfpathlineto{\pgfqpoint{4.920835in}{1.719408in}}%
\pgfpathlineto{\pgfqpoint{4.913358in}{1.709289in}}%
\pgfpathlineto{\pgfqpoint{4.905877in}{1.699199in}}%
\pgfpathlineto{\pgfqpoint{4.898392in}{1.689141in}}%
\pgfpathlineto{\pgfqpoint{4.884729in}{1.691443in}}%
\pgfpathlineto{\pgfqpoint{4.871073in}{1.693768in}}%
\pgfpathlineto{\pgfqpoint{4.857425in}{1.696117in}}%
\pgfpathlineto{\pgfqpoint{4.843785in}{1.698489in}}%
\pgfpathlineto{\pgfqpoint{4.851278in}{1.708357in}}%
\pgfpathlineto{\pgfqpoint{4.858767in}{1.718260in}}%
\pgfpathlineto{\pgfqpoint{4.866251in}{1.728195in}}%
\pgfpathlineto{\pgfqpoint{4.873731in}{1.738158in}}%
\pgfpathclose%
\pgfusepath{fill}%
\end{pgfscope}%
\begin{pgfscope}%
\pgfpathrectangle{\pgfqpoint{1.254980in}{0.150000in}}{\pgfqpoint{5.490039in}{5.490039in}}%
\pgfusepath{clip}%
\pgfsetbuttcap%
\pgfsetroundjoin%
\definecolor{currentfill}{rgb}{0.279574,0.170599,0.479997}%
\pgfsetfillcolor{currentfill}%
\pgfsetfillopacity{0.700000}%
\pgfsetlinewidth{0.000000pt}%
\definecolor{currentstroke}{rgb}{0.000000,0.000000,0.000000}%
\pgfsetstrokecolor{currentstroke}%
\pgfsetdash{}{0pt}%
\pgfpathmoveto{\pgfqpoint{5.435262in}{1.946596in}}%
\pgfpathlineto{\pgfqpoint{5.449083in}{1.945907in}}%
\pgfpathlineto{\pgfqpoint{5.462912in}{1.945241in}}%
\pgfpathlineto{\pgfqpoint{5.476750in}{1.944599in}}%
\pgfpathlineto{\pgfqpoint{5.490597in}{1.943981in}}%
\pgfpathlineto{\pgfqpoint{5.483304in}{1.933643in}}%
\pgfpathlineto{\pgfqpoint{5.476004in}{1.923227in}}%
\pgfpathlineto{\pgfqpoint{5.468698in}{1.912734in}}%
\pgfpathlineto{\pgfqpoint{5.461385in}{1.902167in}}%
\pgfpathlineto{\pgfqpoint{5.447532in}{1.902892in}}%
\pgfpathlineto{\pgfqpoint{5.433687in}{1.903642in}}%
\pgfpathlineto{\pgfqpoint{5.419851in}{1.904415in}}%
\pgfpathlineto{\pgfqpoint{5.406024in}{1.905211in}}%
\pgfpathlineto{\pgfqpoint{5.413343in}{1.915666in}}%
\pgfpathlineto{\pgfqpoint{5.420656in}{1.926049in}}%
\pgfpathlineto{\pgfqpoint{5.427963in}{1.936360in}}%
\pgfpathlineto{\pgfqpoint{5.435262in}{1.946596in}}%
\pgfpathclose%
\pgfusepath{fill}%
\end{pgfscope}%
\begin{pgfscope}%
\pgfpathrectangle{\pgfqpoint{1.254980in}{0.150000in}}{\pgfqpoint{5.490039in}{5.490039in}}%
\pgfusepath{clip}%
\pgfsetbuttcap%
\pgfsetroundjoin%
\definecolor{currentfill}{rgb}{0.220057,0.343307,0.549413}%
\pgfsetfillcolor{currentfill}%
\pgfsetfillopacity{0.700000}%
\pgfsetlinewidth{0.000000pt}%
\definecolor{currentstroke}{rgb}{0.000000,0.000000,0.000000}%
\pgfsetstrokecolor{currentstroke}%
\pgfsetdash{}{0pt}%
\pgfpathmoveto{\pgfqpoint{2.754164in}{2.303304in}}%
\pgfpathlineto{\pgfqpoint{2.767410in}{2.294194in}}%
\pgfpathlineto{\pgfqpoint{2.780659in}{2.285122in}}%
\pgfpathlineto{\pgfqpoint{2.793910in}{2.276087in}}%
\pgfpathlineto{\pgfqpoint{2.807164in}{2.267088in}}%
\pgfpathlineto{\pgfqpoint{2.798484in}{2.277371in}}%
\pgfpathlineto{\pgfqpoint{2.789778in}{2.288152in}}%
\pgfpathlineto{\pgfqpoint{2.781047in}{2.299441in}}%
\pgfpathlineto{\pgfqpoint{2.772290in}{2.311249in}}%
\pgfpathlineto{\pgfqpoint{2.758993in}{2.320612in}}%
\pgfpathlineto{\pgfqpoint{2.745697in}{2.330013in}}%
\pgfpathlineto{\pgfqpoint{2.732404in}{2.339450in}}%
\pgfpathlineto{\pgfqpoint{2.719112in}{2.348926in}}%
\pgfpathlineto{\pgfqpoint{2.727915in}{2.336747in}}%
\pgfpathlineto{\pgfqpoint{2.736690in}{2.325091in}}%
\pgfpathlineto{\pgfqpoint{2.745440in}{2.313946in}}%
\pgfpathlineto{\pgfqpoint{2.754164in}{2.303304in}}%
\pgfpathclose%
\pgfusepath{fill}%
\end{pgfscope}%
\begin{pgfscope}%
\pgfpathrectangle{\pgfqpoint{1.254980in}{0.150000in}}{\pgfqpoint{5.490039in}{5.490039in}}%
\pgfusepath{clip}%
\pgfsetbuttcap%
\pgfsetroundjoin%
\definecolor{currentfill}{rgb}{0.283229,0.120777,0.440584}%
\pgfsetfillcolor{currentfill}%
\pgfsetfillopacity{0.700000}%
\pgfsetlinewidth{0.000000pt}%
\definecolor{currentstroke}{rgb}{0.000000,0.000000,0.000000}%
\pgfsetstrokecolor{currentstroke}%
\pgfsetdash{}{0pt}%
\pgfpathmoveto{\pgfqpoint{3.544300in}{1.833260in}}%
\pgfpathlineto{\pgfqpoint{3.557622in}{1.826782in}}%
\pgfpathlineto{\pgfqpoint{3.570949in}{1.820331in}}%
\pgfpathlineto{\pgfqpoint{3.584279in}{1.813908in}}%
\pgfpathlineto{\pgfqpoint{3.597615in}{1.807511in}}%
\pgfpathlineto{\pgfqpoint{3.589595in}{1.808542in}}%
\pgfpathlineto{\pgfqpoint{3.581564in}{1.809910in}}%
\pgfpathlineto{\pgfqpoint{3.573520in}{1.811623in}}%
\pgfpathlineto{\pgfqpoint{3.565463in}{1.813691in}}%
\pgfpathlineto{\pgfqpoint{3.552100in}{1.820402in}}%
\pgfpathlineto{\pgfqpoint{3.538742in}{1.827141in}}%
\pgfpathlineto{\pgfqpoint{3.525388in}{1.833907in}}%
\pgfpathlineto{\pgfqpoint{3.512038in}{1.840700in}}%
\pgfpathlineto{\pgfqpoint{3.520123in}{1.838312in}}%
\pgfpathlineto{\pgfqpoint{3.528195in}{1.836282in}}%
\pgfpathlineto{\pgfqpoint{3.536254in}{1.834601in}}%
\pgfpathlineto{\pgfqpoint{3.544300in}{1.833260in}}%
\pgfpathclose%
\pgfusepath{fill}%
\end{pgfscope}%
\begin{pgfscope}%
\pgfpathrectangle{\pgfqpoint{1.254980in}{0.150000in}}{\pgfqpoint{5.490039in}{5.490039in}}%
\pgfusepath{clip}%
\pgfsetbuttcap%
\pgfsetroundjoin%
\definecolor{currentfill}{rgb}{0.271305,0.019942,0.347269}%
\pgfsetfillcolor{currentfill}%
\pgfsetfillopacity{0.700000}%
\pgfsetlinewidth{0.000000pt}%
\definecolor{currentstroke}{rgb}{0.000000,0.000000,0.000000}%
\pgfsetstrokecolor{currentstroke}%
\pgfsetdash{}{0pt}%
\pgfpathmoveto{\pgfqpoint{4.566157in}{1.669083in}}%
\pgfpathlineto{\pgfqpoint{4.579700in}{1.665928in}}%
\pgfpathlineto{\pgfqpoint{4.593250in}{1.662796in}}%
\pgfpathlineto{\pgfqpoint{4.606806in}{1.659688in}}%
\pgfpathlineto{\pgfqpoint{4.620370in}{1.656603in}}%
\pgfpathlineto{\pgfqpoint{4.612811in}{1.647933in}}%
\pgfpathlineto{\pgfqpoint{4.605247in}{1.639358in}}%
\pgfpathlineto{\pgfqpoint{4.597680in}{1.630884in}}%
\pgfpathlineto{\pgfqpoint{4.590107in}{1.622517in}}%
\pgfpathlineto{\pgfqpoint{4.576534in}{1.625824in}}%
\pgfpathlineto{\pgfqpoint{4.562967in}{1.629155in}}%
\pgfpathlineto{\pgfqpoint{4.549407in}{1.632510in}}%
\pgfpathlineto{\pgfqpoint{4.535853in}{1.635888in}}%
\pgfpathlineto{\pgfqpoint{4.543436in}{1.644028in}}%
\pgfpathlineto{\pgfqpoint{4.551014in}{1.652277in}}%
\pgfpathlineto{\pgfqpoint{4.558588in}{1.660631in}}%
\pgfpathlineto{\pgfqpoint{4.566157in}{1.669083in}}%
\pgfpathclose%
\pgfusepath{fill}%
\end{pgfscope}%
\begin{pgfscope}%
\pgfpathrectangle{\pgfqpoint{1.254980in}{0.150000in}}{\pgfqpoint{5.490039in}{5.490039in}}%
\pgfusepath{clip}%
\pgfsetbuttcap%
\pgfsetroundjoin%
\definecolor{currentfill}{rgb}{0.280894,0.078907,0.402329}%
\pgfsetfillcolor{currentfill}%
\pgfsetfillopacity{0.700000}%
\pgfsetlinewidth{0.000000pt}%
\definecolor{currentstroke}{rgb}{0.000000,0.000000,0.000000}%
\pgfsetstrokecolor{currentstroke}%
\pgfsetdash{}{0pt}%
\pgfpathmoveto{\pgfqpoint{3.736225in}{1.758779in}}%
\pgfpathlineto{\pgfqpoint{3.749578in}{1.752921in}}%
\pgfpathlineto{\pgfqpoint{3.762937in}{1.747088in}}%
\pgfpathlineto{\pgfqpoint{3.776300in}{1.741282in}}%
\pgfpathlineto{\pgfqpoint{3.789668in}{1.735501in}}%
\pgfpathlineto{\pgfqpoint{3.781766in}{1.734379in}}%
\pgfpathlineto{\pgfqpoint{3.773855in}{1.733551in}}%
\pgfpathlineto{\pgfqpoint{3.765934in}{1.733026in}}%
\pgfpathlineto{\pgfqpoint{3.758002in}{1.732810in}}%
\pgfpathlineto{\pgfqpoint{3.744610in}{1.738892in}}%
\pgfpathlineto{\pgfqpoint{3.731223in}{1.744999in}}%
\pgfpathlineto{\pgfqpoint{3.717841in}{1.751133in}}%
\pgfpathlineto{\pgfqpoint{3.704464in}{1.757292in}}%
\pgfpathlineto{\pgfqpoint{3.712420in}{1.757202in}}%
\pgfpathlineto{\pgfqpoint{3.720366in}{1.757425in}}%
\pgfpathlineto{\pgfqpoint{3.728301in}{1.757953in}}%
\pgfpathlineto{\pgfqpoint{3.736225in}{1.758779in}}%
\pgfpathclose%
\pgfusepath{fill}%
\end{pgfscope}%
\begin{pgfscope}%
\pgfpathrectangle{\pgfqpoint{1.254980in}{0.150000in}}{\pgfqpoint{5.490039in}{5.490039in}}%
\pgfusepath{clip}%
\pgfsetbuttcap%
\pgfsetroundjoin%
\definecolor{currentfill}{rgb}{0.281887,0.150881,0.465405}%
\pgfsetfillcolor{currentfill}%
\pgfsetfillopacity{0.700000}%
\pgfsetlinewidth{0.000000pt}%
\definecolor{currentstroke}{rgb}{0.000000,0.000000,0.000000}%
\pgfsetstrokecolor{currentstroke}%
\pgfsetdash{}{0pt}%
\pgfpathmoveto{\pgfqpoint{5.350801in}{1.908634in}}%
\pgfpathlineto{\pgfqpoint{5.364594in}{1.907743in}}%
\pgfpathlineto{\pgfqpoint{5.378395in}{1.906875in}}%
\pgfpathlineto{\pgfqpoint{5.392205in}{1.906031in}}%
\pgfpathlineto{\pgfqpoint{5.406024in}{1.905211in}}%
\pgfpathlineto{\pgfqpoint{5.398698in}{1.894689in}}%
\pgfpathlineto{\pgfqpoint{5.391365in}{1.884101in}}%
\pgfpathlineto{\pgfqpoint{5.384027in}{1.873450in}}%
\pgfpathlineto{\pgfqpoint{5.376683in}{1.862739in}}%
\pgfpathlineto{\pgfqpoint{5.362857in}{1.863680in}}%
\pgfpathlineto{\pgfqpoint{5.349041in}{1.864644in}}%
\pgfpathlineto{\pgfqpoint{5.335233in}{1.865632in}}%
\pgfpathlineto{\pgfqpoint{5.321433in}{1.866644in}}%
\pgfpathlineto{\pgfqpoint{5.328784in}{1.877229in}}%
\pgfpathlineto{\pgfqpoint{5.336129in}{1.887758in}}%
\pgfpathlineto{\pgfqpoint{5.343468in}{1.898227in}}%
\pgfpathlineto{\pgfqpoint{5.350801in}{1.908634in}}%
\pgfpathclose%
\pgfusepath{fill}%
\end{pgfscope}%
\begin{pgfscope}%
\pgfpathrectangle{\pgfqpoint{1.254980in}{0.150000in}}{\pgfqpoint{5.490039in}{5.490039in}}%
\pgfusepath{clip}%
\pgfsetbuttcap%
\pgfsetroundjoin%
\definecolor{currentfill}{rgb}{0.269944,0.014625,0.341379}%
\pgfsetfillcolor{currentfill}%
\pgfsetfillopacity{0.700000}%
\pgfsetlinewidth{0.000000pt}%
\definecolor{currentstroke}{rgb}{0.000000,0.000000,0.000000}%
\pgfsetstrokecolor{currentstroke}%
\pgfsetdash{}{0pt}%
\pgfpathmoveto{\pgfqpoint{4.204832in}{1.657649in}}%
\pgfpathlineto{\pgfqpoint{4.218283in}{1.653310in}}%
\pgfpathlineto{\pgfqpoint{4.231739in}{1.648997in}}%
\pgfpathlineto{\pgfqpoint{4.245202in}{1.644707in}}%
\pgfpathlineto{\pgfqpoint{4.258671in}{1.640441in}}%
\pgfpathlineto{\pgfqpoint{4.250991in}{1.634583in}}%
\pgfpathlineto{\pgfqpoint{4.243305in}{1.628909in}}%
\pgfpathlineto{\pgfqpoint{4.235614in}{1.623426in}}%
\pgfpathlineto{\pgfqpoint{4.227916in}{1.618140in}}%
\pgfpathlineto{\pgfqpoint{4.214432in}{1.622666in}}%
\pgfpathlineto{\pgfqpoint{4.200954in}{1.627217in}}%
\pgfpathlineto{\pgfqpoint{4.187482in}{1.631791in}}%
\pgfpathlineto{\pgfqpoint{4.174016in}{1.636390in}}%
\pgfpathlineto{\pgfqpoint{4.181729in}{1.641411in}}%
\pgfpathlineto{\pgfqpoint{4.189436in}{1.646632in}}%
\pgfpathlineto{\pgfqpoint{4.197137in}{1.652046in}}%
\pgfpathlineto{\pgfqpoint{4.204832in}{1.657649in}}%
\pgfpathclose%
\pgfusepath{fill}%
\end{pgfscope}%
\begin{pgfscope}%
\pgfpathrectangle{\pgfqpoint{1.254980in}{0.150000in}}{\pgfqpoint{5.490039in}{5.490039in}}%
\pgfusepath{clip}%
\pgfsetbuttcap%
\pgfsetroundjoin%
\definecolor{currentfill}{rgb}{0.162142,0.474838,0.558140}%
\pgfsetfillcolor{currentfill}%
\pgfsetfillopacity{0.700000}%
\pgfsetlinewidth{0.000000pt}%
\definecolor{currentstroke}{rgb}{0.000000,0.000000,0.000000}%
\pgfsetstrokecolor{currentstroke}%
\pgfsetdash{}{0pt}%
\pgfpathmoveto{\pgfqpoint{2.347648in}{2.630824in}}%
\pgfpathlineto{\pgfqpoint{2.360897in}{2.620155in}}%
\pgfpathlineto{\pgfqpoint{2.374148in}{2.609535in}}%
\pgfpathlineto{\pgfqpoint{2.387399in}{2.598962in}}%
\pgfpathlineto{\pgfqpoint{2.400651in}{2.588437in}}%
\pgfpathlineto{\pgfqpoint{2.391526in}{2.603421in}}%
\pgfpathlineto{\pgfqpoint{2.382369in}{2.618976in}}%
\pgfpathlineto{\pgfqpoint{2.373177in}{2.635111in}}%
\pgfpathlineto{\pgfqpoint{2.363951in}{2.651839in}}%
\pgfpathlineto{\pgfqpoint{2.350646in}{2.662753in}}%
\pgfpathlineto{\pgfqpoint{2.337342in}{2.673714in}}%
\pgfpathlineto{\pgfqpoint{2.324038in}{2.684723in}}%
\pgfpathlineto{\pgfqpoint{2.310735in}{2.695781in}}%
\pgfpathlineto{\pgfqpoint{2.320015in}{2.678658in}}%
\pgfpathlineto{\pgfqpoint{2.329261in}{2.662131in}}%
\pgfpathlineto{\pgfqpoint{2.338471in}{2.646190in}}%
\pgfpathlineto{\pgfqpoint{2.347648in}{2.630824in}}%
\pgfpathclose%
\pgfusepath{fill}%
\end{pgfscope}%
\begin{pgfscope}%
\pgfpathrectangle{\pgfqpoint{1.254980in}{0.150000in}}{\pgfqpoint{5.490039in}{5.490039in}}%
\pgfusepath{clip}%
\pgfsetbuttcap%
\pgfsetroundjoin%
\definecolor{currentfill}{rgb}{0.272594,0.025563,0.353093}%
\pgfsetfillcolor{currentfill}%
\pgfsetfillopacity{0.700000}%
\pgfsetlinewidth{0.000000pt}%
\definecolor{currentstroke}{rgb}{0.000000,0.000000,0.000000}%
\pgfsetstrokecolor{currentstroke}%
\pgfsetdash{}{0pt}%
\pgfpathmoveto{\pgfqpoint{4.066497in}{1.674055in}}%
\pgfpathlineto{\pgfqpoint{4.079917in}{1.669262in}}%
\pgfpathlineto{\pgfqpoint{4.093342in}{1.664493in}}%
\pgfpathlineto{\pgfqpoint{4.106773in}{1.659748in}}%
\pgfpathlineto{\pgfqpoint{4.120210in}{1.655028in}}%
\pgfpathlineto{\pgfqpoint{4.112473in}{1.650484in}}%
\pgfpathlineto{\pgfqpoint{4.104730in}{1.646158in}}%
\pgfpathlineto{\pgfqpoint{4.096980in}{1.642056in}}%
\pgfpathlineto{\pgfqpoint{4.089222in}{1.638186in}}%
\pgfpathlineto{\pgfqpoint{4.075768in}{1.643181in}}%
\pgfpathlineto{\pgfqpoint{4.062319in}{1.648199in}}%
\pgfpathlineto{\pgfqpoint{4.048876in}{1.653242in}}%
\pgfpathlineto{\pgfqpoint{4.035438in}{1.658309in}}%
\pgfpathlineto{\pgfqpoint{4.043214in}{1.661900in}}%
\pgfpathlineto{\pgfqpoint{4.050982in}{1.665726in}}%
\pgfpathlineto{\pgfqpoint{4.058743in}{1.669780in}}%
\pgfpathlineto{\pgfqpoint{4.066497in}{1.674055in}}%
\pgfpathclose%
\pgfusepath{fill}%
\end{pgfscope}%
\begin{pgfscope}%
\pgfpathrectangle{\pgfqpoint{1.254980in}{0.150000in}}{\pgfqpoint{5.490039in}{5.490039in}}%
\pgfusepath{clip}%
\pgfsetbuttcap%
\pgfsetroundjoin%
\definecolor{currentfill}{rgb}{0.282884,0.135920,0.453427}%
\pgfsetfillcolor{currentfill}%
\pgfsetfillopacity{0.700000}%
\pgfsetlinewidth{0.000000pt}%
\definecolor{currentstroke}{rgb}{0.000000,0.000000,0.000000}%
\pgfsetstrokecolor{currentstroke}%
\pgfsetdash{}{0pt}%
\pgfpathmoveto{\pgfqpoint{5.266321in}{1.870926in}}%
\pgfpathlineto{\pgfqpoint{5.280087in}{1.869820in}}%
\pgfpathlineto{\pgfqpoint{5.293860in}{1.868738in}}%
\pgfpathlineto{\pgfqpoint{5.307643in}{1.867679in}}%
\pgfpathlineto{\pgfqpoint{5.321433in}{1.866644in}}%
\pgfpathlineto{\pgfqpoint{5.314077in}{1.856004in}}%
\pgfpathlineto{\pgfqpoint{5.306714in}{1.845314in}}%
\pgfpathlineto{\pgfqpoint{5.299345in}{1.834575in}}%
\pgfpathlineto{\pgfqpoint{5.291971in}{1.823791in}}%
\pgfpathlineto{\pgfqpoint{5.278174in}{1.824960in}}%
\pgfpathlineto{\pgfqpoint{5.264385in}{1.826153in}}%
\pgfpathlineto{\pgfqpoint{5.250605in}{1.827369in}}%
\pgfpathlineto{\pgfqpoint{5.236833in}{1.828608in}}%
\pgfpathlineto{\pgfqpoint{5.244213in}{1.839253in}}%
\pgfpathlineto{\pgfqpoint{5.251588in}{1.849857in}}%
\pgfpathlineto{\pgfqpoint{5.258958in}{1.860415in}}%
\pgfpathlineto{\pgfqpoint{5.266321in}{1.870926in}}%
\pgfpathclose%
\pgfusepath{fill}%
\end{pgfscope}%
\begin{pgfscope}%
\pgfpathrectangle{\pgfqpoint{1.254980in}{0.150000in}}{\pgfqpoint{5.490039in}{5.490039in}}%
\pgfusepath{clip}%
\pgfsetbuttcap%
\pgfsetroundjoin%
\definecolor{currentfill}{rgb}{0.276022,0.044167,0.370164}%
\pgfsetfillcolor{currentfill}%
\pgfsetfillopacity{0.700000}%
\pgfsetlinewidth{0.000000pt}%
\definecolor{currentstroke}{rgb}{0.000000,0.000000,0.000000}%
\pgfsetstrokecolor{currentstroke}%
\pgfsetdash{}{0pt}%
\pgfpathmoveto{\pgfqpoint{4.789299in}{1.708211in}}%
\pgfpathlineto{\pgfqpoint{4.802909in}{1.705745in}}%
\pgfpathlineto{\pgfqpoint{4.816527in}{1.703303in}}%
\pgfpathlineto{\pgfqpoint{4.830152in}{1.700884in}}%
\pgfpathlineto{\pgfqpoint{4.843785in}{1.698489in}}%
\pgfpathlineto{\pgfqpoint{4.836288in}{1.688662in}}%
\pgfpathlineto{\pgfqpoint{4.828786in}{1.678880in}}%
\pgfpathlineto{\pgfqpoint{4.821280in}{1.669149in}}%
\pgfpathlineto{\pgfqpoint{4.813769in}{1.659472in}}%
\pgfpathlineto{\pgfqpoint{4.800128in}{1.662065in}}%
\pgfpathlineto{\pgfqpoint{4.786495in}{1.664682in}}%
\pgfpathlineto{\pgfqpoint{4.772869in}{1.667322in}}%
\pgfpathlineto{\pgfqpoint{4.759250in}{1.669985in}}%
\pgfpathlineto{\pgfqpoint{4.766769in}{1.679459in}}%
\pgfpathlineto{\pgfqpoint{4.774283in}{1.688991in}}%
\pgfpathlineto{\pgfqpoint{4.781793in}{1.698577in}}%
\pgfpathlineto{\pgfqpoint{4.789299in}{1.708211in}}%
\pgfpathclose%
\pgfusepath{fill}%
\end{pgfscope}%
\begin{pgfscope}%
\pgfpathrectangle{\pgfqpoint{1.254980in}{0.150000in}}{\pgfqpoint{5.490039in}{5.490039in}}%
\pgfusepath{clip}%
\pgfsetbuttcap%
\pgfsetroundjoin%
\definecolor{currentfill}{rgb}{0.269944,0.014625,0.341379}%
\pgfsetfillcolor{currentfill}%
\pgfsetfillopacity{0.700000}%
\pgfsetlinewidth{0.000000pt}%
\definecolor{currentstroke}{rgb}{0.000000,0.000000,0.000000}%
\pgfsetstrokecolor{currentstroke}%
\pgfsetdash{}{0pt}%
\pgfpathmoveto{\pgfqpoint{4.343214in}{1.649752in}}%
\pgfpathlineto{\pgfqpoint{4.356700in}{1.645854in}}%
\pgfpathlineto{\pgfqpoint{4.370193in}{1.641980in}}%
\pgfpathlineto{\pgfqpoint{4.383692in}{1.638130in}}%
\pgfpathlineto{\pgfqpoint{4.397197in}{1.634304in}}%
\pgfpathlineto{\pgfqpoint{4.389567in}{1.627288in}}%
\pgfpathlineto{\pgfqpoint{4.381931in}{1.620424in}}%
\pgfpathlineto{\pgfqpoint{4.374290in}{1.613718in}}%
\pgfpathlineto{\pgfqpoint{4.366644in}{1.607177in}}%
\pgfpathlineto{\pgfqpoint{4.353126in}{1.611252in}}%
\pgfpathlineto{\pgfqpoint{4.339613in}{1.615350in}}%
\pgfpathlineto{\pgfqpoint{4.326107in}{1.619472in}}%
\pgfpathlineto{\pgfqpoint{4.312608in}{1.623618in}}%
\pgfpathlineto{\pgfqpoint{4.320267in}{1.629907in}}%
\pgfpathlineto{\pgfqpoint{4.327922in}{1.636363in}}%
\pgfpathlineto{\pgfqpoint{4.335571in}{1.642980in}}%
\pgfpathlineto{\pgfqpoint{4.343214in}{1.649752in}}%
\pgfpathclose%
\pgfusepath{fill}%
\end{pgfscope}%
\begin{pgfscope}%
\pgfpathrectangle{\pgfqpoint{1.254980in}{0.150000in}}{\pgfqpoint{5.490039in}{5.490039in}}%
\pgfusepath{clip}%
\pgfsetbuttcap%
\pgfsetroundjoin%
\definecolor{currentfill}{rgb}{0.279574,0.170599,0.479997}%
\pgfsetfillcolor{currentfill}%
\pgfsetfillopacity{0.700000}%
\pgfsetlinewidth{0.000000pt}%
\definecolor{currentstroke}{rgb}{0.000000,0.000000,0.000000}%
\pgfsetstrokecolor{currentstroke}%
\pgfsetdash{}{0pt}%
\pgfpathmoveto{\pgfqpoint{3.352168in}{1.924379in}}%
\pgfpathlineto{\pgfqpoint{3.365468in}{1.917250in}}%
\pgfpathlineto{\pgfqpoint{3.378772in}{1.910150in}}%
\pgfpathlineto{\pgfqpoint{3.392080in}{1.903079in}}%
\pgfpathlineto{\pgfqpoint{3.405392in}{1.896036in}}%
\pgfpathlineto{\pgfqpoint{3.397233in}{1.899438in}}%
\pgfpathlineto{\pgfqpoint{3.389060in}{1.903223in}}%
\pgfpathlineto{\pgfqpoint{3.380870in}{1.907399in}}%
\pgfpathlineto{\pgfqpoint{3.372666in}{1.911976in}}%
\pgfpathlineto{\pgfqpoint{3.359322in}{1.919349in}}%
\pgfpathlineto{\pgfqpoint{3.345982in}{1.926750in}}%
\pgfpathlineto{\pgfqpoint{3.332646in}{1.934180in}}%
\pgfpathlineto{\pgfqpoint{3.319314in}{1.941638in}}%
\pgfpathlineto{\pgfqpoint{3.327551in}{1.936726in}}%
\pgfpathlineto{\pgfqpoint{3.335773in}{1.932218in}}%
\pgfpathlineto{\pgfqpoint{3.343978in}{1.928105in}}%
\pgfpathlineto{\pgfqpoint{3.352168in}{1.924379in}}%
\pgfpathclose%
\pgfusepath{fill}%
\end{pgfscope}%
\begin{pgfscope}%
\pgfpathrectangle{\pgfqpoint{1.254980in}{0.150000in}}{\pgfqpoint{5.490039in}{5.490039in}}%
\pgfusepath{clip}%
\pgfsetbuttcap%
\pgfsetroundjoin%
\definecolor{currentfill}{rgb}{0.283197,0.115680,0.436115}%
\pgfsetfillcolor{currentfill}%
\pgfsetfillopacity{0.700000}%
\pgfsetlinewidth{0.000000pt}%
\definecolor{currentstroke}{rgb}{0.000000,0.000000,0.000000}%
\pgfsetstrokecolor{currentstroke}%
\pgfsetdash{}{0pt}%
\pgfpathmoveto{\pgfqpoint{5.181829in}{1.833802in}}%
\pgfpathlineto{\pgfqpoint{5.195567in}{1.832468in}}%
\pgfpathlineto{\pgfqpoint{5.209314in}{1.831158in}}%
\pgfpathlineto{\pgfqpoint{5.223069in}{1.829871in}}%
\pgfpathlineto{\pgfqpoint{5.236833in}{1.828608in}}%
\pgfpathlineto{\pgfqpoint{5.229447in}{1.817925in}}%
\pgfpathlineto{\pgfqpoint{5.222055in}{1.807206in}}%
\pgfpathlineto{\pgfqpoint{5.214659in}{1.796455in}}%
\pgfpathlineto{\pgfqpoint{5.207257in}{1.785676in}}%
\pgfpathlineto{\pgfqpoint{5.193487in}{1.787086in}}%
\pgfpathlineto{\pgfqpoint{5.179725in}{1.788520in}}%
\pgfpathlineto{\pgfqpoint{5.165972in}{1.789976in}}%
\pgfpathlineto{\pgfqpoint{5.152227in}{1.791457in}}%
\pgfpathlineto{\pgfqpoint{5.159635in}{1.802084in}}%
\pgfpathlineto{\pgfqpoint{5.167038in}{1.812686in}}%
\pgfpathlineto{\pgfqpoint{5.174436in}{1.823260in}}%
\pgfpathlineto{\pgfqpoint{5.181829in}{1.833802in}}%
\pgfpathclose%
\pgfusepath{fill}%
\end{pgfscope}%
\begin{pgfscope}%
\pgfpathrectangle{\pgfqpoint{1.254980in}{0.150000in}}{\pgfqpoint{5.490039in}{5.490039in}}%
\pgfusepath{clip}%
\pgfsetbuttcap%
\pgfsetroundjoin%
\definecolor{currentfill}{rgb}{0.276022,0.044167,0.370164}%
\pgfsetfillcolor{currentfill}%
\pgfsetfillopacity{0.700000}%
\pgfsetlinewidth{0.000000pt}%
\definecolor{currentstroke}{rgb}{0.000000,0.000000,0.000000}%
\pgfsetstrokecolor{currentstroke}%
\pgfsetdash{}{0pt}%
\pgfpathmoveto{\pgfqpoint{3.928136in}{1.699738in}}%
\pgfpathlineto{\pgfqpoint{3.941530in}{1.694472in}}%
\pgfpathlineto{\pgfqpoint{3.954929in}{1.689232in}}%
\pgfpathlineto{\pgfqpoint{3.968333in}{1.684016in}}%
\pgfpathlineto{\pgfqpoint{3.981743in}{1.678825in}}%
\pgfpathlineto{\pgfqpoint{3.973941in}{1.675759in}}%
\pgfpathlineto{\pgfqpoint{3.966130in}{1.672946in}}%
\pgfpathlineto{\pgfqpoint{3.958312in}{1.670393in}}%
\pgfpathlineto{\pgfqpoint{3.950485in}{1.668109in}}%
\pgfpathlineto{\pgfqpoint{3.937055in}{1.673587in}}%
\pgfpathlineto{\pgfqpoint{3.923630in}{1.679090in}}%
\pgfpathlineto{\pgfqpoint{3.910210in}{1.684618in}}%
\pgfpathlineto{\pgfqpoint{3.896796in}{1.690170in}}%
\pgfpathlineto{\pgfqpoint{3.904644in}{1.692162in}}%
\pgfpathlineto{\pgfqpoint{3.912483in}{1.694426in}}%
\pgfpathlineto{\pgfqpoint{3.920314in}{1.696954in}}%
\pgfpathlineto{\pgfqpoint{3.928136in}{1.699738in}}%
\pgfpathclose%
\pgfusepath{fill}%
\end{pgfscope}%
\begin{pgfscope}%
\pgfpathrectangle{\pgfqpoint{1.254980in}{0.150000in}}{\pgfqpoint{5.490039in}{5.490039in}}%
\pgfusepath{clip}%
\pgfsetbuttcap%
\pgfsetroundjoin%
\definecolor{currentfill}{rgb}{0.262138,0.242286,0.520837}%
\pgfsetfillcolor{currentfill}%
\pgfsetfillopacity{0.700000}%
\pgfsetlinewidth{0.000000pt}%
\definecolor{currentstroke}{rgb}{0.000000,0.000000,0.000000}%
\pgfsetstrokecolor{currentstroke}%
\pgfsetdash{}{0pt}%
\pgfpathmoveto{\pgfqpoint{3.106496in}{2.065010in}}%
\pgfpathlineto{\pgfqpoint{3.119771in}{2.057070in}}%
\pgfpathlineto{\pgfqpoint{3.133049in}{2.049162in}}%
\pgfpathlineto{\pgfqpoint{3.146331in}{2.041284in}}%
\pgfpathlineto{\pgfqpoint{3.159616in}{2.033438in}}%
\pgfpathlineto{\pgfqpoint{3.151257in}{2.039783in}}%
\pgfpathlineto{\pgfqpoint{3.142880in}{2.046562in}}%
\pgfpathlineto{\pgfqpoint{3.134483in}{2.053786in}}%
\pgfpathlineto{\pgfqpoint{3.126067in}{2.061463in}}%
\pgfpathlineto{\pgfqpoint{3.112745in}{2.069655in}}%
\pgfpathlineto{\pgfqpoint{3.099426in}{2.077879in}}%
\pgfpathlineto{\pgfqpoint{3.086111in}{2.086133in}}%
\pgfpathlineto{\pgfqpoint{3.072798in}{2.094420in}}%
\pgfpathlineto{\pgfqpoint{3.081253in}{2.086391in}}%
\pgfpathlineto{\pgfqpoint{3.089687in}{2.078819in}}%
\pgfpathlineto{\pgfqpoint{3.098101in}{2.071696in}}%
\pgfpathlineto{\pgfqpoint{3.106496in}{2.065010in}}%
\pgfpathclose%
\pgfusepath{fill}%
\end{pgfscope}%
\begin{pgfscope}%
\pgfpathrectangle{\pgfqpoint{1.254980in}{0.150000in}}{\pgfqpoint{5.490039in}{5.490039in}}%
\pgfusepath{clip}%
\pgfsetbuttcap%
\pgfsetroundjoin%
\definecolor{currentfill}{rgb}{0.223925,0.334994,0.548053}%
\pgfsetfillcolor{currentfill}%
\pgfsetfillopacity{0.700000}%
\pgfsetlinewidth{0.000000pt}%
\definecolor{currentstroke}{rgb}{0.000000,0.000000,0.000000}%
\pgfsetstrokecolor{currentstroke}%
\pgfsetdash{}{0pt}%
\pgfpathmoveto{\pgfqpoint{2.807164in}{2.267088in}}%
\pgfpathlineto{\pgfqpoint{2.820420in}{2.258126in}}%
\pgfpathlineto{\pgfqpoint{2.833679in}{2.249200in}}%
\pgfpathlineto{\pgfqpoint{2.846940in}{2.240309in}}%
\pgfpathlineto{\pgfqpoint{2.860204in}{2.231455in}}%
\pgfpathlineto{\pgfqpoint{2.851566in}{2.241380in}}%
\pgfpathlineto{\pgfqpoint{2.842903in}{2.251798in}}%
\pgfpathlineto{\pgfqpoint{2.834217in}{2.262720in}}%
\pgfpathlineto{\pgfqpoint{2.825504in}{2.274157in}}%
\pgfpathlineto{\pgfqpoint{2.812197in}{2.283376in}}%
\pgfpathlineto{\pgfqpoint{2.798893in}{2.292631in}}%
\pgfpathlineto{\pgfqpoint{2.785590in}{2.301921in}}%
\pgfpathlineto{\pgfqpoint{2.772290in}{2.311249in}}%
\pgfpathlineto{\pgfqpoint{2.781047in}{2.299441in}}%
\pgfpathlineto{\pgfqpoint{2.789778in}{2.288152in}}%
\pgfpathlineto{\pgfqpoint{2.798484in}{2.277371in}}%
\pgfpathlineto{\pgfqpoint{2.807164in}{2.267088in}}%
\pgfpathclose%
\pgfusepath{fill}%
\end{pgfscope}%
\begin{pgfscope}%
\pgfpathrectangle{\pgfqpoint{1.254980in}{0.150000in}}{\pgfqpoint{5.490039in}{5.490039in}}%
\pgfusepath{clip}%
\pgfsetbuttcap%
\pgfsetroundjoin%
\definecolor{currentfill}{rgb}{0.282656,0.100196,0.422160}%
\pgfsetfillcolor{currentfill}%
\pgfsetfillopacity{0.700000}%
\pgfsetlinewidth{0.000000pt}%
\definecolor{currentstroke}{rgb}{0.000000,0.000000,0.000000}%
\pgfsetstrokecolor{currentstroke}%
\pgfsetdash{}{0pt}%
\pgfpathmoveto{\pgfqpoint{5.097328in}{1.797612in}}%
\pgfpathlineto{\pgfqpoint{5.111040in}{1.796038in}}%
\pgfpathlineto{\pgfqpoint{5.124761in}{1.794487in}}%
\pgfpathlineto{\pgfqpoint{5.138490in}{1.792960in}}%
\pgfpathlineto{\pgfqpoint{5.152227in}{1.791457in}}%
\pgfpathlineto{\pgfqpoint{5.144813in}{1.780808in}}%
\pgfpathlineto{\pgfqpoint{5.137395in}{1.770140in}}%
\pgfpathlineto{\pgfqpoint{5.129971in}{1.759459in}}%
\pgfpathlineto{\pgfqpoint{5.122543in}{1.748767in}}%
\pgfpathlineto{\pgfqpoint{5.108799in}{1.750431in}}%
\pgfpathlineto{\pgfqpoint{5.095064in}{1.752117in}}%
\pgfpathlineto{\pgfqpoint{5.081337in}{1.753828in}}%
\pgfpathlineto{\pgfqpoint{5.067617in}{1.755561in}}%
\pgfpathlineto{\pgfqpoint{5.075052in}{1.766089in}}%
\pgfpathlineto{\pgfqpoint{5.082482in}{1.776609in}}%
\pgfpathlineto{\pgfqpoint{5.089908in}{1.787118in}}%
\pgfpathlineto{\pgfqpoint{5.097328in}{1.797612in}}%
\pgfpathclose%
\pgfusepath{fill}%
\end{pgfscope}%
\begin{pgfscope}%
\pgfpathrectangle{\pgfqpoint{1.254980in}{0.150000in}}{\pgfqpoint{5.490039in}{5.490039in}}%
\pgfusepath{clip}%
\pgfsetbuttcap%
\pgfsetroundjoin%
\definecolor{currentfill}{rgb}{0.168126,0.459988,0.558082}%
\pgfsetfillcolor{currentfill}%
\pgfsetfillopacity{0.700000}%
\pgfsetlinewidth{0.000000pt}%
\definecolor{currentstroke}{rgb}{0.000000,0.000000,0.000000}%
\pgfsetstrokecolor{currentstroke}%
\pgfsetdash{}{0pt}%
\pgfpathmoveto{\pgfqpoint{2.400651in}{2.588437in}}%
\pgfpathlineto{\pgfqpoint{2.413904in}{2.577959in}}%
\pgfpathlineto{\pgfqpoint{2.427158in}{2.567527in}}%
\pgfpathlineto{\pgfqpoint{2.440414in}{2.557141in}}%
\pgfpathlineto{\pgfqpoint{2.453670in}{2.546801in}}%
\pgfpathlineto{\pgfqpoint{2.444596in}{2.561404in}}%
\pgfpathlineto{\pgfqpoint{2.435491in}{2.576572in}}%
\pgfpathlineto{\pgfqpoint{2.426352in}{2.592318in}}%
\pgfpathlineto{\pgfqpoint{2.417180in}{2.608651in}}%
\pgfpathlineto{\pgfqpoint{2.403872in}{2.619379in}}%
\pgfpathlineto{\pgfqpoint{2.390564in}{2.630153in}}%
\pgfpathlineto{\pgfqpoint{2.377257in}{2.640972in}}%
\pgfpathlineto{\pgfqpoint{2.363951in}{2.651839in}}%
\pgfpathlineto{\pgfqpoint{2.373177in}{2.635111in}}%
\pgfpathlineto{\pgfqpoint{2.382369in}{2.618976in}}%
\pgfpathlineto{\pgfqpoint{2.391526in}{2.603421in}}%
\pgfpathlineto{\pgfqpoint{2.400651in}{2.588437in}}%
\pgfpathclose%
\pgfusepath{fill}%
\end{pgfscope}%
\begin{pgfscope}%
\pgfpathrectangle{\pgfqpoint{1.254980in}{0.150000in}}{\pgfqpoint{5.490039in}{5.490039in}}%
\pgfusepath{clip}%
\pgfsetbuttcap%
\pgfsetroundjoin%
\definecolor{currentfill}{rgb}{0.269944,0.014625,0.341379}%
\pgfsetfillcolor{currentfill}%
\pgfsetfillopacity{0.700000}%
\pgfsetlinewidth{0.000000pt}%
\definecolor{currentstroke}{rgb}{0.000000,0.000000,0.000000}%
\pgfsetstrokecolor{currentstroke}%
\pgfsetdash{}{0pt}%
\pgfpathmoveto{\pgfqpoint{4.481708in}{1.649636in}}%
\pgfpathlineto{\pgfqpoint{4.495234in}{1.646164in}}%
\pgfpathlineto{\pgfqpoint{4.508767in}{1.642715in}}%
\pgfpathlineto{\pgfqpoint{4.522307in}{1.639290in}}%
\pgfpathlineto{\pgfqpoint{4.535853in}{1.635888in}}%
\pgfpathlineto{\pgfqpoint{4.528266in}{1.627864in}}%
\pgfpathlineto{\pgfqpoint{4.520674in}{1.619962in}}%
\pgfpathlineto{\pgfqpoint{4.513077in}{1.612186in}}%
\pgfpathlineto{\pgfqpoint{4.505476in}{1.604544in}}%
\pgfpathlineto{\pgfqpoint{4.491918in}{1.608182in}}%
\pgfpathlineto{\pgfqpoint{4.478367in}{1.611843in}}%
\pgfpathlineto{\pgfqpoint{4.464822in}{1.615527in}}%
\pgfpathlineto{\pgfqpoint{4.451284in}{1.619235in}}%
\pgfpathlineto{\pgfqpoint{4.458897in}{1.626637in}}%
\pgfpathlineto{\pgfqpoint{4.466505in}{1.634175in}}%
\pgfpathlineto{\pgfqpoint{4.474109in}{1.641843in}}%
\pgfpathlineto{\pgfqpoint{4.481708in}{1.649636in}}%
\pgfpathclose%
\pgfusepath{fill}%
\end{pgfscope}%
\begin{pgfscope}%
\pgfpathrectangle{\pgfqpoint{1.254980in}{0.150000in}}{\pgfqpoint{5.490039in}{5.490039in}}%
\pgfusepath{clip}%
\pgfsetbuttcap%
\pgfsetroundjoin%
\definecolor{currentfill}{rgb}{0.273809,0.031497,0.358853}%
\pgfsetfillcolor{currentfill}%
\pgfsetfillopacity{0.700000}%
\pgfsetlinewidth{0.000000pt}%
\definecolor{currentstroke}{rgb}{0.000000,0.000000,0.000000}%
\pgfsetstrokecolor{currentstroke}%
\pgfsetdash{}{0pt}%
\pgfpathmoveto{\pgfqpoint{4.704848in}{1.680872in}}%
\pgfpathlineto{\pgfqpoint{4.718438in}{1.678115in}}%
\pgfpathlineto{\pgfqpoint{4.732035in}{1.675382in}}%
\pgfpathlineto{\pgfqpoint{4.745639in}{1.672672in}}%
\pgfpathlineto{\pgfqpoint{4.759250in}{1.669985in}}%
\pgfpathlineto{\pgfqpoint{4.751727in}{1.660574in}}%
\pgfpathlineto{\pgfqpoint{4.744200in}{1.651231in}}%
\pgfpathlineto{\pgfqpoint{4.736668in}{1.641961in}}%
\pgfpathlineto{\pgfqpoint{4.729133in}{1.632770in}}%
\pgfpathlineto{\pgfqpoint{4.715512in}{1.635667in}}%
\pgfpathlineto{\pgfqpoint{4.701899in}{1.638588in}}%
\pgfpathlineto{\pgfqpoint{4.688293in}{1.641532in}}%
\pgfpathlineto{\pgfqpoint{4.674694in}{1.644499in}}%
\pgfpathlineto{\pgfqpoint{4.682239in}{1.653475in}}%
\pgfpathlineto{\pgfqpoint{4.689780in}{1.662533in}}%
\pgfpathlineto{\pgfqpoint{4.697316in}{1.671667in}}%
\pgfpathlineto{\pgfqpoint{4.704848in}{1.680872in}}%
\pgfpathclose%
\pgfusepath{fill}%
\end{pgfscope}%
\begin{pgfscope}%
\pgfpathrectangle{\pgfqpoint{1.254980in}{0.150000in}}{\pgfqpoint{5.490039in}{5.490039in}}%
\pgfusepath{clip}%
\pgfsetbuttcap%
\pgfsetroundjoin%
\definecolor{currentfill}{rgb}{0.281446,0.084320,0.407414}%
\pgfsetfillcolor{currentfill}%
\pgfsetfillopacity{0.700000}%
\pgfsetlinewidth{0.000000pt}%
\definecolor{currentstroke}{rgb}{0.000000,0.000000,0.000000}%
\pgfsetstrokecolor{currentstroke}%
\pgfsetdash{}{0pt}%
\pgfpathmoveto{\pgfqpoint{5.012820in}{1.762730in}}%
\pgfpathlineto{\pgfqpoint{5.026508in}{1.760903in}}%
\pgfpathlineto{\pgfqpoint{5.040203in}{1.759099in}}%
\pgfpathlineto{\pgfqpoint{5.053906in}{1.757318in}}%
\pgfpathlineto{\pgfqpoint{5.067617in}{1.755561in}}%
\pgfpathlineto{\pgfqpoint{5.060178in}{1.745031in}}%
\pgfpathlineto{\pgfqpoint{5.052733in}{1.734501in}}%
\pgfpathlineto{\pgfqpoint{5.045284in}{1.723975in}}%
\pgfpathlineto{\pgfqpoint{5.037831in}{1.713459in}}%
\pgfpathlineto{\pgfqpoint{5.024113in}{1.715388in}}%
\pgfpathlineto{\pgfqpoint{5.010403in}{1.717341in}}%
\pgfpathlineto{\pgfqpoint{4.996701in}{1.719318in}}%
\pgfpathlineto{\pgfqpoint{4.983006in}{1.721317in}}%
\pgfpathlineto{\pgfqpoint{4.990467in}{1.731657in}}%
\pgfpathlineto{\pgfqpoint{4.997923in}{1.742008in}}%
\pgfpathlineto{\pgfqpoint{5.005374in}{1.752367in}}%
\pgfpathlineto{\pgfqpoint{5.012820in}{1.762730in}}%
\pgfpathclose%
\pgfusepath{fill}%
\end{pgfscope}%
\begin{pgfscope}%
\pgfpathrectangle{\pgfqpoint{1.254980in}{0.150000in}}{\pgfqpoint{5.490039in}{5.490039in}}%
\pgfusepath{clip}%
\pgfsetbuttcap%
\pgfsetroundjoin%
\definecolor{currentfill}{rgb}{0.283091,0.110553,0.431554}%
\pgfsetfillcolor{currentfill}%
\pgfsetfillopacity{0.700000}%
\pgfsetlinewidth{0.000000pt}%
\definecolor{currentstroke}{rgb}{0.000000,0.000000,0.000000}%
\pgfsetstrokecolor{currentstroke}%
\pgfsetdash{}{0pt}%
\pgfpathmoveto{\pgfqpoint{3.597615in}{1.807511in}}%
\pgfpathlineto{\pgfqpoint{3.610954in}{1.801141in}}%
\pgfpathlineto{\pgfqpoint{3.624299in}{1.794797in}}%
\pgfpathlineto{\pgfqpoint{3.637648in}{1.788480in}}%
\pgfpathlineto{\pgfqpoint{3.651002in}{1.782190in}}%
\pgfpathlineto{\pgfqpoint{3.643009in}{1.782911in}}%
\pgfpathlineto{\pgfqpoint{3.635004in}{1.783966in}}%
\pgfpathlineto{\pgfqpoint{3.626988in}{1.785363in}}%
\pgfpathlineto{\pgfqpoint{3.618959in}{1.787111in}}%
\pgfpathlineto{\pgfqpoint{3.605578in}{1.793716in}}%
\pgfpathlineto{\pgfqpoint{3.592202in}{1.800348in}}%
\pgfpathlineto{\pgfqpoint{3.578830in}{1.807006in}}%
\pgfpathlineto{\pgfqpoint{3.565463in}{1.813691in}}%
\pgfpathlineto{\pgfqpoint{3.573520in}{1.811623in}}%
\pgfpathlineto{\pgfqpoint{3.581564in}{1.809910in}}%
\pgfpathlineto{\pgfqpoint{3.589595in}{1.808542in}}%
\pgfpathlineto{\pgfqpoint{3.597615in}{1.807511in}}%
\pgfpathclose%
\pgfusepath{fill}%
\end{pgfscope}%
\begin{pgfscope}%
\pgfpathrectangle{\pgfqpoint{1.254980in}{0.150000in}}{\pgfqpoint{5.490039in}{5.490039in}}%
\pgfusepath{clip}%
\pgfsetbuttcap%
\pgfsetroundjoin%
\definecolor{currentfill}{rgb}{0.280267,0.073417,0.397163}%
\pgfsetfillcolor{currentfill}%
\pgfsetfillopacity{0.700000}%
\pgfsetlinewidth{0.000000pt}%
\definecolor{currentstroke}{rgb}{0.000000,0.000000,0.000000}%
\pgfsetstrokecolor{currentstroke}%
\pgfsetdash{}{0pt}%
\pgfpathmoveto{\pgfqpoint{3.789668in}{1.735501in}}%
\pgfpathlineto{\pgfqpoint{3.803041in}{1.729745in}}%
\pgfpathlineto{\pgfqpoint{3.816419in}{1.724016in}}%
\pgfpathlineto{\pgfqpoint{3.829802in}{1.718311in}}%
\pgfpathlineto{\pgfqpoint{3.843191in}{1.712633in}}%
\pgfpathlineto{\pgfqpoint{3.835312in}{1.711215in}}%
\pgfpathlineto{\pgfqpoint{3.827424in}{1.710089in}}%
\pgfpathlineto{\pgfqpoint{3.819526in}{1.709261in}}%
\pgfpathlineto{\pgfqpoint{3.811618in}{1.708740in}}%
\pgfpathlineto{\pgfqpoint{3.798207in}{1.714719in}}%
\pgfpathlineto{\pgfqpoint{3.784800in}{1.720724in}}%
\pgfpathlineto{\pgfqpoint{3.771399in}{1.726754in}}%
\pgfpathlineto{\pgfqpoint{3.758002in}{1.732810in}}%
\pgfpathlineto{\pgfqpoint{3.765934in}{1.733026in}}%
\pgfpathlineto{\pgfqpoint{3.773855in}{1.733551in}}%
\pgfpathlineto{\pgfqpoint{3.781766in}{1.734379in}}%
\pgfpathlineto{\pgfqpoint{3.789668in}{1.735501in}}%
\pgfpathclose%
\pgfusepath{fill}%
\end{pgfscope}%
\begin{pgfscope}%
\pgfpathrectangle{\pgfqpoint{1.254980in}{0.150000in}}{\pgfqpoint{5.490039in}{5.490039in}}%
\pgfusepath{clip}%
\pgfsetbuttcap%
\pgfsetroundjoin%
\definecolor{currentfill}{rgb}{0.172719,0.448791,0.557885}%
\pgfsetfillcolor{currentfill}%
\pgfsetfillopacity{0.700000}%
\pgfsetlinewidth{0.000000pt}%
\definecolor{currentstroke}{rgb}{0.000000,0.000000,0.000000}%
\pgfsetstrokecolor{currentstroke}%
\pgfsetdash{}{0pt}%
\pgfpathmoveto{\pgfqpoint{2.453670in}{2.546801in}}%
\pgfpathlineto{\pgfqpoint{2.466928in}{2.536505in}}%
\pgfpathlineto{\pgfqpoint{2.480186in}{2.526254in}}%
\pgfpathlineto{\pgfqpoint{2.493447in}{2.516048in}}%
\pgfpathlineto{\pgfqpoint{2.506708in}{2.505885in}}%
\pgfpathlineto{\pgfqpoint{2.497685in}{2.520108in}}%
\pgfpathlineto{\pgfqpoint{2.488630in}{2.534892in}}%
\pgfpathlineto{\pgfqpoint{2.479544in}{2.550249in}}%
\pgfpathlineto{\pgfqpoint{2.470425in}{2.566189in}}%
\pgfpathlineto{\pgfqpoint{2.457112in}{2.576738in}}%
\pgfpathlineto{\pgfqpoint{2.443800in}{2.587331in}}%
\pgfpathlineto{\pgfqpoint{2.430490in}{2.597969in}}%
\pgfpathlineto{\pgfqpoint{2.417180in}{2.608651in}}%
\pgfpathlineto{\pgfqpoint{2.426352in}{2.592318in}}%
\pgfpathlineto{\pgfqpoint{2.435491in}{2.576572in}}%
\pgfpathlineto{\pgfqpoint{2.444596in}{2.561404in}}%
\pgfpathlineto{\pgfqpoint{2.453670in}{2.546801in}}%
\pgfpathclose%
\pgfusepath{fill}%
\end{pgfscope}%
\begin{pgfscope}%
\pgfpathrectangle{\pgfqpoint{1.254980in}{0.150000in}}{\pgfqpoint{5.490039in}{5.490039in}}%
\pgfusepath{clip}%
\pgfsetbuttcap%
\pgfsetroundjoin%
\definecolor{currentfill}{rgb}{0.278826,0.175490,0.483397}%
\pgfsetfillcolor{currentfill}%
\pgfsetfillopacity{0.700000}%
\pgfsetlinewidth{0.000000pt}%
\definecolor{currentstroke}{rgb}{0.000000,0.000000,0.000000}%
\pgfsetstrokecolor{currentstroke}%
\pgfsetdash{}{0pt}%
\pgfpathmoveto{\pgfqpoint{5.490597in}{1.943981in}}%
\pgfpathlineto{\pgfqpoint{5.504452in}{1.943387in}}%
\pgfpathlineto{\pgfqpoint{5.518317in}{1.942816in}}%
\pgfpathlineto{\pgfqpoint{5.532190in}{1.942269in}}%
\pgfpathlineto{\pgfqpoint{5.524903in}{1.931854in}}%
\pgfpathlineto{\pgfqpoint{5.517608in}{1.921358in}}%
\pgfpathlineto{\pgfqpoint{5.510307in}{1.910783in}}%
\pgfpathlineto{\pgfqpoint{5.502999in}{1.900131in}}%
\pgfpathlineto{\pgfqpoint{5.489119in}{1.900786in}}%
\pgfpathlineto{\pgfqpoint{5.475248in}{1.901464in}}%
\pgfpathlineto{\pgfqpoint{5.461385in}{1.902167in}}%
\pgfpathlineto{\pgfqpoint{5.468698in}{1.912734in}}%
\pgfpathlineto{\pgfqpoint{5.476004in}{1.923227in}}%
\pgfpathlineto{\pgfqpoint{5.483304in}{1.933643in}}%
\pgfpathlineto{\pgfqpoint{5.490597in}{1.943981in}}%
\pgfpathclose%
\pgfusepath{fill}%
\end{pgfscope}%
\begin{pgfscope}%
\pgfpathrectangle{\pgfqpoint{1.254980in}{0.150000in}}{\pgfqpoint{5.490039in}{5.490039in}}%
\pgfusepath{clip}%
\pgfsetbuttcap%
\pgfsetroundjoin%
\definecolor{currentfill}{rgb}{0.279566,0.067836,0.391917}%
\pgfsetfillcolor{currentfill}%
\pgfsetfillopacity{0.700000}%
\pgfsetlinewidth{0.000000pt}%
\definecolor{currentstroke}{rgb}{0.000000,0.000000,0.000000}%
\pgfsetstrokecolor{currentstroke}%
\pgfsetdash{}{0pt}%
\pgfpathmoveto{\pgfqpoint{4.928307in}{1.729550in}}%
\pgfpathlineto{\pgfqpoint{4.941970in}{1.727457in}}%
\pgfpathlineto{\pgfqpoint{4.955641in}{1.725387in}}%
\pgfpathlineto{\pgfqpoint{4.969320in}{1.723341in}}%
\pgfpathlineto{\pgfqpoint{4.983006in}{1.721317in}}%
\pgfpathlineto{\pgfqpoint{4.975541in}{1.710995in}}%
\pgfpathlineto{\pgfqpoint{4.968072in}{1.700692in}}%
\pgfpathlineto{\pgfqpoint{4.960598in}{1.690415in}}%
\pgfpathlineto{\pgfqpoint{4.953120in}{1.680167in}}%
\pgfpathlineto{\pgfqpoint{4.939427in}{1.682376in}}%
\pgfpathlineto{\pgfqpoint{4.925741in}{1.684608in}}%
\pgfpathlineto{\pgfqpoint{4.912062in}{1.686863in}}%
\pgfpathlineto{\pgfqpoint{4.898392in}{1.689141in}}%
\pgfpathlineto{\pgfqpoint{4.905877in}{1.699199in}}%
\pgfpathlineto{\pgfqpoint{4.913358in}{1.709289in}}%
\pgfpathlineto{\pgfqpoint{4.920835in}{1.719408in}}%
\pgfpathlineto{\pgfqpoint{4.928307in}{1.729550in}}%
\pgfpathclose%
\pgfusepath{fill}%
\end{pgfscope}%
\begin{pgfscope}%
\pgfpathrectangle{\pgfqpoint{1.254980in}{0.150000in}}{\pgfqpoint{5.490039in}{5.490039in}}%
\pgfusepath{clip}%
\pgfsetbuttcap%
\pgfsetroundjoin%
\definecolor{currentfill}{rgb}{0.280868,0.160771,0.472899}%
\pgfsetfillcolor{currentfill}%
\pgfsetfillopacity{0.700000}%
\pgfsetlinewidth{0.000000pt}%
\definecolor{currentstroke}{rgb}{0.000000,0.000000,0.000000}%
\pgfsetstrokecolor{currentstroke}%
\pgfsetdash{}{0pt}%
\pgfpathmoveto{\pgfqpoint{3.405392in}{1.896036in}}%
\pgfpathlineto{\pgfqpoint{3.418708in}{1.889021in}}%
\pgfpathlineto{\pgfqpoint{3.432029in}{1.882035in}}%
\pgfpathlineto{\pgfqpoint{3.445353in}{1.875076in}}%
\pgfpathlineto{\pgfqpoint{3.458682in}{1.868146in}}%
\pgfpathlineto{\pgfqpoint{3.450553in}{1.871224in}}%
\pgfpathlineto{\pgfqpoint{3.442410in}{1.874681in}}%
\pgfpathlineto{\pgfqpoint{3.434253in}{1.878526in}}%
\pgfpathlineto{\pgfqpoint{3.426080in}{1.882768in}}%
\pgfpathlineto{\pgfqpoint{3.412720in}{1.890028in}}%
\pgfpathlineto{\pgfqpoint{3.399365in}{1.897316in}}%
\pgfpathlineto{\pgfqpoint{3.386013in}{1.904632in}}%
\pgfpathlineto{\pgfqpoint{3.372666in}{1.911976in}}%
\pgfpathlineto{\pgfqpoint{3.380870in}{1.907399in}}%
\pgfpathlineto{\pgfqpoint{3.389060in}{1.903223in}}%
\pgfpathlineto{\pgfqpoint{3.397233in}{1.899438in}}%
\pgfpathlineto{\pgfqpoint{3.405392in}{1.896036in}}%
\pgfpathclose%
\pgfusepath{fill}%
\end{pgfscope}%
\begin{pgfscope}%
\pgfpathrectangle{\pgfqpoint{1.254980in}{0.150000in}}{\pgfqpoint{5.490039in}{5.490039in}}%
\pgfusepath{clip}%
\pgfsetbuttcap%
\pgfsetroundjoin%
\definecolor{currentfill}{rgb}{0.272594,0.025563,0.353093}%
\pgfsetfillcolor{currentfill}%
\pgfsetfillopacity{0.700000}%
\pgfsetlinewidth{0.000000pt}%
\definecolor{currentstroke}{rgb}{0.000000,0.000000,0.000000}%
\pgfsetstrokecolor{currentstroke}%
\pgfsetdash{}{0pt}%
\pgfpathmoveto{\pgfqpoint{4.120210in}{1.655028in}}%
\pgfpathlineto{\pgfqpoint{4.133653in}{1.650332in}}%
\pgfpathlineto{\pgfqpoint{4.147101in}{1.645661in}}%
\pgfpathlineto{\pgfqpoint{4.160555in}{1.641013in}}%
\pgfpathlineto{\pgfqpoint{4.174016in}{1.636390in}}%
\pgfpathlineto{\pgfqpoint{4.166296in}{1.631577in}}%
\pgfpathlineto{\pgfqpoint{4.158570in}{1.626979in}}%
\pgfpathlineto{\pgfqpoint{4.150837in}{1.622602in}}%
\pgfpathlineto{\pgfqpoint{4.143097in}{1.618453in}}%
\pgfpathlineto{\pgfqpoint{4.129620in}{1.623350in}}%
\pgfpathlineto{\pgfqpoint{4.116148in}{1.628271in}}%
\pgfpathlineto{\pgfqpoint{4.102682in}{1.633217in}}%
\pgfpathlineto{\pgfqpoint{4.089222in}{1.638186in}}%
\pgfpathlineto{\pgfqpoint{4.096980in}{1.642056in}}%
\pgfpathlineto{\pgfqpoint{4.104730in}{1.646158in}}%
\pgfpathlineto{\pgfqpoint{4.112473in}{1.650484in}}%
\pgfpathlineto{\pgfqpoint{4.120210in}{1.655028in}}%
\pgfpathclose%
\pgfusepath{fill}%
\end{pgfscope}%
\begin{pgfscope}%
\pgfpathrectangle{\pgfqpoint{1.254980in}{0.150000in}}{\pgfqpoint{5.490039in}{5.490039in}}%
\pgfusepath{clip}%
\pgfsetbuttcap%
\pgfsetroundjoin%
\definecolor{currentfill}{rgb}{0.269944,0.014625,0.341379}%
\pgfsetfillcolor{currentfill}%
\pgfsetfillopacity{0.700000}%
\pgfsetlinewidth{0.000000pt}%
\definecolor{currentstroke}{rgb}{0.000000,0.000000,0.000000}%
\pgfsetstrokecolor{currentstroke}%
\pgfsetdash{}{0pt}%
\pgfpathmoveto{\pgfqpoint{4.258671in}{1.640441in}}%
\pgfpathlineto{\pgfqpoint{4.272146in}{1.636200in}}%
\pgfpathlineto{\pgfqpoint{4.285627in}{1.631982in}}%
\pgfpathlineto{\pgfqpoint{4.299114in}{1.627788in}}%
\pgfpathlineto{\pgfqpoint{4.312608in}{1.623618in}}%
\pgfpathlineto{\pgfqpoint{4.304942in}{1.617504in}}%
\pgfpathlineto{\pgfqpoint{4.297271in}{1.611571in}}%
\pgfpathlineto{\pgfqpoint{4.289595in}{1.605825in}}%
\pgfpathlineto{\pgfqpoint{4.281913in}{1.600272in}}%
\pgfpathlineto{\pgfqpoint{4.268405in}{1.604703in}}%
\pgfpathlineto{\pgfqpoint{4.254902in}{1.609158in}}%
\pgfpathlineto{\pgfqpoint{4.241406in}{1.613637in}}%
\pgfpathlineto{\pgfqpoint{4.227916in}{1.618140in}}%
\pgfpathlineto{\pgfqpoint{4.235614in}{1.623426in}}%
\pgfpathlineto{\pgfqpoint{4.243305in}{1.628909in}}%
\pgfpathlineto{\pgfqpoint{4.250991in}{1.634583in}}%
\pgfpathlineto{\pgfqpoint{4.258671in}{1.640441in}}%
\pgfpathclose%
\pgfusepath{fill}%
\end{pgfscope}%
\begin{pgfscope}%
\pgfpathrectangle{\pgfqpoint{1.254980in}{0.150000in}}{\pgfqpoint{5.490039in}{5.490039in}}%
\pgfusepath{clip}%
\pgfsetbuttcap%
\pgfsetroundjoin%
\definecolor{currentfill}{rgb}{0.266580,0.228262,0.514349}%
\pgfsetfillcolor{currentfill}%
\pgfsetfillopacity{0.700000}%
\pgfsetlinewidth{0.000000pt}%
\definecolor{currentstroke}{rgb}{0.000000,0.000000,0.000000}%
\pgfsetstrokecolor{currentstroke}%
\pgfsetdash{}{0pt}%
\pgfpathmoveto{\pgfqpoint{3.159616in}{2.033438in}}%
\pgfpathlineto{\pgfqpoint{3.172904in}{2.025622in}}%
\pgfpathlineto{\pgfqpoint{3.186196in}{2.017838in}}%
\pgfpathlineto{\pgfqpoint{3.199492in}{2.010083in}}%
\pgfpathlineto{\pgfqpoint{3.212791in}{2.002359in}}%
\pgfpathlineto{\pgfqpoint{3.204468in}{2.008364in}}%
\pgfpathlineto{\pgfqpoint{3.196127in}{2.014800in}}%
\pgfpathlineto{\pgfqpoint{3.187767in}{2.021675in}}%
\pgfpathlineto{\pgfqpoint{3.179388in}{2.029001in}}%
\pgfpathlineto{\pgfqpoint{3.166053in}{2.037071in}}%
\pgfpathlineto{\pgfqpoint{3.152721in}{2.045171in}}%
\pgfpathlineto{\pgfqpoint{3.139392in}{2.053301in}}%
\pgfpathlineto{\pgfqpoint{3.126067in}{2.061463in}}%
\pgfpathlineto{\pgfqpoint{3.134483in}{2.053786in}}%
\pgfpathlineto{\pgfqpoint{3.142880in}{2.046562in}}%
\pgfpathlineto{\pgfqpoint{3.151257in}{2.039783in}}%
\pgfpathlineto{\pgfqpoint{3.159616in}{2.033438in}}%
\pgfpathclose%
\pgfusepath{fill}%
\end{pgfscope}%
\begin{pgfscope}%
\pgfpathrectangle{\pgfqpoint{1.254980in}{0.150000in}}{\pgfqpoint{5.490039in}{5.490039in}}%
\pgfusepath{clip}%
\pgfsetbuttcap%
\pgfsetroundjoin%
\definecolor{currentfill}{rgb}{0.272594,0.025563,0.353093}%
\pgfsetfillcolor{currentfill}%
\pgfsetfillopacity{0.700000}%
\pgfsetlinewidth{0.000000pt}%
\definecolor{currentstroke}{rgb}{0.000000,0.000000,0.000000}%
\pgfsetstrokecolor{currentstroke}%
\pgfsetdash{}{0pt}%
\pgfpathmoveto{\pgfqpoint{4.620370in}{1.656603in}}%
\pgfpathlineto{\pgfqpoint{4.633940in}{1.653542in}}%
\pgfpathlineto{\pgfqpoint{4.647518in}{1.650504in}}%
\pgfpathlineto{\pgfqpoint{4.661103in}{1.647490in}}%
\pgfpathlineto{\pgfqpoint{4.674694in}{1.644499in}}%
\pgfpathlineto{\pgfqpoint{4.667145in}{1.635610in}}%
\pgfpathlineto{\pgfqpoint{4.659592in}{1.626814in}}%
\pgfpathlineto{\pgfqpoint{4.652034in}{1.618115in}}%
\pgfpathlineto{\pgfqpoint{4.644472in}{1.609520in}}%
\pgfpathlineto{\pgfqpoint{4.630870in}{1.612734in}}%
\pgfpathlineto{\pgfqpoint{4.617276in}{1.615972in}}%
\pgfpathlineto{\pgfqpoint{4.603688in}{1.619232in}}%
\pgfpathlineto{\pgfqpoint{4.590107in}{1.622517in}}%
\pgfpathlineto{\pgfqpoint{4.597680in}{1.630884in}}%
\pgfpathlineto{\pgfqpoint{4.605247in}{1.639358in}}%
\pgfpathlineto{\pgfqpoint{4.612811in}{1.647933in}}%
\pgfpathlineto{\pgfqpoint{4.620370in}{1.656603in}}%
\pgfpathclose%
\pgfusepath{fill}%
\end{pgfscope}%
\begin{pgfscope}%
\pgfpathrectangle{\pgfqpoint{1.254980in}{0.150000in}}{\pgfqpoint{5.490039in}{5.490039in}}%
\pgfusepath{clip}%
\pgfsetbuttcap%
\pgfsetroundjoin%
\definecolor{currentfill}{rgb}{0.229739,0.322361,0.545706}%
\pgfsetfillcolor{currentfill}%
\pgfsetfillopacity{0.700000}%
\pgfsetlinewidth{0.000000pt}%
\definecolor{currentstroke}{rgb}{0.000000,0.000000,0.000000}%
\pgfsetstrokecolor{currentstroke}%
\pgfsetdash{}{0pt}%
\pgfpathmoveto{\pgfqpoint{2.860204in}{2.231455in}}%
\pgfpathlineto{\pgfqpoint{2.873470in}{2.222635in}}%
\pgfpathlineto{\pgfqpoint{2.886739in}{2.213851in}}%
\pgfpathlineto{\pgfqpoint{2.900010in}{2.205102in}}%
\pgfpathlineto{\pgfqpoint{2.913285in}{2.196387in}}%
\pgfpathlineto{\pgfqpoint{2.904688in}{2.205954in}}%
\pgfpathlineto{\pgfqpoint{2.896069in}{2.216010in}}%
\pgfpathlineto{\pgfqpoint{2.887425in}{2.226567in}}%
\pgfpathlineto{\pgfqpoint{2.878757in}{2.237633in}}%
\pgfpathlineto{\pgfqpoint{2.865440in}{2.246712in}}%
\pgfpathlineto{\pgfqpoint{2.852126in}{2.255825in}}%
\pgfpathlineto{\pgfqpoint{2.838814in}{2.264973in}}%
\pgfpathlineto{\pgfqpoint{2.825504in}{2.274157in}}%
\pgfpathlineto{\pgfqpoint{2.834217in}{2.262720in}}%
\pgfpathlineto{\pgfqpoint{2.842903in}{2.251798in}}%
\pgfpathlineto{\pgfqpoint{2.851566in}{2.241380in}}%
\pgfpathlineto{\pgfqpoint{2.860204in}{2.231455in}}%
\pgfpathclose%
\pgfusepath{fill}%
\end{pgfscope}%
\begin{pgfscope}%
\pgfpathrectangle{\pgfqpoint{1.254980in}{0.150000in}}{\pgfqpoint{5.490039in}{5.490039in}}%
\pgfusepath{clip}%
\pgfsetbuttcap%
\pgfsetroundjoin%
\definecolor{currentfill}{rgb}{0.276022,0.044167,0.370164}%
\pgfsetfillcolor{currentfill}%
\pgfsetfillopacity{0.700000}%
\pgfsetlinewidth{0.000000pt}%
\definecolor{currentstroke}{rgb}{0.000000,0.000000,0.000000}%
\pgfsetstrokecolor{currentstroke}%
\pgfsetdash{}{0pt}%
\pgfpathmoveto{\pgfqpoint{3.981743in}{1.678825in}}%
\pgfpathlineto{\pgfqpoint{3.995159in}{1.673659in}}%
\pgfpathlineto{\pgfqpoint{4.008580in}{1.668518in}}%
\pgfpathlineto{\pgfqpoint{4.022006in}{1.663401in}}%
\pgfpathlineto{\pgfqpoint{4.035438in}{1.658309in}}%
\pgfpathlineto{\pgfqpoint{4.027655in}{1.654961in}}%
\pgfpathlineto{\pgfqpoint{4.019864in}{1.651862in}}%
\pgfpathlineto{\pgfqpoint{4.012066in}{1.649021in}}%
\pgfpathlineto{\pgfqpoint{4.004259in}{1.646445in}}%
\pgfpathlineto{\pgfqpoint{3.990808in}{1.651824in}}%
\pgfpathlineto{\pgfqpoint{3.977362in}{1.657228in}}%
\pgfpathlineto{\pgfqpoint{3.963921in}{1.662656in}}%
\pgfpathlineto{\pgfqpoint{3.950485in}{1.668109in}}%
\pgfpathlineto{\pgfqpoint{3.958312in}{1.670393in}}%
\pgfpathlineto{\pgfqpoint{3.966130in}{1.672946in}}%
\pgfpathlineto{\pgfqpoint{3.973941in}{1.675759in}}%
\pgfpathlineto{\pgfqpoint{3.981743in}{1.678825in}}%
\pgfpathclose%
\pgfusepath{fill}%
\end{pgfscope}%
\begin{pgfscope}%
\pgfpathrectangle{\pgfqpoint{1.254980in}{0.150000in}}{\pgfqpoint{5.490039in}{5.490039in}}%
\pgfusepath{clip}%
\pgfsetbuttcap%
\pgfsetroundjoin%
\definecolor{currentfill}{rgb}{0.269944,0.014625,0.341379}%
\pgfsetfillcolor{currentfill}%
\pgfsetfillopacity{0.700000}%
\pgfsetlinewidth{0.000000pt}%
\definecolor{currentstroke}{rgb}{0.000000,0.000000,0.000000}%
\pgfsetstrokecolor{currentstroke}%
\pgfsetdash{}{0pt}%
\pgfpathmoveto{\pgfqpoint{4.397197in}{1.634304in}}%
\pgfpathlineto{\pgfqpoint{4.410709in}{1.630501in}}%
\pgfpathlineto{\pgfqpoint{4.424228in}{1.626722in}}%
\pgfpathlineto{\pgfqpoint{4.437753in}{1.622967in}}%
\pgfpathlineto{\pgfqpoint{4.451284in}{1.619235in}}%
\pgfpathlineto{\pgfqpoint{4.443666in}{1.611976in}}%
\pgfpathlineto{\pgfqpoint{4.436043in}{1.604865in}}%
\pgfpathlineto{\pgfqpoint{4.428415in}{1.597910in}}%
\pgfpathlineto{\pgfqpoint{4.420783in}{1.591115in}}%
\pgfpathlineto{\pgfqpoint{4.407239in}{1.595095in}}%
\pgfpathlineto{\pgfqpoint{4.393701in}{1.599099in}}%
\pgfpathlineto{\pgfqpoint{4.380169in}{1.603126in}}%
\pgfpathlineto{\pgfqpoint{4.366644in}{1.607177in}}%
\pgfpathlineto{\pgfqpoint{4.374290in}{1.613718in}}%
\pgfpathlineto{\pgfqpoint{4.381931in}{1.620424in}}%
\pgfpathlineto{\pgfqpoint{4.389567in}{1.627288in}}%
\pgfpathlineto{\pgfqpoint{4.397197in}{1.634304in}}%
\pgfpathclose%
\pgfusepath{fill}%
\end{pgfscope}%
\begin{pgfscope}%
\pgfpathrectangle{\pgfqpoint{1.254980in}{0.150000in}}{\pgfqpoint{5.490039in}{5.490039in}}%
\pgfusepath{clip}%
\pgfsetbuttcap%
\pgfsetroundjoin%
\definecolor{currentfill}{rgb}{0.280868,0.160771,0.472899}%
\pgfsetfillcolor{currentfill}%
\pgfsetfillopacity{0.700000}%
\pgfsetlinewidth{0.000000pt}%
\definecolor{currentstroke}{rgb}{0.000000,0.000000,0.000000}%
\pgfsetstrokecolor{currentstroke}%
\pgfsetdash{}{0pt}%
\pgfpathmoveto{\pgfqpoint{5.406024in}{1.905211in}}%
\pgfpathlineto{\pgfqpoint{5.419851in}{1.904415in}}%
\pgfpathlineto{\pgfqpoint{5.433687in}{1.903642in}}%
\pgfpathlineto{\pgfqpoint{5.447532in}{1.902892in}}%
\pgfpathlineto{\pgfqpoint{5.461385in}{1.902167in}}%
\pgfpathlineto{\pgfqpoint{5.454066in}{1.891528in}}%
\pgfpathlineto{\pgfqpoint{5.446740in}{1.880821in}}%
\pgfpathlineto{\pgfqpoint{5.439408in}{1.870047in}}%
\pgfpathlineto{\pgfqpoint{5.432070in}{1.859210in}}%
\pgfpathlineto{\pgfqpoint{5.418210in}{1.860057in}}%
\pgfpathlineto{\pgfqpoint{5.404359in}{1.860927in}}%
\pgfpathlineto{\pgfqpoint{5.390516in}{1.861821in}}%
\pgfpathlineto{\pgfqpoint{5.376683in}{1.862739in}}%
\pgfpathlineto{\pgfqpoint{5.384027in}{1.873450in}}%
\pgfpathlineto{\pgfqpoint{5.391365in}{1.884101in}}%
\pgfpathlineto{\pgfqpoint{5.398698in}{1.894689in}}%
\pgfpathlineto{\pgfqpoint{5.406024in}{1.905211in}}%
\pgfpathclose%
\pgfusepath{fill}%
\end{pgfscope}%
\begin{pgfscope}%
\pgfpathrectangle{\pgfqpoint{1.254980in}{0.150000in}}{\pgfqpoint{5.490039in}{5.490039in}}%
\pgfusepath{clip}%
\pgfsetbuttcap%
\pgfsetroundjoin%
\definecolor{currentfill}{rgb}{0.277018,0.050344,0.375715}%
\pgfsetfillcolor{currentfill}%
\pgfsetfillopacity{0.700000}%
\pgfsetlinewidth{0.000000pt}%
\definecolor{currentstroke}{rgb}{0.000000,0.000000,0.000000}%
\pgfsetstrokecolor{currentstroke}%
\pgfsetdash{}{0pt}%
\pgfpathmoveto{\pgfqpoint{4.843785in}{1.698489in}}%
\pgfpathlineto{\pgfqpoint{4.857425in}{1.696117in}}%
\pgfpathlineto{\pgfqpoint{4.871073in}{1.693768in}}%
\pgfpathlineto{\pgfqpoint{4.884729in}{1.691443in}}%
\pgfpathlineto{\pgfqpoint{4.898392in}{1.689141in}}%
\pgfpathlineto{\pgfqpoint{4.890902in}{1.679121in}}%
\pgfpathlineto{\pgfqpoint{4.883408in}{1.669143in}}%
\pgfpathlineto{\pgfqpoint{4.875910in}{1.659212in}}%
\pgfpathlineto{\pgfqpoint{4.868407in}{1.649332in}}%
\pgfpathlineto{\pgfqpoint{4.854737in}{1.651832in}}%
\pgfpathlineto{\pgfqpoint{4.841073in}{1.654355in}}%
\pgfpathlineto{\pgfqpoint{4.827418in}{1.656902in}}%
\pgfpathlineto{\pgfqpoint{4.813769in}{1.659472in}}%
\pgfpathlineto{\pgfqpoint{4.821280in}{1.669149in}}%
\pgfpathlineto{\pgfqpoint{4.828786in}{1.678880in}}%
\pgfpathlineto{\pgfqpoint{4.836288in}{1.688662in}}%
\pgfpathlineto{\pgfqpoint{4.843785in}{1.698489in}}%
\pgfpathclose%
\pgfusepath{fill}%
\end{pgfscope}%
\begin{pgfscope}%
\pgfpathrectangle{\pgfqpoint{1.254980in}{0.150000in}}{\pgfqpoint{5.490039in}{5.490039in}}%
\pgfusepath{clip}%
\pgfsetbuttcap%
\pgfsetroundjoin%
\definecolor{currentfill}{rgb}{0.177423,0.437527,0.557565}%
\pgfsetfillcolor{currentfill}%
\pgfsetfillopacity{0.700000}%
\pgfsetlinewidth{0.000000pt}%
\definecolor{currentstroke}{rgb}{0.000000,0.000000,0.000000}%
\pgfsetstrokecolor{currentstroke}%
\pgfsetdash{}{0pt}%
\pgfpathmoveto{\pgfqpoint{2.506708in}{2.505885in}}%
\pgfpathlineto{\pgfqpoint{2.519971in}{2.495766in}}%
\pgfpathlineto{\pgfqpoint{2.533235in}{2.485689in}}%
\pgfpathlineto{\pgfqpoint{2.546501in}{2.475656in}}%
\pgfpathlineto{\pgfqpoint{2.559768in}{2.465664in}}%
\pgfpathlineto{\pgfqpoint{2.550795in}{2.479508in}}%
\pgfpathlineto{\pgfqpoint{2.541791in}{2.493909in}}%
\pgfpathlineto{\pgfqpoint{2.532755in}{2.508878in}}%
\pgfpathlineto{\pgfqpoint{2.523688in}{2.524426in}}%
\pgfpathlineto{\pgfqpoint{2.510370in}{2.534803in}}%
\pgfpathlineto{\pgfqpoint{2.497054in}{2.545222in}}%
\pgfpathlineto{\pgfqpoint{2.483739in}{2.555684in}}%
\pgfpathlineto{\pgfqpoint{2.470425in}{2.566189in}}%
\pgfpathlineto{\pgfqpoint{2.479544in}{2.550249in}}%
\pgfpathlineto{\pgfqpoint{2.488630in}{2.534892in}}%
\pgfpathlineto{\pgfqpoint{2.497685in}{2.520108in}}%
\pgfpathlineto{\pgfqpoint{2.506708in}{2.505885in}}%
\pgfpathclose%
\pgfusepath{fill}%
\end{pgfscope}%
\begin{pgfscope}%
\pgfpathrectangle{\pgfqpoint{1.254980in}{0.150000in}}{\pgfqpoint{5.490039in}{5.490039in}}%
\pgfusepath{clip}%
\pgfsetbuttcap%
\pgfsetroundjoin%
\definecolor{currentfill}{rgb}{0.282623,0.140926,0.457517}%
\pgfsetfillcolor{currentfill}%
\pgfsetfillopacity{0.700000}%
\pgfsetlinewidth{0.000000pt}%
\definecolor{currentstroke}{rgb}{0.000000,0.000000,0.000000}%
\pgfsetstrokecolor{currentstroke}%
\pgfsetdash{}{0pt}%
\pgfpathmoveto{\pgfqpoint{5.321433in}{1.866644in}}%
\pgfpathlineto{\pgfqpoint{5.335233in}{1.865632in}}%
\pgfpathlineto{\pgfqpoint{5.349041in}{1.864644in}}%
\pgfpathlineto{\pgfqpoint{5.362857in}{1.863680in}}%
\pgfpathlineto{\pgfqpoint{5.376683in}{1.862739in}}%
\pgfpathlineto{\pgfqpoint{5.369332in}{1.851970in}}%
\pgfpathlineto{\pgfqpoint{5.361975in}{1.841147in}}%
\pgfpathlineto{\pgfqpoint{5.354613in}{1.830273in}}%
\pgfpathlineto{\pgfqpoint{5.347245in}{1.819350in}}%
\pgfpathlineto{\pgfqpoint{5.333414in}{1.820425in}}%
\pgfpathlineto{\pgfqpoint{5.319591in}{1.821524in}}%
\pgfpathlineto{\pgfqpoint{5.305777in}{1.822646in}}%
\pgfpathlineto{\pgfqpoint{5.291971in}{1.823791in}}%
\pgfpathlineto{\pgfqpoint{5.299345in}{1.834575in}}%
\pgfpathlineto{\pgfqpoint{5.306714in}{1.845314in}}%
\pgfpathlineto{\pgfqpoint{5.314077in}{1.856004in}}%
\pgfpathlineto{\pgfqpoint{5.321433in}{1.866644in}}%
\pgfpathclose%
\pgfusepath{fill}%
\end{pgfscope}%
\begin{pgfscope}%
\pgfpathrectangle{\pgfqpoint{1.254980in}{0.150000in}}{\pgfqpoint{5.490039in}{5.490039in}}%
\pgfusepath{clip}%
\pgfsetbuttcap%
\pgfsetroundjoin%
\definecolor{currentfill}{rgb}{0.282910,0.105393,0.426902}%
\pgfsetfillcolor{currentfill}%
\pgfsetfillopacity{0.700000}%
\pgfsetlinewidth{0.000000pt}%
\definecolor{currentstroke}{rgb}{0.000000,0.000000,0.000000}%
\pgfsetstrokecolor{currentstroke}%
\pgfsetdash{}{0pt}%
\pgfpathmoveto{\pgfqpoint{3.651002in}{1.782190in}}%
\pgfpathlineto{\pgfqpoint{3.664360in}{1.775926in}}%
\pgfpathlineto{\pgfqpoint{3.677723in}{1.769689in}}%
\pgfpathlineto{\pgfqpoint{3.691091in}{1.763477in}}%
\pgfpathlineto{\pgfqpoint{3.704464in}{1.757292in}}%
\pgfpathlineto{\pgfqpoint{3.696496in}{1.757704in}}%
\pgfpathlineto{\pgfqpoint{3.688518in}{1.758446in}}%
\pgfpathlineto{\pgfqpoint{3.680528in}{1.759527in}}%
\pgfpathlineto{\pgfqpoint{3.672527in}{1.760954in}}%
\pgfpathlineto{\pgfqpoint{3.659128in}{1.767454in}}%
\pgfpathlineto{\pgfqpoint{3.645733in}{1.773980in}}%
\pgfpathlineto{\pgfqpoint{3.632344in}{1.780532in}}%
\pgfpathlineto{\pgfqpoint{3.618959in}{1.787111in}}%
\pgfpathlineto{\pgfqpoint{3.626988in}{1.785363in}}%
\pgfpathlineto{\pgfqpoint{3.635004in}{1.783966in}}%
\pgfpathlineto{\pgfqpoint{3.643009in}{1.782911in}}%
\pgfpathlineto{\pgfqpoint{3.651002in}{1.782190in}}%
\pgfpathclose%
\pgfusepath{fill}%
\end{pgfscope}%
\begin{pgfscope}%
\pgfpathrectangle{\pgfqpoint{1.254980in}{0.150000in}}{\pgfqpoint{5.490039in}{5.490039in}}%
\pgfusepath{clip}%
\pgfsetbuttcap%
\pgfsetroundjoin%
\definecolor{currentfill}{rgb}{0.283187,0.125848,0.444960}%
\pgfsetfillcolor{currentfill}%
\pgfsetfillopacity{0.700000}%
\pgfsetlinewidth{0.000000pt}%
\definecolor{currentstroke}{rgb}{0.000000,0.000000,0.000000}%
\pgfsetstrokecolor{currentstroke}%
\pgfsetdash{}{0pt}%
\pgfpathmoveto{\pgfqpoint{5.236833in}{1.828608in}}%
\pgfpathlineto{\pgfqpoint{5.250605in}{1.827369in}}%
\pgfpathlineto{\pgfqpoint{5.264385in}{1.826153in}}%
\pgfpathlineto{\pgfqpoint{5.278174in}{1.824960in}}%
\pgfpathlineto{\pgfqpoint{5.291971in}{1.823791in}}%
\pgfpathlineto{\pgfqpoint{5.284591in}{1.812966in}}%
\pgfpathlineto{\pgfqpoint{5.277206in}{1.802102in}}%
\pgfpathlineto{\pgfqpoint{5.269816in}{1.791202in}}%
\pgfpathlineto{\pgfqpoint{5.262420in}{1.780271in}}%
\pgfpathlineto{\pgfqpoint{5.248616in}{1.781587in}}%
\pgfpathlineto{\pgfqpoint{5.234821in}{1.782927in}}%
\pgfpathlineto{\pgfqpoint{5.221035in}{1.784290in}}%
\pgfpathlineto{\pgfqpoint{5.207257in}{1.785676in}}%
\pgfpathlineto{\pgfqpoint{5.214659in}{1.796455in}}%
\pgfpathlineto{\pgfqpoint{5.222055in}{1.807206in}}%
\pgfpathlineto{\pgfqpoint{5.229447in}{1.817925in}}%
\pgfpathlineto{\pgfqpoint{5.236833in}{1.828608in}}%
\pgfpathclose%
\pgfusepath{fill}%
\end{pgfscope}%
\begin{pgfscope}%
\pgfpathrectangle{\pgfqpoint{1.254980in}{0.150000in}}{\pgfqpoint{5.490039in}{5.490039in}}%
\pgfusepath{clip}%
\pgfsetbuttcap%
\pgfsetroundjoin%
\definecolor{currentfill}{rgb}{0.269944,0.014625,0.341379}%
\pgfsetfillcolor{currentfill}%
\pgfsetfillopacity{0.700000}%
\pgfsetlinewidth{0.000000pt}%
\definecolor{currentstroke}{rgb}{0.000000,0.000000,0.000000}%
\pgfsetstrokecolor{currentstroke}%
\pgfsetdash{}{0pt}%
\pgfpathmoveto{\pgfqpoint{4.535853in}{1.635888in}}%
\pgfpathlineto{\pgfqpoint{4.549407in}{1.632510in}}%
\pgfpathlineto{\pgfqpoint{4.562967in}{1.629155in}}%
\pgfpathlineto{\pgfqpoint{4.576534in}{1.625824in}}%
\pgfpathlineto{\pgfqpoint{4.590107in}{1.622517in}}%
\pgfpathlineto{\pgfqpoint{4.582531in}{1.614262in}}%
\pgfpathlineto{\pgfqpoint{4.574950in}{1.606125in}}%
\pgfpathlineto{\pgfqpoint{4.567364in}{1.598112in}}%
\pgfpathlineto{\pgfqpoint{4.559775in}{1.590229in}}%
\pgfpathlineto{\pgfqpoint{4.546190in}{1.593773in}}%
\pgfpathlineto{\pgfqpoint{4.532612in}{1.597340in}}%
\pgfpathlineto{\pgfqpoint{4.519040in}{1.600930in}}%
\pgfpathlineto{\pgfqpoint{4.505476in}{1.604544in}}%
\pgfpathlineto{\pgfqpoint{4.513077in}{1.612186in}}%
\pgfpathlineto{\pgfqpoint{4.520674in}{1.619962in}}%
\pgfpathlineto{\pgfqpoint{4.528266in}{1.627864in}}%
\pgfpathlineto{\pgfqpoint{4.535853in}{1.635888in}}%
\pgfpathclose%
\pgfusepath{fill}%
\end{pgfscope}%
\begin{pgfscope}%
\pgfpathrectangle{\pgfqpoint{1.254980in}{0.150000in}}{\pgfqpoint{5.490039in}{5.490039in}}%
\pgfusepath{clip}%
\pgfsetbuttcap%
\pgfsetroundjoin%
\definecolor{currentfill}{rgb}{0.235526,0.309527,0.542944}%
\pgfsetfillcolor{currentfill}%
\pgfsetfillopacity{0.700000}%
\pgfsetlinewidth{0.000000pt}%
\definecolor{currentstroke}{rgb}{0.000000,0.000000,0.000000}%
\pgfsetstrokecolor{currentstroke}%
\pgfsetdash{}{0pt}%
\pgfpathmoveto{\pgfqpoint{2.913285in}{2.196387in}}%
\pgfpathlineto{\pgfqpoint{2.926561in}{2.187706in}}%
\pgfpathlineto{\pgfqpoint{2.939841in}{2.179060in}}%
\pgfpathlineto{\pgfqpoint{2.953124in}{2.170447in}}%
\pgfpathlineto{\pgfqpoint{2.966409in}{2.161868in}}%
\pgfpathlineto{\pgfqpoint{2.957854in}{2.171078in}}%
\pgfpathlineto{\pgfqpoint{2.949276in}{2.180773in}}%
\pgfpathlineto{\pgfqpoint{2.940675in}{2.190964in}}%
\pgfpathlineto{\pgfqpoint{2.932050in}{2.201661in}}%
\pgfpathlineto{\pgfqpoint{2.918723in}{2.210603in}}%
\pgfpathlineto{\pgfqpoint{2.905398in}{2.219579in}}%
\pgfpathlineto{\pgfqpoint{2.892076in}{2.228589in}}%
\pgfpathlineto{\pgfqpoint{2.878757in}{2.237633in}}%
\pgfpathlineto{\pgfqpoint{2.887425in}{2.226567in}}%
\pgfpathlineto{\pgfqpoint{2.896069in}{2.216010in}}%
\pgfpathlineto{\pgfqpoint{2.904688in}{2.205954in}}%
\pgfpathlineto{\pgfqpoint{2.913285in}{2.196387in}}%
\pgfpathclose%
\pgfusepath{fill}%
\end{pgfscope}%
\begin{pgfscope}%
\pgfpathrectangle{\pgfqpoint{1.254980in}{0.150000in}}{\pgfqpoint{5.490039in}{5.490039in}}%
\pgfusepath{clip}%
\pgfsetbuttcap%
\pgfsetroundjoin%
\definecolor{currentfill}{rgb}{0.279566,0.067836,0.391917}%
\pgfsetfillcolor{currentfill}%
\pgfsetfillopacity{0.700000}%
\pgfsetlinewidth{0.000000pt}%
\definecolor{currentstroke}{rgb}{0.000000,0.000000,0.000000}%
\pgfsetstrokecolor{currentstroke}%
\pgfsetdash{}{0pt}%
\pgfpathmoveto{\pgfqpoint{3.843191in}{1.712633in}}%
\pgfpathlineto{\pgfqpoint{3.856584in}{1.706979in}}%
\pgfpathlineto{\pgfqpoint{3.869983in}{1.701351in}}%
\pgfpathlineto{\pgfqpoint{3.883387in}{1.695748in}}%
\pgfpathlineto{\pgfqpoint{3.896796in}{1.690170in}}%
\pgfpathlineto{\pgfqpoint{3.888939in}{1.688458in}}%
\pgfpathlineto{\pgfqpoint{3.881074in}{1.687033in}}%
\pgfpathlineto{\pgfqpoint{3.873199in}{1.685902in}}%
\pgfpathlineto{\pgfqpoint{3.865315in}{1.685075in}}%
\pgfpathlineto{\pgfqpoint{3.851883in}{1.690954in}}%
\pgfpathlineto{\pgfqpoint{3.838456in}{1.696857in}}%
\pgfpathlineto{\pgfqpoint{3.825035in}{1.702786in}}%
\pgfpathlineto{\pgfqpoint{3.811618in}{1.708740in}}%
\pgfpathlineto{\pgfqpoint{3.819526in}{1.709261in}}%
\pgfpathlineto{\pgfqpoint{3.827424in}{1.710089in}}%
\pgfpathlineto{\pgfqpoint{3.835312in}{1.711215in}}%
\pgfpathlineto{\pgfqpoint{3.843191in}{1.712633in}}%
\pgfpathclose%
\pgfusepath{fill}%
\end{pgfscope}%
\begin{pgfscope}%
\pgfpathrectangle{\pgfqpoint{1.254980in}{0.150000in}}{\pgfqpoint{5.490039in}{5.490039in}}%
\pgfusepath{clip}%
\pgfsetbuttcap%
\pgfsetroundjoin%
\definecolor{currentfill}{rgb}{0.274952,0.037752,0.364543}%
\pgfsetfillcolor{currentfill}%
\pgfsetfillopacity{0.700000}%
\pgfsetlinewidth{0.000000pt}%
\definecolor{currentstroke}{rgb}{0.000000,0.000000,0.000000}%
\pgfsetstrokecolor{currentstroke}%
\pgfsetdash{}{0pt}%
\pgfpathmoveto{\pgfqpoint{4.759250in}{1.669985in}}%
\pgfpathlineto{\pgfqpoint{4.772869in}{1.667322in}}%
\pgfpathlineto{\pgfqpoint{4.786495in}{1.664682in}}%
\pgfpathlineto{\pgfqpoint{4.800128in}{1.662065in}}%
\pgfpathlineto{\pgfqpoint{4.813769in}{1.659472in}}%
\pgfpathlineto{\pgfqpoint{4.806255in}{1.649855in}}%
\pgfpathlineto{\pgfqpoint{4.798736in}{1.640303in}}%
\pgfpathlineto{\pgfqpoint{4.791213in}{1.630821in}}%
\pgfpathlineto{\pgfqpoint{4.783686in}{1.621414in}}%
\pgfpathlineto{\pgfqpoint{4.770037in}{1.624218in}}%
\pgfpathlineto{\pgfqpoint{4.756395in}{1.627046in}}%
\pgfpathlineto{\pgfqpoint{4.742760in}{1.629896in}}%
\pgfpathlineto{\pgfqpoint{4.729133in}{1.632770in}}%
\pgfpathlineto{\pgfqpoint{4.736668in}{1.641961in}}%
\pgfpathlineto{\pgfqpoint{4.744200in}{1.651231in}}%
\pgfpathlineto{\pgfqpoint{4.751727in}{1.660574in}}%
\pgfpathlineto{\pgfqpoint{4.759250in}{1.669985in}}%
\pgfpathclose%
\pgfusepath{fill}%
\end{pgfscope}%
\begin{pgfscope}%
\pgfpathrectangle{\pgfqpoint{1.254980in}{0.150000in}}{\pgfqpoint{5.490039in}{5.490039in}}%
\pgfusepath{clip}%
\pgfsetbuttcap%
\pgfsetroundjoin%
\definecolor{currentfill}{rgb}{0.269308,0.218818,0.509577}%
\pgfsetfillcolor{currentfill}%
\pgfsetfillopacity{0.700000}%
\pgfsetlinewidth{0.000000pt}%
\definecolor{currentstroke}{rgb}{0.000000,0.000000,0.000000}%
\pgfsetstrokecolor{currentstroke}%
\pgfsetdash{}{0pt}%
\pgfpathmoveto{\pgfqpoint{3.212791in}{2.002359in}}%
\pgfpathlineto{\pgfqpoint{3.226093in}{1.994665in}}%
\pgfpathlineto{\pgfqpoint{3.239400in}{1.987001in}}%
\pgfpathlineto{\pgfqpoint{3.252709in}{1.979367in}}%
\pgfpathlineto{\pgfqpoint{3.266023in}{1.971763in}}%
\pgfpathlineto{\pgfqpoint{3.257735in}{1.977428in}}%
\pgfpathlineto{\pgfqpoint{3.249429in}{1.983520in}}%
\pgfpathlineto{\pgfqpoint{3.241105in}{1.990048in}}%
\pgfpathlineto{\pgfqpoint{3.232763in}{1.997023in}}%
\pgfpathlineto{\pgfqpoint{3.219414in}{2.004973in}}%
\pgfpathlineto{\pgfqpoint{3.206069in}{2.012952in}}%
\pgfpathlineto{\pgfqpoint{3.192726in}{2.020961in}}%
\pgfpathlineto{\pgfqpoint{3.179388in}{2.029001in}}%
\pgfpathlineto{\pgfqpoint{3.187767in}{2.021675in}}%
\pgfpathlineto{\pgfqpoint{3.196127in}{2.014800in}}%
\pgfpathlineto{\pgfqpoint{3.204468in}{2.008364in}}%
\pgfpathlineto{\pgfqpoint{3.212791in}{2.002359in}}%
\pgfpathclose%
\pgfusepath{fill}%
\end{pgfscope}%
\begin{pgfscope}%
\pgfpathrectangle{\pgfqpoint{1.254980in}{0.150000in}}{\pgfqpoint{5.490039in}{5.490039in}}%
\pgfusepath{clip}%
\pgfsetbuttcap%
\pgfsetroundjoin%
\definecolor{currentfill}{rgb}{0.282910,0.105393,0.426902}%
\pgfsetfillcolor{currentfill}%
\pgfsetfillopacity{0.700000}%
\pgfsetlinewidth{0.000000pt}%
\definecolor{currentstroke}{rgb}{0.000000,0.000000,0.000000}%
\pgfsetstrokecolor{currentstroke}%
\pgfsetdash{}{0pt}%
\pgfpathmoveto{\pgfqpoint{5.152227in}{1.791457in}}%
\pgfpathlineto{\pgfqpoint{5.165972in}{1.789976in}}%
\pgfpathlineto{\pgfqpoint{5.179725in}{1.788520in}}%
\pgfpathlineto{\pgfqpoint{5.193487in}{1.787086in}}%
\pgfpathlineto{\pgfqpoint{5.207257in}{1.785676in}}%
\pgfpathlineto{\pgfqpoint{5.199850in}{1.774872in}}%
\pgfpathlineto{\pgfqpoint{5.192437in}{1.764047in}}%
\pgfpathlineto{\pgfqpoint{5.185020in}{1.753204in}}%
\pgfpathlineto{\pgfqpoint{5.177598in}{1.742347in}}%
\pgfpathlineto{\pgfqpoint{5.163822in}{1.743917in}}%
\pgfpathlineto{\pgfqpoint{5.150054in}{1.745510in}}%
\pgfpathlineto{\pgfqpoint{5.136294in}{1.747127in}}%
\pgfpathlineto{\pgfqpoint{5.122543in}{1.748767in}}%
\pgfpathlineto{\pgfqpoint{5.129971in}{1.759459in}}%
\pgfpathlineto{\pgfqpoint{5.137395in}{1.770140in}}%
\pgfpathlineto{\pgfqpoint{5.144813in}{1.780808in}}%
\pgfpathlineto{\pgfqpoint{5.152227in}{1.791457in}}%
\pgfpathclose%
\pgfusepath{fill}%
\end{pgfscope}%
\begin{pgfscope}%
\pgfpathrectangle{\pgfqpoint{1.254980in}{0.150000in}}{\pgfqpoint{5.490039in}{5.490039in}}%
\pgfusepath{clip}%
\pgfsetbuttcap%
\pgfsetroundjoin%
\definecolor{currentfill}{rgb}{0.281412,0.155834,0.469201}%
\pgfsetfillcolor{currentfill}%
\pgfsetfillopacity{0.700000}%
\pgfsetlinewidth{0.000000pt}%
\definecolor{currentstroke}{rgb}{0.000000,0.000000,0.000000}%
\pgfsetstrokecolor{currentstroke}%
\pgfsetdash{}{0pt}%
\pgfpathmoveto{\pgfqpoint{3.458682in}{1.868146in}}%
\pgfpathlineto{\pgfqpoint{3.472014in}{1.861243in}}%
\pgfpathlineto{\pgfqpoint{3.485351in}{1.854368in}}%
\pgfpathlineto{\pgfqpoint{3.498693in}{1.847520in}}%
\pgfpathlineto{\pgfqpoint{3.512038in}{1.840700in}}%
\pgfpathlineto{\pgfqpoint{3.503939in}{1.843454in}}%
\pgfpathlineto{\pgfqpoint{3.495827in}{1.846584in}}%
\pgfpathlineto{\pgfqpoint{3.487700in}{1.850097in}}%
\pgfpathlineto{\pgfqpoint{3.479559in}{1.854004in}}%
\pgfpathlineto{\pgfqpoint{3.466183in}{1.861154in}}%
\pgfpathlineto{\pgfqpoint{3.452811in}{1.868331in}}%
\pgfpathlineto{\pgfqpoint{3.439443in}{1.875535in}}%
\pgfpathlineto{\pgfqpoint{3.426080in}{1.882768in}}%
\pgfpathlineto{\pgfqpoint{3.434253in}{1.878526in}}%
\pgfpathlineto{\pgfqpoint{3.442410in}{1.874681in}}%
\pgfpathlineto{\pgfqpoint{3.450553in}{1.871224in}}%
\pgfpathlineto{\pgfqpoint{3.458682in}{1.868146in}}%
\pgfpathclose%
\pgfusepath{fill}%
\end{pgfscope}%
\begin{pgfscope}%
\pgfpathrectangle{\pgfqpoint{1.254980in}{0.150000in}}{\pgfqpoint{5.490039in}{5.490039in}}%
\pgfusepath{clip}%
\pgfsetbuttcap%
\pgfsetroundjoin%
\definecolor{currentfill}{rgb}{0.183898,0.422383,0.556944}%
\pgfsetfillcolor{currentfill}%
\pgfsetfillopacity{0.700000}%
\pgfsetlinewidth{0.000000pt}%
\definecolor{currentstroke}{rgb}{0.000000,0.000000,0.000000}%
\pgfsetstrokecolor{currentstroke}%
\pgfsetdash{}{0pt}%
\pgfpathmoveto{\pgfqpoint{2.559768in}{2.465664in}}%
\pgfpathlineto{\pgfqpoint{2.573037in}{2.455714in}}%
\pgfpathlineto{\pgfqpoint{2.586308in}{2.445806in}}%
\pgfpathlineto{\pgfqpoint{2.599580in}{2.435939in}}%
\pgfpathlineto{\pgfqpoint{2.612854in}{2.426112in}}%
\pgfpathlineto{\pgfqpoint{2.603929in}{2.439578in}}%
\pgfpathlineto{\pgfqpoint{2.594974in}{2.453597in}}%
\pgfpathlineto{\pgfqpoint{2.585990in}{2.468179in}}%
\pgfpathlineto{\pgfqpoint{2.576974in}{2.483336in}}%
\pgfpathlineto{\pgfqpoint{2.563650in}{2.493547in}}%
\pgfpathlineto{\pgfqpoint{2.550328in}{2.503799in}}%
\pgfpathlineto{\pgfqpoint{2.537007in}{2.514092in}}%
\pgfpathlineto{\pgfqpoint{2.523688in}{2.524426in}}%
\pgfpathlineto{\pgfqpoint{2.532755in}{2.508878in}}%
\pgfpathlineto{\pgfqpoint{2.541791in}{2.493909in}}%
\pgfpathlineto{\pgfqpoint{2.550795in}{2.479508in}}%
\pgfpathlineto{\pgfqpoint{2.559768in}{2.465664in}}%
\pgfpathclose%
\pgfusepath{fill}%
\end{pgfscope}%
\begin{pgfscope}%
\pgfpathrectangle{\pgfqpoint{1.254980in}{0.150000in}}{\pgfqpoint{5.490039in}{5.490039in}}%
\pgfusepath{clip}%
\pgfsetbuttcap%
\pgfsetroundjoin%
\definecolor{currentfill}{rgb}{0.281924,0.089666,0.412415}%
\pgfsetfillcolor{currentfill}%
\pgfsetfillopacity{0.700000}%
\pgfsetlinewidth{0.000000pt}%
\definecolor{currentstroke}{rgb}{0.000000,0.000000,0.000000}%
\pgfsetstrokecolor{currentstroke}%
\pgfsetdash{}{0pt}%
\pgfpathmoveto{\pgfqpoint{5.067617in}{1.755561in}}%
\pgfpathlineto{\pgfqpoint{5.081337in}{1.753828in}}%
\pgfpathlineto{\pgfqpoint{5.095064in}{1.752117in}}%
\pgfpathlineto{\pgfqpoint{5.108799in}{1.750431in}}%
\pgfpathlineto{\pgfqpoint{5.122543in}{1.748767in}}%
\pgfpathlineto{\pgfqpoint{5.115110in}{1.738069in}}%
\pgfpathlineto{\pgfqpoint{5.107672in}{1.727367in}}%
\pgfpathlineto{\pgfqpoint{5.100229in}{1.716667in}}%
\pgfpathlineto{\pgfqpoint{5.092782in}{1.705973in}}%
\pgfpathlineto{\pgfqpoint{5.079032in}{1.707810in}}%
\pgfpathlineto{\pgfqpoint{5.065291in}{1.709669in}}%
\pgfpathlineto{\pgfqpoint{5.051557in}{1.711552in}}%
\pgfpathlineto{\pgfqpoint{5.037831in}{1.713459in}}%
\pgfpathlineto{\pgfqpoint{5.045284in}{1.723975in}}%
\pgfpathlineto{\pgfqpoint{5.052733in}{1.734501in}}%
\pgfpathlineto{\pgfqpoint{5.060178in}{1.745031in}}%
\pgfpathlineto{\pgfqpoint{5.067617in}{1.755561in}}%
\pgfpathclose%
\pgfusepath{fill}%
\end{pgfscope}%
\begin{pgfscope}%
\pgfpathrectangle{\pgfqpoint{1.254980in}{0.150000in}}{\pgfqpoint{5.490039in}{5.490039in}}%
\pgfusepath{clip}%
\pgfsetbuttcap%
\pgfsetroundjoin%
\definecolor{currentfill}{rgb}{0.272594,0.025563,0.353093}%
\pgfsetfillcolor{currentfill}%
\pgfsetfillopacity{0.700000}%
\pgfsetlinewidth{0.000000pt}%
\definecolor{currentstroke}{rgb}{0.000000,0.000000,0.000000}%
\pgfsetstrokecolor{currentstroke}%
\pgfsetdash{}{0pt}%
\pgfpathmoveto{\pgfqpoint{4.174016in}{1.636390in}}%
\pgfpathlineto{\pgfqpoint{4.187482in}{1.631791in}}%
\pgfpathlineto{\pgfqpoint{4.200954in}{1.627217in}}%
\pgfpathlineto{\pgfqpoint{4.214432in}{1.622666in}}%
\pgfpathlineto{\pgfqpoint{4.227916in}{1.618140in}}%
\pgfpathlineto{\pgfqpoint{4.220213in}{1.613058in}}%
\pgfpathlineto{\pgfqpoint{4.212503in}{1.608187in}}%
\pgfpathlineto{\pgfqpoint{4.204787in}{1.603534in}}%
\pgfpathlineto{\pgfqpoint{4.197065in}{1.599106in}}%
\pgfpathlineto{\pgfqpoint{4.183564in}{1.603907in}}%
\pgfpathlineto{\pgfqpoint{4.170069in}{1.608732in}}%
\pgfpathlineto{\pgfqpoint{4.156580in}{1.613580in}}%
\pgfpathlineto{\pgfqpoint{4.143097in}{1.618453in}}%
\pgfpathlineto{\pgfqpoint{4.150837in}{1.622602in}}%
\pgfpathlineto{\pgfqpoint{4.158570in}{1.626979in}}%
\pgfpathlineto{\pgfqpoint{4.166296in}{1.631577in}}%
\pgfpathlineto{\pgfqpoint{4.174016in}{1.636390in}}%
\pgfpathclose%
\pgfusepath{fill}%
\end{pgfscope}%
\begin{pgfscope}%
\pgfpathrectangle{\pgfqpoint{1.254980in}{0.150000in}}{\pgfqpoint{5.490039in}{5.490039in}}%
\pgfusepath{clip}%
\pgfsetbuttcap%
\pgfsetroundjoin%
\definecolor{currentfill}{rgb}{0.269944,0.014625,0.341379}%
\pgfsetfillcolor{currentfill}%
\pgfsetfillopacity{0.700000}%
\pgfsetlinewidth{0.000000pt}%
\definecolor{currentstroke}{rgb}{0.000000,0.000000,0.000000}%
\pgfsetstrokecolor{currentstroke}%
\pgfsetdash{}{0pt}%
\pgfpathmoveto{\pgfqpoint{4.312608in}{1.623618in}}%
\pgfpathlineto{\pgfqpoint{4.326107in}{1.619472in}}%
\pgfpathlineto{\pgfqpoint{4.339613in}{1.615350in}}%
\pgfpathlineto{\pgfqpoint{4.353126in}{1.611252in}}%
\pgfpathlineto{\pgfqpoint{4.366644in}{1.607177in}}%
\pgfpathlineto{\pgfqpoint{4.358993in}{1.600807in}}%
\pgfpathlineto{\pgfqpoint{4.351337in}{1.594614in}}%
\pgfpathlineto{\pgfqpoint{4.343675in}{1.588605in}}%
\pgfpathlineto{\pgfqpoint{4.336008in}{1.582786in}}%
\pgfpathlineto{\pgfqpoint{4.322475in}{1.587122in}}%
\pgfpathlineto{\pgfqpoint{4.308948in}{1.591482in}}%
\pgfpathlineto{\pgfqpoint{4.295427in}{1.595865in}}%
\pgfpathlineto{\pgfqpoint{4.281913in}{1.600272in}}%
\pgfpathlineto{\pgfqpoint{4.289595in}{1.605825in}}%
\pgfpathlineto{\pgfqpoint{4.297271in}{1.611571in}}%
\pgfpathlineto{\pgfqpoint{4.304942in}{1.617504in}}%
\pgfpathlineto{\pgfqpoint{4.312608in}{1.623618in}}%
\pgfpathclose%
\pgfusepath{fill}%
\end{pgfscope}%
\begin{pgfscope}%
\pgfpathrectangle{\pgfqpoint{1.254980in}{0.150000in}}{\pgfqpoint{5.490039in}{5.490039in}}%
\pgfusepath{clip}%
\pgfsetbuttcap%
\pgfsetroundjoin%
\definecolor{currentfill}{rgb}{0.274952,0.037752,0.364543}%
\pgfsetfillcolor{currentfill}%
\pgfsetfillopacity{0.700000}%
\pgfsetlinewidth{0.000000pt}%
\definecolor{currentstroke}{rgb}{0.000000,0.000000,0.000000}%
\pgfsetstrokecolor{currentstroke}%
\pgfsetdash{}{0pt}%
\pgfpathmoveto{\pgfqpoint{4.035438in}{1.658309in}}%
\pgfpathlineto{\pgfqpoint{4.048876in}{1.653242in}}%
\pgfpathlineto{\pgfqpoint{4.062319in}{1.648199in}}%
\pgfpathlineto{\pgfqpoint{4.075768in}{1.643181in}}%
\pgfpathlineto{\pgfqpoint{4.089222in}{1.638186in}}%
\pgfpathlineto{\pgfqpoint{4.081458in}{1.634556in}}%
\pgfpathlineto{\pgfqpoint{4.073686in}{1.631172in}}%
\pgfpathlineto{\pgfqpoint{4.065907in}{1.628042in}}%
\pgfpathlineto{\pgfqpoint{4.058121in}{1.625174in}}%
\pgfpathlineto{\pgfqpoint{4.044647in}{1.630455in}}%
\pgfpathlineto{\pgfqpoint{4.031179in}{1.635761in}}%
\pgfpathlineto{\pgfqpoint{4.017717in}{1.641091in}}%
\pgfpathlineto{\pgfqpoint{4.004259in}{1.646445in}}%
\pgfpathlineto{\pgfqpoint{4.012066in}{1.649021in}}%
\pgfpathlineto{\pgfqpoint{4.019864in}{1.651862in}}%
\pgfpathlineto{\pgfqpoint{4.027655in}{1.654961in}}%
\pgfpathlineto{\pgfqpoint{4.035438in}{1.658309in}}%
\pgfpathclose%
\pgfusepath{fill}%
\end{pgfscope}%
\begin{pgfscope}%
\pgfpathrectangle{\pgfqpoint{1.254980in}{0.150000in}}{\pgfqpoint{5.490039in}{5.490039in}}%
\pgfusepath{clip}%
\pgfsetbuttcap%
\pgfsetroundjoin%
\definecolor{currentfill}{rgb}{0.272594,0.025563,0.353093}%
\pgfsetfillcolor{currentfill}%
\pgfsetfillopacity{0.700000}%
\pgfsetlinewidth{0.000000pt}%
\definecolor{currentstroke}{rgb}{0.000000,0.000000,0.000000}%
\pgfsetstrokecolor{currentstroke}%
\pgfsetdash{}{0pt}%
\pgfpathmoveto{\pgfqpoint{4.674694in}{1.644499in}}%
\pgfpathlineto{\pgfqpoint{4.688293in}{1.641532in}}%
\pgfpathlineto{\pgfqpoint{4.701899in}{1.638588in}}%
\pgfpathlineto{\pgfqpoint{4.715512in}{1.635667in}}%
\pgfpathlineto{\pgfqpoint{4.729133in}{1.632770in}}%
\pgfpathlineto{\pgfqpoint{4.721593in}{1.623663in}}%
\pgfpathlineto{\pgfqpoint{4.714049in}{1.614644in}}%
\pgfpathlineto{\pgfqpoint{4.706501in}{1.605721in}}%
\pgfpathlineto{\pgfqpoint{4.698948in}{1.596897in}}%
\pgfpathlineto{\pgfqpoint{4.685319in}{1.600018in}}%
\pgfpathlineto{\pgfqpoint{4.671696in}{1.603162in}}%
\pgfpathlineto{\pgfqpoint{4.658081in}{1.606329in}}%
\pgfpathlineto{\pgfqpoint{4.644472in}{1.609520in}}%
\pgfpathlineto{\pgfqpoint{4.652034in}{1.618115in}}%
\pgfpathlineto{\pgfqpoint{4.659592in}{1.626814in}}%
\pgfpathlineto{\pgfqpoint{4.667145in}{1.635610in}}%
\pgfpathlineto{\pgfqpoint{4.674694in}{1.644499in}}%
\pgfpathclose%
\pgfusepath{fill}%
\end{pgfscope}%
\begin{pgfscope}%
\pgfpathrectangle{\pgfqpoint{1.254980in}{0.150000in}}{\pgfqpoint{5.490039in}{5.490039in}}%
\pgfusepath{clip}%
\pgfsetbuttcap%
\pgfsetroundjoin%
\definecolor{currentfill}{rgb}{0.280267,0.073417,0.397163}%
\pgfsetfillcolor{currentfill}%
\pgfsetfillopacity{0.700000}%
\pgfsetlinewidth{0.000000pt}%
\definecolor{currentstroke}{rgb}{0.000000,0.000000,0.000000}%
\pgfsetstrokecolor{currentstroke}%
\pgfsetdash{}{0pt}%
\pgfpathmoveto{\pgfqpoint{4.983006in}{1.721317in}}%
\pgfpathlineto{\pgfqpoint{4.996701in}{1.719318in}}%
\pgfpathlineto{\pgfqpoint{5.010403in}{1.717341in}}%
\pgfpathlineto{\pgfqpoint{5.024113in}{1.715388in}}%
\pgfpathlineto{\pgfqpoint{5.037831in}{1.713459in}}%
\pgfpathlineto{\pgfqpoint{5.030373in}{1.702955in}}%
\pgfpathlineto{\pgfqpoint{5.022910in}{1.692469in}}%
\pgfpathlineto{\pgfqpoint{5.015443in}{1.682004in}}%
\pgfpathlineto{\pgfqpoint{5.007972in}{1.671566in}}%
\pgfpathlineto{\pgfqpoint{4.994248in}{1.673682in}}%
\pgfpathlineto{\pgfqpoint{4.980531in}{1.675820in}}%
\pgfpathlineto{\pgfqpoint{4.966822in}{1.677982in}}%
\pgfpathlineto{\pgfqpoint{4.953120in}{1.680167in}}%
\pgfpathlineto{\pgfqpoint{4.960598in}{1.690415in}}%
\pgfpathlineto{\pgfqpoint{4.968072in}{1.700692in}}%
\pgfpathlineto{\pgfqpoint{4.975541in}{1.710995in}}%
\pgfpathlineto{\pgfqpoint{4.983006in}{1.721317in}}%
\pgfpathclose%
\pgfusepath{fill}%
\end{pgfscope}%
\begin{pgfscope}%
\pgfpathrectangle{\pgfqpoint{1.254980in}{0.150000in}}{\pgfqpoint{5.490039in}{5.490039in}}%
\pgfusepath{clip}%
\pgfsetbuttcap%
\pgfsetroundjoin%
\definecolor{currentfill}{rgb}{0.239346,0.300855,0.540844}%
\pgfsetfillcolor{currentfill}%
\pgfsetfillopacity{0.700000}%
\pgfsetlinewidth{0.000000pt}%
\definecolor{currentstroke}{rgb}{0.000000,0.000000,0.000000}%
\pgfsetstrokecolor{currentstroke}%
\pgfsetdash{}{0pt}%
\pgfpathmoveto{\pgfqpoint{2.966409in}{2.161868in}}%
\pgfpathlineto{\pgfqpoint{2.979697in}{2.153323in}}%
\pgfpathlineto{\pgfqpoint{2.992988in}{2.144810in}}%
\pgfpathlineto{\pgfqpoint{3.006283in}{2.136331in}}%
\pgfpathlineto{\pgfqpoint{3.019580in}{2.127884in}}%
\pgfpathlineto{\pgfqpoint{3.011065in}{2.136737in}}%
\pgfpathlineto{\pgfqpoint{3.002528in}{2.146071in}}%
\pgfpathlineto{\pgfqpoint{2.993968in}{2.155897in}}%
\pgfpathlineto{\pgfqpoint{2.985386in}{2.166226in}}%
\pgfpathlineto{\pgfqpoint{2.972048in}{2.175035in}}%
\pgfpathlineto{\pgfqpoint{2.958713in}{2.183877in}}%
\pgfpathlineto{\pgfqpoint{2.945380in}{2.192753in}}%
\pgfpathlineto{\pgfqpoint{2.932050in}{2.201661in}}%
\pgfpathlineto{\pgfqpoint{2.940675in}{2.190964in}}%
\pgfpathlineto{\pgfqpoint{2.949276in}{2.180773in}}%
\pgfpathlineto{\pgfqpoint{2.957854in}{2.171078in}}%
\pgfpathlineto{\pgfqpoint{2.966409in}{2.161868in}}%
\pgfpathclose%
\pgfusepath{fill}%
\end{pgfscope}%
\begin{pgfscope}%
\pgfpathrectangle{\pgfqpoint{1.254980in}{0.150000in}}{\pgfqpoint{5.490039in}{5.490039in}}%
\pgfusepath{clip}%
\pgfsetbuttcap%
\pgfsetroundjoin%
\definecolor{currentfill}{rgb}{0.269944,0.014625,0.341379}%
\pgfsetfillcolor{currentfill}%
\pgfsetfillopacity{0.700000}%
\pgfsetlinewidth{0.000000pt}%
\definecolor{currentstroke}{rgb}{0.000000,0.000000,0.000000}%
\pgfsetstrokecolor{currentstroke}%
\pgfsetdash{}{0pt}%
\pgfpathmoveto{\pgfqpoint{4.451284in}{1.619235in}}%
\pgfpathlineto{\pgfqpoint{4.464822in}{1.615527in}}%
\pgfpathlineto{\pgfqpoint{4.478367in}{1.611843in}}%
\pgfpathlineto{\pgfqpoint{4.491918in}{1.608182in}}%
\pgfpathlineto{\pgfqpoint{4.505476in}{1.604544in}}%
\pgfpathlineto{\pgfqpoint{4.497870in}{1.597041in}}%
\pgfpathlineto{\pgfqpoint{4.490259in}{1.589684in}}%
\pgfpathlineto{\pgfqpoint{4.482644in}{1.582478in}}%
\pgfpathlineto{\pgfqpoint{4.475024in}{1.575430in}}%
\pgfpathlineto{\pgfqpoint{4.461454in}{1.579316in}}%
\pgfpathlineto{\pgfqpoint{4.447891in}{1.583225in}}%
\pgfpathlineto{\pgfqpoint{4.434333in}{1.587158in}}%
\pgfpathlineto{\pgfqpoint{4.420783in}{1.591115in}}%
\pgfpathlineto{\pgfqpoint{4.428415in}{1.597910in}}%
\pgfpathlineto{\pgfqpoint{4.436043in}{1.604865in}}%
\pgfpathlineto{\pgfqpoint{4.443666in}{1.611976in}}%
\pgfpathlineto{\pgfqpoint{4.451284in}{1.619235in}}%
\pgfpathclose%
\pgfusepath{fill}%
\end{pgfscope}%
\begin{pgfscope}%
\pgfpathrectangle{\pgfqpoint{1.254980in}{0.150000in}}{\pgfqpoint{5.490039in}{5.490039in}}%
\pgfusepath{clip}%
\pgfsetbuttcap%
\pgfsetroundjoin%
\definecolor{currentfill}{rgb}{0.188923,0.410910,0.556326}%
\pgfsetfillcolor{currentfill}%
\pgfsetfillopacity{0.700000}%
\pgfsetlinewidth{0.000000pt}%
\definecolor{currentstroke}{rgb}{0.000000,0.000000,0.000000}%
\pgfsetstrokecolor{currentstroke}%
\pgfsetdash{}{0pt}%
\pgfpathmoveto{\pgfqpoint{2.612854in}{2.426112in}}%
\pgfpathlineto{\pgfqpoint{2.626130in}{2.416326in}}%
\pgfpathlineto{\pgfqpoint{2.639407in}{2.406580in}}%
\pgfpathlineto{\pgfqpoint{2.652686in}{2.396874in}}%
\pgfpathlineto{\pgfqpoint{2.665968in}{2.387207in}}%
\pgfpathlineto{\pgfqpoint{2.657091in}{2.400295in}}%
\pgfpathlineto{\pgfqpoint{2.648185in}{2.413932in}}%
\pgfpathlineto{\pgfqpoint{2.639250in}{2.428129in}}%
\pgfpathlineto{\pgfqpoint{2.630284in}{2.442896in}}%
\pgfpathlineto{\pgfqpoint{2.616954in}{2.452946in}}%
\pgfpathlineto{\pgfqpoint{2.603626in}{2.463036in}}%
\pgfpathlineto{\pgfqpoint{2.590299in}{2.473166in}}%
\pgfpathlineto{\pgfqpoint{2.576974in}{2.483336in}}%
\pgfpathlineto{\pgfqpoint{2.585990in}{2.468179in}}%
\pgfpathlineto{\pgfqpoint{2.594974in}{2.453597in}}%
\pgfpathlineto{\pgfqpoint{2.603929in}{2.439578in}}%
\pgfpathlineto{\pgfqpoint{2.612854in}{2.426112in}}%
\pgfpathclose%
\pgfusepath{fill}%
\end{pgfscope}%
\begin{pgfscope}%
\pgfpathrectangle{\pgfqpoint{1.254980in}{0.150000in}}{\pgfqpoint{5.490039in}{5.490039in}}%
\pgfusepath{clip}%
\pgfsetbuttcap%
\pgfsetroundjoin%
\definecolor{currentfill}{rgb}{0.282656,0.100196,0.422160}%
\pgfsetfillcolor{currentfill}%
\pgfsetfillopacity{0.700000}%
\pgfsetlinewidth{0.000000pt}%
\definecolor{currentstroke}{rgb}{0.000000,0.000000,0.000000}%
\pgfsetstrokecolor{currentstroke}%
\pgfsetdash{}{0pt}%
\pgfpathmoveto{\pgfqpoint{3.704464in}{1.757292in}}%
\pgfpathlineto{\pgfqpoint{3.717841in}{1.751133in}}%
\pgfpathlineto{\pgfqpoint{3.731223in}{1.744999in}}%
\pgfpathlineto{\pgfqpoint{3.744610in}{1.738892in}}%
\pgfpathlineto{\pgfqpoint{3.758002in}{1.732810in}}%
\pgfpathlineto{\pgfqpoint{3.750060in}{1.732913in}}%
\pgfpathlineto{\pgfqpoint{3.742107in}{1.733343in}}%
\pgfpathlineto{\pgfqpoint{3.734143in}{1.734107in}}%
\pgfpathlineto{\pgfqpoint{3.726168in}{1.735215in}}%
\pgfpathlineto{\pgfqpoint{3.712751in}{1.741611in}}%
\pgfpathlineto{\pgfqpoint{3.699338in}{1.748033in}}%
\pgfpathlineto{\pgfqpoint{3.685930in}{1.754481in}}%
\pgfpathlineto{\pgfqpoint{3.672527in}{1.760954in}}%
\pgfpathlineto{\pgfqpoint{3.680528in}{1.759527in}}%
\pgfpathlineto{\pgfqpoint{3.688518in}{1.758446in}}%
\pgfpathlineto{\pgfqpoint{3.696496in}{1.757704in}}%
\pgfpathlineto{\pgfqpoint{3.704464in}{1.757292in}}%
\pgfpathclose%
\pgfusepath{fill}%
\end{pgfscope}%
\begin{pgfscope}%
\pgfpathrectangle{\pgfqpoint{1.254980in}{0.150000in}}{\pgfqpoint{5.490039in}{5.490039in}}%
\pgfusepath{clip}%
\pgfsetbuttcap%
\pgfsetroundjoin%
\definecolor{currentfill}{rgb}{0.280255,0.165693,0.476498}%
\pgfsetfillcolor{currentfill}%
\pgfsetfillopacity{0.700000}%
\pgfsetlinewidth{0.000000pt}%
\definecolor{currentstroke}{rgb}{0.000000,0.000000,0.000000}%
\pgfsetstrokecolor{currentstroke}%
\pgfsetdash{}{0pt}%
\pgfpathmoveto{\pgfqpoint{5.461385in}{1.902167in}}%
\pgfpathlineto{\pgfqpoint{5.475248in}{1.901464in}}%
\pgfpathlineto{\pgfqpoint{5.489119in}{1.900786in}}%
\pgfpathlineto{\pgfqpoint{5.502999in}{1.900131in}}%
\pgfpathlineto{\pgfqpoint{5.495685in}{1.889406in}}%
\pgfpathlineto{\pgfqpoint{5.488364in}{1.878609in}}%
\pgfpathlineto{\pgfqpoint{5.481036in}{1.867743in}}%
\pgfpathlineto{\pgfqpoint{5.473703in}{1.856811in}}%
\pgfpathlineto{\pgfqpoint{5.459816in}{1.857587in}}%
\pgfpathlineto{\pgfqpoint{5.445939in}{1.858387in}}%
\pgfpathlineto{\pgfqpoint{5.432070in}{1.859210in}}%
\pgfpathlineto{\pgfqpoint{5.439408in}{1.870047in}}%
\pgfpathlineto{\pgfqpoint{5.446740in}{1.880821in}}%
\pgfpathlineto{\pgfqpoint{5.454066in}{1.891528in}}%
\pgfpathlineto{\pgfqpoint{5.461385in}{1.902167in}}%
\pgfpathclose%
\pgfusepath{fill}%
\end{pgfscope}%
\begin{pgfscope}%
\pgfpathrectangle{\pgfqpoint{1.254980in}{0.150000in}}{\pgfqpoint{5.490039in}{5.490039in}}%
\pgfusepath{clip}%
\pgfsetbuttcap%
\pgfsetroundjoin%
\definecolor{currentfill}{rgb}{0.271828,0.209303,0.504434}%
\pgfsetfillcolor{currentfill}%
\pgfsetfillopacity{0.700000}%
\pgfsetlinewidth{0.000000pt}%
\definecolor{currentstroke}{rgb}{0.000000,0.000000,0.000000}%
\pgfsetstrokecolor{currentstroke}%
\pgfsetdash{}{0pt}%
\pgfpathmoveto{\pgfqpoint{3.266023in}{1.971763in}}%
\pgfpathlineto{\pgfqpoint{3.279340in}{1.964188in}}%
\pgfpathlineto{\pgfqpoint{3.292661in}{1.956642in}}%
\pgfpathlineto{\pgfqpoint{3.305986in}{1.949126in}}%
\pgfpathlineto{\pgfqpoint{3.319314in}{1.941638in}}%
\pgfpathlineto{\pgfqpoint{3.311060in}{1.946964in}}%
\pgfpathlineto{\pgfqpoint{3.302789in}{1.952713in}}%
\pgfpathlineto{\pgfqpoint{3.294500in}{1.958895in}}%
\pgfpathlineto{\pgfqpoint{3.286194in}{1.965518in}}%
\pgfpathlineto{\pgfqpoint{3.272831in}{1.973351in}}%
\pgfpathlineto{\pgfqpoint{3.259471in}{1.981212in}}%
\pgfpathlineto{\pgfqpoint{3.246115in}{1.989103in}}%
\pgfpathlineto{\pgfqpoint{3.232763in}{1.997023in}}%
\pgfpathlineto{\pgfqpoint{3.241105in}{1.990048in}}%
\pgfpathlineto{\pgfqpoint{3.249429in}{1.983520in}}%
\pgfpathlineto{\pgfqpoint{3.257735in}{1.977428in}}%
\pgfpathlineto{\pgfqpoint{3.266023in}{1.971763in}}%
\pgfpathclose%
\pgfusepath{fill}%
\end{pgfscope}%
\begin{pgfscope}%
\pgfpathrectangle{\pgfqpoint{1.254980in}{0.150000in}}{\pgfqpoint{5.490039in}{5.490039in}}%
\pgfusepath{clip}%
\pgfsetbuttcap%
\pgfsetroundjoin%
\definecolor{currentfill}{rgb}{0.277941,0.056324,0.381191}%
\pgfsetfillcolor{currentfill}%
\pgfsetfillopacity{0.700000}%
\pgfsetlinewidth{0.000000pt}%
\definecolor{currentstroke}{rgb}{0.000000,0.000000,0.000000}%
\pgfsetstrokecolor{currentstroke}%
\pgfsetdash{}{0pt}%
\pgfpathmoveto{\pgfqpoint{4.898392in}{1.689141in}}%
\pgfpathlineto{\pgfqpoint{4.912062in}{1.686863in}}%
\pgfpathlineto{\pgfqpoint{4.925741in}{1.684608in}}%
\pgfpathlineto{\pgfqpoint{4.939427in}{1.682376in}}%
\pgfpathlineto{\pgfqpoint{4.953120in}{1.680167in}}%
\pgfpathlineto{\pgfqpoint{4.945638in}{1.669954in}}%
\pgfpathlineto{\pgfqpoint{4.938151in}{1.659779in}}%
\pgfpathlineto{\pgfqpoint{4.930660in}{1.649648in}}%
\pgfpathlineto{\pgfqpoint{4.923166in}{1.639565in}}%
\pgfpathlineto{\pgfqpoint{4.909465in}{1.641972in}}%
\pgfpathlineto{\pgfqpoint{4.895771in}{1.644402in}}%
\pgfpathlineto{\pgfqpoint{4.882086in}{1.646856in}}%
\pgfpathlineto{\pgfqpoint{4.868407in}{1.649332in}}%
\pgfpathlineto{\pgfqpoint{4.875910in}{1.659212in}}%
\pgfpathlineto{\pgfqpoint{4.883408in}{1.669143in}}%
\pgfpathlineto{\pgfqpoint{4.890902in}{1.679121in}}%
\pgfpathlineto{\pgfqpoint{4.898392in}{1.689141in}}%
\pgfpathclose%
\pgfusepath{fill}%
\end{pgfscope}%
\begin{pgfscope}%
\pgfpathrectangle{\pgfqpoint{1.254980in}{0.150000in}}{\pgfqpoint{5.490039in}{5.490039in}}%
\pgfusepath{clip}%
\pgfsetbuttcap%
\pgfsetroundjoin%
\definecolor{currentfill}{rgb}{0.282290,0.145912,0.461510}%
\pgfsetfillcolor{currentfill}%
\pgfsetfillopacity{0.700000}%
\pgfsetlinewidth{0.000000pt}%
\definecolor{currentstroke}{rgb}{0.000000,0.000000,0.000000}%
\pgfsetstrokecolor{currentstroke}%
\pgfsetdash{}{0pt}%
\pgfpathmoveto{\pgfqpoint{3.512038in}{1.840700in}}%
\pgfpathlineto{\pgfqpoint{3.525388in}{1.833907in}}%
\pgfpathlineto{\pgfqpoint{3.538742in}{1.827141in}}%
\pgfpathlineto{\pgfqpoint{3.552100in}{1.820402in}}%
\pgfpathlineto{\pgfqpoint{3.565463in}{1.813691in}}%
\pgfpathlineto{\pgfqpoint{3.557393in}{1.816121in}}%
\pgfpathlineto{\pgfqpoint{3.549310in}{1.818923in}}%
\pgfpathlineto{\pgfqpoint{3.541214in}{1.822106in}}%
\pgfpathlineto{\pgfqpoint{3.533103in}{1.825678in}}%
\pgfpathlineto{\pgfqpoint{3.519711in}{1.832719in}}%
\pgfpathlineto{\pgfqpoint{3.506323in}{1.839787in}}%
\pgfpathlineto{\pgfqpoint{3.492938in}{1.846882in}}%
\pgfpathlineto{\pgfqpoint{3.479559in}{1.854004in}}%
\pgfpathlineto{\pgfqpoint{3.487700in}{1.850097in}}%
\pgfpathlineto{\pgfqpoint{3.495827in}{1.846584in}}%
\pgfpathlineto{\pgfqpoint{3.503939in}{1.843454in}}%
\pgfpathlineto{\pgfqpoint{3.512038in}{1.840700in}}%
\pgfpathclose%
\pgfusepath{fill}%
\end{pgfscope}%
\begin{pgfscope}%
\pgfpathrectangle{\pgfqpoint{1.254980in}{0.150000in}}{\pgfqpoint{5.490039in}{5.490039in}}%
\pgfusepath{clip}%
\pgfsetbuttcap%
\pgfsetroundjoin%
\definecolor{currentfill}{rgb}{0.278791,0.062145,0.386592}%
\pgfsetfillcolor{currentfill}%
\pgfsetfillopacity{0.700000}%
\pgfsetlinewidth{0.000000pt}%
\definecolor{currentstroke}{rgb}{0.000000,0.000000,0.000000}%
\pgfsetstrokecolor{currentstroke}%
\pgfsetdash{}{0pt}%
\pgfpathmoveto{\pgfqpoint{3.896796in}{1.690170in}}%
\pgfpathlineto{\pgfqpoint{3.910210in}{1.684618in}}%
\pgfpathlineto{\pgfqpoint{3.923630in}{1.679090in}}%
\pgfpathlineto{\pgfqpoint{3.937055in}{1.673587in}}%
\pgfpathlineto{\pgfqpoint{3.950485in}{1.668109in}}%
\pgfpathlineto{\pgfqpoint{3.942650in}{1.666101in}}%
\pgfpathlineto{\pgfqpoint{3.934806in}{1.664377in}}%
\pgfpathlineto{\pgfqpoint{3.926954in}{1.662945in}}%
\pgfpathlineto{\pgfqpoint{3.919092in}{1.661812in}}%
\pgfpathlineto{\pgfqpoint{3.905640in}{1.667590in}}%
\pgfpathlineto{\pgfqpoint{3.892193in}{1.673394in}}%
\pgfpathlineto{\pgfqpoint{3.878751in}{1.679222in}}%
\pgfpathlineto{\pgfqpoint{3.865315in}{1.685075in}}%
\pgfpathlineto{\pgfqpoint{3.873199in}{1.685902in}}%
\pgfpathlineto{\pgfqpoint{3.881074in}{1.687033in}}%
\pgfpathlineto{\pgfqpoint{3.888939in}{1.688458in}}%
\pgfpathlineto{\pgfqpoint{3.896796in}{1.690170in}}%
\pgfpathclose%
\pgfusepath{fill}%
\end{pgfscope}%
\begin{pgfscope}%
\pgfpathrectangle{\pgfqpoint{1.254980in}{0.150000in}}{\pgfqpoint{5.490039in}{5.490039in}}%
\pgfusepath{clip}%
\pgfsetbuttcap%
\pgfsetroundjoin%
\definecolor{currentfill}{rgb}{0.281887,0.150881,0.465405}%
\pgfsetfillcolor{currentfill}%
\pgfsetfillopacity{0.700000}%
\pgfsetlinewidth{0.000000pt}%
\definecolor{currentstroke}{rgb}{0.000000,0.000000,0.000000}%
\pgfsetstrokecolor{currentstroke}%
\pgfsetdash{}{0pt}%
\pgfpathmoveto{\pgfqpoint{5.376683in}{1.862739in}}%
\pgfpathlineto{\pgfqpoint{5.390516in}{1.861821in}}%
\pgfpathlineto{\pgfqpoint{5.404359in}{1.860927in}}%
\pgfpathlineto{\pgfqpoint{5.418210in}{1.860057in}}%
\pgfpathlineto{\pgfqpoint{5.432070in}{1.859210in}}%
\pgfpathlineto{\pgfqpoint{5.424726in}{1.848312in}}%
\pgfpathlineto{\pgfqpoint{5.417375in}{1.837357in}}%
\pgfpathlineto{\pgfqpoint{5.410019in}{1.826346in}}%
\pgfpathlineto{\pgfqpoint{5.402656in}{1.815284in}}%
\pgfpathlineto{\pgfqpoint{5.388791in}{1.816265in}}%
\pgfpathlineto{\pgfqpoint{5.374933in}{1.817270in}}%
\pgfpathlineto{\pgfqpoint{5.361085in}{1.818298in}}%
\pgfpathlineto{\pgfqpoint{5.347245in}{1.819350in}}%
\pgfpathlineto{\pgfqpoint{5.354613in}{1.830273in}}%
\pgfpathlineto{\pgfqpoint{5.361975in}{1.841147in}}%
\pgfpathlineto{\pgfqpoint{5.369332in}{1.851970in}}%
\pgfpathlineto{\pgfqpoint{5.376683in}{1.862739in}}%
\pgfpathclose%
\pgfusepath{fill}%
\end{pgfscope}%
\begin{pgfscope}%
\pgfpathrectangle{\pgfqpoint{1.254980in}{0.150000in}}{\pgfqpoint{5.490039in}{5.490039in}}%
\pgfusepath{clip}%
\pgfsetbuttcap%
\pgfsetroundjoin%
\definecolor{currentfill}{rgb}{0.271305,0.019942,0.347269}%
\pgfsetfillcolor{currentfill}%
\pgfsetfillopacity{0.700000}%
\pgfsetlinewidth{0.000000pt}%
\definecolor{currentstroke}{rgb}{0.000000,0.000000,0.000000}%
\pgfsetstrokecolor{currentstroke}%
\pgfsetdash{}{0pt}%
\pgfpathmoveto{\pgfqpoint{4.590107in}{1.622517in}}%
\pgfpathlineto{\pgfqpoint{4.603688in}{1.619232in}}%
\pgfpathlineto{\pgfqpoint{4.617276in}{1.615972in}}%
\pgfpathlineto{\pgfqpoint{4.630870in}{1.612734in}}%
\pgfpathlineto{\pgfqpoint{4.644472in}{1.609520in}}%
\pgfpathlineto{\pgfqpoint{4.636906in}{1.601034in}}%
\pgfpathlineto{\pgfqpoint{4.629335in}{1.592663in}}%
\pgfpathlineto{\pgfqpoint{4.621761in}{1.584413in}}%
\pgfpathlineto{\pgfqpoint{4.614182in}{1.576289in}}%
\pgfpathlineto{\pgfqpoint{4.600570in}{1.579739in}}%
\pgfpathlineto{\pgfqpoint{4.586964in}{1.583212in}}%
\pgfpathlineto{\pgfqpoint{4.573366in}{1.586709in}}%
\pgfpathlineto{\pgfqpoint{4.559775in}{1.590229in}}%
\pgfpathlineto{\pgfqpoint{4.567364in}{1.598112in}}%
\pgfpathlineto{\pgfqpoint{4.574950in}{1.606125in}}%
\pgfpathlineto{\pgfqpoint{4.582531in}{1.614262in}}%
\pgfpathlineto{\pgfqpoint{4.590107in}{1.622517in}}%
\pgfpathclose%
\pgfusepath{fill}%
\end{pgfscope}%
\begin{pgfscope}%
\pgfpathrectangle{\pgfqpoint{1.254980in}{0.150000in}}{\pgfqpoint{5.490039in}{5.490039in}}%
\pgfusepath{clip}%
\pgfsetbuttcap%
\pgfsetroundjoin%
\definecolor{currentfill}{rgb}{0.283072,0.130895,0.449241}%
\pgfsetfillcolor{currentfill}%
\pgfsetfillopacity{0.700000}%
\pgfsetlinewidth{0.000000pt}%
\definecolor{currentstroke}{rgb}{0.000000,0.000000,0.000000}%
\pgfsetstrokecolor{currentstroke}%
\pgfsetdash{}{0pt}%
\pgfpathmoveto{\pgfqpoint{5.291971in}{1.823791in}}%
\pgfpathlineto{\pgfqpoint{5.305777in}{1.822646in}}%
\pgfpathlineto{\pgfqpoint{5.319591in}{1.821524in}}%
\pgfpathlineto{\pgfqpoint{5.333414in}{1.820425in}}%
\pgfpathlineto{\pgfqpoint{5.347245in}{1.819350in}}%
\pgfpathlineto{\pgfqpoint{5.339871in}{1.808382in}}%
\pgfpathlineto{\pgfqpoint{5.332492in}{1.797372in}}%
\pgfpathlineto{\pgfqpoint{5.325107in}{1.786324in}}%
\pgfpathlineto{\pgfqpoint{5.317717in}{1.775240in}}%
\pgfpathlineto{\pgfqpoint{5.303880in}{1.776463in}}%
\pgfpathlineto{\pgfqpoint{5.290051in}{1.777709in}}%
\pgfpathlineto{\pgfqpoint{5.276231in}{1.778978in}}%
\pgfpathlineto{\pgfqpoint{5.262420in}{1.780271in}}%
\pgfpathlineto{\pgfqpoint{5.269816in}{1.791202in}}%
\pgfpathlineto{\pgfqpoint{5.277206in}{1.802102in}}%
\pgfpathlineto{\pgfqpoint{5.284591in}{1.812966in}}%
\pgfpathlineto{\pgfqpoint{5.291971in}{1.823791in}}%
\pgfpathclose%
\pgfusepath{fill}%
\end{pgfscope}%
\begin{pgfscope}%
\pgfpathrectangle{\pgfqpoint{1.254980in}{0.150000in}}{\pgfqpoint{5.490039in}{5.490039in}}%
\pgfusepath{clip}%
\pgfsetbuttcap%
\pgfsetroundjoin%
\definecolor{currentfill}{rgb}{0.276022,0.044167,0.370164}%
\pgfsetfillcolor{currentfill}%
\pgfsetfillopacity{0.700000}%
\pgfsetlinewidth{0.000000pt}%
\definecolor{currentstroke}{rgb}{0.000000,0.000000,0.000000}%
\pgfsetstrokecolor{currentstroke}%
\pgfsetdash{}{0pt}%
\pgfpathmoveto{\pgfqpoint{4.813769in}{1.659472in}}%
\pgfpathlineto{\pgfqpoint{4.827418in}{1.656902in}}%
\pgfpathlineto{\pgfqpoint{4.841073in}{1.654355in}}%
\pgfpathlineto{\pgfqpoint{4.854737in}{1.651832in}}%
\pgfpathlineto{\pgfqpoint{4.868407in}{1.649332in}}%
\pgfpathlineto{\pgfqpoint{4.860901in}{1.639509in}}%
\pgfpathlineto{\pgfqpoint{4.853390in}{1.629748in}}%
\pgfpathlineto{\pgfqpoint{4.845876in}{1.620053in}}%
\pgfpathlineto{\pgfqpoint{4.838357in}{1.610431in}}%
\pgfpathlineto{\pgfqpoint{4.824678in}{1.613142in}}%
\pgfpathlineto{\pgfqpoint{4.811007in}{1.615876in}}%
\pgfpathlineto{\pgfqpoint{4.797343in}{1.618634in}}%
\pgfpathlineto{\pgfqpoint{4.783686in}{1.621414in}}%
\pgfpathlineto{\pgfqpoint{4.791213in}{1.630821in}}%
\pgfpathlineto{\pgfqpoint{4.798736in}{1.640303in}}%
\pgfpathlineto{\pgfqpoint{4.806255in}{1.649855in}}%
\pgfpathlineto{\pgfqpoint{4.813769in}{1.659472in}}%
\pgfpathclose%
\pgfusepath{fill}%
\end{pgfscope}%
\begin{pgfscope}%
\pgfpathrectangle{\pgfqpoint{1.254980in}{0.150000in}}{\pgfqpoint{5.490039in}{5.490039in}}%
\pgfusepath{clip}%
\pgfsetbuttcap%
\pgfsetroundjoin%
\definecolor{currentfill}{rgb}{0.244972,0.287675,0.537260}%
\pgfsetfillcolor{currentfill}%
\pgfsetfillopacity{0.700000}%
\pgfsetlinewidth{0.000000pt}%
\definecolor{currentstroke}{rgb}{0.000000,0.000000,0.000000}%
\pgfsetstrokecolor{currentstroke}%
\pgfsetdash{}{0pt}%
\pgfpathmoveto{\pgfqpoint{3.019580in}{2.127884in}}%
\pgfpathlineto{\pgfqpoint{3.032880in}{2.119470in}}%
\pgfpathlineto{\pgfqpoint{3.046183in}{2.111088in}}%
\pgfpathlineto{\pgfqpoint{3.059489in}{2.102738in}}%
\pgfpathlineto{\pgfqpoint{3.072798in}{2.094420in}}%
\pgfpathlineto{\pgfqpoint{3.064323in}{2.102916in}}%
\pgfpathlineto{\pgfqpoint{3.055826in}{2.111890in}}%
\pgfpathlineto{\pgfqpoint{3.047308in}{2.121352in}}%
\pgfpathlineto{\pgfqpoint{3.038767in}{2.131313in}}%
\pgfpathlineto{\pgfqpoint{3.025418in}{2.139993in}}%
\pgfpathlineto{\pgfqpoint{3.012071in}{2.148705in}}%
\pgfpathlineto{\pgfqpoint{2.998727in}{2.157449in}}%
\pgfpathlineto{\pgfqpoint{2.985386in}{2.166226in}}%
\pgfpathlineto{\pgfqpoint{2.993968in}{2.155897in}}%
\pgfpathlineto{\pgfqpoint{3.002528in}{2.146071in}}%
\pgfpathlineto{\pgfqpoint{3.011065in}{2.136737in}}%
\pgfpathlineto{\pgfqpoint{3.019580in}{2.127884in}}%
\pgfpathclose%
\pgfusepath{fill}%
\end{pgfscope}%
\begin{pgfscope}%
\pgfpathrectangle{\pgfqpoint{1.254980in}{0.150000in}}{\pgfqpoint{5.490039in}{5.490039in}}%
\pgfusepath{clip}%
\pgfsetbuttcap%
\pgfsetroundjoin%
\definecolor{currentfill}{rgb}{0.194100,0.399323,0.555565}%
\pgfsetfillcolor{currentfill}%
\pgfsetfillopacity{0.700000}%
\pgfsetlinewidth{0.000000pt}%
\definecolor{currentstroke}{rgb}{0.000000,0.000000,0.000000}%
\pgfsetstrokecolor{currentstroke}%
\pgfsetdash{}{0pt}%
\pgfpathmoveto{\pgfqpoint{2.665968in}{2.387207in}}%
\pgfpathlineto{\pgfqpoint{2.679251in}{2.377579in}}%
\pgfpathlineto{\pgfqpoint{2.692536in}{2.367989in}}%
\pgfpathlineto{\pgfqpoint{2.705823in}{2.358438in}}%
\pgfpathlineto{\pgfqpoint{2.719112in}{2.348926in}}%
\pgfpathlineto{\pgfqpoint{2.710282in}{2.361638in}}%
\pgfpathlineto{\pgfqpoint{2.701424in}{2.374894in}}%
\pgfpathlineto{\pgfqpoint{2.692538in}{2.388706in}}%
\pgfpathlineto{\pgfqpoint{2.683622in}{2.403084in}}%
\pgfpathlineto{\pgfqpoint{2.670285in}{2.412979in}}%
\pgfpathlineto{\pgfqpoint{2.656950in}{2.422913in}}%
\pgfpathlineto{\pgfqpoint{2.643616in}{2.432885in}}%
\pgfpathlineto{\pgfqpoint{2.630284in}{2.442896in}}%
\pgfpathlineto{\pgfqpoint{2.639250in}{2.428129in}}%
\pgfpathlineto{\pgfqpoint{2.648185in}{2.413932in}}%
\pgfpathlineto{\pgfqpoint{2.657091in}{2.400295in}}%
\pgfpathlineto{\pgfqpoint{2.665968in}{2.387207in}}%
\pgfpathclose%
\pgfusepath{fill}%
\end{pgfscope}%
\begin{pgfscope}%
\pgfpathrectangle{\pgfqpoint{1.254980in}{0.150000in}}{\pgfqpoint{5.490039in}{5.490039in}}%
\pgfusepath{clip}%
\pgfsetbuttcap%
\pgfsetroundjoin%
\definecolor{currentfill}{rgb}{0.271305,0.019942,0.347269}%
\pgfsetfillcolor{currentfill}%
\pgfsetfillopacity{0.700000}%
\pgfsetlinewidth{0.000000pt}%
\definecolor{currentstroke}{rgb}{0.000000,0.000000,0.000000}%
\pgfsetstrokecolor{currentstroke}%
\pgfsetdash{}{0pt}%
\pgfpathmoveto{\pgfqpoint{4.227916in}{1.618140in}}%
\pgfpathlineto{\pgfqpoint{4.241406in}{1.613637in}}%
\pgfpathlineto{\pgfqpoint{4.254902in}{1.609158in}}%
\pgfpathlineto{\pgfqpoint{4.268405in}{1.604703in}}%
\pgfpathlineto{\pgfqpoint{4.281913in}{1.600272in}}%
\pgfpathlineto{\pgfqpoint{4.274225in}{1.594921in}}%
\pgfpathlineto{\pgfqpoint{4.266531in}{1.589778in}}%
\pgfpathlineto{\pgfqpoint{4.258832in}{1.584849in}}%
\pgfpathlineto{\pgfqpoint{4.251126in}{1.580142in}}%
\pgfpathlineto{\pgfqpoint{4.237602in}{1.584847in}}%
\pgfpathlineto{\pgfqpoint{4.224084in}{1.589576in}}%
\pgfpathlineto{\pgfqpoint{4.210571in}{1.594329in}}%
\pgfpathlineto{\pgfqpoint{4.197065in}{1.599106in}}%
\pgfpathlineto{\pgfqpoint{4.204787in}{1.603534in}}%
\pgfpathlineto{\pgfqpoint{4.212503in}{1.608187in}}%
\pgfpathlineto{\pgfqpoint{4.220213in}{1.613058in}}%
\pgfpathlineto{\pgfqpoint{4.227916in}{1.618140in}}%
\pgfpathclose%
\pgfusepath{fill}%
\end{pgfscope}%
\begin{pgfscope}%
\pgfpathrectangle{\pgfqpoint{1.254980in}{0.150000in}}{\pgfqpoint{5.490039in}{5.490039in}}%
\pgfusepath{clip}%
\pgfsetbuttcap%
\pgfsetroundjoin%
\definecolor{currentfill}{rgb}{0.283197,0.115680,0.436115}%
\pgfsetfillcolor{currentfill}%
\pgfsetfillopacity{0.700000}%
\pgfsetlinewidth{0.000000pt}%
\definecolor{currentstroke}{rgb}{0.000000,0.000000,0.000000}%
\pgfsetstrokecolor{currentstroke}%
\pgfsetdash{}{0pt}%
\pgfpathmoveto{\pgfqpoint{5.207257in}{1.785676in}}%
\pgfpathlineto{\pgfqpoint{5.221035in}{1.784290in}}%
\pgfpathlineto{\pgfqpoint{5.234821in}{1.782927in}}%
\pgfpathlineto{\pgfqpoint{5.248616in}{1.781587in}}%
\pgfpathlineto{\pgfqpoint{5.262420in}{1.780271in}}%
\pgfpathlineto{\pgfqpoint{5.255018in}{1.769312in}}%
\pgfpathlineto{\pgfqpoint{5.247612in}{1.758328in}}%
\pgfpathlineto{\pgfqpoint{5.240201in}{1.747323in}}%
\pgfpathlineto{\pgfqpoint{5.232785in}{1.736301in}}%
\pgfpathlineto{\pgfqpoint{5.218975in}{1.737777in}}%
\pgfpathlineto{\pgfqpoint{5.205175in}{1.739277in}}%
\pgfpathlineto{\pgfqpoint{5.191382in}{1.740800in}}%
\pgfpathlineto{\pgfqpoint{5.177598in}{1.742347in}}%
\pgfpathlineto{\pgfqpoint{5.185020in}{1.753204in}}%
\pgfpathlineto{\pgfqpoint{5.192437in}{1.764047in}}%
\pgfpathlineto{\pgfqpoint{5.199850in}{1.774872in}}%
\pgfpathlineto{\pgfqpoint{5.207257in}{1.785676in}}%
\pgfpathclose%
\pgfusepath{fill}%
\end{pgfscope}%
\begin{pgfscope}%
\pgfpathrectangle{\pgfqpoint{1.254980in}{0.150000in}}{\pgfqpoint{5.490039in}{5.490039in}}%
\pgfusepath{clip}%
\pgfsetbuttcap%
\pgfsetroundjoin%
\definecolor{currentfill}{rgb}{0.269944,0.014625,0.341379}%
\pgfsetfillcolor{currentfill}%
\pgfsetfillopacity{0.700000}%
\pgfsetlinewidth{0.000000pt}%
\definecolor{currentstroke}{rgb}{0.000000,0.000000,0.000000}%
\pgfsetstrokecolor{currentstroke}%
\pgfsetdash{}{0pt}%
\pgfpathmoveto{\pgfqpoint{4.366644in}{1.607177in}}%
\pgfpathlineto{\pgfqpoint{4.380169in}{1.603126in}}%
\pgfpathlineto{\pgfqpoint{4.393701in}{1.599099in}}%
\pgfpathlineto{\pgfqpoint{4.407239in}{1.595095in}}%
\pgfpathlineto{\pgfqpoint{4.420783in}{1.591115in}}%
\pgfpathlineto{\pgfqpoint{4.413145in}{1.584488in}}%
\pgfpathlineto{\pgfqpoint{4.405502in}{1.578035in}}%
\pgfpathlineto{\pgfqpoint{4.397855in}{1.571763in}}%
\pgfpathlineto{\pgfqpoint{4.390202in}{1.565678in}}%
\pgfpathlineto{\pgfqpoint{4.376644in}{1.569920in}}%
\pgfpathlineto{\pgfqpoint{4.363092in}{1.574185in}}%
\pgfpathlineto{\pgfqpoint{4.349547in}{1.578474in}}%
\pgfpathlineto{\pgfqpoint{4.336008in}{1.582786in}}%
\pgfpathlineto{\pgfqpoint{4.343675in}{1.588605in}}%
\pgfpathlineto{\pgfqpoint{4.351337in}{1.594614in}}%
\pgfpathlineto{\pgfqpoint{4.358993in}{1.600807in}}%
\pgfpathlineto{\pgfqpoint{4.366644in}{1.607177in}}%
\pgfpathclose%
\pgfusepath{fill}%
\end{pgfscope}%
\begin{pgfscope}%
\pgfpathrectangle{\pgfqpoint{1.254980in}{0.150000in}}{\pgfqpoint{5.490039in}{5.490039in}}%
\pgfusepath{clip}%
\pgfsetbuttcap%
\pgfsetroundjoin%
\definecolor{currentfill}{rgb}{0.274952,0.037752,0.364543}%
\pgfsetfillcolor{currentfill}%
\pgfsetfillopacity{0.700000}%
\pgfsetlinewidth{0.000000pt}%
\definecolor{currentstroke}{rgb}{0.000000,0.000000,0.000000}%
\pgfsetstrokecolor{currentstroke}%
\pgfsetdash{}{0pt}%
\pgfpathmoveto{\pgfqpoint{4.089222in}{1.638186in}}%
\pgfpathlineto{\pgfqpoint{4.102682in}{1.633217in}}%
\pgfpathlineto{\pgfqpoint{4.116148in}{1.628271in}}%
\pgfpathlineto{\pgfqpoint{4.129620in}{1.623350in}}%
\pgfpathlineto{\pgfqpoint{4.143097in}{1.618453in}}%
\pgfpathlineto{\pgfqpoint{4.135351in}{1.614541in}}%
\pgfpathlineto{\pgfqpoint{4.127598in}{1.610871in}}%
\pgfpathlineto{\pgfqpoint{4.119838in}{1.607452in}}%
\pgfpathlineto{\pgfqpoint{4.112071in}{1.604291in}}%
\pgfpathlineto{\pgfqpoint{4.098575in}{1.609476in}}%
\pgfpathlineto{\pgfqpoint{4.085085in}{1.614684in}}%
\pgfpathlineto{\pgfqpoint{4.071600in}{1.619917in}}%
\pgfpathlineto{\pgfqpoint{4.058121in}{1.625174in}}%
\pgfpathlineto{\pgfqpoint{4.065907in}{1.628042in}}%
\pgfpathlineto{\pgfqpoint{4.073686in}{1.631172in}}%
\pgfpathlineto{\pgfqpoint{4.081458in}{1.634556in}}%
\pgfpathlineto{\pgfqpoint{4.089222in}{1.638186in}}%
\pgfpathclose%
\pgfusepath{fill}%
\end{pgfscope}%
\begin{pgfscope}%
\pgfpathrectangle{\pgfqpoint{1.254980in}{0.150000in}}{\pgfqpoint{5.490039in}{5.490039in}}%
\pgfusepath{clip}%
\pgfsetbuttcap%
\pgfsetroundjoin%
\definecolor{currentfill}{rgb}{0.274128,0.199721,0.498911}%
\pgfsetfillcolor{currentfill}%
\pgfsetfillopacity{0.700000}%
\pgfsetlinewidth{0.000000pt}%
\definecolor{currentstroke}{rgb}{0.000000,0.000000,0.000000}%
\pgfsetstrokecolor{currentstroke}%
\pgfsetdash{}{0pt}%
\pgfpathmoveto{\pgfqpoint{3.319314in}{1.941638in}}%
\pgfpathlineto{\pgfqpoint{3.332646in}{1.934180in}}%
\pgfpathlineto{\pgfqpoint{3.345982in}{1.926750in}}%
\pgfpathlineto{\pgfqpoint{3.359322in}{1.919349in}}%
\pgfpathlineto{\pgfqpoint{3.372666in}{1.911976in}}%
\pgfpathlineto{\pgfqpoint{3.364445in}{1.916963in}}%
\pgfpathlineto{\pgfqpoint{3.356208in}{1.922369in}}%
\pgfpathlineto{\pgfqpoint{3.347954in}{1.928204in}}%
\pgfpathlineto{\pgfqpoint{3.339683in}{1.934478in}}%
\pgfpathlineto{\pgfqpoint{3.326305in}{1.942195in}}%
\pgfpathlineto{\pgfqpoint{3.312931in}{1.949941in}}%
\pgfpathlineto{\pgfqpoint{3.299561in}{1.957715in}}%
\pgfpathlineto{\pgfqpoint{3.286194in}{1.965518in}}%
\pgfpathlineto{\pgfqpoint{3.294500in}{1.958895in}}%
\pgfpathlineto{\pgfqpoint{3.302789in}{1.952713in}}%
\pgfpathlineto{\pgfqpoint{3.311060in}{1.946964in}}%
\pgfpathlineto{\pgfqpoint{3.319314in}{1.941638in}}%
\pgfpathclose%
\pgfusepath{fill}%
\end{pgfscope}%
\begin{pgfscope}%
\pgfpathrectangle{\pgfqpoint{1.254980in}{0.150000in}}{\pgfqpoint{5.490039in}{5.490039in}}%
\pgfusepath{clip}%
\pgfsetbuttcap%
\pgfsetroundjoin%
\definecolor{currentfill}{rgb}{0.282327,0.094955,0.417331}%
\pgfsetfillcolor{currentfill}%
\pgfsetfillopacity{0.700000}%
\pgfsetlinewidth{0.000000pt}%
\definecolor{currentstroke}{rgb}{0.000000,0.000000,0.000000}%
\pgfsetstrokecolor{currentstroke}%
\pgfsetdash{}{0pt}%
\pgfpathmoveto{\pgfqpoint{5.122543in}{1.748767in}}%
\pgfpathlineto{\pgfqpoint{5.136294in}{1.747127in}}%
\pgfpathlineto{\pgfqpoint{5.150054in}{1.745510in}}%
\pgfpathlineto{\pgfqpoint{5.163822in}{1.743917in}}%
\pgfpathlineto{\pgfqpoint{5.177598in}{1.742347in}}%
\pgfpathlineto{\pgfqpoint{5.170171in}{1.731480in}}%
\pgfpathlineto{\pgfqpoint{5.162739in}{1.720608in}}%
\pgfpathlineto{\pgfqpoint{5.155303in}{1.709733in}}%
\pgfpathlineto{\pgfqpoint{5.147862in}{1.698861in}}%
\pgfpathlineto{\pgfqpoint{5.134080in}{1.700604in}}%
\pgfpathlineto{\pgfqpoint{5.120306in}{1.702370in}}%
\pgfpathlineto{\pgfqpoint{5.106540in}{1.704160in}}%
\pgfpathlineto{\pgfqpoint{5.092782in}{1.705973in}}%
\pgfpathlineto{\pgfqpoint{5.100229in}{1.716667in}}%
\pgfpathlineto{\pgfqpoint{5.107672in}{1.727367in}}%
\pgfpathlineto{\pgfqpoint{5.115110in}{1.738069in}}%
\pgfpathlineto{\pgfqpoint{5.122543in}{1.748767in}}%
\pgfpathclose%
\pgfusepath{fill}%
\end{pgfscope}%
\begin{pgfscope}%
\pgfpathrectangle{\pgfqpoint{1.254980in}{0.150000in}}{\pgfqpoint{5.490039in}{5.490039in}}%
\pgfusepath{clip}%
\pgfsetbuttcap%
\pgfsetroundjoin%
\definecolor{currentfill}{rgb}{0.282327,0.094955,0.417331}%
\pgfsetfillcolor{currentfill}%
\pgfsetfillopacity{0.700000}%
\pgfsetlinewidth{0.000000pt}%
\definecolor{currentstroke}{rgb}{0.000000,0.000000,0.000000}%
\pgfsetstrokecolor{currentstroke}%
\pgfsetdash{}{0pt}%
\pgfpathmoveto{\pgfqpoint{3.758002in}{1.732810in}}%
\pgfpathlineto{\pgfqpoint{3.771399in}{1.726754in}}%
\pgfpathlineto{\pgfqpoint{3.784800in}{1.720724in}}%
\pgfpathlineto{\pgfqpoint{3.798207in}{1.714719in}}%
\pgfpathlineto{\pgfqpoint{3.811618in}{1.708740in}}%
\pgfpathlineto{\pgfqpoint{3.803701in}{1.708533in}}%
\pgfpathlineto{\pgfqpoint{3.795773in}{1.708650in}}%
\pgfpathlineto{\pgfqpoint{3.787834in}{1.709098in}}%
\pgfpathlineto{\pgfqpoint{3.779885in}{1.709886in}}%
\pgfpathlineto{\pgfqpoint{3.766449in}{1.716180in}}%
\pgfpathlineto{\pgfqpoint{3.753017in}{1.722499in}}%
\pgfpathlineto{\pgfqpoint{3.739590in}{1.728844in}}%
\pgfpathlineto{\pgfqpoint{3.726168in}{1.735215in}}%
\pgfpathlineto{\pgfqpoint{3.734143in}{1.734107in}}%
\pgfpathlineto{\pgfqpoint{3.742107in}{1.733343in}}%
\pgfpathlineto{\pgfqpoint{3.750060in}{1.732913in}}%
\pgfpathlineto{\pgfqpoint{3.758002in}{1.732810in}}%
\pgfpathclose%
\pgfusepath{fill}%
\end{pgfscope}%
\begin{pgfscope}%
\pgfpathrectangle{\pgfqpoint{1.254980in}{0.150000in}}{\pgfqpoint{5.490039in}{5.490039in}}%
\pgfusepath{clip}%
\pgfsetbuttcap%
\pgfsetroundjoin%
\definecolor{currentfill}{rgb}{0.132444,0.552216,0.553018}%
\pgfsetfillcolor{currentfill}%
\pgfsetfillopacity{0.700000}%
\pgfsetlinewidth{0.000000pt}%
\definecolor{currentstroke}{rgb}{0.000000,0.000000,0.000000}%
\pgfsetstrokecolor{currentstroke}%
\pgfsetdash{}{0pt}%
\pgfpathmoveto{\pgfqpoint{2.204325in}{2.786058in}}%
\pgfpathlineto{\pgfqpoint{2.217626in}{2.774592in}}%
\pgfpathlineto{\pgfqpoint{2.230926in}{2.763179in}}%
\pgfpathlineto{\pgfqpoint{2.244227in}{2.751819in}}%
\pgfpathlineto{\pgfqpoint{2.257528in}{2.740510in}}%
\pgfpathlineto{\pgfqpoint{2.248154in}{2.758644in}}%
\pgfpathlineto{\pgfqpoint{2.238744in}{2.777403in}}%
\pgfpathlineto{\pgfqpoint{2.229294in}{2.796800in}}%
\pgfpathlineto{\pgfqpoint{2.219805in}{2.816846in}}%
\pgfpathlineto{\pgfqpoint{2.206446in}{2.828564in}}%
\pgfpathlineto{\pgfqpoint{2.193087in}{2.840334in}}%
\pgfpathlineto{\pgfqpoint{2.179727in}{2.852157in}}%
\pgfpathlineto{\pgfqpoint{2.166368in}{2.864034in}}%
\pgfpathlineto{\pgfqpoint{2.175917in}{2.843570in}}%
\pgfpathlineto{\pgfqpoint{2.185426in}{2.823762in}}%
\pgfpathlineto{\pgfqpoint{2.194895in}{2.804595in}}%
\pgfpathlineto{\pgfqpoint{2.204325in}{2.786058in}}%
\pgfpathclose%
\pgfusepath{fill}%
\end{pgfscope}%
\begin{pgfscope}%
\pgfpathrectangle{\pgfqpoint{1.254980in}{0.150000in}}{\pgfqpoint{5.490039in}{5.490039in}}%
\pgfusepath{clip}%
\pgfsetbuttcap%
\pgfsetroundjoin%
\definecolor{currentfill}{rgb}{0.273809,0.031497,0.358853}%
\pgfsetfillcolor{currentfill}%
\pgfsetfillopacity{0.700000}%
\pgfsetlinewidth{0.000000pt}%
\definecolor{currentstroke}{rgb}{0.000000,0.000000,0.000000}%
\pgfsetstrokecolor{currentstroke}%
\pgfsetdash{}{0pt}%
\pgfpathmoveto{\pgfqpoint{4.729133in}{1.632770in}}%
\pgfpathlineto{\pgfqpoint{4.742760in}{1.629896in}}%
\pgfpathlineto{\pgfqpoint{4.756395in}{1.627046in}}%
\pgfpathlineto{\pgfqpoint{4.770037in}{1.624218in}}%
\pgfpathlineto{\pgfqpoint{4.783686in}{1.621414in}}%
\pgfpathlineto{\pgfqpoint{4.776155in}{1.612088in}}%
\pgfpathlineto{\pgfqpoint{4.768620in}{1.602848in}}%
\pgfpathlineto{\pgfqpoint{4.761081in}{1.593699in}}%
\pgfpathlineto{\pgfqpoint{4.753538in}{1.584647in}}%
\pgfpathlineto{\pgfqpoint{4.739880in}{1.587675in}}%
\pgfpathlineto{\pgfqpoint{4.726229in}{1.590726in}}%
\pgfpathlineto{\pgfqpoint{4.712585in}{1.593800in}}%
\pgfpathlineto{\pgfqpoint{4.698948in}{1.596897in}}%
\pgfpathlineto{\pgfqpoint{4.706501in}{1.605721in}}%
\pgfpathlineto{\pgfqpoint{4.714049in}{1.614644in}}%
\pgfpathlineto{\pgfqpoint{4.721593in}{1.623663in}}%
\pgfpathlineto{\pgfqpoint{4.729133in}{1.632770in}}%
\pgfpathclose%
\pgfusepath{fill}%
\end{pgfscope}%
\begin{pgfscope}%
\pgfpathrectangle{\pgfqpoint{1.254980in}{0.150000in}}{\pgfqpoint{5.490039in}{5.490039in}}%
\pgfusepath{clip}%
\pgfsetbuttcap%
\pgfsetroundjoin%
\definecolor{currentfill}{rgb}{0.282623,0.140926,0.457517}%
\pgfsetfillcolor{currentfill}%
\pgfsetfillopacity{0.700000}%
\pgfsetlinewidth{0.000000pt}%
\definecolor{currentstroke}{rgb}{0.000000,0.000000,0.000000}%
\pgfsetstrokecolor{currentstroke}%
\pgfsetdash{}{0pt}%
\pgfpathmoveto{\pgfqpoint{3.565463in}{1.813691in}}%
\pgfpathlineto{\pgfqpoint{3.578830in}{1.807006in}}%
\pgfpathlineto{\pgfqpoint{3.592202in}{1.800348in}}%
\pgfpathlineto{\pgfqpoint{3.605578in}{1.793716in}}%
\pgfpathlineto{\pgfqpoint{3.618959in}{1.787111in}}%
\pgfpathlineto{\pgfqpoint{3.610917in}{1.789218in}}%
\pgfpathlineto{\pgfqpoint{3.602863in}{1.791693in}}%
\pgfpathlineto{\pgfqpoint{3.594796in}{1.794544in}}%
\pgfpathlineto{\pgfqpoint{3.586716in}{1.797782in}}%
\pgfpathlineto{\pgfqpoint{3.573306in}{1.804716in}}%
\pgfpathlineto{\pgfqpoint{3.559901in}{1.811677in}}%
\pgfpathlineto{\pgfqpoint{3.546500in}{1.818664in}}%
\pgfpathlineto{\pgfqpoint{3.533103in}{1.825678in}}%
\pgfpathlineto{\pgfqpoint{3.541214in}{1.822106in}}%
\pgfpathlineto{\pgfqpoint{3.549310in}{1.818923in}}%
\pgfpathlineto{\pgfqpoint{3.557393in}{1.816121in}}%
\pgfpathlineto{\pgfqpoint{3.565463in}{1.813691in}}%
\pgfpathclose%
\pgfusepath{fill}%
\end{pgfscope}%
\begin{pgfscope}%
\pgfpathrectangle{\pgfqpoint{1.254980in}{0.150000in}}{\pgfqpoint{5.490039in}{5.490039in}}%
\pgfusepath{clip}%
\pgfsetbuttcap%
\pgfsetroundjoin%
\definecolor{currentfill}{rgb}{0.269944,0.014625,0.341379}%
\pgfsetfillcolor{currentfill}%
\pgfsetfillopacity{0.700000}%
\pgfsetlinewidth{0.000000pt}%
\definecolor{currentstroke}{rgb}{0.000000,0.000000,0.000000}%
\pgfsetstrokecolor{currentstroke}%
\pgfsetdash{}{0pt}%
\pgfpathmoveto{\pgfqpoint{4.505476in}{1.604544in}}%
\pgfpathlineto{\pgfqpoint{4.519040in}{1.600930in}}%
\pgfpathlineto{\pgfqpoint{4.532612in}{1.597340in}}%
\pgfpathlineto{\pgfqpoint{4.546190in}{1.593773in}}%
\pgfpathlineto{\pgfqpoint{4.559775in}{1.590229in}}%
\pgfpathlineto{\pgfqpoint{4.552180in}{1.582483in}}%
\pgfpathlineto{\pgfqpoint{4.544582in}{1.574878in}}%
\pgfpathlineto{\pgfqpoint{4.536978in}{1.567422in}}%
\pgfpathlineto{\pgfqpoint{4.529371in}{1.560120in}}%
\pgfpathlineto{\pgfqpoint{4.515774in}{1.563912in}}%
\pgfpathlineto{\pgfqpoint{4.502184in}{1.567728in}}%
\pgfpathlineto{\pgfqpoint{4.488601in}{1.571567in}}%
\pgfpathlineto{\pgfqpoint{4.475024in}{1.575430in}}%
\pgfpathlineto{\pgfqpoint{4.482644in}{1.582478in}}%
\pgfpathlineto{\pgfqpoint{4.490259in}{1.589684in}}%
\pgfpathlineto{\pgfqpoint{4.497870in}{1.597041in}}%
\pgfpathlineto{\pgfqpoint{4.505476in}{1.604544in}}%
\pgfpathclose%
\pgfusepath{fill}%
\end{pgfscope}%
\begin{pgfscope}%
\pgfpathrectangle{\pgfqpoint{1.254980in}{0.150000in}}{\pgfqpoint{5.490039in}{5.490039in}}%
\pgfusepath{clip}%
\pgfsetbuttcap%
\pgfsetroundjoin%
\definecolor{currentfill}{rgb}{0.199430,0.387607,0.554642}%
\pgfsetfillcolor{currentfill}%
\pgfsetfillopacity{0.700000}%
\pgfsetlinewidth{0.000000pt}%
\definecolor{currentstroke}{rgb}{0.000000,0.000000,0.000000}%
\pgfsetstrokecolor{currentstroke}%
\pgfsetdash{}{0pt}%
\pgfpathmoveto{\pgfqpoint{2.719112in}{2.348926in}}%
\pgfpathlineto{\pgfqpoint{2.732404in}{2.339450in}}%
\pgfpathlineto{\pgfqpoint{2.745697in}{2.330013in}}%
\pgfpathlineto{\pgfqpoint{2.758993in}{2.320612in}}%
\pgfpathlineto{\pgfqpoint{2.772290in}{2.311249in}}%
\pgfpathlineto{\pgfqpoint{2.763507in}{2.323585in}}%
\pgfpathlineto{\pgfqpoint{2.754696in}{2.336462in}}%
\pgfpathlineto{\pgfqpoint{2.745858in}{2.349889in}}%
\pgfpathlineto{\pgfqpoint{2.736991in}{2.363879in}}%
\pgfpathlineto{\pgfqpoint{2.723646in}{2.373625in}}%
\pgfpathlineto{\pgfqpoint{2.710303in}{2.383407in}}%
\pgfpathlineto{\pgfqpoint{2.696962in}{2.393227in}}%
\pgfpathlineto{\pgfqpoint{2.683622in}{2.403084in}}%
\pgfpathlineto{\pgfqpoint{2.692538in}{2.388706in}}%
\pgfpathlineto{\pgfqpoint{2.701424in}{2.374894in}}%
\pgfpathlineto{\pgfqpoint{2.710282in}{2.361638in}}%
\pgfpathlineto{\pgfqpoint{2.719112in}{2.348926in}}%
\pgfpathclose%
\pgfusepath{fill}%
\end{pgfscope}%
\begin{pgfscope}%
\pgfpathrectangle{\pgfqpoint{1.254980in}{0.150000in}}{\pgfqpoint{5.490039in}{5.490039in}}%
\pgfusepath{clip}%
\pgfsetbuttcap%
\pgfsetroundjoin%
\definecolor{currentfill}{rgb}{0.280894,0.078907,0.402329}%
\pgfsetfillcolor{currentfill}%
\pgfsetfillopacity{0.700000}%
\pgfsetlinewidth{0.000000pt}%
\definecolor{currentstroke}{rgb}{0.000000,0.000000,0.000000}%
\pgfsetstrokecolor{currentstroke}%
\pgfsetdash{}{0pt}%
\pgfpathmoveto{\pgfqpoint{5.037831in}{1.713459in}}%
\pgfpathlineto{\pgfqpoint{5.051557in}{1.711552in}}%
\pgfpathlineto{\pgfqpoint{5.065291in}{1.709669in}}%
\pgfpathlineto{\pgfqpoint{5.079032in}{1.707810in}}%
\pgfpathlineto{\pgfqpoint{5.092782in}{1.705973in}}%
\pgfpathlineto{\pgfqpoint{5.085331in}{1.695289in}}%
\pgfpathlineto{\pgfqpoint{5.077875in}{1.684618in}}%
\pgfpathlineto{\pgfqpoint{5.070414in}{1.673966in}}%
\pgfpathlineto{\pgfqpoint{5.062950in}{1.663337in}}%
\pgfpathlineto{\pgfqpoint{5.049193in}{1.665360in}}%
\pgfpathlineto{\pgfqpoint{5.035445in}{1.667405in}}%
\pgfpathlineto{\pgfqpoint{5.021705in}{1.669474in}}%
\pgfpathlineto{\pgfqpoint{5.007972in}{1.671566in}}%
\pgfpathlineto{\pgfqpoint{5.015443in}{1.682004in}}%
\pgfpathlineto{\pgfqpoint{5.022910in}{1.692469in}}%
\pgfpathlineto{\pgfqpoint{5.030373in}{1.702955in}}%
\pgfpathlineto{\pgfqpoint{5.037831in}{1.713459in}}%
\pgfpathclose%
\pgfusepath{fill}%
\end{pgfscope}%
\begin{pgfscope}%
\pgfpathrectangle{\pgfqpoint{1.254980in}{0.150000in}}{\pgfqpoint{5.490039in}{5.490039in}}%
\pgfusepath{clip}%
\pgfsetbuttcap%
\pgfsetroundjoin%
\definecolor{currentfill}{rgb}{0.278791,0.062145,0.386592}%
\pgfsetfillcolor{currentfill}%
\pgfsetfillopacity{0.700000}%
\pgfsetlinewidth{0.000000pt}%
\definecolor{currentstroke}{rgb}{0.000000,0.000000,0.000000}%
\pgfsetstrokecolor{currentstroke}%
\pgfsetdash{}{0pt}%
\pgfpathmoveto{\pgfqpoint{3.950485in}{1.668109in}}%
\pgfpathlineto{\pgfqpoint{3.963921in}{1.662656in}}%
\pgfpathlineto{\pgfqpoint{3.977362in}{1.657228in}}%
\pgfpathlineto{\pgfqpoint{3.990808in}{1.651824in}}%
\pgfpathlineto{\pgfqpoint{4.004259in}{1.646445in}}%
\pgfpathlineto{\pgfqpoint{3.996445in}{1.644142in}}%
\pgfpathlineto{\pgfqpoint{3.988623in}{1.642118in}}%
\pgfpathlineto{\pgfqpoint{3.980792in}{1.640383in}}%
\pgfpathlineto{\pgfqpoint{3.972952in}{1.638945in}}%
\pgfpathlineto{\pgfqpoint{3.959479in}{1.644625in}}%
\pgfpathlineto{\pgfqpoint{3.946012in}{1.650329in}}%
\pgfpathlineto{\pgfqpoint{3.932549in}{1.656058in}}%
\pgfpathlineto{\pgfqpoint{3.919092in}{1.661812in}}%
\pgfpathlineto{\pgfqpoint{3.926954in}{1.662945in}}%
\pgfpathlineto{\pgfqpoint{3.934806in}{1.664377in}}%
\pgfpathlineto{\pgfqpoint{3.942650in}{1.666101in}}%
\pgfpathlineto{\pgfqpoint{3.950485in}{1.668109in}}%
\pgfpathclose%
\pgfusepath{fill}%
\end{pgfscope}%
\begin{pgfscope}%
\pgfpathrectangle{\pgfqpoint{1.254980in}{0.150000in}}{\pgfqpoint{5.490039in}{5.490039in}}%
\pgfusepath{clip}%
\pgfsetbuttcap%
\pgfsetroundjoin%
\definecolor{currentfill}{rgb}{0.248629,0.278775,0.534556}%
\pgfsetfillcolor{currentfill}%
\pgfsetfillopacity{0.700000}%
\pgfsetlinewidth{0.000000pt}%
\definecolor{currentstroke}{rgb}{0.000000,0.000000,0.000000}%
\pgfsetstrokecolor{currentstroke}%
\pgfsetdash{}{0pt}%
\pgfpathmoveto{\pgfqpoint{3.072798in}{2.094420in}}%
\pgfpathlineto{\pgfqpoint{3.086111in}{2.086133in}}%
\pgfpathlineto{\pgfqpoint{3.099426in}{2.077879in}}%
\pgfpathlineto{\pgfqpoint{3.112745in}{2.069655in}}%
\pgfpathlineto{\pgfqpoint{3.126067in}{2.061463in}}%
\pgfpathlineto{\pgfqpoint{3.117630in}{2.069603in}}%
\pgfpathlineto{\pgfqpoint{3.109173in}{2.078218in}}%
\pgfpathlineto{\pgfqpoint{3.100695in}{2.087316in}}%
\pgfpathlineto{\pgfqpoint{3.092196in}{2.096909in}}%
\pgfpathlineto{\pgfqpoint{3.078834in}{2.105463in}}%
\pgfpathlineto{\pgfqpoint{3.065476in}{2.114048in}}%
\pgfpathlineto{\pgfqpoint{3.052120in}{2.122665in}}%
\pgfpathlineto{\pgfqpoint{3.038767in}{2.131313in}}%
\pgfpathlineto{\pgfqpoint{3.047308in}{2.121352in}}%
\pgfpathlineto{\pgfqpoint{3.055826in}{2.111890in}}%
\pgfpathlineto{\pgfqpoint{3.064323in}{2.102916in}}%
\pgfpathlineto{\pgfqpoint{3.072798in}{2.094420in}}%
\pgfpathclose%
\pgfusepath{fill}%
\end{pgfscope}%
\begin{pgfscope}%
\pgfpathrectangle{\pgfqpoint{1.254980in}{0.150000in}}{\pgfqpoint{5.490039in}{5.490039in}}%
\pgfusepath{clip}%
\pgfsetbuttcap%
\pgfsetroundjoin%
\definecolor{currentfill}{rgb}{0.137770,0.537492,0.554906}%
\pgfsetfillcolor{currentfill}%
\pgfsetfillopacity{0.700000}%
\pgfsetlinewidth{0.000000pt}%
\definecolor{currentstroke}{rgb}{0.000000,0.000000,0.000000}%
\pgfsetstrokecolor{currentstroke}%
\pgfsetdash{}{0pt}%
\pgfpathmoveto{\pgfqpoint{2.257528in}{2.740510in}}%
\pgfpathlineto{\pgfqpoint{2.270829in}{2.729252in}}%
\pgfpathlineto{\pgfqpoint{2.284130in}{2.718045in}}%
\pgfpathlineto{\pgfqpoint{2.297432in}{2.706889in}}%
\pgfpathlineto{\pgfqpoint{2.310735in}{2.695781in}}%
\pgfpathlineto{\pgfqpoint{2.301418in}{2.713514in}}%
\pgfpathlineto{\pgfqpoint{2.292065in}{2.731867in}}%
\pgfpathlineto{\pgfqpoint{2.282674in}{2.750853in}}%
\pgfpathlineto{\pgfqpoint{2.273244in}{2.770484in}}%
\pgfpathlineto{\pgfqpoint{2.259884in}{2.781999in}}%
\pgfpathlineto{\pgfqpoint{2.246524in}{2.793564in}}%
\pgfpathlineto{\pgfqpoint{2.233165in}{2.805180in}}%
\pgfpathlineto{\pgfqpoint{2.219805in}{2.816846in}}%
\pgfpathlineto{\pgfqpoint{2.229294in}{2.796800in}}%
\pgfpathlineto{\pgfqpoint{2.238744in}{2.777403in}}%
\pgfpathlineto{\pgfqpoint{2.248154in}{2.758644in}}%
\pgfpathlineto{\pgfqpoint{2.257528in}{2.740510in}}%
\pgfpathclose%
\pgfusepath{fill}%
\end{pgfscope}%
\begin{pgfscope}%
\pgfpathrectangle{\pgfqpoint{1.254980in}{0.150000in}}{\pgfqpoint{5.490039in}{5.490039in}}%
\pgfusepath{clip}%
\pgfsetbuttcap%
\pgfsetroundjoin%
\definecolor{currentfill}{rgb}{0.278791,0.062145,0.386592}%
\pgfsetfillcolor{currentfill}%
\pgfsetfillopacity{0.700000}%
\pgfsetlinewidth{0.000000pt}%
\definecolor{currentstroke}{rgb}{0.000000,0.000000,0.000000}%
\pgfsetstrokecolor{currentstroke}%
\pgfsetdash{}{0pt}%
\pgfpathmoveto{\pgfqpoint{4.953120in}{1.680167in}}%
\pgfpathlineto{\pgfqpoint{4.966822in}{1.677982in}}%
\pgfpathlineto{\pgfqpoint{4.980531in}{1.675820in}}%
\pgfpathlineto{\pgfqpoint{4.994248in}{1.673682in}}%
\pgfpathlineto{\pgfqpoint{5.007972in}{1.671566in}}%
\pgfpathlineto{\pgfqpoint{5.000497in}{1.661159in}}%
\pgfpathlineto{\pgfqpoint{4.993017in}{1.650787in}}%
\pgfpathlineto{\pgfqpoint{4.985534in}{1.640456in}}%
\pgfpathlineto{\pgfqpoint{4.978046in}{1.630169in}}%
\pgfpathlineto{\pgfqpoint{4.964314in}{1.632484in}}%
\pgfpathlineto{\pgfqpoint{4.950590in}{1.634821in}}%
\pgfpathlineto{\pgfqpoint{4.936874in}{1.637181in}}%
\pgfpathlineto{\pgfqpoint{4.923166in}{1.639565in}}%
\pgfpathlineto{\pgfqpoint{4.930660in}{1.649648in}}%
\pgfpathlineto{\pgfqpoint{4.938151in}{1.659779in}}%
\pgfpathlineto{\pgfqpoint{4.945638in}{1.669954in}}%
\pgfpathlineto{\pgfqpoint{4.953120in}{1.680167in}}%
\pgfpathclose%
\pgfusepath{fill}%
\end{pgfscope}%
\begin{pgfscope}%
\pgfpathrectangle{\pgfqpoint{1.254980in}{0.150000in}}{\pgfqpoint{5.490039in}{5.490039in}}%
\pgfusepath{clip}%
\pgfsetbuttcap%
\pgfsetroundjoin%
\definecolor{currentfill}{rgb}{0.271305,0.019942,0.347269}%
\pgfsetfillcolor{currentfill}%
\pgfsetfillopacity{0.700000}%
\pgfsetlinewidth{0.000000pt}%
\definecolor{currentstroke}{rgb}{0.000000,0.000000,0.000000}%
\pgfsetstrokecolor{currentstroke}%
\pgfsetdash{}{0pt}%
\pgfpathmoveto{\pgfqpoint{4.644472in}{1.609520in}}%
\pgfpathlineto{\pgfqpoint{4.658081in}{1.606329in}}%
\pgfpathlineto{\pgfqpoint{4.671696in}{1.603162in}}%
\pgfpathlineto{\pgfqpoint{4.685319in}{1.600018in}}%
\pgfpathlineto{\pgfqpoint{4.698948in}{1.596897in}}%
\pgfpathlineto{\pgfqpoint{4.691392in}{1.588180in}}%
\pgfpathlineto{\pgfqpoint{4.683832in}{1.579574in}}%
\pgfpathlineto{\pgfqpoint{4.676267in}{1.571086in}}%
\pgfpathlineto{\pgfqpoint{4.668699in}{1.562721in}}%
\pgfpathlineto{\pgfqpoint{4.655059in}{1.566078in}}%
\pgfpathlineto{\pgfqpoint{4.641426in}{1.569458in}}%
\pgfpathlineto{\pgfqpoint{4.627801in}{1.572862in}}%
\pgfpathlineto{\pgfqpoint{4.614182in}{1.576289in}}%
\pgfpathlineto{\pgfqpoint{4.621761in}{1.584413in}}%
\pgfpathlineto{\pgfqpoint{4.629335in}{1.592663in}}%
\pgfpathlineto{\pgfqpoint{4.636906in}{1.601034in}}%
\pgfpathlineto{\pgfqpoint{4.644472in}{1.609520in}}%
\pgfpathclose%
\pgfusepath{fill}%
\end{pgfscope}%
\begin{pgfscope}%
\pgfpathrectangle{\pgfqpoint{1.254980in}{0.150000in}}{\pgfqpoint{5.490039in}{5.490039in}}%
\pgfusepath{clip}%
\pgfsetbuttcap%
\pgfsetroundjoin%
\definecolor{currentfill}{rgb}{0.275191,0.194905,0.496005}%
\pgfsetfillcolor{currentfill}%
\pgfsetfillopacity{0.700000}%
\pgfsetlinewidth{0.000000pt}%
\definecolor{currentstroke}{rgb}{0.000000,0.000000,0.000000}%
\pgfsetstrokecolor{currentstroke}%
\pgfsetdash{}{0pt}%
\pgfpathmoveto{\pgfqpoint{3.372666in}{1.911976in}}%
\pgfpathlineto{\pgfqpoint{3.386013in}{1.904632in}}%
\pgfpathlineto{\pgfqpoint{3.399365in}{1.897316in}}%
\pgfpathlineto{\pgfqpoint{3.412720in}{1.890028in}}%
\pgfpathlineto{\pgfqpoint{3.426080in}{1.882768in}}%
\pgfpathlineto{\pgfqpoint{3.417892in}{1.887416in}}%
\pgfpathlineto{\pgfqpoint{3.409688in}{1.892479in}}%
\pgfpathlineto{\pgfqpoint{3.401468in}{1.897968in}}%
\pgfpathlineto{\pgfqpoint{3.393232in}{1.903892in}}%
\pgfpathlineto{\pgfqpoint{3.379839in}{1.911496in}}%
\pgfpathlineto{\pgfqpoint{3.366450in}{1.919128in}}%
\pgfpathlineto{\pgfqpoint{3.353065in}{1.926789in}}%
\pgfpathlineto{\pgfqpoint{3.339683in}{1.934478in}}%
\pgfpathlineto{\pgfqpoint{3.347954in}{1.928204in}}%
\pgfpathlineto{\pgfqpoint{3.356208in}{1.922369in}}%
\pgfpathlineto{\pgfqpoint{3.364445in}{1.916963in}}%
\pgfpathlineto{\pgfqpoint{3.372666in}{1.911976in}}%
\pgfpathclose%
\pgfusepath{fill}%
\end{pgfscope}%
\begin{pgfscope}%
\pgfpathrectangle{\pgfqpoint{1.254980in}{0.150000in}}{\pgfqpoint{5.490039in}{5.490039in}}%
\pgfusepath{clip}%
\pgfsetbuttcap%
\pgfsetroundjoin%
\definecolor{currentfill}{rgb}{0.281412,0.155834,0.469201}%
\pgfsetfillcolor{currentfill}%
\pgfsetfillopacity{0.700000}%
\pgfsetlinewidth{0.000000pt}%
\definecolor{currentstroke}{rgb}{0.000000,0.000000,0.000000}%
\pgfsetstrokecolor{currentstroke}%
\pgfsetdash{}{0pt}%
\pgfpathmoveto{\pgfqpoint{5.432070in}{1.859210in}}%
\pgfpathlineto{\pgfqpoint{5.445939in}{1.858387in}}%
\pgfpathlineto{\pgfqpoint{5.459816in}{1.857587in}}%
\pgfpathlineto{\pgfqpoint{5.473703in}{1.856811in}}%
\pgfpathlineto{\pgfqpoint{5.466363in}{1.845816in}}%
\pgfpathlineto{\pgfqpoint{5.459016in}{1.834761in}}%
\pgfpathlineto{\pgfqpoint{5.451664in}{1.823648in}}%
\pgfpathlineto{\pgfqpoint{5.444306in}{1.812482in}}%
\pgfpathlineto{\pgfqpoint{5.430414in}{1.813392in}}%
\pgfpathlineto{\pgfqpoint{5.416531in}{1.814327in}}%
\pgfpathlineto{\pgfqpoint{5.402656in}{1.815284in}}%
\pgfpathlineto{\pgfqpoint{5.410019in}{1.826346in}}%
\pgfpathlineto{\pgfqpoint{5.417375in}{1.837357in}}%
\pgfpathlineto{\pgfqpoint{5.424726in}{1.848312in}}%
\pgfpathlineto{\pgfqpoint{5.432070in}{1.859210in}}%
\pgfpathclose%
\pgfusepath{fill}%
\end{pgfscope}%
\begin{pgfscope}%
\pgfpathrectangle{\pgfqpoint{1.254980in}{0.150000in}}{\pgfqpoint{5.490039in}{5.490039in}}%
\pgfusepath{clip}%
\pgfsetbuttcap%
\pgfsetroundjoin%
\definecolor{currentfill}{rgb}{0.204903,0.375746,0.553533}%
\pgfsetfillcolor{currentfill}%
\pgfsetfillopacity{0.700000}%
\pgfsetlinewidth{0.000000pt}%
\definecolor{currentstroke}{rgb}{0.000000,0.000000,0.000000}%
\pgfsetstrokecolor{currentstroke}%
\pgfsetdash{}{0pt}%
\pgfpathmoveto{\pgfqpoint{2.772290in}{2.311249in}}%
\pgfpathlineto{\pgfqpoint{2.785590in}{2.301921in}}%
\pgfpathlineto{\pgfqpoint{2.798893in}{2.292631in}}%
\pgfpathlineto{\pgfqpoint{2.812197in}{2.283376in}}%
\pgfpathlineto{\pgfqpoint{2.825504in}{2.274157in}}%
\pgfpathlineto{\pgfqpoint{2.816766in}{2.286119in}}%
\pgfpathlineto{\pgfqpoint{2.808002in}{2.298616in}}%
\pgfpathlineto{\pgfqpoint{2.799211in}{2.311661in}}%
\pgfpathlineto{\pgfqpoint{2.790392in}{2.325263in}}%
\pgfpathlineto{\pgfqpoint{2.777039in}{2.334863in}}%
\pgfpathlineto{\pgfqpoint{2.763687in}{2.344499in}}%
\pgfpathlineto{\pgfqpoint{2.750338in}{2.354171in}}%
\pgfpathlineto{\pgfqpoint{2.736991in}{2.363879in}}%
\pgfpathlineto{\pgfqpoint{2.745858in}{2.349889in}}%
\pgfpathlineto{\pgfqpoint{2.754696in}{2.336462in}}%
\pgfpathlineto{\pgfqpoint{2.763507in}{2.323585in}}%
\pgfpathlineto{\pgfqpoint{2.772290in}{2.311249in}}%
\pgfpathclose%
\pgfusepath{fill}%
\end{pgfscope}%
\begin{pgfscope}%
\pgfpathrectangle{\pgfqpoint{1.254980in}{0.150000in}}{\pgfqpoint{5.490039in}{5.490039in}}%
\pgfusepath{clip}%
\pgfsetbuttcap%
\pgfsetroundjoin%
\definecolor{currentfill}{rgb}{0.143343,0.522773,0.556295}%
\pgfsetfillcolor{currentfill}%
\pgfsetfillopacity{0.700000}%
\pgfsetlinewidth{0.000000pt}%
\definecolor{currentstroke}{rgb}{0.000000,0.000000,0.000000}%
\pgfsetstrokecolor{currentstroke}%
\pgfsetdash{}{0pt}%
\pgfpathmoveto{\pgfqpoint{2.310735in}{2.695781in}}%
\pgfpathlineto{\pgfqpoint{2.324038in}{2.684723in}}%
\pgfpathlineto{\pgfqpoint{2.337342in}{2.673714in}}%
\pgfpathlineto{\pgfqpoint{2.350646in}{2.662753in}}%
\pgfpathlineto{\pgfqpoint{2.363951in}{2.651839in}}%
\pgfpathlineto{\pgfqpoint{2.354690in}{2.669171in}}%
\pgfpathlineto{\pgfqpoint{2.345393in}{2.687120in}}%
\pgfpathlineto{\pgfqpoint{2.336060in}{2.705697in}}%
\pgfpathlineto{\pgfqpoint{2.326689in}{2.724914in}}%
\pgfpathlineto{\pgfqpoint{2.313327in}{2.736234in}}%
\pgfpathlineto{\pgfqpoint{2.299966in}{2.747602in}}%
\pgfpathlineto{\pgfqpoint{2.286605in}{2.759019in}}%
\pgfpathlineto{\pgfqpoint{2.273244in}{2.770484in}}%
\pgfpathlineto{\pgfqpoint{2.282674in}{2.750853in}}%
\pgfpathlineto{\pgfqpoint{2.292065in}{2.731867in}}%
\pgfpathlineto{\pgfqpoint{2.301418in}{2.713514in}}%
\pgfpathlineto{\pgfqpoint{2.310735in}{2.695781in}}%
\pgfpathclose%
\pgfusepath{fill}%
\end{pgfscope}%
\begin{pgfscope}%
\pgfpathrectangle{\pgfqpoint{1.254980in}{0.150000in}}{\pgfqpoint{5.490039in}{5.490039in}}%
\pgfusepath{clip}%
\pgfsetbuttcap%
\pgfsetroundjoin%
\definecolor{currentfill}{rgb}{0.271305,0.019942,0.347269}%
\pgfsetfillcolor{currentfill}%
\pgfsetfillopacity{0.700000}%
\pgfsetlinewidth{0.000000pt}%
\definecolor{currentstroke}{rgb}{0.000000,0.000000,0.000000}%
\pgfsetstrokecolor{currentstroke}%
\pgfsetdash{}{0pt}%
\pgfpathmoveto{\pgfqpoint{4.281913in}{1.600272in}}%
\pgfpathlineto{\pgfqpoint{4.295427in}{1.595865in}}%
\pgfpathlineto{\pgfqpoint{4.308948in}{1.591482in}}%
\pgfpathlineto{\pgfqpoint{4.322475in}{1.587122in}}%
\pgfpathlineto{\pgfqpoint{4.336008in}{1.582786in}}%
\pgfpathlineto{\pgfqpoint{4.328335in}{1.577165in}}%
\pgfpathlineto{\pgfqpoint{4.320657in}{1.571749in}}%
\pgfpathlineto{\pgfqpoint{4.312973in}{1.566545in}}%
\pgfpathlineto{\pgfqpoint{4.305284in}{1.561558in}}%
\pgfpathlineto{\pgfqpoint{4.291735in}{1.566169in}}%
\pgfpathlineto{\pgfqpoint{4.278193in}{1.570803in}}%
\pgfpathlineto{\pgfqpoint{4.264657in}{1.575461in}}%
\pgfpathlineto{\pgfqpoint{4.251126in}{1.580142in}}%
\pgfpathlineto{\pgfqpoint{4.258832in}{1.584849in}}%
\pgfpathlineto{\pgfqpoint{4.266531in}{1.589778in}}%
\pgfpathlineto{\pgfqpoint{4.274225in}{1.594921in}}%
\pgfpathlineto{\pgfqpoint{4.281913in}{1.600272in}}%
\pgfpathclose%
\pgfusepath{fill}%
\end{pgfscope}%
\begin{pgfscope}%
\pgfpathrectangle{\pgfqpoint{1.254980in}{0.150000in}}{\pgfqpoint{5.490039in}{5.490039in}}%
\pgfusepath{clip}%
\pgfsetbuttcap%
\pgfsetroundjoin%
\definecolor{currentfill}{rgb}{0.277018,0.050344,0.375715}%
\pgfsetfillcolor{currentfill}%
\pgfsetfillopacity{0.700000}%
\pgfsetlinewidth{0.000000pt}%
\definecolor{currentstroke}{rgb}{0.000000,0.000000,0.000000}%
\pgfsetstrokecolor{currentstroke}%
\pgfsetdash{}{0pt}%
\pgfpathmoveto{\pgfqpoint{4.868407in}{1.649332in}}%
\pgfpathlineto{\pgfqpoint{4.882086in}{1.646856in}}%
\pgfpathlineto{\pgfqpoint{4.895771in}{1.644402in}}%
\pgfpathlineto{\pgfqpoint{4.909465in}{1.641972in}}%
\pgfpathlineto{\pgfqpoint{4.923166in}{1.639565in}}%
\pgfpathlineto{\pgfqpoint{4.915667in}{1.629536in}}%
\pgfpathlineto{\pgfqpoint{4.908164in}{1.619565in}}%
\pgfpathlineto{\pgfqpoint{4.900657in}{1.609657in}}%
\pgfpathlineto{\pgfqpoint{4.893146in}{1.599818in}}%
\pgfpathlineto{\pgfqpoint{4.879438in}{1.602437in}}%
\pgfpathlineto{\pgfqpoint{4.865737in}{1.605078in}}%
\pgfpathlineto{\pgfqpoint{4.852043in}{1.607743in}}%
\pgfpathlineto{\pgfqpoint{4.838357in}{1.610431in}}%
\pgfpathlineto{\pgfqpoint{4.845876in}{1.620053in}}%
\pgfpathlineto{\pgfqpoint{4.853390in}{1.629748in}}%
\pgfpathlineto{\pgfqpoint{4.860901in}{1.639509in}}%
\pgfpathlineto{\pgfqpoint{4.868407in}{1.649332in}}%
\pgfpathclose%
\pgfusepath{fill}%
\end{pgfscope}%
\begin{pgfscope}%
\pgfpathrectangle{\pgfqpoint{1.254980in}{0.150000in}}{\pgfqpoint{5.490039in}{5.490039in}}%
\pgfusepath{clip}%
\pgfsetbuttcap%
\pgfsetroundjoin%
\definecolor{currentfill}{rgb}{0.273809,0.031497,0.358853}%
\pgfsetfillcolor{currentfill}%
\pgfsetfillopacity{0.700000}%
\pgfsetlinewidth{0.000000pt}%
\definecolor{currentstroke}{rgb}{0.000000,0.000000,0.000000}%
\pgfsetstrokecolor{currentstroke}%
\pgfsetdash{}{0pt}%
\pgfpathmoveto{\pgfqpoint{4.143097in}{1.618453in}}%
\pgfpathlineto{\pgfqpoint{4.156580in}{1.613580in}}%
\pgfpathlineto{\pgfqpoint{4.170069in}{1.608732in}}%
\pgfpathlineto{\pgfqpoint{4.183564in}{1.603907in}}%
\pgfpathlineto{\pgfqpoint{4.197065in}{1.599106in}}%
\pgfpathlineto{\pgfqpoint{4.189336in}{1.594911in}}%
\pgfpathlineto{\pgfqpoint{4.181601in}{1.590956in}}%
\pgfpathlineto{\pgfqpoint{4.173859in}{1.587248in}}%
\pgfpathlineto{\pgfqpoint{4.166111in}{1.583795in}}%
\pgfpathlineto{\pgfqpoint{4.152592in}{1.588883in}}%
\pgfpathlineto{\pgfqpoint{4.139079in}{1.593995in}}%
\pgfpathlineto{\pgfqpoint{4.125572in}{1.599131in}}%
\pgfpathlineto{\pgfqpoint{4.112071in}{1.604291in}}%
\pgfpathlineto{\pgfqpoint{4.119838in}{1.607452in}}%
\pgfpathlineto{\pgfqpoint{4.127598in}{1.610871in}}%
\pgfpathlineto{\pgfqpoint{4.135351in}{1.614541in}}%
\pgfpathlineto{\pgfqpoint{4.143097in}{1.618453in}}%
\pgfpathclose%
\pgfusepath{fill}%
\end{pgfscope}%
\begin{pgfscope}%
\pgfpathrectangle{\pgfqpoint{1.254980in}{0.150000in}}{\pgfqpoint{5.490039in}{5.490039in}}%
\pgfusepath{clip}%
\pgfsetbuttcap%
\pgfsetroundjoin%
\definecolor{currentfill}{rgb}{0.281924,0.089666,0.412415}%
\pgfsetfillcolor{currentfill}%
\pgfsetfillopacity{0.700000}%
\pgfsetlinewidth{0.000000pt}%
\definecolor{currentstroke}{rgb}{0.000000,0.000000,0.000000}%
\pgfsetstrokecolor{currentstroke}%
\pgfsetdash{}{0pt}%
\pgfpathmoveto{\pgfqpoint{3.811618in}{1.708740in}}%
\pgfpathlineto{\pgfqpoint{3.825035in}{1.702786in}}%
\pgfpathlineto{\pgfqpoint{3.838456in}{1.696857in}}%
\pgfpathlineto{\pgfqpoint{3.851883in}{1.690954in}}%
\pgfpathlineto{\pgfqpoint{3.865315in}{1.685075in}}%
\pgfpathlineto{\pgfqpoint{3.857421in}{1.684560in}}%
\pgfpathlineto{\pgfqpoint{3.849517in}{1.684363in}}%
\pgfpathlineto{\pgfqpoint{3.841603in}{1.684495in}}%
\pgfpathlineto{\pgfqpoint{3.833680in}{1.684964in}}%
\pgfpathlineto{\pgfqpoint{3.820224in}{1.691156in}}%
\pgfpathlineto{\pgfqpoint{3.806773in}{1.697374in}}%
\pgfpathlineto{\pgfqpoint{3.793327in}{1.703618in}}%
\pgfpathlineto{\pgfqpoint{3.779885in}{1.709886in}}%
\pgfpathlineto{\pgfqpoint{3.787834in}{1.709098in}}%
\pgfpathlineto{\pgfqpoint{3.795773in}{1.708650in}}%
\pgfpathlineto{\pgfqpoint{3.803701in}{1.708533in}}%
\pgfpathlineto{\pgfqpoint{3.811618in}{1.708740in}}%
\pgfpathclose%
\pgfusepath{fill}%
\end{pgfscope}%
\begin{pgfscope}%
\pgfpathrectangle{\pgfqpoint{1.254980in}{0.150000in}}{\pgfqpoint{5.490039in}{5.490039in}}%
\pgfusepath{clip}%
\pgfsetbuttcap%
\pgfsetroundjoin%
\definecolor{currentfill}{rgb}{0.269944,0.014625,0.341379}%
\pgfsetfillcolor{currentfill}%
\pgfsetfillopacity{0.700000}%
\pgfsetlinewidth{0.000000pt}%
\definecolor{currentstroke}{rgb}{0.000000,0.000000,0.000000}%
\pgfsetstrokecolor{currentstroke}%
\pgfsetdash{}{0pt}%
\pgfpathmoveto{\pgfqpoint{4.420783in}{1.591115in}}%
\pgfpathlineto{\pgfqpoint{4.434333in}{1.587158in}}%
\pgfpathlineto{\pgfqpoint{4.447891in}{1.583225in}}%
\pgfpathlineto{\pgfqpoint{4.461454in}{1.579316in}}%
\pgfpathlineto{\pgfqpoint{4.475024in}{1.575430in}}%
\pgfpathlineto{\pgfqpoint{4.467400in}{1.568546in}}%
\pgfpathlineto{\pgfqpoint{4.459770in}{1.561833in}}%
\pgfpathlineto{\pgfqpoint{4.452136in}{1.555298in}}%
\pgfpathlineto{\pgfqpoint{4.444497in}{1.548947in}}%
\pgfpathlineto{\pgfqpoint{4.430914in}{1.553095in}}%
\pgfpathlineto{\pgfqpoint{4.417337in}{1.557266in}}%
\pgfpathlineto{\pgfqpoint{4.403766in}{1.561460in}}%
\pgfpathlineto{\pgfqpoint{4.390202in}{1.565678in}}%
\pgfpathlineto{\pgfqpoint{4.397855in}{1.571763in}}%
\pgfpathlineto{\pgfqpoint{4.405502in}{1.578035in}}%
\pgfpathlineto{\pgfqpoint{4.413145in}{1.584488in}}%
\pgfpathlineto{\pgfqpoint{4.420783in}{1.591115in}}%
\pgfpathclose%
\pgfusepath{fill}%
\end{pgfscope}%
\begin{pgfscope}%
\pgfpathrectangle{\pgfqpoint{1.254980in}{0.150000in}}{\pgfqpoint{5.490039in}{5.490039in}}%
\pgfusepath{clip}%
\pgfsetbuttcap%
\pgfsetroundjoin%
\definecolor{currentfill}{rgb}{0.283072,0.130895,0.449241}%
\pgfsetfillcolor{currentfill}%
\pgfsetfillopacity{0.700000}%
\pgfsetlinewidth{0.000000pt}%
\definecolor{currentstroke}{rgb}{0.000000,0.000000,0.000000}%
\pgfsetstrokecolor{currentstroke}%
\pgfsetdash{}{0pt}%
\pgfpathmoveto{\pgfqpoint{3.618959in}{1.787111in}}%
\pgfpathlineto{\pgfqpoint{3.632344in}{1.780532in}}%
\pgfpathlineto{\pgfqpoint{3.645733in}{1.773980in}}%
\pgfpathlineto{\pgfqpoint{3.659128in}{1.767454in}}%
\pgfpathlineto{\pgfqpoint{3.672527in}{1.760954in}}%
\pgfpathlineto{\pgfqpoint{3.664513in}{1.762738in}}%
\pgfpathlineto{\pgfqpoint{3.656487in}{1.764885in}}%
\pgfpathlineto{\pgfqpoint{3.648449in}{1.767407in}}%
\pgfpathlineto{\pgfqpoint{3.640398in}{1.770310in}}%
\pgfpathlineto{\pgfqpoint{3.626971in}{1.777139in}}%
\pgfpathlineto{\pgfqpoint{3.613548in}{1.783994in}}%
\pgfpathlineto{\pgfqpoint{3.600130in}{1.790875in}}%
\pgfpathlineto{\pgfqpoint{3.586716in}{1.797782in}}%
\pgfpathlineto{\pgfqpoint{3.594796in}{1.794544in}}%
\pgfpathlineto{\pgfqpoint{3.602863in}{1.791693in}}%
\pgfpathlineto{\pgfqpoint{3.610917in}{1.789218in}}%
\pgfpathlineto{\pgfqpoint{3.618959in}{1.787111in}}%
\pgfpathclose%
\pgfusepath{fill}%
\end{pgfscope}%
\begin{pgfscope}%
\pgfpathrectangle{\pgfqpoint{1.254980in}{0.150000in}}{\pgfqpoint{5.490039in}{5.490039in}}%
\pgfusepath{clip}%
\pgfsetbuttcap%
\pgfsetroundjoin%
\definecolor{currentfill}{rgb}{0.282623,0.140926,0.457517}%
\pgfsetfillcolor{currentfill}%
\pgfsetfillopacity{0.700000}%
\pgfsetlinewidth{0.000000pt}%
\definecolor{currentstroke}{rgb}{0.000000,0.000000,0.000000}%
\pgfsetstrokecolor{currentstroke}%
\pgfsetdash{}{0pt}%
\pgfpathmoveto{\pgfqpoint{5.347245in}{1.819350in}}%
\pgfpathlineto{\pgfqpoint{5.361085in}{1.818298in}}%
\pgfpathlineto{\pgfqpoint{5.374933in}{1.817270in}}%
\pgfpathlineto{\pgfqpoint{5.388791in}{1.816265in}}%
\pgfpathlineto{\pgfqpoint{5.402656in}{1.815284in}}%
\pgfpathlineto{\pgfqpoint{5.395288in}{1.804174in}}%
\pgfpathlineto{\pgfqpoint{5.387915in}{1.793018in}}%
\pgfpathlineto{\pgfqpoint{5.380535in}{1.781820in}}%
\pgfpathlineto{\pgfqpoint{5.373151in}{1.770584in}}%
\pgfpathlineto{\pgfqpoint{5.359279in}{1.771713in}}%
\pgfpathlineto{\pgfqpoint{5.345417in}{1.772866in}}%
\pgfpathlineto{\pgfqpoint{5.331563in}{1.774041in}}%
\pgfpathlineto{\pgfqpoint{5.317717in}{1.775240in}}%
\pgfpathlineto{\pgfqpoint{5.325107in}{1.786324in}}%
\pgfpathlineto{\pgfqpoint{5.332492in}{1.797372in}}%
\pgfpathlineto{\pgfqpoint{5.339871in}{1.808382in}}%
\pgfpathlineto{\pgfqpoint{5.347245in}{1.819350in}}%
\pgfpathclose%
\pgfusepath{fill}%
\end{pgfscope}%
\begin{pgfscope}%
\pgfpathrectangle{\pgfqpoint{1.254980in}{0.150000in}}{\pgfqpoint{5.490039in}{5.490039in}}%
\pgfusepath{clip}%
\pgfsetbuttcap%
\pgfsetroundjoin%
\definecolor{currentfill}{rgb}{0.252194,0.269783,0.531579}%
\pgfsetfillcolor{currentfill}%
\pgfsetfillopacity{0.700000}%
\pgfsetlinewidth{0.000000pt}%
\definecolor{currentstroke}{rgb}{0.000000,0.000000,0.000000}%
\pgfsetstrokecolor{currentstroke}%
\pgfsetdash{}{0pt}%
\pgfpathmoveto{\pgfqpoint{3.126067in}{2.061463in}}%
\pgfpathlineto{\pgfqpoint{3.139392in}{2.053301in}}%
\pgfpathlineto{\pgfqpoint{3.152721in}{2.045171in}}%
\pgfpathlineto{\pgfqpoint{3.166053in}{2.037071in}}%
\pgfpathlineto{\pgfqpoint{3.179388in}{2.029001in}}%
\pgfpathlineto{\pgfqpoint{3.170990in}{2.036786in}}%
\pgfpathlineto{\pgfqpoint{3.162571in}{2.045042in}}%
\pgfpathlineto{\pgfqpoint{3.154133in}{2.053777in}}%
\pgfpathlineto{\pgfqpoint{3.145674in}{2.063003in}}%
\pgfpathlineto{\pgfqpoint{3.132299in}{2.071433in}}%
\pgfpathlineto{\pgfqpoint{3.118928in}{2.079894in}}%
\pgfpathlineto{\pgfqpoint{3.105561in}{2.088386in}}%
\pgfpathlineto{\pgfqpoint{3.092196in}{2.096909in}}%
\pgfpathlineto{\pgfqpoint{3.100695in}{2.087316in}}%
\pgfpathlineto{\pgfqpoint{3.109173in}{2.078218in}}%
\pgfpathlineto{\pgfqpoint{3.117630in}{2.069603in}}%
\pgfpathlineto{\pgfqpoint{3.126067in}{2.061463in}}%
\pgfpathclose%
\pgfusepath{fill}%
\end{pgfscope}%
\begin{pgfscope}%
\pgfpathrectangle{\pgfqpoint{1.254980in}{0.150000in}}{\pgfqpoint{5.490039in}{5.490039in}}%
\pgfusepath{clip}%
\pgfsetbuttcap%
\pgfsetroundjoin%
\definecolor{currentfill}{rgb}{0.283229,0.120777,0.440584}%
\pgfsetfillcolor{currentfill}%
\pgfsetfillopacity{0.700000}%
\pgfsetlinewidth{0.000000pt}%
\definecolor{currentstroke}{rgb}{0.000000,0.000000,0.000000}%
\pgfsetstrokecolor{currentstroke}%
\pgfsetdash{}{0pt}%
\pgfpathmoveto{\pgfqpoint{5.262420in}{1.780271in}}%
\pgfpathlineto{\pgfqpoint{5.276231in}{1.778978in}}%
\pgfpathlineto{\pgfqpoint{5.290051in}{1.777709in}}%
\pgfpathlineto{\pgfqpoint{5.303880in}{1.776463in}}%
\pgfpathlineto{\pgfqpoint{5.317717in}{1.775240in}}%
\pgfpathlineto{\pgfqpoint{5.310322in}{1.764125in}}%
\pgfpathlineto{\pgfqpoint{5.302921in}{1.752983in}}%
\pgfpathlineto{\pgfqpoint{5.295515in}{1.741815in}}%
\pgfpathlineto{\pgfqpoint{5.288105in}{1.730628in}}%
\pgfpathlineto{\pgfqpoint{5.274262in}{1.732011in}}%
\pgfpathlineto{\pgfqpoint{5.260428in}{1.733418in}}%
\pgfpathlineto{\pgfqpoint{5.246602in}{1.734847in}}%
\pgfpathlineto{\pgfqpoint{5.232785in}{1.736301in}}%
\pgfpathlineto{\pgfqpoint{5.240201in}{1.747323in}}%
\pgfpathlineto{\pgfqpoint{5.247612in}{1.758328in}}%
\pgfpathlineto{\pgfqpoint{5.255018in}{1.769312in}}%
\pgfpathlineto{\pgfqpoint{5.262420in}{1.780271in}}%
\pgfpathclose%
\pgfusepath{fill}%
\end{pgfscope}%
\begin{pgfscope}%
\pgfpathrectangle{\pgfqpoint{1.254980in}{0.150000in}}{\pgfqpoint{5.490039in}{5.490039in}}%
\pgfusepath{clip}%
\pgfsetbuttcap%
\pgfsetroundjoin%
\definecolor{currentfill}{rgb}{0.149039,0.508051,0.557250}%
\pgfsetfillcolor{currentfill}%
\pgfsetfillopacity{0.700000}%
\pgfsetlinewidth{0.000000pt}%
\definecolor{currentstroke}{rgb}{0.000000,0.000000,0.000000}%
\pgfsetstrokecolor{currentstroke}%
\pgfsetdash{}{0pt}%
\pgfpathmoveto{\pgfqpoint{2.363951in}{2.651839in}}%
\pgfpathlineto{\pgfqpoint{2.377257in}{2.640972in}}%
\pgfpathlineto{\pgfqpoint{2.390564in}{2.630153in}}%
\pgfpathlineto{\pgfqpoint{2.403872in}{2.619379in}}%
\pgfpathlineto{\pgfqpoint{2.417180in}{2.608651in}}%
\pgfpathlineto{\pgfqpoint{2.407974in}{2.625585in}}%
\pgfpathlineto{\pgfqpoint{2.398732in}{2.643130in}}%
\pgfpathlineto{\pgfqpoint{2.389455in}{2.661299in}}%
\pgfpathlineto{\pgfqpoint{2.380142in}{2.680104in}}%
\pgfpathlineto{\pgfqpoint{2.366778in}{2.691237in}}%
\pgfpathlineto{\pgfqpoint{2.353414in}{2.702416in}}%
\pgfpathlineto{\pgfqpoint{2.340051in}{2.713641in}}%
\pgfpathlineto{\pgfqpoint{2.326689in}{2.724914in}}%
\pgfpathlineto{\pgfqpoint{2.336060in}{2.705697in}}%
\pgfpathlineto{\pgfqpoint{2.345393in}{2.687120in}}%
\pgfpathlineto{\pgfqpoint{2.354690in}{2.669171in}}%
\pgfpathlineto{\pgfqpoint{2.363951in}{2.651839in}}%
\pgfpathclose%
\pgfusepath{fill}%
\end{pgfscope}%
\begin{pgfscope}%
\pgfpathrectangle{\pgfqpoint{1.254980in}{0.150000in}}{\pgfqpoint{5.490039in}{5.490039in}}%
\pgfusepath{clip}%
\pgfsetbuttcap%
\pgfsetroundjoin%
\definecolor{currentfill}{rgb}{0.277941,0.056324,0.381191}%
\pgfsetfillcolor{currentfill}%
\pgfsetfillopacity{0.700000}%
\pgfsetlinewidth{0.000000pt}%
\definecolor{currentstroke}{rgb}{0.000000,0.000000,0.000000}%
\pgfsetstrokecolor{currentstroke}%
\pgfsetdash{}{0pt}%
\pgfpathmoveto{\pgfqpoint{4.004259in}{1.646445in}}%
\pgfpathlineto{\pgfqpoint{4.017717in}{1.641091in}}%
\pgfpathlineto{\pgfqpoint{4.031179in}{1.635761in}}%
\pgfpathlineto{\pgfqpoint{4.044647in}{1.630455in}}%
\pgfpathlineto{\pgfqpoint{4.058121in}{1.625174in}}%
\pgfpathlineto{\pgfqpoint{4.050327in}{1.622574in}}%
\pgfpathlineto{\pgfqpoint{4.042525in}{1.620252in}}%
\pgfpathlineto{\pgfqpoint{4.034715in}{1.618215in}}%
\pgfpathlineto{\pgfqpoint{4.026897in}{1.616470in}}%
\pgfpathlineto{\pgfqpoint{4.013403in}{1.622052in}}%
\pgfpathlineto{\pgfqpoint{3.999914in}{1.627659in}}%
\pgfpathlineto{\pgfqpoint{3.986431in}{1.633290in}}%
\pgfpathlineto{\pgfqpoint{3.972952in}{1.638945in}}%
\pgfpathlineto{\pgfqpoint{3.980792in}{1.640383in}}%
\pgfpathlineto{\pgfqpoint{3.988623in}{1.642118in}}%
\pgfpathlineto{\pgfqpoint{3.996445in}{1.644142in}}%
\pgfpathlineto{\pgfqpoint{4.004259in}{1.646445in}}%
\pgfpathclose%
\pgfusepath{fill}%
\end{pgfscope}%
\begin{pgfscope}%
\pgfpathrectangle{\pgfqpoint{1.254980in}{0.150000in}}{\pgfqpoint{5.490039in}{5.490039in}}%
\pgfusepath{clip}%
\pgfsetbuttcap%
\pgfsetroundjoin%
\definecolor{currentfill}{rgb}{0.271305,0.019942,0.347269}%
\pgfsetfillcolor{currentfill}%
\pgfsetfillopacity{0.700000}%
\pgfsetlinewidth{0.000000pt}%
\definecolor{currentstroke}{rgb}{0.000000,0.000000,0.000000}%
\pgfsetstrokecolor{currentstroke}%
\pgfsetdash{}{0pt}%
\pgfpathmoveto{\pgfqpoint{4.559775in}{1.590229in}}%
\pgfpathlineto{\pgfqpoint{4.573366in}{1.586709in}}%
\pgfpathlineto{\pgfqpoint{4.586964in}{1.583212in}}%
\pgfpathlineto{\pgfqpoint{4.600570in}{1.579739in}}%
\pgfpathlineto{\pgfqpoint{4.614182in}{1.576289in}}%
\pgfpathlineto{\pgfqpoint{4.606598in}{1.568298in}}%
\pgfpathlineto{\pgfqpoint{4.599011in}{1.560446in}}%
\pgfpathlineto{\pgfqpoint{4.591419in}{1.552739in}}%
\pgfpathlineto{\pgfqpoint{4.583823in}{1.545183in}}%
\pgfpathlineto{\pgfqpoint{4.570200in}{1.548882in}}%
\pgfpathlineto{\pgfqpoint{4.556584in}{1.552605in}}%
\pgfpathlineto{\pgfqpoint{4.542974in}{1.556351in}}%
\pgfpathlineto{\pgfqpoint{4.529371in}{1.560120in}}%
\pgfpathlineto{\pgfqpoint{4.536978in}{1.567422in}}%
\pgfpathlineto{\pgfqpoint{4.544582in}{1.574878in}}%
\pgfpathlineto{\pgfqpoint{4.552180in}{1.582483in}}%
\pgfpathlineto{\pgfqpoint{4.559775in}{1.590229in}}%
\pgfpathclose%
\pgfusepath{fill}%
\end{pgfscope}%
\begin{pgfscope}%
\pgfpathrectangle{\pgfqpoint{1.254980in}{0.150000in}}{\pgfqpoint{5.490039in}{5.490039in}}%
\pgfusepath{clip}%
\pgfsetbuttcap%
\pgfsetroundjoin%
\definecolor{currentfill}{rgb}{0.274952,0.037752,0.364543}%
\pgfsetfillcolor{currentfill}%
\pgfsetfillopacity{0.700000}%
\pgfsetlinewidth{0.000000pt}%
\definecolor{currentstroke}{rgb}{0.000000,0.000000,0.000000}%
\pgfsetstrokecolor{currentstroke}%
\pgfsetdash{}{0pt}%
\pgfpathmoveto{\pgfqpoint{4.783686in}{1.621414in}}%
\pgfpathlineto{\pgfqpoint{4.797343in}{1.618634in}}%
\pgfpathlineto{\pgfqpoint{4.811007in}{1.615876in}}%
\pgfpathlineto{\pgfqpoint{4.824678in}{1.613142in}}%
\pgfpathlineto{\pgfqpoint{4.838357in}{1.610431in}}%
\pgfpathlineto{\pgfqpoint{4.830835in}{1.600885in}}%
\pgfpathlineto{\pgfqpoint{4.823308in}{1.591423in}}%
\pgfpathlineto{\pgfqpoint{4.815778in}{1.582048in}}%
\pgfpathlineto{\pgfqpoint{4.808244in}{1.572767in}}%
\pgfpathlineto{\pgfqpoint{4.794556in}{1.575703in}}%
\pgfpathlineto{\pgfqpoint{4.780877in}{1.578661in}}%
\pgfpathlineto{\pgfqpoint{4.767204in}{1.581642in}}%
\pgfpathlineto{\pgfqpoint{4.753538in}{1.584647in}}%
\pgfpathlineto{\pgfqpoint{4.761081in}{1.593699in}}%
\pgfpathlineto{\pgfqpoint{4.768620in}{1.602848in}}%
\pgfpathlineto{\pgfqpoint{4.776155in}{1.612088in}}%
\pgfpathlineto{\pgfqpoint{4.783686in}{1.621414in}}%
\pgfpathclose%
\pgfusepath{fill}%
\end{pgfscope}%
\begin{pgfscope}%
\pgfpathrectangle{\pgfqpoint{1.254980in}{0.150000in}}{\pgfqpoint{5.490039in}{5.490039in}}%
\pgfusepath{clip}%
\pgfsetbuttcap%
\pgfsetroundjoin%
\definecolor{currentfill}{rgb}{0.282910,0.105393,0.426902}%
\pgfsetfillcolor{currentfill}%
\pgfsetfillopacity{0.700000}%
\pgfsetlinewidth{0.000000pt}%
\definecolor{currentstroke}{rgb}{0.000000,0.000000,0.000000}%
\pgfsetstrokecolor{currentstroke}%
\pgfsetdash{}{0pt}%
\pgfpathmoveto{\pgfqpoint{5.177598in}{1.742347in}}%
\pgfpathlineto{\pgfqpoint{5.191382in}{1.740800in}}%
\pgfpathlineto{\pgfqpoint{5.205175in}{1.739277in}}%
\pgfpathlineto{\pgfqpoint{5.218975in}{1.737777in}}%
\pgfpathlineto{\pgfqpoint{5.232785in}{1.736301in}}%
\pgfpathlineto{\pgfqpoint{5.225363in}{1.725265in}}%
\pgfpathlineto{\pgfqpoint{5.217938in}{1.714221in}}%
\pgfpathlineto{\pgfqpoint{5.210507in}{1.703171in}}%
\pgfpathlineto{\pgfqpoint{5.203072in}{1.692121in}}%
\pgfpathlineto{\pgfqpoint{5.189257in}{1.693771in}}%
\pgfpathlineto{\pgfqpoint{5.175451in}{1.695444in}}%
\pgfpathlineto{\pgfqpoint{5.161652in}{1.697141in}}%
\pgfpathlineto{\pgfqpoint{5.147862in}{1.698861in}}%
\pgfpathlineto{\pgfqpoint{5.155303in}{1.709733in}}%
\pgfpathlineto{\pgfqpoint{5.162739in}{1.720608in}}%
\pgfpathlineto{\pgfqpoint{5.170171in}{1.731480in}}%
\pgfpathlineto{\pgfqpoint{5.177598in}{1.742347in}}%
\pgfpathclose%
\pgfusepath{fill}%
\end{pgfscope}%
\begin{pgfscope}%
\pgfpathrectangle{\pgfqpoint{1.254980in}{0.150000in}}{\pgfqpoint{5.490039in}{5.490039in}}%
\pgfusepath{clip}%
\pgfsetbuttcap%
\pgfsetroundjoin%
\definecolor{currentfill}{rgb}{0.210503,0.363727,0.552206}%
\pgfsetfillcolor{currentfill}%
\pgfsetfillopacity{0.700000}%
\pgfsetlinewidth{0.000000pt}%
\definecolor{currentstroke}{rgb}{0.000000,0.000000,0.000000}%
\pgfsetstrokecolor{currentstroke}%
\pgfsetdash{}{0pt}%
\pgfpathmoveto{\pgfqpoint{2.825504in}{2.274157in}}%
\pgfpathlineto{\pgfqpoint{2.838814in}{2.264973in}}%
\pgfpathlineto{\pgfqpoint{2.852126in}{2.255825in}}%
\pgfpathlineto{\pgfqpoint{2.865440in}{2.246712in}}%
\pgfpathlineto{\pgfqpoint{2.878757in}{2.237633in}}%
\pgfpathlineto{\pgfqpoint{2.870064in}{2.249221in}}%
\pgfpathlineto{\pgfqpoint{2.861345in}{2.261340in}}%
\pgfpathlineto{\pgfqpoint{2.852600in}{2.274002in}}%
\pgfpathlineto{\pgfqpoint{2.843829in}{2.287218in}}%
\pgfpathlineto{\pgfqpoint{2.830466in}{2.296677in}}%
\pgfpathlineto{\pgfqpoint{2.817106in}{2.306171in}}%
\pgfpathlineto{\pgfqpoint{2.803748in}{2.315699in}}%
\pgfpathlineto{\pgfqpoint{2.790392in}{2.325263in}}%
\pgfpathlineto{\pgfqpoint{2.799211in}{2.311661in}}%
\pgfpathlineto{\pgfqpoint{2.808002in}{2.298616in}}%
\pgfpathlineto{\pgfqpoint{2.816766in}{2.286119in}}%
\pgfpathlineto{\pgfqpoint{2.825504in}{2.274157in}}%
\pgfpathclose%
\pgfusepath{fill}%
\end{pgfscope}%
\begin{pgfscope}%
\pgfpathrectangle{\pgfqpoint{1.254980in}{0.150000in}}{\pgfqpoint{5.490039in}{5.490039in}}%
\pgfusepath{clip}%
\pgfsetbuttcap%
\pgfsetroundjoin%
\definecolor{currentfill}{rgb}{0.277134,0.185228,0.489898}%
\pgfsetfillcolor{currentfill}%
\pgfsetfillopacity{0.700000}%
\pgfsetlinewidth{0.000000pt}%
\definecolor{currentstroke}{rgb}{0.000000,0.000000,0.000000}%
\pgfsetstrokecolor{currentstroke}%
\pgfsetdash{}{0pt}%
\pgfpathmoveto{\pgfqpoint{3.426080in}{1.882768in}}%
\pgfpathlineto{\pgfqpoint{3.439443in}{1.875535in}}%
\pgfpathlineto{\pgfqpoint{3.452811in}{1.868331in}}%
\pgfpathlineto{\pgfqpoint{3.466183in}{1.861154in}}%
\pgfpathlineto{\pgfqpoint{3.479559in}{1.854004in}}%
\pgfpathlineto{\pgfqpoint{3.471403in}{1.858314in}}%
\pgfpathlineto{\pgfqpoint{3.463232in}{1.863035in}}%
\pgfpathlineto{\pgfqpoint{3.455045in}{1.868178in}}%
\pgfpathlineto{\pgfqpoint{3.446843in}{1.873752in}}%
\pgfpathlineto{\pgfqpoint{3.433434in}{1.881245in}}%
\pgfpathlineto{\pgfqpoint{3.420030in}{1.888766in}}%
\pgfpathlineto{\pgfqpoint{3.406629in}{1.896315in}}%
\pgfpathlineto{\pgfqpoint{3.393232in}{1.903892in}}%
\pgfpathlineto{\pgfqpoint{3.401468in}{1.897968in}}%
\pgfpathlineto{\pgfqpoint{3.409688in}{1.892479in}}%
\pgfpathlineto{\pgfqpoint{3.417892in}{1.887416in}}%
\pgfpathlineto{\pgfqpoint{3.426080in}{1.882768in}}%
\pgfpathclose%
\pgfusepath{fill}%
\end{pgfscope}%
\begin{pgfscope}%
\pgfpathrectangle{\pgfqpoint{1.254980in}{0.150000in}}{\pgfqpoint{5.490039in}{5.490039in}}%
\pgfusepath{clip}%
\pgfsetbuttcap%
\pgfsetroundjoin%
\definecolor{currentfill}{rgb}{0.281446,0.084320,0.407414}%
\pgfsetfillcolor{currentfill}%
\pgfsetfillopacity{0.700000}%
\pgfsetlinewidth{0.000000pt}%
\definecolor{currentstroke}{rgb}{0.000000,0.000000,0.000000}%
\pgfsetstrokecolor{currentstroke}%
\pgfsetdash{}{0pt}%
\pgfpathmoveto{\pgfqpoint{5.092782in}{1.705973in}}%
\pgfpathlineto{\pgfqpoint{5.106540in}{1.704160in}}%
\pgfpathlineto{\pgfqpoint{5.120306in}{1.702370in}}%
\pgfpathlineto{\pgfqpoint{5.134080in}{1.700604in}}%
\pgfpathlineto{\pgfqpoint{5.147862in}{1.698861in}}%
\pgfpathlineto{\pgfqpoint{5.140416in}{1.687995in}}%
\pgfpathlineto{\pgfqpoint{5.132967in}{1.677140in}}%
\pgfpathlineto{\pgfqpoint{5.125512in}{1.666300in}}%
\pgfpathlineto{\pgfqpoint{5.118054in}{1.655480in}}%
\pgfpathlineto{\pgfqpoint{5.104266in}{1.657409in}}%
\pgfpathlineto{\pgfqpoint{5.090486in}{1.659362in}}%
\pgfpathlineto{\pgfqpoint{5.076714in}{1.661338in}}%
\pgfpathlineto{\pgfqpoint{5.062950in}{1.663337in}}%
\pgfpathlineto{\pgfqpoint{5.070414in}{1.673966in}}%
\pgfpathlineto{\pgfqpoint{5.077875in}{1.684618in}}%
\pgfpathlineto{\pgfqpoint{5.085331in}{1.695289in}}%
\pgfpathlineto{\pgfqpoint{5.092782in}{1.705973in}}%
\pgfpathclose%
\pgfusepath{fill}%
\end{pgfscope}%
\begin{pgfscope}%
\pgfpathrectangle{\pgfqpoint{1.254980in}{0.150000in}}{\pgfqpoint{5.490039in}{5.490039in}}%
\pgfusepath{clip}%
\pgfsetbuttcap%
\pgfsetroundjoin%
\definecolor{currentfill}{rgb}{0.154815,0.493313,0.557840}%
\pgfsetfillcolor{currentfill}%
\pgfsetfillopacity{0.700000}%
\pgfsetlinewidth{0.000000pt}%
\definecolor{currentstroke}{rgb}{0.000000,0.000000,0.000000}%
\pgfsetstrokecolor{currentstroke}%
\pgfsetdash{}{0pt}%
\pgfpathmoveto{\pgfqpoint{2.417180in}{2.608651in}}%
\pgfpathlineto{\pgfqpoint{2.430490in}{2.597969in}}%
\pgfpathlineto{\pgfqpoint{2.443800in}{2.587331in}}%
\pgfpathlineto{\pgfqpoint{2.457112in}{2.576738in}}%
\pgfpathlineto{\pgfqpoint{2.470425in}{2.566189in}}%
\pgfpathlineto{\pgfqpoint{2.461272in}{2.582725in}}%
\pgfpathlineto{\pgfqpoint{2.452086in}{2.599869in}}%
\pgfpathlineto{\pgfqpoint{2.442864in}{2.617631in}}%
\pgfpathlineto{\pgfqpoint{2.433608in}{2.636024in}}%
\pgfpathlineto{\pgfqpoint{2.420240in}{2.646977in}}%
\pgfpathlineto{\pgfqpoint{2.406873in}{2.657975in}}%
\pgfpathlineto{\pgfqpoint{2.393507in}{2.669017in}}%
\pgfpathlineto{\pgfqpoint{2.380142in}{2.680104in}}%
\pgfpathlineto{\pgfqpoint{2.389455in}{2.661299in}}%
\pgfpathlineto{\pgfqpoint{2.398732in}{2.643130in}}%
\pgfpathlineto{\pgfqpoint{2.407974in}{2.625585in}}%
\pgfpathlineto{\pgfqpoint{2.417180in}{2.608651in}}%
\pgfpathclose%
\pgfusepath{fill}%
\end{pgfscope}%
\begin{pgfscope}%
\pgfpathrectangle{\pgfqpoint{1.254980in}{0.150000in}}{\pgfqpoint{5.490039in}{5.490039in}}%
\pgfusepath{clip}%
\pgfsetbuttcap%
\pgfsetroundjoin%
\definecolor{currentfill}{rgb}{0.257322,0.256130,0.526563}%
\pgfsetfillcolor{currentfill}%
\pgfsetfillopacity{0.700000}%
\pgfsetlinewidth{0.000000pt}%
\definecolor{currentstroke}{rgb}{0.000000,0.000000,0.000000}%
\pgfsetstrokecolor{currentstroke}%
\pgfsetdash{}{0pt}%
\pgfpathmoveto{\pgfqpoint{3.179388in}{2.029001in}}%
\pgfpathlineto{\pgfqpoint{3.192726in}{2.020961in}}%
\pgfpathlineto{\pgfqpoint{3.206069in}{2.012952in}}%
\pgfpathlineto{\pgfqpoint{3.219414in}{2.004973in}}%
\pgfpathlineto{\pgfqpoint{3.232763in}{1.997023in}}%
\pgfpathlineto{\pgfqpoint{3.224402in}{2.004453in}}%
\pgfpathlineto{\pgfqpoint{3.216022in}{2.012350in}}%
\pgfpathlineto{\pgfqpoint{3.207622in}{2.020723in}}%
\pgfpathlineto{\pgfqpoint{3.199203in}{2.029582in}}%
\pgfpathlineto{\pgfqpoint{3.185815in}{2.037892in}}%
\pgfpathlineto{\pgfqpoint{3.172432in}{2.046232in}}%
\pgfpathlineto{\pgfqpoint{3.159051in}{2.054602in}}%
\pgfpathlineto{\pgfqpoint{3.145674in}{2.063003in}}%
\pgfpathlineto{\pgfqpoint{3.154133in}{2.053777in}}%
\pgfpathlineto{\pgfqpoint{3.162571in}{2.045042in}}%
\pgfpathlineto{\pgfqpoint{3.170990in}{2.036786in}}%
\pgfpathlineto{\pgfqpoint{3.179388in}{2.029001in}}%
\pgfpathclose%
\pgfusepath{fill}%
\end{pgfscope}%
\begin{pgfscope}%
\pgfpathrectangle{\pgfqpoint{1.254980in}{0.150000in}}{\pgfqpoint{5.490039in}{5.490039in}}%
\pgfusepath{clip}%
\pgfsetbuttcap%
\pgfsetroundjoin%
\definecolor{currentfill}{rgb}{0.279566,0.067836,0.391917}%
\pgfsetfillcolor{currentfill}%
\pgfsetfillopacity{0.700000}%
\pgfsetlinewidth{0.000000pt}%
\definecolor{currentstroke}{rgb}{0.000000,0.000000,0.000000}%
\pgfsetstrokecolor{currentstroke}%
\pgfsetdash{}{0pt}%
\pgfpathmoveto{\pgfqpoint{5.007972in}{1.671566in}}%
\pgfpathlineto{\pgfqpoint{5.021705in}{1.669474in}}%
\pgfpathlineto{\pgfqpoint{5.035445in}{1.667405in}}%
\pgfpathlineto{\pgfqpoint{5.049193in}{1.665360in}}%
\pgfpathlineto{\pgfqpoint{5.062950in}{1.663337in}}%
\pgfpathlineto{\pgfqpoint{5.055481in}{1.652736in}}%
\pgfpathlineto{\pgfqpoint{5.048008in}{1.642167in}}%
\pgfpathlineto{\pgfqpoint{5.040531in}{1.631635in}}%
\pgfpathlineto{\pgfqpoint{5.033050in}{1.621145in}}%
\pgfpathlineto{\pgfqpoint{5.019287in}{1.623366in}}%
\pgfpathlineto{\pgfqpoint{5.005532in}{1.625611in}}%
\pgfpathlineto{\pgfqpoint{4.991785in}{1.627879in}}%
\pgfpathlineto{\pgfqpoint{4.978046in}{1.630169in}}%
\pgfpathlineto{\pgfqpoint{4.985534in}{1.640456in}}%
\pgfpathlineto{\pgfqpoint{4.993017in}{1.650787in}}%
\pgfpathlineto{\pgfqpoint{5.000497in}{1.661159in}}%
\pgfpathlineto{\pgfqpoint{5.007972in}{1.671566in}}%
\pgfpathclose%
\pgfusepath{fill}%
\end{pgfscope}%
\begin{pgfscope}%
\pgfpathrectangle{\pgfqpoint{1.254980in}{0.150000in}}{\pgfqpoint{5.490039in}{5.490039in}}%
\pgfusepath{clip}%
\pgfsetbuttcap%
\pgfsetroundjoin%
\definecolor{currentfill}{rgb}{0.281446,0.084320,0.407414}%
\pgfsetfillcolor{currentfill}%
\pgfsetfillopacity{0.700000}%
\pgfsetlinewidth{0.000000pt}%
\definecolor{currentstroke}{rgb}{0.000000,0.000000,0.000000}%
\pgfsetstrokecolor{currentstroke}%
\pgfsetdash{}{0pt}%
\pgfpathmoveto{\pgfqpoint{3.865315in}{1.685075in}}%
\pgfpathlineto{\pgfqpoint{3.878751in}{1.679222in}}%
\pgfpathlineto{\pgfqpoint{3.892193in}{1.673394in}}%
\pgfpathlineto{\pgfqpoint{3.905640in}{1.667590in}}%
\pgfpathlineto{\pgfqpoint{3.919092in}{1.661812in}}%
\pgfpathlineto{\pgfqpoint{3.911221in}{1.660987in}}%
\pgfpathlineto{\pgfqpoint{3.903341in}{1.660478in}}%
\pgfpathlineto{\pgfqpoint{3.895452in}{1.660294in}}%
\pgfpathlineto{\pgfqpoint{3.887553in}{1.660442in}}%
\pgfpathlineto{\pgfqpoint{3.874077in}{1.666535in}}%
\pgfpathlineto{\pgfqpoint{3.860606in}{1.672653in}}%
\pgfpathlineto{\pgfqpoint{3.847141in}{1.678796in}}%
\pgfpathlineto{\pgfqpoint{3.833680in}{1.684964in}}%
\pgfpathlineto{\pgfqpoint{3.841603in}{1.684495in}}%
\pgfpathlineto{\pgfqpoint{3.849517in}{1.684363in}}%
\pgfpathlineto{\pgfqpoint{3.857421in}{1.684560in}}%
\pgfpathlineto{\pgfqpoint{3.865315in}{1.685075in}}%
\pgfpathclose%
\pgfusepath{fill}%
\end{pgfscope}%
\begin{pgfscope}%
\pgfpathrectangle{\pgfqpoint{1.254980in}{0.150000in}}{\pgfqpoint{5.490039in}{5.490039in}}%
\pgfusepath{clip}%
\pgfsetbuttcap%
\pgfsetroundjoin%
\definecolor{currentfill}{rgb}{0.283187,0.125848,0.444960}%
\pgfsetfillcolor{currentfill}%
\pgfsetfillopacity{0.700000}%
\pgfsetlinewidth{0.000000pt}%
\definecolor{currentstroke}{rgb}{0.000000,0.000000,0.000000}%
\pgfsetstrokecolor{currentstroke}%
\pgfsetdash{}{0pt}%
\pgfpathmoveto{\pgfqpoint{3.672527in}{1.760954in}}%
\pgfpathlineto{\pgfqpoint{3.685930in}{1.754481in}}%
\pgfpathlineto{\pgfqpoint{3.699338in}{1.748033in}}%
\pgfpathlineto{\pgfqpoint{3.712751in}{1.741611in}}%
\pgfpathlineto{\pgfqpoint{3.726168in}{1.735215in}}%
\pgfpathlineto{\pgfqpoint{3.718182in}{1.736675in}}%
\pgfpathlineto{\pgfqpoint{3.710184in}{1.738495in}}%
\pgfpathlineto{\pgfqpoint{3.702174in}{1.740686in}}%
\pgfpathlineto{\pgfqpoint{3.694151in}{1.743256in}}%
\pgfpathlineto{\pgfqpoint{3.680706in}{1.749980in}}%
\pgfpathlineto{\pgfqpoint{3.667266in}{1.756731in}}%
\pgfpathlineto{\pgfqpoint{3.653830in}{1.763508in}}%
\pgfpathlineto{\pgfqpoint{3.640398in}{1.770310in}}%
\pgfpathlineto{\pgfqpoint{3.648449in}{1.767407in}}%
\pgfpathlineto{\pgfqpoint{3.656487in}{1.764885in}}%
\pgfpathlineto{\pgfqpoint{3.664513in}{1.762738in}}%
\pgfpathlineto{\pgfqpoint{3.672527in}{1.760954in}}%
\pgfpathclose%
\pgfusepath{fill}%
\end{pgfscope}%
\begin{pgfscope}%
\pgfpathrectangle{\pgfqpoint{1.254980in}{0.150000in}}{\pgfqpoint{5.490039in}{5.490039in}}%
\pgfusepath{clip}%
\pgfsetbuttcap%
\pgfsetroundjoin%
\definecolor{currentfill}{rgb}{0.272594,0.025563,0.353093}%
\pgfsetfillcolor{currentfill}%
\pgfsetfillopacity{0.700000}%
\pgfsetlinewidth{0.000000pt}%
\definecolor{currentstroke}{rgb}{0.000000,0.000000,0.000000}%
\pgfsetstrokecolor{currentstroke}%
\pgfsetdash{}{0pt}%
\pgfpathmoveto{\pgfqpoint{4.698948in}{1.596897in}}%
\pgfpathlineto{\pgfqpoint{4.712585in}{1.593800in}}%
\pgfpathlineto{\pgfqpoint{4.726229in}{1.590726in}}%
\pgfpathlineto{\pgfqpoint{4.739880in}{1.587675in}}%
\pgfpathlineto{\pgfqpoint{4.753538in}{1.584647in}}%
\pgfpathlineto{\pgfqpoint{4.745992in}{1.575698in}}%
\pgfpathlineto{\pgfqpoint{4.738441in}{1.566857in}}%
\pgfpathlineto{\pgfqpoint{4.730886in}{1.558130in}}%
\pgfpathlineto{\pgfqpoint{4.723327in}{1.549524in}}%
\pgfpathlineto{\pgfqpoint{4.709660in}{1.552788in}}%
\pgfpathlineto{\pgfqpoint{4.695999in}{1.556076in}}%
\pgfpathlineto{\pgfqpoint{4.682345in}{1.559387in}}%
\pgfpathlineto{\pgfqpoint{4.668699in}{1.562721in}}%
\pgfpathlineto{\pgfqpoint{4.676267in}{1.571086in}}%
\pgfpathlineto{\pgfqpoint{4.683832in}{1.579574in}}%
\pgfpathlineto{\pgfqpoint{4.691392in}{1.588180in}}%
\pgfpathlineto{\pgfqpoint{4.698948in}{1.596897in}}%
\pgfpathclose%
\pgfusepath{fill}%
\end{pgfscope}%
\begin{pgfscope}%
\pgfpathrectangle{\pgfqpoint{1.254980in}{0.150000in}}{\pgfqpoint{5.490039in}{5.490039in}}%
\pgfusepath{clip}%
\pgfsetbuttcap%
\pgfsetroundjoin%
\definecolor{currentfill}{rgb}{0.271305,0.019942,0.347269}%
\pgfsetfillcolor{currentfill}%
\pgfsetfillopacity{0.700000}%
\pgfsetlinewidth{0.000000pt}%
\definecolor{currentstroke}{rgb}{0.000000,0.000000,0.000000}%
\pgfsetstrokecolor{currentstroke}%
\pgfsetdash{}{0pt}%
\pgfpathmoveto{\pgfqpoint{4.336008in}{1.582786in}}%
\pgfpathlineto{\pgfqpoint{4.349547in}{1.578474in}}%
\pgfpathlineto{\pgfqpoint{4.363092in}{1.574185in}}%
\pgfpathlineto{\pgfqpoint{4.376644in}{1.569920in}}%
\pgfpathlineto{\pgfqpoint{4.390202in}{1.565678in}}%
\pgfpathlineto{\pgfqpoint{4.382544in}{1.559788in}}%
\pgfpathlineto{\pgfqpoint{4.374881in}{1.554099in}}%
\pgfpathlineto{\pgfqpoint{4.367212in}{1.548618in}}%
\pgfpathlineto{\pgfqpoint{4.359538in}{1.543352in}}%
\pgfpathlineto{\pgfqpoint{4.345965in}{1.547869in}}%
\pgfpathlineto{\pgfqpoint{4.332399in}{1.552408in}}%
\pgfpathlineto{\pgfqpoint{4.318838in}{1.556972in}}%
\pgfpathlineto{\pgfqpoint{4.305284in}{1.561558in}}%
\pgfpathlineto{\pgfqpoint{4.312973in}{1.566545in}}%
\pgfpathlineto{\pgfqpoint{4.320657in}{1.571749in}}%
\pgfpathlineto{\pgfqpoint{4.328335in}{1.577165in}}%
\pgfpathlineto{\pgfqpoint{4.336008in}{1.582786in}}%
\pgfpathclose%
\pgfusepath{fill}%
\end{pgfscope}%
\begin{pgfscope}%
\pgfpathrectangle{\pgfqpoint{1.254980in}{0.150000in}}{\pgfqpoint{5.490039in}{5.490039in}}%
\pgfusepath{clip}%
\pgfsetbuttcap%
\pgfsetroundjoin%
\definecolor{currentfill}{rgb}{0.273809,0.031497,0.358853}%
\pgfsetfillcolor{currentfill}%
\pgfsetfillopacity{0.700000}%
\pgfsetlinewidth{0.000000pt}%
\definecolor{currentstroke}{rgb}{0.000000,0.000000,0.000000}%
\pgfsetstrokecolor{currentstroke}%
\pgfsetdash{}{0pt}%
\pgfpathmoveto{\pgfqpoint{4.197065in}{1.599106in}}%
\pgfpathlineto{\pgfqpoint{4.210571in}{1.594329in}}%
\pgfpathlineto{\pgfqpoint{4.224084in}{1.589576in}}%
\pgfpathlineto{\pgfqpoint{4.237602in}{1.584847in}}%
\pgfpathlineto{\pgfqpoint{4.251126in}{1.580142in}}%
\pgfpathlineto{\pgfqpoint{4.243415in}{1.575665in}}%
\pgfpathlineto{\pgfqpoint{4.235697in}{1.571423in}}%
\pgfpathlineto{\pgfqpoint{4.227973in}{1.567426in}}%
\pgfpathlineto{\pgfqpoint{4.220243in}{1.563681in}}%
\pgfpathlineto{\pgfqpoint{4.206701in}{1.568673in}}%
\pgfpathlineto{\pgfqpoint{4.193165in}{1.573690in}}%
\pgfpathlineto{\pgfqpoint{4.179635in}{1.578730in}}%
\pgfpathlineto{\pgfqpoint{4.166111in}{1.583795in}}%
\pgfpathlineto{\pgfqpoint{4.173859in}{1.587248in}}%
\pgfpathlineto{\pgfqpoint{4.181601in}{1.590956in}}%
\pgfpathlineto{\pgfqpoint{4.189336in}{1.594911in}}%
\pgfpathlineto{\pgfqpoint{4.197065in}{1.599106in}}%
\pgfpathclose%
\pgfusepath{fill}%
\end{pgfscope}%
\begin{pgfscope}%
\pgfpathrectangle{\pgfqpoint{1.254980in}{0.150000in}}{\pgfqpoint{5.490039in}{5.490039in}}%
\pgfusepath{clip}%
\pgfsetbuttcap%
\pgfsetroundjoin%
\definecolor{currentfill}{rgb}{0.216210,0.351535,0.550627}%
\pgfsetfillcolor{currentfill}%
\pgfsetfillopacity{0.700000}%
\pgfsetlinewidth{0.000000pt}%
\definecolor{currentstroke}{rgb}{0.000000,0.000000,0.000000}%
\pgfsetstrokecolor{currentstroke}%
\pgfsetdash{}{0pt}%
\pgfpathmoveto{\pgfqpoint{2.878757in}{2.237633in}}%
\pgfpathlineto{\pgfqpoint{2.892076in}{2.228589in}}%
\pgfpathlineto{\pgfqpoint{2.905398in}{2.219579in}}%
\pgfpathlineto{\pgfqpoint{2.918723in}{2.210603in}}%
\pgfpathlineto{\pgfqpoint{2.932050in}{2.201661in}}%
\pgfpathlineto{\pgfqpoint{2.923401in}{2.212876in}}%
\pgfpathlineto{\pgfqpoint{2.914727in}{2.224617in}}%
\pgfpathlineto{\pgfqpoint{2.906028in}{2.236898in}}%
\pgfpathlineto{\pgfqpoint{2.897303in}{2.249728in}}%
\pgfpathlineto{\pgfqpoint{2.883931in}{2.259050in}}%
\pgfpathlineto{\pgfqpoint{2.870561in}{2.268405in}}%
\pgfpathlineto{\pgfqpoint{2.857194in}{2.277794in}}%
\pgfpathlineto{\pgfqpoint{2.843829in}{2.287218in}}%
\pgfpathlineto{\pgfqpoint{2.852600in}{2.274002in}}%
\pgfpathlineto{\pgfqpoint{2.861345in}{2.261340in}}%
\pgfpathlineto{\pgfqpoint{2.870064in}{2.249221in}}%
\pgfpathlineto{\pgfqpoint{2.878757in}{2.237633in}}%
\pgfpathclose%
\pgfusepath{fill}%
\end{pgfscope}%
\begin{pgfscope}%
\pgfpathrectangle{\pgfqpoint{1.254980in}{0.150000in}}{\pgfqpoint{5.490039in}{5.490039in}}%
\pgfusepath{clip}%
\pgfsetbuttcap%
\pgfsetroundjoin%
\definecolor{currentfill}{rgb}{0.269944,0.014625,0.341379}%
\pgfsetfillcolor{currentfill}%
\pgfsetfillopacity{0.700000}%
\pgfsetlinewidth{0.000000pt}%
\definecolor{currentstroke}{rgb}{0.000000,0.000000,0.000000}%
\pgfsetstrokecolor{currentstroke}%
\pgfsetdash{}{0pt}%
\pgfpathmoveto{\pgfqpoint{4.475024in}{1.575430in}}%
\pgfpathlineto{\pgfqpoint{4.488601in}{1.571567in}}%
\pgfpathlineto{\pgfqpoint{4.502184in}{1.567728in}}%
\pgfpathlineto{\pgfqpoint{4.515774in}{1.563912in}}%
\pgfpathlineto{\pgfqpoint{4.529371in}{1.560120in}}%
\pgfpathlineto{\pgfqpoint{4.521759in}{1.552979in}}%
\pgfpathlineto{\pgfqpoint{4.514142in}{1.546006in}}%
\pgfpathlineto{\pgfqpoint{4.506521in}{1.539207in}}%
\pgfpathlineto{\pgfqpoint{4.498895in}{1.532589in}}%
\pgfpathlineto{\pgfqpoint{4.485286in}{1.536644in}}%
\pgfpathlineto{\pgfqpoint{4.471683in}{1.540721in}}%
\pgfpathlineto{\pgfqpoint{4.458087in}{1.544822in}}%
\pgfpathlineto{\pgfqpoint{4.444497in}{1.548947in}}%
\pgfpathlineto{\pgfqpoint{4.452136in}{1.555298in}}%
\pgfpathlineto{\pgfqpoint{4.459770in}{1.561833in}}%
\pgfpathlineto{\pgfqpoint{4.467400in}{1.568546in}}%
\pgfpathlineto{\pgfqpoint{4.475024in}{1.575430in}}%
\pgfpathclose%
\pgfusepath{fill}%
\end{pgfscope}%
\begin{pgfscope}%
\pgfpathrectangle{\pgfqpoint{1.254980in}{0.150000in}}{\pgfqpoint{5.490039in}{5.490039in}}%
\pgfusepath{clip}%
\pgfsetbuttcap%
\pgfsetroundjoin%
\definecolor{currentfill}{rgb}{0.159194,0.482237,0.558073}%
\pgfsetfillcolor{currentfill}%
\pgfsetfillopacity{0.700000}%
\pgfsetlinewidth{0.000000pt}%
\definecolor{currentstroke}{rgb}{0.000000,0.000000,0.000000}%
\pgfsetstrokecolor{currentstroke}%
\pgfsetdash{}{0pt}%
\pgfpathmoveto{\pgfqpoint{2.470425in}{2.566189in}}%
\pgfpathlineto{\pgfqpoint{2.483739in}{2.555684in}}%
\pgfpathlineto{\pgfqpoint{2.497054in}{2.545222in}}%
\pgfpathlineto{\pgfqpoint{2.510370in}{2.534803in}}%
\pgfpathlineto{\pgfqpoint{2.523688in}{2.524426in}}%
\pgfpathlineto{\pgfqpoint{2.514589in}{2.540566in}}%
\pgfpathlineto{\pgfqpoint{2.505456in}{2.557308in}}%
\pgfpathlineto{\pgfqpoint{2.496290in}{2.574665in}}%
\pgfpathlineto{\pgfqpoint{2.487089in}{2.592648in}}%
\pgfpathlineto{\pgfqpoint{2.473717in}{2.603428in}}%
\pgfpathlineto{\pgfqpoint{2.460346in}{2.614250in}}%
\pgfpathlineto{\pgfqpoint{2.446976in}{2.625115in}}%
\pgfpathlineto{\pgfqpoint{2.433608in}{2.636024in}}%
\pgfpathlineto{\pgfqpoint{2.442864in}{2.617631in}}%
\pgfpathlineto{\pgfqpoint{2.452086in}{2.599869in}}%
\pgfpathlineto{\pgfqpoint{2.461272in}{2.582725in}}%
\pgfpathlineto{\pgfqpoint{2.470425in}{2.566189in}}%
\pgfpathclose%
\pgfusepath{fill}%
\end{pgfscope}%
\begin{pgfscope}%
\pgfpathrectangle{\pgfqpoint{1.254980in}{0.150000in}}{\pgfqpoint{5.490039in}{5.490039in}}%
\pgfusepath{clip}%
\pgfsetbuttcap%
\pgfsetroundjoin%
\definecolor{currentfill}{rgb}{0.277941,0.056324,0.381191}%
\pgfsetfillcolor{currentfill}%
\pgfsetfillopacity{0.700000}%
\pgfsetlinewidth{0.000000pt}%
\definecolor{currentstroke}{rgb}{0.000000,0.000000,0.000000}%
\pgfsetstrokecolor{currentstroke}%
\pgfsetdash{}{0pt}%
\pgfpathmoveto{\pgfqpoint{4.923166in}{1.639565in}}%
\pgfpathlineto{\pgfqpoint{4.936874in}{1.637181in}}%
\pgfpathlineto{\pgfqpoint{4.950590in}{1.634821in}}%
\pgfpathlineto{\pgfqpoint{4.964314in}{1.632484in}}%
\pgfpathlineto{\pgfqpoint{4.978046in}{1.630169in}}%
\pgfpathlineto{\pgfqpoint{4.970554in}{1.619933in}}%
\pgfpathlineto{\pgfqpoint{4.963059in}{1.609752in}}%
\pgfpathlineto{\pgfqpoint{4.955559in}{1.599632in}}%
\pgfpathlineto{\pgfqpoint{4.948056in}{1.589576in}}%
\pgfpathlineto{\pgfqpoint{4.934317in}{1.592102in}}%
\pgfpathlineto{\pgfqpoint{4.920586in}{1.594651in}}%
\pgfpathlineto{\pgfqpoint{4.906862in}{1.597223in}}%
\pgfpathlineto{\pgfqpoint{4.893146in}{1.599818in}}%
\pgfpathlineto{\pgfqpoint{4.900657in}{1.609657in}}%
\pgfpathlineto{\pgfqpoint{4.908164in}{1.619565in}}%
\pgfpathlineto{\pgfqpoint{4.915667in}{1.629536in}}%
\pgfpathlineto{\pgfqpoint{4.923166in}{1.639565in}}%
\pgfpathclose%
\pgfusepath{fill}%
\end{pgfscope}%
\begin{pgfscope}%
\pgfpathrectangle{\pgfqpoint{1.254980in}{0.150000in}}{\pgfqpoint{5.490039in}{5.490039in}}%
\pgfusepath{clip}%
\pgfsetbuttcap%
\pgfsetroundjoin%
\definecolor{currentfill}{rgb}{0.277018,0.050344,0.375715}%
\pgfsetfillcolor{currentfill}%
\pgfsetfillopacity{0.700000}%
\pgfsetlinewidth{0.000000pt}%
\definecolor{currentstroke}{rgb}{0.000000,0.000000,0.000000}%
\pgfsetstrokecolor{currentstroke}%
\pgfsetdash{}{0pt}%
\pgfpathmoveto{\pgfqpoint{4.058121in}{1.625174in}}%
\pgfpathlineto{\pgfqpoint{4.071600in}{1.619917in}}%
\pgfpathlineto{\pgfqpoint{4.085085in}{1.614684in}}%
\pgfpathlineto{\pgfqpoint{4.098575in}{1.609476in}}%
\pgfpathlineto{\pgfqpoint{4.112071in}{1.604291in}}%
\pgfpathlineto{\pgfqpoint{4.104296in}{1.601396in}}%
\pgfpathlineto{\pgfqpoint{4.096514in}{1.598775in}}%
\pgfpathlineto{\pgfqpoint{4.088725in}{1.596435in}}%
\pgfpathlineto{\pgfqpoint{4.080928in}{1.594385in}}%
\pgfpathlineto{\pgfqpoint{4.067412in}{1.599870in}}%
\pgfpathlineto{\pgfqpoint{4.053902in}{1.605379in}}%
\pgfpathlineto{\pgfqpoint{4.040397in}{1.610913in}}%
\pgfpathlineto{\pgfqpoint{4.026897in}{1.616470in}}%
\pgfpathlineto{\pgfqpoint{4.034715in}{1.618215in}}%
\pgfpathlineto{\pgfqpoint{4.042525in}{1.620252in}}%
\pgfpathlineto{\pgfqpoint{4.050327in}{1.622574in}}%
\pgfpathlineto{\pgfqpoint{4.058121in}{1.625174in}}%
\pgfpathclose%
\pgfusepath{fill}%
\end{pgfscope}%
\begin{pgfscope}%
\pgfpathrectangle{\pgfqpoint{1.254980in}{0.150000in}}{\pgfqpoint{5.490039in}{5.490039in}}%
\pgfusepath{clip}%
\pgfsetbuttcap%
\pgfsetroundjoin%
\definecolor{currentfill}{rgb}{0.282290,0.145912,0.461510}%
\pgfsetfillcolor{currentfill}%
\pgfsetfillopacity{0.700000}%
\pgfsetlinewidth{0.000000pt}%
\definecolor{currentstroke}{rgb}{0.000000,0.000000,0.000000}%
\pgfsetstrokecolor{currentstroke}%
\pgfsetdash{}{0pt}%
\pgfpathmoveto{\pgfqpoint{5.402656in}{1.815284in}}%
\pgfpathlineto{\pgfqpoint{5.416531in}{1.814327in}}%
\pgfpathlineto{\pgfqpoint{5.430414in}{1.813392in}}%
\pgfpathlineto{\pgfqpoint{5.444306in}{1.812482in}}%
\pgfpathlineto{\pgfqpoint{5.436942in}{1.801264in}}%
\pgfpathlineto{\pgfqpoint{5.429573in}{1.789999in}}%
\pgfpathlineto{\pgfqpoint{5.422197in}{1.778689in}}%
\pgfpathlineto{\pgfqpoint{5.414816in}{1.767338in}}%
\pgfpathlineto{\pgfqpoint{5.400919in}{1.768397in}}%
\pgfpathlineto{\pgfqpoint{5.387031in}{1.769479in}}%
\pgfpathlineto{\pgfqpoint{5.373151in}{1.770584in}}%
\pgfpathlineto{\pgfqpoint{5.380535in}{1.781820in}}%
\pgfpathlineto{\pgfqpoint{5.387915in}{1.793018in}}%
\pgfpathlineto{\pgfqpoint{5.395288in}{1.804174in}}%
\pgfpathlineto{\pgfqpoint{5.402656in}{1.815284in}}%
\pgfpathclose%
\pgfusepath{fill}%
\end{pgfscope}%
\begin{pgfscope}%
\pgfpathrectangle{\pgfqpoint{1.254980in}{0.150000in}}{\pgfqpoint{5.490039in}{5.490039in}}%
\pgfusepath{clip}%
\pgfsetbuttcap%
\pgfsetroundjoin%
\definecolor{currentfill}{rgb}{0.278826,0.175490,0.483397}%
\pgfsetfillcolor{currentfill}%
\pgfsetfillopacity{0.700000}%
\pgfsetlinewidth{0.000000pt}%
\definecolor{currentstroke}{rgb}{0.000000,0.000000,0.000000}%
\pgfsetstrokecolor{currentstroke}%
\pgfsetdash{}{0pt}%
\pgfpathmoveto{\pgfqpoint{3.479559in}{1.854004in}}%
\pgfpathlineto{\pgfqpoint{3.492938in}{1.846882in}}%
\pgfpathlineto{\pgfqpoint{3.506323in}{1.839787in}}%
\pgfpathlineto{\pgfqpoint{3.519711in}{1.832719in}}%
\pgfpathlineto{\pgfqpoint{3.533103in}{1.825678in}}%
\pgfpathlineto{\pgfqpoint{3.524979in}{1.829649in}}%
\pgfpathlineto{\pgfqpoint{3.516840in}{1.834029in}}%
\pgfpathlineto{\pgfqpoint{3.508686in}{1.838826in}}%
\pgfpathlineto{\pgfqpoint{3.500517in}{1.844050in}}%
\pgfpathlineto{\pgfqpoint{3.487092in}{1.851435in}}%
\pgfpathlineto{\pgfqpoint{3.473672in}{1.858847in}}%
\pgfpathlineto{\pgfqpoint{3.460255in}{1.866286in}}%
\pgfpathlineto{\pgfqpoint{3.446843in}{1.873752in}}%
\pgfpathlineto{\pgfqpoint{3.455045in}{1.868178in}}%
\pgfpathlineto{\pgfqpoint{3.463232in}{1.863035in}}%
\pgfpathlineto{\pgfqpoint{3.471403in}{1.858314in}}%
\pgfpathlineto{\pgfqpoint{3.479559in}{1.854004in}}%
\pgfpathclose%
\pgfusepath{fill}%
\end{pgfscope}%
\begin{pgfscope}%
\pgfpathrectangle{\pgfqpoint{1.254980in}{0.150000in}}{\pgfqpoint{5.490039in}{5.490039in}}%
\pgfusepath{clip}%
\pgfsetbuttcap%
\pgfsetroundjoin%
\definecolor{currentfill}{rgb}{0.260571,0.246922,0.522828}%
\pgfsetfillcolor{currentfill}%
\pgfsetfillopacity{0.700000}%
\pgfsetlinewidth{0.000000pt}%
\definecolor{currentstroke}{rgb}{0.000000,0.000000,0.000000}%
\pgfsetstrokecolor{currentstroke}%
\pgfsetdash{}{0pt}%
\pgfpathmoveto{\pgfqpoint{3.232763in}{1.997023in}}%
\pgfpathlineto{\pgfqpoint{3.246115in}{1.989103in}}%
\pgfpathlineto{\pgfqpoint{3.259471in}{1.981212in}}%
\pgfpathlineto{\pgfqpoint{3.272831in}{1.973351in}}%
\pgfpathlineto{\pgfqpoint{3.286194in}{1.965518in}}%
\pgfpathlineto{\pgfqpoint{3.277870in}{1.972594in}}%
\pgfpathlineto{\pgfqpoint{3.269527in}{1.980133in}}%
\pgfpathlineto{\pgfqpoint{3.261166in}{1.988143in}}%
\pgfpathlineto{\pgfqpoint{3.252785in}{1.996636in}}%
\pgfpathlineto{\pgfqpoint{3.239384in}{2.004828in}}%
\pgfpathlineto{\pgfqpoint{3.225987in}{2.013050in}}%
\pgfpathlineto{\pgfqpoint{3.212593in}{2.021301in}}%
\pgfpathlineto{\pgfqpoint{3.199203in}{2.029582in}}%
\pgfpathlineto{\pgfqpoint{3.207622in}{2.020723in}}%
\pgfpathlineto{\pgfqpoint{3.216022in}{2.012350in}}%
\pgfpathlineto{\pgfqpoint{3.224402in}{2.004453in}}%
\pgfpathlineto{\pgfqpoint{3.232763in}{1.997023in}}%
\pgfpathclose%
\pgfusepath{fill}%
\end{pgfscope}%
\begin{pgfscope}%
\pgfpathrectangle{\pgfqpoint{1.254980in}{0.150000in}}{\pgfqpoint{5.490039in}{5.490039in}}%
\pgfusepath{clip}%
\pgfsetbuttcap%
\pgfsetroundjoin%
\definecolor{currentfill}{rgb}{0.271305,0.019942,0.347269}%
\pgfsetfillcolor{currentfill}%
\pgfsetfillopacity{0.700000}%
\pgfsetlinewidth{0.000000pt}%
\definecolor{currentstroke}{rgb}{0.000000,0.000000,0.000000}%
\pgfsetstrokecolor{currentstroke}%
\pgfsetdash{}{0pt}%
\pgfpathmoveto{\pgfqpoint{4.614182in}{1.576289in}}%
\pgfpathlineto{\pgfqpoint{4.627801in}{1.572862in}}%
\pgfpathlineto{\pgfqpoint{4.641426in}{1.569458in}}%
\pgfpathlineto{\pgfqpoint{4.655059in}{1.566078in}}%
\pgfpathlineto{\pgfqpoint{4.668699in}{1.562721in}}%
\pgfpathlineto{\pgfqpoint{4.661126in}{1.554485in}}%
\pgfpathlineto{\pgfqpoint{4.653550in}{1.546385in}}%
\pgfpathlineto{\pgfqpoint{4.645969in}{1.538427in}}%
\pgfpathlineto{\pgfqpoint{4.638384in}{1.530617in}}%
\pgfpathlineto{\pgfqpoint{4.624734in}{1.534224in}}%
\pgfpathlineto{\pgfqpoint{4.611090in}{1.537854in}}%
\pgfpathlineto{\pgfqpoint{4.597453in}{1.541507in}}%
\pgfpathlineto{\pgfqpoint{4.583823in}{1.545183in}}%
\pgfpathlineto{\pgfqpoint{4.591419in}{1.552739in}}%
\pgfpathlineto{\pgfqpoint{4.599011in}{1.560446in}}%
\pgfpathlineto{\pgfqpoint{4.606598in}{1.568298in}}%
\pgfpathlineto{\pgfqpoint{4.614182in}{1.576289in}}%
\pgfpathclose%
\pgfusepath{fill}%
\end{pgfscope}%
\begin{pgfscope}%
\pgfpathrectangle{\pgfqpoint{1.254980in}{0.150000in}}{\pgfqpoint{5.490039in}{5.490039in}}%
\pgfusepath{clip}%
\pgfsetbuttcap%
\pgfsetroundjoin%
\definecolor{currentfill}{rgb}{0.283072,0.130895,0.449241}%
\pgfsetfillcolor{currentfill}%
\pgfsetfillopacity{0.700000}%
\pgfsetlinewidth{0.000000pt}%
\definecolor{currentstroke}{rgb}{0.000000,0.000000,0.000000}%
\pgfsetstrokecolor{currentstroke}%
\pgfsetdash{}{0pt}%
\pgfpathmoveto{\pgfqpoint{5.317717in}{1.775240in}}%
\pgfpathlineto{\pgfqpoint{5.331563in}{1.774041in}}%
\pgfpathlineto{\pgfqpoint{5.345417in}{1.772866in}}%
\pgfpathlineto{\pgfqpoint{5.359279in}{1.771713in}}%
\pgfpathlineto{\pgfqpoint{5.373151in}{1.770584in}}%
\pgfpathlineto{\pgfqpoint{5.365761in}{1.759314in}}%
\pgfpathlineto{\pgfqpoint{5.358365in}{1.748012in}}%
\pgfpathlineto{\pgfqpoint{5.350965in}{1.736682in}}%
\pgfpathlineto{\pgfqpoint{5.343559in}{1.725329in}}%
\pgfpathlineto{\pgfqpoint{5.329683in}{1.726618in}}%
\pgfpathlineto{\pgfqpoint{5.315815in}{1.727932in}}%
\pgfpathlineto{\pgfqpoint{5.301956in}{1.729268in}}%
\pgfpathlineto{\pgfqpoint{5.288105in}{1.730628in}}%
\pgfpathlineto{\pgfqpoint{5.295515in}{1.741815in}}%
\pgfpathlineto{\pgfqpoint{5.302921in}{1.752983in}}%
\pgfpathlineto{\pgfqpoint{5.310322in}{1.764125in}}%
\pgfpathlineto{\pgfqpoint{5.317717in}{1.775240in}}%
\pgfpathclose%
\pgfusepath{fill}%
\end{pgfscope}%
\begin{pgfscope}%
\pgfpathrectangle{\pgfqpoint{1.254980in}{0.150000in}}{\pgfqpoint{5.490039in}{5.490039in}}%
\pgfusepath{clip}%
\pgfsetbuttcap%
\pgfsetroundjoin%
\definecolor{currentfill}{rgb}{0.274952,0.037752,0.364543}%
\pgfsetfillcolor{currentfill}%
\pgfsetfillopacity{0.700000}%
\pgfsetlinewidth{0.000000pt}%
\definecolor{currentstroke}{rgb}{0.000000,0.000000,0.000000}%
\pgfsetstrokecolor{currentstroke}%
\pgfsetdash{}{0pt}%
\pgfpathmoveto{\pgfqpoint{4.838357in}{1.610431in}}%
\pgfpathlineto{\pgfqpoint{4.852043in}{1.607743in}}%
\pgfpathlineto{\pgfqpoint{4.865737in}{1.605078in}}%
\pgfpathlineto{\pgfqpoint{4.879438in}{1.602437in}}%
\pgfpathlineto{\pgfqpoint{4.893146in}{1.599818in}}%
\pgfpathlineto{\pgfqpoint{4.885632in}{1.590054in}}%
\pgfpathlineto{\pgfqpoint{4.878114in}{1.580368in}}%
\pgfpathlineto{\pgfqpoint{4.870591in}{1.570768in}}%
\pgfpathlineto{\pgfqpoint{4.863066in}{1.561258in}}%
\pgfpathlineto{\pgfqpoint{4.849349in}{1.564101in}}%
\pgfpathlineto{\pgfqpoint{4.835640in}{1.566967in}}%
\pgfpathlineto{\pgfqpoint{4.821938in}{1.569855in}}%
\pgfpathlineto{\pgfqpoint{4.808244in}{1.572767in}}%
\pgfpathlineto{\pgfqpoint{4.815778in}{1.582048in}}%
\pgfpathlineto{\pgfqpoint{4.823308in}{1.591423in}}%
\pgfpathlineto{\pgfqpoint{4.830835in}{1.600885in}}%
\pgfpathlineto{\pgfqpoint{4.838357in}{1.610431in}}%
\pgfpathclose%
\pgfusepath{fill}%
\end{pgfscope}%
\begin{pgfscope}%
\pgfpathrectangle{\pgfqpoint{1.254980in}{0.150000in}}{\pgfqpoint{5.490039in}{5.490039in}}%
\pgfusepath{clip}%
\pgfsetbuttcap%
\pgfsetroundjoin%
\definecolor{currentfill}{rgb}{0.221989,0.339161,0.548752}%
\pgfsetfillcolor{currentfill}%
\pgfsetfillopacity{0.700000}%
\pgfsetlinewidth{0.000000pt}%
\definecolor{currentstroke}{rgb}{0.000000,0.000000,0.000000}%
\pgfsetstrokecolor{currentstroke}%
\pgfsetdash{}{0pt}%
\pgfpathmoveto{\pgfqpoint{2.932050in}{2.201661in}}%
\pgfpathlineto{\pgfqpoint{2.945380in}{2.192753in}}%
\pgfpathlineto{\pgfqpoint{2.958713in}{2.183877in}}%
\pgfpathlineto{\pgfqpoint{2.972048in}{2.175035in}}%
\pgfpathlineto{\pgfqpoint{2.985386in}{2.166226in}}%
\pgfpathlineto{\pgfqpoint{2.976780in}{2.177067in}}%
\pgfpathlineto{\pgfqpoint{2.968151in}{2.188432in}}%
\pgfpathlineto{\pgfqpoint{2.959496in}{2.200332in}}%
\pgfpathlineto{\pgfqpoint{2.950817in}{2.212777in}}%
\pgfpathlineto{\pgfqpoint{2.937435in}{2.221965in}}%
\pgfpathlineto{\pgfqpoint{2.924055in}{2.231186in}}%
\pgfpathlineto{\pgfqpoint{2.910678in}{2.240440in}}%
\pgfpathlineto{\pgfqpoint{2.897303in}{2.249728in}}%
\pgfpathlineto{\pgfqpoint{2.906028in}{2.236898in}}%
\pgfpathlineto{\pgfqpoint{2.914727in}{2.224617in}}%
\pgfpathlineto{\pgfqpoint{2.923401in}{2.212876in}}%
\pgfpathlineto{\pgfqpoint{2.932050in}{2.201661in}}%
\pgfpathclose%
\pgfusepath{fill}%
\end{pgfscope}%
\begin{pgfscope}%
\pgfpathrectangle{\pgfqpoint{1.254980in}{0.150000in}}{\pgfqpoint{5.490039in}{5.490039in}}%
\pgfusepath{clip}%
\pgfsetbuttcap%
\pgfsetroundjoin%
\definecolor{currentfill}{rgb}{0.165117,0.467423,0.558141}%
\pgfsetfillcolor{currentfill}%
\pgfsetfillopacity{0.700000}%
\pgfsetlinewidth{0.000000pt}%
\definecolor{currentstroke}{rgb}{0.000000,0.000000,0.000000}%
\pgfsetstrokecolor{currentstroke}%
\pgfsetdash{}{0pt}%
\pgfpathmoveto{\pgfqpoint{2.523688in}{2.524426in}}%
\pgfpathlineto{\pgfqpoint{2.537007in}{2.514092in}}%
\pgfpathlineto{\pgfqpoint{2.550328in}{2.503799in}}%
\pgfpathlineto{\pgfqpoint{2.563650in}{2.493547in}}%
\pgfpathlineto{\pgfqpoint{2.576974in}{2.483336in}}%
\pgfpathlineto{\pgfqpoint{2.567927in}{2.499080in}}%
\pgfpathlineto{\pgfqpoint{2.558847in}{2.515423in}}%
\pgfpathlineto{\pgfqpoint{2.549735in}{2.532375in}}%
\pgfpathlineto{\pgfqpoint{2.540589in}{2.549950in}}%
\pgfpathlineto{\pgfqpoint{2.527212in}{2.560562in}}%
\pgfpathlineto{\pgfqpoint{2.513836in}{2.571216in}}%
\pgfpathlineto{\pgfqpoint{2.500462in}{2.581911in}}%
\pgfpathlineto{\pgfqpoint{2.487089in}{2.592648in}}%
\pgfpathlineto{\pgfqpoint{2.496290in}{2.574665in}}%
\pgfpathlineto{\pgfqpoint{2.505456in}{2.557308in}}%
\pgfpathlineto{\pgfqpoint{2.514589in}{2.540566in}}%
\pgfpathlineto{\pgfqpoint{2.523688in}{2.524426in}}%
\pgfpathclose%
\pgfusepath{fill}%
\end{pgfscope}%
\begin{pgfscope}%
\pgfpathrectangle{\pgfqpoint{1.254980in}{0.150000in}}{\pgfqpoint{5.490039in}{5.490039in}}%
\pgfusepath{clip}%
\pgfsetbuttcap%
\pgfsetroundjoin%
\definecolor{currentfill}{rgb}{0.283091,0.110553,0.431554}%
\pgfsetfillcolor{currentfill}%
\pgfsetfillopacity{0.700000}%
\pgfsetlinewidth{0.000000pt}%
\definecolor{currentstroke}{rgb}{0.000000,0.000000,0.000000}%
\pgfsetstrokecolor{currentstroke}%
\pgfsetdash{}{0pt}%
\pgfpathmoveto{\pgfqpoint{5.232785in}{1.736301in}}%
\pgfpathlineto{\pgfqpoint{5.246602in}{1.734847in}}%
\pgfpathlineto{\pgfqpoint{5.260428in}{1.733418in}}%
\pgfpathlineto{\pgfqpoint{5.274262in}{1.732011in}}%
\pgfpathlineto{\pgfqpoint{5.288105in}{1.730628in}}%
\pgfpathlineto{\pgfqpoint{5.280689in}{1.719424in}}%
\pgfpathlineto{\pgfqpoint{5.273268in}{1.708207in}}%
\pgfpathlineto{\pgfqpoint{5.265843in}{1.696982in}}%
\pgfpathlineto{\pgfqpoint{5.258414in}{1.685753in}}%
\pgfpathlineto{\pgfqpoint{5.244566in}{1.687310in}}%
\pgfpathlineto{\pgfqpoint{5.230726in}{1.688890in}}%
\pgfpathlineto{\pgfqpoint{5.216895in}{1.690494in}}%
\pgfpathlineto{\pgfqpoint{5.203072in}{1.692121in}}%
\pgfpathlineto{\pgfqpoint{5.210507in}{1.703171in}}%
\pgfpathlineto{\pgfqpoint{5.217938in}{1.714221in}}%
\pgfpathlineto{\pgfqpoint{5.225363in}{1.725265in}}%
\pgfpathlineto{\pgfqpoint{5.232785in}{1.736301in}}%
\pgfpathclose%
\pgfusepath{fill}%
\end{pgfscope}%
\begin{pgfscope}%
\pgfpathrectangle{\pgfqpoint{1.254980in}{0.150000in}}{\pgfqpoint{5.490039in}{5.490039in}}%
\pgfusepath{clip}%
\pgfsetbuttcap%
\pgfsetroundjoin%
\definecolor{currentfill}{rgb}{0.283197,0.115680,0.436115}%
\pgfsetfillcolor{currentfill}%
\pgfsetfillopacity{0.700000}%
\pgfsetlinewidth{0.000000pt}%
\definecolor{currentstroke}{rgb}{0.000000,0.000000,0.000000}%
\pgfsetstrokecolor{currentstroke}%
\pgfsetdash{}{0pt}%
\pgfpathmoveto{\pgfqpoint{3.726168in}{1.735215in}}%
\pgfpathlineto{\pgfqpoint{3.739590in}{1.728844in}}%
\pgfpathlineto{\pgfqpoint{3.753017in}{1.722499in}}%
\pgfpathlineto{\pgfqpoint{3.766449in}{1.716180in}}%
\pgfpathlineto{\pgfqpoint{3.779885in}{1.709886in}}%
\pgfpathlineto{\pgfqpoint{3.771925in}{1.711023in}}%
\pgfpathlineto{\pgfqpoint{3.763954in}{1.712517in}}%
\pgfpathlineto{\pgfqpoint{3.755972in}{1.714377in}}%
\pgfpathlineto{\pgfqpoint{3.747978in}{1.716612in}}%
\pgfpathlineto{\pgfqpoint{3.734514in}{1.723235in}}%
\pgfpathlineto{\pgfqpoint{3.721055in}{1.729883in}}%
\pgfpathlineto{\pgfqpoint{3.707601in}{1.736556in}}%
\pgfpathlineto{\pgfqpoint{3.694151in}{1.743256in}}%
\pgfpathlineto{\pgfqpoint{3.702174in}{1.740686in}}%
\pgfpathlineto{\pgfqpoint{3.710184in}{1.738495in}}%
\pgfpathlineto{\pgfqpoint{3.718182in}{1.736675in}}%
\pgfpathlineto{\pgfqpoint{3.726168in}{1.735215in}}%
\pgfpathclose%
\pgfusepath{fill}%
\end{pgfscope}%
\begin{pgfscope}%
\pgfpathrectangle{\pgfqpoint{1.254980in}{0.150000in}}{\pgfqpoint{5.490039in}{5.490039in}}%
\pgfusepath{clip}%
\pgfsetbuttcap%
\pgfsetroundjoin%
\definecolor{currentfill}{rgb}{0.280894,0.078907,0.402329}%
\pgfsetfillcolor{currentfill}%
\pgfsetfillopacity{0.700000}%
\pgfsetlinewidth{0.000000pt}%
\definecolor{currentstroke}{rgb}{0.000000,0.000000,0.000000}%
\pgfsetstrokecolor{currentstroke}%
\pgfsetdash{}{0pt}%
\pgfpathmoveto{\pgfqpoint{3.919092in}{1.661812in}}%
\pgfpathlineto{\pgfqpoint{3.932549in}{1.656058in}}%
\pgfpathlineto{\pgfqpoint{3.946012in}{1.650329in}}%
\pgfpathlineto{\pgfqpoint{3.959479in}{1.644625in}}%
\pgfpathlineto{\pgfqpoint{3.972952in}{1.638945in}}%
\pgfpathlineto{\pgfqpoint{3.965104in}{1.637811in}}%
\pgfpathlineto{\pgfqpoint{3.957247in}{1.636989in}}%
\pgfpathlineto{\pgfqpoint{3.949381in}{1.636489in}}%
\pgfpathlineto{\pgfqpoint{3.941506in}{1.636318in}}%
\pgfpathlineto{\pgfqpoint{3.928010in}{1.642312in}}%
\pgfpathlineto{\pgfqpoint{3.914519in}{1.648331in}}%
\pgfpathlineto{\pgfqpoint{3.901034in}{1.654374in}}%
\pgfpathlineto{\pgfqpoint{3.887553in}{1.660442in}}%
\pgfpathlineto{\pgfqpoint{3.895452in}{1.660294in}}%
\pgfpathlineto{\pgfqpoint{3.903341in}{1.660478in}}%
\pgfpathlineto{\pgfqpoint{3.911221in}{1.660987in}}%
\pgfpathlineto{\pgfqpoint{3.919092in}{1.661812in}}%
\pgfpathclose%
\pgfusepath{fill}%
\end{pgfscope}%
\begin{pgfscope}%
\pgfpathrectangle{\pgfqpoint{1.254980in}{0.150000in}}{\pgfqpoint{5.490039in}{5.490039in}}%
\pgfusepath{clip}%
\pgfsetbuttcap%
\pgfsetroundjoin%
\definecolor{currentfill}{rgb}{0.282327,0.094955,0.417331}%
\pgfsetfillcolor{currentfill}%
\pgfsetfillopacity{0.700000}%
\pgfsetlinewidth{0.000000pt}%
\definecolor{currentstroke}{rgb}{0.000000,0.000000,0.000000}%
\pgfsetstrokecolor{currentstroke}%
\pgfsetdash{}{0pt}%
\pgfpathmoveto{\pgfqpoint{5.147862in}{1.698861in}}%
\pgfpathlineto{\pgfqpoint{5.161652in}{1.697141in}}%
\pgfpathlineto{\pgfqpoint{5.175451in}{1.695444in}}%
\pgfpathlineto{\pgfqpoint{5.189257in}{1.693771in}}%
\pgfpathlineto{\pgfqpoint{5.203072in}{1.692121in}}%
\pgfpathlineto{\pgfqpoint{5.195632in}{1.681073in}}%
\pgfpathlineto{\pgfqpoint{5.188188in}{1.670033in}}%
\pgfpathlineto{\pgfqpoint{5.180740in}{1.659005in}}%
\pgfpathlineto{\pgfqpoint{5.173287in}{1.647994in}}%
\pgfpathlineto{\pgfqpoint{5.159467in}{1.649831in}}%
\pgfpathlineto{\pgfqpoint{5.145654in}{1.651691in}}%
\pgfpathlineto{\pgfqpoint{5.131850in}{1.653574in}}%
\pgfpathlineto{\pgfqpoint{5.118054in}{1.655480in}}%
\pgfpathlineto{\pgfqpoint{5.125512in}{1.666300in}}%
\pgfpathlineto{\pgfqpoint{5.132967in}{1.677140in}}%
\pgfpathlineto{\pgfqpoint{5.140416in}{1.687995in}}%
\pgfpathlineto{\pgfqpoint{5.147862in}{1.698861in}}%
\pgfpathclose%
\pgfusepath{fill}%
\end{pgfscope}%
\begin{pgfscope}%
\pgfpathrectangle{\pgfqpoint{1.254980in}{0.150000in}}{\pgfqpoint{5.490039in}{5.490039in}}%
\pgfusepath{clip}%
\pgfsetbuttcap%
\pgfsetroundjoin%
\definecolor{currentfill}{rgb}{0.273809,0.031497,0.358853}%
\pgfsetfillcolor{currentfill}%
\pgfsetfillopacity{0.700000}%
\pgfsetlinewidth{0.000000pt}%
\definecolor{currentstroke}{rgb}{0.000000,0.000000,0.000000}%
\pgfsetstrokecolor{currentstroke}%
\pgfsetdash{}{0pt}%
\pgfpathmoveto{\pgfqpoint{4.251126in}{1.580142in}}%
\pgfpathlineto{\pgfqpoint{4.264657in}{1.575461in}}%
\pgfpathlineto{\pgfqpoint{4.278193in}{1.570803in}}%
\pgfpathlineto{\pgfqpoint{4.291735in}{1.566169in}}%
\pgfpathlineto{\pgfqpoint{4.305284in}{1.561558in}}%
\pgfpathlineto{\pgfqpoint{4.297588in}{1.556798in}}%
\pgfpathlineto{\pgfqpoint{4.289887in}{1.552271in}}%
\pgfpathlineto{\pgfqpoint{4.282181in}{1.547984in}}%
\pgfpathlineto{\pgfqpoint{4.274468in}{1.543946in}}%
\pgfpathlineto{\pgfqpoint{4.260902in}{1.548844in}}%
\pgfpathlineto{\pgfqpoint{4.247343in}{1.553766in}}%
\pgfpathlineto{\pgfqpoint{4.233790in}{1.558711in}}%
\pgfpathlineto{\pgfqpoint{4.220243in}{1.563681in}}%
\pgfpathlineto{\pgfqpoint{4.227973in}{1.567426in}}%
\pgfpathlineto{\pgfqpoint{4.235697in}{1.571423in}}%
\pgfpathlineto{\pgfqpoint{4.243415in}{1.575665in}}%
\pgfpathlineto{\pgfqpoint{4.251126in}{1.580142in}}%
\pgfpathclose%
\pgfusepath{fill}%
\end{pgfscope}%
\begin{pgfscope}%
\pgfpathrectangle{\pgfqpoint{1.254980in}{0.150000in}}{\pgfqpoint{5.490039in}{5.490039in}}%
\pgfusepath{clip}%
\pgfsetbuttcap%
\pgfsetroundjoin%
\definecolor{currentfill}{rgb}{0.271305,0.019942,0.347269}%
\pgfsetfillcolor{currentfill}%
\pgfsetfillopacity{0.700000}%
\pgfsetlinewidth{0.000000pt}%
\definecolor{currentstroke}{rgb}{0.000000,0.000000,0.000000}%
\pgfsetstrokecolor{currentstroke}%
\pgfsetdash{}{0pt}%
\pgfpathmoveto{\pgfqpoint{4.390202in}{1.565678in}}%
\pgfpathlineto{\pgfqpoint{4.403766in}{1.561460in}}%
\pgfpathlineto{\pgfqpoint{4.417337in}{1.557266in}}%
\pgfpathlineto{\pgfqpoint{4.430914in}{1.553095in}}%
\pgfpathlineto{\pgfqpoint{4.444497in}{1.548947in}}%
\pgfpathlineto{\pgfqpoint{4.436853in}{1.542787in}}%
\pgfpathlineto{\pgfqpoint{4.429204in}{1.536824in}}%
\pgfpathlineto{\pgfqpoint{4.421550in}{1.531067in}}%
\pgfpathlineto{\pgfqpoint{4.413891in}{1.525522in}}%
\pgfpathlineto{\pgfqpoint{4.400294in}{1.529944in}}%
\pgfpathlineto{\pgfqpoint{4.386702in}{1.534390in}}%
\pgfpathlineto{\pgfqpoint{4.373117in}{1.538860in}}%
\pgfpathlineto{\pgfqpoint{4.359538in}{1.543352in}}%
\pgfpathlineto{\pgfqpoint{4.367212in}{1.548618in}}%
\pgfpathlineto{\pgfqpoint{4.374881in}{1.554099in}}%
\pgfpathlineto{\pgfqpoint{4.382544in}{1.559788in}}%
\pgfpathlineto{\pgfqpoint{4.390202in}{1.565678in}}%
\pgfpathclose%
\pgfusepath{fill}%
\end{pgfscope}%
\begin{pgfscope}%
\pgfpathrectangle{\pgfqpoint{1.254980in}{0.150000in}}{\pgfqpoint{5.490039in}{5.490039in}}%
\pgfusepath{clip}%
\pgfsetbuttcap%
\pgfsetroundjoin%
\definecolor{currentfill}{rgb}{0.273809,0.031497,0.358853}%
\pgfsetfillcolor{currentfill}%
\pgfsetfillopacity{0.700000}%
\pgfsetlinewidth{0.000000pt}%
\definecolor{currentstroke}{rgb}{0.000000,0.000000,0.000000}%
\pgfsetstrokecolor{currentstroke}%
\pgfsetdash{}{0pt}%
\pgfpathmoveto{\pgfqpoint{4.753538in}{1.584647in}}%
\pgfpathlineto{\pgfqpoint{4.767204in}{1.581642in}}%
\pgfpathlineto{\pgfqpoint{4.780877in}{1.578661in}}%
\pgfpathlineto{\pgfqpoint{4.794556in}{1.575703in}}%
\pgfpathlineto{\pgfqpoint{4.808244in}{1.572767in}}%
\pgfpathlineto{\pgfqpoint{4.800706in}{1.563586in}}%
\pgfpathlineto{\pgfqpoint{4.793164in}{1.554510in}}%
\pgfpathlineto{\pgfqpoint{4.785619in}{1.545545in}}%
\pgfpathlineto{\pgfqpoint{4.778069in}{1.536696in}}%
\pgfpathlineto{\pgfqpoint{4.764373in}{1.539869in}}%
\pgfpathlineto{\pgfqpoint{4.750684in}{1.543064in}}%
\pgfpathlineto{\pgfqpoint{4.737002in}{1.546282in}}%
\pgfpathlineto{\pgfqpoint{4.723327in}{1.549524in}}%
\pgfpathlineto{\pgfqpoint{4.730886in}{1.558130in}}%
\pgfpathlineto{\pgfqpoint{4.738441in}{1.566857in}}%
\pgfpathlineto{\pgfqpoint{4.745992in}{1.575698in}}%
\pgfpathlineto{\pgfqpoint{4.753538in}{1.584647in}}%
\pgfpathclose%
\pgfusepath{fill}%
\end{pgfscope}%
\begin{pgfscope}%
\pgfpathrectangle{\pgfqpoint{1.254980in}{0.150000in}}{\pgfqpoint{5.490039in}{5.490039in}}%
\pgfusepath{clip}%
\pgfsetbuttcap%
\pgfsetroundjoin%
\definecolor{currentfill}{rgb}{0.280255,0.165693,0.476498}%
\pgfsetfillcolor{currentfill}%
\pgfsetfillopacity{0.700000}%
\pgfsetlinewidth{0.000000pt}%
\definecolor{currentstroke}{rgb}{0.000000,0.000000,0.000000}%
\pgfsetstrokecolor{currentstroke}%
\pgfsetdash{}{0pt}%
\pgfpathmoveto{\pgfqpoint{3.533103in}{1.825678in}}%
\pgfpathlineto{\pgfqpoint{3.546500in}{1.818664in}}%
\pgfpathlineto{\pgfqpoint{3.559901in}{1.811677in}}%
\pgfpathlineto{\pgfqpoint{3.573306in}{1.804716in}}%
\pgfpathlineto{\pgfqpoint{3.586716in}{1.797782in}}%
\pgfpathlineto{\pgfqpoint{3.578622in}{1.801415in}}%
\pgfpathlineto{\pgfqpoint{3.570514in}{1.805453in}}%
\pgfpathlineto{\pgfqpoint{3.562392in}{1.809905in}}%
\pgfpathlineto{\pgfqpoint{3.554255in}{1.814780in}}%
\pgfpathlineto{\pgfqpoint{3.540815in}{1.822058in}}%
\pgfpathlineto{\pgfqpoint{3.527378in}{1.829362in}}%
\pgfpathlineto{\pgfqpoint{3.513945in}{1.836693in}}%
\pgfpathlineto{\pgfqpoint{3.500517in}{1.844050in}}%
\pgfpathlineto{\pgfqpoint{3.508686in}{1.838826in}}%
\pgfpathlineto{\pgfqpoint{3.516840in}{1.834029in}}%
\pgfpathlineto{\pgfqpoint{3.524979in}{1.829649in}}%
\pgfpathlineto{\pgfqpoint{3.533103in}{1.825678in}}%
\pgfpathclose%
\pgfusepath{fill}%
\end{pgfscope}%
\begin{pgfscope}%
\pgfpathrectangle{\pgfqpoint{1.254980in}{0.150000in}}{\pgfqpoint{5.490039in}{5.490039in}}%
\pgfusepath{clip}%
\pgfsetbuttcap%
\pgfsetroundjoin%
\definecolor{currentfill}{rgb}{0.169646,0.456262,0.558030}%
\pgfsetfillcolor{currentfill}%
\pgfsetfillopacity{0.700000}%
\pgfsetlinewidth{0.000000pt}%
\definecolor{currentstroke}{rgb}{0.000000,0.000000,0.000000}%
\pgfsetstrokecolor{currentstroke}%
\pgfsetdash{}{0pt}%
\pgfpathmoveto{\pgfqpoint{2.576974in}{2.483336in}}%
\pgfpathlineto{\pgfqpoint{2.590299in}{2.473166in}}%
\pgfpathlineto{\pgfqpoint{2.603626in}{2.463036in}}%
\pgfpathlineto{\pgfqpoint{2.616954in}{2.452946in}}%
\pgfpathlineto{\pgfqpoint{2.630284in}{2.442896in}}%
\pgfpathlineto{\pgfqpoint{2.621288in}{2.458246in}}%
\pgfpathlineto{\pgfqpoint{2.612261in}{2.474189in}}%
\pgfpathlineto{\pgfqpoint{2.603202in}{2.490739in}}%
\pgfpathlineto{\pgfqpoint{2.594110in}{2.507906in}}%
\pgfpathlineto{\pgfqpoint{2.580728in}{2.518357in}}%
\pgfpathlineto{\pgfqpoint{2.567347in}{2.528847in}}%
\pgfpathlineto{\pgfqpoint{2.553967in}{2.539378in}}%
\pgfpathlineto{\pgfqpoint{2.540589in}{2.549950in}}%
\pgfpathlineto{\pgfqpoint{2.549735in}{2.532375in}}%
\pgfpathlineto{\pgfqpoint{2.558847in}{2.515423in}}%
\pgfpathlineto{\pgfqpoint{2.567927in}{2.499080in}}%
\pgfpathlineto{\pgfqpoint{2.576974in}{2.483336in}}%
\pgfpathclose%
\pgfusepath{fill}%
\end{pgfscope}%
\begin{pgfscope}%
\pgfpathrectangle{\pgfqpoint{1.254980in}{0.150000in}}{\pgfqpoint{5.490039in}{5.490039in}}%
\pgfusepath{clip}%
\pgfsetbuttcap%
\pgfsetroundjoin%
\definecolor{currentfill}{rgb}{0.280267,0.073417,0.397163}%
\pgfsetfillcolor{currentfill}%
\pgfsetfillopacity{0.700000}%
\pgfsetlinewidth{0.000000pt}%
\definecolor{currentstroke}{rgb}{0.000000,0.000000,0.000000}%
\pgfsetstrokecolor{currentstroke}%
\pgfsetdash{}{0pt}%
\pgfpathmoveto{\pgfqpoint{5.062950in}{1.663337in}}%
\pgfpathlineto{\pgfqpoint{5.076714in}{1.661338in}}%
\pgfpathlineto{\pgfqpoint{5.090486in}{1.659362in}}%
\pgfpathlineto{\pgfqpoint{5.104266in}{1.657409in}}%
\pgfpathlineto{\pgfqpoint{5.118054in}{1.655480in}}%
\pgfpathlineto{\pgfqpoint{5.110591in}{1.644684in}}%
\pgfpathlineto{\pgfqpoint{5.103125in}{1.633918in}}%
\pgfpathlineto{\pgfqpoint{5.095654in}{1.623185in}}%
\pgfpathlineto{\pgfqpoint{5.088179in}{1.612490in}}%
\pgfpathlineto{\pgfqpoint{5.074385in}{1.614619in}}%
\pgfpathlineto{\pgfqpoint{5.060599in}{1.616771in}}%
\pgfpathlineto{\pgfqpoint{5.046821in}{1.618947in}}%
\pgfpathlineto{\pgfqpoint{5.033050in}{1.621145in}}%
\pgfpathlineto{\pgfqpoint{5.040531in}{1.631635in}}%
\pgfpathlineto{\pgfqpoint{5.048008in}{1.642167in}}%
\pgfpathlineto{\pgfqpoint{5.055481in}{1.652736in}}%
\pgfpathlineto{\pgfqpoint{5.062950in}{1.663337in}}%
\pgfpathclose%
\pgfusepath{fill}%
\end{pgfscope}%
\begin{pgfscope}%
\pgfpathrectangle{\pgfqpoint{1.254980in}{0.150000in}}{\pgfqpoint{5.490039in}{5.490039in}}%
\pgfusepath{clip}%
\pgfsetbuttcap%
\pgfsetroundjoin%
\definecolor{currentfill}{rgb}{0.263663,0.237631,0.518762}%
\pgfsetfillcolor{currentfill}%
\pgfsetfillopacity{0.700000}%
\pgfsetlinewidth{0.000000pt}%
\definecolor{currentstroke}{rgb}{0.000000,0.000000,0.000000}%
\pgfsetstrokecolor{currentstroke}%
\pgfsetdash{}{0pt}%
\pgfpathmoveto{\pgfqpoint{3.286194in}{1.965518in}}%
\pgfpathlineto{\pgfqpoint{3.299561in}{1.957715in}}%
\pgfpathlineto{\pgfqpoint{3.312931in}{1.949941in}}%
\pgfpathlineto{\pgfqpoint{3.326305in}{1.942195in}}%
\pgfpathlineto{\pgfqpoint{3.339683in}{1.934478in}}%
\pgfpathlineto{\pgfqpoint{3.331395in}{1.941200in}}%
\pgfpathlineto{\pgfqpoint{3.323089in}{1.948380in}}%
\pgfpathlineto{\pgfqpoint{3.314765in}{1.956028in}}%
\pgfpathlineto{\pgfqpoint{3.306422in}{1.964156in}}%
\pgfpathlineto{\pgfqpoint{3.293007in}{1.972233in}}%
\pgfpathlineto{\pgfqpoint{3.279596in}{1.980338in}}%
\pgfpathlineto{\pgfqpoint{3.266189in}{1.988473in}}%
\pgfpathlineto{\pgfqpoint{3.252785in}{1.996636in}}%
\pgfpathlineto{\pgfqpoint{3.261166in}{1.988143in}}%
\pgfpathlineto{\pgfqpoint{3.269527in}{1.980133in}}%
\pgfpathlineto{\pgfqpoint{3.277870in}{1.972594in}}%
\pgfpathlineto{\pgfqpoint{3.286194in}{1.965518in}}%
\pgfpathclose%
\pgfusepath{fill}%
\end{pgfscope}%
\begin{pgfscope}%
\pgfpathrectangle{\pgfqpoint{1.254980in}{0.150000in}}{\pgfqpoint{5.490039in}{5.490039in}}%
\pgfusepath{clip}%
\pgfsetbuttcap%
\pgfsetroundjoin%
\definecolor{currentfill}{rgb}{0.225863,0.330805,0.547314}%
\pgfsetfillcolor{currentfill}%
\pgfsetfillopacity{0.700000}%
\pgfsetlinewidth{0.000000pt}%
\definecolor{currentstroke}{rgb}{0.000000,0.000000,0.000000}%
\pgfsetstrokecolor{currentstroke}%
\pgfsetdash{}{0pt}%
\pgfpathmoveto{\pgfqpoint{2.985386in}{2.166226in}}%
\pgfpathlineto{\pgfqpoint{2.998727in}{2.157449in}}%
\pgfpathlineto{\pgfqpoint{3.012071in}{2.148705in}}%
\pgfpathlineto{\pgfqpoint{3.025418in}{2.139993in}}%
\pgfpathlineto{\pgfqpoint{3.038767in}{2.131313in}}%
\pgfpathlineto{\pgfqpoint{3.030204in}{2.141782in}}%
\pgfpathlineto{\pgfqpoint{3.021618in}{2.152771in}}%
\pgfpathlineto{\pgfqpoint{3.013008in}{2.164290in}}%
\pgfpathlineto{\pgfqpoint{3.004373in}{2.176351in}}%
\pgfpathlineto{\pgfqpoint{2.990980in}{2.185409in}}%
\pgfpathlineto{\pgfqpoint{2.977590in}{2.194499in}}%
\pgfpathlineto{\pgfqpoint{2.964202in}{2.203622in}}%
\pgfpathlineto{\pgfqpoint{2.950817in}{2.212777in}}%
\pgfpathlineto{\pgfqpoint{2.959496in}{2.200332in}}%
\pgfpathlineto{\pgfqpoint{2.968151in}{2.188432in}}%
\pgfpathlineto{\pgfqpoint{2.976780in}{2.177067in}}%
\pgfpathlineto{\pgfqpoint{2.985386in}{2.166226in}}%
\pgfpathclose%
\pgfusepath{fill}%
\end{pgfscope}%
\begin{pgfscope}%
\pgfpathrectangle{\pgfqpoint{1.254980in}{0.150000in}}{\pgfqpoint{5.490039in}{5.490039in}}%
\pgfusepath{clip}%
\pgfsetbuttcap%
\pgfsetroundjoin%
\definecolor{currentfill}{rgb}{0.277018,0.050344,0.375715}%
\pgfsetfillcolor{currentfill}%
\pgfsetfillopacity{0.700000}%
\pgfsetlinewidth{0.000000pt}%
\definecolor{currentstroke}{rgb}{0.000000,0.000000,0.000000}%
\pgfsetstrokecolor{currentstroke}%
\pgfsetdash{}{0pt}%
\pgfpathmoveto{\pgfqpoint{4.112071in}{1.604291in}}%
\pgfpathlineto{\pgfqpoint{4.125572in}{1.599131in}}%
\pgfpathlineto{\pgfqpoint{4.139079in}{1.593995in}}%
\pgfpathlineto{\pgfqpoint{4.152592in}{1.588883in}}%
\pgfpathlineto{\pgfqpoint{4.166111in}{1.583795in}}%
\pgfpathlineto{\pgfqpoint{4.158355in}{1.580604in}}%
\pgfpathlineto{\pgfqpoint{4.150593in}{1.577683in}}%
\pgfpathlineto{\pgfqpoint{4.142824in}{1.575041in}}%
\pgfpathlineto{\pgfqpoint{4.135047in}{1.572685in}}%
\pgfpathlineto{\pgfqpoint{4.121509in}{1.578074in}}%
\pgfpathlineto{\pgfqpoint{4.107977in}{1.583487in}}%
\pgfpathlineto{\pgfqpoint{4.094450in}{1.588924in}}%
\pgfpathlineto{\pgfqpoint{4.080928in}{1.594385in}}%
\pgfpathlineto{\pgfqpoint{4.088725in}{1.596435in}}%
\pgfpathlineto{\pgfqpoint{4.096514in}{1.598775in}}%
\pgfpathlineto{\pgfqpoint{4.104296in}{1.601396in}}%
\pgfpathlineto{\pgfqpoint{4.112071in}{1.604291in}}%
\pgfpathclose%
\pgfusepath{fill}%
\end{pgfscope}%
\begin{pgfscope}%
\pgfpathrectangle{\pgfqpoint{1.254980in}{0.150000in}}{\pgfqpoint{5.490039in}{5.490039in}}%
\pgfusepath{clip}%
\pgfsetbuttcap%
\pgfsetroundjoin%
\definecolor{currentfill}{rgb}{0.271305,0.019942,0.347269}%
\pgfsetfillcolor{currentfill}%
\pgfsetfillopacity{0.700000}%
\pgfsetlinewidth{0.000000pt}%
\definecolor{currentstroke}{rgb}{0.000000,0.000000,0.000000}%
\pgfsetstrokecolor{currentstroke}%
\pgfsetdash{}{0pt}%
\pgfpathmoveto{\pgfqpoint{4.529371in}{1.560120in}}%
\pgfpathlineto{\pgfqpoint{4.542974in}{1.556351in}}%
\pgfpathlineto{\pgfqpoint{4.556584in}{1.552605in}}%
\pgfpathlineto{\pgfqpoint{4.570200in}{1.548882in}}%
\pgfpathlineto{\pgfqpoint{4.583823in}{1.545183in}}%
\pgfpathlineto{\pgfqpoint{4.576223in}{1.537785in}}%
\pgfpathlineto{\pgfqpoint{4.568619in}{1.530551in}}%
\pgfpathlineto{\pgfqpoint{4.561010in}{1.523489in}}%
\pgfpathlineto{\pgfqpoint{4.553397in}{1.516604in}}%
\pgfpathlineto{\pgfqpoint{4.539762in}{1.520565in}}%
\pgfpathlineto{\pgfqpoint{4.526133in}{1.524550in}}%
\pgfpathlineto{\pgfqpoint{4.512511in}{1.528558in}}%
\pgfpathlineto{\pgfqpoint{4.498895in}{1.532589in}}%
\pgfpathlineto{\pgfqpoint{4.506521in}{1.539207in}}%
\pgfpathlineto{\pgfqpoint{4.514142in}{1.546006in}}%
\pgfpathlineto{\pgfqpoint{4.521759in}{1.552979in}}%
\pgfpathlineto{\pgfqpoint{4.529371in}{1.560120in}}%
\pgfpathclose%
\pgfusepath{fill}%
\end{pgfscope}%
\begin{pgfscope}%
\pgfpathrectangle{\pgfqpoint{1.254980in}{0.150000in}}{\pgfqpoint{5.490039in}{5.490039in}}%
\pgfusepath{clip}%
\pgfsetbuttcap%
\pgfsetroundjoin%
\definecolor{currentfill}{rgb}{0.278791,0.062145,0.386592}%
\pgfsetfillcolor{currentfill}%
\pgfsetfillopacity{0.700000}%
\pgfsetlinewidth{0.000000pt}%
\definecolor{currentstroke}{rgb}{0.000000,0.000000,0.000000}%
\pgfsetstrokecolor{currentstroke}%
\pgfsetdash{}{0pt}%
\pgfpathmoveto{\pgfqpoint{4.978046in}{1.630169in}}%
\pgfpathlineto{\pgfqpoint{4.991785in}{1.627879in}}%
\pgfpathlineto{\pgfqpoint{5.005532in}{1.625611in}}%
\pgfpathlineto{\pgfqpoint{5.019287in}{1.623366in}}%
\pgfpathlineto{\pgfqpoint{5.033050in}{1.621145in}}%
\pgfpathlineto{\pgfqpoint{5.025565in}{1.610702in}}%
\pgfpathlineto{\pgfqpoint{5.018076in}{1.600310in}}%
\pgfpathlineto{\pgfqpoint{5.010584in}{1.589976in}}%
\pgfpathlineto{\pgfqpoint{5.003088in}{1.579704in}}%
\pgfpathlineto{\pgfqpoint{4.989318in}{1.582137in}}%
\pgfpathlineto{\pgfqpoint{4.975556in}{1.584594in}}%
\pgfpathlineto{\pgfqpoint{4.961802in}{1.587074in}}%
\pgfpathlineto{\pgfqpoint{4.948056in}{1.589576in}}%
\pgfpathlineto{\pgfqpoint{4.955559in}{1.599632in}}%
\pgfpathlineto{\pgfqpoint{4.963059in}{1.609752in}}%
\pgfpathlineto{\pgfqpoint{4.970554in}{1.619933in}}%
\pgfpathlineto{\pgfqpoint{4.978046in}{1.630169in}}%
\pgfpathclose%
\pgfusepath{fill}%
\end{pgfscope}%
\begin{pgfscope}%
\pgfpathrectangle{\pgfqpoint{1.254980in}{0.150000in}}{\pgfqpoint{5.490039in}{5.490039in}}%
\pgfusepath{clip}%
\pgfsetbuttcap%
\pgfsetroundjoin%
\definecolor{currentfill}{rgb}{0.283091,0.110553,0.431554}%
\pgfsetfillcolor{currentfill}%
\pgfsetfillopacity{0.700000}%
\pgfsetlinewidth{0.000000pt}%
\definecolor{currentstroke}{rgb}{0.000000,0.000000,0.000000}%
\pgfsetstrokecolor{currentstroke}%
\pgfsetdash{}{0pt}%
\pgfpathmoveto{\pgfqpoint{3.779885in}{1.709886in}}%
\pgfpathlineto{\pgfqpoint{3.793327in}{1.703618in}}%
\pgfpathlineto{\pgfqpoint{3.806773in}{1.697374in}}%
\pgfpathlineto{\pgfqpoint{3.820224in}{1.691156in}}%
\pgfpathlineto{\pgfqpoint{3.833680in}{1.684964in}}%
\pgfpathlineto{\pgfqpoint{3.825746in}{1.685777in}}%
\pgfpathlineto{\pgfqpoint{3.817801in}{1.686944in}}%
\pgfpathlineto{\pgfqpoint{3.809845in}{1.688474in}}%
\pgfpathlineto{\pgfqpoint{3.801878in}{1.690376in}}%
\pgfpathlineto{\pgfqpoint{3.788396in}{1.696897in}}%
\pgfpathlineto{\pgfqpoint{3.774919in}{1.703444in}}%
\pgfpathlineto{\pgfqpoint{3.761446in}{1.710015in}}%
\pgfpathlineto{\pgfqpoint{3.747978in}{1.716612in}}%
\pgfpathlineto{\pgfqpoint{3.755972in}{1.714377in}}%
\pgfpathlineto{\pgfqpoint{3.763954in}{1.712517in}}%
\pgfpathlineto{\pgfqpoint{3.771925in}{1.711023in}}%
\pgfpathlineto{\pgfqpoint{3.779885in}{1.709886in}}%
\pgfpathclose%
\pgfusepath{fill}%
\end{pgfscope}%
\begin{pgfscope}%
\pgfpathrectangle{\pgfqpoint{1.254980in}{0.150000in}}{\pgfqpoint{5.490039in}{5.490039in}}%
\pgfusepath{clip}%
\pgfsetbuttcap%
\pgfsetroundjoin%
\definecolor{currentfill}{rgb}{0.174274,0.445044,0.557792}%
\pgfsetfillcolor{currentfill}%
\pgfsetfillopacity{0.700000}%
\pgfsetlinewidth{0.000000pt}%
\definecolor{currentstroke}{rgb}{0.000000,0.000000,0.000000}%
\pgfsetstrokecolor{currentstroke}%
\pgfsetdash{}{0pt}%
\pgfpathmoveto{\pgfqpoint{2.630284in}{2.442896in}}%
\pgfpathlineto{\pgfqpoint{2.643616in}{2.432885in}}%
\pgfpathlineto{\pgfqpoint{2.656950in}{2.422913in}}%
\pgfpathlineto{\pgfqpoint{2.670285in}{2.412979in}}%
\pgfpathlineto{\pgfqpoint{2.683622in}{2.403084in}}%
\pgfpathlineto{\pgfqpoint{2.674677in}{2.418040in}}%
\pgfpathlineto{\pgfqpoint{2.665701in}{2.433586in}}%
\pgfpathlineto{\pgfqpoint{2.656694in}{2.449733in}}%
\pgfpathlineto{\pgfqpoint{2.647656in}{2.466493in}}%
\pgfpathlineto{\pgfqpoint{2.634267in}{2.476788in}}%
\pgfpathlineto{\pgfqpoint{2.620880in}{2.487122in}}%
\pgfpathlineto{\pgfqpoint{2.607494in}{2.497494in}}%
\pgfpathlineto{\pgfqpoint{2.594110in}{2.507906in}}%
\pgfpathlineto{\pgfqpoint{2.603202in}{2.490739in}}%
\pgfpathlineto{\pgfqpoint{2.612261in}{2.474189in}}%
\pgfpathlineto{\pgfqpoint{2.621288in}{2.458246in}}%
\pgfpathlineto{\pgfqpoint{2.630284in}{2.442896in}}%
\pgfpathclose%
\pgfusepath{fill}%
\end{pgfscope}%
\begin{pgfscope}%
\pgfpathrectangle{\pgfqpoint{1.254980in}{0.150000in}}{\pgfqpoint{5.490039in}{5.490039in}}%
\pgfusepath{clip}%
\pgfsetbuttcap%
\pgfsetroundjoin%
\definecolor{currentfill}{rgb}{0.271305,0.019942,0.347269}%
\pgfsetfillcolor{currentfill}%
\pgfsetfillopacity{0.700000}%
\pgfsetlinewidth{0.000000pt}%
\definecolor{currentstroke}{rgb}{0.000000,0.000000,0.000000}%
\pgfsetstrokecolor{currentstroke}%
\pgfsetdash{}{0pt}%
\pgfpathmoveto{\pgfqpoint{4.668699in}{1.562721in}}%
\pgfpathlineto{\pgfqpoint{4.682345in}{1.559387in}}%
\pgfpathlineto{\pgfqpoint{4.695999in}{1.556076in}}%
\pgfpathlineto{\pgfqpoint{4.709660in}{1.552788in}}%
\pgfpathlineto{\pgfqpoint{4.723327in}{1.549524in}}%
\pgfpathlineto{\pgfqpoint{4.715765in}{1.541043in}}%
\pgfpathlineto{\pgfqpoint{4.708199in}{1.532696in}}%
\pgfpathlineto{\pgfqpoint{4.700628in}{1.524486in}}%
\pgfpathlineto{\pgfqpoint{4.693054in}{1.516422in}}%
\pgfpathlineto{\pgfqpoint{4.679377in}{1.519936in}}%
\pgfpathlineto{\pgfqpoint{4.665706in}{1.523473in}}%
\pgfpathlineto{\pgfqpoint{4.652041in}{1.527034in}}%
\pgfpathlineto{\pgfqpoint{4.638384in}{1.530617in}}%
\pgfpathlineto{\pgfqpoint{4.645969in}{1.538427in}}%
\pgfpathlineto{\pgfqpoint{4.653550in}{1.546385in}}%
\pgfpathlineto{\pgfqpoint{4.661126in}{1.554485in}}%
\pgfpathlineto{\pgfqpoint{4.668699in}{1.562721in}}%
\pgfpathclose%
\pgfusepath{fill}%
\end{pgfscope}%
\begin{pgfscope}%
\pgfpathrectangle{\pgfqpoint{1.254980in}{0.150000in}}{\pgfqpoint{5.490039in}{5.490039in}}%
\pgfusepath{clip}%
\pgfsetbuttcap%
\pgfsetroundjoin%
\definecolor{currentfill}{rgb}{0.280267,0.073417,0.397163}%
\pgfsetfillcolor{currentfill}%
\pgfsetfillopacity{0.700000}%
\pgfsetlinewidth{0.000000pt}%
\definecolor{currentstroke}{rgb}{0.000000,0.000000,0.000000}%
\pgfsetstrokecolor{currentstroke}%
\pgfsetdash{}{0pt}%
\pgfpathmoveto{\pgfqpoint{3.972952in}{1.638945in}}%
\pgfpathlineto{\pgfqpoint{3.986431in}{1.633290in}}%
\pgfpathlineto{\pgfqpoint{3.999914in}{1.627659in}}%
\pgfpathlineto{\pgfqpoint{4.013403in}{1.622052in}}%
\pgfpathlineto{\pgfqpoint{4.026897in}{1.616470in}}%
\pgfpathlineto{\pgfqpoint{4.019071in}{1.615027in}}%
\pgfpathlineto{\pgfqpoint{4.011237in}{1.613893in}}%
\pgfpathlineto{\pgfqpoint{4.003394in}{1.613076in}}%
\pgfpathlineto{\pgfqpoint{3.995542in}{1.612586in}}%
\pgfpathlineto{\pgfqpoint{3.982025in}{1.618482in}}%
\pgfpathlineto{\pgfqpoint{3.968514in}{1.624403in}}%
\pgfpathlineto{\pgfqpoint{3.955007in}{1.630348in}}%
\pgfpathlineto{\pgfqpoint{3.941506in}{1.636318in}}%
\pgfpathlineto{\pgfqpoint{3.949381in}{1.636489in}}%
\pgfpathlineto{\pgfqpoint{3.957247in}{1.636989in}}%
\pgfpathlineto{\pgfqpoint{3.965104in}{1.637811in}}%
\pgfpathlineto{\pgfqpoint{3.972952in}{1.638945in}}%
\pgfpathclose%
\pgfusepath{fill}%
\end{pgfscope}%
\begin{pgfscope}%
\pgfpathrectangle{\pgfqpoint{1.254980in}{0.150000in}}{\pgfqpoint{5.490039in}{5.490039in}}%
\pgfusepath{clip}%
\pgfsetbuttcap%
\pgfsetroundjoin%
\definecolor{currentfill}{rgb}{0.282884,0.135920,0.453427}%
\pgfsetfillcolor{currentfill}%
\pgfsetfillopacity{0.700000}%
\pgfsetlinewidth{0.000000pt}%
\definecolor{currentstroke}{rgb}{0.000000,0.000000,0.000000}%
\pgfsetstrokecolor{currentstroke}%
\pgfsetdash{}{0pt}%
\pgfpathmoveto{\pgfqpoint{5.373151in}{1.770584in}}%
\pgfpathlineto{\pgfqpoint{5.387031in}{1.769479in}}%
\pgfpathlineto{\pgfqpoint{5.400919in}{1.768397in}}%
\pgfpathlineto{\pgfqpoint{5.414816in}{1.767338in}}%
\pgfpathlineto{\pgfqpoint{5.407430in}{1.755950in}}%
\pgfpathlineto{\pgfqpoint{5.400039in}{1.744529in}}%
\pgfpathlineto{\pgfqpoint{5.392642in}{1.733077in}}%
\pgfpathlineto{\pgfqpoint{5.385240in}{1.721599in}}%
\pgfpathlineto{\pgfqpoint{5.371338in}{1.722819in}}%
\pgfpathlineto{\pgfqpoint{5.357444in}{1.724062in}}%
\pgfpathlineto{\pgfqpoint{5.343559in}{1.725329in}}%
\pgfpathlineto{\pgfqpoint{5.350965in}{1.736682in}}%
\pgfpathlineto{\pgfqpoint{5.358365in}{1.748012in}}%
\pgfpathlineto{\pgfqpoint{5.365761in}{1.759314in}}%
\pgfpathlineto{\pgfqpoint{5.373151in}{1.770584in}}%
\pgfpathclose%
\pgfusepath{fill}%
\end{pgfscope}%
\begin{pgfscope}%
\pgfpathrectangle{\pgfqpoint{1.254980in}{0.150000in}}{\pgfqpoint{5.490039in}{5.490039in}}%
\pgfusepath{clip}%
\pgfsetbuttcap%
\pgfsetroundjoin%
\definecolor{currentfill}{rgb}{0.276022,0.044167,0.370164}%
\pgfsetfillcolor{currentfill}%
\pgfsetfillopacity{0.700000}%
\pgfsetlinewidth{0.000000pt}%
\definecolor{currentstroke}{rgb}{0.000000,0.000000,0.000000}%
\pgfsetstrokecolor{currentstroke}%
\pgfsetdash{}{0pt}%
\pgfpathmoveto{\pgfqpoint{4.893146in}{1.599818in}}%
\pgfpathlineto{\pgfqpoint{4.906862in}{1.597223in}}%
\pgfpathlineto{\pgfqpoint{4.920586in}{1.594651in}}%
\pgfpathlineto{\pgfqpoint{4.934317in}{1.592102in}}%
\pgfpathlineto{\pgfqpoint{4.948056in}{1.589576in}}%
\pgfpathlineto{\pgfqpoint{4.940549in}{1.579592in}}%
\pgfpathlineto{\pgfqpoint{4.933038in}{1.569684in}}%
\pgfpathlineto{\pgfqpoint{4.925524in}{1.559857in}}%
\pgfpathlineto{\pgfqpoint{4.918006in}{1.550118in}}%
\pgfpathlineto{\pgfqpoint{4.904260in}{1.552868in}}%
\pgfpathlineto{\pgfqpoint{4.890521in}{1.555642in}}%
\pgfpathlineto{\pgfqpoint{4.876790in}{1.558438in}}%
\pgfpathlineto{\pgfqpoint{4.863066in}{1.561258in}}%
\pgfpathlineto{\pgfqpoint{4.870591in}{1.570768in}}%
\pgfpathlineto{\pgfqpoint{4.878114in}{1.580368in}}%
\pgfpathlineto{\pgfqpoint{4.885632in}{1.590054in}}%
\pgfpathlineto{\pgfqpoint{4.893146in}{1.599818in}}%
\pgfpathclose%
\pgfusepath{fill}%
\end{pgfscope}%
\begin{pgfscope}%
\pgfpathrectangle{\pgfqpoint{1.254980in}{0.150000in}}{\pgfqpoint{5.490039in}{5.490039in}}%
\pgfusepath{clip}%
\pgfsetbuttcap%
\pgfsetroundjoin%
\definecolor{currentfill}{rgb}{0.231674,0.318106,0.544834}%
\pgfsetfillcolor{currentfill}%
\pgfsetfillopacity{0.700000}%
\pgfsetlinewidth{0.000000pt}%
\definecolor{currentstroke}{rgb}{0.000000,0.000000,0.000000}%
\pgfsetstrokecolor{currentstroke}%
\pgfsetdash{}{0pt}%
\pgfpathmoveto{\pgfqpoint{3.038767in}{2.131313in}}%
\pgfpathlineto{\pgfqpoint{3.052120in}{2.122665in}}%
\pgfpathlineto{\pgfqpoint{3.065476in}{2.114048in}}%
\pgfpathlineto{\pgfqpoint{3.078834in}{2.105463in}}%
\pgfpathlineto{\pgfqpoint{3.092196in}{2.096909in}}%
\pgfpathlineto{\pgfqpoint{3.083674in}{2.107007in}}%
\pgfpathlineto{\pgfqpoint{3.075131in}{2.117620in}}%
\pgfpathlineto{\pgfqpoint{3.066564in}{2.128759in}}%
\pgfpathlineto{\pgfqpoint{3.057974in}{2.140436in}}%
\pgfpathlineto{\pgfqpoint{3.044569in}{2.149368in}}%
\pgfpathlineto{\pgfqpoint{3.031168in}{2.158330in}}%
\pgfpathlineto{\pgfqpoint{3.017769in}{2.167325in}}%
\pgfpathlineto{\pgfqpoint{3.004373in}{2.176351in}}%
\pgfpathlineto{\pgfqpoint{3.013008in}{2.164290in}}%
\pgfpathlineto{\pgfqpoint{3.021618in}{2.152771in}}%
\pgfpathlineto{\pgfqpoint{3.030204in}{2.141782in}}%
\pgfpathlineto{\pgfqpoint{3.038767in}{2.131313in}}%
\pgfpathclose%
\pgfusepath{fill}%
\end{pgfscope}%
\begin{pgfscope}%
\pgfpathrectangle{\pgfqpoint{1.254980in}{0.150000in}}{\pgfqpoint{5.490039in}{5.490039in}}%
\pgfusepath{clip}%
\pgfsetbuttcap%
\pgfsetroundjoin%
\definecolor{currentfill}{rgb}{0.266580,0.228262,0.514349}%
\pgfsetfillcolor{currentfill}%
\pgfsetfillopacity{0.700000}%
\pgfsetlinewidth{0.000000pt}%
\definecolor{currentstroke}{rgb}{0.000000,0.000000,0.000000}%
\pgfsetstrokecolor{currentstroke}%
\pgfsetdash{}{0pt}%
\pgfpathmoveto{\pgfqpoint{3.339683in}{1.934478in}}%
\pgfpathlineto{\pgfqpoint{3.353065in}{1.926789in}}%
\pgfpathlineto{\pgfqpoint{3.366450in}{1.919128in}}%
\pgfpathlineto{\pgfqpoint{3.379839in}{1.911496in}}%
\pgfpathlineto{\pgfqpoint{3.393232in}{1.903892in}}%
\pgfpathlineto{\pgfqpoint{3.384979in}{1.910260in}}%
\pgfpathlineto{\pgfqpoint{3.376709in}{1.917082in}}%
\pgfpathlineto{\pgfqpoint{3.368422in}{1.924369in}}%
\pgfpathlineto{\pgfqpoint{3.360116in}{1.932131in}}%
\pgfpathlineto{\pgfqpoint{3.346687in}{1.940095in}}%
\pgfpathlineto{\pgfqpoint{3.333262in}{1.948087in}}%
\pgfpathlineto{\pgfqpoint{3.319840in}{1.956107in}}%
\pgfpathlineto{\pgfqpoint{3.306422in}{1.964156in}}%
\pgfpathlineto{\pgfqpoint{3.314765in}{1.956028in}}%
\pgfpathlineto{\pgfqpoint{3.323089in}{1.948380in}}%
\pgfpathlineto{\pgfqpoint{3.331395in}{1.941200in}}%
\pgfpathlineto{\pgfqpoint{3.339683in}{1.934478in}}%
\pgfpathclose%
\pgfusepath{fill}%
\end{pgfscope}%
\begin{pgfscope}%
\pgfpathrectangle{\pgfqpoint{1.254980in}{0.150000in}}{\pgfqpoint{5.490039in}{5.490039in}}%
\pgfusepath{clip}%
\pgfsetbuttcap%
\pgfsetroundjoin%
\definecolor{currentfill}{rgb}{0.280868,0.160771,0.472899}%
\pgfsetfillcolor{currentfill}%
\pgfsetfillopacity{0.700000}%
\pgfsetlinewidth{0.000000pt}%
\definecolor{currentstroke}{rgb}{0.000000,0.000000,0.000000}%
\pgfsetstrokecolor{currentstroke}%
\pgfsetdash{}{0pt}%
\pgfpathmoveto{\pgfqpoint{3.586716in}{1.797782in}}%
\pgfpathlineto{\pgfqpoint{3.600130in}{1.790875in}}%
\pgfpathlineto{\pgfqpoint{3.613548in}{1.783994in}}%
\pgfpathlineto{\pgfqpoint{3.626971in}{1.777139in}}%
\pgfpathlineto{\pgfqpoint{3.640398in}{1.770310in}}%
\pgfpathlineto{\pgfqpoint{3.632334in}{1.773605in}}%
\pgfpathlineto{\pgfqpoint{3.624257in}{1.777302in}}%
\pgfpathlineto{\pgfqpoint{3.616166in}{1.781408in}}%
\pgfpathlineto{\pgfqpoint{3.608061in}{1.785934in}}%
\pgfpathlineto{\pgfqpoint{3.594603in}{1.793106in}}%
\pgfpathlineto{\pgfqpoint{3.581150in}{1.800305in}}%
\pgfpathlineto{\pgfqpoint{3.567701in}{1.807529in}}%
\pgfpathlineto{\pgfqpoint{3.554255in}{1.814780in}}%
\pgfpathlineto{\pgfqpoint{3.562392in}{1.809905in}}%
\pgfpathlineto{\pgfqpoint{3.570514in}{1.805453in}}%
\pgfpathlineto{\pgfqpoint{3.578622in}{1.801415in}}%
\pgfpathlineto{\pgfqpoint{3.586716in}{1.797782in}}%
\pgfpathclose%
\pgfusepath{fill}%
\end{pgfscope}%
\begin{pgfscope}%
\pgfpathrectangle{\pgfqpoint{1.254980in}{0.150000in}}{\pgfqpoint{5.490039in}{5.490039in}}%
\pgfusepath{clip}%
\pgfsetbuttcap%
\pgfsetroundjoin%
\definecolor{currentfill}{rgb}{0.272594,0.025563,0.353093}%
\pgfsetfillcolor{currentfill}%
\pgfsetfillopacity{0.700000}%
\pgfsetlinewidth{0.000000pt}%
\definecolor{currentstroke}{rgb}{0.000000,0.000000,0.000000}%
\pgfsetstrokecolor{currentstroke}%
\pgfsetdash{}{0pt}%
\pgfpathmoveto{\pgfqpoint{4.305284in}{1.561558in}}%
\pgfpathlineto{\pgfqpoint{4.318838in}{1.556972in}}%
\pgfpathlineto{\pgfqpoint{4.332399in}{1.552408in}}%
\pgfpathlineto{\pgfqpoint{4.345965in}{1.547869in}}%
\pgfpathlineto{\pgfqpoint{4.359538in}{1.543352in}}%
\pgfpathlineto{\pgfqpoint{4.351859in}{1.538309in}}%
\pgfpathlineto{\pgfqpoint{4.344174in}{1.533496in}}%
\pgfpathlineto{\pgfqpoint{4.336484in}{1.528920in}}%
\pgfpathlineto{\pgfqpoint{4.328788in}{1.524589in}}%
\pgfpathlineto{\pgfqpoint{4.315199in}{1.529393in}}%
\pgfpathlineto{\pgfqpoint{4.301616in}{1.534220in}}%
\pgfpathlineto{\pgfqpoint{4.288039in}{1.539071in}}%
\pgfpathlineto{\pgfqpoint{4.274468in}{1.543946in}}%
\pgfpathlineto{\pgfqpoint{4.282181in}{1.547984in}}%
\pgfpathlineto{\pgfqpoint{4.289887in}{1.552271in}}%
\pgfpathlineto{\pgfqpoint{4.297588in}{1.556798in}}%
\pgfpathlineto{\pgfqpoint{4.305284in}{1.561558in}}%
\pgfpathclose%
\pgfusepath{fill}%
\end{pgfscope}%
\begin{pgfscope}%
\pgfpathrectangle{\pgfqpoint{1.254980in}{0.150000in}}{\pgfqpoint{5.490039in}{5.490039in}}%
\pgfusepath{clip}%
\pgfsetbuttcap%
\pgfsetroundjoin%
\definecolor{currentfill}{rgb}{0.283197,0.115680,0.436115}%
\pgfsetfillcolor{currentfill}%
\pgfsetfillopacity{0.700000}%
\pgfsetlinewidth{0.000000pt}%
\definecolor{currentstroke}{rgb}{0.000000,0.000000,0.000000}%
\pgfsetstrokecolor{currentstroke}%
\pgfsetdash{}{0pt}%
\pgfpathmoveto{\pgfqpoint{5.288105in}{1.730628in}}%
\pgfpathlineto{\pgfqpoint{5.301956in}{1.729268in}}%
\pgfpathlineto{\pgfqpoint{5.315815in}{1.727932in}}%
\pgfpathlineto{\pgfqpoint{5.329683in}{1.726618in}}%
\pgfpathlineto{\pgfqpoint{5.343559in}{1.725329in}}%
\pgfpathlineto{\pgfqpoint{5.336149in}{1.713955in}}%
\pgfpathlineto{\pgfqpoint{5.328734in}{1.702566in}}%
\pgfpathlineto{\pgfqpoint{5.321314in}{1.691166in}}%
\pgfpathlineto{\pgfqpoint{5.313889in}{1.679757in}}%
\pgfpathlineto{\pgfqpoint{5.300007in}{1.681221in}}%
\pgfpathlineto{\pgfqpoint{5.286134in}{1.682709in}}%
\pgfpathlineto{\pgfqpoint{5.272270in}{1.684219in}}%
\pgfpathlineto{\pgfqpoint{5.258414in}{1.685753in}}%
\pgfpathlineto{\pgfqpoint{5.265843in}{1.696982in}}%
\pgfpathlineto{\pgfqpoint{5.273268in}{1.708207in}}%
\pgfpathlineto{\pgfqpoint{5.280689in}{1.719424in}}%
\pgfpathlineto{\pgfqpoint{5.288105in}{1.730628in}}%
\pgfpathclose%
\pgfusepath{fill}%
\end{pgfscope}%
\begin{pgfscope}%
\pgfpathrectangle{\pgfqpoint{1.254980in}{0.150000in}}{\pgfqpoint{5.490039in}{5.490039in}}%
\pgfusepath{clip}%
\pgfsetbuttcap%
\pgfsetroundjoin%
\definecolor{currentfill}{rgb}{0.271305,0.019942,0.347269}%
\pgfsetfillcolor{currentfill}%
\pgfsetfillopacity{0.700000}%
\pgfsetlinewidth{0.000000pt}%
\definecolor{currentstroke}{rgb}{0.000000,0.000000,0.000000}%
\pgfsetstrokecolor{currentstroke}%
\pgfsetdash{}{0pt}%
\pgfpathmoveto{\pgfqpoint{4.444497in}{1.548947in}}%
\pgfpathlineto{\pgfqpoint{4.458087in}{1.544822in}}%
\pgfpathlineto{\pgfqpoint{4.471683in}{1.540721in}}%
\pgfpathlineto{\pgfqpoint{4.485286in}{1.536644in}}%
\pgfpathlineto{\pgfqpoint{4.498895in}{1.532589in}}%
\pgfpathlineto{\pgfqpoint{4.491264in}{1.526159in}}%
\pgfpathlineto{\pgfqpoint{4.483629in}{1.519923in}}%
\pgfpathlineto{\pgfqpoint{4.475990in}{1.513889in}}%
\pgfpathlineto{\pgfqpoint{4.468345in}{1.508064in}}%
\pgfpathlineto{\pgfqpoint{4.454722in}{1.512394in}}%
\pgfpathlineto{\pgfqpoint{4.441106in}{1.516746in}}%
\pgfpathlineto{\pgfqpoint{4.427495in}{1.521122in}}%
\pgfpathlineto{\pgfqpoint{4.413891in}{1.525522in}}%
\pgfpathlineto{\pgfqpoint{4.421550in}{1.531067in}}%
\pgfpathlineto{\pgfqpoint{4.429204in}{1.536824in}}%
\pgfpathlineto{\pgfqpoint{4.436853in}{1.542787in}}%
\pgfpathlineto{\pgfqpoint{4.444497in}{1.548947in}}%
\pgfpathclose%
\pgfusepath{fill}%
\end{pgfscope}%
\begin{pgfscope}%
\pgfpathrectangle{\pgfqpoint{1.254980in}{0.150000in}}{\pgfqpoint{5.490039in}{5.490039in}}%
\pgfusepath{clip}%
\pgfsetbuttcap%
\pgfsetroundjoin%
\definecolor{currentfill}{rgb}{0.180629,0.429975,0.557282}%
\pgfsetfillcolor{currentfill}%
\pgfsetfillopacity{0.700000}%
\pgfsetlinewidth{0.000000pt}%
\definecolor{currentstroke}{rgb}{0.000000,0.000000,0.000000}%
\pgfsetstrokecolor{currentstroke}%
\pgfsetdash{}{0pt}%
\pgfpathmoveto{\pgfqpoint{2.683622in}{2.403084in}}%
\pgfpathlineto{\pgfqpoint{2.696962in}{2.393227in}}%
\pgfpathlineto{\pgfqpoint{2.710303in}{2.383407in}}%
\pgfpathlineto{\pgfqpoint{2.723646in}{2.373625in}}%
\pgfpathlineto{\pgfqpoint{2.736991in}{2.363879in}}%
\pgfpathlineto{\pgfqpoint{2.728095in}{2.378443in}}%
\pgfpathlineto{\pgfqpoint{2.719170in}{2.393592in}}%
\pgfpathlineto{\pgfqpoint{2.710215in}{2.409338in}}%
\pgfpathlineto{\pgfqpoint{2.701229in}{2.425693in}}%
\pgfpathlineto{\pgfqpoint{2.687833in}{2.435837in}}%
\pgfpathlineto{\pgfqpoint{2.674439in}{2.446018in}}%
\pgfpathlineto{\pgfqpoint{2.661046in}{2.456237in}}%
\pgfpathlineto{\pgfqpoint{2.647656in}{2.466493in}}%
\pgfpathlineto{\pgfqpoint{2.656694in}{2.449733in}}%
\pgfpathlineto{\pgfqpoint{2.665701in}{2.433586in}}%
\pgfpathlineto{\pgfqpoint{2.674677in}{2.418040in}}%
\pgfpathlineto{\pgfqpoint{2.683622in}{2.403084in}}%
\pgfpathclose%
\pgfusepath{fill}%
\end{pgfscope}%
\begin{pgfscope}%
\pgfpathrectangle{\pgfqpoint{1.254980in}{0.150000in}}{\pgfqpoint{5.490039in}{5.490039in}}%
\pgfusepath{clip}%
\pgfsetbuttcap%
\pgfsetroundjoin%
\definecolor{currentfill}{rgb}{0.282656,0.100196,0.422160}%
\pgfsetfillcolor{currentfill}%
\pgfsetfillopacity{0.700000}%
\pgfsetlinewidth{0.000000pt}%
\definecolor{currentstroke}{rgb}{0.000000,0.000000,0.000000}%
\pgfsetstrokecolor{currentstroke}%
\pgfsetdash{}{0pt}%
\pgfpathmoveto{\pgfqpoint{5.203072in}{1.692121in}}%
\pgfpathlineto{\pgfqpoint{5.216895in}{1.690494in}}%
\pgfpathlineto{\pgfqpoint{5.230726in}{1.688890in}}%
\pgfpathlineto{\pgfqpoint{5.244566in}{1.687310in}}%
\pgfpathlineto{\pgfqpoint{5.258414in}{1.685753in}}%
\pgfpathlineto{\pgfqpoint{5.250979in}{1.674524in}}%
\pgfpathlineto{\pgfqpoint{5.243541in}{1.663298in}}%
\pgfpathlineto{\pgfqpoint{5.236098in}{1.652082in}}%
\pgfpathlineto{\pgfqpoint{5.228650in}{1.640879in}}%
\pgfpathlineto{\pgfqpoint{5.214797in}{1.642623in}}%
\pgfpathlineto{\pgfqpoint{5.200952in}{1.644390in}}%
\pgfpathlineto{\pgfqpoint{5.187116in}{1.646180in}}%
\pgfpathlineto{\pgfqpoint{5.173287in}{1.647994in}}%
\pgfpathlineto{\pgfqpoint{5.180740in}{1.659005in}}%
\pgfpathlineto{\pgfqpoint{5.188188in}{1.670033in}}%
\pgfpathlineto{\pgfqpoint{5.195632in}{1.681073in}}%
\pgfpathlineto{\pgfqpoint{5.203072in}{1.692121in}}%
\pgfpathclose%
\pgfusepath{fill}%
\end{pgfscope}%
\begin{pgfscope}%
\pgfpathrectangle{\pgfqpoint{1.254980in}{0.150000in}}{\pgfqpoint{5.490039in}{5.490039in}}%
\pgfusepath{clip}%
\pgfsetbuttcap%
\pgfsetroundjoin%
\definecolor{currentfill}{rgb}{0.276022,0.044167,0.370164}%
\pgfsetfillcolor{currentfill}%
\pgfsetfillopacity{0.700000}%
\pgfsetlinewidth{0.000000pt}%
\definecolor{currentstroke}{rgb}{0.000000,0.000000,0.000000}%
\pgfsetstrokecolor{currentstroke}%
\pgfsetdash{}{0pt}%
\pgfpathmoveto{\pgfqpoint{4.166111in}{1.583795in}}%
\pgfpathlineto{\pgfqpoint{4.179635in}{1.578730in}}%
\pgfpathlineto{\pgfqpoint{4.193165in}{1.573690in}}%
\pgfpathlineto{\pgfqpoint{4.206701in}{1.568673in}}%
\pgfpathlineto{\pgfqpoint{4.220243in}{1.563681in}}%
\pgfpathlineto{\pgfqpoint{4.212506in}{1.560194in}}%
\pgfpathlineto{\pgfqpoint{4.204762in}{1.556974in}}%
\pgfpathlineto{\pgfqpoint{4.197012in}{1.554029in}}%
\pgfpathlineto{\pgfqpoint{4.189255in}{1.551367in}}%
\pgfpathlineto{\pgfqpoint{4.175694in}{1.556661in}}%
\pgfpathlineto{\pgfqpoint{4.162140in}{1.561978in}}%
\pgfpathlineto{\pgfqpoint{4.148591in}{1.567320in}}%
\pgfpathlineto{\pgfqpoint{4.135047in}{1.572685in}}%
\pgfpathlineto{\pgfqpoint{4.142824in}{1.575041in}}%
\pgfpathlineto{\pgfqpoint{4.150593in}{1.577683in}}%
\pgfpathlineto{\pgfqpoint{4.158355in}{1.580604in}}%
\pgfpathlineto{\pgfqpoint{4.166111in}{1.583795in}}%
\pgfpathclose%
\pgfusepath{fill}%
\end{pgfscope}%
\begin{pgfscope}%
\pgfpathrectangle{\pgfqpoint{1.254980in}{0.150000in}}{\pgfqpoint{5.490039in}{5.490039in}}%
\pgfusepath{clip}%
\pgfsetbuttcap%
\pgfsetroundjoin%
\definecolor{currentfill}{rgb}{0.273809,0.031497,0.358853}%
\pgfsetfillcolor{currentfill}%
\pgfsetfillopacity{0.700000}%
\pgfsetlinewidth{0.000000pt}%
\definecolor{currentstroke}{rgb}{0.000000,0.000000,0.000000}%
\pgfsetstrokecolor{currentstroke}%
\pgfsetdash{}{0pt}%
\pgfpathmoveto{\pgfqpoint{4.808244in}{1.572767in}}%
\pgfpathlineto{\pgfqpoint{4.821938in}{1.569855in}}%
\pgfpathlineto{\pgfqpoint{4.835640in}{1.566967in}}%
\pgfpathlineto{\pgfqpoint{4.849349in}{1.564101in}}%
\pgfpathlineto{\pgfqpoint{4.863066in}{1.561258in}}%
\pgfpathlineto{\pgfqpoint{4.855536in}{1.551844in}}%
\pgfpathlineto{\pgfqpoint{4.848003in}{1.542532in}}%
\pgfpathlineto{\pgfqpoint{4.840466in}{1.533328in}}%
\pgfpathlineto{\pgfqpoint{4.832926in}{1.524238in}}%
\pgfpathlineto{\pgfqpoint{4.819201in}{1.527318in}}%
\pgfpathlineto{\pgfqpoint{4.805483in}{1.530421in}}%
\pgfpathlineto{\pgfqpoint{4.791773in}{1.533547in}}%
\pgfpathlineto{\pgfqpoint{4.778069in}{1.536696in}}%
\pgfpathlineto{\pgfqpoint{4.785619in}{1.545545in}}%
\pgfpathlineto{\pgfqpoint{4.793164in}{1.554510in}}%
\pgfpathlineto{\pgfqpoint{4.800706in}{1.563586in}}%
\pgfpathlineto{\pgfqpoint{4.808244in}{1.572767in}}%
\pgfpathclose%
\pgfusepath{fill}%
\end{pgfscope}%
\begin{pgfscope}%
\pgfpathrectangle{\pgfqpoint{1.254980in}{0.150000in}}{\pgfqpoint{5.490039in}{5.490039in}}%
\pgfusepath{clip}%
\pgfsetbuttcap%
\pgfsetroundjoin%
\definecolor{currentfill}{rgb}{0.271305,0.019942,0.347269}%
\pgfsetfillcolor{currentfill}%
\pgfsetfillopacity{0.700000}%
\pgfsetlinewidth{0.000000pt}%
\definecolor{currentstroke}{rgb}{0.000000,0.000000,0.000000}%
\pgfsetstrokecolor{currentstroke}%
\pgfsetdash{}{0pt}%
\pgfpathmoveto{\pgfqpoint{4.583823in}{1.545183in}}%
\pgfpathlineto{\pgfqpoint{4.597453in}{1.541507in}}%
\pgfpathlineto{\pgfqpoint{4.611090in}{1.537854in}}%
\pgfpathlineto{\pgfqpoint{4.624734in}{1.534224in}}%
\pgfpathlineto{\pgfqpoint{4.638384in}{1.530617in}}%
\pgfpathlineto{\pgfqpoint{4.630795in}{1.522962in}}%
\pgfpathlineto{\pgfqpoint{4.623203in}{1.515468in}}%
\pgfpathlineto{\pgfqpoint{4.615606in}{1.508141in}}%
\pgfpathlineto{\pgfqpoint{4.608005in}{1.500989in}}%
\pgfpathlineto{\pgfqpoint{4.594343in}{1.504858in}}%
\pgfpathlineto{\pgfqpoint{4.580688in}{1.508750in}}%
\pgfpathlineto{\pgfqpoint{4.567039in}{1.512665in}}%
\pgfpathlineto{\pgfqpoint{4.553397in}{1.516604in}}%
\pgfpathlineto{\pgfqpoint{4.561010in}{1.523489in}}%
\pgfpathlineto{\pgfqpoint{4.568619in}{1.530551in}}%
\pgfpathlineto{\pgfqpoint{4.576223in}{1.537785in}}%
\pgfpathlineto{\pgfqpoint{4.583823in}{1.545183in}}%
\pgfpathclose%
\pgfusepath{fill}%
\end{pgfscope}%
\begin{pgfscope}%
\pgfpathrectangle{\pgfqpoint{1.254980in}{0.150000in}}{\pgfqpoint{5.490039in}{5.490039in}}%
\pgfusepath{clip}%
\pgfsetbuttcap%
\pgfsetroundjoin%
\definecolor{currentfill}{rgb}{0.281446,0.084320,0.407414}%
\pgfsetfillcolor{currentfill}%
\pgfsetfillopacity{0.700000}%
\pgfsetlinewidth{0.000000pt}%
\definecolor{currentstroke}{rgb}{0.000000,0.000000,0.000000}%
\pgfsetstrokecolor{currentstroke}%
\pgfsetdash{}{0pt}%
\pgfpathmoveto{\pgfqpoint{5.118054in}{1.655480in}}%
\pgfpathlineto{\pgfqpoint{5.131850in}{1.653574in}}%
\pgfpathlineto{\pgfqpoint{5.145654in}{1.651691in}}%
\pgfpathlineto{\pgfqpoint{5.159467in}{1.649831in}}%
\pgfpathlineto{\pgfqpoint{5.173287in}{1.647994in}}%
\pgfpathlineto{\pgfqpoint{5.165830in}{1.637003in}}%
\pgfpathlineto{\pgfqpoint{5.158369in}{1.626039in}}%
\pgfpathlineto{\pgfqpoint{5.150904in}{1.615105in}}%
\pgfpathlineto{\pgfqpoint{5.143436in}{1.604206in}}%
\pgfpathlineto{\pgfqpoint{5.129610in}{1.606243in}}%
\pgfpathlineto{\pgfqpoint{5.115791in}{1.608302in}}%
\pgfpathlineto{\pgfqpoint{5.101981in}{1.610385in}}%
\pgfpathlineto{\pgfqpoint{5.088179in}{1.612490in}}%
\pgfpathlineto{\pgfqpoint{5.095654in}{1.623185in}}%
\pgfpathlineto{\pgfqpoint{5.103125in}{1.633918in}}%
\pgfpathlineto{\pgfqpoint{5.110591in}{1.644684in}}%
\pgfpathlineto{\pgfqpoint{5.118054in}{1.655480in}}%
\pgfpathclose%
\pgfusepath{fill}%
\end{pgfscope}%
\begin{pgfscope}%
\pgfpathrectangle{\pgfqpoint{1.254980in}{0.150000in}}{\pgfqpoint{5.490039in}{5.490039in}}%
\pgfusepath{clip}%
\pgfsetbuttcap%
\pgfsetroundjoin%
\definecolor{currentfill}{rgb}{0.282910,0.105393,0.426902}%
\pgfsetfillcolor{currentfill}%
\pgfsetfillopacity{0.700000}%
\pgfsetlinewidth{0.000000pt}%
\definecolor{currentstroke}{rgb}{0.000000,0.000000,0.000000}%
\pgfsetstrokecolor{currentstroke}%
\pgfsetdash{}{0pt}%
\pgfpathmoveto{\pgfqpoint{3.833680in}{1.684964in}}%
\pgfpathlineto{\pgfqpoint{3.847141in}{1.678796in}}%
\pgfpathlineto{\pgfqpoint{3.860606in}{1.672653in}}%
\pgfpathlineto{\pgfqpoint{3.874077in}{1.666535in}}%
\pgfpathlineto{\pgfqpoint{3.887553in}{1.660442in}}%
\pgfpathlineto{\pgfqpoint{3.879644in}{1.660933in}}%
\pgfpathlineto{\pgfqpoint{3.871725in}{1.661773in}}%
\pgfpathlineto{\pgfqpoint{3.863795in}{1.662973in}}%
\pgfpathlineto{\pgfqpoint{3.855855in}{1.664541in}}%
\pgfpathlineto{\pgfqpoint{3.842354in}{1.670962in}}%
\pgfpathlineto{\pgfqpoint{3.828857in}{1.677408in}}%
\pgfpathlineto{\pgfqpoint{3.815365in}{1.683880in}}%
\pgfpathlineto{\pgfqpoint{3.801878in}{1.690376in}}%
\pgfpathlineto{\pgfqpoint{3.809845in}{1.688474in}}%
\pgfpathlineto{\pgfqpoint{3.817801in}{1.686944in}}%
\pgfpathlineto{\pgfqpoint{3.825746in}{1.685777in}}%
\pgfpathlineto{\pgfqpoint{3.833680in}{1.684964in}}%
\pgfpathclose%
\pgfusepath{fill}%
\end{pgfscope}%
\begin{pgfscope}%
\pgfpathrectangle{\pgfqpoint{1.254980in}{0.150000in}}{\pgfqpoint{5.490039in}{5.490039in}}%
\pgfusepath{clip}%
\pgfsetbuttcap%
\pgfsetroundjoin%
\definecolor{currentfill}{rgb}{0.119512,0.607464,0.540218}%
\pgfsetfillcolor{currentfill}%
\pgfsetfillopacity{0.700000}%
\pgfsetlinewidth{0.000000pt}%
\definecolor{currentstroke}{rgb}{0.000000,0.000000,0.000000}%
\pgfsetstrokecolor{currentstroke}%
\pgfsetdash{}{0pt}%
\pgfpathmoveto{\pgfqpoint{2.166368in}{2.864034in}}%
\pgfpathlineto{\pgfqpoint{2.179727in}{2.852157in}}%
\pgfpathlineto{\pgfqpoint{2.193087in}{2.840334in}}%
\pgfpathlineto{\pgfqpoint{2.206446in}{2.828564in}}%
\pgfpathlineto{\pgfqpoint{2.219805in}{2.816846in}}%
\pgfpathlineto{\pgfqpoint{2.210276in}{2.837555in}}%
\pgfpathlineto{\pgfqpoint{2.200706in}{2.858939in}}%
\pgfpathlineto{\pgfqpoint{2.191095in}{2.881011in}}%
\pgfpathlineto{\pgfqpoint{2.181440in}{2.903783in}}%
\pgfpathlineto{\pgfqpoint{2.168018in}{2.915930in}}%
\pgfpathlineto{\pgfqpoint{2.154596in}{2.928129in}}%
\pgfpathlineto{\pgfqpoint{2.141174in}{2.940381in}}%
\pgfpathlineto{\pgfqpoint{2.127751in}{2.952687in}}%
\pgfpathlineto{\pgfqpoint{2.137470in}{2.929477in}}%
\pgfpathlineto{\pgfqpoint{2.147145in}{2.906974in}}%
\pgfpathlineto{\pgfqpoint{2.156778in}{2.885164in}}%
\pgfpathlineto{\pgfqpoint{2.166368in}{2.864034in}}%
\pgfpathclose%
\pgfusepath{fill}%
\end{pgfscope}%
\begin{pgfscope}%
\pgfpathrectangle{\pgfqpoint{1.254980in}{0.150000in}}{\pgfqpoint{5.490039in}{5.490039in}}%
\pgfusepath{clip}%
\pgfsetbuttcap%
\pgfsetroundjoin%
\definecolor{currentfill}{rgb}{0.235526,0.309527,0.542944}%
\pgfsetfillcolor{currentfill}%
\pgfsetfillopacity{0.700000}%
\pgfsetlinewidth{0.000000pt}%
\definecolor{currentstroke}{rgb}{0.000000,0.000000,0.000000}%
\pgfsetstrokecolor{currentstroke}%
\pgfsetdash{}{0pt}%
\pgfpathmoveto{\pgfqpoint{3.092196in}{2.096909in}}%
\pgfpathlineto{\pgfqpoint{3.105561in}{2.088386in}}%
\pgfpathlineto{\pgfqpoint{3.118928in}{2.079894in}}%
\pgfpathlineto{\pgfqpoint{3.132299in}{2.071433in}}%
\pgfpathlineto{\pgfqpoint{3.145674in}{2.063003in}}%
\pgfpathlineto{\pgfqpoint{3.137193in}{2.072729in}}%
\pgfpathlineto{\pgfqpoint{3.128691in}{2.082967in}}%
\pgfpathlineto{\pgfqpoint{3.120167in}{2.093728in}}%
\pgfpathlineto{\pgfqpoint{3.111620in}{2.105021in}}%
\pgfpathlineto{\pgfqpoint{3.098204in}{2.113829in}}%
\pgfpathlineto{\pgfqpoint{3.084791in}{2.122667in}}%
\pgfpathlineto{\pgfqpoint{3.071381in}{2.131536in}}%
\pgfpathlineto{\pgfqpoint{3.057974in}{2.140436in}}%
\pgfpathlineto{\pgfqpoint{3.066564in}{2.128759in}}%
\pgfpathlineto{\pgfqpoint{3.075131in}{2.117620in}}%
\pgfpathlineto{\pgfqpoint{3.083674in}{2.107007in}}%
\pgfpathlineto{\pgfqpoint{3.092196in}{2.096909in}}%
\pgfpathclose%
\pgfusepath{fill}%
\end{pgfscope}%
\begin{pgfscope}%
\pgfpathrectangle{\pgfqpoint{1.254980in}{0.150000in}}{\pgfqpoint{5.490039in}{5.490039in}}%
\pgfusepath{clip}%
\pgfsetbuttcap%
\pgfsetroundjoin%
\definecolor{currentfill}{rgb}{0.269308,0.218818,0.509577}%
\pgfsetfillcolor{currentfill}%
\pgfsetfillopacity{0.700000}%
\pgfsetlinewidth{0.000000pt}%
\definecolor{currentstroke}{rgb}{0.000000,0.000000,0.000000}%
\pgfsetstrokecolor{currentstroke}%
\pgfsetdash{}{0pt}%
\pgfpathmoveto{\pgfqpoint{3.393232in}{1.903892in}}%
\pgfpathlineto{\pgfqpoint{3.406629in}{1.896315in}}%
\pgfpathlineto{\pgfqpoint{3.420030in}{1.888766in}}%
\pgfpathlineto{\pgfqpoint{3.433434in}{1.881245in}}%
\pgfpathlineto{\pgfqpoint{3.446843in}{1.873752in}}%
\pgfpathlineto{\pgfqpoint{3.438624in}{1.879766in}}%
\pgfpathlineto{\pgfqpoint{3.430389in}{1.886232in}}%
\pgfpathlineto{\pgfqpoint{3.422138in}{1.893158in}}%
\pgfpathlineto{\pgfqpoint{3.413869in}{1.900555in}}%
\pgfpathlineto{\pgfqpoint{3.400425in}{1.908407in}}%
\pgfpathlineto{\pgfqpoint{3.386985in}{1.916288in}}%
\pgfpathlineto{\pgfqpoint{3.373549in}{1.924195in}}%
\pgfpathlineto{\pgfqpoint{3.360116in}{1.932131in}}%
\pgfpathlineto{\pgfqpoint{3.368422in}{1.924369in}}%
\pgfpathlineto{\pgfqpoint{3.376709in}{1.917082in}}%
\pgfpathlineto{\pgfqpoint{3.384979in}{1.910260in}}%
\pgfpathlineto{\pgfqpoint{3.393232in}{1.903892in}}%
\pgfpathclose%
\pgfusepath{fill}%
\end{pgfscope}%
\begin{pgfscope}%
\pgfpathrectangle{\pgfqpoint{1.254980in}{0.150000in}}{\pgfqpoint{5.490039in}{5.490039in}}%
\pgfusepath{clip}%
\pgfsetbuttcap%
\pgfsetroundjoin%
\definecolor{currentfill}{rgb}{0.279566,0.067836,0.391917}%
\pgfsetfillcolor{currentfill}%
\pgfsetfillopacity{0.700000}%
\pgfsetlinewidth{0.000000pt}%
\definecolor{currentstroke}{rgb}{0.000000,0.000000,0.000000}%
\pgfsetstrokecolor{currentstroke}%
\pgfsetdash{}{0pt}%
\pgfpathmoveto{\pgfqpoint{4.026897in}{1.616470in}}%
\pgfpathlineto{\pgfqpoint{4.040397in}{1.610913in}}%
\pgfpathlineto{\pgfqpoint{4.053902in}{1.605379in}}%
\pgfpathlineto{\pgfqpoint{4.067412in}{1.599870in}}%
\pgfpathlineto{\pgfqpoint{4.080928in}{1.594385in}}%
\pgfpathlineto{\pgfqpoint{4.073123in}{1.592632in}}%
\pgfpathlineto{\pgfqpoint{4.065311in}{1.591185in}}%
\pgfpathlineto{\pgfqpoint{4.057490in}{1.590053in}}%
\pgfpathlineto{\pgfqpoint{4.049661in}{1.589242in}}%
\pgfpathlineto{\pgfqpoint{4.036123in}{1.595042in}}%
\pgfpathlineto{\pgfqpoint{4.022591in}{1.600866in}}%
\pgfpathlineto{\pgfqpoint{4.009064in}{1.606714in}}%
\pgfpathlineto{\pgfqpoint{3.995542in}{1.612586in}}%
\pgfpathlineto{\pgfqpoint{4.003394in}{1.613076in}}%
\pgfpathlineto{\pgfqpoint{4.011237in}{1.613893in}}%
\pgfpathlineto{\pgfqpoint{4.019071in}{1.615027in}}%
\pgfpathlineto{\pgfqpoint{4.026897in}{1.616470in}}%
\pgfpathclose%
\pgfusepath{fill}%
\end{pgfscope}%
\begin{pgfscope}%
\pgfpathrectangle{\pgfqpoint{1.254980in}{0.150000in}}{\pgfqpoint{5.490039in}{5.490039in}}%
\pgfusepath{clip}%
\pgfsetbuttcap%
\pgfsetroundjoin%
\definecolor{currentfill}{rgb}{0.185556,0.418570,0.556753}%
\pgfsetfillcolor{currentfill}%
\pgfsetfillopacity{0.700000}%
\pgfsetlinewidth{0.000000pt}%
\definecolor{currentstroke}{rgb}{0.000000,0.000000,0.000000}%
\pgfsetstrokecolor{currentstroke}%
\pgfsetdash{}{0pt}%
\pgfpathmoveto{\pgfqpoint{2.736991in}{2.363879in}}%
\pgfpathlineto{\pgfqpoint{2.750338in}{2.354171in}}%
\pgfpathlineto{\pgfqpoint{2.763687in}{2.344499in}}%
\pgfpathlineto{\pgfqpoint{2.777039in}{2.334863in}}%
\pgfpathlineto{\pgfqpoint{2.790392in}{2.325263in}}%
\pgfpathlineto{\pgfqpoint{2.781546in}{2.339435in}}%
\pgfpathlineto{\pgfqpoint{2.772670in}{2.354188in}}%
\pgfpathlineto{\pgfqpoint{2.763765in}{2.369534in}}%
\pgfpathlineto{\pgfqpoint{2.754831in}{2.385484in}}%
\pgfpathlineto{\pgfqpoint{2.741427in}{2.395482in}}%
\pgfpathlineto{\pgfqpoint{2.728026in}{2.405515in}}%
\pgfpathlineto{\pgfqpoint{2.714626in}{2.415586in}}%
\pgfpathlineto{\pgfqpoint{2.701229in}{2.425693in}}%
\pgfpathlineto{\pgfqpoint{2.710215in}{2.409338in}}%
\pgfpathlineto{\pgfqpoint{2.719170in}{2.393592in}}%
\pgfpathlineto{\pgfqpoint{2.728095in}{2.378443in}}%
\pgfpathlineto{\pgfqpoint{2.736991in}{2.363879in}}%
\pgfpathclose%
\pgfusepath{fill}%
\end{pgfscope}%
\begin{pgfscope}%
\pgfpathrectangle{\pgfqpoint{1.254980in}{0.150000in}}{\pgfqpoint{5.490039in}{5.490039in}}%
\pgfusepath{clip}%
\pgfsetbuttcap%
\pgfsetroundjoin%
\definecolor{currentfill}{rgb}{0.279566,0.067836,0.391917}%
\pgfsetfillcolor{currentfill}%
\pgfsetfillopacity{0.700000}%
\pgfsetlinewidth{0.000000pt}%
\definecolor{currentstroke}{rgb}{0.000000,0.000000,0.000000}%
\pgfsetstrokecolor{currentstroke}%
\pgfsetdash{}{0pt}%
\pgfpathmoveto{\pgfqpoint{5.033050in}{1.621145in}}%
\pgfpathlineto{\pgfqpoint{5.046821in}{1.618947in}}%
\pgfpathlineto{\pgfqpoint{5.060599in}{1.616771in}}%
\pgfpathlineto{\pgfqpoint{5.074385in}{1.614619in}}%
\pgfpathlineto{\pgfqpoint{5.088179in}{1.612490in}}%
\pgfpathlineto{\pgfqpoint{5.080701in}{1.601840in}}%
\pgfpathlineto{\pgfqpoint{5.073218in}{1.591238in}}%
\pgfpathlineto{\pgfqpoint{5.065732in}{1.580690in}}%
\pgfpathlineto{\pgfqpoint{5.058243in}{1.570201in}}%
\pgfpathlineto{\pgfqpoint{5.044442in}{1.572542in}}%
\pgfpathlineto{\pgfqpoint{5.030650in}{1.574906in}}%
\pgfpathlineto{\pgfqpoint{5.016865in}{1.577293in}}%
\pgfpathlineto{\pgfqpoint{5.003088in}{1.579704in}}%
\pgfpathlineto{\pgfqpoint{5.010584in}{1.589976in}}%
\pgfpathlineto{\pgfqpoint{5.018076in}{1.600310in}}%
\pgfpathlineto{\pgfqpoint{5.025565in}{1.610702in}}%
\pgfpathlineto{\pgfqpoint{5.033050in}{1.621145in}}%
\pgfpathclose%
\pgfusepath{fill}%
\end{pgfscope}%
\begin{pgfscope}%
\pgfpathrectangle{\pgfqpoint{1.254980in}{0.150000in}}{\pgfqpoint{5.490039in}{5.490039in}}%
\pgfusepath{clip}%
\pgfsetbuttcap%
\pgfsetroundjoin%
\definecolor{currentfill}{rgb}{0.281887,0.150881,0.465405}%
\pgfsetfillcolor{currentfill}%
\pgfsetfillopacity{0.700000}%
\pgfsetlinewidth{0.000000pt}%
\definecolor{currentstroke}{rgb}{0.000000,0.000000,0.000000}%
\pgfsetstrokecolor{currentstroke}%
\pgfsetdash{}{0pt}%
\pgfpathmoveto{\pgfqpoint{3.640398in}{1.770310in}}%
\pgfpathlineto{\pgfqpoint{3.653830in}{1.763508in}}%
\pgfpathlineto{\pgfqpoint{3.667266in}{1.756731in}}%
\pgfpathlineto{\pgfqpoint{3.680706in}{1.749980in}}%
\pgfpathlineto{\pgfqpoint{3.694151in}{1.743256in}}%
\pgfpathlineto{\pgfqpoint{3.686117in}{1.746213in}}%
\pgfpathlineto{\pgfqpoint{3.678069in}{1.749568in}}%
\pgfpathlineto{\pgfqpoint{3.670009in}{1.753329in}}%
\pgfpathlineto{\pgfqpoint{3.661935in}{1.757507in}}%
\pgfpathlineto{\pgfqpoint{3.648460in}{1.764575in}}%
\pgfpathlineto{\pgfqpoint{3.634990in}{1.771669in}}%
\pgfpathlineto{\pgfqpoint{3.621523in}{1.778788in}}%
\pgfpathlineto{\pgfqpoint{3.608061in}{1.785934in}}%
\pgfpathlineto{\pgfqpoint{3.616166in}{1.781408in}}%
\pgfpathlineto{\pgfqpoint{3.624257in}{1.777302in}}%
\pgfpathlineto{\pgfqpoint{3.632334in}{1.773605in}}%
\pgfpathlineto{\pgfqpoint{3.640398in}{1.770310in}}%
\pgfpathclose%
\pgfusepath{fill}%
\end{pgfscope}%
\begin{pgfscope}%
\pgfpathrectangle{\pgfqpoint{1.254980in}{0.150000in}}{\pgfqpoint{5.490039in}{5.490039in}}%
\pgfusepath{clip}%
\pgfsetbuttcap%
\pgfsetroundjoin%
\definecolor{currentfill}{rgb}{0.272594,0.025563,0.353093}%
\pgfsetfillcolor{currentfill}%
\pgfsetfillopacity{0.700000}%
\pgfsetlinewidth{0.000000pt}%
\definecolor{currentstroke}{rgb}{0.000000,0.000000,0.000000}%
\pgfsetstrokecolor{currentstroke}%
\pgfsetdash{}{0pt}%
\pgfpathmoveto{\pgfqpoint{4.723327in}{1.549524in}}%
\pgfpathlineto{\pgfqpoint{4.737002in}{1.546282in}}%
\pgfpathlineto{\pgfqpoint{4.750684in}{1.543064in}}%
\pgfpathlineto{\pgfqpoint{4.764373in}{1.539869in}}%
\pgfpathlineto{\pgfqpoint{4.778069in}{1.536696in}}%
\pgfpathlineto{\pgfqpoint{4.770517in}{1.527971in}}%
\pgfpathlineto{\pgfqpoint{4.762960in}{1.519375in}}%
\pgfpathlineto{\pgfqpoint{4.755400in}{1.510914in}}%
\pgfpathlineto{\pgfqpoint{4.747836in}{1.502595in}}%
\pgfpathlineto{\pgfqpoint{4.734130in}{1.506017in}}%
\pgfpathlineto{\pgfqpoint{4.720431in}{1.509462in}}%
\pgfpathlineto{\pgfqpoint{4.706739in}{1.512931in}}%
\pgfpathlineto{\pgfqpoint{4.693054in}{1.516422in}}%
\pgfpathlineto{\pgfqpoint{4.700628in}{1.524486in}}%
\pgfpathlineto{\pgfqpoint{4.708199in}{1.532696in}}%
\pgfpathlineto{\pgfqpoint{4.715765in}{1.541043in}}%
\pgfpathlineto{\pgfqpoint{4.723327in}{1.549524in}}%
\pgfpathclose%
\pgfusepath{fill}%
\end{pgfscope}%
\begin{pgfscope}%
\pgfpathrectangle{\pgfqpoint{1.254980in}{0.150000in}}{\pgfqpoint{5.490039in}{5.490039in}}%
\pgfusepath{clip}%
\pgfsetbuttcap%
\pgfsetroundjoin%
\definecolor{currentfill}{rgb}{0.121831,0.589055,0.545623}%
\pgfsetfillcolor{currentfill}%
\pgfsetfillopacity{0.700000}%
\pgfsetlinewidth{0.000000pt}%
\definecolor{currentstroke}{rgb}{0.000000,0.000000,0.000000}%
\pgfsetstrokecolor{currentstroke}%
\pgfsetdash{}{0pt}%
\pgfpathmoveto{\pgfqpoint{2.219805in}{2.816846in}}%
\pgfpathlineto{\pgfqpoint{2.233165in}{2.805180in}}%
\pgfpathlineto{\pgfqpoint{2.246524in}{2.793564in}}%
\pgfpathlineto{\pgfqpoint{2.259884in}{2.781999in}}%
\pgfpathlineto{\pgfqpoint{2.273244in}{2.770484in}}%
\pgfpathlineto{\pgfqpoint{2.263776in}{2.790773in}}%
\pgfpathlineto{\pgfqpoint{2.254267in}{2.811732in}}%
\pgfpathlineto{\pgfqpoint{2.244718in}{2.833374in}}%
\pgfpathlineto{\pgfqpoint{2.235127in}{2.855712in}}%
\pgfpathlineto{\pgfqpoint{2.221705in}{2.867654in}}%
\pgfpathlineto{\pgfqpoint{2.208283in}{2.879646in}}%
\pgfpathlineto{\pgfqpoint{2.194862in}{2.891689in}}%
\pgfpathlineto{\pgfqpoint{2.181440in}{2.903783in}}%
\pgfpathlineto{\pgfqpoint{2.191095in}{2.881011in}}%
\pgfpathlineto{\pgfqpoint{2.200706in}{2.858939in}}%
\pgfpathlineto{\pgfqpoint{2.210276in}{2.837555in}}%
\pgfpathlineto{\pgfqpoint{2.219805in}{2.816846in}}%
\pgfpathclose%
\pgfusepath{fill}%
\end{pgfscope}%
\begin{pgfscope}%
\pgfpathrectangle{\pgfqpoint{1.254980in}{0.150000in}}{\pgfqpoint{5.490039in}{5.490039in}}%
\pgfusepath{clip}%
\pgfsetbuttcap%
\pgfsetroundjoin%
\definecolor{currentfill}{rgb}{0.277018,0.050344,0.375715}%
\pgfsetfillcolor{currentfill}%
\pgfsetfillopacity{0.700000}%
\pgfsetlinewidth{0.000000pt}%
\definecolor{currentstroke}{rgb}{0.000000,0.000000,0.000000}%
\pgfsetstrokecolor{currentstroke}%
\pgfsetdash{}{0pt}%
\pgfpathmoveto{\pgfqpoint{4.948056in}{1.589576in}}%
\pgfpathlineto{\pgfqpoint{4.961802in}{1.587074in}}%
\pgfpathlineto{\pgfqpoint{4.975556in}{1.584594in}}%
\pgfpathlineto{\pgfqpoint{4.989318in}{1.582137in}}%
\pgfpathlineto{\pgfqpoint{5.003088in}{1.579704in}}%
\pgfpathlineto{\pgfqpoint{4.995588in}{1.569499in}}%
\pgfpathlineto{\pgfqpoint{4.988084in}{1.559368in}}%
\pgfpathlineto{\pgfqpoint{4.980577in}{1.549315in}}%
\pgfpathlineto{\pgfqpoint{4.973067in}{1.539346in}}%
\pgfpathlineto{\pgfqpoint{4.959290in}{1.542004in}}%
\pgfpathlineto{\pgfqpoint{4.945521in}{1.544686in}}%
\pgfpathlineto{\pgfqpoint{4.931760in}{1.547390in}}%
\pgfpathlineto{\pgfqpoint{4.918006in}{1.550118in}}%
\pgfpathlineto{\pgfqpoint{4.925524in}{1.559857in}}%
\pgfpathlineto{\pgfqpoint{4.933038in}{1.569684in}}%
\pgfpathlineto{\pgfqpoint{4.940549in}{1.579592in}}%
\pgfpathlineto{\pgfqpoint{4.948056in}{1.589576in}}%
\pgfpathclose%
\pgfusepath{fill}%
\end{pgfscope}%
\begin{pgfscope}%
\pgfpathrectangle{\pgfqpoint{1.254980in}{0.150000in}}{\pgfqpoint{5.490039in}{5.490039in}}%
\pgfusepath{clip}%
\pgfsetbuttcap%
\pgfsetroundjoin%
\definecolor{currentfill}{rgb}{0.272594,0.025563,0.353093}%
\pgfsetfillcolor{currentfill}%
\pgfsetfillopacity{0.700000}%
\pgfsetlinewidth{0.000000pt}%
\definecolor{currentstroke}{rgb}{0.000000,0.000000,0.000000}%
\pgfsetstrokecolor{currentstroke}%
\pgfsetdash{}{0pt}%
\pgfpathmoveto{\pgfqpoint{4.359538in}{1.543352in}}%
\pgfpathlineto{\pgfqpoint{4.373117in}{1.538860in}}%
\pgfpathlineto{\pgfqpoint{4.386702in}{1.534390in}}%
\pgfpathlineto{\pgfqpoint{4.400294in}{1.529944in}}%
\pgfpathlineto{\pgfqpoint{4.413891in}{1.525522in}}%
\pgfpathlineto{\pgfqpoint{4.406227in}{1.520195in}}%
\pgfpathlineto{\pgfqpoint{4.398558in}{1.515096in}}%
\pgfpathlineto{\pgfqpoint{4.390884in}{1.510230in}}%
\pgfpathlineto{\pgfqpoint{4.383204in}{1.505606in}}%
\pgfpathlineto{\pgfqpoint{4.369591in}{1.510317in}}%
\pgfpathlineto{\pgfqpoint{4.355984in}{1.515051in}}%
\pgfpathlineto{\pgfqpoint{4.342383in}{1.519808in}}%
\pgfpathlineto{\pgfqpoint{4.328788in}{1.524589in}}%
\pgfpathlineto{\pgfqpoint{4.336484in}{1.528920in}}%
\pgfpathlineto{\pgfqpoint{4.344174in}{1.533496in}}%
\pgfpathlineto{\pgfqpoint{4.351859in}{1.538309in}}%
\pgfpathlineto{\pgfqpoint{4.359538in}{1.543352in}}%
\pgfpathclose%
\pgfusepath{fill}%
\end{pgfscope}%
\begin{pgfscope}%
\pgfpathrectangle{\pgfqpoint{1.254980in}{0.150000in}}{\pgfqpoint{5.490039in}{5.490039in}}%
\pgfusepath{clip}%
\pgfsetbuttcap%
\pgfsetroundjoin%
\definecolor{currentfill}{rgb}{0.241237,0.296485,0.539709}%
\pgfsetfillcolor{currentfill}%
\pgfsetfillopacity{0.700000}%
\pgfsetlinewidth{0.000000pt}%
\definecolor{currentstroke}{rgb}{0.000000,0.000000,0.000000}%
\pgfsetstrokecolor{currentstroke}%
\pgfsetdash{}{0pt}%
\pgfpathmoveto{\pgfqpoint{3.145674in}{2.063003in}}%
\pgfpathlineto{\pgfqpoint{3.159051in}{2.054602in}}%
\pgfpathlineto{\pgfqpoint{3.172432in}{2.046232in}}%
\pgfpathlineto{\pgfqpoint{3.185815in}{2.037892in}}%
\pgfpathlineto{\pgfqpoint{3.199203in}{2.029582in}}%
\pgfpathlineto{\pgfqpoint{3.190762in}{2.038938in}}%
\pgfpathlineto{\pgfqpoint{3.182301in}{2.048801in}}%
\pgfpathlineto{\pgfqpoint{3.173819in}{2.059183in}}%
\pgfpathlineto{\pgfqpoint{3.165315in}{2.070094in}}%
\pgfpathlineto{\pgfqpoint{3.151886in}{2.078781in}}%
\pgfpathlineto{\pgfqpoint{3.138461in}{2.087497in}}%
\pgfpathlineto{\pgfqpoint{3.125039in}{2.096244in}}%
\pgfpathlineto{\pgfqpoint{3.111620in}{2.105021in}}%
\pgfpathlineto{\pgfqpoint{3.120167in}{2.093728in}}%
\pgfpathlineto{\pgfqpoint{3.128691in}{2.082967in}}%
\pgfpathlineto{\pgfqpoint{3.137193in}{2.072729in}}%
\pgfpathlineto{\pgfqpoint{3.145674in}{2.063003in}}%
\pgfpathclose%
\pgfusepath{fill}%
\end{pgfscope}%
\begin{pgfscope}%
\pgfpathrectangle{\pgfqpoint{1.254980in}{0.150000in}}{\pgfqpoint{5.490039in}{5.490039in}}%
\pgfusepath{clip}%
\pgfsetbuttcap%
\pgfsetroundjoin%
\definecolor{currentfill}{rgb}{0.271305,0.019942,0.347269}%
\pgfsetfillcolor{currentfill}%
\pgfsetfillopacity{0.700000}%
\pgfsetlinewidth{0.000000pt}%
\definecolor{currentstroke}{rgb}{0.000000,0.000000,0.000000}%
\pgfsetstrokecolor{currentstroke}%
\pgfsetdash{}{0pt}%
\pgfpathmoveto{\pgfqpoint{4.498895in}{1.532589in}}%
\pgfpathlineto{\pgfqpoint{4.512511in}{1.528558in}}%
\pgfpathlineto{\pgfqpoint{4.526133in}{1.524550in}}%
\pgfpathlineto{\pgfqpoint{4.539762in}{1.520565in}}%
\pgfpathlineto{\pgfqpoint{4.553397in}{1.516604in}}%
\pgfpathlineto{\pgfqpoint{4.545779in}{1.509903in}}%
\pgfpathlineto{\pgfqpoint{4.538158in}{1.503394in}}%
\pgfpathlineto{\pgfqpoint{4.530531in}{1.497083in}}%
\pgfpathlineto{\pgfqpoint{4.522901in}{1.490978in}}%
\pgfpathlineto{\pgfqpoint{4.509252in}{1.495215in}}%
\pgfpathlineto{\pgfqpoint{4.495610in}{1.499475in}}%
\pgfpathlineto{\pgfqpoint{4.481974in}{1.503758in}}%
\pgfpathlineto{\pgfqpoint{4.468345in}{1.508064in}}%
\pgfpathlineto{\pgfqpoint{4.475990in}{1.513889in}}%
\pgfpathlineto{\pgfqpoint{4.483629in}{1.519923in}}%
\pgfpathlineto{\pgfqpoint{4.491264in}{1.526159in}}%
\pgfpathlineto{\pgfqpoint{4.498895in}{1.532589in}}%
\pgfpathclose%
\pgfusepath{fill}%
\end{pgfscope}%
\begin{pgfscope}%
\pgfpathrectangle{\pgfqpoint{1.254980in}{0.150000in}}{\pgfqpoint{5.490039in}{5.490039in}}%
\pgfusepath{clip}%
\pgfsetbuttcap%
\pgfsetroundjoin%
\definecolor{currentfill}{rgb}{0.274952,0.037752,0.364543}%
\pgfsetfillcolor{currentfill}%
\pgfsetfillopacity{0.700000}%
\pgfsetlinewidth{0.000000pt}%
\definecolor{currentstroke}{rgb}{0.000000,0.000000,0.000000}%
\pgfsetstrokecolor{currentstroke}%
\pgfsetdash{}{0pt}%
\pgfpathmoveto{\pgfqpoint{4.220243in}{1.563681in}}%
\pgfpathlineto{\pgfqpoint{4.233790in}{1.558711in}}%
\pgfpathlineto{\pgfqpoint{4.247343in}{1.553766in}}%
\pgfpathlineto{\pgfqpoint{4.260902in}{1.548844in}}%
\pgfpathlineto{\pgfqpoint{4.274468in}{1.543946in}}%
\pgfpathlineto{\pgfqpoint{4.266749in}{1.540163in}}%
\pgfpathlineto{\pgfqpoint{4.259023in}{1.536644in}}%
\pgfpathlineto{\pgfqpoint{4.251292in}{1.533397in}}%
\pgfpathlineto{\pgfqpoint{4.243554in}{1.530428in}}%
\pgfpathlineto{\pgfqpoint{4.229970in}{1.535627in}}%
\pgfpathlineto{\pgfqpoint{4.216393in}{1.540850in}}%
\pgfpathlineto{\pgfqpoint{4.202821in}{1.546097in}}%
\pgfpathlineto{\pgfqpoint{4.189255in}{1.551367in}}%
\pgfpathlineto{\pgfqpoint{4.197012in}{1.554029in}}%
\pgfpathlineto{\pgfqpoint{4.204762in}{1.556974in}}%
\pgfpathlineto{\pgfqpoint{4.212506in}{1.560194in}}%
\pgfpathlineto{\pgfqpoint{4.220243in}{1.563681in}}%
\pgfpathclose%
\pgfusepath{fill}%
\end{pgfscope}%
\begin{pgfscope}%
\pgfpathrectangle{\pgfqpoint{1.254980in}{0.150000in}}{\pgfqpoint{5.490039in}{5.490039in}}%
\pgfusepath{clip}%
\pgfsetbuttcap%
\pgfsetroundjoin%
\definecolor{currentfill}{rgb}{0.125394,0.574318,0.549086}%
\pgfsetfillcolor{currentfill}%
\pgfsetfillopacity{0.700000}%
\pgfsetlinewidth{0.000000pt}%
\definecolor{currentstroke}{rgb}{0.000000,0.000000,0.000000}%
\pgfsetstrokecolor{currentstroke}%
\pgfsetdash{}{0pt}%
\pgfpathmoveto{\pgfqpoint{2.273244in}{2.770484in}}%
\pgfpathlineto{\pgfqpoint{2.286605in}{2.759019in}}%
\pgfpathlineto{\pgfqpoint{2.299966in}{2.747602in}}%
\pgfpathlineto{\pgfqpoint{2.313327in}{2.736234in}}%
\pgfpathlineto{\pgfqpoint{2.326689in}{2.724914in}}%
\pgfpathlineto{\pgfqpoint{2.317280in}{2.744784in}}%
\pgfpathlineto{\pgfqpoint{2.307831in}{2.765320in}}%
\pgfpathlineto{\pgfqpoint{2.298343in}{2.786534in}}%
\pgfpathlineto{\pgfqpoint{2.288815in}{2.808439in}}%
\pgfpathlineto{\pgfqpoint{2.275393in}{2.820184in}}%
\pgfpathlineto{\pgfqpoint{2.261970in}{2.831978in}}%
\pgfpathlineto{\pgfqpoint{2.248548in}{2.843820in}}%
\pgfpathlineto{\pgfqpoint{2.235127in}{2.855712in}}%
\pgfpathlineto{\pgfqpoint{2.244718in}{2.833374in}}%
\pgfpathlineto{\pgfqpoint{2.254267in}{2.811732in}}%
\pgfpathlineto{\pgfqpoint{2.263776in}{2.790773in}}%
\pgfpathlineto{\pgfqpoint{2.273244in}{2.770484in}}%
\pgfpathclose%
\pgfusepath{fill}%
\end{pgfscope}%
\begin{pgfscope}%
\pgfpathrectangle{\pgfqpoint{1.254980in}{0.150000in}}{\pgfqpoint{5.490039in}{5.490039in}}%
\pgfusepath{clip}%
\pgfsetbuttcap%
\pgfsetroundjoin%
\definecolor{currentfill}{rgb}{0.190631,0.407061,0.556089}%
\pgfsetfillcolor{currentfill}%
\pgfsetfillopacity{0.700000}%
\pgfsetlinewidth{0.000000pt}%
\definecolor{currentstroke}{rgb}{0.000000,0.000000,0.000000}%
\pgfsetstrokecolor{currentstroke}%
\pgfsetdash{}{0pt}%
\pgfpathmoveto{\pgfqpoint{2.790392in}{2.325263in}}%
\pgfpathlineto{\pgfqpoint{2.803748in}{2.315699in}}%
\pgfpathlineto{\pgfqpoint{2.817106in}{2.306171in}}%
\pgfpathlineto{\pgfqpoint{2.830466in}{2.296677in}}%
\pgfpathlineto{\pgfqpoint{2.843829in}{2.287218in}}%
\pgfpathlineto{\pgfqpoint{2.835030in}{2.301000in}}%
\pgfpathlineto{\pgfqpoint{2.826204in}{2.315358in}}%
\pgfpathlineto{\pgfqpoint{2.817349in}{2.330304in}}%
\pgfpathlineto{\pgfqpoint{2.808465in}{2.345850in}}%
\pgfpathlineto{\pgfqpoint{2.795053in}{2.355706in}}%
\pgfpathlineto{\pgfqpoint{2.781644in}{2.365596in}}%
\pgfpathlineto{\pgfqpoint{2.768236in}{2.375522in}}%
\pgfpathlineto{\pgfqpoint{2.754831in}{2.385484in}}%
\pgfpathlineto{\pgfqpoint{2.763765in}{2.369534in}}%
\pgfpathlineto{\pgfqpoint{2.772670in}{2.354188in}}%
\pgfpathlineto{\pgfqpoint{2.781546in}{2.339435in}}%
\pgfpathlineto{\pgfqpoint{2.790392in}{2.325263in}}%
\pgfpathclose%
\pgfusepath{fill}%
\end{pgfscope}%
\begin{pgfscope}%
\pgfpathrectangle{\pgfqpoint{1.254980in}{0.150000in}}{\pgfqpoint{5.490039in}{5.490039in}}%
\pgfusepath{clip}%
\pgfsetbuttcap%
\pgfsetroundjoin%
\definecolor{currentfill}{rgb}{0.283187,0.125848,0.444960}%
\pgfsetfillcolor{currentfill}%
\pgfsetfillopacity{0.700000}%
\pgfsetlinewidth{0.000000pt}%
\definecolor{currentstroke}{rgb}{0.000000,0.000000,0.000000}%
\pgfsetstrokecolor{currentstroke}%
\pgfsetdash{}{0pt}%
\pgfpathmoveto{\pgfqpoint{5.343559in}{1.725329in}}%
\pgfpathlineto{\pgfqpoint{5.357444in}{1.724062in}}%
\pgfpathlineto{\pgfqpoint{5.371338in}{1.722819in}}%
\pgfpathlineto{\pgfqpoint{5.385240in}{1.721599in}}%
\pgfpathlineto{\pgfqpoint{5.377833in}{1.710099in}}%
\pgfpathlineto{\pgfqpoint{5.370422in}{1.698580in}}%
\pgfpathlineto{\pgfqpoint{5.363005in}{1.687048in}}%
\pgfpathlineto{\pgfqpoint{5.355584in}{1.675505in}}%
\pgfpathlineto{\pgfqpoint{5.341677in}{1.676899in}}%
\pgfpathlineto{\pgfqpoint{5.327779in}{1.678317in}}%
\pgfpathlineto{\pgfqpoint{5.313889in}{1.679757in}}%
\pgfpathlineto{\pgfqpoint{5.321314in}{1.691166in}}%
\pgfpathlineto{\pgfqpoint{5.328734in}{1.702566in}}%
\pgfpathlineto{\pgfqpoint{5.336149in}{1.713955in}}%
\pgfpathlineto{\pgfqpoint{5.343559in}{1.725329in}}%
\pgfpathclose%
\pgfusepath{fill}%
\end{pgfscope}%
\begin{pgfscope}%
\pgfpathrectangle{\pgfqpoint{1.254980in}{0.150000in}}{\pgfqpoint{5.490039in}{5.490039in}}%
\pgfusepath{clip}%
\pgfsetbuttcap%
\pgfsetroundjoin%
\definecolor{currentfill}{rgb}{0.282656,0.100196,0.422160}%
\pgfsetfillcolor{currentfill}%
\pgfsetfillopacity{0.700000}%
\pgfsetlinewidth{0.000000pt}%
\definecolor{currentstroke}{rgb}{0.000000,0.000000,0.000000}%
\pgfsetstrokecolor{currentstroke}%
\pgfsetdash{}{0pt}%
\pgfpathmoveto{\pgfqpoint{3.887553in}{1.660442in}}%
\pgfpathlineto{\pgfqpoint{3.901034in}{1.654374in}}%
\pgfpathlineto{\pgfqpoint{3.914519in}{1.648331in}}%
\pgfpathlineto{\pgfqpoint{3.928010in}{1.642312in}}%
\pgfpathlineto{\pgfqpoint{3.941506in}{1.636318in}}%
\pgfpathlineto{\pgfqpoint{3.933622in}{1.636485in}}%
\pgfpathlineto{\pgfqpoint{3.925728in}{1.636999in}}%
\pgfpathlineto{\pgfqpoint{3.917824in}{1.637868in}}%
\pgfpathlineto{\pgfqpoint{3.909910in}{1.639102in}}%
\pgfpathlineto{\pgfqpoint{3.896389in}{1.645425in}}%
\pgfpathlineto{\pgfqpoint{3.882873in}{1.651772in}}%
\pgfpathlineto{\pgfqpoint{3.869361in}{1.658144in}}%
\pgfpathlineto{\pgfqpoint{3.855855in}{1.664541in}}%
\pgfpathlineto{\pgfqpoint{3.863795in}{1.662973in}}%
\pgfpathlineto{\pgfqpoint{3.871725in}{1.661773in}}%
\pgfpathlineto{\pgfqpoint{3.879644in}{1.660933in}}%
\pgfpathlineto{\pgfqpoint{3.887553in}{1.660442in}}%
\pgfpathclose%
\pgfusepath{fill}%
\end{pgfscope}%
\begin{pgfscope}%
\pgfpathrectangle{\pgfqpoint{1.254980in}{0.150000in}}{\pgfqpoint{5.490039in}{5.490039in}}%
\pgfusepath{clip}%
\pgfsetbuttcap%
\pgfsetroundjoin%
\definecolor{currentfill}{rgb}{0.271828,0.209303,0.504434}%
\pgfsetfillcolor{currentfill}%
\pgfsetfillopacity{0.700000}%
\pgfsetlinewidth{0.000000pt}%
\definecolor{currentstroke}{rgb}{0.000000,0.000000,0.000000}%
\pgfsetstrokecolor{currentstroke}%
\pgfsetdash{}{0pt}%
\pgfpathmoveto{\pgfqpoint{3.446843in}{1.873752in}}%
\pgfpathlineto{\pgfqpoint{3.460255in}{1.866286in}}%
\pgfpathlineto{\pgfqpoint{3.473672in}{1.858847in}}%
\pgfpathlineto{\pgfqpoint{3.487092in}{1.851435in}}%
\pgfpathlineto{\pgfqpoint{3.500517in}{1.844050in}}%
\pgfpathlineto{\pgfqpoint{3.492332in}{1.849712in}}%
\pgfpathlineto{\pgfqpoint{3.484132in}{1.855820in}}%
\pgfpathlineto{\pgfqpoint{3.475915in}{1.862385in}}%
\pgfpathlineto{\pgfqpoint{3.467682in}{1.869418in}}%
\pgfpathlineto{\pgfqpoint{3.454223in}{1.877161in}}%
\pgfpathlineto{\pgfqpoint{3.440768in}{1.884932in}}%
\pgfpathlineto{\pgfqpoint{3.427316in}{1.892730in}}%
\pgfpathlineto{\pgfqpoint{3.413869in}{1.900555in}}%
\pgfpathlineto{\pgfqpoint{3.422138in}{1.893158in}}%
\pgfpathlineto{\pgfqpoint{3.430389in}{1.886232in}}%
\pgfpathlineto{\pgfqpoint{3.438624in}{1.879766in}}%
\pgfpathlineto{\pgfqpoint{3.446843in}{1.873752in}}%
\pgfpathclose%
\pgfusepath{fill}%
\end{pgfscope}%
\begin{pgfscope}%
\pgfpathrectangle{\pgfqpoint{1.254980in}{0.150000in}}{\pgfqpoint{5.490039in}{5.490039in}}%
\pgfusepath{clip}%
\pgfsetbuttcap%
\pgfsetroundjoin%
\definecolor{currentfill}{rgb}{0.274952,0.037752,0.364543}%
\pgfsetfillcolor{currentfill}%
\pgfsetfillopacity{0.700000}%
\pgfsetlinewidth{0.000000pt}%
\definecolor{currentstroke}{rgb}{0.000000,0.000000,0.000000}%
\pgfsetstrokecolor{currentstroke}%
\pgfsetdash{}{0pt}%
\pgfpathmoveto{\pgfqpoint{4.863066in}{1.561258in}}%
\pgfpathlineto{\pgfqpoint{4.876790in}{1.558438in}}%
\pgfpathlineto{\pgfqpoint{4.890521in}{1.555642in}}%
\pgfpathlineto{\pgfqpoint{4.904260in}{1.552868in}}%
\pgfpathlineto{\pgfqpoint{4.918006in}{1.550118in}}%
\pgfpathlineto{\pgfqpoint{4.910485in}{1.540471in}}%
\pgfpathlineto{\pgfqpoint{4.902960in}{1.530923in}}%
\pgfpathlineto{\pgfqpoint{4.895431in}{1.521480in}}%
\pgfpathlineto{\pgfqpoint{4.887899in}{1.512148in}}%
\pgfpathlineto{\pgfqpoint{4.874145in}{1.515136in}}%
\pgfpathlineto{\pgfqpoint{4.860398in}{1.518147in}}%
\pgfpathlineto{\pgfqpoint{4.846658in}{1.521181in}}%
\pgfpathlineto{\pgfqpoint{4.832926in}{1.524238in}}%
\pgfpathlineto{\pgfqpoint{4.840466in}{1.533328in}}%
\pgfpathlineto{\pgfqpoint{4.848003in}{1.542532in}}%
\pgfpathlineto{\pgfqpoint{4.855536in}{1.551844in}}%
\pgfpathlineto{\pgfqpoint{4.863066in}{1.561258in}}%
\pgfpathclose%
\pgfusepath{fill}%
\end{pgfscope}%
\begin{pgfscope}%
\pgfpathrectangle{\pgfqpoint{1.254980in}{0.150000in}}{\pgfqpoint{5.490039in}{5.490039in}}%
\pgfusepath{clip}%
\pgfsetbuttcap%
\pgfsetroundjoin%
\definecolor{currentfill}{rgb}{0.271305,0.019942,0.347269}%
\pgfsetfillcolor{currentfill}%
\pgfsetfillopacity{0.700000}%
\pgfsetlinewidth{0.000000pt}%
\definecolor{currentstroke}{rgb}{0.000000,0.000000,0.000000}%
\pgfsetstrokecolor{currentstroke}%
\pgfsetdash{}{0pt}%
\pgfpathmoveto{\pgfqpoint{4.638384in}{1.530617in}}%
\pgfpathlineto{\pgfqpoint{4.652041in}{1.527034in}}%
\pgfpathlineto{\pgfqpoint{4.665706in}{1.523473in}}%
\pgfpathlineto{\pgfqpoint{4.679377in}{1.519936in}}%
\pgfpathlineto{\pgfqpoint{4.693054in}{1.516422in}}%
\pgfpathlineto{\pgfqpoint{4.685477in}{1.508509in}}%
\pgfpathlineto{\pgfqpoint{4.677895in}{1.500754in}}%
\pgfpathlineto{\pgfqpoint{4.670309in}{1.493163in}}%
\pgfpathlineto{\pgfqpoint{4.662720in}{1.485743in}}%
\pgfpathlineto{\pgfqpoint{4.649031in}{1.489520in}}%
\pgfpathlineto{\pgfqpoint{4.635349in}{1.493320in}}%
\pgfpathlineto{\pgfqpoint{4.621674in}{1.497143in}}%
\pgfpathlineto{\pgfqpoint{4.608005in}{1.500989in}}%
\pgfpathlineto{\pgfqpoint{4.615606in}{1.508141in}}%
\pgfpathlineto{\pgfqpoint{4.623203in}{1.515468in}}%
\pgfpathlineto{\pgfqpoint{4.630795in}{1.522962in}}%
\pgfpathlineto{\pgfqpoint{4.638384in}{1.530617in}}%
\pgfpathclose%
\pgfusepath{fill}%
\end{pgfscope}%
\begin{pgfscope}%
\pgfpathrectangle{\pgfqpoint{1.254980in}{0.150000in}}{\pgfqpoint{5.490039in}{5.490039in}}%
\pgfusepath{clip}%
\pgfsetbuttcap%
\pgfsetroundjoin%
\definecolor{currentfill}{rgb}{0.282910,0.105393,0.426902}%
\pgfsetfillcolor{currentfill}%
\pgfsetfillopacity{0.700000}%
\pgfsetlinewidth{0.000000pt}%
\definecolor{currentstroke}{rgb}{0.000000,0.000000,0.000000}%
\pgfsetstrokecolor{currentstroke}%
\pgfsetdash{}{0pt}%
\pgfpathmoveto{\pgfqpoint{5.258414in}{1.685753in}}%
\pgfpathlineto{\pgfqpoint{5.272270in}{1.684219in}}%
\pgfpathlineto{\pgfqpoint{5.286134in}{1.682709in}}%
\pgfpathlineto{\pgfqpoint{5.300007in}{1.681221in}}%
\pgfpathlineto{\pgfqpoint{5.313889in}{1.679757in}}%
\pgfpathlineto{\pgfqpoint{5.306460in}{1.668346in}}%
\pgfpathlineto{\pgfqpoint{5.299026in}{1.656935in}}%
\pgfpathlineto{\pgfqpoint{5.291588in}{1.645530in}}%
\pgfpathlineto{\pgfqpoint{5.284146in}{1.634134in}}%
\pgfpathlineto{\pgfqpoint{5.270259in}{1.635786in}}%
\pgfpathlineto{\pgfqpoint{5.256381in}{1.637460in}}%
\pgfpathlineto{\pgfqpoint{5.242512in}{1.639158in}}%
\pgfpathlineto{\pgfqpoint{5.228650in}{1.640879in}}%
\pgfpathlineto{\pgfqpoint{5.236098in}{1.652082in}}%
\pgfpathlineto{\pgfqpoint{5.243541in}{1.663298in}}%
\pgfpathlineto{\pgfqpoint{5.250979in}{1.674524in}}%
\pgfpathlineto{\pgfqpoint{5.258414in}{1.685753in}}%
\pgfpathclose%
\pgfusepath{fill}%
\end{pgfscope}%
\begin{pgfscope}%
\pgfpathrectangle{\pgfqpoint{1.254980in}{0.150000in}}{\pgfqpoint{5.490039in}{5.490039in}}%
\pgfusepath{clip}%
\pgfsetbuttcap%
\pgfsetroundjoin%
\definecolor{currentfill}{rgb}{0.282290,0.145912,0.461510}%
\pgfsetfillcolor{currentfill}%
\pgfsetfillopacity{0.700000}%
\pgfsetlinewidth{0.000000pt}%
\definecolor{currentstroke}{rgb}{0.000000,0.000000,0.000000}%
\pgfsetstrokecolor{currentstroke}%
\pgfsetdash{}{0pt}%
\pgfpathmoveto{\pgfqpoint{3.694151in}{1.743256in}}%
\pgfpathlineto{\pgfqpoint{3.707601in}{1.736556in}}%
\pgfpathlineto{\pgfqpoint{3.721055in}{1.729883in}}%
\pgfpathlineto{\pgfqpoint{3.734514in}{1.723235in}}%
\pgfpathlineto{\pgfqpoint{3.747978in}{1.716612in}}%
\pgfpathlineto{\pgfqpoint{3.739972in}{1.719232in}}%
\pgfpathlineto{\pgfqpoint{3.731953in}{1.722246in}}%
\pgfpathlineto{\pgfqpoint{3.723923in}{1.725662in}}%
\pgfpathlineto{\pgfqpoint{3.715879in}{1.729491in}}%
\pgfpathlineto{\pgfqpoint{3.702387in}{1.736457in}}%
\pgfpathlineto{\pgfqpoint{3.688899in}{1.743448in}}%
\pgfpathlineto{\pgfqpoint{3.675415in}{1.750464in}}%
\pgfpathlineto{\pgfqpoint{3.661935in}{1.757507in}}%
\pgfpathlineto{\pgfqpoint{3.670009in}{1.753329in}}%
\pgfpathlineto{\pgfqpoint{3.678069in}{1.749568in}}%
\pgfpathlineto{\pgfqpoint{3.686117in}{1.746213in}}%
\pgfpathlineto{\pgfqpoint{3.694151in}{1.743256in}}%
\pgfpathclose%
\pgfusepath{fill}%
\end{pgfscope}%
\begin{pgfscope}%
\pgfpathrectangle{\pgfqpoint{1.254980in}{0.150000in}}{\pgfqpoint{5.490039in}{5.490039in}}%
\pgfusepath{clip}%
\pgfsetbuttcap%
\pgfsetroundjoin%
\definecolor{currentfill}{rgb}{0.129933,0.559582,0.551864}%
\pgfsetfillcolor{currentfill}%
\pgfsetfillopacity{0.700000}%
\pgfsetlinewidth{0.000000pt}%
\definecolor{currentstroke}{rgb}{0.000000,0.000000,0.000000}%
\pgfsetstrokecolor{currentstroke}%
\pgfsetdash{}{0pt}%
\pgfpathmoveto{\pgfqpoint{2.326689in}{2.724914in}}%
\pgfpathlineto{\pgfqpoint{2.340051in}{2.713641in}}%
\pgfpathlineto{\pgfqpoint{2.353414in}{2.702416in}}%
\pgfpathlineto{\pgfqpoint{2.366778in}{2.691237in}}%
\pgfpathlineto{\pgfqpoint{2.380142in}{2.680104in}}%
\pgfpathlineto{\pgfqpoint{2.370791in}{2.699557in}}%
\pgfpathlineto{\pgfqpoint{2.361403in}{2.719671in}}%
\pgfpathlineto{\pgfqpoint{2.351975in}{2.740458in}}%
\pgfpathlineto{\pgfqpoint{2.342509in}{2.761932in}}%
\pgfpathlineto{\pgfqpoint{2.329084in}{2.773488in}}%
\pgfpathlineto{\pgfqpoint{2.315661in}{2.785091in}}%
\pgfpathlineto{\pgfqpoint{2.302238in}{2.796741in}}%
\pgfpathlineto{\pgfqpoint{2.288815in}{2.808439in}}%
\pgfpathlineto{\pgfqpoint{2.298343in}{2.786534in}}%
\pgfpathlineto{\pgfqpoint{2.307831in}{2.765320in}}%
\pgfpathlineto{\pgfqpoint{2.317280in}{2.744784in}}%
\pgfpathlineto{\pgfqpoint{2.326689in}{2.724914in}}%
\pgfpathclose%
\pgfusepath{fill}%
\end{pgfscope}%
\begin{pgfscope}%
\pgfpathrectangle{\pgfqpoint{1.254980in}{0.150000in}}{\pgfqpoint{5.490039in}{5.490039in}}%
\pgfusepath{clip}%
\pgfsetbuttcap%
\pgfsetroundjoin%
\definecolor{currentfill}{rgb}{0.278791,0.062145,0.386592}%
\pgfsetfillcolor{currentfill}%
\pgfsetfillopacity{0.700000}%
\pgfsetlinewidth{0.000000pt}%
\definecolor{currentstroke}{rgb}{0.000000,0.000000,0.000000}%
\pgfsetstrokecolor{currentstroke}%
\pgfsetdash{}{0pt}%
\pgfpathmoveto{\pgfqpoint{4.080928in}{1.594385in}}%
\pgfpathlineto{\pgfqpoint{4.094450in}{1.588924in}}%
\pgfpathlineto{\pgfqpoint{4.107977in}{1.583487in}}%
\pgfpathlineto{\pgfqpoint{4.121509in}{1.578074in}}%
\pgfpathlineto{\pgfqpoint{4.135047in}{1.572685in}}%
\pgfpathlineto{\pgfqpoint{4.127263in}{1.570623in}}%
\pgfpathlineto{\pgfqpoint{4.119471in}{1.568863in}}%
\pgfpathlineto{\pgfqpoint{4.111672in}{1.567414in}}%
\pgfpathlineto{\pgfqpoint{4.103865in}{1.566284in}}%
\pgfpathlineto{\pgfqpoint{4.090306in}{1.571988in}}%
\pgfpathlineto{\pgfqpoint{4.076752in}{1.577715in}}%
\pgfpathlineto{\pgfqpoint{4.063204in}{1.583467in}}%
\pgfpathlineto{\pgfqpoint{4.049661in}{1.589242in}}%
\pgfpathlineto{\pgfqpoint{4.057490in}{1.590053in}}%
\pgfpathlineto{\pgfqpoint{4.065311in}{1.591185in}}%
\pgfpathlineto{\pgfqpoint{4.073123in}{1.592632in}}%
\pgfpathlineto{\pgfqpoint{4.080928in}{1.594385in}}%
\pgfpathclose%
\pgfusepath{fill}%
\end{pgfscope}%
\begin{pgfscope}%
\pgfpathrectangle{\pgfqpoint{1.254980in}{0.150000in}}{\pgfqpoint{5.490039in}{5.490039in}}%
\pgfusepath{clip}%
\pgfsetbuttcap%
\pgfsetroundjoin%
\definecolor{currentfill}{rgb}{0.281924,0.089666,0.412415}%
\pgfsetfillcolor{currentfill}%
\pgfsetfillopacity{0.700000}%
\pgfsetlinewidth{0.000000pt}%
\definecolor{currentstroke}{rgb}{0.000000,0.000000,0.000000}%
\pgfsetstrokecolor{currentstroke}%
\pgfsetdash{}{0pt}%
\pgfpathmoveto{\pgfqpoint{5.173287in}{1.647994in}}%
\pgfpathlineto{\pgfqpoint{5.187116in}{1.646180in}}%
\pgfpathlineto{\pgfqpoint{5.200952in}{1.644390in}}%
\pgfpathlineto{\pgfqpoint{5.214797in}{1.642623in}}%
\pgfpathlineto{\pgfqpoint{5.228650in}{1.640879in}}%
\pgfpathlineto{\pgfqpoint{5.221199in}{1.629693in}}%
\pgfpathlineto{\pgfqpoint{5.213743in}{1.618530in}}%
\pgfpathlineto{\pgfqpoint{5.206284in}{1.607395in}}%
\pgfpathlineto{\pgfqpoint{5.198821in}{1.596291in}}%
\pgfpathlineto{\pgfqpoint{5.184962in}{1.598235in}}%
\pgfpathlineto{\pgfqpoint{5.171112in}{1.600202in}}%
\pgfpathlineto{\pgfqpoint{5.157270in}{1.602193in}}%
\pgfpathlineto{\pgfqpoint{5.143436in}{1.604206in}}%
\pgfpathlineto{\pgfqpoint{5.150904in}{1.615105in}}%
\pgfpathlineto{\pgfqpoint{5.158369in}{1.626039in}}%
\pgfpathlineto{\pgfqpoint{5.165830in}{1.637003in}}%
\pgfpathlineto{\pgfqpoint{5.173287in}{1.647994in}}%
\pgfpathclose%
\pgfusepath{fill}%
\end{pgfscope}%
\begin{pgfscope}%
\pgfpathrectangle{\pgfqpoint{1.254980in}{0.150000in}}{\pgfqpoint{5.490039in}{5.490039in}}%
\pgfusepath{clip}%
\pgfsetbuttcap%
\pgfsetroundjoin%
\definecolor{currentfill}{rgb}{0.195860,0.395433,0.555276}%
\pgfsetfillcolor{currentfill}%
\pgfsetfillopacity{0.700000}%
\pgfsetlinewidth{0.000000pt}%
\definecolor{currentstroke}{rgb}{0.000000,0.000000,0.000000}%
\pgfsetstrokecolor{currentstroke}%
\pgfsetdash{}{0pt}%
\pgfpathmoveto{\pgfqpoint{2.843829in}{2.287218in}}%
\pgfpathlineto{\pgfqpoint{2.857194in}{2.277794in}}%
\pgfpathlineto{\pgfqpoint{2.870561in}{2.268405in}}%
\pgfpathlineto{\pgfqpoint{2.883931in}{2.259050in}}%
\pgfpathlineto{\pgfqpoint{2.897303in}{2.249728in}}%
\pgfpathlineto{\pgfqpoint{2.888552in}{2.263119in}}%
\pgfpathlineto{\pgfqpoint{2.879773in}{2.277083in}}%
\pgfpathlineto{\pgfqpoint{2.870968in}{2.291630in}}%
\pgfpathlineto{\pgfqpoint{2.862134in}{2.306774in}}%
\pgfpathlineto{\pgfqpoint{2.848713in}{2.316491in}}%
\pgfpathlineto{\pgfqpoint{2.835295in}{2.326243in}}%
\pgfpathlineto{\pgfqpoint{2.821879in}{2.336029in}}%
\pgfpathlineto{\pgfqpoint{2.808465in}{2.345850in}}%
\pgfpathlineto{\pgfqpoint{2.817349in}{2.330304in}}%
\pgfpathlineto{\pgfqpoint{2.826204in}{2.315358in}}%
\pgfpathlineto{\pgfqpoint{2.835030in}{2.301000in}}%
\pgfpathlineto{\pgfqpoint{2.843829in}{2.287218in}}%
\pgfpathclose%
\pgfusepath{fill}%
\end{pgfscope}%
\begin{pgfscope}%
\pgfpathrectangle{\pgfqpoint{1.254980in}{0.150000in}}{\pgfqpoint{5.490039in}{5.490039in}}%
\pgfusepath{clip}%
\pgfsetbuttcap%
\pgfsetroundjoin%
\definecolor{currentfill}{rgb}{0.244972,0.287675,0.537260}%
\pgfsetfillcolor{currentfill}%
\pgfsetfillopacity{0.700000}%
\pgfsetlinewidth{0.000000pt}%
\definecolor{currentstroke}{rgb}{0.000000,0.000000,0.000000}%
\pgfsetstrokecolor{currentstroke}%
\pgfsetdash{}{0pt}%
\pgfpathmoveto{\pgfqpoint{3.199203in}{2.029582in}}%
\pgfpathlineto{\pgfqpoint{3.212593in}{2.021301in}}%
\pgfpathlineto{\pgfqpoint{3.225987in}{2.013050in}}%
\pgfpathlineto{\pgfqpoint{3.239384in}{2.004828in}}%
\pgfpathlineto{\pgfqpoint{3.252785in}{1.996636in}}%
\pgfpathlineto{\pgfqpoint{3.244384in}{2.005622in}}%
\pgfpathlineto{\pgfqpoint{3.235963in}{2.015111in}}%
\pgfpathlineto{\pgfqpoint{3.227522in}{2.025115in}}%
\pgfpathlineto{\pgfqpoint{3.219059in}{2.035644in}}%
\pgfpathlineto{\pgfqpoint{3.205618in}{2.044212in}}%
\pgfpathlineto{\pgfqpoint{3.192181in}{2.052810in}}%
\pgfpathlineto{\pgfqpoint{3.178746in}{2.061437in}}%
\pgfpathlineto{\pgfqpoint{3.165315in}{2.070094in}}%
\pgfpathlineto{\pgfqpoint{3.173819in}{2.059183in}}%
\pgfpathlineto{\pgfqpoint{3.182301in}{2.048801in}}%
\pgfpathlineto{\pgfqpoint{3.190762in}{2.038938in}}%
\pgfpathlineto{\pgfqpoint{3.199203in}{2.029582in}}%
\pgfpathclose%
\pgfusepath{fill}%
\end{pgfscope}%
\begin{pgfscope}%
\pgfpathrectangle{\pgfqpoint{1.254980in}{0.150000in}}{\pgfqpoint{5.490039in}{5.490039in}}%
\pgfusepath{clip}%
\pgfsetbuttcap%
\pgfsetroundjoin%
\definecolor{currentfill}{rgb}{0.280267,0.073417,0.397163}%
\pgfsetfillcolor{currentfill}%
\pgfsetfillopacity{0.700000}%
\pgfsetlinewidth{0.000000pt}%
\definecolor{currentstroke}{rgb}{0.000000,0.000000,0.000000}%
\pgfsetstrokecolor{currentstroke}%
\pgfsetdash{}{0pt}%
\pgfpathmoveto{\pgfqpoint{5.088179in}{1.612490in}}%
\pgfpathlineto{\pgfqpoint{5.101981in}{1.610385in}}%
\pgfpathlineto{\pgfqpoint{5.115791in}{1.608302in}}%
\pgfpathlineto{\pgfqpoint{5.129610in}{1.606243in}}%
\pgfpathlineto{\pgfqpoint{5.143436in}{1.604206in}}%
\pgfpathlineto{\pgfqpoint{5.135963in}{1.593348in}}%
\pgfpathlineto{\pgfqpoint{5.128487in}{1.582535in}}%
\pgfpathlineto{\pgfqpoint{5.121007in}{1.571772in}}%
\pgfpathlineto{\pgfqpoint{5.113523in}{1.561066in}}%
\pgfpathlineto{\pgfqpoint{5.099691in}{1.563315in}}%
\pgfpathlineto{\pgfqpoint{5.085867in}{1.565587in}}%
\pgfpathlineto{\pgfqpoint{5.072051in}{1.567882in}}%
\pgfpathlineto{\pgfqpoint{5.058243in}{1.570201in}}%
\pgfpathlineto{\pgfqpoint{5.065732in}{1.580690in}}%
\pgfpathlineto{\pgfqpoint{5.073218in}{1.591238in}}%
\pgfpathlineto{\pgfqpoint{5.080701in}{1.601840in}}%
\pgfpathlineto{\pgfqpoint{5.088179in}{1.612490in}}%
\pgfpathclose%
\pgfusepath{fill}%
\end{pgfscope}%
\begin{pgfscope}%
\pgfpathrectangle{\pgfqpoint{1.254980in}{0.150000in}}{\pgfqpoint{5.490039in}{5.490039in}}%
\pgfusepath{clip}%
\pgfsetbuttcap%
\pgfsetroundjoin%
\definecolor{currentfill}{rgb}{0.272594,0.025563,0.353093}%
\pgfsetfillcolor{currentfill}%
\pgfsetfillopacity{0.700000}%
\pgfsetlinewidth{0.000000pt}%
\definecolor{currentstroke}{rgb}{0.000000,0.000000,0.000000}%
\pgfsetstrokecolor{currentstroke}%
\pgfsetdash{}{0pt}%
\pgfpathmoveto{\pgfqpoint{4.778069in}{1.536696in}}%
\pgfpathlineto{\pgfqpoint{4.791773in}{1.533547in}}%
\pgfpathlineto{\pgfqpoint{4.805483in}{1.530421in}}%
\pgfpathlineto{\pgfqpoint{4.819201in}{1.527318in}}%
\pgfpathlineto{\pgfqpoint{4.832926in}{1.524238in}}%
\pgfpathlineto{\pgfqpoint{4.825382in}{1.515267in}}%
\pgfpathlineto{\pgfqpoint{4.817835in}{1.506423in}}%
\pgfpathlineto{\pgfqpoint{4.810284in}{1.497710in}}%
\pgfpathlineto{\pgfqpoint{4.802730in}{1.489136in}}%
\pgfpathlineto{\pgfqpoint{4.788996in}{1.492466in}}%
\pgfpathlineto{\pgfqpoint{4.775269in}{1.495820in}}%
\pgfpathlineto{\pgfqpoint{4.761549in}{1.499196in}}%
\pgfpathlineto{\pgfqpoint{4.747836in}{1.502595in}}%
\pgfpathlineto{\pgfqpoint{4.755400in}{1.510914in}}%
\pgfpathlineto{\pgfqpoint{4.762960in}{1.519375in}}%
\pgfpathlineto{\pgfqpoint{4.770517in}{1.527971in}}%
\pgfpathlineto{\pgfqpoint{4.778069in}{1.536696in}}%
\pgfpathclose%
\pgfusepath{fill}%
\end{pgfscope}%
\begin{pgfscope}%
\pgfpathrectangle{\pgfqpoint{1.254980in}{0.150000in}}{\pgfqpoint{5.490039in}{5.490039in}}%
\pgfusepath{clip}%
\pgfsetbuttcap%
\pgfsetroundjoin%
\definecolor{currentfill}{rgb}{0.135066,0.544853,0.554029}%
\pgfsetfillcolor{currentfill}%
\pgfsetfillopacity{0.700000}%
\pgfsetlinewidth{0.000000pt}%
\definecolor{currentstroke}{rgb}{0.000000,0.000000,0.000000}%
\pgfsetstrokecolor{currentstroke}%
\pgfsetdash{}{0pt}%
\pgfpathmoveto{\pgfqpoint{2.380142in}{2.680104in}}%
\pgfpathlineto{\pgfqpoint{2.393507in}{2.669017in}}%
\pgfpathlineto{\pgfqpoint{2.406873in}{2.657975in}}%
\pgfpathlineto{\pgfqpoint{2.420240in}{2.646977in}}%
\pgfpathlineto{\pgfqpoint{2.433608in}{2.636024in}}%
\pgfpathlineto{\pgfqpoint{2.424315in}{2.655062in}}%
\pgfpathlineto{\pgfqpoint{2.414985in}{2.674755in}}%
\pgfpathlineto{\pgfqpoint{2.405617in}{2.695117in}}%
\pgfpathlineto{\pgfqpoint{2.396211in}{2.716160in}}%
\pgfpathlineto{\pgfqpoint{2.382784in}{2.727536in}}%
\pgfpathlineto{\pgfqpoint{2.369358in}{2.738956in}}%
\pgfpathlineto{\pgfqpoint{2.355933in}{2.750421in}}%
\pgfpathlineto{\pgfqpoint{2.342509in}{2.761932in}}%
\pgfpathlineto{\pgfqpoint{2.351975in}{2.740458in}}%
\pgfpathlineto{\pgfqpoint{2.361403in}{2.719671in}}%
\pgfpathlineto{\pgfqpoint{2.370791in}{2.699557in}}%
\pgfpathlineto{\pgfqpoint{2.380142in}{2.680104in}}%
\pgfpathclose%
\pgfusepath{fill}%
\end{pgfscope}%
\begin{pgfscope}%
\pgfpathrectangle{\pgfqpoint{1.254980in}{0.150000in}}{\pgfqpoint{5.490039in}{5.490039in}}%
\pgfusepath{clip}%
\pgfsetbuttcap%
\pgfsetroundjoin%
\definecolor{currentfill}{rgb}{0.272594,0.025563,0.353093}%
\pgfsetfillcolor{currentfill}%
\pgfsetfillopacity{0.700000}%
\pgfsetlinewidth{0.000000pt}%
\definecolor{currentstroke}{rgb}{0.000000,0.000000,0.000000}%
\pgfsetstrokecolor{currentstroke}%
\pgfsetdash{}{0pt}%
\pgfpathmoveto{\pgfqpoint{4.413891in}{1.525522in}}%
\pgfpathlineto{\pgfqpoint{4.427495in}{1.521122in}}%
\pgfpathlineto{\pgfqpoint{4.441106in}{1.516746in}}%
\pgfpathlineto{\pgfqpoint{4.454722in}{1.512394in}}%
\pgfpathlineto{\pgfqpoint{4.468345in}{1.508064in}}%
\pgfpathlineto{\pgfqpoint{4.460696in}{1.502455in}}%
\pgfpathlineto{\pgfqpoint{4.453042in}{1.497069in}}%
\pgfpathlineto{\pgfqpoint{4.445382in}{1.491913in}}%
\pgfpathlineto{\pgfqpoint{4.437718in}{1.486996in}}%
\pgfpathlineto{\pgfqpoint{4.424080in}{1.491614in}}%
\pgfpathlineto{\pgfqpoint{4.410449in}{1.496254in}}%
\pgfpathlineto{\pgfqpoint{4.396823in}{1.500919in}}%
\pgfpathlineto{\pgfqpoint{4.383204in}{1.505606in}}%
\pgfpathlineto{\pgfqpoint{4.390884in}{1.510230in}}%
\pgfpathlineto{\pgfqpoint{4.398558in}{1.515096in}}%
\pgfpathlineto{\pgfqpoint{4.406227in}{1.520195in}}%
\pgfpathlineto{\pgfqpoint{4.413891in}{1.525522in}}%
\pgfpathclose%
\pgfusepath{fill}%
\end{pgfscope}%
\begin{pgfscope}%
\pgfpathrectangle{\pgfqpoint{1.254980in}{0.150000in}}{\pgfqpoint{5.490039in}{5.490039in}}%
\pgfusepath{clip}%
\pgfsetbuttcap%
\pgfsetroundjoin%
\definecolor{currentfill}{rgb}{0.274128,0.199721,0.498911}%
\pgfsetfillcolor{currentfill}%
\pgfsetfillopacity{0.700000}%
\pgfsetlinewidth{0.000000pt}%
\definecolor{currentstroke}{rgb}{0.000000,0.000000,0.000000}%
\pgfsetstrokecolor{currentstroke}%
\pgfsetdash{}{0pt}%
\pgfpathmoveto{\pgfqpoint{3.500517in}{1.844050in}}%
\pgfpathlineto{\pgfqpoint{3.513945in}{1.836693in}}%
\pgfpathlineto{\pgfqpoint{3.527378in}{1.829362in}}%
\pgfpathlineto{\pgfqpoint{3.540815in}{1.822058in}}%
\pgfpathlineto{\pgfqpoint{3.554255in}{1.814780in}}%
\pgfpathlineto{\pgfqpoint{3.546104in}{1.820089in}}%
\pgfpathlineto{\pgfqpoint{3.537938in}{1.825840in}}%
\pgfpathlineto{\pgfqpoint{3.529755in}{1.832045in}}%
\pgfpathlineto{\pgfqpoint{3.521557in}{1.838713in}}%
\pgfpathlineto{\pgfqpoint{3.508083in}{1.846349in}}%
\pgfpathlineto{\pgfqpoint{3.494612in}{1.854012in}}%
\pgfpathlineto{\pgfqpoint{3.481145in}{1.861701in}}%
\pgfpathlineto{\pgfqpoint{3.467682in}{1.869418in}}%
\pgfpathlineto{\pgfqpoint{3.475915in}{1.862385in}}%
\pgfpathlineto{\pgfqpoint{3.484132in}{1.855820in}}%
\pgfpathlineto{\pgfqpoint{3.492332in}{1.849712in}}%
\pgfpathlineto{\pgfqpoint{3.500517in}{1.844050in}}%
\pgfpathclose%
\pgfusepath{fill}%
\end{pgfscope}%
\begin{pgfscope}%
\pgfpathrectangle{\pgfqpoint{1.254980in}{0.150000in}}{\pgfqpoint{5.490039in}{5.490039in}}%
\pgfusepath{clip}%
\pgfsetbuttcap%
\pgfsetroundjoin%
\definecolor{currentfill}{rgb}{0.274952,0.037752,0.364543}%
\pgfsetfillcolor{currentfill}%
\pgfsetfillopacity{0.700000}%
\pgfsetlinewidth{0.000000pt}%
\definecolor{currentstroke}{rgb}{0.000000,0.000000,0.000000}%
\pgfsetstrokecolor{currentstroke}%
\pgfsetdash{}{0pt}%
\pgfpathmoveto{\pgfqpoint{4.274468in}{1.543946in}}%
\pgfpathlineto{\pgfqpoint{4.288039in}{1.539071in}}%
\pgfpathlineto{\pgfqpoint{4.301616in}{1.534220in}}%
\pgfpathlineto{\pgfqpoint{4.315199in}{1.529393in}}%
\pgfpathlineto{\pgfqpoint{4.328788in}{1.524589in}}%
\pgfpathlineto{\pgfqpoint{4.321086in}{1.520510in}}%
\pgfpathlineto{\pgfqpoint{4.313378in}{1.516691in}}%
\pgfpathlineto{\pgfqpoint{4.305665in}{1.513141in}}%
\pgfpathlineto{\pgfqpoint{4.297945in}{1.509866in}}%
\pgfpathlineto{\pgfqpoint{4.284338in}{1.514971in}}%
\pgfpathlineto{\pgfqpoint{4.270738in}{1.520100in}}%
\pgfpathlineto{\pgfqpoint{4.257143in}{1.525252in}}%
\pgfpathlineto{\pgfqpoint{4.243554in}{1.530428in}}%
\pgfpathlineto{\pgfqpoint{4.251292in}{1.533397in}}%
\pgfpathlineto{\pgfqpoint{4.259023in}{1.536644in}}%
\pgfpathlineto{\pgfqpoint{4.266749in}{1.540163in}}%
\pgfpathlineto{\pgfqpoint{4.274468in}{1.543946in}}%
\pgfpathclose%
\pgfusepath{fill}%
\end{pgfscope}%
\begin{pgfscope}%
\pgfpathrectangle{\pgfqpoint{1.254980in}{0.150000in}}{\pgfqpoint{5.490039in}{5.490039in}}%
\pgfusepath{clip}%
\pgfsetbuttcap%
\pgfsetroundjoin%
\definecolor{currentfill}{rgb}{0.271305,0.019942,0.347269}%
\pgfsetfillcolor{currentfill}%
\pgfsetfillopacity{0.700000}%
\pgfsetlinewidth{0.000000pt}%
\definecolor{currentstroke}{rgb}{0.000000,0.000000,0.000000}%
\pgfsetstrokecolor{currentstroke}%
\pgfsetdash{}{0pt}%
\pgfpathmoveto{\pgfqpoint{4.553397in}{1.516604in}}%
\pgfpathlineto{\pgfqpoint{4.567039in}{1.512665in}}%
\pgfpathlineto{\pgfqpoint{4.580688in}{1.508750in}}%
\pgfpathlineto{\pgfqpoint{4.594343in}{1.504858in}}%
\pgfpathlineto{\pgfqpoint{4.608005in}{1.500989in}}%
\pgfpathlineto{\pgfqpoint{4.600400in}{1.494018in}}%
\pgfpathlineto{\pgfqpoint{4.592791in}{1.487235in}}%
\pgfpathlineto{\pgfqpoint{4.585177in}{1.480647in}}%
\pgfpathlineto{\pgfqpoint{4.577560in}{1.474261in}}%
\pgfpathlineto{\pgfqpoint{4.563885in}{1.478406in}}%
\pgfpathlineto{\pgfqpoint{4.550217in}{1.482573in}}%
\pgfpathlineto{\pgfqpoint{4.536556in}{1.486764in}}%
\pgfpathlineto{\pgfqpoint{4.522901in}{1.490978in}}%
\pgfpathlineto{\pgfqpoint{4.530531in}{1.497083in}}%
\pgfpathlineto{\pgfqpoint{4.538158in}{1.503394in}}%
\pgfpathlineto{\pgfqpoint{4.545779in}{1.509903in}}%
\pgfpathlineto{\pgfqpoint{4.553397in}{1.516604in}}%
\pgfpathclose%
\pgfusepath{fill}%
\end{pgfscope}%
\begin{pgfscope}%
\pgfpathrectangle{\pgfqpoint{1.254980in}{0.150000in}}{\pgfqpoint{5.490039in}{5.490039in}}%
\pgfusepath{clip}%
\pgfsetbuttcap%
\pgfsetroundjoin%
\definecolor{currentfill}{rgb}{0.282327,0.094955,0.417331}%
\pgfsetfillcolor{currentfill}%
\pgfsetfillopacity{0.700000}%
\pgfsetlinewidth{0.000000pt}%
\definecolor{currentstroke}{rgb}{0.000000,0.000000,0.000000}%
\pgfsetstrokecolor{currentstroke}%
\pgfsetdash{}{0pt}%
\pgfpathmoveto{\pgfqpoint{3.941506in}{1.636318in}}%
\pgfpathlineto{\pgfqpoint{3.955007in}{1.630348in}}%
\pgfpathlineto{\pgfqpoint{3.968514in}{1.624403in}}%
\pgfpathlineto{\pgfqpoint{3.982025in}{1.618482in}}%
\pgfpathlineto{\pgfqpoint{3.995542in}{1.612586in}}%
\pgfpathlineto{\pgfqpoint{3.987681in}{1.612430in}}%
\pgfpathlineto{\pgfqpoint{3.979811in}{1.612617in}}%
\pgfpathlineto{\pgfqpoint{3.971932in}{1.613156in}}%
\pgfpathlineto{\pgfqpoint{3.964044in}{1.614057in}}%
\pgfpathlineto{\pgfqpoint{3.950503in}{1.620282in}}%
\pgfpathlineto{\pgfqpoint{3.936967in}{1.626531in}}%
\pgfpathlineto{\pgfqpoint{3.923436in}{1.632804in}}%
\pgfpathlineto{\pgfqpoint{3.909910in}{1.639102in}}%
\pgfpathlineto{\pgfqpoint{3.917824in}{1.637868in}}%
\pgfpathlineto{\pgfqpoint{3.925728in}{1.636999in}}%
\pgfpathlineto{\pgfqpoint{3.933622in}{1.636485in}}%
\pgfpathlineto{\pgfqpoint{3.941506in}{1.636318in}}%
\pgfpathclose%
\pgfusepath{fill}%
\end{pgfscope}%
\begin{pgfscope}%
\pgfpathrectangle{\pgfqpoint{1.254980in}{0.150000in}}{\pgfqpoint{5.490039in}{5.490039in}}%
\pgfusepath{clip}%
\pgfsetbuttcap%
\pgfsetroundjoin%
\definecolor{currentfill}{rgb}{0.277941,0.056324,0.381191}%
\pgfsetfillcolor{currentfill}%
\pgfsetfillopacity{0.700000}%
\pgfsetlinewidth{0.000000pt}%
\definecolor{currentstroke}{rgb}{0.000000,0.000000,0.000000}%
\pgfsetstrokecolor{currentstroke}%
\pgfsetdash{}{0pt}%
\pgfpathmoveto{\pgfqpoint{5.003088in}{1.579704in}}%
\pgfpathlineto{\pgfqpoint{5.016865in}{1.577293in}}%
\pgfpathlineto{\pgfqpoint{5.030650in}{1.574906in}}%
\pgfpathlineto{\pgfqpoint{5.044442in}{1.572542in}}%
\pgfpathlineto{\pgfqpoint{5.058243in}{1.570201in}}%
\pgfpathlineto{\pgfqpoint{5.050749in}{1.559776in}}%
\pgfpathlineto{\pgfqpoint{5.043253in}{1.549420in}}%
\pgfpathlineto{\pgfqpoint{5.035752in}{1.539140in}}%
\pgfpathlineto{\pgfqpoint{5.028249in}{1.528941in}}%
\pgfpathlineto{\pgfqpoint{5.014442in}{1.531508in}}%
\pgfpathlineto{\pgfqpoint{5.000642in}{1.534097in}}%
\pgfpathlineto{\pgfqpoint{4.986851in}{1.536710in}}%
\pgfpathlineto{\pgfqpoint{4.973067in}{1.539346in}}%
\pgfpathlineto{\pgfqpoint{4.980577in}{1.549315in}}%
\pgfpathlineto{\pgfqpoint{4.988084in}{1.559368in}}%
\pgfpathlineto{\pgfqpoint{4.995588in}{1.569499in}}%
\pgfpathlineto{\pgfqpoint{5.003088in}{1.579704in}}%
\pgfpathclose%
\pgfusepath{fill}%
\end{pgfscope}%
\begin{pgfscope}%
\pgfpathrectangle{\pgfqpoint{1.254980in}{0.150000in}}{\pgfqpoint{5.490039in}{5.490039in}}%
\pgfusepath{clip}%
\pgfsetbuttcap%
\pgfsetroundjoin%
\definecolor{currentfill}{rgb}{0.201239,0.383670,0.554294}%
\pgfsetfillcolor{currentfill}%
\pgfsetfillopacity{0.700000}%
\pgfsetlinewidth{0.000000pt}%
\definecolor{currentstroke}{rgb}{0.000000,0.000000,0.000000}%
\pgfsetstrokecolor{currentstroke}%
\pgfsetdash{}{0pt}%
\pgfpathmoveto{\pgfqpoint{2.897303in}{2.249728in}}%
\pgfpathlineto{\pgfqpoint{2.910678in}{2.240440in}}%
\pgfpathlineto{\pgfqpoint{2.924055in}{2.231186in}}%
\pgfpathlineto{\pgfqpoint{2.937435in}{2.221965in}}%
\pgfpathlineto{\pgfqpoint{2.950817in}{2.212777in}}%
\pgfpathlineto{\pgfqpoint{2.942112in}{2.225779in}}%
\pgfpathlineto{\pgfqpoint{2.933381in}{2.239349in}}%
\pgfpathlineto{\pgfqpoint{2.924624in}{2.253498in}}%
\pgfpathlineto{\pgfqpoint{2.915839in}{2.268240in}}%
\pgfpathlineto{\pgfqpoint{2.902409in}{2.277823in}}%
\pgfpathlineto{\pgfqpoint{2.888982in}{2.287440in}}%
\pgfpathlineto{\pgfqpoint{2.875556in}{2.297090in}}%
\pgfpathlineto{\pgfqpoint{2.862134in}{2.306774in}}%
\pgfpathlineto{\pgfqpoint{2.870968in}{2.291630in}}%
\pgfpathlineto{\pgfqpoint{2.879773in}{2.277083in}}%
\pgfpathlineto{\pgfqpoint{2.888552in}{2.263119in}}%
\pgfpathlineto{\pgfqpoint{2.897303in}{2.249728in}}%
\pgfpathclose%
\pgfusepath{fill}%
\end{pgfscope}%
\begin{pgfscope}%
\pgfpathrectangle{\pgfqpoint{1.254980in}{0.150000in}}{\pgfqpoint{5.490039in}{5.490039in}}%
\pgfusepath{clip}%
\pgfsetbuttcap%
\pgfsetroundjoin%
\definecolor{currentfill}{rgb}{0.282884,0.135920,0.453427}%
\pgfsetfillcolor{currentfill}%
\pgfsetfillopacity{0.700000}%
\pgfsetlinewidth{0.000000pt}%
\definecolor{currentstroke}{rgb}{0.000000,0.000000,0.000000}%
\pgfsetstrokecolor{currentstroke}%
\pgfsetdash{}{0pt}%
\pgfpathmoveto{\pgfqpoint{3.747978in}{1.716612in}}%
\pgfpathlineto{\pgfqpoint{3.761446in}{1.710015in}}%
\pgfpathlineto{\pgfqpoint{3.774919in}{1.703444in}}%
\pgfpathlineto{\pgfqpoint{3.788396in}{1.696897in}}%
\pgfpathlineto{\pgfqpoint{3.801878in}{1.690376in}}%
\pgfpathlineto{\pgfqpoint{3.793900in}{1.692658in}}%
\pgfpathlineto{\pgfqpoint{3.785911in}{1.695331in}}%
\pgfpathlineto{\pgfqpoint{3.777909in}{1.698403in}}%
\pgfpathlineto{\pgfqpoint{3.769895in}{1.701883in}}%
\pgfpathlineto{\pgfqpoint{3.756385in}{1.708747in}}%
\pgfpathlineto{\pgfqpoint{3.742878in}{1.715637in}}%
\pgfpathlineto{\pgfqpoint{3.729377in}{1.722551in}}%
\pgfpathlineto{\pgfqpoint{3.715879in}{1.729491in}}%
\pgfpathlineto{\pgfqpoint{3.723923in}{1.725662in}}%
\pgfpathlineto{\pgfqpoint{3.731953in}{1.722246in}}%
\pgfpathlineto{\pgfqpoint{3.739972in}{1.719232in}}%
\pgfpathlineto{\pgfqpoint{3.747978in}{1.716612in}}%
\pgfpathclose%
\pgfusepath{fill}%
\end{pgfscope}%
\begin{pgfscope}%
\pgfpathrectangle{\pgfqpoint{1.254980in}{0.150000in}}{\pgfqpoint{5.490039in}{5.490039in}}%
\pgfusepath{clip}%
\pgfsetbuttcap%
\pgfsetroundjoin%
\definecolor{currentfill}{rgb}{0.140536,0.530132,0.555659}%
\pgfsetfillcolor{currentfill}%
\pgfsetfillopacity{0.700000}%
\pgfsetlinewidth{0.000000pt}%
\definecolor{currentstroke}{rgb}{0.000000,0.000000,0.000000}%
\pgfsetstrokecolor{currentstroke}%
\pgfsetdash{}{0pt}%
\pgfpathmoveto{\pgfqpoint{2.433608in}{2.636024in}}%
\pgfpathlineto{\pgfqpoint{2.446976in}{2.625115in}}%
\pgfpathlineto{\pgfqpoint{2.460346in}{2.614250in}}%
\pgfpathlineto{\pgfqpoint{2.473717in}{2.603428in}}%
\pgfpathlineto{\pgfqpoint{2.487089in}{2.592648in}}%
\pgfpathlineto{\pgfqpoint{2.477853in}{2.611271in}}%
\pgfpathlineto{\pgfqpoint{2.468581in}{2.630545in}}%
\pgfpathlineto{\pgfqpoint{2.459272in}{2.650483in}}%
\pgfpathlineto{\pgfqpoint{2.449925in}{2.671098in}}%
\pgfpathlineto{\pgfqpoint{2.436495in}{2.682299in}}%
\pgfpathlineto{\pgfqpoint{2.423066in}{2.693542in}}%
\pgfpathlineto{\pgfqpoint{2.409638in}{2.704829in}}%
\pgfpathlineto{\pgfqpoint{2.396211in}{2.716160in}}%
\pgfpathlineto{\pgfqpoint{2.405617in}{2.695117in}}%
\pgfpathlineto{\pgfqpoint{2.414985in}{2.674755in}}%
\pgfpathlineto{\pgfqpoint{2.424315in}{2.655062in}}%
\pgfpathlineto{\pgfqpoint{2.433608in}{2.636024in}}%
\pgfpathclose%
\pgfusepath{fill}%
\end{pgfscope}%
\begin{pgfscope}%
\pgfpathrectangle{\pgfqpoint{1.254980in}{0.150000in}}{\pgfqpoint{5.490039in}{5.490039in}}%
\pgfusepath{clip}%
\pgfsetbuttcap%
\pgfsetroundjoin%
\definecolor{currentfill}{rgb}{0.248629,0.278775,0.534556}%
\pgfsetfillcolor{currentfill}%
\pgfsetfillopacity{0.700000}%
\pgfsetlinewidth{0.000000pt}%
\definecolor{currentstroke}{rgb}{0.000000,0.000000,0.000000}%
\pgfsetstrokecolor{currentstroke}%
\pgfsetdash{}{0pt}%
\pgfpathmoveto{\pgfqpoint{3.252785in}{1.996636in}}%
\pgfpathlineto{\pgfqpoint{3.266189in}{1.988473in}}%
\pgfpathlineto{\pgfqpoint{3.279596in}{1.980338in}}%
\pgfpathlineto{\pgfqpoint{3.293007in}{1.972233in}}%
\pgfpathlineto{\pgfqpoint{3.306422in}{1.964156in}}%
\pgfpathlineto{\pgfqpoint{3.298060in}{1.972772in}}%
\pgfpathlineto{\pgfqpoint{3.289679in}{1.981888in}}%
\pgfpathlineto{\pgfqpoint{3.281278in}{1.991514in}}%
\pgfpathlineto{\pgfqpoint{3.272856in}{2.001661in}}%
\pgfpathlineto{\pgfqpoint{3.259402in}{2.010114in}}%
\pgfpathlineto{\pgfqpoint{3.245951in}{2.018595in}}%
\pgfpathlineto{\pgfqpoint{3.232504in}{2.027105in}}%
\pgfpathlineto{\pgfqpoint{3.219059in}{2.035644in}}%
\pgfpathlineto{\pgfqpoint{3.227522in}{2.025115in}}%
\pgfpathlineto{\pgfqpoint{3.235963in}{2.015111in}}%
\pgfpathlineto{\pgfqpoint{3.244384in}{2.005622in}}%
\pgfpathlineto{\pgfqpoint{3.252785in}{1.996636in}}%
\pgfpathclose%
\pgfusepath{fill}%
\end{pgfscope}%
\begin{pgfscope}%
\pgfpathrectangle{\pgfqpoint{1.254980in}{0.150000in}}{\pgfqpoint{5.490039in}{5.490039in}}%
\pgfusepath{clip}%
\pgfsetbuttcap%
\pgfsetroundjoin%
\definecolor{currentfill}{rgb}{0.278791,0.062145,0.386592}%
\pgfsetfillcolor{currentfill}%
\pgfsetfillopacity{0.700000}%
\pgfsetlinewidth{0.000000pt}%
\definecolor{currentstroke}{rgb}{0.000000,0.000000,0.000000}%
\pgfsetstrokecolor{currentstroke}%
\pgfsetdash{}{0pt}%
\pgfpathmoveto{\pgfqpoint{4.135047in}{1.572685in}}%
\pgfpathlineto{\pgfqpoint{4.148591in}{1.567320in}}%
\pgfpathlineto{\pgfqpoint{4.162140in}{1.561978in}}%
\pgfpathlineto{\pgfqpoint{4.175694in}{1.556661in}}%
\pgfpathlineto{\pgfqpoint{4.189255in}{1.551367in}}%
\pgfpathlineto{\pgfqpoint{4.181491in}{1.548995in}}%
\pgfpathlineto{\pgfqpoint{4.173720in}{1.546923in}}%
\pgfpathlineto{\pgfqpoint{4.165942in}{1.545158in}}%
\pgfpathlineto{\pgfqpoint{4.158156in}{1.543708in}}%
\pgfpathlineto{\pgfqpoint{4.144575in}{1.549316in}}%
\pgfpathlineto{\pgfqpoint{4.131000in}{1.554949in}}%
\pgfpathlineto{\pgfqpoint{4.117430in}{1.560604in}}%
\pgfpathlineto{\pgfqpoint{4.103865in}{1.566284in}}%
\pgfpathlineto{\pgfqpoint{4.111672in}{1.567414in}}%
\pgfpathlineto{\pgfqpoint{4.119471in}{1.568863in}}%
\pgfpathlineto{\pgfqpoint{4.127263in}{1.570623in}}%
\pgfpathlineto{\pgfqpoint{4.135047in}{1.572685in}}%
\pgfpathclose%
\pgfusepath{fill}%
\end{pgfscope}%
\begin{pgfscope}%
\pgfpathrectangle{\pgfqpoint{1.254980in}{0.150000in}}{\pgfqpoint{5.490039in}{5.490039in}}%
\pgfusepath{clip}%
\pgfsetbuttcap%
\pgfsetroundjoin%
\definecolor{currentfill}{rgb}{0.271305,0.019942,0.347269}%
\pgfsetfillcolor{currentfill}%
\pgfsetfillopacity{0.700000}%
\pgfsetlinewidth{0.000000pt}%
\definecolor{currentstroke}{rgb}{0.000000,0.000000,0.000000}%
\pgfsetstrokecolor{currentstroke}%
\pgfsetdash{}{0pt}%
\pgfpathmoveto{\pgfqpoint{4.693054in}{1.516422in}}%
\pgfpathlineto{\pgfqpoint{4.706739in}{1.512931in}}%
\pgfpathlineto{\pgfqpoint{4.720431in}{1.509462in}}%
\pgfpathlineto{\pgfqpoint{4.734130in}{1.506017in}}%
\pgfpathlineto{\pgfqpoint{4.747836in}{1.502595in}}%
\pgfpathlineto{\pgfqpoint{4.740268in}{1.494424in}}%
\pgfpathlineto{\pgfqpoint{4.732697in}{1.486407in}}%
\pgfpathlineto{\pgfqpoint{4.725123in}{1.478552in}}%
\pgfpathlineto{\pgfqpoint{4.717544in}{1.470865in}}%
\pgfpathlineto{\pgfqpoint{4.703828in}{1.474550in}}%
\pgfpathlineto{\pgfqpoint{4.690118in}{1.478258in}}%
\pgfpathlineto{\pgfqpoint{4.676416in}{1.481989in}}%
\pgfpathlineto{\pgfqpoint{4.662720in}{1.485743in}}%
\pgfpathlineto{\pgfqpoint{4.670309in}{1.493163in}}%
\pgfpathlineto{\pgfqpoint{4.677895in}{1.500754in}}%
\pgfpathlineto{\pgfqpoint{4.685477in}{1.508509in}}%
\pgfpathlineto{\pgfqpoint{4.693054in}{1.516422in}}%
\pgfpathclose%
\pgfusepath{fill}%
\end{pgfscope}%
\begin{pgfscope}%
\pgfpathrectangle{\pgfqpoint{1.254980in}{0.150000in}}{\pgfqpoint{5.490039in}{5.490039in}}%
\pgfusepath{clip}%
\pgfsetbuttcap%
\pgfsetroundjoin%
\definecolor{currentfill}{rgb}{0.276022,0.044167,0.370164}%
\pgfsetfillcolor{currentfill}%
\pgfsetfillopacity{0.700000}%
\pgfsetlinewidth{0.000000pt}%
\definecolor{currentstroke}{rgb}{0.000000,0.000000,0.000000}%
\pgfsetstrokecolor{currentstroke}%
\pgfsetdash{}{0pt}%
\pgfpathmoveto{\pgfqpoint{4.918006in}{1.550118in}}%
\pgfpathlineto{\pgfqpoint{4.931760in}{1.547390in}}%
\pgfpathlineto{\pgfqpoint{4.945521in}{1.544686in}}%
\pgfpathlineto{\pgfqpoint{4.959290in}{1.542004in}}%
\pgfpathlineto{\pgfqpoint{4.973067in}{1.539346in}}%
\pgfpathlineto{\pgfqpoint{4.965553in}{1.529466in}}%
\pgfpathlineto{\pgfqpoint{4.958035in}{1.519682in}}%
\pgfpathlineto{\pgfqpoint{4.950515in}{1.509999in}}%
\pgfpathlineto{\pgfqpoint{4.942991in}{1.500424in}}%
\pgfpathlineto{\pgfqpoint{4.929207in}{1.503320in}}%
\pgfpathlineto{\pgfqpoint{4.915430in}{1.506240in}}%
\pgfpathlineto{\pgfqpoint{4.901661in}{1.509182in}}%
\pgfpathlineto{\pgfqpoint{4.887899in}{1.512148in}}%
\pgfpathlineto{\pgfqpoint{4.895431in}{1.521480in}}%
\pgfpathlineto{\pgfqpoint{4.902960in}{1.530923in}}%
\pgfpathlineto{\pgfqpoint{4.910485in}{1.540471in}}%
\pgfpathlineto{\pgfqpoint{4.918006in}{1.550118in}}%
\pgfpathclose%
\pgfusepath{fill}%
\end{pgfscope}%
\begin{pgfscope}%
\pgfpathrectangle{\pgfqpoint{1.254980in}{0.150000in}}{\pgfqpoint{5.490039in}{5.490039in}}%
\pgfusepath{clip}%
\pgfsetbuttcap%
\pgfsetroundjoin%
\definecolor{currentfill}{rgb}{0.283197,0.115680,0.436115}%
\pgfsetfillcolor{currentfill}%
\pgfsetfillopacity{0.700000}%
\pgfsetlinewidth{0.000000pt}%
\definecolor{currentstroke}{rgb}{0.000000,0.000000,0.000000}%
\pgfsetstrokecolor{currentstroke}%
\pgfsetdash{}{0pt}%
\pgfpathmoveto{\pgfqpoint{5.313889in}{1.679757in}}%
\pgfpathlineto{\pgfqpoint{5.327779in}{1.678317in}}%
\pgfpathlineto{\pgfqpoint{5.341677in}{1.676899in}}%
\pgfpathlineto{\pgfqpoint{5.355584in}{1.675505in}}%
\pgfpathlineto{\pgfqpoint{5.348158in}{1.663956in}}%
\pgfpathlineto{\pgfqpoint{5.340728in}{1.652406in}}%
\pgfpathlineto{\pgfqpoint{5.333293in}{1.640859in}}%
\pgfpathlineto{\pgfqpoint{5.325855in}{1.629319in}}%
\pgfpathlineto{\pgfqpoint{5.311943in}{1.630901in}}%
\pgfpathlineto{\pgfqpoint{5.298040in}{1.632506in}}%
\pgfpathlineto{\pgfqpoint{5.284146in}{1.634134in}}%
\pgfpathlineto{\pgfqpoint{5.291588in}{1.645530in}}%
\pgfpathlineto{\pgfqpoint{5.299026in}{1.656935in}}%
\pgfpathlineto{\pgfqpoint{5.306460in}{1.668346in}}%
\pgfpathlineto{\pgfqpoint{5.313889in}{1.679757in}}%
\pgfpathclose%
\pgfusepath{fill}%
\end{pgfscope}%
\begin{pgfscope}%
\pgfpathrectangle{\pgfqpoint{1.254980in}{0.150000in}}{\pgfqpoint{5.490039in}{5.490039in}}%
\pgfusepath{clip}%
\pgfsetbuttcap%
\pgfsetroundjoin%
\definecolor{currentfill}{rgb}{0.144759,0.519093,0.556572}%
\pgfsetfillcolor{currentfill}%
\pgfsetfillopacity{0.700000}%
\pgfsetlinewidth{0.000000pt}%
\definecolor{currentstroke}{rgb}{0.000000,0.000000,0.000000}%
\pgfsetstrokecolor{currentstroke}%
\pgfsetdash{}{0pt}%
\pgfpathmoveto{\pgfqpoint{2.487089in}{2.592648in}}%
\pgfpathlineto{\pgfqpoint{2.500462in}{2.581911in}}%
\pgfpathlineto{\pgfqpoint{2.513836in}{2.571216in}}%
\pgfpathlineto{\pgfqpoint{2.527212in}{2.560562in}}%
\pgfpathlineto{\pgfqpoint{2.540589in}{2.549950in}}%
\pgfpathlineto{\pgfqpoint{2.531408in}{2.568159in}}%
\pgfpathlineto{\pgfqpoint{2.522193in}{2.587015in}}%
\pgfpathlineto{\pgfqpoint{2.512942in}{2.606531in}}%
\pgfpathlineto{\pgfqpoint{2.503655in}{2.626718in}}%
\pgfpathlineto{\pgfqpoint{2.490221in}{2.637750in}}%
\pgfpathlineto{\pgfqpoint{2.476788in}{2.648824in}}%
\pgfpathlineto{\pgfqpoint{2.463356in}{2.659940in}}%
\pgfpathlineto{\pgfqpoint{2.449925in}{2.671098in}}%
\pgfpathlineto{\pgfqpoint{2.459272in}{2.650483in}}%
\pgfpathlineto{\pgfqpoint{2.468581in}{2.630545in}}%
\pgfpathlineto{\pgfqpoint{2.477853in}{2.611271in}}%
\pgfpathlineto{\pgfqpoint{2.487089in}{2.592648in}}%
\pgfpathclose%
\pgfusepath{fill}%
\end{pgfscope}%
\begin{pgfscope}%
\pgfpathrectangle{\pgfqpoint{1.254980in}{0.150000in}}{\pgfqpoint{5.490039in}{5.490039in}}%
\pgfusepath{clip}%
\pgfsetbuttcap%
\pgfsetroundjoin%
\definecolor{currentfill}{rgb}{0.276194,0.190074,0.493001}%
\pgfsetfillcolor{currentfill}%
\pgfsetfillopacity{0.700000}%
\pgfsetlinewidth{0.000000pt}%
\definecolor{currentstroke}{rgb}{0.000000,0.000000,0.000000}%
\pgfsetstrokecolor{currentstroke}%
\pgfsetdash{}{0pt}%
\pgfpathmoveto{\pgfqpoint{3.554255in}{1.814780in}}%
\pgfpathlineto{\pgfqpoint{3.567701in}{1.807529in}}%
\pgfpathlineto{\pgfqpoint{3.581150in}{1.800305in}}%
\pgfpathlineto{\pgfqpoint{3.594603in}{1.793106in}}%
\pgfpathlineto{\pgfqpoint{3.608061in}{1.785934in}}%
\pgfpathlineto{\pgfqpoint{3.599942in}{1.790890in}}%
\pgfpathlineto{\pgfqpoint{3.591809in}{1.796286in}}%
\pgfpathlineto{\pgfqpoint{3.583660in}{1.802130in}}%
\pgfpathlineto{\pgfqpoint{3.575497in}{1.808434in}}%
\pgfpathlineto{\pgfqpoint{3.562006in}{1.815965in}}%
\pgfpathlineto{\pgfqpoint{3.548519in}{1.823521in}}%
\pgfpathlineto{\pgfqpoint{3.535036in}{1.831104in}}%
\pgfpathlineto{\pgfqpoint{3.521557in}{1.838713in}}%
\pgfpathlineto{\pgfqpoint{3.529755in}{1.832045in}}%
\pgfpathlineto{\pgfqpoint{3.537938in}{1.825840in}}%
\pgfpathlineto{\pgfqpoint{3.546104in}{1.820089in}}%
\pgfpathlineto{\pgfqpoint{3.554255in}{1.814780in}}%
\pgfpathclose%
\pgfusepath{fill}%
\end{pgfscope}%
\begin{pgfscope}%
\pgfpathrectangle{\pgfqpoint{1.254980in}{0.150000in}}{\pgfqpoint{5.490039in}{5.490039in}}%
\pgfusepath{clip}%
\pgfsetbuttcap%
\pgfsetroundjoin%
\definecolor{currentfill}{rgb}{0.206756,0.371758,0.553117}%
\pgfsetfillcolor{currentfill}%
\pgfsetfillopacity{0.700000}%
\pgfsetlinewidth{0.000000pt}%
\definecolor{currentstroke}{rgb}{0.000000,0.000000,0.000000}%
\pgfsetstrokecolor{currentstroke}%
\pgfsetdash{}{0pt}%
\pgfpathmoveto{\pgfqpoint{2.950817in}{2.212777in}}%
\pgfpathlineto{\pgfqpoint{2.964202in}{2.203622in}}%
\pgfpathlineto{\pgfqpoint{2.977590in}{2.194499in}}%
\pgfpathlineto{\pgfqpoint{2.990980in}{2.185409in}}%
\pgfpathlineto{\pgfqpoint{3.004373in}{2.176351in}}%
\pgfpathlineto{\pgfqpoint{2.995714in}{2.188964in}}%
\pgfpathlineto{\pgfqpoint{2.987030in}{2.202141in}}%
\pgfpathlineto{\pgfqpoint{2.978320in}{2.215894in}}%
\pgfpathlineto{\pgfqpoint{2.969583in}{2.230234in}}%
\pgfpathlineto{\pgfqpoint{2.956143in}{2.239686in}}%
\pgfpathlineto{\pgfqpoint{2.942706in}{2.249172in}}%
\pgfpathlineto{\pgfqpoint{2.929271in}{2.258689in}}%
\pgfpathlineto{\pgfqpoint{2.915839in}{2.268240in}}%
\pgfpathlineto{\pgfqpoint{2.924624in}{2.253498in}}%
\pgfpathlineto{\pgfqpoint{2.933381in}{2.239349in}}%
\pgfpathlineto{\pgfqpoint{2.942112in}{2.225779in}}%
\pgfpathlineto{\pgfqpoint{2.950817in}{2.212777in}}%
\pgfpathclose%
\pgfusepath{fill}%
\end{pgfscope}%
\begin{pgfscope}%
\pgfpathrectangle{\pgfqpoint{1.254980in}{0.150000in}}{\pgfqpoint{5.490039in}{5.490039in}}%
\pgfusepath{clip}%
\pgfsetbuttcap%
\pgfsetroundjoin%
\definecolor{currentfill}{rgb}{0.282327,0.094955,0.417331}%
\pgfsetfillcolor{currentfill}%
\pgfsetfillopacity{0.700000}%
\pgfsetlinewidth{0.000000pt}%
\definecolor{currentstroke}{rgb}{0.000000,0.000000,0.000000}%
\pgfsetstrokecolor{currentstroke}%
\pgfsetdash{}{0pt}%
\pgfpathmoveto{\pgfqpoint{5.228650in}{1.640879in}}%
\pgfpathlineto{\pgfqpoint{5.242512in}{1.639158in}}%
\pgfpathlineto{\pgfqpoint{5.256381in}{1.637460in}}%
\pgfpathlineto{\pgfqpoint{5.270259in}{1.635786in}}%
\pgfpathlineto{\pgfqpoint{5.284146in}{1.634134in}}%
\pgfpathlineto{\pgfqpoint{5.276699in}{1.622754in}}%
\pgfpathlineto{\pgfqpoint{5.269249in}{1.611392in}}%
\pgfpathlineto{\pgfqpoint{5.261794in}{1.600054in}}%
\pgfpathlineto{\pgfqpoint{5.254336in}{1.588746in}}%
\pgfpathlineto{\pgfqpoint{5.240445in}{1.590597in}}%
\pgfpathlineto{\pgfqpoint{5.226562in}{1.592472in}}%
\pgfpathlineto{\pgfqpoint{5.212687in}{1.594370in}}%
\pgfpathlineto{\pgfqpoint{5.198821in}{1.596291in}}%
\pgfpathlineto{\pgfqpoint{5.206284in}{1.607395in}}%
\pgfpathlineto{\pgfqpoint{5.213743in}{1.618530in}}%
\pgfpathlineto{\pgfqpoint{5.221199in}{1.629693in}}%
\pgfpathlineto{\pgfqpoint{5.228650in}{1.640879in}}%
\pgfpathclose%
\pgfusepath{fill}%
\end{pgfscope}%
\begin{pgfscope}%
\pgfpathrectangle{\pgfqpoint{1.254980in}{0.150000in}}{\pgfqpoint{5.490039in}{5.490039in}}%
\pgfusepath{clip}%
\pgfsetbuttcap%
\pgfsetroundjoin%
\definecolor{currentfill}{rgb}{0.273809,0.031497,0.358853}%
\pgfsetfillcolor{currentfill}%
\pgfsetfillopacity{0.700000}%
\pgfsetlinewidth{0.000000pt}%
\definecolor{currentstroke}{rgb}{0.000000,0.000000,0.000000}%
\pgfsetstrokecolor{currentstroke}%
\pgfsetdash{}{0pt}%
\pgfpathmoveto{\pgfqpoint{4.832926in}{1.524238in}}%
\pgfpathlineto{\pgfqpoint{4.846658in}{1.521181in}}%
\pgfpathlineto{\pgfqpoint{4.860398in}{1.518147in}}%
\pgfpathlineto{\pgfqpoint{4.874145in}{1.515136in}}%
\pgfpathlineto{\pgfqpoint{4.887899in}{1.512148in}}%
\pgfpathlineto{\pgfqpoint{4.880364in}{1.502931in}}%
\pgfpathlineto{\pgfqpoint{4.872826in}{1.493838in}}%
\pgfpathlineto{\pgfqpoint{4.865284in}{1.484873in}}%
\pgfpathlineto{\pgfqpoint{4.857739in}{1.476043in}}%
\pgfpathlineto{\pgfqpoint{4.843976in}{1.479282in}}%
\pgfpathlineto{\pgfqpoint{4.830220in}{1.482544in}}%
\pgfpathlineto{\pgfqpoint{4.816471in}{1.485828in}}%
\pgfpathlineto{\pgfqpoint{4.802730in}{1.489136in}}%
\pgfpathlineto{\pgfqpoint{4.810284in}{1.497710in}}%
\pgfpathlineto{\pgfqpoint{4.817835in}{1.506423in}}%
\pgfpathlineto{\pgfqpoint{4.825382in}{1.515267in}}%
\pgfpathlineto{\pgfqpoint{4.832926in}{1.524238in}}%
\pgfpathclose%
\pgfusepath{fill}%
\end{pgfscope}%
\begin{pgfscope}%
\pgfpathrectangle{\pgfqpoint{1.254980in}{0.150000in}}{\pgfqpoint{5.490039in}{5.490039in}}%
\pgfusepath{clip}%
\pgfsetbuttcap%
\pgfsetroundjoin%
\definecolor{currentfill}{rgb}{0.281924,0.089666,0.412415}%
\pgfsetfillcolor{currentfill}%
\pgfsetfillopacity{0.700000}%
\pgfsetlinewidth{0.000000pt}%
\definecolor{currentstroke}{rgb}{0.000000,0.000000,0.000000}%
\pgfsetstrokecolor{currentstroke}%
\pgfsetdash{}{0pt}%
\pgfpathmoveto{\pgfqpoint{3.995542in}{1.612586in}}%
\pgfpathlineto{\pgfqpoint{4.009064in}{1.606714in}}%
\pgfpathlineto{\pgfqpoint{4.022591in}{1.600866in}}%
\pgfpathlineto{\pgfqpoint{4.036123in}{1.595042in}}%
\pgfpathlineto{\pgfqpoint{4.049661in}{1.589242in}}%
\pgfpathlineto{\pgfqpoint{4.041823in}{1.588763in}}%
\pgfpathlineto{\pgfqpoint{4.033977in}{1.588624in}}%
\pgfpathlineto{\pgfqpoint{4.026122in}{1.588833in}}%
\pgfpathlineto{\pgfqpoint{4.018258in}{1.589400in}}%
\pgfpathlineto{\pgfqpoint{4.004697in}{1.595528in}}%
\pgfpathlineto{\pgfqpoint{3.991141in}{1.601680in}}%
\pgfpathlineto{\pgfqpoint{3.977590in}{1.607856in}}%
\pgfpathlineto{\pgfqpoint{3.964044in}{1.614057in}}%
\pgfpathlineto{\pgfqpoint{3.971932in}{1.613156in}}%
\pgfpathlineto{\pgfqpoint{3.979811in}{1.612617in}}%
\pgfpathlineto{\pgfqpoint{3.987681in}{1.612430in}}%
\pgfpathlineto{\pgfqpoint{3.995542in}{1.612586in}}%
\pgfpathclose%
\pgfusepath{fill}%
\end{pgfscope}%
\begin{pgfscope}%
\pgfpathrectangle{\pgfqpoint{1.254980in}{0.150000in}}{\pgfqpoint{5.490039in}{5.490039in}}%
\pgfusepath{clip}%
\pgfsetbuttcap%
\pgfsetroundjoin%
\definecolor{currentfill}{rgb}{0.272594,0.025563,0.353093}%
\pgfsetfillcolor{currentfill}%
\pgfsetfillopacity{0.700000}%
\pgfsetlinewidth{0.000000pt}%
\definecolor{currentstroke}{rgb}{0.000000,0.000000,0.000000}%
\pgfsetstrokecolor{currentstroke}%
\pgfsetdash{}{0pt}%
\pgfpathmoveto{\pgfqpoint{4.468345in}{1.508064in}}%
\pgfpathlineto{\pgfqpoint{4.481974in}{1.503758in}}%
\pgfpathlineto{\pgfqpoint{4.495610in}{1.499475in}}%
\pgfpathlineto{\pgfqpoint{4.509252in}{1.495215in}}%
\pgfpathlineto{\pgfqpoint{4.522901in}{1.490978in}}%
\pgfpathlineto{\pgfqpoint{4.515265in}{1.485085in}}%
\pgfpathlineto{\pgfqpoint{4.507626in}{1.479412in}}%
\pgfpathlineto{\pgfqpoint{4.499981in}{1.473967in}}%
\pgfpathlineto{\pgfqpoint{4.492332in}{1.468756in}}%
\pgfpathlineto{\pgfqpoint{4.478669in}{1.473281in}}%
\pgfpathlineto{\pgfqpoint{4.465013in}{1.477830in}}%
\pgfpathlineto{\pgfqpoint{4.451362in}{1.482401in}}%
\pgfpathlineto{\pgfqpoint{4.437718in}{1.486996in}}%
\pgfpathlineto{\pgfqpoint{4.445382in}{1.491913in}}%
\pgfpathlineto{\pgfqpoint{4.453042in}{1.497069in}}%
\pgfpathlineto{\pgfqpoint{4.460696in}{1.502455in}}%
\pgfpathlineto{\pgfqpoint{4.468345in}{1.508064in}}%
\pgfpathclose%
\pgfusepath{fill}%
\end{pgfscope}%
\begin{pgfscope}%
\pgfpathrectangle{\pgfqpoint{1.254980in}{0.150000in}}{\pgfqpoint{5.490039in}{5.490039in}}%
\pgfusepath{clip}%
\pgfsetbuttcap%
\pgfsetroundjoin%
\definecolor{currentfill}{rgb}{0.253935,0.265254,0.529983}%
\pgfsetfillcolor{currentfill}%
\pgfsetfillopacity{0.700000}%
\pgfsetlinewidth{0.000000pt}%
\definecolor{currentstroke}{rgb}{0.000000,0.000000,0.000000}%
\pgfsetstrokecolor{currentstroke}%
\pgfsetdash{}{0pt}%
\pgfpathmoveto{\pgfqpoint{3.306422in}{1.964156in}}%
\pgfpathlineto{\pgfqpoint{3.319840in}{1.956107in}}%
\pgfpathlineto{\pgfqpoint{3.333262in}{1.948087in}}%
\pgfpathlineto{\pgfqpoint{3.346687in}{1.940095in}}%
\pgfpathlineto{\pgfqpoint{3.360116in}{1.932131in}}%
\pgfpathlineto{\pgfqpoint{3.351792in}{1.940378in}}%
\pgfpathlineto{\pgfqpoint{3.343450in}{1.949121in}}%
\pgfpathlineto{\pgfqpoint{3.335088in}{1.958370in}}%
\pgfpathlineto{\pgfqpoint{3.326707in}{1.968136in}}%
\pgfpathlineto{\pgfqpoint{3.313239in}{1.976475in}}%
\pgfpathlineto{\pgfqpoint{3.299775in}{1.984842in}}%
\pgfpathlineto{\pgfqpoint{3.286314in}{1.993237in}}%
\pgfpathlineto{\pgfqpoint{3.272856in}{2.001661in}}%
\pgfpathlineto{\pgfqpoint{3.281278in}{1.991514in}}%
\pgfpathlineto{\pgfqpoint{3.289679in}{1.981888in}}%
\pgfpathlineto{\pgfqpoint{3.298060in}{1.972772in}}%
\pgfpathlineto{\pgfqpoint{3.306422in}{1.964156in}}%
\pgfpathclose%
\pgfusepath{fill}%
\end{pgfscope}%
\begin{pgfscope}%
\pgfpathrectangle{\pgfqpoint{1.254980in}{0.150000in}}{\pgfqpoint{5.490039in}{5.490039in}}%
\pgfusepath{clip}%
\pgfsetbuttcap%
\pgfsetroundjoin%
\definecolor{currentfill}{rgb}{0.283072,0.130895,0.449241}%
\pgfsetfillcolor{currentfill}%
\pgfsetfillopacity{0.700000}%
\pgfsetlinewidth{0.000000pt}%
\definecolor{currentstroke}{rgb}{0.000000,0.000000,0.000000}%
\pgfsetstrokecolor{currentstroke}%
\pgfsetdash{}{0pt}%
\pgfpathmoveto{\pgfqpoint{3.801878in}{1.690376in}}%
\pgfpathlineto{\pgfqpoint{3.815365in}{1.683880in}}%
\pgfpathlineto{\pgfqpoint{3.828857in}{1.677408in}}%
\pgfpathlineto{\pgfqpoint{3.842354in}{1.670962in}}%
\pgfpathlineto{\pgfqpoint{3.855855in}{1.664541in}}%
\pgfpathlineto{\pgfqpoint{3.847904in}{1.666486in}}%
\pgfpathlineto{\pgfqpoint{3.839942in}{1.668817in}}%
\pgfpathlineto{\pgfqpoint{3.831969in}{1.671544in}}%
\pgfpathlineto{\pgfqpoint{3.823985in}{1.674677in}}%
\pgfpathlineto{\pgfqpoint{3.810455in}{1.681441in}}%
\pgfpathlineto{\pgfqpoint{3.796931in}{1.688230in}}%
\pgfpathlineto{\pgfqpoint{3.783411in}{1.695044in}}%
\pgfpathlineto{\pgfqpoint{3.769895in}{1.701883in}}%
\pgfpathlineto{\pgfqpoint{3.777909in}{1.698403in}}%
\pgfpathlineto{\pgfqpoint{3.785911in}{1.695331in}}%
\pgfpathlineto{\pgfqpoint{3.793900in}{1.692658in}}%
\pgfpathlineto{\pgfqpoint{3.801878in}{1.690376in}}%
\pgfpathclose%
\pgfusepath{fill}%
\end{pgfscope}%
\begin{pgfscope}%
\pgfpathrectangle{\pgfqpoint{1.254980in}{0.150000in}}{\pgfqpoint{5.490039in}{5.490039in}}%
\pgfusepath{clip}%
\pgfsetbuttcap%
\pgfsetroundjoin%
\definecolor{currentfill}{rgb}{0.274952,0.037752,0.364543}%
\pgfsetfillcolor{currentfill}%
\pgfsetfillopacity{0.700000}%
\pgfsetlinewidth{0.000000pt}%
\definecolor{currentstroke}{rgb}{0.000000,0.000000,0.000000}%
\pgfsetstrokecolor{currentstroke}%
\pgfsetdash{}{0pt}%
\pgfpathmoveto{\pgfqpoint{4.328788in}{1.524589in}}%
\pgfpathlineto{\pgfqpoint{4.342383in}{1.519808in}}%
\pgfpathlineto{\pgfqpoint{4.355984in}{1.515051in}}%
\pgfpathlineto{\pgfqpoint{4.369591in}{1.510317in}}%
\pgfpathlineto{\pgfqpoint{4.383204in}{1.505606in}}%
\pgfpathlineto{\pgfqpoint{4.375519in}{1.501231in}}%
\pgfpathlineto{\pgfqpoint{4.367828in}{1.497113in}}%
\pgfpathlineto{\pgfqpoint{4.360132in}{1.493259in}}%
\pgfpathlineto{\pgfqpoint{4.352430in}{1.489678in}}%
\pgfpathlineto{\pgfqpoint{4.338800in}{1.494690in}}%
\pgfpathlineto{\pgfqpoint{4.325176in}{1.499725in}}%
\pgfpathlineto{\pgfqpoint{4.311557in}{1.504784in}}%
\pgfpathlineto{\pgfqpoint{4.297945in}{1.509866in}}%
\pgfpathlineto{\pgfqpoint{4.305665in}{1.513141in}}%
\pgfpathlineto{\pgfqpoint{4.313378in}{1.516691in}}%
\pgfpathlineto{\pgfqpoint{4.321086in}{1.520510in}}%
\pgfpathlineto{\pgfqpoint{4.328788in}{1.524589in}}%
\pgfpathclose%
\pgfusepath{fill}%
\end{pgfscope}%
\begin{pgfscope}%
\pgfpathrectangle{\pgfqpoint{1.254980in}{0.150000in}}{\pgfqpoint{5.490039in}{5.490039in}}%
\pgfusepath{clip}%
\pgfsetbuttcap%
\pgfsetroundjoin%
\definecolor{currentfill}{rgb}{0.280894,0.078907,0.402329}%
\pgfsetfillcolor{currentfill}%
\pgfsetfillopacity{0.700000}%
\pgfsetlinewidth{0.000000pt}%
\definecolor{currentstroke}{rgb}{0.000000,0.000000,0.000000}%
\pgfsetstrokecolor{currentstroke}%
\pgfsetdash{}{0pt}%
\pgfpathmoveto{\pgfqpoint{5.143436in}{1.604206in}}%
\pgfpathlineto{\pgfqpoint{5.157270in}{1.602193in}}%
\pgfpathlineto{\pgfqpoint{5.171112in}{1.600202in}}%
\pgfpathlineto{\pgfqpoint{5.184962in}{1.598235in}}%
\pgfpathlineto{\pgfqpoint{5.198821in}{1.596291in}}%
\pgfpathlineto{\pgfqpoint{5.191353in}{1.585225in}}%
\pgfpathlineto{\pgfqpoint{5.183883in}{1.574200in}}%
\pgfpathlineto{\pgfqpoint{5.176408in}{1.563223in}}%
\pgfpathlineto{\pgfqpoint{5.168930in}{1.552299in}}%
\pgfpathlineto{\pgfqpoint{5.155066in}{1.554456in}}%
\pgfpathlineto{\pgfqpoint{5.141211in}{1.556636in}}%
\pgfpathlineto{\pgfqpoint{5.127363in}{1.558839in}}%
\pgfpathlineto{\pgfqpoint{5.113523in}{1.561066in}}%
\pgfpathlineto{\pgfqpoint{5.121007in}{1.571772in}}%
\pgfpathlineto{\pgfqpoint{5.128487in}{1.582535in}}%
\pgfpathlineto{\pgfqpoint{5.135963in}{1.593348in}}%
\pgfpathlineto{\pgfqpoint{5.143436in}{1.604206in}}%
\pgfpathclose%
\pgfusepath{fill}%
\end{pgfscope}%
\begin{pgfscope}%
\pgfpathrectangle{\pgfqpoint{1.254980in}{0.150000in}}{\pgfqpoint{5.490039in}{5.490039in}}%
\pgfusepath{clip}%
\pgfsetbuttcap%
\pgfsetroundjoin%
\definecolor{currentfill}{rgb}{0.150476,0.504369,0.557430}%
\pgfsetfillcolor{currentfill}%
\pgfsetfillopacity{0.700000}%
\pgfsetlinewidth{0.000000pt}%
\definecolor{currentstroke}{rgb}{0.000000,0.000000,0.000000}%
\pgfsetstrokecolor{currentstroke}%
\pgfsetdash{}{0pt}%
\pgfpathmoveto{\pgfqpoint{2.540589in}{2.549950in}}%
\pgfpathlineto{\pgfqpoint{2.553967in}{2.539378in}}%
\pgfpathlineto{\pgfqpoint{2.567347in}{2.528847in}}%
\pgfpathlineto{\pgfqpoint{2.580728in}{2.518357in}}%
\pgfpathlineto{\pgfqpoint{2.594110in}{2.507906in}}%
\pgfpathlineto{\pgfqpoint{2.584985in}{2.525703in}}%
\pgfpathlineto{\pgfqpoint{2.575826in}{2.544142in}}%
\pgfpathlineto{\pgfqpoint{2.566632in}{2.563236in}}%
\pgfpathlineto{\pgfqpoint{2.557402in}{2.582997in}}%
\pgfpathlineto{\pgfqpoint{2.543963in}{2.593867in}}%
\pgfpathlineto{\pgfqpoint{2.530526in}{2.604777in}}%
\pgfpathlineto{\pgfqpoint{2.517090in}{2.615727in}}%
\pgfpathlineto{\pgfqpoint{2.503655in}{2.626718in}}%
\pgfpathlineto{\pgfqpoint{2.512942in}{2.606531in}}%
\pgfpathlineto{\pgfqpoint{2.522193in}{2.587015in}}%
\pgfpathlineto{\pgfqpoint{2.531408in}{2.568159in}}%
\pgfpathlineto{\pgfqpoint{2.540589in}{2.549950in}}%
\pgfpathclose%
\pgfusepath{fill}%
\end{pgfscope}%
\begin{pgfscope}%
\pgfpathrectangle{\pgfqpoint{1.254980in}{0.150000in}}{\pgfqpoint{5.490039in}{5.490039in}}%
\pgfusepath{clip}%
\pgfsetbuttcap%
\pgfsetroundjoin%
\definecolor{currentfill}{rgb}{0.271305,0.019942,0.347269}%
\pgfsetfillcolor{currentfill}%
\pgfsetfillopacity{0.700000}%
\pgfsetlinewidth{0.000000pt}%
\definecolor{currentstroke}{rgb}{0.000000,0.000000,0.000000}%
\pgfsetstrokecolor{currentstroke}%
\pgfsetdash{}{0pt}%
\pgfpathmoveto{\pgfqpoint{4.608005in}{1.500989in}}%
\pgfpathlineto{\pgfqpoint{4.621674in}{1.497143in}}%
\pgfpathlineto{\pgfqpoint{4.635349in}{1.493320in}}%
\pgfpathlineto{\pgfqpoint{4.649031in}{1.489520in}}%
\pgfpathlineto{\pgfqpoint{4.662720in}{1.485743in}}%
\pgfpathlineto{\pgfqpoint{4.655127in}{1.478501in}}%
\pgfpathlineto{\pgfqpoint{4.647530in}{1.471444in}}%
\pgfpathlineto{\pgfqpoint{4.639929in}{1.464579in}}%
\pgfpathlineto{\pgfqpoint{4.632324in}{1.457912in}}%
\pgfpathlineto{\pgfqpoint{4.618623in}{1.461965in}}%
\pgfpathlineto{\pgfqpoint{4.604929in}{1.466041in}}%
\pgfpathlineto{\pgfqpoint{4.591241in}{1.470140in}}%
\pgfpathlineto{\pgfqpoint{4.577560in}{1.474261in}}%
\pgfpathlineto{\pgfqpoint{4.585177in}{1.480647in}}%
\pgfpathlineto{\pgfqpoint{4.592791in}{1.487235in}}%
\pgfpathlineto{\pgfqpoint{4.600400in}{1.494018in}}%
\pgfpathlineto{\pgfqpoint{4.608005in}{1.500989in}}%
\pgfpathclose%
\pgfusepath{fill}%
\end{pgfscope}%
\begin{pgfscope}%
\pgfpathrectangle{\pgfqpoint{1.254980in}{0.150000in}}{\pgfqpoint{5.490039in}{5.490039in}}%
\pgfusepath{clip}%
\pgfsetbuttcap%
\pgfsetroundjoin%
\definecolor{currentfill}{rgb}{0.277941,0.056324,0.381191}%
\pgfsetfillcolor{currentfill}%
\pgfsetfillopacity{0.700000}%
\pgfsetlinewidth{0.000000pt}%
\definecolor{currentstroke}{rgb}{0.000000,0.000000,0.000000}%
\pgfsetstrokecolor{currentstroke}%
\pgfsetdash{}{0pt}%
\pgfpathmoveto{\pgfqpoint{4.189255in}{1.551367in}}%
\pgfpathlineto{\pgfqpoint{4.202821in}{1.546097in}}%
\pgfpathlineto{\pgfqpoint{4.216393in}{1.540850in}}%
\pgfpathlineto{\pgfqpoint{4.229970in}{1.535627in}}%
\pgfpathlineto{\pgfqpoint{4.243554in}{1.530428in}}%
\pgfpathlineto{\pgfqpoint{4.235809in}{1.527747in}}%
\pgfpathlineto{\pgfqpoint{4.228058in}{1.525362in}}%
\pgfpathlineto{\pgfqpoint{4.220300in}{1.523280in}}%
\pgfpathlineto{\pgfqpoint{4.212535in}{1.521511in}}%
\pgfpathlineto{\pgfqpoint{4.198932in}{1.527025in}}%
\pgfpathlineto{\pgfqpoint{4.185334in}{1.532562in}}%
\pgfpathlineto{\pgfqpoint{4.171742in}{1.538123in}}%
\pgfpathlineto{\pgfqpoint{4.158156in}{1.543708in}}%
\pgfpathlineto{\pgfqpoint{4.165942in}{1.545158in}}%
\pgfpathlineto{\pgfqpoint{4.173720in}{1.546923in}}%
\pgfpathlineto{\pgfqpoint{4.181491in}{1.548995in}}%
\pgfpathlineto{\pgfqpoint{4.189255in}{1.551367in}}%
\pgfpathclose%
\pgfusepath{fill}%
\end{pgfscope}%
\begin{pgfscope}%
\pgfpathrectangle{\pgfqpoint{1.254980in}{0.150000in}}{\pgfqpoint{5.490039in}{5.490039in}}%
\pgfusepath{clip}%
\pgfsetbuttcap%
\pgfsetroundjoin%
\definecolor{currentfill}{rgb}{0.278791,0.062145,0.386592}%
\pgfsetfillcolor{currentfill}%
\pgfsetfillopacity{0.700000}%
\pgfsetlinewidth{0.000000pt}%
\definecolor{currentstroke}{rgb}{0.000000,0.000000,0.000000}%
\pgfsetstrokecolor{currentstroke}%
\pgfsetdash{}{0pt}%
\pgfpathmoveto{\pgfqpoint{5.058243in}{1.570201in}}%
\pgfpathlineto{\pgfqpoint{5.072051in}{1.567882in}}%
\pgfpathlineto{\pgfqpoint{5.085867in}{1.565587in}}%
\pgfpathlineto{\pgfqpoint{5.099691in}{1.563315in}}%
\pgfpathlineto{\pgfqpoint{5.113523in}{1.561066in}}%
\pgfpathlineto{\pgfqpoint{5.106036in}{1.550420in}}%
\pgfpathlineto{\pgfqpoint{5.098545in}{1.539841in}}%
\pgfpathlineto{\pgfqpoint{5.091051in}{1.529334in}}%
\pgfpathlineto{\pgfqpoint{5.083554in}{1.518904in}}%
\pgfpathlineto{\pgfqpoint{5.069716in}{1.521379in}}%
\pgfpathlineto{\pgfqpoint{5.055886in}{1.523877in}}%
\pgfpathlineto{\pgfqpoint{5.042063in}{1.526397in}}%
\pgfpathlineto{\pgfqpoint{5.028249in}{1.528941in}}%
\pgfpathlineto{\pgfqpoint{5.035752in}{1.539140in}}%
\pgfpathlineto{\pgfqpoint{5.043253in}{1.549420in}}%
\pgfpathlineto{\pgfqpoint{5.050749in}{1.559776in}}%
\pgfpathlineto{\pgfqpoint{5.058243in}{1.570201in}}%
\pgfpathclose%
\pgfusepath{fill}%
\end{pgfscope}%
\begin{pgfscope}%
\pgfpathrectangle{\pgfqpoint{1.254980in}{0.150000in}}{\pgfqpoint{5.490039in}{5.490039in}}%
\pgfusepath{clip}%
\pgfsetbuttcap%
\pgfsetroundjoin%
\definecolor{currentfill}{rgb}{0.210503,0.363727,0.552206}%
\pgfsetfillcolor{currentfill}%
\pgfsetfillopacity{0.700000}%
\pgfsetlinewidth{0.000000pt}%
\definecolor{currentstroke}{rgb}{0.000000,0.000000,0.000000}%
\pgfsetstrokecolor{currentstroke}%
\pgfsetdash{}{0pt}%
\pgfpathmoveto{\pgfqpoint{3.004373in}{2.176351in}}%
\pgfpathlineto{\pgfqpoint{3.017769in}{2.167325in}}%
\pgfpathlineto{\pgfqpoint{3.031168in}{2.158330in}}%
\pgfpathlineto{\pgfqpoint{3.044569in}{2.149368in}}%
\pgfpathlineto{\pgfqpoint{3.057974in}{2.140436in}}%
\pgfpathlineto{\pgfqpoint{3.049359in}{2.152662in}}%
\pgfpathlineto{\pgfqpoint{3.040721in}{2.165447in}}%
\pgfpathlineto{\pgfqpoint{3.032057in}{2.178803in}}%
\pgfpathlineto{\pgfqpoint{3.023368in}{2.192742in}}%
\pgfpathlineto{\pgfqpoint{3.009918in}{2.202068in}}%
\pgfpathlineto{\pgfqpoint{2.996470in}{2.211424in}}%
\pgfpathlineto{\pgfqpoint{2.983025in}{2.220813in}}%
\pgfpathlineto{\pgfqpoint{2.969583in}{2.230234in}}%
\pgfpathlineto{\pgfqpoint{2.978320in}{2.215894in}}%
\pgfpathlineto{\pgfqpoint{2.987030in}{2.202141in}}%
\pgfpathlineto{\pgfqpoint{2.995714in}{2.188964in}}%
\pgfpathlineto{\pgfqpoint{3.004373in}{2.176351in}}%
\pgfpathclose%
\pgfusepath{fill}%
\end{pgfscope}%
\begin{pgfscope}%
\pgfpathrectangle{\pgfqpoint{1.254980in}{0.150000in}}{\pgfqpoint{5.490039in}{5.490039in}}%
\pgfusepath{clip}%
\pgfsetbuttcap%
\pgfsetroundjoin%
\definecolor{currentfill}{rgb}{0.277134,0.185228,0.489898}%
\pgfsetfillcolor{currentfill}%
\pgfsetfillopacity{0.700000}%
\pgfsetlinewidth{0.000000pt}%
\definecolor{currentstroke}{rgb}{0.000000,0.000000,0.000000}%
\pgfsetstrokecolor{currentstroke}%
\pgfsetdash{}{0pt}%
\pgfpathmoveto{\pgfqpoint{3.608061in}{1.785934in}}%
\pgfpathlineto{\pgfqpoint{3.621523in}{1.778788in}}%
\pgfpathlineto{\pgfqpoint{3.634990in}{1.771669in}}%
\pgfpathlineto{\pgfqpoint{3.648460in}{1.764575in}}%
\pgfpathlineto{\pgfqpoint{3.661935in}{1.757507in}}%
\pgfpathlineto{\pgfqpoint{3.653848in}{1.762110in}}%
\pgfpathlineto{\pgfqpoint{3.645747in}{1.767149in}}%
\pgfpathlineto{\pgfqpoint{3.637631in}{1.772634in}}%
\pgfpathlineto{\pgfqpoint{3.629502in}{1.778575in}}%
\pgfpathlineto{\pgfqpoint{3.615994in}{1.786001in}}%
\pgfpathlineto{\pgfqpoint{3.602491in}{1.793452in}}%
\pgfpathlineto{\pgfqpoint{3.588992in}{1.800930in}}%
\pgfpathlineto{\pgfqpoint{3.575497in}{1.808434in}}%
\pgfpathlineto{\pgfqpoint{3.583660in}{1.802130in}}%
\pgfpathlineto{\pgfqpoint{3.591809in}{1.796286in}}%
\pgfpathlineto{\pgfqpoint{3.599942in}{1.790890in}}%
\pgfpathlineto{\pgfqpoint{3.608061in}{1.785934in}}%
\pgfpathclose%
\pgfusepath{fill}%
\end{pgfscope}%
\begin{pgfscope}%
\pgfpathrectangle{\pgfqpoint{1.254980in}{0.150000in}}{\pgfqpoint{5.490039in}{5.490039in}}%
\pgfusepath{clip}%
\pgfsetbuttcap%
\pgfsetroundjoin%
\definecolor{currentfill}{rgb}{0.154815,0.493313,0.557840}%
\pgfsetfillcolor{currentfill}%
\pgfsetfillopacity{0.700000}%
\pgfsetlinewidth{0.000000pt}%
\definecolor{currentstroke}{rgb}{0.000000,0.000000,0.000000}%
\pgfsetstrokecolor{currentstroke}%
\pgfsetdash{}{0pt}%
\pgfpathmoveto{\pgfqpoint{2.594110in}{2.507906in}}%
\pgfpathlineto{\pgfqpoint{2.607494in}{2.497494in}}%
\pgfpathlineto{\pgfqpoint{2.620880in}{2.487122in}}%
\pgfpathlineto{\pgfqpoint{2.634267in}{2.476788in}}%
\pgfpathlineto{\pgfqpoint{2.647656in}{2.466493in}}%
\pgfpathlineto{\pgfqpoint{2.638585in}{2.483879in}}%
\pgfpathlineto{\pgfqpoint{2.629481in}{2.501903in}}%
\pgfpathlineto{\pgfqpoint{2.620343in}{2.520576in}}%
\pgfpathlineto{\pgfqpoint{2.611170in}{2.539913in}}%
\pgfpathlineto{\pgfqpoint{2.597726in}{2.550626in}}%
\pgfpathlineto{\pgfqpoint{2.584283in}{2.561377in}}%
\pgfpathlineto{\pgfqpoint{2.570842in}{2.572168in}}%
\pgfpathlineto{\pgfqpoint{2.557402in}{2.582997in}}%
\pgfpathlineto{\pgfqpoint{2.566632in}{2.563236in}}%
\pgfpathlineto{\pgfqpoint{2.575826in}{2.544142in}}%
\pgfpathlineto{\pgfqpoint{2.584985in}{2.525703in}}%
\pgfpathlineto{\pgfqpoint{2.594110in}{2.507906in}}%
\pgfpathclose%
\pgfusepath{fill}%
\end{pgfscope}%
\begin{pgfscope}%
\pgfpathrectangle{\pgfqpoint{1.254980in}{0.150000in}}{\pgfqpoint{5.490039in}{5.490039in}}%
\pgfusepath{clip}%
\pgfsetbuttcap%
\pgfsetroundjoin%
\definecolor{currentfill}{rgb}{0.272594,0.025563,0.353093}%
\pgfsetfillcolor{currentfill}%
\pgfsetfillopacity{0.700000}%
\pgfsetlinewidth{0.000000pt}%
\definecolor{currentstroke}{rgb}{0.000000,0.000000,0.000000}%
\pgfsetstrokecolor{currentstroke}%
\pgfsetdash{}{0pt}%
\pgfpathmoveto{\pgfqpoint{4.747836in}{1.502595in}}%
\pgfpathlineto{\pgfqpoint{4.761549in}{1.499196in}}%
\pgfpathlineto{\pgfqpoint{4.775269in}{1.495820in}}%
\pgfpathlineto{\pgfqpoint{4.788996in}{1.492466in}}%
\pgfpathlineto{\pgfqpoint{4.802730in}{1.489136in}}%
\pgfpathlineto{\pgfqpoint{4.795172in}{1.480706in}}%
\pgfpathlineto{\pgfqpoint{4.787611in}{1.472428in}}%
\pgfpathlineto{\pgfqpoint{4.780047in}{1.464308in}}%
\pgfpathlineto{\pgfqpoint{4.772479in}{1.456353in}}%
\pgfpathlineto{\pgfqpoint{4.758735in}{1.459946in}}%
\pgfpathlineto{\pgfqpoint{4.744998in}{1.463563in}}%
\pgfpathlineto{\pgfqpoint{4.731268in}{1.467202in}}%
\pgfpathlineto{\pgfqpoint{4.717544in}{1.470865in}}%
\pgfpathlineto{\pgfqpoint{4.725123in}{1.478552in}}%
\pgfpathlineto{\pgfqpoint{4.732697in}{1.486407in}}%
\pgfpathlineto{\pgfqpoint{4.740268in}{1.494424in}}%
\pgfpathlineto{\pgfqpoint{4.747836in}{1.502595in}}%
\pgfpathclose%
\pgfusepath{fill}%
\end{pgfscope}%
\begin{pgfscope}%
\pgfpathrectangle{\pgfqpoint{1.254980in}{0.150000in}}{\pgfqpoint{5.490039in}{5.490039in}}%
\pgfusepath{clip}%
\pgfsetbuttcap%
\pgfsetroundjoin%
\definecolor{currentfill}{rgb}{0.276022,0.044167,0.370164}%
\pgfsetfillcolor{currentfill}%
\pgfsetfillopacity{0.700000}%
\pgfsetlinewidth{0.000000pt}%
\definecolor{currentstroke}{rgb}{0.000000,0.000000,0.000000}%
\pgfsetstrokecolor{currentstroke}%
\pgfsetdash{}{0pt}%
\pgfpathmoveto{\pgfqpoint{4.973067in}{1.539346in}}%
\pgfpathlineto{\pgfqpoint{4.986851in}{1.536710in}}%
\pgfpathlineto{\pgfqpoint{5.000642in}{1.534097in}}%
\pgfpathlineto{\pgfqpoint{5.014442in}{1.531508in}}%
\pgfpathlineto{\pgfqpoint{5.028249in}{1.528941in}}%
\pgfpathlineto{\pgfqpoint{5.020742in}{1.518828in}}%
\pgfpathlineto{\pgfqpoint{5.013232in}{1.508808in}}%
\pgfpathlineto{\pgfqpoint{5.005718in}{1.498885in}}%
\pgfpathlineto{\pgfqpoint{4.998202in}{1.489067in}}%
\pgfpathlineto{\pgfqpoint{4.984388in}{1.491872in}}%
\pgfpathlineto{\pgfqpoint{4.970581in}{1.494700in}}%
\pgfpathlineto{\pgfqpoint{4.956782in}{1.497550in}}%
\pgfpathlineto{\pgfqpoint{4.942991in}{1.500424in}}%
\pgfpathlineto{\pgfqpoint{4.950515in}{1.509999in}}%
\pgfpathlineto{\pgfqpoint{4.958035in}{1.519682in}}%
\pgfpathlineto{\pgfqpoint{4.965553in}{1.529466in}}%
\pgfpathlineto{\pgfqpoint{4.973067in}{1.539346in}}%
\pgfpathclose%
\pgfusepath{fill}%
\end{pgfscope}%
\begin{pgfscope}%
\pgfpathrectangle{\pgfqpoint{1.254980in}{0.150000in}}{\pgfqpoint{5.490039in}{5.490039in}}%
\pgfusepath{clip}%
\pgfsetbuttcap%
\pgfsetroundjoin%
\definecolor{currentfill}{rgb}{0.257322,0.256130,0.526563}%
\pgfsetfillcolor{currentfill}%
\pgfsetfillopacity{0.700000}%
\pgfsetlinewidth{0.000000pt}%
\definecolor{currentstroke}{rgb}{0.000000,0.000000,0.000000}%
\pgfsetstrokecolor{currentstroke}%
\pgfsetdash{}{0pt}%
\pgfpathmoveto{\pgfqpoint{3.360116in}{1.932131in}}%
\pgfpathlineto{\pgfqpoint{3.373549in}{1.924195in}}%
\pgfpathlineto{\pgfqpoint{3.386985in}{1.916288in}}%
\pgfpathlineto{\pgfqpoint{3.400425in}{1.908407in}}%
\pgfpathlineto{\pgfqpoint{3.413869in}{1.900555in}}%
\pgfpathlineto{\pgfqpoint{3.405582in}{1.908433in}}%
\pgfpathlineto{\pgfqpoint{3.397278in}{1.916803in}}%
\pgfpathlineto{\pgfqpoint{3.388955in}{1.925675in}}%
\pgfpathlineto{\pgfqpoint{3.380613in}{1.935061in}}%
\pgfpathlineto{\pgfqpoint{3.367131in}{1.943288in}}%
\pgfpathlineto{\pgfqpoint{3.353653in}{1.951543in}}%
\pgfpathlineto{\pgfqpoint{3.340178in}{1.959826in}}%
\pgfpathlineto{\pgfqpoint{3.326707in}{1.968136in}}%
\pgfpathlineto{\pgfqpoint{3.335088in}{1.958370in}}%
\pgfpathlineto{\pgfqpoint{3.343450in}{1.949121in}}%
\pgfpathlineto{\pgfqpoint{3.351792in}{1.940378in}}%
\pgfpathlineto{\pgfqpoint{3.360116in}{1.932131in}}%
\pgfpathclose%
\pgfusepath{fill}%
\end{pgfscope}%
\begin{pgfscope}%
\pgfpathrectangle{\pgfqpoint{1.254980in}{0.150000in}}{\pgfqpoint{5.490039in}{5.490039in}}%
\pgfusepath{clip}%
\pgfsetbuttcap%
\pgfsetroundjoin%
\definecolor{currentfill}{rgb}{0.283187,0.125848,0.444960}%
\pgfsetfillcolor{currentfill}%
\pgfsetfillopacity{0.700000}%
\pgfsetlinewidth{0.000000pt}%
\definecolor{currentstroke}{rgb}{0.000000,0.000000,0.000000}%
\pgfsetstrokecolor{currentstroke}%
\pgfsetdash{}{0pt}%
\pgfpathmoveto{\pgfqpoint{3.855855in}{1.664541in}}%
\pgfpathlineto{\pgfqpoint{3.869361in}{1.658144in}}%
\pgfpathlineto{\pgfqpoint{3.882873in}{1.651772in}}%
\pgfpathlineto{\pgfqpoint{3.896389in}{1.645425in}}%
\pgfpathlineto{\pgfqpoint{3.909910in}{1.639102in}}%
\pgfpathlineto{\pgfqpoint{3.901985in}{1.640710in}}%
\pgfpathlineto{\pgfqpoint{3.894051in}{1.642701in}}%
\pgfpathlineto{\pgfqpoint{3.886105in}{1.645084in}}%
\pgfpathlineto{\pgfqpoint{3.878149in}{1.647868in}}%
\pgfpathlineto{\pgfqpoint{3.864601in}{1.654533in}}%
\pgfpathlineto{\pgfqpoint{3.851057in}{1.661223in}}%
\pgfpathlineto{\pgfqpoint{3.837519in}{1.667938in}}%
\pgfpathlineto{\pgfqpoint{3.823985in}{1.674677in}}%
\pgfpathlineto{\pgfqpoint{3.831969in}{1.671544in}}%
\pgfpathlineto{\pgfqpoint{3.839942in}{1.668817in}}%
\pgfpathlineto{\pgfqpoint{3.847904in}{1.666486in}}%
\pgfpathlineto{\pgfqpoint{3.855855in}{1.664541in}}%
\pgfpathclose%
\pgfusepath{fill}%
\end{pgfscope}%
\begin{pgfscope}%
\pgfpathrectangle{\pgfqpoint{1.254980in}{0.150000in}}{\pgfqpoint{5.490039in}{5.490039in}}%
\pgfusepath{clip}%
\pgfsetbuttcap%
\pgfsetroundjoin%
\definecolor{currentfill}{rgb}{0.281446,0.084320,0.407414}%
\pgfsetfillcolor{currentfill}%
\pgfsetfillopacity{0.700000}%
\pgfsetlinewidth{0.000000pt}%
\definecolor{currentstroke}{rgb}{0.000000,0.000000,0.000000}%
\pgfsetstrokecolor{currentstroke}%
\pgfsetdash{}{0pt}%
\pgfpathmoveto{\pgfqpoint{4.049661in}{1.589242in}}%
\pgfpathlineto{\pgfqpoint{4.063204in}{1.583467in}}%
\pgfpathlineto{\pgfqpoint{4.076752in}{1.577715in}}%
\pgfpathlineto{\pgfqpoint{4.090306in}{1.571988in}}%
\pgfpathlineto{\pgfqpoint{4.103865in}{1.566284in}}%
\pgfpathlineto{\pgfqpoint{4.096050in}{1.565482in}}%
\pgfpathlineto{\pgfqpoint{4.088227in}{1.565016in}}%
\pgfpathlineto{\pgfqpoint{4.080396in}{1.564895in}}%
\pgfpathlineto{\pgfqpoint{4.072556in}{1.565129in}}%
\pgfpathlineto{\pgfqpoint{4.058973in}{1.571161in}}%
\pgfpathlineto{\pgfqpoint{4.045397in}{1.577216in}}%
\pgfpathlineto{\pgfqpoint{4.031825in}{1.583296in}}%
\pgfpathlineto{\pgfqpoint{4.018258in}{1.589400in}}%
\pgfpathlineto{\pgfqpoint{4.026122in}{1.588833in}}%
\pgfpathlineto{\pgfqpoint{4.033977in}{1.588624in}}%
\pgfpathlineto{\pgfqpoint{4.041823in}{1.588763in}}%
\pgfpathlineto{\pgfqpoint{4.049661in}{1.589242in}}%
\pgfpathclose%
\pgfusepath{fill}%
\end{pgfscope}%
\begin{pgfscope}%
\pgfpathrectangle{\pgfqpoint{1.254980in}{0.150000in}}{\pgfqpoint{5.490039in}{5.490039in}}%
\pgfusepath{clip}%
\pgfsetbuttcap%
\pgfsetroundjoin%
\definecolor{currentfill}{rgb}{0.216210,0.351535,0.550627}%
\pgfsetfillcolor{currentfill}%
\pgfsetfillopacity{0.700000}%
\pgfsetlinewidth{0.000000pt}%
\definecolor{currentstroke}{rgb}{0.000000,0.000000,0.000000}%
\pgfsetstrokecolor{currentstroke}%
\pgfsetdash{}{0pt}%
\pgfpathmoveto{\pgfqpoint{3.057974in}{2.140436in}}%
\pgfpathlineto{\pgfqpoint{3.071381in}{2.131536in}}%
\pgfpathlineto{\pgfqpoint{3.084791in}{2.122667in}}%
\pgfpathlineto{\pgfqpoint{3.098204in}{2.113829in}}%
\pgfpathlineto{\pgfqpoint{3.111620in}{2.105021in}}%
\pgfpathlineto{\pgfqpoint{3.103050in}{2.116859in}}%
\pgfpathlineto{\pgfqpoint{3.094456in}{2.129253in}}%
\pgfpathlineto{\pgfqpoint{3.085839in}{2.142213in}}%
\pgfpathlineto{\pgfqpoint{3.077196in}{2.155753in}}%
\pgfpathlineto{\pgfqpoint{3.063735in}{2.164954in}}%
\pgfpathlineto{\pgfqpoint{3.050277in}{2.174185in}}%
\pgfpathlineto{\pgfqpoint{3.036821in}{2.183448in}}%
\pgfpathlineto{\pgfqpoint{3.023368in}{2.192742in}}%
\pgfpathlineto{\pgfqpoint{3.032057in}{2.178803in}}%
\pgfpathlineto{\pgfqpoint{3.040721in}{2.165447in}}%
\pgfpathlineto{\pgfqpoint{3.049359in}{2.152662in}}%
\pgfpathlineto{\pgfqpoint{3.057974in}{2.140436in}}%
\pgfpathclose%
\pgfusepath{fill}%
\end{pgfscope}%
\begin{pgfscope}%
\pgfpathrectangle{\pgfqpoint{1.254980in}{0.150000in}}{\pgfqpoint{5.490039in}{5.490039in}}%
\pgfusepath{clip}%
\pgfsetbuttcap%
\pgfsetroundjoin%
\definecolor{currentfill}{rgb}{0.282656,0.100196,0.422160}%
\pgfsetfillcolor{currentfill}%
\pgfsetfillopacity{0.700000}%
\pgfsetlinewidth{0.000000pt}%
\definecolor{currentstroke}{rgb}{0.000000,0.000000,0.000000}%
\pgfsetstrokecolor{currentstroke}%
\pgfsetdash{}{0pt}%
\pgfpathmoveto{\pgfqpoint{5.284146in}{1.634134in}}%
\pgfpathlineto{\pgfqpoint{5.298040in}{1.632506in}}%
\pgfpathlineto{\pgfqpoint{5.311943in}{1.630901in}}%
\pgfpathlineto{\pgfqpoint{5.325855in}{1.629319in}}%
\pgfpathlineto{\pgfqpoint{5.318412in}{1.617792in}}%
\pgfpathlineto{\pgfqpoint{5.310965in}{1.606281in}}%
\pgfpathlineto{\pgfqpoint{5.303514in}{1.594792in}}%
\pgfpathlineto{\pgfqpoint{5.296059in}{1.583329in}}%
\pgfpathlineto{\pgfqpoint{5.282143in}{1.585111in}}%
\pgfpathlineto{\pgfqpoint{5.268235in}{1.586917in}}%
\pgfpathlineto{\pgfqpoint{5.254336in}{1.588746in}}%
\pgfpathlineto{\pgfqpoint{5.261794in}{1.600054in}}%
\pgfpathlineto{\pgfqpoint{5.269249in}{1.611392in}}%
\pgfpathlineto{\pgfqpoint{5.276699in}{1.622754in}}%
\pgfpathlineto{\pgfqpoint{5.284146in}{1.634134in}}%
\pgfpathclose%
\pgfusepath{fill}%
\end{pgfscope}%
\begin{pgfscope}%
\pgfpathrectangle{\pgfqpoint{1.254980in}{0.150000in}}{\pgfqpoint{5.490039in}{5.490039in}}%
\pgfusepath{clip}%
\pgfsetbuttcap%
\pgfsetroundjoin%
\definecolor{currentfill}{rgb}{0.160665,0.478540,0.558115}%
\pgfsetfillcolor{currentfill}%
\pgfsetfillopacity{0.700000}%
\pgfsetlinewidth{0.000000pt}%
\definecolor{currentstroke}{rgb}{0.000000,0.000000,0.000000}%
\pgfsetstrokecolor{currentstroke}%
\pgfsetdash{}{0pt}%
\pgfpathmoveto{\pgfqpoint{2.647656in}{2.466493in}}%
\pgfpathlineto{\pgfqpoint{2.661046in}{2.456237in}}%
\pgfpathlineto{\pgfqpoint{2.674439in}{2.446018in}}%
\pgfpathlineto{\pgfqpoint{2.687833in}{2.435837in}}%
\pgfpathlineto{\pgfqpoint{2.701229in}{2.425693in}}%
\pgfpathlineto{\pgfqpoint{2.692211in}{2.442668in}}%
\pgfpathlineto{\pgfqpoint{2.683161in}{2.460277in}}%
\pgfpathlineto{\pgfqpoint{2.674078in}{2.478532in}}%
\pgfpathlineto{\pgfqpoint{2.664962in}{2.497445in}}%
\pgfpathlineto{\pgfqpoint{2.651512in}{2.508005in}}%
\pgfpathlineto{\pgfqpoint{2.638063in}{2.518603in}}%
\pgfpathlineto{\pgfqpoint{2.624616in}{2.529239in}}%
\pgfpathlineto{\pgfqpoint{2.611170in}{2.539913in}}%
\pgfpathlineto{\pgfqpoint{2.620343in}{2.520576in}}%
\pgfpathlineto{\pgfqpoint{2.629481in}{2.501903in}}%
\pgfpathlineto{\pgfqpoint{2.638585in}{2.483879in}}%
\pgfpathlineto{\pgfqpoint{2.647656in}{2.466493in}}%
\pgfpathclose%
\pgfusepath{fill}%
\end{pgfscope}%
\begin{pgfscope}%
\pgfpathrectangle{\pgfqpoint{1.254980in}{0.150000in}}{\pgfqpoint{5.490039in}{5.490039in}}%
\pgfusepath{clip}%
\pgfsetbuttcap%
\pgfsetroundjoin%
\definecolor{currentfill}{rgb}{0.272594,0.025563,0.353093}%
\pgfsetfillcolor{currentfill}%
\pgfsetfillopacity{0.700000}%
\pgfsetlinewidth{0.000000pt}%
\definecolor{currentstroke}{rgb}{0.000000,0.000000,0.000000}%
\pgfsetstrokecolor{currentstroke}%
\pgfsetdash{}{0pt}%
\pgfpathmoveto{\pgfqpoint{4.522901in}{1.490978in}}%
\pgfpathlineto{\pgfqpoint{4.536556in}{1.486764in}}%
\pgfpathlineto{\pgfqpoint{4.550217in}{1.482573in}}%
\pgfpathlineto{\pgfqpoint{4.563885in}{1.478406in}}%
\pgfpathlineto{\pgfqpoint{4.577560in}{1.474261in}}%
\pgfpathlineto{\pgfqpoint{4.569938in}{1.468085in}}%
\pgfpathlineto{\pgfqpoint{4.562312in}{1.462125in}}%
\pgfpathlineto{\pgfqpoint{4.554682in}{1.456389in}}%
\pgfpathlineto{\pgfqpoint{4.547047in}{1.450885in}}%
\pgfpathlineto{\pgfqpoint{4.533359in}{1.455318in}}%
\pgfpathlineto{\pgfqpoint{4.519677in}{1.459775in}}%
\pgfpathlineto{\pgfqpoint{4.506002in}{1.464254in}}%
\pgfpathlineto{\pgfqpoint{4.492332in}{1.468756in}}%
\pgfpathlineto{\pgfqpoint{4.499981in}{1.473967in}}%
\pgfpathlineto{\pgfqpoint{4.507626in}{1.479412in}}%
\pgfpathlineto{\pgfqpoint{4.515265in}{1.485085in}}%
\pgfpathlineto{\pgfqpoint{4.522901in}{1.490978in}}%
\pgfpathclose%
\pgfusepath{fill}%
\end{pgfscope}%
\begin{pgfscope}%
\pgfpathrectangle{\pgfqpoint{1.254980in}{0.150000in}}{\pgfqpoint{5.490039in}{5.490039in}}%
\pgfusepath{clip}%
\pgfsetbuttcap%
\pgfsetroundjoin%
\definecolor{currentfill}{rgb}{0.273809,0.031497,0.358853}%
\pgfsetfillcolor{currentfill}%
\pgfsetfillopacity{0.700000}%
\pgfsetlinewidth{0.000000pt}%
\definecolor{currentstroke}{rgb}{0.000000,0.000000,0.000000}%
\pgfsetstrokecolor{currentstroke}%
\pgfsetdash{}{0pt}%
\pgfpathmoveto{\pgfqpoint{4.383204in}{1.505606in}}%
\pgfpathlineto{\pgfqpoint{4.396823in}{1.500919in}}%
\pgfpathlineto{\pgfqpoint{4.410449in}{1.496254in}}%
\pgfpathlineto{\pgfqpoint{4.424080in}{1.491614in}}%
\pgfpathlineto{\pgfqpoint{4.437718in}{1.486996in}}%
\pgfpathlineto{\pgfqpoint{4.430049in}{1.482324in}}%
\pgfpathlineto{\pgfqpoint{4.422375in}{1.477906in}}%
\pgfpathlineto{\pgfqpoint{4.414696in}{1.473749in}}%
\pgfpathlineto{\pgfqpoint{4.407011in}{1.469862in}}%
\pgfpathlineto{\pgfqpoint{4.393357in}{1.474781in}}%
\pgfpathlineto{\pgfqpoint{4.379709in}{1.479723in}}%
\pgfpathlineto{\pgfqpoint{4.366066in}{1.484689in}}%
\pgfpathlineto{\pgfqpoint{4.352430in}{1.489678in}}%
\pgfpathlineto{\pgfqpoint{4.360132in}{1.493259in}}%
\pgfpathlineto{\pgfqpoint{4.367828in}{1.497113in}}%
\pgfpathlineto{\pgfqpoint{4.375519in}{1.501231in}}%
\pgfpathlineto{\pgfqpoint{4.383204in}{1.505606in}}%
\pgfpathclose%
\pgfusepath{fill}%
\end{pgfscope}%
\begin{pgfscope}%
\pgfpathrectangle{\pgfqpoint{1.254980in}{0.150000in}}{\pgfqpoint{5.490039in}{5.490039in}}%
\pgfusepath{clip}%
\pgfsetbuttcap%
\pgfsetroundjoin%
\definecolor{currentfill}{rgb}{0.274952,0.037752,0.364543}%
\pgfsetfillcolor{currentfill}%
\pgfsetfillopacity{0.700000}%
\pgfsetlinewidth{0.000000pt}%
\definecolor{currentstroke}{rgb}{0.000000,0.000000,0.000000}%
\pgfsetstrokecolor{currentstroke}%
\pgfsetdash{}{0pt}%
\pgfpathmoveto{\pgfqpoint{4.887899in}{1.512148in}}%
\pgfpathlineto{\pgfqpoint{4.901661in}{1.509182in}}%
\pgfpathlineto{\pgfqpoint{4.915430in}{1.506240in}}%
\pgfpathlineto{\pgfqpoint{4.929207in}{1.503320in}}%
\pgfpathlineto{\pgfqpoint{4.942991in}{1.500424in}}%
\pgfpathlineto{\pgfqpoint{4.935464in}{1.490962in}}%
\pgfpathlineto{\pgfqpoint{4.927933in}{1.481619in}}%
\pgfpathlineto{\pgfqpoint{4.920400in}{1.472402in}}%
\pgfpathlineto{\pgfqpoint{4.912863in}{1.463316in}}%
\pgfpathlineto{\pgfqpoint{4.899071in}{1.466464in}}%
\pgfpathlineto{\pgfqpoint{4.885286in}{1.469634in}}%
\pgfpathlineto{\pgfqpoint{4.871509in}{1.472827in}}%
\pgfpathlineto{\pgfqpoint{4.857739in}{1.476043in}}%
\pgfpathlineto{\pgfqpoint{4.865284in}{1.484873in}}%
\pgfpathlineto{\pgfqpoint{4.872826in}{1.493838in}}%
\pgfpathlineto{\pgfqpoint{4.880364in}{1.502931in}}%
\pgfpathlineto{\pgfqpoint{4.887899in}{1.512148in}}%
\pgfpathclose%
\pgfusepath{fill}%
\end{pgfscope}%
\begin{pgfscope}%
\pgfpathrectangle{\pgfqpoint{1.254980in}{0.150000in}}{\pgfqpoint{5.490039in}{5.490039in}}%
\pgfusepath{clip}%
\pgfsetbuttcap%
\pgfsetroundjoin%
\definecolor{currentfill}{rgb}{0.271305,0.019942,0.347269}%
\pgfsetfillcolor{currentfill}%
\pgfsetfillopacity{0.700000}%
\pgfsetlinewidth{0.000000pt}%
\definecolor{currentstroke}{rgb}{0.000000,0.000000,0.000000}%
\pgfsetstrokecolor{currentstroke}%
\pgfsetdash{}{0pt}%
\pgfpathmoveto{\pgfqpoint{4.662720in}{1.485743in}}%
\pgfpathlineto{\pgfqpoint{4.676416in}{1.481989in}}%
\pgfpathlineto{\pgfqpoint{4.690118in}{1.478258in}}%
\pgfpathlineto{\pgfqpoint{4.703828in}{1.474550in}}%
\pgfpathlineto{\pgfqpoint{4.717544in}{1.470865in}}%
\pgfpathlineto{\pgfqpoint{4.709962in}{1.463352in}}%
\pgfpathlineto{\pgfqpoint{4.702377in}{1.456020in}}%
\pgfpathlineto{\pgfqpoint{4.694788in}{1.448877in}}%
\pgfpathlineto{\pgfqpoint{4.687195in}{1.441930in}}%
\pgfpathlineto{\pgfqpoint{4.673467in}{1.445892in}}%
\pgfpathlineto{\pgfqpoint{4.659746in}{1.449876in}}%
\pgfpathlineto{\pgfqpoint{4.646032in}{1.453883in}}%
\pgfpathlineto{\pgfqpoint{4.632324in}{1.457912in}}%
\pgfpathlineto{\pgfqpoint{4.639929in}{1.464579in}}%
\pgfpathlineto{\pgfqpoint{4.647530in}{1.471444in}}%
\pgfpathlineto{\pgfqpoint{4.655127in}{1.478501in}}%
\pgfpathlineto{\pgfqpoint{4.662720in}{1.485743in}}%
\pgfpathclose%
\pgfusepath{fill}%
\end{pgfscope}%
\begin{pgfscope}%
\pgfpathrectangle{\pgfqpoint{1.254980in}{0.150000in}}{\pgfqpoint{5.490039in}{5.490039in}}%
\pgfusepath{clip}%
\pgfsetbuttcap%
\pgfsetroundjoin%
\definecolor{currentfill}{rgb}{0.278826,0.175490,0.483397}%
\pgfsetfillcolor{currentfill}%
\pgfsetfillopacity{0.700000}%
\pgfsetlinewidth{0.000000pt}%
\definecolor{currentstroke}{rgb}{0.000000,0.000000,0.000000}%
\pgfsetstrokecolor{currentstroke}%
\pgfsetdash{}{0pt}%
\pgfpathmoveto{\pgfqpoint{3.661935in}{1.757507in}}%
\pgfpathlineto{\pgfqpoint{3.675415in}{1.750464in}}%
\pgfpathlineto{\pgfqpoint{3.688899in}{1.743448in}}%
\pgfpathlineto{\pgfqpoint{3.702387in}{1.736457in}}%
\pgfpathlineto{\pgfqpoint{3.715879in}{1.729491in}}%
\pgfpathlineto{\pgfqpoint{3.707823in}{1.733743in}}%
\pgfpathlineto{\pgfqpoint{3.699754in}{1.738426in}}%
\pgfpathlineto{\pgfqpoint{3.691671in}{1.743551in}}%
\pgfpathlineto{\pgfqpoint{3.683574in}{1.749128in}}%
\pgfpathlineto{\pgfqpoint{3.670049in}{1.756451in}}%
\pgfpathlineto{\pgfqpoint{3.656529in}{1.763800in}}%
\pgfpathlineto{\pgfqpoint{3.643013in}{1.771174in}}%
\pgfpathlineto{\pgfqpoint{3.629502in}{1.778575in}}%
\pgfpathlineto{\pgfqpoint{3.637631in}{1.772634in}}%
\pgfpathlineto{\pgfqpoint{3.645747in}{1.767149in}}%
\pgfpathlineto{\pgfqpoint{3.653848in}{1.762110in}}%
\pgfpathlineto{\pgfqpoint{3.661935in}{1.757507in}}%
\pgfpathclose%
\pgfusepath{fill}%
\end{pgfscope}%
\begin{pgfscope}%
\pgfpathrectangle{\pgfqpoint{1.254980in}{0.150000in}}{\pgfqpoint{5.490039in}{5.490039in}}%
\pgfusepath{clip}%
\pgfsetbuttcap%
\pgfsetroundjoin%
\definecolor{currentfill}{rgb}{0.277018,0.050344,0.375715}%
\pgfsetfillcolor{currentfill}%
\pgfsetfillopacity{0.700000}%
\pgfsetlinewidth{0.000000pt}%
\definecolor{currentstroke}{rgb}{0.000000,0.000000,0.000000}%
\pgfsetstrokecolor{currentstroke}%
\pgfsetdash{}{0pt}%
\pgfpathmoveto{\pgfqpoint{4.243554in}{1.530428in}}%
\pgfpathlineto{\pgfqpoint{4.257143in}{1.525252in}}%
\pgfpathlineto{\pgfqpoint{4.270738in}{1.520100in}}%
\pgfpathlineto{\pgfqpoint{4.284338in}{1.514971in}}%
\pgfpathlineto{\pgfqpoint{4.297945in}{1.509866in}}%
\pgfpathlineto{\pgfqpoint{4.290219in}{1.506875in}}%
\pgfpathlineto{\pgfqpoint{4.282487in}{1.504177in}}%
\pgfpathlineto{\pgfqpoint{4.274749in}{1.501779in}}%
\pgfpathlineto{\pgfqpoint{4.267004in}{1.499690in}}%
\pgfpathlineto{\pgfqpoint{4.253378in}{1.505110in}}%
\pgfpathlineto{\pgfqpoint{4.239758in}{1.510553in}}%
\pgfpathlineto{\pgfqpoint{4.226144in}{1.516020in}}%
\pgfpathlineto{\pgfqpoint{4.212535in}{1.521511in}}%
\pgfpathlineto{\pgfqpoint{4.220300in}{1.523280in}}%
\pgfpathlineto{\pgfqpoint{4.228058in}{1.525362in}}%
\pgfpathlineto{\pgfqpoint{4.235809in}{1.527747in}}%
\pgfpathlineto{\pgfqpoint{4.243554in}{1.530428in}}%
\pgfpathclose%
\pgfusepath{fill}%
\end{pgfscope}%
\begin{pgfscope}%
\pgfpathrectangle{\pgfqpoint{1.254980in}{0.150000in}}{\pgfqpoint{5.490039in}{5.490039in}}%
\pgfusepath{clip}%
\pgfsetbuttcap%
\pgfsetroundjoin%
\definecolor{currentfill}{rgb}{0.281446,0.084320,0.407414}%
\pgfsetfillcolor{currentfill}%
\pgfsetfillopacity{0.700000}%
\pgfsetlinewidth{0.000000pt}%
\definecolor{currentstroke}{rgb}{0.000000,0.000000,0.000000}%
\pgfsetstrokecolor{currentstroke}%
\pgfsetdash{}{0pt}%
\pgfpathmoveto{\pgfqpoint{5.198821in}{1.596291in}}%
\pgfpathlineto{\pgfqpoint{5.212687in}{1.594370in}}%
\pgfpathlineto{\pgfqpoint{5.226562in}{1.592472in}}%
\pgfpathlineto{\pgfqpoint{5.240445in}{1.590597in}}%
\pgfpathlineto{\pgfqpoint{5.254336in}{1.588746in}}%
\pgfpathlineto{\pgfqpoint{5.246874in}{1.577471in}}%
\pgfpathlineto{\pgfqpoint{5.239408in}{1.566235in}}%
\pgfpathlineto{\pgfqpoint{5.231939in}{1.555043in}}%
\pgfpathlineto{\pgfqpoint{5.224466in}{1.543900in}}%
\pgfpathlineto{\pgfqpoint{5.210570in}{1.545965in}}%
\pgfpathlineto{\pgfqpoint{5.196682in}{1.548054in}}%
\pgfpathlineto{\pgfqpoint{5.182802in}{1.550165in}}%
\pgfpathlineto{\pgfqpoint{5.168930in}{1.552299in}}%
\pgfpathlineto{\pgfqpoint{5.176408in}{1.563223in}}%
\pgfpathlineto{\pgfqpoint{5.183883in}{1.574200in}}%
\pgfpathlineto{\pgfqpoint{5.191353in}{1.585225in}}%
\pgfpathlineto{\pgfqpoint{5.198821in}{1.596291in}}%
\pgfpathclose%
\pgfusepath{fill}%
\end{pgfscope}%
\begin{pgfscope}%
\pgfpathrectangle{\pgfqpoint{1.254980in}{0.150000in}}{\pgfqpoint{5.490039in}{5.490039in}}%
\pgfusepath{clip}%
\pgfsetbuttcap%
\pgfsetroundjoin%
\definecolor{currentfill}{rgb}{0.260571,0.246922,0.522828}%
\pgfsetfillcolor{currentfill}%
\pgfsetfillopacity{0.700000}%
\pgfsetlinewidth{0.000000pt}%
\definecolor{currentstroke}{rgb}{0.000000,0.000000,0.000000}%
\pgfsetstrokecolor{currentstroke}%
\pgfsetdash{}{0pt}%
\pgfpathmoveto{\pgfqpoint{3.413869in}{1.900555in}}%
\pgfpathlineto{\pgfqpoint{3.427316in}{1.892730in}}%
\pgfpathlineto{\pgfqpoint{3.440768in}{1.884932in}}%
\pgfpathlineto{\pgfqpoint{3.454223in}{1.877161in}}%
\pgfpathlineto{\pgfqpoint{3.467682in}{1.869418in}}%
\pgfpathlineto{\pgfqpoint{3.459432in}{1.876928in}}%
\pgfpathlineto{\pgfqpoint{3.451165in}{1.884925in}}%
\pgfpathlineto{\pgfqpoint{3.442880in}{1.893421in}}%
\pgfpathlineto{\pgfqpoint{3.434577in}{1.902427in}}%
\pgfpathlineto{\pgfqpoint{3.421081in}{1.910544in}}%
\pgfpathlineto{\pgfqpoint{3.407588in}{1.918689in}}%
\pgfpathlineto{\pgfqpoint{3.394099in}{1.926861in}}%
\pgfpathlineto{\pgfqpoint{3.380613in}{1.935061in}}%
\pgfpathlineto{\pgfqpoint{3.388955in}{1.925675in}}%
\pgfpathlineto{\pgfqpoint{3.397278in}{1.916803in}}%
\pgfpathlineto{\pgfqpoint{3.405582in}{1.908433in}}%
\pgfpathlineto{\pgfqpoint{3.413869in}{1.900555in}}%
\pgfpathclose%
\pgfusepath{fill}%
\end{pgfscope}%
\begin{pgfscope}%
\pgfpathrectangle{\pgfqpoint{1.254980in}{0.150000in}}{\pgfqpoint{5.490039in}{5.490039in}}%
\pgfusepath{clip}%
\pgfsetbuttcap%
\pgfsetroundjoin%
\definecolor{currentfill}{rgb}{0.165117,0.467423,0.558141}%
\pgfsetfillcolor{currentfill}%
\pgfsetfillopacity{0.700000}%
\pgfsetlinewidth{0.000000pt}%
\definecolor{currentstroke}{rgb}{0.000000,0.000000,0.000000}%
\pgfsetstrokecolor{currentstroke}%
\pgfsetdash{}{0pt}%
\pgfpathmoveto{\pgfqpoint{2.701229in}{2.425693in}}%
\pgfpathlineto{\pgfqpoint{2.714626in}{2.415586in}}%
\pgfpathlineto{\pgfqpoint{2.728026in}{2.405515in}}%
\pgfpathlineto{\pgfqpoint{2.741427in}{2.395482in}}%
\pgfpathlineto{\pgfqpoint{2.754831in}{2.385484in}}%
\pgfpathlineto{\pgfqpoint{2.745866in}{2.402051in}}%
\pgfpathlineto{\pgfqpoint{2.736869in}{2.419246in}}%
\pgfpathlineto{\pgfqpoint{2.727841in}{2.437082in}}%
\pgfpathlineto{\pgfqpoint{2.718780in}{2.455572in}}%
\pgfpathlineto{\pgfqpoint{2.705323in}{2.465986in}}%
\pgfpathlineto{\pgfqpoint{2.691868in}{2.476435in}}%
\pgfpathlineto{\pgfqpoint{2.678414in}{2.486921in}}%
\pgfpathlineto{\pgfqpoint{2.664962in}{2.497445in}}%
\pgfpathlineto{\pgfqpoint{2.674078in}{2.478532in}}%
\pgfpathlineto{\pgfqpoint{2.683161in}{2.460277in}}%
\pgfpathlineto{\pgfqpoint{2.692211in}{2.442668in}}%
\pgfpathlineto{\pgfqpoint{2.701229in}{2.425693in}}%
\pgfpathclose%
\pgfusepath{fill}%
\end{pgfscope}%
\begin{pgfscope}%
\pgfpathrectangle{\pgfqpoint{1.254980in}{0.150000in}}{\pgfqpoint{5.490039in}{5.490039in}}%
\pgfusepath{clip}%
\pgfsetbuttcap%
\pgfsetroundjoin%
\definecolor{currentfill}{rgb}{0.221989,0.339161,0.548752}%
\pgfsetfillcolor{currentfill}%
\pgfsetfillopacity{0.700000}%
\pgfsetlinewidth{0.000000pt}%
\definecolor{currentstroke}{rgb}{0.000000,0.000000,0.000000}%
\pgfsetstrokecolor{currentstroke}%
\pgfsetdash{}{0pt}%
\pgfpathmoveto{\pgfqpoint{3.111620in}{2.105021in}}%
\pgfpathlineto{\pgfqpoint{3.125039in}{2.096244in}}%
\pgfpathlineto{\pgfqpoint{3.138461in}{2.087497in}}%
\pgfpathlineto{\pgfqpoint{3.151886in}{2.078781in}}%
\pgfpathlineto{\pgfqpoint{3.165315in}{2.070094in}}%
\pgfpathlineto{\pgfqpoint{3.156788in}{2.081545in}}%
\pgfpathlineto{\pgfqpoint{3.148239in}{2.093548in}}%
\pgfpathlineto{\pgfqpoint{3.139666in}{2.106114in}}%
\pgfpathlineto{\pgfqpoint{3.131069in}{2.119254in}}%
\pgfpathlineto{\pgfqpoint{3.117597in}{2.128333in}}%
\pgfpathlineto{\pgfqpoint{3.104127in}{2.137443in}}%
\pgfpathlineto{\pgfqpoint{3.090660in}{2.146582in}}%
\pgfpathlineto{\pgfqpoint{3.077196in}{2.155753in}}%
\pgfpathlineto{\pgfqpoint{3.085839in}{2.142213in}}%
\pgfpathlineto{\pgfqpoint{3.094456in}{2.129253in}}%
\pgfpathlineto{\pgfqpoint{3.103050in}{2.116859in}}%
\pgfpathlineto{\pgfqpoint{3.111620in}{2.105021in}}%
\pgfpathclose%
\pgfusepath{fill}%
\end{pgfscope}%
\begin{pgfscope}%
\pgfpathrectangle{\pgfqpoint{1.254980in}{0.150000in}}{\pgfqpoint{5.490039in}{5.490039in}}%
\pgfusepath{clip}%
\pgfsetbuttcap%
\pgfsetroundjoin%
\definecolor{currentfill}{rgb}{0.279566,0.067836,0.391917}%
\pgfsetfillcolor{currentfill}%
\pgfsetfillopacity{0.700000}%
\pgfsetlinewidth{0.000000pt}%
\definecolor{currentstroke}{rgb}{0.000000,0.000000,0.000000}%
\pgfsetstrokecolor{currentstroke}%
\pgfsetdash{}{0pt}%
\pgfpathmoveto{\pgfqpoint{5.113523in}{1.561066in}}%
\pgfpathlineto{\pgfqpoint{5.127363in}{1.558839in}}%
\pgfpathlineto{\pgfqpoint{5.141211in}{1.556636in}}%
\pgfpathlineto{\pgfqpoint{5.155066in}{1.554456in}}%
\pgfpathlineto{\pgfqpoint{5.168930in}{1.552299in}}%
\pgfpathlineto{\pgfqpoint{5.161449in}{1.541432in}}%
\pgfpathlineto{\pgfqpoint{5.153964in}{1.530629in}}%
\pgfpathlineto{\pgfqpoint{5.146476in}{1.519894in}}%
\pgfpathlineto{\pgfqpoint{5.138985in}{1.509233in}}%
\pgfpathlineto{\pgfqpoint{5.125115in}{1.511617in}}%
\pgfpathlineto{\pgfqpoint{5.111254in}{1.514023in}}%
\pgfpathlineto{\pgfqpoint{5.097400in}{1.516452in}}%
\pgfpathlineto{\pgfqpoint{5.083554in}{1.518904in}}%
\pgfpathlineto{\pgfqpoint{5.091051in}{1.529334in}}%
\pgfpathlineto{\pgfqpoint{5.098545in}{1.539841in}}%
\pgfpathlineto{\pgfqpoint{5.106036in}{1.550420in}}%
\pgfpathlineto{\pgfqpoint{5.113523in}{1.561066in}}%
\pgfpathclose%
\pgfusepath{fill}%
\end{pgfscope}%
\begin{pgfscope}%
\pgfpathrectangle{\pgfqpoint{1.254980in}{0.150000in}}{\pgfqpoint{5.490039in}{5.490039in}}%
\pgfusepath{clip}%
\pgfsetbuttcap%
\pgfsetroundjoin%
\definecolor{currentfill}{rgb}{0.272594,0.025563,0.353093}%
\pgfsetfillcolor{currentfill}%
\pgfsetfillopacity{0.700000}%
\pgfsetlinewidth{0.000000pt}%
\definecolor{currentstroke}{rgb}{0.000000,0.000000,0.000000}%
\pgfsetstrokecolor{currentstroke}%
\pgfsetdash{}{0pt}%
\pgfpathmoveto{\pgfqpoint{4.802730in}{1.489136in}}%
\pgfpathlineto{\pgfqpoint{4.816471in}{1.485828in}}%
\pgfpathlineto{\pgfqpoint{4.830220in}{1.482544in}}%
\pgfpathlineto{\pgfqpoint{4.843976in}{1.479282in}}%
\pgfpathlineto{\pgfqpoint{4.857739in}{1.476043in}}%
\pgfpathlineto{\pgfqpoint{4.850190in}{1.467355in}}%
\pgfpathlineto{\pgfqpoint{4.842639in}{1.458815in}}%
\pgfpathlineto{\pgfqpoint{4.835084in}{1.450430in}}%
\pgfpathlineto{\pgfqpoint{4.827526in}{1.442206in}}%
\pgfpathlineto{\pgfqpoint{4.813754in}{1.445708in}}%
\pgfpathlineto{\pgfqpoint{4.799989in}{1.449234in}}%
\pgfpathlineto{\pgfqpoint{4.786230in}{1.452782in}}%
\pgfpathlineto{\pgfqpoint{4.772479in}{1.456353in}}%
\pgfpathlineto{\pgfqpoint{4.780047in}{1.464308in}}%
\pgfpathlineto{\pgfqpoint{4.787611in}{1.472428in}}%
\pgfpathlineto{\pgfqpoint{4.795172in}{1.480706in}}%
\pgfpathlineto{\pgfqpoint{4.802730in}{1.489136in}}%
\pgfpathclose%
\pgfusepath{fill}%
\end{pgfscope}%
\begin{pgfscope}%
\pgfpathrectangle{\pgfqpoint{1.254980in}{0.150000in}}{\pgfqpoint{5.490039in}{5.490039in}}%
\pgfusepath{clip}%
\pgfsetbuttcap%
\pgfsetroundjoin%
\definecolor{currentfill}{rgb}{0.283229,0.120777,0.440584}%
\pgfsetfillcolor{currentfill}%
\pgfsetfillopacity{0.700000}%
\pgfsetlinewidth{0.000000pt}%
\definecolor{currentstroke}{rgb}{0.000000,0.000000,0.000000}%
\pgfsetstrokecolor{currentstroke}%
\pgfsetdash{}{0pt}%
\pgfpathmoveto{\pgfqpoint{3.909910in}{1.639102in}}%
\pgfpathlineto{\pgfqpoint{3.923436in}{1.632804in}}%
\pgfpathlineto{\pgfqpoint{3.936967in}{1.626531in}}%
\pgfpathlineto{\pgfqpoint{3.950503in}{1.620282in}}%
\pgfpathlineto{\pgfqpoint{3.964044in}{1.614057in}}%
\pgfpathlineto{\pgfqpoint{3.956145in}{1.615327in}}%
\pgfpathlineto{\pgfqpoint{3.948237in}{1.616977in}}%
\pgfpathlineto{\pgfqpoint{3.940319in}{1.619016in}}%
\pgfpathlineto{\pgfqpoint{3.932390in}{1.621452in}}%
\pgfpathlineto{\pgfqpoint{3.918822in}{1.628019in}}%
\pgfpathlineto{\pgfqpoint{3.905260in}{1.634611in}}%
\pgfpathlineto{\pgfqpoint{3.891702in}{1.641227in}}%
\pgfpathlineto{\pgfqpoint{3.878149in}{1.647868in}}%
\pgfpathlineto{\pgfqpoint{3.886105in}{1.645084in}}%
\pgfpathlineto{\pgfqpoint{3.894051in}{1.642701in}}%
\pgfpathlineto{\pgfqpoint{3.901985in}{1.640710in}}%
\pgfpathlineto{\pgfqpoint{3.909910in}{1.639102in}}%
\pgfpathclose%
\pgfusepath{fill}%
\end{pgfscope}%
\begin{pgfscope}%
\pgfpathrectangle{\pgfqpoint{1.254980in}{0.150000in}}{\pgfqpoint{5.490039in}{5.490039in}}%
\pgfusepath{clip}%
\pgfsetbuttcap%
\pgfsetroundjoin%
\definecolor{currentfill}{rgb}{0.280894,0.078907,0.402329}%
\pgfsetfillcolor{currentfill}%
\pgfsetfillopacity{0.700000}%
\pgfsetlinewidth{0.000000pt}%
\definecolor{currentstroke}{rgb}{0.000000,0.000000,0.000000}%
\pgfsetstrokecolor{currentstroke}%
\pgfsetdash{}{0pt}%
\pgfpathmoveto{\pgfqpoint{4.103865in}{1.566284in}}%
\pgfpathlineto{\pgfqpoint{4.117430in}{1.560604in}}%
\pgfpathlineto{\pgfqpoint{4.131000in}{1.554949in}}%
\pgfpathlineto{\pgfqpoint{4.144575in}{1.549316in}}%
\pgfpathlineto{\pgfqpoint{4.158156in}{1.543708in}}%
\pgfpathlineto{\pgfqpoint{4.150363in}{1.542583in}}%
\pgfpathlineto{\pgfqpoint{4.142562in}{1.541790in}}%
\pgfpathlineto{\pgfqpoint{4.134753in}{1.541339in}}%
\pgfpathlineto{\pgfqpoint{4.126937in}{1.541239in}}%
\pgfpathlineto{\pgfqpoint{4.113334in}{1.547176in}}%
\pgfpathlineto{\pgfqpoint{4.099736in}{1.553136in}}%
\pgfpathlineto{\pgfqpoint{4.086143in}{1.559120in}}%
\pgfpathlineto{\pgfqpoint{4.072556in}{1.565129in}}%
\pgfpathlineto{\pgfqpoint{4.080396in}{1.564895in}}%
\pgfpathlineto{\pgfqpoint{4.088227in}{1.565016in}}%
\pgfpathlineto{\pgfqpoint{4.096050in}{1.565482in}}%
\pgfpathlineto{\pgfqpoint{4.103865in}{1.566284in}}%
\pgfpathclose%
\pgfusepath{fill}%
\end{pgfscope}%
\begin{pgfscope}%
\pgfpathrectangle{\pgfqpoint{1.254980in}{0.150000in}}{\pgfqpoint{5.490039in}{5.490039in}}%
\pgfusepath{clip}%
\pgfsetbuttcap%
\pgfsetroundjoin%
\definecolor{currentfill}{rgb}{0.277018,0.050344,0.375715}%
\pgfsetfillcolor{currentfill}%
\pgfsetfillopacity{0.700000}%
\pgfsetlinewidth{0.000000pt}%
\definecolor{currentstroke}{rgb}{0.000000,0.000000,0.000000}%
\pgfsetstrokecolor{currentstroke}%
\pgfsetdash{}{0pt}%
\pgfpathmoveto{\pgfqpoint{5.028249in}{1.528941in}}%
\pgfpathlineto{\pgfqpoint{5.042063in}{1.526397in}}%
\pgfpathlineto{\pgfqpoint{5.055886in}{1.523877in}}%
\pgfpathlineto{\pgfqpoint{5.069716in}{1.521379in}}%
\pgfpathlineto{\pgfqpoint{5.083554in}{1.518904in}}%
\pgfpathlineto{\pgfqpoint{5.076054in}{1.508557in}}%
\pgfpathlineto{\pgfqpoint{5.068550in}{1.498300in}}%
\pgfpathlineto{\pgfqpoint{5.061044in}{1.488137in}}%
\pgfpathlineto{\pgfqpoint{5.053534in}{1.478075in}}%
\pgfpathlineto{\pgfqpoint{5.039689in}{1.480789in}}%
\pgfpathlineto{\pgfqpoint{5.025853in}{1.483525in}}%
\pgfpathlineto{\pgfqpoint{5.012023in}{1.486284in}}%
\pgfpathlineto{\pgfqpoint{4.998202in}{1.489067in}}%
\pgfpathlineto{\pgfqpoint{5.005718in}{1.498885in}}%
\pgfpathlineto{\pgfqpoint{5.013232in}{1.508808in}}%
\pgfpathlineto{\pgfqpoint{5.020742in}{1.518828in}}%
\pgfpathlineto{\pgfqpoint{5.028249in}{1.528941in}}%
\pgfpathclose%
\pgfusepath{fill}%
\end{pgfscope}%
\begin{pgfscope}%
\pgfpathrectangle{\pgfqpoint{1.254980in}{0.150000in}}{\pgfqpoint{5.490039in}{5.490039in}}%
\pgfusepath{clip}%
\pgfsetbuttcap%
\pgfsetroundjoin%
\definecolor{currentfill}{rgb}{0.140210,0.665859,0.513427}%
\pgfsetfillcolor{currentfill}%
\pgfsetfillopacity{0.700000}%
\pgfsetlinewidth{0.000000pt}%
\definecolor{currentstroke}{rgb}{0.000000,0.000000,0.000000}%
\pgfsetstrokecolor{currentstroke}%
\pgfsetdash{}{0pt}%
\pgfpathmoveto{\pgfqpoint{2.127751in}{2.952687in}}%
\pgfpathlineto{\pgfqpoint{2.141174in}{2.940381in}}%
\pgfpathlineto{\pgfqpoint{2.154596in}{2.928129in}}%
\pgfpathlineto{\pgfqpoint{2.168018in}{2.915930in}}%
\pgfpathlineto{\pgfqpoint{2.181440in}{2.903783in}}%
\pgfpathlineto{\pgfqpoint{2.171742in}{2.927271in}}%
\pgfpathlineto{\pgfqpoint{2.161999in}{2.951486in}}%
\pgfpathlineto{\pgfqpoint{2.152211in}{2.976443in}}%
\pgfpathlineto{\pgfqpoint{2.142376in}{3.002156in}}%
\pgfpathlineto{\pgfqpoint{2.128888in}{3.014750in}}%
\pgfpathlineto{\pgfqpoint{2.115399in}{3.027398in}}%
\pgfpathlineto{\pgfqpoint{2.101909in}{3.040099in}}%
\pgfpathlineto{\pgfqpoint{2.088419in}{3.052855in}}%
\pgfpathlineto{\pgfqpoint{2.098322in}{3.026685in}}%
\pgfpathlineto{\pgfqpoint{2.108178in}{3.001277in}}%
\pgfpathlineto{\pgfqpoint{2.117987in}{2.976615in}}%
\pgfpathlineto{\pgfqpoint{2.127751in}{2.952687in}}%
\pgfpathclose%
\pgfusepath{fill}%
\end{pgfscope}%
\begin{pgfscope}%
\pgfpathrectangle{\pgfqpoint{1.254980in}{0.150000in}}{\pgfqpoint{5.490039in}{5.490039in}}%
\pgfusepath{clip}%
\pgfsetbuttcap%
\pgfsetroundjoin%
\definecolor{currentfill}{rgb}{0.171176,0.452530,0.557965}%
\pgfsetfillcolor{currentfill}%
\pgfsetfillopacity{0.700000}%
\pgfsetlinewidth{0.000000pt}%
\definecolor{currentstroke}{rgb}{0.000000,0.000000,0.000000}%
\pgfsetstrokecolor{currentstroke}%
\pgfsetdash{}{0pt}%
\pgfpathmoveto{\pgfqpoint{2.754831in}{2.385484in}}%
\pgfpathlineto{\pgfqpoint{2.768236in}{2.375522in}}%
\pgfpathlineto{\pgfqpoint{2.781644in}{2.365596in}}%
\pgfpathlineto{\pgfqpoint{2.795053in}{2.355706in}}%
\pgfpathlineto{\pgfqpoint{2.808465in}{2.345850in}}%
\pgfpathlineto{\pgfqpoint{2.799552in}{2.362008in}}%
\pgfpathlineto{\pgfqpoint{2.790608in}{2.378791in}}%
\pgfpathlineto{\pgfqpoint{2.781633in}{2.396210in}}%
\pgfpathlineto{\pgfqpoint{2.772627in}{2.414279in}}%
\pgfpathlineto{\pgfqpoint{2.759162in}{2.424549in}}%
\pgfpathlineto{\pgfqpoint{2.745700in}{2.434854in}}%
\pgfpathlineto{\pgfqpoint{2.732239in}{2.445195in}}%
\pgfpathlineto{\pgfqpoint{2.718780in}{2.455572in}}%
\pgfpathlineto{\pgfqpoint{2.727841in}{2.437082in}}%
\pgfpathlineto{\pgfqpoint{2.736869in}{2.419246in}}%
\pgfpathlineto{\pgfqpoint{2.745866in}{2.402051in}}%
\pgfpathlineto{\pgfqpoint{2.754831in}{2.385484in}}%
\pgfpathclose%
\pgfusepath{fill}%
\end{pgfscope}%
\begin{pgfscope}%
\pgfpathrectangle{\pgfqpoint{1.254980in}{0.150000in}}{\pgfqpoint{5.490039in}{5.490039in}}%
\pgfusepath{clip}%
\pgfsetbuttcap%
\pgfsetroundjoin%
\definecolor{currentfill}{rgb}{0.273809,0.031497,0.358853}%
\pgfsetfillcolor{currentfill}%
\pgfsetfillopacity{0.700000}%
\pgfsetlinewidth{0.000000pt}%
\definecolor{currentstroke}{rgb}{0.000000,0.000000,0.000000}%
\pgfsetstrokecolor{currentstroke}%
\pgfsetdash{}{0pt}%
\pgfpathmoveto{\pgfqpoint{4.437718in}{1.486996in}}%
\pgfpathlineto{\pgfqpoint{4.451362in}{1.482401in}}%
\pgfpathlineto{\pgfqpoint{4.465013in}{1.477830in}}%
\pgfpathlineto{\pgfqpoint{4.478669in}{1.473281in}}%
\pgfpathlineto{\pgfqpoint{4.492332in}{1.468756in}}%
\pgfpathlineto{\pgfqpoint{4.484679in}{1.463788in}}%
\pgfpathlineto{\pgfqpoint{4.477020in}{1.459070in}}%
\pgfpathlineto{\pgfqpoint{4.469357in}{1.454610in}}%
\pgfpathlineto{\pgfqpoint{4.461689in}{1.450415in}}%
\pgfpathlineto{\pgfqpoint{4.448010in}{1.455242in}}%
\pgfpathlineto{\pgfqpoint{4.434338in}{1.460092in}}%
\pgfpathlineto{\pgfqpoint{4.420671in}{1.464966in}}%
\pgfpathlineto{\pgfqpoint{4.407011in}{1.469862in}}%
\pgfpathlineto{\pgfqpoint{4.414696in}{1.473749in}}%
\pgfpathlineto{\pgfqpoint{4.422375in}{1.477906in}}%
\pgfpathlineto{\pgfqpoint{4.430049in}{1.482324in}}%
\pgfpathlineto{\pgfqpoint{4.437718in}{1.486996in}}%
\pgfpathclose%
\pgfusepath{fill}%
\end{pgfscope}%
\begin{pgfscope}%
\pgfpathrectangle{\pgfqpoint{1.254980in}{0.150000in}}{\pgfqpoint{5.490039in}{5.490039in}}%
\pgfusepath{clip}%
\pgfsetbuttcap%
\pgfsetroundjoin%
\definecolor{currentfill}{rgb}{0.279574,0.170599,0.479997}%
\pgfsetfillcolor{currentfill}%
\pgfsetfillopacity{0.700000}%
\pgfsetlinewidth{0.000000pt}%
\definecolor{currentstroke}{rgb}{0.000000,0.000000,0.000000}%
\pgfsetstrokecolor{currentstroke}%
\pgfsetdash{}{0pt}%
\pgfpathmoveto{\pgfqpoint{3.715879in}{1.729491in}}%
\pgfpathlineto{\pgfqpoint{3.729377in}{1.722551in}}%
\pgfpathlineto{\pgfqpoint{3.742878in}{1.715637in}}%
\pgfpathlineto{\pgfqpoint{3.756385in}{1.708747in}}%
\pgfpathlineto{\pgfqpoint{3.769895in}{1.701883in}}%
\pgfpathlineto{\pgfqpoint{3.761869in}{1.705782in}}%
\pgfpathlineto{\pgfqpoint{3.753831in}{1.710110in}}%
\pgfpathlineto{\pgfqpoint{3.745779in}{1.714876in}}%
\pgfpathlineto{\pgfqpoint{3.737714in}{1.720089in}}%
\pgfpathlineto{\pgfqpoint{3.724173in}{1.727311in}}%
\pgfpathlineto{\pgfqpoint{3.710635in}{1.734558in}}%
\pgfpathlineto{\pgfqpoint{3.697102in}{1.741830in}}%
\pgfpathlineto{\pgfqpoint{3.683574in}{1.749128in}}%
\pgfpathlineto{\pgfqpoint{3.691671in}{1.743551in}}%
\pgfpathlineto{\pgfqpoint{3.699754in}{1.738426in}}%
\pgfpathlineto{\pgfqpoint{3.707823in}{1.733743in}}%
\pgfpathlineto{\pgfqpoint{3.715879in}{1.729491in}}%
\pgfpathclose%
\pgfusepath{fill}%
\end{pgfscope}%
\begin{pgfscope}%
\pgfpathrectangle{\pgfqpoint{1.254980in}{0.150000in}}{\pgfqpoint{5.490039in}{5.490039in}}%
\pgfusepath{clip}%
\pgfsetbuttcap%
\pgfsetroundjoin%
\definecolor{currentfill}{rgb}{0.272594,0.025563,0.353093}%
\pgfsetfillcolor{currentfill}%
\pgfsetfillopacity{0.700000}%
\pgfsetlinewidth{0.000000pt}%
\definecolor{currentstroke}{rgb}{0.000000,0.000000,0.000000}%
\pgfsetstrokecolor{currentstroke}%
\pgfsetdash{}{0pt}%
\pgfpathmoveto{\pgfqpoint{4.577560in}{1.474261in}}%
\pgfpathlineto{\pgfqpoint{4.591241in}{1.470140in}}%
\pgfpathlineto{\pgfqpoint{4.604929in}{1.466041in}}%
\pgfpathlineto{\pgfqpoint{4.618623in}{1.461965in}}%
\pgfpathlineto{\pgfqpoint{4.632324in}{1.457912in}}%
\pgfpathlineto{\pgfqpoint{4.624715in}{1.451452in}}%
\pgfpathlineto{\pgfqpoint{4.617103in}{1.445205in}}%
\pgfpathlineto{\pgfqpoint{4.609486in}{1.439179in}}%
\pgfpathlineto{\pgfqpoint{4.601865in}{1.433381in}}%
\pgfpathlineto{\pgfqpoint{4.588151in}{1.437723in}}%
\pgfpathlineto{\pgfqpoint{4.574443in}{1.442087in}}%
\pgfpathlineto{\pgfqpoint{4.560742in}{1.446475in}}%
\pgfpathlineto{\pgfqpoint{4.547047in}{1.450885in}}%
\pgfpathlineto{\pgfqpoint{4.554682in}{1.456389in}}%
\pgfpathlineto{\pgfqpoint{4.562312in}{1.462125in}}%
\pgfpathlineto{\pgfqpoint{4.569938in}{1.468085in}}%
\pgfpathlineto{\pgfqpoint{4.577560in}{1.474261in}}%
\pgfpathclose%
\pgfusepath{fill}%
\end{pgfscope}%
\begin{pgfscope}%
\pgfpathrectangle{\pgfqpoint{1.254980in}{0.150000in}}{\pgfqpoint{5.490039in}{5.490039in}}%
\pgfusepath{clip}%
\pgfsetbuttcap%
\pgfsetroundjoin%
\definecolor{currentfill}{rgb}{0.263663,0.237631,0.518762}%
\pgfsetfillcolor{currentfill}%
\pgfsetfillopacity{0.700000}%
\pgfsetlinewidth{0.000000pt}%
\definecolor{currentstroke}{rgb}{0.000000,0.000000,0.000000}%
\pgfsetstrokecolor{currentstroke}%
\pgfsetdash{}{0pt}%
\pgfpathmoveto{\pgfqpoint{3.467682in}{1.869418in}}%
\pgfpathlineto{\pgfqpoint{3.481145in}{1.861701in}}%
\pgfpathlineto{\pgfqpoint{3.494612in}{1.854012in}}%
\pgfpathlineto{\pgfqpoint{3.508083in}{1.846349in}}%
\pgfpathlineto{\pgfqpoint{3.521557in}{1.838713in}}%
\pgfpathlineto{\pgfqpoint{3.513343in}{1.845855in}}%
\pgfpathlineto{\pgfqpoint{3.505113in}{1.853481in}}%
\pgfpathlineto{\pgfqpoint{3.496865in}{1.861601in}}%
\pgfpathlineto{\pgfqpoint{3.488600in}{1.870226in}}%
\pgfpathlineto{\pgfqpoint{3.475088in}{1.878236in}}%
\pgfpathlineto{\pgfqpoint{3.461581in}{1.886273in}}%
\pgfpathlineto{\pgfqpoint{3.448077in}{1.894336in}}%
\pgfpathlineto{\pgfqpoint{3.434577in}{1.902427in}}%
\pgfpathlineto{\pgfqpoint{3.442880in}{1.893421in}}%
\pgfpathlineto{\pgfqpoint{3.451165in}{1.884925in}}%
\pgfpathlineto{\pgfqpoint{3.459432in}{1.876928in}}%
\pgfpathlineto{\pgfqpoint{3.467682in}{1.869418in}}%
\pgfpathclose%
\pgfusepath{fill}%
\end{pgfscope}%
\begin{pgfscope}%
\pgfpathrectangle{\pgfqpoint{1.254980in}{0.150000in}}{\pgfqpoint{5.490039in}{5.490039in}}%
\pgfusepath{clip}%
\pgfsetbuttcap%
\pgfsetroundjoin%
\definecolor{currentfill}{rgb}{0.225863,0.330805,0.547314}%
\pgfsetfillcolor{currentfill}%
\pgfsetfillopacity{0.700000}%
\pgfsetlinewidth{0.000000pt}%
\definecolor{currentstroke}{rgb}{0.000000,0.000000,0.000000}%
\pgfsetstrokecolor{currentstroke}%
\pgfsetdash{}{0pt}%
\pgfpathmoveto{\pgfqpoint{3.165315in}{2.070094in}}%
\pgfpathlineto{\pgfqpoint{3.178746in}{2.061437in}}%
\pgfpathlineto{\pgfqpoint{3.192181in}{2.052810in}}%
\pgfpathlineto{\pgfqpoint{3.205618in}{2.044212in}}%
\pgfpathlineto{\pgfqpoint{3.219059in}{2.035644in}}%
\pgfpathlineto{\pgfqpoint{3.210575in}{2.046709in}}%
\pgfpathlineto{\pgfqpoint{3.202069in}{2.058321in}}%
\pgfpathlineto{\pgfqpoint{3.193541in}{2.070493in}}%
\pgfpathlineto{\pgfqpoint{3.184989in}{2.083234in}}%
\pgfpathlineto{\pgfqpoint{3.171505in}{2.092195in}}%
\pgfpathlineto{\pgfqpoint{3.158023in}{2.101185in}}%
\pgfpathlineto{\pgfqpoint{3.144545in}{2.110204in}}%
\pgfpathlineto{\pgfqpoint{3.131069in}{2.119254in}}%
\pgfpathlineto{\pgfqpoint{3.139666in}{2.106114in}}%
\pgfpathlineto{\pgfqpoint{3.148239in}{2.093548in}}%
\pgfpathlineto{\pgfqpoint{3.156788in}{2.081545in}}%
\pgfpathlineto{\pgfqpoint{3.165315in}{2.070094in}}%
\pgfpathclose%
\pgfusepath{fill}%
\end{pgfscope}%
\begin{pgfscope}%
\pgfpathrectangle{\pgfqpoint{1.254980in}{0.150000in}}{\pgfqpoint{5.490039in}{5.490039in}}%
\pgfusepath{clip}%
\pgfsetbuttcap%
\pgfsetroundjoin%
\definecolor{currentfill}{rgb}{0.128087,0.647749,0.523491}%
\pgfsetfillcolor{currentfill}%
\pgfsetfillopacity{0.700000}%
\pgfsetlinewidth{0.000000pt}%
\definecolor{currentstroke}{rgb}{0.000000,0.000000,0.000000}%
\pgfsetstrokecolor{currentstroke}%
\pgfsetdash{}{0pt}%
\pgfpathmoveto{\pgfqpoint{2.181440in}{2.903783in}}%
\pgfpathlineto{\pgfqpoint{2.194862in}{2.891689in}}%
\pgfpathlineto{\pgfqpoint{2.208283in}{2.879646in}}%
\pgfpathlineto{\pgfqpoint{2.221705in}{2.867654in}}%
\pgfpathlineto{\pgfqpoint{2.235127in}{2.855712in}}%
\pgfpathlineto{\pgfqpoint{2.225493in}{2.878760in}}%
\pgfpathlineto{\pgfqpoint{2.215816in}{2.902531in}}%
\pgfpathlineto{\pgfqpoint{2.206095in}{2.927038in}}%
\pgfpathlineto{\pgfqpoint{2.196328in}{2.952296in}}%
\pgfpathlineto{\pgfqpoint{2.182841in}{2.964684in}}%
\pgfpathlineto{\pgfqpoint{2.169353in}{2.977123in}}%
\pgfpathlineto{\pgfqpoint{2.155865in}{2.989613in}}%
\pgfpathlineto{\pgfqpoint{2.142376in}{3.002156in}}%
\pgfpathlineto{\pgfqpoint{2.152211in}{2.976443in}}%
\pgfpathlineto{\pgfqpoint{2.161999in}{2.951486in}}%
\pgfpathlineto{\pgfqpoint{2.171742in}{2.927271in}}%
\pgfpathlineto{\pgfqpoint{2.181440in}{2.903783in}}%
\pgfpathclose%
\pgfusepath{fill}%
\end{pgfscope}%
\begin{pgfscope}%
\pgfpathrectangle{\pgfqpoint{1.254980in}{0.150000in}}{\pgfqpoint{5.490039in}{5.490039in}}%
\pgfusepath{clip}%
\pgfsetbuttcap%
\pgfsetroundjoin%
\definecolor{currentfill}{rgb}{0.277018,0.050344,0.375715}%
\pgfsetfillcolor{currentfill}%
\pgfsetfillopacity{0.700000}%
\pgfsetlinewidth{0.000000pt}%
\definecolor{currentstroke}{rgb}{0.000000,0.000000,0.000000}%
\pgfsetstrokecolor{currentstroke}%
\pgfsetdash{}{0pt}%
\pgfpathmoveto{\pgfqpoint{4.297945in}{1.509866in}}%
\pgfpathlineto{\pgfqpoint{4.311557in}{1.504784in}}%
\pgfpathlineto{\pgfqpoint{4.325176in}{1.499725in}}%
\pgfpathlineto{\pgfqpoint{4.338800in}{1.494690in}}%
\pgfpathlineto{\pgfqpoint{4.352430in}{1.489678in}}%
\pgfpathlineto{\pgfqpoint{4.344723in}{1.486378in}}%
\pgfpathlineto{\pgfqpoint{4.337009in}{1.483366in}}%
\pgfpathlineto{\pgfqpoint{4.329290in}{1.480651in}}%
\pgfpathlineto{\pgfqpoint{4.321565in}{1.478242in}}%
\pgfpathlineto{\pgfqpoint{4.307916in}{1.483569in}}%
\pgfpathlineto{\pgfqpoint{4.294273in}{1.488919in}}%
\pgfpathlineto{\pgfqpoint{4.280636in}{1.494293in}}%
\pgfpathlineto{\pgfqpoint{4.267004in}{1.499690in}}%
\pgfpathlineto{\pgfqpoint{4.274749in}{1.501779in}}%
\pgfpathlineto{\pgfqpoint{4.282487in}{1.504177in}}%
\pgfpathlineto{\pgfqpoint{4.290219in}{1.506875in}}%
\pgfpathlineto{\pgfqpoint{4.297945in}{1.509866in}}%
\pgfpathclose%
\pgfusepath{fill}%
\end{pgfscope}%
\begin{pgfscope}%
\pgfpathrectangle{\pgfqpoint{1.254980in}{0.150000in}}{\pgfqpoint{5.490039in}{5.490039in}}%
\pgfusepath{clip}%
\pgfsetbuttcap%
\pgfsetroundjoin%
\definecolor{currentfill}{rgb}{0.274952,0.037752,0.364543}%
\pgfsetfillcolor{currentfill}%
\pgfsetfillopacity{0.700000}%
\pgfsetlinewidth{0.000000pt}%
\definecolor{currentstroke}{rgb}{0.000000,0.000000,0.000000}%
\pgfsetstrokecolor{currentstroke}%
\pgfsetdash{}{0pt}%
\pgfpathmoveto{\pgfqpoint{4.942991in}{1.500424in}}%
\pgfpathlineto{\pgfqpoint{4.956782in}{1.497550in}}%
\pgfpathlineto{\pgfqpoint{4.970581in}{1.494700in}}%
\pgfpathlineto{\pgfqpoint{4.984388in}{1.491872in}}%
\pgfpathlineto{\pgfqpoint{4.998202in}{1.489067in}}%
\pgfpathlineto{\pgfqpoint{4.990682in}{1.479358in}}%
\pgfpathlineto{\pgfqpoint{4.983159in}{1.469766in}}%
\pgfpathlineto{\pgfqpoint{4.975634in}{1.460296in}}%
\pgfpathlineto{\pgfqpoint{4.968105in}{1.450954in}}%
\pgfpathlineto{\pgfqpoint{4.954284in}{1.454011in}}%
\pgfpathlineto{\pgfqpoint{4.940470in}{1.457090in}}%
\pgfpathlineto{\pgfqpoint{4.926663in}{1.460192in}}%
\pgfpathlineto{\pgfqpoint{4.912863in}{1.463316in}}%
\pgfpathlineto{\pgfqpoint{4.920400in}{1.472402in}}%
\pgfpathlineto{\pgfqpoint{4.927933in}{1.481619in}}%
\pgfpathlineto{\pgfqpoint{4.935464in}{1.490962in}}%
\pgfpathlineto{\pgfqpoint{4.942991in}{1.500424in}}%
\pgfpathclose%
\pgfusepath{fill}%
\end{pgfscope}%
\begin{pgfscope}%
\pgfpathrectangle{\pgfqpoint{1.254980in}{0.150000in}}{\pgfqpoint{5.490039in}{5.490039in}}%
\pgfusepath{clip}%
\pgfsetbuttcap%
\pgfsetroundjoin%
\definecolor{currentfill}{rgb}{0.272594,0.025563,0.353093}%
\pgfsetfillcolor{currentfill}%
\pgfsetfillopacity{0.700000}%
\pgfsetlinewidth{0.000000pt}%
\definecolor{currentstroke}{rgb}{0.000000,0.000000,0.000000}%
\pgfsetstrokecolor{currentstroke}%
\pgfsetdash{}{0pt}%
\pgfpathmoveto{\pgfqpoint{4.717544in}{1.470865in}}%
\pgfpathlineto{\pgfqpoint{4.731268in}{1.467202in}}%
\pgfpathlineto{\pgfqpoint{4.744998in}{1.463563in}}%
\pgfpathlineto{\pgfqpoint{4.758735in}{1.459946in}}%
\pgfpathlineto{\pgfqpoint{4.772479in}{1.456353in}}%
\pgfpathlineto{\pgfqpoint{4.764908in}{1.448568in}}%
\pgfpathlineto{\pgfqpoint{4.757333in}{1.440962in}}%
\pgfpathlineto{\pgfqpoint{4.749755in}{1.433542in}}%
\pgfpathlineto{\pgfqpoint{4.742174in}{1.426313in}}%
\pgfpathlineto{\pgfqpoint{4.728419in}{1.430183in}}%
\pgfpathlineto{\pgfqpoint{4.714671in}{1.434076in}}%
\pgfpathlineto{\pgfqpoint{4.700930in}{1.437992in}}%
\pgfpathlineto{\pgfqpoint{4.687195in}{1.441930in}}%
\pgfpathlineto{\pgfqpoint{4.694788in}{1.448877in}}%
\pgfpathlineto{\pgfqpoint{4.702377in}{1.456020in}}%
\pgfpathlineto{\pgfqpoint{4.709962in}{1.463352in}}%
\pgfpathlineto{\pgfqpoint{4.717544in}{1.470865in}}%
\pgfpathclose%
\pgfusepath{fill}%
\end{pgfscope}%
\begin{pgfscope}%
\pgfpathrectangle{\pgfqpoint{1.254980in}{0.150000in}}{\pgfqpoint{5.490039in}{5.490039in}}%
\pgfusepath{clip}%
\pgfsetbuttcap%
\pgfsetroundjoin%
\definecolor{currentfill}{rgb}{0.281924,0.089666,0.412415}%
\pgfsetfillcolor{currentfill}%
\pgfsetfillopacity{0.700000}%
\pgfsetlinewidth{0.000000pt}%
\definecolor{currentstroke}{rgb}{0.000000,0.000000,0.000000}%
\pgfsetstrokecolor{currentstroke}%
\pgfsetdash{}{0pt}%
\pgfpathmoveto{\pgfqpoint{5.254336in}{1.588746in}}%
\pgfpathlineto{\pgfqpoint{5.268235in}{1.586917in}}%
\pgfpathlineto{\pgfqpoint{5.282143in}{1.585111in}}%
\pgfpathlineto{\pgfqpoint{5.296059in}{1.583329in}}%
\pgfpathlineto{\pgfqpoint{5.288600in}{1.571897in}}%
\pgfpathlineto{\pgfqpoint{5.281138in}{1.560502in}}%
\pgfpathlineto{\pgfqpoint{5.273672in}{1.549149in}}%
\pgfpathlineto{\pgfqpoint{5.266203in}{1.537842in}}%
\pgfpathlineto{\pgfqpoint{5.252282in}{1.539839in}}%
\pgfpathlineto{\pgfqpoint{5.238370in}{1.541858in}}%
\pgfpathlineto{\pgfqpoint{5.224466in}{1.543900in}}%
\pgfpathlineto{\pgfqpoint{5.231939in}{1.555043in}}%
\pgfpathlineto{\pgfqpoint{5.239408in}{1.566235in}}%
\pgfpathlineto{\pgfqpoint{5.246874in}{1.577471in}}%
\pgfpathlineto{\pgfqpoint{5.254336in}{1.588746in}}%
\pgfpathclose%
\pgfusepath{fill}%
\end{pgfscope}%
\begin{pgfscope}%
\pgfpathrectangle{\pgfqpoint{1.254980in}{0.150000in}}{\pgfqpoint{5.490039in}{5.490039in}}%
\pgfusepath{clip}%
\pgfsetbuttcap%
\pgfsetroundjoin%
\definecolor{currentfill}{rgb}{0.175841,0.441290,0.557685}%
\pgfsetfillcolor{currentfill}%
\pgfsetfillopacity{0.700000}%
\pgfsetlinewidth{0.000000pt}%
\definecolor{currentstroke}{rgb}{0.000000,0.000000,0.000000}%
\pgfsetstrokecolor{currentstroke}%
\pgfsetdash{}{0pt}%
\pgfpathmoveto{\pgfqpoint{2.808465in}{2.345850in}}%
\pgfpathlineto{\pgfqpoint{2.821879in}{2.336029in}}%
\pgfpathlineto{\pgfqpoint{2.835295in}{2.326243in}}%
\pgfpathlineto{\pgfqpoint{2.848713in}{2.316491in}}%
\pgfpathlineto{\pgfqpoint{2.862134in}{2.306774in}}%
\pgfpathlineto{\pgfqpoint{2.853271in}{2.322525in}}%
\pgfpathlineto{\pgfqpoint{2.844379in}{2.338896in}}%
\pgfpathlineto{\pgfqpoint{2.835457in}{2.355899in}}%
\pgfpathlineto{\pgfqpoint{2.826504in}{2.373546in}}%
\pgfpathlineto{\pgfqpoint{2.813032in}{2.383678in}}%
\pgfpathlineto{\pgfqpoint{2.799561in}{2.393843in}}%
\pgfpathlineto{\pgfqpoint{2.786093in}{2.404044in}}%
\pgfpathlineto{\pgfqpoint{2.772627in}{2.414279in}}%
\pgfpathlineto{\pgfqpoint{2.781633in}{2.396210in}}%
\pgfpathlineto{\pgfqpoint{2.790608in}{2.378791in}}%
\pgfpathlineto{\pgfqpoint{2.799552in}{2.362008in}}%
\pgfpathlineto{\pgfqpoint{2.808465in}{2.345850in}}%
\pgfpathclose%
\pgfusepath{fill}%
\end{pgfscope}%
\begin{pgfscope}%
\pgfpathrectangle{\pgfqpoint{1.254980in}{0.150000in}}{\pgfqpoint{5.490039in}{5.490039in}}%
\pgfusepath{clip}%
\pgfsetbuttcap%
\pgfsetroundjoin%
\definecolor{currentfill}{rgb}{0.122312,0.633153,0.530398}%
\pgfsetfillcolor{currentfill}%
\pgfsetfillopacity{0.700000}%
\pgfsetlinewidth{0.000000pt}%
\definecolor{currentstroke}{rgb}{0.000000,0.000000,0.000000}%
\pgfsetstrokecolor{currentstroke}%
\pgfsetdash{}{0pt}%
\pgfpathmoveto{\pgfqpoint{2.235127in}{2.855712in}}%
\pgfpathlineto{\pgfqpoint{2.248548in}{2.843820in}}%
\pgfpathlineto{\pgfqpoint{2.261970in}{2.831978in}}%
\pgfpathlineto{\pgfqpoint{2.275393in}{2.820184in}}%
\pgfpathlineto{\pgfqpoint{2.288815in}{2.808439in}}%
\pgfpathlineto{\pgfqpoint{2.279245in}{2.831049in}}%
\pgfpathlineto{\pgfqpoint{2.269633in}{2.854376in}}%
\pgfpathlineto{\pgfqpoint{2.259977in}{2.878436in}}%
\pgfpathlineto{\pgfqpoint{2.250277in}{2.903241in}}%
\pgfpathlineto{\pgfqpoint{2.236790in}{2.915431in}}%
\pgfpathlineto{\pgfqpoint{2.223303in}{2.927670in}}%
\pgfpathlineto{\pgfqpoint{2.209816in}{2.939958in}}%
\pgfpathlineto{\pgfqpoint{2.196328in}{2.952296in}}%
\pgfpathlineto{\pgfqpoint{2.206095in}{2.927038in}}%
\pgfpathlineto{\pgfqpoint{2.215816in}{2.902531in}}%
\pgfpathlineto{\pgfqpoint{2.225493in}{2.878760in}}%
\pgfpathlineto{\pgfqpoint{2.235127in}{2.855712in}}%
\pgfpathclose%
\pgfusepath{fill}%
\end{pgfscope}%
\begin{pgfscope}%
\pgfpathrectangle{\pgfqpoint{1.254980in}{0.150000in}}{\pgfqpoint{5.490039in}{5.490039in}}%
\pgfusepath{clip}%
\pgfsetbuttcap%
\pgfsetroundjoin%
\definecolor{currentfill}{rgb}{0.283091,0.110553,0.431554}%
\pgfsetfillcolor{currentfill}%
\pgfsetfillopacity{0.700000}%
\pgfsetlinewidth{0.000000pt}%
\definecolor{currentstroke}{rgb}{0.000000,0.000000,0.000000}%
\pgfsetstrokecolor{currentstroke}%
\pgfsetdash{}{0pt}%
\pgfpathmoveto{\pgfqpoint{3.964044in}{1.614057in}}%
\pgfpathlineto{\pgfqpoint{3.977590in}{1.607856in}}%
\pgfpathlineto{\pgfqpoint{3.991141in}{1.601680in}}%
\pgfpathlineto{\pgfqpoint{4.004697in}{1.595528in}}%
\pgfpathlineto{\pgfqpoint{4.018258in}{1.589400in}}%
\pgfpathlineto{\pgfqpoint{4.010385in}{1.590334in}}%
\pgfpathlineto{\pgfqpoint{4.002503in}{1.591643in}}%
\pgfpathlineto{\pgfqpoint{3.994611in}{1.593337in}}%
\pgfpathlineto{\pgfqpoint{3.986709in}{1.595425in}}%
\pgfpathlineto{\pgfqpoint{3.973121in}{1.601896in}}%
\pgfpathlineto{\pgfqpoint{3.959539in}{1.608390in}}%
\pgfpathlineto{\pgfqpoint{3.945962in}{1.614909in}}%
\pgfpathlineto{\pgfqpoint{3.932390in}{1.621452in}}%
\pgfpathlineto{\pgfqpoint{3.940319in}{1.619016in}}%
\pgfpathlineto{\pgfqpoint{3.948237in}{1.616977in}}%
\pgfpathlineto{\pgfqpoint{3.956145in}{1.615327in}}%
\pgfpathlineto{\pgfqpoint{3.964044in}{1.614057in}}%
\pgfpathclose%
\pgfusepath{fill}%
\end{pgfscope}%
\begin{pgfscope}%
\pgfpathrectangle{\pgfqpoint{1.254980in}{0.150000in}}{\pgfqpoint{5.490039in}{5.490039in}}%
\pgfusepath{clip}%
\pgfsetbuttcap%
\pgfsetroundjoin%
\definecolor{currentfill}{rgb}{0.280267,0.073417,0.397163}%
\pgfsetfillcolor{currentfill}%
\pgfsetfillopacity{0.700000}%
\pgfsetlinewidth{0.000000pt}%
\definecolor{currentstroke}{rgb}{0.000000,0.000000,0.000000}%
\pgfsetstrokecolor{currentstroke}%
\pgfsetdash{}{0pt}%
\pgfpathmoveto{\pgfqpoint{4.158156in}{1.543708in}}%
\pgfpathlineto{\pgfqpoint{4.171742in}{1.538123in}}%
\pgfpathlineto{\pgfqpoint{4.185334in}{1.532562in}}%
\pgfpathlineto{\pgfqpoint{4.198932in}{1.527025in}}%
\pgfpathlineto{\pgfqpoint{4.212535in}{1.521511in}}%
\pgfpathlineto{\pgfqpoint{4.204763in}{1.520062in}}%
\pgfpathlineto{\pgfqpoint{4.196984in}{1.518943in}}%
\pgfpathlineto{\pgfqpoint{4.189198in}{1.518162in}}%
\pgfpathlineto{\pgfqpoint{4.181404in}{1.517728in}}%
\pgfpathlineto{\pgfqpoint{4.167779in}{1.523570in}}%
\pgfpathlineto{\pgfqpoint{4.154159in}{1.529436in}}%
\pgfpathlineto{\pgfqpoint{4.140545in}{1.535326in}}%
\pgfpathlineto{\pgfqpoint{4.126937in}{1.541239in}}%
\pgfpathlineto{\pgfqpoint{4.134753in}{1.541339in}}%
\pgfpathlineto{\pgfqpoint{4.142562in}{1.541790in}}%
\pgfpathlineto{\pgfqpoint{4.150363in}{1.542583in}}%
\pgfpathlineto{\pgfqpoint{4.158156in}{1.543708in}}%
\pgfpathclose%
\pgfusepath{fill}%
\end{pgfscope}%
\begin{pgfscope}%
\pgfpathrectangle{\pgfqpoint{1.254980in}{0.150000in}}{\pgfqpoint{5.490039in}{5.490039in}}%
\pgfusepath{clip}%
\pgfsetbuttcap%
\pgfsetroundjoin%
\definecolor{currentfill}{rgb}{0.231674,0.318106,0.544834}%
\pgfsetfillcolor{currentfill}%
\pgfsetfillopacity{0.700000}%
\pgfsetlinewidth{0.000000pt}%
\definecolor{currentstroke}{rgb}{0.000000,0.000000,0.000000}%
\pgfsetstrokecolor{currentstroke}%
\pgfsetdash{}{0pt}%
\pgfpathmoveto{\pgfqpoint{3.219059in}{2.035644in}}%
\pgfpathlineto{\pgfqpoint{3.232504in}{2.027105in}}%
\pgfpathlineto{\pgfqpoint{3.245951in}{2.018595in}}%
\pgfpathlineto{\pgfqpoint{3.259402in}{2.010114in}}%
\pgfpathlineto{\pgfqpoint{3.272856in}{2.001661in}}%
\pgfpathlineto{\pgfqpoint{3.264414in}{2.012341in}}%
\pgfpathlineto{\pgfqpoint{3.255951in}{2.023563in}}%
\pgfpathlineto{\pgfqpoint{3.247465in}{2.035341in}}%
\pgfpathlineto{\pgfqpoint{3.238958in}{2.047684in}}%
\pgfpathlineto{\pgfqpoint{3.225461in}{2.056528in}}%
\pgfpathlineto{\pgfqpoint{3.211967in}{2.065401in}}%
\pgfpathlineto{\pgfqpoint{3.198477in}{2.074303in}}%
\pgfpathlineto{\pgfqpoint{3.184989in}{2.083234in}}%
\pgfpathlineto{\pgfqpoint{3.193541in}{2.070493in}}%
\pgfpathlineto{\pgfqpoint{3.202069in}{2.058321in}}%
\pgfpathlineto{\pgfqpoint{3.210575in}{2.046709in}}%
\pgfpathlineto{\pgfqpoint{3.219059in}{2.035644in}}%
\pgfpathclose%
\pgfusepath{fill}%
\end{pgfscope}%
\begin{pgfscope}%
\pgfpathrectangle{\pgfqpoint{1.254980in}{0.150000in}}{\pgfqpoint{5.490039in}{5.490039in}}%
\pgfusepath{clip}%
\pgfsetbuttcap%
\pgfsetroundjoin%
\definecolor{currentfill}{rgb}{0.280267,0.073417,0.397163}%
\pgfsetfillcolor{currentfill}%
\pgfsetfillopacity{0.700000}%
\pgfsetlinewidth{0.000000pt}%
\definecolor{currentstroke}{rgb}{0.000000,0.000000,0.000000}%
\pgfsetstrokecolor{currentstroke}%
\pgfsetdash{}{0pt}%
\pgfpathmoveto{\pgfqpoint{5.168930in}{1.552299in}}%
\pgfpathlineto{\pgfqpoint{5.182802in}{1.550165in}}%
\pgfpathlineto{\pgfqpoint{5.196682in}{1.548054in}}%
\pgfpathlineto{\pgfqpoint{5.210570in}{1.545965in}}%
\pgfpathlineto{\pgfqpoint{5.224466in}{1.543900in}}%
\pgfpathlineto{\pgfqpoint{5.216990in}{1.532812in}}%
\pgfpathlineto{\pgfqpoint{5.209510in}{1.521784in}}%
\pgfpathlineto{\pgfqpoint{5.202028in}{1.510821in}}%
\pgfpathlineto{\pgfqpoint{5.194542in}{1.499929in}}%
\pgfpathlineto{\pgfqpoint{5.180641in}{1.502221in}}%
\pgfpathlineto{\pgfqpoint{5.166747in}{1.504535in}}%
\pgfpathlineto{\pgfqpoint{5.152862in}{1.506873in}}%
\pgfpathlineto{\pgfqpoint{5.138985in}{1.509233in}}%
\pgfpathlineto{\pgfqpoint{5.146476in}{1.519894in}}%
\pgfpathlineto{\pgfqpoint{5.153964in}{1.530629in}}%
\pgfpathlineto{\pgfqpoint{5.161449in}{1.541432in}}%
\pgfpathlineto{\pgfqpoint{5.168930in}{1.552299in}}%
\pgfpathclose%
\pgfusepath{fill}%
\end{pgfscope}%
\begin{pgfscope}%
\pgfpathrectangle{\pgfqpoint{1.254980in}{0.150000in}}{\pgfqpoint{5.490039in}{5.490039in}}%
\pgfusepath{clip}%
\pgfsetbuttcap%
\pgfsetroundjoin%
\definecolor{currentfill}{rgb}{0.273809,0.031497,0.358853}%
\pgfsetfillcolor{currentfill}%
\pgfsetfillopacity{0.700000}%
\pgfsetlinewidth{0.000000pt}%
\definecolor{currentstroke}{rgb}{0.000000,0.000000,0.000000}%
\pgfsetstrokecolor{currentstroke}%
\pgfsetdash{}{0pt}%
\pgfpathmoveto{\pgfqpoint{4.857739in}{1.476043in}}%
\pgfpathlineto{\pgfqpoint{4.871509in}{1.472827in}}%
\pgfpathlineto{\pgfqpoint{4.885286in}{1.469634in}}%
\pgfpathlineto{\pgfqpoint{4.899071in}{1.466464in}}%
\pgfpathlineto{\pgfqpoint{4.912863in}{1.463316in}}%
\pgfpathlineto{\pgfqpoint{4.905324in}{1.454369in}}%
\pgfpathlineto{\pgfqpoint{4.897781in}{1.445567in}}%
\pgfpathlineto{\pgfqpoint{4.890235in}{1.436916in}}%
\pgfpathlineto{\pgfqpoint{4.882687in}{1.428424in}}%
\pgfpathlineto{\pgfqpoint{4.868886in}{1.431835in}}%
\pgfpathlineto{\pgfqpoint{4.855092in}{1.435269in}}%
\pgfpathlineto{\pgfqpoint{4.841306in}{1.438726in}}%
\pgfpathlineto{\pgfqpoint{4.827526in}{1.442206in}}%
\pgfpathlineto{\pgfqpoint{4.835084in}{1.450430in}}%
\pgfpathlineto{\pgfqpoint{4.842639in}{1.458815in}}%
\pgfpathlineto{\pgfqpoint{4.850190in}{1.467355in}}%
\pgfpathlineto{\pgfqpoint{4.857739in}{1.476043in}}%
\pgfpathclose%
\pgfusepath{fill}%
\end{pgfscope}%
\begin{pgfscope}%
\pgfpathrectangle{\pgfqpoint{1.254980in}{0.150000in}}{\pgfqpoint{5.490039in}{5.490039in}}%
\pgfusepath{clip}%
\pgfsetbuttcap%
\pgfsetroundjoin%
\definecolor{currentfill}{rgb}{0.266580,0.228262,0.514349}%
\pgfsetfillcolor{currentfill}%
\pgfsetfillopacity{0.700000}%
\pgfsetlinewidth{0.000000pt}%
\definecolor{currentstroke}{rgb}{0.000000,0.000000,0.000000}%
\pgfsetstrokecolor{currentstroke}%
\pgfsetdash{}{0pt}%
\pgfpathmoveto{\pgfqpoint{3.521557in}{1.838713in}}%
\pgfpathlineto{\pgfqpoint{3.535036in}{1.831104in}}%
\pgfpathlineto{\pgfqpoint{3.548519in}{1.823521in}}%
\pgfpathlineto{\pgfqpoint{3.562006in}{1.815965in}}%
\pgfpathlineto{\pgfqpoint{3.575497in}{1.808434in}}%
\pgfpathlineto{\pgfqpoint{3.567318in}{1.815208in}}%
\pgfpathlineto{\pgfqpoint{3.559123in}{1.822462in}}%
\pgfpathlineto{\pgfqpoint{3.550911in}{1.830207in}}%
\pgfpathlineto{\pgfqpoint{3.542684in}{1.838453in}}%
\pgfpathlineto{\pgfqpoint{3.529157in}{1.846357in}}%
\pgfpathlineto{\pgfqpoint{3.515634in}{1.854287in}}%
\pgfpathlineto{\pgfqpoint{3.502115in}{1.862243in}}%
\pgfpathlineto{\pgfqpoint{3.488600in}{1.870226in}}%
\pgfpathlineto{\pgfqpoint{3.496865in}{1.861601in}}%
\pgfpathlineto{\pgfqpoint{3.505113in}{1.853481in}}%
\pgfpathlineto{\pgfqpoint{3.513343in}{1.845855in}}%
\pgfpathlineto{\pgfqpoint{3.521557in}{1.838713in}}%
\pgfpathclose%
\pgfusepath{fill}%
\end{pgfscope}%
\begin{pgfscope}%
\pgfpathrectangle{\pgfqpoint{1.254980in}{0.150000in}}{\pgfqpoint{5.490039in}{5.490039in}}%
\pgfusepath{clip}%
\pgfsetbuttcap%
\pgfsetroundjoin%
\definecolor{currentfill}{rgb}{0.280868,0.160771,0.472899}%
\pgfsetfillcolor{currentfill}%
\pgfsetfillopacity{0.700000}%
\pgfsetlinewidth{0.000000pt}%
\definecolor{currentstroke}{rgb}{0.000000,0.000000,0.000000}%
\pgfsetstrokecolor{currentstroke}%
\pgfsetdash{}{0pt}%
\pgfpathmoveto{\pgfqpoint{3.769895in}{1.701883in}}%
\pgfpathlineto{\pgfqpoint{3.783411in}{1.695044in}}%
\pgfpathlineto{\pgfqpoint{3.796931in}{1.688230in}}%
\pgfpathlineto{\pgfqpoint{3.810455in}{1.681441in}}%
\pgfpathlineto{\pgfqpoint{3.823985in}{1.674677in}}%
\pgfpathlineto{\pgfqpoint{3.815988in}{1.678224in}}%
\pgfpathlineto{\pgfqpoint{3.807980in}{1.682196in}}%
\pgfpathlineto{\pgfqpoint{3.799959in}{1.686603in}}%
\pgfpathlineto{\pgfqpoint{3.791926in}{1.691453in}}%
\pgfpathlineto{\pgfqpoint{3.778366in}{1.698575in}}%
\pgfpathlineto{\pgfqpoint{3.764811in}{1.705721in}}%
\pgfpathlineto{\pgfqpoint{3.751261in}{1.712893in}}%
\pgfpathlineto{\pgfqpoint{3.737714in}{1.720089in}}%
\pgfpathlineto{\pgfqpoint{3.745779in}{1.714876in}}%
\pgfpathlineto{\pgfqpoint{3.753831in}{1.710110in}}%
\pgfpathlineto{\pgfqpoint{3.761869in}{1.705782in}}%
\pgfpathlineto{\pgfqpoint{3.769895in}{1.701883in}}%
\pgfpathclose%
\pgfusepath{fill}%
\end{pgfscope}%
\begin{pgfscope}%
\pgfpathrectangle{\pgfqpoint{1.254980in}{0.150000in}}{\pgfqpoint{5.490039in}{5.490039in}}%
\pgfusepath{clip}%
\pgfsetbuttcap%
\pgfsetroundjoin%
\definecolor{currentfill}{rgb}{0.119699,0.618490,0.536347}%
\pgfsetfillcolor{currentfill}%
\pgfsetfillopacity{0.700000}%
\pgfsetlinewidth{0.000000pt}%
\definecolor{currentstroke}{rgb}{0.000000,0.000000,0.000000}%
\pgfsetstrokecolor{currentstroke}%
\pgfsetdash{}{0pt}%
\pgfpathmoveto{\pgfqpoint{2.288815in}{2.808439in}}%
\pgfpathlineto{\pgfqpoint{2.302238in}{2.796741in}}%
\pgfpathlineto{\pgfqpoint{2.315661in}{2.785091in}}%
\pgfpathlineto{\pgfqpoint{2.329084in}{2.773488in}}%
\pgfpathlineto{\pgfqpoint{2.342509in}{2.761932in}}%
\pgfpathlineto{\pgfqpoint{2.333001in}{2.784105in}}%
\pgfpathlineto{\pgfqpoint{2.323453in}{2.806991in}}%
\pgfpathlineto{\pgfqpoint{2.313862in}{2.830605in}}%
\pgfpathlineto{\pgfqpoint{2.304228in}{2.854958in}}%
\pgfpathlineto{\pgfqpoint{2.290740in}{2.866958in}}%
\pgfpathlineto{\pgfqpoint{2.277253in}{2.879005in}}%
\pgfpathlineto{\pgfqpoint{2.263765in}{2.891099in}}%
\pgfpathlineto{\pgfqpoint{2.250277in}{2.903241in}}%
\pgfpathlineto{\pgfqpoint{2.259977in}{2.878436in}}%
\pgfpathlineto{\pgfqpoint{2.269633in}{2.854376in}}%
\pgfpathlineto{\pgfqpoint{2.279245in}{2.831049in}}%
\pgfpathlineto{\pgfqpoint{2.288815in}{2.808439in}}%
\pgfpathclose%
\pgfusepath{fill}%
\end{pgfscope}%
\begin{pgfscope}%
\pgfpathrectangle{\pgfqpoint{1.254980in}{0.150000in}}{\pgfqpoint{5.490039in}{5.490039in}}%
\pgfusepath{clip}%
\pgfsetbuttcap%
\pgfsetroundjoin%
\definecolor{currentfill}{rgb}{0.180629,0.429975,0.557282}%
\pgfsetfillcolor{currentfill}%
\pgfsetfillopacity{0.700000}%
\pgfsetlinewidth{0.000000pt}%
\definecolor{currentstroke}{rgb}{0.000000,0.000000,0.000000}%
\pgfsetstrokecolor{currentstroke}%
\pgfsetdash{}{0pt}%
\pgfpathmoveto{\pgfqpoint{2.862134in}{2.306774in}}%
\pgfpathlineto{\pgfqpoint{2.875556in}{2.297090in}}%
\pgfpathlineto{\pgfqpoint{2.888982in}{2.287440in}}%
\pgfpathlineto{\pgfqpoint{2.902409in}{2.277823in}}%
\pgfpathlineto{\pgfqpoint{2.915839in}{2.268240in}}%
\pgfpathlineto{\pgfqpoint{2.907026in}{2.283584in}}%
\pgfpathlineto{\pgfqpoint{2.898185in}{2.299544in}}%
\pgfpathlineto{\pgfqpoint{2.889315in}{2.316132in}}%
\pgfpathlineto{\pgfqpoint{2.880415in}{2.333360in}}%
\pgfpathlineto{\pgfqpoint{2.866934in}{2.343356in}}%
\pgfpathlineto{\pgfqpoint{2.853455in}{2.353386in}}%
\pgfpathlineto{\pgfqpoint{2.839979in}{2.363449in}}%
\pgfpathlineto{\pgfqpoint{2.826504in}{2.373546in}}%
\pgfpathlineto{\pgfqpoint{2.835457in}{2.355899in}}%
\pgfpathlineto{\pgfqpoint{2.844379in}{2.338896in}}%
\pgfpathlineto{\pgfqpoint{2.853271in}{2.322525in}}%
\pgfpathlineto{\pgfqpoint{2.862134in}{2.306774in}}%
\pgfpathclose%
\pgfusepath{fill}%
\end{pgfscope}%
\begin{pgfscope}%
\pgfpathrectangle{\pgfqpoint{1.254980in}{0.150000in}}{\pgfqpoint{5.490039in}{5.490039in}}%
\pgfusepath{clip}%
\pgfsetbuttcap%
\pgfsetroundjoin%
\definecolor{currentfill}{rgb}{0.273809,0.031497,0.358853}%
\pgfsetfillcolor{currentfill}%
\pgfsetfillopacity{0.700000}%
\pgfsetlinewidth{0.000000pt}%
\definecolor{currentstroke}{rgb}{0.000000,0.000000,0.000000}%
\pgfsetstrokecolor{currentstroke}%
\pgfsetdash{}{0pt}%
\pgfpathmoveto{\pgfqpoint{4.492332in}{1.468756in}}%
\pgfpathlineto{\pgfqpoint{4.506002in}{1.464254in}}%
\pgfpathlineto{\pgfqpoint{4.519677in}{1.459775in}}%
\pgfpathlineto{\pgfqpoint{4.533359in}{1.455318in}}%
\pgfpathlineto{\pgfqpoint{4.547047in}{1.450885in}}%
\pgfpathlineto{\pgfqpoint{4.539409in}{1.445620in}}%
\pgfpathlineto{\pgfqpoint{4.531765in}{1.440602in}}%
\pgfpathlineto{\pgfqpoint{4.524118in}{1.435838in}}%
\pgfpathlineto{\pgfqpoint{4.516465in}{1.431337in}}%
\pgfpathlineto{\pgfqpoint{4.502762in}{1.436072in}}%
\pgfpathlineto{\pgfqpoint{4.489065in}{1.440830in}}%
\pgfpathlineto{\pgfqpoint{4.475374in}{1.445611in}}%
\pgfpathlineto{\pgfqpoint{4.461689in}{1.450415in}}%
\pgfpathlineto{\pgfqpoint{4.469357in}{1.454610in}}%
\pgfpathlineto{\pgfqpoint{4.477020in}{1.459070in}}%
\pgfpathlineto{\pgfqpoint{4.484679in}{1.463788in}}%
\pgfpathlineto{\pgfqpoint{4.492332in}{1.468756in}}%
\pgfpathclose%
\pgfusepath{fill}%
\end{pgfscope}%
\begin{pgfscope}%
\pgfpathrectangle{\pgfqpoint{1.254980in}{0.150000in}}{\pgfqpoint{5.490039in}{5.490039in}}%
\pgfusepath{clip}%
\pgfsetbuttcap%
\pgfsetroundjoin%
\definecolor{currentfill}{rgb}{0.277941,0.056324,0.381191}%
\pgfsetfillcolor{currentfill}%
\pgfsetfillopacity{0.700000}%
\pgfsetlinewidth{0.000000pt}%
\definecolor{currentstroke}{rgb}{0.000000,0.000000,0.000000}%
\pgfsetstrokecolor{currentstroke}%
\pgfsetdash{}{0pt}%
\pgfpathmoveto{\pgfqpoint{5.083554in}{1.518904in}}%
\pgfpathlineto{\pgfqpoint{5.097400in}{1.516452in}}%
\pgfpathlineto{\pgfqpoint{5.111254in}{1.514023in}}%
\pgfpathlineto{\pgfqpoint{5.125115in}{1.511617in}}%
\pgfpathlineto{\pgfqpoint{5.138985in}{1.509233in}}%
\pgfpathlineto{\pgfqpoint{5.131490in}{1.498653in}}%
\pgfpathlineto{\pgfqpoint{5.123993in}{1.488158in}}%
\pgfpathlineto{\pgfqpoint{5.116493in}{1.477754in}}%
\pgfpathlineto{\pgfqpoint{5.108989in}{1.467449in}}%
\pgfpathlineto{\pgfqpoint{5.095114in}{1.470071in}}%
\pgfpathlineto{\pgfqpoint{5.081246in}{1.472716in}}%
\pgfpathlineto{\pgfqpoint{5.067386in}{1.475384in}}%
\pgfpathlineto{\pgfqpoint{5.053534in}{1.478075in}}%
\pgfpathlineto{\pgfqpoint{5.061044in}{1.488137in}}%
\pgfpathlineto{\pgfqpoint{5.068550in}{1.498300in}}%
\pgfpathlineto{\pgfqpoint{5.076054in}{1.508557in}}%
\pgfpathlineto{\pgfqpoint{5.083554in}{1.518904in}}%
\pgfpathclose%
\pgfusepath{fill}%
\end{pgfscope}%
\begin{pgfscope}%
\pgfpathrectangle{\pgfqpoint{1.254980in}{0.150000in}}{\pgfqpoint{5.490039in}{5.490039in}}%
\pgfusepath{clip}%
\pgfsetbuttcap%
\pgfsetroundjoin%
\definecolor{currentfill}{rgb}{0.272594,0.025563,0.353093}%
\pgfsetfillcolor{currentfill}%
\pgfsetfillopacity{0.700000}%
\pgfsetlinewidth{0.000000pt}%
\definecolor{currentstroke}{rgb}{0.000000,0.000000,0.000000}%
\pgfsetstrokecolor{currentstroke}%
\pgfsetdash{}{0pt}%
\pgfpathmoveto{\pgfqpoint{4.632324in}{1.457912in}}%
\pgfpathlineto{\pgfqpoint{4.646032in}{1.453883in}}%
\pgfpathlineto{\pgfqpoint{4.659746in}{1.449876in}}%
\pgfpathlineto{\pgfqpoint{4.673467in}{1.445892in}}%
\pgfpathlineto{\pgfqpoint{4.687195in}{1.441930in}}%
\pgfpathlineto{\pgfqpoint{4.679599in}{1.435186in}}%
\pgfpathlineto{\pgfqpoint{4.671999in}{1.428651in}}%
\pgfpathlineto{\pgfqpoint{4.664395in}{1.422335in}}%
\pgfpathlineto{\pgfqpoint{4.656788in}{1.416243in}}%
\pgfpathlineto{\pgfqpoint{4.643047in}{1.420493in}}%
\pgfpathlineto{\pgfqpoint{4.629313in}{1.424766in}}%
\pgfpathlineto{\pgfqpoint{4.615586in}{1.429062in}}%
\pgfpathlineto{\pgfqpoint{4.601865in}{1.433381in}}%
\pgfpathlineto{\pgfqpoint{4.609486in}{1.439179in}}%
\pgfpathlineto{\pgfqpoint{4.617103in}{1.445205in}}%
\pgfpathlineto{\pgfqpoint{4.624715in}{1.451452in}}%
\pgfpathlineto{\pgfqpoint{4.632324in}{1.457912in}}%
\pgfpathclose%
\pgfusepath{fill}%
\end{pgfscope}%
\begin{pgfscope}%
\pgfpathrectangle{\pgfqpoint{1.254980in}{0.150000in}}{\pgfqpoint{5.490039in}{5.490039in}}%
\pgfusepath{clip}%
\pgfsetbuttcap%
\pgfsetroundjoin%
\definecolor{currentfill}{rgb}{0.276022,0.044167,0.370164}%
\pgfsetfillcolor{currentfill}%
\pgfsetfillopacity{0.700000}%
\pgfsetlinewidth{0.000000pt}%
\definecolor{currentstroke}{rgb}{0.000000,0.000000,0.000000}%
\pgfsetstrokecolor{currentstroke}%
\pgfsetdash{}{0pt}%
\pgfpathmoveto{\pgfqpoint{4.352430in}{1.489678in}}%
\pgfpathlineto{\pgfqpoint{4.366066in}{1.484689in}}%
\pgfpathlineto{\pgfqpoint{4.379709in}{1.479723in}}%
\pgfpathlineto{\pgfqpoint{4.393357in}{1.474781in}}%
\pgfpathlineto{\pgfqpoint{4.407011in}{1.469862in}}%
\pgfpathlineto{\pgfqpoint{4.399321in}{1.466251in}}%
\pgfpathlineto{\pgfqpoint{4.391626in}{1.462927in}}%
\pgfpathlineto{\pgfqpoint{4.383925in}{1.459895in}}%
\pgfpathlineto{\pgfqpoint{4.376218in}{1.457166in}}%
\pgfpathlineto{\pgfqpoint{4.362546in}{1.462400in}}%
\pgfpathlineto{\pgfqpoint{4.348880in}{1.467658in}}%
\pgfpathlineto{\pgfqpoint{4.335219in}{1.472938in}}%
\pgfpathlineto{\pgfqpoint{4.321565in}{1.478242in}}%
\pgfpathlineto{\pgfqpoint{4.329290in}{1.480651in}}%
\pgfpathlineto{\pgfqpoint{4.337009in}{1.483366in}}%
\pgfpathlineto{\pgfqpoint{4.344723in}{1.486378in}}%
\pgfpathlineto{\pgfqpoint{4.352430in}{1.489678in}}%
\pgfpathclose%
\pgfusepath{fill}%
\end{pgfscope}%
\begin{pgfscope}%
\pgfpathrectangle{\pgfqpoint{1.254980in}{0.150000in}}{\pgfqpoint{5.490039in}{5.490039in}}%
\pgfusepath{clip}%
\pgfsetbuttcap%
\pgfsetroundjoin%
\definecolor{currentfill}{rgb}{0.120092,0.600104,0.542530}%
\pgfsetfillcolor{currentfill}%
\pgfsetfillopacity{0.700000}%
\pgfsetlinewidth{0.000000pt}%
\definecolor{currentstroke}{rgb}{0.000000,0.000000,0.000000}%
\pgfsetstrokecolor{currentstroke}%
\pgfsetdash{}{0pt}%
\pgfpathmoveto{\pgfqpoint{2.342509in}{2.761932in}}%
\pgfpathlineto{\pgfqpoint{2.355933in}{2.750421in}}%
\pgfpathlineto{\pgfqpoint{2.369358in}{2.738956in}}%
\pgfpathlineto{\pgfqpoint{2.382784in}{2.727536in}}%
\pgfpathlineto{\pgfqpoint{2.396211in}{2.716160in}}%
\pgfpathlineto{\pgfqpoint{2.386765in}{2.737899in}}%
\pgfpathlineto{\pgfqpoint{2.377280in}{2.760346in}}%
\pgfpathlineto{\pgfqpoint{2.367753in}{2.783514in}}%
\pgfpathlineto{\pgfqpoint{2.358184in}{2.807418in}}%
\pgfpathlineto{\pgfqpoint{2.344695in}{2.819235in}}%
\pgfpathlineto{\pgfqpoint{2.331205in}{2.831097in}}%
\pgfpathlineto{\pgfqpoint{2.317717in}{2.843005in}}%
\pgfpathlineto{\pgfqpoint{2.304228in}{2.854958in}}%
\pgfpathlineto{\pgfqpoint{2.313862in}{2.830605in}}%
\pgfpathlineto{\pgfqpoint{2.323453in}{2.806991in}}%
\pgfpathlineto{\pgfqpoint{2.333001in}{2.784105in}}%
\pgfpathlineto{\pgfqpoint{2.342509in}{2.761932in}}%
\pgfpathclose%
\pgfusepath{fill}%
\end{pgfscope}%
\begin{pgfscope}%
\pgfpathrectangle{\pgfqpoint{1.254980in}{0.150000in}}{\pgfqpoint{5.490039in}{5.490039in}}%
\pgfusepath{clip}%
\pgfsetbuttcap%
\pgfsetroundjoin%
\definecolor{currentfill}{rgb}{0.282910,0.105393,0.426902}%
\pgfsetfillcolor{currentfill}%
\pgfsetfillopacity{0.700000}%
\pgfsetlinewidth{0.000000pt}%
\definecolor{currentstroke}{rgb}{0.000000,0.000000,0.000000}%
\pgfsetstrokecolor{currentstroke}%
\pgfsetdash{}{0pt}%
\pgfpathmoveto{\pgfqpoint{4.018258in}{1.589400in}}%
\pgfpathlineto{\pgfqpoint{4.031825in}{1.583296in}}%
\pgfpathlineto{\pgfqpoint{4.045397in}{1.577216in}}%
\pgfpathlineto{\pgfqpoint{4.058973in}{1.571161in}}%
\pgfpathlineto{\pgfqpoint{4.072556in}{1.565129in}}%
\pgfpathlineto{\pgfqpoint{4.064707in}{1.565725in}}%
\pgfpathlineto{\pgfqpoint{4.056849in}{1.566693in}}%
\pgfpathlineto{\pgfqpoint{4.048983in}{1.568043in}}%
\pgfpathlineto{\pgfqpoint{4.041107in}{1.569784in}}%
\pgfpathlineto{\pgfqpoint{4.027500in}{1.576158in}}%
\pgfpathlineto{\pgfqpoint{4.013898in}{1.582557in}}%
\pgfpathlineto{\pgfqpoint{4.000301in}{1.588979in}}%
\pgfpathlineto{\pgfqpoint{3.986709in}{1.595425in}}%
\pgfpathlineto{\pgfqpoint{3.994611in}{1.593337in}}%
\pgfpathlineto{\pgfqpoint{4.002503in}{1.591643in}}%
\pgfpathlineto{\pgfqpoint{4.010385in}{1.590334in}}%
\pgfpathlineto{\pgfqpoint{4.018258in}{1.589400in}}%
\pgfpathclose%
\pgfusepath{fill}%
\end{pgfscope}%
\begin{pgfscope}%
\pgfpathrectangle{\pgfqpoint{1.254980in}{0.150000in}}{\pgfqpoint{5.490039in}{5.490039in}}%
\pgfusepath{clip}%
\pgfsetbuttcap%
\pgfsetroundjoin%
\definecolor{currentfill}{rgb}{0.235526,0.309527,0.542944}%
\pgfsetfillcolor{currentfill}%
\pgfsetfillopacity{0.700000}%
\pgfsetlinewidth{0.000000pt}%
\definecolor{currentstroke}{rgb}{0.000000,0.000000,0.000000}%
\pgfsetstrokecolor{currentstroke}%
\pgfsetdash{}{0pt}%
\pgfpathmoveto{\pgfqpoint{3.272856in}{2.001661in}}%
\pgfpathlineto{\pgfqpoint{3.286314in}{1.993237in}}%
\pgfpathlineto{\pgfqpoint{3.299775in}{1.984842in}}%
\pgfpathlineto{\pgfqpoint{3.313239in}{1.976475in}}%
\pgfpathlineto{\pgfqpoint{3.326707in}{1.968136in}}%
\pgfpathlineto{\pgfqpoint{3.318306in}{1.978431in}}%
\pgfpathlineto{\pgfqpoint{3.309884in}{1.989264in}}%
\pgfpathlineto{\pgfqpoint{3.301442in}{2.000648in}}%
\pgfpathlineto{\pgfqpoint{3.292978in}{2.012594in}}%
\pgfpathlineto{\pgfqpoint{3.279468in}{2.021324in}}%
\pgfpathlineto{\pgfqpoint{3.265961in}{2.030082in}}%
\pgfpathlineto{\pgfqpoint{3.252458in}{2.038869in}}%
\pgfpathlineto{\pgfqpoint{3.238958in}{2.047684in}}%
\pgfpathlineto{\pgfqpoint{3.247465in}{2.035341in}}%
\pgfpathlineto{\pgfqpoint{3.255951in}{2.023563in}}%
\pgfpathlineto{\pgfqpoint{3.264414in}{2.012341in}}%
\pgfpathlineto{\pgfqpoint{3.272856in}{2.001661in}}%
\pgfpathclose%
\pgfusepath{fill}%
\end{pgfscope}%
\begin{pgfscope}%
\pgfpathrectangle{\pgfqpoint{1.254980in}{0.150000in}}{\pgfqpoint{5.490039in}{5.490039in}}%
\pgfusepath{clip}%
\pgfsetbuttcap%
\pgfsetroundjoin%
\definecolor{currentfill}{rgb}{0.272594,0.025563,0.353093}%
\pgfsetfillcolor{currentfill}%
\pgfsetfillopacity{0.700000}%
\pgfsetlinewidth{0.000000pt}%
\definecolor{currentstroke}{rgb}{0.000000,0.000000,0.000000}%
\pgfsetstrokecolor{currentstroke}%
\pgfsetdash{}{0pt}%
\pgfpathmoveto{\pgfqpoint{4.772479in}{1.456353in}}%
\pgfpathlineto{\pgfqpoint{4.786230in}{1.452782in}}%
\pgfpathlineto{\pgfqpoint{4.799989in}{1.449234in}}%
\pgfpathlineto{\pgfqpoint{4.813754in}{1.445708in}}%
\pgfpathlineto{\pgfqpoint{4.827526in}{1.442206in}}%
\pgfpathlineto{\pgfqpoint{4.819965in}{1.434150in}}%
\pgfpathlineto{\pgfqpoint{4.812401in}{1.426269in}}%
\pgfpathlineto{\pgfqpoint{4.804834in}{1.418570in}}%
\pgfpathlineto{\pgfqpoint{4.797263in}{1.411061in}}%
\pgfpathlineto{\pgfqpoint{4.783481in}{1.414840in}}%
\pgfpathlineto{\pgfqpoint{4.769705in}{1.418642in}}%
\pgfpathlineto{\pgfqpoint{4.755936in}{1.422466in}}%
\pgfpathlineto{\pgfqpoint{4.742174in}{1.426313in}}%
\pgfpathlineto{\pgfqpoint{4.749755in}{1.433542in}}%
\pgfpathlineto{\pgfqpoint{4.757333in}{1.440962in}}%
\pgfpathlineto{\pgfqpoint{4.764908in}{1.448568in}}%
\pgfpathlineto{\pgfqpoint{4.772479in}{1.456353in}}%
\pgfpathclose%
\pgfusepath{fill}%
\end{pgfscope}%
\begin{pgfscope}%
\pgfpathrectangle{\pgfqpoint{1.254980in}{0.150000in}}{\pgfqpoint{5.490039in}{5.490039in}}%
\pgfusepath{clip}%
\pgfsetbuttcap%
\pgfsetroundjoin%
\definecolor{currentfill}{rgb}{0.276022,0.044167,0.370164}%
\pgfsetfillcolor{currentfill}%
\pgfsetfillopacity{0.700000}%
\pgfsetlinewidth{0.000000pt}%
\definecolor{currentstroke}{rgb}{0.000000,0.000000,0.000000}%
\pgfsetstrokecolor{currentstroke}%
\pgfsetdash{}{0pt}%
\pgfpathmoveto{\pgfqpoint{4.998202in}{1.489067in}}%
\pgfpathlineto{\pgfqpoint{5.012023in}{1.486284in}}%
\pgfpathlineto{\pgfqpoint{5.025853in}{1.483525in}}%
\pgfpathlineto{\pgfqpoint{5.039689in}{1.480789in}}%
\pgfpathlineto{\pgfqpoint{5.053534in}{1.478075in}}%
\pgfpathlineto{\pgfqpoint{5.046021in}{1.468120in}}%
\pgfpathlineto{\pgfqpoint{5.038506in}{1.458278in}}%
\pgfpathlineto{\pgfqpoint{5.030988in}{1.448554in}}%
\pgfpathlineto{\pgfqpoint{5.023467in}{1.438957in}}%
\pgfpathlineto{\pgfqpoint{5.009615in}{1.441922in}}%
\pgfpathlineto{\pgfqpoint{4.995771in}{1.444910in}}%
\pgfpathlineto{\pgfqpoint{4.981934in}{1.447921in}}%
\pgfpathlineto{\pgfqpoint{4.968105in}{1.450954in}}%
\pgfpathlineto{\pgfqpoint{4.975634in}{1.460296in}}%
\pgfpathlineto{\pgfqpoint{4.983159in}{1.469766in}}%
\pgfpathlineto{\pgfqpoint{4.990682in}{1.479358in}}%
\pgfpathlineto{\pgfqpoint{4.998202in}{1.489067in}}%
\pgfpathclose%
\pgfusepath{fill}%
\end{pgfscope}%
\begin{pgfscope}%
\pgfpathrectangle{\pgfqpoint{1.254980in}{0.150000in}}{\pgfqpoint{5.490039in}{5.490039in}}%
\pgfusepath{clip}%
\pgfsetbuttcap%
\pgfsetroundjoin%
\definecolor{currentfill}{rgb}{0.185556,0.418570,0.556753}%
\pgfsetfillcolor{currentfill}%
\pgfsetfillopacity{0.700000}%
\pgfsetlinewidth{0.000000pt}%
\definecolor{currentstroke}{rgb}{0.000000,0.000000,0.000000}%
\pgfsetstrokecolor{currentstroke}%
\pgfsetdash{}{0pt}%
\pgfpathmoveto{\pgfqpoint{2.915839in}{2.268240in}}%
\pgfpathlineto{\pgfqpoint{2.929271in}{2.258689in}}%
\pgfpathlineto{\pgfqpoint{2.942706in}{2.249172in}}%
\pgfpathlineto{\pgfqpoint{2.956143in}{2.239686in}}%
\pgfpathlineto{\pgfqpoint{2.969583in}{2.230234in}}%
\pgfpathlineto{\pgfqpoint{2.960819in}{2.245173in}}%
\pgfpathlineto{\pgfqpoint{2.952028in}{2.260723in}}%
\pgfpathlineto{\pgfqpoint{2.943209in}{2.276896in}}%
\pgfpathlineto{\pgfqpoint{2.934361in}{2.293705in}}%
\pgfpathlineto{\pgfqpoint{2.920871in}{2.303570in}}%
\pgfpathlineto{\pgfqpoint{2.907383in}{2.313467in}}%
\pgfpathlineto{\pgfqpoint{2.893898in}{2.323397in}}%
\pgfpathlineto{\pgfqpoint{2.880415in}{2.333360in}}%
\pgfpathlineto{\pgfqpoint{2.889315in}{2.316132in}}%
\pgfpathlineto{\pgfqpoint{2.898185in}{2.299544in}}%
\pgfpathlineto{\pgfqpoint{2.907026in}{2.283584in}}%
\pgfpathlineto{\pgfqpoint{2.915839in}{2.268240in}}%
\pgfpathclose%
\pgfusepath{fill}%
\end{pgfscope}%
\begin{pgfscope}%
\pgfpathrectangle{\pgfqpoint{1.254980in}{0.150000in}}{\pgfqpoint{5.490039in}{5.490039in}}%
\pgfusepath{clip}%
\pgfsetbuttcap%
\pgfsetroundjoin%
\definecolor{currentfill}{rgb}{0.269308,0.218818,0.509577}%
\pgfsetfillcolor{currentfill}%
\pgfsetfillopacity{0.700000}%
\pgfsetlinewidth{0.000000pt}%
\definecolor{currentstroke}{rgb}{0.000000,0.000000,0.000000}%
\pgfsetstrokecolor{currentstroke}%
\pgfsetdash{}{0pt}%
\pgfpathmoveto{\pgfqpoint{3.575497in}{1.808434in}}%
\pgfpathlineto{\pgfqpoint{3.588992in}{1.800930in}}%
\pgfpathlineto{\pgfqpoint{3.602491in}{1.793452in}}%
\pgfpathlineto{\pgfqpoint{3.615994in}{1.786001in}}%
\pgfpathlineto{\pgfqpoint{3.629502in}{1.778575in}}%
\pgfpathlineto{\pgfqpoint{3.621357in}{1.784981in}}%
\pgfpathlineto{\pgfqpoint{3.613197in}{1.791864in}}%
\pgfpathlineto{\pgfqpoint{3.605021in}{1.799233in}}%
\pgfpathlineto{\pgfqpoint{3.596830in}{1.807100in}}%
\pgfpathlineto{\pgfqpoint{3.583287in}{1.814899in}}%
\pgfpathlineto{\pgfqpoint{3.569749in}{1.822724in}}%
\pgfpathlineto{\pgfqpoint{3.556214in}{1.830576in}}%
\pgfpathlineto{\pgfqpoint{3.542684in}{1.838453in}}%
\pgfpathlineto{\pgfqpoint{3.550911in}{1.830207in}}%
\pgfpathlineto{\pgfqpoint{3.559123in}{1.822462in}}%
\pgfpathlineto{\pgfqpoint{3.567318in}{1.815208in}}%
\pgfpathlineto{\pgfqpoint{3.575497in}{1.808434in}}%
\pgfpathclose%
\pgfusepath{fill}%
\end{pgfscope}%
\begin{pgfscope}%
\pgfpathrectangle{\pgfqpoint{1.254980in}{0.150000in}}{\pgfqpoint{5.490039in}{5.490039in}}%
\pgfusepath{clip}%
\pgfsetbuttcap%
\pgfsetroundjoin%
\definecolor{currentfill}{rgb}{0.122606,0.585371,0.546557}%
\pgfsetfillcolor{currentfill}%
\pgfsetfillopacity{0.700000}%
\pgfsetlinewidth{0.000000pt}%
\definecolor{currentstroke}{rgb}{0.000000,0.000000,0.000000}%
\pgfsetstrokecolor{currentstroke}%
\pgfsetdash{}{0pt}%
\pgfpathmoveto{\pgfqpoint{2.396211in}{2.716160in}}%
\pgfpathlineto{\pgfqpoint{2.409638in}{2.704829in}}%
\pgfpathlineto{\pgfqpoint{2.423066in}{2.693542in}}%
\pgfpathlineto{\pgfqpoint{2.436495in}{2.682299in}}%
\pgfpathlineto{\pgfqpoint{2.449925in}{2.671098in}}%
\pgfpathlineto{\pgfqpoint{2.440540in}{2.692403in}}%
\pgfpathlineto{\pgfqpoint{2.431117in}{2.714411in}}%
\pgfpathlineto{\pgfqpoint{2.421653in}{2.737137in}}%
\pgfpathlineto{\pgfqpoint{2.412148in}{2.760592in}}%
\pgfpathlineto{\pgfqpoint{2.398656in}{2.772233in}}%
\pgfpathlineto{\pgfqpoint{2.385165in}{2.783917in}}%
\pgfpathlineto{\pgfqpoint{2.371674in}{2.795646in}}%
\pgfpathlineto{\pgfqpoint{2.358184in}{2.807418in}}%
\pgfpathlineto{\pgfqpoint{2.367753in}{2.783514in}}%
\pgfpathlineto{\pgfqpoint{2.377280in}{2.760346in}}%
\pgfpathlineto{\pgfqpoint{2.386765in}{2.737899in}}%
\pgfpathlineto{\pgfqpoint{2.396211in}{2.716160in}}%
\pgfpathclose%
\pgfusepath{fill}%
\end{pgfscope}%
\begin{pgfscope}%
\pgfpathrectangle{\pgfqpoint{1.254980in}{0.150000in}}{\pgfqpoint{5.490039in}{5.490039in}}%
\pgfusepath{clip}%
\pgfsetbuttcap%
\pgfsetroundjoin%
\definecolor{currentfill}{rgb}{0.280267,0.073417,0.397163}%
\pgfsetfillcolor{currentfill}%
\pgfsetfillopacity{0.700000}%
\pgfsetlinewidth{0.000000pt}%
\definecolor{currentstroke}{rgb}{0.000000,0.000000,0.000000}%
\pgfsetstrokecolor{currentstroke}%
\pgfsetdash{}{0pt}%
\pgfpathmoveto{\pgfqpoint{4.212535in}{1.521511in}}%
\pgfpathlineto{\pgfqpoint{4.226144in}{1.516020in}}%
\pgfpathlineto{\pgfqpoint{4.239758in}{1.510553in}}%
\pgfpathlineto{\pgfqpoint{4.253378in}{1.505110in}}%
\pgfpathlineto{\pgfqpoint{4.267004in}{1.499690in}}%
\pgfpathlineto{\pgfqpoint{4.259253in}{1.497918in}}%
\pgfpathlineto{\pgfqpoint{4.251495in}{1.496472in}}%
\pgfpathlineto{\pgfqpoint{4.243730in}{1.495361in}}%
\pgfpathlineto{\pgfqpoint{4.235958in}{1.494593in}}%
\pgfpathlineto{\pgfqpoint{4.222311in}{1.500342in}}%
\pgfpathlineto{\pgfqpoint{4.208670in}{1.506114in}}%
\pgfpathlineto{\pgfqpoint{4.195034in}{1.511909in}}%
\pgfpathlineto{\pgfqpoint{4.181404in}{1.517728in}}%
\pgfpathlineto{\pgfqpoint{4.189198in}{1.518162in}}%
\pgfpathlineto{\pgfqpoint{4.196984in}{1.518943in}}%
\pgfpathlineto{\pgfqpoint{4.204763in}{1.520062in}}%
\pgfpathlineto{\pgfqpoint{4.212535in}{1.521511in}}%
\pgfpathclose%
\pgfusepath{fill}%
\end{pgfscope}%
\begin{pgfscope}%
\pgfpathrectangle{\pgfqpoint{1.254980in}{0.150000in}}{\pgfqpoint{5.490039in}{5.490039in}}%
\pgfusepath{clip}%
\pgfsetbuttcap%
\pgfsetroundjoin%
\definecolor{currentfill}{rgb}{0.281412,0.155834,0.469201}%
\pgfsetfillcolor{currentfill}%
\pgfsetfillopacity{0.700000}%
\pgfsetlinewidth{0.000000pt}%
\definecolor{currentstroke}{rgb}{0.000000,0.000000,0.000000}%
\pgfsetstrokecolor{currentstroke}%
\pgfsetdash{}{0pt}%
\pgfpathmoveto{\pgfqpoint{3.823985in}{1.674677in}}%
\pgfpathlineto{\pgfqpoint{3.837519in}{1.667938in}}%
\pgfpathlineto{\pgfqpoint{3.851057in}{1.661223in}}%
\pgfpathlineto{\pgfqpoint{3.864601in}{1.654533in}}%
\pgfpathlineto{\pgfqpoint{3.878149in}{1.647868in}}%
\pgfpathlineto{\pgfqpoint{3.870181in}{1.651064in}}%
\pgfpathlineto{\pgfqpoint{3.862202in}{1.654680in}}%
\pgfpathlineto{\pgfqpoint{3.854212in}{1.658727in}}%
\pgfpathlineto{\pgfqpoint{3.846209in}{1.663215in}}%
\pgfpathlineto{\pgfqpoint{3.832631in}{1.670238in}}%
\pgfpathlineto{\pgfqpoint{3.819058in}{1.677285in}}%
\pgfpathlineto{\pgfqpoint{3.805490in}{1.684357in}}%
\pgfpathlineto{\pgfqpoint{3.791926in}{1.691453in}}%
\pgfpathlineto{\pgfqpoint{3.799959in}{1.686603in}}%
\pgfpathlineto{\pgfqpoint{3.807980in}{1.682196in}}%
\pgfpathlineto{\pgfqpoint{3.815988in}{1.678224in}}%
\pgfpathlineto{\pgfqpoint{3.823985in}{1.674677in}}%
\pgfpathclose%
\pgfusepath{fill}%
\end{pgfscope}%
\begin{pgfscope}%
\pgfpathrectangle{\pgfqpoint{1.254980in}{0.150000in}}{\pgfqpoint{5.490039in}{5.490039in}}%
\pgfusepath{clip}%
\pgfsetbuttcap%
\pgfsetroundjoin%
\definecolor{currentfill}{rgb}{0.280894,0.078907,0.402329}%
\pgfsetfillcolor{currentfill}%
\pgfsetfillopacity{0.700000}%
\pgfsetlinewidth{0.000000pt}%
\definecolor{currentstroke}{rgb}{0.000000,0.000000,0.000000}%
\pgfsetstrokecolor{currentstroke}%
\pgfsetdash{}{0pt}%
\pgfpathmoveto{\pgfqpoint{5.224466in}{1.543900in}}%
\pgfpathlineto{\pgfqpoint{5.238370in}{1.541858in}}%
\pgfpathlineto{\pgfqpoint{5.252282in}{1.539839in}}%
\pgfpathlineto{\pgfqpoint{5.266203in}{1.537842in}}%
\pgfpathlineto{\pgfqpoint{5.258730in}{1.526588in}}%
\pgfpathlineto{\pgfqpoint{5.251255in}{1.515391in}}%
\pgfpathlineto{\pgfqpoint{5.243776in}{1.504257in}}%
\pgfpathlineto{\pgfqpoint{5.236294in}{1.493191in}}%
\pgfpathlineto{\pgfqpoint{5.222368in}{1.495414in}}%
\pgfpathlineto{\pgfqpoint{5.208451in}{1.497660in}}%
\pgfpathlineto{\pgfqpoint{5.194542in}{1.499929in}}%
\pgfpathlineto{\pgfqpoint{5.202028in}{1.510821in}}%
\pgfpathlineto{\pgfqpoint{5.209510in}{1.521784in}}%
\pgfpathlineto{\pgfqpoint{5.216990in}{1.532812in}}%
\pgfpathlineto{\pgfqpoint{5.224466in}{1.543900in}}%
\pgfpathclose%
\pgfusepath{fill}%
\end{pgfscope}%
\begin{pgfscope}%
\pgfpathrectangle{\pgfqpoint{1.254980in}{0.150000in}}{\pgfqpoint{5.490039in}{5.490039in}}%
\pgfusepath{clip}%
\pgfsetbuttcap%
\pgfsetroundjoin%
\definecolor{currentfill}{rgb}{0.273809,0.031497,0.358853}%
\pgfsetfillcolor{currentfill}%
\pgfsetfillopacity{0.700000}%
\pgfsetlinewidth{0.000000pt}%
\definecolor{currentstroke}{rgb}{0.000000,0.000000,0.000000}%
\pgfsetstrokecolor{currentstroke}%
\pgfsetdash{}{0pt}%
\pgfpathmoveto{\pgfqpoint{4.912863in}{1.463316in}}%
\pgfpathlineto{\pgfqpoint{4.926663in}{1.460192in}}%
\pgfpathlineto{\pgfqpoint{4.940470in}{1.457090in}}%
\pgfpathlineto{\pgfqpoint{4.954284in}{1.454011in}}%
\pgfpathlineto{\pgfqpoint{4.968105in}{1.450954in}}%
\pgfpathlineto{\pgfqpoint{4.960574in}{1.441748in}}%
\pgfpathlineto{\pgfqpoint{4.953040in}{1.432683in}}%
\pgfpathlineto{\pgfqpoint{4.945503in}{1.423767in}}%
\pgfpathlineto{\pgfqpoint{4.937963in}{1.415005in}}%
\pgfpathlineto{\pgfqpoint{4.924133in}{1.418326in}}%
\pgfpathlineto{\pgfqpoint{4.910310in}{1.421669in}}%
\pgfpathlineto{\pgfqpoint{4.896495in}{1.425035in}}%
\pgfpathlineto{\pgfqpoint{4.882687in}{1.428424in}}%
\pgfpathlineto{\pgfqpoint{4.890235in}{1.436916in}}%
\pgfpathlineto{\pgfqpoint{4.897781in}{1.445567in}}%
\pgfpathlineto{\pgfqpoint{4.905324in}{1.454369in}}%
\pgfpathlineto{\pgfqpoint{4.912863in}{1.463316in}}%
\pgfpathclose%
\pgfusepath{fill}%
\end{pgfscope}%
\begin{pgfscope}%
\pgfpathrectangle{\pgfqpoint{1.254980in}{0.150000in}}{\pgfqpoint{5.490039in}{5.490039in}}%
\pgfusepath{clip}%
\pgfsetbuttcap%
\pgfsetroundjoin%
\definecolor{currentfill}{rgb}{0.125394,0.574318,0.549086}%
\pgfsetfillcolor{currentfill}%
\pgfsetfillopacity{0.700000}%
\pgfsetlinewidth{0.000000pt}%
\definecolor{currentstroke}{rgb}{0.000000,0.000000,0.000000}%
\pgfsetstrokecolor{currentstroke}%
\pgfsetdash{}{0pt}%
\pgfpathmoveto{\pgfqpoint{2.449925in}{2.671098in}}%
\pgfpathlineto{\pgfqpoint{2.463356in}{2.659940in}}%
\pgfpathlineto{\pgfqpoint{2.476788in}{2.648824in}}%
\pgfpathlineto{\pgfqpoint{2.490221in}{2.637750in}}%
\pgfpathlineto{\pgfqpoint{2.503655in}{2.626718in}}%
\pgfpathlineto{\pgfqpoint{2.494330in}{2.647591in}}%
\pgfpathlineto{\pgfqpoint{2.484967in}{2.669163in}}%
\pgfpathlineto{\pgfqpoint{2.475565in}{2.691446in}}%
\pgfpathlineto{\pgfqpoint{2.466124in}{2.714455in}}%
\pgfpathlineto{\pgfqpoint{2.452629in}{2.725926in}}%
\pgfpathlineto{\pgfqpoint{2.439134in}{2.737439in}}%
\pgfpathlineto{\pgfqpoint{2.425641in}{2.748994in}}%
\pgfpathlineto{\pgfqpoint{2.412148in}{2.760592in}}%
\pgfpathlineto{\pgfqpoint{2.421653in}{2.737137in}}%
\pgfpathlineto{\pgfqpoint{2.431117in}{2.714411in}}%
\pgfpathlineto{\pgfqpoint{2.440540in}{2.692403in}}%
\pgfpathlineto{\pgfqpoint{2.449925in}{2.671098in}}%
\pgfpathclose%
\pgfusepath{fill}%
\end{pgfscope}%
\begin{pgfscope}%
\pgfpathrectangle{\pgfqpoint{1.254980in}{0.150000in}}{\pgfqpoint{5.490039in}{5.490039in}}%
\pgfusepath{clip}%
\pgfsetbuttcap%
\pgfsetroundjoin%
\definecolor{currentfill}{rgb}{0.273809,0.031497,0.358853}%
\pgfsetfillcolor{currentfill}%
\pgfsetfillopacity{0.700000}%
\pgfsetlinewidth{0.000000pt}%
\definecolor{currentstroke}{rgb}{0.000000,0.000000,0.000000}%
\pgfsetstrokecolor{currentstroke}%
\pgfsetdash{}{0pt}%
\pgfpathmoveto{\pgfqpoint{4.547047in}{1.450885in}}%
\pgfpathlineto{\pgfqpoint{4.560742in}{1.446475in}}%
\pgfpathlineto{\pgfqpoint{4.574443in}{1.442087in}}%
\pgfpathlineto{\pgfqpoint{4.588151in}{1.437723in}}%
\pgfpathlineto{\pgfqpoint{4.601865in}{1.433381in}}%
\pgfpathlineto{\pgfqpoint{4.594241in}{1.427819in}}%
\pgfpathlineto{\pgfqpoint{4.586612in}{1.422500in}}%
\pgfpathlineto{\pgfqpoint{4.578979in}{1.417433in}}%
\pgfpathlineto{\pgfqpoint{4.571342in}{1.412625in}}%
\pgfpathlineto{\pgfqpoint{4.557613in}{1.417268in}}%
\pgfpathlineto{\pgfqpoint{4.543891in}{1.421935in}}%
\pgfpathlineto{\pgfqpoint{4.530175in}{1.426624in}}%
\pgfpathlineto{\pgfqpoint{4.516465in}{1.431337in}}%
\pgfpathlineto{\pgfqpoint{4.524118in}{1.435838in}}%
\pgfpathlineto{\pgfqpoint{4.531765in}{1.440602in}}%
\pgfpathlineto{\pgfqpoint{4.539409in}{1.445620in}}%
\pgfpathlineto{\pgfqpoint{4.547047in}{1.450885in}}%
\pgfpathclose%
\pgfusepath{fill}%
\end{pgfscope}%
\begin{pgfscope}%
\pgfpathrectangle{\pgfqpoint{1.254980in}{0.150000in}}{\pgfqpoint{5.490039in}{5.490039in}}%
\pgfusepath{clip}%
\pgfsetbuttcap%
\pgfsetroundjoin%
\definecolor{currentfill}{rgb}{0.241237,0.296485,0.539709}%
\pgfsetfillcolor{currentfill}%
\pgfsetfillopacity{0.700000}%
\pgfsetlinewidth{0.000000pt}%
\definecolor{currentstroke}{rgb}{0.000000,0.000000,0.000000}%
\pgfsetstrokecolor{currentstroke}%
\pgfsetdash{}{0pt}%
\pgfpathmoveto{\pgfqpoint{3.326707in}{1.968136in}}%
\pgfpathlineto{\pgfqpoint{3.340178in}{1.959826in}}%
\pgfpathlineto{\pgfqpoint{3.353653in}{1.951543in}}%
\pgfpathlineto{\pgfqpoint{3.367131in}{1.943288in}}%
\pgfpathlineto{\pgfqpoint{3.380613in}{1.935061in}}%
\pgfpathlineto{\pgfqpoint{3.372252in}{1.944971in}}%
\pgfpathlineto{\pgfqpoint{3.363872in}{1.955415in}}%
\pgfpathlineto{\pgfqpoint{3.355471in}{1.966407in}}%
\pgfpathlineto{\pgfqpoint{3.347050in}{1.977956in}}%
\pgfpathlineto{\pgfqpoint{3.333527in}{1.986574in}}%
\pgfpathlineto{\pgfqpoint{3.320007in}{1.995219in}}%
\pgfpathlineto{\pgfqpoint{3.306491in}{2.003893in}}%
\pgfpathlineto{\pgfqpoint{3.292978in}{2.012594in}}%
\pgfpathlineto{\pgfqpoint{3.301442in}{2.000648in}}%
\pgfpathlineto{\pgfqpoint{3.309884in}{1.989264in}}%
\pgfpathlineto{\pgfqpoint{3.318306in}{1.978431in}}%
\pgfpathlineto{\pgfqpoint{3.326707in}{1.968136in}}%
\pgfpathclose%
\pgfusepath{fill}%
\end{pgfscope}%
\begin{pgfscope}%
\pgfpathrectangle{\pgfqpoint{1.254980in}{0.150000in}}{\pgfqpoint{5.490039in}{5.490039in}}%
\pgfusepath{clip}%
\pgfsetbuttcap%
\pgfsetroundjoin%
\definecolor{currentfill}{rgb}{0.190631,0.407061,0.556089}%
\pgfsetfillcolor{currentfill}%
\pgfsetfillopacity{0.700000}%
\pgfsetlinewidth{0.000000pt}%
\definecolor{currentstroke}{rgb}{0.000000,0.000000,0.000000}%
\pgfsetstrokecolor{currentstroke}%
\pgfsetdash{}{0pt}%
\pgfpathmoveto{\pgfqpoint{2.969583in}{2.230234in}}%
\pgfpathlineto{\pgfqpoint{2.983025in}{2.220813in}}%
\pgfpathlineto{\pgfqpoint{2.996470in}{2.211424in}}%
\pgfpathlineto{\pgfqpoint{3.009918in}{2.202068in}}%
\pgfpathlineto{\pgfqpoint{3.023368in}{2.192742in}}%
\pgfpathlineto{\pgfqpoint{3.014653in}{2.207276in}}%
\pgfpathlineto{\pgfqpoint{3.005911in}{2.222417in}}%
\pgfpathlineto{\pgfqpoint{2.997142in}{2.238177in}}%
\pgfpathlineto{\pgfqpoint{2.988345in}{2.254568in}}%
\pgfpathlineto{\pgfqpoint{2.974845in}{2.264304in}}%
\pgfpathlineto{\pgfqpoint{2.961348in}{2.274072in}}%
\pgfpathlineto{\pgfqpoint{2.947853in}{2.283873in}}%
\pgfpathlineto{\pgfqpoint{2.934361in}{2.293705in}}%
\pgfpathlineto{\pgfqpoint{2.943209in}{2.276896in}}%
\pgfpathlineto{\pgfqpoint{2.952028in}{2.260723in}}%
\pgfpathlineto{\pgfqpoint{2.960819in}{2.245173in}}%
\pgfpathlineto{\pgfqpoint{2.969583in}{2.230234in}}%
\pgfpathclose%
\pgfusepath{fill}%
\end{pgfscope}%
\begin{pgfscope}%
\pgfpathrectangle{\pgfqpoint{1.254980in}{0.150000in}}{\pgfqpoint{5.490039in}{5.490039in}}%
\pgfusepath{clip}%
\pgfsetbuttcap%
\pgfsetroundjoin%
\definecolor{currentfill}{rgb}{0.276022,0.044167,0.370164}%
\pgfsetfillcolor{currentfill}%
\pgfsetfillopacity{0.700000}%
\pgfsetlinewidth{0.000000pt}%
\definecolor{currentstroke}{rgb}{0.000000,0.000000,0.000000}%
\pgfsetstrokecolor{currentstroke}%
\pgfsetdash{}{0pt}%
\pgfpathmoveto{\pgfqpoint{4.407011in}{1.469862in}}%
\pgfpathlineto{\pgfqpoint{4.420671in}{1.464966in}}%
\pgfpathlineto{\pgfqpoint{4.434338in}{1.460092in}}%
\pgfpathlineto{\pgfqpoint{4.448010in}{1.455242in}}%
\pgfpathlineto{\pgfqpoint{4.461689in}{1.450415in}}%
\pgfpathlineto{\pgfqpoint{4.454016in}{1.446495in}}%
\pgfpathlineto{\pgfqpoint{4.446338in}{1.442857in}}%
\pgfpathlineto{\pgfqpoint{4.438654in}{1.439509in}}%
\pgfpathlineto{\pgfqpoint{4.430966in}{1.436459in}}%
\pgfpathlineto{\pgfqpoint{4.417270in}{1.441601in}}%
\pgfpathlineto{\pgfqpoint{4.403580in}{1.446767in}}%
\pgfpathlineto{\pgfqpoint{4.389896in}{1.451955in}}%
\pgfpathlineto{\pgfqpoint{4.376218in}{1.457166in}}%
\pgfpathlineto{\pgfqpoint{4.383925in}{1.459895in}}%
\pgfpathlineto{\pgfqpoint{4.391626in}{1.462927in}}%
\pgfpathlineto{\pgfqpoint{4.399321in}{1.466251in}}%
\pgfpathlineto{\pgfqpoint{4.407011in}{1.469862in}}%
\pgfpathclose%
\pgfusepath{fill}%
\end{pgfscope}%
\begin{pgfscope}%
\pgfpathrectangle{\pgfqpoint{1.254980in}{0.150000in}}{\pgfqpoint{5.490039in}{5.490039in}}%
\pgfusepath{clip}%
\pgfsetbuttcap%
\pgfsetroundjoin%
\definecolor{currentfill}{rgb}{0.272594,0.025563,0.353093}%
\pgfsetfillcolor{currentfill}%
\pgfsetfillopacity{0.700000}%
\pgfsetlinewidth{0.000000pt}%
\definecolor{currentstroke}{rgb}{0.000000,0.000000,0.000000}%
\pgfsetstrokecolor{currentstroke}%
\pgfsetdash{}{0pt}%
\pgfpathmoveto{\pgfqpoint{4.687195in}{1.441930in}}%
\pgfpathlineto{\pgfqpoint{4.700930in}{1.437992in}}%
\pgfpathlineto{\pgfqpoint{4.714671in}{1.434076in}}%
\pgfpathlineto{\pgfqpoint{4.728419in}{1.430183in}}%
\pgfpathlineto{\pgfqpoint{4.742174in}{1.426313in}}%
\pgfpathlineto{\pgfqpoint{4.734590in}{1.419285in}}%
\pgfpathlineto{\pgfqpoint{4.727002in}{1.412463in}}%
\pgfpathlineto{\pgfqpoint{4.719410in}{1.405855in}}%
\pgfpathlineto{\pgfqpoint{4.711816in}{1.399468in}}%
\pgfpathlineto{\pgfqpoint{4.698049in}{1.403628in}}%
\pgfpathlineto{\pgfqpoint{4.684288in}{1.407810in}}%
\pgfpathlineto{\pgfqpoint{4.670535in}{1.412015in}}%
\pgfpathlineto{\pgfqpoint{4.656788in}{1.416243in}}%
\pgfpathlineto{\pgfqpoint{4.664395in}{1.422335in}}%
\pgfpathlineto{\pgfqpoint{4.671999in}{1.428651in}}%
\pgfpathlineto{\pgfqpoint{4.679599in}{1.435186in}}%
\pgfpathlineto{\pgfqpoint{4.687195in}{1.441930in}}%
\pgfpathclose%
\pgfusepath{fill}%
\end{pgfscope}%
\begin{pgfscope}%
\pgfpathrectangle{\pgfqpoint{1.254980in}{0.150000in}}{\pgfqpoint{5.490039in}{5.490039in}}%
\pgfusepath{clip}%
\pgfsetbuttcap%
\pgfsetroundjoin%
\definecolor{currentfill}{rgb}{0.282656,0.100196,0.422160}%
\pgfsetfillcolor{currentfill}%
\pgfsetfillopacity{0.700000}%
\pgfsetlinewidth{0.000000pt}%
\definecolor{currentstroke}{rgb}{0.000000,0.000000,0.000000}%
\pgfsetstrokecolor{currentstroke}%
\pgfsetdash{}{0pt}%
\pgfpathmoveto{\pgfqpoint{4.072556in}{1.565129in}}%
\pgfpathlineto{\pgfqpoint{4.086143in}{1.559120in}}%
\pgfpathlineto{\pgfqpoint{4.099736in}{1.553136in}}%
\pgfpathlineto{\pgfqpoint{4.113334in}{1.547176in}}%
\pgfpathlineto{\pgfqpoint{4.126937in}{1.541239in}}%
\pgfpathlineto{\pgfqpoint{4.119112in}{1.541498in}}%
\pgfpathlineto{\pgfqpoint{4.111279in}{1.542126in}}%
\pgfpathlineto{\pgfqpoint{4.103437in}{1.543131in}}%
\pgfpathlineto{\pgfqpoint{4.095587in}{1.544524in}}%
\pgfpathlineto{\pgfqpoint{4.081959in}{1.550804in}}%
\pgfpathlineto{\pgfqpoint{4.068337in}{1.557106in}}%
\pgfpathlineto{\pgfqpoint{4.054719in}{1.563433in}}%
\pgfpathlineto{\pgfqpoint{4.041107in}{1.569784in}}%
\pgfpathlineto{\pgfqpoint{4.048983in}{1.568043in}}%
\pgfpathlineto{\pgfqpoint{4.056849in}{1.566693in}}%
\pgfpathlineto{\pgfqpoint{4.064707in}{1.565725in}}%
\pgfpathlineto{\pgfqpoint{4.072556in}{1.565129in}}%
\pgfpathclose%
\pgfusepath{fill}%
\end{pgfscope}%
\begin{pgfscope}%
\pgfpathrectangle{\pgfqpoint{1.254980in}{0.150000in}}{\pgfqpoint{5.490039in}{5.490039in}}%
\pgfusepath{clip}%
\pgfsetbuttcap%
\pgfsetroundjoin%
\definecolor{currentfill}{rgb}{0.271828,0.209303,0.504434}%
\pgfsetfillcolor{currentfill}%
\pgfsetfillopacity{0.700000}%
\pgfsetlinewidth{0.000000pt}%
\definecolor{currentstroke}{rgb}{0.000000,0.000000,0.000000}%
\pgfsetstrokecolor{currentstroke}%
\pgfsetdash{}{0pt}%
\pgfpathmoveto{\pgfqpoint{3.629502in}{1.778575in}}%
\pgfpathlineto{\pgfqpoint{3.643013in}{1.771174in}}%
\pgfpathlineto{\pgfqpoint{3.656529in}{1.763800in}}%
\pgfpathlineto{\pgfqpoint{3.670049in}{1.756451in}}%
\pgfpathlineto{\pgfqpoint{3.683574in}{1.749128in}}%
\pgfpathlineto{\pgfqpoint{3.675462in}{1.755167in}}%
\pgfpathlineto{\pgfqpoint{3.667337in}{1.761679in}}%
\pgfpathlineto{\pgfqpoint{3.659196in}{1.768673in}}%
\pgfpathlineto{\pgfqpoint{3.651040in}{1.776162in}}%
\pgfpathlineto{\pgfqpoint{3.637482in}{1.783858in}}%
\pgfpathlineto{\pgfqpoint{3.623927in}{1.791579in}}%
\pgfpathlineto{\pgfqpoint{3.610376in}{1.799327in}}%
\pgfpathlineto{\pgfqpoint{3.596830in}{1.807100in}}%
\pgfpathlineto{\pgfqpoint{3.605021in}{1.799233in}}%
\pgfpathlineto{\pgfqpoint{3.613197in}{1.791864in}}%
\pgfpathlineto{\pgfqpoint{3.621357in}{1.784981in}}%
\pgfpathlineto{\pgfqpoint{3.629502in}{1.778575in}}%
\pgfpathclose%
\pgfusepath{fill}%
\end{pgfscope}%
\begin{pgfscope}%
\pgfpathrectangle{\pgfqpoint{1.254980in}{0.150000in}}{\pgfqpoint{5.490039in}{5.490039in}}%
\pgfusepath{clip}%
\pgfsetbuttcap%
\pgfsetroundjoin%
\definecolor{currentfill}{rgb}{0.278791,0.062145,0.386592}%
\pgfsetfillcolor{currentfill}%
\pgfsetfillopacity{0.700000}%
\pgfsetlinewidth{0.000000pt}%
\definecolor{currentstroke}{rgb}{0.000000,0.000000,0.000000}%
\pgfsetstrokecolor{currentstroke}%
\pgfsetdash{}{0pt}%
\pgfpathmoveto{\pgfqpoint{5.138985in}{1.509233in}}%
\pgfpathlineto{\pgfqpoint{5.152862in}{1.506873in}}%
\pgfpathlineto{\pgfqpoint{5.166747in}{1.504535in}}%
\pgfpathlineto{\pgfqpoint{5.180641in}{1.502221in}}%
\pgfpathlineto{\pgfqpoint{5.194542in}{1.499929in}}%
\pgfpathlineto{\pgfqpoint{5.187053in}{1.489114in}}%
\pgfpathlineto{\pgfqpoint{5.179561in}{1.478381in}}%
\pgfpathlineto{\pgfqpoint{5.172067in}{1.467737in}}%
\pgfpathlineto{\pgfqpoint{5.164569in}{1.457187in}}%
\pgfpathlineto{\pgfqpoint{5.150663in}{1.459718in}}%
\pgfpathlineto{\pgfqpoint{5.136764in}{1.462272in}}%
\pgfpathlineto{\pgfqpoint{5.122873in}{1.464849in}}%
\pgfpathlineto{\pgfqpoint{5.108989in}{1.467449in}}%
\pgfpathlineto{\pgfqpoint{5.116493in}{1.477754in}}%
\pgfpathlineto{\pgfqpoint{5.123993in}{1.488158in}}%
\pgfpathlineto{\pgfqpoint{5.131490in}{1.498653in}}%
\pgfpathlineto{\pgfqpoint{5.138985in}{1.509233in}}%
\pgfpathclose%
\pgfusepath{fill}%
\end{pgfscope}%
\begin{pgfscope}%
\pgfpathrectangle{\pgfqpoint{1.254980in}{0.150000in}}{\pgfqpoint{5.490039in}{5.490039in}}%
\pgfusepath{clip}%
\pgfsetbuttcap%
\pgfsetroundjoin%
\definecolor{currentfill}{rgb}{0.129933,0.559582,0.551864}%
\pgfsetfillcolor{currentfill}%
\pgfsetfillopacity{0.700000}%
\pgfsetlinewidth{0.000000pt}%
\definecolor{currentstroke}{rgb}{0.000000,0.000000,0.000000}%
\pgfsetstrokecolor{currentstroke}%
\pgfsetdash{}{0pt}%
\pgfpathmoveto{\pgfqpoint{2.503655in}{2.626718in}}%
\pgfpathlineto{\pgfqpoint{2.517090in}{2.615727in}}%
\pgfpathlineto{\pgfqpoint{2.530526in}{2.604777in}}%
\pgfpathlineto{\pgfqpoint{2.543963in}{2.593867in}}%
\pgfpathlineto{\pgfqpoint{2.557402in}{2.582997in}}%
\pgfpathlineto{\pgfqpoint{2.548136in}{2.603440in}}%
\pgfpathlineto{\pgfqpoint{2.538833in}{2.624575in}}%
\pgfpathlineto{\pgfqpoint{2.529493in}{2.646418in}}%
\pgfpathlineto{\pgfqpoint{2.520113in}{2.668982in}}%
\pgfpathlineto{\pgfqpoint{2.506615in}{2.680289in}}%
\pgfpathlineto{\pgfqpoint{2.493117in}{2.691637in}}%
\pgfpathlineto{\pgfqpoint{2.479620in}{2.703025in}}%
\pgfpathlineto{\pgfqpoint{2.466124in}{2.714455in}}%
\pgfpathlineto{\pgfqpoint{2.475565in}{2.691446in}}%
\pgfpathlineto{\pgfqpoint{2.484967in}{2.669163in}}%
\pgfpathlineto{\pgfqpoint{2.494330in}{2.647591in}}%
\pgfpathlineto{\pgfqpoint{2.503655in}{2.626718in}}%
\pgfpathclose%
\pgfusepath{fill}%
\end{pgfscope}%
\begin{pgfscope}%
\pgfpathrectangle{\pgfqpoint{1.254980in}{0.150000in}}{\pgfqpoint{5.490039in}{5.490039in}}%
\pgfusepath{clip}%
\pgfsetbuttcap%
\pgfsetroundjoin%
\definecolor{currentfill}{rgb}{0.282290,0.145912,0.461510}%
\pgfsetfillcolor{currentfill}%
\pgfsetfillopacity{0.700000}%
\pgfsetlinewidth{0.000000pt}%
\definecolor{currentstroke}{rgb}{0.000000,0.000000,0.000000}%
\pgfsetstrokecolor{currentstroke}%
\pgfsetdash{}{0pt}%
\pgfpathmoveto{\pgfqpoint{3.878149in}{1.647868in}}%
\pgfpathlineto{\pgfqpoint{3.891702in}{1.641227in}}%
\pgfpathlineto{\pgfqpoint{3.905260in}{1.634611in}}%
\pgfpathlineto{\pgfqpoint{3.918822in}{1.628019in}}%
\pgfpathlineto{\pgfqpoint{3.932390in}{1.621452in}}%
\pgfpathlineto{\pgfqpoint{3.924450in}{1.624296in}}%
\pgfpathlineto{\pgfqpoint{3.916500in}{1.627557in}}%
\pgfpathlineto{\pgfqpoint{3.908539in}{1.631246in}}%
\pgfpathlineto{\pgfqpoint{3.900566in}{1.635371in}}%
\pgfpathlineto{\pgfqpoint{3.886970in}{1.642295in}}%
\pgfpathlineto{\pgfqpoint{3.873378in}{1.649244in}}%
\pgfpathlineto{\pgfqpoint{3.859791in}{1.656217in}}%
\pgfpathlineto{\pgfqpoint{3.846209in}{1.663215in}}%
\pgfpathlineto{\pgfqpoint{3.854212in}{1.658727in}}%
\pgfpathlineto{\pgfqpoint{3.862202in}{1.654680in}}%
\pgfpathlineto{\pgfqpoint{3.870181in}{1.651064in}}%
\pgfpathlineto{\pgfqpoint{3.878149in}{1.647868in}}%
\pgfpathclose%
\pgfusepath{fill}%
\end{pgfscope}%
\begin{pgfscope}%
\pgfpathrectangle{\pgfqpoint{1.254980in}{0.150000in}}{\pgfqpoint{5.490039in}{5.490039in}}%
\pgfusepath{clip}%
\pgfsetbuttcap%
\pgfsetroundjoin%
\definecolor{currentfill}{rgb}{0.279566,0.067836,0.391917}%
\pgfsetfillcolor{currentfill}%
\pgfsetfillopacity{0.700000}%
\pgfsetlinewidth{0.000000pt}%
\definecolor{currentstroke}{rgb}{0.000000,0.000000,0.000000}%
\pgfsetstrokecolor{currentstroke}%
\pgfsetdash{}{0pt}%
\pgfpathmoveto{\pgfqpoint{4.267004in}{1.499690in}}%
\pgfpathlineto{\pgfqpoint{4.280636in}{1.494293in}}%
\pgfpathlineto{\pgfqpoint{4.294273in}{1.488919in}}%
\pgfpathlineto{\pgfqpoint{4.307916in}{1.483569in}}%
\pgfpathlineto{\pgfqpoint{4.321565in}{1.478242in}}%
\pgfpathlineto{\pgfqpoint{4.313833in}{1.476147in}}%
\pgfpathlineto{\pgfqpoint{4.306095in}{1.474374in}}%
\pgfpathlineto{\pgfqpoint{4.298351in}{1.472933in}}%
\pgfpathlineto{\pgfqpoint{4.290600in}{1.471831in}}%
\pgfpathlineto{\pgfqpoint{4.276931in}{1.477487in}}%
\pgfpathlineto{\pgfqpoint{4.263268in}{1.483166in}}%
\pgfpathlineto{\pgfqpoint{4.249610in}{1.488868in}}%
\pgfpathlineto{\pgfqpoint{4.235958in}{1.494593in}}%
\pgfpathlineto{\pgfqpoint{4.243730in}{1.495361in}}%
\pgfpathlineto{\pgfqpoint{4.251495in}{1.496472in}}%
\pgfpathlineto{\pgfqpoint{4.259253in}{1.497918in}}%
\pgfpathlineto{\pgfqpoint{4.267004in}{1.499690in}}%
\pgfpathclose%
\pgfusepath{fill}%
\end{pgfscope}%
\begin{pgfscope}%
\pgfpathrectangle{\pgfqpoint{1.254980in}{0.150000in}}{\pgfqpoint{5.490039in}{5.490039in}}%
\pgfusepath{clip}%
\pgfsetbuttcap%
\pgfsetroundjoin%
\definecolor{currentfill}{rgb}{0.272594,0.025563,0.353093}%
\pgfsetfillcolor{currentfill}%
\pgfsetfillopacity{0.700000}%
\pgfsetlinewidth{0.000000pt}%
\definecolor{currentstroke}{rgb}{0.000000,0.000000,0.000000}%
\pgfsetstrokecolor{currentstroke}%
\pgfsetdash{}{0pt}%
\pgfpathmoveto{\pgfqpoint{4.827526in}{1.442206in}}%
\pgfpathlineto{\pgfqpoint{4.841306in}{1.438726in}}%
\pgfpathlineto{\pgfqpoint{4.855092in}{1.435269in}}%
\pgfpathlineto{\pgfqpoint{4.868886in}{1.431835in}}%
\pgfpathlineto{\pgfqpoint{4.882687in}{1.428424in}}%
\pgfpathlineto{\pgfqpoint{4.875135in}{1.420096in}}%
\pgfpathlineto{\pgfqpoint{4.867581in}{1.411940in}}%
\pgfpathlineto{\pgfqpoint{4.860024in}{1.403963in}}%
\pgfpathlineto{\pgfqpoint{4.852464in}{1.396171in}}%
\pgfpathlineto{\pgfqpoint{4.838653in}{1.399860in}}%
\pgfpathlineto{\pgfqpoint{4.824850in}{1.403571in}}%
\pgfpathlineto{\pgfqpoint{4.811053in}{1.407304in}}%
\pgfpathlineto{\pgfqpoint{4.797263in}{1.411061in}}%
\pgfpathlineto{\pgfqpoint{4.804834in}{1.418570in}}%
\pgfpathlineto{\pgfqpoint{4.812401in}{1.426269in}}%
\pgfpathlineto{\pgfqpoint{4.819965in}{1.434150in}}%
\pgfpathlineto{\pgfqpoint{4.827526in}{1.442206in}}%
\pgfpathclose%
\pgfusepath{fill}%
\end{pgfscope}%
\begin{pgfscope}%
\pgfpathrectangle{\pgfqpoint{1.254980in}{0.150000in}}{\pgfqpoint{5.490039in}{5.490039in}}%
\pgfusepath{clip}%
\pgfsetbuttcap%
\pgfsetroundjoin%
\definecolor{currentfill}{rgb}{0.277018,0.050344,0.375715}%
\pgfsetfillcolor{currentfill}%
\pgfsetfillopacity{0.700000}%
\pgfsetlinewidth{0.000000pt}%
\definecolor{currentstroke}{rgb}{0.000000,0.000000,0.000000}%
\pgfsetstrokecolor{currentstroke}%
\pgfsetdash{}{0pt}%
\pgfpathmoveto{\pgfqpoint{5.053534in}{1.478075in}}%
\pgfpathlineto{\pgfqpoint{5.067386in}{1.475384in}}%
\pgfpathlineto{\pgfqpoint{5.081246in}{1.472716in}}%
\pgfpathlineto{\pgfqpoint{5.095114in}{1.470071in}}%
\pgfpathlineto{\pgfqpoint{5.108989in}{1.467449in}}%
\pgfpathlineto{\pgfqpoint{5.101483in}{1.457246in}}%
\pgfpathlineto{\pgfqpoint{5.093975in}{1.447154in}}%
\pgfpathlineto{\pgfqpoint{5.086463in}{1.437177in}}%
\pgfpathlineto{\pgfqpoint{5.078949in}{1.427323in}}%
\pgfpathlineto{\pgfqpoint{5.065067in}{1.430197in}}%
\pgfpathlineto{\pgfqpoint{5.051193in}{1.433094in}}%
\pgfpathlineto{\pgfqpoint{5.037326in}{1.436014in}}%
\pgfpathlineto{\pgfqpoint{5.023467in}{1.438957in}}%
\pgfpathlineto{\pgfqpoint{5.030988in}{1.448554in}}%
\pgfpathlineto{\pgfqpoint{5.038506in}{1.458278in}}%
\pgfpathlineto{\pgfqpoint{5.046021in}{1.468120in}}%
\pgfpathlineto{\pgfqpoint{5.053534in}{1.478075in}}%
\pgfpathclose%
\pgfusepath{fill}%
\end{pgfscope}%
\begin{pgfscope}%
\pgfpathrectangle{\pgfqpoint{1.254980in}{0.150000in}}{\pgfqpoint{5.490039in}{5.490039in}}%
\pgfusepath{clip}%
\pgfsetbuttcap%
\pgfsetroundjoin%
\definecolor{currentfill}{rgb}{0.195860,0.395433,0.555276}%
\pgfsetfillcolor{currentfill}%
\pgfsetfillopacity{0.700000}%
\pgfsetlinewidth{0.000000pt}%
\definecolor{currentstroke}{rgb}{0.000000,0.000000,0.000000}%
\pgfsetstrokecolor{currentstroke}%
\pgfsetdash{}{0pt}%
\pgfpathmoveto{\pgfqpoint{3.023368in}{2.192742in}}%
\pgfpathlineto{\pgfqpoint{3.036821in}{2.183448in}}%
\pgfpathlineto{\pgfqpoint{3.050277in}{2.174185in}}%
\pgfpathlineto{\pgfqpoint{3.063735in}{2.164954in}}%
\pgfpathlineto{\pgfqpoint{3.077196in}{2.155753in}}%
\pgfpathlineto{\pgfqpoint{3.068528in}{2.169883in}}%
\pgfpathlineto{\pgfqpoint{3.059835in}{2.184615in}}%
\pgfpathlineto{\pgfqpoint{3.051115in}{2.199962in}}%
\pgfpathlineto{\pgfqpoint{3.042368in}{2.215935in}}%
\pgfpathlineto{\pgfqpoint{3.028859in}{2.225547in}}%
\pgfpathlineto{\pgfqpoint{3.015351in}{2.235189in}}%
\pgfpathlineto{\pgfqpoint{3.001847in}{2.244863in}}%
\pgfpathlineto{\pgfqpoint{2.988345in}{2.254568in}}%
\pgfpathlineto{\pgfqpoint{2.997142in}{2.238177in}}%
\pgfpathlineto{\pgfqpoint{3.005911in}{2.222417in}}%
\pgfpathlineto{\pgfqpoint{3.014653in}{2.207276in}}%
\pgfpathlineto{\pgfqpoint{3.023368in}{2.192742in}}%
\pgfpathclose%
\pgfusepath{fill}%
\end{pgfscope}%
\begin{pgfscope}%
\pgfpathrectangle{\pgfqpoint{1.254980in}{0.150000in}}{\pgfqpoint{5.490039in}{5.490039in}}%
\pgfusepath{clip}%
\pgfsetbuttcap%
\pgfsetroundjoin%
\definecolor{currentfill}{rgb}{0.244972,0.287675,0.537260}%
\pgfsetfillcolor{currentfill}%
\pgfsetfillopacity{0.700000}%
\pgfsetlinewidth{0.000000pt}%
\definecolor{currentstroke}{rgb}{0.000000,0.000000,0.000000}%
\pgfsetstrokecolor{currentstroke}%
\pgfsetdash{}{0pt}%
\pgfpathmoveto{\pgfqpoint{3.380613in}{1.935061in}}%
\pgfpathlineto{\pgfqpoint{3.394099in}{1.926861in}}%
\pgfpathlineto{\pgfqpoint{3.407588in}{1.918689in}}%
\pgfpathlineto{\pgfqpoint{3.421081in}{1.910544in}}%
\pgfpathlineto{\pgfqpoint{3.434577in}{1.902427in}}%
\pgfpathlineto{\pgfqpoint{3.426255in}{1.911952in}}%
\pgfpathlineto{\pgfqpoint{3.417915in}{1.922009in}}%
\pgfpathlineto{\pgfqpoint{3.409555in}{1.932608in}}%
\pgfpathlineto{\pgfqpoint{3.401176in}{1.943761in}}%
\pgfpathlineto{\pgfqpoint{3.387639in}{1.952268in}}%
\pgfpathlineto{\pgfqpoint{3.374106in}{1.960803in}}%
\pgfpathlineto{\pgfqpoint{3.360576in}{1.969366in}}%
\pgfpathlineto{\pgfqpoint{3.347050in}{1.977956in}}%
\pgfpathlineto{\pgfqpoint{3.355471in}{1.966407in}}%
\pgfpathlineto{\pgfqpoint{3.363872in}{1.955415in}}%
\pgfpathlineto{\pgfqpoint{3.372252in}{1.944971in}}%
\pgfpathlineto{\pgfqpoint{3.380613in}{1.935061in}}%
\pgfpathclose%
\pgfusepath{fill}%
\end{pgfscope}%
\begin{pgfscope}%
\pgfpathrectangle{\pgfqpoint{1.254980in}{0.150000in}}{\pgfqpoint{5.490039in}{5.490039in}}%
\pgfusepath{clip}%
\pgfsetbuttcap%
\pgfsetroundjoin%
\definecolor{currentfill}{rgb}{0.135066,0.544853,0.554029}%
\pgfsetfillcolor{currentfill}%
\pgfsetfillopacity{0.700000}%
\pgfsetlinewidth{0.000000pt}%
\definecolor{currentstroke}{rgb}{0.000000,0.000000,0.000000}%
\pgfsetstrokecolor{currentstroke}%
\pgfsetdash{}{0pt}%
\pgfpathmoveto{\pgfqpoint{2.557402in}{2.582997in}}%
\pgfpathlineto{\pgfqpoint{2.570842in}{2.572168in}}%
\pgfpathlineto{\pgfqpoint{2.584283in}{2.561377in}}%
\pgfpathlineto{\pgfqpoint{2.597726in}{2.550626in}}%
\pgfpathlineto{\pgfqpoint{2.611170in}{2.539913in}}%
\pgfpathlineto{\pgfqpoint{2.601963in}{2.559926in}}%
\pgfpathlineto{\pgfqpoint{2.592719in}{2.580627in}}%
\pgfpathlineto{\pgfqpoint{2.583438in}{2.602031in}}%
\pgfpathlineto{\pgfqpoint{2.574120in}{2.624150in}}%
\pgfpathlineto{\pgfqpoint{2.560617in}{2.635299in}}%
\pgfpathlineto{\pgfqpoint{2.547115in}{2.646487in}}%
\pgfpathlineto{\pgfqpoint{2.533613in}{2.657715in}}%
\pgfpathlineto{\pgfqpoint{2.520113in}{2.668982in}}%
\pgfpathlineto{\pgfqpoint{2.529493in}{2.646418in}}%
\pgfpathlineto{\pgfqpoint{2.538833in}{2.624575in}}%
\pgfpathlineto{\pgfqpoint{2.548136in}{2.603440in}}%
\pgfpathlineto{\pgfqpoint{2.557402in}{2.582997in}}%
\pgfpathclose%
\pgfusepath{fill}%
\end{pgfscope}%
\begin{pgfscope}%
\pgfpathrectangle{\pgfqpoint{1.254980in}{0.150000in}}{\pgfqpoint{5.490039in}{5.490039in}}%
\pgfusepath{clip}%
\pgfsetbuttcap%
\pgfsetroundjoin%
\definecolor{currentfill}{rgb}{0.273006,0.204520,0.501721}%
\pgfsetfillcolor{currentfill}%
\pgfsetfillopacity{0.700000}%
\pgfsetlinewidth{0.000000pt}%
\definecolor{currentstroke}{rgb}{0.000000,0.000000,0.000000}%
\pgfsetstrokecolor{currentstroke}%
\pgfsetdash{}{0pt}%
\pgfpathmoveto{\pgfqpoint{3.683574in}{1.749128in}}%
\pgfpathlineto{\pgfqpoint{3.697102in}{1.741830in}}%
\pgfpathlineto{\pgfqpoint{3.710635in}{1.734558in}}%
\pgfpathlineto{\pgfqpoint{3.724173in}{1.727311in}}%
\pgfpathlineto{\pgfqpoint{3.737714in}{1.720089in}}%
\pgfpathlineto{\pgfqpoint{3.729636in}{1.725761in}}%
\pgfpathlineto{\pgfqpoint{3.721544in}{1.731902in}}%
\pgfpathlineto{\pgfqpoint{3.713438in}{1.738522in}}%
\pgfpathlineto{\pgfqpoint{3.705317in}{1.745632in}}%
\pgfpathlineto{\pgfqpoint{3.691741in}{1.753226in}}%
\pgfpathlineto{\pgfqpoint{3.678170in}{1.760846in}}%
\pgfpathlineto{\pgfqpoint{3.664603in}{1.768491in}}%
\pgfpathlineto{\pgfqpoint{3.651040in}{1.776162in}}%
\pgfpathlineto{\pgfqpoint{3.659196in}{1.768673in}}%
\pgfpathlineto{\pgfqpoint{3.667337in}{1.761679in}}%
\pgfpathlineto{\pgfqpoint{3.675462in}{1.755167in}}%
\pgfpathlineto{\pgfqpoint{3.683574in}{1.749128in}}%
\pgfpathclose%
\pgfusepath{fill}%
\end{pgfscope}%
\begin{pgfscope}%
\pgfpathrectangle{\pgfqpoint{1.254980in}{0.150000in}}{\pgfqpoint{5.490039in}{5.490039in}}%
\pgfusepath{clip}%
\pgfsetbuttcap%
\pgfsetroundjoin%
\definecolor{currentfill}{rgb}{0.273809,0.031497,0.358853}%
\pgfsetfillcolor{currentfill}%
\pgfsetfillopacity{0.700000}%
\pgfsetlinewidth{0.000000pt}%
\definecolor{currentstroke}{rgb}{0.000000,0.000000,0.000000}%
\pgfsetstrokecolor{currentstroke}%
\pgfsetdash{}{0pt}%
\pgfpathmoveto{\pgfqpoint{4.601865in}{1.433381in}}%
\pgfpathlineto{\pgfqpoint{4.615586in}{1.429062in}}%
\pgfpathlineto{\pgfqpoint{4.629313in}{1.424766in}}%
\pgfpathlineto{\pgfqpoint{4.643047in}{1.420493in}}%
\pgfpathlineto{\pgfqpoint{4.656788in}{1.416243in}}%
\pgfpathlineto{\pgfqpoint{4.649176in}{1.410383in}}%
\pgfpathlineto{\pgfqpoint{4.641562in}{1.404764in}}%
\pgfpathlineto{\pgfqpoint{4.633943in}{1.399392in}}%
\pgfpathlineto{\pgfqpoint{4.626321in}{1.394277in}}%
\pgfpathlineto{\pgfqpoint{4.612566in}{1.398830in}}%
\pgfpathlineto{\pgfqpoint{4.598819in}{1.403405in}}%
\pgfpathlineto{\pgfqpoint{4.585077in}{1.408004in}}%
\pgfpathlineto{\pgfqpoint{4.571342in}{1.412625in}}%
\pgfpathlineto{\pgfqpoint{4.578979in}{1.417433in}}%
\pgfpathlineto{\pgfqpoint{4.586612in}{1.422500in}}%
\pgfpathlineto{\pgfqpoint{4.594241in}{1.427819in}}%
\pgfpathlineto{\pgfqpoint{4.601865in}{1.433381in}}%
\pgfpathclose%
\pgfusepath{fill}%
\end{pgfscope}%
\begin{pgfscope}%
\pgfpathrectangle{\pgfqpoint{1.254980in}{0.150000in}}{\pgfqpoint{5.490039in}{5.490039in}}%
\pgfusepath{clip}%
\pgfsetbuttcap%
\pgfsetroundjoin%
\definecolor{currentfill}{rgb}{0.274952,0.037752,0.364543}%
\pgfsetfillcolor{currentfill}%
\pgfsetfillopacity{0.700000}%
\pgfsetlinewidth{0.000000pt}%
\definecolor{currentstroke}{rgb}{0.000000,0.000000,0.000000}%
\pgfsetstrokecolor{currentstroke}%
\pgfsetdash{}{0pt}%
\pgfpathmoveto{\pgfqpoint{4.968105in}{1.450954in}}%
\pgfpathlineto{\pgfqpoint{4.981934in}{1.447921in}}%
\pgfpathlineto{\pgfqpoint{4.995771in}{1.444910in}}%
\pgfpathlineto{\pgfqpoint{5.009615in}{1.441922in}}%
\pgfpathlineto{\pgfqpoint{5.023467in}{1.438957in}}%
\pgfpathlineto{\pgfqpoint{5.015943in}{1.429491in}}%
\pgfpathlineto{\pgfqpoint{5.008417in}{1.420163in}}%
\pgfpathlineto{\pgfqpoint{5.000888in}{1.410981in}}%
\pgfpathlineto{\pgfqpoint{4.993356in}{1.401950in}}%
\pgfpathlineto{\pgfqpoint{4.979497in}{1.405179in}}%
\pgfpathlineto{\pgfqpoint{4.965645in}{1.408432in}}%
\pgfpathlineto{\pgfqpoint{4.951800in}{1.411707in}}%
\pgfpathlineto{\pgfqpoint{4.937963in}{1.415005in}}%
\pgfpathlineto{\pgfqpoint{4.945503in}{1.423767in}}%
\pgfpathlineto{\pgfqpoint{4.953040in}{1.432683in}}%
\pgfpathlineto{\pgfqpoint{4.960574in}{1.441748in}}%
\pgfpathlineto{\pgfqpoint{4.968105in}{1.450954in}}%
\pgfpathclose%
\pgfusepath{fill}%
\end{pgfscope}%
\begin{pgfscope}%
\pgfpathrectangle{\pgfqpoint{1.254980in}{0.150000in}}{\pgfqpoint{5.490039in}{5.490039in}}%
\pgfusepath{clip}%
\pgfsetbuttcap%
\pgfsetroundjoin%
\definecolor{currentfill}{rgb}{0.276022,0.044167,0.370164}%
\pgfsetfillcolor{currentfill}%
\pgfsetfillopacity{0.700000}%
\pgfsetlinewidth{0.000000pt}%
\definecolor{currentstroke}{rgb}{0.000000,0.000000,0.000000}%
\pgfsetstrokecolor{currentstroke}%
\pgfsetdash{}{0pt}%
\pgfpathmoveto{\pgfqpoint{4.461689in}{1.450415in}}%
\pgfpathlineto{\pgfqpoint{4.475374in}{1.445611in}}%
\pgfpathlineto{\pgfqpoint{4.489065in}{1.440830in}}%
\pgfpathlineto{\pgfqpoint{4.502762in}{1.436072in}}%
\pgfpathlineto{\pgfqpoint{4.516465in}{1.431337in}}%
\pgfpathlineto{\pgfqpoint{4.508809in}{1.427106in}}%
\pgfpathlineto{\pgfqpoint{4.501147in}{1.423154in}}%
\pgfpathlineto{\pgfqpoint{4.493481in}{1.419489in}}%
\pgfpathlineto{\pgfqpoint{4.485810in}{1.416120in}}%
\pgfpathlineto{\pgfqpoint{4.472090in}{1.421170in}}%
\pgfpathlineto{\pgfqpoint{4.458376in}{1.426244in}}%
\pgfpathlineto{\pgfqpoint{4.444668in}{1.431340in}}%
\pgfpathlineto{\pgfqpoint{4.430966in}{1.436459in}}%
\pgfpathlineto{\pgfqpoint{4.438654in}{1.439509in}}%
\pgfpathlineto{\pgfqpoint{4.446338in}{1.442857in}}%
\pgfpathlineto{\pgfqpoint{4.454016in}{1.446495in}}%
\pgfpathlineto{\pgfqpoint{4.461689in}{1.450415in}}%
\pgfpathclose%
\pgfusepath{fill}%
\end{pgfscope}%
\begin{pgfscope}%
\pgfpathrectangle{\pgfqpoint{1.254980in}{0.150000in}}{\pgfqpoint{5.490039in}{5.490039in}}%
\pgfusepath{clip}%
\pgfsetbuttcap%
\pgfsetroundjoin%
\definecolor{currentfill}{rgb}{0.282327,0.094955,0.417331}%
\pgfsetfillcolor{currentfill}%
\pgfsetfillopacity{0.700000}%
\pgfsetlinewidth{0.000000pt}%
\definecolor{currentstroke}{rgb}{0.000000,0.000000,0.000000}%
\pgfsetstrokecolor{currentstroke}%
\pgfsetdash{}{0pt}%
\pgfpathmoveto{\pgfqpoint{4.126937in}{1.541239in}}%
\pgfpathlineto{\pgfqpoint{4.140545in}{1.535326in}}%
\pgfpathlineto{\pgfqpoint{4.154159in}{1.529436in}}%
\pgfpathlineto{\pgfqpoint{4.167779in}{1.523570in}}%
\pgfpathlineto{\pgfqpoint{4.181404in}{1.517728in}}%
\pgfpathlineto{\pgfqpoint{4.173602in}{1.517650in}}%
\pgfpathlineto{\pgfqpoint{4.165793in}{1.517937in}}%
\pgfpathlineto{\pgfqpoint{4.157975in}{1.518599in}}%
\pgfpathlineto{\pgfqpoint{4.150150in}{1.519644in}}%
\pgfpathlineto{\pgfqpoint{4.136501in}{1.525828in}}%
\pgfpathlineto{\pgfqpoint{4.122858in}{1.532037in}}%
\pgfpathlineto{\pgfqpoint{4.109220in}{1.538269in}}%
\pgfpathlineto{\pgfqpoint{4.095587in}{1.544524in}}%
\pgfpathlineto{\pgfqpoint{4.103437in}{1.543131in}}%
\pgfpathlineto{\pgfqpoint{4.111279in}{1.542126in}}%
\pgfpathlineto{\pgfqpoint{4.119112in}{1.541498in}}%
\pgfpathlineto{\pgfqpoint{4.126937in}{1.541239in}}%
\pgfpathclose%
\pgfusepath{fill}%
\end{pgfscope}%
\begin{pgfscope}%
\pgfpathrectangle{\pgfqpoint{1.254980in}{0.150000in}}{\pgfqpoint{5.490039in}{5.490039in}}%
\pgfusepath{clip}%
\pgfsetbuttcap%
\pgfsetroundjoin%
\definecolor{currentfill}{rgb}{0.140536,0.530132,0.555659}%
\pgfsetfillcolor{currentfill}%
\pgfsetfillopacity{0.700000}%
\pgfsetlinewidth{0.000000pt}%
\definecolor{currentstroke}{rgb}{0.000000,0.000000,0.000000}%
\pgfsetstrokecolor{currentstroke}%
\pgfsetdash{}{0pt}%
\pgfpathmoveto{\pgfqpoint{2.611170in}{2.539913in}}%
\pgfpathlineto{\pgfqpoint{2.624616in}{2.529239in}}%
\pgfpathlineto{\pgfqpoint{2.638063in}{2.518603in}}%
\pgfpathlineto{\pgfqpoint{2.651512in}{2.508005in}}%
\pgfpathlineto{\pgfqpoint{2.664962in}{2.497445in}}%
\pgfpathlineto{\pgfqpoint{2.655812in}{2.517029in}}%
\pgfpathlineto{\pgfqpoint{2.646626in}{2.537297in}}%
\pgfpathlineto{\pgfqpoint{2.637405in}{2.558263in}}%
\pgfpathlineto{\pgfqpoint{2.628147in}{2.579939in}}%
\pgfpathlineto{\pgfqpoint{2.614639in}{2.590935in}}%
\pgfpathlineto{\pgfqpoint{2.601131in}{2.601969in}}%
\pgfpathlineto{\pgfqpoint{2.587625in}{2.613040in}}%
\pgfpathlineto{\pgfqpoint{2.574120in}{2.624150in}}%
\pgfpathlineto{\pgfqpoint{2.583438in}{2.602031in}}%
\pgfpathlineto{\pgfqpoint{2.592719in}{2.580627in}}%
\pgfpathlineto{\pgfqpoint{2.601963in}{2.559926in}}%
\pgfpathlineto{\pgfqpoint{2.611170in}{2.539913in}}%
\pgfpathclose%
\pgfusepath{fill}%
\end{pgfscope}%
\begin{pgfscope}%
\pgfpathrectangle{\pgfqpoint{1.254980in}{0.150000in}}{\pgfqpoint{5.490039in}{5.490039in}}%
\pgfusepath{clip}%
\pgfsetbuttcap%
\pgfsetroundjoin%
\definecolor{currentfill}{rgb}{0.201239,0.383670,0.554294}%
\pgfsetfillcolor{currentfill}%
\pgfsetfillopacity{0.700000}%
\pgfsetlinewidth{0.000000pt}%
\definecolor{currentstroke}{rgb}{0.000000,0.000000,0.000000}%
\pgfsetstrokecolor{currentstroke}%
\pgfsetdash{}{0pt}%
\pgfpathmoveto{\pgfqpoint{3.077196in}{2.155753in}}%
\pgfpathlineto{\pgfqpoint{3.090660in}{2.146582in}}%
\pgfpathlineto{\pgfqpoint{3.104127in}{2.137443in}}%
\pgfpathlineto{\pgfqpoint{3.117597in}{2.128333in}}%
\pgfpathlineto{\pgfqpoint{3.131069in}{2.119254in}}%
\pgfpathlineto{\pgfqpoint{3.122448in}{2.132980in}}%
\pgfpathlineto{\pgfqpoint{3.113802in}{2.147305in}}%
\pgfpathlineto{\pgfqpoint{3.105131in}{2.162239in}}%
\pgfpathlineto{\pgfqpoint{3.096433in}{2.177797in}}%
\pgfpathlineto{\pgfqpoint{3.082913in}{2.187286in}}%
\pgfpathlineto{\pgfqpoint{3.069396in}{2.196805in}}%
\pgfpathlineto{\pgfqpoint{3.055881in}{2.206355in}}%
\pgfpathlineto{\pgfqpoint{3.042368in}{2.215935in}}%
\pgfpathlineto{\pgfqpoint{3.051115in}{2.199962in}}%
\pgfpathlineto{\pgfqpoint{3.059835in}{2.184615in}}%
\pgfpathlineto{\pgfqpoint{3.068528in}{2.169883in}}%
\pgfpathlineto{\pgfqpoint{3.077196in}{2.155753in}}%
\pgfpathclose%
\pgfusepath{fill}%
\end{pgfscope}%
\begin{pgfscope}%
\pgfpathrectangle{\pgfqpoint{1.254980in}{0.150000in}}{\pgfqpoint{5.490039in}{5.490039in}}%
\pgfusepath{clip}%
\pgfsetbuttcap%
\pgfsetroundjoin%
\definecolor{currentfill}{rgb}{0.282623,0.140926,0.457517}%
\pgfsetfillcolor{currentfill}%
\pgfsetfillopacity{0.700000}%
\pgfsetlinewidth{0.000000pt}%
\definecolor{currentstroke}{rgb}{0.000000,0.000000,0.000000}%
\pgfsetstrokecolor{currentstroke}%
\pgfsetdash{}{0pt}%
\pgfpathmoveto{\pgfqpoint{3.932390in}{1.621452in}}%
\pgfpathlineto{\pgfqpoint{3.945962in}{1.614909in}}%
\pgfpathlineto{\pgfqpoint{3.959539in}{1.608390in}}%
\pgfpathlineto{\pgfqpoint{3.973121in}{1.601896in}}%
\pgfpathlineto{\pgfqpoint{3.986709in}{1.595425in}}%
\pgfpathlineto{\pgfqpoint{3.978797in}{1.597918in}}%
\pgfpathlineto{\pgfqpoint{3.970874in}{1.600824in}}%
\pgfpathlineto{\pgfqpoint{3.962942in}{1.604153in}}%
\pgfpathlineto{\pgfqpoint{3.954998in}{1.607916in}}%
\pgfpathlineto{\pgfqpoint{3.941383in}{1.614743in}}%
\pgfpathlineto{\pgfqpoint{3.927773in}{1.621595in}}%
\pgfpathlineto{\pgfqpoint{3.914167in}{1.628471in}}%
\pgfpathlineto{\pgfqpoint{3.900566in}{1.635371in}}%
\pgfpathlineto{\pgfqpoint{3.908539in}{1.631246in}}%
\pgfpathlineto{\pgfqpoint{3.916500in}{1.627557in}}%
\pgfpathlineto{\pgfqpoint{3.924450in}{1.624296in}}%
\pgfpathlineto{\pgfqpoint{3.932390in}{1.621452in}}%
\pgfpathclose%
\pgfusepath{fill}%
\end{pgfscope}%
\begin{pgfscope}%
\pgfpathrectangle{\pgfqpoint{1.254980in}{0.150000in}}{\pgfqpoint{5.490039in}{5.490039in}}%
\pgfusepath{clip}%
\pgfsetbuttcap%
\pgfsetroundjoin%
\definecolor{currentfill}{rgb}{0.272594,0.025563,0.353093}%
\pgfsetfillcolor{currentfill}%
\pgfsetfillopacity{0.700000}%
\pgfsetlinewidth{0.000000pt}%
\definecolor{currentstroke}{rgb}{0.000000,0.000000,0.000000}%
\pgfsetstrokecolor{currentstroke}%
\pgfsetdash{}{0pt}%
\pgfpathmoveto{\pgfqpoint{4.742174in}{1.426313in}}%
\pgfpathlineto{\pgfqpoint{4.755936in}{1.422466in}}%
\pgfpathlineto{\pgfqpoint{4.769705in}{1.418642in}}%
\pgfpathlineto{\pgfqpoint{4.783481in}{1.414840in}}%
\pgfpathlineto{\pgfqpoint{4.797263in}{1.411061in}}%
\pgfpathlineto{\pgfqpoint{4.789690in}{1.403747in}}%
\pgfpathlineto{\pgfqpoint{4.782113in}{1.396637in}}%
\pgfpathlineto{\pgfqpoint{4.774534in}{1.389738in}}%
\pgfpathlineto{\pgfqpoint{4.766951in}{1.383057in}}%
\pgfpathlineto{\pgfqpoint{4.753157in}{1.387126in}}%
\pgfpathlineto{\pgfqpoint{4.739370in}{1.391218in}}%
\pgfpathlineto{\pgfqpoint{4.725589in}{1.395332in}}%
\pgfpathlineto{\pgfqpoint{4.711816in}{1.399468in}}%
\pgfpathlineto{\pgfqpoint{4.719410in}{1.405855in}}%
\pgfpathlineto{\pgfqpoint{4.727002in}{1.412463in}}%
\pgfpathlineto{\pgfqpoint{4.734590in}{1.419285in}}%
\pgfpathlineto{\pgfqpoint{4.742174in}{1.426313in}}%
\pgfpathclose%
\pgfusepath{fill}%
\end{pgfscope}%
\begin{pgfscope}%
\pgfpathrectangle{\pgfqpoint{1.254980in}{0.150000in}}{\pgfqpoint{5.490039in}{5.490039in}}%
\pgfusepath{clip}%
\pgfsetbuttcap%
\pgfsetroundjoin%
\definecolor{currentfill}{rgb}{0.279566,0.067836,0.391917}%
\pgfsetfillcolor{currentfill}%
\pgfsetfillopacity{0.700000}%
\pgfsetlinewidth{0.000000pt}%
\definecolor{currentstroke}{rgb}{0.000000,0.000000,0.000000}%
\pgfsetstrokecolor{currentstroke}%
\pgfsetdash{}{0pt}%
\pgfpathmoveto{\pgfqpoint{5.194542in}{1.499929in}}%
\pgfpathlineto{\pgfqpoint{5.208451in}{1.497660in}}%
\pgfpathlineto{\pgfqpoint{5.222368in}{1.495414in}}%
\pgfpathlineto{\pgfqpoint{5.236294in}{1.493191in}}%
\pgfpathlineto{\pgfqpoint{5.228809in}{1.482200in}}%
\pgfpathlineto{\pgfqpoint{5.221321in}{1.471289in}}%
\pgfpathlineto{\pgfqpoint{5.213830in}{1.460464in}}%
\pgfpathlineto{\pgfqpoint{5.206337in}{1.449731in}}%
\pgfpathlineto{\pgfqpoint{5.192407in}{1.452193in}}%
\pgfpathlineto{\pgfqpoint{5.178484in}{1.454679in}}%
\pgfpathlineto{\pgfqpoint{5.164569in}{1.457187in}}%
\pgfpathlineto{\pgfqpoint{5.172067in}{1.467737in}}%
\pgfpathlineto{\pgfqpoint{5.179561in}{1.478381in}}%
\pgfpathlineto{\pgfqpoint{5.187053in}{1.489114in}}%
\pgfpathlineto{\pgfqpoint{5.194542in}{1.499929in}}%
\pgfpathclose%
\pgfusepath{fill}%
\end{pgfscope}%
\begin{pgfscope}%
\pgfpathrectangle{\pgfqpoint{1.254980in}{0.150000in}}{\pgfqpoint{5.490039in}{5.490039in}}%
\pgfusepath{clip}%
\pgfsetbuttcap%
\pgfsetroundjoin%
\definecolor{currentfill}{rgb}{0.248629,0.278775,0.534556}%
\pgfsetfillcolor{currentfill}%
\pgfsetfillopacity{0.700000}%
\pgfsetlinewidth{0.000000pt}%
\definecolor{currentstroke}{rgb}{0.000000,0.000000,0.000000}%
\pgfsetstrokecolor{currentstroke}%
\pgfsetdash{}{0pt}%
\pgfpathmoveto{\pgfqpoint{3.434577in}{1.902427in}}%
\pgfpathlineto{\pgfqpoint{3.448077in}{1.894336in}}%
\pgfpathlineto{\pgfqpoint{3.461581in}{1.886273in}}%
\pgfpathlineto{\pgfqpoint{3.475088in}{1.878236in}}%
\pgfpathlineto{\pgfqpoint{3.488600in}{1.870226in}}%
\pgfpathlineto{\pgfqpoint{3.480317in}{1.879368in}}%
\pgfpathlineto{\pgfqpoint{3.472016in}{1.889037in}}%
\pgfpathlineto{\pgfqpoint{3.463697in}{1.899244in}}%
\pgfpathlineto{\pgfqpoint{3.455358in}{1.910001in}}%
\pgfpathlineto{\pgfqpoint{3.441807in}{1.918400in}}%
\pgfpathlineto{\pgfqpoint{3.428260in}{1.926827in}}%
\pgfpathlineto{\pgfqpoint{3.414716in}{1.935280in}}%
\pgfpathlineto{\pgfqpoint{3.401176in}{1.943761in}}%
\pgfpathlineto{\pgfqpoint{3.409555in}{1.932608in}}%
\pgfpathlineto{\pgfqpoint{3.417915in}{1.922009in}}%
\pgfpathlineto{\pgfqpoint{3.426255in}{1.911952in}}%
\pgfpathlineto{\pgfqpoint{3.434577in}{1.902427in}}%
\pgfpathclose%
\pgfusepath{fill}%
\end{pgfscope}%
\begin{pgfscope}%
\pgfpathrectangle{\pgfqpoint{1.254980in}{0.150000in}}{\pgfqpoint{5.490039in}{5.490039in}}%
\pgfusepath{clip}%
\pgfsetbuttcap%
\pgfsetroundjoin%
\definecolor{currentfill}{rgb}{0.278791,0.062145,0.386592}%
\pgfsetfillcolor{currentfill}%
\pgfsetfillopacity{0.700000}%
\pgfsetlinewidth{0.000000pt}%
\definecolor{currentstroke}{rgb}{0.000000,0.000000,0.000000}%
\pgfsetstrokecolor{currentstroke}%
\pgfsetdash{}{0pt}%
\pgfpathmoveto{\pgfqpoint{4.321565in}{1.478242in}}%
\pgfpathlineto{\pgfqpoint{4.335219in}{1.472938in}}%
\pgfpathlineto{\pgfqpoint{4.348880in}{1.467658in}}%
\pgfpathlineto{\pgfqpoint{4.362546in}{1.462400in}}%
\pgfpathlineto{\pgfqpoint{4.376218in}{1.457166in}}%
\pgfpathlineto{\pgfqpoint{4.368506in}{1.454747in}}%
\pgfpathlineto{\pgfqpoint{4.360788in}{1.452648in}}%
\pgfpathlineto{\pgfqpoint{4.353064in}{1.450876in}}%
\pgfpathlineto{\pgfqpoint{4.345333in}{1.449441in}}%
\pgfpathlineto{\pgfqpoint{4.331642in}{1.455004in}}%
\pgfpathlineto{\pgfqpoint{4.317956in}{1.460590in}}%
\pgfpathlineto{\pgfqpoint{4.304275in}{1.466199in}}%
\pgfpathlineto{\pgfqpoint{4.290600in}{1.471831in}}%
\pgfpathlineto{\pgfqpoint{4.298351in}{1.472933in}}%
\pgfpathlineto{\pgfqpoint{4.306095in}{1.474374in}}%
\pgfpathlineto{\pgfqpoint{4.313833in}{1.476147in}}%
\pgfpathlineto{\pgfqpoint{4.321565in}{1.478242in}}%
\pgfpathclose%
\pgfusepath{fill}%
\end{pgfscope}%
\begin{pgfscope}%
\pgfpathrectangle{\pgfqpoint{1.254980in}{0.150000in}}{\pgfqpoint{5.490039in}{5.490039in}}%
\pgfusepath{clip}%
\pgfsetbuttcap%
\pgfsetroundjoin%
\definecolor{currentfill}{rgb}{0.144759,0.519093,0.556572}%
\pgfsetfillcolor{currentfill}%
\pgfsetfillopacity{0.700000}%
\pgfsetlinewidth{0.000000pt}%
\definecolor{currentstroke}{rgb}{0.000000,0.000000,0.000000}%
\pgfsetstrokecolor{currentstroke}%
\pgfsetdash{}{0pt}%
\pgfpathmoveto{\pgfqpoint{2.664962in}{2.497445in}}%
\pgfpathlineto{\pgfqpoint{2.678414in}{2.486921in}}%
\pgfpathlineto{\pgfqpoint{2.691868in}{2.476435in}}%
\pgfpathlineto{\pgfqpoint{2.705323in}{2.465986in}}%
\pgfpathlineto{\pgfqpoint{2.718780in}{2.455572in}}%
\pgfpathlineto{\pgfqpoint{2.709686in}{2.474729in}}%
\pgfpathlineto{\pgfqpoint{2.700558in}{2.494565in}}%
\pgfpathlineto{\pgfqpoint{2.691395in}{2.515094in}}%
\pgfpathlineto{\pgfqpoint{2.682197in}{2.536330in}}%
\pgfpathlineto{\pgfqpoint{2.668682in}{2.547177in}}%
\pgfpathlineto{\pgfqpoint{2.655169in}{2.558061in}}%
\pgfpathlineto{\pgfqpoint{2.641657in}{2.568981in}}%
\pgfpathlineto{\pgfqpoint{2.628147in}{2.579939in}}%
\pgfpathlineto{\pgfqpoint{2.637405in}{2.558263in}}%
\pgfpathlineto{\pgfqpoint{2.646626in}{2.537297in}}%
\pgfpathlineto{\pgfqpoint{2.655812in}{2.517029in}}%
\pgfpathlineto{\pgfqpoint{2.664962in}{2.497445in}}%
\pgfpathclose%
\pgfusepath{fill}%
\end{pgfscope}%
\begin{pgfscope}%
\pgfpathrectangle{\pgfqpoint{1.254980in}{0.150000in}}{\pgfqpoint{5.490039in}{5.490039in}}%
\pgfusepath{clip}%
\pgfsetbuttcap%
\pgfsetroundjoin%
\definecolor{currentfill}{rgb}{0.273809,0.031497,0.358853}%
\pgfsetfillcolor{currentfill}%
\pgfsetfillopacity{0.700000}%
\pgfsetlinewidth{0.000000pt}%
\definecolor{currentstroke}{rgb}{0.000000,0.000000,0.000000}%
\pgfsetstrokecolor{currentstroke}%
\pgfsetdash{}{0pt}%
\pgfpathmoveto{\pgfqpoint{4.882687in}{1.428424in}}%
\pgfpathlineto{\pgfqpoint{4.896495in}{1.425035in}}%
\pgfpathlineto{\pgfqpoint{4.910310in}{1.421669in}}%
\pgfpathlineto{\pgfqpoint{4.924133in}{1.418326in}}%
\pgfpathlineto{\pgfqpoint{4.937963in}{1.415005in}}%
\pgfpathlineto{\pgfqpoint{4.930421in}{1.406405in}}%
\pgfpathlineto{\pgfqpoint{4.922876in}{1.397974in}}%
\pgfpathlineto{\pgfqpoint{4.915328in}{1.389718in}}%
\pgfpathlineto{\pgfqpoint{4.907778in}{1.381644in}}%
\pgfpathlineto{\pgfqpoint{4.893938in}{1.385242in}}%
\pgfpathlineto{\pgfqpoint{4.880107in}{1.388862in}}%
\pgfpathlineto{\pgfqpoint{4.866282in}{1.392506in}}%
\pgfpathlineto{\pgfqpoint{4.852464in}{1.396171in}}%
\pgfpathlineto{\pgfqpoint{4.860024in}{1.403963in}}%
\pgfpathlineto{\pgfqpoint{4.867581in}{1.411940in}}%
\pgfpathlineto{\pgfqpoint{4.875135in}{1.420096in}}%
\pgfpathlineto{\pgfqpoint{4.882687in}{1.428424in}}%
\pgfpathclose%
\pgfusepath{fill}%
\end{pgfscope}%
\begin{pgfscope}%
\pgfpathrectangle{\pgfqpoint{1.254980in}{0.150000in}}{\pgfqpoint{5.490039in}{5.490039in}}%
\pgfusepath{clip}%
\pgfsetbuttcap%
\pgfsetroundjoin%
\definecolor{currentfill}{rgb}{0.275191,0.194905,0.496005}%
\pgfsetfillcolor{currentfill}%
\pgfsetfillopacity{0.700000}%
\pgfsetlinewidth{0.000000pt}%
\definecolor{currentstroke}{rgb}{0.000000,0.000000,0.000000}%
\pgfsetstrokecolor{currentstroke}%
\pgfsetdash{}{0pt}%
\pgfpathmoveto{\pgfqpoint{3.737714in}{1.720089in}}%
\pgfpathlineto{\pgfqpoint{3.751261in}{1.712893in}}%
\pgfpathlineto{\pgfqpoint{3.764811in}{1.705721in}}%
\pgfpathlineto{\pgfqpoint{3.778366in}{1.698575in}}%
\pgfpathlineto{\pgfqpoint{3.791926in}{1.691453in}}%
\pgfpathlineto{\pgfqpoint{3.783879in}{1.696759in}}%
\pgfpathlineto{\pgfqpoint{3.775820in}{1.702529in}}%
\pgfpathlineto{\pgfqpoint{3.767747in}{1.708775in}}%
\pgfpathlineto{\pgfqpoint{3.759660in}{1.715506in}}%
\pgfpathlineto{\pgfqpoint{3.746068in}{1.723000in}}%
\pgfpathlineto{\pgfqpoint{3.732480in}{1.730519in}}%
\pgfpathlineto{\pgfqpoint{3.718896in}{1.738063in}}%
\pgfpathlineto{\pgfqpoint{3.705317in}{1.745632in}}%
\pgfpathlineto{\pgfqpoint{3.713438in}{1.738522in}}%
\pgfpathlineto{\pgfqpoint{3.721544in}{1.731902in}}%
\pgfpathlineto{\pgfqpoint{3.729636in}{1.725761in}}%
\pgfpathlineto{\pgfqpoint{3.737714in}{1.720089in}}%
\pgfpathclose%
\pgfusepath{fill}%
\end{pgfscope}%
\begin{pgfscope}%
\pgfpathrectangle{\pgfqpoint{1.254980in}{0.150000in}}{\pgfqpoint{5.490039in}{5.490039in}}%
\pgfusepath{clip}%
\pgfsetbuttcap%
\pgfsetroundjoin%
\definecolor{currentfill}{rgb}{0.204903,0.375746,0.553533}%
\pgfsetfillcolor{currentfill}%
\pgfsetfillopacity{0.700000}%
\pgfsetlinewidth{0.000000pt}%
\definecolor{currentstroke}{rgb}{0.000000,0.000000,0.000000}%
\pgfsetstrokecolor{currentstroke}%
\pgfsetdash{}{0pt}%
\pgfpathmoveto{\pgfqpoint{3.131069in}{2.119254in}}%
\pgfpathlineto{\pgfqpoint{3.144545in}{2.110204in}}%
\pgfpathlineto{\pgfqpoint{3.158023in}{2.101185in}}%
\pgfpathlineto{\pgfqpoint{3.171505in}{2.092195in}}%
\pgfpathlineto{\pgfqpoint{3.184989in}{2.083234in}}%
\pgfpathlineto{\pgfqpoint{3.176414in}{2.096558in}}%
\pgfpathlineto{\pgfqpoint{3.167815in}{2.110475in}}%
\pgfpathlineto{\pgfqpoint{3.159191in}{2.124999in}}%
\pgfpathlineto{\pgfqpoint{3.150542in}{2.140140in}}%
\pgfpathlineto{\pgfqpoint{3.137011in}{2.149510in}}%
\pgfpathlineto{\pgfqpoint{3.123482in}{2.158909in}}%
\pgfpathlineto{\pgfqpoint{3.109957in}{2.168338in}}%
\pgfpathlineto{\pgfqpoint{3.096433in}{2.177797in}}%
\pgfpathlineto{\pgfqpoint{3.105131in}{2.162239in}}%
\pgfpathlineto{\pgfqpoint{3.113802in}{2.147305in}}%
\pgfpathlineto{\pgfqpoint{3.122448in}{2.132980in}}%
\pgfpathlineto{\pgfqpoint{3.131069in}{2.119254in}}%
\pgfpathclose%
\pgfusepath{fill}%
\end{pgfscope}%
\begin{pgfscope}%
\pgfpathrectangle{\pgfqpoint{1.254980in}{0.150000in}}{\pgfqpoint{5.490039in}{5.490039in}}%
\pgfusepath{clip}%
\pgfsetbuttcap%
\pgfsetroundjoin%
\definecolor{currentfill}{rgb}{0.277941,0.056324,0.381191}%
\pgfsetfillcolor{currentfill}%
\pgfsetfillopacity{0.700000}%
\pgfsetlinewidth{0.000000pt}%
\definecolor{currentstroke}{rgb}{0.000000,0.000000,0.000000}%
\pgfsetstrokecolor{currentstroke}%
\pgfsetdash{}{0pt}%
\pgfpathmoveto{\pgfqpoint{5.108989in}{1.467449in}}%
\pgfpathlineto{\pgfqpoint{5.122873in}{1.464849in}}%
\pgfpathlineto{\pgfqpoint{5.136764in}{1.462272in}}%
\pgfpathlineto{\pgfqpoint{5.150663in}{1.459718in}}%
\pgfpathlineto{\pgfqpoint{5.164569in}{1.457187in}}%
\pgfpathlineto{\pgfqpoint{5.157069in}{1.446738in}}%
\pgfpathlineto{\pgfqpoint{5.149567in}{1.436394in}}%
\pgfpathlineto{\pgfqpoint{5.142061in}{1.426164in}}%
\pgfpathlineto{\pgfqpoint{5.134554in}{1.416052in}}%
\pgfpathlineto{\pgfqpoint{5.120641in}{1.418836in}}%
\pgfpathlineto{\pgfqpoint{5.106736in}{1.421642in}}%
\pgfpathlineto{\pgfqpoint{5.092839in}{1.424471in}}%
\pgfpathlineto{\pgfqpoint{5.078949in}{1.427323in}}%
\pgfpathlineto{\pgfqpoint{5.086463in}{1.437177in}}%
\pgfpathlineto{\pgfqpoint{5.093975in}{1.447154in}}%
\pgfpathlineto{\pgfqpoint{5.101483in}{1.457246in}}%
\pgfpathlineto{\pgfqpoint{5.108989in}{1.467449in}}%
\pgfpathclose%
\pgfusepath{fill}%
\end{pgfscope}%
\begin{pgfscope}%
\pgfpathrectangle{\pgfqpoint{1.254980in}{0.150000in}}{\pgfqpoint{5.490039in}{5.490039in}}%
\pgfusepath{clip}%
\pgfsetbuttcap%
\pgfsetroundjoin%
\definecolor{currentfill}{rgb}{0.281924,0.089666,0.412415}%
\pgfsetfillcolor{currentfill}%
\pgfsetfillopacity{0.700000}%
\pgfsetlinewidth{0.000000pt}%
\definecolor{currentstroke}{rgb}{0.000000,0.000000,0.000000}%
\pgfsetstrokecolor{currentstroke}%
\pgfsetdash{}{0pt}%
\pgfpathmoveto{\pgfqpoint{4.181404in}{1.517728in}}%
\pgfpathlineto{\pgfqpoint{4.195034in}{1.511909in}}%
\pgfpathlineto{\pgfqpoint{4.208670in}{1.506114in}}%
\pgfpathlineto{\pgfqpoint{4.222311in}{1.500342in}}%
\pgfpathlineto{\pgfqpoint{4.235958in}{1.494593in}}%
\pgfpathlineto{\pgfqpoint{4.228179in}{1.494178in}}%
\pgfpathlineto{\pgfqpoint{4.220392in}{1.494124in}}%
\pgfpathlineto{\pgfqpoint{4.212598in}{1.494442in}}%
\pgfpathlineto{\pgfqpoint{4.204797in}{1.495139in}}%
\pgfpathlineto{\pgfqpoint{4.191127in}{1.501230in}}%
\pgfpathlineto{\pgfqpoint{4.177463in}{1.507345in}}%
\pgfpathlineto{\pgfqpoint{4.163803in}{1.513482in}}%
\pgfpathlineto{\pgfqpoint{4.150150in}{1.519644in}}%
\pgfpathlineto{\pgfqpoint{4.157975in}{1.518599in}}%
\pgfpathlineto{\pgfqpoint{4.165793in}{1.517937in}}%
\pgfpathlineto{\pgfqpoint{4.173602in}{1.517650in}}%
\pgfpathlineto{\pgfqpoint{4.181404in}{1.517728in}}%
\pgfpathclose%
\pgfusepath{fill}%
\end{pgfscope}%
\begin{pgfscope}%
\pgfpathrectangle{\pgfqpoint{1.254980in}{0.150000in}}{\pgfqpoint{5.490039in}{5.490039in}}%
\pgfusepath{clip}%
\pgfsetbuttcap%
\pgfsetroundjoin%
\definecolor{currentfill}{rgb}{0.252194,0.269783,0.531579}%
\pgfsetfillcolor{currentfill}%
\pgfsetfillopacity{0.700000}%
\pgfsetlinewidth{0.000000pt}%
\definecolor{currentstroke}{rgb}{0.000000,0.000000,0.000000}%
\pgfsetstrokecolor{currentstroke}%
\pgfsetdash{}{0pt}%
\pgfpathmoveto{\pgfqpoint{3.488600in}{1.870226in}}%
\pgfpathlineto{\pgfqpoint{3.502115in}{1.862243in}}%
\pgfpathlineto{\pgfqpoint{3.515634in}{1.854287in}}%
\pgfpathlineto{\pgfqpoint{3.529157in}{1.846357in}}%
\pgfpathlineto{\pgfqpoint{3.542684in}{1.838453in}}%
\pgfpathlineto{\pgfqpoint{3.534439in}{1.847211in}}%
\pgfpathlineto{\pgfqpoint{3.526177in}{1.856493in}}%
\pgfpathlineto{\pgfqpoint{3.517897in}{1.866309in}}%
\pgfpathlineto{\pgfqpoint{3.509598in}{1.876670in}}%
\pgfpathlineto{\pgfqpoint{3.496033in}{1.884963in}}%
\pgfpathlineto{\pgfqpoint{3.482471in}{1.893282in}}%
\pgfpathlineto{\pgfqpoint{3.468913in}{1.901628in}}%
\pgfpathlineto{\pgfqpoint{3.455358in}{1.910001in}}%
\pgfpathlineto{\pgfqpoint{3.463697in}{1.899244in}}%
\pgfpathlineto{\pgfqpoint{3.472016in}{1.889037in}}%
\pgfpathlineto{\pgfqpoint{3.480317in}{1.879368in}}%
\pgfpathlineto{\pgfqpoint{3.488600in}{1.870226in}}%
\pgfpathclose%
\pgfusepath{fill}%
\end{pgfscope}%
\begin{pgfscope}%
\pgfpathrectangle{\pgfqpoint{1.254980in}{0.150000in}}{\pgfqpoint{5.490039in}{5.490039in}}%
\pgfusepath{clip}%
\pgfsetbuttcap%
\pgfsetroundjoin%
\definecolor{currentfill}{rgb}{0.283072,0.130895,0.449241}%
\pgfsetfillcolor{currentfill}%
\pgfsetfillopacity{0.700000}%
\pgfsetlinewidth{0.000000pt}%
\definecolor{currentstroke}{rgb}{0.000000,0.000000,0.000000}%
\pgfsetstrokecolor{currentstroke}%
\pgfsetdash{}{0pt}%
\pgfpathmoveto{\pgfqpoint{3.986709in}{1.595425in}}%
\pgfpathlineto{\pgfqpoint{4.000301in}{1.588979in}}%
\pgfpathlineto{\pgfqpoint{4.013898in}{1.582557in}}%
\pgfpathlineto{\pgfqpoint{4.027500in}{1.576158in}}%
\pgfpathlineto{\pgfqpoint{4.041107in}{1.569784in}}%
\pgfpathlineto{\pgfqpoint{4.033222in}{1.571925in}}%
\pgfpathlineto{\pgfqpoint{4.025327in}{1.574476in}}%
\pgfpathlineto{\pgfqpoint{4.017422in}{1.577447in}}%
\pgfpathlineto{\pgfqpoint{4.009507in}{1.580847in}}%
\pgfpathlineto{\pgfqpoint{3.995873in}{1.587578in}}%
\pgfpathlineto{\pgfqpoint{3.982243in}{1.594334in}}%
\pgfpathlineto{\pgfqpoint{3.968618in}{1.601113in}}%
\pgfpathlineto{\pgfqpoint{3.954998in}{1.607916in}}%
\pgfpathlineto{\pgfqpoint{3.962942in}{1.604153in}}%
\pgfpathlineto{\pgfqpoint{3.970874in}{1.600824in}}%
\pgfpathlineto{\pgfqpoint{3.978797in}{1.597918in}}%
\pgfpathlineto{\pgfqpoint{3.986709in}{1.595425in}}%
\pgfpathclose%
\pgfusepath{fill}%
\end{pgfscope}%
\begin{pgfscope}%
\pgfpathrectangle{\pgfqpoint{1.254980in}{0.150000in}}{\pgfqpoint{5.490039in}{5.490039in}}%
\pgfusepath{clip}%
\pgfsetbuttcap%
\pgfsetroundjoin%
\definecolor{currentfill}{rgb}{0.150476,0.504369,0.557430}%
\pgfsetfillcolor{currentfill}%
\pgfsetfillopacity{0.700000}%
\pgfsetlinewidth{0.000000pt}%
\definecolor{currentstroke}{rgb}{0.000000,0.000000,0.000000}%
\pgfsetstrokecolor{currentstroke}%
\pgfsetdash{}{0pt}%
\pgfpathmoveto{\pgfqpoint{2.718780in}{2.455572in}}%
\pgfpathlineto{\pgfqpoint{2.732239in}{2.445195in}}%
\pgfpathlineto{\pgfqpoint{2.745700in}{2.434854in}}%
\pgfpathlineto{\pgfqpoint{2.759162in}{2.424549in}}%
\pgfpathlineto{\pgfqpoint{2.772627in}{2.414279in}}%
\pgfpathlineto{\pgfqpoint{2.763588in}{2.433009in}}%
\pgfpathlineto{\pgfqpoint{2.754516in}{2.452414in}}%
\pgfpathlineto{\pgfqpoint{2.745411in}{2.472508in}}%
\pgfpathlineto{\pgfqpoint{2.736271in}{2.493302in}}%
\pgfpathlineto{\pgfqpoint{2.722750in}{2.504006in}}%
\pgfpathlineto{\pgfqpoint{2.709231in}{2.514744in}}%
\pgfpathlineto{\pgfqpoint{2.695713in}{2.525519in}}%
\pgfpathlineto{\pgfqpoint{2.682197in}{2.536330in}}%
\pgfpathlineto{\pgfqpoint{2.691395in}{2.515094in}}%
\pgfpathlineto{\pgfqpoint{2.700558in}{2.494565in}}%
\pgfpathlineto{\pgfqpoint{2.709686in}{2.474729in}}%
\pgfpathlineto{\pgfqpoint{2.718780in}{2.455572in}}%
\pgfpathclose%
\pgfusepath{fill}%
\end{pgfscope}%
\begin{pgfscope}%
\pgfpathrectangle{\pgfqpoint{1.254980in}{0.150000in}}{\pgfqpoint{5.490039in}{5.490039in}}%
\pgfusepath{clip}%
\pgfsetbuttcap%
\pgfsetroundjoin%
\definecolor{currentfill}{rgb}{0.273809,0.031497,0.358853}%
\pgfsetfillcolor{currentfill}%
\pgfsetfillopacity{0.700000}%
\pgfsetlinewidth{0.000000pt}%
\definecolor{currentstroke}{rgb}{0.000000,0.000000,0.000000}%
\pgfsetstrokecolor{currentstroke}%
\pgfsetdash{}{0pt}%
\pgfpathmoveto{\pgfqpoint{4.656788in}{1.416243in}}%
\pgfpathlineto{\pgfqpoint{4.670535in}{1.412015in}}%
\pgfpathlineto{\pgfqpoint{4.684288in}{1.407810in}}%
\pgfpathlineto{\pgfqpoint{4.698049in}{1.403628in}}%
\pgfpathlineto{\pgfqpoint{4.711816in}{1.399468in}}%
\pgfpathlineto{\pgfqpoint{4.704217in}{1.393311in}}%
\pgfpathlineto{\pgfqpoint{4.696616in}{1.387391in}}%
\pgfpathlineto{\pgfqpoint{4.689011in}{1.381716in}}%
\pgfpathlineto{\pgfqpoint{4.681403in}{1.376293in}}%
\pgfpathlineto{\pgfqpoint{4.667622in}{1.380755in}}%
\pgfpathlineto{\pgfqpoint{4.653849in}{1.385240in}}%
\pgfpathlineto{\pgfqpoint{4.640081in}{1.389747in}}%
\pgfpathlineto{\pgfqpoint{4.626321in}{1.394277in}}%
\pgfpathlineto{\pgfqpoint{4.633943in}{1.399392in}}%
\pgfpathlineto{\pgfqpoint{4.641562in}{1.404764in}}%
\pgfpathlineto{\pgfqpoint{4.649176in}{1.410383in}}%
\pgfpathlineto{\pgfqpoint{4.656788in}{1.416243in}}%
\pgfpathclose%
\pgfusepath{fill}%
\end{pgfscope}%
\begin{pgfscope}%
\pgfpathrectangle{\pgfqpoint{1.254980in}{0.150000in}}{\pgfqpoint{5.490039in}{5.490039in}}%
\pgfusepath{clip}%
\pgfsetbuttcap%
\pgfsetroundjoin%
\definecolor{currentfill}{rgb}{0.274952,0.037752,0.364543}%
\pgfsetfillcolor{currentfill}%
\pgfsetfillopacity{0.700000}%
\pgfsetlinewidth{0.000000pt}%
\definecolor{currentstroke}{rgb}{0.000000,0.000000,0.000000}%
\pgfsetstrokecolor{currentstroke}%
\pgfsetdash{}{0pt}%
\pgfpathmoveto{\pgfqpoint{4.516465in}{1.431337in}}%
\pgfpathlineto{\pgfqpoint{4.530175in}{1.426624in}}%
\pgfpathlineto{\pgfqpoint{4.543891in}{1.421935in}}%
\pgfpathlineto{\pgfqpoint{4.557613in}{1.417268in}}%
\pgfpathlineto{\pgfqpoint{4.571342in}{1.412625in}}%
\pgfpathlineto{\pgfqpoint{4.563701in}{1.408084in}}%
\pgfpathlineto{\pgfqpoint{4.556055in}{1.403818in}}%
\pgfpathlineto{\pgfqpoint{4.548406in}{1.399836in}}%
\pgfpathlineto{\pgfqpoint{4.540751in}{1.396146in}}%
\pgfpathlineto{\pgfqpoint{4.527007in}{1.401105in}}%
\pgfpathlineto{\pgfqpoint{4.513268in}{1.406087in}}%
\pgfpathlineto{\pgfqpoint{4.499536in}{1.411092in}}%
\pgfpathlineto{\pgfqpoint{4.485810in}{1.416120in}}%
\pgfpathlineto{\pgfqpoint{4.493481in}{1.419489in}}%
\pgfpathlineto{\pgfqpoint{4.501147in}{1.423154in}}%
\pgfpathlineto{\pgfqpoint{4.508809in}{1.427106in}}%
\pgfpathlineto{\pgfqpoint{4.516465in}{1.431337in}}%
\pgfpathclose%
\pgfusepath{fill}%
\end{pgfscope}%
\begin{pgfscope}%
\pgfpathrectangle{\pgfqpoint{1.254980in}{0.150000in}}{\pgfqpoint{5.490039in}{5.490039in}}%
\pgfusepath{clip}%
\pgfsetbuttcap%
\pgfsetroundjoin%
\definecolor{currentfill}{rgb}{0.276022,0.044167,0.370164}%
\pgfsetfillcolor{currentfill}%
\pgfsetfillopacity{0.700000}%
\pgfsetlinewidth{0.000000pt}%
\definecolor{currentstroke}{rgb}{0.000000,0.000000,0.000000}%
\pgfsetstrokecolor{currentstroke}%
\pgfsetdash{}{0pt}%
\pgfpathmoveto{\pgfqpoint{5.023467in}{1.438957in}}%
\pgfpathlineto{\pgfqpoint{5.037326in}{1.436014in}}%
\pgfpathlineto{\pgfqpoint{5.051193in}{1.433094in}}%
\pgfpathlineto{\pgfqpoint{5.065067in}{1.430197in}}%
\pgfpathlineto{\pgfqpoint{5.078949in}{1.427323in}}%
\pgfpathlineto{\pgfqpoint{5.071432in}{1.417597in}}%
\pgfpathlineto{\pgfqpoint{5.063913in}{1.408006in}}%
\pgfpathlineto{\pgfqpoint{5.056392in}{1.398557in}}%
\pgfpathlineto{\pgfqpoint{5.048868in}{1.389256in}}%
\pgfpathlineto{\pgfqpoint{5.034979in}{1.392396in}}%
\pgfpathlineto{\pgfqpoint{5.021097in}{1.395558in}}%
\pgfpathlineto{\pgfqpoint{5.007223in}{1.398742in}}%
\pgfpathlineto{\pgfqpoint{4.993356in}{1.401950in}}%
\pgfpathlineto{\pgfqpoint{5.000888in}{1.410981in}}%
\pgfpathlineto{\pgfqpoint{5.008417in}{1.420163in}}%
\pgfpathlineto{\pgfqpoint{5.015943in}{1.429491in}}%
\pgfpathlineto{\pgfqpoint{5.023467in}{1.438957in}}%
\pgfpathclose%
\pgfusepath{fill}%
\end{pgfscope}%
\begin{pgfscope}%
\pgfpathrectangle{\pgfqpoint{1.254980in}{0.150000in}}{\pgfqpoint{5.490039in}{5.490039in}}%
\pgfusepath{clip}%
\pgfsetbuttcap%
\pgfsetroundjoin%
\definecolor{currentfill}{rgb}{0.272594,0.025563,0.353093}%
\pgfsetfillcolor{currentfill}%
\pgfsetfillopacity{0.700000}%
\pgfsetlinewidth{0.000000pt}%
\definecolor{currentstroke}{rgb}{0.000000,0.000000,0.000000}%
\pgfsetstrokecolor{currentstroke}%
\pgfsetdash{}{0pt}%
\pgfpathmoveto{\pgfqpoint{4.797263in}{1.411061in}}%
\pgfpathlineto{\pgfqpoint{4.811053in}{1.407304in}}%
\pgfpathlineto{\pgfqpoint{4.824850in}{1.403571in}}%
\pgfpathlineto{\pgfqpoint{4.838653in}{1.399860in}}%
\pgfpathlineto{\pgfqpoint{4.852464in}{1.396171in}}%
\pgfpathlineto{\pgfqpoint{4.844901in}{1.388573in}}%
\pgfpathlineto{\pgfqpoint{4.837335in}{1.381175in}}%
\pgfpathlineto{\pgfqpoint{4.829767in}{1.373984in}}%
\pgfpathlineto{\pgfqpoint{4.822196in}{1.367008in}}%
\pgfpathlineto{\pgfqpoint{4.808374in}{1.370987in}}%
\pgfpathlineto{\pgfqpoint{4.794560in}{1.374988in}}%
\pgfpathlineto{\pgfqpoint{4.780752in}{1.379011in}}%
\pgfpathlineto{\pgfqpoint{4.766951in}{1.383057in}}%
\pgfpathlineto{\pgfqpoint{4.774534in}{1.389738in}}%
\pgfpathlineto{\pgfqpoint{4.782113in}{1.396637in}}%
\pgfpathlineto{\pgfqpoint{4.789690in}{1.403747in}}%
\pgfpathlineto{\pgfqpoint{4.797263in}{1.411061in}}%
\pgfpathclose%
\pgfusepath{fill}%
\end{pgfscope}%
\begin{pgfscope}%
\pgfpathrectangle{\pgfqpoint{1.254980in}{0.150000in}}{\pgfqpoint{5.490039in}{5.490039in}}%
\pgfusepath{clip}%
\pgfsetbuttcap%
\pgfsetroundjoin%
\definecolor{currentfill}{rgb}{0.210503,0.363727,0.552206}%
\pgfsetfillcolor{currentfill}%
\pgfsetfillopacity{0.700000}%
\pgfsetlinewidth{0.000000pt}%
\definecolor{currentstroke}{rgb}{0.000000,0.000000,0.000000}%
\pgfsetstrokecolor{currentstroke}%
\pgfsetdash{}{0pt}%
\pgfpathmoveto{\pgfqpoint{3.184989in}{2.083234in}}%
\pgfpathlineto{\pgfqpoint{3.198477in}{2.074303in}}%
\pgfpathlineto{\pgfqpoint{3.211967in}{2.065401in}}%
\pgfpathlineto{\pgfqpoint{3.225461in}{2.056528in}}%
\pgfpathlineto{\pgfqpoint{3.238958in}{2.047684in}}%
\pgfpathlineto{\pgfqpoint{3.230428in}{2.060606in}}%
\pgfpathlineto{\pgfqpoint{3.221875in}{2.074117in}}%
\pgfpathlineto{\pgfqpoint{3.213298in}{2.088229in}}%
\pgfpathlineto{\pgfqpoint{3.204697in}{2.102956in}}%
\pgfpathlineto{\pgfqpoint{3.191154in}{2.112208in}}%
\pgfpathlineto{\pgfqpoint{3.177614in}{2.121490in}}%
\pgfpathlineto{\pgfqpoint{3.164077in}{2.130800in}}%
\pgfpathlineto{\pgfqpoint{3.150542in}{2.140140in}}%
\pgfpathlineto{\pgfqpoint{3.159191in}{2.124999in}}%
\pgfpathlineto{\pgfqpoint{3.167815in}{2.110475in}}%
\pgfpathlineto{\pgfqpoint{3.176414in}{2.096558in}}%
\pgfpathlineto{\pgfqpoint{3.184989in}{2.083234in}}%
\pgfpathclose%
\pgfusepath{fill}%
\end{pgfscope}%
\begin{pgfscope}%
\pgfpathrectangle{\pgfqpoint{1.254980in}{0.150000in}}{\pgfqpoint{5.490039in}{5.490039in}}%
\pgfusepath{clip}%
\pgfsetbuttcap%
\pgfsetroundjoin%
\definecolor{currentfill}{rgb}{0.278791,0.062145,0.386592}%
\pgfsetfillcolor{currentfill}%
\pgfsetfillopacity{0.700000}%
\pgfsetlinewidth{0.000000pt}%
\definecolor{currentstroke}{rgb}{0.000000,0.000000,0.000000}%
\pgfsetstrokecolor{currentstroke}%
\pgfsetdash{}{0pt}%
\pgfpathmoveto{\pgfqpoint{4.376218in}{1.457166in}}%
\pgfpathlineto{\pgfqpoint{4.389896in}{1.451955in}}%
\pgfpathlineto{\pgfqpoint{4.403580in}{1.446767in}}%
\pgfpathlineto{\pgfqpoint{4.417270in}{1.441601in}}%
\pgfpathlineto{\pgfqpoint{4.430966in}{1.436459in}}%
\pgfpathlineto{\pgfqpoint{4.423272in}{1.433717in}}%
\pgfpathlineto{\pgfqpoint{4.415573in}{1.431290in}}%
\pgfpathlineto{\pgfqpoint{4.407868in}{1.429188in}}%
\pgfpathlineto{\pgfqpoint{4.400158in}{1.427419in}}%
\pgfpathlineto{\pgfqpoint{4.386443in}{1.432890in}}%
\pgfpathlineto{\pgfqpoint{4.372734in}{1.438384in}}%
\pgfpathlineto{\pgfqpoint{4.359031in}{1.443901in}}%
\pgfpathlineto{\pgfqpoint{4.345333in}{1.449441in}}%
\pgfpathlineto{\pgfqpoint{4.353064in}{1.450876in}}%
\pgfpathlineto{\pgfqpoint{4.360788in}{1.452648in}}%
\pgfpathlineto{\pgfqpoint{4.368506in}{1.454747in}}%
\pgfpathlineto{\pgfqpoint{4.376218in}{1.457166in}}%
\pgfpathclose%
\pgfusepath{fill}%
\end{pgfscope}%
\begin{pgfscope}%
\pgfpathrectangle{\pgfqpoint{1.254980in}{0.150000in}}{\pgfqpoint{5.490039in}{5.490039in}}%
\pgfusepath{clip}%
\pgfsetbuttcap%
\pgfsetroundjoin%
\definecolor{currentfill}{rgb}{0.277134,0.185228,0.489898}%
\pgfsetfillcolor{currentfill}%
\pgfsetfillopacity{0.700000}%
\pgfsetlinewidth{0.000000pt}%
\definecolor{currentstroke}{rgb}{0.000000,0.000000,0.000000}%
\pgfsetstrokecolor{currentstroke}%
\pgfsetdash{}{0pt}%
\pgfpathmoveto{\pgfqpoint{3.791926in}{1.691453in}}%
\pgfpathlineto{\pgfqpoint{3.805490in}{1.684357in}}%
\pgfpathlineto{\pgfqpoint{3.819058in}{1.677285in}}%
\pgfpathlineto{\pgfqpoint{3.832631in}{1.670238in}}%
\pgfpathlineto{\pgfqpoint{3.846209in}{1.663215in}}%
\pgfpathlineto{\pgfqpoint{3.838194in}{1.668154in}}%
\pgfpathlineto{\pgfqpoint{3.830167in}{1.673554in}}%
\pgfpathlineto{\pgfqpoint{3.822127in}{1.679426in}}%
\pgfpathlineto{\pgfqpoint{3.814073in}{1.685779in}}%
\pgfpathlineto{\pgfqpoint{3.800463in}{1.693174in}}%
\pgfpathlineto{\pgfqpoint{3.786858in}{1.700593in}}%
\pgfpathlineto{\pgfqpoint{3.773257in}{1.708037in}}%
\pgfpathlineto{\pgfqpoint{3.759660in}{1.715506in}}%
\pgfpathlineto{\pgfqpoint{3.767747in}{1.708775in}}%
\pgfpathlineto{\pgfqpoint{3.775820in}{1.702529in}}%
\pgfpathlineto{\pgfqpoint{3.783879in}{1.696759in}}%
\pgfpathlineto{\pgfqpoint{3.791926in}{1.691453in}}%
\pgfpathclose%
\pgfusepath{fill}%
\end{pgfscope}%
\begin{pgfscope}%
\pgfpathrectangle{\pgfqpoint{1.254980in}{0.150000in}}{\pgfqpoint{5.490039in}{5.490039in}}%
\pgfusepath{clip}%
\pgfsetbuttcap%
\pgfsetroundjoin%
\definecolor{currentfill}{rgb}{0.154815,0.493313,0.557840}%
\pgfsetfillcolor{currentfill}%
\pgfsetfillopacity{0.700000}%
\pgfsetlinewidth{0.000000pt}%
\definecolor{currentstroke}{rgb}{0.000000,0.000000,0.000000}%
\pgfsetstrokecolor{currentstroke}%
\pgfsetdash{}{0pt}%
\pgfpathmoveto{\pgfqpoint{2.772627in}{2.414279in}}%
\pgfpathlineto{\pgfqpoint{2.786093in}{2.404044in}}%
\pgfpathlineto{\pgfqpoint{2.799561in}{2.393843in}}%
\pgfpathlineto{\pgfqpoint{2.813032in}{2.383678in}}%
\pgfpathlineto{\pgfqpoint{2.826504in}{2.373546in}}%
\pgfpathlineto{\pgfqpoint{2.817520in}{2.391851in}}%
\pgfpathlineto{\pgfqpoint{2.808504in}{2.410827in}}%
\pgfpathlineto{\pgfqpoint{2.799455in}{2.430486in}}%
\pgfpathlineto{\pgfqpoint{2.790373in}{2.450841in}}%
\pgfpathlineto{\pgfqpoint{2.776845in}{2.461404in}}%
\pgfpathlineto{\pgfqpoint{2.763319in}{2.472002in}}%
\pgfpathlineto{\pgfqpoint{2.749794in}{2.482635in}}%
\pgfpathlineto{\pgfqpoint{2.736271in}{2.493302in}}%
\pgfpathlineto{\pgfqpoint{2.745411in}{2.472508in}}%
\pgfpathlineto{\pgfqpoint{2.754516in}{2.452414in}}%
\pgfpathlineto{\pgfqpoint{2.763588in}{2.433009in}}%
\pgfpathlineto{\pgfqpoint{2.772627in}{2.414279in}}%
\pgfpathclose%
\pgfusepath{fill}%
\end{pgfscope}%
\begin{pgfscope}%
\pgfpathrectangle{\pgfqpoint{1.254980in}{0.150000in}}{\pgfqpoint{5.490039in}{5.490039in}}%
\pgfusepath{clip}%
\pgfsetbuttcap%
\pgfsetroundjoin%
\definecolor{currentfill}{rgb}{0.220124,0.725509,0.466226}%
\pgfsetfillcolor{currentfill}%
\pgfsetfillopacity{0.700000}%
\pgfsetlinewidth{0.000000pt}%
\definecolor{currentstroke}{rgb}{0.000000,0.000000,0.000000}%
\pgfsetstrokecolor{currentstroke}%
\pgfsetdash{}{0pt}%
\pgfpathmoveto{\pgfqpoint{2.088419in}{3.052855in}}%
\pgfpathlineto{\pgfqpoint{2.101909in}{3.040099in}}%
\pgfpathlineto{\pgfqpoint{2.115399in}{3.027398in}}%
\pgfpathlineto{\pgfqpoint{2.128888in}{3.014750in}}%
\pgfpathlineto{\pgfqpoint{2.142376in}{3.002156in}}%
\pgfpathlineto{\pgfqpoint{2.132495in}{3.028638in}}%
\pgfpathlineto{\pgfqpoint{2.122565in}{3.055905in}}%
\pgfpathlineto{\pgfqpoint{2.112585in}{3.083972in}}%
\pgfpathlineto{\pgfqpoint{2.102555in}{3.112852in}}%
\pgfpathlineto{\pgfqpoint{2.088995in}{3.125916in}}%
\pgfpathlineto{\pgfqpoint{2.075434in}{3.139033in}}%
\pgfpathlineto{\pgfqpoint{2.061873in}{3.152205in}}%
\pgfpathlineto{\pgfqpoint{2.048310in}{3.165431in}}%
\pgfpathlineto{\pgfqpoint{2.058414in}{3.136073in}}%
\pgfpathlineto{\pgfqpoint{2.068465in}{3.107534in}}%
\pgfpathlineto{\pgfqpoint{2.078467in}{3.079799in}}%
\pgfpathlineto{\pgfqpoint{2.088419in}{3.052855in}}%
\pgfpathclose%
\pgfusepath{fill}%
\end{pgfscope}%
\begin{pgfscope}%
\pgfpathrectangle{\pgfqpoint{1.254980in}{0.150000in}}{\pgfqpoint{5.490039in}{5.490039in}}%
\pgfusepath{clip}%
\pgfsetbuttcap%
\pgfsetroundjoin%
\definecolor{currentfill}{rgb}{0.255645,0.260703,0.528312}%
\pgfsetfillcolor{currentfill}%
\pgfsetfillopacity{0.700000}%
\pgfsetlinewidth{0.000000pt}%
\definecolor{currentstroke}{rgb}{0.000000,0.000000,0.000000}%
\pgfsetstrokecolor{currentstroke}%
\pgfsetdash{}{0pt}%
\pgfpathmoveto{\pgfqpoint{3.542684in}{1.838453in}}%
\pgfpathlineto{\pgfqpoint{3.556214in}{1.830576in}}%
\pgfpathlineto{\pgfqpoint{3.569749in}{1.822724in}}%
\pgfpathlineto{\pgfqpoint{3.583287in}{1.814899in}}%
\pgfpathlineto{\pgfqpoint{3.596830in}{1.807100in}}%
\pgfpathlineto{\pgfqpoint{3.588622in}{1.815475in}}%
\pgfpathlineto{\pgfqpoint{3.580398in}{1.824370in}}%
\pgfpathlineto{\pgfqpoint{3.572157in}{1.833795in}}%
\pgfpathlineto{\pgfqpoint{3.563898in}{1.843761in}}%
\pgfpathlineto{\pgfqpoint{3.550317in}{1.851949in}}%
\pgfpathlineto{\pgfqpoint{3.536741in}{1.860163in}}%
\pgfpathlineto{\pgfqpoint{3.523168in}{1.868403in}}%
\pgfpathlineto{\pgfqpoint{3.509598in}{1.876670in}}%
\pgfpathlineto{\pgfqpoint{3.517897in}{1.866309in}}%
\pgfpathlineto{\pgfqpoint{3.526177in}{1.856493in}}%
\pgfpathlineto{\pgfqpoint{3.534439in}{1.847211in}}%
\pgfpathlineto{\pgfqpoint{3.542684in}{1.838453in}}%
\pgfpathclose%
\pgfusepath{fill}%
\end{pgfscope}%
\begin{pgfscope}%
\pgfpathrectangle{\pgfqpoint{1.254980in}{0.150000in}}{\pgfqpoint{5.490039in}{5.490039in}}%
\pgfusepath{clip}%
\pgfsetbuttcap%
\pgfsetroundjoin%
\definecolor{currentfill}{rgb}{0.273809,0.031497,0.358853}%
\pgfsetfillcolor{currentfill}%
\pgfsetfillopacity{0.700000}%
\pgfsetlinewidth{0.000000pt}%
\definecolor{currentstroke}{rgb}{0.000000,0.000000,0.000000}%
\pgfsetstrokecolor{currentstroke}%
\pgfsetdash{}{0pt}%
\pgfpathmoveto{\pgfqpoint{4.937963in}{1.415005in}}%
\pgfpathlineto{\pgfqpoint{4.951800in}{1.411707in}}%
\pgfpathlineto{\pgfqpoint{4.965645in}{1.408432in}}%
\pgfpathlineto{\pgfqpoint{4.979497in}{1.405179in}}%
\pgfpathlineto{\pgfqpoint{4.993356in}{1.401950in}}%
\pgfpathlineto{\pgfqpoint{4.985822in}{1.393077in}}%
\pgfpathlineto{\pgfqpoint{4.978286in}{1.384370in}}%
\pgfpathlineto{\pgfqpoint{4.970747in}{1.375835in}}%
\pgfpathlineto{\pgfqpoint{4.963206in}{1.367479in}}%
\pgfpathlineto{\pgfqpoint{4.949338in}{1.370986in}}%
\pgfpathlineto{\pgfqpoint{4.935477in}{1.374516in}}%
\pgfpathlineto{\pgfqpoint{4.921624in}{1.378069in}}%
\pgfpathlineto{\pgfqpoint{4.907778in}{1.381644in}}%
\pgfpathlineto{\pgfqpoint{4.915328in}{1.389718in}}%
\pgfpathlineto{\pgfqpoint{4.922876in}{1.397974in}}%
\pgfpathlineto{\pgfqpoint{4.930421in}{1.406405in}}%
\pgfpathlineto{\pgfqpoint{4.937963in}{1.415005in}}%
\pgfpathclose%
\pgfusepath{fill}%
\end{pgfscope}%
\begin{pgfscope}%
\pgfpathrectangle{\pgfqpoint{1.254980in}{0.150000in}}{\pgfqpoint{5.490039in}{5.490039in}}%
\pgfusepath{clip}%
\pgfsetbuttcap%
\pgfsetroundjoin%
\definecolor{currentfill}{rgb}{0.277941,0.056324,0.381191}%
\pgfsetfillcolor{currentfill}%
\pgfsetfillopacity{0.700000}%
\pgfsetlinewidth{0.000000pt}%
\definecolor{currentstroke}{rgb}{0.000000,0.000000,0.000000}%
\pgfsetstrokecolor{currentstroke}%
\pgfsetdash{}{0pt}%
\pgfpathmoveto{\pgfqpoint{5.164569in}{1.457187in}}%
\pgfpathlineto{\pgfqpoint{5.178484in}{1.454679in}}%
\pgfpathlineto{\pgfqpoint{5.192407in}{1.452193in}}%
\pgfpathlineto{\pgfqpoint{5.206337in}{1.449731in}}%
\pgfpathlineto{\pgfqpoint{5.198841in}{1.439095in}}%
\pgfpathlineto{\pgfqpoint{5.191343in}{1.428564in}}%
\pgfpathlineto{\pgfqpoint{5.183842in}{1.418142in}}%
\pgfpathlineto{\pgfqpoint{5.176339in}{1.407838in}}%
\pgfpathlineto{\pgfqpoint{5.162403in}{1.410553in}}%
\pgfpathlineto{\pgfqpoint{5.148474in}{1.413291in}}%
\pgfpathlineto{\pgfqpoint{5.134554in}{1.416052in}}%
\pgfpathlineto{\pgfqpoint{5.142061in}{1.426164in}}%
\pgfpathlineto{\pgfqpoint{5.149567in}{1.436394in}}%
\pgfpathlineto{\pgfqpoint{5.157069in}{1.446738in}}%
\pgfpathlineto{\pgfqpoint{5.164569in}{1.457187in}}%
\pgfpathclose%
\pgfusepath{fill}%
\end{pgfscope}%
\begin{pgfscope}%
\pgfpathrectangle{\pgfqpoint{1.254980in}{0.150000in}}{\pgfqpoint{5.490039in}{5.490039in}}%
\pgfusepath{clip}%
\pgfsetbuttcap%
\pgfsetroundjoin%
\definecolor{currentfill}{rgb}{0.191090,0.708366,0.482284}%
\pgfsetfillcolor{currentfill}%
\pgfsetfillopacity{0.700000}%
\pgfsetlinewidth{0.000000pt}%
\definecolor{currentstroke}{rgb}{0.000000,0.000000,0.000000}%
\pgfsetstrokecolor{currentstroke}%
\pgfsetdash{}{0pt}%
\pgfpathmoveto{\pgfqpoint{2.142376in}{3.002156in}}%
\pgfpathlineto{\pgfqpoint{2.155865in}{2.989613in}}%
\pgfpathlineto{\pgfqpoint{2.169353in}{2.977123in}}%
\pgfpathlineto{\pgfqpoint{2.182841in}{2.964684in}}%
\pgfpathlineto{\pgfqpoint{2.196328in}{2.952296in}}%
\pgfpathlineto{\pgfqpoint{2.186515in}{2.978319in}}%
\pgfpathlineto{\pgfqpoint{2.176656in}{3.005120in}}%
\pgfpathlineto{\pgfqpoint{2.166748in}{3.032716in}}%
\pgfpathlineto{\pgfqpoint{2.156791in}{3.061120in}}%
\pgfpathlineto{\pgfqpoint{2.143232in}{3.073976in}}%
\pgfpathlineto{\pgfqpoint{2.129674in}{3.086883in}}%
\pgfpathlineto{\pgfqpoint{2.116115in}{3.099841in}}%
\pgfpathlineto{\pgfqpoint{2.102555in}{3.112852in}}%
\pgfpathlineto{\pgfqpoint{2.112585in}{3.083972in}}%
\pgfpathlineto{\pgfqpoint{2.122565in}{3.055905in}}%
\pgfpathlineto{\pgfqpoint{2.132495in}{3.028638in}}%
\pgfpathlineto{\pgfqpoint{2.142376in}{3.002156in}}%
\pgfpathclose%
\pgfusepath{fill}%
\end{pgfscope}%
\begin{pgfscope}%
\pgfpathrectangle{\pgfqpoint{1.254980in}{0.150000in}}{\pgfqpoint{5.490039in}{5.490039in}}%
\pgfusepath{clip}%
\pgfsetbuttcap%
\pgfsetroundjoin%
\definecolor{currentfill}{rgb}{0.283187,0.125848,0.444960}%
\pgfsetfillcolor{currentfill}%
\pgfsetfillopacity{0.700000}%
\pgfsetlinewidth{0.000000pt}%
\definecolor{currentstroke}{rgb}{0.000000,0.000000,0.000000}%
\pgfsetstrokecolor{currentstroke}%
\pgfsetdash{}{0pt}%
\pgfpathmoveto{\pgfqpoint{4.041107in}{1.569784in}}%
\pgfpathlineto{\pgfqpoint{4.054719in}{1.563433in}}%
\pgfpathlineto{\pgfqpoint{4.068337in}{1.557106in}}%
\pgfpathlineto{\pgfqpoint{4.081959in}{1.550804in}}%
\pgfpathlineto{\pgfqpoint{4.095587in}{1.544524in}}%
\pgfpathlineto{\pgfqpoint{4.087728in}{1.546314in}}%
\pgfpathlineto{\pgfqpoint{4.079860in}{1.548510in}}%
\pgfpathlineto{\pgfqpoint{4.071982in}{1.551122in}}%
\pgfpathlineto{\pgfqpoint{4.064095in}{1.554160in}}%
\pgfpathlineto{\pgfqpoint{4.050440in}{1.560796in}}%
\pgfpathlineto{\pgfqpoint{4.036791in}{1.567456in}}%
\pgfpathlineto{\pgfqpoint{4.023147in}{1.574140in}}%
\pgfpathlineto{\pgfqpoint{4.009507in}{1.580847in}}%
\pgfpathlineto{\pgfqpoint{4.017422in}{1.577447in}}%
\pgfpathlineto{\pgfqpoint{4.025327in}{1.574476in}}%
\pgfpathlineto{\pgfqpoint{4.033222in}{1.571925in}}%
\pgfpathlineto{\pgfqpoint{4.041107in}{1.569784in}}%
\pgfpathclose%
\pgfusepath{fill}%
\end{pgfscope}%
\begin{pgfscope}%
\pgfpathrectangle{\pgfqpoint{1.254980in}{0.150000in}}{\pgfqpoint{5.490039in}{5.490039in}}%
\pgfusepath{clip}%
\pgfsetbuttcap%
\pgfsetroundjoin%
\definecolor{currentfill}{rgb}{0.281446,0.084320,0.407414}%
\pgfsetfillcolor{currentfill}%
\pgfsetfillopacity{0.700000}%
\pgfsetlinewidth{0.000000pt}%
\definecolor{currentstroke}{rgb}{0.000000,0.000000,0.000000}%
\pgfsetstrokecolor{currentstroke}%
\pgfsetdash{}{0pt}%
\pgfpathmoveto{\pgfqpoint{4.235958in}{1.494593in}}%
\pgfpathlineto{\pgfqpoint{4.249610in}{1.488868in}}%
\pgfpathlineto{\pgfqpoint{4.263268in}{1.483166in}}%
\pgfpathlineto{\pgfqpoint{4.276931in}{1.477487in}}%
\pgfpathlineto{\pgfqpoint{4.290600in}{1.471831in}}%
\pgfpathlineto{\pgfqpoint{4.282843in}{1.471079in}}%
\pgfpathlineto{\pgfqpoint{4.275079in}{1.470685in}}%
\pgfpathlineto{\pgfqpoint{4.267308in}{1.470658in}}%
\pgfpathlineto{\pgfqpoint{4.259530in}{1.471008in}}%
\pgfpathlineto{\pgfqpoint{4.245838in}{1.477006in}}%
\pgfpathlineto{\pgfqpoint{4.232152in}{1.483027in}}%
\pgfpathlineto{\pgfqpoint{4.218472in}{1.489071in}}%
\pgfpathlineto{\pgfqpoint{4.204797in}{1.495139in}}%
\pgfpathlineto{\pgfqpoint{4.212598in}{1.494442in}}%
\pgfpathlineto{\pgfqpoint{4.220392in}{1.494124in}}%
\pgfpathlineto{\pgfqpoint{4.228179in}{1.494178in}}%
\pgfpathlineto{\pgfqpoint{4.235958in}{1.494593in}}%
\pgfpathclose%
\pgfusepath{fill}%
\end{pgfscope}%
\begin{pgfscope}%
\pgfpathrectangle{\pgfqpoint{1.254980in}{0.150000in}}{\pgfqpoint{5.490039in}{5.490039in}}%
\pgfusepath{clip}%
\pgfsetbuttcap%
\pgfsetroundjoin%
\definecolor{currentfill}{rgb}{0.216210,0.351535,0.550627}%
\pgfsetfillcolor{currentfill}%
\pgfsetfillopacity{0.700000}%
\pgfsetlinewidth{0.000000pt}%
\definecolor{currentstroke}{rgb}{0.000000,0.000000,0.000000}%
\pgfsetstrokecolor{currentstroke}%
\pgfsetdash{}{0pt}%
\pgfpathmoveto{\pgfqpoint{3.238958in}{2.047684in}}%
\pgfpathlineto{\pgfqpoint{3.252458in}{2.038869in}}%
\pgfpathlineto{\pgfqpoint{3.265961in}{2.030082in}}%
\pgfpathlineto{\pgfqpoint{3.279468in}{2.021324in}}%
\pgfpathlineto{\pgfqpoint{3.292978in}{2.012594in}}%
\pgfpathlineto{\pgfqpoint{3.284492in}{2.025114in}}%
\pgfpathlineto{\pgfqpoint{3.275984in}{2.038219in}}%
\pgfpathlineto{\pgfqpoint{3.267453in}{2.051922in}}%
\pgfpathlineto{\pgfqpoint{3.258898in}{2.066234in}}%
\pgfpathlineto{\pgfqpoint{3.245343in}{2.075372in}}%
\pgfpathlineto{\pgfqpoint{3.231792in}{2.084538in}}%
\pgfpathlineto{\pgfqpoint{3.218243in}{2.093732in}}%
\pgfpathlineto{\pgfqpoint{3.204697in}{2.102956in}}%
\pgfpathlineto{\pgfqpoint{3.213298in}{2.088229in}}%
\pgfpathlineto{\pgfqpoint{3.221875in}{2.074117in}}%
\pgfpathlineto{\pgfqpoint{3.230428in}{2.060606in}}%
\pgfpathlineto{\pgfqpoint{3.238958in}{2.047684in}}%
\pgfpathclose%
\pgfusepath{fill}%
\end{pgfscope}%
\begin{pgfscope}%
\pgfpathrectangle{\pgfqpoint{1.254980in}{0.150000in}}{\pgfqpoint{5.490039in}{5.490039in}}%
\pgfusepath{clip}%
\pgfsetbuttcap%
\pgfsetroundjoin%
\definecolor{currentfill}{rgb}{0.160665,0.478540,0.558115}%
\pgfsetfillcolor{currentfill}%
\pgfsetfillopacity{0.700000}%
\pgfsetlinewidth{0.000000pt}%
\definecolor{currentstroke}{rgb}{0.000000,0.000000,0.000000}%
\pgfsetstrokecolor{currentstroke}%
\pgfsetdash{}{0pt}%
\pgfpathmoveto{\pgfqpoint{2.826504in}{2.373546in}}%
\pgfpathlineto{\pgfqpoint{2.839979in}{2.363449in}}%
\pgfpathlineto{\pgfqpoint{2.853455in}{2.353386in}}%
\pgfpathlineto{\pgfqpoint{2.866934in}{2.343356in}}%
\pgfpathlineto{\pgfqpoint{2.880415in}{2.333360in}}%
\pgfpathlineto{\pgfqpoint{2.871484in}{2.351241in}}%
\pgfpathlineto{\pgfqpoint{2.862523in}{2.369787in}}%
\pgfpathlineto{\pgfqpoint{2.853530in}{2.389013in}}%
\pgfpathlineto{\pgfqpoint{2.844505in}{2.408930in}}%
\pgfpathlineto{\pgfqpoint{2.830969in}{2.419357in}}%
\pgfpathlineto{\pgfqpoint{2.817435in}{2.429818in}}%
\pgfpathlineto{\pgfqpoint{2.803903in}{2.440312in}}%
\pgfpathlineto{\pgfqpoint{2.790373in}{2.450841in}}%
\pgfpathlineto{\pgfqpoint{2.799455in}{2.430486in}}%
\pgfpathlineto{\pgfqpoint{2.808504in}{2.410827in}}%
\pgfpathlineto{\pgfqpoint{2.817520in}{2.391851in}}%
\pgfpathlineto{\pgfqpoint{2.826504in}{2.373546in}}%
\pgfpathclose%
\pgfusepath{fill}%
\end{pgfscope}%
\begin{pgfscope}%
\pgfpathrectangle{\pgfqpoint{1.254980in}{0.150000in}}{\pgfqpoint{5.490039in}{5.490039in}}%
\pgfusepath{clip}%
\pgfsetbuttcap%
\pgfsetroundjoin%
\definecolor{currentfill}{rgb}{0.274952,0.037752,0.364543}%
\pgfsetfillcolor{currentfill}%
\pgfsetfillopacity{0.700000}%
\pgfsetlinewidth{0.000000pt}%
\definecolor{currentstroke}{rgb}{0.000000,0.000000,0.000000}%
\pgfsetstrokecolor{currentstroke}%
\pgfsetdash{}{0pt}%
\pgfpathmoveto{\pgfqpoint{4.571342in}{1.412625in}}%
\pgfpathlineto{\pgfqpoint{4.585077in}{1.408004in}}%
\pgfpathlineto{\pgfqpoint{4.598819in}{1.403405in}}%
\pgfpathlineto{\pgfqpoint{4.612566in}{1.398830in}}%
\pgfpathlineto{\pgfqpoint{4.626321in}{1.394277in}}%
\pgfpathlineto{\pgfqpoint{4.618694in}{1.389426in}}%
\pgfpathlineto{\pgfqpoint{4.611064in}{1.384846in}}%
\pgfpathlineto{\pgfqpoint{4.603430in}{1.380547in}}%
\pgfpathlineto{\pgfqpoint{4.595792in}{1.376536in}}%
\pgfpathlineto{\pgfqpoint{4.582023in}{1.381404in}}%
\pgfpathlineto{\pgfqpoint{4.568259in}{1.386295in}}%
\pgfpathlineto{\pgfqpoint{4.554502in}{1.391209in}}%
\pgfpathlineto{\pgfqpoint{4.540751in}{1.396146in}}%
\pgfpathlineto{\pgfqpoint{4.548406in}{1.399836in}}%
\pgfpathlineto{\pgfqpoint{4.556055in}{1.403818in}}%
\pgfpathlineto{\pgfqpoint{4.563701in}{1.408084in}}%
\pgfpathlineto{\pgfqpoint{4.571342in}{1.412625in}}%
\pgfpathclose%
\pgfusepath{fill}%
\end{pgfscope}%
\begin{pgfscope}%
\pgfpathrectangle{\pgfqpoint{1.254980in}{0.150000in}}{\pgfqpoint{5.490039in}{5.490039in}}%
\pgfusepath{clip}%
\pgfsetbuttcap%
\pgfsetroundjoin%
\definecolor{currentfill}{rgb}{0.273809,0.031497,0.358853}%
\pgfsetfillcolor{currentfill}%
\pgfsetfillopacity{0.700000}%
\pgfsetlinewidth{0.000000pt}%
\definecolor{currentstroke}{rgb}{0.000000,0.000000,0.000000}%
\pgfsetstrokecolor{currentstroke}%
\pgfsetdash{}{0pt}%
\pgfpathmoveto{\pgfqpoint{4.711816in}{1.399468in}}%
\pgfpathlineto{\pgfqpoint{4.725589in}{1.395332in}}%
\pgfpathlineto{\pgfqpoint{4.739370in}{1.391218in}}%
\pgfpathlineto{\pgfqpoint{4.753157in}{1.387126in}}%
\pgfpathlineto{\pgfqpoint{4.766951in}{1.383057in}}%
\pgfpathlineto{\pgfqpoint{4.759365in}{1.376603in}}%
\pgfpathlineto{\pgfqpoint{4.751776in}{1.370381in}}%
\pgfpathlineto{\pgfqpoint{4.744184in}{1.364401in}}%
\pgfpathlineto{\pgfqpoint{4.736589in}{1.358671in}}%
\pgfpathlineto{\pgfqpoint{4.722783in}{1.363042in}}%
\pgfpathlineto{\pgfqpoint{4.708983in}{1.367437in}}%
\pgfpathlineto{\pgfqpoint{4.695189in}{1.371853in}}%
\pgfpathlineto{\pgfqpoint{4.681403in}{1.376293in}}%
\pgfpathlineto{\pgfqpoint{4.689011in}{1.381716in}}%
\pgfpathlineto{\pgfqpoint{4.696616in}{1.387391in}}%
\pgfpathlineto{\pgfqpoint{4.704217in}{1.393311in}}%
\pgfpathlineto{\pgfqpoint{4.711816in}{1.399468in}}%
\pgfpathclose%
\pgfusepath{fill}%
\end{pgfscope}%
\begin{pgfscope}%
\pgfpathrectangle{\pgfqpoint{1.254980in}{0.150000in}}{\pgfqpoint{5.490039in}{5.490039in}}%
\pgfusepath{clip}%
\pgfsetbuttcap%
\pgfsetroundjoin%
\definecolor{currentfill}{rgb}{0.170948,0.694384,0.493803}%
\pgfsetfillcolor{currentfill}%
\pgfsetfillopacity{0.700000}%
\pgfsetlinewidth{0.000000pt}%
\definecolor{currentstroke}{rgb}{0.000000,0.000000,0.000000}%
\pgfsetstrokecolor{currentstroke}%
\pgfsetdash{}{0pt}%
\pgfpathmoveto{\pgfqpoint{2.196328in}{2.952296in}}%
\pgfpathlineto{\pgfqpoint{2.209816in}{2.939958in}}%
\pgfpathlineto{\pgfqpoint{2.223303in}{2.927670in}}%
\pgfpathlineto{\pgfqpoint{2.236790in}{2.915431in}}%
\pgfpathlineto{\pgfqpoint{2.250277in}{2.903241in}}%
\pgfpathlineto{\pgfqpoint{2.240533in}{2.928805in}}%
\pgfpathlineto{\pgfqpoint{2.230742in}{2.955144in}}%
\pgfpathlineto{\pgfqpoint{2.220905in}{2.982271in}}%
\pgfpathlineto{\pgfqpoint{2.211019in}{3.010201in}}%
\pgfpathlineto{\pgfqpoint{2.197463in}{3.022856in}}%
\pgfpathlineto{\pgfqpoint{2.183906in}{3.035561in}}%
\pgfpathlineto{\pgfqpoint{2.170348in}{3.048316in}}%
\pgfpathlineto{\pgfqpoint{2.156791in}{3.061120in}}%
\pgfpathlineto{\pgfqpoint{2.166748in}{3.032716in}}%
\pgfpathlineto{\pgfqpoint{2.176656in}{3.005120in}}%
\pgfpathlineto{\pgfqpoint{2.186515in}{2.978319in}}%
\pgfpathlineto{\pgfqpoint{2.196328in}{2.952296in}}%
\pgfpathclose%
\pgfusepath{fill}%
\end{pgfscope}%
\begin{pgfscope}%
\pgfpathrectangle{\pgfqpoint{1.254980in}{0.150000in}}{\pgfqpoint{5.490039in}{5.490039in}}%
\pgfusepath{clip}%
\pgfsetbuttcap%
\pgfsetroundjoin%
\definecolor{currentfill}{rgb}{0.278012,0.180367,0.486697}%
\pgfsetfillcolor{currentfill}%
\pgfsetfillopacity{0.700000}%
\pgfsetlinewidth{0.000000pt}%
\definecolor{currentstroke}{rgb}{0.000000,0.000000,0.000000}%
\pgfsetstrokecolor{currentstroke}%
\pgfsetdash{}{0pt}%
\pgfpathmoveto{\pgfqpoint{3.846209in}{1.663215in}}%
\pgfpathlineto{\pgfqpoint{3.859791in}{1.656217in}}%
\pgfpathlineto{\pgfqpoint{3.873378in}{1.649244in}}%
\pgfpathlineto{\pgfqpoint{3.886970in}{1.642295in}}%
\pgfpathlineto{\pgfqpoint{3.900566in}{1.635371in}}%
\pgfpathlineto{\pgfqpoint{3.892582in}{1.639943in}}%
\pgfpathlineto{\pgfqpoint{3.884586in}{1.644973in}}%
\pgfpathlineto{\pgfqpoint{3.876577in}{1.650471in}}%
\pgfpathlineto{\pgfqpoint{3.868557in}{1.656447in}}%
\pgfpathlineto{\pgfqpoint{3.854929in}{1.663743in}}%
\pgfpathlineto{\pgfqpoint{3.841306in}{1.671064in}}%
\pgfpathlineto{\pgfqpoint{3.827687in}{1.678409in}}%
\pgfpathlineto{\pgfqpoint{3.814073in}{1.685779in}}%
\pgfpathlineto{\pgfqpoint{3.822127in}{1.679426in}}%
\pgfpathlineto{\pgfqpoint{3.830167in}{1.673554in}}%
\pgfpathlineto{\pgfqpoint{3.838194in}{1.668154in}}%
\pgfpathlineto{\pgfqpoint{3.846209in}{1.663215in}}%
\pgfpathclose%
\pgfusepath{fill}%
\end{pgfscope}%
\begin{pgfscope}%
\pgfpathrectangle{\pgfqpoint{1.254980in}{0.150000in}}{\pgfqpoint{5.490039in}{5.490039in}}%
\pgfusepath{clip}%
\pgfsetbuttcap%
\pgfsetroundjoin%
\definecolor{currentfill}{rgb}{0.277941,0.056324,0.381191}%
\pgfsetfillcolor{currentfill}%
\pgfsetfillopacity{0.700000}%
\pgfsetlinewidth{0.000000pt}%
\definecolor{currentstroke}{rgb}{0.000000,0.000000,0.000000}%
\pgfsetstrokecolor{currentstroke}%
\pgfsetdash{}{0pt}%
\pgfpathmoveto{\pgfqpoint{4.430966in}{1.436459in}}%
\pgfpathlineto{\pgfqpoint{4.444668in}{1.431340in}}%
\pgfpathlineto{\pgfqpoint{4.458376in}{1.426244in}}%
\pgfpathlineto{\pgfqpoint{4.472090in}{1.421170in}}%
\pgfpathlineto{\pgfqpoint{4.485810in}{1.416120in}}%
\pgfpathlineto{\pgfqpoint{4.478134in}{1.413054in}}%
\pgfpathlineto{\pgfqpoint{4.470453in}{1.410300in}}%
\pgfpathlineto{\pgfqpoint{4.462767in}{1.407867in}}%
\pgfpathlineto{\pgfqpoint{4.455076in}{1.405764in}}%
\pgfpathlineto{\pgfqpoint{4.441338in}{1.411144in}}%
\pgfpathlineto{\pgfqpoint{4.427606in}{1.416546in}}%
\pgfpathlineto{\pgfqpoint{4.413879in}{1.421971in}}%
\pgfpathlineto{\pgfqpoint{4.400158in}{1.427419in}}%
\pgfpathlineto{\pgfqpoint{4.407868in}{1.429188in}}%
\pgfpathlineto{\pgfqpoint{4.415573in}{1.431290in}}%
\pgfpathlineto{\pgfqpoint{4.423272in}{1.433717in}}%
\pgfpathlineto{\pgfqpoint{4.430966in}{1.436459in}}%
\pgfpathclose%
\pgfusepath{fill}%
\end{pgfscope}%
\begin{pgfscope}%
\pgfpathrectangle{\pgfqpoint{1.254980in}{0.150000in}}{\pgfqpoint{5.490039in}{5.490039in}}%
\pgfusepath{clip}%
\pgfsetbuttcap%
\pgfsetroundjoin%
\definecolor{currentfill}{rgb}{0.276022,0.044167,0.370164}%
\pgfsetfillcolor{currentfill}%
\pgfsetfillopacity{0.700000}%
\pgfsetlinewidth{0.000000pt}%
\definecolor{currentstroke}{rgb}{0.000000,0.000000,0.000000}%
\pgfsetstrokecolor{currentstroke}%
\pgfsetdash{}{0pt}%
\pgfpathmoveto{\pgfqpoint{5.078949in}{1.427323in}}%
\pgfpathlineto{\pgfqpoint{5.092839in}{1.424471in}}%
\pgfpathlineto{\pgfqpoint{5.106736in}{1.421642in}}%
\pgfpathlineto{\pgfqpoint{5.120641in}{1.418836in}}%
\pgfpathlineto{\pgfqpoint{5.134554in}{1.416052in}}%
\pgfpathlineto{\pgfqpoint{5.127044in}{1.406066in}}%
\pgfpathlineto{\pgfqpoint{5.119531in}{1.396212in}}%
\pgfpathlineto{\pgfqpoint{5.112017in}{1.386496in}}%
\pgfpathlineto{\pgfqpoint{5.104500in}{1.376925in}}%
\pgfpathlineto{\pgfqpoint{5.090581in}{1.379974in}}%
\pgfpathlineto{\pgfqpoint{5.076669in}{1.383046in}}%
\pgfpathlineto{\pgfqpoint{5.062765in}{1.386140in}}%
\pgfpathlineto{\pgfqpoint{5.048868in}{1.389256in}}%
\pgfpathlineto{\pgfqpoint{5.056392in}{1.398557in}}%
\pgfpathlineto{\pgfqpoint{5.063913in}{1.408006in}}%
\pgfpathlineto{\pgfqpoint{5.071432in}{1.417597in}}%
\pgfpathlineto{\pgfqpoint{5.078949in}{1.427323in}}%
\pgfpathclose%
\pgfusepath{fill}%
\end{pgfscope}%
\begin{pgfscope}%
\pgfpathrectangle{\pgfqpoint{1.254980in}{0.150000in}}{\pgfqpoint{5.490039in}{5.490039in}}%
\pgfusepath{clip}%
\pgfsetbuttcap%
\pgfsetroundjoin%
\definecolor{currentfill}{rgb}{0.272594,0.025563,0.353093}%
\pgfsetfillcolor{currentfill}%
\pgfsetfillopacity{0.700000}%
\pgfsetlinewidth{0.000000pt}%
\definecolor{currentstroke}{rgb}{0.000000,0.000000,0.000000}%
\pgfsetstrokecolor{currentstroke}%
\pgfsetdash{}{0pt}%
\pgfpathmoveto{\pgfqpoint{4.852464in}{1.396171in}}%
\pgfpathlineto{\pgfqpoint{4.866282in}{1.392506in}}%
\pgfpathlineto{\pgfqpoint{4.880107in}{1.388862in}}%
\pgfpathlineto{\pgfqpoint{4.893938in}{1.385242in}}%
\pgfpathlineto{\pgfqpoint{4.907778in}{1.381644in}}%
\pgfpathlineto{\pgfqpoint{4.900225in}{1.373760in}}%
\pgfpathlineto{\pgfqpoint{4.892669in}{1.366074in}}%
\pgfpathlineto{\pgfqpoint{4.885111in}{1.358591in}}%
\pgfpathlineto{\pgfqpoint{4.877551in}{1.351321in}}%
\pgfpathlineto{\pgfqpoint{4.863702in}{1.355209in}}%
\pgfpathlineto{\pgfqpoint{4.849859in}{1.359119in}}%
\pgfpathlineto{\pgfqpoint{4.836024in}{1.363053in}}%
\pgfpathlineto{\pgfqpoint{4.822196in}{1.367008in}}%
\pgfpathlineto{\pgfqpoint{4.829767in}{1.373984in}}%
\pgfpathlineto{\pgfqpoint{4.837335in}{1.381175in}}%
\pgfpathlineto{\pgfqpoint{4.844901in}{1.388573in}}%
\pgfpathlineto{\pgfqpoint{4.852464in}{1.396171in}}%
\pgfpathclose%
\pgfusepath{fill}%
\end{pgfscope}%
\begin{pgfscope}%
\pgfpathrectangle{\pgfqpoint{1.254980in}{0.150000in}}{\pgfqpoint{5.490039in}{5.490039in}}%
\pgfusepath{clip}%
\pgfsetbuttcap%
\pgfsetroundjoin%
\definecolor{currentfill}{rgb}{0.258965,0.251537,0.524736}%
\pgfsetfillcolor{currentfill}%
\pgfsetfillopacity{0.700000}%
\pgfsetlinewidth{0.000000pt}%
\definecolor{currentstroke}{rgb}{0.000000,0.000000,0.000000}%
\pgfsetstrokecolor{currentstroke}%
\pgfsetdash{}{0pt}%
\pgfpathmoveto{\pgfqpoint{3.596830in}{1.807100in}}%
\pgfpathlineto{\pgfqpoint{3.610376in}{1.799327in}}%
\pgfpathlineto{\pgfqpoint{3.623927in}{1.791579in}}%
\pgfpathlineto{\pgfqpoint{3.637482in}{1.783858in}}%
\pgfpathlineto{\pgfqpoint{3.651040in}{1.776162in}}%
\pgfpathlineto{\pgfqpoint{3.642869in}{1.784154in}}%
\pgfpathlineto{\pgfqpoint{3.634682in}{1.792662in}}%
\pgfpathlineto{\pgfqpoint{3.626478in}{1.801696in}}%
\pgfpathlineto{\pgfqpoint{3.618258in}{1.811268in}}%
\pgfpathlineto{\pgfqpoint{3.604662in}{1.819353in}}%
\pgfpathlineto{\pgfqpoint{3.591070in}{1.827463in}}%
\pgfpathlineto{\pgfqpoint{3.577482in}{1.835599in}}%
\pgfpathlineto{\pgfqpoint{3.563898in}{1.843761in}}%
\pgfpathlineto{\pgfqpoint{3.572157in}{1.833795in}}%
\pgfpathlineto{\pgfqpoint{3.580398in}{1.824370in}}%
\pgfpathlineto{\pgfqpoint{3.588622in}{1.815475in}}%
\pgfpathlineto{\pgfqpoint{3.596830in}{1.807100in}}%
\pgfpathclose%
\pgfusepath{fill}%
\end{pgfscope}%
\begin{pgfscope}%
\pgfpathrectangle{\pgfqpoint{1.254980in}{0.150000in}}{\pgfqpoint{5.490039in}{5.490039in}}%
\pgfusepath{clip}%
\pgfsetbuttcap%
\pgfsetroundjoin%
\definecolor{currentfill}{rgb}{0.150148,0.676631,0.506589}%
\pgfsetfillcolor{currentfill}%
\pgfsetfillopacity{0.700000}%
\pgfsetlinewidth{0.000000pt}%
\definecolor{currentstroke}{rgb}{0.000000,0.000000,0.000000}%
\pgfsetstrokecolor{currentstroke}%
\pgfsetdash{}{0pt}%
\pgfpathmoveto{\pgfqpoint{2.250277in}{2.903241in}}%
\pgfpathlineto{\pgfqpoint{2.263765in}{2.891099in}}%
\pgfpathlineto{\pgfqpoint{2.277253in}{2.879005in}}%
\pgfpathlineto{\pgfqpoint{2.290740in}{2.866958in}}%
\pgfpathlineto{\pgfqpoint{2.304228in}{2.854958in}}%
\pgfpathlineto{\pgfqpoint{2.294551in}{2.880066in}}%
\pgfpathlineto{\pgfqpoint{2.284829in}{2.905943in}}%
\pgfpathlineto{\pgfqpoint{2.275061in}{2.932603in}}%
\pgfpathlineto{\pgfqpoint{2.265246in}{2.960061in}}%
\pgfpathlineto{\pgfqpoint{2.251689in}{2.972525in}}%
\pgfpathlineto{\pgfqpoint{2.238133in}{2.985035in}}%
\pgfpathlineto{\pgfqpoint{2.224576in}{2.997594in}}%
\pgfpathlineto{\pgfqpoint{2.211019in}{3.010201in}}%
\pgfpathlineto{\pgfqpoint{2.220905in}{2.982271in}}%
\pgfpathlineto{\pgfqpoint{2.230742in}{2.955144in}}%
\pgfpathlineto{\pgfqpoint{2.240533in}{2.928805in}}%
\pgfpathlineto{\pgfqpoint{2.250277in}{2.903241in}}%
\pgfpathclose%
\pgfusepath{fill}%
\end{pgfscope}%
\begin{pgfscope}%
\pgfpathrectangle{\pgfqpoint{1.254980in}{0.150000in}}{\pgfqpoint{5.490039in}{5.490039in}}%
\pgfusepath{clip}%
\pgfsetbuttcap%
\pgfsetroundjoin%
\definecolor{currentfill}{rgb}{0.165117,0.467423,0.558141}%
\pgfsetfillcolor{currentfill}%
\pgfsetfillopacity{0.700000}%
\pgfsetlinewidth{0.000000pt}%
\definecolor{currentstroke}{rgb}{0.000000,0.000000,0.000000}%
\pgfsetstrokecolor{currentstroke}%
\pgfsetdash{}{0pt}%
\pgfpathmoveto{\pgfqpoint{2.880415in}{2.333360in}}%
\pgfpathlineto{\pgfqpoint{2.893898in}{2.323397in}}%
\pgfpathlineto{\pgfqpoint{2.907383in}{2.313467in}}%
\pgfpathlineto{\pgfqpoint{2.920871in}{2.303570in}}%
\pgfpathlineto{\pgfqpoint{2.934361in}{2.293705in}}%
\pgfpathlineto{\pgfqpoint{2.925484in}{2.311162in}}%
\pgfpathlineto{\pgfqpoint{2.916576in}{2.329281in}}%
\pgfpathlineto{\pgfqpoint{2.907638in}{2.348074in}}%
\pgfpathlineto{\pgfqpoint{2.898668in}{2.367554in}}%
\pgfpathlineto{\pgfqpoint{2.885124in}{2.377848in}}%
\pgfpathlineto{\pgfqpoint{2.871582in}{2.388176in}}%
\pgfpathlineto{\pgfqpoint{2.858043in}{2.398536in}}%
\pgfpathlineto{\pgfqpoint{2.844505in}{2.408930in}}%
\pgfpathlineto{\pgfqpoint{2.853530in}{2.389013in}}%
\pgfpathlineto{\pgfqpoint{2.862523in}{2.369787in}}%
\pgfpathlineto{\pgfqpoint{2.871484in}{2.351241in}}%
\pgfpathlineto{\pgfqpoint{2.880415in}{2.333360in}}%
\pgfpathclose%
\pgfusepath{fill}%
\end{pgfscope}%
\begin{pgfscope}%
\pgfpathrectangle{\pgfqpoint{1.254980in}{0.150000in}}{\pgfqpoint{5.490039in}{5.490039in}}%
\pgfusepath{clip}%
\pgfsetbuttcap%
\pgfsetroundjoin%
\definecolor{currentfill}{rgb}{0.220057,0.343307,0.549413}%
\pgfsetfillcolor{currentfill}%
\pgfsetfillopacity{0.700000}%
\pgfsetlinewidth{0.000000pt}%
\definecolor{currentstroke}{rgb}{0.000000,0.000000,0.000000}%
\pgfsetstrokecolor{currentstroke}%
\pgfsetdash{}{0pt}%
\pgfpathmoveto{\pgfqpoint{3.292978in}{2.012594in}}%
\pgfpathlineto{\pgfqpoint{3.306491in}{2.003893in}}%
\pgfpathlineto{\pgfqpoint{3.320007in}{1.995219in}}%
\pgfpathlineto{\pgfqpoint{3.333527in}{1.986574in}}%
\pgfpathlineto{\pgfqpoint{3.347050in}{1.977956in}}%
\pgfpathlineto{\pgfqpoint{3.338607in}{1.990075in}}%
\pgfpathlineto{\pgfqpoint{3.330144in}{2.002775in}}%
\pgfpathlineto{\pgfqpoint{3.321658in}{2.016068in}}%
\pgfpathlineto{\pgfqpoint{3.313149in}{2.029966in}}%
\pgfpathlineto{\pgfqpoint{3.299582in}{2.038991in}}%
\pgfpathlineto{\pgfqpoint{3.286018in}{2.048044in}}%
\pgfpathlineto{\pgfqpoint{3.272456in}{2.057125in}}%
\pgfpathlineto{\pgfqpoint{3.258898in}{2.066234in}}%
\pgfpathlineto{\pgfqpoint{3.267453in}{2.051922in}}%
\pgfpathlineto{\pgfqpoint{3.275984in}{2.038219in}}%
\pgfpathlineto{\pgfqpoint{3.284492in}{2.025114in}}%
\pgfpathlineto{\pgfqpoint{3.292978in}{2.012594in}}%
\pgfpathclose%
\pgfusepath{fill}%
\end{pgfscope}%
\begin{pgfscope}%
\pgfpathrectangle{\pgfqpoint{1.254980in}{0.150000in}}{\pgfqpoint{5.490039in}{5.490039in}}%
\pgfusepath{clip}%
\pgfsetbuttcap%
\pgfsetroundjoin%
\definecolor{currentfill}{rgb}{0.283229,0.120777,0.440584}%
\pgfsetfillcolor{currentfill}%
\pgfsetfillopacity{0.700000}%
\pgfsetlinewidth{0.000000pt}%
\definecolor{currentstroke}{rgb}{0.000000,0.000000,0.000000}%
\pgfsetstrokecolor{currentstroke}%
\pgfsetdash{}{0pt}%
\pgfpathmoveto{\pgfqpoint{4.095587in}{1.544524in}}%
\pgfpathlineto{\pgfqpoint{4.109220in}{1.538269in}}%
\pgfpathlineto{\pgfqpoint{4.122858in}{1.532037in}}%
\pgfpathlineto{\pgfqpoint{4.136501in}{1.525828in}}%
\pgfpathlineto{\pgfqpoint{4.150150in}{1.519644in}}%
\pgfpathlineto{\pgfqpoint{4.142316in}{1.521082in}}%
\pgfpathlineto{\pgfqpoint{4.134473in}{1.522923in}}%
\pgfpathlineto{\pgfqpoint{4.126622in}{1.525177in}}%
\pgfpathlineto{\pgfqpoint{4.118762in}{1.527853in}}%
\pgfpathlineto{\pgfqpoint{4.105088in}{1.534394in}}%
\pgfpathlineto{\pgfqpoint{4.091418in}{1.540959in}}%
\pgfpathlineto{\pgfqpoint{4.077754in}{1.547548in}}%
\pgfpathlineto{\pgfqpoint{4.064095in}{1.554160in}}%
\pgfpathlineto{\pgfqpoint{4.071982in}{1.551122in}}%
\pgfpathlineto{\pgfqpoint{4.079860in}{1.548510in}}%
\pgfpathlineto{\pgfqpoint{4.087728in}{1.546314in}}%
\pgfpathlineto{\pgfqpoint{4.095587in}{1.544524in}}%
\pgfpathclose%
\pgfusepath{fill}%
\end{pgfscope}%
\begin{pgfscope}%
\pgfpathrectangle{\pgfqpoint{1.254980in}{0.150000in}}{\pgfqpoint{5.490039in}{5.490039in}}%
\pgfusepath{clip}%
\pgfsetbuttcap%
\pgfsetroundjoin%
\definecolor{currentfill}{rgb}{0.137339,0.662252,0.515571}%
\pgfsetfillcolor{currentfill}%
\pgfsetfillopacity{0.700000}%
\pgfsetlinewidth{0.000000pt}%
\definecolor{currentstroke}{rgb}{0.000000,0.000000,0.000000}%
\pgfsetstrokecolor{currentstroke}%
\pgfsetdash{}{0pt}%
\pgfpathmoveto{\pgfqpoint{2.304228in}{2.854958in}}%
\pgfpathlineto{\pgfqpoint{2.317717in}{2.843005in}}%
\pgfpathlineto{\pgfqpoint{2.331205in}{2.831097in}}%
\pgfpathlineto{\pgfqpoint{2.344695in}{2.819235in}}%
\pgfpathlineto{\pgfqpoint{2.358184in}{2.807418in}}%
\pgfpathlineto{\pgfqpoint{2.348573in}{2.832071in}}%
\pgfpathlineto{\pgfqpoint{2.338918in}{2.857488in}}%
\pgfpathlineto{\pgfqpoint{2.329218in}{2.883683in}}%
\pgfpathlineto{\pgfqpoint{2.319473in}{2.910670in}}%
\pgfpathlineto{\pgfqpoint{2.305916in}{2.922949in}}%
\pgfpathlineto{\pgfqpoint{2.292359in}{2.935274in}}%
\pgfpathlineto{\pgfqpoint{2.278802in}{2.947644in}}%
\pgfpathlineto{\pgfqpoint{2.265246in}{2.960061in}}%
\pgfpathlineto{\pgfqpoint{2.275061in}{2.932603in}}%
\pgfpathlineto{\pgfqpoint{2.284829in}{2.905943in}}%
\pgfpathlineto{\pgfqpoint{2.294551in}{2.880066in}}%
\pgfpathlineto{\pgfqpoint{2.304228in}{2.854958in}}%
\pgfpathclose%
\pgfusepath{fill}%
\end{pgfscope}%
\begin{pgfscope}%
\pgfpathrectangle{\pgfqpoint{1.254980in}{0.150000in}}{\pgfqpoint{5.490039in}{5.490039in}}%
\pgfusepath{clip}%
\pgfsetbuttcap%
\pgfsetroundjoin%
\definecolor{currentfill}{rgb}{0.281446,0.084320,0.407414}%
\pgfsetfillcolor{currentfill}%
\pgfsetfillopacity{0.700000}%
\pgfsetlinewidth{0.000000pt}%
\definecolor{currentstroke}{rgb}{0.000000,0.000000,0.000000}%
\pgfsetstrokecolor{currentstroke}%
\pgfsetdash{}{0pt}%
\pgfpathmoveto{\pgfqpoint{4.290600in}{1.471831in}}%
\pgfpathlineto{\pgfqpoint{4.304275in}{1.466199in}}%
\pgfpathlineto{\pgfqpoint{4.317956in}{1.460590in}}%
\pgfpathlineto{\pgfqpoint{4.331642in}{1.455004in}}%
\pgfpathlineto{\pgfqpoint{4.345333in}{1.449441in}}%
\pgfpathlineto{\pgfqpoint{4.337597in}{1.448351in}}%
\pgfpathlineto{\pgfqpoint{4.329855in}{1.447616in}}%
\pgfpathlineto{\pgfqpoint{4.322105in}{1.447245in}}%
\pgfpathlineto{\pgfqpoint{4.314350in}{1.447247in}}%
\pgfpathlineto{\pgfqpoint{4.300636in}{1.453153in}}%
\pgfpathlineto{\pgfqpoint{4.286929in}{1.459081in}}%
\pgfpathlineto{\pgfqpoint{4.273226in}{1.465033in}}%
\pgfpathlineto{\pgfqpoint{4.259530in}{1.471008in}}%
\pgfpathlineto{\pgfqpoint{4.267308in}{1.470658in}}%
\pgfpathlineto{\pgfqpoint{4.275079in}{1.470685in}}%
\pgfpathlineto{\pgfqpoint{4.282843in}{1.471079in}}%
\pgfpathlineto{\pgfqpoint{4.290600in}{1.471831in}}%
\pgfpathclose%
\pgfusepath{fill}%
\end{pgfscope}%
\begin{pgfscope}%
\pgfpathrectangle{\pgfqpoint{1.254980in}{0.150000in}}{\pgfqpoint{5.490039in}{5.490039in}}%
\pgfusepath{clip}%
\pgfsetbuttcap%
\pgfsetroundjoin%
\definecolor{currentfill}{rgb}{0.274952,0.037752,0.364543}%
\pgfsetfillcolor{currentfill}%
\pgfsetfillopacity{0.700000}%
\pgfsetlinewidth{0.000000pt}%
\definecolor{currentstroke}{rgb}{0.000000,0.000000,0.000000}%
\pgfsetstrokecolor{currentstroke}%
\pgfsetdash{}{0pt}%
\pgfpathmoveto{\pgfqpoint{4.993356in}{1.401950in}}%
\pgfpathlineto{\pgfqpoint{5.007223in}{1.398742in}}%
\pgfpathlineto{\pgfqpoint{5.021097in}{1.395558in}}%
\pgfpathlineto{\pgfqpoint{5.034979in}{1.392396in}}%
\pgfpathlineto{\pgfqpoint{5.048868in}{1.389256in}}%
\pgfpathlineto{\pgfqpoint{5.041342in}{1.380111in}}%
\pgfpathlineto{\pgfqpoint{5.033814in}{1.371127in}}%
\pgfpathlineto{\pgfqpoint{5.026283in}{1.362313in}}%
\pgfpathlineto{\pgfqpoint{5.018750in}{1.353674in}}%
\pgfpathlineto{\pgfqpoint{5.004853in}{1.357092in}}%
\pgfpathlineto{\pgfqpoint{4.990964in}{1.360531in}}%
\pgfpathlineto{\pgfqpoint{4.977081in}{1.363994in}}%
\pgfpathlineto{\pgfqpoint{4.963206in}{1.367479in}}%
\pgfpathlineto{\pgfqpoint{4.970747in}{1.375835in}}%
\pgfpathlineto{\pgfqpoint{4.978286in}{1.384370in}}%
\pgfpathlineto{\pgfqpoint{4.985822in}{1.393077in}}%
\pgfpathlineto{\pgfqpoint{4.993356in}{1.401950in}}%
\pgfpathclose%
\pgfusepath{fill}%
\end{pgfscope}%
\begin{pgfscope}%
\pgfpathrectangle{\pgfqpoint{1.254980in}{0.150000in}}{\pgfqpoint{5.490039in}{5.490039in}}%
\pgfusepath{clip}%
\pgfsetbuttcap%
\pgfsetroundjoin%
\definecolor{currentfill}{rgb}{0.169646,0.456262,0.558030}%
\pgfsetfillcolor{currentfill}%
\pgfsetfillopacity{0.700000}%
\pgfsetlinewidth{0.000000pt}%
\definecolor{currentstroke}{rgb}{0.000000,0.000000,0.000000}%
\pgfsetstrokecolor{currentstroke}%
\pgfsetdash{}{0pt}%
\pgfpathmoveto{\pgfqpoint{2.934361in}{2.293705in}}%
\pgfpathlineto{\pgfqpoint{2.947853in}{2.283873in}}%
\pgfpathlineto{\pgfqpoint{2.961348in}{2.274072in}}%
\pgfpathlineto{\pgfqpoint{2.974845in}{2.264304in}}%
\pgfpathlineto{\pgfqpoint{2.988345in}{2.254568in}}%
\pgfpathlineto{\pgfqpoint{2.979519in}{2.271603in}}%
\pgfpathlineto{\pgfqpoint{2.970665in}{2.289294in}}%
\pgfpathlineto{\pgfqpoint{2.961781in}{2.307655in}}%
\pgfpathlineto{\pgfqpoint{2.952866in}{2.326699in}}%
\pgfpathlineto{\pgfqpoint{2.939313in}{2.336864in}}%
\pgfpathlineto{\pgfqpoint{2.925763in}{2.347062in}}%
\pgfpathlineto{\pgfqpoint{2.912214in}{2.357292in}}%
\pgfpathlineto{\pgfqpoint{2.898668in}{2.367554in}}%
\pgfpathlineto{\pgfqpoint{2.907638in}{2.348074in}}%
\pgfpathlineto{\pgfqpoint{2.916576in}{2.329281in}}%
\pgfpathlineto{\pgfqpoint{2.925484in}{2.311162in}}%
\pgfpathlineto{\pgfqpoint{2.934361in}{2.293705in}}%
\pgfpathclose%
\pgfusepath{fill}%
\end{pgfscope}%
\begin{pgfscope}%
\pgfpathrectangle{\pgfqpoint{1.254980in}{0.150000in}}{\pgfqpoint{5.490039in}{5.490039in}}%
\pgfusepath{clip}%
\pgfsetbuttcap%
\pgfsetroundjoin%
\definecolor{currentfill}{rgb}{0.279574,0.170599,0.479997}%
\pgfsetfillcolor{currentfill}%
\pgfsetfillopacity{0.700000}%
\pgfsetlinewidth{0.000000pt}%
\definecolor{currentstroke}{rgb}{0.000000,0.000000,0.000000}%
\pgfsetstrokecolor{currentstroke}%
\pgfsetdash{}{0pt}%
\pgfpathmoveto{\pgfqpoint{3.900566in}{1.635371in}}%
\pgfpathlineto{\pgfqpoint{3.914167in}{1.628471in}}%
\pgfpathlineto{\pgfqpoint{3.927773in}{1.621595in}}%
\pgfpathlineto{\pgfqpoint{3.941383in}{1.614743in}}%
\pgfpathlineto{\pgfqpoint{3.954998in}{1.607916in}}%
\pgfpathlineto{\pgfqpoint{3.947044in}{1.612122in}}%
\pgfpathlineto{\pgfqpoint{3.939078in}{1.616782in}}%
\pgfpathlineto{\pgfqpoint{3.931101in}{1.621906in}}%
\pgfpathlineto{\pgfqpoint{3.923112in}{1.627505in}}%
\pgfpathlineto{\pgfqpoint{3.909467in}{1.634704in}}%
\pgfpathlineto{\pgfqpoint{3.895825in}{1.641927in}}%
\pgfpathlineto{\pgfqpoint{3.882189in}{1.649175in}}%
\pgfpathlineto{\pgfqpoint{3.868557in}{1.656447in}}%
\pgfpathlineto{\pgfqpoint{3.876577in}{1.650471in}}%
\pgfpathlineto{\pgfqpoint{3.884586in}{1.644973in}}%
\pgfpathlineto{\pgfqpoint{3.892582in}{1.639943in}}%
\pgfpathlineto{\pgfqpoint{3.900566in}{1.635371in}}%
\pgfpathclose%
\pgfusepath{fill}%
\end{pgfscope}%
\begin{pgfscope}%
\pgfpathrectangle{\pgfqpoint{1.254980in}{0.150000in}}{\pgfqpoint{5.490039in}{5.490039in}}%
\pgfusepath{clip}%
\pgfsetbuttcap%
\pgfsetroundjoin%
\definecolor{currentfill}{rgb}{0.128087,0.647749,0.523491}%
\pgfsetfillcolor{currentfill}%
\pgfsetfillopacity{0.700000}%
\pgfsetlinewidth{0.000000pt}%
\definecolor{currentstroke}{rgb}{0.000000,0.000000,0.000000}%
\pgfsetstrokecolor{currentstroke}%
\pgfsetdash{}{0pt}%
\pgfpathmoveto{\pgfqpoint{2.358184in}{2.807418in}}%
\pgfpathlineto{\pgfqpoint{2.371674in}{2.795646in}}%
\pgfpathlineto{\pgfqpoint{2.385165in}{2.783917in}}%
\pgfpathlineto{\pgfqpoint{2.398656in}{2.772233in}}%
\pgfpathlineto{\pgfqpoint{2.412148in}{2.760592in}}%
\pgfpathlineto{\pgfqpoint{2.402602in}{2.784792in}}%
\pgfpathlineto{\pgfqpoint{2.393013in}{2.809751in}}%
\pgfpathlineto{\pgfqpoint{2.383381in}{2.835482in}}%
\pgfpathlineto{\pgfqpoint{2.373705in}{2.862001in}}%
\pgfpathlineto{\pgfqpoint{2.360146in}{2.874102in}}%
\pgfpathlineto{\pgfqpoint{2.346588in}{2.886247in}}%
\pgfpathlineto{\pgfqpoint{2.333031in}{2.898437in}}%
\pgfpathlineto{\pgfqpoint{2.319473in}{2.910670in}}%
\pgfpathlineto{\pgfqpoint{2.329218in}{2.883683in}}%
\pgfpathlineto{\pgfqpoint{2.338918in}{2.857488in}}%
\pgfpathlineto{\pgfqpoint{2.348573in}{2.832071in}}%
\pgfpathlineto{\pgfqpoint{2.358184in}{2.807418in}}%
\pgfpathclose%
\pgfusepath{fill}%
\end{pgfscope}%
\begin{pgfscope}%
\pgfpathrectangle{\pgfqpoint{1.254980in}{0.150000in}}{\pgfqpoint{5.490039in}{5.490039in}}%
\pgfusepath{clip}%
\pgfsetbuttcap%
\pgfsetroundjoin%
\definecolor{currentfill}{rgb}{0.274952,0.037752,0.364543}%
\pgfsetfillcolor{currentfill}%
\pgfsetfillopacity{0.700000}%
\pgfsetlinewidth{0.000000pt}%
\definecolor{currentstroke}{rgb}{0.000000,0.000000,0.000000}%
\pgfsetstrokecolor{currentstroke}%
\pgfsetdash{}{0pt}%
\pgfpathmoveto{\pgfqpoint{4.626321in}{1.394277in}}%
\pgfpathlineto{\pgfqpoint{4.640081in}{1.389747in}}%
\pgfpathlineto{\pgfqpoint{4.653849in}{1.385240in}}%
\pgfpathlineto{\pgfqpoint{4.667622in}{1.380755in}}%
\pgfpathlineto{\pgfqpoint{4.681403in}{1.376293in}}%
\pgfpathlineto{\pgfqpoint{4.673791in}{1.371131in}}%
\pgfpathlineto{\pgfqpoint{4.666175in}{1.366237in}}%
\pgfpathlineto{\pgfqpoint{4.658556in}{1.361620in}}%
\pgfpathlineto{\pgfqpoint{4.650934in}{1.357288in}}%
\pgfpathlineto{\pgfqpoint{4.637139in}{1.362066in}}%
\pgfpathlineto{\pgfqpoint{4.623350in}{1.366867in}}%
\pgfpathlineto{\pgfqpoint{4.609568in}{1.371690in}}%
\pgfpathlineto{\pgfqpoint{4.595792in}{1.376536in}}%
\pgfpathlineto{\pgfqpoint{4.603430in}{1.380547in}}%
\pgfpathlineto{\pgfqpoint{4.611064in}{1.384846in}}%
\pgfpathlineto{\pgfqpoint{4.618694in}{1.389426in}}%
\pgfpathlineto{\pgfqpoint{4.626321in}{1.394277in}}%
\pgfpathclose%
\pgfusepath{fill}%
\end{pgfscope}%
\begin{pgfscope}%
\pgfpathrectangle{\pgfqpoint{1.254980in}{0.150000in}}{\pgfqpoint{5.490039in}{5.490039in}}%
\pgfusepath{clip}%
\pgfsetbuttcap%
\pgfsetroundjoin%
\definecolor{currentfill}{rgb}{0.262138,0.242286,0.520837}%
\pgfsetfillcolor{currentfill}%
\pgfsetfillopacity{0.700000}%
\pgfsetlinewidth{0.000000pt}%
\definecolor{currentstroke}{rgb}{0.000000,0.000000,0.000000}%
\pgfsetstrokecolor{currentstroke}%
\pgfsetdash{}{0pt}%
\pgfpathmoveto{\pgfqpoint{3.651040in}{1.776162in}}%
\pgfpathlineto{\pgfqpoint{3.664603in}{1.768491in}}%
\pgfpathlineto{\pgfqpoint{3.678170in}{1.760846in}}%
\pgfpathlineto{\pgfqpoint{3.691741in}{1.753226in}}%
\pgfpathlineto{\pgfqpoint{3.705317in}{1.745632in}}%
\pgfpathlineto{\pgfqpoint{3.697181in}{1.753242in}}%
\pgfpathlineto{\pgfqpoint{3.689030in}{1.761364in}}%
\pgfpathlineto{\pgfqpoint{3.680864in}{1.770008in}}%
\pgfpathlineto{\pgfqpoint{3.672682in}{1.779185in}}%
\pgfpathlineto{\pgfqpoint{3.659070in}{1.787168in}}%
\pgfpathlineto{\pgfqpoint{3.645462in}{1.795176in}}%
\pgfpathlineto{\pgfqpoint{3.631858in}{1.803209in}}%
\pgfpathlineto{\pgfqpoint{3.618258in}{1.811268in}}%
\pgfpathlineto{\pgfqpoint{3.626478in}{1.801696in}}%
\pgfpathlineto{\pgfqpoint{3.634682in}{1.792662in}}%
\pgfpathlineto{\pgfqpoint{3.642869in}{1.784154in}}%
\pgfpathlineto{\pgfqpoint{3.651040in}{1.776162in}}%
\pgfpathclose%
\pgfusepath{fill}%
\end{pgfscope}%
\begin{pgfscope}%
\pgfpathrectangle{\pgfqpoint{1.254980in}{0.150000in}}{\pgfqpoint{5.490039in}{5.490039in}}%
\pgfusepath{clip}%
\pgfsetbuttcap%
\pgfsetroundjoin%
\definecolor{currentfill}{rgb}{0.273809,0.031497,0.358853}%
\pgfsetfillcolor{currentfill}%
\pgfsetfillopacity{0.700000}%
\pgfsetlinewidth{0.000000pt}%
\definecolor{currentstroke}{rgb}{0.000000,0.000000,0.000000}%
\pgfsetstrokecolor{currentstroke}%
\pgfsetdash{}{0pt}%
\pgfpathmoveto{\pgfqpoint{4.766951in}{1.383057in}}%
\pgfpathlineto{\pgfqpoint{4.780752in}{1.379011in}}%
\pgfpathlineto{\pgfqpoint{4.794560in}{1.374988in}}%
\pgfpathlineto{\pgfqpoint{4.808374in}{1.370987in}}%
\pgfpathlineto{\pgfqpoint{4.822196in}{1.367008in}}%
\pgfpathlineto{\pgfqpoint{4.814622in}{1.360256in}}%
\pgfpathlineto{\pgfqpoint{4.807045in}{1.353733in}}%
\pgfpathlineto{\pgfqpoint{4.799465in}{1.347448in}}%
\pgfpathlineto{\pgfqpoint{4.791883in}{1.341410in}}%
\pgfpathlineto{\pgfqpoint{4.778049in}{1.345691in}}%
\pgfpathlineto{\pgfqpoint{4.764223in}{1.349995in}}%
\pgfpathlineto{\pgfqpoint{4.750403in}{1.354322in}}%
\pgfpathlineto{\pgfqpoint{4.736589in}{1.358671in}}%
\pgfpathlineto{\pgfqpoint{4.744184in}{1.364401in}}%
\pgfpathlineto{\pgfqpoint{4.751776in}{1.370381in}}%
\pgfpathlineto{\pgfqpoint{4.759365in}{1.376603in}}%
\pgfpathlineto{\pgfqpoint{4.766951in}{1.383057in}}%
\pgfpathclose%
\pgfusepath{fill}%
\end{pgfscope}%
\begin{pgfscope}%
\pgfpathrectangle{\pgfqpoint{1.254980in}{0.150000in}}{\pgfqpoint{5.490039in}{5.490039in}}%
\pgfusepath{clip}%
\pgfsetbuttcap%
\pgfsetroundjoin%
\definecolor{currentfill}{rgb}{0.225863,0.330805,0.547314}%
\pgfsetfillcolor{currentfill}%
\pgfsetfillopacity{0.700000}%
\pgfsetlinewidth{0.000000pt}%
\definecolor{currentstroke}{rgb}{0.000000,0.000000,0.000000}%
\pgfsetstrokecolor{currentstroke}%
\pgfsetdash{}{0pt}%
\pgfpathmoveto{\pgfqpoint{3.347050in}{1.977956in}}%
\pgfpathlineto{\pgfqpoint{3.360576in}{1.969366in}}%
\pgfpathlineto{\pgfqpoint{3.374106in}{1.960803in}}%
\pgfpathlineto{\pgfqpoint{3.387639in}{1.952268in}}%
\pgfpathlineto{\pgfqpoint{3.401176in}{1.943761in}}%
\pgfpathlineto{\pgfqpoint{3.392776in}{1.955479in}}%
\pgfpathlineto{\pgfqpoint{3.384356in}{1.967774in}}%
\pgfpathlineto{\pgfqpoint{3.375914in}{1.980658in}}%
\pgfpathlineto{\pgfqpoint{3.367451in}{1.994144in}}%
\pgfpathlineto{\pgfqpoint{3.353871in}{2.003058in}}%
\pgfpathlineto{\pgfqpoint{3.340294in}{2.012000in}}%
\pgfpathlineto{\pgfqpoint{3.326720in}{2.020969in}}%
\pgfpathlineto{\pgfqpoint{3.313149in}{2.029966in}}%
\pgfpathlineto{\pgfqpoint{3.321658in}{2.016068in}}%
\pgfpathlineto{\pgfqpoint{3.330144in}{2.002775in}}%
\pgfpathlineto{\pgfqpoint{3.338607in}{1.990075in}}%
\pgfpathlineto{\pgfqpoint{3.347050in}{1.977956in}}%
\pgfpathclose%
\pgfusepath{fill}%
\end{pgfscope}%
\begin{pgfscope}%
\pgfpathrectangle{\pgfqpoint{1.254980in}{0.150000in}}{\pgfqpoint{5.490039in}{5.490039in}}%
\pgfusepath{clip}%
\pgfsetbuttcap%
\pgfsetroundjoin%
\definecolor{currentfill}{rgb}{0.277941,0.056324,0.381191}%
\pgfsetfillcolor{currentfill}%
\pgfsetfillopacity{0.700000}%
\pgfsetlinewidth{0.000000pt}%
\definecolor{currentstroke}{rgb}{0.000000,0.000000,0.000000}%
\pgfsetstrokecolor{currentstroke}%
\pgfsetdash{}{0pt}%
\pgfpathmoveto{\pgfqpoint{4.485810in}{1.416120in}}%
\pgfpathlineto{\pgfqpoint{4.499536in}{1.411092in}}%
\pgfpathlineto{\pgfqpoint{4.513268in}{1.406087in}}%
\pgfpathlineto{\pgfqpoint{4.527007in}{1.401105in}}%
\pgfpathlineto{\pgfqpoint{4.540751in}{1.396146in}}%
\pgfpathlineto{\pgfqpoint{4.533093in}{1.392756in}}%
\pgfpathlineto{\pgfqpoint{4.525430in}{1.389675in}}%
\pgfpathlineto{\pgfqpoint{4.517762in}{1.386912in}}%
\pgfpathlineto{\pgfqpoint{4.510089in}{1.384475in}}%
\pgfpathlineto{\pgfqpoint{4.496327in}{1.389763in}}%
\pgfpathlineto{\pgfqpoint{4.482571in}{1.395074in}}%
\pgfpathlineto{\pgfqpoint{4.468821in}{1.400408in}}%
\pgfpathlineto{\pgfqpoint{4.455076in}{1.405764in}}%
\pgfpathlineto{\pgfqpoint{4.462767in}{1.407867in}}%
\pgfpathlineto{\pgfqpoint{4.470453in}{1.410300in}}%
\pgfpathlineto{\pgfqpoint{4.478134in}{1.413054in}}%
\pgfpathlineto{\pgfqpoint{4.485810in}{1.416120in}}%
\pgfpathclose%
\pgfusepath{fill}%
\end{pgfscope}%
\begin{pgfscope}%
\pgfpathrectangle{\pgfqpoint{1.254980in}{0.150000in}}{\pgfqpoint{5.490039in}{5.490039in}}%
\pgfusepath{clip}%
\pgfsetbuttcap%
\pgfsetroundjoin%
\definecolor{currentfill}{rgb}{0.277018,0.050344,0.375715}%
\pgfsetfillcolor{currentfill}%
\pgfsetfillopacity{0.700000}%
\pgfsetlinewidth{0.000000pt}%
\definecolor{currentstroke}{rgb}{0.000000,0.000000,0.000000}%
\pgfsetstrokecolor{currentstroke}%
\pgfsetdash{}{0pt}%
\pgfpathmoveto{\pgfqpoint{5.134554in}{1.416052in}}%
\pgfpathlineto{\pgfqpoint{5.148474in}{1.413291in}}%
\pgfpathlineto{\pgfqpoint{5.162403in}{1.410553in}}%
\pgfpathlineto{\pgfqpoint{5.176339in}{1.407838in}}%
\pgfpathlineto{\pgfqpoint{5.168833in}{1.397656in}}%
\pgfpathlineto{\pgfqpoint{5.161326in}{1.387604in}}%
\pgfpathlineto{\pgfqpoint{5.153816in}{1.377687in}}%
\pgfpathlineto{\pgfqpoint{5.146304in}{1.367914in}}%
\pgfpathlineto{\pgfqpoint{5.132362in}{1.370895in}}%
\pgfpathlineto{\pgfqpoint{5.118427in}{1.373899in}}%
\pgfpathlineto{\pgfqpoint{5.104500in}{1.376925in}}%
\pgfpathlineto{\pgfqpoint{5.112017in}{1.386496in}}%
\pgfpathlineto{\pgfqpoint{5.119531in}{1.396212in}}%
\pgfpathlineto{\pgfqpoint{5.127044in}{1.406066in}}%
\pgfpathlineto{\pgfqpoint{5.134554in}{1.416052in}}%
\pgfpathclose%
\pgfusepath{fill}%
\end{pgfscope}%
\begin{pgfscope}%
\pgfpathrectangle{\pgfqpoint{1.254980in}{0.150000in}}{\pgfqpoint{5.490039in}{5.490039in}}%
\pgfusepath{clip}%
\pgfsetbuttcap%
\pgfsetroundjoin%
\definecolor{currentfill}{rgb}{0.122312,0.633153,0.530398}%
\pgfsetfillcolor{currentfill}%
\pgfsetfillopacity{0.700000}%
\pgfsetlinewidth{0.000000pt}%
\definecolor{currentstroke}{rgb}{0.000000,0.000000,0.000000}%
\pgfsetstrokecolor{currentstroke}%
\pgfsetdash{}{0pt}%
\pgfpathmoveto{\pgfqpoint{2.412148in}{2.760592in}}%
\pgfpathlineto{\pgfqpoint{2.425641in}{2.748994in}}%
\pgfpathlineto{\pgfqpoint{2.439134in}{2.737439in}}%
\pgfpathlineto{\pgfqpoint{2.452629in}{2.725926in}}%
\pgfpathlineto{\pgfqpoint{2.466124in}{2.714455in}}%
\pgfpathlineto{\pgfqpoint{2.456642in}{2.738203in}}%
\pgfpathlineto{\pgfqpoint{2.447118in}{2.762705in}}%
\pgfpathlineto{\pgfqpoint{2.437553in}{2.787974in}}%
\pgfpathlineto{\pgfqpoint{2.427944in}{2.814026in}}%
\pgfpathlineto{\pgfqpoint{2.414383in}{2.825956in}}%
\pgfpathlineto{\pgfqpoint{2.400823in}{2.837928in}}%
\pgfpathlineto{\pgfqpoint{2.387264in}{2.849943in}}%
\pgfpathlineto{\pgfqpoint{2.373705in}{2.862001in}}%
\pgfpathlineto{\pgfqpoint{2.383381in}{2.835482in}}%
\pgfpathlineto{\pgfqpoint{2.393013in}{2.809751in}}%
\pgfpathlineto{\pgfqpoint{2.402602in}{2.784792in}}%
\pgfpathlineto{\pgfqpoint{2.412148in}{2.760592in}}%
\pgfpathclose%
\pgfusepath{fill}%
\end{pgfscope}%
\begin{pgfscope}%
\pgfpathrectangle{\pgfqpoint{1.254980in}{0.150000in}}{\pgfqpoint{5.490039in}{5.490039in}}%
\pgfusepath{clip}%
\pgfsetbuttcap%
\pgfsetroundjoin%
\definecolor{currentfill}{rgb}{0.273809,0.031497,0.358853}%
\pgfsetfillcolor{currentfill}%
\pgfsetfillopacity{0.700000}%
\pgfsetlinewidth{0.000000pt}%
\definecolor{currentstroke}{rgb}{0.000000,0.000000,0.000000}%
\pgfsetstrokecolor{currentstroke}%
\pgfsetdash{}{0pt}%
\pgfpathmoveto{\pgfqpoint{4.907778in}{1.381644in}}%
\pgfpathlineto{\pgfqpoint{4.921624in}{1.378069in}}%
\pgfpathlineto{\pgfqpoint{4.935477in}{1.374516in}}%
\pgfpathlineto{\pgfqpoint{4.949338in}{1.370986in}}%
\pgfpathlineto{\pgfqpoint{4.963206in}{1.367479in}}%
\pgfpathlineto{\pgfqpoint{4.955662in}{1.359309in}}%
\pgfpathlineto{\pgfqpoint{4.948117in}{1.351334in}}%
\pgfpathlineto{\pgfqpoint{4.940568in}{1.343559in}}%
\pgfpathlineto{\pgfqpoint{4.933018in}{1.335993in}}%
\pgfpathlineto{\pgfqpoint{4.919141in}{1.339791in}}%
\pgfpathlineto{\pgfqpoint{4.905270in}{1.343612in}}%
\pgfpathlineto{\pgfqpoint{4.891407in}{1.347455in}}%
\pgfpathlineto{\pgfqpoint{4.877551in}{1.351321in}}%
\pgfpathlineto{\pgfqpoint{4.885111in}{1.358591in}}%
\pgfpathlineto{\pgfqpoint{4.892669in}{1.366074in}}%
\pgfpathlineto{\pgfqpoint{4.900225in}{1.373760in}}%
\pgfpathlineto{\pgfqpoint{4.907778in}{1.381644in}}%
\pgfpathclose%
\pgfusepath{fill}%
\end{pgfscope}%
\begin{pgfscope}%
\pgfpathrectangle{\pgfqpoint{1.254980in}{0.150000in}}{\pgfqpoint{5.490039in}{5.490039in}}%
\pgfusepath{clip}%
\pgfsetbuttcap%
\pgfsetroundjoin%
\definecolor{currentfill}{rgb}{0.174274,0.445044,0.557792}%
\pgfsetfillcolor{currentfill}%
\pgfsetfillopacity{0.700000}%
\pgfsetlinewidth{0.000000pt}%
\definecolor{currentstroke}{rgb}{0.000000,0.000000,0.000000}%
\pgfsetstrokecolor{currentstroke}%
\pgfsetdash{}{0pt}%
\pgfpathmoveto{\pgfqpoint{2.988345in}{2.254568in}}%
\pgfpathlineto{\pgfqpoint{3.001847in}{2.244863in}}%
\pgfpathlineto{\pgfqpoint{3.015351in}{2.235189in}}%
\pgfpathlineto{\pgfqpoint{3.028859in}{2.225547in}}%
\pgfpathlineto{\pgfqpoint{3.042368in}{2.215935in}}%
\pgfpathlineto{\pgfqpoint{3.033594in}{2.232549in}}%
\pgfpathlineto{\pgfqpoint{3.024792in}{2.249814in}}%
\pgfpathlineto{\pgfqpoint{3.015960in}{2.267744in}}%
\pgfpathlineto{\pgfqpoint{3.007100in}{2.286353in}}%
\pgfpathlineto{\pgfqpoint{2.993538in}{2.296392in}}%
\pgfpathlineto{\pgfqpoint{2.979978in}{2.306463in}}%
\pgfpathlineto{\pgfqpoint{2.966421in}{2.316565in}}%
\pgfpathlineto{\pgfqpoint{2.952866in}{2.326699in}}%
\pgfpathlineto{\pgfqpoint{2.961781in}{2.307655in}}%
\pgfpathlineto{\pgfqpoint{2.970665in}{2.289294in}}%
\pgfpathlineto{\pgfqpoint{2.979519in}{2.271603in}}%
\pgfpathlineto{\pgfqpoint{2.988345in}{2.254568in}}%
\pgfpathclose%
\pgfusepath{fill}%
\end{pgfscope}%
\begin{pgfscope}%
\pgfpathrectangle{\pgfqpoint{1.254980in}{0.150000in}}{\pgfqpoint{5.490039in}{5.490039in}}%
\pgfusepath{clip}%
\pgfsetbuttcap%
\pgfsetroundjoin%
\definecolor{currentfill}{rgb}{0.283197,0.115680,0.436115}%
\pgfsetfillcolor{currentfill}%
\pgfsetfillopacity{0.700000}%
\pgfsetlinewidth{0.000000pt}%
\definecolor{currentstroke}{rgb}{0.000000,0.000000,0.000000}%
\pgfsetstrokecolor{currentstroke}%
\pgfsetdash{}{0pt}%
\pgfpathmoveto{\pgfqpoint{4.150150in}{1.519644in}}%
\pgfpathlineto{\pgfqpoint{4.163803in}{1.513482in}}%
\pgfpathlineto{\pgfqpoint{4.177463in}{1.507345in}}%
\pgfpathlineto{\pgfqpoint{4.191127in}{1.501230in}}%
\pgfpathlineto{\pgfqpoint{4.204797in}{1.495139in}}%
\pgfpathlineto{\pgfqpoint{4.196987in}{1.496226in}}%
\pgfpathlineto{\pgfqpoint{4.189170in}{1.497712in}}%
\pgfpathlineto{\pgfqpoint{4.181344in}{1.499607in}}%
\pgfpathlineto{\pgfqpoint{4.173511in}{1.501921in}}%
\pgfpathlineto{\pgfqpoint{4.159816in}{1.508369in}}%
\pgfpathlineto{\pgfqpoint{4.146126in}{1.514840in}}%
\pgfpathlineto{\pgfqpoint{4.132441in}{1.521335in}}%
\pgfpathlineto{\pgfqpoint{4.118762in}{1.527853in}}%
\pgfpathlineto{\pgfqpoint{4.126622in}{1.525177in}}%
\pgfpathlineto{\pgfqpoint{4.134473in}{1.522923in}}%
\pgfpathlineto{\pgfqpoint{4.142316in}{1.521082in}}%
\pgfpathlineto{\pgfqpoint{4.150150in}{1.519644in}}%
\pgfpathclose%
\pgfusepath{fill}%
\end{pgfscope}%
\begin{pgfscope}%
\pgfpathrectangle{\pgfqpoint{1.254980in}{0.150000in}}{\pgfqpoint{5.490039in}{5.490039in}}%
\pgfusepath{clip}%
\pgfsetbuttcap%
\pgfsetroundjoin%
\definecolor{currentfill}{rgb}{0.280894,0.078907,0.402329}%
\pgfsetfillcolor{currentfill}%
\pgfsetfillopacity{0.700000}%
\pgfsetlinewidth{0.000000pt}%
\definecolor{currentstroke}{rgb}{0.000000,0.000000,0.000000}%
\pgfsetstrokecolor{currentstroke}%
\pgfsetdash{}{0pt}%
\pgfpathmoveto{\pgfqpoint{4.345333in}{1.449441in}}%
\pgfpathlineto{\pgfqpoint{4.359031in}{1.443901in}}%
\pgfpathlineto{\pgfqpoint{4.372734in}{1.438384in}}%
\pgfpathlineto{\pgfqpoint{4.386443in}{1.432890in}}%
\pgfpathlineto{\pgfqpoint{4.400158in}{1.427419in}}%
\pgfpathlineto{\pgfqpoint{4.392442in}{1.425992in}}%
\pgfpathlineto{\pgfqpoint{4.384721in}{1.424917in}}%
\pgfpathlineto{\pgfqpoint{4.376993in}{1.424201in}}%
\pgfpathlineto{\pgfqpoint{4.369259in}{1.423855in}}%
\pgfpathlineto{\pgfqpoint{4.355523in}{1.429669in}}%
\pgfpathlineto{\pgfqpoint{4.341793in}{1.435505in}}%
\pgfpathlineto{\pgfqpoint{4.328069in}{1.441365in}}%
\pgfpathlineto{\pgfqpoint{4.314350in}{1.447247in}}%
\pgfpathlineto{\pgfqpoint{4.322105in}{1.447245in}}%
\pgfpathlineto{\pgfqpoint{4.329855in}{1.447616in}}%
\pgfpathlineto{\pgfqpoint{4.337597in}{1.448351in}}%
\pgfpathlineto{\pgfqpoint{4.345333in}{1.449441in}}%
\pgfpathclose%
\pgfusepath{fill}%
\end{pgfscope}%
\begin{pgfscope}%
\pgfpathrectangle{\pgfqpoint{1.254980in}{0.150000in}}{\pgfqpoint{5.490039in}{5.490039in}}%
\pgfusepath{clip}%
\pgfsetbuttcap%
\pgfsetroundjoin%
\definecolor{currentfill}{rgb}{0.119699,0.618490,0.536347}%
\pgfsetfillcolor{currentfill}%
\pgfsetfillopacity{0.700000}%
\pgfsetlinewidth{0.000000pt}%
\definecolor{currentstroke}{rgb}{0.000000,0.000000,0.000000}%
\pgfsetstrokecolor{currentstroke}%
\pgfsetdash{}{0pt}%
\pgfpathmoveto{\pgfqpoint{2.466124in}{2.714455in}}%
\pgfpathlineto{\pgfqpoint{2.479620in}{2.703025in}}%
\pgfpathlineto{\pgfqpoint{2.493117in}{2.691637in}}%
\pgfpathlineto{\pgfqpoint{2.506615in}{2.680289in}}%
\pgfpathlineto{\pgfqpoint{2.520113in}{2.668982in}}%
\pgfpathlineto{\pgfqpoint{2.510695in}{2.692280in}}%
\pgfpathlineto{\pgfqpoint{2.501236in}{2.716326in}}%
\pgfpathlineto{\pgfqpoint{2.491736in}{2.741135in}}%
\pgfpathlineto{\pgfqpoint{2.482194in}{2.766721in}}%
\pgfpathlineto{\pgfqpoint{2.468630in}{2.778486in}}%
\pgfpathlineto{\pgfqpoint{2.455067in}{2.790291in}}%
\pgfpathlineto{\pgfqpoint{2.441505in}{2.802138in}}%
\pgfpathlineto{\pgfqpoint{2.427944in}{2.814026in}}%
\pgfpathlineto{\pgfqpoint{2.437553in}{2.787974in}}%
\pgfpathlineto{\pgfqpoint{2.447118in}{2.762705in}}%
\pgfpathlineto{\pgfqpoint{2.456642in}{2.738203in}}%
\pgfpathlineto{\pgfqpoint{2.466124in}{2.714455in}}%
\pgfpathclose%
\pgfusepath{fill}%
\end{pgfscope}%
\begin{pgfscope}%
\pgfpathrectangle{\pgfqpoint{1.254980in}{0.150000in}}{\pgfqpoint{5.490039in}{5.490039in}}%
\pgfusepath{clip}%
\pgfsetbuttcap%
\pgfsetroundjoin%
\definecolor{currentfill}{rgb}{0.229739,0.322361,0.545706}%
\pgfsetfillcolor{currentfill}%
\pgfsetfillopacity{0.700000}%
\pgfsetlinewidth{0.000000pt}%
\definecolor{currentstroke}{rgb}{0.000000,0.000000,0.000000}%
\pgfsetstrokecolor{currentstroke}%
\pgfsetdash{}{0pt}%
\pgfpathmoveto{\pgfqpoint{3.401176in}{1.943761in}}%
\pgfpathlineto{\pgfqpoint{3.414716in}{1.935280in}}%
\pgfpathlineto{\pgfqpoint{3.428260in}{1.926827in}}%
\pgfpathlineto{\pgfqpoint{3.441807in}{1.918400in}}%
\pgfpathlineto{\pgfqpoint{3.455358in}{1.910001in}}%
\pgfpathlineto{\pgfqpoint{3.447000in}{1.921319in}}%
\pgfpathlineto{\pgfqpoint{3.438623in}{1.933210in}}%
\pgfpathlineto{\pgfqpoint{3.430225in}{1.945686in}}%
\pgfpathlineto{\pgfqpoint{3.421806in}{1.958759in}}%
\pgfpathlineto{\pgfqpoint{3.408212in}{1.967565in}}%
\pgfpathlineto{\pgfqpoint{3.394622in}{1.976397in}}%
\pgfpathlineto{\pgfqpoint{3.381035in}{1.985257in}}%
\pgfpathlineto{\pgfqpoint{3.367451in}{1.994144in}}%
\pgfpathlineto{\pgfqpoint{3.375914in}{1.980658in}}%
\pgfpathlineto{\pgfqpoint{3.384356in}{1.967774in}}%
\pgfpathlineto{\pgfqpoint{3.392776in}{1.955479in}}%
\pgfpathlineto{\pgfqpoint{3.401176in}{1.943761in}}%
\pgfpathclose%
\pgfusepath{fill}%
\end{pgfscope}%
\begin{pgfscope}%
\pgfpathrectangle{\pgfqpoint{1.254980in}{0.150000in}}{\pgfqpoint{5.490039in}{5.490039in}}%
\pgfusepath{clip}%
\pgfsetbuttcap%
\pgfsetroundjoin%
\definecolor{currentfill}{rgb}{0.280255,0.165693,0.476498}%
\pgfsetfillcolor{currentfill}%
\pgfsetfillopacity{0.700000}%
\pgfsetlinewidth{0.000000pt}%
\definecolor{currentstroke}{rgb}{0.000000,0.000000,0.000000}%
\pgfsetstrokecolor{currentstroke}%
\pgfsetdash{}{0pt}%
\pgfpathmoveto{\pgfqpoint{3.954998in}{1.607916in}}%
\pgfpathlineto{\pgfqpoint{3.968618in}{1.601113in}}%
\pgfpathlineto{\pgfqpoint{3.982243in}{1.594334in}}%
\pgfpathlineto{\pgfqpoint{3.995873in}{1.587578in}}%
\pgfpathlineto{\pgfqpoint{4.009507in}{1.580847in}}%
\pgfpathlineto{\pgfqpoint{4.001582in}{1.584687in}}%
\pgfpathlineto{\pgfqpoint{3.993646in}{1.588977in}}%
\pgfpathlineto{\pgfqpoint{3.985700in}{1.593728in}}%
\pgfpathlineto{\pgfqpoint{3.977742in}{1.598949in}}%
\pgfpathlineto{\pgfqpoint{3.964077in}{1.606052in}}%
\pgfpathlineto{\pgfqpoint{3.950418in}{1.613179in}}%
\pgfpathlineto{\pgfqpoint{3.936763in}{1.620330in}}%
\pgfpathlineto{\pgfqpoint{3.923112in}{1.627505in}}%
\pgfpathlineto{\pgfqpoint{3.931101in}{1.621906in}}%
\pgfpathlineto{\pgfqpoint{3.939078in}{1.616782in}}%
\pgfpathlineto{\pgfqpoint{3.947044in}{1.612122in}}%
\pgfpathlineto{\pgfqpoint{3.954998in}{1.607916in}}%
\pgfpathclose%
\pgfusepath{fill}%
\end{pgfscope}%
\begin{pgfscope}%
\pgfpathrectangle{\pgfqpoint{1.254980in}{0.150000in}}{\pgfqpoint{5.490039in}{5.490039in}}%
\pgfusepath{clip}%
\pgfsetbuttcap%
\pgfsetroundjoin%
\definecolor{currentfill}{rgb}{0.265145,0.232956,0.516599}%
\pgfsetfillcolor{currentfill}%
\pgfsetfillopacity{0.700000}%
\pgfsetlinewidth{0.000000pt}%
\definecolor{currentstroke}{rgb}{0.000000,0.000000,0.000000}%
\pgfsetstrokecolor{currentstroke}%
\pgfsetdash{}{0pt}%
\pgfpathmoveto{\pgfqpoint{3.705317in}{1.745632in}}%
\pgfpathlineto{\pgfqpoint{3.718896in}{1.738063in}}%
\pgfpathlineto{\pgfqpoint{3.732480in}{1.730519in}}%
\pgfpathlineto{\pgfqpoint{3.746068in}{1.723000in}}%
\pgfpathlineto{\pgfqpoint{3.759660in}{1.715506in}}%
\pgfpathlineto{\pgfqpoint{3.751560in}{1.722734in}}%
\pgfpathlineto{\pgfqpoint{3.743444in}{1.730470in}}%
\pgfpathlineto{\pgfqpoint{3.735314in}{1.738724in}}%
\pgfpathlineto{\pgfqpoint{3.727169in}{1.747508in}}%
\pgfpathlineto{\pgfqpoint{3.713541in}{1.755390in}}%
\pgfpathlineto{\pgfqpoint{3.699917in}{1.763296in}}%
\pgfpathlineto{\pgfqpoint{3.686297in}{1.771228in}}%
\pgfpathlineto{\pgfqpoint{3.672682in}{1.779185in}}%
\pgfpathlineto{\pgfqpoint{3.680864in}{1.770008in}}%
\pgfpathlineto{\pgfqpoint{3.689030in}{1.761364in}}%
\pgfpathlineto{\pgfqpoint{3.697181in}{1.753242in}}%
\pgfpathlineto{\pgfqpoint{3.705317in}{1.745632in}}%
\pgfpathclose%
\pgfusepath{fill}%
\end{pgfscope}%
\begin{pgfscope}%
\pgfpathrectangle{\pgfqpoint{1.254980in}{0.150000in}}{\pgfqpoint{5.490039in}{5.490039in}}%
\pgfusepath{clip}%
\pgfsetbuttcap%
\pgfsetroundjoin%
\definecolor{currentfill}{rgb}{0.274952,0.037752,0.364543}%
\pgfsetfillcolor{currentfill}%
\pgfsetfillopacity{0.700000}%
\pgfsetlinewidth{0.000000pt}%
\definecolor{currentstroke}{rgb}{0.000000,0.000000,0.000000}%
\pgfsetstrokecolor{currentstroke}%
\pgfsetdash{}{0pt}%
\pgfpathmoveto{\pgfqpoint{5.048868in}{1.389256in}}%
\pgfpathlineto{\pgfqpoint{5.062765in}{1.386140in}}%
\pgfpathlineto{\pgfqpoint{5.076669in}{1.383046in}}%
\pgfpathlineto{\pgfqpoint{5.090581in}{1.379974in}}%
\pgfpathlineto{\pgfqpoint{5.104500in}{1.376925in}}%
\pgfpathlineto{\pgfqpoint{5.096981in}{1.367506in}}%
\pgfpathlineto{\pgfqpoint{5.089461in}{1.358246in}}%
\pgfpathlineto{\pgfqpoint{5.081938in}{1.349152in}}%
\pgfpathlineto{\pgfqpoint{5.074413in}{1.340230in}}%
\pgfpathlineto{\pgfqpoint{5.060486in}{1.343558in}}%
\pgfpathlineto{\pgfqpoint{5.046567in}{1.346907in}}%
\pgfpathlineto{\pgfqpoint{5.032655in}{1.350280in}}%
\pgfpathlineto{\pgfqpoint{5.018750in}{1.353674in}}%
\pgfpathlineto{\pgfqpoint{5.026283in}{1.362313in}}%
\pgfpathlineto{\pgfqpoint{5.033814in}{1.371127in}}%
\pgfpathlineto{\pgfqpoint{5.041342in}{1.380111in}}%
\pgfpathlineto{\pgfqpoint{5.048868in}{1.389256in}}%
\pgfpathclose%
\pgfusepath{fill}%
\end{pgfscope}%
\begin{pgfscope}%
\pgfpathrectangle{\pgfqpoint{1.254980in}{0.150000in}}{\pgfqpoint{5.490039in}{5.490039in}}%
\pgfusepath{clip}%
\pgfsetbuttcap%
\pgfsetroundjoin%
\definecolor{currentfill}{rgb}{0.179019,0.433756,0.557430}%
\pgfsetfillcolor{currentfill}%
\pgfsetfillopacity{0.700000}%
\pgfsetlinewidth{0.000000pt}%
\definecolor{currentstroke}{rgb}{0.000000,0.000000,0.000000}%
\pgfsetstrokecolor{currentstroke}%
\pgfsetdash{}{0pt}%
\pgfpathmoveto{\pgfqpoint{3.042368in}{2.215935in}}%
\pgfpathlineto{\pgfqpoint{3.055881in}{2.206355in}}%
\pgfpathlineto{\pgfqpoint{3.069396in}{2.196805in}}%
\pgfpathlineto{\pgfqpoint{3.082913in}{2.187286in}}%
\pgfpathlineto{\pgfqpoint{3.096433in}{2.177797in}}%
\pgfpathlineto{\pgfqpoint{3.087709in}{2.193989in}}%
\pgfpathlineto{\pgfqpoint{3.078958in}{2.210829in}}%
\pgfpathlineto{\pgfqpoint{3.070179in}{2.228329in}}%
\pgfpathlineto{\pgfqpoint{3.061372in}{2.246503in}}%
\pgfpathlineto{\pgfqpoint{3.047800in}{2.256420in}}%
\pgfpathlineto{\pgfqpoint{3.034231in}{2.266367in}}%
\pgfpathlineto{\pgfqpoint{3.020664in}{2.276344in}}%
\pgfpathlineto{\pgfqpoint{3.007100in}{2.286353in}}%
\pgfpathlineto{\pgfqpoint{3.015960in}{2.267744in}}%
\pgfpathlineto{\pgfqpoint{3.024792in}{2.249814in}}%
\pgfpathlineto{\pgfqpoint{3.033594in}{2.232549in}}%
\pgfpathlineto{\pgfqpoint{3.042368in}{2.215935in}}%
\pgfpathclose%
\pgfusepath{fill}%
\end{pgfscope}%
\begin{pgfscope}%
\pgfpathrectangle{\pgfqpoint{1.254980in}{0.150000in}}{\pgfqpoint{5.490039in}{5.490039in}}%
\pgfusepath{clip}%
\pgfsetbuttcap%
\pgfsetroundjoin%
\definecolor{currentfill}{rgb}{0.274952,0.037752,0.364543}%
\pgfsetfillcolor{currentfill}%
\pgfsetfillopacity{0.700000}%
\pgfsetlinewidth{0.000000pt}%
\definecolor{currentstroke}{rgb}{0.000000,0.000000,0.000000}%
\pgfsetstrokecolor{currentstroke}%
\pgfsetdash{}{0pt}%
\pgfpathmoveto{\pgfqpoint{4.681403in}{1.376293in}}%
\pgfpathlineto{\pgfqpoint{4.695189in}{1.371853in}}%
\pgfpathlineto{\pgfqpoint{4.708983in}{1.367437in}}%
\pgfpathlineto{\pgfqpoint{4.722783in}{1.363042in}}%
\pgfpathlineto{\pgfqpoint{4.736589in}{1.358671in}}%
\pgfpathlineto{\pgfqpoint{4.728991in}{1.353198in}}%
\pgfpathlineto{\pgfqpoint{4.721390in}{1.347989in}}%
\pgfpathlineto{\pgfqpoint{4.713785in}{1.343055in}}%
\pgfpathlineto{\pgfqpoint{4.706177in}{1.338402in}}%
\pgfpathlineto{\pgfqpoint{4.692357in}{1.343090in}}%
\pgfpathlineto{\pgfqpoint{4.678543in}{1.347800in}}%
\pgfpathlineto{\pgfqpoint{4.664735in}{1.352533in}}%
\pgfpathlineto{\pgfqpoint{4.650934in}{1.357288in}}%
\pgfpathlineto{\pgfqpoint{4.658556in}{1.361620in}}%
\pgfpathlineto{\pgfqpoint{4.666175in}{1.366237in}}%
\pgfpathlineto{\pgfqpoint{4.673791in}{1.371131in}}%
\pgfpathlineto{\pgfqpoint{4.681403in}{1.376293in}}%
\pgfpathclose%
\pgfusepath{fill}%
\end{pgfscope}%
\begin{pgfscope}%
\pgfpathrectangle{\pgfqpoint{1.254980in}{0.150000in}}{\pgfqpoint{5.490039in}{5.490039in}}%
\pgfusepath{clip}%
\pgfsetbuttcap%
\pgfsetroundjoin%
\definecolor{currentfill}{rgb}{0.119738,0.603785,0.541400}%
\pgfsetfillcolor{currentfill}%
\pgfsetfillopacity{0.700000}%
\pgfsetlinewidth{0.000000pt}%
\definecolor{currentstroke}{rgb}{0.000000,0.000000,0.000000}%
\pgfsetstrokecolor{currentstroke}%
\pgfsetdash{}{0pt}%
\pgfpathmoveto{\pgfqpoint{2.520113in}{2.668982in}}%
\pgfpathlineto{\pgfqpoint{2.533613in}{2.657715in}}%
\pgfpathlineto{\pgfqpoint{2.547115in}{2.646487in}}%
\pgfpathlineto{\pgfqpoint{2.560617in}{2.635299in}}%
\pgfpathlineto{\pgfqpoint{2.574120in}{2.624150in}}%
\pgfpathlineto{\pgfqpoint{2.564764in}{2.646999in}}%
\pgfpathlineto{\pgfqpoint{2.555369in}{2.670592in}}%
\pgfpathlineto{\pgfqpoint{2.545933in}{2.694942in}}%
\pgfpathlineto{\pgfqpoint{2.536457in}{2.720064in}}%
\pgfpathlineto{\pgfqpoint{2.522889in}{2.731669in}}%
\pgfpathlineto{\pgfqpoint{2.509323in}{2.743313in}}%
\pgfpathlineto{\pgfqpoint{2.495758in}{2.754997in}}%
\pgfpathlineto{\pgfqpoint{2.482194in}{2.766721in}}%
\pgfpathlineto{\pgfqpoint{2.491736in}{2.741135in}}%
\pgfpathlineto{\pgfqpoint{2.501236in}{2.716326in}}%
\pgfpathlineto{\pgfqpoint{2.510695in}{2.692280in}}%
\pgfpathlineto{\pgfqpoint{2.520113in}{2.668982in}}%
\pgfpathclose%
\pgfusepath{fill}%
\end{pgfscope}%
\begin{pgfscope}%
\pgfpathrectangle{\pgfqpoint{1.254980in}{0.150000in}}{\pgfqpoint{5.490039in}{5.490039in}}%
\pgfusepath{clip}%
\pgfsetbuttcap%
\pgfsetroundjoin%
\definecolor{currentfill}{rgb}{0.273809,0.031497,0.358853}%
\pgfsetfillcolor{currentfill}%
\pgfsetfillopacity{0.700000}%
\pgfsetlinewidth{0.000000pt}%
\definecolor{currentstroke}{rgb}{0.000000,0.000000,0.000000}%
\pgfsetstrokecolor{currentstroke}%
\pgfsetdash{}{0pt}%
\pgfpathmoveto{\pgfqpoint{4.822196in}{1.367008in}}%
\pgfpathlineto{\pgfqpoint{4.836024in}{1.363053in}}%
\pgfpathlineto{\pgfqpoint{4.849859in}{1.359119in}}%
\pgfpathlineto{\pgfqpoint{4.863702in}{1.355209in}}%
\pgfpathlineto{\pgfqpoint{4.877551in}{1.351321in}}%
\pgfpathlineto{\pgfqpoint{4.869988in}{1.344269in}}%
\pgfpathlineto{\pgfqpoint{4.862422in}{1.337445in}}%
\pgfpathlineto{\pgfqpoint{4.854854in}{1.330856in}}%
\pgfpathlineto{\pgfqpoint{4.847284in}{1.324509in}}%
\pgfpathlineto{\pgfqpoint{4.833423in}{1.328700in}}%
\pgfpathlineto{\pgfqpoint{4.819570in}{1.332914in}}%
\pgfpathlineto{\pgfqpoint{4.805723in}{1.337151in}}%
\pgfpathlineto{\pgfqpoint{4.791883in}{1.341410in}}%
\pgfpathlineto{\pgfqpoint{4.799465in}{1.347448in}}%
\pgfpathlineto{\pgfqpoint{4.807045in}{1.353733in}}%
\pgfpathlineto{\pgfqpoint{4.814622in}{1.360256in}}%
\pgfpathlineto{\pgfqpoint{4.822196in}{1.367008in}}%
\pgfpathclose%
\pgfusepath{fill}%
\end{pgfscope}%
\begin{pgfscope}%
\pgfpathrectangle{\pgfqpoint{1.254980in}{0.150000in}}{\pgfqpoint{5.490039in}{5.490039in}}%
\pgfusepath{clip}%
\pgfsetbuttcap%
\pgfsetroundjoin%
\definecolor{currentfill}{rgb}{0.277018,0.050344,0.375715}%
\pgfsetfillcolor{currentfill}%
\pgfsetfillopacity{0.700000}%
\pgfsetlinewidth{0.000000pt}%
\definecolor{currentstroke}{rgb}{0.000000,0.000000,0.000000}%
\pgfsetstrokecolor{currentstroke}%
\pgfsetdash{}{0pt}%
\pgfpathmoveto{\pgfqpoint{4.540751in}{1.396146in}}%
\pgfpathlineto{\pgfqpoint{4.554502in}{1.391209in}}%
\pgfpathlineto{\pgfqpoint{4.568259in}{1.386295in}}%
\pgfpathlineto{\pgfqpoint{4.582023in}{1.381404in}}%
\pgfpathlineto{\pgfqpoint{4.595792in}{1.376536in}}%
\pgfpathlineto{\pgfqpoint{4.588150in}{1.372822in}}%
\pgfpathlineto{\pgfqpoint{4.580504in}{1.369414in}}%
\pgfpathlineto{\pgfqpoint{4.572853in}{1.366319in}}%
\pgfpathlineto{\pgfqpoint{4.565199in}{1.363548in}}%
\pgfpathlineto{\pgfqpoint{4.551412in}{1.368746in}}%
\pgfpathlineto{\pgfqpoint{4.537632in}{1.373966in}}%
\pgfpathlineto{\pgfqpoint{4.523858in}{1.379209in}}%
\pgfpathlineto{\pgfqpoint{4.510089in}{1.384475in}}%
\pgfpathlineto{\pgfqpoint{4.517762in}{1.386912in}}%
\pgfpathlineto{\pgfqpoint{4.525430in}{1.389675in}}%
\pgfpathlineto{\pgfqpoint{4.533093in}{1.392756in}}%
\pgfpathlineto{\pgfqpoint{4.540751in}{1.396146in}}%
\pgfpathclose%
\pgfusepath{fill}%
\end{pgfscope}%
\begin{pgfscope}%
\pgfpathrectangle{\pgfqpoint{1.254980in}{0.150000in}}{\pgfqpoint{5.490039in}{5.490039in}}%
\pgfusepath{clip}%
\pgfsetbuttcap%
\pgfsetroundjoin%
\definecolor{currentfill}{rgb}{0.283091,0.110553,0.431554}%
\pgfsetfillcolor{currentfill}%
\pgfsetfillopacity{0.700000}%
\pgfsetlinewidth{0.000000pt}%
\definecolor{currentstroke}{rgb}{0.000000,0.000000,0.000000}%
\pgfsetstrokecolor{currentstroke}%
\pgfsetdash{}{0pt}%
\pgfpathmoveto{\pgfqpoint{4.204797in}{1.495139in}}%
\pgfpathlineto{\pgfqpoint{4.218472in}{1.489071in}}%
\pgfpathlineto{\pgfqpoint{4.232152in}{1.483027in}}%
\pgfpathlineto{\pgfqpoint{4.245838in}{1.477006in}}%
\pgfpathlineto{\pgfqpoint{4.259530in}{1.471008in}}%
\pgfpathlineto{\pgfqpoint{4.251744in}{1.471743in}}%
\pgfpathlineto{\pgfqpoint{4.243951in}{1.472875in}}%
\pgfpathlineto{\pgfqpoint{4.236151in}{1.474412in}}%
\pgfpathlineto{\pgfqpoint{4.228342in}{1.476363in}}%
\pgfpathlineto{\pgfqpoint{4.214627in}{1.482718in}}%
\pgfpathlineto{\pgfqpoint{4.200916in}{1.489096in}}%
\pgfpathlineto{\pgfqpoint{4.187211in}{1.495497in}}%
\pgfpathlineto{\pgfqpoint{4.173511in}{1.501921in}}%
\pgfpathlineto{\pgfqpoint{4.181344in}{1.499607in}}%
\pgfpathlineto{\pgfqpoint{4.189170in}{1.497712in}}%
\pgfpathlineto{\pgfqpoint{4.196987in}{1.496226in}}%
\pgfpathlineto{\pgfqpoint{4.204797in}{1.495139in}}%
\pgfpathclose%
\pgfusepath{fill}%
\end{pgfscope}%
\begin{pgfscope}%
\pgfpathrectangle{\pgfqpoint{1.254980in}{0.150000in}}{\pgfqpoint{5.490039in}{5.490039in}}%
\pgfusepath{clip}%
\pgfsetbuttcap%
\pgfsetroundjoin%
\definecolor{currentfill}{rgb}{0.235526,0.309527,0.542944}%
\pgfsetfillcolor{currentfill}%
\pgfsetfillopacity{0.700000}%
\pgfsetlinewidth{0.000000pt}%
\definecolor{currentstroke}{rgb}{0.000000,0.000000,0.000000}%
\pgfsetstrokecolor{currentstroke}%
\pgfsetdash{}{0pt}%
\pgfpathmoveto{\pgfqpoint{3.455358in}{1.910001in}}%
\pgfpathlineto{\pgfqpoint{3.468913in}{1.901628in}}%
\pgfpathlineto{\pgfqpoint{3.482471in}{1.893282in}}%
\pgfpathlineto{\pgfqpoint{3.496033in}{1.884963in}}%
\pgfpathlineto{\pgfqpoint{3.509598in}{1.876670in}}%
\pgfpathlineto{\pgfqpoint{3.501282in}{1.887589in}}%
\pgfpathlineto{\pgfqpoint{3.492946in}{1.899076in}}%
\pgfpathlineto{\pgfqpoint{3.484590in}{1.911144in}}%
\pgfpathlineto{\pgfqpoint{3.476215in}{1.923805in}}%
\pgfpathlineto{\pgfqpoint{3.462608in}{1.932504in}}%
\pgfpathlineto{\pgfqpoint{3.449004in}{1.941229in}}%
\pgfpathlineto{\pgfqpoint{3.435403in}{1.949980in}}%
\pgfpathlineto{\pgfqpoint{3.421806in}{1.958759in}}%
\pgfpathlineto{\pgfqpoint{3.430225in}{1.945686in}}%
\pgfpathlineto{\pgfqpoint{3.438623in}{1.933210in}}%
\pgfpathlineto{\pgfqpoint{3.447000in}{1.921319in}}%
\pgfpathlineto{\pgfqpoint{3.455358in}{1.910001in}}%
\pgfpathclose%
\pgfusepath{fill}%
\end{pgfscope}%
\begin{pgfscope}%
\pgfpathrectangle{\pgfqpoint{1.254980in}{0.150000in}}{\pgfqpoint{5.490039in}{5.490039in}}%
\pgfusepath{clip}%
\pgfsetbuttcap%
\pgfsetroundjoin%
\definecolor{currentfill}{rgb}{0.121831,0.589055,0.545623}%
\pgfsetfillcolor{currentfill}%
\pgfsetfillopacity{0.700000}%
\pgfsetlinewidth{0.000000pt}%
\definecolor{currentstroke}{rgb}{0.000000,0.000000,0.000000}%
\pgfsetstrokecolor{currentstroke}%
\pgfsetdash{}{0pt}%
\pgfpathmoveto{\pgfqpoint{2.574120in}{2.624150in}}%
\pgfpathlineto{\pgfqpoint{2.587625in}{2.613040in}}%
\pgfpathlineto{\pgfqpoint{2.601131in}{2.601969in}}%
\pgfpathlineto{\pgfqpoint{2.614639in}{2.590935in}}%
\pgfpathlineto{\pgfqpoint{2.628147in}{2.579939in}}%
\pgfpathlineto{\pgfqpoint{2.618852in}{2.602341in}}%
\pgfpathlineto{\pgfqpoint{2.609519in}{2.625481in}}%
\pgfpathlineto{\pgfqpoint{2.600147in}{2.649373in}}%
\pgfpathlineto{\pgfqpoint{2.590736in}{2.674033in}}%
\pgfpathlineto{\pgfqpoint{2.577164in}{2.685483in}}%
\pgfpathlineto{\pgfqpoint{2.563594in}{2.696971in}}%
\pgfpathlineto{\pgfqpoint{2.550025in}{2.708498in}}%
\pgfpathlineto{\pgfqpoint{2.536457in}{2.720064in}}%
\pgfpathlineto{\pgfqpoint{2.545933in}{2.694942in}}%
\pgfpathlineto{\pgfqpoint{2.555369in}{2.670592in}}%
\pgfpathlineto{\pgfqpoint{2.564764in}{2.646999in}}%
\pgfpathlineto{\pgfqpoint{2.574120in}{2.624150in}}%
\pgfpathclose%
\pgfusepath{fill}%
\end{pgfscope}%
\begin{pgfscope}%
\pgfpathrectangle{\pgfqpoint{1.254980in}{0.150000in}}{\pgfqpoint{5.490039in}{5.490039in}}%
\pgfusepath{clip}%
\pgfsetbuttcap%
\pgfsetroundjoin%
\definecolor{currentfill}{rgb}{0.273809,0.031497,0.358853}%
\pgfsetfillcolor{currentfill}%
\pgfsetfillopacity{0.700000}%
\pgfsetlinewidth{0.000000pt}%
\definecolor{currentstroke}{rgb}{0.000000,0.000000,0.000000}%
\pgfsetstrokecolor{currentstroke}%
\pgfsetdash{}{0pt}%
\pgfpathmoveto{\pgfqpoint{4.963206in}{1.367479in}}%
\pgfpathlineto{\pgfqpoint{4.977081in}{1.363994in}}%
\pgfpathlineto{\pgfqpoint{4.990964in}{1.360531in}}%
\pgfpathlineto{\pgfqpoint{5.004853in}{1.357092in}}%
\pgfpathlineto{\pgfqpoint{5.018750in}{1.353674in}}%
\pgfpathlineto{\pgfqpoint{5.011216in}{1.345219in}}%
\pgfpathlineto{\pgfqpoint{5.003679in}{1.336954in}}%
\pgfpathlineto{\pgfqpoint{4.996140in}{1.328887in}}%
\pgfpathlineto{\pgfqpoint{4.988599in}{1.321026in}}%
\pgfpathlineto{\pgfqpoint{4.974693in}{1.324734in}}%
\pgfpathlineto{\pgfqpoint{4.960794in}{1.328465in}}%
\pgfpathlineto{\pgfqpoint{4.946903in}{1.332218in}}%
\pgfpathlineto{\pgfqpoint{4.933018in}{1.335993in}}%
\pgfpathlineto{\pgfqpoint{4.940568in}{1.343559in}}%
\pgfpathlineto{\pgfqpoint{4.948117in}{1.351334in}}%
\pgfpathlineto{\pgfqpoint{4.955662in}{1.359309in}}%
\pgfpathlineto{\pgfqpoint{4.963206in}{1.367479in}}%
\pgfpathclose%
\pgfusepath{fill}%
\end{pgfscope}%
\begin{pgfscope}%
\pgfpathrectangle{\pgfqpoint{1.254980in}{0.150000in}}{\pgfqpoint{5.490039in}{5.490039in}}%
\pgfusepath{clip}%
\pgfsetbuttcap%
\pgfsetroundjoin%
\definecolor{currentfill}{rgb}{0.267968,0.223549,0.512008}%
\pgfsetfillcolor{currentfill}%
\pgfsetfillopacity{0.700000}%
\pgfsetlinewidth{0.000000pt}%
\definecolor{currentstroke}{rgb}{0.000000,0.000000,0.000000}%
\pgfsetstrokecolor{currentstroke}%
\pgfsetdash{}{0pt}%
\pgfpathmoveto{\pgfqpoint{3.759660in}{1.715506in}}%
\pgfpathlineto{\pgfqpoint{3.773257in}{1.708037in}}%
\pgfpathlineto{\pgfqpoint{3.786858in}{1.700593in}}%
\pgfpathlineto{\pgfqpoint{3.800463in}{1.693174in}}%
\pgfpathlineto{\pgfqpoint{3.814073in}{1.685779in}}%
\pgfpathlineto{\pgfqpoint{3.806006in}{1.692626in}}%
\pgfpathlineto{\pgfqpoint{3.797926in}{1.699975in}}%
\pgfpathlineto{\pgfqpoint{3.789831in}{1.707840in}}%
\pgfpathlineto{\pgfqpoint{3.781723in}{1.716230in}}%
\pgfpathlineto{\pgfqpoint{3.768078in}{1.724012in}}%
\pgfpathlineto{\pgfqpoint{3.754437in}{1.731819in}}%
\pgfpathlineto{\pgfqpoint{3.740801in}{1.739651in}}%
\pgfpathlineto{\pgfqpoint{3.727169in}{1.747508in}}%
\pgfpathlineto{\pgfqpoint{3.735314in}{1.738724in}}%
\pgfpathlineto{\pgfqpoint{3.743444in}{1.730470in}}%
\pgfpathlineto{\pgfqpoint{3.751560in}{1.722734in}}%
\pgfpathlineto{\pgfqpoint{3.759660in}{1.715506in}}%
\pgfpathclose%
\pgfusepath{fill}%
\end{pgfscope}%
\begin{pgfscope}%
\pgfpathrectangle{\pgfqpoint{1.254980in}{0.150000in}}{\pgfqpoint{5.490039in}{5.490039in}}%
\pgfusepath{clip}%
\pgfsetbuttcap%
\pgfsetroundjoin%
\definecolor{currentfill}{rgb}{0.183898,0.422383,0.556944}%
\pgfsetfillcolor{currentfill}%
\pgfsetfillopacity{0.700000}%
\pgfsetlinewidth{0.000000pt}%
\definecolor{currentstroke}{rgb}{0.000000,0.000000,0.000000}%
\pgfsetstrokecolor{currentstroke}%
\pgfsetdash{}{0pt}%
\pgfpathmoveto{\pgfqpoint{3.096433in}{2.177797in}}%
\pgfpathlineto{\pgfqpoint{3.109957in}{2.168338in}}%
\pgfpathlineto{\pgfqpoint{3.123482in}{2.158909in}}%
\pgfpathlineto{\pgfqpoint{3.137011in}{2.149510in}}%
\pgfpathlineto{\pgfqpoint{3.150542in}{2.140140in}}%
\pgfpathlineto{\pgfqpoint{3.141868in}{2.155912in}}%
\pgfpathlineto{\pgfqpoint{3.133167in}{2.172328in}}%
\pgfpathlineto{\pgfqpoint{3.124439in}{2.189399in}}%
\pgfpathlineto{\pgfqpoint{3.115684in}{2.207140in}}%
\pgfpathlineto{\pgfqpoint{3.102102in}{2.216936in}}%
\pgfpathlineto{\pgfqpoint{3.088523in}{2.226761in}}%
\pgfpathlineto{\pgfqpoint{3.074946in}{2.236617in}}%
\pgfpathlineto{\pgfqpoint{3.061372in}{2.246503in}}%
\pgfpathlineto{\pgfqpoint{3.070179in}{2.228329in}}%
\pgfpathlineto{\pgfqpoint{3.078958in}{2.210829in}}%
\pgfpathlineto{\pgfqpoint{3.087709in}{2.193989in}}%
\pgfpathlineto{\pgfqpoint{3.096433in}{2.177797in}}%
\pgfpathclose%
\pgfusepath{fill}%
\end{pgfscope}%
\begin{pgfscope}%
\pgfpathrectangle{\pgfqpoint{1.254980in}{0.150000in}}{\pgfqpoint{5.490039in}{5.490039in}}%
\pgfusepath{clip}%
\pgfsetbuttcap%
\pgfsetroundjoin%
\definecolor{currentfill}{rgb}{0.280267,0.073417,0.397163}%
\pgfsetfillcolor{currentfill}%
\pgfsetfillopacity{0.700000}%
\pgfsetlinewidth{0.000000pt}%
\definecolor{currentstroke}{rgb}{0.000000,0.000000,0.000000}%
\pgfsetstrokecolor{currentstroke}%
\pgfsetdash{}{0pt}%
\pgfpathmoveto{\pgfqpoint{4.400158in}{1.427419in}}%
\pgfpathlineto{\pgfqpoint{4.413879in}{1.421971in}}%
\pgfpathlineto{\pgfqpoint{4.427606in}{1.416546in}}%
\pgfpathlineto{\pgfqpoint{4.441338in}{1.411144in}}%
\pgfpathlineto{\pgfqpoint{4.455076in}{1.405764in}}%
\pgfpathlineto{\pgfqpoint{4.447380in}{1.404000in}}%
\pgfpathlineto{\pgfqpoint{4.439679in}{1.402583in}}%
\pgfpathlineto{\pgfqpoint{4.431971in}{1.401524in}}%
\pgfpathlineto{\pgfqpoint{4.424259in}{1.400830in}}%
\pgfpathlineto{\pgfqpoint{4.410500in}{1.406552in}}%
\pgfpathlineto{\pgfqpoint{4.396748in}{1.412297in}}%
\pgfpathlineto{\pgfqpoint{4.383001in}{1.418065in}}%
\pgfpathlineto{\pgfqpoint{4.369259in}{1.423855in}}%
\pgfpathlineto{\pgfqpoint{4.376993in}{1.424201in}}%
\pgfpathlineto{\pgfqpoint{4.384721in}{1.424917in}}%
\pgfpathlineto{\pgfqpoint{4.392442in}{1.425992in}}%
\pgfpathlineto{\pgfqpoint{4.400158in}{1.427419in}}%
\pgfpathclose%
\pgfusepath{fill}%
\end{pgfscope}%
\begin{pgfscope}%
\pgfpathrectangle{\pgfqpoint{1.254980in}{0.150000in}}{\pgfqpoint{5.490039in}{5.490039in}}%
\pgfusepath{clip}%
\pgfsetbuttcap%
\pgfsetroundjoin%
\definecolor{currentfill}{rgb}{0.281412,0.155834,0.469201}%
\pgfsetfillcolor{currentfill}%
\pgfsetfillopacity{0.700000}%
\pgfsetlinewidth{0.000000pt}%
\definecolor{currentstroke}{rgb}{0.000000,0.000000,0.000000}%
\pgfsetstrokecolor{currentstroke}%
\pgfsetdash{}{0pt}%
\pgfpathmoveto{\pgfqpoint{4.009507in}{1.580847in}}%
\pgfpathlineto{\pgfqpoint{4.023147in}{1.574140in}}%
\pgfpathlineto{\pgfqpoint{4.036791in}{1.567456in}}%
\pgfpathlineto{\pgfqpoint{4.050440in}{1.560796in}}%
\pgfpathlineto{\pgfqpoint{4.064095in}{1.554160in}}%
\pgfpathlineto{\pgfqpoint{4.056198in}{1.557635in}}%
\pgfpathlineto{\pgfqpoint{4.048291in}{1.561555in}}%
\pgfpathlineto{\pgfqpoint{4.040374in}{1.565932in}}%
\pgfpathlineto{\pgfqpoint{4.032446in}{1.570776in}}%
\pgfpathlineto{\pgfqpoint{4.018763in}{1.577784in}}%
\pgfpathlineto{\pgfqpoint{4.005085in}{1.584815in}}%
\pgfpathlineto{\pgfqpoint{3.991411in}{1.591870in}}%
\pgfpathlineto{\pgfqpoint{3.977742in}{1.598949in}}%
\pgfpathlineto{\pgfqpoint{3.985700in}{1.593728in}}%
\pgfpathlineto{\pgfqpoint{3.993646in}{1.588977in}}%
\pgfpathlineto{\pgfqpoint{4.001582in}{1.584687in}}%
\pgfpathlineto{\pgfqpoint{4.009507in}{1.580847in}}%
\pgfpathclose%
\pgfusepath{fill}%
\end{pgfscope}%
\begin{pgfscope}%
\pgfpathrectangle{\pgfqpoint{1.254980in}{0.150000in}}{\pgfqpoint{5.490039in}{5.490039in}}%
\pgfusepath{clip}%
\pgfsetbuttcap%
\pgfsetroundjoin%
\definecolor{currentfill}{rgb}{0.125394,0.574318,0.549086}%
\pgfsetfillcolor{currentfill}%
\pgfsetfillopacity{0.700000}%
\pgfsetlinewidth{0.000000pt}%
\definecolor{currentstroke}{rgb}{0.000000,0.000000,0.000000}%
\pgfsetstrokecolor{currentstroke}%
\pgfsetdash{}{0pt}%
\pgfpathmoveto{\pgfqpoint{2.628147in}{2.579939in}}%
\pgfpathlineto{\pgfqpoint{2.641657in}{2.568981in}}%
\pgfpathlineto{\pgfqpoint{2.655169in}{2.558061in}}%
\pgfpathlineto{\pgfqpoint{2.668682in}{2.547177in}}%
\pgfpathlineto{\pgfqpoint{2.682197in}{2.536330in}}%
\pgfpathlineto{\pgfqpoint{2.672962in}{2.558284in}}%
\pgfpathlineto{\pgfqpoint{2.663691in}{2.580973in}}%
\pgfpathlineto{\pgfqpoint{2.654382in}{2.604409in}}%
\pgfpathlineto{\pgfqpoint{2.645034in}{2.628608in}}%
\pgfpathlineto{\pgfqpoint{2.631457in}{2.639908in}}%
\pgfpathlineto{\pgfqpoint{2.617882in}{2.651245in}}%
\pgfpathlineto{\pgfqpoint{2.604308in}{2.662620in}}%
\pgfpathlineto{\pgfqpoint{2.590736in}{2.674033in}}%
\pgfpathlineto{\pgfqpoint{2.600147in}{2.649373in}}%
\pgfpathlineto{\pgfqpoint{2.609519in}{2.625481in}}%
\pgfpathlineto{\pgfqpoint{2.618852in}{2.602341in}}%
\pgfpathlineto{\pgfqpoint{2.628147in}{2.579939in}}%
\pgfpathclose%
\pgfusepath{fill}%
\end{pgfscope}%
\begin{pgfscope}%
\pgfpathrectangle{\pgfqpoint{1.254980in}{0.150000in}}{\pgfqpoint{5.490039in}{5.490039in}}%
\pgfusepath{clip}%
\pgfsetbuttcap%
\pgfsetroundjoin%
\definecolor{currentfill}{rgb}{0.276022,0.044167,0.370164}%
\pgfsetfillcolor{currentfill}%
\pgfsetfillopacity{0.700000}%
\pgfsetlinewidth{0.000000pt}%
\definecolor{currentstroke}{rgb}{0.000000,0.000000,0.000000}%
\pgfsetstrokecolor{currentstroke}%
\pgfsetdash{}{0pt}%
\pgfpathmoveto{\pgfqpoint{5.104500in}{1.376925in}}%
\pgfpathlineto{\pgfqpoint{5.118427in}{1.373899in}}%
\pgfpathlineto{\pgfqpoint{5.132362in}{1.370895in}}%
\pgfpathlineto{\pgfqpoint{5.146304in}{1.367914in}}%
\pgfpathlineto{\pgfqpoint{5.138790in}{1.358290in}}%
\pgfpathlineto{\pgfqpoint{5.131275in}{1.348822in}}%
\pgfpathlineto{\pgfqpoint{5.123757in}{1.339518in}}%
\pgfpathlineto{\pgfqpoint{5.116238in}{1.330384in}}%
\pgfpathlineto{\pgfqpoint{5.102289in}{1.333644in}}%
\pgfpathlineto{\pgfqpoint{5.088347in}{1.336926in}}%
\pgfpathlineto{\pgfqpoint{5.074413in}{1.340230in}}%
\pgfpathlineto{\pgfqpoint{5.081938in}{1.349152in}}%
\pgfpathlineto{\pgfqpoint{5.089461in}{1.358246in}}%
\pgfpathlineto{\pgfqpoint{5.096981in}{1.367506in}}%
\pgfpathlineto{\pgfqpoint{5.104500in}{1.376925in}}%
\pgfpathclose%
\pgfusepath{fill}%
\end{pgfscope}%
\begin{pgfscope}%
\pgfpathrectangle{\pgfqpoint{1.254980in}{0.150000in}}{\pgfqpoint{5.490039in}{5.490039in}}%
\pgfusepath{clip}%
\pgfsetbuttcap%
\pgfsetroundjoin%
\definecolor{currentfill}{rgb}{0.239346,0.300855,0.540844}%
\pgfsetfillcolor{currentfill}%
\pgfsetfillopacity{0.700000}%
\pgfsetlinewidth{0.000000pt}%
\definecolor{currentstroke}{rgb}{0.000000,0.000000,0.000000}%
\pgfsetstrokecolor{currentstroke}%
\pgfsetdash{}{0pt}%
\pgfpathmoveto{\pgfqpoint{3.509598in}{1.876670in}}%
\pgfpathlineto{\pgfqpoint{3.523168in}{1.868403in}}%
\pgfpathlineto{\pgfqpoint{3.536741in}{1.860163in}}%
\pgfpathlineto{\pgfqpoint{3.550317in}{1.851949in}}%
\pgfpathlineto{\pgfqpoint{3.563898in}{1.843761in}}%
\pgfpathlineto{\pgfqpoint{3.555621in}{1.854281in}}%
\pgfpathlineto{\pgfqpoint{3.547326in}{1.865365in}}%
\pgfpathlineto{\pgfqpoint{3.539013in}{1.877026in}}%
\pgfpathlineto{\pgfqpoint{3.530681in}{1.889275in}}%
\pgfpathlineto{\pgfqpoint{3.517059in}{1.897868in}}%
\pgfpathlineto{\pgfqpoint{3.503441in}{1.906488in}}%
\pgfpathlineto{\pgfqpoint{3.489826in}{1.915133in}}%
\pgfpathlineto{\pgfqpoint{3.476215in}{1.923805in}}%
\pgfpathlineto{\pgfqpoint{3.484590in}{1.911144in}}%
\pgfpathlineto{\pgfqpoint{3.492946in}{1.899076in}}%
\pgfpathlineto{\pgfqpoint{3.501282in}{1.887589in}}%
\pgfpathlineto{\pgfqpoint{3.509598in}{1.876670in}}%
\pgfpathclose%
\pgfusepath{fill}%
\end{pgfscope}%
\begin{pgfscope}%
\pgfpathrectangle{\pgfqpoint{1.254980in}{0.150000in}}{\pgfqpoint{5.490039in}{5.490039in}}%
\pgfusepath{clip}%
\pgfsetbuttcap%
\pgfsetroundjoin%
\definecolor{currentfill}{rgb}{0.274952,0.037752,0.364543}%
\pgfsetfillcolor{currentfill}%
\pgfsetfillopacity{0.700000}%
\pgfsetlinewidth{0.000000pt}%
\definecolor{currentstroke}{rgb}{0.000000,0.000000,0.000000}%
\pgfsetstrokecolor{currentstroke}%
\pgfsetdash{}{0pt}%
\pgfpathmoveto{\pgfqpoint{4.736589in}{1.358671in}}%
\pgfpathlineto{\pgfqpoint{4.750403in}{1.354322in}}%
\pgfpathlineto{\pgfqpoint{4.764223in}{1.349995in}}%
\pgfpathlineto{\pgfqpoint{4.778049in}{1.345691in}}%
\pgfpathlineto{\pgfqpoint{4.791883in}{1.341410in}}%
\pgfpathlineto{\pgfqpoint{4.784297in}{1.335625in}}%
\pgfpathlineto{\pgfqpoint{4.776709in}{1.330103in}}%
\pgfpathlineto{\pgfqpoint{4.769119in}{1.324850in}}%
\pgfpathlineto{\pgfqpoint{4.761525in}{1.319876in}}%
\pgfpathlineto{\pgfqpoint{4.747678in}{1.324474in}}%
\pgfpathlineto{\pgfqpoint{4.733838in}{1.329094in}}%
\pgfpathlineto{\pgfqpoint{4.720004in}{1.333737in}}%
\pgfpathlineto{\pgfqpoint{4.706177in}{1.338402in}}%
\pgfpathlineto{\pgfqpoint{4.713785in}{1.343055in}}%
\pgfpathlineto{\pgfqpoint{4.721390in}{1.347989in}}%
\pgfpathlineto{\pgfqpoint{4.728991in}{1.353198in}}%
\pgfpathlineto{\pgfqpoint{4.736589in}{1.358671in}}%
\pgfpathclose%
\pgfusepath{fill}%
\end{pgfscope}%
\begin{pgfscope}%
\pgfpathrectangle{\pgfqpoint{1.254980in}{0.150000in}}{\pgfqpoint{5.490039in}{5.490039in}}%
\pgfusepath{clip}%
\pgfsetbuttcap%
\pgfsetroundjoin%
\definecolor{currentfill}{rgb}{0.188923,0.410910,0.556326}%
\pgfsetfillcolor{currentfill}%
\pgfsetfillopacity{0.700000}%
\pgfsetlinewidth{0.000000pt}%
\definecolor{currentstroke}{rgb}{0.000000,0.000000,0.000000}%
\pgfsetstrokecolor{currentstroke}%
\pgfsetdash{}{0pt}%
\pgfpathmoveto{\pgfqpoint{3.150542in}{2.140140in}}%
\pgfpathlineto{\pgfqpoint{3.164077in}{2.130800in}}%
\pgfpathlineto{\pgfqpoint{3.177614in}{2.121490in}}%
\pgfpathlineto{\pgfqpoint{3.191154in}{2.112208in}}%
\pgfpathlineto{\pgfqpoint{3.204697in}{2.102956in}}%
\pgfpathlineto{\pgfqpoint{3.196071in}{2.118309in}}%
\pgfpathlineto{\pgfqpoint{3.187419in}{2.134300in}}%
\pgfpathlineto{\pgfqpoint{3.178742in}{2.150943in}}%
\pgfpathlineto{\pgfqpoint{3.170039in}{2.168251in}}%
\pgfpathlineto{\pgfqpoint{3.156446in}{2.177929in}}%
\pgfpathlineto{\pgfqpoint{3.142856in}{2.187636in}}%
\pgfpathlineto{\pgfqpoint{3.129269in}{2.197373in}}%
\pgfpathlineto{\pgfqpoint{3.115684in}{2.207140in}}%
\pgfpathlineto{\pgfqpoint{3.124439in}{2.189399in}}%
\pgfpathlineto{\pgfqpoint{3.133167in}{2.172328in}}%
\pgfpathlineto{\pgfqpoint{3.141868in}{2.155912in}}%
\pgfpathlineto{\pgfqpoint{3.150542in}{2.140140in}}%
\pgfpathclose%
\pgfusepath{fill}%
\end{pgfscope}%
\begin{pgfscope}%
\pgfpathrectangle{\pgfqpoint{1.254980in}{0.150000in}}{\pgfqpoint{5.490039in}{5.490039in}}%
\pgfusepath{clip}%
\pgfsetbuttcap%
\pgfsetroundjoin%
\definecolor{currentfill}{rgb}{0.277018,0.050344,0.375715}%
\pgfsetfillcolor{currentfill}%
\pgfsetfillopacity{0.700000}%
\pgfsetlinewidth{0.000000pt}%
\definecolor{currentstroke}{rgb}{0.000000,0.000000,0.000000}%
\pgfsetstrokecolor{currentstroke}%
\pgfsetdash{}{0pt}%
\pgfpathmoveto{\pgfqpoint{4.595792in}{1.376536in}}%
\pgfpathlineto{\pgfqpoint{4.609568in}{1.371690in}}%
\pgfpathlineto{\pgfqpoint{4.623350in}{1.366867in}}%
\pgfpathlineto{\pgfqpoint{4.637139in}{1.362066in}}%
\pgfpathlineto{\pgfqpoint{4.650934in}{1.357288in}}%
\pgfpathlineto{\pgfqpoint{4.643307in}{1.353250in}}%
\pgfpathlineto{\pgfqpoint{4.635677in}{1.349514in}}%
\pgfpathlineto{\pgfqpoint{4.628044in}{1.346089in}}%
\pgfpathlineto{\pgfqpoint{4.620406in}{1.342984in}}%
\pgfpathlineto{\pgfqpoint{4.606595in}{1.348091in}}%
\pgfpathlineto{\pgfqpoint{4.592790in}{1.353221in}}%
\pgfpathlineto{\pgfqpoint{4.578991in}{1.358373in}}%
\pgfpathlineto{\pgfqpoint{4.565199in}{1.363548in}}%
\pgfpathlineto{\pgfqpoint{4.572853in}{1.366319in}}%
\pgfpathlineto{\pgfqpoint{4.580504in}{1.369414in}}%
\pgfpathlineto{\pgfqpoint{4.588150in}{1.372822in}}%
\pgfpathlineto{\pgfqpoint{4.595792in}{1.376536in}}%
\pgfpathclose%
\pgfusepath{fill}%
\end{pgfscope}%
\begin{pgfscope}%
\pgfpathrectangle{\pgfqpoint{1.254980in}{0.150000in}}{\pgfqpoint{5.490039in}{5.490039in}}%
\pgfusepath{clip}%
\pgfsetbuttcap%
\pgfsetroundjoin%
\definecolor{currentfill}{rgb}{0.273809,0.031497,0.358853}%
\pgfsetfillcolor{currentfill}%
\pgfsetfillopacity{0.700000}%
\pgfsetlinewidth{0.000000pt}%
\definecolor{currentstroke}{rgb}{0.000000,0.000000,0.000000}%
\pgfsetstrokecolor{currentstroke}%
\pgfsetdash{}{0pt}%
\pgfpathmoveto{\pgfqpoint{4.877551in}{1.351321in}}%
\pgfpathlineto{\pgfqpoint{4.891407in}{1.347455in}}%
\pgfpathlineto{\pgfqpoint{4.905270in}{1.343612in}}%
\pgfpathlineto{\pgfqpoint{4.919141in}{1.339791in}}%
\pgfpathlineto{\pgfqpoint{4.933018in}{1.335993in}}%
\pgfpathlineto{\pgfqpoint{4.925466in}{1.328643in}}%
\pgfpathlineto{\pgfqpoint{4.917911in}{1.321517in}}%
\pgfpathlineto{\pgfqpoint{4.910354in}{1.314623in}}%
\pgfpathlineto{\pgfqpoint{4.902795in}{1.307967in}}%
\pgfpathlineto{\pgfqpoint{4.888907in}{1.312069in}}%
\pgfpathlineto{\pgfqpoint{4.875026in}{1.316193in}}%
\pgfpathlineto{\pgfqpoint{4.861151in}{1.320340in}}%
\pgfpathlineto{\pgfqpoint{4.847284in}{1.324509in}}%
\pgfpathlineto{\pgfqpoint{4.854854in}{1.330856in}}%
\pgfpathlineto{\pgfqpoint{4.862422in}{1.337445in}}%
\pgfpathlineto{\pgfqpoint{4.869988in}{1.344269in}}%
\pgfpathlineto{\pgfqpoint{4.877551in}{1.351321in}}%
\pgfpathclose%
\pgfusepath{fill}%
\end{pgfscope}%
\begin{pgfscope}%
\pgfpathrectangle{\pgfqpoint{1.254980in}{0.150000in}}{\pgfqpoint{5.490039in}{5.490039in}}%
\pgfusepath{clip}%
\pgfsetbuttcap%
\pgfsetroundjoin%
\definecolor{currentfill}{rgb}{0.282910,0.105393,0.426902}%
\pgfsetfillcolor{currentfill}%
\pgfsetfillopacity{0.700000}%
\pgfsetlinewidth{0.000000pt}%
\definecolor{currentstroke}{rgb}{0.000000,0.000000,0.000000}%
\pgfsetstrokecolor{currentstroke}%
\pgfsetdash{}{0pt}%
\pgfpathmoveto{\pgfqpoint{4.259530in}{1.471008in}}%
\pgfpathlineto{\pgfqpoint{4.273226in}{1.465033in}}%
\pgfpathlineto{\pgfqpoint{4.286929in}{1.459081in}}%
\pgfpathlineto{\pgfqpoint{4.300636in}{1.453153in}}%
\pgfpathlineto{\pgfqpoint{4.314350in}{1.447247in}}%
\pgfpathlineto{\pgfqpoint{4.306587in}{1.447632in}}%
\pgfpathlineto{\pgfqpoint{4.298818in}{1.448408in}}%
\pgfpathlineto{\pgfqpoint{4.291042in}{1.449587in}}%
\pgfpathlineto{\pgfqpoint{4.283259in}{1.451176in}}%
\pgfpathlineto{\pgfqpoint{4.269521in}{1.457439in}}%
\pgfpathlineto{\pgfqpoint{4.255790in}{1.463724in}}%
\pgfpathlineto{\pgfqpoint{4.242063in}{1.470032in}}%
\pgfpathlineto{\pgfqpoint{4.228342in}{1.476363in}}%
\pgfpathlineto{\pgfqpoint{4.236151in}{1.474412in}}%
\pgfpathlineto{\pgfqpoint{4.243951in}{1.472875in}}%
\pgfpathlineto{\pgfqpoint{4.251744in}{1.471743in}}%
\pgfpathlineto{\pgfqpoint{4.259530in}{1.471008in}}%
\pgfpathclose%
\pgfusepath{fill}%
\end{pgfscope}%
\begin{pgfscope}%
\pgfpathrectangle{\pgfqpoint{1.254980in}{0.150000in}}{\pgfqpoint{5.490039in}{5.490039in}}%
\pgfusepath{clip}%
\pgfsetbuttcap%
\pgfsetroundjoin%
\definecolor{currentfill}{rgb}{0.270595,0.214069,0.507052}%
\pgfsetfillcolor{currentfill}%
\pgfsetfillopacity{0.700000}%
\pgfsetlinewidth{0.000000pt}%
\definecolor{currentstroke}{rgb}{0.000000,0.000000,0.000000}%
\pgfsetstrokecolor{currentstroke}%
\pgfsetdash{}{0pt}%
\pgfpathmoveto{\pgfqpoint{3.814073in}{1.685779in}}%
\pgfpathlineto{\pgfqpoint{3.827687in}{1.678409in}}%
\pgfpathlineto{\pgfqpoint{3.841306in}{1.671064in}}%
\pgfpathlineto{\pgfqpoint{3.854929in}{1.663743in}}%
\pgfpathlineto{\pgfqpoint{3.868557in}{1.656447in}}%
\pgfpathlineto{\pgfqpoint{3.860523in}{1.662911in}}%
\pgfpathlineto{\pgfqpoint{3.852477in}{1.669876in}}%
\pgfpathlineto{\pgfqpoint{3.844417in}{1.677351in}}%
\pgfpathlineto{\pgfqpoint{3.836343in}{1.685348in}}%
\pgfpathlineto{\pgfqpoint{3.822682in}{1.693032in}}%
\pgfpathlineto{\pgfqpoint{3.809024in}{1.700740in}}%
\pgfpathlineto{\pgfqpoint{3.795371in}{1.708473in}}%
\pgfpathlineto{\pgfqpoint{3.781723in}{1.716230in}}%
\pgfpathlineto{\pgfqpoint{3.789831in}{1.707840in}}%
\pgfpathlineto{\pgfqpoint{3.797926in}{1.699975in}}%
\pgfpathlineto{\pgfqpoint{3.806006in}{1.692626in}}%
\pgfpathlineto{\pgfqpoint{3.814073in}{1.685779in}}%
\pgfpathclose%
\pgfusepath{fill}%
\end{pgfscope}%
\begin{pgfscope}%
\pgfpathrectangle{\pgfqpoint{1.254980in}{0.150000in}}{\pgfqpoint{5.490039in}{5.490039in}}%
\pgfusepath{clip}%
\pgfsetbuttcap%
\pgfsetroundjoin%
\definecolor{currentfill}{rgb}{0.129933,0.559582,0.551864}%
\pgfsetfillcolor{currentfill}%
\pgfsetfillopacity{0.700000}%
\pgfsetlinewidth{0.000000pt}%
\definecolor{currentstroke}{rgb}{0.000000,0.000000,0.000000}%
\pgfsetstrokecolor{currentstroke}%
\pgfsetdash{}{0pt}%
\pgfpathmoveto{\pgfqpoint{2.682197in}{2.536330in}}%
\pgfpathlineto{\pgfqpoint{2.695713in}{2.525519in}}%
\pgfpathlineto{\pgfqpoint{2.709231in}{2.514744in}}%
\pgfpathlineto{\pgfqpoint{2.722750in}{2.504006in}}%
\pgfpathlineto{\pgfqpoint{2.736271in}{2.493302in}}%
\pgfpathlineto{\pgfqpoint{2.727096in}{2.514812in}}%
\pgfpathlineto{\pgfqpoint{2.717886in}{2.537051in}}%
\pgfpathlineto{\pgfqpoint{2.708638in}{2.560032in}}%
\pgfpathlineto{\pgfqpoint{2.699353in}{2.583770in}}%
\pgfpathlineto{\pgfqpoint{2.685771in}{2.594925in}}%
\pgfpathlineto{\pgfqpoint{2.672191in}{2.606116in}}%
\pgfpathlineto{\pgfqpoint{2.658612in}{2.617344in}}%
\pgfpathlineto{\pgfqpoint{2.645034in}{2.628608in}}%
\pgfpathlineto{\pgfqpoint{2.654382in}{2.604409in}}%
\pgfpathlineto{\pgfqpoint{2.663691in}{2.580973in}}%
\pgfpathlineto{\pgfqpoint{2.672962in}{2.558284in}}%
\pgfpathlineto{\pgfqpoint{2.682197in}{2.536330in}}%
\pgfpathclose%
\pgfusepath{fill}%
\end{pgfscope}%
\begin{pgfscope}%
\pgfpathrectangle{\pgfqpoint{1.254980in}{0.150000in}}{\pgfqpoint{5.490039in}{5.490039in}}%
\pgfusepath{clip}%
\pgfsetbuttcap%
\pgfsetroundjoin%
\definecolor{currentfill}{rgb}{0.281887,0.150881,0.465405}%
\pgfsetfillcolor{currentfill}%
\pgfsetfillopacity{0.700000}%
\pgfsetlinewidth{0.000000pt}%
\definecolor{currentstroke}{rgb}{0.000000,0.000000,0.000000}%
\pgfsetstrokecolor{currentstroke}%
\pgfsetdash{}{0pt}%
\pgfpathmoveto{\pgfqpoint{4.064095in}{1.554160in}}%
\pgfpathlineto{\pgfqpoint{4.077754in}{1.547548in}}%
\pgfpathlineto{\pgfqpoint{4.091418in}{1.540959in}}%
\pgfpathlineto{\pgfqpoint{4.105088in}{1.534394in}}%
\pgfpathlineto{\pgfqpoint{4.118762in}{1.527853in}}%
\pgfpathlineto{\pgfqpoint{4.110893in}{1.530961in}}%
\pgfpathlineto{\pgfqpoint{4.103014in}{1.534512in}}%
\pgfpathlineto{\pgfqpoint{4.095126in}{1.538516in}}%
\pgfpathlineto{\pgfqpoint{4.087228in}{1.542983in}}%
\pgfpathlineto{\pgfqpoint{4.073525in}{1.549896in}}%
\pgfpathlineto{\pgfqpoint{4.059828in}{1.556832in}}%
\pgfpathlineto{\pgfqpoint{4.046135in}{1.563792in}}%
\pgfpathlineto{\pgfqpoint{4.032446in}{1.570776in}}%
\pgfpathlineto{\pgfqpoint{4.040374in}{1.565932in}}%
\pgfpathlineto{\pgfqpoint{4.048291in}{1.561555in}}%
\pgfpathlineto{\pgfqpoint{4.056198in}{1.557635in}}%
\pgfpathlineto{\pgfqpoint{4.064095in}{1.554160in}}%
\pgfpathclose%
\pgfusepath{fill}%
\end{pgfscope}%
\begin{pgfscope}%
\pgfpathrectangle{\pgfqpoint{1.254980in}{0.150000in}}{\pgfqpoint{5.490039in}{5.490039in}}%
\pgfusepath{clip}%
\pgfsetbuttcap%
\pgfsetroundjoin%
\definecolor{currentfill}{rgb}{0.279566,0.067836,0.391917}%
\pgfsetfillcolor{currentfill}%
\pgfsetfillopacity{0.700000}%
\pgfsetlinewidth{0.000000pt}%
\definecolor{currentstroke}{rgb}{0.000000,0.000000,0.000000}%
\pgfsetstrokecolor{currentstroke}%
\pgfsetdash{}{0pt}%
\pgfpathmoveto{\pgfqpoint{4.455076in}{1.405764in}}%
\pgfpathlineto{\pgfqpoint{4.468821in}{1.400408in}}%
\pgfpathlineto{\pgfqpoint{4.482571in}{1.395074in}}%
\pgfpathlineto{\pgfqpoint{4.496327in}{1.389763in}}%
\pgfpathlineto{\pgfqpoint{4.510089in}{1.384475in}}%
\pgfpathlineto{\pgfqpoint{4.502412in}{1.382373in}}%
\pgfpathlineto{\pgfqpoint{4.494730in}{1.380615in}}%
\pgfpathlineto{\pgfqpoint{4.487043in}{1.379211in}}%
\pgfpathlineto{\pgfqpoint{4.479351in}{1.378169in}}%
\pgfpathlineto{\pgfqpoint{4.465569in}{1.383800in}}%
\pgfpathlineto{\pgfqpoint{4.451793in}{1.389454in}}%
\pgfpathlineto{\pgfqpoint{4.438023in}{1.395131in}}%
\pgfpathlineto{\pgfqpoint{4.424259in}{1.400830in}}%
\pgfpathlineto{\pgfqpoint{4.431971in}{1.401524in}}%
\pgfpathlineto{\pgfqpoint{4.439679in}{1.402583in}}%
\pgfpathlineto{\pgfqpoint{4.447380in}{1.404000in}}%
\pgfpathlineto{\pgfqpoint{4.455076in}{1.405764in}}%
\pgfpathclose%
\pgfusepath{fill}%
\end{pgfscope}%
\begin{pgfscope}%
\pgfpathrectangle{\pgfqpoint{1.254980in}{0.150000in}}{\pgfqpoint{5.490039in}{5.490039in}}%
\pgfusepath{clip}%
\pgfsetbuttcap%
\pgfsetroundjoin%
\definecolor{currentfill}{rgb}{0.274952,0.037752,0.364543}%
\pgfsetfillcolor{currentfill}%
\pgfsetfillopacity{0.700000}%
\pgfsetlinewidth{0.000000pt}%
\definecolor{currentstroke}{rgb}{0.000000,0.000000,0.000000}%
\pgfsetstrokecolor{currentstroke}%
\pgfsetdash{}{0pt}%
\pgfpathmoveto{\pgfqpoint{5.018750in}{1.353674in}}%
\pgfpathlineto{\pgfqpoint{5.032655in}{1.350280in}}%
\pgfpathlineto{\pgfqpoint{5.046567in}{1.346907in}}%
\pgfpathlineto{\pgfqpoint{5.060486in}{1.343558in}}%
\pgfpathlineto{\pgfqpoint{5.074413in}{1.340230in}}%
\pgfpathlineto{\pgfqpoint{5.066886in}{1.331489in}}%
\pgfpathlineto{\pgfqpoint{5.059358in}{1.322935in}}%
\pgfpathlineto{\pgfqpoint{5.051828in}{1.314575in}}%
\pgfpathlineto{\pgfqpoint{5.044296in}{1.306417in}}%
\pgfpathlineto{\pgfqpoint{5.030361in}{1.310036in}}%
\pgfpathlineto{\pgfqpoint{5.016433in}{1.313676in}}%
\pgfpathlineto{\pgfqpoint{5.002513in}{1.317340in}}%
\pgfpathlineto{\pgfqpoint{4.988599in}{1.321026in}}%
\pgfpathlineto{\pgfqpoint{4.996140in}{1.328887in}}%
\pgfpathlineto{\pgfqpoint{5.003679in}{1.336954in}}%
\pgfpathlineto{\pgfqpoint{5.011216in}{1.345219in}}%
\pgfpathlineto{\pgfqpoint{5.018750in}{1.353674in}}%
\pgfpathclose%
\pgfusepath{fill}%
\end{pgfscope}%
\begin{pgfscope}%
\pgfpathrectangle{\pgfqpoint{1.254980in}{0.150000in}}{\pgfqpoint{5.490039in}{5.490039in}}%
\pgfusepath{clip}%
\pgfsetbuttcap%
\pgfsetroundjoin%
\definecolor{currentfill}{rgb}{0.194100,0.399323,0.555565}%
\pgfsetfillcolor{currentfill}%
\pgfsetfillopacity{0.700000}%
\pgfsetlinewidth{0.000000pt}%
\definecolor{currentstroke}{rgb}{0.000000,0.000000,0.000000}%
\pgfsetstrokecolor{currentstroke}%
\pgfsetdash{}{0pt}%
\pgfpathmoveto{\pgfqpoint{3.204697in}{2.102956in}}%
\pgfpathlineto{\pgfqpoint{3.218243in}{2.093732in}}%
\pgfpathlineto{\pgfqpoint{3.231792in}{2.084538in}}%
\pgfpathlineto{\pgfqpoint{3.245343in}{2.075372in}}%
\pgfpathlineto{\pgfqpoint{3.258898in}{2.066234in}}%
\pgfpathlineto{\pgfqpoint{3.250320in}{2.081168in}}%
\pgfpathlineto{\pgfqpoint{3.241718in}{2.096737in}}%
\pgfpathlineto{\pgfqpoint{3.233090in}{2.112952in}}%
\pgfpathlineto{\pgfqpoint{3.224437in}{2.129828in}}%
\pgfpathlineto{\pgfqpoint{3.210833in}{2.139391in}}%
\pgfpathlineto{\pgfqpoint{3.197232in}{2.148982in}}%
\pgfpathlineto{\pgfqpoint{3.183634in}{2.158602in}}%
\pgfpathlineto{\pgfqpoint{3.170039in}{2.168251in}}%
\pgfpathlineto{\pgfqpoint{3.178742in}{2.150943in}}%
\pgfpathlineto{\pgfqpoint{3.187419in}{2.134300in}}%
\pgfpathlineto{\pgfqpoint{3.196071in}{2.118309in}}%
\pgfpathlineto{\pgfqpoint{3.204697in}{2.102956in}}%
\pgfpathclose%
\pgfusepath{fill}%
\end{pgfscope}%
\begin{pgfscope}%
\pgfpathrectangle{\pgfqpoint{1.254980in}{0.150000in}}{\pgfqpoint{5.490039in}{5.490039in}}%
\pgfusepath{clip}%
\pgfsetbuttcap%
\pgfsetroundjoin%
\definecolor{currentfill}{rgb}{0.243113,0.292092,0.538516}%
\pgfsetfillcolor{currentfill}%
\pgfsetfillopacity{0.700000}%
\pgfsetlinewidth{0.000000pt}%
\definecolor{currentstroke}{rgb}{0.000000,0.000000,0.000000}%
\pgfsetstrokecolor{currentstroke}%
\pgfsetdash{}{0pt}%
\pgfpathmoveto{\pgfqpoint{3.563898in}{1.843761in}}%
\pgfpathlineto{\pgfqpoint{3.577482in}{1.835599in}}%
\pgfpathlineto{\pgfqpoint{3.591070in}{1.827463in}}%
\pgfpathlineto{\pgfqpoint{3.604662in}{1.819353in}}%
\pgfpathlineto{\pgfqpoint{3.618258in}{1.811268in}}%
\pgfpathlineto{\pgfqpoint{3.610021in}{1.821389in}}%
\pgfpathlineto{\pgfqpoint{3.601767in}{1.832071in}}%
\pgfpathlineto{\pgfqpoint{3.593494in}{1.843325in}}%
\pgfpathlineto{\pgfqpoint{3.585204in}{1.855163in}}%
\pgfpathlineto{\pgfqpoint{3.571567in}{1.863652in}}%
\pgfpathlineto{\pgfqpoint{3.557935in}{1.872167in}}%
\pgfpathlineto{\pgfqpoint{3.544306in}{1.880708in}}%
\pgfpathlineto{\pgfqpoint{3.530681in}{1.889275in}}%
\pgfpathlineto{\pgfqpoint{3.539013in}{1.877026in}}%
\pgfpathlineto{\pgfqpoint{3.547326in}{1.865365in}}%
\pgfpathlineto{\pgfqpoint{3.555621in}{1.854281in}}%
\pgfpathlineto{\pgfqpoint{3.563898in}{1.843761in}}%
\pgfpathclose%
\pgfusepath{fill}%
\end{pgfscope}%
\begin{pgfscope}%
\pgfpathrectangle{\pgfqpoint{1.254980in}{0.150000in}}{\pgfqpoint{5.490039in}{5.490039in}}%
\pgfusepath{clip}%
\pgfsetbuttcap%
\pgfsetroundjoin%
\definecolor{currentfill}{rgb}{0.133743,0.548535,0.553541}%
\pgfsetfillcolor{currentfill}%
\pgfsetfillopacity{0.700000}%
\pgfsetlinewidth{0.000000pt}%
\definecolor{currentstroke}{rgb}{0.000000,0.000000,0.000000}%
\pgfsetstrokecolor{currentstroke}%
\pgfsetdash{}{0pt}%
\pgfpathmoveto{\pgfqpoint{2.736271in}{2.493302in}}%
\pgfpathlineto{\pgfqpoint{2.749794in}{2.482635in}}%
\pgfpathlineto{\pgfqpoint{2.763319in}{2.472002in}}%
\pgfpathlineto{\pgfqpoint{2.776845in}{2.461404in}}%
\pgfpathlineto{\pgfqpoint{2.790373in}{2.450841in}}%
\pgfpathlineto{\pgfqpoint{2.781257in}{2.471907in}}%
\pgfpathlineto{\pgfqpoint{2.772106in}{2.493696in}}%
\pgfpathlineto{\pgfqpoint{2.762919in}{2.516224in}}%
\pgfpathlineto{\pgfqpoint{2.753696in}{2.539503in}}%
\pgfpathlineto{\pgfqpoint{2.740108in}{2.550518in}}%
\pgfpathlineto{\pgfqpoint{2.726521in}{2.561567in}}%
\pgfpathlineto{\pgfqpoint{2.712936in}{2.572651in}}%
\pgfpathlineto{\pgfqpoint{2.699353in}{2.583770in}}%
\pgfpathlineto{\pgfqpoint{2.708638in}{2.560032in}}%
\pgfpathlineto{\pgfqpoint{2.717886in}{2.537051in}}%
\pgfpathlineto{\pgfqpoint{2.727096in}{2.514812in}}%
\pgfpathlineto{\pgfqpoint{2.736271in}{2.493302in}}%
\pgfpathclose%
\pgfusepath{fill}%
\end{pgfscope}%
\begin{pgfscope}%
\pgfpathrectangle{\pgfqpoint{1.254980in}{0.150000in}}{\pgfqpoint{5.490039in}{5.490039in}}%
\pgfusepath{clip}%
\pgfsetbuttcap%
\pgfsetroundjoin%
\definecolor{currentfill}{rgb}{0.273006,0.204520,0.501721}%
\pgfsetfillcolor{currentfill}%
\pgfsetfillopacity{0.700000}%
\pgfsetlinewidth{0.000000pt}%
\definecolor{currentstroke}{rgb}{0.000000,0.000000,0.000000}%
\pgfsetstrokecolor{currentstroke}%
\pgfsetdash{}{0pt}%
\pgfpathmoveto{\pgfqpoint{3.868557in}{1.656447in}}%
\pgfpathlineto{\pgfqpoint{3.882189in}{1.649175in}}%
\pgfpathlineto{\pgfqpoint{3.895825in}{1.641927in}}%
\pgfpathlineto{\pgfqpoint{3.909467in}{1.634704in}}%
\pgfpathlineto{\pgfqpoint{3.923112in}{1.627505in}}%
\pgfpathlineto{\pgfqpoint{3.915111in}{1.633588in}}%
\pgfpathlineto{\pgfqpoint{3.907098in}{1.640167in}}%
\pgfpathlineto{\pgfqpoint{3.899072in}{1.647254in}}%
\pgfpathlineto{\pgfqpoint{3.891033in}{1.654858in}}%
\pgfpathlineto{\pgfqpoint{3.877354in}{1.662444in}}%
\pgfpathlineto{\pgfqpoint{3.863680in}{1.670054in}}%
\pgfpathlineto{\pgfqpoint{3.850009in}{1.677689in}}%
\pgfpathlineto{\pgfqpoint{3.836343in}{1.685348in}}%
\pgfpathlineto{\pgfqpoint{3.844417in}{1.677351in}}%
\pgfpathlineto{\pgfqpoint{3.852477in}{1.669876in}}%
\pgfpathlineto{\pgfqpoint{3.860523in}{1.662911in}}%
\pgfpathlineto{\pgfqpoint{3.868557in}{1.656447in}}%
\pgfpathclose%
\pgfusepath{fill}%
\end{pgfscope}%
\begin{pgfscope}%
\pgfpathrectangle{\pgfqpoint{1.254980in}{0.150000in}}{\pgfqpoint{5.490039in}{5.490039in}}%
\pgfusepath{clip}%
\pgfsetbuttcap%
\pgfsetroundjoin%
\definecolor{currentfill}{rgb}{0.273809,0.031497,0.358853}%
\pgfsetfillcolor{currentfill}%
\pgfsetfillopacity{0.700000}%
\pgfsetlinewidth{0.000000pt}%
\definecolor{currentstroke}{rgb}{0.000000,0.000000,0.000000}%
\pgfsetstrokecolor{currentstroke}%
\pgfsetdash{}{0pt}%
\pgfpathmoveto{\pgfqpoint{4.791883in}{1.341410in}}%
\pgfpathlineto{\pgfqpoint{4.805723in}{1.337151in}}%
\pgfpathlineto{\pgfqpoint{4.819570in}{1.332914in}}%
\pgfpathlineto{\pgfqpoint{4.833423in}{1.328700in}}%
\pgfpathlineto{\pgfqpoint{4.847284in}{1.324509in}}%
\pgfpathlineto{\pgfqpoint{4.839711in}{1.318413in}}%
\pgfpathlineto{\pgfqpoint{4.832136in}{1.312575in}}%
\pgfpathlineto{\pgfqpoint{4.824558in}{1.307005in}}%
\pgfpathlineto{\pgfqpoint{4.816978in}{1.301709in}}%
\pgfpathlineto{\pgfqpoint{4.803105in}{1.306217in}}%
\pgfpathlineto{\pgfqpoint{4.789238in}{1.310748in}}%
\pgfpathlineto{\pgfqpoint{4.775378in}{1.315301in}}%
\pgfpathlineto{\pgfqpoint{4.761525in}{1.319876in}}%
\pgfpathlineto{\pgfqpoint{4.769119in}{1.324850in}}%
\pgfpathlineto{\pgfqpoint{4.776709in}{1.330103in}}%
\pgfpathlineto{\pgfqpoint{4.784297in}{1.335625in}}%
\pgfpathlineto{\pgfqpoint{4.791883in}{1.341410in}}%
\pgfpathclose%
\pgfusepath{fill}%
\end{pgfscope}%
\begin{pgfscope}%
\pgfpathrectangle{\pgfqpoint{1.254980in}{0.150000in}}{\pgfqpoint{5.490039in}{5.490039in}}%
\pgfusepath{clip}%
\pgfsetbuttcap%
\pgfsetroundjoin%
\definecolor{currentfill}{rgb}{0.282656,0.100196,0.422160}%
\pgfsetfillcolor{currentfill}%
\pgfsetfillopacity{0.700000}%
\pgfsetlinewidth{0.000000pt}%
\definecolor{currentstroke}{rgb}{0.000000,0.000000,0.000000}%
\pgfsetstrokecolor{currentstroke}%
\pgfsetdash{}{0pt}%
\pgfpathmoveto{\pgfqpoint{4.314350in}{1.447247in}}%
\pgfpathlineto{\pgfqpoint{4.328069in}{1.441365in}}%
\pgfpathlineto{\pgfqpoint{4.341793in}{1.435505in}}%
\pgfpathlineto{\pgfqpoint{4.355523in}{1.429669in}}%
\pgfpathlineto{\pgfqpoint{4.369259in}{1.423855in}}%
\pgfpathlineto{\pgfqpoint{4.361519in}{1.423889in}}%
\pgfpathlineto{\pgfqpoint{4.353773in}{1.424310in}}%
\pgfpathlineto{\pgfqpoint{4.346020in}{1.425130in}}%
\pgfpathlineto{\pgfqpoint{4.338261in}{1.426358in}}%
\pgfpathlineto{\pgfqpoint{4.324502in}{1.432528in}}%
\pgfpathlineto{\pgfqpoint{4.310749in}{1.438721in}}%
\pgfpathlineto{\pgfqpoint{4.297001in}{1.444937in}}%
\pgfpathlineto{\pgfqpoint{4.283259in}{1.451176in}}%
\pgfpathlineto{\pgfqpoint{4.291042in}{1.449587in}}%
\pgfpathlineto{\pgfqpoint{4.298818in}{1.448408in}}%
\pgfpathlineto{\pgfqpoint{4.306587in}{1.447632in}}%
\pgfpathlineto{\pgfqpoint{4.314350in}{1.447247in}}%
\pgfpathclose%
\pgfusepath{fill}%
\end{pgfscope}%
\begin{pgfscope}%
\pgfpathrectangle{\pgfqpoint{1.254980in}{0.150000in}}{\pgfqpoint{5.490039in}{5.490039in}}%
\pgfusepath{clip}%
\pgfsetbuttcap%
\pgfsetroundjoin%
\definecolor{currentfill}{rgb}{0.276022,0.044167,0.370164}%
\pgfsetfillcolor{currentfill}%
\pgfsetfillopacity{0.700000}%
\pgfsetlinewidth{0.000000pt}%
\definecolor{currentstroke}{rgb}{0.000000,0.000000,0.000000}%
\pgfsetstrokecolor{currentstroke}%
\pgfsetdash{}{0pt}%
\pgfpathmoveto{\pgfqpoint{4.650934in}{1.357288in}}%
\pgfpathlineto{\pgfqpoint{4.664735in}{1.352533in}}%
\pgfpathlineto{\pgfqpoint{4.678543in}{1.347800in}}%
\pgfpathlineto{\pgfqpoint{4.692357in}{1.343090in}}%
\pgfpathlineto{\pgfqpoint{4.706177in}{1.338402in}}%
\pgfpathlineto{\pgfqpoint{4.698566in}{1.334040in}}%
\pgfpathlineto{\pgfqpoint{4.690952in}{1.329976in}}%
\pgfpathlineto{\pgfqpoint{4.683334in}{1.326220in}}%
\pgfpathlineto{\pgfqpoint{4.675713in}{1.322780in}}%
\pgfpathlineto{\pgfqpoint{4.661877in}{1.327797in}}%
\pgfpathlineto{\pgfqpoint{4.648047in}{1.332837in}}%
\pgfpathlineto{\pgfqpoint{4.634224in}{1.337899in}}%
\pgfpathlineto{\pgfqpoint{4.620406in}{1.342984in}}%
\pgfpathlineto{\pgfqpoint{4.628044in}{1.346089in}}%
\pgfpathlineto{\pgfqpoint{4.635677in}{1.349514in}}%
\pgfpathlineto{\pgfqpoint{4.643307in}{1.353250in}}%
\pgfpathlineto{\pgfqpoint{4.650934in}{1.357288in}}%
\pgfpathclose%
\pgfusepath{fill}%
\end{pgfscope}%
\begin{pgfscope}%
\pgfpathrectangle{\pgfqpoint{1.254980in}{0.150000in}}{\pgfqpoint{5.490039in}{5.490039in}}%
\pgfusepath{clip}%
\pgfsetbuttcap%
\pgfsetroundjoin%
\definecolor{currentfill}{rgb}{0.139147,0.533812,0.555298}%
\pgfsetfillcolor{currentfill}%
\pgfsetfillopacity{0.700000}%
\pgfsetlinewidth{0.000000pt}%
\definecolor{currentstroke}{rgb}{0.000000,0.000000,0.000000}%
\pgfsetstrokecolor{currentstroke}%
\pgfsetdash{}{0pt}%
\pgfpathmoveto{\pgfqpoint{2.790373in}{2.450841in}}%
\pgfpathlineto{\pgfqpoint{2.803903in}{2.440312in}}%
\pgfpathlineto{\pgfqpoint{2.817435in}{2.429818in}}%
\pgfpathlineto{\pgfqpoint{2.830969in}{2.419357in}}%
\pgfpathlineto{\pgfqpoint{2.844505in}{2.408930in}}%
\pgfpathlineto{\pgfqpoint{2.835446in}{2.429552in}}%
\pgfpathlineto{\pgfqpoint{2.826354in}{2.450894in}}%
\pgfpathlineto{\pgfqpoint{2.817227in}{2.472969in}}%
\pgfpathlineto{\pgfqpoint{2.808065in}{2.495791in}}%
\pgfpathlineto{\pgfqpoint{2.794470in}{2.506668in}}%
\pgfpathlineto{\pgfqpoint{2.780877in}{2.517579in}}%
\pgfpathlineto{\pgfqpoint{2.767286in}{2.528524in}}%
\pgfpathlineto{\pgfqpoint{2.753696in}{2.539503in}}%
\pgfpathlineto{\pgfqpoint{2.762919in}{2.516224in}}%
\pgfpathlineto{\pgfqpoint{2.772106in}{2.493696in}}%
\pgfpathlineto{\pgfqpoint{2.781257in}{2.471907in}}%
\pgfpathlineto{\pgfqpoint{2.790373in}{2.450841in}}%
\pgfpathclose%
\pgfusepath{fill}%
\end{pgfscope}%
\begin{pgfscope}%
\pgfpathrectangle{\pgfqpoint{1.254980in}{0.150000in}}{\pgfqpoint{5.490039in}{5.490039in}}%
\pgfusepath{clip}%
\pgfsetbuttcap%
\pgfsetroundjoin%
\definecolor{currentfill}{rgb}{0.282623,0.140926,0.457517}%
\pgfsetfillcolor{currentfill}%
\pgfsetfillopacity{0.700000}%
\pgfsetlinewidth{0.000000pt}%
\definecolor{currentstroke}{rgb}{0.000000,0.000000,0.000000}%
\pgfsetstrokecolor{currentstroke}%
\pgfsetdash{}{0pt}%
\pgfpathmoveto{\pgfqpoint{4.118762in}{1.527853in}}%
\pgfpathlineto{\pgfqpoint{4.132441in}{1.521335in}}%
\pgfpathlineto{\pgfqpoint{4.146126in}{1.514840in}}%
\pgfpathlineto{\pgfqpoint{4.159816in}{1.508369in}}%
\pgfpathlineto{\pgfqpoint{4.173511in}{1.501921in}}%
\pgfpathlineto{\pgfqpoint{4.165668in}{1.504664in}}%
\pgfpathlineto{\pgfqpoint{4.157817in}{1.507846in}}%
\pgfpathlineto{\pgfqpoint{4.149957in}{1.511476in}}%
\pgfpathlineto{\pgfqpoint{4.142088in}{1.515566in}}%
\pgfpathlineto{\pgfqpoint{4.128366in}{1.522385in}}%
\pgfpathlineto{\pgfqpoint{4.114648in}{1.529228in}}%
\pgfpathlineto{\pgfqpoint{4.100936in}{1.536094in}}%
\pgfpathlineto{\pgfqpoint{4.087228in}{1.542983in}}%
\pgfpathlineto{\pgfqpoint{4.095126in}{1.538516in}}%
\pgfpathlineto{\pgfqpoint{4.103014in}{1.534512in}}%
\pgfpathlineto{\pgfqpoint{4.110893in}{1.530961in}}%
\pgfpathlineto{\pgfqpoint{4.118762in}{1.527853in}}%
\pgfpathclose%
\pgfusepath{fill}%
\end{pgfscope}%
\begin{pgfscope}%
\pgfpathrectangle{\pgfqpoint{1.254980in}{0.150000in}}{\pgfqpoint{5.490039in}{5.490039in}}%
\pgfusepath{clip}%
\pgfsetbuttcap%
\pgfsetroundjoin%
\definecolor{currentfill}{rgb}{0.199430,0.387607,0.554642}%
\pgfsetfillcolor{currentfill}%
\pgfsetfillopacity{0.700000}%
\pgfsetlinewidth{0.000000pt}%
\definecolor{currentstroke}{rgb}{0.000000,0.000000,0.000000}%
\pgfsetstrokecolor{currentstroke}%
\pgfsetdash{}{0pt}%
\pgfpathmoveto{\pgfqpoint{3.258898in}{2.066234in}}%
\pgfpathlineto{\pgfqpoint{3.272456in}{2.057125in}}%
\pgfpathlineto{\pgfqpoint{3.286018in}{2.048044in}}%
\pgfpathlineto{\pgfqpoint{3.299582in}{2.038991in}}%
\pgfpathlineto{\pgfqpoint{3.313149in}{2.029966in}}%
\pgfpathlineto{\pgfqpoint{3.304618in}{2.044482in}}%
\pgfpathlineto{\pgfqpoint{3.296063in}{2.059628in}}%
\pgfpathlineto{\pgfqpoint{3.287485in}{2.075417in}}%
\pgfpathlineto{\pgfqpoint{3.278881in}{2.091861in}}%
\pgfpathlineto{\pgfqpoint{3.265266in}{2.101311in}}%
\pgfpathlineto{\pgfqpoint{3.251653in}{2.110788in}}%
\pgfpathlineto{\pgfqpoint{3.238044in}{2.120294in}}%
\pgfpathlineto{\pgfqpoint{3.224437in}{2.129828in}}%
\pgfpathlineto{\pgfqpoint{3.233090in}{2.112952in}}%
\pgfpathlineto{\pgfqpoint{3.241718in}{2.096737in}}%
\pgfpathlineto{\pgfqpoint{3.250320in}{2.081168in}}%
\pgfpathlineto{\pgfqpoint{3.258898in}{2.066234in}}%
\pgfpathclose%
\pgfusepath{fill}%
\end{pgfscope}%
\begin{pgfscope}%
\pgfpathrectangle{\pgfqpoint{1.254980in}{0.150000in}}{\pgfqpoint{5.490039in}{5.490039in}}%
\pgfusepath{clip}%
\pgfsetbuttcap%
\pgfsetroundjoin%
\definecolor{currentfill}{rgb}{0.273809,0.031497,0.358853}%
\pgfsetfillcolor{currentfill}%
\pgfsetfillopacity{0.700000}%
\pgfsetlinewidth{0.000000pt}%
\definecolor{currentstroke}{rgb}{0.000000,0.000000,0.000000}%
\pgfsetstrokecolor{currentstroke}%
\pgfsetdash{}{0pt}%
\pgfpathmoveto{\pgfqpoint{4.933018in}{1.335993in}}%
\pgfpathlineto{\pgfqpoint{4.946903in}{1.332218in}}%
\pgfpathlineto{\pgfqpoint{4.960794in}{1.328465in}}%
\pgfpathlineto{\pgfqpoint{4.974693in}{1.324734in}}%
\pgfpathlineto{\pgfqpoint{4.988599in}{1.321026in}}%
\pgfpathlineto{\pgfqpoint{4.981057in}{1.313377in}}%
\pgfpathlineto{\pgfqpoint{4.973512in}{1.305948in}}%
\pgfpathlineto{\pgfqpoint{4.965966in}{1.298748in}}%
\pgfpathlineto{\pgfqpoint{4.958418in}{1.291784in}}%
\pgfpathlineto{\pgfqpoint{4.944501in}{1.295796in}}%
\pgfpathlineto{\pgfqpoint{4.930592in}{1.299831in}}%
\pgfpathlineto{\pgfqpoint{4.916690in}{1.303888in}}%
\pgfpathlineto{\pgfqpoint{4.902795in}{1.307967in}}%
\pgfpathlineto{\pgfqpoint{4.910354in}{1.314623in}}%
\pgfpathlineto{\pgfqpoint{4.917911in}{1.321517in}}%
\pgfpathlineto{\pgfqpoint{4.925466in}{1.328643in}}%
\pgfpathlineto{\pgfqpoint{4.933018in}{1.335993in}}%
\pgfpathclose%
\pgfusepath{fill}%
\end{pgfscope}%
\begin{pgfscope}%
\pgfpathrectangle{\pgfqpoint{1.254980in}{0.150000in}}{\pgfqpoint{5.490039in}{5.490039in}}%
\pgfusepath{clip}%
\pgfsetbuttcap%
\pgfsetroundjoin%
\definecolor{currentfill}{rgb}{0.246811,0.283237,0.535941}%
\pgfsetfillcolor{currentfill}%
\pgfsetfillopacity{0.700000}%
\pgfsetlinewidth{0.000000pt}%
\definecolor{currentstroke}{rgb}{0.000000,0.000000,0.000000}%
\pgfsetstrokecolor{currentstroke}%
\pgfsetdash{}{0pt}%
\pgfpathmoveto{\pgfqpoint{3.618258in}{1.811268in}}%
\pgfpathlineto{\pgfqpoint{3.631858in}{1.803209in}}%
\pgfpathlineto{\pgfqpoint{3.645462in}{1.795176in}}%
\pgfpathlineto{\pgfqpoint{3.659070in}{1.787168in}}%
\pgfpathlineto{\pgfqpoint{3.672682in}{1.779185in}}%
\pgfpathlineto{\pgfqpoint{3.664483in}{1.788908in}}%
\pgfpathlineto{\pgfqpoint{3.656268in}{1.799188in}}%
\pgfpathlineto{\pgfqpoint{3.648036in}{1.810035in}}%
\pgfpathlineto{\pgfqpoint{3.639786in}{1.821463in}}%
\pgfpathlineto{\pgfqpoint{3.626135in}{1.829850in}}%
\pgfpathlineto{\pgfqpoint{3.612487in}{1.838262in}}%
\pgfpathlineto{\pgfqpoint{3.598844in}{1.846700in}}%
\pgfpathlineto{\pgfqpoint{3.585204in}{1.855163in}}%
\pgfpathlineto{\pgfqpoint{3.593494in}{1.843325in}}%
\pgfpathlineto{\pgfqpoint{3.601767in}{1.832071in}}%
\pgfpathlineto{\pgfqpoint{3.610021in}{1.821389in}}%
\pgfpathlineto{\pgfqpoint{3.618258in}{1.811268in}}%
\pgfpathclose%
\pgfusepath{fill}%
\end{pgfscope}%
\begin{pgfscope}%
\pgfpathrectangle{\pgfqpoint{1.254980in}{0.150000in}}{\pgfqpoint{5.490039in}{5.490039in}}%
\pgfusepath{clip}%
\pgfsetbuttcap%
\pgfsetroundjoin%
\definecolor{currentfill}{rgb}{0.279566,0.067836,0.391917}%
\pgfsetfillcolor{currentfill}%
\pgfsetfillopacity{0.700000}%
\pgfsetlinewidth{0.000000pt}%
\definecolor{currentstroke}{rgb}{0.000000,0.000000,0.000000}%
\pgfsetstrokecolor{currentstroke}%
\pgfsetdash{}{0pt}%
\pgfpathmoveto{\pgfqpoint{4.510089in}{1.384475in}}%
\pgfpathlineto{\pgfqpoint{4.523858in}{1.379209in}}%
\pgfpathlineto{\pgfqpoint{4.537632in}{1.373966in}}%
\pgfpathlineto{\pgfqpoint{4.551412in}{1.368746in}}%
\pgfpathlineto{\pgfqpoint{4.565199in}{1.363548in}}%
\pgfpathlineto{\pgfqpoint{4.557540in}{1.361109in}}%
\pgfpathlineto{\pgfqpoint{4.549876in}{1.359010in}}%
\pgfpathlineto{\pgfqpoint{4.542209in}{1.357261in}}%
\pgfpathlineto{\pgfqpoint{4.534536in}{1.355872in}}%
\pgfpathlineto{\pgfqpoint{4.520731in}{1.361412in}}%
\pgfpathlineto{\pgfqpoint{4.506932in}{1.366975in}}%
\pgfpathlineto{\pgfqpoint{4.493138in}{1.372561in}}%
\pgfpathlineto{\pgfqpoint{4.479351in}{1.378169in}}%
\pgfpathlineto{\pgfqpoint{4.487043in}{1.379211in}}%
\pgfpathlineto{\pgfqpoint{4.494730in}{1.380615in}}%
\pgfpathlineto{\pgfqpoint{4.502412in}{1.382373in}}%
\pgfpathlineto{\pgfqpoint{4.510089in}{1.384475in}}%
\pgfpathclose%
\pgfusepath{fill}%
\end{pgfscope}%
\begin{pgfscope}%
\pgfpathrectangle{\pgfqpoint{1.254980in}{0.150000in}}{\pgfqpoint{5.490039in}{5.490039in}}%
\pgfusepath{clip}%
\pgfsetbuttcap%
\pgfsetroundjoin%
\definecolor{currentfill}{rgb}{0.360741,0.785964,0.387814}%
\pgfsetfillcolor{currentfill}%
\pgfsetfillopacity{0.700000}%
\pgfsetlinewidth{0.000000pt}%
\definecolor{currentstroke}{rgb}{0.000000,0.000000,0.000000}%
\pgfsetstrokecolor{currentstroke}%
\pgfsetdash{}{0pt}%
\pgfpathmoveto{\pgfqpoint{2.048310in}{3.165431in}}%
\pgfpathlineto{\pgfqpoint{2.061873in}{3.152205in}}%
\pgfpathlineto{\pgfqpoint{2.075434in}{3.139033in}}%
\pgfpathlineto{\pgfqpoint{2.088995in}{3.125916in}}%
\pgfpathlineto{\pgfqpoint{2.102555in}{3.112852in}}%
\pgfpathlineto{\pgfqpoint{2.092474in}{3.142562in}}%
\pgfpathlineto{\pgfqpoint{2.082341in}{3.173116in}}%
\pgfpathlineto{\pgfqpoint{2.072154in}{3.204531in}}%
\pgfpathlineto{\pgfqpoint{2.061912in}{3.236822in}}%
\pgfpathlineto{\pgfqpoint{2.048275in}{3.250377in}}%
\pgfpathlineto{\pgfqpoint{2.034638in}{3.263987in}}%
\pgfpathlineto{\pgfqpoint{2.020999in}{3.277651in}}%
\pgfpathlineto{\pgfqpoint{2.007359in}{3.291370in}}%
\pgfpathlineto{\pgfqpoint{2.017680in}{3.258578in}}%
\pgfpathlineto{\pgfqpoint{2.027944in}{3.226668in}}%
\pgfpathlineto{\pgfqpoint{2.038154in}{3.195624in}}%
\pgfpathlineto{\pgfqpoint{2.048310in}{3.165431in}}%
\pgfpathclose%
\pgfusepath{fill}%
\end{pgfscope}%
\begin{pgfscope}%
\pgfpathrectangle{\pgfqpoint{1.254980in}{0.150000in}}{\pgfqpoint{5.490039in}{5.490039in}}%
\pgfusepath{clip}%
\pgfsetbuttcap%
\pgfsetroundjoin%
\definecolor{currentfill}{rgb}{0.274952,0.037752,0.364543}%
\pgfsetfillcolor{currentfill}%
\pgfsetfillopacity{0.700000}%
\pgfsetlinewidth{0.000000pt}%
\definecolor{currentstroke}{rgb}{0.000000,0.000000,0.000000}%
\pgfsetstrokecolor{currentstroke}%
\pgfsetdash{}{0pt}%
\pgfpathmoveto{\pgfqpoint{5.074413in}{1.340230in}}%
\pgfpathlineto{\pgfqpoint{5.088347in}{1.336926in}}%
\pgfpathlineto{\pgfqpoint{5.102289in}{1.333644in}}%
\pgfpathlineto{\pgfqpoint{5.116238in}{1.330384in}}%
\pgfpathlineto{\pgfqpoint{5.108717in}{1.321427in}}%
\pgfpathlineto{\pgfqpoint{5.101195in}{1.312656in}}%
\pgfpathlineto{\pgfqpoint{5.093671in}{1.304076in}}%
\pgfpathlineto{\pgfqpoint{5.086145in}{1.295696in}}%
\pgfpathlineto{\pgfqpoint{5.072188in}{1.299247in}}%
\pgfpathlineto{\pgfqpoint{5.058239in}{1.302821in}}%
\pgfpathlineto{\pgfqpoint{5.044296in}{1.306417in}}%
\pgfpathlineto{\pgfqpoint{5.051828in}{1.314575in}}%
\pgfpathlineto{\pgfqpoint{5.059358in}{1.322935in}}%
\pgfpathlineto{\pgfqpoint{5.066886in}{1.331489in}}%
\pgfpathlineto{\pgfqpoint{5.074413in}{1.340230in}}%
\pgfpathclose%
\pgfusepath{fill}%
\end{pgfscope}%
\begin{pgfscope}%
\pgfpathrectangle{\pgfqpoint{1.254980in}{0.150000in}}{\pgfqpoint{5.490039in}{5.490039in}}%
\pgfusepath{clip}%
\pgfsetbuttcap%
\pgfsetroundjoin%
\definecolor{currentfill}{rgb}{0.143343,0.522773,0.556295}%
\pgfsetfillcolor{currentfill}%
\pgfsetfillopacity{0.700000}%
\pgfsetlinewidth{0.000000pt}%
\definecolor{currentstroke}{rgb}{0.000000,0.000000,0.000000}%
\pgfsetstrokecolor{currentstroke}%
\pgfsetdash{}{0pt}%
\pgfpathmoveto{\pgfqpoint{2.844505in}{2.408930in}}%
\pgfpathlineto{\pgfqpoint{2.858043in}{2.398536in}}%
\pgfpathlineto{\pgfqpoint{2.871582in}{2.388176in}}%
\pgfpathlineto{\pgfqpoint{2.885124in}{2.377848in}}%
\pgfpathlineto{\pgfqpoint{2.898668in}{2.367554in}}%
\pgfpathlineto{\pgfqpoint{2.889667in}{2.387734in}}%
\pgfpathlineto{\pgfqpoint{2.880632in}{2.408630in}}%
\pgfpathlineto{\pgfqpoint{2.871564in}{2.430253in}}%
\pgfpathlineto{\pgfqpoint{2.862462in}{2.452618in}}%
\pgfpathlineto{\pgfqpoint{2.848860in}{2.463362in}}%
\pgfpathlineto{\pgfqpoint{2.835260in}{2.474138in}}%
\pgfpathlineto{\pgfqpoint{2.821662in}{2.484948in}}%
\pgfpathlineto{\pgfqpoint{2.808065in}{2.495791in}}%
\pgfpathlineto{\pgfqpoint{2.817227in}{2.472969in}}%
\pgfpathlineto{\pgfqpoint{2.826354in}{2.450894in}}%
\pgfpathlineto{\pgfqpoint{2.835446in}{2.429552in}}%
\pgfpathlineto{\pgfqpoint{2.844505in}{2.408930in}}%
\pgfpathclose%
\pgfusepath{fill}%
\end{pgfscope}%
\begin{pgfscope}%
\pgfpathrectangle{\pgfqpoint{1.254980in}{0.150000in}}{\pgfqpoint{5.490039in}{5.490039in}}%
\pgfusepath{clip}%
\pgfsetbuttcap%
\pgfsetroundjoin%
\definecolor{currentfill}{rgb}{0.319809,0.770914,0.411152}%
\pgfsetfillcolor{currentfill}%
\pgfsetfillopacity{0.700000}%
\pgfsetlinewidth{0.000000pt}%
\definecolor{currentstroke}{rgb}{0.000000,0.000000,0.000000}%
\pgfsetstrokecolor{currentstroke}%
\pgfsetdash{}{0pt}%
\pgfpathmoveto{\pgfqpoint{2.102555in}{3.112852in}}%
\pgfpathlineto{\pgfqpoint{2.116115in}{3.099841in}}%
\pgfpathlineto{\pgfqpoint{2.129674in}{3.086883in}}%
\pgfpathlineto{\pgfqpoint{2.143232in}{3.073976in}}%
\pgfpathlineto{\pgfqpoint{2.156791in}{3.061120in}}%
\pgfpathlineto{\pgfqpoint{2.146783in}{3.090348in}}%
\pgfpathlineto{\pgfqpoint{2.136725in}{3.120416in}}%
\pgfpathlineto{\pgfqpoint{2.126615in}{3.151338in}}%
\pgfpathlineto{\pgfqpoint{2.116451in}{3.183131in}}%
\pgfpathlineto{\pgfqpoint{2.102817in}{3.196475in}}%
\pgfpathlineto{\pgfqpoint{2.089183in}{3.209872in}}%
\pgfpathlineto{\pgfqpoint{2.075548in}{3.223321in}}%
\pgfpathlineto{\pgfqpoint{2.061912in}{3.236822in}}%
\pgfpathlineto{\pgfqpoint{2.072154in}{3.204531in}}%
\pgfpathlineto{\pgfqpoint{2.082341in}{3.173116in}}%
\pgfpathlineto{\pgfqpoint{2.092474in}{3.142562in}}%
\pgfpathlineto{\pgfqpoint{2.102555in}{3.112852in}}%
\pgfpathclose%
\pgfusepath{fill}%
\end{pgfscope}%
\begin{pgfscope}%
\pgfpathrectangle{\pgfqpoint{1.254980in}{0.150000in}}{\pgfqpoint{5.490039in}{5.490039in}}%
\pgfusepath{clip}%
\pgfsetbuttcap%
\pgfsetroundjoin%
\definecolor{currentfill}{rgb}{0.274128,0.199721,0.498911}%
\pgfsetfillcolor{currentfill}%
\pgfsetfillopacity{0.700000}%
\pgfsetlinewidth{0.000000pt}%
\definecolor{currentstroke}{rgb}{0.000000,0.000000,0.000000}%
\pgfsetstrokecolor{currentstroke}%
\pgfsetdash{}{0pt}%
\pgfpathmoveto{\pgfqpoint{3.923112in}{1.627505in}}%
\pgfpathlineto{\pgfqpoint{3.936763in}{1.620330in}}%
\pgfpathlineto{\pgfqpoint{3.950418in}{1.613179in}}%
\pgfpathlineto{\pgfqpoint{3.964077in}{1.606052in}}%
\pgfpathlineto{\pgfqpoint{3.977742in}{1.598949in}}%
\pgfpathlineto{\pgfqpoint{3.969773in}{1.604651in}}%
\pgfpathlineto{\pgfqpoint{3.961792in}{1.610846in}}%
\pgfpathlineto{\pgfqpoint{3.953799in}{1.617543in}}%
\pgfpathlineto{\pgfqpoint{3.945794in}{1.624755in}}%
\pgfpathlineto{\pgfqpoint{3.932097in}{1.632244in}}%
\pgfpathlineto{\pgfqpoint{3.918405in}{1.639758in}}%
\pgfpathlineto{\pgfqpoint{3.904717in}{1.647296in}}%
\pgfpathlineto{\pgfqpoint{3.891033in}{1.654858in}}%
\pgfpathlineto{\pgfqpoint{3.899072in}{1.647254in}}%
\pgfpathlineto{\pgfqpoint{3.907098in}{1.640167in}}%
\pgfpathlineto{\pgfqpoint{3.915111in}{1.633588in}}%
\pgfpathlineto{\pgfqpoint{3.923112in}{1.627505in}}%
\pgfpathclose%
\pgfusepath{fill}%
\end{pgfscope}%
\begin{pgfscope}%
\pgfpathrectangle{\pgfqpoint{1.254980in}{0.150000in}}{\pgfqpoint{5.490039in}{5.490039in}}%
\pgfusepath{clip}%
\pgfsetbuttcap%
\pgfsetroundjoin%
\definecolor{currentfill}{rgb}{0.203063,0.379716,0.553925}%
\pgfsetfillcolor{currentfill}%
\pgfsetfillopacity{0.700000}%
\pgfsetlinewidth{0.000000pt}%
\definecolor{currentstroke}{rgb}{0.000000,0.000000,0.000000}%
\pgfsetstrokecolor{currentstroke}%
\pgfsetdash{}{0pt}%
\pgfpathmoveto{\pgfqpoint{3.313149in}{2.029966in}}%
\pgfpathlineto{\pgfqpoint{3.326720in}{2.020969in}}%
\pgfpathlineto{\pgfqpoint{3.340294in}{2.012000in}}%
\pgfpathlineto{\pgfqpoint{3.353871in}{2.003058in}}%
\pgfpathlineto{\pgfqpoint{3.367451in}{1.994144in}}%
\pgfpathlineto{\pgfqpoint{3.358966in}{2.008242in}}%
\pgfpathlineto{\pgfqpoint{3.350458in}{2.022967in}}%
\pgfpathlineto{\pgfqpoint{3.341927in}{2.038329in}}%
\pgfpathlineto{\pgfqpoint{3.333373in}{2.054343in}}%
\pgfpathlineto{\pgfqpoint{3.319745in}{2.063681in}}%
\pgfpathlineto{\pgfqpoint{3.306121in}{2.073047in}}%
\pgfpathlineto{\pgfqpoint{3.292499in}{2.082440in}}%
\pgfpathlineto{\pgfqpoint{3.278881in}{2.091861in}}%
\pgfpathlineto{\pgfqpoint{3.287485in}{2.075417in}}%
\pgfpathlineto{\pgfqpoint{3.296063in}{2.059628in}}%
\pgfpathlineto{\pgfqpoint{3.304618in}{2.044482in}}%
\pgfpathlineto{\pgfqpoint{3.313149in}{2.029966in}}%
\pgfpathclose%
\pgfusepath{fill}%
\end{pgfscope}%
\begin{pgfscope}%
\pgfpathrectangle{\pgfqpoint{1.254980in}{0.150000in}}{\pgfqpoint{5.490039in}{5.490039in}}%
\pgfusepath{clip}%
\pgfsetbuttcap%
\pgfsetroundjoin%
\definecolor{currentfill}{rgb}{0.288921,0.758394,0.428426}%
\pgfsetfillcolor{currentfill}%
\pgfsetfillopacity{0.700000}%
\pgfsetlinewidth{0.000000pt}%
\definecolor{currentstroke}{rgb}{0.000000,0.000000,0.000000}%
\pgfsetstrokecolor{currentstroke}%
\pgfsetdash{}{0pt}%
\pgfpathmoveto{\pgfqpoint{2.156791in}{3.061120in}}%
\pgfpathlineto{\pgfqpoint{2.170348in}{3.048316in}}%
\pgfpathlineto{\pgfqpoint{2.183906in}{3.035561in}}%
\pgfpathlineto{\pgfqpoint{2.197463in}{3.022856in}}%
\pgfpathlineto{\pgfqpoint{2.211019in}{3.010201in}}%
\pgfpathlineto{\pgfqpoint{2.201085in}{3.038949in}}%
\pgfpathlineto{\pgfqpoint{2.191101in}{3.068532in}}%
\pgfpathlineto{\pgfqpoint{2.181066in}{3.098963in}}%
\pgfpathlineto{\pgfqpoint{2.170979in}{3.130260in}}%
\pgfpathlineto{\pgfqpoint{2.157348in}{3.143402in}}%
\pgfpathlineto{\pgfqpoint{2.143716in}{3.156595in}}%
\pgfpathlineto{\pgfqpoint{2.130084in}{3.169837in}}%
\pgfpathlineto{\pgfqpoint{2.116451in}{3.183131in}}%
\pgfpathlineto{\pgfqpoint{2.126615in}{3.151338in}}%
\pgfpathlineto{\pgfqpoint{2.136725in}{3.120416in}}%
\pgfpathlineto{\pgfqpoint{2.146783in}{3.090348in}}%
\pgfpathlineto{\pgfqpoint{2.156791in}{3.061120in}}%
\pgfpathclose%
\pgfusepath{fill}%
\end{pgfscope}%
\begin{pgfscope}%
\pgfpathrectangle{\pgfqpoint{1.254980in}{0.150000in}}{\pgfqpoint{5.490039in}{5.490039in}}%
\pgfusepath{clip}%
\pgfsetbuttcap%
\pgfsetroundjoin%
\definecolor{currentfill}{rgb}{0.282327,0.094955,0.417331}%
\pgfsetfillcolor{currentfill}%
\pgfsetfillopacity{0.700000}%
\pgfsetlinewidth{0.000000pt}%
\definecolor{currentstroke}{rgb}{0.000000,0.000000,0.000000}%
\pgfsetstrokecolor{currentstroke}%
\pgfsetdash{}{0pt}%
\pgfpathmoveto{\pgfqpoint{4.369259in}{1.423855in}}%
\pgfpathlineto{\pgfqpoint{4.383001in}{1.418065in}}%
\pgfpathlineto{\pgfqpoint{4.396748in}{1.412297in}}%
\pgfpathlineto{\pgfqpoint{4.410500in}{1.406552in}}%
\pgfpathlineto{\pgfqpoint{4.424259in}{1.400830in}}%
\pgfpathlineto{\pgfqpoint{4.416541in}{1.400512in}}%
\pgfpathlineto{\pgfqpoint{4.408817in}{1.400579in}}%
\pgfpathlineto{\pgfqpoint{4.401087in}{1.401040in}}%
\pgfpathlineto{\pgfqpoint{4.393351in}{1.401906in}}%
\pgfpathlineto{\pgfqpoint{4.379570in}{1.407985in}}%
\pgfpathlineto{\pgfqpoint{4.365795in}{1.414086in}}%
\pgfpathlineto{\pgfqpoint{4.352025in}{1.420211in}}%
\pgfpathlineto{\pgfqpoint{4.338261in}{1.426358in}}%
\pgfpathlineto{\pgfqpoint{4.346020in}{1.425130in}}%
\pgfpathlineto{\pgfqpoint{4.353773in}{1.424310in}}%
\pgfpathlineto{\pgfqpoint{4.361519in}{1.423889in}}%
\pgfpathlineto{\pgfqpoint{4.369259in}{1.423855in}}%
\pgfpathclose%
\pgfusepath{fill}%
\end{pgfscope}%
\begin{pgfscope}%
\pgfpathrectangle{\pgfqpoint{1.254980in}{0.150000in}}{\pgfqpoint{5.490039in}{5.490039in}}%
\pgfusepath{clip}%
\pgfsetbuttcap%
\pgfsetroundjoin%
\definecolor{currentfill}{rgb}{0.282884,0.135920,0.453427}%
\pgfsetfillcolor{currentfill}%
\pgfsetfillopacity{0.700000}%
\pgfsetlinewidth{0.000000pt}%
\definecolor{currentstroke}{rgb}{0.000000,0.000000,0.000000}%
\pgfsetstrokecolor{currentstroke}%
\pgfsetdash{}{0pt}%
\pgfpathmoveto{\pgfqpoint{4.173511in}{1.501921in}}%
\pgfpathlineto{\pgfqpoint{4.187211in}{1.495497in}}%
\pgfpathlineto{\pgfqpoint{4.200916in}{1.489096in}}%
\pgfpathlineto{\pgfqpoint{4.214627in}{1.482718in}}%
\pgfpathlineto{\pgfqpoint{4.228342in}{1.476363in}}%
\pgfpathlineto{\pgfqpoint{4.220526in}{1.478740in}}%
\pgfpathlineto{\pgfqpoint{4.212702in}{1.481553in}}%
\pgfpathlineto{\pgfqpoint{4.204869in}{1.484810in}}%
\pgfpathlineto{\pgfqpoint{4.197028in}{1.488524in}}%
\pgfpathlineto{\pgfqpoint{4.183285in}{1.495250in}}%
\pgfpathlineto{\pgfqpoint{4.169548in}{1.501999in}}%
\pgfpathlineto{\pgfqpoint{4.155815in}{1.508771in}}%
\pgfpathlineto{\pgfqpoint{4.142088in}{1.515566in}}%
\pgfpathlineto{\pgfqpoint{4.149957in}{1.511476in}}%
\pgfpathlineto{\pgfqpoint{4.157817in}{1.507846in}}%
\pgfpathlineto{\pgfqpoint{4.165668in}{1.504664in}}%
\pgfpathlineto{\pgfqpoint{4.173511in}{1.501921in}}%
\pgfpathclose%
\pgfusepath{fill}%
\end{pgfscope}%
\begin{pgfscope}%
\pgfpathrectangle{\pgfqpoint{1.254980in}{0.150000in}}{\pgfqpoint{5.490039in}{5.490039in}}%
\pgfusepath{clip}%
\pgfsetbuttcap%
\pgfsetroundjoin%
\definecolor{currentfill}{rgb}{0.250425,0.274290,0.533103}%
\pgfsetfillcolor{currentfill}%
\pgfsetfillopacity{0.700000}%
\pgfsetlinewidth{0.000000pt}%
\definecolor{currentstroke}{rgb}{0.000000,0.000000,0.000000}%
\pgfsetstrokecolor{currentstroke}%
\pgfsetdash{}{0pt}%
\pgfpathmoveto{\pgfqpoint{3.672682in}{1.779185in}}%
\pgfpathlineto{\pgfqpoint{3.686297in}{1.771228in}}%
\pgfpathlineto{\pgfqpoint{3.699917in}{1.763296in}}%
\pgfpathlineto{\pgfqpoint{3.713541in}{1.755390in}}%
\pgfpathlineto{\pgfqpoint{3.727169in}{1.747508in}}%
\pgfpathlineto{\pgfqpoint{3.719008in}{1.756833in}}%
\pgfpathlineto{\pgfqpoint{3.710832in}{1.766710in}}%
\pgfpathlineto{\pgfqpoint{3.702639in}{1.777152in}}%
\pgfpathlineto{\pgfqpoint{3.694430in}{1.788170in}}%
\pgfpathlineto{\pgfqpoint{3.680763in}{1.796455in}}%
\pgfpathlineto{\pgfqpoint{3.667100in}{1.804766in}}%
\pgfpathlineto{\pgfqpoint{3.653441in}{1.813102in}}%
\pgfpathlineto{\pgfqpoint{3.639786in}{1.821463in}}%
\pgfpathlineto{\pgfqpoint{3.648036in}{1.810035in}}%
\pgfpathlineto{\pgfqpoint{3.656268in}{1.799188in}}%
\pgfpathlineto{\pgfqpoint{3.664483in}{1.788908in}}%
\pgfpathlineto{\pgfqpoint{3.672682in}{1.779185in}}%
\pgfpathclose%
\pgfusepath{fill}%
\end{pgfscope}%
\begin{pgfscope}%
\pgfpathrectangle{\pgfqpoint{1.254980in}{0.150000in}}{\pgfqpoint{5.490039in}{5.490039in}}%
\pgfusepath{clip}%
\pgfsetbuttcap%
\pgfsetroundjoin%
\definecolor{currentfill}{rgb}{0.276022,0.044167,0.370164}%
\pgfsetfillcolor{currentfill}%
\pgfsetfillopacity{0.700000}%
\pgfsetlinewidth{0.000000pt}%
\definecolor{currentstroke}{rgb}{0.000000,0.000000,0.000000}%
\pgfsetstrokecolor{currentstroke}%
\pgfsetdash{}{0pt}%
\pgfpathmoveto{\pgfqpoint{4.706177in}{1.338402in}}%
\pgfpathlineto{\pgfqpoint{4.720004in}{1.333737in}}%
\pgfpathlineto{\pgfqpoint{4.733838in}{1.329094in}}%
\pgfpathlineto{\pgfqpoint{4.747678in}{1.324474in}}%
\pgfpathlineto{\pgfqpoint{4.761525in}{1.319876in}}%
\pgfpathlineto{\pgfqpoint{4.753928in}{1.315189in}}%
\pgfpathlineto{\pgfqpoint{4.746329in}{1.310798in}}%
\pgfpathlineto{\pgfqpoint{4.738727in}{1.306710in}}%
\pgfpathlineto{\pgfqpoint{4.731121in}{1.302935in}}%
\pgfpathlineto{\pgfqpoint{4.717260in}{1.307863in}}%
\pgfpathlineto{\pgfqpoint{4.703404in}{1.312813in}}%
\pgfpathlineto{\pgfqpoint{4.689556in}{1.317785in}}%
\pgfpathlineto{\pgfqpoint{4.675713in}{1.322780in}}%
\pgfpathlineto{\pgfqpoint{4.683334in}{1.326220in}}%
\pgfpathlineto{\pgfqpoint{4.690952in}{1.329976in}}%
\pgfpathlineto{\pgfqpoint{4.698566in}{1.334040in}}%
\pgfpathlineto{\pgfqpoint{4.706177in}{1.338402in}}%
\pgfpathclose%
\pgfusepath{fill}%
\end{pgfscope}%
\begin{pgfscope}%
\pgfpathrectangle{\pgfqpoint{1.254980in}{0.150000in}}{\pgfqpoint{5.490039in}{5.490039in}}%
\pgfusepath{clip}%
\pgfsetbuttcap%
\pgfsetroundjoin%
\definecolor{currentfill}{rgb}{0.149039,0.508051,0.557250}%
\pgfsetfillcolor{currentfill}%
\pgfsetfillopacity{0.700000}%
\pgfsetlinewidth{0.000000pt}%
\definecolor{currentstroke}{rgb}{0.000000,0.000000,0.000000}%
\pgfsetstrokecolor{currentstroke}%
\pgfsetdash{}{0pt}%
\pgfpathmoveto{\pgfqpoint{2.898668in}{2.367554in}}%
\pgfpathlineto{\pgfqpoint{2.912214in}{2.357292in}}%
\pgfpathlineto{\pgfqpoint{2.925763in}{2.347062in}}%
\pgfpathlineto{\pgfqpoint{2.939313in}{2.336864in}}%
\pgfpathlineto{\pgfqpoint{2.952866in}{2.326699in}}%
\pgfpathlineto{\pgfqpoint{2.943920in}{2.346439in}}%
\pgfpathlineto{\pgfqpoint{2.934943in}{2.366888in}}%
\pgfpathlineto{\pgfqpoint{2.925933in}{2.388061in}}%
\pgfpathlineto{\pgfqpoint{2.916890in}{2.409971in}}%
\pgfpathlineto{\pgfqpoint{2.903280in}{2.420585in}}%
\pgfpathlineto{\pgfqpoint{2.889672in}{2.431230in}}%
\pgfpathlineto{\pgfqpoint{2.876066in}{2.441908in}}%
\pgfpathlineto{\pgfqpoint{2.862462in}{2.452618in}}%
\pgfpathlineto{\pgfqpoint{2.871564in}{2.430253in}}%
\pgfpathlineto{\pgfqpoint{2.880632in}{2.408630in}}%
\pgfpathlineto{\pgfqpoint{2.889667in}{2.387734in}}%
\pgfpathlineto{\pgfqpoint{2.898668in}{2.367554in}}%
\pgfpathclose%
\pgfusepath{fill}%
\end{pgfscope}%
\begin{pgfscope}%
\pgfpathrectangle{\pgfqpoint{1.254980in}{0.150000in}}{\pgfqpoint{5.490039in}{5.490039in}}%
\pgfusepath{clip}%
\pgfsetbuttcap%
\pgfsetroundjoin%
\definecolor{currentfill}{rgb}{0.273809,0.031497,0.358853}%
\pgfsetfillcolor{currentfill}%
\pgfsetfillopacity{0.700000}%
\pgfsetlinewidth{0.000000pt}%
\definecolor{currentstroke}{rgb}{0.000000,0.000000,0.000000}%
\pgfsetstrokecolor{currentstroke}%
\pgfsetdash{}{0pt}%
\pgfpathmoveto{\pgfqpoint{4.847284in}{1.324509in}}%
\pgfpathlineto{\pgfqpoint{4.861151in}{1.320340in}}%
\pgfpathlineto{\pgfqpoint{4.875026in}{1.316193in}}%
\pgfpathlineto{\pgfqpoint{4.888907in}{1.312069in}}%
\pgfpathlineto{\pgfqpoint{4.902795in}{1.307967in}}%
\pgfpathlineto{\pgfqpoint{4.895234in}{1.301559in}}%
\pgfpathlineto{\pgfqpoint{4.887671in}{1.295407in}}%
\pgfpathlineto{\pgfqpoint{4.880105in}{1.289518in}}%
\pgfpathlineto{\pgfqpoint{4.872538in}{1.283901in}}%
\pgfpathlineto{\pgfqpoint{4.858638in}{1.288319in}}%
\pgfpathlineto{\pgfqpoint{4.844744in}{1.292760in}}%
\pgfpathlineto{\pgfqpoint{4.830858in}{1.297224in}}%
\pgfpathlineto{\pgfqpoint{4.816978in}{1.301709in}}%
\pgfpathlineto{\pgfqpoint{4.824558in}{1.307005in}}%
\pgfpathlineto{\pgfqpoint{4.832136in}{1.312575in}}%
\pgfpathlineto{\pgfqpoint{4.839711in}{1.318413in}}%
\pgfpathlineto{\pgfqpoint{4.847284in}{1.324509in}}%
\pgfpathclose%
\pgfusepath{fill}%
\end{pgfscope}%
\begin{pgfscope}%
\pgfpathrectangle{\pgfqpoint{1.254980in}{0.150000in}}{\pgfqpoint{5.490039in}{5.490039in}}%
\pgfusepath{clip}%
\pgfsetbuttcap%
\pgfsetroundjoin%
\definecolor{currentfill}{rgb}{0.252899,0.742211,0.448284}%
\pgfsetfillcolor{currentfill}%
\pgfsetfillopacity{0.700000}%
\pgfsetlinewidth{0.000000pt}%
\definecolor{currentstroke}{rgb}{0.000000,0.000000,0.000000}%
\pgfsetstrokecolor{currentstroke}%
\pgfsetdash{}{0pt}%
\pgfpathmoveto{\pgfqpoint{2.211019in}{3.010201in}}%
\pgfpathlineto{\pgfqpoint{2.224576in}{2.997594in}}%
\pgfpathlineto{\pgfqpoint{2.238133in}{2.985035in}}%
\pgfpathlineto{\pgfqpoint{2.251689in}{2.972525in}}%
\pgfpathlineto{\pgfqpoint{2.265246in}{2.960061in}}%
\pgfpathlineto{\pgfqpoint{2.255383in}{2.988332in}}%
\pgfpathlineto{\pgfqpoint{2.245472in}{3.017431in}}%
\pgfpathlineto{\pgfqpoint{2.235512in}{3.047374in}}%
\pgfpathlineto{\pgfqpoint{2.225500in}{3.078177in}}%
\pgfpathlineto{\pgfqpoint{2.211871in}{3.091125in}}%
\pgfpathlineto{\pgfqpoint{2.198240in}{3.104122in}}%
\pgfpathlineto{\pgfqpoint{2.184610in}{3.117166in}}%
\pgfpathlineto{\pgfqpoint{2.170979in}{3.130260in}}%
\pgfpathlineto{\pgfqpoint{2.181066in}{3.098963in}}%
\pgfpathlineto{\pgfqpoint{2.191101in}{3.068532in}}%
\pgfpathlineto{\pgfqpoint{2.201085in}{3.038949in}}%
\pgfpathlineto{\pgfqpoint{2.211019in}{3.010201in}}%
\pgfpathclose%
\pgfusepath{fill}%
\end{pgfscope}%
\begin{pgfscope}%
\pgfpathrectangle{\pgfqpoint{1.254980in}{0.150000in}}{\pgfqpoint{5.490039in}{5.490039in}}%
\pgfusepath{clip}%
\pgfsetbuttcap%
\pgfsetroundjoin%
\definecolor{currentfill}{rgb}{0.278791,0.062145,0.386592}%
\pgfsetfillcolor{currentfill}%
\pgfsetfillopacity{0.700000}%
\pgfsetlinewidth{0.000000pt}%
\definecolor{currentstroke}{rgb}{0.000000,0.000000,0.000000}%
\pgfsetstrokecolor{currentstroke}%
\pgfsetdash{}{0pt}%
\pgfpathmoveto{\pgfqpoint{4.565199in}{1.363548in}}%
\pgfpathlineto{\pgfqpoint{4.578991in}{1.358373in}}%
\pgfpathlineto{\pgfqpoint{4.592790in}{1.353221in}}%
\pgfpathlineto{\pgfqpoint{4.606595in}{1.348091in}}%
\pgfpathlineto{\pgfqpoint{4.620406in}{1.342984in}}%
\pgfpathlineto{\pgfqpoint{4.612765in}{1.340206in}}%
\pgfpathlineto{\pgfqpoint{4.605120in}{1.337767in}}%
\pgfpathlineto{\pgfqpoint{4.597470in}{1.335673in}}%
\pgfpathlineto{\pgfqpoint{4.589817in}{1.333935in}}%
\pgfpathlineto{\pgfqpoint{4.575988in}{1.339386in}}%
\pgfpathlineto{\pgfqpoint{4.562165in}{1.344859in}}%
\pgfpathlineto{\pgfqpoint{4.548347in}{1.350354in}}%
\pgfpathlineto{\pgfqpoint{4.534536in}{1.355872in}}%
\pgfpathlineto{\pgfqpoint{4.542209in}{1.357261in}}%
\pgfpathlineto{\pgfqpoint{4.549876in}{1.359010in}}%
\pgfpathlineto{\pgfqpoint{4.557540in}{1.361109in}}%
\pgfpathlineto{\pgfqpoint{4.565199in}{1.363548in}}%
\pgfpathclose%
\pgfusepath{fill}%
\end{pgfscope}%
\begin{pgfscope}%
\pgfpathrectangle{\pgfqpoint{1.254980in}{0.150000in}}{\pgfqpoint{5.490039in}{5.490039in}}%
\pgfusepath{clip}%
\pgfsetbuttcap%
\pgfsetroundjoin%
\definecolor{currentfill}{rgb}{0.273809,0.031497,0.358853}%
\pgfsetfillcolor{currentfill}%
\pgfsetfillopacity{0.700000}%
\pgfsetlinewidth{0.000000pt}%
\definecolor{currentstroke}{rgb}{0.000000,0.000000,0.000000}%
\pgfsetstrokecolor{currentstroke}%
\pgfsetdash{}{0pt}%
\pgfpathmoveto{\pgfqpoint{4.988599in}{1.321026in}}%
\pgfpathlineto{\pgfqpoint{5.002513in}{1.317340in}}%
\pgfpathlineto{\pgfqpoint{5.016433in}{1.313676in}}%
\pgfpathlineto{\pgfqpoint{5.030361in}{1.310036in}}%
\pgfpathlineto{\pgfqpoint{5.044296in}{1.306417in}}%
\pgfpathlineto{\pgfqpoint{5.036763in}{1.298469in}}%
\pgfpathlineto{\pgfqpoint{5.029228in}{1.290738in}}%
\pgfpathlineto{\pgfqpoint{5.021691in}{1.283232in}}%
\pgfpathlineto{\pgfqpoint{5.014153in}{1.275959in}}%
\pgfpathlineto{\pgfqpoint{5.000208in}{1.279882in}}%
\pgfpathlineto{\pgfqpoint{4.986271in}{1.283827in}}%
\pgfpathlineto{\pgfqpoint{4.972341in}{1.287794in}}%
\pgfpathlineto{\pgfqpoint{4.958418in}{1.291784in}}%
\pgfpathlineto{\pgfqpoint{4.965966in}{1.298748in}}%
\pgfpathlineto{\pgfqpoint{4.973512in}{1.305948in}}%
\pgfpathlineto{\pgfqpoint{4.981057in}{1.313377in}}%
\pgfpathlineto{\pgfqpoint{4.988599in}{1.321026in}}%
\pgfpathclose%
\pgfusepath{fill}%
\end{pgfscope}%
\begin{pgfscope}%
\pgfpathrectangle{\pgfqpoint{1.254980in}{0.150000in}}{\pgfqpoint{5.490039in}{5.490039in}}%
\pgfusepath{clip}%
\pgfsetbuttcap%
\pgfsetroundjoin%
\definecolor{currentfill}{rgb}{0.208623,0.367752,0.552675}%
\pgfsetfillcolor{currentfill}%
\pgfsetfillopacity{0.700000}%
\pgfsetlinewidth{0.000000pt}%
\definecolor{currentstroke}{rgb}{0.000000,0.000000,0.000000}%
\pgfsetstrokecolor{currentstroke}%
\pgfsetdash{}{0pt}%
\pgfpathmoveto{\pgfqpoint{3.367451in}{1.994144in}}%
\pgfpathlineto{\pgfqpoint{3.381035in}{1.985257in}}%
\pgfpathlineto{\pgfqpoint{3.394622in}{1.976397in}}%
\pgfpathlineto{\pgfqpoint{3.408212in}{1.967565in}}%
\pgfpathlineto{\pgfqpoint{3.421806in}{1.958759in}}%
\pgfpathlineto{\pgfqpoint{3.413366in}{1.972441in}}%
\pgfpathlineto{\pgfqpoint{3.404904in}{1.986744in}}%
\pgfpathlineto{\pgfqpoint{3.396421in}{2.001681in}}%
\pgfpathlineto{\pgfqpoint{3.387914in}{2.017265in}}%
\pgfpathlineto{\pgfqpoint{3.374274in}{2.026494in}}%
\pgfpathlineto{\pgfqpoint{3.360637in}{2.035750in}}%
\pgfpathlineto{\pgfqpoint{3.347003in}{2.045033in}}%
\pgfpathlineto{\pgfqpoint{3.333373in}{2.054343in}}%
\pgfpathlineto{\pgfqpoint{3.341927in}{2.038329in}}%
\pgfpathlineto{\pgfqpoint{3.350458in}{2.022967in}}%
\pgfpathlineto{\pgfqpoint{3.358966in}{2.008242in}}%
\pgfpathlineto{\pgfqpoint{3.367451in}{1.994144in}}%
\pgfpathclose%
\pgfusepath{fill}%
\end{pgfscope}%
\begin{pgfscope}%
\pgfpathrectangle{\pgfqpoint{1.254980in}{0.150000in}}{\pgfqpoint{5.490039in}{5.490039in}}%
\pgfusepath{clip}%
\pgfsetbuttcap%
\pgfsetroundjoin%
\definecolor{currentfill}{rgb}{0.220124,0.725509,0.466226}%
\pgfsetfillcolor{currentfill}%
\pgfsetfillopacity{0.700000}%
\pgfsetlinewidth{0.000000pt}%
\definecolor{currentstroke}{rgb}{0.000000,0.000000,0.000000}%
\pgfsetstrokecolor{currentstroke}%
\pgfsetdash{}{0pt}%
\pgfpathmoveto{\pgfqpoint{2.265246in}{2.960061in}}%
\pgfpathlineto{\pgfqpoint{2.278802in}{2.947644in}}%
\pgfpathlineto{\pgfqpoint{2.292359in}{2.935274in}}%
\pgfpathlineto{\pgfqpoint{2.305916in}{2.922949in}}%
\pgfpathlineto{\pgfqpoint{2.319473in}{2.910670in}}%
\pgfpathlineto{\pgfqpoint{2.309682in}{2.938466in}}%
\pgfpathlineto{\pgfqpoint{2.299843in}{2.967084in}}%
\pgfpathlineto{\pgfqpoint{2.289955in}{2.996540in}}%
\pgfpathlineto{\pgfqpoint{2.280019in}{3.026850in}}%
\pgfpathlineto{\pgfqpoint{2.266389in}{3.039612in}}%
\pgfpathlineto{\pgfqpoint{2.252760in}{3.052421in}}%
\pgfpathlineto{\pgfqpoint{2.239130in}{3.065275in}}%
\pgfpathlineto{\pgfqpoint{2.225500in}{3.078177in}}%
\pgfpathlineto{\pgfqpoint{2.235512in}{3.047374in}}%
\pgfpathlineto{\pgfqpoint{2.245472in}{3.017431in}}%
\pgfpathlineto{\pgfqpoint{2.255383in}{2.988332in}}%
\pgfpathlineto{\pgfqpoint{2.265246in}{2.960061in}}%
\pgfpathclose%
\pgfusepath{fill}%
\end{pgfscope}%
\begin{pgfscope}%
\pgfpathrectangle{\pgfqpoint{1.254980in}{0.150000in}}{\pgfqpoint{5.490039in}{5.490039in}}%
\pgfusepath{clip}%
\pgfsetbuttcap%
\pgfsetroundjoin%
\definecolor{currentfill}{rgb}{0.276194,0.190074,0.493001}%
\pgfsetfillcolor{currentfill}%
\pgfsetfillopacity{0.700000}%
\pgfsetlinewidth{0.000000pt}%
\definecolor{currentstroke}{rgb}{0.000000,0.000000,0.000000}%
\pgfsetstrokecolor{currentstroke}%
\pgfsetdash{}{0pt}%
\pgfpathmoveto{\pgfqpoint{3.977742in}{1.598949in}}%
\pgfpathlineto{\pgfqpoint{3.991411in}{1.591870in}}%
\pgfpathlineto{\pgfqpoint{4.005085in}{1.584815in}}%
\pgfpathlineto{\pgfqpoint{4.018763in}{1.577784in}}%
\pgfpathlineto{\pgfqpoint{4.032446in}{1.570776in}}%
\pgfpathlineto{\pgfqpoint{4.024508in}{1.576098in}}%
\pgfpathlineto{\pgfqpoint{4.016559in}{1.581907in}}%
\pgfpathlineto{\pgfqpoint{4.008599in}{1.588216in}}%
\pgfpathlineto{\pgfqpoint{4.000627in}{1.595035in}}%
\pgfpathlineto{\pgfqpoint{3.986912in}{1.602429in}}%
\pgfpathlineto{\pgfqpoint{3.973201in}{1.609847in}}%
\pgfpathlineto{\pgfqpoint{3.959495in}{1.617289in}}%
\pgfpathlineto{\pgfqpoint{3.945794in}{1.624755in}}%
\pgfpathlineto{\pgfqpoint{3.953799in}{1.617543in}}%
\pgfpathlineto{\pgfqpoint{3.961792in}{1.610846in}}%
\pgfpathlineto{\pgfqpoint{3.969773in}{1.604651in}}%
\pgfpathlineto{\pgfqpoint{3.977742in}{1.598949in}}%
\pgfpathclose%
\pgfusepath{fill}%
\end{pgfscope}%
\begin{pgfscope}%
\pgfpathrectangle{\pgfqpoint{1.254980in}{0.150000in}}{\pgfqpoint{5.490039in}{5.490039in}}%
\pgfusepath{clip}%
\pgfsetbuttcap%
\pgfsetroundjoin%
\definecolor{currentfill}{rgb}{0.153364,0.497000,0.557724}%
\pgfsetfillcolor{currentfill}%
\pgfsetfillopacity{0.700000}%
\pgfsetlinewidth{0.000000pt}%
\definecolor{currentstroke}{rgb}{0.000000,0.000000,0.000000}%
\pgfsetstrokecolor{currentstroke}%
\pgfsetdash{}{0pt}%
\pgfpathmoveto{\pgfqpoint{2.952866in}{2.326699in}}%
\pgfpathlineto{\pgfqpoint{2.966421in}{2.316565in}}%
\pgfpathlineto{\pgfqpoint{2.979978in}{2.306463in}}%
\pgfpathlineto{\pgfqpoint{2.993538in}{2.296392in}}%
\pgfpathlineto{\pgfqpoint{3.007100in}{2.286353in}}%
\pgfpathlineto{\pgfqpoint{2.998209in}{2.305653in}}%
\pgfpathlineto{\pgfqpoint{2.989288in}{2.325657in}}%
\pgfpathlineto{\pgfqpoint{2.980335in}{2.346381in}}%
\pgfpathlineto{\pgfqpoint{2.971351in}{2.367837in}}%
\pgfpathlineto{\pgfqpoint{2.957732in}{2.378323in}}%
\pgfpathlineto{\pgfqpoint{2.944116in}{2.388841in}}%
\pgfpathlineto{\pgfqpoint{2.930502in}{2.399390in}}%
\pgfpathlineto{\pgfqpoint{2.916890in}{2.409971in}}%
\pgfpathlineto{\pgfqpoint{2.925933in}{2.388061in}}%
\pgfpathlineto{\pgfqpoint{2.934943in}{2.366888in}}%
\pgfpathlineto{\pgfqpoint{2.943920in}{2.346439in}}%
\pgfpathlineto{\pgfqpoint{2.952866in}{2.326699in}}%
\pgfpathclose%
\pgfusepath{fill}%
\end{pgfscope}%
\begin{pgfscope}%
\pgfpathrectangle{\pgfqpoint{1.254980in}{0.150000in}}{\pgfqpoint{5.490039in}{5.490039in}}%
\pgfusepath{clip}%
\pgfsetbuttcap%
\pgfsetroundjoin%
\definecolor{currentfill}{rgb}{0.253935,0.265254,0.529983}%
\pgfsetfillcolor{currentfill}%
\pgfsetfillopacity{0.700000}%
\pgfsetlinewidth{0.000000pt}%
\definecolor{currentstroke}{rgb}{0.000000,0.000000,0.000000}%
\pgfsetstrokecolor{currentstroke}%
\pgfsetdash{}{0pt}%
\pgfpathmoveto{\pgfqpoint{3.727169in}{1.747508in}}%
\pgfpathlineto{\pgfqpoint{3.740801in}{1.739651in}}%
\pgfpathlineto{\pgfqpoint{3.754437in}{1.731819in}}%
\pgfpathlineto{\pgfqpoint{3.768078in}{1.724012in}}%
\pgfpathlineto{\pgfqpoint{3.781723in}{1.716230in}}%
\pgfpathlineto{\pgfqpoint{3.773599in}{1.725158in}}%
\pgfpathlineto{\pgfqpoint{3.765460in}{1.734634in}}%
\pgfpathlineto{\pgfqpoint{3.757306in}{1.744670in}}%
\pgfpathlineto{\pgfqpoint{3.749136in}{1.755278in}}%
\pgfpathlineto{\pgfqpoint{3.735453in}{1.763463in}}%
\pgfpathlineto{\pgfqpoint{3.721775in}{1.771674in}}%
\pgfpathlineto{\pgfqpoint{3.708100in}{1.779909in}}%
\pgfpathlineto{\pgfqpoint{3.694430in}{1.788170in}}%
\pgfpathlineto{\pgfqpoint{3.702639in}{1.777152in}}%
\pgfpathlineto{\pgfqpoint{3.710832in}{1.766710in}}%
\pgfpathlineto{\pgfqpoint{3.719008in}{1.756833in}}%
\pgfpathlineto{\pgfqpoint{3.727169in}{1.747508in}}%
\pgfpathclose%
\pgfusepath{fill}%
\end{pgfscope}%
\begin{pgfscope}%
\pgfpathrectangle{\pgfqpoint{1.254980in}{0.150000in}}{\pgfqpoint{5.490039in}{5.490039in}}%
\pgfusepath{clip}%
\pgfsetbuttcap%
\pgfsetroundjoin%
\definecolor{currentfill}{rgb}{0.196571,0.711827,0.479221}%
\pgfsetfillcolor{currentfill}%
\pgfsetfillopacity{0.700000}%
\pgfsetlinewidth{0.000000pt}%
\definecolor{currentstroke}{rgb}{0.000000,0.000000,0.000000}%
\pgfsetstrokecolor{currentstroke}%
\pgfsetdash{}{0pt}%
\pgfpathmoveto{\pgfqpoint{2.319473in}{2.910670in}}%
\pgfpathlineto{\pgfqpoint{2.333031in}{2.898437in}}%
\pgfpathlineto{\pgfqpoint{2.346588in}{2.886247in}}%
\pgfpathlineto{\pgfqpoint{2.360146in}{2.874102in}}%
\pgfpathlineto{\pgfqpoint{2.373705in}{2.862001in}}%
\pgfpathlineto{\pgfqpoint{2.363983in}{2.889322in}}%
\pgfpathlineto{\pgfqpoint{2.354215in}{2.917461in}}%
\pgfpathlineto{\pgfqpoint{2.344400in}{2.946432in}}%
\pgfpathlineto{\pgfqpoint{2.334537in}{2.976252in}}%
\pgfpathlineto{\pgfqpoint{2.320907in}{2.988835in}}%
\pgfpathlineto{\pgfqpoint{2.307278in}{3.001462in}}%
\pgfpathlineto{\pgfqpoint{2.293648in}{3.014133in}}%
\pgfpathlineto{\pgfqpoint{2.280019in}{3.026850in}}%
\pgfpathlineto{\pgfqpoint{2.289955in}{2.996540in}}%
\pgfpathlineto{\pgfqpoint{2.299843in}{2.967084in}}%
\pgfpathlineto{\pgfqpoint{2.309682in}{2.938466in}}%
\pgfpathlineto{\pgfqpoint{2.319473in}{2.910670in}}%
\pgfpathclose%
\pgfusepath{fill}%
\end{pgfscope}%
\begin{pgfscope}%
\pgfpathrectangle{\pgfqpoint{1.254980in}{0.150000in}}{\pgfqpoint{5.490039in}{5.490039in}}%
\pgfusepath{clip}%
\pgfsetbuttcap%
\pgfsetroundjoin%
\definecolor{currentfill}{rgb}{0.283072,0.130895,0.449241}%
\pgfsetfillcolor{currentfill}%
\pgfsetfillopacity{0.700000}%
\pgfsetlinewidth{0.000000pt}%
\definecolor{currentstroke}{rgb}{0.000000,0.000000,0.000000}%
\pgfsetstrokecolor{currentstroke}%
\pgfsetdash{}{0pt}%
\pgfpathmoveto{\pgfqpoint{4.228342in}{1.476363in}}%
\pgfpathlineto{\pgfqpoint{4.242063in}{1.470032in}}%
\pgfpathlineto{\pgfqpoint{4.255790in}{1.463724in}}%
\pgfpathlineto{\pgfqpoint{4.269521in}{1.457439in}}%
\pgfpathlineto{\pgfqpoint{4.283259in}{1.451176in}}%
\pgfpathlineto{\pgfqpoint{4.275468in}{1.453188in}}%
\pgfpathlineto{\pgfqpoint{4.267669in}{1.455631in}}%
\pgfpathlineto{\pgfqpoint{4.259863in}{1.458515in}}%
\pgfpathlineto{\pgfqpoint{4.252049in}{1.461852in}}%
\pgfpathlineto{\pgfqpoint{4.238286in}{1.468486in}}%
\pgfpathlineto{\pgfqpoint{4.224528in}{1.475142in}}%
\pgfpathlineto{\pgfqpoint{4.210776in}{1.481821in}}%
\pgfpathlineto{\pgfqpoint{4.197028in}{1.488524in}}%
\pgfpathlineto{\pgfqpoint{4.204869in}{1.484810in}}%
\pgfpathlineto{\pgfqpoint{4.212702in}{1.481553in}}%
\pgfpathlineto{\pgfqpoint{4.220526in}{1.478740in}}%
\pgfpathlineto{\pgfqpoint{4.228342in}{1.476363in}}%
\pgfpathclose%
\pgfusepath{fill}%
\end{pgfscope}%
\begin{pgfscope}%
\pgfpathrectangle{\pgfqpoint{1.254980in}{0.150000in}}{\pgfqpoint{5.490039in}{5.490039in}}%
\pgfusepath{clip}%
\pgfsetbuttcap%
\pgfsetroundjoin%
\definecolor{currentfill}{rgb}{0.281924,0.089666,0.412415}%
\pgfsetfillcolor{currentfill}%
\pgfsetfillopacity{0.700000}%
\pgfsetlinewidth{0.000000pt}%
\definecolor{currentstroke}{rgb}{0.000000,0.000000,0.000000}%
\pgfsetstrokecolor{currentstroke}%
\pgfsetdash{}{0pt}%
\pgfpathmoveto{\pgfqpoint{4.424259in}{1.400830in}}%
\pgfpathlineto{\pgfqpoint{4.438023in}{1.395131in}}%
\pgfpathlineto{\pgfqpoint{4.451793in}{1.389454in}}%
\pgfpathlineto{\pgfqpoint{4.465569in}{1.383800in}}%
\pgfpathlineto{\pgfqpoint{4.479351in}{1.378169in}}%
\pgfpathlineto{\pgfqpoint{4.471653in}{1.377500in}}%
\pgfpathlineto{\pgfqpoint{4.463951in}{1.377212in}}%
\pgfpathlineto{\pgfqpoint{4.456243in}{1.377315in}}%
\pgfpathlineto{\pgfqpoint{4.448529in}{1.377819in}}%
\pgfpathlineto{\pgfqpoint{4.434726in}{1.383806in}}%
\pgfpathlineto{\pgfqpoint{4.420929in}{1.389817in}}%
\pgfpathlineto{\pgfqpoint{4.407137in}{1.395850in}}%
\pgfpathlineto{\pgfqpoint{4.393351in}{1.401906in}}%
\pgfpathlineto{\pgfqpoint{4.401087in}{1.401040in}}%
\pgfpathlineto{\pgfqpoint{4.408817in}{1.400579in}}%
\pgfpathlineto{\pgfqpoint{4.416541in}{1.400512in}}%
\pgfpathlineto{\pgfqpoint{4.424259in}{1.400830in}}%
\pgfpathclose%
\pgfusepath{fill}%
\end{pgfscope}%
\begin{pgfscope}%
\pgfpathrectangle{\pgfqpoint{1.254980in}{0.150000in}}{\pgfqpoint{5.490039in}{5.490039in}}%
\pgfusepath{clip}%
\pgfsetbuttcap%
\pgfsetroundjoin%
\definecolor{currentfill}{rgb}{0.214298,0.355619,0.551184}%
\pgfsetfillcolor{currentfill}%
\pgfsetfillopacity{0.700000}%
\pgfsetlinewidth{0.000000pt}%
\definecolor{currentstroke}{rgb}{0.000000,0.000000,0.000000}%
\pgfsetstrokecolor{currentstroke}%
\pgfsetdash{}{0pt}%
\pgfpathmoveto{\pgfqpoint{3.421806in}{1.958759in}}%
\pgfpathlineto{\pgfqpoint{3.435403in}{1.949980in}}%
\pgfpathlineto{\pgfqpoint{3.449004in}{1.941229in}}%
\pgfpathlineto{\pgfqpoint{3.462608in}{1.932504in}}%
\pgfpathlineto{\pgfqpoint{3.476215in}{1.923805in}}%
\pgfpathlineto{\pgfqpoint{3.467820in}{1.937071in}}%
\pgfpathlineto{\pgfqpoint{3.459403in}{1.950953in}}%
\pgfpathlineto{\pgfqpoint{3.450966in}{1.965466in}}%
\pgfpathlineto{\pgfqpoint{3.442506in}{1.980621in}}%
\pgfpathlineto{\pgfqpoint{3.428853in}{1.989742in}}%
\pgfpathlineto{\pgfqpoint{3.415203in}{1.998889in}}%
\pgfpathlineto{\pgfqpoint{3.401557in}{2.008064in}}%
\pgfpathlineto{\pgfqpoint{3.387914in}{2.017265in}}%
\pgfpathlineto{\pgfqpoint{3.396421in}{2.001681in}}%
\pgfpathlineto{\pgfqpoint{3.404904in}{1.986744in}}%
\pgfpathlineto{\pgfqpoint{3.413366in}{1.972441in}}%
\pgfpathlineto{\pgfqpoint{3.421806in}{1.958759in}}%
\pgfpathclose%
\pgfusepath{fill}%
\end{pgfscope}%
\begin{pgfscope}%
\pgfpathrectangle{\pgfqpoint{1.254980in}{0.150000in}}{\pgfqpoint{5.490039in}{5.490039in}}%
\pgfusepath{clip}%
\pgfsetbuttcap%
\pgfsetroundjoin%
\definecolor{currentfill}{rgb}{0.157729,0.485932,0.558013}%
\pgfsetfillcolor{currentfill}%
\pgfsetfillopacity{0.700000}%
\pgfsetlinewidth{0.000000pt}%
\definecolor{currentstroke}{rgb}{0.000000,0.000000,0.000000}%
\pgfsetstrokecolor{currentstroke}%
\pgfsetdash{}{0pt}%
\pgfpathmoveto{\pgfqpoint{3.007100in}{2.286353in}}%
\pgfpathlineto{\pgfqpoint{3.020664in}{2.276344in}}%
\pgfpathlineto{\pgfqpoint{3.034231in}{2.266367in}}%
\pgfpathlineto{\pgfqpoint{3.047800in}{2.256420in}}%
\pgfpathlineto{\pgfqpoint{3.061372in}{2.246503in}}%
\pgfpathlineto{\pgfqpoint{3.052535in}{2.265364in}}%
\pgfpathlineto{\pgfqpoint{3.043669in}{2.284925in}}%
\pgfpathlineto{\pgfqpoint{3.034773in}{2.305200in}}%
\pgfpathlineto{\pgfqpoint{3.025845in}{2.326203in}}%
\pgfpathlineto{\pgfqpoint{3.012218in}{2.336565in}}%
\pgfpathlineto{\pgfqpoint{2.998594in}{2.346958in}}%
\pgfpathlineto{\pgfqpoint{2.984971in}{2.357382in}}%
\pgfpathlineto{\pgfqpoint{2.971351in}{2.367837in}}%
\pgfpathlineto{\pgfqpoint{2.980335in}{2.346381in}}%
\pgfpathlineto{\pgfqpoint{2.989288in}{2.325657in}}%
\pgfpathlineto{\pgfqpoint{2.998209in}{2.305653in}}%
\pgfpathlineto{\pgfqpoint{3.007100in}{2.286353in}}%
\pgfpathclose%
\pgfusepath{fill}%
\end{pgfscope}%
\begin{pgfscope}%
\pgfpathrectangle{\pgfqpoint{1.254980in}{0.150000in}}{\pgfqpoint{5.490039in}{5.490039in}}%
\pgfusepath{clip}%
\pgfsetbuttcap%
\pgfsetroundjoin%
\definecolor{currentfill}{rgb}{0.170948,0.694384,0.493803}%
\pgfsetfillcolor{currentfill}%
\pgfsetfillopacity{0.700000}%
\pgfsetlinewidth{0.000000pt}%
\definecolor{currentstroke}{rgb}{0.000000,0.000000,0.000000}%
\pgfsetstrokecolor{currentstroke}%
\pgfsetdash{}{0pt}%
\pgfpathmoveto{\pgfqpoint{2.373705in}{2.862001in}}%
\pgfpathlineto{\pgfqpoint{2.387264in}{2.849943in}}%
\pgfpathlineto{\pgfqpoint{2.400823in}{2.837928in}}%
\pgfpathlineto{\pgfqpoint{2.414383in}{2.825956in}}%
\pgfpathlineto{\pgfqpoint{2.427944in}{2.814026in}}%
\pgfpathlineto{\pgfqpoint{2.418291in}{2.840875in}}%
\pgfpathlineto{\pgfqpoint{2.408593in}{2.868536in}}%
\pgfpathlineto{\pgfqpoint{2.398850in}{2.897024in}}%
\pgfpathlineto{\pgfqpoint{2.389059in}{2.926355in}}%
\pgfpathlineto{\pgfqpoint{2.375428in}{2.938765in}}%
\pgfpathlineto{\pgfqpoint{2.361797in}{2.951218in}}%
\pgfpathlineto{\pgfqpoint{2.348167in}{2.963713in}}%
\pgfpathlineto{\pgfqpoint{2.334537in}{2.976252in}}%
\pgfpathlineto{\pgfqpoint{2.344400in}{2.946432in}}%
\pgfpathlineto{\pgfqpoint{2.354215in}{2.917461in}}%
\pgfpathlineto{\pgfqpoint{2.363983in}{2.889322in}}%
\pgfpathlineto{\pgfqpoint{2.373705in}{2.862001in}}%
\pgfpathclose%
\pgfusepath{fill}%
\end{pgfscope}%
\begin{pgfscope}%
\pgfpathrectangle{\pgfqpoint{1.254980in}{0.150000in}}{\pgfqpoint{5.490039in}{5.490039in}}%
\pgfusepath{clip}%
\pgfsetbuttcap%
\pgfsetroundjoin%
\definecolor{currentfill}{rgb}{0.276022,0.044167,0.370164}%
\pgfsetfillcolor{currentfill}%
\pgfsetfillopacity{0.700000}%
\pgfsetlinewidth{0.000000pt}%
\definecolor{currentstroke}{rgb}{0.000000,0.000000,0.000000}%
\pgfsetstrokecolor{currentstroke}%
\pgfsetdash{}{0pt}%
\pgfpathmoveto{\pgfqpoint{4.761525in}{1.319876in}}%
\pgfpathlineto{\pgfqpoint{4.775378in}{1.315301in}}%
\pgfpathlineto{\pgfqpoint{4.789238in}{1.310748in}}%
\pgfpathlineto{\pgfqpoint{4.803105in}{1.306217in}}%
\pgfpathlineto{\pgfqpoint{4.816978in}{1.301709in}}%
\pgfpathlineto{\pgfqpoint{4.809395in}{1.296697in}}%
\pgfpathlineto{\pgfqpoint{4.801810in}{1.291978in}}%
\pgfpathlineto{\pgfqpoint{4.794222in}{1.287558in}}%
\pgfpathlineto{\pgfqpoint{4.786632in}{1.283449in}}%
\pgfpathlineto{\pgfqpoint{4.772745in}{1.288287in}}%
\pgfpathlineto{\pgfqpoint{4.758864in}{1.293147in}}%
\pgfpathlineto{\pgfqpoint{4.744989in}{1.298030in}}%
\pgfpathlineto{\pgfqpoint{4.731121in}{1.302935in}}%
\pgfpathlineto{\pgfqpoint{4.738727in}{1.306710in}}%
\pgfpathlineto{\pgfqpoint{4.746329in}{1.310798in}}%
\pgfpathlineto{\pgfqpoint{4.753928in}{1.315189in}}%
\pgfpathlineto{\pgfqpoint{4.761525in}{1.319876in}}%
\pgfpathclose%
\pgfusepath{fill}%
\end{pgfscope}%
\begin{pgfscope}%
\pgfpathrectangle{\pgfqpoint{1.254980in}{0.150000in}}{\pgfqpoint{5.490039in}{5.490039in}}%
\pgfusepath{clip}%
\pgfsetbuttcap%
\pgfsetroundjoin%
\definecolor{currentfill}{rgb}{0.274952,0.037752,0.364543}%
\pgfsetfillcolor{currentfill}%
\pgfsetfillopacity{0.700000}%
\pgfsetlinewidth{0.000000pt}%
\definecolor{currentstroke}{rgb}{0.000000,0.000000,0.000000}%
\pgfsetstrokecolor{currentstroke}%
\pgfsetdash{}{0pt}%
\pgfpathmoveto{\pgfqpoint{4.902795in}{1.307967in}}%
\pgfpathlineto{\pgfqpoint{4.916690in}{1.303888in}}%
\pgfpathlineto{\pgfqpoint{4.930592in}{1.299831in}}%
\pgfpathlineto{\pgfqpoint{4.944501in}{1.295796in}}%
\pgfpathlineto{\pgfqpoint{4.958418in}{1.291784in}}%
\pgfpathlineto{\pgfqpoint{4.950867in}{1.285064in}}%
\pgfpathlineto{\pgfqpoint{4.943316in}{1.278596in}}%
\pgfpathlineto{\pgfqpoint{4.935762in}{1.272388in}}%
\pgfpathlineto{\pgfqpoint{4.928207in}{1.266449in}}%
\pgfpathlineto{\pgfqpoint{4.914279in}{1.270779in}}%
\pgfpathlineto{\pgfqpoint{4.900359in}{1.275130in}}%
\pgfpathlineto{\pgfqpoint{4.886445in}{1.279504in}}%
\pgfpathlineto{\pgfqpoint{4.872538in}{1.283901in}}%
\pgfpathlineto{\pgfqpoint{4.880105in}{1.289518in}}%
\pgfpathlineto{\pgfqpoint{4.887671in}{1.295407in}}%
\pgfpathlineto{\pgfqpoint{4.895234in}{1.301559in}}%
\pgfpathlineto{\pgfqpoint{4.902795in}{1.307967in}}%
\pgfpathclose%
\pgfusepath{fill}%
\end{pgfscope}%
\begin{pgfscope}%
\pgfpathrectangle{\pgfqpoint{1.254980in}{0.150000in}}{\pgfqpoint{5.490039in}{5.490039in}}%
\pgfusepath{clip}%
\pgfsetbuttcap%
\pgfsetroundjoin%
\definecolor{currentfill}{rgb}{0.278012,0.180367,0.486697}%
\pgfsetfillcolor{currentfill}%
\pgfsetfillopacity{0.700000}%
\pgfsetlinewidth{0.000000pt}%
\definecolor{currentstroke}{rgb}{0.000000,0.000000,0.000000}%
\pgfsetstrokecolor{currentstroke}%
\pgfsetdash{}{0pt}%
\pgfpathmoveto{\pgfqpoint{4.032446in}{1.570776in}}%
\pgfpathlineto{\pgfqpoint{4.046135in}{1.563792in}}%
\pgfpathlineto{\pgfqpoint{4.059828in}{1.556832in}}%
\pgfpathlineto{\pgfqpoint{4.073525in}{1.549896in}}%
\pgfpathlineto{\pgfqpoint{4.087228in}{1.542983in}}%
\pgfpathlineto{\pgfqpoint{4.079320in}{1.547924in}}%
\pgfpathlineto{\pgfqpoint{4.071402in}{1.553349in}}%
\pgfpathlineto{\pgfqpoint{4.063473in}{1.559270in}}%
\pgfpathlineto{\pgfqpoint{4.055533in}{1.565697in}}%
\pgfpathlineto{\pgfqpoint{4.041800in}{1.572996in}}%
\pgfpathlineto{\pgfqpoint{4.028071in}{1.580319in}}%
\pgfpathlineto{\pgfqpoint{4.014347in}{1.587665in}}%
\pgfpathlineto{\pgfqpoint{4.000627in}{1.595035in}}%
\pgfpathlineto{\pgfqpoint{4.008599in}{1.588216in}}%
\pgfpathlineto{\pgfqpoint{4.016559in}{1.581907in}}%
\pgfpathlineto{\pgfqpoint{4.024508in}{1.576098in}}%
\pgfpathlineto{\pgfqpoint{4.032446in}{1.570776in}}%
\pgfpathclose%
\pgfusepath{fill}%
\end{pgfscope}%
\begin{pgfscope}%
\pgfpathrectangle{\pgfqpoint{1.254980in}{0.150000in}}{\pgfqpoint{5.490039in}{5.490039in}}%
\pgfusepath{clip}%
\pgfsetbuttcap%
\pgfsetroundjoin%
\definecolor{currentfill}{rgb}{0.278791,0.062145,0.386592}%
\pgfsetfillcolor{currentfill}%
\pgfsetfillopacity{0.700000}%
\pgfsetlinewidth{0.000000pt}%
\definecolor{currentstroke}{rgb}{0.000000,0.000000,0.000000}%
\pgfsetstrokecolor{currentstroke}%
\pgfsetdash{}{0pt}%
\pgfpathmoveto{\pgfqpoint{4.620406in}{1.342984in}}%
\pgfpathlineto{\pgfqpoint{4.634224in}{1.337899in}}%
\pgfpathlineto{\pgfqpoint{4.648047in}{1.332837in}}%
\pgfpathlineto{\pgfqpoint{4.661877in}{1.327797in}}%
\pgfpathlineto{\pgfqpoint{4.675713in}{1.322780in}}%
\pgfpathlineto{\pgfqpoint{4.668089in}{1.319665in}}%
\pgfpathlineto{\pgfqpoint{4.660461in}{1.316883in}}%
\pgfpathlineto{\pgfqpoint{4.652829in}{1.314445in}}%
\pgfpathlineto{\pgfqpoint{4.645194in}{1.312359in}}%
\pgfpathlineto{\pgfqpoint{4.631341in}{1.317720in}}%
\pgfpathlineto{\pgfqpoint{4.617493in}{1.323102in}}%
\pgfpathlineto{\pgfqpoint{4.603652in}{1.328508in}}%
\pgfpathlineto{\pgfqpoint{4.589817in}{1.333935in}}%
\pgfpathlineto{\pgfqpoint{4.597470in}{1.335673in}}%
\pgfpathlineto{\pgfqpoint{4.605120in}{1.337767in}}%
\pgfpathlineto{\pgfqpoint{4.612765in}{1.340206in}}%
\pgfpathlineto{\pgfqpoint{4.620406in}{1.342984in}}%
\pgfpathclose%
\pgfusepath{fill}%
\end{pgfscope}%
\begin{pgfscope}%
\pgfpathrectangle{\pgfqpoint{1.254980in}{0.150000in}}{\pgfqpoint{5.490039in}{5.490039in}}%
\pgfusepath{clip}%
\pgfsetbuttcap%
\pgfsetroundjoin%
\definecolor{currentfill}{rgb}{0.274952,0.037752,0.364543}%
\pgfsetfillcolor{currentfill}%
\pgfsetfillopacity{0.700000}%
\pgfsetlinewidth{0.000000pt}%
\definecolor{currentstroke}{rgb}{0.000000,0.000000,0.000000}%
\pgfsetstrokecolor{currentstroke}%
\pgfsetdash{}{0pt}%
\pgfpathmoveto{\pgfqpoint{5.044296in}{1.306417in}}%
\pgfpathlineto{\pgfqpoint{5.058239in}{1.302821in}}%
\pgfpathlineto{\pgfqpoint{5.072188in}{1.299247in}}%
\pgfpathlineto{\pgfqpoint{5.086145in}{1.295696in}}%
\pgfpathlineto{\pgfqpoint{5.078618in}{1.287523in}}%
\pgfpathlineto{\pgfqpoint{5.071090in}{1.279565in}}%
\pgfpathlineto{\pgfqpoint{5.063560in}{1.271830in}}%
\pgfpathlineto{\pgfqpoint{5.056029in}{1.264324in}}%
\pgfpathlineto{\pgfqpoint{5.042063in}{1.268180in}}%
\pgfpathlineto{\pgfqpoint{5.028104in}{1.272058in}}%
\pgfpathlineto{\pgfqpoint{5.014153in}{1.275959in}}%
\pgfpathlineto{\pgfqpoint{5.021691in}{1.283232in}}%
\pgfpathlineto{\pgfqpoint{5.029228in}{1.290738in}}%
\pgfpathlineto{\pgfqpoint{5.036763in}{1.298469in}}%
\pgfpathlineto{\pgfqpoint{5.044296in}{1.306417in}}%
\pgfpathclose%
\pgfusepath{fill}%
\end{pgfscope}%
\begin{pgfscope}%
\pgfpathrectangle{\pgfqpoint{1.254980in}{0.150000in}}{\pgfqpoint{5.490039in}{5.490039in}}%
\pgfusepath{clip}%
\pgfsetbuttcap%
\pgfsetroundjoin%
\definecolor{currentfill}{rgb}{0.153894,0.680203,0.504172}%
\pgfsetfillcolor{currentfill}%
\pgfsetfillopacity{0.700000}%
\pgfsetlinewidth{0.000000pt}%
\definecolor{currentstroke}{rgb}{0.000000,0.000000,0.000000}%
\pgfsetstrokecolor{currentstroke}%
\pgfsetdash{}{0pt}%
\pgfpathmoveto{\pgfqpoint{2.427944in}{2.814026in}}%
\pgfpathlineto{\pgfqpoint{2.441505in}{2.802138in}}%
\pgfpathlineto{\pgfqpoint{2.455067in}{2.790291in}}%
\pgfpathlineto{\pgfqpoint{2.468630in}{2.778486in}}%
\pgfpathlineto{\pgfqpoint{2.482194in}{2.766721in}}%
\pgfpathlineto{\pgfqpoint{2.472609in}{2.793099in}}%
\pgfpathlineto{\pgfqpoint{2.462980in}{2.820284in}}%
\pgfpathlineto{\pgfqpoint{2.453306in}{2.848291in}}%
\pgfpathlineto{\pgfqpoint{2.443587in}{2.877136in}}%
\pgfpathlineto{\pgfqpoint{2.429955in}{2.889379in}}%
\pgfpathlineto{\pgfqpoint{2.416322in}{2.901663in}}%
\pgfpathlineto{\pgfqpoint{2.402690in}{2.913988in}}%
\pgfpathlineto{\pgfqpoint{2.389059in}{2.926355in}}%
\pgfpathlineto{\pgfqpoint{2.398850in}{2.897024in}}%
\pgfpathlineto{\pgfqpoint{2.408593in}{2.868536in}}%
\pgfpathlineto{\pgfqpoint{2.418291in}{2.840875in}}%
\pgfpathlineto{\pgfqpoint{2.427944in}{2.814026in}}%
\pgfpathclose%
\pgfusepath{fill}%
\end{pgfscope}%
\begin{pgfscope}%
\pgfpathrectangle{\pgfqpoint{1.254980in}{0.150000in}}{\pgfqpoint{5.490039in}{5.490039in}}%
\pgfusepath{clip}%
\pgfsetbuttcap%
\pgfsetroundjoin%
\definecolor{currentfill}{rgb}{0.257322,0.256130,0.526563}%
\pgfsetfillcolor{currentfill}%
\pgfsetfillopacity{0.700000}%
\pgfsetlinewidth{0.000000pt}%
\definecolor{currentstroke}{rgb}{0.000000,0.000000,0.000000}%
\pgfsetstrokecolor{currentstroke}%
\pgfsetdash{}{0pt}%
\pgfpathmoveto{\pgfqpoint{3.781723in}{1.716230in}}%
\pgfpathlineto{\pgfqpoint{3.795371in}{1.708473in}}%
\pgfpathlineto{\pgfqpoint{3.809024in}{1.700740in}}%
\pgfpathlineto{\pgfqpoint{3.822682in}{1.693032in}}%
\pgfpathlineto{\pgfqpoint{3.836343in}{1.685348in}}%
\pgfpathlineto{\pgfqpoint{3.828256in}{1.693879in}}%
\pgfpathlineto{\pgfqpoint{3.820154in}{1.702953in}}%
\pgfpathlineto{\pgfqpoint{3.812038in}{1.712584in}}%
\pgfpathlineto{\pgfqpoint{3.803906in}{1.722783in}}%
\pgfpathlineto{\pgfqpoint{3.790208in}{1.730870in}}%
\pgfpathlineto{\pgfqpoint{3.776513in}{1.738981in}}%
\pgfpathlineto{\pgfqpoint{3.762823in}{1.747117in}}%
\pgfpathlineto{\pgfqpoint{3.749136in}{1.755278in}}%
\pgfpathlineto{\pgfqpoint{3.757306in}{1.744670in}}%
\pgfpathlineto{\pgfqpoint{3.765460in}{1.734634in}}%
\pgfpathlineto{\pgfqpoint{3.773599in}{1.725158in}}%
\pgfpathlineto{\pgfqpoint{3.781723in}{1.716230in}}%
\pgfpathclose%
\pgfusepath{fill}%
\end{pgfscope}%
\begin{pgfscope}%
\pgfpathrectangle{\pgfqpoint{1.254980in}{0.150000in}}{\pgfqpoint{5.490039in}{5.490039in}}%
\pgfusepath{clip}%
\pgfsetbuttcap%
\pgfsetroundjoin%
\definecolor{currentfill}{rgb}{0.163625,0.471133,0.558148}%
\pgfsetfillcolor{currentfill}%
\pgfsetfillopacity{0.700000}%
\pgfsetlinewidth{0.000000pt}%
\definecolor{currentstroke}{rgb}{0.000000,0.000000,0.000000}%
\pgfsetstrokecolor{currentstroke}%
\pgfsetdash{}{0pt}%
\pgfpathmoveto{\pgfqpoint{3.061372in}{2.246503in}}%
\pgfpathlineto{\pgfqpoint{3.074946in}{2.236617in}}%
\pgfpathlineto{\pgfqpoint{3.088523in}{2.226761in}}%
\pgfpathlineto{\pgfqpoint{3.102102in}{2.216936in}}%
\pgfpathlineto{\pgfqpoint{3.115684in}{2.207140in}}%
\pgfpathlineto{\pgfqpoint{3.106901in}{2.225562in}}%
\pgfpathlineto{\pgfqpoint{3.098089in}{2.244680in}}%
\pgfpathlineto{\pgfqpoint{3.089248in}{2.264508in}}%
\pgfpathlineto{\pgfqpoint{3.080377in}{2.285058in}}%
\pgfpathlineto{\pgfqpoint{3.066741in}{2.295299in}}%
\pgfpathlineto{\pgfqpoint{3.053107in}{2.305570in}}%
\pgfpathlineto{\pgfqpoint{3.039475in}{2.315871in}}%
\pgfpathlineto{\pgfqpoint{3.025845in}{2.326203in}}%
\pgfpathlineto{\pgfqpoint{3.034773in}{2.305200in}}%
\pgfpathlineto{\pgfqpoint{3.043669in}{2.284925in}}%
\pgfpathlineto{\pgfqpoint{3.052535in}{2.265364in}}%
\pgfpathlineto{\pgfqpoint{3.061372in}{2.246503in}}%
\pgfpathclose%
\pgfusepath{fill}%
\end{pgfscope}%
\begin{pgfscope}%
\pgfpathrectangle{\pgfqpoint{1.254980in}{0.150000in}}{\pgfqpoint{5.490039in}{5.490039in}}%
\pgfusepath{clip}%
\pgfsetbuttcap%
\pgfsetroundjoin%
\definecolor{currentfill}{rgb}{0.218130,0.347432,0.550038}%
\pgfsetfillcolor{currentfill}%
\pgfsetfillopacity{0.700000}%
\pgfsetlinewidth{0.000000pt}%
\definecolor{currentstroke}{rgb}{0.000000,0.000000,0.000000}%
\pgfsetstrokecolor{currentstroke}%
\pgfsetdash{}{0pt}%
\pgfpathmoveto{\pgfqpoint{3.476215in}{1.923805in}}%
\pgfpathlineto{\pgfqpoint{3.489826in}{1.915133in}}%
\pgfpathlineto{\pgfqpoint{3.503441in}{1.906488in}}%
\pgfpathlineto{\pgfqpoint{3.517059in}{1.897868in}}%
\pgfpathlineto{\pgfqpoint{3.530681in}{1.889275in}}%
\pgfpathlineto{\pgfqpoint{3.522329in}{1.902125in}}%
\pgfpathlineto{\pgfqpoint{3.513957in}{1.915588in}}%
\pgfpathlineto{\pgfqpoint{3.505564in}{1.929676in}}%
\pgfpathlineto{\pgfqpoint{3.497151in}{1.944402in}}%
\pgfpathlineto{\pgfqpoint{3.483485in}{1.953417in}}%
\pgfpathlineto{\pgfqpoint{3.469822in}{1.962458in}}%
\pgfpathlineto{\pgfqpoint{3.456162in}{1.971526in}}%
\pgfpathlineto{\pgfqpoint{3.442506in}{1.980621in}}%
\pgfpathlineto{\pgfqpoint{3.450966in}{1.965466in}}%
\pgfpathlineto{\pgfqpoint{3.459403in}{1.950953in}}%
\pgfpathlineto{\pgfqpoint{3.467820in}{1.937071in}}%
\pgfpathlineto{\pgfqpoint{3.476215in}{1.923805in}}%
\pgfpathclose%
\pgfusepath{fill}%
\end{pgfscope}%
\begin{pgfscope}%
\pgfpathrectangle{\pgfqpoint{1.254980in}{0.150000in}}{\pgfqpoint{5.490039in}{5.490039in}}%
\pgfusepath{clip}%
\pgfsetbuttcap%
\pgfsetroundjoin%
\definecolor{currentfill}{rgb}{0.283187,0.125848,0.444960}%
\pgfsetfillcolor{currentfill}%
\pgfsetfillopacity{0.700000}%
\pgfsetlinewidth{0.000000pt}%
\definecolor{currentstroke}{rgb}{0.000000,0.000000,0.000000}%
\pgfsetstrokecolor{currentstroke}%
\pgfsetdash{}{0pt}%
\pgfpathmoveto{\pgfqpoint{4.283259in}{1.451176in}}%
\pgfpathlineto{\pgfqpoint{4.297001in}{1.444937in}}%
\pgfpathlineto{\pgfqpoint{4.310749in}{1.438721in}}%
\pgfpathlineto{\pgfqpoint{4.324502in}{1.432528in}}%
\pgfpathlineto{\pgfqpoint{4.338261in}{1.426358in}}%
\pgfpathlineto{\pgfqpoint{4.330495in}{1.428004in}}%
\pgfpathlineto{\pgfqpoint{4.322721in}{1.430077in}}%
\pgfpathlineto{\pgfqpoint{4.314941in}{1.432589in}}%
\pgfpathlineto{\pgfqpoint{4.307154in}{1.435550in}}%
\pgfpathlineto{\pgfqpoint{4.293370in}{1.442091in}}%
\pgfpathlineto{\pgfqpoint{4.279591in}{1.448655in}}%
\pgfpathlineto{\pgfqpoint{4.265818in}{1.455242in}}%
\pgfpathlineto{\pgfqpoint{4.252049in}{1.461852in}}%
\pgfpathlineto{\pgfqpoint{4.259863in}{1.458515in}}%
\pgfpathlineto{\pgfqpoint{4.267669in}{1.455631in}}%
\pgfpathlineto{\pgfqpoint{4.275468in}{1.453188in}}%
\pgfpathlineto{\pgfqpoint{4.283259in}{1.451176in}}%
\pgfpathclose%
\pgfusepath{fill}%
\end{pgfscope}%
\begin{pgfscope}%
\pgfpathrectangle{\pgfqpoint{1.254980in}{0.150000in}}{\pgfqpoint{5.490039in}{5.490039in}}%
\pgfusepath{clip}%
\pgfsetbuttcap%
\pgfsetroundjoin%
\definecolor{currentfill}{rgb}{0.140210,0.665859,0.513427}%
\pgfsetfillcolor{currentfill}%
\pgfsetfillopacity{0.700000}%
\pgfsetlinewidth{0.000000pt}%
\definecolor{currentstroke}{rgb}{0.000000,0.000000,0.000000}%
\pgfsetstrokecolor{currentstroke}%
\pgfsetdash{}{0pt}%
\pgfpathmoveto{\pgfqpoint{2.482194in}{2.766721in}}%
\pgfpathlineto{\pgfqpoint{2.495758in}{2.754997in}}%
\pgfpathlineto{\pgfqpoint{2.509323in}{2.743313in}}%
\pgfpathlineto{\pgfqpoint{2.522889in}{2.731669in}}%
\pgfpathlineto{\pgfqpoint{2.536457in}{2.720064in}}%
\pgfpathlineto{\pgfqpoint{2.526939in}{2.745973in}}%
\pgfpathlineto{\pgfqpoint{2.517378in}{2.772683in}}%
\pgfpathlineto{\pgfqpoint{2.507774in}{2.800210in}}%
\pgfpathlineto{\pgfqpoint{2.498125in}{2.828570in}}%
\pgfpathlineto{\pgfqpoint{2.484490in}{2.840651in}}%
\pgfpathlineto{\pgfqpoint{2.470855in}{2.852772in}}%
\pgfpathlineto{\pgfqpoint{2.457221in}{2.864934in}}%
\pgfpathlineto{\pgfqpoint{2.443587in}{2.877136in}}%
\pgfpathlineto{\pgfqpoint{2.453306in}{2.848291in}}%
\pgfpathlineto{\pgfqpoint{2.462980in}{2.820284in}}%
\pgfpathlineto{\pgfqpoint{2.472609in}{2.793099in}}%
\pgfpathlineto{\pgfqpoint{2.482194in}{2.766721in}}%
\pgfpathclose%
\pgfusepath{fill}%
\end{pgfscope}%
\begin{pgfscope}%
\pgfpathrectangle{\pgfqpoint{1.254980in}{0.150000in}}{\pgfqpoint{5.490039in}{5.490039in}}%
\pgfusepath{clip}%
\pgfsetbuttcap%
\pgfsetroundjoin%
\definecolor{currentfill}{rgb}{0.281446,0.084320,0.407414}%
\pgfsetfillcolor{currentfill}%
\pgfsetfillopacity{0.700000}%
\pgfsetlinewidth{0.000000pt}%
\definecolor{currentstroke}{rgb}{0.000000,0.000000,0.000000}%
\pgfsetstrokecolor{currentstroke}%
\pgfsetdash{}{0pt}%
\pgfpathmoveto{\pgfqpoint{4.479351in}{1.378169in}}%
\pgfpathlineto{\pgfqpoint{4.493138in}{1.372561in}}%
\pgfpathlineto{\pgfqpoint{4.506932in}{1.366975in}}%
\pgfpathlineto{\pgfqpoint{4.520731in}{1.361412in}}%
\pgfpathlineto{\pgfqpoint{4.534536in}{1.355872in}}%
\pgfpathlineto{\pgfqpoint{4.526859in}{1.354851in}}%
\pgfpathlineto{\pgfqpoint{4.519177in}{1.354208in}}%
\pgfpathlineto{\pgfqpoint{4.511491in}{1.353952in}}%
\pgfpathlineto{\pgfqpoint{4.503799in}{1.354094in}}%
\pgfpathlineto{\pgfqpoint{4.489973in}{1.359991in}}%
\pgfpathlineto{\pgfqpoint{4.476153in}{1.365911in}}%
\pgfpathlineto{\pgfqpoint{4.462338in}{1.371853in}}%
\pgfpathlineto{\pgfqpoint{4.448529in}{1.377819in}}%
\pgfpathlineto{\pgfqpoint{4.456243in}{1.377315in}}%
\pgfpathlineto{\pgfqpoint{4.463951in}{1.377212in}}%
\pgfpathlineto{\pgfqpoint{4.471653in}{1.377500in}}%
\pgfpathlineto{\pgfqpoint{4.479351in}{1.378169in}}%
\pgfpathclose%
\pgfusepath{fill}%
\end{pgfscope}%
\begin{pgfscope}%
\pgfpathrectangle{\pgfqpoint{1.254980in}{0.150000in}}{\pgfqpoint{5.490039in}{5.490039in}}%
\pgfusepath{clip}%
\pgfsetbuttcap%
\pgfsetroundjoin%
\definecolor{currentfill}{rgb}{0.278826,0.175490,0.483397}%
\pgfsetfillcolor{currentfill}%
\pgfsetfillopacity{0.700000}%
\pgfsetlinewidth{0.000000pt}%
\definecolor{currentstroke}{rgb}{0.000000,0.000000,0.000000}%
\pgfsetstrokecolor{currentstroke}%
\pgfsetdash{}{0pt}%
\pgfpathmoveto{\pgfqpoint{4.087228in}{1.542983in}}%
\pgfpathlineto{\pgfqpoint{4.100936in}{1.536094in}}%
\pgfpathlineto{\pgfqpoint{4.114648in}{1.529228in}}%
\pgfpathlineto{\pgfqpoint{4.128366in}{1.522385in}}%
\pgfpathlineto{\pgfqpoint{4.142088in}{1.515566in}}%
\pgfpathlineto{\pgfqpoint{4.134209in}{1.520127in}}%
\pgfpathlineto{\pgfqpoint{4.126321in}{1.525168in}}%
\pgfpathlineto{\pgfqpoint{4.118423in}{1.530700in}}%
\pgfpathlineto{\pgfqpoint{4.110515in}{1.536735in}}%
\pgfpathlineto{\pgfqpoint{4.096763in}{1.543940in}}%
\pgfpathlineto{\pgfqpoint{4.083015in}{1.551169in}}%
\pgfpathlineto{\pgfqpoint{4.069272in}{1.558421in}}%
\pgfpathlineto{\pgfqpoint{4.055533in}{1.565697in}}%
\pgfpathlineto{\pgfqpoint{4.063473in}{1.559270in}}%
\pgfpathlineto{\pgfqpoint{4.071402in}{1.553349in}}%
\pgfpathlineto{\pgfqpoint{4.079320in}{1.547924in}}%
\pgfpathlineto{\pgfqpoint{4.087228in}{1.542983in}}%
\pgfpathclose%
\pgfusepath{fill}%
\end{pgfscope}%
\begin{pgfscope}%
\pgfpathrectangle{\pgfqpoint{1.254980in}{0.150000in}}{\pgfqpoint{5.490039in}{5.490039in}}%
\pgfusepath{clip}%
\pgfsetbuttcap%
\pgfsetroundjoin%
\definecolor{currentfill}{rgb}{0.168126,0.459988,0.558082}%
\pgfsetfillcolor{currentfill}%
\pgfsetfillopacity{0.700000}%
\pgfsetlinewidth{0.000000pt}%
\definecolor{currentstroke}{rgb}{0.000000,0.000000,0.000000}%
\pgfsetstrokecolor{currentstroke}%
\pgfsetdash{}{0pt}%
\pgfpathmoveto{\pgfqpoint{3.115684in}{2.207140in}}%
\pgfpathlineto{\pgfqpoint{3.129269in}{2.197373in}}%
\pgfpathlineto{\pgfqpoint{3.142856in}{2.187636in}}%
\pgfpathlineto{\pgfqpoint{3.156446in}{2.177929in}}%
\pgfpathlineto{\pgfqpoint{3.170039in}{2.168251in}}%
\pgfpathlineto{\pgfqpoint{3.161308in}{2.186236in}}%
\pgfpathlineto{\pgfqpoint{3.152549in}{2.204913in}}%
\pgfpathlineto{\pgfqpoint{3.143763in}{2.224293in}}%
\pgfpathlineto{\pgfqpoint{3.134947in}{2.244393in}}%
\pgfpathlineto{\pgfqpoint{3.121301in}{2.254515in}}%
\pgfpathlineto{\pgfqpoint{3.107657in}{2.264666in}}%
\pgfpathlineto{\pgfqpoint{3.094016in}{2.274848in}}%
\pgfpathlineto{\pgfqpoint{3.080377in}{2.285058in}}%
\pgfpathlineto{\pgfqpoint{3.089248in}{2.264508in}}%
\pgfpathlineto{\pgfqpoint{3.098089in}{2.244680in}}%
\pgfpathlineto{\pgfqpoint{3.106901in}{2.225562in}}%
\pgfpathlineto{\pgfqpoint{3.115684in}{2.207140in}}%
\pgfpathclose%
\pgfusepath{fill}%
\end{pgfscope}%
\begin{pgfscope}%
\pgfpathrectangle{\pgfqpoint{1.254980in}{0.150000in}}{\pgfqpoint{5.490039in}{5.490039in}}%
\pgfusepath{clip}%
\pgfsetbuttcap%
\pgfsetroundjoin%
\definecolor{currentfill}{rgb}{0.130067,0.651384,0.521608}%
\pgfsetfillcolor{currentfill}%
\pgfsetfillopacity{0.700000}%
\pgfsetlinewidth{0.000000pt}%
\definecolor{currentstroke}{rgb}{0.000000,0.000000,0.000000}%
\pgfsetstrokecolor{currentstroke}%
\pgfsetdash{}{0pt}%
\pgfpathmoveto{\pgfqpoint{2.536457in}{2.720064in}}%
\pgfpathlineto{\pgfqpoint{2.550025in}{2.708498in}}%
\pgfpathlineto{\pgfqpoint{2.563594in}{2.696971in}}%
\pgfpathlineto{\pgfqpoint{2.577164in}{2.685483in}}%
\pgfpathlineto{\pgfqpoint{2.590736in}{2.674033in}}%
\pgfpathlineto{\pgfqpoint{2.581284in}{2.699474in}}%
\pgfpathlineto{\pgfqpoint{2.571790in}{2.725711in}}%
\pgfpathlineto{\pgfqpoint{2.562254in}{2.752760in}}%
\pgfpathlineto{\pgfqpoint{2.552675in}{2.780636in}}%
\pgfpathlineto{\pgfqpoint{2.539036in}{2.792561in}}%
\pgfpathlineto{\pgfqpoint{2.525398in}{2.804525in}}%
\pgfpathlineto{\pgfqpoint{2.511761in}{2.816528in}}%
\pgfpathlineto{\pgfqpoint{2.498125in}{2.828570in}}%
\pgfpathlineto{\pgfqpoint{2.507774in}{2.800210in}}%
\pgfpathlineto{\pgfqpoint{2.517378in}{2.772683in}}%
\pgfpathlineto{\pgfqpoint{2.526939in}{2.745973in}}%
\pgfpathlineto{\pgfqpoint{2.536457in}{2.720064in}}%
\pgfpathclose%
\pgfusepath{fill}%
\end{pgfscope}%
\begin{pgfscope}%
\pgfpathrectangle{\pgfqpoint{1.254980in}{0.150000in}}{\pgfqpoint{5.490039in}{5.490039in}}%
\pgfusepath{clip}%
\pgfsetbuttcap%
\pgfsetroundjoin%
\definecolor{currentfill}{rgb}{0.276022,0.044167,0.370164}%
\pgfsetfillcolor{currentfill}%
\pgfsetfillopacity{0.700000}%
\pgfsetlinewidth{0.000000pt}%
\definecolor{currentstroke}{rgb}{0.000000,0.000000,0.000000}%
\pgfsetstrokecolor{currentstroke}%
\pgfsetdash{}{0pt}%
\pgfpathmoveto{\pgfqpoint{4.816978in}{1.301709in}}%
\pgfpathlineto{\pgfqpoint{4.830858in}{1.297224in}}%
\pgfpathlineto{\pgfqpoint{4.844744in}{1.292760in}}%
\pgfpathlineto{\pgfqpoint{4.858638in}{1.288319in}}%
\pgfpathlineto{\pgfqpoint{4.872538in}{1.283901in}}%
\pgfpathlineto{\pgfqpoint{4.864968in}{1.278564in}}%
\pgfpathlineto{\pgfqpoint{4.857397in}{1.273515in}}%
\pgfpathlineto{\pgfqpoint{4.849823in}{1.268764in}}%
\pgfpathlineto{\pgfqpoint{4.842247in}{1.264320in}}%
\pgfpathlineto{\pgfqpoint{4.828333in}{1.269068in}}%
\pgfpathlineto{\pgfqpoint{4.814426in}{1.273840in}}%
\pgfpathlineto{\pgfqpoint{4.800526in}{1.278633in}}%
\pgfpathlineto{\pgfqpoint{4.786632in}{1.283449in}}%
\pgfpathlineto{\pgfqpoint{4.794222in}{1.287558in}}%
\pgfpathlineto{\pgfqpoint{4.801810in}{1.291978in}}%
\pgfpathlineto{\pgfqpoint{4.809395in}{1.296697in}}%
\pgfpathlineto{\pgfqpoint{4.816978in}{1.301709in}}%
\pgfpathclose%
\pgfusepath{fill}%
\end{pgfscope}%
\begin{pgfscope}%
\pgfpathrectangle{\pgfqpoint{1.254980in}{0.150000in}}{\pgfqpoint{5.490039in}{5.490039in}}%
\pgfusepath{clip}%
\pgfsetbuttcap%
\pgfsetroundjoin%
\definecolor{currentfill}{rgb}{0.260571,0.246922,0.522828}%
\pgfsetfillcolor{currentfill}%
\pgfsetfillopacity{0.700000}%
\pgfsetlinewidth{0.000000pt}%
\definecolor{currentstroke}{rgb}{0.000000,0.000000,0.000000}%
\pgfsetstrokecolor{currentstroke}%
\pgfsetdash{}{0pt}%
\pgfpathmoveto{\pgfqpoint{3.836343in}{1.685348in}}%
\pgfpathlineto{\pgfqpoint{3.850009in}{1.677689in}}%
\pgfpathlineto{\pgfqpoint{3.863680in}{1.670054in}}%
\pgfpathlineto{\pgfqpoint{3.877354in}{1.662444in}}%
\pgfpathlineto{\pgfqpoint{3.891033in}{1.654858in}}%
\pgfpathlineto{\pgfqpoint{3.882981in}{1.662991in}}%
\pgfpathlineto{\pgfqpoint{3.874916in}{1.671665in}}%
\pgfpathlineto{\pgfqpoint{3.866836in}{1.680891in}}%
\pgfpathlineto{\pgfqpoint{3.858743in}{1.690681in}}%
\pgfpathlineto{\pgfqpoint{3.845027in}{1.698670in}}%
\pgfpathlineto{\pgfqpoint{3.831316in}{1.706683in}}%
\pgfpathlineto{\pgfqpoint{3.817609in}{1.714721in}}%
\pgfpathlineto{\pgfqpoint{3.803906in}{1.722783in}}%
\pgfpathlineto{\pgfqpoint{3.812038in}{1.712584in}}%
\pgfpathlineto{\pgfqpoint{3.820154in}{1.702953in}}%
\pgfpathlineto{\pgfqpoint{3.828256in}{1.693879in}}%
\pgfpathlineto{\pgfqpoint{3.836343in}{1.685348in}}%
\pgfpathclose%
\pgfusepath{fill}%
\end{pgfscope}%
\begin{pgfscope}%
\pgfpathrectangle{\pgfqpoint{1.254980in}{0.150000in}}{\pgfqpoint{5.490039in}{5.490039in}}%
\pgfusepath{clip}%
\pgfsetbuttcap%
\pgfsetroundjoin%
\definecolor{currentfill}{rgb}{0.274952,0.037752,0.364543}%
\pgfsetfillcolor{currentfill}%
\pgfsetfillopacity{0.700000}%
\pgfsetlinewidth{0.000000pt}%
\definecolor{currentstroke}{rgb}{0.000000,0.000000,0.000000}%
\pgfsetstrokecolor{currentstroke}%
\pgfsetdash{}{0pt}%
\pgfpathmoveto{\pgfqpoint{4.958418in}{1.291784in}}%
\pgfpathlineto{\pgfqpoint{4.972341in}{1.287794in}}%
\pgfpathlineto{\pgfqpoint{4.986271in}{1.283827in}}%
\pgfpathlineto{\pgfqpoint{5.000208in}{1.279882in}}%
\pgfpathlineto{\pgfqpoint{5.014153in}{1.275959in}}%
\pgfpathlineto{\pgfqpoint{5.006613in}{1.268926in}}%
\pgfpathlineto{\pgfqpoint{4.999072in}{1.262143in}}%
\pgfpathlineto{\pgfqpoint{4.991529in}{1.255616in}}%
\pgfpathlineto{\pgfqpoint{4.983985in}{1.249355in}}%
\pgfpathlineto{\pgfqpoint{4.970030in}{1.253595in}}%
\pgfpathlineto{\pgfqpoint{4.956082in}{1.257857in}}%
\pgfpathlineto{\pgfqpoint{4.942141in}{1.262142in}}%
\pgfpathlineto{\pgfqpoint{4.928207in}{1.266449in}}%
\pgfpathlineto{\pgfqpoint{4.935762in}{1.272388in}}%
\pgfpathlineto{\pgfqpoint{4.943316in}{1.278596in}}%
\pgfpathlineto{\pgfqpoint{4.950867in}{1.285064in}}%
\pgfpathlineto{\pgfqpoint{4.958418in}{1.291784in}}%
\pgfpathclose%
\pgfusepath{fill}%
\end{pgfscope}%
\begin{pgfscope}%
\pgfpathrectangle{\pgfqpoint{1.254980in}{0.150000in}}{\pgfqpoint{5.490039in}{5.490039in}}%
\pgfusepath{clip}%
\pgfsetbuttcap%
\pgfsetroundjoin%
\definecolor{currentfill}{rgb}{0.223925,0.334994,0.548053}%
\pgfsetfillcolor{currentfill}%
\pgfsetfillopacity{0.700000}%
\pgfsetlinewidth{0.000000pt}%
\definecolor{currentstroke}{rgb}{0.000000,0.000000,0.000000}%
\pgfsetstrokecolor{currentstroke}%
\pgfsetdash{}{0pt}%
\pgfpathmoveto{\pgfqpoint{3.530681in}{1.889275in}}%
\pgfpathlineto{\pgfqpoint{3.544306in}{1.880708in}}%
\pgfpathlineto{\pgfqpoint{3.557935in}{1.872167in}}%
\pgfpathlineto{\pgfqpoint{3.571567in}{1.863652in}}%
\pgfpathlineto{\pgfqpoint{3.585204in}{1.855163in}}%
\pgfpathlineto{\pgfqpoint{3.576894in}{1.867598in}}%
\pgfpathlineto{\pgfqpoint{3.568566in}{1.880641in}}%
\pgfpathlineto{\pgfqpoint{3.560218in}{1.894306in}}%
\pgfpathlineto{\pgfqpoint{3.551850in}{1.908604in}}%
\pgfpathlineto{\pgfqpoint{3.538170in}{1.917514in}}%
\pgfpathlineto{\pgfqpoint{3.524494in}{1.926451in}}%
\pgfpathlineto{\pgfqpoint{3.510820in}{1.935413in}}%
\pgfpathlineto{\pgfqpoint{3.497151in}{1.944402in}}%
\pgfpathlineto{\pgfqpoint{3.505564in}{1.929676in}}%
\pgfpathlineto{\pgfqpoint{3.513957in}{1.915588in}}%
\pgfpathlineto{\pgfqpoint{3.522329in}{1.902125in}}%
\pgfpathlineto{\pgfqpoint{3.530681in}{1.889275in}}%
\pgfpathclose%
\pgfusepath{fill}%
\end{pgfscope}%
\begin{pgfscope}%
\pgfpathrectangle{\pgfqpoint{1.254980in}{0.150000in}}{\pgfqpoint{5.490039in}{5.490039in}}%
\pgfusepath{clip}%
\pgfsetbuttcap%
\pgfsetroundjoin%
\definecolor{currentfill}{rgb}{0.277941,0.056324,0.381191}%
\pgfsetfillcolor{currentfill}%
\pgfsetfillopacity{0.700000}%
\pgfsetlinewidth{0.000000pt}%
\definecolor{currentstroke}{rgb}{0.000000,0.000000,0.000000}%
\pgfsetstrokecolor{currentstroke}%
\pgfsetdash{}{0pt}%
\pgfpathmoveto{\pgfqpoint{4.675713in}{1.322780in}}%
\pgfpathlineto{\pgfqpoint{4.689556in}{1.317785in}}%
\pgfpathlineto{\pgfqpoint{4.703404in}{1.312813in}}%
\pgfpathlineto{\pgfqpoint{4.717260in}{1.307863in}}%
\pgfpathlineto{\pgfqpoint{4.731121in}{1.302935in}}%
\pgfpathlineto{\pgfqpoint{4.723513in}{1.299482in}}%
\pgfpathlineto{\pgfqpoint{4.715902in}{1.296359in}}%
\pgfpathlineto{\pgfqpoint{4.708287in}{1.293576in}}%
\pgfpathlineto{\pgfqpoint{4.700670in}{1.291142in}}%
\pgfpathlineto{\pgfqpoint{4.686791in}{1.296412in}}%
\pgfpathlineto{\pgfqpoint{4.672919in}{1.301706in}}%
\pgfpathlineto{\pgfqpoint{4.659054in}{1.307021in}}%
\pgfpathlineto{\pgfqpoint{4.645194in}{1.312359in}}%
\pgfpathlineto{\pgfqpoint{4.652829in}{1.314445in}}%
\pgfpathlineto{\pgfqpoint{4.660461in}{1.316883in}}%
\pgfpathlineto{\pgfqpoint{4.668089in}{1.319665in}}%
\pgfpathlineto{\pgfqpoint{4.675713in}{1.322780in}}%
\pgfpathclose%
\pgfusepath{fill}%
\end{pgfscope}%
\begin{pgfscope}%
\pgfpathrectangle{\pgfqpoint{1.254980in}{0.150000in}}{\pgfqpoint{5.490039in}{5.490039in}}%
\pgfusepath{clip}%
\pgfsetbuttcap%
\pgfsetroundjoin%
\definecolor{currentfill}{rgb}{0.123444,0.636809,0.528763}%
\pgfsetfillcolor{currentfill}%
\pgfsetfillopacity{0.700000}%
\pgfsetlinewidth{0.000000pt}%
\definecolor{currentstroke}{rgb}{0.000000,0.000000,0.000000}%
\pgfsetstrokecolor{currentstroke}%
\pgfsetdash{}{0pt}%
\pgfpathmoveto{\pgfqpoint{2.590736in}{2.674033in}}%
\pgfpathlineto{\pgfqpoint{2.604308in}{2.662620in}}%
\pgfpathlineto{\pgfqpoint{2.617882in}{2.651245in}}%
\pgfpathlineto{\pgfqpoint{2.631457in}{2.639908in}}%
\pgfpathlineto{\pgfqpoint{2.645034in}{2.628608in}}%
\pgfpathlineto{\pgfqpoint{2.635647in}{2.653582in}}%
\pgfpathlineto{\pgfqpoint{2.626219in}{2.679348in}}%
\pgfpathlineto{\pgfqpoint{2.616751in}{2.705920in}}%
\pgfpathlineto{\pgfqpoint{2.607240in}{2.733314in}}%
\pgfpathlineto{\pgfqpoint{2.593597in}{2.745088in}}%
\pgfpathlineto{\pgfqpoint{2.579955in}{2.756900in}}%
\pgfpathlineto{\pgfqpoint{2.566315in}{2.768749in}}%
\pgfpathlineto{\pgfqpoint{2.552675in}{2.780636in}}%
\pgfpathlineto{\pgfqpoint{2.562254in}{2.752760in}}%
\pgfpathlineto{\pgfqpoint{2.571790in}{2.725711in}}%
\pgfpathlineto{\pgfqpoint{2.581284in}{2.699474in}}%
\pgfpathlineto{\pgfqpoint{2.590736in}{2.674033in}}%
\pgfpathclose%
\pgfusepath{fill}%
\end{pgfscope}%
\begin{pgfscope}%
\pgfpathrectangle{\pgfqpoint{1.254980in}{0.150000in}}{\pgfqpoint{5.490039in}{5.490039in}}%
\pgfusepath{clip}%
\pgfsetbuttcap%
\pgfsetroundjoin%
\definecolor{currentfill}{rgb}{0.283229,0.120777,0.440584}%
\pgfsetfillcolor{currentfill}%
\pgfsetfillopacity{0.700000}%
\pgfsetlinewidth{0.000000pt}%
\definecolor{currentstroke}{rgb}{0.000000,0.000000,0.000000}%
\pgfsetstrokecolor{currentstroke}%
\pgfsetdash{}{0pt}%
\pgfpathmoveto{\pgfqpoint{4.338261in}{1.426358in}}%
\pgfpathlineto{\pgfqpoint{4.352025in}{1.420211in}}%
\pgfpathlineto{\pgfqpoint{4.365795in}{1.414086in}}%
\pgfpathlineto{\pgfqpoint{4.379570in}{1.407985in}}%
\pgfpathlineto{\pgfqpoint{4.393351in}{1.401906in}}%
\pgfpathlineto{\pgfqpoint{4.385608in}{1.403186in}}%
\pgfpathlineto{\pgfqpoint{4.377860in}{1.404891in}}%
\pgfpathlineto{\pgfqpoint{4.370104in}{1.407030in}}%
\pgfpathlineto{\pgfqpoint{4.362343in}{1.409614in}}%
\pgfpathlineto{\pgfqpoint{4.348537in}{1.416063in}}%
\pgfpathlineto{\pgfqpoint{4.334737in}{1.422536in}}%
\pgfpathlineto{\pgfqpoint{4.320943in}{1.429031in}}%
\pgfpathlineto{\pgfqpoint{4.307154in}{1.435550in}}%
\pgfpathlineto{\pgfqpoint{4.314941in}{1.432589in}}%
\pgfpathlineto{\pgfqpoint{4.322721in}{1.430077in}}%
\pgfpathlineto{\pgfqpoint{4.330495in}{1.428004in}}%
\pgfpathlineto{\pgfqpoint{4.338261in}{1.426358in}}%
\pgfpathclose%
\pgfusepath{fill}%
\end{pgfscope}%
\begin{pgfscope}%
\pgfpathrectangle{\pgfqpoint{1.254980in}{0.150000in}}{\pgfqpoint{5.490039in}{5.490039in}}%
\pgfusepath{clip}%
\pgfsetbuttcap%
\pgfsetroundjoin%
\definecolor{currentfill}{rgb}{0.172719,0.448791,0.557885}%
\pgfsetfillcolor{currentfill}%
\pgfsetfillopacity{0.700000}%
\pgfsetlinewidth{0.000000pt}%
\definecolor{currentstroke}{rgb}{0.000000,0.000000,0.000000}%
\pgfsetstrokecolor{currentstroke}%
\pgfsetdash{}{0pt}%
\pgfpathmoveto{\pgfqpoint{3.170039in}{2.168251in}}%
\pgfpathlineto{\pgfqpoint{3.183634in}{2.158602in}}%
\pgfpathlineto{\pgfqpoint{3.197232in}{2.148982in}}%
\pgfpathlineto{\pgfqpoint{3.210833in}{2.139391in}}%
\pgfpathlineto{\pgfqpoint{3.224437in}{2.129828in}}%
\pgfpathlineto{\pgfqpoint{3.215758in}{2.147377in}}%
\pgfpathlineto{\pgfqpoint{3.207052in}{2.165612in}}%
\pgfpathlineto{\pgfqpoint{3.198319in}{2.184547in}}%
\pgfpathlineto{\pgfqpoint{3.189558in}{2.204196in}}%
\pgfpathlineto{\pgfqpoint{3.175901in}{2.214202in}}%
\pgfpathlineto{\pgfqpoint{3.162247in}{2.224236in}}%
\pgfpathlineto{\pgfqpoint{3.148596in}{2.234300in}}%
\pgfpathlineto{\pgfqpoint{3.134947in}{2.244393in}}%
\pgfpathlineto{\pgfqpoint{3.143763in}{2.224293in}}%
\pgfpathlineto{\pgfqpoint{3.152549in}{2.204913in}}%
\pgfpathlineto{\pgfqpoint{3.161308in}{2.186236in}}%
\pgfpathlineto{\pgfqpoint{3.170039in}{2.168251in}}%
\pgfpathclose%
\pgfusepath{fill}%
\end{pgfscope}%
\begin{pgfscope}%
\pgfpathrectangle{\pgfqpoint{1.254980in}{0.150000in}}{\pgfqpoint{5.490039in}{5.490039in}}%
\pgfusepath{clip}%
\pgfsetbuttcap%
\pgfsetroundjoin%
\definecolor{currentfill}{rgb}{0.281446,0.084320,0.407414}%
\pgfsetfillcolor{currentfill}%
\pgfsetfillopacity{0.700000}%
\pgfsetlinewidth{0.000000pt}%
\definecolor{currentstroke}{rgb}{0.000000,0.000000,0.000000}%
\pgfsetstrokecolor{currentstroke}%
\pgfsetdash{}{0pt}%
\pgfpathmoveto{\pgfqpoint{4.534536in}{1.355872in}}%
\pgfpathlineto{\pgfqpoint{4.548347in}{1.350354in}}%
\pgfpathlineto{\pgfqpoint{4.562165in}{1.344859in}}%
\pgfpathlineto{\pgfqpoint{4.575988in}{1.339386in}}%
\pgfpathlineto{\pgfqpoint{4.589817in}{1.333935in}}%
\pgfpathlineto{\pgfqpoint{4.582159in}{1.332563in}}%
\pgfpathlineto{\pgfqpoint{4.574497in}{1.331565in}}%
\pgfpathlineto{\pgfqpoint{4.566831in}{1.330951in}}%
\pgfpathlineto{\pgfqpoint{4.559161in}{1.330730in}}%
\pgfpathlineto{\pgfqpoint{4.545312in}{1.336537in}}%
\pgfpathlineto{\pgfqpoint{4.531468in}{1.342367in}}%
\pgfpathlineto{\pgfqpoint{4.517631in}{1.348219in}}%
\pgfpathlineto{\pgfqpoint{4.503799in}{1.354094in}}%
\pgfpathlineto{\pgfqpoint{4.511491in}{1.353952in}}%
\pgfpathlineto{\pgfqpoint{4.519177in}{1.354208in}}%
\pgfpathlineto{\pgfqpoint{4.526859in}{1.354851in}}%
\pgfpathlineto{\pgfqpoint{4.534536in}{1.355872in}}%
\pgfpathclose%
\pgfusepath{fill}%
\end{pgfscope}%
\begin{pgfscope}%
\pgfpathrectangle{\pgfqpoint{1.254980in}{0.150000in}}{\pgfqpoint{5.490039in}{5.490039in}}%
\pgfusepath{clip}%
\pgfsetbuttcap%
\pgfsetroundjoin%
\definecolor{currentfill}{rgb}{0.120081,0.622161,0.534946}%
\pgfsetfillcolor{currentfill}%
\pgfsetfillopacity{0.700000}%
\pgfsetlinewidth{0.000000pt}%
\definecolor{currentstroke}{rgb}{0.000000,0.000000,0.000000}%
\pgfsetstrokecolor{currentstroke}%
\pgfsetdash{}{0pt}%
\pgfpathmoveto{\pgfqpoint{2.645034in}{2.628608in}}%
\pgfpathlineto{\pgfqpoint{2.658612in}{2.617344in}}%
\pgfpathlineto{\pgfqpoint{2.672191in}{2.606116in}}%
\pgfpathlineto{\pgfqpoint{2.685771in}{2.594925in}}%
\pgfpathlineto{\pgfqpoint{2.699353in}{2.583770in}}%
\pgfpathlineto{\pgfqpoint{2.690030in}{2.608280in}}%
\pgfpathlineto{\pgfqpoint{2.680667in}{2.633576in}}%
\pgfpathlineto{\pgfqpoint{2.671265in}{2.659673in}}%
\pgfpathlineto{\pgfqpoint{2.661822in}{2.686586in}}%
\pgfpathlineto{\pgfqpoint{2.648175in}{2.698213in}}%
\pgfpathlineto{\pgfqpoint{2.634529in}{2.709877in}}%
\pgfpathlineto{\pgfqpoint{2.620884in}{2.721577in}}%
\pgfpathlineto{\pgfqpoint{2.607240in}{2.733314in}}%
\pgfpathlineto{\pgfqpoint{2.616751in}{2.705920in}}%
\pgfpathlineto{\pgfqpoint{2.626219in}{2.679348in}}%
\pgfpathlineto{\pgfqpoint{2.635647in}{2.653582in}}%
\pgfpathlineto{\pgfqpoint{2.645034in}{2.628608in}}%
\pgfpathclose%
\pgfusepath{fill}%
\end{pgfscope}%
\begin{pgfscope}%
\pgfpathrectangle{\pgfqpoint{1.254980in}{0.150000in}}{\pgfqpoint{5.490039in}{5.490039in}}%
\pgfusepath{clip}%
\pgfsetbuttcap%
\pgfsetroundjoin%
\definecolor{currentfill}{rgb}{0.280255,0.165693,0.476498}%
\pgfsetfillcolor{currentfill}%
\pgfsetfillopacity{0.700000}%
\pgfsetlinewidth{0.000000pt}%
\definecolor{currentstroke}{rgb}{0.000000,0.000000,0.000000}%
\pgfsetstrokecolor{currentstroke}%
\pgfsetdash{}{0pt}%
\pgfpathmoveto{\pgfqpoint{4.142088in}{1.515566in}}%
\pgfpathlineto{\pgfqpoint{4.155815in}{1.508771in}}%
\pgfpathlineto{\pgfqpoint{4.169548in}{1.501999in}}%
\pgfpathlineto{\pgfqpoint{4.183285in}{1.495250in}}%
\pgfpathlineto{\pgfqpoint{4.197028in}{1.488524in}}%
\pgfpathlineto{\pgfqpoint{4.189178in}{1.492704in}}%
\pgfpathlineto{\pgfqpoint{4.181319in}{1.497360in}}%
\pgfpathlineto{\pgfqpoint{4.173451in}{1.502505in}}%
\pgfpathlineto{\pgfqpoint{4.165574in}{1.508148in}}%
\pgfpathlineto{\pgfqpoint{4.151802in}{1.515260in}}%
\pgfpathlineto{\pgfqpoint{4.138035in}{1.522395in}}%
\pgfpathlineto{\pgfqpoint{4.124273in}{1.529553in}}%
\pgfpathlineto{\pgfqpoint{4.110515in}{1.536735in}}%
\pgfpathlineto{\pgfqpoint{4.118423in}{1.530700in}}%
\pgfpathlineto{\pgfqpoint{4.126321in}{1.525168in}}%
\pgfpathlineto{\pgfqpoint{4.134209in}{1.520127in}}%
\pgfpathlineto{\pgfqpoint{4.142088in}{1.515566in}}%
\pgfpathclose%
\pgfusepath{fill}%
\end{pgfscope}%
\begin{pgfscope}%
\pgfpathrectangle{\pgfqpoint{1.254980in}{0.150000in}}{\pgfqpoint{5.490039in}{5.490039in}}%
\pgfusepath{clip}%
\pgfsetbuttcap%
\pgfsetroundjoin%
\definecolor{currentfill}{rgb}{0.227802,0.326594,0.546532}%
\pgfsetfillcolor{currentfill}%
\pgfsetfillopacity{0.700000}%
\pgfsetlinewidth{0.000000pt}%
\definecolor{currentstroke}{rgb}{0.000000,0.000000,0.000000}%
\pgfsetstrokecolor{currentstroke}%
\pgfsetdash{}{0pt}%
\pgfpathmoveto{\pgfqpoint{3.585204in}{1.855163in}}%
\pgfpathlineto{\pgfqpoint{3.598844in}{1.846700in}}%
\pgfpathlineto{\pgfqpoint{3.612487in}{1.838262in}}%
\pgfpathlineto{\pgfqpoint{3.626135in}{1.829850in}}%
\pgfpathlineto{\pgfqpoint{3.639786in}{1.821463in}}%
\pgfpathlineto{\pgfqpoint{3.631519in}{1.833483in}}%
\pgfpathlineto{\pgfqpoint{3.623233in}{1.846107in}}%
\pgfpathlineto{\pgfqpoint{3.614929in}{1.859349in}}%
\pgfpathlineto{\pgfqpoint{3.606605in}{1.873219in}}%
\pgfpathlineto{\pgfqpoint{3.592911in}{1.882027in}}%
\pgfpathlineto{\pgfqpoint{3.579221in}{1.890860in}}%
\pgfpathlineto{\pgfqpoint{3.565534in}{1.899719in}}%
\pgfpathlineto{\pgfqpoint{3.551850in}{1.908604in}}%
\pgfpathlineto{\pgfqpoint{3.560218in}{1.894306in}}%
\pgfpathlineto{\pgfqpoint{3.568566in}{1.880641in}}%
\pgfpathlineto{\pgfqpoint{3.576894in}{1.867598in}}%
\pgfpathlineto{\pgfqpoint{3.585204in}{1.855163in}}%
\pgfpathclose%
\pgfusepath{fill}%
\end{pgfscope}%
\begin{pgfscope}%
\pgfpathrectangle{\pgfqpoint{1.254980in}{0.150000in}}{\pgfqpoint{5.490039in}{5.490039in}}%
\pgfusepath{clip}%
\pgfsetbuttcap%
\pgfsetroundjoin%
\definecolor{currentfill}{rgb}{0.263663,0.237631,0.518762}%
\pgfsetfillcolor{currentfill}%
\pgfsetfillopacity{0.700000}%
\pgfsetlinewidth{0.000000pt}%
\definecolor{currentstroke}{rgb}{0.000000,0.000000,0.000000}%
\pgfsetstrokecolor{currentstroke}%
\pgfsetdash{}{0pt}%
\pgfpathmoveto{\pgfqpoint{3.891033in}{1.654858in}}%
\pgfpathlineto{\pgfqpoint{3.904717in}{1.647296in}}%
\pgfpathlineto{\pgfqpoint{3.918405in}{1.639758in}}%
\pgfpathlineto{\pgfqpoint{3.932097in}{1.632244in}}%
\pgfpathlineto{\pgfqpoint{3.945794in}{1.624755in}}%
\pgfpathlineto{\pgfqpoint{3.937777in}{1.632491in}}%
\pgfpathlineto{\pgfqpoint{3.929746in}{1.640765in}}%
\pgfpathlineto{\pgfqpoint{3.921703in}{1.649586in}}%
\pgfpathlineto{\pgfqpoint{3.913646in}{1.658968in}}%
\pgfpathlineto{\pgfqpoint{3.899914in}{1.666860in}}%
\pgfpathlineto{\pgfqpoint{3.886186in}{1.674776in}}%
\pgfpathlineto{\pgfqpoint{3.872462in}{1.682717in}}%
\pgfpathlineto{\pgfqpoint{3.858743in}{1.690681in}}%
\pgfpathlineto{\pgfqpoint{3.866836in}{1.680891in}}%
\pgfpathlineto{\pgfqpoint{3.874916in}{1.671665in}}%
\pgfpathlineto{\pgfqpoint{3.882981in}{1.662991in}}%
\pgfpathlineto{\pgfqpoint{3.891033in}{1.654858in}}%
\pgfpathclose%
\pgfusepath{fill}%
\end{pgfscope}%
\begin{pgfscope}%
\pgfpathrectangle{\pgfqpoint{1.254980in}{0.150000in}}{\pgfqpoint{5.490039in}{5.490039in}}%
\pgfusepath{clip}%
\pgfsetbuttcap%
\pgfsetroundjoin%
\definecolor{currentfill}{rgb}{0.276022,0.044167,0.370164}%
\pgfsetfillcolor{currentfill}%
\pgfsetfillopacity{0.700000}%
\pgfsetlinewidth{0.000000pt}%
\definecolor{currentstroke}{rgb}{0.000000,0.000000,0.000000}%
\pgfsetstrokecolor{currentstroke}%
\pgfsetdash{}{0pt}%
\pgfpathmoveto{\pgfqpoint{4.872538in}{1.283901in}}%
\pgfpathlineto{\pgfqpoint{4.886445in}{1.279504in}}%
\pgfpathlineto{\pgfqpoint{4.900359in}{1.275130in}}%
\pgfpathlineto{\pgfqpoint{4.914279in}{1.270779in}}%
\pgfpathlineto{\pgfqpoint{4.928207in}{1.266449in}}%
\pgfpathlineto{\pgfqpoint{4.920650in}{1.260787in}}%
\pgfpathlineto{\pgfqpoint{4.913091in}{1.255410in}}%
\pgfpathlineto{\pgfqpoint{4.905530in}{1.250327in}}%
\pgfpathlineto{\pgfqpoint{4.897968in}{1.245547in}}%
\pgfpathlineto{\pgfqpoint{4.884028in}{1.250207in}}%
\pgfpathlineto{\pgfqpoint{4.870094in}{1.254889in}}%
\pgfpathlineto{\pgfqpoint{4.856167in}{1.259593in}}%
\pgfpathlineto{\pgfqpoint{4.842247in}{1.264320in}}%
\pgfpathlineto{\pgfqpoint{4.849823in}{1.268764in}}%
\pgfpathlineto{\pgfqpoint{4.857397in}{1.273515in}}%
\pgfpathlineto{\pgfqpoint{4.864968in}{1.278564in}}%
\pgfpathlineto{\pgfqpoint{4.872538in}{1.283901in}}%
\pgfpathclose%
\pgfusepath{fill}%
\end{pgfscope}%
\begin{pgfscope}%
\pgfpathrectangle{\pgfqpoint{1.254980in}{0.150000in}}{\pgfqpoint{5.490039in}{5.490039in}}%
\pgfusepath{clip}%
\pgfsetbuttcap%
\pgfsetroundjoin%
\definecolor{currentfill}{rgb}{0.274952,0.037752,0.364543}%
\pgfsetfillcolor{currentfill}%
\pgfsetfillopacity{0.700000}%
\pgfsetlinewidth{0.000000pt}%
\definecolor{currentstroke}{rgb}{0.000000,0.000000,0.000000}%
\pgfsetstrokecolor{currentstroke}%
\pgfsetdash{}{0pt}%
\pgfpathmoveto{\pgfqpoint{5.014153in}{1.275959in}}%
\pgfpathlineto{\pgfqpoint{5.028104in}{1.272058in}}%
\pgfpathlineto{\pgfqpoint{5.042063in}{1.268180in}}%
\pgfpathlineto{\pgfqpoint{5.056029in}{1.264324in}}%
\pgfpathlineto{\pgfqpoint{5.048497in}{1.257058in}}%
\pgfpathlineto{\pgfqpoint{5.040963in}{1.250037in}}%
\pgfpathlineto{\pgfqpoint{5.033429in}{1.243271in}}%
\pgfpathlineto{\pgfqpoint{5.025893in}{1.236767in}}%
\pgfpathlineto{\pgfqpoint{5.011917in}{1.240941in}}%
\pgfpathlineto{\pgfqpoint{4.997948in}{1.245137in}}%
\pgfpathlineto{\pgfqpoint{4.983985in}{1.249355in}}%
\pgfpathlineto{\pgfqpoint{4.991529in}{1.255616in}}%
\pgfpathlineto{\pgfqpoint{4.999072in}{1.262143in}}%
\pgfpathlineto{\pgfqpoint{5.006613in}{1.268926in}}%
\pgfpathlineto{\pgfqpoint{5.014153in}{1.275959in}}%
\pgfpathclose%
\pgfusepath{fill}%
\end{pgfscope}%
\begin{pgfscope}%
\pgfpathrectangle{\pgfqpoint{1.254980in}{0.150000in}}{\pgfqpoint{5.490039in}{5.490039in}}%
\pgfusepath{clip}%
\pgfsetbuttcap%
\pgfsetroundjoin%
\definecolor{currentfill}{rgb}{0.177423,0.437527,0.557565}%
\pgfsetfillcolor{currentfill}%
\pgfsetfillopacity{0.700000}%
\pgfsetlinewidth{0.000000pt}%
\definecolor{currentstroke}{rgb}{0.000000,0.000000,0.000000}%
\pgfsetstrokecolor{currentstroke}%
\pgfsetdash{}{0pt}%
\pgfpathmoveto{\pgfqpoint{3.224437in}{2.129828in}}%
\pgfpathlineto{\pgfqpoint{3.238044in}{2.120294in}}%
\pgfpathlineto{\pgfqpoint{3.251653in}{2.110788in}}%
\pgfpathlineto{\pgfqpoint{3.265266in}{2.101311in}}%
\pgfpathlineto{\pgfqpoint{3.278881in}{2.091861in}}%
\pgfpathlineto{\pgfqpoint{3.270253in}{2.108974in}}%
\pgfpathlineto{\pgfqpoint{3.261598in}{2.126769in}}%
\pgfpathlineto{\pgfqpoint{3.252918in}{2.145260in}}%
\pgfpathlineto{\pgfqpoint{3.244210in}{2.164459in}}%
\pgfpathlineto{\pgfqpoint{3.230543in}{2.174350in}}%
\pgfpathlineto{\pgfqpoint{3.216878in}{2.184270in}}%
\pgfpathlineto{\pgfqpoint{3.203217in}{2.194219in}}%
\pgfpathlineto{\pgfqpoint{3.189558in}{2.204196in}}%
\pgfpathlineto{\pgfqpoint{3.198319in}{2.184547in}}%
\pgfpathlineto{\pgfqpoint{3.207052in}{2.165612in}}%
\pgfpathlineto{\pgfqpoint{3.215758in}{2.147377in}}%
\pgfpathlineto{\pgfqpoint{3.224437in}{2.129828in}}%
\pgfpathclose%
\pgfusepath{fill}%
\end{pgfscope}%
\begin{pgfscope}%
\pgfpathrectangle{\pgfqpoint{1.254980in}{0.150000in}}{\pgfqpoint{5.490039in}{5.490039in}}%
\pgfusepath{clip}%
\pgfsetbuttcap%
\pgfsetroundjoin%
\definecolor{currentfill}{rgb}{0.277941,0.056324,0.381191}%
\pgfsetfillcolor{currentfill}%
\pgfsetfillopacity{0.700000}%
\pgfsetlinewidth{0.000000pt}%
\definecolor{currentstroke}{rgb}{0.000000,0.000000,0.000000}%
\pgfsetstrokecolor{currentstroke}%
\pgfsetdash{}{0pt}%
\pgfpathmoveto{\pgfqpoint{4.731121in}{1.302935in}}%
\pgfpathlineto{\pgfqpoint{4.744989in}{1.298030in}}%
\pgfpathlineto{\pgfqpoint{4.758864in}{1.293147in}}%
\pgfpathlineto{\pgfqpoint{4.772745in}{1.288287in}}%
\pgfpathlineto{\pgfqpoint{4.786632in}{1.283449in}}%
\pgfpathlineto{\pgfqpoint{4.779039in}{1.279657in}}%
\pgfpathlineto{\pgfqpoint{4.771444in}{1.276193in}}%
\pgfpathlineto{\pgfqpoint{4.763846in}{1.273064in}}%
\pgfpathlineto{\pgfqpoint{4.756245in}{1.270282in}}%
\pgfpathlineto{\pgfqpoint{4.742342in}{1.275463in}}%
\pgfpathlineto{\pgfqpoint{4.728445in}{1.280667in}}%
\pgfpathlineto{\pgfqpoint{4.714554in}{1.285893in}}%
\pgfpathlineto{\pgfqpoint{4.700670in}{1.291142in}}%
\pgfpathlineto{\pgfqpoint{4.708287in}{1.293576in}}%
\pgfpathlineto{\pgfqpoint{4.715902in}{1.296359in}}%
\pgfpathlineto{\pgfqpoint{4.723513in}{1.299482in}}%
\pgfpathlineto{\pgfqpoint{4.731121in}{1.302935in}}%
\pgfpathclose%
\pgfusepath{fill}%
\end{pgfscope}%
\begin{pgfscope}%
\pgfpathrectangle{\pgfqpoint{1.254980in}{0.150000in}}{\pgfqpoint{5.490039in}{5.490039in}}%
\pgfusepath{clip}%
\pgfsetbuttcap%
\pgfsetroundjoin%
\definecolor{currentfill}{rgb}{0.119512,0.607464,0.540218}%
\pgfsetfillcolor{currentfill}%
\pgfsetfillopacity{0.700000}%
\pgfsetlinewidth{0.000000pt}%
\definecolor{currentstroke}{rgb}{0.000000,0.000000,0.000000}%
\pgfsetstrokecolor{currentstroke}%
\pgfsetdash{}{0pt}%
\pgfpathmoveto{\pgfqpoint{2.699353in}{2.583770in}}%
\pgfpathlineto{\pgfqpoint{2.712936in}{2.572651in}}%
\pgfpathlineto{\pgfqpoint{2.726521in}{2.561567in}}%
\pgfpathlineto{\pgfqpoint{2.740108in}{2.550518in}}%
\pgfpathlineto{\pgfqpoint{2.753696in}{2.539503in}}%
\pgfpathlineto{\pgfqpoint{2.744436in}{2.563549in}}%
\pgfpathlineto{\pgfqpoint{2.735138in}{2.588376in}}%
\pgfpathlineto{\pgfqpoint{2.725801in}{2.613999in}}%
\pgfpathlineto{\pgfqpoint{2.716424in}{2.640433in}}%
\pgfpathlineto{\pgfqpoint{2.702772in}{2.651918in}}%
\pgfpathlineto{\pgfqpoint{2.689121in}{2.663439in}}%
\pgfpathlineto{\pgfqpoint{2.675471in}{2.674994in}}%
\pgfpathlineto{\pgfqpoint{2.661822in}{2.686586in}}%
\pgfpathlineto{\pgfqpoint{2.671265in}{2.659673in}}%
\pgfpathlineto{\pgfqpoint{2.680667in}{2.633576in}}%
\pgfpathlineto{\pgfqpoint{2.690030in}{2.608280in}}%
\pgfpathlineto{\pgfqpoint{2.699353in}{2.583770in}}%
\pgfpathclose%
\pgfusepath{fill}%
\end{pgfscope}%
\begin{pgfscope}%
\pgfpathrectangle{\pgfqpoint{1.254980in}{0.150000in}}{\pgfqpoint{5.490039in}{5.490039in}}%
\pgfusepath{clip}%
\pgfsetbuttcap%
\pgfsetroundjoin%
\definecolor{currentfill}{rgb}{0.283197,0.115680,0.436115}%
\pgfsetfillcolor{currentfill}%
\pgfsetfillopacity{0.700000}%
\pgfsetlinewidth{0.000000pt}%
\definecolor{currentstroke}{rgb}{0.000000,0.000000,0.000000}%
\pgfsetstrokecolor{currentstroke}%
\pgfsetdash{}{0pt}%
\pgfpathmoveto{\pgfqpoint{4.393351in}{1.401906in}}%
\pgfpathlineto{\pgfqpoint{4.407137in}{1.395850in}}%
\pgfpathlineto{\pgfqpoint{4.420929in}{1.389817in}}%
\pgfpathlineto{\pgfqpoint{4.434726in}{1.383806in}}%
\pgfpathlineto{\pgfqpoint{4.448529in}{1.377819in}}%
\pgfpathlineto{\pgfqpoint{4.440810in}{1.378733in}}%
\pgfpathlineto{\pgfqpoint{4.433085in}{1.380069in}}%
\pgfpathlineto{\pgfqpoint{4.425354in}{1.381835in}}%
\pgfpathlineto{\pgfqpoint{4.417618in}{1.384042in}}%
\pgfpathlineto{\pgfqpoint{4.403791in}{1.390401in}}%
\pgfpathlineto{\pgfqpoint{4.389969in}{1.396782in}}%
\pgfpathlineto{\pgfqpoint{4.376153in}{1.403187in}}%
\pgfpathlineto{\pgfqpoint{4.362343in}{1.409614in}}%
\pgfpathlineto{\pgfqpoint{4.370104in}{1.407030in}}%
\pgfpathlineto{\pgfqpoint{4.377860in}{1.404891in}}%
\pgfpathlineto{\pgfqpoint{4.385608in}{1.403186in}}%
\pgfpathlineto{\pgfqpoint{4.393351in}{1.401906in}}%
\pgfpathclose%
\pgfusepath{fill}%
\end{pgfscope}%
\begin{pgfscope}%
\pgfpathrectangle{\pgfqpoint{1.254980in}{0.150000in}}{\pgfqpoint{5.490039in}{5.490039in}}%
\pgfusepath{clip}%
\pgfsetbuttcap%
\pgfsetroundjoin%
\definecolor{currentfill}{rgb}{0.231674,0.318106,0.544834}%
\pgfsetfillcolor{currentfill}%
\pgfsetfillopacity{0.700000}%
\pgfsetlinewidth{0.000000pt}%
\definecolor{currentstroke}{rgb}{0.000000,0.000000,0.000000}%
\pgfsetstrokecolor{currentstroke}%
\pgfsetdash{}{0pt}%
\pgfpathmoveto{\pgfqpoint{3.639786in}{1.821463in}}%
\pgfpathlineto{\pgfqpoint{3.653441in}{1.813102in}}%
\pgfpathlineto{\pgfqpoint{3.667100in}{1.804766in}}%
\pgfpathlineto{\pgfqpoint{3.680763in}{1.796455in}}%
\pgfpathlineto{\pgfqpoint{3.694430in}{1.788170in}}%
\pgfpathlineto{\pgfqpoint{3.686203in}{1.799775in}}%
\pgfpathlineto{\pgfqpoint{3.677960in}{1.811981in}}%
\pgfpathlineto{\pgfqpoint{3.669698in}{1.824800in}}%
\pgfpathlineto{\pgfqpoint{3.661418in}{1.838244in}}%
\pgfpathlineto{\pgfqpoint{3.647709in}{1.846950in}}%
\pgfpathlineto{\pgfqpoint{3.634004in}{1.855681in}}%
\pgfpathlineto{\pgfqpoint{3.620303in}{1.864438in}}%
\pgfpathlineto{\pgfqpoint{3.606605in}{1.873219in}}%
\pgfpathlineto{\pgfqpoint{3.614929in}{1.859349in}}%
\pgfpathlineto{\pgfqpoint{3.623233in}{1.846107in}}%
\pgfpathlineto{\pgfqpoint{3.631519in}{1.833483in}}%
\pgfpathlineto{\pgfqpoint{3.639786in}{1.821463in}}%
\pgfpathclose%
\pgfusepath{fill}%
\end{pgfscope}%
\begin{pgfscope}%
\pgfpathrectangle{\pgfqpoint{1.254980in}{0.150000in}}{\pgfqpoint{5.490039in}{5.490039in}}%
\pgfusepath{clip}%
\pgfsetbuttcap%
\pgfsetroundjoin%
\definecolor{currentfill}{rgb}{0.280894,0.078907,0.402329}%
\pgfsetfillcolor{currentfill}%
\pgfsetfillopacity{0.700000}%
\pgfsetlinewidth{0.000000pt}%
\definecolor{currentstroke}{rgb}{0.000000,0.000000,0.000000}%
\pgfsetstrokecolor{currentstroke}%
\pgfsetdash{}{0pt}%
\pgfpathmoveto{\pgfqpoint{4.589817in}{1.333935in}}%
\pgfpathlineto{\pgfqpoint{4.603652in}{1.328508in}}%
\pgfpathlineto{\pgfqpoint{4.617493in}{1.323102in}}%
\pgfpathlineto{\pgfqpoint{4.631341in}{1.317720in}}%
\pgfpathlineto{\pgfqpoint{4.645194in}{1.312359in}}%
\pgfpathlineto{\pgfqpoint{4.637555in}{1.310635in}}%
\pgfpathlineto{\pgfqpoint{4.629913in}{1.309281in}}%
\pgfpathlineto{\pgfqpoint{4.622266in}{1.308309in}}%
\pgfpathlineto{\pgfqpoint{4.614616in}{1.307726in}}%
\pgfpathlineto{\pgfqpoint{4.600743in}{1.313444in}}%
\pgfpathlineto{\pgfqpoint{4.586877in}{1.319183in}}%
\pgfpathlineto{\pgfqpoint{4.573016in}{1.324946in}}%
\pgfpathlineto{\pgfqpoint{4.559161in}{1.330730in}}%
\pgfpathlineto{\pgfqpoint{4.566831in}{1.330951in}}%
\pgfpathlineto{\pgfqpoint{4.574497in}{1.331565in}}%
\pgfpathlineto{\pgfqpoint{4.582159in}{1.332563in}}%
\pgfpathlineto{\pgfqpoint{4.589817in}{1.333935in}}%
\pgfpathclose%
\pgfusepath{fill}%
\end{pgfscope}%
\begin{pgfscope}%
\pgfpathrectangle{\pgfqpoint{1.254980in}{0.150000in}}{\pgfqpoint{5.490039in}{5.490039in}}%
\pgfusepath{clip}%
\pgfsetbuttcap%
\pgfsetroundjoin%
\definecolor{currentfill}{rgb}{0.266580,0.228262,0.514349}%
\pgfsetfillcolor{currentfill}%
\pgfsetfillopacity{0.700000}%
\pgfsetlinewidth{0.000000pt}%
\definecolor{currentstroke}{rgb}{0.000000,0.000000,0.000000}%
\pgfsetstrokecolor{currentstroke}%
\pgfsetdash{}{0pt}%
\pgfpathmoveto{\pgfqpoint{3.945794in}{1.624755in}}%
\pgfpathlineto{\pgfqpoint{3.959495in}{1.617289in}}%
\pgfpathlineto{\pgfqpoint{3.973201in}{1.609847in}}%
\pgfpathlineto{\pgfqpoint{3.986912in}{1.602429in}}%
\pgfpathlineto{\pgfqpoint{4.000627in}{1.595035in}}%
\pgfpathlineto{\pgfqpoint{3.992643in}{1.602376in}}%
\pgfpathlineto{\pgfqpoint{3.984648in}{1.610249in}}%
\pgfpathlineto{\pgfqpoint{3.976640in}{1.618666in}}%
\pgfpathlineto{\pgfqpoint{3.968619in}{1.627640in}}%
\pgfpathlineto{\pgfqpoint{3.954869in}{1.635436in}}%
\pgfpathlineto{\pgfqpoint{3.941124in}{1.643256in}}%
\pgfpathlineto{\pgfqpoint{3.927383in}{1.651100in}}%
\pgfpathlineto{\pgfqpoint{3.913646in}{1.658968in}}%
\pgfpathlineto{\pgfqpoint{3.921703in}{1.649586in}}%
\pgfpathlineto{\pgfqpoint{3.929746in}{1.640765in}}%
\pgfpathlineto{\pgfqpoint{3.937777in}{1.632491in}}%
\pgfpathlineto{\pgfqpoint{3.945794in}{1.624755in}}%
\pgfpathclose%
\pgfusepath{fill}%
\end{pgfscope}%
\begin{pgfscope}%
\pgfpathrectangle{\pgfqpoint{1.254980in}{0.150000in}}{\pgfqpoint{5.490039in}{5.490039in}}%
\pgfusepath{clip}%
\pgfsetbuttcap%
\pgfsetroundjoin%
\definecolor{currentfill}{rgb}{0.121148,0.592739,0.544641}%
\pgfsetfillcolor{currentfill}%
\pgfsetfillopacity{0.700000}%
\pgfsetlinewidth{0.000000pt}%
\definecolor{currentstroke}{rgb}{0.000000,0.000000,0.000000}%
\pgfsetstrokecolor{currentstroke}%
\pgfsetdash{}{0pt}%
\pgfpathmoveto{\pgfqpoint{2.753696in}{2.539503in}}%
\pgfpathlineto{\pgfqpoint{2.767286in}{2.528524in}}%
\pgfpathlineto{\pgfqpoint{2.780877in}{2.517579in}}%
\pgfpathlineto{\pgfqpoint{2.794470in}{2.506668in}}%
\pgfpathlineto{\pgfqpoint{2.808065in}{2.495791in}}%
\pgfpathlineto{\pgfqpoint{2.798867in}{2.519375in}}%
\pgfpathlineto{\pgfqpoint{2.789632in}{2.543734in}}%
\pgfpathlineto{\pgfqpoint{2.780359in}{2.568884in}}%
\pgfpathlineto{\pgfqpoint{2.771048in}{2.594840in}}%
\pgfpathlineto{\pgfqpoint{2.757390in}{2.606187in}}%
\pgfpathlineto{\pgfqpoint{2.743733in}{2.617568in}}%
\pgfpathlineto{\pgfqpoint{2.730078in}{2.628983in}}%
\pgfpathlineto{\pgfqpoint{2.716424in}{2.640433in}}%
\pgfpathlineto{\pgfqpoint{2.725801in}{2.613999in}}%
\pgfpathlineto{\pgfqpoint{2.735138in}{2.588376in}}%
\pgfpathlineto{\pgfqpoint{2.744436in}{2.563549in}}%
\pgfpathlineto{\pgfqpoint{2.753696in}{2.539503in}}%
\pgfpathclose%
\pgfusepath{fill}%
\end{pgfscope}%
\begin{pgfscope}%
\pgfpathrectangle{\pgfqpoint{1.254980in}{0.150000in}}{\pgfqpoint{5.490039in}{5.490039in}}%
\pgfusepath{clip}%
\pgfsetbuttcap%
\pgfsetroundjoin%
\definecolor{currentfill}{rgb}{0.280868,0.160771,0.472899}%
\pgfsetfillcolor{currentfill}%
\pgfsetfillopacity{0.700000}%
\pgfsetlinewidth{0.000000pt}%
\definecolor{currentstroke}{rgb}{0.000000,0.000000,0.000000}%
\pgfsetstrokecolor{currentstroke}%
\pgfsetdash{}{0pt}%
\pgfpathmoveto{\pgfqpoint{4.197028in}{1.488524in}}%
\pgfpathlineto{\pgfqpoint{4.210776in}{1.481821in}}%
\pgfpathlineto{\pgfqpoint{4.224528in}{1.475142in}}%
\pgfpathlineto{\pgfqpoint{4.238286in}{1.468486in}}%
\pgfpathlineto{\pgfqpoint{4.252049in}{1.461852in}}%
\pgfpathlineto{\pgfqpoint{4.244227in}{1.465652in}}%
\pgfpathlineto{\pgfqpoint{4.236397in}{1.469924in}}%
\pgfpathlineto{\pgfqpoint{4.228558in}{1.474681in}}%
\pgfpathlineto{\pgfqpoint{4.220711in}{1.479933in}}%
\pgfpathlineto{\pgfqpoint{4.206919in}{1.486952in}}%
\pgfpathlineto{\pgfqpoint{4.193132in}{1.493994in}}%
\pgfpathlineto{\pgfqpoint{4.179350in}{1.501059in}}%
\pgfpathlineto{\pgfqpoint{4.165574in}{1.508148in}}%
\pgfpathlineto{\pgfqpoint{4.173451in}{1.502505in}}%
\pgfpathlineto{\pgfqpoint{4.181319in}{1.497360in}}%
\pgfpathlineto{\pgfqpoint{4.189178in}{1.492704in}}%
\pgfpathlineto{\pgfqpoint{4.197028in}{1.488524in}}%
\pgfpathclose%
\pgfusepath{fill}%
\end{pgfscope}%
\begin{pgfscope}%
\pgfpathrectangle{\pgfqpoint{1.254980in}{0.150000in}}{\pgfqpoint{5.490039in}{5.490039in}}%
\pgfusepath{clip}%
\pgfsetbuttcap%
\pgfsetroundjoin%
\definecolor{currentfill}{rgb}{0.182256,0.426184,0.557120}%
\pgfsetfillcolor{currentfill}%
\pgfsetfillopacity{0.700000}%
\pgfsetlinewidth{0.000000pt}%
\definecolor{currentstroke}{rgb}{0.000000,0.000000,0.000000}%
\pgfsetstrokecolor{currentstroke}%
\pgfsetdash{}{0pt}%
\pgfpathmoveto{\pgfqpoint{3.278881in}{2.091861in}}%
\pgfpathlineto{\pgfqpoint{3.292499in}{2.082440in}}%
\pgfpathlineto{\pgfqpoint{3.306121in}{2.073047in}}%
\pgfpathlineto{\pgfqpoint{3.319745in}{2.063681in}}%
\pgfpathlineto{\pgfqpoint{3.333373in}{2.054343in}}%
\pgfpathlineto{\pgfqpoint{3.324794in}{2.071021in}}%
\pgfpathlineto{\pgfqpoint{3.316190in}{2.088377in}}%
\pgfpathlineto{\pgfqpoint{3.307562in}{2.106423in}}%
\pgfpathlineto{\pgfqpoint{3.298907in}{2.125173in}}%
\pgfpathlineto{\pgfqpoint{3.285229in}{2.134953in}}%
\pgfpathlineto{\pgfqpoint{3.271553in}{2.144760in}}%
\pgfpathlineto{\pgfqpoint{3.257880in}{2.154595in}}%
\pgfpathlineto{\pgfqpoint{3.244210in}{2.164459in}}%
\pgfpathlineto{\pgfqpoint{3.252918in}{2.145260in}}%
\pgfpathlineto{\pgfqpoint{3.261598in}{2.126769in}}%
\pgfpathlineto{\pgfqpoint{3.270253in}{2.108974in}}%
\pgfpathlineto{\pgfqpoint{3.278881in}{2.091861in}}%
\pgfpathclose%
\pgfusepath{fill}%
\end{pgfscope}%
\begin{pgfscope}%
\pgfpathrectangle{\pgfqpoint{1.254980in}{0.150000in}}{\pgfqpoint{5.490039in}{5.490039in}}%
\pgfusepath{clip}%
\pgfsetbuttcap%
\pgfsetroundjoin%
\definecolor{currentfill}{rgb}{0.124395,0.578002,0.548287}%
\pgfsetfillcolor{currentfill}%
\pgfsetfillopacity{0.700000}%
\pgfsetlinewidth{0.000000pt}%
\definecolor{currentstroke}{rgb}{0.000000,0.000000,0.000000}%
\pgfsetstrokecolor{currentstroke}%
\pgfsetdash{}{0pt}%
\pgfpathmoveto{\pgfqpoint{2.808065in}{2.495791in}}%
\pgfpathlineto{\pgfqpoint{2.821662in}{2.484948in}}%
\pgfpathlineto{\pgfqpoint{2.835260in}{2.474138in}}%
\pgfpathlineto{\pgfqpoint{2.848860in}{2.463362in}}%
\pgfpathlineto{\pgfqpoint{2.862462in}{2.452618in}}%
\pgfpathlineto{\pgfqpoint{2.853325in}{2.475741in}}%
\pgfpathlineto{\pgfqpoint{2.844153in}{2.499634in}}%
\pgfpathlineto{\pgfqpoint{2.834944in}{2.524312in}}%
\pgfpathlineto{\pgfqpoint{2.825697in}{2.549792in}}%
\pgfpathlineto{\pgfqpoint{2.812033in}{2.561004in}}%
\pgfpathlineto{\pgfqpoint{2.798370in}{2.572249in}}%
\pgfpathlineto{\pgfqpoint{2.784708in}{2.583528in}}%
\pgfpathlineto{\pgfqpoint{2.771048in}{2.594840in}}%
\pgfpathlineto{\pgfqpoint{2.780359in}{2.568884in}}%
\pgfpathlineto{\pgfqpoint{2.789632in}{2.543734in}}%
\pgfpathlineto{\pgfqpoint{2.798867in}{2.519375in}}%
\pgfpathlineto{\pgfqpoint{2.808065in}{2.495791in}}%
\pgfpathclose%
\pgfusepath{fill}%
\end{pgfscope}%
\begin{pgfscope}%
\pgfpathrectangle{\pgfqpoint{1.254980in}{0.150000in}}{\pgfqpoint{5.490039in}{5.490039in}}%
\pgfusepath{clip}%
\pgfsetbuttcap%
\pgfsetroundjoin%
\definecolor{currentfill}{rgb}{0.276022,0.044167,0.370164}%
\pgfsetfillcolor{currentfill}%
\pgfsetfillopacity{0.700000}%
\pgfsetlinewidth{0.000000pt}%
\definecolor{currentstroke}{rgb}{0.000000,0.000000,0.000000}%
\pgfsetstrokecolor{currentstroke}%
\pgfsetdash{}{0pt}%
\pgfpathmoveto{\pgfqpoint{4.928207in}{1.266449in}}%
\pgfpathlineto{\pgfqpoint{4.942141in}{1.262142in}}%
\pgfpathlineto{\pgfqpoint{4.956082in}{1.257857in}}%
\pgfpathlineto{\pgfqpoint{4.970030in}{1.253595in}}%
\pgfpathlineto{\pgfqpoint{4.983985in}{1.249355in}}%
\pgfpathlineto{\pgfqpoint{4.976440in}{1.243367in}}%
\pgfpathlineto{\pgfqpoint{4.968893in}{1.237661in}}%
\pgfpathlineto{\pgfqpoint{4.961345in}{1.232246in}}%
\pgfpathlineto{\pgfqpoint{4.953796in}{1.227130in}}%
\pgfpathlineto{\pgfqpoint{4.939829in}{1.231701in}}%
\pgfpathlineto{\pgfqpoint{4.925868in}{1.236294in}}%
\pgfpathlineto{\pgfqpoint{4.911915in}{1.240909in}}%
\pgfpathlineto{\pgfqpoint{4.897968in}{1.245547in}}%
\pgfpathlineto{\pgfqpoint{4.905530in}{1.250327in}}%
\pgfpathlineto{\pgfqpoint{4.913091in}{1.255410in}}%
\pgfpathlineto{\pgfqpoint{4.920650in}{1.260787in}}%
\pgfpathlineto{\pgfqpoint{4.928207in}{1.266449in}}%
\pgfpathclose%
\pgfusepath{fill}%
\end{pgfscope}%
\begin{pgfscope}%
\pgfpathrectangle{\pgfqpoint{1.254980in}{0.150000in}}{\pgfqpoint{5.490039in}{5.490039in}}%
\pgfusepath{clip}%
\pgfsetbuttcap%
\pgfsetroundjoin%
\definecolor{currentfill}{rgb}{0.237441,0.305202,0.541921}%
\pgfsetfillcolor{currentfill}%
\pgfsetfillopacity{0.700000}%
\pgfsetlinewidth{0.000000pt}%
\definecolor{currentstroke}{rgb}{0.000000,0.000000,0.000000}%
\pgfsetstrokecolor{currentstroke}%
\pgfsetdash{}{0pt}%
\pgfpathmoveto{\pgfqpoint{3.694430in}{1.788170in}}%
\pgfpathlineto{\pgfqpoint{3.708100in}{1.779909in}}%
\pgfpathlineto{\pgfqpoint{3.721775in}{1.771674in}}%
\pgfpathlineto{\pgfqpoint{3.735453in}{1.763463in}}%
\pgfpathlineto{\pgfqpoint{3.749136in}{1.755278in}}%
\pgfpathlineto{\pgfqpoint{3.740950in}{1.766470in}}%
\pgfpathlineto{\pgfqpoint{3.732747in}{1.778258in}}%
\pgfpathlineto{\pgfqpoint{3.724527in}{1.790655in}}%
\pgfpathlineto{\pgfqpoint{3.716290in}{1.803672in}}%
\pgfpathlineto{\pgfqpoint{3.702567in}{1.812277in}}%
\pgfpathlineto{\pgfqpoint{3.688847in}{1.820908in}}%
\pgfpathlineto{\pgfqpoint{3.675131in}{1.829563in}}%
\pgfpathlineto{\pgfqpoint{3.661418in}{1.838244in}}%
\pgfpathlineto{\pgfqpoint{3.669698in}{1.824800in}}%
\pgfpathlineto{\pgfqpoint{3.677960in}{1.811981in}}%
\pgfpathlineto{\pgfqpoint{3.686203in}{1.799775in}}%
\pgfpathlineto{\pgfqpoint{3.694430in}{1.788170in}}%
\pgfpathclose%
\pgfusepath{fill}%
\end{pgfscope}%
\begin{pgfscope}%
\pgfpathrectangle{\pgfqpoint{1.254980in}{0.150000in}}{\pgfqpoint{5.490039in}{5.490039in}}%
\pgfusepath{clip}%
\pgfsetbuttcap%
\pgfsetroundjoin%
\definecolor{currentfill}{rgb}{0.277941,0.056324,0.381191}%
\pgfsetfillcolor{currentfill}%
\pgfsetfillopacity{0.700000}%
\pgfsetlinewidth{0.000000pt}%
\definecolor{currentstroke}{rgb}{0.000000,0.000000,0.000000}%
\pgfsetstrokecolor{currentstroke}%
\pgfsetdash{}{0pt}%
\pgfpathmoveto{\pgfqpoint{4.786632in}{1.283449in}}%
\pgfpathlineto{\pgfqpoint{4.800526in}{1.278633in}}%
\pgfpathlineto{\pgfqpoint{4.814426in}{1.273840in}}%
\pgfpathlineto{\pgfqpoint{4.828333in}{1.269068in}}%
\pgfpathlineto{\pgfqpoint{4.842247in}{1.264320in}}%
\pgfpathlineto{\pgfqpoint{4.834669in}{1.260189in}}%
\pgfpathlineto{\pgfqpoint{4.827089in}{1.256383in}}%
\pgfpathlineto{\pgfqpoint{4.819506in}{1.252910in}}%
\pgfpathlineto{\pgfqpoint{4.811922in}{1.249778in}}%
\pgfpathlineto{\pgfqpoint{4.797993in}{1.254871in}}%
\pgfpathlineto{\pgfqpoint{4.784070in}{1.259985in}}%
\pgfpathlineto{\pgfqpoint{4.770155in}{1.265122in}}%
\pgfpathlineto{\pgfqpoint{4.756245in}{1.270282in}}%
\pgfpathlineto{\pgfqpoint{4.763846in}{1.273064in}}%
\pgfpathlineto{\pgfqpoint{4.771444in}{1.276193in}}%
\pgfpathlineto{\pgfqpoint{4.779039in}{1.279657in}}%
\pgfpathlineto{\pgfqpoint{4.786632in}{1.283449in}}%
\pgfpathclose%
\pgfusepath{fill}%
\end{pgfscope}%
\begin{pgfscope}%
\pgfpathrectangle{\pgfqpoint{1.254980in}{0.150000in}}{\pgfqpoint{5.490039in}{5.490039in}}%
\pgfusepath{clip}%
\pgfsetbuttcap%
\pgfsetroundjoin%
\definecolor{currentfill}{rgb}{0.187231,0.414746,0.556547}%
\pgfsetfillcolor{currentfill}%
\pgfsetfillopacity{0.700000}%
\pgfsetlinewidth{0.000000pt}%
\definecolor{currentstroke}{rgb}{0.000000,0.000000,0.000000}%
\pgfsetstrokecolor{currentstroke}%
\pgfsetdash{}{0pt}%
\pgfpathmoveto{\pgfqpoint{3.333373in}{2.054343in}}%
\pgfpathlineto{\pgfqpoint{3.347003in}{2.045033in}}%
\pgfpathlineto{\pgfqpoint{3.360637in}{2.035750in}}%
\pgfpathlineto{\pgfqpoint{3.374274in}{2.026494in}}%
\pgfpathlineto{\pgfqpoint{3.387914in}{2.017265in}}%
\pgfpathlineto{\pgfqpoint{3.379384in}{2.033509in}}%
\pgfpathlineto{\pgfqpoint{3.370830in}{2.050426in}}%
\pgfpathlineto{\pgfqpoint{3.362252in}{2.068028in}}%
\pgfpathlineto{\pgfqpoint{3.353649in}{2.086331in}}%
\pgfpathlineto{\pgfqpoint{3.339959in}{2.096000in}}%
\pgfpathlineto{\pgfqpoint{3.326272in}{2.105697in}}%
\pgfpathlineto{\pgfqpoint{3.312588in}{2.115421in}}%
\pgfpathlineto{\pgfqpoint{3.298907in}{2.125173in}}%
\pgfpathlineto{\pgfqpoint{3.307562in}{2.106423in}}%
\pgfpathlineto{\pgfqpoint{3.316190in}{2.088377in}}%
\pgfpathlineto{\pgfqpoint{3.324794in}{2.071021in}}%
\pgfpathlineto{\pgfqpoint{3.333373in}{2.054343in}}%
\pgfpathclose%
\pgfusepath{fill}%
\end{pgfscope}%
\begin{pgfscope}%
\pgfpathrectangle{\pgfqpoint{1.254980in}{0.150000in}}{\pgfqpoint{5.490039in}{5.490039in}}%
\pgfusepath{clip}%
\pgfsetbuttcap%
\pgfsetroundjoin%
\definecolor{currentfill}{rgb}{0.283091,0.110553,0.431554}%
\pgfsetfillcolor{currentfill}%
\pgfsetfillopacity{0.700000}%
\pgfsetlinewidth{0.000000pt}%
\definecolor{currentstroke}{rgb}{0.000000,0.000000,0.000000}%
\pgfsetstrokecolor{currentstroke}%
\pgfsetdash{}{0pt}%
\pgfpathmoveto{\pgfqpoint{4.448529in}{1.377819in}}%
\pgfpathlineto{\pgfqpoint{4.462338in}{1.371853in}}%
\pgfpathlineto{\pgfqpoint{4.476153in}{1.365911in}}%
\pgfpathlineto{\pgfqpoint{4.489973in}{1.359991in}}%
\pgfpathlineto{\pgfqpoint{4.503799in}{1.354094in}}%
\pgfpathlineto{\pgfqpoint{4.496102in}{1.354643in}}%
\pgfpathlineto{\pgfqpoint{4.488400in}{1.355609in}}%
\pgfpathlineto{\pgfqpoint{4.480693in}{1.357002in}}%
\pgfpathlineto{\pgfqpoint{4.472980in}{1.358833in}}%
\pgfpathlineto{\pgfqpoint{4.459131in}{1.365102in}}%
\pgfpathlineto{\pgfqpoint{4.445288in}{1.371392in}}%
\pgfpathlineto{\pgfqpoint{4.431450in}{1.377706in}}%
\pgfpathlineto{\pgfqpoint{4.417618in}{1.384042in}}%
\pgfpathlineto{\pgfqpoint{4.425354in}{1.381835in}}%
\pgfpathlineto{\pgfqpoint{4.433085in}{1.380069in}}%
\pgfpathlineto{\pgfqpoint{4.440810in}{1.378733in}}%
\pgfpathlineto{\pgfqpoint{4.448529in}{1.377819in}}%
\pgfpathclose%
\pgfusepath{fill}%
\end{pgfscope}%
\begin{pgfscope}%
\pgfpathrectangle{\pgfqpoint{1.254980in}{0.150000in}}{\pgfqpoint{5.490039in}{5.490039in}}%
\pgfusepath{clip}%
\pgfsetbuttcap%
\pgfsetroundjoin%
\definecolor{currentfill}{rgb}{0.269308,0.218818,0.509577}%
\pgfsetfillcolor{currentfill}%
\pgfsetfillopacity{0.700000}%
\pgfsetlinewidth{0.000000pt}%
\definecolor{currentstroke}{rgb}{0.000000,0.000000,0.000000}%
\pgfsetstrokecolor{currentstroke}%
\pgfsetdash{}{0pt}%
\pgfpathmoveto{\pgfqpoint{4.000627in}{1.595035in}}%
\pgfpathlineto{\pgfqpoint{4.014347in}{1.587665in}}%
\pgfpathlineto{\pgfqpoint{4.028071in}{1.580319in}}%
\pgfpathlineto{\pgfqpoint{4.041800in}{1.572996in}}%
\pgfpathlineto{\pgfqpoint{4.055533in}{1.565697in}}%
\pgfpathlineto{\pgfqpoint{4.047583in}{1.572641in}}%
\pgfpathlineto{\pgfqpoint{4.039621in}{1.580114in}}%
\pgfpathlineto{\pgfqpoint{4.031647in}{1.588128in}}%
\pgfpathlineto{\pgfqpoint{4.023662in}{1.596693in}}%
\pgfpathlineto{\pgfqpoint{4.009894in}{1.604394in}}%
\pgfpathlineto{\pgfqpoint{3.996131in}{1.612119in}}%
\pgfpathlineto{\pgfqpoint{3.982373in}{1.619867in}}%
\pgfpathlineto{\pgfqpoint{3.968619in}{1.627640in}}%
\pgfpathlineto{\pgfqpoint{3.976640in}{1.618666in}}%
\pgfpathlineto{\pgfqpoint{3.984648in}{1.610249in}}%
\pgfpathlineto{\pgfqpoint{3.992643in}{1.602376in}}%
\pgfpathlineto{\pgfqpoint{4.000627in}{1.595035in}}%
\pgfpathclose%
\pgfusepath{fill}%
\end{pgfscope}%
\begin{pgfscope}%
\pgfpathrectangle{\pgfqpoint{1.254980in}{0.150000in}}{\pgfqpoint{5.490039in}{5.490039in}}%
\pgfusepath{clip}%
\pgfsetbuttcap%
\pgfsetroundjoin%
\definecolor{currentfill}{rgb}{0.281412,0.155834,0.469201}%
\pgfsetfillcolor{currentfill}%
\pgfsetfillopacity{0.700000}%
\pgfsetlinewidth{0.000000pt}%
\definecolor{currentstroke}{rgb}{0.000000,0.000000,0.000000}%
\pgfsetstrokecolor{currentstroke}%
\pgfsetdash{}{0pt}%
\pgfpathmoveto{\pgfqpoint{4.252049in}{1.461852in}}%
\pgfpathlineto{\pgfqpoint{4.265818in}{1.455242in}}%
\pgfpathlineto{\pgfqpoint{4.279591in}{1.448655in}}%
\pgfpathlineto{\pgfqpoint{4.293370in}{1.442091in}}%
\pgfpathlineto{\pgfqpoint{4.307154in}{1.435550in}}%
\pgfpathlineto{\pgfqpoint{4.299359in}{1.438969in}}%
\pgfpathlineto{\pgfqpoint{4.291556in}{1.442857in}}%
\pgfpathlineto{\pgfqpoint{4.283746in}{1.447226in}}%
\pgfpathlineto{\pgfqpoint{4.275927in}{1.452086in}}%
\pgfpathlineto{\pgfqpoint{4.262115in}{1.459013in}}%
\pgfpathlineto{\pgfqpoint{4.248309in}{1.465963in}}%
\pgfpathlineto{\pgfqpoint{4.234507in}{1.472936in}}%
\pgfpathlineto{\pgfqpoint{4.220711in}{1.479933in}}%
\pgfpathlineto{\pgfqpoint{4.228558in}{1.474681in}}%
\pgfpathlineto{\pgfqpoint{4.236397in}{1.469924in}}%
\pgfpathlineto{\pgfqpoint{4.244227in}{1.465652in}}%
\pgfpathlineto{\pgfqpoint{4.252049in}{1.461852in}}%
\pgfpathclose%
\pgfusepath{fill}%
\end{pgfscope}%
\begin{pgfscope}%
\pgfpathrectangle{\pgfqpoint{1.254980in}{0.150000in}}{\pgfqpoint{5.490039in}{5.490039in}}%
\pgfusepath{clip}%
\pgfsetbuttcap%
\pgfsetroundjoin%
\definecolor{currentfill}{rgb}{0.127568,0.566949,0.550556}%
\pgfsetfillcolor{currentfill}%
\pgfsetfillopacity{0.700000}%
\pgfsetlinewidth{0.000000pt}%
\definecolor{currentstroke}{rgb}{0.000000,0.000000,0.000000}%
\pgfsetstrokecolor{currentstroke}%
\pgfsetdash{}{0pt}%
\pgfpathmoveto{\pgfqpoint{2.862462in}{2.452618in}}%
\pgfpathlineto{\pgfqpoint{2.876066in}{2.441908in}}%
\pgfpathlineto{\pgfqpoint{2.889672in}{2.431230in}}%
\pgfpathlineto{\pgfqpoint{2.903280in}{2.420585in}}%
\pgfpathlineto{\pgfqpoint{2.916890in}{2.409971in}}%
\pgfpathlineto{\pgfqpoint{2.907813in}{2.432633in}}%
\pgfpathlineto{\pgfqpoint{2.898702in}{2.456061in}}%
\pgfpathlineto{\pgfqpoint{2.889555in}{2.480270in}}%
\pgfpathlineto{\pgfqpoint{2.880373in}{2.505274in}}%
\pgfpathlineto{\pgfqpoint{2.866701in}{2.516355in}}%
\pgfpathlineto{\pgfqpoint{2.853031in}{2.527468in}}%
\pgfpathlineto{\pgfqpoint{2.839363in}{2.538613in}}%
\pgfpathlineto{\pgfqpoint{2.825697in}{2.549792in}}%
\pgfpathlineto{\pgfqpoint{2.834944in}{2.524312in}}%
\pgfpathlineto{\pgfqpoint{2.844153in}{2.499634in}}%
\pgfpathlineto{\pgfqpoint{2.853325in}{2.475741in}}%
\pgfpathlineto{\pgfqpoint{2.862462in}{2.452618in}}%
\pgfpathclose%
\pgfusepath{fill}%
\end{pgfscope}%
\begin{pgfscope}%
\pgfpathrectangle{\pgfqpoint{1.254980in}{0.150000in}}{\pgfqpoint{5.490039in}{5.490039in}}%
\pgfusepath{clip}%
\pgfsetbuttcap%
\pgfsetroundjoin%
\definecolor{currentfill}{rgb}{0.280267,0.073417,0.397163}%
\pgfsetfillcolor{currentfill}%
\pgfsetfillopacity{0.700000}%
\pgfsetlinewidth{0.000000pt}%
\definecolor{currentstroke}{rgb}{0.000000,0.000000,0.000000}%
\pgfsetstrokecolor{currentstroke}%
\pgfsetdash{}{0pt}%
\pgfpathmoveto{\pgfqpoint{4.645194in}{1.312359in}}%
\pgfpathlineto{\pgfqpoint{4.659054in}{1.307021in}}%
\pgfpathlineto{\pgfqpoint{4.672919in}{1.301706in}}%
\pgfpathlineto{\pgfqpoint{4.686791in}{1.296412in}}%
\pgfpathlineto{\pgfqpoint{4.700670in}{1.291142in}}%
\pgfpathlineto{\pgfqpoint{4.693049in}{1.289065in}}%
\pgfpathlineto{\pgfqpoint{4.685425in}{1.287357in}}%
\pgfpathlineto{\pgfqpoint{4.677797in}{1.286025in}}%
\pgfpathlineto{\pgfqpoint{4.670167in}{1.285081in}}%
\pgfpathlineto{\pgfqpoint{4.656270in}{1.290708in}}%
\pgfpathlineto{\pgfqpoint{4.642379in}{1.296359in}}%
\pgfpathlineto{\pgfqpoint{4.628495in}{1.302031in}}%
\pgfpathlineto{\pgfqpoint{4.614616in}{1.307726in}}%
\pgfpathlineto{\pgfqpoint{4.622266in}{1.308309in}}%
\pgfpathlineto{\pgfqpoint{4.629913in}{1.309281in}}%
\pgfpathlineto{\pgfqpoint{4.637555in}{1.310635in}}%
\pgfpathlineto{\pgfqpoint{4.645194in}{1.312359in}}%
\pgfpathclose%
\pgfusepath{fill}%
\end{pgfscope}%
\begin{pgfscope}%
\pgfpathrectangle{\pgfqpoint{1.254980in}{0.150000in}}{\pgfqpoint{5.490039in}{5.490039in}}%
\pgfusepath{clip}%
\pgfsetbuttcap%
\pgfsetroundjoin%
\definecolor{currentfill}{rgb}{0.555484,0.840254,0.269281}%
\pgfsetfillcolor{currentfill}%
\pgfsetfillopacity{0.700000}%
\pgfsetlinewidth{0.000000pt}%
\definecolor{currentstroke}{rgb}{0.000000,0.000000,0.000000}%
\pgfsetstrokecolor{currentstroke}%
\pgfsetdash{}{0pt}%
\pgfpathmoveto{\pgfqpoint{2.007359in}{3.291370in}}%
\pgfpathlineto{\pgfqpoint{2.020999in}{3.277651in}}%
\pgfpathlineto{\pgfqpoint{2.034638in}{3.263987in}}%
\pgfpathlineto{\pgfqpoint{2.048275in}{3.250377in}}%
\pgfpathlineto{\pgfqpoint{2.061912in}{3.236822in}}%
\pgfpathlineto{\pgfqpoint{2.051615in}{3.270006in}}%
\pgfpathlineto{\pgfqpoint{2.041261in}{3.304100in}}%
\pgfpathlineto{\pgfqpoint{2.030848in}{3.339119in}}%
\pgfpathlineto{\pgfqpoint{2.020377in}{3.375081in}}%
\pgfpathlineto{\pgfqpoint{2.006658in}{3.389151in}}%
\pgfpathlineto{\pgfqpoint{1.992938in}{3.403276in}}%
\pgfpathlineto{\pgfqpoint{1.979217in}{3.417456in}}%
\pgfpathlineto{\pgfqpoint{1.965494in}{3.431691in}}%
\pgfpathlineto{\pgfqpoint{1.976050in}{3.395204in}}%
\pgfpathlineto{\pgfqpoint{1.986546in}{3.359666in}}%
\pgfpathlineto{\pgfqpoint{1.996981in}{3.325060in}}%
\pgfpathlineto{\pgfqpoint{2.007359in}{3.291370in}}%
\pgfpathclose%
\pgfusepath{fill}%
\end{pgfscope}%
\begin{pgfscope}%
\pgfpathrectangle{\pgfqpoint{1.254980in}{0.150000in}}{\pgfqpoint{5.490039in}{5.490039in}}%
\pgfusepath{clip}%
\pgfsetbuttcap%
\pgfsetroundjoin%
\definecolor{currentfill}{rgb}{0.190631,0.407061,0.556089}%
\pgfsetfillcolor{currentfill}%
\pgfsetfillopacity{0.700000}%
\pgfsetlinewidth{0.000000pt}%
\definecolor{currentstroke}{rgb}{0.000000,0.000000,0.000000}%
\pgfsetstrokecolor{currentstroke}%
\pgfsetdash{}{0pt}%
\pgfpathmoveto{\pgfqpoint{3.387914in}{2.017265in}}%
\pgfpathlineto{\pgfqpoint{3.401557in}{2.008064in}}%
\pgfpathlineto{\pgfqpoint{3.415203in}{1.998889in}}%
\pgfpathlineto{\pgfqpoint{3.428853in}{1.989742in}}%
\pgfpathlineto{\pgfqpoint{3.442506in}{1.980621in}}%
\pgfpathlineto{\pgfqpoint{3.434024in}{1.996431in}}%
\pgfpathlineto{\pgfqpoint{3.425519in}{2.012909in}}%
\pgfpathlineto{\pgfqpoint{3.416991in}{2.030069in}}%
\pgfpathlineto{\pgfqpoint{3.408439in}{2.047925in}}%
\pgfpathlineto{\pgfqpoint{3.394737in}{2.057486in}}%
\pgfpathlineto{\pgfqpoint{3.381038in}{2.067074in}}%
\pgfpathlineto{\pgfqpoint{3.367342in}{2.076689in}}%
\pgfpathlineto{\pgfqpoint{3.353649in}{2.086331in}}%
\pgfpathlineto{\pgfqpoint{3.362252in}{2.068028in}}%
\pgfpathlineto{\pgfqpoint{3.370830in}{2.050426in}}%
\pgfpathlineto{\pgfqpoint{3.379384in}{2.033509in}}%
\pgfpathlineto{\pgfqpoint{3.387914in}{2.017265in}}%
\pgfpathclose%
\pgfusepath{fill}%
\end{pgfscope}%
\begin{pgfscope}%
\pgfpathrectangle{\pgfqpoint{1.254980in}{0.150000in}}{\pgfqpoint{5.490039in}{5.490039in}}%
\pgfusepath{clip}%
\pgfsetbuttcap%
\pgfsetroundjoin%
\definecolor{currentfill}{rgb}{0.241237,0.296485,0.539709}%
\pgfsetfillcolor{currentfill}%
\pgfsetfillopacity{0.700000}%
\pgfsetlinewidth{0.000000pt}%
\definecolor{currentstroke}{rgb}{0.000000,0.000000,0.000000}%
\pgfsetstrokecolor{currentstroke}%
\pgfsetdash{}{0pt}%
\pgfpathmoveto{\pgfqpoint{3.749136in}{1.755278in}}%
\pgfpathlineto{\pgfqpoint{3.762823in}{1.747117in}}%
\pgfpathlineto{\pgfqpoint{3.776513in}{1.738981in}}%
\pgfpathlineto{\pgfqpoint{3.790208in}{1.730870in}}%
\pgfpathlineto{\pgfqpoint{3.803906in}{1.722783in}}%
\pgfpathlineto{\pgfqpoint{3.795760in}{1.733562in}}%
\pgfpathlineto{\pgfqpoint{3.787597in}{1.744932in}}%
\pgfpathlineto{\pgfqpoint{3.779419in}{1.756907in}}%
\pgfpathlineto{\pgfqpoint{3.771223in}{1.769499in}}%
\pgfpathlineto{\pgfqpoint{3.757484in}{1.778005in}}%
\pgfpathlineto{\pgfqpoint{3.743749in}{1.786536in}}%
\pgfpathlineto{\pgfqpoint{3.730018in}{1.795091in}}%
\pgfpathlineto{\pgfqpoint{3.716290in}{1.803672in}}%
\pgfpathlineto{\pgfqpoint{3.724527in}{1.790655in}}%
\pgfpathlineto{\pgfqpoint{3.732747in}{1.778258in}}%
\pgfpathlineto{\pgfqpoint{3.740950in}{1.766470in}}%
\pgfpathlineto{\pgfqpoint{3.749136in}{1.755278in}}%
\pgfpathclose%
\pgfusepath{fill}%
\end{pgfscope}%
\begin{pgfscope}%
\pgfpathrectangle{\pgfqpoint{1.254980in}{0.150000in}}{\pgfqpoint{5.490039in}{5.490039in}}%
\pgfusepath{clip}%
\pgfsetbuttcap%
\pgfsetroundjoin%
\definecolor{currentfill}{rgb}{0.506271,0.828786,0.300362}%
\pgfsetfillcolor{currentfill}%
\pgfsetfillopacity{0.700000}%
\pgfsetlinewidth{0.000000pt}%
\definecolor{currentstroke}{rgb}{0.000000,0.000000,0.000000}%
\pgfsetstrokecolor{currentstroke}%
\pgfsetdash{}{0pt}%
\pgfpathmoveto{\pgfqpoint{2.061912in}{3.236822in}}%
\pgfpathlineto{\pgfqpoint{2.075548in}{3.223321in}}%
\pgfpathlineto{\pgfqpoint{2.089183in}{3.209872in}}%
\pgfpathlineto{\pgfqpoint{2.102817in}{3.196475in}}%
\pgfpathlineto{\pgfqpoint{2.116451in}{3.183131in}}%
\pgfpathlineto{\pgfqpoint{2.106233in}{3.215811in}}%
\pgfpathlineto{\pgfqpoint{2.095959in}{3.249394in}}%
\pgfpathlineto{\pgfqpoint{2.085629in}{3.283897in}}%
\pgfpathlineto{\pgfqpoint{2.075241in}{3.319337in}}%
\pgfpathlineto{\pgfqpoint{2.061526in}{3.333194in}}%
\pgfpathlineto{\pgfqpoint{2.047811in}{3.347103in}}%
\pgfpathlineto{\pgfqpoint{2.034095in}{3.361066in}}%
\pgfpathlineto{\pgfqpoint{2.020377in}{3.375081in}}%
\pgfpathlineto{\pgfqpoint{2.030848in}{3.339119in}}%
\pgfpathlineto{\pgfqpoint{2.041261in}{3.304100in}}%
\pgfpathlineto{\pgfqpoint{2.051615in}{3.270006in}}%
\pgfpathlineto{\pgfqpoint{2.061912in}{3.236822in}}%
\pgfpathclose%
\pgfusepath{fill}%
\end{pgfscope}%
\begin{pgfscope}%
\pgfpathrectangle{\pgfqpoint{1.254980in}{0.150000in}}{\pgfqpoint{5.490039in}{5.490039in}}%
\pgfusepath{clip}%
\pgfsetbuttcap%
\pgfsetroundjoin%
\definecolor{currentfill}{rgb}{0.132444,0.552216,0.553018}%
\pgfsetfillcolor{currentfill}%
\pgfsetfillopacity{0.700000}%
\pgfsetlinewidth{0.000000pt}%
\definecolor{currentstroke}{rgb}{0.000000,0.000000,0.000000}%
\pgfsetstrokecolor{currentstroke}%
\pgfsetdash{}{0pt}%
\pgfpathmoveto{\pgfqpoint{2.916890in}{2.409971in}}%
\pgfpathlineto{\pgfqpoint{2.930502in}{2.399390in}}%
\pgfpathlineto{\pgfqpoint{2.944116in}{2.388841in}}%
\pgfpathlineto{\pgfqpoint{2.957732in}{2.378323in}}%
\pgfpathlineto{\pgfqpoint{2.971351in}{2.367837in}}%
\pgfpathlineto{\pgfqpoint{2.962333in}{2.390040in}}%
\pgfpathlineto{\pgfqpoint{2.953282in}{2.413004in}}%
\pgfpathlineto{\pgfqpoint{2.944197in}{2.436743in}}%
\pgfpathlineto{\pgfqpoint{2.935077in}{2.461273in}}%
\pgfpathlineto{\pgfqpoint{2.921398in}{2.472226in}}%
\pgfpathlineto{\pgfqpoint{2.907721in}{2.483210in}}%
\pgfpathlineto{\pgfqpoint{2.894046in}{2.494226in}}%
\pgfpathlineto{\pgfqpoint{2.880373in}{2.505274in}}%
\pgfpathlineto{\pgfqpoint{2.889555in}{2.480270in}}%
\pgfpathlineto{\pgfqpoint{2.898702in}{2.456061in}}%
\pgfpathlineto{\pgfqpoint{2.907813in}{2.432633in}}%
\pgfpathlineto{\pgfqpoint{2.916890in}{2.409971in}}%
\pgfpathclose%
\pgfusepath{fill}%
\end{pgfscope}%
\begin{pgfscope}%
\pgfpathrectangle{\pgfqpoint{1.254980in}{0.150000in}}{\pgfqpoint{5.490039in}{5.490039in}}%
\pgfusepath{clip}%
\pgfsetbuttcap%
\pgfsetroundjoin%
\definecolor{currentfill}{rgb}{0.274952,0.037752,0.364543}%
\pgfsetfillcolor{currentfill}%
\pgfsetfillopacity{0.700000}%
\pgfsetlinewidth{0.000000pt}%
\definecolor{currentstroke}{rgb}{0.000000,0.000000,0.000000}%
\pgfsetstrokecolor{currentstroke}%
\pgfsetdash{}{0pt}%
\pgfpathmoveto{\pgfqpoint{4.983985in}{1.249355in}}%
\pgfpathlineto{\pgfqpoint{4.997948in}{1.245137in}}%
\pgfpathlineto{\pgfqpoint{5.011917in}{1.240941in}}%
\pgfpathlineto{\pgfqpoint{5.025893in}{1.236767in}}%
\pgfpathlineto{\pgfqpoint{5.018356in}{1.230535in}}%
\pgfpathlineto{\pgfqpoint{5.010818in}{1.224583in}}%
\pgfpathlineto{\pgfqpoint{5.003279in}{1.218918in}}%
\pgfpathlineto{\pgfqpoint{4.995738in}{1.213550in}}%
\pgfpathlineto{\pgfqpoint{4.981751in}{1.218055in}}%
\pgfpathlineto{\pgfqpoint{4.967770in}{1.222581in}}%
\pgfpathlineto{\pgfqpoint{4.953796in}{1.227130in}}%
\pgfpathlineto{\pgfqpoint{4.961345in}{1.232246in}}%
\pgfpathlineto{\pgfqpoint{4.968893in}{1.237661in}}%
\pgfpathlineto{\pgfqpoint{4.976440in}{1.243367in}}%
\pgfpathlineto{\pgfqpoint{4.983985in}{1.249355in}}%
\pgfpathclose%
\pgfusepath{fill}%
\end{pgfscope}%
\begin{pgfscope}%
\pgfpathrectangle{\pgfqpoint{1.254980in}{0.150000in}}{\pgfqpoint{5.490039in}{5.490039in}}%
\pgfusepath{clip}%
\pgfsetbuttcap%
\pgfsetroundjoin%
\definecolor{currentfill}{rgb}{0.271828,0.209303,0.504434}%
\pgfsetfillcolor{currentfill}%
\pgfsetfillopacity{0.700000}%
\pgfsetlinewidth{0.000000pt}%
\definecolor{currentstroke}{rgb}{0.000000,0.000000,0.000000}%
\pgfsetstrokecolor{currentstroke}%
\pgfsetdash{}{0pt}%
\pgfpathmoveto{\pgfqpoint{4.055533in}{1.565697in}}%
\pgfpathlineto{\pgfqpoint{4.069272in}{1.558421in}}%
\pgfpathlineto{\pgfqpoint{4.083015in}{1.551169in}}%
\pgfpathlineto{\pgfqpoint{4.096763in}{1.543940in}}%
\pgfpathlineto{\pgfqpoint{4.110515in}{1.536735in}}%
\pgfpathlineto{\pgfqpoint{4.102597in}{1.543284in}}%
\pgfpathlineto{\pgfqpoint{4.094668in}{1.550357in}}%
\pgfpathlineto{\pgfqpoint{4.086728in}{1.557967in}}%
\pgfpathlineto{\pgfqpoint{4.078777in}{1.566125in}}%
\pgfpathlineto{\pgfqpoint{4.064991in}{1.573731in}}%
\pgfpathlineto{\pgfqpoint{4.051210in}{1.581362in}}%
\pgfpathlineto{\pgfqpoint{4.037434in}{1.589015in}}%
\pgfpathlineto{\pgfqpoint{4.023662in}{1.596693in}}%
\pgfpathlineto{\pgfqpoint{4.031647in}{1.588128in}}%
\pgfpathlineto{\pgfqpoint{4.039621in}{1.580114in}}%
\pgfpathlineto{\pgfqpoint{4.047583in}{1.572641in}}%
\pgfpathlineto{\pgfqpoint{4.055533in}{1.565697in}}%
\pgfpathclose%
\pgfusepath{fill}%
\end{pgfscope}%
\begin{pgfscope}%
\pgfpathrectangle{\pgfqpoint{1.254980in}{0.150000in}}{\pgfqpoint{5.490039in}{5.490039in}}%
\pgfusepath{clip}%
\pgfsetbuttcap%
\pgfsetroundjoin%
\definecolor{currentfill}{rgb}{0.458674,0.816363,0.329727}%
\pgfsetfillcolor{currentfill}%
\pgfsetfillopacity{0.700000}%
\pgfsetlinewidth{0.000000pt}%
\definecolor{currentstroke}{rgb}{0.000000,0.000000,0.000000}%
\pgfsetstrokecolor{currentstroke}%
\pgfsetdash{}{0pt}%
\pgfpathmoveto{\pgfqpoint{2.116451in}{3.183131in}}%
\pgfpathlineto{\pgfqpoint{2.130084in}{3.169837in}}%
\pgfpathlineto{\pgfqpoint{2.143716in}{3.156595in}}%
\pgfpathlineto{\pgfqpoint{2.157348in}{3.143402in}}%
\pgfpathlineto{\pgfqpoint{2.170979in}{3.130260in}}%
\pgfpathlineto{\pgfqpoint{2.160839in}{3.162438in}}%
\pgfpathlineto{\pgfqpoint{2.150645in}{3.195513in}}%
\pgfpathlineto{\pgfqpoint{2.140395in}{3.229502in}}%
\pgfpathlineto{\pgfqpoint{2.130089in}{3.264423in}}%
\pgfpathlineto{\pgfqpoint{2.116378in}{3.278076in}}%
\pgfpathlineto{\pgfqpoint{2.102667in}{3.291779in}}%
\pgfpathlineto{\pgfqpoint{2.088954in}{3.305532in}}%
\pgfpathlineto{\pgfqpoint{2.075241in}{3.319337in}}%
\pgfpathlineto{\pgfqpoint{2.085629in}{3.283897in}}%
\pgfpathlineto{\pgfqpoint{2.095959in}{3.249394in}}%
\pgfpathlineto{\pgfqpoint{2.106233in}{3.215811in}}%
\pgfpathlineto{\pgfqpoint{2.116451in}{3.183131in}}%
\pgfpathclose%
\pgfusepath{fill}%
\end{pgfscope}%
\begin{pgfscope}%
\pgfpathrectangle{\pgfqpoint{1.254980in}{0.150000in}}{\pgfqpoint{5.490039in}{5.490039in}}%
\pgfusepath{clip}%
\pgfsetbuttcap%
\pgfsetroundjoin%
\definecolor{currentfill}{rgb}{0.282910,0.105393,0.426902}%
\pgfsetfillcolor{currentfill}%
\pgfsetfillopacity{0.700000}%
\pgfsetlinewidth{0.000000pt}%
\definecolor{currentstroke}{rgb}{0.000000,0.000000,0.000000}%
\pgfsetstrokecolor{currentstroke}%
\pgfsetdash{}{0pt}%
\pgfpathmoveto{\pgfqpoint{4.503799in}{1.354094in}}%
\pgfpathlineto{\pgfqpoint{4.517631in}{1.348219in}}%
\pgfpathlineto{\pgfqpoint{4.531468in}{1.342367in}}%
\pgfpathlineto{\pgfqpoint{4.545312in}{1.336537in}}%
\pgfpathlineto{\pgfqpoint{4.559161in}{1.330730in}}%
\pgfpathlineto{\pgfqpoint{4.551486in}{1.330914in}}%
\pgfpathlineto{\pgfqpoint{4.543806in}{1.331511in}}%
\pgfpathlineto{\pgfqpoint{4.536121in}{1.332531in}}%
\pgfpathlineto{\pgfqpoint{4.528432in}{1.333986in}}%
\pgfpathlineto{\pgfqpoint{4.514560in}{1.340164in}}%
\pgfpathlineto{\pgfqpoint{4.500695in}{1.346364in}}%
\pgfpathlineto{\pgfqpoint{4.486835in}{1.352588in}}%
\pgfpathlineto{\pgfqpoint{4.472980in}{1.358833in}}%
\pgfpathlineto{\pgfqpoint{4.480693in}{1.357002in}}%
\pgfpathlineto{\pgfqpoint{4.488400in}{1.355609in}}%
\pgfpathlineto{\pgfqpoint{4.496102in}{1.354643in}}%
\pgfpathlineto{\pgfqpoint{4.503799in}{1.354094in}}%
\pgfpathclose%
\pgfusepath{fill}%
\end{pgfscope}%
\begin{pgfscope}%
\pgfpathrectangle{\pgfqpoint{1.254980in}{0.150000in}}{\pgfqpoint{5.490039in}{5.490039in}}%
\pgfusepath{clip}%
\pgfsetbuttcap%
\pgfsetroundjoin%
\definecolor{currentfill}{rgb}{0.277018,0.050344,0.375715}%
\pgfsetfillcolor{currentfill}%
\pgfsetfillopacity{0.700000}%
\pgfsetlinewidth{0.000000pt}%
\definecolor{currentstroke}{rgb}{0.000000,0.000000,0.000000}%
\pgfsetstrokecolor{currentstroke}%
\pgfsetdash{}{0pt}%
\pgfpathmoveto{\pgfqpoint{4.842247in}{1.264320in}}%
\pgfpathlineto{\pgfqpoint{4.856167in}{1.259593in}}%
\pgfpathlineto{\pgfqpoint{4.870094in}{1.254889in}}%
\pgfpathlineto{\pgfqpoint{4.884028in}{1.250207in}}%
\pgfpathlineto{\pgfqpoint{4.897968in}{1.245547in}}%
\pgfpathlineto{\pgfqpoint{4.890404in}{1.241078in}}%
\pgfpathlineto{\pgfqpoint{4.882838in}{1.236930in}}%
\pgfpathlineto{\pgfqpoint{4.875270in}{1.233111in}}%
\pgfpathlineto{\pgfqpoint{4.867701in}{1.229630in}}%
\pgfpathlineto{\pgfqpoint{4.853746in}{1.234634in}}%
\pgfpathlineto{\pgfqpoint{4.839798in}{1.239660in}}%
\pgfpathlineto{\pgfqpoint{4.825857in}{1.244708in}}%
\pgfpathlineto{\pgfqpoint{4.811922in}{1.249778in}}%
\pgfpathlineto{\pgfqpoint{4.819506in}{1.252910in}}%
\pgfpathlineto{\pgfqpoint{4.827089in}{1.256383in}}%
\pgfpathlineto{\pgfqpoint{4.834669in}{1.260189in}}%
\pgfpathlineto{\pgfqpoint{4.842247in}{1.264320in}}%
\pgfpathclose%
\pgfusepath{fill}%
\end{pgfscope}%
\begin{pgfscope}%
\pgfpathrectangle{\pgfqpoint{1.254980in}{0.150000in}}{\pgfqpoint{5.490039in}{5.490039in}}%
\pgfusepath{clip}%
\pgfsetbuttcap%
\pgfsetroundjoin%
\definecolor{currentfill}{rgb}{0.282290,0.145912,0.461510}%
\pgfsetfillcolor{currentfill}%
\pgfsetfillopacity{0.700000}%
\pgfsetlinewidth{0.000000pt}%
\definecolor{currentstroke}{rgb}{0.000000,0.000000,0.000000}%
\pgfsetstrokecolor{currentstroke}%
\pgfsetdash{}{0pt}%
\pgfpathmoveto{\pgfqpoint{4.307154in}{1.435550in}}%
\pgfpathlineto{\pgfqpoint{4.320943in}{1.429031in}}%
\pgfpathlineto{\pgfqpoint{4.334737in}{1.422536in}}%
\pgfpathlineto{\pgfqpoint{4.348537in}{1.416063in}}%
\pgfpathlineto{\pgfqpoint{4.362343in}{1.409614in}}%
\pgfpathlineto{\pgfqpoint{4.354574in}{1.412653in}}%
\pgfpathlineto{\pgfqpoint{4.346798in}{1.416157in}}%
\pgfpathlineto{\pgfqpoint{4.339015in}{1.420139in}}%
\pgfpathlineto{\pgfqpoint{4.331225in}{1.424607in}}%
\pgfpathlineto{\pgfqpoint{4.317393in}{1.431443in}}%
\pgfpathlineto{\pgfqpoint{4.303566in}{1.438301in}}%
\pgfpathlineto{\pgfqpoint{4.289744in}{1.445182in}}%
\pgfpathlineto{\pgfqpoint{4.275927in}{1.452086in}}%
\pgfpathlineto{\pgfqpoint{4.283746in}{1.447226in}}%
\pgfpathlineto{\pgfqpoint{4.291556in}{1.442857in}}%
\pgfpathlineto{\pgfqpoint{4.299359in}{1.438969in}}%
\pgfpathlineto{\pgfqpoint{4.307154in}{1.435550in}}%
\pgfpathclose%
\pgfusepath{fill}%
\end{pgfscope}%
\begin{pgfscope}%
\pgfpathrectangle{\pgfqpoint{1.254980in}{0.150000in}}{\pgfqpoint{5.490039in}{5.490039in}}%
\pgfusepath{clip}%
\pgfsetbuttcap%
\pgfsetroundjoin%
\definecolor{currentfill}{rgb}{0.412913,0.803041,0.357269}%
\pgfsetfillcolor{currentfill}%
\pgfsetfillopacity{0.700000}%
\pgfsetlinewidth{0.000000pt}%
\definecolor{currentstroke}{rgb}{0.000000,0.000000,0.000000}%
\pgfsetstrokecolor{currentstroke}%
\pgfsetdash{}{0pt}%
\pgfpathmoveto{\pgfqpoint{2.170979in}{3.130260in}}%
\pgfpathlineto{\pgfqpoint{2.184610in}{3.117166in}}%
\pgfpathlineto{\pgfqpoint{2.198240in}{3.104122in}}%
\pgfpathlineto{\pgfqpoint{2.211871in}{3.091125in}}%
\pgfpathlineto{\pgfqpoint{2.225500in}{3.078177in}}%
\pgfpathlineto{\pgfqpoint{2.215437in}{3.109854in}}%
\pgfpathlineto{\pgfqpoint{2.205321in}{3.142424in}}%
\pgfpathlineto{\pgfqpoint{2.195152in}{3.175902in}}%
\pgfpathlineto{\pgfqpoint{2.184927in}{3.210305in}}%
\pgfpathlineto{\pgfqpoint{2.171218in}{3.223762in}}%
\pgfpathlineto{\pgfqpoint{2.157509in}{3.237266in}}%
\pgfpathlineto{\pgfqpoint{2.143799in}{3.250820in}}%
\pgfpathlineto{\pgfqpoint{2.130089in}{3.264423in}}%
\pgfpathlineto{\pgfqpoint{2.140395in}{3.229502in}}%
\pgfpathlineto{\pgfqpoint{2.150645in}{3.195513in}}%
\pgfpathlineto{\pgfqpoint{2.160839in}{3.162438in}}%
\pgfpathlineto{\pgfqpoint{2.170979in}{3.130260in}}%
\pgfpathclose%
\pgfusepath{fill}%
\end{pgfscope}%
\begin{pgfscope}%
\pgfpathrectangle{\pgfqpoint{1.254980in}{0.150000in}}{\pgfqpoint{5.490039in}{5.490039in}}%
\pgfusepath{clip}%
\pgfsetbuttcap%
\pgfsetroundjoin%
\definecolor{currentfill}{rgb}{0.136408,0.541173,0.554483}%
\pgfsetfillcolor{currentfill}%
\pgfsetfillopacity{0.700000}%
\pgfsetlinewidth{0.000000pt}%
\definecolor{currentstroke}{rgb}{0.000000,0.000000,0.000000}%
\pgfsetstrokecolor{currentstroke}%
\pgfsetdash{}{0pt}%
\pgfpathmoveto{\pgfqpoint{2.971351in}{2.367837in}}%
\pgfpathlineto{\pgfqpoint{2.984971in}{2.357382in}}%
\pgfpathlineto{\pgfqpoint{2.998594in}{2.346958in}}%
\pgfpathlineto{\pgfqpoint{3.012218in}{2.336565in}}%
\pgfpathlineto{\pgfqpoint{3.025845in}{2.326203in}}%
\pgfpathlineto{\pgfqpoint{3.016886in}{2.347948in}}%
\pgfpathlineto{\pgfqpoint{3.007895in}{2.370449in}}%
\pgfpathlineto{\pgfqpoint{2.998870in}{2.393720in}}%
\pgfpathlineto{\pgfqpoint{2.989812in}{2.417778in}}%
\pgfpathlineto{\pgfqpoint{2.976125in}{2.428605in}}%
\pgfpathlineto{\pgfqpoint{2.962440in}{2.439463in}}%
\pgfpathlineto{\pgfqpoint{2.948757in}{2.450353in}}%
\pgfpathlineto{\pgfqpoint{2.935077in}{2.461273in}}%
\pgfpathlineto{\pgfqpoint{2.944197in}{2.436743in}}%
\pgfpathlineto{\pgfqpoint{2.953282in}{2.413004in}}%
\pgfpathlineto{\pgfqpoint{2.962333in}{2.390040in}}%
\pgfpathlineto{\pgfqpoint{2.971351in}{2.367837in}}%
\pgfpathclose%
\pgfusepath{fill}%
\end{pgfscope}%
\begin{pgfscope}%
\pgfpathrectangle{\pgfqpoint{1.254980in}{0.150000in}}{\pgfqpoint{5.490039in}{5.490039in}}%
\pgfusepath{clip}%
\pgfsetbuttcap%
\pgfsetroundjoin%
\definecolor{currentfill}{rgb}{0.195860,0.395433,0.555276}%
\pgfsetfillcolor{currentfill}%
\pgfsetfillopacity{0.700000}%
\pgfsetlinewidth{0.000000pt}%
\definecolor{currentstroke}{rgb}{0.000000,0.000000,0.000000}%
\pgfsetstrokecolor{currentstroke}%
\pgfsetdash{}{0pt}%
\pgfpathmoveto{\pgfqpoint{3.442506in}{1.980621in}}%
\pgfpathlineto{\pgfqpoint{3.456162in}{1.971526in}}%
\pgfpathlineto{\pgfqpoint{3.469822in}{1.962458in}}%
\pgfpathlineto{\pgfqpoint{3.483485in}{1.953417in}}%
\pgfpathlineto{\pgfqpoint{3.497151in}{1.944402in}}%
\pgfpathlineto{\pgfqpoint{3.488716in}{1.959779in}}%
\pgfpathlineto{\pgfqpoint{3.480259in}{1.975820in}}%
\pgfpathlineto{\pgfqpoint{3.471780in}{1.992538in}}%
\pgfpathlineto{\pgfqpoint{3.463278in}{2.009947in}}%
\pgfpathlineto{\pgfqpoint{3.449564in}{2.019402in}}%
\pgfpathlineto{\pgfqpoint{3.435853in}{2.028883in}}%
\pgfpathlineto{\pgfqpoint{3.422144in}{2.038390in}}%
\pgfpathlineto{\pgfqpoint{3.408439in}{2.047925in}}%
\pgfpathlineto{\pgfqpoint{3.416991in}{2.030069in}}%
\pgfpathlineto{\pgfqpoint{3.425519in}{2.012909in}}%
\pgfpathlineto{\pgfqpoint{3.434024in}{1.996431in}}%
\pgfpathlineto{\pgfqpoint{3.442506in}{1.980621in}}%
\pgfpathclose%
\pgfusepath{fill}%
\end{pgfscope}%
\begin{pgfscope}%
\pgfpathrectangle{\pgfqpoint{1.254980in}{0.150000in}}{\pgfqpoint{5.490039in}{5.490039in}}%
\pgfusepath{clip}%
\pgfsetbuttcap%
\pgfsetroundjoin%
\definecolor{currentfill}{rgb}{0.280267,0.073417,0.397163}%
\pgfsetfillcolor{currentfill}%
\pgfsetfillopacity{0.700000}%
\pgfsetlinewidth{0.000000pt}%
\definecolor{currentstroke}{rgb}{0.000000,0.000000,0.000000}%
\pgfsetstrokecolor{currentstroke}%
\pgfsetdash{}{0pt}%
\pgfpathmoveto{\pgfqpoint{4.700670in}{1.291142in}}%
\pgfpathlineto{\pgfqpoint{4.714554in}{1.285893in}}%
\pgfpathlineto{\pgfqpoint{4.728445in}{1.280667in}}%
\pgfpathlineto{\pgfqpoint{4.742342in}{1.275463in}}%
\pgfpathlineto{\pgfqpoint{4.756245in}{1.270282in}}%
\pgfpathlineto{\pgfqpoint{4.748641in}{1.267853in}}%
\pgfpathlineto{\pgfqpoint{4.741035in}{1.265789in}}%
\pgfpathlineto{\pgfqpoint{4.733426in}{1.264099in}}%
\pgfpathlineto{\pgfqpoint{4.725814in}{1.262792in}}%
\pgfpathlineto{\pgfqpoint{4.711893in}{1.268331in}}%
\pgfpathlineto{\pgfqpoint{4.697978in}{1.273892in}}%
\pgfpathlineto{\pgfqpoint{4.684069in}{1.279475in}}%
\pgfpathlineto{\pgfqpoint{4.670167in}{1.285081in}}%
\pgfpathlineto{\pgfqpoint{4.677797in}{1.286025in}}%
\pgfpathlineto{\pgfqpoint{4.685425in}{1.287357in}}%
\pgfpathlineto{\pgfqpoint{4.693049in}{1.289065in}}%
\pgfpathlineto{\pgfqpoint{4.700670in}{1.291142in}}%
\pgfpathclose%
\pgfusepath{fill}%
\end{pgfscope}%
\begin{pgfscope}%
\pgfpathrectangle{\pgfqpoint{1.254980in}{0.150000in}}{\pgfqpoint{5.490039in}{5.490039in}}%
\pgfusepath{clip}%
\pgfsetbuttcap%
\pgfsetroundjoin%
\definecolor{currentfill}{rgb}{0.244972,0.287675,0.537260}%
\pgfsetfillcolor{currentfill}%
\pgfsetfillopacity{0.700000}%
\pgfsetlinewidth{0.000000pt}%
\definecolor{currentstroke}{rgb}{0.000000,0.000000,0.000000}%
\pgfsetstrokecolor{currentstroke}%
\pgfsetdash{}{0pt}%
\pgfpathmoveto{\pgfqpoint{3.803906in}{1.722783in}}%
\pgfpathlineto{\pgfqpoint{3.817609in}{1.714721in}}%
\pgfpathlineto{\pgfqpoint{3.831316in}{1.706683in}}%
\pgfpathlineto{\pgfqpoint{3.845027in}{1.698670in}}%
\pgfpathlineto{\pgfqpoint{3.858743in}{1.690681in}}%
\pgfpathlineto{\pgfqpoint{3.850634in}{1.701047in}}%
\pgfpathlineto{\pgfqpoint{3.842511in}{1.712000in}}%
\pgfpathlineto{\pgfqpoint{3.834373in}{1.723554in}}%
\pgfpathlineto{\pgfqpoint{3.826219in}{1.735720in}}%
\pgfpathlineto{\pgfqpoint{3.812464in}{1.744128in}}%
\pgfpathlineto{\pgfqpoint{3.798713in}{1.752560in}}%
\pgfpathlineto{\pgfqpoint{3.784966in}{1.761017in}}%
\pgfpathlineto{\pgfqpoint{3.771223in}{1.769499in}}%
\pgfpathlineto{\pgfqpoint{3.779419in}{1.756907in}}%
\pgfpathlineto{\pgfqpoint{3.787597in}{1.744932in}}%
\pgfpathlineto{\pgfqpoint{3.795760in}{1.733562in}}%
\pgfpathlineto{\pgfqpoint{3.803906in}{1.722783in}}%
\pgfpathclose%
\pgfusepath{fill}%
\end{pgfscope}%
\begin{pgfscope}%
\pgfpathrectangle{\pgfqpoint{1.254980in}{0.150000in}}{\pgfqpoint{5.490039in}{5.490039in}}%
\pgfusepath{clip}%
\pgfsetbuttcap%
\pgfsetroundjoin%
\definecolor{currentfill}{rgb}{0.369214,0.788888,0.382914}%
\pgfsetfillcolor{currentfill}%
\pgfsetfillopacity{0.700000}%
\pgfsetlinewidth{0.000000pt}%
\definecolor{currentstroke}{rgb}{0.000000,0.000000,0.000000}%
\pgfsetstrokecolor{currentstroke}%
\pgfsetdash{}{0pt}%
\pgfpathmoveto{\pgfqpoint{2.225500in}{3.078177in}}%
\pgfpathlineto{\pgfqpoint{2.239130in}{3.065275in}}%
\pgfpathlineto{\pgfqpoint{2.252760in}{3.052421in}}%
\pgfpathlineto{\pgfqpoint{2.266389in}{3.039612in}}%
\pgfpathlineto{\pgfqpoint{2.280019in}{3.026850in}}%
\pgfpathlineto{\pgfqpoint{2.270031in}{3.058030in}}%
\pgfpathlineto{\pgfqpoint{2.259993in}{3.090096in}}%
\pgfpathlineto{\pgfqpoint{2.249902in}{3.123065in}}%
\pgfpathlineto{\pgfqpoint{2.239757in}{3.156953in}}%
\pgfpathlineto{\pgfqpoint{2.226050in}{3.170221in}}%
\pgfpathlineto{\pgfqpoint{2.212342in}{3.183535in}}%
\pgfpathlineto{\pgfqpoint{2.198635in}{3.196896in}}%
\pgfpathlineto{\pgfqpoint{2.184927in}{3.210305in}}%
\pgfpathlineto{\pgfqpoint{2.195152in}{3.175902in}}%
\pgfpathlineto{\pgfqpoint{2.205321in}{3.142424in}}%
\pgfpathlineto{\pgfqpoint{2.215437in}{3.109854in}}%
\pgfpathlineto{\pgfqpoint{2.225500in}{3.078177in}}%
\pgfpathclose%
\pgfusepath{fill}%
\end{pgfscope}%
\begin{pgfscope}%
\pgfpathrectangle{\pgfqpoint{1.254980in}{0.150000in}}{\pgfqpoint{5.490039in}{5.490039in}}%
\pgfusepath{clip}%
\pgfsetbuttcap%
\pgfsetroundjoin%
\definecolor{currentfill}{rgb}{0.327796,0.773980,0.406640}%
\pgfsetfillcolor{currentfill}%
\pgfsetfillopacity{0.700000}%
\pgfsetlinewidth{0.000000pt}%
\definecolor{currentstroke}{rgb}{0.000000,0.000000,0.000000}%
\pgfsetstrokecolor{currentstroke}%
\pgfsetdash{}{0pt}%
\pgfpathmoveto{\pgfqpoint{2.280019in}{3.026850in}}%
\pgfpathlineto{\pgfqpoint{2.293648in}{3.014133in}}%
\pgfpathlineto{\pgfqpoint{2.307278in}{3.001462in}}%
\pgfpathlineto{\pgfqpoint{2.320907in}{2.988835in}}%
\pgfpathlineto{\pgfqpoint{2.334537in}{2.976252in}}%
\pgfpathlineto{\pgfqpoint{2.324625in}{3.006936in}}%
\pgfpathlineto{\pgfqpoint{2.314662in}{3.038500in}}%
\pgfpathlineto{\pgfqpoint{2.304649in}{3.070962in}}%
\pgfpathlineto{\pgfqpoint{2.294583in}{3.104336in}}%
\pgfpathlineto{\pgfqpoint{2.280876in}{3.117423in}}%
\pgfpathlineto{\pgfqpoint{2.267170in}{3.130554in}}%
\pgfpathlineto{\pgfqpoint{2.253463in}{3.143731in}}%
\pgfpathlineto{\pgfqpoint{2.239757in}{3.156953in}}%
\pgfpathlineto{\pgfqpoint{2.249902in}{3.123065in}}%
\pgfpathlineto{\pgfqpoint{2.259993in}{3.090096in}}%
\pgfpathlineto{\pgfqpoint{2.270031in}{3.058030in}}%
\pgfpathlineto{\pgfqpoint{2.280019in}{3.026850in}}%
\pgfpathclose%
\pgfusepath{fill}%
\end{pgfscope}%
\begin{pgfscope}%
\pgfpathrectangle{\pgfqpoint{1.254980in}{0.150000in}}{\pgfqpoint{5.490039in}{5.490039in}}%
\pgfusepath{clip}%
\pgfsetbuttcap%
\pgfsetroundjoin%
\definecolor{currentfill}{rgb}{0.141935,0.526453,0.555991}%
\pgfsetfillcolor{currentfill}%
\pgfsetfillopacity{0.700000}%
\pgfsetlinewidth{0.000000pt}%
\definecolor{currentstroke}{rgb}{0.000000,0.000000,0.000000}%
\pgfsetstrokecolor{currentstroke}%
\pgfsetdash{}{0pt}%
\pgfpathmoveto{\pgfqpoint{3.025845in}{2.326203in}}%
\pgfpathlineto{\pgfqpoint{3.039475in}{2.315871in}}%
\pgfpathlineto{\pgfqpoint{3.053107in}{2.305570in}}%
\pgfpathlineto{\pgfqpoint{3.066741in}{2.295299in}}%
\pgfpathlineto{\pgfqpoint{3.080377in}{2.285058in}}%
\pgfpathlineto{\pgfqpoint{3.071475in}{2.306346in}}%
\pgfpathlineto{\pgfqpoint{3.062542in}{2.328385in}}%
\pgfpathlineto{\pgfqpoint{3.053577in}{2.351190in}}%
\pgfpathlineto{\pgfqpoint{3.044580in}{2.374775in}}%
\pgfpathlineto{\pgfqpoint{3.030884in}{2.385480in}}%
\pgfpathlineto{\pgfqpoint{3.017191in}{2.396215in}}%
\pgfpathlineto{\pgfqpoint{3.003500in}{2.406981in}}%
\pgfpathlineto{\pgfqpoint{2.989812in}{2.417778in}}%
\pgfpathlineto{\pgfqpoint{2.998870in}{2.393720in}}%
\pgfpathlineto{\pgfqpoint{3.007895in}{2.370449in}}%
\pgfpathlineto{\pgfqpoint{3.016886in}{2.347948in}}%
\pgfpathlineto{\pgfqpoint{3.025845in}{2.326203in}}%
\pgfpathclose%
\pgfusepath{fill}%
\end{pgfscope}%
\begin{pgfscope}%
\pgfpathrectangle{\pgfqpoint{1.254980in}{0.150000in}}{\pgfqpoint{5.490039in}{5.490039in}}%
\pgfusepath{clip}%
\pgfsetbuttcap%
\pgfsetroundjoin%
\definecolor{currentfill}{rgb}{0.273006,0.204520,0.501721}%
\pgfsetfillcolor{currentfill}%
\pgfsetfillopacity{0.700000}%
\pgfsetlinewidth{0.000000pt}%
\definecolor{currentstroke}{rgb}{0.000000,0.000000,0.000000}%
\pgfsetstrokecolor{currentstroke}%
\pgfsetdash{}{0pt}%
\pgfpathmoveto{\pgfqpoint{4.110515in}{1.536735in}}%
\pgfpathlineto{\pgfqpoint{4.124273in}{1.529553in}}%
\pgfpathlineto{\pgfqpoint{4.138035in}{1.522395in}}%
\pgfpathlineto{\pgfqpoint{4.151802in}{1.515260in}}%
\pgfpathlineto{\pgfqpoint{4.165574in}{1.508148in}}%
\pgfpathlineto{\pgfqpoint{4.157687in}{1.514301in}}%
\pgfpathlineto{\pgfqpoint{4.149790in}{1.520975in}}%
\pgfpathlineto{\pgfqpoint{4.141883in}{1.528181in}}%
\pgfpathlineto{\pgfqpoint{4.133965in}{1.535931in}}%
\pgfpathlineto{\pgfqpoint{4.120161in}{1.543445in}}%
\pgfpathlineto{\pgfqpoint{4.106362in}{1.550981in}}%
\pgfpathlineto{\pgfqpoint{4.092567in}{1.558541in}}%
\pgfpathlineto{\pgfqpoint{4.078777in}{1.566125in}}%
\pgfpathlineto{\pgfqpoint{4.086728in}{1.557967in}}%
\pgfpathlineto{\pgfqpoint{4.094668in}{1.550357in}}%
\pgfpathlineto{\pgfqpoint{4.102597in}{1.543284in}}%
\pgfpathlineto{\pgfqpoint{4.110515in}{1.536735in}}%
\pgfpathclose%
\pgfusepath{fill}%
\end{pgfscope}%
\begin{pgfscope}%
\pgfpathrectangle{\pgfqpoint{1.254980in}{0.150000in}}{\pgfqpoint{5.490039in}{5.490039in}}%
\pgfusepath{clip}%
\pgfsetbuttcap%
\pgfsetroundjoin%
\definecolor{currentfill}{rgb}{0.201239,0.383670,0.554294}%
\pgfsetfillcolor{currentfill}%
\pgfsetfillopacity{0.700000}%
\pgfsetlinewidth{0.000000pt}%
\definecolor{currentstroke}{rgb}{0.000000,0.000000,0.000000}%
\pgfsetstrokecolor{currentstroke}%
\pgfsetdash{}{0pt}%
\pgfpathmoveto{\pgfqpoint{3.497151in}{1.944402in}}%
\pgfpathlineto{\pgfqpoint{3.510820in}{1.935413in}}%
\pgfpathlineto{\pgfqpoint{3.524494in}{1.926451in}}%
\pgfpathlineto{\pgfqpoint{3.538170in}{1.917514in}}%
\pgfpathlineto{\pgfqpoint{3.551850in}{1.908604in}}%
\pgfpathlineto{\pgfqpoint{3.543462in}{1.923548in}}%
\pgfpathlineto{\pgfqpoint{3.535052in}{1.939153in}}%
\pgfpathlineto{\pgfqpoint{3.526621in}{1.955430in}}%
\pgfpathlineto{\pgfqpoint{3.518169in}{1.972393in}}%
\pgfpathlineto{\pgfqpoint{3.504441in}{1.981742in}}%
\pgfpathlineto{\pgfqpoint{3.490717in}{1.991118in}}%
\pgfpathlineto{\pgfqpoint{3.476996in}{2.000519in}}%
\pgfpathlineto{\pgfqpoint{3.463278in}{2.009947in}}%
\pgfpathlineto{\pgfqpoint{3.471780in}{1.992538in}}%
\pgfpathlineto{\pgfqpoint{3.480259in}{1.975820in}}%
\pgfpathlineto{\pgfqpoint{3.488716in}{1.959779in}}%
\pgfpathlineto{\pgfqpoint{3.497151in}{1.944402in}}%
\pgfpathclose%
\pgfusepath{fill}%
\end{pgfscope}%
\begin{pgfscope}%
\pgfpathrectangle{\pgfqpoint{1.254980in}{0.150000in}}{\pgfqpoint{5.490039in}{5.490039in}}%
\pgfusepath{clip}%
\pgfsetbuttcap%
\pgfsetroundjoin%
\definecolor{currentfill}{rgb}{0.282656,0.100196,0.422160}%
\pgfsetfillcolor{currentfill}%
\pgfsetfillopacity{0.700000}%
\pgfsetlinewidth{0.000000pt}%
\definecolor{currentstroke}{rgb}{0.000000,0.000000,0.000000}%
\pgfsetstrokecolor{currentstroke}%
\pgfsetdash{}{0pt}%
\pgfpathmoveto{\pgfqpoint{4.559161in}{1.330730in}}%
\pgfpathlineto{\pgfqpoint{4.573016in}{1.324946in}}%
\pgfpathlineto{\pgfqpoint{4.586877in}{1.319183in}}%
\pgfpathlineto{\pgfqpoint{4.600743in}{1.313444in}}%
\pgfpathlineto{\pgfqpoint{4.614616in}{1.307726in}}%
\pgfpathlineto{\pgfqpoint{4.606962in}{1.307544in}}%
\pgfpathlineto{\pgfqpoint{4.599303in}{1.307772in}}%
\pgfpathlineto{\pgfqpoint{4.591641in}{1.308419in}}%
\pgfpathlineto{\pgfqpoint{4.583974in}{1.309497in}}%
\pgfpathlineto{\pgfqpoint{4.570080in}{1.315586in}}%
\pgfpathlineto{\pgfqpoint{4.556191in}{1.321697in}}%
\pgfpathlineto{\pgfqpoint{4.542309in}{1.327830in}}%
\pgfpathlineto{\pgfqpoint{4.528432in}{1.333986in}}%
\pgfpathlineto{\pgfqpoint{4.536121in}{1.332531in}}%
\pgfpathlineto{\pgfqpoint{4.543806in}{1.331511in}}%
\pgfpathlineto{\pgfqpoint{4.551486in}{1.330914in}}%
\pgfpathlineto{\pgfqpoint{4.559161in}{1.330730in}}%
\pgfpathclose%
\pgfusepath{fill}%
\end{pgfscope}%
\begin{pgfscope}%
\pgfpathrectangle{\pgfqpoint{1.254980in}{0.150000in}}{\pgfqpoint{5.490039in}{5.490039in}}%
\pgfusepath{clip}%
\pgfsetbuttcap%
\pgfsetroundjoin%
\definecolor{currentfill}{rgb}{0.296479,0.761561,0.424223}%
\pgfsetfillcolor{currentfill}%
\pgfsetfillopacity{0.700000}%
\pgfsetlinewidth{0.000000pt}%
\definecolor{currentstroke}{rgb}{0.000000,0.000000,0.000000}%
\pgfsetstrokecolor{currentstroke}%
\pgfsetdash{}{0pt}%
\pgfpathmoveto{\pgfqpoint{2.334537in}{2.976252in}}%
\pgfpathlineto{\pgfqpoint{2.348167in}{2.963713in}}%
\pgfpathlineto{\pgfqpoint{2.361797in}{2.951218in}}%
\pgfpathlineto{\pgfqpoint{2.375428in}{2.938765in}}%
\pgfpathlineto{\pgfqpoint{2.389059in}{2.926355in}}%
\pgfpathlineto{\pgfqpoint{2.379221in}{2.956545in}}%
\pgfpathlineto{\pgfqpoint{2.369333in}{2.987610in}}%
\pgfpathlineto{\pgfqpoint{2.359396in}{3.019566in}}%
\pgfpathlineto{\pgfqpoint{2.349408in}{3.052430in}}%
\pgfpathlineto{\pgfqpoint{2.335701in}{3.065341in}}%
\pgfpathlineto{\pgfqpoint{2.321995in}{3.078296in}}%
\pgfpathlineto{\pgfqpoint{2.308289in}{3.091294in}}%
\pgfpathlineto{\pgfqpoint{2.294583in}{3.104336in}}%
\pgfpathlineto{\pgfqpoint{2.304649in}{3.070962in}}%
\pgfpathlineto{\pgfqpoint{2.314662in}{3.038500in}}%
\pgfpathlineto{\pgfqpoint{2.324625in}{3.006936in}}%
\pgfpathlineto{\pgfqpoint{2.334537in}{2.976252in}}%
\pgfpathclose%
\pgfusepath{fill}%
\end{pgfscope}%
\begin{pgfscope}%
\pgfpathrectangle{\pgfqpoint{1.254980in}{0.150000in}}{\pgfqpoint{5.490039in}{5.490039in}}%
\pgfusepath{clip}%
\pgfsetbuttcap%
\pgfsetroundjoin%
\definecolor{currentfill}{rgb}{0.282623,0.140926,0.457517}%
\pgfsetfillcolor{currentfill}%
\pgfsetfillopacity{0.700000}%
\pgfsetlinewidth{0.000000pt}%
\definecolor{currentstroke}{rgb}{0.000000,0.000000,0.000000}%
\pgfsetstrokecolor{currentstroke}%
\pgfsetdash{}{0pt}%
\pgfpathmoveto{\pgfqpoint{4.362343in}{1.409614in}}%
\pgfpathlineto{\pgfqpoint{4.376153in}{1.403187in}}%
\pgfpathlineto{\pgfqpoint{4.389969in}{1.396782in}}%
\pgfpathlineto{\pgfqpoint{4.403791in}{1.390401in}}%
\pgfpathlineto{\pgfqpoint{4.417618in}{1.384042in}}%
\pgfpathlineto{\pgfqpoint{4.409874in}{1.386701in}}%
\pgfpathlineto{\pgfqpoint{4.402125in}{1.389822in}}%
\pgfpathlineto{\pgfqpoint{4.394369in}{1.393416in}}%
\pgfpathlineto{\pgfqpoint{4.386606in}{1.397493in}}%
\pgfpathlineto{\pgfqpoint{4.372753in}{1.404238in}}%
\pgfpathlineto{\pgfqpoint{4.358905in}{1.411005in}}%
\pgfpathlineto{\pgfqpoint{4.345063in}{1.417795in}}%
\pgfpathlineto{\pgfqpoint{4.331225in}{1.424607in}}%
\pgfpathlineto{\pgfqpoint{4.339015in}{1.420139in}}%
\pgfpathlineto{\pgfqpoint{4.346798in}{1.416157in}}%
\pgfpathlineto{\pgfqpoint{4.354574in}{1.412653in}}%
\pgfpathlineto{\pgfqpoint{4.362343in}{1.409614in}}%
\pgfpathclose%
\pgfusepath{fill}%
\end{pgfscope}%
\begin{pgfscope}%
\pgfpathrectangle{\pgfqpoint{1.254980in}{0.150000in}}{\pgfqpoint{5.490039in}{5.490039in}}%
\pgfusepath{clip}%
\pgfsetbuttcap%
\pgfsetroundjoin%
\definecolor{currentfill}{rgb}{0.277018,0.050344,0.375715}%
\pgfsetfillcolor{currentfill}%
\pgfsetfillopacity{0.700000}%
\pgfsetlinewidth{0.000000pt}%
\definecolor{currentstroke}{rgb}{0.000000,0.000000,0.000000}%
\pgfsetstrokecolor{currentstroke}%
\pgfsetdash{}{0pt}%
\pgfpathmoveto{\pgfqpoint{4.897968in}{1.245547in}}%
\pgfpathlineto{\pgfqpoint{4.911915in}{1.240909in}}%
\pgfpathlineto{\pgfqpoint{4.925868in}{1.236294in}}%
\pgfpathlineto{\pgfqpoint{4.939829in}{1.231701in}}%
\pgfpathlineto{\pgfqpoint{4.953796in}{1.227130in}}%
\pgfpathlineto{\pgfqpoint{4.946245in}{1.222322in}}%
\pgfpathlineto{\pgfqpoint{4.938693in}{1.217831in}}%
\pgfpathlineto{\pgfqpoint{4.931140in}{1.213667in}}%
\pgfpathlineto{\pgfqpoint{4.923585in}{1.209837in}}%
\pgfpathlineto{\pgfqpoint{4.909604in}{1.214752in}}%
\pgfpathlineto{\pgfqpoint{4.895630in}{1.219690in}}%
\pgfpathlineto{\pgfqpoint{4.881662in}{1.224649in}}%
\pgfpathlineto{\pgfqpoint{4.867701in}{1.229630in}}%
\pgfpathlineto{\pgfqpoint{4.875270in}{1.233111in}}%
\pgfpathlineto{\pgfqpoint{4.882838in}{1.236930in}}%
\pgfpathlineto{\pgfqpoint{4.890404in}{1.241078in}}%
\pgfpathlineto{\pgfqpoint{4.897968in}{1.245547in}}%
\pgfpathclose%
\pgfusepath{fill}%
\end{pgfscope}%
\begin{pgfscope}%
\pgfpathrectangle{\pgfqpoint{1.254980in}{0.150000in}}{\pgfqpoint{5.490039in}{5.490039in}}%
\pgfusepath{clip}%
\pgfsetbuttcap%
\pgfsetroundjoin%
\definecolor{currentfill}{rgb}{0.248629,0.278775,0.534556}%
\pgfsetfillcolor{currentfill}%
\pgfsetfillopacity{0.700000}%
\pgfsetlinewidth{0.000000pt}%
\definecolor{currentstroke}{rgb}{0.000000,0.000000,0.000000}%
\pgfsetstrokecolor{currentstroke}%
\pgfsetdash{}{0pt}%
\pgfpathmoveto{\pgfqpoint{3.858743in}{1.690681in}}%
\pgfpathlineto{\pgfqpoint{3.872462in}{1.682717in}}%
\pgfpathlineto{\pgfqpoint{3.886186in}{1.674776in}}%
\pgfpathlineto{\pgfqpoint{3.899914in}{1.666860in}}%
\pgfpathlineto{\pgfqpoint{3.913646in}{1.658968in}}%
\pgfpathlineto{\pgfqpoint{3.905576in}{1.668921in}}%
\pgfpathlineto{\pgfqpoint{3.897491in}{1.679458in}}%
\pgfpathlineto{\pgfqpoint{3.889392in}{1.690591in}}%
\pgfpathlineto{\pgfqpoint{3.881278in}{1.702332in}}%
\pgfpathlineto{\pgfqpoint{3.867507in}{1.710643in}}%
\pgfpathlineto{\pgfqpoint{3.853740in}{1.718977in}}%
\pgfpathlineto{\pgfqpoint{3.839978in}{1.727337in}}%
\pgfpathlineto{\pgfqpoint{3.826219in}{1.735720in}}%
\pgfpathlineto{\pgfqpoint{3.834373in}{1.723554in}}%
\pgfpathlineto{\pgfqpoint{3.842511in}{1.712000in}}%
\pgfpathlineto{\pgfqpoint{3.850634in}{1.701047in}}%
\pgfpathlineto{\pgfqpoint{3.858743in}{1.690681in}}%
\pgfpathclose%
\pgfusepath{fill}%
\end{pgfscope}%
\begin{pgfscope}%
\pgfpathrectangle{\pgfqpoint{1.254980in}{0.150000in}}{\pgfqpoint{5.490039in}{5.490039in}}%
\pgfusepath{clip}%
\pgfsetbuttcap%
\pgfsetroundjoin%
\definecolor{currentfill}{rgb}{0.146180,0.515413,0.556823}%
\pgfsetfillcolor{currentfill}%
\pgfsetfillopacity{0.700000}%
\pgfsetlinewidth{0.000000pt}%
\definecolor{currentstroke}{rgb}{0.000000,0.000000,0.000000}%
\pgfsetstrokecolor{currentstroke}%
\pgfsetdash{}{0pt}%
\pgfpathmoveto{\pgfqpoint{3.080377in}{2.285058in}}%
\pgfpathlineto{\pgfqpoint{3.094016in}{2.274848in}}%
\pgfpathlineto{\pgfqpoint{3.107657in}{2.264666in}}%
\pgfpathlineto{\pgfqpoint{3.121301in}{2.254515in}}%
\pgfpathlineto{\pgfqpoint{3.134947in}{2.244393in}}%
\pgfpathlineto{\pgfqpoint{3.126102in}{2.265224in}}%
\pgfpathlineto{\pgfqpoint{3.117226in}{2.286802in}}%
\pgfpathlineto{\pgfqpoint{3.108320in}{2.309141in}}%
\pgfpathlineto{\pgfqpoint{3.099382in}{2.332256in}}%
\pgfpathlineto{\pgfqpoint{3.085678in}{2.342841in}}%
\pgfpathlineto{\pgfqpoint{3.071976in}{2.353456in}}%
\pgfpathlineto{\pgfqpoint{3.058277in}{2.364101in}}%
\pgfpathlineto{\pgfqpoint{3.044580in}{2.374775in}}%
\pgfpathlineto{\pgfqpoint{3.053577in}{2.351190in}}%
\pgfpathlineto{\pgfqpoint{3.062542in}{2.328385in}}%
\pgfpathlineto{\pgfqpoint{3.071475in}{2.306346in}}%
\pgfpathlineto{\pgfqpoint{3.080377in}{2.285058in}}%
\pgfpathclose%
\pgfusepath{fill}%
\end{pgfscope}%
\begin{pgfscope}%
\pgfpathrectangle{\pgfqpoint{1.254980in}{0.150000in}}{\pgfqpoint{5.490039in}{5.490039in}}%
\pgfusepath{clip}%
\pgfsetbuttcap%
\pgfsetroundjoin%
\definecolor{currentfill}{rgb}{0.259857,0.745492,0.444467}%
\pgfsetfillcolor{currentfill}%
\pgfsetfillopacity{0.700000}%
\pgfsetlinewidth{0.000000pt}%
\definecolor{currentstroke}{rgb}{0.000000,0.000000,0.000000}%
\pgfsetstrokecolor{currentstroke}%
\pgfsetdash{}{0pt}%
\pgfpathmoveto{\pgfqpoint{2.389059in}{2.926355in}}%
\pgfpathlineto{\pgfqpoint{2.402690in}{2.913988in}}%
\pgfpathlineto{\pgfqpoint{2.416322in}{2.901663in}}%
\pgfpathlineto{\pgfqpoint{2.429955in}{2.889379in}}%
\pgfpathlineto{\pgfqpoint{2.443587in}{2.877136in}}%
\pgfpathlineto{\pgfqpoint{2.433822in}{2.906833in}}%
\pgfpathlineto{\pgfqpoint{2.424008in}{2.937400in}}%
\pgfpathlineto{\pgfqpoint{2.414146in}{2.968853in}}%
\pgfpathlineto{\pgfqpoint{2.404235in}{3.001207in}}%
\pgfpathlineto{\pgfqpoint{2.390528in}{3.013950in}}%
\pgfpathlineto{\pgfqpoint{2.376821in}{3.026734in}}%
\pgfpathlineto{\pgfqpoint{2.363114in}{3.039561in}}%
\pgfpathlineto{\pgfqpoint{2.349408in}{3.052430in}}%
\pgfpathlineto{\pgfqpoint{2.359396in}{3.019566in}}%
\pgfpathlineto{\pgfqpoint{2.369333in}{2.987610in}}%
\pgfpathlineto{\pgfqpoint{2.379221in}{2.956545in}}%
\pgfpathlineto{\pgfqpoint{2.389059in}{2.926355in}}%
\pgfpathclose%
\pgfusepath{fill}%
\end{pgfscope}%
\begin{pgfscope}%
\pgfpathrectangle{\pgfqpoint{1.254980in}{0.150000in}}{\pgfqpoint{5.490039in}{5.490039in}}%
\pgfusepath{clip}%
\pgfsetbuttcap%
\pgfsetroundjoin%
\definecolor{currentfill}{rgb}{0.279566,0.067836,0.391917}%
\pgfsetfillcolor{currentfill}%
\pgfsetfillopacity{0.700000}%
\pgfsetlinewidth{0.000000pt}%
\definecolor{currentstroke}{rgb}{0.000000,0.000000,0.000000}%
\pgfsetstrokecolor{currentstroke}%
\pgfsetdash{}{0pt}%
\pgfpathmoveto{\pgfqpoint{4.756245in}{1.270282in}}%
\pgfpathlineto{\pgfqpoint{4.770155in}{1.265122in}}%
\pgfpathlineto{\pgfqpoint{4.784070in}{1.259985in}}%
\pgfpathlineto{\pgfqpoint{4.797993in}{1.254871in}}%
\pgfpathlineto{\pgfqpoint{4.811922in}{1.249778in}}%
\pgfpathlineto{\pgfqpoint{4.804335in}{1.246998in}}%
\pgfpathlineto{\pgfqpoint{4.796745in}{1.244578in}}%
\pgfpathlineto{\pgfqpoint{4.789154in}{1.242529in}}%
\pgfpathlineto{\pgfqpoint{4.781560in}{1.240859in}}%
\pgfpathlineto{\pgfqpoint{4.767614in}{1.246309in}}%
\pgfpathlineto{\pgfqpoint{4.753675in}{1.251781in}}%
\pgfpathlineto{\pgfqpoint{4.739741in}{1.257275in}}%
\pgfpathlineto{\pgfqpoint{4.725814in}{1.262792in}}%
\pgfpathlineto{\pgfqpoint{4.733426in}{1.264099in}}%
\pgfpathlineto{\pgfqpoint{4.741035in}{1.265789in}}%
\pgfpathlineto{\pgfqpoint{4.748641in}{1.267853in}}%
\pgfpathlineto{\pgfqpoint{4.756245in}{1.270282in}}%
\pgfpathclose%
\pgfusepath{fill}%
\end{pgfscope}%
\begin{pgfscope}%
\pgfpathrectangle{\pgfqpoint{1.254980in}{0.150000in}}{\pgfqpoint{5.490039in}{5.490039in}}%
\pgfusepath{clip}%
\pgfsetbuttcap%
\pgfsetroundjoin%
\definecolor{currentfill}{rgb}{0.204903,0.375746,0.553533}%
\pgfsetfillcolor{currentfill}%
\pgfsetfillopacity{0.700000}%
\pgfsetlinewidth{0.000000pt}%
\definecolor{currentstroke}{rgb}{0.000000,0.000000,0.000000}%
\pgfsetstrokecolor{currentstroke}%
\pgfsetdash{}{0pt}%
\pgfpathmoveto{\pgfqpoint{3.551850in}{1.908604in}}%
\pgfpathlineto{\pgfqpoint{3.565534in}{1.899719in}}%
\pgfpathlineto{\pgfqpoint{3.579221in}{1.890860in}}%
\pgfpathlineto{\pgfqpoint{3.592911in}{1.882027in}}%
\pgfpathlineto{\pgfqpoint{3.606605in}{1.873219in}}%
\pgfpathlineto{\pgfqpoint{3.598262in}{1.887732in}}%
\pgfpathlineto{\pgfqpoint{3.589899in}{1.902901in}}%
\pgfpathlineto{\pgfqpoint{3.581516in}{1.918737in}}%
\pgfpathlineto{\pgfqpoint{3.573111in}{1.935255in}}%
\pgfpathlineto{\pgfqpoint{3.559370in}{1.944501in}}%
\pgfpathlineto{\pgfqpoint{3.545633in}{1.953772in}}%
\pgfpathlineto{\pgfqpoint{3.531899in}{1.963070in}}%
\pgfpathlineto{\pgfqpoint{3.518169in}{1.972393in}}%
\pgfpathlineto{\pgfqpoint{3.526621in}{1.955430in}}%
\pgfpathlineto{\pgfqpoint{3.535052in}{1.939153in}}%
\pgfpathlineto{\pgfqpoint{3.543462in}{1.923548in}}%
\pgfpathlineto{\pgfqpoint{3.551850in}{1.908604in}}%
\pgfpathclose%
\pgfusepath{fill}%
\end{pgfscope}%
\begin{pgfscope}%
\pgfpathrectangle{\pgfqpoint{1.254980in}{0.150000in}}{\pgfqpoint{5.490039in}{5.490039in}}%
\pgfusepath{clip}%
\pgfsetbuttcap%
\pgfsetroundjoin%
\definecolor{currentfill}{rgb}{0.275191,0.194905,0.496005}%
\pgfsetfillcolor{currentfill}%
\pgfsetfillopacity{0.700000}%
\pgfsetlinewidth{0.000000pt}%
\definecolor{currentstroke}{rgb}{0.000000,0.000000,0.000000}%
\pgfsetstrokecolor{currentstroke}%
\pgfsetdash{}{0pt}%
\pgfpathmoveto{\pgfqpoint{4.165574in}{1.508148in}}%
\pgfpathlineto{\pgfqpoint{4.179350in}{1.501059in}}%
\pgfpathlineto{\pgfqpoint{4.193132in}{1.493994in}}%
\pgfpathlineto{\pgfqpoint{4.206919in}{1.486952in}}%
\pgfpathlineto{\pgfqpoint{4.220711in}{1.479933in}}%
\pgfpathlineto{\pgfqpoint{4.212854in}{1.485690in}}%
\pgfpathlineto{\pgfqpoint{4.204988in}{1.491965in}}%
\pgfpathlineto{\pgfqpoint{4.197113in}{1.498768in}}%
\pgfpathlineto{\pgfqpoint{4.189229in}{1.506111in}}%
\pgfpathlineto{\pgfqpoint{4.175406in}{1.513531in}}%
\pgfpathlineto{\pgfqpoint{4.161587in}{1.520975in}}%
\pgfpathlineto{\pgfqpoint{4.147774in}{1.528442in}}%
\pgfpathlineto{\pgfqpoint{4.133965in}{1.535931in}}%
\pgfpathlineto{\pgfqpoint{4.141883in}{1.528181in}}%
\pgfpathlineto{\pgfqpoint{4.149790in}{1.520975in}}%
\pgfpathlineto{\pgfqpoint{4.157687in}{1.514301in}}%
\pgfpathlineto{\pgfqpoint{4.165574in}{1.508148in}}%
\pgfpathclose%
\pgfusepath{fill}%
\end{pgfscope}%
\begin{pgfscope}%
\pgfpathrectangle{\pgfqpoint{1.254980in}{0.150000in}}{\pgfqpoint{5.490039in}{5.490039in}}%
\pgfusepath{clip}%
\pgfsetbuttcap%
\pgfsetroundjoin%
\definecolor{currentfill}{rgb}{0.232815,0.732247,0.459277}%
\pgfsetfillcolor{currentfill}%
\pgfsetfillopacity{0.700000}%
\pgfsetlinewidth{0.000000pt}%
\definecolor{currentstroke}{rgb}{0.000000,0.000000,0.000000}%
\pgfsetstrokecolor{currentstroke}%
\pgfsetdash{}{0pt}%
\pgfpathmoveto{\pgfqpoint{2.443587in}{2.877136in}}%
\pgfpathlineto{\pgfqpoint{2.457221in}{2.864934in}}%
\pgfpathlineto{\pgfqpoint{2.470855in}{2.852772in}}%
\pgfpathlineto{\pgfqpoint{2.484490in}{2.840651in}}%
\pgfpathlineto{\pgfqpoint{2.498125in}{2.828570in}}%
\pgfpathlineto{\pgfqpoint{2.488431in}{2.857777in}}%
\pgfpathlineto{\pgfqpoint{2.478691in}{2.887848in}}%
\pgfpathlineto{\pgfqpoint{2.468903in}{2.918798in}}%
\pgfpathlineto{\pgfqpoint{2.459067in}{2.950645in}}%
\pgfpathlineto{\pgfqpoint{2.445359in}{2.963225in}}%
\pgfpathlineto{\pgfqpoint{2.431650in}{2.975845in}}%
\pgfpathlineto{\pgfqpoint{2.417942in}{2.988505in}}%
\pgfpathlineto{\pgfqpoint{2.404235in}{3.001207in}}%
\pgfpathlineto{\pgfqpoint{2.414146in}{2.968853in}}%
\pgfpathlineto{\pgfqpoint{2.424008in}{2.937400in}}%
\pgfpathlineto{\pgfqpoint{2.433822in}{2.906833in}}%
\pgfpathlineto{\pgfqpoint{2.443587in}{2.877136in}}%
\pgfpathclose%
\pgfusepath{fill}%
\end{pgfscope}%
\begin{pgfscope}%
\pgfpathrectangle{\pgfqpoint{1.254980in}{0.150000in}}{\pgfqpoint{5.490039in}{5.490039in}}%
\pgfusepath{clip}%
\pgfsetbuttcap%
\pgfsetroundjoin%
\definecolor{currentfill}{rgb}{0.150476,0.504369,0.557430}%
\pgfsetfillcolor{currentfill}%
\pgfsetfillopacity{0.700000}%
\pgfsetlinewidth{0.000000pt}%
\definecolor{currentstroke}{rgb}{0.000000,0.000000,0.000000}%
\pgfsetstrokecolor{currentstroke}%
\pgfsetdash{}{0pt}%
\pgfpathmoveto{\pgfqpoint{3.134947in}{2.244393in}}%
\pgfpathlineto{\pgfqpoint{3.148596in}{2.234300in}}%
\pgfpathlineto{\pgfqpoint{3.162247in}{2.224236in}}%
\pgfpathlineto{\pgfqpoint{3.175901in}{2.214202in}}%
\pgfpathlineto{\pgfqpoint{3.189558in}{2.204196in}}%
\pgfpathlineto{\pgfqpoint{3.180768in}{2.224572in}}%
\pgfpathlineto{\pgfqpoint{3.171949in}{2.245690in}}%
\pgfpathlineto{\pgfqpoint{3.163100in}{2.267565in}}%
\pgfpathlineto{\pgfqpoint{3.154221in}{2.290209in}}%
\pgfpathlineto{\pgfqpoint{3.140508in}{2.300677in}}%
\pgfpathlineto{\pgfqpoint{3.126797in}{2.311174in}}%
\pgfpathlineto{\pgfqpoint{3.113088in}{2.321700in}}%
\pgfpathlineto{\pgfqpoint{3.099382in}{2.332256in}}%
\pgfpathlineto{\pgfqpoint{3.108320in}{2.309141in}}%
\pgfpathlineto{\pgfqpoint{3.117226in}{2.286802in}}%
\pgfpathlineto{\pgfqpoint{3.126102in}{2.265224in}}%
\pgfpathlineto{\pgfqpoint{3.134947in}{2.244393in}}%
\pgfpathclose%
\pgfusepath{fill}%
\end{pgfscope}%
\begin{pgfscope}%
\pgfpathrectangle{\pgfqpoint{1.254980in}{0.150000in}}{\pgfqpoint{5.490039in}{5.490039in}}%
\pgfusepath{clip}%
\pgfsetbuttcap%
\pgfsetroundjoin%
\definecolor{currentfill}{rgb}{0.252194,0.269783,0.531579}%
\pgfsetfillcolor{currentfill}%
\pgfsetfillopacity{0.700000}%
\pgfsetlinewidth{0.000000pt}%
\definecolor{currentstroke}{rgb}{0.000000,0.000000,0.000000}%
\pgfsetstrokecolor{currentstroke}%
\pgfsetdash{}{0pt}%
\pgfpathmoveto{\pgfqpoint{3.913646in}{1.658968in}}%
\pgfpathlineto{\pgfqpoint{3.927383in}{1.651100in}}%
\pgfpathlineto{\pgfqpoint{3.941124in}{1.643256in}}%
\pgfpathlineto{\pgfqpoint{3.954869in}{1.635436in}}%
\pgfpathlineto{\pgfqpoint{3.968619in}{1.627640in}}%
\pgfpathlineto{\pgfqpoint{3.960585in}{1.637181in}}%
\pgfpathlineto{\pgfqpoint{3.952538in}{1.647301in}}%
\pgfpathlineto{\pgfqpoint{3.944478in}{1.658013in}}%
\pgfpathlineto{\pgfqpoint{3.936403in}{1.669330in}}%
\pgfpathlineto{\pgfqpoint{3.922616in}{1.677544in}}%
\pgfpathlineto{\pgfqpoint{3.908832in}{1.685783in}}%
\pgfpathlineto{\pgfqpoint{3.895053in}{1.694045in}}%
\pgfpathlineto{\pgfqpoint{3.881278in}{1.702332in}}%
\pgfpathlineto{\pgfqpoint{3.889392in}{1.690591in}}%
\pgfpathlineto{\pgfqpoint{3.897491in}{1.679458in}}%
\pgfpathlineto{\pgfqpoint{3.905576in}{1.668921in}}%
\pgfpathlineto{\pgfqpoint{3.913646in}{1.658968in}}%
\pgfpathclose%
\pgfusepath{fill}%
\end{pgfscope}%
\begin{pgfscope}%
\pgfpathrectangle{\pgfqpoint{1.254980in}{0.150000in}}{\pgfqpoint{5.490039in}{5.490039in}}%
\pgfusepath{clip}%
\pgfsetbuttcap%
\pgfsetroundjoin%
\definecolor{currentfill}{rgb}{0.282327,0.094955,0.417331}%
\pgfsetfillcolor{currentfill}%
\pgfsetfillopacity{0.700000}%
\pgfsetlinewidth{0.000000pt}%
\definecolor{currentstroke}{rgb}{0.000000,0.000000,0.000000}%
\pgfsetstrokecolor{currentstroke}%
\pgfsetdash{}{0pt}%
\pgfpathmoveto{\pgfqpoint{4.614616in}{1.307726in}}%
\pgfpathlineto{\pgfqpoint{4.628495in}{1.302031in}}%
\pgfpathlineto{\pgfqpoint{4.642379in}{1.296359in}}%
\pgfpathlineto{\pgfqpoint{4.656270in}{1.290708in}}%
\pgfpathlineto{\pgfqpoint{4.670167in}{1.285081in}}%
\pgfpathlineto{\pgfqpoint{4.662533in}{1.284532in}}%
\pgfpathlineto{\pgfqpoint{4.654895in}{1.284391in}}%
\pgfpathlineto{\pgfqpoint{4.647253in}{1.284666in}}%
\pgfpathlineto{\pgfqpoint{4.639608in}{1.285367in}}%
\pgfpathlineto{\pgfqpoint{4.625691in}{1.291366in}}%
\pgfpathlineto{\pgfqpoint{4.611780in}{1.297388in}}%
\pgfpathlineto{\pgfqpoint{4.597874in}{1.303431in}}%
\pgfpathlineto{\pgfqpoint{4.583974in}{1.309497in}}%
\pgfpathlineto{\pgfqpoint{4.591641in}{1.308419in}}%
\pgfpathlineto{\pgfqpoint{4.599303in}{1.307772in}}%
\pgfpathlineto{\pgfqpoint{4.606962in}{1.307544in}}%
\pgfpathlineto{\pgfqpoint{4.614616in}{1.307726in}}%
\pgfpathclose%
\pgfusepath{fill}%
\end{pgfscope}%
\begin{pgfscope}%
\pgfpathrectangle{\pgfqpoint{1.254980in}{0.150000in}}{\pgfqpoint{5.490039in}{5.490039in}}%
\pgfusepath{clip}%
\pgfsetbuttcap%
\pgfsetroundjoin%
\definecolor{currentfill}{rgb}{0.282884,0.135920,0.453427}%
\pgfsetfillcolor{currentfill}%
\pgfsetfillopacity{0.700000}%
\pgfsetlinewidth{0.000000pt}%
\definecolor{currentstroke}{rgb}{0.000000,0.000000,0.000000}%
\pgfsetstrokecolor{currentstroke}%
\pgfsetdash{}{0pt}%
\pgfpathmoveto{\pgfqpoint{4.417618in}{1.384042in}}%
\pgfpathlineto{\pgfqpoint{4.431450in}{1.377706in}}%
\pgfpathlineto{\pgfqpoint{4.445288in}{1.371392in}}%
\pgfpathlineto{\pgfqpoint{4.459131in}{1.365102in}}%
\pgfpathlineto{\pgfqpoint{4.472980in}{1.358833in}}%
\pgfpathlineto{\pgfqpoint{4.465262in}{1.361112in}}%
\pgfpathlineto{\pgfqpoint{4.457538in}{1.363849in}}%
\pgfpathlineto{\pgfqpoint{4.449807in}{1.367056in}}%
\pgfpathlineto{\pgfqpoint{4.442071in}{1.370742in}}%
\pgfpathlineto{\pgfqpoint{4.428197in}{1.377396in}}%
\pgfpathlineto{\pgfqpoint{4.414328in}{1.384072in}}%
\pgfpathlineto{\pgfqpoint{4.400464in}{1.390771in}}%
\pgfpathlineto{\pgfqpoint{4.386606in}{1.397493in}}%
\pgfpathlineto{\pgfqpoint{4.394369in}{1.393416in}}%
\pgfpathlineto{\pgfqpoint{4.402125in}{1.389822in}}%
\pgfpathlineto{\pgfqpoint{4.409874in}{1.386701in}}%
\pgfpathlineto{\pgfqpoint{4.417618in}{1.384042in}}%
\pgfpathclose%
\pgfusepath{fill}%
\end{pgfscope}%
\begin{pgfscope}%
\pgfpathrectangle{\pgfqpoint{1.254980in}{0.150000in}}{\pgfqpoint{5.490039in}{5.490039in}}%
\pgfusepath{clip}%
\pgfsetbuttcap%
\pgfsetroundjoin%
\definecolor{currentfill}{rgb}{0.202219,0.715272,0.476084}%
\pgfsetfillcolor{currentfill}%
\pgfsetfillopacity{0.700000}%
\pgfsetlinewidth{0.000000pt}%
\definecolor{currentstroke}{rgb}{0.000000,0.000000,0.000000}%
\pgfsetstrokecolor{currentstroke}%
\pgfsetdash{}{0pt}%
\pgfpathmoveto{\pgfqpoint{2.498125in}{2.828570in}}%
\pgfpathlineto{\pgfqpoint{2.511761in}{2.816528in}}%
\pgfpathlineto{\pgfqpoint{2.525398in}{2.804525in}}%
\pgfpathlineto{\pgfqpoint{2.539036in}{2.792561in}}%
\pgfpathlineto{\pgfqpoint{2.552675in}{2.780636in}}%
\pgfpathlineto{\pgfqpoint{2.543052in}{2.809354in}}%
\pgfpathlineto{\pgfqpoint{2.533384in}{2.838930in}}%
\pgfpathlineto{\pgfqpoint{2.523669in}{2.869381in}}%
\pgfpathlineto{\pgfqpoint{2.513908in}{2.900722in}}%
\pgfpathlineto{\pgfqpoint{2.500197in}{2.913144in}}%
\pgfpathlineto{\pgfqpoint{2.486487in}{2.925605in}}%
\pgfpathlineto{\pgfqpoint{2.472777in}{2.938105in}}%
\pgfpathlineto{\pgfqpoint{2.459067in}{2.950645in}}%
\pgfpathlineto{\pgfqpoint{2.468903in}{2.918798in}}%
\pgfpathlineto{\pgfqpoint{2.478691in}{2.887848in}}%
\pgfpathlineto{\pgfqpoint{2.488431in}{2.857777in}}%
\pgfpathlineto{\pgfqpoint{2.498125in}{2.828570in}}%
\pgfpathclose%
\pgfusepath{fill}%
\end{pgfscope}%
\begin{pgfscope}%
\pgfpathrectangle{\pgfqpoint{1.254980in}{0.150000in}}{\pgfqpoint{5.490039in}{5.490039in}}%
\pgfusepath{clip}%
\pgfsetbuttcap%
\pgfsetroundjoin%
\definecolor{currentfill}{rgb}{0.277018,0.050344,0.375715}%
\pgfsetfillcolor{currentfill}%
\pgfsetfillopacity{0.700000}%
\pgfsetlinewidth{0.000000pt}%
\definecolor{currentstroke}{rgb}{0.000000,0.000000,0.000000}%
\pgfsetstrokecolor{currentstroke}%
\pgfsetdash{}{0pt}%
\pgfpathmoveto{\pgfqpoint{4.953796in}{1.227130in}}%
\pgfpathlineto{\pgfqpoint{4.967770in}{1.222581in}}%
\pgfpathlineto{\pgfqpoint{4.981751in}{1.218055in}}%
\pgfpathlineto{\pgfqpoint{4.995738in}{1.213550in}}%
\pgfpathlineto{\pgfqpoint{4.988197in}{1.208488in}}%
\pgfpathlineto{\pgfqpoint{4.980655in}{1.203741in}}%
\pgfpathlineto{\pgfqpoint{4.973112in}{1.199316in}}%
\pgfpathlineto{\pgfqpoint{4.965568in}{1.195225in}}%
\pgfpathlineto{\pgfqpoint{4.951567in}{1.200074in}}%
\pgfpathlineto{\pgfqpoint{4.937573in}{1.204944in}}%
\pgfpathlineto{\pgfqpoint{4.923585in}{1.209837in}}%
\pgfpathlineto{\pgfqpoint{4.931140in}{1.213667in}}%
\pgfpathlineto{\pgfqpoint{4.938693in}{1.217831in}}%
\pgfpathlineto{\pgfqpoint{4.946245in}{1.222322in}}%
\pgfpathlineto{\pgfqpoint{4.953796in}{1.227130in}}%
\pgfpathclose%
\pgfusepath{fill}%
\end{pgfscope}%
\begin{pgfscope}%
\pgfpathrectangle{\pgfqpoint{1.254980in}{0.150000in}}{\pgfqpoint{5.490039in}{5.490039in}}%
\pgfusepath{clip}%
\pgfsetbuttcap%
\pgfsetroundjoin%
\definecolor{currentfill}{rgb}{0.210503,0.363727,0.552206}%
\pgfsetfillcolor{currentfill}%
\pgfsetfillopacity{0.700000}%
\pgfsetlinewidth{0.000000pt}%
\definecolor{currentstroke}{rgb}{0.000000,0.000000,0.000000}%
\pgfsetstrokecolor{currentstroke}%
\pgfsetdash{}{0pt}%
\pgfpathmoveto{\pgfqpoint{3.606605in}{1.873219in}}%
\pgfpathlineto{\pgfqpoint{3.620303in}{1.864438in}}%
\pgfpathlineto{\pgfqpoint{3.634004in}{1.855681in}}%
\pgfpathlineto{\pgfqpoint{3.647709in}{1.846950in}}%
\pgfpathlineto{\pgfqpoint{3.661418in}{1.838244in}}%
\pgfpathlineto{\pgfqpoint{3.653120in}{1.852326in}}%
\pgfpathlineto{\pgfqpoint{3.644802in}{1.867058in}}%
\pgfpathlineto{\pgfqpoint{3.636465in}{1.882455in}}%
\pgfpathlineto{\pgfqpoint{3.628108in}{1.898528in}}%
\pgfpathlineto{\pgfqpoint{3.614353in}{1.907672in}}%
\pgfpathlineto{\pgfqpoint{3.600602in}{1.916841in}}%
\pgfpathlineto{\pgfqpoint{3.586855in}{1.926035in}}%
\pgfpathlineto{\pgfqpoint{3.573111in}{1.935255in}}%
\pgfpathlineto{\pgfqpoint{3.581516in}{1.918737in}}%
\pgfpathlineto{\pgfqpoint{3.589899in}{1.902901in}}%
\pgfpathlineto{\pgfqpoint{3.598262in}{1.887732in}}%
\pgfpathlineto{\pgfqpoint{3.606605in}{1.873219in}}%
\pgfpathclose%
\pgfusepath{fill}%
\end{pgfscope}%
\begin{pgfscope}%
\pgfpathrectangle{\pgfqpoint{1.254980in}{0.150000in}}{\pgfqpoint{5.490039in}{5.490039in}}%
\pgfusepath{clip}%
\pgfsetbuttcap%
\pgfsetroundjoin%
\definecolor{currentfill}{rgb}{0.180653,0.701402,0.488189}%
\pgfsetfillcolor{currentfill}%
\pgfsetfillopacity{0.700000}%
\pgfsetlinewidth{0.000000pt}%
\definecolor{currentstroke}{rgb}{0.000000,0.000000,0.000000}%
\pgfsetstrokecolor{currentstroke}%
\pgfsetdash{}{0pt}%
\pgfpathmoveto{\pgfqpoint{2.552675in}{2.780636in}}%
\pgfpathlineto{\pgfqpoint{2.566315in}{2.768749in}}%
\pgfpathlineto{\pgfqpoint{2.579955in}{2.756900in}}%
\pgfpathlineto{\pgfqpoint{2.593597in}{2.745088in}}%
\pgfpathlineto{\pgfqpoint{2.607240in}{2.733314in}}%
\pgfpathlineto{\pgfqpoint{2.597686in}{2.761545in}}%
\pgfpathlineto{\pgfqpoint{2.588089in}{2.790628in}}%
\pgfpathlineto{\pgfqpoint{2.578447in}{2.820581in}}%
\pgfpathlineto{\pgfqpoint{2.568759in}{2.851418in}}%
\pgfpathlineto{\pgfqpoint{2.555045in}{2.863687in}}%
\pgfpathlineto{\pgfqpoint{2.541332in}{2.875994in}}%
\pgfpathlineto{\pgfqpoint{2.527620in}{2.888339in}}%
\pgfpathlineto{\pgfqpoint{2.513908in}{2.900722in}}%
\pgfpathlineto{\pgfqpoint{2.523669in}{2.869381in}}%
\pgfpathlineto{\pgfqpoint{2.533384in}{2.838930in}}%
\pgfpathlineto{\pgfqpoint{2.543052in}{2.809354in}}%
\pgfpathlineto{\pgfqpoint{2.552675in}{2.780636in}}%
\pgfpathclose%
\pgfusepath{fill}%
\end{pgfscope}%
\begin{pgfscope}%
\pgfpathrectangle{\pgfqpoint{1.254980in}{0.150000in}}{\pgfqpoint{5.490039in}{5.490039in}}%
\pgfusepath{clip}%
\pgfsetbuttcap%
\pgfsetroundjoin%
\definecolor{currentfill}{rgb}{0.154815,0.493313,0.557840}%
\pgfsetfillcolor{currentfill}%
\pgfsetfillopacity{0.700000}%
\pgfsetlinewidth{0.000000pt}%
\definecolor{currentstroke}{rgb}{0.000000,0.000000,0.000000}%
\pgfsetstrokecolor{currentstroke}%
\pgfsetdash{}{0pt}%
\pgfpathmoveto{\pgfqpoint{3.189558in}{2.204196in}}%
\pgfpathlineto{\pgfqpoint{3.203217in}{2.194219in}}%
\pgfpathlineto{\pgfqpoint{3.216878in}{2.184270in}}%
\pgfpathlineto{\pgfqpoint{3.230543in}{2.174350in}}%
\pgfpathlineto{\pgfqpoint{3.244210in}{2.164459in}}%
\pgfpathlineto{\pgfqpoint{3.235475in}{2.184381in}}%
\pgfpathlineto{\pgfqpoint{3.226712in}{2.205040in}}%
\pgfpathlineto{\pgfqpoint{3.217920in}{2.226450in}}%
\pgfpathlineto{\pgfqpoint{3.209099in}{2.248626in}}%
\pgfpathlineto{\pgfqpoint{3.195376in}{2.258979in}}%
\pgfpathlineto{\pgfqpoint{3.181655in}{2.269361in}}%
\pgfpathlineto{\pgfqpoint{3.167937in}{2.279771in}}%
\pgfpathlineto{\pgfqpoint{3.154221in}{2.290209in}}%
\pgfpathlineto{\pgfqpoint{3.163100in}{2.267565in}}%
\pgfpathlineto{\pgfqpoint{3.171949in}{2.245690in}}%
\pgfpathlineto{\pgfqpoint{3.180768in}{2.224572in}}%
\pgfpathlineto{\pgfqpoint{3.189558in}{2.204196in}}%
\pgfpathclose%
\pgfusepath{fill}%
\end{pgfscope}%
\begin{pgfscope}%
\pgfpathrectangle{\pgfqpoint{1.254980in}{0.150000in}}{\pgfqpoint{5.490039in}{5.490039in}}%
\pgfusepath{clip}%
\pgfsetbuttcap%
\pgfsetroundjoin%
\definecolor{currentfill}{rgb}{0.279566,0.067836,0.391917}%
\pgfsetfillcolor{currentfill}%
\pgfsetfillopacity{0.700000}%
\pgfsetlinewidth{0.000000pt}%
\definecolor{currentstroke}{rgb}{0.000000,0.000000,0.000000}%
\pgfsetstrokecolor{currentstroke}%
\pgfsetdash{}{0pt}%
\pgfpathmoveto{\pgfqpoint{4.811922in}{1.249778in}}%
\pgfpathlineto{\pgfqpoint{4.825857in}{1.244708in}}%
\pgfpathlineto{\pgfqpoint{4.839798in}{1.239660in}}%
\pgfpathlineto{\pgfqpoint{4.853746in}{1.234634in}}%
\pgfpathlineto{\pgfqpoint{4.867701in}{1.229630in}}%
\pgfpathlineto{\pgfqpoint{4.860130in}{1.226498in}}%
\pgfpathlineto{\pgfqpoint{4.852557in}{1.223722in}}%
\pgfpathlineto{\pgfqpoint{4.844982in}{1.221314in}}%
\pgfpathlineto{\pgfqpoint{4.837406in}{1.219282in}}%
\pgfpathlineto{\pgfqpoint{4.823435in}{1.224643in}}%
\pgfpathlineto{\pgfqpoint{4.809470in}{1.230026in}}%
\pgfpathlineto{\pgfqpoint{4.795512in}{1.235432in}}%
\pgfpathlineto{\pgfqpoint{4.781560in}{1.240859in}}%
\pgfpathlineto{\pgfqpoint{4.789154in}{1.242529in}}%
\pgfpathlineto{\pgfqpoint{4.796745in}{1.244578in}}%
\pgfpathlineto{\pgfqpoint{4.804335in}{1.246998in}}%
\pgfpathlineto{\pgfqpoint{4.811922in}{1.249778in}}%
\pgfpathclose%
\pgfusepath{fill}%
\end{pgfscope}%
\begin{pgfscope}%
\pgfpathrectangle{\pgfqpoint{1.254980in}{0.150000in}}{\pgfqpoint{5.490039in}{5.490039in}}%
\pgfusepath{clip}%
\pgfsetbuttcap%
\pgfsetroundjoin%
\definecolor{currentfill}{rgb}{0.276194,0.190074,0.493001}%
\pgfsetfillcolor{currentfill}%
\pgfsetfillopacity{0.700000}%
\pgfsetlinewidth{0.000000pt}%
\definecolor{currentstroke}{rgb}{0.000000,0.000000,0.000000}%
\pgfsetstrokecolor{currentstroke}%
\pgfsetdash{}{0pt}%
\pgfpathmoveto{\pgfqpoint{4.220711in}{1.479933in}}%
\pgfpathlineto{\pgfqpoint{4.234507in}{1.472936in}}%
\pgfpathlineto{\pgfqpoint{4.248309in}{1.465963in}}%
\pgfpathlineto{\pgfqpoint{4.262115in}{1.459013in}}%
\pgfpathlineto{\pgfqpoint{4.275927in}{1.452086in}}%
\pgfpathlineto{\pgfqpoint{4.268100in}{1.457449in}}%
\pgfpathlineto{\pgfqpoint{4.260265in}{1.463324in}}%
\pgfpathlineto{\pgfqpoint{4.252421in}{1.469724in}}%
\pgfpathlineto{\pgfqpoint{4.244569in}{1.476661in}}%
\pgfpathlineto{\pgfqpoint{4.230726in}{1.483989in}}%
\pgfpathlineto{\pgfqpoint{4.216889in}{1.491340in}}%
\pgfpathlineto{\pgfqpoint{4.203056in}{1.498714in}}%
\pgfpathlineto{\pgfqpoint{4.189229in}{1.506111in}}%
\pgfpathlineto{\pgfqpoint{4.197113in}{1.498768in}}%
\pgfpathlineto{\pgfqpoint{4.204988in}{1.491965in}}%
\pgfpathlineto{\pgfqpoint{4.212854in}{1.485690in}}%
\pgfpathlineto{\pgfqpoint{4.220711in}{1.479933in}}%
\pgfpathclose%
\pgfusepath{fill}%
\end{pgfscope}%
\begin{pgfscope}%
\pgfpathrectangle{\pgfqpoint{1.254980in}{0.150000in}}{\pgfqpoint{5.490039in}{5.490039in}}%
\pgfusepath{clip}%
\pgfsetbuttcap%
\pgfsetroundjoin%
\definecolor{currentfill}{rgb}{0.255645,0.260703,0.528312}%
\pgfsetfillcolor{currentfill}%
\pgfsetfillopacity{0.700000}%
\pgfsetlinewidth{0.000000pt}%
\definecolor{currentstroke}{rgb}{0.000000,0.000000,0.000000}%
\pgfsetstrokecolor{currentstroke}%
\pgfsetdash{}{0pt}%
\pgfpathmoveto{\pgfqpoint{3.968619in}{1.627640in}}%
\pgfpathlineto{\pgfqpoint{3.982373in}{1.619867in}}%
\pgfpathlineto{\pgfqpoint{3.996131in}{1.612119in}}%
\pgfpathlineto{\pgfqpoint{4.009894in}{1.604394in}}%
\pgfpathlineto{\pgfqpoint{4.023662in}{1.596693in}}%
\pgfpathlineto{\pgfqpoint{4.015664in}{1.605822in}}%
\pgfpathlineto{\pgfqpoint{4.007654in}{1.615526in}}%
\pgfpathlineto{\pgfqpoint{3.999631in}{1.625819in}}%
\pgfpathlineto{\pgfqpoint{3.991595in}{1.636711in}}%
\pgfpathlineto{\pgfqpoint{3.977791in}{1.644830in}}%
\pgfpathlineto{\pgfqpoint{3.963991in}{1.652973in}}%
\pgfpathlineto{\pgfqpoint{3.950195in}{1.661140in}}%
\pgfpathlineto{\pgfqpoint{3.936403in}{1.669330in}}%
\pgfpathlineto{\pgfqpoint{3.944478in}{1.658013in}}%
\pgfpathlineto{\pgfqpoint{3.952538in}{1.647301in}}%
\pgfpathlineto{\pgfqpoint{3.960585in}{1.637181in}}%
\pgfpathlineto{\pgfqpoint{3.968619in}{1.627640in}}%
\pgfpathclose%
\pgfusepath{fill}%
\end{pgfscope}%
\begin{pgfscope}%
\pgfpathrectangle{\pgfqpoint{1.254980in}{0.150000in}}{\pgfqpoint{5.490039in}{5.490039in}}%
\pgfusepath{clip}%
\pgfsetbuttcap%
\pgfsetroundjoin%
\definecolor{currentfill}{rgb}{0.162016,0.687316,0.499129}%
\pgfsetfillcolor{currentfill}%
\pgfsetfillopacity{0.700000}%
\pgfsetlinewidth{0.000000pt}%
\definecolor{currentstroke}{rgb}{0.000000,0.000000,0.000000}%
\pgfsetstrokecolor{currentstroke}%
\pgfsetdash{}{0pt}%
\pgfpathmoveto{\pgfqpoint{2.607240in}{2.733314in}}%
\pgfpathlineto{\pgfqpoint{2.620884in}{2.721577in}}%
\pgfpathlineto{\pgfqpoint{2.634529in}{2.709877in}}%
\pgfpathlineto{\pgfqpoint{2.648175in}{2.698213in}}%
\pgfpathlineto{\pgfqpoint{2.661822in}{2.686586in}}%
\pgfpathlineto{\pgfqpoint{2.652337in}{2.714331in}}%
\pgfpathlineto{\pgfqpoint{2.642810in}{2.742923in}}%
\pgfpathlineto{\pgfqpoint{2.633239in}{2.772378in}}%
\pgfpathlineto{\pgfqpoint{2.623624in}{2.802713in}}%
\pgfpathlineto{\pgfqpoint{2.609907in}{2.814834in}}%
\pgfpathlineto{\pgfqpoint{2.596190in}{2.826992in}}%
\pgfpathlineto{\pgfqpoint{2.582474in}{2.839186in}}%
\pgfpathlineto{\pgfqpoint{2.568759in}{2.851418in}}%
\pgfpathlineto{\pgfqpoint{2.578447in}{2.820581in}}%
\pgfpathlineto{\pgfqpoint{2.588089in}{2.790628in}}%
\pgfpathlineto{\pgfqpoint{2.597686in}{2.761545in}}%
\pgfpathlineto{\pgfqpoint{2.607240in}{2.733314in}}%
\pgfpathclose%
\pgfusepath{fill}%
\end{pgfscope}%
\begin{pgfscope}%
\pgfpathrectangle{\pgfqpoint{1.254980in}{0.150000in}}{\pgfqpoint{5.490039in}{5.490039in}}%
\pgfusepath{clip}%
\pgfsetbuttcap%
\pgfsetroundjoin%
\definecolor{currentfill}{rgb}{0.283072,0.130895,0.449241}%
\pgfsetfillcolor{currentfill}%
\pgfsetfillopacity{0.700000}%
\pgfsetlinewidth{0.000000pt}%
\definecolor{currentstroke}{rgb}{0.000000,0.000000,0.000000}%
\pgfsetstrokecolor{currentstroke}%
\pgfsetdash{}{0pt}%
\pgfpathmoveto{\pgfqpoint{4.472980in}{1.358833in}}%
\pgfpathlineto{\pgfqpoint{4.486835in}{1.352588in}}%
\pgfpathlineto{\pgfqpoint{4.500695in}{1.346364in}}%
\pgfpathlineto{\pgfqpoint{4.514560in}{1.340164in}}%
\pgfpathlineto{\pgfqpoint{4.528432in}{1.333986in}}%
\pgfpathlineto{\pgfqpoint{4.520737in}{1.335884in}}%
\pgfpathlineto{\pgfqpoint{4.513038in}{1.338238in}}%
\pgfpathlineto{\pgfqpoint{4.505333in}{1.341057in}}%
\pgfpathlineto{\pgfqpoint{4.497623in}{1.344352in}}%
\pgfpathlineto{\pgfqpoint{4.483727in}{1.350916in}}%
\pgfpathlineto{\pgfqpoint{4.469836in}{1.357502in}}%
\pgfpathlineto{\pgfqpoint{4.455951in}{1.364111in}}%
\pgfpathlineto{\pgfqpoint{4.442071in}{1.370742in}}%
\pgfpathlineto{\pgfqpoint{4.449807in}{1.367056in}}%
\pgfpathlineto{\pgfqpoint{4.457538in}{1.363849in}}%
\pgfpathlineto{\pgfqpoint{4.465262in}{1.361112in}}%
\pgfpathlineto{\pgfqpoint{4.472980in}{1.358833in}}%
\pgfpathclose%
\pgfusepath{fill}%
\end{pgfscope}%
\begin{pgfscope}%
\pgfpathrectangle{\pgfqpoint{1.254980in}{0.150000in}}{\pgfqpoint{5.490039in}{5.490039in}}%
\pgfusepath{clip}%
\pgfsetbuttcap%
\pgfsetroundjoin%
\definecolor{currentfill}{rgb}{0.214298,0.355619,0.551184}%
\pgfsetfillcolor{currentfill}%
\pgfsetfillopacity{0.700000}%
\pgfsetlinewidth{0.000000pt}%
\definecolor{currentstroke}{rgb}{0.000000,0.000000,0.000000}%
\pgfsetstrokecolor{currentstroke}%
\pgfsetdash{}{0pt}%
\pgfpathmoveto{\pgfqpoint{3.661418in}{1.838244in}}%
\pgfpathlineto{\pgfqpoint{3.675131in}{1.829563in}}%
\pgfpathlineto{\pgfqpoint{3.688847in}{1.820908in}}%
\pgfpathlineto{\pgfqpoint{3.702567in}{1.812277in}}%
\pgfpathlineto{\pgfqpoint{3.716290in}{1.803672in}}%
\pgfpathlineto{\pgfqpoint{3.708036in}{1.817323in}}%
\pgfpathlineto{\pgfqpoint{3.699763in}{1.831621in}}%
\pgfpathlineto{\pgfqpoint{3.691471in}{1.846578in}}%
\pgfpathlineto{\pgfqpoint{3.683160in}{1.862207in}}%
\pgfpathlineto{\pgfqpoint{3.669392in}{1.871250in}}%
\pgfpathlineto{\pgfqpoint{3.655627in}{1.880317in}}%
\pgfpathlineto{\pgfqpoint{3.641865in}{1.889410in}}%
\pgfpathlineto{\pgfqpoint{3.628108in}{1.898528in}}%
\pgfpathlineto{\pgfqpoint{3.636465in}{1.882455in}}%
\pgfpathlineto{\pgfqpoint{3.644802in}{1.867058in}}%
\pgfpathlineto{\pgfqpoint{3.653120in}{1.852326in}}%
\pgfpathlineto{\pgfqpoint{3.661418in}{1.838244in}}%
\pgfpathclose%
\pgfusepath{fill}%
\end{pgfscope}%
\begin{pgfscope}%
\pgfpathrectangle{\pgfqpoint{1.254980in}{0.150000in}}{\pgfqpoint{5.490039in}{5.490039in}}%
\pgfusepath{clip}%
\pgfsetbuttcap%
\pgfsetroundjoin%
\definecolor{currentfill}{rgb}{0.160665,0.478540,0.558115}%
\pgfsetfillcolor{currentfill}%
\pgfsetfillopacity{0.700000}%
\pgfsetlinewidth{0.000000pt}%
\definecolor{currentstroke}{rgb}{0.000000,0.000000,0.000000}%
\pgfsetstrokecolor{currentstroke}%
\pgfsetdash{}{0pt}%
\pgfpathmoveto{\pgfqpoint{3.244210in}{2.164459in}}%
\pgfpathlineto{\pgfqpoint{3.257880in}{2.154595in}}%
\pgfpathlineto{\pgfqpoint{3.271553in}{2.144760in}}%
\pgfpathlineto{\pgfqpoint{3.285229in}{2.134953in}}%
\pgfpathlineto{\pgfqpoint{3.298907in}{2.125173in}}%
\pgfpathlineto{\pgfqpoint{3.290226in}{2.144642in}}%
\pgfpathlineto{\pgfqpoint{3.281517in}{2.164843in}}%
\pgfpathlineto{\pgfqpoint{3.272781in}{2.185790in}}%
\pgfpathlineto{\pgfqpoint{3.264017in}{2.207498in}}%
\pgfpathlineto{\pgfqpoint{3.250284in}{2.217738in}}%
\pgfpathlineto{\pgfqpoint{3.236553in}{2.228006in}}%
\pgfpathlineto{\pgfqpoint{3.222825in}{2.238302in}}%
\pgfpathlineto{\pgfqpoint{3.209099in}{2.248626in}}%
\pgfpathlineto{\pgfqpoint{3.217920in}{2.226450in}}%
\pgfpathlineto{\pgfqpoint{3.226712in}{2.205040in}}%
\pgfpathlineto{\pgfqpoint{3.235475in}{2.184381in}}%
\pgfpathlineto{\pgfqpoint{3.244210in}{2.164459in}}%
\pgfpathclose%
\pgfusepath{fill}%
\end{pgfscope}%
\begin{pgfscope}%
\pgfpathrectangle{\pgfqpoint{1.254980in}{0.150000in}}{\pgfqpoint{5.490039in}{5.490039in}}%
\pgfusepath{clip}%
\pgfsetbuttcap%
\pgfsetroundjoin%
\definecolor{currentfill}{rgb}{0.281924,0.089666,0.412415}%
\pgfsetfillcolor{currentfill}%
\pgfsetfillopacity{0.700000}%
\pgfsetlinewidth{0.000000pt}%
\definecolor{currentstroke}{rgb}{0.000000,0.000000,0.000000}%
\pgfsetstrokecolor{currentstroke}%
\pgfsetdash{}{0pt}%
\pgfpathmoveto{\pgfqpoint{4.670167in}{1.285081in}}%
\pgfpathlineto{\pgfqpoint{4.684069in}{1.279475in}}%
\pgfpathlineto{\pgfqpoint{4.697978in}{1.273892in}}%
\pgfpathlineto{\pgfqpoint{4.711893in}{1.268331in}}%
\pgfpathlineto{\pgfqpoint{4.725814in}{1.262792in}}%
\pgfpathlineto{\pgfqpoint{4.718199in}{1.261878in}}%
\pgfpathlineto{\pgfqpoint{4.710581in}{1.261367in}}%
\pgfpathlineto{\pgfqpoint{4.702961in}{1.261269in}}%
\pgfpathlineto{\pgfqpoint{4.695336in}{1.261594in}}%
\pgfpathlineto{\pgfqpoint{4.681396in}{1.267504in}}%
\pgfpathlineto{\pgfqpoint{4.667461in}{1.273436in}}%
\pgfpathlineto{\pgfqpoint{4.653532in}{1.279391in}}%
\pgfpathlineto{\pgfqpoint{4.639608in}{1.285367in}}%
\pgfpathlineto{\pgfqpoint{4.647253in}{1.284666in}}%
\pgfpathlineto{\pgfqpoint{4.654895in}{1.284391in}}%
\pgfpathlineto{\pgfqpoint{4.662533in}{1.284532in}}%
\pgfpathlineto{\pgfqpoint{4.670167in}{1.285081in}}%
\pgfpathclose%
\pgfusepath{fill}%
\end{pgfscope}%
\begin{pgfscope}%
\pgfpathrectangle{\pgfqpoint{1.254980in}{0.150000in}}{\pgfqpoint{5.490039in}{5.490039in}}%
\pgfusepath{clip}%
\pgfsetbuttcap%
\pgfsetroundjoin%
\definecolor{currentfill}{rgb}{0.146616,0.673050,0.508936}%
\pgfsetfillcolor{currentfill}%
\pgfsetfillopacity{0.700000}%
\pgfsetlinewidth{0.000000pt}%
\definecolor{currentstroke}{rgb}{0.000000,0.000000,0.000000}%
\pgfsetstrokecolor{currentstroke}%
\pgfsetdash{}{0pt}%
\pgfpathmoveto{\pgfqpoint{2.661822in}{2.686586in}}%
\pgfpathlineto{\pgfqpoint{2.675471in}{2.674994in}}%
\pgfpathlineto{\pgfqpoint{2.689121in}{2.663439in}}%
\pgfpathlineto{\pgfqpoint{2.702772in}{2.651918in}}%
\pgfpathlineto{\pgfqpoint{2.716424in}{2.640433in}}%
\pgfpathlineto{\pgfqpoint{2.707007in}{2.667694in}}%
\pgfpathlineto{\pgfqpoint{2.697549in}{2.695796in}}%
\pgfpathlineto{\pgfqpoint{2.688048in}{2.724756in}}%
\pgfpathlineto{\pgfqpoint{2.678505in}{2.754590in}}%
\pgfpathlineto{\pgfqpoint{2.664783in}{2.766567in}}%
\pgfpathlineto{\pgfqpoint{2.651062in}{2.778580in}}%
\pgfpathlineto{\pgfqpoint{2.637343in}{2.790628in}}%
\pgfpathlineto{\pgfqpoint{2.623624in}{2.802713in}}%
\pgfpathlineto{\pgfqpoint{2.633239in}{2.772378in}}%
\pgfpathlineto{\pgfqpoint{2.642810in}{2.742923in}}%
\pgfpathlineto{\pgfqpoint{2.652337in}{2.714331in}}%
\pgfpathlineto{\pgfqpoint{2.661822in}{2.686586in}}%
\pgfpathclose%
\pgfusepath{fill}%
\end{pgfscope}%
\begin{pgfscope}%
\pgfpathrectangle{\pgfqpoint{1.254980in}{0.150000in}}{\pgfqpoint{5.490039in}{5.490039in}}%
\pgfusepath{clip}%
\pgfsetbuttcap%
\pgfsetroundjoin%
\definecolor{currentfill}{rgb}{0.278012,0.180367,0.486697}%
\pgfsetfillcolor{currentfill}%
\pgfsetfillopacity{0.700000}%
\pgfsetlinewidth{0.000000pt}%
\definecolor{currentstroke}{rgb}{0.000000,0.000000,0.000000}%
\pgfsetstrokecolor{currentstroke}%
\pgfsetdash{}{0pt}%
\pgfpathmoveto{\pgfqpoint{4.275927in}{1.452086in}}%
\pgfpathlineto{\pgfqpoint{4.289744in}{1.445182in}}%
\pgfpathlineto{\pgfqpoint{4.303566in}{1.438301in}}%
\pgfpathlineto{\pgfqpoint{4.317393in}{1.431443in}}%
\pgfpathlineto{\pgfqpoint{4.331225in}{1.424607in}}%
\pgfpathlineto{\pgfqpoint{4.323427in}{1.429574in}}%
\pgfpathlineto{\pgfqpoint{4.315622in}{1.435051in}}%
\pgfpathlineto{\pgfqpoint{4.307808in}{1.441048in}}%
\pgfpathlineto{\pgfqpoint{4.299987in}{1.447578in}}%
\pgfpathlineto{\pgfqpoint{4.286125in}{1.454814in}}%
\pgfpathlineto{\pgfqpoint{4.272268in}{1.462074in}}%
\pgfpathlineto{\pgfqpoint{4.258416in}{1.469356in}}%
\pgfpathlineto{\pgfqpoint{4.244569in}{1.476661in}}%
\pgfpathlineto{\pgfqpoint{4.252421in}{1.469724in}}%
\pgfpathlineto{\pgfqpoint{4.260265in}{1.463324in}}%
\pgfpathlineto{\pgfqpoint{4.268100in}{1.457449in}}%
\pgfpathlineto{\pgfqpoint{4.275927in}{1.452086in}}%
\pgfpathclose%
\pgfusepath{fill}%
\end{pgfscope}%
\begin{pgfscope}%
\pgfpathrectangle{\pgfqpoint{1.254980in}{0.150000in}}{\pgfqpoint{5.490039in}{5.490039in}}%
\pgfusepath{clip}%
\pgfsetbuttcap%
\pgfsetroundjoin%
\definecolor{currentfill}{rgb}{0.258965,0.251537,0.524736}%
\pgfsetfillcolor{currentfill}%
\pgfsetfillopacity{0.700000}%
\pgfsetlinewidth{0.000000pt}%
\definecolor{currentstroke}{rgb}{0.000000,0.000000,0.000000}%
\pgfsetstrokecolor{currentstroke}%
\pgfsetdash{}{0pt}%
\pgfpathmoveto{\pgfqpoint{4.023662in}{1.596693in}}%
\pgfpathlineto{\pgfqpoint{4.037434in}{1.589015in}}%
\pgfpathlineto{\pgfqpoint{4.051210in}{1.581362in}}%
\pgfpathlineto{\pgfqpoint{4.064991in}{1.573731in}}%
\pgfpathlineto{\pgfqpoint{4.078777in}{1.566125in}}%
\pgfpathlineto{\pgfqpoint{4.070814in}{1.574842in}}%
\pgfpathlineto{\pgfqpoint{4.062840in}{1.584131in}}%
\pgfpathlineto{\pgfqpoint{4.054854in}{1.594003in}}%
\pgfpathlineto{\pgfqpoint{4.046856in}{1.604472in}}%
\pgfpathlineto{\pgfqpoint{4.033034in}{1.612496in}}%
\pgfpathlineto{\pgfqpoint{4.019217in}{1.620544in}}%
\pgfpathlineto{\pgfqpoint{4.005404in}{1.628616in}}%
\pgfpathlineto{\pgfqpoint{3.991595in}{1.636711in}}%
\pgfpathlineto{\pgfqpoint{3.999631in}{1.625819in}}%
\pgfpathlineto{\pgfqpoint{4.007654in}{1.615526in}}%
\pgfpathlineto{\pgfqpoint{4.015664in}{1.605822in}}%
\pgfpathlineto{\pgfqpoint{4.023662in}{1.596693in}}%
\pgfpathclose%
\pgfusepath{fill}%
\end{pgfscope}%
\begin{pgfscope}%
\pgfpathrectangle{\pgfqpoint{1.254980in}{0.150000in}}{\pgfqpoint{5.490039in}{5.490039in}}%
\pgfusepath{clip}%
\pgfsetbuttcap%
\pgfsetroundjoin%
\definecolor{currentfill}{rgb}{0.278791,0.062145,0.386592}%
\pgfsetfillcolor{currentfill}%
\pgfsetfillopacity{0.700000}%
\pgfsetlinewidth{0.000000pt}%
\definecolor{currentstroke}{rgb}{0.000000,0.000000,0.000000}%
\pgfsetstrokecolor{currentstroke}%
\pgfsetdash{}{0pt}%
\pgfpathmoveto{\pgfqpoint{4.867701in}{1.229630in}}%
\pgfpathlineto{\pgfqpoint{4.881662in}{1.224649in}}%
\pgfpathlineto{\pgfqpoint{4.895630in}{1.219690in}}%
\pgfpathlineto{\pgfqpoint{4.909604in}{1.214752in}}%
\pgfpathlineto{\pgfqpoint{4.923585in}{1.209837in}}%
\pgfpathlineto{\pgfqpoint{4.916029in}{1.206352in}}%
\pgfpathlineto{\pgfqpoint{4.908472in}{1.203221in}}%
\pgfpathlineto{\pgfqpoint{4.900913in}{1.200453in}}%
\pgfpathlineto{\pgfqpoint{4.893353in}{1.198058in}}%
\pgfpathlineto{\pgfqpoint{4.879357in}{1.203331in}}%
\pgfpathlineto{\pgfqpoint{4.865367in}{1.208626in}}%
\pgfpathlineto{\pgfqpoint{4.851383in}{1.213943in}}%
\pgfpathlineto{\pgfqpoint{4.837406in}{1.219282in}}%
\pgfpathlineto{\pgfqpoint{4.844982in}{1.221314in}}%
\pgfpathlineto{\pgfqpoint{4.852557in}{1.223722in}}%
\pgfpathlineto{\pgfqpoint{4.860130in}{1.226498in}}%
\pgfpathlineto{\pgfqpoint{4.867701in}{1.229630in}}%
\pgfpathclose%
\pgfusepath{fill}%
\end{pgfscope}%
\begin{pgfscope}%
\pgfpathrectangle{\pgfqpoint{1.254980in}{0.150000in}}{\pgfqpoint{5.490039in}{5.490039in}}%
\pgfusepath{clip}%
\pgfsetbuttcap%
\pgfsetroundjoin%
\definecolor{currentfill}{rgb}{0.165117,0.467423,0.558141}%
\pgfsetfillcolor{currentfill}%
\pgfsetfillopacity{0.700000}%
\pgfsetlinewidth{0.000000pt}%
\definecolor{currentstroke}{rgb}{0.000000,0.000000,0.000000}%
\pgfsetstrokecolor{currentstroke}%
\pgfsetdash{}{0pt}%
\pgfpathmoveto{\pgfqpoint{3.298907in}{2.125173in}}%
\pgfpathlineto{\pgfqpoint{3.312588in}{2.115421in}}%
\pgfpathlineto{\pgfqpoint{3.326272in}{2.105697in}}%
\pgfpathlineto{\pgfqpoint{3.339959in}{2.096000in}}%
\pgfpathlineto{\pgfqpoint{3.353649in}{2.086331in}}%
\pgfpathlineto{\pgfqpoint{3.345021in}{2.105347in}}%
\pgfpathlineto{\pgfqpoint{3.336367in}{2.125091in}}%
\pgfpathlineto{\pgfqpoint{3.327686in}{2.145576in}}%
\pgfpathlineto{\pgfqpoint{3.318977in}{2.166817in}}%
\pgfpathlineto{\pgfqpoint{3.305233in}{2.176946in}}%
\pgfpathlineto{\pgfqpoint{3.291492in}{2.187102in}}%
\pgfpathlineto{\pgfqpoint{3.277753in}{2.197287in}}%
\pgfpathlineto{\pgfqpoint{3.264017in}{2.207498in}}%
\pgfpathlineto{\pgfqpoint{3.272781in}{2.185790in}}%
\pgfpathlineto{\pgfqpoint{3.281517in}{2.164843in}}%
\pgfpathlineto{\pgfqpoint{3.290226in}{2.144642in}}%
\pgfpathlineto{\pgfqpoint{3.298907in}{2.125173in}}%
\pgfpathclose%
\pgfusepath{fill}%
\end{pgfscope}%
\begin{pgfscope}%
\pgfpathrectangle{\pgfqpoint{1.254980in}{0.150000in}}{\pgfqpoint{5.490039in}{5.490039in}}%
\pgfusepath{clip}%
\pgfsetbuttcap%
\pgfsetroundjoin%
\definecolor{currentfill}{rgb}{0.134692,0.658636,0.517649}%
\pgfsetfillcolor{currentfill}%
\pgfsetfillopacity{0.700000}%
\pgfsetlinewidth{0.000000pt}%
\definecolor{currentstroke}{rgb}{0.000000,0.000000,0.000000}%
\pgfsetstrokecolor{currentstroke}%
\pgfsetdash{}{0pt}%
\pgfpathmoveto{\pgfqpoint{2.716424in}{2.640433in}}%
\pgfpathlineto{\pgfqpoint{2.730078in}{2.628983in}}%
\pgfpathlineto{\pgfqpoint{2.743733in}{2.617568in}}%
\pgfpathlineto{\pgfqpoint{2.757390in}{2.606187in}}%
\pgfpathlineto{\pgfqpoint{2.771048in}{2.594840in}}%
\pgfpathlineto{\pgfqpoint{2.761698in}{2.621618in}}%
\pgfpathlineto{\pgfqpoint{2.752308in}{2.649232in}}%
\pgfpathlineto{\pgfqpoint{2.742876in}{2.677698in}}%
\pgfpathlineto{\pgfqpoint{2.733403in}{2.707033in}}%
\pgfpathlineto{\pgfqpoint{2.719677in}{2.718870in}}%
\pgfpathlineto{\pgfqpoint{2.705951in}{2.730742in}}%
\pgfpathlineto{\pgfqpoint{2.692227in}{2.742649in}}%
\pgfpathlineto{\pgfqpoint{2.678505in}{2.754590in}}%
\pgfpathlineto{\pgfqpoint{2.688048in}{2.724756in}}%
\pgfpathlineto{\pgfqpoint{2.697549in}{2.695796in}}%
\pgfpathlineto{\pgfqpoint{2.707007in}{2.667694in}}%
\pgfpathlineto{\pgfqpoint{2.716424in}{2.640433in}}%
\pgfpathclose%
\pgfusepath{fill}%
\end{pgfscope}%
\begin{pgfscope}%
\pgfpathrectangle{\pgfqpoint{1.254980in}{0.150000in}}{\pgfqpoint{5.490039in}{5.490039in}}%
\pgfusepath{clip}%
\pgfsetbuttcap%
\pgfsetroundjoin%
\definecolor{currentfill}{rgb}{0.220057,0.343307,0.549413}%
\pgfsetfillcolor{currentfill}%
\pgfsetfillopacity{0.700000}%
\pgfsetlinewidth{0.000000pt}%
\definecolor{currentstroke}{rgb}{0.000000,0.000000,0.000000}%
\pgfsetstrokecolor{currentstroke}%
\pgfsetdash{}{0pt}%
\pgfpathmoveto{\pgfqpoint{3.716290in}{1.803672in}}%
\pgfpathlineto{\pgfqpoint{3.730018in}{1.795091in}}%
\pgfpathlineto{\pgfqpoint{3.743749in}{1.786536in}}%
\pgfpathlineto{\pgfqpoint{3.757484in}{1.778005in}}%
\pgfpathlineto{\pgfqpoint{3.771223in}{1.769499in}}%
\pgfpathlineto{\pgfqpoint{3.763011in}{1.782720in}}%
\pgfpathlineto{\pgfqpoint{3.754782in}{1.796583in}}%
\pgfpathlineto{\pgfqpoint{3.746535in}{1.811101in}}%
\pgfpathlineto{\pgfqpoint{3.738270in}{1.826288in}}%
\pgfpathlineto{\pgfqpoint{3.724487in}{1.835230in}}%
\pgfpathlineto{\pgfqpoint{3.710708in}{1.844198in}}%
\pgfpathlineto{\pgfqpoint{3.696932in}{1.853190in}}%
\pgfpathlineto{\pgfqpoint{3.683160in}{1.862207in}}%
\pgfpathlineto{\pgfqpoint{3.691471in}{1.846578in}}%
\pgfpathlineto{\pgfqpoint{3.699763in}{1.831621in}}%
\pgfpathlineto{\pgfqpoint{3.708036in}{1.817323in}}%
\pgfpathlineto{\pgfqpoint{3.716290in}{1.803672in}}%
\pgfpathclose%
\pgfusepath{fill}%
\end{pgfscope}%
\begin{pgfscope}%
\pgfpathrectangle{\pgfqpoint{1.254980in}{0.150000in}}{\pgfqpoint{5.490039in}{5.490039in}}%
\pgfusepath{clip}%
\pgfsetbuttcap%
\pgfsetroundjoin%
\definecolor{currentfill}{rgb}{0.283187,0.125848,0.444960}%
\pgfsetfillcolor{currentfill}%
\pgfsetfillopacity{0.700000}%
\pgfsetlinewidth{0.000000pt}%
\definecolor{currentstroke}{rgb}{0.000000,0.000000,0.000000}%
\pgfsetstrokecolor{currentstroke}%
\pgfsetdash{}{0pt}%
\pgfpathmoveto{\pgfqpoint{4.528432in}{1.333986in}}%
\pgfpathlineto{\pgfqpoint{4.542309in}{1.327830in}}%
\pgfpathlineto{\pgfqpoint{4.556191in}{1.321697in}}%
\pgfpathlineto{\pgfqpoint{4.570080in}{1.315586in}}%
\pgfpathlineto{\pgfqpoint{4.583974in}{1.309497in}}%
\pgfpathlineto{\pgfqpoint{4.576303in}{1.311016in}}%
\pgfpathlineto{\pgfqpoint{4.568627in}{1.312986in}}%
\pgfpathlineto{\pgfqpoint{4.560946in}{1.315418in}}%
\pgfpathlineto{\pgfqpoint{4.553261in}{1.318322in}}%
\pgfpathlineto{\pgfqpoint{4.539343in}{1.324796in}}%
\pgfpathlineto{\pgfqpoint{4.525431in}{1.331292in}}%
\pgfpathlineto{\pgfqpoint{4.511524in}{1.337811in}}%
\pgfpathlineto{\pgfqpoint{4.497623in}{1.344352in}}%
\pgfpathlineto{\pgfqpoint{4.505333in}{1.341057in}}%
\pgfpathlineto{\pgfqpoint{4.513038in}{1.338238in}}%
\pgfpathlineto{\pgfqpoint{4.520737in}{1.335884in}}%
\pgfpathlineto{\pgfqpoint{4.528432in}{1.333986in}}%
\pgfpathclose%
\pgfusepath{fill}%
\end{pgfscope}%
\begin{pgfscope}%
\pgfpathrectangle{\pgfqpoint{1.254980in}{0.150000in}}{\pgfqpoint{5.490039in}{5.490039in}}%
\pgfusepath{clip}%
\pgfsetbuttcap%
\pgfsetroundjoin%
\definecolor{currentfill}{rgb}{0.126326,0.644107,0.525311}%
\pgfsetfillcolor{currentfill}%
\pgfsetfillopacity{0.700000}%
\pgfsetlinewidth{0.000000pt}%
\definecolor{currentstroke}{rgb}{0.000000,0.000000,0.000000}%
\pgfsetstrokecolor{currentstroke}%
\pgfsetdash{}{0pt}%
\pgfpathmoveto{\pgfqpoint{2.771048in}{2.594840in}}%
\pgfpathlineto{\pgfqpoint{2.784708in}{2.583528in}}%
\pgfpathlineto{\pgfqpoint{2.798370in}{2.572249in}}%
\pgfpathlineto{\pgfqpoint{2.812033in}{2.561004in}}%
\pgfpathlineto{\pgfqpoint{2.825697in}{2.549792in}}%
\pgfpathlineto{\pgfqpoint{2.816413in}{2.576088in}}%
\pgfpathlineto{\pgfqpoint{2.807089in}{2.603214in}}%
\pgfpathlineto{\pgfqpoint{2.797726in}{2.631189in}}%
\pgfpathlineto{\pgfqpoint{2.788322in}{2.660026in}}%
\pgfpathlineto{\pgfqpoint{2.774590in}{2.671727in}}%
\pgfpathlineto{\pgfqpoint{2.760860in}{2.683462in}}%
\pgfpathlineto{\pgfqpoint{2.747131in}{2.695230in}}%
\pgfpathlineto{\pgfqpoint{2.733403in}{2.707033in}}%
\pgfpathlineto{\pgfqpoint{2.742876in}{2.677698in}}%
\pgfpathlineto{\pgfqpoint{2.752308in}{2.649232in}}%
\pgfpathlineto{\pgfqpoint{2.761698in}{2.621618in}}%
\pgfpathlineto{\pgfqpoint{2.771048in}{2.594840in}}%
\pgfpathclose%
\pgfusepath{fill}%
\end{pgfscope}%
\begin{pgfscope}%
\pgfpathrectangle{\pgfqpoint{1.254980in}{0.150000in}}{\pgfqpoint{5.490039in}{5.490039in}}%
\pgfusepath{clip}%
\pgfsetbuttcap%
\pgfsetroundjoin%
\definecolor{currentfill}{rgb}{0.281924,0.089666,0.412415}%
\pgfsetfillcolor{currentfill}%
\pgfsetfillopacity{0.700000}%
\pgfsetlinewidth{0.000000pt}%
\definecolor{currentstroke}{rgb}{0.000000,0.000000,0.000000}%
\pgfsetstrokecolor{currentstroke}%
\pgfsetdash{}{0pt}%
\pgfpathmoveto{\pgfqpoint{4.725814in}{1.262792in}}%
\pgfpathlineto{\pgfqpoint{4.739741in}{1.257275in}}%
\pgfpathlineto{\pgfqpoint{4.753675in}{1.251781in}}%
\pgfpathlineto{\pgfqpoint{4.767614in}{1.246309in}}%
\pgfpathlineto{\pgfqpoint{4.781560in}{1.240859in}}%
\pgfpathlineto{\pgfqpoint{4.773964in}{1.239579in}}%
\pgfpathlineto{\pgfqpoint{4.766365in}{1.238699in}}%
\pgfpathlineto{\pgfqpoint{4.758764in}{1.238228in}}%
\pgfpathlineto{\pgfqpoint{4.751160in}{1.238176in}}%
\pgfpathlineto{\pgfqpoint{4.737195in}{1.243997in}}%
\pgfpathlineto{\pgfqpoint{4.723236in}{1.249841in}}%
\pgfpathlineto{\pgfqpoint{4.709283in}{1.255706in}}%
\pgfpathlineto{\pgfqpoint{4.695336in}{1.261594in}}%
\pgfpathlineto{\pgfqpoint{4.702961in}{1.261269in}}%
\pgfpathlineto{\pgfqpoint{4.710581in}{1.261367in}}%
\pgfpathlineto{\pgfqpoint{4.718199in}{1.261878in}}%
\pgfpathlineto{\pgfqpoint{4.725814in}{1.262792in}}%
\pgfpathclose%
\pgfusepath{fill}%
\end{pgfscope}%
\begin{pgfscope}%
\pgfpathrectangle{\pgfqpoint{1.254980in}{0.150000in}}{\pgfqpoint{5.490039in}{5.490039in}}%
\pgfusepath{clip}%
\pgfsetbuttcap%
\pgfsetroundjoin%
\definecolor{currentfill}{rgb}{0.169646,0.456262,0.558030}%
\pgfsetfillcolor{currentfill}%
\pgfsetfillopacity{0.700000}%
\pgfsetlinewidth{0.000000pt}%
\definecolor{currentstroke}{rgb}{0.000000,0.000000,0.000000}%
\pgfsetstrokecolor{currentstroke}%
\pgfsetdash{}{0pt}%
\pgfpathmoveto{\pgfqpoint{3.353649in}{2.086331in}}%
\pgfpathlineto{\pgfqpoint{3.367342in}{2.076689in}}%
\pgfpathlineto{\pgfqpoint{3.381038in}{2.067074in}}%
\pgfpathlineto{\pgfqpoint{3.394737in}{2.057486in}}%
\pgfpathlineto{\pgfqpoint{3.408439in}{2.047925in}}%
\pgfpathlineto{\pgfqpoint{3.399863in}{2.066489in}}%
\pgfpathlineto{\pgfqpoint{3.391261in}{2.085776in}}%
\pgfpathlineto{\pgfqpoint{3.382634in}{2.105800in}}%
\pgfpathlineto{\pgfqpoint{3.373981in}{2.126575in}}%
\pgfpathlineto{\pgfqpoint{3.360226in}{2.136595in}}%
\pgfpathlineto{\pgfqpoint{3.346474in}{2.146642in}}%
\pgfpathlineto{\pgfqpoint{3.332724in}{2.156716in}}%
\pgfpathlineto{\pgfqpoint{3.318977in}{2.166817in}}%
\pgfpathlineto{\pgfqpoint{3.327686in}{2.145576in}}%
\pgfpathlineto{\pgfqpoint{3.336367in}{2.125091in}}%
\pgfpathlineto{\pgfqpoint{3.345021in}{2.105347in}}%
\pgfpathlineto{\pgfqpoint{3.353649in}{2.086331in}}%
\pgfpathclose%
\pgfusepath{fill}%
\end{pgfscope}%
\begin{pgfscope}%
\pgfpathrectangle{\pgfqpoint{1.254980in}{0.150000in}}{\pgfqpoint{5.490039in}{5.490039in}}%
\pgfusepath{clip}%
\pgfsetbuttcap%
\pgfsetroundjoin%
\definecolor{currentfill}{rgb}{0.278826,0.175490,0.483397}%
\pgfsetfillcolor{currentfill}%
\pgfsetfillopacity{0.700000}%
\pgfsetlinewidth{0.000000pt}%
\definecolor{currentstroke}{rgb}{0.000000,0.000000,0.000000}%
\pgfsetstrokecolor{currentstroke}%
\pgfsetdash{}{0pt}%
\pgfpathmoveto{\pgfqpoint{4.331225in}{1.424607in}}%
\pgfpathlineto{\pgfqpoint{4.345063in}{1.417795in}}%
\pgfpathlineto{\pgfqpoint{4.358905in}{1.411005in}}%
\pgfpathlineto{\pgfqpoint{4.372753in}{1.404238in}}%
\pgfpathlineto{\pgfqpoint{4.386606in}{1.397493in}}%
\pgfpathlineto{\pgfqpoint{4.378836in}{1.402065in}}%
\pgfpathlineto{\pgfqpoint{4.371060in}{1.407143in}}%
\pgfpathlineto{\pgfqpoint{4.363276in}{1.412738in}}%
\pgfpathlineto{\pgfqpoint{4.355484in}{1.418861in}}%
\pgfpathlineto{\pgfqpoint{4.341602in}{1.426006in}}%
\pgfpathlineto{\pgfqpoint{4.327726in}{1.433174in}}%
\pgfpathlineto{\pgfqpoint{4.313854in}{1.440365in}}%
\pgfpathlineto{\pgfqpoint{4.299987in}{1.447578in}}%
\pgfpathlineto{\pgfqpoint{4.307808in}{1.441048in}}%
\pgfpathlineto{\pgfqpoint{4.315622in}{1.435051in}}%
\pgfpathlineto{\pgfqpoint{4.323427in}{1.429574in}}%
\pgfpathlineto{\pgfqpoint{4.331225in}{1.424607in}}%
\pgfpathclose%
\pgfusepath{fill}%
\end{pgfscope}%
\begin{pgfscope}%
\pgfpathrectangle{\pgfqpoint{1.254980in}{0.150000in}}{\pgfqpoint{5.490039in}{5.490039in}}%
\pgfusepath{clip}%
\pgfsetbuttcap%
\pgfsetroundjoin%
\definecolor{currentfill}{rgb}{0.262138,0.242286,0.520837}%
\pgfsetfillcolor{currentfill}%
\pgfsetfillopacity{0.700000}%
\pgfsetlinewidth{0.000000pt}%
\definecolor{currentstroke}{rgb}{0.000000,0.000000,0.000000}%
\pgfsetstrokecolor{currentstroke}%
\pgfsetdash{}{0pt}%
\pgfpathmoveto{\pgfqpoint{4.078777in}{1.566125in}}%
\pgfpathlineto{\pgfqpoint{4.092567in}{1.558541in}}%
\pgfpathlineto{\pgfqpoint{4.106362in}{1.550981in}}%
\pgfpathlineto{\pgfqpoint{4.120161in}{1.543445in}}%
\pgfpathlineto{\pgfqpoint{4.133965in}{1.535931in}}%
\pgfpathlineto{\pgfqpoint{4.126037in}{1.544238in}}%
\pgfpathlineto{\pgfqpoint{4.118098in}{1.553111in}}%
\pgfpathlineto{\pgfqpoint{4.110148in}{1.562564in}}%
\pgfpathlineto{\pgfqpoint{4.102186in}{1.572610in}}%
\pgfpathlineto{\pgfqpoint{4.088347in}{1.580540in}}%
\pgfpathlineto{\pgfqpoint{4.074512in}{1.588494in}}%
\pgfpathlineto{\pgfqpoint{4.060682in}{1.596471in}}%
\pgfpathlineto{\pgfqpoint{4.046856in}{1.604472in}}%
\pgfpathlineto{\pgfqpoint{4.054854in}{1.594003in}}%
\pgfpathlineto{\pgfqpoint{4.062840in}{1.584131in}}%
\pgfpathlineto{\pgfqpoint{4.070814in}{1.574842in}}%
\pgfpathlineto{\pgfqpoint{4.078777in}{1.566125in}}%
\pgfpathclose%
\pgfusepath{fill}%
\end{pgfscope}%
\begin{pgfscope}%
\pgfpathrectangle{\pgfqpoint{1.254980in}{0.150000in}}{\pgfqpoint{5.490039in}{5.490039in}}%
\pgfusepath{clip}%
\pgfsetbuttcap%
\pgfsetroundjoin%
\definecolor{currentfill}{rgb}{0.223925,0.334994,0.548053}%
\pgfsetfillcolor{currentfill}%
\pgfsetfillopacity{0.700000}%
\pgfsetlinewidth{0.000000pt}%
\definecolor{currentstroke}{rgb}{0.000000,0.000000,0.000000}%
\pgfsetstrokecolor{currentstroke}%
\pgfsetdash{}{0pt}%
\pgfpathmoveto{\pgfqpoint{3.771223in}{1.769499in}}%
\pgfpathlineto{\pgfqpoint{3.784966in}{1.761017in}}%
\pgfpathlineto{\pgfqpoint{3.798713in}{1.752560in}}%
\pgfpathlineto{\pgfqpoint{3.812464in}{1.744128in}}%
\pgfpathlineto{\pgfqpoint{3.826219in}{1.735720in}}%
\pgfpathlineto{\pgfqpoint{3.818049in}{1.748511in}}%
\pgfpathlineto{\pgfqpoint{3.809862in}{1.761941in}}%
\pgfpathlineto{\pgfqpoint{3.801659in}{1.776020in}}%
\pgfpathlineto{\pgfqpoint{3.793438in}{1.790764in}}%
\pgfpathlineto{\pgfqpoint{3.779641in}{1.799608in}}%
\pgfpathlineto{\pgfqpoint{3.765847in}{1.808477in}}%
\pgfpathlineto{\pgfqpoint{3.752056in}{1.817370in}}%
\pgfpathlineto{\pgfqpoint{3.738270in}{1.826288in}}%
\pgfpathlineto{\pgfqpoint{3.746535in}{1.811101in}}%
\pgfpathlineto{\pgfqpoint{3.754782in}{1.796583in}}%
\pgfpathlineto{\pgfqpoint{3.763011in}{1.782720in}}%
\pgfpathlineto{\pgfqpoint{3.771223in}{1.769499in}}%
\pgfpathclose%
\pgfusepath{fill}%
\end{pgfscope}%
\begin{pgfscope}%
\pgfpathrectangle{\pgfqpoint{1.254980in}{0.150000in}}{\pgfqpoint{5.490039in}{5.490039in}}%
\pgfusepath{clip}%
\pgfsetbuttcap%
\pgfsetroundjoin%
\definecolor{currentfill}{rgb}{0.121380,0.629492,0.531973}%
\pgfsetfillcolor{currentfill}%
\pgfsetfillopacity{0.700000}%
\pgfsetlinewidth{0.000000pt}%
\definecolor{currentstroke}{rgb}{0.000000,0.000000,0.000000}%
\pgfsetstrokecolor{currentstroke}%
\pgfsetdash{}{0pt}%
\pgfpathmoveto{\pgfqpoint{2.825697in}{2.549792in}}%
\pgfpathlineto{\pgfqpoint{2.839363in}{2.538613in}}%
\pgfpathlineto{\pgfqpoint{2.853031in}{2.527468in}}%
\pgfpathlineto{\pgfqpoint{2.866701in}{2.516355in}}%
\pgfpathlineto{\pgfqpoint{2.880373in}{2.505274in}}%
\pgfpathlineto{\pgfqpoint{2.871153in}{2.531089in}}%
\pgfpathlineto{\pgfqpoint{2.861895in}{2.557730in}}%
\pgfpathlineto{\pgfqpoint{2.852599in}{2.585213in}}%
\pgfpathlineto{\pgfqpoint{2.843264in}{2.613554in}}%
\pgfpathlineto{\pgfqpoint{2.829526in}{2.625123in}}%
\pgfpathlineto{\pgfqpoint{2.815790in}{2.636724in}}%
\pgfpathlineto{\pgfqpoint{2.802056in}{2.648358in}}%
\pgfpathlineto{\pgfqpoint{2.788322in}{2.660026in}}%
\pgfpathlineto{\pgfqpoint{2.797726in}{2.631189in}}%
\pgfpathlineto{\pgfqpoint{2.807089in}{2.603214in}}%
\pgfpathlineto{\pgfqpoint{2.816413in}{2.576088in}}%
\pgfpathlineto{\pgfqpoint{2.825697in}{2.549792in}}%
\pgfpathclose%
\pgfusepath{fill}%
\end{pgfscope}%
\begin{pgfscope}%
\pgfpathrectangle{\pgfqpoint{1.254980in}{0.150000in}}{\pgfqpoint{5.490039in}{5.490039in}}%
\pgfusepath{clip}%
\pgfsetbuttcap%
\pgfsetroundjoin%
\definecolor{currentfill}{rgb}{0.278791,0.062145,0.386592}%
\pgfsetfillcolor{currentfill}%
\pgfsetfillopacity{0.700000}%
\pgfsetlinewidth{0.000000pt}%
\definecolor{currentstroke}{rgb}{0.000000,0.000000,0.000000}%
\pgfsetstrokecolor{currentstroke}%
\pgfsetdash{}{0pt}%
\pgfpathmoveto{\pgfqpoint{4.923585in}{1.209837in}}%
\pgfpathlineto{\pgfqpoint{4.937573in}{1.204944in}}%
\pgfpathlineto{\pgfqpoint{4.951567in}{1.200074in}}%
\pgfpathlineto{\pgfqpoint{4.965568in}{1.195225in}}%
\pgfpathlineto{\pgfqpoint{4.958022in}{1.191476in}}%
\pgfpathlineto{\pgfqpoint{4.950476in}{1.188077in}}%
\pgfpathlineto{\pgfqpoint{4.942929in}{1.185040in}}%
\pgfpathlineto{\pgfqpoint{4.935381in}{1.182372in}}%
\pgfpathlineto{\pgfqpoint{4.921365in}{1.187579in}}%
\pgfpathlineto{\pgfqpoint{4.907356in}{1.192807in}}%
\pgfpathlineto{\pgfqpoint{4.893353in}{1.198058in}}%
\pgfpathlineto{\pgfqpoint{4.900913in}{1.200453in}}%
\pgfpathlineto{\pgfqpoint{4.908472in}{1.203221in}}%
\pgfpathlineto{\pgfqpoint{4.916029in}{1.206352in}}%
\pgfpathlineto{\pgfqpoint{4.923585in}{1.209837in}}%
\pgfpathclose%
\pgfusepath{fill}%
\end{pgfscope}%
\begin{pgfscope}%
\pgfpathrectangle{\pgfqpoint{1.254980in}{0.150000in}}{\pgfqpoint{5.490039in}{5.490039in}}%
\pgfusepath{clip}%
\pgfsetbuttcap%
\pgfsetroundjoin%
\definecolor{currentfill}{rgb}{0.174274,0.445044,0.557792}%
\pgfsetfillcolor{currentfill}%
\pgfsetfillopacity{0.700000}%
\pgfsetlinewidth{0.000000pt}%
\definecolor{currentstroke}{rgb}{0.000000,0.000000,0.000000}%
\pgfsetstrokecolor{currentstroke}%
\pgfsetdash{}{0pt}%
\pgfpathmoveto{\pgfqpoint{3.408439in}{2.047925in}}%
\pgfpathlineto{\pgfqpoint{3.422144in}{2.038390in}}%
\pgfpathlineto{\pgfqpoint{3.435853in}{2.028883in}}%
\pgfpathlineto{\pgfqpoint{3.449564in}{2.019402in}}%
\pgfpathlineto{\pgfqpoint{3.463278in}{2.009947in}}%
\pgfpathlineto{\pgfqpoint{3.454753in}{2.028061in}}%
\pgfpathlineto{\pgfqpoint{3.446204in}{2.046892in}}%
\pgfpathlineto{\pgfqpoint{3.437630in}{2.066456in}}%
\pgfpathlineto{\pgfqpoint{3.429031in}{2.086766in}}%
\pgfpathlineto{\pgfqpoint{3.415264in}{2.096678in}}%
\pgfpathlineto{\pgfqpoint{3.401500in}{2.106617in}}%
\pgfpathlineto{\pgfqpoint{3.387739in}{2.116583in}}%
\pgfpathlineto{\pgfqpoint{3.373981in}{2.126575in}}%
\pgfpathlineto{\pgfqpoint{3.382634in}{2.105800in}}%
\pgfpathlineto{\pgfqpoint{3.391261in}{2.085776in}}%
\pgfpathlineto{\pgfqpoint{3.399863in}{2.066489in}}%
\pgfpathlineto{\pgfqpoint{3.408439in}{2.047925in}}%
\pgfpathclose%
\pgfusepath{fill}%
\end{pgfscope}%
\begin{pgfscope}%
\pgfpathrectangle{\pgfqpoint{1.254980in}{0.150000in}}{\pgfqpoint{5.490039in}{5.490039in}}%
\pgfusepath{clip}%
\pgfsetbuttcap%
\pgfsetroundjoin%
\definecolor{currentfill}{rgb}{0.283229,0.120777,0.440584}%
\pgfsetfillcolor{currentfill}%
\pgfsetfillopacity{0.700000}%
\pgfsetlinewidth{0.000000pt}%
\definecolor{currentstroke}{rgb}{0.000000,0.000000,0.000000}%
\pgfsetstrokecolor{currentstroke}%
\pgfsetdash{}{0pt}%
\pgfpathmoveto{\pgfqpoint{4.583974in}{1.309497in}}%
\pgfpathlineto{\pgfqpoint{4.597874in}{1.303431in}}%
\pgfpathlineto{\pgfqpoint{4.611780in}{1.297388in}}%
\pgfpathlineto{\pgfqpoint{4.625691in}{1.291366in}}%
\pgfpathlineto{\pgfqpoint{4.639608in}{1.285367in}}%
\pgfpathlineto{\pgfqpoint{4.631959in}{1.286506in}}%
\pgfpathlineto{\pgfqpoint{4.624307in}{1.288092in}}%
\pgfpathlineto{\pgfqpoint{4.616650in}{1.290136in}}%
\pgfpathlineto{\pgfqpoint{4.608989in}{1.292650in}}%
\pgfpathlineto{\pgfqpoint{4.595048in}{1.299034in}}%
\pgfpathlineto{\pgfqpoint{4.581114in}{1.305441in}}%
\pgfpathlineto{\pgfqpoint{4.567185in}{1.311870in}}%
\pgfpathlineto{\pgfqpoint{4.553261in}{1.318322in}}%
\pgfpathlineto{\pgfqpoint{4.560946in}{1.315418in}}%
\pgfpathlineto{\pgfqpoint{4.568627in}{1.312986in}}%
\pgfpathlineto{\pgfqpoint{4.576303in}{1.311016in}}%
\pgfpathlineto{\pgfqpoint{4.583974in}{1.309497in}}%
\pgfpathclose%
\pgfusepath{fill}%
\end{pgfscope}%
\begin{pgfscope}%
\pgfpathrectangle{\pgfqpoint{1.254980in}{0.150000in}}{\pgfqpoint{5.490039in}{5.490039in}}%
\pgfusepath{clip}%
\pgfsetbuttcap%
\pgfsetroundjoin%
\definecolor{currentfill}{rgb}{0.119483,0.614817,0.537692}%
\pgfsetfillcolor{currentfill}%
\pgfsetfillopacity{0.700000}%
\pgfsetlinewidth{0.000000pt}%
\definecolor{currentstroke}{rgb}{0.000000,0.000000,0.000000}%
\pgfsetstrokecolor{currentstroke}%
\pgfsetdash{}{0pt}%
\pgfpathmoveto{\pgfqpoint{2.880373in}{2.505274in}}%
\pgfpathlineto{\pgfqpoint{2.894046in}{2.494226in}}%
\pgfpathlineto{\pgfqpoint{2.907721in}{2.483210in}}%
\pgfpathlineto{\pgfqpoint{2.921398in}{2.472226in}}%
\pgfpathlineto{\pgfqpoint{2.935077in}{2.461273in}}%
\pgfpathlineto{\pgfqpoint{2.925921in}{2.486609in}}%
\pgfpathlineto{\pgfqpoint{2.916728in}{2.512766in}}%
\pgfpathlineto{\pgfqpoint{2.907499in}{2.539759in}}%
\pgfpathlineto{\pgfqpoint{2.898231in}{2.567605in}}%
\pgfpathlineto{\pgfqpoint{2.884487in}{2.579044in}}%
\pgfpathlineto{\pgfqpoint{2.870744in}{2.590515in}}%
\pgfpathlineto{\pgfqpoint{2.857003in}{2.602019in}}%
\pgfpathlineto{\pgfqpoint{2.843264in}{2.613554in}}%
\pgfpathlineto{\pgfqpoint{2.852599in}{2.585213in}}%
\pgfpathlineto{\pgfqpoint{2.861895in}{2.557730in}}%
\pgfpathlineto{\pgfqpoint{2.871153in}{2.531089in}}%
\pgfpathlineto{\pgfqpoint{2.880373in}{2.505274in}}%
\pgfpathclose%
\pgfusepath{fill}%
\end{pgfscope}%
\begin{pgfscope}%
\pgfpathrectangle{\pgfqpoint{1.254980in}{0.150000in}}{\pgfqpoint{5.490039in}{5.490039in}}%
\pgfusepath{clip}%
\pgfsetbuttcap%
\pgfsetroundjoin%
\definecolor{currentfill}{rgb}{0.281446,0.084320,0.407414}%
\pgfsetfillcolor{currentfill}%
\pgfsetfillopacity{0.700000}%
\pgfsetlinewidth{0.000000pt}%
\definecolor{currentstroke}{rgb}{0.000000,0.000000,0.000000}%
\pgfsetstrokecolor{currentstroke}%
\pgfsetdash{}{0pt}%
\pgfpathmoveto{\pgfqpoint{4.781560in}{1.240859in}}%
\pgfpathlineto{\pgfqpoint{4.795512in}{1.235432in}}%
\pgfpathlineto{\pgfqpoint{4.809470in}{1.230026in}}%
\pgfpathlineto{\pgfqpoint{4.823435in}{1.224643in}}%
\pgfpathlineto{\pgfqpoint{4.837406in}{1.219282in}}%
\pgfpathlineto{\pgfqpoint{4.829827in}{1.217635in}}%
\pgfpathlineto{\pgfqpoint{4.822247in}{1.216385in}}%
\pgfpathlineto{\pgfqpoint{4.814664in}{1.215541in}}%
\pgfpathlineto{\pgfqpoint{4.807080in}{1.215113in}}%
\pgfpathlineto{\pgfqpoint{4.793091in}{1.220846in}}%
\pgfpathlineto{\pgfqpoint{4.779108in}{1.226600in}}%
\pgfpathlineto{\pgfqpoint{4.765131in}{1.232377in}}%
\pgfpathlineto{\pgfqpoint{4.751160in}{1.238176in}}%
\pgfpathlineto{\pgfqpoint{4.758764in}{1.238228in}}%
\pgfpathlineto{\pgfqpoint{4.766365in}{1.238699in}}%
\pgfpathlineto{\pgfqpoint{4.773964in}{1.239579in}}%
\pgfpathlineto{\pgfqpoint{4.781560in}{1.240859in}}%
\pgfpathclose%
\pgfusepath{fill}%
\end{pgfscope}%
\begin{pgfscope}%
\pgfpathrectangle{\pgfqpoint{1.254980in}{0.150000in}}{\pgfqpoint{5.490039in}{5.490039in}}%
\pgfusepath{clip}%
\pgfsetbuttcap%
\pgfsetroundjoin%
\definecolor{currentfill}{rgb}{0.229739,0.322361,0.545706}%
\pgfsetfillcolor{currentfill}%
\pgfsetfillopacity{0.700000}%
\pgfsetlinewidth{0.000000pt}%
\definecolor{currentstroke}{rgb}{0.000000,0.000000,0.000000}%
\pgfsetstrokecolor{currentstroke}%
\pgfsetdash{}{0pt}%
\pgfpathmoveto{\pgfqpoint{3.826219in}{1.735720in}}%
\pgfpathlineto{\pgfqpoint{3.839978in}{1.727337in}}%
\pgfpathlineto{\pgfqpoint{3.853740in}{1.718977in}}%
\pgfpathlineto{\pgfqpoint{3.867507in}{1.710643in}}%
\pgfpathlineto{\pgfqpoint{3.881278in}{1.702332in}}%
\pgfpathlineto{\pgfqpoint{3.873149in}{1.714694in}}%
\pgfpathlineto{\pgfqpoint{3.865005in}{1.727689in}}%
\pgfpathlineto{\pgfqpoint{3.856844in}{1.741332in}}%
\pgfpathlineto{\pgfqpoint{3.848667in}{1.755633in}}%
\pgfpathlineto{\pgfqpoint{3.834854in}{1.764380in}}%
\pgfpathlineto{\pgfqpoint{3.821045in}{1.773150in}}%
\pgfpathlineto{\pgfqpoint{3.807240in}{1.781945in}}%
\pgfpathlineto{\pgfqpoint{3.793438in}{1.790764in}}%
\pgfpathlineto{\pgfqpoint{3.801659in}{1.776020in}}%
\pgfpathlineto{\pgfqpoint{3.809862in}{1.761941in}}%
\pgfpathlineto{\pgfqpoint{3.818049in}{1.748511in}}%
\pgfpathlineto{\pgfqpoint{3.826219in}{1.735720in}}%
\pgfpathclose%
\pgfusepath{fill}%
\end{pgfscope}%
\begin{pgfscope}%
\pgfpathrectangle{\pgfqpoint{1.254980in}{0.150000in}}{\pgfqpoint{5.490039in}{5.490039in}}%
\pgfusepath{clip}%
\pgfsetbuttcap%
\pgfsetroundjoin%
\definecolor{currentfill}{rgb}{0.265145,0.232956,0.516599}%
\pgfsetfillcolor{currentfill}%
\pgfsetfillopacity{0.700000}%
\pgfsetlinewidth{0.000000pt}%
\definecolor{currentstroke}{rgb}{0.000000,0.000000,0.000000}%
\pgfsetstrokecolor{currentstroke}%
\pgfsetdash{}{0pt}%
\pgfpathmoveto{\pgfqpoint{4.133965in}{1.535931in}}%
\pgfpathlineto{\pgfqpoint{4.147774in}{1.528442in}}%
\pgfpathlineto{\pgfqpoint{4.161587in}{1.520975in}}%
\pgfpathlineto{\pgfqpoint{4.175406in}{1.513531in}}%
\pgfpathlineto{\pgfqpoint{4.189229in}{1.506111in}}%
\pgfpathlineto{\pgfqpoint{4.181334in}{1.514006in}}%
\pgfpathlineto{\pgfqpoint{4.173429in}{1.522465in}}%
\pgfpathlineto{\pgfqpoint{4.165514in}{1.531499in}}%
\pgfpathlineto{\pgfqpoint{4.157588in}{1.541121in}}%
\pgfpathlineto{\pgfqpoint{4.143731in}{1.548958in}}%
\pgfpathlineto{\pgfqpoint{4.129878in}{1.556819in}}%
\pgfpathlineto{\pgfqpoint{4.116030in}{1.564702in}}%
\pgfpathlineto{\pgfqpoint{4.102186in}{1.572610in}}%
\pgfpathlineto{\pgfqpoint{4.110148in}{1.562564in}}%
\pgfpathlineto{\pgfqpoint{4.118098in}{1.553111in}}%
\pgfpathlineto{\pgfqpoint{4.126037in}{1.544238in}}%
\pgfpathlineto{\pgfqpoint{4.133965in}{1.535931in}}%
\pgfpathclose%
\pgfusepath{fill}%
\end{pgfscope}%
\begin{pgfscope}%
\pgfpathrectangle{\pgfqpoint{1.254980in}{0.150000in}}{\pgfqpoint{5.490039in}{5.490039in}}%
\pgfusepath{clip}%
\pgfsetbuttcap%
\pgfsetroundjoin%
\definecolor{currentfill}{rgb}{0.280255,0.165693,0.476498}%
\pgfsetfillcolor{currentfill}%
\pgfsetfillopacity{0.700000}%
\pgfsetlinewidth{0.000000pt}%
\definecolor{currentstroke}{rgb}{0.000000,0.000000,0.000000}%
\pgfsetstrokecolor{currentstroke}%
\pgfsetdash{}{0pt}%
\pgfpathmoveto{\pgfqpoint{4.386606in}{1.397493in}}%
\pgfpathlineto{\pgfqpoint{4.400464in}{1.390771in}}%
\pgfpathlineto{\pgfqpoint{4.414328in}{1.384072in}}%
\pgfpathlineto{\pgfqpoint{4.428197in}{1.377396in}}%
\pgfpathlineto{\pgfqpoint{4.442071in}{1.370742in}}%
\pgfpathlineto{\pgfqpoint{4.434329in}{1.374919in}}%
\pgfpathlineto{\pgfqpoint{4.426580in}{1.379598in}}%
\pgfpathlineto{\pgfqpoint{4.418825in}{1.384790in}}%
\pgfpathlineto{\pgfqpoint{4.411063in}{1.390507in}}%
\pgfpathlineto{\pgfqpoint{4.397161in}{1.397562in}}%
\pgfpathlineto{\pgfqpoint{4.383264in}{1.404639in}}%
\pgfpathlineto{\pgfqpoint{4.369371in}{1.411738in}}%
\pgfpathlineto{\pgfqpoint{4.355484in}{1.418861in}}%
\pgfpathlineto{\pgfqpoint{4.363276in}{1.412738in}}%
\pgfpathlineto{\pgfqpoint{4.371060in}{1.407143in}}%
\pgfpathlineto{\pgfqpoint{4.378836in}{1.402065in}}%
\pgfpathlineto{\pgfqpoint{4.386606in}{1.397493in}}%
\pgfpathclose%
\pgfusepath{fill}%
\end{pgfscope}%
\begin{pgfscope}%
\pgfpathrectangle{\pgfqpoint{1.254980in}{0.150000in}}{\pgfqpoint{5.490039in}{5.490039in}}%
\pgfusepath{clip}%
\pgfsetbuttcap%
\pgfsetroundjoin%
\definecolor{currentfill}{rgb}{0.119738,0.603785,0.541400}%
\pgfsetfillcolor{currentfill}%
\pgfsetfillopacity{0.700000}%
\pgfsetlinewidth{0.000000pt}%
\definecolor{currentstroke}{rgb}{0.000000,0.000000,0.000000}%
\pgfsetstrokecolor{currentstroke}%
\pgfsetdash{}{0pt}%
\pgfpathmoveto{\pgfqpoint{2.935077in}{2.461273in}}%
\pgfpathlineto{\pgfqpoint{2.948757in}{2.450353in}}%
\pgfpathlineto{\pgfqpoint{2.962440in}{2.439463in}}%
\pgfpathlineto{\pgfqpoint{2.976125in}{2.428605in}}%
\pgfpathlineto{\pgfqpoint{2.989812in}{2.417778in}}%
\pgfpathlineto{\pgfqpoint{2.980719in}{2.442635in}}%
\pgfpathlineto{\pgfqpoint{2.971590in}{2.468309in}}%
\pgfpathlineto{\pgfqpoint{2.962426in}{2.494814in}}%
\pgfpathlineto{\pgfqpoint{2.953224in}{2.522166in}}%
\pgfpathlineto{\pgfqpoint{2.939473in}{2.533479in}}%
\pgfpathlineto{\pgfqpoint{2.925724in}{2.544823in}}%
\pgfpathlineto{\pgfqpoint{2.911976in}{2.556198in}}%
\pgfpathlineto{\pgfqpoint{2.898231in}{2.567605in}}%
\pgfpathlineto{\pgfqpoint{2.907499in}{2.539759in}}%
\pgfpathlineto{\pgfqpoint{2.916728in}{2.512766in}}%
\pgfpathlineto{\pgfqpoint{2.925921in}{2.486609in}}%
\pgfpathlineto{\pgfqpoint{2.935077in}{2.461273in}}%
\pgfpathclose%
\pgfusepath{fill}%
\end{pgfscope}%
\begin{pgfscope}%
\pgfpathrectangle{\pgfqpoint{1.254980in}{0.150000in}}{\pgfqpoint{5.490039in}{5.490039in}}%
\pgfusepath{clip}%
\pgfsetbuttcap%
\pgfsetroundjoin%
\definecolor{currentfill}{rgb}{0.179019,0.433756,0.557430}%
\pgfsetfillcolor{currentfill}%
\pgfsetfillopacity{0.700000}%
\pgfsetlinewidth{0.000000pt}%
\definecolor{currentstroke}{rgb}{0.000000,0.000000,0.000000}%
\pgfsetstrokecolor{currentstroke}%
\pgfsetdash{}{0pt}%
\pgfpathmoveto{\pgfqpoint{3.463278in}{2.009947in}}%
\pgfpathlineto{\pgfqpoint{3.476996in}{2.000519in}}%
\pgfpathlineto{\pgfqpoint{3.490717in}{1.991118in}}%
\pgfpathlineto{\pgfqpoint{3.504441in}{1.981742in}}%
\pgfpathlineto{\pgfqpoint{3.518169in}{1.972393in}}%
\pgfpathlineto{\pgfqpoint{3.509693in}{1.990056in}}%
\pgfpathlineto{\pgfqpoint{3.501195in}{2.008432in}}%
\pgfpathlineto{\pgfqpoint{3.492673in}{2.027536in}}%
\pgfpathlineto{\pgfqpoint{3.484127in}{2.047382in}}%
\pgfpathlineto{\pgfqpoint{3.470349in}{2.057189in}}%
\pgfpathlineto{\pgfqpoint{3.456573in}{2.067021in}}%
\pgfpathlineto{\pgfqpoint{3.442800in}{2.076880in}}%
\pgfpathlineto{\pgfqpoint{3.429031in}{2.086766in}}%
\pgfpathlineto{\pgfqpoint{3.437630in}{2.066456in}}%
\pgfpathlineto{\pgfqpoint{3.446204in}{2.046892in}}%
\pgfpathlineto{\pgfqpoint{3.454753in}{2.028061in}}%
\pgfpathlineto{\pgfqpoint{3.463278in}{2.009947in}}%
\pgfpathclose%
\pgfusepath{fill}%
\end{pgfscope}%
\begin{pgfscope}%
\pgfpathrectangle{\pgfqpoint{1.254980in}{0.150000in}}{\pgfqpoint{5.490039in}{5.490039in}}%
\pgfusepath{clip}%
\pgfsetbuttcap%
\pgfsetroundjoin%
\definecolor{currentfill}{rgb}{0.783315,0.879285,0.125405}%
\pgfsetfillcolor{currentfill}%
\pgfsetfillopacity{0.700000}%
\pgfsetlinewidth{0.000000pt}%
\definecolor{currentstroke}{rgb}{0.000000,0.000000,0.000000}%
\pgfsetstrokecolor{currentstroke}%
\pgfsetdash{}{0pt}%
\pgfpathmoveto{\pgfqpoint{1.965494in}{3.431691in}}%
\pgfpathlineto{\pgfqpoint{1.979217in}{3.417456in}}%
\pgfpathlineto{\pgfqpoint{1.992938in}{3.403276in}}%
\pgfpathlineto{\pgfqpoint{2.006658in}{3.389151in}}%
\pgfpathlineto{\pgfqpoint{2.020377in}{3.375081in}}%
\pgfpathlineto{\pgfqpoint{2.009845in}{3.412005in}}%
\pgfpathlineto{\pgfqpoint{1.999251in}{3.449906in}}%
\pgfpathlineto{\pgfqpoint{1.988595in}{3.488804in}}%
\pgfpathlineto{\pgfqpoint{1.977874in}{3.528717in}}%
\pgfpathlineto{\pgfqpoint{1.964067in}{3.543326in}}%
\pgfpathlineto{\pgfqpoint{1.950259in}{3.557990in}}%
\pgfpathlineto{\pgfqpoint{1.936450in}{3.572711in}}%
\pgfpathlineto{\pgfqpoint{1.922638in}{3.587488in}}%
\pgfpathlineto{\pgfqpoint{1.933450in}{3.547025in}}%
\pgfpathlineto{\pgfqpoint{1.944195in}{3.507584in}}%
\pgfpathlineto{\pgfqpoint{1.954877in}{3.469145in}}%
\pgfpathlineto{\pgfqpoint{1.965494in}{3.431691in}}%
\pgfpathclose%
\pgfusepath{fill}%
\end{pgfscope}%
\begin{pgfscope}%
\pgfpathrectangle{\pgfqpoint{1.254980in}{0.150000in}}{\pgfqpoint{5.490039in}{5.490039in}}%
\pgfusepath{clip}%
\pgfsetbuttcap%
\pgfsetroundjoin%
\definecolor{currentfill}{rgb}{0.121831,0.589055,0.545623}%
\pgfsetfillcolor{currentfill}%
\pgfsetfillopacity{0.700000}%
\pgfsetlinewidth{0.000000pt}%
\definecolor{currentstroke}{rgb}{0.000000,0.000000,0.000000}%
\pgfsetstrokecolor{currentstroke}%
\pgfsetdash{}{0pt}%
\pgfpathmoveto{\pgfqpoint{2.989812in}{2.417778in}}%
\pgfpathlineto{\pgfqpoint{3.003500in}{2.406981in}}%
\pgfpathlineto{\pgfqpoint{3.017191in}{2.396215in}}%
\pgfpathlineto{\pgfqpoint{3.030884in}{2.385480in}}%
\pgfpathlineto{\pgfqpoint{3.044580in}{2.374775in}}%
\pgfpathlineto{\pgfqpoint{3.035548in}{2.399156in}}%
\pgfpathlineto{\pgfqpoint{3.026483in}{2.424348in}}%
\pgfpathlineto{\pgfqpoint{3.017383in}{2.450366in}}%
\pgfpathlineto{\pgfqpoint{3.008247in}{2.477225in}}%
\pgfpathlineto{\pgfqpoint{2.994488in}{2.488414in}}%
\pgfpathlineto{\pgfqpoint{2.980732in}{2.499634in}}%
\pgfpathlineto{\pgfqpoint{2.966977in}{2.510885in}}%
\pgfpathlineto{\pgfqpoint{2.953224in}{2.522166in}}%
\pgfpathlineto{\pgfqpoint{2.962426in}{2.494814in}}%
\pgfpathlineto{\pgfqpoint{2.971590in}{2.468309in}}%
\pgfpathlineto{\pgfqpoint{2.980719in}{2.442635in}}%
\pgfpathlineto{\pgfqpoint{2.989812in}{2.417778in}}%
\pgfpathclose%
\pgfusepath{fill}%
\end{pgfscope}%
\begin{pgfscope}%
\pgfpathrectangle{\pgfqpoint{1.254980in}{0.150000in}}{\pgfqpoint{5.490039in}{5.490039in}}%
\pgfusepath{clip}%
\pgfsetbuttcap%
\pgfsetroundjoin%
\definecolor{currentfill}{rgb}{0.730889,0.871916,0.156029}%
\pgfsetfillcolor{currentfill}%
\pgfsetfillopacity{0.700000}%
\pgfsetlinewidth{0.000000pt}%
\definecolor{currentstroke}{rgb}{0.000000,0.000000,0.000000}%
\pgfsetstrokecolor{currentstroke}%
\pgfsetdash{}{0pt}%
\pgfpathmoveto{\pgfqpoint{2.020377in}{3.375081in}}%
\pgfpathlineto{\pgfqpoint{2.034095in}{3.361066in}}%
\pgfpathlineto{\pgfqpoint{2.047811in}{3.347103in}}%
\pgfpathlineto{\pgfqpoint{2.061526in}{3.333194in}}%
\pgfpathlineto{\pgfqpoint{2.075241in}{3.319337in}}%
\pgfpathlineto{\pgfqpoint{2.064793in}{3.355732in}}%
\pgfpathlineto{\pgfqpoint{2.054286in}{3.393099in}}%
\pgfpathlineto{\pgfqpoint{2.043717in}{3.431456in}}%
\pgfpathlineto{\pgfqpoint{2.033085in}{3.470821in}}%
\pgfpathlineto{\pgfqpoint{2.019284in}{3.485215in}}%
\pgfpathlineto{\pgfqpoint{2.005482in}{3.499662in}}%
\pgfpathlineto{\pgfqpoint{1.991679in}{3.514162in}}%
\pgfpathlineto{\pgfqpoint{1.977874in}{3.528717in}}%
\pgfpathlineto{\pgfqpoint{1.988595in}{3.488804in}}%
\pgfpathlineto{\pgfqpoint{1.999251in}{3.449906in}}%
\pgfpathlineto{\pgfqpoint{2.009845in}{3.412005in}}%
\pgfpathlineto{\pgfqpoint{2.020377in}{3.375081in}}%
\pgfpathclose%
\pgfusepath{fill}%
\end{pgfscope}%
\begin{pgfscope}%
\pgfpathrectangle{\pgfqpoint{1.254980in}{0.150000in}}{\pgfqpoint{5.490039in}{5.490039in}}%
\pgfusepath{clip}%
\pgfsetbuttcap%
\pgfsetroundjoin%
\definecolor{currentfill}{rgb}{0.283197,0.115680,0.436115}%
\pgfsetfillcolor{currentfill}%
\pgfsetfillopacity{0.700000}%
\pgfsetlinewidth{0.000000pt}%
\definecolor{currentstroke}{rgb}{0.000000,0.000000,0.000000}%
\pgfsetstrokecolor{currentstroke}%
\pgfsetdash{}{0pt}%
\pgfpathmoveto{\pgfqpoint{4.639608in}{1.285367in}}%
\pgfpathlineto{\pgfqpoint{4.653532in}{1.279391in}}%
\pgfpathlineto{\pgfqpoint{4.667461in}{1.273436in}}%
\pgfpathlineto{\pgfqpoint{4.681396in}{1.267504in}}%
\pgfpathlineto{\pgfqpoint{4.695336in}{1.261594in}}%
\pgfpathlineto{\pgfqpoint{4.687709in}{1.262352in}}%
\pgfpathlineto{\pgfqpoint{4.680078in}{1.263555in}}%
\pgfpathlineto{\pgfqpoint{4.672444in}{1.265212in}}%
\pgfpathlineto{\pgfqpoint{4.664807in}{1.267334in}}%
\pgfpathlineto{\pgfqpoint{4.650844in}{1.273629in}}%
\pgfpathlineto{\pgfqpoint{4.636886in}{1.279947in}}%
\pgfpathlineto{\pgfqpoint{4.622935in}{1.286287in}}%
\pgfpathlineto{\pgfqpoint{4.608989in}{1.292650in}}%
\pgfpathlineto{\pgfqpoint{4.616650in}{1.290136in}}%
\pgfpathlineto{\pgfqpoint{4.624307in}{1.288092in}}%
\pgfpathlineto{\pgfqpoint{4.631959in}{1.286506in}}%
\pgfpathlineto{\pgfqpoint{4.639608in}{1.285367in}}%
\pgfpathclose%
\pgfusepath{fill}%
\end{pgfscope}%
\begin{pgfscope}%
\pgfpathrectangle{\pgfqpoint{1.254980in}{0.150000in}}{\pgfqpoint{5.490039in}{5.490039in}}%
\pgfusepath{clip}%
\pgfsetbuttcap%
\pgfsetroundjoin%
\definecolor{currentfill}{rgb}{0.233603,0.313828,0.543914}%
\pgfsetfillcolor{currentfill}%
\pgfsetfillopacity{0.700000}%
\pgfsetlinewidth{0.000000pt}%
\definecolor{currentstroke}{rgb}{0.000000,0.000000,0.000000}%
\pgfsetstrokecolor{currentstroke}%
\pgfsetdash{}{0pt}%
\pgfpathmoveto{\pgfqpoint{3.881278in}{1.702332in}}%
\pgfpathlineto{\pgfqpoint{3.895053in}{1.694045in}}%
\pgfpathlineto{\pgfqpoint{3.908832in}{1.685783in}}%
\pgfpathlineto{\pgfqpoint{3.922616in}{1.677544in}}%
\pgfpathlineto{\pgfqpoint{3.936403in}{1.669330in}}%
\pgfpathlineto{\pgfqpoint{3.928314in}{1.681263in}}%
\pgfpathlineto{\pgfqpoint{3.920211in}{1.693826in}}%
\pgfpathlineto{\pgfqpoint{3.912092in}{1.707031in}}%
\pgfpathlineto{\pgfqpoint{3.903958in}{1.720891in}}%
\pgfpathlineto{\pgfqpoint{3.890130in}{1.729540in}}%
\pgfpathlineto{\pgfqpoint{3.876305in}{1.738214in}}%
\pgfpathlineto{\pgfqpoint{3.862484in}{1.746912in}}%
\pgfpathlineto{\pgfqpoint{3.848667in}{1.755633in}}%
\pgfpathlineto{\pgfqpoint{3.856844in}{1.741332in}}%
\pgfpathlineto{\pgfqpoint{3.865005in}{1.727689in}}%
\pgfpathlineto{\pgfqpoint{3.873149in}{1.714694in}}%
\pgfpathlineto{\pgfqpoint{3.881278in}{1.702332in}}%
\pgfpathclose%
\pgfusepath{fill}%
\end{pgfscope}%
\begin{pgfscope}%
\pgfpathrectangle{\pgfqpoint{1.254980in}{0.150000in}}{\pgfqpoint{5.490039in}{5.490039in}}%
\pgfusepath{clip}%
\pgfsetbuttcap%
\pgfsetroundjoin%
\definecolor{currentfill}{rgb}{0.266580,0.228262,0.514349}%
\pgfsetfillcolor{currentfill}%
\pgfsetfillopacity{0.700000}%
\pgfsetlinewidth{0.000000pt}%
\definecolor{currentstroke}{rgb}{0.000000,0.000000,0.000000}%
\pgfsetstrokecolor{currentstroke}%
\pgfsetdash{}{0pt}%
\pgfpathmoveto{\pgfqpoint{4.189229in}{1.506111in}}%
\pgfpathlineto{\pgfqpoint{4.203056in}{1.498714in}}%
\pgfpathlineto{\pgfqpoint{4.216889in}{1.491340in}}%
\pgfpathlineto{\pgfqpoint{4.230726in}{1.483989in}}%
\pgfpathlineto{\pgfqpoint{4.244569in}{1.476661in}}%
\pgfpathlineto{\pgfqpoint{4.236707in}{1.484145in}}%
\pgfpathlineto{\pgfqpoint{4.228836in}{1.492189in}}%
\pgfpathlineto{\pgfqpoint{4.220955in}{1.500804in}}%
\pgfpathlineto{\pgfqpoint{4.213064in}{1.510003in}}%
\pgfpathlineto{\pgfqpoint{4.199188in}{1.517748in}}%
\pgfpathlineto{\pgfqpoint{4.185317in}{1.525516in}}%
\pgfpathlineto{\pgfqpoint{4.171450in}{1.533307in}}%
\pgfpathlineto{\pgfqpoint{4.157588in}{1.541121in}}%
\pgfpathlineto{\pgfqpoint{4.165514in}{1.531499in}}%
\pgfpathlineto{\pgfqpoint{4.173429in}{1.522465in}}%
\pgfpathlineto{\pgfqpoint{4.181334in}{1.514006in}}%
\pgfpathlineto{\pgfqpoint{4.189229in}{1.506111in}}%
\pgfpathclose%
\pgfusepath{fill}%
\end{pgfscope}%
\begin{pgfscope}%
\pgfpathrectangle{\pgfqpoint{1.254980in}{0.150000in}}{\pgfqpoint{5.490039in}{5.490039in}}%
\pgfusepath{clip}%
\pgfsetbuttcap%
\pgfsetroundjoin%
\definecolor{currentfill}{rgb}{0.182256,0.426184,0.557120}%
\pgfsetfillcolor{currentfill}%
\pgfsetfillopacity{0.700000}%
\pgfsetlinewidth{0.000000pt}%
\definecolor{currentstroke}{rgb}{0.000000,0.000000,0.000000}%
\pgfsetstrokecolor{currentstroke}%
\pgfsetdash{}{0pt}%
\pgfpathmoveto{\pgfqpoint{3.518169in}{1.972393in}}%
\pgfpathlineto{\pgfqpoint{3.531899in}{1.963070in}}%
\pgfpathlineto{\pgfqpoint{3.545633in}{1.953772in}}%
\pgfpathlineto{\pgfqpoint{3.559370in}{1.944501in}}%
\pgfpathlineto{\pgfqpoint{3.573111in}{1.935255in}}%
\pgfpathlineto{\pgfqpoint{3.564685in}{1.952468in}}%
\pgfpathlineto{\pgfqpoint{3.556237in}{1.970391in}}%
\pgfpathlineto{\pgfqpoint{3.547766in}{1.989036in}}%
\pgfpathlineto{\pgfqpoint{3.539273in}{2.008418in}}%
\pgfpathlineto{\pgfqpoint{3.525482in}{2.018120in}}%
\pgfpathlineto{\pgfqpoint{3.511694in}{2.027848in}}%
\pgfpathlineto{\pgfqpoint{3.497909in}{2.037602in}}%
\pgfpathlineto{\pgfqpoint{3.484127in}{2.047382in}}%
\pgfpathlineto{\pgfqpoint{3.492673in}{2.027536in}}%
\pgfpathlineto{\pgfqpoint{3.501195in}{2.008432in}}%
\pgfpathlineto{\pgfqpoint{3.509693in}{1.990056in}}%
\pgfpathlineto{\pgfqpoint{3.518169in}{1.972393in}}%
\pgfpathclose%
\pgfusepath{fill}%
\end{pgfscope}%
\begin{pgfscope}%
\pgfpathrectangle{\pgfqpoint{1.254980in}{0.150000in}}{\pgfqpoint{5.490039in}{5.490039in}}%
\pgfusepath{clip}%
\pgfsetbuttcap%
\pgfsetroundjoin%
\definecolor{currentfill}{rgb}{0.678489,0.863742,0.189503}%
\pgfsetfillcolor{currentfill}%
\pgfsetfillopacity{0.700000}%
\pgfsetlinewidth{0.000000pt}%
\definecolor{currentstroke}{rgb}{0.000000,0.000000,0.000000}%
\pgfsetstrokecolor{currentstroke}%
\pgfsetdash{}{0pt}%
\pgfpathmoveto{\pgfqpoint{2.075241in}{3.319337in}}%
\pgfpathlineto{\pgfqpoint{2.088954in}{3.305532in}}%
\pgfpathlineto{\pgfqpoint{2.102667in}{3.291779in}}%
\pgfpathlineto{\pgfqpoint{2.116378in}{3.278076in}}%
\pgfpathlineto{\pgfqpoint{2.130089in}{3.264423in}}%
\pgfpathlineto{\pgfqpoint{2.119726in}{3.300292in}}%
\pgfpathlineto{\pgfqpoint{2.109303in}{3.337127in}}%
\pgfpathlineto{\pgfqpoint{2.098821in}{3.374946in}}%
\pgfpathlineto{\pgfqpoint{2.088277in}{3.413766in}}%
\pgfpathlineto{\pgfqpoint{2.074481in}{3.427953in}}%
\pgfpathlineto{\pgfqpoint{2.060683in}{3.442191in}}%
\pgfpathlineto{\pgfqpoint{2.046885in}{3.456480in}}%
\pgfpathlineto{\pgfqpoint{2.033085in}{3.470821in}}%
\pgfpathlineto{\pgfqpoint{2.043717in}{3.431456in}}%
\pgfpathlineto{\pgfqpoint{2.054286in}{3.393099in}}%
\pgfpathlineto{\pgfqpoint{2.064793in}{3.355732in}}%
\pgfpathlineto{\pgfqpoint{2.075241in}{3.319337in}}%
\pgfpathclose%
\pgfusepath{fill}%
\end{pgfscope}%
\begin{pgfscope}%
\pgfpathrectangle{\pgfqpoint{1.254980in}{0.150000in}}{\pgfqpoint{5.490039in}{5.490039in}}%
\pgfusepath{clip}%
\pgfsetbuttcap%
\pgfsetroundjoin%
\definecolor{currentfill}{rgb}{0.280868,0.160771,0.472899}%
\pgfsetfillcolor{currentfill}%
\pgfsetfillopacity{0.700000}%
\pgfsetlinewidth{0.000000pt}%
\definecolor{currentstroke}{rgb}{0.000000,0.000000,0.000000}%
\pgfsetstrokecolor{currentstroke}%
\pgfsetdash{}{0pt}%
\pgfpathmoveto{\pgfqpoint{4.442071in}{1.370742in}}%
\pgfpathlineto{\pgfqpoint{4.455951in}{1.364111in}}%
\pgfpathlineto{\pgfqpoint{4.469836in}{1.357502in}}%
\pgfpathlineto{\pgfqpoint{4.483727in}{1.350916in}}%
\pgfpathlineto{\pgfqpoint{4.497623in}{1.344352in}}%
\pgfpathlineto{\pgfqpoint{4.489907in}{1.348134in}}%
\pgfpathlineto{\pgfqpoint{4.482185in}{1.352415in}}%
\pgfpathlineto{\pgfqpoint{4.474458in}{1.357205in}}%
\pgfpathlineto{\pgfqpoint{4.466724in}{1.362515in}}%
\pgfpathlineto{\pgfqpoint{4.452801in}{1.369479in}}%
\pgfpathlineto{\pgfqpoint{4.438883in}{1.376466in}}%
\pgfpathlineto{\pgfqpoint{4.424971in}{1.383475in}}%
\pgfpathlineto{\pgfqpoint{4.411063in}{1.390507in}}%
\pgfpathlineto{\pgfqpoint{4.418825in}{1.384790in}}%
\pgfpathlineto{\pgfqpoint{4.426580in}{1.379598in}}%
\pgfpathlineto{\pgfqpoint{4.434329in}{1.374919in}}%
\pgfpathlineto{\pgfqpoint{4.442071in}{1.370742in}}%
\pgfpathclose%
\pgfusepath{fill}%
\end{pgfscope}%
\begin{pgfscope}%
\pgfpathrectangle{\pgfqpoint{1.254980in}{0.150000in}}{\pgfqpoint{5.490039in}{5.490039in}}%
\pgfusepath{clip}%
\pgfsetbuttcap%
\pgfsetroundjoin%
\definecolor{currentfill}{rgb}{0.280894,0.078907,0.402329}%
\pgfsetfillcolor{currentfill}%
\pgfsetfillopacity{0.700000}%
\pgfsetlinewidth{0.000000pt}%
\definecolor{currentstroke}{rgb}{0.000000,0.000000,0.000000}%
\pgfsetstrokecolor{currentstroke}%
\pgfsetdash{}{0pt}%
\pgfpathmoveto{\pgfqpoint{4.837406in}{1.219282in}}%
\pgfpathlineto{\pgfqpoint{4.851383in}{1.213943in}}%
\pgfpathlineto{\pgfqpoint{4.865367in}{1.208626in}}%
\pgfpathlineto{\pgfqpoint{4.879357in}{1.203331in}}%
\pgfpathlineto{\pgfqpoint{4.893353in}{1.198058in}}%
\pgfpathlineto{\pgfqpoint{4.885791in}{1.196046in}}%
\pgfpathlineto{\pgfqpoint{4.878228in}{1.194426in}}%
\pgfpathlineto{\pgfqpoint{4.870664in}{1.193208in}}%
\pgfpathlineto{\pgfqpoint{4.863098in}{1.192403in}}%
\pgfpathlineto{\pgfqpoint{4.849084in}{1.198048in}}%
\pgfpathlineto{\pgfqpoint{4.835077in}{1.203714in}}%
\pgfpathlineto{\pgfqpoint{4.821075in}{1.209402in}}%
\pgfpathlineto{\pgfqpoint{4.807080in}{1.215113in}}%
\pgfpathlineto{\pgfqpoint{4.814664in}{1.215541in}}%
\pgfpathlineto{\pgfqpoint{4.822247in}{1.216385in}}%
\pgfpathlineto{\pgfqpoint{4.829827in}{1.217635in}}%
\pgfpathlineto{\pgfqpoint{4.837406in}{1.219282in}}%
\pgfpathclose%
\pgfusepath{fill}%
\end{pgfscope}%
\begin{pgfscope}%
\pgfpathrectangle{\pgfqpoint{1.254980in}{0.150000in}}{\pgfqpoint{5.490039in}{5.490039in}}%
\pgfusepath{clip}%
\pgfsetbuttcap%
\pgfsetroundjoin%
\definecolor{currentfill}{rgb}{0.626579,0.854645,0.223353}%
\pgfsetfillcolor{currentfill}%
\pgfsetfillopacity{0.700000}%
\pgfsetlinewidth{0.000000pt}%
\definecolor{currentstroke}{rgb}{0.000000,0.000000,0.000000}%
\pgfsetstrokecolor{currentstroke}%
\pgfsetdash{}{0pt}%
\pgfpathmoveto{\pgfqpoint{2.130089in}{3.264423in}}%
\pgfpathlineto{\pgfqpoint{2.143799in}{3.250820in}}%
\pgfpathlineto{\pgfqpoint{2.157509in}{3.237266in}}%
\pgfpathlineto{\pgfqpoint{2.171218in}{3.223762in}}%
\pgfpathlineto{\pgfqpoint{2.184927in}{3.210305in}}%
\pgfpathlineto{\pgfqpoint{2.174646in}{3.245651in}}%
\pgfpathlineto{\pgfqpoint{2.164307in}{3.281956in}}%
\pgfpathlineto{\pgfqpoint{2.153910in}{3.319239in}}%
\pgfpathlineto{\pgfqpoint{2.143453in}{3.357517in}}%
\pgfpathlineto{\pgfqpoint{2.129661in}{3.371506in}}%
\pgfpathlineto{\pgfqpoint{2.115867in}{3.385543in}}%
\pgfpathlineto{\pgfqpoint{2.102073in}{3.399630in}}%
\pgfpathlineto{\pgfqpoint{2.088277in}{3.413766in}}%
\pgfpathlineto{\pgfqpoint{2.098821in}{3.374946in}}%
\pgfpathlineto{\pgfqpoint{2.109303in}{3.337127in}}%
\pgfpathlineto{\pgfqpoint{2.119726in}{3.300292in}}%
\pgfpathlineto{\pgfqpoint{2.130089in}{3.264423in}}%
\pgfpathclose%
\pgfusepath{fill}%
\end{pgfscope}%
\begin{pgfscope}%
\pgfpathrectangle{\pgfqpoint{1.254980in}{0.150000in}}{\pgfqpoint{5.490039in}{5.490039in}}%
\pgfusepath{clip}%
\pgfsetbuttcap%
\pgfsetroundjoin%
\definecolor{currentfill}{rgb}{0.125394,0.574318,0.549086}%
\pgfsetfillcolor{currentfill}%
\pgfsetfillopacity{0.700000}%
\pgfsetlinewidth{0.000000pt}%
\definecolor{currentstroke}{rgb}{0.000000,0.000000,0.000000}%
\pgfsetstrokecolor{currentstroke}%
\pgfsetdash{}{0pt}%
\pgfpathmoveto{\pgfqpoint{3.044580in}{2.374775in}}%
\pgfpathlineto{\pgfqpoint{3.058277in}{2.364101in}}%
\pgfpathlineto{\pgfqpoint{3.071976in}{2.353456in}}%
\pgfpathlineto{\pgfqpoint{3.085678in}{2.342841in}}%
\pgfpathlineto{\pgfqpoint{3.099382in}{2.332256in}}%
\pgfpathlineto{\pgfqpoint{3.090412in}{2.356161in}}%
\pgfpathlineto{\pgfqpoint{3.081409in}{2.380872in}}%
\pgfpathlineto{\pgfqpoint{3.072371in}{2.406404in}}%
\pgfpathlineto{\pgfqpoint{3.063300in}{2.432772in}}%
\pgfpathlineto{\pgfqpoint{3.049534in}{2.443840in}}%
\pgfpathlineto{\pgfqpoint{3.035769in}{2.454938in}}%
\pgfpathlineto{\pgfqpoint{3.022007in}{2.466067in}}%
\pgfpathlineto{\pgfqpoint{3.008247in}{2.477225in}}%
\pgfpathlineto{\pgfqpoint{3.017383in}{2.450366in}}%
\pgfpathlineto{\pgfqpoint{3.026483in}{2.424348in}}%
\pgfpathlineto{\pgfqpoint{3.035548in}{2.399156in}}%
\pgfpathlineto{\pgfqpoint{3.044580in}{2.374775in}}%
\pgfpathclose%
\pgfusepath{fill}%
\end{pgfscope}%
\begin{pgfscope}%
\pgfpathrectangle{\pgfqpoint{1.254980in}{0.150000in}}{\pgfqpoint{5.490039in}{5.490039in}}%
\pgfusepath{clip}%
\pgfsetbuttcap%
\pgfsetroundjoin%
\definecolor{currentfill}{rgb}{0.575563,0.844566,0.256415}%
\pgfsetfillcolor{currentfill}%
\pgfsetfillopacity{0.700000}%
\pgfsetlinewidth{0.000000pt}%
\definecolor{currentstroke}{rgb}{0.000000,0.000000,0.000000}%
\pgfsetstrokecolor{currentstroke}%
\pgfsetdash{}{0pt}%
\pgfpathmoveto{\pgfqpoint{2.184927in}{3.210305in}}%
\pgfpathlineto{\pgfqpoint{2.198635in}{3.196896in}}%
\pgfpathlineto{\pgfqpoint{2.212342in}{3.183535in}}%
\pgfpathlineto{\pgfqpoint{2.226050in}{3.170221in}}%
\pgfpathlineto{\pgfqpoint{2.239757in}{3.156953in}}%
\pgfpathlineto{\pgfqpoint{2.229557in}{3.191777in}}%
\pgfpathlineto{\pgfqpoint{2.219301in}{3.227555in}}%
\pgfpathlineto{\pgfqpoint{2.208988in}{3.264304in}}%
\pgfpathlineto{\pgfqpoint{2.198617in}{3.302043in}}%
\pgfpathlineto{\pgfqpoint{2.184827in}{3.315841in}}%
\pgfpathlineto{\pgfqpoint{2.171037in}{3.329685in}}%
\pgfpathlineto{\pgfqpoint{2.157245in}{3.343577in}}%
\pgfpathlineto{\pgfqpoint{2.143453in}{3.357517in}}%
\pgfpathlineto{\pgfqpoint{2.153910in}{3.319239in}}%
\pgfpathlineto{\pgfqpoint{2.164307in}{3.281956in}}%
\pgfpathlineto{\pgfqpoint{2.174646in}{3.245651in}}%
\pgfpathlineto{\pgfqpoint{2.184927in}{3.210305in}}%
\pgfpathclose%
\pgfusepath{fill}%
\end{pgfscope}%
\begin{pgfscope}%
\pgfpathrectangle{\pgfqpoint{1.254980in}{0.150000in}}{\pgfqpoint{5.490039in}{5.490039in}}%
\pgfusepath{clip}%
\pgfsetbuttcap%
\pgfsetroundjoin%
\definecolor{currentfill}{rgb}{0.237441,0.305202,0.541921}%
\pgfsetfillcolor{currentfill}%
\pgfsetfillopacity{0.700000}%
\pgfsetlinewidth{0.000000pt}%
\definecolor{currentstroke}{rgb}{0.000000,0.000000,0.000000}%
\pgfsetstrokecolor{currentstroke}%
\pgfsetdash{}{0pt}%
\pgfpathmoveto{\pgfqpoint{3.936403in}{1.669330in}}%
\pgfpathlineto{\pgfqpoint{3.950195in}{1.661140in}}%
\pgfpathlineto{\pgfqpoint{3.963991in}{1.652973in}}%
\pgfpathlineto{\pgfqpoint{3.977791in}{1.644830in}}%
\pgfpathlineto{\pgfqpoint{3.991595in}{1.636711in}}%
\pgfpathlineto{\pgfqpoint{3.983546in}{1.648216in}}%
\pgfpathlineto{\pgfqpoint{3.975482in}{1.660346in}}%
\pgfpathlineto{\pgfqpoint{3.967405in}{1.673114in}}%
\pgfpathlineto{\pgfqpoint{3.959313in}{1.686533in}}%
\pgfpathlineto{\pgfqpoint{3.945468in}{1.695087in}}%
\pgfpathlineto{\pgfqpoint{3.931628in}{1.703664in}}%
\pgfpathlineto{\pgfqpoint{3.917791in}{1.712266in}}%
\pgfpathlineto{\pgfqpoint{3.903958in}{1.720891in}}%
\pgfpathlineto{\pgfqpoint{3.912092in}{1.707031in}}%
\pgfpathlineto{\pgfqpoint{3.920211in}{1.693826in}}%
\pgfpathlineto{\pgfqpoint{3.928314in}{1.681263in}}%
\pgfpathlineto{\pgfqpoint{3.936403in}{1.669330in}}%
\pgfpathclose%
\pgfusepath{fill}%
\end{pgfscope}%
\begin{pgfscope}%
\pgfpathrectangle{\pgfqpoint{1.254980in}{0.150000in}}{\pgfqpoint{5.490039in}{5.490039in}}%
\pgfusepath{clip}%
\pgfsetbuttcap%
\pgfsetroundjoin%
\definecolor{currentfill}{rgb}{0.187231,0.414746,0.556547}%
\pgfsetfillcolor{currentfill}%
\pgfsetfillopacity{0.700000}%
\pgfsetlinewidth{0.000000pt}%
\definecolor{currentstroke}{rgb}{0.000000,0.000000,0.000000}%
\pgfsetstrokecolor{currentstroke}%
\pgfsetdash{}{0pt}%
\pgfpathmoveto{\pgfqpoint{3.573111in}{1.935255in}}%
\pgfpathlineto{\pgfqpoint{3.586855in}{1.926035in}}%
\pgfpathlineto{\pgfqpoint{3.600602in}{1.916841in}}%
\pgfpathlineto{\pgfqpoint{3.614353in}{1.907672in}}%
\pgfpathlineto{\pgfqpoint{3.628108in}{1.898528in}}%
\pgfpathlineto{\pgfqpoint{3.619730in}{1.915293in}}%
\pgfpathlineto{\pgfqpoint{3.611331in}{1.932761in}}%
\pgfpathlineto{\pgfqpoint{3.602911in}{1.950949in}}%
\pgfpathlineto{\pgfqpoint{3.594469in}{1.969868in}}%
\pgfpathlineto{\pgfqpoint{3.580665in}{1.979467in}}%
\pgfpathlineto{\pgfqpoint{3.566864in}{1.989092in}}%
\pgfpathlineto{\pgfqpoint{3.553067in}{1.998742in}}%
\pgfpathlineto{\pgfqpoint{3.539273in}{2.008418in}}%
\pgfpathlineto{\pgfqpoint{3.547766in}{1.989036in}}%
\pgfpathlineto{\pgfqpoint{3.556237in}{1.970391in}}%
\pgfpathlineto{\pgfqpoint{3.564685in}{1.952468in}}%
\pgfpathlineto{\pgfqpoint{3.573111in}{1.935255in}}%
\pgfpathclose%
\pgfusepath{fill}%
\end{pgfscope}%
\begin{pgfscope}%
\pgfpathrectangle{\pgfqpoint{1.254980in}{0.150000in}}{\pgfqpoint{5.490039in}{5.490039in}}%
\pgfusepath{clip}%
\pgfsetbuttcap%
\pgfsetroundjoin%
\definecolor{currentfill}{rgb}{0.525776,0.833491,0.288127}%
\pgfsetfillcolor{currentfill}%
\pgfsetfillopacity{0.700000}%
\pgfsetlinewidth{0.000000pt}%
\definecolor{currentstroke}{rgb}{0.000000,0.000000,0.000000}%
\pgfsetstrokecolor{currentstroke}%
\pgfsetdash{}{0pt}%
\pgfpathmoveto{\pgfqpoint{2.239757in}{3.156953in}}%
\pgfpathlineto{\pgfqpoint{2.253463in}{3.143731in}}%
\pgfpathlineto{\pgfqpoint{2.267170in}{3.130554in}}%
\pgfpathlineto{\pgfqpoint{2.280876in}{3.117423in}}%
\pgfpathlineto{\pgfqpoint{2.294583in}{3.104336in}}%
\pgfpathlineto{\pgfqpoint{2.284463in}{3.138642in}}%
\pgfpathlineto{\pgfqpoint{2.274289in}{3.173894in}}%
\pgfpathlineto{\pgfqpoint{2.264059in}{3.210113in}}%
\pgfpathlineto{\pgfqpoint{2.253773in}{3.247314in}}%
\pgfpathlineto{\pgfqpoint{2.239985in}{3.260928in}}%
\pgfpathlineto{\pgfqpoint{2.226196in}{3.274587in}}%
\pgfpathlineto{\pgfqpoint{2.212407in}{3.288292in}}%
\pgfpathlineto{\pgfqpoint{2.198617in}{3.302043in}}%
\pgfpathlineto{\pgfqpoint{2.208988in}{3.264304in}}%
\pgfpathlineto{\pgfqpoint{2.219301in}{3.227555in}}%
\pgfpathlineto{\pgfqpoint{2.229557in}{3.191777in}}%
\pgfpathlineto{\pgfqpoint{2.239757in}{3.156953in}}%
\pgfpathclose%
\pgfusepath{fill}%
\end{pgfscope}%
\begin{pgfscope}%
\pgfpathrectangle{\pgfqpoint{1.254980in}{0.150000in}}{\pgfqpoint{5.490039in}{5.490039in}}%
\pgfusepath{clip}%
\pgfsetbuttcap%
\pgfsetroundjoin%
\definecolor{currentfill}{rgb}{0.269308,0.218818,0.509577}%
\pgfsetfillcolor{currentfill}%
\pgfsetfillopacity{0.700000}%
\pgfsetlinewidth{0.000000pt}%
\definecolor{currentstroke}{rgb}{0.000000,0.000000,0.000000}%
\pgfsetstrokecolor{currentstroke}%
\pgfsetdash{}{0pt}%
\pgfpathmoveto{\pgfqpoint{4.244569in}{1.476661in}}%
\pgfpathlineto{\pgfqpoint{4.258416in}{1.469356in}}%
\pgfpathlineto{\pgfqpoint{4.272268in}{1.462074in}}%
\pgfpathlineto{\pgfqpoint{4.286125in}{1.454814in}}%
\pgfpathlineto{\pgfqpoint{4.299987in}{1.447578in}}%
\pgfpathlineto{\pgfqpoint{4.292157in}{1.454652in}}%
\pgfpathlineto{\pgfqpoint{4.284318in}{1.462281in}}%
\pgfpathlineto{\pgfqpoint{4.276470in}{1.470478in}}%
\pgfpathlineto{\pgfqpoint{4.268614in}{1.479254in}}%
\pgfpathlineto{\pgfqpoint{4.254719in}{1.486907in}}%
\pgfpathlineto{\pgfqpoint{4.240829in}{1.494583in}}%
\pgfpathlineto{\pgfqpoint{4.226944in}{1.502281in}}%
\pgfpathlineto{\pgfqpoint{4.213064in}{1.510003in}}%
\pgfpathlineto{\pgfqpoint{4.220955in}{1.500804in}}%
\pgfpathlineto{\pgfqpoint{4.228836in}{1.492189in}}%
\pgfpathlineto{\pgfqpoint{4.236707in}{1.484145in}}%
\pgfpathlineto{\pgfqpoint{4.244569in}{1.476661in}}%
\pgfpathclose%
\pgfusepath{fill}%
\end{pgfscope}%
\begin{pgfscope}%
\pgfpathrectangle{\pgfqpoint{1.254980in}{0.150000in}}{\pgfqpoint{5.490039in}{5.490039in}}%
\pgfusepath{clip}%
\pgfsetbuttcap%
\pgfsetroundjoin%
\definecolor{currentfill}{rgb}{0.283091,0.110553,0.431554}%
\pgfsetfillcolor{currentfill}%
\pgfsetfillopacity{0.700000}%
\pgfsetlinewidth{0.000000pt}%
\definecolor{currentstroke}{rgb}{0.000000,0.000000,0.000000}%
\pgfsetstrokecolor{currentstroke}%
\pgfsetdash{}{0pt}%
\pgfpathmoveto{\pgfqpoint{4.695336in}{1.261594in}}%
\pgfpathlineto{\pgfqpoint{4.709283in}{1.255706in}}%
\pgfpathlineto{\pgfqpoint{4.723236in}{1.249841in}}%
\pgfpathlineto{\pgfqpoint{4.737195in}{1.243997in}}%
\pgfpathlineto{\pgfqpoint{4.751160in}{1.238176in}}%
\pgfpathlineto{\pgfqpoint{4.743553in}{1.238555in}}%
\pgfpathlineto{\pgfqpoint{4.735944in}{1.239373in}}%
\pgfpathlineto{\pgfqpoint{4.728332in}{1.240643in}}%
\pgfpathlineto{\pgfqpoint{4.720717in}{1.242374in}}%
\pgfpathlineto{\pgfqpoint{4.706731in}{1.248581in}}%
\pgfpathlineto{\pgfqpoint{4.692750in}{1.254810in}}%
\pgfpathlineto{\pgfqpoint{4.678776in}{1.261061in}}%
\pgfpathlineto{\pgfqpoint{4.664807in}{1.267334in}}%
\pgfpathlineto{\pgfqpoint{4.672444in}{1.265212in}}%
\pgfpathlineto{\pgfqpoint{4.680078in}{1.263555in}}%
\pgfpathlineto{\pgfqpoint{4.687709in}{1.262352in}}%
\pgfpathlineto{\pgfqpoint{4.695336in}{1.261594in}}%
\pgfpathclose%
\pgfusepath{fill}%
\end{pgfscope}%
\begin{pgfscope}%
\pgfpathrectangle{\pgfqpoint{1.254980in}{0.150000in}}{\pgfqpoint{5.490039in}{5.490039in}}%
\pgfusepath{clip}%
\pgfsetbuttcap%
\pgfsetroundjoin%
\definecolor{currentfill}{rgb}{0.128729,0.563265,0.551229}%
\pgfsetfillcolor{currentfill}%
\pgfsetfillopacity{0.700000}%
\pgfsetlinewidth{0.000000pt}%
\definecolor{currentstroke}{rgb}{0.000000,0.000000,0.000000}%
\pgfsetstrokecolor{currentstroke}%
\pgfsetdash{}{0pt}%
\pgfpathmoveto{\pgfqpoint{3.099382in}{2.332256in}}%
\pgfpathlineto{\pgfqpoint{3.113088in}{2.321700in}}%
\pgfpathlineto{\pgfqpoint{3.126797in}{2.311174in}}%
\pgfpathlineto{\pgfqpoint{3.140508in}{2.300677in}}%
\pgfpathlineto{\pgfqpoint{3.154221in}{2.290209in}}%
\pgfpathlineto{\pgfqpoint{3.145311in}{2.313640in}}%
\pgfpathlineto{\pgfqpoint{3.136369in}{2.337871in}}%
\pgfpathlineto{\pgfqpoint{3.127394in}{2.362918in}}%
\pgfpathlineto{\pgfqpoint{3.118386in}{2.388796in}}%
\pgfpathlineto{\pgfqpoint{3.104611in}{2.399746in}}%
\pgfpathlineto{\pgfqpoint{3.090839in}{2.410725in}}%
\pgfpathlineto{\pgfqpoint{3.077068in}{2.421734in}}%
\pgfpathlineto{\pgfqpoint{3.063300in}{2.432772in}}%
\pgfpathlineto{\pgfqpoint{3.072371in}{2.406404in}}%
\pgfpathlineto{\pgfqpoint{3.081409in}{2.380872in}}%
\pgfpathlineto{\pgfqpoint{3.090412in}{2.356161in}}%
\pgfpathlineto{\pgfqpoint{3.099382in}{2.332256in}}%
\pgfpathclose%
\pgfusepath{fill}%
\end{pgfscope}%
\begin{pgfscope}%
\pgfpathrectangle{\pgfqpoint{1.254980in}{0.150000in}}{\pgfqpoint{5.490039in}{5.490039in}}%
\pgfusepath{clip}%
\pgfsetbuttcap%
\pgfsetroundjoin%
\definecolor{currentfill}{rgb}{0.477504,0.821444,0.318195}%
\pgfsetfillcolor{currentfill}%
\pgfsetfillopacity{0.700000}%
\pgfsetlinewidth{0.000000pt}%
\definecolor{currentstroke}{rgb}{0.000000,0.000000,0.000000}%
\pgfsetstrokecolor{currentstroke}%
\pgfsetdash{}{0pt}%
\pgfpathmoveto{\pgfqpoint{2.294583in}{3.104336in}}%
\pgfpathlineto{\pgfqpoint{2.308289in}{3.091294in}}%
\pgfpathlineto{\pgfqpoint{2.321995in}{3.078296in}}%
\pgfpathlineto{\pgfqpoint{2.335701in}{3.065341in}}%
\pgfpathlineto{\pgfqpoint{2.349408in}{3.052430in}}%
\pgfpathlineto{\pgfqpoint{2.339367in}{3.086218in}}%
\pgfpathlineto{\pgfqpoint{2.329274in}{3.120948in}}%
\pgfpathlineto{\pgfqpoint{2.319126in}{3.156637in}}%
\pgfpathlineto{\pgfqpoint{2.308923in}{3.193303in}}%
\pgfpathlineto{\pgfqpoint{2.295136in}{3.206740in}}%
\pgfpathlineto{\pgfqpoint{2.281348in}{3.220220in}}%
\pgfpathlineto{\pgfqpoint{2.267561in}{3.233745in}}%
\pgfpathlineto{\pgfqpoint{2.253773in}{3.247314in}}%
\pgfpathlineto{\pgfqpoint{2.264059in}{3.210113in}}%
\pgfpathlineto{\pgfqpoint{2.274289in}{3.173894in}}%
\pgfpathlineto{\pgfqpoint{2.284463in}{3.138642in}}%
\pgfpathlineto{\pgfqpoint{2.294583in}{3.104336in}}%
\pgfpathclose%
\pgfusepath{fill}%
\end{pgfscope}%
\begin{pgfscope}%
\pgfpathrectangle{\pgfqpoint{1.254980in}{0.150000in}}{\pgfqpoint{5.490039in}{5.490039in}}%
\pgfusepath{clip}%
\pgfsetbuttcap%
\pgfsetroundjoin%
\definecolor{currentfill}{rgb}{0.281412,0.155834,0.469201}%
\pgfsetfillcolor{currentfill}%
\pgfsetfillopacity{0.700000}%
\pgfsetlinewidth{0.000000pt}%
\definecolor{currentstroke}{rgb}{0.000000,0.000000,0.000000}%
\pgfsetstrokecolor{currentstroke}%
\pgfsetdash{}{0pt}%
\pgfpathmoveto{\pgfqpoint{4.497623in}{1.344352in}}%
\pgfpathlineto{\pgfqpoint{4.511524in}{1.337811in}}%
\pgfpathlineto{\pgfqpoint{4.525431in}{1.331292in}}%
\pgfpathlineto{\pgfqpoint{4.539343in}{1.324796in}}%
\pgfpathlineto{\pgfqpoint{4.553261in}{1.318322in}}%
\pgfpathlineto{\pgfqpoint{4.545571in}{1.321709in}}%
\pgfpathlineto{\pgfqpoint{4.537876in}{1.325591in}}%
\pgfpathlineto{\pgfqpoint{4.530176in}{1.329979in}}%
\pgfpathlineto{\pgfqpoint{4.522470in}{1.334883in}}%
\pgfpathlineto{\pgfqpoint{4.508525in}{1.341758in}}%
\pgfpathlineto{\pgfqpoint{4.494586in}{1.348654in}}%
\pgfpathlineto{\pgfqpoint{4.480653in}{1.355574in}}%
\pgfpathlineto{\pgfqpoint{4.466724in}{1.362515in}}%
\pgfpathlineto{\pgfqpoint{4.474458in}{1.357205in}}%
\pgfpathlineto{\pgfqpoint{4.482185in}{1.352415in}}%
\pgfpathlineto{\pgfqpoint{4.489907in}{1.348134in}}%
\pgfpathlineto{\pgfqpoint{4.497623in}{1.344352in}}%
\pgfpathclose%
\pgfusepath{fill}%
\end{pgfscope}%
\begin{pgfscope}%
\pgfpathrectangle{\pgfqpoint{1.254980in}{0.150000in}}{\pgfqpoint{5.490039in}{5.490039in}}%
\pgfusepath{clip}%
\pgfsetbuttcap%
\pgfsetroundjoin%
\definecolor{currentfill}{rgb}{0.280894,0.078907,0.402329}%
\pgfsetfillcolor{currentfill}%
\pgfsetfillopacity{0.700000}%
\pgfsetlinewidth{0.000000pt}%
\definecolor{currentstroke}{rgb}{0.000000,0.000000,0.000000}%
\pgfsetstrokecolor{currentstroke}%
\pgfsetdash{}{0pt}%
\pgfpathmoveto{\pgfqpoint{4.893353in}{1.198058in}}%
\pgfpathlineto{\pgfqpoint{4.907356in}{1.192807in}}%
\pgfpathlineto{\pgfqpoint{4.921365in}{1.187579in}}%
\pgfpathlineto{\pgfqpoint{4.935381in}{1.182372in}}%
\pgfpathlineto{\pgfqpoint{4.927832in}{1.180085in}}%
\pgfpathlineto{\pgfqpoint{4.920281in}{1.178188in}}%
\pgfpathlineto{\pgfqpoint{4.912730in}{1.176690in}}%
\pgfpathlineto{\pgfqpoint{4.905177in}{1.175603in}}%
\pgfpathlineto{\pgfqpoint{4.891145in}{1.181181in}}%
\pgfpathlineto{\pgfqpoint{4.877118in}{1.186781in}}%
\pgfpathlineto{\pgfqpoint{4.863098in}{1.192403in}}%
\pgfpathlineto{\pgfqpoint{4.870664in}{1.193208in}}%
\pgfpathlineto{\pgfqpoint{4.878228in}{1.194426in}}%
\pgfpathlineto{\pgfqpoint{4.885791in}{1.196046in}}%
\pgfpathlineto{\pgfqpoint{4.893353in}{1.198058in}}%
\pgfpathclose%
\pgfusepath{fill}%
\end{pgfscope}%
\begin{pgfscope}%
\pgfpathrectangle{\pgfqpoint{1.254980in}{0.150000in}}{\pgfqpoint{5.490039in}{5.490039in}}%
\pgfusepath{clip}%
\pgfsetbuttcap%
\pgfsetroundjoin%
\definecolor{currentfill}{rgb}{0.430983,0.808473,0.346476}%
\pgfsetfillcolor{currentfill}%
\pgfsetfillopacity{0.700000}%
\pgfsetlinewidth{0.000000pt}%
\definecolor{currentstroke}{rgb}{0.000000,0.000000,0.000000}%
\pgfsetstrokecolor{currentstroke}%
\pgfsetdash{}{0pt}%
\pgfpathmoveto{\pgfqpoint{2.349408in}{3.052430in}}%
\pgfpathlineto{\pgfqpoint{2.363114in}{3.039561in}}%
\pgfpathlineto{\pgfqpoint{2.376821in}{3.026734in}}%
\pgfpathlineto{\pgfqpoint{2.390528in}{3.013950in}}%
\pgfpathlineto{\pgfqpoint{2.404235in}{3.001207in}}%
\pgfpathlineto{\pgfqpoint{2.394273in}{3.034480in}}%
\pgfpathlineto{\pgfqpoint{2.384259in}{3.068689in}}%
\pgfpathlineto{\pgfqpoint{2.374192in}{3.103851in}}%
\pgfpathlineto{\pgfqpoint{2.364071in}{3.139984in}}%
\pgfpathlineto{\pgfqpoint{2.350284in}{3.153250in}}%
\pgfpathlineto{\pgfqpoint{2.336497in}{3.166559in}}%
\pgfpathlineto{\pgfqpoint{2.322710in}{3.179909in}}%
\pgfpathlineto{\pgfqpoint{2.308923in}{3.193303in}}%
\pgfpathlineto{\pgfqpoint{2.319126in}{3.156637in}}%
\pgfpathlineto{\pgfqpoint{2.329274in}{3.120948in}}%
\pgfpathlineto{\pgfqpoint{2.339367in}{3.086218in}}%
\pgfpathlineto{\pgfqpoint{2.349408in}{3.052430in}}%
\pgfpathclose%
\pgfusepath{fill}%
\end{pgfscope}%
\begin{pgfscope}%
\pgfpathrectangle{\pgfqpoint{1.254980in}{0.150000in}}{\pgfqpoint{5.490039in}{5.490039in}}%
\pgfusepath{clip}%
\pgfsetbuttcap%
\pgfsetroundjoin%
\definecolor{currentfill}{rgb}{0.192357,0.403199,0.555836}%
\pgfsetfillcolor{currentfill}%
\pgfsetfillopacity{0.700000}%
\pgfsetlinewidth{0.000000pt}%
\definecolor{currentstroke}{rgb}{0.000000,0.000000,0.000000}%
\pgfsetstrokecolor{currentstroke}%
\pgfsetdash{}{0pt}%
\pgfpathmoveto{\pgfqpoint{3.628108in}{1.898528in}}%
\pgfpathlineto{\pgfqpoint{3.641865in}{1.889410in}}%
\pgfpathlineto{\pgfqpoint{3.655627in}{1.880317in}}%
\pgfpathlineto{\pgfqpoint{3.669392in}{1.871250in}}%
\pgfpathlineto{\pgfqpoint{3.683160in}{1.862207in}}%
\pgfpathlineto{\pgfqpoint{3.674830in}{1.878523in}}%
\pgfpathlineto{\pgfqpoint{3.666479in}{1.895539in}}%
\pgfpathlineto{\pgfqpoint{3.658108in}{1.913269in}}%
\pgfpathlineto{\pgfqpoint{3.649717in}{1.931727in}}%
\pgfpathlineto{\pgfqpoint{3.635900in}{1.941224in}}%
\pgfpathlineto{\pgfqpoint{3.622086in}{1.950747in}}%
\pgfpathlineto{\pgfqpoint{3.608276in}{1.960295in}}%
\pgfpathlineto{\pgfqpoint{3.594469in}{1.969868in}}%
\pgfpathlineto{\pgfqpoint{3.602911in}{1.950949in}}%
\pgfpathlineto{\pgfqpoint{3.611331in}{1.932761in}}%
\pgfpathlineto{\pgfqpoint{3.619730in}{1.915293in}}%
\pgfpathlineto{\pgfqpoint{3.628108in}{1.898528in}}%
\pgfpathclose%
\pgfusepath{fill}%
\end{pgfscope}%
\begin{pgfscope}%
\pgfpathrectangle{\pgfqpoint{1.254980in}{0.150000in}}{\pgfqpoint{5.490039in}{5.490039in}}%
\pgfusepath{clip}%
\pgfsetbuttcap%
\pgfsetroundjoin%
\definecolor{currentfill}{rgb}{0.133743,0.548535,0.553541}%
\pgfsetfillcolor{currentfill}%
\pgfsetfillopacity{0.700000}%
\pgfsetlinewidth{0.000000pt}%
\definecolor{currentstroke}{rgb}{0.000000,0.000000,0.000000}%
\pgfsetstrokecolor{currentstroke}%
\pgfsetdash{}{0pt}%
\pgfpathmoveto{\pgfqpoint{3.154221in}{2.290209in}}%
\pgfpathlineto{\pgfqpoint{3.167937in}{2.279771in}}%
\pgfpathlineto{\pgfqpoint{3.181655in}{2.269361in}}%
\pgfpathlineto{\pgfqpoint{3.195376in}{2.258979in}}%
\pgfpathlineto{\pgfqpoint{3.209099in}{2.248626in}}%
\pgfpathlineto{\pgfqpoint{3.200248in}{2.271583in}}%
\pgfpathlineto{\pgfqpoint{3.191366in}{2.295335in}}%
\pgfpathlineto{\pgfqpoint{3.182452in}{2.319898in}}%
\pgfpathlineto{\pgfqpoint{3.173507in}{2.345288in}}%
\pgfpathlineto{\pgfqpoint{3.159723in}{2.356121in}}%
\pgfpathlineto{\pgfqpoint{3.145942in}{2.366984in}}%
\pgfpathlineto{\pgfqpoint{3.132163in}{2.377875in}}%
\pgfpathlineto{\pgfqpoint{3.118386in}{2.388796in}}%
\pgfpathlineto{\pgfqpoint{3.127394in}{2.362918in}}%
\pgfpathlineto{\pgfqpoint{3.136369in}{2.337871in}}%
\pgfpathlineto{\pgfqpoint{3.145311in}{2.313640in}}%
\pgfpathlineto{\pgfqpoint{3.154221in}{2.290209in}}%
\pgfpathclose%
\pgfusepath{fill}%
\end{pgfscope}%
\begin{pgfscope}%
\pgfpathrectangle{\pgfqpoint{1.254980in}{0.150000in}}{\pgfqpoint{5.490039in}{5.490039in}}%
\pgfusepath{clip}%
\pgfsetbuttcap%
\pgfsetroundjoin%
\definecolor{currentfill}{rgb}{0.241237,0.296485,0.539709}%
\pgfsetfillcolor{currentfill}%
\pgfsetfillopacity{0.700000}%
\pgfsetlinewidth{0.000000pt}%
\definecolor{currentstroke}{rgb}{0.000000,0.000000,0.000000}%
\pgfsetstrokecolor{currentstroke}%
\pgfsetdash{}{0pt}%
\pgfpathmoveto{\pgfqpoint{3.991595in}{1.636711in}}%
\pgfpathlineto{\pgfqpoint{4.005404in}{1.628616in}}%
\pgfpathlineto{\pgfqpoint{4.019217in}{1.620544in}}%
\pgfpathlineto{\pgfqpoint{4.033034in}{1.612496in}}%
\pgfpathlineto{\pgfqpoint{4.046856in}{1.604472in}}%
\pgfpathlineto{\pgfqpoint{4.038845in}{1.615549in}}%
\pgfpathlineto{\pgfqpoint{4.030821in}{1.627247in}}%
\pgfpathlineto{\pgfqpoint{4.022783in}{1.639579in}}%
\pgfpathlineto{\pgfqpoint{4.014732in}{1.652557in}}%
\pgfpathlineto{\pgfqpoint{4.000871in}{1.661016in}}%
\pgfpathlineto{\pgfqpoint{3.987014in}{1.669498in}}%
\pgfpathlineto{\pgfqpoint{3.973162in}{1.678004in}}%
\pgfpathlineto{\pgfqpoint{3.959313in}{1.686533in}}%
\pgfpathlineto{\pgfqpoint{3.967405in}{1.673114in}}%
\pgfpathlineto{\pgfqpoint{3.975482in}{1.660346in}}%
\pgfpathlineto{\pgfqpoint{3.983546in}{1.648216in}}%
\pgfpathlineto{\pgfqpoint{3.991595in}{1.636711in}}%
\pgfpathclose%
\pgfusepath{fill}%
\end{pgfscope}%
\begin{pgfscope}%
\pgfpathrectangle{\pgfqpoint{1.254980in}{0.150000in}}{\pgfqpoint{5.490039in}{5.490039in}}%
\pgfusepath{clip}%
\pgfsetbuttcap%
\pgfsetroundjoin%
\definecolor{currentfill}{rgb}{0.386433,0.794644,0.372886}%
\pgfsetfillcolor{currentfill}%
\pgfsetfillopacity{0.700000}%
\pgfsetlinewidth{0.000000pt}%
\definecolor{currentstroke}{rgb}{0.000000,0.000000,0.000000}%
\pgfsetstrokecolor{currentstroke}%
\pgfsetdash{}{0pt}%
\pgfpathmoveto{\pgfqpoint{2.404235in}{3.001207in}}%
\pgfpathlineto{\pgfqpoint{2.417942in}{2.988505in}}%
\pgfpathlineto{\pgfqpoint{2.431650in}{2.975845in}}%
\pgfpathlineto{\pgfqpoint{2.445359in}{2.963225in}}%
\pgfpathlineto{\pgfqpoint{2.459067in}{2.950645in}}%
\pgfpathlineto{\pgfqpoint{2.449182in}{2.983405in}}%
\pgfpathlineto{\pgfqpoint{2.439247in}{3.017095in}}%
\pgfpathlineto{\pgfqpoint{2.429260in}{3.051732in}}%
\pgfpathlineto{\pgfqpoint{2.419220in}{3.087334in}}%
\pgfpathlineto{\pgfqpoint{2.405432in}{3.100435in}}%
\pgfpathlineto{\pgfqpoint{2.391645in}{3.113577in}}%
\pgfpathlineto{\pgfqpoint{2.377858in}{3.126760in}}%
\pgfpathlineto{\pgfqpoint{2.364071in}{3.139984in}}%
\pgfpathlineto{\pgfqpoint{2.374192in}{3.103851in}}%
\pgfpathlineto{\pgfqpoint{2.384259in}{3.068689in}}%
\pgfpathlineto{\pgfqpoint{2.394273in}{3.034480in}}%
\pgfpathlineto{\pgfqpoint{2.404235in}{3.001207in}}%
\pgfpathclose%
\pgfusepath{fill}%
\end{pgfscope}%
\begin{pgfscope}%
\pgfpathrectangle{\pgfqpoint{1.254980in}{0.150000in}}{\pgfqpoint{5.490039in}{5.490039in}}%
\pgfusepath{clip}%
\pgfsetbuttcap%
\pgfsetroundjoin%
\definecolor{currentfill}{rgb}{0.271828,0.209303,0.504434}%
\pgfsetfillcolor{currentfill}%
\pgfsetfillopacity{0.700000}%
\pgfsetlinewidth{0.000000pt}%
\definecolor{currentstroke}{rgb}{0.000000,0.000000,0.000000}%
\pgfsetstrokecolor{currentstroke}%
\pgfsetdash{}{0pt}%
\pgfpathmoveto{\pgfqpoint{4.299987in}{1.447578in}}%
\pgfpathlineto{\pgfqpoint{4.313854in}{1.440365in}}%
\pgfpathlineto{\pgfqpoint{4.327726in}{1.433174in}}%
\pgfpathlineto{\pgfqpoint{4.341602in}{1.426006in}}%
\pgfpathlineto{\pgfqpoint{4.355484in}{1.418861in}}%
\pgfpathlineto{\pgfqpoint{4.347685in}{1.425524in}}%
\pgfpathlineto{\pgfqpoint{4.339878in}{1.432739in}}%
\pgfpathlineto{\pgfqpoint{4.332063in}{1.440518in}}%
\pgfpathlineto{\pgfqpoint{4.324240in}{1.448872in}}%
\pgfpathlineto{\pgfqpoint{4.310326in}{1.456433in}}%
\pgfpathlineto{\pgfqpoint{4.296417in}{1.464017in}}%
\pgfpathlineto{\pgfqpoint{4.282513in}{1.471624in}}%
\pgfpathlineto{\pgfqpoint{4.268614in}{1.479254in}}%
\pgfpathlineto{\pgfqpoint{4.276470in}{1.470478in}}%
\pgfpathlineto{\pgfqpoint{4.284318in}{1.462281in}}%
\pgfpathlineto{\pgfqpoint{4.292157in}{1.454652in}}%
\pgfpathlineto{\pgfqpoint{4.299987in}{1.447578in}}%
\pgfpathclose%
\pgfusepath{fill}%
\end{pgfscope}%
\begin{pgfscope}%
\pgfpathrectangle{\pgfqpoint{1.254980in}{0.150000in}}{\pgfqpoint{5.490039in}{5.490039in}}%
\pgfusepath{clip}%
\pgfsetbuttcap%
\pgfsetroundjoin%
\definecolor{currentfill}{rgb}{0.352360,0.783011,0.392636}%
\pgfsetfillcolor{currentfill}%
\pgfsetfillopacity{0.700000}%
\pgfsetlinewidth{0.000000pt}%
\definecolor{currentstroke}{rgb}{0.000000,0.000000,0.000000}%
\pgfsetstrokecolor{currentstroke}%
\pgfsetdash{}{0pt}%
\pgfpathmoveto{\pgfqpoint{2.459067in}{2.950645in}}%
\pgfpathlineto{\pgfqpoint{2.472777in}{2.938105in}}%
\pgfpathlineto{\pgfqpoint{2.486487in}{2.925605in}}%
\pgfpathlineto{\pgfqpoint{2.500197in}{2.913144in}}%
\pgfpathlineto{\pgfqpoint{2.513908in}{2.900722in}}%
\pgfpathlineto{\pgfqpoint{2.504099in}{2.932971in}}%
\pgfpathlineto{\pgfqpoint{2.494240in}{2.966143in}}%
\pgfpathlineto{\pgfqpoint{2.484332in}{3.000258in}}%
\pgfpathlineto{\pgfqpoint{2.474373in}{3.035331in}}%
\pgfpathlineto{\pgfqpoint{2.460584in}{3.048272in}}%
\pgfpathlineto{\pgfqpoint{2.446796in}{3.061253in}}%
\pgfpathlineto{\pgfqpoint{2.433008in}{3.074274in}}%
\pgfpathlineto{\pgfqpoint{2.419220in}{3.087334in}}%
\pgfpathlineto{\pgfqpoint{2.429260in}{3.051732in}}%
\pgfpathlineto{\pgfqpoint{2.439247in}{3.017095in}}%
\pgfpathlineto{\pgfqpoint{2.449182in}{2.983405in}}%
\pgfpathlineto{\pgfqpoint{2.459067in}{2.950645in}}%
\pgfpathclose%
\pgfusepath{fill}%
\end{pgfscope}%
\begin{pgfscope}%
\pgfpathrectangle{\pgfqpoint{1.254980in}{0.150000in}}{\pgfqpoint{5.490039in}{5.490039in}}%
\pgfusepath{clip}%
\pgfsetbuttcap%
\pgfsetroundjoin%
\definecolor{currentfill}{rgb}{0.282910,0.105393,0.426902}%
\pgfsetfillcolor{currentfill}%
\pgfsetfillopacity{0.700000}%
\pgfsetlinewidth{0.000000pt}%
\definecolor{currentstroke}{rgb}{0.000000,0.000000,0.000000}%
\pgfsetstrokecolor{currentstroke}%
\pgfsetdash{}{0pt}%
\pgfpathmoveto{\pgfqpoint{4.751160in}{1.238176in}}%
\pgfpathlineto{\pgfqpoint{4.765131in}{1.232377in}}%
\pgfpathlineto{\pgfqpoint{4.779108in}{1.226600in}}%
\pgfpathlineto{\pgfqpoint{4.793091in}{1.220846in}}%
\pgfpathlineto{\pgfqpoint{4.807080in}{1.215113in}}%
\pgfpathlineto{\pgfqpoint{4.799493in}{1.215111in}}%
\pgfpathlineto{\pgfqpoint{4.791904in}{1.215546in}}%
\pgfpathlineto{\pgfqpoint{4.784314in}{1.216428in}}%
\pgfpathlineto{\pgfqpoint{4.776720in}{1.217768in}}%
\pgfpathlineto{\pgfqpoint{4.762711in}{1.223887in}}%
\pgfpathlineto{\pgfqpoint{4.748707in}{1.230027in}}%
\pgfpathlineto{\pgfqpoint{4.734709in}{1.236189in}}%
\pgfpathlineto{\pgfqpoint{4.720717in}{1.242374in}}%
\pgfpathlineto{\pgfqpoint{4.728332in}{1.240643in}}%
\pgfpathlineto{\pgfqpoint{4.735944in}{1.239373in}}%
\pgfpathlineto{\pgfqpoint{4.743553in}{1.238555in}}%
\pgfpathlineto{\pgfqpoint{4.751160in}{1.238176in}}%
\pgfpathclose%
\pgfusepath{fill}%
\end{pgfscope}%
\begin{pgfscope}%
\pgfpathrectangle{\pgfqpoint{1.254980in}{0.150000in}}{\pgfqpoint{5.490039in}{5.490039in}}%
\pgfusepath{clip}%
\pgfsetbuttcap%
\pgfsetroundjoin%
\definecolor{currentfill}{rgb}{0.137770,0.537492,0.554906}%
\pgfsetfillcolor{currentfill}%
\pgfsetfillopacity{0.700000}%
\pgfsetlinewidth{0.000000pt}%
\definecolor{currentstroke}{rgb}{0.000000,0.000000,0.000000}%
\pgfsetstrokecolor{currentstroke}%
\pgfsetdash{}{0pt}%
\pgfpathmoveto{\pgfqpoint{3.209099in}{2.248626in}}%
\pgfpathlineto{\pgfqpoint{3.222825in}{2.238302in}}%
\pgfpathlineto{\pgfqpoint{3.236553in}{2.228006in}}%
\pgfpathlineto{\pgfqpoint{3.250284in}{2.217738in}}%
\pgfpathlineto{\pgfqpoint{3.264017in}{2.207498in}}%
\pgfpathlineto{\pgfqpoint{3.255224in}{2.229982in}}%
\pgfpathlineto{\pgfqpoint{3.246401in}{2.253257in}}%
\pgfpathlineto{\pgfqpoint{3.237548in}{2.277337in}}%
\pgfpathlineto{\pgfqpoint{3.228664in}{2.302239in}}%
\pgfpathlineto{\pgfqpoint{3.214871in}{2.312958in}}%
\pgfpathlineto{\pgfqpoint{3.201081in}{2.323706in}}%
\pgfpathlineto{\pgfqpoint{3.187292in}{2.334483in}}%
\pgfpathlineto{\pgfqpoint{3.173507in}{2.345288in}}%
\pgfpathlineto{\pgfqpoint{3.182452in}{2.319898in}}%
\pgfpathlineto{\pgfqpoint{3.191366in}{2.295335in}}%
\pgfpathlineto{\pgfqpoint{3.200248in}{2.271583in}}%
\pgfpathlineto{\pgfqpoint{3.209099in}{2.248626in}}%
\pgfpathclose%
\pgfusepath{fill}%
\end{pgfscope}%
\begin{pgfscope}%
\pgfpathrectangle{\pgfqpoint{1.254980in}{0.150000in}}{\pgfqpoint{5.490039in}{5.490039in}}%
\pgfusepath{clip}%
\pgfsetbuttcap%
\pgfsetroundjoin%
\definecolor{currentfill}{rgb}{0.282290,0.145912,0.461510}%
\pgfsetfillcolor{currentfill}%
\pgfsetfillopacity{0.700000}%
\pgfsetlinewidth{0.000000pt}%
\definecolor{currentstroke}{rgb}{0.000000,0.000000,0.000000}%
\pgfsetstrokecolor{currentstroke}%
\pgfsetdash{}{0pt}%
\pgfpathmoveto{\pgfqpoint{4.553261in}{1.318322in}}%
\pgfpathlineto{\pgfqpoint{4.567185in}{1.311870in}}%
\pgfpathlineto{\pgfqpoint{4.581114in}{1.305441in}}%
\pgfpathlineto{\pgfqpoint{4.595048in}{1.299034in}}%
\pgfpathlineto{\pgfqpoint{4.608989in}{1.292650in}}%
\pgfpathlineto{\pgfqpoint{4.601323in}{1.295642in}}%
\pgfpathlineto{\pgfqpoint{4.593654in}{1.299126in}}%
\pgfpathlineto{\pgfqpoint{4.585980in}{1.303111in}}%
\pgfpathlineto{\pgfqpoint{4.578301in}{1.307610in}}%
\pgfpathlineto{\pgfqpoint{4.564335in}{1.314395in}}%
\pgfpathlineto{\pgfqpoint{4.550375in}{1.321202in}}%
\pgfpathlineto{\pgfqpoint{4.536419in}{1.328031in}}%
\pgfpathlineto{\pgfqpoint{4.522470in}{1.334883in}}%
\pgfpathlineto{\pgfqpoint{4.530176in}{1.329979in}}%
\pgfpathlineto{\pgfqpoint{4.537876in}{1.325591in}}%
\pgfpathlineto{\pgfqpoint{4.545571in}{1.321709in}}%
\pgfpathlineto{\pgfqpoint{4.553261in}{1.318322in}}%
\pgfpathclose%
\pgfusepath{fill}%
\end{pgfscope}%
\begin{pgfscope}%
\pgfpathrectangle{\pgfqpoint{1.254980in}{0.150000in}}{\pgfqpoint{5.490039in}{5.490039in}}%
\pgfusepath{clip}%
\pgfsetbuttcap%
\pgfsetroundjoin%
\definecolor{currentfill}{rgb}{0.197636,0.391528,0.554969}%
\pgfsetfillcolor{currentfill}%
\pgfsetfillopacity{0.700000}%
\pgfsetlinewidth{0.000000pt}%
\definecolor{currentstroke}{rgb}{0.000000,0.000000,0.000000}%
\pgfsetstrokecolor{currentstroke}%
\pgfsetdash{}{0pt}%
\pgfpathmoveto{\pgfqpoint{3.683160in}{1.862207in}}%
\pgfpathlineto{\pgfqpoint{3.696932in}{1.853190in}}%
\pgfpathlineto{\pgfqpoint{3.710708in}{1.844198in}}%
\pgfpathlineto{\pgfqpoint{3.724487in}{1.835230in}}%
\pgfpathlineto{\pgfqpoint{3.738270in}{1.826288in}}%
\pgfpathlineto{\pgfqpoint{3.729986in}{1.842156in}}%
\pgfpathlineto{\pgfqpoint{3.721683in}{1.858720in}}%
\pgfpathlineto{\pgfqpoint{3.713361in}{1.875993in}}%
\pgfpathlineto{\pgfqpoint{3.705018in}{1.893989in}}%
\pgfpathlineto{\pgfqpoint{3.691188in}{1.903386in}}%
\pgfpathlineto{\pgfqpoint{3.677361in}{1.912808in}}%
\pgfpathlineto{\pgfqpoint{3.663537in}{1.922255in}}%
\pgfpathlineto{\pgfqpoint{3.649717in}{1.931727in}}%
\pgfpathlineto{\pgfqpoint{3.658108in}{1.913269in}}%
\pgfpathlineto{\pgfqpoint{3.666479in}{1.895539in}}%
\pgfpathlineto{\pgfqpoint{3.674830in}{1.878523in}}%
\pgfpathlineto{\pgfqpoint{3.683160in}{1.862207in}}%
\pgfpathclose%
\pgfusepath{fill}%
\end{pgfscope}%
\begin{pgfscope}%
\pgfpathrectangle{\pgfqpoint{1.254980in}{0.150000in}}{\pgfqpoint{5.490039in}{5.490039in}}%
\pgfusepath{clip}%
\pgfsetbuttcap%
\pgfsetroundjoin%
\definecolor{currentfill}{rgb}{0.311925,0.767822,0.415586}%
\pgfsetfillcolor{currentfill}%
\pgfsetfillopacity{0.700000}%
\pgfsetlinewidth{0.000000pt}%
\definecolor{currentstroke}{rgb}{0.000000,0.000000,0.000000}%
\pgfsetstrokecolor{currentstroke}%
\pgfsetdash{}{0pt}%
\pgfpathmoveto{\pgfqpoint{2.513908in}{2.900722in}}%
\pgfpathlineto{\pgfqpoint{2.527620in}{2.888339in}}%
\pgfpathlineto{\pgfqpoint{2.541332in}{2.875994in}}%
\pgfpathlineto{\pgfqpoint{2.555045in}{2.863687in}}%
\pgfpathlineto{\pgfqpoint{2.568759in}{2.851418in}}%
\pgfpathlineto{\pgfqpoint{2.559025in}{2.883157in}}%
\pgfpathlineto{\pgfqpoint{2.549243in}{2.915814in}}%
\pgfpathlineto{\pgfqpoint{2.539412in}{2.949407in}}%
\pgfpathlineto{\pgfqpoint{2.529532in}{2.983953in}}%
\pgfpathlineto{\pgfqpoint{2.515741in}{2.996740in}}%
\pgfpathlineto{\pgfqpoint{2.501951in}{3.009565in}}%
\pgfpathlineto{\pgfqpoint{2.488162in}{3.022428in}}%
\pgfpathlineto{\pgfqpoint{2.474373in}{3.035331in}}%
\pgfpathlineto{\pgfqpoint{2.484332in}{3.000258in}}%
\pgfpathlineto{\pgfqpoint{2.494240in}{2.966143in}}%
\pgfpathlineto{\pgfqpoint{2.504099in}{2.932971in}}%
\pgfpathlineto{\pgfqpoint{2.513908in}{2.900722in}}%
\pgfpathclose%
\pgfusepath{fill}%
\end{pgfscope}%
\begin{pgfscope}%
\pgfpathrectangle{\pgfqpoint{1.254980in}{0.150000in}}{\pgfqpoint{5.490039in}{5.490039in}}%
\pgfusepath{clip}%
\pgfsetbuttcap%
\pgfsetroundjoin%
\definecolor{currentfill}{rgb}{0.244972,0.287675,0.537260}%
\pgfsetfillcolor{currentfill}%
\pgfsetfillopacity{0.700000}%
\pgfsetlinewidth{0.000000pt}%
\definecolor{currentstroke}{rgb}{0.000000,0.000000,0.000000}%
\pgfsetstrokecolor{currentstroke}%
\pgfsetdash{}{0pt}%
\pgfpathmoveto{\pgfqpoint{4.046856in}{1.604472in}}%
\pgfpathlineto{\pgfqpoint{4.060682in}{1.596471in}}%
\pgfpathlineto{\pgfqpoint{4.074512in}{1.588494in}}%
\pgfpathlineto{\pgfqpoint{4.088347in}{1.580540in}}%
\pgfpathlineto{\pgfqpoint{4.102186in}{1.572610in}}%
\pgfpathlineto{\pgfqpoint{4.094213in}{1.583259in}}%
\pgfpathlineto{\pgfqpoint{4.086227in}{1.594525in}}%
\pgfpathlineto{\pgfqpoint{4.078229in}{1.606421in}}%
\pgfpathlineto{\pgfqpoint{4.070219in}{1.618959in}}%
\pgfpathlineto{\pgfqpoint{4.056341in}{1.627323in}}%
\pgfpathlineto{\pgfqpoint{4.042467in}{1.635711in}}%
\pgfpathlineto{\pgfqpoint{4.028598in}{1.644122in}}%
\pgfpathlineto{\pgfqpoint{4.014732in}{1.652557in}}%
\pgfpathlineto{\pgfqpoint{4.022783in}{1.639579in}}%
\pgfpathlineto{\pgfqpoint{4.030821in}{1.627247in}}%
\pgfpathlineto{\pgfqpoint{4.038845in}{1.615549in}}%
\pgfpathlineto{\pgfqpoint{4.046856in}{1.604472in}}%
\pgfpathclose%
\pgfusepath{fill}%
\end{pgfscope}%
\begin{pgfscope}%
\pgfpathrectangle{\pgfqpoint{1.254980in}{0.150000in}}{\pgfqpoint{5.490039in}{5.490039in}}%
\pgfusepath{clip}%
\pgfsetbuttcap%
\pgfsetroundjoin%
\definecolor{currentfill}{rgb}{0.141935,0.526453,0.555991}%
\pgfsetfillcolor{currentfill}%
\pgfsetfillopacity{0.700000}%
\pgfsetlinewidth{0.000000pt}%
\definecolor{currentstroke}{rgb}{0.000000,0.000000,0.000000}%
\pgfsetstrokecolor{currentstroke}%
\pgfsetdash{}{0pt}%
\pgfpathmoveto{\pgfqpoint{3.264017in}{2.207498in}}%
\pgfpathlineto{\pgfqpoint{3.277753in}{2.197287in}}%
\pgfpathlineto{\pgfqpoint{3.291492in}{2.187102in}}%
\pgfpathlineto{\pgfqpoint{3.305233in}{2.176946in}}%
\pgfpathlineto{\pgfqpoint{3.318977in}{2.166817in}}%
\pgfpathlineto{\pgfqpoint{3.310241in}{2.188829in}}%
\pgfpathlineto{\pgfqpoint{3.301477in}{2.211627in}}%
\pgfpathlineto{\pgfqpoint{3.292683in}{2.235226in}}%
\pgfpathlineto{\pgfqpoint{3.283859in}{2.259641in}}%
\pgfpathlineto{\pgfqpoint{3.270057in}{2.270248in}}%
\pgfpathlineto{\pgfqpoint{3.256257in}{2.280884in}}%
\pgfpathlineto{\pgfqpoint{3.242459in}{2.291547in}}%
\pgfpathlineto{\pgfqpoint{3.228664in}{2.302239in}}%
\pgfpathlineto{\pgfqpoint{3.237548in}{2.277337in}}%
\pgfpathlineto{\pgfqpoint{3.246401in}{2.253257in}}%
\pgfpathlineto{\pgfqpoint{3.255224in}{2.229982in}}%
\pgfpathlineto{\pgfqpoint{3.264017in}{2.207498in}}%
\pgfpathclose%
\pgfusepath{fill}%
\end{pgfscope}%
\begin{pgfscope}%
\pgfpathrectangle{\pgfqpoint{1.254980in}{0.150000in}}{\pgfqpoint{5.490039in}{5.490039in}}%
\pgfusepath{clip}%
\pgfsetbuttcap%
\pgfsetroundjoin%
\definecolor{currentfill}{rgb}{0.281477,0.755203,0.432552}%
\pgfsetfillcolor{currentfill}%
\pgfsetfillopacity{0.700000}%
\pgfsetlinewidth{0.000000pt}%
\definecolor{currentstroke}{rgb}{0.000000,0.000000,0.000000}%
\pgfsetstrokecolor{currentstroke}%
\pgfsetdash{}{0pt}%
\pgfpathmoveto{\pgfqpoint{2.568759in}{2.851418in}}%
\pgfpathlineto{\pgfqpoint{2.582474in}{2.839186in}}%
\pgfpathlineto{\pgfqpoint{2.596190in}{2.826992in}}%
\pgfpathlineto{\pgfqpoint{2.609907in}{2.814834in}}%
\pgfpathlineto{\pgfqpoint{2.623624in}{2.802713in}}%
\pgfpathlineto{\pgfqpoint{2.613963in}{2.833944in}}%
\pgfpathlineto{\pgfqpoint{2.604257in}{2.866088in}}%
\pgfpathlineto{\pgfqpoint{2.594502in}{2.899161in}}%
\pgfpathlineto{\pgfqpoint{2.584700in}{2.933182in}}%
\pgfpathlineto{\pgfqpoint{2.570907in}{2.945819in}}%
\pgfpathlineto{\pgfqpoint{2.557114in}{2.958493in}}%
\pgfpathlineto{\pgfqpoint{2.543323in}{2.971204in}}%
\pgfpathlineto{\pgfqpoint{2.529532in}{2.983953in}}%
\pgfpathlineto{\pgfqpoint{2.539412in}{2.949407in}}%
\pgfpathlineto{\pgfqpoint{2.549243in}{2.915814in}}%
\pgfpathlineto{\pgfqpoint{2.559025in}{2.883157in}}%
\pgfpathlineto{\pgfqpoint{2.568759in}{2.851418in}}%
\pgfpathclose%
\pgfusepath{fill}%
\end{pgfscope}%
\begin{pgfscope}%
\pgfpathrectangle{\pgfqpoint{1.254980in}{0.150000in}}{\pgfqpoint{5.490039in}{5.490039in}}%
\pgfusepath{clip}%
\pgfsetbuttcap%
\pgfsetroundjoin%
\definecolor{currentfill}{rgb}{0.273006,0.204520,0.501721}%
\pgfsetfillcolor{currentfill}%
\pgfsetfillopacity{0.700000}%
\pgfsetlinewidth{0.000000pt}%
\definecolor{currentstroke}{rgb}{0.000000,0.000000,0.000000}%
\pgfsetstrokecolor{currentstroke}%
\pgfsetdash{}{0pt}%
\pgfpathmoveto{\pgfqpoint{4.355484in}{1.418861in}}%
\pgfpathlineto{\pgfqpoint{4.369371in}{1.411738in}}%
\pgfpathlineto{\pgfqpoint{4.383264in}{1.404639in}}%
\pgfpathlineto{\pgfqpoint{4.397161in}{1.397562in}}%
\pgfpathlineto{\pgfqpoint{4.411063in}{1.390507in}}%
\pgfpathlineto{\pgfqpoint{4.403294in}{1.396760in}}%
\pgfpathlineto{\pgfqpoint{4.395518in}{1.403561in}}%
\pgfpathlineto{\pgfqpoint{4.387735in}{1.410922in}}%
\pgfpathlineto{\pgfqpoint{4.379944in}{1.418854in}}%
\pgfpathlineto{\pgfqpoint{4.366011in}{1.426324in}}%
\pgfpathlineto{\pgfqpoint{4.352082in}{1.433817in}}%
\pgfpathlineto{\pgfqpoint{4.338159in}{1.441333in}}%
\pgfpathlineto{\pgfqpoint{4.324240in}{1.448872in}}%
\pgfpathlineto{\pgfqpoint{4.332063in}{1.440518in}}%
\pgfpathlineto{\pgfqpoint{4.339878in}{1.432739in}}%
\pgfpathlineto{\pgfqpoint{4.347685in}{1.425524in}}%
\pgfpathlineto{\pgfqpoint{4.355484in}{1.418861in}}%
\pgfpathclose%
\pgfusepath{fill}%
\end{pgfscope}%
\begin{pgfscope}%
\pgfpathrectangle{\pgfqpoint{1.254980in}{0.150000in}}{\pgfqpoint{5.490039in}{5.490039in}}%
\pgfusepath{clip}%
\pgfsetbuttcap%
\pgfsetroundjoin%
\definecolor{currentfill}{rgb}{0.201239,0.383670,0.554294}%
\pgfsetfillcolor{currentfill}%
\pgfsetfillopacity{0.700000}%
\pgfsetlinewidth{0.000000pt}%
\definecolor{currentstroke}{rgb}{0.000000,0.000000,0.000000}%
\pgfsetstrokecolor{currentstroke}%
\pgfsetdash{}{0pt}%
\pgfpathmoveto{\pgfqpoint{3.738270in}{1.826288in}}%
\pgfpathlineto{\pgfqpoint{3.752056in}{1.817370in}}%
\pgfpathlineto{\pgfqpoint{3.765847in}{1.808477in}}%
\pgfpathlineto{\pgfqpoint{3.779641in}{1.799608in}}%
\pgfpathlineto{\pgfqpoint{3.793438in}{1.790764in}}%
\pgfpathlineto{\pgfqpoint{3.785200in}{1.806186in}}%
\pgfpathlineto{\pgfqpoint{3.776944in}{1.822298in}}%
\pgfpathlineto{\pgfqpoint{3.768669in}{1.839115in}}%
\pgfpathlineto{\pgfqpoint{3.760375in}{1.856650in}}%
\pgfpathlineto{\pgfqpoint{3.746531in}{1.865948in}}%
\pgfpathlineto{\pgfqpoint{3.732690in}{1.875270in}}%
\pgfpathlineto{\pgfqpoint{3.718852in}{1.884617in}}%
\pgfpathlineto{\pgfqpoint{3.705018in}{1.893989in}}%
\pgfpathlineto{\pgfqpoint{3.713361in}{1.875993in}}%
\pgfpathlineto{\pgfqpoint{3.721683in}{1.858720in}}%
\pgfpathlineto{\pgfqpoint{3.729986in}{1.842156in}}%
\pgfpathlineto{\pgfqpoint{3.738270in}{1.826288in}}%
\pgfpathclose%
\pgfusepath{fill}%
\end{pgfscope}%
\begin{pgfscope}%
\pgfpathrectangle{\pgfqpoint{1.254980in}{0.150000in}}{\pgfqpoint{5.490039in}{5.490039in}}%
\pgfusepath{clip}%
\pgfsetbuttcap%
\pgfsetroundjoin%
\definecolor{currentfill}{rgb}{0.246070,0.738910,0.452024}%
\pgfsetfillcolor{currentfill}%
\pgfsetfillopacity{0.700000}%
\pgfsetlinewidth{0.000000pt}%
\definecolor{currentstroke}{rgb}{0.000000,0.000000,0.000000}%
\pgfsetstrokecolor{currentstroke}%
\pgfsetdash{}{0pt}%
\pgfpathmoveto{\pgfqpoint{2.623624in}{2.802713in}}%
\pgfpathlineto{\pgfqpoint{2.637343in}{2.790628in}}%
\pgfpathlineto{\pgfqpoint{2.651062in}{2.778580in}}%
\pgfpathlineto{\pgfqpoint{2.664783in}{2.766567in}}%
\pgfpathlineto{\pgfqpoint{2.678505in}{2.754590in}}%
\pgfpathlineto{\pgfqpoint{2.668917in}{2.785315in}}%
\pgfpathlineto{\pgfqpoint{2.659284in}{2.816947in}}%
\pgfpathlineto{\pgfqpoint{2.649605in}{2.849502in}}%
\pgfpathlineto{\pgfqpoint{2.639879in}{2.883000in}}%
\pgfpathlineto{\pgfqpoint{2.626083in}{2.895491in}}%
\pgfpathlineto{\pgfqpoint{2.612288in}{2.908018in}}%
\pgfpathlineto{\pgfqpoint{2.598493in}{2.920582in}}%
\pgfpathlineto{\pgfqpoint{2.584700in}{2.933182in}}%
\pgfpathlineto{\pgfqpoint{2.594502in}{2.899161in}}%
\pgfpathlineto{\pgfqpoint{2.604257in}{2.866088in}}%
\pgfpathlineto{\pgfqpoint{2.613963in}{2.833944in}}%
\pgfpathlineto{\pgfqpoint{2.623624in}{2.802713in}}%
\pgfpathclose%
\pgfusepath{fill}%
\end{pgfscope}%
\begin{pgfscope}%
\pgfpathrectangle{\pgfqpoint{1.254980in}{0.150000in}}{\pgfqpoint{5.490039in}{5.490039in}}%
\pgfusepath{clip}%
\pgfsetbuttcap%
\pgfsetroundjoin%
\definecolor{currentfill}{rgb}{0.282656,0.100196,0.422160}%
\pgfsetfillcolor{currentfill}%
\pgfsetfillopacity{0.700000}%
\pgfsetlinewidth{0.000000pt}%
\definecolor{currentstroke}{rgb}{0.000000,0.000000,0.000000}%
\pgfsetstrokecolor{currentstroke}%
\pgfsetdash{}{0pt}%
\pgfpathmoveto{\pgfqpoint{4.807080in}{1.215113in}}%
\pgfpathlineto{\pgfqpoint{4.821075in}{1.209402in}}%
\pgfpathlineto{\pgfqpoint{4.835077in}{1.203714in}}%
\pgfpathlineto{\pgfqpoint{4.849084in}{1.198048in}}%
\pgfpathlineto{\pgfqpoint{4.863098in}{1.192403in}}%
\pgfpathlineto{\pgfqpoint{4.855531in}{1.192021in}}%
\pgfpathlineto{\pgfqpoint{4.847962in}{1.192072in}}%
\pgfpathlineto{\pgfqpoint{4.840391in}{1.192567in}}%
\pgfpathlineto{\pgfqpoint{4.832819in}{1.193516in}}%
\pgfpathlineto{\pgfqpoint{4.818785in}{1.199546in}}%
\pgfpathlineto{\pgfqpoint{4.804758in}{1.205598in}}%
\pgfpathlineto{\pgfqpoint{4.790736in}{1.211672in}}%
\pgfpathlineto{\pgfqpoint{4.776720in}{1.217768in}}%
\pgfpathlineto{\pgfqpoint{4.784314in}{1.216428in}}%
\pgfpathlineto{\pgfqpoint{4.791904in}{1.215546in}}%
\pgfpathlineto{\pgfqpoint{4.799493in}{1.215111in}}%
\pgfpathlineto{\pgfqpoint{4.807080in}{1.215113in}}%
\pgfpathclose%
\pgfusepath{fill}%
\end{pgfscope}%
\begin{pgfscope}%
\pgfpathrectangle{\pgfqpoint{1.254980in}{0.150000in}}{\pgfqpoint{5.490039in}{5.490039in}}%
\pgfusepath{clip}%
\pgfsetbuttcap%
\pgfsetroundjoin%
\definecolor{currentfill}{rgb}{0.282623,0.140926,0.457517}%
\pgfsetfillcolor{currentfill}%
\pgfsetfillopacity{0.700000}%
\pgfsetlinewidth{0.000000pt}%
\definecolor{currentstroke}{rgb}{0.000000,0.000000,0.000000}%
\pgfsetstrokecolor{currentstroke}%
\pgfsetdash{}{0pt}%
\pgfpathmoveto{\pgfqpoint{4.608989in}{1.292650in}}%
\pgfpathlineto{\pgfqpoint{4.622935in}{1.286287in}}%
\pgfpathlineto{\pgfqpoint{4.636886in}{1.279947in}}%
\pgfpathlineto{\pgfqpoint{4.650844in}{1.273629in}}%
\pgfpathlineto{\pgfqpoint{4.664807in}{1.267334in}}%
\pgfpathlineto{\pgfqpoint{4.657165in}{1.269932in}}%
\pgfpathlineto{\pgfqpoint{4.649521in}{1.273017in}}%
\pgfpathlineto{\pgfqpoint{4.641872in}{1.276600in}}%
\pgfpathlineto{\pgfqpoint{4.634219in}{1.280693in}}%
\pgfpathlineto{\pgfqpoint{4.620231in}{1.287389in}}%
\pgfpathlineto{\pgfqpoint{4.606249in}{1.294107in}}%
\pgfpathlineto{\pgfqpoint{4.592272in}{1.300847in}}%
\pgfpathlineto{\pgfqpoint{4.578301in}{1.307610in}}%
\pgfpathlineto{\pgfqpoint{4.585980in}{1.303111in}}%
\pgfpathlineto{\pgfqpoint{4.593654in}{1.299126in}}%
\pgfpathlineto{\pgfqpoint{4.601323in}{1.295642in}}%
\pgfpathlineto{\pgfqpoint{4.608989in}{1.292650in}}%
\pgfpathclose%
\pgfusepath{fill}%
\end{pgfscope}%
\begin{pgfscope}%
\pgfpathrectangle{\pgfqpoint{1.254980in}{0.150000in}}{\pgfqpoint{5.490039in}{5.490039in}}%
\pgfusepath{clip}%
\pgfsetbuttcap%
\pgfsetroundjoin%
\definecolor{currentfill}{rgb}{0.248629,0.278775,0.534556}%
\pgfsetfillcolor{currentfill}%
\pgfsetfillopacity{0.700000}%
\pgfsetlinewidth{0.000000pt}%
\definecolor{currentstroke}{rgb}{0.000000,0.000000,0.000000}%
\pgfsetstrokecolor{currentstroke}%
\pgfsetdash{}{0pt}%
\pgfpathmoveto{\pgfqpoint{4.102186in}{1.572610in}}%
\pgfpathlineto{\pgfqpoint{4.116030in}{1.564702in}}%
\pgfpathlineto{\pgfqpoint{4.129878in}{1.556819in}}%
\pgfpathlineto{\pgfqpoint{4.143731in}{1.548958in}}%
\pgfpathlineto{\pgfqpoint{4.157588in}{1.541121in}}%
\pgfpathlineto{\pgfqpoint{4.149652in}{1.551343in}}%
\pgfpathlineto{\pgfqpoint{4.141704in}{1.562178in}}%
\pgfpathlineto{\pgfqpoint{4.133744in}{1.573638in}}%
\pgfpathlineto{\pgfqpoint{4.125773in}{1.585736in}}%
\pgfpathlineto{\pgfqpoint{4.111878in}{1.594007in}}%
\pgfpathlineto{\pgfqpoint{4.097987in}{1.602301in}}%
\pgfpathlineto{\pgfqpoint{4.084101in}{1.610618in}}%
\pgfpathlineto{\pgfqpoint{4.070219in}{1.618959in}}%
\pgfpathlineto{\pgfqpoint{4.078229in}{1.606421in}}%
\pgfpathlineto{\pgfqpoint{4.086227in}{1.594525in}}%
\pgfpathlineto{\pgfqpoint{4.094213in}{1.583259in}}%
\pgfpathlineto{\pgfqpoint{4.102186in}{1.572610in}}%
\pgfpathclose%
\pgfusepath{fill}%
\end{pgfscope}%
\begin{pgfscope}%
\pgfpathrectangle{\pgfqpoint{1.254980in}{0.150000in}}{\pgfqpoint{5.490039in}{5.490039in}}%
\pgfusepath{clip}%
\pgfsetbuttcap%
\pgfsetroundjoin%
\definecolor{currentfill}{rgb}{0.147607,0.511733,0.557049}%
\pgfsetfillcolor{currentfill}%
\pgfsetfillopacity{0.700000}%
\pgfsetlinewidth{0.000000pt}%
\definecolor{currentstroke}{rgb}{0.000000,0.000000,0.000000}%
\pgfsetstrokecolor{currentstroke}%
\pgfsetdash{}{0pt}%
\pgfpathmoveto{\pgfqpoint{3.318977in}{2.166817in}}%
\pgfpathlineto{\pgfqpoint{3.332724in}{2.156716in}}%
\pgfpathlineto{\pgfqpoint{3.346474in}{2.146642in}}%
\pgfpathlineto{\pgfqpoint{3.360226in}{2.136595in}}%
\pgfpathlineto{\pgfqpoint{3.373981in}{2.126575in}}%
\pgfpathlineto{\pgfqpoint{3.365301in}{2.148117in}}%
\pgfpathlineto{\pgfqpoint{3.356594in}{2.170439in}}%
\pgfpathlineto{\pgfqpoint{3.347858in}{2.193557in}}%
\pgfpathlineto{\pgfqpoint{3.339094in}{2.217486in}}%
\pgfpathlineto{\pgfqpoint{3.325282in}{2.227983in}}%
\pgfpathlineto{\pgfqpoint{3.311472in}{2.238508in}}%
\pgfpathlineto{\pgfqpoint{3.297664in}{2.249061in}}%
\pgfpathlineto{\pgfqpoint{3.283859in}{2.259641in}}%
\pgfpathlineto{\pgfqpoint{3.292683in}{2.235226in}}%
\pgfpathlineto{\pgfqpoint{3.301477in}{2.211627in}}%
\pgfpathlineto{\pgfqpoint{3.310241in}{2.188829in}}%
\pgfpathlineto{\pgfqpoint{3.318977in}{2.166817in}}%
\pgfpathclose%
\pgfusepath{fill}%
\end{pgfscope}%
\begin{pgfscope}%
\pgfpathrectangle{\pgfqpoint{1.254980in}{0.150000in}}{\pgfqpoint{5.490039in}{5.490039in}}%
\pgfusepath{clip}%
\pgfsetbuttcap%
\pgfsetroundjoin%
\definecolor{currentfill}{rgb}{0.220124,0.725509,0.466226}%
\pgfsetfillcolor{currentfill}%
\pgfsetfillopacity{0.700000}%
\pgfsetlinewidth{0.000000pt}%
\definecolor{currentstroke}{rgb}{0.000000,0.000000,0.000000}%
\pgfsetstrokecolor{currentstroke}%
\pgfsetdash{}{0pt}%
\pgfpathmoveto{\pgfqpoint{2.678505in}{2.754590in}}%
\pgfpathlineto{\pgfqpoint{2.692227in}{2.742649in}}%
\pgfpathlineto{\pgfqpoint{2.705951in}{2.730742in}}%
\pgfpathlineto{\pgfqpoint{2.719677in}{2.718870in}}%
\pgfpathlineto{\pgfqpoint{2.733403in}{2.707033in}}%
\pgfpathlineto{\pgfqpoint{2.723887in}{2.737253in}}%
\pgfpathlineto{\pgfqpoint{2.714327in}{2.768374in}}%
\pgfpathlineto{\pgfqpoint{2.704723in}{2.800414in}}%
\pgfpathlineto{\pgfqpoint{2.695073in}{2.833389in}}%
\pgfpathlineto{\pgfqpoint{2.681273in}{2.845739in}}%
\pgfpathlineto{\pgfqpoint{2.667474in}{2.858124in}}%
\pgfpathlineto{\pgfqpoint{2.653676in}{2.870544in}}%
\pgfpathlineto{\pgfqpoint{2.639879in}{2.883000in}}%
\pgfpathlineto{\pgfqpoint{2.649605in}{2.849502in}}%
\pgfpathlineto{\pgfqpoint{2.659284in}{2.816947in}}%
\pgfpathlineto{\pgfqpoint{2.668917in}{2.785315in}}%
\pgfpathlineto{\pgfqpoint{2.678505in}{2.754590in}}%
\pgfpathclose%
\pgfusepath{fill}%
\end{pgfscope}%
\begin{pgfscope}%
\pgfpathrectangle{\pgfqpoint{1.254980in}{0.150000in}}{\pgfqpoint{5.490039in}{5.490039in}}%
\pgfusepath{clip}%
\pgfsetbuttcap%
\pgfsetroundjoin%
\definecolor{currentfill}{rgb}{0.206756,0.371758,0.553117}%
\pgfsetfillcolor{currentfill}%
\pgfsetfillopacity{0.700000}%
\pgfsetlinewidth{0.000000pt}%
\definecolor{currentstroke}{rgb}{0.000000,0.000000,0.000000}%
\pgfsetstrokecolor{currentstroke}%
\pgfsetdash{}{0pt}%
\pgfpathmoveto{\pgfqpoint{3.793438in}{1.790764in}}%
\pgfpathlineto{\pgfqpoint{3.807240in}{1.781945in}}%
\pgfpathlineto{\pgfqpoint{3.821045in}{1.773150in}}%
\pgfpathlineto{\pgfqpoint{3.834854in}{1.764380in}}%
\pgfpathlineto{\pgfqpoint{3.848667in}{1.755633in}}%
\pgfpathlineto{\pgfqpoint{3.840474in}{1.770608in}}%
\pgfpathlineto{\pgfqpoint{3.832263in}{1.786269in}}%
\pgfpathlineto{\pgfqpoint{3.824035in}{1.802631in}}%
\pgfpathlineto{\pgfqpoint{3.815789in}{1.819706in}}%
\pgfpathlineto{\pgfqpoint{3.801930in}{1.828906in}}%
\pgfpathlineto{\pgfqpoint{3.788075in}{1.838129in}}%
\pgfpathlineto{\pgfqpoint{3.774223in}{1.847378in}}%
\pgfpathlineto{\pgfqpoint{3.760375in}{1.856650in}}%
\pgfpathlineto{\pgfqpoint{3.768669in}{1.839115in}}%
\pgfpathlineto{\pgfqpoint{3.776944in}{1.822298in}}%
\pgfpathlineto{\pgfqpoint{3.785200in}{1.806186in}}%
\pgfpathlineto{\pgfqpoint{3.793438in}{1.790764in}}%
\pgfpathclose%
\pgfusepath{fill}%
\end{pgfscope}%
\begin{pgfscope}%
\pgfpathrectangle{\pgfqpoint{1.254980in}{0.150000in}}{\pgfqpoint{5.490039in}{5.490039in}}%
\pgfusepath{clip}%
\pgfsetbuttcap%
\pgfsetroundjoin%
\definecolor{currentfill}{rgb}{0.275191,0.194905,0.496005}%
\pgfsetfillcolor{currentfill}%
\pgfsetfillopacity{0.700000}%
\pgfsetlinewidth{0.000000pt}%
\definecolor{currentstroke}{rgb}{0.000000,0.000000,0.000000}%
\pgfsetstrokecolor{currentstroke}%
\pgfsetdash{}{0pt}%
\pgfpathmoveto{\pgfqpoint{4.411063in}{1.390507in}}%
\pgfpathlineto{\pgfqpoint{4.424971in}{1.383475in}}%
\pgfpathlineto{\pgfqpoint{4.438883in}{1.376466in}}%
\pgfpathlineto{\pgfqpoint{4.452801in}{1.369479in}}%
\pgfpathlineto{\pgfqpoint{4.466724in}{1.362515in}}%
\pgfpathlineto{\pgfqpoint{4.458985in}{1.368358in}}%
\pgfpathlineto{\pgfqpoint{4.451239in}{1.374745in}}%
\pgfpathlineto{\pgfqpoint{4.443486in}{1.381688in}}%
\pgfpathlineto{\pgfqpoint{4.435727in}{1.389198in}}%
\pgfpathlineto{\pgfqpoint{4.421774in}{1.396578in}}%
\pgfpathlineto{\pgfqpoint{4.407825in}{1.403981in}}%
\pgfpathlineto{\pgfqpoint{4.393882in}{1.411406in}}%
\pgfpathlineto{\pgfqpoint{4.379944in}{1.418854in}}%
\pgfpathlineto{\pgfqpoint{4.387735in}{1.410922in}}%
\pgfpathlineto{\pgfqpoint{4.395518in}{1.403561in}}%
\pgfpathlineto{\pgfqpoint{4.403294in}{1.396760in}}%
\pgfpathlineto{\pgfqpoint{4.411063in}{1.390507in}}%
\pgfpathclose%
\pgfusepath{fill}%
\end{pgfscope}%
\begin{pgfscope}%
\pgfpathrectangle{\pgfqpoint{1.254980in}{0.150000in}}{\pgfqpoint{5.490039in}{5.490039in}}%
\pgfusepath{clip}%
\pgfsetbuttcap%
\pgfsetroundjoin%
\definecolor{currentfill}{rgb}{0.151918,0.500685,0.557587}%
\pgfsetfillcolor{currentfill}%
\pgfsetfillopacity{0.700000}%
\pgfsetlinewidth{0.000000pt}%
\definecolor{currentstroke}{rgb}{0.000000,0.000000,0.000000}%
\pgfsetstrokecolor{currentstroke}%
\pgfsetdash{}{0pt}%
\pgfpathmoveto{\pgfqpoint{3.373981in}{2.126575in}}%
\pgfpathlineto{\pgfqpoint{3.387739in}{2.116583in}}%
\pgfpathlineto{\pgfqpoint{3.401500in}{2.106617in}}%
\pgfpathlineto{\pgfqpoint{3.415264in}{2.096678in}}%
\pgfpathlineto{\pgfqpoint{3.429031in}{2.086766in}}%
\pgfpathlineto{\pgfqpoint{3.420406in}{2.107837in}}%
\pgfpathlineto{\pgfqpoint{3.411755in}{2.129685in}}%
\pgfpathlineto{\pgfqpoint{3.403077in}{2.152323in}}%
\pgfpathlineto{\pgfqpoint{3.394371in}{2.175767in}}%
\pgfpathlineto{\pgfqpoint{3.380548in}{2.186156in}}%
\pgfpathlineto{\pgfqpoint{3.366727in}{2.196572in}}%
\pgfpathlineto{\pgfqpoint{3.352910in}{2.207015in}}%
\pgfpathlineto{\pgfqpoint{3.339094in}{2.217486in}}%
\pgfpathlineto{\pgfqpoint{3.347858in}{2.193557in}}%
\pgfpathlineto{\pgfqpoint{3.356594in}{2.170439in}}%
\pgfpathlineto{\pgfqpoint{3.365301in}{2.148117in}}%
\pgfpathlineto{\pgfqpoint{3.373981in}{2.126575in}}%
\pgfpathclose%
\pgfusepath{fill}%
\end{pgfscope}%
\begin{pgfscope}%
\pgfpathrectangle{\pgfqpoint{1.254980in}{0.150000in}}{\pgfqpoint{5.490039in}{5.490039in}}%
\pgfusepath{clip}%
\pgfsetbuttcap%
\pgfsetroundjoin%
\definecolor{currentfill}{rgb}{0.196571,0.711827,0.479221}%
\pgfsetfillcolor{currentfill}%
\pgfsetfillopacity{0.700000}%
\pgfsetlinewidth{0.000000pt}%
\definecolor{currentstroke}{rgb}{0.000000,0.000000,0.000000}%
\pgfsetstrokecolor{currentstroke}%
\pgfsetdash{}{0pt}%
\pgfpathmoveto{\pgfqpoint{2.733403in}{2.707033in}}%
\pgfpathlineto{\pgfqpoint{2.747131in}{2.695230in}}%
\pgfpathlineto{\pgfqpoint{2.760860in}{2.683462in}}%
\pgfpathlineto{\pgfqpoint{2.774590in}{2.671727in}}%
\pgfpathlineto{\pgfqpoint{2.788322in}{2.660026in}}%
\pgfpathlineto{\pgfqpoint{2.778877in}{2.689742in}}%
\pgfpathlineto{\pgfqpoint{2.769389in}{2.720355in}}%
\pgfpathlineto{\pgfqpoint{2.759858in}{2.751880in}}%
\pgfpathlineto{\pgfqpoint{2.750283in}{2.784335in}}%
\pgfpathlineto{\pgfqpoint{2.736479in}{2.796547in}}%
\pgfpathlineto{\pgfqpoint{2.722676in}{2.808793in}}%
\pgfpathlineto{\pgfqpoint{2.708874in}{2.821074in}}%
\pgfpathlineto{\pgfqpoint{2.695073in}{2.833389in}}%
\pgfpathlineto{\pgfqpoint{2.704723in}{2.800414in}}%
\pgfpathlineto{\pgfqpoint{2.714327in}{2.768374in}}%
\pgfpathlineto{\pgfqpoint{2.723887in}{2.737253in}}%
\pgfpathlineto{\pgfqpoint{2.733403in}{2.707033in}}%
\pgfpathclose%
\pgfusepath{fill}%
\end{pgfscope}%
\begin{pgfscope}%
\pgfpathrectangle{\pgfqpoint{1.254980in}{0.150000in}}{\pgfqpoint{5.490039in}{5.490039in}}%
\pgfusepath{clip}%
\pgfsetbuttcap%
\pgfsetroundjoin%
\definecolor{currentfill}{rgb}{0.252194,0.269783,0.531579}%
\pgfsetfillcolor{currentfill}%
\pgfsetfillopacity{0.700000}%
\pgfsetlinewidth{0.000000pt}%
\definecolor{currentstroke}{rgb}{0.000000,0.000000,0.000000}%
\pgfsetstrokecolor{currentstroke}%
\pgfsetdash{}{0pt}%
\pgfpathmoveto{\pgfqpoint{4.157588in}{1.541121in}}%
\pgfpathlineto{\pgfqpoint{4.171450in}{1.533307in}}%
\pgfpathlineto{\pgfqpoint{4.185317in}{1.525516in}}%
\pgfpathlineto{\pgfqpoint{4.199188in}{1.517748in}}%
\pgfpathlineto{\pgfqpoint{4.213064in}{1.510003in}}%
\pgfpathlineto{\pgfqpoint{4.205163in}{1.519798in}}%
\pgfpathlineto{\pgfqpoint{4.197252in}{1.530202in}}%
\pgfpathlineto{\pgfqpoint{4.189330in}{1.541227in}}%
\pgfpathlineto{\pgfqpoint{4.181397in}{1.552886in}}%
\pgfpathlineto{\pgfqpoint{4.167484in}{1.561064in}}%
\pgfpathlineto{\pgfqpoint{4.153576in}{1.569265in}}%
\pgfpathlineto{\pgfqpoint{4.139672in}{1.577489in}}%
\pgfpathlineto{\pgfqpoint{4.125773in}{1.585736in}}%
\pgfpathlineto{\pgfqpoint{4.133744in}{1.573638in}}%
\pgfpathlineto{\pgfqpoint{4.141704in}{1.562178in}}%
\pgfpathlineto{\pgfqpoint{4.149652in}{1.551343in}}%
\pgfpathlineto{\pgfqpoint{4.157588in}{1.541121in}}%
\pgfpathclose%
\pgfusepath{fill}%
\end{pgfscope}%
\begin{pgfscope}%
\pgfpathrectangle{\pgfqpoint{1.254980in}{0.150000in}}{\pgfqpoint{5.490039in}{5.490039in}}%
\pgfusepath{clip}%
\pgfsetbuttcap%
\pgfsetroundjoin%
\definecolor{currentfill}{rgb}{0.282656,0.100196,0.422160}%
\pgfsetfillcolor{currentfill}%
\pgfsetfillopacity{0.700000}%
\pgfsetlinewidth{0.000000pt}%
\definecolor{currentstroke}{rgb}{0.000000,0.000000,0.000000}%
\pgfsetstrokecolor{currentstroke}%
\pgfsetdash{}{0pt}%
\pgfpathmoveto{\pgfqpoint{4.863098in}{1.192403in}}%
\pgfpathlineto{\pgfqpoint{4.877118in}{1.186781in}}%
\pgfpathlineto{\pgfqpoint{4.891145in}{1.181181in}}%
\pgfpathlineto{\pgfqpoint{4.905177in}{1.175603in}}%
\pgfpathlineto{\pgfqpoint{4.897624in}{1.174935in}}%
\pgfpathlineto{\pgfqpoint{4.890069in}{1.174699in}}%
\pgfpathlineto{\pgfqpoint{4.882513in}{1.174903in}}%
\pgfpathlineto{\pgfqpoint{4.874956in}{1.175558in}}%
\pgfpathlineto{\pgfqpoint{4.860904in}{1.181522in}}%
\pgfpathlineto{\pgfqpoint{4.846859in}{1.187508in}}%
\pgfpathlineto{\pgfqpoint{4.832819in}{1.193516in}}%
\pgfpathlineto{\pgfqpoint{4.840391in}{1.192567in}}%
\pgfpathlineto{\pgfqpoint{4.847962in}{1.192072in}}%
\pgfpathlineto{\pgfqpoint{4.855531in}{1.192021in}}%
\pgfpathlineto{\pgfqpoint{4.863098in}{1.192403in}}%
\pgfpathclose%
\pgfusepath{fill}%
\end{pgfscope}%
\begin{pgfscope}%
\pgfpathrectangle{\pgfqpoint{1.254980in}{0.150000in}}{\pgfqpoint{5.490039in}{5.490039in}}%
\pgfusepath{clip}%
\pgfsetbuttcap%
\pgfsetroundjoin%
\definecolor{currentfill}{rgb}{0.282884,0.135920,0.453427}%
\pgfsetfillcolor{currentfill}%
\pgfsetfillopacity{0.700000}%
\pgfsetlinewidth{0.000000pt}%
\definecolor{currentstroke}{rgb}{0.000000,0.000000,0.000000}%
\pgfsetstrokecolor{currentstroke}%
\pgfsetdash{}{0pt}%
\pgfpathmoveto{\pgfqpoint{4.664807in}{1.267334in}}%
\pgfpathlineto{\pgfqpoint{4.678776in}{1.261061in}}%
\pgfpathlineto{\pgfqpoint{4.692750in}{1.254810in}}%
\pgfpathlineto{\pgfqpoint{4.706731in}{1.248581in}}%
\pgfpathlineto{\pgfqpoint{4.720717in}{1.242374in}}%
\pgfpathlineto{\pgfqpoint{4.713099in}{1.244577in}}%
\pgfpathlineto{\pgfqpoint{4.705478in}{1.247264in}}%
\pgfpathlineto{\pgfqpoint{4.697853in}{1.250445in}}%
\pgfpathlineto{\pgfqpoint{4.690226in}{1.254132in}}%
\pgfpathlineto{\pgfqpoint{4.676216in}{1.260739in}}%
\pgfpathlineto{\pgfqpoint{4.662211in}{1.267368in}}%
\pgfpathlineto{\pgfqpoint{4.648212in}{1.274020in}}%
\pgfpathlineto{\pgfqpoint{4.634219in}{1.280693in}}%
\pgfpathlineto{\pgfqpoint{4.641872in}{1.276600in}}%
\pgfpathlineto{\pgfqpoint{4.649521in}{1.273017in}}%
\pgfpathlineto{\pgfqpoint{4.657165in}{1.269932in}}%
\pgfpathlineto{\pgfqpoint{4.664807in}{1.267334in}}%
\pgfpathclose%
\pgfusepath{fill}%
\end{pgfscope}%
\begin{pgfscope}%
\pgfpathrectangle{\pgfqpoint{1.254980in}{0.150000in}}{\pgfqpoint{5.490039in}{5.490039in}}%
\pgfusepath{clip}%
\pgfsetbuttcap%
\pgfsetroundjoin%
\definecolor{currentfill}{rgb}{0.175707,0.697900,0.491033}%
\pgfsetfillcolor{currentfill}%
\pgfsetfillopacity{0.700000}%
\pgfsetlinewidth{0.000000pt}%
\definecolor{currentstroke}{rgb}{0.000000,0.000000,0.000000}%
\pgfsetstrokecolor{currentstroke}%
\pgfsetdash{}{0pt}%
\pgfpathmoveto{\pgfqpoint{2.788322in}{2.660026in}}%
\pgfpathlineto{\pgfqpoint{2.802056in}{2.648358in}}%
\pgfpathlineto{\pgfqpoint{2.815790in}{2.636724in}}%
\pgfpathlineto{\pgfqpoint{2.829526in}{2.625123in}}%
\pgfpathlineto{\pgfqpoint{2.843264in}{2.613554in}}%
\pgfpathlineto{\pgfqpoint{2.833888in}{2.642769in}}%
\pgfpathlineto{\pgfqpoint{2.824472in}{2.672874in}}%
\pgfpathlineto{\pgfqpoint{2.815013in}{2.703886in}}%
\pgfpathlineto{\pgfqpoint{2.805511in}{2.735823in}}%
\pgfpathlineto{\pgfqpoint{2.791702in}{2.747901in}}%
\pgfpathlineto{\pgfqpoint{2.777895in}{2.760012in}}%
\pgfpathlineto{\pgfqpoint{2.764088in}{2.772157in}}%
\pgfpathlineto{\pgfqpoint{2.750283in}{2.784335in}}%
\pgfpathlineto{\pgfqpoint{2.759858in}{2.751880in}}%
\pgfpathlineto{\pgfqpoint{2.769389in}{2.720355in}}%
\pgfpathlineto{\pgfqpoint{2.778877in}{2.689742in}}%
\pgfpathlineto{\pgfqpoint{2.788322in}{2.660026in}}%
\pgfpathclose%
\pgfusepath{fill}%
\end{pgfscope}%
\begin{pgfscope}%
\pgfpathrectangle{\pgfqpoint{1.254980in}{0.150000in}}{\pgfqpoint{5.490039in}{5.490039in}}%
\pgfusepath{clip}%
\pgfsetbuttcap%
\pgfsetroundjoin%
\definecolor{currentfill}{rgb}{0.210503,0.363727,0.552206}%
\pgfsetfillcolor{currentfill}%
\pgfsetfillopacity{0.700000}%
\pgfsetlinewidth{0.000000pt}%
\definecolor{currentstroke}{rgb}{0.000000,0.000000,0.000000}%
\pgfsetstrokecolor{currentstroke}%
\pgfsetdash{}{0pt}%
\pgfpathmoveto{\pgfqpoint{3.848667in}{1.755633in}}%
\pgfpathlineto{\pgfqpoint{3.862484in}{1.746912in}}%
\pgfpathlineto{\pgfqpoint{3.876305in}{1.738214in}}%
\pgfpathlineto{\pgfqpoint{3.890130in}{1.729540in}}%
\pgfpathlineto{\pgfqpoint{3.903958in}{1.720891in}}%
\pgfpathlineto{\pgfqpoint{3.895809in}{1.735420in}}%
\pgfpathlineto{\pgfqpoint{3.887643in}{1.750631in}}%
\pgfpathlineto{\pgfqpoint{3.879460in}{1.766537in}}%
\pgfpathlineto{\pgfqpoint{3.871261in}{1.783153in}}%
\pgfpathlineto{\pgfqpoint{3.857387in}{1.792255in}}%
\pgfpathlineto{\pgfqpoint{3.843517in}{1.801381in}}%
\pgfpathlineto{\pgfqpoint{3.829651in}{1.810532in}}%
\pgfpathlineto{\pgfqpoint{3.815789in}{1.819706in}}%
\pgfpathlineto{\pgfqpoint{3.824035in}{1.802631in}}%
\pgfpathlineto{\pgfqpoint{3.832263in}{1.786269in}}%
\pgfpathlineto{\pgfqpoint{3.840474in}{1.770608in}}%
\pgfpathlineto{\pgfqpoint{3.848667in}{1.755633in}}%
\pgfpathclose%
\pgfusepath{fill}%
\end{pgfscope}%
\begin{pgfscope}%
\pgfpathrectangle{\pgfqpoint{1.254980in}{0.150000in}}{\pgfqpoint{5.490039in}{5.490039in}}%
\pgfusepath{clip}%
\pgfsetbuttcap%
\pgfsetroundjoin%
\definecolor{currentfill}{rgb}{0.156270,0.489624,0.557936}%
\pgfsetfillcolor{currentfill}%
\pgfsetfillopacity{0.700000}%
\pgfsetlinewidth{0.000000pt}%
\definecolor{currentstroke}{rgb}{0.000000,0.000000,0.000000}%
\pgfsetstrokecolor{currentstroke}%
\pgfsetdash{}{0pt}%
\pgfpathmoveto{\pgfqpoint{3.429031in}{2.086766in}}%
\pgfpathlineto{\pgfqpoint{3.442800in}{2.076880in}}%
\pgfpathlineto{\pgfqpoint{3.456573in}{2.067021in}}%
\pgfpathlineto{\pgfqpoint{3.470349in}{2.057189in}}%
\pgfpathlineto{\pgfqpoint{3.484127in}{2.047382in}}%
\pgfpathlineto{\pgfqpoint{3.475557in}{2.067985in}}%
\pgfpathlineto{\pgfqpoint{3.466961in}{2.089358in}}%
\pgfpathlineto{\pgfqpoint{3.458340in}{2.111517in}}%
\pgfpathlineto{\pgfqpoint{3.449692in}{2.134477in}}%
\pgfpathlineto{\pgfqpoint{3.435857in}{2.144760in}}%
\pgfpathlineto{\pgfqpoint{3.422026in}{2.155069in}}%
\pgfpathlineto{\pgfqpoint{3.408197in}{2.165404in}}%
\pgfpathlineto{\pgfqpoint{3.394371in}{2.175767in}}%
\pgfpathlineto{\pgfqpoint{3.403077in}{2.152323in}}%
\pgfpathlineto{\pgfqpoint{3.411755in}{2.129685in}}%
\pgfpathlineto{\pgfqpoint{3.420406in}{2.107837in}}%
\pgfpathlineto{\pgfqpoint{3.429031in}{2.086766in}}%
\pgfpathclose%
\pgfusepath{fill}%
\end{pgfscope}%
\begin{pgfscope}%
\pgfpathrectangle{\pgfqpoint{1.254980in}{0.150000in}}{\pgfqpoint{5.490039in}{5.490039in}}%
\pgfusepath{clip}%
\pgfsetbuttcap%
\pgfsetroundjoin%
\definecolor{currentfill}{rgb}{0.276194,0.190074,0.493001}%
\pgfsetfillcolor{currentfill}%
\pgfsetfillopacity{0.700000}%
\pgfsetlinewidth{0.000000pt}%
\definecolor{currentstroke}{rgb}{0.000000,0.000000,0.000000}%
\pgfsetstrokecolor{currentstroke}%
\pgfsetdash{}{0pt}%
\pgfpathmoveto{\pgfqpoint{4.466724in}{1.362515in}}%
\pgfpathlineto{\pgfqpoint{4.480653in}{1.355574in}}%
\pgfpathlineto{\pgfqpoint{4.494586in}{1.348654in}}%
\pgfpathlineto{\pgfqpoint{4.508525in}{1.341758in}}%
\pgfpathlineto{\pgfqpoint{4.522470in}{1.334883in}}%
\pgfpathlineto{\pgfqpoint{4.514759in}{1.340316in}}%
\pgfpathlineto{\pgfqpoint{4.507042in}{1.346289in}}%
\pgfpathlineto{\pgfqpoint{4.499319in}{1.352814in}}%
\pgfpathlineto{\pgfqpoint{4.491590in}{1.359903in}}%
\pgfpathlineto{\pgfqpoint{4.477617in}{1.367193in}}%
\pgfpathlineto{\pgfqpoint{4.463648in}{1.374506in}}%
\pgfpathlineto{\pgfqpoint{4.449685in}{1.381841in}}%
\pgfpathlineto{\pgfqpoint{4.435727in}{1.389198in}}%
\pgfpathlineto{\pgfqpoint{4.443486in}{1.381688in}}%
\pgfpathlineto{\pgfqpoint{4.451239in}{1.374745in}}%
\pgfpathlineto{\pgfqpoint{4.458985in}{1.368358in}}%
\pgfpathlineto{\pgfqpoint{4.466724in}{1.362515in}}%
\pgfpathclose%
\pgfusepath{fill}%
\end{pgfscope}%
\begin{pgfscope}%
\pgfpathrectangle{\pgfqpoint{1.254980in}{0.150000in}}{\pgfqpoint{5.490039in}{5.490039in}}%
\pgfusepath{clip}%
\pgfsetbuttcap%
\pgfsetroundjoin%
\definecolor{currentfill}{rgb}{0.157851,0.683765,0.501686}%
\pgfsetfillcolor{currentfill}%
\pgfsetfillopacity{0.700000}%
\pgfsetlinewidth{0.000000pt}%
\definecolor{currentstroke}{rgb}{0.000000,0.000000,0.000000}%
\pgfsetstrokecolor{currentstroke}%
\pgfsetdash{}{0pt}%
\pgfpathmoveto{\pgfqpoint{2.843264in}{2.613554in}}%
\pgfpathlineto{\pgfqpoint{2.857003in}{2.602019in}}%
\pgfpathlineto{\pgfqpoint{2.870744in}{2.590515in}}%
\pgfpathlineto{\pgfqpoint{2.884487in}{2.579044in}}%
\pgfpathlineto{\pgfqpoint{2.898231in}{2.567605in}}%
\pgfpathlineto{\pgfqpoint{2.888924in}{2.596320in}}%
\pgfpathlineto{\pgfqpoint{2.879577in}{2.625919in}}%
\pgfpathlineto{\pgfqpoint{2.870190in}{2.656420in}}%
\pgfpathlineto{\pgfqpoint{2.860761in}{2.687839in}}%
\pgfpathlineto{\pgfqpoint{2.846946in}{2.699786in}}%
\pgfpathlineto{\pgfqpoint{2.833133in}{2.711766in}}%
\pgfpathlineto{\pgfqpoint{2.819322in}{2.723778in}}%
\pgfpathlineto{\pgfqpoint{2.805511in}{2.735823in}}%
\pgfpathlineto{\pgfqpoint{2.815013in}{2.703886in}}%
\pgfpathlineto{\pgfqpoint{2.824472in}{2.672874in}}%
\pgfpathlineto{\pgfqpoint{2.833888in}{2.642769in}}%
\pgfpathlineto{\pgfqpoint{2.843264in}{2.613554in}}%
\pgfpathclose%
\pgfusepath{fill}%
\end{pgfscope}%
\begin{pgfscope}%
\pgfpathrectangle{\pgfqpoint{1.254980in}{0.150000in}}{\pgfqpoint{5.490039in}{5.490039in}}%
\pgfusepath{clip}%
\pgfsetbuttcap%
\pgfsetroundjoin%
\definecolor{currentfill}{rgb}{0.255645,0.260703,0.528312}%
\pgfsetfillcolor{currentfill}%
\pgfsetfillopacity{0.700000}%
\pgfsetlinewidth{0.000000pt}%
\definecolor{currentstroke}{rgb}{0.000000,0.000000,0.000000}%
\pgfsetstrokecolor{currentstroke}%
\pgfsetdash{}{0pt}%
\pgfpathmoveto{\pgfqpoint{4.213064in}{1.510003in}}%
\pgfpathlineto{\pgfqpoint{4.226944in}{1.502281in}}%
\pgfpathlineto{\pgfqpoint{4.240829in}{1.494583in}}%
\pgfpathlineto{\pgfqpoint{4.254719in}{1.486907in}}%
\pgfpathlineto{\pgfqpoint{4.268614in}{1.479254in}}%
\pgfpathlineto{\pgfqpoint{4.260748in}{1.488623in}}%
\pgfpathlineto{\pgfqpoint{4.252872in}{1.498596in}}%
\pgfpathlineto{\pgfqpoint{4.244987in}{1.509186in}}%
\pgfpathlineto{\pgfqpoint{4.237091in}{1.520406in}}%
\pgfpathlineto{\pgfqpoint{4.223161in}{1.528492in}}%
\pgfpathlineto{\pgfqpoint{4.209235in}{1.536600in}}%
\pgfpathlineto{\pgfqpoint{4.195314in}{1.544732in}}%
\pgfpathlineto{\pgfqpoint{4.181397in}{1.552886in}}%
\pgfpathlineto{\pgfqpoint{4.189330in}{1.541227in}}%
\pgfpathlineto{\pgfqpoint{4.197252in}{1.530202in}}%
\pgfpathlineto{\pgfqpoint{4.205163in}{1.519798in}}%
\pgfpathlineto{\pgfqpoint{4.213064in}{1.510003in}}%
\pgfpathclose%
\pgfusepath{fill}%
\end{pgfscope}%
\begin{pgfscope}%
\pgfpathrectangle{\pgfqpoint{1.254980in}{0.150000in}}{\pgfqpoint{5.490039in}{5.490039in}}%
\pgfusepath{clip}%
\pgfsetbuttcap%
\pgfsetroundjoin%
\definecolor{currentfill}{rgb}{0.160665,0.478540,0.558115}%
\pgfsetfillcolor{currentfill}%
\pgfsetfillopacity{0.700000}%
\pgfsetlinewidth{0.000000pt}%
\definecolor{currentstroke}{rgb}{0.000000,0.000000,0.000000}%
\pgfsetstrokecolor{currentstroke}%
\pgfsetdash{}{0pt}%
\pgfpathmoveto{\pgfqpoint{3.484127in}{2.047382in}}%
\pgfpathlineto{\pgfqpoint{3.497909in}{2.037602in}}%
\pgfpathlineto{\pgfqpoint{3.511694in}{2.027848in}}%
\pgfpathlineto{\pgfqpoint{3.525482in}{2.018120in}}%
\pgfpathlineto{\pgfqpoint{3.539273in}{2.008418in}}%
\pgfpathlineto{\pgfqpoint{3.530756in}{2.028552in}}%
\pgfpathlineto{\pgfqpoint{3.522214in}{2.049452in}}%
\pgfpathlineto{\pgfqpoint{3.513648in}{2.071134in}}%
\pgfpathlineto{\pgfqpoint{3.505057in}{2.093611in}}%
\pgfpathlineto{\pgfqpoint{3.491211in}{2.103788in}}%
\pgfpathlineto{\pgfqpoint{3.477368in}{2.113992in}}%
\pgfpathlineto{\pgfqpoint{3.463529in}{2.124221in}}%
\pgfpathlineto{\pgfqpoint{3.449692in}{2.134477in}}%
\pgfpathlineto{\pgfqpoint{3.458340in}{2.111517in}}%
\pgfpathlineto{\pgfqpoint{3.466961in}{2.089358in}}%
\pgfpathlineto{\pgfqpoint{3.475557in}{2.067985in}}%
\pgfpathlineto{\pgfqpoint{3.484127in}{2.047382in}}%
\pgfpathclose%
\pgfusepath{fill}%
\end{pgfscope}%
\begin{pgfscope}%
\pgfpathrectangle{\pgfqpoint{1.254980in}{0.150000in}}{\pgfqpoint{5.490039in}{5.490039in}}%
\pgfusepath{clip}%
\pgfsetbuttcap%
\pgfsetroundjoin%
\definecolor{currentfill}{rgb}{0.216210,0.351535,0.550627}%
\pgfsetfillcolor{currentfill}%
\pgfsetfillopacity{0.700000}%
\pgfsetlinewidth{0.000000pt}%
\definecolor{currentstroke}{rgb}{0.000000,0.000000,0.000000}%
\pgfsetstrokecolor{currentstroke}%
\pgfsetdash{}{0pt}%
\pgfpathmoveto{\pgfqpoint{3.903958in}{1.720891in}}%
\pgfpathlineto{\pgfqpoint{3.917791in}{1.712266in}}%
\pgfpathlineto{\pgfqpoint{3.931628in}{1.703664in}}%
\pgfpathlineto{\pgfqpoint{3.945468in}{1.695087in}}%
\pgfpathlineto{\pgfqpoint{3.959313in}{1.686533in}}%
\pgfpathlineto{\pgfqpoint{3.951206in}{1.700617in}}%
\pgfpathlineto{\pgfqpoint{3.943084in}{1.715378in}}%
\pgfpathlineto{\pgfqpoint{3.934946in}{1.730830in}}%
\pgfpathlineto{\pgfqpoint{3.926792in}{1.746987in}}%
\pgfpathlineto{\pgfqpoint{3.912904in}{1.755993in}}%
\pgfpathlineto{\pgfqpoint{3.899019in}{1.765022in}}%
\pgfpathlineto{\pgfqpoint{3.885138in}{1.774076in}}%
\pgfpathlineto{\pgfqpoint{3.871261in}{1.783153in}}%
\pgfpathlineto{\pgfqpoint{3.879460in}{1.766537in}}%
\pgfpathlineto{\pgfqpoint{3.887643in}{1.750631in}}%
\pgfpathlineto{\pgfqpoint{3.895809in}{1.735420in}}%
\pgfpathlineto{\pgfqpoint{3.903958in}{1.720891in}}%
\pgfpathclose%
\pgfusepath{fill}%
\end{pgfscope}%
\begin{pgfscope}%
\pgfpathrectangle{\pgfqpoint{1.254980in}{0.150000in}}{\pgfqpoint{5.490039in}{5.490039in}}%
\pgfusepath{clip}%
\pgfsetbuttcap%
\pgfsetroundjoin%
\definecolor{currentfill}{rgb}{0.283072,0.130895,0.449241}%
\pgfsetfillcolor{currentfill}%
\pgfsetfillopacity{0.700000}%
\pgfsetlinewidth{0.000000pt}%
\definecolor{currentstroke}{rgb}{0.000000,0.000000,0.000000}%
\pgfsetstrokecolor{currentstroke}%
\pgfsetdash{}{0pt}%
\pgfpathmoveto{\pgfqpoint{4.720717in}{1.242374in}}%
\pgfpathlineto{\pgfqpoint{4.734709in}{1.236189in}}%
\pgfpathlineto{\pgfqpoint{4.748707in}{1.230027in}}%
\pgfpathlineto{\pgfqpoint{4.762711in}{1.223887in}}%
\pgfpathlineto{\pgfqpoint{4.776720in}{1.217768in}}%
\pgfpathlineto{\pgfqpoint{4.769125in}{1.219577in}}%
\pgfpathlineto{\pgfqpoint{4.761527in}{1.221866in}}%
\pgfpathlineto{\pgfqpoint{4.753926in}{1.224645in}}%
\pgfpathlineto{\pgfqpoint{4.746323in}{1.227926in}}%
\pgfpathlineto{\pgfqpoint{4.732290in}{1.234444in}}%
\pgfpathlineto{\pgfqpoint{4.718263in}{1.240985in}}%
\pgfpathlineto{\pgfqpoint{4.704242in}{1.247547in}}%
\pgfpathlineto{\pgfqpoint{4.690226in}{1.254132in}}%
\pgfpathlineto{\pgfqpoint{4.697853in}{1.250445in}}%
\pgfpathlineto{\pgfqpoint{4.705478in}{1.247264in}}%
\pgfpathlineto{\pgfqpoint{4.713099in}{1.244577in}}%
\pgfpathlineto{\pgfqpoint{4.720717in}{1.242374in}}%
\pgfpathclose%
\pgfusepath{fill}%
\end{pgfscope}%
\begin{pgfscope}%
\pgfpathrectangle{\pgfqpoint{1.254980in}{0.150000in}}{\pgfqpoint{5.490039in}{5.490039in}}%
\pgfusepath{clip}%
\pgfsetbuttcap%
\pgfsetroundjoin%
\definecolor{currentfill}{rgb}{0.143303,0.669459,0.511215}%
\pgfsetfillcolor{currentfill}%
\pgfsetfillopacity{0.700000}%
\pgfsetlinewidth{0.000000pt}%
\definecolor{currentstroke}{rgb}{0.000000,0.000000,0.000000}%
\pgfsetstrokecolor{currentstroke}%
\pgfsetdash{}{0pt}%
\pgfpathmoveto{\pgfqpoint{2.898231in}{2.567605in}}%
\pgfpathlineto{\pgfqpoint{2.911976in}{2.556198in}}%
\pgfpathlineto{\pgfqpoint{2.925724in}{2.544823in}}%
\pgfpathlineto{\pgfqpoint{2.939473in}{2.533479in}}%
\pgfpathlineto{\pgfqpoint{2.953224in}{2.522166in}}%
\pgfpathlineto{\pgfqpoint{2.943985in}{2.550381in}}%
\pgfpathlineto{\pgfqpoint{2.934707in}{2.579476in}}%
\pgfpathlineto{\pgfqpoint{2.925390in}{2.609467in}}%
\pgfpathlineto{\pgfqpoint{2.916032in}{2.640370in}}%
\pgfpathlineto{\pgfqpoint{2.902212in}{2.652190in}}%
\pgfpathlineto{\pgfqpoint{2.888394in}{2.664041in}}%
\pgfpathlineto{\pgfqpoint{2.874576in}{2.675924in}}%
\pgfpathlineto{\pgfqpoint{2.860761in}{2.687839in}}%
\pgfpathlineto{\pgfqpoint{2.870190in}{2.656420in}}%
\pgfpathlineto{\pgfqpoint{2.879577in}{2.625919in}}%
\pgfpathlineto{\pgfqpoint{2.888924in}{2.596320in}}%
\pgfpathlineto{\pgfqpoint{2.898231in}{2.567605in}}%
\pgfpathclose%
\pgfusepath{fill}%
\end{pgfscope}%
\begin{pgfscope}%
\pgfpathrectangle{\pgfqpoint{1.254980in}{0.150000in}}{\pgfqpoint{5.490039in}{5.490039in}}%
\pgfusepath{clip}%
\pgfsetbuttcap%
\pgfsetroundjoin%
\definecolor{currentfill}{rgb}{0.278012,0.180367,0.486697}%
\pgfsetfillcolor{currentfill}%
\pgfsetfillopacity{0.700000}%
\pgfsetlinewidth{0.000000pt}%
\definecolor{currentstroke}{rgb}{0.000000,0.000000,0.000000}%
\pgfsetstrokecolor{currentstroke}%
\pgfsetdash{}{0pt}%
\pgfpathmoveto{\pgfqpoint{4.522470in}{1.334883in}}%
\pgfpathlineto{\pgfqpoint{4.536419in}{1.328031in}}%
\pgfpathlineto{\pgfqpoint{4.550375in}{1.321202in}}%
\pgfpathlineto{\pgfqpoint{4.564335in}{1.314395in}}%
\pgfpathlineto{\pgfqpoint{4.578301in}{1.307610in}}%
\pgfpathlineto{\pgfqpoint{4.570617in}{1.312633in}}%
\pgfpathlineto{\pgfqpoint{4.562929in}{1.318192in}}%
\pgfpathlineto{\pgfqpoint{4.555235in}{1.324300in}}%
\pgfpathlineto{\pgfqpoint{4.547536in}{1.330967in}}%
\pgfpathlineto{\pgfqpoint{4.533542in}{1.338168in}}%
\pgfpathlineto{\pgfqpoint{4.519553in}{1.345390in}}%
\pgfpathlineto{\pgfqpoint{4.505569in}{1.352636in}}%
\pgfpathlineto{\pgfqpoint{4.491590in}{1.359903in}}%
\pgfpathlineto{\pgfqpoint{4.499319in}{1.352814in}}%
\pgfpathlineto{\pgfqpoint{4.507042in}{1.346289in}}%
\pgfpathlineto{\pgfqpoint{4.514759in}{1.340316in}}%
\pgfpathlineto{\pgfqpoint{4.522470in}{1.334883in}}%
\pgfpathclose%
\pgfusepath{fill}%
\end{pgfscope}%
\begin{pgfscope}%
\pgfpathrectangle{\pgfqpoint{1.254980in}{0.150000in}}{\pgfqpoint{5.490039in}{5.490039in}}%
\pgfusepath{clip}%
\pgfsetbuttcap%
\pgfsetroundjoin%
\definecolor{currentfill}{rgb}{0.132268,0.655014,0.519661}%
\pgfsetfillcolor{currentfill}%
\pgfsetfillopacity{0.700000}%
\pgfsetlinewidth{0.000000pt}%
\definecolor{currentstroke}{rgb}{0.000000,0.000000,0.000000}%
\pgfsetstrokecolor{currentstroke}%
\pgfsetdash{}{0pt}%
\pgfpathmoveto{\pgfqpoint{2.953224in}{2.522166in}}%
\pgfpathlineto{\pgfqpoint{2.966977in}{2.510885in}}%
\pgfpathlineto{\pgfqpoint{2.980732in}{2.499634in}}%
\pgfpathlineto{\pgfqpoint{2.994488in}{2.488414in}}%
\pgfpathlineto{\pgfqpoint{3.008247in}{2.477225in}}%
\pgfpathlineto{\pgfqpoint{2.999074in}{2.504943in}}%
\pgfpathlineto{\pgfqpoint{2.989864in}{2.533534in}}%
\pgfpathlineto{\pgfqpoint{2.980616in}{2.563016in}}%
\pgfpathlineto{\pgfqpoint{2.971329in}{2.593406in}}%
\pgfpathlineto{\pgfqpoint{2.957503in}{2.605100in}}%
\pgfpathlineto{\pgfqpoint{2.943678in}{2.616826in}}%
\pgfpathlineto{\pgfqpoint{2.929854in}{2.628582in}}%
\pgfpathlineto{\pgfqpoint{2.916032in}{2.640370in}}%
\pgfpathlineto{\pgfqpoint{2.925390in}{2.609467in}}%
\pgfpathlineto{\pgfqpoint{2.934707in}{2.579476in}}%
\pgfpathlineto{\pgfqpoint{2.943985in}{2.550381in}}%
\pgfpathlineto{\pgfqpoint{2.953224in}{2.522166in}}%
\pgfpathclose%
\pgfusepath{fill}%
\end{pgfscope}%
\begin{pgfscope}%
\pgfpathrectangle{\pgfqpoint{1.254980in}{0.150000in}}{\pgfqpoint{5.490039in}{5.490039in}}%
\pgfusepath{clip}%
\pgfsetbuttcap%
\pgfsetroundjoin%
\definecolor{currentfill}{rgb}{0.165117,0.467423,0.558141}%
\pgfsetfillcolor{currentfill}%
\pgfsetfillopacity{0.700000}%
\pgfsetlinewidth{0.000000pt}%
\definecolor{currentstroke}{rgb}{0.000000,0.000000,0.000000}%
\pgfsetstrokecolor{currentstroke}%
\pgfsetdash{}{0pt}%
\pgfpathmoveto{\pgfqpoint{3.539273in}{2.008418in}}%
\pgfpathlineto{\pgfqpoint{3.553067in}{1.998742in}}%
\pgfpathlineto{\pgfqpoint{3.566864in}{1.989092in}}%
\pgfpathlineto{\pgfqpoint{3.580665in}{1.979467in}}%
\pgfpathlineto{\pgfqpoint{3.594469in}{1.969868in}}%
\pgfpathlineto{\pgfqpoint{3.586004in}{1.989535in}}%
\pgfpathlineto{\pgfqpoint{3.577516in}{2.009962in}}%
\pgfpathlineto{\pgfqpoint{3.569004in}{2.031167in}}%
\pgfpathlineto{\pgfqpoint{3.560469in}{2.053162in}}%
\pgfpathlineto{\pgfqpoint{3.546611in}{2.063235in}}%
\pgfpathlineto{\pgfqpoint{3.532757in}{2.073335in}}%
\pgfpathlineto{\pgfqpoint{3.518905in}{2.083460in}}%
\pgfpathlineto{\pgfqpoint{3.505057in}{2.093611in}}%
\pgfpathlineto{\pgfqpoint{3.513648in}{2.071134in}}%
\pgfpathlineto{\pgfqpoint{3.522214in}{2.049452in}}%
\pgfpathlineto{\pgfqpoint{3.530756in}{2.028552in}}%
\pgfpathlineto{\pgfqpoint{3.539273in}{2.008418in}}%
\pgfpathclose%
\pgfusepath{fill}%
\end{pgfscope}%
\begin{pgfscope}%
\pgfpathrectangle{\pgfqpoint{1.254980in}{0.150000in}}{\pgfqpoint{5.490039in}{5.490039in}}%
\pgfusepath{clip}%
\pgfsetbuttcap%
\pgfsetroundjoin%
\definecolor{currentfill}{rgb}{0.258965,0.251537,0.524736}%
\pgfsetfillcolor{currentfill}%
\pgfsetfillopacity{0.700000}%
\pgfsetlinewidth{0.000000pt}%
\definecolor{currentstroke}{rgb}{0.000000,0.000000,0.000000}%
\pgfsetstrokecolor{currentstroke}%
\pgfsetdash{}{0pt}%
\pgfpathmoveto{\pgfqpoint{4.268614in}{1.479254in}}%
\pgfpathlineto{\pgfqpoint{4.282513in}{1.471624in}}%
\pgfpathlineto{\pgfqpoint{4.296417in}{1.464017in}}%
\pgfpathlineto{\pgfqpoint{4.310326in}{1.456433in}}%
\pgfpathlineto{\pgfqpoint{4.324240in}{1.448872in}}%
\pgfpathlineto{\pgfqpoint{4.316408in}{1.457814in}}%
\pgfpathlineto{\pgfqpoint{4.308568in}{1.467357in}}%
\pgfpathlineto{\pgfqpoint{4.300718in}{1.477513in}}%
\pgfpathlineto{\pgfqpoint{4.292859in}{1.488294in}}%
\pgfpathlineto{\pgfqpoint{4.278910in}{1.496288in}}%
\pgfpathlineto{\pgfqpoint{4.264966in}{1.504304in}}%
\pgfpathlineto{\pgfqpoint{4.251026in}{1.512344in}}%
\pgfpathlineto{\pgfqpoint{4.237091in}{1.520406in}}%
\pgfpathlineto{\pgfqpoint{4.244987in}{1.509186in}}%
\pgfpathlineto{\pgfqpoint{4.252872in}{1.498596in}}%
\pgfpathlineto{\pgfqpoint{4.260748in}{1.488623in}}%
\pgfpathlineto{\pgfqpoint{4.268614in}{1.479254in}}%
\pgfpathclose%
\pgfusepath{fill}%
\end{pgfscope}%
\begin{pgfscope}%
\pgfpathrectangle{\pgfqpoint{1.254980in}{0.150000in}}{\pgfqpoint{5.490039in}{5.490039in}}%
\pgfusepath{clip}%
\pgfsetbuttcap%
\pgfsetroundjoin%
\definecolor{currentfill}{rgb}{0.993248,0.906157,0.143936}%
\pgfsetfillcolor{currentfill}%
\pgfsetfillopacity{0.700000}%
\pgfsetlinewidth{0.000000pt}%
\definecolor{currentstroke}{rgb}{0.000000,0.000000,0.000000}%
\pgfsetstrokecolor{currentstroke}%
\pgfsetdash{}{0pt}%
\pgfpathmoveto{\pgfqpoint{1.922638in}{3.587488in}}%
\pgfpathlineto{\pgfqpoint{1.936450in}{3.572711in}}%
\pgfpathlineto{\pgfqpoint{1.950259in}{3.557990in}}%
\pgfpathlineto{\pgfqpoint{1.964067in}{3.543326in}}%
\pgfpathlineto{\pgfqpoint{1.977874in}{3.528717in}}%
\pgfpathlineto{\pgfqpoint{1.967087in}{3.569663in}}%
\pgfpathlineto{\pgfqpoint{1.956234in}{3.611661in}}%
\pgfpathlineto{\pgfqpoint{1.945312in}{3.654732in}}%
\pgfpathlineto{\pgfqpoint{1.931435in}{3.669763in}}%
\pgfpathlineto{\pgfqpoint{1.917557in}{3.684850in}}%
\pgfpathlineto{\pgfqpoint{1.903677in}{3.699993in}}%
\pgfpathlineto{\pgfqpoint{1.889796in}{3.715194in}}%
\pgfpathlineto{\pgfqpoint{1.900813in}{3.671553in}}%
\pgfpathlineto{\pgfqpoint{1.911760in}{3.628991in}}%
\pgfpathlineto{\pgfqpoint{1.922638in}{3.587488in}}%
\pgfpathclose%
\pgfusepath{fill}%
\end{pgfscope}%
\begin{pgfscope}%
\pgfpathrectangle{\pgfqpoint{1.254980in}{0.150000in}}{\pgfqpoint{5.490039in}{5.490039in}}%
\pgfusepath{clip}%
\pgfsetbuttcap%
\pgfsetroundjoin%
\definecolor{currentfill}{rgb}{0.220057,0.343307,0.549413}%
\pgfsetfillcolor{currentfill}%
\pgfsetfillopacity{0.700000}%
\pgfsetlinewidth{0.000000pt}%
\definecolor{currentstroke}{rgb}{0.000000,0.000000,0.000000}%
\pgfsetstrokecolor{currentstroke}%
\pgfsetdash{}{0pt}%
\pgfpathmoveto{\pgfqpoint{3.959313in}{1.686533in}}%
\pgfpathlineto{\pgfqpoint{3.973162in}{1.678004in}}%
\pgfpathlineto{\pgfqpoint{3.987014in}{1.669498in}}%
\pgfpathlineto{\pgfqpoint{4.000871in}{1.661016in}}%
\pgfpathlineto{\pgfqpoint{4.014732in}{1.652557in}}%
\pgfpathlineto{\pgfqpoint{4.006668in}{1.666195in}}%
\pgfpathlineto{\pgfqpoint{3.998588in}{1.680507in}}%
\pgfpathlineto{\pgfqpoint{3.990494in}{1.695506in}}%
\pgfpathlineto{\pgfqpoint{3.982385in}{1.711205in}}%
\pgfpathlineto{\pgfqpoint{3.968481in}{1.720115in}}%
\pgfpathlineto{\pgfqpoint{3.954581in}{1.729048in}}%
\pgfpathlineto{\pgfqpoint{3.940685in}{1.738006in}}%
\pgfpathlineto{\pgfqpoint{3.926792in}{1.746987in}}%
\pgfpathlineto{\pgfqpoint{3.934946in}{1.730830in}}%
\pgfpathlineto{\pgfqpoint{3.943084in}{1.715378in}}%
\pgfpathlineto{\pgfqpoint{3.951206in}{1.700617in}}%
\pgfpathlineto{\pgfqpoint{3.959313in}{1.686533in}}%
\pgfpathclose%
\pgfusepath{fill}%
\end{pgfscope}%
\begin{pgfscope}%
\pgfpathrectangle{\pgfqpoint{1.254980in}{0.150000in}}{\pgfqpoint{5.490039in}{5.490039in}}%
\pgfusepath{clip}%
\pgfsetbuttcap%
\pgfsetroundjoin%
\definecolor{currentfill}{rgb}{0.945636,0.899815,0.112838}%
\pgfsetfillcolor{currentfill}%
\pgfsetfillopacity{0.700000}%
\pgfsetlinewidth{0.000000pt}%
\definecolor{currentstroke}{rgb}{0.000000,0.000000,0.000000}%
\pgfsetstrokecolor{currentstroke}%
\pgfsetdash{}{0pt}%
\pgfpathmoveto{\pgfqpoint{1.977874in}{3.528717in}}%
\pgfpathlineto{\pgfqpoint{1.991679in}{3.514162in}}%
\pgfpathlineto{\pgfqpoint{2.005482in}{3.499662in}}%
\pgfpathlineto{\pgfqpoint{2.019284in}{3.485215in}}%
\pgfpathlineto{\pgfqpoint{2.033085in}{3.470821in}}%
\pgfpathlineto{\pgfqpoint{2.022389in}{3.511214in}}%
\pgfpathlineto{\pgfqpoint{2.011628in}{3.552652in}}%
\pgfpathlineto{\pgfqpoint{2.000800in}{3.595155in}}%
\pgfpathlineto{\pgfqpoint{1.986930in}{3.609968in}}%
\pgfpathlineto{\pgfqpoint{1.973059in}{3.624835in}}%
\pgfpathlineto{\pgfqpoint{1.959186in}{3.639756in}}%
\pgfpathlineto{\pgfqpoint{1.945312in}{3.654732in}}%
\pgfpathlineto{\pgfqpoint{1.956234in}{3.611661in}}%
\pgfpathlineto{\pgfqpoint{1.967087in}{3.569663in}}%
\pgfpathlineto{\pgfqpoint{1.977874in}{3.528717in}}%
\pgfpathclose%
\pgfusepath{fill}%
\end{pgfscope}%
\begin{pgfscope}%
\pgfpathrectangle{\pgfqpoint{1.254980in}{0.150000in}}{\pgfqpoint{5.490039in}{5.490039in}}%
\pgfusepath{clip}%
\pgfsetbuttcap%
\pgfsetroundjoin%
\definecolor{currentfill}{rgb}{0.886271,0.892374,0.095374}%
\pgfsetfillcolor{currentfill}%
\pgfsetfillopacity{0.700000}%
\pgfsetlinewidth{0.000000pt}%
\definecolor{currentstroke}{rgb}{0.000000,0.000000,0.000000}%
\pgfsetstrokecolor{currentstroke}%
\pgfsetdash{}{0pt}%
\pgfpathmoveto{\pgfqpoint{2.033085in}{3.470821in}}%
\pgfpathlineto{\pgfqpoint{2.046885in}{3.456480in}}%
\pgfpathlineto{\pgfqpoint{2.060683in}{3.442191in}}%
\pgfpathlineto{\pgfqpoint{2.074481in}{3.427953in}}%
\pgfpathlineto{\pgfqpoint{2.088277in}{3.413766in}}%
\pgfpathlineto{\pgfqpoint{2.077671in}{3.453607in}}%
\pgfpathlineto{\pgfqpoint{2.067001in}{3.494488in}}%
\pgfpathlineto{\pgfqpoint{2.056266in}{3.536427in}}%
\pgfpathlineto{\pgfqpoint{2.042401in}{3.551031in}}%
\pgfpathlineto{\pgfqpoint{2.028536in}{3.565687in}}%
\pgfpathlineto{\pgfqpoint{2.014669in}{3.580395in}}%
\pgfpathlineto{\pgfqpoint{2.000800in}{3.595155in}}%
\pgfpathlineto{\pgfqpoint{2.011628in}{3.552652in}}%
\pgfpathlineto{\pgfqpoint{2.022389in}{3.511214in}}%
\pgfpathlineto{\pgfqpoint{2.033085in}{3.470821in}}%
\pgfpathclose%
\pgfusepath{fill}%
\end{pgfscope}%
\begin{pgfscope}%
\pgfpathrectangle{\pgfqpoint{1.254980in}{0.150000in}}{\pgfqpoint{5.490039in}{5.490039in}}%
\pgfusepath{clip}%
\pgfsetbuttcap%
\pgfsetroundjoin%
\definecolor{currentfill}{rgb}{0.124780,0.640461,0.527068}%
\pgfsetfillcolor{currentfill}%
\pgfsetfillopacity{0.700000}%
\pgfsetlinewidth{0.000000pt}%
\definecolor{currentstroke}{rgb}{0.000000,0.000000,0.000000}%
\pgfsetstrokecolor{currentstroke}%
\pgfsetdash{}{0pt}%
\pgfpathmoveto{\pgfqpoint{3.008247in}{2.477225in}}%
\pgfpathlineto{\pgfqpoint{3.022007in}{2.466067in}}%
\pgfpathlineto{\pgfqpoint{3.035769in}{2.454938in}}%
\pgfpathlineto{\pgfqpoint{3.049534in}{2.443840in}}%
\pgfpathlineto{\pgfqpoint{3.063300in}{2.432772in}}%
\pgfpathlineto{\pgfqpoint{3.054193in}{2.459993in}}%
\pgfpathlineto{\pgfqpoint{3.045050in}{2.488082in}}%
\pgfpathlineto{\pgfqpoint{3.035870in}{2.517057in}}%
\pgfpathlineto{\pgfqpoint{3.026653in}{2.546933in}}%
\pgfpathlineto{\pgfqpoint{3.012819in}{2.558506in}}%
\pgfpathlineto{\pgfqpoint{2.998988in}{2.570109in}}%
\pgfpathlineto{\pgfqpoint{2.985158in}{2.581742in}}%
\pgfpathlineto{\pgfqpoint{2.971329in}{2.593406in}}%
\pgfpathlineto{\pgfqpoint{2.980616in}{2.563016in}}%
\pgfpathlineto{\pgfqpoint{2.989864in}{2.533534in}}%
\pgfpathlineto{\pgfqpoint{2.999074in}{2.504943in}}%
\pgfpathlineto{\pgfqpoint{3.008247in}{2.477225in}}%
\pgfpathclose%
\pgfusepath{fill}%
\end{pgfscope}%
\begin{pgfscope}%
\pgfpathrectangle{\pgfqpoint{1.254980in}{0.150000in}}{\pgfqpoint{5.490039in}{5.490039in}}%
\pgfusepath{clip}%
\pgfsetbuttcap%
\pgfsetroundjoin%
\definecolor{currentfill}{rgb}{0.283187,0.125848,0.444960}%
\pgfsetfillcolor{currentfill}%
\pgfsetfillopacity{0.700000}%
\pgfsetlinewidth{0.000000pt}%
\definecolor{currentstroke}{rgb}{0.000000,0.000000,0.000000}%
\pgfsetstrokecolor{currentstroke}%
\pgfsetdash{}{0pt}%
\pgfpathmoveto{\pgfqpoint{4.776720in}{1.217768in}}%
\pgfpathlineto{\pgfqpoint{4.790736in}{1.211672in}}%
\pgfpathlineto{\pgfqpoint{4.804758in}{1.205598in}}%
\pgfpathlineto{\pgfqpoint{4.818785in}{1.199546in}}%
\pgfpathlineto{\pgfqpoint{4.832819in}{1.193516in}}%
\pgfpathlineto{\pgfqpoint{4.825245in}{1.194930in}}%
\pgfpathlineto{\pgfqpoint{4.817669in}{1.196820in}}%
\pgfpathlineto{\pgfqpoint{4.810091in}{1.199197in}}%
\pgfpathlineto{\pgfqpoint{4.802511in}{1.202073in}}%
\pgfpathlineto{\pgfqpoint{4.788456in}{1.208503in}}%
\pgfpathlineto{\pgfqpoint{4.774406in}{1.214955in}}%
\pgfpathlineto{\pgfqpoint{4.760361in}{1.221429in}}%
\pgfpathlineto{\pgfqpoint{4.746323in}{1.227926in}}%
\pgfpathlineto{\pgfqpoint{4.753926in}{1.224645in}}%
\pgfpathlineto{\pgfqpoint{4.761527in}{1.221866in}}%
\pgfpathlineto{\pgfqpoint{4.769125in}{1.219577in}}%
\pgfpathlineto{\pgfqpoint{4.776720in}{1.217768in}}%
\pgfpathclose%
\pgfusepath{fill}%
\end{pgfscope}%
\begin{pgfscope}%
\pgfpathrectangle{\pgfqpoint{1.254980in}{0.150000in}}{\pgfqpoint{5.490039in}{5.490039in}}%
\pgfusepath{clip}%
\pgfsetbuttcap%
\pgfsetroundjoin%
\definecolor{currentfill}{rgb}{0.835270,0.886029,0.102646}%
\pgfsetfillcolor{currentfill}%
\pgfsetfillopacity{0.700000}%
\pgfsetlinewidth{0.000000pt}%
\definecolor{currentstroke}{rgb}{0.000000,0.000000,0.000000}%
\pgfsetstrokecolor{currentstroke}%
\pgfsetdash{}{0pt}%
\pgfpathmoveto{\pgfqpoint{2.088277in}{3.413766in}}%
\pgfpathlineto{\pgfqpoint{2.102073in}{3.399630in}}%
\pgfpathlineto{\pgfqpoint{2.115867in}{3.385543in}}%
\pgfpathlineto{\pgfqpoint{2.129661in}{3.371506in}}%
\pgfpathlineto{\pgfqpoint{2.143453in}{3.357517in}}%
\pgfpathlineto{\pgfqpoint{2.132935in}{3.396810in}}%
\pgfpathlineto{\pgfqpoint{2.122355in}{3.437135in}}%
\pgfpathlineto{\pgfqpoint{2.111712in}{3.478513in}}%
\pgfpathlineto{\pgfqpoint{2.097852in}{3.492917in}}%
\pgfpathlineto{\pgfqpoint{2.083991in}{3.507370in}}%
\pgfpathlineto{\pgfqpoint{2.070129in}{3.521873in}}%
\pgfpathlineto{\pgfqpoint{2.056266in}{3.536427in}}%
\pgfpathlineto{\pgfqpoint{2.067001in}{3.494488in}}%
\pgfpathlineto{\pgfqpoint{2.077671in}{3.453607in}}%
\pgfpathlineto{\pgfqpoint{2.088277in}{3.413766in}}%
\pgfpathclose%
\pgfusepath{fill}%
\end{pgfscope}%
\begin{pgfscope}%
\pgfpathrectangle{\pgfqpoint{1.254980in}{0.150000in}}{\pgfqpoint{5.490039in}{5.490039in}}%
\pgfusepath{clip}%
\pgfsetbuttcap%
\pgfsetroundjoin%
\definecolor{currentfill}{rgb}{0.169646,0.456262,0.558030}%
\pgfsetfillcolor{currentfill}%
\pgfsetfillopacity{0.700000}%
\pgfsetlinewidth{0.000000pt}%
\definecolor{currentstroke}{rgb}{0.000000,0.000000,0.000000}%
\pgfsetstrokecolor{currentstroke}%
\pgfsetdash{}{0pt}%
\pgfpathmoveto{\pgfqpoint{3.594469in}{1.969868in}}%
\pgfpathlineto{\pgfqpoint{3.608276in}{1.960295in}}%
\pgfpathlineto{\pgfqpoint{3.622086in}{1.950747in}}%
\pgfpathlineto{\pgfqpoint{3.635900in}{1.941224in}}%
\pgfpathlineto{\pgfqpoint{3.649717in}{1.931727in}}%
\pgfpathlineto{\pgfqpoint{3.641303in}{1.950926in}}%
\pgfpathlineto{\pgfqpoint{3.632868in}{1.970883in}}%
\pgfpathlineto{\pgfqpoint{3.624410in}{1.991611in}}%
\pgfpathlineto{\pgfqpoint{3.615929in}{2.013125in}}%
\pgfpathlineto{\pgfqpoint{3.602059in}{2.023096in}}%
\pgfpathlineto{\pgfqpoint{3.588193in}{2.033092in}}%
\pgfpathlineto{\pgfqpoint{3.574329in}{2.043114in}}%
\pgfpathlineto{\pgfqpoint{3.560469in}{2.053162in}}%
\pgfpathlineto{\pgfqpoint{3.569004in}{2.031167in}}%
\pgfpathlineto{\pgfqpoint{3.577516in}{2.009962in}}%
\pgfpathlineto{\pgfqpoint{3.586004in}{1.989535in}}%
\pgfpathlineto{\pgfqpoint{3.594469in}{1.969868in}}%
\pgfpathclose%
\pgfusepath{fill}%
\end{pgfscope}%
\begin{pgfscope}%
\pgfpathrectangle{\pgfqpoint{1.254980in}{0.150000in}}{\pgfqpoint{5.490039in}{5.490039in}}%
\pgfusepath{clip}%
\pgfsetbuttcap%
\pgfsetroundjoin%
\definecolor{currentfill}{rgb}{0.783315,0.879285,0.125405}%
\pgfsetfillcolor{currentfill}%
\pgfsetfillopacity{0.700000}%
\pgfsetlinewidth{0.000000pt}%
\definecolor{currentstroke}{rgb}{0.000000,0.000000,0.000000}%
\pgfsetstrokecolor{currentstroke}%
\pgfsetdash{}{0pt}%
\pgfpathmoveto{\pgfqpoint{2.143453in}{3.357517in}}%
\pgfpathlineto{\pgfqpoint{2.157245in}{3.343577in}}%
\pgfpathlineto{\pgfqpoint{2.171037in}{3.329685in}}%
\pgfpathlineto{\pgfqpoint{2.184827in}{3.315841in}}%
\pgfpathlineto{\pgfqpoint{2.198617in}{3.302043in}}%
\pgfpathlineto{\pgfqpoint{2.188187in}{3.340790in}}%
\pgfpathlineto{\pgfqpoint{2.177695in}{3.380563in}}%
\pgfpathlineto{\pgfqpoint{2.167142in}{3.421381in}}%
\pgfpathlineto{\pgfqpoint{2.153286in}{3.435592in}}%
\pgfpathlineto{\pgfqpoint{2.139428in}{3.449851in}}%
\pgfpathlineto{\pgfqpoint{2.125570in}{3.464158in}}%
\pgfpathlineto{\pgfqpoint{2.111712in}{3.478513in}}%
\pgfpathlineto{\pgfqpoint{2.122355in}{3.437135in}}%
\pgfpathlineto{\pgfqpoint{2.132935in}{3.396810in}}%
\pgfpathlineto{\pgfqpoint{2.143453in}{3.357517in}}%
\pgfpathclose%
\pgfusepath{fill}%
\end{pgfscope}%
\begin{pgfscope}%
\pgfpathrectangle{\pgfqpoint{1.254980in}{0.150000in}}{\pgfqpoint{5.490039in}{5.490039in}}%
\pgfusepath{clip}%
\pgfsetbuttcap%
\pgfsetroundjoin%
\definecolor{currentfill}{rgb}{0.278826,0.175490,0.483397}%
\pgfsetfillcolor{currentfill}%
\pgfsetfillopacity{0.700000}%
\pgfsetlinewidth{0.000000pt}%
\definecolor{currentstroke}{rgb}{0.000000,0.000000,0.000000}%
\pgfsetstrokecolor{currentstroke}%
\pgfsetdash{}{0pt}%
\pgfpathmoveto{\pgfqpoint{4.578301in}{1.307610in}}%
\pgfpathlineto{\pgfqpoint{4.592272in}{1.300847in}}%
\pgfpathlineto{\pgfqpoint{4.606249in}{1.294107in}}%
\pgfpathlineto{\pgfqpoint{4.620231in}{1.287389in}}%
\pgfpathlineto{\pgfqpoint{4.634219in}{1.280693in}}%
\pgfpathlineto{\pgfqpoint{4.626562in}{1.285307in}}%
\pgfpathlineto{\pgfqpoint{4.618901in}{1.290453in}}%
\pgfpathlineto{\pgfqpoint{4.611236in}{1.296143in}}%
\pgfpathlineto{\pgfqpoint{4.603566in}{1.302389in}}%
\pgfpathlineto{\pgfqpoint{4.589550in}{1.309500in}}%
\pgfpathlineto{\pgfqpoint{4.575540in}{1.316633in}}%
\pgfpathlineto{\pgfqpoint{4.561536in}{1.323789in}}%
\pgfpathlineto{\pgfqpoint{4.547536in}{1.330967in}}%
\pgfpathlineto{\pgfqpoint{4.555235in}{1.324300in}}%
\pgfpathlineto{\pgfqpoint{4.562929in}{1.318192in}}%
\pgfpathlineto{\pgfqpoint{4.570617in}{1.312633in}}%
\pgfpathlineto{\pgfqpoint{4.578301in}{1.307610in}}%
\pgfpathclose%
\pgfusepath{fill}%
\end{pgfscope}%
\begin{pgfscope}%
\pgfpathrectangle{\pgfqpoint{1.254980in}{0.150000in}}{\pgfqpoint{5.490039in}{5.490039in}}%
\pgfusepath{clip}%
\pgfsetbuttcap%
\pgfsetroundjoin%
\definecolor{currentfill}{rgb}{0.121380,0.629492,0.531973}%
\pgfsetfillcolor{currentfill}%
\pgfsetfillopacity{0.700000}%
\pgfsetlinewidth{0.000000pt}%
\definecolor{currentstroke}{rgb}{0.000000,0.000000,0.000000}%
\pgfsetstrokecolor{currentstroke}%
\pgfsetdash{}{0pt}%
\pgfpathmoveto{\pgfqpoint{3.063300in}{2.432772in}}%
\pgfpathlineto{\pgfqpoint{3.077068in}{2.421734in}}%
\pgfpathlineto{\pgfqpoint{3.090839in}{2.410725in}}%
\pgfpathlineto{\pgfqpoint{3.104611in}{2.399746in}}%
\pgfpathlineto{\pgfqpoint{3.118386in}{2.388796in}}%
\pgfpathlineto{\pgfqpoint{3.109344in}{2.415521in}}%
\pgfpathlineto{\pgfqpoint{3.100267in}{2.443110in}}%
\pgfpathlineto{\pgfqpoint{3.091154in}{2.471579in}}%
\pgfpathlineto{\pgfqpoint{3.082005in}{2.500944in}}%
\pgfpathlineto{\pgfqpoint{3.068164in}{2.512397in}}%
\pgfpathlineto{\pgfqpoint{3.054325in}{2.523879in}}%
\pgfpathlineto{\pgfqpoint{3.040488in}{2.535391in}}%
\pgfpathlineto{\pgfqpoint{3.026653in}{2.546933in}}%
\pgfpathlineto{\pgfqpoint{3.035870in}{2.517057in}}%
\pgfpathlineto{\pgfqpoint{3.045050in}{2.488082in}}%
\pgfpathlineto{\pgfqpoint{3.054193in}{2.459993in}}%
\pgfpathlineto{\pgfqpoint{3.063300in}{2.432772in}}%
\pgfpathclose%
\pgfusepath{fill}%
\end{pgfscope}%
\begin{pgfscope}%
\pgfpathrectangle{\pgfqpoint{1.254980in}{0.150000in}}{\pgfqpoint{5.490039in}{5.490039in}}%
\pgfusepath{clip}%
\pgfsetbuttcap%
\pgfsetroundjoin%
\definecolor{currentfill}{rgb}{0.262138,0.242286,0.520837}%
\pgfsetfillcolor{currentfill}%
\pgfsetfillopacity{0.700000}%
\pgfsetlinewidth{0.000000pt}%
\definecolor{currentstroke}{rgb}{0.000000,0.000000,0.000000}%
\pgfsetstrokecolor{currentstroke}%
\pgfsetdash{}{0pt}%
\pgfpathmoveto{\pgfqpoint{4.324240in}{1.448872in}}%
\pgfpathlineto{\pgfqpoint{4.338159in}{1.441333in}}%
\pgfpathlineto{\pgfqpoint{4.352082in}{1.433817in}}%
\pgfpathlineto{\pgfqpoint{4.366011in}{1.426324in}}%
\pgfpathlineto{\pgfqpoint{4.379944in}{1.418854in}}%
\pgfpathlineto{\pgfqpoint{4.372145in}{1.427370in}}%
\pgfpathlineto{\pgfqpoint{4.364339in}{1.436483in}}%
\pgfpathlineto{\pgfqpoint{4.356524in}{1.446204in}}%
\pgfpathlineto{\pgfqpoint{4.348701in}{1.456547in}}%
\pgfpathlineto{\pgfqpoint{4.334733in}{1.464450in}}%
\pgfpathlineto{\pgfqpoint{4.320770in}{1.472375in}}%
\pgfpathlineto{\pgfqpoint{4.306812in}{1.480323in}}%
\pgfpathlineto{\pgfqpoint{4.292859in}{1.488294in}}%
\pgfpathlineto{\pgfqpoint{4.300718in}{1.477513in}}%
\pgfpathlineto{\pgfqpoint{4.308568in}{1.467357in}}%
\pgfpathlineto{\pgfqpoint{4.316408in}{1.457814in}}%
\pgfpathlineto{\pgfqpoint{4.324240in}{1.448872in}}%
\pgfpathclose%
\pgfusepath{fill}%
\end{pgfscope}%
\begin{pgfscope}%
\pgfpathrectangle{\pgfqpoint{1.254980in}{0.150000in}}{\pgfqpoint{5.490039in}{5.490039in}}%
\pgfusepath{clip}%
\pgfsetbuttcap%
\pgfsetroundjoin%
\definecolor{currentfill}{rgb}{0.730889,0.871916,0.156029}%
\pgfsetfillcolor{currentfill}%
\pgfsetfillopacity{0.700000}%
\pgfsetlinewidth{0.000000pt}%
\definecolor{currentstroke}{rgb}{0.000000,0.000000,0.000000}%
\pgfsetstrokecolor{currentstroke}%
\pgfsetdash{}{0pt}%
\pgfpathmoveto{\pgfqpoint{2.198617in}{3.302043in}}%
\pgfpathlineto{\pgfqpoint{2.212407in}{3.288292in}}%
\pgfpathlineto{\pgfqpoint{2.226196in}{3.274587in}}%
\pgfpathlineto{\pgfqpoint{2.239985in}{3.260928in}}%
\pgfpathlineto{\pgfqpoint{2.253773in}{3.247314in}}%
\pgfpathlineto{\pgfqpoint{2.243428in}{3.285517in}}%
\pgfpathlineto{\pgfqpoint{2.233024in}{3.324740in}}%
\pgfpathlineto{\pgfqpoint{2.222560in}{3.365001in}}%
\pgfpathlineto{\pgfqpoint{2.208707in}{3.379027in}}%
\pgfpathlineto{\pgfqpoint{2.194852in}{3.393099in}}%
\pgfpathlineto{\pgfqpoint{2.180997in}{3.407216in}}%
\pgfpathlineto{\pgfqpoint{2.167142in}{3.421381in}}%
\pgfpathlineto{\pgfqpoint{2.177695in}{3.380563in}}%
\pgfpathlineto{\pgfqpoint{2.188187in}{3.340790in}}%
\pgfpathlineto{\pgfqpoint{2.198617in}{3.302043in}}%
\pgfpathclose%
\pgfusepath{fill}%
\end{pgfscope}%
\begin{pgfscope}%
\pgfpathrectangle{\pgfqpoint{1.254980in}{0.150000in}}{\pgfqpoint{5.490039in}{5.490039in}}%
\pgfusepath{clip}%
\pgfsetbuttcap%
\pgfsetroundjoin%
\definecolor{currentfill}{rgb}{0.223925,0.334994,0.548053}%
\pgfsetfillcolor{currentfill}%
\pgfsetfillopacity{0.700000}%
\pgfsetlinewidth{0.000000pt}%
\definecolor{currentstroke}{rgb}{0.000000,0.000000,0.000000}%
\pgfsetstrokecolor{currentstroke}%
\pgfsetdash{}{0pt}%
\pgfpathmoveto{\pgfqpoint{4.014732in}{1.652557in}}%
\pgfpathlineto{\pgfqpoint{4.028598in}{1.644122in}}%
\pgfpathlineto{\pgfqpoint{4.042467in}{1.635711in}}%
\pgfpathlineto{\pgfqpoint{4.056341in}{1.627323in}}%
\pgfpathlineto{\pgfqpoint{4.070219in}{1.618959in}}%
\pgfpathlineto{\pgfqpoint{4.062195in}{1.632153in}}%
\pgfpathlineto{\pgfqpoint{4.054158in}{1.646016in}}%
\pgfpathlineto{\pgfqpoint{4.046107in}{1.660561in}}%
\pgfpathlineto{\pgfqpoint{4.038042in}{1.675802in}}%
\pgfpathlineto{\pgfqpoint{4.024122in}{1.684617in}}%
\pgfpathlineto{\pgfqpoint{4.010206in}{1.693456in}}%
\pgfpathlineto{\pgfqpoint{3.996294in}{1.702319in}}%
\pgfpathlineto{\pgfqpoint{3.982385in}{1.711205in}}%
\pgfpathlineto{\pgfqpoint{3.990494in}{1.695506in}}%
\pgfpathlineto{\pgfqpoint{3.998588in}{1.680507in}}%
\pgfpathlineto{\pgfqpoint{4.006668in}{1.666195in}}%
\pgfpathlineto{\pgfqpoint{4.014732in}{1.652557in}}%
\pgfpathclose%
\pgfusepath{fill}%
\end{pgfscope}%
\begin{pgfscope}%
\pgfpathrectangle{\pgfqpoint{1.254980in}{0.150000in}}{\pgfqpoint{5.490039in}{5.490039in}}%
\pgfusepath{clip}%
\pgfsetbuttcap%
\pgfsetroundjoin%
\definecolor{currentfill}{rgb}{0.678489,0.863742,0.189503}%
\pgfsetfillcolor{currentfill}%
\pgfsetfillopacity{0.700000}%
\pgfsetlinewidth{0.000000pt}%
\definecolor{currentstroke}{rgb}{0.000000,0.000000,0.000000}%
\pgfsetstrokecolor{currentstroke}%
\pgfsetdash{}{0pt}%
\pgfpathmoveto{\pgfqpoint{2.253773in}{3.247314in}}%
\pgfpathlineto{\pgfqpoint{2.267561in}{3.233745in}}%
\pgfpathlineto{\pgfqpoint{2.281348in}{3.220220in}}%
\pgfpathlineto{\pgfqpoint{2.295136in}{3.206740in}}%
\pgfpathlineto{\pgfqpoint{2.308923in}{3.193303in}}%
\pgfpathlineto{\pgfqpoint{2.298663in}{3.230964in}}%
\pgfpathlineto{\pgfqpoint{2.288346in}{3.269639in}}%
\pgfpathlineto{\pgfqpoint{2.277970in}{3.309347in}}%
\pgfpathlineto{\pgfqpoint{2.264118in}{3.323194in}}%
\pgfpathlineto{\pgfqpoint{2.250266in}{3.337085in}}%
\pgfpathlineto{\pgfqpoint{2.236413in}{3.351021in}}%
\pgfpathlineto{\pgfqpoint{2.222560in}{3.365001in}}%
\pgfpathlineto{\pgfqpoint{2.233024in}{3.324740in}}%
\pgfpathlineto{\pgfqpoint{2.243428in}{3.285517in}}%
\pgfpathlineto{\pgfqpoint{2.253773in}{3.247314in}}%
\pgfpathclose%
\pgfusepath{fill}%
\end{pgfscope}%
\begin{pgfscope}%
\pgfpathrectangle{\pgfqpoint{1.254980in}{0.150000in}}{\pgfqpoint{5.490039in}{5.490039in}}%
\pgfusepath{clip}%
\pgfsetbuttcap%
\pgfsetroundjoin%
\definecolor{currentfill}{rgb}{0.174274,0.445044,0.557792}%
\pgfsetfillcolor{currentfill}%
\pgfsetfillopacity{0.700000}%
\pgfsetlinewidth{0.000000pt}%
\definecolor{currentstroke}{rgb}{0.000000,0.000000,0.000000}%
\pgfsetstrokecolor{currentstroke}%
\pgfsetdash{}{0pt}%
\pgfpathmoveto{\pgfqpoint{3.649717in}{1.931727in}}%
\pgfpathlineto{\pgfqpoint{3.663537in}{1.922255in}}%
\pgfpathlineto{\pgfqpoint{3.677361in}{1.912808in}}%
\pgfpathlineto{\pgfqpoint{3.691188in}{1.903386in}}%
\pgfpathlineto{\pgfqpoint{3.705018in}{1.893989in}}%
\pgfpathlineto{\pgfqpoint{3.696655in}{1.912723in}}%
\pgfpathlineto{\pgfqpoint{3.688272in}{1.932208in}}%
\pgfpathlineto{\pgfqpoint{3.679866in}{1.952461in}}%
\pgfpathlineto{\pgfqpoint{3.671439in}{1.973494in}}%
\pgfpathlineto{\pgfqpoint{3.657557in}{1.983364in}}%
\pgfpathlineto{\pgfqpoint{3.643678in}{1.993259in}}%
\pgfpathlineto{\pgfqpoint{3.629802in}{2.003179in}}%
\pgfpathlineto{\pgfqpoint{3.615929in}{2.013125in}}%
\pgfpathlineto{\pgfqpoint{3.624410in}{1.991611in}}%
\pgfpathlineto{\pgfqpoint{3.632868in}{1.970883in}}%
\pgfpathlineto{\pgfqpoint{3.641303in}{1.950926in}}%
\pgfpathlineto{\pgfqpoint{3.649717in}{1.931727in}}%
\pgfpathclose%
\pgfusepath{fill}%
\end{pgfscope}%
\begin{pgfscope}%
\pgfpathrectangle{\pgfqpoint{1.254980in}{0.150000in}}{\pgfqpoint{5.490039in}{5.490039in}}%
\pgfusepath{clip}%
\pgfsetbuttcap%
\pgfsetroundjoin%
\definecolor{currentfill}{rgb}{0.119483,0.614817,0.537692}%
\pgfsetfillcolor{currentfill}%
\pgfsetfillopacity{0.700000}%
\pgfsetlinewidth{0.000000pt}%
\definecolor{currentstroke}{rgb}{0.000000,0.000000,0.000000}%
\pgfsetstrokecolor{currentstroke}%
\pgfsetdash{}{0pt}%
\pgfpathmoveto{\pgfqpoint{3.118386in}{2.388796in}}%
\pgfpathlineto{\pgfqpoint{3.132163in}{2.377875in}}%
\pgfpathlineto{\pgfqpoint{3.145942in}{2.366984in}}%
\pgfpathlineto{\pgfqpoint{3.159723in}{2.356121in}}%
\pgfpathlineto{\pgfqpoint{3.173507in}{2.345288in}}%
\pgfpathlineto{\pgfqpoint{3.164528in}{2.371519in}}%
\pgfpathlineto{\pgfqpoint{3.155516in}{2.398609in}}%
\pgfpathlineto{\pgfqpoint{3.146469in}{2.426572in}}%
\pgfpathlineto{\pgfqpoint{3.137388in}{2.455427in}}%
\pgfpathlineto{\pgfqpoint{3.123539in}{2.466762in}}%
\pgfpathlineto{\pgfqpoint{3.109692in}{2.478127in}}%
\pgfpathlineto{\pgfqpoint{3.095848in}{2.489521in}}%
\pgfpathlineto{\pgfqpoint{3.082005in}{2.500944in}}%
\pgfpathlineto{\pgfqpoint{3.091154in}{2.471579in}}%
\pgfpathlineto{\pgfqpoint{3.100267in}{2.443110in}}%
\pgfpathlineto{\pgfqpoint{3.109344in}{2.415521in}}%
\pgfpathlineto{\pgfqpoint{3.118386in}{2.388796in}}%
\pgfpathclose%
\pgfusepath{fill}%
\end{pgfscope}%
\begin{pgfscope}%
\pgfpathrectangle{\pgfqpoint{1.254980in}{0.150000in}}{\pgfqpoint{5.490039in}{5.490039in}}%
\pgfusepath{clip}%
\pgfsetbuttcap%
\pgfsetroundjoin%
\definecolor{currentfill}{rgb}{0.283229,0.120777,0.440584}%
\pgfsetfillcolor{currentfill}%
\pgfsetfillopacity{0.700000}%
\pgfsetlinewidth{0.000000pt}%
\definecolor{currentstroke}{rgb}{0.000000,0.000000,0.000000}%
\pgfsetstrokecolor{currentstroke}%
\pgfsetdash{}{0pt}%
\pgfpathmoveto{\pgfqpoint{4.832819in}{1.193516in}}%
\pgfpathlineto{\pgfqpoint{4.846859in}{1.187508in}}%
\pgfpathlineto{\pgfqpoint{4.860904in}{1.181522in}}%
\pgfpathlineto{\pgfqpoint{4.874956in}{1.175558in}}%
\pgfpathlineto{\pgfqpoint{4.867398in}{1.176676in}}%
\pgfpathlineto{\pgfqpoint{4.859838in}{1.178268in}}%
\pgfpathlineto{\pgfqpoint{4.852277in}{1.180343in}}%
\pgfpathlineto{\pgfqpoint{4.844714in}{1.182914in}}%
\pgfpathlineto{\pgfqpoint{4.830641in}{1.189278in}}%
\pgfpathlineto{\pgfqpoint{4.816573in}{1.195664in}}%
\pgfpathlineto{\pgfqpoint{4.802511in}{1.202073in}}%
\pgfpathlineto{\pgfqpoint{4.810091in}{1.199197in}}%
\pgfpathlineto{\pgfqpoint{4.817669in}{1.196820in}}%
\pgfpathlineto{\pgfqpoint{4.825245in}{1.194930in}}%
\pgfpathlineto{\pgfqpoint{4.832819in}{1.193516in}}%
\pgfpathclose%
\pgfusepath{fill}%
\end{pgfscope}%
\begin{pgfscope}%
\pgfpathrectangle{\pgfqpoint{1.254980in}{0.150000in}}{\pgfqpoint{5.490039in}{5.490039in}}%
\pgfusepath{clip}%
\pgfsetbuttcap%
\pgfsetroundjoin%
\definecolor{currentfill}{rgb}{0.626579,0.854645,0.223353}%
\pgfsetfillcolor{currentfill}%
\pgfsetfillopacity{0.700000}%
\pgfsetlinewidth{0.000000pt}%
\definecolor{currentstroke}{rgb}{0.000000,0.000000,0.000000}%
\pgfsetstrokecolor{currentstroke}%
\pgfsetdash{}{0pt}%
\pgfpathmoveto{\pgfqpoint{2.308923in}{3.193303in}}%
\pgfpathlineto{\pgfqpoint{2.322710in}{3.179909in}}%
\pgfpathlineto{\pgfqpoint{2.336497in}{3.166559in}}%
\pgfpathlineto{\pgfqpoint{2.350284in}{3.153250in}}%
\pgfpathlineto{\pgfqpoint{2.364071in}{3.139984in}}%
\pgfpathlineto{\pgfqpoint{2.353895in}{3.177106in}}%
\pgfpathlineto{\pgfqpoint{2.343663in}{3.215236in}}%
\pgfpathlineto{\pgfqpoint{2.333373in}{3.254392in}}%
\pgfpathlineto{\pgfqpoint{2.319523in}{3.268066in}}%
\pgfpathlineto{\pgfqpoint{2.305672in}{3.281783in}}%
\pgfpathlineto{\pgfqpoint{2.291821in}{3.295543in}}%
\pgfpathlineto{\pgfqpoint{2.277970in}{3.309347in}}%
\pgfpathlineto{\pgfqpoint{2.288346in}{3.269639in}}%
\pgfpathlineto{\pgfqpoint{2.298663in}{3.230964in}}%
\pgfpathlineto{\pgfqpoint{2.308923in}{3.193303in}}%
\pgfpathclose%
\pgfusepath{fill}%
\end{pgfscope}%
\begin{pgfscope}%
\pgfpathrectangle{\pgfqpoint{1.254980in}{0.150000in}}{\pgfqpoint{5.490039in}{5.490039in}}%
\pgfusepath{clip}%
\pgfsetbuttcap%
\pgfsetroundjoin%
\definecolor{currentfill}{rgb}{0.575563,0.844566,0.256415}%
\pgfsetfillcolor{currentfill}%
\pgfsetfillopacity{0.700000}%
\pgfsetlinewidth{0.000000pt}%
\definecolor{currentstroke}{rgb}{0.000000,0.000000,0.000000}%
\pgfsetstrokecolor{currentstroke}%
\pgfsetdash{}{0pt}%
\pgfpathmoveto{\pgfqpoint{2.364071in}{3.139984in}}%
\pgfpathlineto{\pgfqpoint{2.377858in}{3.126760in}}%
\pgfpathlineto{\pgfqpoint{2.391645in}{3.113577in}}%
\pgfpathlineto{\pgfqpoint{2.405432in}{3.100435in}}%
\pgfpathlineto{\pgfqpoint{2.419220in}{3.087334in}}%
\pgfpathlineto{\pgfqpoint{2.409127in}{3.123919in}}%
\pgfpathlineto{\pgfqpoint{2.398979in}{3.161506in}}%
\pgfpathlineto{\pgfqpoint{2.388775in}{3.200111in}}%
\pgfpathlineto{\pgfqpoint{2.374924in}{3.213619in}}%
\pgfpathlineto{\pgfqpoint{2.361074in}{3.227169in}}%
\pgfpathlineto{\pgfqpoint{2.347224in}{3.240759in}}%
\pgfpathlineto{\pgfqpoint{2.333373in}{3.254392in}}%
\pgfpathlineto{\pgfqpoint{2.343663in}{3.215236in}}%
\pgfpathlineto{\pgfqpoint{2.353895in}{3.177106in}}%
\pgfpathlineto{\pgfqpoint{2.364071in}{3.139984in}}%
\pgfpathclose%
\pgfusepath{fill}%
\end{pgfscope}%
\begin{pgfscope}%
\pgfpathrectangle{\pgfqpoint{1.254980in}{0.150000in}}{\pgfqpoint{5.490039in}{5.490039in}}%
\pgfusepath{clip}%
\pgfsetbuttcap%
\pgfsetroundjoin%
\definecolor{currentfill}{rgb}{0.229739,0.322361,0.545706}%
\pgfsetfillcolor{currentfill}%
\pgfsetfillopacity{0.700000}%
\pgfsetlinewidth{0.000000pt}%
\definecolor{currentstroke}{rgb}{0.000000,0.000000,0.000000}%
\pgfsetstrokecolor{currentstroke}%
\pgfsetdash{}{0pt}%
\pgfpathmoveto{\pgfqpoint{4.070219in}{1.618959in}}%
\pgfpathlineto{\pgfqpoint{4.084101in}{1.610618in}}%
\pgfpathlineto{\pgfqpoint{4.097987in}{1.602301in}}%
\pgfpathlineto{\pgfqpoint{4.111878in}{1.594007in}}%
\pgfpathlineto{\pgfqpoint{4.125773in}{1.585736in}}%
\pgfpathlineto{\pgfqpoint{4.117789in}{1.598486in}}%
\pgfpathlineto{\pgfqpoint{4.109793in}{1.611901in}}%
\pgfpathlineto{\pgfqpoint{4.101784in}{1.625993in}}%
\pgfpathlineto{\pgfqpoint{4.093762in}{1.640777in}}%
\pgfpathlineto{\pgfqpoint{4.079826in}{1.649498in}}%
\pgfpathlineto{\pgfqpoint{4.065894in}{1.658243in}}%
\pgfpathlineto{\pgfqpoint{4.051966in}{1.667011in}}%
\pgfpathlineto{\pgfqpoint{4.038042in}{1.675802in}}%
\pgfpathlineto{\pgfqpoint{4.046107in}{1.660561in}}%
\pgfpathlineto{\pgfqpoint{4.054158in}{1.646016in}}%
\pgfpathlineto{\pgfqpoint{4.062195in}{1.632153in}}%
\pgfpathlineto{\pgfqpoint{4.070219in}{1.618959in}}%
\pgfpathclose%
\pgfusepath{fill}%
\end{pgfscope}%
\begin{pgfscope}%
\pgfpathrectangle{\pgfqpoint{1.254980in}{0.150000in}}{\pgfqpoint{5.490039in}{5.490039in}}%
\pgfusepath{clip}%
\pgfsetbuttcap%
\pgfsetroundjoin%
\definecolor{currentfill}{rgb}{0.265145,0.232956,0.516599}%
\pgfsetfillcolor{currentfill}%
\pgfsetfillopacity{0.700000}%
\pgfsetlinewidth{0.000000pt}%
\definecolor{currentstroke}{rgb}{0.000000,0.000000,0.000000}%
\pgfsetstrokecolor{currentstroke}%
\pgfsetdash{}{0pt}%
\pgfpathmoveto{\pgfqpoint{4.379944in}{1.418854in}}%
\pgfpathlineto{\pgfqpoint{4.393882in}{1.411406in}}%
\pgfpathlineto{\pgfqpoint{4.407825in}{1.403981in}}%
\pgfpathlineto{\pgfqpoint{4.421774in}{1.396578in}}%
\pgfpathlineto{\pgfqpoint{4.435727in}{1.389198in}}%
\pgfpathlineto{\pgfqpoint{4.427961in}{1.397289in}}%
\pgfpathlineto{\pgfqpoint{4.420187in}{1.405971in}}%
\pgfpathlineto{\pgfqpoint{4.412406in}{1.415259in}}%
\pgfpathlineto{\pgfqpoint{4.404618in}{1.425164in}}%
\pgfpathlineto{\pgfqpoint{4.390631in}{1.432976in}}%
\pgfpathlineto{\pgfqpoint{4.376650in}{1.440810in}}%
\pgfpathlineto{\pgfqpoint{4.362673in}{1.448667in}}%
\pgfpathlineto{\pgfqpoint{4.348701in}{1.456547in}}%
\pgfpathlineto{\pgfqpoint{4.356524in}{1.446204in}}%
\pgfpathlineto{\pgfqpoint{4.364339in}{1.436483in}}%
\pgfpathlineto{\pgfqpoint{4.372145in}{1.427370in}}%
\pgfpathlineto{\pgfqpoint{4.379944in}{1.418854in}}%
\pgfpathclose%
\pgfusepath{fill}%
\end{pgfscope}%
\begin{pgfscope}%
\pgfpathrectangle{\pgfqpoint{1.254980in}{0.150000in}}{\pgfqpoint{5.490039in}{5.490039in}}%
\pgfusepath{clip}%
\pgfsetbuttcap%
\pgfsetroundjoin%
\definecolor{currentfill}{rgb}{0.280255,0.165693,0.476498}%
\pgfsetfillcolor{currentfill}%
\pgfsetfillopacity{0.700000}%
\pgfsetlinewidth{0.000000pt}%
\definecolor{currentstroke}{rgb}{0.000000,0.000000,0.000000}%
\pgfsetstrokecolor{currentstroke}%
\pgfsetdash{}{0pt}%
\pgfpathmoveto{\pgfqpoint{4.634219in}{1.280693in}}%
\pgfpathlineto{\pgfqpoint{4.648212in}{1.274020in}}%
\pgfpathlineto{\pgfqpoint{4.662211in}{1.267368in}}%
\pgfpathlineto{\pgfqpoint{4.676216in}{1.260739in}}%
\pgfpathlineto{\pgfqpoint{4.690226in}{1.254132in}}%
\pgfpathlineto{\pgfqpoint{4.682595in}{1.258336in}}%
\pgfpathlineto{\pgfqpoint{4.674960in}{1.263069in}}%
\pgfpathlineto{\pgfqpoint{4.667322in}{1.268341in}}%
\pgfpathlineto{\pgfqpoint{4.659680in}{1.274166in}}%
\pgfpathlineto{\pgfqpoint{4.645644in}{1.281189in}}%
\pgfpathlineto{\pgfqpoint{4.631612in}{1.288233in}}%
\pgfpathlineto{\pgfqpoint{4.617586in}{1.295300in}}%
\pgfpathlineto{\pgfqpoint{4.603566in}{1.302389in}}%
\pgfpathlineto{\pgfqpoint{4.611236in}{1.296143in}}%
\pgfpathlineto{\pgfqpoint{4.618901in}{1.290453in}}%
\pgfpathlineto{\pgfqpoint{4.626562in}{1.285307in}}%
\pgfpathlineto{\pgfqpoint{4.634219in}{1.280693in}}%
\pgfpathclose%
\pgfusepath{fill}%
\end{pgfscope}%
\begin{pgfscope}%
\pgfpathrectangle{\pgfqpoint{1.254980in}{0.150000in}}{\pgfqpoint{5.490039in}{5.490039in}}%
\pgfusepath{clip}%
\pgfsetbuttcap%
\pgfsetroundjoin%
\definecolor{currentfill}{rgb}{0.120092,0.600104,0.542530}%
\pgfsetfillcolor{currentfill}%
\pgfsetfillopacity{0.700000}%
\pgfsetlinewidth{0.000000pt}%
\definecolor{currentstroke}{rgb}{0.000000,0.000000,0.000000}%
\pgfsetstrokecolor{currentstroke}%
\pgfsetdash{}{0pt}%
\pgfpathmoveto{\pgfqpoint{3.173507in}{2.345288in}}%
\pgfpathlineto{\pgfqpoint{3.187292in}{2.334483in}}%
\pgfpathlineto{\pgfqpoint{3.201081in}{2.323706in}}%
\pgfpathlineto{\pgfqpoint{3.214871in}{2.312958in}}%
\pgfpathlineto{\pgfqpoint{3.228664in}{2.302239in}}%
\pgfpathlineto{\pgfqpoint{3.219748in}{2.327977in}}%
\pgfpathlineto{\pgfqpoint{3.210800in}{2.354568in}}%
\pgfpathlineto{\pgfqpoint{3.201818in}{2.382028in}}%
\pgfpathlineto{\pgfqpoint{3.192803in}{2.410374in}}%
\pgfpathlineto{\pgfqpoint{3.178946in}{2.421594in}}%
\pgfpathlineto{\pgfqpoint{3.165091in}{2.432843in}}%
\pgfpathlineto{\pgfqpoint{3.151238in}{2.444121in}}%
\pgfpathlineto{\pgfqpoint{3.137388in}{2.455427in}}%
\pgfpathlineto{\pgfqpoint{3.146469in}{2.426572in}}%
\pgfpathlineto{\pgfqpoint{3.155516in}{2.398609in}}%
\pgfpathlineto{\pgfqpoint{3.164528in}{2.371519in}}%
\pgfpathlineto{\pgfqpoint{3.173507in}{2.345288in}}%
\pgfpathclose%
\pgfusepath{fill}%
\end{pgfscope}%
\begin{pgfscope}%
\pgfpathrectangle{\pgfqpoint{1.254980in}{0.150000in}}{\pgfqpoint{5.490039in}{5.490039in}}%
\pgfusepath{clip}%
\pgfsetbuttcap%
\pgfsetroundjoin%
\definecolor{currentfill}{rgb}{0.525776,0.833491,0.288127}%
\pgfsetfillcolor{currentfill}%
\pgfsetfillopacity{0.700000}%
\pgfsetlinewidth{0.000000pt}%
\definecolor{currentstroke}{rgb}{0.000000,0.000000,0.000000}%
\pgfsetstrokecolor{currentstroke}%
\pgfsetdash{}{0pt}%
\pgfpathmoveto{\pgfqpoint{2.419220in}{3.087334in}}%
\pgfpathlineto{\pgfqpoint{2.433008in}{3.074274in}}%
\pgfpathlineto{\pgfqpoint{2.446796in}{3.061253in}}%
\pgfpathlineto{\pgfqpoint{2.460584in}{3.048272in}}%
\pgfpathlineto{\pgfqpoint{2.474373in}{3.035331in}}%
\pgfpathlineto{\pgfqpoint{2.464361in}{3.071381in}}%
\pgfpathlineto{\pgfqpoint{2.454296in}{3.108426in}}%
\pgfpathlineto{\pgfqpoint{2.444176in}{3.146484in}}%
\pgfpathlineto{\pgfqpoint{2.430326in}{3.159831in}}%
\pgfpathlineto{\pgfqpoint{2.416475in}{3.173217in}}%
\pgfpathlineto{\pgfqpoint{2.402625in}{3.186644in}}%
\pgfpathlineto{\pgfqpoint{2.388775in}{3.200111in}}%
\pgfpathlineto{\pgfqpoint{2.398979in}{3.161506in}}%
\pgfpathlineto{\pgfqpoint{2.409127in}{3.123919in}}%
\pgfpathlineto{\pgfqpoint{2.419220in}{3.087334in}}%
\pgfpathclose%
\pgfusepath{fill}%
\end{pgfscope}%
\begin{pgfscope}%
\pgfpathrectangle{\pgfqpoint{1.254980in}{0.150000in}}{\pgfqpoint{5.490039in}{5.490039in}}%
\pgfusepath{clip}%
\pgfsetbuttcap%
\pgfsetroundjoin%
\definecolor{currentfill}{rgb}{0.179019,0.433756,0.557430}%
\pgfsetfillcolor{currentfill}%
\pgfsetfillopacity{0.700000}%
\pgfsetlinewidth{0.000000pt}%
\definecolor{currentstroke}{rgb}{0.000000,0.000000,0.000000}%
\pgfsetstrokecolor{currentstroke}%
\pgfsetdash{}{0pt}%
\pgfpathmoveto{\pgfqpoint{3.705018in}{1.893989in}}%
\pgfpathlineto{\pgfqpoint{3.718852in}{1.884617in}}%
\pgfpathlineto{\pgfqpoint{3.732690in}{1.875270in}}%
\pgfpathlineto{\pgfqpoint{3.746531in}{1.865948in}}%
\pgfpathlineto{\pgfqpoint{3.760375in}{1.856650in}}%
\pgfpathlineto{\pgfqpoint{3.752062in}{1.874919in}}%
\pgfpathlineto{\pgfqpoint{3.743729in}{1.893934in}}%
\pgfpathlineto{\pgfqpoint{3.735375in}{1.913712in}}%
\pgfpathlineto{\pgfqpoint{3.727000in}{1.934266in}}%
\pgfpathlineto{\pgfqpoint{3.713105in}{1.944036in}}%
\pgfpathlineto{\pgfqpoint{3.699213in}{1.953830in}}%
\pgfpathlineto{\pgfqpoint{3.685324in}{1.963650in}}%
\pgfpathlineto{\pgfqpoint{3.671439in}{1.973494in}}%
\pgfpathlineto{\pgfqpoint{3.679866in}{1.952461in}}%
\pgfpathlineto{\pgfqpoint{3.688272in}{1.932208in}}%
\pgfpathlineto{\pgfqpoint{3.696655in}{1.912723in}}%
\pgfpathlineto{\pgfqpoint{3.705018in}{1.893989in}}%
\pgfpathclose%
\pgfusepath{fill}%
\end{pgfscope}%
\begin{pgfscope}%
\pgfpathrectangle{\pgfqpoint{1.254980in}{0.150000in}}{\pgfqpoint{5.490039in}{5.490039in}}%
\pgfusepath{clip}%
\pgfsetbuttcap%
\pgfsetroundjoin%
\definecolor{currentfill}{rgb}{0.477504,0.821444,0.318195}%
\pgfsetfillcolor{currentfill}%
\pgfsetfillopacity{0.700000}%
\pgfsetlinewidth{0.000000pt}%
\definecolor{currentstroke}{rgb}{0.000000,0.000000,0.000000}%
\pgfsetstrokecolor{currentstroke}%
\pgfsetdash{}{0pt}%
\pgfpathmoveto{\pgfqpoint{2.474373in}{3.035331in}}%
\pgfpathlineto{\pgfqpoint{2.488162in}{3.022428in}}%
\pgfpathlineto{\pgfqpoint{2.501951in}{3.009565in}}%
\pgfpathlineto{\pgfqpoint{2.515741in}{2.996740in}}%
\pgfpathlineto{\pgfqpoint{2.529532in}{2.983953in}}%
\pgfpathlineto{\pgfqpoint{2.519600in}{3.019470in}}%
\pgfpathlineto{\pgfqpoint{2.509617in}{3.055975in}}%
\pgfpathlineto{\pgfqpoint{2.499581in}{3.093488in}}%
\pgfpathlineto{\pgfqpoint{2.485729in}{3.106679in}}%
\pgfpathlineto{\pgfqpoint{2.471878in}{3.119908in}}%
\pgfpathlineto{\pgfqpoint{2.458027in}{3.133177in}}%
\pgfpathlineto{\pgfqpoint{2.444176in}{3.146484in}}%
\pgfpathlineto{\pgfqpoint{2.454296in}{3.108426in}}%
\pgfpathlineto{\pgfqpoint{2.464361in}{3.071381in}}%
\pgfpathlineto{\pgfqpoint{2.474373in}{3.035331in}}%
\pgfpathclose%
\pgfusepath{fill}%
\end{pgfscope}%
\begin{pgfscope}%
\pgfpathrectangle{\pgfqpoint{1.254980in}{0.150000in}}{\pgfqpoint{5.490039in}{5.490039in}}%
\pgfusepath{clip}%
\pgfsetbuttcap%
\pgfsetroundjoin%
\definecolor{currentfill}{rgb}{0.121831,0.589055,0.545623}%
\pgfsetfillcolor{currentfill}%
\pgfsetfillopacity{0.700000}%
\pgfsetlinewidth{0.000000pt}%
\definecolor{currentstroke}{rgb}{0.000000,0.000000,0.000000}%
\pgfsetstrokecolor{currentstroke}%
\pgfsetdash{}{0pt}%
\pgfpathmoveto{\pgfqpoint{3.228664in}{2.302239in}}%
\pgfpathlineto{\pgfqpoint{3.242459in}{2.291547in}}%
\pgfpathlineto{\pgfqpoint{3.256257in}{2.280884in}}%
\pgfpathlineto{\pgfqpoint{3.270057in}{2.270248in}}%
\pgfpathlineto{\pgfqpoint{3.283859in}{2.259641in}}%
\pgfpathlineto{\pgfqpoint{3.275005in}{2.284887in}}%
\pgfpathlineto{\pgfqpoint{3.266120in}{2.310981in}}%
\pgfpathlineto{\pgfqpoint{3.257202in}{2.337939in}}%
\pgfpathlineto{\pgfqpoint{3.248252in}{2.365777in}}%
\pgfpathlineto{\pgfqpoint{3.234387in}{2.376884in}}%
\pgfpathlineto{\pgfqpoint{3.220523in}{2.388019in}}%
\pgfpathlineto{\pgfqpoint{3.206662in}{2.399182in}}%
\pgfpathlineto{\pgfqpoint{3.192803in}{2.410374in}}%
\pgfpathlineto{\pgfqpoint{3.201818in}{2.382028in}}%
\pgfpathlineto{\pgfqpoint{3.210800in}{2.354568in}}%
\pgfpathlineto{\pgfqpoint{3.219748in}{2.327977in}}%
\pgfpathlineto{\pgfqpoint{3.228664in}{2.302239in}}%
\pgfpathclose%
\pgfusepath{fill}%
\end{pgfscope}%
\begin{pgfscope}%
\pgfpathrectangle{\pgfqpoint{1.254980in}{0.150000in}}{\pgfqpoint{5.490039in}{5.490039in}}%
\pgfusepath{clip}%
\pgfsetbuttcap%
\pgfsetroundjoin%
\definecolor{currentfill}{rgb}{0.233603,0.313828,0.543914}%
\pgfsetfillcolor{currentfill}%
\pgfsetfillopacity{0.700000}%
\pgfsetlinewidth{0.000000pt}%
\definecolor{currentstroke}{rgb}{0.000000,0.000000,0.000000}%
\pgfsetstrokecolor{currentstroke}%
\pgfsetdash{}{0pt}%
\pgfpathmoveto{\pgfqpoint{4.125773in}{1.585736in}}%
\pgfpathlineto{\pgfqpoint{4.139672in}{1.577489in}}%
\pgfpathlineto{\pgfqpoint{4.153576in}{1.569265in}}%
\pgfpathlineto{\pgfqpoint{4.167484in}{1.561064in}}%
\pgfpathlineto{\pgfqpoint{4.181397in}{1.552886in}}%
\pgfpathlineto{\pgfqpoint{4.173452in}{1.565192in}}%
\pgfpathlineto{\pgfqpoint{4.165496in}{1.578159in}}%
\pgfpathlineto{\pgfqpoint{4.157528in}{1.591799in}}%
\pgfpathlineto{\pgfqpoint{4.149548in}{1.606127in}}%
\pgfpathlineto{\pgfqpoint{4.135596in}{1.614755in}}%
\pgfpathlineto{\pgfqpoint{4.121647in}{1.623405in}}%
\pgfpathlineto{\pgfqpoint{4.107702in}{1.632080in}}%
\pgfpathlineto{\pgfqpoint{4.093762in}{1.640777in}}%
\pgfpathlineto{\pgfqpoint{4.101784in}{1.625993in}}%
\pgfpathlineto{\pgfqpoint{4.109793in}{1.611901in}}%
\pgfpathlineto{\pgfqpoint{4.117789in}{1.598486in}}%
\pgfpathlineto{\pgfqpoint{4.125773in}{1.585736in}}%
\pgfpathclose%
\pgfusepath{fill}%
\end{pgfscope}%
\begin{pgfscope}%
\pgfpathrectangle{\pgfqpoint{1.254980in}{0.150000in}}{\pgfqpoint{5.490039in}{5.490039in}}%
\pgfusepath{clip}%
\pgfsetbuttcap%
\pgfsetroundjoin%
\definecolor{currentfill}{rgb}{0.440137,0.811138,0.340967}%
\pgfsetfillcolor{currentfill}%
\pgfsetfillopacity{0.700000}%
\pgfsetlinewidth{0.000000pt}%
\definecolor{currentstroke}{rgb}{0.000000,0.000000,0.000000}%
\pgfsetstrokecolor{currentstroke}%
\pgfsetdash{}{0pt}%
\pgfpathmoveto{\pgfqpoint{2.529532in}{2.983953in}}%
\pgfpathlineto{\pgfqpoint{2.543323in}{2.971204in}}%
\pgfpathlineto{\pgfqpoint{2.557114in}{2.958493in}}%
\pgfpathlineto{\pgfqpoint{2.570907in}{2.945819in}}%
\pgfpathlineto{\pgfqpoint{2.584700in}{2.933182in}}%
\pgfpathlineto{\pgfqpoint{2.574848in}{2.968167in}}%
\pgfpathlineto{\pgfqpoint{2.564945in}{3.004136in}}%
\pgfpathlineto{\pgfqpoint{2.554991in}{3.041105in}}%
\pgfpathlineto{\pgfqpoint{2.541138in}{3.054145in}}%
\pgfpathlineto{\pgfqpoint{2.527285in}{3.067221in}}%
\pgfpathlineto{\pgfqpoint{2.513433in}{3.080336in}}%
\pgfpathlineto{\pgfqpoint{2.499581in}{3.093488in}}%
\pgfpathlineto{\pgfqpoint{2.509617in}{3.055975in}}%
\pgfpathlineto{\pgfqpoint{2.519600in}{3.019470in}}%
\pgfpathlineto{\pgfqpoint{2.529532in}{2.983953in}}%
\pgfpathclose%
\pgfusepath{fill}%
\end{pgfscope}%
\begin{pgfscope}%
\pgfpathrectangle{\pgfqpoint{1.254980in}{0.150000in}}{\pgfqpoint{5.490039in}{5.490039in}}%
\pgfusepath{clip}%
\pgfsetbuttcap%
\pgfsetroundjoin%
\definecolor{currentfill}{rgb}{0.182256,0.426184,0.557120}%
\pgfsetfillcolor{currentfill}%
\pgfsetfillopacity{0.700000}%
\pgfsetlinewidth{0.000000pt}%
\definecolor{currentstroke}{rgb}{0.000000,0.000000,0.000000}%
\pgfsetstrokecolor{currentstroke}%
\pgfsetdash{}{0pt}%
\pgfpathmoveto{\pgfqpoint{3.760375in}{1.856650in}}%
\pgfpathlineto{\pgfqpoint{3.774223in}{1.847378in}}%
\pgfpathlineto{\pgfqpoint{3.788075in}{1.838129in}}%
\pgfpathlineto{\pgfqpoint{3.801930in}{1.828906in}}%
\pgfpathlineto{\pgfqpoint{3.815789in}{1.819706in}}%
\pgfpathlineto{\pgfqpoint{3.807524in}{1.837510in}}%
\pgfpathlineto{\pgfqpoint{3.799240in}{1.856057in}}%
\pgfpathlineto{\pgfqpoint{3.790937in}{1.875360in}}%
\pgfpathlineto{\pgfqpoint{3.782615in}{1.895436in}}%
\pgfpathlineto{\pgfqpoint{3.768706in}{1.905106in}}%
\pgfpathlineto{\pgfqpoint{3.754801in}{1.914801in}}%
\pgfpathlineto{\pgfqpoint{3.740899in}{1.924521in}}%
\pgfpathlineto{\pgfqpoint{3.727000in}{1.934266in}}%
\pgfpathlineto{\pgfqpoint{3.735375in}{1.913712in}}%
\pgfpathlineto{\pgfqpoint{3.743729in}{1.893934in}}%
\pgfpathlineto{\pgfqpoint{3.752062in}{1.874919in}}%
\pgfpathlineto{\pgfqpoint{3.760375in}{1.856650in}}%
\pgfpathclose%
\pgfusepath{fill}%
\end{pgfscope}%
\begin{pgfscope}%
\pgfpathrectangle{\pgfqpoint{1.254980in}{0.150000in}}{\pgfqpoint{5.490039in}{5.490039in}}%
\pgfusepath{clip}%
\pgfsetbuttcap%
\pgfsetroundjoin%
\definecolor{currentfill}{rgb}{0.266580,0.228262,0.514349}%
\pgfsetfillcolor{currentfill}%
\pgfsetfillopacity{0.700000}%
\pgfsetlinewidth{0.000000pt}%
\definecolor{currentstroke}{rgb}{0.000000,0.000000,0.000000}%
\pgfsetstrokecolor{currentstroke}%
\pgfsetdash{}{0pt}%
\pgfpathmoveto{\pgfqpoint{4.435727in}{1.389198in}}%
\pgfpathlineto{\pgfqpoint{4.449685in}{1.381841in}}%
\pgfpathlineto{\pgfqpoint{4.463648in}{1.374506in}}%
\pgfpathlineto{\pgfqpoint{4.477617in}{1.367193in}}%
\pgfpathlineto{\pgfqpoint{4.491590in}{1.359903in}}%
\pgfpathlineto{\pgfqpoint{4.483856in}{1.367568in}}%
\pgfpathlineto{\pgfqpoint{4.476114in}{1.375821in}}%
\pgfpathlineto{\pgfqpoint{4.468367in}{1.384675in}}%
\pgfpathlineto{\pgfqpoint{4.460612in}{1.394143in}}%
\pgfpathlineto{\pgfqpoint{4.446606in}{1.401864in}}%
\pgfpathlineto{\pgfqpoint{4.432605in}{1.409608in}}%
\pgfpathlineto{\pgfqpoint{4.418609in}{1.417375in}}%
\pgfpathlineto{\pgfqpoint{4.404618in}{1.425164in}}%
\pgfpathlineto{\pgfqpoint{4.412406in}{1.415259in}}%
\pgfpathlineto{\pgfqpoint{4.420187in}{1.405971in}}%
\pgfpathlineto{\pgfqpoint{4.427961in}{1.397289in}}%
\pgfpathlineto{\pgfqpoint{4.435727in}{1.389198in}}%
\pgfpathclose%
\pgfusepath{fill}%
\end{pgfscope}%
\begin{pgfscope}%
\pgfpathrectangle{\pgfqpoint{1.254980in}{0.150000in}}{\pgfqpoint{5.490039in}{5.490039in}}%
\pgfusepath{clip}%
\pgfsetbuttcap%
\pgfsetroundjoin%
\definecolor{currentfill}{rgb}{0.280868,0.160771,0.472899}%
\pgfsetfillcolor{currentfill}%
\pgfsetfillopacity{0.700000}%
\pgfsetlinewidth{0.000000pt}%
\definecolor{currentstroke}{rgb}{0.000000,0.000000,0.000000}%
\pgfsetstrokecolor{currentstroke}%
\pgfsetdash{}{0pt}%
\pgfpathmoveto{\pgfqpoint{4.690226in}{1.254132in}}%
\pgfpathlineto{\pgfqpoint{4.704242in}{1.247547in}}%
\pgfpathlineto{\pgfqpoint{4.718263in}{1.240985in}}%
\pgfpathlineto{\pgfqpoint{4.732290in}{1.234444in}}%
\pgfpathlineto{\pgfqpoint{4.746323in}{1.227926in}}%
\pgfpathlineto{\pgfqpoint{4.738717in}{1.231720in}}%
\pgfpathlineto{\pgfqpoint{4.731108in}{1.236039in}}%
\pgfpathlineto{\pgfqpoint{4.723496in}{1.240895in}}%
\pgfpathlineto{\pgfqpoint{4.715882in}{1.246299in}}%
\pgfpathlineto{\pgfqpoint{4.701823in}{1.253233in}}%
\pgfpathlineto{\pgfqpoint{4.687770in}{1.260188in}}%
\pgfpathlineto{\pgfqpoint{4.673723in}{1.267166in}}%
\pgfpathlineto{\pgfqpoint{4.659680in}{1.274166in}}%
\pgfpathlineto{\pgfqpoint{4.667322in}{1.268341in}}%
\pgfpathlineto{\pgfqpoint{4.674960in}{1.263069in}}%
\pgfpathlineto{\pgfqpoint{4.682595in}{1.258336in}}%
\pgfpathlineto{\pgfqpoint{4.690226in}{1.254132in}}%
\pgfpathclose%
\pgfusepath{fill}%
\end{pgfscope}%
\begin{pgfscope}%
\pgfpathrectangle{\pgfqpoint{1.254980in}{0.150000in}}{\pgfqpoint{5.490039in}{5.490039in}}%
\pgfusepath{clip}%
\pgfsetbuttcap%
\pgfsetroundjoin%
\definecolor{currentfill}{rgb}{0.395174,0.797475,0.367757}%
\pgfsetfillcolor{currentfill}%
\pgfsetfillopacity{0.700000}%
\pgfsetlinewidth{0.000000pt}%
\definecolor{currentstroke}{rgb}{0.000000,0.000000,0.000000}%
\pgfsetstrokecolor{currentstroke}%
\pgfsetdash{}{0pt}%
\pgfpathmoveto{\pgfqpoint{2.584700in}{2.933182in}}%
\pgfpathlineto{\pgfqpoint{2.598493in}{2.920582in}}%
\pgfpathlineto{\pgfqpoint{2.612288in}{2.908018in}}%
\pgfpathlineto{\pgfqpoint{2.626083in}{2.895491in}}%
\pgfpathlineto{\pgfqpoint{2.639879in}{2.883000in}}%
\pgfpathlineto{\pgfqpoint{2.630106in}{2.917456in}}%
\pgfpathlineto{\pgfqpoint{2.620283in}{2.952888in}}%
\pgfpathlineto{\pgfqpoint{2.610410in}{2.989316in}}%
\pgfpathlineto{\pgfqpoint{2.596554in}{3.002209in}}%
\pgfpathlineto{\pgfqpoint{2.582700in}{3.015138in}}%
\pgfpathlineto{\pgfqpoint{2.568845in}{3.028103in}}%
\pgfpathlineto{\pgfqpoint{2.554991in}{3.041105in}}%
\pgfpathlineto{\pgfqpoint{2.564945in}{3.004136in}}%
\pgfpathlineto{\pgfqpoint{2.574848in}{2.968167in}}%
\pgfpathlineto{\pgfqpoint{2.584700in}{2.933182in}}%
\pgfpathclose%
\pgfusepath{fill}%
\end{pgfscope}%
\begin{pgfscope}%
\pgfpathrectangle{\pgfqpoint{1.254980in}{0.150000in}}{\pgfqpoint{5.490039in}{5.490039in}}%
\pgfusepath{clip}%
\pgfsetbuttcap%
\pgfsetroundjoin%
\definecolor{currentfill}{rgb}{0.125394,0.574318,0.549086}%
\pgfsetfillcolor{currentfill}%
\pgfsetfillopacity{0.700000}%
\pgfsetlinewidth{0.000000pt}%
\definecolor{currentstroke}{rgb}{0.000000,0.000000,0.000000}%
\pgfsetstrokecolor{currentstroke}%
\pgfsetdash{}{0pt}%
\pgfpathmoveto{\pgfqpoint{3.283859in}{2.259641in}}%
\pgfpathlineto{\pgfqpoint{3.297664in}{2.249061in}}%
\pgfpathlineto{\pgfqpoint{3.311472in}{2.238508in}}%
\pgfpathlineto{\pgfqpoint{3.325282in}{2.227983in}}%
\pgfpathlineto{\pgfqpoint{3.339094in}{2.217486in}}%
\pgfpathlineto{\pgfqpoint{3.330301in}{2.242241in}}%
\pgfpathlineto{\pgfqpoint{3.321477in}{2.267839in}}%
\pgfpathlineto{\pgfqpoint{3.312623in}{2.294296in}}%
\pgfpathlineto{\pgfqpoint{3.303738in}{2.321627in}}%
\pgfpathlineto{\pgfqpoint{3.289863in}{2.332623in}}%
\pgfpathlineto{\pgfqpoint{3.275990in}{2.343646in}}%
\pgfpathlineto{\pgfqpoint{3.262120in}{2.354697in}}%
\pgfpathlineto{\pgfqpoint{3.248252in}{2.365777in}}%
\pgfpathlineto{\pgfqpoint{3.257202in}{2.337939in}}%
\pgfpathlineto{\pgfqpoint{3.266120in}{2.310981in}}%
\pgfpathlineto{\pgfqpoint{3.275005in}{2.284887in}}%
\pgfpathlineto{\pgfqpoint{3.283859in}{2.259641in}}%
\pgfpathclose%
\pgfusepath{fill}%
\end{pgfscope}%
\begin{pgfscope}%
\pgfpathrectangle{\pgfqpoint{1.254980in}{0.150000in}}{\pgfqpoint{5.490039in}{5.490039in}}%
\pgfusepath{clip}%
\pgfsetbuttcap%
\pgfsetroundjoin%
\definecolor{currentfill}{rgb}{0.360741,0.785964,0.387814}%
\pgfsetfillcolor{currentfill}%
\pgfsetfillopacity{0.700000}%
\pgfsetlinewidth{0.000000pt}%
\definecolor{currentstroke}{rgb}{0.000000,0.000000,0.000000}%
\pgfsetstrokecolor{currentstroke}%
\pgfsetdash{}{0pt}%
\pgfpathmoveto{\pgfqpoint{2.639879in}{2.883000in}}%
\pgfpathlineto{\pgfqpoint{2.653676in}{2.870544in}}%
\pgfpathlineto{\pgfqpoint{2.667474in}{2.858124in}}%
\pgfpathlineto{\pgfqpoint{2.681273in}{2.845739in}}%
\pgfpathlineto{\pgfqpoint{2.695073in}{2.833389in}}%
\pgfpathlineto{\pgfqpoint{2.685376in}{2.867317in}}%
\pgfpathlineto{\pgfqpoint{2.675632in}{2.902217in}}%
\pgfpathlineto{\pgfqpoint{2.665839in}{2.938105in}}%
\pgfpathlineto{\pgfqpoint{2.651981in}{2.950854in}}%
\pgfpathlineto{\pgfqpoint{2.638123in}{2.963639in}}%
\pgfpathlineto{\pgfqpoint{2.624266in}{2.976460in}}%
\pgfpathlineto{\pgfqpoint{2.610410in}{2.989316in}}%
\pgfpathlineto{\pgfqpoint{2.620283in}{2.952888in}}%
\pgfpathlineto{\pgfqpoint{2.630106in}{2.917456in}}%
\pgfpathlineto{\pgfqpoint{2.639879in}{2.883000in}}%
\pgfpathclose%
\pgfusepath{fill}%
\end{pgfscope}%
\begin{pgfscope}%
\pgfpathrectangle{\pgfqpoint{1.254980in}{0.150000in}}{\pgfqpoint{5.490039in}{5.490039in}}%
\pgfusepath{clip}%
\pgfsetbuttcap%
\pgfsetroundjoin%
\definecolor{currentfill}{rgb}{0.237441,0.305202,0.541921}%
\pgfsetfillcolor{currentfill}%
\pgfsetfillopacity{0.700000}%
\pgfsetlinewidth{0.000000pt}%
\definecolor{currentstroke}{rgb}{0.000000,0.000000,0.000000}%
\pgfsetstrokecolor{currentstroke}%
\pgfsetdash{}{0pt}%
\pgfpathmoveto{\pgfqpoint{4.181397in}{1.552886in}}%
\pgfpathlineto{\pgfqpoint{4.195314in}{1.544732in}}%
\pgfpathlineto{\pgfqpoint{4.209235in}{1.536600in}}%
\pgfpathlineto{\pgfqpoint{4.223161in}{1.528492in}}%
\pgfpathlineto{\pgfqpoint{4.237091in}{1.520406in}}%
\pgfpathlineto{\pgfqpoint{4.229185in}{1.532269in}}%
\pgfpathlineto{\pgfqpoint{4.221269in}{1.544788in}}%
\pgfpathlineto{\pgfqpoint{4.213341in}{1.557977in}}%
\pgfpathlineto{\pgfqpoint{4.205402in}{1.571848in}}%
\pgfpathlineto{\pgfqpoint{4.191432in}{1.580383in}}%
\pgfpathlineto{\pgfqpoint{4.177467in}{1.588941in}}%
\pgfpathlineto{\pgfqpoint{4.163505in}{1.597522in}}%
\pgfpathlineto{\pgfqpoint{4.149548in}{1.606127in}}%
\pgfpathlineto{\pgfqpoint{4.157528in}{1.591799in}}%
\pgfpathlineto{\pgfqpoint{4.165496in}{1.578159in}}%
\pgfpathlineto{\pgfqpoint{4.173452in}{1.565192in}}%
\pgfpathlineto{\pgfqpoint{4.181397in}{1.552886in}}%
\pgfpathclose%
\pgfusepath{fill}%
\end{pgfscope}%
\begin{pgfscope}%
\pgfpathrectangle{\pgfqpoint{1.254980in}{0.150000in}}{\pgfqpoint{5.490039in}{5.490039in}}%
\pgfusepath{clip}%
\pgfsetbuttcap%
\pgfsetroundjoin%
\definecolor{currentfill}{rgb}{0.187231,0.414746,0.556547}%
\pgfsetfillcolor{currentfill}%
\pgfsetfillopacity{0.700000}%
\pgfsetlinewidth{0.000000pt}%
\definecolor{currentstroke}{rgb}{0.000000,0.000000,0.000000}%
\pgfsetstrokecolor{currentstroke}%
\pgfsetdash{}{0pt}%
\pgfpathmoveto{\pgfqpoint{3.815789in}{1.819706in}}%
\pgfpathlineto{\pgfqpoint{3.829651in}{1.810532in}}%
\pgfpathlineto{\pgfqpoint{3.843517in}{1.801381in}}%
\pgfpathlineto{\pgfqpoint{3.857387in}{1.792255in}}%
\pgfpathlineto{\pgfqpoint{3.871261in}{1.783153in}}%
\pgfpathlineto{\pgfqpoint{3.863044in}{1.800493in}}%
\pgfpathlineto{\pgfqpoint{3.854809in}{1.818571in}}%
\pgfpathlineto{\pgfqpoint{3.846555in}{1.837401in}}%
\pgfpathlineto{\pgfqpoint{3.838283in}{1.856998in}}%
\pgfpathlineto{\pgfqpoint{3.824361in}{1.866571in}}%
\pgfpathlineto{\pgfqpoint{3.810442in}{1.876168in}}%
\pgfpathlineto{\pgfqpoint{3.796527in}{1.885790in}}%
\pgfpathlineto{\pgfqpoint{3.782615in}{1.895436in}}%
\pgfpathlineto{\pgfqpoint{3.790937in}{1.875360in}}%
\pgfpathlineto{\pgfqpoint{3.799240in}{1.856057in}}%
\pgfpathlineto{\pgfqpoint{3.807524in}{1.837510in}}%
\pgfpathlineto{\pgfqpoint{3.815789in}{1.819706in}}%
\pgfpathclose%
\pgfusepath{fill}%
\end{pgfscope}%
\begin{pgfscope}%
\pgfpathrectangle{\pgfqpoint{1.254980in}{0.150000in}}{\pgfqpoint{5.490039in}{5.490039in}}%
\pgfusepath{clip}%
\pgfsetbuttcap%
\pgfsetroundjoin%
\definecolor{currentfill}{rgb}{0.128729,0.563265,0.551229}%
\pgfsetfillcolor{currentfill}%
\pgfsetfillopacity{0.700000}%
\pgfsetlinewidth{0.000000pt}%
\definecolor{currentstroke}{rgb}{0.000000,0.000000,0.000000}%
\pgfsetstrokecolor{currentstroke}%
\pgfsetdash{}{0pt}%
\pgfpathmoveto{\pgfqpoint{3.339094in}{2.217486in}}%
\pgfpathlineto{\pgfqpoint{3.352910in}{2.207015in}}%
\pgfpathlineto{\pgfqpoint{3.366727in}{2.196572in}}%
\pgfpathlineto{\pgfqpoint{3.380548in}{2.186156in}}%
\pgfpathlineto{\pgfqpoint{3.394371in}{2.175767in}}%
\pgfpathlineto{\pgfqpoint{3.385638in}{2.200032in}}%
\pgfpathlineto{\pgfqpoint{3.376875in}{2.225135in}}%
\pgfpathlineto{\pgfqpoint{3.368083in}{2.251092in}}%
\pgfpathlineto{\pgfqpoint{3.359261in}{2.277917in}}%
\pgfpathlineto{\pgfqpoint{3.345376in}{2.288804in}}%
\pgfpathlineto{\pgfqpoint{3.331494in}{2.299717in}}%
\pgfpathlineto{\pgfqpoint{3.317615in}{2.310658in}}%
\pgfpathlineto{\pgfqpoint{3.303738in}{2.321627in}}%
\pgfpathlineto{\pgfqpoint{3.312623in}{2.294296in}}%
\pgfpathlineto{\pgfqpoint{3.321477in}{2.267839in}}%
\pgfpathlineto{\pgfqpoint{3.330301in}{2.242241in}}%
\pgfpathlineto{\pgfqpoint{3.339094in}{2.217486in}}%
\pgfpathclose%
\pgfusepath{fill}%
\end{pgfscope}%
\begin{pgfscope}%
\pgfpathrectangle{\pgfqpoint{1.254980in}{0.150000in}}{\pgfqpoint{5.490039in}{5.490039in}}%
\pgfusepath{clip}%
\pgfsetbuttcap%
\pgfsetroundjoin%
\definecolor{currentfill}{rgb}{0.269308,0.218818,0.509577}%
\pgfsetfillcolor{currentfill}%
\pgfsetfillopacity{0.700000}%
\pgfsetlinewidth{0.000000pt}%
\definecolor{currentstroke}{rgb}{0.000000,0.000000,0.000000}%
\pgfsetstrokecolor{currentstroke}%
\pgfsetdash{}{0pt}%
\pgfpathmoveto{\pgfqpoint{4.491590in}{1.359903in}}%
\pgfpathlineto{\pgfqpoint{4.505569in}{1.352636in}}%
\pgfpathlineto{\pgfqpoint{4.519553in}{1.345390in}}%
\pgfpathlineto{\pgfqpoint{4.533542in}{1.338168in}}%
\pgfpathlineto{\pgfqpoint{4.547536in}{1.330967in}}%
\pgfpathlineto{\pgfqpoint{4.539832in}{1.338207in}}%
\pgfpathlineto{\pgfqpoint{4.532122in}{1.346031in}}%
\pgfpathlineto{\pgfqpoint{4.524406in}{1.354451in}}%
\pgfpathlineto{\pgfqpoint{4.516685in}{1.363481in}}%
\pgfpathlineto{\pgfqpoint{4.502659in}{1.371113in}}%
\pgfpathlineto{\pgfqpoint{4.488639in}{1.378767in}}%
\pgfpathlineto{\pgfqpoint{4.474623in}{1.386444in}}%
\pgfpathlineto{\pgfqpoint{4.460612in}{1.394143in}}%
\pgfpathlineto{\pgfqpoint{4.468367in}{1.384675in}}%
\pgfpathlineto{\pgfqpoint{4.476114in}{1.375821in}}%
\pgfpathlineto{\pgfqpoint{4.483856in}{1.367568in}}%
\pgfpathlineto{\pgfqpoint{4.491590in}{1.359903in}}%
\pgfpathclose%
\pgfusepath{fill}%
\end{pgfscope}%
\begin{pgfscope}%
\pgfpathrectangle{\pgfqpoint{1.254980in}{0.150000in}}{\pgfqpoint{5.490039in}{5.490039in}}%
\pgfusepath{clip}%
\pgfsetbuttcap%
\pgfsetroundjoin%
\definecolor{currentfill}{rgb}{0.319809,0.770914,0.411152}%
\pgfsetfillcolor{currentfill}%
\pgfsetfillopacity{0.700000}%
\pgfsetlinewidth{0.000000pt}%
\definecolor{currentstroke}{rgb}{0.000000,0.000000,0.000000}%
\pgfsetstrokecolor{currentstroke}%
\pgfsetdash{}{0pt}%
\pgfpathmoveto{\pgfqpoint{2.695073in}{2.833389in}}%
\pgfpathlineto{\pgfqpoint{2.708874in}{2.821074in}}%
\pgfpathlineto{\pgfqpoint{2.722676in}{2.808793in}}%
\pgfpathlineto{\pgfqpoint{2.736479in}{2.796547in}}%
\pgfpathlineto{\pgfqpoint{2.750283in}{2.784335in}}%
\pgfpathlineto{\pgfqpoint{2.740662in}{2.817737in}}%
\pgfpathlineto{\pgfqpoint{2.730996in}{2.852105in}}%
\pgfpathlineto{\pgfqpoint{2.721282in}{2.887455in}}%
\pgfpathlineto{\pgfqpoint{2.707420in}{2.900065in}}%
\pgfpathlineto{\pgfqpoint{2.693559in}{2.912710in}}%
\pgfpathlineto{\pgfqpoint{2.679699in}{2.925390in}}%
\pgfpathlineto{\pgfqpoint{2.665839in}{2.938105in}}%
\pgfpathlineto{\pgfqpoint{2.675632in}{2.902217in}}%
\pgfpathlineto{\pgfqpoint{2.685376in}{2.867317in}}%
\pgfpathlineto{\pgfqpoint{2.695073in}{2.833389in}}%
\pgfpathclose%
\pgfusepath{fill}%
\end{pgfscope}%
\begin{pgfscope}%
\pgfpathrectangle{\pgfqpoint{1.254980in}{0.150000in}}{\pgfqpoint{5.490039in}{5.490039in}}%
\pgfusepath{clip}%
\pgfsetbuttcap%
\pgfsetroundjoin%
\definecolor{currentfill}{rgb}{0.281412,0.155834,0.469201}%
\pgfsetfillcolor{currentfill}%
\pgfsetfillopacity{0.700000}%
\pgfsetlinewidth{0.000000pt}%
\definecolor{currentstroke}{rgb}{0.000000,0.000000,0.000000}%
\pgfsetstrokecolor{currentstroke}%
\pgfsetdash{}{0pt}%
\pgfpathmoveto{\pgfqpoint{4.746323in}{1.227926in}}%
\pgfpathlineto{\pgfqpoint{4.760361in}{1.221429in}}%
\pgfpathlineto{\pgfqpoint{4.774406in}{1.214955in}}%
\pgfpathlineto{\pgfqpoint{4.788456in}{1.208503in}}%
\pgfpathlineto{\pgfqpoint{4.802511in}{1.202073in}}%
\pgfpathlineto{\pgfqpoint{4.794930in}{1.205457in}}%
\pgfpathlineto{\pgfqpoint{4.787346in}{1.209363in}}%
\pgfpathlineto{\pgfqpoint{4.779760in}{1.213802in}}%
\pgfpathlineto{\pgfqpoint{4.772171in}{1.218785in}}%
\pgfpathlineto{\pgfqpoint{4.758091in}{1.225631in}}%
\pgfpathlineto{\pgfqpoint{4.744015in}{1.232498in}}%
\pgfpathlineto{\pgfqpoint{4.729946in}{1.239387in}}%
\pgfpathlineto{\pgfqpoint{4.715882in}{1.246299in}}%
\pgfpathlineto{\pgfqpoint{4.723496in}{1.240895in}}%
\pgfpathlineto{\pgfqpoint{4.731108in}{1.236039in}}%
\pgfpathlineto{\pgfqpoint{4.738717in}{1.231720in}}%
\pgfpathlineto{\pgfqpoint{4.746323in}{1.227926in}}%
\pgfpathclose%
\pgfusepath{fill}%
\end{pgfscope}%
\begin{pgfscope}%
\pgfpathrectangle{\pgfqpoint{1.254980in}{0.150000in}}{\pgfqpoint{5.490039in}{5.490039in}}%
\pgfusepath{clip}%
\pgfsetbuttcap%
\pgfsetroundjoin%
\definecolor{currentfill}{rgb}{0.288921,0.758394,0.428426}%
\pgfsetfillcolor{currentfill}%
\pgfsetfillopacity{0.700000}%
\pgfsetlinewidth{0.000000pt}%
\definecolor{currentstroke}{rgb}{0.000000,0.000000,0.000000}%
\pgfsetstrokecolor{currentstroke}%
\pgfsetdash{}{0pt}%
\pgfpathmoveto{\pgfqpoint{2.750283in}{2.784335in}}%
\pgfpathlineto{\pgfqpoint{2.764088in}{2.772157in}}%
\pgfpathlineto{\pgfqpoint{2.777895in}{2.760012in}}%
\pgfpathlineto{\pgfqpoint{2.791702in}{2.747901in}}%
\pgfpathlineto{\pgfqpoint{2.805511in}{2.735823in}}%
\pgfpathlineto{\pgfqpoint{2.795966in}{2.768701in}}%
\pgfpathlineto{\pgfqpoint{2.786376in}{2.802537in}}%
\pgfpathlineto{\pgfqpoint{2.776739in}{2.837351in}}%
\pgfpathlineto{\pgfqpoint{2.762873in}{2.849827in}}%
\pgfpathlineto{\pgfqpoint{2.749009in}{2.862335in}}%
\pgfpathlineto{\pgfqpoint{2.735145in}{2.874878in}}%
\pgfpathlineto{\pgfqpoint{2.721282in}{2.887455in}}%
\pgfpathlineto{\pgfqpoint{2.730996in}{2.852105in}}%
\pgfpathlineto{\pgfqpoint{2.740662in}{2.817737in}}%
\pgfpathlineto{\pgfqpoint{2.750283in}{2.784335in}}%
\pgfpathclose%
\pgfusepath{fill}%
\end{pgfscope}%
\begin{pgfscope}%
\pgfpathrectangle{\pgfqpoint{1.254980in}{0.150000in}}{\pgfqpoint{5.490039in}{5.490039in}}%
\pgfusepath{clip}%
\pgfsetbuttcap%
\pgfsetroundjoin%
\definecolor{currentfill}{rgb}{0.192357,0.403199,0.555836}%
\pgfsetfillcolor{currentfill}%
\pgfsetfillopacity{0.700000}%
\pgfsetlinewidth{0.000000pt}%
\definecolor{currentstroke}{rgb}{0.000000,0.000000,0.000000}%
\pgfsetstrokecolor{currentstroke}%
\pgfsetdash{}{0pt}%
\pgfpathmoveto{\pgfqpoint{3.871261in}{1.783153in}}%
\pgfpathlineto{\pgfqpoint{3.885138in}{1.774076in}}%
\pgfpathlineto{\pgfqpoint{3.899019in}{1.765022in}}%
\pgfpathlineto{\pgfqpoint{3.912904in}{1.755993in}}%
\pgfpathlineto{\pgfqpoint{3.926792in}{1.746987in}}%
\pgfpathlineto{\pgfqpoint{3.918622in}{1.763864in}}%
\pgfpathlineto{\pgfqpoint{3.910435in}{1.781473in}}%
\pgfpathlineto{\pgfqpoint{3.902231in}{1.799831in}}%
\pgfpathlineto{\pgfqpoint{3.894008in}{1.818951in}}%
\pgfpathlineto{\pgfqpoint{3.880072in}{1.828427in}}%
\pgfpathlineto{\pgfqpoint{3.866139in}{1.837926in}}%
\pgfpathlineto{\pgfqpoint{3.852209in}{1.847450in}}%
\pgfpathlineto{\pgfqpoint{3.838283in}{1.856998in}}%
\pgfpathlineto{\pgfqpoint{3.846555in}{1.837401in}}%
\pgfpathlineto{\pgfqpoint{3.854809in}{1.818571in}}%
\pgfpathlineto{\pgfqpoint{3.863044in}{1.800493in}}%
\pgfpathlineto{\pgfqpoint{3.871261in}{1.783153in}}%
\pgfpathclose%
\pgfusepath{fill}%
\end{pgfscope}%
\begin{pgfscope}%
\pgfpathrectangle{\pgfqpoint{1.254980in}{0.150000in}}{\pgfqpoint{5.490039in}{5.490039in}}%
\pgfusepath{clip}%
\pgfsetbuttcap%
\pgfsetroundjoin%
\definecolor{currentfill}{rgb}{0.133743,0.548535,0.553541}%
\pgfsetfillcolor{currentfill}%
\pgfsetfillopacity{0.700000}%
\pgfsetlinewidth{0.000000pt}%
\definecolor{currentstroke}{rgb}{0.000000,0.000000,0.000000}%
\pgfsetstrokecolor{currentstroke}%
\pgfsetdash{}{0pt}%
\pgfpathmoveto{\pgfqpoint{3.394371in}{2.175767in}}%
\pgfpathlineto{\pgfqpoint{3.408197in}{2.165404in}}%
\pgfpathlineto{\pgfqpoint{3.422026in}{2.155069in}}%
\pgfpathlineto{\pgfqpoint{3.435857in}{2.144760in}}%
\pgfpathlineto{\pgfqpoint{3.449692in}{2.134477in}}%
\pgfpathlineto{\pgfqpoint{3.441017in}{2.158254in}}%
\pgfpathlineto{\pgfqpoint{3.432314in}{2.182863in}}%
\pgfpathlineto{\pgfqpoint{3.423583in}{2.208320in}}%
\pgfpathlineto{\pgfqpoint{3.414823in}{2.234641in}}%
\pgfpathlineto{\pgfqpoint{3.400929in}{2.245420in}}%
\pgfpathlineto{\pgfqpoint{3.387037in}{2.256226in}}%
\pgfpathlineto{\pgfqpoint{3.373147in}{2.267058in}}%
\pgfpathlineto{\pgfqpoint{3.359261in}{2.277917in}}%
\pgfpathlineto{\pgfqpoint{3.368083in}{2.251092in}}%
\pgfpathlineto{\pgfqpoint{3.376875in}{2.225135in}}%
\pgfpathlineto{\pgfqpoint{3.385638in}{2.200032in}}%
\pgfpathlineto{\pgfqpoint{3.394371in}{2.175767in}}%
\pgfpathclose%
\pgfusepath{fill}%
\end{pgfscope}%
\begin{pgfscope}%
\pgfpathrectangle{\pgfqpoint{1.254980in}{0.150000in}}{\pgfqpoint{5.490039in}{5.490039in}}%
\pgfusepath{clip}%
\pgfsetbuttcap%
\pgfsetroundjoin%
\definecolor{currentfill}{rgb}{0.241237,0.296485,0.539709}%
\pgfsetfillcolor{currentfill}%
\pgfsetfillopacity{0.700000}%
\pgfsetlinewidth{0.000000pt}%
\definecolor{currentstroke}{rgb}{0.000000,0.000000,0.000000}%
\pgfsetstrokecolor{currentstroke}%
\pgfsetdash{}{0pt}%
\pgfpathmoveto{\pgfqpoint{4.237091in}{1.520406in}}%
\pgfpathlineto{\pgfqpoint{4.251026in}{1.512344in}}%
\pgfpathlineto{\pgfqpoint{4.264966in}{1.504304in}}%
\pgfpathlineto{\pgfqpoint{4.278910in}{1.496288in}}%
\pgfpathlineto{\pgfqpoint{4.292859in}{1.488294in}}%
\pgfpathlineto{\pgfqpoint{4.284990in}{1.499714in}}%
\pgfpathlineto{\pgfqpoint{4.277112in}{1.511786in}}%
\pgfpathlineto{\pgfqpoint{4.269224in}{1.524523in}}%
\pgfpathlineto{\pgfqpoint{4.261325in}{1.537939in}}%
\pgfpathlineto{\pgfqpoint{4.247338in}{1.546382in}}%
\pgfpathlineto{\pgfqpoint{4.233355in}{1.554847in}}%
\pgfpathlineto{\pgfqpoint{4.219376in}{1.563336in}}%
\pgfpathlineto{\pgfqpoint{4.205402in}{1.571848in}}%
\pgfpathlineto{\pgfqpoint{4.213341in}{1.557977in}}%
\pgfpathlineto{\pgfqpoint{4.221269in}{1.544788in}}%
\pgfpathlineto{\pgfqpoint{4.229185in}{1.532269in}}%
\pgfpathlineto{\pgfqpoint{4.237091in}{1.520406in}}%
\pgfpathclose%
\pgfusepath{fill}%
\end{pgfscope}%
\begin{pgfscope}%
\pgfpathrectangle{\pgfqpoint{1.254980in}{0.150000in}}{\pgfqpoint{5.490039in}{5.490039in}}%
\pgfusepath{clip}%
\pgfsetbuttcap%
\pgfsetroundjoin%
\definecolor{currentfill}{rgb}{0.259857,0.745492,0.444467}%
\pgfsetfillcolor{currentfill}%
\pgfsetfillopacity{0.700000}%
\pgfsetlinewidth{0.000000pt}%
\definecolor{currentstroke}{rgb}{0.000000,0.000000,0.000000}%
\pgfsetstrokecolor{currentstroke}%
\pgfsetdash{}{0pt}%
\pgfpathmoveto{\pgfqpoint{2.805511in}{2.735823in}}%
\pgfpathlineto{\pgfqpoint{2.819322in}{2.723778in}}%
\pgfpathlineto{\pgfqpoint{2.833133in}{2.711766in}}%
\pgfpathlineto{\pgfqpoint{2.846946in}{2.699786in}}%
\pgfpathlineto{\pgfqpoint{2.860761in}{2.687839in}}%
\pgfpathlineto{\pgfqpoint{2.851289in}{2.720194in}}%
\pgfpathlineto{\pgfqpoint{2.841774in}{2.753502in}}%
\pgfpathlineto{\pgfqpoint{2.832215in}{2.787781in}}%
\pgfpathlineto{\pgfqpoint{2.818344in}{2.800124in}}%
\pgfpathlineto{\pgfqpoint{2.804475in}{2.812500in}}%
\pgfpathlineto{\pgfqpoint{2.790606in}{2.824909in}}%
\pgfpathlineto{\pgfqpoint{2.776739in}{2.837351in}}%
\pgfpathlineto{\pgfqpoint{2.786376in}{2.802537in}}%
\pgfpathlineto{\pgfqpoint{2.795966in}{2.768701in}}%
\pgfpathlineto{\pgfqpoint{2.805511in}{2.735823in}}%
\pgfpathclose%
\pgfusepath{fill}%
\end{pgfscope}%
\begin{pgfscope}%
\pgfpathrectangle{\pgfqpoint{1.254980in}{0.150000in}}{\pgfqpoint{5.490039in}{5.490039in}}%
\pgfusepath{clip}%
\pgfsetbuttcap%
\pgfsetroundjoin%
\definecolor{currentfill}{rgb}{0.271828,0.209303,0.504434}%
\pgfsetfillcolor{currentfill}%
\pgfsetfillopacity{0.700000}%
\pgfsetlinewidth{0.000000pt}%
\definecolor{currentstroke}{rgb}{0.000000,0.000000,0.000000}%
\pgfsetstrokecolor{currentstroke}%
\pgfsetdash{}{0pt}%
\pgfpathmoveto{\pgfqpoint{4.547536in}{1.330967in}}%
\pgfpathlineto{\pgfqpoint{4.561536in}{1.323789in}}%
\pgfpathlineto{\pgfqpoint{4.575540in}{1.316633in}}%
\pgfpathlineto{\pgfqpoint{4.589550in}{1.309500in}}%
\pgfpathlineto{\pgfqpoint{4.603566in}{1.302389in}}%
\pgfpathlineto{\pgfqpoint{4.595891in}{1.309203in}}%
\pgfpathlineto{\pgfqpoint{4.588212in}{1.316598in}}%
\pgfpathlineto{\pgfqpoint{4.580527in}{1.324585in}}%
\pgfpathlineto{\pgfqpoint{4.572838in}{1.333178in}}%
\pgfpathlineto{\pgfqpoint{4.558792in}{1.340720in}}%
\pgfpathlineto{\pgfqpoint{4.544751in}{1.348285in}}%
\pgfpathlineto{\pgfqpoint{4.530716in}{1.355872in}}%
\pgfpathlineto{\pgfqpoint{4.516685in}{1.363481in}}%
\pgfpathlineto{\pgfqpoint{4.524406in}{1.354451in}}%
\pgfpathlineto{\pgfqpoint{4.532122in}{1.346031in}}%
\pgfpathlineto{\pgfqpoint{4.539832in}{1.338207in}}%
\pgfpathlineto{\pgfqpoint{4.547536in}{1.330967in}}%
\pgfpathclose%
\pgfusepath{fill}%
\end{pgfscope}%
\begin{pgfscope}%
\pgfpathrectangle{\pgfqpoint{1.254980in}{0.150000in}}{\pgfqpoint{5.490039in}{5.490039in}}%
\pgfusepath{clip}%
\pgfsetbuttcap%
\pgfsetroundjoin%
\definecolor{currentfill}{rgb}{0.281887,0.150881,0.465405}%
\pgfsetfillcolor{currentfill}%
\pgfsetfillopacity{0.700000}%
\pgfsetlinewidth{0.000000pt}%
\definecolor{currentstroke}{rgb}{0.000000,0.000000,0.000000}%
\pgfsetstrokecolor{currentstroke}%
\pgfsetdash{}{0pt}%
\pgfpathmoveto{\pgfqpoint{4.802511in}{1.202073in}}%
\pgfpathlineto{\pgfqpoint{4.816573in}{1.195664in}}%
\pgfpathlineto{\pgfqpoint{4.830641in}{1.189278in}}%
\pgfpathlineto{\pgfqpoint{4.844714in}{1.182914in}}%
\pgfpathlineto{\pgfqpoint{4.837150in}{1.185992in}}%
\pgfpathlineto{\pgfqpoint{4.829584in}{1.189588in}}%
\pgfpathlineto{\pgfqpoint{4.822017in}{1.193714in}}%
\pgfpathlineto{\pgfqpoint{4.814447in}{1.198381in}}%
\pgfpathlineto{\pgfqpoint{4.800350in}{1.205161in}}%
\pgfpathlineto{\pgfqpoint{4.786258in}{1.211962in}}%
\pgfpathlineto{\pgfqpoint{4.772171in}{1.218785in}}%
\pgfpathlineto{\pgfqpoint{4.779760in}{1.213802in}}%
\pgfpathlineto{\pgfqpoint{4.787346in}{1.209363in}}%
\pgfpathlineto{\pgfqpoint{4.794930in}{1.205457in}}%
\pgfpathlineto{\pgfqpoint{4.802511in}{1.202073in}}%
\pgfpathclose%
\pgfusepath{fill}%
\end{pgfscope}%
\begin{pgfscope}%
\pgfpathrectangle{\pgfqpoint{1.254980in}{0.150000in}}{\pgfqpoint{5.490039in}{5.490039in}}%
\pgfusepath{clip}%
\pgfsetbuttcap%
\pgfsetroundjoin%
\definecolor{currentfill}{rgb}{0.137770,0.537492,0.554906}%
\pgfsetfillcolor{currentfill}%
\pgfsetfillopacity{0.700000}%
\pgfsetlinewidth{0.000000pt}%
\definecolor{currentstroke}{rgb}{0.000000,0.000000,0.000000}%
\pgfsetstrokecolor{currentstroke}%
\pgfsetdash{}{0pt}%
\pgfpathmoveto{\pgfqpoint{3.449692in}{2.134477in}}%
\pgfpathlineto{\pgfqpoint{3.463529in}{2.124221in}}%
\pgfpathlineto{\pgfqpoint{3.477368in}{2.113992in}}%
\pgfpathlineto{\pgfqpoint{3.491211in}{2.103788in}}%
\pgfpathlineto{\pgfqpoint{3.505057in}{2.093611in}}%
\pgfpathlineto{\pgfqpoint{3.496440in}{2.116900in}}%
\pgfpathlineto{\pgfqpoint{3.487796in}{2.141016in}}%
\pgfpathlineto{\pgfqpoint{3.479125in}{2.165975in}}%
\pgfpathlineto{\pgfqpoint{3.470426in}{2.191793in}}%
\pgfpathlineto{\pgfqpoint{3.456522in}{2.202465in}}%
\pgfpathlineto{\pgfqpoint{3.442619in}{2.213164in}}%
\pgfpathlineto{\pgfqpoint{3.428720in}{2.223889in}}%
\pgfpathlineto{\pgfqpoint{3.414823in}{2.234641in}}%
\pgfpathlineto{\pgfqpoint{3.423583in}{2.208320in}}%
\pgfpathlineto{\pgfqpoint{3.432314in}{2.182863in}}%
\pgfpathlineto{\pgfqpoint{3.441017in}{2.158254in}}%
\pgfpathlineto{\pgfqpoint{3.449692in}{2.134477in}}%
\pgfpathclose%
\pgfusepath{fill}%
\end{pgfscope}%
\begin{pgfscope}%
\pgfpathrectangle{\pgfqpoint{1.254980in}{0.150000in}}{\pgfqpoint{5.490039in}{5.490039in}}%
\pgfusepath{clip}%
\pgfsetbuttcap%
\pgfsetroundjoin%
\definecolor{currentfill}{rgb}{0.226397,0.728888,0.462789}%
\pgfsetfillcolor{currentfill}%
\pgfsetfillopacity{0.700000}%
\pgfsetlinewidth{0.000000pt}%
\definecolor{currentstroke}{rgb}{0.000000,0.000000,0.000000}%
\pgfsetstrokecolor{currentstroke}%
\pgfsetdash{}{0pt}%
\pgfpathmoveto{\pgfqpoint{2.860761in}{2.687839in}}%
\pgfpathlineto{\pgfqpoint{2.874576in}{2.675924in}}%
\pgfpathlineto{\pgfqpoint{2.888394in}{2.664041in}}%
\pgfpathlineto{\pgfqpoint{2.902212in}{2.652190in}}%
\pgfpathlineto{\pgfqpoint{2.916032in}{2.640370in}}%
\pgfpathlineto{\pgfqpoint{2.906634in}{2.672204in}}%
\pgfpathlineto{\pgfqpoint{2.897193in}{2.704985in}}%
\pgfpathlineto{\pgfqpoint{2.887709in}{2.738731in}}%
\pgfpathlineto{\pgfqpoint{2.873834in}{2.750946in}}%
\pgfpathlineto{\pgfqpoint{2.859959in}{2.763192in}}%
\pgfpathlineto{\pgfqpoint{2.846086in}{2.775470in}}%
\pgfpathlineto{\pgfqpoint{2.832215in}{2.787781in}}%
\pgfpathlineto{\pgfqpoint{2.841774in}{2.753502in}}%
\pgfpathlineto{\pgfqpoint{2.851289in}{2.720194in}}%
\pgfpathlineto{\pgfqpoint{2.860761in}{2.687839in}}%
\pgfpathclose%
\pgfusepath{fill}%
\end{pgfscope}%
\begin{pgfscope}%
\pgfpathrectangle{\pgfqpoint{1.254980in}{0.150000in}}{\pgfqpoint{5.490039in}{5.490039in}}%
\pgfusepath{clip}%
\pgfsetbuttcap%
\pgfsetroundjoin%
\definecolor{currentfill}{rgb}{0.195860,0.395433,0.555276}%
\pgfsetfillcolor{currentfill}%
\pgfsetfillopacity{0.700000}%
\pgfsetlinewidth{0.000000pt}%
\definecolor{currentstroke}{rgb}{0.000000,0.000000,0.000000}%
\pgfsetstrokecolor{currentstroke}%
\pgfsetdash{}{0pt}%
\pgfpathmoveto{\pgfqpoint{3.926792in}{1.746987in}}%
\pgfpathlineto{\pgfqpoint{3.940685in}{1.738006in}}%
\pgfpathlineto{\pgfqpoint{3.954581in}{1.729048in}}%
\pgfpathlineto{\pgfqpoint{3.968481in}{1.720115in}}%
\pgfpathlineto{\pgfqpoint{3.982385in}{1.711205in}}%
\pgfpathlineto{\pgfqpoint{3.974261in}{1.727618in}}%
\pgfpathlineto{\pgfqpoint{3.966121in}{1.744761in}}%
\pgfpathlineto{\pgfqpoint{3.957964in}{1.762647in}}%
\pgfpathlineto{\pgfqpoint{3.949791in}{1.781290in}}%
\pgfpathlineto{\pgfqpoint{3.935840in}{1.790669in}}%
\pgfpathlineto{\pgfqpoint{3.921892in}{1.800073in}}%
\pgfpathlineto{\pgfqpoint{3.907949in}{1.809500in}}%
\pgfpathlineto{\pgfqpoint{3.894008in}{1.818951in}}%
\pgfpathlineto{\pgfqpoint{3.902231in}{1.799831in}}%
\pgfpathlineto{\pgfqpoint{3.910435in}{1.781473in}}%
\pgfpathlineto{\pgfqpoint{3.918622in}{1.763864in}}%
\pgfpathlineto{\pgfqpoint{3.926792in}{1.746987in}}%
\pgfpathclose%
\pgfusepath{fill}%
\end{pgfscope}%
\begin{pgfscope}%
\pgfpathrectangle{\pgfqpoint{1.254980in}{0.150000in}}{\pgfqpoint{5.490039in}{5.490039in}}%
\pgfusepath{clip}%
\pgfsetbuttcap%
\pgfsetroundjoin%
\definecolor{currentfill}{rgb}{0.244972,0.287675,0.537260}%
\pgfsetfillcolor{currentfill}%
\pgfsetfillopacity{0.700000}%
\pgfsetlinewidth{0.000000pt}%
\definecolor{currentstroke}{rgb}{0.000000,0.000000,0.000000}%
\pgfsetstrokecolor{currentstroke}%
\pgfsetdash{}{0pt}%
\pgfpathmoveto{\pgfqpoint{4.292859in}{1.488294in}}%
\pgfpathlineto{\pgfqpoint{4.306812in}{1.480323in}}%
\pgfpathlineto{\pgfqpoint{4.320770in}{1.472375in}}%
\pgfpathlineto{\pgfqpoint{4.334733in}{1.464450in}}%
\pgfpathlineto{\pgfqpoint{4.348701in}{1.456547in}}%
\pgfpathlineto{\pgfqpoint{4.340869in}{1.467525in}}%
\pgfpathlineto{\pgfqpoint{4.333028in}{1.479150in}}%
\pgfpathlineto{\pgfqpoint{4.325178in}{1.491436in}}%
\pgfpathlineto{\pgfqpoint{4.317318in}{1.504397in}}%
\pgfpathlineto{\pgfqpoint{4.303313in}{1.512748in}}%
\pgfpathlineto{\pgfqpoint{4.289313in}{1.521122in}}%
\pgfpathlineto{\pgfqpoint{4.275317in}{1.529519in}}%
\pgfpathlineto{\pgfqpoint{4.261325in}{1.537939in}}%
\pgfpathlineto{\pgfqpoint{4.269224in}{1.524523in}}%
\pgfpathlineto{\pgfqpoint{4.277112in}{1.511786in}}%
\pgfpathlineto{\pgfqpoint{4.284990in}{1.499714in}}%
\pgfpathlineto{\pgfqpoint{4.292859in}{1.488294in}}%
\pgfpathclose%
\pgfusepath{fill}%
\end{pgfscope}%
\begin{pgfscope}%
\pgfpathrectangle{\pgfqpoint{1.254980in}{0.150000in}}{\pgfqpoint{5.490039in}{5.490039in}}%
\pgfusepath{clip}%
\pgfsetbuttcap%
\pgfsetroundjoin%
\definecolor{currentfill}{rgb}{0.202219,0.715272,0.476084}%
\pgfsetfillcolor{currentfill}%
\pgfsetfillopacity{0.700000}%
\pgfsetlinewidth{0.000000pt}%
\definecolor{currentstroke}{rgb}{0.000000,0.000000,0.000000}%
\pgfsetstrokecolor{currentstroke}%
\pgfsetdash{}{0pt}%
\pgfpathmoveto{\pgfqpoint{2.916032in}{2.640370in}}%
\pgfpathlineto{\pgfqpoint{2.929854in}{2.628582in}}%
\pgfpathlineto{\pgfqpoint{2.943678in}{2.616826in}}%
\pgfpathlineto{\pgfqpoint{2.957503in}{2.605100in}}%
\pgfpathlineto{\pgfqpoint{2.971329in}{2.593406in}}%
\pgfpathlineto{\pgfqpoint{2.962002in}{2.624719in}}%
\pgfpathlineto{\pgfqpoint{2.952635in}{2.656975in}}%
\pgfpathlineto{\pgfqpoint{2.943226in}{2.690189in}}%
\pgfpathlineto{\pgfqpoint{2.929344in}{2.702278in}}%
\pgfpathlineto{\pgfqpoint{2.915465in}{2.714397in}}%
\pgfpathlineto{\pgfqpoint{2.901586in}{2.726549in}}%
\pgfpathlineto{\pgfqpoint{2.887709in}{2.738731in}}%
\pgfpathlineto{\pgfqpoint{2.897193in}{2.704985in}}%
\pgfpathlineto{\pgfqpoint{2.906634in}{2.672204in}}%
\pgfpathlineto{\pgfqpoint{2.916032in}{2.640370in}}%
\pgfpathclose%
\pgfusepath{fill}%
\end{pgfscope}%
\begin{pgfscope}%
\pgfpathrectangle{\pgfqpoint{1.254980in}{0.150000in}}{\pgfqpoint{5.490039in}{5.490039in}}%
\pgfusepath{clip}%
\pgfsetbuttcap%
\pgfsetroundjoin%
\definecolor{currentfill}{rgb}{0.141935,0.526453,0.555991}%
\pgfsetfillcolor{currentfill}%
\pgfsetfillopacity{0.700000}%
\pgfsetlinewidth{0.000000pt}%
\definecolor{currentstroke}{rgb}{0.000000,0.000000,0.000000}%
\pgfsetstrokecolor{currentstroke}%
\pgfsetdash{}{0pt}%
\pgfpathmoveto{\pgfqpoint{3.505057in}{2.093611in}}%
\pgfpathlineto{\pgfqpoint{3.518905in}{2.083460in}}%
\pgfpathlineto{\pgfqpoint{3.532757in}{2.073335in}}%
\pgfpathlineto{\pgfqpoint{3.546611in}{2.063235in}}%
\pgfpathlineto{\pgfqpoint{3.560469in}{2.053162in}}%
\pgfpathlineto{\pgfqpoint{3.551908in}{2.075964in}}%
\pgfpathlineto{\pgfqpoint{3.543322in}{2.099588in}}%
\pgfpathlineto{\pgfqpoint{3.534711in}{2.124049in}}%
\pgfpathlineto{\pgfqpoint{3.526072in}{2.149365in}}%
\pgfpathlineto{\pgfqpoint{3.512157in}{2.159933in}}%
\pgfpathlineto{\pgfqpoint{3.498244in}{2.170526in}}%
\pgfpathlineto{\pgfqpoint{3.484334in}{2.181146in}}%
\pgfpathlineto{\pgfqpoint{3.470426in}{2.191793in}}%
\pgfpathlineto{\pgfqpoint{3.479125in}{2.165975in}}%
\pgfpathlineto{\pgfqpoint{3.487796in}{2.141016in}}%
\pgfpathlineto{\pgfqpoint{3.496440in}{2.116900in}}%
\pgfpathlineto{\pgfqpoint{3.505057in}{2.093611in}}%
\pgfpathclose%
\pgfusepath{fill}%
\end{pgfscope}%
\begin{pgfscope}%
\pgfpathrectangle{\pgfqpoint{1.254980in}{0.150000in}}{\pgfqpoint{5.490039in}{5.490039in}}%
\pgfusepath{clip}%
\pgfsetbuttcap%
\pgfsetroundjoin%
\definecolor{currentfill}{rgb}{0.273006,0.204520,0.501721}%
\pgfsetfillcolor{currentfill}%
\pgfsetfillopacity{0.700000}%
\pgfsetlinewidth{0.000000pt}%
\definecolor{currentstroke}{rgb}{0.000000,0.000000,0.000000}%
\pgfsetstrokecolor{currentstroke}%
\pgfsetdash{}{0pt}%
\pgfpathmoveto{\pgfqpoint{4.603566in}{1.302389in}}%
\pgfpathlineto{\pgfqpoint{4.617586in}{1.295300in}}%
\pgfpathlineto{\pgfqpoint{4.631612in}{1.288233in}}%
\pgfpathlineto{\pgfqpoint{4.645644in}{1.281189in}}%
\pgfpathlineto{\pgfqpoint{4.659680in}{1.274166in}}%
\pgfpathlineto{\pgfqpoint{4.652035in}{1.280556in}}%
\pgfpathlineto{\pgfqpoint{4.644385in}{1.287521in}}%
\pgfpathlineto{\pgfqpoint{4.636731in}{1.295076in}}%
\pgfpathlineto{\pgfqpoint{4.629072in}{1.303232in}}%
\pgfpathlineto{\pgfqpoint{4.615006in}{1.310685in}}%
\pgfpathlineto{\pgfqpoint{4.600945in}{1.318160in}}%
\pgfpathlineto{\pgfqpoint{4.586889in}{1.325658in}}%
\pgfpathlineto{\pgfqpoint{4.572838in}{1.333178in}}%
\pgfpathlineto{\pgfqpoint{4.580527in}{1.324585in}}%
\pgfpathlineto{\pgfqpoint{4.588212in}{1.316598in}}%
\pgfpathlineto{\pgfqpoint{4.595891in}{1.309203in}}%
\pgfpathlineto{\pgfqpoint{4.603566in}{1.302389in}}%
\pgfpathclose%
\pgfusepath{fill}%
\end{pgfscope}%
\begin{pgfscope}%
\pgfpathrectangle{\pgfqpoint{1.254980in}{0.150000in}}{\pgfqpoint{5.490039in}{5.490039in}}%
\pgfusepath{clip}%
\pgfsetbuttcap%
\pgfsetroundjoin%
\definecolor{currentfill}{rgb}{0.201239,0.383670,0.554294}%
\pgfsetfillcolor{currentfill}%
\pgfsetfillopacity{0.700000}%
\pgfsetlinewidth{0.000000pt}%
\definecolor{currentstroke}{rgb}{0.000000,0.000000,0.000000}%
\pgfsetstrokecolor{currentstroke}%
\pgfsetdash{}{0pt}%
\pgfpathmoveto{\pgfqpoint{3.982385in}{1.711205in}}%
\pgfpathlineto{\pgfqpoint{3.996294in}{1.702319in}}%
\pgfpathlineto{\pgfqpoint{4.010206in}{1.693456in}}%
\pgfpathlineto{\pgfqpoint{4.024122in}{1.684617in}}%
\pgfpathlineto{\pgfqpoint{4.038042in}{1.675802in}}%
\pgfpathlineto{\pgfqpoint{4.029962in}{1.691754in}}%
\pgfpathlineto{\pgfqpoint{4.021868in}{1.708430in}}%
\pgfpathlineto{\pgfqpoint{4.013758in}{1.725844in}}%
\pgfpathlineto{\pgfqpoint{4.005632in}{1.744012in}}%
\pgfpathlineto{\pgfqpoint{3.991666in}{1.753295in}}%
\pgfpathlineto{\pgfqpoint{3.977704in}{1.762603in}}%
\pgfpathlineto{\pgfqpoint{3.963746in}{1.771935in}}%
\pgfpathlineto{\pgfqpoint{3.949791in}{1.781290in}}%
\pgfpathlineto{\pgfqpoint{3.957964in}{1.762647in}}%
\pgfpathlineto{\pgfqpoint{3.966121in}{1.744761in}}%
\pgfpathlineto{\pgfqpoint{3.974261in}{1.727618in}}%
\pgfpathlineto{\pgfqpoint{3.982385in}{1.711205in}}%
\pgfpathclose%
\pgfusepath{fill}%
\end{pgfscope}%
\begin{pgfscope}%
\pgfpathrectangle{\pgfqpoint{1.254980in}{0.150000in}}{\pgfqpoint{5.490039in}{5.490039in}}%
\pgfusepath{clip}%
\pgfsetbuttcap%
\pgfsetroundjoin%
\definecolor{currentfill}{rgb}{0.180653,0.701402,0.488189}%
\pgfsetfillcolor{currentfill}%
\pgfsetfillopacity{0.700000}%
\pgfsetlinewidth{0.000000pt}%
\definecolor{currentstroke}{rgb}{0.000000,0.000000,0.000000}%
\pgfsetstrokecolor{currentstroke}%
\pgfsetdash{}{0pt}%
\pgfpathmoveto{\pgfqpoint{2.971329in}{2.593406in}}%
\pgfpathlineto{\pgfqpoint{2.985158in}{2.581742in}}%
\pgfpathlineto{\pgfqpoint{2.998988in}{2.570109in}}%
\pgfpathlineto{\pgfqpoint{3.012819in}{2.558506in}}%
\pgfpathlineto{\pgfqpoint{3.026653in}{2.546933in}}%
\pgfpathlineto{\pgfqpoint{3.017397in}{2.577729in}}%
\pgfpathlineto{\pgfqpoint{3.008101in}{2.609460in}}%
\pgfpathlineto{\pgfqpoint{2.998765in}{2.642145in}}%
\pgfpathlineto{\pgfqpoint{2.984878in}{2.654110in}}%
\pgfpathlineto{\pgfqpoint{2.970992in}{2.666106in}}%
\pgfpathlineto{\pgfqpoint{2.957108in}{2.678132in}}%
\pgfpathlineto{\pgfqpoint{2.943226in}{2.690189in}}%
\pgfpathlineto{\pgfqpoint{2.952635in}{2.656975in}}%
\pgfpathlineto{\pgfqpoint{2.962002in}{2.624719in}}%
\pgfpathlineto{\pgfqpoint{2.971329in}{2.593406in}}%
\pgfpathclose%
\pgfusepath{fill}%
\end{pgfscope}%
\begin{pgfscope}%
\pgfpathrectangle{\pgfqpoint{1.254980in}{0.150000in}}{\pgfqpoint{5.490039in}{5.490039in}}%
\pgfusepath{clip}%
\pgfsetbuttcap%
\pgfsetroundjoin%
\definecolor{currentfill}{rgb}{0.248629,0.278775,0.534556}%
\pgfsetfillcolor{currentfill}%
\pgfsetfillopacity{0.700000}%
\pgfsetlinewidth{0.000000pt}%
\definecolor{currentstroke}{rgb}{0.000000,0.000000,0.000000}%
\pgfsetstrokecolor{currentstroke}%
\pgfsetdash{}{0pt}%
\pgfpathmoveto{\pgfqpoint{4.348701in}{1.456547in}}%
\pgfpathlineto{\pgfqpoint{4.362673in}{1.448667in}}%
\pgfpathlineto{\pgfqpoint{4.376650in}{1.440810in}}%
\pgfpathlineto{\pgfqpoint{4.390631in}{1.432976in}}%
\pgfpathlineto{\pgfqpoint{4.404618in}{1.425164in}}%
\pgfpathlineto{\pgfqpoint{4.396821in}{1.435700in}}%
\pgfpathlineto{\pgfqpoint{4.389017in}{1.446878in}}%
\pgfpathlineto{\pgfqpoint{4.381204in}{1.458714in}}%
\pgfpathlineto{\pgfqpoint{4.373383in}{1.471220in}}%
\pgfpathlineto{\pgfqpoint{4.359360in}{1.479480in}}%
\pgfpathlineto{\pgfqpoint{4.345342in}{1.487763in}}%
\pgfpathlineto{\pgfqpoint{4.331328in}{1.496068in}}%
\pgfpathlineto{\pgfqpoint{4.317318in}{1.504397in}}%
\pgfpathlineto{\pgfqpoint{4.325178in}{1.491436in}}%
\pgfpathlineto{\pgfqpoint{4.333028in}{1.479150in}}%
\pgfpathlineto{\pgfqpoint{4.340869in}{1.467525in}}%
\pgfpathlineto{\pgfqpoint{4.348701in}{1.456547in}}%
\pgfpathclose%
\pgfusepath{fill}%
\end{pgfscope}%
\begin{pgfscope}%
\pgfpathrectangle{\pgfqpoint{1.254980in}{0.150000in}}{\pgfqpoint{5.490039in}{5.490039in}}%
\pgfusepath{clip}%
\pgfsetbuttcap%
\pgfsetroundjoin%
\definecolor{currentfill}{rgb}{0.146180,0.515413,0.556823}%
\pgfsetfillcolor{currentfill}%
\pgfsetfillopacity{0.700000}%
\pgfsetlinewidth{0.000000pt}%
\definecolor{currentstroke}{rgb}{0.000000,0.000000,0.000000}%
\pgfsetstrokecolor{currentstroke}%
\pgfsetdash{}{0pt}%
\pgfpathmoveto{\pgfqpoint{3.560469in}{2.053162in}}%
\pgfpathlineto{\pgfqpoint{3.574329in}{2.043114in}}%
\pgfpathlineto{\pgfqpoint{3.588193in}{2.033092in}}%
\pgfpathlineto{\pgfqpoint{3.602059in}{2.023096in}}%
\pgfpathlineto{\pgfqpoint{3.615929in}{2.013125in}}%
\pgfpathlineto{\pgfqpoint{3.607424in}{2.035440in}}%
\pgfpathlineto{\pgfqpoint{3.598895in}{2.058573in}}%
\pgfpathlineto{\pgfqpoint{3.590342in}{2.082539in}}%
\pgfpathlineto{\pgfqpoint{3.581763in}{2.107353in}}%
\pgfpathlineto{\pgfqpoint{3.567836in}{2.117817in}}%
\pgfpathlineto{\pgfqpoint{3.553912in}{2.128307in}}%
\pgfpathlineto{\pgfqpoint{3.539991in}{2.138823in}}%
\pgfpathlineto{\pgfqpoint{3.526072in}{2.149365in}}%
\pgfpathlineto{\pgfqpoint{3.534711in}{2.124049in}}%
\pgfpathlineto{\pgfqpoint{3.543322in}{2.099588in}}%
\pgfpathlineto{\pgfqpoint{3.551908in}{2.075964in}}%
\pgfpathlineto{\pgfqpoint{3.560469in}{2.053162in}}%
\pgfpathclose%
\pgfusepath{fill}%
\end{pgfscope}%
\begin{pgfscope}%
\pgfpathrectangle{\pgfqpoint{1.254980in}{0.150000in}}{\pgfqpoint{5.490039in}{5.490039in}}%
\pgfusepath{clip}%
\pgfsetbuttcap%
\pgfsetroundjoin%
\definecolor{currentfill}{rgb}{0.162016,0.687316,0.499129}%
\pgfsetfillcolor{currentfill}%
\pgfsetfillopacity{0.700000}%
\pgfsetlinewidth{0.000000pt}%
\definecolor{currentstroke}{rgb}{0.000000,0.000000,0.000000}%
\pgfsetstrokecolor{currentstroke}%
\pgfsetdash{}{0pt}%
\pgfpathmoveto{\pgfqpoint{3.026653in}{2.546933in}}%
\pgfpathlineto{\pgfqpoint{3.040488in}{2.535391in}}%
\pgfpathlineto{\pgfqpoint{3.054325in}{2.523879in}}%
\pgfpathlineto{\pgfqpoint{3.068164in}{2.512397in}}%
\pgfpathlineto{\pgfqpoint{3.082005in}{2.500944in}}%
\pgfpathlineto{\pgfqpoint{3.072819in}{2.531222in}}%
\pgfpathlineto{\pgfqpoint{3.063594in}{2.562431in}}%
\pgfpathlineto{\pgfqpoint{3.054331in}{2.594587in}}%
\pgfpathlineto{\pgfqpoint{3.040437in}{2.606431in}}%
\pgfpathlineto{\pgfqpoint{3.026545in}{2.618306in}}%
\pgfpathlineto{\pgfqpoint{3.012654in}{2.630210in}}%
\pgfpathlineto{\pgfqpoint{2.998765in}{2.642145in}}%
\pgfpathlineto{\pgfqpoint{3.008101in}{2.609460in}}%
\pgfpathlineto{\pgfqpoint{3.017397in}{2.577729in}}%
\pgfpathlineto{\pgfqpoint{3.026653in}{2.546933in}}%
\pgfpathclose%
\pgfusepath{fill}%
\end{pgfscope}%
\begin{pgfscope}%
\pgfpathrectangle{\pgfqpoint{1.254980in}{0.150000in}}{\pgfqpoint{5.490039in}{5.490039in}}%
\pgfusepath{clip}%
\pgfsetbuttcap%
\pgfsetroundjoin%
\definecolor{currentfill}{rgb}{0.204903,0.375746,0.553533}%
\pgfsetfillcolor{currentfill}%
\pgfsetfillopacity{0.700000}%
\pgfsetlinewidth{0.000000pt}%
\definecolor{currentstroke}{rgb}{0.000000,0.000000,0.000000}%
\pgfsetstrokecolor{currentstroke}%
\pgfsetdash{}{0pt}%
\pgfpathmoveto{\pgfqpoint{4.038042in}{1.675802in}}%
\pgfpathlineto{\pgfqpoint{4.051966in}{1.667011in}}%
\pgfpathlineto{\pgfqpoint{4.065894in}{1.658243in}}%
\pgfpathlineto{\pgfqpoint{4.079826in}{1.649498in}}%
\pgfpathlineto{\pgfqpoint{4.093762in}{1.640777in}}%
\pgfpathlineto{\pgfqpoint{4.085726in}{1.656267in}}%
\pgfpathlineto{\pgfqpoint{4.077677in}{1.672477in}}%
\pgfpathlineto{\pgfqpoint{4.069613in}{1.689421in}}%
\pgfpathlineto{\pgfqpoint{4.061535in}{1.707113in}}%
\pgfpathlineto{\pgfqpoint{4.047553in}{1.716302in}}%
\pgfpathlineto{\pgfqpoint{4.033576in}{1.725515in}}%
\pgfpathlineto{\pgfqpoint{4.019602in}{1.734751in}}%
\pgfpathlineto{\pgfqpoint{4.005632in}{1.744012in}}%
\pgfpathlineto{\pgfqpoint{4.013758in}{1.725844in}}%
\pgfpathlineto{\pgfqpoint{4.021868in}{1.708430in}}%
\pgfpathlineto{\pgfqpoint{4.029962in}{1.691754in}}%
\pgfpathlineto{\pgfqpoint{4.038042in}{1.675802in}}%
\pgfpathclose%
\pgfusepath{fill}%
\end{pgfscope}%
\begin{pgfscope}%
\pgfpathrectangle{\pgfqpoint{1.254980in}{0.150000in}}{\pgfqpoint{5.490039in}{5.490039in}}%
\pgfusepath{clip}%
\pgfsetbuttcap%
\pgfsetroundjoin%
\definecolor{currentfill}{rgb}{0.275191,0.194905,0.496005}%
\pgfsetfillcolor{currentfill}%
\pgfsetfillopacity{0.700000}%
\pgfsetlinewidth{0.000000pt}%
\definecolor{currentstroke}{rgb}{0.000000,0.000000,0.000000}%
\pgfsetstrokecolor{currentstroke}%
\pgfsetdash{}{0pt}%
\pgfpathmoveto{\pgfqpoint{4.659680in}{1.274166in}}%
\pgfpathlineto{\pgfqpoint{4.673723in}{1.267166in}}%
\pgfpathlineto{\pgfqpoint{4.687770in}{1.260188in}}%
\pgfpathlineto{\pgfqpoint{4.701823in}{1.253233in}}%
\pgfpathlineto{\pgfqpoint{4.715882in}{1.246299in}}%
\pgfpathlineto{\pgfqpoint{4.708264in}{1.252263in}}%
\pgfpathlineto{\pgfqpoint{4.700643in}{1.258800in}}%
\pgfpathlineto{\pgfqpoint{4.693018in}{1.265922in}}%
\pgfpathlineto{\pgfqpoint{4.685390in}{1.273641in}}%
\pgfpathlineto{\pgfqpoint{4.671303in}{1.281005in}}%
\pgfpathlineto{\pgfqpoint{4.657221in}{1.288392in}}%
\pgfpathlineto{\pgfqpoint{4.643144in}{1.295801in}}%
\pgfpathlineto{\pgfqpoint{4.629072in}{1.303232in}}%
\pgfpathlineto{\pgfqpoint{4.636731in}{1.295076in}}%
\pgfpathlineto{\pgfqpoint{4.644385in}{1.287521in}}%
\pgfpathlineto{\pgfqpoint{4.652035in}{1.280556in}}%
\pgfpathlineto{\pgfqpoint{4.659680in}{1.274166in}}%
\pgfpathclose%
\pgfusepath{fill}%
\end{pgfscope}%
\begin{pgfscope}%
\pgfpathrectangle{\pgfqpoint{1.254980in}{0.150000in}}{\pgfqpoint{5.490039in}{5.490039in}}%
\pgfusepath{clip}%
\pgfsetbuttcap%
\pgfsetroundjoin%
\definecolor{currentfill}{rgb}{0.151918,0.500685,0.557587}%
\pgfsetfillcolor{currentfill}%
\pgfsetfillopacity{0.700000}%
\pgfsetlinewidth{0.000000pt}%
\definecolor{currentstroke}{rgb}{0.000000,0.000000,0.000000}%
\pgfsetstrokecolor{currentstroke}%
\pgfsetdash{}{0pt}%
\pgfpathmoveto{\pgfqpoint{3.615929in}{2.013125in}}%
\pgfpathlineto{\pgfqpoint{3.629802in}{2.003179in}}%
\pgfpathlineto{\pgfqpoint{3.643678in}{1.993259in}}%
\pgfpathlineto{\pgfqpoint{3.657557in}{1.983364in}}%
\pgfpathlineto{\pgfqpoint{3.671439in}{1.973494in}}%
\pgfpathlineto{\pgfqpoint{3.662989in}{1.995325in}}%
\pgfpathlineto{\pgfqpoint{3.654516in}{2.017967in}}%
\pgfpathlineto{\pgfqpoint{3.646019in}{2.041438in}}%
\pgfpathlineto{\pgfqpoint{3.637499in}{2.065752in}}%
\pgfpathlineto{\pgfqpoint{3.623560in}{2.076114in}}%
\pgfpathlineto{\pgfqpoint{3.609625in}{2.086501in}}%
\pgfpathlineto{\pgfqpoint{3.595692in}{2.096914in}}%
\pgfpathlineto{\pgfqpoint{3.581763in}{2.107353in}}%
\pgfpathlineto{\pgfqpoint{3.590342in}{2.082539in}}%
\pgfpathlineto{\pgfqpoint{3.598895in}{2.058573in}}%
\pgfpathlineto{\pgfqpoint{3.607424in}{2.035440in}}%
\pgfpathlineto{\pgfqpoint{3.615929in}{2.013125in}}%
\pgfpathclose%
\pgfusepath{fill}%
\end{pgfscope}%
\begin{pgfscope}%
\pgfpathrectangle{\pgfqpoint{1.254980in}{0.150000in}}{\pgfqpoint{5.490039in}{5.490039in}}%
\pgfusepath{clip}%
\pgfsetbuttcap%
\pgfsetroundjoin%
\definecolor{currentfill}{rgb}{0.146616,0.673050,0.508936}%
\pgfsetfillcolor{currentfill}%
\pgfsetfillopacity{0.700000}%
\pgfsetlinewidth{0.000000pt}%
\definecolor{currentstroke}{rgb}{0.000000,0.000000,0.000000}%
\pgfsetstrokecolor{currentstroke}%
\pgfsetdash{}{0pt}%
\pgfpathmoveto{\pgfqpoint{3.082005in}{2.500944in}}%
\pgfpathlineto{\pgfqpoint{3.095848in}{2.489521in}}%
\pgfpathlineto{\pgfqpoint{3.109692in}{2.478127in}}%
\pgfpathlineto{\pgfqpoint{3.123539in}{2.466762in}}%
\pgfpathlineto{\pgfqpoint{3.137388in}{2.455427in}}%
\pgfpathlineto{\pgfqpoint{3.128270in}{2.485189in}}%
\pgfpathlineto{\pgfqpoint{3.119116in}{2.515877in}}%
\pgfpathlineto{\pgfqpoint{3.109924in}{2.547506in}}%
\pgfpathlineto{\pgfqpoint{3.096023in}{2.559232in}}%
\pgfpathlineto{\pgfqpoint{3.082124in}{2.570987in}}%
\pgfpathlineto{\pgfqpoint{3.068226in}{2.582772in}}%
\pgfpathlineto{\pgfqpoint{3.054331in}{2.594587in}}%
\pgfpathlineto{\pgfqpoint{3.063594in}{2.562431in}}%
\pgfpathlineto{\pgfqpoint{3.072819in}{2.531222in}}%
\pgfpathlineto{\pgfqpoint{3.082005in}{2.500944in}}%
\pgfpathclose%
\pgfusepath{fill}%
\end{pgfscope}%
\begin{pgfscope}%
\pgfpathrectangle{\pgfqpoint{1.254980in}{0.150000in}}{\pgfqpoint{5.490039in}{5.490039in}}%
\pgfusepath{clip}%
\pgfsetbuttcap%
\pgfsetroundjoin%
\definecolor{currentfill}{rgb}{0.252194,0.269783,0.531579}%
\pgfsetfillcolor{currentfill}%
\pgfsetfillopacity{0.700000}%
\pgfsetlinewidth{0.000000pt}%
\definecolor{currentstroke}{rgb}{0.000000,0.000000,0.000000}%
\pgfsetstrokecolor{currentstroke}%
\pgfsetdash{}{0pt}%
\pgfpathmoveto{\pgfqpoint{4.404618in}{1.425164in}}%
\pgfpathlineto{\pgfqpoint{4.418609in}{1.417375in}}%
\pgfpathlineto{\pgfqpoint{4.432605in}{1.409608in}}%
\pgfpathlineto{\pgfqpoint{4.446606in}{1.401864in}}%
\pgfpathlineto{\pgfqpoint{4.460612in}{1.394143in}}%
\pgfpathlineto{\pgfqpoint{4.452850in}{1.404236in}}%
\pgfpathlineto{\pgfqpoint{4.445082in}{1.414969in}}%
\pgfpathlineto{\pgfqpoint{4.437305in}{1.426354in}}%
\pgfpathlineto{\pgfqpoint{4.429522in}{1.438406in}}%
\pgfpathlineto{\pgfqpoint{4.415480in}{1.446575in}}%
\pgfpathlineto{\pgfqpoint{4.401443in}{1.454767in}}%
\pgfpathlineto{\pgfqpoint{4.387411in}{1.462982in}}%
\pgfpathlineto{\pgfqpoint{4.373383in}{1.471220in}}%
\pgfpathlineto{\pgfqpoint{4.381204in}{1.458714in}}%
\pgfpathlineto{\pgfqpoint{4.389017in}{1.446878in}}%
\pgfpathlineto{\pgfqpoint{4.396821in}{1.435700in}}%
\pgfpathlineto{\pgfqpoint{4.404618in}{1.425164in}}%
\pgfpathclose%
\pgfusepath{fill}%
\end{pgfscope}%
\begin{pgfscope}%
\pgfpathrectangle{\pgfqpoint{1.254980in}{0.150000in}}{\pgfqpoint{5.490039in}{5.490039in}}%
\pgfusepath{clip}%
\pgfsetbuttcap%
\pgfsetroundjoin%
\definecolor{currentfill}{rgb}{0.210503,0.363727,0.552206}%
\pgfsetfillcolor{currentfill}%
\pgfsetfillopacity{0.700000}%
\pgfsetlinewidth{0.000000pt}%
\definecolor{currentstroke}{rgb}{0.000000,0.000000,0.000000}%
\pgfsetstrokecolor{currentstroke}%
\pgfsetdash{}{0pt}%
\pgfpathmoveto{\pgfqpoint{4.093762in}{1.640777in}}%
\pgfpathlineto{\pgfqpoint{4.107702in}{1.632080in}}%
\pgfpathlineto{\pgfqpoint{4.121647in}{1.623405in}}%
\pgfpathlineto{\pgfqpoint{4.135596in}{1.614755in}}%
\pgfpathlineto{\pgfqpoint{4.149548in}{1.606127in}}%
\pgfpathlineto{\pgfqpoint{4.141556in}{1.621156in}}%
\pgfpathlineto{\pgfqpoint{4.133550in}{1.636900in}}%
\pgfpathlineto{\pgfqpoint{4.125531in}{1.653373in}}%
\pgfpathlineto{\pgfqpoint{4.117499in}{1.670591in}}%
\pgfpathlineto{\pgfqpoint{4.103502in}{1.679686in}}%
\pgfpathlineto{\pgfqpoint{4.089509in}{1.688805in}}%
\pgfpathlineto{\pgfqpoint{4.075520in}{1.697947in}}%
\pgfpathlineto{\pgfqpoint{4.061535in}{1.707113in}}%
\pgfpathlineto{\pgfqpoint{4.069613in}{1.689421in}}%
\pgfpathlineto{\pgfqpoint{4.077677in}{1.672477in}}%
\pgfpathlineto{\pgfqpoint{4.085726in}{1.656267in}}%
\pgfpathlineto{\pgfqpoint{4.093762in}{1.640777in}}%
\pgfpathclose%
\pgfusepath{fill}%
\end{pgfscope}%
\begin{pgfscope}%
\pgfpathrectangle{\pgfqpoint{1.254980in}{0.150000in}}{\pgfqpoint{5.490039in}{5.490039in}}%
\pgfusepath{clip}%
\pgfsetbuttcap%
\pgfsetroundjoin%
\definecolor{currentfill}{rgb}{0.137339,0.662252,0.515571}%
\pgfsetfillcolor{currentfill}%
\pgfsetfillopacity{0.700000}%
\pgfsetlinewidth{0.000000pt}%
\definecolor{currentstroke}{rgb}{0.000000,0.000000,0.000000}%
\pgfsetstrokecolor{currentstroke}%
\pgfsetdash{}{0pt}%
\pgfpathmoveto{\pgfqpoint{3.137388in}{2.455427in}}%
\pgfpathlineto{\pgfqpoint{3.151238in}{2.444121in}}%
\pgfpathlineto{\pgfqpoint{3.165091in}{2.432843in}}%
\pgfpathlineto{\pgfqpoint{3.178946in}{2.421594in}}%
\pgfpathlineto{\pgfqpoint{3.192803in}{2.410374in}}%
\pgfpathlineto{\pgfqpoint{3.183753in}{2.439622in}}%
\pgfpathlineto{\pgfqpoint{3.174668in}{2.469789in}}%
\pgfpathlineto{\pgfqpoint{3.165546in}{2.500893in}}%
\pgfpathlineto{\pgfqpoint{3.151638in}{2.512503in}}%
\pgfpathlineto{\pgfqpoint{3.137731in}{2.524141in}}%
\pgfpathlineto{\pgfqpoint{3.123826in}{2.535809in}}%
\pgfpathlineto{\pgfqpoint{3.109924in}{2.547506in}}%
\pgfpathlineto{\pgfqpoint{3.119116in}{2.515877in}}%
\pgfpathlineto{\pgfqpoint{3.128270in}{2.485189in}}%
\pgfpathlineto{\pgfqpoint{3.137388in}{2.455427in}}%
\pgfpathclose%
\pgfusepath{fill}%
\end{pgfscope}%
\begin{pgfscope}%
\pgfpathrectangle{\pgfqpoint{1.254980in}{0.150000in}}{\pgfqpoint{5.490039in}{5.490039in}}%
\pgfusepath{clip}%
\pgfsetbuttcap%
\pgfsetroundjoin%
\definecolor{currentfill}{rgb}{0.156270,0.489624,0.557936}%
\pgfsetfillcolor{currentfill}%
\pgfsetfillopacity{0.700000}%
\pgfsetlinewidth{0.000000pt}%
\definecolor{currentstroke}{rgb}{0.000000,0.000000,0.000000}%
\pgfsetstrokecolor{currentstroke}%
\pgfsetdash{}{0pt}%
\pgfpathmoveto{\pgfqpoint{3.671439in}{1.973494in}}%
\pgfpathlineto{\pgfqpoint{3.685324in}{1.963650in}}%
\pgfpathlineto{\pgfqpoint{3.699213in}{1.953830in}}%
\pgfpathlineto{\pgfqpoint{3.713105in}{1.944036in}}%
\pgfpathlineto{\pgfqpoint{3.727000in}{1.934266in}}%
\pgfpathlineto{\pgfqpoint{3.718604in}{1.955612in}}%
\pgfpathlineto{\pgfqpoint{3.710186in}{1.977765in}}%
\pgfpathlineto{\pgfqpoint{3.701746in}{2.000741in}}%
\pgfpathlineto{\pgfqpoint{3.693282in}{2.024556in}}%
\pgfpathlineto{\pgfqpoint{3.679332in}{2.034817in}}%
\pgfpathlineto{\pgfqpoint{3.665384in}{2.045103in}}%
\pgfpathlineto{\pgfqpoint{3.651440in}{2.055415in}}%
\pgfpathlineto{\pgfqpoint{3.637499in}{2.065752in}}%
\pgfpathlineto{\pgfqpoint{3.646019in}{2.041438in}}%
\pgfpathlineto{\pgfqpoint{3.654516in}{2.017967in}}%
\pgfpathlineto{\pgfqpoint{3.662989in}{1.995325in}}%
\pgfpathlineto{\pgfqpoint{3.671439in}{1.973494in}}%
\pgfpathclose%
\pgfusepath{fill}%
\end{pgfscope}%
\begin{pgfscope}%
\pgfpathrectangle{\pgfqpoint{1.254980in}{0.150000in}}{\pgfqpoint{5.490039in}{5.490039in}}%
\pgfusepath{clip}%
\pgfsetbuttcap%
\pgfsetroundjoin%
\definecolor{currentfill}{rgb}{0.276194,0.190074,0.493001}%
\pgfsetfillcolor{currentfill}%
\pgfsetfillopacity{0.700000}%
\pgfsetlinewidth{0.000000pt}%
\definecolor{currentstroke}{rgb}{0.000000,0.000000,0.000000}%
\pgfsetstrokecolor{currentstroke}%
\pgfsetdash{}{0pt}%
\pgfpathmoveto{\pgfqpoint{4.715882in}{1.246299in}}%
\pgfpathlineto{\pgfqpoint{4.729946in}{1.239387in}}%
\pgfpathlineto{\pgfqpoint{4.744015in}{1.232498in}}%
\pgfpathlineto{\pgfqpoint{4.758091in}{1.225631in}}%
\pgfpathlineto{\pgfqpoint{4.772171in}{1.218785in}}%
\pgfpathlineto{\pgfqpoint{4.764581in}{1.224325in}}%
\pgfpathlineto{\pgfqpoint{4.756987in}{1.230433in}}%
\pgfpathlineto{\pgfqpoint{4.749391in}{1.237122in}}%
\pgfpathlineto{\pgfqpoint{4.741792in}{1.244405in}}%
\pgfpathlineto{\pgfqpoint{4.727684in}{1.251681in}}%
\pgfpathlineto{\pgfqpoint{4.713581in}{1.258979in}}%
\pgfpathlineto{\pgfqpoint{4.699483in}{1.266299in}}%
\pgfpathlineto{\pgfqpoint{4.685390in}{1.273641in}}%
\pgfpathlineto{\pgfqpoint{4.693018in}{1.265922in}}%
\pgfpathlineto{\pgfqpoint{4.700643in}{1.258800in}}%
\pgfpathlineto{\pgfqpoint{4.708264in}{1.252263in}}%
\pgfpathlineto{\pgfqpoint{4.715882in}{1.246299in}}%
\pgfpathclose%
\pgfusepath{fill}%
\end{pgfscope}%
\begin{pgfscope}%
\pgfpathrectangle{\pgfqpoint{1.254980in}{0.150000in}}{\pgfqpoint{5.490039in}{5.490039in}}%
\pgfusepath{clip}%
\pgfsetbuttcap%
\pgfsetroundjoin%
\definecolor{currentfill}{rgb}{0.128087,0.647749,0.523491}%
\pgfsetfillcolor{currentfill}%
\pgfsetfillopacity{0.700000}%
\pgfsetlinewidth{0.000000pt}%
\definecolor{currentstroke}{rgb}{0.000000,0.000000,0.000000}%
\pgfsetstrokecolor{currentstroke}%
\pgfsetdash{}{0pt}%
\pgfpathmoveto{\pgfqpoint{3.192803in}{2.410374in}}%
\pgfpathlineto{\pgfqpoint{3.206662in}{2.399182in}}%
\pgfpathlineto{\pgfqpoint{3.220523in}{2.388019in}}%
\pgfpathlineto{\pgfqpoint{3.234387in}{2.376884in}}%
\pgfpathlineto{\pgfqpoint{3.248252in}{2.365777in}}%
\pgfpathlineto{\pgfqpoint{3.239269in}{2.394511in}}%
\pgfpathlineto{\pgfqpoint{3.230251in}{2.424160in}}%
\pgfpathlineto{\pgfqpoint{3.221199in}{2.454739in}}%
\pgfpathlineto{\pgfqpoint{3.207283in}{2.466235in}}%
\pgfpathlineto{\pgfqpoint{3.193368in}{2.477759in}}%
\pgfpathlineto{\pgfqpoint{3.179456in}{2.489311in}}%
\pgfpathlineto{\pgfqpoint{3.165546in}{2.500893in}}%
\pgfpathlineto{\pgfqpoint{3.174668in}{2.469789in}}%
\pgfpathlineto{\pgfqpoint{3.183753in}{2.439622in}}%
\pgfpathlineto{\pgfqpoint{3.192803in}{2.410374in}}%
\pgfpathclose%
\pgfusepath{fill}%
\end{pgfscope}%
\begin{pgfscope}%
\pgfpathrectangle{\pgfqpoint{1.254980in}{0.150000in}}{\pgfqpoint{5.490039in}{5.490039in}}%
\pgfusepath{clip}%
\pgfsetbuttcap%
\pgfsetroundjoin%
\definecolor{currentfill}{rgb}{0.255645,0.260703,0.528312}%
\pgfsetfillcolor{currentfill}%
\pgfsetfillopacity{0.700000}%
\pgfsetlinewidth{0.000000pt}%
\definecolor{currentstroke}{rgb}{0.000000,0.000000,0.000000}%
\pgfsetstrokecolor{currentstroke}%
\pgfsetdash{}{0pt}%
\pgfpathmoveto{\pgfqpoint{4.460612in}{1.394143in}}%
\pgfpathlineto{\pgfqpoint{4.474623in}{1.386444in}}%
\pgfpathlineto{\pgfqpoint{4.488639in}{1.378767in}}%
\pgfpathlineto{\pgfqpoint{4.502659in}{1.371113in}}%
\pgfpathlineto{\pgfqpoint{4.516685in}{1.363481in}}%
\pgfpathlineto{\pgfqpoint{4.508957in}{1.373133in}}%
\pgfpathlineto{\pgfqpoint{4.501223in}{1.383420in}}%
\pgfpathlineto{\pgfqpoint{4.493482in}{1.394356in}}%
\pgfpathlineto{\pgfqpoint{4.485735in}{1.405953in}}%
\pgfpathlineto{\pgfqpoint{4.471674in}{1.414032in}}%
\pgfpathlineto{\pgfqpoint{4.457619in}{1.422134in}}%
\pgfpathlineto{\pgfqpoint{4.443568in}{1.430259in}}%
\pgfpathlineto{\pgfqpoint{4.429522in}{1.438406in}}%
\pgfpathlineto{\pgfqpoint{4.437305in}{1.426354in}}%
\pgfpathlineto{\pgfqpoint{4.445082in}{1.414969in}}%
\pgfpathlineto{\pgfqpoint{4.452850in}{1.404236in}}%
\pgfpathlineto{\pgfqpoint{4.460612in}{1.394143in}}%
\pgfpathclose%
\pgfusepath{fill}%
\end{pgfscope}%
\begin{pgfscope}%
\pgfpathrectangle{\pgfqpoint{1.254980in}{0.150000in}}{\pgfqpoint{5.490039in}{5.490039in}}%
\pgfusepath{clip}%
\pgfsetbuttcap%
\pgfsetroundjoin%
\definecolor{currentfill}{rgb}{0.160665,0.478540,0.558115}%
\pgfsetfillcolor{currentfill}%
\pgfsetfillopacity{0.700000}%
\pgfsetlinewidth{0.000000pt}%
\definecolor{currentstroke}{rgb}{0.000000,0.000000,0.000000}%
\pgfsetstrokecolor{currentstroke}%
\pgfsetdash{}{0pt}%
\pgfpathmoveto{\pgfqpoint{3.727000in}{1.934266in}}%
\pgfpathlineto{\pgfqpoint{3.740899in}{1.924521in}}%
\pgfpathlineto{\pgfqpoint{3.754801in}{1.914801in}}%
\pgfpathlineto{\pgfqpoint{3.768706in}{1.905106in}}%
\pgfpathlineto{\pgfqpoint{3.782615in}{1.895436in}}%
\pgfpathlineto{\pgfqpoint{3.774271in}{1.916298in}}%
\pgfpathlineto{\pgfqpoint{3.765907in}{1.937963in}}%
\pgfpathlineto{\pgfqpoint{3.757522in}{1.960445in}}%
\pgfpathlineto{\pgfqpoint{3.749115in}{1.983761in}}%
\pgfpathlineto{\pgfqpoint{3.735152in}{1.993922in}}%
\pgfpathlineto{\pgfqpoint{3.721192in}{2.004108in}}%
\pgfpathlineto{\pgfqpoint{3.707236in}{2.014320in}}%
\pgfpathlineto{\pgfqpoint{3.693282in}{2.024556in}}%
\pgfpathlineto{\pgfqpoint{3.701746in}{2.000741in}}%
\pgfpathlineto{\pgfqpoint{3.710186in}{1.977765in}}%
\pgfpathlineto{\pgfqpoint{3.718604in}{1.955612in}}%
\pgfpathlineto{\pgfqpoint{3.727000in}{1.934266in}}%
\pgfpathclose%
\pgfusepath{fill}%
\end{pgfscope}%
\begin{pgfscope}%
\pgfpathrectangle{\pgfqpoint{1.254980in}{0.150000in}}{\pgfqpoint{5.490039in}{5.490039in}}%
\pgfusepath{clip}%
\pgfsetbuttcap%
\pgfsetroundjoin%
\definecolor{currentfill}{rgb}{0.214298,0.355619,0.551184}%
\pgfsetfillcolor{currentfill}%
\pgfsetfillopacity{0.700000}%
\pgfsetlinewidth{0.000000pt}%
\definecolor{currentstroke}{rgb}{0.000000,0.000000,0.000000}%
\pgfsetstrokecolor{currentstroke}%
\pgfsetdash{}{0pt}%
\pgfpathmoveto{\pgfqpoint{4.149548in}{1.606127in}}%
\pgfpathlineto{\pgfqpoint{4.163505in}{1.597522in}}%
\pgfpathlineto{\pgfqpoint{4.177467in}{1.588941in}}%
\pgfpathlineto{\pgfqpoint{4.191432in}{1.580383in}}%
\pgfpathlineto{\pgfqpoint{4.205402in}{1.571848in}}%
\pgfpathlineto{\pgfqpoint{4.197452in}{1.586416in}}%
\pgfpathlineto{\pgfqpoint{4.189489in}{1.601695in}}%
\pgfpathlineto{\pgfqpoint{4.181514in}{1.617699in}}%
\pgfpathlineto{\pgfqpoint{4.173527in}{1.634443in}}%
\pgfpathlineto{\pgfqpoint{4.159514in}{1.643445in}}%
\pgfpathlineto{\pgfqpoint{4.145505in}{1.652470in}}%
\pgfpathlineto{\pgfqpoint{4.131500in}{1.661519in}}%
\pgfpathlineto{\pgfqpoint{4.117499in}{1.670591in}}%
\pgfpathlineto{\pgfqpoint{4.125531in}{1.653373in}}%
\pgfpathlineto{\pgfqpoint{4.133550in}{1.636900in}}%
\pgfpathlineto{\pgfqpoint{4.141556in}{1.621156in}}%
\pgfpathlineto{\pgfqpoint{4.149548in}{1.606127in}}%
\pgfpathclose%
\pgfusepath{fill}%
\end{pgfscope}%
\begin{pgfscope}%
\pgfpathrectangle{\pgfqpoint{1.254980in}{0.150000in}}{\pgfqpoint{5.490039in}{5.490039in}}%
\pgfusepath{clip}%
\pgfsetbuttcap%
\pgfsetroundjoin%
\definecolor{currentfill}{rgb}{0.122312,0.633153,0.530398}%
\pgfsetfillcolor{currentfill}%
\pgfsetfillopacity{0.700000}%
\pgfsetlinewidth{0.000000pt}%
\definecolor{currentstroke}{rgb}{0.000000,0.000000,0.000000}%
\pgfsetstrokecolor{currentstroke}%
\pgfsetdash{}{0pt}%
\pgfpathmoveto{\pgfqpoint{3.248252in}{2.365777in}}%
\pgfpathlineto{\pgfqpoint{3.262120in}{2.354697in}}%
\pgfpathlineto{\pgfqpoint{3.275990in}{2.343646in}}%
\pgfpathlineto{\pgfqpoint{3.289863in}{2.332623in}}%
\pgfpathlineto{\pgfqpoint{3.303738in}{2.321627in}}%
\pgfpathlineto{\pgfqpoint{3.294820in}{2.349849in}}%
\pgfpathlineto{\pgfqpoint{3.285869in}{2.378980in}}%
\pgfpathlineto{\pgfqpoint{3.276885in}{2.409036in}}%
\pgfpathlineto{\pgfqpoint{3.262960in}{2.420420in}}%
\pgfpathlineto{\pgfqpoint{3.249038in}{2.431832in}}%
\pgfpathlineto{\pgfqpoint{3.235117in}{2.443271in}}%
\pgfpathlineto{\pgfqpoint{3.221199in}{2.454739in}}%
\pgfpathlineto{\pgfqpoint{3.230251in}{2.424160in}}%
\pgfpathlineto{\pgfqpoint{3.239269in}{2.394511in}}%
\pgfpathlineto{\pgfqpoint{3.248252in}{2.365777in}}%
\pgfpathclose%
\pgfusepath{fill}%
\end{pgfscope}%
\begin{pgfscope}%
\pgfpathrectangle{\pgfqpoint{1.254980in}{0.150000in}}{\pgfqpoint{5.490039in}{5.490039in}}%
\pgfusepath{clip}%
\pgfsetbuttcap%
\pgfsetroundjoin%
\definecolor{currentfill}{rgb}{0.277134,0.185228,0.489898}%
\pgfsetfillcolor{currentfill}%
\pgfsetfillopacity{0.700000}%
\pgfsetlinewidth{0.000000pt}%
\definecolor{currentstroke}{rgb}{0.000000,0.000000,0.000000}%
\pgfsetstrokecolor{currentstroke}%
\pgfsetdash{}{0pt}%
\pgfpathmoveto{\pgfqpoint{4.772171in}{1.218785in}}%
\pgfpathlineto{\pgfqpoint{4.786258in}{1.211962in}}%
\pgfpathlineto{\pgfqpoint{4.800350in}{1.205161in}}%
\pgfpathlineto{\pgfqpoint{4.814447in}{1.198381in}}%
\pgfpathlineto{\pgfqpoint{4.806876in}{1.203602in}}%
\pgfpathlineto{\pgfqpoint{4.799303in}{1.209389in}}%
\pgfpathlineto{\pgfqpoint{4.791728in}{1.215754in}}%
\pgfpathlineto{\pgfqpoint{4.784151in}{1.222709in}}%
\pgfpathlineto{\pgfqpoint{4.770026in}{1.229919in}}%
\pgfpathlineto{\pgfqpoint{4.755906in}{1.237151in}}%
\pgfpathlineto{\pgfqpoint{4.741792in}{1.244405in}}%
\pgfpathlineto{\pgfqpoint{4.749391in}{1.237122in}}%
\pgfpathlineto{\pgfqpoint{4.756987in}{1.230433in}}%
\pgfpathlineto{\pgfqpoint{4.764581in}{1.224325in}}%
\pgfpathlineto{\pgfqpoint{4.772171in}{1.218785in}}%
\pgfpathclose%
\pgfusepath{fill}%
\end{pgfscope}%
\begin{pgfscope}%
\pgfpathrectangle{\pgfqpoint{1.254980in}{0.150000in}}{\pgfqpoint{5.490039in}{5.490039in}}%
\pgfusepath{clip}%
\pgfsetbuttcap%
\pgfsetroundjoin%
\definecolor{currentfill}{rgb}{0.165117,0.467423,0.558141}%
\pgfsetfillcolor{currentfill}%
\pgfsetfillopacity{0.700000}%
\pgfsetlinewidth{0.000000pt}%
\definecolor{currentstroke}{rgb}{0.000000,0.000000,0.000000}%
\pgfsetstrokecolor{currentstroke}%
\pgfsetdash{}{0pt}%
\pgfpathmoveto{\pgfqpoint{3.782615in}{1.895436in}}%
\pgfpathlineto{\pgfqpoint{3.796527in}{1.885790in}}%
\pgfpathlineto{\pgfqpoint{3.810442in}{1.876168in}}%
\pgfpathlineto{\pgfqpoint{3.824361in}{1.866571in}}%
\pgfpathlineto{\pgfqpoint{3.838283in}{1.856998in}}%
\pgfpathlineto{\pgfqpoint{3.829992in}{1.877378in}}%
\pgfpathlineto{\pgfqpoint{3.821681in}{1.898555in}}%
\pgfpathlineto{\pgfqpoint{3.813350in}{1.920545in}}%
\pgfpathlineto{\pgfqpoint{3.804998in}{1.943363in}}%
\pgfpathlineto{\pgfqpoint{3.791022in}{1.953425in}}%
\pgfpathlineto{\pgfqpoint{3.777050in}{1.963513in}}%
\pgfpathlineto{\pgfqpoint{3.763081in}{1.973624in}}%
\pgfpathlineto{\pgfqpoint{3.749115in}{1.983761in}}%
\pgfpathlineto{\pgfqpoint{3.757522in}{1.960445in}}%
\pgfpathlineto{\pgfqpoint{3.765907in}{1.937963in}}%
\pgfpathlineto{\pgfqpoint{3.774271in}{1.916298in}}%
\pgfpathlineto{\pgfqpoint{3.782615in}{1.895436in}}%
\pgfpathclose%
\pgfusepath{fill}%
\end{pgfscope}%
\begin{pgfscope}%
\pgfpathrectangle{\pgfqpoint{1.254980in}{0.150000in}}{\pgfqpoint{5.490039in}{5.490039in}}%
\pgfusepath{clip}%
\pgfsetbuttcap%
\pgfsetroundjoin%
\definecolor{currentfill}{rgb}{0.258965,0.251537,0.524736}%
\pgfsetfillcolor{currentfill}%
\pgfsetfillopacity{0.700000}%
\pgfsetlinewidth{0.000000pt}%
\definecolor{currentstroke}{rgb}{0.000000,0.000000,0.000000}%
\pgfsetstrokecolor{currentstroke}%
\pgfsetdash{}{0pt}%
\pgfpathmoveto{\pgfqpoint{4.516685in}{1.363481in}}%
\pgfpathlineto{\pgfqpoint{4.530716in}{1.355872in}}%
\pgfpathlineto{\pgfqpoint{4.544751in}{1.348285in}}%
\pgfpathlineto{\pgfqpoint{4.558792in}{1.340720in}}%
\pgfpathlineto{\pgfqpoint{4.572838in}{1.333178in}}%
\pgfpathlineto{\pgfqpoint{4.565143in}{1.342389in}}%
\pgfpathlineto{\pgfqpoint{4.557443in}{1.352230in}}%
\pgfpathlineto{\pgfqpoint{4.549737in}{1.362716in}}%
\pgfpathlineto{\pgfqpoint{4.542025in}{1.373860in}}%
\pgfpathlineto{\pgfqpoint{4.527945in}{1.381849in}}%
\pgfpathlineto{\pgfqpoint{4.513870in}{1.389861in}}%
\pgfpathlineto{\pgfqpoint{4.499800in}{1.397896in}}%
\pgfpathlineto{\pgfqpoint{4.485735in}{1.405953in}}%
\pgfpathlineto{\pgfqpoint{4.493482in}{1.394356in}}%
\pgfpathlineto{\pgfqpoint{4.501223in}{1.383420in}}%
\pgfpathlineto{\pgfqpoint{4.508957in}{1.373133in}}%
\pgfpathlineto{\pgfqpoint{4.516685in}{1.363481in}}%
\pgfpathclose%
\pgfusepath{fill}%
\end{pgfscope}%
\begin{pgfscope}%
\pgfpathrectangle{\pgfqpoint{1.254980in}{0.150000in}}{\pgfqpoint{5.490039in}{5.490039in}}%
\pgfusepath{clip}%
\pgfsetbuttcap%
\pgfsetroundjoin%
\definecolor{currentfill}{rgb}{0.119699,0.618490,0.536347}%
\pgfsetfillcolor{currentfill}%
\pgfsetfillopacity{0.700000}%
\pgfsetlinewidth{0.000000pt}%
\definecolor{currentstroke}{rgb}{0.000000,0.000000,0.000000}%
\pgfsetstrokecolor{currentstroke}%
\pgfsetdash{}{0pt}%
\pgfpathmoveto{\pgfqpoint{3.303738in}{2.321627in}}%
\pgfpathlineto{\pgfqpoint{3.317615in}{2.310658in}}%
\pgfpathlineto{\pgfqpoint{3.331494in}{2.299717in}}%
\pgfpathlineto{\pgfqpoint{3.345376in}{2.288804in}}%
\pgfpathlineto{\pgfqpoint{3.359261in}{2.277917in}}%
\pgfpathlineto{\pgfqpoint{3.350407in}{2.305629in}}%
\pgfpathlineto{\pgfqpoint{3.341523in}{2.334243in}}%
\pgfpathlineto{\pgfqpoint{3.332606in}{2.363778in}}%
\pgfpathlineto{\pgfqpoint{3.318672in}{2.375051in}}%
\pgfpathlineto{\pgfqpoint{3.304741in}{2.386352in}}%
\pgfpathlineto{\pgfqpoint{3.290812in}{2.397680in}}%
\pgfpathlineto{\pgfqpoint{3.276885in}{2.409036in}}%
\pgfpathlineto{\pgfqpoint{3.285869in}{2.378980in}}%
\pgfpathlineto{\pgfqpoint{3.294820in}{2.349849in}}%
\pgfpathlineto{\pgfqpoint{3.303738in}{2.321627in}}%
\pgfpathclose%
\pgfusepath{fill}%
\end{pgfscope}%
\begin{pgfscope}%
\pgfpathrectangle{\pgfqpoint{1.254980in}{0.150000in}}{\pgfqpoint{5.490039in}{5.490039in}}%
\pgfusepath{clip}%
\pgfsetbuttcap%
\pgfsetroundjoin%
\definecolor{currentfill}{rgb}{0.220057,0.343307,0.549413}%
\pgfsetfillcolor{currentfill}%
\pgfsetfillopacity{0.700000}%
\pgfsetlinewidth{0.000000pt}%
\definecolor{currentstroke}{rgb}{0.000000,0.000000,0.000000}%
\pgfsetstrokecolor{currentstroke}%
\pgfsetdash{}{0pt}%
\pgfpathmoveto{\pgfqpoint{4.205402in}{1.571848in}}%
\pgfpathlineto{\pgfqpoint{4.219376in}{1.563336in}}%
\pgfpathlineto{\pgfqpoint{4.233355in}{1.554847in}}%
\pgfpathlineto{\pgfqpoint{4.247338in}{1.546382in}}%
\pgfpathlineto{\pgfqpoint{4.261325in}{1.537939in}}%
\pgfpathlineto{\pgfqpoint{4.253415in}{1.552047in}}%
\pgfpathlineto{\pgfqpoint{4.245495in}{1.566861in}}%
\pgfpathlineto{\pgfqpoint{4.237563in}{1.582396in}}%
\pgfpathlineto{\pgfqpoint{4.229620in}{1.598666in}}%
\pgfpathlineto{\pgfqpoint{4.215590in}{1.607576in}}%
\pgfpathlineto{\pgfqpoint{4.201565in}{1.616508in}}%
\pgfpathlineto{\pgfqpoint{4.187544in}{1.625464in}}%
\pgfpathlineto{\pgfqpoint{4.173527in}{1.634443in}}%
\pgfpathlineto{\pgfqpoint{4.181514in}{1.617699in}}%
\pgfpathlineto{\pgfqpoint{4.189489in}{1.601695in}}%
\pgfpathlineto{\pgfqpoint{4.197452in}{1.586416in}}%
\pgfpathlineto{\pgfqpoint{4.205402in}{1.571848in}}%
\pgfpathclose%
\pgfusepath{fill}%
\end{pgfscope}%
\begin{pgfscope}%
\pgfpathrectangle{\pgfqpoint{1.254980in}{0.150000in}}{\pgfqpoint{5.490039in}{5.490039in}}%
\pgfusepath{clip}%
\pgfsetbuttcap%
\pgfsetroundjoin%
\definecolor{currentfill}{rgb}{0.119512,0.607464,0.540218}%
\pgfsetfillcolor{currentfill}%
\pgfsetfillopacity{0.700000}%
\pgfsetlinewidth{0.000000pt}%
\definecolor{currentstroke}{rgb}{0.000000,0.000000,0.000000}%
\pgfsetstrokecolor{currentstroke}%
\pgfsetdash{}{0pt}%
\pgfpathmoveto{\pgfqpoint{3.359261in}{2.277917in}}%
\pgfpathlineto{\pgfqpoint{3.373147in}{2.267058in}}%
\pgfpathlineto{\pgfqpoint{3.387037in}{2.256226in}}%
\pgfpathlineto{\pgfqpoint{3.400929in}{2.245420in}}%
\pgfpathlineto{\pgfqpoint{3.414823in}{2.234641in}}%
\pgfpathlineto{\pgfqpoint{3.406033in}{2.261843in}}%
\pgfpathlineto{\pgfqpoint{3.397213in}{2.289943in}}%
\pgfpathlineto{\pgfqpoint{3.388362in}{2.318956in}}%
\pgfpathlineto{\pgfqpoint{3.374420in}{2.330121in}}%
\pgfpathlineto{\pgfqpoint{3.360479in}{2.341313in}}%
\pgfpathlineto{\pgfqpoint{3.346541in}{2.352532in}}%
\pgfpathlineto{\pgfqpoint{3.332606in}{2.363778in}}%
\pgfpathlineto{\pgfqpoint{3.341523in}{2.334243in}}%
\pgfpathlineto{\pgfqpoint{3.350407in}{2.305629in}}%
\pgfpathlineto{\pgfqpoint{3.359261in}{2.277917in}}%
\pgfpathclose%
\pgfusepath{fill}%
\end{pgfscope}%
\begin{pgfscope}%
\pgfpathrectangle{\pgfqpoint{1.254980in}{0.150000in}}{\pgfqpoint{5.490039in}{5.490039in}}%
\pgfusepath{clip}%
\pgfsetbuttcap%
\pgfsetroundjoin%
\definecolor{currentfill}{rgb}{0.168126,0.459988,0.558082}%
\pgfsetfillcolor{currentfill}%
\pgfsetfillopacity{0.700000}%
\pgfsetlinewidth{0.000000pt}%
\definecolor{currentstroke}{rgb}{0.000000,0.000000,0.000000}%
\pgfsetstrokecolor{currentstroke}%
\pgfsetdash{}{0pt}%
\pgfpathmoveto{\pgfqpoint{3.838283in}{1.856998in}}%
\pgfpathlineto{\pgfqpoint{3.852209in}{1.847450in}}%
\pgfpathlineto{\pgfqpoint{3.866139in}{1.837926in}}%
\pgfpathlineto{\pgfqpoint{3.880072in}{1.828427in}}%
\pgfpathlineto{\pgfqpoint{3.894008in}{1.818951in}}%
\pgfpathlineto{\pgfqpoint{3.885768in}{1.838849in}}%
\pgfpathlineto{\pgfqpoint{3.877509in}{1.859539in}}%
\pgfpathlineto{\pgfqpoint{3.869231in}{1.881036in}}%
\pgfpathlineto{\pgfqpoint{3.860933in}{1.903358in}}%
\pgfpathlineto{\pgfqpoint{3.846944in}{1.913323in}}%
\pgfpathlineto{\pgfqpoint{3.832959in}{1.923312in}}%
\pgfpathlineto{\pgfqpoint{3.818977in}{1.933325in}}%
\pgfpathlineto{\pgfqpoint{3.804998in}{1.943363in}}%
\pgfpathlineto{\pgfqpoint{3.813350in}{1.920545in}}%
\pgfpathlineto{\pgfqpoint{3.821681in}{1.898555in}}%
\pgfpathlineto{\pgfqpoint{3.829992in}{1.877378in}}%
\pgfpathlineto{\pgfqpoint{3.838283in}{1.856998in}}%
\pgfpathclose%
\pgfusepath{fill}%
\end{pgfscope}%
\begin{pgfscope}%
\pgfpathrectangle{\pgfqpoint{1.254980in}{0.150000in}}{\pgfqpoint{5.490039in}{5.490039in}}%
\pgfusepath{clip}%
\pgfsetbuttcap%
\pgfsetroundjoin%
\definecolor{currentfill}{rgb}{0.260571,0.246922,0.522828}%
\pgfsetfillcolor{currentfill}%
\pgfsetfillopacity{0.700000}%
\pgfsetlinewidth{0.000000pt}%
\definecolor{currentstroke}{rgb}{0.000000,0.000000,0.000000}%
\pgfsetstrokecolor{currentstroke}%
\pgfsetdash{}{0pt}%
\pgfpathmoveto{\pgfqpoint{4.572838in}{1.333178in}}%
\pgfpathlineto{\pgfqpoint{4.586889in}{1.325658in}}%
\pgfpathlineto{\pgfqpoint{4.600945in}{1.318160in}}%
\pgfpathlineto{\pgfqpoint{4.615006in}{1.310685in}}%
\pgfpathlineto{\pgfqpoint{4.629072in}{1.303232in}}%
\pgfpathlineto{\pgfqpoint{4.621410in}{1.312001in}}%
\pgfpathlineto{\pgfqpoint{4.613742in}{1.321398in}}%
\pgfpathlineto{\pgfqpoint{4.606070in}{1.331435in}}%
\pgfpathlineto{\pgfqpoint{4.598392in}{1.342125in}}%
\pgfpathlineto{\pgfqpoint{4.584293in}{1.350025in}}%
\pgfpathlineto{\pgfqpoint{4.570199in}{1.357947in}}%
\pgfpathlineto{\pgfqpoint{4.556109in}{1.365892in}}%
\pgfpathlineto{\pgfqpoint{4.542025in}{1.373860in}}%
\pgfpathlineto{\pgfqpoint{4.549737in}{1.362716in}}%
\pgfpathlineto{\pgfqpoint{4.557443in}{1.352230in}}%
\pgfpathlineto{\pgfqpoint{4.565143in}{1.342389in}}%
\pgfpathlineto{\pgfqpoint{4.572838in}{1.333178in}}%
\pgfpathclose%
\pgfusepath{fill}%
\end{pgfscope}%
\begin{pgfscope}%
\pgfpathrectangle{\pgfqpoint{1.254980in}{0.150000in}}{\pgfqpoint{5.490039in}{5.490039in}}%
\pgfusepath{clip}%
\pgfsetbuttcap%
\pgfsetroundjoin%
\definecolor{currentfill}{rgb}{0.223925,0.334994,0.548053}%
\pgfsetfillcolor{currentfill}%
\pgfsetfillopacity{0.700000}%
\pgfsetlinewidth{0.000000pt}%
\definecolor{currentstroke}{rgb}{0.000000,0.000000,0.000000}%
\pgfsetstrokecolor{currentstroke}%
\pgfsetdash{}{0pt}%
\pgfpathmoveto{\pgfqpoint{4.261325in}{1.537939in}}%
\pgfpathlineto{\pgfqpoint{4.275317in}{1.529519in}}%
\pgfpathlineto{\pgfqpoint{4.289313in}{1.521122in}}%
\pgfpathlineto{\pgfqpoint{4.303313in}{1.512748in}}%
\pgfpathlineto{\pgfqpoint{4.317318in}{1.504397in}}%
\pgfpathlineto{\pgfqpoint{4.309449in}{1.518045in}}%
\pgfpathlineto{\pgfqpoint{4.301569in}{1.532396in}}%
\pgfpathlineto{\pgfqpoint{4.293680in}{1.547462in}}%
\pgfpathlineto{\pgfqpoint{4.285780in}{1.563259in}}%
\pgfpathlineto{\pgfqpoint{4.271733in}{1.572076in}}%
\pgfpathlineto{\pgfqpoint{4.257691in}{1.580916in}}%
\pgfpathlineto{\pgfqpoint{4.243653in}{1.589780in}}%
\pgfpathlineto{\pgfqpoint{4.229620in}{1.598666in}}%
\pgfpathlineto{\pgfqpoint{4.237563in}{1.582396in}}%
\pgfpathlineto{\pgfqpoint{4.245495in}{1.566861in}}%
\pgfpathlineto{\pgfqpoint{4.253415in}{1.552047in}}%
\pgfpathlineto{\pgfqpoint{4.261325in}{1.537939in}}%
\pgfpathclose%
\pgfusepath{fill}%
\end{pgfscope}%
\begin{pgfscope}%
\pgfpathrectangle{\pgfqpoint{1.254980in}{0.150000in}}{\pgfqpoint{5.490039in}{5.490039in}}%
\pgfusepath{clip}%
\pgfsetbuttcap%
\pgfsetroundjoin%
\definecolor{currentfill}{rgb}{0.121148,0.592739,0.544641}%
\pgfsetfillcolor{currentfill}%
\pgfsetfillopacity{0.700000}%
\pgfsetlinewidth{0.000000pt}%
\definecolor{currentstroke}{rgb}{0.000000,0.000000,0.000000}%
\pgfsetstrokecolor{currentstroke}%
\pgfsetdash{}{0pt}%
\pgfpathmoveto{\pgfqpoint{3.414823in}{2.234641in}}%
\pgfpathlineto{\pgfqpoint{3.428720in}{2.223889in}}%
\pgfpathlineto{\pgfqpoint{3.442619in}{2.213164in}}%
\pgfpathlineto{\pgfqpoint{3.456522in}{2.202465in}}%
\pgfpathlineto{\pgfqpoint{3.470426in}{2.191793in}}%
\pgfpathlineto{\pgfqpoint{3.461699in}{2.218486in}}%
\pgfpathlineto{\pgfqpoint{3.452943in}{2.246071in}}%
\pgfpathlineto{\pgfqpoint{3.444157in}{2.274565in}}%
\pgfpathlineto{\pgfqpoint{3.430205in}{2.285623in}}%
\pgfpathlineto{\pgfqpoint{3.416255in}{2.296707in}}%
\pgfpathlineto{\pgfqpoint{3.402307in}{2.307818in}}%
\pgfpathlineto{\pgfqpoint{3.388362in}{2.318956in}}%
\pgfpathlineto{\pgfqpoint{3.397213in}{2.289943in}}%
\pgfpathlineto{\pgfqpoint{3.406033in}{2.261843in}}%
\pgfpathlineto{\pgfqpoint{3.414823in}{2.234641in}}%
\pgfpathclose%
\pgfusepath{fill}%
\end{pgfscope}%
\begin{pgfscope}%
\pgfpathrectangle{\pgfqpoint{1.254980in}{0.150000in}}{\pgfqpoint{5.490039in}{5.490039in}}%
\pgfusepath{clip}%
\pgfsetbuttcap%
\pgfsetroundjoin%
\definecolor{currentfill}{rgb}{0.172719,0.448791,0.557885}%
\pgfsetfillcolor{currentfill}%
\pgfsetfillopacity{0.700000}%
\pgfsetlinewidth{0.000000pt}%
\definecolor{currentstroke}{rgb}{0.000000,0.000000,0.000000}%
\pgfsetstrokecolor{currentstroke}%
\pgfsetdash{}{0pt}%
\pgfpathmoveto{\pgfqpoint{3.894008in}{1.818951in}}%
\pgfpathlineto{\pgfqpoint{3.907949in}{1.809500in}}%
\pgfpathlineto{\pgfqpoint{3.921892in}{1.800073in}}%
\pgfpathlineto{\pgfqpoint{3.935840in}{1.790669in}}%
\pgfpathlineto{\pgfqpoint{3.949791in}{1.781290in}}%
\pgfpathlineto{\pgfqpoint{3.941600in}{1.800706in}}%
\pgfpathlineto{\pgfqpoint{3.933393in}{1.820910in}}%
\pgfpathlineto{\pgfqpoint{3.925166in}{1.841917in}}%
\pgfpathlineto{\pgfqpoint{3.916922in}{1.863742in}}%
\pgfpathlineto{\pgfqpoint{3.902920in}{1.873610in}}%
\pgfpathlineto{\pgfqpoint{3.888921in}{1.883502in}}%
\pgfpathlineto{\pgfqpoint{3.874925in}{1.893418in}}%
\pgfpathlineto{\pgfqpoint{3.860933in}{1.903358in}}%
\pgfpathlineto{\pgfqpoint{3.869231in}{1.881036in}}%
\pgfpathlineto{\pgfqpoint{3.877509in}{1.859539in}}%
\pgfpathlineto{\pgfqpoint{3.885768in}{1.838849in}}%
\pgfpathlineto{\pgfqpoint{3.894008in}{1.818951in}}%
\pgfpathclose%
\pgfusepath{fill}%
\end{pgfscope}%
\begin{pgfscope}%
\pgfpathrectangle{\pgfqpoint{1.254980in}{0.150000in}}{\pgfqpoint{5.490039in}{5.490039in}}%
\pgfusepath{clip}%
\pgfsetbuttcap%
\pgfsetroundjoin%
\definecolor{currentfill}{rgb}{0.123463,0.581687,0.547445}%
\pgfsetfillcolor{currentfill}%
\pgfsetfillopacity{0.700000}%
\pgfsetlinewidth{0.000000pt}%
\definecolor{currentstroke}{rgb}{0.000000,0.000000,0.000000}%
\pgfsetstrokecolor{currentstroke}%
\pgfsetdash{}{0pt}%
\pgfpathmoveto{\pgfqpoint{3.470426in}{2.191793in}}%
\pgfpathlineto{\pgfqpoint{3.484334in}{2.181146in}}%
\pgfpathlineto{\pgfqpoint{3.498244in}{2.170526in}}%
\pgfpathlineto{\pgfqpoint{3.512157in}{2.159933in}}%
\pgfpathlineto{\pgfqpoint{3.526072in}{2.149365in}}%
\pgfpathlineto{\pgfqpoint{3.517407in}{2.175551in}}%
\pgfpathlineto{\pgfqpoint{3.508713in}{2.202623in}}%
\pgfpathlineto{\pgfqpoint{3.499991in}{2.230599in}}%
\pgfpathlineto{\pgfqpoint{3.486029in}{2.241551in}}%
\pgfpathlineto{\pgfqpoint{3.472069in}{2.252529in}}%
\pgfpathlineto{\pgfqpoint{3.458112in}{2.263534in}}%
\pgfpathlineto{\pgfqpoint{3.444157in}{2.274565in}}%
\pgfpathlineto{\pgfqpoint{3.452943in}{2.246071in}}%
\pgfpathlineto{\pgfqpoint{3.461699in}{2.218486in}}%
\pgfpathlineto{\pgfqpoint{3.470426in}{2.191793in}}%
\pgfpathclose%
\pgfusepath{fill}%
\end{pgfscope}%
\begin{pgfscope}%
\pgfpathrectangle{\pgfqpoint{1.254980in}{0.150000in}}{\pgfqpoint{5.490039in}{5.490039in}}%
\pgfusepath{clip}%
\pgfsetbuttcap%
\pgfsetroundjoin%
\definecolor{currentfill}{rgb}{0.227802,0.326594,0.546532}%
\pgfsetfillcolor{currentfill}%
\pgfsetfillopacity{0.700000}%
\pgfsetlinewidth{0.000000pt}%
\definecolor{currentstroke}{rgb}{0.000000,0.000000,0.000000}%
\pgfsetstrokecolor{currentstroke}%
\pgfsetdash{}{0pt}%
\pgfpathmoveto{\pgfqpoint{4.317318in}{1.504397in}}%
\pgfpathlineto{\pgfqpoint{4.331328in}{1.496068in}}%
\pgfpathlineto{\pgfqpoint{4.345342in}{1.487763in}}%
\pgfpathlineto{\pgfqpoint{4.359360in}{1.479480in}}%
\pgfpathlineto{\pgfqpoint{4.373383in}{1.471220in}}%
\pgfpathlineto{\pgfqpoint{4.365553in}{1.484409in}}%
\pgfpathlineto{\pgfqpoint{4.357714in}{1.498296in}}%
\pgfpathlineto{\pgfqpoint{4.349865in}{1.512894in}}%
\pgfpathlineto{\pgfqpoint{4.342007in}{1.528218in}}%
\pgfpathlineto{\pgfqpoint{4.327944in}{1.536944in}}%
\pgfpathlineto{\pgfqpoint{4.313885in}{1.545693in}}%
\pgfpathlineto{\pgfqpoint{4.299830in}{1.554464in}}%
\pgfpathlineto{\pgfqpoint{4.285780in}{1.563259in}}%
\pgfpathlineto{\pgfqpoint{4.293680in}{1.547462in}}%
\pgfpathlineto{\pgfqpoint{4.301569in}{1.532396in}}%
\pgfpathlineto{\pgfqpoint{4.309449in}{1.518045in}}%
\pgfpathlineto{\pgfqpoint{4.317318in}{1.504397in}}%
\pgfpathclose%
\pgfusepath{fill}%
\end{pgfscope}%
\begin{pgfscope}%
\pgfpathrectangle{\pgfqpoint{1.254980in}{0.150000in}}{\pgfqpoint{5.490039in}{5.490039in}}%
\pgfusepath{clip}%
\pgfsetbuttcap%
\pgfsetroundjoin%
\definecolor{currentfill}{rgb}{0.263663,0.237631,0.518762}%
\pgfsetfillcolor{currentfill}%
\pgfsetfillopacity{0.700000}%
\pgfsetlinewidth{0.000000pt}%
\definecolor{currentstroke}{rgb}{0.000000,0.000000,0.000000}%
\pgfsetstrokecolor{currentstroke}%
\pgfsetdash{}{0pt}%
\pgfpathmoveto{\pgfqpoint{4.629072in}{1.303232in}}%
\pgfpathlineto{\pgfqpoint{4.643144in}{1.295801in}}%
\pgfpathlineto{\pgfqpoint{4.657221in}{1.288392in}}%
\pgfpathlineto{\pgfqpoint{4.671303in}{1.281005in}}%
\pgfpathlineto{\pgfqpoint{4.685390in}{1.273641in}}%
\pgfpathlineto{\pgfqpoint{4.677758in}{1.281970in}}%
\pgfpathlineto{\pgfqpoint{4.670123in}{1.290922in}}%
\pgfpathlineto{\pgfqpoint{4.662483in}{1.300509in}}%
\pgfpathlineto{\pgfqpoint{4.654839in}{1.310746in}}%
\pgfpathlineto{\pgfqpoint{4.640720in}{1.318557in}}%
\pgfpathlineto{\pgfqpoint{4.626606in}{1.326391in}}%
\pgfpathlineto{\pgfqpoint{4.612497in}{1.334247in}}%
\pgfpathlineto{\pgfqpoint{4.598392in}{1.342125in}}%
\pgfpathlineto{\pgfqpoint{4.606070in}{1.331435in}}%
\pgfpathlineto{\pgfqpoint{4.613742in}{1.321398in}}%
\pgfpathlineto{\pgfqpoint{4.621410in}{1.312001in}}%
\pgfpathlineto{\pgfqpoint{4.629072in}{1.303232in}}%
\pgfpathclose%
\pgfusepath{fill}%
\end{pgfscope}%
\begin{pgfscope}%
\pgfpathrectangle{\pgfqpoint{1.254980in}{0.150000in}}{\pgfqpoint{5.490039in}{5.490039in}}%
\pgfusepath{clip}%
\pgfsetbuttcap%
\pgfsetroundjoin%
\definecolor{currentfill}{rgb}{0.177423,0.437527,0.557565}%
\pgfsetfillcolor{currentfill}%
\pgfsetfillopacity{0.700000}%
\pgfsetlinewidth{0.000000pt}%
\definecolor{currentstroke}{rgb}{0.000000,0.000000,0.000000}%
\pgfsetstrokecolor{currentstroke}%
\pgfsetdash{}{0pt}%
\pgfpathmoveto{\pgfqpoint{3.949791in}{1.781290in}}%
\pgfpathlineto{\pgfqpoint{3.963746in}{1.771935in}}%
\pgfpathlineto{\pgfqpoint{3.977704in}{1.762603in}}%
\pgfpathlineto{\pgfqpoint{3.991666in}{1.753295in}}%
\pgfpathlineto{\pgfqpoint{4.005632in}{1.744012in}}%
\pgfpathlineto{\pgfqpoint{3.997491in}{1.762947in}}%
\pgfpathlineto{\pgfqpoint{3.989333in}{1.782665in}}%
\pgfpathlineto{\pgfqpoint{3.981158in}{1.803182in}}%
\pgfpathlineto{\pgfqpoint{3.972966in}{1.824512in}}%
\pgfpathlineto{\pgfqpoint{3.958950in}{1.834283in}}%
\pgfpathlineto{\pgfqpoint{3.944937in}{1.844079in}}%
\pgfpathlineto{\pgfqpoint{3.930928in}{1.853898in}}%
\pgfpathlineto{\pgfqpoint{3.916922in}{1.863742in}}%
\pgfpathlineto{\pgfqpoint{3.925166in}{1.841917in}}%
\pgfpathlineto{\pgfqpoint{3.933393in}{1.820910in}}%
\pgfpathlineto{\pgfqpoint{3.941600in}{1.800706in}}%
\pgfpathlineto{\pgfqpoint{3.949791in}{1.781290in}}%
\pgfpathclose%
\pgfusepath{fill}%
\end{pgfscope}%
\begin{pgfscope}%
\pgfpathrectangle{\pgfqpoint{1.254980in}{0.150000in}}{\pgfqpoint{5.490039in}{5.490039in}}%
\pgfusepath{clip}%
\pgfsetbuttcap%
\pgfsetroundjoin%
\definecolor{currentfill}{rgb}{0.126453,0.570633,0.549841}%
\pgfsetfillcolor{currentfill}%
\pgfsetfillopacity{0.700000}%
\pgfsetlinewidth{0.000000pt}%
\definecolor{currentstroke}{rgb}{0.000000,0.000000,0.000000}%
\pgfsetstrokecolor{currentstroke}%
\pgfsetdash{}{0pt}%
\pgfpathmoveto{\pgfqpoint{3.526072in}{2.149365in}}%
\pgfpathlineto{\pgfqpoint{3.539991in}{2.138823in}}%
\pgfpathlineto{\pgfqpoint{3.553912in}{2.128307in}}%
\pgfpathlineto{\pgfqpoint{3.567836in}{2.117817in}}%
\pgfpathlineto{\pgfqpoint{3.581763in}{2.107353in}}%
\pgfpathlineto{\pgfqpoint{3.573157in}{2.133032in}}%
\pgfpathlineto{\pgfqpoint{3.564526in}{2.159593in}}%
\pgfpathlineto{\pgfqpoint{3.555867in}{2.187051in}}%
\pgfpathlineto{\pgfqpoint{3.541894in}{2.197899in}}%
\pgfpathlineto{\pgfqpoint{3.527924in}{2.208773in}}%
\pgfpathlineto{\pgfqpoint{3.513956in}{2.219673in}}%
\pgfpathlineto{\pgfqpoint{3.499991in}{2.230599in}}%
\pgfpathlineto{\pgfqpoint{3.508713in}{2.202623in}}%
\pgfpathlineto{\pgfqpoint{3.517407in}{2.175551in}}%
\pgfpathlineto{\pgfqpoint{3.526072in}{2.149365in}}%
\pgfpathclose%
\pgfusepath{fill}%
\end{pgfscope}%
\begin{pgfscope}%
\pgfpathrectangle{\pgfqpoint{1.254980in}{0.150000in}}{\pgfqpoint{5.490039in}{5.490039in}}%
\pgfusepath{clip}%
\pgfsetbuttcap%
\pgfsetroundjoin%
\definecolor{currentfill}{rgb}{0.231674,0.318106,0.544834}%
\pgfsetfillcolor{currentfill}%
\pgfsetfillopacity{0.700000}%
\pgfsetlinewidth{0.000000pt}%
\definecolor{currentstroke}{rgb}{0.000000,0.000000,0.000000}%
\pgfsetstrokecolor{currentstroke}%
\pgfsetdash{}{0pt}%
\pgfpathmoveto{\pgfqpoint{4.373383in}{1.471220in}}%
\pgfpathlineto{\pgfqpoint{4.387411in}{1.462982in}}%
\pgfpathlineto{\pgfqpoint{4.401443in}{1.454767in}}%
\pgfpathlineto{\pgfqpoint{4.415480in}{1.446575in}}%
\pgfpathlineto{\pgfqpoint{4.429522in}{1.438406in}}%
\pgfpathlineto{\pgfqpoint{4.421730in}{1.451136in}}%
\pgfpathlineto{\pgfqpoint{4.413930in}{1.464560in}}%
\pgfpathlineto{\pgfqpoint{4.406122in}{1.478691in}}%
\pgfpathlineto{\pgfqpoint{4.398304in}{1.493543in}}%
\pgfpathlineto{\pgfqpoint{4.384224in}{1.502178in}}%
\pgfpathlineto{\pgfqpoint{4.370147in}{1.510835in}}%
\pgfpathlineto{\pgfqpoint{4.356075in}{1.519515in}}%
\pgfpathlineto{\pgfqpoint{4.342007in}{1.528218in}}%
\pgfpathlineto{\pgfqpoint{4.349865in}{1.512894in}}%
\pgfpathlineto{\pgfqpoint{4.357714in}{1.498296in}}%
\pgfpathlineto{\pgfqpoint{4.365553in}{1.484409in}}%
\pgfpathlineto{\pgfqpoint{4.373383in}{1.471220in}}%
\pgfpathclose%
\pgfusepath{fill}%
\end{pgfscope}%
\begin{pgfscope}%
\pgfpathrectangle{\pgfqpoint{1.254980in}{0.150000in}}{\pgfqpoint{5.490039in}{5.490039in}}%
\pgfusepath{clip}%
\pgfsetbuttcap%
\pgfsetroundjoin%
\definecolor{currentfill}{rgb}{0.266580,0.228262,0.514349}%
\pgfsetfillcolor{currentfill}%
\pgfsetfillopacity{0.700000}%
\pgfsetlinewidth{0.000000pt}%
\definecolor{currentstroke}{rgb}{0.000000,0.000000,0.000000}%
\pgfsetstrokecolor{currentstroke}%
\pgfsetdash{}{0pt}%
\pgfpathmoveto{\pgfqpoint{4.685390in}{1.273641in}}%
\pgfpathlineto{\pgfqpoint{4.699483in}{1.266299in}}%
\pgfpathlineto{\pgfqpoint{4.713581in}{1.258979in}}%
\pgfpathlineto{\pgfqpoint{4.727684in}{1.251681in}}%
\pgfpathlineto{\pgfqpoint{4.741792in}{1.244405in}}%
\pgfpathlineto{\pgfqpoint{4.734191in}{1.252293in}}%
\pgfpathlineto{\pgfqpoint{4.726586in}{1.260800in}}%
\pgfpathlineto{\pgfqpoint{4.718978in}{1.269939in}}%
\pgfpathlineto{\pgfqpoint{4.711367in}{1.279723in}}%
\pgfpathlineto{\pgfqpoint{4.697228in}{1.287445in}}%
\pgfpathlineto{\pgfqpoint{4.683093in}{1.295190in}}%
\pgfpathlineto{\pgfqpoint{4.668964in}{1.302957in}}%
\pgfpathlineto{\pgfqpoint{4.654839in}{1.310746in}}%
\pgfpathlineto{\pgfqpoint{4.662483in}{1.300509in}}%
\pgfpathlineto{\pgfqpoint{4.670123in}{1.290922in}}%
\pgfpathlineto{\pgfqpoint{4.677758in}{1.281970in}}%
\pgfpathlineto{\pgfqpoint{4.685390in}{1.273641in}}%
\pgfpathclose%
\pgfusepath{fill}%
\end{pgfscope}%
\begin{pgfscope}%
\pgfpathrectangle{\pgfqpoint{1.254980in}{0.150000in}}{\pgfqpoint{5.490039in}{5.490039in}}%
\pgfusepath{clip}%
\pgfsetbuttcap%
\pgfsetroundjoin%
\definecolor{currentfill}{rgb}{0.182256,0.426184,0.557120}%
\pgfsetfillcolor{currentfill}%
\pgfsetfillopacity{0.700000}%
\pgfsetlinewidth{0.000000pt}%
\definecolor{currentstroke}{rgb}{0.000000,0.000000,0.000000}%
\pgfsetstrokecolor{currentstroke}%
\pgfsetdash{}{0pt}%
\pgfpathmoveto{\pgfqpoint{4.005632in}{1.744012in}}%
\pgfpathlineto{\pgfqpoint{4.019602in}{1.734751in}}%
\pgfpathlineto{\pgfqpoint{4.033576in}{1.725515in}}%
\pgfpathlineto{\pgfqpoint{4.047553in}{1.716302in}}%
\pgfpathlineto{\pgfqpoint{4.061535in}{1.707113in}}%
\pgfpathlineto{\pgfqpoint{4.053441in}{1.725568in}}%
\pgfpathlineto{\pgfqpoint{4.045332in}{1.744802in}}%
\pgfpathlineto{\pgfqpoint{4.037208in}{1.764829in}}%
\pgfpathlineto{\pgfqpoint{4.029067in}{1.785664in}}%
\pgfpathlineto{\pgfqpoint{4.015036in}{1.795341in}}%
\pgfpathlineto{\pgfqpoint{4.001009in}{1.805041in}}%
\pgfpathlineto{\pgfqpoint{3.986986in}{1.814764in}}%
\pgfpathlineto{\pgfqpoint{3.972966in}{1.824512in}}%
\pgfpathlineto{\pgfqpoint{3.981158in}{1.803182in}}%
\pgfpathlineto{\pgfqpoint{3.989333in}{1.782665in}}%
\pgfpathlineto{\pgfqpoint{3.997491in}{1.762947in}}%
\pgfpathlineto{\pgfqpoint{4.005632in}{1.744012in}}%
\pgfpathclose%
\pgfusepath{fill}%
\end{pgfscope}%
\begin{pgfscope}%
\pgfpathrectangle{\pgfqpoint{1.254980in}{0.150000in}}{\pgfqpoint{5.490039in}{5.490039in}}%
\pgfusepath{clip}%
\pgfsetbuttcap%
\pgfsetroundjoin%
\definecolor{currentfill}{rgb}{0.131172,0.555899,0.552459}%
\pgfsetfillcolor{currentfill}%
\pgfsetfillopacity{0.700000}%
\pgfsetlinewidth{0.000000pt}%
\definecolor{currentstroke}{rgb}{0.000000,0.000000,0.000000}%
\pgfsetstrokecolor{currentstroke}%
\pgfsetdash{}{0pt}%
\pgfpathmoveto{\pgfqpoint{3.581763in}{2.107353in}}%
\pgfpathlineto{\pgfqpoint{3.595692in}{2.096914in}}%
\pgfpathlineto{\pgfqpoint{3.609625in}{2.086501in}}%
\pgfpathlineto{\pgfqpoint{3.623560in}{2.076114in}}%
\pgfpathlineto{\pgfqpoint{3.637499in}{2.065752in}}%
\pgfpathlineto{\pgfqpoint{3.628953in}{2.090925in}}%
\pgfpathlineto{\pgfqpoint{3.620382in}{2.116975in}}%
\pgfpathlineto{\pgfqpoint{3.611785in}{2.143917in}}%
\pgfpathlineto{\pgfqpoint{3.597801in}{2.154662in}}%
\pgfpathlineto{\pgfqpoint{3.583820in}{2.165433in}}%
\pgfpathlineto{\pgfqpoint{3.569842in}{2.176229in}}%
\pgfpathlineto{\pgfqpoint{3.555867in}{2.187051in}}%
\pgfpathlineto{\pgfqpoint{3.564526in}{2.159593in}}%
\pgfpathlineto{\pgfqpoint{3.573157in}{2.133032in}}%
\pgfpathlineto{\pgfqpoint{3.581763in}{2.107353in}}%
\pgfpathclose%
\pgfusepath{fill}%
\end{pgfscope}%
\begin{pgfscope}%
\pgfpathrectangle{\pgfqpoint{1.254980in}{0.150000in}}{\pgfqpoint{5.490039in}{5.490039in}}%
\pgfusepath{clip}%
\pgfsetbuttcap%
\pgfsetroundjoin%
\definecolor{currentfill}{rgb}{0.235526,0.309527,0.542944}%
\pgfsetfillcolor{currentfill}%
\pgfsetfillopacity{0.700000}%
\pgfsetlinewidth{0.000000pt}%
\definecolor{currentstroke}{rgb}{0.000000,0.000000,0.000000}%
\pgfsetstrokecolor{currentstroke}%
\pgfsetdash{}{0pt}%
\pgfpathmoveto{\pgfqpoint{4.429522in}{1.438406in}}%
\pgfpathlineto{\pgfqpoint{4.443568in}{1.430259in}}%
\pgfpathlineto{\pgfqpoint{4.457619in}{1.422134in}}%
\pgfpathlineto{\pgfqpoint{4.471674in}{1.414032in}}%
\pgfpathlineto{\pgfqpoint{4.485735in}{1.405953in}}%
\pgfpathlineto{\pgfqpoint{4.477980in}{1.418225in}}%
\pgfpathlineto{\pgfqpoint{4.470219in}{1.431186in}}%
\pgfpathlineto{\pgfqpoint{4.462450in}{1.444850in}}%
\pgfpathlineto{\pgfqpoint{4.454673in}{1.459230in}}%
\pgfpathlineto{\pgfqpoint{4.440574in}{1.467775in}}%
\pgfpathlineto{\pgfqpoint{4.426480in}{1.476341in}}%
\pgfpathlineto{\pgfqpoint{4.412390in}{1.484931in}}%
\pgfpathlineto{\pgfqpoint{4.398304in}{1.493543in}}%
\pgfpathlineto{\pgfqpoint{4.406122in}{1.478691in}}%
\pgfpathlineto{\pgfqpoint{4.413930in}{1.464560in}}%
\pgfpathlineto{\pgfqpoint{4.421730in}{1.451136in}}%
\pgfpathlineto{\pgfqpoint{4.429522in}{1.438406in}}%
\pgfpathclose%
\pgfusepath{fill}%
\end{pgfscope}%
\begin{pgfscope}%
\pgfpathrectangle{\pgfqpoint{1.254980in}{0.150000in}}{\pgfqpoint{5.490039in}{5.490039in}}%
\pgfusepath{clip}%
\pgfsetbuttcap%
\pgfsetroundjoin%
\definecolor{currentfill}{rgb}{0.185556,0.418570,0.556753}%
\pgfsetfillcolor{currentfill}%
\pgfsetfillopacity{0.700000}%
\pgfsetlinewidth{0.000000pt}%
\definecolor{currentstroke}{rgb}{0.000000,0.000000,0.000000}%
\pgfsetstrokecolor{currentstroke}%
\pgfsetdash{}{0pt}%
\pgfpathmoveto{\pgfqpoint{4.061535in}{1.707113in}}%
\pgfpathlineto{\pgfqpoint{4.075520in}{1.697947in}}%
\pgfpathlineto{\pgfqpoint{4.089509in}{1.688805in}}%
\pgfpathlineto{\pgfqpoint{4.103502in}{1.679686in}}%
\pgfpathlineto{\pgfqpoint{4.117499in}{1.670591in}}%
\pgfpathlineto{\pgfqpoint{4.109453in}{1.688567in}}%
\pgfpathlineto{\pgfqpoint{4.101392in}{1.707316in}}%
\pgfpathlineto{\pgfqpoint{4.093316in}{1.726854in}}%
\pgfpathlineto{\pgfqpoint{4.085225in}{1.747196in}}%
\pgfpathlineto{\pgfqpoint{4.071180in}{1.756778in}}%
\pgfpathlineto{\pgfqpoint{4.057139in}{1.766383in}}%
\pgfpathlineto{\pgfqpoint{4.043101in}{1.776012in}}%
\pgfpathlineto{\pgfqpoint{4.029067in}{1.785664in}}%
\pgfpathlineto{\pgfqpoint{4.037208in}{1.764829in}}%
\pgfpathlineto{\pgfqpoint{4.045332in}{1.744802in}}%
\pgfpathlineto{\pgfqpoint{4.053441in}{1.725568in}}%
\pgfpathlineto{\pgfqpoint{4.061535in}{1.707113in}}%
\pgfpathclose%
\pgfusepath{fill}%
\end{pgfscope}%
\begin{pgfscope}%
\pgfpathrectangle{\pgfqpoint{1.254980in}{0.150000in}}{\pgfqpoint{5.490039in}{5.490039in}}%
\pgfusepath{clip}%
\pgfsetbuttcap%
\pgfsetroundjoin%
\definecolor{currentfill}{rgb}{0.135066,0.544853,0.554029}%
\pgfsetfillcolor{currentfill}%
\pgfsetfillopacity{0.700000}%
\pgfsetlinewidth{0.000000pt}%
\definecolor{currentstroke}{rgb}{0.000000,0.000000,0.000000}%
\pgfsetstrokecolor{currentstroke}%
\pgfsetdash{}{0pt}%
\pgfpathmoveto{\pgfqpoint{3.637499in}{2.065752in}}%
\pgfpathlineto{\pgfqpoint{3.651440in}{2.055415in}}%
\pgfpathlineto{\pgfqpoint{3.665384in}{2.045103in}}%
\pgfpathlineto{\pgfqpoint{3.679332in}{2.034817in}}%
\pgfpathlineto{\pgfqpoint{3.693282in}{2.024556in}}%
\pgfpathlineto{\pgfqpoint{3.684795in}{2.049225in}}%
\pgfpathlineto{\pgfqpoint{3.676284in}{2.074765in}}%
\pgfpathlineto{\pgfqpoint{3.667748in}{2.101192in}}%
\pgfpathlineto{\pgfqpoint{3.653753in}{2.111835in}}%
\pgfpathlineto{\pgfqpoint{3.639761in}{2.122504in}}%
\pgfpathlineto{\pgfqpoint{3.625771in}{2.133198in}}%
\pgfpathlineto{\pgfqpoint{3.611785in}{2.143917in}}%
\pgfpathlineto{\pgfqpoint{3.620382in}{2.116975in}}%
\pgfpathlineto{\pgfqpoint{3.628953in}{2.090925in}}%
\pgfpathlineto{\pgfqpoint{3.637499in}{2.065752in}}%
\pgfpathclose%
\pgfusepath{fill}%
\end{pgfscope}%
\begin{pgfscope}%
\pgfpathrectangle{\pgfqpoint{1.254980in}{0.150000in}}{\pgfqpoint{5.490039in}{5.490039in}}%
\pgfusepath{clip}%
\pgfsetbuttcap%
\pgfsetroundjoin%
\definecolor{currentfill}{rgb}{0.267968,0.223549,0.512008}%
\pgfsetfillcolor{currentfill}%
\pgfsetfillopacity{0.700000}%
\pgfsetlinewidth{0.000000pt}%
\definecolor{currentstroke}{rgb}{0.000000,0.000000,0.000000}%
\pgfsetstrokecolor{currentstroke}%
\pgfsetdash{}{0pt}%
\pgfpathmoveto{\pgfqpoint{4.741792in}{1.244405in}}%
\pgfpathlineto{\pgfqpoint{4.755906in}{1.237151in}}%
\pgfpathlineto{\pgfqpoint{4.770026in}{1.229919in}}%
\pgfpathlineto{\pgfqpoint{4.784151in}{1.222709in}}%
\pgfpathlineto{\pgfqpoint{4.776571in}{1.230267in}}%
\pgfpathlineto{\pgfqpoint{4.768989in}{1.238441in}}%
\pgfpathlineto{\pgfqpoint{4.761404in}{1.247244in}}%
\pgfpathlineto{\pgfqpoint{4.753817in}{1.256688in}}%
\pgfpathlineto{\pgfqpoint{4.739662in}{1.264344in}}%
\pgfpathlineto{\pgfqpoint{4.725512in}{1.272023in}}%
\pgfpathlineto{\pgfqpoint{4.711367in}{1.279723in}}%
\pgfpathlineto{\pgfqpoint{4.718978in}{1.269939in}}%
\pgfpathlineto{\pgfqpoint{4.726586in}{1.260800in}}%
\pgfpathlineto{\pgfqpoint{4.734191in}{1.252293in}}%
\pgfpathlineto{\pgfqpoint{4.741792in}{1.244405in}}%
\pgfpathclose%
\pgfusepath{fill}%
\end{pgfscope}%
\begin{pgfscope}%
\pgfpathrectangle{\pgfqpoint{1.254980in}{0.150000in}}{\pgfqpoint{5.490039in}{5.490039in}}%
\pgfusepath{clip}%
\pgfsetbuttcap%
\pgfsetroundjoin%
\definecolor{currentfill}{rgb}{0.139147,0.533812,0.555298}%
\pgfsetfillcolor{currentfill}%
\pgfsetfillopacity{0.700000}%
\pgfsetlinewidth{0.000000pt}%
\definecolor{currentstroke}{rgb}{0.000000,0.000000,0.000000}%
\pgfsetstrokecolor{currentstroke}%
\pgfsetdash{}{0pt}%
\pgfpathmoveto{\pgfqpoint{3.693282in}{2.024556in}}%
\pgfpathlineto{\pgfqpoint{3.707236in}{2.014320in}}%
\pgfpathlineto{\pgfqpoint{3.721192in}{2.004108in}}%
\pgfpathlineto{\pgfqpoint{3.735152in}{1.993922in}}%
\pgfpathlineto{\pgfqpoint{3.749115in}{1.983761in}}%
\pgfpathlineto{\pgfqpoint{3.740685in}{2.007926in}}%
\pgfpathlineto{\pgfqpoint{3.732232in}{2.032957in}}%
\pgfpathlineto{\pgfqpoint{3.723756in}{2.058871in}}%
\pgfpathlineto{\pgfqpoint{3.709750in}{2.069414in}}%
\pgfpathlineto{\pgfqpoint{3.695746in}{2.079981in}}%
\pgfpathlineto{\pgfqpoint{3.681745in}{2.090574in}}%
\pgfpathlineto{\pgfqpoint{3.667748in}{2.101192in}}%
\pgfpathlineto{\pgfqpoint{3.676284in}{2.074765in}}%
\pgfpathlineto{\pgfqpoint{3.684795in}{2.049225in}}%
\pgfpathlineto{\pgfqpoint{3.693282in}{2.024556in}}%
\pgfpathclose%
\pgfusepath{fill}%
\end{pgfscope}%
\begin{pgfscope}%
\pgfpathrectangle{\pgfqpoint{1.254980in}{0.150000in}}{\pgfqpoint{5.490039in}{5.490039in}}%
\pgfusepath{clip}%
\pgfsetbuttcap%
\pgfsetroundjoin%
\definecolor{currentfill}{rgb}{0.190631,0.407061,0.556089}%
\pgfsetfillcolor{currentfill}%
\pgfsetfillopacity{0.700000}%
\pgfsetlinewidth{0.000000pt}%
\definecolor{currentstroke}{rgb}{0.000000,0.000000,0.000000}%
\pgfsetstrokecolor{currentstroke}%
\pgfsetdash{}{0pt}%
\pgfpathmoveto{\pgfqpoint{4.117499in}{1.670591in}}%
\pgfpathlineto{\pgfqpoint{4.131500in}{1.661519in}}%
\pgfpathlineto{\pgfqpoint{4.145505in}{1.652470in}}%
\pgfpathlineto{\pgfqpoint{4.159514in}{1.643445in}}%
\pgfpathlineto{\pgfqpoint{4.173527in}{1.634443in}}%
\pgfpathlineto{\pgfqpoint{4.165526in}{1.651940in}}%
\pgfpathlineto{\pgfqpoint{4.157513in}{1.670206in}}%
\pgfpathlineto{\pgfqpoint{4.149485in}{1.689256in}}%
\pgfpathlineto{\pgfqpoint{4.141444in}{1.709105in}}%
\pgfpathlineto{\pgfqpoint{4.127383in}{1.718593in}}%
\pgfpathlineto{\pgfqpoint{4.113327in}{1.728104in}}%
\pgfpathlineto{\pgfqpoint{4.099274in}{1.737638in}}%
\pgfpathlineto{\pgfqpoint{4.085225in}{1.747196in}}%
\pgfpathlineto{\pgfqpoint{4.093316in}{1.726854in}}%
\pgfpathlineto{\pgfqpoint{4.101392in}{1.707316in}}%
\pgfpathlineto{\pgfqpoint{4.109453in}{1.688567in}}%
\pgfpathlineto{\pgfqpoint{4.117499in}{1.670591in}}%
\pgfpathclose%
\pgfusepath{fill}%
\end{pgfscope}%
\begin{pgfscope}%
\pgfpathrectangle{\pgfqpoint{1.254980in}{0.150000in}}{\pgfqpoint{5.490039in}{5.490039in}}%
\pgfusepath{clip}%
\pgfsetbuttcap%
\pgfsetroundjoin%
\definecolor{currentfill}{rgb}{0.239346,0.300855,0.540844}%
\pgfsetfillcolor{currentfill}%
\pgfsetfillopacity{0.700000}%
\pgfsetlinewidth{0.000000pt}%
\definecolor{currentstroke}{rgb}{0.000000,0.000000,0.000000}%
\pgfsetstrokecolor{currentstroke}%
\pgfsetdash{}{0pt}%
\pgfpathmoveto{\pgfqpoint{4.485735in}{1.405953in}}%
\pgfpathlineto{\pgfqpoint{4.499800in}{1.397896in}}%
\pgfpathlineto{\pgfqpoint{4.513870in}{1.389861in}}%
\pgfpathlineto{\pgfqpoint{4.527945in}{1.381849in}}%
\pgfpathlineto{\pgfqpoint{4.542025in}{1.373860in}}%
\pgfpathlineto{\pgfqpoint{4.534307in}{1.385674in}}%
\pgfpathlineto{\pgfqpoint{4.526582in}{1.398172in}}%
\pgfpathlineto{\pgfqpoint{4.518851in}{1.411370in}}%
\pgfpathlineto{\pgfqpoint{4.511114in}{1.425279in}}%
\pgfpathlineto{\pgfqpoint{4.496997in}{1.433733in}}%
\pgfpathlineto{\pgfqpoint{4.482884in}{1.442210in}}%
\pgfpathlineto{\pgfqpoint{4.468776in}{1.450709in}}%
\pgfpathlineto{\pgfqpoint{4.454673in}{1.459230in}}%
\pgfpathlineto{\pgfqpoint{4.462450in}{1.444850in}}%
\pgfpathlineto{\pgfqpoint{4.470219in}{1.431186in}}%
\pgfpathlineto{\pgfqpoint{4.477980in}{1.418225in}}%
\pgfpathlineto{\pgfqpoint{4.485735in}{1.405953in}}%
\pgfpathclose%
\pgfusepath{fill}%
\end{pgfscope}%
\begin{pgfscope}%
\pgfpathrectangle{\pgfqpoint{1.254980in}{0.150000in}}{\pgfqpoint{5.490039in}{5.490039in}}%
\pgfusepath{clip}%
\pgfsetbuttcap%
\pgfsetroundjoin%
\definecolor{currentfill}{rgb}{0.144759,0.519093,0.556572}%
\pgfsetfillcolor{currentfill}%
\pgfsetfillopacity{0.700000}%
\pgfsetlinewidth{0.000000pt}%
\definecolor{currentstroke}{rgb}{0.000000,0.000000,0.000000}%
\pgfsetstrokecolor{currentstroke}%
\pgfsetdash{}{0pt}%
\pgfpathmoveto{\pgfqpoint{3.749115in}{1.983761in}}%
\pgfpathlineto{\pgfqpoint{3.763081in}{1.973624in}}%
\pgfpathlineto{\pgfqpoint{3.777050in}{1.963513in}}%
\pgfpathlineto{\pgfqpoint{3.791022in}{1.953425in}}%
\pgfpathlineto{\pgfqpoint{3.804998in}{1.943363in}}%
\pgfpathlineto{\pgfqpoint{3.796625in}{1.967026in}}%
\pgfpathlineto{\pgfqpoint{3.788230in}{1.991549in}}%
\pgfpathlineto{\pgfqpoint{3.779812in}{2.016949in}}%
\pgfpathlineto{\pgfqpoint{3.765794in}{2.027392in}}%
\pgfpathlineto{\pgfqpoint{3.751778in}{2.037860in}}%
\pgfpathlineto{\pgfqpoint{3.737766in}{2.048353in}}%
\pgfpathlineto{\pgfqpoint{3.723756in}{2.058871in}}%
\pgfpathlineto{\pgfqpoint{3.732232in}{2.032957in}}%
\pgfpathlineto{\pgfqpoint{3.740685in}{2.007926in}}%
\pgfpathlineto{\pgfqpoint{3.749115in}{1.983761in}}%
\pgfpathclose%
\pgfusepath{fill}%
\end{pgfscope}%
\begin{pgfscope}%
\pgfpathrectangle{\pgfqpoint{1.254980in}{0.150000in}}{\pgfqpoint{5.490039in}{5.490039in}}%
\pgfusepath{clip}%
\pgfsetbuttcap%
\pgfsetroundjoin%
\definecolor{currentfill}{rgb}{0.195860,0.395433,0.555276}%
\pgfsetfillcolor{currentfill}%
\pgfsetfillopacity{0.700000}%
\pgfsetlinewidth{0.000000pt}%
\definecolor{currentstroke}{rgb}{0.000000,0.000000,0.000000}%
\pgfsetstrokecolor{currentstroke}%
\pgfsetdash{}{0pt}%
\pgfpathmoveto{\pgfqpoint{4.173527in}{1.634443in}}%
\pgfpathlineto{\pgfqpoint{4.187544in}{1.625464in}}%
\pgfpathlineto{\pgfqpoint{4.201565in}{1.616508in}}%
\pgfpathlineto{\pgfqpoint{4.215590in}{1.607576in}}%
\pgfpathlineto{\pgfqpoint{4.229620in}{1.598666in}}%
\pgfpathlineto{\pgfqpoint{4.221665in}{1.615685in}}%
\pgfpathlineto{\pgfqpoint{4.213697in}{1.633468in}}%
\pgfpathlineto{\pgfqpoint{4.205717in}{1.652031in}}%
\pgfpathlineto{\pgfqpoint{4.197723in}{1.671388in}}%
\pgfpathlineto{\pgfqpoint{4.183648in}{1.680782in}}%
\pgfpathlineto{\pgfqpoint{4.169576in}{1.690200in}}%
\pgfpathlineto{\pgfqpoint{4.155508in}{1.699641in}}%
\pgfpathlineto{\pgfqpoint{4.141444in}{1.709105in}}%
\pgfpathlineto{\pgfqpoint{4.149485in}{1.689256in}}%
\pgfpathlineto{\pgfqpoint{4.157513in}{1.670206in}}%
\pgfpathlineto{\pgfqpoint{4.165526in}{1.651940in}}%
\pgfpathlineto{\pgfqpoint{4.173527in}{1.634443in}}%
\pgfpathclose%
\pgfusepath{fill}%
\end{pgfscope}%
\begin{pgfscope}%
\pgfpathrectangle{\pgfqpoint{1.254980in}{0.150000in}}{\pgfqpoint{5.490039in}{5.490039in}}%
\pgfusepath{clip}%
\pgfsetbuttcap%
\pgfsetroundjoin%
\definecolor{currentfill}{rgb}{0.243113,0.292092,0.538516}%
\pgfsetfillcolor{currentfill}%
\pgfsetfillopacity{0.700000}%
\pgfsetlinewidth{0.000000pt}%
\definecolor{currentstroke}{rgb}{0.000000,0.000000,0.000000}%
\pgfsetstrokecolor{currentstroke}%
\pgfsetdash{}{0pt}%
\pgfpathmoveto{\pgfqpoint{4.542025in}{1.373860in}}%
\pgfpathlineto{\pgfqpoint{4.556109in}{1.365892in}}%
\pgfpathlineto{\pgfqpoint{4.570199in}{1.357947in}}%
\pgfpathlineto{\pgfqpoint{4.584293in}{1.350025in}}%
\pgfpathlineto{\pgfqpoint{4.598392in}{1.342125in}}%
\pgfpathlineto{\pgfqpoint{4.590710in}{1.353481in}}%
\pgfpathlineto{\pgfqpoint{4.583022in}{1.365518in}}%
\pgfpathlineto{\pgfqpoint{4.575328in}{1.378249in}}%
\pgfpathlineto{\pgfqpoint{4.567629in}{1.391688in}}%
\pgfpathlineto{\pgfqpoint{4.553493in}{1.400052in}}%
\pgfpathlineto{\pgfqpoint{4.539362in}{1.408439in}}%
\pgfpathlineto{\pgfqpoint{4.525235in}{1.416848in}}%
\pgfpathlineto{\pgfqpoint{4.511114in}{1.425279in}}%
\pgfpathlineto{\pgfqpoint{4.518851in}{1.411370in}}%
\pgfpathlineto{\pgfqpoint{4.526582in}{1.398172in}}%
\pgfpathlineto{\pgfqpoint{4.534307in}{1.385674in}}%
\pgfpathlineto{\pgfqpoint{4.542025in}{1.373860in}}%
\pgfpathclose%
\pgfusepath{fill}%
\end{pgfscope}%
\begin{pgfscope}%
\pgfpathrectangle{\pgfqpoint{1.254980in}{0.150000in}}{\pgfqpoint{5.490039in}{5.490039in}}%
\pgfusepath{clip}%
\pgfsetbuttcap%
\pgfsetroundjoin%
\definecolor{currentfill}{rgb}{0.149039,0.508051,0.557250}%
\pgfsetfillcolor{currentfill}%
\pgfsetfillopacity{0.700000}%
\pgfsetlinewidth{0.000000pt}%
\definecolor{currentstroke}{rgb}{0.000000,0.000000,0.000000}%
\pgfsetstrokecolor{currentstroke}%
\pgfsetdash{}{0pt}%
\pgfpathmoveto{\pgfqpoint{3.804998in}{1.943363in}}%
\pgfpathlineto{\pgfqpoint{3.818977in}{1.933325in}}%
\pgfpathlineto{\pgfqpoint{3.832959in}{1.923312in}}%
\pgfpathlineto{\pgfqpoint{3.846944in}{1.913323in}}%
\pgfpathlineto{\pgfqpoint{3.860933in}{1.903358in}}%
\pgfpathlineto{\pgfqpoint{3.852615in}{1.926519in}}%
\pgfpathlineto{\pgfqpoint{3.844277in}{1.950535in}}%
\pgfpathlineto{\pgfqpoint{3.835917in}{1.975423in}}%
\pgfpathlineto{\pgfqpoint{3.821886in}{1.985768in}}%
\pgfpathlineto{\pgfqpoint{3.807858in}{1.996137in}}%
\pgfpathlineto{\pgfqpoint{3.793834in}{2.006531in}}%
\pgfpathlineto{\pgfqpoint{3.779812in}{2.016949in}}%
\pgfpathlineto{\pgfqpoint{3.788230in}{1.991549in}}%
\pgfpathlineto{\pgfqpoint{3.796625in}{1.967026in}}%
\pgfpathlineto{\pgfqpoint{3.804998in}{1.943363in}}%
\pgfpathclose%
\pgfusepath{fill}%
\end{pgfscope}%
\begin{pgfscope}%
\pgfpathrectangle{\pgfqpoint{1.254980in}{0.150000in}}{\pgfqpoint{5.490039in}{5.490039in}}%
\pgfusepath{clip}%
\pgfsetbuttcap%
\pgfsetroundjoin%
\definecolor{currentfill}{rgb}{0.199430,0.387607,0.554642}%
\pgfsetfillcolor{currentfill}%
\pgfsetfillopacity{0.700000}%
\pgfsetlinewidth{0.000000pt}%
\definecolor{currentstroke}{rgb}{0.000000,0.000000,0.000000}%
\pgfsetstrokecolor{currentstroke}%
\pgfsetdash{}{0pt}%
\pgfpathmoveto{\pgfqpoint{4.229620in}{1.598666in}}%
\pgfpathlineto{\pgfqpoint{4.243653in}{1.589780in}}%
\pgfpathlineto{\pgfqpoint{4.257691in}{1.580916in}}%
\pgfpathlineto{\pgfqpoint{4.271733in}{1.572076in}}%
\pgfpathlineto{\pgfqpoint{4.285780in}{1.563259in}}%
\pgfpathlineto{\pgfqpoint{4.277868in}{1.579800in}}%
\pgfpathlineto{\pgfqpoint{4.269946in}{1.597101in}}%
\pgfpathlineto{\pgfqpoint{4.262012in}{1.615177in}}%
\pgfpathlineto{\pgfqpoint{4.254066in}{1.634042in}}%
\pgfpathlineto{\pgfqpoint{4.239974in}{1.643344in}}%
\pgfpathlineto{\pgfqpoint{4.225887in}{1.652668in}}%
\pgfpathlineto{\pgfqpoint{4.211803in}{1.662016in}}%
\pgfpathlineto{\pgfqpoint{4.197723in}{1.671388in}}%
\pgfpathlineto{\pgfqpoint{4.205717in}{1.652031in}}%
\pgfpathlineto{\pgfqpoint{4.213697in}{1.633468in}}%
\pgfpathlineto{\pgfqpoint{4.221665in}{1.615685in}}%
\pgfpathlineto{\pgfqpoint{4.229620in}{1.598666in}}%
\pgfpathclose%
\pgfusepath{fill}%
\end{pgfscope}%
\begin{pgfscope}%
\pgfpathrectangle{\pgfqpoint{1.254980in}{0.150000in}}{\pgfqpoint{5.490039in}{5.490039in}}%
\pgfusepath{clip}%
\pgfsetbuttcap%
\pgfsetroundjoin%
\definecolor{currentfill}{rgb}{0.246811,0.283237,0.535941}%
\pgfsetfillcolor{currentfill}%
\pgfsetfillopacity{0.700000}%
\pgfsetlinewidth{0.000000pt}%
\definecolor{currentstroke}{rgb}{0.000000,0.000000,0.000000}%
\pgfsetstrokecolor{currentstroke}%
\pgfsetdash{}{0pt}%
\pgfpathmoveto{\pgfqpoint{4.598392in}{1.342125in}}%
\pgfpathlineto{\pgfqpoint{4.612497in}{1.334247in}}%
\pgfpathlineto{\pgfqpoint{4.626606in}{1.326391in}}%
\pgfpathlineto{\pgfqpoint{4.640720in}{1.318557in}}%
\pgfpathlineto{\pgfqpoint{4.654839in}{1.310746in}}%
\pgfpathlineto{\pgfqpoint{4.647191in}{1.321645in}}%
\pgfpathlineto{\pgfqpoint{4.639539in}{1.333220in}}%
\pgfpathlineto{\pgfqpoint{4.631881in}{1.345485in}}%
\pgfpathlineto{\pgfqpoint{4.624219in}{1.358454in}}%
\pgfpathlineto{\pgfqpoint{4.610064in}{1.366729in}}%
\pgfpathlineto{\pgfqpoint{4.595914in}{1.375026in}}%
\pgfpathlineto{\pgfqpoint{4.581769in}{1.383346in}}%
\pgfpathlineto{\pgfqpoint{4.567629in}{1.391688in}}%
\pgfpathlineto{\pgfqpoint{4.575328in}{1.378249in}}%
\pgfpathlineto{\pgfqpoint{4.583022in}{1.365518in}}%
\pgfpathlineto{\pgfqpoint{4.590710in}{1.353481in}}%
\pgfpathlineto{\pgfqpoint{4.598392in}{1.342125in}}%
\pgfpathclose%
\pgfusepath{fill}%
\end{pgfscope}%
\begin{pgfscope}%
\pgfpathrectangle{\pgfqpoint{1.254980in}{0.150000in}}{\pgfqpoint{5.490039in}{5.490039in}}%
\pgfusepath{clip}%
\pgfsetbuttcap%
\pgfsetroundjoin%
\definecolor{currentfill}{rgb}{0.153364,0.497000,0.557724}%
\pgfsetfillcolor{currentfill}%
\pgfsetfillopacity{0.700000}%
\pgfsetlinewidth{0.000000pt}%
\definecolor{currentstroke}{rgb}{0.000000,0.000000,0.000000}%
\pgfsetstrokecolor{currentstroke}%
\pgfsetdash{}{0pt}%
\pgfpathmoveto{\pgfqpoint{3.860933in}{1.903358in}}%
\pgfpathlineto{\pgfqpoint{3.874925in}{1.893418in}}%
\pgfpathlineto{\pgfqpoint{3.888921in}{1.883502in}}%
\pgfpathlineto{\pgfqpoint{3.902920in}{1.873610in}}%
\pgfpathlineto{\pgfqpoint{3.916922in}{1.863742in}}%
\pgfpathlineto{\pgfqpoint{3.908658in}{1.886402in}}%
\pgfpathlineto{\pgfqpoint{3.900376in}{1.909912in}}%
\pgfpathlineto{\pgfqpoint{3.892073in}{1.934289in}}%
\pgfpathlineto{\pgfqpoint{3.878029in}{1.944536in}}%
\pgfpathlineto{\pgfqpoint{3.863988in}{1.954807in}}%
\pgfpathlineto{\pgfqpoint{3.849951in}{1.965103in}}%
\pgfpathlineto{\pgfqpoint{3.835917in}{1.975423in}}%
\pgfpathlineto{\pgfqpoint{3.844277in}{1.950535in}}%
\pgfpathlineto{\pgfqpoint{3.852615in}{1.926519in}}%
\pgfpathlineto{\pgfqpoint{3.860933in}{1.903358in}}%
\pgfpathclose%
\pgfusepath{fill}%
\end{pgfscope}%
\begin{pgfscope}%
\pgfpathrectangle{\pgfqpoint{1.254980in}{0.150000in}}{\pgfqpoint{5.490039in}{5.490039in}}%
\pgfusepath{clip}%
\pgfsetbuttcap%
\pgfsetroundjoin%
\definecolor{currentfill}{rgb}{0.204903,0.375746,0.553533}%
\pgfsetfillcolor{currentfill}%
\pgfsetfillopacity{0.700000}%
\pgfsetlinewidth{0.000000pt}%
\definecolor{currentstroke}{rgb}{0.000000,0.000000,0.000000}%
\pgfsetstrokecolor{currentstroke}%
\pgfsetdash{}{0pt}%
\pgfpathmoveto{\pgfqpoint{4.285780in}{1.563259in}}%
\pgfpathlineto{\pgfqpoint{4.299830in}{1.554464in}}%
\pgfpathlineto{\pgfqpoint{4.313885in}{1.545693in}}%
\pgfpathlineto{\pgfqpoint{4.327944in}{1.536944in}}%
\pgfpathlineto{\pgfqpoint{4.342007in}{1.528218in}}%
\pgfpathlineto{\pgfqpoint{4.334139in}{1.544283in}}%
\pgfpathlineto{\pgfqpoint{4.326261in}{1.561102in}}%
\pgfpathlineto{\pgfqpoint{4.318372in}{1.578691in}}%
\pgfpathlineto{\pgfqpoint{4.310472in}{1.597066in}}%
\pgfpathlineto{\pgfqpoint{4.296364in}{1.606275in}}%
\pgfpathlineto{\pgfqpoint{4.282261in}{1.615508in}}%
\pgfpathlineto{\pgfqpoint{4.268161in}{1.624763in}}%
\pgfpathlineto{\pgfqpoint{4.254066in}{1.634042in}}%
\pgfpathlineto{\pgfqpoint{4.262012in}{1.615177in}}%
\pgfpathlineto{\pgfqpoint{4.269946in}{1.597101in}}%
\pgfpathlineto{\pgfqpoint{4.277868in}{1.579800in}}%
\pgfpathlineto{\pgfqpoint{4.285780in}{1.563259in}}%
\pgfpathclose%
\pgfusepath{fill}%
\end{pgfscope}%
\begin{pgfscope}%
\pgfpathrectangle{\pgfqpoint{1.254980in}{0.150000in}}{\pgfqpoint{5.490039in}{5.490039in}}%
\pgfusepath{clip}%
\pgfsetbuttcap%
\pgfsetroundjoin%
\definecolor{currentfill}{rgb}{0.157729,0.485932,0.558013}%
\pgfsetfillcolor{currentfill}%
\pgfsetfillopacity{0.700000}%
\pgfsetlinewidth{0.000000pt}%
\definecolor{currentstroke}{rgb}{0.000000,0.000000,0.000000}%
\pgfsetstrokecolor{currentstroke}%
\pgfsetdash{}{0pt}%
\pgfpathmoveto{\pgfqpoint{3.916922in}{1.863742in}}%
\pgfpathlineto{\pgfqpoint{3.930928in}{1.853898in}}%
\pgfpathlineto{\pgfqpoint{3.944937in}{1.844079in}}%
\pgfpathlineto{\pgfqpoint{3.958950in}{1.834283in}}%
\pgfpathlineto{\pgfqpoint{3.972966in}{1.824512in}}%
\pgfpathlineto{\pgfqpoint{3.964756in}{1.846672in}}%
\pgfpathlineto{\pgfqpoint{3.956527in}{1.869677in}}%
\pgfpathlineto{\pgfqpoint{3.948280in}{1.893543in}}%
\pgfpathlineto{\pgfqpoint{3.934223in}{1.903693in}}%
\pgfpathlineto{\pgfqpoint{3.920170in}{1.913868in}}%
\pgfpathlineto{\pgfqpoint{3.906120in}{1.924066in}}%
\pgfpathlineto{\pgfqpoint{3.892073in}{1.934289in}}%
\pgfpathlineto{\pgfqpoint{3.900376in}{1.909912in}}%
\pgfpathlineto{\pgfqpoint{3.908658in}{1.886402in}}%
\pgfpathlineto{\pgfqpoint{3.916922in}{1.863742in}}%
\pgfpathclose%
\pgfusepath{fill}%
\end{pgfscope}%
\begin{pgfscope}%
\pgfpathrectangle{\pgfqpoint{1.254980in}{0.150000in}}{\pgfqpoint{5.490039in}{5.490039in}}%
\pgfusepath{clip}%
\pgfsetbuttcap%
\pgfsetroundjoin%
\definecolor{currentfill}{rgb}{0.250425,0.274290,0.533103}%
\pgfsetfillcolor{currentfill}%
\pgfsetfillopacity{0.700000}%
\pgfsetlinewidth{0.000000pt}%
\definecolor{currentstroke}{rgb}{0.000000,0.000000,0.000000}%
\pgfsetstrokecolor{currentstroke}%
\pgfsetdash{}{0pt}%
\pgfpathmoveto{\pgfqpoint{4.654839in}{1.310746in}}%
\pgfpathlineto{\pgfqpoint{4.668964in}{1.302957in}}%
\pgfpathlineto{\pgfqpoint{4.683093in}{1.295190in}}%
\pgfpathlineto{\pgfqpoint{4.697228in}{1.287445in}}%
\pgfpathlineto{\pgfqpoint{4.711367in}{1.279723in}}%
\pgfpathlineto{\pgfqpoint{4.703753in}{1.290165in}}%
\pgfpathlineto{\pgfqpoint{4.696134in}{1.301279in}}%
\pgfpathlineto{\pgfqpoint{4.688513in}{1.313078in}}%
\pgfpathlineto{\pgfqpoint{4.680887in}{1.325577in}}%
\pgfpathlineto{\pgfqpoint{4.666712in}{1.333763in}}%
\pgfpathlineto{\pgfqpoint{4.652543in}{1.341971in}}%
\pgfpathlineto{\pgfqpoint{4.638379in}{1.350202in}}%
\pgfpathlineto{\pgfqpoint{4.624219in}{1.358454in}}%
\pgfpathlineto{\pgfqpoint{4.631881in}{1.345485in}}%
\pgfpathlineto{\pgfqpoint{4.639539in}{1.333220in}}%
\pgfpathlineto{\pgfqpoint{4.647191in}{1.321645in}}%
\pgfpathlineto{\pgfqpoint{4.654839in}{1.310746in}}%
\pgfpathclose%
\pgfusepath{fill}%
\end{pgfscope}%
\begin{pgfscope}%
\pgfpathrectangle{\pgfqpoint{1.254980in}{0.150000in}}{\pgfqpoint{5.490039in}{5.490039in}}%
\pgfusepath{clip}%
\pgfsetbuttcap%
\pgfsetroundjoin%
\definecolor{currentfill}{rgb}{0.208623,0.367752,0.552675}%
\pgfsetfillcolor{currentfill}%
\pgfsetfillopacity{0.700000}%
\pgfsetlinewidth{0.000000pt}%
\definecolor{currentstroke}{rgb}{0.000000,0.000000,0.000000}%
\pgfsetstrokecolor{currentstroke}%
\pgfsetdash{}{0pt}%
\pgfpathmoveto{\pgfqpoint{4.342007in}{1.528218in}}%
\pgfpathlineto{\pgfqpoint{4.356075in}{1.519515in}}%
\pgfpathlineto{\pgfqpoint{4.370147in}{1.510835in}}%
\pgfpathlineto{\pgfqpoint{4.384224in}{1.502178in}}%
\pgfpathlineto{\pgfqpoint{4.398304in}{1.493543in}}%
\pgfpathlineto{\pgfqpoint{4.390478in}{1.509131in}}%
\pgfpathlineto{\pgfqpoint{4.382643in}{1.525469in}}%
\pgfpathlineto{\pgfqpoint{4.374798in}{1.542573in}}%
\pgfpathlineto{\pgfqpoint{4.366944in}{1.560456in}}%
\pgfpathlineto{\pgfqpoint{4.352820in}{1.569574in}}%
\pgfpathlineto{\pgfqpoint{4.338700in}{1.578715in}}%
\pgfpathlineto{\pgfqpoint{4.324584in}{1.587879in}}%
\pgfpathlineto{\pgfqpoint{4.310472in}{1.597066in}}%
\pgfpathlineto{\pgfqpoint{4.318372in}{1.578691in}}%
\pgfpathlineto{\pgfqpoint{4.326261in}{1.561102in}}%
\pgfpathlineto{\pgfqpoint{4.334139in}{1.544283in}}%
\pgfpathlineto{\pgfqpoint{4.342007in}{1.528218in}}%
\pgfpathclose%
\pgfusepath{fill}%
\end{pgfscope}%
\begin{pgfscope}%
\pgfpathrectangle{\pgfqpoint{1.254980in}{0.150000in}}{\pgfqpoint{5.490039in}{5.490039in}}%
\pgfusepath{clip}%
\pgfsetbuttcap%
\pgfsetroundjoin%
\definecolor{currentfill}{rgb}{0.162142,0.474838,0.558140}%
\pgfsetfillcolor{currentfill}%
\pgfsetfillopacity{0.700000}%
\pgfsetlinewidth{0.000000pt}%
\definecolor{currentstroke}{rgb}{0.000000,0.000000,0.000000}%
\pgfsetstrokecolor{currentstroke}%
\pgfsetdash{}{0pt}%
\pgfpathmoveto{\pgfqpoint{3.972966in}{1.824512in}}%
\pgfpathlineto{\pgfqpoint{3.986986in}{1.814764in}}%
\pgfpathlineto{\pgfqpoint{4.001009in}{1.805041in}}%
\pgfpathlineto{\pgfqpoint{4.015036in}{1.795341in}}%
\pgfpathlineto{\pgfqpoint{4.029067in}{1.785664in}}%
\pgfpathlineto{\pgfqpoint{4.020909in}{1.807325in}}%
\pgfpathlineto{\pgfqpoint{4.012734in}{1.829825in}}%
\pgfpathlineto{\pgfqpoint{4.004541in}{1.853182in}}%
\pgfpathlineto{\pgfqpoint{3.990471in}{1.863237in}}%
\pgfpathlineto{\pgfqpoint{3.976404in}{1.873315in}}%
\pgfpathlineto{\pgfqpoint{3.962340in}{1.883417in}}%
\pgfpathlineto{\pgfqpoint{3.948280in}{1.893543in}}%
\pgfpathlineto{\pgfqpoint{3.956527in}{1.869677in}}%
\pgfpathlineto{\pgfqpoint{3.964756in}{1.846672in}}%
\pgfpathlineto{\pgfqpoint{3.972966in}{1.824512in}}%
\pgfpathclose%
\pgfusepath{fill}%
\end{pgfscope}%
\begin{pgfscope}%
\pgfpathrectangle{\pgfqpoint{1.254980in}{0.150000in}}{\pgfqpoint{5.490039in}{5.490039in}}%
\pgfusepath{clip}%
\pgfsetbuttcap%
\pgfsetroundjoin%
\definecolor{currentfill}{rgb}{0.253935,0.265254,0.529983}%
\pgfsetfillcolor{currentfill}%
\pgfsetfillopacity{0.700000}%
\pgfsetlinewidth{0.000000pt}%
\definecolor{currentstroke}{rgb}{0.000000,0.000000,0.000000}%
\pgfsetstrokecolor{currentstroke}%
\pgfsetdash{}{0pt}%
\pgfpathmoveto{\pgfqpoint{4.711367in}{1.279723in}}%
\pgfpathlineto{\pgfqpoint{4.725512in}{1.272023in}}%
\pgfpathlineto{\pgfqpoint{4.739662in}{1.264344in}}%
\pgfpathlineto{\pgfqpoint{4.753817in}{1.256688in}}%
\pgfpathlineto{\pgfqpoint{4.746227in}{1.266788in}}%
\pgfpathlineto{\pgfqpoint{4.738634in}{1.277556in}}%
\pgfpathlineto{\pgfqpoint{4.731038in}{1.289006in}}%
\pgfpathlineto{\pgfqpoint{4.723439in}{1.301153in}}%
\pgfpathlineto{\pgfqpoint{4.709250in}{1.309272in}}%
\pgfpathlineto{\pgfqpoint{4.695066in}{1.317414in}}%
\pgfpathlineto{\pgfqpoint{4.680887in}{1.325577in}}%
\pgfpathlineto{\pgfqpoint{4.688513in}{1.313078in}}%
\pgfpathlineto{\pgfqpoint{4.696134in}{1.301279in}}%
\pgfpathlineto{\pgfqpoint{4.703753in}{1.290165in}}%
\pgfpathlineto{\pgfqpoint{4.711367in}{1.279723in}}%
\pgfpathclose%
\pgfusepath{fill}%
\end{pgfscope}%
\begin{pgfscope}%
\pgfpathrectangle{\pgfqpoint{1.254980in}{0.150000in}}{\pgfqpoint{5.490039in}{5.490039in}}%
\pgfusepath{clip}%
\pgfsetbuttcap%
\pgfsetroundjoin%
\definecolor{currentfill}{rgb}{0.166617,0.463708,0.558119}%
\pgfsetfillcolor{currentfill}%
\pgfsetfillopacity{0.700000}%
\pgfsetlinewidth{0.000000pt}%
\definecolor{currentstroke}{rgb}{0.000000,0.000000,0.000000}%
\pgfsetstrokecolor{currentstroke}%
\pgfsetdash{}{0pt}%
\pgfpathmoveto{\pgfqpoint{4.029067in}{1.785664in}}%
\pgfpathlineto{\pgfqpoint{4.043101in}{1.776012in}}%
\pgfpathlineto{\pgfqpoint{4.057139in}{1.766383in}}%
\pgfpathlineto{\pgfqpoint{4.071180in}{1.756778in}}%
\pgfpathlineto{\pgfqpoint{4.085225in}{1.747196in}}%
\pgfpathlineto{\pgfqpoint{4.077119in}{1.768358in}}%
\pgfpathlineto{\pgfqpoint{4.068997in}{1.790355in}}%
\pgfpathlineto{\pgfqpoint{4.060858in}{1.813204in}}%
\pgfpathlineto{\pgfqpoint{4.046773in}{1.823163in}}%
\pgfpathlineto{\pgfqpoint{4.032693in}{1.833145in}}%
\pgfpathlineto{\pgfqpoint{4.018615in}{1.843152in}}%
\pgfpathlineto{\pgfqpoint{4.004541in}{1.853182in}}%
\pgfpathlineto{\pgfqpoint{4.012734in}{1.829825in}}%
\pgfpathlineto{\pgfqpoint{4.020909in}{1.807325in}}%
\pgfpathlineto{\pgfqpoint{4.029067in}{1.785664in}}%
\pgfpathclose%
\pgfusepath{fill}%
\end{pgfscope}%
\begin{pgfscope}%
\pgfpathrectangle{\pgfqpoint{1.254980in}{0.150000in}}{\pgfqpoint{5.490039in}{5.490039in}}%
\pgfusepath{clip}%
\pgfsetbuttcap%
\pgfsetroundjoin%
\definecolor{currentfill}{rgb}{0.212395,0.359683,0.551710}%
\pgfsetfillcolor{currentfill}%
\pgfsetfillopacity{0.700000}%
\pgfsetlinewidth{0.000000pt}%
\definecolor{currentstroke}{rgb}{0.000000,0.000000,0.000000}%
\pgfsetstrokecolor{currentstroke}%
\pgfsetdash{}{0pt}%
\pgfpathmoveto{\pgfqpoint{4.398304in}{1.493543in}}%
\pgfpathlineto{\pgfqpoint{4.412390in}{1.484931in}}%
\pgfpathlineto{\pgfqpoint{4.426480in}{1.476341in}}%
\pgfpathlineto{\pgfqpoint{4.440574in}{1.467775in}}%
\pgfpathlineto{\pgfqpoint{4.454673in}{1.459230in}}%
\pgfpathlineto{\pgfqpoint{4.446888in}{1.474342in}}%
\pgfpathlineto{\pgfqpoint{4.439095in}{1.490200in}}%
\pgfpathlineto{\pgfqpoint{4.431293in}{1.506819in}}%
\pgfpathlineto{\pgfqpoint{4.423483in}{1.524213in}}%
\pgfpathlineto{\pgfqpoint{4.409342in}{1.533239in}}%
\pgfpathlineto{\pgfqpoint{4.395205in}{1.542289in}}%
\pgfpathlineto{\pgfqpoint{4.381072in}{1.551361in}}%
\pgfpathlineto{\pgfqpoint{4.366944in}{1.560456in}}%
\pgfpathlineto{\pgfqpoint{4.374798in}{1.542573in}}%
\pgfpathlineto{\pgfqpoint{4.382643in}{1.525469in}}%
\pgfpathlineto{\pgfqpoint{4.390478in}{1.509131in}}%
\pgfpathlineto{\pgfqpoint{4.398304in}{1.493543in}}%
\pgfpathclose%
\pgfusepath{fill}%
\end{pgfscope}%
\begin{pgfscope}%
\pgfpathrectangle{\pgfqpoint{1.254980in}{0.150000in}}{\pgfqpoint{5.490039in}{5.490039in}}%
\pgfusepath{clip}%
\pgfsetbuttcap%
\pgfsetroundjoin%
\definecolor{currentfill}{rgb}{0.169646,0.456262,0.558030}%
\pgfsetfillcolor{currentfill}%
\pgfsetfillopacity{0.700000}%
\pgfsetlinewidth{0.000000pt}%
\definecolor{currentstroke}{rgb}{0.000000,0.000000,0.000000}%
\pgfsetstrokecolor{currentstroke}%
\pgfsetdash{}{0pt}%
\pgfpathmoveto{\pgfqpoint{4.085225in}{1.747196in}}%
\pgfpathlineto{\pgfqpoint{4.099274in}{1.737638in}}%
\pgfpathlineto{\pgfqpoint{4.113327in}{1.728104in}}%
\pgfpathlineto{\pgfqpoint{4.127383in}{1.718593in}}%
\pgfpathlineto{\pgfqpoint{4.141444in}{1.709105in}}%
\pgfpathlineto{\pgfqpoint{4.133388in}{1.729769in}}%
\pgfpathlineto{\pgfqpoint{4.125317in}{1.751263in}}%
\pgfpathlineto{\pgfqpoint{4.117231in}{1.773604in}}%
\pgfpathlineto{\pgfqpoint{4.103132in}{1.783468in}}%
\pgfpathlineto{\pgfqpoint{4.089037in}{1.793356in}}%
\pgfpathlineto{\pgfqpoint{4.074946in}{1.803268in}}%
\pgfpathlineto{\pgfqpoint{4.060858in}{1.813204in}}%
\pgfpathlineto{\pgfqpoint{4.068997in}{1.790355in}}%
\pgfpathlineto{\pgfqpoint{4.077119in}{1.768358in}}%
\pgfpathlineto{\pgfqpoint{4.085225in}{1.747196in}}%
\pgfpathclose%
\pgfusepath{fill}%
\end{pgfscope}%
\begin{pgfscope}%
\pgfpathrectangle{\pgfqpoint{1.254980in}{0.150000in}}{\pgfqpoint{5.490039in}{5.490039in}}%
\pgfusepath{clip}%
\pgfsetbuttcap%
\pgfsetroundjoin%
\definecolor{currentfill}{rgb}{0.218130,0.347432,0.550038}%
\pgfsetfillcolor{currentfill}%
\pgfsetfillopacity{0.700000}%
\pgfsetlinewidth{0.000000pt}%
\definecolor{currentstroke}{rgb}{0.000000,0.000000,0.000000}%
\pgfsetstrokecolor{currentstroke}%
\pgfsetdash{}{0pt}%
\pgfpathmoveto{\pgfqpoint{4.454673in}{1.459230in}}%
\pgfpathlineto{\pgfqpoint{4.468776in}{1.450709in}}%
\pgfpathlineto{\pgfqpoint{4.482884in}{1.442210in}}%
\pgfpathlineto{\pgfqpoint{4.496997in}{1.433733in}}%
\pgfpathlineto{\pgfqpoint{4.511114in}{1.425279in}}%
\pgfpathlineto{\pgfqpoint{4.503369in}{1.439916in}}%
\pgfpathlineto{\pgfqpoint{4.495617in}{1.455293in}}%
\pgfpathlineto{\pgfqpoint{4.487858in}{1.471427in}}%
\pgfpathlineto{\pgfqpoint{4.480091in}{1.488332in}}%
\pgfpathlineto{\pgfqpoint{4.465932in}{1.497268in}}%
\pgfpathlineto{\pgfqpoint{4.451778in}{1.506227in}}%
\pgfpathlineto{\pgfqpoint{4.437628in}{1.515208in}}%
\pgfpathlineto{\pgfqpoint{4.423483in}{1.524213in}}%
\pgfpathlineto{\pgfqpoint{4.431293in}{1.506819in}}%
\pgfpathlineto{\pgfqpoint{4.439095in}{1.490200in}}%
\pgfpathlineto{\pgfqpoint{4.446888in}{1.474342in}}%
\pgfpathlineto{\pgfqpoint{4.454673in}{1.459230in}}%
\pgfpathclose%
\pgfusepath{fill}%
\end{pgfscope}%
\begin{pgfscope}%
\pgfpathrectangle{\pgfqpoint{1.254980in}{0.150000in}}{\pgfqpoint{5.490039in}{5.490039in}}%
\pgfusepath{clip}%
\pgfsetbuttcap%
\pgfsetroundjoin%
\definecolor{currentfill}{rgb}{0.174274,0.445044,0.557792}%
\pgfsetfillcolor{currentfill}%
\pgfsetfillopacity{0.700000}%
\pgfsetlinewidth{0.000000pt}%
\definecolor{currentstroke}{rgb}{0.000000,0.000000,0.000000}%
\pgfsetstrokecolor{currentstroke}%
\pgfsetdash{}{0pt}%
\pgfpathmoveto{\pgfqpoint{4.141444in}{1.709105in}}%
\pgfpathlineto{\pgfqpoint{4.155508in}{1.699641in}}%
\pgfpathlineto{\pgfqpoint{4.169576in}{1.690200in}}%
\pgfpathlineto{\pgfqpoint{4.183648in}{1.680782in}}%
\pgfpathlineto{\pgfqpoint{4.197723in}{1.671388in}}%
\pgfpathlineto{\pgfqpoint{4.189717in}{1.691554in}}%
\pgfpathlineto{\pgfqpoint{4.181696in}{1.712546in}}%
\pgfpathlineto{\pgfqpoint{4.173662in}{1.734380in}}%
\pgfpathlineto{\pgfqpoint{4.159548in}{1.744151in}}%
\pgfpathlineto{\pgfqpoint{4.145439in}{1.753945in}}%
\pgfpathlineto{\pgfqpoint{4.131333in}{1.763763in}}%
\pgfpathlineto{\pgfqpoint{4.117231in}{1.773604in}}%
\pgfpathlineto{\pgfqpoint{4.125317in}{1.751263in}}%
\pgfpathlineto{\pgfqpoint{4.133388in}{1.729769in}}%
\pgfpathlineto{\pgfqpoint{4.141444in}{1.709105in}}%
\pgfpathclose%
\pgfusepath{fill}%
\end{pgfscope}%
\begin{pgfscope}%
\pgfpathrectangle{\pgfqpoint{1.254980in}{0.150000in}}{\pgfqpoint{5.490039in}{5.490039in}}%
\pgfusepath{clip}%
\pgfsetbuttcap%
\pgfsetroundjoin%
\definecolor{currentfill}{rgb}{0.221989,0.339161,0.548752}%
\pgfsetfillcolor{currentfill}%
\pgfsetfillopacity{0.700000}%
\pgfsetlinewidth{0.000000pt}%
\definecolor{currentstroke}{rgb}{0.000000,0.000000,0.000000}%
\pgfsetstrokecolor{currentstroke}%
\pgfsetdash{}{0pt}%
\pgfpathmoveto{\pgfqpoint{4.511114in}{1.425279in}}%
\pgfpathlineto{\pgfqpoint{4.525235in}{1.416848in}}%
\pgfpathlineto{\pgfqpoint{4.539362in}{1.408439in}}%
\pgfpathlineto{\pgfqpoint{4.553493in}{1.400052in}}%
\pgfpathlineto{\pgfqpoint{4.567629in}{1.391688in}}%
\pgfpathlineto{\pgfqpoint{4.559923in}{1.405849in}}%
\pgfpathlineto{\pgfqpoint{4.552211in}{1.420747in}}%
\pgfpathlineto{\pgfqpoint{4.544493in}{1.436397in}}%
\pgfpathlineto{\pgfqpoint{4.536768in}{1.452813in}}%
\pgfpathlineto{\pgfqpoint{4.522592in}{1.461659in}}%
\pgfpathlineto{\pgfqpoint{4.508421in}{1.470527in}}%
\pgfpathlineto{\pgfqpoint{4.494253in}{1.479418in}}%
\pgfpathlineto{\pgfqpoint{4.480091in}{1.488332in}}%
\pgfpathlineto{\pgfqpoint{4.487858in}{1.471427in}}%
\pgfpathlineto{\pgfqpoint{4.495617in}{1.455293in}}%
\pgfpathlineto{\pgfqpoint{4.503369in}{1.439916in}}%
\pgfpathlineto{\pgfqpoint{4.511114in}{1.425279in}}%
\pgfpathclose%
\pgfusepath{fill}%
\end{pgfscope}%
\begin{pgfscope}%
\pgfpathrectangle{\pgfqpoint{1.254980in}{0.150000in}}{\pgfqpoint{5.490039in}{5.490039in}}%
\pgfusepath{clip}%
\pgfsetbuttcap%
\pgfsetroundjoin%
\definecolor{currentfill}{rgb}{0.179019,0.433756,0.557430}%
\pgfsetfillcolor{currentfill}%
\pgfsetfillopacity{0.700000}%
\pgfsetlinewidth{0.000000pt}%
\definecolor{currentstroke}{rgb}{0.000000,0.000000,0.000000}%
\pgfsetstrokecolor{currentstroke}%
\pgfsetdash{}{0pt}%
\pgfpathmoveto{\pgfqpoint{4.197723in}{1.671388in}}%
\pgfpathlineto{\pgfqpoint{4.211803in}{1.662016in}}%
\pgfpathlineto{\pgfqpoint{4.225887in}{1.652668in}}%
\pgfpathlineto{\pgfqpoint{4.239974in}{1.643344in}}%
\pgfpathlineto{\pgfqpoint{4.254066in}{1.634042in}}%
\pgfpathlineto{\pgfqpoint{4.246107in}{1.653712in}}%
\pgfpathlineto{\pgfqpoint{4.238136in}{1.674203in}}%
\pgfpathlineto{\pgfqpoint{4.230152in}{1.695530in}}%
\pgfpathlineto{\pgfqpoint{4.216024in}{1.705208in}}%
\pgfpathlineto{\pgfqpoint{4.201899in}{1.714909in}}%
\pgfpathlineto{\pgfqpoint{4.187779in}{1.724633in}}%
\pgfpathlineto{\pgfqpoint{4.173662in}{1.734380in}}%
\pgfpathlineto{\pgfqpoint{4.181696in}{1.712546in}}%
\pgfpathlineto{\pgfqpoint{4.189717in}{1.691554in}}%
\pgfpathlineto{\pgfqpoint{4.197723in}{1.671388in}}%
\pgfpathclose%
\pgfusepath{fill}%
\end{pgfscope}%
\begin{pgfscope}%
\pgfpathrectangle{\pgfqpoint{1.254980in}{0.150000in}}{\pgfqpoint{5.490039in}{5.490039in}}%
\pgfusepath{clip}%
\pgfsetbuttcap%
\pgfsetroundjoin%
\definecolor{currentfill}{rgb}{0.225863,0.330805,0.547314}%
\pgfsetfillcolor{currentfill}%
\pgfsetfillopacity{0.700000}%
\pgfsetlinewidth{0.000000pt}%
\definecolor{currentstroke}{rgb}{0.000000,0.000000,0.000000}%
\pgfsetstrokecolor{currentstroke}%
\pgfsetdash{}{0pt}%
\pgfpathmoveto{\pgfqpoint{4.567629in}{1.391688in}}%
\pgfpathlineto{\pgfqpoint{4.581769in}{1.383346in}}%
\pgfpathlineto{\pgfqpoint{4.595914in}{1.375026in}}%
\pgfpathlineto{\pgfqpoint{4.610064in}{1.366729in}}%
\pgfpathlineto{\pgfqpoint{4.624219in}{1.358454in}}%
\pgfpathlineto{\pgfqpoint{4.616552in}{1.372141in}}%
\pgfpathlineto{\pgfqpoint{4.608879in}{1.386560in}}%
\pgfpathlineto{\pgfqpoint{4.601201in}{1.401726in}}%
\pgfpathlineto{\pgfqpoint{4.593517in}{1.417654in}}%
\pgfpathlineto{\pgfqpoint{4.579323in}{1.426410in}}%
\pgfpathlineto{\pgfqpoint{4.565134in}{1.435189in}}%
\pgfpathlineto{\pgfqpoint{4.550949in}{1.443990in}}%
\pgfpathlineto{\pgfqpoint{4.536768in}{1.452813in}}%
\pgfpathlineto{\pgfqpoint{4.544493in}{1.436397in}}%
\pgfpathlineto{\pgfqpoint{4.552211in}{1.420747in}}%
\pgfpathlineto{\pgfqpoint{4.559923in}{1.405849in}}%
\pgfpathlineto{\pgfqpoint{4.567629in}{1.391688in}}%
\pgfpathclose%
\pgfusepath{fill}%
\end{pgfscope}%
\begin{pgfscope}%
\pgfpathrectangle{\pgfqpoint{1.254980in}{0.150000in}}{\pgfqpoint{5.490039in}{5.490039in}}%
\pgfusepath{clip}%
\pgfsetbuttcap%
\pgfsetroundjoin%
\definecolor{currentfill}{rgb}{0.183898,0.422383,0.556944}%
\pgfsetfillcolor{currentfill}%
\pgfsetfillopacity{0.700000}%
\pgfsetlinewidth{0.000000pt}%
\definecolor{currentstroke}{rgb}{0.000000,0.000000,0.000000}%
\pgfsetstrokecolor{currentstroke}%
\pgfsetdash{}{0pt}%
\pgfpathmoveto{\pgfqpoint{4.254066in}{1.634042in}}%
\pgfpathlineto{\pgfqpoint{4.268161in}{1.624763in}}%
\pgfpathlineto{\pgfqpoint{4.282261in}{1.615508in}}%
\pgfpathlineto{\pgfqpoint{4.296364in}{1.606275in}}%
\pgfpathlineto{\pgfqpoint{4.310472in}{1.597066in}}%
\pgfpathlineto{\pgfqpoint{4.302561in}{1.616240in}}%
\pgfpathlineto{\pgfqpoint{4.294638in}{1.636230in}}%
\pgfpathlineto{\pgfqpoint{4.286704in}{1.657052in}}%
\pgfpathlineto{\pgfqpoint{4.272560in}{1.666637in}}%
\pgfpathlineto{\pgfqpoint{4.258420in}{1.676245in}}%
\pgfpathlineto{\pgfqpoint{4.244284in}{1.685876in}}%
\pgfpathlineto{\pgfqpoint{4.230152in}{1.695530in}}%
\pgfpathlineto{\pgfqpoint{4.238136in}{1.674203in}}%
\pgfpathlineto{\pgfqpoint{4.246107in}{1.653712in}}%
\pgfpathlineto{\pgfqpoint{4.254066in}{1.634042in}}%
\pgfpathclose%
\pgfusepath{fill}%
\end{pgfscope}%
\begin{pgfscope}%
\pgfpathrectangle{\pgfqpoint{1.254980in}{0.150000in}}{\pgfqpoint{5.490039in}{5.490039in}}%
\pgfusepath{clip}%
\pgfsetbuttcap%
\pgfsetroundjoin%
\definecolor{currentfill}{rgb}{0.229739,0.322361,0.545706}%
\pgfsetfillcolor{currentfill}%
\pgfsetfillopacity{0.700000}%
\pgfsetlinewidth{0.000000pt}%
\definecolor{currentstroke}{rgb}{0.000000,0.000000,0.000000}%
\pgfsetstrokecolor{currentstroke}%
\pgfsetdash{}{0pt}%
\pgfpathmoveto{\pgfqpoint{4.624219in}{1.358454in}}%
\pgfpathlineto{\pgfqpoint{4.638379in}{1.350202in}}%
\pgfpathlineto{\pgfqpoint{4.652543in}{1.341971in}}%
\pgfpathlineto{\pgfqpoint{4.666712in}{1.333763in}}%
\pgfpathlineto{\pgfqpoint{4.680887in}{1.325577in}}%
\pgfpathlineto{\pgfqpoint{4.673257in}{1.338790in}}%
\pgfpathlineto{\pgfqpoint{4.665622in}{1.352730in}}%
\pgfpathlineto{\pgfqpoint{4.657983in}{1.367413in}}%
\pgfpathlineto{\pgfqpoint{4.650340in}{1.382854in}}%
\pgfpathlineto{\pgfqpoint{4.636127in}{1.391520in}}%
\pgfpathlineto{\pgfqpoint{4.621919in}{1.400209in}}%
\pgfpathlineto{\pgfqpoint{4.607716in}{1.408920in}}%
\pgfpathlineto{\pgfqpoint{4.593517in}{1.417654in}}%
\pgfpathlineto{\pgfqpoint{4.601201in}{1.401726in}}%
\pgfpathlineto{\pgfqpoint{4.608879in}{1.386560in}}%
\pgfpathlineto{\pgfqpoint{4.616552in}{1.372141in}}%
\pgfpathlineto{\pgfqpoint{4.624219in}{1.358454in}}%
\pgfpathclose%
\pgfusepath{fill}%
\end{pgfscope}%
\begin{pgfscope}%
\pgfpathrectangle{\pgfqpoint{1.254980in}{0.150000in}}{\pgfqpoint{5.490039in}{5.490039in}}%
\pgfusepath{clip}%
\pgfsetbuttcap%
\pgfsetroundjoin%
\definecolor{currentfill}{rgb}{0.187231,0.414746,0.556547}%
\pgfsetfillcolor{currentfill}%
\pgfsetfillopacity{0.700000}%
\pgfsetlinewidth{0.000000pt}%
\definecolor{currentstroke}{rgb}{0.000000,0.000000,0.000000}%
\pgfsetstrokecolor{currentstroke}%
\pgfsetdash{}{0pt}%
\pgfpathmoveto{\pgfqpoint{4.310472in}{1.597066in}}%
\pgfpathlineto{\pgfqpoint{4.324584in}{1.587879in}}%
\pgfpathlineto{\pgfqpoint{4.338700in}{1.578715in}}%
\pgfpathlineto{\pgfqpoint{4.352820in}{1.569574in}}%
\pgfpathlineto{\pgfqpoint{4.366944in}{1.560456in}}%
\pgfpathlineto{\pgfqpoint{4.359079in}{1.579136in}}%
\pgfpathlineto{\pgfqpoint{4.351204in}{1.598626in}}%
\pgfpathlineto{\pgfqpoint{4.343318in}{1.618943in}}%
\pgfpathlineto{\pgfqpoint{4.329158in}{1.628436in}}%
\pgfpathlineto{\pgfqpoint{4.315003in}{1.637952in}}%
\pgfpathlineto{\pgfqpoint{4.300851in}{1.647490in}}%
\pgfpathlineto{\pgfqpoint{4.286704in}{1.657052in}}%
\pgfpathlineto{\pgfqpoint{4.294638in}{1.636230in}}%
\pgfpathlineto{\pgfqpoint{4.302561in}{1.616240in}}%
\pgfpathlineto{\pgfqpoint{4.310472in}{1.597066in}}%
\pgfpathclose%
\pgfusepath{fill}%
\end{pgfscope}%
\begin{pgfscope}%
\pgfpathrectangle{\pgfqpoint{1.254980in}{0.150000in}}{\pgfqpoint{5.490039in}{5.490039in}}%
\pgfusepath{clip}%
\pgfsetbuttcap%
\pgfsetroundjoin%
\definecolor{currentfill}{rgb}{0.233603,0.313828,0.543914}%
\pgfsetfillcolor{currentfill}%
\pgfsetfillopacity{0.700000}%
\pgfsetlinewidth{0.000000pt}%
\definecolor{currentstroke}{rgb}{0.000000,0.000000,0.000000}%
\pgfsetstrokecolor{currentstroke}%
\pgfsetdash{}{0pt}%
\pgfpathmoveto{\pgfqpoint{4.680887in}{1.325577in}}%
\pgfpathlineto{\pgfqpoint{4.695066in}{1.317414in}}%
\pgfpathlineto{\pgfqpoint{4.709250in}{1.309272in}}%
\pgfpathlineto{\pgfqpoint{4.723439in}{1.301153in}}%
\pgfpathlineto{\pgfqpoint{4.715836in}{1.314010in}}%
\pgfpathlineto{\pgfqpoint{4.708230in}{1.327592in}}%
\pgfpathlineto{\pgfqpoint{4.700619in}{1.341913in}}%
\pgfpathlineto{\pgfqpoint{4.693005in}{1.356988in}}%
\pgfpathlineto{\pgfqpoint{4.678779in}{1.365587in}}%
\pgfpathlineto{\pgfqpoint{4.664557in}{1.374209in}}%
\pgfpathlineto{\pgfqpoint{4.650340in}{1.382854in}}%
\pgfpathlineto{\pgfqpoint{4.657983in}{1.367413in}}%
\pgfpathlineto{\pgfqpoint{4.665622in}{1.352730in}}%
\pgfpathlineto{\pgfqpoint{4.673257in}{1.338790in}}%
\pgfpathlineto{\pgfqpoint{4.680887in}{1.325577in}}%
\pgfpathclose%
\pgfusepath{fill}%
\end{pgfscope}%
\begin{pgfscope}%
\pgfpathrectangle{\pgfqpoint{1.254980in}{0.150000in}}{\pgfqpoint{5.490039in}{5.490039in}}%
\pgfusepath{clip}%
\pgfsetbuttcap%
\pgfsetroundjoin%
\definecolor{currentfill}{rgb}{0.192357,0.403199,0.555836}%
\pgfsetfillcolor{currentfill}%
\pgfsetfillopacity{0.700000}%
\pgfsetlinewidth{0.000000pt}%
\definecolor{currentstroke}{rgb}{0.000000,0.000000,0.000000}%
\pgfsetstrokecolor{currentstroke}%
\pgfsetdash{}{0pt}%
\pgfpathmoveto{\pgfqpoint{4.366944in}{1.560456in}}%
\pgfpathlineto{\pgfqpoint{4.381072in}{1.551361in}}%
\pgfpathlineto{\pgfqpoint{4.395205in}{1.542289in}}%
\pgfpathlineto{\pgfqpoint{4.409342in}{1.533239in}}%
\pgfpathlineto{\pgfqpoint{4.423483in}{1.524213in}}%
\pgfpathlineto{\pgfqpoint{4.415663in}{1.542397in}}%
\pgfpathlineto{\pgfqpoint{4.407835in}{1.561388in}}%
\pgfpathlineto{\pgfqpoint{4.399996in}{1.581201in}}%
\pgfpathlineto{\pgfqpoint{4.385820in}{1.590603in}}%
\pgfpathlineto{\pgfqpoint{4.371649in}{1.600027in}}%
\pgfpathlineto{\pgfqpoint{4.357481in}{1.609473in}}%
\pgfpathlineto{\pgfqpoint{4.343318in}{1.618943in}}%
\pgfpathlineto{\pgfqpoint{4.351204in}{1.598626in}}%
\pgfpathlineto{\pgfqpoint{4.359079in}{1.579136in}}%
\pgfpathlineto{\pgfqpoint{4.366944in}{1.560456in}}%
\pgfpathclose%
\pgfusepath{fill}%
\end{pgfscope}%
\begin{pgfscope}%
\pgfpathrectangle{\pgfqpoint{1.254980in}{0.150000in}}{\pgfqpoint{5.490039in}{5.490039in}}%
\pgfusepath{clip}%
\pgfsetbuttcap%
\pgfsetroundjoin%
\definecolor{currentfill}{rgb}{0.195860,0.395433,0.555276}%
\pgfsetfillcolor{currentfill}%
\pgfsetfillopacity{0.700000}%
\pgfsetlinewidth{0.000000pt}%
\definecolor{currentstroke}{rgb}{0.000000,0.000000,0.000000}%
\pgfsetstrokecolor{currentstroke}%
\pgfsetdash{}{0pt}%
\pgfpathmoveto{\pgfqpoint{4.423483in}{1.524213in}}%
\pgfpathlineto{\pgfqpoint{4.437628in}{1.515208in}}%
\pgfpathlineto{\pgfqpoint{4.451778in}{1.506227in}}%
\pgfpathlineto{\pgfqpoint{4.465932in}{1.497268in}}%
\pgfpathlineto{\pgfqpoint{4.480091in}{1.488332in}}%
\pgfpathlineto{\pgfqpoint{4.472315in}{1.506023in}}%
\pgfpathlineto{\pgfqpoint{4.464532in}{1.524515in}}%
\pgfpathlineto{\pgfqpoint{4.456740in}{1.543825in}}%
\pgfpathlineto{\pgfqpoint{4.442548in}{1.553135in}}%
\pgfpathlineto{\pgfqpoint{4.428360in}{1.562468in}}%
\pgfpathlineto{\pgfqpoint{4.414176in}{1.571823in}}%
\pgfpathlineto{\pgfqpoint{4.399996in}{1.581201in}}%
\pgfpathlineto{\pgfqpoint{4.407835in}{1.561388in}}%
\pgfpathlineto{\pgfqpoint{4.415663in}{1.542397in}}%
\pgfpathlineto{\pgfqpoint{4.423483in}{1.524213in}}%
\pgfpathclose%
\pgfusepath{fill}%
\end{pgfscope}%
\begin{pgfscope}%
\pgfpathrectangle{\pgfqpoint{1.254980in}{0.150000in}}{\pgfqpoint{5.490039in}{5.490039in}}%
\pgfusepath{clip}%
\pgfsetbuttcap%
\pgfsetroundjoin%
\definecolor{currentfill}{rgb}{0.201239,0.383670,0.554294}%
\pgfsetfillcolor{currentfill}%
\pgfsetfillopacity{0.700000}%
\pgfsetlinewidth{0.000000pt}%
\definecolor{currentstroke}{rgb}{0.000000,0.000000,0.000000}%
\pgfsetstrokecolor{currentstroke}%
\pgfsetdash{}{0pt}%
\pgfpathmoveto{\pgfqpoint{4.480091in}{1.488332in}}%
\pgfpathlineto{\pgfqpoint{4.494253in}{1.479418in}}%
\pgfpathlineto{\pgfqpoint{4.508421in}{1.470527in}}%
\pgfpathlineto{\pgfqpoint{4.522592in}{1.461659in}}%
\pgfpathlineto{\pgfqpoint{4.536768in}{1.452813in}}%
\pgfpathlineto{\pgfqpoint{4.529036in}{1.470010in}}%
\pgfpathlineto{\pgfqpoint{4.521297in}{1.488005in}}%
\pgfpathlineto{\pgfqpoint{4.513550in}{1.506812in}}%
\pgfpathlineto{\pgfqpoint{4.499341in}{1.516031in}}%
\pgfpathlineto{\pgfqpoint{4.485137in}{1.525273in}}%
\pgfpathlineto{\pgfqpoint{4.470936in}{1.534538in}}%
\pgfpathlineto{\pgfqpoint{4.456740in}{1.543825in}}%
\pgfpathlineto{\pgfqpoint{4.464532in}{1.524515in}}%
\pgfpathlineto{\pgfqpoint{4.472315in}{1.506023in}}%
\pgfpathlineto{\pgfqpoint{4.480091in}{1.488332in}}%
\pgfpathclose%
\pgfusepath{fill}%
\end{pgfscope}%
\begin{pgfscope}%
\pgfpathrectangle{\pgfqpoint{1.254980in}{0.150000in}}{\pgfqpoint{5.490039in}{5.490039in}}%
\pgfusepath{clip}%
\pgfsetbuttcap%
\pgfsetroundjoin%
\definecolor{currentfill}{rgb}{0.204903,0.375746,0.553533}%
\pgfsetfillcolor{currentfill}%
\pgfsetfillopacity{0.700000}%
\pgfsetlinewidth{0.000000pt}%
\definecolor{currentstroke}{rgb}{0.000000,0.000000,0.000000}%
\pgfsetstrokecolor{currentstroke}%
\pgfsetdash{}{0pt}%
\pgfpathmoveto{\pgfqpoint{4.536768in}{1.452813in}}%
\pgfpathlineto{\pgfqpoint{4.550949in}{1.443990in}}%
\pgfpathlineto{\pgfqpoint{4.565134in}{1.435189in}}%
\pgfpathlineto{\pgfqpoint{4.579323in}{1.426410in}}%
\pgfpathlineto{\pgfqpoint{4.593517in}{1.417654in}}%
\pgfpathlineto{\pgfqpoint{4.585828in}{1.434359in}}%
\pgfpathlineto{\pgfqpoint{4.578132in}{1.451856in}}%
\pgfpathlineto{\pgfqpoint{4.570429in}{1.470160in}}%
\pgfpathlineto{\pgfqpoint{4.556203in}{1.479289in}}%
\pgfpathlineto{\pgfqpoint{4.541981in}{1.488441in}}%
\pgfpathlineto{\pgfqpoint{4.527764in}{1.497615in}}%
\pgfpathlineto{\pgfqpoint{4.513550in}{1.506812in}}%
\pgfpathlineto{\pgfqpoint{4.521297in}{1.488005in}}%
\pgfpathlineto{\pgfqpoint{4.529036in}{1.470010in}}%
\pgfpathlineto{\pgfqpoint{4.536768in}{1.452813in}}%
\pgfpathclose%
\pgfusepath{fill}%
\end{pgfscope}%
\begin{pgfscope}%
\pgfpathrectangle{\pgfqpoint{1.254980in}{0.150000in}}{\pgfqpoint{5.490039in}{5.490039in}}%
\pgfusepath{clip}%
\pgfsetbuttcap%
\pgfsetroundjoin%
\definecolor{currentfill}{rgb}{0.208623,0.367752,0.552675}%
\pgfsetfillcolor{currentfill}%
\pgfsetfillopacity{0.700000}%
\pgfsetlinewidth{0.000000pt}%
\definecolor{currentstroke}{rgb}{0.000000,0.000000,0.000000}%
\pgfsetstrokecolor{currentstroke}%
\pgfsetdash{}{0pt}%
\pgfpathmoveto{\pgfqpoint{4.593517in}{1.417654in}}%
\pgfpathlineto{\pgfqpoint{4.607716in}{1.408920in}}%
\pgfpathlineto{\pgfqpoint{4.621919in}{1.400209in}}%
\pgfpathlineto{\pgfqpoint{4.636127in}{1.391520in}}%
\pgfpathlineto{\pgfqpoint{4.650340in}{1.382854in}}%
\pgfpathlineto{\pgfqpoint{4.642691in}{1.399066in}}%
\pgfpathlineto{\pgfqpoint{4.635038in}{1.416066in}}%
\pgfpathlineto{\pgfqpoint{4.627379in}{1.433869in}}%
\pgfpathlineto{\pgfqpoint{4.613135in}{1.442908in}}%
\pgfpathlineto{\pgfqpoint{4.598895in}{1.451970in}}%
\pgfpathlineto{\pgfqpoint{4.584660in}{1.461054in}}%
\pgfpathlineto{\pgfqpoint{4.570429in}{1.470160in}}%
\pgfpathlineto{\pgfqpoint{4.578132in}{1.451856in}}%
\pgfpathlineto{\pgfqpoint{4.585828in}{1.434359in}}%
\pgfpathlineto{\pgfqpoint{4.593517in}{1.417654in}}%
\pgfpathclose%
\pgfusepath{fill}%
\end{pgfscope}%
\begin{pgfscope}%
\pgfpathrectangle{\pgfqpoint{1.254980in}{0.150000in}}{\pgfqpoint{5.490039in}{5.490039in}}%
\pgfusepath{clip}%
\pgfsetbuttcap%
\pgfsetroundjoin%
\definecolor{currentfill}{rgb}{0.212395,0.359683,0.551710}%
\pgfsetfillcolor{currentfill}%
\pgfsetfillopacity{0.700000}%
\pgfsetlinewidth{0.000000pt}%
\definecolor{currentstroke}{rgb}{0.000000,0.000000,0.000000}%
\pgfsetstrokecolor{currentstroke}%
\pgfsetdash{}{0pt}%
\pgfpathmoveto{\pgfqpoint{4.650340in}{1.382854in}}%
\pgfpathlineto{\pgfqpoint{4.664557in}{1.374209in}}%
\pgfpathlineto{\pgfqpoint{4.678779in}{1.365587in}}%
\pgfpathlineto{\pgfqpoint{4.693005in}{1.356988in}}%
\pgfpathlineto{\pgfqpoint{4.685387in}{1.372831in}}%
\pgfpathlineto{\pgfqpoint{4.677764in}{1.389459in}}%
\pgfpathlineto{\pgfqpoint{4.670137in}{1.406886in}}%
\pgfpathlineto{\pgfqpoint{4.655880in}{1.415858in}}%
\pgfpathlineto{\pgfqpoint{4.641627in}{1.424852in}}%
\pgfpathlineto{\pgfqpoint{4.627379in}{1.433869in}}%
\pgfpathlineto{\pgfqpoint{4.635038in}{1.416066in}}%
\pgfpathlineto{\pgfqpoint{4.642691in}{1.399066in}}%
\pgfpathlineto{\pgfqpoint{4.650340in}{1.382854in}}%
\pgfpathclose%
\pgfusepath{fill}%
\end{pgfscope}%
\begin{pgfscope}%
\pgfsetbuttcap%
\pgfsetmiterjoin%
\definecolor{currentfill}{rgb}{1.000000,1.000000,1.000000}%
\pgfsetfillcolor{currentfill}%
\pgfsetfillopacity{0.800000}%
\pgfsetlinewidth{1.003750pt}%
\definecolor{currentstroke}{rgb}{0.800000,0.800000,0.800000}%
\pgfsetstrokecolor{currentstroke}%
\pgfsetstrokeopacity{0.800000}%
\pgfsetdash{}{0pt}%
\pgfpathmoveto{\pgfqpoint{5.541867in}{5.121213in}}%
\pgfpathlineto{\pgfqpoint{6.647797in}{5.121213in}}%
\pgfpathquadraticcurveto{\pgfqpoint{6.675575in}{5.121213in}}{\pgfqpoint{6.675575in}{5.148991in}}%
\pgfpathlineto{\pgfqpoint{6.675575in}{5.542817in}}%
\pgfpathquadraticcurveto{\pgfqpoint{6.675575in}{5.570595in}}{\pgfqpoint{6.647797in}{5.570595in}}%
\pgfpathlineto{\pgfqpoint{5.541867in}{5.570595in}}%
\pgfpathquadraticcurveto{\pgfqpoint{5.514090in}{5.570595in}}{\pgfqpoint{5.514090in}{5.542817in}}%
\pgfpathlineto{\pgfqpoint{5.514090in}{5.148991in}}%
\pgfpathquadraticcurveto{\pgfqpoint{5.514090in}{5.121213in}}{\pgfqpoint{5.541867in}{5.121213in}}%
\pgfpathlineto{\pgfqpoint{5.541867in}{5.121213in}}%
\pgfpathclose%
\pgfusepath{stroke,fill}%
\end{pgfscope}%
\begin{pgfscope}%
\pgfsetrectcap%
\pgfsetroundjoin%
\pgfsetlinewidth{1.505625pt}%
\definecolor{currentstroke}{rgb}{1.000000,0.000000,0.000000}%
\pgfsetstrokecolor{currentstroke}%
\pgfsetdash{}{0pt}%
\pgfpathmoveto{\pgfqpoint{5.569645in}{5.458127in}}%
\pgfpathlineto{\pgfqpoint{5.708534in}{5.458127in}}%
\pgfpathlineto{\pgfqpoint{5.847423in}{5.458127in}}%
\pgfusepath{stroke}%
\end{pgfscope}%
\begin{pgfscope}%
\pgfsetbuttcap%
\pgfsetroundjoin%
\definecolor{currentfill}{rgb}{1.000000,0.000000,0.000000}%
\pgfsetfillcolor{currentfill}%
\pgfsetlinewidth{1.003750pt}%
\definecolor{currentstroke}{rgb}{1.000000,0.000000,0.000000}%
\pgfsetstrokecolor{currentstroke}%
\pgfsetdash{}{0pt}%
\pgfsys@defobject{currentmarker}{\pgfqpoint{-0.041667in}{-0.041667in}}{\pgfqpoint{0.041667in}{0.041667in}}{%
\pgfpathmoveto{\pgfqpoint{0.000000in}{-0.041667in}}%
\pgfpathcurveto{\pgfqpoint{0.011050in}{-0.041667in}}{\pgfqpoint{0.021649in}{-0.037276in}}{\pgfqpoint{0.029463in}{-0.029463in}}%
\pgfpathcurveto{\pgfqpoint{0.037276in}{-0.021649in}}{\pgfqpoint{0.041667in}{-0.011050in}}{\pgfqpoint{0.041667in}{0.000000in}}%
\pgfpathcurveto{\pgfqpoint{0.041667in}{0.011050in}}{\pgfqpoint{0.037276in}{0.021649in}}{\pgfqpoint{0.029463in}{0.029463in}}%
\pgfpathcurveto{\pgfqpoint{0.021649in}{0.037276in}}{\pgfqpoint{0.011050in}{0.041667in}}{\pgfqpoint{0.000000in}{0.041667in}}%
\pgfpathcurveto{\pgfqpoint{-0.011050in}{0.041667in}}{\pgfqpoint{-0.021649in}{0.037276in}}{\pgfqpoint{-0.029463in}{0.029463in}}%
\pgfpathcurveto{\pgfqpoint{-0.037276in}{0.021649in}}{\pgfqpoint{-0.041667in}{0.011050in}}{\pgfqpoint{-0.041667in}{0.000000in}}%
\pgfpathcurveto{\pgfqpoint{-0.041667in}{-0.011050in}}{\pgfqpoint{-0.037276in}{-0.021649in}}{\pgfqpoint{-0.029463in}{-0.029463in}}%
\pgfpathcurveto{\pgfqpoint{-0.021649in}{-0.037276in}}{\pgfqpoint{-0.011050in}{-0.041667in}}{\pgfqpoint{0.000000in}{-0.041667in}}%
\pgfpathlineto{\pgfqpoint{0.000000in}{-0.041667in}}%
\pgfpathclose%
\pgfusepath{stroke,fill}%
}%
\begin{pgfscope}%
\pgfsys@transformshift{5.708534in}{5.458127in}%
\pgfsys@useobject{currentmarker}{}%
\end{pgfscope}%
\end{pgfscope}%
\begin{pgfscope}%
\definecolor{textcolor}{rgb}{0.000000,0.000000,0.000000}%
\pgfsetstrokecolor{textcolor}%
\pgfsetfillcolor{textcolor}%
\pgftext[x=5.958534in,y=5.409516in,left,base]{\color{textcolor}\sffamily\fontsize{10.000000}{12.000000}\selectfont Iterations}%
\end{pgfscope}%
\begin{pgfscope}%
\pgfsetbuttcap%
\pgfsetroundjoin%
\definecolor{currentfill}{rgb}{0.000000,0.000000,1.000000}%
\pgfsetfillcolor{currentfill}%
\pgfsetlinewidth{1.003750pt}%
\definecolor{currentstroke}{rgb}{0.000000,0.000000,1.000000}%
\pgfsetstrokecolor{currentstroke}%
\pgfsetdash{}{0pt}%
\pgfsys@defobject{currentmarker}{\pgfqpoint{-0.069444in}{-0.069444in}}{\pgfqpoint{0.069444in}{0.069444in}}{%
\pgfpathmoveto{\pgfqpoint{0.000000in}{-0.069444in}}%
\pgfpathcurveto{\pgfqpoint{0.018417in}{-0.069444in}}{\pgfqpoint{0.036082in}{-0.062127in}}{\pgfqpoint{0.049105in}{-0.049105in}}%
\pgfpathcurveto{\pgfqpoint{0.062127in}{-0.036082in}}{\pgfqpoint{0.069444in}{-0.018417in}}{\pgfqpoint{0.069444in}{0.000000in}}%
\pgfpathcurveto{\pgfqpoint{0.069444in}{0.018417in}}{\pgfqpoint{0.062127in}{0.036082in}}{\pgfqpoint{0.049105in}{0.049105in}}%
\pgfpathcurveto{\pgfqpoint{0.036082in}{0.062127in}}{\pgfqpoint{0.018417in}{0.069444in}}{\pgfqpoint{0.000000in}{0.069444in}}%
\pgfpathcurveto{\pgfqpoint{-0.018417in}{0.069444in}}{\pgfqpoint{-0.036082in}{0.062127in}}{\pgfqpoint{-0.049105in}{0.049105in}}%
\pgfpathcurveto{\pgfqpoint{-0.062127in}{0.036082in}}{\pgfqpoint{-0.069444in}{0.018417in}}{\pgfqpoint{-0.069444in}{0.000000in}}%
\pgfpathcurveto{\pgfqpoint{-0.069444in}{-0.018417in}}{\pgfqpoint{-0.062127in}{-0.036082in}}{\pgfqpoint{-0.049105in}{-0.049105in}}%
\pgfpathcurveto{\pgfqpoint{-0.036082in}{-0.062127in}}{\pgfqpoint{-0.018417in}{-0.069444in}}{\pgfqpoint{0.000000in}{-0.069444in}}%
\pgfpathlineto{\pgfqpoint{0.000000in}{-0.069444in}}%
\pgfpathclose%
\pgfusepath{stroke,fill}%
}%
\begin{pgfscope}%
\pgfsys@transformshift{5.708534in}{5.242117in}%
\pgfsys@useobject{currentmarker}{}%
\end{pgfscope}%
\end{pgfscope}%
\begin{pgfscope}%
\definecolor{textcolor}{rgb}{0.000000,0.000000,0.000000}%
\pgfsetstrokecolor{textcolor}%
\pgfsetfillcolor{textcolor}%
\pgftext[x=5.958534in,y=5.205659in,left,base]{\color{textcolor}\sffamily\fontsize{10.000000}{12.000000}\selectfont Minimum}%
\end{pgfscope}%
\end{pgfpicture}%
\makeatother%
\endgroup%
}
        \caption{3D graf funkcie}
        \label{fig:newton_vpravo}
    \end{subfigure}

    \label{fig:newton_komplet}
\end{figure}


\newpage
\subsubsection{MSG bez nulování $\beta$}


\noindent \textbf{Počiatočný bod} $x^{[0]} = [0; 0]$ \\

\begin{table}[H]
    \centering
    \begin{tabular}{cccc}
        \toprule
        \textbf{Iterácia} & \textbf{Bod } $x^{[k]} = [x;y]$ & \textbf{Hodnota } $f(x^{[k]})$ & \textbf{Norma } $\|\nabla f\|$ \\
        \midrule
        0  & $[0.000000;\; 0.000000]$   & 2.000000 & -- \\
        1  & $[0.000000;\; -0.500000]$  & 1.669031 & 1.000000 \\
        2  & $[0.250000;\; -0.553265]$  & 1.607017 & 0.511223 \\
        \dots & \dots & \dots & \dots \\
        18 & $[0.390938;\; -0.743091]$ & 1.569002 & 0.011786 \\
        19 & $[0.391029;\; -0.742336]$ & 1.569001 & 0.003799 \\
        20 & $[0.387794;\; -0.740333]$ & 1.568997 & 0.003664 \\
        \bottomrule
    \end{tabular}
    \caption{Priebeh MSG pre $x^{[0]} = [0;0]$.}
\end{table}

Tu sa ukázalo, že vynechanie pravidelného nulovania parametra $\beta$ má na konvergenciu negatívny vplyv. Počet iterácií narástol na 20, čo je o tretinu viac ako v prípade s resetom (kde to bolo 15). Je vidieť, že bez reštartu sa v smere hľadania postupne kumulujú chyby a metóda stráca svoju efektivitu, kým sa konečne trafí do presného minima.


\begin{figure}[H]
    \centering

    \begin{subfigure}{0.48\textwidth}
        \centering
        \resizebox{\linewidth}{!}{\input{grafy/cg_contour_10.pgf}}
        \caption{Pohľad zhora (Vrstevnice)}
        \label{fig:newton_vlavo}
    \end{subfigure}
    \hfill
    \begin{subfigure}{0.48\textwidth}
        \centering
        \resizebox{\linewidth}{!}{%% Creator: Matplotlib, PGF backend
%%
%% To include the figure in your LaTeX document, write
%%   \input{<filename>.pgf}
%%
%% Make sure the required packages are loaded in your preamble
%%   \usepackage{pgf}
%%
%% Also ensure that all the required font packages are loaded; for instance,
%% the lmodern package is sometimes necessary when using math font.
%%   \usepackage{lmodern}
%%
%% Figures using additional raster images can only be included by \input if
%% they are in the same directory as the main LaTeX file. For loading figures
%% from other directories you can use the `import` package
%%   \usepackage{import}
%%
%% and then include the figures with
%%   \import{<path to file>}{<filename>.pgf}
%%
%% Matplotlib used the following preamble
%%   
%%   \usepackage{fontspec}
%%   \setmainfont{DejaVuSerif.ttf}[Path=\detokenize{/home/radimek/Documents/projekt_mat_prog/mat_prog_kernel/lib/python3.12/site-packages/matplotlib/mpl-data/fonts/ttf/}]
%%   \setsansfont{DejaVuSans.ttf}[Path=\detokenize{/home/radimek/Documents/projekt_mat_prog/mat_prog_kernel/lib/python3.12/site-packages/matplotlib/mpl-data/fonts/ttf/}]
%%   \setmonofont{DejaVuSansMono.ttf}[Path=\detokenize{/home/radimek/Documents/projekt_mat_prog/mat_prog_kernel/lib/python3.12/site-packages/matplotlib/mpl-data/fonts/ttf/}]
%%   \makeatletter\@ifpackageloaded{underscore}{}{\usepackage[strings]{underscore}}\makeatother
%%
\begingroup%
\makeatletter%
\begin{pgfpicture}%
\pgfpathrectangle{\pgfpointorigin}{\pgfqpoint{8.000000in}{6.000000in}}%
\pgfusepath{use as bounding box, clip}%
\begin{pgfscope}%
\pgfsetbuttcap%
\pgfsetmiterjoin%
\definecolor{currentfill}{rgb}{1.000000,1.000000,1.000000}%
\pgfsetfillcolor{currentfill}%
\pgfsetlinewidth{0.000000pt}%
\definecolor{currentstroke}{rgb}{1.000000,1.000000,1.000000}%
\pgfsetstrokecolor{currentstroke}%
\pgfsetdash{}{0pt}%
\pgfpathmoveto{\pgfqpoint{0.000000in}{0.000000in}}%
\pgfpathlineto{\pgfqpoint{8.000000in}{0.000000in}}%
\pgfpathlineto{\pgfqpoint{8.000000in}{6.000000in}}%
\pgfpathlineto{\pgfqpoint{0.000000in}{6.000000in}}%
\pgfpathlineto{\pgfqpoint{0.000000in}{0.000000in}}%
\pgfpathclose%
\pgfusepath{fill}%
\end{pgfscope}%
\begin{pgfscope}%
\pgfsetbuttcap%
\pgfsetmiterjoin%
\definecolor{currentfill}{rgb}{1.000000,1.000000,1.000000}%
\pgfsetfillcolor{currentfill}%
\pgfsetlinewidth{0.000000pt}%
\definecolor{currentstroke}{rgb}{0.000000,0.000000,0.000000}%
\pgfsetstrokecolor{currentstroke}%
\pgfsetstrokeopacity{0.000000}%
\pgfsetdash{}{0pt}%
\pgfpathmoveto{\pgfqpoint{1.254980in}{0.150000in}}%
\pgfpathlineto{\pgfqpoint{6.745020in}{0.150000in}}%
\pgfpathlineto{\pgfqpoint{6.745020in}{5.640039in}}%
\pgfpathlineto{\pgfqpoint{1.254980in}{5.640039in}}%
\pgfpathlineto{\pgfqpoint{1.254980in}{0.150000in}}%
\pgfpathclose%
\pgfusepath{fill}%
\end{pgfscope}%
\begin{pgfscope}%
\pgfsetbuttcap%
\pgfsetmiterjoin%
\definecolor{currentfill}{rgb}{0.950000,0.950000,0.950000}%
\pgfsetfillcolor{currentfill}%
\pgfsetfillopacity{0.500000}%
\pgfsetlinewidth{1.003750pt}%
\definecolor{currentstroke}{rgb}{0.950000,0.950000,0.950000}%
\pgfsetstrokecolor{currentstroke}%
\pgfsetstrokeopacity{0.500000}%
\pgfsetdash{}{0pt}%
\pgfpathmoveto{\pgfqpoint{1.669516in}{1.503668in}}%
\pgfpathlineto{\pgfqpoint{3.482506in}{3.023352in}}%
\pgfpathlineto{\pgfqpoint{3.457304in}{5.215008in}}%
\pgfpathlineto{\pgfqpoint{1.557553in}{3.828657in}}%
\pgfusepath{stroke,fill}%
\end{pgfscope}%
\begin{pgfscope}%
\pgfsetbuttcap%
\pgfsetmiterjoin%
\definecolor{currentfill}{rgb}{0.900000,0.900000,0.900000}%
\pgfsetfillcolor{currentfill}%
\pgfsetfillopacity{0.500000}%
\pgfsetlinewidth{1.003750pt}%
\definecolor{currentstroke}{rgb}{0.900000,0.900000,0.900000}%
\pgfsetstrokecolor{currentstroke}%
\pgfsetstrokeopacity{0.500000}%
\pgfsetdash{}{0pt}%
\pgfpathmoveto{\pgfqpoint{3.482506in}{3.023352in}}%
\pgfpathlineto{\pgfqpoint{6.391709in}{2.177762in}}%
\pgfpathlineto{\pgfqpoint{6.495528in}{4.444907in}}%
\pgfpathlineto{\pgfqpoint{3.457304in}{5.215008in}}%
\pgfusepath{stroke,fill}%
\end{pgfscope}%
\begin{pgfscope}%
\pgfsetbuttcap%
\pgfsetmiterjoin%
\definecolor{currentfill}{rgb}{0.925000,0.925000,0.925000}%
\pgfsetfillcolor{currentfill}%
\pgfsetfillopacity{0.500000}%
\pgfsetlinewidth{1.003750pt}%
\definecolor{currentstroke}{rgb}{0.925000,0.925000,0.925000}%
\pgfsetstrokecolor{currentstroke}%
\pgfsetstrokeopacity{0.500000}%
\pgfsetdash{}{0pt}%
\pgfpathmoveto{\pgfqpoint{1.669516in}{1.503668in}}%
\pgfpathlineto{\pgfqpoint{4.753413in}{0.496467in}}%
\pgfpathlineto{\pgfqpoint{6.391709in}{2.177762in}}%
\pgfpathlineto{\pgfqpoint{3.482506in}{3.023352in}}%
\pgfusepath{stroke,fill}%
\end{pgfscope}%
\begin{pgfscope}%
\pgfsetrectcap%
\pgfsetroundjoin%
\pgfsetlinewidth{0.803000pt}%
\definecolor{currentstroke}{rgb}{0.000000,0.000000,0.000000}%
\pgfsetstrokecolor{currentstroke}%
\pgfsetdash{}{0pt}%
\pgfpathmoveto{\pgfqpoint{1.669516in}{1.503668in}}%
\pgfpathlineto{\pgfqpoint{4.753413in}{0.496467in}}%
\pgfusepath{stroke}%
\end{pgfscope}%
\begin{pgfscope}%
\definecolor{textcolor}{rgb}{0.000000,0.000000,0.000000}%
\pgfsetstrokecolor{textcolor}%
\pgfsetfillcolor{textcolor}%
\pgftext[x=2.945156in,y=0.524780in,,]{\color{textcolor}\sffamily\fontsize{10.000000}{12.000000}\selectfont x}%
\end{pgfscope}%
\begin{pgfscope}%
\pgfsetbuttcap%
\pgfsetroundjoin%
\pgfsetlinewidth{0.803000pt}%
\definecolor{currentstroke}{rgb}{0.690196,0.690196,0.690196}%
\pgfsetstrokecolor{currentstroke}%
\pgfsetdash{}{0pt}%
\pgfpathmoveto{\pgfqpoint{1.856293in}{1.442666in}}%
\pgfpathlineto{\pgfqpoint{3.659435in}{2.971926in}}%
\pgfpathlineto{\pgfqpoint{3.641714in}{5.168266in}}%
\pgfusepath{stroke}%
\end{pgfscope}%
\begin{pgfscope}%
\pgfsetbuttcap%
\pgfsetroundjoin%
\pgfsetlinewidth{0.803000pt}%
\definecolor{currentstroke}{rgb}{0.690196,0.690196,0.690196}%
\pgfsetstrokecolor{currentstroke}%
\pgfsetdash{}{0pt}%
\pgfpathmoveto{\pgfqpoint{2.179349in}{1.337156in}}%
\pgfpathlineto{\pgfqpoint{3.965233in}{2.883042in}}%
\pgfpathlineto{\pgfqpoint{3.960553in}{5.087450in}}%
\pgfusepath{stroke}%
\end{pgfscope}%
\begin{pgfscope}%
\pgfsetbuttcap%
\pgfsetroundjoin%
\pgfsetlinewidth{0.803000pt}%
\definecolor{currentstroke}{rgb}{0.690196,0.690196,0.690196}%
\pgfsetstrokecolor{currentstroke}%
\pgfsetdash{}{0pt}%
\pgfpathmoveto{\pgfqpoint{2.506079in}{1.230446in}}%
\pgfpathlineto{\pgfqpoint{4.274222in}{2.793232in}}%
\pgfpathlineto{\pgfqpoint{4.282861in}{5.005754in}}%
\pgfusepath{stroke}%
\end{pgfscope}%
\begin{pgfscope}%
\pgfsetbuttcap%
\pgfsetroundjoin%
\pgfsetlinewidth{0.803000pt}%
\definecolor{currentstroke}{rgb}{0.690196,0.690196,0.690196}%
\pgfsetstrokecolor{currentstroke}%
\pgfsetdash{}{0pt}%
\pgfpathmoveto{\pgfqpoint{2.836546in}{1.122516in}}%
\pgfpathlineto{\pgfqpoint{4.586450in}{2.702479in}}%
\pgfpathlineto{\pgfqpoint{4.608697in}{4.923164in}}%
\pgfusepath{stroke}%
\end{pgfscope}%
\begin{pgfscope}%
\pgfsetbuttcap%
\pgfsetroundjoin%
\pgfsetlinewidth{0.803000pt}%
\definecolor{currentstroke}{rgb}{0.690196,0.690196,0.690196}%
\pgfsetstrokecolor{currentstroke}%
\pgfsetdash{}{0pt}%
\pgfpathmoveto{\pgfqpoint{3.170814in}{1.013344in}}%
\pgfpathlineto{\pgfqpoint{4.901969in}{2.610770in}}%
\pgfpathlineto{\pgfqpoint{4.938117in}{4.839666in}}%
\pgfusepath{stroke}%
\end{pgfscope}%
\begin{pgfscope}%
\pgfsetbuttcap%
\pgfsetroundjoin%
\pgfsetlinewidth{0.803000pt}%
\definecolor{currentstroke}{rgb}{0.690196,0.690196,0.690196}%
\pgfsetstrokecolor{currentstroke}%
\pgfsetdash{}{0pt}%
\pgfpathmoveto{\pgfqpoint{3.508950in}{0.902909in}}%
\pgfpathlineto{\pgfqpoint{5.220832in}{2.518089in}}%
\pgfpathlineto{\pgfqpoint{5.271181in}{4.755244in}}%
\pgfusepath{stroke}%
\end{pgfscope}%
\begin{pgfscope}%
\pgfsetbuttcap%
\pgfsetroundjoin%
\pgfsetlinewidth{0.803000pt}%
\definecolor{currentstroke}{rgb}{0.690196,0.690196,0.690196}%
\pgfsetstrokecolor{currentstroke}%
\pgfsetdash{}{0pt}%
\pgfpathmoveto{\pgfqpoint{3.851022in}{0.791188in}}%
\pgfpathlineto{\pgfqpoint{5.543092in}{2.424421in}}%
\pgfpathlineto{\pgfqpoint{5.607950in}{4.669883in}}%
\pgfusepath{stroke}%
\end{pgfscope}%
\begin{pgfscope}%
\pgfsetbuttcap%
\pgfsetroundjoin%
\pgfsetlinewidth{0.803000pt}%
\definecolor{currentstroke}{rgb}{0.690196,0.690196,0.690196}%
\pgfsetstrokecolor{currentstroke}%
\pgfsetdash{}{0pt}%
\pgfpathmoveto{\pgfqpoint{4.197097in}{0.678160in}}%
\pgfpathlineto{\pgfqpoint{5.868803in}{2.329750in}}%
\pgfpathlineto{\pgfqpoint{5.948486in}{4.583567in}}%
\pgfusepath{stroke}%
\end{pgfscope}%
\begin{pgfscope}%
\pgfsetbuttcap%
\pgfsetroundjoin%
\pgfsetlinewidth{0.803000pt}%
\definecolor{currentstroke}{rgb}{0.690196,0.690196,0.690196}%
\pgfsetstrokecolor{currentstroke}%
\pgfsetdash{}{0pt}%
\pgfpathmoveto{\pgfqpoint{4.547248in}{0.563801in}}%
\pgfpathlineto{\pgfqpoint{6.198022in}{2.234059in}}%
\pgfpathlineto{\pgfqpoint{6.292853in}{4.496280in}}%
\pgfusepath{stroke}%
\end{pgfscope}%
\begin{pgfscope}%
\pgfsetrectcap%
\pgfsetroundjoin%
\pgfsetlinewidth{0.803000pt}%
\definecolor{currentstroke}{rgb}{0.000000,0.000000,0.000000}%
\pgfsetstrokecolor{currentstroke}%
\pgfsetdash{}{0pt}%
\pgfpathmoveto{\pgfqpoint{1.871995in}{1.455983in}}%
\pgfpathlineto{\pgfqpoint{1.824823in}{1.415976in}}%
\pgfusepath{stroke}%
\end{pgfscope}%
\begin{pgfscope}%
\definecolor{textcolor}{rgb}{0.000000,0.000000,0.000000}%
\pgfsetstrokecolor{textcolor}%
\pgfsetfillcolor{textcolor}%
\pgftext[x=1.751850in,y=1.224727in,,top]{\color{textcolor}\sffamily\fontsize{10.000000}{12.000000}\selectfont \ensuremath{-}1.00}%
\end{pgfscope}%
\begin{pgfscope}%
\pgfsetrectcap%
\pgfsetroundjoin%
\pgfsetlinewidth{0.803000pt}%
\definecolor{currentstroke}{rgb}{0.000000,0.000000,0.000000}%
\pgfsetstrokecolor{currentstroke}%
\pgfsetdash{}{0pt}%
\pgfpathmoveto{\pgfqpoint{2.194907in}{1.350624in}}%
\pgfpathlineto{\pgfqpoint{2.148165in}{1.310163in}}%
\pgfusepath{stroke}%
\end{pgfscope}%
\begin{pgfscope}%
\definecolor{textcolor}{rgb}{0.000000,0.000000,0.000000}%
\pgfsetstrokecolor{textcolor}%
\pgfsetfillcolor{textcolor}%
\pgftext[x=2.075126in,y=1.117716in,,top]{\color{textcolor}\sffamily\fontsize{10.000000}{12.000000}\selectfont \ensuremath{-}0.75}%
\end{pgfscope}%
\begin{pgfscope}%
\pgfsetrectcap%
\pgfsetroundjoin%
\pgfsetlinewidth{0.803000pt}%
\definecolor{currentstroke}{rgb}{0.000000,0.000000,0.000000}%
\pgfsetstrokecolor{currentstroke}%
\pgfsetdash{}{0pt}%
\pgfpathmoveto{\pgfqpoint{2.521490in}{1.244068in}}%
\pgfpathlineto{\pgfqpoint{2.475190in}{1.203145in}}%
\pgfusepath{stroke}%
\end{pgfscope}%
\begin{pgfscope}%
\definecolor{textcolor}{rgb}{0.000000,0.000000,0.000000}%
\pgfsetstrokecolor{textcolor}%
\pgfsetfillcolor{textcolor}%
\pgftext[x=2.402087in,y=1.009485in,,top]{\color{textcolor}\sffamily\fontsize{10.000000}{12.000000}\selectfont \ensuremath{-}0.50}%
\end{pgfscope}%
\begin{pgfscope}%
\pgfsetrectcap%
\pgfsetroundjoin%
\pgfsetlinewidth{0.803000pt}%
\definecolor{currentstroke}{rgb}{0.000000,0.000000,0.000000}%
\pgfsetstrokecolor{currentstroke}%
\pgfsetdash{}{0pt}%
\pgfpathmoveto{\pgfqpoint{2.851805in}{1.136293in}}%
\pgfpathlineto{\pgfqpoint{2.805960in}{1.094901in}}%
\pgfusepath{stroke}%
\end{pgfscope}%
\begin{pgfscope}%
\definecolor{textcolor}{rgb}{0.000000,0.000000,0.000000}%
\pgfsetstrokecolor{textcolor}%
\pgfsetfillcolor{textcolor}%
\pgftext[x=2.732798in,y=0.900014in,,top]{\color{textcolor}\sffamily\fontsize{10.000000}{12.000000}\selectfont \ensuremath{-}0.25}%
\end{pgfscope}%
\begin{pgfscope}%
\pgfsetrectcap%
\pgfsetroundjoin%
\pgfsetlinewidth{0.803000pt}%
\definecolor{currentstroke}{rgb}{0.000000,0.000000,0.000000}%
\pgfsetstrokecolor{currentstroke}%
\pgfsetdash{}{0pt}%
\pgfpathmoveto{\pgfqpoint{3.185917in}{1.027280in}}%
\pgfpathlineto{\pgfqpoint{3.140542in}{0.985410in}}%
\pgfusepath{stroke}%
\end{pgfscope}%
\begin{pgfscope}%
\definecolor{textcolor}{rgb}{0.000000,0.000000,0.000000}%
\pgfsetstrokecolor{textcolor}%
\pgfsetfillcolor{textcolor}%
\pgftext[x=3.067323in,y=0.789279in,,top]{\color{textcolor}\sffamily\fontsize{10.000000}{12.000000}\selectfont 0.00}%
\end{pgfscope}%
\begin{pgfscope}%
\pgfsetrectcap%
\pgfsetroundjoin%
\pgfsetlinewidth{0.803000pt}%
\definecolor{currentstroke}{rgb}{0.000000,0.000000,0.000000}%
\pgfsetstrokecolor{currentstroke}%
\pgfsetdash{}{0pt}%
\pgfpathmoveto{\pgfqpoint{3.523892in}{0.917006in}}%
\pgfpathlineto{\pgfqpoint{3.479000in}{0.874650in}}%
\pgfusepath{stroke}%
\end{pgfscope}%
\begin{pgfscope}%
\definecolor{textcolor}{rgb}{0.000000,0.000000,0.000000}%
\pgfsetstrokecolor{textcolor}%
\pgfsetfillcolor{textcolor}%
\pgftext[x=3.405729in,y=0.677260in,,top]{\color{textcolor}\sffamily\fontsize{10.000000}{12.000000}\selectfont 0.25}%
\end{pgfscope}%
\begin{pgfscope}%
\pgfsetrectcap%
\pgfsetroundjoin%
\pgfsetlinewidth{0.803000pt}%
\definecolor{currentstroke}{rgb}{0.000000,0.000000,0.000000}%
\pgfsetstrokecolor{currentstroke}%
\pgfsetdash{}{0pt}%
\pgfpathmoveto{\pgfqpoint{3.865798in}{0.805450in}}%
\pgfpathlineto{\pgfqpoint{3.821404in}{0.762600in}}%
\pgfusepath{stroke}%
\end{pgfscope}%
\begin{pgfscope}%
\definecolor{textcolor}{rgb}{0.000000,0.000000,0.000000}%
\pgfsetstrokecolor{textcolor}%
\pgfsetfillcolor{textcolor}%
\pgftext[x=3.748083in,y=0.563934in,,top]{\color{textcolor}\sffamily\fontsize{10.000000}{12.000000}\selectfont 0.50}%
\end{pgfscope}%
\begin{pgfscope}%
\pgfsetrectcap%
\pgfsetroundjoin%
\pgfsetlinewidth{0.803000pt}%
\definecolor{currentstroke}{rgb}{0.000000,0.000000,0.000000}%
\pgfsetstrokecolor{currentstroke}%
\pgfsetdash{}{0pt}%
\pgfpathmoveto{\pgfqpoint{4.211703in}{0.692590in}}%
\pgfpathlineto{\pgfqpoint{4.167821in}{0.649236in}}%
\pgfusepath{stroke}%
\end{pgfscope}%
\begin{pgfscope}%
\definecolor{textcolor}{rgb}{0.000000,0.000000,0.000000}%
\pgfsetstrokecolor{textcolor}%
\pgfsetfillcolor{textcolor}%
\pgftext[x=4.094455in,y=0.449278in,,top]{\color{textcolor}\sffamily\fontsize{10.000000}{12.000000}\selectfont 0.75}%
\end{pgfscope}%
\begin{pgfscope}%
\pgfsetrectcap%
\pgfsetroundjoin%
\pgfsetlinewidth{0.803000pt}%
\definecolor{currentstroke}{rgb}{0.000000,0.000000,0.000000}%
\pgfsetstrokecolor{currentstroke}%
\pgfsetdash{}{0pt}%
\pgfpathmoveto{\pgfqpoint{4.561678in}{0.578401in}}%
\pgfpathlineto{\pgfqpoint{4.518323in}{0.534535in}}%
\pgfusepath{stroke}%
\end{pgfscope}%
\begin{pgfscope}%
\definecolor{textcolor}{rgb}{0.000000,0.000000,0.000000}%
\pgfsetstrokecolor{textcolor}%
\pgfsetfillcolor{textcolor}%
\pgftext[x=4.444916in,y=0.333269in,,top]{\color{textcolor}\sffamily\fontsize{10.000000}{12.000000}\selectfont 1.00}%
\end{pgfscope}%
\begin{pgfscope}%
\pgfsetrectcap%
\pgfsetroundjoin%
\pgfsetlinewidth{0.803000pt}%
\definecolor{currentstroke}{rgb}{0.000000,0.000000,0.000000}%
\pgfsetstrokecolor{currentstroke}%
\pgfsetdash{}{0pt}%
\pgfpathmoveto{\pgfqpoint{6.391709in}{2.177762in}}%
\pgfpathlineto{\pgfqpoint{4.753413in}{0.496467in}}%
\pgfusepath{stroke}%
\end{pgfscope}%
\begin{pgfscope}%
\definecolor{textcolor}{rgb}{0.000000,0.000000,0.000000}%
\pgfsetstrokecolor{textcolor}%
\pgfsetfillcolor{textcolor}%
\pgftext[x=5.983676in,y=0.985873in,,]{\color{textcolor}\sffamily\fontsize{10.000000}{12.000000}\selectfont y}%
\end{pgfscope}%
\begin{pgfscope}%
\pgfsetbuttcap%
\pgfsetroundjoin%
\pgfsetlinewidth{0.803000pt}%
\definecolor{currentstroke}{rgb}{0.690196,0.690196,0.690196}%
\pgfsetstrokecolor{currentstroke}%
\pgfsetdash{}{0pt}%
\pgfpathmoveto{\pgfqpoint{1.688926in}{3.924526in}}%
\pgfpathlineto{\pgfqpoint{1.794447in}{1.608387in}}%
\pgfpathlineto{\pgfqpoint{4.866770in}{0.612800in}}%
\pgfusepath{stroke}%
\end{pgfscope}%
\begin{pgfscope}%
\pgfsetbuttcap%
\pgfsetroundjoin%
\pgfsetlinewidth{0.803000pt}%
\definecolor{currentstroke}{rgb}{0.690196,0.690196,0.690196}%
\pgfsetstrokecolor{currentstroke}%
\pgfsetdash{}{0pt}%
\pgfpathmoveto{\pgfqpoint{1.925189in}{4.096941in}}%
\pgfpathlineto{\pgfqpoint{2.019290in}{1.796855in}}%
\pgfpathlineto{\pgfqpoint{5.070611in}{0.821990in}}%
\pgfusepath{stroke}%
\end{pgfscope}%
\begin{pgfscope}%
\pgfsetbuttcap%
\pgfsetroundjoin%
\pgfsetlinewidth{0.803000pt}%
\definecolor{currentstroke}{rgb}{0.690196,0.690196,0.690196}%
\pgfsetstrokecolor{currentstroke}%
\pgfsetdash{}{0pt}%
\pgfpathmoveto{\pgfqpoint{2.156470in}{4.265719in}}%
\pgfpathlineto{\pgfqpoint{2.239596in}{1.981520in}}%
\pgfpathlineto{\pgfqpoint{5.270122in}{1.026738in}}%
\pgfusepath{stroke}%
\end{pgfscope}%
\begin{pgfscope}%
\pgfsetbuttcap%
\pgfsetroundjoin%
\pgfsetlinewidth{0.803000pt}%
\definecolor{currentstroke}{rgb}{0.690196,0.690196,0.690196}%
\pgfsetstrokecolor{currentstroke}%
\pgfsetdash{}{0pt}%
\pgfpathmoveto{\pgfqpoint{2.382925in}{4.430975in}}%
\pgfpathlineto{\pgfqpoint{2.455502in}{2.162497in}}%
\pgfpathlineto{\pgfqpoint{5.465441in}{1.227183in}}%
\pgfusepath{stroke}%
\end{pgfscope}%
\begin{pgfscope}%
\pgfsetbuttcap%
\pgfsetroundjoin%
\pgfsetlinewidth{0.803000pt}%
\definecolor{currentstroke}{rgb}{0.690196,0.690196,0.690196}%
\pgfsetstrokecolor{currentstroke}%
\pgfsetdash{}{0pt}%
\pgfpathmoveto{\pgfqpoint{2.604702in}{4.592818in}}%
\pgfpathlineto{\pgfqpoint{2.667139in}{2.339894in}}%
\pgfpathlineto{\pgfqpoint{5.656697in}{1.423459in}}%
\pgfusepath{stroke}%
\end{pgfscope}%
\begin{pgfscope}%
\pgfsetbuttcap%
\pgfsetroundjoin%
\pgfsetlinewidth{0.803000pt}%
\definecolor{currentstroke}{rgb}{0.690196,0.690196,0.690196}%
\pgfsetstrokecolor{currentstroke}%
\pgfsetdash{}{0pt}%
\pgfpathmoveto{\pgfqpoint{2.821945in}{4.751352in}}%
\pgfpathlineto{\pgfqpoint{2.874630in}{2.513818in}}%
\pgfpathlineto{\pgfqpoint{5.844017in}{1.615695in}}%
\pgfusepath{stroke}%
\end{pgfscope}%
\begin{pgfscope}%
\pgfsetbuttcap%
\pgfsetroundjoin%
\pgfsetlinewidth{0.803000pt}%
\definecolor{currentstroke}{rgb}{0.690196,0.690196,0.690196}%
\pgfsetstrokecolor{currentstroke}%
\pgfsetdash{}{0pt}%
\pgfpathmoveto{\pgfqpoint{3.034792in}{4.906678in}}%
\pgfpathlineto{\pgfqpoint{3.078098in}{2.684369in}}%
\pgfpathlineto{\pgfqpoint{6.027520in}{1.804014in}}%
\pgfusepath{stroke}%
\end{pgfscope}%
\begin{pgfscope}%
\pgfsetbuttcap%
\pgfsetroundjoin%
\pgfsetlinewidth{0.803000pt}%
\definecolor{currentstroke}{rgb}{0.690196,0.690196,0.690196}%
\pgfsetstrokecolor{currentstroke}%
\pgfsetdash{}{0pt}%
\pgfpathmoveto{\pgfqpoint{3.243375in}{5.058892in}}%
\pgfpathlineto{\pgfqpoint{3.277658in}{2.851644in}}%
\pgfpathlineto{\pgfqpoint{6.207323in}{1.988536in}}%
\pgfusepath{stroke}%
\end{pgfscope}%
\begin{pgfscope}%
\pgfsetrectcap%
\pgfsetroundjoin%
\pgfsetlinewidth{0.803000pt}%
\definecolor{currentstroke}{rgb}{0.000000,0.000000,0.000000}%
\pgfsetstrokecolor{currentstroke}%
\pgfsetdash{}{0pt}%
\pgfpathmoveto{\pgfqpoint{4.840880in}{0.621189in}}%
\pgfpathlineto{\pgfqpoint{4.918618in}{0.595998in}}%
\pgfusepath{stroke}%
\end{pgfscope}%
\begin{pgfscope}%
\definecolor{textcolor}{rgb}{0.000000,0.000000,0.000000}%
\pgfsetstrokecolor{textcolor}%
\pgfsetfillcolor{textcolor}%
\pgftext[x=5.045633in,y=0.426401in,,top]{\color{textcolor}\sffamily\fontsize{10.000000}{12.000000}\selectfont \ensuremath{-}1.0}%
\end{pgfscope}%
\begin{pgfscope}%
\pgfsetrectcap%
\pgfsetroundjoin%
\pgfsetlinewidth{0.803000pt}%
\definecolor{currentstroke}{rgb}{0.000000,0.000000,0.000000}%
\pgfsetstrokecolor{currentstroke}%
\pgfsetdash{}{0pt}%
\pgfpathmoveto{\pgfqpoint{5.044911in}{0.830201in}}%
\pgfpathlineto{\pgfqpoint{5.122076in}{0.805548in}}%
\pgfusepath{stroke}%
\end{pgfscope}%
\begin{pgfscope}%
\definecolor{textcolor}{rgb}{0.000000,0.000000,0.000000}%
\pgfsetstrokecolor{textcolor}%
\pgfsetfillcolor{textcolor}%
\pgftext[x=5.247634in,y=0.637677in,,top]{\color{textcolor}\sffamily\fontsize{10.000000}{12.000000}\selectfont \ensuremath{-}0.8}%
\end{pgfscope}%
\begin{pgfscope}%
\pgfsetrectcap%
\pgfsetroundjoin%
\pgfsetlinewidth{0.803000pt}%
\definecolor{currentstroke}{rgb}{0.000000,0.000000,0.000000}%
\pgfsetstrokecolor{currentstroke}%
\pgfsetdash{}{0pt}%
\pgfpathmoveto{\pgfqpoint{5.244611in}{1.034775in}}%
\pgfpathlineto{\pgfqpoint{5.321208in}{1.010643in}}%
\pgfusepath{stroke}%
\end{pgfscope}%
\begin{pgfscope}%
\definecolor{textcolor}{rgb}{0.000000,0.000000,0.000000}%
\pgfsetstrokecolor{textcolor}%
\pgfsetfillcolor{textcolor}%
\pgftext[x=5.445344in,y=0.844463in,,top]{\color{textcolor}\sffamily\fontsize{10.000000}{12.000000}\selectfont \ensuremath{-}0.6}%
\end{pgfscope}%
\begin{pgfscope}%
\pgfsetrectcap%
\pgfsetroundjoin%
\pgfsetlinewidth{0.803000pt}%
\definecolor{currentstroke}{rgb}{0.000000,0.000000,0.000000}%
\pgfsetstrokecolor{currentstroke}%
\pgfsetdash{}{0pt}%
\pgfpathmoveto{\pgfqpoint{5.440116in}{1.235052in}}%
\pgfpathlineto{\pgfqpoint{5.516153in}{1.211424in}}%
\pgfusepath{stroke}%
\end{pgfscope}%
\begin{pgfscope}%
\definecolor{textcolor}{rgb}{0.000000,0.000000,0.000000}%
\pgfsetstrokecolor{textcolor}%
\pgfsetfillcolor{textcolor}%
\pgftext[x=5.638896in,y=1.046902in,,top]{\color{textcolor}\sffamily\fontsize{10.000000}{12.000000}\selectfont \ensuremath{-}0.4}%
\end{pgfscope}%
\begin{pgfscope}%
\pgfsetrectcap%
\pgfsetroundjoin%
\pgfsetlinewidth{0.803000pt}%
\definecolor{currentstroke}{rgb}{0.000000,0.000000,0.000000}%
\pgfsetstrokecolor{currentstroke}%
\pgfsetdash{}{0pt}%
\pgfpathmoveto{\pgfqpoint{5.631557in}{1.431165in}}%
\pgfpathlineto{\pgfqpoint{5.707039in}{1.408026in}}%
\pgfusepath{stroke}%
\end{pgfscope}%
\begin{pgfscope}%
\definecolor{textcolor}{rgb}{0.000000,0.000000,0.000000}%
\pgfsetstrokecolor{textcolor}%
\pgfsetfillcolor{textcolor}%
\pgftext[x=5.828422in,y=1.245130in,,top]{\color{textcolor}\sffamily\fontsize{10.000000}{12.000000}\selectfont \ensuremath{-}0.2}%
\end{pgfscope}%
\begin{pgfscope}%
\pgfsetrectcap%
\pgfsetroundjoin%
\pgfsetlinewidth{0.803000pt}%
\definecolor{currentstroke}{rgb}{0.000000,0.000000,0.000000}%
\pgfsetstrokecolor{currentstroke}%
\pgfsetdash{}{0pt}%
\pgfpathmoveto{\pgfqpoint{5.819059in}{1.623244in}}%
\pgfpathlineto{\pgfqpoint{5.893994in}{1.600579in}}%
\pgfusepath{stroke}%
\end{pgfscope}%
\begin{pgfscope}%
\definecolor{textcolor}{rgb}{0.000000,0.000000,0.000000}%
\pgfsetstrokecolor{textcolor}%
\pgfsetfillcolor{textcolor}%
\pgftext[x=6.014046in,y=1.439275in,,top]{\color{textcolor}\sffamily\fontsize{10.000000}{12.000000}\selectfont 0.0}%
\end{pgfscope}%
\begin{pgfscope}%
\pgfsetrectcap%
\pgfsetroundjoin%
\pgfsetlinewidth{0.803000pt}%
\definecolor{currentstroke}{rgb}{0.000000,0.000000,0.000000}%
\pgfsetstrokecolor{currentstroke}%
\pgfsetdash{}{0pt}%
\pgfpathmoveto{\pgfqpoint{6.002742in}{1.811410in}}%
\pgfpathlineto{\pgfqpoint{6.077136in}{1.789205in}}%
\pgfusepath{stroke}%
\end{pgfscope}%
\begin{pgfscope}%
\definecolor{textcolor}{rgb}{0.000000,0.000000,0.000000}%
\pgfsetstrokecolor{textcolor}%
\pgfsetfillcolor{textcolor}%
\pgftext[x=6.195886in,y=1.629464in,,top]{\color{textcolor}\sffamily\fontsize{10.000000}{12.000000}\selectfont 0.2}%
\end{pgfscope}%
\begin{pgfscope}%
\pgfsetrectcap%
\pgfsetroundjoin%
\pgfsetlinewidth{0.803000pt}%
\definecolor{currentstroke}{rgb}{0.000000,0.000000,0.000000}%
\pgfsetstrokecolor{currentstroke}%
\pgfsetdash{}{0pt}%
\pgfpathmoveto{\pgfqpoint{6.182723in}{1.995784in}}%
\pgfpathlineto{\pgfqpoint{6.256582in}{1.974024in}}%
\pgfusepath{stroke}%
\end{pgfscope}%
\begin{pgfscope}%
\definecolor{textcolor}{rgb}{0.000000,0.000000,0.000000}%
\pgfsetstrokecolor{textcolor}%
\pgfsetfillcolor{textcolor}%
\pgftext[x=6.374057in,y=1.815816in,,top]{\color{textcolor}\sffamily\fontsize{10.000000}{12.000000}\selectfont 0.4}%
\end{pgfscope}%
\begin{pgfscope}%
\pgfsetrectcap%
\pgfsetroundjoin%
\pgfsetlinewidth{0.803000pt}%
\definecolor{currentstroke}{rgb}{0.000000,0.000000,0.000000}%
\pgfsetstrokecolor{currentstroke}%
\pgfsetdash{}{0pt}%
\pgfpathmoveto{\pgfqpoint{6.391709in}{2.177762in}}%
\pgfpathlineto{\pgfqpoint{6.495528in}{4.444907in}}%
\pgfusepath{stroke}%
\end{pgfscope}%
\begin{pgfscope}%
\definecolor{textcolor}{rgb}{0.000000,0.000000,0.000000}%
\pgfsetstrokecolor{textcolor}%
\pgfsetfillcolor{textcolor}%
\pgftext[x=7.004475in,y=3.361793in,,,rotate=87.378092]{\color{textcolor}\sffamily\fontsize{10.000000}{12.000000}\selectfont f(x,y)}%
\end{pgfscope}%
\begin{pgfscope}%
\pgfsetbuttcap%
\pgfsetroundjoin%
\pgfsetlinewidth{0.803000pt}%
\definecolor{currentstroke}{rgb}{0.690196,0.690196,0.690196}%
\pgfsetstrokecolor{currentstroke}%
\pgfsetdash{}{0pt}%
\pgfpathmoveto{\pgfqpoint{6.406399in}{2.498562in}}%
\pgfpathlineto{\pgfqpoint{3.478934in}{3.334033in}}%
\pgfpathlineto{\pgfqpoint{1.653696in}{1.832177in}}%
\pgfusepath{stroke}%
\end{pgfscope}%
\begin{pgfscope}%
\pgfsetbuttcap%
\pgfsetroundjoin%
\pgfsetlinewidth{0.803000pt}%
\definecolor{currentstroke}{rgb}{0.690196,0.690196,0.690196}%
\pgfsetstrokecolor{currentstroke}%
\pgfsetdash{}{0pt}%
\pgfpathmoveto{\pgfqpoint{6.426005in}{2.926700in}}%
\pgfpathlineto{\pgfqpoint{3.474169in}{3.748377in}}%
\pgfpathlineto{\pgfqpoint{1.632572in}{2.270847in}}%
\pgfusepath{stroke}%
\end{pgfscope}%
\begin{pgfscope}%
\pgfsetbuttcap%
\pgfsetroundjoin%
\pgfsetlinewidth{0.803000pt}%
\definecolor{currentstroke}{rgb}{0.690196,0.690196,0.690196}%
\pgfsetstrokecolor{currentstroke}%
\pgfsetdash{}{0pt}%
\pgfpathmoveto{\pgfqpoint{6.445943in}{3.362098in}}%
\pgfpathlineto{\pgfqpoint{3.469328in}{4.169408in}}%
\pgfpathlineto{\pgfqpoint{1.611075in}{2.717242in}}%
\pgfusepath{stroke}%
\end{pgfscope}%
\begin{pgfscope}%
\pgfsetbuttcap%
\pgfsetroundjoin%
\pgfsetlinewidth{0.803000pt}%
\definecolor{currentstroke}{rgb}{0.690196,0.690196,0.690196}%
\pgfsetstrokecolor{currentstroke}%
\pgfsetdash{}{0pt}%
\pgfpathmoveto{\pgfqpoint{6.466222in}{3.804941in}}%
\pgfpathlineto{\pgfqpoint{3.464407in}{4.597288in}}%
\pgfpathlineto{\pgfqpoint{1.589196in}{3.171568in}}%
\pgfusepath{stroke}%
\end{pgfscope}%
\begin{pgfscope}%
\pgfsetbuttcap%
\pgfsetroundjoin%
\pgfsetlinewidth{0.803000pt}%
\definecolor{currentstroke}{rgb}{0.690196,0.690196,0.690196}%
\pgfsetstrokecolor{currentstroke}%
\pgfsetdash{}{0pt}%
\pgfpathmoveto{\pgfqpoint{6.486851in}{4.255422in}}%
\pgfpathlineto{\pgfqpoint{3.459406in}{5.032186in}}%
\pgfpathlineto{\pgfqpoint{1.566925in}{3.634036in}}%
\pgfusepath{stroke}%
\end{pgfscope}%
\begin{pgfscope}%
\pgfsetrectcap%
\pgfsetroundjoin%
\pgfsetlinewidth{0.803000pt}%
\definecolor{currentstroke}{rgb}{0.000000,0.000000,0.000000}%
\pgfsetstrokecolor{currentstroke}%
\pgfsetdash{}{0pt}%
\pgfpathmoveto{\pgfqpoint{6.381822in}{2.505576in}}%
\pgfpathlineto{\pgfqpoint{6.455612in}{2.484517in}}%
\pgfusepath{stroke}%
\end{pgfscope}%
\begin{pgfscope}%
\definecolor{textcolor}{rgb}{0.000000,0.000000,0.000000}%
\pgfsetstrokecolor{textcolor}%
\pgfsetfillcolor{textcolor}%
\pgftext[x=6.661676in,y=2.534355in,,top]{\color{textcolor}\sffamily\fontsize{10.000000}{12.000000}\selectfont 2}%
\end{pgfscope}%
\begin{pgfscope}%
\pgfsetrectcap%
\pgfsetroundjoin%
\pgfsetlinewidth{0.803000pt}%
\definecolor{currentstroke}{rgb}{0.000000,0.000000,0.000000}%
\pgfsetstrokecolor{currentstroke}%
\pgfsetdash{}{0pt}%
\pgfpathmoveto{\pgfqpoint{6.401214in}{2.933601in}}%
\pgfpathlineto{\pgfqpoint{6.475647in}{2.912882in}}%
\pgfusepath{stroke}%
\end{pgfscope}%
\begin{pgfscope}%
\definecolor{textcolor}{rgb}{0.000000,0.000000,0.000000}%
\pgfsetstrokecolor{textcolor}%
\pgfsetfillcolor{textcolor}%
\pgftext[x=6.683383in,y=2.961916in,,top]{\color{textcolor}\sffamily\fontsize{10.000000}{12.000000}\selectfont 3}%
\end{pgfscope}%
\begin{pgfscope}%
\pgfsetrectcap%
\pgfsetroundjoin%
\pgfsetlinewidth{0.803000pt}%
\definecolor{currentstroke}{rgb}{0.000000,0.000000,0.000000}%
\pgfsetstrokecolor{currentstroke}%
\pgfsetdash{}{0pt}%
\pgfpathmoveto{\pgfqpoint{6.420934in}{3.368881in}}%
\pgfpathlineto{\pgfqpoint{6.496023in}{3.348515in}}%
\pgfusepath{stroke}%
\end{pgfscope}%
\begin{pgfscope}%
\definecolor{textcolor}{rgb}{0.000000,0.000000,0.000000}%
\pgfsetstrokecolor{textcolor}%
\pgfsetfillcolor{textcolor}%
\pgftext[x=6.705458in,y=3.396711in,,top]{\color{textcolor}\sffamily\fontsize{10.000000}{12.000000}\selectfont 4}%
\end{pgfscope}%
\begin{pgfscope}%
\pgfsetrectcap%
\pgfsetroundjoin%
\pgfsetlinewidth{0.803000pt}%
\definecolor{currentstroke}{rgb}{0.000000,0.000000,0.000000}%
\pgfsetstrokecolor{currentstroke}%
\pgfsetdash{}{0pt}%
\pgfpathmoveto{\pgfqpoint{6.440991in}{3.811601in}}%
\pgfpathlineto{\pgfqpoint{6.516747in}{3.791604in}}%
\pgfusepath{stroke}%
\end{pgfscope}%
\begin{pgfscope}%
\definecolor{textcolor}{rgb}{0.000000,0.000000,0.000000}%
\pgfsetstrokecolor{textcolor}%
\pgfsetfillcolor{textcolor}%
\pgftext[x=6.727909in,y=3.838926in,,top]{\color{textcolor}\sffamily\fontsize{10.000000}{12.000000}\selectfont 5}%
\end{pgfscope}%
\begin{pgfscope}%
\pgfsetrectcap%
\pgfsetroundjoin%
\pgfsetlinewidth{0.803000pt}%
\definecolor{currentstroke}{rgb}{0.000000,0.000000,0.000000}%
\pgfsetstrokecolor{currentstroke}%
\pgfsetdash{}{0pt}%
\pgfpathmoveto{\pgfqpoint{6.461394in}{4.261954in}}%
\pgfpathlineto{\pgfqpoint{6.537829in}{4.242342in}}%
\pgfusepath{stroke}%
\end{pgfscope}%
\begin{pgfscope}%
\definecolor{textcolor}{rgb}{0.000000,0.000000,0.000000}%
\pgfsetstrokecolor{textcolor}%
\pgfsetfillcolor{textcolor}%
\pgftext[x=6.750746in,y=4.288752in,,top]{\color{textcolor}\sffamily\fontsize{10.000000}{12.000000}\selectfont 6}%
\end{pgfscope}%
\begin{pgfscope}%
\pgfpathrectangle{\pgfqpoint{1.254980in}{0.150000in}}{\pgfqpoint{5.490039in}{5.490039in}}%
\pgfusepath{clip}%
\pgfsetrectcap%
\pgfsetroundjoin%
\pgfsetlinewidth{1.505625pt}%
\definecolor{currentstroke}{rgb}{1.000000,0.000000,0.000000}%
\pgfsetstrokecolor{currentstroke}%
\pgfsetdash{}{0pt}%
\pgfpathmoveto{\pgfqpoint{4.323943in}{2.397921in}}%
\pgfpathlineto{\pgfqpoint{3.818694in}{1.796349in}}%
\pgfpathlineto{\pgfqpoint{4.096329in}{1.615481in}}%
\pgfpathlineto{\pgfqpoint{4.020526in}{1.278201in}}%
\pgfpathlineto{\pgfqpoint{4.095598in}{1.220519in}}%
\pgfpathlineto{\pgfqpoint{4.212645in}{1.226359in}}%
\pgfpathlineto{\pgfqpoint{4.111851in}{1.272754in}}%
\pgfpathlineto{\pgfqpoint{4.134800in}{1.320760in}}%
\pgfpathlineto{\pgfqpoint{4.073423in}{1.343880in}}%
\pgfpathlineto{\pgfqpoint{4.077168in}{1.370665in}}%
\pgfpathlineto{\pgfqpoint{4.087741in}{1.369111in}}%
\pgfpathlineto{\pgfqpoint{4.092030in}{1.364848in}}%
\pgfpathlineto{\pgfqpoint{4.091351in}{1.347592in}}%
\pgfpathlineto{\pgfqpoint{4.093183in}{1.347595in}}%
\pgfpathlineto{\pgfqpoint{4.095965in}{1.350403in}}%
\pgfpathlineto{\pgfqpoint{4.095054in}{1.355138in}}%
\pgfpathlineto{\pgfqpoint{4.084623in}{1.363180in}}%
\pgfpathlineto{\pgfqpoint{4.083232in}{1.360191in}}%
\pgfpathlineto{\pgfqpoint{4.089581in}{1.354259in}}%
\pgfpathlineto{\pgfqpoint{4.090492in}{1.354967in}}%
\pgfpathlineto{\pgfqpoint{4.088194in}{1.358330in}}%
\pgfpathlineto{\pgfqpoint{4.087632in}{1.358278in}}%
\pgfusepath{stroke}%
\end{pgfscope}%
\begin{pgfscope}%
\pgfpathrectangle{\pgfqpoint{1.254980in}{0.150000in}}{\pgfqpoint{5.490039in}{5.490039in}}%
\pgfusepath{clip}%
\pgfsetbuttcap%
\pgfsetroundjoin%
\definecolor{currentfill}{rgb}{1.000000,0.000000,0.000000}%
\pgfsetfillcolor{currentfill}%
\pgfsetlinewidth{1.003750pt}%
\definecolor{currentstroke}{rgb}{1.000000,0.000000,0.000000}%
\pgfsetstrokecolor{currentstroke}%
\pgfsetdash{}{0pt}%
\pgfsys@defobject{currentmarker}{\pgfqpoint{-0.041667in}{-0.041667in}}{\pgfqpoint{0.041667in}{0.041667in}}{%
\pgfpathmoveto{\pgfqpoint{0.000000in}{-0.041667in}}%
\pgfpathcurveto{\pgfqpoint{0.011050in}{-0.041667in}}{\pgfqpoint{0.021649in}{-0.037276in}}{\pgfqpoint{0.029463in}{-0.029463in}}%
\pgfpathcurveto{\pgfqpoint{0.037276in}{-0.021649in}}{\pgfqpoint{0.041667in}{-0.011050in}}{\pgfqpoint{0.041667in}{0.000000in}}%
\pgfpathcurveto{\pgfqpoint{0.041667in}{0.011050in}}{\pgfqpoint{0.037276in}{0.021649in}}{\pgfqpoint{0.029463in}{0.029463in}}%
\pgfpathcurveto{\pgfqpoint{0.021649in}{0.037276in}}{\pgfqpoint{0.011050in}{0.041667in}}{\pgfqpoint{0.000000in}{0.041667in}}%
\pgfpathcurveto{\pgfqpoint{-0.011050in}{0.041667in}}{\pgfqpoint{-0.021649in}{0.037276in}}{\pgfqpoint{-0.029463in}{0.029463in}}%
\pgfpathcurveto{\pgfqpoint{-0.037276in}{0.021649in}}{\pgfqpoint{-0.041667in}{0.011050in}}{\pgfqpoint{-0.041667in}{0.000000in}}%
\pgfpathcurveto{\pgfqpoint{-0.041667in}{-0.011050in}}{\pgfqpoint{-0.037276in}{-0.021649in}}{\pgfqpoint{-0.029463in}{-0.029463in}}%
\pgfpathcurveto{\pgfqpoint{-0.021649in}{-0.037276in}}{\pgfqpoint{-0.011050in}{-0.041667in}}{\pgfqpoint{0.000000in}{-0.041667in}}%
\pgfpathlineto{\pgfqpoint{0.000000in}{-0.041667in}}%
\pgfpathclose%
\pgfusepath{stroke,fill}%
}%
\begin{pgfscope}%
\pgfsys@transformshift{4.323943in}{2.397921in}%
\pgfsys@useobject{currentmarker}{}%
\end{pgfscope}%
\begin{pgfscope}%
\pgfsys@transformshift{3.818694in}{1.796349in}%
\pgfsys@useobject{currentmarker}{}%
\end{pgfscope}%
\begin{pgfscope}%
\pgfsys@transformshift{4.096329in}{1.615481in}%
\pgfsys@useobject{currentmarker}{}%
\end{pgfscope}%
\begin{pgfscope}%
\pgfsys@transformshift{4.020526in}{1.278201in}%
\pgfsys@useobject{currentmarker}{}%
\end{pgfscope}%
\begin{pgfscope}%
\pgfsys@transformshift{4.095598in}{1.220519in}%
\pgfsys@useobject{currentmarker}{}%
\end{pgfscope}%
\begin{pgfscope}%
\pgfsys@transformshift{4.212645in}{1.226359in}%
\pgfsys@useobject{currentmarker}{}%
\end{pgfscope}%
\begin{pgfscope}%
\pgfsys@transformshift{4.111851in}{1.272754in}%
\pgfsys@useobject{currentmarker}{}%
\end{pgfscope}%
\begin{pgfscope}%
\pgfsys@transformshift{4.134800in}{1.320760in}%
\pgfsys@useobject{currentmarker}{}%
\end{pgfscope}%
\begin{pgfscope}%
\pgfsys@transformshift{4.073423in}{1.343880in}%
\pgfsys@useobject{currentmarker}{}%
\end{pgfscope}%
\begin{pgfscope}%
\pgfsys@transformshift{4.077168in}{1.370665in}%
\pgfsys@useobject{currentmarker}{}%
\end{pgfscope}%
\begin{pgfscope}%
\pgfsys@transformshift{4.087741in}{1.369111in}%
\pgfsys@useobject{currentmarker}{}%
\end{pgfscope}%
\begin{pgfscope}%
\pgfsys@transformshift{4.092030in}{1.364848in}%
\pgfsys@useobject{currentmarker}{}%
\end{pgfscope}%
\begin{pgfscope}%
\pgfsys@transformshift{4.091351in}{1.347592in}%
\pgfsys@useobject{currentmarker}{}%
\end{pgfscope}%
\begin{pgfscope}%
\pgfsys@transformshift{4.093183in}{1.347595in}%
\pgfsys@useobject{currentmarker}{}%
\end{pgfscope}%
\begin{pgfscope}%
\pgfsys@transformshift{4.095965in}{1.350403in}%
\pgfsys@useobject{currentmarker}{}%
\end{pgfscope}%
\begin{pgfscope}%
\pgfsys@transformshift{4.095054in}{1.355138in}%
\pgfsys@useobject{currentmarker}{}%
\end{pgfscope}%
\begin{pgfscope}%
\pgfsys@transformshift{4.084623in}{1.363180in}%
\pgfsys@useobject{currentmarker}{}%
\end{pgfscope}%
\begin{pgfscope}%
\pgfsys@transformshift{4.083232in}{1.360191in}%
\pgfsys@useobject{currentmarker}{}%
\end{pgfscope}%
\begin{pgfscope}%
\pgfsys@transformshift{4.089581in}{1.354259in}%
\pgfsys@useobject{currentmarker}{}%
\end{pgfscope}%
\begin{pgfscope}%
\pgfsys@transformshift{4.090492in}{1.354967in}%
\pgfsys@useobject{currentmarker}{}%
\end{pgfscope}%
\begin{pgfscope}%
\pgfsys@transformshift{4.088194in}{1.358330in}%
\pgfsys@useobject{currentmarker}{}%
\end{pgfscope}%
\begin{pgfscope}%
\pgfsys@transformshift{4.087632in}{1.358278in}%
\pgfsys@useobject{currentmarker}{}%
\end{pgfscope}%
\end{pgfscope}%
\begin{pgfscope}%
\pgfpathrectangle{\pgfqpoint{1.254980in}{0.150000in}}{\pgfqpoint{5.490039in}{5.490039in}}%
\pgfusepath{clip}%
\pgfsetbuttcap%
\pgfsetroundjoin%
\definecolor{currentfill}{rgb}{0.000000,0.000000,1.000000}%
\pgfsetfillcolor{currentfill}%
\pgfsetlinewidth{1.003750pt}%
\definecolor{currentstroke}{rgb}{0.000000,0.000000,1.000000}%
\pgfsetstrokecolor{currentstroke}%
\pgfsetdash{}{0pt}%
\pgfsys@defobject{currentmarker}{\pgfqpoint{-0.069444in}{-0.069444in}}{\pgfqpoint{0.069444in}{0.069444in}}{%
\pgfpathmoveto{\pgfqpoint{0.000000in}{-0.069444in}}%
\pgfpathcurveto{\pgfqpoint{0.018417in}{-0.069444in}}{\pgfqpoint{0.036082in}{-0.062127in}}{\pgfqpoint{0.049105in}{-0.049105in}}%
\pgfpathcurveto{\pgfqpoint{0.062127in}{-0.036082in}}{\pgfqpoint{0.069444in}{-0.018417in}}{\pgfqpoint{0.069444in}{0.000000in}}%
\pgfpathcurveto{\pgfqpoint{0.069444in}{0.018417in}}{\pgfqpoint{0.062127in}{0.036082in}}{\pgfqpoint{0.049105in}{0.049105in}}%
\pgfpathcurveto{\pgfqpoint{0.036082in}{0.062127in}}{\pgfqpoint{0.018417in}{0.069444in}}{\pgfqpoint{0.000000in}{0.069444in}}%
\pgfpathcurveto{\pgfqpoint{-0.018417in}{0.069444in}}{\pgfqpoint{-0.036082in}{0.062127in}}{\pgfqpoint{-0.049105in}{0.049105in}}%
\pgfpathcurveto{\pgfqpoint{-0.062127in}{0.036082in}}{\pgfqpoint{-0.069444in}{0.018417in}}{\pgfqpoint{-0.069444in}{0.000000in}}%
\pgfpathcurveto{\pgfqpoint{-0.069444in}{-0.018417in}}{\pgfqpoint{-0.062127in}{-0.036082in}}{\pgfqpoint{-0.049105in}{-0.049105in}}%
\pgfpathcurveto{\pgfqpoint{-0.036082in}{-0.062127in}}{\pgfqpoint{-0.018417in}{-0.069444in}}{\pgfqpoint{0.000000in}{-0.069444in}}%
\pgfpathlineto{\pgfqpoint{0.000000in}{-0.069444in}}%
\pgfpathclose%
\pgfusepath{stroke,fill}%
}%
\begin{pgfscope}%
\pgfsys@transformshift{4.087632in}{1.358278in}%
\pgfsys@useobject{currentmarker}{}%
\end{pgfscope}%
\end{pgfscope}%
\begin{pgfscope}%
\pgfpathrectangle{\pgfqpoint{1.254980in}{0.150000in}}{\pgfqpoint{5.490039in}{5.490039in}}%
\pgfusepath{clip}%
\pgfsetbuttcap%
\pgfsetroundjoin%
\definecolor{currentfill}{rgb}{0.136408,0.541173,0.554483}%
\pgfsetfillcolor{currentfill}%
\pgfsetfillopacity{0.700000}%
\pgfsetlinewidth{0.000000pt}%
\definecolor{currentstroke}{rgb}{0.000000,0.000000,0.000000}%
\pgfsetstrokecolor{currentstroke}%
\pgfsetdash{}{0pt}%
\pgfpathmoveto{\pgfqpoint{3.975071in}{3.728190in}}%
\pgfpathlineto{\pgfqpoint{3.987802in}{3.713500in}}%
\pgfpathlineto{\pgfqpoint{4.000533in}{3.698976in}}%
\pgfpathlineto{\pgfqpoint{4.013263in}{3.684619in}}%
\pgfpathlineto{\pgfqpoint{4.025994in}{3.670426in}}%
\pgfpathlineto{\pgfqpoint{4.033351in}{3.686859in}}%
\pgfpathlineto{\pgfqpoint{4.040706in}{3.703453in}}%
\pgfpathlineto{\pgfqpoint{4.048058in}{3.720208in}}%
\pgfpathlineto{\pgfqpoint{4.035334in}{3.734583in}}%
\pgfpathlineto{\pgfqpoint{4.022609in}{3.749122in}}%
\pgfpathlineto{\pgfqpoint{4.009885in}{3.763828in}}%
\pgfpathlineto{\pgfqpoint{3.997160in}{3.778701in}}%
\pgfpathlineto{\pgfqpoint{3.989800in}{3.761698in}}%
\pgfpathlineto{\pgfqpoint{3.982437in}{3.744862in}}%
\pgfpathlineto{\pgfqpoint{3.975071in}{3.728190in}}%
\pgfpathclose%
\pgfusepath{fill}%
\end{pgfscope}%
\begin{pgfscope}%
\pgfpathrectangle{\pgfqpoint{1.254980in}{0.150000in}}{\pgfqpoint{5.490039in}{5.490039in}}%
\pgfusepath{clip}%
\pgfsetbuttcap%
\pgfsetroundjoin%
\definecolor{currentfill}{rgb}{0.128729,0.563265,0.551229}%
\pgfsetfillcolor{currentfill}%
\pgfsetfillopacity{0.700000}%
\pgfsetlinewidth{0.000000pt}%
\definecolor{currentstroke}{rgb}{0.000000,0.000000,0.000000}%
\pgfsetstrokecolor{currentstroke}%
\pgfsetdash{}{0pt}%
\pgfpathmoveto{\pgfqpoint{3.924143in}{3.788641in}}%
\pgfpathlineto{\pgfqpoint{3.936876in}{3.773273in}}%
\pgfpathlineto{\pgfqpoint{3.949609in}{3.758076in}}%
\pgfpathlineto{\pgfqpoint{3.962340in}{3.743049in}}%
\pgfpathlineto{\pgfqpoint{3.975071in}{3.728190in}}%
\pgfpathlineto{\pgfqpoint{3.982437in}{3.744862in}}%
\pgfpathlineto{\pgfqpoint{3.989800in}{3.761698in}}%
\pgfpathlineto{\pgfqpoint{3.997160in}{3.778701in}}%
\pgfpathlineto{\pgfqpoint{3.984435in}{3.793742in}}%
\pgfpathlineto{\pgfqpoint{3.971709in}{3.808952in}}%
\pgfpathlineto{\pgfqpoint{3.958983in}{3.824332in}}%
\pgfpathlineto{\pgfqpoint{3.946256in}{3.839883in}}%
\pgfpathlineto{\pgfqpoint{3.938888in}{3.822632in}}%
\pgfpathlineto{\pgfqpoint{3.931517in}{3.805552in}}%
\pgfpathlineto{\pgfqpoint{3.924143in}{3.788641in}}%
\pgfpathclose%
\pgfusepath{fill}%
\end{pgfscope}%
\begin{pgfscope}%
\pgfpathrectangle{\pgfqpoint{1.254980in}{0.150000in}}{\pgfqpoint{5.490039in}{5.490039in}}%
\pgfusepath{clip}%
\pgfsetbuttcap%
\pgfsetroundjoin%
\definecolor{currentfill}{rgb}{0.144759,0.519093,0.556572}%
\pgfsetfillcolor{currentfill}%
\pgfsetfillopacity{0.700000}%
\pgfsetlinewidth{0.000000pt}%
\definecolor{currentstroke}{rgb}{0.000000,0.000000,0.000000}%
\pgfsetstrokecolor{currentstroke}%
\pgfsetdash{}{0pt}%
\pgfpathmoveto{\pgfqpoint{4.025994in}{3.670426in}}%
\pgfpathlineto{\pgfqpoint{4.038724in}{3.656398in}}%
\pgfpathlineto{\pgfqpoint{4.051455in}{3.642533in}}%
\pgfpathlineto{\pgfqpoint{4.064186in}{3.628830in}}%
\pgfpathlineto{\pgfqpoint{4.076918in}{3.615288in}}%
\pgfpathlineto{\pgfqpoint{4.084267in}{3.631484in}}%
\pgfpathlineto{\pgfqpoint{4.091614in}{3.647835in}}%
\pgfpathlineto{\pgfqpoint{4.098958in}{3.664345in}}%
\pgfpathlineto{\pgfqpoint{4.086232in}{3.678067in}}%
\pgfpathlineto{\pgfqpoint{4.073507in}{3.691952in}}%
\pgfpathlineto{\pgfqpoint{4.060782in}{3.705998in}}%
\pgfpathlineto{\pgfqpoint{4.048058in}{3.720208in}}%
\pgfpathlineto{\pgfqpoint{4.040706in}{3.703453in}}%
\pgfpathlineto{\pgfqpoint{4.033351in}{3.686859in}}%
\pgfpathlineto{\pgfqpoint{4.025994in}{3.670426in}}%
\pgfpathclose%
\pgfusepath{fill}%
\end{pgfscope}%
\begin{pgfscope}%
\pgfpathrectangle{\pgfqpoint{1.254980in}{0.150000in}}{\pgfqpoint{5.490039in}{5.490039in}}%
\pgfusepath{clip}%
\pgfsetbuttcap%
\pgfsetroundjoin%
\definecolor{currentfill}{rgb}{0.121831,0.589055,0.545623}%
\pgfsetfillcolor{currentfill}%
\pgfsetfillopacity{0.700000}%
\pgfsetlinewidth{0.000000pt}%
\definecolor{currentstroke}{rgb}{0.000000,0.000000,0.000000}%
\pgfsetstrokecolor{currentstroke}%
\pgfsetdash{}{0pt}%
\pgfpathmoveto{\pgfqpoint{3.873201in}{3.851841in}}%
\pgfpathlineto{\pgfqpoint{3.885939in}{3.835780in}}%
\pgfpathlineto{\pgfqpoint{3.898675in}{3.819893in}}%
\pgfpathlineto{\pgfqpoint{3.911409in}{3.804181in}}%
\pgfpathlineto{\pgfqpoint{3.924143in}{3.788641in}}%
\pgfpathlineto{\pgfqpoint{3.931517in}{3.805552in}}%
\pgfpathlineto{\pgfqpoint{3.938888in}{3.822632in}}%
\pgfpathlineto{\pgfqpoint{3.946256in}{3.839883in}}%
\pgfpathlineto{\pgfqpoint{3.933528in}{3.855606in}}%
\pgfpathlineto{\pgfqpoint{3.920799in}{3.871502in}}%
\pgfpathlineto{\pgfqpoint{3.908069in}{3.887572in}}%
\pgfpathlineto{\pgfqpoint{3.895338in}{3.903818in}}%
\pgfpathlineto{\pgfqpoint{3.887962in}{3.886318in}}%
\pgfpathlineto{\pgfqpoint{3.880583in}{3.868993in}}%
\pgfpathlineto{\pgfqpoint{3.873201in}{3.851841in}}%
\pgfpathclose%
\pgfusepath{fill}%
\end{pgfscope}%
\begin{pgfscope}%
\pgfpathrectangle{\pgfqpoint{1.254980in}{0.150000in}}{\pgfqpoint{5.490039in}{5.490039in}}%
\pgfusepath{clip}%
\pgfsetbuttcap%
\pgfsetroundjoin%
\definecolor{currentfill}{rgb}{0.153364,0.497000,0.557724}%
\pgfsetfillcolor{currentfill}%
\pgfsetfillopacity{0.700000}%
\pgfsetlinewidth{0.000000pt}%
\definecolor{currentstroke}{rgb}{0.000000,0.000000,0.000000}%
\pgfsetstrokecolor{currentstroke}%
\pgfsetdash{}{0pt}%
\pgfpathmoveto{\pgfqpoint{4.076918in}{3.615288in}}%
\pgfpathlineto{\pgfqpoint{4.089651in}{3.601908in}}%
\pgfpathlineto{\pgfqpoint{4.102384in}{3.588686in}}%
\pgfpathlineto{\pgfqpoint{4.115118in}{3.575624in}}%
\pgfpathlineto{\pgfqpoint{4.127853in}{3.562720in}}%
\pgfpathlineto{\pgfqpoint{4.135193in}{3.578678in}}%
\pgfpathlineto{\pgfqpoint{4.142531in}{3.594789in}}%
\pgfpathlineto{\pgfqpoint{4.149867in}{3.611053in}}%
\pgfpathlineto{\pgfqpoint{4.137138in}{3.624138in}}%
\pgfpathlineto{\pgfqpoint{4.124411in}{3.637381in}}%
\pgfpathlineto{\pgfqpoint{4.111684in}{3.650783in}}%
\pgfpathlineto{\pgfqpoint{4.098958in}{3.664345in}}%
\pgfpathlineto{\pgfqpoint{4.091614in}{3.647835in}}%
\pgfpathlineto{\pgfqpoint{4.084267in}{3.631484in}}%
\pgfpathlineto{\pgfqpoint{4.076918in}{3.615288in}}%
\pgfpathclose%
\pgfusepath{fill}%
\end{pgfscope}%
\begin{pgfscope}%
\pgfpathrectangle{\pgfqpoint{1.254980in}{0.150000in}}{\pgfqpoint{5.490039in}{5.490039in}}%
\pgfusepath{clip}%
\pgfsetbuttcap%
\pgfsetroundjoin%
\definecolor{currentfill}{rgb}{0.119483,0.614817,0.537692}%
\pgfsetfillcolor{currentfill}%
\pgfsetfillopacity{0.700000}%
\pgfsetlinewidth{0.000000pt}%
\definecolor{currentstroke}{rgb}{0.000000,0.000000,0.000000}%
\pgfsetstrokecolor{currentstroke}%
\pgfsetdash{}{0pt}%
\pgfpathmoveto{\pgfqpoint{3.822238in}{3.917858in}}%
\pgfpathlineto{\pgfqpoint{3.834981in}{3.901086in}}%
\pgfpathlineto{\pgfqpoint{3.847723in}{3.884493in}}%
\pgfpathlineto{\pgfqpoint{3.860463in}{3.868079in}}%
\pgfpathlineto{\pgfqpoint{3.873201in}{3.851841in}}%
\pgfpathlineto{\pgfqpoint{3.880583in}{3.868993in}}%
\pgfpathlineto{\pgfqpoint{3.887962in}{3.886318in}}%
\pgfpathlineto{\pgfqpoint{3.895338in}{3.903818in}}%
\pgfpathlineto{\pgfqpoint{3.882605in}{3.920240in}}%
\pgfpathlineto{\pgfqpoint{3.869871in}{3.936839in}}%
\pgfpathlineto{\pgfqpoint{3.857135in}{3.953616in}}%
\pgfpathlineto{\pgfqpoint{3.844398in}{3.970573in}}%
\pgfpathlineto{\pgfqpoint{3.837014in}{3.952823in}}%
\pgfpathlineto{\pgfqpoint{3.829627in}{3.935252in}}%
\pgfpathlineto{\pgfqpoint{3.822238in}{3.917858in}}%
\pgfpathclose%
\pgfusepath{fill}%
\end{pgfscope}%
\begin{pgfscope}%
\pgfpathrectangle{\pgfqpoint{1.254980in}{0.150000in}}{\pgfqpoint{5.490039in}{5.490039in}}%
\pgfusepath{clip}%
\pgfsetbuttcap%
\pgfsetroundjoin%
\definecolor{currentfill}{rgb}{0.160665,0.478540,0.558115}%
\pgfsetfillcolor{currentfill}%
\pgfsetfillopacity{0.700000}%
\pgfsetlinewidth{0.000000pt}%
\definecolor{currentstroke}{rgb}{0.000000,0.000000,0.000000}%
\pgfsetstrokecolor{currentstroke}%
\pgfsetdash{}{0pt}%
\pgfpathmoveto{\pgfqpoint{4.127853in}{3.562720in}}%
\pgfpathlineto{\pgfqpoint{4.140589in}{3.549973in}}%
\pgfpathlineto{\pgfqpoint{4.153326in}{3.537382in}}%
\pgfpathlineto{\pgfqpoint{4.166064in}{3.524947in}}%
\pgfpathlineto{\pgfqpoint{4.178804in}{3.512667in}}%
\pgfpathlineto{\pgfqpoint{4.186136in}{3.528389in}}%
\pgfpathlineto{\pgfqpoint{4.193466in}{3.544259in}}%
\pgfpathlineto{\pgfqpoint{4.200793in}{3.560279in}}%
\pgfpathlineto{\pgfqpoint{4.188059in}{3.572740in}}%
\pgfpathlineto{\pgfqpoint{4.175327in}{3.585355in}}%
\pgfpathlineto{\pgfqpoint{4.162596in}{3.598126in}}%
\pgfpathlineto{\pgfqpoint{4.149867in}{3.611053in}}%
\pgfpathlineto{\pgfqpoint{4.142531in}{3.594789in}}%
\pgfpathlineto{\pgfqpoint{4.135193in}{3.578678in}}%
\pgfpathlineto{\pgfqpoint{4.127853in}{3.562720in}}%
\pgfpathclose%
\pgfusepath{fill}%
\end{pgfscope}%
\begin{pgfscope}%
\pgfpathrectangle{\pgfqpoint{1.254980in}{0.150000in}}{\pgfqpoint{5.490039in}{5.490039in}}%
\pgfusepath{clip}%
\pgfsetbuttcap%
\pgfsetroundjoin%
\definecolor{currentfill}{rgb}{0.124780,0.640461,0.527068}%
\pgfsetfillcolor{currentfill}%
\pgfsetfillopacity{0.700000}%
\pgfsetlinewidth{0.000000pt}%
\definecolor{currentstroke}{rgb}{0.000000,0.000000,0.000000}%
\pgfsetstrokecolor{currentstroke}%
\pgfsetdash{}{0pt}%
\pgfpathmoveto{\pgfqpoint{3.771243in}{3.986762in}}%
\pgfpathlineto{\pgfqpoint{3.783995in}{3.969261in}}%
\pgfpathlineto{\pgfqpoint{3.796745in}{3.951945in}}%
\pgfpathlineto{\pgfqpoint{3.809492in}{3.934811in}}%
\pgfpathlineto{\pgfqpoint{3.822238in}{3.917858in}}%
\pgfpathlineto{\pgfqpoint{3.829627in}{3.935252in}}%
\pgfpathlineto{\pgfqpoint{3.837014in}{3.952823in}}%
\pgfpathlineto{\pgfqpoint{3.844398in}{3.970573in}}%
\pgfpathlineto{\pgfqpoint{3.831658in}{3.987711in}}%
\pgfpathlineto{\pgfqpoint{3.818916in}{4.005030in}}%
\pgfpathlineto{\pgfqpoint{3.806172in}{4.022532in}}%
\pgfpathlineto{\pgfqpoint{3.793426in}{4.040219in}}%
\pgfpathlineto{\pgfqpoint{3.786035in}{4.022217in}}%
\pgfpathlineto{\pgfqpoint{3.778641in}{4.004399in}}%
\pgfpathlineto{\pgfqpoint{3.771243in}{3.986762in}}%
\pgfpathclose%
\pgfusepath{fill}%
\end{pgfscope}%
\begin{pgfscope}%
\pgfpathrectangle{\pgfqpoint{1.254980in}{0.150000in}}{\pgfqpoint{5.490039in}{5.490039in}}%
\pgfusepath{clip}%
\pgfsetbuttcap%
\pgfsetroundjoin%
\definecolor{currentfill}{rgb}{0.168126,0.459988,0.558082}%
\pgfsetfillcolor{currentfill}%
\pgfsetfillopacity{0.700000}%
\pgfsetlinewidth{0.000000pt}%
\definecolor{currentstroke}{rgb}{0.000000,0.000000,0.000000}%
\pgfsetstrokecolor{currentstroke}%
\pgfsetdash{}{0pt}%
\pgfpathmoveto{\pgfqpoint{4.178804in}{3.512667in}}%
\pgfpathlineto{\pgfqpoint{4.191546in}{3.500540in}}%
\pgfpathlineto{\pgfqpoint{4.204289in}{3.488567in}}%
\pgfpathlineto{\pgfqpoint{4.217034in}{3.476746in}}%
\pgfpathlineto{\pgfqpoint{4.229780in}{3.465077in}}%
\pgfpathlineto{\pgfqpoint{4.237104in}{3.480564in}}%
\pgfpathlineto{\pgfqpoint{4.244424in}{3.496194in}}%
\pgfpathlineto{\pgfqpoint{4.251743in}{3.511971in}}%
\pgfpathlineto{\pgfqpoint{4.239003in}{3.523819in}}%
\pgfpathlineto{\pgfqpoint{4.226264in}{3.535820in}}%
\pgfpathlineto{\pgfqpoint{4.213528in}{3.547973in}}%
\pgfpathlineto{\pgfqpoint{4.200793in}{3.560279in}}%
\pgfpathlineto{\pgfqpoint{4.193466in}{3.544259in}}%
\pgfpathlineto{\pgfqpoint{4.186136in}{3.528389in}}%
\pgfpathlineto{\pgfqpoint{4.178804in}{3.512667in}}%
\pgfpathclose%
\pgfusepath{fill}%
\end{pgfscope}%
\begin{pgfscope}%
\pgfpathrectangle{\pgfqpoint{1.254980in}{0.150000in}}{\pgfqpoint{5.490039in}{5.490039in}}%
\pgfusepath{clip}%
\pgfsetbuttcap%
\pgfsetroundjoin%
\definecolor{currentfill}{rgb}{0.143303,0.669459,0.511215}%
\pgfsetfillcolor{currentfill}%
\pgfsetfillopacity{0.700000}%
\pgfsetlinewidth{0.000000pt}%
\definecolor{currentstroke}{rgb}{0.000000,0.000000,0.000000}%
\pgfsetstrokecolor{currentstroke}%
\pgfsetdash{}{0pt}%
\pgfpathmoveto{\pgfqpoint{3.720209in}{4.058625in}}%
\pgfpathlineto{\pgfqpoint{3.732972in}{4.040378in}}%
\pgfpathlineto{\pgfqpoint{3.745732in}{4.022319in}}%
\pgfpathlineto{\pgfqpoint{3.758489in}{4.004447in}}%
\pgfpathlineto{\pgfqpoint{3.771243in}{3.986762in}}%
\pgfpathlineto{\pgfqpoint{3.778641in}{4.004399in}}%
\pgfpathlineto{\pgfqpoint{3.786035in}{4.022217in}}%
\pgfpathlineto{\pgfqpoint{3.793426in}{4.040219in}}%
\pgfpathlineto{\pgfqpoint{3.780678in}{4.058090in}}%
\pgfpathlineto{\pgfqpoint{3.767926in}{4.076148in}}%
\pgfpathlineto{\pgfqpoint{3.755172in}{4.094394in}}%
\pgfpathlineto{\pgfqpoint{3.742415in}{4.112829in}}%
\pgfpathlineto{\pgfqpoint{3.735017in}{4.094574in}}%
\pgfpathlineto{\pgfqpoint{3.727615in}{4.076507in}}%
\pgfpathlineto{\pgfqpoint{3.720209in}{4.058625in}}%
\pgfpathclose%
\pgfusepath{fill}%
\end{pgfscope}%
\begin{pgfscope}%
\pgfpathrectangle{\pgfqpoint{1.254980in}{0.150000in}}{\pgfqpoint{5.490039in}{5.490039in}}%
\pgfusepath{clip}%
\pgfsetbuttcap%
\pgfsetroundjoin%
\definecolor{currentfill}{rgb}{0.135066,0.544853,0.554029}%
\pgfsetfillcolor{currentfill}%
\pgfsetfillopacity{0.700000}%
\pgfsetlinewidth{0.000000pt}%
\definecolor{currentstroke}{rgb}{0.000000,0.000000,0.000000}%
\pgfsetstrokecolor{currentstroke}%
\pgfsetdash{}{0pt}%
\pgfpathmoveto{\pgfqpoint{3.894618in}{3.722629in}}%
\pgfpathlineto{\pgfqpoint{3.907360in}{3.707491in}}%
\pgfpathlineto{\pgfqpoint{3.920100in}{3.692523in}}%
\pgfpathlineto{\pgfqpoint{3.932840in}{3.677725in}}%
\pgfpathlineto{\pgfqpoint{3.945580in}{3.663095in}}%
\pgfpathlineto{\pgfqpoint{3.952957in}{3.679135in}}%
\pgfpathlineto{\pgfqpoint{3.960332in}{3.695329in}}%
\pgfpathlineto{\pgfqpoint{3.967703in}{3.711680in}}%
\pgfpathlineto{\pgfqpoint{3.975071in}{3.728190in}}%
\pgfpathlineto{\pgfqpoint{3.962340in}{3.743049in}}%
\pgfpathlineto{\pgfqpoint{3.949609in}{3.758076in}}%
\pgfpathlineto{\pgfqpoint{3.936876in}{3.773273in}}%
\pgfpathlineto{\pgfqpoint{3.924143in}{3.788641in}}%
\pgfpathlineto{\pgfqpoint{3.916767in}{3.771895in}}%
\pgfpathlineto{\pgfqpoint{3.909387in}{3.755313in}}%
\pgfpathlineto{\pgfqpoint{3.902004in}{3.738892in}}%
\pgfpathlineto{\pgfqpoint{3.894618in}{3.722629in}}%
\pgfpathclose%
\pgfusepath{fill}%
\end{pgfscope}%
\begin{pgfscope}%
\pgfpathrectangle{\pgfqpoint{1.254980in}{0.150000in}}{\pgfqpoint{5.490039in}{5.490039in}}%
\pgfusepath{clip}%
\pgfsetbuttcap%
\pgfsetroundjoin%
\definecolor{currentfill}{rgb}{0.143343,0.522773,0.556295}%
\pgfsetfillcolor{currentfill}%
\pgfsetfillopacity{0.700000}%
\pgfsetlinewidth{0.000000pt}%
\definecolor{currentstroke}{rgb}{0.000000,0.000000,0.000000}%
\pgfsetstrokecolor{currentstroke}%
\pgfsetdash{}{0pt}%
\pgfpathmoveto{\pgfqpoint{3.945580in}{3.663095in}}%
\pgfpathlineto{\pgfqpoint{3.958319in}{3.648634in}}%
\pgfpathlineto{\pgfqpoint{3.971058in}{3.634338in}}%
\pgfpathlineto{\pgfqpoint{3.983797in}{3.620209in}}%
\pgfpathlineto{\pgfqpoint{3.996536in}{3.606244in}}%
\pgfpathlineto{\pgfqpoint{4.003905in}{3.622062in}}%
\pgfpathlineto{\pgfqpoint{4.011271in}{3.638030in}}%
\pgfpathlineto{\pgfqpoint{4.018634in}{3.654150in}}%
\pgfpathlineto{\pgfqpoint{4.025994in}{3.670426in}}%
\pgfpathlineto{\pgfqpoint{4.013263in}{3.684619in}}%
\pgfpathlineto{\pgfqpoint{4.000533in}{3.698976in}}%
\pgfpathlineto{\pgfqpoint{3.987802in}{3.713500in}}%
\pgfpathlineto{\pgfqpoint{3.975071in}{3.728190in}}%
\pgfpathlineto{\pgfqpoint{3.967703in}{3.711680in}}%
\pgfpathlineto{\pgfqpoint{3.960332in}{3.695329in}}%
\pgfpathlineto{\pgfqpoint{3.952957in}{3.679135in}}%
\pgfpathlineto{\pgfqpoint{3.945580in}{3.663095in}}%
\pgfpathclose%
\pgfusepath{fill}%
\end{pgfscope}%
\begin{pgfscope}%
\pgfpathrectangle{\pgfqpoint{1.254980in}{0.150000in}}{\pgfqpoint{5.490039in}{5.490039in}}%
\pgfusepath{clip}%
\pgfsetbuttcap%
\pgfsetroundjoin%
\definecolor{currentfill}{rgb}{0.126453,0.570633,0.549841}%
\pgfsetfillcolor{currentfill}%
\pgfsetfillopacity{0.700000}%
\pgfsetlinewidth{0.000000pt}%
\definecolor{currentstroke}{rgb}{0.000000,0.000000,0.000000}%
\pgfsetstrokecolor{currentstroke}%
\pgfsetdash{}{0pt}%
\pgfpathmoveto{\pgfqpoint{3.843643in}{3.784909in}}%
\pgfpathlineto{\pgfqpoint{3.856389in}{3.769078in}}%
\pgfpathlineto{\pgfqpoint{3.869133in}{3.753422in}}%
\pgfpathlineto{\pgfqpoint{3.881876in}{3.737939in}}%
\pgfpathlineto{\pgfqpoint{3.894618in}{3.722629in}}%
\pgfpathlineto{\pgfqpoint{3.902004in}{3.738892in}}%
\pgfpathlineto{\pgfqpoint{3.909387in}{3.755313in}}%
\pgfpathlineto{\pgfqpoint{3.916767in}{3.771895in}}%
\pgfpathlineto{\pgfqpoint{3.924143in}{3.788641in}}%
\pgfpathlineto{\pgfqpoint{3.911409in}{3.804181in}}%
\pgfpathlineto{\pgfqpoint{3.898675in}{3.819893in}}%
\pgfpathlineto{\pgfqpoint{3.885939in}{3.835780in}}%
\pgfpathlineto{\pgfqpoint{3.873201in}{3.851841in}}%
\pgfpathlineto{\pgfqpoint{3.865817in}{3.834859in}}%
\pgfpathlineto{\pgfqpoint{3.858429in}{3.818045in}}%
\pgfpathlineto{\pgfqpoint{3.851038in}{3.801396in}}%
\pgfpathlineto{\pgfqpoint{3.843643in}{3.784909in}}%
\pgfpathclose%
\pgfusepath{fill}%
\end{pgfscope}%
\begin{pgfscope}%
\pgfpathrectangle{\pgfqpoint{1.254980in}{0.150000in}}{\pgfqpoint{5.490039in}{5.490039in}}%
\pgfusepath{clip}%
\pgfsetbuttcap%
\pgfsetroundjoin%
\definecolor{currentfill}{rgb}{0.151918,0.500685,0.557587}%
\pgfsetfillcolor{currentfill}%
\pgfsetfillopacity{0.700000}%
\pgfsetlinewidth{0.000000pt}%
\definecolor{currentstroke}{rgb}{0.000000,0.000000,0.000000}%
\pgfsetstrokecolor{currentstroke}%
\pgfsetdash{}{0pt}%
\pgfpathmoveto{\pgfqpoint{3.996536in}{3.606244in}}%
\pgfpathlineto{\pgfqpoint{4.009275in}{3.592444in}}%
\pgfpathlineto{\pgfqpoint{4.022015in}{3.578806in}}%
\pgfpathlineto{\pgfqpoint{4.034755in}{3.565331in}}%
\pgfpathlineto{\pgfqpoint{4.047495in}{3.552017in}}%
\pgfpathlineto{\pgfqpoint{4.054855in}{3.567613in}}%
\pgfpathlineto{\pgfqpoint{4.062212in}{3.583355in}}%
\pgfpathlineto{\pgfqpoint{4.069567in}{3.599246in}}%
\pgfpathlineto{\pgfqpoint{4.076918in}{3.615288in}}%
\pgfpathlineto{\pgfqpoint{4.064186in}{3.628830in}}%
\pgfpathlineto{\pgfqpoint{4.051455in}{3.642533in}}%
\pgfpathlineto{\pgfqpoint{4.038724in}{3.656398in}}%
\pgfpathlineto{\pgfqpoint{4.025994in}{3.670426in}}%
\pgfpathlineto{\pgfqpoint{4.018634in}{3.654150in}}%
\pgfpathlineto{\pgfqpoint{4.011271in}{3.638030in}}%
\pgfpathlineto{\pgfqpoint{4.003905in}{3.622062in}}%
\pgfpathlineto{\pgfqpoint{3.996536in}{3.606244in}}%
\pgfpathclose%
\pgfusepath{fill}%
\end{pgfscope}%
\begin{pgfscope}%
\pgfpathrectangle{\pgfqpoint{1.254980in}{0.150000in}}{\pgfqpoint{5.490039in}{5.490039in}}%
\pgfusepath{clip}%
\pgfsetbuttcap%
\pgfsetroundjoin%
\definecolor{currentfill}{rgb}{0.175841,0.441290,0.557685}%
\pgfsetfillcolor{currentfill}%
\pgfsetfillopacity{0.700000}%
\pgfsetlinewidth{0.000000pt}%
\definecolor{currentstroke}{rgb}{0.000000,0.000000,0.000000}%
\pgfsetstrokecolor{currentstroke}%
\pgfsetdash{}{0pt}%
\pgfpathmoveto{\pgfqpoint{4.229780in}{3.465077in}}%
\pgfpathlineto{\pgfqpoint{4.242529in}{3.453559in}}%
\pgfpathlineto{\pgfqpoint{4.255280in}{3.442190in}}%
\pgfpathlineto{\pgfqpoint{4.268033in}{3.430972in}}%
\pgfpathlineto{\pgfqpoint{4.280789in}{3.419902in}}%
\pgfpathlineto{\pgfqpoint{4.288103in}{3.435154in}}%
\pgfpathlineto{\pgfqpoint{4.295415in}{3.450545in}}%
\pgfpathlineto{\pgfqpoint{4.302725in}{3.466079in}}%
\pgfpathlineto{\pgfqpoint{4.289976in}{3.477328in}}%
\pgfpathlineto{\pgfqpoint{4.277229in}{3.488726in}}%
\pgfpathlineto{\pgfqpoint{4.264485in}{3.500273in}}%
\pgfpathlineto{\pgfqpoint{4.251743in}{3.511971in}}%
\pgfpathlineto{\pgfqpoint{4.244424in}{3.496194in}}%
\pgfpathlineto{\pgfqpoint{4.237104in}{3.480564in}}%
\pgfpathlineto{\pgfqpoint{4.229780in}{3.465077in}}%
\pgfpathclose%
\pgfusepath{fill}%
\end{pgfscope}%
\begin{pgfscope}%
\pgfpathrectangle{\pgfqpoint{1.254980in}{0.150000in}}{\pgfqpoint{5.490039in}{5.490039in}}%
\pgfusepath{clip}%
\pgfsetbuttcap%
\pgfsetroundjoin%
\definecolor{currentfill}{rgb}{0.120565,0.596422,0.543611}%
\pgfsetfillcolor{currentfill}%
\pgfsetfillopacity{0.700000}%
\pgfsetlinewidth{0.000000pt}%
\definecolor{currentstroke}{rgb}{0.000000,0.000000,0.000000}%
\pgfsetstrokecolor{currentstroke}%
\pgfsetdash{}{0pt}%
\pgfpathmoveto{\pgfqpoint{3.792646in}{3.850001in}}%
\pgfpathlineto{\pgfqpoint{3.805398in}{3.833461in}}%
\pgfpathlineto{\pgfqpoint{3.818148in}{3.817099in}}%
\pgfpathlineto{\pgfqpoint{3.830896in}{3.800916in}}%
\pgfpathlineto{\pgfqpoint{3.843643in}{3.784909in}}%
\pgfpathlineto{\pgfqpoint{3.851038in}{3.801396in}}%
\pgfpathlineto{\pgfqpoint{3.858429in}{3.818045in}}%
\pgfpathlineto{\pgfqpoint{3.865817in}{3.834859in}}%
\pgfpathlineto{\pgfqpoint{3.873201in}{3.851841in}}%
\pgfpathlineto{\pgfqpoint{3.860463in}{3.868079in}}%
\pgfpathlineto{\pgfqpoint{3.847723in}{3.884493in}}%
\pgfpathlineto{\pgfqpoint{3.834981in}{3.901086in}}%
\pgfpathlineto{\pgfqpoint{3.822238in}{3.917858in}}%
\pgfpathlineto{\pgfqpoint{3.814845in}{3.900639in}}%
\pgfpathlineto{\pgfqpoint{3.807449in}{3.883591in}}%
\pgfpathlineto{\pgfqpoint{3.800049in}{3.866712in}}%
\pgfpathlineto{\pgfqpoint{3.792646in}{3.850001in}}%
\pgfpathclose%
\pgfusepath{fill}%
\end{pgfscope}%
\begin{pgfscope}%
\pgfpathrectangle{\pgfqpoint{1.254980in}{0.150000in}}{\pgfqpoint{5.490039in}{5.490039in}}%
\pgfusepath{clip}%
\pgfsetbuttcap%
\pgfsetroundjoin%
\definecolor{currentfill}{rgb}{0.160665,0.478540,0.558115}%
\pgfsetfillcolor{currentfill}%
\pgfsetfillopacity{0.700000}%
\pgfsetlinewidth{0.000000pt}%
\definecolor{currentstroke}{rgb}{0.000000,0.000000,0.000000}%
\pgfsetstrokecolor{currentstroke}%
\pgfsetdash{}{0pt}%
\pgfpathmoveto{\pgfqpoint{4.047495in}{3.552017in}}%
\pgfpathlineto{\pgfqpoint{4.060236in}{3.538863in}}%
\pgfpathlineto{\pgfqpoint{4.072978in}{3.525868in}}%
\pgfpathlineto{\pgfqpoint{4.085721in}{3.513033in}}%
\pgfpathlineto{\pgfqpoint{4.098464in}{3.500355in}}%
\pgfpathlineto{\pgfqpoint{4.105815in}{3.515731in}}%
\pgfpathlineto{\pgfqpoint{4.113164in}{3.531248in}}%
\pgfpathlineto{\pgfqpoint{4.120509in}{3.546911in}}%
\pgfpathlineto{\pgfqpoint{4.127853in}{3.562720in}}%
\pgfpathlineto{\pgfqpoint{4.115118in}{3.575624in}}%
\pgfpathlineto{\pgfqpoint{4.102384in}{3.588686in}}%
\pgfpathlineto{\pgfqpoint{4.089651in}{3.601908in}}%
\pgfpathlineto{\pgfqpoint{4.076918in}{3.615288in}}%
\pgfpathlineto{\pgfqpoint{4.069567in}{3.599246in}}%
\pgfpathlineto{\pgfqpoint{4.062212in}{3.583355in}}%
\pgfpathlineto{\pgfqpoint{4.054855in}{3.567613in}}%
\pgfpathlineto{\pgfqpoint{4.047495in}{3.552017in}}%
\pgfpathclose%
\pgfusepath{fill}%
\end{pgfscope}%
\begin{pgfscope}%
\pgfpathrectangle{\pgfqpoint{1.254980in}{0.150000in}}{\pgfqpoint{5.490039in}{5.490039in}}%
\pgfusepath{clip}%
\pgfsetbuttcap%
\pgfsetroundjoin%
\definecolor{currentfill}{rgb}{0.180653,0.701402,0.488189}%
\pgfsetfillcolor{currentfill}%
\pgfsetfillopacity{0.700000}%
\pgfsetlinewidth{0.000000pt}%
\definecolor{currentstroke}{rgb}{0.000000,0.000000,0.000000}%
\pgfsetstrokecolor{currentstroke}%
\pgfsetdash{}{0pt}%
\pgfpathmoveto{\pgfqpoint{3.669127in}{4.133528in}}%
\pgfpathlineto{\pgfqpoint{3.681902in}{4.114513in}}%
\pgfpathlineto{\pgfqpoint{3.694675in}{4.095692in}}%
\pgfpathlineto{\pgfqpoint{3.707443in}{4.077063in}}%
\pgfpathlineto{\pgfqpoint{3.720209in}{4.058625in}}%
\pgfpathlineto{\pgfqpoint{3.727615in}{4.076507in}}%
\pgfpathlineto{\pgfqpoint{3.735017in}{4.094574in}}%
\pgfpathlineto{\pgfqpoint{3.742415in}{4.112829in}}%
\pgfpathlineto{\pgfqpoint{3.729655in}{4.131453in}}%
\pgfpathlineto{\pgfqpoint{3.716892in}{4.150270in}}%
\pgfpathlineto{\pgfqpoint{3.704126in}{4.169278in}}%
\pgfpathlineto{\pgfqpoint{3.691356in}{4.188481in}}%
\pgfpathlineto{\pgfqpoint{3.683950in}{4.169972in}}%
\pgfpathlineto{\pgfqpoint{3.676540in}{4.151655in}}%
\pgfpathlineto{\pgfqpoint{3.669127in}{4.133528in}}%
\pgfpathclose%
\pgfusepath{fill}%
\end{pgfscope}%
\begin{pgfscope}%
\pgfpathrectangle{\pgfqpoint{1.254980in}{0.150000in}}{\pgfqpoint{5.490039in}{5.490039in}}%
\pgfusepath{clip}%
\pgfsetbuttcap%
\pgfsetroundjoin%
\definecolor{currentfill}{rgb}{0.120081,0.622161,0.534946}%
\pgfsetfillcolor{currentfill}%
\pgfsetfillopacity{0.700000}%
\pgfsetlinewidth{0.000000pt}%
\definecolor{currentstroke}{rgb}{0.000000,0.000000,0.000000}%
\pgfsetstrokecolor{currentstroke}%
\pgfsetdash{}{0pt}%
\pgfpathmoveto{\pgfqpoint{3.741619in}{3.917975in}}%
\pgfpathlineto{\pgfqpoint{3.754379in}{3.900707in}}%
\pgfpathlineto{\pgfqpoint{3.767137in}{3.883623in}}%
\pgfpathlineto{\pgfqpoint{3.779893in}{3.866722in}}%
\pgfpathlineto{\pgfqpoint{3.792646in}{3.850001in}}%
\pgfpathlineto{\pgfqpoint{3.800049in}{3.866712in}}%
\pgfpathlineto{\pgfqpoint{3.807449in}{3.883591in}}%
\pgfpathlineto{\pgfqpoint{3.814845in}{3.900639in}}%
\pgfpathlineto{\pgfqpoint{3.822238in}{3.917858in}}%
\pgfpathlineto{\pgfqpoint{3.809492in}{3.934811in}}%
\pgfpathlineto{\pgfqpoint{3.796745in}{3.951945in}}%
\pgfpathlineto{\pgfqpoint{3.783995in}{3.969261in}}%
\pgfpathlineto{\pgfqpoint{3.771243in}{3.986762in}}%
\pgfpathlineto{\pgfqpoint{3.763842in}{3.969303in}}%
\pgfpathlineto{\pgfqpoint{3.756438in}{3.952021in}}%
\pgfpathlineto{\pgfqpoint{3.749030in}{3.934912in}}%
\pgfpathlineto{\pgfqpoint{3.741619in}{3.917975in}}%
\pgfpathclose%
\pgfusepath{fill}%
\end{pgfscope}%
\begin{pgfscope}%
\pgfpathrectangle{\pgfqpoint{1.254980in}{0.150000in}}{\pgfqpoint{5.490039in}{5.490039in}}%
\pgfusepath{clip}%
\pgfsetbuttcap%
\pgfsetroundjoin%
\definecolor{currentfill}{rgb}{0.168126,0.459988,0.558082}%
\pgfsetfillcolor{currentfill}%
\pgfsetfillopacity{0.700000}%
\pgfsetlinewidth{0.000000pt}%
\definecolor{currentstroke}{rgb}{0.000000,0.000000,0.000000}%
\pgfsetstrokecolor{currentstroke}%
\pgfsetdash{}{0pt}%
\pgfpathmoveto{\pgfqpoint{4.098464in}{3.500355in}}%
\pgfpathlineto{\pgfqpoint{4.111209in}{3.487834in}}%
\pgfpathlineto{\pgfqpoint{4.123955in}{3.475470in}}%
\pgfpathlineto{\pgfqpoint{4.136702in}{3.463260in}}%
\pgfpathlineto{\pgfqpoint{4.149451in}{3.451206in}}%
\pgfpathlineto{\pgfqpoint{4.156793in}{3.466361in}}%
\pgfpathlineto{\pgfqpoint{4.164133in}{3.481655in}}%
\pgfpathlineto{\pgfqpoint{4.171470in}{3.497089in}}%
\pgfpathlineto{\pgfqpoint{4.178804in}{3.512667in}}%
\pgfpathlineto{\pgfqpoint{4.166064in}{3.524947in}}%
\pgfpathlineto{\pgfqpoint{4.153326in}{3.537382in}}%
\pgfpathlineto{\pgfqpoint{4.140589in}{3.549973in}}%
\pgfpathlineto{\pgfqpoint{4.127853in}{3.562720in}}%
\pgfpathlineto{\pgfqpoint{4.120509in}{3.546911in}}%
\pgfpathlineto{\pgfqpoint{4.113164in}{3.531248in}}%
\pgfpathlineto{\pgfqpoint{4.105815in}{3.515731in}}%
\pgfpathlineto{\pgfqpoint{4.098464in}{3.500355in}}%
\pgfpathclose%
\pgfusepath{fill}%
\end{pgfscope}%
\begin{pgfscope}%
\pgfpathrectangle{\pgfqpoint{1.254980in}{0.150000in}}{\pgfqpoint{5.490039in}{5.490039in}}%
\pgfusepath{clip}%
\pgfsetbuttcap%
\pgfsetroundjoin%
\definecolor{currentfill}{rgb}{0.182256,0.426184,0.557120}%
\pgfsetfillcolor{currentfill}%
\pgfsetfillopacity{0.700000}%
\pgfsetlinewidth{0.000000pt}%
\definecolor{currentstroke}{rgb}{0.000000,0.000000,0.000000}%
\pgfsetstrokecolor{currentstroke}%
\pgfsetdash{}{0pt}%
\pgfpathmoveto{\pgfqpoint{4.280789in}{3.419902in}}%
\pgfpathlineto{\pgfqpoint{4.293546in}{3.408980in}}%
\pgfpathlineto{\pgfqpoint{4.306307in}{3.398205in}}%
\pgfpathlineto{\pgfqpoint{4.319070in}{3.387577in}}%
\pgfpathlineto{\pgfqpoint{4.331835in}{3.377095in}}%
\pgfpathlineto{\pgfqpoint{4.339141in}{3.392113in}}%
\pgfpathlineto{\pgfqpoint{4.346444in}{3.407266in}}%
\pgfpathlineto{\pgfqpoint{4.353744in}{3.422557in}}%
\pgfpathlineto{\pgfqpoint{4.340985in}{3.433218in}}%
\pgfpathlineto{\pgfqpoint{4.328229in}{3.444024in}}%
\pgfpathlineto{\pgfqpoint{4.315476in}{3.454978in}}%
\pgfpathlineto{\pgfqpoint{4.302725in}{3.466079in}}%
\pgfpathlineto{\pgfqpoint{4.295415in}{3.450545in}}%
\pgfpathlineto{\pgfqpoint{4.288103in}{3.435154in}}%
\pgfpathlineto{\pgfqpoint{4.280789in}{3.419902in}}%
\pgfpathclose%
\pgfusepath{fill}%
\end{pgfscope}%
\begin{pgfscope}%
\pgfpathrectangle{\pgfqpoint{1.254980in}{0.150000in}}{\pgfqpoint{5.490039in}{5.490039in}}%
\pgfusepath{clip}%
\pgfsetbuttcap%
\pgfsetroundjoin%
\definecolor{currentfill}{rgb}{0.130067,0.651384,0.521608}%
\pgfsetfillcolor{currentfill}%
\pgfsetfillopacity{0.700000}%
\pgfsetlinewidth{0.000000pt}%
\definecolor{currentstroke}{rgb}{0.000000,0.000000,0.000000}%
\pgfsetstrokecolor{currentstroke}%
\pgfsetdash{}{0pt}%
\pgfpathmoveto{\pgfqpoint{3.690553in}{3.988904in}}%
\pgfpathlineto{\pgfqpoint{3.703323in}{3.970891in}}%
\pgfpathlineto{\pgfqpoint{3.716091in}{3.953066in}}%
\pgfpathlineto{\pgfqpoint{3.728856in}{3.935427in}}%
\pgfpathlineto{\pgfqpoint{3.741619in}{3.917975in}}%
\pgfpathlineto{\pgfqpoint{3.749030in}{3.934912in}}%
\pgfpathlineto{\pgfqpoint{3.756438in}{3.952021in}}%
\pgfpathlineto{\pgfqpoint{3.763842in}{3.969303in}}%
\pgfpathlineto{\pgfqpoint{3.771243in}{3.986762in}}%
\pgfpathlineto{\pgfqpoint{3.758489in}{4.004447in}}%
\pgfpathlineto{\pgfqpoint{3.745732in}{4.022319in}}%
\pgfpathlineto{\pgfqpoint{3.732972in}{4.040378in}}%
\pgfpathlineto{\pgfqpoint{3.720209in}{4.058625in}}%
\pgfpathlineto{\pgfqpoint{3.712800in}{4.040927in}}%
\pgfpathlineto{\pgfqpoint{3.705388in}{4.023409in}}%
\pgfpathlineto{\pgfqpoint{3.697972in}{4.006069in}}%
\pgfpathlineto{\pgfqpoint{3.690553in}{3.988904in}}%
\pgfpathclose%
\pgfusepath{fill}%
\end{pgfscope}%
\begin{pgfscope}%
\pgfpathrectangle{\pgfqpoint{1.254980in}{0.150000in}}{\pgfqpoint{5.490039in}{5.490039in}}%
\pgfusepath{clip}%
\pgfsetbuttcap%
\pgfsetroundjoin%
\definecolor{currentfill}{rgb}{0.175841,0.441290,0.557685}%
\pgfsetfillcolor{currentfill}%
\pgfsetfillopacity{0.700000}%
\pgfsetlinewidth{0.000000pt}%
\definecolor{currentstroke}{rgb}{0.000000,0.000000,0.000000}%
\pgfsetstrokecolor{currentstroke}%
\pgfsetdash{}{0pt}%
\pgfpathmoveto{\pgfqpoint{4.149451in}{3.451206in}}%
\pgfpathlineto{\pgfqpoint{4.162201in}{3.439305in}}%
\pgfpathlineto{\pgfqpoint{4.174954in}{3.427557in}}%
\pgfpathlineto{\pgfqpoint{4.187707in}{3.415962in}}%
\pgfpathlineto{\pgfqpoint{4.200463in}{3.404518in}}%
\pgfpathlineto{\pgfqpoint{4.207796in}{3.419454in}}%
\pgfpathlineto{\pgfqpoint{4.215127in}{3.434524in}}%
\pgfpathlineto{\pgfqpoint{4.222455in}{3.449731in}}%
\pgfpathlineto{\pgfqpoint{4.229780in}{3.465077in}}%
\pgfpathlineto{\pgfqpoint{4.217034in}{3.476746in}}%
\pgfpathlineto{\pgfqpoint{4.204289in}{3.488567in}}%
\pgfpathlineto{\pgfqpoint{4.191546in}{3.500540in}}%
\pgfpathlineto{\pgfqpoint{4.178804in}{3.512667in}}%
\pgfpathlineto{\pgfqpoint{4.171470in}{3.497089in}}%
\pgfpathlineto{\pgfqpoint{4.164133in}{3.481655in}}%
\pgfpathlineto{\pgfqpoint{4.156793in}{3.466361in}}%
\pgfpathlineto{\pgfqpoint{4.149451in}{3.451206in}}%
\pgfpathclose%
\pgfusepath{fill}%
\end{pgfscope}%
\begin{pgfscope}%
\pgfpathrectangle{\pgfqpoint{1.254980in}{0.150000in}}{\pgfqpoint{5.490039in}{5.490039in}}%
\pgfusepath{clip}%
\pgfsetbuttcap%
\pgfsetroundjoin%
\definecolor{currentfill}{rgb}{0.226397,0.728888,0.462789}%
\pgfsetfillcolor{currentfill}%
\pgfsetfillopacity{0.700000}%
\pgfsetlinewidth{0.000000pt}%
\definecolor{currentstroke}{rgb}{0.000000,0.000000,0.000000}%
\pgfsetstrokecolor{currentstroke}%
\pgfsetdash{}{0pt}%
\pgfpathmoveto{\pgfqpoint{3.617986in}{4.211550in}}%
\pgfpathlineto{\pgfqpoint{3.630777in}{4.191747in}}%
\pgfpathlineto{\pgfqpoint{3.643564in}{4.172143in}}%
\pgfpathlineto{\pgfqpoint{3.656347in}{4.152737in}}%
\pgfpathlineto{\pgfqpoint{3.669127in}{4.133528in}}%
\pgfpathlineto{\pgfqpoint{3.676540in}{4.151655in}}%
\pgfpathlineto{\pgfqpoint{3.683950in}{4.169972in}}%
\pgfpathlineto{\pgfqpoint{3.691356in}{4.188481in}}%
\pgfpathlineto{\pgfqpoint{3.678582in}{4.207879in}}%
\pgfpathlineto{\pgfqpoint{3.665805in}{4.227474in}}%
\pgfpathlineto{\pgfqpoint{3.653023in}{4.247267in}}%
\pgfpathlineto{\pgfqpoint{3.640238in}{4.267258in}}%
\pgfpathlineto{\pgfqpoint{3.632825in}{4.248493in}}%
\pgfpathlineto{\pgfqpoint{3.625407in}{4.229924in}}%
\pgfpathlineto{\pgfqpoint{3.617986in}{4.211550in}}%
\pgfpathclose%
\pgfusepath{fill}%
\end{pgfscope}%
\begin{pgfscope}%
\pgfpathrectangle{\pgfqpoint{1.254980in}{0.150000in}}{\pgfqpoint{5.490039in}{5.490039in}}%
\pgfusepath{clip}%
\pgfsetbuttcap%
\pgfsetroundjoin%
\definecolor{currentfill}{rgb}{0.188923,0.410910,0.556326}%
\pgfsetfillcolor{currentfill}%
\pgfsetfillopacity{0.700000}%
\pgfsetlinewidth{0.000000pt}%
\definecolor{currentstroke}{rgb}{0.000000,0.000000,0.000000}%
\pgfsetstrokecolor{currentstroke}%
\pgfsetdash{}{0pt}%
\pgfpathmoveto{\pgfqpoint{4.331835in}{3.377095in}}%
\pgfpathlineto{\pgfqpoint{4.344604in}{3.366759in}}%
\pgfpathlineto{\pgfqpoint{4.357376in}{3.356567in}}%
\pgfpathlineto{\pgfqpoint{4.370150in}{3.346518in}}%
\pgfpathlineto{\pgfqpoint{4.382928in}{3.336614in}}%
\pgfpathlineto{\pgfqpoint{4.390224in}{3.351397in}}%
\pgfpathlineto{\pgfqpoint{4.397518in}{3.366312in}}%
\pgfpathlineto{\pgfqpoint{4.404809in}{3.381361in}}%
\pgfpathlineto{\pgfqpoint{4.392039in}{3.391445in}}%
\pgfpathlineto{\pgfqpoint{4.379271in}{3.401671in}}%
\pgfpathlineto{\pgfqpoint{4.366506in}{3.412042in}}%
\pgfpathlineto{\pgfqpoint{4.353744in}{3.422557in}}%
\pgfpathlineto{\pgfqpoint{4.346444in}{3.407266in}}%
\pgfpathlineto{\pgfqpoint{4.339141in}{3.392113in}}%
\pgfpathlineto{\pgfqpoint{4.331835in}{3.377095in}}%
\pgfpathclose%
\pgfusepath{fill}%
\end{pgfscope}%
\begin{pgfscope}%
\pgfpathrectangle{\pgfqpoint{1.254980in}{0.150000in}}{\pgfqpoint{5.490039in}{5.490039in}}%
\pgfusepath{clip}%
\pgfsetbuttcap%
\pgfsetroundjoin%
\definecolor{currentfill}{rgb}{0.143343,0.522773,0.556295}%
\pgfsetfillcolor{currentfill}%
\pgfsetfillopacity{0.700000}%
\pgfsetlinewidth{0.000000pt}%
\definecolor{currentstroke}{rgb}{0.000000,0.000000,0.000000}%
\pgfsetstrokecolor{currentstroke}%
\pgfsetdash{}{0pt}%
\pgfpathmoveto{\pgfqpoint{3.865042in}{3.659119in}}%
\pgfpathlineto{\pgfqpoint{3.877793in}{3.644194in}}%
\pgfpathlineto{\pgfqpoint{3.890542in}{3.629439in}}%
\pgfpathlineto{\pgfqpoint{3.903291in}{3.614853in}}%
\pgfpathlineto{\pgfqpoint{3.916039in}{3.600436in}}%
\pgfpathlineto{\pgfqpoint{3.923429in}{3.615880in}}%
\pgfpathlineto{\pgfqpoint{3.930816in}{3.631470in}}%
\pgfpathlineto{\pgfqpoint{3.938200in}{3.647208in}}%
\pgfpathlineto{\pgfqpoint{3.945580in}{3.663095in}}%
\pgfpathlineto{\pgfqpoint{3.932840in}{3.677725in}}%
\pgfpathlineto{\pgfqpoint{3.920100in}{3.692523in}}%
\pgfpathlineto{\pgfqpoint{3.907360in}{3.707491in}}%
\pgfpathlineto{\pgfqpoint{3.894618in}{3.722629in}}%
\pgfpathlineto{\pgfqpoint{3.887229in}{3.706523in}}%
\pgfpathlineto{\pgfqpoint{3.879837in}{3.690571in}}%
\pgfpathlineto{\pgfqpoint{3.872441in}{3.674770in}}%
\pgfpathlineto{\pgfqpoint{3.865042in}{3.659119in}}%
\pgfpathclose%
\pgfusepath{fill}%
\end{pgfscope}%
\begin{pgfscope}%
\pgfpathrectangle{\pgfqpoint{1.254980in}{0.150000in}}{\pgfqpoint{5.490039in}{5.490039in}}%
\pgfusepath{clip}%
\pgfsetbuttcap%
\pgfsetroundjoin%
\definecolor{currentfill}{rgb}{0.151918,0.500685,0.557587}%
\pgfsetfillcolor{currentfill}%
\pgfsetfillopacity{0.700000}%
\pgfsetlinewidth{0.000000pt}%
\definecolor{currentstroke}{rgb}{0.000000,0.000000,0.000000}%
\pgfsetstrokecolor{currentstroke}%
\pgfsetdash{}{0pt}%
\pgfpathmoveto{\pgfqpoint{3.916039in}{3.600436in}}%
\pgfpathlineto{\pgfqpoint{3.928787in}{3.586186in}}%
\pgfpathlineto{\pgfqpoint{3.941535in}{3.572103in}}%
\pgfpathlineto{\pgfqpoint{3.954283in}{3.558185in}}%
\pgfpathlineto{\pgfqpoint{3.967031in}{3.544432in}}%
\pgfpathlineto{\pgfqpoint{3.974412in}{3.559671in}}%
\pgfpathlineto{\pgfqpoint{3.981790in}{3.575051in}}%
\pgfpathlineto{\pgfqpoint{3.989165in}{3.590575in}}%
\pgfpathlineto{\pgfqpoint{3.996536in}{3.606244in}}%
\pgfpathlineto{\pgfqpoint{3.983797in}{3.620209in}}%
\pgfpathlineto{\pgfqpoint{3.971058in}{3.634338in}}%
\pgfpathlineto{\pgfqpoint{3.958319in}{3.648634in}}%
\pgfpathlineto{\pgfqpoint{3.945580in}{3.663095in}}%
\pgfpathlineto{\pgfqpoint{3.938200in}{3.647208in}}%
\pgfpathlineto{\pgfqpoint{3.930816in}{3.631470in}}%
\pgfpathlineto{\pgfqpoint{3.923429in}{3.615880in}}%
\pgfpathlineto{\pgfqpoint{3.916039in}{3.600436in}}%
\pgfpathclose%
\pgfusepath{fill}%
\end{pgfscope}%
\begin{pgfscope}%
\pgfpathrectangle{\pgfqpoint{1.254980in}{0.150000in}}{\pgfqpoint{5.490039in}{5.490039in}}%
\pgfusepath{clip}%
\pgfsetbuttcap%
\pgfsetroundjoin%
\definecolor{currentfill}{rgb}{0.133743,0.548535,0.553541}%
\pgfsetfillcolor{currentfill}%
\pgfsetfillopacity{0.700000}%
\pgfsetlinewidth{0.000000pt}%
\definecolor{currentstroke}{rgb}{0.000000,0.000000,0.000000}%
\pgfsetstrokecolor{currentstroke}%
\pgfsetdash{}{0pt}%
\pgfpathmoveto{\pgfqpoint{3.814032in}{3.720545in}}%
\pgfpathlineto{\pgfqpoint{3.826787in}{3.704928in}}%
\pgfpathlineto{\pgfqpoint{3.839540in}{3.689485in}}%
\pgfpathlineto{\pgfqpoint{3.852291in}{3.674216in}}%
\pgfpathlineto{\pgfqpoint{3.865042in}{3.659119in}}%
\pgfpathlineto{\pgfqpoint{3.872441in}{3.674770in}}%
\pgfpathlineto{\pgfqpoint{3.879837in}{3.690571in}}%
\pgfpathlineto{\pgfqpoint{3.887229in}{3.706523in}}%
\pgfpathlineto{\pgfqpoint{3.894618in}{3.722629in}}%
\pgfpathlineto{\pgfqpoint{3.881876in}{3.737939in}}%
\pgfpathlineto{\pgfqpoint{3.869133in}{3.753422in}}%
\pgfpathlineto{\pgfqpoint{3.856389in}{3.769078in}}%
\pgfpathlineto{\pgfqpoint{3.843643in}{3.784909in}}%
\pgfpathlineto{\pgfqpoint{3.836246in}{3.768583in}}%
\pgfpathlineto{\pgfqpoint{3.828845in}{3.752415in}}%
\pgfpathlineto{\pgfqpoint{3.821440in}{3.736403in}}%
\pgfpathlineto{\pgfqpoint{3.814032in}{3.720545in}}%
\pgfpathclose%
\pgfusepath{fill}%
\end{pgfscope}%
\begin{pgfscope}%
\pgfpathrectangle{\pgfqpoint{1.254980in}{0.150000in}}{\pgfqpoint{5.490039in}{5.490039in}}%
\pgfusepath{clip}%
\pgfsetbuttcap%
\pgfsetroundjoin%
\definecolor{currentfill}{rgb}{0.182256,0.426184,0.557120}%
\pgfsetfillcolor{currentfill}%
\pgfsetfillopacity{0.700000}%
\pgfsetlinewidth{0.000000pt}%
\definecolor{currentstroke}{rgb}{0.000000,0.000000,0.000000}%
\pgfsetstrokecolor{currentstroke}%
\pgfsetdash{}{0pt}%
\pgfpathmoveto{\pgfqpoint{4.200463in}{3.404518in}}%
\pgfpathlineto{\pgfqpoint{4.213221in}{3.393224in}}%
\pgfpathlineto{\pgfqpoint{4.225981in}{3.382081in}}%
\pgfpathlineto{\pgfqpoint{4.238743in}{3.371087in}}%
\pgfpathlineto{\pgfqpoint{4.251507in}{3.360242in}}%
\pgfpathlineto{\pgfqpoint{4.258831in}{3.374960in}}%
\pgfpathlineto{\pgfqpoint{4.266153in}{3.389807in}}%
\pgfpathlineto{\pgfqpoint{4.273472in}{3.404787in}}%
\pgfpathlineto{\pgfqpoint{4.280789in}{3.419902in}}%
\pgfpathlineto{\pgfqpoint{4.268033in}{3.430972in}}%
\pgfpathlineto{\pgfqpoint{4.255280in}{3.442190in}}%
\pgfpathlineto{\pgfqpoint{4.242529in}{3.453559in}}%
\pgfpathlineto{\pgfqpoint{4.229780in}{3.465077in}}%
\pgfpathlineto{\pgfqpoint{4.222455in}{3.449731in}}%
\pgfpathlineto{\pgfqpoint{4.215127in}{3.434524in}}%
\pgfpathlineto{\pgfqpoint{4.207796in}{3.419454in}}%
\pgfpathlineto{\pgfqpoint{4.200463in}{3.404518in}}%
\pgfpathclose%
\pgfusepath{fill}%
\end{pgfscope}%
\begin{pgfscope}%
\pgfpathrectangle{\pgfqpoint{1.254980in}{0.150000in}}{\pgfqpoint{5.490039in}{5.490039in}}%
\pgfusepath{clip}%
\pgfsetbuttcap%
\pgfsetroundjoin%
\definecolor{currentfill}{rgb}{0.153894,0.680203,0.504172}%
\pgfsetfillcolor{currentfill}%
\pgfsetfillopacity{0.700000}%
\pgfsetlinewidth{0.000000pt}%
\definecolor{currentstroke}{rgb}{0.000000,0.000000,0.000000}%
\pgfsetstrokecolor{currentstroke}%
\pgfsetdash{}{0pt}%
\pgfpathmoveto{\pgfqpoint{3.639438in}{4.062867in}}%
\pgfpathlineto{\pgfqpoint{3.652221in}{4.044088in}}%
\pgfpathlineto{\pgfqpoint{3.665002in}{4.025502in}}%
\pgfpathlineto{\pgfqpoint{3.677779in}{4.007108in}}%
\pgfpathlineto{\pgfqpoint{3.690553in}{3.988904in}}%
\pgfpathlineto{\pgfqpoint{3.697972in}{4.006069in}}%
\pgfpathlineto{\pgfqpoint{3.705388in}{4.023409in}}%
\pgfpathlineto{\pgfqpoint{3.712800in}{4.040927in}}%
\pgfpathlineto{\pgfqpoint{3.720209in}{4.058625in}}%
\pgfpathlineto{\pgfqpoint{3.707443in}{4.077063in}}%
\pgfpathlineto{\pgfqpoint{3.694675in}{4.095692in}}%
\pgfpathlineto{\pgfqpoint{3.681902in}{4.114513in}}%
\pgfpathlineto{\pgfqpoint{3.669127in}{4.133528in}}%
\pgfpathlineto{\pgfqpoint{3.661710in}{4.115588in}}%
\pgfpathlineto{\pgfqpoint{3.654290in}{4.097833in}}%
\pgfpathlineto{\pgfqpoint{3.646866in}{4.080260in}}%
\pgfpathlineto{\pgfqpoint{3.639438in}{4.062867in}}%
\pgfpathclose%
\pgfusepath{fill}%
\end{pgfscope}%
\begin{pgfscope}%
\pgfpathrectangle{\pgfqpoint{1.254980in}{0.150000in}}{\pgfqpoint{5.490039in}{5.490039in}}%
\pgfusepath{clip}%
\pgfsetbuttcap%
\pgfsetroundjoin%
\definecolor{currentfill}{rgb}{0.159194,0.482237,0.558073}%
\pgfsetfillcolor{currentfill}%
\pgfsetfillopacity{0.700000}%
\pgfsetlinewidth{0.000000pt}%
\definecolor{currentstroke}{rgb}{0.000000,0.000000,0.000000}%
\pgfsetstrokecolor{currentstroke}%
\pgfsetdash{}{0pt}%
\pgfpathmoveto{\pgfqpoint{3.967031in}{3.544432in}}%
\pgfpathlineto{\pgfqpoint{3.979779in}{3.530843in}}%
\pgfpathlineto{\pgfqpoint{3.992528in}{3.517416in}}%
\pgfpathlineto{\pgfqpoint{4.005276in}{3.504152in}}%
\pgfpathlineto{\pgfqpoint{4.018026in}{3.491048in}}%
\pgfpathlineto{\pgfqpoint{4.025398in}{3.506082in}}%
\pgfpathlineto{\pgfqpoint{4.032766in}{3.521254in}}%
\pgfpathlineto{\pgfqpoint{4.040132in}{3.536564in}}%
\pgfpathlineto{\pgfqpoint{4.047495in}{3.552017in}}%
\pgfpathlineto{\pgfqpoint{4.034755in}{3.565331in}}%
\pgfpathlineto{\pgfqpoint{4.022015in}{3.578806in}}%
\pgfpathlineto{\pgfqpoint{4.009275in}{3.592444in}}%
\pgfpathlineto{\pgfqpoint{3.996536in}{3.606244in}}%
\pgfpathlineto{\pgfqpoint{3.989165in}{3.590575in}}%
\pgfpathlineto{\pgfqpoint{3.981790in}{3.575051in}}%
\pgfpathlineto{\pgfqpoint{3.974412in}{3.559671in}}%
\pgfpathlineto{\pgfqpoint{3.967031in}{3.544432in}}%
\pgfpathclose%
\pgfusepath{fill}%
\end{pgfscope}%
\begin{pgfscope}%
\pgfpathrectangle{\pgfqpoint{1.254980in}{0.150000in}}{\pgfqpoint{5.490039in}{5.490039in}}%
\pgfusepath{clip}%
\pgfsetbuttcap%
\pgfsetroundjoin%
\definecolor{currentfill}{rgb}{0.125394,0.574318,0.549086}%
\pgfsetfillcolor{currentfill}%
\pgfsetfillopacity{0.700000}%
\pgfsetlinewidth{0.000000pt}%
\definecolor{currentstroke}{rgb}{0.000000,0.000000,0.000000}%
\pgfsetstrokecolor{currentstroke}%
\pgfsetdash{}{0pt}%
\pgfpathmoveto{\pgfqpoint{3.763000in}{3.784780in}}%
\pgfpathlineto{\pgfqpoint{3.775761in}{3.768454in}}%
\pgfpathlineto{\pgfqpoint{3.788520in}{3.752307in}}%
\pgfpathlineto{\pgfqpoint{3.801277in}{3.736338in}}%
\pgfpathlineto{\pgfqpoint{3.814032in}{3.720545in}}%
\pgfpathlineto{\pgfqpoint{3.821440in}{3.736403in}}%
\pgfpathlineto{\pgfqpoint{3.828845in}{3.752415in}}%
\pgfpathlineto{\pgfqpoint{3.836246in}{3.768583in}}%
\pgfpathlineto{\pgfqpoint{3.843643in}{3.784909in}}%
\pgfpathlineto{\pgfqpoint{3.830896in}{3.800916in}}%
\pgfpathlineto{\pgfqpoint{3.818148in}{3.817099in}}%
\pgfpathlineto{\pgfqpoint{3.805398in}{3.833461in}}%
\pgfpathlineto{\pgfqpoint{3.792646in}{3.850001in}}%
\pgfpathlineto{\pgfqpoint{3.785240in}{3.833454in}}%
\pgfpathlineto{\pgfqpoint{3.777830in}{3.817070in}}%
\pgfpathlineto{\pgfqpoint{3.770417in}{3.800846in}}%
\pgfpathlineto{\pgfqpoint{3.763000in}{3.784780in}}%
\pgfpathclose%
\pgfusepath{fill}%
\end{pgfscope}%
\begin{pgfscope}%
\pgfpathrectangle{\pgfqpoint{1.254980in}{0.150000in}}{\pgfqpoint{5.490039in}{5.490039in}}%
\pgfusepath{clip}%
\pgfsetbuttcap%
\pgfsetroundjoin%
\definecolor{currentfill}{rgb}{0.168126,0.459988,0.558082}%
\pgfsetfillcolor{currentfill}%
\pgfsetfillopacity{0.700000}%
\pgfsetlinewidth{0.000000pt}%
\definecolor{currentstroke}{rgb}{0.000000,0.000000,0.000000}%
\pgfsetstrokecolor{currentstroke}%
\pgfsetdash{}{0pt}%
\pgfpathmoveto{\pgfqpoint{4.018026in}{3.491048in}}%
\pgfpathlineto{\pgfqpoint{4.030776in}{3.478105in}}%
\pgfpathlineto{\pgfqpoint{4.043527in}{3.465321in}}%
\pgfpathlineto{\pgfqpoint{4.056278in}{3.452696in}}%
\pgfpathlineto{\pgfqpoint{4.069031in}{3.440228in}}%
\pgfpathlineto{\pgfqpoint{4.076394in}{3.455058in}}%
\pgfpathlineto{\pgfqpoint{4.083753in}{3.470021in}}%
\pgfpathlineto{\pgfqpoint{4.091110in}{3.485119in}}%
\pgfpathlineto{\pgfqpoint{4.098464in}{3.500355in}}%
\pgfpathlineto{\pgfqpoint{4.085721in}{3.513033in}}%
\pgfpathlineto{\pgfqpoint{4.072978in}{3.525868in}}%
\pgfpathlineto{\pgfqpoint{4.060236in}{3.538863in}}%
\pgfpathlineto{\pgfqpoint{4.047495in}{3.552017in}}%
\pgfpathlineto{\pgfqpoint{4.040132in}{3.536564in}}%
\pgfpathlineto{\pgfqpoint{4.032766in}{3.521254in}}%
\pgfpathlineto{\pgfqpoint{4.025398in}{3.506082in}}%
\pgfpathlineto{\pgfqpoint{4.018026in}{3.491048in}}%
\pgfpathclose%
\pgfusepath{fill}%
\end{pgfscope}%
\begin{pgfscope}%
\pgfpathrectangle{\pgfqpoint{1.254980in}{0.150000in}}{\pgfqpoint{5.490039in}{5.490039in}}%
\pgfusepath{clip}%
\pgfsetbuttcap%
\pgfsetroundjoin%
\definecolor{currentfill}{rgb}{0.120092,0.600104,0.542530}%
\pgfsetfillcolor{currentfill}%
\pgfsetfillopacity{0.700000}%
\pgfsetlinewidth{0.000000pt}%
\definecolor{currentstroke}{rgb}{0.000000,0.000000,0.000000}%
\pgfsetstrokecolor{currentstroke}%
\pgfsetdash{}{0pt}%
\pgfpathmoveto{\pgfqpoint{3.711938in}{3.851892in}}%
\pgfpathlineto{\pgfqpoint{3.724707in}{3.834840in}}%
\pgfpathlineto{\pgfqpoint{3.737474in}{3.817971in}}%
\pgfpathlineto{\pgfqpoint{3.750238in}{3.801285in}}%
\pgfpathlineto{\pgfqpoint{3.763000in}{3.784780in}}%
\pgfpathlineto{\pgfqpoint{3.770417in}{3.800846in}}%
\pgfpathlineto{\pgfqpoint{3.777830in}{3.817070in}}%
\pgfpathlineto{\pgfqpoint{3.785240in}{3.833454in}}%
\pgfpathlineto{\pgfqpoint{3.792646in}{3.850001in}}%
\pgfpathlineto{\pgfqpoint{3.779893in}{3.866722in}}%
\pgfpathlineto{\pgfqpoint{3.767137in}{3.883623in}}%
\pgfpathlineto{\pgfqpoint{3.754379in}{3.900707in}}%
\pgfpathlineto{\pgfqpoint{3.741619in}{3.917975in}}%
\pgfpathlineto{\pgfqpoint{3.734204in}{3.901207in}}%
\pgfpathlineto{\pgfqpoint{3.726786in}{3.884605in}}%
\pgfpathlineto{\pgfqpoint{3.719364in}{3.868167in}}%
\pgfpathlineto{\pgfqpoint{3.711938in}{3.851892in}}%
\pgfpathclose%
\pgfusepath{fill}%
\end{pgfscope}%
\begin{pgfscope}%
\pgfpathrectangle{\pgfqpoint{1.254980in}{0.150000in}}{\pgfqpoint{5.490039in}{5.490039in}}%
\pgfusepath{clip}%
\pgfsetbuttcap%
\pgfsetroundjoin%
\definecolor{currentfill}{rgb}{0.296479,0.761561,0.424223}%
\pgfsetfillcolor{currentfill}%
\pgfsetfillopacity{0.700000}%
\pgfsetlinewidth{0.000000pt}%
\definecolor{currentstroke}{rgb}{0.000000,0.000000,0.000000}%
\pgfsetstrokecolor{currentstroke}%
\pgfsetdash{}{0pt}%
\pgfpathmoveto{\pgfqpoint{3.566779in}{4.292778in}}%
\pgfpathlineto{\pgfqpoint{3.579587in}{4.272165in}}%
\pgfpathlineto{\pgfqpoint{3.592392in}{4.251757in}}%
\pgfpathlineto{\pgfqpoint{3.605191in}{4.231553in}}%
\pgfpathlineto{\pgfqpoint{3.617986in}{4.211550in}}%
\pgfpathlineto{\pgfqpoint{3.625407in}{4.229924in}}%
\pgfpathlineto{\pgfqpoint{3.632825in}{4.248493in}}%
\pgfpathlineto{\pgfqpoint{3.640238in}{4.267258in}}%
\pgfpathlineto{\pgfqpoint{3.627449in}{4.287451in}}%
\pgfpathlineto{\pgfqpoint{3.614655in}{4.307846in}}%
\pgfpathlineto{\pgfqpoint{3.601856in}{4.328444in}}%
\pgfpathlineto{\pgfqpoint{3.589053in}{4.349247in}}%
\pgfpathlineto{\pgfqpoint{3.581632in}{4.330223in}}%
\pgfpathlineto{\pgfqpoint{3.574207in}{4.311401in}}%
\pgfpathlineto{\pgfqpoint{3.566779in}{4.292778in}}%
\pgfpathclose%
\pgfusepath{fill}%
\end{pgfscope}%
\begin{pgfscope}%
\pgfpathrectangle{\pgfqpoint{1.254980in}{0.150000in}}{\pgfqpoint{5.490039in}{5.490039in}}%
\pgfusepath{clip}%
\pgfsetbuttcap%
\pgfsetroundjoin%
\definecolor{currentfill}{rgb}{0.194100,0.399323,0.555565}%
\pgfsetfillcolor{currentfill}%
\pgfsetfillopacity{0.700000}%
\pgfsetlinewidth{0.000000pt}%
\definecolor{currentstroke}{rgb}{0.000000,0.000000,0.000000}%
\pgfsetstrokecolor{currentstroke}%
\pgfsetdash{}{0pt}%
\pgfpathmoveto{\pgfqpoint{4.382928in}{3.336614in}}%
\pgfpathlineto{\pgfqpoint{4.395709in}{3.326851in}}%
\pgfpathlineto{\pgfqpoint{4.408493in}{3.317231in}}%
\pgfpathlineto{\pgfqpoint{4.421281in}{3.307753in}}%
\pgfpathlineto{\pgfqpoint{4.434073in}{3.298415in}}%
\pgfpathlineto{\pgfqpoint{4.441359in}{3.312965in}}%
\pgfpathlineto{\pgfqpoint{4.448644in}{3.327642in}}%
\pgfpathlineto{\pgfqpoint{4.455926in}{3.342450in}}%
\pgfpathlineto{\pgfqpoint{4.443142in}{3.351966in}}%
\pgfpathlineto{\pgfqpoint{4.430361in}{3.361623in}}%
\pgfpathlineto{\pgfqpoint{4.417583in}{3.371421in}}%
\pgfpathlineto{\pgfqpoint{4.404809in}{3.381361in}}%
\pgfpathlineto{\pgfqpoint{4.397518in}{3.366312in}}%
\pgfpathlineto{\pgfqpoint{4.390224in}{3.351397in}}%
\pgfpathlineto{\pgfqpoint{4.382928in}{3.336614in}}%
\pgfpathclose%
\pgfusepath{fill}%
\end{pgfscope}%
\begin{pgfscope}%
\pgfpathrectangle{\pgfqpoint{1.254980in}{0.150000in}}{\pgfqpoint{5.490039in}{5.490039in}}%
\pgfusepath{clip}%
\pgfsetbuttcap%
\pgfsetroundjoin%
\definecolor{currentfill}{rgb}{0.188923,0.410910,0.556326}%
\pgfsetfillcolor{currentfill}%
\pgfsetfillopacity{0.700000}%
\pgfsetlinewidth{0.000000pt}%
\definecolor{currentstroke}{rgb}{0.000000,0.000000,0.000000}%
\pgfsetstrokecolor{currentstroke}%
\pgfsetdash{}{0pt}%
\pgfpathmoveto{\pgfqpoint{4.251507in}{3.360242in}}%
\pgfpathlineto{\pgfqpoint{4.264274in}{3.349545in}}%
\pgfpathlineto{\pgfqpoint{4.277044in}{3.338995in}}%
\pgfpathlineto{\pgfqpoint{4.289816in}{3.328591in}}%
\pgfpathlineto{\pgfqpoint{4.302591in}{3.318333in}}%
\pgfpathlineto{\pgfqpoint{4.309906in}{3.332833in}}%
\pgfpathlineto{\pgfqpoint{4.317218in}{3.347458in}}%
\pgfpathlineto{\pgfqpoint{4.324528in}{3.362211in}}%
\pgfpathlineto{\pgfqpoint{4.331835in}{3.377095in}}%
\pgfpathlineto{\pgfqpoint{4.319070in}{3.387577in}}%
\pgfpathlineto{\pgfqpoint{4.306307in}{3.398205in}}%
\pgfpathlineto{\pgfqpoint{4.293546in}{3.408980in}}%
\pgfpathlineto{\pgfqpoint{4.280789in}{3.419902in}}%
\pgfpathlineto{\pgfqpoint{4.273472in}{3.404787in}}%
\pgfpathlineto{\pgfqpoint{4.266153in}{3.389807in}}%
\pgfpathlineto{\pgfqpoint{4.258831in}{3.374960in}}%
\pgfpathlineto{\pgfqpoint{4.251507in}{3.360242in}}%
\pgfpathclose%
\pgfusepath{fill}%
\end{pgfscope}%
\begin{pgfscope}%
\pgfpathrectangle{\pgfqpoint{1.254980in}{0.150000in}}{\pgfqpoint{5.490039in}{5.490039in}}%
\pgfusepath{clip}%
\pgfsetbuttcap%
\pgfsetroundjoin%
\definecolor{currentfill}{rgb}{0.175841,0.441290,0.557685}%
\pgfsetfillcolor{currentfill}%
\pgfsetfillopacity{0.700000}%
\pgfsetlinewidth{0.000000pt}%
\definecolor{currentstroke}{rgb}{0.000000,0.000000,0.000000}%
\pgfsetstrokecolor{currentstroke}%
\pgfsetdash{}{0pt}%
\pgfpathmoveto{\pgfqpoint{4.069031in}{3.440228in}}%
\pgfpathlineto{\pgfqpoint{4.081785in}{3.427918in}}%
\pgfpathlineto{\pgfqpoint{4.094540in}{3.415763in}}%
\pgfpathlineto{\pgfqpoint{4.107296in}{3.403764in}}%
\pgfpathlineto{\pgfqpoint{4.120054in}{3.391919in}}%
\pgfpathlineto{\pgfqpoint{4.127408in}{3.406544in}}%
\pgfpathlineto{\pgfqpoint{4.134758in}{3.421299in}}%
\pgfpathlineto{\pgfqpoint{4.142106in}{3.436186in}}%
\pgfpathlineto{\pgfqpoint{4.149451in}{3.451206in}}%
\pgfpathlineto{\pgfqpoint{4.136702in}{3.463260in}}%
\pgfpathlineto{\pgfqpoint{4.123955in}{3.475470in}}%
\pgfpathlineto{\pgfqpoint{4.111209in}{3.487834in}}%
\pgfpathlineto{\pgfqpoint{4.098464in}{3.500355in}}%
\pgfpathlineto{\pgfqpoint{4.091110in}{3.485119in}}%
\pgfpathlineto{\pgfqpoint{4.083753in}{3.470021in}}%
\pgfpathlineto{\pgfqpoint{4.076394in}{3.455058in}}%
\pgfpathlineto{\pgfqpoint{4.069031in}{3.440228in}}%
\pgfpathclose%
\pgfusepath{fill}%
\end{pgfscope}%
\begin{pgfscope}%
\pgfpathrectangle{\pgfqpoint{1.254980in}{0.150000in}}{\pgfqpoint{5.490039in}{5.490039in}}%
\pgfusepath{clip}%
\pgfsetbuttcap%
\pgfsetroundjoin%
\definecolor{currentfill}{rgb}{0.196571,0.711827,0.479221}%
\pgfsetfillcolor{currentfill}%
\pgfsetfillopacity{0.700000}%
\pgfsetlinewidth{0.000000pt}%
\definecolor{currentstroke}{rgb}{0.000000,0.000000,0.000000}%
\pgfsetstrokecolor{currentstroke}%
\pgfsetdash{}{0pt}%
\pgfpathmoveto{\pgfqpoint{3.588265in}{4.139944in}}%
\pgfpathlineto{\pgfqpoint{3.601064in}{4.120378in}}%
\pgfpathlineto{\pgfqpoint{3.613859in}{4.101011in}}%
\pgfpathlineto{\pgfqpoint{3.626650in}{4.081841in}}%
\pgfpathlineto{\pgfqpoint{3.639438in}{4.062867in}}%
\pgfpathlineto{\pgfqpoint{3.646866in}{4.080260in}}%
\pgfpathlineto{\pgfqpoint{3.654290in}{4.097833in}}%
\pgfpathlineto{\pgfqpoint{3.661710in}{4.115588in}}%
\pgfpathlineto{\pgfqpoint{3.669127in}{4.133528in}}%
\pgfpathlineto{\pgfqpoint{3.656347in}{4.152737in}}%
\pgfpathlineto{\pgfqpoint{3.643564in}{4.172143in}}%
\pgfpathlineto{\pgfqpoint{3.630777in}{4.191747in}}%
\pgfpathlineto{\pgfqpoint{3.617986in}{4.211550in}}%
\pgfpathlineto{\pgfqpoint{3.610562in}{4.193367in}}%
\pgfpathlineto{\pgfqpoint{3.603133in}{4.175373in}}%
\pgfpathlineto{\pgfqpoint{3.595701in}{4.157566in}}%
\pgfpathlineto{\pgfqpoint{3.588265in}{4.139944in}}%
\pgfpathclose%
\pgfusepath{fill}%
\end{pgfscope}%
\begin{pgfscope}%
\pgfpathrectangle{\pgfqpoint{1.254980in}{0.150000in}}{\pgfqpoint{5.490039in}{5.490039in}}%
\pgfusepath{clip}%
\pgfsetbuttcap%
\pgfsetroundjoin%
\definecolor{currentfill}{rgb}{0.121380,0.629492,0.531973}%
\pgfsetfillcolor{currentfill}%
\pgfsetfillopacity{0.700000}%
\pgfsetlinewidth{0.000000pt}%
\definecolor{currentstroke}{rgb}{0.000000,0.000000,0.000000}%
\pgfsetstrokecolor{currentstroke}%
\pgfsetdash{}{0pt}%
\pgfpathmoveto{\pgfqpoint{3.660837in}{3.921955in}}%
\pgfpathlineto{\pgfqpoint{3.673616in}{3.904159in}}%
\pgfpathlineto{\pgfqpoint{3.686393in}{3.886550in}}%
\pgfpathlineto{\pgfqpoint{3.699167in}{3.869128in}}%
\pgfpathlineto{\pgfqpoint{3.711938in}{3.851892in}}%
\pgfpathlineto{\pgfqpoint{3.719364in}{3.868167in}}%
\pgfpathlineto{\pgfqpoint{3.726786in}{3.884605in}}%
\pgfpathlineto{\pgfqpoint{3.734204in}{3.901207in}}%
\pgfpathlineto{\pgfqpoint{3.741619in}{3.917975in}}%
\pgfpathlineto{\pgfqpoint{3.728856in}{3.935427in}}%
\pgfpathlineto{\pgfqpoint{3.716091in}{3.953066in}}%
\pgfpathlineto{\pgfqpoint{3.703323in}{3.970891in}}%
\pgfpathlineto{\pgfqpoint{3.690553in}{3.988904in}}%
\pgfpathlineto{\pgfqpoint{3.683129in}{3.971913in}}%
\pgfpathlineto{\pgfqpoint{3.675702in}{3.955093in}}%
\pgfpathlineto{\pgfqpoint{3.668272in}{3.938441in}}%
\pgfpathlineto{\pgfqpoint{3.660837in}{3.921955in}}%
\pgfpathclose%
\pgfusepath{fill}%
\end{pgfscope}%
\begin{pgfscope}%
\pgfpathrectangle{\pgfqpoint{1.254980in}{0.150000in}}{\pgfqpoint{5.490039in}{5.490039in}}%
\pgfusepath{clip}%
\pgfsetbuttcap%
\pgfsetroundjoin%
\definecolor{currentfill}{rgb}{0.183898,0.422383,0.556944}%
\pgfsetfillcolor{currentfill}%
\pgfsetfillopacity{0.700000}%
\pgfsetlinewidth{0.000000pt}%
\definecolor{currentstroke}{rgb}{0.000000,0.000000,0.000000}%
\pgfsetstrokecolor{currentstroke}%
\pgfsetdash{}{0pt}%
\pgfpathmoveto{\pgfqpoint{4.120054in}{3.391919in}}%
\pgfpathlineto{\pgfqpoint{4.132814in}{3.380227in}}%
\pgfpathlineto{\pgfqpoint{4.145575in}{3.368689in}}%
\pgfpathlineto{\pgfqpoint{4.158338in}{3.357303in}}%
\pgfpathlineto{\pgfqpoint{4.171103in}{3.346068in}}%
\pgfpathlineto{\pgfqpoint{4.178447in}{3.360490in}}%
\pgfpathlineto{\pgfqpoint{4.185789in}{3.375038in}}%
\pgfpathlineto{\pgfqpoint{4.193127in}{3.389713in}}%
\pgfpathlineto{\pgfqpoint{4.200463in}{3.404518in}}%
\pgfpathlineto{\pgfqpoint{4.187707in}{3.415962in}}%
\pgfpathlineto{\pgfqpoint{4.174954in}{3.427557in}}%
\pgfpathlineto{\pgfqpoint{4.162201in}{3.439305in}}%
\pgfpathlineto{\pgfqpoint{4.149451in}{3.451206in}}%
\pgfpathlineto{\pgfqpoint{4.142106in}{3.436186in}}%
\pgfpathlineto{\pgfqpoint{4.134758in}{3.421299in}}%
\pgfpathlineto{\pgfqpoint{4.127408in}{3.406544in}}%
\pgfpathlineto{\pgfqpoint{4.120054in}{3.391919in}}%
\pgfpathclose%
\pgfusepath{fill}%
\end{pgfscope}%
\begin{pgfscope}%
\pgfpathrectangle{\pgfqpoint{1.254980in}{0.150000in}}{\pgfqpoint{5.490039in}{5.490039in}}%
\pgfusepath{clip}%
\pgfsetbuttcap%
\pgfsetroundjoin%
\definecolor{currentfill}{rgb}{0.195860,0.395433,0.555276}%
\pgfsetfillcolor{currentfill}%
\pgfsetfillopacity{0.700000}%
\pgfsetlinewidth{0.000000pt}%
\definecolor{currentstroke}{rgb}{0.000000,0.000000,0.000000}%
\pgfsetstrokecolor{currentstroke}%
\pgfsetdash{}{0pt}%
\pgfpathmoveto{\pgfqpoint{4.302591in}{3.318333in}}%
\pgfpathlineto{\pgfqpoint{4.315369in}{3.308221in}}%
\pgfpathlineto{\pgfqpoint{4.328150in}{3.298253in}}%
\pgfpathlineto{\pgfqpoint{4.340934in}{3.288429in}}%
\pgfpathlineto{\pgfqpoint{4.353721in}{3.278748in}}%
\pgfpathlineto{\pgfqpoint{4.361026in}{3.293029in}}%
\pgfpathlineto{\pgfqpoint{4.368329in}{3.307432in}}%
\pgfpathlineto{\pgfqpoint{4.375630in}{3.321959in}}%
\pgfpathlineto{\pgfqpoint{4.382928in}{3.336614in}}%
\pgfpathlineto{\pgfqpoint{4.370150in}{3.346518in}}%
\pgfpathlineto{\pgfqpoint{4.357376in}{3.356567in}}%
\pgfpathlineto{\pgfqpoint{4.344604in}{3.366759in}}%
\pgfpathlineto{\pgfqpoint{4.331835in}{3.377095in}}%
\pgfpathlineto{\pgfqpoint{4.324528in}{3.362211in}}%
\pgfpathlineto{\pgfqpoint{4.317218in}{3.347458in}}%
\pgfpathlineto{\pgfqpoint{4.309906in}{3.332833in}}%
\pgfpathlineto{\pgfqpoint{4.302591in}{3.318333in}}%
\pgfpathclose%
\pgfusepath{fill}%
\end{pgfscope}%
\begin{pgfscope}%
\pgfpathrectangle{\pgfqpoint{1.254980in}{0.150000in}}{\pgfqpoint{5.490039in}{5.490039in}}%
\pgfusepath{clip}%
\pgfsetbuttcap%
\pgfsetroundjoin%
\definecolor{currentfill}{rgb}{0.134692,0.658636,0.517649}%
\pgfsetfillcolor{currentfill}%
\pgfsetfillopacity{0.700000}%
\pgfsetlinewidth{0.000000pt}%
\definecolor{currentstroke}{rgb}{0.000000,0.000000,0.000000}%
\pgfsetstrokecolor{currentstroke}%
\pgfsetdash{}{0pt}%
\pgfpathmoveto{\pgfqpoint{3.609687in}{3.995047in}}%
\pgfpathlineto{\pgfqpoint{3.622480in}{3.976486in}}%
\pgfpathlineto{\pgfqpoint{3.635269in}{3.958118in}}%
\pgfpathlineto{\pgfqpoint{3.648054in}{3.939941in}}%
\pgfpathlineto{\pgfqpoint{3.660837in}{3.921955in}}%
\pgfpathlineto{\pgfqpoint{3.668272in}{3.938441in}}%
\pgfpathlineto{\pgfqpoint{3.675702in}{3.955093in}}%
\pgfpathlineto{\pgfqpoint{3.683129in}{3.971913in}}%
\pgfpathlineto{\pgfqpoint{3.690553in}{3.988904in}}%
\pgfpathlineto{\pgfqpoint{3.677779in}{4.007108in}}%
\pgfpathlineto{\pgfqpoint{3.665002in}{4.025502in}}%
\pgfpathlineto{\pgfqpoint{3.652221in}{4.044088in}}%
\pgfpathlineto{\pgfqpoint{3.639438in}{4.062867in}}%
\pgfpathlineto{\pgfqpoint{3.632006in}{4.045651in}}%
\pgfpathlineto{\pgfqpoint{3.624570in}{4.028611in}}%
\pgfpathlineto{\pgfqpoint{3.617131in}{4.011744in}}%
\pgfpathlineto{\pgfqpoint{3.609687in}{3.995047in}}%
\pgfpathclose%
\pgfusepath{fill}%
\end{pgfscope}%
\begin{pgfscope}%
\pgfpathrectangle{\pgfqpoint{1.254980in}{0.150000in}}{\pgfqpoint{5.490039in}{5.490039in}}%
\pgfusepath{clip}%
\pgfsetbuttcap%
\pgfsetroundjoin%
\definecolor{currentfill}{rgb}{0.199430,0.387607,0.554642}%
\pgfsetfillcolor{currentfill}%
\pgfsetfillopacity{0.700000}%
\pgfsetlinewidth{0.000000pt}%
\definecolor{currentstroke}{rgb}{0.000000,0.000000,0.000000}%
\pgfsetstrokecolor{currentstroke}%
\pgfsetdash{}{0pt}%
\pgfpathmoveto{\pgfqpoint{4.434073in}{3.298415in}}%
\pgfpathlineto{\pgfqpoint{4.446868in}{3.289217in}}%
\pgfpathlineto{\pgfqpoint{4.459667in}{3.280159in}}%
\pgfpathlineto{\pgfqpoint{4.472470in}{3.271240in}}%
\pgfpathlineto{\pgfqpoint{4.485277in}{3.262459in}}%
\pgfpathlineto{\pgfqpoint{4.492554in}{3.276776in}}%
\pgfpathlineto{\pgfqpoint{4.499829in}{3.291216in}}%
\pgfpathlineto{\pgfqpoint{4.507101in}{3.305783in}}%
\pgfpathlineto{\pgfqpoint{4.494302in}{3.314741in}}%
\pgfpathlineto{\pgfqpoint{4.481506in}{3.323838in}}%
\pgfpathlineto{\pgfqpoint{4.468714in}{3.333074in}}%
\pgfpathlineto{\pgfqpoint{4.455926in}{3.342450in}}%
\pgfpathlineto{\pgfqpoint{4.448644in}{3.327642in}}%
\pgfpathlineto{\pgfqpoint{4.441359in}{3.312965in}}%
\pgfpathlineto{\pgfqpoint{4.434073in}{3.298415in}}%
\pgfpathclose%
\pgfusepath{fill}%
\end{pgfscope}%
\begin{pgfscope}%
\pgfpathrectangle{\pgfqpoint{1.254980in}{0.150000in}}{\pgfqpoint{5.490039in}{5.490039in}}%
\pgfusepath{clip}%
\pgfsetbuttcap%
\pgfsetroundjoin%
\definecolor{currentfill}{rgb}{0.369214,0.788888,0.382914}%
\pgfsetfillcolor{currentfill}%
\pgfsetfillopacity{0.700000}%
\pgfsetlinewidth{0.000000pt}%
\definecolor{currentstroke}{rgb}{0.000000,0.000000,0.000000}%
\pgfsetstrokecolor{currentstroke}%
\pgfsetdash{}{0pt}%
\pgfpathmoveto{\pgfqpoint{3.515493in}{4.377303in}}%
\pgfpathlineto{\pgfqpoint{3.528322in}{4.355857in}}%
\pgfpathlineto{\pgfqpoint{3.541146in}{4.334622in}}%
\pgfpathlineto{\pgfqpoint{3.553965in}{4.313596in}}%
\pgfpathlineto{\pgfqpoint{3.566779in}{4.292778in}}%
\pgfpathlineto{\pgfqpoint{3.574207in}{4.311401in}}%
\pgfpathlineto{\pgfqpoint{3.581632in}{4.330223in}}%
\pgfpathlineto{\pgfqpoint{3.589053in}{4.349247in}}%
\pgfpathlineto{\pgfqpoint{3.576245in}{4.370256in}}%
\pgfpathlineto{\pgfqpoint{3.563432in}{4.391474in}}%
\pgfpathlineto{\pgfqpoint{3.550613in}{4.412900in}}%
\pgfpathlineto{\pgfqpoint{3.537790in}{4.434538in}}%
\pgfpathlineto{\pgfqpoint{3.530362in}{4.415254in}}%
\pgfpathlineto{\pgfqpoint{3.522929in}{4.396177in}}%
\pgfpathlineto{\pgfqpoint{3.515493in}{4.377303in}}%
\pgfpathclose%
\pgfusepath{fill}%
\end{pgfscope}%
\begin{pgfscope}%
\pgfpathrectangle{\pgfqpoint{1.254980in}{0.150000in}}{\pgfqpoint{5.490039in}{5.490039in}}%
\pgfusepath{clip}%
\pgfsetbuttcap%
\pgfsetroundjoin%
\definecolor{currentfill}{rgb}{0.150476,0.504369,0.557430}%
\pgfsetfillcolor{currentfill}%
\pgfsetfillopacity{0.700000}%
\pgfsetlinewidth{0.000000pt}%
\definecolor{currentstroke}{rgb}{0.000000,0.000000,0.000000}%
\pgfsetstrokecolor{currentstroke}%
\pgfsetdash{}{0pt}%
\pgfpathmoveto{\pgfqpoint{3.835412in}{3.597967in}}%
\pgfpathlineto{\pgfqpoint{3.848171in}{3.583238in}}%
\pgfpathlineto{\pgfqpoint{3.860930in}{3.568679in}}%
\pgfpathlineto{\pgfqpoint{3.873688in}{3.554289in}}%
\pgfpathlineto{\pgfqpoint{3.886445in}{3.540068in}}%
\pgfpathlineto{\pgfqpoint{3.893849in}{3.554952in}}%
\pgfpathlineto{\pgfqpoint{3.901249in}{3.569974in}}%
\pgfpathlineto{\pgfqpoint{3.908646in}{3.585134in}}%
\pgfpathlineto{\pgfqpoint{3.916039in}{3.600436in}}%
\pgfpathlineto{\pgfqpoint{3.903291in}{3.614853in}}%
\pgfpathlineto{\pgfqpoint{3.890542in}{3.629439in}}%
\pgfpathlineto{\pgfqpoint{3.877793in}{3.644194in}}%
\pgfpathlineto{\pgfqpoint{3.865042in}{3.659119in}}%
\pgfpathlineto{\pgfqpoint{3.857640in}{3.643615in}}%
\pgfpathlineto{\pgfqpoint{3.850234in}{3.628257in}}%
\pgfpathlineto{\pgfqpoint{3.842825in}{3.613041in}}%
\pgfpathlineto{\pgfqpoint{3.835412in}{3.597967in}}%
\pgfpathclose%
\pgfusepath{fill}%
\end{pgfscope}%
\begin{pgfscope}%
\pgfpathrectangle{\pgfqpoint{1.254980in}{0.150000in}}{\pgfqpoint{5.490039in}{5.490039in}}%
\pgfusepath{clip}%
\pgfsetbuttcap%
\pgfsetroundjoin%
\definecolor{currentfill}{rgb}{0.252899,0.742211,0.448284}%
\pgfsetfillcolor{currentfill}%
\pgfsetfillopacity{0.700000}%
\pgfsetlinewidth{0.000000pt}%
\definecolor{currentstroke}{rgb}{0.000000,0.000000,0.000000}%
\pgfsetstrokecolor{currentstroke}%
\pgfsetdash{}{0pt}%
\pgfpathmoveto{\pgfqpoint{3.537025in}{4.220220in}}%
\pgfpathlineto{\pgfqpoint{3.549842in}{4.199846in}}%
\pgfpathlineto{\pgfqpoint{3.562654in}{4.179676in}}%
\pgfpathlineto{\pgfqpoint{3.575462in}{4.159709in}}%
\pgfpathlineto{\pgfqpoint{3.588265in}{4.139944in}}%
\pgfpathlineto{\pgfqpoint{3.595701in}{4.157566in}}%
\pgfpathlineto{\pgfqpoint{3.603133in}{4.175373in}}%
\pgfpathlineto{\pgfqpoint{3.610562in}{4.193367in}}%
\pgfpathlineto{\pgfqpoint{3.617986in}{4.211550in}}%
\pgfpathlineto{\pgfqpoint{3.605191in}{4.231553in}}%
\pgfpathlineto{\pgfqpoint{3.592392in}{4.251757in}}%
\pgfpathlineto{\pgfqpoint{3.579587in}{4.272165in}}%
\pgfpathlineto{\pgfqpoint{3.566779in}{4.292778in}}%
\pgfpathlineto{\pgfqpoint{3.559346in}{4.274351in}}%
\pgfpathlineto{\pgfqpoint{3.551910in}{4.256117in}}%
\pgfpathlineto{\pgfqpoint{3.544469in}{4.238074in}}%
\pgfpathlineto{\pgfqpoint{3.537025in}{4.220220in}}%
\pgfpathclose%
\pgfusepath{fill}%
\end{pgfscope}%
\begin{pgfscope}%
\pgfpathrectangle{\pgfqpoint{1.254980in}{0.150000in}}{\pgfqpoint{5.490039in}{5.490039in}}%
\pgfusepath{clip}%
\pgfsetbuttcap%
\pgfsetroundjoin%
\definecolor{currentfill}{rgb}{0.141935,0.526453,0.555991}%
\pgfsetfillcolor{currentfill}%
\pgfsetfillopacity{0.700000}%
\pgfsetlinewidth{0.000000pt}%
\definecolor{currentstroke}{rgb}{0.000000,0.000000,0.000000}%
\pgfsetstrokecolor{currentstroke}%
\pgfsetdash{}{0pt}%
\pgfpathmoveto{\pgfqpoint{3.784364in}{3.658605in}}%
\pgfpathlineto{\pgfqpoint{3.797128in}{3.643185in}}%
\pgfpathlineto{\pgfqpoint{3.809890in}{3.627940in}}%
\pgfpathlineto{\pgfqpoint{3.822651in}{3.612867in}}%
\pgfpathlineto{\pgfqpoint{3.835412in}{3.597967in}}%
\pgfpathlineto{\pgfqpoint{3.842825in}{3.613041in}}%
\pgfpathlineto{\pgfqpoint{3.850234in}{3.628257in}}%
\pgfpathlineto{\pgfqpoint{3.857640in}{3.643615in}}%
\pgfpathlineto{\pgfqpoint{3.865042in}{3.659119in}}%
\pgfpathlineto{\pgfqpoint{3.852291in}{3.674216in}}%
\pgfpathlineto{\pgfqpoint{3.839540in}{3.689485in}}%
\pgfpathlineto{\pgfqpoint{3.826787in}{3.704928in}}%
\pgfpathlineto{\pgfqpoint{3.814032in}{3.720545in}}%
\pgfpathlineto{\pgfqpoint{3.806621in}{3.704838in}}%
\pgfpathlineto{\pgfqpoint{3.799206in}{3.689281in}}%
\pgfpathlineto{\pgfqpoint{3.791787in}{3.673871in}}%
\pgfpathlineto{\pgfqpoint{3.784364in}{3.658605in}}%
\pgfpathclose%
\pgfusepath{fill}%
\end{pgfscope}%
\begin{pgfscope}%
\pgfpathrectangle{\pgfqpoint{1.254980in}{0.150000in}}{\pgfqpoint{5.490039in}{5.490039in}}%
\pgfusepath{clip}%
\pgfsetbuttcap%
\pgfsetroundjoin%
\definecolor{currentfill}{rgb}{0.159194,0.482237,0.558073}%
\pgfsetfillcolor{currentfill}%
\pgfsetfillopacity{0.700000}%
\pgfsetlinewidth{0.000000pt}%
\definecolor{currentstroke}{rgb}{0.000000,0.000000,0.000000}%
\pgfsetstrokecolor{currentstroke}%
\pgfsetdash{}{0pt}%
\pgfpathmoveto{\pgfqpoint{3.886445in}{3.540068in}}%
\pgfpathlineto{\pgfqpoint{3.899203in}{3.526014in}}%
\pgfpathlineto{\pgfqpoint{3.911960in}{3.512126in}}%
\pgfpathlineto{\pgfqpoint{3.924717in}{3.498404in}}%
\pgfpathlineto{\pgfqpoint{3.937474in}{3.484846in}}%
\pgfpathlineto{\pgfqpoint{3.944869in}{3.499541in}}%
\pgfpathlineto{\pgfqpoint{3.952259in}{3.514369in}}%
\pgfpathlineto{\pgfqpoint{3.959647in}{3.529332in}}%
\pgfpathlineto{\pgfqpoint{3.967031in}{3.544432in}}%
\pgfpathlineto{\pgfqpoint{3.954283in}{3.558185in}}%
\pgfpathlineto{\pgfqpoint{3.941535in}{3.572103in}}%
\pgfpathlineto{\pgfqpoint{3.928787in}{3.586186in}}%
\pgfpathlineto{\pgfqpoint{3.916039in}{3.600436in}}%
\pgfpathlineto{\pgfqpoint{3.908646in}{3.585134in}}%
\pgfpathlineto{\pgfqpoint{3.901249in}{3.569974in}}%
\pgfpathlineto{\pgfqpoint{3.893849in}{3.554952in}}%
\pgfpathlineto{\pgfqpoint{3.886445in}{3.540068in}}%
\pgfpathclose%
\pgfusepath{fill}%
\end{pgfscope}%
\begin{pgfscope}%
\pgfpathrectangle{\pgfqpoint{1.254980in}{0.150000in}}{\pgfqpoint{5.490039in}{5.490039in}}%
\pgfusepath{clip}%
\pgfsetbuttcap%
\pgfsetroundjoin%
\definecolor{currentfill}{rgb}{0.132444,0.552216,0.553018}%
\pgfsetfillcolor{currentfill}%
\pgfsetfillopacity{0.700000}%
\pgfsetlinewidth{0.000000pt}%
\definecolor{currentstroke}{rgb}{0.000000,0.000000,0.000000}%
\pgfsetstrokecolor{currentstroke}%
\pgfsetdash{}{0pt}%
\pgfpathmoveto{\pgfqpoint{3.733296in}{3.722049in}}%
\pgfpathlineto{\pgfqpoint{3.746065in}{3.705921in}}%
\pgfpathlineto{\pgfqpoint{3.758833in}{3.689972in}}%
\pgfpathlineto{\pgfqpoint{3.771600in}{3.674201in}}%
\pgfpathlineto{\pgfqpoint{3.784364in}{3.658605in}}%
\pgfpathlineto{\pgfqpoint{3.791787in}{3.673871in}}%
\pgfpathlineto{\pgfqpoint{3.799206in}{3.689281in}}%
\pgfpathlineto{\pgfqpoint{3.806621in}{3.704838in}}%
\pgfpathlineto{\pgfqpoint{3.814032in}{3.720545in}}%
\pgfpathlineto{\pgfqpoint{3.801277in}{3.736338in}}%
\pgfpathlineto{\pgfqpoint{3.788520in}{3.752307in}}%
\pgfpathlineto{\pgfqpoint{3.775761in}{3.768454in}}%
\pgfpathlineto{\pgfqpoint{3.763000in}{3.784780in}}%
\pgfpathlineto{\pgfqpoint{3.755580in}{3.768869in}}%
\pgfpathlineto{\pgfqpoint{3.748156in}{3.753112in}}%
\pgfpathlineto{\pgfqpoint{3.740728in}{3.737506in}}%
\pgfpathlineto{\pgfqpoint{3.733296in}{3.722049in}}%
\pgfpathclose%
\pgfusepath{fill}%
\end{pgfscope}%
\begin{pgfscope}%
\pgfpathrectangle{\pgfqpoint{1.254980in}{0.150000in}}{\pgfqpoint{5.490039in}{5.490039in}}%
\pgfusepath{clip}%
\pgfsetbuttcap%
\pgfsetroundjoin%
\definecolor{currentfill}{rgb}{0.190631,0.407061,0.556089}%
\pgfsetfillcolor{currentfill}%
\pgfsetfillopacity{0.700000}%
\pgfsetlinewidth{0.000000pt}%
\definecolor{currentstroke}{rgb}{0.000000,0.000000,0.000000}%
\pgfsetstrokecolor{currentstroke}%
\pgfsetdash{}{0pt}%
\pgfpathmoveto{\pgfqpoint{4.171103in}{3.346068in}}%
\pgfpathlineto{\pgfqpoint{4.183870in}{3.334984in}}%
\pgfpathlineto{\pgfqpoint{4.196639in}{3.324050in}}%
\pgfpathlineto{\pgfqpoint{4.209411in}{3.313265in}}%
\pgfpathlineto{\pgfqpoint{4.222185in}{3.302628in}}%
\pgfpathlineto{\pgfqpoint{4.229519in}{3.316848in}}%
\pgfpathlineto{\pgfqpoint{4.236851in}{3.331188in}}%
\pgfpathlineto{\pgfqpoint{4.244181in}{3.345653in}}%
\pgfpathlineto{\pgfqpoint{4.251507in}{3.360242in}}%
\pgfpathlineto{\pgfqpoint{4.238743in}{3.371087in}}%
\pgfpathlineto{\pgfqpoint{4.225981in}{3.382081in}}%
\pgfpathlineto{\pgfqpoint{4.213221in}{3.393224in}}%
\pgfpathlineto{\pgfqpoint{4.200463in}{3.404518in}}%
\pgfpathlineto{\pgfqpoint{4.193127in}{3.389713in}}%
\pgfpathlineto{\pgfqpoint{4.185789in}{3.375038in}}%
\pgfpathlineto{\pgfqpoint{4.178447in}{3.360490in}}%
\pgfpathlineto{\pgfqpoint{4.171103in}{3.346068in}}%
\pgfpathclose%
\pgfusepath{fill}%
\end{pgfscope}%
\begin{pgfscope}%
\pgfpathrectangle{\pgfqpoint{1.254980in}{0.150000in}}{\pgfqpoint{5.490039in}{5.490039in}}%
\pgfusepath{clip}%
\pgfsetbuttcap%
\pgfsetroundjoin%
\definecolor{currentfill}{rgb}{0.166617,0.463708,0.558119}%
\pgfsetfillcolor{currentfill}%
\pgfsetfillopacity{0.700000}%
\pgfsetlinewidth{0.000000pt}%
\definecolor{currentstroke}{rgb}{0.000000,0.000000,0.000000}%
\pgfsetstrokecolor{currentstroke}%
\pgfsetdash{}{0pt}%
\pgfpathmoveto{\pgfqpoint{3.937474in}{3.484846in}}%
\pgfpathlineto{\pgfqpoint{3.950232in}{3.471452in}}%
\pgfpathlineto{\pgfqpoint{3.962989in}{3.458220in}}%
\pgfpathlineto{\pgfqpoint{3.975747in}{3.445150in}}%
\pgfpathlineto{\pgfqpoint{3.988506in}{3.432242in}}%
\pgfpathlineto{\pgfqpoint{3.995891in}{3.446748in}}%
\pgfpathlineto{\pgfqpoint{4.003272in}{3.461383in}}%
\pgfpathlineto{\pgfqpoint{4.010651in}{3.476149in}}%
\pgfpathlineto{\pgfqpoint{4.018026in}{3.491048in}}%
\pgfpathlineto{\pgfqpoint{4.005276in}{3.504152in}}%
\pgfpathlineto{\pgfqpoint{3.992528in}{3.517416in}}%
\pgfpathlineto{\pgfqpoint{3.979779in}{3.530843in}}%
\pgfpathlineto{\pgfqpoint{3.967031in}{3.544432in}}%
\pgfpathlineto{\pgfqpoint{3.959647in}{3.529332in}}%
\pgfpathlineto{\pgfqpoint{3.952259in}{3.514369in}}%
\pgfpathlineto{\pgfqpoint{3.944869in}{3.499541in}}%
\pgfpathlineto{\pgfqpoint{3.937474in}{3.484846in}}%
\pgfpathclose%
\pgfusepath{fill}%
\end{pgfscope}%
\begin{pgfscope}%
\pgfpathrectangle{\pgfqpoint{1.254980in}{0.150000in}}{\pgfqpoint{5.490039in}{5.490039in}}%
\pgfusepath{clip}%
\pgfsetbuttcap%
\pgfsetroundjoin%
\definecolor{currentfill}{rgb}{0.124395,0.578002,0.548287}%
\pgfsetfillcolor{currentfill}%
\pgfsetfillopacity{0.700000}%
\pgfsetlinewidth{0.000000pt}%
\definecolor{currentstroke}{rgb}{0.000000,0.000000,0.000000}%
\pgfsetstrokecolor{currentstroke}%
\pgfsetdash{}{0pt}%
\pgfpathmoveto{\pgfqpoint{3.682197in}{3.788367in}}%
\pgfpathlineto{\pgfqpoint{3.694975in}{3.771514in}}%
\pgfpathlineto{\pgfqpoint{3.707751in}{3.754844in}}%
\pgfpathlineto{\pgfqpoint{3.720524in}{3.738356in}}%
\pgfpathlineto{\pgfqpoint{3.733296in}{3.722049in}}%
\pgfpathlineto{\pgfqpoint{3.740728in}{3.737506in}}%
\pgfpathlineto{\pgfqpoint{3.748156in}{3.753112in}}%
\pgfpathlineto{\pgfqpoint{3.755580in}{3.768869in}}%
\pgfpathlineto{\pgfqpoint{3.763000in}{3.784780in}}%
\pgfpathlineto{\pgfqpoint{3.750238in}{3.801285in}}%
\pgfpathlineto{\pgfqpoint{3.737474in}{3.817971in}}%
\pgfpathlineto{\pgfqpoint{3.724707in}{3.834840in}}%
\pgfpathlineto{\pgfqpoint{3.711938in}{3.851892in}}%
\pgfpathlineto{\pgfqpoint{3.704509in}{3.835776in}}%
\pgfpathlineto{\pgfqpoint{3.697075in}{3.819818in}}%
\pgfpathlineto{\pgfqpoint{3.689638in}{3.804016in}}%
\pgfpathlineto{\pgfqpoint{3.682197in}{3.788367in}}%
\pgfpathclose%
\pgfusepath{fill}%
\end{pgfscope}%
\begin{pgfscope}%
\pgfpathrectangle{\pgfqpoint{1.254980in}{0.150000in}}{\pgfqpoint{5.490039in}{5.490039in}}%
\pgfusepath{clip}%
\pgfsetbuttcap%
\pgfsetroundjoin%
\definecolor{currentfill}{rgb}{0.162016,0.687316,0.499129}%
\pgfsetfillcolor{currentfill}%
\pgfsetfillopacity{0.700000}%
\pgfsetlinewidth{0.000000pt}%
\definecolor{currentstroke}{rgb}{0.000000,0.000000,0.000000}%
\pgfsetstrokecolor{currentstroke}%
\pgfsetdash{}{0pt}%
\pgfpathmoveto{\pgfqpoint{3.558480in}{4.071248in}}%
\pgfpathlineto{\pgfqpoint{3.571287in}{4.051902in}}%
\pgfpathlineto{\pgfqpoint{3.584091in}{4.032754in}}%
\pgfpathlineto{\pgfqpoint{3.596891in}{4.013802in}}%
\pgfpathlineto{\pgfqpoint{3.609687in}{3.995047in}}%
\pgfpathlineto{\pgfqpoint{3.617131in}{4.011744in}}%
\pgfpathlineto{\pgfqpoint{3.624570in}{4.028611in}}%
\pgfpathlineto{\pgfqpoint{3.632006in}{4.045651in}}%
\pgfpathlineto{\pgfqpoint{3.639438in}{4.062867in}}%
\pgfpathlineto{\pgfqpoint{3.626650in}{4.081841in}}%
\pgfpathlineto{\pgfqpoint{3.613859in}{4.101011in}}%
\pgfpathlineto{\pgfqpoint{3.601064in}{4.120378in}}%
\pgfpathlineto{\pgfqpoint{3.588265in}{4.139944in}}%
\pgfpathlineto{\pgfqpoint{3.580825in}{4.122503in}}%
\pgfpathlineto{\pgfqpoint{3.573381in}{4.105241in}}%
\pgfpathlineto{\pgfqpoint{3.565932in}{4.088157in}}%
\pgfpathlineto{\pgfqpoint{3.558480in}{4.071248in}}%
\pgfpathclose%
\pgfusepath{fill}%
\end{pgfscope}%
\begin{pgfscope}%
\pgfpathrectangle{\pgfqpoint{1.254980in}{0.150000in}}{\pgfqpoint{5.490039in}{5.490039in}}%
\pgfusepath{clip}%
\pgfsetbuttcap%
\pgfsetroundjoin%
\definecolor{currentfill}{rgb}{0.175841,0.441290,0.557685}%
\pgfsetfillcolor{currentfill}%
\pgfsetfillopacity{0.700000}%
\pgfsetlinewidth{0.000000pt}%
\definecolor{currentstroke}{rgb}{0.000000,0.000000,0.000000}%
\pgfsetstrokecolor{currentstroke}%
\pgfsetdash{}{0pt}%
\pgfpathmoveto{\pgfqpoint{3.988506in}{3.432242in}}%
\pgfpathlineto{\pgfqpoint{4.001265in}{3.419493in}}%
\pgfpathlineto{\pgfqpoint{4.014026in}{3.406903in}}%
\pgfpathlineto{\pgfqpoint{4.026787in}{3.394472in}}%
\pgfpathlineto{\pgfqpoint{4.039549in}{3.382199in}}%
\pgfpathlineto{\pgfqpoint{4.046924in}{3.396517in}}%
\pgfpathlineto{\pgfqpoint{4.054296in}{3.410960in}}%
\pgfpathlineto{\pgfqpoint{4.061665in}{3.425530in}}%
\pgfpathlineto{\pgfqpoint{4.069031in}{3.440228in}}%
\pgfpathlineto{\pgfqpoint{4.056278in}{3.452696in}}%
\pgfpathlineto{\pgfqpoint{4.043527in}{3.465321in}}%
\pgfpathlineto{\pgfqpoint{4.030776in}{3.478105in}}%
\pgfpathlineto{\pgfqpoint{4.018026in}{3.491048in}}%
\pgfpathlineto{\pgfqpoint{4.010651in}{3.476149in}}%
\pgfpathlineto{\pgfqpoint{4.003272in}{3.461383in}}%
\pgfpathlineto{\pgfqpoint{3.995891in}{3.446748in}}%
\pgfpathlineto{\pgfqpoint{3.988506in}{3.432242in}}%
\pgfpathclose%
\pgfusepath{fill}%
\end{pgfscope}%
\begin{pgfscope}%
\pgfpathrectangle{\pgfqpoint{1.254980in}{0.150000in}}{\pgfqpoint{5.490039in}{5.490039in}}%
\pgfusepath{clip}%
\pgfsetbuttcap%
\pgfsetroundjoin%
\definecolor{currentfill}{rgb}{0.201239,0.383670,0.554294}%
\pgfsetfillcolor{currentfill}%
\pgfsetfillopacity{0.700000}%
\pgfsetlinewidth{0.000000pt}%
\definecolor{currentstroke}{rgb}{0.000000,0.000000,0.000000}%
\pgfsetstrokecolor{currentstroke}%
\pgfsetdash{}{0pt}%
\pgfpathmoveto{\pgfqpoint{4.353721in}{3.278748in}}%
\pgfpathlineto{\pgfqpoint{4.366512in}{3.269209in}}%
\pgfpathlineto{\pgfqpoint{4.379306in}{3.259813in}}%
\pgfpathlineto{\pgfqpoint{4.392103in}{3.250558in}}%
\pgfpathlineto{\pgfqpoint{4.404904in}{3.241443in}}%
\pgfpathlineto{\pgfqpoint{4.412200in}{3.255507in}}%
\pgfpathlineto{\pgfqpoint{4.419493in}{3.269688in}}%
\pgfpathlineto{\pgfqpoint{4.426784in}{3.283990in}}%
\pgfpathlineto{\pgfqpoint{4.434073in}{3.298415in}}%
\pgfpathlineto{\pgfqpoint{4.421281in}{3.307753in}}%
\pgfpathlineto{\pgfqpoint{4.408493in}{3.317231in}}%
\pgfpathlineto{\pgfqpoint{4.395709in}{3.326851in}}%
\pgfpathlineto{\pgfqpoint{4.382928in}{3.336614in}}%
\pgfpathlineto{\pgfqpoint{4.375630in}{3.321959in}}%
\pgfpathlineto{\pgfqpoint{4.368329in}{3.307432in}}%
\pgfpathlineto{\pgfqpoint{4.361026in}{3.293029in}}%
\pgfpathlineto{\pgfqpoint{4.353721in}{3.278748in}}%
\pgfpathclose%
\pgfusepath{fill}%
\end{pgfscope}%
\begin{pgfscope}%
\pgfpathrectangle{\pgfqpoint{1.254980in}{0.150000in}}{\pgfqpoint{5.490039in}{5.490039in}}%
\pgfusepath{clip}%
\pgfsetbuttcap%
\pgfsetroundjoin%
\definecolor{currentfill}{rgb}{0.119512,0.607464,0.540218}%
\pgfsetfillcolor{currentfill}%
\pgfsetfillopacity{0.700000}%
\pgfsetlinewidth{0.000000pt}%
\definecolor{currentstroke}{rgb}{0.000000,0.000000,0.000000}%
\pgfsetstrokecolor{currentstroke}%
\pgfsetdash{}{0pt}%
\pgfpathmoveto{\pgfqpoint{3.631058in}{3.857633in}}%
\pgfpathlineto{\pgfqpoint{3.643847in}{3.840036in}}%
\pgfpathlineto{\pgfqpoint{3.656633in}{3.822627in}}%
\pgfpathlineto{\pgfqpoint{3.669416in}{3.805404in}}%
\pgfpathlineto{\pgfqpoint{3.682197in}{3.788367in}}%
\pgfpathlineto{\pgfqpoint{3.689638in}{3.804016in}}%
\pgfpathlineto{\pgfqpoint{3.697075in}{3.819818in}}%
\pgfpathlineto{\pgfqpoint{3.704509in}{3.835776in}}%
\pgfpathlineto{\pgfqpoint{3.711938in}{3.851892in}}%
\pgfpathlineto{\pgfqpoint{3.699167in}{3.869128in}}%
\pgfpathlineto{\pgfqpoint{3.686393in}{3.886550in}}%
\pgfpathlineto{\pgfqpoint{3.673616in}{3.904159in}}%
\pgfpathlineto{\pgfqpoint{3.660837in}{3.921955in}}%
\pgfpathlineto{\pgfqpoint{3.653398in}{3.905634in}}%
\pgfpathlineto{\pgfqpoint{3.645956in}{3.889474in}}%
\pgfpathlineto{\pgfqpoint{3.638509in}{3.873475in}}%
\pgfpathlineto{\pgfqpoint{3.631058in}{3.857633in}}%
\pgfpathclose%
\pgfusepath{fill}%
\end{pgfscope}%
\begin{pgfscope}%
\pgfpathrectangle{\pgfqpoint{1.254980in}{0.150000in}}{\pgfqpoint{5.490039in}{5.490039in}}%
\pgfusepath{clip}%
\pgfsetbuttcap%
\pgfsetroundjoin%
\definecolor{currentfill}{rgb}{0.197636,0.391528,0.554969}%
\pgfsetfillcolor{currentfill}%
\pgfsetfillopacity{0.700000}%
\pgfsetlinewidth{0.000000pt}%
\definecolor{currentstroke}{rgb}{0.000000,0.000000,0.000000}%
\pgfsetstrokecolor{currentstroke}%
\pgfsetdash{}{0pt}%
\pgfpathmoveto{\pgfqpoint{4.222185in}{3.302628in}}%
\pgfpathlineto{\pgfqpoint{4.234961in}{3.292139in}}%
\pgfpathlineto{\pgfqpoint{4.247740in}{3.281798in}}%
\pgfpathlineto{\pgfqpoint{4.260521in}{3.271603in}}%
\pgfpathlineto{\pgfqpoint{4.273306in}{3.261553in}}%
\pgfpathlineto{\pgfqpoint{4.280631in}{3.275570in}}%
\pgfpathlineto{\pgfqpoint{4.287954in}{3.289705in}}%
\pgfpathlineto{\pgfqpoint{4.295274in}{3.303958in}}%
\pgfpathlineto{\pgfqpoint{4.302591in}{3.318333in}}%
\pgfpathlineto{\pgfqpoint{4.289816in}{3.328591in}}%
\pgfpathlineto{\pgfqpoint{4.277044in}{3.338995in}}%
\pgfpathlineto{\pgfqpoint{4.264274in}{3.349545in}}%
\pgfpathlineto{\pgfqpoint{4.251507in}{3.360242in}}%
\pgfpathlineto{\pgfqpoint{4.244181in}{3.345653in}}%
\pgfpathlineto{\pgfqpoint{4.236851in}{3.331188in}}%
\pgfpathlineto{\pgfqpoint{4.229519in}{3.316848in}}%
\pgfpathlineto{\pgfqpoint{4.222185in}{3.302628in}}%
\pgfpathclose%
\pgfusepath{fill}%
\end{pgfscope}%
\begin{pgfscope}%
\pgfpathrectangle{\pgfqpoint{1.254980in}{0.150000in}}{\pgfqpoint{5.490039in}{5.490039in}}%
\pgfusepath{clip}%
\pgfsetbuttcap%
\pgfsetroundjoin%
\definecolor{currentfill}{rgb}{0.204903,0.375746,0.553533}%
\pgfsetfillcolor{currentfill}%
\pgfsetfillopacity{0.700000}%
\pgfsetlinewidth{0.000000pt}%
\definecolor{currentstroke}{rgb}{0.000000,0.000000,0.000000}%
\pgfsetstrokecolor{currentstroke}%
\pgfsetdash{}{0pt}%
\pgfpathmoveto{\pgfqpoint{4.485277in}{3.262459in}}%
\pgfpathlineto{\pgfqpoint{4.498088in}{3.253816in}}%
\pgfpathlineto{\pgfqpoint{4.510903in}{3.245311in}}%
\pgfpathlineto{\pgfqpoint{4.523722in}{3.236942in}}%
\pgfpathlineto{\pgfqpoint{4.536546in}{3.228710in}}%
\pgfpathlineto{\pgfqpoint{4.543813in}{3.242794in}}%
\pgfpathlineto{\pgfqpoint{4.551078in}{3.256997in}}%
\pgfpathlineto{\pgfqpoint{4.558341in}{3.271323in}}%
\pgfpathlineto{\pgfqpoint{4.545525in}{3.279733in}}%
\pgfpathlineto{\pgfqpoint{4.532713in}{3.288279in}}%
\pgfpathlineto{\pgfqpoint{4.519905in}{3.296962in}}%
\pgfpathlineto{\pgfqpoint{4.507101in}{3.305783in}}%
\pgfpathlineto{\pgfqpoint{4.499829in}{3.291216in}}%
\pgfpathlineto{\pgfqpoint{4.492554in}{3.276776in}}%
\pgfpathlineto{\pgfqpoint{4.485277in}{3.262459in}}%
\pgfpathclose%
\pgfusepath{fill}%
\end{pgfscope}%
\begin{pgfscope}%
\pgfpathrectangle{\pgfqpoint{1.254980in}{0.150000in}}{\pgfqpoint{5.490039in}{5.490039in}}%
\pgfusepath{clip}%
\pgfsetbuttcap%
\pgfsetroundjoin%
\definecolor{currentfill}{rgb}{0.319809,0.770914,0.411152}%
\pgfsetfillcolor{currentfill}%
\pgfsetfillopacity{0.700000}%
\pgfsetlinewidth{0.000000pt}%
\definecolor{currentstroke}{rgb}{0.000000,0.000000,0.000000}%
\pgfsetstrokecolor{currentstroke}%
\pgfsetdash{}{0pt}%
\pgfpathmoveto{\pgfqpoint{3.485707in}{4.303787in}}%
\pgfpathlineto{\pgfqpoint{3.498544in}{4.282582in}}%
\pgfpathlineto{\pgfqpoint{3.511376in}{4.261587in}}%
\pgfpathlineto{\pgfqpoint{3.524203in}{4.240800in}}%
\pgfpathlineto{\pgfqpoint{3.537025in}{4.220220in}}%
\pgfpathlineto{\pgfqpoint{3.544469in}{4.238074in}}%
\pgfpathlineto{\pgfqpoint{3.551910in}{4.256117in}}%
\pgfpathlineto{\pgfqpoint{3.559346in}{4.274351in}}%
\pgfpathlineto{\pgfqpoint{3.566779in}{4.292778in}}%
\pgfpathlineto{\pgfqpoint{3.553965in}{4.313596in}}%
\pgfpathlineto{\pgfqpoint{3.541146in}{4.334622in}}%
\pgfpathlineto{\pgfqpoint{3.528322in}{4.355857in}}%
\pgfpathlineto{\pgfqpoint{3.515493in}{4.377303in}}%
\pgfpathlineto{\pgfqpoint{3.508053in}{4.358629in}}%
\pgfpathlineto{\pgfqpoint{3.500608in}{4.340154in}}%
\pgfpathlineto{\pgfqpoint{3.493160in}{4.321874in}}%
\pgfpathlineto{\pgfqpoint{3.485707in}{4.303787in}}%
\pgfpathclose%
\pgfusepath{fill}%
\end{pgfscope}%
\begin{pgfscope}%
\pgfpathrectangle{\pgfqpoint{1.254980in}{0.150000in}}{\pgfqpoint{5.490039in}{5.490039in}}%
\pgfusepath{clip}%
\pgfsetbuttcap%
\pgfsetroundjoin%
\definecolor{currentfill}{rgb}{0.183898,0.422383,0.556944}%
\pgfsetfillcolor{currentfill}%
\pgfsetfillopacity{0.700000}%
\pgfsetlinewidth{0.000000pt}%
\definecolor{currentstroke}{rgb}{0.000000,0.000000,0.000000}%
\pgfsetstrokecolor{currentstroke}%
\pgfsetdash{}{0pt}%
\pgfpathmoveto{\pgfqpoint{4.039549in}{3.382199in}}%
\pgfpathlineto{\pgfqpoint{4.052312in}{3.370082in}}%
\pgfpathlineto{\pgfqpoint{4.065076in}{3.358121in}}%
\pgfpathlineto{\pgfqpoint{4.077842in}{3.346315in}}%
\pgfpathlineto{\pgfqpoint{4.090609in}{3.334664in}}%
\pgfpathlineto{\pgfqpoint{4.097975in}{3.348795in}}%
\pgfpathlineto{\pgfqpoint{4.105338in}{3.363046in}}%
\pgfpathlineto{\pgfqpoint{4.112698in}{3.377420in}}%
\pgfpathlineto{\pgfqpoint{4.120054in}{3.391919in}}%
\pgfpathlineto{\pgfqpoint{4.107296in}{3.403764in}}%
\pgfpathlineto{\pgfqpoint{4.094540in}{3.415763in}}%
\pgfpathlineto{\pgfqpoint{4.081785in}{3.427918in}}%
\pgfpathlineto{\pgfqpoint{4.069031in}{3.440228in}}%
\pgfpathlineto{\pgfqpoint{4.061665in}{3.425530in}}%
\pgfpathlineto{\pgfqpoint{4.054296in}{3.410960in}}%
\pgfpathlineto{\pgfqpoint{4.046924in}{3.396517in}}%
\pgfpathlineto{\pgfqpoint{4.039549in}{3.382199in}}%
\pgfpathclose%
\pgfusepath{fill}%
\end{pgfscope}%
\begin{pgfscope}%
\pgfpathrectangle{\pgfqpoint{1.254980in}{0.150000in}}{\pgfqpoint{5.490039in}{5.490039in}}%
\pgfusepath{clip}%
\pgfsetbuttcap%
\pgfsetroundjoin%
\definecolor{currentfill}{rgb}{0.208030,0.718701,0.472873}%
\pgfsetfillcolor{currentfill}%
\pgfsetfillopacity{0.700000}%
\pgfsetlinewidth{0.000000pt}%
\definecolor{currentstroke}{rgb}{0.000000,0.000000,0.000000}%
\pgfsetstrokecolor{currentstroke}%
\pgfsetdash{}{0pt}%
\pgfpathmoveto{\pgfqpoint{3.507204in}{4.150644in}}%
\pgfpathlineto{\pgfqpoint{3.520030in}{4.130490in}}%
\pgfpathlineto{\pgfqpoint{3.532851in}{4.110541in}}%
\pgfpathlineto{\pgfqpoint{3.545668in}{4.090794in}}%
\pgfpathlineto{\pgfqpoint{3.558480in}{4.071248in}}%
\pgfpathlineto{\pgfqpoint{3.565932in}{4.088157in}}%
\pgfpathlineto{\pgfqpoint{3.573381in}{4.105241in}}%
\pgfpathlineto{\pgfqpoint{3.580825in}{4.122503in}}%
\pgfpathlineto{\pgfqpoint{3.588265in}{4.139944in}}%
\pgfpathlineto{\pgfqpoint{3.575462in}{4.159709in}}%
\pgfpathlineto{\pgfqpoint{3.562654in}{4.179676in}}%
\pgfpathlineto{\pgfqpoint{3.549842in}{4.199846in}}%
\pgfpathlineto{\pgfqpoint{3.537025in}{4.220220in}}%
\pgfpathlineto{\pgfqpoint{3.529576in}{4.202553in}}%
\pgfpathlineto{\pgfqpoint{3.522123in}{4.185069in}}%
\pgfpathlineto{\pgfqpoint{3.514666in}{4.167767in}}%
\pgfpathlineto{\pgfqpoint{3.507204in}{4.150644in}}%
\pgfpathclose%
\pgfusepath{fill}%
\end{pgfscope}%
\begin{pgfscope}%
\pgfpathrectangle{\pgfqpoint{1.254980in}{0.150000in}}{\pgfqpoint{5.490039in}{5.490039in}}%
\pgfusepath{clip}%
\pgfsetbuttcap%
\pgfsetroundjoin%
\definecolor{currentfill}{rgb}{0.123444,0.636809,0.528763}%
\pgfsetfillcolor{currentfill}%
\pgfsetfillopacity{0.700000}%
\pgfsetlinewidth{0.000000pt}%
\definecolor{currentstroke}{rgb}{0.000000,0.000000,0.000000}%
\pgfsetstrokecolor{currentstroke}%
\pgfsetdash{}{0pt}%
\pgfpathmoveto{\pgfqpoint{3.579872in}{3.929922in}}%
\pgfpathlineto{\pgfqpoint{3.592673in}{3.911562in}}%
\pgfpathlineto{\pgfqpoint{3.605472in}{3.893394in}}%
\pgfpathlineto{\pgfqpoint{3.618267in}{3.875418in}}%
\pgfpathlineto{\pgfqpoint{3.631058in}{3.857633in}}%
\pgfpathlineto{\pgfqpoint{3.638509in}{3.873475in}}%
\pgfpathlineto{\pgfqpoint{3.645956in}{3.889474in}}%
\pgfpathlineto{\pgfqpoint{3.653398in}{3.905634in}}%
\pgfpathlineto{\pgfqpoint{3.660837in}{3.921955in}}%
\pgfpathlineto{\pgfqpoint{3.648054in}{3.939941in}}%
\pgfpathlineto{\pgfqpoint{3.635269in}{3.958118in}}%
\pgfpathlineto{\pgfqpoint{3.622480in}{3.976486in}}%
\pgfpathlineto{\pgfqpoint{3.609687in}{3.995047in}}%
\pgfpathlineto{\pgfqpoint{3.602240in}{3.978519in}}%
\pgfpathlineto{\pgfqpoint{3.594788in}{3.962156in}}%
\pgfpathlineto{\pgfqpoint{3.587332in}{3.945958in}}%
\pgfpathlineto{\pgfqpoint{3.579872in}{3.929922in}}%
\pgfpathclose%
\pgfusepath{fill}%
\end{pgfscope}%
\begin{pgfscope}%
\pgfpathrectangle{\pgfqpoint{1.254980in}{0.150000in}}{\pgfqpoint{5.490039in}{5.490039in}}%
\pgfusepath{clip}%
\pgfsetbuttcap%
\pgfsetroundjoin%
\definecolor{currentfill}{rgb}{0.190631,0.407061,0.556089}%
\pgfsetfillcolor{currentfill}%
\pgfsetfillopacity{0.700000}%
\pgfsetlinewidth{0.000000pt}%
\definecolor{currentstroke}{rgb}{0.000000,0.000000,0.000000}%
\pgfsetstrokecolor{currentstroke}%
\pgfsetdash{}{0pt}%
\pgfpathmoveto{\pgfqpoint{4.090609in}{3.334664in}}%
\pgfpathlineto{\pgfqpoint{4.103378in}{3.323166in}}%
\pgfpathlineto{\pgfqpoint{4.116149in}{3.311821in}}%
\pgfpathlineto{\pgfqpoint{4.128922in}{3.300628in}}%
\pgfpathlineto{\pgfqpoint{4.141696in}{3.289587in}}%
\pgfpathlineto{\pgfqpoint{4.149052in}{3.303530in}}%
\pgfpathlineto{\pgfqpoint{4.156406in}{3.317590in}}%
\pgfpathlineto{\pgfqpoint{4.163756in}{3.331768in}}%
\pgfpathlineto{\pgfqpoint{4.171103in}{3.346068in}}%
\pgfpathlineto{\pgfqpoint{4.158338in}{3.357303in}}%
\pgfpathlineto{\pgfqpoint{4.145575in}{3.368689in}}%
\pgfpathlineto{\pgfqpoint{4.132814in}{3.380227in}}%
\pgfpathlineto{\pgfqpoint{4.120054in}{3.391919in}}%
\pgfpathlineto{\pgfqpoint{4.112698in}{3.377420in}}%
\pgfpathlineto{\pgfqpoint{4.105338in}{3.363046in}}%
\pgfpathlineto{\pgfqpoint{4.097975in}{3.348795in}}%
\pgfpathlineto{\pgfqpoint{4.090609in}{3.334664in}}%
\pgfpathclose%
\pgfusepath{fill}%
\end{pgfscope}%
\begin{pgfscope}%
\pgfpathrectangle{\pgfqpoint{1.254980in}{0.150000in}}{\pgfqpoint{5.490039in}{5.490039in}}%
\pgfusepath{clip}%
\pgfsetbuttcap%
\pgfsetroundjoin%
\definecolor{currentfill}{rgb}{0.206756,0.371758,0.553117}%
\pgfsetfillcolor{currentfill}%
\pgfsetfillopacity{0.700000}%
\pgfsetlinewidth{0.000000pt}%
\definecolor{currentstroke}{rgb}{0.000000,0.000000,0.000000}%
\pgfsetstrokecolor{currentstroke}%
\pgfsetdash{}{0pt}%
\pgfpathmoveto{\pgfqpoint{4.404904in}{3.241443in}}%
\pgfpathlineto{\pgfqpoint{4.417709in}{3.232469in}}%
\pgfpathlineto{\pgfqpoint{4.430518in}{3.223634in}}%
\pgfpathlineto{\pgfqpoint{4.443330in}{3.214939in}}%
\pgfpathlineto{\pgfqpoint{4.456147in}{3.206381in}}%
\pgfpathlineto{\pgfqpoint{4.463433in}{3.220227in}}%
\pgfpathlineto{\pgfqpoint{4.470716in}{3.234187in}}%
\pgfpathlineto{\pgfqpoint{4.477998in}{3.248264in}}%
\pgfpathlineto{\pgfqpoint{4.485277in}{3.262459in}}%
\pgfpathlineto{\pgfqpoint{4.472470in}{3.271240in}}%
\pgfpathlineto{\pgfqpoint{4.459667in}{3.280159in}}%
\pgfpathlineto{\pgfqpoint{4.446868in}{3.289217in}}%
\pgfpathlineto{\pgfqpoint{4.434073in}{3.298415in}}%
\pgfpathlineto{\pgfqpoint{4.426784in}{3.283990in}}%
\pgfpathlineto{\pgfqpoint{4.419493in}{3.269688in}}%
\pgfpathlineto{\pgfqpoint{4.412200in}{3.255507in}}%
\pgfpathlineto{\pgfqpoint{4.404904in}{3.241443in}}%
\pgfpathclose%
\pgfusepath{fill}%
\end{pgfscope}%
\begin{pgfscope}%
\pgfpathrectangle{\pgfqpoint{1.254980in}{0.150000in}}{\pgfqpoint{5.490039in}{5.490039in}}%
\pgfusepath{clip}%
\pgfsetbuttcap%
\pgfsetroundjoin%
\definecolor{currentfill}{rgb}{0.203063,0.379716,0.553925}%
\pgfsetfillcolor{currentfill}%
\pgfsetfillopacity{0.700000}%
\pgfsetlinewidth{0.000000pt}%
\definecolor{currentstroke}{rgb}{0.000000,0.000000,0.000000}%
\pgfsetstrokecolor{currentstroke}%
\pgfsetdash{}{0pt}%
\pgfpathmoveto{\pgfqpoint{4.273306in}{3.261553in}}%
\pgfpathlineto{\pgfqpoint{4.286093in}{3.251649in}}%
\pgfpathlineto{\pgfqpoint{4.298884in}{3.241889in}}%
\pgfpathlineto{\pgfqpoint{4.311677in}{3.232273in}}%
\pgfpathlineto{\pgfqpoint{4.324474in}{3.222800in}}%
\pgfpathlineto{\pgfqpoint{4.331790in}{3.236615in}}%
\pgfpathlineto{\pgfqpoint{4.339103in}{3.250543in}}%
\pgfpathlineto{\pgfqpoint{4.346413in}{3.264586in}}%
\pgfpathlineto{\pgfqpoint{4.353721in}{3.278748in}}%
\pgfpathlineto{\pgfqpoint{4.340934in}{3.288429in}}%
\pgfpathlineto{\pgfqpoint{4.328150in}{3.298253in}}%
\pgfpathlineto{\pgfqpoint{4.315369in}{3.308221in}}%
\pgfpathlineto{\pgfqpoint{4.302591in}{3.318333in}}%
\pgfpathlineto{\pgfqpoint{4.295274in}{3.303958in}}%
\pgfpathlineto{\pgfqpoint{4.287954in}{3.289705in}}%
\pgfpathlineto{\pgfqpoint{4.280631in}{3.275570in}}%
\pgfpathlineto{\pgfqpoint{4.273306in}{3.261553in}}%
\pgfpathclose%
\pgfusepath{fill}%
\end{pgfscope}%
\begin{pgfscope}%
\pgfpathrectangle{\pgfqpoint{1.254980in}{0.150000in}}{\pgfqpoint{5.490039in}{5.490039in}}%
\pgfusepath{clip}%
\pgfsetbuttcap%
\pgfsetroundjoin%
\definecolor{currentfill}{rgb}{0.149039,0.508051,0.557250}%
\pgfsetfillcolor{currentfill}%
\pgfsetfillopacity{0.700000}%
\pgfsetlinewidth{0.000000pt}%
\definecolor{currentstroke}{rgb}{0.000000,0.000000,0.000000}%
\pgfsetstrokecolor{currentstroke}%
\pgfsetdash{}{0pt}%
\pgfpathmoveto{\pgfqpoint{3.754636in}{3.598956in}}%
\pgfpathlineto{\pgfqpoint{3.767410in}{3.583717in}}%
\pgfpathlineto{\pgfqpoint{3.780181in}{3.568652in}}%
\pgfpathlineto{\pgfqpoint{3.792952in}{3.553760in}}%
\pgfpathlineto{\pgfqpoint{3.805722in}{3.539039in}}%
\pgfpathlineto{\pgfqpoint{3.813150in}{3.553570in}}%
\pgfpathlineto{\pgfqpoint{3.820574in}{3.568233in}}%
\pgfpathlineto{\pgfqpoint{3.827995in}{3.583032in}}%
\pgfpathlineto{\pgfqpoint{3.835412in}{3.597967in}}%
\pgfpathlineto{\pgfqpoint{3.822651in}{3.612867in}}%
\pgfpathlineto{\pgfqpoint{3.809890in}{3.627940in}}%
\pgfpathlineto{\pgfqpoint{3.797128in}{3.643185in}}%
\pgfpathlineto{\pgfqpoint{3.784364in}{3.658605in}}%
\pgfpathlineto{\pgfqpoint{3.776938in}{3.643483in}}%
\pgfpathlineto{\pgfqpoint{3.769508in}{3.628503in}}%
\pgfpathlineto{\pgfqpoint{3.762074in}{3.613661in}}%
\pgfpathlineto{\pgfqpoint{3.754636in}{3.598956in}}%
\pgfpathclose%
\pgfusepath{fill}%
\end{pgfscope}%
\begin{pgfscope}%
\pgfpathrectangle{\pgfqpoint{1.254980in}{0.150000in}}{\pgfqpoint{5.490039in}{5.490039in}}%
\pgfusepath{clip}%
\pgfsetbuttcap%
\pgfsetroundjoin%
\definecolor{currentfill}{rgb}{0.157729,0.485932,0.558013}%
\pgfsetfillcolor{currentfill}%
\pgfsetfillopacity{0.700000}%
\pgfsetlinewidth{0.000000pt}%
\definecolor{currentstroke}{rgb}{0.000000,0.000000,0.000000}%
\pgfsetstrokecolor{currentstroke}%
\pgfsetdash{}{0pt}%
\pgfpathmoveto{\pgfqpoint{3.805722in}{3.539039in}}%
\pgfpathlineto{\pgfqpoint{3.818491in}{3.524490in}}%
\pgfpathlineto{\pgfqpoint{3.831259in}{3.510111in}}%
\pgfpathlineto{\pgfqpoint{3.844027in}{3.495901in}}%
\pgfpathlineto{\pgfqpoint{3.856795in}{3.481859in}}%
\pgfpathlineto{\pgfqpoint{3.864213in}{3.496216in}}%
\pgfpathlineto{\pgfqpoint{3.871627in}{3.510702in}}%
\pgfpathlineto{\pgfqpoint{3.879038in}{3.525318in}}%
\pgfpathlineto{\pgfqpoint{3.886445in}{3.540068in}}%
\pgfpathlineto{\pgfqpoint{3.873688in}{3.554289in}}%
\pgfpathlineto{\pgfqpoint{3.860930in}{3.568679in}}%
\pgfpathlineto{\pgfqpoint{3.848171in}{3.583238in}}%
\pgfpathlineto{\pgfqpoint{3.835412in}{3.597967in}}%
\pgfpathlineto{\pgfqpoint{3.827995in}{3.583032in}}%
\pgfpathlineto{\pgfqpoint{3.820574in}{3.568233in}}%
\pgfpathlineto{\pgfqpoint{3.813150in}{3.553570in}}%
\pgfpathlineto{\pgfqpoint{3.805722in}{3.539039in}}%
\pgfpathclose%
\pgfusepath{fill}%
\end{pgfscope}%
\begin{pgfscope}%
\pgfpathrectangle{\pgfqpoint{1.254980in}{0.150000in}}{\pgfqpoint{5.490039in}{5.490039in}}%
\pgfusepath{clip}%
\pgfsetbuttcap%
\pgfsetroundjoin%
\definecolor{currentfill}{rgb}{0.139147,0.533812,0.555298}%
\pgfsetfillcolor{currentfill}%
\pgfsetfillopacity{0.700000}%
\pgfsetlinewidth{0.000000pt}%
\definecolor{currentstroke}{rgb}{0.000000,0.000000,0.000000}%
\pgfsetstrokecolor{currentstroke}%
\pgfsetdash{}{0pt}%
\pgfpathmoveto{\pgfqpoint{3.703529in}{3.661676in}}%
\pgfpathlineto{\pgfqpoint{3.716308in}{3.645729in}}%
\pgfpathlineto{\pgfqpoint{3.729086in}{3.629961in}}%
\pgfpathlineto{\pgfqpoint{3.741862in}{3.614371in}}%
\pgfpathlineto{\pgfqpoint{3.754636in}{3.598956in}}%
\pgfpathlineto{\pgfqpoint{3.762074in}{3.613661in}}%
\pgfpathlineto{\pgfqpoint{3.769508in}{3.628503in}}%
\pgfpathlineto{\pgfqpoint{3.776938in}{3.643483in}}%
\pgfpathlineto{\pgfqpoint{3.784364in}{3.658605in}}%
\pgfpathlineto{\pgfqpoint{3.771600in}{3.674201in}}%
\pgfpathlineto{\pgfqpoint{3.758833in}{3.689972in}}%
\pgfpathlineto{\pgfqpoint{3.746065in}{3.705921in}}%
\pgfpathlineto{\pgfqpoint{3.733296in}{3.722049in}}%
\pgfpathlineto{\pgfqpoint{3.725860in}{3.706740in}}%
\pgfpathlineto{\pgfqpoint{3.718420in}{3.691576in}}%
\pgfpathlineto{\pgfqpoint{3.710977in}{3.676555in}}%
\pgfpathlineto{\pgfqpoint{3.703529in}{3.661676in}}%
\pgfpathclose%
\pgfusepath{fill}%
\end{pgfscope}%
\begin{pgfscope}%
\pgfpathrectangle{\pgfqpoint{1.254980in}{0.150000in}}{\pgfqpoint{5.490039in}{5.490039in}}%
\pgfusepath{clip}%
\pgfsetbuttcap%
\pgfsetroundjoin%
\definecolor{currentfill}{rgb}{0.166617,0.463708,0.558119}%
\pgfsetfillcolor{currentfill}%
\pgfsetfillopacity{0.700000}%
\pgfsetlinewidth{0.000000pt}%
\definecolor{currentstroke}{rgb}{0.000000,0.000000,0.000000}%
\pgfsetstrokecolor{currentstroke}%
\pgfsetdash{}{0pt}%
\pgfpathmoveto{\pgfqpoint{3.856795in}{3.481859in}}%
\pgfpathlineto{\pgfqpoint{3.869562in}{3.467985in}}%
\pgfpathlineto{\pgfqpoint{3.882328in}{3.454276in}}%
\pgfpathlineto{\pgfqpoint{3.895095in}{3.440733in}}%
\pgfpathlineto{\pgfqpoint{3.907862in}{3.427354in}}%
\pgfpathlineto{\pgfqpoint{3.915270in}{3.441538in}}%
\pgfpathlineto{\pgfqpoint{3.922675in}{3.455846in}}%
\pgfpathlineto{\pgfqpoint{3.930077in}{3.470282in}}%
\pgfpathlineto{\pgfqpoint{3.937474in}{3.484846in}}%
\pgfpathlineto{\pgfqpoint{3.924717in}{3.498404in}}%
\pgfpathlineto{\pgfqpoint{3.911960in}{3.512126in}}%
\pgfpathlineto{\pgfqpoint{3.899203in}{3.526014in}}%
\pgfpathlineto{\pgfqpoint{3.886445in}{3.540068in}}%
\pgfpathlineto{\pgfqpoint{3.879038in}{3.525318in}}%
\pgfpathlineto{\pgfqpoint{3.871627in}{3.510702in}}%
\pgfpathlineto{\pgfqpoint{3.864213in}{3.496216in}}%
\pgfpathlineto{\pgfqpoint{3.856795in}{3.481859in}}%
\pgfpathclose%
\pgfusepath{fill}%
\end{pgfscope}%
\begin{pgfscope}%
\pgfpathrectangle{\pgfqpoint{1.254980in}{0.150000in}}{\pgfqpoint{5.490039in}{5.490039in}}%
\pgfusepath{clip}%
\pgfsetbuttcap%
\pgfsetroundjoin%
\definecolor{currentfill}{rgb}{0.140210,0.665859,0.513427}%
\pgfsetfillcolor{currentfill}%
\pgfsetfillopacity{0.700000}%
\pgfsetlinewidth{0.000000pt}%
\definecolor{currentstroke}{rgb}{0.000000,0.000000,0.000000}%
\pgfsetstrokecolor{currentstroke}%
\pgfsetdash{}{0pt}%
\pgfpathmoveto{\pgfqpoint{3.528627in}{4.005316in}}%
\pgfpathlineto{\pgfqpoint{3.541444in}{3.986172in}}%
\pgfpathlineto{\pgfqpoint{3.554257in}{3.967226in}}%
\pgfpathlineto{\pgfqpoint{3.567066in}{3.948476in}}%
\pgfpathlineto{\pgfqpoint{3.579872in}{3.929922in}}%
\pgfpathlineto{\pgfqpoint{3.587332in}{3.945958in}}%
\pgfpathlineto{\pgfqpoint{3.594788in}{3.962156in}}%
\pgfpathlineto{\pgfqpoint{3.602240in}{3.978519in}}%
\pgfpathlineto{\pgfqpoint{3.609687in}{3.995047in}}%
\pgfpathlineto{\pgfqpoint{3.596891in}{4.013802in}}%
\pgfpathlineto{\pgfqpoint{3.584091in}{4.032754in}}%
\pgfpathlineto{\pgfqpoint{3.571287in}{4.051902in}}%
\pgfpathlineto{\pgfqpoint{3.558480in}{4.071248in}}%
\pgfpathlineto{\pgfqpoint{3.551023in}{4.054511in}}%
\pgfpathlineto{\pgfqpoint{3.543562in}{4.037945in}}%
\pgfpathlineto{\pgfqpoint{3.536097in}{4.021548in}}%
\pgfpathlineto{\pgfqpoint{3.528627in}{4.005316in}}%
\pgfpathclose%
\pgfusepath{fill}%
\end{pgfscope}%
\begin{pgfscope}%
\pgfpathrectangle{\pgfqpoint{1.254980in}{0.150000in}}{\pgfqpoint{5.490039in}{5.490039in}}%
\pgfusepath{clip}%
\pgfsetbuttcap%
\pgfsetroundjoin%
\definecolor{currentfill}{rgb}{0.210503,0.363727,0.552206}%
\pgfsetfillcolor{currentfill}%
\pgfsetfillopacity{0.700000}%
\pgfsetlinewidth{0.000000pt}%
\definecolor{currentstroke}{rgb}{0.000000,0.000000,0.000000}%
\pgfsetstrokecolor{currentstroke}%
\pgfsetdash{}{0pt}%
\pgfpathmoveto{\pgfqpoint{4.536546in}{3.228710in}}%
\pgfpathlineto{\pgfqpoint{4.549375in}{3.220613in}}%
\pgfpathlineto{\pgfqpoint{4.562208in}{3.212651in}}%
\pgfpathlineto{\pgfqpoint{4.575045in}{3.204824in}}%
\pgfpathlineto{\pgfqpoint{4.587888in}{3.197131in}}%
\pgfpathlineto{\pgfqpoint{4.595145in}{3.210982in}}%
\pgfpathlineto{\pgfqpoint{4.602400in}{3.224949in}}%
\pgfpathlineto{\pgfqpoint{4.609653in}{3.239034in}}%
\pgfpathlineto{\pgfqpoint{4.596818in}{3.246905in}}%
\pgfpathlineto{\pgfqpoint{4.583988in}{3.254909in}}%
\pgfpathlineto{\pgfqpoint{4.571162in}{3.263048in}}%
\pgfpathlineto{\pgfqpoint{4.558341in}{3.271323in}}%
\pgfpathlineto{\pgfqpoint{4.551078in}{3.256997in}}%
\pgfpathlineto{\pgfqpoint{4.543813in}{3.242794in}}%
\pgfpathlineto{\pgfqpoint{4.536546in}{3.228710in}}%
\pgfpathclose%
\pgfusepath{fill}%
\end{pgfscope}%
\begin{pgfscope}%
\pgfpathrectangle{\pgfqpoint{1.254980in}{0.150000in}}{\pgfqpoint{5.490039in}{5.490039in}}%
\pgfusepath{clip}%
\pgfsetbuttcap%
\pgfsetroundjoin%
\definecolor{currentfill}{rgb}{0.129933,0.559582,0.551864}%
\pgfsetfillcolor{currentfill}%
\pgfsetfillopacity{0.700000}%
\pgfsetlinewidth{0.000000pt}%
\definecolor{currentstroke}{rgb}{0.000000,0.000000,0.000000}%
\pgfsetstrokecolor{currentstroke}%
\pgfsetdash{}{0pt}%
\pgfpathmoveto{\pgfqpoint{3.652391in}{3.727266in}}%
\pgfpathlineto{\pgfqpoint{3.665179in}{3.710595in}}%
\pgfpathlineto{\pgfqpoint{3.677964in}{3.694108in}}%
\pgfpathlineto{\pgfqpoint{3.690748in}{3.677801in}}%
\pgfpathlineto{\pgfqpoint{3.703529in}{3.661676in}}%
\pgfpathlineto{\pgfqpoint{3.710977in}{3.676555in}}%
\pgfpathlineto{\pgfqpoint{3.718420in}{3.691576in}}%
\pgfpathlineto{\pgfqpoint{3.725860in}{3.706740in}}%
\pgfpathlineto{\pgfqpoint{3.733296in}{3.722049in}}%
\pgfpathlineto{\pgfqpoint{3.720524in}{3.738356in}}%
\pgfpathlineto{\pgfqpoint{3.707751in}{3.754844in}}%
\pgfpathlineto{\pgfqpoint{3.694975in}{3.771514in}}%
\pgfpathlineto{\pgfqpoint{3.682197in}{3.788367in}}%
\pgfpathlineto{\pgfqpoint{3.674751in}{3.772870in}}%
\pgfpathlineto{\pgfqpoint{3.667302in}{3.757522in}}%
\pgfpathlineto{\pgfqpoint{3.659849in}{3.742321in}}%
\pgfpathlineto{\pgfqpoint{3.652391in}{3.727266in}}%
\pgfpathclose%
\pgfusepath{fill}%
\end{pgfscope}%
\begin{pgfscope}%
\pgfpathrectangle{\pgfqpoint{1.254980in}{0.150000in}}{\pgfqpoint{5.490039in}{5.490039in}}%
\pgfusepath{clip}%
\pgfsetbuttcap%
\pgfsetroundjoin%
\definecolor{currentfill}{rgb}{0.174274,0.445044,0.557792}%
\pgfsetfillcolor{currentfill}%
\pgfsetfillopacity{0.700000}%
\pgfsetlinewidth{0.000000pt}%
\definecolor{currentstroke}{rgb}{0.000000,0.000000,0.000000}%
\pgfsetstrokecolor{currentstroke}%
\pgfsetdash{}{0pt}%
\pgfpathmoveto{\pgfqpoint{3.907862in}{3.427354in}}%
\pgfpathlineto{\pgfqpoint{3.920629in}{3.414139in}}%
\pgfpathlineto{\pgfqpoint{3.933396in}{3.401086in}}%
\pgfpathlineto{\pgfqpoint{3.946164in}{3.388195in}}%
\pgfpathlineto{\pgfqpoint{3.958932in}{3.375464in}}%
\pgfpathlineto{\pgfqpoint{3.966331in}{3.389475in}}%
\pgfpathlineto{\pgfqpoint{3.973726in}{3.403607in}}%
\pgfpathlineto{\pgfqpoint{3.981118in}{3.417862in}}%
\pgfpathlineto{\pgfqpoint{3.988506in}{3.432242in}}%
\pgfpathlineto{\pgfqpoint{3.975747in}{3.445150in}}%
\pgfpathlineto{\pgfqpoint{3.962989in}{3.458220in}}%
\pgfpathlineto{\pgfqpoint{3.950232in}{3.471452in}}%
\pgfpathlineto{\pgfqpoint{3.937474in}{3.484846in}}%
\pgfpathlineto{\pgfqpoint{3.930077in}{3.470282in}}%
\pgfpathlineto{\pgfqpoint{3.922675in}{3.455846in}}%
\pgfpathlineto{\pgfqpoint{3.915270in}{3.441538in}}%
\pgfpathlineto{\pgfqpoint{3.907862in}{3.427354in}}%
\pgfpathclose%
\pgfusepath{fill}%
\end{pgfscope}%
\begin{pgfscope}%
\pgfpathrectangle{\pgfqpoint{1.254980in}{0.150000in}}{\pgfqpoint{5.490039in}{5.490039in}}%
\pgfusepath{clip}%
\pgfsetbuttcap%
\pgfsetroundjoin%
\definecolor{currentfill}{rgb}{0.197636,0.391528,0.554969}%
\pgfsetfillcolor{currentfill}%
\pgfsetfillopacity{0.700000}%
\pgfsetlinewidth{0.000000pt}%
\definecolor{currentstroke}{rgb}{0.000000,0.000000,0.000000}%
\pgfsetstrokecolor{currentstroke}%
\pgfsetdash{}{0pt}%
\pgfpathmoveto{\pgfqpoint{4.141696in}{3.289587in}}%
\pgfpathlineto{\pgfqpoint{4.154473in}{3.278696in}}%
\pgfpathlineto{\pgfqpoint{4.167251in}{3.267954in}}%
\pgfpathlineto{\pgfqpoint{4.180032in}{3.257362in}}%
\pgfpathlineto{\pgfqpoint{4.192816in}{3.246919in}}%
\pgfpathlineto{\pgfqpoint{4.200163in}{3.260675in}}%
\pgfpathlineto{\pgfqpoint{4.207506in}{3.274544in}}%
\pgfpathlineto{\pgfqpoint{4.214847in}{3.288527in}}%
\pgfpathlineto{\pgfqpoint{4.222185in}{3.302628in}}%
\pgfpathlineto{\pgfqpoint{4.209411in}{3.313265in}}%
\pgfpathlineto{\pgfqpoint{4.196639in}{3.324050in}}%
\pgfpathlineto{\pgfqpoint{4.183870in}{3.334984in}}%
\pgfpathlineto{\pgfqpoint{4.171103in}{3.346068in}}%
\pgfpathlineto{\pgfqpoint{4.163756in}{3.331768in}}%
\pgfpathlineto{\pgfqpoint{4.156406in}{3.317590in}}%
\pgfpathlineto{\pgfqpoint{4.149052in}{3.303530in}}%
\pgfpathlineto{\pgfqpoint{4.141696in}{3.289587in}}%
\pgfpathclose%
\pgfusepath{fill}%
\end{pgfscope}%
\begin{pgfscope}%
\pgfpathrectangle{\pgfqpoint{1.254980in}{0.150000in}}{\pgfqpoint{5.490039in}{5.490039in}}%
\pgfusepath{clip}%
\pgfsetbuttcap%
\pgfsetroundjoin%
\definecolor{currentfill}{rgb}{0.266941,0.748751,0.440573}%
\pgfsetfillcolor{currentfill}%
\pgfsetfillopacity{0.700000}%
\pgfsetlinewidth{0.000000pt}%
\definecolor{currentstroke}{rgb}{0.000000,0.000000,0.000000}%
\pgfsetstrokecolor{currentstroke}%
\pgfsetdash{}{0pt}%
\pgfpathmoveto{\pgfqpoint{3.455852in}{4.233324in}}%
\pgfpathlineto{\pgfqpoint{3.468698in}{4.212341in}}%
\pgfpathlineto{\pgfqpoint{3.481538in}{4.191567in}}%
\pgfpathlineto{\pgfqpoint{3.494374in}{4.171002in}}%
\pgfpathlineto{\pgfqpoint{3.507204in}{4.150644in}}%
\pgfpathlineto{\pgfqpoint{3.514666in}{4.167767in}}%
\pgfpathlineto{\pgfqpoint{3.522123in}{4.185069in}}%
\pgfpathlineto{\pgfqpoint{3.529576in}{4.202553in}}%
\pgfpathlineto{\pgfqpoint{3.537025in}{4.220220in}}%
\pgfpathlineto{\pgfqpoint{3.524203in}{4.240800in}}%
\pgfpathlineto{\pgfqpoint{3.511376in}{4.261587in}}%
\pgfpathlineto{\pgfqpoint{3.498544in}{4.282582in}}%
\pgfpathlineto{\pgfqpoint{3.485707in}{4.303787in}}%
\pgfpathlineto{\pgfqpoint{3.478250in}{4.285891in}}%
\pgfpathlineto{\pgfqpoint{3.470788in}{4.268184in}}%
\pgfpathlineto{\pgfqpoint{3.463322in}{4.250662in}}%
\pgfpathlineto{\pgfqpoint{3.455852in}{4.233324in}}%
\pgfpathclose%
\pgfusepath{fill}%
\end{pgfscope}%
\begin{pgfscope}%
\pgfpathrectangle{\pgfqpoint{1.254980in}{0.150000in}}{\pgfqpoint{5.490039in}{5.490039in}}%
\pgfusepath{clip}%
\pgfsetbuttcap%
\pgfsetroundjoin%
\definecolor{currentfill}{rgb}{0.182256,0.426184,0.557120}%
\pgfsetfillcolor{currentfill}%
\pgfsetfillopacity{0.700000}%
\pgfsetlinewidth{0.000000pt}%
\definecolor{currentstroke}{rgb}{0.000000,0.000000,0.000000}%
\pgfsetstrokecolor{currentstroke}%
\pgfsetdash{}{0pt}%
\pgfpathmoveto{\pgfqpoint{3.958932in}{3.375464in}}%
\pgfpathlineto{\pgfqpoint{3.971701in}{3.362894in}}%
\pgfpathlineto{\pgfqpoint{3.984471in}{3.350483in}}%
\pgfpathlineto{\pgfqpoint{3.997242in}{3.338230in}}%
\pgfpathlineto{\pgfqpoint{4.010014in}{3.326134in}}%
\pgfpathlineto{\pgfqpoint{4.017402in}{3.339973in}}%
\pgfpathlineto{\pgfqpoint{4.024788in}{3.353929in}}%
\pgfpathlineto{\pgfqpoint{4.032170in}{3.368003in}}%
\pgfpathlineto{\pgfqpoint{4.039549in}{3.382199in}}%
\pgfpathlineto{\pgfqpoint{4.026787in}{3.394472in}}%
\pgfpathlineto{\pgfqpoint{4.014026in}{3.406903in}}%
\pgfpathlineto{\pgfqpoint{4.001265in}{3.419493in}}%
\pgfpathlineto{\pgfqpoint{3.988506in}{3.432242in}}%
\pgfpathlineto{\pgfqpoint{3.981118in}{3.417862in}}%
\pgfpathlineto{\pgfqpoint{3.973726in}{3.403607in}}%
\pgfpathlineto{\pgfqpoint{3.966331in}{3.389475in}}%
\pgfpathlineto{\pgfqpoint{3.958932in}{3.375464in}}%
\pgfpathclose%
\pgfusepath{fill}%
\end{pgfscope}%
\begin{pgfscope}%
\pgfpathrectangle{\pgfqpoint{1.254980in}{0.150000in}}{\pgfqpoint{5.490039in}{5.490039in}}%
\pgfusepath{clip}%
\pgfsetbuttcap%
\pgfsetroundjoin%
\definecolor{currentfill}{rgb}{0.121831,0.589055,0.545623}%
\pgfsetfillcolor{currentfill}%
\pgfsetfillopacity{0.700000}%
\pgfsetlinewidth{0.000000pt}%
\definecolor{currentstroke}{rgb}{0.000000,0.000000,0.000000}%
\pgfsetstrokecolor{currentstroke}%
\pgfsetdash{}{0pt}%
\pgfpathmoveto{\pgfqpoint{3.601213in}{3.795801in}}%
\pgfpathlineto{\pgfqpoint{3.614012in}{3.778387in}}%
\pgfpathlineto{\pgfqpoint{3.626808in}{3.761161in}}%
\pgfpathlineto{\pgfqpoint{3.639601in}{3.744121in}}%
\pgfpathlineto{\pgfqpoint{3.652391in}{3.727266in}}%
\pgfpathlineto{\pgfqpoint{3.659849in}{3.742321in}}%
\pgfpathlineto{\pgfqpoint{3.667302in}{3.757522in}}%
\pgfpathlineto{\pgfqpoint{3.674751in}{3.772870in}}%
\pgfpathlineto{\pgfqpoint{3.682197in}{3.788367in}}%
\pgfpathlineto{\pgfqpoint{3.669416in}{3.805404in}}%
\pgfpathlineto{\pgfqpoint{3.656633in}{3.822627in}}%
\pgfpathlineto{\pgfqpoint{3.643847in}{3.840036in}}%
\pgfpathlineto{\pgfqpoint{3.631058in}{3.857633in}}%
\pgfpathlineto{\pgfqpoint{3.623604in}{3.841946in}}%
\pgfpathlineto{\pgfqpoint{3.616144in}{3.826413in}}%
\pgfpathlineto{\pgfqpoint{3.608681in}{3.811032in}}%
\pgfpathlineto{\pgfqpoint{3.601213in}{3.795801in}}%
\pgfpathclose%
\pgfusepath{fill}%
\end{pgfscope}%
\begin{pgfscope}%
\pgfpathrectangle{\pgfqpoint{1.254980in}{0.150000in}}{\pgfqpoint{5.490039in}{5.490039in}}%
\pgfusepath{clip}%
\pgfsetbuttcap%
\pgfsetroundjoin%
\definecolor{currentfill}{rgb}{0.210503,0.363727,0.552206}%
\pgfsetfillcolor{currentfill}%
\pgfsetfillopacity{0.700000}%
\pgfsetlinewidth{0.000000pt}%
\definecolor{currentstroke}{rgb}{0.000000,0.000000,0.000000}%
\pgfsetstrokecolor{currentstroke}%
\pgfsetdash{}{0pt}%
\pgfpathmoveto{\pgfqpoint{4.324474in}{3.222800in}}%
\pgfpathlineto{\pgfqpoint{4.337274in}{3.213470in}}%
\pgfpathlineto{\pgfqpoint{4.350078in}{3.204281in}}%
\pgfpathlineto{\pgfqpoint{4.362885in}{3.195234in}}%
\pgfpathlineto{\pgfqpoint{4.375696in}{3.186327in}}%
\pgfpathlineto{\pgfqpoint{4.383002in}{3.199940in}}%
\pgfpathlineto{\pgfqpoint{4.390305in}{3.213662in}}%
\pgfpathlineto{\pgfqpoint{4.397606in}{3.227496in}}%
\pgfpathlineto{\pgfqpoint{4.404904in}{3.241443in}}%
\pgfpathlineto{\pgfqpoint{4.392103in}{3.250558in}}%
\pgfpathlineto{\pgfqpoint{4.379306in}{3.259813in}}%
\pgfpathlineto{\pgfqpoint{4.366512in}{3.269209in}}%
\pgfpathlineto{\pgfqpoint{4.353721in}{3.278748in}}%
\pgfpathlineto{\pgfqpoint{4.346413in}{3.264586in}}%
\pgfpathlineto{\pgfqpoint{4.339103in}{3.250543in}}%
\pgfpathlineto{\pgfqpoint{4.331790in}{3.236615in}}%
\pgfpathlineto{\pgfqpoint{4.324474in}{3.222800in}}%
\pgfpathclose%
\pgfusepath{fill}%
\end{pgfscope}%
\begin{pgfscope}%
\pgfpathrectangle{\pgfqpoint{1.254980in}{0.150000in}}{\pgfqpoint{5.490039in}{5.490039in}}%
\pgfusepath{clip}%
\pgfsetbuttcap%
\pgfsetroundjoin%
\definecolor{currentfill}{rgb}{0.212395,0.359683,0.551710}%
\pgfsetfillcolor{currentfill}%
\pgfsetfillopacity{0.700000}%
\pgfsetlinewidth{0.000000pt}%
\definecolor{currentstroke}{rgb}{0.000000,0.000000,0.000000}%
\pgfsetstrokecolor{currentstroke}%
\pgfsetdash{}{0pt}%
\pgfpathmoveto{\pgfqpoint{4.456147in}{3.206381in}}%
\pgfpathlineto{\pgfqpoint{4.468968in}{3.197962in}}%
\pgfpathlineto{\pgfqpoint{4.481793in}{3.189680in}}%
\pgfpathlineto{\pgfqpoint{4.494622in}{3.181534in}}%
\pgfpathlineto{\pgfqpoint{4.507456in}{3.173525in}}%
\pgfpathlineto{\pgfqpoint{4.514732in}{3.187153in}}%
\pgfpathlineto{\pgfqpoint{4.522006in}{3.200892in}}%
\pgfpathlineto{\pgfqpoint{4.529277in}{3.214744in}}%
\pgfpathlineto{\pgfqpoint{4.536546in}{3.228710in}}%
\pgfpathlineto{\pgfqpoint{4.523722in}{3.236942in}}%
\pgfpathlineto{\pgfqpoint{4.510903in}{3.245311in}}%
\pgfpathlineto{\pgfqpoint{4.498088in}{3.253816in}}%
\pgfpathlineto{\pgfqpoint{4.485277in}{3.262459in}}%
\pgfpathlineto{\pgfqpoint{4.477998in}{3.248264in}}%
\pgfpathlineto{\pgfqpoint{4.470716in}{3.234187in}}%
\pgfpathlineto{\pgfqpoint{4.463433in}{3.220227in}}%
\pgfpathlineto{\pgfqpoint{4.456147in}{3.206381in}}%
\pgfpathclose%
\pgfusepath{fill}%
\end{pgfscope}%
\begin{pgfscope}%
\pgfpathrectangle{\pgfqpoint{1.254980in}{0.150000in}}{\pgfqpoint{5.490039in}{5.490039in}}%
\pgfusepath{clip}%
\pgfsetbuttcap%
\pgfsetroundjoin%
\definecolor{currentfill}{rgb}{0.175707,0.697900,0.491033}%
\pgfsetfillcolor{currentfill}%
\pgfsetfillopacity{0.700000}%
\pgfsetlinewidth{0.000000pt}%
\definecolor{currentstroke}{rgb}{0.000000,0.000000,0.000000}%
\pgfsetstrokecolor{currentstroke}%
\pgfsetdash{}{0pt}%
\pgfpathmoveto{\pgfqpoint{3.477314in}{4.083900in}}%
\pgfpathlineto{\pgfqpoint{3.490149in}{4.063950in}}%
\pgfpathlineto{\pgfqpoint{3.502980in}{4.044204in}}%
\pgfpathlineto{\pgfqpoint{3.515805in}{4.024660in}}%
\pgfpathlineto{\pgfqpoint{3.528627in}{4.005316in}}%
\pgfpathlineto{\pgfqpoint{3.536097in}{4.021548in}}%
\pgfpathlineto{\pgfqpoint{3.543562in}{4.037945in}}%
\pgfpathlineto{\pgfqpoint{3.551023in}{4.054511in}}%
\pgfpathlineto{\pgfqpoint{3.558480in}{4.071248in}}%
\pgfpathlineto{\pgfqpoint{3.545668in}{4.090794in}}%
\pgfpathlineto{\pgfqpoint{3.532851in}{4.110541in}}%
\pgfpathlineto{\pgfqpoint{3.520030in}{4.130490in}}%
\pgfpathlineto{\pgfqpoint{3.507204in}{4.150644in}}%
\pgfpathlineto{\pgfqpoint{3.499739in}{4.133698in}}%
\pgfpathlineto{\pgfqpoint{3.492268in}{4.116927in}}%
\pgfpathlineto{\pgfqpoint{3.484794in}{4.100328in}}%
\pgfpathlineto{\pgfqpoint{3.477314in}{4.083900in}}%
\pgfpathclose%
\pgfusepath{fill}%
\end{pgfscope}%
\begin{pgfscope}%
\pgfpathrectangle{\pgfqpoint{1.254980in}{0.150000in}}{\pgfqpoint{5.490039in}{5.490039in}}%
\pgfusepath{clip}%
\pgfsetbuttcap%
\pgfsetroundjoin%
\definecolor{currentfill}{rgb}{0.204903,0.375746,0.553533}%
\pgfsetfillcolor{currentfill}%
\pgfsetfillopacity{0.700000}%
\pgfsetlinewidth{0.000000pt}%
\definecolor{currentstroke}{rgb}{0.000000,0.000000,0.000000}%
\pgfsetstrokecolor{currentstroke}%
\pgfsetdash{}{0pt}%
\pgfpathmoveto{\pgfqpoint{4.192816in}{3.246919in}}%
\pgfpathlineto{\pgfqpoint{4.205602in}{3.236623in}}%
\pgfpathlineto{\pgfqpoint{4.218390in}{3.226474in}}%
\pgfpathlineto{\pgfqpoint{4.231182in}{3.216471in}}%
\pgfpathlineto{\pgfqpoint{4.243976in}{3.206614in}}%
\pgfpathlineto{\pgfqpoint{4.251313in}{3.220184in}}%
\pgfpathlineto{\pgfqpoint{4.258647in}{3.233862in}}%
\pgfpathlineto{\pgfqpoint{4.265978in}{3.247651in}}%
\pgfpathlineto{\pgfqpoint{4.273306in}{3.261553in}}%
\pgfpathlineto{\pgfqpoint{4.260521in}{3.271603in}}%
\pgfpathlineto{\pgfqpoint{4.247740in}{3.281798in}}%
\pgfpathlineto{\pgfqpoint{4.234961in}{3.292139in}}%
\pgfpathlineto{\pgfqpoint{4.222185in}{3.302628in}}%
\pgfpathlineto{\pgfqpoint{4.214847in}{3.288527in}}%
\pgfpathlineto{\pgfqpoint{4.207506in}{3.274544in}}%
\pgfpathlineto{\pgfqpoint{4.200163in}{3.260675in}}%
\pgfpathlineto{\pgfqpoint{4.192816in}{3.246919in}}%
\pgfpathclose%
\pgfusepath{fill}%
\end{pgfscope}%
\begin{pgfscope}%
\pgfpathrectangle{\pgfqpoint{1.254980in}{0.150000in}}{\pgfqpoint{5.490039in}{5.490039in}}%
\pgfusepath{clip}%
\pgfsetbuttcap%
\pgfsetroundjoin%
\definecolor{currentfill}{rgb}{0.190631,0.407061,0.556089}%
\pgfsetfillcolor{currentfill}%
\pgfsetfillopacity{0.700000}%
\pgfsetlinewidth{0.000000pt}%
\definecolor{currentstroke}{rgb}{0.000000,0.000000,0.000000}%
\pgfsetstrokecolor{currentstroke}%
\pgfsetdash{}{0pt}%
\pgfpathmoveto{\pgfqpoint{4.010014in}{3.326134in}}%
\pgfpathlineto{\pgfqpoint{4.022786in}{3.314195in}}%
\pgfpathlineto{\pgfqpoint{4.035561in}{3.302412in}}%
\pgfpathlineto{\pgfqpoint{4.048336in}{3.290784in}}%
\pgfpathlineto{\pgfqpoint{4.061113in}{3.279311in}}%
\pgfpathlineto{\pgfqpoint{4.068492in}{3.292978in}}%
\pgfpathlineto{\pgfqpoint{4.075868in}{3.306758in}}%
\pgfpathlineto{\pgfqpoint{4.083240in}{3.320652in}}%
\pgfpathlineto{\pgfqpoint{4.090609in}{3.334664in}}%
\pgfpathlineto{\pgfqpoint{4.077842in}{3.346315in}}%
\pgfpathlineto{\pgfqpoint{4.065076in}{3.358121in}}%
\pgfpathlineto{\pgfqpoint{4.052312in}{3.370082in}}%
\pgfpathlineto{\pgfqpoint{4.039549in}{3.382199in}}%
\pgfpathlineto{\pgfqpoint{4.032170in}{3.368003in}}%
\pgfpathlineto{\pgfqpoint{4.024788in}{3.353929in}}%
\pgfpathlineto{\pgfqpoint{4.017402in}{3.339973in}}%
\pgfpathlineto{\pgfqpoint{4.010014in}{3.326134in}}%
\pgfpathclose%
\pgfusepath{fill}%
\end{pgfscope}%
\begin{pgfscope}%
\pgfpathrectangle{\pgfqpoint{1.254980in}{0.150000in}}{\pgfqpoint{5.490039in}{5.490039in}}%
\pgfusepath{clip}%
\pgfsetbuttcap%
\pgfsetroundjoin%
\definecolor{currentfill}{rgb}{0.119483,0.614817,0.537692}%
\pgfsetfillcolor{currentfill}%
\pgfsetfillopacity{0.700000}%
\pgfsetlinewidth{0.000000pt}%
\definecolor{currentstroke}{rgb}{0.000000,0.000000,0.000000}%
\pgfsetstrokecolor{currentstroke}%
\pgfsetdash{}{0pt}%
\pgfpathmoveto{\pgfqpoint{3.549987in}{3.867356in}}%
\pgfpathlineto{\pgfqpoint{3.562799in}{3.849179in}}%
\pgfpathlineto{\pgfqpoint{3.575607in}{3.831196in}}%
\pgfpathlineto{\pgfqpoint{3.588412in}{3.813403in}}%
\pgfpathlineto{\pgfqpoint{3.601213in}{3.795801in}}%
\pgfpathlineto{\pgfqpoint{3.608681in}{3.811032in}}%
\pgfpathlineto{\pgfqpoint{3.616144in}{3.826413in}}%
\pgfpathlineto{\pgfqpoint{3.623604in}{3.841946in}}%
\pgfpathlineto{\pgfqpoint{3.631058in}{3.857633in}}%
\pgfpathlineto{\pgfqpoint{3.618267in}{3.875418in}}%
\pgfpathlineto{\pgfqpoint{3.605472in}{3.893394in}}%
\pgfpathlineto{\pgfqpoint{3.592673in}{3.911562in}}%
\pgfpathlineto{\pgfqpoint{3.579872in}{3.929922in}}%
\pgfpathlineto{\pgfqpoint{3.572407in}{3.914046in}}%
\pgfpathlineto{\pgfqpoint{3.564938in}{3.898327in}}%
\pgfpathlineto{\pgfqpoint{3.557465in}{3.882764in}}%
\pgfpathlineto{\pgfqpoint{3.549987in}{3.867356in}}%
\pgfpathclose%
\pgfusepath{fill}%
\end{pgfscope}%
\begin{pgfscope}%
\pgfpathrectangle{\pgfqpoint{1.254980in}{0.150000in}}{\pgfqpoint{5.490039in}{5.490039in}}%
\pgfusepath{clip}%
\pgfsetbuttcap%
\pgfsetroundjoin%
\definecolor{currentfill}{rgb}{0.214298,0.355619,0.551184}%
\pgfsetfillcolor{currentfill}%
\pgfsetfillopacity{0.700000}%
\pgfsetlinewidth{0.000000pt}%
\definecolor{currentstroke}{rgb}{0.000000,0.000000,0.000000}%
\pgfsetstrokecolor{currentstroke}%
\pgfsetdash{}{0pt}%
\pgfpathmoveto{\pgfqpoint{4.587888in}{3.197131in}}%
\pgfpathlineto{\pgfqpoint{4.600735in}{3.189572in}}%
\pgfpathlineto{\pgfqpoint{4.613587in}{3.182146in}}%
\pgfpathlineto{\pgfqpoint{4.626445in}{3.174852in}}%
\pgfpathlineto{\pgfqpoint{4.639308in}{3.167690in}}%
\pgfpathlineto{\pgfqpoint{4.646554in}{3.181309in}}%
\pgfpathlineto{\pgfqpoint{4.653799in}{3.195039in}}%
\pgfpathlineto{\pgfqpoint{4.661042in}{3.208883in}}%
\pgfpathlineto{\pgfqpoint{4.648187in}{3.216222in}}%
\pgfpathlineto{\pgfqpoint{4.635337in}{3.223694in}}%
\pgfpathlineto{\pgfqpoint{4.622493in}{3.231297in}}%
\pgfpathlineto{\pgfqpoint{4.609653in}{3.239034in}}%
\pgfpathlineto{\pgfqpoint{4.602400in}{3.224949in}}%
\pgfpathlineto{\pgfqpoint{4.595145in}{3.210982in}}%
\pgfpathlineto{\pgfqpoint{4.587888in}{3.197131in}}%
\pgfpathclose%
\pgfusepath{fill}%
\end{pgfscope}%
\begin{pgfscope}%
\pgfpathrectangle{\pgfqpoint{1.254980in}{0.150000in}}{\pgfqpoint{5.490039in}{5.490039in}}%
\pgfusepath{clip}%
\pgfsetbuttcap%
\pgfsetroundjoin%
\definecolor{currentfill}{rgb}{0.197636,0.391528,0.554969}%
\pgfsetfillcolor{currentfill}%
\pgfsetfillopacity{0.700000}%
\pgfsetlinewidth{0.000000pt}%
\definecolor{currentstroke}{rgb}{0.000000,0.000000,0.000000}%
\pgfsetstrokecolor{currentstroke}%
\pgfsetdash{}{0pt}%
\pgfpathmoveto{\pgfqpoint{4.061113in}{3.279311in}}%
\pgfpathlineto{\pgfqpoint{4.073892in}{3.267990in}}%
\pgfpathlineto{\pgfqpoint{4.086672in}{3.256823in}}%
\pgfpathlineto{\pgfqpoint{4.099455in}{3.245807in}}%
\pgfpathlineto{\pgfqpoint{4.112239in}{3.234943in}}%
\pgfpathlineto{\pgfqpoint{4.119608in}{3.248438in}}%
\pgfpathlineto{\pgfqpoint{4.126974in}{3.262043in}}%
\pgfpathlineto{\pgfqpoint{4.134337in}{3.275758in}}%
\pgfpathlineto{\pgfqpoint{4.141696in}{3.289587in}}%
\pgfpathlineto{\pgfqpoint{4.128922in}{3.300628in}}%
\pgfpathlineto{\pgfqpoint{4.116149in}{3.311821in}}%
\pgfpathlineto{\pgfqpoint{4.103378in}{3.323166in}}%
\pgfpathlineto{\pgfqpoint{4.090609in}{3.334664in}}%
\pgfpathlineto{\pgfqpoint{4.083240in}{3.320652in}}%
\pgfpathlineto{\pgfqpoint{4.075868in}{3.306758in}}%
\pgfpathlineto{\pgfqpoint{4.068492in}{3.292978in}}%
\pgfpathlineto{\pgfqpoint{4.061113in}{3.279311in}}%
\pgfpathclose%
\pgfusepath{fill}%
\end{pgfscope}%
\begin{pgfscope}%
\pgfpathrectangle{\pgfqpoint{1.254980in}{0.150000in}}{\pgfqpoint{5.490039in}{5.490039in}}%
\pgfusepath{clip}%
\pgfsetbuttcap%
\pgfsetroundjoin%
\definecolor{currentfill}{rgb}{0.128087,0.647749,0.523491}%
\pgfsetfillcolor{currentfill}%
\pgfsetfillopacity{0.700000}%
\pgfsetlinewidth{0.000000pt}%
\definecolor{currentstroke}{rgb}{0.000000,0.000000,0.000000}%
\pgfsetstrokecolor{currentstroke}%
\pgfsetdash{}{0pt}%
\pgfpathmoveto{\pgfqpoint{3.498703in}{3.942012in}}%
\pgfpathlineto{\pgfqpoint{3.511530in}{3.923052in}}%
\pgfpathlineto{\pgfqpoint{3.524353in}{3.904291in}}%
\pgfpathlineto{\pgfqpoint{3.537172in}{3.885726in}}%
\pgfpathlineto{\pgfqpoint{3.549987in}{3.867356in}}%
\pgfpathlineto{\pgfqpoint{3.557465in}{3.882764in}}%
\pgfpathlineto{\pgfqpoint{3.564938in}{3.898327in}}%
\pgfpathlineto{\pgfqpoint{3.572407in}{3.914046in}}%
\pgfpathlineto{\pgfqpoint{3.579872in}{3.929922in}}%
\pgfpathlineto{\pgfqpoint{3.567066in}{3.948476in}}%
\pgfpathlineto{\pgfqpoint{3.554257in}{3.967226in}}%
\pgfpathlineto{\pgfqpoint{3.541444in}{3.986172in}}%
\pgfpathlineto{\pgfqpoint{3.528627in}{4.005316in}}%
\pgfpathlineto{\pgfqpoint{3.521153in}{3.989249in}}%
\pgfpathlineto{\pgfqpoint{3.513674in}{3.973344in}}%
\pgfpathlineto{\pgfqpoint{3.506191in}{3.957598in}}%
\pgfpathlineto{\pgfqpoint{3.498703in}{3.942012in}}%
\pgfpathclose%
\pgfusepath{fill}%
\end{pgfscope}%
\begin{pgfscope}%
\pgfpathrectangle{\pgfqpoint{1.254980in}{0.150000in}}{\pgfqpoint{5.490039in}{5.490039in}}%
\pgfusepath{clip}%
\pgfsetbuttcap%
\pgfsetroundjoin%
\definecolor{currentfill}{rgb}{0.216210,0.351535,0.550627}%
\pgfsetfillcolor{currentfill}%
\pgfsetfillopacity{0.700000}%
\pgfsetlinewidth{0.000000pt}%
\definecolor{currentstroke}{rgb}{0.000000,0.000000,0.000000}%
\pgfsetstrokecolor{currentstroke}%
\pgfsetdash{}{0pt}%
\pgfpathmoveto{\pgfqpoint{4.375696in}{3.186327in}}%
\pgfpathlineto{\pgfqpoint{4.388510in}{3.177561in}}%
\pgfpathlineto{\pgfqpoint{4.401329in}{3.168934in}}%
\pgfpathlineto{\pgfqpoint{4.414151in}{3.160446in}}%
\pgfpathlineto{\pgfqpoint{4.426978in}{3.152096in}}%
\pgfpathlineto{\pgfqpoint{4.434274in}{3.165507in}}%
\pgfpathlineto{\pgfqpoint{4.441567in}{3.179024in}}%
\pgfpathlineto{\pgfqpoint{4.448858in}{3.192648in}}%
\pgfpathlineto{\pgfqpoint{4.456147in}{3.206381in}}%
\pgfpathlineto{\pgfqpoint{4.443330in}{3.214939in}}%
\pgfpathlineto{\pgfqpoint{4.430518in}{3.223634in}}%
\pgfpathlineto{\pgfqpoint{4.417709in}{3.232469in}}%
\pgfpathlineto{\pgfqpoint{4.404904in}{3.241443in}}%
\pgfpathlineto{\pgfqpoint{4.397606in}{3.227496in}}%
\pgfpathlineto{\pgfqpoint{4.390305in}{3.213662in}}%
\pgfpathlineto{\pgfqpoint{4.383002in}{3.199940in}}%
\pgfpathlineto{\pgfqpoint{4.375696in}{3.186327in}}%
\pgfpathclose%
\pgfusepath{fill}%
\end{pgfscope}%
\begin{pgfscope}%
\pgfpathrectangle{\pgfqpoint{1.254980in}{0.150000in}}{\pgfqpoint{5.490039in}{5.490039in}}%
\pgfusepath{clip}%
\pgfsetbuttcap%
\pgfsetroundjoin%
\definecolor{currentfill}{rgb}{0.154815,0.493313,0.557840}%
\pgfsetfillcolor{currentfill}%
\pgfsetfillopacity{0.700000}%
\pgfsetlinewidth{0.000000pt}%
\definecolor{currentstroke}{rgb}{0.000000,0.000000,0.000000}%
\pgfsetstrokecolor{currentstroke}%
\pgfsetdash{}{0pt}%
\pgfpathmoveto{\pgfqpoint{3.724844in}{3.541474in}}%
\pgfpathlineto{\pgfqpoint{3.737628in}{3.526399in}}%
\pgfpathlineto{\pgfqpoint{3.750410in}{3.511498in}}%
\pgfpathlineto{\pgfqpoint{3.763191in}{3.496769in}}%
\pgfpathlineto{\pgfqpoint{3.775971in}{3.482213in}}%
\pgfpathlineto{\pgfqpoint{3.783415in}{3.496229in}}%
\pgfpathlineto{\pgfqpoint{3.790854in}{3.510371in}}%
\pgfpathlineto{\pgfqpoint{3.798290in}{3.524640in}}%
\pgfpathlineto{\pgfqpoint{3.805722in}{3.539039in}}%
\pgfpathlineto{\pgfqpoint{3.792952in}{3.553760in}}%
\pgfpathlineto{\pgfqpoint{3.780181in}{3.568652in}}%
\pgfpathlineto{\pgfqpoint{3.767410in}{3.583717in}}%
\pgfpathlineto{\pgfqpoint{3.754636in}{3.598956in}}%
\pgfpathlineto{\pgfqpoint{3.747195in}{3.584387in}}%
\pgfpathlineto{\pgfqpoint{3.739749in}{3.569952in}}%
\pgfpathlineto{\pgfqpoint{3.732299in}{3.555648in}}%
\pgfpathlineto{\pgfqpoint{3.724844in}{3.541474in}}%
\pgfpathclose%
\pgfusepath{fill}%
\end{pgfscope}%
\begin{pgfscope}%
\pgfpathrectangle{\pgfqpoint{1.254980in}{0.150000in}}{\pgfqpoint{5.490039in}{5.490039in}}%
\pgfusepath{clip}%
\pgfsetbuttcap%
\pgfsetroundjoin%
\definecolor{currentfill}{rgb}{0.226397,0.728888,0.462789}%
\pgfsetfillcolor{currentfill}%
\pgfsetfillopacity{0.700000}%
\pgfsetlinewidth{0.000000pt}%
\definecolor{currentstroke}{rgb}{0.000000,0.000000,0.000000}%
\pgfsetstrokecolor{currentstroke}%
\pgfsetdash{}{0pt}%
\pgfpathmoveto{\pgfqpoint{3.425924in}{4.165764in}}%
\pgfpathlineto{\pgfqpoint{3.438779in}{4.144985in}}%
\pgfpathlineto{\pgfqpoint{3.451629in}{4.124416in}}%
\pgfpathlineto{\pgfqpoint{3.464474in}{4.104055in}}%
\pgfpathlineto{\pgfqpoint{3.477314in}{4.083900in}}%
\pgfpathlineto{\pgfqpoint{3.484794in}{4.100328in}}%
\pgfpathlineto{\pgfqpoint{3.492268in}{4.116927in}}%
\pgfpathlineto{\pgfqpoint{3.499739in}{4.133698in}}%
\pgfpathlineto{\pgfqpoint{3.507204in}{4.150644in}}%
\pgfpathlineto{\pgfqpoint{3.494374in}{4.171002in}}%
\pgfpathlineto{\pgfqpoint{3.481538in}{4.191567in}}%
\pgfpathlineto{\pgfqpoint{3.468698in}{4.212341in}}%
\pgfpathlineto{\pgfqpoint{3.455852in}{4.233324in}}%
\pgfpathlineto{\pgfqpoint{3.448377in}{4.216167in}}%
\pgfpathlineto{\pgfqpoint{3.440897in}{4.199190in}}%
\pgfpathlineto{\pgfqpoint{3.433413in}{4.182389in}}%
\pgfpathlineto{\pgfqpoint{3.425924in}{4.165764in}}%
\pgfpathclose%
\pgfusepath{fill}%
\end{pgfscope}%
\begin{pgfscope}%
\pgfpathrectangle{\pgfqpoint{1.254980in}{0.150000in}}{\pgfqpoint{5.490039in}{5.490039in}}%
\pgfusepath{clip}%
\pgfsetbuttcap%
\pgfsetroundjoin%
\definecolor{currentfill}{rgb}{0.163625,0.471133,0.558148}%
\pgfsetfillcolor{currentfill}%
\pgfsetfillopacity{0.700000}%
\pgfsetlinewidth{0.000000pt}%
\definecolor{currentstroke}{rgb}{0.000000,0.000000,0.000000}%
\pgfsetstrokecolor{currentstroke}%
\pgfsetdash{}{0pt}%
\pgfpathmoveto{\pgfqpoint{3.775971in}{3.482213in}}%
\pgfpathlineto{\pgfqpoint{3.788750in}{3.467828in}}%
\pgfpathlineto{\pgfqpoint{3.801528in}{3.453612in}}%
\pgfpathlineto{\pgfqpoint{3.814306in}{3.439565in}}%
\pgfpathlineto{\pgfqpoint{3.827083in}{3.425687in}}%
\pgfpathlineto{\pgfqpoint{3.834517in}{3.439546in}}%
\pgfpathlineto{\pgfqpoint{3.841947in}{3.453526in}}%
\pgfpathlineto{\pgfqpoint{3.849373in}{3.467630in}}%
\pgfpathlineto{\pgfqpoint{3.856795in}{3.481859in}}%
\pgfpathlineto{\pgfqpoint{3.844027in}{3.495901in}}%
\pgfpathlineto{\pgfqpoint{3.831259in}{3.510111in}}%
\pgfpathlineto{\pgfqpoint{3.818491in}{3.524490in}}%
\pgfpathlineto{\pgfqpoint{3.805722in}{3.539039in}}%
\pgfpathlineto{\pgfqpoint{3.798290in}{3.524640in}}%
\pgfpathlineto{\pgfqpoint{3.790854in}{3.510371in}}%
\pgfpathlineto{\pgfqpoint{3.783415in}{3.496229in}}%
\pgfpathlineto{\pgfqpoint{3.775971in}{3.482213in}}%
\pgfpathclose%
\pgfusepath{fill}%
\end{pgfscope}%
\begin{pgfscope}%
\pgfpathrectangle{\pgfqpoint{1.254980in}{0.150000in}}{\pgfqpoint{5.490039in}{5.490039in}}%
\pgfusepath{clip}%
\pgfsetbuttcap%
\pgfsetroundjoin%
\definecolor{currentfill}{rgb}{0.212395,0.359683,0.551710}%
\pgfsetfillcolor{currentfill}%
\pgfsetfillopacity{0.700000}%
\pgfsetlinewidth{0.000000pt}%
\definecolor{currentstroke}{rgb}{0.000000,0.000000,0.000000}%
\pgfsetstrokecolor{currentstroke}%
\pgfsetdash{}{0pt}%
\pgfpathmoveto{\pgfqpoint{4.243976in}{3.206614in}}%
\pgfpathlineto{\pgfqpoint{4.256773in}{3.196903in}}%
\pgfpathlineto{\pgfqpoint{4.269573in}{3.187335in}}%
\pgfpathlineto{\pgfqpoint{4.282376in}{3.177911in}}%
\pgfpathlineto{\pgfqpoint{4.295183in}{3.168631in}}%
\pgfpathlineto{\pgfqpoint{4.302510in}{3.182014in}}%
\pgfpathlineto{\pgfqpoint{4.309834in}{3.195501in}}%
\pgfpathlineto{\pgfqpoint{4.317156in}{3.209096in}}%
\pgfpathlineto{\pgfqpoint{4.324474in}{3.222800in}}%
\pgfpathlineto{\pgfqpoint{4.311677in}{3.232273in}}%
\pgfpathlineto{\pgfqpoint{4.298884in}{3.241889in}}%
\pgfpathlineto{\pgfqpoint{4.286093in}{3.251649in}}%
\pgfpathlineto{\pgfqpoint{4.273306in}{3.261553in}}%
\pgfpathlineto{\pgfqpoint{4.265978in}{3.247651in}}%
\pgfpathlineto{\pgfqpoint{4.258647in}{3.233862in}}%
\pgfpathlineto{\pgfqpoint{4.251313in}{3.220184in}}%
\pgfpathlineto{\pgfqpoint{4.243976in}{3.206614in}}%
\pgfpathclose%
\pgfusepath{fill}%
\end{pgfscope}%
\begin{pgfscope}%
\pgfpathrectangle{\pgfqpoint{1.254980in}{0.150000in}}{\pgfqpoint{5.490039in}{5.490039in}}%
\pgfusepath{clip}%
\pgfsetbuttcap%
\pgfsetroundjoin%
\definecolor{currentfill}{rgb}{0.146180,0.515413,0.556823}%
\pgfsetfillcolor{currentfill}%
\pgfsetfillopacity{0.700000}%
\pgfsetlinewidth{0.000000pt}%
\definecolor{currentstroke}{rgb}{0.000000,0.000000,0.000000}%
\pgfsetstrokecolor{currentstroke}%
\pgfsetdash{}{0pt}%
\pgfpathmoveto{\pgfqpoint{3.673696in}{3.603535in}}%
\pgfpathlineto{\pgfqpoint{3.686486in}{3.587753in}}%
\pgfpathlineto{\pgfqpoint{3.699274in}{3.572150in}}%
\pgfpathlineto{\pgfqpoint{3.712060in}{3.556724in}}%
\pgfpathlineto{\pgfqpoint{3.724844in}{3.541474in}}%
\pgfpathlineto{\pgfqpoint{3.732299in}{3.555648in}}%
\pgfpathlineto{\pgfqpoint{3.739749in}{3.569952in}}%
\pgfpathlineto{\pgfqpoint{3.747195in}{3.584387in}}%
\pgfpathlineto{\pgfqpoint{3.754636in}{3.598956in}}%
\pgfpathlineto{\pgfqpoint{3.741862in}{3.614371in}}%
\pgfpathlineto{\pgfqpoint{3.729086in}{3.629961in}}%
\pgfpathlineto{\pgfqpoint{3.716308in}{3.645729in}}%
\pgfpathlineto{\pgfqpoint{3.703529in}{3.661676in}}%
\pgfpathlineto{\pgfqpoint{3.696077in}{3.646936in}}%
\pgfpathlineto{\pgfqpoint{3.688621in}{3.632333in}}%
\pgfpathlineto{\pgfqpoint{3.681161in}{3.617867in}}%
\pgfpathlineto{\pgfqpoint{3.673696in}{3.603535in}}%
\pgfpathclose%
\pgfusepath{fill}%
\end{pgfscope}%
\begin{pgfscope}%
\pgfpathrectangle{\pgfqpoint{1.254980in}{0.150000in}}{\pgfqpoint{5.490039in}{5.490039in}}%
\pgfusepath{clip}%
\pgfsetbuttcap%
\pgfsetroundjoin%
\definecolor{currentfill}{rgb}{0.218130,0.347432,0.550038}%
\pgfsetfillcolor{currentfill}%
\pgfsetfillopacity{0.700000}%
\pgfsetlinewidth{0.000000pt}%
\definecolor{currentstroke}{rgb}{0.000000,0.000000,0.000000}%
\pgfsetstrokecolor{currentstroke}%
\pgfsetdash{}{0pt}%
\pgfpathmoveto{\pgfqpoint{4.507456in}{3.173525in}}%
\pgfpathlineto{\pgfqpoint{4.520294in}{3.165651in}}%
\pgfpathlineto{\pgfqpoint{4.533137in}{3.157912in}}%
\pgfpathlineto{\pgfqpoint{4.545985in}{3.150308in}}%
\pgfpathlineto{\pgfqpoint{4.558838in}{3.142838in}}%
\pgfpathlineto{\pgfqpoint{4.566104in}{3.156250in}}%
\pgfpathlineto{\pgfqpoint{4.573367in}{3.169767in}}%
\pgfpathlineto{\pgfqpoint{4.580629in}{3.183394in}}%
\pgfpathlineto{\pgfqpoint{4.587888in}{3.197131in}}%
\pgfpathlineto{\pgfqpoint{4.575045in}{3.204824in}}%
\pgfpathlineto{\pgfqpoint{4.562208in}{3.212651in}}%
\pgfpathlineto{\pgfqpoint{4.549375in}{3.220613in}}%
\pgfpathlineto{\pgfqpoint{4.536546in}{3.228710in}}%
\pgfpathlineto{\pgfqpoint{4.529277in}{3.214744in}}%
\pgfpathlineto{\pgfqpoint{4.522006in}{3.200892in}}%
\pgfpathlineto{\pgfqpoint{4.514732in}{3.187153in}}%
\pgfpathlineto{\pgfqpoint{4.507456in}{3.173525in}}%
\pgfpathclose%
\pgfusepath{fill}%
\end{pgfscope}%
\begin{pgfscope}%
\pgfpathrectangle{\pgfqpoint{1.254980in}{0.150000in}}{\pgfqpoint{5.490039in}{5.490039in}}%
\pgfusepath{clip}%
\pgfsetbuttcap%
\pgfsetroundjoin%
\definecolor{currentfill}{rgb}{0.172719,0.448791,0.557885}%
\pgfsetfillcolor{currentfill}%
\pgfsetfillopacity{0.700000}%
\pgfsetlinewidth{0.000000pt}%
\definecolor{currentstroke}{rgb}{0.000000,0.000000,0.000000}%
\pgfsetstrokecolor{currentstroke}%
\pgfsetdash{}{0pt}%
\pgfpathmoveto{\pgfqpoint{3.827083in}{3.425687in}}%
\pgfpathlineto{\pgfqpoint{3.839860in}{3.411975in}}%
\pgfpathlineto{\pgfqpoint{3.852637in}{3.398430in}}%
\pgfpathlineto{\pgfqpoint{3.865414in}{3.385050in}}%
\pgfpathlineto{\pgfqpoint{3.878191in}{3.371834in}}%
\pgfpathlineto{\pgfqpoint{3.885614in}{3.385535in}}%
\pgfpathlineto{\pgfqpoint{3.893034in}{3.399355in}}%
\pgfpathlineto{\pgfqpoint{3.900450in}{3.413294in}}%
\pgfpathlineto{\pgfqpoint{3.907862in}{3.427354in}}%
\pgfpathlineto{\pgfqpoint{3.895095in}{3.440733in}}%
\pgfpathlineto{\pgfqpoint{3.882328in}{3.454276in}}%
\pgfpathlineto{\pgfqpoint{3.869562in}{3.467985in}}%
\pgfpathlineto{\pgfqpoint{3.856795in}{3.481859in}}%
\pgfpathlineto{\pgfqpoint{3.849373in}{3.467630in}}%
\pgfpathlineto{\pgfqpoint{3.841947in}{3.453526in}}%
\pgfpathlineto{\pgfqpoint{3.834517in}{3.439546in}}%
\pgfpathlineto{\pgfqpoint{3.827083in}{3.425687in}}%
\pgfpathclose%
\pgfusepath{fill}%
\end{pgfscope}%
\begin{pgfscope}%
\pgfpathrectangle{\pgfqpoint{1.254980in}{0.150000in}}{\pgfqpoint{5.490039in}{5.490039in}}%
\pgfusepath{clip}%
\pgfsetbuttcap%
\pgfsetroundjoin%
\definecolor{currentfill}{rgb}{0.136408,0.541173,0.554483}%
\pgfsetfillcolor{currentfill}%
\pgfsetfillopacity{0.700000}%
\pgfsetlinewidth{0.000000pt}%
\definecolor{currentstroke}{rgb}{0.000000,0.000000,0.000000}%
\pgfsetstrokecolor{currentstroke}%
\pgfsetdash{}{0pt}%
\pgfpathmoveto{\pgfqpoint{3.622517in}{3.668464in}}%
\pgfpathlineto{\pgfqpoint{3.635315in}{3.651959in}}%
\pgfpathlineto{\pgfqpoint{3.648111in}{3.635636in}}%
\pgfpathlineto{\pgfqpoint{3.660905in}{3.619495in}}%
\pgfpathlineto{\pgfqpoint{3.673696in}{3.603535in}}%
\pgfpathlineto{\pgfqpoint{3.681161in}{3.617867in}}%
\pgfpathlineto{\pgfqpoint{3.688621in}{3.632333in}}%
\pgfpathlineto{\pgfqpoint{3.696077in}{3.646936in}}%
\pgfpathlineto{\pgfqpoint{3.703529in}{3.661676in}}%
\pgfpathlineto{\pgfqpoint{3.690748in}{3.677801in}}%
\pgfpathlineto{\pgfqpoint{3.677964in}{3.694108in}}%
\pgfpathlineto{\pgfqpoint{3.665179in}{3.710595in}}%
\pgfpathlineto{\pgfqpoint{3.652391in}{3.727266in}}%
\pgfpathlineto{\pgfqpoint{3.644929in}{3.712355in}}%
\pgfpathlineto{\pgfqpoint{3.637463in}{3.697585in}}%
\pgfpathlineto{\pgfqpoint{3.629992in}{3.682955in}}%
\pgfpathlineto{\pgfqpoint{3.622517in}{3.668464in}}%
\pgfpathclose%
\pgfusepath{fill}%
\end{pgfscope}%
\begin{pgfscope}%
\pgfpathrectangle{\pgfqpoint{1.254980in}{0.150000in}}{\pgfqpoint{5.490039in}{5.490039in}}%
\pgfusepath{clip}%
\pgfsetbuttcap%
\pgfsetroundjoin%
\definecolor{currentfill}{rgb}{0.182256,0.426184,0.557120}%
\pgfsetfillcolor{currentfill}%
\pgfsetfillopacity{0.700000}%
\pgfsetlinewidth{0.000000pt}%
\definecolor{currentstroke}{rgb}{0.000000,0.000000,0.000000}%
\pgfsetstrokecolor{currentstroke}%
\pgfsetdash{}{0pt}%
\pgfpathmoveto{\pgfqpoint{3.878191in}{3.371834in}}%
\pgfpathlineto{\pgfqpoint{3.890968in}{3.358781in}}%
\pgfpathlineto{\pgfqpoint{3.903745in}{3.345891in}}%
\pgfpathlineto{\pgfqpoint{3.916523in}{3.333162in}}%
\pgfpathlineto{\pgfqpoint{3.929301in}{3.320594in}}%
\pgfpathlineto{\pgfqpoint{3.936715in}{3.334139in}}%
\pgfpathlineto{\pgfqpoint{3.944124in}{3.347798in}}%
\pgfpathlineto{\pgfqpoint{3.951530in}{3.361573in}}%
\pgfpathlineto{\pgfqpoint{3.958932in}{3.375464in}}%
\pgfpathlineto{\pgfqpoint{3.946164in}{3.388195in}}%
\pgfpathlineto{\pgfqpoint{3.933396in}{3.401086in}}%
\pgfpathlineto{\pgfqpoint{3.920629in}{3.414139in}}%
\pgfpathlineto{\pgfqpoint{3.907862in}{3.427354in}}%
\pgfpathlineto{\pgfqpoint{3.900450in}{3.413294in}}%
\pgfpathlineto{\pgfqpoint{3.893034in}{3.399355in}}%
\pgfpathlineto{\pgfqpoint{3.885614in}{3.385535in}}%
\pgfpathlineto{\pgfqpoint{3.878191in}{3.371834in}}%
\pgfpathclose%
\pgfusepath{fill}%
\end{pgfscope}%
\begin{pgfscope}%
\pgfpathrectangle{\pgfqpoint{1.254980in}{0.150000in}}{\pgfqpoint{5.490039in}{5.490039in}}%
\pgfusepath{clip}%
\pgfsetbuttcap%
\pgfsetroundjoin%
\definecolor{currentfill}{rgb}{0.206756,0.371758,0.553117}%
\pgfsetfillcolor{currentfill}%
\pgfsetfillopacity{0.700000}%
\pgfsetlinewidth{0.000000pt}%
\definecolor{currentstroke}{rgb}{0.000000,0.000000,0.000000}%
\pgfsetstrokecolor{currentstroke}%
\pgfsetdash{}{0pt}%
\pgfpathmoveto{\pgfqpoint{4.112239in}{3.234943in}}%
\pgfpathlineto{\pgfqpoint{4.125025in}{3.224229in}}%
\pgfpathlineto{\pgfqpoint{4.137814in}{3.213665in}}%
\pgfpathlineto{\pgfqpoint{4.150604in}{3.203250in}}%
\pgfpathlineto{\pgfqpoint{4.163398in}{3.192983in}}%
\pgfpathlineto{\pgfqpoint{4.170757in}{3.206308in}}%
\pgfpathlineto{\pgfqpoint{4.178113in}{3.219737in}}%
\pgfpathlineto{\pgfqpoint{4.185466in}{3.233273in}}%
\pgfpathlineto{\pgfqpoint{4.192816in}{3.246919in}}%
\pgfpathlineto{\pgfqpoint{4.180032in}{3.257362in}}%
\pgfpathlineto{\pgfqpoint{4.167251in}{3.267954in}}%
\pgfpathlineto{\pgfqpoint{4.154473in}{3.278696in}}%
\pgfpathlineto{\pgfqpoint{4.141696in}{3.289587in}}%
\pgfpathlineto{\pgfqpoint{4.134337in}{3.275758in}}%
\pgfpathlineto{\pgfqpoint{4.126974in}{3.262043in}}%
\pgfpathlineto{\pgfqpoint{4.119608in}{3.248438in}}%
\pgfpathlineto{\pgfqpoint{4.112239in}{3.234943in}}%
\pgfpathclose%
\pgfusepath{fill}%
\end{pgfscope}%
\begin{pgfscope}%
\pgfpathrectangle{\pgfqpoint{1.254980in}{0.150000in}}{\pgfqpoint{5.490039in}{5.490039in}}%
\pgfusepath{clip}%
\pgfsetbuttcap%
\pgfsetroundjoin%
\definecolor{currentfill}{rgb}{0.126453,0.570633,0.549841}%
\pgfsetfillcolor{currentfill}%
\pgfsetfillopacity{0.700000}%
\pgfsetlinewidth{0.000000pt}%
\definecolor{currentstroke}{rgb}{0.000000,0.000000,0.000000}%
\pgfsetstrokecolor{currentstroke}%
\pgfsetdash{}{0pt}%
\pgfpathmoveto{\pgfqpoint{3.571299in}{3.736335in}}%
\pgfpathlineto{\pgfqpoint{3.584108in}{3.719087in}}%
\pgfpathlineto{\pgfqpoint{3.596914in}{3.702027in}}%
\pgfpathlineto{\pgfqpoint{3.609717in}{3.685153in}}%
\pgfpathlineto{\pgfqpoint{3.622517in}{3.668464in}}%
\pgfpathlineto{\pgfqpoint{3.629992in}{3.682955in}}%
\pgfpathlineto{\pgfqpoint{3.637463in}{3.697585in}}%
\pgfpathlineto{\pgfqpoint{3.644929in}{3.712355in}}%
\pgfpathlineto{\pgfqpoint{3.652391in}{3.727266in}}%
\pgfpathlineto{\pgfqpoint{3.639601in}{3.744121in}}%
\pgfpathlineto{\pgfqpoint{3.626808in}{3.761161in}}%
\pgfpathlineto{\pgfqpoint{3.614012in}{3.778387in}}%
\pgfpathlineto{\pgfqpoint{3.601213in}{3.795801in}}%
\pgfpathlineto{\pgfqpoint{3.593741in}{3.780717in}}%
\pgfpathlineto{\pgfqpoint{3.586265in}{3.765779in}}%
\pgfpathlineto{\pgfqpoint{3.578784in}{3.750986in}}%
\pgfpathlineto{\pgfqpoint{3.571299in}{3.736335in}}%
\pgfpathclose%
\pgfusepath{fill}%
\end{pgfscope}%
\begin{pgfscope}%
\pgfpathrectangle{\pgfqpoint{1.254980in}{0.150000in}}{\pgfqpoint{5.490039in}{5.490039in}}%
\pgfusepath{clip}%
\pgfsetbuttcap%
\pgfsetroundjoin%
\definecolor{currentfill}{rgb}{0.150148,0.676631,0.506589}%
\pgfsetfillcolor{currentfill}%
\pgfsetfillopacity{0.700000}%
\pgfsetlinewidth{0.000000pt}%
\definecolor{currentstroke}{rgb}{0.000000,0.000000,0.000000}%
\pgfsetstrokecolor{currentstroke}%
\pgfsetdash{}{0pt}%
\pgfpathmoveto{\pgfqpoint{3.447350in}{4.019853in}}%
\pgfpathlineto{\pgfqpoint{3.460195in}{4.000089in}}%
\pgfpathlineto{\pgfqpoint{3.473036in}{3.980529in}}%
\pgfpathlineto{\pgfqpoint{3.485871in}{3.961170in}}%
\pgfpathlineto{\pgfqpoint{3.498703in}{3.942012in}}%
\pgfpathlineto{\pgfqpoint{3.506191in}{3.957598in}}%
\pgfpathlineto{\pgfqpoint{3.513674in}{3.973344in}}%
\pgfpathlineto{\pgfqpoint{3.521153in}{3.989249in}}%
\pgfpathlineto{\pgfqpoint{3.528627in}{4.005316in}}%
\pgfpathlineto{\pgfqpoint{3.515805in}{4.024660in}}%
\pgfpathlineto{\pgfqpoint{3.502980in}{4.044204in}}%
\pgfpathlineto{\pgfqpoint{3.490149in}{4.063950in}}%
\pgfpathlineto{\pgfqpoint{3.477314in}{4.083900in}}%
\pgfpathlineto{\pgfqpoint{3.469830in}{4.067641in}}%
\pgfpathlineto{\pgfqpoint{3.462342in}{4.051548in}}%
\pgfpathlineto{\pgfqpoint{3.454848in}{4.035619in}}%
\pgfpathlineto{\pgfqpoint{3.447350in}{4.019853in}}%
\pgfpathclose%
\pgfusepath{fill}%
\end{pgfscope}%
\begin{pgfscope}%
\pgfpathrectangle{\pgfqpoint{1.254980in}{0.150000in}}{\pgfqpoint{5.490039in}{5.490039in}}%
\pgfusepath{clip}%
\pgfsetbuttcap%
\pgfsetroundjoin%
\definecolor{currentfill}{rgb}{0.190631,0.407061,0.556089}%
\pgfsetfillcolor{currentfill}%
\pgfsetfillopacity{0.700000}%
\pgfsetlinewidth{0.000000pt}%
\definecolor{currentstroke}{rgb}{0.000000,0.000000,0.000000}%
\pgfsetstrokecolor{currentstroke}%
\pgfsetdash{}{0pt}%
\pgfpathmoveto{\pgfqpoint{3.929301in}{3.320594in}}%
\pgfpathlineto{\pgfqpoint{3.942080in}{3.308186in}}%
\pgfpathlineto{\pgfqpoint{3.954860in}{3.295937in}}%
\pgfpathlineto{\pgfqpoint{3.967641in}{3.283846in}}%
\pgfpathlineto{\pgfqpoint{3.980422in}{3.271913in}}%
\pgfpathlineto{\pgfqpoint{3.987826in}{3.285302in}}%
\pgfpathlineto{\pgfqpoint{3.995225in}{3.298800in}}%
\pgfpathlineto{\pgfqpoint{4.002621in}{3.312411in}}%
\pgfpathlineto{\pgfqpoint{4.010014in}{3.326134in}}%
\pgfpathlineto{\pgfqpoint{3.997242in}{3.338230in}}%
\pgfpathlineto{\pgfqpoint{3.984471in}{3.350483in}}%
\pgfpathlineto{\pgfqpoint{3.971701in}{3.362894in}}%
\pgfpathlineto{\pgfqpoint{3.958932in}{3.375464in}}%
\pgfpathlineto{\pgfqpoint{3.951530in}{3.361573in}}%
\pgfpathlineto{\pgfqpoint{3.944124in}{3.347798in}}%
\pgfpathlineto{\pgfqpoint{3.936715in}{3.334139in}}%
\pgfpathlineto{\pgfqpoint{3.929301in}{3.320594in}}%
\pgfpathclose%
\pgfusepath{fill}%
\end{pgfscope}%
\begin{pgfscope}%
\pgfpathrectangle{\pgfqpoint{1.254980in}{0.150000in}}{\pgfqpoint{5.490039in}{5.490039in}}%
\pgfusepath{clip}%
\pgfsetbuttcap%
\pgfsetroundjoin%
\definecolor{currentfill}{rgb}{0.218130,0.347432,0.550038}%
\pgfsetfillcolor{currentfill}%
\pgfsetfillopacity{0.700000}%
\pgfsetlinewidth{0.000000pt}%
\definecolor{currentstroke}{rgb}{0.000000,0.000000,0.000000}%
\pgfsetstrokecolor{currentstroke}%
\pgfsetdash{}{0pt}%
\pgfpathmoveto{\pgfqpoint{4.639308in}{3.167690in}}%
\pgfpathlineto{\pgfqpoint{4.652175in}{3.160660in}}%
\pgfpathlineto{\pgfqpoint{4.665049in}{3.153761in}}%
\pgfpathlineto{\pgfqpoint{4.677928in}{3.146993in}}%
\pgfpathlineto{\pgfqpoint{4.690812in}{3.140355in}}%
\pgfpathlineto{\pgfqpoint{4.698048in}{3.153741in}}%
\pgfpathlineto{\pgfqpoint{4.705283in}{3.167234in}}%
\pgfpathlineto{\pgfqpoint{4.712515in}{3.180839in}}%
\pgfpathlineto{\pgfqpoint{4.699639in}{3.187654in}}%
\pgfpathlineto{\pgfqpoint{4.686768in}{3.194600in}}%
\pgfpathlineto{\pgfqpoint{4.673902in}{3.201676in}}%
\pgfpathlineto{\pgfqpoint{4.661042in}{3.208883in}}%
\pgfpathlineto{\pgfqpoint{4.653799in}{3.195039in}}%
\pgfpathlineto{\pgfqpoint{4.646554in}{3.181309in}}%
\pgfpathlineto{\pgfqpoint{4.639308in}{3.167690in}}%
\pgfpathclose%
\pgfusepath{fill}%
\end{pgfscope}%
\begin{pgfscope}%
\pgfpathrectangle{\pgfqpoint{1.254980in}{0.150000in}}{\pgfqpoint{5.490039in}{5.490039in}}%
\pgfusepath{clip}%
\pgfsetbuttcap%
\pgfsetroundjoin%
\definecolor{currentfill}{rgb}{0.218130,0.347432,0.550038}%
\pgfsetfillcolor{currentfill}%
\pgfsetfillopacity{0.700000}%
\pgfsetlinewidth{0.000000pt}%
\definecolor{currentstroke}{rgb}{0.000000,0.000000,0.000000}%
\pgfsetstrokecolor{currentstroke}%
\pgfsetdash{}{0pt}%
\pgfpathmoveto{\pgfqpoint{4.295183in}{3.168631in}}%
\pgfpathlineto{\pgfqpoint{4.307993in}{3.159492in}}%
\pgfpathlineto{\pgfqpoint{4.320806in}{3.150496in}}%
\pgfpathlineto{\pgfqpoint{4.333623in}{3.141641in}}%
\pgfpathlineto{\pgfqpoint{4.346444in}{3.132927in}}%
\pgfpathlineto{\pgfqpoint{4.353761in}{3.146123in}}%
\pgfpathlineto{\pgfqpoint{4.361075in}{3.159421in}}%
\pgfpathlineto{\pgfqpoint{4.368387in}{3.172821in}}%
\pgfpathlineto{\pgfqpoint{4.375696in}{3.186327in}}%
\pgfpathlineto{\pgfqpoint{4.362885in}{3.195234in}}%
\pgfpathlineto{\pgfqpoint{4.350078in}{3.204281in}}%
\pgfpathlineto{\pgfqpoint{4.337274in}{3.213470in}}%
\pgfpathlineto{\pgfqpoint{4.324474in}{3.222800in}}%
\pgfpathlineto{\pgfqpoint{4.317156in}{3.209096in}}%
\pgfpathlineto{\pgfqpoint{4.309834in}{3.195501in}}%
\pgfpathlineto{\pgfqpoint{4.302510in}{3.182014in}}%
\pgfpathlineto{\pgfqpoint{4.295183in}{3.168631in}}%
\pgfpathclose%
\pgfusepath{fill}%
\end{pgfscope}%
\begin{pgfscope}%
\pgfpathrectangle{\pgfqpoint{1.254980in}{0.150000in}}{\pgfqpoint{5.490039in}{5.490039in}}%
\pgfusepath{clip}%
\pgfsetbuttcap%
\pgfsetroundjoin%
\definecolor{currentfill}{rgb}{0.220057,0.343307,0.549413}%
\pgfsetfillcolor{currentfill}%
\pgfsetfillopacity{0.700000}%
\pgfsetlinewidth{0.000000pt}%
\definecolor{currentstroke}{rgb}{0.000000,0.000000,0.000000}%
\pgfsetstrokecolor{currentstroke}%
\pgfsetdash{}{0pt}%
\pgfpathmoveto{\pgfqpoint{4.426978in}{3.152096in}}%
\pgfpathlineto{\pgfqpoint{4.439808in}{3.143884in}}%
\pgfpathlineto{\pgfqpoint{4.452643in}{3.135809in}}%
\pgfpathlineto{\pgfqpoint{4.465483in}{3.127871in}}%
\pgfpathlineto{\pgfqpoint{4.478327in}{3.120069in}}%
\pgfpathlineto{\pgfqpoint{4.485613in}{3.133279in}}%
\pgfpathlineto{\pgfqpoint{4.492896in}{3.146590in}}%
\pgfpathlineto{\pgfqpoint{4.500177in}{3.160004in}}%
\pgfpathlineto{\pgfqpoint{4.507456in}{3.173525in}}%
\pgfpathlineto{\pgfqpoint{4.494622in}{3.181534in}}%
\pgfpathlineto{\pgfqpoint{4.481793in}{3.189680in}}%
\pgfpathlineto{\pgfqpoint{4.468968in}{3.197962in}}%
\pgfpathlineto{\pgfqpoint{4.456147in}{3.206381in}}%
\pgfpathlineto{\pgfqpoint{4.448858in}{3.192648in}}%
\pgfpathlineto{\pgfqpoint{4.441567in}{3.179024in}}%
\pgfpathlineto{\pgfqpoint{4.434274in}{3.165507in}}%
\pgfpathlineto{\pgfqpoint{4.426978in}{3.152096in}}%
\pgfpathclose%
\pgfusepath{fill}%
\end{pgfscope}%
\begin{pgfscope}%
\pgfpathrectangle{\pgfqpoint{1.254980in}{0.150000in}}{\pgfqpoint{5.490039in}{5.490039in}}%
\pgfusepath{clip}%
\pgfsetbuttcap%
\pgfsetroundjoin%
\definecolor{currentfill}{rgb}{0.120565,0.596422,0.543611}%
\pgfsetfillcolor{currentfill}%
\pgfsetfillopacity{0.700000}%
\pgfsetlinewidth{0.000000pt}%
\definecolor{currentstroke}{rgb}{0.000000,0.000000,0.000000}%
\pgfsetstrokecolor{currentstroke}%
\pgfsetdash{}{0pt}%
\pgfpathmoveto{\pgfqpoint{3.520031in}{3.807222in}}%
\pgfpathlineto{\pgfqpoint{3.532853in}{3.789213in}}%
\pgfpathlineto{\pgfqpoint{3.545671in}{3.771396in}}%
\pgfpathlineto{\pgfqpoint{3.558487in}{3.753771in}}%
\pgfpathlineto{\pgfqpoint{3.571299in}{3.736335in}}%
\pgfpathlineto{\pgfqpoint{3.578784in}{3.750986in}}%
\pgfpathlineto{\pgfqpoint{3.586265in}{3.765779in}}%
\pgfpathlineto{\pgfqpoint{3.593741in}{3.780717in}}%
\pgfpathlineto{\pgfqpoint{3.601213in}{3.795801in}}%
\pgfpathlineto{\pgfqpoint{3.588412in}{3.813403in}}%
\pgfpathlineto{\pgfqpoint{3.575607in}{3.831196in}}%
\pgfpathlineto{\pgfqpoint{3.562799in}{3.849179in}}%
\pgfpathlineto{\pgfqpoint{3.549987in}{3.867356in}}%
\pgfpathlineto{\pgfqpoint{3.542505in}{3.852099in}}%
\pgfpathlineto{\pgfqpoint{3.535018in}{3.836992in}}%
\pgfpathlineto{\pgfqpoint{3.527527in}{3.822034in}}%
\pgfpathlineto{\pgfqpoint{3.520031in}{3.807222in}}%
\pgfpathclose%
\pgfusepath{fill}%
\end{pgfscope}%
\begin{pgfscope}%
\pgfpathrectangle{\pgfqpoint{1.254980in}{0.150000in}}{\pgfqpoint{5.490039in}{5.490039in}}%
\pgfusepath{clip}%
\pgfsetbuttcap%
\pgfsetroundjoin%
\definecolor{currentfill}{rgb}{0.212395,0.359683,0.551710}%
\pgfsetfillcolor{currentfill}%
\pgfsetfillopacity{0.700000}%
\pgfsetlinewidth{0.000000pt}%
\definecolor{currentstroke}{rgb}{0.000000,0.000000,0.000000}%
\pgfsetstrokecolor{currentstroke}%
\pgfsetdash{}{0pt}%
\pgfpathmoveto{\pgfqpoint{4.163398in}{3.192983in}}%
\pgfpathlineto{\pgfqpoint{4.176193in}{3.182864in}}%
\pgfpathlineto{\pgfqpoint{4.188992in}{3.172892in}}%
\pgfpathlineto{\pgfqpoint{4.201793in}{3.163067in}}%
\pgfpathlineto{\pgfqpoint{4.214597in}{3.153386in}}%
\pgfpathlineto{\pgfqpoint{4.221946in}{3.166540in}}%
\pgfpathlineto{\pgfqpoint{4.229293in}{3.179794in}}%
\pgfpathlineto{\pgfqpoint{4.236636in}{3.193152in}}%
\pgfpathlineto{\pgfqpoint{4.243976in}{3.206614in}}%
\pgfpathlineto{\pgfqpoint{4.231182in}{3.216471in}}%
\pgfpathlineto{\pgfqpoint{4.218390in}{3.226474in}}%
\pgfpathlineto{\pgfqpoint{4.205602in}{3.236623in}}%
\pgfpathlineto{\pgfqpoint{4.192816in}{3.246919in}}%
\pgfpathlineto{\pgfqpoint{4.185466in}{3.233273in}}%
\pgfpathlineto{\pgfqpoint{4.178113in}{3.219737in}}%
\pgfpathlineto{\pgfqpoint{4.170757in}{3.206308in}}%
\pgfpathlineto{\pgfqpoint{4.163398in}{3.192983in}}%
\pgfpathclose%
\pgfusepath{fill}%
\end{pgfscope}%
\begin{pgfscope}%
\pgfpathrectangle{\pgfqpoint{1.254980in}{0.150000in}}{\pgfqpoint{5.490039in}{5.490039in}}%
\pgfusepath{clip}%
\pgfsetbuttcap%
\pgfsetroundjoin%
\definecolor{currentfill}{rgb}{0.197636,0.391528,0.554969}%
\pgfsetfillcolor{currentfill}%
\pgfsetfillopacity{0.700000}%
\pgfsetlinewidth{0.000000pt}%
\definecolor{currentstroke}{rgb}{0.000000,0.000000,0.000000}%
\pgfsetstrokecolor{currentstroke}%
\pgfsetdash{}{0pt}%
\pgfpathmoveto{\pgfqpoint{3.980422in}{3.271913in}}%
\pgfpathlineto{\pgfqpoint{3.993205in}{3.260136in}}%
\pgfpathlineto{\pgfqpoint{4.005989in}{3.248515in}}%
\pgfpathlineto{\pgfqpoint{4.018775in}{3.237048in}}%
\pgfpathlineto{\pgfqpoint{4.031562in}{3.225736in}}%
\pgfpathlineto{\pgfqpoint{4.038955in}{3.238969in}}%
\pgfpathlineto{\pgfqpoint{4.046345in}{3.252308in}}%
\pgfpathlineto{\pgfqpoint{4.053731in}{3.265755in}}%
\pgfpathlineto{\pgfqpoint{4.061113in}{3.279311in}}%
\pgfpathlineto{\pgfqpoint{4.048336in}{3.290784in}}%
\pgfpathlineto{\pgfqpoint{4.035561in}{3.302412in}}%
\pgfpathlineto{\pgfqpoint{4.022786in}{3.314195in}}%
\pgfpathlineto{\pgfqpoint{4.010014in}{3.326134in}}%
\pgfpathlineto{\pgfqpoint{4.002621in}{3.312411in}}%
\pgfpathlineto{\pgfqpoint{3.995225in}{3.298800in}}%
\pgfpathlineto{\pgfqpoint{3.987826in}{3.285302in}}%
\pgfpathlineto{\pgfqpoint{3.980422in}{3.271913in}}%
\pgfpathclose%
\pgfusepath{fill}%
\end{pgfscope}%
\begin{pgfscope}%
\pgfpathrectangle{\pgfqpoint{1.254980in}{0.150000in}}{\pgfqpoint{5.490039in}{5.490039in}}%
\pgfusepath{clip}%
\pgfsetbuttcap%
\pgfsetroundjoin%
\definecolor{currentfill}{rgb}{0.221989,0.339161,0.548752}%
\pgfsetfillcolor{currentfill}%
\pgfsetfillopacity{0.700000}%
\pgfsetlinewidth{0.000000pt}%
\definecolor{currentstroke}{rgb}{0.000000,0.000000,0.000000}%
\pgfsetstrokecolor{currentstroke}%
\pgfsetdash{}{0pt}%
\pgfpathmoveto{\pgfqpoint{4.558838in}{3.142838in}}%
\pgfpathlineto{\pgfqpoint{4.571695in}{3.135502in}}%
\pgfpathlineto{\pgfqpoint{4.584558in}{3.128298in}}%
\pgfpathlineto{\pgfqpoint{4.597425in}{3.121227in}}%
\pgfpathlineto{\pgfqpoint{4.610298in}{3.114289in}}%
\pgfpathlineto{\pgfqpoint{4.617554in}{3.127483in}}%
\pgfpathlineto{\pgfqpoint{4.624807in}{3.140780in}}%
\pgfpathlineto{\pgfqpoint{4.632059in}{3.154181in}}%
\pgfpathlineto{\pgfqpoint{4.639308in}{3.167690in}}%
\pgfpathlineto{\pgfqpoint{4.626445in}{3.174852in}}%
\pgfpathlineto{\pgfqpoint{4.613587in}{3.182146in}}%
\pgfpathlineto{\pgfqpoint{4.600735in}{3.189572in}}%
\pgfpathlineto{\pgfqpoint{4.587888in}{3.197131in}}%
\pgfpathlineto{\pgfqpoint{4.580629in}{3.183394in}}%
\pgfpathlineto{\pgfqpoint{4.573367in}{3.169767in}}%
\pgfpathlineto{\pgfqpoint{4.566104in}{3.156250in}}%
\pgfpathlineto{\pgfqpoint{4.558838in}{3.142838in}}%
\pgfpathclose%
\pgfusepath{fill}%
\end{pgfscope}%
\begin{pgfscope}%
\pgfpathrectangle{\pgfqpoint{1.254980in}{0.150000in}}{\pgfqpoint{5.490039in}{5.490039in}}%
\pgfusepath{clip}%
\pgfsetbuttcap%
\pgfsetroundjoin%
\definecolor{currentfill}{rgb}{0.191090,0.708366,0.482284}%
\pgfsetfillcolor{currentfill}%
\pgfsetfillopacity{0.700000}%
\pgfsetlinewidth{0.000000pt}%
\definecolor{currentstroke}{rgb}{0.000000,0.000000,0.000000}%
\pgfsetstrokecolor{currentstroke}%
\pgfsetdash{}{0pt}%
\pgfpathmoveto{\pgfqpoint{3.395919in}{4.100970in}}%
\pgfpathlineto{\pgfqpoint{3.408785in}{4.080378in}}%
\pgfpathlineto{\pgfqpoint{3.421645in}{4.059996in}}%
\pgfpathlineto{\pgfqpoint{3.434500in}{4.039822in}}%
\pgfpathlineto{\pgfqpoint{3.447350in}{4.019853in}}%
\pgfpathlineto{\pgfqpoint{3.454848in}{4.035619in}}%
\pgfpathlineto{\pgfqpoint{3.462342in}{4.051548in}}%
\pgfpathlineto{\pgfqpoint{3.469830in}{4.067641in}}%
\pgfpathlineto{\pgfqpoint{3.477314in}{4.083900in}}%
\pgfpathlineto{\pgfqpoint{3.464474in}{4.104055in}}%
\pgfpathlineto{\pgfqpoint{3.451629in}{4.124416in}}%
\pgfpathlineto{\pgfqpoint{3.438779in}{4.144985in}}%
\pgfpathlineto{\pgfqpoint{3.425924in}{4.165764in}}%
\pgfpathlineto{\pgfqpoint{3.418430in}{4.149311in}}%
\pgfpathlineto{\pgfqpoint{3.410931in}{4.133029in}}%
\pgfpathlineto{\pgfqpoint{3.403428in}{4.116916in}}%
\pgfpathlineto{\pgfqpoint{3.395919in}{4.100970in}}%
\pgfpathclose%
\pgfusepath{fill}%
\end{pgfscope}%
\begin{pgfscope}%
\pgfpathrectangle{\pgfqpoint{1.254980in}{0.150000in}}{\pgfqpoint{5.490039in}{5.490039in}}%
\pgfusepath{clip}%
\pgfsetbuttcap%
\pgfsetroundjoin%
\definecolor{currentfill}{rgb}{0.120638,0.625828,0.533488}%
\pgfsetfillcolor{currentfill}%
\pgfsetfillopacity{0.700000}%
\pgfsetlinewidth{0.000000pt}%
\definecolor{currentstroke}{rgb}{0.000000,0.000000,0.000000}%
\pgfsetstrokecolor{currentstroke}%
\pgfsetdash{}{0pt}%
\pgfpathmoveto{\pgfqpoint{3.468704in}{3.881208in}}%
\pgfpathlineto{\pgfqpoint{3.481542in}{3.862417in}}%
\pgfpathlineto{\pgfqpoint{3.494376in}{3.843823in}}%
\pgfpathlineto{\pgfqpoint{3.507205in}{3.825425in}}%
\pgfpathlineto{\pgfqpoint{3.520031in}{3.807222in}}%
\pgfpathlineto{\pgfqpoint{3.527527in}{3.822034in}}%
\pgfpathlineto{\pgfqpoint{3.535018in}{3.836992in}}%
\pgfpathlineto{\pgfqpoint{3.542505in}{3.852099in}}%
\pgfpathlineto{\pgfqpoint{3.549987in}{3.867356in}}%
\pgfpathlineto{\pgfqpoint{3.537172in}{3.885726in}}%
\pgfpathlineto{\pgfqpoint{3.524353in}{3.904291in}}%
\pgfpathlineto{\pgfqpoint{3.511530in}{3.923052in}}%
\pgfpathlineto{\pgfqpoint{3.498703in}{3.942012in}}%
\pgfpathlineto{\pgfqpoint{3.491210in}{3.926581in}}%
\pgfpathlineto{\pgfqpoint{3.483713in}{3.911305in}}%
\pgfpathlineto{\pgfqpoint{3.476211in}{3.896181in}}%
\pgfpathlineto{\pgfqpoint{3.468704in}{3.881208in}}%
\pgfpathclose%
\pgfusepath{fill}%
\end{pgfscope}%
\begin{pgfscope}%
\pgfpathrectangle{\pgfqpoint{1.254980in}{0.150000in}}{\pgfqpoint{5.490039in}{5.490039in}}%
\pgfusepath{clip}%
\pgfsetbuttcap%
\pgfsetroundjoin%
\definecolor{currentfill}{rgb}{0.206756,0.371758,0.553117}%
\pgfsetfillcolor{currentfill}%
\pgfsetfillopacity{0.700000}%
\pgfsetlinewidth{0.000000pt}%
\definecolor{currentstroke}{rgb}{0.000000,0.000000,0.000000}%
\pgfsetstrokecolor{currentstroke}%
\pgfsetdash{}{0pt}%
\pgfpathmoveto{\pgfqpoint{4.031562in}{3.225736in}}%
\pgfpathlineto{\pgfqpoint{4.044351in}{3.214578in}}%
\pgfpathlineto{\pgfqpoint{4.057141in}{3.203572in}}%
\pgfpathlineto{\pgfqpoint{4.069933in}{3.192718in}}%
\pgfpathlineto{\pgfqpoint{4.082727in}{3.182015in}}%
\pgfpathlineto{\pgfqpoint{4.090111in}{3.195092in}}%
\pgfpathlineto{\pgfqpoint{4.097490in}{3.208272in}}%
\pgfpathlineto{\pgfqpoint{4.104866in}{3.221555in}}%
\pgfpathlineto{\pgfqpoint{4.112239in}{3.234943in}}%
\pgfpathlineto{\pgfqpoint{4.099455in}{3.245807in}}%
\pgfpathlineto{\pgfqpoint{4.086672in}{3.256823in}}%
\pgfpathlineto{\pgfqpoint{4.073892in}{3.267990in}}%
\pgfpathlineto{\pgfqpoint{4.061113in}{3.279311in}}%
\pgfpathlineto{\pgfqpoint{4.053731in}{3.265755in}}%
\pgfpathlineto{\pgfqpoint{4.046345in}{3.252308in}}%
\pgfpathlineto{\pgfqpoint{4.038955in}{3.238969in}}%
\pgfpathlineto{\pgfqpoint{4.031562in}{3.225736in}}%
\pgfpathclose%
\pgfusepath{fill}%
\end{pgfscope}%
\begin{pgfscope}%
\pgfpathrectangle{\pgfqpoint{1.254980in}{0.150000in}}{\pgfqpoint{5.490039in}{5.490039in}}%
\pgfusepath{clip}%
\pgfsetbuttcap%
\pgfsetroundjoin%
\definecolor{currentfill}{rgb}{0.162142,0.474838,0.558140}%
\pgfsetfillcolor{currentfill}%
\pgfsetfillopacity{0.700000}%
\pgfsetlinewidth{0.000000pt}%
\definecolor{currentstroke}{rgb}{0.000000,0.000000,0.000000}%
\pgfsetstrokecolor{currentstroke}%
\pgfsetdash{}{0pt}%
\pgfpathmoveto{\pgfqpoint{3.694986in}{3.486043in}}%
\pgfpathlineto{\pgfqpoint{3.707779in}{3.471116in}}%
\pgfpathlineto{\pgfqpoint{3.720572in}{3.456363in}}%
\pgfpathlineto{\pgfqpoint{3.733363in}{3.441782in}}%
\pgfpathlineto{\pgfqpoint{3.746154in}{3.427373in}}%
\pgfpathlineto{\pgfqpoint{3.753614in}{3.440903in}}%
\pgfpathlineto{\pgfqpoint{3.761071in}{3.454552in}}%
\pgfpathlineto{\pgfqpoint{3.768523in}{3.468321in}}%
\pgfpathlineto{\pgfqpoint{3.775971in}{3.482213in}}%
\pgfpathlineto{\pgfqpoint{3.763191in}{3.496769in}}%
\pgfpathlineto{\pgfqpoint{3.750410in}{3.511498in}}%
\pgfpathlineto{\pgfqpoint{3.737628in}{3.526399in}}%
\pgfpathlineto{\pgfqpoint{3.724844in}{3.541474in}}%
\pgfpathlineto{\pgfqpoint{3.717386in}{3.527428in}}%
\pgfpathlineto{\pgfqpoint{3.709924in}{3.513509in}}%
\pgfpathlineto{\pgfqpoint{3.702457in}{3.499714in}}%
\pgfpathlineto{\pgfqpoint{3.694986in}{3.486043in}}%
\pgfpathclose%
\pgfusepath{fill}%
\end{pgfscope}%
\begin{pgfscope}%
\pgfpathrectangle{\pgfqpoint{1.254980in}{0.150000in}}{\pgfqpoint{5.490039in}{5.490039in}}%
\pgfusepath{clip}%
\pgfsetbuttcap%
\pgfsetroundjoin%
\definecolor{currentfill}{rgb}{0.223925,0.334994,0.548053}%
\pgfsetfillcolor{currentfill}%
\pgfsetfillopacity{0.700000}%
\pgfsetlinewidth{0.000000pt}%
\definecolor{currentstroke}{rgb}{0.000000,0.000000,0.000000}%
\pgfsetstrokecolor{currentstroke}%
\pgfsetdash{}{0pt}%
\pgfpathmoveto{\pgfqpoint{4.346444in}{3.132927in}}%
\pgfpathlineto{\pgfqpoint{4.359268in}{3.124352in}}%
\pgfpathlineto{\pgfqpoint{4.372097in}{3.115917in}}%
\pgfpathlineto{\pgfqpoint{4.384929in}{3.107621in}}%
\pgfpathlineto{\pgfqpoint{4.397766in}{3.099463in}}%
\pgfpathlineto{\pgfqpoint{4.405073in}{3.112474in}}%
\pgfpathlineto{\pgfqpoint{4.412377in}{3.125581in}}%
\pgfpathlineto{\pgfqpoint{4.419679in}{3.138788in}}%
\pgfpathlineto{\pgfqpoint{4.426978in}{3.152096in}}%
\pgfpathlineto{\pgfqpoint{4.414151in}{3.160446in}}%
\pgfpathlineto{\pgfqpoint{4.401329in}{3.168934in}}%
\pgfpathlineto{\pgfqpoint{4.388510in}{3.177561in}}%
\pgfpathlineto{\pgfqpoint{4.375696in}{3.186327in}}%
\pgfpathlineto{\pgfqpoint{4.368387in}{3.172821in}}%
\pgfpathlineto{\pgfqpoint{4.361075in}{3.159421in}}%
\pgfpathlineto{\pgfqpoint{4.353761in}{3.146123in}}%
\pgfpathlineto{\pgfqpoint{4.346444in}{3.132927in}}%
\pgfpathclose%
\pgfusepath{fill}%
\end{pgfscope}%
\begin{pgfscope}%
\pgfpathrectangle{\pgfqpoint{1.254980in}{0.150000in}}{\pgfqpoint{5.490039in}{5.490039in}}%
\pgfusepath{clip}%
\pgfsetbuttcap%
\pgfsetroundjoin%
\definecolor{currentfill}{rgb}{0.151918,0.500685,0.557587}%
\pgfsetfillcolor{currentfill}%
\pgfsetfillopacity{0.700000}%
\pgfsetlinewidth{0.000000pt}%
\definecolor{currentstroke}{rgb}{0.000000,0.000000,0.000000}%
\pgfsetstrokecolor{currentstroke}%
\pgfsetdash{}{0pt}%
\pgfpathmoveto{\pgfqpoint{3.643795in}{3.547511in}}%
\pgfpathlineto{\pgfqpoint{3.656595in}{3.531878in}}%
\pgfpathlineto{\pgfqpoint{3.669394in}{3.516423in}}%
\pgfpathlineto{\pgfqpoint{3.682190in}{3.501145in}}%
\pgfpathlineto{\pgfqpoint{3.694986in}{3.486043in}}%
\pgfpathlineto{\pgfqpoint{3.702457in}{3.499714in}}%
\pgfpathlineto{\pgfqpoint{3.709924in}{3.513509in}}%
\pgfpathlineto{\pgfqpoint{3.717386in}{3.527428in}}%
\pgfpathlineto{\pgfqpoint{3.724844in}{3.541474in}}%
\pgfpathlineto{\pgfqpoint{3.712060in}{3.556724in}}%
\pgfpathlineto{\pgfqpoint{3.699274in}{3.572150in}}%
\pgfpathlineto{\pgfqpoint{3.686486in}{3.587753in}}%
\pgfpathlineto{\pgfqpoint{3.673696in}{3.603535in}}%
\pgfpathlineto{\pgfqpoint{3.666228in}{3.589334in}}%
\pgfpathlineto{\pgfqpoint{3.658754in}{3.575265in}}%
\pgfpathlineto{\pgfqpoint{3.651277in}{3.561324in}}%
\pgfpathlineto{\pgfqpoint{3.643795in}{3.547511in}}%
\pgfpathclose%
\pgfusepath{fill}%
\end{pgfscope}%
\begin{pgfscope}%
\pgfpathrectangle{\pgfqpoint{1.254980in}{0.150000in}}{\pgfqpoint{5.490039in}{5.490039in}}%
\pgfusepath{clip}%
\pgfsetbuttcap%
\pgfsetroundjoin%
\definecolor{currentfill}{rgb}{0.171176,0.452530,0.557965}%
\pgfsetfillcolor{currentfill}%
\pgfsetfillopacity{0.700000}%
\pgfsetlinewidth{0.000000pt}%
\definecolor{currentstroke}{rgb}{0.000000,0.000000,0.000000}%
\pgfsetstrokecolor{currentstroke}%
\pgfsetdash{}{0pt}%
\pgfpathmoveto{\pgfqpoint{3.746154in}{3.427373in}}%
\pgfpathlineto{\pgfqpoint{3.758944in}{3.413135in}}%
\pgfpathlineto{\pgfqpoint{3.771732in}{3.399067in}}%
\pgfpathlineto{\pgfqpoint{3.784521in}{3.385168in}}%
\pgfpathlineto{\pgfqpoint{3.797309in}{3.371436in}}%
\pgfpathlineto{\pgfqpoint{3.804758in}{3.384825in}}%
\pgfpathlineto{\pgfqpoint{3.812204in}{3.398328in}}%
\pgfpathlineto{\pgfqpoint{3.819646in}{3.411948in}}%
\pgfpathlineto{\pgfqpoint{3.827083in}{3.425687in}}%
\pgfpathlineto{\pgfqpoint{3.814306in}{3.439565in}}%
\pgfpathlineto{\pgfqpoint{3.801528in}{3.453612in}}%
\pgfpathlineto{\pgfqpoint{3.788750in}{3.467828in}}%
\pgfpathlineto{\pgfqpoint{3.775971in}{3.482213in}}%
\pgfpathlineto{\pgfqpoint{3.768523in}{3.468321in}}%
\pgfpathlineto{\pgfqpoint{3.761071in}{3.454552in}}%
\pgfpathlineto{\pgfqpoint{3.753614in}{3.440903in}}%
\pgfpathlineto{\pgfqpoint{3.746154in}{3.427373in}}%
\pgfpathclose%
\pgfusepath{fill}%
\end{pgfscope}%
\begin{pgfscope}%
\pgfpathrectangle{\pgfqpoint{1.254980in}{0.150000in}}{\pgfqpoint{5.490039in}{5.490039in}}%
\pgfusepath{clip}%
\pgfsetbuttcap%
\pgfsetroundjoin%
\definecolor{currentfill}{rgb}{0.220057,0.343307,0.549413}%
\pgfsetfillcolor{currentfill}%
\pgfsetfillopacity{0.700000}%
\pgfsetlinewidth{0.000000pt}%
\definecolor{currentstroke}{rgb}{0.000000,0.000000,0.000000}%
\pgfsetstrokecolor{currentstroke}%
\pgfsetdash{}{0pt}%
\pgfpathmoveto{\pgfqpoint{4.214597in}{3.153386in}}%
\pgfpathlineto{\pgfqpoint{4.227404in}{3.143851in}}%
\pgfpathlineto{\pgfqpoint{4.240214in}{3.134460in}}%
\pgfpathlineto{\pgfqpoint{4.253027in}{3.125213in}}%
\pgfpathlineto{\pgfqpoint{4.265843in}{3.116109in}}%
\pgfpathlineto{\pgfqpoint{4.273183in}{3.129092in}}%
\pgfpathlineto{\pgfqpoint{4.280519in}{3.142172in}}%
\pgfpathlineto{\pgfqpoint{4.287853in}{3.155351in}}%
\pgfpathlineto{\pgfqpoint{4.295183in}{3.168631in}}%
\pgfpathlineto{\pgfqpoint{4.282376in}{3.177911in}}%
\pgfpathlineto{\pgfqpoint{4.269573in}{3.187335in}}%
\pgfpathlineto{\pgfqpoint{4.256773in}{3.196903in}}%
\pgfpathlineto{\pgfqpoint{4.243976in}{3.206614in}}%
\pgfpathlineto{\pgfqpoint{4.236636in}{3.193152in}}%
\pgfpathlineto{\pgfqpoint{4.229293in}{3.179794in}}%
\pgfpathlineto{\pgfqpoint{4.221946in}{3.166540in}}%
\pgfpathlineto{\pgfqpoint{4.214597in}{3.153386in}}%
\pgfpathclose%
\pgfusepath{fill}%
\end{pgfscope}%
\begin{pgfscope}%
\pgfpathrectangle{\pgfqpoint{1.254980in}{0.150000in}}{\pgfqpoint{5.490039in}{5.490039in}}%
\pgfusepath{clip}%
\pgfsetbuttcap%
\pgfsetroundjoin%
\definecolor{currentfill}{rgb}{0.141935,0.526453,0.555991}%
\pgfsetfillcolor{currentfill}%
\pgfsetfillopacity{0.700000}%
\pgfsetlinewidth{0.000000pt}%
\definecolor{currentstroke}{rgb}{0.000000,0.000000,0.000000}%
\pgfsetstrokecolor{currentstroke}%
\pgfsetdash{}{0pt}%
\pgfpathmoveto{\pgfqpoint{3.592573in}{3.611846in}}%
\pgfpathlineto{\pgfqpoint{3.605382in}{3.595489in}}%
\pgfpathlineto{\pgfqpoint{3.618188in}{3.579316in}}%
\pgfpathlineto{\pgfqpoint{3.630993in}{3.563323in}}%
\pgfpathlineto{\pgfqpoint{3.643795in}{3.547511in}}%
\pgfpathlineto{\pgfqpoint{3.651277in}{3.561324in}}%
\pgfpathlineto{\pgfqpoint{3.658754in}{3.575265in}}%
\pgfpathlineto{\pgfqpoint{3.666228in}{3.589334in}}%
\pgfpathlineto{\pgfqpoint{3.673696in}{3.603535in}}%
\pgfpathlineto{\pgfqpoint{3.660905in}{3.619495in}}%
\pgfpathlineto{\pgfqpoint{3.648111in}{3.635636in}}%
\pgfpathlineto{\pgfqpoint{3.635315in}{3.651959in}}%
\pgfpathlineto{\pgfqpoint{3.622517in}{3.668464in}}%
\pgfpathlineto{\pgfqpoint{3.615038in}{3.654109in}}%
\pgfpathlineto{\pgfqpoint{3.607554in}{3.639889in}}%
\pgfpathlineto{\pgfqpoint{3.600066in}{3.625801in}}%
\pgfpathlineto{\pgfqpoint{3.592573in}{3.611846in}}%
\pgfpathclose%
\pgfusepath{fill}%
\end{pgfscope}%
\begin{pgfscope}%
\pgfpathrectangle{\pgfqpoint{1.254980in}{0.150000in}}{\pgfqpoint{5.490039in}{5.490039in}}%
\pgfusepath{clip}%
\pgfsetbuttcap%
\pgfsetroundjoin%
\definecolor{currentfill}{rgb}{0.179019,0.433756,0.557430}%
\pgfsetfillcolor{currentfill}%
\pgfsetfillopacity{0.700000}%
\pgfsetlinewidth{0.000000pt}%
\definecolor{currentstroke}{rgb}{0.000000,0.000000,0.000000}%
\pgfsetstrokecolor{currentstroke}%
\pgfsetdash{}{0pt}%
\pgfpathmoveto{\pgfqpoint{3.797309in}{3.371436in}}%
\pgfpathlineto{\pgfqpoint{3.810096in}{3.357872in}}%
\pgfpathlineto{\pgfqpoint{3.822883in}{3.344473in}}%
\pgfpathlineto{\pgfqpoint{3.835670in}{3.331240in}}%
\pgfpathlineto{\pgfqpoint{3.848458in}{3.318171in}}%
\pgfpathlineto{\pgfqpoint{3.855897in}{3.331418in}}%
\pgfpathlineto{\pgfqpoint{3.863332in}{3.344777in}}%
\pgfpathlineto{\pgfqpoint{3.870763in}{3.358248in}}%
\pgfpathlineto{\pgfqpoint{3.878191in}{3.371834in}}%
\pgfpathlineto{\pgfqpoint{3.865414in}{3.385050in}}%
\pgfpathlineto{\pgfqpoint{3.852637in}{3.398430in}}%
\pgfpathlineto{\pgfqpoint{3.839860in}{3.411975in}}%
\pgfpathlineto{\pgfqpoint{3.827083in}{3.425687in}}%
\pgfpathlineto{\pgfqpoint{3.819646in}{3.411948in}}%
\pgfpathlineto{\pgfqpoint{3.812204in}{3.398328in}}%
\pgfpathlineto{\pgfqpoint{3.804758in}{3.384825in}}%
\pgfpathlineto{\pgfqpoint{3.797309in}{3.371436in}}%
\pgfpathclose%
\pgfusepath{fill}%
\end{pgfscope}%
\begin{pgfscope}%
\pgfpathrectangle{\pgfqpoint{1.254980in}{0.150000in}}{\pgfqpoint{5.490039in}{5.490039in}}%
\pgfusepath{clip}%
\pgfsetbuttcap%
\pgfsetroundjoin%
\definecolor{currentfill}{rgb}{0.225863,0.330805,0.547314}%
\pgfsetfillcolor{currentfill}%
\pgfsetfillopacity{0.700000}%
\pgfsetlinewidth{0.000000pt}%
\definecolor{currentstroke}{rgb}{0.000000,0.000000,0.000000}%
\pgfsetstrokecolor{currentstroke}%
\pgfsetdash{}{0pt}%
\pgfpathmoveto{\pgfqpoint{4.478327in}{3.120069in}}%
\pgfpathlineto{\pgfqpoint{4.491175in}{3.112403in}}%
\pgfpathlineto{\pgfqpoint{4.504028in}{3.104872in}}%
\pgfpathlineto{\pgfqpoint{4.516886in}{3.097475in}}%
\pgfpathlineto{\pgfqpoint{4.529749in}{3.090212in}}%
\pgfpathlineto{\pgfqpoint{4.537025in}{3.103220in}}%
\pgfpathlineto{\pgfqpoint{4.544298in}{3.116326in}}%
\pgfpathlineto{\pgfqpoint{4.551569in}{3.129531in}}%
\pgfpathlineto{\pgfqpoint{4.558838in}{3.142838in}}%
\pgfpathlineto{\pgfqpoint{4.545985in}{3.150308in}}%
\pgfpathlineto{\pgfqpoint{4.533137in}{3.157912in}}%
\pgfpathlineto{\pgfqpoint{4.520294in}{3.165651in}}%
\pgfpathlineto{\pgfqpoint{4.507456in}{3.173525in}}%
\pgfpathlineto{\pgfqpoint{4.500177in}{3.160004in}}%
\pgfpathlineto{\pgfqpoint{4.492896in}{3.146590in}}%
\pgfpathlineto{\pgfqpoint{4.485613in}{3.133279in}}%
\pgfpathlineto{\pgfqpoint{4.478327in}{3.120069in}}%
\pgfpathclose%
\pgfusepath{fill}%
\end{pgfscope}%
\begin{pgfscope}%
\pgfpathrectangle{\pgfqpoint{1.254980in}{0.150000in}}{\pgfqpoint{5.490039in}{5.490039in}}%
\pgfusepath{clip}%
\pgfsetbuttcap%
\pgfsetroundjoin%
\definecolor{currentfill}{rgb}{0.221989,0.339161,0.548752}%
\pgfsetfillcolor{currentfill}%
\pgfsetfillopacity{0.700000}%
\pgfsetlinewidth{0.000000pt}%
\definecolor{currentstroke}{rgb}{0.000000,0.000000,0.000000}%
\pgfsetstrokecolor{currentstroke}%
\pgfsetdash{}{0pt}%
\pgfpathmoveto{\pgfqpoint{4.690812in}{3.140355in}}%
\pgfpathlineto{\pgfqpoint{4.703702in}{3.133846in}}%
\pgfpathlineto{\pgfqpoint{4.716598in}{3.127467in}}%
\pgfpathlineto{\pgfqpoint{4.729499in}{3.121217in}}%
\pgfpathlineto{\pgfqpoint{4.742407in}{3.115095in}}%
\pgfpathlineto{\pgfqpoint{4.749633in}{3.128249in}}%
\pgfpathlineto{\pgfqpoint{4.756856in}{3.141506in}}%
\pgfpathlineto{\pgfqpoint{4.764078in}{3.154870in}}%
\pgfpathlineto{\pgfqpoint{4.751179in}{3.161169in}}%
\pgfpathlineto{\pgfqpoint{4.738285in}{3.167597in}}%
\pgfpathlineto{\pgfqpoint{4.725397in}{3.174153in}}%
\pgfpathlineto{\pgfqpoint{4.712515in}{3.180839in}}%
\pgfpathlineto{\pgfqpoint{4.705283in}{3.167234in}}%
\pgfpathlineto{\pgfqpoint{4.698048in}{3.153741in}}%
\pgfpathlineto{\pgfqpoint{4.690812in}{3.140355in}}%
\pgfpathclose%
\pgfusepath{fill}%
\end{pgfscope}%
\begin{pgfscope}%
\pgfpathrectangle{\pgfqpoint{1.254980in}{0.150000in}}{\pgfqpoint{5.490039in}{5.490039in}}%
\pgfusepath{clip}%
\pgfsetbuttcap%
\pgfsetroundjoin%
\definecolor{currentfill}{rgb}{0.134692,0.658636,0.517649}%
\pgfsetfillcolor{currentfill}%
\pgfsetfillopacity{0.700000}%
\pgfsetlinewidth{0.000000pt}%
\definecolor{currentstroke}{rgb}{0.000000,0.000000,0.000000}%
\pgfsetstrokecolor{currentstroke}%
\pgfsetdash{}{0pt}%
\pgfpathmoveto{\pgfqpoint{3.417309in}{3.958375in}}%
\pgfpathlineto{\pgfqpoint{3.430165in}{3.938780in}}%
\pgfpathlineto{\pgfqpoint{3.443016in}{3.919388in}}%
\pgfpathlineto{\pgfqpoint{3.455862in}{3.900198in}}%
\pgfpathlineto{\pgfqpoint{3.468704in}{3.881208in}}%
\pgfpathlineto{\pgfqpoint{3.476211in}{3.896181in}}%
\pgfpathlineto{\pgfqpoint{3.483713in}{3.911305in}}%
\pgfpathlineto{\pgfqpoint{3.491210in}{3.926581in}}%
\pgfpathlineto{\pgfqpoint{3.498703in}{3.942012in}}%
\pgfpathlineto{\pgfqpoint{3.485871in}{3.961170in}}%
\pgfpathlineto{\pgfqpoint{3.473036in}{3.980529in}}%
\pgfpathlineto{\pgfqpoint{3.460195in}{4.000089in}}%
\pgfpathlineto{\pgfqpoint{3.447350in}{4.019853in}}%
\pgfpathlineto{\pgfqpoint{3.439847in}{4.004248in}}%
\pgfpathlineto{\pgfqpoint{3.432339in}{3.988801in}}%
\pgfpathlineto{\pgfqpoint{3.424827in}{3.973511in}}%
\pgfpathlineto{\pgfqpoint{3.417309in}{3.958375in}}%
\pgfpathclose%
\pgfusepath{fill}%
\end{pgfscope}%
\begin{pgfscope}%
\pgfpathrectangle{\pgfqpoint{1.254980in}{0.150000in}}{\pgfqpoint{5.490039in}{5.490039in}}%
\pgfusepath{clip}%
\pgfsetbuttcap%
\pgfsetroundjoin%
\definecolor{currentfill}{rgb}{0.132444,0.552216,0.553018}%
\pgfsetfillcolor{currentfill}%
\pgfsetfillopacity{0.700000}%
\pgfsetlinewidth{0.000000pt}%
\definecolor{currentstroke}{rgb}{0.000000,0.000000,0.000000}%
\pgfsetstrokecolor{currentstroke}%
\pgfsetdash{}{0pt}%
\pgfpathmoveto{\pgfqpoint{3.541311in}{3.679119in}}%
\pgfpathlineto{\pgfqpoint{3.554131in}{3.662021in}}%
\pgfpathlineto{\pgfqpoint{3.566947in}{3.645110in}}%
\pgfpathlineto{\pgfqpoint{3.579762in}{3.628385in}}%
\pgfpathlineto{\pgfqpoint{3.592573in}{3.611846in}}%
\pgfpathlineto{\pgfqpoint{3.600066in}{3.625801in}}%
\pgfpathlineto{\pgfqpoint{3.607554in}{3.639889in}}%
\pgfpathlineto{\pgfqpoint{3.615038in}{3.654109in}}%
\pgfpathlineto{\pgfqpoint{3.622517in}{3.668464in}}%
\pgfpathlineto{\pgfqpoint{3.609717in}{3.685153in}}%
\pgfpathlineto{\pgfqpoint{3.596914in}{3.702027in}}%
\pgfpathlineto{\pgfqpoint{3.584108in}{3.719087in}}%
\pgfpathlineto{\pgfqpoint{3.571299in}{3.736335in}}%
\pgfpathlineto{\pgfqpoint{3.563809in}{3.721824in}}%
\pgfpathlineto{\pgfqpoint{3.556314in}{3.707452in}}%
\pgfpathlineto{\pgfqpoint{3.548815in}{3.693218in}}%
\pgfpathlineto{\pgfqpoint{3.541311in}{3.679119in}}%
\pgfpathclose%
\pgfusepath{fill}%
\end{pgfscope}%
\begin{pgfscope}%
\pgfpathrectangle{\pgfqpoint{1.254980in}{0.150000in}}{\pgfqpoint{5.490039in}{5.490039in}}%
\pgfusepath{clip}%
\pgfsetbuttcap%
\pgfsetroundjoin%
\definecolor{currentfill}{rgb}{0.188923,0.410910,0.556326}%
\pgfsetfillcolor{currentfill}%
\pgfsetfillopacity{0.700000}%
\pgfsetlinewidth{0.000000pt}%
\definecolor{currentstroke}{rgb}{0.000000,0.000000,0.000000}%
\pgfsetstrokecolor{currentstroke}%
\pgfsetdash{}{0pt}%
\pgfpathmoveto{\pgfqpoint{3.848458in}{3.318171in}}%
\pgfpathlineto{\pgfqpoint{3.861245in}{3.305265in}}%
\pgfpathlineto{\pgfqpoint{3.874033in}{3.292521in}}%
\pgfpathlineto{\pgfqpoint{3.886821in}{3.279939in}}%
\pgfpathlineto{\pgfqpoint{3.899609in}{3.267517in}}%
\pgfpathlineto{\pgfqpoint{3.907038in}{3.280624in}}%
\pgfpathlineto{\pgfqpoint{3.914463in}{3.293838in}}%
\pgfpathlineto{\pgfqpoint{3.921884in}{3.307161in}}%
\pgfpathlineto{\pgfqpoint{3.929301in}{3.320594in}}%
\pgfpathlineto{\pgfqpoint{3.916523in}{3.333162in}}%
\pgfpathlineto{\pgfqpoint{3.903745in}{3.345891in}}%
\pgfpathlineto{\pgfqpoint{3.890968in}{3.358781in}}%
\pgfpathlineto{\pgfqpoint{3.878191in}{3.371834in}}%
\pgfpathlineto{\pgfqpoint{3.870763in}{3.358248in}}%
\pgfpathlineto{\pgfqpoint{3.863332in}{3.344777in}}%
\pgfpathlineto{\pgfqpoint{3.855897in}{3.331418in}}%
\pgfpathlineto{\pgfqpoint{3.848458in}{3.318171in}}%
\pgfpathclose%
\pgfusepath{fill}%
\end{pgfscope}%
\begin{pgfscope}%
\pgfpathrectangle{\pgfqpoint{1.254980in}{0.150000in}}{\pgfqpoint{5.490039in}{5.490039in}}%
\pgfusepath{clip}%
\pgfsetbuttcap%
\pgfsetroundjoin%
\definecolor{currentfill}{rgb}{0.214298,0.355619,0.551184}%
\pgfsetfillcolor{currentfill}%
\pgfsetfillopacity{0.700000}%
\pgfsetlinewidth{0.000000pt}%
\definecolor{currentstroke}{rgb}{0.000000,0.000000,0.000000}%
\pgfsetstrokecolor{currentstroke}%
\pgfsetdash{}{0pt}%
\pgfpathmoveto{\pgfqpoint{4.082727in}{3.182015in}}%
\pgfpathlineto{\pgfqpoint{4.095524in}{3.171462in}}%
\pgfpathlineto{\pgfqpoint{4.108322in}{3.161060in}}%
\pgfpathlineto{\pgfqpoint{4.121123in}{3.150806in}}%
\pgfpathlineto{\pgfqpoint{4.133926in}{3.140700in}}%
\pgfpathlineto{\pgfqpoint{4.141299in}{3.153622in}}%
\pgfpathlineto{\pgfqpoint{4.148669in}{3.166643in}}%
\pgfpathlineto{\pgfqpoint{4.156035in}{3.179762in}}%
\pgfpathlineto{\pgfqpoint{4.163398in}{3.192983in}}%
\pgfpathlineto{\pgfqpoint{4.150604in}{3.203250in}}%
\pgfpathlineto{\pgfqpoint{4.137814in}{3.213665in}}%
\pgfpathlineto{\pgfqpoint{4.125025in}{3.224229in}}%
\pgfpathlineto{\pgfqpoint{4.112239in}{3.234943in}}%
\pgfpathlineto{\pgfqpoint{4.104866in}{3.221555in}}%
\pgfpathlineto{\pgfqpoint{4.097490in}{3.208272in}}%
\pgfpathlineto{\pgfqpoint{4.090111in}{3.195092in}}%
\pgfpathlineto{\pgfqpoint{4.082727in}{3.182015in}}%
\pgfpathclose%
\pgfusepath{fill}%
\end{pgfscope}%
\begin{pgfscope}%
\pgfpathrectangle{\pgfqpoint{1.254980in}{0.150000in}}{\pgfqpoint{5.490039in}{5.490039in}}%
\pgfusepath{clip}%
\pgfsetbuttcap%
\pgfsetroundjoin%
\definecolor{currentfill}{rgb}{0.225863,0.330805,0.547314}%
\pgfsetfillcolor{currentfill}%
\pgfsetfillopacity{0.700000}%
\pgfsetlinewidth{0.000000pt}%
\definecolor{currentstroke}{rgb}{0.000000,0.000000,0.000000}%
\pgfsetstrokecolor{currentstroke}%
\pgfsetdash{}{0pt}%
\pgfpathmoveto{\pgfqpoint{4.610298in}{3.114289in}}%
\pgfpathlineto{\pgfqpoint{4.623177in}{3.107481in}}%
\pgfpathlineto{\pgfqpoint{4.636060in}{3.100805in}}%
\pgfpathlineto{\pgfqpoint{4.648950in}{3.094260in}}%
\pgfpathlineto{\pgfqpoint{4.661845in}{3.087845in}}%
\pgfpathlineto{\pgfqpoint{4.669090in}{3.100822in}}%
\pgfpathlineto{\pgfqpoint{4.676333in}{3.113898in}}%
\pgfpathlineto{\pgfqpoint{4.683573in}{3.127075in}}%
\pgfpathlineto{\pgfqpoint{4.690812in}{3.140355in}}%
\pgfpathlineto{\pgfqpoint{4.677928in}{3.146993in}}%
\pgfpathlineto{\pgfqpoint{4.665049in}{3.153761in}}%
\pgfpathlineto{\pgfqpoint{4.652175in}{3.160660in}}%
\pgfpathlineto{\pgfqpoint{4.639308in}{3.167690in}}%
\pgfpathlineto{\pgfqpoint{4.632059in}{3.154181in}}%
\pgfpathlineto{\pgfqpoint{4.624807in}{3.140780in}}%
\pgfpathlineto{\pgfqpoint{4.617554in}{3.127483in}}%
\pgfpathlineto{\pgfqpoint{4.610298in}{3.114289in}}%
\pgfpathclose%
\pgfusepath{fill}%
\end{pgfscope}%
\begin{pgfscope}%
\pgfpathrectangle{\pgfqpoint{1.254980in}{0.150000in}}{\pgfqpoint{5.490039in}{5.490039in}}%
\pgfusepath{clip}%
\pgfsetbuttcap%
\pgfsetroundjoin%
\definecolor{currentfill}{rgb}{0.197636,0.391528,0.554969}%
\pgfsetfillcolor{currentfill}%
\pgfsetfillopacity{0.700000}%
\pgfsetlinewidth{0.000000pt}%
\definecolor{currentstroke}{rgb}{0.000000,0.000000,0.000000}%
\pgfsetstrokecolor{currentstroke}%
\pgfsetdash{}{0pt}%
\pgfpathmoveto{\pgfqpoint{3.899609in}{3.267517in}}%
\pgfpathlineto{\pgfqpoint{3.912399in}{3.255256in}}%
\pgfpathlineto{\pgfqpoint{3.925189in}{3.243153in}}%
\pgfpathlineto{\pgfqpoint{3.937980in}{3.231208in}}%
\pgfpathlineto{\pgfqpoint{3.950772in}{3.219421in}}%
\pgfpathlineto{\pgfqpoint{3.958190in}{3.232387in}}%
\pgfpathlineto{\pgfqpoint{3.965605in}{3.245457in}}%
\pgfpathlineto{\pgfqpoint{3.973015in}{3.258632in}}%
\pgfpathlineto{\pgfqpoint{3.980422in}{3.271913in}}%
\pgfpathlineto{\pgfqpoint{3.967641in}{3.283846in}}%
\pgfpathlineto{\pgfqpoint{3.954860in}{3.295937in}}%
\pgfpathlineto{\pgfqpoint{3.942080in}{3.308186in}}%
\pgfpathlineto{\pgfqpoint{3.929301in}{3.320594in}}%
\pgfpathlineto{\pgfqpoint{3.921884in}{3.307161in}}%
\pgfpathlineto{\pgfqpoint{3.914463in}{3.293838in}}%
\pgfpathlineto{\pgfqpoint{3.907038in}{3.280624in}}%
\pgfpathlineto{\pgfqpoint{3.899609in}{3.267517in}}%
\pgfpathclose%
\pgfusepath{fill}%
\end{pgfscope}%
\begin{pgfscope}%
\pgfpathrectangle{\pgfqpoint{1.254980in}{0.150000in}}{\pgfqpoint{5.490039in}{5.490039in}}%
\pgfusepath{clip}%
\pgfsetbuttcap%
\pgfsetroundjoin%
\definecolor{currentfill}{rgb}{0.123463,0.581687,0.547445}%
\pgfsetfillcolor{currentfill}%
\pgfsetfillopacity{0.700000}%
\pgfsetlinewidth{0.000000pt}%
\definecolor{currentstroke}{rgb}{0.000000,0.000000,0.000000}%
\pgfsetstrokecolor{currentstroke}%
\pgfsetdash{}{0pt}%
\pgfpathmoveto{\pgfqpoint{3.489999in}{3.749406in}}%
\pgfpathlineto{\pgfqpoint{3.502832in}{3.731548in}}%
\pgfpathlineto{\pgfqpoint{3.515662in}{3.713881in}}%
\pgfpathlineto{\pgfqpoint{3.528488in}{3.696405in}}%
\pgfpathlineto{\pgfqpoint{3.541311in}{3.679119in}}%
\pgfpathlineto{\pgfqpoint{3.548815in}{3.693218in}}%
\pgfpathlineto{\pgfqpoint{3.556314in}{3.707452in}}%
\pgfpathlineto{\pgfqpoint{3.563809in}{3.721824in}}%
\pgfpathlineto{\pgfqpoint{3.571299in}{3.736335in}}%
\pgfpathlineto{\pgfqpoint{3.558487in}{3.753771in}}%
\pgfpathlineto{\pgfqpoint{3.545671in}{3.771396in}}%
\pgfpathlineto{\pgfqpoint{3.532853in}{3.789213in}}%
\pgfpathlineto{\pgfqpoint{3.520031in}{3.807222in}}%
\pgfpathlineto{\pgfqpoint{3.512530in}{3.792556in}}%
\pgfpathlineto{\pgfqpoint{3.505025in}{3.778032in}}%
\pgfpathlineto{\pgfqpoint{3.497514in}{3.763649in}}%
\pgfpathlineto{\pgfqpoint{3.489999in}{3.749406in}}%
\pgfpathclose%
\pgfusepath{fill}%
\end{pgfscope}%
\begin{pgfscope}%
\pgfpathrectangle{\pgfqpoint{1.254980in}{0.150000in}}{\pgfqpoint{5.490039in}{5.490039in}}%
\pgfusepath{clip}%
\pgfsetbuttcap%
\pgfsetroundjoin%
\definecolor{currentfill}{rgb}{0.225863,0.330805,0.547314}%
\pgfsetfillcolor{currentfill}%
\pgfsetfillopacity{0.700000}%
\pgfsetlinewidth{0.000000pt}%
\definecolor{currentstroke}{rgb}{0.000000,0.000000,0.000000}%
\pgfsetstrokecolor{currentstroke}%
\pgfsetdash{}{0pt}%
\pgfpathmoveto{\pgfqpoint{4.265843in}{3.116109in}}%
\pgfpathlineto{\pgfqpoint{4.278663in}{3.107148in}}%
\pgfpathlineto{\pgfqpoint{4.291487in}{3.098328in}}%
\pgfpathlineto{\pgfqpoint{4.304314in}{3.089649in}}%
\pgfpathlineto{\pgfqpoint{4.317144in}{3.081111in}}%
\pgfpathlineto{\pgfqpoint{4.324474in}{3.093923in}}%
\pgfpathlineto{\pgfqpoint{4.331800in}{3.106828in}}%
\pgfpathlineto{\pgfqpoint{4.339123in}{3.119829in}}%
\pgfpathlineto{\pgfqpoint{4.346444in}{3.132927in}}%
\pgfpathlineto{\pgfqpoint{4.333623in}{3.141641in}}%
\pgfpathlineto{\pgfqpoint{4.320806in}{3.150496in}}%
\pgfpathlineto{\pgfqpoint{4.307993in}{3.159492in}}%
\pgfpathlineto{\pgfqpoint{4.295183in}{3.168631in}}%
\pgfpathlineto{\pgfqpoint{4.287853in}{3.155351in}}%
\pgfpathlineto{\pgfqpoint{4.280519in}{3.142172in}}%
\pgfpathlineto{\pgfqpoint{4.273183in}{3.129092in}}%
\pgfpathlineto{\pgfqpoint{4.265843in}{3.116109in}}%
\pgfpathclose%
\pgfusepath{fill}%
\end{pgfscope}%
\begin{pgfscope}%
\pgfpathrectangle{\pgfqpoint{1.254980in}{0.150000in}}{\pgfqpoint{5.490039in}{5.490039in}}%
\pgfusepath{clip}%
\pgfsetbuttcap%
\pgfsetroundjoin%
\definecolor{currentfill}{rgb}{0.227802,0.326594,0.546532}%
\pgfsetfillcolor{currentfill}%
\pgfsetfillopacity{0.700000}%
\pgfsetlinewidth{0.000000pt}%
\definecolor{currentstroke}{rgb}{0.000000,0.000000,0.000000}%
\pgfsetstrokecolor{currentstroke}%
\pgfsetdash{}{0pt}%
\pgfpathmoveto{\pgfqpoint{4.397766in}{3.099463in}}%
\pgfpathlineto{\pgfqpoint{4.410606in}{3.091443in}}%
\pgfpathlineto{\pgfqpoint{4.423451in}{3.083560in}}%
\pgfpathlineto{\pgfqpoint{4.436301in}{3.075814in}}%
\pgfpathlineto{\pgfqpoint{4.449155in}{3.068204in}}%
\pgfpathlineto{\pgfqpoint{4.456452in}{3.081029in}}%
\pgfpathlineto{\pgfqpoint{4.463746in}{3.093946in}}%
\pgfpathlineto{\pgfqpoint{4.471038in}{3.106959in}}%
\pgfpathlineto{\pgfqpoint{4.478327in}{3.120069in}}%
\pgfpathlineto{\pgfqpoint{4.465483in}{3.127871in}}%
\pgfpathlineto{\pgfqpoint{4.452643in}{3.135809in}}%
\pgfpathlineto{\pgfqpoint{4.439808in}{3.143884in}}%
\pgfpathlineto{\pgfqpoint{4.426978in}{3.152096in}}%
\pgfpathlineto{\pgfqpoint{4.419679in}{3.138788in}}%
\pgfpathlineto{\pgfqpoint{4.412377in}{3.125581in}}%
\pgfpathlineto{\pgfqpoint{4.405073in}{3.112474in}}%
\pgfpathlineto{\pgfqpoint{4.397766in}{3.099463in}}%
\pgfpathclose%
\pgfusepath{fill}%
\end{pgfscope}%
\begin{pgfscope}%
\pgfpathrectangle{\pgfqpoint{1.254980in}{0.150000in}}{\pgfqpoint{5.490039in}{5.490039in}}%
\pgfusepath{clip}%
\pgfsetbuttcap%
\pgfsetroundjoin%
\definecolor{currentfill}{rgb}{0.166383,0.690856,0.496502}%
\pgfsetfillcolor{currentfill}%
\pgfsetfillopacity{0.700000}%
\pgfsetlinewidth{0.000000pt}%
\definecolor{currentstroke}{rgb}{0.000000,0.000000,0.000000}%
\pgfsetstrokecolor{currentstroke}%
\pgfsetdash{}{0pt}%
\pgfpathmoveto{\pgfqpoint{3.365835in}{4.038814in}}%
\pgfpathlineto{\pgfqpoint{3.378711in}{4.018393in}}%
\pgfpathlineto{\pgfqpoint{3.391583in}{3.998180in}}%
\pgfpathlineto{\pgfqpoint{3.404448in}{3.978175in}}%
\pgfpathlineto{\pgfqpoint{3.417309in}{3.958375in}}%
\pgfpathlineto{\pgfqpoint{3.424827in}{3.973511in}}%
\pgfpathlineto{\pgfqpoint{3.432339in}{3.988801in}}%
\pgfpathlineto{\pgfqpoint{3.439847in}{4.004248in}}%
\pgfpathlineto{\pgfqpoint{3.447350in}{4.019853in}}%
\pgfpathlineto{\pgfqpoint{3.434500in}{4.039822in}}%
\pgfpathlineto{\pgfqpoint{3.421645in}{4.059996in}}%
\pgfpathlineto{\pgfqpoint{3.408785in}{4.080378in}}%
\pgfpathlineto{\pgfqpoint{3.395919in}{4.100970in}}%
\pgfpathlineto{\pgfqpoint{3.388406in}{4.085188in}}%
\pgfpathlineto{\pgfqpoint{3.380887in}{4.069570in}}%
\pgfpathlineto{\pgfqpoint{3.373364in}{4.054112in}}%
\pgfpathlineto{\pgfqpoint{3.365835in}{4.038814in}}%
\pgfpathclose%
\pgfusepath{fill}%
\end{pgfscope}%
\begin{pgfscope}%
\pgfpathrectangle{\pgfqpoint{1.254980in}{0.150000in}}{\pgfqpoint{5.490039in}{5.490039in}}%
\pgfusepath{clip}%
\pgfsetbuttcap%
\pgfsetroundjoin%
\definecolor{currentfill}{rgb}{0.220057,0.343307,0.549413}%
\pgfsetfillcolor{currentfill}%
\pgfsetfillopacity{0.700000}%
\pgfsetlinewidth{0.000000pt}%
\definecolor{currentstroke}{rgb}{0.000000,0.000000,0.000000}%
\pgfsetstrokecolor{currentstroke}%
\pgfsetdash{}{0pt}%
\pgfpathmoveto{\pgfqpoint{4.133926in}{3.140700in}}%
\pgfpathlineto{\pgfqpoint{4.146732in}{3.130743in}}%
\pgfpathlineto{\pgfqpoint{4.159540in}{3.120932in}}%
\pgfpathlineto{\pgfqpoint{4.172352in}{3.111267in}}%
\pgfpathlineto{\pgfqpoint{4.185166in}{3.101748in}}%
\pgfpathlineto{\pgfqpoint{4.192528in}{3.114515in}}%
\pgfpathlineto{\pgfqpoint{4.199888in}{3.127376in}}%
\pgfpathlineto{\pgfqpoint{4.207244in}{3.140332in}}%
\pgfpathlineto{\pgfqpoint{4.214597in}{3.153386in}}%
\pgfpathlineto{\pgfqpoint{4.201793in}{3.163067in}}%
\pgfpathlineto{\pgfqpoint{4.188992in}{3.172892in}}%
\pgfpathlineto{\pgfqpoint{4.176193in}{3.182864in}}%
\pgfpathlineto{\pgfqpoint{4.163398in}{3.192983in}}%
\pgfpathlineto{\pgfqpoint{4.156035in}{3.179762in}}%
\pgfpathlineto{\pgfqpoint{4.148669in}{3.166643in}}%
\pgfpathlineto{\pgfqpoint{4.141299in}{3.153622in}}%
\pgfpathlineto{\pgfqpoint{4.133926in}{3.140700in}}%
\pgfpathclose%
\pgfusepath{fill}%
\end{pgfscope}%
\begin{pgfscope}%
\pgfpathrectangle{\pgfqpoint{1.254980in}{0.150000in}}{\pgfqpoint{5.490039in}{5.490039in}}%
\pgfusepath{clip}%
\pgfsetbuttcap%
\pgfsetroundjoin%
\definecolor{currentfill}{rgb}{0.204903,0.375746,0.553533}%
\pgfsetfillcolor{currentfill}%
\pgfsetfillopacity{0.700000}%
\pgfsetlinewidth{0.000000pt}%
\definecolor{currentstroke}{rgb}{0.000000,0.000000,0.000000}%
\pgfsetstrokecolor{currentstroke}%
\pgfsetdash{}{0pt}%
\pgfpathmoveto{\pgfqpoint{3.950772in}{3.219421in}}%
\pgfpathlineto{\pgfqpoint{3.963565in}{3.207790in}}%
\pgfpathlineto{\pgfqpoint{3.976359in}{3.196315in}}%
\pgfpathlineto{\pgfqpoint{3.989155in}{3.184994in}}%
\pgfpathlineto{\pgfqpoint{4.001952in}{3.173828in}}%
\pgfpathlineto{\pgfqpoint{4.009360in}{3.186655in}}%
\pgfpathlineto{\pgfqpoint{4.016765in}{3.199581in}}%
\pgfpathlineto{\pgfqpoint{4.024165in}{3.212607in}}%
\pgfpathlineto{\pgfqpoint{4.031562in}{3.225736in}}%
\pgfpathlineto{\pgfqpoint{4.018775in}{3.237048in}}%
\pgfpathlineto{\pgfqpoint{4.005989in}{3.248515in}}%
\pgfpathlineto{\pgfqpoint{3.993205in}{3.260136in}}%
\pgfpathlineto{\pgfqpoint{3.980422in}{3.271913in}}%
\pgfpathlineto{\pgfqpoint{3.973015in}{3.258632in}}%
\pgfpathlineto{\pgfqpoint{3.965605in}{3.245457in}}%
\pgfpathlineto{\pgfqpoint{3.958190in}{3.232387in}}%
\pgfpathlineto{\pgfqpoint{3.950772in}{3.219421in}}%
\pgfpathclose%
\pgfusepath{fill}%
\end{pgfscope}%
\begin{pgfscope}%
\pgfpathrectangle{\pgfqpoint{1.254980in}{0.150000in}}{\pgfqpoint{5.490039in}{5.490039in}}%
\pgfusepath{clip}%
\pgfsetbuttcap%
\pgfsetroundjoin%
\definecolor{currentfill}{rgb}{0.119423,0.611141,0.538982}%
\pgfsetfillcolor{currentfill}%
\pgfsetfillopacity{0.700000}%
\pgfsetlinewidth{0.000000pt}%
\definecolor{currentstroke}{rgb}{0.000000,0.000000,0.000000}%
\pgfsetstrokecolor{currentstroke}%
\pgfsetdash{}{0pt}%
\pgfpathmoveto{\pgfqpoint{3.438628in}{3.822789in}}%
\pgfpathlineto{\pgfqpoint{3.451477in}{3.804149in}}%
\pgfpathlineto{\pgfqpoint{3.464322in}{3.785706in}}%
\pgfpathlineto{\pgfqpoint{3.477162in}{3.767459in}}%
\pgfpathlineto{\pgfqpoint{3.489999in}{3.749406in}}%
\pgfpathlineto{\pgfqpoint{3.497514in}{3.763649in}}%
\pgfpathlineto{\pgfqpoint{3.505025in}{3.778032in}}%
\pgfpathlineto{\pgfqpoint{3.512530in}{3.792556in}}%
\pgfpathlineto{\pgfqpoint{3.520031in}{3.807222in}}%
\pgfpathlineto{\pgfqpoint{3.507205in}{3.825425in}}%
\pgfpathlineto{\pgfqpoint{3.494376in}{3.843823in}}%
\pgfpathlineto{\pgfqpoint{3.481542in}{3.862417in}}%
\pgfpathlineto{\pgfqpoint{3.468704in}{3.881208in}}%
\pgfpathlineto{\pgfqpoint{3.461193in}{3.866384in}}%
\pgfpathlineto{\pgfqpoint{3.453676in}{3.851707in}}%
\pgfpathlineto{\pgfqpoint{3.446155in}{3.837176in}}%
\pgfpathlineto{\pgfqpoint{3.438628in}{3.822789in}}%
\pgfpathclose%
\pgfusepath{fill}%
\end{pgfscope}%
\begin{pgfscope}%
\pgfpathrectangle{\pgfqpoint{1.254980in}{0.150000in}}{\pgfqpoint{5.490039in}{5.490039in}}%
\pgfusepath{clip}%
\pgfsetbuttcap%
\pgfsetroundjoin%
\definecolor{currentfill}{rgb}{0.229739,0.322361,0.545706}%
\pgfsetfillcolor{currentfill}%
\pgfsetfillopacity{0.700000}%
\pgfsetlinewidth{0.000000pt}%
\definecolor{currentstroke}{rgb}{0.000000,0.000000,0.000000}%
\pgfsetstrokecolor{currentstroke}%
\pgfsetdash{}{0pt}%
\pgfpathmoveto{\pgfqpoint{4.529749in}{3.090212in}}%
\pgfpathlineto{\pgfqpoint{4.542617in}{3.083083in}}%
\pgfpathlineto{\pgfqpoint{4.555490in}{3.076087in}}%
\pgfpathlineto{\pgfqpoint{4.568368in}{3.069224in}}%
\pgfpathlineto{\pgfqpoint{4.581251in}{3.062492in}}%
\pgfpathlineto{\pgfqpoint{4.588517in}{3.075299in}}%
\pgfpathlineto{\pgfqpoint{4.595780in}{3.088199in}}%
\pgfpathlineto{\pgfqpoint{4.603040in}{3.101195in}}%
\pgfpathlineto{\pgfqpoint{4.610298in}{3.114289in}}%
\pgfpathlineto{\pgfqpoint{4.597425in}{3.121227in}}%
\pgfpathlineto{\pgfqpoint{4.584558in}{3.128298in}}%
\pgfpathlineto{\pgfqpoint{4.571695in}{3.135502in}}%
\pgfpathlineto{\pgfqpoint{4.558838in}{3.142838in}}%
\pgfpathlineto{\pgfqpoint{4.551569in}{3.129531in}}%
\pgfpathlineto{\pgfqpoint{4.544298in}{3.116326in}}%
\pgfpathlineto{\pgfqpoint{4.537025in}{3.103220in}}%
\pgfpathlineto{\pgfqpoint{4.529749in}{3.090212in}}%
\pgfpathclose%
\pgfusepath{fill}%
\end{pgfscope}%
\begin{pgfscope}%
\pgfpathrectangle{\pgfqpoint{1.254980in}{0.150000in}}{\pgfqpoint{5.490039in}{5.490039in}}%
\pgfusepath{clip}%
\pgfsetbuttcap%
\pgfsetroundjoin%
\definecolor{currentfill}{rgb}{0.223925,0.334994,0.548053}%
\pgfsetfillcolor{currentfill}%
\pgfsetfillopacity{0.700000}%
\pgfsetlinewidth{0.000000pt}%
\definecolor{currentstroke}{rgb}{0.000000,0.000000,0.000000}%
\pgfsetstrokecolor{currentstroke}%
\pgfsetdash{}{0pt}%
\pgfpathmoveto{\pgfqpoint{4.742407in}{3.115095in}}%
\pgfpathlineto{\pgfqpoint{4.755321in}{3.109101in}}%
\pgfpathlineto{\pgfqpoint{4.768241in}{3.103235in}}%
\pgfpathlineto{\pgfqpoint{4.781167in}{3.097496in}}%
\pgfpathlineto{\pgfqpoint{4.794099in}{3.091884in}}%
\pgfpathlineto{\pgfqpoint{4.801314in}{3.104804in}}%
\pgfpathlineto{\pgfqpoint{4.808526in}{3.117825in}}%
\pgfpathlineto{\pgfqpoint{4.815737in}{3.130949in}}%
\pgfpathlineto{\pgfqpoint{4.802813in}{3.136739in}}%
\pgfpathlineto{\pgfqpoint{4.789895in}{3.142655in}}%
\pgfpathlineto{\pgfqpoint{4.776984in}{3.148699in}}%
\pgfpathlineto{\pgfqpoint{4.764078in}{3.154870in}}%
\pgfpathlineto{\pgfqpoint{4.756856in}{3.141506in}}%
\pgfpathlineto{\pgfqpoint{4.749633in}{3.128249in}}%
\pgfpathlineto{\pgfqpoint{4.742407in}{3.115095in}}%
\pgfpathclose%
\pgfusepath{fill}%
\end{pgfscope}%
\begin{pgfscope}%
\pgfpathrectangle{\pgfqpoint{1.254980in}{0.150000in}}{\pgfqpoint{5.490039in}{5.490039in}}%
\pgfusepath{clip}%
\pgfsetbuttcap%
\pgfsetroundjoin%
\definecolor{currentfill}{rgb}{0.168126,0.459988,0.558082}%
\pgfsetfillcolor{currentfill}%
\pgfsetfillopacity{0.700000}%
\pgfsetlinewidth{0.000000pt}%
\definecolor{currentstroke}{rgb}{0.000000,0.000000,0.000000}%
\pgfsetstrokecolor{currentstroke}%
\pgfsetdash{}{0pt}%
\pgfpathmoveto{\pgfqpoint{3.665057in}{3.432560in}}%
\pgfpathlineto{\pgfqpoint{3.677861in}{3.417765in}}%
\pgfpathlineto{\pgfqpoint{3.690665in}{3.403143in}}%
\pgfpathlineto{\pgfqpoint{3.703467in}{3.388694in}}%
\pgfpathlineto{\pgfqpoint{3.716269in}{3.374416in}}%
\pgfpathlineto{\pgfqpoint{3.723747in}{3.387484in}}%
\pgfpathlineto{\pgfqpoint{3.731220in}{3.400666in}}%
\pgfpathlineto{\pgfqpoint{3.738689in}{3.413962in}}%
\pgfpathlineto{\pgfqpoint{3.746154in}{3.427373in}}%
\pgfpathlineto{\pgfqpoint{3.733363in}{3.441782in}}%
\pgfpathlineto{\pgfqpoint{3.720572in}{3.456363in}}%
\pgfpathlineto{\pgfqpoint{3.707779in}{3.471116in}}%
\pgfpathlineto{\pgfqpoint{3.694986in}{3.486043in}}%
\pgfpathlineto{\pgfqpoint{3.687510in}{3.472494in}}%
\pgfpathlineto{\pgfqpoint{3.680030in}{3.459064in}}%
\pgfpathlineto{\pgfqpoint{3.672545in}{3.445754in}}%
\pgfpathlineto{\pgfqpoint{3.665057in}{3.432560in}}%
\pgfpathclose%
\pgfusepath{fill}%
\end{pgfscope}%
\begin{pgfscope}%
\pgfpathrectangle{\pgfqpoint{1.254980in}{0.150000in}}{\pgfqpoint{5.490039in}{5.490039in}}%
\pgfusepath{clip}%
\pgfsetbuttcap%
\pgfsetroundjoin%
\definecolor{currentfill}{rgb}{0.212395,0.359683,0.551710}%
\pgfsetfillcolor{currentfill}%
\pgfsetfillopacity{0.700000}%
\pgfsetlinewidth{0.000000pt}%
\definecolor{currentstroke}{rgb}{0.000000,0.000000,0.000000}%
\pgfsetstrokecolor{currentstroke}%
\pgfsetdash{}{0pt}%
\pgfpathmoveto{\pgfqpoint{4.001952in}{3.173828in}}%
\pgfpathlineto{\pgfqpoint{4.014751in}{3.162815in}}%
\pgfpathlineto{\pgfqpoint{4.027552in}{3.151955in}}%
\pgfpathlineto{\pgfqpoint{4.040354in}{3.141247in}}%
\pgfpathlineto{\pgfqpoint{4.053159in}{3.130690in}}%
\pgfpathlineto{\pgfqpoint{4.060557in}{3.143376in}}%
\pgfpathlineto{\pgfqpoint{4.067951in}{3.156159in}}%
\pgfpathlineto{\pgfqpoint{4.075341in}{3.169037in}}%
\pgfpathlineto{\pgfqpoint{4.082727in}{3.182015in}}%
\pgfpathlineto{\pgfqpoint{4.069933in}{3.192718in}}%
\pgfpathlineto{\pgfqpoint{4.057141in}{3.203572in}}%
\pgfpathlineto{\pgfqpoint{4.044351in}{3.214578in}}%
\pgfpathlineto{\pgfqpoint{4.031562in}{3.225736in}}%
\pgfpathlineto{\pgfqpoint{4.024165in}{3.212607in}}%
\pgfpathlineto{\pgfqpoint{4.016765in}{3.199581in}}%
\pgfpathlineto{\pgfqpoint{4.009360in}{3.186655in}}%
\pgfpathlineto{\pgfqpoint{4.001952in}{3.173828in}}%
\pgfpathclose%
\pgfusepath{fill}%
\end{pgfscope}%
\begin{pgfscope}%
\pgfpathrectangle{\pgfqpoint{1.254980in}{0.150000in}}{\pgfqpoint{5.490039in}{5.490039in}}%
\pgfusepath{clip}%
\pgfsetbuttcap%
\pgfsetroundjoin%
\definecolor{currentfill}{rgb}{0.157729,0.485932,0.558013}%
\pgfsetfillcolor{currentfill}%
\pgfsetfillopacity{0.700000}%
\pgfsetlinewidth{0.000000pt}%
\definecolor{currentstroke}{rgb}{0.000000,0.000000,0.000000}%
\pgfsetstrokecolor{currentstroke}%
\pgfsetdash{}{0pt}%
\pgfpathmoveto{\pgfqpoint{3.613821in}{3.493500in}}%
\pgfpathlineto{\pgfqpoint{3.626633in}{3.477999in}}%
\pgfpathlineto{\pgfqpoint{3.639442in}{3.462677in}}%
\pgfpathlineto{\pgfqpoint{3.652250in}{3.447530in}}%
\pgfpathlineto{\pgfqpoint{3.665057in}{3.432560in}}%
\pgfpathlineto{\pgfqpoint{3.672545in}{3.445754in}}%
\pgfpathlineto{\pgfqpoint{3.680030in}{3.459064in}}%
\pgfpathlineto{\pgfqpoint{3.687510in}{3.472494in}}%
\pgfpathlineto{\pgfqpoint{3.694986in}{3.486043in}}%
\pgfpathlineto{\pgfqpoint{3.682190in}{3.501145in}}%
\pgfpathlineto{\pgfqpoint{3.669394in}{3.516423in}}%
\pgfpathlineto{\pgfqpoint{3.656595in}{3.531878in}}%
\pgfpathlineto{\pgfqpoint{3.643795in}{3.547511in}}%
\pgfpathlineto{\pgfqpoint{3.636308in}{3.533824in}}%
\pgfpathlineto{\pgfqpoint{3.628817in}{3.520260in}}%
\pgfpathlineto{\pgfqpoint{3.621322in}{3.506820in}}%
\pgfpathlineto{\pgfqpoint{3.613821in}{3.493500in}}%
\pgfpathclose%
\pgfusepath{fill}%
\end{pgfscope}%
\begin{pgfscope}%
\pgfpathrectangle{\pgfqpoint{1.254980in}{0.150000in}}{\pgfqpoint{5.490039in}{5.490039in}}%
\pgfusepath{clip}%
\pgfsetbuttcap%
\pgfsetroundjoin%
\definecolor{currentfill}{rgb}{0.177423,0.437527,0.557565}%
\pgfsetfillcolor{currentfill}%
\pgfsetfillopacity{0.700000}%
\pgfsetlinewidth{0.000000pt}%
\definecolor{currentstroke}{rgb}{0.000000,0.000000,0.000000}%
\pgfsetstrokecolor{currentstroke}%
\pgfsetdash{}{0pt}%
\pgfpathmoveto{\pgfqpoint{3.716269in}{3.374416in}}%
\pgfpathlineto{\pgfqpoint{3.729069in}{3.360310in}}%
\pgfpathlineto{\pgfqpoint{3.741869in}{3.346373in}}%
\pgfpathlineto{\pgfqpoint{3.754668in}{3.332604in}}%
\pgfpathlineto{\pgfqpoint{3.767467in}{3.319004in}}%
\pgfpathlineto{\pgfqpoint{3.774934in}{3.331947in}}%
\pgfpathlineto{\pgfqpoint{3.782396in}{3.344999in}}%
\pgfpathlineto{\pgfqpoint{3.789854in}{3.358162in}}%
\pgfpathlineto{\pgfqpoint{3.797309in}{3.371436in}}%
\pgfpathlineto{\pgfqpoint{3.784521in}{3.385168in}}%
\pgfpathlineto{\pgfqpoint{3.771732in}{3.399067in}}%
\pgfpathlineto{\pgfqpoint{3.758944in}{3.413135in}}%
\pgfpathlineto{\pgfqpoint{3.746154in}{3.427373in}}%
\pgfpathlineto{\pgfqpoint{3.738689in}{3.413962in}}%
\pgfpathlineto{\pgfqpoint{3.731220in}{3.400666in}}%
\pgfpathlineto{\pgfqpoint{3.723747in}{3.387484in}}%
\pgfpathlineto{\pgfqpoint{3.716269in}{3.374416in}}%
\pgfpathclose%
\pgfusepath{fill}%
\end{pgfscope}%
\begin{pgfscope}%
\pgfpathrectangle{\pgfqpoint{1.254980in}{0.150000in}}{\pgfqpoint{5.490039in}{5.490039in}}%
\pgfusepath{clip}%
\pgfsetbuttcap%
\pgfsetroundjoin%
\definecolor{currentfill}{rgb}{0.227802,0.326594,0.546532}%
\pgfsetfillcolor{currentfill}%
\pgfsetfillopacity{0.700000}%
\pgfsetlinewidth{0.000000pt}%
\definecolor{currentstroke}{rgb}{0.000000,0.000000,0.000000}%
\pgfsetstrokecolor{currentstroke}%
\pgfsetdash{}{0pt}%
\pgfpathmoveto{\pgfqpoint{4.661845in}{3.087845in}}%
\pgfpathlineto{\pgfqpoint{4.674745in}{3.081559in}}%
\pgfpathlineto{\pgfqpoint{4.687652in}{3.075403in}}%
\pgfpathlineto{\pgfqpoint{4.700564in}{3.069376in}}%
\pgfpathlineto{\pgfqpoint{4.713482in}{3.063477in}}%
\pgfpathlineto{\pgfqpoint{4.720717in}{3.076237in}}%
\pgfpathlineto{\pgfqpoint{4.727949in}{3.089092in}}%
\pgfpathlineto{\pgfqpoint{4.735179in}{3.102044in}}%
\pgfpathlineto{\pgfqpoint{4.742407in}{3.115095in}}%
\pgfpathlineto{\pgfqpoint{4.729499in}{3.121217in}}%
\pgfpathlineto{\pgfqpoint{4.716598in}{3.127467in}}%
\pgfpathlineto{\pgfqpoint{4.703702in}{3.133846in}}%
\pgfpathlineto{\pgfqpoint{4.690812in}{3.140355in}}%
\pgfpathlineto{\pgfqpoint{4.683573in}{3.127075in}}%
\pgfpathlineto{\pgfqpoint{4.676333in}{3.113898in}}%
\pgfpathlineto{\pgfqpoint{4.669090in}{3.100822in}}%
\pgfpathlineto{\pgfqpoint{4.661845in}{3.087845in}}%
\pgfpathclose%
\pgfusepath{fill}%
\end{pgfscope}%
\begin{pgfscope}%
\pgfpathrectangle{\pgfqpoint{1.254980in}{0.150000in}}{\pgfqpoint{5.490039in}{5.490039in}}%
\pgfusepath{clip}%
\pgfsetbuttcap%
\pgfsetroundjoin%
\definecolor{currentfill}{rgb}{0.124780,0.640461,0.527068}%
\pgfsetfillcolor{currentfill}%
\pgfsetfillopacity{0.700000}%
\pgfsetlinewidth{0.000000pt}%
\definecolor{currentstroke}{rgb}{0.000000,0.000000,0.000000}%
\pgfsetstrokecolor{currentstroke}%
\pgfsetdash{}{0pt}%
\pgfpathmoveto{\pgfqpoint{3.387188in}{3.899350in}}%
\pgfpathlineto{\pgfqpoint{3.400055in}{3.879907in}}%
\pgfpathlineto{\pgfqpoint{3.412917in}{3.860667in}}%
\pgfpathlineto{\pgfqpoint{3.425775in}{3.841628in}}%
\pgfpathlineto{\pgfqpoint{3.438628in}{3.822789in}}%
\pgfpathlineto{\pgfqpoint{3.446155in}{3.837176in}}%
\pgfpathlineto{\pgfqpoint{3.453676in}{3.851707in}}%
\pgfpathlineto{\pgfqpoint{3.461193in}{3.866384in}}%
\pgfpathlineto{\pgfqpoint{3.468704in}{3.881208in}}%
\pgfpathlineto{\pgfqpoint{3.455862in}{3.900198in}}%
\pgfpathlineto{\pgfqpoint{3.443016in}{3.919388in}}%
\pgfpathlineto{\pgfqpoint{3.430165in}{3.938780in}}%
\pgfpathlineto{\pgfqpoint{3.417309in}{3.958375in}}%
\pgfpathlineto{\pgfqpoint{3.409786in}{3.943394in}}%
\pgfpathlineto{\pgfqpoint{3.402259in}{3.928563in}}%
\pgfpathlineto{\pgfqpoint{3.394726in}{3.913883in}}%
\pgfpathlineto{\pgfqpoint{3.387188in}{3.899350in}}%
\pgfpathclose%
\pgfusepath{fill}%
\end{pgfscope}%
\begin{pgfscope}%
\pgfpathrectangle{\pgfqpoint{1.254980in}{0.150000in}}{\pgfqpoint{5.490039in}{5.490039in}}%
\pgfusepath{clip}%
\pgfsetbuttcap%
\pgfsetroundjoin%
\definecolor{currentfill}{rgb}{0.231674,0.318106,0.544834}%
\pgfsetfillcolor{currentfill}%
\pgfsetfillopacity{0.700000}%
\pgfsetlinewidth{0.000000pt}%
\definecolor{currentstroke}{rgb}{0.000000,0.000000,0.000000}%
\pgfsetstrokecolor{currentstroke}%
\pgfsetdash{}{0pt}%
\pgfpathmoveto{\pgfqpoint{4.317144in}{3.081111in}}%
\pgfpathlineto{\pgfqpoint{4.329979in}{3.072713in}}%
\pgfpathlineto{\pgfqpoint{4.342817in}{3.064455in}}%
\pgfpathlineto{\pgfqpoint{4.355660in}{3.056335in}}%
\pgfpathlineto{\pgfqpoint{4.368506in}{3.048353in}}%
\pgfpathlineto{\pgfqpoint{4.375826in}{3.060995in}}%
\pgfpathlineto{\pgfqpoint{4.383142in}{3.073726in}}%
\pgfpathlineto{\pgfqpoint{4.390455in}{3.086548in}}%
\pgfpathlineto{\pgfqpoint{4.397766in}{3.099463in}}%
\pgfpathlineto{\pgfqpoint{4.384929in}{3.107621in}}%
\pgfpathlineto{\pgfqpoint{4.372097in}{3.115917in}}%
\pgfpathlineto{\pgfqpoint{4.359268in}{3.124352in}}%
\pgfpathlineto{\pgfqpoint{4.346444in}{3.132927in}}%
\pgfpathlineto{\pgfqpoint{4.339123in}{3.119829in}}%
\pgfpathlineto{\pgfqpoint{4.331800in}{3.106828in}}%
\pgfpathlineto{\pgfqpoint{4.324474in}{3.093923in}}%
\pgfpathlineto{\pgfqpoint{4.317144in}{3.081111in}}%
\pgfpathclose%
\pgfusepath{fill}%
\end{pgfscope}%
\begin{pgfscope}%
\pgfpathrectangle{\pgfqpoint{1.254980in}{0.150000in}}{\pgfqpoint{5.490039in}{5.490039in}}%
\pgfusepath{clip}%
\pgfsetbuttcap%
\pgfsetroundjoin%
\definecolor{currentfill}{rgb}{0.149039,0.508051,0.557250}%
\pgfsetfillcolor{currentfill}%
\pgfsetfillopacity{0.700000}%
\pgfsetlinewidth{0.000000pt}%
\definecolor{currentstroke}{rgb}{0.000000,0.000000,0.000000}%
\pgfsetstrokecolor{currentstroke}%
\pgfsetdash{}{0pt}%
\pgfpathmoveto{\pgfqpoint{3.562554in}{3.557305in}}%
\pgfpathlineto{\pgfqpoint{3.575375in}{3.541082in}}%
\pgfpathlineto{\pgfqpoint{3.588192in}{3.525040in}}%
\pgfpathlineto{\pgfqpoint{3.601008in}{3.509180in}}%
\pgfpathlineto{\pgfqpoint{3.613821in}{3.493500in}}%
\pgfpathlineto{\pgfqpoint{3.621322in}{3.506820in}}%
\pgfpathlineto{\pgfqpoint{3.628817in}{3.520260in}}%
\pgfpathlineto{\pgfqpoint{3.636308in}{3.533824in}}%
\pgfpathlineto{\pgfqpoint{3.643795in}{3.547511in}}%
\pgfpathlineto{\pgfqpoint{3.630993in}{3.563323in}}%
\pgfpathlineto{\pgfqpoint{3.618188in}{3.579316in}}%
\pgfpathlineto{\pgfqpoint{3.605382in}{3.595489in}}%
\pgfpathlineto{\pgfqpoint{3.592573in}{3.611846in}}%
\pgfpathlineto{\pgfqpoint{3.585075in}{3.598020in}}%
\pgfpathlineto{\pgfqpoint{3.577573in}{3.584322in}}%
\pgfpathlineto{\pgfqpoint{3.570066in}{3.570751in}}%
\pgfpathlineto{\pgfqpoint{3.562554in}{3.557305in}}%
\pgfpathclose%
\pgfusepath{fill}%
\end{pgfscope}%
\begin{pgfscope}%
\pgfpathrectangle{\pgfqpoint{1.254980in}{0.150000in}}{\pgfqpoint{5.490039in}{5.490039in}}%
\pgfusepath{clip}%
\pgfsetbuttcap%
\pgfsetroundjoin%
\definecolor{currentfill}{rgb}{0.225863,0.330805,0.547314}%
\pgfsetfillcolor{currentfill}%
\pgfsetfillopacity{0.700000}%
\pgfsetlinewidth{0.000000pt}%
\definecolor{currentstroke}{rgb}{0.000000,0.000000,0.000000}%
\pgfsetstrokecolor{currentstroke}%
\pgfsetdash{}{0pt}%
\pgfpathmoveto{\pgfqpoint{4.185166in}{3.101748in}}%
\pgfpathlineto{\pgfqpoint{4.197982in}{3.092374in}}%
\pgfpathlineto{\pgfqpoint{4.210803in}{3.083144in}}%
\pgfpathlineto{\pgfqpoint{4.223626in}{3.074058in}}%
\pgfpathlineto{\pgfqpoint{4.236452in}{3.065115in}}%
\pgfpathlineto{\pgfqpoint{4.243805in}{3.077726in}}%
\pgfpathlineto{\pgfqpoint{4.251155in}{3.090428in}}%
\pgfpathlineto{\pgfqpoint{4.258501in}{3.103222in}}%
\pgfpathlineto{\pgfqpoint{4.265843in}{3.116109in}}%
\pgfpathlineto{\pgfqpoint{4.253027in}{3.125213in}}%
\pgfpathlineto{\pgfqpoint{4.240214in}{3.134460in}}%
\pgfpathlineto{\pgfqpoint{4.227404in}{3.143851in}}%
\pgfpathlineto{\pgfqpoint{4.214597in}{3.153386in}}%
\pgfpathlineto{\pgfqpoint{4.207244in}{3.140332in}}%
\pgfpathlineto{\pgfqpoint{4.199888in}{3.127376in}}%
\pgfpathlineto{\pgfqpoint{4.192528in}{3.114515in}}%
\pgfpathlineto{\pgfqpoint{4.185166in}{3.101748in}}%
\pgfpathclose%
\pgfusepath{fill}%
\end{pgfscope}%
\begin{pgfscope}%
\pgfpathrectangle{\pgfqpoint{1.254980in}{0.150000in}}{\pgfqpoint{5.490039in}{5.490039in}}%
\pgfusepath{clip}%
\pgfsetbuttcap%
\pgfsetroundjoin%
\definecolor{currentfill}{rgb}{0.185556,0.418570,0.556753}%
\pgfsetfillcolor{currentfill}%
\pgfsetfillopacity{0.700000}%
\pgfsetlinewidth{0.000000pt}%
\definecolor{currentstroke}{rgb}{0.000000,0.000000,0.000000}%
\pgfsetstrokecolor{currentstroke}%
\pgfsetdash{}{0pt}%
\pgfpathmoveto{\pgfqpoint{3.767467in}{3.319004in}}%
\pgfpathlineto{\pgfqpoint{3.780265in}{3.305570in}}%
\pgfpathlineto{\pgfqpoint{3.793063in}{3.292303in}}%
\pgfpathlineto{\pgfqpoint{3.805861in}{3.279200in}}%
\pgfpathlineto{\pgfqpoint{3.818659in}{3.266261in}}%
\pgfpathlineto{\pgfqpoint{3.826115in}{3.279080in}}%
\pgfpathlineto{\pgfqpoint{3.833567in}{3.292003in}}%
\pgfpathlineto{\pgfqpoint{3.841014in}{3.305033in}}%
\pgfpathlineto{\pgfqpoint{3.848458in}{3.318171in}}%
\pgfpathlineto{\pgfqpoint{3.835670in}{3.331240in}}%
\pgfpathlineto{\pgfqpoint{3.822883in}{3.344473in}}%
\pgfpathlineto{\pgfqpoint{3.810096in}{3.357872in}}%
\pgfpathlineto{\pgfqpoint{3.797309in}{3.371436in}}%
\pgfpathlineto{\pgfqpoint{3.789854in}{3.358162in}}%
\pgfpathlineto{\pgfqpoint{3.782396in}{3.344999in}}%
\pgfpathlineto{\pgfqpoint{3.774934in}{3.331947in}}%
\pgfpathlineto{\pgfqpoint{3.767467in}{3.319004in}}%
\pgfpathclose%
\pgfusepath{fill}%
\end{pgfscope}%
\begin{pgfscope}%
\pgfpathrectangle{\pgfqpoint{1.254980in}{0.150000in}}{\pgfqpoint{5.490039in}{5.490039in}}%
\pgfusepath{clip}%
\pgfsetbuttcap%
\pgfsetroundjoin%
\definecolor{currentfill}{rgb}{0.233603,0.313828,0.543914}%
\pgfsetfillcolor{currentfill}%
\pgfsetfillopacity{0.700000}%
\pgfsetlinewidth{0.000000pt}%
\definecolor{currentstroke}{rgb}{0.000000,0.000000,0.000000}%
\pgfsetstrokecolor{currentstroke}%
\pgfsetdash{}{0pt}%
\pgfpathmoveto{\pgfqpoint{4.449155in}{3.068204in}}%
\pgfpathlineto{\pgfqpoint{4.462013in}{3.060730in}}%
\pgfpathlineto{\pgfqpoint{4.474876in}{3.053390in}}%
\pgfpathlineto{\pgfqpoint{4.487745in}{3.046185in}}%
\pgfpathlineto{\pgfqpoint{4.500618in}{3.039115in}}%
\pgfpathlineto{\pgfqpoint{4.507905in}{3.051753in}}%
\pgfpathlineto{\pgfqpoint{4.515189in}{3.064481in}}%
\pgfpathlineto{\pgfqpoint{4.522470in}{3.077300in}}%
\pgfpathlineto{\pgfqpoint{4.529749in}{3.090212in}}%
\pgfpathlineto{\pgfqpoint{4.516886in}{3.097475in}}%
\pgfpathlineto{\pgfqpoint{4.504028in}{3.104872in}}%
\pgfpathlineto{\pgfqpoint{4.491175in}{3.112403in}}%
\pgfpathlineto{\pgfqpoint{4.478327in}{3.120069in}}%
\pgfpathlineto{\pgfqpoint{4.471038in}{3.106959in}}%
\pgfpathlineto{\pgfqpoint{4.463746in}{3.093946in}}%
\pgfpathlineto{\pgfqpoint{4.456452in}{3.081029in}}%
\pgfpathlineto{\pgfqpoint{4.449155in}{3.068204in}}%
\pgfpathclose%
\pgfusepath{fill}%
\end{pgfscope}%
\begin{pgfscope}%
\pgfpathrectangle{\pgfqpoint{1.254980in}{0.150000in}}{\pgfqpoint{5.490039in}{5.490039in}}%
\pgfusepath{clip}%
\pgfsetbuttcap%
\pgfsetroundjoin%
\definecolor{currentfill}{rgb}{0.137770,0.537492,0.554906}%
\pgfsetfillcolor{currentfill}%
\pgfsetfillopacity{0.700000}%
\pgfsetlinewidth{0.000000pt}%
\definecolor{currentstroke}{rgb}{0.000000,0.000000,0.000000}%
\pgfsetstrokecolor{currentstroke}%
\pgfsetdash{}{0pt}%
\pgfpathmoveto{\pgfqpoint{3.511247in}{3.624047in}}%
\pgfpathlineto{\pgfqpoint{3.524078in}{3.607082in}}%
\pgfpathlineto{\pgfqpoint{3.536906in}{3.590304in}}%
\pgfpathlineto{\pgfqpoint{3.549732in}{3.573713in}}%
\pgfpathlineto{\pgfqpoint{3.562554in}{3.557305in}}%
\pgfpathlineto{\pgfqpoint{3.570066in}{3.570751in}}%
\pgfpathlineto{\pgfqpoint{3.577573in}{3.584322in}}%
\pgfpathlineto{\pgfqpoint{3.585075in}{3.598020in}}%
\pgfpathlineto{\pgfqpoint{3.592573in}{3.611846in}}%
\pgfpathlineto{\pgfqpoint{3.579762in}{3.628385in}}%
\pgfpathlineto{\pgfqpoint{3.566947in}{3.645110in}}%
\pgfpathlineto{\pgfqpoint{3.554131in}{3.662021in}}%
\pgfpathlineto{\pgfqpoint{3.541311in}{3.679119in}}%
\pgfpathlineto{\pgfqpoint{3.533802in}{3.665154in}}%
\pgfpathlineto{\pgfqpoint{3.526288in}{3.651321in}}%
\pgfpathlineto{\pgfqpoint{3.518770in}{3.637619in}}%
\pgfpathlineto{\pgfqpoint{3.511247in}{3.624047in}}%
\pgfpathclose%
\pgfusepath{fill}%
\end{pgfscope}%
\begin{pgfscope}%
\pgfpathrectangle{\pgfqpoint{1.254980in}{0.150000in}}{\pgfqpoint{5.490039in}{5.490039in}}%
\pgfusepath{clip}%
\pgfsetbuttcap%
\pgfsetroundjoin%
\definecolor{currentfill}{rgb}{0.194100,0.399323,0.555565}%
\pgfsetfillcolor{currentfill}%
\pgfsetfillopacity{0.700000}%
\pgfsetlinewidth{0.000000pt}%
\definecolor{currentstroke}{rgb}{0.000000,0.000000,0.000000}%
\pgfsetstrokecolor{currentstroke}%
\pgfsetdash{}{0pt}%
\pgfpathmoveto{\pgfqpoint{3.818659in}{3.266261in}}%
\pgfpathlineto{\pgfqpoint{3.831457in}{3.253486in}}%
\pgfpathlineto{\pgfqpoint{3.844256in}{3.240873in}}%
\pgfpathlineto{\pgfqpoint{3.857055in}{3.228421in}}%
\pgfpathlineto{\pgfqpoint{3.869854in}{3.216130in}}%
\pgfpathlineto{\pgfqpoint{3.877299in}{3.228824in}}%
\pgfpathlineto{\pgfqpoint{3.884740in}{3.241619in}}%
\pgfpathlineto{\pgfqpoint{3.892177in}{3.254516in}}%
\pgfpathlineto{\pgfqpoint{3.899609in}{3.267517in}}%
\pgfpathlineto{\pgfqpoint{3.886821in}{3.279939in}}%
\pgfpathlineto{\pgfqpoint{3.874033in}{3.292521in}}%
\pgfpathlineto{\pgfqpoint{3.861245in}{3.305265in}}%
\pgfpathlineto{\pgfqpoint{3.848458in}{3.318171in}}%
\pgfpathlineto{\pgfqpoint{3.841014in}{3.305033in}}%
\pgfpathlineto{\pgfqpoint{3.833567in}{3.292003in}}%
\pgfpathlineto{\pgfqpoint{3.826115in}{3.279080in}}%
\pgfpathlineto{\pgfqpoint{3.818659in}{3.266261in}}%
\pgfpathclose%
\pgfusepath{fill}%
\end{pgfscope}%
\begin{pgfscope}%
\pgfpathrectangle{\pgfqpoint{1.254980in}{0.150000in}}{\pgfqpoint{5.490039in}{5.490039in}}%
\pgfusepath{clip}%
\pgfsetbuttcap%
\pgfsetroundjoin%
\definecolor{currentfill}{rgb}{0.220057,0.343307,0.549413}%
\pgfsetfillcolor{currentfill}%
\pgfsetfillopacity{0.700000}%
\pgfsetlinewidth{0.000000pt}%
\definecolor{currentstroke}{rgb}{0.000000,0.000000,0.000000}%
\pgfsetstrokecolor{currentstroke}%
\pgfsetdash{}{0pt}%
\pgfpathmoveto{\pgfqpoint{4.053159in}{3.130690in}}%
\pgfpathlineto{\pgfqpoint{4.065965in}{3.120283in}}%
\pgfpathlineto{\pgfqpoint{4.078774in}{3.110026in}}%
\pgfpathlineto{\pgfqpoint{4.091585in}{3.099918in}}%
\pgfpathlineto{\pgfqpoint{4.104399in}{3.089958in}}%
\pgfpathlineto{\pgfqpoint{4.111786in}{3.102505in}}%
\pgfpathlineto{\pgfqpoint{4.119170in}{3.115143in}}%
\pgfpathlineto{\pgfqpoint{4.126550in}{3.127874in}}%
\pgfpathlineto{\pgfqpoint{4.133926in}{3.140700in}}%
\pgfpathlineto{\pgfqpoint{4.121123in}{3.150806in}}%
\pgfpathlineto{\pgfqpoint{4.108322in}{3.161060in}}%
\pgfpathlineto{\pgfqpoint{4.095524in}{3.171462in}}%
\pgfpathlineto{\pgfqpoint{4.082727in}{3.182015in}}%
\pgfpathlineto{\pgfqpoint{4.075341in}{3.169037in}}%
\pgfpathlineto{\pgfqpoint{4.067951in}{3.156159in}}%
\pgfpathlineto{\pgfqpoint{4.060557in}{3.143376in}}%
\pgfpathlineto{\pgfqpoint{4.053159in}{3.130690in}}%
\pgfpathclose%
\pgfusepath{fill}%
\end{pgfscope}%
\begin{pgfscope}%
\pgfpathrectangle{\pgfqpoint{1.254980in}{0.150000in}}{\pgfqpoint{5.490039in}{5.490039in}}%
\pgfusepath{clip}%
\pgfsetbuttcap%
\pgfsetroundjoin%
\definecolor{currentfill}{rgb}{0.128729,0.563265,0.551229}%
\pgfsetfillcolor{currentfill}%
\pgfsetfillopacity{0.700000}%
\pgfsetlinewidth{0.000000pt}%
\definecolor{currentstroke}{rgb}{0.000000,0.000000,0.000000}%
\pgfsetstrokecolor{currentstroke}%
\pgfsetdash{}{0pt}%
\pgfpathmoveto{\pgfqpoint{3.459889in}{3.693801in}}%
\pgfpathlineto{\pgfqpoint{3.472733in}{3.676076in}}%
\pgfpathlineto{\pgfqpoint{3.485575in}{3.658543in}}%
\pgfpathlineto{\pgfqpoint{3.498412in}{3.641200in}}%
\pgfpathlineto{\pgfqpoint{3.511247in}{3.624047in}}%
\pgfpathlineto{\pgfqpoint{3.518770in}{3.637619in}}%
\pgfpathlineto{\pgfqpoint{3.526288in}{3.651321in}}%
\pgfpathlineto{\pgfqpoint{3.533802in}{3.665154in}}%
\pgfpathlineto{\pgfqpoint{3.541311in}{3.679119in}}%
\pgfpathlineto{\pgfqpoint{3.528488in}{3.696405in}}%
\pgfpathlineto{\pgfqpoint{3.515662in}{3.713881in}}%
\pgfpathlineto{\pgfqpoint{3.502832in}{3.731548in}}%
\pgfpathlineto{\pgfqpoint{3.489999in}{3.749406in}}%
\pgfpathlineto{\pgfqpoint{3.482479in}{3.735302in}}%
\pgfpathlineto{\pgfqpoint{3.474954in}{3.721334in}}%
\pgfpathlineto{\pgfqpoint{3.467424in}{3.707501in}}%
\pgfpathlineto{\pgfqpoint{3.459889in}{3.693801in}}%
\pgfpathclose%
\pgfusepath{fill}%
\end{pgfscope}%
\begin{pgfscope}%
\pgfpathrectangle{\pgfqpoint{1.254980in}{0.150000in}}{\pgfqpoint{5.490039in}{5.490039in}}%
\pgfusepath{clip}%
\pgfsetbuttcap%
\pgfsetroundjoin%
\definecolor{currentfill}{rgb}{0.233603,0.313828,0.543914}%
\pgfsetfillcolor{currentfill}%
\pgfsetfillopacity{0.700000}%
\pgfsetlinewidth{0.000000pt}%
\definecolor{currentstroke}{rgb}{0.000000,0.000000,0.000000}%
\pgfsetstrokecolor{currentstroke}%
\pgfsetdash{}{0pt}%
\pgfpathmoveto{\pgfqpoint{4.581251in}{3.062492in}}%
\pgfpathlineto{\pgfqpoint{4.594140in}{3.055892in}}%
\pgfpathlineto{\pgfqpoint{4.607034in}{3.049423in}}%
\pgfpathlineto{\pgfqpoint{4.619934in}{3.043085in}}%
\pgfpathlineto{\pgfqpoint{4.632839in}{3.036878in}}%
\pgfpathlineto{\pgfqpoint{4.640094in}{3.049482in}}%
\pgfpathlineto{\pgfqpoint{4.647347in}{3.062177in}}%
\pgfpathlineto{\pgfqpoint{4.654597in}{3.074964in}}%
\pgfpathlineto{\pgfqpoint{4.661845in}{3.087845in}}%
\pgfpathlineto{\pgfqpoint{4.648950in}{3.094260in}}%
\pgfpathlineto{\pgfqpoint{4.636060in}{3.100805in}}%
\pgfpathlineto{\pgfqpoint{4.623177in}{3.107481in}}%
\pgfpathlineto{\pgfqpoint{4.610298in}{3.114289in}}%
\pgfpathlineto{\pgfqpoint{4.603040in}{3.101195in}}%
\pgfpathlineto{\pgfqpoint{4.595780in}{3.088199in}}%
\pgfpathlineto{\pgfqpoint{4.588517in}{3.075299in}}%
\pgfpathlineto{\pgfqpoint{4.581251in}{3.062492in}}%
\pgfpathclose%
\pgfusepath{fill}%
\end{pgfscope}%
\begin{pgfscope}%
\pgfpathrectangle{\pgfqpoint{1.254980in}{0.150000in}}{\pgfqpoint{5.490039in}{5.490039in}}%
\pgfusepath{clip}%
\pgfsetbuttcap%
\pgfsetroundjoin%
\definecolor{currentfill}{rgb}{0.146616,0.673050,0.508936}%
\pgfsetfillcolor{currentfill}%
\pgfsetfillopacity{0.700000}%
\pgfsetlinewidth{0.000000pt}%
\definecolor{currentstroke}{rgb}{0.000000,0.000000,0.000000}%
\pgfsetstrokecolor{currentstroke}%
\pgfsetdash{}{0pt}%
\pgfpathmoveto{\pgfqpoint{3.335668in}{3.979180in}}%
\pgfpathlineto{\pgfqpoint{3.348556in}{3.958911in}}%
\pgfpathlineto{\pgfqpoint{3.361438in}{3.938851in}}%
\pgfpathlineto{\pgfqpoint{3.374316in}{3.918998in}}%
\pgfpathlineto{\pgfqpoint{3.387188in}{3.899350in}}%
\pgfpathlineto{\pgfqpoint{3.394726in}{3.913883in}}%
\pgfpathlineto{\pgfqpoint{3.402259in}{3.928563in}}%
\pgfpathlineto{\pgfqpoint{3.409786in}{3.943394in}}%
\pgfpathlineto{\pgfqpoint{3.417309in}{3.958375in}}%
\pgfpathlineto{\pgfqpoint{3.404448in}{3.978175in}}%
\pgfpathlineto{\pgfqpoint{3.391583in}{3.998180in}}%
\pgfpathlineto{\pgfqpoint{3.378711in}{4.018393in}}%
\pgfpathlineto{\pgfqpoint{3.365835in}{4.038814in}}%
\pgfpathlineto{\pgfqpoint{3.358301in}{4.023674in}}%
\pgfpathlineto{\pgfqpoint{3.350762in}{4.008689in}}%
\pgfpathlineto{\pgfqpoint{3.343218in}{3.993858in}}%
\pgfpathlineto{\pgfqpoint{3.335668in}{3.979180in}}%
\pgfpathclose%
\pgfusepath{fill}%
\end{pgfscope}%
\begin{pgfscope}%
\pgfpathrectangle{\pgfqpoint{1.254980in}{0.150000in}}{\pgfqpoint{5.490039in}{5.490039in}}%
\pgfusepath{clip}%
\pgfsetbuttcap%
\pgfsetroundjoin%
\definecolor{currentfill}{rgb}{0.203063,0.379716,0.553925}%
\pgfsetfillcolor{currentfill}%
\pgfsetfillopacity{0.700000}%
\pgfsetlinewidth{0.000000pt}%
\definecolor{currentstroke}{rgb}{0.000000,0.000000,0.000000}%
\pgfsetstrokecolor{currentstroke}%
\pgfsetdash{}{0pt}%
\pgfpathmoveto{\pgfqpoint{3.869854in}{3.216130in}}%
\pgfpathlineto{\pgfqpoint{3.882654in}{3.203999in}}%
\pgfpathlineto{\pgfqpoint{3.895455in}{3.192027in}}%
\pgfpathlineto{\pgfqpoint{3.908256in}{3.180212in}}%
\pgfpathlineto{\pgfqpoint{3.921059in}{3.168555in}}%
\pgfpathlineto{\pgfqpoint{3.928493in}{3.181124in}}%
\pgfpathlineto{\pgfqpoint{3.935923in}{3.193791in}}%
\pgfpathlineto{\pgfqpoint{3.943349in}{3.206556in}}%
\pgfpathlineto{\pgfqpoint{3.950772in}{3.219421in}}%
\pgfpathlineto{\pgfqpoint{3.937980in}{3.231208in}}%
\pgfpathlineto{\pgfqpoint{3.925189in}{3.243153in}}%
\pgfpathlineto{\pgfqpoint{3.912399in}{3.255256in}}%
\pgfpathlineto{\pgfqpoint{3.899609in}{3.267517in}}%
\pgfpathlineto{\pgfqpoint{3.892177in}{3.254516in}}%
\pgfpathlineto{\pgfqpoint{3.884740in}{3.241619in}}%
\pgfpathlineto{\pgfqpoint{3.877299in}{3.228824in}}%
\pgfpathlineto{\pgfqpoint{3.869854in}{3.216130in}}%
\pgfpathclose%
\pgfusepath{fill}%
\end{pgfscope}%
\begin{pgfscope}%
\pgfpathrectangle{\pgfqpoint{1.254980in}{0.150000in}}{\pgfqpoint{5.490039in}{5.490039in}}%
\pgfusepath{clip}%
\pgfsetbuttcap%
\pgfsetroundjoin%
\definecolor{currentfill}{rgb}{0.231674,0.318106,0.544834}%
\pgfsetfillcolor{currentfill}%
\pgfsetfillopacity{0.700000}%
\pgfsetlinewidth{0.000000pt}%
\definecolor{currentstroke}{rgb}{0.000000,0.000000,0.000000}%
\pgfsetstrokecolor{currentstroke}%
\pgfsetdash{}{0pt}%
\pgfpathmoveto{\pgfqpoint{4.236452in}{3.065115in}}%
\pgfpathlineto{\pgfqpoint{4.249282in}{3.056314in}}%
\pgfpathlineto{\pgfqpoint{4.262116in}{3.047655in}}%
\pgfpathlineto{\pgfqpoint{4.274953in}{3.039137in}}%
\pgfpathlineto{\pgfqpoint{4.287794in}{3.030760in}}%
\pgfpathlineto{\pgfqpoint{4.295136in}{3.043217in}}%
\pgfpathlineto{\pgfqpoint{4.302475in}{3.055760in}}%
\pgfpathlineto{\pgfqpoint{4.309811in}{3.068391in}}%
\pgfpathlineto{\pgfqpoint{4.317144in}{3.081111in}}%
\pgfpathlineto{\pgfqpoint{4.304314in}{3.089649in}}%
\pgfpathlineto{\pgfqpoint{4.291487in}{3.098328in}}%
\pgfpathlineto{\pgfqpoint{4.278663in}{3.107148in}}%
\pgfpathlineto{\pgfqpoint{4.265843in}{3.116109in}}%
\pgfpathlineto{\pgfqpoint{4.258501in}{3.103222in}}%
\pgfpathlineto{\pgfqpoint{4.251155in}{3.090428in}}%
\pgfpathlineto{\pgfqpoint{4.243805in}{3.077726in}}%
\pgfpathlineto{\pgfqpoint{4.236452in}{3.065115in}}%
\pgfpathclose%
\pgfusepath{fill}%
\end{pgfscope}%
\begin{pgfscope}%
\pgfpathrectangle{\pgfqpoint{1.254980in}{0.150000in}}{\pgfqpoint{5.490039in}{5.490039in}}%
\pgfusepath{clip}%
\pgfsetbuttcap%
\pgfsetroundjoin%
\definecolor{currentfill}{rgb}{0.235526,0.309527,0.542944}%
\pgfsetfillcolor{currentfill}%
\pgfsetfillopacity{0.700000}%
\pgfsetlinewidth{0.000000pt}%
\definecolor{currentstroke}{rgb}{0.000000,0.000000,0.000000}%
\pgfsetstrokecolor{currentstroke}%
\pgfsetdash{}{0pt}%
\pgfpathmoveto{\pgfqpoint{4.368506in}{3.048353in}}%
\pgfpathlineto{\pgfqpoint{4.381357in}{3.040510in}}%
\pgfpathlineto{\pgfqpoint{4.394212in}{3.032803in}}%
\pgfpathlineto{\pgfqpoint{4.407072in}{3.025233in}}%
\pgfpathlineto{\pgfqpoint{4.419936in}{3.017800in}}%
\pgfpathlineto{\pgfqpoint{4.427245in}{3.030271in}}%
\pgfpathlineto{\pgfqpoint{4.434551in}{3.042827in}}%
\pgfpathlineto{\pgfqpoint{4.441854in}{3.055471in}}%
\pgfpathlineto{\pgfqpoint{4.449155in}{3.068204in}}%
\pgfpathlineto{\pgfqpoint{4.436301in}{3.075814in}}%
\pgfpathlineto{\pgfqpoint{4.423451in}{3.083560in}}%
\pgfpathlineto{\pgfqpoint{4.410606in}{3.091443in}}%
\pgfpathlineto{\pgfqpoint{4.397766in}{3.099463in}}%
\pgfpathlineto{\pgfqpoint{4.390455in}{3.086548in}}%
\pgfpathlineto{\pgfqpoint{4.383142in}{3.073726in}}%
\pgfpathlineto{\pgfqpoint{4.375826in}{3.060995in}}%
\pgfpathlineto{\pgfqpoint{4.368506in}{3.048353in}}%
\pgfpathclose%
\pgfusepath{fill}%
\end{pgfscope}%
\begin{pgfscope}%
\pgfpathrectangle{\pgfqpoint{1.254980in}{0.150000in}}{\pgfqpoint{5.490039in}{5.490039in}}%
\pgfusepath{clip}%
\pgfsetbuttcap%
\pgfsetroundjoin%
\definecolor{currentfill}{rgb}{0.225863,0.330805,0.547314}%
\pgfsetfillcolor{currentfill}%
\pgfsetfillopacity{0.700000}%
\pgfsetlinewidth{0.000000pt}%
\definecolor{currentstroke}{rgb}{0.000000,0.000000,0.000000}%
\pgfsetstrokecolor{currentstroke}%
\pgfsetdash{}{0pt}%
\pgfpathmoveto{\pgfqpoint{4.794099in}{3.091884in}}%
\pgfpathlineto{\pgfqpoint{4.807038in}{3.086397in}}%
\pgfpathlineto{\pgfqpoint{4.819983in}{3.081037in}}%
\pgfpathlineto{\pgfqpoint{4.832935in}{3.075803in}}%
\pgfpathlineto{\pgfqpoint{4.845894in}{3.070693in}}%
\pgfpathlineto{\pgfqpoint{4.853097in}{3.083381in}}%
\pgfpathlineto{\pgfqpoint{4.860299in}{3.096165in}}%
\pgfpathlineto{\pgfqpoint{4.867499in}{3.109049in}}%
\pgfpathlineto{\pgfqpoint{4.854548in}{3.114336in}}%
\pgfpathlineto{\pgfqpoint{4.841605in}{3.119748in}}%
\pgfpathlineto{\pgfqpoint{4.828668in}{3.125285in}}%
\pgfpathlineto{\pgfqpoint{4.815737in}{3.130949in}}%
\pgfpathlineto{\pgfqpoint{4.808526in}{3.117825in}}%
\pgfpathlineto{\pgfqpoint{4.801314in}{3.104804in}}%
\pgfpathlineto{\pgfqpoint{4.794099in}{3.091884in}}%
\pgfpathclose%
\pgfusepath{fill}%
\end{pgfscope}%
\begin{pgfscope}%
\pgfpathrectangle{\pgfqpoint{1.254980in}{0.150000in}}{\pgfqpoint{5.490039in}{5.490039in}}%
\pgfusepath{clip}%
\pgfsetbuttcap%
\pgfsetroundjoin%
\definecolor{currentfill}{rgb}{0.121148,0.592739,0.544641}%
\pgfsetfillcolor{currentfill}%
\pgfsetfillopacity{0.700000}%
\pgfsetlinewidth{0.000000pt}%
\definecolor{currentstroke}{rgb}{0.000000,0.000000,0.000000}%
\pgfsetstrokecolor{currentstroke}%
\pgfsetdash{}{0pt}%
\pgfpathmoveto{\pgfqpoint{3.408471in}{3.766648in}}%
\pgfpathlineto{\pgfqpoint{3.421331in}{3.748142in}}%
\pgfpathlineto{\pgfqpoint{3.434188in}{3.729833in}}%
\pgfpathlineto{\pgfqpoint{3.447040in}{3.711720in}}%
\pgfpathlineto{\pgfqpoint{3.459889in}{3.693801in}}%
\pgfpathlineto{\pgfqpoint{3.467424in}{3.707501in}}%
\pgfpathlineto{\pgfqpoint{3.474954in}{3.721334in}}%
\pgfpathlineto{\pgfqpoint{3.482479in}{3.735302in}}%
\pgfpathlineto{\pgfqpoint{3.489999in}{3.749406in}}%
\pgfpathlineto{\pgfqpoint{3.477162in}{3.767459in}}%
\pgfpathlineto{\pgfqpoint{3.464322in}{3.785706in}}%
\pgfpathlineto{\pgfqpoint{3.451477in}{3.804149in}}%
\pgfpathlineto{\pgfqpoint{3.438628in}{3.822789in}}%
\pgfpathlineto{\pgfqpoint{3.431096in}{3.808544in}}%
\pgfpathlineto{\pgfqpoint{3.423560in}{3.794440in}}%
\pgfpathlineto{\pgfqpoint{3.416018in}{3.780475in}}%
\pgfpathlineto{\pgfqpoint{3.408471in}{3.766648in}}%
\pgfpathclose%
\pgfusepath{fill}%
\end{pgfscope}%
\begin{pgfscope}%
\pgfpathrectangle{\pgfqpoint{1.254980in}{0.150000in}}{\pgfqpoint{5.490039in}{5.490039in}}%
\pgfusepath{clip}%
\pgfsetbuttcap%
\pgfsetroundjoin%
\definecolor{currentfill}{rgb}{0.227802,0.326594,0.546532}%
\pgfsetfillcolor{currentfill}%
\pgfsetfillopacity{0.700000}%
\pgfsetlinewidth{0.000000pt}%
\definecolor{currentstroke}{rgb}{0.000000,0.000000,0.000000}%
\pgfsetstrokecolor{currentstroke}%
\pgfsetdash{}{0pt}%
\pgfpathmoveto{\pgfqpoint{4.104399in}{3.089958in}}%
\pgfpathlineto{\pgfqpoint{4.117215in}{3.080145in}}%
\pgfpathlineto{\pgfqpoint{4.130033in}{3.070480in}}%
\pgfpathlineto{\pgfqpoint{4.142855in}{3.060961in}}%
\pgfpathlineto{\pgfqpoint{4.155679in}{3.051587in}}%
\pgfpathlineto{\pgfqpoint{4.163056in}{3.063994in}}%
\pgfpathlineto{\pgfqpoint{4.170429in}{3.076489in}}%
\pgfpathlineto{\pgfqpoint{4.177799in}{3.089073in}}%
\pgfpathlineto{\pgfqpoint{4.185166in}{3.101748in}}%
\pgfpathlineto{\pgfqpoint{4.172352in}{3.111267in}}%
\pgfpathlineto{\pgfqpoint{4.159540in}{3.120932in}}%
\pgfpathlineto{\pgfqpoint{4.146732in}{3.130743in}}%
\pgfpathlineto{\pgfqpoint{4.133926in}{3.140700in}}%
\pgfpathlineto{\pgfqpoint{4.126550in}{3.127874in}}%
\pgfpathlineto{\pgfqpoint{4.119170in}{3.115143in}}%
\pgfpathlineto{\pgfqpoint{4.111786in}{3.102505in}}%
\pgfpathlineto{\pgfqpoint{4.104399in}{3.089958in}}%
\pgfpathclose%
\pgfusepath{fill}%
\end{pgfscope}%
\begin{pgfscope}%
\pgfpathrectangle{\pgfqpoint{1.254980in}{0.150000in}}{\pgfqpoint{5.490039in}{5.490039in}}%
\pgfusepath{clip}%
\pgfsetbuttcap%
\pgfsetroundjoin%
\definecolor{currentfill}{rgb}{0.212395,0.359683,0.551710}%
\pgfsetfillcolor{currentfill}%
\pgfsetfillopacity{0.700000}%
\pgfsetlinewidth{0.000000pt}%
\definecolor{currentstroke}{rgb}{0.000000,0.000000,0.000000}%
\pgfsetstrokecolor{currentstroke}%
\pgfsetdash{}{0pt}%
\pgfpathmoveto{\pgfqpoint{3.921059in}{3.168555in}}%
\pgfpathlineto{\pgfqpoint{3.933862in}{3.157054in}}%
\pgfpathlineto{\pgfqpoint{3.946667in}{3.145709in}}%
\pgfpathlineto{\pgfqpoint{3.959474in}{3.134519in}}%
\pgfpathlineto{\pgfqpoint{3.972282in}{3.123483in}}%
\pgfpathlineto{\pgfqpoint{3.979705in}{3.135928in}}%
\pgfpathlineto{\pgfqpoint{3.987125in}{3.148466in}}%
\pgfpathlineto{\pgfqpoint{3.994541in}{3.161099in}}%
\pgfpathlineto{\pgfqpoint{4.001952in}{3.173828in}}%
\pgfpathlineto{\pgfqpoint{3.989155in}{3.184994in}}%
\pgfpathlineto{\pgfqpoint{3.976359in}{3.196315in}}%
\pgfpathlineto{\pgfqpoint{3.963565in}{3.207790in}}%
\pgfpathlineto{\pgfqpoint{3.950772in}{3.219421in}}%
\pgfpathlineto{\pgfqpoint{3.943349in}{3.206556in}}%
\pgfpathlineto{\pgfqpoint{3.935923in}{3.193791in}}%
\pgfpathlineto{\pgfqpoint{3.928493in}{3.181124in}}%
\pgfpathlineto{\pgfqpoint{3.921059in}{3.168555in}}%
\pgfpathclose%
\pgfusepath{fill}%
\end{pgfscope}%
\begin{pgfscope}%
\pgfpathrectangle{\pgfqpoint{1.254980in}{0.150000in}}{\pgfqpoint{5.490039in}{5.490039in}}%
\pgfusepath{clip}%
\pgfsetbuttcap%
\pgfsetroundjoin%
\definecolor{currentfill}{rgb}{0.231674,0.318106,0.544834}%
\pgfsetfillcolor{currentfill}%
\pgfsetfillopacity{0.700000}%
\pgfsetlinewidth{0.000000pt}%
\definecolor{currentstroke}{rgb}{0.000000,0.000000,0.000000}%
\pgfsetstrokecolor{currentstroke}%
\pgfsetdash{}{0pt}%
\pgfpathmoveto{\pgfqpoint{4.713482in}{3.063477in}}%
\pgfpathlineto{\pgfqpoint{4.726407in}{3.057706in}}%
\pgfpathlineto{\pgfqpoint{4.739338in}{3.052062in}}%
\pgfpathlineto{\pgfqpoint{4.752275in}{3.046546in}}%
\pgfpathlineto{\pgfqpoint{4.765218in}{3.041157in}}%
\pgfpathlineto{\pgfqpoint{4.772442in}{3.053700in}}%
\pgfpathlineto{\pgfqpoint{4.779663in}{3.066334in}}%
\pgfpathlineto{\pgfqpoint{4.786882in}{3.079061in}}%
\pgfpathlineto{\pgfqpoint{4.794099in}{3.091884in}}%
\pgfpathlineto{\pgfqpoint{4.781167in}{3.097496in}}%
\pgfpathlineto{\pgfqpoint{4.768241in}{3.103235in}}%
\pgfpathlineto{\pgfqpoint{4.755321in}{3.109101in}}%
\pgfpathlineto{\pgfqpoint{4.742407in}{3.115095in}}%
\pgfpathlineto{\pgfqpoint{4.735179in}{3.102044in}}%
\pgfpathlineto{\pgfqpoint{4.727949in}{3.089092in}}%
\pgfpathlineto{\pgfqpoint{4.720717in}{3.076237in}}%
\pgfpathlineto{\pgfqpoint{4.713482in}{3.063477in}}%
\pgfpathclose%
\pgfusepath{fill}%
\end{pgfscope}%
\begin{pgfscope}%
\pgfpathrectangle{\pgfqpoint{1.254980in}{0.150000in}}{\pgfqpoint{5.490039in}{5.490039in}}%
\pgfusepath{clip}%
\pgfsetbuttcap%
\pgfsetroundjoin%
\definecolor{currentfill}{rgb}{0.235526,0.309527,0.542944}%
\pgfsetfillcolor{currentfill}%
\pgfsetfillopacity{0.700000}%
\pgfsetlinewidth{0.000000pt}%
\definecolor{currentstroke}{rgb}{0.000000,0.000000,0.000000}%
\pgfsetstrokecolor{currentstroke}%
\pgfsetdash{}{0pt}%
\pgfpathmoveto{\pgfqpoint{4.500618in}{3.039115in}}%
\pgfpathlineto{\pgfqpoint{4.513496in}{3.032177in}}%
\pgfpathlineto{\pgfqpoint{4.526379in}{3.025373in}}%
\pgfpathlineto{\pgfqpoint{4.539267in}{3.018701in}}%
\pgfpathlineto{\pgfqpoint{4.552161in}{3.012162in}}%
\pgfpathlineto{\pgfqpoint{4.559438in}{3.024614in}}%
\pgfpathlineto{\pgfqpoint{4.566712in}{3.037152in}}%
\pgfpathlineto{\pgfqpoint{4.573983in}{3.049777in}}%
\pgfpathlineto{\pgfqpoint{4.581251in}{3.062492in}}%
\pgfpathlineto{\pgfqpoint{4.568368in}{3.069224in}}%
\pgfpathlineto{\pgfqpoint{4.555490in}{3.076087in}}%
\pgfpathlineto{\pgfqpoint{4.542617in}{3.083083in}}%
\pgfpathlineto{\pgfqpoint{4.529749in}{3.090212in}}%
\pgfpathlineto{\pgfqpoint{4.522470in}{3.077300in}}%
\pgfpathlineto{\pgfqpoint{4.515189in}{3.064481in}}%
\pgfpathlineto{\pgfqpoint{4.507905in}{3.051753in}}%
\pgfpathlineto{\pgfqpoint{4.500618in}{3.039115in}}%
\pgfpathclose%
\pgfusepath{fill}%
\end{pgfscope}%
\begin{pgfscope}%
\pgfpathrectangle{\pgfqpoint{1.254980in}{0.150000in}}{\pgfqpoint{5.490039in}{5.490039in}}%
\pgfusepath{clip}%
\pgfsetbuttcap%
\pgfsetroundjoin%
\definecolor{currentfill}{rgb}{0.120081,0.622161,0.534946}%
\pgfsetfillcolor{currentfill}%
\pgfsetfillopacity{0.700000}%
\pgfsetlinewidth{0.000000pt}%
\definecolor{currentstroke}{rgb}{0.000000,0.000000,0.000000}%
\pgfsetstrokecolor{currentstroke}%
\pgfsetdash{}{0pt}%
\pgfpathmoveto{\pgfqpoint{3.356983in}{3.842671in}}%
\pgfpathlineto{\pgfqpoint{3.369862in}{3.823363in}}%
\pgfpathlineto{\pgfqpoint{3.382736in}{3.804257in}}%
\pgfpathlineto{\pgfqpoint{3.395606in}{3.785353in}}%
\pgfpathlineto{\pgfqpoint{3.408471in}{3.766648in}}%
\pgfpathlineto{\pgfqpoint{3.416018in}{3.780475in}}%
\pgfpathlineto{\pgfqpoint{3.423560in}{3.794440in}}%
\pgfpathlineto{\pgfqpoint{3.431096in}{3.808544in}}%
\pgfpathlineto{\pgfqpoint{3.438628in}{3.822789in}}%
\pgfpathlineto{\pgfqpoint{3.425775in}{3.841628in}}%
\pgfpathlineto{\pgfqpoint{3.412917in}{3.860667in}}%
\pgfpathlineto{\pgfqpoint{3.400055in}{3.879907in}}%
\pgfpathlineto{\pgfqpoint{3.387188in}{3.899350in}}%
\pgfpathlineto{\pgfqpoint{3.379644in}{3.884965in}}%
\pgfpathlineto{\pgfqpoint{3.372096in}{3.870724in}}%
\pgfpathlineto{\pgfqpoint{3.364542in}{3.856626in}}%
\pgfpathlineto{\pgfqpoint{3.356983in}{3.842671in}}%
\pgfpathclose%
\pgfusepath{fill}%
\end{pgfscope}%
\begin{pgfscope}%
\pgfpathrectangle{\pgfqpoint{1.254980in}{0.150000in}}{\pgfqpoint{5.490039in}{5.490039in}}%
\pgfusepath{clip}%
\pgfsetbuttcap%
\pgfsetroundjoin%
\definecolor{currentfill}{rgb}{0.174274,0.445044,0.557792}%
\pgfsetfillcolor{currentfill}%
\pgfsetfillopacity{0.700000}%
\pgfsetlinewidth{0.000000pt}%
\definecolor{currentstroke}{rgb}{0.000000,0.000000,0.000000}%
\pgfsetstrokecolor{currentstroke}%
\pgfsetdash{}{0pt}%
\pgfpathmoveto{\pgfqpoint{3.635054in}{3.380929in}}%
\pgfpathlineto{\pgfqpoint{3.647871in}{3.366249in}}%
\pgfpathlineto{\pgfqpoint{3.660686in}{3.351743in}}%
\pgfpathlineto{\pgfqpoint{3.673500in}{3.337409in}}%
\pgfpathlineto{\pgfqpoint{3.686313in}{3.323247in}}%
\pgfpathlineto{\pgfqpoint{3.693808in}{3.335876in}}%
\pgfpathlineto{\pgfqpoint{3.701300in}{3.348614in}}%
\pgfpathlineto{\pgfqpoint{3.708787in}{3.361460in}}%
\pgfpathlineto{\pgfqpoint{3.716269in}{3.374416in}}%
\pgfpathlineto{\pgfqpoint{3.703467in}{3.388694in}}%
\pgfpathlineto{\pgfqpoint{3.690665in}{3.403143in}}%
\pgfpathlineto{\pgfqpoint{3.677861in}{3.417765in}}%
\pgfpathlineto{\pgfqpoint{3.665057in}{3.432560in}}%
\pgfpathlineto{\pgfqpoint{3.657563in}{3.419482in}}%
\pgfpathlineto{\pgfqpoint{3.650065in}{3.406519in}}%
\pgfpathlineto{\pgfqpoint{3.642562in}{3.393668in}}%
\pgfpathlineto{\pgfqpoint{3.635054in}{3.380929in}}%
\pgfpathclose%
\pgfusepath{fill}%
\end{pgfscope}%
\begin{pgfscope}%
\pgfpathrectangle{\pgfqpoint{1.254980in}{0.150000in}}{\pgfqpoint{5.490039in}{5.490039in}}%
\pgfusepath{clip}%
\pgfsetbuttcap%
\pgfsetroundjoin%
\definecolor{currentfill}{rgb}{0.163625,0.471133,0.558148}%
\pgfsetfillcolor{currentfill}%
\pgfsetfillopacity{0.700000}%
\pgfsetlinewidth{0.000000pt}%
\definecolor{currentstroke}{rgb}{0.000000,0.000000,0.000000}%
\pgfsetstrokecolor{currentstroke}%
\pgfsetdash{}{0pt}%
\pgfpathmoveto{\pgfqpoint{3.583773in}{3.441407in}}%
\pgfpathlineto{\pgfqpoint{3.596596in}{3.426021in}}%
\pgfpathlineto{\pgfqpoint{3.609417in}{3.410814in}}%
\pgfpathlineto{\pgfqpoint{3.622237in}{3.395784in}}%
\pgfpathlineto{\pgfqpoint{3.635054in}{3.380929in}}%
\pgfpathlineto{\pgfqpoint{3.642562in}{3.393668in}}%
\pgfpathlineto{\pgfqpoint{3.650065in}{3.406519in}}%
\pgfpathlineto{\pgfqpoint{3.657563in}{3.419482in}}%
\pgfpathlineto{\pgfqpoint{3.665057in}{3.432560in}}%
\pgfpathlineto{\pgfqpoint{3.652250in}{3.447530in}}%
\pgfpathlineto{\pgfqpoint{3.639442in}{3.462677in}}%
\pgfpathlineto{\pgfqpoint{3.626633in}{3.477999in}}%
\pgfpathlineto{\pgfqpoint{3.613821in}{3.493500in}}%
\pgfpathlineto{\pgfqpoint{3.606316in}{3.480301in}}%
\pgfpathlineto{\pgfqpoint{3.598807in}{3.467219in}}%
\pgfpathlineto{\pgfqpoint{3.591292in}{3.454255in}}%
\pgfpathlineto{\pgfqpoint{3.583773in}{3.441407in}}%
\pgfpathclose%
\pgfusepath{fill}%
\end{pgfscope}%
\begin{pgfscope}%
\pgfpathrectangle{\pgfqpoint{1.254980in}{0.150000in}}{\pgfqpoint{5.490039in}{5.490039in}}%
\pgfusepath{clip}%
\pgfsetbuttcap%
\pgfsetroundjoin%
\definecolor{currentfill}{rgb}{0.220057,0.343307,0.549413}%
\pgfsetfillcolor{currentfill}%
\pgfsetfillopacity{0.700000}%
\pgfsetlinewidth{0.000000pt}%
\definecolor{currentstroke}{rgb}{0.000000,0.000000,0.000000}%
\pgfsetstrokecolor{currentstroke}%
\pgfsetdash{}{0pt}%
\pgfpathmoveto{\pgfqpoint{3.972282in}{3.123483in}}%
\pgfpathlineto{\pgfqpoint{3.985091in}{3.112600in}}%
\pgfpathlineto{\pgfqpoint{3.997902in}{3.101870in}}%
\pgfpathlineto{\pgfqpoint{4.010715in}{3.091292in}}%
\pgfpathlineto{\pgfqpoint{4.023530in}{3.080865in}}%
\pgfpathlineto{\pgfqpoint{4.030943in}{3.093186in}}%
\pgfpathlineto{\pgfqpoint{4.038352in}{3.105596in}}%
\pgfpathlineto{\pgfqpoint{4.045757in}{3.118097in}}%
\pgfpathlineto{\pgfqpoint{4.053159in}{3.130690in}}%
\pgfpathlineto{\pgfqpoint{4.040354in}{3.141247in}}%
\pgfpathlineto{\pgfqpoint{4.027552in}{3.151955in}}%
\pgfpathlineto{\pgfqpoint{4.014751in}{3.162815in}}%
\pgfpathlineto{\pgfqpoint{4.001952in}{3.173828in}}%
\pgfpathlineto{\pgfqpoint{3.994541in}{3.161099in}}%
\pgfpathlineto{\pgfqpoint{3.987125in}{3.148466in}}%
\pgfpathlineto{\pgfqpoint{3.979705in}{3.135928in}}%
\pgfpathlineto{\pgfqpoint{3.972282in}{3.123483in}}%
\pgfpathclose%
\pgfusepath{fill}%
\end{pgfscope}%
\begin{pgfscope}%
\pgfpathrectangle{\pgfqpoint{1.254980in}{0.150000in}}{\pgfqpoint{5.490039in}{5.490039in}}%
\pgfusepath{clip}%
\pgfsetbuttcap%
\pgfsetroundjoin%
\definecolor{currentfill}{rgb}{0.182256,0.426184,0.557120}%
\pgfsetfillcolor{currentfill}%
\pgfsetfillopacity{0.700000}%
\pgfsetlinewidth{0.000000pt}%
\definecolor{currentstroke}{rgb}{0.000000,0.000000,0.000000}%
\pgfsetstrokecolor{currentstroke}%
\pgfsetdash{}{0pt}%
\pgfpathmoveto{\pgfqpoint{3.686313in}{3.323247in}}%
\pgfpathlineto{\pgfqpoint{3.699124in}{3.309255in}}%
\pgfpathlineto{\pgfqpoint{3.711936in}{3.295433in}}%
\pgfpathlineto{\pgfqpoint{3.724746in}{3.281780in}}%
\pgfpathlineto{\pgfqpoint{3.737556in}{3.268295in}}%
\pgfpathlineto{\pgfqpoint{3.745040in}{3.280815in}}%
\pgfpathlineto{\pgfqpoint{3.752520in}{3.293439in}}%
\pgfpathlineto{\pgfqpoint{3.759996in}{3.306169in}}%
\pgfpathlineto{\pgfqpoint{3.767467in}{3.319004in}}%
\pgfpathlineto{\pgfqpoint{3.754668in}{3.332604in}}%
\pgfpathlineto{\pgfqpoint{3.741869in}{3.346373in}}%
\pgfpathlineto{\pgfqpoint{3.729069in}{3.360310in}}%
\pgfpathlineto{\pgfqpoint{3.716269in}{3.374416in}}%
\pgfpathlineto{\pgfqpoint{3.708787in}{3.361460in}}%
\pgfpathlineto{\pgfqpoint{3.701300in}{3.348614in}}%
\pgfpathlineto{\pgfqpoint{3.693808in}{3.335876in}}%
\pgfpathlineto{\pgfqpoint{3.686313in}{3.323247in}}%
\pgfpathclose%
\pgfusepath{fill}%
\end{pgfscope}%
\begin{pgfscope}%
\pgfpathrectangle{\pgfqpoint{1.254980in}{0.150000in}}{\pgfqpoint{5.490039in}{5.490039in}}%
\pgfusepath{clip}%
\pgfsetbuttcap%
\pgfsetroundjoin%
\definecolor{currentfill}{rgb}{0.154815,0.493313,0.557840}%
\pgfsetfillcolor{currentfill}%
\pgfsetfillopacity{0.700000}%
\pgfsetlinewidth{0.000000pt}%
\definecolor{currentstroke}{rgb}{0.000000,0.000000,0.000000}%
\pgfsetstrokecolor{currentstroke}%
\pgfsetdash{}{0pt}%
\pgfpathmoveto{\pgfqpoint{3.532459in}{3.504748in}}%
\pgfpathlineto{\pgfqpoint{3.545291in}{3.488640in}}%
\pgfpathlineto{\pgfqpoint{3.558120in}{3.472715in}}%
\pgfpathlineto{\pgfqpoint{3.570948in}{3.456971in}}%
\pgfpathlineto{\pgfqpoint{3.583773in}{3.441407in}}%
\pgfpathlineto{\pgfqpoint{3.591292in}{3.454255in}}%
\pgfpathlineto{\pgfqpoint{3.598807in}{3.467219in}}%
\pgfpathlineto{\pgfqpoint{3.606316in}{3.480301in}}%
\pgfpathlineto{\pgfqpoint{3.613821in}{3.493500in}}%
\pgfpathlineto{\pgfqpoint{3.601008in}{3.509180in}}%
\pgfpathlineto{\pgfqpoint{3.588192in}{3.525040in}}%
\pgfpathlineto{\pgfqpoint{3.575375in}{3.541082in}}%
\pgfpathlineto{\pgfqpoint{3.562554in}{3.557305in}}%
\pgfpathlineto{\pgfqpoint{3.555038in}{3.543984in}}%
\pgfpathlineto{\pgfqpoint{3.547516in}{3.530784in}}%
\pgfpathlineto{\pgfqpoint{3.539990in}{3.517706in}}%
\pgfpathlineto{\pgfqpoint{3.532459in}{3.504748in}}%
\pgfpathclose%
\pgfusepath{fill}%
\end{pgfscope}%
\begin{pgfscope}%
\pgfpathrectangle{\pgfqpoint{1.254980in}{0.150000in}}{\pgfqpoint{5.490039in}{5.490039in}}%
\pgfusepath{clip}%
\pgfsetbuttcap%
\pgfsetroundjoin%
\definecolor{currentfill}{rgb}{0.235526,0.309527,0.542944}%
\pgfsetfillcolor{currentfill}%
\pgfsetfillopacity{0.700000}%
\pgfsetlinewidth{0.000000pt}%
\definecolor{currentstroke}{rgb}{0.000000,0.000000,0.000000}%
\pgfsetstrokecolor{currentstroke}%
\pgfsetdash{}{0pt}%
\pgfpathmoveto{\pgfqpoint{4.632839in}{3.036878in}}%
\pgfpathlineto{\pgfqpoint{4.645750in}{3.030799in}}%
\pgfpathlineto{\pgfqpoint{4.658667in}{3.024851in}}%
\pgfpathlineto{\pgfqpoint{4.671590in}{3.019031in}}%
\pgfpathlineto{\pgfqpoint{4.684520in}{3.013339in}}%
\pgfpathlineto{\pgfqpoint{4.691764in}{3.025742in}}%
\pgfpathlineto{\pgfqpoint{4.699006in}{3.038232in}}%
\pgfpathlineto{\pgfqpoint{4.706245in}{3.050809in}}%
\pgfpathlineto{\pgfqpoint{4.713482in}{3.063477in}}%
\pgfpathlineto{\pgfqpoint{4.700564in}{3.069376in}}%
\pgfpathlineto{\pgfqpoint{4.687652in}{3.075403in}}%
\pgfpathlineto{\pgfqpoint{4.674745in}{3.081559in}}%
\pgfpathlineto{\pgfqpoint{4.661845in}{3.087845in}}%
\pgfpathlineto{\pgfqpoint{4.654597in}{3.074964in}}%
\pgfpathlineto{\pgfqpoint{4.647347in}{3.062177in}}%
\pgfpathlineto{\pgfqpoint{4.640094in}{3.049482in}}%
\pgfpathlineto{\pgfqpoint{4.632839in}{3.036878in}}%
\pgfpathclose%
\pgfusepath{fill}%
\end{pgfscope}%
\begin{pgfscope}%
\pgfpathrectangle{\pgfqpoint{1.254980in}{0.150000in}}{\pgfqpoint{5.490039in}{5.490039in}}%
\pgfusepath{clip}%
\pgfsetbuttcap%
\pgfsetroundjoin%
\definecolor{currentfill}{rgb}{0.237441,0.305202,0.541921}%
\pgfsetfillcolor{currentfill}%
\pgfsetfillopacity{0.700000}%
\pgfsetlinewidth{0.000000pt}%
\definecolor{currentstroke}{rgb}{0.000000,0.000000,0.000000}%
\pgfsetstrokecolor{currentstroke}%
\pgfsetdash{}{0pt}%
\pgfpathmoveto{\pgfqpoint{4.287794in}{3.030760in}}%
\pgfpathlineto{\pgfqpoint{4.300638in}{3.022523in}}%
\pgfpathlineto{\pgfqpoint{4.313487in}{3.014425in}}%
\pgfpathlineto{\pgfqpoint{4.326339in}{3.006466in}}%
\pgfpathlineto{\pgfqpoint{4.339196in}{2.998646in}}%
\pgfpathlineto{\pgfqpoint{4.346529in}{3.010947in}}%
\pgfpathlineto{\pgfqpoint{4.353858in}{3.023331in}}%
\pgfpathlineto{\pgfqpoint{4.361184in}{3.035800in}}%
\pgfpathlineto{\pgfqpoint{4.368506in}{3.048353in}}%
\pgfpathlineto{\pgfqpoint{4.355660in}{3.056335in}}%
\pgfpathlineto{\pgfqpoint{4.342817in}{3.064455in}}%
\pgfpathlineto{\pgfqpoint{4.329979in}{3.072713in}}%
\pgfpathlineto{\pgfqpoint{4.317144in}{3.081111in}}%
\pgfpathlineto{\pgfqpoint{4.309811in}{3.068391in}}%
\pgfpathlineto{\pgfqpoint{4.302475in}{3.055760in}}%
\pgfpathlineto{\pgfqpoint{4.295136in}{3.043217in}}%
\pgfpathlineto{\pgfqpoint{4.287794in}{3.030760in}}%
\pgfpathclose%
\pgfusepath{fill}%
\end{pgfscope}%
\begin{pgfscope}%
\pgfpathrectangle{\pgfqpoint{1.254980in}{0.150000in}}{\pgfqpoint{5.490039in}{5.490039in}}%
\pgfusepath{clip}%
\pgfsetbuttcap%
\pgfsetroundjoin%
\definecolor{currentfill}{rgb}{0.192357,0.403199,0.555836}%
\pgfsetfillcolor{currentfill}%
\pgfsetfillopacity{0.700000}%
\pgfsetlinewidth{0.000000pt}%
\definecolor{currentstroke}{rgb}{0.000000,0.000000,0.000000}%
\pgfsetstrokecolor{currentstroke}%
\pgfsetdash{}{0pt}%
\pgfpathmoveto{\pgfqpoint{3.737556in}{3.268295in}}%
\pgfpathlineto{\pgfqpoint{3.750365in}{3.254976in}}%
\pgfpathlineto{\pgfqpoint{3.763175in}{3.241823in}}%
\pgfpathlineto{\pgfqpoint{3.775984in}{3.228835in}}%
\pgfpathlineto{\pgfqpoint{3.788793in}{3.216011in}}%
\pgfpathlineto{\pgfqpoint{3.796266in}{3.228423in}}%
\pgfpathlineto{\pgfqpoint{3.803735in}{3.240935in}}%
\pgfpathlineto{\pgfqpoint{3.811199in}{3.253547in}}%
\pgfpathlineto{\pgfqpoint{3.818659in}{3.266261in}}%
\pgfpathlineto{\pgfqpoint{3.805861in}{3.279200in}}%
\pgfpathlineto{\pgfqpoint{3.793063in}{3.292303in}}%
\pgfpathlineto{\pgfqpoint{3.780265in}{3.305570in}}%
\pgfpathlineto{\pgfqpoint{3.767467in}{3.319004in}}%
\pgfpathlineto{\pgfqpoint{3.759996in}{3.306169in}}%
\pgfpathlineto{\pgfqpoint{3.752520in}{3.293439in}}%
\pgfpathlineto{\pgfqpoint{3.745040in}{3.280815in}}%
\pgfpathlineto{\pgfqpoint{3.737556in}{3.268295in}}%
\pgfpathclose%
\pgfusepath{fill}%
\end{pgfscope}%
\begin{pgfscope}%
\pgfpathrectangle{\pgfqpoint{1.254980in}{0.150000in}}{\pgfqpoint{5.490039in}{5.490039in}}%
\pgfusepath{clip}%
\pgfsetbuttcap%
\pgfsetroundjoin%
\definecolor{currentfill}{rgb}{0.233603,0.313828,0.543914}%
\pgfsetfillcolor{currentfill}%
\pgfsetfillopacity{0.700000}%
\pgfsetlinewidth{0.000000pt}%
\definecolor{currentstroke}{rgb}{0.000000,0.000000,0.000000}%
\pgfsetstrokecolor{currentstroke}%
\pgfsetdash{}{0pt}%
\pgfpathmoveto{\pgfqpoint{4.155679in}{3.051587in}}%
\pgfpathlineto{\pgfqpoint{4.168506in}{3.042358in}}%
\pgfpathlineto{\pgfqpoint{4.181336in}{3.033274in}}%
\pgfpathlineto{\pgfqpoint{4.194170in}{3.024333in}}%
\pgfpathlineto{\pgfqpoint{4.207006in}{3.015535in}}%
\pgfpathlineto{\pgfqpoint{4.214373in}{3.027803in}}%
\pgfpathlineto{\pgfqpoint{4.221736in}{3.040155in}}%
\pgfpathlineto{\pgfqpoint{4.229096in}{3.052591in}}%
\pgfpathlineto{\pgfqpoint{4.236452in}{3.065115in}}%
\pgfpathlineto{\pgfqpoint{4.223626in}{3.074058in}}%
\pgfpathlineto{\pgfqpoint{4.210803in}{3.083144in}}%
\pgfpathlineto{\pgfqpoint{4.197982in}{3.092374in}}%
\pgfpathlineto{\pgfqpoint{4.185166in}{3.101748in}}%
\pgfpathlineto{\pgfqpoint{4.177799in}{3.089073in}}%
\pgfpathlineto{\pgfqpoint{4.170429in}{3.076489in}}%
\pgfpathlineto{\pgfqpoint{4.163056in}{3.063994in}}%
\pgfpathlineto{\pgfqpoint{4.155679in}{3.051587in}}%
\pgfpathclose%
\pgfusepath{fill}%
\end{pgfscope}%
\begin{pgfscope}%
\pgfpathrectangle{\pgfqpoint{1.254980in}{0.150000in}}{\pgfqpoint{5.490039in}{5.490039in}}%
\pgfusepath{clip}%
\pgfsetbuttcap%
\pgfsetroundjoin%
\definecolor{currentfill}{rgb}{0.144759,0.519093,0.556572}%
\pgfsetfillcolor{currentfill}%
\pgfsetfillopacity{0.700000}%
\pgfsetlinewidth{0.000000pt}%
\definecolor{currentstroke}{rgb}{0.000000,0.000000,0.000000}%
\pgfsetstrokecolor{currentstroke}%
\pgfsetdash{}{0pt}%
\pgfpathmoveto{\pgfqpoint{3.481103in}{3.571024in}}%
\pgfpathlineto{\pgfqpoint{3.493947in}{3.554175in}}%
\pgfpathlineto{\pgfqpoint{3.506787in}{3.537514in}}%
\pgfpathlineto{\pgfqpoint{3.519624in}{3.521039in}}%
\pgfpathlineto{\pgfqpoint{3.532459in}{3.504748in}}%
\pgfpathlineto{\pgfqpoint{3.539990in}{3.517706in}}%
\pgfpathlineto{\pgfqpoint{3.547516in}{3.530784in}}%
\pgfpathlineto{\pgfqpoint{3.555038in}{3.543984in}}%
\pgfpathlineto{\pgfqpoint{3.562554in}{3.557305in}}%
\pgfpathlineto{\pgfqpoint{3.549732in}{3.573713in}}%
\pgfpathlineto{\pgfqpoint{3.536906in}{3.590304in}}%
\pgfpathlineto{\pgfqpoint{3.524078in}{3.607082in}}%
\pgfpathlineto{\pgfqpoint{3.511247in}{3.624047in}}%
\pgfpathlineto{\pgfqpoint{3.503718in}{3.610603in}}%
\pgfpathlineto{\pgfqpoint{3.496185in}{3.597285in}}%
\pgfpathlineto{\pgfqpoint{3.488647in}{3.584092in}}%
\pgfpathlineto{\pgfqpoint{3.481103in}{3.571024in}}%
\pgfpathclose%
\pgfusepath{fill}%
\end{pgfscope}%
\begin{pgfscope}%
\pgfpathrectangle{\pgfqpoint{1.254980in}{0.150000in}}{\pgfqpoint{5.490039in}{5.490039in}}%
\pgfusepath{clip}%
\pgfsetbuttcap%
\pgfsetroundjoin%
\definecolor{currentfill}{rgb}{0.239346,0.300855,0.540844}%
\pgfsetfillcolor{currentfill}%
\pgfsetfillopacity{0.700000}%
\pgfsetlinewidth{0.000000pt}%
\definecolor{currentstroke}{rgb}{0.000000,0.000000,0.000000}%
\pgfsetstrokecolor{currentstroke}%
\pgfsetdash{}{0pt}%
\pgfpathmoveto{\pgfqpoint{4.419936in}{3.017800in}}%
\pgfpathlineto{\pgfqpoint{4.432805in}{3.010502in}}%
\pgfpathlineto{\pgfqpoint{4.445678in}{3.003338in}}%
\pgfpathlineto{\pgfqpoint{4.458556in}{2.996310in}}%
\pgfpathlineto{\pgfqpoint{4.471440in}{2.989415in}}%
\pgfpathlineto{\pgfqpoint{4.478739in}{3.001716in}}%
\pgfpathlineto{\pgfqpoint{4.486035in}{3.014098in}}%
\pgfpathlineto{\pgfqpoint{4.493328in}{3.026564in}}%
\pgfpathlineto{\pgfqpoint{4.500618in}{3.039115in}}%
\pgfpathlineto{\pgfqpoint{4.487745in}{3.046185in}}%
\pgfpathlineto{\pgfqpoint{4.474876in}{3.053390in}}%
\pgfpathlineto{\pgfqpoint{4.462013in}{3.060730in}}%
\pgfpathlineto{\pgfqpoint{4.449155in}{3.068204in}}%
\pgfpathlineto{\pgfqpoint{4.441854in}{3.055471in}}%
\pgfpathlineto{\pgfqpoint{4.434551in}{3.042827in}}%
\pgfpathlineto{\pgfqpoint{4.427245in}{3.030271in}}%
\pgfpathlineto{\pgfqpoint{4.419936in}{3.017800in}}%
\pgfpathclose%
\pgfusepath{fill}%
\end{pgfscope}%
\begin{pgfscope}%
\pgfpathrectangle{\pgfqpoint{1.254980in}{0.150000in}}{\pgfqpoint{5.490039in}{5.490039in}}%
\pgfusepath{clip}%
\pgfsetbuttcap%
\pgfsetroundjoin%
\definecolor{currentfill}{rgb}{0.201239,0.383670,0.554294}%
\pgfsetfillcolor{currentfill}%
\pgfsetfillopacity{0.700000}%
\pgfsetlinewidth{0.000000pt}%
\definecolor{currentstroke}{rgb}{0.000000,0.000000,0.000000}%
\pgfsetstrokecolor{currentstroke}%
\pgfsetdash{}{0pt}%
\pgfpathmoveto{\pgfqpoint{3.788793in}{3.216011in}}%
\pgfpathlineto{\pgfqpoint{3.801602in}{3.203351in}}%
\pgfpathlineto{\pgfqpoint{3.814412in}{3.190852in}}%
\pgfpathlineto{\pgfqpoint{3.827221in}{3.178515in}}%
\pgfpathlineto{\pgfqpoint{3.840032in}{3.166339in}}%
\pgfpathlineto{\pgfqpoint{3.847494in}{3.178642in}}%
\pgfpathlineto{\pgfqpoint{3.854951in}{3.191041in}}%
\pgfpathlineto{\pgfqpoint{3.862405in}{3.203536in}}%
\pgfpathlineto{\pgfqpoint{3.869854in}{3.216130in}}%
\pgfpathlineto{\pgfqpoint{3.857055in}{3.228421in}}%
\pgfpathlineto{\pgfqpoint{3.844256in}{3.240873in}}%
\pgfpathlineto{\pgfqpoint{3.831457in}{3.253486in}}%
\pgfpathlineto{\pgfqpoint{3.818659in}{3.266261in}}%
\pgfpathlineto{\pgfqpoint{3.811199in}{3.253547in}}%
\pgfpathlineto{\pgfqpoint{3.803735in}{3.240935in}}%
\pgfpathlineto{\pgfqpoint{3.796266in}{3.228423in}}%
\pgfpathlineto{\pgfqpoint{3.788793in}{3.216011in}}%
\pgfpathclose%
\pgfusepath{fill}%
\end{pgfscope}%
\begin{pgfscope}%
\pgfpathrectangle{\pgfqpoint{1.254980in}{0.150000in}}{\pgfqpoint{5.490039in}{5.490039in}}%
\pgfusepath{clip}%
\pgfsetbuttcap%
\pgfsetroundjoin%
\definecolor{currentfill}{rgb}{0.132268,0.655014,0.519661}%
\pgfsetfillcolor{currentfill}%
\pgfsetfillopacity{0.700000}%
\pgfsetlinewidth{0.000000pt}%
\definecolor{currentstroke}{rgb}{0.000000,0.000000,0.000000}%
\pgfsetstrokecolor{currentstroke}%
\pgfsetdash{}{0pt}%
\pgfpathmoveto{\pgfqpoint{3.305414in}{3.921959in}}%
\pgfpathlineto{\pgfqpoint{3.318315in}{3.901826in}}%
\pgfpathlineto{\pgfqpoint{3.331209in}{3.881901in}}%
\pgfpathlineto{\pgfqpoint{3.344099in}{3.862183in}}%
\pgfpathlineto{\pgfqpoint{3.356983in}{3.842671in}}%
\pgfpathlineto{\pgfqpoint{3.364542in}{3.856626in}}%
\pgfpathlineto{\pgfqpoint{3.372096in}{3.870724in}}%
\pgfpathlineto{\pgfqpoint{3.379644in}{3.884965in}}%
\pgfpathlineto{\pgfqpoint{3.387188in}{3.899350in}}%
\pgfpathlineto{\pgfqpoint{3.374316in}{3.918998in}}%
\pgfpathlineto{\pgfqpoint{3.361438in}{3.938851in}}%
\pgfpathlineto{\pgfqpoint{3.348556in}{3.958911in}}%
\pgfpathlineto{\pgfqpoint{3.335668in}{3.979180in}}%
\pgfpathlineto{\pgfqpoint{3.328113in}{3.964652in}}%
\pgfpathlineto{\pgfqpoint{3.320552in}{3.950274in}}%
\pgfpathlineto{\pgfqpoint{3.312986in}{3.936043in}}%
\pgfpathlineto{\pgfqpoint{3.305414in}{3.921959in}}%
\pgfpathclose%
\pgfusepath{fill}%
\end{pgfscope}%
\begin{pgfscope}%
\pgfpathrectangle{\pgfqpoint{1.254980in}{0.150000in}}{\pgfqpoint{5.490039in}{5.490039in}}%
\pgfusepath{clip}%
\pgfsetbuttcap%
\pgfsetroundjoin%
\definecolor{currentfill}{rgb}{0.133743,0.548535,0.553541}%
\pgfsetfillcolor{currentfill}%
\pgfsetfillopacity{0.700000}%
\pgfsetlinewidth{0.000000pt}%
\definecolor{currentstroke}{rgb}{0.000000,0.000000,0.000000}%
\pgfsetstrokecolor{currentstroke}%
\pgfsetdash{}{0pt}%
\pgfpathmoveto{\pgfqpoint{3.429697in}{3.640311in}}%
\pgfpathlineto{\pgfqpoint{3.442554in}{3.622702in}}%
\pgfpathlineto{\pgfqpoint{3.455407in}{3.605286in}}%
\pgfpathlineto{\pgfqpoint{3.468257in}{3.588060in}}%
\pgfpathlineto{\pgfqpoint{3.481103in}{3.571024in}}%
\pgfpathlineto{\pgfqpoint{3.488647in}{3.584092in}}%
\pgfpathlineto{\pgfqpoint{3.496185in}{3.597285in}}%
\pgfpathlineto{\pgfqpoint{3.503718in}{3.610603in}}%
\pgfpathlineto{\pgfqpoint{3.511247in}{3.624047in}}%
\pgfpathlineto{\pgfqpoint{3.498412in}{3.641200in}}%
\pgfpathlineto{\pgfqpoint{3.485575in}{3.658543in}}%
\pgfpathlineto{\pgfqpoint{3.472733in}{3.676076in}}%
\pgfpathlineto{\pgfqpoint{3.459889in}{3.693801in}}%
\pgfpathlineto{\pgfqpoint{3.452348in}{3.680234in}}%
\pgfpathlineto{\pgfqpoint{3.444803in}{3.666797in}}%
\pgfpathlineto{\pgfqpoint{3.437253in}{3.653490in}}%
\pgfpathlineto{\pgfqpoint{3.429697in}{3.640311in}}%
\pgfpathclose%
\pgfusepath{fill}%
\end{pgfscope}%
\begin{pgfscope}%
\pgfpathrectangle{\pgfqpoint{1.254980in}{0.150000in}}{\pgfqpoint{5.490039in}{5.490039in}}%
\pgfusepath{clip}%
\pgfsetbuttcap%
\pgfsetroundjoin%
\definecolor{currentfill}{rgb}{0.227802,0.326594,0.546532}%
\pgfsetfillcolor{currentfill}%
\pgfsetfillopacity{0.700000}%
\pgfsetlinewidth{0.000000pt}%
\definecolor{currentstroke}{rgb}{0.000000,0.000000,0.000000}%
\pgfsetstrokecolor{currentstroke}%
\pgfsetdash{}{0pt}%
\pgfpathmoveto{\pgfqpoint{4.023530in}{3.080865in}}%
\pgfpathlineto{\pgfqpoint{4.036347in}{3.070588in}}%
\pgfpathlineto{\pgfqpoint{4.049166in}{3.060460in}}%
\pgfpathlineto{\pgfqpoint{4.061988in}{3.050482in}}%
\pgfpathlineto{\pgfqpoint{4.074812in}{3.040652in}}%
\pgfpathlineto{\pgfqpoint{4.082214in}{3.052849in}}%
\pgfpathlineto{\pgfqpoint{4.089613in}{3.065131in}}%
\pgfpathlineto{\pgfqpoint{4.097008in}{3.077500in}}%
\pgfpathlineto{\pgfqpoint{4.104399in}{3.089958in}}%
\pgfpathlineto{\pgfqpoint{4.091585in}{3.099918in}}%
\pgfpathlineto{\pgfqpoint{4.078774in}{3.110026in}}%
\pgfpathlineto{\pgfqpoint{4.065965in}{3.120283in}}%
\pgfpathlineto{\pgfqpoint{4.053159in}{3.130690in}}%
\pgfpathlineto{\pgfqpoint{4.045757in}{3.118097in}}%
\pgfpathlineto{\pgfqpoint{4.038352in}{3.105596in}}%
\pgfpathlineto{\pgfqpoint{4.030943in}{3.093186in}}%
\pgfpathlineto{\pgfqpoint{4.023530in}{3.080865in}}%
\pgfpathclose%
\pgfusepath{fill}%
\end{pgfscope}%
\begin{pgfscope}%
\pgfpathrectangle{\pgfqpoint{1.254980in}{0.150000in}}{\pgfqpoint{5.490039in}{5.490039in}}%
\pgfusepath{clip}%
\pgfsetbuttcap%
\pgfsetroundjoin%
\definecolor{currentfill}{rgb}{0.227802,0.326594,0.546532}%
\pgfsetfillcolor{currentfill}%
\pgfsetfillopacity{0.700000}%
\pgfsetlinewidth{0.000000pt}%
\definecolor{currentstroke}{rgb}{0.000000,0.000000,0.000000}%
\pgfsetstrokecolor{currentstroke}%
\pgfsetdash{}{0pt}%
\pgfpathmoveto{\pgfqpoint{4.845894in}{3.070693in}}%
\pgfpathlineto{\pgfqpoint{4.858859in}{3.065708in}}%
\pgfpathlineto{\pgfqpoint{4.871832in}{3.060848in}}%
\pgfpathlineto{\pgfqpoint{4.884811in}{3.056112in}}%
\pgfpathlineto{\pgfqpoint{4.897798in}{3.051499in}}%
\pgfpathlineto{\pgfqpoint{4.904990in}{3.063954in}}%
\pgfpathlineto{\pgfqpoint{4.912180in}{3.076502in}}%
\pgfpathlineto{\pgfqpoint{4.919368in}{3.089144in}}%
\pgfpathlineto{\pgfqpoint{4.906390in}{3.093935in}}%
\pgfpathlineto{\pgfqpoint{4.893419in}{3.098849in}}%
\pgfpathlineto{\pgfqpoint{4.880455in}{3.103886in}}%
\pgfpathlineto{\pgfqpoint{4.867499in}{3.109049in}}%
\pgfpathlineto{\pgfqpoint{4.860299in}{3.096165in}}%
\pgfpathlineto{\pgfqpoint{4.853097in}{3.083381in}}%
\pgfpathlineto{\pgfqpoint{4.845894in}{3.070693in}}%
\pgfpathclose%
\pgfusepath{fill}%
\end{pgfscope}%
\begin{pgfscope}%
\pgfpathrectangle{\pgfqpoint{1.254980in}{0.150000in}}{\pgfqpoint{5.490039in}{5.490039in}}%
\pgfusepath{clip}%
\pgfsetbuttcap%
\pgfsetroundjoin%
\definecolor{currentfill}{rgb}{0.239346,0.300855,0.540844}%
\pgfsetfillcolor{currentfill}%
\pgfsetfillopacity{0.700000}%
\pgfsetlinewidth{0.000000pt}%
\definecolor{currentstroke}{rgb}{0.000000,0.000000,0.000000}%
\pgfsetstrokecolor{currentstroke}%
\pgfsetdash{}{0pt}%
\pgfpathmoveto{\pgfqpoint{4.552161in}{3.012162in}}%
\pgfpathlineto{\pgfqpoint{4.565060in}{3.005754in}}%
\pgfpathlineto{\pgfqpoint{4.577965in}{2.999477in}}%
\pgfpathlineto{\pgfqpoint{4.590875in}{2.993331in}}%
\pgfpathlineto{\pgfqpoint{4.603791in}{2.987315in}}%
\pgfpathlineto{\pgfqpoint{4.611057in}{2.999581in}}%
\pgfpathlineto{\pgfqpoint{4.618321in}{3.011929in}}%
\pgfpathlineto{\pgfqpoint{4.625581in}{3.024360in}}%
\pgfpathlineto{\pgfqpoint{4.632839in}{3.036878in}}%
\pgfpathlineto{\pgfqpoint{4.619934in}{3.043085in}}%
\pgfpathlineto{\pgfqpoint{4.607034in}{3.049423in}}%
\pgfpathlineto{\pgfqpoint{4.594140in}{3.055892in}}%
\pgfpathlineto{\pgfqpoint{4.581251in}{3.062492in}}%
\pgfpathlineto{\pgfqpoint{4.573983in}{3.049777in}}%
\pgfpathlineto{\pgfqpoint{4.566712in}{3.037152in}}%
\pgfpathlineto{\pgfqpoint{4.559438in}{3.024614in}}%
\pgfpathlineto{\pgfqpoint{4.552161in}{3.012162in}}%
\pgfpathclose%
\pgfusepath{fill}%
\end{pgfscope}%
\begin{pgfscope}%
\pgfpathrectangle{\pgfqpoint{1.254980in}{0.150000in}}{\pgfqpoint{5.490039in}{5.490039in}}%
\pgfusepath{clip}%
\pgfsetbuttcap%
\pgfsetroundjoin%
\definecolor{currentfill}{rgb}{0.233603,0.313828,0.543914}%
\pgfsetfillcolor{currentfill}%
\pgfsetfillopacity{0.700000}%
\pgfsetlinewidth{0.000000pt}%
\definecolor{currentstroke}{rgb}{0.000000,0.000000,0.000000}%
\pgfsetstrokecolor{currentstroke}%
\pgfsetdash{}{0pt}%
\pgfpathmoveto{\pgfqpoint{4.765218in}{3.041157in}}%
\pgfpathlineto{\pgfqpoint{4.778168in}{3.035894in}}%
\pgfpathlineto{\pgfqpoint{4.791124in}{3.030756in}}%
\pgfpathlineto{\pgfqpoint{4.804088in}{3.025745in}}%
\pgfpathlineto{\pgfqpoint{4.817057in}{3.020859in}}%
\pgfpathlineto{\pgfqpoint{4.824270in}{3.033184in}}%
\pgfpathlineto{\pgfqpoint{4.831480in}{3.045597in}}%
\pgfpathlineto{\pgfqpoint{4.838688in}{3.058099in}}%
\pgfpathlineto{\pgfqpoint{4.845894in}{3.070693in}}%
\pgfpathlineto{\pgfqpoint{4.832935in}{3.075803in}}%
\pgfpathlineto{\pgfqpoint{4.819983in}{3.081037in}}%
\pgfpathlineto{\pgfqpoint{4.807038in}{3.086397in}}%
\pgfpathlineto{\pgfqpoint{4.794099in}{3.091884in}}%
\pgfpathlineto{\pgfqpoint{4.786882in}{3.079061in}}%
\pgfpathlineto{\pgfqpoint{4.779663in}{3.066334in}}%
\pgfpathlineto{\pgfqpoint{4.772442in}{3.053700in}}%
\pgfpathlineto{\pgfqpoint{4.765218in}{3.041157in}}%
\pgfpathclose%
\pgfusepath{fill}%
\end{pgfscope}%
\begin{pgfscope}%
\pgfpathrectangle{\pgfqpoint{1.254980in}{0.150000in}}{\pgfqpoint{5.490039in}{5.490039in}}%
\pgfusepath{clip}%
\pgfsetbuttcap%
\pgfsetroundjoin%
\definecolor{currentfill}{rgb}{0.210503,0.363727,0.552206}%
\pgfsetfillcolor{currentfill}%
\pgfsetfillopacity{0.700000}%
\pgfsetlinewidth{0.000000pt}%
\definecolor{currentstroke}{rgb}{0.000000,0.000000,0.000000}%
\pgfsetstrokecolor{currentstroke}%
\pgfsetdash{}{0pt}%
\pgfpathmoveto{\pgfqpoint{3.840032in}{3.166339in}}%
\pgfpathlineto{\pgfqpoint{3.852843in}{3.154322in}}%
\pgfpathlineto{\pgfqpoint{3.865654in}{3.142464in}}%
\pgfpathlineto{\pgfqpoint{3.878467in}{3.130764in}}%
\pgfpathlineto{\pgfqpoint{3.891280in}{3.119222in}}%
\pgfpathlineto{\pgfqpoint{3.898731in}{3.131416in}}%
\pgfpathlineto{\pgfqpoint{3.906178in}{3.143702in}}%
\pgfpathlineto{\pgfqpoint{3.913620in}{3.156082in}}%
\pgfpathlineto{\pgfqpoint{3.921059in}{3.168555in}}%
\pgfpathlineto{\pgfqpoint{3.908256in}{3.180212in}}%
\pgfpathlineto{\pgfqpoint{3.895455in}{3.192027in}}%
\pgfpathlineto{\pgfqpoint{3.882654in}{3.203999in}}%
\pgfpathlineto{\pgfqpoint{3.869854in}{3.216130in}}%
\pgfpathlineto{\pgfqpoint{3.862405in}{3.203536in}}%
\pgfpathlineto{\pgfqpoint{3.854951in}{3.191041in}}%
\pgfpathlineto{\pgfqpoint{3.847494in}{3.178642in}}%
\pgfpathlineto{\pgfqpoint{3.840032in}{3.166339in}}%
\pgfpathclose%
\pgfusepath{fill}%
\end{pgfscope}%
\begin{pgfscope}%
\pgfpathrectangle{\pgfqpoint{1.254980in}{0.150000in}}{\pgfqpoint{5.490039in}{5.490039in}}%
\pgfusepath{clip}%
\pgfsetbuttcap%
\pgfsetroundjoin%
\definecolor{currentfill}{rgb}{0.124395,0.578002,0.548287}%
\pgfsetfillcolor{currentfill}%
\pgfsetfillopacity{0.700000}%
\pgfsetlinewidth{0.000000pt}%
\definecolor{currentstroke}{rgb}{0.000000,0.000000,0.000000}%
\pgfsetstrokecolor{currentstroke}%
\pgfsetdash{}{0pt}%
\pgfpathmoveto{\pgfqpoint{3.378229in}{3.712688in}}%
\pgfpathlineto{\pgfqpoint{3.391103in}{3.694300in}}%
\pgfpathlineto{\pgfqpoint{3.403971in}{3.676108in}}%
\pgfpathlineto{\pgfqpoint{3.416836in}{3.658112in}}%
\pgfpathlineto{\pgfqpoint{3.429697in}{3.640311in}}%
\pgfpathlineto{\pgfqpoint{3.437253in}{3.653490in}}%
\pgfpathlineto{\pgfqpoint{3.444803in}{3.666797in}}%
\pgfpathlineto{\pgfqpoint{3.452348in}{3.680234in}}%
\pgfpathlineto{\pgfqpoint{3.459889in}{3.693801in}}%
\pgfpathlineto{\pgfqpoint{3.447040in}{3.711720in}}%
\pgfpathlineto{\pgfqpoint{3.434188in}{3.729833in}}%
\pgfpathlineto{\pgfqpoint{3.421331in}{3.748142in}}%
\pgfpathlineto{\pgfqpoint{3.408471in}{3.766648in}}%
\pgfpathlineto{\pgfqpoint{3.400918in}{3.752957in}}%
\pgfpathlineto{\pgfqpoint{3.393361in}{3.739401in}}%
\pgfpathlineto{\pgfqpoint{3.385798in}{3.725979in}}%
\pgfpathlineto{\pgfqpoint{3.378229in}{3.712688in}}%
\pgfpathclose%
\pgfusepath{fill}%
\end{pgfscope}%
\begin{pgfscope}%
\pgfpathrectangle{\pgfqpoint{1.254980in}{0.150000in}}{\pgfqpoint{5.490039in}{5.490039in}}%
\pgfusepath{clip}%
\pgfsetbuttcap%
\pgfsetroundjoin%
\definecolor{currentfill}{rgb}{0.239346,0.300855,0.540844}%
\pgfsetfillcolor{currentfill}%
\pgfsetfillopacity{0.700000}%
\pgfsetlinewidth{0.000000pt}%
\definecolor{currentstroke}{rgb}{0.000000,0.000000,0.000000}%
\pgfsetstrokecolor{currentstroke}%
\pgfsetdash{}{0pt}%
\pgfpathmoveto{\pgfqpoint{4.207006in}{3.015535in}}%
\pgfpathlineto{\pgfqpoint{4.219847in}{3.006880in}}%
\pgfpathlineto{\pgfqpoint{4.232690in}{2.998366in}}%
\pgfpathlineto{\pgfqpoint{4.245538in}{2.989993in}}%
\pgfpathlineto{\pgfqpoint{4.258389in}{2.981761in}}%
\pgfpathlineto{\pgfqpoint{4.265745in}{2.993890in}}%
\pgfpathlineto{\pgfqpoint{4.273098in}{3.006098in}}%
\pgfpathlineto{\pgfqpoint{4.280448in}{3.018388in}}%
\pgfpathlineto{\pgfqpoint{4.287794in}{3.030760in}}%
\pgfpathlineto{\pgfqpoint{4.274953in}{3.039137in}}%
\pgfpathlineto{\pgfqpoint{4.262116in}{3.047655in}}%
\pgfpathlineto{\pgfqpoint{4.249282in}{3.056314in}}%
\pgfpathlineto{\pgfqpoint{4.236452in}{3.065115in}}%
\pgfpathlineto{\pgfqpoint{4.229096in}{3.052591in}}%
\pgfpathlineto{\pgfqpoint{4.221736in}{3.040155in}}%
\pgfpathlineto{\pgfqpoint{4.214373in}{3.027803in}}%
\pgfpathlineto{\pgfqpoint{4.207006in}{3.015535in}}%
\pgfpathclose%
\pgfusepath{fill}%
\end{pgfscope}%
\begin{pgfscope}%
\pgfpathrectangle{\pgfqpoint{1.254980in}{0.150000in}}{\pgfqpoint{5.490039in}{5.490039in}}%
\pgfusepath{clip}%
\pgfsetbuttcap%
\pgfsetroundjoin%
\definecolor{currentfill}{rgb}{0.241237,0.296485,0.539709}%
\pgfsetfillcolor{currentfill}%
\pgfsetfillopacity{0.700000}%
\pgfsetlinewidth{0.000000pt}%
\definecolor{currentstroke}{rgb}{0.000000,0.000000,0.000000}%
\pgfsetstrokecolor{currentstroke}%
\pgfsetdash{}{0pt}%
\pgfpathmoveto{\pgfqpoint{4.339196in}{2.998646in}}%
\pgfpathlineto{\pgfqpoint{4.352057in}{2.990963in}}%
\pgfpathlineto{\pgfqpoint{4.364922in}{2.983417in}}%
\pgfpathlineto{\pgfqpoint{4.377792in}{2.976008in}}%
\pgfpathlineto{\pgfqpoint{4.390667in}{2.968735in}}%
\pgfpathlineto{\pgfqpoint{4.397989in}{2.980882in}}%
\pgfpathlineto{\pgfqpoint{4.405308in}{2.993107in}}%
\pgfpathlineto{\pgfqpoint{4.412623in}{3.005412in}}%
\pgfpathlineto{\pgfqpoint{4.419936in}{3.017800in}}%
\pgfpathlineto{\pgfqpoint{4.407072in}{3.025233in}}%
\pgfpathlineto{\pgfqpoint{4.394212in}{3.032803in}}%
\pgfpathlineto{\pgfqpoint{4.381357in}{3.040510in}}%
\pgfpathlineto{\pgfqpoint{4.368506in}{3.048353in}}%
\pgfpathlineto{\pgfqpoint{4.361184in}{3.035800in}}%
\pgfpathlineto{\pgfqpoint{4.353858in}{3.023331in}}%
\pgfpathlineto{\pgfqpoint{4.346529in}{3.010947in}}%
\pgfpathlineto{\pgfqpoint{4.339196in}{2.998646in}}%
\pgfpathclose%
\pgfusepath{fill}%
\end{pgfscope}%
\begin{pgfscope}%
\pgfpathrectangle{\pgfqpoint{1.254980in}{0.150000in}}{\pgfqpoint{5.490039in}{5.490039in}}%
\pgfusepath{clip}%
\pgfsetbuttcap%
\pgfsetroundjoin%
\definecolor{currentfill}{rgb}{0.233603,0.313828,0.543914}%
\pgfsetfillcolor{currentfill}%
\pgfsetfillopacity{0.700000}%
\pgfsetlinewidth{0.000000pt}%
\definecolor{currentstroke}{rgb}{0.000000,0.000000,0.000000}%
\pgfsetstrokecolor{currentstroke}%
\pgfsetdash{}{0pt}%
\pgfpathmoveto{\pgfqpoint{4.074812in}{3.040652in}}%
\pgfpathlineto{\pgfqpoint{4.087638in}{3.030969in}}%
\pgfpathlineto{\pgfqpoint{4.100467in}{3.021434in}}%
\pgfpathlineto{\pgfqpoint{4.113299in}{3.012044in}}%
\pgfpathlineto{\pgfqpoint{4.126133in}{3.002800in}}%
\pgfpathlineto{\pgfqpoint{4.133525in}{3.014873in}}%
\pgfpathlineto{\pgfqpoint{4.140914in}{3.027028in}}%
\pgfpathlineto{\pgfqpoint{4.148298in}{3.039265in}}%
\pgfpathlineto{\pgfqpoint{4.155679in}{3.051587in}}%
\pgfpathlineto{\pgfqpoint{4.142855in}{3.060961in}}%
\pgfpathlineto{\pgfqpoint{4.130033in}{3.070480in}}%
\pgfpathlineto{\pgfqpoint{4.117215in}{3.080145in}}%
\pgfpathlineto{\pgfqpoint{4.104399in}{3.089958in}}%
\pgfpathlineto{\pgfqpoint{4.097008in}{3.077500in}}%
\pgfpathlineto{\pgfqpoint{4.089613in}{3.065131in}}%
\pgfpathlineto{\pgfqpoint{4.082214in}{3.052849in}}%
\pgfpathlineto{\pgfqpoint{4.074812in}{3.040652in}}%
\pgfpathclose%
\pgfusepath{fill}%
\end{pgfscope}%
\begin{pgfscope}%
\pgfpathrectangle{\pgfqpoint{1.254980in}{0.150000in}}{\pgfqpoint{5.490039in}{5.490039in}}%
\pgfusepath{clip}%
\pgfsetbuttcap%
\pgfsetroundjoin%
\definecolor{currentfill}{rgb}{0.218130,0.347432,0.550038}%
\pgfsetfillcolor{currentfill}%
\pgfsetfillopacity{0.700000}%
\pgfsetlinewidth{0.000000pt}%
\definecolor{currentstroke}{rgb}{0.000000,0.000000,0.000000}%
\pgfsetstrokecolor{currentstroke}%
\pgfsetdash{}{0pt}%
\pgfpathmoveto{\pgfqpoint{3.891280in}{3.119222in}}%
\pgfpathlineto{\pgfqpoint{3.904095in}{3.107835in}}%
\pgfpathlineto{\pgfqpoint{3.916911in}{3.096605in}}%
\pgfpathlineto{\pgfqpoint{3.929728in}{3.085529in}}%
\pgfpathlineto{\pgfqpoint{3.942547in}{3.074607in}}%
\pgfpathlineto{\pgfqpoint{3.949987in}{3.086693in}}%
\pgfpathlineto{\pgfqpoint{3.957422in}{3.098867in}}%
\pgfpathlineto{\pgfqpoint{3.964854in}{3.111130in}}%
\pgfpathlineto{\pgfqpoint{3.972282in}{3.123483in}}%
\pgfpathlineto{\pgfqpoint{3.959474in}{3.134519in}}%
\pgfpathlineto{\pgfqpoint{3.946667in}{3.145709in}}%
\pgfpathlineto{\pgfqpoint{3.933862in}{3.157054in}}%
\pgfpathlineto{\pgfqpoint{3.921059in}{3.168555in}}%
\pgfpathlineto{\pgfqpoint{3.913620in}{3.156082in}}%
\pgfpathlineto{\pgfqpoint{3.906178in}{3.143702in}}%
\pgfpathlineto{\pgfqpoint{3.898731in}{3.131416in}}%
\pgfpathlineto{\pgfqpoint{3.891280in}{3.119222in}}%
\pgfpathclose%
\pgfusepath{fill}%
\end{pgfscope}%
\begin{pgfscope}%
\pgfpathrectangle{\pgfqpoint{1.254980in}{0.150000in}}{\pgfqpoint{5.490039in}{5.490039in}}%
\pgfusepath{clip}%
\pgfsetbuttcap%
\pgfsetroundjoin%
\definecolor{currentfill}{rgb}{0.237441,0.305202,0.541921}%
\pgfsetfillcolor{currentfill}%
\pgfsetfillopacity{0.700000}%
\pgfsetlinewidth{0.000000pt}%
\definecolor{currentstroke}{rgb}{0.000000,0.000000,0.000000}%
\pgfsetstrokecolor{currentstroke}%
\pgfsetdash{}{0pt}%
\pgfpathmoveto{\pgfqpoint{4.684520in}{3.013339in}}%
\pgfpathlineto{\pgfqpoint{4.697455in}{3.007776in}}%
\pgfpathlineto{\pgfqpoint{4.710396in}{3.002340in}}%
\pgfpathlineto{\pgfqpoint{4.723344in}{2.997031in}}%
\pgfpathlineto{\pgfqpoint{4.736298in}{2.991849in}}%
\pgfpathlineto{\pgfqpoint{4.743532in}{3.004051in}}%
\pgfpathlineto{\pgfqpoint{4.750763in}{3.016334in}}%
\pgfpathlineto{\pgfqpoint{4.757992in}{3.028702in}}%
\pgfpathlineto{\pgfqpoint{4.765218in}{3.041157in}}%
\pgfpathlineto{\pgfqpoint{4.752275in}{3.046546in}}%
\pgfpathlineto{\pgfqpoint{4.739338in}{3.052062in}}%
\pgfpathlineto{\pgfqpoint{4.726407in}{3.057706in}}%
\pgfpathlineto{\pgfqpoint{4.713482in}{3.063477in}}%
\pgfpathlineto{\pgfqpoint{4.706245in}{3.050809in}}%
\pgfpathlineto{\pgfqpoint{4.699006in}{3.038232in}}%
\pgfpathlineto{\pgfqpoint{4.691764in}{3.025742in}}%
\pgfpathlineto{\pgfqpoint{4.684520in}{3.013339in}}%
\pgfpathclose%
\pgfusepath{fill}%
\end{pgfscope}%
\begin{pgfscope}%
\pgfpathrectangle{\pgfqpoint{1.254980in}{0.150000in}}{\pgfqpoint{5.490039in}{5.490039in}}%
\pgfusepath{clip}%
\pgfsetbuttcap%
\pgfsetroundjoin%
\definecolor{currentfill}{rgb}{0.243113,0.292092,0.538516}%
\pgfsetfillcolor{currentfill}%
\pgfsetfillopacity{0.700000}%
\pgfsetlinewidth{0.000000pt}%
\definecolor{currentstroke}{rgb}{0.000000,0.000000,0.000000}%
\pgfsetstrokecolor{currentstroke}%
\pgfsetdash{}{0pt}%
\pgfpathmoveto{\pgfqpoint{4.471440in}{2.989415in}}%
\pgfpathlineto{\pgfqpoint{4.484328in}{2.982654in}}%
\pgfpathlineto{\pgfqpoint{4.497222in}{2.976027in}}%
\pgfpathlineto{\pgfqpoint{4.510120in}{2.969531in}}%
\pgfpathlineto{\pgfqpoint{4.523024in}{2.963168in}}%
\pgfpathlineto{\pgfqpoint{4.530313in}{2.975298in}}%
\pgfpathlineto{\pgfqpoint{4.537599in}{2.987506in}}%
\pgfpathlineto{\pgfqpoint{4.544881in}{2.999793in}}%
\pgfpathlineto{\pgfqpoint{4.552161in}{3.012162in}}%
\pgfpathlineto{\pgfqpoint{4.539267in}{3.018701in}}%
\pgfpathlineto{\pgfqpoint{4.526379in}{3.025373in}}%
\pgfpathlineto{\pgfqpoint{4.513496in}{3.032177in}}%
\pgfpathlineto{\pgfqpoint{4.500618in}{3.039115in}}%
\pgfpathlineto{\pgfqpoint{4.493328in}{3.026564in}}%
\pgfpathlineto{\pgfqpoint{4.486035in}{3.014098in}}%
\pgfpathlineto{\pgfqpoint{4.478739in}{3.001716in}}%
\pgfpathlineto{\pgfqpoint{4.471440in}{2.989415in}}%
\pgfpathclose%
\pgfusepath{fill}%
\end{pgfscope}%
\begin{pgfscope}%
\pgfpathrectangle{\pgfqpoint{1.254980in}{0.150000in}}{\pgfqpoint{5.490039in}{5.490039in}}%
\pgfusepath{clip}%
\pgfsetbuttcap%
\pgfsetroundjoin%
\definecolor{currentfill}{rgb}{0.119512,0.607464,0.540218}%
\pgfsetfillcolor{currentfill}%
\pgfsetfillopacity{0.700000}%
\pgfsetlinewidth{0.000000pt}%
\definecolor{currentstroke}{rgb}{0.000000,0.000000,0.000000}%
\pgfsetstrokecolor{currentstroke}%
\pgfsetdash{}{0pt}%
\pgfpathmoveto{\pgfqpoint{3.326691in}{3.788240in}}%
\pgfpathlineto{\pgfqpoint{3.339583in}{3.769050in}}%
\pgfpathlineto{\pgfqpoint{3.352470in}{3.750062in}}%
\pgfpathlineto{\pgfqpoint{3.365352in}{3.731275in}}%
\pgfpathlineto{\pgfqpoint{3.378229in}{3.712688in}}%
\pgfpathlineto{\pgfqpoint{3.385798in}{3.725979in}}%
\pgfpathlineto{\pgfqpoint{3.393361in}{3.739401in}}%
\pgfpathlineto{\pgfqpoint{3.400918in}{3.752957in}}%
\pgfpathlineto{\pgfqpoint{3.408471in}{3.766648in}}%
\pgfpathlineto{\pgfqpoint{3.395606in}{3.785353in}}%
\pgfpathlineto{\pgfqpoint{3.382736in}{3.804257in}}%
\pgfpathlineto{\pgfqpoint{3.369862in}{3.823363in}}%
\pgfpathlineto{\pgfqpoint{3.356983in}{3.842671in}}%
\pgfpathlineto{\pgfqpoint{3.349418in}{3.828856in}}%
\pgfpathlineto{\pgfqpoint{3.341848in}{3.815180in}}%
\pgfpathlineto{\pgfqpoint{3.334272in}{3.801642in}}%
\pgfpathlineto{\pgfqpoint{3.326691in}{3.788240in}}%
\pgfpathclose%
\pgfusepath{fill}%
\end{pgfscope}%
\begin{pgfscope}%
\pgfpathrectangle{\pgfqpoint{1.254980in}{0.150000in}}{\pgfqpoint{5.490039in}{5.490039in}}%
\pgfusepath{clip}%
\pgfsetbuttcap%
\pgfsetroundjoin%
\definecolor{currentfill}{rgb}{0.169646,0.456262,0.558030}%
\pgfsetfillcolor{currentfill}%
\pgfsetfillopacity{0.700000}%
\pgfsetlinewidth{0.000000pt}%
\definecolor{currentstroke}{rgb}{0.000000,0.000000,0.000000}%
\pgfsetstrokecolor{currentstroke}%
\pgfsetdash{}{0pt}%
\pgfpathmoveto{\pgfqpoint{3.553646in}{3.391144in}}%
\pgfpathlineto{\pgfqpoint{3.566482in}{3.375858in}}%
\pgfpathlineto{\pgfqpoint{3.579315in}{3.360750in}}%
\pgfpathlineto{\pgfqpoint{3.592147in}{3.345819in}}%
\pgfpathlineto{\pgfqpoint{3.604976in}{3.331064in}}%
\pgfpathlineto{\pgfqpoint{3.612503in}{3.343369in}}%
\pgfpathlineto{\pgfqpoint{3.620025in}{3.355781in}}%
\pgfpathlineto{\pgfqpoint{3.627542in}{3.368300in}}%
\pgfpathlineto{\pgfqpoint{3.635054in}{3.380929in}}%
\pgfpathlineto{\pgfqpoint{3.622237in}{3.395784in}}%
\pgfpathlineto{\pgfqpoint{3.609417in}{3.410814in}}%
\pgfpathlineto{\pgfqpoint{3.596596in}{3.426021in}}%
\pgfpathlineto{\pgfqpoint{3.583773in}{3.441407in}}%
\pgfpathlineto{\pgfqpoint{3.576249in}{3.428672in}}%
\pgfpathlineto{\pgfqpoint{3.568720in}{3.416051in}}%
\pgfpathlineto{\pgfqpoint{3.561186in}{3.403542in}}%
\pgfpathlineto{\pgfqpoint{3.553646in}{3.391144in}}%
\pgfpathclose%
\pgfusepath{fill}%
\end{pgfscope}%
\begin{pgfscope}%
\pgfpathrectangle{\pgfqpoint{1.254980in}{0.150000in}}{\pgfqpoint{5.490039in}{5.490039in}}%
\pgfusepath{clip}%
\pgfsetbuttcap%
\pgfsetroundjoin%
\definecolor{currentfill}{rgb}{0.179019,0.433756,0.557430}%
\pgfsetfillcolor{currentfill}%
\pgfsetfillopacity{0.700000}%
\pgfsetlinewidth{0.000000pt}%
\definecolor{currentstroke}{rgb}{0.000000,0.000000,0.000000}%
\pgfsetstrokecolor{currentstroke}%
\pgfsetdash{}{0pt}%
\pgfpathmoveto{\pgfqpoint{3.604976in}{3.331064in}}%
\pgfpathlineto{\pgfqpoint{3.617805in}{3.316483in}}%
\pgfpathlineto{\pgfqpoint{3.630632in}{3.302076in}}%
\pgfpathlineto{\pgfqpoint{3.643458in}{3.287841in}}%
\pgfpathlineto{\pgfqpoint{3.656282in}{3.273778in}}%
\pgfpathlineto{\pgfqpoint{3.663797in}{3.285990in}}%
\pgfpathlineto{\pgfqpoint{3.671307in}{3.298305in}}%
\pgfpathlineto{\pgfqpoint{3.678812in}{3.310723in}}%
\pgfpathlineto{\pgfqpoint{3.686313in}{3.323247in}}%
\pgfpathlineto{\pgfqpoint{3.673500in}{3.337409in}}%
\pgfpathlineto{\pgfqpoint{3.660686in}{3.351743in}}%
\pgfpathlineto{\pgfqpoint{3.647871in}{3.366249in}}%
\pgfpathlineto{\pgfqpoint{3.635054in}{3.380929in}}%
\pgfpathlineto{\pgfqpoint{3.627542in}{3.368300in}}%
\pgfpathlineto{\pgfqpoint{3.620025in}{3.355781in}}%
\pgfpathlineto{\pgfqpoint{3.612503in}{3.343369in}}%
\pgfpathlineto{\pgfqpoint{3.604976in}{3.331064in}}%
\pgfpathclose%
\pgfusepath{fill}%
\end{pgfscope}%
\begin{pgfscope}%
\pgfpathrectangle{\pgfqpoint{1.254980in}{0.150000in}}{\pgfqpoint{5.490039in}{5.490039in}}%
\pgfusepath{clip}%
\pgfsetbuttcap%
\pgfsetroundjoin%
\definecolor{currentfill}{rgb}{0.159194,0.482237,0.558073}%
\pgfsetfillcolor{currentfill}%
\pgfsetfillopacity{0.700000}%
\pgfsetlinewidth{0.000000pt}%
\definecolor{currentstroke}{rgb}{0.000000,0.000000,0.000000}%
\pgfsetstrokecolor{currentstroke}%
\pgfsetdash{}{0pt}%
\pgfpathmoveto{\pgfqpoint{3.502283in}{3.454086in}}%
\pgfpathlineto{\pgfqpoint{3.515128in}{3.438078in}}%
\pgfpathlineto{\pgfqpoint{3.527970in}{3.422253in}}%
\pgfpathlineto{\pgfqpoint{3.540809in}{3.406608in}}%
\pgfpathlineto{\pgfqpoint{3.553646in}{3.391144in}}%
\pgfpathlineto{\pgfqpoint{3.561186in}{3.403542in}}%
\pgfpathlineto{\pgfqpoint{3.568720in}{3.416051in}}%
\pgfpathlineto{\pgfqpoint{3.576249in}{3.428672in}}%
\pgfpathlineto{\pgfqpoint{3.583773in}{3.441407in}}%
\pgfpathlineto{\pgfqpoint{3.570948in}{3.456971in}}%
\pgfpathlineto{\pgfqpoint{3.558120in}{3.472715in}}%
\pgfpathlineto{\pgfqpoint{3.545291in}{3.488640in}}%
\pgfpathlineto{\pgfqpoint{3.532459in}{3.504748in}}%
\pgfpathlineto{\pgfqpoint{3.524923in}{3.491908in}}%
\pgfpathlineto{\pgfqpoint{3.517381in}{3.479185in}}%
\pgfpathlineto{\pgfqpoint{3.509835in}{3.466578in}}%
\pgfpathlineto{\pgfqpoint{3.502283in}{3.454086in}}%
\pgfpathclose%
\pgfusepath{fill}%
\end{pgfscope}%
\begin{pgfscope}%
\pgfpathrectangle{\pgfqpoint{1.254980in}{0.150000in}}{\pgfqpoint{5.490039in}{5.490039in}}%
\pgfusepath{clip}%
\pgfsetbuttcap%
\pgfsetroundjoin%
\definecolor{currentfill}{rgb}{0.188923,0.410910,0.556326}%
\pgfsetfillcolor{currentfill}%
\pgfsetfillopacity{0.700000}%
\pgfsetlinewidth{0.000000pt}%
\definecolor{currentstroke}{rgb}{0.000000,0.000000,0.000000}%
\pgfsetstrokecolor{currentstroke}%
\pgfsetdash{}{0pt}%
\pgfpathmoveto{\pgfqpoint{3.656282in}{3.273778in}}%
\pgfpathlineto{\pgfqpoint{3.669106in}{3.259886in}}%
\pgfpathlineto{\pgfqpoint{3.681929in}{3.246163in}}%
\pgfpathlineto{\pgfqpoint{3.694751in}{3.232609in}}%
\pgfpathlineto{\pgfqpoint{3.707573in}{3.219222in}}%
\pgfpathlineto{\pgfqpoint{3.715075in}{3.231341in}}%
\pgfpathlineto{\pgfqpoint{3.722573in}{3.243559in}}%
\pgfpathlineto{\pgfqpoint{3.730067in}{3.255876in}}%
\pgfpathlineto{\pgfqpoint{3.737556in}{3.268295in}}%
\pgfpathlineto{\pgfqpoint{3.724746in}{3.281780in}}%
\pgfpathlineto{\pgfqpoint{3.711936in}{3.295433in}}%
\pgfpathlineto{\pgfqpoint{3.699124in}{3.309255in}}%
\pgfpathlineto{\pgfqpoint{3.686313in}{3.323247in}}%
\pgfpathlineto{\pgfqpoint{3.678812in}{3.310723in}}%
\pgfpathlineto{\pgfqpoint{3.671307in}{3.298305in}}%
\pgfpathlineto{\pgfqpoint{3.663797in}{3.285990in}}%
\pgfpathlineto{\pgfqpoint{3.656282in}{3.273778in}}%
\pgfpathclose%
\pgfusepath{fill}%
\end{pgfscope}%
\begin{pgfscope}%
\pgfpathrectangle{\pgfqpoint{1.254980in}{0.150000in}}{\pgfqpoint{5.490039in}{5.490039in}}%
\pgfusepath{clip}%
\pgfsetbuttcap%
\pgfsetroundjoin%
\definecolor{currentfill}{rgb}{0.225863,0.330805,0.547314}%
\pgfsetfillcolor{currentfill}%
\pgfsetfillopacity{0.700000}%
\pgfsetlinewidth{0.000000pt}%
\definecolor{currentstroke}{rgb}{0.000000,0.000000,0.000000}%
\pgfsetstrokecolor{currentstroke}%
\pgfsetdash{}{0pt}%
\pgfpathmoveto{\pgfqpoint{3.942547in}{3.074607in}}%
\pgfpathlineto{\pgfqpoint{3.955367in}{3.063838in}}%
\pgfpathlineto{\pgfqpoint{3.968189in}{3.053222in}}%
\pgfpathlineto{\pgfqpoint{3.981013in}{3.042758in}}%
\pgfpathlineto{\pgfqpoint{3.993838in}{3.032445in}}%
\pgfpathlineto{\pgfqpoint{4.001267in}{3.044423in}}%
\pgfpathlineto{\pgfqpoint{4.008692in}{3.056485in}}%
\pgfpathlineto{\pgfqpoint{4.016113in}{3.068631in}}%
\pgfpathlineto{\pgfqpoint{4.023530in}{3.080865in}}%
\pgfpathlineto{\pgfqpoint{4.010715in}{3.091292in}}%
\pgfpathlineto{\pgfqpoint{3.997902in}{3.101870in}}%
\pgfpathlineto{\pgfqpoint{3.985091in}{3.112600in}}%
\pgfpathlineto{\pgfqpoint{3.972282in}{3.123483in}}%
\pgfpathlineto{\pgfqpoint{3.964854in}{3.111130in}}%
\pgfpathlineto{\pgfqpoint{3.957422in}{3.098867in}}%
\pgfpathlineto{\pgfqpoint{3.949987in}{3.086693in}}%
\pgfpathlineto{\pgfqpoint{3.942547in}{3.074607in}}%
\pgfpathclose%
\pgfusepath{fill}%
\end{pgfscope}%
\begin{pgfscope}%
\pgfpathrectangle{\pgfqpoint{1.254980in}{0.150000in}}{\pgfqpoint{5.490039in}{5.490039in}}%
\pgfusepath{clip}%
\pgfsetbuttcap%
\pgfsetroundjoin%
\definecolor{currentfill}{rgb}{0.241237,0.296485,0.539709}%
\pgfsetfillcolor{currentfill}%
\pgfsetfillopacity{0.700000}%
\pgfsetlinewidth{0.000000pt}%
\definecolor{currentstroke}{rgb}{0.000000,0.000000,0.000000}%
\pgfsetstrokecolor{currentstroke}%
\pgfsetdash{}{0pt}%
\pgfpathmoveto{\pgfqpoint{4.603791in}{2.987315in}}%
\pgfpathlineto{\pgfqpoint{4.616713in}{2.981428in}}%
\pgfpathlineto{\pgfqpoint{4.629640in}{2.975672in}}%
\pgfpathlineto{\pgfqpoint{4.642574in}{2.970044in}}%
\pgfpathlineto{\pgfqpoint{4.655514in}{2.964544in}}%
\pgfpathlineto{\pgfqpoint{4.662770in}{2.976624in}}%
\pgfpathlineto{\pgfqpoint{4.670022in}{2.988782in}}%
\pgfpathlineto{\pgfqpoint{4.677272in}{3.001020in}}%
\pgfpathlineto{\pgfqpoint{4.684520in}{3.013339in}}%
\pgfpathlineto{\pgfqpoint{4.671590in}{3.019031in}}%
\pgfpathlineto{\pgfqpoint{4.658667in}{3.024851in}}%
\pgfpathlineto{\pgfqpoint{4.645750in}{3.030799in}}%
\pgfpathlineto{\pgfqpoint{4.632839in}{3.036878in}}%
\pgfpathlineto{\pgfqpoint{4.625581in}{3.024360in}}%
\pgfpathlineto{\pgfqpoint{4.618321in}{3.011929in}}%
\pgfpathlineto{\pgfqpoint{4.611057in}{2.999581in}}%
\pgfpathlineto{\pgfqpoint{4.603791in}{2.987315in}}%
\pgfpathclose%
\pgfusepath{fill}%
\end{pgfscope}%
\begin{pgfscope}%
\pgfpathrectangle{\pgfqpoint{1.254980in}{0.150000in}}{\pgfqpoint{5.490039in}{5.490039in}}%
\pgfusepath{clip}%
\pgfsetbuttcap%
\pgfsetroundjoin%
\definecolor{currentfill}{rgb}{0.243113,0.292092,0.538516}%
\pgfsetfillcolor{currentfill}%
\pgfsetfillopacity{0.700000}%
\pgfsetlinewidth{0.000000pt}%
\definecolor{currentstroke}{rgb}{0.000000,0.000000,0.000000}%
\pgfsetstrokecolor{currentstroke}%
\pgfsetdash{}{0pt}%
\pgfpathmoveto{\pgfqpoint{4.258389in}{2.981761in}}%
\pgfpathlineto{\pgfqpoint{4.271244in}{2.973670in}}%
\pgfpathlineto{\pgfqpoint{4.284102in}{2.965717in}}%
\pgfpathlineto{\pgfqpoint{4.296965in}{2.957903in}}%
\pgfpathlineto{\pgfqpoint{4.309832in}{2.950228in}}%
\pgfpathlineto{\pgfqpoint{4.317178in}{2.962217in}}%
\pgfpathlineto{\pgfqpoint{4.324521in}{2.974282in}}%
\pgfpathlineto{\pgfqpoint{4.331860in}{2.986424in}}%
\pgfpathlineto{\pgfqpoint{4.339196in}{2.998646in}}%
\pgfpathlineto{\pgfqpoint{4.326339in}{3.006466in}}%
\pgfpathlineto{\pgfqpoint{4.313487in}{3.014425in}}%
\pgfpathlineto{\pgfqpoint{4.300638in}{3.022523in}}%
\pgfpathlineto{\pgfqpoint{4.287794in}{3.030760in}}%
\pgfpathlineto{\pgfqpoint{4.280448in}{3.018388in}}%
\pgfpathlineto{\pgfqpoint{4.273098in}{3.006098in}}%
\pgfpathlineto{\pgfqpoint{4.265745in}{2.993890in}}%
\pgfpathlineto{\pgfqpoint{4.258389in}{2.981761in}}%
\pgfpathclose%
\pgfusepath{fill}%
\end{pgfscope}%
\begin{pgfscope}%
\pgfpathrectangle{\pgfqpoint{1.254980in}{0.150000in}}{\pgfqpoint{5.490039in}{5.490039in}}%
\pgfusepath{clip}%
\pgfsetbuttcap%
\pgfsetroundjoin%
\definecolor{currentfill}{rgb}{0.149039,0.508051,0.557250}%
\pgfsetfillcolor{currentfill}%
\pgfsetfillopacity{0.700000}%
\pgfsetlinewidth{0.000000pt}%
\definecolor{currentstroke}{rgb}{0.000000,0.000000,0.000000}%
\pgfsetstrokecolor{currentstroke}%
\pgfsetdash{}{0pt}%
\pgfpathmoveto{\pgfqpoint{3.450878in}{3.519962in}}%
\pgfpathlineto{\pgfqpoint{3.463734in}{3.503214in}}%
\pgfpathlineto{\pgfqpoint{3.476586in}{3.486652in}}%
\pgfpathlineto{\pgfqpoint{3.489436in}{3.470277in}}%
\pgfpathlineto{\pgfqpoint{3.502283in}{3.454086in}}%
\pgfpathlineto{\pgfqpoint{3.509835in}{3.466578in}}%
\pgfpathlineto{\pgfqpoint{3.517381in}{3.479185in}}%
\pgfpathlineto{\pgfqpoint{3.524923in}{3.491908in}}%
\pgfpathlineto{\pgfqpoint{3.532459in}{3.504748in}}%
\pgfpathlineto{\pgfqpoint{3.519624in}{3.521039in}}%
\pgfpathlineto{\pgfqpoint{3.506787in}{3.537514in}}%
\pgfpathlineto{\pgfqpoint{3.493947in}{3.554175in}}%
\pgfpathlineto{\pgfqpoint{3.481103in}{3.571024in}}%
\pgfpathlineto{\pgfqpoint{3.473555in}{3.558078in}}%
\pgfpathlineto{\pgfqpoint{3.466001in}{3.545253in}}%
\pgfpathlineto{\pgfqpoint{3.458442in}{3.532548in}}%
\pgfpathlineto{\pgfqpoint{3.450878in}{3.519962in}}%
\pgfpathclose%
\pgfusepath{fill}%
\end{pgfscope}%
\begin{pgfscope}%
\pgfpathrectangle{\pgfqpoint{1.254980in}{0.150000in}}{\pgfqpoint{5.490039in}{5.490039in}}%
\pgfusepath{clip}%
\pgfsetbuttcap%
\pgfsetroundjoin%
\definecolor{currentfill}{rgb}{0.229739,0.322361,0.545706}%
\pgfsetfillcolor{currentfill}%
\pgfsetfillopacity{0.700000}%
\pgfsetlinewidth{0.000000pt}%
\definecolor{currentstroke}{rgb}{0.000000,0.000000,0.000000}%
\pgfsetstrokecolor{currentstroke}%
\pgfsetdash{}{0pt}%
\pgfpathmoveto{\pgfqpoint{4.897798in}{3.051499in}}%
\pgfpathlineto{\pgfqpoint{4.910791in}{3.047010in}}%
\pgfpathlineto{\pgfqpoint{4.923792in}{3.042643in}}%
\pgfpathlineto{\pgfqpoint{4.936800in}{3.038399in}}%
\pgfpathlineto{\pgfqpoint{4.949816in}{3.034278in}}%
\pgfpathlineto{\pgfqpoint{4.956997in}{3.046500in}}%
\pgfpathlineto{\pgfqpoint{4.964175in}{3.058810in}}%
\pgfpathlineto{\pgfqpoint{4.971351in}{3.071212in}}%
\pgfpathlineto{\pgfqpoint{4.958344in}{3.075512in}}%
\pgfpathlineto{\pgfqpoint{4.945345in}{3.079933in}}%
\pgfpathlineto{\pgfqpoint{4.932353in}{3.084477in}}%
\pgfpathlineto{\pgfqpoint{4.919368in}{3.089144in}}%
\pgfpathlineto{\pgfqpoint{4.912180in}{3.076502in}}%
\pgfpathlineto{\pgfqpoint{4.904990in}{3.063954in}}%
\pgfpathlineto{\pgfqpoint{4.897798in}{3.051499in}}%
\pgfpathclose%
\pgfusepath{fill}%
\end{pgfscope}%
\begin{pgfscope}%
\pgfpathrectangle{\pgfqpoint{1.254980in}{0.150000in}}{\pgfqpoint{5.490039in}{5.490039in}}%
\pgfusepath{clip}%
\pgfsetbuttcap%
\pgfsetroundjoin%
\definecolor{currentfill}{rgb}{0.197636,0.391528,0.554969}%
\pgfsetfillcolor{currentfill}%
\pgfsetfillopacity{0.700000}%
\pgfsetlinewidth{0.000000pt}%
\definecolor{currentstroke}{rgb}{0.000000,0.000000,0.000000}%
\pgfsetstrokecolor{currentstroke}%
\pgfsetdash{}{0pt}%
\pgfpathmoveto{\pgfqpoint{3.707573in}{3.219222in}}%
\pgfpathlineto{\pgfqpoint{3.720394in}{3.206003in}}%
\pgfpathlineto{\pgfqpoint{3.733215in}{3.192949in}}%
\pgfpathlineto{\pgfqpoint{3.746035in}{3.180060in}}%
\pgfpathlineto{\pgfqpoint{3.758856in}{3.167335in}}%
\pgfpathlineto{\pgfqpoint{3.766347in}{3.179361in}}%
\pgfpathlineto{\pgfqpoint{3.773833in}{3.191481in}}%
\pgfpathlineto{\pgfqpoint{3.781315in}{3.203698in}}%
\pgfpathlineto{\pgfqpoint{3.788793in}{3.216011in}}%
\pgfpathlineto{\pgfqpoint{3.775984in}{3.228835in}}%
\pgfpathlineto{\pgfqpoint{3.763175in}{3.241823in}}%
\pgfpathlineto{\pgfqpoint{3.750365in}{3.254976in}}%
\pgfpathlineto{\pgfqpoint{3.737556in}{3.268295in}}%
\pgfpathlineto{\pgfqpoint{3.730067in}{3.255876in}}%
\pgfpathlineto{\pgfqpoint{3.722573in}{3.243559in}}%
\pgfpathlineto{\pgfqpoint{3.715075in}{3.231341in}}%
\pgfpathlineto{\pgfqpoint{3.707573in}{3.219222in}}%
\pgfpathclose%
\pgfusepath{fill}%
\end{pgfscope}%
\begin{pgfscope}%
\pgfpathrectangle{\pgfqpoint{1.254980in}{0.150000in}}{\pgfqpoint{5.490039in}{5.490039in}}%
\pgfusepath{clip}%
\pgfsetbuttcap%
\pgfsetroundjoin%
\definecolor{currentfill}{rgb}{0.239346,0.300855,0.540844}%
\pgfsetfillcolor{currentfill}%
\pgfsetfillopacity{0.700000}%
\pgfsetlinewidth{0.000000pt}%
\definecolor{currentstroke}{rgb}{0.000000,0.000000,0.000000}%
\pgfsetstrokecolor{currentstroke}%
\pgfsetdash{}{0pt}%
\pgfpathmoveto{\pgfqpoint{4.126133in}{3.002800in}}%
\pgfpathlineto{\pgfqpoint{4.138971in}{2.993701in}}%
\pgfpathlineto{\pgfqpoint{4.151812in}{2.984747in}}%
\pgfpathlineto{\pgfqpoint{4.164656in}{2.975935in}}%
\pgfpathlineto{\pgfqpoint{4.177503in}{2.967267in}}%
\pgfpathlineto{\pgfqpoint{4.184884in}{2.979216in}}%
\pgfpathlineto{\pgfqpoint{4.192262in}{2.991243in}}%
\pgfpathlineto{\pgfqpoint{4.199636in}{3.003349in}}%
\pgfpathlineto{\pgfqpoint{4.207006in}{3.015535in}}%
\pgfpathlineto{\pgfqpoint{4.194170in}{3.024333in}}%
\pgfpathlineto{\pgfqpoint{4.181336in}{3.033274in}}%
\pgfpathlineto{\pgfqpoint{4.168506in}{3.042358in}}%
\pgfpathlineto{\pgfqpoint{4.155679in}{3.051587in}}%
\pgfpathlineto{\pgfqpoint{4.148298in}{3.039265in}}%
\pgfpathlineto{\pgfqpoint{4.140914in}{3.027028in}}%
\pgfpathlineto{\pgfqpoint{4.133525in}{3.014873in}}%
\pgfpathlineto{\pgfqpoint{4.126133in}{3.002800in}}%
\pgfpathclose%
\pgfusepath{fill}%
\end{pgfscope}%
\begin{pgfscope}%
\pgfpathrectangle{\pgfqpoint{1.254980in}{0.150000in}}{\pgfqpoint{5.490039in}{5.490039in}}%
\pgfusepath{clip}%
\pgfsetbuttcap%
\pgfsetroundjoin%
\definecolor{currentfill}{rgb}{0.124780,0.640461,0.527068}%
\pgfsetfillcolor{currentfill}%
\pgfsetfillopacity{0.700000}%
\pgfsetlinewidth{0.000000pt}%
\definecolor{currentstroke}{rgb}{0.000000,0.000000,0.000000}%
\pgfsetstrokecolor{currentstroke}%
\pgfsetdash{}{0pt}%
\pgfpathmoveto{\pgfqpoint{3.275072in}{3.867055in}}%
\pgfpathlineto{\pgfqpoint{3.287985in}{3.847040in}}%
\pgfpathlineto{\pgfqpoint{3.300892in}{3.827234in}}%
\pgfpathlineto{\pgfqpoint{3.313794in}{3.807634in}}%
\pgfpathlineto{\pgfqpoint{3.326691in}{3.788240in}}%
\pgfpathlineto{\pgfqpoint{3.334272in}{3.801642in}}%
\pgfpathlineto{\pgfqpoint{3.341848in}{3.815180in}}%
\pgfpathlineto{\pgfqpoint{3.349418in}{3.828856in}}%
\pgfpathlineto{\pgfqpoint{3.356983in}{3.842671in}}%
\pgfpathlineto{\pgfqpoint{3.344099in}{3.862183in}}%
\pgfpathlineto{\pgfqpoint{3.331209in}{3.881901in}}%
\pgfpathlineto{\pgfqpoint{3.318315in}{3.901826in}}%
\pgfpathlineto{\pgfqpoint{3.305414in}{3.921959in}}%
\pgfpathlineto{\pgfqpoint{3.297837in}{3.908020in}}%
\pgfpathlineto{\pgfqpoint{3.290254in}{3.894223in}}%
\pgfpathlineto{\pgfqpoint{3.282666in}{3.880569in}}%
\pgfpathlineto{\pgfqpoint{3.275072in}{3.867055in}}%
\pgfpathclose%
\pgfusepath{fill}%
\end{pgfscope}%
\begin{pgfscope}%
\pgfpathrectangle{\pgfqpoint{1.254980in}{0.150000in}}{\pgfqpoint{5.490039in}{5.490039in}}%
\pgfusepath{clip}%
\pgfsetbuttcap%
\pgfsetroundjoin%
\definecolor{currentfill}{rgb}{0.244972,0.287675,0.537260}%
\pgfsetfillcolor{currentfill}%
\pgfsetfillopacity{0.700000}%
\pgfsetlinewidth{0.000000pt}%
\definecolor{currentstroke}{rgb}{0.000000,0.000000,0.000000}%
\pgfsetstrokecolor{currentstroke}%
\pgfsetdash{}{0pt}%
\pgfpathmoveto{\pgfqpoint{4.390667in}{2.968735in}}%
\pgfpathlineto{\pgfqpoint{4.403546in}{2.961598in}}%
\pgfpathlineto{\pgfqpoint{4.416429in}{2.954595in}}%
\pgfpathlineto{\pgfqpoint{4.429318in}{2.947728in}}%
\pgfpathlineto{\pgfqpoint{4.442212in}{2.940994in}}%
\pgfpathlineto{\pgfqpoint{4.449524in}{2.952986in}}%
\pgfpathlineto{\pgfqpoint{4.456832in}{2.965052in}}%
\pgfpathlineto{\pgfqpoint{4.464138in}{2.977195in}}%
\pgfpathlineto{\pgfqpoint{4.471440in}{2.989415in}}%
\pgfpathlineto{\pgfqpoint{4.458556in}{2.996310in}}%
\pgfpathlineto{\pgfqpoint{4.445678in}{3.003338in}}%
\pgfpathlineto{\pgfqpoint{4.432805in}{3.010502in}}%
\pgfpathlineto{\pgfqpoint{4.419936in}{3.017800in}}%
\pgfpathlineto{\pgfqpoint{4.412623in}{3.005412in}}%
\pgfpathlineto{\pgfqpoint{4.405308in}{2.993107in}}%
\pgfpathlineto{\pgfqpoint{4.397989in}{2.980882in}}%
\pgfpathlineto{\pgfqpoint{4.390667in}{2.968735in}}%
\pgfpathclose%
\pgfusepath{fill}%
\end{pgfscope}%
\begin{pgfscope}%
\pgfpathrectangle{\pgfqpoint{1.254980in}{0.150000in}}{\pgfqpoint{5.490039in}{5.490039in}}%
\pgfusepath{clip}%
\pgfsetbuttcap%
\pgfsetroundjoin%
\definecolor{currentfill}{rgb}{0.233603,0.313828,0.543914}%
\pgfsetfillcolor{currentfill}%
\pgfsetfillopacity{0.700000}%
\pgfsetlinewidth{0.000000pt}%
\definecolor{currentstroke}{rgb}{0.000000,0.000000,0.000000}%
\pgfsetstrokecolor{currentstroke}%
\pgfsetdash{}{0pt}%
\pgfpathmoveto{\pgfqpoint{4.817057in}{3.020859in}}%
\pgfpathlineto{\pgfqpoint{4.830034in}{3.016097in}}%
\pgfpathlineto{\pgfqpoint{4.843018in}{3.011460in}}%
\pgfpathlineto{\pgfqpoint{4.856009in}{3.006947in}}%
\pgfpathlineto{\pgfqpoint{4.869007in}{3.002557in}}%
\pgfpathlineto{\pgfqpoint{4.876208in}{3.014666in}}%
\pgfpathlineto{\pgfqpoint{4.883407in}{3.026857in}}%
\pgfpathlineto{\pgfqpoint{4.890603in}{3.039134in}}%
\pgfpathlineto{\pgfqpoint{4.897798in}{3.051499in}}%
\pgfpathlineto{\pgfqpoint{4.884811in}{3.056112in}}%
\pgfpathlineto{\pgfqpoint{4.871832in}{3.060848in}}%
\pgfpathlineto{\pgfqpoint{4.858859in}{3.065708in}}%
\pgfpathlineto{\pgfqpoint{4.845894in}{3.070693in}}%
\pgfpathlineto{\pgfqpoint{4.838688in}{3.058099in}}%
\pgfpathlineto{\pgfqpoint{4.831480in}{3.045597in}}%
\pgfpathlineto{\pgfqpoint{4.824270in}{3.033184in}}%
\pgfpathlineto{\pgfqpoint{4.817057in}{3.020859in}}%
\pgfpathclose%
\pgfusepath{fill}%
\end{pgfscope}%
\begin{pgfscope}%
\pgfpathrectangle{\pgfqpoint{1.254980in}{0.150000in}}{\pgfqpoint{5.490039in}{5.490039in}}%
\pgfusepath{clip}%
\pgfsetbuttcap%
\pgfsetroundjoin%
\definecolor{currentfill}{rgb}{0.139147,0.533812,0.555298}%
\pgfsetfillcolor{currentfill}%
\pgfsetfillopacity{0.700000}%
\pgfsetlinewidth{0.000000pt}%
\definecolor{currentstroke}{rgb}{0.000000,0.000000,0.000000}%
\pgfsetstrokecolor{currentstroke}%
\pgfsetdash{}{0pt}%
\pgfpathmoveto{\pgfqpoint{3.399420in}{3.588848in}}%
\pgfpathlineto{\pgfqpoint{3.412290in}{3.571340in}}%
\pgfpathlineto{\pgfqpoint{3.425156in}{3.554024in}}%
\pgfpathlineto{\pgfqpoint{3.438019in}{3.536898in}}%
\pgfpathlineto{\pgfqpoint{3.450878in}{3.519962in}}%
\pgfpathlineto{\pgfqpoint{3.458442in}{3.532548in}}%
\pgfpathlineto{\pgfqpoint{3.466001in}{3.545253in}}%
\pgfpathlineto{\pgfqpoint{3.473555in}{3.558078in}}%
\pgfpathlineto{\pgfqpoint{3.481103in}{3.571024in}}%
\pgfpathlineto{\pgfqpoint{3.468257in}{3.588060in}}%
\pgfpathlineto{\pgfqpoint{3.455407in}{3.605286in}}%
\pgfpathlineto{\pgfqpoint{3.442554in}{3.622702in}}%
\pgfpathlineto{\pgfqpoint{3.429697in}{3.640311in}}%
\pgfpathlineto{\pgfqpoint{3.422136in}{3.627258in}}%
\pgfpathlineto{\pgfqpoint{3.414569in}{3.614331in}}%
\pgfpathlineto{\pgfqpoint{3.406998in}{3.601528in}}%
\pgfpathlineto{\pgfqpoint{3.399420in}{3.588848in}}%
\pgfpathclose%
\pgfusepath{fill}%
\end{pgfscope}%
\begin{pgfscope}%
\pgfpathrectangle{\pgfqpoint{1.254980in}{0.150000in}}{\pgfqpoint{5.490039in}{5.490039in}}%
\pgfusepath{clip}%
\pgfsetbuttcap%
\pgfsetroundjoin%
\definecolor{currentfill}{rgb}{0.206756,0.371758,0.553117}%
\pgfsetfillcolor{currentfill}%
\pgfsetfillopacity{0.700000}%
\pgfsetlinewidth{0.000000pt}%
\definecolor{currentstroke}{rgb}{0.000000,0.000000,0.000000}%
\pgfsetstrokecolor{currentstroke}%
\pgfsetdash{}{0pt}%
\pgfpathmoveto{\pgfqpoint{3.758856in}{3.167335in}}%
\pgfpathlineto{\pgfqpoint{3.771677in}{3.154773in}}%
\pgfpathlineto{\pgfqpoint{3.784497in}{3.142373in}}%
\pgfpathlineto{\pgfqpoint{3.797319in}{3.130135in}}%
\pgfpathlineto{\pgfqpoint{3.810140in}{3.118057in}}%
\pgfpathlineto{\pgfqpoint{3.817620in}{3.129990in}}%
\pgfpathlineto{\pgfqpoint{3.825095in}{3.142014in}}%
\pgfpathlineto{\pgfqpoint{3.832566in}{3.154130in}}%
\pgfpathlineto{\pgfqpoint{3.840032in}{3.166339in}}%
\pgfpathlineto{\pgfqpoint{3.827221in}{3.178515in}}%
\pgfpathlineto{\pgfqpoint{3.814412in}{3.190852in}}%
\pgfpathlineto{\pgfqpoint{3.801602in}{3.203351in}}%
\pgfpathlineto{\pgfqpoint{3.788793in}{3.216011in}}%
\pgfpathlineto{\pgfqpoint{3.781315in}{3.203698in}}%
\pgfpathlineto{\pgfqpoint{3.773833in}{3.191481in}}%
\pgfpathlineto{\pgfqpoint{3.766347in}{3.179361in}}%
\pgfpathlineto{\pgfqpoint{3.758856in}{3.167335in}}%
\pgfpathclose%
\pgfusepath{fill}%
\end{pgfscope}%
\begin{pgfscope}%
\pgfpathrectangle{\pgfqpoint{1.254980in}{0.150000in}}{\pgfqpoint{5.490039in}{5.490039in}}%
\pgfusepath{clip}%
\pgfsetbuttcap%
\pgfsetroundjoin%
\definecolor{currentfill}{rgb}{0.233603,0.313828,0.543914}%
\pgfsetfillcolor{currentfill}%
\pgfsetfillopacity{0.700000}%
\pgfsetlinewidth{0.000000pt}%
\definecolor{currentstroke}{rgb}{0.000000,0.000000,0.000000}%
\pgfsetstrokecolor{currentstroke}%
\pgfsetdash{}{0pt}%
\pgfpathmoveto{\pgfqpoint{3.993838in}{3.032445in}}%
\pgfpathlineto{\pgfqpoint{4.006666in}{3.022283in}}%
\pgfpathlineto{\pgfqpoint{4.019496in}{3.012269in}}%
\pgfpathlineto{\pgfqpoint{4.032328in}{3.002405in}}%
\pgfpathlineto{\pgfqpoint{4.045163in}{2.992689in}}%
\pgfpathlineto{\pgfqpoint{4.052581in}{3.004559in}}%
\pgfpathlineto{\pgfqpoint{4.059995in}{3.016508in}}%
\pgfpathlineto{\pgfqpoint{4.067405in}{3.028539in}}%
\pgfpathlineto{\pgfqpoint{4.074812in}{3.040652in}}%
\pgfpathlineto{\pgfqpoint{4.061988in}{3.050482in}}%
\pgfpathlineto{\pgfqpoint{4.049166in}{3.060460in}}%
\pgfpathlineto{\pgfqpoint{4.036347in}{3.070588in}}%
\pgfpathlineto{\pgfqpoint{4.023530in}{3.080865in}}%
\pgfpathlineto{\pgfqpoint{4.016113in}{3.068631in}}%
\pgfpathlineto{\pgfqpoint{4.008692in}{3.056485in}}%
\pgfpathlineto{\pgfqpoint{4.001267in}{3.044423in}}%
\pgfpathlineto{\pgfqpoint{3.993838in}{3.032445in}}%
\pgfpathclose%
\pgfusepath{fill}%
\end{pgfscope}%
\begin{pgfscope}%
\pgfpathrectangle{\pgfqpoint{1.254980in}{0.150000in}}{\pgfqpoint{5.490039in}{5.490039in}}%
\pgfusepath{clip}%
\pgfsetbuttcap%
\pgfsetroundjoin%
\definecolor{currentfill}{rgb}{0.244972,0.287675,0.537260}%
\pgfsetfillcolor{currentfill}%
\pgfsetfillopacity{0.700000}%
\pgfsetlinewidth{0.000000pt}%
\definecolor{currentstroke}{rgb}{0.000000,0.000000,0.000000}%
\pgfsetstrokecolor{currentstroke}%
\pgfsetdash{}{0pt}%
\pgfpathmoveto{\pgfqpoint{4.523024in}{2.963168in}}%
\pgfpathlineto{\pgfqpoint{4.535934in}{2.956936in}}%
\pgfpathlineto{\pgfqpoint{4.548849in}{2.950836in}}%
\pgfpathlineto{\pgfqpoint{4.561770in}{2.944866in}}%
\pgfpathlineto{\pgfqpoint{4.574696in}{2.939027in}}%
\pgfpathlineto{\pgfqpoint{4.581974in}{2.950986in}}%
\pgfpathlineto{\pgfqpoint{4.589250in}{2.963019in}}%
\pgfpathlineto{\pgfqpoint{4.596522in}{2.975128in}}%
\pgfpathlineto{\pgfqpoint{4.603791in}{2.987315in}}%
\pgfpathlineto{\pgfqpoint{4.590875in}{2.993331in}}%
\pgfpathlineto{\pgfqpoint{4.577965in}{2.999477in}}%
\pgfpathlineto{\pgfqpoint{4.565060in}{3.005754in}}%
\pgfpathlineto{\pgfqpoint{4.552161in}{3.012162in}}%
\pgfpathlineto{\pgfqpoint{4.544881in}{2.999793in}}%
\pgfpathlineto{\pgfqpoint{4.537599in}{2.987506in}}%
\pgfpathlineto{\pgfqpoint{4.530313in}{2.975298in}}%
\pgfpathlineto{\pgfqpoint{4.523024in}{2.963168in}}%
\pgfpathclose%
\pgfusepath{fill}%
\end{pgfscope}%
\begin{pgfscope}%
\pgfpathrectangle{\pgfqpoint{1.254980in}{0.150000in}}{\pgfqpoint{5.490039in}{5.490039in}}%
\pgfusepath{clip}%
\pgfsetbuttcap%
\pgfsetroundjoin%
\definecolor{currentfill}{rgb}{0.128729,0.563265,0.551229}%
\pgfsetfillcolor{currentfill}%
\pgfsetfillopacity{0.700000}%
\pgfsetlinewidth{0.000000pt}%
\definecolor{currentstroke}{rgb}{0.000000,0.000000,0.000000}%
\pgfsetstrokecolor{currentstroke}%
\pgfsetdash{}{0pt}%
\pgfpathmoveto{\pgfqpoint{3.347901in}{3.660823in}}%
\pgfpathlineto{\pgfqpoint{3.360787in}{3.642535in}}%
\pgfpathlineto{\pgfqpoint{3.373669in}{3.624444in}}%
\pgfpathlineto{\pgfqpoint{3.386547in}{3.606549in}}%
\pgfpathlineto{\pgfqpoint{3.399420in}{3.588848in}}%
\pgfpathlineto{\pgfqpoint{3.406998in}{3.601528in}}%
\pgfpathlineto{\pgfqpoint{3.414569in}{3.614331in}}%
\pgfpathlineto{\pgfqpoint{3.422136in}{3.627258in}}%
\pgfpathlineto{\pgfqpoint{3.429697in}{3.640311in}}%
\pgfpathlineto{\pgfqpoint{3.416836in}{3.658112in}}%
\pgfpathlineto{\pgfqpoint{3.403971in}{3.676108in}}%
\pgfpathlineto{\pgfqpoint{3.391103in}{3.694300in}}%
\pgfpathlineto{\pgfqpoint{3.378229in}{3.712688in}}%
\pgfpathlineto{\pgfqpoint{3.370656in}{3.699529in}}%
\pgfpathlineto{\pgfqpoint{3.363076in}{3.686499in}}%
\pgfpathlineto{\pgfqpoint{3.355492in}{3.673597in}}%
\pgfpathlineto{\pgfqpoint{3.347901in}{3.660823in}}%
\pgfpathclose%
\pgfusepath{fill}%
\end{pgfscope}%
\begin{pgfscope}%
\pgfpathrectangle{\pgfqpoint{1.254980in}{0.150000in}}{\pgfqpoint{5.490039in}{5.490039in}}%
\pgfusepath{clip}%
\pgfsetbuttcap%
\pgfsetroundjoin%
\definecolor{currentfill}{rgb}{0.239346,0.300855,0.540844}%
\pgfsetfillcolor{currentfill}%
\pgfsetfillopacity{0.700000}%
\pgfsetlinewidth{0.000000pt}%
\definecolor{currentstroke}{rgb}{0.000000,0.000000,0.000000}%
\pgfsetstrokecolor{currentstroke}%
\pgfsetdash{}{0pt}%
\pgfpathmoveto{\pgfqpoint{4.736298in}{2.991849in}}%
\pgfpathlineto{\pgfqpoint{4.749259in}{2.986794in}}%
\pgfpathlineto{\pgfqpoint{4.762227in}{2.981864in}}%
\pgfpathlineto{\pgfqpoint{4.775201in}{2.977060in}}%
\pgfpathlineto{\pgfqpoint{4.788182in}{2.972382in}}%
\pgfpathlineto{\pgfqpoint{4.795405in}{2.984381in}}%
\pgfpathlineto{\pgfqpoint{4.802625in}{2.996459in}}%
\pgfpathlineto{\pgfqpoint{4.809842in}{3.008618in}}%
\pgfpathlineto{\pgfqpoint{4.817057in}{3.020859in}}%
\pgfpathlineto{\pgfqpoint{4.804088in}{3.025745in}}%
\pgfpathlineto{\pgfqpoint{4.791124in}{3.030756in}}%
\pgfpathlineto{\pgfqpoint{4.778168in}{3.035894in}}%
\pgfpathlineto{\pgfqpoint{4.765218in}{3.041157in}}%
\pgfpathlineto{\pgfqpoint{4.757992in}{3.028702in}}%
\pgfpathlineto{\pgfqpoint{4.750763in}{3.016334in}}%
\pgfpathlineto{\pgfqpoint{4.743532in}{3.004051in}}%
\pgfpathlineto{\pgfqpoint{4.736298in}{2.991849in}}%
\pgfpathclose%
\pgfusepath{fill}%
\end{pgfscope}%
\begin{pgfscope}%
\pgfpathrectangle{\pgfqpoint{1.254980in}{0.150000in}}{\pgfqpoint{5.490039in}{5.490039in}}%
\pgfusepath{clip}%
\pgfsetbuttcap%
\pgfsetroundjoin%
\definecolor{currentfill}{rgb}{0.216210,0.351535,0.550627}%
\pgfsetfillcolor{currentfill}%
\pgfsetfillopacity{0.700000}%
\pgfsetlinewidth{0.000000pt}%
\definecolor{currentstroke}{rgb}{0.000000,0.000000,0.000000}%
\pgfsetstrokecolor{currentstroke}%
\pgfsetdash{}{0pt}%
\pgfpathmoveto{\pgfqpoint{3.810140in}{3.118057in}}%
\pgfpathlineto{\pgfqpoint{3.822963in}{3.106139in}}%
\pgfpathlineto{\pgfqpoint{3.835786in}{3.094380in}}%
\pgfpathlineto{\pgfqpoint{3.848609in}{3.082779in}}%
\pgfpathlineto{\pgfqpoint{3.861434in}{3.071335in}}%
\pgfpathlineto{\pgfqpoint{3.868902in}{3.083175in}}%
\pgfpathlineto{\pgfqpoint{3.876366in}{3.095102in}}%
\pgfpathlineto{\pgfqpoint{3.883825in}{3.107117in}}%
\pgfpathlineto{\pgfqpoint{3.891280in}{3.119222in}}%
\pgfpathlineto{\pgfqpoint{3.878467in}{3.130764in}}%
\pgfpathlineto{\pgfqpoint{3.865654in}{3.142464in}}%
\pgfpathlineto{\pgfqpoint{3.852843in}{3.154322in}}%
\pgfpathlineto{\pgfqpoint{3.840032in}{3.166339in}}%
\pgfpathlineto{\pgfqpoint{3.832566in}{3.154130in}}%
\pgfpathlineto{\pgfqpoint{3.825095in}{3.142014in}}%
\pgfpathlineto{\pgfqpoint{3.817620in}{3.129990in}}%
\pgfpathlineto{\pgfqpoint{3.810140in}{3.118057in}}%
\pgfpathclose%
\pgfusepath{fill}%
\end{pgfscope}%
\begin{pgfscope}%
\pgfpathrectangle{\pgfqpoint{1.254980in}{0.150000in}}{\pgfqpoint{5.490039in}{5.490039in}}%
\pgfusepath{clip}%
\pgfsetbuttcap%
\pgfsetroundjoin%
\definecolor{currentfill}{rgb}{0.244972,0.287675,0.537260}%
\pgfsetfillcolor{currentfill}%
\pgfsetfillopacity{0.700000}%
\pgfsetlinewidth{0.000000pt}%
\definecolor{currentstroke}{rgb}{0.000000,0.000000,0.000000}%
\pgfsetstrokecolor{currentstroke}%
\pgfsetdash{}{0pt}%
\pgfpathmoveto{\pgfqpoint{4.177503in}{2.967267in}}%
\pgfpathlineto{\pgfqpoint{4.190353in}{2.958742in}}%
\pgfpathlineto{\pgfqpoint{4.203207in}{2.950358in}}%
\pgfpathlineto{\pgfqpoint{4.216065in}{2.942115in}}%
\pgfpathlineto{\pgfqpoint{4.228926in}{2.934013in}}%
\pgfpathlineto{\pgfqpoint{4.236298in}{2.945838in}}%
\pgfpathlineto{\pgfqpoint{4.243665in}{2.957737in}}%
\pgfpathlineto{\pgfqpoint{4.251029in}{2.969711in}}%
\pgfpathlineto{\pgfqpoint{4.258389in}{2.981761in}}%
\pgfpathlineto{\pgfqpoint{4.245538in}{2.989993in}}%
\pgfpathlineto{\pgfqpoint{4.232690in}{2.998366in}}%
\pgfpathlineto{\pgfqpoint{4.219847in}{3.006880in}}%
\pgfpathlineto{\pgfqpoint{4.207006in}{3.015535in}}%
\pgfpathlineto{\pgfqpoint{4.199636in}{3.003349in}}%
\pgfpathlineto{\pgfqpoint{4.192262in}{2.991243in}}%
\pgfpathlineto{\pgfqpoint{4.184884in}{2.979216in}}%
\pgfpathlineto{\pgfqpoint{4.177503in}{2.967267in}}%
\pgfpathclose%
\pgfusepath{fill}%
\end{pgfscope}%
\begin{pgfscope}%
\pgfpathrectangle{\pgfqpoint{1.254980in}{0.150000in}}{\pgfqpoint{5.490039in}{5.490039in}}%
\pgfusepath{clip}%
\pgfsetbuttcap%
\pgfsetroundjoin%
\definecolor{currentfill}{rgb}{0.248629,0.278775,0.534556}%
\pgfsetfillcolor{currentfill}%
\pgfsetfillopacity{0.700000}%
\pgfsetlinewidth{0.000000pt}%
\definecolor{currentstroke}{rgb}{0.000000,0.000000,0.000000}%
\pgfsetstrokecolor{currentstroke}%
\pgfsetdash{}{0pt}%
\pgfpathmoveto{\pgfqpoint{4.309832in}{2.950228in}}%
\pgfpathlineto{\pgfqpoint{4.322703in}{2.942690in}}%
\pgfpathlineto{\pgfqpoint{4.335579in}{2.935290in}}%
\pgfpathlineto{\pgfqpoint{4.348459in}{2.928026in}}%
\pgfpathlineto{\pgfqpoint{4.361344in}{2.920899in}}%
\pgfpathlineto{\pgfqpoint{4.368680in}{2.932748in}}%
\pgfpathlineto{\pgfqpoint{4.376012in}{2.944670in}}%
\pgfpathlineto{\pgfqpoint{4.383341in}{2.956665in}}%
\pgfpathlineto{\pgfqpoint{4.390667in}{2.968735in}}%
\pgfpathlineto{\pgfqpoint{4.377792in}{2.976008in}}%
\pgfpathlineto{\pgfqpoint{4.364922in}{2.983417in}}%
\pgfpathlineto{\pgfqpoint{4.352057in}{2.990963in}}%
\pgfpathlineto{\pgfqpoint{4.339196in}{2.998646in}}%
\pgfpathlineto{\pgfqpoint{4.331860in}{2.986424in}}%
\pgfpathlineto{\pgfqpoint{4.324521in}{2.974282in}}%
\pgfpathlineto{\pgfqpoint{4.317178in}{2.962217in}}%
\pgfpathlineto{\pgfqpoint{4.309832in}{2.950228in}}%
\pgfpathclose%
\pgfusepath{fill}%
\end{pgfscope}%
\begin{pgfscope}%
\pgfpathrectangle{\pgfqpoint{1.254980in}{0.150000in}}{\pgfqpoint{5.490039in}{5.490039in}}%
\pgfusepath{clip}%
\pgfsetbuttcap%
\pgfsetroundjoin%
\definecolor{currentfill}{rgb}{0.121148,0.592739,0.544641}%
\pgfsetfillcolor{currentfill}%
\pgfsetfillopacity{0.700000}%
\pgfsetlinewidth{0.000000pt}%
\definecolor{currentstroke}{rgb}{0.000000,0.000000,0.000000}%
\pgfsetstrokecolor{currentstroke}%
\pgfsetdash{}{0pt}%
\pgfpathmoveto{\pgfqpoint{3.296310in}{3.735970in}}%
\pgfpathlineto{\pgfqpoint{3.309215in}{3.716881in}}%
\pgfpathlineto{\pgfqpoint{3.322115in}{3.697994in}}%
\pgfpathlineto{\pgfqpoint{3.335011in}{3.679309in}}%
\pgfpathlineto{\pgfqpoint{3.347901in}{3.660823in}}%
\pgfpathlineto{\pgfqpoint{3.355492in}{3.673597in}}%
\pgfpathlineto{\pgfqpoint{3.363076in}{3.686499in}}%
\pgfpathlineto{\pgfqpoint{3.370656in}{3.699529in}}%
\pgfpathlineto{\pgfqpoint{3.378229in}{3.712688in}}%
\pgfpathlineto{\pgfqpoint{3.365352in}{3.731275in}}%
\pgfpathlineto{\pgfqpoint{3.352470in}{3.750062in}}%
\pgfpathlineto{\pgfqpoint{3.339583in}{3.769050in}}%
\pgfpathlineto{\pgfqpoint{3.326691in}{3.788240in}}%
\pgfpathlineto{\pgfqpoint{3.319105in}{3.774973in}}%
\pgfpathlineto{\pgfqpoint{3.311512in}{3.761840in}}%
\pgfpathlineto{\pgfqpoint{3.303914in}{3.748840in}}%
\pgfpathlineto{\pgfqpoint{3.296310in}{3.735970in}}%
\pgfpathclose%
\pgfusepath{fill}%
\end{pgfscope}%
\begin{pgfscope}%
\pgfpathrectangle{\pgfqpoint{1.254980in}{0.150000in}}{\pgfqpoint{5.490039in}{5.490039in}}%
\pgfusepath{clip}%
\pgfsetbuttcap%
\pgfsetroundjoin%
\definecolor{currentfill}{rgb}{0.239346,0.300855,0.540844}%
\pgfsetfillcolor{currentfill}%
\pgfsetfillopacity{0.700000}%
\pgfsetlinewidth{0.000000pt}%
\definecolor{currentstroke}{rgb}{0.000000,0.000000,0.000000}%
\pgfsetstrokecolor{currentstroke}%
\pgfsetdash{}{0pt}%
\pgfpathmoveto{\pgfqpoint{4.045163in}{2.992689in}}%
\pgfpathlineto{\pgfqpoint{4.058000in}{2.983121in}}%
\pgfpathlineto{\pgfqpoint{4.070839in}{2.973699in}}%
\pgfpathlineto{\pgfqpoint{4.083682in}{2.964424in}}%
\pgfpathlineto{\pgfqpoint{4.096527in}{2.955294in}}%
\pgfpathlineto{\pgfqpoint{4.103934in}{2.967056in}}%
\pgfpathlineto{\pgfqpoint{4.111338in}{2.978893in}}%
\pgfpathlineto{\pgfqpoint{4.118738in}{2.990807in}}%
\pgfpathlineto{\pgfqpoint{4.126133in}{3.002800in}}%
\pgfpathlineto{\pgfqpoint{4.113299in}{3.012044in}}%
\pgfpathlineto{\pgfqpoint{4.100467in}{3.021434in}}%
\pgfpathlineto{\pgfqpoint{4.087638in}{3.030969in}}%
\pgfpathlineto{\pgfqpoint{4.074812in}{3.040652in}}%
\pgfpathlineto{\pgfqpoint{4.067405in}{3.028539in}}%
\pgfpathlineto{\pgfqpoint{4.059995in}{3.016508in}}%
\pgfpathlineto{\pgfqpoint{4.052581in}{3.004559in}}%
\pgfpathlineto{\pgfqpoint{4.045163in}{2.992689in}}%
\pgfpathclose%
\pgfusepath{fill}%
\end{pgfscope}%
\begin{pgfscope}%
\pgfpathrectangle{\pgfqpoint{1.254980in}{0.150000in}}{\pgfqpoint{5.490039in}{5.490039in}}%
\pgfusepath{clip}%
\pgfsetbuttcap%
\pgfsetroundjoin%
\definecolor{currentfill}{rgb}{0.223925,0.334994,0.548053}%
\pgfsetfillcolor{currentfill}%
\pgfsetfillopacity{0.700000}%
\pgfsetlinewidth{0.000000pt}%
\definecolor{currentstroke}{rgb}{0.000000,0.000000,0.000000}%
\pgfsetstrokecolor{currentstroke}%
\pgfsetdash{}{0pt}%
\pgfpathmoveto{\pgfqpoint{3.861434in}{3.071335in}}%
\pgfpathlineto{\pgfqpoint{3.874260in}{3.060047in}}%
\pgfpathlineto{\pgfqpoint{3.887087in}{3.048915in}}%
\pgfpathlineto{\pgfqpoint{3.899915in}{3.037938in}}%
\pgfpathlineto{\pgfqpoint{3.912745in}{3.027114in}}%
\pgfpathlineto{\pgfqpoint{3.920202in}{3.038862in}}%
\pgfpathlineto{\pgfqpoint{3.927654in}{3.050693in}}%
\pgfpathlineto{\pgfqpoint{3.935103in}{3.062607in}}%
\pgfpathlineto{\pgfqpoint{3.942547in}{3.074607in}}%
\pgfpathlineto{\pgfqpoint{3.929728in}{3.085529in}}%
\pgfpathlineto{\pgfqpoint{3.916911in}{3.096605in}}%
\pgfpathlineto{\pgfqpoint{3.904095in}{3.107835in}}%
\pgfpathlineto{\pgfqpoint{3.891280in}{3.119222in}}%
\pgfpathlineto{\pgfqpoint{3.883825in}{3.107117in}}%
\pgfpathlineto{\pgfqpoint{3.876366in}{3.095102in}}%
\pgfpathlineto{\pgfqpoint{3.868902in}{3.083175in}}%
\pgfpathlineto{\pgfqpoint{3.861434in}{3.071335in}}%
\pgfpathclose%
\pgfusepath{fill}%
\end{pgfscope}%
\begin{pgfscope}%
\pgfpathrectangle{\pgfqpoint{1.254980in}{0.150000in}}{\pgfqpoint{5.490039in}{5.490039in}}%
\pgfusepath{clip}%
\pgfsetbuttcap%
\pgfsetroundjoin%
\definecolor{currentfill}{rgb}{0.243113,0.292092,0.538516}%
\pgfsetfillcolor{currentfill}%
\pgfsetfillopacity{0.700000}%
\pgfsetlinewidth{0.000000pt}%
\definecolor{currentstroke}{rgb}{0.000000,0.000000,0.000000}%
\pgfsetstrokecolor{currentstroke}%
\pgfsetdash{}{0pt}%
\pgfpathmoveto{\pgfqpoint{4.655514in}{2.964544in}}%
\pgfpathlineto{\pgfqpoint{4.668460in}{2.959173in}}%
\pgfpathlineto{\pgfqpoint{4.681412in}{2.953929in}}%
\pgfpathlineto{\pgfqpoint{4.694371in}{2.948812in}}%
\pgfpathlineto{\pgfqpoint{4.707336in}{2.943823in}}%
\pgfpathlineto{\pgfqpoint{4.714581in}{2.955717in}}%
\pgfpathlineto{\pgfqpoint{4.721823in}{2.967684in}}%
\pgfpathlineto{\pgfqpoint{4.729062in}{2.979728in}}%
\pgfpathlineto{\pgfqpoint{4.736298in}{2.991849in}}%
\pgfpathlineto{\pgfqpoint{4.723344in}{2.997031in}}%
\pgfpathlineto{\pgfqpoint{4.710396in}{3.002340in}}%
\pgfpathlineto{\pgfqpoint{4.697455in}{3.007776in}}%
\pgfpathlineto{\pgfqpoint{4.684520in}{3.013339in}}%
\pgfpathlineto{\pgfqpoint{4.677272in}{3.001020in}}%
\pgfpathlineto{\pgfqpoint{4.670022in}{2.988782in}}%
\pgfpathlineto{\pgfqpoint{4.662770in}{2.976624in}}%
\pgfpathlineto{\pgfqpoint{4.655514in}{2.964544in}}%
\pgfpathclose%
\pgfusepath{fill}%
\end{pgfscope}%
\begin{pgfscope}%
\pgfpathrectangle{\pgfqpoint{1.254980in}{0.150000in}}{\pgfqpoint{5.490039in}{5.490039in}}%
\pgfusepath{clip}%
\pgfsetbuttcap%
\pgfsetroundjoin%
\definecolor{currentfill}{rgb}{0.248629,0.278775,0.534556}%
\pgfsetfillcolor{currentfill}%
\pgfsetfillopacity{0.700000}%
\pgfsetlinewidth{0.000000pt}%
\definecolor{currentstroke}{rgb}{0.000000,0.000000,0.000000}%
\pgfsetstrokecolor{currentstroke}%
\pgfsetdash{}{0pt}%
\pgfpathmoveto{\pgfqpoint{4.442212in}{2.940994in}}%
\pgfpathlineto{\pgfqpoint{4.455110in}{2.934394in}}%
\pgfpathlineto{\pgfqpoint{4.468014in}{2.927927in}}%
\pgfpathlineto{\pgfqpoint{4.480923in}{2.921592in}}%
\pgfpathlineto{\pgfqpoint{4.493838in}{2.915390in}}%
\pgfpathlineto{\pgfqpoint{4.501139in}{2.927227in}}%
\pgfpathlineto{\pgfqpoint{4.508438in}{2.939134in}}%
\pgfpathlineto{\pgfqpoint{4.515733in}{2.951114in}}%
\pgfpathlineto{\pgfqpoint{4.523024in}{2.963168in}}%
\pgfpathlineto{\pgfqpoint{4.510120in}{2.969531in}}%
\pgfpathlineto{\pgfqpoint{4.497222in}{2.976027in}}%
\pgfpathlineto{\pgfqpoint{4.484328in}{2.982654in}}%
\pgfpathlineto{\pgfqpoint{4.471440in}{2.989415in}}%
\pgfpathlineto{\pgfqpoint{4.464138in}{2.977195in}}%
\pgfpathlineto{\pgfqpoint{4.456832in}{2.965052in}}%
\pgfpathlineto{\pgfqpoint{4.449524in}{2.952986in}}%
\pgfpathlineto{\pgfqpoint{4.442212in}{2.940994in}}%
\pgfpathclose%
\pgfusepath{fill}%
\end{pgfscope}%
\begin{pgfscope}%
\pgfpathrectangle{\pgfqpoint{1.254980in}{0.150000in}}{\pgfqpoint{5.490039in}{5.490039in}}%
\pgfusepath{clip}%
\pgfsetbuttcap%
\pgfsetroundjoin%
\definecolor{currentfill}{rgb}{0.229739,0.322361,0.545706}%
\pgfsetfillcolor{currentfill}%
\pgfsetfillopacity{0.700000}%
\pgfsetlinewidth{0.000000pt}%
\definecolor{currentstroke}{rgb}{0.000000,0.000000,0.000000}%
\pgfsetstrokecolor{currentstroke}%
\pgfsetdash{}{0pt}%
\pgfpathmoveto{\pgfqpoint{4.949816in}{3.034278in}}%
\pgfpathlineto{\pgfqpoint{4.962839in}{3.030278in}}%
\pgfpathlineto{\pgfqpoint{4.975870in}{3.026400in}}%
\pgfpathlineto{\pgfqpoint{4.988909in}{3.022643in}}%
\pgfpathlineto{\pgfqpoint{5.001956in}{3.019007in}}%
\pgfpathlineto{\pgfqpoint{5.009124in}{3.030996in}}%
\pgfpathlineto{\pgfqpoint{5.016290in}{3.043069in}}%
\pgfpathlineto{\pgfqpoint{5.023455in}{3.055230in}}%
\pgfpathlineto{\pgfqpoint{5.010417in}{3.059044in}}%
\pgfpathlineto{\pgfqpoint{4.997388in}{3.062979in}}%
\pgfpathlineto{\pgfqpoint{4.984366in}{3.067035in}}%
\pgfpathlineto{\pgfqpoint{4.971351in}{3.071212in}}%
\pgfpathlineto{\pgfqpoint{4.964175in}{3.058810in}}%
\pgfpathlineto{\pgfqpoint{4.956997in}{3.046500in}}%
\pgfpathlineto{\pgfqpoint{4.949816in}{3.034278in}}%
\pgfpathclose%
\pgfusepath{fill}%
\end{pgfscope}%
\begin{pgfscope}%
\pgfpathrectangle{\pgfqpoint{1.254980in}{0.150000in}}{\pgfqpoint{5.490039in}{5.490039in}}%
\pgfusepath{clip}%
\pgfsetbuttcap%
\pgfsetroundjoin%
\definecolor{currentfill}{rgb}{0.174274,0.445044,0.557792}%
\pgfsetfillcolor{currentfill}%
\pgfsetfillopacity{0.700000}%
\pgfsetlinewidth{0.000000pt}%
\definecolor{currentstroke}{rgb}{0.000000,0.000000,0.000000}%
\pgfsetstrokecolor{currentstroke}%
\pgfsetdash{}{0pt}%
\pgfpathmoveto{\pgfqpoint{3.523439in}{3.342635in}}%
\pgfpathlineto{\pgfqpoint{3.536287in}{3.327432in}}%
\pgfpathlineto{\pgfqpoint{3.549133in}{3.312408in}}%
\pgfpathlineto{\pgfqpoint{3.561977in}{3.297560in}}%
\pgfpathlineto{\pgfqpoint{3.574820in}{3.282888in}}%
\pgfpathlineto{\pgfqpoint{3.582367in}{3.294777in}}%
\pgfpathlineto{\pgfqpoint{3.589908in}{3.306769in}}%
\pgfpathlineto{\pgfqpoint{3.597445in}{3.318864in}}%
\pgfpathlineto{\pgfqpoint{3.604976in}{3.331064in}}%
\pgfpathlineto{\pgfqpoint{3.592147in}{3.345819in}}%
\pgfpathlineto{\pgfqpoint{3.579315in}{3.360750in}}%
\pgfpathlineto{\pgfqpoint{3.566482in}{3.375858in}}%
\pgfpathlineto{\pgfqpoint{3.553646in}{3.391144in}}%
\pgfpathlineto{\pgfqpoint{3.546102in}{3.378855in}}%
\pgfpathlineto{\pgfqpoint{3.538553in}{3.366674in}}%
\pgfpathlineto{\pgfqpoint{3.530999in}{3.354601in}}%
\pgfpathlineto{\pgfqpoint{3.523439in}{3.342635in}}%
\pgfpathclose%
\pgfusepath{fill}%
\end{pgfscope}%
\begin{pgfscope}%
\pgfpathrectangle{\pgfqpoint{1.254980in}{0.150000in}}{\pgfqpoint{5.490039in}{5.490039in}}%
\pgfusepath{clip}%
\pgfsetbuttcap%
\pgfsetroundjoin%
\definecolor{currentfill}{rgb}{0.183898,0.422383,0.556944}%
\pgfsetfillcolor{currentfill}%
\pgfsetfillopacity{0.700000}%
\pgfsetlinewidth{0.000000pt}%
\definecolor{currentstroke}{rgb}{0.000000,0.000000,0.000000}%
\pgfsetstrokecolor{currentstroke}%
\pgfsetdash{}{0pt}%
\pgfpathmoveto{\pgfqpoint{3.574820in}{3.282888in}}%
\pgfpathlineto{\pgfqpoint{3.587661in}{3.268390in}}%
\pgfpathlineto{\pgfqpoint{3.600500in}{3.254066in}}%
\pgfpathlineto{\pgfqpoint{3.613338in}{3.239915in}}%
\pgfpathlineto{\pgfqpoint{3.626175in}{3.225935in}}%
\pgfpathlineto{\pgfqpoint{3.633709in}{3.237747in}}%
\pgfpathlineto{\pgfqpoint{3.641238in}{3.249658in}}%
\pgfpathlineto{\pgfqpoint{3.648763in}{3.261668in}}%
\pgfpathlineto{\pgfqpoint{3.656282in}{3.273778in}}%
\pgfpathlineto{\pgfqpoint{3.643458in}{3.287841in}}%
\pgfpathlineto{\pgfqpoint{3.630632in}{3.302076in}}%
\pgfpathlineto{\pgfqpoint{3.617805in}{3.316483in}}%
\pgfpathlineto{\pgfqpoint{3.604976in}{3.331064in}}%
\pgfpathlineto{\pgfqpoint{3.597445in}{3.318864in}}%
\pgfpathlineto{\pgfqpoint{3.589908in}{3.306769in}}%
\pgfpathlineto{\pgfqpoint{3.582367in}{3.294777in}}%
\pgfpathlineto{\pgfqpoint{3.574820in}{3.282888in}}%
\pgfpathclose%
\pgfusepath{fill}%
\end{pgfscope}%
\begin{pgfscope}%
\pgfpathrectangle{\pgfqpoint{1.254980in}{0.150000in}}{\pgfqpoint{5.490039in}{5.490039in}}%
\pgfusepath{clip}%
\pgfsetbuttcap%
\pgfsetroundjoin%
\definecolor{currentfill}{rgb}{0.165117,0.467423,0.558141}%
\pgfsetfillcolor{currentfill}%
\pgfsetfillopacity{0.700000}%
\pgfsetlinewidth{0.000000pt}%
\definecolor{currentstroke}{rgb}{0.000000,0.000000,0.000000}%
\pgfsetstrokecolor{currentstroke}%
\pgfsetdash{}{0pt}%
\pgfpathmoveto{\pgfqpoint{3.472025in}{3.405243in}}%
\pgfpathlineto{\pgfqpoint{3.484882in}{3.389319in}}%
\pgfpathlineto{\pgfqpoint{3.497737in}{3.373577in}}%
\pgfpathlineto{\pgfqpoint{3.510589in}{3.358016in}}%
\pgfpathlineto{\pgfqpoint{3.523439in}{3.342635in}}%
\pgfpathlineto{\pgfqpoint{3.530999in}{3.354601in}}%
\pgfpathlineto{\pgfqpoint{3.538553in}{3.366674in}}%
\pgfpathlineto{\pgfqpoint{3.546102in}{3.378855in}}%
\pgfpathlineto{\pgfqpoint{3.553646in}{3.391144in}}%
\pgfpathlineto{\pgfqpoint{3.540809in}{3.406608in}}%
\pgfpathlineto{\pgfqpoint{3.527970in}{3.422253in}}%
\pgfpathlineto{\pgfqpoint{3.515128in}{3.438078in}}%
\pgfpathlineto{\pgfqpoint{3.502283in}{3.454086in}}%
\pgfpathlineto{\pgfqpoint{3.494727in}{3.441707in}}%
\pgfpathlineto{\pgfqpoint{3.487165in}{3.429441in}}%
\pgfpathlineto{\pgfqpoint{3.479598in}{3.417287in}}%
\pgfpathlineto{\pgfqpoint{3.472025in}{3.405243in}}%
\pgfpathclose%
\pgfusepath{fill}%
\end{pgfscope}%
\begin{pgfscope}%
\pgfpathrectangle{\pgfqpoint{1.254980in}{0.150000in}}{\pgfqpoint{5.490039in}{5.490039in}}%
\pgfusepath{clip}%
\pgfsetbuttcap%
\pgfsetroundjoin%
\definecolor{currentfill}{rgb}{0.235526,0.309527,0.542944}%
\pgfsetfillcolor{currentfill}%
\pgfsetfillopacity{0.700000}%
\pgfsetlinewidth{0.000000pt}%
\definecolor{currentstroke}{rgb}{0.000000,0.000000,0.000000}%
\pgfsetstrokecolor{currentstroke}%
\pgfsetdash{}{0pt}%
\pgfpathmoveto{\pgfqpoint{4.869007in}{3.002557in}}%
\pgfpathlineto{\pgfqpoint{4.882012in}{2.998291in}}%
\pgfpathlineto{\pgfqpoint{4.895024in}{2.994148in}}%
\pgfpathlineto{\pgfqpoint{4.908044in}{2.990128in}}%
\pgfpathlineto{\pgfqpoint{4.921072in}{2.986230in}}%
\pgfpathlineto{\pgfqpoint{4.928261in}{2.998121in}}%
\pgfpathlineto{\pgfqpoint{4.935449in}{3.010090in}}%
\pgfpathlineto{\pgfqpoint{4.942634in}{3.022142in}}%
\pgfpathlineto{\pgfqpoint{4.949816in}{3.034278in}}%
\pgfpathlineto{\pgfqpoint{4.936800in}{3.038399in}}%
\pgfpathlineto{\pgfqpoint{4.923792in}{3.042643in}}%
\pgfpathlineto{\pgfqpoint{4.910791in}{3.047010in}}%
\pgfpathlineto{\pgfqpoint{4.897798in}{3.051499in}}%
\pgfpathlineto{\pgfqpoint{4.890603in}{3.039134in}}%
\pgfpathlineto{\pgfqpoint{4.883407in}{3.026857in}}%
\pgfpathlineto{\pgfqpoint{4.876208in}{3.014666in}}%
\pgfpathlineto{\pgfqpoint{4.869007in}{3.002557in}}%
\pgfpathclose%
\pgfusepath{fill}%
\end{pgfscope}%
\begin{pgfscope}%
\pgfpathrectangle{\pgfqpoint{1.254980in}{0.150000in}}{\pgfqpoint{5.490039in}{5.490039in}}%
\pgfusepath{clip}%
\pgfsetbuttcap%
\pgfsetroundjoin%
\definecolor{currentfill}{rgb}{0.194100,0.399323,0.555565}%
\pgfsetfillcolor{currentfill}%
\pgfsetfillopacity{0.700000}%
\pgfsetlinewidth{0.000000pt}%
\definecolor{currentstroke}{rgb}{0.000000,0.000000,0.000000}%
\pgfsetstrokecolor{currentstroke}%
\pgfsetdash{}{0pt}%
\pgfpathmoveto{\pgfqpoint{3.626175in}{3.225935in}}%
\pgfpathlineto{\pgfqpoint{3.639011in}{3.212126in}}%
\pgfpathlineto{\pgfqpoint{3.651846in}{3.198486in}}%
\pgfpathlineto{\pgfqpoint{3.664680in}{3.185015in}}%
\pgfpathlineto{\pgfqpoint{3.677514in}{3.171711in}}%
\pgfpathlineto{\pgfqpoint{3.685036in}{3.183447in}}%
\pgfpathlineto{\pgfqpoint{3.692553in}{3.195276in}}%
\pgfpathlineto{\pgfqpoint{3.700065in}{3.207201in}}%
\pgfpathlineto{\pgfqpoint{3.707573in}{3.219222in}}%
\pgfpathlineto{\pgfqpoint{3.694751in}{3.232609in}}%
\pgfpathlineto{\pgfqpoint{3.681929in}{3.246163in}}%
\pgfpathlineto{\pgfqpoint{3.669106in}{3.259886in}}%
\pgfpathlineto{\pgfqpoint{3.656282in}{3.273778in}}%
\pgfpathlineto{\pgfqpoint{3.648763in}{3.261668in}}%
\pgfpathlineto{\pgfqpoint{3.641238in}{3.249658in}}%
\pgfpathlineto{\pgfqpoint{3.633709in}{3.237747in}}%
\pgfpathlineto{\pgfqpoint{3.626175in}{3.225935in}}%
\pgfpathclose%
\pgfusepath{fill}%
\end{pgfscope}%
\begin{pgfscope}%
\pgfpathrectangle{\pgfqpoint{1.254980in}{0.150000in}}{\pgfqpoint{5.490039in}{5.490039in}}%
\pgfusepath{clip}%
\pgfsetbuttcap%
\pgfsetroundjoin%
\definecolor{currentfill}{rgb}{0.231674,0.318106,0.544834}%
\pgfsetfillcolor{currentfill}%
\pgfsetfillopacity{0.700000}%
\pgfsetlinewidth{0.000000pt}%
\definecolor{currentstroke}{rgb}{0.000000,0.000000,0.000000}%
\pgfsetstrokecolor{currentstroke}%
\pgfsetdash{}{0pt}%
\pgfpathmoveto{\pgfqpoint{3.912745in}{3.027114in}}%
\pgfpathlineto{\pgfqpoint{3.925576in}{3.016444in}}%
\pgfpathlineto{\pgfqpoint{3.938409in}{3.005927in}}%
\pgfpathlineto{\pgfqpoint{3.951244in}{2.995561in}}%
\pgfpathlineto{\pgfqpoint{3.964081in}{2.985347in}}%
\pgfpathlineto{\pgfqpoint{3.971527in}{2.997002in}}%
\pgfpathlineto{\pgfqpoint{3.978968in}{3.008736in}}%
\pgfpathlineto{\pgfqpoint{3.986405in}{3.020550in}}%
\pgfpathlineto{\pgfqpoint{3.993838in}{3.032445in}}%
\pgfpathlineto{\pgfqpoint{3.981013in}{3.042758in}}%
\pgfpathlineto{\pgfqpoint{3.968189in}{3.053222in}}%
\pgfpathlineto{\pgfqpoint{3.955367in}{3.063838in}}%
\pgfpathlineto{\pgfqpoint{3.942547in}{3.074607in}}%
\pgfpathlineto{\pgfqpoint{3.935103in}{3.062607in}}%
\pgfpathlineto{\pgfqpoint{3.927654in}{3.050693in}}%
\pgfpathlineto{\pgfqpoint{3.920202in}{3.038862in}}%
\pgfpathlineto{\pgfqpoint{3.912745in}{3.027114in}}%
\pgfpathclose%
\pgfusepath{fill}%
\end{pgfscope}%
\begin{pgfscope}%
\pgfpathrectangle{\pgfqpoint{1.254980in}{0.150000in}}{\pgfqpoint{5.490039in}{5.490039in}}%
\pgfusepath{clip}%
\pgfsetbuttcap%
\pgfsetroundjoin%
\definecolor{currentfill}{rgb}{0.120638,0.625828,0.533488}%
\pgfsetfillcolor{currentfill}%
\pgfsetfillopacity{0.700000}%
\pgfsetlinewidth{0.000000pt}%
\definecolor{currentstroke}{rgb}{0.000000,0.000000,0.000000}%
\pgfsetstrokecolor{currentstroke}%
\pgfsetdash{}{0pt}%
\pgfpathmoveto{\pgfqpoint{3.244637in}{3.814379in}}%
\pgfpathlineto{\pgfqpoint{3.257563in}{3.794466in}}%
\pgfpathlineto{\pgfqpoint{3.270484in}{3.774761in}}%
\pgfpathlineto{\pgfqpoint{3.283400in}{3.755263in}}%
\pgfpathlineto{\pgfqpoint{3.296310in}{3.735970in}}%
\pgfpathlineto{\pgfqpoint{3.303914in}{3.748840in}}%
\pgfpathlineto{\pgfqpoint{3.311512in}{3.761840in}}%
\pgfpathlineto{\pgfqpoint{3.319105in}{3.774973in}}%
\pgfpathlineto{\pgfqpoint{3.326691in}{3.788240in}}%
\pgfpathlineto{\pgfqpoint{3.313794in}{3.807634in}}%
\pgfpathlineto{\pgfqpoint{3.300892in}{3.827234in}}%
\pgfpathlineto{\pgfqpoint{3.287985in}{3.847040in}}%
\pgfpathlineto{\pgfqpoint{3.275072in}{3.867055in}}%
\pgfpathlineto{\pgfqpoint{3.267472in}{3.853680in}}%
\pgfpathlineto{\pgfqpoint{3.259866in}{3.840444in}}%
\pgfpathlineto{\pgfqpoint{3.252254in}{3.827344in}}%
\pgfpathlineto{\pgfqpoint{3.244637in}{3.814379in}}%
\pgfpathclose%
\pgfusepath{fill}%
\end{pgfscope}%
\begin{pgfscope}%
\pgfpathrectangle{\pgfqpoint{1.254980in}{0.150000in}}{\pgfqpoint{5.490039in}{5.490039in}}%
\pgfusepath{clip}%
\pgfsetbuttcap%
\pgfsetroundjoin%
\definecolor{currentfill}{rgb}{0.248629,0.278775,0.534556}%
\pgfsetfillcolor{currentfill}%
\pgfsetfillopacity{0.700000}%
\pgfsetlinewidth{0.000000pt}%
\definecolor{currentstroke}{rgb}{0.000000,0.000000,0.000000}%
\pgfsetstrokecolor{currentstroke}%
\pgfsetdash{}{0pt}%
\pgfpathmoveto{\pgfqpoint{4.574696in}{2.939027in}}%
\pgfpathlineto{\pgfqpoint{4.587629in}{2.933317in}}%
\pgfpathlineto{\pgfqpoint{4.600567in}{2.927737in}}%
\pgfpathlineto{\pgfqpoint{4.613511in}{2.922285in}}%
\pgfpathlineto{\pgfqpoint{4.626462in}{2.916962in}}%
\pgfpathlineto{\pgfqpoint{4.633729in}{2.928751in}}%
\pgfpathlineto{\pgfqpoint{4.640994in}{2.940609in}}%
\pgfpathlineto{\pgfqpoint{4.648255in}{2.952540in}}%
\pgfpathlineto{\pgfqpoint{4.655514in}{2.964544in}}%
\pgfpathlineto{\pgfqpoint{4.642574in}{2.970044in}}%
\pgfpathlineto{\pgfqpoint{4.629640in}{2.975672in}}%
\pgfpathlineto{\pgfqpoint{4.616713in}{2.981428in}}%
\pgfpathlineto{\pgfqpoint{4.603791in}{2.987315in}}%
\pgfpathlineto{\pgfqpoint{4.596522in}{2.975128in}}%
\pgfpathlineto{\pgfqpoint{4.589250in}{2.963019in}}%
\pgfpathlineto{\pgfqpoint{4.581974in}{2.950986in}}%
\pgfpathlineto{\pgfqpoint{4.574696in}{2.939027in}}%
\pgfpathclose%
\pgfusepath{fill}%
\end{pgfscope}%
\begin{pgfscope}%
\pgfpathrectangle{\pgfqpoint{1.254980in}{0.150000in}}{\pgfqpoint{5.490039in}{5.490039in}}%
\pgfusepath{clip}%
\pgfsetbuttcap%
\pgfsetroundjoin%
\definecolor{currentfill}{rgb}{0.154815,0.493313,0.557840}%
\pgfsetfillcolor{currentfill}%
\pgfsetfillopacity{0.700000}%
\pgfsetlinewidth{0.000000pt}%
\definecolor{currentstroke}{rgb}{0.000000,0.000000,0.000000}%
\pgfsetstrokecolor{currentstroke}%
\pgfsetdash{}{0pt}%
\pgfpathmoveto{\pgfqpoint{3.420567in}{3.470785in}}%
\pgfpathlineto{\pgfqpoint{3.433436in}{3.454120in}}%
\pgfpathlineto{\pgfqpoint{3.446302in}{3.437642in}}%
\pgfpathlineto{\pgfqpoint{3.459165in}{3.421350in}}%
\pgfpathlineto{\pgfqpoint{3.472025in}{3.405243in}}%
\pgfpathlineto{\pgfqpoint{3.479598in}{3.417287in}}%
\pgfpathlineto{\pgfqpoint{3.487165in}{3.429441in}}%
\pgfpathlineto{\pgfqpoint{3.494727in}{3.441707in}}%
\pgfpathlineto{\pgfqpoint{3.502283in}{3.454086in}}%
\pgfpathlineto{\pgfqpoint{3.489436in}{3.470277in}}%
\pgfpathlineto{\pgfqpoint{3.476586in}{3.486652in}}%
\pgfpathlineto{\pgfqpoint{3.463734in}{3.503214in}}%
\pgfpathlineto{\pgfqpoint{3.450878in}{3.519962in}}%
\pgfpathlineto{\pgfqpoint{3.443308in}{3.507494in}}%
\pgfpathlineto{\pgfqpoint{3.435733in}{3.495142in}}%
\pgfpathlineto{\pgfqpoint{3.428153in}{3.482906in}}%
\pgfpathlineto{\pgfqpoint{3.420567in}{3.470785in}}%
\pgfpathclose%
\pgfusepath{fill}%
\end{pgfscope}%
\begin{pgfscope}%
\pgfpathrectangle{\pgfqpoint{1.254980in}{0.150000in}}{\pgfqpoint{5.490039in}{5.490039in}}%
\pgfusepath{clip}%
\pgfsetbuttcap%
\pgfsetroundjoin%
\definecolor{currentfill}{rgb}{0.250425,0.274290,0.533103}%
\pgfsetfillcolor{currentfill}%
\pgfsetfillopacity{0.700000}%
\pgfsetlinewidth{0.000000pt}%
\definecolor{currentstroke}{rgb}{0.000000,0.000000,0.000000}%
\pgfsetstrokecolor{currentstroke}%
\pgfsetdash{}{0pt}%
\pgfpathmoveto{\pgfqpoint{4.228926in}{2.934013in}}%
\pgfpathlineto{\pgfqpoint{4.241792in}{2.926050in}}%
\pgfpathlineto{\pgfqpoint{4.254661in}{2.918228in}}%
\pgfpathlineto{\pgfqpoint{4.267534in}{2.910544in}}%
\pgfpathlineto{\pgfqpoint{4.280411in}{2.902998in}}%
\pgfpathlineto{\pgfqpoint{4.287772in}{2.914700in}}%
\pgfpathlineto{\pgfqpoint{4.295129in}{2.926471in}}%
\pgfpathlineto{\pgfqpoint{4.302482in}{2.938313in}}%
\pgfpathlineto{\pgfqpoint{4.309832in}{2.950228in}}%
\pgfpathlineto{\pgfqpoint{4.296965in}{2.957903in}}%
\pgfpathlineto{\pgfqpoint{4.284102in}{2.965717in}}%
\pgfpathlineto{\pgfqpoint{4.271244in}{2.973670in}}%
\pgfpathlineto{\pgfqpoint{4.258389in}{2.981761in}}%
\pgfpathlineto{\pgfqpoint{4.251029in}{2.969711in}}%
\pgfpathlineto{\pgfqpoint{4.243665in}{2.957737in}}%
\pgfpathlineto{\pgfqpoint{4.236298in}{2.945838in}}%
\pgfpathlineto{\pgfqpoint{4.228926in}{2.934013in}}%
\pgfpathclose%
\pgfusepath{fill}%
\end{pgfscope}%
\begin{pgfscope}%
\pgfpathrectangle{\pgfqpoint{1.254980in}{0.150000in}}{\pgfqpoint{5.490039in}{5.490039in}}%
\pgfusepath{clip}%
\pgfsetbuttcap%
\pgfsetroundjoin%
\definecolor{currentfill}{rgb}{0.204903,0.375746,0.553533}%
\pgfsetfillcolor{currentfill}%
\pgfsetfillopacity{0.700000}%
\pgfsetlinewidth{0.000000pt}%
\definecolor{currentstroke}{rgb}{0.000000,0.000000,0.000000}%
\pgfsetstrokecolor{currentstroke}%
\pgfsetdash{}{0pt}%
\pgfpathmoveto{\pgfqpoint{3.677514in}{3.171711in}}%
\pgfpathlineto{\pgfqpoint{3.690347in}{3.158575in}}%
\pgfpathlineto{\pgfqpoint{3.703180in}{3.145604in}}%
\pgfpathlineto{\pgfqpoint{3.716013in}{3.132798in}}%
\pgfpathlineto{\pgfqpoint{3.728845in}{3.120156in}}%
\pgfpathlineto{\pgfqpoint{3.736355in}{3.131814in}}%
\pgfpathlineto{\pgfqpoint{3.743860in}{3.143562in}}%
\pgfpathlineto{\pgfqpoint{3.751360in}{3.155402in}}%
\pgfpathlineto{\pgfqpoint{3.758856in}{3.167335in}}%
\pgfpathlineto{\pgfqpoint{3.746035in}{3.180060in}}%
\pgfpathlineto{\pgfqpoint{3.733215in}{3.192949in}}%
\pgfpathlineto{\pgfqpoint{3.720394in}{3.206003in}}%
\pgfpathlineto{\pgfqpoint{3.707573in}{3.219222in}}%
\pgfpathlineto{\pgfqpoint{3.700065in}{3.207201in}}%
\pgfpathlineto{\pgfqpoint{3.692553in}{3.195276in}}%
\pgfpathlineto{\pgfqpoint{3.685036in}{3.183447in}}%
\pgfpathlineto{\pgfqpoint{3.677514in}{3.171711in}}%
\pgfpathclose%
\pgfusepath{fill}%
\end{pgfscope}%
\begin{pgfscope}%
\pgfpathrectangle{\pgfqpoint{1.254980in}{0.150000in}}{\pgfqpoint{5.490039in}{5.490039in}}%
\pgfusepath{clip}%
\pgfsetbuttcap%
\pgfsetroundjoin%
\definecolor{currentfill}{rgb}{0.244972,0.287675,0.537260}%
\pgfsetfillcolor{currentfill}%
\pgfsetfillopacity{0.700000}%
\pgfsetlinewidth{0.000000pt}%
\definecolor{currentstroke}{rgb}{0.000000,0.000000,0.000000}%
\pgfsetstrokecolor{currentstroke}%
\pgfsetdash{}{0pt}%
\pgfpathmoveto{\pgfqpoint{4.096527in}{2.955294in}}%
\pgfpathlineto{\pgfqpoint{4.109375in}{2.946309in}}%
\pgfpathlineto{\pgfqpoint{4.122226in}{2.937469in}}%
\pgfpathlineto{\pgfqpoint{4.135081in}{2.928772in}}%
\pgfpathlineto{\pgfqpoint{4.147938in}{2.920218in}}%
\pgfpathlineto{\pgfqpoint{4.155335in}{2.931871in}}%
\pgfpathlineto{\pgfqpoint{4.162728in}{2.943596in}}%
\pgfpathlineto{\pgfqpoint{4.170117in}{2.955394in}}%
\pgfpathlineto{\pgfqpoint{4.177503in}{2.967267in}}%
\pgfpathlineto{\pgfqpoint{4.164656in}{2.975935in}}%
\pgfpathlineto{\pgfqpoint{4.151812in}{2.984747in}}%
\pgfpathlineto{\pgfqpoint{4.138971in}{2.993701in}}%
\pgfpathlineto{\pgfqpoint{4.126133in}{3.002800in}}%
\pgfpathlineto{\pgfqpoint{4.118738in}{2.990807in}}%
\pgfpathlineto{\pgfqpoint{4.111338in}{2.978893in}}%
\pgfpathlineto{\pgfqpoint{4.103934in}{2.967056in}}%
\pgfpathlineto{\pgfqpoint{4.096527in}{2.955294in}}%
\pgfpathclose%
\pgfusepath{fill}%
\end{pgfscope}%
\begin{pgfscope}%
\pgfpathrectangle{\pgfqpoint{1.254980in}{0.150000in}}{\pgfqpoint{5.490039in}{5.490039in}}%
\pgfusepath{clip}%
\pgfsetbuttcap%
\pgfsetroundjoin%
\definecolor{currentfill}{rgb}{0.252194,0.269783,0.531579}%
\pgfsetfillcolor{currentfill}%
\pgfsetfillopacity{0.700000}%
\pgfsetlinewidth{0.000000pt}%
\definecolor{currentstroke}{rgb}{0.000000,0.000000,0.000000}%
\pgfsetstrokecolor{currentstroke}%
\pgfsetdash{}{0pt}%
\pgfpathmoveto{\pgfqpoint{4.361344in}{2.920899in}}%
\pgfpathlineto{\pgfqpoint{4.374233in}{2.913907in}}%
\pgfpathlineto{\pgfqpoint{4.387127in}{2.907050in}}%
\pgfpathlineto{\pgfqpoint{4.400026in}{2.900327in}}%
\pgfpathlineto{\pgfqpoint{4.412930in}{2.893739in}}%
\pgfpathlineto{\pgfqpoint{4.420256in}{2.905449in}}%
\pgfpathlineto{\pgfqpoint{4.427578in}{2.917227in}}%
\pgfpathlineto{\pgfqpoint{4.434896in}{2.929075in}}%
\pgfpathlineto{\pgfqpoint{4.442212in}{2.940994in}}%
\pgfpathlineto{\pgfqpoint{4.429318in}{2.947728in}}%
\pgfpathlineto{\pgfqpoint{4.416429in}{2.954595in}}%
\pgfpathlineto{\pgfqpoint{4.403546in}{2.961598in}}%
\pgfpathlineto{\pgfqpoint{4.390667in}{2.968735in}}%
\pgfpathlineto{\pgfqpoint{4.383341in}{2.956665in}}%
\pgfpathlineto{\pgfqpoint{4.376012in}{2.944670in}}%
\pgfpathlineto{\pgfqpoint{4.368680in}{2.932748in}}%
\pgfpathlineto{\pgfqpoint{4.361344in}{2.920899in}}%
\pgfpathclose%
\pgfusepath{fill}%
\end{pgfscope}%
\begin{pgfscope}%
\pgfpathrectangle{\pgfqpoint{1.254980in}{0.150000in}}{\pgfqpoint{5.490039in}{5.490039in}}%
\pgfusepath{clip}%
\pgfsetbuttcap%
\pgfsetroundjoin%
\definecolor{currentfill}{rgb}{0.143343,0.522773,0.556295}%
\pgfsetfillcolor{currentfill}%
\pgfsetfillopacity{0.700000}%
\pgfsetlinewidth{0.000000pt}%
\definecolor{currentstroke}{rgb}{0.000000,0.000000,0.000000}%
\pgfsetstrokecolor{currentstroke}%
\pgfsetdash{}{0pt}%
\pgfpathmoveto{\pgfqpoint{3.369056in}{3.539335in}}%
\pgfpathlineto{\pgfqpoint{3.381940in}{3.521911in}}%
\pgfpathlineto{\pgfqpoint{3.394819in}{3.504679in}}%
\pgfpathlineto{\pgfqpoint{3.407695in}{3.487637in}}%
\pgfpathlineto{\pgfqpoint{3.420567in}{3.470785in}}%
\pgfpathlineto{\pgfqpoint{3.428153in}{3.482906in}}%
\pgfpathlineto{\pgfqpoint{3.435733in}{3.495142in}}%
\pgfpathlineto{\pgfqpoint{3.443308in}{3.507494in}}%
\pgfpathlineto{\pgfqpoint{3.450878in}{3.519962in}}%
\pgfpathlineto{\pgfqpoint{3.438019in}{3.536898in}}%
\pgfpathlineto{\pgfqpoint{3.425156in}{3.554024in}}%
\pgfpathlineto{\pgfqpoint{3.412290in}{3.571340in}}%
\pgfpathlineto{\pgfqpoint{3.399420in}{3.588848in}}%
\pgfpathlineto{\pgfqpoint{3.391838in}{3.576290in}}%
\pgfpathlineto{\pgfqpoint{3.384250in}{3.563852in}}%
\pgfpathlineto{\pgfqpoint{3.376656in}{3.551534in}}%
\pgfpathlineto{\pgfqpoint{3.369056in}{3.539335in}}%
\pgfpathclose%
\pgfusepath{fill}%
\end{pgfscope}%
\begin{pgfscope}%
\pgfpathrectangle{\pgfqpoint{1.254980in}{0.150000in}}{\pgfqpoint{5.490039in}{5.490039in}}%
\pgfusepath{clip}%
\pgfsetbuttcap%
\pgfsetroundjoin%
\definecolor{currentfill}{rgb}{0.241237,0.296485,0.539709}%
\pgfsetfillcolor{currentfill}%
\pgfsetfillopacity{0.700000}%
\pgfsetlinewidth{0.000000pt}%
\definecolor{currentstroke}{rgb}{0.000000,0.000000,0.000000}%
\pgfsetstrokecolor{currentstroke}%
\pgfsetdash{}{0pt}%
\pgfpathmoveto{\pgfqpoint{4.788182in}{2.972382in}}%
\pgfpathlineto{\pgfqpoint{4.801170in}{2.967828in}}%
\pgfpathlineto{\pgfqpoint{4.814165in}{2.963399in}}%
\pgfpathlineto{\pgfqpoint{4.827167in}{2.959094in}}%
\pgfpathlineto{\pgfqpoint{4.840176in}{2.954912in}}%
\pgfpathlineto{\pgfqpoint{4.847388in}{2.966710in}}%
\pgfpathlineto{\pgfqpoint{4.854597in}{2.978582in}}%
\pgfpathlineto{\pgfqpoint{4.861803in}{2.990530in}}%
\pgfpathlineto{\pgfqpoint{4.869007in}{3.002557in}}%
\pgfpathlineto{\pgfqpoint{4.856009in}{3.006947in}}%
\pgfpathlineto{\pgfqpoint{4.843018in}{3.011460in}}%
\pgfpathlineto{\pgfqpoint{4.830034in}{3.016097in}}%
\pgfpathlineto{\pgfqpoint{4.817057in}{3.020859in}}%
\pgfpathlineto{\pgfqpoint{4.809842in}{3.008618in}}%
\pgfpathlineto{\pgfqpoint{4.802625in}{2.996459in}}%
\pgfpathlineto{\pgfqpoint{4.795405in}{2.984381in}}%
\pgfpathlineto{\pgfqpoint{4.788182in}{2.972382in}}%
\pgfpathclose%
\pgfusepath{fill}%
\end{pgfscope}%
\begin{pgfscope}%
\pgfpathrectangle{\pgfqpoint{1.254980in}{0.150000in}}{\pgfqpoint{5.490039in}{5.490039in}}%
\pgfusepath{clip}%
\pgfsetbuttcap%
\pgfsetroundjoin%
\definecolor{currentfill}{rgb}{0.214298,0.355619,0.551184}%
\pgfsetfillcolor{currentfill}%
\pgfsetfillopacity{0.700000}%
\pgfsetlinewidth{0.000000pt}%
\definecolor{currentstroke}{rgb}{0.000000,0.000000,0.000000}%
\pgfsetstrokecolor{currentstroke}%
\pgfsetdash{}{0pt}%
\pgfpathmoveto{\pgfqpoint{3.728845in}{3.120156in}}%
\pgfpathlineto{\pgfqpoint{3.741678in}{3.107677in}}%
\pgfpathlineto{\pgfqpoint{3.754511in}{3.095360in}}%
\pgfpathlineto{\pgfqpoint{3.767344in}{3.083205in}}%
\pgfpathlineto{\pgfqpoint{3.780177in}{3.071210in}}%
\pgfpathlineto{\pgfqpoint{3.787675in}{3.082791in}}%
\pgfpathlineto{\pgfqpoint{3.795168in}{3.094459in}}%
\pgfpathlineto{\pgfqpoint{3.802656in}{3.106214in}}%
\pgfpathlineto{\pgfqpoint{3.810140in}{3.118057in}}%
\pgfpathlineto{\pgfqpoint{3.797319in}{3.130135in}}%
\pgfpathlineto{\pgfqpoint{3.784497in}{3.142373in}}%
\pgfpathlineto{\pgfqpoint{3.771677in}{3.154773in}}%
\pgfpathlineto{\pgfqpoint{3.758856in}{3.167335in}}%
\pgfpathlineto{\pgfqpoint{3.751360in}{3.155402in}}%
\pgfpathlineto{\pgfqpoint{3.743860in}{3.143562in}}%
\pgfpathlineto{\pgfqpoint{3.736355in}{3.131814in}}%
\pgfpathlineto{\pgfqpoint{3.728845in}{3.120156in}}%
\pgfpathclose%
\pgfusepath{fill}%
\end{pgfscope}%
\begin{pgfscope}%
\pgfpathrectangle{\pgfqpoint{1.254980in}{0.150000in}}{\pgfqpoint{5.490039in}{5.490039in}}%
\pgfusepath{clip}%
\pgfsetbuttcap%
\pgfsetroundjoin%
\definecolor{currentfill}{rgb}{0.239346,0.300855,0.540844}%
\pgfsetfillcolor{currentfill}%
\pgfsetfillopacity{0.700000}%
\pgfsetlinewidth{0.000000pt}%
\definecolor{currentstroke}{rgb}{0.000000,0.000000,0.000000}%
\pgfsetstrokecolor{currentstroke}%
\pgfsetdash{}{0pt}%
\pgfpathmoveto{\pgfqpoint{3.964081in}{2.985347in}}%
\pgfpathlineto{\pgfqpoint{3.976920in}{2.975283in}}%
\pgfpathlineto{\pgfqpoint{3.989760in}{2.965368in}}%
\pgfpathlineto{\pgfqpoint{4.002603in}{2.955602in}}%
\pgfpathlineto{\pgfqpoint{4.015449in}{2.945985in}}%
\pgfpathlineto{\pgfqpoint{4.022883in}{2.957547in}}%
\pgfpathlineto{\pgfqpoint{4.030314in}{2.969185in}}%
\pgfpathlineto{\pgfqpoint{4.037740in}{2.980898in}}%
\pgfpathlineto{\pgfqpoint{4.045163in}{2.992689in}}%
\pgfpathlineto{\pgfqpoint{4.032328in}{3.002405in}}%
\pgfpathlineto{\pgfqpoint{4.019496in}{3.012269in}}%
\pgfpathlineto{\pgfqpoint{4.006666in}{3.022283in}}%
\pgfpathlineto{\pgfqpoint{3.993838in}{3.032445in}}%
\pgfpathlineto{\pgfqpoint{3.986405in}{3.020550in}}%
\pgfpathlineto{\pgfqpoint{3.978968in}{3.008736in}}%
\pgfpathlineto{\pgfqpoint{3.971527in}{2.997002in}}%
\pgfpathlineto{\pgfqpoint{3.964081in}{2.985347in}}%
\pgfpathclose%
\pgfusepath{fill}%
\end{pgfscope}%
\begin{pgfscope}%
\pgfpathrectangle{\pgfqpoint{1.254980in}{0.150000in}}{\pgfqpoint{5.490039in}{5.490039in}}%
\pgfusepath{clip}%
\pgfsetbuttcap%
\pgfsetroundjoin%
\definecolor{currentfill}{rgb}{0.133743,0.548535,0.553541}%
\pgfsetfillcolor{currentfill}%
\pgfsetfillopacity{0.700000}%
\pgfsetlinewidth{0.000000pt}%
\definecolor{currentstroke}{rgb}{0.000000,0.000000,0.000000}%
\pgfsetstrokecolor{currentstroke}%
\pgfsetdash{}{0pt}%
\pgfpathmoveto{\pgfqpoint{3.317483in}{3.610973in}}%
\pgfpathlineto{\pgfqpoint{3.330383in}{3.592770in}}%
\pgfpathlineto{\pgfqpoint{3.343278in}{3.574763in}}%
\pgfpathlineto{\pgfqpoint{3.356169in}{3.556952in}}%
\pgfpathlineto{\pgfqpoint{3.369056in}{3.539335in}}%
\pgfpathlineto{\pgfqpoint{3.376656in}{3.551534in}}%
\pgfpathlineto{\pgfqpoint{3.384250in}{3.563852in}}%
\pgfpathlineto{\pgfqpoint{3.391838in}{3.576290in}}%
\pgfpathlineto{\pgfqpoint{3.399420in}{3.588848in}}%
\pgfpathlineto{\pgfqpoint{3.386547in}{3.606549in}}%
\pgfpathlineto{\pgfqpoint{3.373669in}{3.624444in}}%
\pgfpathlineto{\pgfqpoint{3.360787in}{3.642535in}}%
\pgfpathlineto{\pgfqpoint{3.347901in}{3.660823in}}%
\pgfpathlineto{\pgfqpoint{3.340305in}{3.648174in}}%
\pgfpathlineto{\pgfqpoint{3.332703in}{3.635650in}}%
\pgfpathlineto{\pgfqpoint{3.325096in}{3.623251in}}%
\pgfpathlineto{\pgfqpoint{3.317483in}{3.610973in}}%
\pgfpathclose%
\pgfusepath{fill}%
\end{pgfscope}%
\begin{pgfscope}%
\pgfpathrectangle{\pgfqpoint{1.254980in}{0.150000in}}{\pgfqpoint{5.490039in}{5.490039in}}%
\pgfusepath{clip}%
\pgfsetbuttcap%
\pgfsetroundjoin%
\definecolor{currentfill}{rgb}{0.252194,0.269783,0.531579}%
\pgfsetfillcolor{currentfill}%
\pgfsetfillopacity{0.700000}%
\pgfsetlinewidth{0.000000pt}%
\definecolor{currentstroke}{rgb}{0.000000,0.000000,0.000000}%
\pgfsetstrokecolor{currentstroke}%
\pgfsetdash{}{0pt}%
\pgfpathmoveto{\pgfqpoint{4.493838in}{2.915390in}}%
\pgfpathlineto{\pgfqpoint{4.506758in}{2.909320in}}%
\pgfpathlineto{\pgfqpoint{4.519683in}{2.903380in}}%
\pgfpathlineto{\pgfqpoint{4.532614in}{2.897571in}}%
\pgfpathlineto{\pgfqpoint{4.545551in}{2.891893in}}%
\pgfpathlineto{\pgfqpoint{4.552843in}{2.903574in}}%
\pgfpathlineto{\pgfqpoint{4.560130in}{2.915323in}}%
\pgfpathlineto{\pgfqpoint{4.567415in}{2.927139in}}%
\pgfpathlineto{\pgfqpoint{4.574696in}{2.939027in}}%
\pgfpathlineto{\pgfqpoint{4.561770in}{2.944866in}}%
\pgfpathlineto{\pgfqpoint{4.548849in}{2.950836in}}%
\pgfpathlineto{\pgfqpoint{4.535934in}{2.956936in}}%
\pgfpathlineto{\pgfqpoint{4.523024in}{2.963168in}}%
\pgfpathlineto{\pgfqpoint{4.515733in}{2.951114in}}%
\pgfpathlineto{\pgfqpoint{4.508438in}{2.939134in}}%
\pgfpathlineto{\pgfqpoint{4.501139in}{2.927227in}}%
\pgfpathlineto{\pgfqpoint{4.493838in}{2.915390in}}%
\pgfpathclose%
\pgfusepath{fill}%
\end{pgfscope}%
\begin{pgfscope}%
\pgfpathrectangle{\pgfqpoint{1.254980in}{0.150000in}}{\pgfqpoint{5.490039in}{5.490039in}}%
\pgfusepath{clip}%
\pgfsetbuttcap%
\pgfsetroundjoin%
\definecolor{currentfill}{rgb}{0.244972,0.287675,0.537260}%
\pgfsetfillcolor{currentfill}%
\pgfsetfillopacity{0.700000}%
\pgfsetlinewidth{0.000000pt}%
\definecolor{currentstroke}{rgb}{0.000000,0.000000,0.000000}%
\pgfsetstrokecolor{currentstroke}%
\pgfsetdash{}{0pt}%
\pgfpathmoveto{\pgfqpoint{4.707336in}{2.943823in}}%
\pgfpathlineto{\pgfqpoint{4.720308in}{2.938960in}}%
\pgfpathlineto{\pgfqpoint{4.733286in}{2.934222in}}%
\pgfpathlineto{\pgfqpoint{4.746271in}{2.929611in}}%
\pgfpathlineto{\pgfqpoint{4.759263in}{2.925125in}}%
\pgfpathlineto{\pgfqpoint{4.766497in}{2.936832in}}%
\pgfpathlineto{\pgfqpoint{4.773728in}{2.948609in}}%
\pgfpathlineto{\pgfqpoint{4.780957in}{2.960458in}}%
\pgfpathlineto{\pgfqpoint{4.788182in}{2.972382in}}%
\pgfpathlineto{\pgfqpoint{4.775201in}{2.977060in}}%
\pgfpathlineto{\pgfqpoint{4.762227in}{2.981864in}}%
\pgfpathlineto{\pgfqpoint{4.749259in}{2.986794in}}%
\pgfpathlineto{\pgfqpoint{4.736298in}{2.991849in}}%
\pgfpathlineto{\pgfqpoint{4.729062in}{2.979728in}}%
\pgfpathlineto{\pgfqpoint{4.721823in}{2.967684in}}%
\pgfpathlineto{\pgfqpoint{4.714581in}{2.955717in}}%
\pgfpathlineto{\pgfqpoint{4.707336in}{2.943823in}}%
\pgfpathclose%
\pgfusepath{fill}%
\end{pgfscope}%
\begin{pgfscope}%
\pgfpathrectangle{\pgfqpoint{1.254980in}{0.150000in}}{\pgfqpoint{5.490039in}{5.490039in}}%
\pgfusepath{clip}%
\pgfsetbuttcap%
\pgfsetroundjoin%
\definecolor{currentfill}{rgb}{0.221989,0.339161,0.548752}%
\pgfsetfillcolor{currentfill}%
\pgfsetfillopacity{0.700000}%
\pgfsetlinewidth{0.000000pt}%
\definecolor{currentstroke}{rgb}{0.000000,0.000000,0.000000}%
\pgfsetstrokecolor{currentstroke}%
\pgfsetdash{}{0pt}%
\pgfpathmoveto{\pgfqpoint{3.780177in}{3.071210in}}%
\pgfpathlineto{\pgfqpoint{3.793011in}{3.059374in}}%
\pgfpathlineto{\pgfqpoint{3.805846in}{3.047698in}}%
\pgfpathlineto{\pgfqpoint{3.818681in}{3.036180in}}%
\pgfpathlineto{\pgfqpoint{3.831517in}{3.024818in}}%
\pgfpathlineto{\pgfqpoint{3.839003in}{3.036323in}}%
\pgfpathlineto{\pgfqpoint{3.846485in}{3.047910in}}%
\pgfpathlineto{\pgfqpoint{3.853962in}{3.059580in}}%
\pgfpathlineto{\pgfqpoint{3.861434in}{3.071335in}}%
\pgfpathlineto{\pgfqpoint{3.848609in}{3.082779in}}%
\pgfpathlineto{\pgfqpoint{3.835786in}{3.094380in}}%
\pgfpathlineto{\pgfqpoint{3.822963in}{3.106139in}}%
\pgfpathlineto{\pgfqpoint{3.810140in}{3.118057in}}%
\pgfpathlineto{\pgfqpoint{3.802656in}{3.106214in}}%
\pgfpathlineto{\pgfqpoint{3.795168in}{3.094459in}}%
\pgfpathlineto{\pgfqpoint{3.787675in}{3.082791in}}%
\pgfpathlineto{\pgfqpoint{3.780177in}{3.071210in}}%
\pgfpathclose%
\pgfusepath{fill}%
\end{pgfscope}%
\begin{pgfscope}%
\pgfpathrectangle{\pgfqpoint{1.254980in}{0.150000in}}{\pgfqpoint{5.490039in}{5.490039in}}%
\pgfusepath{clip}%
\pgfsetbuttcap%
\pgfsetroundjoin%
\definecolor{currentfill}{rgb}{0.250425,0.274290,0.533103}%
\pgfsetfillcolor{currentfill}%
\pgfsetfillopacity{0.700000}%
\pgfsetlinewidth{0.000000pt}%
\definecolor{currentstroke}{rgb}{0.000000,0.000000,0.000000}%
\pgfsetstrokecolor{currentstroke}%
\pgfsetdash{}{0pt}%
\pgfpathmoveto{\pgfqpoint{4.147938in}{2.920218in}}%
\pgfpathlineto{\pgfqpoint{4.160800in}{2.911806in}}%
\pgfpathlineto{\pgfqpoint{4.173664in}{2.903536in}}%
\pgfpathlineto{\pgfqpoint{4.186532in}{2.895408in}}%
\pgfpathlineto{\pgfqpoint{4.199404in}{2.887420in}}%
\pgfpathlineto{\pgfqpoint{4.206791in}{2.898964in}}%
\pgfpathlineto{\pgfqpoint{4.214173in}{2.910577in}}%
\pgfpathlineto{\pgfqpoint{4.221552in}{2.922259in}}%
\pgfpathlineto{\pgfqpoint{4.228926in}{2.934013in}}%
\pgfpathlineto{\pgfqpoint{4.216065in}{2.942115in}}%
\pgfpathlineto{\pgfqpoint{4.203207in}{2.950358in}}%
\pgfpathlineto{\pgfqpoint{4.190353in}{2.958742in}}%
\pgfpathlineto{\pgfqpoint{4.177503in}{2.967267in}}%
\pgfpathlineto{\pgfqpoint{4.170117in}{2.955394in}}%
\pgfpathlineto{\pgfqpoint{4.162728in}{2.943596in}}%
\pgfpathlineto{\pgfqpoint{4.155335in}{2.931871in}}%
\pgfpathlineto{\pgfqpoint{4.147938in}{2.920218in}}%
\pgfpathclose%
\pgfusepath{fill}%
\end{pgfscope}%
\begin{pgfscope}%
\pgfpathrectangle{\pgfqpoint{1.254980in}{0.150000in}}{\pgfqpoint{5.490039in}{5.490039in}}%
\pgfusepath{clip}%
\pgfsetbuttcap%
\pgfsetroundjoin%
\definecolor{currentfill}{rgb}{0.253935,0.265254,0.529983}%
\pgfsetfillcolor{currentfill}%
\pgfsetfillopacity{0.700000}%
\pgfsetlinewidth{0.000000pt}%
\definecolor{currentstroke}{rgb}{0.000000,0.000000,0.000000}%
\pgfsetstrokecolor{currentstroke}%
\pgfsetdash{}{0pt}%
\pgfpathmoveto{\pgfqpoint{4.280411in}{2.902998in}}%
\pgfpathlineto{\pgfqpoint{4.293293in}{2.895590in}}%
\pgfpathlineto{\pgfqpoint{4.306179in}{2.888320in}}%
\pgfpathlineto{\pgfqpoint{4.319070in}{2.881186in}}%
\pgfpathlineto{\pgfqpoint{4.331965in}{2.874188in}}%
\pgfpathlineto{\pgfqpoint{4.339315in}{2.885765in}}%
\pgfpathlineto{\pgfqpoint{4.346661in}{2.897409in}}%
\pgfpathlineto{\pgfqpoint{4.354004in}{2.909119in}}%
\pgfpathlineto{\pgfqpoint{4.361344in}{2.920899in}}%
\pgfpathlineto{\pgfqpoint{4.348459in}{2.928026in}}%
\pgfpathlineto{\pgfqpoint{4.335579in}{2.935290in}}%
\pgfpathlineto{\pgfqpoint{4.322703in}{2.942690in}}%
\pgfpathlineto{\pgfqpoint{4.309832in}{2.950228in}}%
\pgfpathlineto{\pgfqpoint{4.302482in}{2.938313in}}%
\pgfpathlineto{\pgfqpoint{4.295129in}{2.926471in}}%
\pgfpathlineto{\pgfqpoint{4.287772in}{2.914700in}}%
\pgfpathlineto{\pgfqpoint{4.280411in}{2.902998in}}%
\pgfpathclose%
\pgfusepath{fill}%
\end{pgfscope}%
\begin{pgfscope}%
\pgfpathrectangle{\pgfqpoint{1.254980in}{0.150000in}}{\pgfqpoint{5.490039in}{5.490039in}}%
\pgfusepath{clip}%
\pgfsetbuttcap%
\pgfsetroundjoin%
\definecolor{currentfill}{rgb}{0.123463,0.581687,0.547445}%
\pgfsetfillcolor{currentfill}%
\pgfsetfillopacity{0.700000}%
\pgfsetlinewidth{0.000000pt}%
\definecolor{currentstroke}{rgb}{0.000000,0.000000,0.000000}%
\pgfsetstrokecolor{currentstroke}%
\pgfsetdash{}{0pt}%
\pgfpathmoveto{\pgfqpoint{3.265836in}{3.685784in}}%
\pgfpathlineto{\pgfqpoint{3.278755in}{3.666779in}}%
\pgfpathlineto{\pgfqpoint{3.291669in}{3.647977in}}%
\pgfpathlineto{\pgfqpoint{3.304579in}{3.629375in}}%
\pgfpathlineto{\pgfqpoint{3.317483in}{3.610973in}}%
\pgfpathlineto{\pgfqpoint{3.325096in}{3.623251in}}%
\pgfpathlineto{\pgfqpoint{3.332703in}{3.635650in}}%
\pgfpathlineto{\pgfqpoint{3.340305in}{3.648174in}}%
\pgfpathlineto{\pgfqpoint{3.347901in}{3.660823in}}%
\pgfpathlineto{\pgfqpoint{3.335011in}{3.679309in}}%
\pgfpathlineto{\pgfqpoint{3.322115in}{3.697994in}}%
\pgfpathlineto{\pgfqpoint{3.309215in}{3.716881in}}%
\pgfpathlineto{\pgfqpoint{3.296310in}{3.735970in}}%
\pgfpathlineto{\pgfqpoint{3.288700in}{3.723231in}}%
\pgfpathlineto{\pgfqpoint{3.281085in}{3.710621in}}%
\pgfpathlineto{\pgfqpoint{3.273464in}{3.698139in}}%
\pgfpathlineto{\pgfqpoint{3.265836in}{3.685784in}}%
\pgfpathclose%
\pgfusepath{fill}%
\end{pgfscope}%
\begin{pgfscope}%
\pgfpathrectangle{\pgfqpoint{1.254980in}{0.150000in}}{\pgfqpoint{5.490039in}{5.490039in}}%
\pgfusepath{clip}%
\pgfsetbuttcap%
\pgfsetroundjoin%
\definecolor{currentfill}{rgb}{0.229739,0.322361,0.545706}%
\pgfsetfillcolor{currentfill}%
\pgfsetfillopacity{0.700000}%
\pgfsetlinewidth{0.000000pt}%
\definecolor{currentstroke}{rgb}{0.000000,0.000000,0.000000}%
\pgfsetstrokecolor{currentstroke}%
\pgfsetdash{}{0pt}%
\pgfpathmoveto{\pgfqpoint{5.001956in}{3.019007in}}%
\pgfpathlineto{\pgfqpoint{5.015010in}{3.015492in}}%
\pgfpathlineto{\pgfqpoint{5.028072in}{3.012097in}}%
\pgfpathlineto{\pgfqpoint{5.041143in}{3.008822in}}%
\pgfpathlineto{\pgfqpoint{5.054222in}{3.005667in}}%
\pgfpathlineto{\pgfqpoint{5.061378in}{3.017422in}}%
\pgfpathlineto{\pgfqpoint{5.068532in}{3.029258in}}%
\pgfpathlineto{\pgfqpoint{5.075684in}{3.041178in}}%
\pgfpathlineto{\pgfqpoint{5.062614in}{3.044511in}}%
\pgfpathlineto{\pgfqpoint{5.049553in}{3.047964in}}%
\pgfpathlineto{\pgfqpoint{5.036500in}{3.051537in}}%
\pgfpathlineto{\pgfqpoint{5.023455in}{3.055230in}}%
\pgfpathlineto{\pgfqpoint{5.016290in}{3.043069in}}%
\pgfpathlineto{\pgfqpoint{5.009124in}{3.030996in}}%
\pgfpathlineto{\pgfqpoint{5.001956in}{3.019007in}}%
\pgfpathclose%
\pgfusepath{fill}%
\end{pgfscope}%
\begin{pgfscope}%
\pgfpathrectangle{\pgfqpoint{1.254980in}{0.150000in}}{\pgfqpoint{5.490039in}{5.490039in}}%
\pgfusepath{clip}%
\pgfsetbuttcap%
\pgfsetroundjoin%
\definecolor{currentfill}{rgb}{0.244972,0.287675,0.537260}%
\pgfsetfillcolor{currentfill}%
\pgfsetfillopacity{0.700000}%
\pgfsetlinewidth{0.000000pt}%
\definecolor{currentstroke}{rgb}{0.000000,0.000000,0.000000}%
\pgfsetstrokecolor{currentstroke}%
\pgfsetdash{}{0pt}%
\pgfpathmoveto{\pgfqpoint{4.015449in}{2.945985in}}%
\pgfpathlineto{\pgfqpoint{4.028297in}{2.936515in}}%
\pgfpathlineto{\pgfqpoint{4.041147in}{2.927192in}}%
\pgfpathlineto{\pgfqpoint{4.054000in}{2.918015in}}%
\pgfpathlineto{\pgfqpoint{4.066857in}{2.908984in}}%
\pgfpathlineto{\pgfqpoint{4.074280in}{2.920454in}}%
\pgfpathlineto{\pgfqpoint{4.081700in}{2.931995in}}%
\pgfpathlineto{\pgfqpoint{4.089115in}{2.943608in}}%
\pgfpathlineto{\pgfqpoint{4.096527in}{2.955294in}}%
\pgfpathlineto{\pgfqpoint{4.083682in}{2.964424in}}%
\pgfpathlineto{\pgfqpoint{4.070839in}{2.973699in}}%
\pgfpathlineto{\pgfqpoint{4.058000in}{2.983121in}}%
\pgfpathlineto{\pgfqpoint{4.045163in}{2.992689in}}%
\pgfpathlineto{\pgfqpoint{4.037740in}{2.980898in}}%
\pgfpathlineto{\pgfqpoint{4.030314in}{2.969185in}}%
\pgfpathlineto{\pgfqpoint{4.022883in}{2.957547in}}%
\pgfpathlineto{\pgfqpoint{4.015449in}{2.945985in}}%
\pgfpathclose%
\pgfusepath{fill}%
\end{pgfscope}%
\begin{pgfscope}%
\pgfpathrectangle{\pgfqpoint{1.254980in}{0.150000in}}{\pgfqpoint{5.490039in}{5.490039in}}%
\pgfusepath{clip}%
\pgfsetbuttcap%
\pgfsetroundjoin%
\definecolor{currentfill}{rgb}{0.250425,0.274290,0.533103}%
\pgfsetfillcolor{currentfill}%
\pgfsetfillopacity{0.700000}%
\pgfsetlinewidth{0.000000pt}%
\definecolor{currentstroke}{rgb}{0.000000,0.000000,0.000000}%
\pgfsetstrokecolor{currentstroke}%
\pgfsetdash{}{0pt}%
\pgfpathmoveto{\pgfqpoint{4.626462in}{2.916962in}}%
\pgfpathlineto{\pgfqpoint{4.639418in}{2.911768in}}%
\pgfpathlineto{\pgfqpoint{4.652381in}{2.906701in}}%
\pgfpathlineto{\pgfqpoint{4.665351in}{2.901761in}}%
\pgfpathlineto{\pgfqpoint{4.678327in}{2.896948in}}%
\pgfpathlineto{\pgfqpoint{4.685584in}{2.908565in}}%
\pgfpathlineto{\pgfqpoint{4.692837in}{2.920249in}}%
\pgfpathlineto{\pgfqpoint{4.700088in}{2.932001in}}%
\pgfpathlineto{\pgfqpoint{4.707336in}{2.943823in}}%
\pgfpathlineto{\pgfqpoint{4.694371in}{2.948812in}}%
\pgfpathlineto{\pgfqpoint{4.681412in}{2.953929in}}%
\pgfpathlineto{\pgfqpoint{4.668460in}{2.959173in}}%
\pgfpathlineto{\pgfqpoint{4.655514in}{2.964544in}}%
\pgfpathlineto{\pgfqpoint{4.648255in}{2.952540in}}%
\pgfpathlineto{\pgfqpoint{4.640994in}{2.940609in}}%
\pgfpathlineto{\pgfqpoint{4.633729in}{2.928751in}}%
\pgfpathlineto{\pgfqpoint{4.626462in}{2.916962in}}%
\pgfpathclose%
\pgfusepath{fill}%
\end{pgfscope}%
\begin{pgfscope}%
\pgfpathrectangle{\pgfqpoint{1.254980in}{0.150000in}}{\pgfqpoint{5.490039in}{5.490039in}}%
\pgfusepath{clip}%
\pgfsetbuttcap%
\pgfsetroundjoin%
\definecolor{currentfill}{rgb}{0.229739,0.322361,0.545706}%
\pgfsetfillcolor{currentfill}%
\pgfsetfillopacity{0.700000}%
\pgfsetlinewidth{0.000000pt}%
\definecolor{currentstroke}{rgb}{0.000000,0.000000,0.000000}%
\pgfsetstrokecolor{currentstroke}%
\pgfsetdash{}{0pt}%
\pgfpathmoveto{\pgfqpoint{3.831517in}{3.024818in}}%
\pgfpathlineto{\pgfqpoint{3.844355in}{3.013614in}}%
\pgfpathlineto{\pgfqpoint{3.857193in}{3.002564in}}%
\pgfpathlineto{\pgfqpoint{3.870033in}{2.991670in}}%
\pgfpathlineto{\pgfqpoint{3.882874in}{2.980930in}}%
\pgfpathlineto{\pgfqpoint{3.890349in}{2.992357in}}%
\pgfpathlineto{\pgfqpoint{3.897819in}{3.003863in}}%
\pgfpathlineto{\pgfqpoint{3.905284in}{3.015448in}}%
\pgfpathlineto{\pgfqpoint{3.912745in}{3.027114in}}%
\pgfpathlineto{\pgfqpoint{3.899915in}{3.037938in}}%
\pgfpathlineto{\pgfqpoint{3.887087in}{3.048915in}}%
\pgfpathlineto{\pgfqpoint{3.874260in}{3.060047in}}%
\pgfpathlineto{\pgfqpoint{3.861434in}{3.071335in}}%
\pgfpathlineto{\pgfqpoint{3.853962in}{3.059580in}}%
\pgfpathlineto{\pgfqpoint{3.846485in}{3.047910in}}%
\pgfpathlineto{\pgfqpoint{3.839003in}{3.036323in}}%
\pgfpathlineto{\pgfqpoint{3.831517in}{3.024818in}}%
\pgfpathclose%
\pgfusepath{fill}%
\end{pgfscope}%
\begin{pgfscope}%
\pgfpathrectangle{\pgfqpoint{1.254980in}{0.150000in}}{\pgfqpoint{5.490039in}{5.490039in}}%
\pgfusepath{clip}%
\pgfsetbuttcap%
\pgfsetroundjoin%
\definecolor{currentfill}{rgb}{0.235526,0.309527,0.542944}%
\pgfsetfillcolor{currentfill}%
\pgfsetfillopacity{0.700000}%
\pgfsetlinewidth{0.000000pt}%
\definecolor{currentstroke}{rgb}{0.000000,0.000000,0.000000}%
\pgfsetstrokecolor{currentstroke}%
\pgfsetdash{}{0pt}%
\pgfpathmoveto{\pgfqpoint{4.921072in}{2.986230in}}%
\pgfpathlineto{\pgfqpoint{4.934107in}{2.982454in}}%
\pgfpathlineto{\pgfqpoint{4.947150in}{2.978800in}}%
\pgfpathlineto{\pgfqpoint{4.960200in}{2.975267in}}%
\pgfpathlineto{\pgfqpoint{4.973259in}{2.971855in}}%
\pgfpathlineto{\pgfqpoint{4.980437in}{2.983527in}}%
\pgfpathlineto{\pgfqpoint{4.987612in}{2.995275in}}%
\pgfpathlineto{\pgfqpoint{4.994785in}{3.007101in}}%
\pgfpathlineto{\pgfqpoint{5.001956in}{3.019007in}}%
\pgfpathlineto{\pgfqpoint{4.988909in}{3.022643in}}%
\pgfpathlineto{\pgfqpoint{4.975870in}{3.026400in}}%
\pgfpathlineto{\pgfqpoint{4.962839in}{3.030278in}}%
\pgfpathlineto{\pgfqpoint{4.949816in}{3.034278in}}%
\pgfpathlineto{\pgfqpoint{4.942634in}{3.022142in}}%
\pgfpathlineto{\pgfqpoint{4.935449in}{3.010090in}}%
\pgfpathlineto{\pgfqpoint{4.928261in}{2.998121in}}%
\pgfpathlineto{\pgfqpoint{4.921072in}{2.986230in}}%
\pgfpathclose%
\pgfusepath{fill}%
\end{pgfscope}%
\begin{pgfscope}%
\pgfpathrectangle{\pgfqpoint{1.254980in}{0.150000in}}{\pgfqpoint{5.490039in}{5.490039in}}%
\pgfusepath{clip}%
\pgfsetbuttcap%
\pgfsetroundjoin%
\definecolor{currentfill}{rgb}{0.253935,0.265254,0.529983}%
\pgfsetfillcolor{currentfill}%
\pgfsetfillopacity{0.700000}%
\pgfsetlinewidth{0.000000pt}%
\definecolor{currentstroke}{rgb}{0.000000,0.000000,0.000000}%
\pgfsetstrokecolor{currentstroke}%
\pgfsetdash{}{0pt}%
\pgfpathmoveto{\pgfqpoint{4.412930in}{2.893739in}}%
\pgfpathlineto{\pgfqpoint{4.425839in}{2.887284in}}%
\pgfpathlineto{\pgfqpoint{4.438753in}{2.880963in}}%
\pgfpathlineto{\pgfqpoint{4.451673in}{2.874774in}}%
\pgfpathlineto{\pgfqpoint{4.464598in}{2.868717in}}%
\pgfpathlineto{\pgfqpoint{4.471913in}{2.880288in}}%
\pgfpathlineto{\pgfqpoint{4.479225in}{2.891922in}}%
\pgfpathlineto{\pgfqpoint{4.486533in}{2.903623in}}%
\pgfpathlineto{\pgfqpoint{4.493838in}{2.915390in}}%
\pgfpathlineto{\pgfqpoint{4.480923in}{2.921592in}}%
\pgfpathlineto{\pgfqpoint{4.468014in}{2.927927in}}%
\pgfpathlineto{\pgfqpoint{4.455110in}{2.934394in}}%
\pgfpathlineto{\pgfqpoint{4.442212in}{2.940994in}}%
\pgfpathlineto{\pgfqpoint{4.434896in}{2.929075in}}%
\pgfpathlineto{\pgfqpoint{4.427578in}{2.917227in}}%
\pgfpathlineto{\pgfqpoint{4.420256in}{2.905449in}}%
\pgfpathlineto{\pgfqpoint{4.412930in}{2.893739in}}%
\pgfpathclose%
\pgfusepath{fill}%
\end{pgfscope}%
\begin{pgfscope}%
\pgfpathrectangle{\pgfqpoint{1.254980in}{0.150000in}}{\pgfqpoint{5.490039in}{5.490039in}}%
\pgfusepath{clip}%
\pgfsetbuttcap%
\pgfsetroundjoin%
\definecolor{currentfill}{rgb}{0.179019,0.433756,0.557430}%
\pgfsetfillcolor{currentfill}%
\pgfsetfillopacity{0.700000}%
\pgfsetlinewidth{0.000000pt}%
\definecolor{currentstroke}{rgb}{0.000000,0.000000,0.000000}%
\pgfsetstrokecolor{currentstroke}%
\pgfsetdash{}{0pt}%
\pgfpathmoveto{\pgfqpoint{3.493149in}{3.295812in}}%
\pgfpathlineto{\pgfqpoint{3.506010in}{3.280677in}}%
\pgfpathlineto{\pgfqpoint{3.518869in}{3.265720in}}%
\pgfpathlineto{\pgfqpoint{3.531727in}{3.250939in}}%
\pgfpathlineto{\pgfqpoint{3.544582in}{3.236334in}}%
\pgfpathlineto{\pgfqpoint{3.552149in}{3.247824in}}%
\pgfpathlineto{\pgfqpoint{3.559711in}{3.259412in}}%
\pgfpathlineto{\pgfqpoint{3.567268in}{3.271100in}}%
\pgfpathlineto{\pgfqpoint{3.574820in}{3.282888in}}%
\pgfpathlineto{\pgfqpoint{3.561977in}{3.297560in}}%
\pgfpathlineto{\pgfqpoint{3.549133in}{3.312408in}}%
\pgfpathlineto{\pgfqpoint{3.536287in}{3.327432in}}%
\pgfpathlineto{\pgfqpoint{3.523439in}{3.342635in}}%
\pgfpathlineto{\pgfqpoint{3.515875in}{3.330773in}}%
\pgfpathlineto{\pgfqpoint{3.508305in}{3.319016in}}%
\pgfpathlineto{\pgfqpoint{3.500730in}{3.307363in}}%
\pgfpathlineto{\pgfqpoint{3.493149in}{3.295812in}}%
\pgfpathclose%
\pgfusepath{fill}%
\end{pgfscope}%
\begin{pgfscope}%
\pgfpathrectangle{\pgfqpoint{1.254980in}{0.150000in}}{\pgfqpoint{5.490039in}{5.490039in}}%
\pgfusepath{clip}%
\pgfsetbuttcap%
\pgfsetroundjoin%
\definecolor{currentfill}{rgb}{0.188923,0.410910,0.556326}%
\pgfsetfillcolor{currentfill}%
\pgfsetfillopacity{0.700000}%
\pgfsetlinewidth{0.000000pt}%
\definecolor{currentstroke}{rgb}{0.000000,0.000000,0.000000}%
\pgfsetstrokecolor{currentstroke}%
\pgfsetdash{}{0pt}%
\pgfpathmoveto{\pgfqpoint{3.544582in}{3.236334in}}%
\pgfpathlineto{\pgfqpoint{3.557436in}{3.221904in}}%
\pgfpathlineto{\pgfqpoint{3.570288in}{3.207647in}}%
\pgfpathlineto{\pgfqpoint{3.583139in}{3.193562in}}%
\pgfpathlineto{\pgfqpoint{3.595989in}{3.179649in}}%
\pgfpathlineto{\pgfqpoint{3.603543in}{3.191078in}}%
\pgfpathlineto{\pgfqpoint{3.611092in}{3.202601in}}%
\pgfpathlineto{\pgfqpoint{3.618636in}{3.214220in}}%
\pgfpathlineto{\pgfqpoint{3.626175in}{3.225935in}}%
\pgfpathlineto{\pgfqpoint{3.613338in}{3.239915in}}%
\pgfpathlineto{\pgfqpoint{3.600500in}{3.254066in}}%
\pgfpathlineto{\pgfqpoint{3.587661in}{3.268390in}}%
\pgfpathlineto{\pgfqpoint{3.574820in}{3.282888in}}%
\pgfpathlineto{\pgfqpoint{3.567268in}{3.271100in}}%
\pgfpathlineto{\pgfqpoint{3.559711in}{3.259412in}}%
\pgfpathlineto{\pgfqpoint{3.552149in}{3.247824in}}%
\pgfpathlineto{\pgfqpoint{3.544582in}{3.236334in}}%
\pgfpathclose%
\pgfusepath{fill}%
\end{pgfscope}%
\begin{pgfscope}%
\pgfpathrectangle{\pgfqpoint{1.254980in}{0.150000in}}{\pgfqpoint{5.490039in}{5.490039in}}%
\pgfusepath{clip}%
\pgfsetbuttcap%
\pgfsetroundjoin%
\definecolor{currentfill}{rgb}{0.119423,0.611141,0.538982}%
\pgfsetfillcolor{currentfill}%
\pgfsetfillopacity{0.700000}%
\pgfsetlinewidth{0.000000pt}%
\definecolor{currentstroke}{rgb}{0.000000,0.000000,0.000000}%
\pgfsetstrokecolor{currentstroke}%
\pgfsetdash{}{0pt}%
\pgfpathmoveto{\pgfqpoint{3.214106in}{3.763854in}}%
\pgfpathlineto{\pgfqpoint{3.227047in}{3.744026in}}%
\pgfpathlineto{\pgfqpoint{3.239982in}{3.724406in}}%
\pgfpathlineto{\pgfqpoint{3.252912in}{3.704992in}}%
\pgfpathlineto{\pgfqpoint{3.265836in}{3.685784in}}%
\pgfpathlineto{\pgfqpoint{3.273464in}{3.698139in}}%
\pgfpathlineto{\pgfqpoint{3.281085in}{3.710621in}}%
\pgfpathlineto{\pgfqpoint{3.288700in}{3.723231in}}%
\pgfpathlineto{\pgfqpoint{3.296310in}{3.735970in}}%
\pgfpathlineto{\pgfqpoint{3.283400in}{3.755263in}}%
\pgfpathlineto{\pgfqpoint{3.270484in}{3.774761in}}%
\pgfpathlineto{\pgfqpoint{3.257563in}{3.794466in}}%
\pgfpathlineto{\pgfqpoint{3.244637in}{3.814379in}}%
\pgfpathlineto{\pgfqpoint{3.237013in}{3.801549in}}%
\pgfpathlineto{\pgfqpoint{3.229383in}{3.788852in}}%
\pgfpathlineto{\pgfqpoint{3.221748in}{3.776287in}}%
\pgfpathlineto{\pgfqpoint{3.214106in}{3.763854in}}%
\pgfpathclose%
\pgfusepath{fill}%
\end{pgfscope}%
\begin{pgfscope}%
\pgfpathrectangle{\pgfqpoint{1.254980in}{0.150000in}}{\pgfqpoint{5.490039in}{5.490039in}}%
\pgfusepath{clip}%
\pgfsetbuttcap%
\pgfsetroundjoin%
\definecolor{currentfill}{rgb}{0.169646,0.456262,0.558030}%
\pgfsetfillcolor{currentfill}%
\pgfsetfillopacity{0.700000}%
\pgfsetlinewidth{0.000000pt}%
\definecolor{currentstroke}{rgb}{0.000000,0.000000,0.000000}%
\pgfsetstrokecolor{currentstroke}%
\pgfsetdash{}{0pt}%
\pgfpathmoveto{\pgfqpoint{3.441681in}{3.358151in}}%
\pgfpathlineto{\pgfqpoint{3.454552in}{3.342294in}}%
\pgfpathlineto{\pgfqpoint{3.467420in}{3.326620in}}%
\pgfpathlineto{\pgfqpoint{3.480286in}{3.311126in}}%
\pgfpathlineto{\pgfqpoint{3.493149in}{3.295812in}}%
\pgfpathlineto{\pgfqpoint{3.500730in}{3.307363in}}%
\pgfpathlineto{\pgfqpoint{3.508305in}{3.319016in}}%
\pgfpathlineto{\pgfqpoint{3.515875in}{3.330773in}}%
\pgfpathlineto{\pgfqpoint{3.523439in}{3.342635in}}%
\pgfpathlineto{\pgfqpoint{3.510589in}{3.358016in}}%
\pgfpathlineto{\pgfqpoint{3.497737in}{3.373577in}}%
\pgfpathlineto{\pgfqpoint{3.484882in}{3.389319in}}%
\pgfpathlineto{\pgfqpoint{3.472025in}{3.405243in}}%
\pgfpathlineto{\pgfqpoint{3.464447in}{3.393308in}}%
\pgfpathlineto{\pgfqpoint{3.456864in}{3.381482in}}%
\pgfpathlineto{\pgfqpoint{3.449275in}{3.369763in}}%
\pgfpathlineto{\pgfqpoint{3.441681in}{3.358151in}}%
\pgfpathclose%
\pgfusepath{fill}%
\end{pgfscope}%
\begin{pgfscope}%
\pgfpathrectangle{\pgfqpoint{1.254980in}{0.150000in}}{\pgfqpoint{5.490039in}{5.490039in}}%
\pgfusepath{clip}%
\pgfsetbuttcap%
\pgfsetroundjoin%
\definecolor{currentfill}{rgb}{0.241237,0.296485,0.539709}%
\pgfsetfillcolor{currentfill}%
\pgfsetfillopacity{0.700000}%
\pgfsetlinewidth{0.000000pt}%
\definecolor{currentstroke}{rgb}{0.000000,0.000000,0.000000}%
\pgfsetstrokecolor{currentstroke}%
\pgfsetdash{}{0pt}%
\pgfpathmoveto{\pgfqpoint{4.840176in}{2.954912in}}%
\pgfpathlineto{\pgfqpoint{4.853193in}{2.950854in}}%
\pgfpathlineto{\pgfqpoint{4.866217in}{2.946919in}}%
\pgfpathlineto{\pgfqpoint{4.879248in}{2.943107in}}%
\pgfpathlineto{\pgfqpoint{4.892287in}{2.939417in}}%
\pgfpathlineto{\pgfqpoint{4.899488in}{2.951012in}}%
\pgfpathlineto{\pgfqpoint{4.906685in}{2.962678in}}%
\pgfpathlineto{\pgfqpoint{4.913880in}{2.974417in}}%
\pgfpathlineto{\pgfqpoint{4.921072in}{2.986230in}}%
\pgfpathlineto{\pgfqpoint{4.908044in}{2.990128in}}%
\pgfpathlineto{\pgfqpoint{4.895024in}{2.994148in}}%
\pgfpathlineto{\pgfqpoint{4.882012in}{2.998291in}}%
\pgfpathlineto{\pgfqpoint{4.869007in}{3.002557in}}%
\pgfpathlineto{\pgfqpoint{4.861803in}{2.990530in}}%
\pgfpathlineto{\pgfqpoint{4.854597in}{2.978582in}}%
\pgfpathlineto{\pgfqpoint{4.847388in}{2.966710in}}%
\pgfpathlineto{\pgfqpoint{4.840176in}{2.954912in}}%
\pgfpathclose%
\pgfusepath{fill}%
\end{pgfscope}%
\begin{pgfscope}%
\pgfpathrectangle{\pgfqpoint{1.254980in}{0.150000in}}{\pgfqpoint{5.490039in}{5.490039in}}%
\pgfusepath{clip}%
\pgfsetbuttcap%
\pgfsetroundjoin%
\definecolor{currentfill}{rgb}{0.199430,0.387607,0.554642}%
\pgfsetfillcolor{currentfill}%
\pgfsetfillopacity{0.700000}%
\pgfsetlinewidth{0.000000pt}%
\definecolor{currentstroke}{rgb}{0.000000,0.000000,0.000000}%
\pgfsetstrokecolor{currentstroke}%
\pgfsetdash{}{0pt}%
\pgfpathmoveto{\pgfqpoint{3.595989in}{3.179649in}}%
\pgfpathlineto{\pgfqpoint{3.608837in}{3.165907in}}%
\pgfpathlineto{\pgfqpoint{3.621685in}{3.152335in}}%
\pgfpathlineto{\pgfqpoint{3.634532in}{3.138930in}}%
\pgfpathlineto{\pgfqpoint{3.647378in}{3.125694in}}%
\pgfpathlineto{\pgfqpoint{3.654920in}{3.137062in}}%
\pgfpathlineto{\pgfqpoint{3.662456in}{3.148520in}}%
\pgfpathlineto{\pgfqpoint{3.669988in}{3.160069in}}%
\pgfpathlineto{\pgfqpoint{3.677514in}{3.171711in}}%
\pgfpathlineto{\pgfqpoint{3.664680in}{3.185015in}}%
\pgfpathlineto{\pgfqpoint{3.651846in}{3.198486in}}%
\pgfpathlineto{\pgfqpoint{3.639011in}{3.212126in}}%
\pgfpathlineto{\pgfqpoint{3.626175in}{3.225935in}}%
\pgfpathlineto{\pgfqpoint{3.618636in}{3.214220in}}%
\pgfpathlineto{\pgfqpoint{3.611092in}{3.202601in}}%
\pgfpathlineto{\pgfqpoint{3.603543in}{3.191078in}}%
\pgfpathlineto{\pgfqpoint{3.595989in}{3.179649in}}%
\pgfpathclose%
\pgfusepath{fill}%
\end{pgfscope}%
\begin{pgfscope}%
\pgfpathrectangle{\pgfqpoint{1.254980in}{0.150000in}}{\pgfqpoint{5.490039in}{5.490039in}}%
\pgfusepath{clip}%
\pgfsetbuttcap%
\pgfsetroundjoin%
\definecolor{currentfill}{rgb}{0.159194,0.482237,0.558073}%
\pgfsetfillcolor{currentfill}%
\pgfsetfillopacity{0.700000}%
\pgfsetlinewidth{0.000000pt}%
\definecolor{currentstroke}{rgb}{0.000000,0.000000,0.000000}%
\pgfsetstrokecolor{currentstroke}%
\pgfsetdash{}{0pt}%
\pgfpathmoveto{\pgfqpoint{3.390169in}{3.423424in}}%
\pgfpathlineto{\pgfqpoint{3.403052in}{3.406827in}}%
\pgfpathlineto{\pgfqpoint{3.415931in}{3.390416in}}%
\pgfpathlineto{\pgfqpoint{3.428808in}{3.374192in}}%
\pgfpathlineto{\pgfqpoint{3.441681in}{3.358151in}}%
\pgfpathlineto{\pgfqpoint{3.449275in}{3.369763in}}%
\pgfpathlineto{\pgfqpoint{3.456864in}{3.381482in}}%
\pgfpathlineto{\pgfqpoint{3.464447in}{3.393308in}}%
\pgfpathlineto{\pgfqpoint{3.472025in}{3.405243in}}%
\pgfpathlineto{\pgfqpoint{3.459165in}{3.421350in}}%
\pgfpathlineto{\pgfqpoint{3.446302in}{3.437642in}}%
\pgfpathlineto{\pgfqpoint{3.433436in}{3.454120in}}%
\pgfpathlineto{\pgfqpoint{3.420567in}{3.470785in}}%
\pgfpathlineto{\pgfqpoint{3.412976in}{3.458776in}}%
\pgfpathlineto{\pgfqpoint{3.405379in}{3.446881in}}%
\pgfpathlineto{\pgfqpoint{3.397777in}{3.435097in}}%
\pgfpathlineto{\pgfqpoint{3.390169in}{3.423424in}}%
\pgfpathclose%
\pgfusepath{fill}%
\end{pgfscope}%
\begin{pgfscope}%
\pgfpathrectangle{\pgfqpoint{1.254980in}{0.150000in}}{\pgfqpoint{5.490039in}{5.490039in}}%
\pgfusepath{clip}%
\pgfsetbuttcap%
\pgfsetroundjoin%
\definecolor{currentfill}{rgb}{0.237441,0.305202,0.541921}%
\pgfsetfillcolor{currentfill}%
\pgfsetfillopacity{0.700000}%
\pgfsetlinewidth{0.000000pt}%
\definecolor{currentstroke}{rgb}{0.000000,0.000000,0.000000}%
\pgfsetstrokecolor{currentstroke}%
\pgfsetdash{}{0pt}%
\pgfpathmoveto{\pgfqpoint{3.882874in}{2.980930in}}%
\pgfpathlineto{\pgfqpoint{3.895717in}{2.970342in}}%
\pgfpathlineto{\pgfqpoint{3.908561in}{2.959908in}}%
\pgfpathlineto{\pgfqpoint{3.921407in}{2.949625in}}%
\pgfpathlineto{\pgfqpoint{3.934255in}{2.939493in}}%
\pgfpathlineto{\pgfqpoint{3.941718in}{2.950844in}}%
\pgfpathlineto{\pgfqpoint{3.949177in}{2.962269in}}%
\pgfpathlineto{\pgfqpoint{3.956631in}{2.973770in}}%
\pgfpathlineto{\pgfqpoint{3.964081in}{2.985347in}}%
\pgfpathlineto{\pgfqpoint{3.951244in}{2.995561in}}%
\pgfpathlineto{\pgfqpoint{3.938409in}{3.005927in}}%
\pgfpathlineto{\pgfqpoint{3.925576in}{3.016444in}}%
\pgfpathlineto{\pgfqpoint{3.912745in}{3.027114in}}%
\pgfpathlineto{\pgfqpoint{3.905284in}{3.015448in}}%
\pgfpathlineto{\pgfqpoint{3.897819in}{3.003863in}}%
\pgfpathlineto{\pgfqpoint{3.890349in}{2.992357in}}%
\pgfpathlineto{\pgfqpoint{3.882874in}{2.980930in}}%
\pgfpathclose%
\pgfusepath{fill}%
\end{pgfscope}%
\begin{pgfscope}%
\pgfpathrectangle{\pgfqpoint{1.254980in}{0.150000in}}{\pgfqpoint{5.490039in}{5.490039in}}%
\pgfusepath{clip}%
\pgfsetbuttcap%
\pgfsetroundjoin%
\definecolor{currentfill}{rgb}{0.253935,0.265254,0.529983}%
\pgfsetfillcolor{currentfill}%
\pgfsetfillopacity{0.700000}%
\pgfsetlinewidth{0.000000pt}%
\definecolor{currentstroke}{rgb}{0.000000,0.000000,0.000000}%
\pgfsetstrokecolor{currentstroke}%
\pgfsetdash{}{0pt}%
\pgfpathmoveto{\pgfqpoint{4.545551in}{2.891893in}}%
\pgfpathlineto{\pgfqpoint{4.558494in}{2.886344in}}%
\pgfpathlineto{\pgfqpoint{4.571443in}{2.880925in}}%
\pgfpathlineto{\pgfqpoint{4.584398in}{2.875635in}}%
\pgfpathlineto{\pgfqpoint{4.597359in}{2.870473in}}%
\pgfpathlineto{\pgfqpoint{4.604639in}{2.881999in}}%
\pgfpathlineto{\pgfqpoint{4.611917in}{2.893588in}}%
\pgfpathlineto{\pgfqpoint{4.619191in}{2.905242in}}%
\pgfpathlineto{\pgfqpoint{4.626462in}{2.916962in}}%
\pgfpathlineto{\pgfqpoint{4.613511in}{2.922285in}}%
\pgfpathlineto{\pgfqpoint{4.600567in}{2.927737in}}%
\pgfpathlineto{\pgfqpoint{4.587629in}{2.933317in}}%
\pgfpathlineto{\pgfqpoint{4.574696in}{2.939027in}}%
\pgfpathlineto{\pgfqpoint{4.567415in}{2.927139in}}%
\pgfpathlineto{\pgfqpoint{4.560130in}{2.915323in}}%
\pgfpathlineto{\pgfqpoint{4.552843in}{2.903574in}}%
\pgfpathlineto{\pgfqpoint{4.545551in}{2.891893in}}%
\pgfpathclose%
\pgfusepath{fill}%
\end{pgfscope}%
\begin{pgfscope}%
\pgfpathrectangle{\pgfqpoint{1.254980in}{0.150000in}}{\pgfqpoint{5.490039in}{5.490039in}}%
\pgfusepath{clip}%
\pgfsetbuttcap%
\pgfsetroundjoin%
\definecolor{currentfill}{rgb}{0.255645,0.260703,0.528312}%
\pgfsetfillcolor{currentfill}%
\pgfsetfillopacity{0.700000}%
\pgfsetlinewidth{0.000000pt}%
\definecolor{currentstroke}{rgb}{0.000000,0.000000,0.000000}%
\pgfsetstrokecolor{currentstroke}%
\pgfsetdash{}{0pt}%
\pgfpathmoveto{\pgfqpoint{4.199404in}{2.887420in}}%
\pgfpathlineto{\pgfqpoint{4.212280in}{2.879572in}}%
\pgfpathlineto{\pgfqpoint{4.225160in}{2.871863in}}%
\pgfpathlineto{\pgfqpoint{4.238043in}{2.864294in}}%
\pgfpathlineto{\pgfqpoint{4.250931in}{2.856862in}}%
\pgfpathlineto{\pgfqpoint{4.258307in}{2.868298in}}%
\pgfpathlineto{\pgfqpoint{4.265679in}{2.879799in}}%
\pgfpathlineto{\pgfqpoint{4.273047in}{2.891365in}}%
\pgfpathlineto{\pgfqpoint{4.280411in}{2.902998in}}%
\pgfpathlineto{\pgfqpoint{4.267534in}{2.910544in}}%
\pgfpathlineto{\pgfqpoint{4.254661in}{2.918228in}}%
\pgfpathlineto{\pgfqpoint{4.241792in}{2.926050in}}%
\pgfpathlineto{\pgfqpoint{4.228926in}{2.934013in}}%
\pgfpathlineto{\pgfqpoint{4.221552in}{2.922259in}}%
\pgfpathlineto{\pgfqpoint{4.214173in}{2.910577in}}%
\pgfpathlineto{\pgfqpoint{4.206791in}{2.898964in}}%
\pgfpathlineto{\pgfqpoint{4.199404in}{2.887420in}}%
\pgfpathclose%
\pgfusepath{fill}%
\end{pgfscope}%
\begin{pgfscope}%
\pgfpathrectangle{\pgfqpoint{1.254980in}{0.150000in}}{\pgfqpoint{5.490039in}{5.490039in}}%
\pgfusepath{clip}%
\pgfsetbuttcap%
\pgfsetroundjoin%
\definecolor{currentfill}{rgb}{0.210503,0.363727,0.552206}%
\pgfsetfillcolor{currentfill}%
\pgfsetfillopacity{0.700000}%
\pgfsetlinewidth{0.000000pt}%
\definecolor{currentstroke}{rgb}{0.000000,0.000000,0.000000}%
\pgfsetstrokecolor{currentstroke}%
\pgfsetdash{}{0pt}%
\pgfpathmoveto{\pgfqpoint{3.647378in}{3.125694in}}%
\pgfpathlineto{\pgfqpoint{3.660224in}{3.112624in}}%
\pgfpathlineto{\pgfqpoint{3.673069in}{3.099721in}}%
\pgfpathlineto{\pgfqpoint{3.685914in}{3.086982in}}%
\pgfpathlineto{\pgfqpoint{3.698759in}{3.074407in}}%
\pgfpathlineto{\pgfqpoint{3.706288in}{3.085713in}}%
\pgfpathlineto{\pgfqpoint{3.713812in}{3.097106in}}%
\pgfpathlineto{\pgfqpoint{3.721331in}{3.108587in}}%
\pgfpathlineto{\pgfqpoint{3.728845in}{3.120156in}}%
\pgfpathlineto{\pgfqpoint{3.716013in}{3.132798in}}%
\pgfpathlineto{\pgfqpoint{3.703180in}{3.145604in}}%
\pgfpathlineto{\pgfqpoint{3.690347in}{3.158575in}}%
\pgfpathlineto{\pgfqpoint{3.677514in}{3.171711in}}%
\pgfpathlineto{\pgfqpoint{3.669988in}{3.160069in}}%
\pgfpathlineto{\pgfqpoint{3.662456in}{3.148520in}}%
\pgfpathlineto{\pgfqpoint{3.654920in}{3.137062in}}%
\pgfpathlineto{\pgfqpoint{3.647378in}{3.125694in}}%
\pgfpathclose%
\pgfusepath{fill}%
\end{pgfscope}%
\begin{pgfscope}%
\pgfpathrectangle{\pgfqpoint{1.254980in}{0.150000in}}{\pgfqpoint{5.490039in}{5.490039in}}%
\pgfusepath{clip}%
\pgfsetbuttcap%
\pgfsetroundjoin%
\definecolor{currentfill}{rgb}{0.250425,0.274290,0.533103}%
\pgfsetfillcolor{currentfill}%
\pgfsetfillopacity{0.700000}%
\pgfsetlinewidth{0.000000pt}%
\definecolor{currentstroke}{rgb}{0.000000,0.000000,0.000000}%
\pgfsetstrokecolor{currentstroke}%
\pgfsetdash{}{0pt}%
\pgfpathmoveto{\pgfqpoint{4.066857in}{2.908984in}}%
\pgfpathlineto{\pgfqpoint{4.079715in}{2.900098in}}%
\pgfpathlineto{\pgfqpoint{4.092577in}{2.891356in}}%
\pgfpathlineto{\pgfqpoint{4.105443in}{2.882757in}}%
\pgfpathlineto{\pgfqpoint{4.118311in}{2.874302in}}%
\pgfpathlineto{\pgfqpoint{4.125724in}{2.885679in}}%
\pgfpathlineto{\pgfqpoint{4.133133in}{2.897123in}}%
\pgfpathlineto{\pgfqpoint{4.140538in}{2.908636in}}%
\pgfpathlineto{\pgfqpoint{4.147938in}{2.920218in}}%
\pgfpathlineto{\pgfqpoint{4.135081in}{2.928772in}}%
\pgfpathlineto{\pgfqpoint{4.122226in}{2.937469in}}%
\pgfpathlineto{\pgfqpoint{4.109375in}{2.946309in}}%
\pgfpathlineto{\pgfqpoint{4.096527in}{2.955294in}}%
\pgfpathlineto{\pgfqpoint{4.089115in}{2.943608in}}%
\pgfpathlineto{\pgfqpoint{4.081700in}{2.931995in}}%
\pgfpathlineto{\pgfqpoint{4.074280in}{2.920454in}}%
\pgfpathlineto{\pgfqpoint{4.066857in}{2.908984in}}%
\pgfpathclose%
\pgfusepath{fill}%
\end{pgfscope}%
\begin{pgfscope}%
\pgfpathrectangle{\pgfqpoint{1.254980in}{0.150000in}}{\pgfqpoint{5.490039in}{5.490039in}}%
\pgfusepath{clip}%
\pgfsetbuttcap%
\pgfsetroundjoin%
\definecolor{currentfill}{rgb}{0.149039,0.508051,0.557250}%
\pgfsetfillcolor{currentfill}%
\pgfsetfillopacity{0.700000}%
\pgfsetlinewidth{0.000000pt}%
\definecolor{currentstroke}{rgb}{0.000000,0.000000,0.000000}%
\pgfsetstrokecolor{currentstroke}%
\pgfsetdash{}{0pt}%
\pgfpathmoveto{\pgfqpoint{3.338602in}{3.491704in}}%
\pgfpathlineto{\pgfqpoint{3.351499in}{3.474348in}}%
\pgfpathlineto{\pgfqpoint{3.364393in}{3.457183in}}%
\pgfpathlineto{\pgfqpoint{3.377283in}{3.440209in}}%
\pgfpathlineto{\pgfqpoint{3.390169in}{3.423424in}}%
\pgfpathlineto{\pgfqpoint{3.397777in}{3.435097in}}%
\pgfpathlineto{\pgfqpoint{3.405379in}{3.446881in}}%
\pgfpathlineto{\pgfqpoint{3.412976in}{3.458776in}}%
\pgfpathlineto{\pgfqpoint{3.420567in}{3.470785in}}%
\pgfpathlineto{\pgfqpoint{3.407695in}{3.487637in}}%
\pgfpathlineto{\pgfqpoint{3.394819in}{3.504679in}}%
\pgfpathlineto{\pgfqpoint{3.381940in}{3.521911in}}%
\pgfpathlineto{\pgfqpoint{3.369056in}{3.539335in}}%
\pgfpathlineto{\pgfqpoint{3.361451in}{3.527253in}}%
\pgfpathlineto{\pgfqpoint{3.353841in}{3.515288in}}%
\pgfpathlineto{\pgfqpoint{3.346224in}{3.503439in}}%
\pgfpathlineto{\pgfqpoint{3.338602in}{3.491704in}}%
\pgfpathclose%
\pgfusepath{fill}%
\end{pgfscope}%
\begin{pgfscope}%
\pgfpathrectangle{\pgfqpoint{1.254980in}{0.150000in}}{\pgfqpoint{5.490039in}{5.490039in}}%
\pgfusepath{clip}%
\pgfsetbuttcap%
\pgfsetroundjoin%
\definecolor{currentfill}{rgb}{0.257322,0.256130,0.526563}%
\pgfsetfillcolor{currentfill}%
\pgfsetfillopacity{0.700000}%
\pgfsetlinewidth{0.000000pt}%
\definecolor{currentstroke}{rgb}{0.000000,0.000000,0.000000}%
\pgfsetstrokecolor{currentstroke}%
\pgfsetdash{}{0pt}%
\pgfpathmoveto{\pgfqpoint{4.331965in}{2.874188in}}%
\pgfpathlineto{\pgfqpoint{4.344864in}{2.867326in}}%
\pgfpathlineto{\pgfqpoint{4.357769in}{2.860599in}}%
\pgfpathlineto{\pgfqpoint{4.370678in}{2.854006in}}%
\pgfpathlineto{\pgfqpoint{4.383592in}{2.847548in}}%
\pgfpathlineto{\pgfqpoint{4.390932in}{2.859001in}}%
\pgfpathlineto{\pgfqpoint{4.398269in}{2.870517in}}%
\pgfpathlineto{\pgfqpoint{4.405601in}{2.882095in}}%
\pgfpathlineto{\pgfqpoint{4.412930in}{2.893739in}}%
\pgfpathlineto{\pgfqpoint{4.400026in}{2.900327in}}%
\pgfpathlineto{\pgfqpoint{4.387127in}{2.907050in}}%
\pgfpathlineto{\pgfqpoint{4.374233in}{2.913907in}}%
\pgfpathlineto{\pgfqpoint{4.361344in}{2.920899in}}%
\pgfpathlineto{\pgfqpoint{4.354004in}{2.909119in}}%
\pgfpathlineto{\pgfqpoint{4.346661in}{2.897409in}}%
\pgfpathlineto{\pgfqpoint{4.339315in}{2.885765in}}%
\pgfpathlineto{\pgfqpoint{4.331965in}{2.874188in}}%
\pgfpathclose%
\pgfusepath{fill}%
\end{pgfscope}%
\begin{pgfscope}%
\pgfpathrectangle{\pgfqpoint{1.254980in}{0.150000in}}{\pgfqpoint{5.490039in}{5.490039in}}%
\pgfusepath{clip}%
\pgfsetbuttcap%
\pgfsetroundjoin%
\definecolor{currentfill}{rgb}{0.246811,0.283237,0.535941}%
\pgfsetfillcolor{currentfill}%
\pgfsetfillopacity{0.700000}%
\pgfsetlinewidth{0.000000pt}%
\definecolor{currentstroke}{rgb}{0.000000,0.000000,0.000000}%
\pgfsetstrokecolor{currentstroke}%
\pgfsetdash{}{0pt}%
\pgfpathmoveto{\pgfqpoint{4.759263in}{2.925125in}}%
\pgfpathlineto{\pgfqpoint{4.772263in}{2.920763in}}%
\pgfpathlineto{\pgfqpoint{4.785269in}{2.916526in}}%
\pgfpathlineto{\pgfqpoint{4.798282in}{2.912414in}}%
\pgfpathlineto{\pgfqpoint{4.811302in}{2.908425in}}%
\pgfpathlineto{\pgfqpoint{4.818525in}{2.919946in}}%
\pgfpathlineto{\pgfqpoint{4.825745in}{2.931532in}}%
\pgfpathlineto{\pgfqpoint{4.832962in}{2.943187in}}%
\pgfpathlineto{\pgfqpoint{4.840176in}{2.954912in}}%
\pgfpathlineto{\pgfqpoint{4.827167in}{2.959094in}}%
\pgfpathlineto{\pgfqpoint{4.814165in}{2.963399in}}%
\pgfpathlineto{\pgfqpoint{4.801170in}{2.967828in}}%
\pgfpathlineto{\pgfqpoint{4.788182in}{2.972382in}}%
\pgfpathlineto{\pgfqpoint{4.780957in}{2.960458in}}%
\pgfpathlineto{\pgfqpoint{4.773728in}{2.948609in}}%
\pgfpathlineto{\pgfqpoint{4.766497in}{2.936832in}}%
\pgfpathlineto{\pgfqpoint{4.759263in}{2.925125in}}%
\pgfpathclose%
\pgfusepath{fill}%
\end{pgfscope}%
\begin{pgfscope}%
\pgfpathrectangle{\pgfqpoint{1.254980in}{0.150000in}}{\pgfqpoint{5.490039in}{5.490039in}}%
\pgfusepath{clip}%
\pgfsetbuttcap%
\pgfsetroundjoin%
\definecolor{currentfill}{rgb}{0.218130,0.347432,0.550038}%
\pgfsetfillcolor{currentfill}%
\pgfsetfillopacity{0.700000}%
\pgfsetlinewidth{0.000000pt}%
\definecolor{currentstroke}{rgb}{0.000000,0.000000,0.000000}%
\pgfsetstrokecolor{currentstroke}%
\pgfsetdash{}{0pt}%
\pgfpathmoveto{\pgfqpoint{3.698759in}{3.074407in}}%
\pgfpathlineto{\pgfqpoint{3.711604in}{3.061995in}}%
\pgfpathlineto{\pgfqpoint{3.724449in}{3.049745in}}%
\pgfpathlineto{\pgfqpoint{3.737294in}{3.037657in}}%
\pgfpathlineto{\pgfqpoint{3.750140in}{3.025729in}}%
\pgfpathlineto{\pgfqpoint{3.757656in}{3.036974in}}%
\pgfpathlineto{\pgfqpoint{3.765168in}{3.048302in}}%
\pgfpathlineto{\pgfqpoint{3.772675in}{3.059714in}}%
\pgfpathlineto{\pgfqpoint{3.780177in}{3.071210in}}%
\pgfpathlineto{\pgfqpoint{3.767344in}{3.083205in}}%
\pgfpathlineto{\pgfqpoint{3.754511in}{3.095360in}}%
\pgfpathlineto{\pgfqpoint{3.741678in}{3.107677in}}%
\pgfpathlineto{\pgfqpoint{3.728845in}{3.120156in}}%
\pgfpathlineto{\pgfqpoint{3.721331in}{3.108587in}}%
\pgfpathlineto{\pgfqpoint{3.713812in}{3.097106in}}%
\pgfpathlineto{\pgfqpoint{3.706288in}{3.085713in}}%
\pgfpathlineto{\pgfqpoint{3.698759in}{3.074407in}}%
\pgfpathclose%
\pgfusepath{fill}%
\end{pgfscope}%
\begin{pgfscope}%
\pgfpathrectangle{\pgfqpoint{1.254980in}{0.150000in}}{\pgfqpoint{5.490039in}{5.490039in}}%
\pgfusepath{clip}%
\pgfsetbuttcap%
\pgfsetroundjoin%
\definecolor{currentfill}{rgb}{0.137770,0.537492,0.554906}%
\pgfsetfillcolor{currentfill}%
\pgfsetfillopacity{0.700000}%
\pgfsetlinewidth{0.000000pt}%
\definecolor{currentstroke}{rgb}{0.000000,0.000000,0.000000}%
\pgfsetstrokecolor{currentstroke}%
\pgfsetdash{}{0pt}%
\pgfpathmoveto{\pgfqpoint{3.286972in}{3.563072in}}%
\pgfpathlineto{\pgfqpoint{3.299886in}{3.544936in}}%
\pgfpathlineto{\pgfqpoint{3.312796in}{3.526997in}}%
\pgfpathlineto{\pgfqpoint{3.325701in}{3.509254in}}%
\pgfpathlineto{\pgfqpoint{3.338602in}{3.491704in}}%
\pgfpathlineto{\pgfqpoint{3.346224in}{3.503439in}}%
\pgfpathlineto{\pgfqpoint{3.353841in}{3.515288in}}%
\pgfpathlineto{\pgfqpoint{3.361451in}{3.527253in}}%
\pgfpathlineto{\pgfqpoint{3.369056in}{3.539335in}}%
\pgfpathlineto{\pgfqpoint{3.356169in}{3.556952in}}%
\pgfpathlineto{\pgfqpoint{3.343278in}{3.574763in}}%
\pgfpathlineto{\pgfqpoint{3.330383in}{3.592770in}}%
\pgfpathlineto{\pgfqpoint{3.317483in}{3.610973in}}%
\pgfpathlineto{\pgfqpoint{3.309864in}{3.598818in}}%
\pgfpathlineto{\pgfqpoint{3.302239in}{3.586783in}}%
\pgfpathlineto{\pgfqpoint{3.294608in}{3.574868in}}%
\pgfpathlineto{\pgfqpoint{3.286972in}{3.563072in}}%
\pgfpathclose%
\pgfusepath{fill}%
\end{pgfscope}%
\begin{pgfscope}%
\pgfpathrectangle{\pgfqpoint{1.254980in}{0.150000in}}{\pgfqpoint{5.490039in}{5.490039in}}%
\pgfusepath{clip}%
\pgfsetbuttcap%
\pgfsetroundjoin%
\definecolor{currentfill}{rgb}{0.244972,0.287675,0.537260}%
\pgfsetfillcolor{currentfill}%
\pgfsetfillopacity{0.700000}%
\pgfsetlinewidth{0.000000pt}%
\definecolor{currentstroke}{rgb}{0.000000,0.000000,0.000000}%
\pgfsetstrokecolor{currentstroke}%
\pgfsetdash{}{0pt}%
\pgfpathmoveto{\pgfqpoint{3.934255in}{2.939493in}}%
\pgfpathlineto{\pgfqpoint{3.947105in}{2.929512in}}%
\pgfpathlineto{\pgfqpoint{3.959957in}{2.919680in}}%
\pgfpathlineto{\pgfqpoint{3.972811in}{2.909998in}}%
\pgfpathlineto{\pgfqpoint{3.985668in}{2.900463in}}%
\pgfpathlineto{\pgfqpoint{3.993120in}{2.911737in}}%
\pgfpathlineto{\pgfqpoint{4.000567in}{2.923081in}}%
\pgfpathlineto{\pgfqpoint{4.008010in}{2.934496in}}%
\pgfpathlineto{\pgfqpoint{4.015449in}{2.945985in}}%
\pgfpathlineto{\pgfqpoint{4.002603in}{2.955602in}}%
\pgfpathlineto{\pgfqpoint{3.989760in}{2.965368in}}%
\pgfpathlineto{\pgfqpoint{3.976920in}{2.975283in}}%
\pgfpathlineto{\pgfqpoint{3.964081in}{2.985347in}}%
\pgfpathlineto{\pgfqpoint{3.956631in}{2.973770in}}%
\pgfpathlineto{\pgfqpoint{3.949177in}{2.962269in}}%
\pgfpathlineto{\pgfqpoint{3.941718in}{2.950844in}}%
\pgfpathlineto{\pgfqpoint{3.934255in}{2.939493in}}%
\pgfpathclose%
\pgfusepath{fill}%
\end{pgfscope}%
\begin{pgfscope}%
\pgfpathrectangle{\pgfqpoint{1.254980in}{0.150000in}}{\pgfqpoint{5.490039in}{5.490039in}}%
\pgfusepath{clip}%
\pgfsetbuttcap%
\pgfsetroundjoin%
\definecolor{currentfill}{rgb}{0.257322,0.256130,0.526563}%
\pgfsetfillcolor{currentfill}%
\pgfsetfillopacity{0.700000}%
\pgfsetlinewidth{0.000000pt}%
\definecolor{currentstroke}{rgb}{0.000000,0.000000,0.000000}%
\pgfsetstrokecolor{currentstroke}%
\pgfsetdash{}{0pt}%
\pgfpathmoveto{\pgfqpoint{4.464598in}{2.868717in}}%
\pgfpathlineto{\pgfqpoint{4.477528in}{2.862792in}}%
\pgfpathlineto{\pgfqpoint{4.490464in}{2.856998in}}%
\pgfpathlineto{\pgfqpoint{4.503406in}{2.851335in}}%
\pgfpathlineto{\pgfqpoint{4.516353in}{2.845802in}}%
\pgfpathlineto{\pgfqpoint{4.523658in}{2.857233in}}%
\pgfpathlineto{\pgfqpoint{4.530959in}{2.868724in}}%
\pgfpathlineto{\pgfqpoint{4.538257in}{2.880277in}}%
\pgfpathlineto{\pgfqpoint{4.545551in}{2.891893in}}%
\pgfpathlineto{\pgfqpoint{4.532614in}{2.897571in}}%
\pgfpathlineto{\pgfqpoint{4.519683in}{2.903380in}}%
\pgfpathlineto{\pgfqpoint{4.506758in}{2.909320in}}%
\pgfpathlineto{\pgfqpoint{4.493838in}{2.915390in}}%
\pgfpathlineto{\pgfqpoint{4.486533in}{2.903623in}}%
\pgfpathlineto{\pgfqpoint{4.479225in}{2.891922in}}%
\pgfpathlineto{\pgfqpoint{4.471913in}{2.880288in}}%
\pgfpathlineto{\pgfqpoint{4.464598in}{2.868717in}}%
\pgfpathclose%
\pgfusepath{fill}%
\end{pgfscope}%
\begin{pgfscope}%
\pgfpathrectangle{\pgfqpoint{1.254980in}{0.150000in}}{\pgfqpoint{5.490039in}{5.490039in}}%
\pgfusepath{clip}%
\pgfsetbuttcap%
\pgfsetroundjoin%
\definecolor{currentfill}{rgb}{0.250425,0.274290,0.533103}%
\pgfsetfillcolor{currentfill}%
\pgfsetfillopacity{0.700000}%
\pgfsetlinewidth{0.000000pt}%
\definecolor{currentstroke}{rgb}{0.000000,0.000000,0.000000}%
\pgfsetstrokecolor{currentstroke}%
\pgfsetdash{}{0pt}%
\pgfpathmoveto{\pgfqpoint{4.678327in}{2.896948in}}%
\pgfpathlineto{\pgfqpoint{4.691309in}{2.892261in}}%
\pgfpathlineto{\pgfqpoint{4.704298in}{2.887701in}}%
\pgfpathlineto{\pgfqpoint{4.717294in}{2.883267in}}%
\pgfpathlineto{\pgfqpoint{4.730297in}{2.878957in}}%
\pgfpathlineto{\pgfqpoint{4.737544in}{2.890404in}}%
\pgfpathlineto{\pgfqpoint{4.744787in}{2.901912in}}%
\pgfpathlineto{\pgfqpoint{4.752027in}{2.913485in}}%
\pgfpathlineto{\pgfqpoint{4.759263in}{2.925125in}}%
\pgfpathlineto{\pgfqpoint{4.746271in}{2.929611in}}%
\pgfpathlineto{\pgfqpoint{4.733286in}{2.934222in}}%
\pgfpathlineto{\pgfqpoint{4.720308in}{2.938960in}}%
\pgfpathlineto{\pgfqpoint{4.707336in}{2.943823in}}%
\pgfpathlineto{\pgfqpoint{4.700088in}{2.932001in}}%
\pgfpathlineto{\pgfqpoint{4.692837in}{2.920249in}}%
\pgfpathlineto{\pgfqpoint{4.685584in}{2.908565in}}%
\pgfpathlineto{\pgfqpoint{4.678327in}{2.896948in}}%
\pgfpathclose%
\pgfusepath{fill}%
\end{pgfscope}%
\begin{pgfscope}%
\pgfpathrectangle{\pgfqpoint{1.254980in}{0.150000in}}{\pgfqpoint{5.490039in}{5.490039in}}%
\pgfusepath{clip}%
\pgfsetbuttcap%
\pgfsetroundjoin%
\definecolor{currentfill}{rgb}{0.227802,0.326594,0.546532}%
\pgfsetfillcolor{currentfill}%
\pgfsetfillopacity{0.700000}%
\pgfsetlinewidth{0.000000pt}%
\definecolor{currentstroke}{rgb}{0.000000,0.000000,0.000000}%
\pgfsetstrokecolor{currentstroke}%
\pgfsetdash{}{0pt}%
\pgfpathmoveto{\pgfqpoint{3.750140in}{3.025729in}}%
\pgfpathlineto{\pgfqpoint{3.762986in}{3.013961in}}%
\pgfpathlineto{\pgfqpoint{3.775832in}{3.002351in}}%
\pgfpathlineto{\pgfqpoint{3.788680in}{2.990900in}}%
\pgfpathlineto{\pgfqpoint{3.801528in}{2.979606in}}%
\pgfpathlineto{\pgfqpoint{3.809032in}{2.990790in}}%
\pgfpathlineto{\pgfqpoint{3.816532in}{3.002053in}}%
\pgfpathlineto{\pgfqpoint{3.824027in}{3.013396in}}%
\pgfpathlineto{\pgfqpoint{3.831517in}{3.024818in}}%
\pgfpathlineto{\pgfqpoint{3.818681in}{3.036180in}}%
\pgfpathlineto{\pgfqpoint{3.805846in}{3.047698in}}%
\pgfpathlineto{\pgfqpoint{3.793011in}{3.059374in}}%
\pgfpathlineto{\pgfqpoint{3.780177in}{3.071210in}}%
\pgfpathlineto{\pgfqpoint{3.772675in}{3.059714in}}%
\pgfpathlineto{\pgfqpoint{3.765168in}{3.048302in}}%
\pgfpathlineto{\pgfqpoint{3.757656in}{3.036974in}}%
\pgfpathlineto{\pgfqpoint{3.750140in}{3.025729in}}%
\pgfpathclose%
\pgfusepath{fill}%
\end{pgfscope}%
\begin{pgfscope}%
\pgfpathrectangle{\pgfqpoint{1.254980in}{0.150000in}}{\pgfqpoint{5.490039in}{5.490039in}}%
\pgfusepath{clip}%
\pgfsetbuttcap%
\pgfsetroundjoin%
\definecolor{currentfill}{rgb}{0.229739,0.322361,0.545706}%
\pgfsetfillcolor{currentfill}%
\pgfsetfillopacity{0.700000}%
\pgfsetlinewidth{0.000000pt}%
\definecolor{currentstroke}{rgb}{0.000000,0.000000,0.000000}%
\pgfsetstrokecolor{currentstroke}%
\pgfsetdash{}{0pt}%
\pgfpathmoveto{\pgfqpoint{5.054222in}{3.005667in}}%
\pgfpathlineto{\pgfqpoint{5.067309in}{3.002631in}}%
\pgfpathlineto{\pgfqpoint{5.080404in}{2.999715in}}%
\pgfpathlineto{\pgfqpoint{5.093508in}{2.996917in}}%
\pgfpathlineto{\pgfqpoint{5.106620in}{2.994238in}}%
\pgfpathlineto{\pgfqpoint{5.113764in}{3.005759in}}%
\pgfpathlineto{\pgfqpoint{5.120905in}{3.017357in}}%
\pgfpathlineto{\pgfqpoint{5.128044in}{3.029035in}}%
\pgfpathlineto{\pgfqpoint{5.114941in}{3.031892in}}%
\pgfpathlineto{\pgfqpoint{5.101847in}{3.034869in}}%
\pgfpathlineto{\pgfqpoint{5.088761in}{3.037964in}}%
\pgfpathlineto{\pgfqpoint{5.075684in}{3.041178in}}%
\pgfpathlineto{\pgfqpoint{5.068532in}{3.029258in}}%
\pgfpathlineto{\pgfqpoint{5.061378in}{3.017422in}}%
\pgfpathlineto{\pgfqpoint{5.054222in}{3.005667in}}%
\pgfpathclose%
\pgfusepath{fill}%
\end{pgfscope}%
\begin{pgfscope}%
\pgfpathrectangle{\pgfqpoint{1.254980in}{0.150000in}}{\pgfqpoint{5.490039in}{5.490039in}}%
\pgfusepath{clip}%
\pgfsetbuttcap%
\pgfsetroundjoin%
\definecolor{currentfill}{rgb}{0.235526,0.309527,0.542944}%
\pgfsetfillcolor{currentfill}%
\pgfsetfillopacity{0.700000}%
\pgfsetlinewidth{0.000000pt}%
\definecolor{currentstroke}{rgb}{0.000000,0.000000,0.000000}%
\pgfsetstrokecolor{currentstroke}%
\pgfsetdash{}{0pt}%
\pgfpathmoveto{\pgfqpoint{4.973259in}{2.971855in}}%
\pgfpathlineto{\pgfqpoint{4.986325in}{2.968563in}}%
\pgfpathlineto{\pgfqpoint{4.999400in}{2.965392in}}%
\pgfpathlineto{\pgfqpoint{5.012482in}{2.962342in}}%
\pgfpathlineto{\pgfqpoint{5.025573in}{2.959411in}}%
\pgfpathlineto{\pgfqpoint{5.032739in}{2.970865in}}%
\pgfpathlineto{\pgfqpoint{5.039902in}{2.982391in}}%
\pgfpathlineto{\pgfqpoint{5.047063in}{2.993991in}}%
\pgfpathlineto{\pgfqpoint{5.054222in}{3.005667in}}%
\pgfpathlineto{\pgfqpoint{5.041143in}{3.008822in}}%
\pgfpathlineto{\pgfqpoint{5.028072in}{3.012097in}}%
\pgfpathlineto{\pgfqpoint{5.015010in}{3.015492in}}%
\pgfpathlineto{\pgfqpoint{5.001956in}{3.019007in}}%
\pgfpathlineto{\pgfqpoint{4.994785in}{3.007101in}}%
\pgfpathlineto{\pgfqpoint{4.987612in}{2.995275in}}%
\pgfpathlineto{\pgfqpoint{4.980437in}{2.983527in}}%
\pgfpathlineto{\pgfqpoint{4.973259in}{2.971855in}}%
\pgfpathclose%
\pgfusepath{fill}%
\end{pgfscope}%
\begin{pgfscope}%
\pgfpathrectangle{\pgfqpoint{1.254980in}{0.150000in}}{\pgfqpoint{5.490039in}{5.490039in}}%
\pgfusepath{clip}%
\pgfsetbuttcap%
\pgfsetroundjoin%
\definecolor{currentfill}{rgb}{0.127568,0.566949,0.550556}%
\pgfsetfillcolor{currentfill}%
\pgfsetfillopacity{0.700000}%
\pgfsetlinewidth{0.000000pt}%
\definecolor{currentstroke}{rgb}{0.000000,0.000000,0.000000}%
\pgfsetstrokecolor{currentstroke}%
\pgfsetdash{}{0pt}%
\pgfpathmoveto{\pgfqpoint{3.235267in}{3.637612in}}%
\pgfpathlineto{\pgfqpoint{3.248201in}{3.618675in}}%
\pgfpathlineto{\pgfqpoint{3.261129in}{3.599940in}}%
\pgfpathlineto{\pgfqpoint{3.274053in}{3.581407in}}%
\pgfpathlineto{\pgfqpoint{3.286972in}{3.563072in}}%
\pgfpathlineto{\pgfqpoint{3.294608in}{3.574868in}}%
\pgfpathlineto{\pgfqpoint{3.302239in}{3.586783in}}%
\pgfpathlineto{\pgfqpoint{3.309864in}{3.598818in}}%
\pgfpathlineto{\pgfqpoint{3.317483in}{3.610973in}}%
\pgfpathlineto{\pgfqpoint{3.304579in}{3.629375in}}%
\pgfpathlineto{\pgfqpoint{3.291669in}{3.647977in}}%
\pgfpathlineto{\pgfqpoint{3.278755in}{3.666779in}}%
\pgfpathlineto{\pgfqpoint{3.265836in}{3.685784in}}%
\pgfpathlineto{\pgfqpoint{3.258203in}{3.673554in}}%
\pgfpathlineto{\pgfqpoint{3.250564in}{3.661450in}}%
\pgfpathlineto{\pgfqpoint{3.242918in}{3.649469in}}%
\pgfpathlineto{\pgfqpoint{3.235267in}{3.637612in}}%
\pgfpathclose%
\pgfusepath{fill}%
\end{pgfscope}%
\begin{pgfscope}%
\pgfpathrectangle{\pgfqpoint{1.254980in}{0.150000in}}{\pgfqpoint{5.490039in}{5.490039in}}%
\pgfusepath{clip}%
\pgfsetbuttcap%
\pgfsetroundjoin%
\definecolor{currentfill}{rgb}{0.255645,0.260703,0.528312}%
\pgfsetfillcolor{currentfill}%
\pgfsetfillopacity{0.700000}%
\pgfsetlinewidth{0.000000pt}%
\definecolor{currentstroke}{rgb}{0.000000,0.000000,0.000000}%
\pgfsetstrokecolor{currentstroke}%
\pgfsetdash{}{0pt}%
\pgfpathmoveto{\pgfqpoint{4.118311in}{2.874302in}}%
\pgfpathlineto{\pgfqpoint{4.131183in}{2.865989in}}%
\pgfpathlineto{\pgfqpoint{4.144058in}{2.857818in}}%
\pgfpathlineto{\pgfqpoint{4.156937in}{2.849788in}}%
\pgfpathlineto{\pgfqpoint{4.169819in}{2.841898in}}%
\pgfpathlineto{\pgfqpoint{4.177222in}{2.853183in}}%
\pgfpathlineto{\pgfqpoint{4.184620in}{2.864530in}}%
\pgfpathlineto{\pgfqpoint{4.192014in}{2.875942in}}%
\pgfpathlineto{\pgfqpoint{4.199404in}{2.887420in}}%
\pgfpathlineto{\pgfqpoint{4.186532in}{2.895408in}}%
\pgfpathlineto{\pgfqpoint{4.173664in}{2.903536in}}%
\pgfpathlineto{\pgfqpoint{4.160800in}{2.911806in}}%
\pgfpathlineto{\pgfqpoint{4.147938in}{2.920218in}}%
\pgfpathlineto{\pgfqpoint{4.140538in}{2.908636in}}%
\pgfpathlineto{\pgfqpoint{4.133133in}{2.897123in}}%
\pgfpathlineto{\pgfqpoint{4.125724in}{2.885679in}}%
\pgfpathlineto{\pgfqpoint{4.118311in}{2.874302in}}%
\pgfpathclose%
\pgfusepath{fill}%
\end{pgfscope}%
\begin{pgfscope}%
\pgfpathrectangle{\pgfqpoint{1.254980in}{0.150000in}}{\pgfqpoint{5.490039in}{5.490039in}}%
\pgfusepath{clip}%
\pgfsetbuttcap%
\pgfsetroundjoin%
\definecolor{currentfill}{rgb}{0.258965,0.251537,0.524736}%
\pgfsetfillcolor{currentfill}%
\pgfsetfillopacity{0.700000}%
\pgfsetlinewidth{0.000000pt}%
\definecolor{currentstroke}{rgb}{0.000000,0.000000,0.000000}%
\pgfsetstrokecolor{currentstroke}%
\pgfsetdash{}{0pt}%
\pgfpathmoveto{\pgfqpoint{4.250931in}{2.856862in}}%
\pgfpathlineto{\pgfqpoint{4.263823in}{2.849569in}}%
\pgfpathlineto{\pgfqpoint{4.276720in}{2.842412in}}%
\pgfpathlineto{\pgfqpoint{4.289621in}{2.835393in}}%
\pgfpathlineto{\pgfqpoint{4.302526in}{2.828509in}}%
\pgfpathlineto{\pgfqpoint{4.309891in}{2.839837in}}%
\pgfpathlineto{\pgfqpoint{4.317253in}{2.851225in}}%
\pgfpathlineto{\pgfqpoint{4.324611in}{2.862675in}}%
\pgfpathlineto{\pgfqpoint{4.331965in}{2.874188in}}%
\pgfpathlineto{\pgfqpoint{4.319070in}{2.881186in}}%
\pgfpathlineto{\pgfqpoint{4.306179in}{2.888320in}}%
\pgfpathlineto{\pgfqpoint{4.293293in}{2.895590in}}%
\pgfpathlineto{\pgfqpoint{4.280411in}{2.902998in}}%
\pgfpathlineto{\pgfqpoint{4.273047in}{2.891365in}}%
\pgfpathlineto{\pgfqpoint{4.265679in}{2.879799in}}%
\pgfpathlineto{\pgfqpoint{4.258307in}{2.868298in}}%
\pgfpathlineto{\pgfqpoint{4.250931in}{2.856862in}}%
\pgfpathclose%
\pgfusepath{fill}%
\end{pgfscope}%
\begin{pgfscope}%
\pgfpathrectangle{\pgfqpoint{1.254980in}{0.150000in}}{\pgfqpoint{5.490039in}{5.490039in}}%
\pgfusepath{clip}%
\pgfsetbuttcap%
\pgfsetroundjoin%
\definecolor{currentfill}{rgb}{0.253935,0.265254,0.529983}%
\pgfsetfillcolor{currentfill}%
\pgfsetfillopacity{0.700000}%
\pgfsetlinewidth{0.000000pt}%
\definecolor{currentstroke}{rgb}{0.000000,0.000000,0.000000}%
\pgfsetstrokecolor{currentstroke}%
\pgfsetdash{}{0pt}%
\pgfpathmoveto{\pgfqpoint{4.597359in}{2.870473in}}%
\pgfpathlineto{\pgfqpoint{4.610326in}{2.865440in}}%
\pgfpathlineto{\pgfqpoint{4.623300in}{2.860534in}}%
\pgfpathlineto{\pgfqpoint{4.636280in}{2.855756in}}%
\pgfpathlineto{\pgfqpoint{4.649266in}{2.851104in}}%
\pgfpathlineto{\pgfqpoint{4.656536in}{2.862475in}}%
\pgfpathlineto{\pgfqpoint{4.663803in}{2.873904in}}%
\pgfpathlineto{\pgfqpoint{4.671066in}{2.885395in}}%
\pgfpathlineto{\pgfqpoint{4.678327in}{2.896948in}}%
\pgfpathlineto{\pgfqpoint{4.665351in}{2.901761in}}%
\pgfpathlineto{\pgfqpoint{4.652381in}{2.906701in}}%
\pgfpathlineto{\pgfqpoint{4.639418in}{2.911768in}}%
\pgfpathlineto{\pgfqpoint{4.626462in}{2.916962in}}%
\pgfpathlineto{\pgfqpoint{4.619191in}{2.905242in}}%
\pgfpathlineto{\pgfqpoint{4.611917in}{2.893588in}}%
\pgfpathlineto{\pgfqpoint{4.604639in}{2.881999in}}%
\pgfpathlineto{\pgfqpoint{4.597359in}{2.870473in}}%
\pgfpathclose%
\pgfusepath{fill}%
\end{pgfscope}%
\begin{pgfscope}%
\pgfpathrectangle{\pgfqpoint{1.254980in}{0.150000in}}{\pgfqpoint{5.490039in}{5.490039in}}%
\pgfusepath{clip}%
\pgfsetbuttcap%
\pgfsetroundjoin%
\definecolor{currentfill}{rgb}{0.250425,0.274290,0.533103}%
\pgfsetfillcolor{currentfill}%
\pgfsetfillopacity{0.700000}%
\pgfsetlinewidth{0.000000pt}%
\definecolor{currentstroke}{rgb}{0.000000,0.000000,0.000000}%
\pgfsetstrokecolor{currentstroke}%
\pgfsetdash{}{0pt}%
\pgfpathmoveto{\pgfqpoint{3.985668in}{2.900463in}}%
\pgfpathlineto{\pgfqpoint{3.998527in}{2.891076in}}%
\pgfpathlineto{\pgfqpoint{4.011389in}{2.881836in}}%
\pgfpathlineto{\pgfqpoint{4.024253in}{2.872742in}}%
\pgfpathlineto{\pgfqpoint{4.037120in}{2.863794in}}%
\pgfpathlineto{\pgfqpoint{4.044560in}{2.874990in}}%
\pgfpathlineto{\pgfqpoint{4.051997in}{2.886253in}}%
\pgfpathlineto{\pgfqpoint{4.059429in}{2.897584in}}%
\pgfpathlineto{\pgfqpoint{4.066857in}{2.908984in}}%
\pgfpathlineto{\pgfqpoint{4.054000in}{2.918015in}}%
\pgfpathlineto{\pgfqpoint{4.041147in}{2.927192in}}%
\pgfpathlineto{\pgfqpoint{4.028297in}{2.936515in}}%
\pgfpathlineto{\pgfqpoint{4.015449in}{2.945985in}}%
\pgfpathlineto{\pgfqpoint{4.008010in}{2.934496in}}%
\pgfpathlineto{\pgfqpoint{4.000567in}{2.923081in}}%
\pgfpathlineto{\pgfqpoint{3.993120in}{2.911737in}}%
\pgfpathlineto{\pgfqpoint{3.985668in}{2.900463in}}%
\pgfpathclose%
\pgfusepath{fill}%
\end{pgfscope}%
\begin{pgfscope}%
\pgfpathrectangle{\pgfqpoint{1.254980in}{0.150000in}}{\pgfqpoint{5.490039in}{5.490039in}}%
\pgfusepath{clip}%
\pgfsetbuttcap%
\pgfsetroundjoin%
\definecolor{currentfill}{rgb}{0.241237,0.296485,0.539709}%
\pgfsetfillcolor{currentfill}%
\pgfsetfillopacity{0.700000}%
\pgfsetlinewidth{0.000000pt}%
\definecolor{currentstroke}{rgb}{0.000000,0.000000,0.000000}%
\pgfsetstrokecolor{currentstroke}%
\pgfsetdash{}{0pt}%
\pgfpathmoveto{\pgfqpoint{4.892287in}{2.939417in}}%
\pgfpathlineto{\pgfqpoint{4.905334in}{2.935850in}}%
\pgfpathlineto{\pgfqpoint{4.918389in}{2.932404in}}%
\pgfpathlineto{\pgfqpoint{4.931451in}{2.929079in}}%
\pgfpathlineto{\pgfqpoint{4.944521in}{2.925876in}}%
\pgfpathlineto{\pgfqpoint{4.951710in}{2.937268in}}%
\pgfpathlineto{\pgfqpoint{4.958895in}{2.948727in}}%
\pgfpathlineto{\pgfqpoint{4.966078in}{2.960255in}}%
\pgfpathlineto{\pgfqpoint{4.973259in}{2.971855in}}%
\pgfpathlineto{\pgfqpoint{4.960200in}{2.975267in}}%
\pgfpathlineto{\pgfqpoint{4.947150in}{2.978800in}}%
\pgfpathlineto{\pgfqpoint{4.934107in}{2.982454in}}%
\pgfpathlineto{\pgfqpoint{4.921072in}{2.986230in}}%
\pgfpathlineto{\pgfqpoint{4.913880in}{2.974417in}}%
\pgfpathlineto{\pgfqpoint{4.906685in}{2.962678in}}%
\pgfpathlineto{\pgfqpoint{4.899488in}{2.951012in}}%
\pgfpathlineto{\pgfqpoint{4.892287in}{2.939417in}}%
\pgfpathclose%
\pgfusepath{fill}%
\end{pgfscope}%
\begin{pgfscope}%
\pgfpathrectangle{\pgfqpoint{1.254980in}{0.150000in}}{\pgfqpoint{5.490039in}{5.490039in}}%
\pgfusepath{clip}%
\pgfsetbuttcap%
\pgfsetroundjoin%
\definecolor{currentfill}{rgb}{0.235526,0.309527,0.542944}%
\pgfsetfillcolor{currentfill}%
\pgfsetfillopacity{0.700000}%
\pgfsetlinewidth{0.000000pt}%
\definecolor{currentstroke}{rgb}{0.000000,0.000000,0.000000}%
\pgfsetstrokecolor{currentstroke}%
\pgfsetdash{}{0pt}%
\pgfpathmoveto{\pgfqpoint{3.801528in}{2.979606in}}%
\pgfpathlineto{\pgfqpoint{3.814377in}{2.968468in}}%
\pgfpathlineto{\pgfqpoint{3.827227in}{2.957486in}}%
\pgfpathlineto{\pgfqpoint{3.840079in}{2.946659in}}%
\pgfpathlineto{\pgfqpoint{3.852932in}{2.935985in}}%
\pgfpathlineto{\pgfqpoint{3.860424in}{2.947109in}}%
\pgfpathlineto{\pgfqpoint{3.867912in}{2.958306in}}%
\pgfpathlineto{\pgfqpoint{3.875395in}{2.969580in}}%
\pgfpathlineto{\pgfqpoint{3.882874in}{2.980930in}}%
\pgfpathlineto{\pgfqpoint{3.870033in}{2.991670in}}%
\pgfpathlineto{\pgfqpoint{3.857193in}{3.002564in}}%
\pgfpathlineto{\pgfqpoint{3.844355in}{3.013614in}}%
\pgfpathlineto{\pgfqpoint{3.831517in}{3.024818in}}%
\pgfpathlineto{\pgfqpoint{3.824027in}{3.013396in}}%
\pgfpathlineto{\pgfqpoint{3.816532in}{3.002053in}}%
\pgfpathlineto{\pgfqpoint{3.809032in}{2.990790in}}%
\pgfpathlineto{\pgfqpoint{3.801528in}{2.979606in}}%
\pgfpathclose%
\pgfusepath{fill}%
\end{pgfscope}%
\begin{pgfscope}%
\pgfpathrectangle{\pgfqpoint{1.254980in}{0.150000in}}{\pgfqpoint{5.490039in}{5.490039in}}%
\pgfusepath{clip}%
\pgfsetbuttcap%
\pgfsetroundjoin%
\definecolor{currentfill}{rgb}{0.258965,0.251537,0.524736}%
\pgfsetfillcolor{currentfill}%
\pgfsetfillopacity{0.700000}%
\pgfsetlinewidth{0.000000pt}%
\definecolor{currentstroke}{rgb}{0.000000,0.000000,0.000000}%
\pgfsetstrokecolor{currentstroke}%
\pgfsetdash{}{0pt}%
\pgfpathmoveto{\pgfqpoint{4.383592in}{2.847548in}}%
\pgfpathlineto{\pgfqpoint{4.396512in}{2.841223in}}%
\pgfpathlineto{\pgfqpoint{4.409436in}{2.835032in}}%
\pgfpathlineto{\pgfqpoint{4.422366in}{2.828973in}}%
\pgfpathlineto{\pgfqpoint{4.435302in}{2.823046in}}%
\pgfpathlineto{\pgfqpoint{4.442631in}{2.834375in}}%
\pgfpathlineto{\pgfqpoint{4.449957in}{2.845762in}}%
\pgfpathlineto{\pgfqpoint{4.457279in}{2.857209in}}%
\pgfpathlineto{\pgfqpoint{4.464598in}{2.868717in}}%
\pgfpathlineto{\pgfqpoint{4.451673in}{2.874774in}}%
\pgfpathlineto{\pgfqpoint{4.438753in}{2.880963in}}%
\pgfpathlineto{\pgfqpoint{4.425839in}{2.887284in}}%
\pgfpathlineto{\pgfqpoint{4.412930in}{2.893739in}}%
\pgfpathlineto{\pgfqpoint{4.405601in}{2.882095in}}%
\pgfpathlineto{\pgfqpoint{4.398269in}{2.870517in}}%
\pgfpathlineto{\pgfqpoint{4.390932in}{2.859001in}}%
\pgfpathlineto{\pgfqpoint{4.383592in}{2.847548in}}%
\pgfpathclose%
\pgfusepath{fill}%
\end{pgfscope}%
\begin{pgfscope}%
\pgfpathrectangle{\pgfqpoint{1.254980in}{0.150000in}}{\pgfqpoint{5.490039in}{5.490039in}}%
\pgfusepath{clip}%
\pgfsetbuttcap%
\pgfsetroundjoin%
\definecolor{currentfill}{rgb}{0.120092,0.600104,0.542530}%
\pgfsetfillcolor{currentfill}%
\pgfsetfillopacity{0.700000}%
\pgfsetlinewidth{0.000000pt}%
\definecolor{currentstroke}{rgb}{0.000000,0.000000,0.000000}%
\pgfsetstrokecolor{currentstroke}%
\pgfsetdash{}{0pt}%
\pgfpathmoveto{\pgfqpoint{3.183477in}{3.715410in}}%
\pgfpathlineto{\pgfqpoint{3.196433in}{3.695650in}}%
\pgfpathlineto{\pgfqpoint{3.209383in}{3.676098in}}%
\pgfpathlineto{\pgfqpoint{3.222328in}{3.656752in}}%
\pgfpathlineto{\pgfqpoint{3.235267in}{3.637612in}}%
\pgfpathlineto{\pgfqpoint{3.242918in}{3.649469in}}%
\pgfpathlineto{\pgfqpoint{3.250564in}{3.661450in}}%
\pgfpathlineto{\pgfqpoint{3.258203in}{3.673554in}}%
\pgfpathlineto{\pgfqpoint{3.265836in}{3.685784in}}%
\pgfpathlineto{\pgfqpoint{3.252912in}{3.704992in}}%
\pgfpathlineto{\pgfqpoint{3.239982in}{3.724406in}}%
\pgfpathlineto{\pgfqpoint{3.227047in}{3.744026in}}%
\pgfpathlineto{\pgfqpoint{3.214106in}{3.763854in}}%
\pgfpathlineto{\pgfqpoint{3.206458in}{3.751550in}}%
\pgfpathlineto{\pgfqpoint{3.198804in}{3.739376in}}%
\pgfpathlineto{\pgfqpoint{3.191144in}{3.727329in}}%
\pgfpathlineto{\pgfqpoint{3.183477in}{3.715410in}}%
\pgfpathclose%
\pgfusepath{fill}%
\end{pgfscope}%
\begin{pgfscope}%
\pgfpathrectangle{\pgfqpoint{1.254980in}{0.150000in}}{\pgfqpoint{5.490039in}{5.490039in}}%
\pgfusepath{clip}%
\pgfsetbuttcap%
\pgfsetroundjoin%
\definecolor{currentfill}{rgb}{0.183898,0.422383,0.556944}%
\pgfsetfillcolor{currentfill}%
\pgfsetfillopacity{0.700000}%
\pgfsetlinewidth{0.000000pt}%
\definecolor{currentstroke}{rgb}{0.000000,0.000000,0.000000}%
\pgfsetstrokecolor{currentstroke}%
\pgfsetdash{}{0pt}%
\pgfpathmoveto{\pgfqpoint{3.462773in}{3.250617in}}%
\pgfpathlineto{\pgfqpoint{3.475648in}{3.235533in}}%
\pgfpathlineto{\pgfqpoint{3.488520in}{3.220627in}}%
\pgfpathlineto{\pgfqpoint{3.501391in}{3.205897in}}%
\pgfpathlineto{\pgfqpoint{3.514260in}{3.191343in}}%
\pgfpathlineto{\pgfqpoint{3.521849in}{3.202447in}}%
\pgfpathlineto{\pgfqpoint{3.529432in}{3.213647in}}%
\pgfpathlineto{\pgfqpoint{3.537009in}{3.224942in}}%
\pgfpathlineto{\pgfqpoint{3.544582in}{3.236334in}}%
\pgfpathlineto{\pgfqpoint{3.531727in}{3.250939in}}%
\pgfpathlineto{\pgfqpoint{3.518869in}{3.265720in}}%
\pgfpathlineto{\pgfqpoint{3.506010in}{3.280677in}}%
\pgfpathlineto{\pgfqpoint{3.493149in}{3.295812in}}%
\pgfpathlineto{\pgfqpoint{3.485563in}{3.284363in}}%
\pgfpathlineto{\pgfqpoint{3.477972in}{3.273014in}}%
\pgfpathlineto{\pgfqpoint{3.470375in}{3.261766in}}%
\pgfpathlineto{\pgfqpoint{3.462773in}{3.250617in}}%
\pgfpathclose%
\pgfusepath{fill}%
\end{pgfscope}%
\begin{pgfscope}%
\pgfpathrectangle{\pgfqpoint{1.254980in}{0.150000in}}{\pgfqpoint{5.490039in}{5.490039in}}%
\pgfusepath{clip}%
\pgfsetbuttcap%
\pgfsetroundjoin%
\definecolor{currentfill}{rgb}{0.174274,0.445044,0.557792}%
\pgfsetfillcolor{currentfill}%
\pgfsetfillopacity{0.700000}%
\pgfsetlinewidth{0.000000pt}%
\definecolor{currentstroke}{rgb}{0.000000,0.000000,0.000000}%
\pgfsetstrokecolor{currentstroke}%
\pgfsetdash{}{0pt}%
\pgfpathmoveto{\pgfqpoint{3.411249in}{3.312752in}}%
\pgfpathlineto{\pgfqpoint{3.424134in}{3.296946in}}%
\pgfpathlineto{\pgfqpoint{3.437016in}{3.281323in}}%
\pgfpathlineto{\pgfqpoint{3.449895in}{3.265880in}}%
\pgfpathlineto{\pgfqpoint{3.462773in}{3.250617in}}%
\pgfpathlineto{\pgfqpoint{3.470375in}{3.261766in}}%
\pgfpathlineto{\pgfqpoint{3.477972in}{3.273014in}}%
\pgfpathlineto{\pgfqpoint{3.485563in}{3.284363in}}%
\pgfpathlineto{\pgfqpoint{3.493149in}{3.295812in}}%
\pgfpathlineto{\pgfqpoint{3.480286in}{3.311126in}}%
\pgfpathlineto{\pgfqpoint{3.467420in}{3.326620in}}%
\pgfpathlineto{\pgfqpoint{3.454552in}{3.342294in}}%
\pgfpathlineto{\pgfqpoint{3.441681in}{3.358151in}}%
\pgfpathlineto{\pgfqpoint{3.434081in}{3.346645in}}%
\pgfpathlineto{\pgfqpoint{3.426476in}{3.335244in}}%
\pgfpathlineto{\pgfqpoint{3.418865in}{3.323946in}}%
\pgfpathlineto{\pgfqpoint{3.411249in}{3.312752in}}%
\pgfpathclose%
\pgfusepath{fill}%
\end{pgfscope}%
\begin{pgfscope}%
\pgfpathrectangle{\pgfqpoint{1.254980in}{0.150000in}}{\pgfqpoint{5.490039in}{5.490039in}}%
\pgfusepath{clip}%
\pgfsetbuttcap%
\pgfsetroundjoin%
\definecolor{currentfill}{rgb}{0.194100,0.399323,0.555565}%
\pgfsetfillcolor{currentfill}%
\pgfsetfillopacity{0.700000}%
\pgfsetlinewidth{0.000000pt}%
\definecolor{currentstroke}{rgb}{0.000000,0.000000,0.000000}%
\pgfsetstrokecolor{currentstroke}%
\pgfsetdash{}{0pt}%
\pgfpathmoveto{\pgfqpoint{3.514260in}{3.191343in}}%
\pgfpathlineto{\pgfqpoint{3.527127in}{3.176964in}}%
\pgfpathlineto{\pgfqpoint{3.539993in}{3.162758in}}%
\pgfpathlineto{\pgfqpoint{3.552857in}{3.148725in}}%
\pgfpathlineto{\pgfqpoint{3.565720in}{3.134863in}}%
\pgfpathlineto{\pgfqpoint{3.573295in}{3.145922in}}%
\pgfpathlineto{\pgfqpoint{3.580865in}{3.157072in}}%
\pgfpathlineto{\pgfqpoint{3.588429in}{3.168314in}}%
\pgfpathlineto{\pgfqpoint{3.595989in}{3.179649in}}%
\pgfpathlineto{\pgfqpoint{3.583139in}{3.193562in}}%
\pgfpathlineto{\pgfqpoint{3.570288in}{3.207647in}}%
\pgfpathlineto{\pgfqpoint{3.557436in}{3.221904in}}%
\pgfpathlineto{\pgfqpoint{3.544582in}{3.236334in}}%
\pgfpathlineto{\pgfqpoint{3.537009in}{3.224942in}}%
\pgfpathlineto{\pgfqpoint{3.529432in}{3.213647in}}%
\pgfpathlineto{\pgfqpoint{3.521849in}{3.202447in}}%
\pgfpathlineto{\pgfqpoint{3.514260in}{3.191343in}}%
\pgfpathclose%
\pgfusepath{fill}%
\end{pgfscope}%
\begin{pgfscope}%
\pgfpathrectangle{\pgfqpoint{1.254980in}{0.150000in}}{\pgfqpoint{5.490039in}{5.490039in}}%
\pgfusepath{clip}%
\pgfsetbuttcap%
\pgfsetroundjoin%
\definecolor{currentfill}{rgb}{0.246811,0.283237,0.535941}%
\pgfsetfillcolor{currentfill}%
\pgfsetfillopacity{0.700000}%
\pgfsetlinewidth{0.000000pt}%
\definecolor{currentstroke}{rgb}{0.000000,0.000000,0.000000}%
\pgfsetstrokecolor{currentstroke}%
\pgfsetdash{}{0pt}%
\pgfpathmoveto{\pgfqpoint{4.811302in}{2.908425in}}%
\pgfpathlineto{\pgfqpoint{4.824330in}{2.904560in}}%
\pgfpathlineto{\pgfqpoint{4.837365in}{2.900818in}}%
\pgfpathlineto{\pgfqpoint{4.850408in}{2.897198in}}%
\pgfpathlineto{\pgfqpoint{4.863459in}{2.893702in}}%
\pgfpathlineto{\pgfqpoint{4.870670in}{2.905035in}}%
\pgfpathlineto{\pgfqpoint{4.877879in}{2.916431in}}%
\pgfpathlineto{\pgfqpoint{4.885085in}{2.927891in}}%
\pgfpathlineto{\pgfqpoint{4.892287in}{2.939417in}}%
\pgfpathlineto{\pgfqpoint{4.879248in}{2.943107in}}%
\pgfpathlineto{\pgfqpoint{4.866217in}{2.946919in}}%
\pgfpathlineto{\pgfqpoint{4.853193in}{2.950854in}}%
\pgfpathlineto{\pgfqpoint{4.840176in}{2.954912in}}%
\pgfpathlineto{\pgfqpoint{4.832962in}{2.943187in}}%
\pgfpathlineto{\pgfqpoint{4.825745in}{2.931532in}}%
\pgfpathlineto{\pgfqpoint{4.818525in}{2.919946in}}%
\pgfpathlineto{\pgfqpoint{4.811302in}{2.908425in}}%
\pgfpathclose%
\pgfusepath{fill}%
\end{pgfscope}%
\begin{pgfscope}%
\pgfpathrectangle{\pgfqpoint{1.254980in}{0.150000in}}{\pgfqpoint{5.490039in}{5.490039in}}%
\pgfusepath{clip}%
\pgfsetbuttcap%
\pgfsetroundjoin%
\definecolor{currentfill}{rgb}{0.163625,0.471133,0.558148}%
\pgfsetfillcolor{currentfill}%
\pgfsetfillopacity{0.700000}%
\pgfsetlinewidth{0.000000pt}%
\definecolor{currentstroke}{rgb}{0.000000,0.000000,0.000000}%
\pgfsetstrokecolor{currentstroke}%
\pgfsetdash{}{0pt}%
\pgfpathmoveto{\pgfqpoint{3.359679in}{3.377821in}}%
\pgfpathlineto{\pgfqpoint{3.372577in}{3.361274in}}%
\pgfpathlineto{\pgfqpoint{3.385471in}{3.344915in}}%
\pgfpathlineto{\pgfqpoint{3.398361in}{3.328742in}}%
\pgfpathlineto{\pgfqpoint{3.411249in}{3.312752in}}%
\pgfpathlineto{\pgfqpoint{3.418865in}{3.323946in}}%
\pgfpathlineto{\pgfqpoint{3.426476in}{3.335244in}}%
\pgfpathlineto{\pgfqpoint{3.434081in}{3.346645in}}%
\pgfpathlineto{\pgfqpoint{3.441681in}{3.358151in}}%
\pgfpathlineto{\pgfqpoint{3.428808in}{3.374192in}}%
\pgfpathlineto{\pgfqpoint{3.415931in}{3.390416in}}%
\pgfpathlineto{\pgfqpoint{3.403052in}{3.406827in}}%
\pgfpathlineto{\pgfqpoint{3.390169in}{3.423424in}}%
\pgfpathlineto{\pgfqpoint{3.382555in}{3.411860in}}%
\pgfpathlineto{\pgfqpoint{3.374936in}{3.400406in}}%
\pgfpathlineto{\pgfqpoint{3.367310in}{3.389059in}}%
\pgfpathlineto{\pgfqpoint{3.359679in}{3.377821in}}%
\pgfpathclose%
\pgfusepath{fill}%
\end{pgfscope}%
\begin{pgfscope}%
\pgfpathrectangle{\pgfqpoint{1.254980in}{0.150000in}}{\pgfqpoint{5.490039in}{5.490039in}}%
\pgfusepath{clip}%
\pgfsetbuttcap%
\pgfsetroundjoin%
\definecolor{currentfill}{rgb}{0.204903,0.375746,0.553533}%
\pgfsetfillcolor{currentfill}%
\pgfsetfillopacity{0.700000}%
\pgfsetlinewidth{0.000000pt}%
\definecolor{currentstroke}{rgb}{0.000000,0.000000,0.000000}%
\pgfsetstrokecolor{currentstroke}%
\pgfsetdash{}{0pt}%
\pgfpathmoveto{\pgfqpoint{3.565720in}{3.134863in}}%
\pgfpathlineto{\pgfqpoint{3.578582in}{3.121172in}}%
\pgfpathlineto{\pgfqpoint{3.591443in}{3.107651in}}%
\pgfpathlineto{\pgfqpoint{3.604303in}{3.094298in}}%
\pgfpathlineto{\pgfqpoint{3.617162in}{3.081112in}}%
\pgfpathlineto{\pgfqpoint{3.624724in}{3.092126in}}%
\pgfpathlineto{\pgfqpoint{3.632280in}{3.103227in}}%
\pgfpathlineto{\pgfqpoint{3.639832in}{3.114416in}}%
\pgfpathlineto{\pgfqpoint{3.647378in}{3.125694in}}%
\pgfpathlineto{\pgfqpoint{3.634532in}{3.138930in}}%
\pgfpathlineto{\pgfqpoint{3.621685in}{3.152335in}}%
\pgfpathlineto{\pgfqpoint{3.608837in}{3.165907in}}%
\pgfpathlineto{\pgfqpoint{3.595989in}{3.179649in}}%
\pgfpathlineto{\pgfqpoint{3.588429in}{3.168314in}}%
\pgfpathlineto{\pgfqpoint{3.580865in}{3.157072in}}%
\pgfpathlineto{\pgfqpoint{3.573295in}{3.145922in}}%
\pgfpathlineto{\pgfqpoint{3.565720in}{3.134863in}}%
\pgfpathclose%
\pgfusepath{fill}%
\end{pgfscope}%
\begin{pgfscope}%
\pgfpathrectangle{\pgfqpoint{1.254980in}{0.150000in}}{\pgfqpoint{5.490039in}{5.490039in}}%
\pgfusepath{clip}%
\pgfsetbuttcap%
\pgfsetroundjoin%
\definecolor{currentfill}{rgb}{0.243113,0.292092,0.538516}%
\pgfsetfillcolor{currentfill}%
\pgfsetfillopacity{0.700000}%
\pgfsetlinewidth{0.000000pt}%
\definecolor{currentstroke}{rgb}{0.000000,0.000000,0.000000}%
\pgfsetstrokecolor{currentstroke}%
\pgfsetdash{}{0pt}%
\pgfpathmoveto{\pgfqpoint{3.852932in}{2.935985in}}%
\pgfpathlineto{\pgfqpoint{3.865786in}{2.925466in}}%
\pgfpathlineto{\pgfqpoint{3.878642in}{2.915098in}}%
\pgfpathlineto{\pgfqpoint{3.891500in}{2.904883in}}%
\pgfpathlineto{\pgfqpoint{3.904359in}{2.894818in}}%
\pgfpathlineto{\pgfqpoint{3.911840in}{2.905880in}}%
\pgfpathlineto{\pgfqpoint{3.919316in}{2.917012in}}%
\pgfpathlineto{\pgfqpoint{3.926788in}{2.928217in}}%
\pgfpathlineto{\pgfqpoint{3.934255in}{2.939493in}}%
\pgfpathlineto{\pgfqpoint{3.921407in}{2.949625in}}%
\pgfpathlineto{\pgfqpoint{3.908561in}{2.959908in}}%
\pgfpathlineto{\pgfqpoint{3.895717in}{2.970342in}}%
\pgfpathlineto{\pgfqpoint{3.882874in}{2.980930in}}%
\pgfpathlineto{\pgfqpoint{3.875395in}{2.969580in}}%
\pgfpathlineto{\pgfqpoint{3.867912in}{2.958306in}}%
\pgfpathlineto{\pgfqpoint{3.860424in}{2.947109in}}%
\pgfpathlineto{\pgfqpoint{3.852932in}{2.935985in}}%
\pgfpathclose%
\pgfusepath{fill}%
\end{pgfscope}%
\begin{pgfscope}%
\pgfpathrectangle{\pgfqpoint{1.254980in}{0.150000in}}{\pgfqpoint{5.490039in}{5.490039in}}%
\pgfusepath{clip}%
\pgfsetbuttcap%
\pgfsetroundjoin%
\definecolor{currentfill}{rgb}{0.258965,0.251537,0.524736}%
\pgfsetfillcolor{currentfill}%
\pgfsetfillopacity{0.700000}%
\pgfsetlinewidth{0.000000pt}%
\definecolor{currentstroke}{rgb}{0.000000,0.000000,0.000000}%
\pgfsetstrokecolor{currentstroke}%
\pgfsetdash{}{0pt}%
\pgfpathmoveto{\pgfqpoint{4.516353in}{2.845802in}}%
\pgfpathlineto{\pgfqpoint{4.529306in}{2.840399in}}%
\pgfpathlineto{\pgfqpoint{4.542266in}{2.835126in}}%
\pgfpathlineto{\pgfqpoint{4.555231in}{2.829981in}}%
\pgfpathlineto{\pgfqpoint{4.568203in}{2.824966in}}%
\pgfpathlineto{\pgfqpoint{4.575497in}{2.836256in}}%
\pgfpathlineto{\pgfqpoint{4.582788in}{2.847603in}}%
\pgfpathlineto{\pgfqpoint{4.590075in}{2.859009in}}%
\pgfpathlineto{\pgfqpoint{4.597359in}{2.870473in}}%
\pgfpathlineto{\pgfqpoint{4.584398in}{2.875635in}}%
\pgfpathlineto{\pgfqpoint{4.571443in}{2.880925in}}%
\pgfpathlineto{\pgfqpoint{4.558494in}{2.886344in}}%
\pgfpathlineto{\pgfqpoint{4.545551in}{2.891893in}}%
\pgfpathlineto{\pgfqpoint{4.538257in}{2.880277in}}%
\pgfpathlineto{\pgfqpoint{4.530959in}{2.868724in}}%
\pgfpathlineto{\pgfqpoint{4.523658in}{2.857233in}}%
\pgfpathlineto{\pgfqpoint{4.516353in}{2.845802in}}%
\pgfpathclose%
\pgfusepath{fill}%
\end{pgfscope}%
\begin{pgfscope}%
\pgfpathrectangle{\pgfqpoint{1.254980in}{0.150000in}}{\pgfqpoint{5.490039in}{5.490039in}}%
\pgfusepath{clip}%
\pgfsetbuttcap%
\pgfsetroundjoin%
\definecolor{currentfill}{rgb}{0.258965,0.251537,0.524736}%
\pgfsetfillcolor{currentfill}%
\pgfsetfillopacity{0.700000}%
\pgfsetlinewidth{0.000000pt}%
\definecolor{currentstroke}{rgb}{0.000000,0.000000,0.000000}%
\pgfsetstrokecolor{currentstroke}%
\pgfsetdash{}{0pt}%
\pgfpathmoveto{\pgfqpoint{4.169819in}{2.841898in}}%
\pgfpathlineto{\pgfqpoint{4.182706in}{2.834149in}}%
\pgfpathlineto{\pgfqpoint{4.195596in}{2.826539in}}%
\pgfpathlineto{\pgfqpoint{4.208490in}{2.819068in}}%
\pgfpathlineto{\pgfqpoint{4.221389in}{2.811736in}}%
\pgfpathlineto{\pgfqpoint{4.228780in}{2.822927in}}%
\pgfpathlineto{\pgfqpoint{4.236168in}{2.834178in}}%
\pgfpathlineto{\pgfqpoint{4.243551in}{2.845489in}}%
\pgfpathlineto{\pgfqpoint{4.250931in}{2.856862in}}%
\pgfpathlineto{\pgfqpoint{4.238043in}{2.864294in}}%
\pgfpathlineto{\pgfqpoint{4.225160in}{2.871863in}}%
\pgfpathlineto{\pgfqpoint{4.212280in}{2.879572in}}%
\pgfpathlineto{\pgfqpoint{4.199404in}{2.887420in}}%
\pgfpathlineto{\pgfqpoint{4.192014in}{2.875942in}}%
\pgfpathlineto{\pgfqpoint{4.184620in}{2.864530in}}%
\pgfpathlineto{\pgfqpoint{4.177222in}{2.853183in}}%
\pgfpathlineto{\pgfqpoint{4.169819in}{2.841898in}}%
\pgfpathclose%
\pgfusepath{fill}%
\end{pgfscope}%
\begin{pgfscope}%
\pgfpathrectangle{\pgfqpoint{1.254980in}{0.150000in}}{\pgfqpoint{5.490039in}{5.490039in}}%
\pgfusepath{clip}%
\pgfsetbuttcap%
\pgfsetroundjoin%
\definecolor{currentfill}{rgb}{0.214298,0.355619,0.551184}%
\pgfsetfillcolor{currentfill}%
\pgfsetfillopacity{0.700000}%
\pgfsetlinewidth{0.000000pt}%
\definecolor{currentstroke}{rgb}{0.000000,0.000000,0.000000}%
\pgfsetstrokecolor{currentstroke}%
\pgfsetdash{}{0pt}%
\pgfpathmoveto{\pgfqpoint{3.617162in}{3.081112in}}%
\pgfpathlineto{\pgfqpoint{3.630021in}{3.068094in}}%
\pgfpathlineto{\pgfqpoint{3.642879in}{3.055241in}}%
\pgfpathlineto{\pgfqpoint{3.655737in}{3.042553in}}%
\pgfpathlineto{\pgfqpoint{3.668594in}{3.030029in}}%
\pgfpathlineto{\pgfqpoint{3.676143in}{3.040998in}}%
\pgfpathlineto{\pgfqpoint{3.683687in}{3.052050in}}%
\pgfpathlineto{\pgfqpoint{3.691225in}{3.063186in}}%
\pgfpathlineto{\pgfqpoint{3.698759in}{3.074407in}}%
\pgfpathlineto{\pgfqpoint{3.685914in}{3.086982in}}%
\pgfpathlineto{\pgfqpoint{3.673069in}{3.099721in}}%
\pgfpathlineto{\pgfqpoint{3.660224in}{3.112624in}}%
\pgfpathlineto{\pgfqpoint{3.647378in}{3.125694in}}%
\pgfpathlineto{\pgfqpoint{3.639832in}{3.114416in}}%
\pgfpathlineto{\pgfqpoint{3.632280in}{3.103227in}}%
\pgfpathlineto{\pgfqpoint{3.624724in}{3.092126in}}%
\pgfpathlineto{\pgfqpoint{3.617162in}{3.081112in}}%
\pgfpathclose%
\pgfusepath{fill}%
\end{pgfscope}%
\begin{pgfscope}%
\pgfpathrectangle{\pgfqpoint{1.254980in}{0.150000in}}{\pgfqpoint{5.490039in}{5.490039in}}%
\pgfusepath{clip}%
\pgfsetbuttcap%
\pgfsetroundjoin%
\definecolor{currentfill}{rgb}{0.153364,0.497000,0.557724}%
\pgfsetfillcolor{currentfill}%
\pgfsetfillopacity{0.700000}%
\pgfsetlinewidth{0.000000pt}%
\definecolor{currentstroke}{rgb}{0.000000,0.000000,0.000000}%
\pgfsetstrokecolor{currentstroke}%
\pgfsetdash{}{0pt}%
\pgfpathmoveto{\pgfqpoint{3.308055in}{3.445897in}}%
\pgfpathlineto{\pgfqpoint{3.320967in}{3.428592in}}%
\pgfpathlineto{\pgfqpoint{3.333875in}{3.411478in}}%
\pgfpathlineto{\pgfqpoint{3.346779in}{3.394555in}}%
\pgfpathlineto{\pgfqpoint{3.359679in}{3.377821in}}%
\pgfpathlineto{\pgfqpoint{3.367310in}{3.389059in}}%
\pgfpathlineto{\pgfqpoint{3.374936in}{3.400406in}}%
\pgfpathlineto{\pgfqpoint{3.382555in}{3.411860in}}%
\pgfpathlineto{\pgfqpoint{3.390169in}{3.423424in}}%
\pgfpathlineto{\pgfqpoint{3.377283in}{3.440209in}}%
\pgfpathlineto{\pgfqpoint{3.364393in}{3.457183in}}%
\pgfpathlineto{\pgfqpoint{3.351499in}{3.474348in}}%
\pgfpathlineto{\pgfqpoint{3.338602in}{3.491704in}}%
\pgfpathlineto{\pgfqpoint{3.330974in}{3.480084in}}%
\pgfpathlineto{\pgfqpoint{3.323340in}{3.468576in}}%
\pgfpathlineto{\pgfqpoint{3.315700in}{3.457181in}}%
\pgfpathlineto{\pgfqpoint{3.308055in}{3.445897in}}%
\pgfpathclose%
\pgfusepath{fill}%
\end{pgfscope}%
\begin{pgfscope}%
\pgfpathrectangle{\pgfqpoint{1.254980in}{0.150000in}}{\pgfqpoint{5.490039in}{5.490039in}}%
\pgfusepath{clip}%
\pgfsetbuttcap%
\pgfsetroundjoin%
\definecolor{currentfill}{rgb}{0.255645,0.260703,0.528312}%
\pgfsetfillcolor{currentfill}%
\pgfsetfillopacity{0.700000}%
\pgfsetlinewidth{0.000000pt}%
\definecolor{currentstroke}{rgb}{0.000000,0.000000,0.000000}%
\pgfsetstrokecolor{currentstroke}%
\pgfsetdash{}{0pt}%
\pgfpathmoveto{\pgfqpoint{4.037120in}{2.863794in}}%
\pgfpathlineto{\pgfqpoint{4.049990in}{2.854991in}}%
\pgfpathlineto{\pgfqpoint{4.062863in}{2.846332in}}%
\pgfpathlineto{\pgfqpoint{4.075739in}{2.837816in}}%
\pgfpathlineto{\pgfqpoint{4.088618in}{2.829444in}}%
\pgfpathlineto{\pgfqpoint{4.096048in}{2.840563in}}%
\pgfpathlineto{\pgfqpoint{4.103473in}{2.851745in}}%
\pgfpathlineto{\pgfqpoint{4.110894in}{2.862991in}}%
\pgfpathlineto{\pgfqpoint{4.118311in}{2.874302in}}%
\pgfpathlineto{\pgfqpoint{4.105443in}{2.882757in}}%
\pgfpathlineto{\pgfqpoint{4.092577in}{2.891356in}}%
\pgfpathlineto{\pgfqpoint{4.079715in}{2.900098in}}%
\pgfpathlineto{\pgfqpoint{4.066857in}{2.908984in}}%
\pgfpathlineto{\pgfqpoint{4.059429in}{2.897584in}}%
\pgfpathlineto{\pgfqpoint{4.051997in}{2.886253in}}%
\pgfpathlineto{\pgfqpoint{4.044560in}{2.874990in}}%
\pgfpathlineto{\pgfqpoint{4.037120in}{2.863794in}}%
\pgfpathclose%
\pgfusepath{fill}%
\end{pgfscope}%
\begin{pgfscope}%
\pgfpathrectangle{\pgfqpoint{1.254980in}{0.150000in}}{\pgfqpoint{5.490039in}{5.490039in}}%
\pgfusepath{clip}%
\pgfsetbuttcap%
\pgfsetroundjoin%
\definecolor{currentfill}{rgb}{0.260571,0.246922,0.522828}%
\pgfsetfillcolor{currentfill}%
\pgfsetfillopacity{0.700000}%
\pgfsetlinewidth{0.000000pt}%
\definecolor{currentstroke}{rgb}{0.000000,0.000000,0.000000}%
\pgfsetstrokecolor{currentstroke}%
\pgfsetdash{}{0pt}%
\pgfpathmoveto{\pgfqpoint{4.302526in}{2.828509in}}%
\pgfpathlineto{\pgfqpoint{4.315436in}{2.821762in}}%
\pgfpathlineto{\pgfqpoint{4.328351in}{2.815149in}}%
\pgfpathlineto{\pgfqpoint{4.341271in}{2.808671in}}%
\pgfpathlineto{\pgfqpoint{4.354196in}{2.802327in}}%
\pgfpathlineto{\pgfqpoint{4.361550in}{2.813546in}}%
\pgfpathlineto{\pgfqpoint{4.368902in}{2.824822in}}%
\pgfpathlineto{\pgfqpoint{4.376249in}{2.836155in}}%
\pgfpathlineto{\pgfqpoint{4.383592in}{2.847548in}}%
\pgfpathlineto{\pgfqpoint{4.370678in}{2.854006in}}%
\pgfpathlineto{\pgfqpoint{4.357769in}{2.860599in}}%
\pgfpathlineto{\pgfqpoint{4.344864in}{2.867326in}}%
\pgfpathlineto{\pgfqpoint{4.331965in}{2.874188in}}%
\pgfpathlineto{\pgfqpoint{4.324611in}{2.862675in}}%
\pgfpathlineto{\pgfqpoint{4.317253in}{2.851225in}}%
\pgfpathlineto{\pgfqpoint{4.309891in}{2.839837in}}%
\pgfpathlineto{\pgfqpoint{4.302526in}{2.828509in}}%
\pgfpathclose%
\pgfusepath{fill}%
\end{pgfscope}%
\begin{pgfscope}%
\pgfpathrectangle{\pgfqpoint{1.254980in}{0.150000in}}{\pgfqpoint{5.490039in}{5.490039in}}%
\pgfusepath{clip}%
\pgfsetbuttcap%
\pgfsetroundjoin%
\definecolor{currentfill}{rgb}{0.250425,0.274290,0.533103}%
\pgfsetfillcolor{currentfill}%
\pgfsetfillopacity{0.700000}%
\pgfsetlinewidth{0.000000pt}%
\definecolor{currentstroke}{rgb}{0.000000,0.000000,0.000000}%
\pgfsetstrokecolor{currentstroke}%
\pgfsetdash{}{0pt}%
\pgfpathmoveto{\pgfqpoint{4.730297in}{2.878957in}}%
\pgfpathlineto{\pgfqpoint{4.743307in}{2.874773in}}%
\pgfpathlineto{\pgfqpoint{4.756325in}{2.870714in}}%
\pgfpathlineto{\pgfqpoint{4.769349in}{2.866778in}}%
\pgfpathlineto{\pgfqpoint{4.782380in}{2.862967in}}%
\pgfpathlineto{\pgfqpoint{4.789616in}{2.874242in}}%
\pgfpathlineto{\pgfqpoint{4.796848in}{2.885575in}}%
\pgfpathlineto{\pgfqpoint{4.804077in}{2.896969in}}%
\pgfpathlineto{\pgfqpoint{4.811302in}{2.908425in}}%
\pgfpathlineto{\pgfqpoint{4.798282in}{2.912414in}}%
\pgfpathlineto{\pgfqpoint{4.785269in}{2.916526in}}%
\pgfpathlineto{\pgfqpoint{4.772263in}{2.920763in}}%
\pgfpathlineto{\pgfqpoint{4.759263in}{2.925125in}}%
\pgfpathlineto{\pgfqpoint{4.752027in}{2.913485in}}%
\pgfpathlineto{\pgfqpoint{4.744787in}{2.901912in}}%
\pgfpathlineto{\pgfqpoint{4.737544in}{2.890404in}}%
\pgfpathlineto{\pgfqpoint{4.730297in}{2.878957in}}%
\pgfpathclose%
\pgfusepath{fill}%
\end{pgfscope}%
\begin{pgfscope}%
\pgfpathrectangle{\pgfqpoint{1.254980in}{0.150000in}}{\pgfqpoint{5.490039in}{5.490039in}}%
\pgfusepath{clip}%
\pgfsetbuttcap%
\pgfsetroundjoin%
\definecolor{currentfill}{rgb}{0.223925,0.334994,0.548053}%
\pgfsetfillcolor{currentfill}%
\pgfsetfillopacity{0.700000}%
\pgfsetlinewidth{0.000000pt}%
\definecolor{currentstroke}{rgb}{0.000000,0.000000,0.000000}%
\pgfsetstrokecolor{currentstroke}%
\pgfsetdash{}{0pt}%
\pgfpathmoveto{\pgfqpoint{3.668594in}{3.030029in}}%
\pgfpathlineto{\pgfqpoint{3.681452in}{3.017669in}}%
\pgfpathlineto{\pgfqpoint{3.694309in}{3.005470in}}%
\pgfpathlineto{\pgfqpoint{3.707167in}{2.993433in}}%
\pgfpathlineto{\pgfqpoint{3.720025in}{2.981557in}}%
\pgfpathlineto{\pgfqpoint{3.727561in}{2.992480in}}%
\pgfpathlineto{\pgfqpoint{3.735092in}{3.003482in}}%
\pgfpathlineto{\pgfqpoint{3.742618in}{3.014565in}}%
\pgfpathlineto{\pgfqpoint{3.750140in}{3.025729in}}%
\pgfpathlineto{\pgfqpoint{3.737294in}{3.037657in}}%
\pgfpathlineto{\pgfqpoint{3.724449in}{3.049745in}}%
\pgfpathlineto{\pgfqpoint{3.711604in}{3.061995in}}%
\pgfpathlineto{\pgfqpoint{3.698759in}{3.074407in}}%
\pgfpathlineto{\pgfqpoint{3.691225in}{3.063186in}}%
\pgfpathlineto{\pgfqpoint{3.683687in}{3.052050in}}%
\pgfpathlineto{\pgfqpoint{3.676143in}{3.040998in}}%
\pgfpathlineto{\pgfqpoint{3.668594in}{3.030029in}}%
\pgfpathclose%
\pgfusepath{fill}%
\end{pgfscope}%
\begin{pgfscope}%
\pgfpathrectangle{\pgfqpoint{1.254980in}{0.150000in}}{\pgfqpoint{5.490039in}{5.490039in}}%
\pgfusepath{clip}%
\pgfsetbuttcap%
\pgfsetroundjoin%
\definecolor{currentfill}{rgb}{0.141935,0.526453,0.555991}%
\pgfsetfillcolor{currentfill}%
\pgfsetfillopacity{0.700000}%
\pgfsetlinewidth{0.000000pt}%
\definecolor{currentstroke}{rgb}{0.000000,0.000000,0.000000}%
\pgfsetstrokecolor{currentstroke}%
\pgfsetdash{}{0pt}%
\pgfpathmoveto{\pgfqpoint{3.256365in}{3.517061in}}%
\pgfpathlineto{\pgfqpoint{3.269294in}{3.498976in}}%
\pgfpathlineto{\pgfqpoint{3.282218in}{3.481088in}}%
\pgfpathlineto{\pgfqpoint{3.295139in}{3.463395in}}%
\pgfpathlineto{\pgfqpoint{3.308055in}{3.445897in}}%
\pgfpathlineto{\pgfqpoint{3.315700in}{3.457181in}}%
\pgfpathlineto{\pgfqpoint{3.323340in}{3.468576in}}%
\pgfpathlineto{\pgfqpoint{3.330974in}{3.480084in}}%
\pgfpathlineto{\pgfqpoint{3.338602in}{3.491704in}}%
\pgfpathlineto{\pgfqpoint{3.325701in}{3.509254in}}%
\pgfpathlineto{\pgfqpoint{3.312796in}{3.526997in}}%
\pgfpathlineto{\pgfqpoint{3.299886in}{3.544936in}}%
\pgfpathlineto{\pgfqpoint{3.286972in}{3.563072in}}%
\pgfpathlineto{\pgfqpoint{3.279329in}{3.551394in}}%
\pgfpathlineto{\pgfqpoint{3.271680in}{3.539834in}}%
\pgfpathlineto{\pgfqpoint{3.264025in}{3.528389in}}%
\pgfpathlineto{\pgfqpoint{3.256365in}{3.517061in}}%
\pgfpathclose%
\pgfusepath{fill}%
\end{pgfscope}%
\begin{pgfscope}%
\pgfpathrectangle{\pgfqpoint{1.254980in}{0.150000in}}{\pgfqpoint{5.490039in}{5.490039in}}%
\pgfusepath{clip}%
\pgfsetbuttcap%
\pgfsetroundjoin%
\definecolor{currentfill}{rgb}{0.227802,0.326594,0.546532}%
\pgfsetfillcolor{currentfill}%
\pgfsetfillopacity{0.700000}%
\pgfsetlinewidth{0.000000pt}%
\definecolor{currentstroke}{rgb}{0.000000,0.000000,0.000000}%
\pgfsetstrokecolor{currentstroke}%
\pgfsetdash{}{0pt}%
\pgfpathmoveto{\pgfqpoint{5.106620in}{2.994238in}}%
\pgfpathlineto{\pgfqpoint{5.119741in}{2.991677in}}%
\pgfpathlineto{\pgfqpoint{5.132871in}{2.989234in}}%
\pgfpathlineto{\pgfqpoint{5.146009in}{2.986909in}}%
\pgfpathlineto{\pgfqpoint{5.159157in}{2.984701in}}%
\pgfpathlineto{\pgfqpoint{5.166288in}{2.995988in}}%
\pgfpathlineto{\pgfqpoint{5.173416in}{3.007348in}}%
\pgfpathlineto{\pgfqpoint{5.180542in}{3.018783in}}%
\pgfpathlineto{\pgfqpoint{5.167405in}{3.021170in}}%
\pgfpathlineto{\pgfqpoint{5.154276in}{3.023674in}}%
\pgfpathlineto{\pgfqpoint{5.141156in}{3.026295in}}%
\pgfpathlineto{\pgfqpoint{5.128044in}{3.029035in}}%
\pgfpathlineto{\pgfqpoint{5.120905in}{3.017357in}}%
\pgfpathlineto{\pgfqpoint{5.113764in}{3.005759in}}%
\pgfpathlineto{\pgfqpoint{5.106620in}{2.994238in}}%
\pgfpathclose%
\pgfusepath{fill}%
\end{pgfscope}%
\begin{pgfscope}%
\pgfpathrectangle{\pgfqpoint{1.254980in}{0.150000in}}{\pgfqpoint{5.490039in}{5.490039in}}%
\pgfusepath{clip}%
\pgfsetbuttcap%
\pgfsetroundjoin%
\definecolor{currentfill}{rgb}{0.250425,0.274290,0.533103}%
\pgfsetfillcolor{currentfill}%
\pgfsetfillopacity{0.700000}%
\pgfsetlinewidth{0.000000pt}%
\definecolor{currentstroke}{rgb}{0.000000,0.000000,0.000000}%
\pgfsetstrokecolor{currentstroke}%
\pgfsetdash{}{0pt}%
\pgfpathmoveto{\pgfqpoint{3.904359in}{2.894818in}}%
\pgfpathlineto{\pgfqpoint{3.917221in}{2.884904in}}%
\pgfpathlineto{\pgfqpoint{3.930084in}{2.875139in}}%
\pgfpathlineto{\pgfqpoint{3.942950in}{2.865524in}}%
\pgfpathlineto{\pgfqpoint{3.955818in}{2.856057in}}%
\pgfpathlineto{\pgfqpoint{3.963287in}{2.867057in}}%
\pgfpathlineto{\pgfqpoint{3.970752in}{2.878124in}}%
\pgfpathlineto{\pgfqpoint{3.978212in}{2.889259in}}%
\pgfpathlineto{\pgfqpoint{3.985668in}{2.900463in}}%
\pgfpathlineto{\pgfqpoint{3.972811in}{2.909998in}}%
\pgfpathlineto{\pgfqpoint{3.959957in}{2.919680in}}%
\pgfpathlineto{\pgfqpoint{3.947105in}{2.929512in}}%
\pgfpathlineto{\pgfqpoint{3.934255in}{2.939493in}}%
\pgfpathlineto{\pgfqpoint{3.926788in}{2.928217in}}%
\pgfpathlineto{\pgfqpoint{3.919316in}{2.917012in}}%
\pgfpathlineto{\pgfqpoint{3.911840in}{2.905880in}}%
\pgfpathlineto{\pgfqpoint{3.904359in}{2.894818in}}%
\pgfpathclose%
\pgfusepath{fill}%
\end{pgfscope}%
\begin{pgfscope}%
\pgfpathrectangle{\pgfqpoint{1.254980in}{0.150000in}}{\pgfqpoint{5.490039in}{5.490039in}}%
\pgfusepath{clip}%
\pgfsetbuttcap%
\pgfsetroundjoin%
\definecolor{currentfill}{rgb}{0.260571,0.246922,0.522828}%
\pgfsetfillcolor{currentfill}%
\pgfsetfillopacity{0.700000}%
\pgfsetlinewidth{0.000000pt}%
\definecolor{currentstroke}{rgb}{0.000000,0.000000,0.000000}%
\pgfsetstrokecolor{currentstroke}%
\pgfsetdash{}{0pt}%
\pgfpathmoveto{\pgfqpoint{4.435302in}{2.823046in}}%
\pgfpathlineto{\pgfqpoint{4.448242in}{2.817251in}}%
\pgfpathlineto{\pgfqpoint{4.461189in}{2.811587in}}%
\pgfpathlineto{\pgfqpoint{4.474141in}{2.806054in}}%
\pgfpathlineto{\pgfqpoint{4.487099in}{2.800652in}}%
\pgfpathlineto{\pgfqpoint{4.494418in}{2.811856in}}%
\pgfpathlineto{\pgfqpoint{4.501733in}{2.823115in}}%
\pgfpathlineto{\pgfqpoint{4.509045in}{2.834430in}}%
\pgfpathlineto{\pgfqpoint{4.516353in}{2.845802in}}%
\pgfpathlineto{\pgfqpoint{4.503406in}{2.851335in}}%
\pgfpathlineto{\pgfqpoint{4.490464in}{2.856998in}}%
\pgfpathlineto{\pgfqpoint{4.477528in}{2.862792in}}%
\pgfpathlineto{\pgfqpoint{4.464598in}{2.868717in}}%
\pgfpathlineto{\pgfqpoint{4.457279in}{2.857209in}}%
\pgfpathlineto{\pgfqpoint{4.449957in}{2.845762in}}%
\pgfpathlineto{\pgfqpoint{4.442631in}{2.834375in}}%
\pgfpathlineto{\pgfqpoint{4.435302in}{2.823046in}}%
\pgfpathclose%
\pgfusepath{fill}%
\end{pgfscope}%
\begin{pgfscope}%
\pgfpathrectangle{\pgfqpoint{1.254980in}{0.150000in}}{\pgfqpoint{5.490039in}{5.490039in}}%
\pgfusepath{clip}%
\pgfsetbuttcap%
\pgfsetroundjoin%
\definecolor{currentfill}{rgb}{0.233603,0.313828,0.543914}%
\pgfsetfillcolor{currentfill}%
\pgfsetfillopacity{0.700000}%
\pgfsetlinewidth{0.000000pt}%
\definecolor{currentstroke}{rgb}{0.000000,0.000000,0.000000}%
\pgfsetstrokecolor{currentstroke}%
\pgfsetdash{}{0pt}%
\pgfpathmoveto{\pgfqpoint{5.025573in}{2.959411in}}%
\pgfpathlineto{\pgfqpoint{5.038673in}{2.956599in}}%
\pgfpathlineto{\pgfqpoint{5.051780in}{2.953907in}}%
\pgfpathlineto{\pgfqpoint{5.064897in}{2.951334in}}%
\pgfpathlineto{\pgfqpoint{5.078022in}{2.948879in}}%
\pgfpathlineto{\pgfqpoint{5.085175in}{2.960115in}}%
\pgfpathlineto{\pgfqpoint{5.092326in}{2.971418in}}%
\pgfpathlineto{\pgfqpoint{5.099474in}{2.982792in}}%
\pgfpathlineto{\pgfqpoint{5.106620in}{2.994238in}}%
\pgfpathlineto{\pgfqpoint{5.093508in}{2.996917in}}%
\pgfpathlineto{\pgfqpoint{5.080404in}{2.999715in}}%
\pgfpathlineto{\pgfqpoint{5.067309in}{3.002631in}}%
\pgfpathlineto{\pgfqpoint{5.054222in}{3.005667in}}%
\pgfpathlineto{\pgfqpoint{5.047063in}{2.993991in}}%
\pgfpathlineto{\pgfqpoint{5.039902in}{2.982391in}}%
\pgfpathlineto{\pgfqpoint{5.032739in}{2.970865in}}%
\pgfpathlineto{\pgfqpoint{5.025573in}{2.959411in}}%
\pgfpathclose%
\pgfusepath{fill}%
\end{pgfscope}%
\begin{pgfscope}%
\pgfpathrectangle{\pgfqpoint{1.254980in}{0.150000in}}{\pgfqpoint{5.490039in}{5.490039in}}%
\pgfusepath{clip}%
\pgfsetbuttcap%
\pgfsetroundjoin%
\definecolor{currentfill}{rgb}{0.255645,0.260703,0.528312}%
\pgfsetfillcolor{currentfill}%
\pgfsetfillopacity{0.700000}%
\pgfsetlinewidth{0.000000pt}%
\definecolor{currentstroke}{rgb}{0.000000,0.000000,0.000000}%
\pgfsetstrokecolor{currentstroke}%
\pgfsetdash{}{0pt}%
\pgfpathmoveto{\pgfqpoint{4.649266in}{2.851104in}}%
\pgfpathlineto{\pgfqpoint{4.662260in}{2.846579in}}%
\pgfpathlineto{\pgfqpoint{4.675260in}{2.842181in}}%
\pgfpathlineto{\pgfqpoint{4.688267in}{2.837908in}}%
\pgfpathlineto{\pgfqpoint{4.701280in}{2.833760in}}%
\pgfpathlineto{\pgfqpoint{4.708540in}{2.844975in}}%
\pgfpathlineto{\pgfqpoint{4.715796in}{2.856245in}}%
\pgfpathlineto{\pgfqpoint{4.723048in}{2.867572in}}%
\pgfpathlineto{\pgfqpoint{4.730297in}{2.878957in}}%
\pgfpathlineto{\pgfqpoint{4.717294in}{2.883267in}}%
\pgfpathlineto{\pgfqpoint{4.704298in}{2.887701in}}%
\pgfpathlineto{\pgfqpoint{4.691309in}{2.892261in}}%
\pgfpathlineto{\pgfqpoint{4.678327in}{2.896948in}}%
\pgfpathlineto{\pgfqpoint{4.671066in}{2.885395in}}%
\pgfpathlineto{\pgfqpoint{4.663803in}{2.873904in}}%
\pgfpathlineto{\pgfqpoint{4.656536in}{2.862475in}}%
\pgfpathlineto{\pgfqpoint{4.649266in}{2.851104in}}%
\pgfpathclose%
\pgfusepath{fill}%
\end{pgfscope}%
\begin{pgfscope}%
\pgfpathrectangle{\pgfqpoint{1.254980in}{0.150000in}}{\pgfqpoint{5.490039in}{5.490039in}}%
\pgfusepath{clip}%
\pgfsetbuttcap%
\pgfsetroundjoin%
\definecolor{currentfill}{rgb}{0.233603,0.313828,0.543914}%
\pgfsetfillcolor{currentfill}%
\pgfsetfillopacity{0.700000}%
\pgfsetlinewidth{0.000000pt}%
\definecolor{currentstroke}{rgb}{0.000000,0.000000,0.000000}%
\pgfsetstrokecolor{currentstroke}%
\pgfsetdash{}{0pt}%
\pgfpathmoveto{\pgfqpoint{3.720025in}{2.981557in}}%
\pgfpathlineto{\pgfqpoint{3.732884in}{2.969840in}}%
\pgfpathlineto{\pgfqpoint{3.745743in}{2.958282in}}%
\pgfpathlineto{\pgfqpoint{3.758602in}{2.946882in}}%
\pgfpathlineto{\pgfqpoint{3.771463in}{2.935639in}}%
\pgfpathlineto{\pgfqpoint{3.778986in}{2.946517in}}%
\pgfpathlineto{\pgfqpoint{3.786505in}{2.957470in}}%
\pgfpathlineto{\pgfqpoint{3.794019in}{2.968499in}}%
\pgfpathlineto{\pgfqpoint{3.801528in}{2.979606in}}%
\pgfpathlineto{\pgfqpoint{3.788680in}{2.990900in}}%
\pgfpathlineto{\pgfqpoint{3.775832in}{3.002351in}}%
\pgfpathlineto{\pgfqpoint{3.762986in}{3.013961in}}%
\pgfpathlineto{\pgfqpoint{3.750140in}{3.025729in}}%
\pgfpathlineto{\pgfqpoint{3.742618in}{3.014565in}}%
\pgfpathlineto{\pgfqpoint{3.735092in}{3.003482in}}%
\pgfpathlineto{\pgfqpoint{3.727561in}{2.992480in}}%
\pgfpathlineto{\pgfqpoint{3.720025in}{2.981557in}}%
\pgfpathclose%
\pgfusepath{fill}%
\end{pgfscope}%
\begin{pgfscope}%
\pgfpathrectangle{\pgfqpoint{1.254980in}{0.150000in}}{\pgfqpoint{5.490039in}{5.490039in}}%
\pgfusepath{clip}%
\pgfsetbuttcap%
\pgfsetroundjoin%
\definecolor{currentfill}{rgb}{0.131172,0.555899,0.552459}%
\pgfsetfillcolor{currentfill}%
\pgfsetfillopacity{0.700000}%
\pgfsetlinewidth{0.000000pt}%
\definecolor{currentstroke}{rgb}{0.000000,0.000000,0.000000}%
\pgfsetstrokecolor{currentstroke}%
\pgfsetdash{}{0pt}%
\pgfpathmoveto{\pgfqpoint{3.204599in}{3.591396in}}%
\pgfpathlineto{\pgfqpoint{3.217548in}{3.572510in}}%
\pgfpathlineto{\pgfqpoint{3.230492in}{3.553826in}}%
\pgfpathlineto{\pgfqpoint{3.243431in}{3.535344in}}%
\pgfpathlineto{\pgfqpoint{3.256365in}{3.517061in}}%
\pgfpathlineto{\pgfqpoint{3.264025in}{3.528389in}}%
\pgfpathlineto{\pgfqpoint{3.271680in}{3.539834in}}%
\pgfpathlineto{\pgfqpoint{3.279329in}{3.551394in}}%
\pgfpathlineto{\pgfqpoint{3.286972in}{3.563072in}}%
\pgfpathlineto{\pgfqpoint{3.274053in}{3.581407in}}%
\pgfpathlineto{\pgfqpoint{3.261129in}{3.599940in}}%
\pgfpathlineto{\pgfqpoint{3.248201in}{3.618675in}}%
\pgfpathlineto{\pgfqpoint{3.235267in}{3.637612in}}%
\pgfpathlineto{\pgfqpoint{3.227609in}{3.625876in}}%
\pgfpathlineto{\pgfqpoint{3.219945in}{3.614263in}}%
\pgfpathlineto{\pgfqpoint{3.212275in}{3.602769in}}%
\pgfpathlineto{\pgfqpoint{3.204599in}{3.591396in}}%
\pgfpathclose%
\pgfusepath{fill}%
\end{pgfscope}%
\begin{pgfscope}%
\pgfpathrectangle{\pgfqpoint{1.254980in}{0.150000in}}{\pgfqpoint{5.490039in}{5.490039in}}%
\pgfusepath{clip}%
\pgfsetbuttcap%
\pgfsetroundjoin%
\definecolor{currentfill}{rgb}{0.241237,0.296485,0.539709}%
\pgfsetfillcolor{currentfill}%
\pgfsetfillopacity{0.700000}%
\pgfsetlinewidth{0.000000pt}%
\definecolor{currentstroke}{rgb}{0.000000,0.000000,0.000000}%
\pgfsetstrokecolor{currentstroke}%
\pgfsetdash{}{0pt}%
\pgfpathmoveto{\pgfqpoint{4.944521in}{2.925876in}}%
\pgfpathlineto{\pgfqpoint{4.957600in}{2.922793in}}%
\pgfpathlineto{\pgfqpoint{4.970686in}{2.919831in}}%
\pgfpathlineto{\pgfqpoint{4.983781in}{2.916989in}}%
\pgfpathlineto{\pgfqpoint{4.996884in}{2.914267in}}%
\pgfpathlineto{\pgfqpoint{5.004060in}{2.925457in}}%
\pgfpathlineto{\pgfqpoint{5.011234in}{2.936709in}}%
\pgfpathlineto{\pgfqpoint{5.018405in}{2.948026in}}%
\pgfpathlineto{\pgfqpoint{5.025573in}{2.959411in}}%
\pgfpathlineto{\pgfqpoint{5.012482in}{2.962342in}}%
\pgfpathlineto{\pgfqpoint{4.999400in}{2.965392in}}%
\pgfpathlineto{\pgfqpoint{4.986325in}{2.968563in}}%
\pgfpathlineto{\pgfqpoint{4.973259in}{2.971855in}}%
\pgfpathlineto{\pgfqpoint{4.966078in}{2.960255in}}%
\pgfpathlineto{\pgfqpoint{4.958895in}{2.948727in}}%
\pgfpathlineto{\pgfqpoint{4.951710in}{2.937268in}}%
\pgfpathlineto{\pgfqpoint{4.944521in}{2.925876in}}%
\pgfpathclose%
\pgfusepath{fill}%
\end{pgfscope}%
\begin{pgfscope}%
\pgfpathrectangle{\pgfqpoint{1.254980in}{0.150000in}}{\pgfqpoint{5.490039in}{5.490039in}}%
\pgfusepath{clip}%
\pgfsetbuttcap%
\pgfsetroundjoin%
\definecolor{currentfill}{rgb}{0.260571,0.246922,0.522828}%
\pgfsetfillcolor{currentfill}%
\pgfsetfillopacity{0.700000}%
\pgfsetlinewidth{0.000000pt}%
\definecolor{currentstroke}{rgb}{0.000000,0.000000,0.000000}%
\pgfsetstrokecolor{currentstroke}%
\pgfsetdash{}{0pt}%
\pgfpathmoveto{\pgfqpoint{4.088618in}{2.829444in}}%
\pgfpathlineto{\pgfqpoint{4.101501in}{2.821214in}}%
\pgfpathlineto{\pgfqpoint{4.114387in}{2.813126in}}%
\pgfpathlineto{\pgfqpoint{4.127276in}{2.805179in}}%
\pgfpathlineto{\pgfqpoint{4.140170in}{2.797373in}}%
\pgfpathlineto{\pgfqpoint{4.147588in}{2.808415in}}%
\pgfpathlineto{\pgfqpoint{4.155003in}{2.819516in}}%
\pgfpathlineto{\pgfqpoint{4.162413in}{2.830676in}}%
\pgfpathlineto{\pgfqpoint{4.169819in}{2.841898in}}%
\pgfpathlineto{\pgfqpoint{4.156937in}{2.849788in}}%
\pgfpathlineto{\pgfqpoint{4.144058in}{2.857818in}}%
\pgfpathlineto{\pgfqpoint{4.131183in}{2.865989in}}%
\pgfpathlineto{\pgfqpoint{4.118311in}{2.874302in}}%
\pgfpathlineto{\pgfqpoint{4.110894in}{2.862991in}}%
\pgfpathlineto{\pgfqpoint{4.103473in}{2.851745in}}%
\pgfpathlineto{\pgfqpoint{4.096048in}{2.840563in}}%
\pgfpathlineto{\pgfqpoint{4.088618in}{2.829444in}}%
\pgfpathclose%
\pgfusepath{fill}%
\end{pgfscope}%
\begin{pgfscope}%
\pgfpathrectangle{\pgfqpoint{1.254980in}{0.150000in}}{\pgfqpoint{5.490039in}{5.490039in}}%
\pgfusepath{clip}%
\pgfsetbuttcap%
\pgfsetroundjoin%
\definecolor{currentfill}{rgb}{0.262138,0.242286,0.520837}%
\pgfsetfillcolor{currentfill}%
\pgfsetfillopacity{0.700000}%
\pgfsetlinewidth{0.000000pt}%
\definecolor{currentstroke}{rgb}{0.000000,0.000000,0.000000}%
\pgfsetstrokecolor{currentstroke}%
\pgfsetdash{}{0pt}%
\pgfpathmoveto{\pgfqpoint{4.221389in}{2.811736in}}%
\pgfpathlineto{\pgfqpoint{4.234291in}{2.804541in}}%
\pgfpathlineto{\pgfqpoint{4.247198in}{2.797484in}}%
\pgfpathlineto{\pgfqpoint{4.260110in}{2.790563in}}%
\pgfpathlineto{\pgfqpoint{4.273026in}{2.783778in}}%
\pgfpathlineto{\pgfqpoint{4.280407in}{2.794877in}}%
\pgfpathlineto{\pgfqpoint{4.287784in}{2.806030in}}%
\pgfpathlineto{\pgfqpoint{4.295157in}{2.817241in}}%
\pgfpathlineto{\pgfqpoint{4.302526in}{2.828509in}}%
\pgfpathlineto{\pgfqpoint{4.289621in}{2.835393in}}%
\pgfpathlineto{\pgfqpoint{4.276720in}{2.842412in}}%
\pgfpathlineto{\pgfqpoint{4.263823in}{2.849569in}}%
\pgfpathlineto{\pgfqpoint{4.250931in}{2.856862in}}%
\pgfpathlineto{\pgfqpoint{4.243551in}{2.845489in}}%
\pgfpathlineto{\pgfqpoint{4.236168in}{2.834178in}}%
\pgfpathlineto{\pgfqpoint{4.228780in}{2.822927in}}%
\pgfpathlineto{\pgfqpoint{4.221389in}{2.811736in}}%
\pgfpathclose%
\pgfusepath{fill}%
\end{pgfscope}%
\begin{pgfscope}%
\pgfpathrectangle{\pgfqpoint{1.254980in}{0.150000in}}{\pgfqpoint{5.490039in}{5.490039in}}%
\pgfusepath{clip}%
\pgfsetbuttcap%
\pgfsetroundjoin%
\definecolor{currentfill}{rgb}{0.258965,0.251537,0.524736}%
\pgfsetfillcolor{currentfill}%
\pgfsetfillopacity{0.700000}%
\pgfsetlinewidth{0.000000pt}%
\definecolor{currentstroke}{rgb}{0.000000,0.000000,0.000000}%
\pgfsetstrokecolor{currentstroke}%
\pgfsetdash{}{0pt}%
\pgfpathmoveto{\pgfqpoint{4.568203in}{2.824966in}}%
\pgfpathlineto{\pgfqpoint{4.581180in}{2.820078in}}%
\pgfpathlineto{\pgfqpoint{4.594165in}{2.815318in}}%
\pgfpathlineto{\pgfqpoint{4.607155in}{2.810686in}}%
\pgfpathlineto{\pgfqpoint{4.620152in}{2.806180in}}%
\pgfpathlineto{\pgfqpoint{4.627436in}{2.817331in}}%
\pgfpathlineto{\pgfqpoint{4.634716in}{2.828534in}}%
\pgfpathlineto{\pgfqpoint{4.641993in}{2.839791in}}%
\pgfpathlineto{\pgfqpoint{4.649266in}{2.851104in}}%
\pgfpathlineto{\pgfqpoint{4.636280in}{2.855756in}}%
\pgfpathlineto{\pgfqpoint{4.623300in}{2.860534in}}%
\pgfpathlineto{\pgfqpoint{4.610326in}{2.865440in}}%
\pgfpathlineto{\pgfqpoint{4.597359in}{2.870473in}}%
\pgfpathlineto{\pgfqpoint{4.590075in}{2.859009in}}%
\pgfpathlineto{\pgfqpoint{4.582788in}{2.847603in}}%
\pgfpathlineto{\pgfqpoint{4.575497in}{2.836256in}}%
\pgfpathlineto{\pgfqpoint{4.568203in}{2.824966in}}%
\pgfpathclose%
\pgfusepath{fill}%
\end{pgfscope}%
\begin{pgfscope}%
\pgfpathrectangle{\pgfqpoint{1.254980in}{0.150000in}}{\pgfqpoint{5.490039in}{5.490039in}}%
\pgfusepath{clip}%
\pgfsetbuttcap%
\pgfsetroundjoin%
\definecolor{currentfill}{rgb}{0.246811,0.283237,0.535941}%
\pgfsetfillcolor{currentfill}%
\pgfsetfillopacity{0.700000}%
\pgfsetlinewidth{0.000000pt}%
\definecolor{currentstroke}{rgb}{0.000000,0.000000,0.000000}%
\pgfsetstrokecolor{currentstroke}%
\pgfsetdash{}{0pt}%
\pgfpathmoveto{\pgfqpoint{4.863459in}{2.893702in}}%
\pgfpathlineto{\pgfqpoint{4.876517in}{2.890327in}}%
\pgfpathlineto{\pgfqpoint{4.889583in}{2.887074in}}%
\pgfpathlineto{\pgfqpoint{4.902657in}{2.883943in}}%
\pgfpathlineto{\pgfqpoint{4.915739in}{2.880933in}}%
\pgfpathlineto{\pgfqpoint{4.922939in}{2.892079in}}%
\pgfpathlineto{\pgfqpoint{4.930136in}{2.903283in}}%
\pgfpathlineto{\pgfqpoint{4.937330in}{2.914548in}}%
\pgfpathlineto{\pgfqpoint{4.944521in}{2.925876in}}%
\pgfpathlineto{\pgfqpoint{4.931451in}{2.929079in}}%
\pgfpathlineto{\pgfqpoint{4.918389in}{2.932404in}}%
\pgfpathlineto{\pgfqpoint{4.905334in}{2.935850in}}%
\pgfpathlineto{\pgfqpoint{4.892287in}{2.939417in}}%
\pgfpathlineto{\pgfqpoint{4.885085in}{2.927891in}}%
\pgfpathlineto{\pgfqpoint{4.877879in}{2.916431in}}%
\pgfpathlineto{\pgfqpoint{4.870670in}{2.905035in}}%
\pgfpathlineto{\pgfqpoint{4.863459in}{2.893702in}}%
\pgfpathclose%
\pgfusepath{fill}%
\end{pgfscope}%
\begin{pgfscope}%
\pgfpathrectangle{\pgfqpoint{1.254980in}{0.150000in}}{\pgfqpoint{5.490039in}{5.490039in}}%
\pgfusepath{clip}%
\pgfsetbuttcap%
\pgfsetroundjoin%
\definecolor{currentfill}{rgb}{0.255645,0.260703,0.528312}%
\pgfsetfillcolor{currentfill}%
\pgfsetfillopacity{0.700000}%
\pgfsetlinewidth{0.000000pt}%
\definecolor{currentstroke}{rgb}{0.000000,0.000000,0.000000}%
\pgfsetstrokecolor{currentstroke}%
\pgfsetdash{}{0pt}%
\pgfpathmoveto{\pgfqpoint{3.955818in}{2.856057in}}%
\pgfpathlineto{\pgfqpoint{3.968688in}{2.846737in}}%
\pgfpathlineto{\pgfqpoint{3.981561in}{2.837564in}}%
\pgfpathlineto{\pgfqpoint{3.994436in}{2.828538in}}%
\pgfpathlineto{\pgfqpoint{4.007315in}{2.819657in}}%
\pgfpathlineto{\pgfqpoint{4.014772in}{2.830596in}}%
\pgfpathlineto{\pgfqpoint{4.022226in}{2.841598in}}%
\pgfpathlineto{\pgfqpoint{4.029675in}{2.852663in}}%
\pgfpathlineto{\pgfqpoint{4.037120in}{2.863794in}}%
\pgfpathlineto{\pgfqpoint{4.024253in}{2.872742in}}%
\pgfpathlineto{\pgfqpoint{4.011389in}{2.881836in}}%
\pgfpathlineto{\pgfqpoint{3.998527in}{2.891076in}}%
\pgfpathlineto{\pgfqpoint{3.985668in}{2.900463in}}%
\pgfpathlineto{\pgfqpoint{3.978212in}{2.889259in}}%
\pgfpathlineto{\pgfqpoint{3.970752in}{2.878124in}}%
\pgfpathlineto{\pgfqpoint{3.963287in}{2.867057in}}%
\pgfpathlineto{\pgfqpoint{3.955818in}{2.856057in}}%
\pgfpathclose%
\pgfusepath{fill}%
\end{pgfscope}%
\begin{pgfscope}%
\pgfpathrectangle{\pgfqpoint{1.254980in}{0.150000in}}{\pgfqpoint{5.490039in}{5.490039in}}%
\pgfusepath{clip}%
\pgfsetbuttcap%
\pgfsetroundjoin%
\definecolor{currentfill}{rgb}{0.241237,0.296485,0.539709}%
\pgfsetfillcolor{currentfill}%
\pgfsetfillopacity{0.700000}%
\pgfsetlinewidth{0.000000pt}%
\definecolor{currentstroke}{rgb}{0.000000,0.000000,0.000000}%
\pgfsetstrokecolor{currentstroke}%
\pgfsetdash{}{0pt}%
\pgfpathmoveto{\pgfqpoint{3.771463in}{2.935639in}}%
\pgfpathlineto{\pgfqpoint{3.784324in}{2.924553in}}%
\pgfpathlineto{\pgfqpoint{3.797187in}{2.913622in}}%
\pgfpathlineto{\pgfqpoint{3.810050in}{2.902846in}}%
\pgfpathlineto{\pgfqpoint{3.822915in}{2.892224in}}%
\pgfpathlineto{\pgfqpoint{3.830426in}{2.903056in}}%
\pgfpathlineto{\pgfqpoint{3.837933in}{2.913960in}}%
\pgfpathlineto{\pgfqpoint{3.845435in}{2.924936in}}%
\pgfpathlineto{\pgfqpoint{3.852932in}{2.935985in}}%
\pgfpathlineto{\pgfqpoint{3.840079in}{2.946659in}}%
\pgfpathlineto{\pgfqpoint{3.827227in}{2.957486in}}%
\pgfpathlineto{\pgfqpoint{3.814377in}{2.968468in}}%
\pgfpathlineto{\pgfqpoint{3.801528in}{2.979606in}}%
\pgfpathlineto{\pgfqpoint{3.794019in}{2.968499in}}%
\pgfpathlineto{\pgfqpoint{3.786505in}{2.957470in}}%
\pgfpathlineto{\pgfqpoint{3.778986in}{2.946517in}}%
\pgfpathlineto{\pgfqpoint{3.771463in}{2.935639in}}%
\pgfpathclose%
\pgfusepath{fill}%
\end{pgfscope}%
\begin{pgfscope}%
\pgfpathrectangle{\pgfqpoint{1.254980in}{0.150000in}}{\pgfqpoint{5.490039in}{5.490039in}}%
\pgfusepath{clip}%
\pgfsetbuttcap%
\pgfsetroundjoin%
\definecolor{currentfill}{rgb}{0.263663,0.237631,0.518762}%
\pgfsetfillcolor{currentfill}%
\pgfsetfillopacity{0.700000}%
\pgfsetlinewidth{0.000000pt}%
\definecolor{currentstroke}{rgb}{0.000000,0.000000,0.000000}%
\pgfsetstrokecolor{currentstroke}%
\pgfsetdash{}{0pt}%
\pgfpathmoveto{\pgfqpoint{4.354196in}{2.802327in}}%
\pgfpathlineto{\pgfqpoint{4.367125in}{2.796117in}}%
\pgfpathlineto{\pgfqpoint{4.380060in}{2.790040in}}%
\pgfpathlineto{\pgfqpoint{4.393001in}{2.784096in}}%
\pgfpathlineto{\pgfqpoint{4.405946in}{2.778284in}}%
\pgfpathlineto{\pgfqpoint{4.413291in}{2.789394in}}%
\pgfpathlineto{\pgfqpoint{4.420632in}{2.800557in}}%
\pgfpathlineto{\pgfqpoint{4.427968in}{2.811774in}}%
\pgfpathlineto{\pgfqpoint{4.435302in}{2.823046in}}%
\pgfpathlineto{\pgfqpoint{4.422366in}{2.828973in}}%
\pgfpathlineto{\pgfqpoint{4.409436in}{2.835032in}}%
\pgfpathlineto{\pgfqpoint{4.396512in}{2.841223in}}%
\pgfpathlineto{\pgfqpoint{4.383592in}{2.847548in}}%
\pgfpathlineto{\pgfqpoint{4.376249in}{2.836155in}}%
\pgfpathlineto{\pgfqpoint{4.368902in}{2.824822in}}%
\pgfpathlineto{\pgfqpoint{4.361550in}{2.813546in}}%
\pgfpathlineto{\pgfqpoint{4.354196in}{2.802327in}}%
\pgfpathclose%
\pgfusepath{fill}%
\end{pgfscope}%
\begin{pgfscope}%
\pgfpathrectangle{\pgfqpoint{1.254980in}{0.150000in}}{\pgfqpoint{5.490039in}{5.490039in}}%
\pgfusepath{clip}%
\pgfsetbuttcap%
\pgfsetroundjoin%
\definecolor{currentfill}{rgb}{0.121831,0.589055,0.545623}%
\pgfsetfillcolor{currentfill}%
\pgfsetfillopacity{0.700000}%
\pgfsetlinewidth{0.000000pt}%
\definecolor{currentstroke}{rgb}{0.000000,0.000000,0.000000}%
\pgfsetstrokecolor{currentstroke}%
\pgfsetdash{}{0pt}%
\pgfpathmoveto{\pgfqpoint{3.152746in}{3.668989in}}%
\pgfpathlineto{\pgfqpoint{3.165718in}{3.649280in}}%
\pgfpathlineto{\pgfqpoint{3.178684in}{3.629779in}}%
\pgfpathlineto{\pgfqpoint{3.191644in}{3.610485in}}%
\pgfpathlineto{\pgfqpoint{3.204599in}{3.591396in}}%
\pgfpathlineto{\pgfqpoint{3.212275in}{3.602769in}}%
\pgfpathlineto{\pgfqpoint{3.219945in}{3.614263in}}%
\pgfpathlineto{\pgfqpoint{3.227609in}{3.625876in}}%
\pgfpathlineto{\pgfqpoint{3.235267in}{3.637612in}}%
\pgfpathlineto{\pgfqpoint{3.222328in}{3.656752in}}%
\pgfpathlineto{\pgfqpoint{3.209383in}{3.676098in}}%
\pgfpathlineto{\pgfqpoint{3.196433in}{3.695650in}}%
\pgfpathlineto{\pgfqpoint{3.183477in}{3.715410in}}%
\pgfpathlineto{\pgfqpoint{3.175804in}{3.703618in}}%
\pgfpathlineto{\pgfqpoint{3.168124in}{3.691950in}}%
\pgfpathlineto{\pgfqpoint{3.160439in}{3.680408in}}%
\pgfpathlineto{\pgfqpoint{3.152746in}{3.668989in}}%
\pgfpathclose%
\pgfusepath{fill}%
\end{pgfscope}%
\begin{pgfscope}%
\pgfpathrectangle{\pgfqpoint{1.254980in}{0.150000in}}{\pgfqpoint{5.490039in}{5.490039in}}%
\pgfusepath{clip}%
\pgfsetbuttcap%
\pgfsetroundjoin%
\definecolor{currentfill}{rgb}{0.188923,0.410910,0.556326}%
\pgfsetfillcolor{currentfill}%
\pgfsetfillopacity{0.700000}%
\pgfsetlinewidth{0.000000pt}%
\definecolor{currentstroke}{rgb}{0.000000,0.000000,0.000000}%
\pgfsetstrokecolor{currentstroke}%
\pgfsetdash{}{0pt}%
\pgfpathmoveto{\pgfqpoint{3.432307in}{3.207001in}}%
\pgfpathlineto{\pgfqpoint{3.445197in}{3.191952in}}%
\pgfpathlineto{\pgfqpoint{3.458084in}{3.177081in}}%
\pgfpathlineto{\pgfqpoint{3.470968in}{3.162386in}}%
\pgfpathlineto{\pgfqpoint{3.483851in}{3.147867in}}%
\pgfpathlineto{\pgfqpoint{3.491462in}{3.158597in}}%
\pgfpathlineto{\pgfqpoint{3.499067in}{3.169419in}}%
\pgfpathlineto{\pgfqpoint{3.506666in}{3.180334in}}%
\pgfpathlineto{\pgfqpoint{3.514260in}{3.191343in}}%
\pgfpathlineto{\pgfqpoint{3.501391in}{3.205897in}}%
\pgfpathlineto{\pgfqpoint{3.488520in}{3.220627in}}%
\pgfpathlineto{\pgfqpoint{3.475648in}{3.235533in}}%
\pgfpathlineto{\pgfqpoint{3.462773in}{3.250617in}}%
\pgfpathlineto{\pgfqpoint{3.455165in}{3.239567in}}%
\pgfpathlineto{\pgfqpoint{3.447551in}{3.228615in}}%
\pgfpathlineto{\pgfqpoint{3.439932in}{3.217760in}}%
\pgfpathlineto{\pgfqpoint{3.432307in}{3.207001in}}%
\pgfpathclose%
\pgfusepath{fill}%
\end{pgfscope}%
\begin{pgfscope}%
\pgfpathrectangle{\pgfqpoint{1.254980in}{0.150000in}}{\pgfqpoint{5.490039in}{5.490039in}}%
\pgfusepath{clip}%
\pgfsetbuttcap%
\pgfsetroundjoin%
\definecolor{currentfill}{rgb}{0.177423,0.437527,0.557565}%
\pgfsetfillcolor{currentfill}%
\pgfsetfillopacity{0.700000}%
\pgfsetlinewidth{0.000000pt}%
\definecolor{currentstroke}{rgb}{0.000000,0.000000,0.000000}%
\pgfsetstrokecolor{currentstroke}%
\pgfsetdash{}{0pt}%
\pgfpathmoveto{\pgfqpoint{3.380726in}{3.268997in}}%
\pgfpathlineto{\pgfqpoint{3.393625in}{3.253226in}}%
\pgfpathlineto{\pgfqpoint{3.406522in}{3.237637in}}%
\pgfpathlineto{\pgfqpoint{3.419416in}{3.222229in}}%
\pgfpathlineto{\pgfqpoint{3.432307in}{3.207001in}}%
\pgfpathlineto{\pgfqpoint{3.439932in}{3.217760in}}%
\pgfpathlineto{\pgfqpoint{3.447551in}{3.228615in}}%
\pgfpathlineto{\pgfqpoint{3.455165in}{3.239567in}}%
\pgfpathlineto{\pgfqpoint{3.462773in}{3.250617in}}%
\pgfpathlineto{\pgfqpoint{3.449895in}{3.265880in}}%
\pgfpathlineto{\pgfqpoint{3.437016in}{3.281323in}}%
\pgfpathlineto{\pgfqpoint{3.424134in}{3.296946in}}%
\pgfpathlineto{\pgfqpoint{3.411249in}{3.312752in}}%
\pgfpathlineto{\pgfqpoint{3.403627in}{3.301661in}}%
\pgfpathlineto{\pgfqpoint{3.395999in}{3.290672in}}%
\pgfpathlineto{\pgfqpoint{3.388365in}{3.279784in}}%
\pgfpathlineto{\pgfqpoint{3.380726in}{3.268997in}}%
\pgfpathclose%
\pgfusepath{fill}%
\end{pgfscope}%
\begin{pgfscope}%
\pgfpathrectangle{\pgfqpoint{1.254980in}{0.150000in}}{\pgfqpoint{5.490039in}{5.490039in}}%
\pgfusepath{clip}%
\pgfsetbuttcap%
\pgfsetroundjoin%
\definecolor{currentfill}{rgb}{0.252194,0.269783,0.531579}%
\pgfsetfillcolor{currentfill}%
\pgfsetfillopacity{0.700000}%
\pgfsetlinewidth{0.000000pt}%
\definecolor{currentstroke}{rgb}{0.000000,0.000000,0.000000}%
\pgfsetstrokecolor{currentstroke}%
\pgfsetdash{}{0pt}%
\pgfpathmoveto{\pgfqpoint{4.782380in}{2.862967in}}%
\pgfpathlineto{\pgfqpoint{4.795419in}{2.859279in}}%
\pgfpathlineto{\pgfqpoint{4.808466in}{2.855714in}}%
\pgfpathlineto{\pgfqpoint{4.821520in}{2.852273in}}%
\pgfpathlineto{\pgfqpoint{4.834581in}{2.848953in}}%
\pgfpathlineto{\pgfqpoint{4.841806in}{2.860056in}}%
\pgfpathlineto{\pgfqpoint{4.849026in}{2.871214in}}%
\pgfpathlineto{\pgfqpoint{4.856244in}{2.882429in}}%
\pgfpathlineto{\pgfqpoint{4.863459in}{2.893702in}}%
\pgfpathlineto{\pgfqpoint{4.850408in}{2.897198in}}%
\pgfpathlineto{\pgfqpoint{4.837365in}{2.900818in}}%
\pgfpathlineto{\pgfqpoint{4.824330in}{2.904560in}}%
\pgfpathlineto{\pgfqpoint{4.811302in}{2.908425in}}%
\pgfpathlineto{\pgfqpoint{4.804077in}{2.896969in}}%
\pgfpathlineto{\pgfqpoint{4.796848in}{2.885575in}}%
\pgfpathlineto{\pgfqpoint{4.789616in}{2.874242in}}%
\pgfpathlineto{\pgfqpoint{4.782380in}{2.862967in}}%
\pgfpathclose%
\pgfusepath{fill}%
\end{pgfscope}%
\begin{pgfscope}%
\pgfpathrectangle{\pgfqpoint{1.254980in}{0.150000in}}{\pgfqpoint{5.490039in}{5.490039in}}%
\pgfusepath{clip}%
\pgfsetbuttcap%
\pgfsetroundjoin%
\definecolor{currentfill}{rgb}{0.199430,0.387607,0.554642}%
\pgfsetfillcolor{currentfill}%
\pgfsetfillopacity{0.700000}%
\pgfsetlinewidth{0.000000pt}%
\definecolor{currentstroke}{rgb}{0.000000,0.000000,0.000000}%
\pgfsetstrokecolor{currentstroke}%
\pgfsetdash{}{0pt}%
\pgfpathmoveto{\pgfqpoint{3.483851in}{3.147867in}}%
\pgfpathlineto{\pgfqpoint{3.496733in}{3.133523in}}%
\pgfpathlineto{\pgfqpoint{3.509612in}{3.119352in}}%
\pgfpathlineto{\pgfqpoint{3.522490in}{3.105354in}}%
\pgfpathlineto{\pgfqpoint{3.535367in}{3.091527in}}%
\pgfpathlineto{\pgfqpoint{3.542964in}{3.102228in}}%
\pgfpathlineto{\pgfqpoint{3.550554in}{3.113017in}}%
\pgfpathlineto{\pgfqpoint{3.558140in}{3.123895in}}%
\pgfpathlineto{\pgfqpoint{3.565720in}{3.134863in}}%
\pgfpathlineto{\pgfqpoint{3.552857in}{3.148725in}}%
\pgfpathlineto{\pgfqpoint{3.539993in}{3.162758in}}%
\pgfpathlineto{\pgfqpoint{3.527127in}{3.176964in}}%
\pgfpathlineto{\pgfqpoint{3.514260in}{3.191343in}}%
\pgfpathlineto{\pgfqpoint{3.506666in}{3.180334in}}%
\pgfpathlineto{\pgfqpoint{3.499067in}{3.169419in}}%
\pgfpathlineto{\pgfqpoint{3.491462in}{3.158597in}}%
\pgfpathlineto{\pgfqpoint{3.483851in}{3.147867in}}%
\pgfpathclose%
\pgfusepath{fill}%
\end{pgfscope}%
\begin{pgfscope}%
\pgfpathrectangle{\pgfqpoint{1.254980in}{0.150000in}}{\pgfqpoint{5.490039in}{5.490039in}}%
\pgfusepath{clip}%
\pgfsetbuttcap%
\pgfsetroundjoin%
\definecolor{currentfill}{rgb}{0.168126,0.459988,0.558082}%
\pgfsetfillcolor{currentfill}%
\pgfsetfillopacity{0.700000}%
\pgfsetlinewidth{0.000000pt}%
\definecolor{currentstroke}{rgb}{0.000000,0.000000,0.000000}%
\pgfsetstrokecolor{currentstroke}%
\pgfsetdash{}{0pt}%
\pgfpathmoveto{\pgfqpoint{3.329097in}{3.333926in}}%
\pgfpathlineto{\pgfqpoint{3.342009in}{3.317415in}}%
\pgfpathlineto{\pgfqpoint{3.354918in}{3.301090in}}%
\pgfpathlineto{\pgfqpoint{3.367823in}{3.284951in}}%
\pgfpathlineto{\pgfqpoint{3.380726in}{3.268997in}}%
\pgfpathlineto{\pgfqpoint{3.388365in}{3.279784in}}%
\pgfpathlineto{\pgfqpoint{3.395999in}{3.290672in}}%
\pgfpathlineto{\pgfqpoint{3.403627in}{3.301661in}}%
\pgfpathlineto{\pgfqpoint{3.411249in}{3.312752in}}%
\pgfpathlineto{\pgfqpoint{3.398361in}{3.328742in}}%
\pgfpathlineto{\pgfqpoint{3.385471in}{3.344915in}}%
\pgfpathlineto{\pgfqpoint{3.372577in}{3.361274in}}%
\pgfpathlineto{\pgfqpoint{3.359679in}{3.377821in}}%
\pgfpathlineto{\pgfqpoint{3.352043in}{3.366688in}}%
\pgfpathlineto{\pgfqpoint{3.344400in}{3.355662in}}%
\pgfpathlineto{\pgfqpoint{3.336751in}{3.344742in}}%
\pgfpathlineto{\pgfqpoint{3.329097in}{3.333926in}}%
\pgfpathclose%
\pgfusepath{fill}%
\end{pgfscope}%
\begin{pgfscope}%
\pgfpathrectangle{\pgfqpoint{1.254980in}{0.150000in}}{\pgfqpoint{5.490039in}{5.490039in}}%
\pgfusepath{clip}%
\pgfsetbuttcap%
\pgfsetroundjoin%
\definecolor{currentfill}{rgb}{0.210503,0.363727,0.552206}%
\pgfsetfillcolor{currentfill}%
\pgfsetfillopacity{0.700000}%
\pgfsetlinewidth{0.000000pt}%
\definecolor{currentstroke}{rgb}{0.000000,0.000000,0.000000}%
\pgfsetstrokecolor{currentstroke}%
\pgfsetdash{}{0pt}%
\pgfpathmoveto{\pgfqpoint{3.535367in}{3.091527in}}%
\pgfpathlineto{\pgfqpoint{3.548243in}{3.077871in}}%
\pgfpathlineto{\pgfqpoint{3.561117in}{3.064385in}}%
\pgfpathlineto{\pgfqpoint{3.573991in}{3.051067in}}%
\pgfpathlineto{\pgfqpoint{3.586863in}{3.037917in}}%
\pgfpathlineto{\pgfqpoint{3.594446in}{3.048588in}}%
\pgfpathlineto{\pgfqpoint{3.602023in}{3.059344in}}%
\pgfpathlineto{\pgfqpoint{3.609595in}{3.070185in}}%
\pgfpathlineto{\pgfqpoint{3.617162in}{3.081112in}}%
\pgfpathlineto{\pgfqpoint{3.604303in}{3.094298in}}%
\pgfpathlineto{\pgfqpoint{3.591443in}{3.107651in}}%
\pgfpathlineto{\pgfqpoint{3.578582in}{3.121172in}}%
\pgfpathlineto{\pgfqpoint{3.565720in}{3.134863in}}%
\pgfpathlineto{\pgfqpoint{3.558140in}{3.123895in}}%
\pgfpathlineto{\pgfqpoint{3.550554in}{3.113017in}}%
\pgfpathlineto{\pgfqpoint{3.542964in}{3.102228in}}%
\pgfpathlineto{\pgfqpoint{3.535367in}{3.091527in}}%
\pgfpathclose%
\pgfusepath{fill}%
\end{pgfscope}%
\begin{pgfscope}%
\pgfpathrectangle{\pgfqpoint{1.254980in}{0.150000in}}{\pgfqpoint{5.490039in}{5.490039in}}%
\pgfusepath{clip}%
\pgfsetbuttcap%
\pgfsetroundjoin%
\definecolor{currentfill}{rgb}{0.262138,0.242286,0.520837}%
\pgfsetfillcolor{currentfill}%
\pgfsetfillopacity{0.700000}%
\pgfsetlinewidth{0.000000pt}%
\definecolor{currentstroke}{rgb}{0.000000,0.000000,0.000000}%
\pgfsetstrokecolor{currentstroke}%
\pgfsetdash{}{0pt}%
\pgfpathmoveto{\pgfqpoint{4.487099in}{2.800652in}}%
\pgfpathlineto{\pgfqpoint{4.500062in}{2.795379in}}%
\pgfpathlineto{\pgfqpoint{4.513032in}{2.790236in}}%
\pgfpathlineto{\pgfqpoint{4.526008in}{2.785222in}}%
\pgfpathlineto{\pgfqpoint{4.538990in}{2.780337in}}%
\pgfpathlineto{\pgfqpoint{4.546298in}{2.791417in}}%
\pgfpathlineto{\pgfqpoint{4.553603in}{2.802547in}}%
\pgfpathlineto{\pgfqpoint{4.560905in}{2.813730in}}%
\pgfpathlineto{\pgfqpoint{4.568203in}{2.824966in}}%
\pgfpathlineto{\pgfqpoint{4.555231in}{2.829981in}}%
\pgfpathlineto{\pgfqpoint{4.542266in}{2.835126in}}%
\pgfpathlineto{\pgfqpoint{4.529306in}{2.840399in}}%
\pgfpathlineto{\pgfqpoint{4.516353in}{2.845802in}}%
\pgfpathlineto{\pgfqpoint{4.509045in}{2.834430in}}%
\pgfpathlineto{\pgfqpoint{4.501733in}{2.823115in}}%
\pgfpathlineto{\pgfqpoint{4.494418in}{2.811856in}}%
\pgfpathlineto{\pgfqpoint{4.487099in}{2.800652in}}%
\pgfpathclose%
\pgfusepath{fill}%
\end{pgfscope}%
\begin{pgfscope}%
\pgfpathrectangle{\pgfqpoint{1.254980in}{0.150000in}}{\pgfqpoint{5.490039in}{5.490039in}}%
\pgfusepath{clip}%
\pgfsetbuttcap%
\pgfsetroundjoin%
\definecolor{currentfill}{rgb}{0.248629,0.278775,0.534556}%
\pgfsetfillcolor{currentfill}%
\pgfsetfillopacity{0.700000}%
\pgfsetlinewidth{0.000000pt}%
\definecolor{currentstroke}{rgb}{0.000000,0.000000,0.000000}%
\pgfsetstrokecolor{currentstroke}%
\pgfsetdash{}{0pt}%
\pgfpathmoveto{\pgfqpoint{3.822915in}{2.892224in}}%
\pgfpathlineto{\pgfqpoint{3.835782in}{2.881756in}}%
\pgfpathlineto{\pgfqpoint{3.848650in}{2.871440in}}%
\pgfpathlineto{\pgfqpoint{3.861519in}{2.861276in}}%
\pgfpathlineto{\pgfqpoint{3.874390in}{2.851263in}}%
\pgfpathlineto{\pgfqpoint{3.881890in}{2.862049in}}%
\pgfpathlineto{\pgfqpoint{3.889384in}{2.872904in}}%
\pgfpathlineto{\pgfqpoint{3.896874in}{2.883826in}}%
\pgfpathlineto{\pgfqpoint{3.904359in}{2.894818in}}%
\pgfpathlineto{\pgfqpoint{3.891500in}{2.904883in}}%
\pgfpathlineto{\pgfqpoint{3.878642in}{2.915098in}}%
\pgfpathlineto{\pgfqpoint{3.865786in}{2.925466in}}%
\pgfpathlineto{\pgfqpoint{3.852932in}{2.935985in}}%
\pgfpathlineto{\pgfqpoint{3.845435in}{2.924936in}}%
\pgfpathlineto{\pgfqpoint{3.837933in}{2.913960in}}%
\pgfpathlineto{\pgfqpoint{3.830426in}{2.903056in}}%
\pgfpathlineto{\pgfqpoint{3.822915in}{2.892224in}}%
\pgfpathclose%
\pgfusepath{fill}%
\end{pgfscope}%
\begin{pgfscope}%
\pgfpathrectangle{\pgfqpoint{1.254980in}{0.150000in}}{\pgfqpoint{5.490039in}{5.490039in}}%
\pgfusepath{clip}%
\pgfsetbuttcap%
\pgfsetroundjoin%
\definecolor{currentfill}{rgb}{0.263663,0.237631,0.518762}%
\pgfsetfillcolor{currentfill}%
\pgfsetfillopacity{0.700000}%
\pgfsetlinewidth{0.000000pt}%
\definecolor{currentstroke}{rgb}{0.000000,0.000000,0.000000}%
\pgfsetstrokecolor{currentstroke}%
\pgfsetdash{}{0pt}%
\pgfpathmoveto{\pgfqpoint{4.140170in}{2.797373in}}%
\pgfpathlineto{\pgfqpoint{4.153067in}{2.789707in}}%
\pgfpathlineto{\pgfqpoint{4.165968in}{2.782180in}}%
\pgfpathlineto{\pgfqpoint{4.178873in}{2.774792in}}%
\pgfpathlineto{\pgfqpoint{4.191782in}{2.767543in}}%
\pgfpathlineto{\pgfqpoint{4.199190in}{2.778508in}}%
\pgfpathlineto{\pgfqpoint{4.206593in}{2.789527in}}%
\pgfpathlineto{\pgfqpoint{4.213993in}{2.800603in}}%
\pgfpathlineto{\pgfqpoint{4.221389in}{2.811736in}}%
\pgfpathlineto{\pgfqpoint{4.208490in}{2.819068in}}%
\pgfpathlineto{\pgfqpoint{4.195596in}{2.826539in}}%
\pgfpathlineto{\pgfqpoint{4.182706in}{2.834149in}}%
\pgfpathlineto{\pgfqpoint{4.169819in}{2.841898in}}%
\pgfpathlineto{\pgfqpoint{4.162413in}{2.830676in}}%
\pgfpathlineto{\pgfqpoint{4.155003in}{2.819516in}}%
\pgfpathlineto{\pgfqpoint{4.147588in}{2.808415in}}%
\pgfpathlineto{\pgfqpoint{4.140170in}{2.797373in}}%
\pgfpathclose%
\pgfusepath{fill}%
\end{pgfscope}%
\begin{pgfscope}%
\pgfpathrectangle{\pgfqpoint{1.254980in}{0.150000in}}{\pgfqpoint{5.490039in}{5.490039in}}%
\pgfusepath{clip}%
\pgfsetbuttcap%
\pgfsetroundjoin%
\definecolor{currentfill}{rgb}{0.156270,0.489624,0.557936}%
\pgfsetfillcolor{currentfill}%
\pgfsetfillopacity{0.700000}%
\pgfsetlinewidth{0.000000pt}%
\definecolor{currentstroke}{rgb}{0.000000,0.000000,0.000000}%
\pgfsetstrokecolor{currentstroke}%
\pgfsetdash{}{0pt}%
\pgfpathmoveto{\pgfqpoint{3.277411in}{3.401864in}}%
\pgfpathlineto{\pgfqpoint{3.290338in}{3.384593in}}%
\pgfpathlineto{\pgfqpoint{3.303262in}{3.367514in}}%
\pgfpathlineto{\pgfqpoint{3.316181in}{3.350625in}}%
\pgfpathlineto{\pgfqpoint{3.329097in}{3.333926in}}%
\pgfpathlineto{\pgfqpoint{3.336751in}{3.344742in}}%
\pgfpathlineto{\pgfqpoint{3.344400in}{3.355662in}}%
\pgfpathlineto{\pgfqpoint{3.352043in}{3.366688in}}%
\pgfpathlineto{\pgfqpoint{3.359679in}{3.377821in}}%
\pgfpathlineto{\pgfqpoint{3.346779in}{3.394555in}}%
\pgfpathlineto{\pgfqpoint{3.333875in}{3.411478in}}%
\pgfpathlineto{\pgfqpoint{3.320967in}{3.428592in}}%
\pgfpathlineto{\pgfqpoint{3.308055in}{3.445897in}}%
\pgfpathlineto{\pgfqpoint{3.300403in}{3.434724in}}%
\pgfpathlineto{\pgfqpoint{3.292745in}{3.423661in}}%
\pgfpathlineto{\pgfqpoint{3.285081in}{3.412708in}}%
\pgfpathlineto{\pgfqpoint{3.277411in}{3.401864in}}%
\pgfpathclose%
\pgfusepath{fill}%
\end{pgfscope}%
\begin{pgfscope}%
\pgfpathrectangle{\pgfqpoint{1.254980in}{0.150000in}}{\pgfqpoint{5.490039in}{5.490039in}}%
\pgfusepath{clip}%
\pgfsetbuttcap%
\pgfsetroundjoin%
\definecolor{currentfill}{rgb}{0.220057,0.343307,0.549413}%
\pgfsetfillcolor{currentfill}%
\pgfsetfillopacity{0.700000}%
\pgfsetlinewidth{0.000000pt}%
\definecolor{currentstroke}{rgb}{0.000000,0.000000,0.000000}%
\pgfsetstrokecolor{currentstroke}%
\pgfsetdash{}{0pt}%
\pgfpathmoveto{\pgfqpoint{3.586863in}{3.037917in}}%
\pgfpathlineto{\pgfqpoint{3.599735in}{3.024934in}}%
\pgfpathlineto{\pgfqpoint{3.612607in}{3.012116in}}%
\pgfpathlineto{\pgfqpoint{3.625478in}{2.999464in}}%
\pgfpathlineto{\pgfqpoint{3.638349in}{2.986975in}}%
\pgfpathlineto{\pgfqpoint{3.645918in}{2.997617in}}%
\pgfpathlineto{\pgfqpoint{3.653482in}{3.008340in}}%
\pgfpathlineto{\pgfqpoint{3.661041in}{3.019144in}}%
\pgfpathlineto{\pgfqpoint{3.668594in}{3.030029in}}%
\pgfpathlineto{\pgfqpoint{3.655737in}{3.042553in}}%
\pgfpathlineto{\pgfqpoint{3.642879in}{3.055241in}}%
\pgfpathlineto{\pgfqpoint{3.630021in}{3.068094in}}%
\pgfpathlineto{\pgfqpoint{3.617162in}{3.081112in}}%
\pgfpathlineto{\pgfqpoint{3.609595in}{3.070185in}}%
\pgfpathlineto{\pgfqpoint{3.602023in}{3.059344in}}%
\pgfpathlineto{\pgfqpoint{3.594446in}{3.048588in}}%
\pgfpathlineto{\pgfqpoint{3.586863in}{3.037917in}}%
\pgfpathclose%
\pgfusepath{fill}%
\end{pgfscope}%
\begin{pgfscope}%
\pgfpathrectangle{\pgfqpoint{1.254980in}{0.150000in}}{\pgfqpoint{5.490039in}{5.490039in}}%
\pgfusepath{clip}%
\pgfsetbuttcap%
\pgfsetroundjoin%
\definecolor{currentfill}{rgb}{0.260571,0.246922,0.522828}%
\pgfsetfillcolor{currentfill}%
\pgfsetfillopacity{0.700000}%
\pgfsetlinewidth{0.000000pt}%
\definecolor{currentstroke}{rgb}{0.000000,0.000000,0.000000}%
\pgfsetstrokecolor{currentstroke}%
\pgfsetdash{}{0pt}%
\pgfpathmoveto{\pgfqpoint{4.007315in}{2.819657in}}%
\pgfpathlineto{\pgfqpoint{4.020196in}{2.810921in}}%
\pgfpathlineto{\pgfqpoint{4.033080in}{2.802329in}}%
\pgfpathlineto{\pgfqpoint{4.045967in}{2.793881in}}%
\pgfpathlineto{\pgfqpoint{4.058857in}{2.785577in}}%
\pgfpathlineto{\pgfqpoint{4.066304in}{2.796454in}}%
\pgfpathlineto{\pgfqpoint{4.073746in}{2.807390in}}%
\pgfpathlineto{\pgfqpoint{4.081184in}{2.818387in}}%
\pgfpathlineto{\pgfqpoint{4.088618in}{2.829444in}}%
\pgfpathlineto{\pgfqpoint{4.075739in}{2.837816in}}%
\pgfpathlineto{\pgfqpoint{4.062863in}{2.846332in}}%
\pgfpathlineto{\pgfqpoint{4.049990in}{2.854991in}}%
\pgfpathlineto{\pgfqpoint{4.037120in}{2.863794in}}%
\pgfpathlineto{\pgfqpoint{4.029675in}{2.852663in}}%
\pgfpathlineto{\pgfqpoint{4.022226in}{2.841598in}}%
\pgfpathlineto{\pgfqpoint{4.014772in}{2.830596in}}%
\pgfpathlineto{\pgfqpoint{4.007315in}{2.819657in}}%
\pgfpathclose%
\pgfusepath{fill}%
\end{pgfscope}%
\begin{pgfscope}%
\pgfpathrectangle{\pgfqpoint{1.254980in}{0.150000in}}{\pgfqpoint{5.490039in}{5.490039in}}%
\pgfusepath{clip}%
\pgfsetbuttcap%
\pgfsetroundjoin%
\definecolor{currentfill}{rgb}{0.265145,0.232956,0.516599}%
\pgfsetfillcolor{currentfill}%
\pgfsetfillopacity{0.700000}%
\pgfsetlinewidth{0.000000pt}%
\definecolor{currentstroke}{rgb}{0.000000,0.000000,0.000000}%
\pgfsetstrokecolor{currentstroke}%
\pgfsetdash{}{0pt}%
\pgfpathmoveto{\pgfqpoint{4.273026in}{2.783778in}}%
\pgfpathlineto{\pgfqpoint{4.285946in}{2.777129in}}%
\pgfpathlineto{\pgfqpoint{4.298872in}{2.770616in}}%
\pgfpathlineto{\pgfqpoint{4.311802in}{2.764237in}}%
\pgfpathlineto{\pgfqpoint{4.324737in}{2.757992in}}%
\pgfpathlineto{\pgfqpoint{4.332108in}{2.768997in}}%
\pgfpathlineto{\pgfqpoint{4.339474in}{2.780054in}}%
\pgfpathlineto{\pgfqpoint{4.346837in}{2.791163in}}%
\pgfpathlineto{\pgfqpoint{4.354196in}{2.802327in}}%
\pgfpathlineto{\pgfqpoint{4.341271in}{2.808671in}}%
\pgfpathlineto{\pgfqpoint{4.328351in}{2.815149in}}%
\pgfpathlineto{\pgfqpoint{4.315436in}{2.821762in}}%
\pgfpathlineto{\pgfqpoint{4.302526in}{2.828509in}}%
\pgfpathlineto{\pgfqpoint{4.295157in}{2.817241in}}%
\pgfpathlineto{\pgfqpoint{4.287784in}{2.806030in}}%
\pgfpathlineto{\pgfqpoint{4.280407in}{2.794877in}}%
\pgfpathlineto{\pgfqpoint{4.273026in}{2.783778in}}%
\pgfpathclose%
\pgfusepath{fill}%
\end{pgfscope}%
\begin{pgfscope}%
\pgfpathrectangle{\pgfqpoint{1.254980in}{0.150000in}}{\pgfqpoint{5.490039in}{5.490039in}}%
\pgfusepath{clip}%
\pgfsetbuttcap%
\pgfsetroundjoin%
\definecolor{currentfill}{rgb}{0.255645,0.260703,0.528312}%
\pgfsetfillcolor{currentfill}%
\pgfsetfillopacity{0.700000}%
\pgfsetlinewidth{0.000000pt}%
\definecolor{currentstroke}{rgb}{0.000000,0.000000,0.000000}%
\pgfsetstrokecolor{currentstroke}%
\pgfsetdash{}{0pt}%
\pgfpathmoveto{\pgfqpoint{4.701280in}{2.833760in}}%
\pgfpathlineto{\pgfqpoint{4.714301in}{2.829738in}}%
\pgfpathlineto{\pgfqpoint{4.727329in}{2.825840in}}%
\pgfpathlineto{\pgfqpoint{4.740364in}{2.822066in}}%
\pgfpathlineto{\pgfqpoint{4.753407in}{2.818417in}}%
\pgfpathlineto{\pgfqpoint{4.760655in}{2.829475in}}%
\pgfpathlineto{\pgfqpoint{4.767900in}{2.840585in}}%
\pgfpathlineto{\pgfqpoint{4.775142in}{2.851749in}}%
\pgfpathlineto{\pgfqpoint{4.782380in}{2.862967in}}%
\pgfpathlineto{\pgfqpoint{4.769349in}{2.866778in}}%
\pgfpathlineto{\pgfqpoint{4.756325in}{2.870714in}}%
\pgfpathlineto{\pgfqpoint{4.743307in}{2.874773in}}%
\pgfpathlineto{\pgfqpoint{4.730297in}{2.878957in}}%
\pgfpathlineto{\pgfqpoint{4.723048in}{2.867572in}}%
\pgfpathlineto{\pgfqpoint{4.715796in}{2.856245in}}%
\pgfpathlineto{\pgfqpoint{4.708540in}{2.844975in}}%
\pgfpathlineto{\pgfqpoint{4.701280in}{2.833760in}}%
\pgfpathclose%
\pgfusepath{fill}%
\end{pgfscope}%
\begin{pgfscope}%
\pgfpathrectangle{\pgfqpoint{1.254980in}{0.150000in}}{\pgfqpoint{5.490039in}{5.490039in}}%
\pgfusepath{clip}%
\pgfsetbuttcap%
\pgfsetroundjoin%
\definecolor{currentfill}{rgb}{0.233603,0.313828,0.543914}%
\pgfsetfillcolor{currentfill}%
\pgfsetfillopacity{0.700000}%
\pgfsetlinewidth{0.000000pt}%
\definecolor{currentstroke}{rgb}{0.000000,0.000000,0.000000}%
\pgfsetstrokecolor{currentstroke}%
\pgfsetdash{}{0pt}%
\pgfpathmoveto{\pgfqpoint{5.078022in}{2.948879in}}%
\pgfpathlineto{\pgfqpoint{5.091155in}{2.946543in}}%
\pgfpathlineto{\pgfqpoint{5.104298in}{2.944325in}}%
\pgfpathlineto{\pgfqpoint{5.117449in}{2.942225in}}%
\pgfpathlineto{\pgfqpoint{5.130609in}{2.940243in}}%
\pgfpathlineto{\pgfqpoint{5.137750in}{2.951259in}}%
\pgfpathlineto{\pgfqpoint{5.144888in}{2.962339in}}%
\pgfpathlineto{\pgfqpoint{5.152024in}{2.973486in}}%
\pgfpathlineto{\pgfqpoint{5.159157in}{2.984701in}}%
\pgfpathlineto{\pgfqpoint{5.146009in}{2.986909in}}%
\pgfpathlineto{\pgfqpoint{5.132871in}{2.989234in}}%
\pgfpathlineto{\pgfqpoint{5.119741in}{2.991677in}}%
\pgfpathlineto{\pgfqpoint{5.106620in}{2.994238in}}%
\pgfpathlineto{\pgfqpoint{5.099474in}{2.982792in}}%
\pgfpathlineto{\pgfqpoint{5.092326in}{2.971418in}}%
\pgfpathlineto{\pgfqpoint{5.085175in}{2.960115in}}%
\pgfpathlineto{\pgfqpoint{5.078022in}{2.948879in}}%
\pgfpathclose%
\pgfusepath{fill}%
\end{pgfscope}%
\begin{pgfscope}%
\pgfpathrectangle{\pgfqpoint{1.254980in}{0.150000in}}{\pgfqpoint{5.490039in}{5.490039in}}%
\pgfusepath{clip}%
\pgfsetbuttcap%
\pgfsetroundjoin%
\definecolor{currentfill}{rgb}{0.225863,0.330805,0.547314}%
\pgfsetfillcolor{currentfill}%
\pgfsetfillopacity{0.700000}%
\pgfsetlinewidth{0.000000pt}%
\definecolor{currentstroke}{rgb}{0.000000,0.000000,0.000000}%
\pgfsetstrokecolor{currentstroke}%
\pgfsetdash{}{0pt}%
\pgfpathmoveto{\pgfqpoint{5.159157in}{2.984701in}}%
\pgfpathlineto{\pgfqpoint{5.172313in}{2.982611in}}%
\pgfpathlineto{\pgfqpoint{5.185479in}{2.980637in}}%
\pgfpathlineto{\pgfqpoint{5.198654in}{2.978781in}}%
\pgfpathlineto{\pgfqpoint{5.211838in}{2.977041in}}%
\pgfpathlineto{\pgfqpoint{5.218955in}{2.988092in}}%
\pgfpathlineto{\pgfqpoint{5.226071in}{2.999213in}}%
\pgfpathlineto{\pgfqpoint{5.233184in}{3.010407in}}%
\pgfpathlineto{\pgfqpoint{5.220010in}{3.012326in}}%
\pgfpathlineto{\pgfqpoint{5.206845in}{3.014362in}}%
\pgfpathlineto{\pgfqpoint{5.193689in}{3.016514in}}%
\pgfpathlineto{\pgfqpoint{5.180542in}{3.018783in}}%
\pgfpathlineto{\pgfqpoint{5.173416in}{3.007348in}}%
\pgfpathlineto{\pgfqpoint{5.166288in}{2.995988in}}%
\pgfpathlineto{\pgfqpoint{5.159157in}{2.984701in}}%
\pgfpathclose%
\pgfusepath{fill}%
\end{pgfscope}%
\begin{pgfscope}%
\pgfpathrectangle{\pgfqpoint{1.254980in}{0.150000in}}{\pgfqpoint{5.490039in}{5.490039in}}%
\pgfusepath{clip}%
\pgfsetbuttcap%
\pgfsetroundjoin%
\definecolor{currentfill}{rgb}{0.146180,0.515413,0.556823}%
\pgfsetfillcolor{currentfill}%
\pgfsetfillopacity{0.700000}%
\pgfsetlinewidth{0.000000pt}%
\definecolor{currentstroke}{rgb}{0.000000,0.000000,0.000000}%
\pgfsetstrokecolor{currentstroke}%
\pgfsetdash{}{0pt}%
\pgfpathmoveto{\pgfqpoint{3.225658in}{3.472889in}}%
\pgfpathlineto{\pgfqpoint{3.238603in}{3.454839in}}%
\pgfpathlineto{\pgfqpoint{3.251544in}{3.436985in}}%
\pgfpathlineto{\pgfqpoint{3.264479in}{3.419327in}}%
\pgfpathlineto{\pgfqpoint{3.277411in}{3.401864in}}%
\pgfpathlineto{\pgfqpoint{3.285081in}{3.412708in}}%
\pgfpathlineto{\pgfqpoint{3.292745in}{3.423661in}}%
\pgfpathlineto{\pgfqpoint{3.300403in}{3.434724in}}%
\pgfpathlineto{\pgfqpoint{3.308055in}{3.445897in}}%
\pgfpathlineto{\pgfqpoint{3.295139in}{3.463395in}}%
\pgfpathlineto{\pgfqpoint{3.282218in}{3.481088in}}%
\pgfpathlineto{\pgfqpoint{3.269294in}{3.498976in}}%
\pgfpathlineto{\pgfqpoint{3.256365in}{3.517061in}}%
\pgfpathlineto{\pgfqpoint{3.248697in}{3.505847in}}%
\pgfpathlineto{\pgfqpoint{3.241024in}{3.494748in}}%
\pgfpathlineto{\pgfqpoint{3.233344in}{3.483762in}}%
\pgfpathlineto{\pgfqpoint{3.225658in}{3.472889in}}%
\pgfpathclose%
\pgfusepath{fill}%
\end{pgfscope}%
\begin{pgfscope}%
\pgfpathrectangle{\pgfqpoint{1.254980in}{0.150000in}}{\pgfqpoint{5.490039in}{5.490039in}}%
\pgfusepath{clip}%
\pgfsetbuttcap%
\pgfsetroundjoin%
\definecolor{currentfill}{rgb}{0.229739,0.322361,0.545706}%
\pgfsetfillcolor{currentfill}%
\pgfsetfillopacity{0.700000}%
\pgfsetlinewidth{0.000000pt}%
\definecolor{currentstroke}{rgb}{0.000000,0.000000,0.000000}%
\pgfsetstrokecolor{currentstroke}%
\pgfsetdash{}{0pt}%
\pgfpathmoveto{\pgfqpoint{3.638349in}{2.986975in}}%
\pgfpathlineto{\pgfqpoint{3.651219in}{2.974650in}}%
\pgfpathlineto{\pgfqpoint{3.664090in}{2.962487in}}%
\pgfpathlineto{\pgfqpoint{3.676961in}{2.950485in}}%
\pgfpathlineto{\pgfqpoint{3.689832in}{2.938644in}}%
\pgfpathlineto{\pgfqpoint{3.697387in}{2.949257in}}%
\pgfpathlineto{\pgfqpoint{3.704938in}{2.959946in}}%
\pgfpathlineto{\pgfqpoint{3.712484in}{2.970712in}}%
\pgfpathlineto{\pgfqpoint{3.720025in}{2.981557in}}%
\pgfpathlineto{\pgfqpoint{3.707167in}{2.993433in}}%
\pgfpathlineto{\pgfqpoint{3.694309in}{3.005470in}}%
\pgfpathlineto{\pgfqpoint{3.681452in}{3.017669in}}%
\pgfpathlineto{\pgfqpoint{3.668594in}{3.030029in}}%
\pgfpathlineto{\pgfqpoint{3.661041in}{3.019144in}}%
\pgfpathlineto{\pgfqpoint{3.653482in}{3.008340in}}%
\pgfpathlineto{\pgfqpoint{3.645918in}{2.997617in}}%
\pgfpathlineto{\pgfqpoint{3.638349in}{2.986975in}}%
\pgfpathclose%
\pgfusepath{fill}%
\end{pgfscope}%
\begin{pgfscope}%
\pgfpathrectangle{\pgfqpoint{1.254980in}{0.150000in}}{\pgfqpoint{5.490039in}{5.490039in}}%
\pgfusepath{clip}%
\pgfsetbuttcap%
\pgfsetroundjoin%
\definecolor{currentfill}{rgb}{0.239346,0.300855,0.540844}%
\pgfsetfillcolor{currentfill}%
\pgfsetfillopacity{0.700000}%
\pgfsetlinewidth{0.000000pt}%
\definecolor{currentstroke}{rgb}{0.000000,0.000000,0.000000}%
\pgfsetstrokecolor{currentstroke}%
\pgfsetdash{}{0pt}%
\pgfpathmoveto{\pgfqpoint{4.996884in}{2.914267in}}%
\pgfpathlineto{\pgfqpoint{5.009995in}{2.911665in}}%
\pgfpathlineto{\pgfqpoint{5.023115in}{2.909182in}}%
\pgfpathlineto{\pgfqpoint{5.036243in}{2.906818in}}%
\pgfpathlineto{\pgfqpoint{5.049381in}{2.904573in}}%
\pgfpathlineto{\pgfqpoint{5.056545in}{2.915559in}}%
\pgfpathlineto{\pgfqpoint{5.063707in}{2.926604in}}%
\pgfpathlineto{\pgfqpoint{5.070866in}{2.937710in}}%
\pgfpathlineto{\pgfqpoint{5.078022in}{2.948879in}}%
\pgfpathlineto{\pgfqpoint{5.064897in}{2.951334in}}%
\pgfpathlineto{\pgfqpoint{5.051780in}{2.953907in}}%
\pgfpathlineto{\pgfqpoint{5.038673in}{2.956599in}}%
\pgfpathlineto{\pgfqpoint{5.025573in}{2.959411in}}%
\pgfpathlineto{\pgfqpoint{5.018405in}{2.948026in}}%
\pgfpathlineto{\pgfqpoint{5.011234in}{2.936709in}}%
\pgfpathlineto{\pgfqpoint{5.004060in}{2.925457in}}%
\pgfpathlineto{\pgfqpoint{4.996884in}{2.914267in}}%
\pgfpathclose%
\pgfusepath{fill}%
\end{pgfscope}%
\begin{pgfscope}%
\pgfpathrectangle{\pgfqpoint{1.254980in}{0.150000in}}{\pgfqpoint{5.490039in}{5.490039in}}%
\pgfusepath{clip}%
\pgfsetbuttcap%
\pgfsetroundjoin%
\definecolor{currentfill}{rgb}{0.265145,0.232956,0.516599}%
\pgfsetfillcolor{currentfill}%
\pgfsetfillopacity{0.700000}%
\pgfsetlinewidth{0.000000pt}%
\definecolor{currentstroke}{rgb}{0.000000,0.000000,0.000000}%
\pgfsetstrokecolor{currentstroke}%
\pgfsetdash{}{0pt}%
\pgfpathmoveto{\pgfqpoint{4.405946in}{2.778284in}}%
\pgfpathlineto{\pgfqpoint{4.418898in}{2.772603in}}%
\pgfpathlineto{\pgfqpoint{4.431854in}{2.767054in}}%
\pgfpathlineto{\pgfqpoint{4.444817in}{2.761636in}}%
\pgfpathlineto{\pgfqpoint{4.457785in}{2.756348in}}%
\pgfpathlineto{\pgfqpoint{4.465119in}{2.767350in}}%
\pgfpathlineto{\pgfqpoint{4.472449in}{2.778400in}}%
\pgfpathlineto{\pgfqpoint{4.479776in}{2.789500in}}%
\pgfpathlineto{\pgfqpoint{4.487099in}{2.800652in}}%
\pgfpathlineto{\pgfqpoint{4.474141in}{2.806054in}}%
\pgfpathlineto{\pgfqpoint{4.461189in}{2.811587in}}%
\pgfpathlineto{\pgfqpoint{4.448242in}{2.817251in}}%
\pgfpathlineto{\pgfqpoint{4.435302in}{2.823046in}}%
\pgfpathlineto{\pgfqpoint{4.427968in}{2.811774in}}%
\pgfpathlineto{\pgfqpoint{4.420632in}{2.800557in}}%
\pgfpathlineto{\pgfqpoint{4.413291in}{2.789394in}}%
\pgfpathlineto{\pgfqpoint{4.405946in}{2.778284in}}%
\pgfpathclose%
\pgfusepath{fill}%
\end{pgfscope}%
\begin{pgfscope}%
\pgfpathrectangle{\pgfqpoint{1.254980in}{0.150000in}}{\pgfqpoint{5.490039in}{5.490039in}}%
\pgfusepath{clip}%
\pgfsetbuttcap%
\pgfsetroundjoin%
\definecolor{currentfill}{rgb}{0.253935,0.265254,0.529983}%
\pgfsetfillcolor{currentfill}%
\pgfsetfillopacity{0.700000}%
\pgfsetlinewidth{0.000000pt}%
\definecolor{currentstroke}{rgb}{0.000000,0.000000,0.000000}%
\pgfsetstrokecolor{currentstroke}%
\pgfsetdash{}{0pt}%
\pgfpathmoveto{\pgfqpoint{3.874390in}{2.851263in}}%
\pgfpathlineto{\pgfqpoint{3.887264in}{2.841400in}}%
\pgfpathlineto{\pgfqpoint{3.900139in}{2.831687in}}%
\pgfpathlineto{\pgfqpoint{3.913016in}{2.822123in}}%
\pgfpathlineto{\pgfqpoint{3.925896in}{2.812708in}}%
\pgfpathlineto{\pgfqpoint{3.933383in}{2.823449in}}%
\pgfpathlineto{\pgfqpoint{3.940866in}{2.834253in}}%
\pgfpathlineto{\pgfqpoint{3.948344in}{2.845122in}}%
\pgfpathlineto{\pgfqpoint{3.955818in}{2.856057in}}%
\pgfpathlineto{\pgfqpoint{3.942950in}{2.865524in}}%
\pgfpathlineto{\pgfqpoint{3.930084in}{2.875139in}}%
\pgfpathlineto{\pgfqpoint{3.917221in}{2.884904in}}%
\pgfpathlineto{\pgfqpoint{3.904359in}{2.894818in}}%
\pgfpathlineto{\pgfqpoint{3.896874in}{2.883826in}}%
\pgfpathlineto{\pgfqpoint{3.889384in}{2.872904in}}%
\pgfpathlineto{\pgfqpoint{3.881890in}{2.862049in}}%
\pgfpathlineto{\pgfqpoint{3.874390in}{2.851263in}}%
\pgfpathclose%
\pgfusepath{fill}%
\end{pgfscope}%
\begin{pgfscope}%
\pgfpathrectangle{\pgfqpoint{1.254980in}{0.150000in}}{\pgfqpoint{5.490039in}{5.490039in}}%
\pgfusepath{clip}%
\pgfsetbuttcap%
\pgfsetroundjoin%
\definecolor{currentfill}{rgb}{0.260571,0.246922,0.522828}%
\pgfsetfillcolor{currentfill}%
\pgfsetfillopacity{0.700000}%
\pgfsetlinewidth{0.000000pt}%
\definecolor{currentstroke}{rgb}{0.000000,0.000000,0.000000}%
\pgfsetstrokecolor{currentstroke}%
\pgfsetdash{}{0pt}%
\pgfpathmoveto{\pgfqpoint{4.620152in}{2.806180in}}%
\pgfpathlineto{\pgfqpoint{4.633156in}{2.801802in}}%
\pgfpathlineto{\pgfqpoint{4.646167in}{2.797549in}}%
\pgfpathlineto{\pgfqpoint{4.659185in}{2.793422in}}%
\pgfpathlineto{\pgfqpoint{4.672209in}{2.789421in}}%
\pgfpathlineto{\pgfqpoint{4.679482in}{2.800431in}}%
\pgfpathlineto{\pgfqpoint{4.686752in}{2.811490in}}%
\pgfpathlineto{\pgfqpoint{4.694018in}{2.822599in}}%
\pgfpathlineto{\pgfqpoint{4.701280in}{2.833760in}}%
\pgfpathlineto{\pgfqpoint{4.688267in}{2.837908in}}%
\pgfpathlineto{\pgfqpoint{4.675260in}{2.842181in}}%
\pgfpathlineto{\pgfqpoint{4.662260in}{2.846579in}}%
\pgfpathlineto{\pgfqpoint{4.649266in}{2.851104in}}%
\pgfpathlineto{\pgfqpoint{4.641993in}{2.839791in}}%
\pgfpathlineto{\pgfqpoint{4.634716in}{2.828534in}}%
\pgfpathlineto{\pgfqpoint{4.627436in}{2.817331in}}%
\pgfpathlineto{\pgfqpoint{4.620152in}{2.806180in}}%
\pgfpathclose%
\pgfusepath{fill}%
\end{pgfscope}%
\begin{pgfscope}%
\pgfpathrectangle{\pgfqpoint{1.254980in}{0.150000in}}{\pgfqpoint{5.490039in}{5.490039in}}%
\pgfusepath{clip}%
\pgfsetbuttcap%
\pgfsetroundjoin%
\definecolor{currentfill}{rgb}{0.135066,0.544853,0.554029}%
\pgfsetfillcolor{currentfill}%
\pgfsetfillopacity{0.700000}%
\pgfsetlinewidth{0.000000pt}%
\definecolor{currentstroke}{rgb}{0.000000,0.000000,0.000000}%
\pgfsetstrokecolor{currentstroke}%
\pgfsetdash{}{0pt}%
\pgfpathmoveto{\pgfqpoint{3.173829in}{3.547086in}}%
\pgfpathlineto{\pgfqpoint{3.186794in}{3.528234in}}%
\pgfpathlineto{\pgfqpoint{3.199754in}{3.509586in}}%
\pgfpathlineto{\pgfqpoint{3.212709in}{3.491137in}}%
\pgfpathlineto{\pgfqpoint{3.225658in}{3.472889in}}%
\pgfpathlineto{\pgfqpoint{3.233344in}{3.483762in}}%
\pgfpathlineto{\pgfqpoint{3.241024in}{3.494748in}}%
\pgfpathlineto{\pgfqpoint{3.248697in}{3.505847in}}%
\pgfpathlineto{\pgfqpoint{3.256365in}{3.517061in}}%
\pgfpathlineto{\pgfqpoint{3.243431in}{3.535344in}}%
\pgfpathlineto{\pgfqpoint{3.230492in}{3.553826in}}%
\pgfpathlineto{\pgfqpoint{3.217548in}{3.572510in}}%
\pgfpathlineto{\pgfqpoint{3.204599in}{3.591396in}}%
\pgfpathlineto{\pgfqpoint{3.196916in}{3.580141in}}%
\pgfpathlineto{\pgfqpoint{3.189226in}{3.569005in}}%
\pgfpathlineto{\pgfqpoint{3.181531in}{3.557987in}}%
\pgfpathlineto{\pgfqpoint{3.173829in}{3.547086in}}%
\pgfpathclose%
\pgfusepath{fill}%
\end{pgfscope}%
\begin{pgfscope}%
\pgfpathrectangle{\pgfqpoint{1.254980in}{0.150000in}}{\pgfqpoint{5.490039in}{5.490039in}}%
\pgfusepath{clip}%
\pgfsetbuttcap%
\pgfsetroundjoin%
\definecolor{currentfill}{rgb}{0.237441,0.305202,0.541921}%
\pgfsetfillcolor{currentfill}%
\pgfsetfillopacity{0.700000}%
\pgfsetlinewidth{0.000000pt}%
\definecolor{currentstroke}{rgb}{0.000000,0.000000,0.000000}%
\pgfsetstrokecolor{currentstroke}%
\pgfsetdash{}{0pt}%
\pgfpathmoveto{\pgfqpoint{3.689832in}{2.938644in}}%
\pgfpathlineto{\pgfqpoint{3.702703in}{2.926963in}}%
\pgfpathlineto{\pgfqpoint{3.715575in}{2.915440in}}%
\pgfpathlineto{\pgfqpoint{3.728447in}{2.904076in}}%
\pgfpathlineto{\pgfqpoint{3.741320in}{2.892869in}}%
\pgfpathlineto{\pgfqpoint{3.748863in}{2.903451in}}%
\pgfpathlineto{\pgfqpoint{3.756401in}{2.914107in}}%
\pgfpathlineto{\pgfqpoint{3.763934in}{2.924836in}}%
\pgfpathlineto{\pgfqpoint{3.771463in}{2.935639in}}%
\pgfpathlineto{\pgfqpoint{3.758602in}{2.946882in}}%
\pgfpathlineto{\pgfqpoint{3.745743in}{2.958282in}}%
\pgfpathlineto{\pgfqpoint{3.732884in}{2.969840in}}%
\pgfpathlineto{\pgfqpoint{3.720025in}{2.981557in}}%
\pgfpathlineto{\pgfqpoint{3.712484in}{2.970712in}}%
\pgfpathlineto{\pgfqpoint{3.704938in}{2.959946in}}%
\pgfpathlineto{\pgfqpoint{3.697387in}{2.949257in}}%
\pgfpathlineto{\pgfqpoint{3.689832in}{2.938644in}}%
\pgfpathclose%
\pgfusepath{fill}%
\end{pgfscope}%
\begin{pgfscope}%
\pgfpathrectangle{\pgfqpoint{1.254980in}{0.150000in}}{\pgfqpoint{5.490039in}{5.490039in}}%
\pgfusepath{clip}%
\pgfsetbuttcap%
\pgfsetroundjoin%
\definecolor{currentfill}{rgb}{0.244972,0.287675,0.537260}%
\pgfsetfillcolor{currentfill}%
\pgfsetfillopacity{0.700000}%
\pgfsetlinewidth{0.000000pt}%
\definecolor{currentstroke}{rgb}{0.000000,0.000000,0.000000}%
\pgfsetstrokecolor{currentstroke}%
\pgfsetdash{}{0pt}%
\pgfpathmoveto{\pgfqpoint{4.915739in}{2.880933in}}%
\pgfpathlineto{\pgfqpoint{4.928829in}{2.878044in}}%
\pgfpathlineto{\pgfqpoint{4.941927in}{2.875275in}}%
\pgfpathlineto{\pgfqpoint{4.955033in}{2.872627in}}%
\pgfpathlineto{\pgfqpoint{4.968148in}{2.870098in}}%
\pgfpathlineto{\pgfqpoint{4.975336in}{2.881057in}}%
\pgfpathlineto{\pgfqpoint{4.982522in}{2.892069in}}%
\pgfpathlineto{\pgfqpoint{4.989704in}{2.903139in}}%
\pgfpathlineto{\pgfqpoint{4.996884in}{2.914267in}}%
\pgfpathlineto{\pgfqpoint{4.983781in}{2.916989in}}%
\pgfpathlineto{\pgfqpoint{4.970686in}{2.919831in}}%
\pgfpathlineto{\pgfqpoint{4.957600in}{2.922793in}}%
\pgfpathlineto{\pgfqpoint{4.944521in}{2.925876in}}%
\pgfpathlineto{\pgfqpoint{4.937330in}{2.914548in}}%
\pgfpathlineto{\pgfqpoint{4.930136in}{2.903283in}}%
\pgfpathlineto{\pgfqpoint{4.922939in}{2.892079in}}%
\pgfpathlineto{\pgfqpoint{4.915739in}{2.880933in}}%
\pgfpathclose%
\pgfusepath{fill}%
\end{pgfscope}%
\begin{pgfscope}%
\pgfpathrectangle{\pgfqpoint{1.254980in}{0.150000in}}{\pgfqpoint{5.490039in}{5.490039in}}%
\pgfusepath{clip}%
\pgfsetbuttcap%
\pgfsetroundjoin%
\definecolor{currentfill}{rgb}{0.263663,0.237631,0.518762}%
\pgfsetfillcolor{currentfill}%
\pgfsetfillopacity{0.700000}%
\pgfsetlinewidth{0.000000pt}%
\definecolor{currentstroke}{rgb}{0.000000,0.000000,0.000000}%
\pgfsetstrokecolor{currentstroke}%
\pgfsetdash{}{0pt}%
\pgfpathmoveto{\pgfqpoint{4.058857in}{2.785577in}}%
\pgfpathlineto{\pgfqpoint{4.071751in}{2.777414in}}%
\pgfpathlineto{\pgfqpoint{4.084648in}{2.769394in}}%
\pgfpathlineto{\pgfqpoint{4.097549in}{2.761514in}}%
\pgfpathlineto{\pgfqpoint{4.110453in}{2.753776in}}%
\pgfpathlineto{\pgfqpoint{4.117889in}{2.764591in}}%
\pgfpathlineto{\pgfqpoint{4.125320in}{2.775462in}}%
\pgfpathlineto{\pgfqpoint{4.132747in}{2.786389in}}%
\pgfpathlineto{\pgfqpoint{4.140170in}{2.797373in}}%
\pgfpathlineto{\pgfqpoint{4.127276in}{2.805179in}}%
\pgfpathlineto{\pgfqpoint{4.114387in}{2.813126in}}%
\pgfpathlineto{\pgfqpoint{4.101501in}{2.821214in}}%
\pgfpathlineto{\pgfqpoint{4.088618in}{2.829444in}}%
\pgfpathlineto{\pgfqpoint{4.081184in}{2.818387in}}%
\pgfpathlineto{\pgfqpoint{4.073746in}{2.807390in}}%
\pgfpathlineto{\pgfqpoint{4.066304in}{2.796454in}}%
\pgfpathlineto{\pgfqpoint{4.058857in}{2.785577in}}%
\pgfpathclose%
\pgfusepath{fill}%
\end{pgfscope}%
\begin{pgfscope}%
\pgfpathrectangle{\pgfqpoint{1.254980in}{0.150000in}}{\pgfqpoint{5.490039in}{5.490039in}}%
\pgfusepath{clip}%
\pgfsetbuttcap%
\pgfsetroundjoin%
\definecolor{currentfill}{rgb}{0.266580,0.228262,0.514349}%
\pgfsetfillcolor{currentfill}%
\pgfsetfillopacity{0.700000}%
\pgfsetlinewidth{0.000000pt}%
\definecolor{currentstroke}{rgb}{0.000000,0.000000,0.000000}%
\pgfsetstrokecolor{currentstroke}%
\pgfsetdash{}{0pt}%
\pgfpathmoveto{\pgfqpoint{4.191782in}{2.767543in}}%
\pgfpathlineto{\pgfqpoint{4.204695in}{2.760432in}}%
\pgfpathlineto{\pgfqpoint{4.217613in}{2.753457in}}%
\pgfpathlineto{\pgfqpoint{4.230535in}{2.746620in}}%
\pgfpathlineto{\pgfqpoint{4.243461in}{2.739919in}}%
\pgfpathlineto{\pgfqpoint{4.250859in}{2.750806in}}%
\pgfpathlineto{\pgfqpoint{4.258252in}{2.761744in}}%
\pgfpathlineto{\pgfqpoint{4.265641in}{2.772734in}}%
\pgfpathlineto{\pgfqpoint{4.273026in}{2.783778in}}%
\pgfpathlineto{\pgfqpoint{4.260110in}{2.790563in}}%
\pgfpathlineto{\pgfqpoint{4.247198in}{2.797484in}}%
\pgfpathlineto{\pgfqpoint{4.234291in}{2.804541in}}%
\pgfpathlineto{\pgfqpoint{4.221389in}{2.811736in}}%
\pgfpathlineto{\pgfqpoint{4.213993in}{2.800603in}}%
\pgfpathlineto{\pgfqpoint{4.206593in}{2.789527in}}%
\pgfpathlineto{\pgfqpoint{4.199190in}{2.778508in}}%
\pgfpathlineto{\pgfqpoint{4.191782in}{2.767543in}}%
\pgfpathclose%
\pgfusepath{fill}%
\end{pgfscope}%
\begin{pgfscope}%
\pgfpathrectangle{\pgfqpoint{1.254980in}{0.150000in}}{\pgfqpoint{5.490039in}{5.490039in}}%
\pgfusepath{clip}%
\pgfsetbuttcap%
\pgfsetroundjoin%
\definecolor{currentfill}{rgb}{0.263663,0.237631,0.518762}%
\pgfsetfillcolor{currentfill}%
\pgfsetfillopacity{0.700000}%
\pgfsetlinewidth{0.000000pt}%
\definecolor{currentstroke}{rgb}{0.000000,0.000000,0.000000}%
\pgfsetstrokecolor{currentstroke}%
\pgfsetdash{}{0pt}%
\pgfpathmoveto{\pgfqpoint{4.538990in}{2.780337in}}%
\pgfpathlineto{\pgfqpoint{4.551978in}{2.775580in}}%
\pgfpathlineto{\pgfqpoint{4.564973in}{2.770950in}}%
\pgfpathlineto{\pgfqpoint{4.577974in}{2.766449in}}%
\pgfpathlineto{\pgfqpoint{4.590982in}{2.762074in}}%
\pgfpathlineto{\pgfqpoint{4.598280in}{2.773029in}}%
\pgfpathlineto{\pgfqpoint{4.605574in}{2.784031in}}%
\pgfpathlineto{\pgfqpoint{4.612865in}{2.795081in}}%
\pgfpathlineto{\pgfqpoint{4.620152in}{2.806180in}}%
\pgfpathlineto{\pgfqpoint{4.607155in}{2.810686in}}%
\pgfpathlineto{\pgfqpoint{4.594165in}{2.815318in}}%
\pgfpathlineto{\pgfqpoint{4.581180in}{2.820078in}}%
\pgfpathlineto{\pgfqpoint{4.568203in}{2.824966in}}%
\pgfpathlineto{\pgfqpoint{4.560905in}{2.813730in}}%
\pgfpathlineto{\pgfqpoint{4.553603in}{2.802547in}}%
\pgfpathlineto{\pgfqpoint{4.546298in}{2.791417in}}%
\pgfpathlineto{\pgfqpoint{4.538990in}{2.780337in}}%
\pgfpathclose%
\pgfusepath{fill}%
\end{pgfscope}%
\begin{pgfscope}%
\pgfpathrectangle{\pgfqpoint{1.254980in}{0.150000in}}{\pgfqpoint{5.490039in}{5.490039in}}%
\pgfusepath{clip}%
\pgfsetbuttcap%
\pgfsetroundjoin%
\definecolor{currentfill}{rgb}{0.250425,0.274290,0.533103}%
\pgfsetfillcolor{currentfill}%
\pgfsetfillopacity{0.700000}%
\pgfsetlinewidth{0.000000pt}%
\definecolor{currentstroke}{rgb}{0.000000,0.000000,0.000000}%
\pgfsetstrokecolor{currentstroke}%
\pgfsetdash{}{0pt}%
\pgfpathmoveto{\pgfqpoint{4.834581in}{2.848953in}}%
\pgfpathlineto{\pgfqpoint{4.847651in}{2.845756in}}%
\pgfpathlineto{\pgfqpoint{4.860728in}{2.842681in}}%
\pgfpathlineto{\pgfqpoint{4.873813in}{2.839728in}}%
\pgfpathlineto{\pgfqpoint{4.886907in}{2.836896in}}%
\pgfpathlineto{\pgfqpoint{4.894119in}{2.847827in}}%
\pgfpathlineto{\pgfqpoint{4.901329in}{2.858809in}}%
\pgfpathlineto{\pgfqpoint{4.908535in}{2.869843in}}%
\pgfpathlineto{\pgfqpoint{4.915739in}{2.880933in}}%
\pgfpathlineto{\pgfqpoint{4.902657in}{2.883943in}}%
\pgfpathlineto{\pgfqpoint{4.889583in}{2.887074in}}%
\pgfpathlineto{\pgfqpoint{4.876517in}{2.890327in}}%
\pgfpathlineto{\pgfqpoint{4.863459in}{2.893702in}}%
\pgfpathlineto{\pgfqpoint{4.856244in}{2.882429in}}%
\pgfpathlineto{\pgfqpoint{4.849026in}{2.871214in}}%
\pgfpathlineto{\pgfqpoint{4.841806in}{2.860056in}}%
\pgfpathlineto{\pgfqpoint{4.834581in}{2.848953in}}%
\pgfpathclose%
\pgfusepath{fill}%
\end{pgfscope}%
\begin{pgfscope}%
\pgfpathrectangle{\pgfqpoint{1.254980in}{0.150000in}}{\pgfqpoint{5.490039in}{5.490039in}}%
\pgfusepath{clip}%
\pgfsetbuttcap%
\pgfsetroundjoin%
\definecolor{currentfill}{rgb}{0.266580,0.228262,0.514349}%
\pgfsetfillcolor{currentfill}%
\pgfsetfillopacity{0.700000}%
\pgfsetlinewidth{0.000000pt}%
\definecolor{currentstroke}{rgb}{0.000000,0.000000,0.000000}%
\pgfsetstrokecolor{currentstroke}%
\pgfsetdash{}{0pt}%
\pgfpathmoveto{\pgfqpoint{4.324737in}{2.757992in}}%
\pgfpathlineto{\pgfqpoint{4.337677in}{2.751881in}}%
\pgfpathlineto{\pgfqpoint{4.350623in}{2.745903in}}%
\pgfpathlineto{\pgfqpoint{4.363574in}{2.740058in}}%
\pgfpathlineto{\pgfqpoint{4.376530in}{2.734345in}}%
\pgfpathlineto{\pgfqpoint{4.383890in}{2.745257in}}%
\pgfpathlineto{\pgfqpoint{4.391246in}{2.756216in}}%
\pgfpathlineto{\pgfqpoint{4.398598in}{2.767225in}}%
\pgfpathlineto{\pgfqpoint{4.405946in}{2.778284in}}%
\pgfpathlineto{\pgfqpoint{4.393001in}{2.784096in}}%
\pgfpathlineto{\pgfqpoint{4.380060in}{2.790040in}}%
\pgfpathlineto{\pgfqpoint{4.367125in}{2.796117in}}%
\pgfpathlineto{\pgfqpoint{4.354196in}{2.802327in}}%
\pgfpathlineto{\pgfqpoint{4.346837in}{2.791163in}}%
\pgfpathlineto{\pgfqpoint{4.339474in}{2.780054in}}%
\pgfpathlineto{\pgfqpoint{4.332108in}{2.768997in}}%
\pgfpathlineto{\pgfqpoint{4.324737in}{2.757992in}}%
\pgfpathclose%
\pgfusepath{fill}%
\end{pgfscope}%
\begin{pgfscope}%
\pgfpathrectangle{\pgfqpoint{1.254980in}{0.150000in}}{\pgfqpoint{5.490039in}{5.490039in}}%
\pgfusepath{clip}%
\pgfsetbuttcap%
\pgfsetroundjoin%
\definecolor{currentfill}{rgb}{0.258965,0.251537,0.524736}%
\pgfsetfillcolor{currentfill}%
\pgfsetfillopacity{0.700000}%
\pgfsetlinewidth{0.000000pt}%
\definecolor{currentstroke}{rgb}{0.000000,0.000000,0.000000}%
\pgfsetstrokecolor{currentstroke}%
\pgfsetdash{}{0pt}%
\pgfpathmoveto{\pgfqpoint{3.925896in}{2.812708in}}%
\pgfpathlineto{\pgfqpoint{3.938778in}{2.803440in}}%
\pgfpathlineto{\pgfqpoint{3.951662in}{2.794319in}}%
\pgfpathlineto{\pgfqpoint{3.964549in}{2.785344in}}%
\pgfpathlineto{\pgfqpoint{3.977439in}{2.776515in}}%
\pgfpathlineto{\pgfqpoint{3.984914in}{2.787210in}}%
\pgfpathlineto{\pgfqpoint{3.992386in}{2.797965in}}%
\pgfpathlineto{\pgfqpoint{3.999852in}{2.808780in}}%
\pgfpathlineto{\pgfqpoint{4.007315in}{2.819657in}}%
\pgfpathlineto{\pgfqpoint{3.994436in}{2.828538in}}%
\pgfpathlineto{\pgfqpoint{3.981561in}{2.837564in}}%
\pgfpathlineto{\pgfqpoint{3.968688in}{2.846737in}}%
\pgfpathlineto{\pgfqpoint{3.955818in}{2.856057in}}%
\pgfpathlineto{\pgfqpoint{3.948344in}{2.845122in}}%
\pgfpathlineto{\pgfqpoint{3.940866in}{2.834253in}}%
\pgfpathlineto{\pgfqpoint{3.933383in}{2.823449in}}%
\pgfpathlineto{\pgfqpoint{3.925896in}{2.812708in}}%
\pgfpathclose%
\pgfusepath{fill}%
\end{pgfscope}%
\begin{pgfscope}%
\pgfpathrectangle{\pgfqpoint{1.254980in}{0.150000in}}{\pgfqpoint{5.490039in}{5.490039in}}%
\pgfusepath{clip}%
\pgfsetbuttcap%
\pgfsetroundjoin%
\definecolor{currentfill}{rgb}{0.244972,0.287675,0.537260}%
\pgfsetfillcolor{currentfill}%
\pgfsetfillopacity{0.700000}%
\pgfsetlinewidth{0.000000pt}%
\definecolor{currentstroke}{rgb}{0.000000,0.000000,0.000000}%
\pgfsetstrokecolor{currentstroke}%
\pgfsetdash{}{0pt}%
\pgfpathmoveto{\pgfqpoint{3.741320in}{2.892869in}}%
\pgfpathlineto{\pgfqpoint{3.754194in}{2.881818in}}%
\pgfpathlineto{\pgfqpoint{3.767069in}{2.870923in}}%
\pgfpathlineto{\pgfqpoint{3.779945in}{2.860183in}}%
\pgfpathlineto{\pgfqpoint{3.792823in}{2.849597in}}%
\pgfpathlineto{\pgfqpoint{3.800353in}{2.860150in}}%
\pgfpathlineto{\pgfqpoint{3.807879in}{2.870771in}}%
\pgfpathlineto{\pgfqpoint{3.815399in}{2.881463in}}%
\pgfpathlineto{\pgfqpoint{3.822915in}{2.892224in}}%
\pgfpathlineto{\pgfqpoint{3.810050in}{2.902846in}}%
\pgfpathlineto{\pgfqpoint{3.797187in}{2.913622in}}%
\pgfpathlineto{\pgfqpoint{3.784324in}{2.924553in}}%
\pgfpathlineto{\pgfqpoint{3.771463in}{2.935639in}}%
\pgfpathlineto{\pgfqpoint{3.763934in}{2.924836in}}%
\pgfpathlineto{\pgfqpoint{3.756401in}{2.914107in}}%
\pgfpathlineto{\pgfqpoint{3.748863in}{2.903451in}}%
\pgfpathlineto{\pgfqpoint{3.741320in}{2.892869in}}%
\pgfpathclose%
\pgfusepath{fill}%
\end{pgfscope}%
\begin{pgfscope}%
\pgfpathrectangle{\pgfqpoint{1.254980in}{0.150000in}}{\pgfqpoint{5.490039in}{5.490039in}}%
\pgfusepath{clip}%
\pgfsetbuttcap%
\pgfsetroundjoin%
\definecolor{currentfill}{rgb}{0.124395,0.578002,0.548287}%
\pgfsetfillcolor{currentfill}%
\pgfsetfillopacity{0.700000}%
\pgfsetlinewidth{0.000000pt}%
\definecolor{currentstroke}{rgb}{0.000000,0.000000,0.000000}%
\pgfsetstrokecolor{currentstroke}%
\pgfsetdash{}{0pt}%
\pgfpathmoveto{\pgfqpoint{3.121911in}{3.624542in}}%
\pgfpathlineto{\pgfqpoint{3.134899in}{3.604867in}}%
\pgfpathlineto{\pgfqpoint{3.147881in}{3.585400in}}%
\pgfpathlineto{\pgfqpoint{3.160858in}{3.566141in}}%
\pgfpathlineto{\pgfqpoint{3.173829in}{3.547086in}}%
\pgfpathlineto{\pgfqpoint{3.181531in}{3.557987in}}%
\pgfpathlineto{\pgfqpoint{3.189226in}{3.569005in}}%
\pgfpathlineto{\pgfqpoint{3.196916in}{3.580141in}}%
\pgfpathlineto{\pgfqpoint{3.204599in}{3.591396in}}%
\pgfpathlineto{\pgfqpoint{3.191644in}{3.610485in}}%
\pgfpathlineto{\pgfqpoint{3.178684in}{3.629779in}}%
\pgfpathlineto{\pgfqpoint{3.165718in}{3.649280in}}%
\pgfpathlineto{\pgfqpoint{3.152746in}{3.668989in}}%
\pgfpathlineto{\pgfqpoint{3.145047in}{3.657694in}}%
\pgfpathlineto{\pgfqpoint{3.137342in}{3.646522in}}%
\pgfpathlineto{\pgfqpoint{3.129629in}{3.635471in}}%
\pgfpathlineto{\pgfqpoint{3.121911in}{3.624542in}}%
\pgfpathclose%
\pgfusepath{fill}%
\end{pgfscope}%
\begin{pgfscope}%
\pgfpathrectangle{\pgfqpoint{1.254980in}{0.150000in}}{\pgfqpoint{5.490039in}{5.490039in}}%
\pgfusepath{clip}%
\pgfsetbuttcap%
\pgfsetroundjoin%
\definecolor{currentfill}{rgb}{0.255645,0.260703,0.528312}%
\pgfsetfillcolor{currentfill}%
\pgfsetfillopacity{0.700000}%
\pgfsetlinewidth{0.000000pt}%
\definecolor{currentstroke}{rgb}{0.000000,0.000000,0.000000}%
\pgfsetstrokecolor{currentstroke}%
\pgfsetdash{}{0pt}%
\pgfpathmoveto{\pgfqpoint{4.753407in}{2.818417in}}%
\pgfpathlineto{\pgfqpoint{4.766457in}{2.814891in}}%
\pgfpathlineto{\pgfqpoint{4.779514in}{2.811489in}}%
\pgfpathlineto{\pgfqpoint{4.792579in}{2.808209in}}%
\pgfpathlineto{\pgfqpoint{4.805652in}{2.805052in}}%
\pgfpathlineto{\pgfqpoint{4.812889in}{2.815954in}}%
\pgfpathlineto{\pgfqpoint{4.820123in}{2.826904in}}%
\pgfpathlineto{\pgfqpoint{4.827354in}{2.837903in}}%
\pgfpathlineto{\pgfqpoint{4.834581in}{2.848953in}}%
\pgfpathlineto{\pgfqpoint{4.821520in}{2.852273in}}%
\pgfpathlineto{\pgfqpoint{4.808466in}{2.855714in}}%
\pgfpathlineto{\pgfqpoint{4.795419in}{2.859279in}}%
\pgfpathlineto{\pgfqpoint{4.782380in}{2.862967in}}%
\pgfpathlineto{\pgfqpoint{4.775142in}{2.851749in}}%
\pgfpathlineto{\pgfqpoint{4.767900in}{2.840585in}}%
\pgfpathlineto{\pgfqpoint{4.760655in}{2.829475in}}%
\pgfpathlineto{\pgfqpoint{4.753407in}{2.818417in}}%
\pgfpathclose%
\pgfusepath{fill}%
\end{pgfscope}%
\begin{pgfscope}%
\pgfpathrectangle{\pgfqpoint{1.254980in}{0.150000in}}{\pgfqpoint{5.490039in}{5.490039in}}%
\pgfusepath{clip}%
\pgfsetbuttcap%
\pgfsetroundjoin%
\definecolor{currentfill}{rgb}{0.192357,0.403199,0.555836}%
\pgfsetfillcolor{currentfill}%
\pgfsetfillopacity{0.700000}%
\pgfsetlinewidth{0.000000pt}%
\definecolor{currentstroke}{rgb}{0.000000,0.000000,0.000000}%
\pgfsetstrokecolor{currentstroke}%
\pgfsetdash{}{0pt}%
\pgfpathmoveto{\pgfqpoint{3.401750in}{3.164924in}}%
\pgfpathlineto{\pgfqpoint{3.414655in}{3.149894in}}%
\pgfpathlineto{\pgfqpoint{3.427556in}{3.135041in}}%
\pgfpathlineto{\pgfqpoint{3.440456in}{3.120365in}}%
\pgfpathlineto{\pgfqpoint{3.453354in}{3.105865in}}%
\pgfpathlineto{\pgfqpoint{3.460987in}{3.116229in}}%
\pgfpathlineto{\pgfqpoint{3.468614in}{3.126684in}}%
\pgfpathlineto{\pgfqpoint{3.476236in}{3.137230in}}%
\pgfpathlineto{\pgfqpoint{3.483851in}{3.147867in}}%
\pgfpathlineto{\pgfqpoint{3.470968in}{3.162386in}}%
\pgfpathlineto{\pgfqpoint{3.458084in}{3.177081in}}%
\pgfpathlineto{\pgfqpoint{3.445197in}{3.191952in}}%
\pgfpathlineto{\pgfqpoint{3.432307in}{3.207001in}}%
\pgfpathlineto{\pgfqpoint{3.424677in}{3.196339in}}%
\pgfpathlineto{\pgfqpoint{3.417040in}{3.185772in}}%
\pgfpathlineto{\pgfqpoint{3.409398in}{3.175301in}}%
\pgfpathlineto{\pgfqpoint{3.401750in}{3.164924in}}%
\pgfpathclose%
\pgfusepath{fill}%
\end{pgfscope}%
\begin{pgfscope}%
\pgfpathrectangle{\pgfqpoint{1.254980in}{0.150000in}}{\pgfqpoint{5.490039in}{5.490039in}}%
\pgfusepath{clip}%
\pgfsetbuttcap%
\pgfsetroundjoin%
\definecolor{currentfill}{rgb}{0.182256,0.426184,0.557120}%
\pgfsetfillcolor{currentfill}%
\pgfsetfillopacity{0.700000}%
\pgfsetlinewidth{0.000000pt}%
\definecolor{currentstroke}{rgb}{0.000000,0.000000,0.000000}%
\pgfsetstrokecolor{currentstroke}%
\pgfsetdash{}{0pt}%
\pgfpathmoveto{\pgfqpoint{3.350108in}{3.226845in}}%
\pgfpathlineto{\pgfqpoint{3.363023in}{3.211092in}}%
\pgfpathlineto{\pgfqpoint{3.375935in}{3.195522in}}%
\pgfpathlineto{\pgfqpoint{3.388844in}{3.180133in}}%
\pgfpathlineto{\pgfqpoint{3.401750in}{3.164924in}}%
\pgfpathlineto{\pgfqpoint{3.409398in}{3.175301in}}%
\pgfpathlineto{\pgfqpoint{3.417040in}{3.185772in}}%
\pgfpathlineto{\pgfqpoint{3.424677in}{3.196339in}}%
\pgfpathlineto{\pgfqpoint{3.432307in}{3.207001in}}%
\pgfpathlineto{\pgfqpoint{3.419416in}{3.222229in}}%
\pgfpathlineto{\pgfqpoint{3.406522in}{3.237637in}}%
\pgfpathlineto{\pgfqpoint{3.393625in}{3.253226in}}%
\pgfpathlineto{\pgfqpoint{3.380726in}{3.268997in}}%
\pgfpathlineto{\pgfqpoint{3.373080in}{3.258310in}}%
\pgfpathlineto{\pgfqpoint{3.365429in}{3.247723in}}%
\pgfpathlineto{\pgfqpoint{3.357771in}{3.237234in}}%
\pgfpathlineto{\pgfqpoint{3.350108in}{3.226845in}}%
\pgfpathclose%
\pgfusepath{fill}%
\end{pgfscope}%
\begin{pgfscope}%
\pgfpathrectangle{\pgfqpoint{1.254980in}{0.150000in}}{\pgfqpoint{5.490039in}{5.490039in}}%
\pgfusepath{clip}%
\pgfsetbuttcap%
\pgfsetroundjoin%
\definecolor{currentfill}{rgb}{0.203063,0.379716,0.553925}%
\pgfsetfillcolor{currentfill}%
\pgfsetfillopacity{0.700000}%
\pgfsetlinewidth{0.000000pt}%
\definecolor{currentstroke}{rgb}{0.000000,0.000000,0.000000}%
\pgfsetstrokecolor{currentstroke}%
\pgfsetdash{}{0pt}%
\pgfpathmoveto{\pgfqpoint{3.453354in}{3.105865in}}%
\pgfpathlineto{\pgfqpoint{3.466249in}{3.091540in}}%
\pgfpathlineto{\pgfqpoint{3.479144in}{3.077388in}}%
\pgfpathlineto{\pgfqpoint{3.492036in}{3.063409in}}%
\pgfpathlineto{\pgfqpoint{3.504927in}{3.049602in}}%
\pgfpathlineto{\pgfqpoint{3.512545in}{3.059953in}}%
\pgfpathlineto{\pgfqpoint{3.520158in}{3.070390in}}%
\pgfpathlineto{\pgfqpoint{3.527765in}{3.080915in}}%
\pgfpathlineto{\pgfqpoint{3.535367in}{3.091527in}}%
\pgfpathlineto{\pgfqpoint{3.522490in}{3.105354in}}%
\pgfpathlineto{\pgfqpoint{3.509612in}{3.119352in}}%
\pgfpathlineto{\pgfqpoint{3.496733in}{3.133523in}}%
\pgfpathlineto{\pgfqpoint{3.483851in}{3.147867in}}%
\pgfpathlineto{\pgfqpoint{3.476236in}{3.137230in}}%
\pgfpathlineto{\pgfqpoint{3.468614in}{3.126684in}}%
\pgfpathlineto{\pgfqpoint{3.460987in}{3.116229in}}%
\pgfpathlineto{\pgfqpoint{3.453354in}{3.105865in}}%
\pgfpathclose%
\pgfusepath{fill}%
\end{pgfscope}%
\begin{pgfscope}%
\pgfpathrectangle{\pgfqpoint{1.254980in}{0.150000in}}{\pgfqpoint{5.490039in}{5.490039in}}%
\pgfusepath{clip}%
\pgfsetbuttcap%
\pgfsetroundjoin%
\definecolor{currentfill}{rgb}{0.171176,0.452530,0.557965}%
\pgfsetfillcolor{currentfill}%
\pgfsetfillopacity{0.700000}%
\pgfsetlinewidth{0.000000pt}%
\definecolor{currentstroke}{rgb}{0.000000,0.000000,0.000000}%
\pgfsetstrokecolor{currentstroke}%
\pgfsetdash{}{0pt}%
\pgfpathmoveto{\pgfqpoint{3.298417in}{3.291700in}}%
\pgfpathlineto{\pgfqpoint{3.311345in}{3.275207in}}%
\pgfpathlineto{\pgfqpoint{3.324269in}{3.258901in}}%
\pgfpathlineto{\pgfqpoint{3.337190in}{3.242781in}}%
\pgfpathlineto{\pgfqpoint{3.350108in}{3.226845in}}%
\pgfpathlineto{\pgfqpoint{3.357771in}{3.237234in}}%
\pgfpathlineto{\pgfqpoint{3.365429in}{3.247723in}}%
\pgfpathlineto{\pgfqpoint{3.373080in}{3.258310in}}%
\pgfpathlineto{\pgfqpoint{3.380726in}{3.268997in}}%
\pgfpathlineto{\pgfqpoint{3.367823in}{3.284951in}}%
\pgfpathlineto{\pgfqpoint{3.354918in}{3.301090in}}%
\pgfpathlineto{\pgfqpoint{3.342009in}{3.317415in}}%
\pgfpathlineto{\pgfqpoint{3.329097in}{3.333926in}}%
\pgfpathlineto{\pgfqpoint{3.321436in}{3.323214in}}%
\pgfpathlineto{\pgfqpoint{3.313769in}{3.312607in}}%
\pgfpathlineto{\pgfqpoint{3.306096in}{3.302102in}}%
\pgfpathlineto{\pgfqpoint{3.298417in}{3.291700in}}%
\pgfpathclose%
\pgfusepath{fill}%
\end{pgfscope}%
\begin{pgfscope}%
\pgfpathrectangle{\pgfqpoint{1.254980in}{0.150000in}}{\pgfqpoint{5.490039in}{5.490039in}}%
\pgfusepath{clip}%
\pgfsetbuttcap%
\pgfsetroundjoin%
\definecolor{currentfill}{rgb}{0.266580,0.228262,0.514349}%
\pgfsetfillcolor{currentfill}%
\pgfsetfillopacity{0.700000}%
\pgfsetlinewidth{0.000000pt}%
\definecolor{currentstroke}{rgb}{0.000000,0.000000,0.000000}%
\pgfsetstrokecolor{currentstroke}%
\pgfsetdash{}{0pt}%
\pgfpathmoveto{\pgfqpoint{4.457785in}{2.756348in}}%
\pgfpathlineto{\pgfqpoint{4.470759in}{2.751191in}}%
\pgfpathlineto{\pgfqpoint{4.483739in}{2.746162in}}%
\pgfpathlineto{\pgfqpoint{4.496725in}{2.741263in}}%
\pgfpathlineto{\pgfqpoint{4.509718in}{2.736493in}}%
\pgfpathlineto{\pgfqpoint{4.517042in}{2.747385in}}%
\pgfpathlineto{\pgfqpoint{4.524361in}{2.758322in}}%
\pgfpathlineto{\pgfqpoint{4.531677in}{2.769306in}}%
\pgfpathlineto{\pgfqpoint{4.538990in}{2.780337in}}%
\pgfpathlineto{\pgfqpoint{4.526008in}{2.785222in}}%
\pgfpathlineto{\pgfqpoint{4.513032in}{2.790236in}}%
\pgfpathlineto{\pgfqpoint{4.500062in}{2.795379in}}%
\pgfpathlineto{\pgfqpoint{4.487099in}{2.800652in}}%
\pgfpathlineto{\pgfqpoint{4.479776in}{2.789500in}}%
\pgfpathlineto{\pgfqpoint{4.472449in}{2.778400in}}%
\pgfpathlineto{\pgfqpoint{4.465119in}{2.767350in}}%
\pgfpathlineto{\pgfqpoint{4.457785in}{2.756348in}}%
\pgfpathclose%
\pgfusepath{fill}%
\end{pgfscope}%
\begin{pgfscope}%
\pgfpathrectangle{\pgfqpoint{1.254980in}{0.150000in}}{\pgfqpoint{5.490039in}{5.490039in}}%
\pgfusepath{clip}%
\pgfsetbuttcap%
\pgfsetroundjoin%
\definecolor{currentfill}{rgb}{0.214298,0.355619,0.551184}%
\pgfsetfillcolor{currentfill}%
\pgfsetfillopacity{0.700000}%
\pgfsetlinewidth{0.000000pt}%
\definecolor{currentstroke}{rgb}{0.000000,0.000000,0.000000}%
\pgfsetstrokecolor{currentstroke}%
\pgfsetdash{}{0pt}%
\pgfpathmoveto{\pgfqpoint{3.504927in}{3.049602in}}%
\pgfpathlineto{\pgfqpoint{3.517817in}{3.035965in}}%
\pgfpathlineto{\pgfqpoint{3.530705in}{3.022498in}}%
\pgfpathlineto{\pgfqpoint{3.543593in}{3.009199in}}%
\pgfpathlineto{\pgfqpoint{3.556480in}{2.996068in}}%
\pgfpathlineto{\pgfqpoint{3.564084in}{3.006406in}}%
\pgfpathlineto{\pgfqpoint{3.571682in}{3.016827in}}%
\pgfpathlineto{\pgfqpoint{3.579275in}{3.027330in}}%
\pgfpathlineto{\pgfqpoint{3.586863in}{3.037917in}}%
\pgfpathlineto{\pgfqpoint{3.573991in}{3.051067in}}%
\pgfpathlineto{\pgfqpoint{3.561117in}{3.064385in}}%
\pgfpathlineto{\pgfqpoint{3.548243in}{3.077871in}}%
\pgfpathlineto{\pgfqpoint{3.535367in}{3.091527in}}%
\pgfpathlineto{\pgfqpoint{3.527765in}{3.080915in}}%
\pgfpathlineto{\pgfqpoint{3.520158in}{3.070390in}}%
\pgfpathlineto{\pgfqpoint{3.512545in}{3.059953in}}%
\pgfpathlineto{\pgfqpoint{3.504927in}{3.049602in}}%
\pgfpathclose%
\pgfusepath{fill}%
\end{pgfscope}%
\begin{pgfscope}%
\pgfpathrectangle{\pgfqpoint{1.254980in}{0.150000in}}{\pgfqpoint{5.490039in}{5.490039in}}%
\pgfusepath{clip}%
\pgfsetbuttcap%
\pgfsetroundjoin%
\definecolor{currentfill}{rgb}{0.252194,0.269783,0.531579}%
\pgfsetfillcolor{currentfill}%
\pgfsetfillopacity{0.700000}%
\pgfsetlinewidth{0.000000pt}%
\definecolor{currentstroke}{rgb}{0.000000,0.000000,0.000000}%
\pgfsetstrokecolor{currentstroke}%
\pgfsetdash{}{0pt}%
\pgfpathmoveto{\pgfqpoint{3.792823in}{2.849597in}}%
\pgfpathlineto{\pgfqpoint{3.805701in}{2.839164in}}%
\pgfpathlineto{\pgfqpoint{3.818581in}{2.828884in}}%
\pgfpathlineto{\pgfqpoint{3.831463in}{2.818756in}}%
\pgfpathlineto{\pgfqpoint{3.844347in}{2.808779in}}%
\pgfpathlineto{\pgfqpoint{3.851865in}{2.819302in}}%
\pgfpathlineto{\pgfqpoint{3.859378in}{2.829890in}}%
\pgfpathlineto{\pgfqpoint{3.866887in}{2.840543in}}%
\pgfpathlineto{\pgfqpoint{3.874390in}{2.851263in}}%
\pgfpathlineto{\pgfqpoint{3.861519in}{2.861276in}}%
\pgfpathlineto{\pgfqpoint{3.848650in}{2.871440in}}%
\pgfpathlineto{\pgfqpoint{3.835782in}{2.881756in}}%
\pgfpathlineto{\pgfqpoint{3.822915in}{2.892224in}}%
\pgfpathlineto{\pgfqpoint{3.815399in}{2.881463in}}%
\pgfpathlineto{\pgfqpoint{3.807879in}{2.870771in}}%
\pgfpathlineto{\pgfqpoint{3.800353in}{2.860150in}}%
\pgfpathlineto{\pgfqpoint{3.792823in}{2.849597in}}%
\pgfpathclose%
\pgfusepath{fill}%
\end{pgfscope}%
\begin{pgfscope}%
\pgfpathrectangle{\pgfqpoint{1.254980in}{0.150000in}}{\pgfqpoint{5.490039in}{5.490039in}}%
\pgfusepath{clip}%
\pgfsetbuttcap%
\pgfsetroundjoin%
\definecolor{currentfill}{rgb}{0.267968,0.223549,0.512008}%
\pgfsetfillcolor{currentfill}%
\pgfsetfillopacity{0.700000}%
\pgfsetlinewidth{0.000000pt}%
\definecolor{currentstroke}{rgb}{0.000000,0.000000,0.000000}%
\pgfsetstrokecolor{currentstroke}%
\pgfsetdash{}{0pt}%
\pgfpathmoveto{\pgfqpoint{4.110453in}{2.753776in}}%
\pgfpathlineto{\pgfqpoint{4.123361in}{2.746177in}}%
\pgfpathlineto{\pgfqpoint{4.136273in}{2.738718in}}%
\pgfpathlineto{\pgfqpoint{4.149189in}{2.731399in}}%
\pgfpathlineto{\pgfqpoint{4.162108in}{2.724217in}}%
\pgfpathlineto{\pgfqpoint{4.169533in}{2.734971in}}%
\pgfpathlineto{\pgfqpoint{4.176954in}{2.745776in}}%
\pgfpathlineto{\pgfqpoint{4.184370in}{2.756633in}}%
\pgfpathlineto{\pgfqpoint{4.191782in}{2.767543in}}%
\pgfpathlineto{\pgfqpoint{4.178873in}{2.774792in}}%
\pgfpathlineto{\pgfqpoint{4.165968in}{2.782180in}}%
\pgfpathlineto{\pgfqpoint{4.153067in}{2.789707in}}%
\pgfpathlineto{\pgfqpoint{4.140170in}{2.797373in}}%
\pgfpathlineto{\pgfqpoint{4.132747in}{2.786389in}}%
\pgfpathlineto{\pgfqpoint{4.125320in}{2.775462in}}%
\pgfpathlineto{\pgfqpoint{4.117889in}{2.764591in}}%
\pgfpathlineto{\pgfqpoint{4.110453in}{2.753776in}}%
\pgfpathclose%
\pgfusepath{fill}%
\end{pgfscope}%
\begin{pgfscope}%
\pgfpathrectangle{\pgfqpoint{1.254980in}{0.150000in}}{\pgfqpoint{5.490039in}{5.490039in}}%
\pgfusepath{clip}%
\pgfsetbuttcap%
\pgfsetroundjoin%
\definecolor{currentfill}{rgb}{0.231674,0.318106,0.544834}%
\pgfsetfillcolor{currentfill}%
\pgfsetfillopacity{0.700000}%
\pgfsetlinewidth{0.000000pt}%
\definecolor{currentstroke}{rgb}{0.000000,0.000000,0.000000}%
\pgfsetstrokecolor{currentstroke}%
\pgfsetdash{}{0pt}%
\pgfpathmoveto{\pgfqpoint{5.130609in}{2.940243in}}%
\pgfpathlineto{\pgfqpoint{5.143779in}{2.938378in}}%
\pgfpathlineto{\pgfqpoint{5.156957in}{2.936630in}}%
\pgfpathlineto{\pgfqpoint{5.170145in}{2.934999in}}%
\pgfpathlineto{\pgfqpoint{5.183342in}{2.933484in}}%
\pgfpathlineto{\pgfqpoint{5.190470in}{2.944281in}}%
\pgfpathlineto{\pgfqpoint{5.197595in}{2.955137in}}%
\pgfpathlineto{\pgfqpoint{5.204718in}{2.966057in}}%
\pgfpathlineto{\pgfqpoint{5.211838in}{2.977041in}}%
\pgfpathlineto{\pgfqpoint{5.198654in}{2.978781in}}%
\pgfpathlineto{\pgfqpoint{5.185479in}{2.980637in}}%
\pgfpathlineto{\pgfqpoint{5.172313in}{2.982611in}}%
\pgfpathlineto{\pgfqpoint{5.159157in}{2.984701in}}%
\pgfpathlineto{\pgfqpoint{5.152024in}{2.973486in}}%
\pgfpathlineto{\pgfqpoint{5.144888in}{2.962339in}}%
\pgfpathlineto{\pgfqpoint{5.137750in}{2.951259in}}%
\pgfpathlineto{\pgfqpoint{5.130609in}{2.940243in}}%
\pgfpathclose%
\pgfusepath{fill}%
\end{pgfscope}%
\begin{pgfscope}%
\pgfpathrectangle{\pgfqpoint{1.254980in}{0.150000in}}{\pgfqpoint{5.490039in}{5.490039in}}%
\pgfusepath{clip}%
\pgfsetbuttcap%
\pgfsetroundjoin%
\definecolor{currentfill}{rgb}{0.223925,0.334994,0.548053}%
\pgfsetfillcolor{currentfill}%
\pgfsetfillopacity{0.700000}%
\pgfsetlinewidth{0.000000pt}%
\definecolor{currentstroke}{rgb}{0.000000,0.000000,0.000000}%
\pgfsetstrokecolor{currentstroke}%
\pgfsetdash{}{0pt}%
\pgfpathmoveto{\pgfqpoint{5.211838in}{2.977041in}}%
\pgfpathlineto{\pgfqpoint{5.225031in}{2.975417in}}%
\pgfpathlineto{\pgfqpoint{5.238234in}{2.973909in}}%
\pgfpathlineto{\pgfqpoint{5.251447in}{2.972517in}}%
\pgfpathlineto{\pgfqpoint{5.264669in}{2.971241in}}%
\pgfpathlineto{\pgfqpoint{5.271773in}{2.982057in}}%
\pgfpathlineto{\pgfqpoint{5.278874in}{2.992939in}}%
\pgfpathlineto{\pgfqpoint{5.285974in}{3.003889in}}%
\pgfpathlineto{\pgfqpoint{5.272762in}{3.005345in}}%
\pgfpathlineto{\pgfqpoint{5.259560in}{3.006916in}}%
\pgfpathlineto{\pgfqpoint{5.246367in}{3.008604in}}%
\pgfpathlineto{\pgfqpoint{5.233184in}{3.010407in}}%
\pgfpathlineto{\pgfqpoint{5.226071in}{2.999213in}}%
\pgfpathlineto{\pgfqpoint{5.218955in}{2.988092in}}%
\pgfpathlineto{\pgfqpoint{5.211838in}{2.977041in}}%
\pgfpathclose%
\pgfusepath{fill}%
\end{pgfscope}%
\begin{pgfscope}%
\pgfpathrectangle{\pgfqpoint{1.254980in}{0.150000in}}{\pgfqpoint{5.490039in}{5.490039in}}%
\pgfusepath{clip}%
\pgfsetbuttcap%
\pgfsetroundjoin%
\definecolor{currentfill}{rgb}{0.160665,0.478540,0.558115}%
\pgfsetfillcolor{currentfill}%
\pgfsetfillopacity{0.700000}%
\pgfsetlinewidth{0.000000pt}%
\definecolor{currentstroke}{rgb}{0.000000,0.000000,0.000000}%
\pgfsetstrokecolor{currentstroke}%
\pgfsetdash{}{0pt}%
\pgfpathmoveto{\pgfqpoint{3.246668in}{3.359564in}}%
\pgfpathlineto{\pgfqpoint{3.259611in}{3.342312in}}%
\pgfpathlineto{\pgfqpoint{3.272551in}{3.325251in}}%
\pgfpathlineto{\pgfqpoint{3.285486in}{3.308381in}}%
\pgfpathlineto{\pgfqpoint{3.298417in}{3.291700in}}%
\pgfpathlineto{\pgfqpoint{3.306096in}{3.302102in}}%
\pgfpathlineto{\pgfqpoint{3.313769in}{3.312607in}}%
\pgfpathlineto{\pgfqpoint{3.321436in}{3.323214in}}%
\pgfpathlineto{\pgfqpoint{3.329097in}{3.333926in}}%
\pgfpathlineto{\pgfqpoint{3.316181in}{3.350625in}}%
\pgfpathlineto{\pgfqpoint{3.303262in}{3.367514in}}%
\pgfpathlineto{\pgfqpoint{3.290338in}{3.384593in}}%
\pgfpathlineto{\pgfqpoint{3.277411in}{3.401864in}}%
\pgfpathlineto{\pgfqpoint{3.269735in}{3.391128in}}%
\pgfpathlineto{\pgfqpoint{3.262052in}{3.380499in}}%
\pgfpathlineto{\pgfqpoint{3.254363in}{3.369978in}}%
\pgfpathlineto{\pgfqpoint{3.246668in}{3.359564in}}%
\pgfpathclose%
\pgfusepath{fill}%
\end{pgfscope}%
\begin{pgfscope}%
\pgfpathrectangle{\pgfqpoint{1.254980in}{0.150000in}}{\pgfqpoint{5.490039in}{5.490039in}}%
\pgfusepath{clip}%
\pgfsetbuttcap%
\pgfsetroundjoin%
\definecolor{currentfill}{rgb}{0.260571,0.246922,0.522828}%
\pgfsetfillcolor{currentfill}%
\pgfsetfillopacity{0.700000}%
\pgfsetlinewidth{0.000000pt}%
\definecolor{currentstroke}{rgb}{0.000000,0.000000,0.000000}%
\pgfsetstrokecolor{currentstroke}%
\pgfsetdash{}{0pt}%
\pgfpathmoveto{\pgfqpoint{4.672209in}{2.789421in}}%
\pgfpathlineto{\pgfqpoint{4.685241in}{2.785545in}}%
\pgfpathlineto{\pgfqpoint{4.698279in}{2.781794in}}%
\pgfpathlineto{\pgfqpoint{4.711325in}{2.778167in}}%
\pgfpathlineto{\pgfqpoint{4.724378in}{2.774664in}}%
\pgfpathlineto{\pgfqpoint{4.731641in}{2.785533in}}%
\pgfpathlineto{\pgfqpoint{4.738900in}{2.796447in}}%
\pgfpathlineto{\pgfqpoint{4.746155in}{2.807408in}}%
\pgfpathlineto{\pgfqpoint{4.753407in}{2.818417in}}%
\pgfpathlineto{\pgfqpoint{4.740364in}{2.822066in}}%
\pgfpathlineto{\pgfqpoint{4.727329in}{2.825840in}}%
\pgfpathlineto{\pgfqpoint{4.714301in}{2.829738in}}%
\pgfpathlineto{\pgfqpoint{4.701280in}{2.833760in}}%
\pgfpathlineto{\pgfqpoint{4.694018in}{2.822599in}}%
\pgfpathlineto{\pgfqpoint{4.686752in}{2.811490in}}%
\pgfpathlineto{\pgfqpoint{4.679482in}{2.800431in}}%
\pgfpathlineto{\pgfqpoint{4.672209in}{2.789421in}}%
\pgfpathclose%
\pgfusepath{fill}%
\end{pgfscope}%
\begin{pgfscope}%
\pgfpathrectangle{\pgfqpoint{1.254980in}{0.150000in}}{\pgfqpoint{5.490039in}{5.490039in}}%
\pgfusepath{clip}%
\pgfsetbuttcap%
\pgfsetroundjoin%
\definecolor{currentfill}{rgb}{0.223925,0.334994,0.548053}%
\pgfsetfillcolor{currentfill}%
\pgfsetfillopacity{0.700000}%
\pgfsetlinewidth{0.000000pt}%
\definecolor{currentstroke}{rgb}{0.000000,0.000000,0.000000}%
\pgfsetstrokecolor{currentstroke}%
\pgfsetdash{}{0pt}%
\pgfpathmoveto{\pgfqpoint{3.556480in}{2.996068in}}%
\pgfpathlineto{\pgfqpoint{3.569365in}{2.983104in}}%
\pgfpathlineto{\pgfqpoint{3.582251in}{2.970306in}}%
\pgfpathlineto{\pgfqpoint{3.595135in}{2.957673in}}%
\pgfpathlineto{\pgfqpoint{3.608020in}{2.945204in}}%
\pgfpathlineto{\pgfqpoint{3.615610in}{2.955529in}}%
\pgfpathlineto{\pgfqpoint{3.623195in}{2.965932in}}%
\pgfpathlineto{\pgfqpoint{3.630774in}{2.976414in}}%
\pgfpathlineto{\pgfqpoint{3.638349in}{2.986975in}}%
\pgfpathlineto{\pgfqpoint{3.625478in}{2.999464in}}%
\pgfpathlineto{\pgfqpoint{3.612607in}{3.012116in}}%
\pgfpathlineto{\pgfqpoint{3.599735in}{3.024934in}}%
\pgfpathlineto{\pgfqpoint{3.586863in}{3.037917in}}%
\pgfpathlineto{\pgfqpoint{3.579275in}{3.027330in}}%
\pgfpathlineto{\pgfqpoint{3.571682in}{3.016827in}}%
\pgfpathlineto{\pgfqpoint{3.564084in}{3.006406in}}%
\pgfpathlineto{\pgfqpoint{3.556480in}{2.996068in}}%
\pgfpathclose%
\pgfusepath{fill}%
\end{pgfscope}%
\begin{pgfscope}%
\pgfpathrectangle{\pgfqpoint{1.254980in}{0.150000in}}{\pgfqpoint{5.490039in}{5.490039in}}%
\pgfusepath{clip}%
\pgfsetbuttcap%
\pgfsetroundjoin%
\definecolor{currentfill}{rgb}{0.269308,0.218818,0.509577}%
\pgfsetfillcolor{currentfill}%
\pgfsetfillopacity{0.700000}%
\pgfsetlinewidth{0.000000pt}%
\definecolor{currentstroke}{rgb}{0.000000,0.000000,0.000000}%
\pgfsetstrokecolor{currentstroke}%
\pgfsetdash{}{0pt}%
\pgfpathmoveto{\pgfqpoint{4.243461in}{2.739919in}}%
\pgfpathlineto{\pgfqpoint{4.256393in}{2.733354in}}%
\pgfpathlineto{\pgfqpoint{4.269328in}{2.726923in}}%
\pgfpathlineto{\pgfqpoint{4.282269in}{2.720628in}}%
\pgfpathlineto{\pgfqpoint{4.295215in}{2.714467in}}%
\pgfpathlineto{\pgfqpoint{4.302602in}{2.725276in}}%
\pgfpathlineto{\pgfqpoint{4.309984in}{2.736132in}}%
\pgfpathlineto{\pgfqpoint{4.317363in}{2.747037in}}%
\pgfpathlineto{\pgfqpoint{4.324737in}{2.757992in}}%
\pgfpathlineto{\pgfqpoint{4.311802in}{2.764237in}}%
\pgfpathlineto{\pgfqpoint{4.298872in}{2.770616in}}%
\pgfpathlineto{\pgfqpoint{4.285946in}{2.777129in}}%
\pgfpathlineto{\pgfqpoint{4.273026in}{2.783778in}}%
\pgfpathlineto{\pgfqpoint{4.265641in}{2.772734in}}%
\pgfpathlineto{\pgfqpoint{4.258252in}{2.761744in}}%
\pgfpathlineto{\pgfqpoint{4.250859in}{2.750806in}}%
\pgfpathlineto{\pgfqpoint{4.243461in}{2.739919in}}%
\pgfpathclose%
\pgfusepath{fill}%
\end{pgfscope}%
\begin{pgfscope}%
\pgfpathrectangle{\pgfqpoint{1.254980in}{0.150000in}}{\pgfqpoint{5.490039in}{5.490039in}}%
\pgfusepath{clip}%
\pgfsetbuttcap%
\pgfsetroundjoin%
\definecolor{currentfill}{rgb}{0.263663,0.237631,0.518762}%
\pgfsetfillcolor{currentfill}%
\pgfsetfillopacity{0.700000}%
\pgfsetlinewidth{0.000000pt}%
\definecolor{currentstroke}{rgb}{0.000000,0.000000,0.000000}%
\pgfsetstrokecolor{currentstroke}%
\pgfsetdash{}{0pt}%
\pgfpathmoveto{\pgfqpoint{3.977439in}{2.776515in}}%
\pgfpathlineto{\pgfqpoint{3.990331in}{2.767831in}}%
\pgfpathlineto{\pgfqpoint{4.003227in}{2.759291in}}%
\pgfpathlineto{\pgfqpoint{4.016125in}{2.750895in}}%
\pgfpathlineto{\pgfqpoint{4.029027in}{2.742642in}}%
\pgfpathlineto{\pgfqpoint{4.036491in}{2.753291in}}%
\pgfpathlineto{\pgfqpoint{4.043951in}{2.763996in}}%
\pgfpathlineto{\pgfqpoint{4.051406in}{2.774758in}}%
\pgfpathlineto{\pgfqpoint{4.058857in}{2.785577in}}%
\pgfpathlineto{\pgfqpoint{4.045967in}{2.793881in}}%
\pgfpathlineto{\pgfqpoint{4.033080in}{2.802329in}}%
\pgfpathlineto{\pgfqpoint{4.020196in}{2.810921in}}%
\pgfpathlineto{\pgfqpoint{4.007315in}{2.819657in}}%
\pgfpathlineto{\pgfqpoint{3.999852in}{2.808780in}}%
\pgfpathlineto{\pgfqpoint{3.992386in}{2.797965in}}%
\pgfpathlineto{\pgfqpoint{3.984914in}{2.787210in}}%
\pgfpathlineto{\pgfqpoint{3.977439in}{2.776515in}}%
\pgfpathclose%
\pgfusepath{fill}%
\end{pgfscope}%
\begin{pgfscope}%
\pgfpathrectangle{\pgfqpoint{1.254980in}{0.150000in}}{\pgfqpoint{5.490039in}{5.490039in}}%
\pgfusepath{clip}%
\pgfsetbuttcap%
\pgfsetroundjoin%
\definecolor{currentfill}{rgb}{0.237441,0.305202,0.541921}%
\pgfsetfillcolor{currentfill}%
\pgfsetfillopacity{0.700000}%
\pgfsetlinewidth{0.000000pt}%
\definecolor{currentstroke}{rgb}{0.000000,0.000000,0.000000}%
\pgfsetstrokecolor{currentstroke}%
\pgfsetdash{}{0pt}%
\pgfpathmoveto{\pgfqpoint{5.049381in}{2.904573in}}%
\pgfpathlineto{\pgfqpoint{5.062527in}{2.902446in}}%
\pgfpathlineto{\pgfqpoint{5.075681in}{2.900438in}}%
\pgfpathlineto{\pgfqpoint{5.088845in}{2.898548in}}%
\pgfpathlineto{\pgfqpoint{5.102018in}{2.896775in}}%
\pgfpathlineto{\pgfqpoint{5.109170in}{2.907557in}}%
\pgfpathlineto{\pgfqpoint{5.116319in}{2.918394in}}%
\pgfpathlineto{\pgfqpoint{5.123466in}{2.929289in}}%
\pgfpathlineto{\pgfqpoint{5.130609in}{2.940243in}}%
\pgfpathlineto{\pgfqpoint{5.117449in}{2.942225in}}%
\pgfpathlineto{\pgfqpoint{5.104298in}{2.944325in}}%
\pgfpathlineto{\pgfqpoint{5.091155in}{2.946543in}}%
\pgfpathlineto{\pgfqpoint{5.078022in}{2.948879in}}%
\pgfpathlineto{\pgfqpoint{5.070866in}{2.937710in}}%
\pgfpathlineto{\pgfqpoint{5.063707in}{2.926604in}}%
\pgfpathlineto{\pgfqpoint{5.056545in}{2.915559in}}%
\pgfpathlineto{\pgfqpoint{5.049381in}{2.904573in}}%
\pgfpathclose%
\pgfusepath{fill}%
\end{pgfscope}%
\begin{pgfscope}%
\pgfpathrectangle{\pgfqpoint{1.254980in}{0.150000in}}{\pgfqpoint{5.490039in}{5.490039in}}%
\pgfusepath{clip}%
\pgfsetbuttcap%
\pgfsetroundjoin%
\definecolor{currentfill}{rgb}{0.149039,0.508051,0.557250}%
\pgfsetfillcolor{currentfill}%
\pgfsetfillopacity{0.700000}%
\pgfsetlinewidth{0.000000pt}%
\definecolor{currentstroke}{rgb}{0.000000,0.000000,0.000000}%
\pgfsetstrokecolor{currentstroke}%
\pgfsetdash{}{0pt}%
\pgfpathmoveto{\pgfqpoint{3.194850in}{3.430517in}}%
\pgfpathlineto{\pgfqpoint{3.207812in}{3.412485in}}%
\pgfpathlineto{\pgfqpoint{3.220768in}{3.394649in}}%
\pgfpathlineto{\pgfqpoint{3.233720in}{3.377010in}}%
\pgfpathlineto{\pgfqpoint{3.246668in}{3.359564in}}%
\pgfpathlineto{\pgfqpoint{3.254363in}{3.369978in}}%
\pgfpathlineto{\pgfqpoint{3.262052in}{3.380499in}}%
\pgfpathlineto{\pgfqpoint{3.269735in}{3.391128in}}%
\pgfpathlineto{\pgfqpoint{3.277411in}{3.401864in}}%
\pgfpathlineto{\pgfqpoint{3.264479in}{3.419327in}}%
\pgfpathlineto{\pgfqpoint{3.251544in}{3.436985in}}%
\pgfpathlineto{\pgfqpoint{3.238603in}{3.454839in}}%
\pgfpathlineto{\pgfqpoint{3.225658in}{3.472889in}}%
\pgfpathlineto{\pgfqpoint{3.217966in}{3.462129in}}%
\pgfpathlineto{\pgfqpoint{3.210267in}{3.451480in}}%
\pgfpathlineto{\pgfqpoint{3.202562in}{3.440943in}}%
\pgfpathlineto{\pgfqpoint{3.194850in}{3.430517in}}%
\pgfpathclose%
\pgfusepath{fill}%
\end{pgfscope}%
\begin{pgfscope}%
\pgfpathrectangle{\pgfqpoint{1.254980in}{0.150000in}}{\pgfqpoint{5.490039in}{5.490039in}}%
\pgfusepath{clip}%
\pgfsetbuttcap%
\pgfsetroundjoin%
\definecolor{currentfill}{rgb}{0.233603,0.313828,0.543914}%
\pgfsetfillcolor{currentfill}%
\pgfsetfillopacity{0.700000}%
\pgfsetlinewidth{0.000000pt}%
\definecolor{currentstroke}{rgb}{0.000000,0.000000,0.000000}%
\pgfsetstrokecolor{currentstroke}%
\pgfsetdash{}{0pt}%
\pgfpathmoveto{\pgfqpoint{3.608020in}{2.945204in}}%
\pgfpathlineto{\pgfqpoint{3.620904in}{2.932898in}}%
\pgfpathlineto{\pgfqpoint{3.633788in}{2.920755in}}%
\pgfpathlineto{\pgfqpoint{3.646672in}{2.908773in}}%
\pgfpathlineto{\pgfqpoint{3.659556in}{2.896951in}}%
\pgfpathlineto{\pgfqpoint{3.667133in}{2.907262in}}%
\pgfpathlineto{\pgfqpoint{3.674704in}{2.917648in}}%
\pgfpathlineto{\pgfqpoint{3.682270in}{2.928108in}}%
\pgfpathlineto{\pgfqpoint{3.689832in}{2.938644in}}%
\pgfpathlineto{\pgfqpoint{3.676961in}{2.950485in}}%
\pgfpathlineto{\pgfqpoint{3.664090in}{2.962487in}}%
\pgfpathlineto{\pgfqpoint{3.651219in}{2.974650in}}%
\pgfpathlineto{\pgfqpoint{3.638349in}{2.986975in}}%
\pgfpathlineto{\pgfqpoint{3.630774in}{2.976414in}}%
\pgfpathlineto{\pgfqpoint{3.623195in}{2.965932in}}%
\pgfpathlineto{\pgfqpoint{3.615610in}{2.955529in}}%
\pgfpathlineto{\pgfqpoint{3.608020in}{2.945204in}}%
\pgfpathclose%
\pgfusepath{fill}%
\end{pgfscope}%
\begin{pgfscope}%
\pgfpathrectangle{\pgfqpoint{1.254980in}{0.150000in}}{\pgfqpoint{5.490039in}{5.490039in}}%
\pgfusepath{clip}%
\pgfsetbuttcap%
\pgfsetroundjoin%
\definecolor{currentfill}{rgb}{0.244972,0.287675,0.537260}%
\pgfsetfillcolor{currentfill}%
\pgfsetfillopacity{0.700000}%
\pgfsetlinewidth{0.000000pt}%
\definecolor{currentstroke}{rgb}{0.000000,0.000000,0.000000}%
\pgfsetstrokecolor{currentstroke}%
\pgfsetdash{}{0pt}%
\pgfpathmoveto{\pgfqpoint{4.968148in}{2.870098in}}%
\pgfpathlineto{\pgfqpoint{4.981271in}{2.867690in}}%
\pgfpathlineto{\pgfqpoint{4.994403in}{2.865401in}}%
\pgfpathlineto{\pgfqpoint{5.007544in}{2.863231in}}%
\pgfpathlineto{\pgfqpoint{5.020693in}{2.861180in}}%
\pgfpathlineto{\pgfqpoint{5.027869in}{2.871950in}}%
\pgfpathlineto{\pgfqpoint{5.035043in}{2.882771in}}%
\pgfpathlineto{\pgfqpoint{5.042213in}{2.893644in}}%
\pgfpathlineto{\pgfqpoint{5.049381in}{2.904573in}}%
\pgfpathlineto{\pgfqpoint{5.036243in}{2.906818in}}%
\pgfpathlineto{\pgfqpoint{5.023115in}{2.909182in}}%
\pgfpathlineto{\pgfqpoint{5.009995in}{2.911665in}}%
\pgfpathlineto{\pgfqpoint{4.996884in}{2.914267in}}%
\pgfpathlineto{\pgfqpoint{4.989704in}{2.903139in}}%
\pgfpathlineto{\pgfqpoint{4.982522in}{2.892069in}}%
\pgfpathlineto{\pgfqpoint{4.975336in}{2.881057in}}%
\pgfpathlineto{\pgfqpoint{4.968148in}{2.870098in}}%
\pgfpathclose%
\pgfusepath{fill}%
\end{pgfscope}%
\begin{pgfscope}%
\pgfpathrectangle{\pgfqpoint{1.254980in}{0.150000in}}{\pgfqpoint{5.490039in}{5.490039in}}%
\pgfusepath{clip}%
\pgfsetbuttcap%
\pgfsetroundjoin%
\definecolor{currentfill}{rgb}{0.269308,0.218818,0.509577}%
\pgfsetfillcolor{currentfill}%
\pgfsetfillopacity{0.700000}%
\pgfsetlinewidth{0.000000pt}%
\definecolor{currentstroke}{rgb}{0.000000,0.000000,0.000000}%
\pgfsetstrokecolor{currentstroke}%
\pgfsetdash{}{0pt}%
\pgfpathmoveto{\pgfqpoint{4.376530in}{2.734345in}}%
\pgfpathlineto{\pgfqpoint{4.389491in}{2.728764in}}%
\pgfpathlineto{\pgfqpoint{4.402458in}{2.723314in}}%
\pgfpathlineto{\pgfqpoint{4.415431in}{2.717995in}}%
\pgfpathlineto{\pgfqpoint{4.428410in}{2.712807in}}%
\pgfpathlineto{\pgfqpoint{4.435760in}{2.723625in}}%
\pgfpathlineto{\pgfqpoint{4.443105in}{2.734487in}}%
\pgfpathlineto{\pgfqpoint{4.450447in}{2.745395in}}%
\pgfpathlineto{\pgfqpoint{4.457785in}{2.756348in}}%
\pgfpathlineto{\pgfqpoint{4.444817in}{2.761636in}}%
\pgfpathlineto{\pgfqpoint{4.431854in}{2.767054in}}%
\pgfpathlineto{\pgfqpoint{4.418898in}{2.772603in}}%
\pgfpathlineto{\pgfqpoint{4.405946in}{2.778284in}}%
\pgfpathlineto{\pgfqpoint{4.398598in}{2.767225in}}%
\pgfpathlineto{\pgfqpoint{4.391246in}{2.756216in}}%
\pgfpathlineto{\pgfqpoint{4.383890in}{2.745257in}}%
\pgfpathlineto{\pgfqpoint{4.376530in}{2.734345in}}%
\pgfpathclose%
\pgfusepath{fill}%
\end{pgfscope}%
\begin{pgfscope}%
\pgfpathrectangle{\pgfqpoint{1.254980in}{0.150000in}}{\pgfqpoint{5.490039in}{5.490039in}}%
\pgfusepath{clip}%
\pgfsetbuttcap%
\pgfsetroundjoin%
\definecolor{currentfill}{rgb}{0.258965,0.251537,0.524736}%
\pgfsetfillcolor{currentfill}%
\pgfsetfillopacity{0.700000}%
\pgfsetlinewidth{0.000000pt}%
\definecolor{currentstroke}{rgb}{0.000000,0.000000,0.000000}%
\pgfsetstrokecolor{currentstroke}%
\pgfsetdash{}{0pt}%
\pgfpathmoveto{\pgfqpoint{3.844347in}{2.808779in}}%
\pgfpathlineto{\pgfqpoint{3.857232in}{2.798952in}}%
\pgfpathlineto{\pgfqpoint{3.870119in}{2.789275in}}%
\pgfpathlineto{\pgfqpoint{3.883008in}{2.779747in}}%
\pgfpathlineto{\pgfqpoint{3.895900in}{2.770368in}}%
\pgfpathlineto{\pgfqpoint{3.903406in}{2.780860in}}%
\pgfpathlineto{\pgfqpoint{3.910907in}{2.791414in}}%
\pgfpathlineto{\pgfqpoint{3.918404in}{2.802030in}}%
\pgfpathlineto{\pgfqpoint{3.925896in}{2.812708in}}%
\pgfpathlineto{\pgfqpoint{3.913016in}{2.822123in}}%
\pgfpathlineto{\pgfqpoint{3.900139in}{2.831687in}}%
\pgfpathlineto{\pgfqpoint{3.887264in}{2.841400in}}%
\pgfpathlineto{\pgfqpoint{3.874390in}{2.851263in}}%
\pgfpathlineto{\pgfqpoint{3.866887in}{2.840543in}}%
\pgfpathlineto{\pgfqpoint{3.859378in}{2.829890in}}%
\pgfpathlineto{\pgfqpoint{3.851865in}{2.819302in}}%
\pgfpathlineto{\pgfqpoint{3.844347in}{2.808779in}}%
\pgfpathclose%
\pgfusepath{fill}%
\end{pgfscope}%
\begin{pgfscope}%
\pgfpathrectangle{\pgfqpoint{1.254980in}{0.150000in}}{\pgfqpoint{5.490039in}{5.490039in}}%
\pgfusepath{clip}%
\pgfsetbuttcap%
\pgfsetroundjoin%
\definecolor{currentfill}{rgb}{0.263663,0.237631,0.518762}%
\pgfsetfillcolor{currentfill}%
\pgfsetfillopacity{0.700000}%
\pgfsetlinewidth{0.000000pt}%
\definecolor{currentstroke}{rgb}{0.000000,0.000000,0.000000}%
\pgfsetstrokecolor{currentstroke}%
\pgfsetdash{}{0pt}%
\pgfpathmoveto{\pgfqpoint{4.590982in}{2.762074in}}%
\pgfpathlineto{\pgfqpoint{4.603996in}{2.757826in}}%
\pgfpathlineto{\pgfqpoint{4.617017in}{2.753704in}}%
\pgfpathlineto{\pgfqpoint{4.630045in}{2.749708in}}%
\pgfpathlineto{\pgfqpoint{4.643080in}{2.745838in}}%
\pgfpathlineto{\pgfqpoint{4.650368in}{2.756668in}}%
\pgfpathlineto{\pgfqpoint{4.657652in}{2.767541in}}%
\pgfpathlineto{\pgfqpoint{4.664932in}{2.778458in}}%
\pgfpathlineto{\pgfqpoint{4.672209in}{2.789421in}}%
\pgfpathlineto{\pgfqpoint{4.659185in}{2.793422in}}%
\pgfpathlineto{\pgfqpoint{4.646167in}{2.797549in}}%
\pgfpathlineto{\pgfqpoint{4.633156in}{2.801802in}}%
\pgfpathlineto{\pgfqpoint{4.620152in}{2.806180in}}%
\pgfpathlineto{\pgfqpoint{4.612865in}{2.795081in}}%
\pgfpathlineto{\pgfqpoint{4.605574in}{2.784031in}}%
\pgfpathlineto{\pgfqpoint{4.598280in}{2.773029in}}%
\pgfpathlineto{\pgfqpoint{4.590982in}{2.762074in}}%
\pgfpathclose%
\pgfusepath{fill}%
\end{pgfscope}%
\begin{pgfscope}%
\pgfpathrectangle{\pgfqpoint{1.254980in}{0.150000in}}{\pgfqpoint{5.490039in}{5.490039in}}%
\pgfusepath{clip}%
\pgfsetbuttcap%
\pgfsetroundjoin%
\definecolor{currentfill}{rgb}{0.137770,0.537492,0.554906}%
\pgfsetfillcolor{currentfill}%
\pgfsetfillopacity{0.700000}%
\pgfsetlinewidth{0.000000pt}%
\definecolor{currentstroke}{rgb}{0.000000,0.000000,0.000000}%
\pgfsetstrokecolor{currentstroke}%
\pgfsetdash{}{0pt}%
\pgfpathmoveto{\pgfqpoint{3.142954in}{3.504642in}}%
\pgfpathlineto{\pgfqpoint{3.155936in}{3.485808in}}%
\pgfpathlineto{\pgfqpoint{3.168912in}{3.467177in}}%
\pgfpathlineto{\pgfqpoint{3.181884in}{3.448747in}}%
\pgfpathlineto{\pgfqpoint{3.194850in}{3.430517in}}%
\pgfpathlineto{\pgfqpoint{3.202562in}{3.440943in}}%
\pgfpathlineto{\pgfqpoint{3.210267in}{3.451480in}}%
\pgfpathlineto{\pgfqpoint{3.217966in}{3.462129in}}%
\pgfpathlineto{\pgfqpoint{3.225658in}{3.472889in}}%
\pgfpathlineto{\pgfqpoint{3.212709in}{3.491137in}}%
\pgfpathlineto{\pgfqpoint{3.199754in}{3.509586in}}%
\pgfpathlineto{\pgfqpoint{3.186794in}{3.528234in}}%
\pgfpathlineto{\pgfqpoint{3.173829in}{3.547086in}}%
\pgfpathlineto{\pgfqpoint{3.166120in}{3.536301in}}%
\pgfpathlineto{\pgfqpoint{3.158404in}{3.525633in}}%
\pgfpathlineto{\pgfqpoint{3.150682in}{3.515080in}}%
\pgfpathlineto{\pgfqpoint{3.142954in}{3.504642in}}%
\pgfpathclose%
\pgfusepath{fill}%
\end{pgfscope}%
\begin{pgfscope}%
\pgfpathrectangle{\pgfqpoint{1.254980in}{0.150000in}}{\pgfqpoint{5.490039in}{5.490039in}}%
\pgfusepath{clip}%
\pgfsetbuttcap%
\pgfsetroundjoin%
\definecolor{currentfill}{rgb}{0.250425,0.274290,0.533103}%
\pgfsetfillcolor{currentfill}%
\pgfsetfillopacity{0.700000}%
\pgfsetlinewidth{0.000000pt}%
\definecolor{currentstroke}{rgb}{0.000000,0.000000,0.000000}%
\pgfsetstrokecolor{currentstroke}%
\pgfsetdash{}{0pt}%
\pgfpathmoveto{\pgfqpoint{4.886907in}{2.836896in}}%
\pgfpathlineto{\pgfqpoint{4.900008in}{2.834185in}}%
\pgfpathlineto{\pgfqpoint{4.913118in}{2.831594in}}%
\pgfpathlineto{\pgfqpoint{4.926236in}{2.829124in}}%
\pgfpathlineto{\pgfqpoint{4.939362in}{2.826774in}}%
\pgfpathlineto{\pgfqpoint{4.946563in}{2.837533in}}%
\pgfpathlineto{\pgfqpoint{4.953762in}{2.848339in}}%
\pgfpathlineto{\pgfqpoint{4.960956in}{2.859193in}}%
\pgfpathlineto{\pgfqpoint{4.968148in}{2.870098in}}%
\pgfpathlineto{\pgfqpoint{4.955033in}{2.872627in}}%
\pgfpathlineto{\pgfqpoint{4.941927in}{2.875275in}}%
\pgfpathlineto{\pgfqpoint{4.928829in}{2.878044in}}%
\pgfpathlineto{\pgfqpoint{4.915739in}{2.880933in}}%
\pgfpathlineto{\pgfqpoint{4.908535in}{2.869843in}}%
\pgfpathlineto{\pgfqpoint{4.901329in}{2.858809in}}%
\pgfpathlineto{\pgfqpoint{4.894119in}{2.847827in}}%
\pgfpathlineto{\pgfqpoint{4.886907in}{2.836896in}}%
\pgfpathclose%
\pgfusepath{fill}%
\end{pgfscope}%
\begin{pgfscope}%
\pgfpathrectangle{\pgfqpoint{1.254980in}{0.150000in}}{\pgfqpoint{5.490039in}{5.490039in}}%
\pgfusepath{clip}%
\pgfsetbuttcap%
\pgfsetroundjoin%
\definecolor{currentfill}{rgb}{0.241237,0.296485,0.539709}%
\pgfsetfillcolor{currentfill}%
\pgfsetfillopacity{0.700000}%
\pgfsetlinewidth{0.000000pt}%
\definecolor{currentstroke}{rgb}{0.000000,0.000000,0.000000}%
\pgfsetstrokecolor{currentstroke}%
\pgfsetdash{}{0pt}%
\pgfpathmoveto{\pgfqpoint{3.659556in}{2.896951in}}%
\pgfpathlineto{\pgfqpoint{3.672441in}{2.885290in}}%
\pgfpathlineto{\pgfqpoint{3.685326in}{2.873787in}}%
\pgfpathlineto{\pgfqpoint{3.698211in}{2.862442in}}%
\pgfpathlineto{\pgfqpoint{3.711098in}{2.851255in}}%
\pgfpathlineto{\pgfqpoint{3.718661in}{2.861552in}}%
\pgfpathlineto{\pgfqpoint{3.726219in}{2.871919in}}%
\pgfpathlineto{\pgfqpoint{3.733772in}{2.882358in}}%
\pgfpathlineto{\pgfqpoint{3.741320in}{2.892869in}}%
\pgfpathlineto{\pgfqpoint{3.728447in}{2.904076in}}%
\pgfpathlineto{\pgfqpoint{3.715575in}{2.915440in}}%
\pgfpathlineto{\pgfqpoint{3.702703in}{2.926963in}}%
\pgfpathlineto{\pgfqpoint{3.689832in}{2.938644in}}%
\pgfpathlineto{\pgfqpoint{3.682270in}{2.928108in}}%
\pgfpathlineto{\pgfqpoint{3.674704in}{2.917648in}}%
\pgfpathlineto{\pgfqpoint{3.667133in}{2.907262in}}%
\pgfpathlineto{\pgfqpoint{3.659556in}{2.896951in}}%
\pgfpathclose%
\pgfusepath{fill}%
\end{pgfscope}%
\begin{pgfscope}%
\pgfpathrectangle{\pgfqpoint{1.254980in}{0.150000in}}{\pgfqpoint{5.490039in}{5.490039in}}%
\pgfusepath{clip}%
\pgfsetbuttcap%
\pgfsetroundjoin%
\definecolor{currentfill}{rgb}{0.269308,0.218818,0.509577}%
\pgfsetfillcolor{currentfill}%
\pgfsetfillopacity{0.700000}%
\pgfsetlinewidth{0.000000pt}%
\definecolor{currentstroke}{rgb}{0.000000,0.000000,0.000000}%
\pgfsetstrokecolor{currentstroke}%
\pgfsetdash{}{0pt}%
\pgfpathmoveto{\pgfqpoint{4.162108in}{2.724217in}}%
\pgfpathlineto{\pgfqpoint{4.175033in}{2.717173in}}%
\pgfpathlineto{\pgfqpoint{4.187961in}{2.710267in}}%
\pgfpathlineto{\pgfqpoint{4.200894in}{2.703498in}}%
\pgfpathlineto{\pgfqpoint{4.213831in}{2.696864in}}%
\pgfpathlineto{\pgfqpoint{4.221245in}{2.707556in}}%
\pgfpathlineto{\pgfqpoint{4.228655in}{2.718295in}}%
\pgfpathlineto{\pgfqpoint{4.236060in}{2.729082in}}%
\pgfpathlineto{\pgfqpoint{4.243461in}{2.739919in}}%
\pgfpathlineto{\pgfqpoint{4.230535in}{2.746620in}}%
\pgfpathlineto{\pgfqpoint{4.217613in}{2.753457in}}%
\pgfpathlineto{\pgfqpoint{4.204695in}{2.760432in}}%
\pgfpathlineto{\pgfqpoint{4.191782in}{2.767543in}}%
\pgfpathlineto{\pgfqpoint{4.184370in}{2.756633in}}%
\pgfpathlineto{\pgfqpoint{4.176954in}{2.745776in}}%
\pgfpathlineto{\pgfqpoint{4.169533in}{2.734971in}}%
\pgfpathlineto{\pgfqpoint{4.162108in}{2.724217in}}%
\pgfpathclose%
\pgfusepath{fill}%
\end{pgfscope}%
\begin{pgfscope}%
\pgfpathrectangle{\pgfqpoint{1.254980in}{0.150000in}}{\pgfqpoint{5.490039in}{5.490039in}}%
\pgfusepath{clip}%
\pgfsetbuttcap%
\pgfsetroundjoin%
\definecolor{currentfill}{rgb}{0.267968,0.223549,0.512008}%
\pgfsetfillcolor{currentfill}%
\pgfsetfillopacity{0.700000}%
\pgfsetlinewidth{0.000000pt}%
\definecolor{currentstroke}{rgb}{0.000000,0.000000,0.000000}%
\pgfsetstrokecolor{currentstroke}%
\pgfsetdash{}{0pt}%
\pgfpathmoveto{\pgfqpoint{4.029027in}{2.742642in}}%
\pgfpathlineto{\pgfqpoint{4.041932in}{2.734532in}}%
\pgfpathlineto{\pgfqpoint{4.054840in}{2.726563in}}%
\pgfpathlineto{\pgfqpoint{4.067752in}{2.718736in}}%
\pgfpathlineto{\pgfqpoint{4.080667in}{2.711050in}}%
\pgfpathlineto{\pgfqpoint{4.088120in}{2.721652in}}%
\pgfpathlineto{\pgfqpoint{4.095569in}{2.732307in}}%
\pgfpathlineto{\pgfqpoint{4.103013in}{2.743015in}}%
\pgfpathlineto{\pgfqpoint{4.110453in}{2.753776in}}%
\pgfpathlineto{\pgfqpoint{4.097549in}{2.761514in}}%
\pgfpathlineto{\pgfqpoint{4.084648in}{2.769394in}}%
\pgfpathlineto{\pgfqpoint{4.071751in}{2.777414in}}%
\pgfpathlineto{\pgfqpoint{4.058857in}{2.785577in}}%
\pgfpathlineto{\pgfqpoint{4.051406in}{2.774758in}}%
\pgfpathlineto{\pgfqpoint{4.043951in}{2.763996in}}%
\pgfpathlineto{\pgfqpoint{4.036491in}{2.753291in}}%
\pgfpathlineto{\pgfqpoint{4.029027in}{2.742642in}}%
\pgfpathclose%
\pgfusepath{fill}%
\end{pgfscope}%
\begin{pgfscope}%
\pgfpathrectangle{\pgfqpoint{1.254980in}{0.150000in}}{\pgfqpoint{5.490039in}{5.490039in}}%
\pgfusepath{clip}%
\pgfsetbuttcap%
\pgfsetroundjoin%
\definecolor{currentfill}{rgb}{0.266580,0.228262,0.514349}%
\pgfsetfillcolor{currentfill}%
\pgfsetfillopacity{0.700000}%
\pgfsetlinewidth{0.000000pt}%
\definecolor{currentstroke}{rgb}{0.000000,0.000000,0.000000}%
\pgfsetstrokecolor{currentstroke}%
\pgfsetdash{}{0pt}%
\pgfpathmoveto{\pgfqpoint{4.509718in}{2.736493in}}%
\pgfpathlineto{\pgfqpoint{4.522717in}{2.731851in}}%
\pgfpathlineto{\pgfqpoint{4.535722in}{2.727337in}}%
\pgfpathlineto{\pgfqpoint{4.548733in}{2.722951in}}%
\pgfpathlineto{\pgfqpoint{4.561751in}{2.718691in}}%
\pgfpathlineto{\pgfqpoint{4.569065in}{2.729474in}}%
\pgfpathlineto{\pgfqpoint{4.576374in}{2.740298in}}%
\pgfpathlineto{\pgfqpoint{4.583680in}{2.751164in}}%
\pgfpathlineto{\pgfqpoint{4.590982in}{2.762074in}}%
\pgfpathlineto{\pgfqpoint{4.577974in}{2.766449in}}%
\pgfpathlineto{\pgfqpoint{4.564973in}{2.770950in}}%
\pgfpathlineto{\pgfqpoint{4.551978in}{2.775580in}}%
\pgfpathlineto{\pgfqpoint{4.538990in}{2.780337in}}%
\pgfpathlineto{\pgfqpoint{4.531677in}{2.769306in}}%
\pgfpathlineto{\pgfqpoint{4.524361in}{2.758322in}}%
\pgfpathlineto{\pgfqpoint{4.517042in}{2.747385in}}%
\pgfpathlineto{\pgfqpoint{4.509718in}{2.736493in}}%
\pgfpathclose%
\pgfusepath{fill}%
\end{pgfscope}%
\begin{pgfscope}%
\pgfpathrectangle{\pgfqpoint{1.254980in}{0.150000in}}{\pgfqpoint{5.490039in}{5.490039in}}%
\pgfusepath{clip}%
\pgfsetbuttcap%
\pgfsetroundjoin%
\definecolor{currentfill}{rgb}{0.255645,0.260703,0.528312}%
\pgfsetfillcolor{currentfill}%
\pgfsetfillopacity{0.700000}%
\pgfsetlinewidth{0.000000pt}%
\definecolor{currentstroke}{rgb}{0.000000,0.000000,0.000000}%
\pgfsetstrokecolor{currentstroke}%
\pgfsetdash{}{0pt}%
\pgfpathmoveto{\pgfqpoint{4.805652in}{2.805052in}}%
\pgfpathlineto{\pgfqpoint{4.818732in}{2.802018in}}%
\pgfpathlineto{\pgfqpoint{4.831821in}{2.799105in}}%
\pgfpathlineto{\pgfqpoint{4.844917in}{2.796314in}}%
\pgfpathlineto{\pgfqpoint{4.858022in}{2.793645in}}%
\pgfpathlineto{\pgfqpoint{4.865248in}{2.804390in}}%
\pgfpathlineto{\pgfqpoint{4.872471in}{2.815179in}}%
\pgfpathlineto{\pgfqpoint{4.879691in}{2.826014in}}%
\pgfpathlineto{\pgfqpoint{4.886907in}{2.836896in}}%
\pgfpathlineto{\pgfqpoint{4.873813in}{2.839728in}}%
\pgfpathlineto{\pgfqpoint{4.860728in}{2.842681in}}%
\pgfpathlineto{\pgfqpoint{4.847651in}{2.845756in}}%
\pgfpathlineto{\pgfqpoint{4.834581in}{2.848953in}}%
\pgfpathlineto{\pgfqpoint{4.827354in}{2.837903in}}%
\pgfpathlineto{\pgfqpoint{4.820123in}{2.826904in}}%
\pgfpathlineto{\pgfqpoint{4.812889in}{2.815954in}}%
\pgfpathlineto{\pgfqpoint{4.805652in}{2.805052in}}%
\pgfpathclose%
\pgfusepath{fill}%
\end{pgfscope}%
\begin{pgfscope}%
\pgfpathrectangle{\pgfqpoint{1.254980in}{0.150000in}}{\pgfqpoint{5.490039in}{5.490039in}}%
\pgfusepath{clip}%
\pgfsetbuttcap%
\pgfsetroundjoin%
\definecolor{currentfill}{rgb}{0.270595,0.214069,0.507052}%
\pgfsetfillcolor{currentfill}%
\pgfsetfillopacity{0.700000}%
\pgfsetlinewidth{0.000000pt}%
\definecolor{currentstroke}{rgb}{0.000000,0.000000,0.000000}%
\pgfsetstrokecolor{currentstroke}%
\pgfsetdash{}{0pt}%
\pgfpathmoveto{\pgfqpoint{4.295215in}{2.714467in}}%
\pgfpathlineto{\pgfqpoint{4.308166in}{2.708439in}}%
\pgfpathlineto{\pgfqpoint{4.321122in}{2.702545in}}%
\pgfpathlineto{\pgfqpoint{4.334083in}{2.696784in}}%
\pgfpathlineto{\pgfqpoint{4.347049in}{2.691155in}}%
\pgfpathlineto{\pgfqpoint{4.354425in}{2.701886in}}%
\pgfpathlineto{\pgfqpoint{4.361798in}{2.712661in}}%
\pgfpathlineto{\pgfqpoint{4.369166in}{2.723480in}}%
\pgfpathlineto{\pgfqpoint{4.376530in}{2.734345in}}%
\pgfpathlineto{\pgfqpoint{4.363574in}{2.740058in}}%
\pgfpathlineto{\pgfqpoint{4.350623in}{2.745903in}}%
\pgfpathlineto{\pgfqpoint{4.337677in}{2.751881in}}%
\pgfpathlineto{\pgfqpoint{4.324737in}{2.757992in}}%
\pgfpathlineto{\pgfqpoint{4.317363in}{2.747037in}}%
\pgfpathlineto{\pgfqpoint{4.309984in}{2.736132in}}%
\pgfpathlineto{\pgfqpoint{4.302602in}{2.725276in}}%
\pgfpathlineto{\pgfqpoint{4.295215in}{2.714467in}}%
\pgfpathclose%
\pgfusepath{fill}%
\end{pgfscope}%
\begin{pgfscope}%
\pgfpathrectangle{\pgfqpoint{1.254980in}{0.150000in}}{\pgfqpoint{5.490039in}{5.490039in}}%
\pgfusepath{clip}%
\pgfsetbuttcap%
\pgfsetroundjoin%
\definecolor{currentfill}{rgb}{0.263663,0.237631,0.518762}%
\pgfsetfillcolor{currentfill}%
\pgfsetfillopacity{0.700000}%
\pgfsetlinewidth{0.000000pt}%
\definecolor{currentstroke}{rgb}{0.000000,0.000000,0.000000}%
\pgfsetstrokecolor{currentstroke}%
\pgfsetdash{}{0pt}%
\pgfpathmoveto{\pgfqpoint{3.895900in}{2.770368in}}%
\pgfpathlineto{\pgfqpoint{3.908794in}{2.761136in}}%
\pgfpathlineto{\pgfqpoint{3.921690in}{2.752051in}}%
\pgfpathlineto{\pgfqpoint{3.934589in}{2.743112in}}%
\pgfpathlineto{\pgfqpoint{3.947490in}{2.734319in}}%
\pgfpathlineto{\pgfqpoint{3.954984in}{2.744782in}}%
\pgfpathlineto{\pgfqpoint{3.962474in}{2.755302in}}%
\pgfpathlineto{\pgfqpoint{3.969958in}{2.765879in}}%
\pgfpathlineto{\pgfqpoint{3.977439in}{2.776515in}}%
\pgfpathlineto{\pgfqpoint{3.964549in}{2.785344in}}%
\pgfpathlineto{\pgfqpoint{3.951662in}{2.794319in}}%
\pgfpathlineto{\pgfqpoint{3.938778in}{2.803440in}}%
\pgfpathlineto{\pgfqpoint{3.925896in}{2.812708in}}%
\pgfpathlineto{\pgfqpoint{3.918404in}{2.802030in}}%
\pgfpathlineto{\pgfqpoint{3.910907in}{2.791414in}}%
\pgfpathlineto{\pgfqpoint{3.903406in}{2.780860in}}%
\pgfpathlineto{\pgfqpoint{3.895900in}{2.770368in}}%
\pgfpathclose%
\pgfusepath{fill}%
\end{pgfscope}%
\begin{pgfscope}%
\pgfpathrectangle{\pgfqpoint{1.254980in}{0.150000in}}{\pgfqpoint{5.490039in}{5.490039in}}%
\pgfusepath{clip}%
\pgfsetbuttcap%
\pgfsetroundjoin%
\definecolor{currentfill}{rgb}{0.127568,0.566949,0.550556}%
\pgfsetfillcolor{currentfill}%
\pgfsetfillopacity{0.700000}%
\pgfsetlinewidth{0.000000pt}%
\definecolor{currentstroke}{rgb}{0.000000,0.000000,0.000000}%
\pgfsetstrokecolor{currentstroke}%
\pgfsetdash{}{0pt}%
\pgfpathmoveto{\pgfqpoint{3.090967in}{3.582026in}}%
\pgfpathlineto{\pgfqpoint{3.103973in}{3.562370in}}%
\pgfpathlineto{\pgfqpoint{3.116972in}{3.542921in}}%
\pgfpathlineto{\pgfqpoint{3.129966in}{3.523679in}}%
\pgfpathlineto{\pgfqpoint{3.142954in}{3.504642in}}%
\pgfpathlineto{\pgfqpoint{3.150682in}{3.515080in}}%
\pgfpathlineto{\pgfqpoint{3.158404in}{3.525633in}}%
\pgfpathlineto{\pgfqpoint{3.166120in}{3.536301in}}%
\pgfpathlineto{\pgfqpoint{3.173829in}{3.547086in}}%
\pgfpathlineto{\pgfqpoint{3.160858in}{3.566141in}}%
\pgfpathlineto{\pgfqpoint{3.147881in}{3.585400in}}%
\pgfpathlineto{\pgfqpoint{3.134899in}{3.604867in}}%
\pgfpathlineto{\pgfqpoint{3.121911in}{3.624542in}}%
\pgfpathlineto{\pgfqpoint{3.114185in}{3.613733in}}%
\pgfpathlineto{\pgfqpoint{3.106452in}{3.603044in}}%
\pgfpathlineto{\pgfqpoint{3.098713in}{3.592476in}}%
\pgfpathlineto{\pgfqpoint{3.090967in}{3.582026in}}%
\pgfpathclose%
\pgfusepath{fill}%
\end{pgfscope}%
\begin{pgfscope}%
\pgfpathrectangle{\pgfqpoint{1.254980in}{0.150000in}}{\pgfqpoint{5.490039in}{5.490039in}}%
\pgfusepath{clip}%
\pgfsetbuttcap%
\pgfsetroundjoin%
\definecolor{currentfill}{rgb}{0.250425,0.274290,0.533103}%
\pgfsetfillcolor{currentfill}%
\pgfsetfillopacity{0.700000}%
\pgfsetlinewidth{0.000000pt}%
\definecolor{currentstroke}{rgb}{0.000000,0.000000,0.000000}%
\pgfsetstrokecolor{currentstroke}%
\pgfsetdash{}{0pt}%
\pgfpathmoveto{\pgfqpoint{3.711098in}{2.851255in}}%
\pgfpathlineto{\pgfqpoint{3.723985in}{2.840224in}}%
\pgfpathlineto{\pgfqpoint{3.736872in}{2.829349in}}%
\pgfpathlineto{\pgfqpoint{3.749761in}{2.818629in}}%
\pgfpathlineto{\pgfqpoint{3.762651in}{2.808063in}}%
\pgfpathlineto{\pgfqpoint{3.770202in}{2.818346in}}%
\pgfpathlineto{\pgfqpoint{3.777747in}{2.828695in}}%
\pgfpathlineto{\pgfqpoint{3.785287in}{2.839112in}}%
\pgfpathlineto{\pgfqpoint{3.792823in}{2.849597in}}%
\pgfpathlineto{\pgfqpoint{3.779945in}{2.860183in}}%
\pgfpathlineto{\pgfqpoint{3.767069in}{2.870923in}}%
\pgfpathlineto{\pgfqpoint{3.754194in}{2.881818in}}%
\pgfpathlineto{\pgfqpoint{3.741320in}{2.892869in}}%
\pgfpathlineto{\pgfqpoint{3.733772in}{2.882358in}}%
\pgfpathlineto{\pgfqpoint{3.726219in}{2.871919in}}%
\pgfpathlineto{\pgfqpoint{3.718661in}{2.861552in}}%
\pgfpathlineto{\pgfqpoint{3.711098in}{2.851255in}}%
\pgfpathclose%
\pgfusepath{fill}%
\end{pgfscope}%
\begin{pgfscope}%
\pgfpathrectangle{\pgfqpoint{1.254980in}{0.150000in}}{\pgfqpoint{5.490039in}{5.490039in}}%
\pgfusepath{clip}%
\pgfsetbuttcap%
\pgfsetroundjoin%
\definecolor{currentfill}{rgb}{0.260571,0.246922,0.522828}%
\pgfsetfillcolor{currentfill}%
\pgfsetfillopacity{0.700000}%
\pgfsetlinewidth{0.000000pt}%
\definecolor{currentstroke}{rgb}{0.000000,0.000000,0.000000}%
\pgfsetstrokecolor{currentstroke}%
\pgfsetdash{}{0pt}%
\pgfpathmoveto{\pgfqpoint{4.724378in}{2.774664in}}%
\pgfpathlineto{\pgfqpoint{4.737439in}{2.771285in}}%
\pgfpathlineto{\pgfqpoint{4.750507in}{2.768029in}}%
\pgfpathlineto{\pgfqpoint{4.763583in}{2.764896in}}%
\pgfpathlineto{\pgfqpoint{4.776667in}{2.761887in}}%
\pgfpathlineto{\pgfqpoint{4.783918in}{2.772615in}}%
\pgfpathlineto{\pgfqpoint{4.791167in}{2.783384in}}%
\pgfpathlineto{\pgfqpoint{4.798411in}{2.794196in}}%
\pgfpathlineto{\pgfqpoint{4.805652in}{2.805052in}}%
\pgfpathlineto{\pgfqpoint{4.792579in}{2.808209in}}%
\pgfpathlineto{\pgfqpoint{4.779514in}{2.811489in}}%
\pgfpathlineto{\pgfqpoint{4.766457in}{2.814891in}}%
\pgfpathlineto{\pgfqpoint{4.753407in}{2.818417in}}%
\pgfpathlineto{\pgfqpoint{4.746155in}{2.807408in}}%
\pgfpathlineto{\pgfqpoint{4.738900in}{2.796447in}}%
\pgfpathlineto{\pgfqpoint{4.731641in}{2.785533in}}%
\pgfpathlineto{\pgfqpoint{4.724378in}{2.774664in}}%
\pgfpathclose%
\pgfusepath{fill}%
\end{pgfscope}%
\begin{pgfscope}%
\pgfpathrectangle{\pgfqpoint{1.254980in}{0.150000in}}{\pgfqpoint{5.490039in}{5.490039in}}%
\pgfusepath{clip}%
\pgfsetbuttcap%
\pgfsetroundjoin%
\definecolor{currentfill}{rgb}{0.197636,0.391528,0.554969}%
\pgfsetfillcolor{currentfill}%
\pgfsetfillopacity{0.700000}%
\pgfsetlinewidth{0.000000pt}%
\definecolor{currentstroke}{rgb}{0.000000,0.000000,0.000000}%
\pgfsetstrokecolor{currentstroke}%
\pgfsetdash{}{0pt}%
\pgfpathmoveto{\pgfqpoint{3.371099in}{3.124354in}}%
\pgfpathlineto{\pgfqpoint{3.384019in}{3.109327in}}%
\pgfpathlineto{\pgfqpoint{3.396936in}{3.094477in}}%
\pgfpathlineto{\pgfqpoint{3.409851in}{3.079804in}}%
\pgfpathlineto{\pgfqpoint{3.422764in}{3.065307in}}%
\pgfpathlineto{\pgfqpoint{3.430420in}{3.075313in}}%
\pgfpathlineto{\pgfqpoint{3.438070in}{3.085407in}}%
\pgfpathlineto{\pgfqpoint{3.445715in}{3.095591in}}%
\pgfpathlineto{\pgfqpoint{3.453354in}{3.105865in}}%
\pgfpathlineto{\pgfqpoint{3.440456in}{3.120365in}}%
\pgfpathlineto{\pgfqpoint{3.427556in}{3.135041in}}%
\pgfpathlineto{\pgfqpoint{3.414655in}{3.149894in}}%
\pgfpathlineto{\pgfqpoint{3.401750in}{3.164924in}}%
\pgfpathlineto{\pgfqpoint{3.394097in}{3.154641in}}%
\pgfpathlineto{\pgfqpoint{3.386437in}{3.144452in}}%
\pgfpathlineto{\pgfqpoint{3.378771in}{3.134357in}}%
\pgfpathlineto{\pgfqpoint{3.371099in}{3.124354in}}%
\pgfpathclose%
\pgfusepath{fill}%
\end{pgfscope}%
\begin{pgfscope}%
\pgfpathrectangle{\pgfqpoint{1.254980in}{0.150000in}}{\pgfqpoint{5.490039in}{5.490039in}}%
\pgfusepath{clip}%
\pgfsetbuttcap%
\pgfsetroundjoin%
\definecolor{currentfill}{rgb}{0.185556,0.418570,0.556753}%
\pgfsetfillcolor{currentfill}%
\pgfsetfillopacity{0.700000}%
\pgfsetlinewidth{0.000000pt}%
\definecolor{currentstroke}{rgb}{0.000000,0.000000,0.000000}%
\pgfsetstrokecolor{currentstroke}%
\pgfsetdash{}{0pt}%
\pgfpathmoveto{\pgfqpoint{3.319394in}{3.186265in}}%
\pgfpathlineto{\pgfqpoint{3.332325in}{3.170515in}}%
\pgfpathlineto{\pgfqpoint{3.345252in}{3.154947in}}%
\pgfpathlineto{\pgfqpoint{3.358177in}{3.139561in}}%
\pgfpathlineto{\pgfqpoint{3.371099in}{3.124354in}}%
\pgfpathlineto{\pgfqpoint{3.378771in}{3.134357in}}%
\pgfpathlineto{\pgfqpoint{3.386437in}{3.144452in}}%
\pgfpathlineto{\pgfqpoint{3.394097in}{3.154641in}}%
\pgfpathlineto{\pgfqpoint{3.401750in}{3.164924in}}%
\pgfpathlineto{\pgfqpoint{3.388844in}{3.180133in}}%
\pgfpathlineto{\pgfqpoint{3.375935in}{3.195522in}}%
\pgfpathlineto{\pgfqpoint{3.363023in}{3.211092in}}%
\pgfpathlineto{\pgfqpoint{3.350108in}{3.226845in}}%
\pgfpathlineto{\pgfqpoint{3.342439in}{3.216554in}}%
\pgfpathlineto{\pgfqpoint{3.334763in}{3.206360in}}%
\pgfpathlineto{\pgfqpoint{3.327082in}{3.196264in}}%
\pgfpathlineto{\pgfqpoint{3.319394in}{3.186265in}}%
\pgfpathclose%
\pgfusepath{fill}%
\end{pgfscope}%
\begin{pgfscope}%
\pgfpathrectangle{\pgfqpoint{1.254980in}{0.150000in}}{\pgfqpoint{5.490039in}{5.490039in}}%
\pgfusepath{clip}%
\pgfsetbuttcap%
\pgfsetroundjoin%
\definecolor{currentfill}{rgb}{0.227802,0.326594,0.546532}%
\pgfsetfillcolor{currentfill}%
\pgfsetfillopacity{0.700000}%
\pgfsetlinewidth{0.000000pt}%
\definecolor{currentstroke}{rgb}{0.000000,0.000000,0.000000}%
\pgfsetstrokecolor{currentstroke}%
\pgfsetdash{}{0pt}%
\pgfpathmoveto{\pgfqpoint{5.183342in}{2.933484in}}%
\pgfpathlineto{\pgfqpoint{5.196549in}{2.932086in}}%
\pgfpathlineto{\pgfqpoint{5.209765in}{2.930805in}}%
\pgfpathlineto{\pgfqpoint{5.222991in}{2.929639in}}%
\pgfpathlineto{\pgfqpoint{5.236226in}{2.928589in}}%
\pgfpathlineto{\pgfqpoint{5.243341in}{2.939165in}}%
\pgfpathlineto{\pgfqpoint{5.250453in}{2.949797in}}%
\pgfpathlineto{\pgfqpoint{5.257562in}{2.960489in}}%
\pgfpathlineto{\pgfqpoint{5.264669in}{2.971241in}}%
\pgfpathlineto{\pgfqpoint{5.251447in}{2.972517in}}%
\pgfpathlineto{\pgfqpoint{5.238234in}{2.973909in}}%
\pgfpathlineto{\pgfqpoint{5.225031in}{2.975417in}}%
\pgfpathlineto{\pgfqpoint{5.211838in}{2.977041in}}%
\pgfpathlineto{\pgfqpoint{5.204718in}{2.966057in}}%
\pgfpathlineto{\pgfqpoint{5.197595in}{2.955137in}}%
\pgfpathlineto{\pgfqpoint{5.190470in}{2.944281in}}%
\pgfpathlineto{\pgfqpoint{5.183342in}{2.933484in}}%
\pgfpathclose%
\pgfusepath{fill}%
\end{pgfscope}%
\begin{pgfscope}%
\pgfpathrectangle{\pgfqpoint{1.254980in}{0.150000in}}{\pgfqpoint{5.490039in}{5.490039in}}%
\pgfusepath{clip}%
\pgfsetbuttcap%
\pgfsetroundjoin%
\definecolor{currentfill}{rgb}{0.206756,0.371758,0.553117}%
\pgfsetfillcolor{currentfill}%
\pgfsetfillopacity{0.700000}%
\pgfsetlinewidth{0.000000pt}%
\definecolor{currentstroke}{rgb}{0.000000,0.000000,0.000000}%
\pgfsetstrokecolor{currentstroke}%
\pgfsetdash{}{0pt}%
\pgfpathmoveto{\pgfqpoint{3.422764in}{3.065307in}}%
\pgfpathlineto{\pgfqpoint{3.435675in}{3.050984in}}%
\pgfpathlineto{\pgfqpoint{3.448584in}{3.036836in}}%
\pgfpathlineto{\pgfqpoint{3.461491in}{3.022860in}}%
\pgfpathlineto{\pgfqpoint{3.474397in}{3.009055in}}%
\pgfpathlineto{\pgfqpoint{3.482038in}{3.019064in}}%
\pgfpathlineto{\pgfqpoint{3.489673in}{3.029158in}}%
\pgfpathlineto{\pgfqpoint{3.497303in}{3.039337in}}%
\pgfpathlineto{\pgfqpoint{3.504927in}{3.049602in}}%
\pgfpathlineto{\pgfqpoint{3.492036in}{3.063409in}}%
\pgfpathlineto{\pgfqpoint{3.479144in}{3.077388in}}%
\pgfpathlineto{\pgfqpoint{3.466249in}{3.091540in}}%
\pgfpathlineto{\pgfqpoint{3.453354in}{3.105865in}}%
\pgfpathlineto{\pgfqpoint{3.445715in}{3.095591in}}%
\pgfpathlineto{\pgfqpoint{3.438070in}{3.085407in}}%
\pgfpathlineto{\pgfqpoint{3.430420in}{3.075313in}}%
\pgfpathlineto{\pgfqpoint{3.422764in}{3.065307in}}%
\pgfpathclose%
\pgfusepath{fill}%
\end{pgfscope}%
\begin{pgfscope}%
\pgfpathrectangle{\pgfqpoint{1.254980in}{0.150000in}}{\pgfqpoint{5.490039in}{5.490039in}}%
\pgfusepath{clip}%
\pgfsetbuttcap%
\pgfsetroundjoin%
\definecolor{currentfill}{rgb}{0.221989,0.339161,0.548752}%
\pgfsetfillcolor{currentfill}%
\pgfsetfillopacity{0.700000}%
\pgfsetlinewidth{0.000000pt}%
\definecolor{currentstroke}{rgb}{0.000000,0.000000,0.000000}%
\pgfsetstrokecolor{currentstroke}%
\pgfsetdash{}{0pt}%
\pgfpathmoveto{\pgfqpoint{5.264669in}{2.971241in}}%
\pgfpathlineto{\pgfqpoint{5.277901in}{2.970080in}}%
\pgfpathlineto{\pgfqpoint{5.291142in}{2.969034in}}%
\pgfpathlineto{\pgfqpoint{5.304394in}{2.968104in}}%
\pgfpathlineto{\pgfqpoint{5.317655in}{2.967287in}}%
\pgfpathlineto{\pgfqpoint{5.324746in}{2.977867in}}%
\pgfpathlineto{\pgfqpoint{5.331833in}{2.988509in}}%
\pgfpathlineto{\pgfqpoint{5.338919in}{2.999216in}}%
\pgfpathlineto{\pgfqpoint{5.325668in}{3.000212in}}%
\pgfpathlineto{\pgfqpoint{5.312426in}{3.001323in}}%
\pgfpathlineto{\pgfqpoint{5.299195in}{3.002548in}}%
\pgfpathlineto{\pgfqpoint{5.285974in}{3.003889in}}%
\pgfpathlineto{\pgfqpoint{5.278874in}{2.992939in}}%
\pgfpathlineto{\pgfqpoint{5.271773in}{2.982057in}}%
\pgfpathlineto{\pgfqpoint{5.264669in}{2.971241in}}%
\pgfpathclose%
\pgfusepath{fill}%
\end{pgfscope}%
\begin{pgfscope}%
\pgfpathrectangle{\pgfqpoint{1.254980in}{0.150000in}}{\pgfqpoint{5.490039in}{5.490039in}}%
\pgfusepath{clip}%
\pgfsetbuttcap%
\pgfsetroundjoin%
\definecolor{currentfill}{rgb}{0.269308,0.218818,0.509577}%
\pgfsetfillcolor{currentfill}%
\pgfsetfillopacity{0.700000}%
\pgfsetlinewidth{0.000000pt}%
\definecolor{currentstroke}{rgb}{0.000000,0.000000,0.000000}%
\pgfsetstrokecolor{currentstroke}%
\pgfsetdash{}{0pt}%
\pgfpathmoveto{\pgfqpoint{4.428410in}{2.712807in}}%
\pgfpathlineto{\pgfqpoint{4.441394in}{2.707749in}}%
\pgfpathlineto{\pgfqpoint{4.454385in}{2.702820in}}%
\pgfpathlineto{\pgfqpoint{4.467381in}{2.698021in}}%
\pgfpathlineto{\pgfqpoint{4.480384in}{2.693350in}}%
\pgfpathlineto{\pgfqpoint{4.487723in}{2.704075in}}%
\pgfpathlineto{\pgfqpoint{4.495059in}{2.714839in}}%
\pgfpathlineto{\pgfqpoint{4.502390in}{2.725645in}}%
\pgfpathlineto{\pgfqpoint{4.509718in}{2.736493in}}%
\pgfpathlineto{\pgfqpoint{4.496725in}{2.741263in}}%
\pgfpathlineto{\pgfqpoint{4.483739in}{2.746162in}}%
\pgfpathlineto{\pgfqpoint{4.470759in}{2.751191in}}%
\pgfpathlineto{\pgfqpoint{4.457785in}{2.756348in}}%
\pgfpathlineto{\pgfqpoint{4.450447in}{2.745395in}}%
\pgfpathlineto{\pgfqpoint{4.443105in}{2.734487in}}%
\pgfpathlineto{\pgfqpoint{4.435760in}{2.723625in}}%
\pgfpathlineto{\pgfqpoint{4.428410in}{2.712807in}}%
\pgfpathclose%
\pgfusepath{fill}%
\end{pgfscope}%
\begin{pgfscope}%
\pgfpathrectangle{\pgfqpoint{1.254980in}{0.150000in}}{\pgfqpoint{5.490039in}{5.490039in}}%
\pgfusepath{clip}%
\pgfsetbuttcap%
\pgfsetroundjoin%
\definecolor{currentfill}{rgb}{0.174274,0.445044,0.557792}%
\pgfsetfillcolor{currentfill}%
\pgfsetfillopacity{0.700000}%
\pgfsetlinewidth{0.000000pt}%
\definecolor{currentstroke}{rgb}{0.000000,0.000000,0.000000}%
\pgfsetstrokecolor{currentstroke}%
\pgfsetdash{}{0pt}%
\pgfpathmoveto{\pgfqpoint{3.267638in}{3.251111in}}%
\pgfpathlineto{\pgfqpoint{3.280583in}{3.234621in}}%
\pgfpathlineto{\pgfqpoint{3.293523in}{3.218317in}}%
\pgfpathlineto{\pgfqpoint{3.306460in}{3.202199in}}%
\pgfpathlineto{\pgfqpoint{3.319394in}{3.186265in}}%
\pgfpathlineto{\pgfqpoint{3.327082in}{3.196264in}}%
\pgfpathlineto{\pgfqpoint{3.334763in}{3.206360in}}%
\pgfpathlineto{\pgfqpoint{3.342439in}{3.216554in}}%
\pgfpathlineto{\pgfqpoint{3.350108in}{3.226845in}}%
\pgfpathlineto{\pgfqpoint{3.337190in}{3.242781in}}%
\pgfpathlineto{\pgfqpoint{3.324269in}{3.258901in}}%
\pgfpathlineto{\pgfqpoint{3.311345in}{3.275207in}}%
\pgfpathlineto{\pgfqpoint{3.298417in}{3.291700in}}%
\pgfpathlineto{\pgfqpoint{3.290732in}{3.281400in}}%
\pgfpathlineto{\pgfqpoint{3.283040in}{3.271202in}}%
\pgfpathlineto{\pgfqpoint{3.275343in}{3.261106in}}%
\pgfpathlineto{\pgfqpoint{3.267638in}{3.251111in}}%
\pgfpathclose%
\pgfusepath{fill}%
\end{pgfscope}%
\begin{pgfscope}%
\pgfpathrectangle{\pgfqpoint{1.254980in}{0.150000in}}{\pgfqpoint{5.490039in}{5.490039in}}%
\pgfusepath{clip}%
\pgfsetbuttcap%
\pgfsetroundjoin%
\definecolor{currentfill}{rgb}{0.218130,0.347432,0.550038}%
\pgfsetfillcolor{currentfill}%
\pgfsetfillopacity{0.700000}%
\pgfsetlinewidth{0.000000pt}%
\definecolor{currentstroke}{rgb}{0.000000,0.000000,0.000000}%
\pgfsetstrokecolor{currentstroke}%
\pgfsetdash{}{0pt}%
\pgfpathmoveto{\pgfqpoint{3.474397in}{3.009055in}}%
\pgfpathlineto{\pgfqpoint{3.487302in}{2.995422in}}%
\pgfpathlineto{\pgfqpoint{3.500205in}{2.981958in}}%
\pgfpathlineto{\pgfqpoint{3.513107in}{2.968662in}}%
\pgfpathlineto{\pgfqpoint{3.526008in}{2.955535in}}%
\pgfpathlineto{\pgfqpoint{3.533634in}{2.965546in}}%
\pgfpathlineto{\pgfqpoint{3.541255in}{2.975639in}}%
\pgfpathlineto{\pgfqpoint{3.548870in}{2.985813in}}%
\pgfpathlineto{\pgfqpoint{3.556480in}{2.996068in}}%
\pgfpathlineto{\pgfqpoint{3.543593in}{3.009199in}}%
\pgfpathlineto{\pgfqpoint{3.530705in}{3.022498in}}%
\pgfpathlineto{\pgfqpoint{3.517817in}{3.035965in}}%
\pgfpathlineto{\pgfqpoint{3.504927in}{3.049602in}}%
\pgfpathlineto{\pgfqpoint{3.497303in}{3.039337in}}%
\pgfpathlineto{\pgfqpoint{3.489673in}{3.029158in}}%
\pgfpathlineto{\pgfqpoint{3.482038in}{3.019064in}}%
\pgfpathlineto{\pgfqpoint{3.474397in}{3.009055in}}%
\pgfpathclose%
\pgfusepath{fill}%
\end{pgfscope}%
\begin{pgfscope}%
\pgfpathrectangle{\pgfqpoint{1.254980in}{0.150000in}}{\pgfqpoint{5.490039in}{5.490039in}}%
\pgfusepath{clip}%
\pgfsetbuttcap%
\pgfsetroundjoin%
\definecolor{currentfill}{rgb}{0.255645,0.260703,0.528312}%
\pgfsetfillcolor{currentfill}%
\pgfsetfillopacity{0.700000}%
\pgfsetlinewidth{0.000000pt}%
\definecolor{currentstroke}{rgb}{0.000000,0.000000,0.000000}%
\pgfsetstrokecolor{currentstroke}%
\pgfsetdash{}{0pt}%
\pgfpathmoveto{\pgfqpoint{3.762651in}{2.808063in}}%
\pgfpathlineto{\pgfqpoint{3.775543in}{2.797650in}}%
\pgfpathlineto{\pgfqpoint{3.788436in}{2.787390in}}%
\pgfpathlineto{\pgfqpoint{3.801330in}{2.777282in}}%
\pgfpathlineto{\pgfqpoint{3.814226in}{2.767325in}}%
\pgfpathlineto{\pgfqpoint{3.821763in}{2.777594in}}%
\pgfpathlineto{\pgfqpoint{3.829296in}{2.787925in}}%
\pgfpathlineto{\pgfqpoint{3.836824in}{2.798320in}}%
\pgfpathlineto{\pgfqpoint{3.844347in}{2.808779in}}%
\pgfpathlineto{\pgfqpoint{3.831463in}{2.818756in}}%
\pgfpathlineto{\pgfqpoint{3.818581in}{2.828884in}}%
\pgfpathlineto{\pgfqpoint{3.805701in}{2.839164in}}%
\pgfpathlineto{\pgfqpoint{3.792823in}{2.849597in}}%
\pgfpathlineto{\pgfqpoint{3.785287in}{2.839112in}}%
\pgfpathlineto{\pgfqpoint{3.777747in}{2.828695in}}%
\pgfpathlineto{\pgfqpoint{3.770202in}{2.818346in}}%
\pgfpathlineto{\pgfqpoint{3.762651in}{2.808063in}}%
\pgfpathclose%
\pgfusepath{fill}%
\end{pgfscope}%
\begin{pgfscope}%
\pgfpathrectangle{\pgfqpoint{1.254980in}{0.150000in}}{\pgfqpoint{5.490039in}{5.490039in}}%
\pgfusepath{clip}%
\pgfsetbuttcap%
\pgfsetroundjoin%
\definecolor{currentfill}{rgb}{0.235526,0.309527,0.542944}%
\pgfsetfillcolor{currentfill}%
\pgfsetfillopacity{0.700000}%
\pgfsetlinewidth{0.000000pt}%
\definecolor{currentstroke}{rgb}{0.000000,0.000000,0.000000}%
\pgfsetstrokecolor{currentstroke}%
\pgfsetdash{}{0pt}%
\pgfpathmoveto{\pgfqpoint{5.102018in}{2.896775in}}%
\pgfpathlineto{\pgfqpoint{5.115200in}{2.895120in}}%
\pgfpathlineto{\pgfqpoint{5.128391in}{2.893582in}}%
\pgfpathlineto{\pgfqpoint{5.141592in}{2.892162in}}%
\pgfpathlineto{\pgfqpoint{5.154802in}{2.890858in}}%
\pgfpathlineto{\pgfqpoint{5.161941in}{2.901435in}}%
\pgfpathlineto{\pgfqpoint{5.169078in}{2.912064in}}%
\pgfpathlineto{\pgfqpoint{5.176212in}{2.922746in}}%
\pgfpathlineto{\pgfqpoint{5.183342in}{2.933484in}}%
\pgfpathlineto{\pgfqpoint{5.170145in}{2.934999in}}%
\pgfpathlineto{\pgfqpoint{5.156957in}{2.936630in}}%
\pgfpathlineto{\pgfqpoint{5.143779in}{2.938378in}}%
\pgfpathlineto{\pgfqpoint{5.130609in}{2.940243in}}%
\pgfpathlineto{\pgfqpoint{5.123466in}{2.929289in}}%
\pgfpathlineto{\pgfqpoint{5.116319in}{2.918394in}}%
\pgfpathlineto{\pgfqpoint{5.109170in}{2.907557in}}%
\pgfpathlineto{\pgfqpoint{5.102018in}{2.896775in}}%
\pgfpathclose%
\pgfusepath{fill}%
\end{pgfscope}%
\begin{pgfscope}%
\pgfpathrectangle{\pgfqpoint{1.254980in}{0.150000in}}{\pgfqpoint{5.490039in}{5.490039in}}%
\pgfusepath{clip}%
\pgfsetbuttcap%
\pgfsetroundjoin%
\definecolor{currentfill}{rgb}{0.270595,0.214069,0.507052}%
\pgfsetfillcolor{currentfill}%
\pgfsetfillopacity{0.700000}%
\pgfsetlinewidth{0.000000pt}%
\definecolor{currentstroke}{rgb}{0.000000,0.000000,0.000000}%
\pgfsetstrokecolor{currentstroke}%
\pgfsetdash{}{0pt}%
\pgfpathmoveto{\pgfqpoint{4.080667in}{2.711050in}}%
\pgfpathlineto{\pgfqpoint{4.093586in}{2.703503in}}%
\pgfpathlineto{\pgfqpoint{4.106509in}{2.696097in}}%
\pgfpathlineto{\pgfqpoint{4.119436in}{2.688829in}}%
\pgfpathlineto{\pgfqpoint{4.132367in}{2.681700in}}%
\pgfpathlineto{\pgfqpoint{4.139809in}{2.692256in}}%
\pgfpathlineto{\pgfqpoint{4.147246in}{2.702861in}}%
\pgfpathlineto{\pgfqpoint{4.154680in}{2.713514in}}%
\pgfpathlineto{\pgfqpoint{4.162108in}{2.724217in}}%
\pgfpathlineto{\pgfqpoint{4.149189in}{2.731399in}}%
\pgfpathlineto{\pgfqpoint{4.136273in}{2.738718in}}%
\pgfpathlineto{\pgfqpoint{4.123361in}{2.746177in}}%
\pgfpathlineto{\pgfqpoint{4.110453in}{2.753776in}}%
\pgfpathlineto{\pgfqpoint{4.103013in}{2.743015in}}%
\pgfpathlineto{\pgfqpoint{4.095569in}{2.732307in}}%
\pgfpathlineto{\pgfqpoint{4.088120in}{2.721652in}}%
\pgfpathlineto{\pgfqpoint{4.080667in}{2.711050in}}%
\pgfpathclose%
\pgfusepath{fill}%
\end{pgfscope}%
\begin{pgfscope}%
\pgfpathrectangle{\pgfqpoint{1.254980in}{0.150000in}}{\pgfqpoint{5.490039in}{5.490039in}}%
\pgfusepath{clip}%
\pgfsetbuttcap%
\pgfsetroundjoin%
\definecolor{currentfill}{rgb}{0.163625,0.471133,0.558148}%
\pgfsetfillcolor{currentfill}%
\pgfsetfillopacity{0.700000}%
\pgfsetlinewidth{0.000000pt}%
\definecolor{currentstroke}{rgb}{0.000000,0.000000,0.000000}%
\pgfsetstrokecolor{currentstroke}%
\pgfsetdash{}{0pt}%
\pgfpathmoveto{\pgfqpoint{3.215823in}{3.318968in}}%
\pgfpathlineto{\pgfqpoint{3.228783in}{3.301717in}}%
\pgfpathlineto{\pgfqpoint{3.241739in}{3.284658in}}%
\pgfpathlineto{\pgfqpoint{3.254691in}{3.267790in}}%
\pgfpathlineto{\pgfqpoint{3.267638in}{3.251111in}}%
\pgfpathlineto{\pgfqpoint{3.275343in}{3.261106in}}%
\pgfpathlineto{\pgfqpoint{3.283040in}{3.271202in}}%
\pgfpathlineto{\pgfqpoint{3.290732in}{3.281400in}}%
\pgfpathlineto{\pgfqpoint{3.298417in}{3.291700in}}%
\pgfpathlineto{\pgfqpoint{3.285486in}{3.308381in}}%
\pgfpathlineto{\pgfqpoint{3.272551in}{3.325251in}}%
\pgfpathlineto{\pgfqpoint{3.259611in}{3.342312in}}%
\pgfpathlineto{\pgfqpoint{3.246668in}{3.359564in}}%
\pgfpathlineto{\pgfqpoint{3.238966in}{3.349256in}}%
\pgfpathlineto{\pgfqpoint{3.231258in}{3.339055in}}%
\pgfpathlineto{\pgfqpoint{3.223544in}{3.328958in}}%
\pgfpathlineto{\pgfqpoint{3.215823in}{3.318968in}}%
\pgfpathclose%
\pgfusepath{fill}%
\end{pgfscope}%
\begin{pgfscope}%
\pgfpathrectangle{\pgfqpoint{1.254980in}{0.150000in}}{\pgfqpoint{5.490039in}{5.490039in}}%
\pgfusepath{clip}%
\pgfsetbuttcap%
\pgfsetroundjoin%
\definecolor{currentfill}{rgb}{0.263663,0.237631,0.518762}%
\pgfsetfillcolor{currentfill}%
\pgfsetfillopacity{0.700000}%
\pgfsetlinewidth{0.000000pt}%
\definecolor{currentstroke}{rgb}{0.000000,0.000000,0.000000}%
\pgfsetstrokecolor{currentstroke}%
\pgfsetdash{}{0pt}%
\pgfpathmoveto{\pgfqpoint{4.643080in}{2.745838in}}%
\pgfpathlineto{\pgfqpoint{4.656122in}{2.742093in}}%
\pgfpathlineto{\pgfqpoint{4.669172in}{2.738473in}}%
\pgfpathlineto{\pgfqpoint{4.682228in}{2.734977in}}%
\pgfpathlineto{\pgfqpoint{4.695292in}{2.731605in}}%
\pgfpathlineto{\pgfqpoint{4.702569in}{2.742310in}}%
\pgfpathlineto{\pgfqpoint{4.709843in}{2.753054in}}%
\pgfpathlineto{\pgfqpoint{4.717112in}{2.763838in}}%
\pgfpathlineto{\pgfqpoint{4.724378in}{2.774664in}}%
\pgfpathlineto{\pgfqpoint{4.711325in}{2.778167in}}%
\pgfpathlineto{\pgfqpoint{4.698279in}{2.781794in}}%
\pgfpathlineto{\pgfqpoint{4.685241in}{2.785545in}}%
\pgfpathlineto{\pgfqpoint{4.672209in}{2.789421in}}%
\pgfpathlineto{\pgfqpoint{4.664932in}{2.778458in}}%
\pgfpathlineto{\pgfqpoint{4.657652in}{2.767541in}}%
\pgfpathlineto{\pgfqpoint{4.650368in}{2.756668in}}%
\pgfpathlineto{\pgfqpoint{4.643080in}{2.745838in}}%
\pgfpathclose%
\pgfusepath{fill}%
\end{pgfscope}%
\begin{pgfscope}%
\pgfpathrectangle{\pgfqpoint{1.254980in}{0.150000in}}{\pgfqpoint{5.490039in}{5.490039in}}%
\pgfusepath{clip}%
\pgfsetbuttcap%
\pgfsetroundjoin%
\definecolor{currentfill}{rgb}{0.271828,0.209303,0.504434}%
\pgfsetfillcolor{currentfill}%
\pgfsetfillopacity{0.700000}%
\pgfsetlinewidth{0.000000pt}%
\definecolor{currentstroke}{rgb}{0.000000,0.000000,0.000000}%
\pgfsetstrokecolor{currentstroke}%
\pgfsetdash{}{0pt}%
\pgfpathmoveto{\pgfqpoint{4.213831in}{2.696864in}}%
\pgfpathlineto{\pgfqpoint{4.226773in}{2.690367in}}%
\pgfpathlineto{\pgfqpoint{4.239719in}{2.684005in}}%
\pgfpathlineto{\pgfqpoint{4.252671in}{2.677778in}}%
\pgfpathlineto{\pgfqpoint{4.265627in}{2.671684in}}%
\pgfpathlineto{\pgfqpoint{4.273030in}{2.682314in}}%
\pgfpathlineto{\pgfqpoint{4.280429in}{2.692987in}}%
\pgfpathlineto{\pgfqpoint{4.287824in}{2.703704in}}%
\pgfpathlineto{\pgfqpoint{4.295215in}{2.714467in}}%
\pgfpathlineto{\pgfqpoint{4.282269in}{2.720628in}}%
\pgfpathlineto{\pgfqpoint{4.269328in}{2.726923in}}%
\pgfpathlineto{\pgfqpoint{4.256393in}{2.733354in}}%
\pgfpathlineto{\pgfqpoint{4.243461in}{2.739919in}}%
\pgfpathlineto{\pgfqpoint{4.236060in}{2.729082in}}%
\pgfpathlineto{\pgfqpoint{4.228655in}{2.718295in}}%
\pgfpathlineto{\pgfqpoint{4.221245in}{2.707556in}}%
\pgfpathlineto{\pgfqpoint{4.213831in}{2.696864in}}%
\pgfpathclose%
\pgfusepath{fill}%
\end{pgfscope}%
\begin{pgfscope}%
\pgfpathrectangle{\pgfqpoint{1.254980in}{0.150000in}}{\pgfqpoint{5.490039in}{5.490039in}}%
\pgfusepath{clip}%
\pgfsetbuttcap%
\pgfsetroundjoin%
\definecolor{currentfill}{rgb}{0.267968,0.223549,0.512008}%
\pgfsetfillcolor{currentfill}%
\pgfsetfillopacity{0.700000}%
\pgfsetlinewidth{0.000000pt}%
\definecolor{currentstroke}{rgb}{0.000000,0.000000,0.000000}%
\pgfsetstrokecolor{currentstroke}%
\pgfsetdash{}{0pt}%
\pgfpathmoveto{\pgfqpoint{3.947490in}{2.734319in}}%
\pgfpathlineto{\pgfqpoint{3.960394in}{2.725672in}}%
\pgfpathlineto{\pgfqpoint{3.973301in}{2.717168in}}%
\pgfpathlineto{\pgfqpoint{3.986211in}{2.708808in}}%
\pgfpathlineto{\pgfqpoint{3.999124in}{2.700592in}}%
\pgfpathlineto{\pgfqpoint{4.006607in}{2.711024in}}%
\pgfpathlineto{\pgfqpoint{4.014085in}{2.721509in}}%
\pgfpathlineto{\pgfqpoint{4.021558in}{2.732049in}}%
\pgfpathlineto{\pgfqpoint{4.029027in}{2.742642in}}%
\pgfpathlineto{\pgfqpoint{4.016125in}{2.750895in}}%
\pgfpathlineto{\pgfqpoint{4.003227in}{2.759291in}}%
\pgfpathlineto{\pgfqpoint{3.990331in}{2.767831in}}%
\pgfpathlineto{\pgfqpoint{3.977439in}{2.776515in}}%
\pgfpathlineto{\pgfqpoint{3.969958in}{2.765879in}}%
\pgfpathlineto{\pgfqpoint{3.962474in}{2.755302in}}%
\pgfpathlineto{\pgfqpoint{3.954984in}{2.744782in}}%
\pgfpathlineto{\pgfqpoint{3.947490in}{2.734319in}}%
\pgfpathclose%
\pgfusepath{fill}%
\end{pgfscope}%
\begin{pgfscope}%
\pgfpathrectangle{\pgfqpoint{1.254980in}{0.150000in}}{\pgfqpoint{5.490039in}{5.490039in}}%
\pgfusepath{clip}%
\pgfsetbuttcap%
\pgfsetroundjoin%
\definecolor{currentfill}{rgb}{0.227802,0.326594,0.546532}%
\pgfsetfillcolor{currentfill}%
\pgfsetfillopacity{0.700000}%
\pgfsetlinewidth{0.000000pt}%
\definecolor{currentstroke}{rgb}{0.000000,0.000000,0.000000}%
\pgfsetstrokecolor{currentstroke}%
\pgfsetdash{}{0pt}%
\pgfpathmoveto{\pgfqpoint{3.526008in}{2.955535in}}%
\pgfpathlineto{\pgfqpoint{3.538908in}{2.942574in}}%
\pgfpathlineto{\pgfqpoint{3.551808in}{2.929780in}}%
\pgfpathlineto{\pgfqpoint{3.564707in}{2.917150in}}%
\pgfpathlineto{\pgfqpoint{3.577605in}{2.904685in}}%
\pgfpathlineto{\pgfqpoint{3.585217in}{2.914699in}}%
\pgfpathlineto{\pgfqpoint{3.592823in}{2.924790in}}%
\pgfpathlineto{\pgfqpoint{3.600424in}{2.934958in}}%
\pgfpathlineto{\pgfqpoint{3.608020in}{2.945204in}}%
\pgfpathlineto{\pgfqpoint{3.595135in}{2.957673in}}%
\pgfpathlineto{\pgfqpoint{3.582251in}{2.970306in}}%
\pgfpathlineto{\pgfqpoint{3.569365in}{2.983104in}}%
\pgfpathlineto{\pgfqpoint{3.556480in}{2.996068in}}%
\pgfpathlineto{\pgfqpoint{3.548870in}{2.985813in}}%
\pgfpathlineto{\pgfqpoint{3.541255in}{2.975639in}}%
\pgfpathlineto{\pgfqpoint{3.533634in}{2.965546in}}%
\pgfpathlineto{\pgfqpoint{3.526008in}{2.955535in}}%
\pgfpathclose%
\pgfusepath{fill}%
\end{pgfscope}%
\begin{pgfscope}%
\pgfpathrectangle{\pgfqpoint{1.254980in}{0.150000in}}{\pgfqpoint{5.490039in}{5.490039in}}%
\pgfusepath{clip}%
\pgfsetbuttcap%
\pgfsetroundjoin%
\definecolor{currentfill}{rgb}{0.243113,0.292092,0.538516}%
\pgfsetfillcolor{currentfill}%
\pgfsetfillopacity{0.700000}%
\pgfsetlinewidth{0.000000pt}%
\definecolor{currentstroke}{rgb}{0.000000,0.000000,0.000000}%
\pgfsetstrokecolor{currentstroke}%
\pgfsetdash{}{0pt}%
\pgfpathmoveto{\pgfqpoint{5.020693in}{2.861180in}}%
\pgfpathlineto{\pgfqpoint{5.033851in}{2.859248in}}%
\pgfpathlineto{\pgfqpoint{5.047018in}{2.857434in}}%
\pgfpathlineto{\pgfqpoint{5.060194in}{2.855738in}}%
\pgfpathlineto{\pgfqpoint{5.073379in}{2.854160in}}%
\pgfpathlineto{\pgfqpoint{5.080543in}{2.864741in}}%
\pgfpathlineto{\pgfqpoint{5.087705in}{2.875369in}}%
\pgfpathlineto{\pgfqpoint{5.094863in}{2.886047in}}%
\pgfpathlineto{\pgfqpoint{5.102018in}{2.896775in}}%
\pgfpathlineto{\pgfqpoint{5.088845in}{2.898548in}}%
\pgfpathlineto{\pgfqpoint{5.075681in}{2.900438in}}%
\pgfpathlineto{\pgfqpoint{5.062527in}{2.902446in}}%
\pgfpathlineto{\pgfqpoint{5.049381in}{2.904573in}}%
\pgfpathlineto{\pgfqpoint{5.042213in}{2.893644in}}%
\pgfpathlineto{\pgfqpoint{5.035043in}{2.882771in}}%
\pgfpathlineto{\pgfqpoint{5.027869in}{2.871950in}}%
\pgfpathlineto{\pgfqpoint{5.020693in}{2.861180in}}%
\pgfpathclose%
\pgfusepath{fill}%
\end{pgfscope}%
\begin{pgfscope}%
\pgfpathrectangle{\pgfqpoint{1.254980in}{0.150000in}}{\pgfqpoint{5.490039in}{5.490039in}}%
\pgfusepath{clip}%
\pgfsetbuttcap%
\pgfsetroundjoin%
\definecolor{currentfill}{rgb}{0.153364,0.497000,0.557724}%
\pgfsetfillcolor{currentfill}%
\pgfsetfillopacity{0.700000}%
\pgfsetlinewidth{0.000000pt}%
\definecolor{currentstroke}{rgb}{0.000000,0.000000,0.000000}%
\pgfsetstrokecolor{currentstroke}%
\pgfsetdash{}{0pt}%
\pgfpathmoveto{\pgfqpoint{3.163937in}{3.389914in}}%
\pgfpathlineto{\pgfqpoint{3.176916in}{3.371883in}}%
\pgfpathlineto{\pgfqpoint{3.189889in}{3.354049in}}%
\pgfpathlineto{\pgfqpoint{3.202858in}{3.336411in}}%
\pgfpathlineto{\pgfqpoint{3.215823in}{3.318968in}}%
\pgfpathlineto{\pgfqpoint{3.223544in}{3.328958in}}%
\pgfpathlineto{\pgfqpoint{3.231258in}{3.339055in}}%
\pgfpathlineto{\pgfqpoint{3.238966in}{3.349256in}}%
\pgfpathlineto{\pgfqpoint{3.246668in}{3.359564in}}%
\pgfpathlineto{\pgfqpoint{3.233720in}{3.377010in}}%
\pgfpathlineto{\pgfqpoint{3.220768in}{3.394649in}}%
\pgfpathlineto{\pgfqpoint{3.207812in}{3.412485in}}%
\pgfpathlineto{\pgfqpoint{3.194850in}{3.430517in}}%
\pgfpathlineto{\pgfqpoint{3.187132in}{3.420201in}}%
\pgfpathlineto{\pgfqpoint{3.179407in}{3.409996in}}%
\pgfpathlineto{\pgfqpoint{3.171675in}{3.399900in}}%
\pgfpathlineto{\pgfqpoint{3.163937in}{3.389914in}}%
\pgfpathclose%
\pgfusepath{fill}%
\end{pgfscope}%
\begin{pgfscope}%
\pgfpathrectangle{\pgfqpoint{1.254980in}{0.150000in}}{\pgfqpoint{5.490039in}{5.490039in}}%
\pgfusepath{clip}%
\pgfsetbuttcap%
\pgfsetroundjoin%
\definecolor{currentfill}{rgb}{0.248629,0.278775,0.534556}%
\pgfsetfillcolor{currentfill}%
\pgfsetfillopacity{0.700000}%
\pgfsetlinewidth{0.000000pt}%
\definecolor{currentstroke}{rgb}{0.000000,0.000000,0.000000}%
\pgfsetstrokecolor{currentstroke}%
\pgfsetdash{}{0pt}%
\pgfpathmoveto{\pgfqpoint{4.939362in}{2.826774in}}%
\pgfpathlineto{\pgfqpoint{4.952497in}{2.824544in}}%
\pgfpathlineto{\pgfqpoint{4.965640in}{2.822434in}}%
\pgfpathlineto{\pgfqpoint{4.978793in}{2.820443in}}%
\pgfpathlineto{\pgfqpoint{4.991954in}{2.818571in}}%
\pgfpathlineto{\pgfqpoint{4.999143in}{2.829156in}}%
\pgfpathlineto{\pgfqpoint{5.006330in}{2.839785in}}%
\pgfpathlineto{\pgfqpoint{5.013513in}{2.850459in}}%
\pgfpathlineto{\pgfqpoint{5.020693in}{2.861180in}}%
\pgfpathlineto{\pgfqpoint{5.007544in}{2.863231in}}%
\pgfpathlineto{\pgfqpoint{4.994403in}{2.865401in}}%
\pgfpathlineto{\pgfqpoint{4.981271in}{2.867690in}}%
\pgfpathlineto{\pgfqpoint{4.968148in}{2.870098in}}%
\pgfpathlineto{\pgfqpoint{4.960956in}{2.859193in}}%
\pgfpathlineto{\pgfqpoint{4.953762in}{2.848339in}}%
\pgfpathlineto{\pgfqpoint{4.946563in}{2.837533in}}%
\pgfpathlineto{\pgfqpoint{4.939362in}{2.826774in}}%
\pgfpathclose%
\pgfusepath{fill}%
\end{pgfscope}%
\begin{pgfscope}%
\pgfpathrectangle{\pgfqpoint{1.254980in}{0.150000in}}{\pgfqpoint{5.490039in}{5.490039in}}%
\pgfusepath{clip}%
\pgfsetbuttcap%
\pgfsetroundjoin%
\definecolor{currentfill}{rgb}{0.237441,0.305202,0.541921}%
\pgfsetfillcolor{currentfill}%
\pgfsetfillopacity{0.700000}%
\pgfsetlinewidth{0.000000pt}%
\definecolor{currentstroke}{rgb}{0.000000,0.000000,0.000000}%
\pgfsetstrokecolor{currentstroke}%
\pgfsetdash{}{0pt}%
\pgfpathmoveto{\pgfqpoint{3.577605in}{2.904685in}}%
\pgfpathlineto{\pgfqpoint{3.590503in}{2.892383in}}%
\pgfpathlineto{\pgfqpoint{3.603401in}{2.880243in}}%
\pgfpathlineto{\pgfqpoint{3.616299in}{2.868265in}}%
\pgfpathlineto{\pgfqpoint{3.629198in}{2.856447in}}%
\pgfpathlineto{\pgfqpoint{3.636795in}{2.866463in}}%
\pgfpathlineto{\pgfqpoint{3.644388in}{2.876552in}}%
\pgfpathlineto{\pgfqpoint{3.651975in}{2.886715in}}%
\pgfpathlineto{\pgfqpoint{3.659556in}{2.896951in}}%
\pgfpathlineto{\pgfqpoint{3.646672in}{2.908773in}}%
\pgfpathlineto{\pgfqpoint{3.633788in}{2.920755in}}%
\pgfpathlineto{\pgfqpoint{3.620904in}{2.932898in}}%
\pgfpathlineto{\pgfqpoint{3.608020in}{2.945204in}}%
\pgfpathlineto{\pgfqpoint{3.600424in}{2.934958in}}%
\pgfpathlineto{\pgfqpoint{3.592823in}{2.924790in}}%
\pgfpathlineto{\pgfqpoint{3.585217in}{2.914699in}}%
\pgfpathlineto{\pgfqpoint{3.577605in}{2.904685in}}%
\pgfpathclose%
\pgfusepath{fill}%
\end{pgfscope}%
\begin{pgfscope}%
\pgfpathrectangle{\pgfqpoint{1.254980in}{0.150000in}}{\pgfqpoint{5.490039in}{5.490039in}}%
\pgfusepath{clip}%
\pgfsetbuttcap%
\pgfsetroundjoin%
\definecolor{currentfill}{rgb}{0.271828,0.209303,0.504434}%
\pgfsetfillcolor{currentfill}%
\pgfsetfillopacity{0.700000}%
\pgfsetlinewidth{0.000000pt}%
\definecolor{currentstroke}{rgb}{0.000000,0.000000,0.000000}%
\pgfsetstrokecolor{currentstroke}%
\pgfsetdash{}{0pt}%
\pgfpathmoveto{\pgfqpoint{4.347049in}{2.691155in}}%
\pgfpathlineto{\pgfqpoint{4.360021in}{2.685657in}}%
\pgfpathlineto{\pgfqpoint{4.372999in}{2.680291in}}%
\pgfpathlineto{\pgfqpoint{4.385982in}{2.675057in}}%
\pgfpathlineto{\pgfqpoint{4.398971in}{2.669952in}}%
\pgfpathlineto{\pgfqpoint{4.406337in}{2.680605in}}%
\pgfpathlineto{\pgfqpoint{4.413699in}{2.691298in}}%
\pgfpathlineto{\pgfqpoint{4.421056in}{2.702032in}}%
\pgfpathlineto{\pgfqpoint{4.428410in}{2.712807in}}%
\pgfpathlineto{\pgfqpoint{4.415431in}{2.717995in}}%
\pgfpathlineto{\pgfqpoint{4.402458in}{2.723314in}}%
\pgfpathlineto{\pgfqpoint{4.389491in}{2.728764in}}%
\pgfpathlineto{\pgfqpoint{4.376530in}{2.734345in}}%
\pgfpathlineto{\pgfqpoint{4.369166in}{2.723480in}}%
\pgfpathlineto{\pgfqpoint{4.361798in}{2.712661in}}%
\pgfpathlineto{\pgfqpoint{4.354425in}{2.701886in}}%
\pgfpathlineto{\pgfqpoint{4.347049in}{2.691155in}}%
\pgfpathclose%
\pgfusepath{fill}%
\end{pgfscope}%
\begin{pgfscope}%
\pgfpathrectangle{\pgfqpoint{1.254980in}{0.150000in}}{\pgfqpoint{5.490039in}{5.490039in}}%
\pgfusepath{clip}%
\pgfsetbuttcap%
\pgfsetroundjoin%
\definecolor{currentfill}{rgb}{0.267968,0.223549,0.512008}%
\pgfsetfillcolor{currentfill}%
\pgfsetfillopacity{0.700000}%
\pgfsetlinewidth{0.000000pt}%
\definecolor{currentstroke}{rgb}{0.000000,0.000000,0.000000}%
\pgfsetstrokecolor{currentstroke}%
\pgfsetdash{}{0pt}%
\pgfpathmoveto{\pgfqpoint{4.561751in}{2.718691in}}%
\pgfpathlineto{\pgfqpoint{4.574776in}{2.714558in}}%
\pgfpathlineto{\pgfqpoint{4.587808in}{2.710552in}}%
\pgfpathlineto{\pgfqpoint{4.600846in}{2.706672in}}%
\pgfpathlineto{\pgfqpoint{4.613892in}{2.702917in}}%
\pgfpathlineto{\pgfqpoint{4.621195in}{2.713590in}}%
\pgfpathlineto{\pgfqpoint{4.628494in}{2.724300in}}%
\pgfpathlineto{\pgfqpoint{4.635789in}{2.735049in}}%
\pgfpathlineto{\pgfqpoint{4.643080in}{2.745838in}}%
\pgfpathlineto{\pgfqpoint{4.630045in}{2.749708in}}%
\pgfpathlineto{\pgfqpoint{4.617017in}{2.753704in}}%
\pgfpathlineto{\pgfqpoint{4.603996in}{2.757826in}}%
\pgfpathlineto{\pgfqpoint{4.590982in}{2.762074in}}%
\pgfpathlineto{\pgfqpoint{4.583680in}{2.751164in}}%
\pgfpathlineto{\pgfqpoint{4.576374in}{2.740298in}}%
\pgfpathlineto{\pgfqpoint{4.569065in}{2.729474in}}%
\pgfpathlineto{\pgfqpoint{4.561751in}{2.718691in}}%
\pgfpathclose%
\pgfusepath{fill}%
\end{pgfscope}%
\begin{pgfscope}%
\pgfpathrectangle{\pgfqpoint{1.254980in}{0.150000in}}{\pgfqpoint{5.490039in}{5.490039in}}%
\pgfusepath{clip}%
\pgfsetbuttcap%
\pgfsetroundjoin%
\definecolor{currentfill}{rgb}{0.262138,0.242286,0.520837}%
\pgfsetfillcolor{currentfill}%
\pgfsetfillopacity{0.700000}%
\pgfsetlinewidth{0.000000pt}%
\definecolor{currentstroke}{rgb}{0.000000,0.000000,0.000000}%
\pgfsetstrokecolor{currentstroke}%
\pgfsetdash{}{0pt}%
\pgfpathmoveto{\pgfqpoint{3.814226in}{2.767325in}}%
\pgfpathlineto{\pgfqpoint{3.827124in}{2.757519in}}%
\pgfpathlineto{\pgfqpoint{3.840023in}{2.747862in}}%
\pgfpathlineto{\pgfqpoint{3.852925in}{2.738355in}}%
\pgfpathlineto{\pgfqpoint{3.865828in}{2.728996in}}%
\pgfpathlineto{\pgfqpoint{3.873354in}{2.739250in}}%
\pgfpathlineto{\pgfqpoint{3.880874in}{2.749563in}}%
\pgfpathlineto{\pgfqpoint{3.888389in}{2.759935in}}%
\pgfpathlineto{\pgfqpoint{3.895900in}{2.770368in}}%
\pgfpathlineto{\pgfqpoint{3.883008in}{2.779747in}}%
\pgfpathlineto{\pgfqpoint{3.870119in}{2.789275in}}%
\pgfpathlineto{\pgfqpoint{3.857232in}{2.798952in}}%
\pgfpathlineto{\pgfqpoint{3.844347in}{2.808779in}}%
\pgfpathlineto{\pgfqpoint{3.836824in}{2.798320in}}%
\pgfpathlineto{\pgfqpoint{3.829296in}{2.787925in}}%
\pgfpathlineto{\pgfqpoint{3.821763in}{2.777594in}}%
\pgfpathlineto{\pgfqpoint{3.814226in}{2.767325in}}%
\pgfpathclose%
\pgfusepath{fill}%
\end{pgfscope}%
\begin{pgfscope}%
\pgfpathrectangle{\pgfqpoint{1.254980in}{0.150000in}}{\pgfqpoint{5.490039in}{5.490039in}}%
\pgfusepath{clip}%
\pgfsetbuttcap%
\pgfsetroundjoin%
\definecolor{currentfill}{rgb}{0.141935,0.526453,0.555991}%
\pgfsetfillcolor{currentfill}%
\pgfsetfillopacity{0.700000}%
\pgfsetlinewidth{0.000000pt}%
\definecolor{currentstroke}{rgb}{0.000000,0.000000,0.000000}%
\pgfsetstrokecolor{currentstroke}%
\pgfsetdash{}{0pt}%
\pgfpathmoveto{\pgfqpoint{3.111970in}{3.464033in}}%
\pgfpathlineto{\pgfqpoint{3.124970in}{3.445201in}}%
\pgfpathlineto{\pgfqpoint{3.137964in}{3.426571in}}%
\pgfpathlineto{\pgfqpoint{3.150953in}{3.408143in}}%
\pgfpathlineto{\pgfqpoint{3.163937in}{3.389914in}}%
\pgfpathlineto{\pgfqpoint{3.171675in}{3.399900in}}%
\pgfpathlineto{\pgfqpoint{3.179407in}{3.409996in}}%
\pgfpathlineto{\pgfqpoint{3.187132in}{3.420201in}}%
\pgfpathlineto{\pgfqpoint{3.194850in}{3.430517in}}%
\pgfpathlineto{\pgfqpoint{3.181884in}{3.448747in}}%
\pgfpathlineto{\pgfqpoint{3.168912in}{3.467177in}}%
\pgfpathlineto{\pgfqpoint{3.155936in}{3.485808in}}%
\pgfpathlineto{\pgfqpoint{3.142954in}{3.504642in}}%
\pgfpathlineto{\pgfqpoint{3.135218in}{3.494318in}}%
\pgfpathlineto{\pgfqpoint{3.127476in}{3.484109in}}%
\pgfpathlineto{\pgfqpoint{3.119726in}{3.474014in}}%
\pgfpathlineto{\pgfqpoint{3.111970in}{3.464033in}}%
\pgfpathclose%
\pgfusepath{fill}%
\end{pgfscope}%
\begin{pgfscope}%
\pgfpathrectangle{\pgfqpoint{1.254980in}{0.150000in}}{\pgfqpoint{5.490039in}{5.490039in}}%
\pgfusepath{clip}%
\pgfsetbuttcap%
\pgfsetroundjoin%
\definecolor{currentfill}{rgb}{0.253935,0.265254,0.529983}%
\pgfsetfillcolor{currentfill}%
\pgfsetfillopacity{0.700000}%
\pgfsetlinewidth{0.000000pt}%
\definecolor{currentstroke}{rgb}{0.000000,0.000000,0.000000}%
\pgfsetstrokecolor{currentstroke}%
\pgfsetdash{}{0pt}%
\pgfpathmoveto{\pgfqpoint{4.858022in}{2.793645in}}%
\pgfpathlineto{\pgfqpoint{4.871134in}{2.791096in}}%
\pgfpathlineto{\pgfqpoint{4.884255in}{2.788669in}}%
\pgfpathlineto{\pgfqpoint{4.897385in}{2.786362in}}%
\pgfpathlineto{\pgfqpoint{4.910522in}{2.784175in}}%
\pgfpathlineto{\pgfqpoint{4.917737in}{2.794763in}}%
\pgfpathlineto{\pgfqpoint{4.924949in}{2.805391in}}%
\pgfpathlineto{\pgfqpoint{4.932157in}{2.816061in}}%
\pgfpathlineto{\pgfqpoint{4.939362in}{2.826774in}}%
\pgfpathlineto{\pgfqpoint{4.926236in}{2.829124in}}%
\pgfpathlineto{\pgfqpoint{4.913118in}{2.831594in}}%
\pgfpathlineto{\pgfqpoint{4.900008in}{2.834185in}}%
\pgfpathlineto{\pgfqpoint{4.886907in}{2.836896in}}%
\pgfpathlineto{\pgfqpoint{4.879691in}{2.826014in}}%
\pgfpathlineto{\pgfqpoint{4.872471in}{2.815179in}}%
\pgfpathlineto{\pgfqpoint{4.865248in}{2.804390in}}%
\pgfpathlineto{\pgfqpoint{4.858022in}{2.793645in}}%
\pgfpathclose%
\pgfusepath{fill}%
\end{pgfscope}%
\begin{pgfscope}%
\pgfpathrectangle{\pgfqpoint{1.254980in}{0.150000in}}{\pgfqpoint{5.490039in}{5.490039in}}%
\pgfusepath{clip}%
\pgfsetbuttcap%
\pgfsetroundjoin%
\definecolor{currentfill}{rgb}{0.246811,0.283237,0.535941}%
\pgfsetfillcolor{currentfill}%
\pgfsetfillopacity{0.700000}%
\pgfsetlinewidth{0.000000pt}%
\definecolor{currentstroke}{rgb}{0.000000,0.000000,0.000000}%
\pgfsetstrokecolor{currentstroke}%
\pgfsetdash{}{0pt}%
\pgfpathmoveto{\pgfqpoint{3.629198in}{2.856447in}}%
\pgfpathlineto{\pgfqpoint{3.642096in}{2.844789in}}%
\pgfpathlineto{\pgfqpoint{3.654994in}{2.833291in}}%
\pgfpathlineto{\pgfqpoint{3.667893in}{2.821950in}}%
\pgfpathlineto{\pgfqpoint{3.680793in}{2.810767in}}%
\pgfpathlineto{\pgfqpoint{3.688377in}{2.820785in}}%
\pgfpathlineto{\pgfqpoint{3.695956in}{2.830872in}}%
\pgfpathlineto{\pgfqpoint{3.703529in}{2.841028in}}%
\pgfpathlineto{\pgfqpoint{3.711098in}{2.851255in}}%
\pgfpathlineto{\pgfqpoint{3.698211in}{2.862442in}}%
\pgfpathlineto{\pgfqpoint{3.685326in}{2.873787in}}%
\pgfpathlineto{\pgfqpoint{3.672441in}{2.885290in}}%
\pgfpathlineto{\pgfqpoint{3.659556in}{2.896951in}}%
\pgfpathlineto{\pgfqpoint{3.651975in}{2.886715in}}%
\pgfpathlineto{\pgfqpoint{3.644388in}{2.876552in}}%
\pgfpathlineto{\pgfqpoint{3.636795in}{2.866463in}}%
\pgfpathlineto{\pgfqpoint{3.629198in}{2.856447in}}%
\pgfpathclose%
\pgfusepath{fill}%
\end{pgfscope}%
\begin{pgfscope}%
\pgfpathrectangle{\pgfqpoint{1.254980in}{0.150000in}}{\pgfqpoint{5.490039in}{5.490039in}}%
\pgfusepath{clip}%
\pgfsetbuttcap%
\pgfsetroundjoin%
\definecolor{currentfill}{rgb}{0.273006,0.204520,0.501721}%
\pgfsetfillcolor{currentfill}%
\pgfsetfillopacity{0.700000}%
\pgfsetlinewidth{0.000000pt}%
\definecolor{currentstroke}{rgb}{0.000000,0.000000,0.000000}%
\pgfsetstrokecolor{currentstroke}%
\pgfsetdash{}{0pt}%
\pgfpathmoveto{\pgfqpoint{4.132367in}{2.681700in}}%
\pgfpathlineto{\pgfqpoint{4.145302in}{2.674708in}}%
\pgfpathlineto{\pgfqpoint{4.158241in}{2.667855in}}%
\pgfpathlineto{\pgfqpoint{4.171184in}{2.661137in}}%
\pgfpathlineto{\pgfqpoint{4.184132in}{2.654557in}}%
\pgfpathlineto{\pgfqpoint{4.191563in}{2.665066in}}%
\pgfpathlineto{\pgfqpoint{4.198990in}{2.675620in}}%
\pgfpathlineto{\pgfqpoint{4.206413in}{2.686220in}}%
\pgfpathlineto{\pgfqpoint{4.213831in}{2.696864in}}%
\pgfpathlineto{\pgfqpoint{4.200894in}{2.703498in}}%
\pgfpathlineto{\pgfqpoint{4.187961in}{2.710267in}}%
\pgfpathlineto{\pgfqpoint{4.175033in}{2.717173in}}%
\pgfpathlineto{\pgfqpoint{4.162108in}{2.724217in}}%
\pgfpathlineto{\pgfqpoint{4.154680in}{2.713514in}}%
\pgfpathlineto{\pgfqpoint{4.147246in}{2.702861in}}%
\pgfpathlineto{\pgfqpoint{4.139809in}{2.692256in}}%
\pgfpathlineto{\pgfqpoint{4.132367in}{2.681700in}}%
\pgfpathclose%
\pgfusepath{fill}%
\end{pgfscope}%
\begin{pgfscope}%
\pgfpathrectangle{\pgfqpoint{1.254980in}{0.150000in}}{\pgfqpoint{5.490039in}{5.490039in}}%
\pgfusepath{clip}%
\pgfsetbuttcap%
\pgfsetroundjoin%
\definecolor{currentfill}{rgb}{0.270595,0.214069,0.507052}%
\pgfsetfillcolor{currentfill}%
\pgfsetfillopacity{0.700000}%
\pgfsetlinewidth{0.000000pt}%
\definecolor{currentstroke}{rgb}{0.000000,0.000000,0.000000}%
\pgfsetstrokecolor{currentstroke}%
\pgfsetdash{}{0pt}%
\pgfpathmoveto{\pgfqpoint{3.999124in}{2.700592in}}%
\pgfpathlineto{\pgfqpoint{4.012041in}{2.692518in}}%
\pgfpathlineto{\pgfqpoint{4.024960in}{2.684586in}}%
\pgfpathlineto{\pgfqpoint{4.037884in}{2.676795in}}%
\pgfpathlineto{\pgfqpoint{4.050810in}{2.669145in}}%
\pgfpathlineto{\pgfqpoint{4.058281in}{2.679547in}}%
\pgfpathlineto{\pgfqpoint{4.065748in}{2.689998in}}%
\pgfpathlineto{\pgfqpoint{4.073210in}{2.700498in}}%
\pgfpathlineto{\pgfqpoint{4.080667in}{2.711050in}}%
\pgfpathlineto{\pgfqpoint{4.067752in}{2.718736in}}%
\pgfpathlineto{\pgfqpoint{4.054840in}{2.726563in}}%
\pgfpathlineto{\pgfqpoint{4.041932in}{2.734532in}}%
\pgfpathlineto{\pgfqpoint{4.029027in}{2.742642in}}%
\pgfpathlineto{\pgfqpoint{4.021558in}{2.732049in}}%
\pgfpathlineto{\pgfqpoint{4.014085in}{2.721509in}}%
\pgfpathlineto{\pgfqpoint{4.006607in}{2.711024in}}%
\pgfpathlineto{\pgfqpoint{3.999124in}{2.700592in}}%
\pgfpathclose%
\pgfusepath{fill}%
\end{pgfscope}%
\begin{pgfscope}%
\pgfpathrectangle{\pgfqpoint{1.254980in}{0.150000in}}{\pgfqpoint{5.490039in}{5.490039in}}%
\pgfusepath{clip}%
\pgfsetbuttcap%
\pgfsetroundjoin%
\definecolor{currentfill}{rgb}{0.270595,0.214069,0.507052}%
\pgfsetfillcolor{currentfill}%
\pgfsetfillopacity{0.700000}%
\pgfsetlinewidth{0.000000pt}%
\definecolor{currentstroke}{rgb}{0.000000,0.000000,0.000000}%
\pgfsetstrokecolor{currentstroke}%
\pgfsetdash{}{0pt}%
\pgfpathmoveto{\pgfqpoint{4.480384in}{2.693350in}}%
\pgfpathlineto{\pgfqpoint{4.493393in}{2.688808in}}%
\pgfpathlineto{\pgfqpoint{4.506409in}{2.684394in}}%
\pgfpathlineto{\pgfqpoint{4.519431in}{2.680107in}}%
\pgfpathlineto{\pgfqpoint{4.532459in}{2.675947in}}%
\pgfpathlineto{\pgfqpoint{4.539788in}{2.686578in}}%
\pgfpathlineto{\pgfqpoint{4.547113in}{2.697244in}}%
\pgfpathlineto{\pgfqpoint{4.554434in}{2.707948in}}%
\pgfpathlineto{\pgfqpoint{4.561751in}{2.718691in}}%
\pgfpathlineto{\pgfqpoint{4.548733in}{2.722951in}}%
\pgfpathlineto{\pgfqpoint{4.535722in}{2.727337in}}%
\pgfpathlineto{\pgfqpoint{4.522717in}{2.731851in}}%
\pgfpathlineto{\pgfqpoint{4.509718in}{2.736493in}}%
\pgfpathlineto{\pgfqpoint{4.502390in}{2.725645in}}%
\pgfpathlineto{\pgfqpoint{4.495059in}{2.714839in}}%
\pgfpathlineto{\pgfqpoint{4.487723in}{2.704075in}}%
\pgfpathlineto{\pgfqpoint{4.480384in}{2.693350in}}%
\pgfpathclose%
\pgfusepath{fill}%
\end{pgfscope}%
\begin{pgfscope}%
\pgfpathrectangle{\pgfqpoint{1.254980in}{0.150000in}}{\pgfqpoint{5.490039in}{5.490039in}}%
\pgfusepath{clip}%
\pgfsetbuttcap%
\pgfsetroundjoin%
\definecolor{currentfill}{rgb}{0.258965,0.251537,0.524736}%
\pgfsetfillcolor{currentfill}%
\pgfsetfillopacity{0.700000}%
\pgfsetlinewidth{0.000000pt}%
\definecolor{currentstroke}{rgb}{0.000000,0.000000,0.000000}%
\pgfsetstrokecolor{currentstroke}%
\pgfsetdash{}{0pt}%
\pgfpathmoveto{\pgfqpoint{4.776667in}{2.761887in}}%
\pgfpathlineto{\pgfqpoint{4.789758in}{2.758999in}}%
\pgfpathlineto{\pgfqpoint{4.802858in}{2.756234in}}%
\pgfpathlineto{\pgfqpoint{4.815965in}{2.753590in}}%
\pgfpathlineto{\pgfqpoint{4.829080in}{2.751068in}}%
\pgfpathlineto{\pgfqpoint{4.836321in}{2.761655in}}%
\pgfpathlineto{\pgfqpoint{4.843558in}{2.772279in}}%
\pgfpathlineto{\pgfqpoint{4.850792in}{2.782942in}}%
\pgfpathlineto{\pgfqpoint{4.858022in}{2.793645in}}%
\pgfpathlineto{\pgfqpoint{4.844917in}{2.796314in}}%
\pgfpathlineto{\pgfqpoint{4.831821in}{2.799105in}}%
\pgfpathlineto{\pgfqpoint{4.818732in}{2.802018in}}%
\pgfpathlineto{\pgfqpoint{4.805652in}{2.805052in}}%
\pgfpathlineto{\pgfqpoint{4.798411in}{2.794196in}}%
\pgfpathlineto{\pgfqpoint{4.791167in}{2.783384in}}%
\pgfpathlineto{\pgfqpoint{4.783918in}{2.772615in}}%
\pgfpathlineto{\pgfqpoint{4.776667in}{2.761887in}}%
\pgfpathclose%
\pgfusepath{fill}%
\end{pgfscope}%
\begin{pgfscope}%
\pgfpathrectangle{\pgfqpoint{1.254980in}{0.150000in}}{\pgfqpoint{5.490039in}{5.490039in}}%
\pgfusepath{clip}%
\pgfsetbuttcap%
\pgfsetroundjoin%
\definecolor{currentfill}{rgb}{0.273006,0.204520,0.501721}%
\pgfsetfillcolor{currentfill}%
\pgfsetfillopacity{0.700000}%
\pgfsetlinewidth{0.000000pt}%
\definecolor{currentstroke}{rgb}{0.000000,0.000000,0.000000}%
\pgfsetstrokecolor{currentstroke}%
\pgfsetdash{}{0pt}%
\pgfpathmoveto{\pgfqpoint{4.265627in}{2.671684in}}%
\pgfpathlineto{\pgfqpoint{4.278588in}{2.665725in}}%
\pgfpathlineto{\pgfqpoint{4.291555in}{2.659899in}}%
\pgfpathlineto{\pgfqpoint{4.304526in}{2.654206in}}%
\pgfpathlineto{\pgfqpoint{4.317503in}{2.648645in}}%
\pgfpathlineto{\pgfqpoint{4.324896in}{2.659212in}}%
\pgfpathlineto{\pgfqpoint{4.332285in}{2.669819in}}%
\pgfpathlineto{\pgfqpoint{4.339669in}{2.680466in}}%
\pgfpathlineto{\pgfqpoint{4.347049in}{2.691155in}}%
\pgfpathlineto{\pgfqpoint{4.334083in}{2.696784in}}%
\pgfpathlineto{\pgfqpoint{4.321122in}{2.702545in}}%
\pgfpathlineto{\pgfqpoint{4.308166in}{2.708439in}}%
\pgfpathlineto{\pgfqpoint{4.295215in}{2.714467in}}%
\pgfpathlineto{\pgfqpoint{4.287824in}{2.703704in}}%
\pgfpathlineto{\pgfqpoint{4.280429in}{2.692987in}}%
\pgfpathlineto{\pgfqpoint{4.273030in}{2.682314in}}%
\pgfpathlineto{\pgfqpoint{4.265627in}{2.671684in}}%
\pgfpathclose%
\pgfusepath{fill}%
\end{pgfscope}%
\begin{pgfscope}%
\pgfpathrectangle{\pgfqpoint{1.254980in}{0.150000in}}{\pgfqpoint{5.490039in}{5.490039in}}%
\pgfusepath{clip}%
\pgfsetbuttcap%
\pgfsetroundjoin%
\definecolor{currentfill}{rgb}{0.129933,0.559582,0.551864}%
\pgfsetfillcolor{currentfill}%
\pgfsetfillopacity{0.700000}%
\pgfsetlinewidth{0.000000pt}%
\definecolor{currentstroke}{rgb}{0.000000,0.000000,0.000000}%
\pgfsetstrokecolor{currentstroke}%
\pgfsetdash{}{0pt}%
\pgfpathmoveto{\pgfqpoint{3.059912in}{3.541413in}}%
\pgfpathlineto{\pgfqpoint{3.072936in}{3.521757in}}%
\pgfpathlineto{\pgfqpoint{3.085953in}{3.502310in}}%
\pgfpathlineto{\pgfqpoint{3.098965in}{3.483069in}}%
\pgfpathlineto{\pgfqpoint{3.111970in}{3.464033in}}%
\pgfpathlineto{\pgfqpoint{3.119726in}{3.474014in}}%
\pgfpathlineto{\pgfqpoint{3.127476in}{3.484109in}}%
\pgfpathlineto{\pgfqpoint{3.135218in}{3.494318in}}%
\pgfpathlineto{\pgfqpoint{3.142954in}{3.504642in}}%
\pgfpathlineto{\pgfqpoint{3.129966in}{3.523679in}}%
\pgfpathlineto{\pgfqpoint{3.116972in}{3.542921in}}%
\pgfpathlineto{\pgfqpoint{3.103973in}{3.562370in}}%
\pgfpathlineto{\pgfqpoint{3.090967in}{3.582026in}}%
\pgfpathlineto{\pgfqpoint{3.083214in}{3.571696in}}%
\pgfpathlineto{\pgfqpoint{3.075454in}{3.561483in}}%
\pgfpathlineto{\pgfqpoint{3.067686in}{3.551389in}}%
\pgfpathlineto{\pgfqpoint{3.059912in}{3.541413in}}%
\pgfpathclose%
\pgfusepath{fill}%
\end{pgfscope}%
\begin{pgfscope}%
\pgfpathrectangle{\pgfqpoint{1.254980in}{0.150000in}}{\pgfqpoint{5.490039in}{5.490039in}}%
\pgfusepath{clip}%
\pgfsetbuttcap%
\pgfsetroundjoin%
\definecolor{currentfill}{rgb}{0.266580,0.228262,0.514349}%
\pgfsetfillcolor{currentfill}%
\pgfsetfillopacity{0.700000}%
\pgfsetlinewidth{0.000000pt}%
\definecolor{currentstroke}{rgb}{0.000000,0.000000,0.000000}%
\pgfsetstrokecolor{currentstroke}%
\pgfsetdash{}{0pt}%
\pgfpathmoveto{\pgfqpoint{3.865828in}{2.728996in}}%
\pgfpathlineto{\pgfqpoint{3.878734in}{2.719784in}}%
\pgfpathlineto{\pgfqpoint{3.891643in}{2.710720in}}%
\pgfpathlineto{\pgfqpoint{3.904553in}{2.701802in}}%
\pgfpathlineto{\pgfqpoint{3.917467in}{2.693030in}}%
\pgfpathlineto{\pgfqpoint{3.924980in}{2.703269in}}%
\pgfpathlineto{\pgfqpoint{3.932488in}{2.713563in}}%
\pgfpathlineto{\pgfqpoint{3.939991in}{2.723913in}}%
\pgfpathlineto{\pgfqpoint{3.947490in}{2.734319in}}%
\pgfpathlineto{\pgfqpoint{3.934589in}{2.743112in}}%
\pgfpathlineto{\pgfqpoint{3.921690in}{2.752051in}}%
\pgfpathlineto{\pgfqpoint{3.908794in}{2.761136in}}%
\pgfpathlineto{\pgfqpoint{3.895900in}{2.770368in}}%
\pgfpathlineto{\pgfqpoint{3.888389in}{2.759935in}}%
\pgfpathlineto{\pgfqpoint{3.880874in}{2.749563in}}%
\pgfpathlineto{\pgfqpoint{3.873354in}{2.739250in}}%
\pgfpathlineto{\pgfqpoint{3.865828in}{2.728996in}}%
\pgfpathclose%
\pgfusepath{fill}%
\end{pgfscope}%
\begin{pgfscope}%
\pgfpathrectangle{\pgfqpoint{1.254980in}{0.150000in}}{\pgfqpoint{5.490039in}{5.490039in}}%
\pgfusepath{clip}%
\pgfsetbuttcap%
\pgfsetroundjoin%
\definecolor{currentfill}{rgb}{0.253935,0.265254,0.529983}%
\pgfsetfillcolor{currentfill}%
\pgfsetfillopacity{0.700000}%
\pgfsetlinewidth{0.000000pt}%
\definecolor{currentstroke}{rgb}{0.000000,0.000000,0.000000}%
\pgfsetstrokecolor{currentstroke}%
\pgfsetdash{}{0pt}%
\pgfpathmoveto{\pgfqpoint{3.680793in}{2.810767in}}%
\pgfpathlineto{\pgfqpoint{3.693693in}{2.799740in}}%
\pgfpathlineto{\pgfqpoint{3.706594in}{2.788869in}}%
\pgfpathlineto{\pgfqpoint{3.719497in}{2.778153in}}%
\pgfpathlineto{\pgfqpoint{3.732400in}{2.767591in}}%
\pgfpathlineto{\pgfqpoint{3.739970in}{2.777611in}}%
\pgfpathlineto{\pgfqpoint{3.747536in}{2.787696in}}%
\pgfpathlineto{\pgfqpoint{3.755096in}{2.797846in}}%
\pgfpathlineto{\pgfqpoint{3.762651in}{2.808063in}}%
\pgfpathlineto{\pgfqpoint{3.749761in}{2.818629in}}%
\pgfpathlineto{\pgfqpoint{3.736872in}{2.829349in}}%
\pgfpathlineto{\pgfqpoint{3.723985in}{2.840224in}}%
\pgfpathlineto{\pgfqpoint{3.711098in}{2.851255in}}%
\pgfpathlineto{\pgfqpoint{3.703529in}{2.841028in}}%
\pgfpathlineto{\pgfqpoint{3.695956in}{2.830872in}}%
\pgfpathlineto{\pgfqpoint{3.688377in}{2.820785in}}%
\pgfpathlineto{\pgfqpoint{3.680793in}{2.810767in}}%
\pgfpathclose%
\pgfusepath{fill}%
\end{pgfscope}%
\begin{pgfscope}%
\pgfpathrectangle{\pgfqpoint{1.254980in}{0.150000in}}{\pgfqpoint{5.490039in}{5.490039in}}%
\pgfusepath{clip}%
\pgfsetbuttcap%
\pgfsetroundjoin%
\definecolor{currentfill}{rgb}{0.225863,0.330805,0.547314}%
\pgfsetfillcolor{currentfill}%
\pgfsetfillopacity{0.700000}%
\pgfsetlinewidth{0.000000pt}%
\definecolor{currentstroke}{rgb}{0.000000,0.000000,0.000000}%
\pgfsetstrokecolor{currentstroke}%
\pgfsetdash{}{0pt}%
\pgfpathmoveto{\pgfqpoint{5.236226in}{2.928589in}}%
\pgfpathlineto{\pgfqpoint{5.249472in}{2.927654in}}%
\pgfpathlineto{\pgfqpoint{5.262727in}{2.926835in}}%
\pgfpathlineto{\pgfqpoint{5.275992in}{2.926131in}}%
\pgfpathlineto{\pgfqpoint{5.289267in}{2.925541in}}%
\pgfpathlineto{\pgfqpoint{5.296369in}{2.935897in}}%
\pgfpathlineto{\pgfqpoint{5.303467in}{2.946304in}}%
\pgfpathlineto{\pgfqpoint{5.310563in}{2.956767in}}%
\pgfpathlineto{\pgfqpoint{5.317655in}{2.967287in}}%
\pgfpathlineto{\pgfqpoint{5.304394in}{2.968104in}}%
\pgfpathlineto{\pgfqpoint{5.291142in}{2.969034in}}%
\pgfpathlineto{\pgfqpoint{5.277901in}{2.970080in}}%
\pgfpathlineto{\pgfqpoint{5.264669in}{2.971241in}}%
\pgfpathlineto{\pgfqpoint{5.257562in}{2.960489in}}%
\pgfpathlineto{\pgfqpoint{5.250453in}{2.949797in}}%
\pgfpathlineto{\pgfqpoint{5.243341in}{2.939165in}}%
\pgfpathlineto{\pgfqpoint{5.236226in}{2.928589in}}%
\pgfpathclose%
\pgfusepath{fill}%
\end{pgfscope}%
\begin{pgfscope}%
\pgfpathrectangle{\pgfqpoint{1.254980in}{0.150000in}}{\pgfqpoint{5.490039in}{5.490039in}}%
\pgfusepath{clip}%
\pgfsetbuttcap%
\pgfsetroundjoin%
\definecolor{currentfill}{rgb}{0.263663,0.237631,0.518762}%
\pgfsetfillcolor{currentfill}%
\pgfsetfillopacity{0.700000}%
\pgfsetlinewidth{0.000000pt}%
\definecolor{currentstroke}{rgb}{0.000000,0.000000,0.000000}%
\pgfsetstrokecolor{currentstroke}%
\pgfsetdash{}{0pt}%
\pgfpathmoveto{\pgfqpoint{4.695292in}{2.731605in}}%
\pgfpathlineto{\pgfqpoint{4.708363in}{2.728358in}}%
\pgfpathlineto{\pgfqpoint{4.721442in}{2.725233in}}%
\pgfpathlineto{\pgfqpoint{4.734529in}{2.722232in}}%
\pgfpathlineto{\pgfqpoint{4.747623in}{2.719354in}}%
\pgfpathlineto{\pgfqpoint{4.754890in}{2.729933in}}%
\pgfpathlineto{\pgfqpoint{4.762152in}{2.740547in}}%
\pgfpathlineto{\pgfqpoint{4.769412in}{2.751198in}}%
\pgfpathlineto{\pgfqpoint{4.776667in}{2.761887in}}%
\pgfpathlineto{\pgfqpoint{4.763583in}{2.764896in}}%
\pgfpathlineto{\pgfqpoint{4.750507in}{2.768029in}}%
\pgfpathlineto{\pgfqpoint{4.737439in}{2.771285in}}%
\pgfpathlineto{\pgfqpoint{4.724378in}{2.774664in}}%
\pgfpathlineto{\pgfqpoint{4.717112in}{2.763838in}}%
\pgfpathlineto{\pgfqpoint{4.709843in}{2.753054in}}%
\pgfpathlineto{\pgfqpoint{4.702569in}{2.742310in}}%
\pgfpathlineto{\pgfqpoint{4.695292in}{2.731605in}}%
\pgfpathclose%
\pgfusepath{fill}%
\end{pgfscope}%
\begin{pgfscope}%
\pgfpathrectangle{\pgfqpoint{1.254980in}{0.150000in}}{\pgfqpoint{5.490039in}{5.490039in}}%
\pgfusepath{clip}%
\pgfsetbuttcap%
\pgfsetroundjoin%
\definecolor{currentfill}{rgb}{0.218130,0.347432,0.550038}%
\pgfsetfillcolor{currentfill}%
\pgfsetfillopacity{0.700000}%
\pgfsetlinewidth{0.000000pt}%
\definecolor{currentstroke}{rgb}{0.000000,0.000000,0.000000}%
\pgfsetstrokecolor{currentstroke}%
\pgfsetdash{}{0pt}%
\pgfpathmoveto{\pgfqpoint{5.317655in}{2.967287in}}%
\pgfpathlineto{\pgfqpoint{5.330927in}{2.966586in}}%
\pgfpathlineto{\pgfqpoint{5.344209in}{2.965999in}}%
\pgfpathlineto{\pgfqpoint{5.357501in}{2.965526in}}%
\pgfpathlineto{\pgfqpoint{5.370804in}{2.965167in}}%
\pgfpathlineto{\pgfqpoint{5.377880in}{2.975510in}}%
\pgfpathlineto{\pgfqpoint{5.384953in}{2.985911in}}%
\pgfpathlineto{\pgfqpoint{5.392024in}{2.996373in}}%
\pgfpathlineto{\pgfqpoint{5.378732in}{2.996913in}}%
\pgfpathlineto{\pgfqpoint{5.365451in}{2.997566in}}%
\pgfpathlineto{\pgfqpoint{5.352180in}{2.998334in}}%
\pgfpathlineto{\pgfqpoint{5.338919in}{2.999216in}}%
\pgfpathlineto{\pgfqpoint{5.331833in}{2.988509in}}%
\pgfpathlineto{\pgfqpoint{5.324746in}{2.977867in}}%
\pgfpathlineto{\pgfqpoint{5.317655in}{2.967287in}}%
\pgfpathclose%
\pgfusepath{fill}%
\end{pgfscope}%
\begin{pgfscope}%
\pgfpathrectangle{\pgfqpoint{1.254980in}{0.150000in}}{\pgfqpoint{5.490039in}{5.490039in}}%
\pgfusepath{clip}%
\pgfsetbuttcap%
\pgfsetroundjoin%
\definecolor{currentfill}{rgb}{0.233603,0.313828,0.543914}%
\pgfsetfillcolor{currentfill}%
\pgfsetfillopacity{0.700000}%
\pgfsetlinewidth{0.000000pt}%
\definecolor{currentstroke}{rgb}{0.000000,0.000000,0.000000}%
\pgfsetstrokecolor{currentstroke}%
\pgfsetdash{}{0pt}%
\pgfpathmoveto{\pgfqpoint{5.154802in}{2.890858in}}%
\pgfpathlineto{\pgfqpoint{5.168021in}{2.889670in}}%
\pgfpathlineto{\pgfqpoint{5.181251in}{2.888599in}}%
\pgfpathlineto{\pgfqpoint{5.194489in}{2.887644in}}%
\pgfpathlineto{\pgfqpoint{5.207738in}{2.886805in}}%
\pgfpathlineto{\pgfqpoint{5.214865in}{2.897177in}}%
\pgfpathlineto{\pgfqpoint{5.221988in}{2.907597in}}%
\pgfpathlineto{\pgfqpoint{5.229109in}{2.918067in}}%
\pgfpathlineto{\pgfqpoint{5.236226in}{2.928589in}}%
\pgfpathlineto{\pgfqpoint{5.222991in}{2.929639in}}%
\pgfpathlineto{\pgfqpoint{5.209765in}{2.930805in}}%
\pgfpathlineto{\pgfqpoint{5.196549in}{2.932086in}}%
\pgfpathlineto{\pgfqpoint{5.183342in}{2.933484in}}%
\pgfpathlineto{\pgfqpoint{5.176212in}{2.922746in}}%
\pgfpathlineto{\pgfqpoint{5.169078in}{2.912064in}}%
\pgfpathlineto{\pgfqpoint{5.161941in}{2.901435in}}%
\pgfpathlineto{\pgfqpoint{5.154802in}{2.890858in}}%
\pgfpathclose%
\pgfusepath{fill}%
\end{pgfscope}%
\begin{pgfscope}%
\pgfpathrectangle{\pgfqpoint{1.254980in}{0.150000in}}{\pgfqpoint{5.490039in}{5.490039in}}%
\pgfusepath{clip}%
\pgfsetbuttcap%
\pgfsetroundjoin%
\definecolor{currentfill}{rgb}{0.199430,0.387607,0.554642}%
\pgfsetfillcolor{currentfill}%
\pgfsetfillopacity{0.700000}%
\pgfsetlinewidth{0.000000pt}%
\definecolor{currentstroke}{rgb}{0.000000,0.000000,0.000000}%
\pgfsetstrokecolor{currentstroke}%
\pgfsetdash{}{0pt}%
\pgfpathmoveto{\pgfqpoint{3.340351in}{3.085271in}}%
\pgfpathlineto{\pgfqpoint{3.353286in}{3.070230in}}%
\pgfpathlineto{\pgfqpoint{3.366220in}{3.055366in}}%
\pgfpathlineto{\pgfqpoint{3.379151in}{3.040680in}}%
\pgfpathlineto{\pgfqpoint{3.392079in}{3.026170in}}%
\pgfpathlineto{\pgfqpoint{3.399759in}{3.035822in}}%
\pgfpathlineto{\pgfqpoint{3.407434in}{3.045562in}}%
\pgfpathlineto{\pgfqpoint{3.415102in}{3.055390in}}%
\pgfpathlineto{\pgfqpoint{3.422764in}{3.065307in}}%
\pgfpathlineto{\pgfqpoint{3.409851in}{3.079804in}}%
\pgfpathlineto{\pgfqpoint{3.396936in}{3.094477in}}%
\pgfpathlineto{\pgfqpoint{3.384019in}{3.109327in}}%
\pgfpathlineto{\pgfqpoint{3.371099in}{3.124354in}}%
\pgfpathlineto{\pgfqpoint{3.363421in}{3.114445in}}%
\pgfpathlineto{\pgfqpoint{3.355737in}{3.104628in}}%
\pgfpathlineto{\pgfqpoint{3.348047in}{3.094903in}}%
\pgfpathlineto{\pgfqpoint{3.340351in}{3.085271in}}%
\pgfpathclose%
\pgfusepath{fill}%
\end{pgfscope}%
\begin{pgfscope}%
\pgfpathrectangle{\pgfqpoint{1.254980in}{0.150000in}}{\pgfqpoint{5.490039in}{5.490039in}}%
\pgfusepath{clip}%
\pgfsetbuttcap%
\pgfsetroundjoin%
\definecolor{currentfill}{rgb}{0.188923,0.410910,0.556326}%
\pgfsetfillcolor{currentfill}%
\pgfsetfillopacity{0.700000}%
\pgfsetlinewidth{0.000000pt}%
\definecolor{currentstroke}{rgb}{0.000000,0.000000,0.000000}%
\pgfsetstrokecolor{currentstroke}%
\pgfsetdash{}{0pt}%
\pgfpathmoveto{\pgfqpoint{3.288580in}{3.147236in}}%
\pgfpathlineto{\pgfqpoint{3.301527in}{3.131472in}}%
\pgfpathlineto{\pgfqpoint{3.314471in}{3.115891in}}%
\pgfpathlineto{\pgfqpoint{3.327412in}{3.100491in}}%
\pgfpathlineto{\pgfqpoint{3.340351in}{3.085271in}}%
\pgfpathlineto{\pgfqpoint{3.348047in}{3.094903in}}%
\pgfpathlineto{\pgfqpoint{3.355737in}{3.104628in}}%
\pgfpathlineto{\pgfqpoint{3.363421in}{3.114445in}}%
\pgfpathlineto{\pgfqpoint{3.371099in}{3.124354in}}%
\pgfpathlineto{\pgfqpoint{3.358177in}{3.139561in}}%
\pgfpathlineto{\pgfqpoint{3.345252in}{3.154947in}}%
\pgfpathlineto{\pgfqpoint{3.332325in}{3.170515in}}%
\pgfpathlineto{\pgfqpoint{3.319394in}{3.186265in}}%
\pgfpathlineto{\pgfqpoint{3.311700in}{3.176363in}}%
\pgfpathlineto{\pgfqpoint{3.304000in}{3.166558in}}%
\pgfpathlineto{\pgfqpoint{3.296293in}{3.156849in}}%
\pgfpathlineto{\pgfqpoint{3.288580in}{3.147236in}}%
\pgfpathclose%
\pgfusepath{fill}%
\end{pgfscope}%
\begin{pgfscope}%
\pgfpathrectangle{\pgfqpoint{1.254980in}{0.150000in}}{\pgfqpoint{5.490039in}{5.490039in}}%
\pgfusepath{clip}%
\pgfsetbuttcap%
\pgfsetroundjoin%
\definecolor{currentfill}{rgb}{0.273006,0.204520,0.501721}%
\pgfsetfillcolor{currentfill}%
\pgfsetfillopacity{0.700000}%
\pgfsetlinewidth{0.000000pt}%
\definecolor{currentstroke}{rgb}{0.000000,0.000000,0.000000}%
\pgfsetstrokecolor{currentstroke}%
\pgfsetdash{}{0pt}%
\pgfpathmoveto{\pgfqpoint{4.398971in}{2.669952in}}%
\pgfpathlineto{\pgfqpoint{4.411966in}{2.664978in}}%
\pgfpathlineto{\pgfqpoint{4.424967in}{2.660134in}}%
\pgfpathlineto{\pgfqpoint{4.437974in}{2.655418in}}%
\pgfpathlineto{\pgfqpoint{4.450987in}{2.650832in}}%
\pgfpathlineto{\pgfqpoint{4.458342in}{2.661407in}}%
\pgfpathlineto{\pgfqpoint{4.465694in}{2.672017in}}%
\pgfpathlineto{\pgfqpoint{4.473041in}{2.682665in}}%
\pgfpathlineto{\pgfqpoint{4.480384in}{2.693350in}}%
\pgfpathlineto{\pgfqpoint{4.467381in}{2.698021in}}%
\pgfpathlineto{\pgfqpoint{4.454385in}{2.702820in}}%
\pgfpathlineto{\pgfqpoint{4.441394in}{2.707749in}}%
\pgfpathlineto{\pgfqpoint{4.428410in}{2.712807in}}%
\pgfpathlineto{\pgfqpoint{4.421056in}{2.702032in}}%
\pgfpathlineto{\pgfqpoint{4.413699in}{2.691298in}}%
\pgfpathlineto{\pgfqpoint{4.406337in}{2.680605in}}%
\pgfpathlineto{\pgfqpoint{4.398971in}{2.669952in}}%
\pgfpathclose%
\pgfusepath{fill}%
\end{pgfscope}%
\begin{pgfscope}%
\pgfpathrectangle{\pgfqpoint{1.254980in}{0.150000in}}{\pgfqpoint{5.490039in}{5.490039in}}%
\pgfusepath{clip}%
\pgfsetbuttcap%
\pgfsetroundjoin%
\definecolor{currentfill}{rgb}{0.210503,0.363727,0.552206}%
\pgfsetfillcolor{currentfill}%
\pgfsetfillopacity{0.700000}%
\pgfsetlinewidth{0.000000pt}%
\definecolor{currentstroke}{rgb}{0.000000,0.000000,0.000000}%
\pgfsetstrokecolor{currentstroke}%
\pgfsetdash{}{0pt}%
\pgfpathmoveto{\pgfqpoint{3.392079in}{3.026170in}}%
\pgfpathlineto{\pgfqpoint{3.405006in}{3.011834in}}%
\pgfpathlineto{\pgfqpoint{3.417930in}{2.997672in}}%
\pgfpathlineto{\pgfqpoint{3.430853in}{2.983683in}}%
\pgfpathlineto{\pgfqpoint{3.443775in}{2.969866in}}%
\pgfpathlineto{\pgfqpoint{3.451439in}{2.979537in}}%
\pgfpathlineto{\pgfqpoint{3.459098in}{2.989292in}}%
\pgfpathlineto{\pgfqpoint{3.466750in}{2.999131in}}%
\pgfpathlineto{\pgfqpoint{3.474397in}{3.009055in}}%
\pgfpathlineto{\pgfqpoint{3.461491in}{3.022860in}}%
\pgfpathlineto{\pgfqpoint{3.448584in}{3.036836in}}%
\pgfpathlineto{\pgfqpoint{3.435675in}{3.050984in}}%
\pgfpathlineto{\pgfqpoint{3.422764in}{3.065307in}}%
\pgfpathlineto{\pgfqpoint{3.415102in}{3.055390in}}%
\pgfpathlineto{\pgfqpoint{3.407434in}{3.045562in}}%
\pgfpathlineto{\pgfqpoint{3.399759in}{3.035822in}}%
\pgfpathlineto{\pgfqpoint{3.392079in}{3.026170in}}%
\pgfpathclose%
\pgfusepath{fill}%
\end{pgfscope}%
\begin{pgfscope}%
\pgfpathrectangle{\pgfqpoint{1.254980in}{0.150000in}}{\pgfqpoint{5.490039in}{5.490039in}}%
\pgfusepath{clip}%
\pgfsetbuttcap%
\pgfsetroundjoin%
\definecolor{currentfill}{rgb}{0.177423,0.437527,0.557565}%
\pgfsetfillcolor{currentfill}%
\pgfsetfillopacity{0.700000}%
\pgfsetlinewidth{0.000000pt}%
\definecolor{currentstroke}{rgb}{0.000000,0.000000,0.000000}%
\pgfsetstrokecolor{currentstroke}%
\pgfsetdash{}{0pt}%
\pgfpathmoveto{\pgfqpoint{3.236757in}{3.212138in}}%
\pgfpathlineto{\pgfqpoint{3.249718in}{3.195634in}}%
\pgfpathlineto{\pgfqpoint{3.262675in}{3.179316in}}%
\pgfpathlineto{\pgfqpoint{3.275629in}{3.163184in}}%
\pgfpathlineto{\pgfqpoint{3.288580in}{3.147236in}}%
\pgfpathlineto{\pgfqpoint{3.296293in}{3.156849in}}%
\pgfpathlineto{\pgfqpoint{3.304000in}{3.166558in}}%
\pgfpathlineto{\pgfqpoint{3.311700in}{3.176363in}}%
\pgfpathlineto{\pgfqpoint{3.319394in}{3.186265in}}%
\pgfpathlineto{\pgfqpoint{3.306460in}{3.202199in}}%
\pgfpathlineto{\pgfqpoint{3.293523in}{3.218317in}}%
\pgfpathlineto{\pgfqpoint{3.280583in}{3.234621in}}%
\pgfpathlineto{\pgfqpoint{3.267638in}{3.251111in}}%
\pgfpathlineto{\pgfqpoint{3.259928in}{3.241217in}}%
\pgfpathlineto{\pgfqpoint{3.252211in}{3.231424in}}%
\pgfpathlineto{\pgfqpoint{3.244487in}{3.221731in}}%
\pgfpathlineto{\pgfqpoint{3.236757in}{3.212138in}}%
\pgfpathclose%
\pgfusepath{fill}%
\end{pgfscope}%
\begin{pgfscope}%
\pgfpathrectangle{\pgfqpoint{1.254980in}{0.150000in}}{\pgfqpoint{5.490039in}{5.490039in}}%
\pgfusepath{clip}%
\pgfsetbuttcap%
\pgfsetroundjoin%
\definecolor{currentfill}{rgb}{0.241237,0.296485,0.539709}%
\pgfsetfillcolor{currentfill}%
\pgfsetfillopacity{0.700000}%
\pgfsetlinewidth{0.000000pt}%
\definecolor{currentstroke}{rgb}{0.000000,0.000000,0.000000}%
\pgfsetstrokecolor{currentstroke}%
\pgfsetdash{}{0pt}%
\pgfpathmoveto{\pgfqpoint{5.073379in}{2.854160in}}%
\pgfpathlineto{\pgfqpoint{5.086573in}{2.852699in}}%
\pgfpathlineto{\pgfqpoint{5.099777in}{2.851356in}}%
\pgfpathlineto{\pgfqpoint{5.112990in}{2.850131in}}%
\pgfpathlineto{\pgfqpoint{5.126212in}{2.849022in}}%
\pgfpathlineto{\pgfqpoint{5.133365in}{2.859414in}}%
\pgfpathlineto{\pgfqpoint{5.140514in}{2.869849in}}%
\pgfpathlineto{\pgfqpoint{5.147659in}{2.880330in}}%
\pgfpathlineto{\pgfqpoint{5.154802in}{2.890858in}}%
\pgfpathlineto{\pgfqpoint{5.141592in}{2.892162in}}%
\pgfpathlineto{\pgfqpoint{5.128391in}{2.893582in}}%
\pgfpathlineto{\pgfqpoint{5.115200in}{2.895120in}}%
\pgfpathlineto{\pgfqpoint{5.102018in}{2.896775in}}%
\pgfpathlineto{\pgfqpoint{5.094863in}{2.886047in}}%
\pgfpathlineto{\pgfqpoint{5.087705in}{2.875369in}}%
\pgfpathlineto{\pgfqpoint{5.080543in}{2.864741in}}%
\pgfpathlineto{\pgfqpoint{5.073379in}{2.854160in}}%
\pgfpathclose%
\pgfusepath{fill}%
\end{pgfscope}%
\begin{pgfscope}%
\pgfpathrectangle{\pgfqpoint{1.254980in}{0.150000in}}{\pgfqpoint{5.490039in}{5.490039in}}%
\pgfusepath{clip}%
\pgfsetbuttcap%
\pgfsetroundjoin%
\definecolor{currentfill}{rgb}{0.273006,0.204520,0.501721}%
\pgfsetfillcolor{currentfill}%
\pgfsetfillopacity{0.700000}%
\pgfsetlinewidth{0.000000pt}%
\definecolor{currentstroke}{rgb}{0.000000,0.000000,0.000000}%
\pgfsetstrokecolor{currentstroke}%
\pgfsetdash{}{0pt}%
\pgfpathmoveto{\pgfqpoint{4.050810in}{2.669145in}}%
\pgfpathlineto{\pgfqpoint{4.063740in}{2.661636in}}%
\pgfpathlineto{\pgfqpoint{4.076674in}{2.654266in}}%
\pgfpathlineto{\pgfqpoint{4.089612in}{2.647035in}}%
\pgfpathlineto{\pgfqpoint{4.102554in}{2.639942in}}%
\pgfpathlineto{\pgfqpoint{4.110014in}{2.650313in}}%
\pgfpathlineto{\pgfqpoint{4.117469in}{2.660729in}}%
\pgfpathlineto{\pgfqpoint{4.124920in}{2.671191in}}%
\pgfpathlineto{\pgfqpoint{4.132367in}{2.681700in}}%
\pgfpathlineto{\pgfqpoint{4.119436in}{2.688829in}}%
\pgfpathlineto{\pgfqpoint{4.106509in}{2.696097in}}%
\pgfpathlineto{\pgfqpoint{4.093586in}{2.703503in}}%
\pgfpathlineto{\pgfqpoint{4.080667in}{2.711050in}}%
\pgfpathlineto{\pgfqpoint{4.073210in}{2.700498in}}%
\pgfpathlineto{\pgfqpoint{4.065748in}{2.689998in}}%
\pgfpathlineto{\pgfqpoint{4.058281in}{2.679547in}}%
\pgfpathlineto{\pgfqpoint{4.050810in}{2.669145in}}%
\pgfpathclose%
\pgfusepath{fill}%
\end{pgfscope}%
\begin{pgfscope}%
\pgfpathrectangle{\pgfqpoint{1.254980in}{0.150000in}}{\pgfqpoint{5.490039in}{5.490039in}}%
\pgfusepath{clip}%
\pgfsetbuttcap%
\pgfsetroundjoin%
\definecolor{currentfill}{rgb}{0.258965,0.251537,0.524736}%
\pgfsetfillcolor{currentfill}%
\pgfsetfillopacity{0.700000}%
\pgfsetlinewidth{0.000000pt}%
\definecolor{currentstroke}{rgb}{0.000000,0.000000,0.000000}%
\pgfsetstrokecolor{currentstroke}%
\pgfsetdash{}{0pt}%
\pgfpathmoveto{\pgfqpoint{3.732400in}{2.767591in}}%
\pgfpathlineto{\pgfqpoint{3.745304in}{2.757183in}}%
\pgfpathlineto{\pgfqpoint{3.758210in}{2.746928in}}%
\pgfpathlineto{\pgfqpoint{3.771117in}{2.736824in}}%
\pgfpathlineto{\pgfqpoint{3.784026in}{2.726872in}}%
\pgfpathlineto{\pgfqpoint{3.791583in}{2.736893in}}%
\pgfpathlineto{\pgfqpoint{3.799136in}{2.746975in}}%
\pgfpathlineto{\pgfqpoint{3.806683in}{2.757119in}}%
\pgfpathlineto{\pgfqpoint{3.814226in}{2.767325in}}%
\pgfpathlineto{\pgfqpoint{3.801330in}{2.777282in}}%
\pgfpathlineto{\pgfqpoint{3.788436in}{2.787390in}}%
\pgfpathlineto{\pgfqpoint{3.775543in}{2.797650in}}%
\pgfpathlineto{\pgfqpoint{3.762651in}{2.808063in}}%
\pgfpathlineto{\pgfqpoint{3.755096in}{2.797846in}}%
\pgfpathlineto{\pgfqpoint{3.747536in}{2.787696in}}%
\pgfpathlineto{\pgfqpoint{3.739970in}{2.777611in}}%
\pgfpathlineto{\pgfqpoint{3.732400in}{2.767591in}}%
\pgfpathclose%
\pgfusepath{fill}%
\end{pgfscope}%
\begin{pgfscope}%
\pgfpathrectangle{\pgfqpoint{1.254980in}{0.150000in}}{\pgfqpoint{5.490039in}{5.490039in}}%
\pgfusepath{clip}%
\pgfsetbuttcap%
\pgfsetroundjoin%
\definecolor{currentfill}{rgb}{0.221989,0.339161,0.548752}%
\pgfsetfillcolor{currentfill}%
\pgfsetfillopacity{0.700000}%
\pgfsetlinewidth{0.000000pt}%
\definecolor{currentstroke}{rgb}{0.000000,0.000000,0.000000}%
\pgfsetstrokecolor{currentstroke}%
\pgfsetdash{}{0pt}%
\pgfpathmoveto{\pgfqpoint{3.443775in}{2.969866in}}%
\pgfpathlineto{\pgfqpoint{3.456694in}{2.956220in}}%
\pgfpathlineto{\pgfqpoint{3.469613in}{2.942743in}}%
\pgfpathlineto{\pgfqpoint{3.482530in}{2.929435in}}%
\pgfpathlineto{\pgfqpoint{3.495446in}{2.916295in}}%
\pgfpathlineto{\pgfqpoint{3.503095in}{2.925985in}}%
\pgfpathlineto{\pgfqpoint{3.510739in}{2.935754in}}%
\pgfpathlineto{\pgfqpoint{3.518376in}{2.945604in}}%
\pgfpathlineto{\pgfqpoint{3.526008in}{2.955535in}}%
\pgfpathlineto{\pgfqpoint{3.513107in}{2.968662in}}%
\pgfpathlineto{\pgfqpoint{3.500205in}{2.981958in}}%
\pgfpathlineto{\pgfqpoint{3.487302in}{2.995422in}}%
\pgfpathlineto{\pgfqpoint{3.474397in}{3.009055in}}%
\pgfpathlineto{\pgfqpoint{3.466750in}{2.999131in}}%
\pgfpathlineto{\pgfqpoint{3.459098in}{2.989292in}}%
\pgfpathlineto{\pgfqpoint{3.451439in}{2.979537in}}%
\pgfpathlineto{\pgfqpoint{3.443775in}{2.969866in}}%
\pgfpathclose%
\pgfusepath{fill}%
\end{pgfscope}%
\begin{pgfscope}%
\pgfpathrectangle{\pgfqpoint{1.254980in}{0.150000in}}{\pgfqpoint{5.490039in}{5.490039in}}%
\pgfusepath{clip}%
\pgfsetbuttcap%
\pgfsetroundjoin%
\definecolor{currentfill}{rgb}{0.267968,0.223549,0.512008}%
\pgfsetfillcolor{currentfill}%
\pgfsetfillopacity{0.700000}%
\pgfsetlinewidth{0.000000pt}%
\definecolor{currentstroke}{rgb}{0.000000,0.000000,0.000000}%
\pgfsetstrokecolor{currentstroke}%
\pgfsetdash{}{0pt}%
\pgfpathmoveto{\pgfqpoint{4.613892in}{2.702917in}}%
\pgfpathlineto{\pgfqpoint{4.626944in}{2.699288in}}%
\pgfpathlineto{\pgfqpoint{4.640004in}{2.695783in}}%
\pgfpathlineto{\pgfqpoint{4.653071in}{2.692403in}}%
\pgfpathlineto{\pgfqpoint{4.666145in}{2.689148in}}%
\pgfpathlineto{\pgfqpoint{4.673438in}{2.699711in}}%
\pgfpathlineto{\pgfqpoint{4.680726in}{2.710307in}}%
\pgfpathlineto{\pgfqpoint{4.688011in}{2.720938in}}%
\pgfpathlineto{\pgfqpoint{4.695292in}{2.731605in}}%
\pgfpathlineto{\pgfqpoint{4.682228in}{2.734977in}}%
\pgfpathlineto{\pgfqpoint{4.669172in}{2.738473in}}%
\pgfpathlineto{\pgfqpoint{4.656122in}{2.742093in}}%
\pgfpathlineto{\pgfqpoint{4.643080in}{2.745838in}}%
\pgfpathlineto{\pgfqpoint{4.635789in}{2.735049in}}%
\pgfpathlineto{\pgfqpoint{4.628494in}{2.724300in}}%
\pgfpathlineto{\pgfqpoint{4.621195in}{2.713590in}}%
\pgfpathlineto{\pgfqpoint{4.613892in}{2.702917in}}%
\pgfpathclose%
\pgfusepath{fill}%
\end{pgfscope}%
\begin{pgfscope}%
\pgfpathrectangle{\pgfqpoint{1.254980in}{0.150000in}}{\pgfqpoint{5.490039in}{5.490039in}}%
\pgfusepath{clip}%
\pgfsetbuttcap%
\pgfsetroundjoin%
\definecolor{currentfill}{rgb}{0.274128,0.199721,0.498911}%
\pgfsetfillcolor{currentfill}%
\pgfsetfillopacity{0.700000}%
\pgfsetlinewidth{0.000000pt}%
\definecolor{currentstroke}{rgb}{0.000000,0.000000,0.000000}%
\pgfsetstrokecolor{currentstroke}%
\pgfsetdash{}{0pt}%
\pgfpathmoveto{\pgfqpoint{4.184132in}{2.654557in}}%
\pgfpathlineto{\pgfqpoint{4.197085in}{2.648112in}}%
\pgfpathlineto{\pgfqpoint{4.210042in}{2.641802in}}%
\pgfpathlineto{\pgfqpoint{4.223004in}{2.635627in}}%
\pgfpathlineto{\pgfqpoint{4.235971in}{2.629587in}}%
\pgfpathlineto{\pgfqpoint{4.243392in}{2.640050in}}%
\pgfpathlineto{\pgfqpoint{4.250808in}{2.650553in}}%
\pgfpathlineto{\pgfqpoint{4.258219in}{2.661098in}}%
\pgfpathlineto{\pgfqpoint{4.265627in}{2.671684in}}%
\pgfpathlineto{\pgfqpoint{4.252671in}{2.677778in}}%
\pgfpathlineto{\pgfqpoint{4.239719in}{2.684005in}}%
\pgfpathlineto{\pgfqpoint{4.226773in}{2.690367in}}%
\pgfpathlineto{\pgfqpoint{4.213831in}{2.696864in}}%
\pgfpathlineto{\pgfqpoint{4.206413in}{2.686220in}}%
\pgfpathlineto{\pgfqpoint{4.198990in}{2.675620in}}%
\pgfpathlineto{\pgfqpoint{4.191563in}{2.665066in}}%
\pgfpathlineto{\pgfqpoint{4.184132in}{2.654557in}}%
\pgfpathclose%
\pgfusepath{fill}%
\end{pgfscope}%
\begin{pgfscope}%
\pgfpathrectangle{\pgfqpoint{1.254980in}{0.150000in}}{\pgfqpoint{5.490039in}{5.490039in}}%
\pgfusepath{clip}%
\pgfsetbuttcap%
\pgfsetroundjoin%
\definecolor{currentfill}{rgb}{0.166617,0.463708,0.558119}%
\pgfsetfillcolor{currentfill}%
\pgfsetfillopacity{0.700000}%
\pgfsetlinewidth{0.000000pt}%
\definecolor{currentstroke}{rgb}{0.000000,0.000000,0.000000}%
\pgfsetstrokecolor{currentstroke}%
\pgfsetdash{}{0pt}%
\pgfpathmoveto{\pgfqpoint{3.184872in}{3.280052in}}%
\pgfpathlineto{\pgfqpoint{3.197850in}{3.262787in}}%
\pgfpathlineto{\pgfqpoint{3.210823in}{3.245714in}}%
\pgfpathlineto{\pgfqpoint{3.223792in}{3.228832in}}%
\pgfpathlineto{\pgfqpoint{3.236757in}{3.212138in}}%
\pgfpathlineto{\pgfqpoint{3.244487in}{3.221731in}}%
\pgfpathlineto{\pgfqpoint{3.252211in}{3.231424in}}%
\pgfpathlineto{\pgfqpoint{3.259928in}{3.241217in}}%
\pgfpathlineto{\pgfqpoint{3.267638in}{3.251111in}}%
\pgfpathlineto{\pgfqpoint{3.254691in}{3.267790in}}%
\pgfpathlineto{\pgfqpoint{3.241739in}{3.284658in}}%
\pgfpathlineto{\pgfqpoint{3.228783in}{3.301717in}}%
\pgfpathlineto{\pgfqpoint{3.215823in}{3.318968in}}%
\pgfpathlineto{\pgfqpoint{3.208095in}{3.309082in}}%
\pgfpathlineto{\pgfqpoint{3.200361in}{3.299301in}}%
\pgfpathlineto{\pgfqpoint{3.192620in}{3.289624in}}%
\pgfpathlineto{\pgfqpoint{3.184872in}{3.280052in}}%
\pgfpathclose%
\pgfusepath{fill}%
\end{pgfscope}%
\begin{pgfscope}%
\pgfpathrectangle{\pgfqpoint{1.254980in}{0.150000in}}{\pgfqpoint{5.490039in}{5.490039in}}%
\pgfusepath{clip}%
\pgfsetbuttcap%
\pgfsetroundjoin%
\definecolor{currentfill}{rgb}{0.270595,0.214069,0.507052}%
\pgfsetfillcolor{currentfill}%
\pgfsetfillopacity{0.700000}%
\pgfsetlinewidth{0.000000pt}%
\definecolor{currentstroke}{rgb}{0.000000,0.000000,0.000000}%
\pgfsetstrokecolor{currentstroke}%
\pgfsetdash{}{0pt}%
\pgfpathmoveto{\pgfqpoint{3.917467in}{2.693030in}}%
\pgfpathlineto{\pgfqpoint{3.930383in}{2.684402in}}%
\pgfpathlineto{\pgfqpoint{3.943302in}{2.675920in}}%
\pgfpathlineto{\pgfqpoint{3.956224in}{2.667581in}}%
\pgfpathlineto{\pgfqpoint{3.969148in}{2.659385in}}%
\pgfpathlineto{\pgfqpoint{3.976649in}{2.669610in}}%
\pgfpathlineto{\pgfqpoint{3.984146in}{2.679885in}}%
\pgfpathlineto{\pgfqpoint{3.991637in}{2.690213in}}%
\pgfpathlineto{\pgfqpoint{3.999124in}{2.700592in}}%
\pgfpathlineto{\pgfqpoint{3.986211in}{2.708808in}}%
\pgfpathlineto{\pgfqpoint{3.973301in}{2.717168in}}%
\pgfpathlineto{\pgfqpoint{3.960394in}{2.725672in}}%
\pgfpathlineto{\pgfqpoint{3.947490in}{2.734319in}}%
\pgfpathlineto{\pgfqpoint{3.939991in}{2.723913in}}%
\pgfpathlineto{\pgfqpoint{3.932488in}{2.713563in}}%
\pgfpathlineto{\pgfqpoint{3.924980in}{2.703269in}}%
\pgfpathlineto{\pgfqpoint{3.917467in}{2.693030in}}%
\pgfpathclose%
\pgfusepath{fill}%
\end{pgfscope}%
\begin{pgfscope}%
\pgfpathrectangle{\pgfqpoint{1.254980in}{0.150000in}}{\pgfqpoint{5.490039in}{5.490039in}}%
\pgfusepath{clip}%
\pgfsetbuttcap%
\pgfsetroundjoin%
\definecolor{currentfill}{rgb}{0.246811,0.283237,0.535941}%
\pgfsetfillcolor{currentfill}%
\pgfsetfillopacity{0.700000}%
\pgfsetlinewidth{0.000000pt}%
\definecolor{currentstroke}{rgb}{0.000000,0.000000,0.000000}%
\pgfsetstrokecolor{currentstroke}%
\pgfsetdash{}{0pt}%
\pgfpathmoveto{\pgfqpoint{4.991954in}{2.818571in}}%
\pgfpathlineto{\pgfqpoint{5.005124in}{2.816817in}}%
\pgfpathlineto{\pgfqpoint{5.018302in}{2.815182in}}%
\pgfpathlineto{\pgfqpoint{5.031490in}{2.813666in}}%
\pgfpathlineto{\pgfqpoint{5.044687in}{2.812267in}}%
\pgfpathlineto{\pgfqpoint{5.051865in}{2.822679in}}%
\pgfpathlineto{\pgfqpoint{5.059040in}{2.833131in}}%
\pgfpathlineto{\pgfqpoint{5.066211in}{2.843624in}}%
\pgfpathlineto{\pgfqpoint{5.073379in}{2.854160in}}%
\pgfpathlineto{\pgfqpoint{5.060194in}{2.855738in}}%
\pgfpathlineto{\pgfqpoint{5.047018in}{2.857434in}}%
\pgfpathlineto{\pgfqpoint{5.033851in}{2.859248in}}%
\pgfpathlineto{\pgfqpoint{5.020693in}{2.861180in}}%
\pgfpathlineto{\pgfqpoint{5.013513in}{2.850459in}}%
\pgfpathlineto{\pgfqpoint{5.006330in}{2.839785in}}%
\pgfpathlineto{\pgfqpoint{4.999143in}{2.829156in}}%
\pgfpathlineto{\pgfqpoint{4.991954in}{2.818571in}}%
\pgfpathclose%
\pgfusepath{fill}%
\end{pgfscope}%
\begin{pgfscope}%
\pgfpathrectangle{\pgfqpoint{1.254980in}{0.150000in}}{\pgfqpoint{5.490039in}{5.490039in}}%
\pgfusepath{clip}%
\pgfsetbuttcap%
\pgfsetroundjoin%
\definecolor{currentfill}{rgb}{0.231674,0.318106,0.544834}%
\pgfsetfillcolor{currentfill}%
\pgfsetfillopacity{0.700000}%
\pgfsetlinewidth{0.000000pt}%
\definecolor{currentstroke}{rgb}{0.000000,0.000000,0.000000}%
\pgfsetstrokecolor{currentstroke}%
\pgfsetdash{}{0pt}%
\pgfpathmoveto{\pgfqpoint{3.495446in}{2.916295in}}%
\pgfpathlineto{\pgfqpoint{3.508361in}{2.903322in}}%
\pgfpathlineto{\pgfqpoint{3.521276in}{2.890515in}}%
\pgfpathlineto{\pgfqpoint{3.534189in}{2.877873in}}%
\pgfpathlineto{\pgfqpoint{3.547103in}{2.865395in}}%
\pgfpathlineto{\pgfqpoint{3.554737in}{2.875103in}}%
\pgfpathlineto{\pgfqpoint{3.562365in}{2.884887in}}%
\pgfpathlineto{\pgfqpoint{3.569988in}{2.894748in}}%
\pgfpathlineto{\pgfqpoint{3.577605in}{2.904685in}}%
\pgfpathlineto{\pgfqpoint{3.564707in}{2.917150in}}%
\pgfpathlineto{\pgfqpoint{3.551808in}{2.929780in}}%
\pgfpathlineto{\pgfqpoint{3.538908in}{2.942574in}}%
\pgfpathlineto{\pgfqpoint{3.526008in}{2.955535in}}%
\pgfpathlineto{\pgfqpoint{3.518376in}{2.945604in}}%
\pgfpathlineto{\pgfqpoint{3.510739in}{2.935754in}}%
\pgfpathlineto{\pgfqpoint{3.503095in}{2.925985in}}%
\pgfpathlineto{\pgfqpoint{3.495446in}{2.916295in}}%
\pgfpathclose%
\pgfusepath{fill}%
\end{pgfscope}%
\begin{pgfscope}%
\pgfpathrectangle{\pgfqpoint{1.254980in}{0.150000in}}{\pgfqpoint{5.490039in}{5.490039in}}%
\pgfusepath{clip}%
\pgfsetbuttcap%
\pgfsetroundjoin%
\definecolor{currentfill}{rgb}{0.156270,0.489624,0.557936}%
\pgfsetfillcolor{currentfill}%
\pgfsetfillopacity{0.700000}%
\pgfsetlinewidth{0.000000pt}%
\definecolor{currentstroke}{rgb}{0.000000,0.000000,0.000000}%
\pgfsetstrokecolor{currentstroke}%
\pgfsetdash{}{0pt}%
\pgfpathmoveto{\pgfqpoint{3.132915in}{3.351058in}}%
\pgfpathlineto{\pgfqpoint{3.145912in}{3.333012in}}%
\pgfpathlineto{\pgfqpoint{3.158903in}{3.315163in}}%
\pgfpathlineto{\pgfqpoint{3.171890in}{3.297511in}}%
\pgfpathlineto{\pgfqpoint{3.184872in}{3.280052in}}%
\pgfpathlineto{\pgfqpoint{3.192620in}{3.289624in}}%
\pgfpathlineto{\pgfqpoint{3.200361in}{3.299301in}}%
\pgfpathlineto{\pgfqpoint{3.208095in}{3.309082in}}%
\pgfpathlineto{\pgfqpoint{3.215823in}{3.318968in}}%
\pgfpathlineto{\pgfqpoint{3.202858in}{3.336411in}}%
\pgfpathlineto{\pgfqpoint{3.189889in}{3.354049in}}%
\pgfpathlineto{\pgfqpoint{3.176916in}{3.371883in}}%
\pgfpathlineto{\pgfqpoint{3.163937in}{3.389914in}}%
\pgfpathlineto{\pgfqpoint{3.156192in}{3.380036in}}%
\pgfpathlineto{\pgfqpoint{3.148440in}{3.370268in}}%
\pgfpathlineto{\pgfqpoint{3.140681in}{3.360609in}}%
\pgfpathlineto{\pgfqpoint{3.132915in}{3.351058in}}%
\pgfpathclose%
\pgfusepath{fill}%
\end{pgfscope}%
\begin{pgfscope}%
\pgfpathrectangle{\pgfqpoint{1.254980in}{0.150000in}}{\pgfqpoint{5.490039in}{5.490039in}}%
\pgfusepath{clip}%
\pgfsetbuttcap%
\pgfsetroundjoin%
\definecolor{currentfill}{rgb}{0.252194,0.269783,0.531579}%
\pgfsetfillcolor{currentfill}%
\pgfsetfillopacity{0.700000}%
\pgfsetlinewidth{0.000000pt}%
\definecolor{currentstroke}{rgb}{0.000000,0.000000,0.000000}%
\pgfsetstrokecolor{currentstroke}%
\pgfsetdash{}{0pt}%
\pgfpathmoveto{\pgfqpoint{4.910522in}{2.784175in}}%
\pgfpathlineto{\pgfqpoint{4.923669in}{2.782108in}}%
\pgfpathlineto{\pgfqpoint{4.936823in}{2.780161in}}%
\pgfpathlineto{\pgfqpoint{4.949987in}{2.778333in}}%
\pgfpathlineto{\pgfqpoint{4.963160in}{2.776624in}}%
\pgfpathlineto{\pgfqpoint{4.970363in}{2.787055in}}%
\pgfpathlineto{\pgfqpoint{4.977564in}{2.797522in}}%
\pgfpathlineto{\pgfqpoint{4.984760in}{2.808026in}}%
\pgfpathlineto{\pgfqpoint{4.991954in}{2.818571in}}%
\pgfpathlineto{\pgfqpoint{4.978793in}{2.820443in}}%
\pgfpathlineto{\pgfqpoint{4.965640in}{2.822434in}}%
\pgfpathlineto{\pgfqpoint{4.952497in}{2.824544in}}%
\pgfpathlineto{\pgfqpoint{4.939362in}{2.826774in}}%
\pgfpathlineto{\pgfqpoint{4.932157in}{2.816061in}}%
\pgfpathlineto{\pgfqpoint{4.924949in}{2.805391in}}%
\pgfpathlineto{\pgfqpoint{4.917737in}{2.794763in}}%
\pgfpathlineto{\pgfqpoint{4.910522in}{2.784175in}}%
\pgfpathclose%
\pgfusepath{fill}%
\end{pgfscope}%
\begin{pgfscope}%
\pgfpathrectangle{\pgfqpoint{1.254980in}{0.150000in}}{\pgfqpoint{5.490039in}{5.490039in}}%
\pgfusepath{clip}%
\pgfsetbuttcap%
\pgfsetroundjoin%
\definecolor{currentfill}{rgb}{0.274128,0.199721,0.498911}%
\pgfsetfillcolor{currentfill}%
\pgfsetfillopacity{0.700000}%
\pgfsetlinewidth{0.000000pt}%
\definecolor{currentstroke}{rgb}{0.000000,0.000000,0.000000}%
\pgfsetstrokecolor{currentstroke}%
\pgfsetdash{}{0pt}%
\pgfpathmoveto{\pgfqpoint{4.317503in}{2.648645in}}%
\pgfpathlineto{\pgfqpoint{4.330486in}{2.643216in}}%
\pgfpathlineto{\pgfqpoint{4.343474in}{2.637919in}}%
\pgfpathlineto{\pgfqpoint{4.356467in}{2.632753in}}%
\pgfpathlineto{\pgfqpoint{4.369467in}{2.627717in}}%
\pgfpathlineto{\pgfqpoint{4.376849in}{2.638221in}}%
\pgfpathlineto{\pgfqpoint{4.384227in}{2.648761in}}%
\pgfpathlineto{\pgfqpoint{4.391601in}{2.659338in}}%
\pgfpathlineto{\pgfqpoint{4.398971in}{2.669952in}}%
\pgfpathlineto{\pgfqpoint{4.385982in}{2.675057in}}%
\pgfpathlineto{\pgfqpoint{4.372999in}{2.680291in}}%
\pgfpathlineto{\pgfqpoint{4.360021in}{2.685657in}}%
\pgfpathlineto{\pgfqpoint{4.347049in}{2.691155in}}%
\pgfpathlineto{\pgfqpoint{4.339669in}{2.680466in}}%
\pgfpathlineto{\pgfqpoint{4.332285in}{2.669819in}}%
\pgfpathlineto{\pgfqpoint{4.324896in}{2.659212in}}%
\pgfpathlineto{\pgfqpoint{4.317503in}{2.648645in}}%
\pgfpathclose%
\pgfusepath{fill}%
\end{pgfscope}%
\begin{pgfscope}%
\pgfpathrectangle{\pgfqpoint{1.254980in}{0.150000in}}{\pgfqpoint{5.490039in}{5.490039in}}%
\pgfusepath{clip}%
\pgfsetbuttcap%
\pgfsetroundjoin%
\definecolor{currentfill}{rgb}{0.241237,0.296485,0.539709}%
\pgfsetfillcolor{currentfill}%
\pgfsetfillopacity{0.700000}%
\pgfsetlinewidth{0.000000pt}%
\definecolor{currentstroke}{rgb}{0.000000,0.000000,0.000000}%
\pgfsetstrokecolor{currentstroke}%
\pgfsetdash{}{0pt}%
\pgfpathmoveto{\pgfqpoint{3.547103in}{2.865395in}}%
\pgfpathlineto{\pgfqpoint{3.560015in}{2.853081in}}%
\pgfpathlineto{\pgfqpoint{3.572928in}{2.840929in}}%
\pgfpathlineto{\pgfqpoint{3.585840in}{2.828939in}}%
\pgfpathlineto{\pgfqpoint{3.598752in}{2.817109in}}%
\pgfpathlineto{\pgfqpoint{3.606372in}{2.826835in}}%
\pgfpathlineto{\pgfqpoint{3.613986in}{2.836633in}}%
\pgfpathlineto{\pgfqpoint{3.621594in}{2.846504in}}%
\pgfpathlineto{\pgfqpoint{3.629198in}{2.856447in}}%
\pgfpathlineto{\pgfqpoint{3.616299in}{2.868265in}}%
\pgfpathlineto{\pgfqpoint{3.603401in}{2.880243in}}%
\pgfpathlineto{\pgfqpoint{3.590503in}{2.892383in}}%
\pgfpathlineto{\pgfqpoint{3.577605in}{2.904685in}}%
\pgfpathlineto{\pgfqpoint{3.569988in}{2.894748in}}%
\pgfpathlineto{\pgfqpoint{3.562365in}{2.884887in}}%
\pgfpathlineto{\pgfqpoint{3.554737in}{2.875103in}}%
\pgfpathlineto{\pgfqpoint{3.547103in}{2.865395in}}%
\pgfpathclose%
\pgfusepath{fill}%
\end{pgfscope}%
\begin{pgfscope}%
\pgfpathrectangle{\pgfqpoint{1.254980in}{0.150000in}}{\pgfqpoint{5.490039in}{5.490039in}}%
\pgfusepath{clip}%
\pgfsetbuttcap%
\pgfsetroundjoin%
\definecolor{currentfill}{rgb}{0.270595,0.214069,0.507052}%
\pgfsetfillcolor{currentfill}%
\pgfsetfillopacity{0.700000}%
\pgfsetlinewidth{0.000000pt}%
\definecolor{currentstroke}{rgb}{0.000000,0.000000,0.000000}%
\pgfsetstrokecolor{currentstroke}%
\pgfsetdash{}{0pt}%
\pgfpathmoveto{\pgfqpoint{4.532459in}{2.675947in}}%
\pgfpathlineto{\pgfqpoint{4.545494in}{2.671915in}}%
\pgfpathlineto{\pgfqpoint{4.558536in}{2.668008in}}%
\pgfpathlineto{\pgfqpoint{4.571585in}{2.664228in}}%
\pgfpathlineto{\pgfqpoint{4.584641in}{2.660574in}}%
\pgfpathlineto{\pgfqpoint{4.591959in}{2.671110in}}%
\pgfpathlineto{\pgfqpoint{4.599274in}{2.681678in}}%
\pgfpathlineto{\pgfqpoint{4.606585in}{2.692280in}}%
\pgfpathlineto{\pgfqpoint{4.613892in}{2.702917in}}%
\pgfpathlineto{\pgfqpoint{4.600846in}{2.706672in}}%
\pgfpathlineto{\pgfqpoint{4.587808in}{2.710552in}}%
\pgfpathlineto{\pgfqpoint{4.574776in}{2.714558in}}%
\pgfpathlineto{\pgfqpoint{4.561751in}{2.718691in}}%
\pgfpathlineto{\pgfqpoint{4.554434in}{2.707948in}}%
\pgfpathlineto{\pgfqpoint{4.547113in}{2.697244in}}%
\pgfpathlineto{\pgfqpoint{4.539788in}{2.686578in}}%
\pgfpathlineto{\pgfqpoint{4.532459in}{2.675947in}}%
\pgfpathclose%
\pgfusepath{fill}%
\end{pgfscope}%
\begin{pgfscope}%
\pgfpathrectangle{\pgfqpoint{1.254980in}{0.150000in}}{\pgfqpoint{5.490039in}{5.490039in}}%
\pgfusepath{clip}%
\pgfsetbuttcap%
\pgfsetroundjoin%
\definecolor{currentfill}{rgb}{0.265145,0.232956,0.516599}%
\pgfsetfillcolor{currentfill}%
\pgfsetfillopacity{0.700000}%
\pgfsetlinewidth{0.000000pt}%
\definecolor{currentstroke}{rgb}{0.000000,0.000000,0.000000}%
\pgfsetstrokecolor{currentstroke}%
\pgfsetdash{}{0pt}%
\pgfpathmoveto{\pgfqpoint{3.784026in}{2.726872in}}%
\pgfpathlineto{\pgfqpoint{3.796936in}{2.717070in}}%
\pgfpathlineto{\pgfqpoint{3.809849in}{2.707418in}}%
\pgfpathlineto{\pgfqpoint{3.822763in}{2.697915in}}%
\pgfpathlineto{\pgfqpoint{3.835679in}{2.688561in}}%
\pgfpathlineto{\pgfqpoint{3.843224in}{2.698583in}}%
\pgfpathlineto{\pgfqpoint{3.850764in}{2.708663in}}%
\pgfpathlineto{\pgfqpoint{3.858298in}{2.718800in}}%
\pgfpathlineto{\pgfqpoint{3.865828in}{2.728996in}}%
\pgfpathlineto{\pgfqpoint{3.852925in}{2.738355in}}%
\pgfpathlineto{\pgfqpoint{3.840023in}{2.747862in}}%
\pgfpathlineto{\pgfqpoint{3.827124in}{2.757519in}}%
\pgfpathlineto{\pgfqpoint{3.814226in}{2.767325in}}%
\pgfpathlineto{\pgfqpoint{3.806683in}{2.757119in}}%
\pgfpathlineto{\pgfqpoint{3.799136in}{2.746975in}}%
\pgfpathlineto{\pgfqpoint{3.791583in}{2.736893in}}%
\pgfpathlineto{\pgfqpoint{3.784026in}{2.726872in}}%
\pgfpathclose%
\pgfusepath{fill}%
\end{pgfscope}%
\begin{pgfscope}%
\pgfpathrectangle{\pgfqpoint{1.254980in}{0.150000in}}{\pgfqpoint{5.490039in}{5.490039in}}%
\pgfusepath{clip}%
\pgfsetbuttcap%
\pgfsetroundjoin%
\definecolor{currentfill}{rgb}{0.258965,0.251537,0.524736}%
\pgfsetfillcolor{currentfill}%
\pgfsetfillopacity{0.700000}%
\pgfsetlinewidth{0.000000pt}%
\definecolor{currentstroke}{rgb}{0.000000,0.000000,0.000000}%
\pgfsetstrokecolor{currentstroke}%
\pgfsetdash{}{0pt}%
\pgfpathmoveto{\pgfqpoint{4.829080in}{2.751068in}}%
\pgfpathlineto{\pgfqpoint{4.842204in}{2.748667in}}%
\pgfpathlineto{\pgfqpoint{4.855336in}{2.746387in}}%
\pgfpathlineto{\pgfqpoint{4.868476in}{2.744227in}}%
\pgfpathlineto{\pgfqpoint{4.881625in}{2.742188in}}%
\pgfpathlineto{\pgfqpoint{4.888855in}{2.752633in}}%
\pgfpathlineto{\pgfqpoint{4.896081in}{2.763112in}}%
\pgfpathlineto{\pgfqpoint{4.903303in}{2.773625in}}%
\pgfpathlineto{\pgfqpoint{4.910522in}{2.784175in}}%
\pgfpathlineto{\pgfqpoint{4.897385in}{2.786362in}}%
\pgfpathlineto{\pgfqpoint{4.884255in}{2.788669in}}%
\pgfpathlineto{\pgfqpoint{4.871134in}{2.791096in}}%
\pgfpathlineto{\pgfqpoint{4.858022in}{2.793645in}}%
\pgfpathlineto{\pgfqpoint{4.850792in}{2.782942in}}%
\pgfpathlineto{\pgfqpoint{4.843558in}{2.772279in}}%
\pgfpathlineto{\pgfqpoint{4.836321in}{2.761655in}}%
\pgfpathlineto{\pgfqpoint{4.829080in}{2.751068in}}%
\pgfpathclose%
\pgfusepath{fill}%
\end{pgfscope}%
\begin{pgfscope}%
\pgfpathrectangle{\pgfqpoint{1.254980in}{0.150000in}}{\pgfqpoint{5.490039in}{5.490039in}}%
\pgfusepath{clip}%
\pgfsetbuttcap%
\pgfsetroundjoin%
\definecolor{currentfill}{rgb}{0.144759,0.519093,0.556572}%
\pgfsetfillcolor{currentfill}%
\pgfsetfillopacity{0.700000}%
\pgfsetlinewidth{0.000000pt}%
\definecolor{currentstroke}{rgb}{0.000000,0.000000,0.000000}%
\pgfsetstrokecolor{currentstroke}%
\pgfsetdash{}{0pt}%
\pgfpathmoveto{\pgfqpoint{3.080875in}{3.425237in}}%
\pgfpathlineto{\pgfqpoint{3.093894in}{3.406390in}}%
\pgfpathlineto{\pgfqpoint{3.106906in}{3.387745in}}%
\pgfpathlineto{\pgfqpoint{3.119913in}{3.369301in}}%
\pgfpathlineto{\pgfqpoint{3.132915in}{3.351058in}}%
\pgfpathlineto{\pgfqpoint{3.140681in}{3.360609in}}%
\pgfpathlineto{\pgfqpoint{3.148440in}{3.370268in}}%
\pgfpathlineto{\pgfqpoint{3.156192in}{3.380036in}}%
\pgfpathlineto{\pgfqpoint{3.163937in}{3.389914in}}%
\pgfpathlineto{\pgfqpoint{3.150953in}{3.408143in}}%
\pgfpathlineto{\pgfqpoint{3.137964in}{3.426571in}}%
\pgfpathlineto{\pgfqpoint{3.124970in}{3.445201in}}%
\pgfpathlineto{\pgfqpoint{3.111970in}{3.464033in}}%
\pgfpathlineto{\pgfqpoint{3.104207in}{3.454165in}}%
\pgfpathlineto{\pgfqpoint{3.096437in}{3.444409in}}%
\pgfpathlineto{\pgfqpoint{3.088660in}{3.434767in}}%
\pgfpathlineto{\pgfqpoint{3.080875in}{3.425237in}}%
\pgfpathclose%
\pgfusepath{fill}%
\end{pgfscope}%
\begin{pgfscope}%
\pgfpathrectangle{\pgfqpoint{1.254980in}{0.150000in}}{\pgfqpoint{5.490039in}{5.490039in}}%
\pgfusepath{clip}%
\pgfsetbuttcap%
\pgfsetroundjoin%
\definecolor{currentfill}{rgb}{0.250425,0.274290,0.533103}%
\pgfsetfillcolor{currentfill}%
\pgfsetfillopacity{0.700000}%
\pgfsetlinewidth{0.000000pt}%
\definecolor{currentstroke}{rgb}{0.000000,0.000000,0.000000}%
\pgfsetstrokecolor{currentstroke}%
\pgfsetdash{}{0pt}%
\pgfpathmoveto{\pgfqpoint{3.598752in}{2.817109in}}%
\pgfpathlineto{\pgfqpoint{3.611665in}{2.805440in}}%
\pgfpathlineto{\pgfqpoint{3.624577in}{2.793929in}}%
\pgfpathlineto{\pgfqpoint{3.637490in}{2.782577in}}%
\pgfpathlineto{\pgfqpoint{3.650404in}{2.771382in}}%
\pgfpathlineto{\pgfqpoint{3.658009in}{2.781125in}}%
\pgfpathlineto{\pgfqpoint{3.665609in}{2.790937in}}%
\pgfpathlineto{\pgfqpoint{3.673204in}{2.800818in}}%
\pgfpathlineto{\pgfqpoint{3.680793in}{2.810767in}}%
\pgfpathlineto{\pgfqpoint{3.667893in}{2.821950in}}%
\pgfpathlineto{\pgfqpoint{3.654994in}{2.833291in}}%
\pgfpathlineto{\pgfqpoint{3.642096in}{2.844789in}}%
\pgfpathlineto{\pgfqpoint{3.629198in}{2.856447in}}%
\pgfpathlineto{\pgfqpoint{3.621594in}{2.846504in}}%
\pgfpathlineto{\pgfqpoint{3.613986in}{2.836633in}}%
\pgfpathlineto{\pgfqpoint{3.606372in}{2.826835in}}%
\pgfpathlineto{\pgfqpoint{3.598752in}{2.817109in}}%
\pgfpathclose%
\pgfusepath{fill}%
\end{pgfscope}%
\begin{pgfscope}%
\pgfpathrectangle{\pgfqpoint{1.254980in}{0.150000in}}{\pgfqpoint{5.490039in}{5.490039in}}%
\pgfusepath{clip}%
\pgfsetbuttcap%
\pgfsetroundjoin%
\definecolor{currentfill}{rgb}{0.275191,0.194905,0.496005}%
\pgfsetfillcolor{currentfill}%
\pgfsetfillopacity{0.700000}%
\pgfsetlinewidth{0.000000pt}%
\definecolor{currentstroke}{rgb}{0.000000,0.000000,0.000000}%
\pgfsetstrokecolor{currentstroke}%
\pgfsetdash{}{0pt}%
\pgfpathmoveto{\pgfqpoint{4.102554in}{2.639942in}}%
\pgfpathlineto{\pgfqpoint{4.115500in}{2.632988in}}%
\pgfpathlineto{\pgfqpoint{4.128451in}{2.626171in}}%
\pgfpathlineto{\pgfqpoint{4.141405in}{2.619490in}}%
\pgfpathlineto{\pgfqpoint{4.154364in}{2.612947in}}%
\pgfpathlineto{\pgfqpoint{4.161813in}{2.623286in}}%
\pgfpathlineto{\pgfqpoint{4.169257in}{2.633667in}}%
\pgfpathlineto{\pgfqpoint{4.176697in}{2.644091in}}%
\pgfpathlineto{\pgfqpoint{4.184132in}{2.654557in}}%
\pgfpathlineto{\pgfqpoint{4.171184in}{2.661137in}}%
\pgfpathlineto{\pgfqpoint{4.158241in}{2.667855in}}%
\pgfpathlineto{\pgfqpoint{4.145302in}{2.674708in}}%
\pgfpathlineto{\pgfqpoint{4.132367in}{2.681700in}}%
\pgfpathlineto{\pgfqpoint{4.124920in}{2.671191in}}%
\pgfpathlineto{\pgfqpoint{4.117469in}{2.660729in}}%
\pgfpathlineto{\pgfqpoint{4.110014in}{2.650313in}}%
\pgfpathlineto{\pgfqpoint{4.102554in}{2.639942in}}%
\pgfpathclose%
\pgfusepath{fill}%
\end{pgfscope}%
\begin{pgfscope}%
\pgfpathrectangle{\pgfqpoint{1.254980in}{0.150000in}}{\pgfqpoint{5.490039in}{5.490039in}}%
\pgfusepath{clip}%
\pgfsetbuttcap%
\pgfsetroundjoin%
\definecolor{currentfill}{rgb}{0.273006,0.204520,0.501721}%
\pgfsetfillcolor{currentfill}%
\pgfsetfillopacity{0.700000}%
\pgfsetlinewidth{0.000000pt}%
\definecolor{currentstroke}{rgb}{0.000000,0.000000,0.000000}%
\pgfsetstrokecolor{currentstroke}%
\pgfsetdash{}{0pt}%
\pgfpathmoveto{\pgfqpoint{3.969148in}{2.659385in}}%
\pgfpathlineto{\pgfqpoint{3.982076in}{2.651332in}}%
\pgfpathlineto{\pgfqpoint{3.995008in}{2.643421in}}%
\pgfpathlineto{\pgfqpoint{4.007942in}{2.635651in}}%
\pgfpathlineto{\pgfqpoint{4.020880in}{2.628022in}}%
\pgfpathlineto{\pgfqpoint{4.028370in}{2.638232in}}%
\pgfpathlineto{\pgfqpoint{4.035854in}{2.648488in}}%
\pgfpathlineto{\pgfqpoint{4.043335in}{2.658793in}}%
\pgfpathlineto{\pgfqpoint{4.050810in}{2.669145in}}%
\pgfpathlineto{\pgfqpoint{4.037884in}{2.676795in}}%
\pgfpathlineto{\pgfqpoint{4.024960in}{2.684586in}}%
\pgfpathlineto{\pgfqpoint{4.012041in}{2.692518in}}%
\pgfpathlineto{\pgfqpoint{3.999124in}{2.700592in}}%
\pgfpathlineto{\pgfqpoint{3.991637in}{2.690213in}}%
\pgfpathlineto{\pgfqpoint{3.984146in}{2.679885in}}%
\pgfpathlineto{\pgfqpoint{3.976649in}{2.669610in}}%
\pgfpathlineto{\pgfqpoint{3.969148in}{2.659385in}}%
\pgfpathclose%
\pgfusepath{fill}%
\end{pgfscope}%
\begin{pgfscope}%
\pgfpathrectangle{\pgfqpoint{1.254980in}{0.150000in}}{\pgfqpoint{5.490039in}{5.490039in}}%
\pgfusepath{clip}%
\pgfsetbuttcap%
\pgfsetroundjoin%
\definecolor{currentfill}{rgb}{0.273006,0.204520,0.501721}%
\pgfsetfillcolor{currentfill}%
\pgfsetfillopacity{0.700000}%
\pgfsetlinewidth{0.000000pt}%
\definecolor{currentstroke}{rgb}{0.000000,0.000000,0.000000}%
\pgfsetstrokecolor{currentstroke}%
\pgfsetdash{}{0pt}%
\pgfpathmoveto{\pgfqpoint{4.450987in}{2.650832in}}%
\pgfpathlineto{\pgfqpoint{4.464006in}{2.646374in}}%
\pgfpathlineto{\pgfqpoint{4.477032in}{2.642044in}}%
\pgfpathlineto{\pgfqpoint{4.490064in}{2.637842in}}%
\pgfpathlineto{\pgfqpoint{4.503103in}{2.633767in}}%
\pgfpathlineto{\pgfqpoint{4.510448in}{2.644263in}}%
\pgfpathlineto{\pgfqpoint{4.517789in}{2.654791in}}%
\pgfpathlineto{\pgfqpoint{4.525126in}{2.665352in}}%
\pgfpathlineto{\pgfqpoint{4.532459in}{2.675947in}}%
\pgfpathlineto{\pgfqpoint{4.519431in}{2.680107in}}%
\pgfpathlineto{\pgfqpoint{4.506409in}{2.684394in}}%
\pgfpathlineto{\pgfqpoint{4.493393in}{2.688808in}}%
\pgfpathlineto{\pgfqpoint{4.480384in}{2.693350in}}%
\pgfpathlineto{\pgfqpoint{4.473041in}{2.682665in}}%
\pgfpathlineto{\pgfqpoint{4.465694in}{2.672017in}}%
\pgfpathlineto{\pgfqpoint{4.458342in}{2.661407in}}%
\pgfpathlineto{\pgfqpoint{4.450987in}{2.650832in}}%
\pgfpathclose%
\pgfusepath{fill}%
\end{pgfscope}%
\begin{pgfscope}%
\pgfpathrectangle{\pgfqpoint{1.254980in}{0.150000in}}{\pgfqpoint{5.490039in}{5.490039in}}%
\pgfusepath{clip}%
\pgfsetbuttcap%
\pgfsetroundjoin%
\definecolor{currentfill}{rgb}{0.263663,0.237631,0.518762}%
\pgfsetfillcolor{currentfill}%
\pgfsetfillopacity{0.700000}%
\pgfsetlinewidth{0.000000pt}%
\definecolor{currentstroke}{rgb}{0.000000,0.000000,0.000000}%
\pgfsetstrokecolor{currentstroke}%
\pgfsetdash{}{0pt}%
\pgfpathmoveto{\pgfqpoint{4.747623in}{2.719354in}}%
\pgfpathlineto{\pgfqpoint{4.760725in}{2.716598in}}%
\pgfpathlineto{\pgfqpoint{4.773835in}{2.713964in}}%
\pgfpathlineto{\pgfqpoint{4.786953in}{2.711453in}}%
\pgfpathlineto{\pgfqpoint{4.800080in}{2.709062in}}%
\pgfpathlineto{\pgfqpoint{4.807336in}{2.719516in}}%
\pgfpathlineto{\pgfqpoint{4.814588in}{2.730000in}}%
\pgfpathlineto{\pgfqpoint{4.821836in}{2.740517in}}%
\pgfpathlineto{\pgfqpoint{4.829080in}{2.751068in}}%
\pgfpathlineto{\pgfqpoint{4.815965in}{2.753590in}}%
\pgfpathlineto{\pgfqpoint{4.802858in}{2.756234in}}%
\pgfpathlineto{\pgfqpoint{4.789758in}{2.758999in}}%
\pgfpathlineto{\pgfqpoint{4.776667in}{2.761887in}}%
\pgfpathlineto{\pgfqpoint{4.769412in}{2.751198in}}%
\pgfpathlineto{\pgfqpoint{4.762152in}{2.740547in}}%
\pgfpathlineto{\pgfqpoint{4.754890in}{2.729933in}}%
\pgfpathlineto{\pgfqpoint{4.747623in}{2.719354in}}%
\pgfpathclose%
\pgfusepath{fill}%
\end{pgfscope}%
\begin{pgfscope}%
\pgfpathrectangle{\pgfqpoint{1.254980in}{0.150000in}}{\pgfqpoint{5.490039in}{5.490039in}}%
\pgfusepath{clip}%
\pgfsetbuttcap%
\pgfsetroundjoin%
\definecolor{currentfill}{rgb}{0.276194,0.190074,0.493001}%
\pgfsetfillcolor{currentfill}%
\pgfsetfillopacity{0.700000}%
\pgfsetlinewidth{0.000000pt}%
\definecolor{currentstroke}{rgb}{0.000000,0.000000,0.000000}%
\pgfsetstrokecolor{currentstroke}%
\pgfsetdash{}{0pt}%
\pgfpathmoveto{\pgfqpoint{4.235971in}{2.629587in}}%
\pgfpathlineto{\pgfqpoint{4.248943in}{2.623680in}}%
\pgfpathlineto{\pgfqpoint{4.261920in}{2.617907in}}%
\pgfpathlineto{\pgfqpoint{4.274902in}{2.612267in}}%
\pgfpathlineto{\pgfqpoint{4.287890in}{2.606759in}}%
\pgfpathlineto{\pgfqpoint{4.295300in}{2.617175in}}%
\pgfpathlineto{\pgfqpoint{4.302705in}{2.627628in}}%
\pgfpathlineto{\pgfqpoint{4.310106in}{2.638117in}}%
\pgfpathlineto{\pgfqpoint{4.317503in}{2.648645in}}%
\pgfpathlineto{\pgfqpoint{4.304526in}{2.654206in}}%
\pgfpathlineto{\pgfqpoint{4.291555in}{2.659899in}}%
\pgfpathlineto{\pgfqpoint{4.278588in}{2.665725in}}%
\pgfpathlineto{\pgfqpoint{4.265627in}{2.671684in}}%
\pgfpathlineto{\pgfqpoint{4.258219in}{2.661098in}}%
\pgfpathlineto{\pgfqpoint{4.250808in}{2.650553in}}%
\pgfpathlineto{\pgfqpoint{4.243392in}{2.640050in}}%
\pgfpathlineto{\pgfqpoint{4.235971in}{2.629587in}}%
\pgfpathclose%
\pgfusepath{fill}%
\end{pgfscope}%
\begin{pgfscope}%
\pgfpathrectangle{\pgfqpoint{1.254980in}{0.150000in}}{\pgfqpoint{5.490039in}{5.490039in}}%
\pgfusepath{clip}%
\pgfsetbuttcap%
\pgfsetroundjoin%
\definecolor{currentfill}{rgb}{0.132444,0.552216,0.553018}%
\pgfsetfillcolor{currentfill}%
\pgfsetfillopacity{0.700000}%
\pgfsetlinewidth{0.000000pt}%
\definecolor{currentstroke}{rgb}{0.000000,0.000000,0.000000}%
\pgfsetstrokecolor{currentstroke}%
\pgfsetdash{}{0pt}%
\pgfpathmoveto{\pgfqpoint{3.028742in}{3.502680in}}%
\pgfpathlineto{\pgfqpoint{3.041785in}{3.483008in}}%
\pgfpathlineto{\pgfqpoint{3.054821in}{3.463545in}}%
\pgfpathlineto{\pgfqpoint{3.067851in}{3.444289in}}%
\pgfpathlineto{\pgfqpoint{3.080875in}{3.425237in}}%
\pgfpathlineto{\pgfqpoint{3.088660in}{3.434767in}}%
\pgfpathlineto{\pgfqpoint{3.096437in}{3.444409in}}%
\pgfpathlineto{\pgfqpoint{3.104207in}{3.454165in}}%
\pgfpathlineto{\pgfqpoint{3.111970in}{3.464033in}}%
\pgfpathlineto{\pgfqpoint{3.098965in}{3.483069in}}%
\pgfpathlineto{\pgfqpoint{3.085953in}{3.502310in}}%
\pgfpathlineto{\pgfqpoint{3.072936in}{3.521757in}}%
\pgfpathlineto{\pgfqpoint{3.059912in}{3.541413in}}%
\pgfpathlineto{\pgfqpoint{3.052130in}{3.531554in}}%
\pgfpathlineto{\pgfqpoint{3.044341in}{3.521812in}}%
\pgfpathlineto{\pgfqpoint{3.036545in}{3.512188in}}%
\pgfpathlineto{\pgfqpoint{3.028742in}{3.502680in}}%
\pgfpathclose%
\pgfusepath{fill}%
\end{pgfscope}%
\begin{pgfscope}%
\pgfpathrectangle{\pgfqpoint{1.254980in}{0.150000in}}{\pgfqpoint{5.490039in}{5.490039in}}%
\pgfusepath{clip}%
\pgfsetbuttcap%
\pgfsetroundjoin%
\definecolor{currentfill}{rgb}{0.269308,0.218818,0.509577}%
\pgfsetfillcolor{currentfill}%
\pgfsetfillopacity{0.700000}%
\pgfsetlinewidth{0.000000pt}%
\definecolor{currentstroke}{rgb}{0.000000,0.000000,0.000000}%
\pgfsetstrokecolor{currentstroke}%
\pgfsetdash{}{0pt}%
\pgfpathmoveto{\pgfqpoint{3.835679in}{2.688561in}}%
\pgfpathlineto{\pgfqpoint{3.848598in}{2.679354in}}%
\pgfpathlineto{\pgfqpoint{3.861518in}{2.670295in}}%
\pgfpathlineto{\pgfqpoint{3.874441in}{2.661382in}}%
\pgfpathlineto{\pgfqpoint{3.887367in}{2.652615in}}%
\pgfpathlineto{\pgfqpoint{3.894899in}{2.662638in}}%
\pgfpathlineto{\pgfqpoint{3.902427in}{2.672714in}}%
\pgfpathlineto{\pgfqpoint{3.909949in}{2.682845in}}%
\pgfpathlineto{\pgfqpoint{3.917467in}{2.693030in}}%
\pgfpathlineto{\pgfqpoint{3.904553in}{2.701802in}}%
\pgfpathlineto{\pgfqpoint{3.891643in}{2.710720in}}%
\pgfpathlineto{\pgfqpoint{3.878734in}{2.719784in}}%
\pgfpathlineto{\pgfqpoint{3.865828in}{2.728996in}}%
\pgfpathlineto{\pgfqpoint{3.858298in}{2.718800in}}%
\pgfpathlineto{\pgfqpoint{3.850764in}{2.708663in}}%
\pgfpathlineto{\pgfqpoint{3.843224in}{2.698583in}}%
\pgfpathlineto{\pgfqpoint{3.835679in}{2.688561in}}%
\pgfpathclose%
\pgfusepath{fill}%
\end{pgfscope}%
\begin{pgfscope}%
\pgfpathrectangle{\pgfqpoint{1.254980in}{0.150000in}}{\pgfqpoint{5.490039in}{5.490039in}}%
\pgfusepath{clip}%
\pgfsetbuttcap%
\pgfsetroundjoin%
\definecolor{currentfill}{rgb}{0.221989,0.339161,0.548752}%
\pgfsetfillcolor{currentfill}%
\pgfsetfillopacity{0.700000}%
\pgfsetlinewidth{0.000000pt}%
\definecolor{currentstroke}{rgb}{0.000000,0.000000,0.000000}%
\pgfsetstrokecolor{currentstroke}%
\pgfsetdash{}{0pt}%
\pgfpathmoveto{\pgfqpoint{5.289267in}{2.925541in}}%
\pgfpathlineto{\pgfqpoint{5.302553in}{2.925067in}}%
\pgfpathlineto{\pgfqpoint{5.315849in}{2.924707in}}%
\pgfpathlineto{\pgfqpoint{5.329155in}{2.924461in}}%
\pgfpathlineto{\pgfqpoint{5.342471in}{2.924330in}}%
\pgfpathlineto{\pgfqpoint{5.349559in}{2.934463in}}%
\pgfpathlineto{\pgfqpoint{5.356643in}{2.944646in}}%
\pgfpathlineto{\pgfqpoint{5.363725in}{2.954880in}}%
\pgfpathlineto{\pgfqpoint{5.370804in}{2.965167in}}%
\pgfpathlineto{\pgfqpoint{5.357501in}{2.965526in}}%
\pgfpathlineto{\pgfqpoint{5.344209in}{2.965999in}}%
\pgfpathlineto{\pgfqpoint{5.330927in}{2.966586in}}%
\pgfpathlineto{\pgfqpoint{5.317655in}{2.967287in}}%
\pgfpathlineto{\pgfqpoint{5.310563in}{2.956767in}}%
\pgfpathlineto{\pgfqpoint{5.303467in}{2.946304in}}%
\pgfpathlineto{\pgfqpoint{5.296369in}{2.935897in}}%
\pgfpathlineto{\pgfqpoint{5.289267in}{2.925541in}}%
\pgfpathclose%
\pgfusepath{fill}%
\end{pgfscope}%
\begin{pgfscope}%
\pgfpathrectangle{\pgfqpoint{1.254980in}{0.150000in}}{\pgfqpoint{5.490039in}{5.490039in}}%
\pgfusepath{clip}%
\pgfsetbuttcap%
\pgfsetroundjoin%
\definecolor{currentfill}{rgb}{0.257322,0.256130,0.526563}%
\pgfsetfillcolor{currentfill}%
\pgfsetfillopacity{0.700000}%
\pgfsetlinewidth{0.000000pt}%
\definecolor{currentstroke}{rgb}{0.000000,0.000000,0.000000}%
\pgfsetstrokecolor{currentstroke}%
\pgfsetdash{}{0pt}%
\pgfpathmoveto{\pgfqpoint{3.650404in}{2.771382in}}%
\pgfpathlineto{\pgfqpoint{3.663318in}{2.760343in}}%
\pgfpathlineto{\pgfqpoint{3.676233in}{2.749461in}}%
\pgfpathlineto{\pgfqpoint{3.689149in}{2.738733in}}%
\pgfpathlineto{\pgfqpoint{3.702065in}{2.728160in}}%
\pgfpathlineto{\pgfqpoint{3.709657in}{2.737921in}}%
\pgfpathlineto{\pgfqpoint{3.717243in}{2.747747in}}%
\pgfpathlineto{\pgfqpoint{3.724824in}{2.757637in}}%
\pgfpathlineto{\pgfqpoint{3.732400in}{2.767591in}}%
\pgfpathlineto{\pgfqpoint{3.719497in}{2.778153in}}%
\pgfpathlineto{\pgfqpoint{3.706594in}{2.788869in}}%
\pgfpathlineto{\pgfqpoint{3.693693in}{2.799740in}}%
\pgfpathlineto{\pgfqpoint{3.680793in}{2.810767in}}%
\pgfpathlineto{\pgfqpoint{3.673204in}{2.800818in}}%
\pgfpathlineto{\pgfqpoint{3.665609in}{2.790937in}}%
\pgfpathlineto{\pgfqpoint{3.658009in}{2.781125in}}%
\pgfpathlineto{\pgfqpoint{3.650404in}{2.771382in}}%
\pgfpathclose%
\pgfusepath{fill}%
\end{pgfscope}%
\begin{pgfscope}%
\pgfpathrectangle{\pgfqpoint{1.254980in}{0.150000in}}{\pgfqpoint{5.490039in}{5.490039in}}%
\pgfusepath{clip}%
\pgfsetbuttcap%
\pgfsetroundjoin%
\definecolor{currentfill}{rgb}{0.216210,0.351535,0.550627}%
\pgfsetfillcolor{currentfill}%
\pgfsetfillopacity{0.700000}%
\pgfsetlinewidth{0.000000pt}%
\definecolor{currentstroke}{rgb}{0.000000,0.000000,0.000000}%
\pgfsetstrokecolor{currentstroke}%
\pgfsetdash{}{0pt}%
\pgfpathmoveto{\pgfqpoint{5.370804in}{2.965167in}}%
\pgfpathlineto{\pgfqpoint{5.384116in}{2.964922in}}%
\pgfpathlineto{\pgfqpoint{5.397440in}{2.964790in}}%
\pgfpathlineto{\pgfqpoint{5.410774in}{2.964772in}}%
\pgfpathlineto{\pgfqpoint{5.424119in}{2.964867in}}%
\pgfpathlineto{\pgfqpoint{5.431180in}{2.974972in}}%
\pgfpathlineto{\pgfqpoint{5.438239in}{2.985132in}}%
\pgfpathlineto{\pgfqpoint{5.445295in}{2.995349in}}%
\pgfpathlineto{\pgfqpoint{5.431962in}{2.995435in}}%
\pgfpathlineto{\pgfqpoint{5.418639in}{2.995635in}}%
\pgfpathlineto{\pgfqpoint{5.405326in}{2.995947in}}%
\pgfpathlineto{\pgfqpoint{5.392024in}{2.996373in}}%
\pgfpathlineto{\pgfqpoint{5.384953in}{2.985911in}}%
\pgfpathlineto{\pgfqpoint{5.377880in}{2.975510in}}%
\pgfpathlineto{\pgfqpoint{5.370804in}{2.965167in}}%
\pgfpathclose%
\pgfusepath{fill}%
\end{pgfscope}%
\begin{pgfscope}%
\pgfpathrectangle{\pgfqpoint{1.254980in}{0.150000in}}{\pgfqpoint{5.490039in}{5.490039in}}%
\pgfusepath{clip}%
\pgfsetbuttcap%
\pgfsetroundjoin%
\definecolor{currentfill}{rgb}{0.229739,0.322361,0.545706}%
\pgfsetfillcolor{currentfill}%
\pgfsetfillopacity{0.700000}%
\pgfsetlinewidth{0.000000pt}%
\definecolor{currentstroke}{rgb}{0.000000,0.000000,0.000000}%
\pgfsetstrokecolor{currentstroke}%
\pgfsetdash{}{0pt}%
\pgfpathmoveto{\pgfqpoint{5.207738in}{2.886805in}}%
\pgfpathlineto{\pgfqpoint{5.220996in}{2.886081in}}%
\pgfpathlineto{\pgfqpoint{5.234265in}{2.885474in}}%
\pgfpathlineto{\pgfqpoint{5.247543in}{2.884981in}}%
\pgfpathlineto{\pgfqpoint{5.260832in}{2.884603in}}%
\pgfpathlineto{\pgfqpoint{5.267946in}{2.894770in}}%
\pgfpathlineto{\pgfqpoint{5.275056in}{2.904980in}}%
\pgfpathlineto{\pgfqpoint{5.282163in}{2.915237in}}%
\pgfpathlineto{\pgfqpoint{5.289267in}{2.925541in}}%
\pgfpathlineto{\pgfqpoint{5.275992in}{2.926131in}}%
\pgfpathlineto{\pgfqpoint{5.262727in}{2.926835in}}%
\pgfpathlineto{\pgfqpoint{5.249472in}{2.927654in}}%
\pgfpathlineto{\pgfqpoint{5.236226in}{2.928589in}}%
\pgfpathlineto{\pgfqpoint{5.229109in}{2.918067in}}%
\pgfpathlineto{\pgfqpoint{5.221988in}{2.907597in}}%
\pgfpathlineto{\pgfqpoint{5.214865in}{2.897177in}}%
\pgfpathlineto{\pgfqpoint{5.207738in}{2.886805in}}%
\pgfpathclose%
\pgfusepath{fill}%
\end{pgfscope}%
\begin{pgfscope}%
\pgfpathrectangle{\pgfqpoint{1.254980in}{0.150000in}}{\pgfqpoint{5.490039in}{5.490039in}}%
\pgfusepath{clip}%
\pgfsetbuttcap%
\pgfsetroundjoin%
\definecolor{currentfill}{rgb}{0.266580,0.228262,0.514349}%
\pgfsetfillcolor{currentfill}%
\pgfsetfillopacity{0.700000}%
\pgfsetlinewidth{0.000000pt}%
\definecolor{currentstroke}{rgb}{0.000000,0.000000,0.000000}%
\pgfsetstrokecolor{currentstroke}%
\pgfsetdash{}{0pt}%
\pgfpathmoveto{\pgfqpoint{4.666145in}{2.689148in}}%
\pgfpathlineto{\pgfqpoint{4.679227in}{2.686016in}}%
\pgfpathlineto{\pgfqpoint{4.692317in}{2.683008in}}%
\pgfpathlineto{\pgfqpoint{4.705414in}{2.680123in}}%
\pgfpathlineto{\pgfqpoint{4.718518in}{2.677360in}}%
\pgfpathlineto{\pgfqpoint{4.725800in}{2.687813in}}%
\pgfpathlineto{\pgfqpoint{4.733079in}{2.698295in}}%
\pgfpathlineto{\pgfqpoint{4.740353in}{2.708808in}}%
\pgfpathlineto{\pgfqpoint{4.747623in}{2.719354in}}%
\pgfpathlineto{\pgfqpoint{4.734529in}{2.722232in}}%
\pgfpathlineto{\pgfqpoint{4.721442in}{2.725233in}}%
\pgfpathlineto{\pgfqpoint{4.708363in}{2.728358in}}%
\pgfpathlineto{\pgfqpoint{4.695292in}{2.731605in}}%
\pgfpathlineto{\pgfqpoint{4.688011in}{2.720938in}}%
\pgfpathlineto{\pgfqpoint{4.680726in}{2.710307in}}%
\pgfpathlineto{\pgfqpoint{4.673438in}{2.699711in}}%
\pgfpathlineto{\pgfqpoint{4.666145in}{2.689148in}}%
\pgfpathclose%
\pgfusepath{fill}%
\end{pgfscope}%
\begin{pgfscope}%
\pgfpathrectangle{\pgfqpoint{1.254980in}{0.150000in}}{\pgfqpoint{5.490039in}{5.490039in}}%
\pgfusepath{clip}%
\pgfsetbuttcap%
\pgfsetroundjoin%
\definecolor{currentfill}{rgb}{0.237441,0.305202,0.541921}%
\pgfsetfillcolor{currentfill}%
\pgfsetfillopacity{0.700000}%
\pgfsetlinewidth{0.000000pt}%
\definecolor{currentstroke}{rgb}{0.000000,0.000000,0.000000}%
\pgfsetstrokecolor{currentstroke}%
\pgfsetdash{}{0pt}%
\pgfpathmoveto{\pgfqpoint{5.126212in}{2.849022in}}%
\pgfpathlineto{\pgfqpoint{5.139444in}{2.848030in}}%
\pgfpathlineto{\pgfqpoint{5.152686in}{2.847154in}}%
\pgfpathlineto{\pgfqpoint{5.165938in}{2.846395in}}%
\pgfpathlineto{\pgfqpoint{5.179199in}{2.845751in}}%
\pgfpathlineto{\pgfqpoint{5.186339in}{2.855953in}}%
\pgfpathlineto{\pgfqpoint{5.193475in}{2.866195in}}%
\pgfpathlineto{\pgfqpoint{5.200608in}{2.876478in}}%
\pgfpathlineto{\pgfqpoint{5.207738in}{2.886805in}}%
\pgfpathlineto{\pgfqpoint{5.194489in}{2.887644in}}%
\pgfpathlineto{\pgfqpoint{5.181251in}{2.888599in}}%
\pgfpathlineto{\pgfqpoint{5.168021in}{2.889670in}}%
\pgfpathlineto{\pgfqpoint{5.154802in}{2.890858in}}%
\pgfpathlineto{\pgfqpoint{5.147659in}{2.880330in}}%
\pgfpathlineto{\pgfqpoint{5.140514in}{2.869849in}}%
\pgfpathlineto{\pgfqpoint{5.133365in}{2.859414in}}%
\pgfpathlineto{\pgfqpoint{5.126212in}{2.849022in}}%
\pgfpathclose%
\pgfusepath{fill}%
\end{pgfscope}%
\begin{pgfscope}%
\pgfpathrectangle{\pgfqpoint{1.254980in}{0.150000in}}{\pgfqpoint{5.490039in}{5.490039in}}%
\pgfusepath{clip}%
\pgfsetbuttcap%
\pgfsetroundjoin%
\definecolor{currentfill}{rgb}{0.275191,0.194905,0.496005}%
\pgfsetfillcolor{currentfill}%
\pgfsetfillopacity{0.700000}%
\pgfsetlinewidth{0.000000pt}%
\definecolor{currentstroke}{rgb}{0.000000,0.000000,0.000000}%
\pgfsetstrokecolor{currentstroke}%
\pgfsetdash{}{0pt}%
\pgfpathmoveto{\pgfqpoint{4.369467in}{2.627717in}}%
\pgfpathlineto{\pgfqpoint{4.382472in}{2.622811in}}%
\pgfpathlineto{\pgfqpoint{4.395483in}{2.618035in}}%
\pgfpathlineto{\pgfqpoint{4.408500in}{2.613389in}}%
\pgfpathlineto{\pgfqpoint{4.421523in}{2.608871in}}%
\pgfpathlineto{\pgfqpoint{4.428896in}{2.619312in}}%
\pgfpathlineto{\pgfqpoint{4.436263in}{2.629786in}}%
\pgfpathlineto{\pgfqpoint{4.443627in}{2.640292in}}%
\pgfpathlineto{\pgfqpoint{4.450987in}{2.650832in}}%
\pgfpathlineto{\pgfqpoint{4.437974in}{2.655418in}}%
\pgfpathlineto{\pgfqpoint{4.424967in}{2.660134in}}%
\pgfpathlineto{\pgfqpoint{4.411966in}{2.664978in}}%
\pgfpathlineto{\pgfqpoint{4.398971in}{2.669952in}}%
\pgfpathlineto{\pgfqpoint{4.391601in}{2.659338in}}%
\pgfpathlineto{\pgfqpoint{4.384227in}{2.648761in}}%
\pgfpathlineto{\pgfqpoint{4.376849in}{2.638221in}}%
\pgfpathlineto{\pgfqpoint{4.369467in}{2.627717in}}%
\pgfpathclose%
\pgfusepath{fill}%
\end{pgfscope}%
\begin{pgfscope}%
\pgfpathrectangle{\pgfqpoint{1.254980in}{0.150000in}}{\pgfqpoint{5.490039in}{5.490039in}}%
\pgfusepath{clip}%
\pgfsetbuttcap%
\pgfsetroundjoin%
\definecolor{currentfill}{rgb}{0.203063,0.379716,0.553925}%
\pgfsetfillcolor{currentfill}%
\pgfsetfillopacity{0.700000}%
\pgfsetlinewidth{0.000000pt}%
\definecolor{currentstroke}{rgb}{0.000000,0.000000,0.000000}%
\pgfsetstrokecolor{currentstroke}%
\pgfsetdash{}{0pt}%
\pgfpathmoveto{\pgfqpoint{3.309502in}{3.047660in}}%
\pgfpathlineto{\pgfqpoint{3.322454in}{3.032589in}}%
\pgfpathlineto{\pgfqpoint{3.335404in}{3.017697in}}%
\pgfpathlineto{\pgfqpoint{3.348352in}{3.002981in}}%
\pgfpathlineto{\pgfqpoint{3.361297in}{2.988441in}}%
\pgfpathlineto{\pgfqpoint{3.369002in}{2.997742in}}%
\pgfpathlineto{\pgfqpoint{3.376700in}{3.007130in}}%
\pgfpathlineto{\pgfqpoint{3.384393in}{3.016606in}}%
\pgfpathlineto{\pgfqpoint{3.392079in}{3.026170in}}%
\pgfpathlineto{\pgfqpoint{3.379151in}{3.040680in}}%
\pgfpathlineto{\pgfqpoint{3.366220in}{3.055366in}}%
\pgfpathlineto{\pgfqpoint{3.353286in}{3.070230in}}%
\pgfpathlineto{\pgfqpoint{3.340351in}{3.085271in}}%
\pgfpathlineto{\pgfqpoint{3.332648in}{3.075730in}}%
\pgfpathlineto{\pgfqpoint{3.324939in}{3.066282in}}%
\pgfpathlineto{\pgfqpoint{3.317224in}{3.056925in}}%
\pgfpathlineto{\pgfqpoint{3.309502in}{3.047660in}}%
\pgfpathclose%
\pgfusepath{fill}%
\end{pgfscope}%
\begin{pgfscope}%
\pgfpathrectangle{\pgfqpoint{1.254980in}{0.150000in}}{\pgfqpoint{5.490039in}{5.490039in}}%
\pgfusepath{clip}%
\pgfsetbuttcap%
\pgfsetroundjoin%
\definecolor{currentfill}{rgb}{0.192357,0.403199,0.555836}%
\pgfsetfillcolor{currentfill}%
\pgfsetfillopacity{0.700000}%
\pgfsetlinewidth{0.000000pt}%
\definecolor{currentstroke}{rgb}{0.000000,0.000000,0.000000}%
\pgfsetstrokecolor{currentstroke}%
\pgfsetdash{}{0pt}%
\pgfpathmoveto{\pgfqpoint{3.257663in}{3.109745in}}%
\pgfpathlineto{\pgfqpoint{3.270627in}{3.093951in}}%
\pgfpathlineto{\pgfqpoint{3.283588in}{3.078340in}}%
\pgfpathlineto{\pgfqpoint{3.296547in}{3.062910in}}%
\pgfpathlineto{\pgfqpoint{3.309502in}{3.047660in}}%
\pgfpathlineto{\pgfqpoint{3.317224in}{3.056925in}}%
\pgfpathlineto{\pgfqpoint{3.324939in}{3.066282in}}%
\pgfpathlineto{\pgfqpoint{3.332648in}{3.075730in}}%
\pgfpathlineto{\pgfqpoint{3.340351in}{3.085271in}}%
\pgfpathlineto{\pgfqpoint{3.327412in}{3.100491in}}%
\pgfpathlineto{\pgfqpoint{3.314471in}{3.115891in}}%
\pgfpathlineto{\pgfqpoint{3.301527in}{3.131472in}}%
\pgfpathlineto{\pgfqpoint{3.288580in}{3.147236in}}%
\pgfpathlineto{\pgfqpoint{3.280860in}{3.137720in}}%
\pgfpathlineto{\pgfqpoint{3.273134in}{3.128299in}}%
\pgfpathlineto{\pgfqpoint{3.265402in}{3.118974in}}%
\pgfpathlineto{\pgfqpoint{3.257663in}{3.109745in}}%
\pgfpathclose%
\pgfusepath{fill}%
\end{pgfscope}%
\begin{pgfscope}%
\pgfpathrectangle{\pgfqpoint{1.254980in}{0.150000in}}{\pgfqpoint{5.490039in}{5.490039in}}%
\pgfusepath{clip}%
\pgfsetbuttcap%
\pgfsetroundjoin%
\definecolor{currentfill}{rgb}{0.214298,0.355619,0.551184}%
\pgfsetfillcolor{currentfill}%
\pgfsetfillopacity{0.700000}%
\pgfsetlinewidth{0.000000pt}%
\definecolor{currentstroke}{rgb}{0.000000,0.000000,0.000000}%
\pgfsetstrokecolor{currentstroke}%
\pgfsetdash{}{0pt}%
\pgfpathmoveto{\pgfqpoint{3.361297in}{2.988441in}}%
\pgfpathlineto{\pgfqpoint{3.374240in}{2.974077in}}%
\pgfpathlineto{\pgfqpoint{3.387181in}{2.959886in}}%
\pgfpathlineto{\pgfqpoint{3.400120in}{2.945868in}}%
\pgfpathlineto{\pgfqpoint{3.413057in}{2.932021in}}%
\pgfpathlineto{\pgfqpoint{3.420746in}{2.941357in}}%
\pgfpathlineto{\pgfqpoint{3.428428in}{2.950776in}}%
\pgfpathlineto{\pgfqpoint{3.436104in}{2.960279in}}%
\pgfpathlineto{\pgfqpoint{3.443775in}{2.969866in}}%
\pgfpathlineto{\pgfqpoint{3.430853in}{2.983683in}}%
\pgfpathlineto{\pgfqpoint{3.417930in}{2.997672in}}%
\pgfpathlineto{\pgfqpoint{3.405006in}{3.011834in}}%
\pgfpathlineto{\pgfqpoint{3.392079in}{3.026170in}}%
\pgfpathlineto{\pgfqpoint{3.384393in}{3.016606in}}%
\pgfpathlineto{\pgfqpoint{3.376700in}{3.007130in}}%
\pgfpathlineto{\pgfqpoint{3.369002in}{2.997742in}}%
\pgfpathlineto{\pgfqpoint{3.361297in}{2.988441in}}%
\pgfpathclose%
\pgfusepath{fill}%
\end{pgfscope}%
\begin{pgfscope}%
\pgfpathrectangle{\pgfqpoint{1.254980in}{0.150000in}}{\pgfqpoint{5.490039in}{5.490039in}}%
\pgfusepath{clip}%
\pgfsetbuttcap%
\pgfsetroundjoin%
\definecolor{currentfill}{rgb}{0.244972,0.287675,0.537260}%
\pgfsetfillcolor{currentfill}%
\pgfsetfillopacity{0.700000}%
\pgfsetlinewidth{0.000000pt}%
\definecolor{currentstroke}{rgb}{0.000000,0.000000,0.000000}%
\pgfsetstrokecolor{currentstroke}%
\pgfsetdash{}{0pt}%
\pgfpathmoveto{\pgfqpoint{5.044687in}{2.812267in}}%
\pgfpathlineto{\pgfqpoint{5.057894in}{2.810986in}}%
\pgfpathlineto{\pgfqpoint{5.071109in}{2.809823in}}%
\pgfpathlineto{\pgfqpoint{5.084335in}{2.808777in}}%
\pgfpathlineto{\pgfqpoint{5.097569in}{2.807848in}}%
\pgfpathlineto{\pgfqpoint{5.104735in}{2.818086in}}%
\pgfpathlineto{\pgfqpoint{5.111898in}{2.828360in}}%
\pgfpathlineto{\pgfqpoint{5.119057in}{2.838671in}}%
\pgfpathlineto{\pgfqpoint{5.126212in}{2.849022in}}%
\pgfpathlineto{\pgfqpoint{5.112990in}{2.850131in}}%
\pgfpathlineto{\pgfqpoint{5.099777in}{2.851356in}}%
\pgfpathlineto{\pgfqpoint{5.086573in}{2.852699in}}%
\pgfpathlineto{\pgfqpoint{5.073379in}{2.854160in}}%
\pgfpathlineto{\pgfqpoint{5.066211in}{2.843624in}}%
\pgfpathlineto{\pgfqpoint{5.059040in}{2.833131in}}%
\pgfpathlineto{\pgfqpoint{5.051865in}{2.822679in}}%
\pgfpathlineto{\pgfqpoint{5.044687in}{2.812267in}}%
\pgfpathclose%
\pgfusepath{fill}%
\end{pgfscope}%
\begin{pgfscope}%
\pgfpathrectangle{\pgfqpoint{1.254980in}{0.150000in}}{\pgfqpoint{5.490039in}{5.490039in}}%
\pgfusepath{clip}%
\pgfsetbuttcap%
\pgfsetroundjoin%
\definecolor{currentfill}{rgb}{0.275191,0.194905,0.496005}%
\pgfsetfillcolor{currentfill}%
\pgfsetfillopacity{0.700000}%
\pgfsetlinewidth{0.000000pt}%
\definecolor{currentstroke}{rgb}{0.000000,0.000000,0.000000}%
\pgfsetstrokecolor{currentstroke}%
\pgfsetdash{}{0pt}%
\pgfpathmoveto{\pgfqpoint{4.020880in}{2.628022in}}%
\pgfpathlineto{\pgfqpoint{4.033822in}{2.620534in}}%
\pgfpathlineto{\pgfqpoint{4.046767in}{2.613185in}}%
\pgfpathlineto{\pgfqpoint{4.059717in}{2.605975in}}%
\pgfpathlineto{\pgfqpoint{4.072670in}{2.598904in}}%
\pgfpathlineto{\pgfqpoint{4.080148in}{2.609098in}}%
\pgfpathlineto{\pgfqpoint{4.087621in}{2.619335in}}%
\pgfpathlineto{\pgfqpoint{4.095090in}{2.629616in}}%
\pgfpathlineto{\pgfqpoint{4.102554in}{2.639942in}}%
\pgfpathlineto{\pgfqpoint{4.089612in}{2.647035in}}%
\pgfpathlineto{\pgfqpoint{4.076674in}{2.654266in}}%
\pgfpathlineto{\pgfqpoint{4.063740in}{2.661636in}}%
\pgfpathlineto{\pgfqpoint{4.050810in}{2.669145in}}%
\pgfpathlineto{\pgfqpoint{4.043335in}{2.658793in}}%
\pgfpathlineto{\pgfqpoint{4.035854in}{2.648488in}}%
\pgfpathlineto{\pgfqpoint{4.028370in}{2.638232in}}%
\pgfpathlineto{\pgfqpoint{4.020880in}{2.628022in}}%
\pgfpathclose%
\pgfusepath{fill}%
\end{pgfscope}%
\begin{pgfscope}%
\pgfpathrectangle{\pgfqpoint{1.254980in}{0.150000in}}{\pgfqpoint{5.490039in}{5.490039in}}%
\pgfusepath{clip}%
\pgfsetbuttcap%
\pgfsetroundjoin%
\definecolor{currentfill}{rgb}{0.180629,0.429975,0.557282}%
\pgfsetfillcolor{currentfill}%
\pgfsetfillopacity{0.700000}%
\pgfsetlinewidth{0.000000pt}%
\definecolor{currentstroke}{rgb}{0.000000,0.000000,0.000000}%
\pgfsetstrokecolor{currentstroke}%
\pgfsetdash{}{0pt}%
\pgfpathmoveto{\pgfqpoint{3.205770in}{3.174769in}}%
\pgfpathlineto{\pgfqpoint{3.218749in}{3.158233in}}%
\pgfpathlineto{\pgfqpoint{3.231724in}{3.141885in}}%
\pgfpathlineto{\pgfqpoint{3.244695in}{3.125723in}}%
\pgfpathlineto{\pgfqpoint{3.257663in}{3.109745in}}%
\pgfpathlineto{\pgfqpoint{3.265402in}{3.118974in}}%
\pgfpathlineto{\pgfqpoint{3.273134in}{3.128299in}}%
\pgfpathlineto{\pgfqpoint{3.280860in}{3.137720in}}%
\pgfpathlineto{\pgfqpoint{3.288580in}{3.147236in}}%
\pgfpathlineto{\pgfqpoint{3.275629in}{3.163184in}}%
\pgfpathlineto{\pgfqpoint{3.262675in}{3.179316in}}%
\pgfpathlineto{\pgfqpoint{3.249718in}{3.195634in}}%
\pgfpathlineto{\pgfqpoint{3.236757in}{3.212138in}}%
\pgfpathlineto{\pgfqpoint{3.229020in}{3.202646in}}%
\pgfpathlineto{\pgfqpoint{3.221277in}{3.193254in}}%
\pgfpathlineto{\pgfqpoint{3.213527in}{3.183961in}}%
\pgfpathlineto{\pgfqpoint{3.205770in}{3.174769in}}%
\pgfpathclose%
\pgfusepath{fill}%
\end{pgfscope}%
\begin{pgfscope}%
\pgfpathrectangle{\pgfqpoint{1.254980in}{0.150000in}}{\pgfqpoint{5.490039in}{5.490039in}}%
\pgfusepath{clip}%
\pgfsetbuttcap%
\pgfsetroundjoin%
\definecolor{currentfill}{rgb}{0.270595,0.214069,0.507052}%
\pgfsetfillcolor{currentfill}%
\pgfsetfillopacity{0.700000}%
\pgfsetlinewidth{0.000000pt}%
\definecolor{currentstroke}{rgb}{0.000000,0.000000,0.000000}%
\pgfsetstrokecolor{currentstroke}%
\pgfsetdash{}{0pt}%
\pgfpathmoveto{\pgfqpoint{4.584641in}{2.660574in}}%
\pgfpathlineto{\pgfqpoint{4.597704in}{2.657045in}}%
\pgfpathlineto{\pgfqpoint{4.610774in}{2.653640in}}%
\pgfpathlineto{\pgfqpoint{4.623851in}{2.650361in}}%
\pgfpathlineto{\pgfqpoint{4.636936in}{2.647206in}}%
\pgfpathlineto{\pgfqpoint{4.644244in}{2.657647in}}%
\pgfpathlineto{\pgfqpoint{4.651548in}{2.668117in}}%
\pgfpathlineto{\pgfqpoint{4.658849in}{2.678617in}}%
\pgfpathlineto{\pgfqpoint{4.666145in}{2.689148in}}%
\pgfpathlineto{\pgfqpoint{4.653071in}{2.692403in}}%
\pgfpathlineto{\pgfqpoint{4.640004in}{2.695783in}}%
\pgfpathlineto{\pgfqpoint{4.626944in}{2.699288in}}%
\pgfpathlineto{\pgfqpoint{4.613892in}{2.702917in}}%
\pgfpathlineto{\pgfqpoint{4.606585in}{2.692280in}}%
\pgfpathlineto{\pgfqpoint{4.599274in}{2.681678in}}%
\pgfpathlineto{\pgfqpoint{4.591959in}{2.671110in}}%
\pgfpathlineto{\pgfqpoint{4.584641in}{2.660574in}}%
\pgfpathclose%
\pgfusepath{fill}%
\end{pgfscope}%
\begin{pgfscope}%
\pgfpathrectangle{\pgfqpoint{1.254980in}{0.150000in}}{\pgfqpoint{5.490039in}{5.490039in}}%
\pgfusepath{clip}%
\pgfsetbuttcap%
\pgfsetroundjoin%
\definecolor{currentfill}{rgb}{0.262138,0.242286,0.520837}%
\pgfsetfillcolor{currentfill}%
\pgfsetfillopacity{0.700000}%
\pgfsetlinewidth{0.000000pt}%
\definecolor{currentstroke}{rgb}{0.000000,0.000000,0.000000}%
\pgfsetstrokecolor{currentstroke}%
\pgfsetdash{}{0pt}%
\pgfpathmoveto{\pgfqpoint{3.702065in}{2.728160in}}%
\pgfpathlineto{\pgfqpoint{3.714983in}{2.717741in}}%
\pgfpathlineto{\pgfqpoint{3.727902in}{2.707474in}}%
\pgfpathlineto{\pgfqpoint{3.740823in}{2.697359in}}%
\pgfpathlineto{\pgfqpoint{3.753745in}{2.687396in}}%
\pgfpathlineto{\pgfqpoint{3.761323in}{2.697174in}}%
\pgfpathlineto{\pgfqpoint{3.768896in}{2.707013in}}%
\pgfpathlineto{\pgfqpoint{3.776463in}{2.716912in}}%
\pgfpathlineto{\pgfqpoint{3.784026in}{2.726872in}}%
\pgfpathlineto{\pgfqpoint{3.771117in}{2.736824in}}%
\pgfpathlineto{\pgfqpoint{3.758210in}{2.746928in}}%
\pgfpathlineto{\pgfqpoint{3.745304in}{2.757183in}}%
\pgfpathlineto{\pgfqpoint{3.732400in}{2.767591in}}%
\pgfpathlineto{\pgfqpoint{3.724824in}{2.757637in}}%
\pgfpathlineto{\pgfqpoint{3.717243in}{2.747747in}}%
\pgfpathlineto{\pgfqpoint{3.709657in}{2.737921in}}%
\pgfpathlineto{\pgfqpoint{3.702065in}{2.728160in}}%
\pgfpathclose%
\pgfusepath{fill}%
\end{pgfscope}%
\begin{pgfscope}%
\pgfpathrectangle{\pgfqpoint{1.254980in}{0.150000in}}{\pgfqpoint{5.490039in}{5.490039in}}%
\pgfusepath{clip}%
\pgfsetbuttcap%
\pgfsetroundjoin%
\definecolor{currentfill}{rgb}{0.225863,0.330805,0.547314}%
\pgfsetfillcolor{currentfill}%
\pgfsetfillopacity{0.700000}%
\pgfsetlinewidth{0.000000pt}%
\definecolor{currentstroke}{rgb}{0.000000,0.000000,0.000000}%
\pgfsetstrokecolor{currentstroke}%
\pgfsetdash{}{0pt}%
\pgfpathmoveto{\pgfqpoint{3.413057in}{2.932021in}}%
\pgfpathlineto{\pgfqpoint{3.425993in}{2.918346in}}%
\pgfpathlineto{\pgfqpoint{3.438927in}{2.904841in}}%
\pgfpathlineto{\pgfqpoint{3.451860in}{2.891504in}}%
\pgfpathlineto{\pgfqpoint{3.464791in}{2.878336in}}%
\pgfpathlineto{\pgfqpoint{3.472464in}{2.887706in}}%
\pgfpathlineto{\pgfqpoint{3.480131in}{2.897156in}}%
\pgfpathlineto{\pgfqpoint{3.487791in}{2.906685in}}%
\pgfpathlineto{\pgfqpoint{3.495446in}{2.916295in}}%
\pgfpathlineto{\pgfqpoint{3.482530in}{2.929435in}}%
\pgfpathlineto{\pgfqpoint{3.469613in}{2.942743in}}%
\pgfpathlineto{\pgfqpoint{3.456694in}{2.956220in}}%
\pgfpathlineto{\pgfqpoint{3.443775in}{2.969866in}}%
\pgfpathlineto{\pgfqpoint{3.436104in}{2.960279in}}%
\pgfpathlineto{\pgfqpoint{3.428428in}{2.950776in}}%
\pgfpathlineto{\pgfqpoint{3.420746in}{2.941357in}}%
\pgfpathlineto{\pgfqpoint{3.413057in}{2.932021in}}%
\pgfpathclose%
\pgfusepath{fill}%
\end{pgfscope}%
\begin{pgfscope}%
\pgfpathrectangle{\pgfqpoint{1.254980in}{0.150000in}}{\pgfqpoint{5.490039in}{5.490039in}}%
\pgfusepath{clip}%
\pgfsetbuttcap%
\pgfsetroundjoin%
\definecolor{currentfill}{rgb}{0.276194,0.190074,0.493001}%
\pgfsetfillcolor{currentfill}%
\pgfsetfillopacity{0.700000}%
\pgfsetlinewidth{0.000000pt}%
\definecolor{currentstroke}{rgb}{0.000000,0.000000,0.000000}%
\pgfsetstrokecolor{currentstroke}%
\pgfsetdash{}{0pt}%
\pgfpathmoveto{\pgfqpoint{4.154364in}{2.612947in}}%
\pgfpathlineto{\pgfqpoint{4.167327in}{2.606539in}}%
\pgfpathlineto{\pgfqpoint{4.180295in}{2.600266in}}%
\pgfpathlineto{\pgfqpoint{4.193268in}{2.594129in}}%
\pgfpathlineto{\pgfqpoint{4.206246in}{2.588125in}}%
\pgfpathlineto{\pgfqpoint{4.213684in}{2.598433in}}%
\pgfpathlineto{\pgfqpoint{4.221117in}{2.608779in}}%
\pgfpathlineto{\pgfqpoint{4.228546in}{2.619164in}}%
\pgfpathlineto{\pgfqpoint{4.235971in}{2.629587in}}%
\pgfpathlineto{\pgfqpoint{4.223004in}{2.635627in}}%
\pgfpathlineto{\pgfqpoint{4.210042in}{2.641802in}}%
\pgfpathlineto{\pgfqpoint{4.197085in}{2.648112in}}%
\pgfpathlineto{\pgfqpoint{4.184132in}{2.654557in}}%
\pgfpathlineto{\pgfqpoint{4.176697in}{2.644091in}}%
\pgfpathlineto{\pgfqpoint{4.169257in}{2.633667in}}%
\pgfpathlineto{\pgfqpoint{4.161813in}{2.623286in}}%
\pgfpathlineto{\pgfqpoint{4.154364in}{2.612947in}}%
\pgfpathclose%
\pgfusepath{fill}%
\end{pgfscope}%
\begin{pgfscope}%
\pgfpathrectangle{\pgfqpoint{1.254980in}{0.150000in}}{\pgfqpoint{5.490039in}{5.490039in}}%
\pgfusepath{clip}%
\pgfsetbuttcap%
\pgfsetroundjoin%
\definecolor{currentfill}{rgb}{0.250425,0.274290,0.533103}%
\pgfsetfillcolor{currentfill}%
\pgfsetfillopacity{0.700000}%
\pgfsetlinewidth{0.000000pt}%
\definecolor{currentstroke}{rgb}{0.000000,0.000000,0.000000}%
\pgfsetstrokecolor{currentstroke}%
\pgfsetdash{}{0pt}%
\pgfpathmoveto{\pgfqpoint{4.963160in}{2.776624in}}%
\pgfpathlineto{\pgfqpoint{4.976341in}{2.775035in}}%
\pgfpathlineto{\pgfqpoint{4.989531in}{2.773564in}}%
\pgfpathlineto{\pgfqpoint{5.002731in}{2.772211in}}%
\pgfpathlineto{\pgfqpoint{5.015940in}{2.770976in}}%
\pgfpathlineto{\pgfqpoint{5.023132in}{2.781249in}}%
\pgfpathlineto{\pgfqpoint{5.030321in}{2.791554in}}%
\pgfpathlineto{\pgfqpoint{5.037506in}{2.801892in}}%
\pgfpathlineto{\pgfqpoint{5.044687in}{2.812267in}}%
\pgfpathlineto{\pgfqpoint{5.031490in}{2.813666in}}%
\pgfpathlineto{\pgfqpoint{5.018302in}{2.815182in}}%
\pgfpathlineto{\pgfqpoint{5.005124in}{2.816817in}}%
\pgfpathlineto{\pgfqpoint{4.991954in}{2.818571in}}%
\pgfpathlineto{\pgfqpoint{4.984760in}{2.808026in}}%
\pgfpathlineto{\pgfqpoint{4.977564in}{2.797522in}}%
\pgfpathlineto{\pgfqpoint{4.970363in}{2.787055in}}%
\pgfpathlineto{\pgfqpoint{4.963160in}{2.776624in}}%
\pgfpathclose%
\pgfusepath{fill}%
\end{pgfscope}%
\begin{pgfscope}%
\pgfpathrectangle{\pgfqpoint{1.254980in}{0.150000in}}{\pgfqpoint{5.490039in}{5.490039in}}%
\pgfusepath{clip}%
\pgfsetbuttcap%
\pgfsetroundjoin%
\definecolor{currentfill}{rgb}{0.273006,0.204520,0.501721}%
\pgfsetfillcolor{currentfill}%
\pgfsetfillopacity{0.700000}%
\pgfsetlinewidth{0.000000pt}%
\definecolor{currentstroke}{rgb}{0.000000,0.000000,0.000000}%
\pgfsetstrokecolor{currentstroke}%
\pgfsetdash{}{0pt}%
\pgfpathmoveto{\pgfqpoint{3.887367in}{2.652615in}}%
\pgfpathlineto{\pgfqpoint{3.900295in}{2.643992in}}%
\pgfpathlineto{\pgfqpoint{3.913226in}{2.635515in}}%
\pgfpathlineto{\pgfqpoint{3.926160in}{2.627181in}}%
\pgfpathlineto{\pgfqpoint{3.939097in}{2.618991in}}%
\pgfpathlineto{\pgfqpoint{3.946617in}{2.629014in}}%
\pgfpathlineto{\pgfqpoint{3.954132in}{2.639088in}}%
\pgfpathlineto{\pgfqpoint{3.961643in}{2.649211in}}%
\pgfpathlineto{\pgfqpoint{3.969148in}{2.659385in}}%
\pgfpathlineto{\pgfqpoint{3.956224in}{2.667581in}}%
\pgfpathlineto{\pgfqpoint{3.943302in}{2.675920in}}%
\pgfpathlineto{\pgfqpoint{3.930383in}{2.684402in}}%
\pgfpathlineto{\pgfqpoint{3.917467in}{2.693030in}}%
\pgfpathlineto{\pgfqpoint{3.909949in}{2.682845in}}%
\pgfpathlineto{\pgfqpoint{3.902427in}{2.672714in}}%
\pgfpathlineto{\pgfqpoint{3.894899in}{2.662638in}}%
\pgfpathlineto{\pgfqpoint{3.887367in}{2.652615in}}%
\pgfpathclose%
\pgfusepath{fill}%
\end{pgfscope}%
\begin{pgfscope}%
\pgfpathrectangle{\pgfqpoint{1.254980in}{0.150000in}}{\pgfqpoint{5.490039in}{5.490039in}}%
\pgfusepath{clip}%
\pgfsetbuttcap%
\pgfsetroundjoin%
\definecolor{currentfill}{rgb}{0.169646,0.456262,0.558030}%
\pgfsetfillcolor{currentfill}%
\pgfsetfillopacity{0.700000}%
\pgfsetlinewidth{0.000000pt}%
\definecolor{currentstroke}{rgb}{0.000000,0.000000,0.000000}%
\pgfsetstrokecolor{currentstroke}%
\pgfsetdash{}{0pt}%
\pgfpathmoveto{\pgfqpoint{3.153813in}{3.242806in}}%
\pgfpathlineto{\pgfqpoint{3.166808in}{3.225510in}}%
\pgfpathlineto{\pgfqpoint{3.179800in}{3.208406in}}%
\pgfpathlineto{\pgfqpoint{3.192787in}{3.191493in}}%
\pgfpathlineto{\pgfqpoint{3.205770in}{3.174769in}}%
\pgfpathlineto{\pgfqpoint{3.213527in}{3.183961in}}%
\pgfpathlineto{\pgfqpoint{3.221277in}{3.193254in}}%
\pgfpathlineto{\pgfqpoint{3.229020in}{3.202646in}}%
\pgfpathlineto{\pgfqpoint{3.236757in}{3.212138in}}%
\pgfpathlineto{\pgfqpoint{3.223792in}{3.228832in}}%
\pgfpathlineto{\pgfqpoint{3.210823in}{3.245714in}}%
\pgfpathlineto{\pgfqpoint{3.197850in}{3.262787in}}%
\pgfpathlineto{\pgfqpoint{3.184872in}{3.280052in}}%
\pgfpathlineto{\pgfqpoint{3.177118in}{3.270585in}}%
\pgfpathlineto{\pgfqpoint{3.169356in}{3.261221in}}%
\pgfpathlineto{\pgfqpoint{3.161588in}{3.251961in}}%
\pgfpathlineto{\pgfqpoint{3.153813in}{3.242806in}}%
\pgfpathclose%
\pgfusepath{fill}%
\end{pgfscope}%
\begin{pgfscope}%
\pgfpathrectangle{\pgfqpoint{1.254980in}{0.150000in}}{\pgfqpoint{5.490039in}{5.490039in}}%
\pgfusepath{clip}%
\pgfsetbuttcap%
\pgfsetroundjoin%
\definecolor{currentfill}{rgb}{0.235526,0.309527,0.542944}%
\pgfsetfillcolor{currentfill}%
\pgfsetfillopacity{0.700000}%
\pgfsetlinewidth{0.000000pt}%
\definecolor{currentstroke}{rgb}{0.000000,0.000000,0.000000}%
\pgfsetstrokecolor{currentstroke}%
\pgfsetdash{}{0pt}%
\pgfpathmoveto{\pgfqpoint{3.464791in}{2.878336in}}%
\pgfpathlineto{\pgfqpoint{3.477722in}{2.865334in}}%
\pgfpathlineto{\pgfqpoint{3.490652in}{2.852499in}}%
\pgfpathlineto{\pgfqpoint{3.503581in}{2.839829in}}%
\pgfpathlineto{\pgfqpoint{3.516509in}{2.827323in}}%
\pgfpathlineto{\pgfqpoint{3.524166in}{2.836727in}}%
\pgfpathlineto{\pgfqpoint{3.531817in}{2.846207in}}%
\pgfpathlineto{\pgfqpoint{3.539463in}{2.855763in}}%
\pgfpathlineto{\pgfqpoint{3.547103in}{2.865395in}}%
\pgfpathlineto{\pgfqpoint{3.534189in}{2.877873in}}%
\pgfpathlineto{\pgfqpoint{3.521276in}{2.890515in}}%
\pgfpathlineto{\pgfqpoint{3.508361in}{2.903322in}}%
\pgfpathlineto{\pgfqpoint{3.495446in}{2.916295in}}%
\pgfpathlineto{\pgfqpoint{3.487791in}{2.906685in}}%
\pgfpathlineto{\pgfqpoint{3.480131in}{2.897156in}}%
\pgfpathlineto{\pgfqpoint{3.472464in}{2.887706in}}%
\pgfpathlineto{\pgfqpoint{3.464791in}{2.878336in}}%
\pgfpathclose%
\pgfusepath{fill}%
\end{pgfscope}%
\begin{pgfscope}%
\pgfpathrectangle{\pgfqpoint{1.254980in}{0.150000in}}{\pgfqpoint{5.490039in}{5.490039in}}%
\pgfusepath{clip}%
\pgfsetbuttcap%
\pgfsetroundjoin%
\definecolor{currentfill}{rgb}{0.257322,0.256130,0.526563}%
\pgfsetfillcolor{currentfill}%
\pgfsetfillopacity{0.700000}%
\pgfsetlinewidth{0.000000pt}%
\definecolor{currentstroke}{rgb}{0.000000,0.000000,0.000000}%
\pgfsetstrokecolor{currentstroke}%
\pgfsetdash{}{0pt}%
\pgfpathmoveto{\pgfqpoint{4.881625in}{2.742188in}}%
\pgfpathlineto{\pgfqpoint{4.894783in}{2.740269in}}%
\pgfpathlineto{\pgfqpoint{4.907949in}{2.738470in}}%
\pgfpathlineto{\pgfqpoint{4.921123in}{2.736790in}}%
\pgfpathlineto{\pgfqpoint{4.934307in}{2.735230in}}%
\pgfpathlineto{\pgfqpoint{4.941526in}{2.745533in}}%
\pgfpathlineto{\pgfqpoint{4.948741in}{2.755865in}}%
\pgfpathlineto{\pgfqpoint{4.955952in}{2.766228in}}%
\pgfpathlineto{\pgfqpoint{4.963160in}{2.776624in}}%
\pgfpathlineto{\pgfqpoint{4.949987in}{2.778333in}}%
\pgfpathlineto{\pgfqpoint{4.936823in}{2.780161in}}%
\pgfpathlineto{\pgfqpoint{4.923669in}{2.782108in}}%
\pgfpathlineto{\pgfqpoint{4.910522in}{2.784175in}}%
\pgfpathlineto{\pgfqpoint{4.903303in}{2.773625in}}%
\pgfpathlineto{\pgfqpoint{4.896081in}{2.763112in}}%
\pgfpathlineto{\pgfqpoint{4.888855in}{2.752633in}}%
\pgfpathlineto{\pgfqpoint{4.881625in}{2.742188in}}%
\pgfpathclose%
\pgfusepath{fill}%
\end{pgfscope}%
\begin{pgfscope}%
\pgfpathrectangle{\pgfqpoint{1.254980in}{0.150000in}}{\pgfqpoint{5.490039in}{5.490039in}}%
\pgfusepath{clip}%
\pgfsetbuttcap%
\pgfsetroundjoin%
\definecolor{currentfill}{rgb}{0.276194,0.190074,0.493001}%
\pgfsetfillcolor{currentfill}%
\pgfsetfillopacity{0.700000}%
\pgfsetlinewidth{0.000000pt}%
\definecolor{currentstroke}{rgb}{0.000000,0.000000,0.000000}%
\pgfsetstrokecolor{currentstroke}%
\pgfsetdash{}{0pt}%
\pgfpathmoveto{\pgfqpoint{4.287890in}{2.606759in}}%
\pgfpathlineto{\pgfqpoint{4.300883in}{2.601383in}}%
\pgfpathlineto{\pgfqpoint{4.313881in}{2.596139in}}%
\pgfpathlineto{\pgfqpoint{4.326885in}{2.591025in}}%
\pgfpathlineto{\pgfqpoint{4.339895in}{2.586043in}}%
\pgfpathlineto{\pgfqpoint{4.347294in}{2.596411in}}%
\pgfpathlineto{\pgfqpoint{4.354689in}{2.606813in}}%
\pgfpathlineto{\pgfqpoint{4.362080in}{2.617248in}}%
\pgfpathlineto{\pgfqpoint{4.369467in}{2.627717in}}%
\pgfpathlineto{\pgfqpoint{4.356467in}{2.632753in}}%
\pgfpathlineto{\pgfqpoint{4.343474in}{2.637919in}}%
\pgfpathlineto{\pgfqpoint{4.330486in}{2.643216in}}%
\pgfpathlineto{\pgfqpoint{4.317503in}{2.648645in}}%
\pgfpathlineto{\pgfqpoint{4.310106in}{2.638117in}}%
\pgfpathlineto{\pgfqpoint{4.302705in}{2.627628in}}%
\pgfpathlineto{\pgfqpoint{4.295300in}{2.617175in}}%
\pgfpathlineto{\pgfqpoint{4.287890in}{2.606759in}}%
\pgfpathclose%
\pgfusepath{fill}%
\end{pgfscope}%
\begin{pgfscope}%
\pgfpathrectangle{\pgfqpoint{1.254980in}{0.150000in}}{\pgfqpoint{5.490039in}{5.490039in}}%
\pgfusepath{clip}%
\pgfsetbuttcap%
\pgfsetroundjoin%
\definecolor{currentfill}{rgb}{0.157729,0.485932,0.558013}%
\pgfsetfillcolor{currentfill}%
\pgfsetfillopacity{0.700000}%
\pgfsetlinewidth{0.000000pt}%
\definecolor{currentstroke}{rgb}{0.000000,0.000000,0.000000}%
\pgfsetstrokecolor{currentstroke}%
\pgfsetdash{}{0pt}%
\pgfpathmoveto{\pgfqpoint{3.101781in}{3.313936in}}%
\pgfpathlineto{\pgfqpoint{3.114797in}{3.295859in}}%
\pgfpathlineto{\pgfqpoint{3.127807in}{3.277979in}}%
\pgfpathlineto{\pgfqpoint{3.140812in}{3.260295in}}%
\pgfpathlineto{\pgfqpoint{3.153813in}{3.242806in}}%
\pgfpathlineto{\pgfqpoint{3.161588in}{3.251961in}}%
\pgfpathlineto{\pgfqpoint{3.169356in}{3.261221in}}%
\pgfpathlineto{\pgfqpoint{3.177118in}{3.270585in}}%
\pgfpathlineto{\pgfqpoint{3.184872in}{3.280052in}}%
\pgfpathlineto{\pgfqpoint{3.171890in}{3.297511in}}%
\pgfpathlineto{\pgfqpoint{3.158903in}{3.315163in}}%
\pgfpathlineto{\pgfqpoint{3.145912in}{3.333012in}}%
\pgfpathlineto{\pgfqpoint{3.132915in}{3.351058in}}%
\pgfpathlineto{\pgfqpoint{3.125142in}{3.341615in}}%
\pgfpathlineto{\pgfqpoint{3.117362in}{3.332280in}}%
\pgfpathlineto{\pgfqpoint{3.109575in}{3.323054in}}%
\pgfpathlineto{\pgfqpoint{3.101781in}{3.313936in}}%
\pgfpathclose%
\pgfusepath{fill}%
\end{pgfscope}%
\begin{pgfscope}%
\pgfpathrectangle{\pgfqpoint{1.254980in}{0.150000in}}{\pgfqpoint{5.490039in}{5.490039in}}%
\pgfusepath{clip}%
\pgfsetbuttcap%
\pgfsetroundjoin%
\definecolor{currentfill}{rgb}{0.273006,0.204520,0.501721}%
\pgfsetfillcolor{currentfill}%
\pgfsetfillopacity{0.700000}%
\pgfsetlinewidth{0.000000pt}%
\definecolor{currentstroke}{rgb}{0.000000,0.000000,0.000000}%
\pgfsetstrokecolor{currentstroke}%
\pgfsetdash{}{0pt}%
\pgfpathmoveto{\pgfqpoint{4.503103in}{2.633767in}}%
\pgfpathlineto{\pgfqpoint{4.516148in}{2.629818in}}%
\pgfpathlineto{\pgfqpoint{4.529200in}{2.625997in}}%
\pgfpathlineto{\pgfqpoint{4.542259in}{2.622301in}}%
\pgfpathlineto{\pgfqpoint{4.555325in}{2.618732in}}%
\pgfpathlineto{\pgfqpoint{4.562660in}{2.629149in}}%
\pgfpathlineto{\pgfqpoint{4.569991in}{2.639594in}}%
\pgfpathlineto{\pgfqpoint{4.577318in}{2.650069in}}%
\pgfpathlineto{\pgfqpoint{4.584641in}{2.660574in}}%
\pgfpathlineto{\pgfqpoint{4.571585in}{2.664228in}}%
\pgfpathlineto{\pgfqpoint{4.558536in}{2.668008in}}%
\pgfpathlineto{\pgfqpoint{4.545494in}{2.671915in}}%
\pgfpathlineto{\pgfqpoint{4.532459in}{2.675947in}}%
\pgfpathlineto{\pgfqpoint{4.525126in}{2.665352in}}%
\pgfpathlineto{\pgfqpoint{4.517789in}{2.654791in}}%
\pgfpathlineto{\pgfqpoint{4.510448in}{2.644263in}}%
\pgfpathlineto{\pgfqpoint{4.503103in}{2.633767in}}%
\pgfpathclose%
\pgfusepath{fill}%
\end{pgfscope}%
\begin{pgfscope}%
\pgfpathrectangle{\pgfqpoint{1.254980in}{0.150000in}}{\pgfqpoint{5.490039in}{5.490039in}}%
\pgfusepath{clip}%
\pgfsetbuttcap%
\pgfsetroundjoin%
\definecolor{currentfill}{rgb}{0.244972,0.287675,0.537260}%
\pgfsetfillcolor{currentfill}%
\pgfsetfillopacity{0.700000}%
\pgfsetlinewidth{0.000000pt}%
\definecolor{currentstroke}{rgb}{0.000000,0.000000,0.000000}%
\pgfsetstrokecolor{currentstroke}%
\pgfsetdash{}{0pt}%
\pgfpathmoveto{\pgfqpoint{3.516509in}{2.827323in}}%
\pgfpathlineto{\pgfqpoint{3.529437in}{2.814980in}}%
\pgfpathlineto{\pgfqpoint{3.542364in}{2.802800in}}%
\pgfpathlineto{\pgfqpoint{3.555291in}{2.790782in}}%
\pgfpathlineto{\pgfqpoint{3.568218in}{2.778925in}}%
\pgfpathlineto{\pgfqpoint{3.575860in}{2.788363in}}%
\pgfpathlineto{\pgfqpoint{3.583497in}{2.797873in}}%
\pgfpathlineto{\pgfqpoint{3.591127in}{2.807455in}}%
\pgfpathlineto{\pgfqpoint{3.598752in}{2.817109in}}%
\pgfpathlineto{\pgfqpoint{3.585840in}{2.828939in}}%
\pgfpathlineto{\pgfqpoint{3.572928in}{2.840929in}}%
\pgfpathlineto{\pgfqpoint{3.560015in}{2.853081in}}%
\pgfpathlineto{\pgfqpoint{3.547103in}{2.865395in}}%
\pgfpathlineto{\pgfqpoint{3.539463in}{2.855763in}}%
\pgfpathlineto{\pgfqpoint{3.531817in}{2.846207in}}%
\pgfpathlineto{\pgfqpoint{3.524166in}{2.836727in}}%
\pgfpathlineto{\pgfqpoint{3.516509in}{2.827323in}}%
\pgfpathclose%
\pgfusepath{fill}%
\end{pgfscope}%
\begin{pgfscope}%
\pgfpathrectangle{\pgfqpoint{1.254980in}{0.150000in}}{\pgfqpoint{5.490039in}{5.490039in}}%
\pgfusepath{clip}%
\pgfsetbuttcap%
\pgfsetroundjoin%
\definecolor{currentfill}{rgb}{0.267968,0.223549,0.512008}%
\pgfsetfillcolor{currentfill}%
\pgfsetfillopacity{0.700000}%
\pgfsetlinewidth{0.000000pt}%
\definecolor{currentstroke}{rgb}{0.000000,0.000000,0.000000}%
\pgfsetstrokecolor{currentstroke}%
\pgfsetdash{}{0pt}%
\pgfpathmoveto{\pgfqpoint{3.753745in}{2.687396in}}%
\pgfpathlineto{\pgfqpoint{3.766668in}{2.677583in}}%
\pgfpathlineto{\pgfqpoint{3.779594in}{2.667920in}}%
\pgfpathlineto{\pgfqpoint{3.792521in}{2.658406in}}%
\pgfpathlineto{\pgfqpoint{3.805450in}{2.649041in}}%
\pgfpathlineto{\pgfqpoint{3.813015in}{2.658836in}}%
\pgfpathlineto{\pgfqpoint{3.820575in}{2.668688in}}%
\pgfpathlineto{\pgfqpoint{3.828129in}{2.678596in}}%
\pgfpathlineto{\pgfqpoint{3.835679in}{2.688561in}}%
\pgfpathlineto{\pgfqpoint{3.822763in}{2.697915in}}%
\pgfpathlineto{\pgfqpoint{3.809849in}{2.707418in}}%
\pgfpathlineto{\pgfqpoint{3.796936in}{2.717070in}}%
\pgfpathlineto{\pgfqpoint{3.784026in}{2.726872in}}%
\pgfpathlineto{\pgfqpoint{3.776463in}{2.716912in}}%
\pgfpathlineto{\pgfqpoint{3.768896in}{2.707013in}}%
\pgfpathlineto{\pgfqpoint{3.761323in}{2.697174in}}%
\pgfpathlineto{\pgfqpoint{3.753745in}{2.687396in}}%
\pgfpathclose%
\pgfusepath{fill}%
\end{pgfscope}%
\begin{pgfscope}%
\pgfpathrectangle{\pgfqpoint{1.254980in}{0.150000in}}{\pgfqpoint{5.490039in}{5.490039in}}%
\pgfusepath{clip}%
\pgfsetbuttcap%
\pgfsetroundjoin%
\definecolor{currentfill}{rgb}{0.262138,0.242286,0.520837}%
\pgfsetfillcolor{currentfill}%
\pgfsetfillopacity{0.700000}%
\pgfsetlinewidth{0.000000pt}%
\definecolor{currentstroke}{rgb}{0.000000,0.000000,0.000000}%
\pgfsetstrokecolor{currentstroke}%
\pgfsetdash{}{0pt}%
\pgfpathmoveto{\pgfqpoint{4.800080in}{2.709062in}}%
\pgfpathlineto{\pgfqpoint{4.813214in}{2.706794in}}%
\pgfpathlineto{\pgfqpoint{4.826357in}{2.704646in}}%
\pgfpathlineto{\pgfqpoint{4.839508in}{2.702619in}}%
\pgfpathlineto{\pgfqpoint{4.852668in}{2.700712in}}%
\pgfpathlineto{\pgfqpoint{4.859913in}{2.711038in}}%
\pgfpathlineto{\pgfqpoint{4.867154in}{2.721392in}}%
\pgfpathlineto{\pgfqpoint{4.874392in}{2.731775in}}%
\pgfpathlineto{\pgfqpoint{4.881625in}{2.742188in}}%
\pgfpathlineto{\pgfqpoint{4.868476in}{2.744227in}}%
\pgfpathlineto{\pgfqpoint{4.855336in}{2.746387in}}%
\pgfpathlineto{\pgfqpoint{4.842204in}{2.748667in}}%
\pgfpathlineto{\pgfqpoint{4.829080in}{2.751068in}}%
\pgfpathlineto{\pgfqpoint{4.821836in}{2.740517in}}%
\pgfpathlineto{\pgfqpoint{4.814588in}{2.730000in}}%
\pgfpathlineto{\pgfqpoint{4.807336in}{2.719516in}}%
\pgfpathlineto{\pgfqpoint{4.800080in}{2.709062in}}%
\pgfpathclose%
\pgfusepath{fill}%
\end{pgfscope}%
\begin{pgfscope}%
\pgfpathrectangle{\pgfqpoint{1.254980in}{0.150000in}}{\pgfqpoint{5.490039in}{5.490039in}}%
\pgfusepath{clip}%
\pgfsetbuttcap%
\pgfsetroundjoin%
\definecolor{currentfill}{rgb}{0.146180,0.515413,0.556823}%
\pgfsetfillcolor{currentfill}%
\pgfsetfillopacity{0.700000}%
\pgfsetlinewidth{0.000000pt}%
\definecolor{currentstroke}{rgb}{0.000000,0.000000,0.000000}%
\pgfsetstrokecolor{currentstroke}%
\pgfsetdash{}{0pt}%
\pgfpathmoveto{\pgfqpoint{3.049665in}{3.388243in}}%
\pgfpathlineto{\pgfqpoint{3.062703in}{3.369364in}}%
\pgfpathlineto{\pgfqpoint{3.075735in}{3.350687in}}%
\pgfpathlineto{\pgfqpoint{3.088761in}{3.332212in}}%
\pgfpathlineto{\pgfqpoint{3.101781in}{3.313936in}}%
\pgfpathlineto{\pgfqpoint{3.109575in}{3.323054in}}%
\pgfpathlineto{\pgfqpoint{3.117362in}{3.332280in}}%
\pgfpathlineto{\pgfqpoint{3.125142in}{3.341615in}}%
\pgfpathlineto{\pgfqpoint{3.132915in}{3.351058in}}%
\pgfpathlineto{\pgfqpoint{3.119913in}{3.369301in}}%
\pgfpathlineto{\pgfqpoint{3.106906in}{3.387745in}}%
\pgfpathlineto{\pgfqpoint{3.093894in}{3.406390in}}%
\pgfpathlineto{\pgfqpoint{3.080875in}{3.425237in}}%
\pgfpathlineto{\pgfqpoint{3.073084in}{3.415820in}}%
\pgfpathlineto{\pgfqpoint{3.065285in}{3.406515in}}%
\pgfpathlineto{\pgfqpoint{3.057479in}{3.397323in}}%
\pgfpathlineto{\pgfqpoint{3.049665in}{3.388243in}}%
\pgfpathclose%
\pgfusepath{fill}%
\end{pgfscope}%
\begin{pgfscope}%
\pgfpathrectangle{\pgfqpoint{1.254980in}{0.150000in}}{\pgfqpoint{5.490039in}{5.490039in}}%
\pgfusepath{clip}%
\pgfsetbuttcap%
\pgfsetroundjoin%
\definecolor{currentfill}{rgb}{0.277134,0.185228,0.489898}%
\pgfsetfillcolor{currentfill}%
\pgfsetfillopacity{0.700000}%
\pgfsetlinewidth{0.000000pt}%
\definecolor{currentstroke}{rgb}{0.000000,0.000000,0.000000}%
\pgfsetstrokecolor{currentstroke}%
\pgfsetdash{}{0pt}%
\pgfpathmoveto{\pgfqpoint{4.072670in}{2.598904in}}%
\pgfpathlineto{\pgfqpoint{4.085627in}{2.591970in}}%
\pgfpathlineto{\pgfqpoint{4.098588in}{2.585175in}}%
\pgfpathlineto{\pgfqpoint{4.111554in}{2.578516in}}%
\pgfpathlineto{\pgfqpoint{4.124524in}{2.571994in}}%
\pgfpathlineto{\pgfqpoint{4.131991in}{2.582172in}}%
\pgfpathlineto{\pgfqpoint{4.139453in}{2.592390in}}%
\pgfpathlineto{\pgfqpoint{4.146911in}{2.602648in}}%
\pgfpathlineto{\pgfqpoint{4.154364in}{2.612947in}}%
\pgfpathlineto{\pgfqpoint{4.141405in}{2.619490in}}%
\pgfpathlineto{\pgfqpoint{4.128451in}{2.626171in}}%
\pgfpathlineto{\pgfqpoint{4.115500in}{2.632988in}}%
\pgfpathlineto{\pgfqpoint{4.102554in}{2.639942in}}%
\pgfpathlineto{\pgfqpoint{4.095090in}{2.629616in}}%
\pgfpathlineto{\pgfqpoint{4.087621in}{2.619335in}}%
\pgfpathlineto{\pgfqpoint{4.080148in}{2.609098in}}%
\pgfpathlineto{\pgfqpoint{4.072670in}{2.598904in}}%
\pgfpathclose%
\pgfusepath{fill}%
\end{pgfscope}%
\begin{pgfscope}%
\pgfpathrectangle{\pgfqpoint{1.254980in}{0.150000in}}{\pgfqpoint{5.490039in}{5.490039in}}%
\pgfusepath{clip}%
\pgfsetbuttcap%
\pgfsetroundjoin%
\definecolor{currentfill}{rgb}{0.252194,0.269783,0.531579}%
\pgfsetfillcolor{currentfill}%
\pgfsetfillopacity{0.700000}%
\pgfsetlinewidth{0.000000pt}%
\definecolor{currentstroke}{rgb}{0.000000,0.000000,0.000000}%
\pgfsetstrokecolor{currentstroke}%
\pgfsetdash{}{0pt}%
\pgfpathmoveto{\pgfqpoint{3.568218in}{2.778925in}}%
\pgfpathlineto{\pgfqpoint{3.581145in}{2.767227in}}%
\pgfpathlineto{\pgfqpoint{3.594073in}{2.755689in}}%
\pgfpathlineto{\pgfqpoint{3.607000in}{2.744309in}}%
\pgfpathlineto{\pgfqpoint{3.619928in}{2.733087in}}%
\pgfpathlineto{\pgfqpoint{3.627555in}{2.742559in}}%
\pgfpathlineto{\pgfqpoint{3.635177in}{2.752098in}}%
\pgfpathlineto{\pgfqpoint{3.642793in}{2.761706in}}%
\pgfpathlineto{\pgfqpoint{3.650404in}{2.771382in}}%
\pgfpathlineto{\pgfqpoint{3.637490in}{2.782577in}}%
\pgfpathlineto{\pgfqpoint{3.624577in}{2.793929in}}%
\pgfpathlineto{\pgfqpoint{3.611665in}{2.805440in}}%
\pgfpathlineto{\pgfqpoint{3.598752in}{2.817109in}}%
\pgfpathlineto{\pgfqpoint{3.591127in}{2.807455in}}%
\pgfpathlineto{\pgfqpoint{3.583497in}{2.797873in}}%
\pgfpathlineto{\pgfqpoint{3.575860in}{2.788363in}}%
\pgfpathlineto{\pgfqpoint{3.568218in}{2.778925in}}%
\pgfpathclose%
\pgfusepath{fill}%
\end{pgfscope}%
\begin{pgfscope}%
\pgfpathrectangle{\pgfqpoint{1.254980in}{0.150000in}}{\pgfqpoint{5.490039in}{5.490039in}}%
\pgfusepath{clip}%
\pgfsetbuttcap%
\pgfsetroundjoin%
\definecolor{currentfill}{rgb}{0.275191,0.194905,0.496005}%
\pgfsetfillcolor{currentfill}%
\pgfsetfillopacity{0.700000}%
\pgfsetlinewidth{0.000000pt}%
\definecolor{currentstroke}{rgb}{0.000000,0.000000,0.000000}%
\pgfsetstrokecolor{currentstroke}%
\pgfsetdash{}{0pt}%
\pgfpathmoveto{\pgfqpoint{3.939097in}{2.618991in}}%
\pgfpathlineto{\pgfqpoint{3.952037in}{2.610943in}}%
\pgfpathlineto{\pgfqpoint{3.964980in}{2.603037in}}%
\pgfpathlineto{\pgfqpoint{3.977926in}{2.595273in}}%
\pgfpathlineto{\pgfqpoint{3.990876in}{2.587649in}}%
\pgfpathlineto{\pgfqpoint{3.998384in}{2.597674in}}%
\pgfpathlineto{\pgfqpoint{4.005888in}{2.607744in}}%
\pgfpathlineto{\pgfqpoint{4.013386in}{2.617860in}}%
\pgfpathlineto{\pgfqpoint{4.020880in}{2.628022in}}%
\pgfpathlineto{\pgfqpoint{4.007942in}{2.635651in}}%
\pgfpathlineto{\pgfqpoint{3.995008in}{2.643421in}}%
\pgfpathlineto{\pgfqpoint{3.982076in}{2.651332in}}%
\pgfpathlineto{\pgfqpoint{3.969148in}{2.659385in}}%
\pgfpathlineto{\pgfqpoint{3.961643in}{2.649211in}}%
\pgfpathlineto{\pgfqpoint{3.954132in}{2.639088in}}%
\pgfpathlineto{\pgfqpoint{3.946617in}{2.629014in}}%
\pgfpathlineto{\pgfqpoint{3.939097in}{2.618991in}}%
\pgfpathclose%
\pgfusepath{fill}%
\end{pgfscope}%
\begin{pgfscope}%
\pgfpathrectangle{\pgfqpoint{1.254980in}{0.150000in}}{\pgfqpoint{5.490039in}{5.490039in}}%
\pgfusepath{clip}%
\pgfsetbuttcap%
\pgfsetroundjoin%
\definecolor{currentfill}{rgb}{0.275191,0.194905,0.496005}%
\pgfsetfillcolor{currentfill}%
\pgfsetfillopacity{0.700000}%
\pgfsetlinewidth{0.000000pt}%
\definecolor{currentstroke}{rgb}{0.000000,0.000000,0.000000}%
\pgfsetstrokecolor{currentstroke}%
\pgfsetdash{}{0pt}%
\pgfpathmoveto{\pgfqpoint{4.421523in}{2.608871in}}%
\pgfpathlineto{\pgfqpoint{4.434553in}{2.604482in}}%
\pgfpathlineto{\pgfqpoint{4.447589in}{2.600221in}}%
\pgfpathlineto{\pgfqpoint{4.460631in}{2.596088in}}%
\pgfpathlineto{\pgfqpoint{4.473680in}{2.592082in}}%
\pgfpathlineto{\pgfqpoint{4.481042in}{2.602460in}}%
\pgfpathlineto{\pgfqpoint{4.488400in}{2.612866in}}%
\pgfpathlineto{\pgfqpoint{4.495753in}{2.623301in}}%
\pgfpathlineto{\pgfqpoint{4.503103in}{2.633767in}}%
\pgfpathlineto{\pgfqpoint{4.490064in}{2.637842in}}%
\pgfpathlineto{\pgfqpoint{4.477032in}{2.642044in}}%
\pgfpathlineto{\pgfqpoint{4.464006in}{2.646374in}}%
\pgfpathlineto{\pgfqpoint{4.450987in}{2.650832in}}%
\pgfpathlineto{\pgfqpoint{4.443627in}{2.640292in}}%
\pgfpathlineto{\pgfqpoint{4.436263in}{2.629786in}}%
\pgfpathlineto{\pgfqpoint{4.428896in}{2.619312in}}%
\pgfpathlineto{\pgfqpoint{4.421523in}{2.608871in}}%
\pgfpathclose%
\pgfusepath{fill}%
\end{pgfscope}%
\begin{pgfscope}%
\pgfpathrectangle{\pgfqpoint{1.254980in}{0.150000in}}{\pgfqpoint{5.490039in}{5.490039in}}%
\pgfusepath{clip}%
\pgfsetbuttcap%
\pgfsetroundjoin%
\definecolor{currentfill}{rgb}{0.266580,0.228262,0.514349}%
\pgfsetfillcolor{currentfill}%
\pgfsetfillopacity{0.700000}%
\pgfsetlinewidth{0.000000pt}%
\definecolor{currentstroke}{rgb}{0.000000,0.000000,0.000000}%
\pgfsetstrokecolor{currentstroke}%
\pgfsetdash{}{0pt}%
\pgfpathmoveto{\pgfqpoint{4.718518in}{2.677360in}}%
\pgfpathlineto{\pgfqpoint{4.731631in}{2.674721in}}%
\pgfpathlineto{\pgfqpoint{4.744752in}{2.672204in}}%
\pgfpathlineto{\pgfqpoint{4.757880in}{2.669808in}}%
\pgfpathlineto{\pgfqpoint{4.771017in}{2.667535in}}%
\pgfpathlineto{\pgfqpoint{4.778289in}{2.677877in}}%
\pgfpathlineto{\pgfqpoint{4.785556in}{2.688244in}}%
\pgfpathlineto{\pgfqpoint{4.792820in}{2.698639in}}%
\pgfpathlineto{\pgfqpoint{4.800080in}{2.709062in}}%
\pgfpathlineto{\pgfqpoint{4.786953in}{2.711453in}}%
\pgfpathlineto{\pgfqpoint{4.773835in}{2.713964in}}%
\pgfpathlineto{\pgfqpoint{4.760725in}{2.716598in}}%
\pgfpathlineto{\pgfqpoint{4.747623in}{2.719354in}}%
\pgfpathlineto{\pgfqpoint{4.740353in}{2.708808in}}%
\pgfpathlineto{\pgfqpoint{4.733079in}{2.698295in}}%
\pgfpathlineto{\pgfqpoint{4.725800in}{2.687813in}}%
\pgfpathlineto{\pgfqpoint{4.718518in}{2.677360in}}%
\pgfpathclose%
\pgfusepath{fill}%
\end{pgfscope}%
\begin{pgfscope}%
\pgfpathrectangle{\pgfqpoint{1.254980in}{0.150000in}}{\pgfqpoint{5.490039in}{5.490039in}}%
\pgfusepath{clip}%
\pgfsetbuttcap%
\pgfsetroundjoin%
\definecolor{currentfill}{rgb}{0.278012,0.180367,0.486697}%
\pgfsetfillcolor{currentfill}%
\pgfsetfillopacity{0.700000}%
\pgfsetlinewidth{0.000000pt}%
\definecolor{currentstroke}{rgb}{0.000000,0.000000,0.000000}%
\pgfsetstrokecolor{currentstroke}%
\pgfsetdash{}{0pt}%
\pgfpathmoveto{\pgfqpoint{4.206246in}{2.588125in}}%
\pgfpathlineto{\pgfqpoint{4.219228in}{2.582256in}}%
\pgfpathlineto{\pgfqpoint{4.232216in}{2.576520in}}%
\pgfpathlineto{\pgfqpoint{4.245209in}{2.570917in}}%
\pgfpathlineto{\pgfqpoint{4.258207in}{2.565446in}}%
\pgfpathlineto{\pgfqpoint{4.265634in}{2.575723in}}%
\pgfpathlineto{\pgfqpoint{4.273057in}{2.586034in}}%
\pgfpathlineto{\pgfqpoint{4.280476in}{2.596379in}}%
\pgfpathlineto{\pgfqpoint{4.287890in}{2.606759in}}%
\pgfpathlineto{\pgfqpoint{4.274902in}{2.612267in}}%
\pgfpathlineto{\pgfqpoint{4.261920in}{2.617907in}}%
\pgfpathlineto{\pgfqpoint{4.248943in}{2.623680in}}%
\pgfpathlineto{\pgfqpoint{4.235971in}{2.629587in}}%
\pgfpathlineto{\pgfqpoint{4.228546in}{2.619164in}}%
\pgfpathlineto{\pgfqpoint{4.221117in}{2.608779in}}%
\pgfpathlineto{\pgfqpoint{4.213684in}{2.598433in}}%
\pgfpathlineto{\pgfqpoint{4.206246in}{2.588125in}}%
\pgfpathclose%
\pgfusepath{fill}%
\end{pgfscope}%
\begin{pgfscope}%
\pgfpathrectangle{\pgfqpoint{1.254980in}{0.150000in}}{\pgfqpoint{5.490039in}{5.490039in}}%
\pgfusepath{clip}%
\pgfsetbuttcap%
\pgfsetroundjoin%
\definecolor{currentfill}{rgb}{0.218130,0.347432,0.550038}%
\pgfsetfillcolor{currentfill}%
\pgfsetfillopacity{0.700000}%
\pgfsetlinewidth{0.000000pt}%
\definecolor{currentstroke}{rgb}{0.000000,0.000000,0.000000}%
\pgfsetstrokecolor{currentstroke}%
\pgfsetdash{}{0pt}%
\pgfpathmoveto{\pgfqpoint{5.342471in}{2.924330in}}%
\pgfpathlineto{\pgfqpoint{5.355798in}{2.924312in}}%
\pgfpathlineto{\pgfqpoint{5.369136in}{2.924409in}}%
\pgfpathlineto{\pgfqpoint{5.382484in}{2.924619in}}%
\pgfpathlineto{\pgfqpoint{5.395844in}{2.924942in}}%
\pgfpathlineto{\pgfqpoint{5.402917in}{2.934853in}}%
\pgfpathlineto{\pgfqpoint{5.409987in}{2.944810in}}%
\pgfpathlineto{\pgfqpoint{5.417055in}{2.954813in}}%
\pgfpathlineto{\pgfqpoint{5.424119in}{2.964867in}}%
\pgfpathlineto{\pgfqpoint{5.410774in}{2.964772in}}%
\pgfpathlineto{\pgfqpoint{5.397440in}{2.964790in}}%
\pgfpathlineto{\pgfqpoint{5.384116in}{2.964922in}}%
\pgfpathlineto{\pgfqpoint{5.370804in}{2.965167in}}%
\pgfpathlineto{\pgfqpoint{5.363725in}{2.954880in}}%
\pgfpathlineto{\pgfqpoint{5.356643in}{2.944646in}}%
\pgfpathlineto{\pgfqpoint{5.349559in}{2.934463in}}%
\pgfpathlineto{\pgfqpoint{5.342471in}{2.924330in}}%
\pgfpathclose%
\pgfusepath{fill}%
\end{pgfscope}%
\begin{pgfscope}%
\pgfpathrectangle{\pgfqpoint{1.254980in}{0.150000in}}{\pgfqpoint{5.490039in}{5.490039in}}%
\pgfusepath{clip}%
\pgfsetbuttcap%
\pgfsetroundjoin%
\definecolor{currentfill}{rgb}{0.225863,0.330805,0.547314}%
\pgfsetfillcolor{currentfill}%
\pgfsetfillopacity{0.700000}%
\pgfsetlinewidth{0.000000pt}%
\definecolor{currentstroke}{rgb}{0.000000,0.000000,0.000000}%
\pgfsetstrokecolor{currentstroke}%
\pgfsetdash{}{0pt}%
\pgfpathmoveto{\pgfqpoint{5.260832in}{2.884603in}}%
\pgfpathlineto{\pgfqpoint{5.274131in}{2.884341in}}%
\pgfpathlineto{\pgfqpoint{5.287440in}{2.884193in}}%
\pgfpathlineto{\pgfqpoint{5.300760in}{2.884159in}}%
\pgfpathlineto{\pgfqpoint{5.314090in}{2.884240in}}%
\pgfpathlineto{\pgfqpoint{5.321190in}{2.894200in}}%
\pgfpathlineto{\pgfqpoint{5.328287in}{2.904200in}}%
\pgfpathlineto{\pgfqpoint{5.335381in}{2.914243in}}%
\pgfpathlineto{\pgfqpoint{5.342471in}{2.924330in}}%
\pgfpathlineto{\pgfqpoint{5.329155in}{2.924461in}}%
\pgfpathlineto{\pgfqpoint{5.315849in}{2.924707in}}%
\pgfpathlineto{\pgfqpoint{5.302553in}{2.925067in}}%
\pgfpathlineto{\pgfqpoint{5.289267in}{2.925541in}}%
\pgfpathlineto{\pgfqpoint{5.282163in}{2.915237in}}%
\pgfpathlineto{\pgfqpoint{5.275056in}{2.904980in}}%
\pgfpathlineto{\pgfqpoint{5.267946in}{2.894770in}}%
\pgfpathlineto{\pgfqpoint{5.260832in}{2.884603in}}%
\pgfpathclose%
\pgfusepath{fill}%
\end{pgfscope}%
\begin{pgfscope}%
\pgfpathrectangle{\pgfqpoint{1.254980in}{0.150000in}}{\pgfqpoint{5.490039in}{5.490039in}}%
\pgfusepath{clip}%
\pgfsetbuttcap%
\pgfsetroundjoin%
\definecolor{currentfill}{rgb}{0.271828,0.209303,0.504434}%
\pgfsetfillcolor{currentfill}%
\pgfsetfillopacity{0.700000}%
\pgfsetlinewidth{0.000000pt}%
\definecolor{currentstroke}{rgb}{0.000000,0.000000,0.000000}%
\pgfsetstrokecolor{currentstroke}%
\pgfsetdash{}{0pt}%
\pgfpathmoveto{\pgfqpoint{3.805450in}{2.649041in}}%
\pgfpathlineto{\pgfqpoint{3.818381in}{2.639823in}}%
\pgfpathlineto{\pgfqpoint{3.831315in}{2.630753in}}%
\pgfpathlineto{\pgfqpoint{3.844250in}{2.621829in}}%
\pgfpathlineto{\pgfqpoint{3.857188in}{2.613052in}}%
\pgfpathlineto{\pgfqpoint{3.864741in}{2.622863in}}%
\pgfpathlineto{\pgfqpoint{3.872288in}{2.632728in}}%
\pgfpathlineto{\pgfqpoint{3.879830in}{2.642645in}}%
\pgfpathlineto{\pgfqpoint{3.887367in}{2.652615in}}%
\pgfpathlineto{\pgfqpoint{3.874441in}{2.661382in}}%
\pgfpathlineto{\pgfqpoint{3.861518in}{2.670295in}}%
\pgfpathlineto{\pgfqpoint{3.848598in}{2.679354in}}%
\pgfpathlineto{\pgfqpoint{3.835679in}{2.688561in}}%
\pgfpathlineto{\pgfqpoint{3.828129in}{2.678596in}}%
\pgfpathlineto{\pgfqpoint{3.820575in}{2.668688in}}%
\pgfpathlineto{\pgfqpoint{3.813015in}{2.658836in}}%
\pgfpathlineto{\pgfqpoint{3.805450in}{2.649041in}}%
\pgfpathclose%
\pgfusepath{fill}%
\end{pgfscope}%
\begin{pgfscope}%
\pgfpathrectangle{\pgfqpoint{1.254980in}{0.150000in}}{\pgfqpoint{5.490039in}{5.490039in}}%
\pgfusepath{clip}%
\pgfsetbuttcap%
\pgfsetroundjoin%
\definecolor{currentfill}{rgb}{0.135066,0.544853,0.554029}%
\pgfsetfillcolor{currentfill}%
\pgfsetfillopacity{0.700000}%
\pgfsetlinewidth{0.000000pt}%
\definecolor{currentstroke}{rgb}{0.000000,0.000000,0.000000}%
\pgfsetstrokecolor{currentstroke}%
\pgfsetdash{}{0pt}%
\pgfpathmoveto{\pgfqpoint{2.997453in}{3.465815in}}%
\pgfpathlineto{\pgfqpoint{3.010516in}{3.446111in}}%
\pgfpathlineto{\pgfqpoint{3.023572in}{3.426615in}}%
\pgfpathlineto{\pgfqpoint{3.036622in}{3.407326in}}%
\pgfpathlineto{\pgfqpoint{3.049665in}{3.388243in}}%
\pgfpathlineto{\pgfqpoint{3.057479in}{3.397323in}}%
\pgfpathlineto{\pgfqpoint{3.065285in}{3.406515in}}%
\pgfpathlineto{\pgfqpoint{3.073084in}{3.415820in}}%
\pgfpathlineto{\pgfqpoint{3.080875in}{3.425237in}}%
\pgfpathlineto{\pgfqpoint{3.067851in}{3.444289in}}%
\pgfpathlineto{\pgfqpoint{3.054821in}{3.463545in}}%
\pgfpathlineto{\pgfqpoint{3.041785in}{3.483008in}}%
\pgfpathlineto{\pgfqpoint{3.028742in}{3.502680in}}%
\pgfpathlineto{\pgfqpoint{3.020931in}{3.493289in}}%
\pgfpathlineto{\pgfqpoint{3.013112in}{3.484014in}}%
\pgfpathlineto{\pgfqpoint{3.005287in}{3.474856in}}%
\pgfpathlineto{\pgfqpoint{2.997453in}{3.465815in}}%
\pgfpathclose%
\pgfusepath{fill}%
\end{pgfscope}%
\begin{pgfscope}%
\pgfpathrectangle{\pgfqpoint{1.254980in}{0.150000in}}{\pgfqpoint{5.490039in}{5.490039in}}%
\pgfusepath{clip}%
\pgfsetbuttcap%
\pgfsetroundjoin%
\definecolor{currentfill}{rgb}{0.210503,0.363727,0.552206}%
\pgfsetfillcolor{currentfill}%
\pgfsetfillopacity{0.700000}%
\pgfsetlinewidth{0.000000pt}%
\definecolor{currentstroke}{rgb}{0.000000,0.000000,0.000000}%
\pgfsetstrokecolor{currentstroke}%
\pgfsetdash{}{0pt}%
\pgfpathmoveto{\pgfqpoint{5.424119in}{2.964867in}}%
\pgfpathlineto{\pgfqpoint{5.437475in}{2.965075in}}%
\pgfpathlineto{\pgfqpoint{5.450841in}{2.965396in}}%
\pgfpathlineto{\pgfqpoint{5.464219in}{2.965830in}}%
\pgfpathlineto{\pgfqpoint{5.477607in}{2.966376in}}%
\pgfpathlineto{\pgfqpoint{5.484654in}{2.976244in}}%
\pgfpathlineto{\pgfqpoint{5.491698in}{2.986162in}}%
\pgfpathlineto{\pgfqpoint{5.498739in}{2.996133in}}%
\pgfpathlineto{\pgfqpoint{5.485362in}{2.995768in}}%
\pgfpathlineto{\pgfqpoint{5.471995in}{2.995516in}}%
\pgfpathlineto{\pgfqpoint{5.458640in}{2.995376in}}%
\pgfpathlineto{\pgfqpoint{5.445295in}{2.995349in}}%
\pgfpathlineto{\pgfqpoint{5.438239in}{2.985132in}}%
\pgfpathlineto{\pgfqpoint{5.431180in}{2.974972in}}%
\pgfpathlineto{\pgfqpoint{5.424119in}{2.964867in}}%
\pgfpathclose%
\pgfusepath{fill}%
\end{pgfscope}%
\begin{pgfscope}%
\pgfpathrectangle{\pgfqpoint{1.254980in}{0.150000in}}{\pgfqpoint{5.490039in}{5.490039in}}%
\pgfusepath{clip}%
\pgfsetbuttcap%
\pgfsetroundjoin%
\definecolor{currentfill}{rgb}{0.258965,0.251537,0.524736}%
\pgfsetfillcolor{currentfill}%
\pgfsetfillopacity{0.700000}%
\pgfsetlinewidth{0.000000pt}%
\definecolor{currentstroke}{rgb}{0.000000,0.000000,0.000000}%
\pgfsetstrokecolor{currentstroke}%
\pgfsetdash{}{0pt}%
\pgfpathmoveto{\pgfqpoint{3.619928in}{2.733087in}}%
\pgfpathlineto{\pgfqpoint{3.632856in}{2.722021in}}%
\pgfpathlineto{\pgfqpoint{3.645785in}{2.711111in}}%
\pgfpathlineto{\pgfqpoint{3.658715in}{2.700356in}}%
\pgfpathlineto{\pgfqpoint{3.671646in}{2.689756in}}%
\pgfpathlineto{\pgfqpoint{3.679259in}{2.699261in}}%
\pgfpathlineto{\pgfqpoint{3.686866in}{2.708830in}}%
\pgfpathlineto{\pgfqpoint{3.694469in}{2.718463in}}%
\pgfpathlineto{\pgfqpoint{3.702065in}{2.728160in}}%
\pgfpathlineto{\pgfqpoint{3.689149in}{2.738733in}}%
\pgfpathlineto{\pgfqpoint{3.676233in}{2.749461in}}%
\pgfpathlineto{\pgfqpoint{3.663318in}{2.760343in}}%
\pgfpathlineto{\pgfqpoint{3.650404in}{2.771382in}}%
\pgfpathlineto{\pgfqpoint{3.642793in}{2.761706in}}%
\pgfpathlineto{\pgfqpoint{3.635177in}{2.752098in}}%
\pgfpathlineto{\pgfqpoint{3.627555in}{2.742559in}}%
\pgfpathlineto{\pgfqpoint{3.619928in}{2.733087in}}%
\pgfpathclose%
\pgfusepath{fill}%
\end{pgfscope}%
\begin{pgfscope}%
\pgfpathrectangle{\pgfqpoint{1.254980in}{0.150000in}}{\pgfqpoint{5.490039in}{5.490039in}}%
\pgfusepath{clip}%
\pgfsetbuttcap%
\pgfsetroundjoin%
\definecolor{currentfill}{rgb}{0.233603,0.313828,0.543914}%
\pgfsetfillcolor{currentfill}%
\pgfsetfillopacity{0.700000}%
\pgfsetlinewidth{0.000000pt}%
\definecolor{currentstroke}{rgb}{0.000000,0.000000,0.000000}%
\pgfsetstrokecolor{currentstroke}%
\pgfsetdash{}{0pt}%
\pgfpathmoveto{\pgfqpoint{5.179199in}{2.845751in}}%
\pgfpathlineto{\pgfqpoint{5.192470in}{2.845224in}}%
\pgfpathlineto{\pgfqpoint{5.205752in}{2.844812in}}%
\pgfpathlineto{\pgfqpoint{5.219043in}{2.844515in}}%
\pgfpathlineto{\pgfqpoint{5.232345in}{2.844334in}}%
\pgfpathlineto{\pgfqpoint{5.239472in}{2.854346in}}%
\pgfpathlineto{\pgfqpoint{5.246595in}{2.864393in}}%
\pgfpathlineto{\pgfqpoint{5.253715in}{2.874479in}}%
\pgfpathlineto{\pgfqpoint{5.260832in}{2.884603in}}%
\pgfpathlineto{\pgfqpoint{5.247543in}{2.884981in}}%
\pgfpathlineto{\pgfqpoint{5.234265in}{2.885474in}}%
\pgfpathlineto{\pgfqpoint{5.220996in}{2.886081in}}%
\pgfpathlineto{\pgfqpoint{5.207738in}{2.886805in}}%
\pgfpathlineto{\pgfqpoint{5.200608in}{2.876478in}}%
\pgfpathlineto{\pgfqpoint{5.193475in}{2.866195in}}%
\pgfpathlineto{\pgfqpoint{5.186339in}{2.855953in}}%
\pgfpathlineto{\pgfqpoint{5.179199in}{2.845751in}}%
\pgfpathclose%
\pgfusepath{fill}%
\end{pgfscope}%
\begin{pgfscope}%
\pgfpathrectangle{\pgfqpoint{1.254980in}{0.150000in}}{\pgfqpoint{5.490039in}{5.490039in}}%
\pgfusepath{clip}%
\pgfsetbuttcap%
\pgfsetroundjoin%
\definecolor{currentfill}{rgb}{0.270595,0.214069,0.507052}%
\pgfsetfillcolor{currentfill}%
\pgfsetfillopacity{0.700000}%
\pgfsetlinewidth{0.000000pt}%
\definecolor{currentstroke}{rgb}{0.000000,0.000000,0.000000}%
\pgfsetstrokecolor{currentstroke}%
\pgfsetdash{}{0pt}%
\pgfpathmoveto{\pgfqpoint{4.636936in}{2.647206in}}%
\pgfpathlineto{\pgfqpoint{4.650028in}{2.644174in}}%
\pgfpathlineto{\pgfqpoint{4.663127in}{2.641267in}}%
\pgfpathlineto{\pgfqpoint{4.676235in}{2.638482in}}%
\pgfpathlineto{\pgfqpoint{4.689350in}{2.635821in}}%
\pgfpathlineto{\pgfqpoint{4.696648in}{2.646168in}}%
\pgfpathlineto{\pgfqpoint{4.703942in}{2.656539in}}%
\pgfpathlineto{\pgfqpoint{4.711232in}{2.666936in}}%
\pgfpathlineto{\pgfqpoint{4.718518in}{2.677360in}}%
\pgfpathlineto{\pgfqpoint{4.705414in}{2.680123in}}%
\pgfpathlineto{\pgfqpoint{4.692317in}{2.683008in}}%
\pgfpathlineto{\pgfqpoint{4.679227in}{2.686016in}}%
\pgfpathlineto{\pgfqpoint{4.666145in}{2.689148in}}%
\pgfpathlineto{\pgfqpoint{4.658849in}{2.678617in}}%
\pgfpathlineto{\pgfqpoint{4.651548in}{2.668117in}}%
\pgfpathlineto{\pgfqpoint{4.644244in}{2.657647in}}%
\pgfpathlineto{\pgfqpoint{4.636936in}{2.647206in}}%
\pgfpathclose%
\pgfusepath{fill}%
\end{pgfscope}%
\begin{pgfscope}%
\pgfpathrectangle{\pgfqpoint{1.254980in}{0.150000in}}{\pgfqpoint{5.490039in}{5.490039in}}%
\pgfusepath{clip}%
\pgfsetbuttcap%
\pgfsetroundjoin%
\definecolor{currentfill}{rgb}{0.241237,0.296485,0.539709}%
\pgfsetfillcolor{currentfill}%
\pgfsetfillopacity{0.700000}%
\pgfsetlinewidth{0.000000pt}%
\definecolor{currentstroke}{rgb}{0.000000,0.000000,0.000000}%
\pgfsetstrokecolor{currentstroke}%
\pgfsetdash{}{0pt}%
\pgfpathmoveto{\pgfqpoint{5.097569in}{2.807848in}}%
\pgfpathlineto{\pgfqpoint{5.110813in}{2.807036in}}%
\pgfpathlineto{\pgfqpoint{5.124067in}{2.806340in}}%
\pgfpathlineto{\pgfqpoint{5.137331in}{2.805761in}}%
\pgfpathlineto{\pgfqpoint{5.150605in}{2.805298in}}%
\pgfpathlineto{\pgfqpoint{5.157759in}{2.815362in}}%
\pgfpathlineto{\pgfqpoint{5.164909in}{2.825457in}}%
\pgfpathlineto{\pgfqpoint{5.172056in}{2.835587in}}%
\pgfpathlineto{\pgfqpoint{5.179199in}{2.845751in}}%
\pgfpathlineto{\pgfqpoint{5.165938in}{2.846395in}}%
\pgfpathlineto{\pgfqpoint{5.152686in}{2.847154in}}%
\pgfpathlineto{\pgfqpoint{5.139444in}{2.848030in}}%
\pgfpathlineto{\pgfqpoint{5.126212in}{2.849022in}}%
\pgfpathlineto{\pgfqpoint{5.119057in}{2.838671in}}%
\pgfpathlineto{\pgfqpoint{5.111898in}{2.828360in}}%
\pgfpathlineto{\pgfqpoint{5.104735in}{2.818086in}}%
\pgfpathlineto{\pgfqpoint{5.097569in}{2.807848in}}%
\pgfpathclose%
\pgfusepath{fill}%
\end{pgfscope}%
\begin{pgfscope}%
\pgfpathrectangle{\pgfqpoint{1.254980in}{0.150000in}}{\pgfqpoint{5.490039in}{5.490039in}}%
\pgfusepath{clip}%
\pgfsetbuttcap%
\pgfsetroundjoin%
\definecolor{currentfill}{rgb}{0.277134,0.185228,0.489898}%
\pgfsetfillcolor{currentfill}%
\pgfsetfillopacity{0.700000}%
\pgfsetlinewidth{0.000000pt}%
\definecolor{currentstroke}{rgb}{0.000000,0.000000,0.000000}%
\pgfsetstrokecolor{currentstroke}%
\pgfsetdash{}{0pt}%
\pgfpathmoveto{\pgfqpoint{4.339895in}{2.586043in}}%
\pgfpathlineto{\pgfqpoint{4.352910in}{2.581190in}}%
\pgfpathlineto{\pgfqpoint{4.365932in}{2.576468in}}%
\pgfpathlineto{\pgfqpoint{4.378959in}{2.571874in}}%
\pgfpathlineto{\pgfqpoint{4.391993in}{2.567410in}}%
\pgfpathlineto{\pgfqpoint{4.399382in}{2.577731in}}%
\pgfpathlineto{\pgfqpoint{4.406767in}{2.588082in}}%
\pgfpathlineto{\pgfqpoint{4.414147in}{2.598461in}}%
\pgfpathlineto{\pgfqpoint{4.421523in}{2.608871in}}%
\pgfpathlineto{\pgfqpoint{4.408500in}{2.613389in}}%
\pgfpathlineto{\pgfqpoint{4.395483in}{2.618035in}}%
\pgfpathlineto{\pgfqpoint{4.382472in}{2.622811in}}%
\pgfpathlineto{\pgfqpoint{4.369467in}{2.627717in}}%
\pgfpathlineto{\pgfqpoint{4.362080in}{2.617248in}}%
\pgfpathlineto{\pgfqpoint{4.354689in}{2.606813in}}%
\pgfpathlineto{\pgfqpoint{4.347294in}{2.596411in}}%
\pgfpathlineto{\pgfqpoint{4.339895in}{2.586043in}}%
\pgfpathclose%
\pgfusepath{fill}%
\end{pgfscope}%
\begin{pgfscope}%
\pgfpathrectangle{\pgfqpoint{1.254980in}{0.150000in}}{\pgfqpoint{5.490039in}{5.490039in}}%
\pgfusepath{clip}%
\pgfsetbuttcap%
\pgfsetroundjoin%
\definecolor{currentfill}{rgb}{0.248629,0.278775,0.534556}%
\pgfsetfillcolor{currentfill}%
\pgfsetfillopacity{0.700000}%
\pgfsetlinewidth{0.000000pt}%
\definecolor{currentstroke}{rgb}{0.000000,0.000000,0.000000}%
\pgfsetstrokecolor{currentstroke}%
\pgfsetdash{}{0pt}%
\pgfpathmoveto{\pgfqpoint{5.015940in}{2.770976in}}%
\pgfpathlineto{\pgfqpoint{5.029158in}{2.769859in}}%
\pgfpathlineto{\pgfqpoint{5.042385in}{2.768860in}}%
\pgfpathlineto{\pgfqpoint{5.055622in}{2.767979in}}%
\pgfpathlineto{\pgfqpoint{5.068869in}{2.767214in}}%
\pgfpathlineto{\pgfqpoint{5.076049in}{2.777328in}}%
\pgfpathlineto{\pgfqpoint{5.083226in}{2.787471in}}%
\pgfpathlineto{\pgfqpoint{5.090400in}{2.797643in}}%
\pgfpathlineto{\pgfqpoint{5.097569in}{2.807848in}}%
\pgfpathlineto{\pgfqpoint{5.084335in}{2.808777in}}%
\pgfpathlineto{\pgfqpoint{5.071109in}{2.809823in}}%
\pgfpathlineto{\pgfqpoint{5.057894in}{2.810986in}}%
\pgfpathlineto{\pgfqpoint{5.044687in}{2.812267in}}%
\pgfpathlineto{\pgfqpoint{5.037506in}{2.801892in}}%
\pgfpathlineto{\pgfqpoint{5.030321in}{2.791554in}}%
\pgfpathlineto{\pgfqpoint{5.023132in}{2.781249in}}%
\pgfpathlineto{\pgfqpoint{5.015940in}{2.770976in}}%
\pgfpathclose%
\pgfusepath{fill}%
\end{pgfscope}%
\begin{pgfscope}%
\pgfpathrectangle{\pgfqpoint{1.254980in}{0.150000in}}{\pgfqpoint{5.490039in}{5.490039in}}%
\pgfusepath{clip}%
\pgfsetbuttcap%
\pgfsetroundjoin%
\definecolor{currentfill}{rgb}{0.206756,0.371758,0.553117}%
\pgfsetfillcolor{currentfill}%
\pgfsetfillopacity{0.700000}%
\pgfsetlinewidth{0.000000pt}%
\definecolor{currentstroke}{rgb}{0.000000,0.000000,0.000000}%
\pgfsetstrokecolor{currentstroke}%
\pgfsetdash{}{0pt}%
\pgfpathmoveto{\pgfqpoint{3.278550in}{3.011519in}}%
\pgfpathlineto{\pgfqpoint{3.291520in}{2.996403in}}%
\pgfpathlineto{\pgfqpoint{3.304487in}{2.981464in}}%
\pgfpathlineto{\pgfqpoint{3.317452in}{2.966703in}}%
\pgfpathlineto{\pgfqpoint{3.330414in}{2.952118in}}%
\pgfpathlineto{\pgfqpoint{3.338144in}{2.961067in}}%
\pgfpathlineto{\pgfqpoint{3.345868in}{2.970104in}}%
\pgfpathlineto{\pgfqpoint{3.353586in}{2.979229in}}%
\pgfpathlineto{\pgfqpoint{3.361297in}{2.988441in}}%
\pgfpathlineto{\pgfqpoint{3.348352in}{3.002981in}}%
\pgfpathlineto{\pgfqpoint{3.335404in}{3.017697in}}%
\pgfpathlineto{\pgfqpoint{3.322454in}{3.032589in}}%
\pgfpathlineto{\pgfqpoint{3.309502in}{3.047660in}}%
\pgfpathlineto{\pgfqpoint{3.301774in}{3.038487in}}%
\pgfpathlineto{\pgfqpoint{3.294039in}{3.029406in}}%
\pgfpathlineto{\pgfqpoint{3.286298in}{3.020416in}}%
\pgfpathlineto{\pgfqpoint{3.278550in}{3.011519in}}%
\pgfpathclose%
\pgfusepath{fill}%
\end{pgfscope}%
\begin{pgfscope}%
\pgfpathrectangle{\pgfqpoint{1.254980in}{0.150000in}}{\pgfqpoint{5.490039in}{5.490039in}}%
\pgfusepath{clip}%
\pgfsetbuttcap%
\pgfsetroundjoin%
\definecolor{currentfill}{rgb}{0.277134,0.185228,0.489898}%
\pgfsetfillcolor{currentfill}%
\pgfsetfillopacity{0.700000}%
\pgfsetlinewidth{0.000000pt}%
\definecolor{currentstroke}{rgb}{0.000000,0.000000,0.000000}%
\pgfsetstrokecolor{currentstroke}%
\pgfsetdash{}{0pt}%
\pgfpathmoveto{\pgfqpoint{3.990876in}{2.587649in}}%
\pgfpathlineto{\pgfqpoint{4.003829in}{2.580166in}}%
\pgfpathlineto{\pgfqpoint{4.016786in}{2.572823in}}%
\pgfpathlineto{\pgfqpoint{4.029747in}{2.565619in}}%
\pgfpathlineto{\pgfqpoint{4.042712in}{2.558553in}}%
\pgfpathlineto{\pgfqpoint{4.050208in}{2.568578in}}%
\pgfpathlineto{\pgfqpoint{4.057700in}{2.578644in}}%
\pgfpathlineto{\pgfqpoint{4.065187in}{2.588753in}}%
\pgfpathlineto{\pgfqpoint{4.072670in}{2.598904in}}%
\pgfpathlineto{\pgfqpoint{4.059717in}{2.605975in}}%
\pgfpathlineto{\pgfqpoint{4.046767in}{2.613185in}}%
\pgfpathlineto{\pgfqpoint{4.033822in}{2.620534in}}%
\pgfpathlineto{\pgfqpoint{4.020880in}{2.628022in}}%
\pgfpathlineto{\pgfqpoint{4.013386in}{2.617860in}}%
\pgfpathlineto{\pgfqpoint{4.005888in}{2.607744in}}%
\pgfpathlineto{\pgfqpoint{3.998384in}{2.597674in}}%
\pgfpathlineto{\pgfqpoint{3.990876in}{2.587649in}}%
\pgfpathclose%
\pgfusepath{fill}%
\end{pgfscope}%
\begin{pgfscope}%
\pgfpathrectangle{\pgfqpoint{1.254980in}{0.150000in}}{\pgfqpoint{5.490039in}{5.490039in}}%
\pgfusepath{clip}%
\pgfsetbuttcap%
\pgfsetroundjoin%
\definecolor{currentfill}{rgb}{0.194100,0.399323,0.555565}%
\pgfsetfillcolor{currentfill}%
\pgfsetfillopacity{0.700000}%
\pgfsetlinewidth{0.000000pt}%
\definecolor{currentstroke}{rgb}{0.000000,0.000000,0.000000}%
\pgfsetstrokecolor{currentstroke}%
\pgfsetdash{}{0pt}%
\pgfpathmoveto{\pgfqpoint{3.226639in}{3.073789in}}%
\pgfpathlineto{\pgfqpoint{3.239622in}{3.057949in}}%
\pgfpathlineto{\pgfqpoint{3.252601in}{3.042291in}}%
\pgfpathlineto{\pgfqpoint{3.265577in}{3.026815in}}%
\pgfpathlineto{\pgfqpoint{3.278550in}{3.011519in}}%
\pgfpathlineto{\pgfqpoint{3.286298in}{3.020416in}}%
\pgfpathlineto{\pgfqpoint{3.294039in}{3.029406in}}%
\pgfpathlineto{\pgfqpoint{3.301774in}{3.038487in}}%
\pgfpathlineto{\pgfqpoint{3.309502in}{3.047660in}}%
\pgfpathlineto{\pgfqpoint{3.296547in}{3.062910in}}%
\pgfpathlineto{\pgfqpoint{3.283588in}{3.078340in}}%
\pgfpathlineto{\pgfqpoint{3.270627in}{3.093951in}}%
\pgfpathlineto{\pgfqpoint{3.257663in}{3.109745in}}%
\pgfpathlineto{\pgfqpoint{3.249917in}{3.100612in}}%
\pgfpathlineto{\pgfqpoint{3.242165in}{3.091575in}}%
\pgfpathlineto{\pgfqpoint{3.234405in}{3.082634in}}%
\pgfpathlineto{\pgfqpoint{3.226639in}{3.073789in}}%
\pgfpathclose%
\pgfusepath{fill}%
\end{pgfscope}%
\begin{pgfscope}%
\pgfpathrectangle{\pgfqpoint{1.254980in}{0.150000in}}{\pgfqpoint{5.490039in}{5.490039in}}%
\pgfusepath{clip}%
\pgfsetbuttcap%
\pgfsetroundjoin%
\definecolor{currentfill}{rgb}{0.273006,0.204520,0.501721}%
\pgfsetfillcolor{currentfill}%
\pgfsetfillopacity{0.700000}%
\pgfsetlinewidth{0.000000pt}%
\definecolor{currentstroke}{rgb}{0.000000,0.000000,0.000000}%
\pgfsetstrokecolor{currentstroke}%
\pgfsetdash{}{0pt}%
\pgfpathmoveto{\pgfqpoint{4.555325in}{2.618732in}}%
\pgfpathlineto{\pgfqpoint{4.568398in}{2.615287in}}%
\pgfpathlineto{\pgfqpoint{4.581479in}{2.611968in}}%
\pgfpathlineto{\pgfqpoint{4.594566in}{2.608773in}}%
\pgfpathlineto{\pgfqpoint{4.607661in}{2.605703in}}%
\pgfpathlineto{\pgfqpoint{4.614986in}{2.616041in}}%
\pgfpathlineto{\pgfqpoint{4.622307in}{2.626404in}}%
\pgfpathlineto{\pgfqpoint{4.629623in}{2.636792in}}%
\pgfpathlineto{\pgfqpoint{4.636936in}{2.647206in}}%
\pgfpathlineto{\pgfqpoint{4.623851in}{2.650361in}}%
\pgfpathlineto{\pgfqpoint{4.610774in}{2.653640in}}%
\pgfpathlineto{\pgfqpoint{4.597704in}{2.657045in}}%
\pgfpathlineto{\pgfqpoint{4.584641in}{2.660574in}}%
\pgfpathlineto{\pgfqpoint{4.577318in}{2.650069in}}%
\pgfpathlineto{\pgfqpoint{4.569991in}{2.639594in}}%
\pgfpathlineto{\pgfqpoint{4.562660in}{2.629149in}}%
\pgfpathlineto{\pgfqpoint{4.555325in}{2.618732in}}%
\pgfpathclose%
\pgfusepath{fill}%
\end{pgfscope}%
\begin{pgfscope}%
\pgfpathrectangle{\pgfqpoint{1.254980in}{0.150000in}}{\pgfqpoint{5.490039in}{5.490039in}}%
\pgfusepath{clip}%
\pgfsetbuttcap%
\pgfsetroundjoin%
\definecolor{currentfill}{rgb}{0.218130,0.347432,0.550038}%
\pgfsetfillcolor{currentfill}%
\pgfsetfillopacity{0.700000}%
\pgfsetlinewidth{0.000000pt}%
\definecolor{currentstroke}{rgb}{0.000000,0.000000,0.000000}%
\pgfsetstrokecolor{currentstroke}%
\pgfsetdash{}{0pt}%
\pgfpathmoveto{\pgfqpoint{3.330414in}{2.952118in}}%
\pgfpathlineto{\pgfqpoint{3.343374in}{2.937708in}}%
\pgfpathlineto{\pgfqpoint{3.356332in}{2.923472in}}%
\pgfpathlineto{\pgfqpoint{3.369287in}{2.909408in}}%
\pgfpathlineto{\pgfqpoint{3.382241in}{2.895517in}}%
\pgfpathlineto{\pgfqpoint{3.389955in}{2.904517in}}%
\pgfpathlineto{\pgfqpoint{3.397662in}{2.913602in}}%
\pgfpathlineto{\pgfqpoint{3.405362in}{2.922770in}}%
\pgfpathlineto{\pgfqpoint{3.413057in}{2.932021in}}%
\pgfpathlineto{\pgfqpoint{3.400120in}{2.945868in}}%
\pgfpathlineto{\pgfqpoint{3.387181in}{2.959886in}}%
\pgfpathlineto{\pgfqpoint{3.374240in}{2.974077in}}%
\pgfpathlineto{\pgfqpoint{3.361297in}{2.988441in}}%
\pgfpathlineto{\pgfqpoint{3.353586in}{2.979229in}}%
\pgfpathlineto{\pgfqpoint{3.345868in}{2.970104in}}%
\pgfpathlineto{\pgfqpoint{3.338144in}{2.961067in}}%
\pgfpathlineto{\pgfqpoint{3.330414in}{2.952118in}}%
\pgfpathclose%
\pgfusepath{fill}%
\end{pgfscope}%
\begin{pgfscope}%
\pgfpathrectangle{\pgfqpoint{1.254980in}{0.150000in}}{\pgfqpoint{5.490039in}{5.490039in}}%
\pgfusepath{clip}%
\pgfsetbuttcap%
\pgfsetroundjoin%
\definecolor{currentfill}{rgb}{0.278826,0.175490,0.483397}%
\pgfsetfillcolor{currentfill}%
\pgfsetfillopacity{0.700000}%
\pgfsetlinewidth{0.000000pt}%
\definecolor{currentstroke}{rgb}{0.000000,0.000000,0.000000}%
\pgfsetstrokecolor{currentstroke}%
\pgfsetdash{}{0pt}%
\pgfpathmoveto{\pgfqpoint{4.124524in}{2.571994in}}%
\pgfpathlineto{\pgfqpoint{4.137498in}{2.565607in}}%
\pgfpathlineto{\pgfqpoint{4.150477in}{2.559356in}}%
\pgfpathlineto{\pgfqpoint{4.163461in}{2.553240in}}%
\pgfpathlineto{\pgfqpoint{4.176450in}{2.547259in}}%
\pgfpathlineto{\pgfqpoint{4.183905in}{2.557421in}}%
\pgfpathlineto{\pgfqpoint{4.191357in}{2.567620in}}%
\pgfpathlineto{\pgfqpoint{4.198804in}{2.577854in}}%
\pgfpathlineto{\pgfqpoint{4.206246in}{2.588125in}}%
\pgfpathlineto{\pgfqpoint{4.193268in}{2.594129in}}%
\pgfpathlineto{\pgfqpoint{4.180295in}{2.600266in}}%
\pgfpathlineto{\pgfqpoint{4.167327in}{2.606539in}}%
\pgfpathlineto{\pgfqpoint{4.154364in}{2.612947in}}%
\pgfpathlineto{\pgfqpoint{4.146911in}{2.602648in}}%
\pgfpathlineto{\pgfqpoint{4.139453in}{2.592390in}}%
\pgfpathlineto{\pgfqpoint{4.131991in}{2.582172in}}%
\pgfpathlineto{\pgfqpoint{4.124524in}{2.571994in}}%
\pgfpathclose%
\pgfusepath{fill}%
\end{pgfscope}%
\begin{pgfscope}%
\pgfpathrectangle{\pgfqpoint{1.254980in}{0.150000in}}{\pgfqpoint{5.490039in}{5.490039in}}%
\pgfusepath{clip}%
\pgfsetbuttcap%
\pgfsetroundjoin%
\definecolor{currentfill}{rgb}{0.265145,0.232956,0.516599}%
\pgfsetfillcolor{currentfill}%
\pgfsetfillopacity{0.700000}%
\pgfsetlinewidth{0.000000pt}%
\definecolor{currentstroke}{rgb}{0.000000,0.000000,0.000000}%
\pgfsetstrokecolor{currentstroke}%
\pgfsetdash{}{0pt}%
\pgfpathmoveto{\pgfqpoint{3.671646in}{2.689756in}}%
\pgfpathlineto{\pgfqpoint{3.684578in}{2.679309in}}%
\pgfpathlineto{\pgfqpoint{3.697511in}{2.669015in}}%
\pgfpathlineto{\pgfqpoint{3.710445in}{2.658874in}}%
\pgfpathlineto{\pgfqpoint{3.723380in}{2.648883in}}%
\pgfpathlineto{\pgfqpoint{3.730979in}{2.658421in}}%
\pgfpathlineto{\pgfqpoint{3.738573in}{2.668019in}}%
\pgfpathlineto{\pgfqpoint{3.746162in}{2.677677in}}%
\pgfpathlineto{\pgfqpoint{3.753745in}{2.687396in}}%
\pgfpathlineto{\pgfqpoint{3.740823in}{2.697359in}}%
\pgfpathlineto{\pgfqpoint{3.727902in}{2.707474in}}%
\pgfpathlineto{\pgfqpoint{3.714983in}{2.717741in}}%
\pgfpathlineto{\pgfqpoint{3.702065in}{2.728160in}}%
\pgfpathlineto{\pgfqpoint{3.694469in}{2.718463in}}%
\pgfpathlineto{\pgfqpoint{3.686866in}{2.708830in}}%
\pgfpathlineto{\pgfqpoint{3.679259in}{2.699261in}}%
\pgfpathlineto{\pgfqpoint{3.671646in}{2.689756in}}%
\pgfpathclose%
\pgfusepath{fill}%
\end{pgfscope}%
\begin{pgfscope}%
\pgfpathrectangle{\pgfqpoint{1.254980in}{0.150000in}}{\pgfqpoint{5.490039in}{5.490039in}}%
\pgfusepath{clip}%
\pgfsetbuttcap%
\pgfsetroundjoin%
\definecolor{currentfill}{rgb}{0.183898,0.422383,0.556944}%
\pgfsetfillcolor{currentfill}%
\pgfsetfillopacity{0.700000}%
\pgfsetlinewidth{0.000000pt}%
\definecolor{currentstroke}{rgb}{0.000000,0.000000,0.000000}%
\pgfsetstrokecolor{currentstroke}%
\pgfsetdash{}{0pt}%
\pgfpathmoveto{\pgfqpoint{3.174673in}{3.138999in}}%
\pgfpathlineto{\pgfqpoint{3.187671in}{3.122417in}}%
\pgfpathlineto{\pgfqpoint{3.200664in}{3.106022in}}%
\pgfpathlineto{\pgfqpoint{3.213654in}{3.089813in}}%
\pgfpathlineto{\pgfqpoint{3.226639in}{3.073789in}}%
\pgfpathlineto{\pgfqpoint{3.234405in}{3.082634in}}%
\pgfpathlineto{\pgfqpoint{3.242165in}{3.091575in}}%
\pgfpathlineto{\pgfqpoint{3.249917in}{3.100612in}}%
\pgfpathlineto{\pgfqpoint{3.257663in}{3.109745in}}%
\pgfpathlineto{\pgfqpoint{3.244695in}{3.125723in}}%
\pgfpathlineto{\pgfqpoint{3.231724in}{3.141885in}}%
\pgfpathlineto{\pgfqpoint{3.218749in}{3.158233in}}%
\pgfpathlineto{\pgfqpoint{3.205770in}{3.174769in}}%
\pgfpathlineto{\pgfqpoint{3.198006in}{3.165676in}}%
\pgfpathlineto{\pgfqpoint{3.190235in}{3.156684in}}%
\pgfpathlineto{\pgfqpoint{3.182458in}{3.147791in}}%
\pgfpathlineto{\pgfqpoint{3.174673in}{3.138999in}}%
\pgfpathclose%
\pgfusepath{fill}%
\end{pgfscope}%
\begin{pgfscope}%
\pgfpathrectangle{\pgfqpoint{1.254980in}{0.150000in}}{\pgfqpoint{5.490039in}{5.490039in}}%
\pgfusepath{clip}%
\pgfsetbuttcap%
\pgfsetroundjoin%
\definecolor{currentfill}{rgb}{0.253935,0.265254,0.529983}%
\pgfsetfillcolor{currentfill}%
\pgfsetfillopacity{0.700000}%
\pgfsetlinewidth{0.000000pt}%
\definecolor{currentstroke}{rgb}{0.000000,0.000000,0.000000}%
\pgfsetstrokecolor{currentstroke}%
\pgfsetdash{}{0pt}%
\pgfpathmoveto{\pgfqpoint{4.934307in}{2.735230in}}%
\pgfpathlineto{\pgfqpoint{4.947500in}{2.733788in}}%
\pgfpathlineto{\pgfqpoint{4.960701in}{2.732466in}}%
\pgfpathlineto{\pgfqpoint{4.973912in}{2.731261in}}%
\pgfpathlineto{\pgfqpoint{4.987132in}{2.730175in}}%
\pgfpathlineto{\pgfqpoint{4.994340in}{2.740335in}}%
\pgfpathlineto{\pgfqpoint{5.001544in}{2.750521in}}%
\pgfpathlineto{\pgfqpoint{5.008744in}{2.760734in}}%
\pgfpathlineto{\pgfqpoint{5.015940in}{2.770976in}}%
\pgfpathlineto{\pgfqpoint{5.002731in}{2.772211in}}%
\pgfpathlineto{\pgfqpoint{4.989531in}{2.773564in}}%
\pgfpathlineto{\pgfqpoint{4.976341in}{2.775035in}}%
\pgfpathlineto{\pgfqpoint{4.963160in}{2.776624in}}%
\pgfpathlineto{\pgfqpoint{4.955952in}{2.766228in}}%
\pgfpathlineto{\pgfqpoint{4.948741in}{2.755865in}}%
\pgfpathlineto{\pgfqpoint{4.941526in}{2.745533in}}%
\pgfpathlineto{\pgfqpoint{4.934307in}{2.735230in}}%
\pgfpathclose%
\pgfusepath{fill}%
\end{pgfscope}%
\begin{pgfscope}%
\pgfpathrectangle{\pgfqpoint{1.254980in}{0.150000in}}{\pgfqpoint{5.490039in}{5.490039in}}%
\pgfusepath{clip}%
\pgfsetbuttcap%
\pgfsetroundjoin%
\definecolor{currentfill}{rgb}{0.227802,0.326594,0.546532}%
\pgfsetfillcolor{currentfill}%
\pgfsetfillopacity{0.700000}%
\pgfsetlinewidth{0.000000pt}%
\definecolor{currentstroke}{rgb}{0.000000,0.000000,0.000000}%
\pgfsetstrokecolor{currentstroke}%
\pgfsetdash{}{0pt}%
\pgfpathmoveto{\pgfqpoint{3.382241in}{2.895517in}}%
\pgfpathlineto{\pgfqpoint{3.395194in}{2.881797in}}%
\pgfpathlineto{\pgfqpoint{3.408144in}{2.868247in}}%
\pgfpathlineto{\pgfqpoint{3.421093in}{2.854866in}}%
\pgfpathlineto{\pgfqpoint{3.434041in}{2.841653in}}%
\pgfpathlineto{\pgfqpoint{3.441738in}{2.850703in}}%
\pgfpathlineto{\pgfqpoint{3.449428in}{2.859834in}}%
\pgfpathlineto{\pgfqpoint{3.457113in}{2.869045in}}%
\pgfpathlineto{\pgfqpoint{3.464791in}{2.878336in}}%
\pgfpathlineto{\pgfqpoint{3.451860in}{2.891504in}}%
\pgfpathlineto{\pgfqpoint{3.438927in}{2.904841in}}%
\pgfpathlineto{\pgfqpoint{3.425993in}{2.918346in}}%
\pgfpathlineto{\pgfqpoint{3.413057in}{2.932021in}}%
\pgfpathlineto{\pgfqpoint{3.405362in}{2.922770in}}%
\pgfpathlineto{\pgfqpoint{3.397662in}{2.913602in}}%
\pgfpathlineto{\pgfqpoint{3.389955in}{2.904517in}}%
\pgfpathlineto{\pgfqpoint{3.382241in}{2.895517in}}%
\pgfpathclose%
\pgfusepath{fill}%
\end{pgfscope}%
\begin{pgfscope}%
\pgfpathrectangle{\pgfqpoint{1.254980in}{0.150000in}}{\pgfqpoint{5.490039in}{5.490039in}}%
\pgfusepath{clip}%
\pgfsetbuttcap%
\pgfsetroundjoin%
\definecolor{currentfill}{rgb}{0.275191,0.194905,0.496005}%
\pgfsetfillcolor{currentfill}%
\pgfsetfillopacity{0.700000}%
\pgfsetlinewidth{0.000000pt}%
\definecolor{currentstroke}{rgb}{0.000000,0.000000,0.000000}%
\pgfsetstrokecolor{currentstroke}%
\pgfsetdash{}{0pt}%
\pgfpathmoveto{\pgfqpoint{3.857188in}{2.613052in}}%
\pgfpathlineto{\pgfqpoint{3.870129in}{2.604419in}}%
\pgfpathlineto{\pgfqpoint{3.883072in}{2.595931in}}%
\pgfpathlineto{\pgfqpoint{3.896019in}{2.587587in}}%
\pgfpathlineto{\pgfqpoint{3.908968in}{2.579386in}}%
\pgfpathlineto{\pgfqpoint{3.916507in}{2.589214in}}%
\pgfpathlineto{\pgfqpoint{3.924042in}{2.599091in}}%
\pgfpathlineto{\pgfqpoint{3.931572in}{2.609016in}}%
\pgfpathlineto{\pgfqpoint{3.939097in}{2.618991in}}%
\pgfpathlineto{\pgfqpoint{3.926160in}{2.627181in}}%
\pgfpathlineto{\pgfqpoint{3.913226in}{2.635515in}}%
\pgfpathlineto{\pgfqpoint{3.900295in}{2.643992in}}%
\pgfpathlineto{\pgfqpoint{3.887367in}{2.652615in}}%
\pgfpathlineto{\pgfqpoint{3.879830in}{2.642645in}}%
\pgfpathlineto{\pgfqpoint{3.872288in}{2.632728in}}%
\pgfpathlineto{\pgfqpoint{3.864741in}{2.622863in}}%
\pgfpathlineto{\pgfqpoint{3.857188in}{2.613052in}}%
\pgfpathclose%
\pgfusepath{fill}%
\end{pgfscope}%
\begin{pgfscope}%
\pgfpathrectangle{\pgfqpoint{1.254980in}{0.150000in}}{\pgfqpoint{5.490039in}{5.490039in}}%
\pgfusepath{clip}%
\pgfsetbuttcap%
\pgfsetroundjoin%
\definecolor{currentfill}{rgb}{0.171176,0.452530,0.557965}%
\pgfsetfillcolor{currentfill}%
\pgfsetfillopacity{0.700000}%
\pgfsetlinewidth{0.000000pt}%
\definecolor{currentstroke}{rgb}{0.000000,0.000000,0.000000}%
\pgfsetstrokecolor{currentstroke}%
\pgfsetdash{}{0pt}%
\pgfpathmoveto{\pgfqpoint{3.122641in}{3.207225in}}%
\pgfpathlineto{\pgfqpoint{3.135656in}{3.189882in}}%
\pgfpathlineto{\pgfqpoint{3.148666in}{3.172730in}}%
\pgfpathlineto{\pgfqpoint{3.161672in}{3.155770in}}%
\pgfpathlineto{\pgfqpoint{3.174673in}{3.138999in}}%
\pgfpathlineto{\pgfqpoint{3.182458in}{3.147791in}}%
\pgfpathlineto{\pgfqpoint{3.190235in}{3.156684in}}%
\pgfpathlineto{\pgfqpoint{3.198006in}{3.165676in}}%
\pgfpathlineto{\pgfqpoint{3.205770in}{3.174769in}}%
\pgfpathlineto{\pgfqpoint{3.192787in}{3.191493in}}%
\pgfpathlineto{\pgfqpoint{3.179800in}{3.208406in}}%
\pgfpathlineto{\pgfqpoint{3.166808in}{3.225510in}}%
\pgfpathlineto{\pgfqpoint{3.153813in}{3.242806in}}%
\pgfpathlineto{\pgfqpoint{3.146030in}{3.233755in}}%
\pgfpathlineto{\pgfqpoint{3.138241in}{3.224807in}}%
\pgfpathlineto{\pgfqpoint{3.130444in}{3.215964in}}%
\pgfpathlineto{\pgfqpoint{3.122641in}{3.207225in}}%
\pgfpathclose%
\pgfusepath{fill}%
\end{pgfscope}%
\begin{pgfscope}%
\pgfpathrectangle{\pgfqpoint{1.254980in}{0.150000in}}{\pgfqpoint{5.490039in}{5.490039in}}%
\pgfusepath{clip}%
\pgfsetbuttcap%
\pgfsetroundjoin%
\definecolor{currentfill}{rgb}{0.239346,0.300855,0.540844}%
\pgfsetfillcolor{currentfill}%
\pgfsetfillopacity{0.700000}%
\pgfsetlinewidth{0.000000pt}%
\definecolor{currentstroke}{rgb}{0.000000,0.000000,0.000000}%
\pgfsetstrokecolor{currentstroke}%
\pgfsetdash{}{0pt}%
\pgfpathmoveto{\pgfqpoint{3.434041in}{2.841653in}}%
\pgfpathlineto{\pgfqpoint{3.446988in}{2.828607in}}%
\pgfpathlineto{\pgfqpoint{3.459933in}{2.815727in}}%
\pgfpathlineto{\pgfqpoint{3.472878in}{2.803013in}}%
\pgfpathlineto{\pgfqpoint{3.485822in}{2.790463in}}%
\pgfpathlineto{\pgfqpoint{3.493503in}{2.799564in}}%
\pgfpathlineto{\pgfqpoint{3.501178in}{2.808741in}}%
\pgfpathlineto{\pgfqpoint{3.508846in}{2.817994in}}%
\pgfpathlineto{\pgfqpoint{3.516509in}{2.827323in}}%
\pgfpathlineto{\pgfqpoint{3.503581in}{2.839829in}}%
\pgfpathlineto{\pgfqpoint{3.490652in}{2.852499in}}%
\pgfpathlineto{\pgfqpoint{3.477722in}{2.865334in}}%
\pgfpathlineto{\pgfqpoint{3.464791in}{2.878336in}}%
\pgfpathlineto{\pgfqpoint{3.457113in}{2.869045in}}%
\pgfpathlineto{\pgfqpoint{3.449428in}{2.859834in}}%
\pgfpathlineto{\pgfqpoint{3.441738in}{2.850703in}}%
\pgfpathlineto{\pgfqpoint{3.434041in}{2.841653in}}%
\pgfpathclose%
\pgfusepath{fill}%
\end{pgfscope}%
\begin{pgfscope}%
\pgfpathrectangle{\pgfqpoint{1.254980in}{0.150000in}}{\pgfqpoint{5.490039in}{5.490039in}}%
\pgfusepath{clip}%
\pgfsetbuttcap%
\pgfsetroundjoin%
\definecolor{currentfill}{rgb}{0.260571,0.246922,0.522828}%
\pgfsetfillcolor{currentfill}%
\pgfsetfillopacity{0.700000}%
\pgfsetlinewidth{0.000000pt}%
\definecolor{currentstroke}{rgb}{0.000000,0.000000,0.000000}%
\pgfsetstrokecolor{currentstroke}%
\pgfsetdash{}{0pt}%
\pgfpathmoveto{\pgfqpoint{4.852668in}{2.700712in}}%
\pgfpathlineto{\pgfqpoint{4.865836in}{2.698925in}}%
\pgfpathlineto{\pgfqpoint{4.879013in}{2.697259in}}%
\pgfpathlineto{\pgfqpoint{4.892199in}{2.695711in}}%
\pgfpathlineto{\pgfqpoint{4.905393in}{2.694284in}}%
\pgfpathlineto{\pgfqpoint{4.912628in}{2.704484in}}%
\pgfpathlineto{\pgfqpoint{4.919858in}{2.714707in}}%
\pgfpathlineto{\pgfqpoint{4.927085in}{2.724955in}}%
\pgfpathlineto{\pgfqpoint{4.934307in}{2.735230in}}%
\pgfpathlineto{\pgfqpoint{4.921123in}{2.736790in}}%
\pgfpathlineto{\pgfqpoint{4.907949in}{2.738470in}}%
\pgfpathlineto{\pgfqpoint{4.894783in}{2.740269in}}%
\pgfpathlineto{\pgfqpoint{4.881625in}{2.742188in}}%
\pgfpathlineto{\pgfqpoint{4.874392in}{2.731775in}}%
\pgfpathlineto{\pgfqpoint{4.867154in}{2.721392in}}%
\pgfpathlineto{\pgfqpoint{4.859913in}{2.711038in}}%
\pgfpathlineto{\pgfqpoint{4.852668in}{2.700712in}}%
\pgfpathclose%
\pgfusepath{fill}%
\end{pgfscope}%
\begin{pgfscope}%
\pgfpathrectangle{\pgfqpoint{1.254980in}{0.150000in}}{\pgfqpoint{5.490039in}{5.490039in}}%
\pgfusepath{clip}%
\pgfsetbuttcap%
\pgfsetroundjoin%
\definecolor{currentfill}{rgb}{0.278826,0.175490,0.483397}%
\pgfsetfillcolor{currentfill}%
\pgfsetfillopacity{0.700000}%
\pgfsetlinewidth{0.000000pt}%
\definecolor{currentstroke}{rgb}{0.000000,0.000000,0.000000}%
\pgfsetstrokecolor{currentstroke}%
\pgfsetdash{}{0pt}%
\pgfpathmoveto{\pgfqpoint{4.258207in}{2.565446in}}%
\pgfpathlineto{\pgfqpoint{4.271210in}{2.560108in}}%
\pgfpathlineto{\pgfqpoint{4.284219in}{2.554901in}}%
\pgfpathlineto{\pgfqpoint{4.297234in}{2.549825in}}%
\pgfpathlineto{\pgfqpoint{4.310254in}{2.544880in}}%
\pgfpathlineto{\pgfqpoint{4.317671in}{2.555125in}}%
\pgfpathlineto{\pgfqpoint{4.325083in}{2.565400in}}%
\pgfpathlineto{\pgfqpoint{4.332491in}{2.575706in}}%
\pgfpathlineto{\pgfqpoint{4.339895in}{2.586043in}}%
\pgfpathlineto{\pgfqpoint{4.326885in}{2.591025in}}%
\pgfpathlineto{\pgfqpoint{4.313881in}{2.596139in}}%
\pgfpathlineto{\pgfqpoint{4.300883in}{2.601383in}}%
\pgfpathlineto{\pgfqpoint{4.287890in}{2.606759in}}%
\pgfpathlineto{\pgfqpoint{4.280476in}{2.596379in}}%
\pgfpathlineto{\pgfqpoint{4.273057in}{2.586034in}}%
\pgfpathlineto{\pgfqpoint{4.265634in}{2.575723in}}%
\pgfpathlineto{\pgfqpoint{4.258207in}{2.565446in}}%
\pgfpathclose%
\pgfusepath{fill}%
\end{pgfscope}%
\begin{pgfscope}%
\pgfpathrectangle{\pgfqpoint{1.254980in}{0.150000in}}{\pgfqpoint{5.490039in}{5.490039in}}%
\pgfusepath{clip}%
\pgfsetbuttcap%
\pgfsetroundjoin%
\definecolor{currentfill}{rgb}{0.275191,0.194905,0.496005}%
\pgfsetfillcolor{currentfill}%
\pgfsetfillopacity{0.700000}%
\pgfsetlinewidth{0.000000pt}%
\definecolor{currentstroke}{rgb}{0.000000,0.000000,0.000000}%
\pgfsetstrokecolor{currentstroke}%
\pgfsetdash{}{0pt}%
\pgfpathmoveto{\pgfqpoint{4.473680in}{2.592082in}}%
\pgfpathlineto{\pgfqpoint{4.486736in}{2.588203in}}%
\pgfpathlineto{\pgfqpoint{4.499798in}{2.584450in}}%
\pgfpathlineto{\pgfqpoint{4.512867in}{2.580824in}}%
\pgfpathlineto{\pgfqpoint{4.525944in}{2.577323in}}%
\pgfpathlineto{\pgfqpoint{4.533295in}{2.587638in}}%
\pgfpathlineto{\pgfqpoint{4.540643in}{2.597977in}}%
\pgfpathlineto{\pgfqpoint{4.547986in}{2.608341in}}%
\pgfpathlineto{\pgfqpoint{4.555325in}{2.618732in}}%
\pgfpathlineto{\pgfqpoint{4.542259in}{2.622301in}}%
\pgfpathlineto{\pgfqpoint{4.529200in}{2.625997in}}%
\pgfpathlineto{\pgfqpoint{4.516148in}{2.629818in}}%
\pgfpathlineto{\pgfqpoint{4.503103in}{2.633767in}}%
\pgfpathlineto{\pgfqpoint{4.495753in}{2.623301in}}%
\pgfpathlineto{\pgfqpoint{4.488400in}{2.612866in}}%
\pgfpathlineto{\pgfqpoint{4.481042in}{2.602460in}}%
\pgfpathlineto{\pgfqpoint{4.473680in}{2.592082in}}%
\pgfpathclose%
\pgfusepath{fill}%
\end{pgfscope}%
\begin{pgfscope}%
\pgfpathrectangle{\pgfqpoint{1.254980in}{0.150000in}}{\pgfqpoint{5.490039in}{5.490039in}}%
\pgfusepath{clip}%
\pgfsetbuttcap%
\pgfsetroundjoin%
\definecolor{currentfill}{rgb}{0.160665,0.478540,0.558115}%
\pgfsetfillcolor{currentfill}%
\pgfsetfillopacity{0.700000}%
\pgfsetlinewidth{0.000000pt}%
\definecolor{currentstroke}{rgb}{0.000000,0.000000,0.000000}%
\pgfsetstrokecolor{currentstroke}%
\pgfsetdash{}{0pt}%
\pgfpathmoveto{\pgfqpoint{3.070532in}{3.278546in}}%
\pgfpathlineto{\pgfqpoint{3.083567in}{3.260421in}}%
\pgfpathlineto{\pgfqpoint{3.096597in}{3.242494in}}%
\pgfpathlineto{\pgfqpoint{3.109621in}{3.224762in}}%
\pgfpathlineto{\pgfqpoint{3.122641in}{3.207225in}}%
\pgfpathlineto{\pgfqpoint{3.130444in}{3.215964in}}%
\pgfpathlineto{\pgfqpoint{3.138241in}{3.224807in}}%
\pgfpathlineto{\pgfqpoint{3.146030in}{3.233755in}}%
\pgfpathlineto{\pgfqpoint{3.153813in}{3.242806in}}%
\pgfpathlineto{\pgfqpoint{3.140812in}{3.260295in}}%
\pgfpathlineto{\pgfqpoint{3.127807in}{3.277979in}}%
\pgfpathlineto{\pgfqpoint{3.114797in}{3.295859in}}%
\pgfpathlineto{\pgfqpoint{3.101781in}{3.313936in}}%
\pgfpathlineto{\pgfqpoint{3.093980in}{3.304926in}}%
\pgfpathlineto{\pgfqpoint{3.086171in}{3.296025in}}%
\pgfpathlineto{\pgfqpoint{3.078355in}{3.287231in}}%
\pgfpathlineto{\pgfqpoint{3.070532in}{3.278546in}}%
\pgfpathclose%
\pgfusepath{fill}%
\end{pgfscope}%
\begin{pgfscope}%
\pgfpathrectangle{\pgfqpoint{1.254980in}{0.150000in}}{\pgfqpoint{5.490039in}{5.490039in}}%
\pgfusepath{clip}%
\pgfsetbuttcap%
\pgfsetroundjoin%
\definecolor{currentfill}{rgb}{0.246811,0.283237,0.535941}%
\pgfsetfillcolor{currentfill}%
\pgfsetfillopacity{0.700000}%
\pgfsetlinewidth{0.000000pt}%
\definecolor{currentstroke}{rgb}{0.000000,0.000000,0.000000}%
\pgfsetstrokecolor{currentstroke}%
\pgfsetdash{}{0pt}%
\pgfpathmoveto{\pgfqpoint{3.485822in}{2.790463in}}%
\pgfpathlineto{\pgfqpoint{3.498766in}{2.778076in}}%
\pgfpathlineto{\pgfqpoint{3.511709in}{2.765852in}}%
\pgfpathlineto{\pgfqpoint{3.524651in}{2.753790in}}%
\pgfpathlineto{\pgfqpoint{3.537593in}{2.741889in}}%
\pgfpathlineto{\pgfqpoint{3.545258in}{2.751040in}}%
\pgfpathlineto{\pgfqpoint{3.552917in}{2.760263in}}%
\pgfpathlineto{\pgfqpoint{3.560571in}{2.769558in}}%
\pgfpathlineto{\pgfqpoint{3.568218in}{2.778925in}}%
\pgfpathlineto{\pgfqpoint{3.555291in}{2.790782in}}%
\pgfpathlineto{\pgfqpoint{3.542364in}{2.802800in}}%
\pgfpathlineto{\pgfqpoint{3.529437in}{2.814980in}}%
\pgfpathlineto{\pgfqpoint{3.516509in}{2.827323in}}%
\pgfpathlineto{\pgfqpoint{3.508846in}{2.817994in}}%
\pgfpathlineto{\pgfqpoint{3.501178in}{2.808741in}}%
\pgfpathlineto{\pgfqpoint{3.493503in}{2.799564in}}%
\pgfpathlineto{\pgfqpoint{3.485822in}{2.790463in}}%
\pgfpathclose%
\pgfusepath{fill}%
\end{pgfscope}%
\begin{pgfscope}%
\pgfpathrectangle{\pgfqpoint{1.254980in}{0.150000in}}{\pgfqpoint{5.490039in}{5.490039in}}%
\pgfusepath{clip}%
\pgfsetbuttcap%
\pgfsetroundjoin%
\definecolor{currentfill}{rgb}{0.270595,0.214069,0.507052}%
\pgfsetfillcolor{currentfill}%
\pgfsetfillopacity{0.700000}%
\pgfsetlinewidth{0.000000pt}%
\definecolor{currentstroke}{rgb}{0.000000,0.000000,0.000000}%
\pgfsetstrokecolor{currentstroke}%
\pgfsetdash{}{0pt}%
\pgfpathmoveto{\pgfqpoint{3.723380in}{2.648883in}}%
\pgfpathlineto{\pgfqpoint{3.736317in}{2.639044in}}%
\pgfpathlineto{\pgfqpoint{3.749256in}{2.629354in}}%
\pgfpathlineto{\pgfqpoint{3.762197in}{2.619814in}}%
\pgfpathlineto{\pgfqpoint{3.775139in}{2.610422in}}%
\pgfpathlineto{\pgfqpoint{3.782724in}{2.619993in}}%
\pgfpathlineto{\pgfqpoint{3.790305in}{2.629619in}}%
\pgfpathlineto{\pgfqpoint{3.797880in}{2.639302in}}%
\pgfpathlineto{\pgfqpoint{3.805450in}{2.649041in}}%
\pgfpathlineto{\pgfqpoint{3.792521in}{2.658406in}}%
\pgfpathlineto{\pgfqpoint{3.779594in}{2.667920in}}%
\pgfpathlineto{\pgfqpoint{3.766668in}{2.677583in}}%
\pgfpathlineto{\pgfqpoint{3.753745in}{2.687396in}}%
\pgfpathlineto{\pgfqpoint{3.746162in}{2.677677in}}%
\pgfpathlineto{\pgfqpoint{3.738573in}{2.668019in}}%
\pgfpathlineto{\pgfqpoint{3.730979in}{2.658421in}}%
\pgfpathlineto{\pgfqpoint{3.723380in}{2.648883in}}%
\pgfpathclose%
\pgfusepath{fill}%
\end{pgfscope}%
\begin{pgfscope}%
\pgfpathrectangle{\pgfqpoint{1.254980in}{0.150000in}}{\pgfqpoint{5.490039in}{5.490039in}}%
\pgfusepath{clip}%
\pgfsetbuttcap%
\pgfsetroundjoin%
\definecolor{currentfill}{rgb}{0.265145,0.232956,0.516599}%
\pgfsetfillcolor{currentfill}%
\pgfsetfillopacity{0.700000}%
\pgfsetlinewidth{0.000000pt}%
\definecolor{currentstroke}{rgb}{0.000000,0.000000,0.000000}%
\pgfsetstrokecolor{currentstroke}%
\pgfsetdash{}{0pt}%
\pgfpathmoveto{\pgfqpoint{4.771017in}{2.667535in}}%
\pgfpathlineto{\pgfqpoint{4.784162in}{2.665382in}}%
\pgfpathlineto{\pgfqpoint{4.797315in}{2.663351in}}%
\pgfpathlineto{\pgfqpoint{4.810477in}{2.661441in}}%
\pgfpathlineto{\pgfqpoint{4.823647in}{2.659651in}}%
\pgfpathlineto{\pgfqpoint{4.830908in}{2.669882in}}%
\pgfpathlineto{\pgfqpoint{4.838166in}{2.680135in}}%
\pgfpathlineto{\pgfqpoint{4.845419in}{2.690411in}}%
\pgfpathlineto{\pgfqpoint{4.852668in}{2.700712in}}%
\pgfpathlineto{\pgfqpoint{4.839508in}{2.702619in}}%
\pgfpathlineto{\pgfqpoint{4.826357in}{2.704646in}}%
\pgfpathlineto{\pgfqpoint{4.813214in}{2.706794in}}%
\pgfpathlineto{\pgfqpoint{4.800080in}{2.709062in}}%
\pgfpathlineto{\pgfqpoint{4.792820in}{2.698639in}}%
\pgfpathlineto{\pgfqpoint{4.785556in}{2.688244in}}%
\pgfpathlineto{\pgfqpoint{4.778289in}{2.677877in}}%
\pgfpathlineto{\pgfqpoint{4.771017in}{2.667535in}}%
\pgfpathclose%
\pgfusepath{fill}%
\end{pgfscope}%
\begin{pgfscope}%
\pgfpathrectangle{\pgfqpoint{1.254980in}{0.150000in}}{\pgfqpoint{5.490039in}{5.490039in}}%
\pgfusepath{clip}%
\pgfsetbuttcap%
\pgfsetroundjoin%
\definecolor{currentfill}{rgb}{0.278826,0.175490,0.483397}%
\pgfsetfillcolor{currentfill}%
\pgfsetfillopacity{0.700000}%
\pgfsetlinewidth{0.000000pt}%
\definecolor{currentstroke}{rgb}{0.000000,0.000000,0.000000}%
\pgfsetstrokecolor{currentstroke}%
\pgfsetdash{}{0pt}%
\pgfpathmoveto{\pgfqpoint{4.042712in}{2.558553in}}%
\pgfpathlineto{\pgfqpoint{4.055680in}{2.551626in}}%
\pgfpathlineto{\pgfqpoint{4.068653in}{2.544836in}}%
\pgfpathlineto{\pgfqpoint{4.081630in}{2.538183in}}%
\pgfpathlineto{\pgfqpoint{4.094611in}{2.531666in}}%
\pgfpathlineto{\pgfqpoint{4.102096in}{2.541691in}}%
\pgfpathlineto{\pgfqpoint{4.109577in}{2.551753in}}%
\pgfpathlineto{\pgfqpoint{4.117053in}{2.561854in}}%
\pgfpathlineto{\pgfqpoint{4.124524in}{2.571994in}}%
\pgfpathlineto{\pgfqpoint{4.111554in}{2.578516in}}%
\pgfpathlineto{\pgfqpoint{4.098588in}{2.585175in}}%
\pgfpathlineto{\pgfqpoint{4.085627in}{2.591970in}}%
\pgfpathlineto{\pgfqpoint{4.072670in}{2.598904in}}%
\pgfpathlineto{\pgfqpoint{4.065187in}{2.588753in}}%
\pgfpathlineto{\pgfqpoint{4.057700in}{2.578644in}}%
\pgfpathlineto{\pgfqpoint{4.050208in}{2.568578in}}%
\pgfpathlineto{\pgfqpoint{4.042712in}{2.558553in}}%
\pgfpathclose%
\pgfusepath{fill}%
\end{pgfscope}%
\begin{pgfscope}%
\pgfpathrectangle{\pgfqpoint{1.254980in}{0.150000in}}{\pgfqpoint{5.490039in}{5.490039in}}%
\pgfusepath{clip}%
\pgfsetbuttcap%
\pgfsetroundjoin%
\definecolor{currentfill}{rgb}{0.149039,0.508051,0.557250}%
\pgfsetfillcolor{currentfill}%
\pgfsetfillopacity{0.700000}%
\pgfsetlinewidth{0.000000pt}%
\definecolor{currentstroke}{rgb}{0.000000,0.000000,0.000000}%
\pgfsetstrokecolor{currentstroke}%
\pgfsetdash{}{0pt}%
\pgfpathmoveto{\pgfqpoint{3.018336in}{3.353047in}}%
\pgfpathlineto{\pgfqpoint{3.031394in}{3.334119in}}%
\pgfpathlineto{\pgfqpoint{3.044446in}{3.315394in}}%
\pgfpathlineto{\pgfqpoint{3.057492in}{3.296870in}}%
\pgfpathlineto{\pgfqpoint{3.070532in}{3.278546in}}%
\pgfpathlineto{\pgfqpoint{3.078355in}{3.287231in}}%
\pgfpathlineto{\pgfqpoint{3.086171in}{3.296025in}}%
\pgfpathlineto{\pgfqpoint{3.093980in}{3.304926in}}%
\pgfpathlineto{\pgfqpoint{3.101781in}{3.313936in}}%
\pgfpathlineto{\pgfqpoint{3.088761in}{3.332212in}}%
\pgfpathlineto{\pgfqpoint{3.075735in}{3.350687in}}%
\pgfpathlineto{\pgfqpoint{3.062703in}{3.369364in}}%
\pgfpathlineto{\pgfqpoint{3.049665in}{3.388243in}}%
\pgfpathlineto{\pgfqpoint{3.041844in}{3.379275in}}%
\pgfpathlineto{\pgfqpoint{3.034016in}{3.370420in}}%
\pgfpathlineto{\pgfqpoint{3.026180in}{3.361677in}}%
\pgfpathlineto{\pgfqpoint{3.018336in}{3.353047in}}%
\pgfpathclose%
\pgfusepath{fill}%
\end{pgfscope}%
\begin{pgfscope}%
\pgfpathrectangle{\pgfqpoint{1.254980in}{0.150000in}}{\pgfqpoint{5.490039in}{5.490039in}}%
\pgfusepath{clip}%
\pgfsetbuttcap%
\pgfsetroundjoin%
\definecolor{currentfill}{rgb}{0.277134,0.185228,0.489898}%
\pgfsetfillcolor{currentfill}%
\pgfsetfillopacity{0.700000}%
\pgfsetlinewidth{0.000000pt}%
\definecolor{currentstroke}{rgb}{0.000000,0.000000,0.000000}%
\pgfsetstrokecolor{currentstroke}%
\pgfsetdash{}{0pt}%
\pgfpathmoveto{\pgfqpoint{3.908968in}{2.579386in}}%
\pgfpathlineto{\pgfqpoint{3.921920in}{2.571328in}}%
\pgfpathlineto{\pgfqpoint{3.934875in}{2.563412in}}%
\pgfpathlineto{\pgfqpoint{3.947833in}{2.555637in}}%
\pgfpathlineto{\pgfqpoint{3.960795in}{2.548004in}}%
\pgfpathlineto{\pgfqpoint{3.968323in}{2.557848in}}%
\pgfpathlineto{\pgfqpoint{3.975845in}{2.567737in}}%
\pgfpathlineto{\pgfqpoint{3.983363in}{2.577670in}}%
\pgfpathlineto{\pgfqpoint{3.990876in}{2.587649in}}%
\pgfpathlineto{\pgfqpoint{3.977926in}{2.595273in}}%
\pgfpathlineto{\pgfqpoint{3.964980in}{2.603037in}}%
\pgfpathlineto{\pgfqpoint{3.952037in}{2.610943in}}%
\pgfpathlineto{\pgfqpoint{3.939097in}{2.618991in}}%
\pgfpathlineto{\pgfqpoint{3.931572in}{2.609016in}}%
\pgfpathlineto{\pgfqpoint{3.924042in}{2.599091in}}%
\pgfpathlineto{\pgfqpoint{3.916507in}{2.589214in}}%
\pgfpathlineto{\pgfqpoint{3.908968in}{2.579386in}}%
\pgfpathclose%
\pgfusepath{fill}%
\end{pgfscope}%
\begin{pgfscope}%
\pgfpathrectangle{\pgfqpoint{1.254980in}{0.150000in}}{\pgfqpoint{5.490039in}{5.490039in}}%
\pgfusepath{clip}%
\pgfsetbuttcap%
\pgfsetroundjoin%
\definecolor{currentfill}{rgb}{0.277134,0.185228,0.489898}%
\pgfsetfillcolor{currentfill}%
\pgfsetfillopacity{0.700000}%
\pgfsetlinewidth{0.000000pt}%
\definecolor{currentstroke}{rgb}{0.000000,0.000000,0.000000}%
\pgfsetstrokecolor{currentstroke}%
\pgfsetdash{}{0pt}%
\pgfpathmoveto{\pgfqpoint{4.391993in}{2.567410in}}%
\pgfpathlineto{\pgfqpoint{4.405033in}{2.563074in}}%
\pgfpathlineto{\pgfqpoint{4.418079in}{2.558867in}}%
\pgfpathlineto{\pgfqpoint{4.431131in}{2.554787in}}%
\pgfpathlineto{\pgfqpoint{4.444190in}{2.550834in}}%
\pgfpathlineto{\pgfqpoint{4.451569in}{2.561108in}}%
\pgfpathlineto{\pgfqpoint{4.458944in}{2.571407in}}%
\pgfpathlineto{\pgfqpoint{4.466314in}{2.581731in}}%
\pgfpathlineto{\pgfqpoint{4.473680in}{2.592082in}}%
\pgfpathlineto{\pgfqpoint{4.460631in}{2.596088in}}%
\pgfpathlineto{\pgfqpoint{4.447589in}{2.600221in}}%
\pgfpathlineto{\pgfqpoint{4.434553in}{2.604482in}}%
\pgfpathlineto{\pgfqpoint{4.421523in}{2.608871in}}%
\pgfpathlineto{\pgfqpoint{4.414147in}{2.598461in}}%
\pgfpathlineto{\pgfqpoint{4.406767in}{2.588082in}}%
\pgfpathlineto{\pgfqpoint{4.399382in}{2.577731in}}%
\pgfpathlineto{\pgfqpoint{4.391993in}{2.567410in}}%
\pgfpathclose%
\pgfusepath{fill}%
\end{pgfscope}%
\begin{pgfscope}%
\pgfpathrectangle{\pgfqpoint{1.254980in}{0.150000in}}{\pgfqpoint{5.490039in}{5.490039in}}%
\pgfusepath{clip}%
\pgfsetbuttcap%
\pgfsetroundjoin%
\definecolor{currentfill}{rgb}{0.269308,0.218818,0.509577}%
\pgfsetfillcolor{currentfill}%
\pgfsetfillopacity{0.700000}%
\pgfsetlinewidth{0.000000pt}%
\definecolor{currentstroke}{rgb}{0.000000,0.000000,0.000000}%
\pgfsetstrokecolor{currentstroke}%
\pgfsetdash{}{0pt}%
\pgfpathmoveto{\pgfqpoint{4.689350in}{2.635821in}}%
\pgfpathlineto{\pgfqpoint{4.702473in}{2.633282in}}%
\pgfpathlineto{\pgfqpoint{4.715604in}{2.630866in}}%
\pgfpathlineto{\pgfqpoint{4.728743in}{2.628572in}}%
\pgfpathlineto{\pgfqpoint{4.741890in}{2.626399in}}%
\pgfpathlineto{\pgfqpoint{4.749178in}{2.636651in}}%
\pgfpathlineto{\pgfqpoint{4.756462in}{2.646923in}}%
\pgfpathlineto{\pgfqpoint{4.763741in}{2.657217in}}%
\pgfpathlineto{\pgfqpoint{4.771017in}{2.667535in}}%
\pgfpathlineto{\pgfqpoint{4.757880in}{2.669808in}}%
\pgfpathlineto{\pgfqpoint{4.744752in}{2.672204in}}%
\pgfpathlineto{\pgfqpoint{4.731631in}{2.674721in}}%
\pgfpathlineto{\pgfqpoint{4.718518in}{2.677360in}}%
\pgfpathlineto{\pgfqpoint{4.711232in}{2.666936in}}%
\pgfpathlineto{\pgfqpoint{4.703942in}{2.656539in}}%
\pgfpathlineto{\pgfqpoint{4.696648in}{2.646168in}}%
\pgfpathlineto{\pgfqpoint{4.689350in}{2.635821in}}%
\pgfpathclose%
\pgfusepath{fill}%
\end{pgfscope}%
\begin{pgfscope}%
\pgfpathrectangle{\pgfqpoint{1.254980in}{0.150000in}}{\pgfqpoint{5.490039in}{5.490039in}}%
\pgfusepath{clip}%
\pgfsetbuttcap%
\pgfsetroundjoin%
\definecolor{currentfill}{rgb}{0.255645,0.260703,0.528312}%
\pgfsetfillcolor{currentfill}%
\pgfsetfillopacity{0.700000}%
\pgfsetlinewidth{0.000000pt}%
\definecolor{currentstroke}{rgb}{0.000000,0.000000,0.000000}%
\pgfsetstrokecolor{currentstroke}%
\pgfsetdash{}{0pt}%
\pgfpathmoveto{\pgfqpoint{3.537593in}{2.741889in}}%
\pgfpathlineto{\pgfqpoint{3.550536in}{2.730148in}}%
\pgfpathlineto{\pgfqpoint{3.563478in}{2.718566in}}%
\pgfpathlineto{\pgfqpoint{3.576420in}{2.707143in}}%
\pgfpathlineto{\pgfqpoint{3.589363in}{2.695877in}}%
\pgfpathlineto{\pgfqpoint{3.597013in}{2.705078in}}%
\pgfpathlineto{\pgfqpoint{3.604657in}{2.714346in}}%
\pgfpathlineto{\pgfqpoint{3.612295in}{2.723682in}}%
\pgfpathlineto{\pgfqpoint{3.619928in}{2.733087in}}%
\pgfpathlineto{\pgfqpoint{3.607000in}{2.744309in}}%
\pgfpathlineto{\pgfqpoint{3.594073in}{2.755689in}}%
\pgfpathlineto{\pgfqpoint{3.581145in}{2.767227in}}%
\pgfpathlineto{\pgfqpoint{3.568218in}{2.778925in}}%
\pgfpathlineto{\pgfqpoint{3.560571in}{2.769558in}}%
\pgfpathlineto{\pgfqpoint{3.552917in}{2.760263in}}%
\pgfpathlineto{\pgfqpoint{3.545258in}{2.751040in}}%
\pgfpathlineto{\pgfqpoint{3.537593in}{2.741889in}}%
\pgfpathclose%
\pgfusepath{fill}%
\end{pgfscope}%
\begin{pgfscope}%
\pgfpathrectangle{\pgfqpoint{1.254980in}{0.150000in}}{\pgfqpoint{5.490039in}{5.490039in}}%
\pgfusepath{clip}%
\pgfsetbuttcap%
\pgfsetroundjoin%
\definecolor{currentfill}{rgb}{0.214298,0.355619,0.551184}%
\pgfsetfillcolor{currentfill}%
\pgfsetfillopacity{0.700000}%
\pgfsetlinewidth{0.000000pt}%
\definecolor{currentstroke}{rgb}{0.000000,0.000000,0.000000}%
\pgfsetstrokecolor{currentstroke}%
\pgfsetdash{}{0pt}%
\pgfpathmoveto{\pgfqpoint{5.395844in}{2.924942in}}%
\pgfpathlineto{\pgfqpoint{5.409214in}{2.925378in}}%
\pgfpathlineto{\pgfqpoint{5.422595in}{2.925928in}}%
\pgfpathlineto{\pgfqpoint{5.435987in}{2.926591in}}%
\pgfpathlineto{\pgfqpoint{5.449391in}{2.927366in}}%
\pgfpathlineto{\pgfqpoint{5.456449in}{2.937055in}}%
\pgfpathlineto{\pgfqpoint{5.463505in}{2.946784in}}%
\pgfpathlineto{\pgfqpoint{5.470558in}{2.956557in}}%
\pgfpathlineto{\pgfqpoint{5.477607in}{2.966376in}}%
\pgfpathlineto{\pgfqpoint{5.464219in}{2.965830in}}%
\pgfpathlineto{\pgfqpoint{5.450841in}{2.965396in}}%
\pgfpathlineto{\pgfqpoint{5.437475in}{2.965075in}}%
\pgfpathlineto{\pgfqpoint{5.424119in}{2.964867in}}%
\pgfpathlineto{\pgfqpoint{5.417055in}{2.954813in}}%
\pgfpathlineto{\pgfqpoint{5.409987in}{2.944810in}}%
\pgfpathlineto{\pgfqpoint{5.402917in}{2.934853in}}%
\pgfpathlineto{\pgfqpoint{5.395844in}{2.924942in}}%
\pgfpathclose%
\pgfusepath{fill}%
\end{pgfscope}%
\begin{pgfscope}%
\pgfpathrectangle{\pgfqpoint{1.254980in}{0.150000in}}{\pgfqpoint{5.490039in}{5.490039in}}%
\pgfusepath{clip}%
\pgfsetbuttcap%
\pgfsetroundjoin%
\definecolor{currentfill}{rgb}{0.279574,0.170599,0.479997}%
\pgfsetfillcolor{currentfill}%
\pgfsetfillopacity{0.700000}%
\pgfsetlinewidth{0.000000pt}%
\definecolor{currentstroke}{rgb}{0.000000,0.000000,0.000000}%
\pgfsetstrokecolor{currentstroke}%
\pgfsetdash{}{0pt}%
\pgfpathmoveto{\pgfqpoint{4.176450in}{2.547259in}}%
\pgfpathlineto{\pgfqpoint{4.189443in}{2.541411in}}%
\pgfpathlineto{\pgfqpoint{4.202441in}{2.535697in}}%
\pgfpathlineto{\pgfqpoint{4.215445in}{2.530116in}}%
\pgfpathlineto{\pgfqpoint{4.228453in}{2.524667in}}%
\pgfpathlineto{\pgfqpoint{4.235898in}{2.534814in}}%
\pgfpathlineto{\pgfqpoint{4.243339in}{2.544992in}}%
\pgfpathlineto{\pgfqpoint{4.250775in}{2.555203in}}%
\pgfpathlineto{\pgfqpoint{4.258207in}{2.565446in}}%
\pgfpathlineto{\pgfqpoint{4.245209in}{2.570917in}}%
\pgfpathlineto{\pgfqpoint{4.232216in}{2.576520in}}%
\pgfpathlineto{\pgfqpoint{4.219228in}{2.582256in}}%
\pgfpathlineto{\pgfqpoint{4.206246in}{2.588125in}}%
\pgfpathlineto{\pgfqpoint{4.198804in}{2.577854in}}%
\pgfpathlineto{\pgfqpoint{4.191357in}{2.567620in}}%
\pgfpathlineto{\pgfqpoint{4.183905in}{2.557421in}}%
\pgfpathlineto{\pgfqpoint{4.176450in}{2.547259in}}%
\pgfpathclose%
\pgfusepath{fill}%
\end{pgfscope}%
\begin{pgfscope}%
\pgfpathrectangle{\pgfqpoint{1.254980in}{0.150000in}}{\pgfqpoint{5.490039in}{5.490039in}}%
\pgfusepath{clip}%
\pgfsetbuttcap%
\pgfsetroundjoin%
\definecolor{currentfill}{rgb}{0.221989,0.339161,0.548752}%
\pgfsetfillcolor{currentfill}%
\pgfsetfillopacity{0.700000}%
\pgfsetlinewidth{0.000000pt}%
\definecolor{currentstroke}{rgb}{0.000000,0.000000,0.000000}%
\pgfsetstrokecolor{currentstroke}%
\pgfsetdash{}{0pt}%
\pgfpathmoveto{\pgfqpoint{5.314090in}{2.884240in}}%
\pgfpathlineto{\pgfqpoint{5.327431in}{2.884435in}}%
\pgfpathlineto{\pgfqpoint{5.340782in}{2.884744in}}%
\pgfpathlineto{\pgfqpoint{5.354145in}{2.885166in}}%
\pgfpathlineto{\pgfqpoint{5.367518in}{2.885703in}}%
\pgfpathlineto{\pgfqpoint{5.374604in}{2.895456in}}%
\pgfpathlineto{\pgfqpoint{5.381687in}{2.905246in}}%
\pgfpathlineto{\pgfqpoint{5.388767in}{2.915073in}}%
\pgfpathlineto{\pgfqpoint{5.395844in}{2.924942in}}%
\pgfpathlineto{\pgfqpoint{5.382484in}{2.924619in}}%
\pgfpathlineto{\pgfqpoint{5.369136in}{2.924409in}}%
\pgfpathlineto{\pgfqpoint{5.355798in}{2.924312in}}%
\pgfpathlineto{\pgfqpoint{5.342471in}{2.924330in}}%
\pgfpathlineto{\pgfqpoint{5.335381in}{2.914243in}}%
\pgfpathlineto{\pgfqpoint{5.328287in}{2.904200in}}%
\pgfpathlineto{\pgfqpoint{5.321190in}{2.894200in}}%
\pgfpathlineto{\pgfqpoint{5.314090in}{2.884240in}}%
\pgfpathclose%
\pgfusepath{fill}%
\end{pgfscope}%
\begin{pgfscope}%
\pgfpathrectangle{\pgfqpoint{1.254980in}{0.150000in}}{\pgfqpoint{5.490039in}{5.490039in}}%
\pgfusepath{clip}%
\pgfsetbuttcap%
\pgfsetroundjoin%
\definecolor{currentfill}{rgb}{0.206756,0.371758,0.553117}%
\pgfsetfillcolor{currentfill}%
\pgfsetfillopacity{0.700000}%
\pgfsetlinewidth{0.000000pt}%
\definecolor{currentstroke}{rgb}{0.000000,0.000000,0.000000}%
\pgfsetstrokecolor{currentstroke}%
\pgfsetdash{}{0pt}%
\pgfpathmoveto{\pgfqpoint{5.477607in}{2.966376in}}%
\pgfpathlineto{\pgfqpoint{5.491007in}{2.967035in}}%
\pgfpathlineto{\pgfqpoint{5.504418in}{2.967807in}}%
\pgfpathlineto{\pgfqpoint{5.517841in}{2.968690in}}%
\pgfpathlineto{\pgfqpoint{5.531275in}{2.969685in}}%
\pgfpathlineto{\pgfqpoint{5.538306in}{2.979314in}}%
\pgfpathlineto{\pgfqpoint{5.545335in}{2.988989in}}%
\pgfpathlineto{\pgfqpoint{5.552360in}{2.998713in}}%
\pgfpathlineto{\pgfqpoint{5.538938in}{2.997900in}}%
\pgfpathlineto{\pgfqpoint{5.525527in}{2.997199in}}%
\pgfpathlineto{\pgfqpoint{5.512127in}{2.996610in}}%
\pgfpathlineto{\pgfqpoint{5.498739in}{2.996133in}}%
\pgfpathlineto{\pgfqpoint{5.491698in}{2.986162in}}%
\pgfpathlineto{\pgfqpoint{5.484654in}{2.976244in}}%
\pgfpathlineto{\pgfqpoint{5.477607in}{2.966376in}}%
\pgfpathclose%
\pgfusepath{fill}%
\end{pgfscope}%
\begin{pgfscope}%
\pgfpathrectangle{\pgfqpoint{1.254980in}{0.150000in}}{\pgfqpoint{5.490039in}{5.490039in}}%
\pgfusepath{clip}%
\pgfsetbuttcap%
\pgfsetroundjoin%
\definecolor{currentfill}{rgb}{0.229739,0.322361,0.545706}%
\pgfsetfillcolor{currentfill}%
\pgfsetfillopacity{0.700000}%
\pgfsetlinewidth{0.000000pt}%
\definecolor{currentstroke}{rgb}{0.000000,0.000000,0.000000}%
\pgfsetstrokecolor{currentstroke}%
\pgfsetdash{}{0pt}%
\pgfpathmoveto{\pgfqpoint{5.232345in}{2.844334in}}%
\pgfpathlineto{\pgfqpoint{5.245657in}{2.844268in}}%
\pgfpathlineto{\pgfqpoint{5.258979in}{2.844317in}}%
\pgfpathlineto{\pgfqpoint{5.272312in}{2.844480in}}%
\pgfpathlineto{\pgfqpoint{5.285655in}{2.844758in}}%
\pgfpathlineto{\pgfqpoint{5.292769in}{2.854579in}}%
\pgfpathlineto{\pgfqpoint{5.299880in}{2.864431in}}%
\pgfpathlineto{\pgfqpoint{5.306987in}{2.874318in}}%
\pgfpathlineto{\pgfqpoint{5.314090in}{2.884240in}}%
\pgfpathlineto{\pgfqpoint{5.300760in}{2.884159in}}%
\pgfpathlineto{\pgfqpoint{5.287440in}{2.884193in}}%
\pgfpathlineto{\pgfqpoint{5.274131in}{2.884341in}}%
\pgfpathlineto{\pgfqpoint{5.260832in}{2.884603in}}%
\pgfpathlineto{\pgfqpoint{5.253715in}{2.874479in}}%
\pgfpathlineto{\pgfqpoint{5.246595in}{2.864393in}}%
\pgfpathlineto{\pgfqpoint{5.239472in}{2.854346in}}%
\pgfpathlineto{\pgfqpoint{5.232345in}{2.844334in}}%
\pgfpathclose%
\pgfusepath{fill}%
\end{pgfscope}%
\begin{pgfscope}%
\pgfpathrectangle{\pgfqpoint{1.254980in}{0.150000in}}{\pgfqpoint{5.490039in}{5.490039in}}%
\pgfusepath{clip}%
\pgfsetbuttcap%
\pgfsetroundjoin%
\definecolor{currentfill}{rgb}{0.274128,0.199721,0.498911}%
\pgfsetfillcolor{currentfill}%
\pgfsetfillopacity{0.700000}%
\pgfsetlinewidth{0.000000pt}%
\definecolor{currentstroke}{rgb}{0.000000,0.000000,0.000000}%
\pgfsetstrokecolor{currentstroke}%
\pgfsetdash{}{0pt}%
\pgfpathmoveto{\pgfqpoint{3.775139in}{2.610422in}}%
\pgfpathlineto{\pgfqpoint{3.788083in}{2.601178in}}%
\pgfpathlineto{\pgfqpoint{3.801030in}{2.592081in}}%
\pgfpathlineto{\pgfqpoint{3.813978in}{2.583131in}}%
\pgfpathlineto{\pgfqpoint{3.826929in}{2.574327in}}%
\pgfpathlineto{\pgfqpoint{3.834502in}{2.583930in}}%
\pgfpathlineto{\pgfqpoint{3.842069in}{2.593585in}}%
\pgfpathlineto{\pgfqpoint{3.849631in}{2.603292in}}%
\pgfpathlineto{\pgfqpoint{3.857188in}{2.613052in}}%
\pgfpathlineto{\pgfqpoint{3.844250in}{2.621829in}}%
\pgfpathlineto{\pgfqpoint{3.831315in}{2.630753in}}%
\pgfpathlineto{\pgfqpoint{3.818381in}{2.639823in}}%
\pgfpathlineto{\pgfqpoint{3.805450in}{2.649041in}}%
\pgfpathlineto{\pgfqpoint{3.797880in}{2.639302in}}%
\pgfpathlineto{\pgfqpoint{3.790305in}{2.629619in}}%
\pgfpathlineto{\pgfqpoint{3.782724in}{2.619993in}}%
\pgfpathlineto{\pgfqpoint{3.775139in}{2.610422in}}%
\pgfpathclose%
\pgfusepath{fill}%
\end{pgfscope}%
\begin{pgfscope}%
\pgfpathrectangle{\pgfqpoint{1.254980in}{0.150000in}}{\pgfqpoint{5.490039in}{5.490039in}}%
\pgfusepath{clip}%
\pgfsetbuttcap%
\pgfsetroundjoin%
\definecolor{currentfill}{rgb}{0.273006,0.204520,0.501721}%
\pgfsetfillcolor{currentfill}%
\pgfsetfillopacity{0.700000}%
\pgfsetlinewidth{0.000000pt}%
\definecolor{currentstroke}{rgb}{0.000000,0.000000,0.000000}%
\pgfsetstrokecolor{currentstroke}%
\pgfsetdash{}{0pt}%
\pgfpathmoveto{\pgfqpoint{4.607661in}{2.605703in}}%
\pgfpathlineto{\pgfqpoint{4.620763in}{2.602757in}}%
\pgfpathlineto{\pgfqpoint{4.633873in}{2.599935in}}%
\pgfpathlineto{\pgfqpoint{4.646991in}{2.597235in}}%
\pgfpathlineto{\pgfqpoint{4.660116in}{2.594660in}}%
\pgfpathlineto{\pgfqpoint{4.667431in}{2.604918in}}%
\pgfpathlineto{\pgfqpoint{4.674741in}{2.615197in}}%
\pgfpathlineto{\pgfqpoint{4.682048in}{2.625498in}}%
\pgfpathlineto{\pgfqpoint{4.689350in}{2.635821in}}%
\pgfpathlineto{\pgfqpoint{4.676235in}{2.638482in}}%
\pgfpathlineto{\pgfqpoint{4.663127in}{2.641267in}}%
\pgfpathlineto{\pgfqpoint{4.650028in}{2.644174in}}%
\pgfpathlineto{\pgfqpoint{4.636936in}{2.647206in}}%
\pgfpathlineto{\pgfqpoint{4.629623in}{2.636792in}}%
\pgfpathlineto{\pgfqpoint{4.622307in}{2.626404in}}%
\pgfpathlineto{\pgfqpoint{4.614986in}{2.616041in}}%
\pgfpathlineto{\pgfqpoint{4.607661in}{2.605703in}}%
\pgfpathclose%
\pgfusepath{fill}%
\end{pgfscope}%
\begin{pgfscope}%
\pgfpathrectangle{\pgfqpoint{1.254980in}{0.150000in}}{\pgfqpoint{5.490039in}{5.490039in}}%
\pgfusepath{clip}%
\pgfsetbuttcap%
\pgfsetroundjoin%
\definecolor{currentfill}{rgb}{0.237441,0.305202,0.541921}%
\pgfsetfillcolor{currentfill}%
\pgfsetfillopacity{0.700000}%
\pgfsetlinewidth{0.000000pt}%
\definecolor{currentstroke}{rgb}{0.000000,0.000000,0.000000}%
\pgfsetstrokecolor{currentstroke}%
\pgfsetdash{}{0pt}%
\pgfpathmoveto{\pgfqpoint{5.150605in}{2.805298in}}%
\pgfpathlineto{\pgfqpoint{5.163889in}{2.804951in}}%
\pgfpathlineto{\pgfqpoint{5.177182in}{2.804720in}}%
\pgfpathlineto{\pgfqpoint{5.190486in}{2.804604in}}%
\pgfpathlineto{\pgfqpoint{5.203801in}{2.804604in}}%
\pgfpathlineto{\pgfqpoint{5.210942in}{2.814493in}}%
\pgfpathlineto{\pgfqpoint{5.218080in}{2.824409in}}%
\pgfpathlineto{\pgfqpoint{5.225214in}{2.834356in}}%
\pgfpathlineto{\pgfqpoint{5.232345in}{2.844334in}}%
\pgfpathlineto{\pgfqpoint{5.219043in}{2.844515in}}%
\pgfpathlineto{\pgfqpoint{5.205752in}{2.844812in}}%
\pgfpathlineto{\pgfqpoint{5.192470in}{2.845224in}}%
\pgfpathlineto{\pgfqpoint{5.179199in}{2.845751in}}%
\pgfpathlineto{\pgfqpoint{5.172056in}{2.835587in}}%
\pgfpathlineto{\pgfqpoint{5.164909in}{2.825457in}}%
\pgfpathlineto{\pgfqpoint{5.157759in}{2.815362in}}%
\pgfpathlineto{\pgfqpoint{5.150605in}{2.805298in}}%
\pgfpathclose%
\pgfusepath{fill}%
\end{pgfscope}%
\begin{pgfscope}%
\pgfpathrectangle{\pgfqpoint{1.254980in}{0.150000in}}{\pgfqpoint{5.490039in}{5.490039in}}%
\pgfusepath{clip}%
\pgfsetbuttcap%
\pgfsetroundjoin%
\definecolor{currentfill}{rgb}{0.136408,0.541173,0.554483}%
\pgfsetfillcolor{currentfill}%
\pgfsetfillopacity{0.700000}%
\pgfsetlinewidth{0.000000pt}%
\definecolor{currentstroke}{rgb}{0.000000,0.000000,0.000000}%
\pgfsetstrokecolor{currentstroke}%
\pgfsetdash{}{0pt}%
\pgfpathmoveto{\pgfqpoint{2.966042in}{3.430815in}}%
\pgfpathlineto{\pgfqpoint{2.979125in}{3.411062in}}%
\pgfpathlineto{\pgfqpoint{2.992202in}{3.391517in}}%
\pgfpathlineto{\pgfqpoint{3.005272in}{3.372179in}}%
\pgfpathlineto{\pgfqpoint{3.018336in}{3.353047in}}%
\pgfpathlineto{\pgfqpoint{3.026180in}{3.361677in}}%
\pgfpathlineto{\pgfqpoint{3.034016in}{3.370420in}}%
\pgfpathlineto{\pgfqpoint{3.041844in}{3.379275in}}%
\pgfpathlineto{\pgfqpoint{3.049665in}{3.388243in}}%
\pgfpathlineto{\pgfqpoint{3.036622in}{3.407326in}}%
\pgfpathlineto{\pgfqpoint{3.023572in}{3.426615in}}%
\pgfpathlineto{\pgfqpoint{3.010516in}{3.446111in}}%
\pgfpathlineto{\pgfqpoint{2.997453in}{3.465815in}}%
\pgfpathlineto{\pgfqpoint{2.989612in}{3.456890in}}%
\pgfpathlineto{\pgfqpoint{2.981763in}{3.448081in}}%
\pgfpathlineto{\pgfqpoint{2.973906in}{3.439390in}}%
\pgfpathlineto{\pgfqpoint{2.966042in}{3.430815in}}%
\pgfpathclose%
\pgfusepath{fill}%
\end{pgfscope}%
\begin{pgfscope}%
\pgfpathrectangle{\pgfqpoint{1.254980in}{0.150000in}}{\pgfqpoint{5.490039in}{5.490039in}}%
\pgfusepath{clip}%
\pgfsetbuttcap%
\pgfsetroundjoin%
\definecolor{currentfill}{rgb}{0.262138,0.242286,0.520837}%
\pgfsetfillcolor{currentfill}%
\pgfsetfillopacity{0.700000}%
\pgfsetlinewidth{0.000000pt}%
\definecolor{currentstroke}{rgb}{0.000000,0.000000,0.000000}%
\pgfsetstrokecolor{currentstroke}%
\pgfsetdash{}{0pt}%
\pgfpathmoveto{\pgfqpoint{3.589363in}{2.695877in}}%
\pgfpathlineto{\pgfqpoint{3.602306in}{2.684768in}}%
\pgfpathlineto{\pgfqpoint{3.615250in}{2.673815in}}%
\pgfpathlineto{\pgfqpoint{3.628194in}{2.663017in}}%
\pgfpathlineto{\pgfqpoint{3.641139in}{2.652374in}}%
\pgfpathlineto{\pgfqpoint{3.648774in}{2.661624in}}%
\pgfpathlineto{\pgfqpoint{3.656404in}{2.670937in}}%
\pgfpathlineto{\pgfqpoint{3.664028in}{2.680315in}}%
\pgfpathlineto{\pgfqpoint{3.671646in}{2.689756in}}%
\pgfpathlineto{\pgfqpoint{3.658715in}{2.700356in}}%
\pgfpathlineto{\pgfqpoint{3.645785in}{2.711111in}}%
\pgfpathlineto{\pgfqpoint{3.632856in}{2.722021in}}%
\pgfpathlineto{\pgfqpoint{3.619928in}{2.733087in}}%
\pgfpathlineto{\pgfqpoint{3.612295in}{2.723682in}}%
\pgfpathlineto{\pgfqpoint{3.604657in}{2.714346in}}%
\pgfpathlineto{\pgfqpoint{3.597013in}{2.705078in}}%
\pgfpathlineto{\pgfqpoint{3.589363in}{2.695877in}}%
\pgfpathclose%
\pgfusepath{fill}%
\end{pgfscope}%
\begin{pgfscope}%
\pgfpathrectangle{\pgfqpoint{1.254980in}{0.150000in}}{\pgfqpoint{5.490039in}{5.490039in}}%
\pgfusepath{clip}%
\pgfsetbuttcap%
\pgfsetroundjoin%
\definecolor{currentfill}{rgb}{0.244972,0.287675,0.537260}%
\pgfsetfillcolor{currentfill}%
\pgfsetfillopacity{0.700000}%
\pgfsetlinewidth{0.000000pt}%
\definecolor{currentstroke}{rgb}{0.000000,0.000000,0.000000}%
\pgfsetstrokecolor{currentstroke}%
\pgfsetdash{}{0pt}%
\pgfpathmoveto{\pgfqpoint{5.068869in}{2.767214in}}%
\pgfpathlineto{\pgfqpoint{5.082125in}{2.766567in}}%
\pgfpathlineto{\pgfqpoint{5.095391in}{2.766036in}}%
\pgfpathlineto{\pgfqpoint{5.108666in}{2.765622in}}%
\pgfpathlineto{\pgfqpoint{5.121952in}{2.765324in}}%
\pgfpathlineto{\pgfqpoint{5.129121in}{2.775279in}}%
\pgfpathlineto{\pgfqpoint{5.136286in}{2.785258in}}%
\pgfpathlineto{\pgfqpoint{5.143447in}{2.795264in}}%
\pgfpathlineto{\pgfqpoint{5.150605in}{2.805298in}}%
\pgfpathlineto{\pgfqpoint{5.137331in}{2.805761in}}%
\pgfpathlineto{\pgfqpoint{5.124067in}{2.806340in}}%
\pgfpathlineto{\pgfqpoint{5.110813in}{2.807036in}}%
\pgfpathlineto{\pgfqpoint{5.097569in}{2.807848in}}%
\pgfpathlineto{\pgfqpoint{5.090400in}{2.797643in}}%
\pgfpathlineto{\pgfqpoint{5.083226in}{2.787471in}}%
\pgfpathlineto{\pgfqpoint{5.076049in}{2.777328in}}%
\pgfpathlineto{\pgfqpoint{5.068869in}{2.767214in}}%
\pgfpathclose%
\pgfusepath{fill}%
\end{pgfscope}%
\begin{pgfscope}%
\pgfpathrectangle{\pgfqpoint{1.254980in}{0.150000in}}{\pgfqpoint{5.490039in}{5.490039in}}%
\pgfusepath{clip}%
\pgfsetbuttcap%
\pgfsetroundjoin%
\definecolor{currentfill}{rgb}{0.278826,0.175490,0.483397}%
\pgfsetfillcolor{currentfill}%
\pgfsetfillopacity{0.700000}%
\pgfsetlinewidth{0.000000pt}%
\definecolor{currentstroke}{rgb}{0.000000,0.000000,0.000000}%
\pgfsetstrokecolor{currentstroke}%
\pgfsetdash{}{0pt}%
\pgfpathmoveto{\pgfqpoint{4.310254in}{2.544880in}}%
\pgfpathlineto{\pgfqpoint{4.323280in}{2.540065in}}%
\pgfpathlineto{\pgfqpoint{4.336311in}{2.535381in}}%
\pgfpathlineto{\pgfqpoint{4.349349in}{2.530825in}}%
\pgfpathlineto{\pgfqpoint{4.362393in}{2.526399in}}%
\pgfpathlineto{\pgfqpoint{4.369799in}{2.536612in}}%
\pgfpathlineto{\pgfqpoint{4.377202in}{2.546851in}}%
\pgfpathlineto{\pgfqpoint{4.384599in}{2.557117in}}%
\pgfpathlineto{\pgfqpoint{4.391993in}{2.567410in}}%
\pgfpathlineto{\pgfqpoint{4.378959in}{2.571874in}}%
\pgfpathlineto{\pgfqpoint{4.365932in}{2.576468in}}%
\pgfpathlineto{\pgfqpoint{4.352910in}{2.581190in}}%
\pgfpathlineto{\pgfqpoint{4.339895in}{2.586043in}}%
\pgfpathlineto{\pgfqpoint{4.332491in}{2.575706in}}%
\pgfpathlineto{\pgfqpoint{4.325083in}{2.565400in}}%
\pgfpathlineto{\pgfqpoint{4.317671in}{2.555125in}}%
\pgfpathlineto{\pgfqpoint{4.310254in}{2.544880in}}%
\pgfpathclose%
\pgfusepath{fill}%
\end{pgfscope}%
\begin{pgfscope}%
\pgfpathrectangle{\pgfqpoint{1.254980in}{0.150000in}}{\pgfqpoint{5.490039in}{5.490039in}}%
\pgfusepath{clip}%
\pgfsetbuttcap%
\pgfsetroundjoin%
\definecolor{currentfill}{rgb}{0.252194,0.269783,0.531579}%
\pgfsetfillcolor{currentfill}%
\pgfsetfillopacity{0.700000}%
\pgfsetlinewidth{0.000000pt}%
\definecolor{currentstroke}{rgb}{0.000000,0.000000,0.000000}%
\pgfsetstrokecolor{currentstroke}%
\pgfsetdash{}{0pt}%
\pgfpathmoveto{\pgfqpoint{4.987132in}{2.730175in}}%
\pgfpathlineto{\pgfqpoint{5.000362in}{2.729207in}}%
\pgfpathlineto{\pgfqpoint{5.013601in}{2.728357in}}%
\pgfpathlineto{\pgfqpoint{5.026849in}{2.727625in}}%
\pgfpathlineto{\pgfqpoint{5.040107in}{2.727009in}}%
\pgfpathlineto{\pgfqpoint{5.047303in}{2.737026in}}%
\pgfpathlineto{\pgfqpoint{5.054496in}{2.747065in}}%
\pgfpathlineto{\pgfqpoint{5.061684in}{2.757127in}}%
\pgfpathlineto{\pgfqpoint{5.068869in}{2.767214in}}%
\pgfpathlineto{\pgfqpoint{5.055622in}{2.767979in}}%
\pgfpathlineto{\pgfqpoint{5.042385in}{2.768860in}}%
\pgfpathlineto{\pgfqpoint{5.029158in}{2.769859in}}%
\pgfpathlineto{\pgfqpoint{5.015940in}{2.770976in}}%
\pgfpathlineto{\pgfqpoint{5.008744in}{2.760734in}}%
\pgfpathlineto{\pgfqpoint{5.001544in}{2.750521in}}%
\pgfpathlineto{\pgfqpoint{4.994340in}{2.740335in}}%
\pgfpathlineto{\pgfqpoint{4.987132in}{2.730175in}}%
\pgfpathclose%
\pgfusepath{fill}%
\end{pgfscope}%
\begin{pgfscope}%
\pgfpathrectangle{\pgfqpoint{1.254980in}{0.150000in}}{\pgfqpoint{5.490039in}{5.490039in}}%
\pgfusepath{clip}%
\pgfsetbuttcap%
\pgfsetroundjoin%
\definecolor{currentfill}{rgb}{0.275191,0.194905,0.496005}%
\pgfsetfillcolor{currentfill}%
\pgfsetfillopacity{0.700000}%
\pgfsetlinewidth{0.000000pt}%
\definecolor{currentstroke}{rgb}{0.000000,0.000000,0.000000}%
\pgfsetstrokecolor{currentstroke}%
\pgfsetdash{}{0pt}%
\pgfpathmoveto{\pgfqpoint{4.525944in}{2.577323in}}%
\pgfpathlineto{\pgfqpoint{4.539027in}{2.573949in}}%
\pgfpathlineto{\pgfqpoint{4.552117in}{2.570699in}}%
\pgfpathlineto{\pgfqpoint{4.565215in}{2.567574in}}%
\pgfpathlineto{\pgfqpoint{4.578320in}{2.564573in}}%
\pgfpathlineto{\pgfqpoint{4.585661in}{2.574824in}}%
\pgfpathlineto{\pgfqpoint{4.592999in}{2.585095in}}%
\pgfpathlineto{\pgfqpoint{4.600332in}{2.595388in}}%
\pgfpathlineto{\pgfqpoint{4.607661in}{2.605703in}}%
\pgfpathlineto{\pgfqpoint{4.594566in}{2.608773in}}%
\pgfpathlineto{\pgfqpoint{4.581479in}{2.611968in}}%
\pgfpathlineto{\pgfqpoint{4.568398in}{2.615287in}}%
\pgfpathlineto{\pgfqpoint{4.555325in}{2.618732in}}%
\pgfpathlineto{\pgfqpoint{4.547986in}{2.608341in}}%
\pgfpathlineto{\pgfqpoint{4.540643in}{2.597977in}}%
\pgfpathlineto{\pgfqpoint{4.533295in}{2.587638in}}%
\pgfpathlineto{\pgfqpoint{4.525944in}{2.577323in}}%
\pgfpathclose%
\pgfusepath{fill}%
\end{pgfscope}%
\begin{pgfscope}%
\pgfpathrectangle{\pgfqpoint{1.254980in}{0.150000in}}{\pgfqpoint{5.490039in}{5.490039in}}%
\pgfusepath{clip}%
\pgfsetbuttcap%
\pgfsetroundjoin%
\definecolor{currentfill}{rgb}{0.278826,0.175490,0.483397}%
\pgfsetfillcolor{currentfill}%
\pgfsetfillopacity{0.700000}%
\pgfsetlinewidth{0.000000pt}%
\definecolor{currentstroke}{rgb}{0.000000,0.000000,0.000000}%
\pgfsetstrokecolor{currentstroke}%
\pgfsetdash{}{0pt}%
\pgfpathmoveto{\pgfqpoint{3.960795in}{2.548004in}}%
\pgfpathlineto{\pgfqpoint{3.973760in}{2.540510in}}%
\pgfpathlineto{\pgfqpoint{3.986729in}{2.533157in}}%
\pgfpathlineto{\pgfqpoint{3.999702in}{2.525943in}}%
\pgfpathlineto{\pgfqpoint{4.012678in}{2.518868in}}%
\pgfpathlineto{\pgfqpoint{4.020193in}{2.528727in}}%
\pgfpathlineto{\pgfqpoint{4.027704in}{2.538628in}}%
\pgfpathlineto{\pgfqpoint{4.035210in}{2.548570in}}%
\pgfpathlineto{\pgfqpoint{4.042712in}{2.558553in}}%
\pgfpathlineto{\pgfqpoint{4.029747in}{2.565619in}}%
\pgfpathlineto{\pgfqpoint{4.016786in}{2.572823in}}%
\pgfpathlineto{\pgfqpoint{4.003829in}{2.580166in}}%
\pgfpathlineto{\pgfqpoint{3.990876in}{2.587649in}}%
\pgfpathlineto{\pgfqpoint{3.983363in}{2.577670in}}%
\pgfpathlineto{\pgfqpoint{3.975845in}{2.567737in}}%
\pgfpathlineto{\pgfqpoint{3.968323in}{2.557848in}}%
\pgfpathlineto{\pgfqpoint{3.960795in}{2.548004in}}%
\pgfpathclose%
\pgfusepath{fill}%
\end{pgfscope}%
\begin{pgfscope}%
\pgfpathrectangle{\pgfqpoint{1.254980in}{0.150000in}}{\pgfqpoint{5.490039in}{5.490039in}}%
\pgfusepath{clip}%
\pgfsetbuttcap%
\pgfsetroundjoin%
\definecolor{currentfill}{rgb}{0.280255,0.165693,0.476498}%
\pgfsetfillcolor{currentfill}%
\pgfsetfillopacity{0.700000}%
\pgfsetlinewidth{0.000000pt}%
\definecolor{currentstroke}{rgb}{0.000000,0.000000,0.000000}%
\pgfsetstrokecolor{currentstroke}%
\pgfsetdash{}{0pt}%
\pgfpathmoveto{\pgfqpoint{4.094611in}{2.531666in}}%
\pgfpathlineto{\pgfqpoint{4.107596in}{2.525286in}}%
\pgfpathlineto{\pgfqpoint{4.120586in}{2.519041in}}%
\pgfpathlineto{\pgfqpoint{4.133581in}{2.512931in}}%
\pgfpathlineto{\pgfqpoint{4.146581in}{2.506956in}}%
\pgfpathlineto{\pgfqpoint{4.154055in}{2.516980in}}%
\pgfpathlineto{\pgfqpoint{4.161524in}{2.527038in}}%
\pgfpathlineto{\pgfqpoint{4.168989in}{2.537131in}}%
\pgfpathlineto{\pgfqpoint{4.176450in}{2.547259in}}%
\pgfpathlineto{\pgfqpoint{4.163461in}{2.553240in}}%
\pgfpathlineto{\pgfqpoint{4.150477in}{2.559356in}}%
\pgfpathlineto{\pgfqpoint{4.137498in}{2.565607in}}%
\pgfpathlineto{\pgfqpoint{4.124524in}{2.571994in}}%
\pgfpathlineto{\pgfqpoint{4.117053in}{2.561854in}}%
\pgfpathlineto{\pgfqpoint{4.109577in}{2.551753in}}%
\pgfpathlineto{\pgfqpoint{4.102096in}{2.541691in}}%
\pgfpathlineto{\pgfqpoint{4.094611in}{2.531666in}}%
\pgfpathclose%
\pgfusepath{fill}%
\end{pgfscope}%
\begin{pgfscope}%
\pgfpathrectangle{\pgfqpoint{1.254980in}{0.150000in}}{\pgfqpoint{5.490039in}{5.490039in}}%
\pgfusepath{clip}%
\pgfsetbuttcap%
\pgfsetroundjoin%
\definecolor{currentfill}{rgb}{0.208623,0.367752,0.552675}%
\pgfsetfillcolor{currentfill}%
\pgfsetfillopacity{0.700000}%
\pgfsetlinewidth{0.000000pt}%
\definecolor{currentstroke}{rgb}{0.000000,0.000000,0.000000}%
\pgfsetstrokecolor{currentstroke}%
\pgfsetdash{}{0pt}%
\pgfpathmoveto{\pgfqpoint{3.247491in}{2.976853in}}%
\pgfpathlineto{\pgfqpoint{3.260479in}{2.961674in}}%
\pgfpathlineto{\pgfqpoint{3.273465in}{2.946674in}}%
\pgfpathlineto{\pgfqpoint{3.286447in}{2.931851in}}%
\pgfpathlineto{\pgfqpoint{3.299427in}{2.917204in}}%
\pgfpathlineto{\pgfqpoint{3.307184in}{2.925800in}}%
\pgfpathlineto{\pgfqpoint{3.314934in}{2.934484in}}%
\pgfpathlineto{\pgfqpoint{3.322677in}{2.943257in}}%
\pgfpathlineto{\pgfqpoint{3.330414in}{2.952118in}}%
\pgfpathlineto{\pgfqpoint{3.317452in}{2.966703in}}%
\pgfpathlineto{\pgfqpoint{3.304487in}{2.981464in}}%
\pgfpathlineto{\pgfqpoint{3.291520in}{2.996403in}}%
\pgfpathlineto{\pgfqpoint{3.278550in}{3.011519in}}%
\pgfpathlineto{\pgfqpoint{3.270795in}{3.002714in}}%
\pgfpathlineto{\pgfqpoint{3.263034in}{2.994001in}}%
\pgfpathlineto{\pgfqpoint{3.255266in}{2.985381in}}%
\pgfpathlineto{\pgfqpoint{3.247491in}{2.976853in}}%
\pgfpathclose%
\pgfusepath{fill}%
\end{pgfscope}%
\begin{pgfscope}%
\pgfpathrectangle{\pgfqpoint{1.254980in}{0.150000in}}{\pgfqpoint{5.490039in}{5.490039in}}%
\pgfusepath{clip}%
\pgfsetbuttcap%
\pgfsetroundjoin%
\definecolor{currentfill}{rgb}{0.257322,0.256130,0.526563}%
\pgfsetfillcolor{currentfill}%
\pgfsetfillopacity{0.700000}%
\pgfsetlinewidth{0.000000pt}%
\definecolor{currentstroke}{rgb}{0.000000,0.000000,0.000000}%
\pgfsetstrokecolor{currentstroke}%
\pgfsetdash{}{0pt}%
\pgfpathmoveto{\pgfqpoint{4.905393in}{2.694284in}}%
\pgfpathlineto{\pgfqpoint{4.918597in}{2.692975in}}%
\pgfpathlineto{\pgfqpoint{4.931810in}{2.691785in}}%
\pgfpathlineto{\pgfqpoint{4.945031in}{2.690714in}}%
\pgfpathlineto{\pgfqpoint{4.958263in}{2.689762in}}%
\pgfpathlineto{\pgfqpoint{4.965486in}{2.699834in}}%
\pgfpathlineto{\pgfqpoint{4.972706in}{2.709926in}}%
\pgfpathlineto{\pgfqpoint{4.979921in}{2.720039in}}%
\pgfpathlineto{\pgfqpoint{4.987132in}{2.730175in}}%
\pgfpathlineto{\pgfqpoint{4.973912in}{2.731261in}}%
\pgfpathlineto{\pgfqpoint{4.960701in}{2.732466in}}%
\pgfpathlineto{\pgfqpoint{4.947500in}{2.733788in}}%
\pgfpathlineto{\pgfqpoint{4.934307in}{2.735230in}}%
\pgfpathlineto{\pgfqpoint{4.927085in}{2.724955in}}%
\pgfpathlineto{\pgfqpoint{4.919858in}{2.714707in}}%
\pgfpathlineto{\pgfqpoint{4.912628in}{2.704484in}}%
\pgfpathlineto{\pgfqpoint{4.905393in}{2.694284in}}%
\pgfpathclose%
\pgfusepath{fill}%
\end{pgfscope}%
\begin{pgfscope}%
\pgfpathrectangle{\pgfqpoint{1.254980in}{0.150000in}}{\pgfqpoint{5.490039in}{5.490039in}}%
\pgfusepath{clip}%
\pgfsetbuttcap%
\pgfsetroundjoin%
\definecolor{currentfill}{rgb}{0.197636,0.391528,0.554969}%
\pgfsetfillcolor{currentfill}%
\pgfsetfillopacity{0.700000}%
\pgfsetlinewidth{0.000000pt}%
\definecolor{currentstroke}{rgb}{0.000000,0.000000,0.000000}%
\pgfsetstrokecolor{currentstroke}%
\pgfsetdash{}{0pt}%
\pgfpathmoveto{\pgfqpoint{3.195507in}{3.039373in}}%
\pgfpathlineto{\pgfqpoint{3.208508in}{3.023470in}}%
\pgfpathlineto{\pgfqpoint{3.221506in}{3.007750in}}%
\pgfpathlineto{\pgfqpoint{3.234500in}{2.992211in}}%
\pgfpathlineto{\pgfqpoint{3.247491in}{2.976853in}}%
\pgfpathlineto{\pgfqpoint{3.255266in}{2.985381in}}%
\pgfpathlineto{\pgfqpoint{3.263034in}{2.994001in}}%
\pgfpathlineto{\pgfqpoint{3.270795in}{3.002714in}}%
\pgfpathlineto{\pgfqpoint{3.278550in}{3.011519in}}%
\pgfpathlineto{\pgfqpoint{3.265577in}{3.026815in}}%
\pgfpathlineto{\pgfqpoint{3.252601in}{3.042291in}}%
\pgfpathlineto{\pgfqpoint{3.239622in}{3.057949in}}%
\pgfpathlineto{\pgfqpoint{3.226639in}{3.073789in}}%
\pgfpathlineto{\pgfqpoint{3.218867in}{3.065040in}}%
\pgfpathlineto{\pgfqpoint{3.211087in}{3.056387in}}%
\pgfpathlineto{\pgfqpoint{3.203300in}{3.047832in}}%
\pgfpathlineto{\pgfqpoint{3.195507in}{3.039373in}}%
\pgfpathclose%
\pgfusepath{fill}%
\end{pgfscope}%
\begin{pgfscope}%
\pgfpathrectangle{\pgfqpoint{1.254980in}{0.150000in}}{\pgfqpoint{5.490039in}{5.490039in}}%
\pgfusepath{clip}%
\pgfsetbuttcap%
\pgfsetroundjoin%
\definecolor{currentfill}{rgb}{0.220057,0.343307,0.549413}%
\pgfsetfillcolor{currentfill}%
\pgfsetfillopacity{0.700000}%
\pgfsetlinewidth{0.000000pt}%
\definecolor{currentstroke}{rgb}{0.000000,0.000000,0.000000}%
\pgfsetstrokecolor{currentstroke}%
\pgfsetdash{}{0pt}%
\pgfpathmoveto{\pgfqpoint{3.299427in}{2.917204in}}%
\pgfpathlineto{\pgfqpoint{3.312405in}{2.902733in}}%
\pgfpathlineto{\pgfqpoint{3.325380in}{2.888435in}}%
\pgfpathlineto{\pgfqpoint{3.338353in}{2.874311in}}%
\pgfpathlineto{\pgfqpoint{3.351325in}{2.860359in}}%
\pgfpathlineto{\pgfqpoint{3.359063in}{2.869022in}}%
\pgfpathlineto{\pgfqpoint{3.366796in}{2.877769in}}%
\pgfpathlineto{\pgfqpoint{3.374522in}{2.886601in}}%
\pgfpathlineto{\pgfqpoint{3.382241in}{2.895517in}}%
\pgfpathlineto{\pgfqpoint{3.369287in}{2.909408in}}%
\pgfpathlineto{\pgfqpoint{3.356332in}{2.923472in}}%
\pgfpathlineto{\pgfqpoint{3.343374in}{2.937708in}}%
\pgfpathlineto{\pgfqpoint{3.330414in}{2.952118in}}%
\pgfpathlineto{\pgfqpoint{3.322677in}{2.943257in}}%
\pgfpathlineto{\pgfqpoint{3.314934in}{2.934484in}}%
\pgfpathlineto{\pgfqpoint{3.307184in}{2.925800in}}%
\pgfpathlineto{\pgfqpoint{3.299427in}{2.917204in}}%
\pgfpathclose%
\pgfusepath{fill}%
\end{pgfscope}%
\begin{pgfscope}%
\pgfpathrectangle{\pgfqpoint{1.254980in}{0.150000in}}{\pgfqpoint{5.490039in}{5.490039in}}%
\pgfusepath{clip}%
\pgfsetbuttcap%
\pgfsetroundjoin%
\definecolor{currentfill}{rgb}{0.267968,0.223549,0.512008}%
\pgfsetfillcolor{currentfill}%
\pgfsetfillopacity{0.700000}%
\pgfsetlinewidth{0.000000pt}%
\definecolor{currentstroke}{rgb}{0.000000,0.000000,0.000000}%
\pgfsetstrokecolor{currentstroke}%
\pgfsetdash{}{0pt}%
\pgfpathmoveto{\pgfqpoint{3.641139in}{2.652374in}}%
\pgfpathlineto{\pgfqpoint{3.654085in}{2.641885in}}%
\pgfpathlineto{\pgfqpoint{3.667032in}{2.631548in}}%
\pgfpathlineto{\pgfqpoint{3.679981in}{2.621364in}}%
\pgfpathlineto{\pgfqpoint{3.692930in}{2.611331in}}%
\pgfpathlineto{\pgfqpoint{3.700551in}{2.620629in}}%
\pgfpathlineto{\pgfqpoint{3.708166in}{2.629987in}}%
\pgfpathlineto{\pgfqpoint{3.715776in}{2.639405in}}%
\pgfpathlineto{\pgfqpoint{3.723380in}{2.648883in}}%
\pgfpathlineto{\pgfqpoint{3.710445in}{2.658874in}}%
\pgfpathlineto{\pgfqpoint{3.697511in}{2.669015in}}%
\pgfpathlineto{\pgfqpoint{3.684578in}{2.679309in}}%
\pgfpathlineto{\pgfqpoint{3.671646in}{2.689756in}}%
\pgfpathlineto{\pgfqpoint{3.664028in}{2.680315in}}%
\pgfpathlineto{\pgfqpoint{3.656404in}{2.670937in}}%
\pgfpathlineto{\pgfqpoint{3.648774in}{2.661624in}}%
\pgfpathlineto{\pgfqpoint{3.641139in}{2.652374in}}%
\pgfpathclose%
\pgfusepath{fill}%
\end{pgfscope}%
\begin{pgfscope}%
\pgfpathrectangle{\pgfqpoint{1.254980in}{0.150000in}}{\pgfqpoint{5.490039in}{5.490039in}}%
\pgfusepath{clip}%
\pgfsetbuttcap%
\pgfsetroundjoin%
\definecolor{currentfill}{rgb}{0.276194,0.190074,0.493001}%
\pgfsetfillcolor{currentfill}%
\pgfsetfillopacity{0.700000}%
\pgfsetlinewidth{0.000000pt}%
\definecolor{currentstroke}{rgb}{0.000000,0.000000,0.000000}%
\pgfsetstrokecolor{currentstroke}%
\pgfsetdash{}{0pt}%
\pgfpathmoveto{\pgfqpoint{3.826929in}{2.574327in}}%
\pgfpathlineto{\pgfqpoint{3.839883in}{2.565668in}}%
\pgfpathlineto{\pgfqpoint{3.852839in}{2.557154in}}%
\pgfpathlineto{\pgfqpoint{3.865798in}{2.548784in}}%
\pgfpathlineto{\pgfqpoint{3.878759in}{2.540557in}}%
\pgfpathlineto{\pgfqpoint{3.886319in}{2.550192in}}%
\pgfpathlineto{\pgfqpoint{3.893874in}{2.559875in}}%
\pgfpathlineto{\pgfqpoint{3.901423in}{2.569606in}}%
\pgfpathlineto{\pgfqpoint{3.908968in}{2.579386in}}%
\pgfpathlineto{\pgfqpoint{3.896019in}{2.587587in}}%
\pgfpathlineto{\pgfqpoint{3.883072in}{2.595931in}}%
\pgfpathlineto{\pgfqpoint{3.870129in}{2.604419in}}%
\pgfpathlineto{\pgfqpoint{3.857188in}{2.613052in}}%
\pgfpathlineto{\pgfqpoint{3.849631in}{2.603292in}}%
\pgfpathlineto{\pgfqpoint{3.842069in}{2.593585in}}%
\pgfpathlineto{\pgfqpoint{3.834502in}{2.583930in}}%
\pgfpathlineto{\pgfqpoint{3.826929in}{2.574327in}}%
\pgfpathclose%
\pgfusepath{fill}%
\end{pgfscope}%
\begin{pgfscope}%
\pgfpathrectangle{\pgfqpoint{1.254980in}{0.150000in}}{\pgfqpoint{5.490039in}{5.490039in}}%
\pgfusepath{clip}%
\pgfsetbuttcap%
\pgfsetroundjoin%
\definecolor{currentfill}{rgb}{0.185556,0.418570,0.556753}%
\pgfsetfillcolor{currentfill}%
\pgfsetfillopacity{0.700000}%
\pgfsetlinewidth{0.000000pt}%
\definecolor{currentstroke}{rgb}{0.000000,0.000000,0.000000}%
\pgfsetstrokecolor{currentstroke}%
\pgfsetdash{}{0pt}%
\pgfpathmoveto{\pgfqpoint{3.143464in}{3.104835in}}%
\pgfpathlineto{\pgfqpoint{3.156481in}{3.088189in}}%
\pgfpathlineto{\pgfqpoint{3.169493in}{3.071731in}}%
\pgfpathlineto{\pgfqpoint{3.182502in}{3.055459in}}%
\pgfpathlineto{\pgfqpoint{3.195507in}{3.039373in}}%
\pgfpathlineto{\pgfqpoint{3.203300in}{3.047832in}}%
\pgfpathlineto{\pgfqpoint{3.211087in}{3.056387in}}%
\pgfpathlineto{\pgfqpoint{3.218867in}{3.065040in}}%
\pgfpathlineto{\pgfqpoint{3.226639in}{3.073789in}}%
\pgfpathlineto{\pgfqpoint{3.213654in}{3.089813in}}%
\pgfpathlineto{\pgfqpoint{3.200664in}{3.106022in}}%
\pgfpathlineto{\pgfqpoint{3.187671in}{3.122417in}}%
\pgfpathlineto{\pgfqpoint{3.174673in}{3.138999in}}%
\pgfpathlineto{\pgfqpoint{3.166882in}{3.130307in}}%
\pgfpathlineto{\pgfqpoint{3.159083in}{3.121716in}}%
\pgfpathlineto{\pgfqpoint{3.151277in}{3.113225in}}%
\pgfpathlineto{\pgfqpoint{3.143464in}{3.104835in}}%
\pgfpathclose%
\pgfusepath{fill}%
\end{pgfscope}%
\begin{pgfscope}%
\pgfpathrectangle{\pgfqpoint{1.254980in}{0.150000in}}{\pgfqpoint{5.490039in}{5.490039in}}%
\pgfusepath{clip}%
\pgfsetbuttcap%
\pgfsetroundjoin%
\definecolor{currentfill}{rgb}{0.231674,0.318106,0.544834}%
\pgfsetfillcolor{currentfill}%
\pgfsetfillopacity{0.700000}%
\pgfsetlinewidth{0.000000pt}%
\definecolor{currentstroke}{rgb}{0.000000,0.000000,0.000000}%
\pgfsetstrokecolor{currentstroke}%
\pgfsetdash{}{0pt}%
\pgfpathmoveto{\pgfqpoint{3.351325in}{2.860359in}}%
\pgfpathlineto{\pgfqpoint{3.364294in}{2.846578in}}%
\pgfpathlineto{\pgfqpoint{3.377262in}{2.832967in}}%
\pgfpathlineto{\pgfqpoint{3.390228in}{2.819525in}}%
\pgfpathlineto{\pgfqpoint{3.403192in}{2.806251in}}%
\pgfpathlineto{\pgfqpoint{3.410914in}{2.814981in}}%
\pgfpathlineto{\pgfqpoint{3.418629in}{2.823791in}}%
\pgfpathlineto{\pgfqpoint{3.426338in}{2.832682in}}%
\pgfpathlineto{\pgfqpoint{3.434041in}{2.841653in}}%
\pgfpathlineto{\pgfqpoint{3.421093in}{2.854866in}}%
\pgfpathlineto{\pgfqpoint{3.408144in}{2.868247in}}%
\pgfpathlineto{\pgfqpoint{3.395194in}{2.881797in}}%
\pgfpathlineto{\pgfqpoint{3.382241in}{2.895517in}}%
\pgfpathlineto{\pgfqpoint{3.374522in}{2.886601in}}%
\pgfpathlineto{\pgfqpoint{3.366796in}{2.877769in}}%
\pgfpathlineto{\pgfqpoint{3.359063in}{2.869022in}}%
\pgfpathlineto{\pgfqpoint{3.351325in}{2.860359in}}%
\pgfpathclose%
\pgfusepath{fill}%
\end{pgfscope}%
\begin{pgfscope}%
\pgfpathrectangle{\pgfqpoint{1.254980in}{0.150000in}}{\pgfqpoint{5.490039in}{5.490039in}}%
\pgfusepath{clip}%
\pgfsetbuttcap%
\pgfsetroundjoin%
\definecolor{currentfill}{rgb}{0.263663,0.237631,0.518762}%
\pgfsetfillcolor{currentfill}%
\pgfsetfillopacity{0.700000}%
\pgfsetlinewidth{0.000000pt}%
\definecolor{currentstroke}{rgb}{0.000000,0.000000,0.000000}%
\pgfsetstrokecolor{currentstroke}%
\pgfsetdash{}{0pt}%
\pgfpathmoveto{\pgfqpoint{4.823647in}{2.659651in}}%
\pgfpathlineto{\pgfqpoint{4.836826in}{2.657982in}}%
\pgfpathlineto{\pgfqpoint{4.850014in}{2.656432in}}%
\pgfpathlineto{\pgfqpoint{4.863210in}{2.655002in}}%
\pgfpathlineto{\pgfqpoint{4.876416in}{2.653692in}}%
\pgfpathlineto{\pgfqpoint{4.883666in}{2.663812in}}%
\pgfpathlineto{\pgfqpoint{4.890913in}{2.673949in}}%
\pgfpathlineto{\pgfqpoint{4.898155in}{2.684106in}}%
\pgfpathlineto{\pgfqpoint{4.905393in}{2.694284in}}%
\pgfpathlineto{\pgfqpoint{4.892199in}{2.695711in}}%
\pgfpathlineto{\pgfqpoint{4.879013in}{2.697259in}}%
\pgfpathlineto{\pgfqpoint{4.865836in}{2.698925in}}%
\pgfpathlineto{\pgfqpoint{4.852668in}{2.700712in}}%
\pgfpathlineto{\pgfqpoint{4.845419in}{2.690411in}}%
\pgfpathlineto{\pgfqpoint{4.838166in}{2.680135in}}%
\pgfpathlineto{\pgfqpoint{4.830908in}{2.669882in}}%
\pgfpathlineto{\pgfqpoint{4.823647in}{2.659651in}}%
\pgfpathclose%
\pgfusepath{fill}%
\end{pgfscope}%
\begin{pgfscope}%
\pgfpathrectangle{\pgfqpoint{1.254980in}{0.150000in}}{\pgfqpoint{5.490039in}{5.490039in}}%
\pgfusepath{clip}%
\pgfsetbuttcap%
\pgfsetroundjoin%
\definecolor{currentfill}{rgb}{0.280255,0.165693,0.476498}%
\pgfsetfillcolor{currentfill}%
\pgfsetfillopacity{0.700000}%
\pgfsetlinewidth{0.000000pt}%
\definecolor{currentstroke}{rgb}{0.000000,0.000000,0.000000}%
\pgfsetstrokecolor{currentstroke}%
\pgfsetdash{}{0pt}%
\pgfpathmoveto{\pgfqpoint{4.228453in}{2.524667in}}%
\pgfpathlineto{\pgfqpoint{4.241467in}{2.519351in}}%
\pgfpathlineto{\pgfqpoint{4.254487in}{2.514166in}}%
\pgfpathlineto{\pgfqpoint{4.267512in}{2.509112in}}%
\pgfpathlineto{\pgfqpoint{4.280542in}{2.504189in}}%
\pgfpathlineto{\pgfqpoint{4.287977in}{2.514320in}}%
\pgfpathlineto{\pgfqpoint{4.295407in}{2.524478in}}%
\pgfpathlineto{\pgfqpoint{4.302833in}{2.534665in}}%
\pgfpathlineto{\pgfqpoint{4.310254in}{2.544880in}}%
\pgfpathlineto{\pgfqpoint{4.297234in}{2.549825in}}%
\pgfpathlineto{\pgfqpoint{4.284219in}{2.554901in}}%
\pgfpathlineto{\pgfqpoint{4.271210in}{2.560108in}}%
\pgfpathlineto{\pgfqpoint{4.258207in}{2.565446in}}%
\pgfpathlineto{\pgfqpoint{4.250775in}{2.555203in}}%
\pgfpathlineto{\pgfqpoint{4.243339in}{2.544992in}}%
\pgfpathlineto{\pgfqpoint{4.235898in}{2.534814in}}%
\pgfpathlineto{\pgfqpoint{4.228453in}{2.524667in}}%
\pgfpathclose%
\pgfusepath{fill}%
\end{pgfscope}%
\begin{pgfscope}%
\pgfpathrectangle{\pgfqpoint{1.254980in}{0.150000in}}{\pgfqpoint{5.490039in}{5.490039in}}%
\pgfusepath{clip}%
\pgfsetbuttcap%
\pgfsetroundjoin%
\definecolor{currentfill}{rgb}{0.174274,0.445044,0.557792}%
\pgfsetfillcolor{currentfill}%
\pgfsetfillopacity{0.700000}%
\pgfsetlinewidth{0.000000pt}%
\definecolor{currentstroke}{rgb}{0.000000,0.000000,0.000000}%
\pgfsetstrokecolor{currentstroke}%
\pgfsetdash{}{0pt}%
\pgfpathmoveto{\pgfqpoint{3.091353in}{3.173316in}}%
\pgfpathlineto{\pgfqpoint{3.104388in}{3.155908in}}%
\pgfpathlineto{\pgfqpoint{3.117418in}{3.138693in}}%
\pgfpathlineto{\pgfqpoint{3.130443in}{3.121669in}}%
\pgfpathlineto{\pgfqpoint{3.143464in}{3.104835in}}%
\pgfpathlineto{\pgfqpoint{3.151277in}{3.113225in}}%
\pgfpathlineto{\pgfqpoint{3.159083in}{3.121716in}}%
\pgfpathlineto{\pgfqpoint{3.166882in}{3.130307in}}%
\pgfpathlineto{\pgfqpoint{3.174673in}{3.138999in}}%
\pgfpathlineto{\pgfqpoint{3.161672in}{3.155770in}}%
\pgfpathlineto{\pgfqpoint{3.148666in}{3.172730in}}%
\pgfpathlineto{\pgfqpoint{3.135656in}{3.189882in}}%
\pgfpathlineto{\pgfqpoint{3.122641in}{3.207225in}}%
\pgfpathlineto{\pgfqpoint{3.114830in}{3.198591in}}%
\pgfpathlineto{\pgfqpoint{3.107012in}{3.190061in}}%
\pgfpathlineto{\pgfqpoint{3.099186in}{3.181636in}}%
\pgfpathlineto{\pgfqpoint{3.091353in}{3.173316in}}%
\pgfpathclose%
\pgfusepath{fill}%
\end{pgfscope}%
\begin{pgfscope}%
\pgfpathrectangle{\pgfqpoint{1.254980in}{0.150000in}}{\pgfqpoint{5.490039in}{5.490039in}}%
\pgfusepath{clip}%
\pgfsetbuttcap%
\pgfsetroundjoin%
\definecolor{currentfill}{rgb}{0.277134,0.185228,0.489898}%
\pgfsetfillcolor{currentfill}%
\pgfsetfillopacity{0.700000}%
\pgfsetlinewidth{0.000000pt}%
\definecolor{currentstroke}{rgb}{0.000000,0.000000,0.000000}%
\pgfsetstrokecolor{currentstroke}%
\pgfsetdash{}{0pt}%
\pgfpathmoveto{\pgfqpoint{4.444190in}{2.550834in}}%
\pgfpathlineto{\pgfqpoint{4.457256in}{2.547009in}}%
\pgfpathlineto{\pgfqpoint{4.470328in}{2.543310in}}%
\pgfpathlineto{\pgfqpoint{4.483408in}{2.539738in}}%
\pgfpathlineto{\pgfqpoint{4.496494in}{2.536291in}}%
\pgfpathlineto{\pgfqpoint{4.503863in}{2.546517in}}%
\pgfpathlineto{\pgfqpoint{4.511227in}{2.556764in}}%
\pgfpathlineto{\pgfqpoint{4.518588in}{2.567032in}}%
\pgfpathlineto{\pgfqpoint{4.525944in}{2.577323in}}%
\pgfpathlineto{\pgfqpoint{4.512867in}{2.580824in}}%
\pgfpathlineto{\pgfqpoint{4.499798in}{2.584450in}}%
\pgfpathlineto{\pgfqpoint{4.486736in}{2.588203in}}%
\pgfpathlineto{\pgfqpoint{4.473680in}{2.592082in}}%
\pgfpathlineto{\pgfqpoint{4.466314in}{2.581731in}}%
\pgfpathlineto{\pgfqpoint{4.458944in}{2.571407in}}%
\pgfpathlineto{\pgfqpoint{4.451569in}{2.561108in}}%
\pgfpathlineto{\pgfqpoint{4.444190in}{2.550834in}}%
\pgfpathclose%
\pgfusepath{fill}%
\end{pgfscope}%
\begin{pgfscope}%
\pgfpathrectangle{\pgfqpoint{1.254980in}{0.150000in}}{\pgfqpoint{5.490039in}{5.490039in}}%
\pgfusepath{clip}%
\pgfsetbuttcap%
\pgfsetroundjoin%
\definecolor{currentfill}{rgb}{0.241237,0.296485,0.539709}%
\pgfsetfillcolor{currentfill}%
\pgfsetfillopacity{0.700000}%
\pgfsetlinewidth{0.000000pt}%
\definecolor{currentstroke}{rgb}{0.000000,0.000000,0.000000}%
\pgfsetstrokecolor{currentstroke}%
\pgfsetdash{}{0pt}%
\pgfpathmoveto{\pgfqpoint{3.403192in}{2.806251in}}%
\pgfpathlineto{\pgfqpoint{3.416156in}{2.793145in}}%
\pgfpathlineto{\pgfqpoint{3.429118in}{2.780205in}}%
\pgfpathlineto{\pgfqpoint{3.442079in}{2.767430in}}%
\pgfpathlineto{\pgfqpoint{3.455039in}{2.754820in}}%
\pgfpathlineto{\pgfqpoint{3.462744in}{2.763616in}}%
\pgfpathlineto{\pgfqpoint{3.470443in}{2.772488in}}%
\pgfpathlineto{\pgfqpoint{3.478136in}{2.781437in}}%
\pgfpathlineto{\pgfqpoint{3.485822in}{2.790463in}}%
\pgfpathlineto{\pgfqpoint{3.472878in}{2.803013in}}%
\pgfpathlineto{\pgfqpoint{3.459933in}{2.815727in}}%
\pgfpathlineto{\pgfqpoint{3.446988in}{2.828607in}}%
\pgfpathlineto{\pgfqpoint{3.434041in}{2.841653in}}%
\pgfpathlineto{\pgfqpoint{3.426338in}{2.832682in}}%
\pgfpathlineto{\pgfqpoint{3.418629in}{2.823791in}}%
\pgfpathlineto{\pgfqpoint{3.410914in}{2.814981in}}%
\pgfpathlineto{\pgfqpoint{3.403192in}{2.806251in}}%
\pgfpathclose%
\pgfusepath{fill}%
\end{pgfscope}%
\begin{pgfscope}%
\pgfpathrectangle{\pgfqpoint{1.254980in}{0.150000in}}{\pgfqpoint{5.490039in}{5.490039in}}%
\pgfusepath{clip}%
\pgfsetbuttcap%
\pgfsetroundjoin%
\definecolor{currentfill}{rgb}{0.267968,0.223549,0.512008}%
\pgfsetfillcolor{currentfill}%
\pgfsetfillopacity{0.700000}%
\pgfsetlinewidth{0.000000pt}%
\definecolor{currentstroke}{rgb}{0.000000,0.000000,0.000000}%
\pgfsetstrokecolor{currentstroke}%
\pgfsetdash{}{0pt}%
\pgfpathmoveto{\pgfqpoint{4.741890in}{2.626399in}}%
\pgfpathlineto{\pgfqpoint{4.755045in}{2.624348in}}%
\pgfpathlineto{\pgfqpoint{4.768209in}{2.622419in}}%
\pgfpathlineto{\pgfqpoint{4.781381in}{2.620610in}}%
\pgfpathlineto{\pgfqpoint{4.794562in}{2.618921in}}%
\pgfpathlineto{\pgfqpoint{4.801839in}{2.629077in}}%
\pgfpathlineto{\pgfqpoint{4.809113in}{2.639250in}}%
\pgfpathlineto{\pgfqpoint{4.816382in}{2.649441in}}%
\pgfpathlineto{\pgfqpoint{4.823647in}{2.659651in}}%
\pgfpathlineto{\pgfqpoint{4.810477in}{2.661441in}}%
\pgfpathlineto{\pgfqpoint{4.797315in}{2.663351in}}%
\pgfpathlineto{\pgfqpoint{4.784162in}{2.665382in}}%
\pgfpathlineto{\pgfqpoint{4.771017in}{2.667535in}}%
\pgfpathlineto{\pgfqpoint{4.763741in}{2.657217in}}%
\pgfpathlineto{\pgfqpoint{4.756462in}{2.646923in}}%
\pgfpathlineto{\pgfqpoint{4.749178in}{2.636651in}}%
\pgfpathlineto{\pgfqpoint{4.741890in}{2.626399in}}%
\pgfpathclose%
\pgfusepath{fill}%
\end{pgfscope}%
\begin{pgfscope}%
\pgfpathrectangle{\pgfqpoint{1.254980in}{0.150000in}}{\pgfqpoint{5.490039in}{5.490039in}}%
\pgfusepath{clip}%
\pgfsetbuttcap%
\pgfsetroundjoin%
\definecolor{currentfill}{rgb}{0.162142,0.474838,0.558140}%
\pgfsetfillcolor{currentfill}%
\pgfsetfillopacity{0.700000}%
\pgfsetlinewidth{0.000000pt}%
\definecolor{currentstroke}{rgb}{0.000000,0.000000,0.000000}%
\pgfsetstrokecolor{currentstroke}%
\pgfsetdash{}{0pt}%
\pgfpathmoveto{\pgfqpoint{3.039163in}{3.244894in}}%
\pgfpathlineto{\pgfqpoint{3.052219in}{3.226705in}}%
\pgfpathlineto{\pgfqpoint{3.065269in}{3.208713in}}%
\pgfpathlineto{\pgfqpoint{3.078313in}{3.190917in}}%
\pgfpathlineto{\pgfqpoint{3.091353in}{3.173316in}}%
\pgfpathlineto{\pgfqpoint{3.099186in}{3.181636in}}%
\pgfpathlineto{\pgfqpoint{3.107012in}{3.190061in}}%
\pgfpathlineto{\pgfqpoint{3.114830in}{3.198591in}}%
\pgfpathlineto{\pgfqpoint{3.122641in}{3.207225in}}%
\pgfpathlineto{\pgfqpoint{3.109621in}{3.224762in}}%
\pgfpathlineto{\pgfqpoint{3.096597in}{3.242494in}}%
\pgfpathlineto{\pgfqpoint{3.083567in}{3.260421in}}%
\pgfpathlineto{\pgfqpoint{3.070532in}{3.278546in}}%
\pgfpathlineto{\pgfqpoint{3.062701in}{3.269970in}}%
\pgfpathlineto{\pgfqpoint{3.054863in}{3.261502in}}%
\pgfpathlineto{\pgfqpoint{3.047017in}{3.253144in}}%
\pgfpathlineto{\pgfqpoint{3.039163in}{3.244894in}}%
\pgfpathclose%
\pgfusepath{fill}%
\end{pgfscope}%
\begin{pgfscope}%
\pgfpathrectangle{\pgfqpoint{1.254980in}{0.150000in}}{\pgfqpoint{5.490039in}{5.490039in}}%
\pgfusepath{clip}%
\pgfsetbuttcap%
\pgfsetroundjoin%
\definecolor{currentfill}{rgb}{0.271828,0.209303,0.504434}%
\pgfsetfillcolor{currentfill}%
\pgfsetfillopacity{0.700000}%
\pgfsetlinewidth{0.000000pt}%
\definecolor{currentstroke}{rgb}{0.000000,0.000000,0.000000}%
\pgfsetstrokecolor{currentstroke}%
\pgfsetdash{}{0pt}%
\pgfpathmoveto{\pgfqpoint{3.692930in}{2.611331in}}%
\pgfpathlineto{\pgfqpoint{3.705881in}{2.601449in}}%
\pgfpathlineto{\pgfqpoint{3.718834in}{2.591716in}}%
\pgfpathlineto{\pgfqpoint{3.731788in}{2.582134in}}%
\pgfpathlineto{\pgfqpoint{3.744744in}{2.572700in}}%
\pgfpathlineto{\pgfqpoint{3.752351in}{2.582046in}}%
\pgfpathlineto{\pgfqpoint{3.759952in}{2.591449in}}%
\pgfpathlineto{\pgfqpoint{3.767548in}{2.600907in}}%
\pgfpathlineto{\pgfqpoint{3.775139in}{2.610422in}}%
\pgfpathlineto{\pgfqpoint{3.762197in}{2.619814in}}%
\pgfpathlineto{\pgfqpoint{3.749256in}{2.629354in}}%
\pgfpathlineto{\pgfqpoint{3.736317in}{2.639044in}}%
\pgfpathlineto{\pgfqpoint{3.723380in}{2.648883in}}%
\pgfpathlineto{\pgfqpoint{3.715776in}{2.639405in}}%
\pgfpathlineto{\pgfqpoint{3.708166in}{2.629987in}}%
\pgfpathlineto{\pgfqpoint{3.700551in}{2.620629in}}%
\pgfpathlineto{\pgfqpoint{3.692930in}{2.611331in}}%
\pgfpathclose%
\pgfusepath{fill}%
\end{pgfscope}%
\begin{pgfscope}%
\pgfpathrectangle{\pgfqpoint{1.254980in}{0.150000in}}{\pgfqpoint{5.490039in}{5.490039in}}%
\pgfusepath{clip}%
\pgfsetbuttcap%
\pgfsetroundjoin%
\definecolor{currentfill}{rgb}{0.250425,0.274290,0.533103}%
\pgfsetfillcolor{currentfill}%
\pgfsetfillopacity{0.700000}%
\pgfsetlinewidth{0.000000pt}%
\definecolor{currentstroke}{rgb}{0.000000,0.000000,0.000000}%
\pgfsetstrokecolor{currentstroke}%
\pgfsetdash{}{0pt}%
\pgfpathmoveto{\pgfqpoint{3.455039in}{2.754820in}}%
\pgfpathlineto{\pgfqpoint{3.467999in}{2.742374in}}%
\pgfpathlineto{\pgfqpoint{3.480958in}{2.730090in}}%
\pgfpathlineto{\pgfqpoint{3.493916in}{2.717968in}}%
\pgfpathlineto{\pgfqpoint{3.506874in}{2.706007in}}%
\pgfpathlineto{\pgfqpoint{3.514563in}{2.714869in}}%
\pgfpathlineto{\pgfqpoint{3.522246in}{2.723803in}}%
\pgfpathlineto{\pgfqpoint{3.529923in}{2.732810in}}%
\pgfpathlineto{\pgfqpoint{3.537593in}{2.741889in}}%
\pgfpathlineto{\pgfqpoint{3.524651in}{2.753790in}}%
\pgfpathlineto{\pgfqpoint{3.511709in}{2.765852in}}%
\pgfpathlineto{\pgfqpoint{3.498766in}{2.778076in}}%
\pgfpathlineto{\pgfqpoint{3.485822in}{2.790463in}}%
\pgfpathlineto{\pgfqpoint{3.478136in}{2.781437in}}%
\pgfpathlineto{\pgfqpoint{3.470443in}{2.772488in}}%
\pgfpathlineto{\pgfqpoint{3.462744in}{2.763616in}}%
\pgfpathlineto{\pgfqpoint{3.455039in}{2.754820in}}%
\pgfpathclose%
\pgfusepath{fill}%
\end{pgfscope}%
\begin{pgfscope}%
\pgfpathrectangle{\pgfqpoint{1.254980in}{0.150000in}}{\pgfqpoint{5.490039in}{5.490039in}}%
\pgfusepath{clip}%
\pgfsetbuttcap%
\pgfsetroundjoin%
\definecolor{currentfill}{rgb}{0.280255,0.165693,0.476498}%
\pgfsetfillcolor{currentfill}%
\pgfsetfillopacity{0.700000}%
\pgfsetlinewidth{0.000000pt}%
\definecolor{currentstroke}{rgb}{0.000000,0.000000,0.000000}%
\pgfsetstrokecolor{currentstroke}%
\pgfsetdash{}{0pt}%
\pgfpathmoveto{\pgfqpoint{4.012678in}{2.518868in}}%
\pgfpathlineto{\pgfqpoint{4.025658in}{2.511930in}}%
\pgfpathlineto{\pgfqpoint{4.038642in}{2.505131in}}%
\pgfpathlineto{\pgfqpoint{4.051630in}{2.498468in}}%
\pgfpathlineto{\pgfqpoint{4.064623in}{2.491942in}}%
\pgfpathlineto{\pgfqpoint{4.072127in}{2.501817in}}%
\pgfpathlineto{\pgfqpoint{4.079626in}{2.511730in}}%
\pgfpathlineto{\pgfqpoint{4.087121in}{2.521679in}}%
\pgfpathlineto{\pgfqpoint{4.094611in}{2.531666in}}%
\pgfpathlineto{\pgfqpoint{4.081630in}{2.538183in}}%
\pgfpathlineto{\pgfqpoint{4.068653in}{2.544836in}}%
\pgfpathlineto{\pgfqpoint{4.055680in}{2.551626in}}%
\pgfpathlineto{\pgfqpoint{4.042712in}{2.558553in}}%
\pgfpathlineto{\pgfqpoint{4.035210in}{2.548570in}}%
\pgfpathlineto{\pgfqpoint{4.027704in}{2.538628in}}%
\pgfpathlineto{\pgfqpoint{4.020193in}{2.528727in}}%
\pgfpathlineto{\pgfqpoint{4.012678in}{2.518868in}}%
\pgfpathclose%
\pgfusepath{fill}%
\end{pgfscope}%
\begin{pgfscope}%
\pgfpathrectangle{\pgfqpoint{1.254980in}{0.150000in}}{\pgfqpoint{5.490039in}{5.490039in}}%
\pgfusepath{clip}%
\pgfsetbuttcap%
\pgfsetroundjoin%
\definecolor{currentfill}{rgb}{0.278826,0.175490,0.483397}%
\pgfsetfillcolor{currentfill}%
\pgfsetfillopacity{0.700000}%
\pgfsetlinewidth{0.000000pt}%
\definecolor{currentstroke}{rgb}{0.000000,0.000000,0.000000}%
\pgfsetstrokecolor{currentstroke}%
\pgfsetdash{}{0pt}%
\pgfpathmoveto{\pgfqpoint{4.362393in}{2.526399in}}%
\pgfpathlineto{\pgfqpoint{4.375443in}{2.522101in}}%
\pgfpathlineto{\pgfqpoint{4.388499in}{2.517931in}}%
\pgfpathlineto{\pgfqpoint{4.401562in}{2.513889in}}%
\pgfpathlineto{\pgfqpoint{4.414631in}{2.509975in}}%
\pgfpathlineto{\pgfqpoint{4.422028in}{2.520156in}}%
\pgfpathlineto{\pgfqpoint{4.429419in}{2.530359in}}%
\pgfpathlineto{\pgfqpoint{4.436807in}{2.540585in}}%
\pgfpathlineto{\pgfqpoint{4.444190in}{2.550834in}}%
\pgfpathlineto{\pgfqpoint{4.431131in}{2.554787in}}%
\pgfpathlineto{\pgfqpoint{4.418079in}{2.558867in}}%
\pgfpathlineto{\pgfqpoint{4.405033in}{2.563074in}}%
\pgfpathlineto{\pgfqpoint{4.391993in}{2.567410in}}%
\pgfpathlineto{\pgfqpoint{4.384599in}{2.557117in}}%
\pgfpathlineto{\pgfqpoint{4.377202in}{2.546851in}}%
\pgfpathlineto{\pgfqpoint{4.369799in}{2.536612in}}%
\pgfpathlineto{\pgfqpoint{4.362393in}{2.526399in}}%
\pgfpathclose%
\pgfusepath{fill}%
\end{pgfscope}%
\begin{pgfscope}%
\pgfpathrectangle{\pgfqpoint{1.254980in}{0.150000in}}{\pgfqpoint{5.490039in}{5.490039in}}%
\pgfusepath{clip}%
\pgfsetbuttcap%
\pgfsetroundjoin%
\definecolor{currentfill}{rgb}{0.271828,0.209303,0.504434}%
\pgfsetfillcolor{currentfill}%
\pgfsetfillopacity{0.700000}%
\pgfsetlinewidth{0.000000pt}%
\definecolor{currentstroke}{rgb}{0.000000,0.000000,0.000000}%
\pgfsetstrokecolor{currentstroke}%
\pgfsetdash{}{0pt}%
\pgfpathmoveto{\pgfqpoint{4.660116in}{2.594660in}}%
\pgfpathlineto{\pgfqpoint{4.673249in}{2.592206in}}%
\pgfpathlineto{\pgfqpoint{4.686390in}{2.589876in}}%
\pgfpathlineto{\pgfqpoint{4.699539in}{2.587667in}}%
\pgfpathlineto{\pgfqpoint{4.712697in}{2.585580in}}%
\pgfpathlineto{\pgfqpoint{4.720001in}{2.595759in}}%
\pgfpathlineto{\pgfqpoint{4.727302in}{2.605955in}}%
\pgfpathlineto{\pgfqpoint{4.734598in}{2.616168in}}%
\pgfpathlineto{\pgfqpoint{4.741890in}{2.626399in}}%
\pgfpathlineto{\pgfqpoint{4.728743in}{2.628572in}}%
\pgfpathlineto{\pgfqpoint{4.715604in}{2.630866in}}%
\pgfpathlineto{\pgfqpoint{4.702473in}{2.633282in}}%
\pgfpathlineto{\pgfqpoint{4.689350in}{2.635821in}}%
\pgfpathlineto{\pgfqpoint{4.682048in}{2.625498in}}%
\pgfpathlineto{\pgfqpoint{4.674741in}{2.615197in}}%
\pgfpathlineto{\pgfqpoint{4.667431in}{2.604918in}}%
\pgfpathlineto{\pgfqpoint{4.660116in}{2.594660in}}%
\pgfpathclose%
\pgfusepath{fill}%
\end{pgfscope}%
\begin{pgfscope}%
\pgfpathrectangle{\pgfqpoint{1.254980in}{0.150000in}}{\pgfqpoint{5.490039in}{5.490039in}}%
\pgfusepath{clip}%
\pgfsetbuttcap%
\pgfsetroundjoin%
\definecolor{currentfill}{rgb}{0.278826,0.175490,0.483397}%
\pgfsetfillcolor{currentfill}%
\pgfsetfillopacity{0.700000}%
\pgfsetlinewidth{0.000000pt}%
\definecolor{currentstroke}{rgb}{0.000000,0.000000,0.000000}%
\pgfsetstrokecolor{currentstroke}%
\pgfsetdash{}{0pt}%
\pgfpathmoveto{\pgfqpoint{3.878759in}{2.540557in}}%
\pgfpathlineto{\pgfqpoint{3.891724in}{2.532473in}}%
\pgfpathlineto{\pgfqpoint{3.904692in}{2.524531in}}%
\pgfpathlineto{\pgfqpoint{3.917662in}{2.516731in}}%
\pgfpathlineto{\pgfqpoint{3.930636in}{2.509072in}}%
\pgfpathlineto{\pgfqpoint{3.938183in}{2.518738in}}%
\pgfpathlineto{\pgfqpoint{3.945725in}{2.528449in}}%
\pgfpathlineto{\pgfqpoint{3.953263in}{2.538204in}}%
\pgfpathlineto{\pgfqpoint{3.960795in}{2.548004in}}%
\pgfpathlineto{\pgfqpoint{3.947833in}{2.555637in}}%
\pgfpathlineto{\pgfqpoint{3.934875in}{2.563412in}}%
\pgfpathlineto{\pgfqpoint{3.921920in}{2.571328in}}%
\pgfpathlineto{\pgfqpoint{3.908968in}{2.579386in}}%
\pgfpathlineto{\pgfqpoint{3.901423in}{2.569606in}}%
\pgfpathlineto{\pgfqpoint{3.893874in}{2.559875in}}%
\pgfpathlineto{\pgfqpoint{3.886319in}{2.550192in}}%
\pgfpathlineto{\pgfqpoint{3.878759in}{2.540557in}}%
\pgfpathclose%
\pgfusepath{fill}%
\end{pgfscope}%
\begin{pgfscope}%
\pgfpathrectangle{\pgfqpoint{1.254980in}{0.150000in}}{\pgfqpoint{5.490039in}{5.490039in}}%
\pgfusepath{clip}%
\pgfsetbuttcap%
\pgfsetroundjoin%
\definecolor{currentfill}{rgb}{0.210503,0.363727,0.552206}%
\pgfsetfillcolor{currentfill}%
\pgfsetfillopacity{0.700000}%
\pgfsetlinewidth{0.000000pt}%
\definecolor{currentstroke}{rgb}{0.000000,0.000000,0.000000}%
\pgfsetstrokecolor{currentstroke}%
\pgfsetdash{}{0pt}%
\pgfpathmoveto{\pgfqpoint{5.449391in}{2.927366in}}%
\pgfpathlineto{\pgfqpoint{5.462805in}{2.928255in}}%
\pgfpathlineto{\pgfqpoint{5.476231in}{2.929255in}}%
\pgfpathlineto{\pgfqpoint{5.489669in}{2.930368in}}%
\pgfpathlineto{\pgfqpoint{5.503118in}{2.931594in}}%
\pgfpathlineto{\pgfqpoint{5.510162in}{2.941058in}}%
\pgfpathlineto{\pgfqpoint{5.517203in}{2.950560in}}%
\pgfpathlineto{\pgfqpoint{5.524240in}{2.960102in}}%
\pgfpathlineto{\pgfqpoint{5.531275in}{2.969685in}}%
\pgfpathlineto{\pgfqpoint{5.517841in}{2.968690in}}%
\pgfpathlineto{\pgfqpoint{5.504418in}{2.967807in}}%
\pgfpathlineto{\pgfqpoint{5.491007in}{2.967035in}}%
\pgfpathlineto{\pgfqpoint{5.477607in}{2.966376in}}%
\pgfpathlineto{\pgfqpoint{5.470558in}{2.956557in}}%
\pgfpathlineto{\pgfqpoint{5.463505in}{2.946784in}}%
\pgfpathlineto{\pgfqpoint{5.456449in}{2.937055in}}%
\pgfpathlineto{\pgfqpoint{5.449391in}{2.927366in}}%
\pgfpathclose%
\pgfusepath{fill}%
\end{pgfscope}%
\begin{pgfscope}%
\pgfpathrectangle{\pgfqpoint{1.254980in}{0.150000in}}{\pgfqpoint{5.490039in}{5.490039in}}%
\pgfusepath{clip}%
\pgfsetbuttcap%
\pgfsetroundjoin%
\definecolor{currentfill}{rgb}{0.150476,0.504369,0.557430}%
\pgfsetfillcolor{currentfill}%
\pgfsetfillopacity{0.700000}%
\pgfsetlinewidth{0.000000pt}%
\definecolor{currentstroke}{rgb}{0.000000,0.000000,0.000000}%
\pgfsetstrokecolor{currentstroke}%
\pgfsetdash{}{0pt}%
\pgfpathmoveto{\pgfqpoint{2.986884in}{3.319655in}}%
\pgfpathlineto{\pgfqpoint{2.999963in}{3.300662in}}%
\pgfpathlineto{\pgfqpoint{3.013035in}{3.281872in}}%
\pgfpathlineto{\pgfqpoint{3.026102in}{3.263283in}}%
\pgfpathlineto{\pgfqpoint{3.039163in}{3.244894in}}%
\pgfpathlineto{\pgfqpoint{3.047017in}{3.253144in}}%
\pgfpathlineto{\pgfqpoint{3.054863in}{3.261502in}}%
\pgfpathlineto{\pgfqpoint{3.062701in}{3.269970in}}%
\pgfpathlineto{\pgfqpoint{3.070532in}{3.278546in}}%
\pgfpathlineto{\pgfqpoint{3.057492in}{3.296870in}}%
\pgfpathlineto{\pgfqpoint{3.044446in}{3.315394in}}%
\pgfpathlineto{\pgfqpoint{3.031394in}{3.334119in}}%
\pgfpathlineto{\pgfqpoint{3.018336in}{3.353047in}}%
\pgfpathlineto{\pgfqpoint{3.010485in}{3.344529in}}%
\pgfpathlineto{\pgfqpoint{3.002626in}{3.336125in}}%
\pgfpathlineto{\pgfqpoint{2.994759in}{3.327833in}}%
\pgfpathlineto{\pgfqpoint{2.986884in}{3.319655in}}%
\pgfpathclose%
\pgfusepath{fill}%
\end{pgfscope}%
\begin{pgfscope}%
\pgfpathrectangle{\pgfqpoint{1.254980in}{0.150000in}}{\pgfqpoint{5.490039in}{5.490039in}}%
\pgfusepath{clip}%
\pgfsetbuttcap%
\pgfsetroundjoin%
\definecolor{currentfill}{rgb}{0.280868,0.160771,0.472899}%
\pgfsetfillcolor{currentfill}%
\pgfsetfillopacity{0.700000}%
\pgfsetlinewidth{0.000000pt}%
\definecolor{currentstroke}{rgb}{0.000000,0.000000,0.000000}%
\pgfsetstrokecolor{currentstroke}%
\pgfsetdash{}{0pt}%
\pgfpathmoveto{\pgfqpoint{4.146581in}{2.506956in}}%
\pgfpathlineto{\pgfqpoint{4.159585in}{2.501114in}}%
\pgfpathlineto{\pgfqpoint{4.172594in}{2.495406in}}%
\pgfpathlineto{\pgfqpoint{4.185608in}{2.489831in}}%
\pgfpathlineto{\pgfqpoint{4.198628in}{2.484389in}}%
\pgfpathlineto{\pgfqpoint{4.206091in}{2.494413in}}%
\pgfpathlineto{\pgfqpoint{4.213550in}{2.504467in}}%
\pgfpathlineto{\pgfqpoint{4.221004in}{2.514552in}}%
\pgfpathlineto{\pgfqpoint{4.228453in}{2.524667in}}%
\pgfpathlineto{\pgfqpoint{4.215445in}{2.530116in}}%
\pgfpathlineto{\pgfqpoint{4.202441in}{2.535697in}}%
\pgfpathlineto{\pgfqpoint{4.189443in}{2.541411in}}%
\pgfpathlineto{\pgfqpoint{4.176450in}{2.547259in}}%
\pgfpathlineto{\pgfqpoint{4.168989in}{2.537131in}}%
\pgfpathlineto{\pgfqpoint{4.161524in}{2.527038in}}%
\pgfpathlineto{\pgfqpoint{4.154055in}{2.516980in}}%
\pgfpathlineto{\pgfqpoint{4.146581in}{2.506956in}}%
\pgfpathclose%
\pgfusepath{fill}%
\end{pgfscope}%
\begin{pgfscope}%
\pgfpathrectangle{\pgfqpoint{1.254980in}{0.150000in}}{\pgfqpoint{5.490039in}{5.490039in}}%
\pgfusepath{clip}%
\pgfsetbuttcap%
\pgfsetroundjoin%
\definecolor{currentfill}{rgb}{0.218130,0.347432,0.550038}%
\pgfsetfillcolor{currentfill}%
\pgfsetfillopacity{0.700000}%
\pgfsetlinewidth{0.000000pt}%
\definecolor{currentstroke}{rgb}{0.000000,0.000000,0.000000}%
\pgfsetstrokecolor{currentstroke}%
\pgfsetdash{}{0pt}%
\pgfpathmoveto{\pgfqpoint{5.367518in}{2.885703in}}%
\pgfpathlineto{\pgfqpoint{5.380902in}{2.886353in}}%
\pgfpathlineto{\pgfqpoint{5.394297in}{2.887116in}}%
\pgfpathlineto{\pgfqpoint{5.407704in}{2.887992in}}%
\pgfpathlineto{\pgfqpoint{5.421121in}{2.888981in}}%
\pgfpathlineto{\pgfqpoint{5.428194in}{2.898527in}}%
\pgfpathlineto{\pgfqpoint{5.435263in}{2.908105in}}%
\pgfpathlineto{\pgfqpoint{5.442328in}{2.917717in}}%
\pgfpathlineto{\pgfqpoint{5.449391in}{2.927366in}}%
\pgfpathlineto{\pgfqpoint{5.435987in}{2.926591in}}%
\pgfpathlineto{\pgfqpoint{5.422595in}{2.925928in}}%
\pgfpathlineto{\pgfqpoint{5.409214in}{2.925378in}}%
\pgfpathlineto{\pgfqpoint{5.395844in}{2.924942in}}%
\pgfpathlineto{\pgfqpoint{5.388767in}{2.915073in}}%
\pgfpathlineto{\pgfqpoint{5.381687in}{2.905246in}}%
\pgfpathlineto{\pgfqpoint{5.374604in}{2.895456in}}%
\pgfpathlineto{\pgfqpoint{5.367518in}{2.885703in}}%
\pgfpathclose%
\pgfusepath{fill}%
\end{pgfscope}%
\begin{pgfscope}%
\pgfpathrectangle{\pgfqpoint{1.254980in}{0.150000in}}{\pgfqpoint{5.490039in}{5.490039in}}%
\pgfusepath{clip}%
\pgfsetbuttcap%
\pgfsetroundjoin%
\definecolor{currentfill}{rgb}{0.257322,0.256130,0.526563}%
\pgfsetfillcolor{currentfill}%
\pgfsetfillopacity{0.700000}%
\pgfsetlinewidth{0.000000pt}%
\definecolor{currentstroke}{rgb}{0.000000,0.000000,0.000000}%
\pgfsetstrokecolor{currentstroke}%
\pgfsetdash{}{0pt}%
\pgfpathmoveto{\pgfqpoint{3.506874in}{2.706007in}}%
\pgfpathlineto{\pgfqpoint{3.519832in}{2.694207in}}%
\pgfpathlineto{\pgfqpoint{3.532790in}{2.682566in}}%
\pgfpathlineto{\pgfqpoint{3.545748in}{2.671083in}}%
\pgfpathlineto{\pgfqpoint{3.558706in}{2.659758in}}%
\pgfpathlineto{\pgfqpoint{3.566379in}{2.668685in}}%
\pgfpathlineto{\pgfqpoint{3.574046in}{2.677681in}}%
\pgfpathlineto{\pgfqpoint{3.581707in}{2.686745in}}%
\pgfpathlineto{\pgfqpoint{3.589363in}{2.695877in}}%
\pgfpathlineto{\pgfqpoint{3.576420in}{2.707143in}}%
\pgfpathlineto{\pgfqpoint{3.563478in}{2.718566in}}%
\pgfpathlineto{\pgfqpoint{3.550536in}{2.730148in}}%
\pgfpathlineto{\pgfqpoint{3.537593in}{2.741889in}}%
\pgfpathlineto{\pgfqpoint{3.529923in}{2.732810in}}%
\pgfpathlineto{\pgfqpoint{3.522246in}{2.723803in}}%
\pgfpathlineto{\pgfqpoint{3.514563in}{2.714869in}}%
\pgfpathlineto{\pgfqpoint{3.506874in}{2.706007in}}%
\pgfpathclose%
\pgfusepath{fill}%
\end{pgfscope}%
\begin{pgfscope}%
\pgfpathrectangle{\pgfqpoint{1.254980in}{0.150000in}}{\pgfqpoint{5.490039in}{5.490039in}}%
\pgfusepath{clip}%
\pgfsetbuttcap%
\pgfsetroundjoin%
\definecolor{currentfill}{rgb}{0.225863,0.330805,0.547314}%
\pgfsetfillcolor{currentfill}%
\pgfsetfillopacity{0.700000}%
\pgfsetlinewidth{0.000000pt}%
\definecolor{currentstroke}{rgb}{0.000000,0.000000,0.000000}%
\pgfsetstrokecolor{currentstroke}%
\pgfsetdash{}{0pt}%
\pgfpathmoveto{\pgfqpoint{5.285655in}{2.844758in}}%
\pgfpathlineto{\pgfqpoint{5.299009in}{2.845150in}}%
\pgfpathlineto{\pgfqpoint{5.312374in}{2.845656in}}%
\pgfpathlineto{\pgfqpoint{5.325750in}{2.846276in}}%
\pgfpathlineto{\pgfqpoint{5.339137in}{2.847011in}}%
\pgfpathlineto{\pgfqpoint{5.346238in}{2.856640in}}%
\pgfpathlineto{\pgfqpoint{5.353335in}{2.866297in}}%
\pgfpathlineto{\pgfqpoint{5.360428in}{2.875984in}}%
\pgfpathlineto{\pgfqpoint{5.367518in}{2.885703in}}%
\pgfpathlineto{\pgfqpoint{5.354145in}{2.885166in}}%
\pgfpathlineto{\pgfqpoint{5.340782in}{2.884744in}}%
\pgfpathlineto{\pgfqpoint{5.327431in}{2.884435in}}%
\pgfpathlineto{\pgfqpoint{5.314090in}{2.884240in}}%
\pgfpathlineto{\pgfqpoint{5.306987in}{2.874318in}}%
\pgfpathlineto{\pgfqpoint{5.299880in}{2.864431in}}%
\pgfpathlineto{\pgfqpoint{5.292769in}{2.854579in}}%
\pgfpathlineto{\pgfqpoint{5.285655in}{2.844758in}}%
\pgfpathclose%
\pgfusepath{fill}%
\end{pgfscope}%
\begin{pgfscope}%
\pgfpathrectangle{\pgfqpoint{1.254980in}{0.150000in}}{\pgfqpoint{5.490039in}{5.490039in}}%
\pgfusepath{clip}%
\pgfsetbuttcap%
\pgfsetroundjoin%
\definecolor{currentfill}{rgb}{0.201239,0.383670,0.554294}%
\pgfsetfillcolor{currentfill}%
\pgfsetfillopacity{0.700000}%
\pgfsetlinewidth{0.000000pt}%
\definecolor{currentstroke}{rgb}{0.000000,0.000000,0.000000}%
\pgfsetstrokecolor{currentstroke}%
\pgfsetdash{}{0pt}%
\pgfpathmoveto{\pgfqpoint{5.531275in}{2.969685in}}%
\pgfpathlineto{\pgfqpoint{5.544720in}{2.970793in}}%
\pgfpathlineto{\pgfqpoint{5.558177in}{2.972012in}}%
\pgfpathlineto{\pgfqpoint{5.571646in}{2.973342in}}%
\pgfpathlineto{\pgfqpoint{5.585127in}{2.974785in}}%
\pgfpathlineto{\pgfqpoint{5.592143in}{2.984173in}}%
\pgfpathlineto{\pgfqpoint{5.599155in}{2.993604in}}%
\pgfpathlineto{\pgfqpoint{5.606165in}{3.003081in}}%
\pgfpathlineto{\pgfqpoint{5.592696in}{3.001822in}}%
\pgfpathlineto{\pgfqpoint{5.579239in}{3.000674in}}%
\pgfpathlineto{\pgfqpoint{5.565794in}{2.999638in}}%
\pgfpathlineto{\pgfqpoint{5.552360in}{2.998713in}}%
\pgfpathlineto{\pgfqpoint{5.545335in}{2.988989in}}%
\pgfpathlineto{\pgfqpoint{5.538306in}{2.979314in}}%
\pgfpathlineto{\pgfqpoint{5.531275in}{2.969685in}}%
\pgfpathclose%
\pgfusepath{fill}%
\end{pgfscope}%
\begin{pgfscope}%
\pgfpathrectangle{\pgfqpoint{1.254980in}{0.150000in}}{\pgfqpoint{5.490039in}{5.490039in}}%
\pgfusepath{clip}%
\pgfsetbuttcap%
\pgfsetroundjoin%
\definecolor{currentfill}{rgb}{0.233603,0.313828,0.543914}%
\pgfsetfillcolor{currentfill}%
\pgfsetfillopacity{0.700000}%
\pgfsetlinewidth{0.000000pt}%
\definecolor{currentstroke}{rgb}{0.000000,0.000000,0.000000}%
\pgfsetstrokecolor{currentstroke}%
\pgfsetdash{}{0pt}%
\pgfpathmoveto{\pgfqpoint{5.203801in}{2.804604in}}%
\pgfpathlineto{\pgfqpoint{5.217125in}{2.804719in}}%
\pgfpathlineto{\pgfqpoint{5.230460in}{2.804949in}}%
\pgfpathlineto{\pgfqpoint{5.243806in}{2.805294in}}%
\pgfpathlineto{\pgfqpoint{5.257162in}{2.805753in}}%
\pgfpathlineto{\pgfqpoint{5.264291in}{2.815466in}}%
\pgfpathlineto{\pgfqpoint{5.271416in}{2.825204in}}%
\pgfpathlineto{\pgfqpoint{5.278538in}{2.834967in}}%
\pgfpathlineto{\pgfqpoint{5.285655in}{2.844758in}}%
\pgfpathlineto{\pgfqpoint{5.272312in}{2.844480in}}%
\pgfpathlineto{\pgfqpoint{5.258979in}{2.844317in}}%
\pgfpathlineto{\pgfqpoint{5.245657in}{2.844268in}}%
\pgfpathlineto{\pgfqpoint{5.232345in}{2.844334in}}%
\pgfpathlineto{\pgfqpoint{5.225214in}{2.834356in}}%
\pgfpathlineto{\pgfqpoint{5.218080in}{2.824409in}}%
\pgfpathlineto{\pgfqpoint{5.210942in}{2.814493in}}%
\pgfpathlineto{\pgfqpoint{5.203801in}{2.804604in}}%
\pgfpathclose%
\pgfusepath{fill}%
\end{pgfscope}%
\begin{pgfscope}%
\pgfpathrectangle{\pgfqpoint{1.254980in}{0.150000in}}{\pgfqpoint{5.490039in}{5.490039in}}%
\pgfusepath{clip}%
\pgfsetbuttcap%
\pgfsetroundjoin%
\definecolor{currentfill}{rgb}{0.275191,0.194905,0.496005}%
\pgfsetfillcolor{currentfill}%
\pgfsetfillopacity{0.700000}%
\pgfsetlinewidth{0.000000pt}%
\definecolor{currentstroke}{rgb}{0.000000,0.000000,0.000000}%
\pgfsetstrokecolor{currentstroke}%
\pgfsetdash{}{0pt}%
\pgfpathmoveto{\pgfqpoint{4.578320in}{2.564573in}}%
\pgfpathlineto{\pgfqpoint{4.591432in}{2.561697in}}%
\pgfpathlineto{\pgfqpoint{4.604552in}{2.558944in}}%
\pgfpathlineto{\pgfqpoint{4.617679in}{2.556315in}}%
\pgfpathlineto{\pgfqpoint{4.630815in}{2.553809in}}%
\pgfpathlineto{\pgfqpoint{4.638146in}{2.563996in}}%
\pgfpathlineto{\pgfqpoint{4.645474in}{2.574199in}}%
\pgfpathlineto{\pgfqpoint{4.652797in}{2.584420in}}%
\pgfpathlineto{\pgfqpoint{4.660116in}{2.594660in}}%
\pgfpathlineto{\pgfqpoint{4.646991in}{2.597235in}}%
\pgfpathlineto{\pgfqpoint{4.633873in}{2.599935in}}%
\pgfpathlineto{\pgfqpoint{4.620763in}{2.602757in}}%
\pgfpathlineto{\pgfqpoint{4.607661in}{2.605703in}}%
\pgfpathlineto{\pgfqpoint{4.600332in}{2.595388in}}%
\pgfpathlineto{\pgfqpoint{4.592999in}{2.585095in}}%
\pgfpathlineto{\pgfqpoint{4.585661in}{2.574824in}}%
\pgfpathlineto{\pgfqpoint{4.578320in}{2.564573in}}%
\pgfpathclose%
\pgfusepath{fill}%
\end{pgfscope}%
\begin{pgfscope}%
\pgfpathrectangle{\pgfqpoint{1.254980in}{0.150000in}}{\pgfqpoint{5.490039in}{5.490039in}}%
\pgfusepath{clip}%
\pgfsetbuttcap%
\pgfsetroundjoin%
\definecolor{currentfill}{rgb}{0.275191,0.194905,0.496005}%
\pgfsetfillcolor{currentfill}%
\pgfsetfillopacity{0.700000}%
\pgfsetlinewidth{0.000000pt}%
\definecolor{currentstroke}{rgb}{0.000000,0.000000,0.000000}%
\pgfsetstrokecolor{currentstroke}%
\pgfsetdash{}{0pt}%
\pgfpathmoveto{\pgfqpoint{3.744744in}{2.572700in}}%
\pgfpathlineto{\pgfqpoint{3.757702in}{2.563414in}}%
\pgfpathlineto{\pgfqpoint{3.770662in}{2.554275in}}%
\pgfpathlineto{\pgfqpoint{3.783624in}{2.545283in}}%
\pgfpathlineto{\pgfqpoint{3.796588in}{2.536437in}}%
\pgfpathlineto{\pgfqpoint{3.804181in}{2.545831in}}%
\pgfpathlineto{\pgfqpoint{3.811769in}{2.555278in}}%
\pgfpathlineto{\pgfqpoint{3.819352in}{2.564776in}}%
\pgfpathlineto{\pgfqpoint{3.826929in}{2.574327in}}%
\pgfpathlineto{\pgfqpoint{3.813978in}{2.583131in}}%
\pgfpathlineto{\pgfqpoint{3.801030in}{2.592081in}}%
\pgfpathlineto{\pgfqpoint{3.788083in}{2.601178in}}%
\pgfpathlineto{\pgfqpoint{3.775139in}{2.610422in}}%
\pgfpathlineto{\pgfqpoint{3.767548in}{2.600907in}}%
\pgfpathlineto{\pgfqpoint{3.759952in}{2.591449in}}%
\pgfpathlineto{\pgfqpoint{3.752351in}{2.582046in}}%
\pgfpathlineto{\pgfqpoint{3.744744in}{2.572700in}}%
\pgfpathclose%
\pgfusepath{fill}%
\end{pgfscope}%
\begin{pgfscope}%
\pgfpathrectangle{\pgfqpoint{1.254980in}{0.150000in}}{\pgfqpoint{5.490039in}{5.490039in}}%
\pgfusepath{clip}%
\pgfsetbuttcap%
\pgfsetroundjoin%
\definecolor{currentfill}{rgb}{0.241237,0.296485,0.539709}%
\pgfsetfillcolor{currentfill}%
\pgfsetfillopacity{0.700000}%
\pgfsetlinewidth{0.000000pt}%
\definecolor{currentstroke}{rgb}{0.000000,0.000000,0.000000}%
\pgfsetstrokecolor{currentstroke}%
\pgfsetdash{}{0pt}%
\pgfpathmoveto{\pgfqpoint{5.121952in}{2.765324in}}%
\pgfpathlineto{\pgfqpoint{5.135248in}{2.765142in}}%
\pgfpathlineto{\pgfqpoint{5.148554in}{2.765076in}}%
\pgfpathlineto{\pgfqpoint{5.161870in}{2.765126in}}%
\pgfpathlineto{\pgfqpoint{5.175196in}{2.765292in}}%
\pgfpathlineto{\pgfqpoint{5.182353in}{2.775087in}}%
\pgfpathlineto{\pgfqpoint{5.189506in}{2.784903in}}%
\pgfpathlineto{\pgfqpoint{5.196655in}{2.794741in}}%
\pgfpathlineto{\pgfqpoint{5.203801in}{2.804604in}}%
\pgfpathlineto{\pgfqpoint{5.190486in}{2.804604in}}%
\pgfpathlineto{\pgfqpoint{5.177182in}{2.804720in}}%
\pgfpathlineto{\pgfqpoint{5.163889in}{2.804951in}}%
\pgfpathlineto{\pgfqpoint{5.150605in}{2.805298in}}%
\pgfpathlineto{\pgfqpoint{5.143447in}{2.795264in}}%
\pgfpathlineto{\pgfqpoint{5.136286in}{2.785258in}}%
\pgfpathlineto{\pgfqpoint{5.129121in}{2.775279in}}%
\pgfpathlineto{\pgfqpoint{5.121952in}{2.765324in}}%
\pgfpathclose%
\pgfusepath{fill}%
\end{pgfscope}%
\begin{pgfscope}%
\pgfpathrectangle{\pgfqpoint{1.254980in}{0.150000in}}{\pgfqpoint{5.490039in}{5.490039in}}%
\pgfusepath{clip}%
\pgfsetbuttcap%
\pgfsetroundjoin%
\definecolor{currentfill}{rgb}{0.139147,0.533812,0.555298}%
\pgfsetfillcolor{currentfill}%
\pgfsetfillopacity{0.700000}%
\pgfsetlinewidth{0.000000pt}%
\definecolor{currentstroke}{rgb}{0.000000,0.000000,0.000000}%
\pgfsetstrokecolor{currentstroke}%
\pgfsetdash{}{0pt}%
\pgfpathmoveto{\pgfqpoint{2.934504in}{3.397687in}}%
\pgfpathlineto{\pgfqpoint{2.947609in}{3.377867in}}%
\pgfpathlineto{\pgfqpoint{2.960707in}{3.358256in}}%
\pgfpathlineto{\pgfqpoint{2.973799in}{3.338853in}}%
\pgfpathlineto{\pgfqpoint{2.986884in}{3.319655in}}%
\pgfpathlineto{\pgfqpoint{2.994759in}{3.327833in}}%
\pgfpathlineto{\pgfqpoint{3.002626in}{3.336125in}}%
\pgfpathlineto{\pgfqpoint{3.010485in}{3.344529in}}%
\pgfpathlineto{\pgfqpoint{3.018336in}{3.353047in}}%
\pgfpathlineto{\pgfqpoint{3.005272in}{3.372179in}}%
\pgfpathlineto{\pgfqpoint{2.992202in}{3.391517in}}%
\pgfpathlineto{\pgfqpoint{2.979125in}{3.411062in}}%
\pgfpathlineto{\pgfqpoint{2.966042in}{3.430815in}}%
\pgfpathlineto{\pgfqpoint{2.958169in}{3.422357in}}%
\pgfpathlineto{\pgfqpoint{2.950289in}{3.414016in}}%
\pgfpathlineto{\pgfqpoint{2.942400in}{3.405792in}}%
\pgfpathlineto{\pgfqpoint{2.934504in}{3.397687in}}%
\pgfpathclose%
\pgfusepath{fill}%
\end{pgfscope}%
\begin{pgfscope}%
\pgfpathrectangle{\pgfqpoint{1.254980in}{0.150000in}}{\pgfqpoint{5.490039in}{5.490039in}}%
\pgfusepath{clip}%
\pgfsetbuttcap%
\pgfsetroundjoin%
\definecolor{currentfill}{rgb}{0.248629,0.278775,0.534556}%
\pgfsetfillcolor{currentfill}%
\pgfsetfillopacity{0.700000}%
\pgfsetlinewidth{0.000000pt}%
\definecolor{currentstroke}{rgb}{0.000000,0.000000,0.000000}%
\pgfsetstrokecolor{currentstroke}%
\pgfsetdash{}{0pt}%
\pgfpathmoveto{\pgfqpoint{5.040107in}{2.727009in}}%
\pgfpathlineto{\pgfqpoint{5.053375in}{2.726511in}}%
\pgfpathlineto{\pgfqpoint{5.066652in}{2.726130in}}%
\pgfpathlineto{\pgfqpoint{5.079940in}{2.725865in}}%
\pgfpathlineto{\pgfqpoint{5.093237in}{2.725717in}}%
\pgfpathlineto{\pgfqpoint{5.100422in}{2.735590in}}%
\pgfpathlineto{\pgfqpoint{5.107602in}{2.745481in}}%
\pgfpathlineto{\pgfqpoint{5.114779in}{2.755392in}}%
\pgfpathlineto{\pgfqpoint{5.121952in}{2.765324in}}%
\pgfpathlineto{\pgfqpoint{5.108666in}{2.765622in}}%
\pgfpathlineto{\pgfqpoint{5.095391in}{2.766036in}}%
\pgfpathlineto{\pgfqpoint{5.082125in}{2.766567in}}%
\pgfpathlineto{\pgfqpoint{5.068869in}{2.767214in}}%
\pgfpathlineto{\pgfqpoint{5.061684in}{2.757127in}}%
\pgfpathlineto{\pgfqpoint{5.054496in}{2.747065in}}%
\pgfpathlineto{\pgfqpoint{5.047303in}{2.737026in}}%
\pgfpathlineto{\pgfqpoint{5.040107in}{2.727009in}}%
\pgfpathclose%
\pgfusepath{fill}%
\end{pgfscope}%
\begin{pgfscope}%
\pgfpathrectangle{\pgfqpoint{1.254980in}{0.150000in}}{\pgfqpoint{5.490039in}{5.490039in}}%
\pgfusepath{clip}%
\pgfsetbuttcap%
\pgfsetroundjoin%
\definecolor{currentfill}{rgb}{0.280255,0.165693,0.476498}%
\pgfsetfillcolor{currentfill}%
\pgfsetfillopacity{0.700000}%
\pgfsetlinewidth{0.000000pt}%
\definecolor{currentstroke}{rgb}{0.000000,0.000000,0.000000}%
\pgfsetstrokecolor{currentstroke}%
\pgfsetdash{}{0pt}%
\pgfpathmoveto{\pgfqpoint{4.280542in}{2.504189in}}%
\pgfpathlineto{\pgfqpoint{4.293579in}{2.499396in}}%
\pgfpathlineto{\pgfqpoint{4.306621in}{2.494734in}}%
\pgfpathlineto{\pgfqpoint{4.319669in}{2.490201in}}%
\pgfpathlineto{\pgfqpoint{4.332723in}{2.485796in}}%
\pgfpathlineto{\pgfqpoint{4.340147in}{2.495910in}}%
\pgfpathlineto{\pgfqpoint{4.347567in}{2.506049in}}%
\pgfpathlineto{\pgfqpoint{4.354982in}{2.516211in}}%
\pgfpathlineto{\pgfqpoint{4.362393in}{2.526399in}}%
\pgfpathlineto{\pgfqpoint{4.349349in}{2.530825in}}%
\pgfpathlineto{\pgfqpoint{4.336311in}{2.535381in}}%
\pgfpathlineto{\pgfqpoint{4.323280in}{2.540065in}}%
\pgfpathlineto{\pgfqpoint{4.310254in}{2.544880in}}%
\pgfpathlineto{\pgfqpoint{4.302833in}{2.534665in}}%
\pgfpathlineto{\pgfqpoint{4.295407in}{2.524478in}}%
\pgfpathlineto{\pgfqpoint{4.287977in}{2.514320in}}%
\pgfpathlineto{\pgfqpoint{4.280542in}{2.504189in}}%
\pgfpathclose%
\pgfusepath{fill}%
\end{pgfscope}%
\begin{pgfscope}%
\pgfpathrectangle{\pgfqpoint{1.254980in}{0.150000in}}{\pgfqpoint{5.490039in}{5.490039in}}%
\pgfusepath{clip}%
\pgfsetbuttcap%
\pgfsetroundjoin%
\definecolor{currentfill}{rgb}{0.263663,0.237631,0.518762}%
\pgfsetfillcolor{currentfill}%
\pgfsetfillopacity{0.700000}%
\pgfsetlinewidth{0.000000pt}%
\definecolor{currentstroke}{rgb}{0.000000,0.000000,0.000000}%
\pgfsetstrokecolor{currentstroke}%
\pgfsetdash{}{0pt}%
\pgfpathmoveto{\pgfqpoint{3.558706in}{2.659758in}}%
\pgfpathlineto{\pgfqpoint{3.571665in}{2.648590in}}%
\pgfpathlineto{\pgfqpoint{3.584623in}{2.637578in}}%
\pgfpathlineto{\pgfqpoint{3.597583in}{2.626721in}}%
\pgfpathlineto{\pgfqpoint{3.610543in}{2.616019in}}%
\pgfpathlineto{\pgfqpoint{3.618201in}{2.625011in}}%
\pgfpathlineto{\pgfqpoint{3.625852in}{2.634068in}}%
\pgfpathlineto{\pgfqpoint{3.633499in}{2.643189in}}%
\pgfpathlineto{\pgfqpoint{3.641139in}{2.652374in}}%
\pgfpathlineto{\pgfqpoint{3.628194in}{2.663017in}}%
\pgfpathlineto{\pgfqpoint{3.615250in}{2.673815in}}%
\pgfpathlineto{\pgfqpoint{3.602306in}{2.684768in}}%
\pgfpathlineto{\pgfqpoint{3.589363in}{2.695877in}}%
\pgfpathlineto{\pgfqpoint{3.581707in}{2.686745in}}%
\pgfpathlineto{\pgfqpoint{3.574046in}{2.677681in}}%
\pgfpathlineto{\pgfqpoint{3.566379in}{2.668685in}}%
\pgfpathlineto{\pgfqpoint{3.558706in}{2.659758in}}%
\pgfpathclose%
\pgfusepath{fill}%
\end{pgfscope}%
\begin{pgfscope}%
\pgfpathrectangle{\pgfqpoint{1.254980in}{0.150000in}}{\pgfqpoint{5.490039in}{5.490039in}}%
\pgfusepath{clip}%
\pgfsetbuttcap%
\pgfsetroundjoin%
\definecolor{currentfill}{rgb}{0.255645,0.260703,0.528312}%
\pgfsetfillcolor{currentfill}%
\pgfsetfillopacity{0.700000}%
\pgfsetlinewidth{0.000000pt}%
\definecolor{currentstroke}{rgb}{0.000000,0.000000,0.000000}%
\pgfsetstrokecolor{currentstroke}%
\pgfsetdash{}{0pt}%
\pgfpathmoveto{\pgfqpoint{4.958263in}{2.689762in}}%
\pgfpathlineto{\pgfqpoint{4.971503in}{2.688927in}}%
\pgfpathlineto{\pgfqpoint{4.984753in}{2.688210in}}%
\pgfpathlineto{\pgfqpoint{4.998013in}{2.687611in}}%
\pgfpathlineto{\pgfqpoint{5.011282in}{2.687130in}}%
\pgfpathlineto{\pgfqpoint{5.018494in}{2.697074in}}%
\pgfpathlineto{\pgfqpoint{5.025702in}{2.707035in}}%
\pgfpathlineto{\pgfqpoint{5.032907in}{2.717013in}}%
\pgfpathlineto{\pgfqpoint{5.040107in}{2.727009in}}%
\pgfpathlineto{\pgfqpoint{5.026849in}{2.727625in}}%
\pgfpathlineto{\pgfqpoint{5.013601in}{2.728357in}}%
\pgfpathlineto{\pgfqpoint{5.000362in}{2.729207in}}%
\pgfpathlineto{\pgfqpoint{4.987132in}{2.730175in}}%
\pgfpathlineto{\pgfqpoint{4.979921in}{2.720039in}}%
\pgfpathlineto{\pgfqpoint{4.972706in}{2.709926in}}%
\pgfpathlineto{\pgfqpoint{4.965486in}{2.699834in}}%
\pgfpathlineto{\pgfqpoint{4.958263in}{2.689762in}}%
\pgfpathclose%
\pgfusepath{fill}%
\end{pgfscope}%
\begin{pgfscope}%
\pgfpathrectangle{\pgfqpoint{1.254980in}{0.150000in}}{\pgfqpoint{5.490039in}{5.490039in}}%
\pgfusepath{clip}%
\pgfsetbuttcap%
\pgfsetroundjoin%
\definecolor{currentfill}{rgb}{0.277134,0.185228,0.489898}%
\pgfsetfillcolor{currentfill}%
\pgfsetfillopacity{0.700000}%
\pgfsetlinewidth{0.000000pt}%
\definecolor{currentstroke}{rgb}{0.000000,0.000000,0.000000}%
\pgfsetstrokecolor{currentstroke}%
\pgfsetdash{}{0pt}%
\pgfpathmoveto{\pgfqpoint{4.496494in}{2.536291in}}%
\pgfpathlineto{\pgfqpoint{4.509587in}{2.532970in}}%
\pgfpathlineto{\pgfqpoint{4.522687in}{2.529774in}}%
\pgfpathlineto{\pgfqpoint{4.535795in}{2.526703in}}%
\pgfpathlineto{\pgfqpoint{4.548910in}{2.523757in}}%
\pgfpathlineto{\pgfqpoint{4.556269in}{2.533934in}}%
\pgfpathlineto{\pgfqpoint{4.563623in}{2.544129in}}%
\pgfpathlineto{\pgfqpoint{4.570974in}{2.554342in}}%
\pgfpathlineto{\pgfqpoint{4.578320in}{2.564573in}}%
\pgfpathlineto{\pgfqpoint{4.565215in}{2.567574in}}%
\pgfpathlineto{\pgfqpoint{4.552117in}{2.570699in}}%
\pgfpathlineto{\pgfqpoint{4.539027in}{2.573949in}}%
\pgfpathlineto{\pgfqpoint{4.525944in}{2.577323in}}%
\pgfpathlineto{\pgfqpoint{4.518588in}{2.567032in}}%
\pgfpathlineto{\pgfqpoint{4.511227in}{2.556764in}}%
\pgfpathlineto{\pgfqpoint{4.503863in}{2.546517in}}%
\pgfpathlineto{\pgfqpoint{4.496494in}{2.536291in}}%
\pgfpathclose%
\pgfusepath{fill}%
\end{pgfscope}%
\begin{pgfscope}%
\pgfpathrectangle{\pgfqpoint{1.254980in}{0.150000in}}{\pgfqpoint{5.490039in}{5.490039in}}%
\pgfusepath{clip}%
\pgfsetbuttcap%
\pgfsetroundjoin%
\definecolor{currentfill}{rgb}{0.280255,0.165693,0.476498}%
\pgfsetfillcolor{currentfill}%
\pgfsetfillopacity{0.700000}%
\pgfsetlinewidth{0.000000pt}%
\definecolor{currentstroke}{rgb}{0.000000,0.000000,0.000000}%
\pgfsetstrokecolor{currentstroke}%
\pgfsetdash{}{0pt}%
\pgfpathmoveto{\pgfqpoint{3.930636in}{2.509072in}}%
\pgfpathlineto{\pgfqpoint{3.943614in}{2.501553in}}%
\pgfpathlineto{\pgfqpoint{3.956594in}{2.494174in}}%
\pgfpathlineto{\pgfqpoint{3.969579in}{2.486934in}}%
\pgfpathlineto{\pgfqpoint{3.982567in}{2.479833in}}%
\pgfpathlineto{\pgfqpoint{3.990102in}{2.489531in}}%
\pgfpathlineto{\pgfqpoint{3.997632in}{2.499269in}}%
\pgfpathlineto{\pgfqpoint{4.005157in}{2.509048in}}%
\pgfpathlineto{\pgfqpoint{4.012678in}{2.518868in}}%
\pgfpathlineto{\pgfqpoint{3.999702in}{2.525943in}}%
\pgfpathlineto{\pgfqpoint{3.986729in}{2.533157in}}%
\pgfpathlineto{\pgfqpoint{3.973760in}{2.540510in}}%
\pgfpathlineto{\pgfqpoint{3.960795in}{2.548004in}}%
\pgfpathlineto{\pgfqpoint{3.953263in}{2.538204in}}%
\pgfpathlineto{\pgfqpoint{3.945725in}{2.528449in}}%
\pgfpathlineto{\pgfqpoint{3.938183in}{2.518738in}}%
\pgfpathlineto{\pgfqpoint{3.930636in}{2.509072in}}%
\pgfpathclose%
\pgfusepath{fill}%
\end{pgfscope}%
\begin{pgfscope}%
\pgfpathrectangle{\pgfqpoint{1.254980in}{0.150000in}}{\pgfqpoint{5.490039in}{5.490039in}}%
\pgfusepath{clip}%
\pgfsetbuttcap%
\pgfsetroundjoin%
\definecolor{currentfill}{rgb}{0.280868,0.160771,0.472899}%
\pgfsetfillcolor{currentfill}%
\pgfsetfillopacity{0.700000}%
\pgfsetlinewidth{0.000000pt}%
\definecolor{currentstroke}{rgb}{0.000000,0.000000,0.000000}%
\pgfsetstrokecolor{currentstroke}%
\pgfsetdash{}{0pt}%
\pgfpathmoveto{\pgfqpoint{4.064623in}{2.491942in}}%
\pgfpathlineto{\pgfqpoint{4.077620in}{2.485552in}}%
\pgfpathlineto{\pgfqpoint{4.090621in}{2.479297in}}%
\pgfpathlineto{\pgfqpoint{4.103627in}{2.473178in}}%
\pgfpathlineto{\pgfqpoint{4.116637in}{2.467193in}}%
\pgfpathlineto{\pgfqpoint{4.124130in}{2.477084in}}%
\pgfpathlineto{\pgfqpoint{4.131618in}{2.487008in}}%
\pgfpathlineto{\pgfqpoint{4.139102in}{2.496965in}}%
\pgfpathlineto{\pgfqpoint{4.146581in}{2.506956in}}%
\pgfpathlineto{\pgfqpoint{4.133581in}{2.512931in}}%
\pgfpathlineto{\pgfqpoint{4.120586in}{2.519041in}}%
\pgfpathlineto{\pgfqpoint{4.107596in}{2.525286in}}%
\pgfpathlineto{\pgfqpoint{4.094611in}{2.531666in}}%
\pgfpathlineto{\pgfqpoint{4.087121in}{2.521679in}}%
\pgfpathlineto{\pgfqpoint{4.079626in}{2.511730in}}%
\pgfpathlineto{\pgfqpoint{4.072127in}{2.501817in}}%
\pgfpathlineto{\pgfqpoint{4.064623in}{2.491942in}}%
\pgfpathclose%
\pgfusepath{fill}%
\end{pgfscope}%
\begin{pgfscope}%
\pgfpathrectangle{\pgfqpoint{1.254980in}{0.150000in}}{\pgfqpoint{5.490039in}{5.490039in}}%
\pgfusepath{clip}%
\pgfsetbuttcap%
\pgfsetroundjoin%
\definecolor{currentfill}{rgb}{0.260571,0.246922,0.522828}%
\pgfsetfillcolor{currentfill}%
\pgfsetfillopacity{0.700000}%
\pgfsetlinewidth{0.000000pt}%
\definecolor{currentstroke}{rgb}{0.000000,0.000000,0.000000}%
\pgfsetstrokecolor{currentstroke}%
\pgfsetdash{}{0pt}%
\pgfpathmoveto{\pgfqpoint{4.876416in}{2.653692in}}%
\pgfpathlineto{\pgfqpoint{4.889630in}{2.652501in}}%
\pgfpathlineto{\pgfqpoint{4.902853in}{2.651429in}}%
\pgfpathlineto{\pgfqpoint{4.916086in}{2.650475in}}%
\pgfpathlineto{\pgfqpoint{4.929328in}{2.649640in}}%
\pgfpathlineto{\pgfqpoint{4.936568in}{2.659648in}}%
\pgfpathlineto{\pgfqpoint{4.943803in}{2.669670in}}%
\pgfpathlineto{\pgfqpoint{4.951035in}{2.679707in}}%
\pgfpathlineto{\pgfqpoint{4.958263in}{2.689762in}}%
\pgfpathlineto{\pgfqpoint{4.945031in}{2.690714in}}%
\pgfpathlineto{\pgfqpoint{4.931810in}{2.691785in}}%
\pgfpathlineto{\pgfqpoint{4.918597in}{2.692975in}}%
\pgfpathlineto{\pgfqpoint{4.905393in}{2.694284in}}%
\pgfpathlineto{\pgfqpoint{4.898155in}{2.684106in}}%
\pgfpathlineto{\pgfqpoint{4.890913in}{2.673949in}}%
\pgfpathlineto{\pgfqpoint{4.883666in}{2.663812in}}%
\pgfpathlineto{\pgfqpoint{4.876416in}{2.653692in}}%
\pgfpathclose%
\pgfusepath{fill}%
\end{pgfscope}%
\begin{pgfscope}%
\pgfpathrectangle{\pgfqpoint{1.254980in}{0.150000in}}{\pgfqpoint{5.490039in}{5.490039in}}%
\pgfusepath{clip}%
\pgfsetbuttcap%
\pgfsetroundjoin%
\definecolor{currentfill}{rgb}{0.269308,0.218818,0.509577}%
\pgfsetfillcolor{currentfill}%
\pgfsetfillopacity{0.700000}%
\pgfsetlinewidth{0.000000pt}%
\definecolor{currentstroke}{rgb}{0.000000,0.000000,0.000000}%
\pgfsetstrokecolor{currentstroke}%
\pgfsetdash{}{0pt}%
\pgfpathmoveto{\pgfqpoint{3.610543in}{2.616019in}}%
\pgfpathlineto{\pgfqpoint{3.623504in}{2.605471in}}%
\pgfpathlineto{\pgfqpoint{3.636466in}{2.595076in}}%
\pgfpathlineto{\pgfqpoint{3.649428in}{2.584833in}}%
\pgfpathlineto{\pgfqpoint{3.662393in}{2.574742in}}%
\pgfpathlineto{\pgfqpoint{3.670035in}{2.583798in}}%
\pgfpathlineto{\pgfqpoint{3.677673in}{2.592915in}}%
\pgfpathlineto{\pgfqpoint{3.685304in}{2.602093in}}%
\pgfpathlineto{\pgfqpoint{3.692930in}{2.611331in}}%
\pgfpathlineto{\pgfqpoint{3.679981in}{2.621364in}}%
\pgfpathlineto{\pgfqpoint{3.667032in}{2.631548in}}%
\pgfpathlineto{\pgfqpoint{3.654085in}{2.641885in}}%
\pgfpathlineto{\pgfqpoint{3.641139in}{2.652374in}}%
\pgfpathlineto{\pgfqpoint{3.633499in}{2.643189in}}%
\pgfpathlineto{\pgfqpoint{3.625852in}{2.634068in}}%
\pgfpathlineto{\pgfqpoint{3.618201in}{2.625011in}}%
\pgfpathlineto{\pgfqpoint{3.610543in}{2.616019in}}%
\pgfpathclose%
\pgfusepath{fill}%
\end{pgfscope}%
\begin{pgfscope}%
\pgfpathrectangle{\pgfqpoint{1.254980in}{0.150000in}}{\pgfqpoint{5.490039in}{5.490039in}}%
\pgfusepath{clip}%
\pgfsetbuttcap%
\pgfsetroundjoin%
\definecolor{currentfill}{rgb}{0.210503,0.363727,0.552206}%
\pgfsetfillcolor{currentfill}%
\pgfsetfillopacity{0.700000}%
\pgfsetlinewidth{0.000000pt}%
\definecolor{currentstroke}{rgb}{0.000000,0.000000,0.000000}%
\pgfsetstrokecolor{currentstroke}%
\pgfsetdash{}{0pt}%
\pgfpathmoveto{\pgfqpoint{3.216322in}{2.943677in}}%
\pgfpathlineto{\pgfqpoint{3.229329in}{2.928420in}}%
\pgfpathlineto{\pgfqpoint{3.242334in}{2.913341in}}%
\pgfpathlineto{\pgfqpoint{3.255335in}{2.898440in}}%
\pgfpathlineto{\pgfqpoint{3.268333in}{2.883716in}}%
\pgfpathlineto{\pgfqpoint{3.276117in}{2.891953in}}%
\pgfpathlineto{\pgfqpoint{3.283894in}{2.900281in}}%
\pgfpathlineto{\pgfqpoint{3.291664in}{2.908698in}}%
\pgfpathlineto{\pgfqpoint{3.299427in}{2.917204in}}%
\pgfpathlineto{\pgfqpoint{3.286447in}{2.931851in}}%
\pgfpathlineto{\pgfqpoint{3.273465in}{2.946674in}}%
\pgfpathlineto{\pgfqpoint{3.260479in}{2.961674in}}%
\pgfpathlineto{\pgfqpoint{3.247491in}{2.976853in}}%
\pgfpathlineto{\pgfqpoint{3.239709in}{2.968419in}}%
\pgfpathlineto{\pgfqpoint{3.231921in}{2.960077in}}%
\pgfpathlineto{\pgfqpoint{3.224125in}{2.951830in}}%
\pgfpathlineto{\pgfqpoint{3.216322in}{2.943677in}}%
\pgfpathclose%
\pgfusepath{fill}%
\end{pgfscope}%
\begin{pgfscope}%
\pgfpathrectangle{\pgfqpoint{1.254980in}{0.150000in}}{\pgfqpoint{5.490039in}{5.490039in}}%
\pgfusepath{clip}%
\pgfsetbuttcap%
\pgfsetroundjoin%
\definecolor{currentfill}{rgb}{0.278012,0.180367,0.486697}%
\pgfsetfillcolor{currentfill}%
\pgfsetfillopacity{0.700000}%
\pgfsetlinewidth{0.000000pt}%
\definecolor{currentstroke}{rgb}{0.000000,0.000000,0.000000}%
\pgfsetstrokecolor{currentstroke}%
\pgfsetdash{}{0pt}%
\pgfpathmoveto{\pgfqpoint{3.796588in}{2.536437in}}%
\pgfpathlineto{\pgfqpoint{3.809555in}{2.527736in}}%
\pgfpathlineto{\pgfqpoint{3.822524in}{2.519180in}}%
\pgfpathlineto{\pgfqpoint{3.835496in}{2.510768in}}%
\pgfpathlineto{\pgfqpoint{3.848470in}{2.502499in}}%
\pgfpathlineto{\pgfqpoint{3.856050in}{2.511941in}}%
\pgfpathlineto{\pgfqpoint{3.863625in}{2.521432in}}%
\pgfpathlineto{\pgfqpoint{3.871195in}{2.530970in}}%
\pgfpathlineto{\pgfqpoint{3.878759in}{2.540557in}}%
\pgfpathlineto{\pgfqpoint{3.865798in}{2.548784in}}%
\pgfpathlineto{\pgfqpoint{3.852839in}{2.557154in}}%
\pgfpathlineto{\pgfqpoint{3.839883in}{2.565668in}}%
\pgfpathlineto{\pgfqpoint{3.826929in}{2.574327in}}%
\pgfpathlineto{\pgfqpoint{3.819352in}{2.564776in}}%
\pgfpathlineto{\pgfqpoint{3.811769in}{2.555278in}}%
\pgfpathlineto{\pgfqpoint{3.804181in}{2.545831in}}%
\pgfpathlineto{\pgfqpoint{3.796588in}{2.536437in}}%
\pgfpathclose%
\pgfusepath{fill}%
\end{pgfscope}%
\begin{pgfscope}%
\pgfpathrectangle{\pgfqpoint{1.254980in}{0.150000in}}{\pgfqpoint{5.490039in}{5.490039in}}%
\pgfusepath{clip}%
\pgfsetbuttcap%
\pgfsetroundjoin%
\definecolor{currentfill}{rgb}{0.199430,0.387607,0.554642}%
\pgfsetfillcolor{currentfill}%
\pgfsetfillopacity{0.700000}%
\pgfsetlinewidth{0.000000pt}%
\definecolor{currentstroke}{rgb}{0.000000,0.000000,0.000000}%
\pgfsetstrokecolor{currentstroke}%
\pgfsetdash{}{0pt}%
\pgfpathmoveto{\pgfqpoint{3.164260in}{3.006511in}}%
\pgfpathlineto{\pgfqpoint{3.177281in}{2.990529in}}%
\pgfpathlineto{\pgfqpoint{3.190298in}{2.974730in}}%
\pgfpathlineto{\pgfqpoint{3.203312in}{2.959113in}}%
\pgfpathlineto{\pgfqpoint{3.216322in}{2.943677in}}%
\pgfpathlineto{\pgfqpoint{3.224125in}{2.951830in}}%
\pgfpathlineto{\pgfqpoint{3.231921in}{2.960077in}}%
\pgfpathlineto{\pgfqpoint{3.239709in}{2.968419in}}%
\pgfpathlineto{\pgfqpoint{3.247491in}{2.976853in}}%
\pgfpathlineto{\pgfqpoint{3.234500in}{2.992211in}}%
\pgfpathlineto{\pgfqpoint{3.221506in}{3.007750in}}%
\pgfpathlineto{\pgfqpoint{3.208508in}{3.023470in}}%
\pgfpathlineto{\pgfqpoint{3.195507in}{3.039373in}}%
\pgfpathlineto{\pgfqpoint{3.187706in}{3.031011in}}%
\pgfpathlineto{\pgfqpoint{3.179898in}{3.022746in}}%
\pgfpathlineto{\pgfqpoint{3.172083in}{3.014580in}}%
\pgfpathlineto{\pgfqpoint{3.164260in}{3.006511in}}%
\pgfpathclose%
\pgfusepath{fill}%
\end{pgfscope}%
\begin{pgfscope}%
\pgfpathrectangle{\pgfqpoint{1.254980in}{0.150000in}}{\pgfqpoint{5.490039in}{5.490039in}}%
\pgfusepath{clip}%
\pgfsetbuttcap%
\pgfsetroundjoin%
\definecolor{currentfill}{rgb}{0.221989,0.339161,0.548752}%
\pgfsetfillcolor{currentfill}%
\pgfsetfillopacity{0.700000}%
\pgfsetlinewidth{0.000000pt}%
\definecolor{currentstroke}{rgb}{0.000000,0.000000,0.000000}%
\pgfsetstrokecolor{currentstroke}%
\pgfsetdash{}{0pt}%
\pgfpathmoveto{\pgfqpoint{3.268333in}{2.883716in}}%
\pgfpathlineto{\pgfqpoint{3.281329in}{2.869167in}}%
\pgfpathlineto{\pgfqpoint{3.294323in}{2.854792in}}%
\pgfpathlineto{\pgfqpoint{3.307314in}{2.840590in}}%
\pgfpathlineto{\pgfqpoint{3.320303in}{2.826560in}}%
\pgfpathlineto{\pgfqpoint{3.328069in}{2.834881in}}%
\pgfpathlineto{\pgfqpoint{3.335827in}{2.843288in}}%
\pgfpathlineto{\pgfqpoint{3.343579in}{2.851781in}}%
\pgfpathlineto{\pgfqpoint{3.351325in}{2.860359in}}%
\pgfpathlineto{\pgfqpoint{3.338353in}{2.874311in}}%
\pgfpathlineto{\pgfqpoint{3.325380in}{2.888435in}}%
\pgfpathlineto{\pgfqpoint{3.312405in}{2.902733in}}%
\pgfpathlineto{\pgfqpoint{3.299427in}{2.917204in}}%
\pgfpathlineto{\pgfqpoint{3.291664in}{2.908698in}}%
\pgfpathlineto{\pgfqpoint{3.283894in}{2.900281in}}%
\pgfpathlineto{\pgfqpoint{3.276117in}{2.891953in}}%
\pgfpathlineto{\pgfqpoint{3.268333in}{2.883716in}}%
\pgfpathclose%
\pgfusepath{fill}%
\end{pgfscope}%
\begin{pgfscope}%
\pgfpathrectangle{\pgfqpoint{1.254980in}{0.150000in}}{\pgfqpoint{5.490039in}{5.490039in}}%
\pgfusepath{clip}%
\pgfsetbuttcap%
\pgfsetroundjoin%
\definecolor{currentfill}{rgb}{0.266580,0.228262,0.514349}%
\pgfsetfillcolor{currentfill}%
\pgfsetfillopacity{0.700000}%
\pgfsetlinewidth{0.000000pt}%
\definecolor{currentstroke}{rgb}{0.000000,0.000000,0.000000}%
\pgfsetstrokecolor{currentstroke}%
\pgfsetdash{}{0pt}%
\pgfpathmoveto{\pgfqpoint{4.794562in}{2.618921in}}%
\pgfpathlineto{\pgfqpoint{4.807751in}{2.617354in}}%
\pgfpathlineto{\pgfqpoint{4.820949in}{2.615906in}}%
\pgfpathlineto{\pgfqpoint{4.834156in}{2.614578in}}%
\pgfpathlineto{\pgfqpoint{4.847371in}{2.613369in}}%
\pgfpathlineto{\pgfqpoint{4.854639in}{2.623429in}}%
\pgfpathlineto{\pgfqpoint{4.861902in}{2.633502in}}%
\pgfpathlineto{\pgfqpoint{4.869161in}{2.643589in}}%
\pgfpathlineto{\pgfqpoint{4.876416in}{2.653692in}}%
\pgfpathlineto{\pgfqpoint{4.863210in}{2.655002in}}%
\pgfpathlineto{\pgfqpoint{4.850014in}{2.656432in}}%
\pgfpathlineto{\pgfqpoint{4.836826in}{2.657982in}}%
\pgfpathlineto{\pgfqpoint{4.823647in}{2.659651in}}%
\pgfpathlineto{\pgfqpoint{4.816382in}{2.649441in}}%
\pgfpathlineto{\pgfqpoint{4.809113in}{2.639250in}}%
\pgfpathlineto{\pgfqpoint{4.801839in}{2.629077in}}%
\pgfpathlineto{\pgfqpoint{4.794562in}{2.618921in}}%
\pgfpathclose%
\pgfusepath{fill}%
\end{pgfscope}%
\begin{pgfscope}%
\pgfpathrectangle{\pgfqpoint{1.254980in}{0.150000in}}{\pgfqpoint{5.490039in}{5.490039in}}%
\pgfusepath{clip}%
\pgfsetbuttcap%
\pgfsetroundjoin%
\definecolor{currentfill}{rgb}{0.281412,0.155834,0.469201}%
\pgfsetfillcolor{currentfill}%
\pgfsetfillopacity{0.700000}%
\pgfsetlinewidth{0.000000pt}%
\definecolor{currentstroke}{rgb}{0.000000,0.000000,0.000000}%
\pgfsetstrokecolor{currentstroke}%
\pgfsetdash{}{0pt}%
\pgfpathmoveto{\pgfqpoint{4.198628in}{2.484389in}}%
\pgfpathlineto{\pgfqpoint{4.211652in}{2.479079in}}%
\pgfpathlineto{\pgfqpoint{4.224682in}{2.473901in}}%
\pgfpathlineto{\pgfqpoint{4.237718in}{2.468853in}}%
\pgfpathlineto{\pgfqpoint{4.250759in}{2.463937in}}%
\pgfpathlineto{\pgfqpoint{4.258211in}{2.473960in}}%
\pgfpathlineto{\pgfqpoint{4.265660in}{2.484010in}}%
\pgfpathlineto{\pgfqpoint{4.273103in}{2.494086in}}%
\pgfpathlineto{\pgfqpoint{4.280542in}{2.504189in}}%
\pgfpathlineto{\pgfqpoint{4.267512in}{2.509112in}}%
\pgfpathlineto{\pgfqpoint{4.254487in}{2.514166in}}%
\pgfpathlineto{\pgfqpoint{4.241467in}{2.519351in}}%
\pgfpathlineto{\pgfqpoint{4.228453in}{2.524667in}}%
\pgfpathlineto{\pgfqpoint{4.221004in}{2.514552in}}%
\pgfpathlineto{\pgfqpoint{4.213550in}{2.504467in}}%
\pgfpathlineto{\pgfqpoint{4.206091in}{2.494413in}}%
\pgfpathlineto{\pgfqpoint{4.198628in}{2.484389in}}%
\pgfpathclose%
\pgfusepath{fill}%
\end{pgfscope}%
\begin{pgfscope}%
\pgfpathrectangle{\pgfqpoint{1.254980in}{0.150000in}}{\pgfqpoint{5.490039in}{5.490039in}}%
\pgfusepath{clip}%
\pgfsetbuttcap%
\pgfsetroundjoin%
\definecolor{currentfill}{rgb}{0.187231,0.414746,0.556547}%
\pgfsetfillcolor{currentfill}%
\pgfsetfillopacity{0.700000}%
\pgfsetlinewidth{0.000000pt}%
\definecolor{currentstroke}{rgb}{0.000000,0.000000,0.000000}%
\pgfsetstrokecolor{currentstroke}%
\pgfsetdash{}{0pt}%
\pgfpathmoveto{\pgfqpoint{3.112138in}{3.072291in}}%
\pgfpathlineto{\pgfqpoint{3.125175in}{3.055566in}}%
\pgfpathlineto{\pgfqpoint{3.138208in}{3.039028in}}%
\pgfpathlineto{\pgfqpoint{3.151236in}{3.022677in}}%
\pgfpathlineto{\pgfqpoint{3.164260in}{3.006511in}}%
\pgfpathlineto{\pgfqpoint{3.172083in}{3.014580in}}%
\pgfpathlineto{\pgfqpoint{3.179898in}{3.022746in}}%
\pgfpathlineto{\pgfqpoint{3.187706in}{3.031011in}}%
\pgfpathlineto{\pgfqpoint{3.195507in}{3.039373in}}%
\pgfpathlineto{\pgfqpoint{3.182502in}{3.055459in}}%
\pgfpathlineto{\pgfqpoint{3.169493in}{3.071731in}}%
\pgfpathlineto{\pgfqpoint{3.156481in}{3.088189in}}%
\pgfpathlineto{\pgfqpoint{3.143464in}{3.104835in}}%
\pgfpathlineto{\pgfqpoint{3.135644in}{3.096546in}}%
\pgfpathlineto{\pgfqpoint{3.127816in}{3.088359in}}%
\pgfpathlineto{\pgfqpoint{3.119981in}{3.080274in}}%
\pgfpathlineto{\pgfqpoint{3.112138in}{3.072291in}}%
\pgfpathclose%
\pgfusepath{fill}%
\end{pgfscope}%
\begin{pgfscope}%
\pgfpathrectangle{\pgfqpoint{1.254980in}{0.150000in}}{\pgfqpoint{5.490039in}{5.490039in}}%
\pgfusepath{clip}%
\pgfsetbuttcap%
\pgfsetroundjoin%
\definecolor{currentfill}{rgb}{0.278826,0.175490,0.483397}%
\pgfsetfillcolor{currentfill}%
\pgfsetfillopacity{0.700000}%
\pgfsetlinewidth{0.000000pt}%
\definecolor{currentstroke}{rgb}{0.000000,0.000000,0.000000}%
\pgfsetstrokecolor{currentstroke}%
\pgfsetdash{}{0pt}%
\pgfpathmoveto{\pgfqpoint{4.414631in}{2.509975in}}%
\pgfpathlineto{\pgfqpoint{4.427707in}{2.506188in}}%
\pgfpathlineto{\pgfqpoint{4.440789in}{2.502527in}}%
\pgfpathlineto{\pgfqpoint{4.453879in}{2.498993in}}%
\pgfpathlineto{\pgfqpoint{4.466975in}{2.495585in}}%
\pgfpathlineto{\pgfqpoint{4.474361in}{2.505733in}}%
\pgfpathlineto{\pgfqpoint{4.481743in}{2.515900in}}%
\pgfpathlineto{\pgfqpoint{4.489121in}{2.526086in}}%
\pgfpathlineto{\pgfqpoint{4.496494in}{2.536291in}}%
\pgfpathlineto{\pgfqpoint{4.483408in}{2.539738in}}%
\pgfpathlineto{\pgfqpoint{4.470328in}{2.543310in}}%
\pgfpathlineto{\pgfqpoint{4.457256in}{2.547009in}}%
\pgfpathlineto{\pgfqpoint{4.444190in}{2.550834in}}%
\pgfpathlineto{\pgfqpoint{4.436807in}{2.540585in}}%
\pgfpathlineto{\pgfqpoint{4.429419in}{2.530359in}}%
\pgfpathlineto{\pgfqpoint{4.422028in}{2.520156in}}%
\pgfpathlineto{\pgfqpoint{4.414631in}{2.509975in}}%
\pgfpathclose%
\pgfusepath{fill}%
\end{pgfscope}%
\begin{pgfscope}%
\pgfpathrectangle{\pgfqpoint{1.254980in}{0.150000in}}{\pgfqpoint{5.490039in}{5.490039in}}%
\pgfusepath{clip}%
\pgfsetbuttcap%
\pgfsetroundjoin%
\definecolor{currentfill}{rgb}{0.233603,0.313828,0.543914}%
\pgfsetfillcolor{currentfill}%
\pgfsetfillopacity{0.700000}%
\pgfsetlinewidth{0.000000pt}%
\definecolor{currentstroke}{rgb}{0.000000,0.000000,0.000000}%
\pgfsetstrokecolor{currentstroke}%
\pgfsetdash{}{0pt}%
\pgfpathmoveto{\pgfqpoint{3.320303in}{2.826560in}}%
\pgfpathlineto{\pgfqpoint{3.333291in}{2.812702in}}%
\pgfpathlineto{\pgfqpoint{3.346276in}{2.799014in}}%
\pgfpathlineto{\pgfqpoint{3.359260in}{2.785495in}}%
\pgfpathlineto{\pgfqpoint{3.372242in}{2.772145in}}%
\pgfpathlineto{\pgfqpoint{3.379989in}{2.780549in}}%
\pgfpathlineto{\pgfqpoint{3.387730in}{2.789035in}}%
\pgfpathlineto{\pgfqpoint{3.395464in}{2.797602in}}%
\pgfpathlineto{\pgfqpoint{3.403192in}{2.806251in}}%
\pgfpathlineto{\pgfqpoint{3.390228in}{2.819525in}}%
\pgfpathlineto{\pgfqpoint{3.377262in}{2.832967in}}%
\pgfpathlineto{\pgfqpoint{3.364294in}{2.846578in}}%
\pgfpathlineto{\pgfqpoint{3.351325in}{2.860359in}}%
\pgfpathlineto{\pgfqpoint{3.343579in}{2.851781in}}%
\pgfpathlineto{\pgfqpoint{3.335827in}{2.843288in}}%
\pgfpathlineto{\pgfqpoint{3.328069in}{2.834881in}}%
\pgfpathlineto{\pgfqpoint{3.320303in}{2.826560in}}%
\pgfpathclose%
\pgfusepath{fill}%
\end{pgfscope}%
\begin{pgfscope}%
\pgfpathrectangle{\pgfqpoint{1.254980in}{0.150000in}}{\pgfqpoint{5.490039in}{5.490039in}}%
\pgfusepath{clip}%
\pgfsetbuttcap%
\pgfsetroundjoin%
\definecolor{currentfill}{rgb}{0.175841,0.441290,0.557685}%
\pgfsetfillcolor{currentfill}%
\pgfsetfillopacity{0.700000}%
\pgfsetlinewidth{0.000000pt}%
\definecolor{currentstroke}{rgb}{0.000000,0.000000,0.000000}%
\pgfsetstrokecolor{currentstroke}%
\pgfsetdash{}{0pt}%
\pgfpathmoveto{\pgfqpoint{3.059945in}{3.141093in}}%
\pgfpathlineto{\pgfqpoint{3.073001in}{3.123605in}}%
\pgfpathlineto{\pgfqpoint{3.086051in}{3.106309in}}%
\pgfpathlineto{\pgfqpoint{3.099097in}{3.089205in}}%
\pgfpathlineto{\pgfqpoint{3.112138in}{3.072291in}}%
\pgfpathlineto{\pgfqpoint{3.119981in}{3.080274in}}%
\pgfpathlineto{\pgfqpoint{3.127816in}{3.088359in}}%
\pgfpathlineto{\pgfqpoint{3.135644in}{3.096546in}}%
\pgfpathlineto{\pgfqpoint{3.143464in}{3.104835in}}%
\pgfpathlineto{\pgfqpoint{3.130443in}{3.121669in}}%
\pgfpathlineto{\pgfqpoint{3.117418in}{3.138693in}}%
\pgfpathlineto{\pgfqpoint{3.104388in}{3.155908in}}%
\pgfpathlineto{\pgfqpoint{3.091353in}{3.173316in}}%
\pgfpathlineto{\pgfqpoint{3.083512in}{3.165101in}}%
\pgfpathlineto{\pgfqpoint{3.075664in}{3.156992in}}%
\pgfpathlineto{\pgfqpoint{3.067809in}{3.148989in}}%
\pgfpathlineto{\pgfqpoint{3.059945in}{3.141093in}}%
\pgfpathclose%
\pgfusepath{fill}%
\end{pgfscope}%
\begin{pgfscope}%
\pgfpathrectangle{\pgfqpoint{1.254980in}{0.150000in}}{\pgfqpoint{5.490039in}{5.490039in}}%
\pgfusepath{clip}%
\pgfsetbuttcap%
\pgfsetroundjoin%
\definecolor{currentfill}{rgb}{0.270595,0.214069,0.507052}%
\pgfsetfillcolor{currentfill}%
\pgfsetfillopacity{0.700000}%
\pgfsetlinewidth{0.000000pt}%
\definecolor{currentstroke}{rgb}{0.000000,0.000000,0.000000}%
\pgfsetstrokecolor{currentstroke}%
\pgfsetdash{}{0pt}%
\pgfpathmoveto{\pgfqpoint{4.712697in}{2.585580in}}%
\pgfpathlineto{\pgfqpoint{4.725862in}{2.583615in}}%
\pgfpathlineto{\pgfqpoint{4.739036in}{2.581771in}}%
\pgfpathlineto{\pgfqpoint{4.752218in}{2.580048in}}%
\pgfpathlineto{\pgfqpoint{4.765409in}{2.578446in}}%
\pgfpathlineto{\pgfqpoint{4.772703in}{2.588545in}}%
\pgfpathlineto{\pgfqpoint{4.779994in}{2.598656in}}%
\pgfpathlineto{\pgfqpoint{4.787280in}{2.608781in}}%
\pgfpathlineto{\pgfqpoint{4.794562in}{2.618921in}}%
\pgfpathlineto{\pgfqpoint{4.781381in}{2.620610in}}%
\pgfpathlineto{\pgfqpoint{4.768209in}{2.622419in}}%
\pgfpathlineto{\pgfqpoint{4.755045in}{2.624348in}}%
\pgfpathlineto{\pgfqpoint{4.741890in}{2.626399in}}%
\pgfpathlineto{\pgfqpoint{4.734598in}{2.616168in}}%
\pgfpathlineto{\pgfqpoint{4.727302in}{2.605955in}}%
\pgfpathlineto{\pgfqpoint{4.720001in}{2.595759in}}%
\pgfpathlineto{\pgfqpoint{4.712697in}{2.585580in}}%
\pgfpathclose%
\pgfusepath{fill}%
\end{pgfscope}%
\begin{pgfscope}%
\pgfpathrectangle{\pgfqpoint{1.254980in}{0.150000in}}{\pgfqpoint{5.490039in}{5.490039in}}%
\pgfusepath{clip}%
\pgfsetbuttcap%
\pgfsetroundjoin%
\definecolor{currentfill}{rgb}{0.243113,0.292092,0.538516}%
\pgfsetfillcolor{currentfill}%
\pgfsetfillopacity{0.700000}%
\pgfsetlinewidth{0.000000pt}%
\definecolor{currentstroke}{rgb}{0.000000,0.000000,0.000000}%
\pgfsetstrokecolor{currentstroke}%
\pgfsetdash{}{0pt}%
\pgfpathmoveto{\pgfqpoint{3.372242in}{2.772145in}}%
\pgfpathlineto{\pgfqpoint{3.385222in}{2.758962in}}%
\pgfpathlineto{\pgfqpoint{3.398202in}{2.745946in}}%
\pgfpathlineto{\pgfqpoint{3.411180in}{2.733095in}}%
\pgfpathlineto{\pgfqpoint{3.424157in}{2.720409in}}%
\pgfpathlineto{\pgfqpoint{3.431887in}{2.728895in}}%
\pgfpathlineto{\pgfqpoint{3.439611in}{2.737459in}}%
\pgfpathlineto{\pgfqpoint{3.447328in}{2.746101in}}%
\pgfpathlineto{\pgfqpoint{3.455039in}{2.754820in}}%
\pgfpathlineto{\pgfqpoint{3.442079in}{2.767430in}}%
\pgfpathlineto{\pgfqpoint{3.429118in}{2.780205in}}%
\pgfpathlineto{\pgfqpoint{3.416156in}{2.793145in}}%
\pgfpathlineto{\pgfqpoint{3.403192in}{2.806251in}}%
\pgfpathlineto{\pgfqpoint{3.395464in}{2.797602in}}%
\pgfpathlineto{\pgfqpoint{3.387730in}{2.789035in}}%
\pgfpathlineto{\pgfqpoint{3.379989in}{2.780549in}}%
\pgfpathlineto{\pgfqpoint{3.372242in}{2.772145in}}%
\pgfpathclose%
\pgfusepath{fill}%
\end{pgfscope}%
\begin{pgfscope}%
\pgfpathrectangle{\pgfqpoint{1.254980in}{0.150000in}}{\pgfqpoint{5.490039in}{5.490039in}}%
\pgfusepath{clip}%
\pgfsetbuttcap%
\pgfsetroundjoin%
\definecolor{currentfill}{rgb}{0.274128,0.199721,0.498911}%
\pgfsetfillcolor{currentfill}%
\pgfsetfillopacity{0.700000}%
\pgfsetlinewidth{0.000000pt}%
\definecolor{currentstroke}{rgb}{0.000000,0.000000,0.000000}%
\pgfsetstrokecolor{currentstroke}%
\pgfsetdash{}{0pt}%
\pgfpathmoveto{\pgfqpoint{3.662393in}{2.574742in}}%
\pgfpathlineto{\pgfqpoint{3.675358in}{2.564801in}}%
\pgfpathlineto{\pgfqpoint{3.688325in}{2.555011in}}%
\pgfpathlineto{\pgfqpoint{3.701293in}{2.545370in}}%
\pgfpathlineto{\pgfqpoint{3.714263in}{2.535878in}}%
\pgfpathlineto{\pgfqpoint{3.721892in}{2.544999in}}%
\pgfpathlineto{\pgfqpoint{3.729514in}{2.554176in}}%
\pgfpathlineto{\pgfqpoint{3.737132in}{2.563410in}}%
\pgfpathlineto{\pgfqpoint{3.744744in}{2.572700in}}%
\pgfpathlineto{\pgfqpoint{3.731788in}{2.582134in}}%
\pgfpathlineto{\pgfqpoint{3.718834in}{2.591716in}}%
\pgfpathlineto{\pgfqpoint{3.705881in}{2.601449in}}%
\pgfpathlineto{\pgfqpoint{3.692930in}{2.611331in}}%
\pgfpathlineto{\pgfqpoint{3.685304in}{2.602093in}}%
\pgfpathlineto{\pgfqpoint{3.677673in}{2.592915in}}%
\pgfpathlineto{\pgfqpoint{3.670035in}{2.583798in}}%
\pgfpathlineto{\pgfqpoint{3.662393in}{2.574742in}}%
\pgfpathclose%
\pgfusepath{fill}%
\end{pgfscope}%
\begin{pgfscope}%
\pgfpathrectangle{\pgfqpoint{1.254980in}{0.150000in}}{\pgfqpoint{5.490039in}{5.490039in}}%
\pgfusepath{clip}%
\pgfsetbuttcap%
\pgfsetroundjoin%
\definecolor{currentfill}{rgb}{0.163625,0.471133,0.558148}%
\pgfsetfillcolor{currentfill}%
\pgfsetfillopacity{0.700000}%
\pgfsetlinewidth{0.000000pt}%
\definecolor{currentstroke}{rgb}{0.000000,0.000000,0.000000}%
\pgfsetstrokecolor{currentstroke}%
\pgfsetdash{}{0pt}%
\pgfpathmoveto{\pgfqpoint{3.007671in}{3.212996in}}%
\pgfpathlineto{\pgfqpoint{3.020748in}{3.194725in}}%
\pgfpathlineto{\pgfqpoint{3.033819in}{3.176651in}}%
\pgfpathlineto{\pgfqpoint{3.046885in}{3.158775in}}%
\pgfpathlineto{\pgfqpoint{3.059945in}{3.141093in}}%
\pgfpathlineto{\pgfqpoint{3.067809in}{3.148989in}}%
\pgfpathlineto{\pgfqpoint{3.075664in}{3.156992in}}%
\pgfpathlineto{\pgfqpoint{3.083512in}{3.165101in}}%
\pgfpathlineto{\pgfqpoint{3.091353in}{3.173316in}}%
\pgfpathlineto{\pgfqpoint{3.078313in}{3.190917in}}%
\pgfpathlineto{\pgfqpoint{3.065269in}{3.208713in}}%
\pgfpathlineto{\pgfqpoint{3.052219in}{3.226705in}}%
\pgfpathlineto{\pgfqpoint{3.039163in}{3.244894in}}%
\pgfpathlineto{\pgfqpoint{3.031302in}{3.236754in}}%
\pgfpathlineto{\pgfqpoint{3.023433in}{3.228724in}}%
\pgfpathlineto{\pgfqpoint{3.015556in}{3.220805in}}%
\pgfpathlineto{\pgfqpoint{3.007671in}{3.212996in}}%
\pgfpathclose%
\pgfusepath{fill}%
\end{pgfscope}%
\begin{pgfscope}%
\pgfpathrectangle{\pgfqpoint{1.254980in}{0.150000in}}{\pgfqpoint{5.490039in}{5.490039in}}%
\pgfusepath{clip}%
\pgfsetbuttcap%
\pgfsetroundjoin%
\definecolor{currentfill}{rgb}{0.252194,0.269783,0.531579}%
\pgfsetfillcolor{currentfill}%
\pgfsetfillopacity{0.700000}%
\pgfsetlinewidth{0.000000pt}%
\definecolor{currentstroke}{rgb}{0.000000,0.000000,0.000000}%
\pgfsetstrokecolor{currentstroke}%
\pgfsetdash{}{0pt}%
\pgfpathmoveto{\pgfqpoint{3.424157in}{2.720409in}}%
\pgfpathlineto{\pgfqpoint{3.437134in}{2.707887in}}%
\pgfpathlineto{\pgfqpoint{3.450109in}{2.695527in}}%
\pgfpathlineto{\pgfqpoint{3.463084in}{2.683330in}}%
\pgfpathlineto{\pgfqpoint{3.476059in}{2.671293in}}%
\pgfpathlineto{\pgfqpoint{3.483772in}{2.679861in}}%
\pgfpathlineto{\pgfqpoint{3.491479in}{2.688503in}}%
\pgfpathlineto{\pgfqpoint{3.499180in}{2.697219in}}%
\pgfpathlineto{\pgfqpoint{3.506874in}{2.706007in}}%
\pgfpathlineto{\pgfqpoint{3.493916in}{2.717968in}}%
\pgfpathlineto{\pgfqpoint{3.480958in}{2.730090in}}%
\pgfpathlineto{\pgfqpoint{3.467999in}{2.742374in}}%
\pgfpathlineto{\pgfqpoint{3.455039in}{2.754820in}}%
\pgfpathlineto{\pgfqpoint{3.447328in}{2.746101in}}%
\pgfpathlineto{\pgfqpoint{3.439611in}{2.737459in}}%
\pgfpathlineto{\pgfqpoint{3.431887in}{2.728895in}}%
\pgfpathlineto{\pgfqpoint{3.424157in}{2.720409in}}%
\pgfpathclose%
\pgfusepath{fill}%
\end{pgfscope}%
\begin{pgfscope}%
\pgfpathrectangle{\pgfqpoint{1.254980in}{0.150000in}}{\pgfqpoint{5.490039in}{5.490039in}}%
\pgfusepath{clip}%
\pgfsetbuttcap%
\pgfsetroundjoin%
\definecolor{currentfill}{rgb}{0.281412,0.155834,0.469201}%
\pgfsetfillcolor{currentfill}%
\pgfsetfillopacity{0.700000}%
\pgfsetlinewidth{0.000000pt}%
\definecolor{currentstroke}{rgb}{0.000000,0.000000,0.000000}%
\pgfsetstrokecolor{currentstroke}%
\pgfsetdash{}{0pt}%
\pgfpathmoveto{\pgfqpoint{3.982567in}{2.479833in}}%
\pgfpathlineto{\pgfqpoint{3.995559in}{2.472870in}}%
\pgfpathlineto{\pgfqpoint{4.008555in}{2.466045in}}%
\pgfpathlineto{\pgfqpoint{4.021555in}{2.459358in}}%
\pgfpathlineto{\pgfqpoint{4.034559in}{2.452806in}}%
\pgfpathlineto{\pgfqpoint{4.042082in}{2.462535in}}%
\pgfpathlineto{\pgfqpoint{4.049601in}{2.472301in}}%
\pgfpathlineto{\pgfqpoint{4.057114in}{2.482103in}}%
\pgfpathlineto{\pgfqpoint{4.064623in}{2.491942in}}%
\pgfpathlineto{\pgfqpoint{4.051630in}{2.498468in}}%
\pgfpathlineto{\pgfqpoint{4.038642in}{2.505131in}}%
\pgfpathlineto{\pgfqpoint{4.025658in}{2.511930in}}%
\pgfpathlineto{\pgfqpoint{4.012678in}{2.518868in}}%
\pgfpathlineto{\pgfqpoint{4.005157in}{2.509048in}}%
\pgfpathlineto{\pgfqpoint{3.997632in}{2.499269in}}%
\pgfpathlineto{\pgfqpoint{3.990102in}{2.489531in}}%
\pgfpathlineto{\pgfqpoint{3.982567in}{2.479833in}}%
\pgfpathclose%
\pgfusepath{fill}%
\end{pgfscope}%
\begin{pgfscope}%
\pgfpathrectangle{\pgfqpoint{1.254980in}{0.150000in}}{\pgfqpoint{5.490039in}{5.490039in}}%
\pgfusepath{clip}%
\pgfsetbuttcap%
\pgfsetroundjoin%
\definecolor{currentfill}{rgb}{0.280255,0.165693,0.476498}%
\pgfsetfillcolor{currentfill}%
\pgfsetfillopacity{0.700000}%
\pgfsetlinewidth{0.000000pt}%
\definecolor{currentstroke}{rgb}{0.000000,0.000000,0.000000}%
\pgfsetstrokecolor{currentstroke}%
\pgfsetdash{}{0pt}%
\pgfpathmoveto{\pgfqpoint{4.332723in}{2.485796in}}%
\pgfpathlineto{\pgfqpoint{4.345783in}{2.481521in}}%
\pgfpathlineto{\pgfqpoint{4.358849in}{2.477374in}}%
\pgfpathlineto{\pgfqpoint{4.371922in}{2.473355in}}%
\pgfpathlineto{\pgfqpoint{4.385001in}{2.469463in}}%
\pgfpathlineto{\pgfqpoint{4.392416in}{2.479560in}}%
\pgfpathlineto{\pgfqpoint{4.399825in}{2.489678in}}%
\pgfpathlineto{\pgfqpoint{4.407230in}{2.499816in}}%
\pgfpathlineto{\pgfqpoint{4.414631in}{2.509975in}}%
\pgfpathlineto{\pgfqpoint{4.401562in}{2.513889in}}%
\pgfpathlineto{\pgfqpoint{4.388499in}{2.517931in}}%
\pgfpathlineto{\pgfqpoint{4.375443in}{2.522101in}}%
\pgfpathlineto{\pgfqpoint{4.362393in}{2.526399in}}%
\pgfpathlineto{\pgfqpoint{4.354982in}{2.516211in}}%
\pgfpathlineto{\pgfqpoint{4.347567in}{2.506049in}}%
\pgfpathlineto{\pgfqpoint{4.340147in}{2.495910in}}%
\pgfpathlineto{\pgfqpoint{4.332723in}{2.485796in}}%
\pgfpathclose%
\pgfusepath{fill}%
\end{pgfscope}%
\begin{pgfscope}%
\pgfpathrectangle{\pgfqpoint{1.254980in}{0.150000in}}{\pgfqpoint{5.490039in}{5.490039in}}%
\pgfusepath{clip}%
\pgfsetbuttcap%
\pgfsetroundjoin%
\definecolor{currentfill}{rgb}{0.274128,0.199721,0.498911}%
\pgfsetfillcolor{currentfill}%
\pgfsetfillopacity{0.700000}%
\pgfsetlinewidth{0.000000pt}%
\definecolor{currentstroke}{rgb}{0.000000,0.000000,0.000000}%
\pgfsetstrokecolor{currentstroke}%
\pgfsetdash{}{0pt}%
\pgfpathmoveto{\pgfqpoint{4.630815in}{2.553809in}}%
\pgfpathlineto{\pgfqpoint{4.643958in}{2.551425in}}%
\pgfpathlineto{\pgfqpoint{4.657109in}{2.549165in}}%
\pgfpathlineto{\pgfqpoint{4.670268in}{2.547026in}}%
\pgfpathlineto{\pgfqpoint{4.683435in}{2.545010in}}%
\pgfpathlineto{\pgfqpoint{4.690757in}{2.555132in}}%
\pgfpathlineto{\pgfqpoint{4.698074in}{2.565268in}}%
\pgfpathlineto{\pgfqpoint{4.705388in}{2.575417in}}%
\pgfpathlineto{\pgfqpoint{4.712697in}{2.585580in}}%
\pgfpathlineto{\pgfqpoint{4.699539in}{2.587667in}}%
\pgfpathlineto{\pgfqpoint{4.686390in}{2.589876in}}%
\pgfpathlineto{\pgfqpoint{4.673249in}{2.592206in}}%
\pgfpathlineto{\pgfqpoint{4.660116in}{2.594660in}}%
\pgfpathlineto{\pgfqpoint{4.652797in}{2.584420in}}%
\pgfpathlineto{\pgfqpoint{4.645474in}{2.574199in}}%
\pgfpathlineto{\pgfqpoint{4.638146in}{2.563996in}}%
\pgfpathlineto{\pgfqpoint{4.630815in}{2.553809in}}%
\pgfpathclose%
\pgfusepath{fill}%
\end{pgfscope}%
\begin{pgfscope}%
\pgfpathrectangle{\pgfqpoint{1.254980in}{0.150000in}}{\pgfqpoint{5.490039in}{5.490039in}}%
\pgfusepath{clip}%
\pgfsetbuttcap%
\pgfsetroundjoin%
\definecolor{currentfill}{rgb}{0.279574,0.170599,0.479997}%
\pgfsetfillcolor{currentfill}%
\pgfsetfillopacity{0.700000}%
\pgfsetlinewidth{0.000000pt}%
\definecolor{currentstroke}{rgb}{0.000000,0.000000,0.000000}%
\pgfsetstrokecolor{currentstroke}%
\pgfsetdash{}{0pt}%
\pgfpathmoveto{\pgfqpoint{3.848470in}{2.502499in}}%
\pgfpathlineto{\pgfqpoint{3.861447in}{2.494374in}}%
\pgfpathlineto{\pgfqpoint{3.874428in}{2.486390in}}%
\pgfpathlineto{\pgfqpoint{3.887411in}{2.478549in}}%
\pgfpathlineto{\pgfqpoint{3.900397in}{2.470848in}}%
\pgfpathlineto{\pgfqpoint{3.907965in}{2.480337in}}%
\pgfpathlineto{\pgfqpoint{3.915527in}{2.489871in}}%
\pgfpathlineto{\pgfqpoint{3.923084in}{2.499449in}}%
\pgfpathlineto{\pgfqpoint{3.930636in}{2.509072in}}%
\pgfpathlineto{\pgfqpoint{3.917662in}{2.516731in}}%
\pgfpathlineto{\pgfqpoint{3.904692in}{2.524531in}}%
\pgfpathlineto{\pgfqpoint{3.891724in}{2.532473in}}%
\pgfpathlineto{\pgfqpoint{3.878759in}{2.540557in}}%
\pgfpathlineto{\pgfqpoint{3.871195in}{2.530970in}}%
\pgfpathlineto{\pgfqpoint{3.863625in}{2.521432in}}%
\pgfpathlineto{\pgfqpoint{3.856050in}{2.511941in}}%
\pgfpathlineto{\pgfqpoint{3.848470in}{2.502499in}}%
\pgfpathclose%
\pgfusepath{fill}%
\end{pgfscope}%
\begin{pgfscope}%
\pgfpathrectangle{\pgfqpoint{1.254980in}{0.150000in}}{\pgfqpoint{5.490039in}{5.490039in}}%
\pgfusepath{clip}%
\pgfsetbuttcap%
\pgfsetroundjoin%
\definecolor{currentfill}{rgb}{0.204903,0.375746,0.553533}%
\pgfsetfillcolor{currentfill}%
\pgfsetfillopacity{0.700000}%
\pgfsetlinewidth{0.000000pt}%
\definecolor{currentstroke}{rgb}{0.000000,0.000000,0.000000}%
\pgfsetstrokecolor{currentstroke}%
\pgfsetdash{}{0pt}%
\pgfpathmoveto{\pgfqpoint{5.503118in}{2.931594in}}%
\pgfpathlineto{\pgfqpoint{5.516578in}{2.932931in}}%
\pgfpathlineto{\pgfqpoint{5.530051in}{2.934380in}}%
\pgfpathlineto{\pgfqpoint{5.543535in}{2.935942in}}%
\pgfpathlineto{\pgfqpoint{5.557031in}{2.937615in}}%
\pgfpathlineto{\pgfqpoint{5.564060in}{2.946854in}}%
\pgfpathlineto{\pgfqpoint{5.571086in}{2.956128in}}%
\pgfpathlineto{\pgfqpoint{5.578108in}{2.965437in}}%
\pgfpathlineto{\pgfqpoint{5.585127in}{2.974785in}}%
\pgfpathlineto{\pgfqpoint{5.571646in}{2.973342in}}%
\pgfpathlineto{\pgfqpoint{5.558177in}{2.972012in}}%
\pgfpathlineto{\pgfqpoint{5.544720in}{2.970793in}}%
\pgfpathlineto{\pgfqpoint{5.531275in}{2.969685in}}%
\pgfpathlineto{\pgfqpoint{5.524240in}{2.960102in}}%
\pgfpathlineto{\pgfqpoint{5.517203in}{2.950560in}}%
\pgfpathlineto{\pgfqpoint{5.510162in}{2.941058in}}%
\pgfpathlineto{\pgfqpoint{5.503118in}{2.931594in}}%
\pgfpathclose%
\pgfusepath{fill}%
\end{pgfscope}%
\begin{pgfscope}%
\pgfpathrectangle{\pgfqpoint{1.254980in}{0.150000in}}{\pgfqpoint{5.490039in}{5.490039in}}%
\pgfusepath{clip}%
\pgfsetbuttcap%
\pgfsetroundjoin%
\definecolor{currentfill}{rgb}{0.212395,0.359683,0.551710}%
\pgfsetfillcolor{currentfill}%
\pgfsetfillopacity{0.700000}%
\pgfsetlinewidth{0.000000pt}%
\definecolor{currentstroke}{rgb}{0.000000,0.000000,0.000000}%
\pgfsetstrokecolor{currentstroke}%
\pgfsetdash{}{0pt}%
\pgfpathmoveto{\pgfqpoint{5.421121in}{2.888981in}}%
\pgfpathlineto{\pgfqpoint{5.434550in}{2.890084in}}%
\pgfpathlineto{\pgfqpoint{5.447991in}{2.891298in}}%
\pgfpathlineto{\pgfqpoint{5.461443in}{2.892626in}}%
\pgfpathlineto{\pgfqpoint{5.474906in}{2.894066in}}%
\pgfpathlineto{\pgfqpoint{5.481965in}{2.903403in}}%
\pgfpathlineto{\pgfqpoint{5.489019in}{2.912768in}}%
\pgfpathlineto{\pgfqpoint{5.496070in}{2.922164in}}%
\pgfpathlineto{\pgfqpoint{5.503118in}{2.931594in}}%
\pgfpathlineto{\pgfqpoint{5.489669in}{2.930368in}}%
\pgfpathlineto{\pgfqpoint{5.476231in}{2.929255in}}%
\pgfpathlineto{\pgfqpoint{5.462805in}{2.928255in}}%
\pgfpathlineto{\pgfqpoint{5.449391in}{2.927366in}}%
\pgfpathlineto{\pgfqpoint{5.442328in}{2.917717in}}%
\pgfpathlineto{\pgfqpoint{5.435263in}{2.908105in}}%
\pgfpathlineto{\pgfqpoint{5.428194in}{2.898527in}}%
\pgfpathlineto{\pgfqpoint{5.421121in}{2.888981in}}%
\pgfpathclose%
\pgfusepath{fill}%
\end{pgfscope}%
\begin{pgfscope}%
\pgfpathrectangle{\pgfqpoint{1.254980in}{0.150000in}}{\pgfqpoint{5.490039in}{5.490039in}}%
\pgfusepath{clip}%
\pgfsetbuttcap%
\pgfsetroundjoin%
\definecolor{currentfill}{rgb}{0.281887,0.150881,0.465405}%
\pgfsetfillcolor{currentfill}%
\pgfsetfillopacity{0.700000}%
\pgfsetlinewidth{0.000000pt}%
\definecolor{currentstroke}{rgb}{0.000000,0.000000,0.000000}%
\pgfsetstrokecolor{currentstroke}%
\pgfsetdash{}{0pt}%
\pgfpathmoveto{\pgfqpoint{4.116637in}{2.467193in}}%
\pgfpathlineto{\pgfqpoint{4.129653in}{2.461342in}}%
\pgfpathlineto{\pgfqpoint{4.142673in}{2.455625in}}%
\pgfpathlineto{\pgfqpoint{4.155698in}{2.450041in}}%
\pgfpathlineto{\pgfqpoint{4.168728in}{2.444590in}}%
\pgfpathlineto{\pgfqpoint{4.176210in}{2.454496in}}%
\pgfpathlineto{\pgfqpoint{4.183687in}{2.464431in}}%
\pgfpathlineto{\pgfqpoint{4.191160in}{2.474395in}}%
\pgfpathlineto{\pgfqpoint{4.198628in}{2.484389in}}%
\pgfpathlineto{\pgfqpoint{4.185608in}{2.489831in}}%
\pgfpathlineto{\pgfqpoint{4.172594in}{2.495406in}}%
\pgfpathlineto{\pgfqpoint{4.159585in}{2.501114in}}%
\pgfpathlineto{\pgfqpoint{4.146581in}{2.506956in}}%
\pgfpathlineto{\pgfqpoint{4.139102in}{2.496965in}}%
\pgfpathlineto{\pgfqpoint{4.131618in}{2.487008in}}%
\pgfpathlineto{\pgfqpoint{4.124130in}{2.477084in}}%
\pgfpathlineto{\pgfqpoint{4.116637in}{2.467193in}}%
\pgfpathclose%
\pgfusepath{fill}%
\end{pgfscope}%
\begin{pgfscope}%
\pgfpathrectangle{\pgfqpoint{1.254980in}{0.150000in}}{\pgfqpoint{5.490039in}{5.490039in}}%
\pgfusepath{clip}%
\pgfsetbuttcap%
\pgfsetroundjoin%
\definecolor{currentfill}{rgb}{0.221989,0.339161,0.548752}%
\pgfsetfillcolor{currentfill}%
\pgfsetfillopacity{0.700000}%
\pgfsetlinewidth{0.000000pt}%
\definecolor{currentstroke}{rgb}{0.000000,0.000000,0.000000}%
\pgfsetstrokecolor{currentstroke}%
\pgfsetdash{}{0pt}%
\pgfpathmoveto{\pgfqpoint{5.339137in}{2.847011in}}%
\pgfpathlineto{\pgfqpoint{5.352535in}{2.847858in}}%
\pgfpathlineto{\pgfqpoint{5.365943in}{2.848820in}}%
\pgfpathlineto{\pgfqpoint{5.379364in}{2.849894in}}%
\pgfpathlineto{\pgfqpoint{5.392795in}{2.851082in}}%
\pgfpathlineto{\pgfqpoint{5.399882in}{2.860519in}}%
\pgfpathlineto{\pgfqpoint{5.406966in}{2.869979in}}%
\pgfpathlineto{\pgfqpoint{5.414045in}{2.879466in}}%
\pgfpathlineto{\pgfqpoint{5.421121in}{2.888981in}}%
\pgfpathlineto{\pgfqpoint{5.407704in}{2.887992in}}%
\pgfpathlineto{\pgfqpoint{5.394297in}{2.887116in}}%
\pgfpathlineto{\pgfqpoint{5.380902in}{2.886353in}}%
\pgfpathlineto{\pgfqpoint{5.367518in}{2.885703in}}%
\pgfpathlineto{\pgfqpoint{5.360428in}{2.875984in}}%
\pgfpathlineto{\pgfqpoint{5.353335in}{2.866297in}}%
\pgfpathlineto{\pgfqpoint{5.346238in}{2.856640in}}%
\pgfpathlineto{\pgfqpoint{5.339137in}{2.847011in}}%
\pgfpathclose%
\pgfusepath{fill}%
\end{pgfscope}%
\begin{pgfscope}%
\pgfpathrectangle{\pgfqpoint{1.254980in}{0.150000in}}{\pgfqpoint{5.490039in}{5.490039in}}%
\pgfusepath{clip}%
\pgfsetbuttcap%
\pgfsetroundjoin%
\definecolor{currentfill}{rgb}{0.229739,0.322361,0.545706}%
\pgfsetfillcolor{currentfill}%
\pgfsetfillopacity{0.700000}%
\pgfsetlinewidth{0.000000pt}%
\definecolor{currentstroke}{rgb}{0.000000,0.000000,0.000000}%
\pgfsetstrokecolor{currentstroke}%
\pgfsetdash{}{0pt}%
\pgfpathmoveto{\pgfqpoint{5.257162in}{2.805753in}}%
\pgfpathlineto{\pgfqpoint{5.270529in}{2.806327in}}%
\pgfpathlineto{\pgfqpoint{5.283907in}{2.807016in}}%
\pgfpathlineto{\pgfqpoint{5.297296in}{2.807818in}}%
\pgfpathlineto{\pgfqpoint{5.310696in}{2.808735in}}%
\pgfpathlineto{\pgfqpoint{5.317812in}{2.818271in}}%
\pgfpathlineto{\pgfqpoint{5.324924in}{2.827828in}}%
\pgfpathlineto{\pgfqpoint{5.332032in}{2.837407in}}%
\pgfpathlineto{\pgfqpoint{5.339137in}{2.847011in}}%
\pgfpathlineto{\pgfqpoint{5.325750in}{2.846276in}}%
\pgfpathlineto{\pgfqpoint{5.312374in}{2.845656in}}%
\pgfpathlineto{\pgfqpoint{5.299009in}{2.845150in}}%
\pgfpathlineto{\pgfqpoint{5.285655in}{2.844758in}}%
\pgfpathlineto{\pgfqpoint{5.278538in}{2.834967in}}%
\pgfpathlineto{\pgfqpoint{5.271416in}{2.825204in}}%
\pgfpathlineto{\pgfqpoint{5.264291in}{2.815466in}}%
\pgfpathlineto{\pgfqpoint{5.257162in}{2.805753in}}%
\pgfpathclose%
\pgfusepath{fill}%
\end{pgfscope}%
\begin{pgfscope}%
\pgfpathrectangle{\pgfqpoint{1.254980in}{0.150000in}}{\pgfqpoint{5.490039in}{5.490039in}}%
\pgfusepath{clip}%
\pgfsetbuttcap%
\pgfsetroundjoin%
\definecolor{currentfill}{rgb}{0.260571,0.246922,0.522828}%
\pgfsetfillcolor{currentfill}%
\pgfsetfillopacity{0.700000}%
\pgfsetlinewidth{0.000000pt}%
\definecolor{currentstroke}{rgb}{0.000000,0.000000,0.000000}%
\pgfsetstrokecolor{currentstroke}%
\pgfsetdash{}{0pt}%
\pgfpathmoveto{\pgfqpoint{3.476059in}{2.671293in}}%
\pgfpathlineto{\pgfqpoint{3.489033in}{2.659417in}}%
\pgfpathlineto{\pgfqpoint{3.502007in}{2.647701in}}%
\pgfpathlineto{\pgfqpoint{3.514981in}{2.636143in}}%
\pgfpathlineto{\pgfqpoint{3.527955in}{2.624743in}}%
\pgfpathlineto{\pgfqpoint{3.535652in}{2.633392in}}%
\pgfpathlineto{\pgfqpoint{3.543343in}{2.642111in}}%
\pgfpathlineto{\pgfqpoint{3.551027in}{2.650900in}}%
\pgfpathlineto{\pgfqpoint{3.558706in}{2.659758in}}%
\pgfpathlineto{\pgfqpoint{3.545748in}{2.671083in}}%
\pgfpathlineto{\pgfqpoint{3.532790in}{2.682566in}}%
\pgfpathlineto{\pgfqpoint{3.519832in}{2.694207in}}%
\pgfpathlineto{\pgfqpoint{3.506874in}{2.706007in}}%
\pgfpathlineto{\pgfqpoint{3.499180in}{2.697219in}}%
\pgfpathlineto{\pgfqpoint{3.491479in}{2.688503in}}%
\pgfpathlineto{\pgfqpoint{3.483772in}{2.679861in}}%
\pgfpathlineto{\pgfqpoint{3.476059in}{2.671293in}}%
\pgfpathclose%
\pgfusepath{fill}%
\end{pgfscope}%
\begin{pgfscope}%
\pgfpathrectangle{\pgfqpoint{1.254980in}{0.150000in}}{\pgfqpoint{5.490039in}{5.490039in}}%
\pgfusepath{clip}%
\pgfsetbuttcap%
\pgfsetroundjoin%
\definecolor{currentfill}{rgb}{0.151918,0.500685,0.557587}%
\pgfsetfillcolor{currentfill}%
\pgfsetfillopacity{0.700000}%
\pgfsetlinewidth{0.000000pt}%
\definecolor{currentstroke}{rgb}{0.000000,0.000000,0.000000}%
\pgfsetstrokecolor{currentstroke}%
\pgfsetdash{}{0pt}%
\pgfpathmoveto{\pgfqpoint{2.955305in}{3.288084in}}%
\pgfpathlineto{\pgfqpoint{2.968405in}{3.269008in}}%
\pgfpathlineto{\pgfqpoint{2.981500in}{3.250136in}}%
\pgfpathlineto{\pgfqpoint{2.994588in}{3.231466in}}%
\pgfpathlineto{\pgfqpoint{3.007671in}{3.212996in}}%
\pgfpathlineto{\pgfqpoint{3.015556in}{3.220805in}}%
\pgfpathlineto{\pgfqpoint{3.023433in}{3.228724in}}%
\pgfpathlineto{\pgfqpoint{3.031302in}{3.236754in}}%
\pgfpathlineto{\pgfqpoint{3.039163in}{3.244894in}}%
\pgfpathlineto{\pgfqpoint{3.026102in}{3.263283in}}%
\pgfpathlineto{\pgfqpoint{3.013035in}{3.281872in}}%
\pgfpathlineto{\pgfqpoint{2.999963in}{3.300662in}}%
\pgfpathlineto{\pgfqpoint{2.986884in}{3.319655in}}%
\pgfpathlineto{\pgfqpoint{2.979001in}{3.311591in}}%
\pgfpathlineto{\pgfqpoint{2.971110in}{3.303641in}}%
\pgfpathlineto{\pgfqpoint{2.963212in}{3.295805in}}%
\pgfpathlineto{\pgfqpoint{2.955305in}{3.288084in}}%
\pgfpathclose%
\pgfusepath{fill}%
\end{pgfscope}%
\begin{pgfscope}%
\pgfpathrectangle{\pgfqpoint{1.254980in}{0.150000in}}{\pgfqpoint{5.490039in}{5.490039in}}%
\pgfusepath{clip}%
\pgfsetbuttcap%
\pgfsetroundjoin%
\definecolor{currentfill}{rgb}{0.197636,0.391528,0.554969}%
\pgfsetfillcolor{currentfill}%
\pgfsetfillopacity{0.700000}%
\pgfsetlinewidth{0.000000pt}%
\definecolor{currentstroke}{rgb}{0.000000,0.000000,0.000000}%
\pgfsetstrokecolor{currentstroke}%
\pgfsetdash{}{0pt}%
\pgfpathmoveto{\pgfqpoint{5.585127in}{2.974785in}}%
\pgfpathlineto{\pgfqpoint{5.598619in}{2.976338in}}%
\pgfpathlineto{\pgfqpoint{5.612124in}{2.978003in}}%
\pgfpathlineto{\pgfqpoint{5.625640in}{2.979779in}}%
\pgfpathlineto{\pgfqpoint{5.639169in}{2.981666in}}%
\pgfpathlineto{\pgfqpoint{5.646169in}{2.990813in}}%
\pgfpathlineto{\pgfqpoint{5.653166in}{3.000000in}}%
\pgfpathlineto{\pgfqpoint{5.660159in}{3.009229in}}%
\pgfpathlineto{\pgfqpoint{5.646643in}{3.007525in}}%
\pgfpathlineto{\pgfqpoint{5.633138in}{3.005933in}}%
\pgfpathlineto{\pgfqpoint{5.619646in}{3.004452in}}%
\pgfpathlineto{\pgfqpoint{5.606165in}{3.003081in}}%
\pgfpathlineto{\pgfqpoint{5.599155in}{2.993604in}}%
\pgfpathlineto{\pgfqpoint{5.592143in}{2.984173in}}%
\pgfpathlineto{\pgfqpoint{5.585127in}{2.974785in}}%
\pgfpathclose%
\pgfusepath{fill}%
\end{pgfscope}%
\begin{pgfscope}%
\pgfpathrectangle{\pgfqpoint{1.254980in}{0.150000in}}{\pgfqpoint{5.490039in}{5.490039in}}%
\pgfusepath{clip}%
\pgfsetbuttcap%
\pgfsetroundjoin%
\definecolor{currentfill}{rgb}{0.237441,0.305202,0.541921}%
\pgfsetfillcolor{currentfill}%
\pgfsetfillopacity{0.700000}%
\pgfsetlinewidth{0.000000pt}%
\definecolor{currentstroke}{rgb}{0.000000,0.000000,0.000000}%
\pgfsetstrokecolor{currentstroke}%
\pgfsetdash{}{0pt}%
\pgfpathmoveto{\pgfqpoint{5.175196in}{2.765292in}}%
\pgfpathlineto{\pgfqpoint{5.188533in}{2.765573in}}%
\pgfpathlineto{\pgfqpoint{5.201881in}{2.765969in}}%
\pgfpathlineto{\pgfqpoint{5.215239in}{2.766480in}}%
\pgfpathlineto{\pgfqpoint{5.228607in}{2.767105in}}%
\pgfpathlineto{\pgfqpoint{5.235752in}{2.776740in}}%
\pgfpathlineto{\pgfqpoint{5.242893in}{2.786392in}}%
\pgfpathlineto{\pgfqpoint{5.250029in}{2.796063in}}%
\pgfpathlineto{\pgfqpoint{5.257162in}{2.805753in}}%
\pgfpathlineto{\pgfqpoint{5.243806in}{2.805294in}}%
\pgfpathlineto{\pgfqpoint{5.230460in}{2.804949in}}%
\pgfpathlineto{\pgfqpoint{5.217125in}{2.804719in}}%
\pgfpathlineto{\pgfqpoint{5.203801in}{2.804604in}}%
\pgfpathlineto{\pgfqpoint{5.196655in}{2.794741in}}%
\pgfpathlineto{\pgfqpoint{5.189506in}{2.784903in}}%
\pgfpathlineto{\pgfqpoint{5.182353in}{2.775087in}}%
\pgfpathlineto{\pgfqpoint{5.175196in}{2.765292in}}%
\pgfpathclose%
\pgfusepath{fill}%
\end{pgfscope}%
\begin{pgfscope}%
\pgfpathrectangle{\pgfqpoint{1.254980in}{0.150000in}}{\pgfqpoint{5.490039in}{5.490039in}}%
\pgfusepath{clip}%
\pgfsetbuttcap%
\pgfsetroundjoin%
\definecolor{currentfill}{rgb}{0.277134,0.185228,0.489898}%
\pgfsetfillcolor{currentfill}%
\pgfsetfillopacity{0.700000}%
\pgfsetlinewidth{0.000000pt}%
\definecolor{currentstroke}{rgb}{0.000000,0.000000,0.000000}%
\pgfsetstrokecolor{currentstroke}%
\pgfsetdash{}{0pt}%
\pgfpathmoveto{\pgfqpoint{4.548910in}{2.523757in}}%
\pgfpathlineto{\pgfqpoint{4.562032in}{2.520934in}}%
\pgfpathlineto{\pgfqpoint{4.575162in}{2.518236in}}%
\pgfpathlineto{\pgfqpoint{4.588299in}{2.515661in}}%
\pgfpathlineto{\pgfqpoint{4.601444in}{2.513209in}}%
\pgfpathlineto{\pgfqpoint{4.608793in}{2.523338in}}%
\pgfpathlineto{\pgfqpoint{4.616138in}{2.533481in}}%
\pgfpathlineto{\pgfqpoint{4.623479in}{2.543637in}}%
\pgfpathlineto{\pgfqpoint{4.630815in}{2.553809in}}%
\pgfpathlineto{\pgfqpoint{4.617679in}{2.556315in}}%
\pgfpathlineto{\pgfqpoint{4.604552in}{2.558944in}}%
\pgfpathlineto{\pgfqpoint{4.591432in}{2.561697in}}%
\pgfpathlineto{\pgfqpoint{4.578320in}{2.564573in}}%
\pgfpathlineto{\pgfqpoint{4.570974in}{2.554342in}}%
\pgfpathlineto{\pgfqpoint{4.563623in}{2.544129in}}%
\pgfpathlineto{\pgfqpoint{4.556269in}{2.533934in}}%
\pgfpathlineto{\pgfqpoint{4.548910in}{2.523757in}}%
\pgfpathclose%
\pgfusepath{fill}%
\end{pgfscope}%
\begin{pgfscope}%
\pgfpathrectangle{\pgfqpoint{1.254980in}{0.150000in}}{\pgfqpoint{5.490039in}{5.490039in}}%
\pgfusepath{clip}%
\pgfsetbuttcap%
\pgfsetroundjoin%
\definecolor{currentfill}{rgb}{0.244972,0.287675,0.537260}%
\pgfsetfillcolor{currentfill}%
\pgfsetfillopacity{0.700000}%
\pgfsetlinewidth{0.000000pt}%
\definecolor{currentstroke}{rgb}{0.000000,0.000000,0.000000}%
\pgfsetstrokecolor{currentstroke}%
\pgfsetdash{}{0pt}%
\pgfpathmoveto{\pgfqpoint{5.093237in}{2.725717in}}%
\pgfpathlineto{\pgfqpoint{5.106544in}{2.725685in}}%
\pgfpathlineto{\pgfqpoint{5.119862in}{2.725769in}}%
\pgfpathlineto{\pgfqpoint{5.133190in}{2.725969in}}%
\pgfpathlineto{\pgfqpoint{5.146528in}{2.726285in}}%
\pgfpathlineto{\pgfqpoint{5.153701in}{2.736014in}}%
\pgfpathlineto{\pgfqpoint{5.160870in}{2.745757in}}%
\pgfpathlineto{\pgfqpoint{5.168035in}{2.755516in}}%
\pgfpathlineto{\pgfqpoint{5.175196in}{2.765292in}}%
\pgfpathlineto{\pgfqpoint{5.161870in}{2.765126in}}%
\pgfpathlineto{\pgfqpoint{5.148554in}{2.765076in}}%
\pgfpathlineto{\pgfqpoint{5.135248in}{2.765142in}}%
\pgfpathlineto{\pgfqpoint{5.121952in}{2.765324in}}%
\pgfpathlineto{\pgfqpoint{5.114779in}{2.755392in}}%
\pgfpathlineto{\pgfqpoint{5.107602in}{2.745481in}}%
\pgfpathlineto{\pgfqpoint{5.100422in}{2.735590in}}%
\pgfpathlineto{\pgfqpoint{5.093237in}{2.725717in}}%
\pgfpathclose%
\pgfusepath{fill}%
\end{pgfscope}%
\begin{pgfscope}%
\pgfpathrectangle{\pgfqpoint{1.254980in}{0.150000in}}{\pgfqpoint{5.490039in}{5.490039in}}%
\pgfusepath{clip}%
\pgfsetbuttcap%
\pgfsetroundjoin%
\definecolor{currentfill}{rgb}{0.277134,0.185228,0.489898}%
\pgfsetfillcolor{currentfill}%
\pgfsetfillopacity{0.700000}%
\pgfsetlinewidth{0.000000pt}%
\definecolor{currentstroke}{rgb}{0.000000,0.000000,0.000000}%
\pgfsetstrokecolor{currentstroke}%
\pgfsetdash{}{0pt}%
\pgfpathmoveto{\pgfqpoint{3.714263in}{2.535878in}}%
\pgfpathlineto{\pgfqpoint{3.727235in}{2.526534in}}%
\pgfpathlineto{\pgfqpoint{3.740209in}{2.517338in}}%
\pgfpathlineto{\pgfqpoint{3.753184in}{2.508288in}}%
\pgfpathlineto{\pgfqpoint{3.766162in}{2.499384in}}%
\pgfpathlineto{\pgfqpoint{3.773777in}{2.508568in}}%
\pgfpathlineto{\pgfqpoint{3.781386in}{2.517805in}}%
\pgfpathlineto{\pgfqpoint{3.788990in}{2.527095in}}%
\pgfpathlineto{\pgfqpoint{3.796588in}{2.536437in}}%
\pgfpathlineto{\pgfqpoint{3.783624in}{2.545283in}}%
\pgfpathlineto{\pgfqpoint{3.770662in}{2.554275in}}%
\pgfpathlineto{\pgfqpoint{3.757702in}{2.563414in}}%
\pgfpathlineto{\pgfqpoint{3.744744in}{2.572700in}}%
\pgfpathlineto{\pgfqpoint{3.737132in}{2.563410in}}%
\pgfpathlineto{\pgfqpoint{3.729514in}{2.554176in}}%
\pgfpathlineto{\pgfqpoint{3.721892in}{2.544999in}}%
\pgfpathlineto{\pgfqpoint{3.714263in}{2.535878in}}%
\pgfpathclose%
\pgfusepath{fill}%
\end{pgfscope}%
\begin{pgfscope}%
\pgfpathrectangle{\pgfqpoint{1.254980in}{0.150000in}}{\pgfqpoint{5.490039in}{5.490039in}}%
\pgfusepath{clip}%
\pgfsetbuttcap%
\pgfsetroundjoin%
\definecolor{currentfill}{rgb}{0.281412,0.155834,0.469201}%
\pgfsetfillcolor{currentfill}%
\pgfsetfillopacity{0.700000}%
\pgfsetlinewidth{0.000000pt}%
\definecolor{currentstroke}{rgb}{0.000000,0.000000,0.000000}%
\pgfsetstrokecolor{currentstroke}%
\pgfsetdash{}{0pt}%
\pgfpathmoveto{\pgfqpoint{4.250759in}{2.463937in}}%
\pgfpathlineto{\pgfqpoint{4.263805in}{2.459151in}}%
\pgfpathlineto{\pgfqpoint{4.276858in}{2.454495in}}%
\pgfpathlineto{\pgfqpoint{4.289916in}{2.449969in}}%
\pgfpathlineto{\pgfqpoint{4.302981in}{2.445571in}}%
\pgfpathlineto{\pgfqpoint{4.310423in}{2.455594in}}%
\pgfpathlineto{\pgfqpoint{4.317861in}{2.465638in}}%
\pgfpathlineto{\pgfqpoint{4.325294in}{2.475706in}}%
\pgfpathlineto{\pgfqpoint{4.332723in}{2.485796in}}%
\pgfpathlineto{\pgfqpoint{4.319669in}{2.490201in}}%
\pgfpathlineto{\pgfqpoint{4.306621in}{2.494734in}}%
\pgfpathlineto{\pgfqpoint{4.293579in}{2.499396in}}%
\pgfpathlineto{\pgfqpoint{4.280542in}{2.504189in}}%
\pgfpathlineto{\pgfqpoint{4.273103in}{2.494086in}}%
\pgfpathlineto{\pgfqpoint{4.265660in}{2.484010in}}%
\pgfpathlineto{\pgfqpoint{4.258211in}{2.473960in}}%
\pgfpathlineto{\pgfqpoint{4.250759in}{2.463937in}}%
\pgfpathclose%
\pgfusepath{fill}%
\end{pgfscope}%
\begin{pgfscope}%
\pgfpathrectangle{\pgfqpoint{1.254980in}{0.150000in}}{\pgfqpoint{5.490039in}{5.490039in}}%
\pgfusepath{clip}%
\pgfsetbuttcap%
\pgfsetroundjoin%
\definecolor{currentfill}{rgb}{0.252194,0.269783,0.531579}%
\pgfsetfillcolor{currentfill}%
\pgfsetfillopacity{0.700000}%
\pgfsetlinewidth{0.000000pt}%
\definecolor{currentstroke}{rgb}{0.000000,0.000000,0.000000}%
\pgfsetstrokecolor{currentstroke}%
\pgfsetdash{}{0pt}%
\pgfpathmoveto{\pgfqpoint{5.011282in}{2.687130in}}%
\pgfpathlineto{\pgfqpoint{5.024560in}{2.686765in}}%
\pgfpathlineto{\pgfqpoint{5.037849in}{2.686518in}}%
\pgfpathlineto{\pgfqpoint{5.051148in}{2.686387in}}%
\pgfpathlineto{\pgfqpoint{5.064456in}{2.686374in}}%
\pgfpathlineto{\pgfqpoint{5.071658in}{2.696190in}}%
\pgfpathlineto{\pgfqpoint{5.078855in}{2.706018in}}%
\pgfpathlineto{\pgfqpoint{5.086048in}{2.715860in}}%
\pgfpathlineto{\pgfqpoint{5.093237in}{2.725717in}}%
\pgfpathlineto{\pgfqpoint{5.079940in}{2.725865in}}%
\pgfpathlineto{\pgfqpoint{5.066652in}{2.726130in}}%
\pgfpathlineto{\pgfqpoint{5.053375in}{2.726511in}}%
\pgfpathlineto{\pgfqpoint{5.040107in}{2.727009in}}%
\pgfpathlineto{\pgfqpoint{5.032907in}{2.717013in}}%
\pgfpathlineto{\pgfqpoint{5.025702in}{2.707035in}}%
\pgfpathlineto{\pgfqpoint{5.018494in}{2.697074in}}%
\pgfpathlineto{\pgfqpoint{5.011282in}{2.687130in}}%
\pgfpathclose%
\pgfusepath{fill}%
\end{pgfscope}%
\begin{pgfscope}%
\pgfpathrectangle{\pgfqpoint{1.254980in}{0.150000in}}{\pgfqpoint{5.490039in}{5.490039in}}%
\pgfusepath{clip}%
\pgfsetbuttcap%
\pgfsetroundjoin%
\definecolor{currentfill}{rgb}{0.266580,0.228262,0.514349}%
\pgfsetfillcolor{currentfill}%
\pgfsetfillopacity{0.700000}%
\pgfsetlinewidth{0.000000pt}%
\definecolor{currentstroke}{rgb}{0.000000,0.000000,0.000000}%
\pgfsetstrokecolor{currentstroke}%
\pgfsetdash{}{0pt}%
\pgfpathmoveto{\pgfqpoint{3.527955in}{2.624743in}}%
\pgfpathlineto{\pgfqpoint{3.540929in}{2.613500in}}%
\pgfpathlineto{\pgfqpoint{3.553904in}{2.602413in}}%
\pgfpathlineto{\pgfqpoint{3.566879in}{2.591482in}}%
\pgfpathlineto{\pgfqpoint{3.579854in}{2.580705in}}%
\pgfpathlineto{\pgfqpoint{3.587535in}{2.589435in}}%
\pgfpathlineto{\pgfqpoint{3.595210in}{2.598231in}}%
\pgfpathlineto{\pgfqpoint{3.602879in}{2.607093in}}%
\pgfpathlineto{\pgfqpoint{3.610543in}{2.616019in}}%
\pgfpathlineto{\pgfqpoint{3.597583in}{2.626721in}}%
\pgfpathlineto{\pgfqpoint{3.584623in}{2.637578in}}%
\pgfpathlineto{\pgfqpoint{3.571665in}{2.648590in}}%
\pgfpathlineto{\pgfqpoint{3.558706in}{2.659758in}}%
\pgfpathlineto{\pgfqpoint{3.551027in}{2.650900in}}%
\pgfpathlineto{\pgfqpoint{3.543343in}{2.642111in}}%
\pgfpathlineto{\pgfqpoint{3.535652in}{2.633392in}}%
\pgfpathlineto{\pgfqpoint{3.527955in}{2.624743in}}%
\pgfpathclose%
\pgfusepath{fill}%
\end{pgfscope}%
\begin{pgfscope}%
\pgfpathrectangle{\pgfqpoint{1.254980in}{0.150000in}}{\pgfqpoint{5.490039in}{5.490039in}}%
\pgfusepath{clip}%
\pgfsetbuttcap%
\pgfsetroundjoin%
\definecolor{currentfill}{rgb}{0.258965,0.251537,0.524736}%
\pgfsetfillcolor{currentfill}%
\pgfsetfillopacity{0.700000}%
\pgfsetlinewidth{0.000000pt}%
\definecolor{currentstroke}{rgb}{0.000000,0.000000,0.000000}%
\pgfsetstrokecolor{currentstroke}%
\pgfsetdash{}{0pt}%
\pgfpathmoveto{\pgfqpoint{4.929328in}{2.649640in}}%
\pgfpathlineto{\pgfqpoint{4.942579in}{2.648924in}}%
\pgfpathlineto{\pgfqpoint{4.955840in}{2.648325in}}%
\pgfpathlineto{\pgfqpoint{4.969110in}{2.647844in}}%
\pgfpathlineto{\pgfqpoint{4.982390in}{2.647481in}}%
\pgfpathlineto{\pgfqpoint{4.989619in}{2.657376in}}%
\pgfpathlineto{\pgfqpoint{4.996844in}{2.667282in}}%
\pgfpathlineto{\pgfqpoint{5.004065in}{2.677199in}}%
\pgfpathlineto{\pgfqpoint{5.011282in}{2.687130in}}%
\pgfpathlineto{\pgfqpoint{4.998013in}{2.687611in}}%
\pgfpathlineto{\pgfqpoint{4.984753in}{2.688210in}}%
\pgfpathlineto{\pgfqpoint{4.971503in}{2.688927in}}%
\pgfpathlineto{\pgfqpoint{4.958263in}{2.689762in}}%
\pgfpathlineto{\pgfqpoint{4.951035in}{2.679707in}}%
\pgfpathlineto{\pgfqpoint{4.943803in}{2.669670in}}%
\pgfpathlineto{\pgfqpoint{4.936568in}{2.659648in}}%
\pgfpathlineto{\pgfqpoint{4.929328in}{2.649640in}}%
\pgfpathclose%
\pgfusepath{fill}%
\end{pgfscope}%
\begin{pgfscope}%
\pgfpathrectangle{\pgfqpoint{1.254980in}{0.150000in}}{\pgfqpoint{5.490039in}{5.490039in}}%
\pgfusepath{clip}%
\pgfsetbuttcap%
\pgfsetroundjoin%
\definecolor{currentfill}{rgb}{0.140536,0.530132,0.555659}%
\pgfsetfillcolor{currentfill}%
\pgfsetfillopacity{0.700000}%
\pgfsetlinewidth{0.000000pt}%
\definecolor{currentstroke}{rgb}{0.000000,0.000000,0.000000}%
\pgfsetstrokecolor{currentstroke}%
\pgfsetdash{}{0pt}%
\pgfpathmoveto{\pgfqpoint{2.902835in}{3.366446in}}%
\pgfpathlineto{\pgfqpoint{2.915963in}{3.346543in}}%
\pgfpathlineto{\pgfqpoint{2.929083in}{3.326850in}}%
\pgfpathlineto{\pgfqpoint{2.942197in}{3.307364in}}%
\pgfpathlineto{\pgfqpoint{2.955305in}{3.288084in}}%
\pgfpathlineto{\pgfqpoint{2.963212in}{3.295805in}}%
\pgfpathlineto{\pgfqpoint{2.971110in}{3.303641in}}%
\pgfpathlineto{\pgfqpoint{2.979001in}{3.311591in}}%
\pgfpathlineto{\pgfqpoint{2.986884in}{3.319655in}}%
\pgfpathlineto{\pgfqpoint{2.973799in}{3.338853in}}%
\pgfpathlineto{\pgfqpoint{2.960707in}{3.358256in}}%
\pgfpathlineto{\pgfqpoint{2.947609in}{3.377867in}}%
\pgfpathlineto{\pgfqpoint{2.934504in}{3.397687in}}%
\pgfpathlineto{\pgfqpoint{2.926599in}{3.389699in}}%
\pgfpathlineto{\pgfqpoint{2.918686in}{3.381829in}}%
\pgfpathlineto{\pgfqpoint{2.910765in}{3.374078in}}%
\pgfpathlineto{\pgfqpoint{2.902835in}{3.366446in}}%
\pgfpathclose%
\pgfusepath{fill}%
\end{pgfscope}%
\begin{pgfscope}%
\pgfpathrectangle{\pgfqpoint{1.254980in}{0.150000in}}{\pgfqpoint{5.490039in}{5.490039in}}%
\pgfusepath{clip}%
\pgfsetbuttcap%
\pgfsetroundjoin%
\definecolor{currentfill}{rgb}{0.278826,0.175490,0.483397}%
\pgfsetfillcolor{currentfill}%
\pgfsetfillopacity{0.700000}%
\pgfsetlinewidth{0.000000pt}%
\definecolor{currentstroke}{rgb}{0.000000,0.000000,0.000000}%
\pgfsetstrokecolor{currentstroke}%
\pgfsetdash{}{0pt}%
\pgfpathmoveto{\pgfqpoint{4.466975in}{2.495585in}}%
\pgfpathlineto{\pgfqpoint{4.480078in}{2.492302in}}%
\pgfpathlineto{\pgfqpoint{4.493188in}{2.489145in}}%
\pgfpathlineto{\pgfqpoint{4.506305in}{2.486112in}}%
\pgfpathlineto{\pgfqpoint{4.519430in}{2.483204in}}%
\pgfpathlineto{\pgfqpoint{4.526807in}{2.493320in}}%
\pgfpathlineto{\pgfqpoint{4.534179in}{2.503450in}}%
\pgfpathlineto{\pgfqpoint{4.541546in}{2.513596in}}%
\pgfpathlineto{\pgfqpoint{4.548910in}{2.523757in}}%
\pgfpathlineto{\pgfqpoint{4.535795in}{2.526703in}}%
\pgfpathlineto{\pgfqpoint{4.522687in}{2.529774in}}%
\pgfpathlineto{\pgfqpoint{4.509587in}{2.532970in}}%
\pgfpathlineto{\pgfqpoint{4.496494in}{2.536291in}}%
\pgfpathlineto{\pgfqpoint{4.489121in}{2.526086in}}%
\pgfpathlineto{\pgfqpoint{4.481743in}{2.515900in}}%
\pgfpathlineto{\pgfqpoint{4.474361in}{2.505733in}}%
\pgfpathlineto{\pgfqpoint{4.466975in}{2.495585in}}%
\pgfpathclose%
\pgfusepath{fill}%
\end{pgfscope}%
\begin{pgfscope}%
\pgfpathrectangle{\pgfqpoint{1.254980in}{0.150000in}}{\pgfqpoint{5.490039in}{5.490039in}}%
\pgfusepath{clip}%
\pgfsetbuttcap%
\pgfsetroundjoin%
\definecolor{currentfill}{rgb}{0.281412,0.155834,0.469201}%
\pgfsetfillcolor{currentfill}%
\pgfsetfillopacity{0.700000}%
\pgfsetlinewidth{0.000000pt}%
\definecolor{currentstroke}{rgb}{0.000000,0.000000,0.000000}%
\pgfsetstrokecolor{currentstroke}%
\pgfsetdash{}{0pt}%
\pgfpathmoveto{\pgfqpoint{3.900397in}{2.470848in}}%
\pgfpathlineto{\pgfqpoint{3.913387in}{2.463288in}}%
\pgfpathlineto{\pgfqpoint{3.926380in}{2.455868in}}%
\pgfpathlineto{\pgfqpoint{3.939377in}{2.448587in}}%
\pgfpathlineto{\pgfqpoint{3.952377in}{2.441445in}}%
\pgfpathlineto{\pgfqpoint{3.959932in}{2.450981in}}%
\pgfpathlineto{\pgfqpoint{3.967482in}{2.460558in}}%
\pgfpathlineto{\pgfqpoint{3.975027in}{2.470176in}}%
\pgfpathlineto{\pgfqpoint{3.982567in}{2.479833in}}%
\pgfpathlineto{\pgfqpoint{3.969579in}{2.486934in}}%
\pgfpathlineto{\pgfqpoint{3.956594in}{2.494174in}}%
\pgfpathlineto{\pgfqpoint{3.943614in}{2.501553in}}%
\pgfpathlineto{\pgfqpoint{3.930636in}{2.509072in}}%
\pgfpathlineto{\pgfqpoint{3.923084in}{2.499449in}}%
\pgfpathlineto{\pgfqpoint{3.915527in}{2.489871in}}%
\pgfpathlineto{\pgfqpoint{3.907965in}{2.480337in}}%
\pgfpathlineto{\pgfqpoint{3.900397in}{2.470848in}}%
\pgfpathclose%
\pgfusepath{fill}%
\end{pgfscope}%
\begin{pgfscope}%
\pgfpathrectangle{\pgfqpoint{1.254980in}{0.150000in}}{\pgfqpoint{5.490039in}{5.490039in}}%
\pgfusepath{clip}%
\pgfsetbuttcap%
\pgfsetroundjoin%
\definecolor{currentfill}{rgb}{0.281887,0.150881,0.465405}%
\pgfsetfillcolor{currentfill}%
\pgfsetfillopacity{0.700000}%
\pgfsetlinewidth{0.000000pt}%
\definecolor{currentstroke}{rgb}{0.000000,0.000000,0.000000}%
\pgfsetstrokecolor{currentstroke}%
\pgfsetdash{}{0pt}%
\pgfpathmoveto{\pgfqpoint{4.034559in}{2.452806in}}%
\pgfpathlineto{\pgfqpoint{4.047567in}{2.446391in}}%
\pgfpathlineto{\pgfqpoint{4.060580in}{2.440112in}}%
\pgfpathlineto{\pgfqpoint{4.073597in}{2.433967in}}%
\pgfpathlineto{\pgfqpoint{4.086619in}{2.427957in}}%
\pgfpathlineto{\pgfqpoint{4.094131in}{2.437717in}}%
\pgfpathlineto{\pgfqpoint{4.101638in}{2.447510in}}%
\pgfpathlineto{\pgfqpoint{4.109140in}{2.457335in}}%
\pgfpathlineto{\pgfqpoint{4.116637in}{2.467193in}}%
\pgfpathlineto{\pgfqpoint{4.103627in}{2.473178in}}%
\pgfpathlineto{\pgfqpoint{4.090621in}{2.479297in}}%
\pgfpathlineto{\pgfqpoint{4.077620in}{2.485552in}}%
\pgfpathlineto{\pgfqpoint{4.064623in}{2.491942in}}%
\pgfpathlineto{\pgfqpoint{4.057114in}{2.482103in}}%
\pgfpathlineto{\pgfqpoint{4.049601in}{2.472301in}}%
\pgfpathlineto{\pgfqpoint{4.042082in}{2.462535in}}%
\pgfpathlineto{\pgfqpoint{4.034559in}{2.452806in}}%
\pgfpathclose%
\pgfusepath{fill}%
\end{pgfscope}%
\begin{pgfscope}%
\pgfpathrectangle{\pgfqpoint{1.254980in}{0.150000in}}{\pgfqpoint{5.490039in}{5.490039in}}%
\pgfusepath{clip}%
\pgfsetbuttcap%
\pgfsetroundjoin%
\definecolor{currentfill}{rgb}{0.263663,0.237631,0.518762}%
\pgfsetfillcolor{currentfill}%
\pgfsetfillopacity{0.700000}%
\pgfsetlinewidth{0.000000pt}%
\definecolor{currentstroke}{rgb}{0.000000,0.000000,0.000000}%
\pgfsetstrokecolor{currentstroke}%
\pgfsetdash{}{0pt}%
\pgfpathmoveto{\pgfqpoint{4.847371in}{2.613369in}}%
\pgfpathlineto{\pgfqpoint{4.860596in}{2.612280in}}%
\pgfpathlineto{\pgfqpoint{4.873830in}{2.611310in}}%
\pgfpathlineto{\pgfqpoint{4.887073in}{2.610459in}}%
\pgfpathlineto{\pgfqpoint{4.900325in}{2.609726in}}%
\pgfpathlineto{\pgfqpoint{4.907582in}{2.619690in}}%
\pgfpathlineto{\pgfqpoint{4.914835in}{2.629663in}}%
\pgfpathlineto{\pgfqpoint{4.922083in}{2.639646in}}%
\pgfpathlineto{\pgfqpoint{4.929328in}{2.649640in}}%
\pgfpathlineto{\pgfqpoint{4.916086in}{2.650475in}}%
\pgfpathlineto{\pgfqpoint{4.902853in}{2.651429in}}%
\pgfpathlineto{\pgfqpoint{4.889630in}{2.652501in}}%
\pgfpathlineto{\pgfqpoint{4.876416in}{2.653692in}}%
\pgfpathlineto{\pgfqpoint{4.869161in}{2.643589in}}%
\pgfpathlineto{\pgfqpoint{4.861902in}{2.633502in}}%
\pgfpathlineto{\pgfqpoint{4.854639in}{2.623429in}}%
\pgfpathlineto{\pgfqpoint{4.847371in}{2.613369in}}%
\pgfpathclose%
\pgfusepath{fill}%
\end{pgfscope}%
\begin{pgfscope}%
\pgfpathrectangle{\pgfqpoint{1.254980in}{0.150000in}}{\pgfqpoint{5.490039in}{5.490039in}}%
\pgfusepath{clip}%
\pgfsetbuttcap%
\pgfsetroundjoin%
\definecolor{currentfill}{rgb}{0.269308,0.218818,0.509577}%
\pgfsetfillcolor{currentfill}%
\pgfsetfillopacity{0.700000}%
\pgfsetlinewidth{0.000000pt}%
\definecolor{currentstroke}{rgb}{0.000000,0.000000,0.000000}%
\pgfsetstrokecolor{currentstroke}%
\pgfsetdash{}{0pt}%
\pgfpathmoveto{\pgfqpoint{4.765409in}{2.578446in}}%
\pgfpathlineto{\pgfqpoint{4.778608in}{2.576964in}}%
\pgfpathlineto{\pgfqpoint{4.791816in}{2.575603in}}%
\pgfpathlineto{\pgfqpoint{4.805033in}{2.574361in}}%
\pgfpathlineto{\pgfqpoint{4.818259in}{2.573239in}}%
\pgfpathlineto{\pgfqpoint{4.825544in}{2.583257in}}%
\pgfpathlineto{\pgfqpoint{4.832824in}{2.593285in}}%
\pgfpathlineto{\pgfqpoint{4.840100in}{2.603322in}}%
\pgfpathlineto{\pgfqpoint{4.847371in}{2.613369in}}%
\pgfpathlineto{\pgfqpoint{4.834156in}{2.614578in}}%
\pgfpathlineto{\pgfqpoint{4.820949in}{2.615906in}}%
\pgfpathlineto{\pgfqpoint{4.807751in}{2.617354in}}%
\pgfpathlineto{\pgfqpoint{4.794562in}{2.618921in}}%
\pgfpathlineto{\pgfqpoint{4.787280in}{2.608781in}}%
\pgfpathlineto{\pgfqpoint{4.779994in}{2.598656in}}%
\pgfpathlineto{\pgfqpoint{4.772703in}{2.588545in}}%
\pgfpathlineto{\pgfqpoint{4.765409in}{2.578446in}}%
\pgfpathclose%
\pgfusepath{fill}%
\end{pgfscope}%
\begin{pgfscope}%
\pgfpathrectangle{\pgfqpoint{1.254980in}{0.150000in}}{\pgfqpoint{5.490039in}{5.490039in}}%
\pgfusepath{clip}%
\pgfsetbuttcap%
\pgfsetroundjoin%
\definecolor{currentfill}{rgb}{0.279574,0.170599,0.479997}%
\pgfsetfillcolor{currentfill}%
\pgfsetfillopacity{0.700000}%
\pgfsetlinewidth{0.000000pt}%
\definecolor{currentstroke}{rgb}{0.000000,0.000000,0.000000}%
\pgfsetstrokecolor{currentstroke}%
\pgfsetdash{}{0pt}%
\pgfpathmoveto{\pgfqpoint{3.766162in}{2.499384in}}%
\pgfpathlineto{\pgfqpoint{3.779143in}{2.490626in}}%
\pgfpathlineto{\pgfqpoint{3.792125in}{2.482012in}}%
\pgfpathlineto{\pgfqpoint{3.805110in}{2.473543in}}%
\pgfpathlineto{\pgfqpoint{3.818098in}{2.465217in}}%
\pgfpathlineto{\pgfqpoint{3.825699in}{2.474464in}}%
\pgfpathlineto{\pgfqpoint{3.833295in}{2.483761in}}%
\pgfpathlineto{\pgfqpoint{3.840885in}{2.493106in}}%
\pgfpathlineto{\pgfqpoint{3.848470in}{2.502499in}}%
\pgfpathlineto{\pgfqpoint{3.835496in}{2.510768in}}%
\pgfpathlineto{\pgfqpoint{3.822524in}{2.519180in}}%
\pgfpathlineto{\pgfqpoint{3.809555in}{2.527736in}}%
\pgfpathlineto{\pgfqpoint{3.796588in}{2.536437in}}%
\pgfpathlineto{\pgfqpoint{3.788990in}{2.527095in}}%
\pgfpathlineto{\pgfqpoint{3.781386in}{2.517805in}}%
\pgfpathlineto{\pgfqpoint{3.773777in}{2.508568in}}%
\pgfpathlineto{\pgfqpoint{3.766162in}{2.499384in}}%
\pgfpathclose%
\pgfusepath{fill}%
\end{pgfscope}%
\begin{pgfscope}%
\pgfpathrectangle{\pgfqpoint{1.254980in}{0.150000in}}{\pgfqpoint{5.490039in}{5.490039in}}%
\pgfusepath{clip}%
\pgfsetbuttcap%
\pgfsetroundjoin%
\definecolor{currentfill}{rgb}{0.271828,0.209303,0.504434}%
\pgfsetfillcolor{currentfill}%
\pgfsetfillopacity{0.700000}%
\pgfsetlinewidth{0.000000pt}%
\definecolor{currentstroke}{rgb}{0.000000,0.000000,0.000000}%
\pgfsetstrokecolor{currentstroke}%
\pgfsetdash{}{0pt}%
\pgfpathmoveto{\pgfqpoint{3.579854in}{2.580705in}}%
\pgfpathlineto{\pgfqpoint{3.592831in}{2.570082in}}%
\pgfpathlineto{\pgfqpoint{3.605808in}{2.559613in}}%
\pgfpathlineto{\pgfqpoint{3.618786in}{2.549296in}}%
\pgfpathlineto{\pgfqpoint{3.631765in}{2.539130in}}%
\pgfpathlineto{\pgfqpoint{3.639430in}{2.547940in}}%
\pgfpathlineto{\pgfqpoint{3.647090in}{2.556813in}}%
\pgfpathlineto{\pgfqpoint{3.654744in}{2.565747in}}%
\pgfpathlineto{\pgfqpoint{3.662393in}{2.574742in}}%
\pgfpathlineto{\pgfqpoint{3.649428in}{2.584833in}}%
\pgfpathlineto{\pgfqpoint{3.636466in}{2.595076in}}%
\pgfpathlineto{\pgfqpoint{3.623504in}{2.605471in}}%
\pgfpathlineto{\pgfqpoint{3.610543in}{2.616019in}}%
\pgfpathlineto{\pgfqpoint{3.602879in}{2.607093in}}%
\pgfpathlineto{\pgfqpoint{3.595210in}{2.598231in}}%
\pgfpathlineto{\pgfqpoint{3.587535in}{2.589435in}}%
\pgfpathlineto{\pgfqpoint{3.579854in}{2.580705in}}%
\pgfpathclose%
\pgfusepath{fill}%
\end{pgfscope}%
\begin{pgfscope}%
\pgfpathrectangle{\pgfqpoint{1.254980in}{0.150000in}}{\pgfqpoint{5.490039in}{5.490039in}}%
\pgfusepath{clip}%
\pgfsetbuttcap%
\pgfsetroundjoin%
\definecolor{currentfill}{rgb}{0.281887,0.150881,0.465405}%
\pgfsetfillcolor{currentfill}%
\pgfsetfillopacity{0.700000}%
\pgfsetlinewidth{0.000000pt}%
\definecolor{currentstroke}{rgb}{0.000000,0.000000,0.000000}%
\pgfsetstrokecolor{currentstroke}%
\pgfsetdash{}{0pt}%
\pgfpathmoveto{\pgfqpoint{4.168728in}{2.444590in}}%
\pgfpathlineto{\pgfqpoint{4.181764in}{2.439270in}}%
\pgfpathlineto{\pgfqpoint{4.194804in}{2.434083in}}%
\pgfpathlineto{\pgfqpoint{4.207850in}{2.429027in}}%
\pgfpathlineto{\pgfqpoint{4.220902in}{2.424101in}}%
\pgfpathlineto{\pgfqpoint{4.228373in}{2.434022in}}%
\pgfpathlineto{\pgfqpoint{4.235840in}{2.443968in}}%
\pgfpathlineto{\pgfqpoint{4.243302in}{2.453940in}}%
\pgfpathlineto{\pgfqpoint{4.250759in}{2.463937in}}%
\pgfpathlineto{\pgfqpoint{4.237718in}{2.468853in}}%
\pgfpathlineto{\pgfqpoint{4.224682in}{2.473901in}}%
\pgfpathlineto{\pgfqpoint{4.211652in}{2.479079in}}%
\pgfpathlineto{\pgfqpoint{4.198628in}{2.484389in}}%
\pgfpathlineto{\pgfqpoint{4.191160in}{2.474395in}}%
\pgfpathlineto{\pgfqpoint{4.183687in}{2.464431in}}%
\pgfpathlineto{\pgfqpoint{4.176210in}{2.454496in}}%
\pgfpathlineto{\pgfqpoint{4.168728in}{2.444590in}}%
\pgfpathclose%
\pgfusepath{fill}%
\end{pgfscope}%
\begin{pgfscope}%
\pgfpathrectangle{\pgfqpoint{1.254980in}{0.150000in}}{\pgfqpoint{5.490039in}{5.490039in}}%
\pgfusepath{clip}%
\pgfsetbuttcap%
\pgfsetroundjoin%
\definecolor{currentfill}{rgb}{0.280255,0.165693,0.476498}%
\pgfsetfillcolor{currentfill}%
\pgfsetfillopacity{0.700000}%
\pgfsetlinewidth{0.000000pt}%
\definecolor{currentstroke}{rgb}{0.000000,0.000000,0.000000}%
\pgfsetstrokecolor{currentstroke}%
\pgfsetdash{}{0pt}%
\pgfpathmoveto{\pgfqpoint{4.385001in}{2.469463in}}%
\pgfpathlineto{\pgfqpoint{4.398087in}{2.465698in}}%
\pgfpathlineto{\pgfqpoint{4.411180in}{2.462060in}}%
\pgfpathlineto{\pgfqpoint{4.424279in}{2.458549in}}%
\pgfpathlineto{\pgfqpoint{4.437385in}{2.455163in}}%
\pgfpathlineto{\pgfqpoint{4.444789in}{2.465244in}}%
\pgfpathlineto{\pgfqpoint{4.452189in}{2.475341in}}%
\pgfpathlineto{\pgfqpoint{4.459584in}{2.485454in}}%
\pgfpathlineto{\pgfqpoint{4.466975in}{2.495585in}}%
\pgfpathlineto{\pgfqpoint{4.453879in}{2.498993in}}%
\pgfpathlineto{\pgfqpoint{4.440789in}{2.502527in}}%
\pgfpathlineto{\pgfqpoint{4.427707in}{2.506188in}}%
\pgfpathlineto{\pgfqpoint{4.414631in}{2.509975in}}%
\pgfpathlineto{\pgfqpoint{4.407230in}{2.499816in}}%
\pgfpathlineto{\pgfqpoint{4.399825in}{2.489678in}}%
\pgfpathlineto{\pgfqpoint{4.392416in}{2.479560in}}%
\pgfpathlineto{\pgfqpoint{4.385001in}{2.469463in}}%
\pgfpathclose%
\pgfusepath{fill}%
\end{pgfscope}%
\begin{pgfscope}%
\pgfpathrectangle{\pgfqpoint{1.254980in}{0.150000in}}{\pgfqpoint{5.490039in}{5.490039in}}%
\pgfusepath{clip}%
\pgfsetbuttcap%
\pgfsetroundjoin%
\definecolor{currentfill}{rgb}{0.212395,0.359683,0.551710}%
\pgfsetfillcolor{currentfill}%
\pgfsetfillopacity{0.700000}%
\pgfsetlinewidth{0.000000pt}%
\definecolor{currentstroke}{rgb}{0.000000,0.000000,0.000000}%
\pgfsetstrokecolor{currentstroke}%
\pgfsetdash{}{0pt}%
\pgfpathmoveto{\pgfqpoint{3.185040in}{2.912014in}}%
\pgfpathlineto{\pgfqpoint{3.198066in}{2.896662in}}%
\pgfpathlineto{\pgfqpoint{3.211090in}{2.881489in}}%
\pgfpathlineto{\pgfqpoint{3.224111in}{2.866494in}}%
\pgfpathlineto{\pgfqpoint{3.237129in}{2.851675in}}%
\pgfpathlineto{\pgfqpoint{3.244941in}{2.859548in}}%
\pgfpathlineto{\pgfqpoint{3.252745in}{2.867513in}}%
\pgfpathlineto{\pgfqpoint{3.260543in}{2.875569in}}%
\pgfpathlineto{\pgfqpoint{3.268333in}{2.883716in}}%
\pgfpathlineto{\pgfqpoint{3.255335in}{2.898440in}}%
\pgfpathlineto{\pgfqpoint{3.242334in}{2.913341in}}%
\pgfpathlineto{\pgfqpoint{3.229329in}{2.928420in}}%
\pgfpathlineto{\pgfqpoint{3.216322in}{2.943677in}}%
\pgfpathlineto{\pgfqpoint{3.208512in}{2.935618in}}%
\pgfpathlineto{\pgfqpoint{3.200695in}{2.927654in}}%
\pgfpathlineto{\pgfqpoint{3.192871in}{2.919786in}}%
\pgfpathlineto{\pgfqpoint{3.185040in}{2.912014in}}%
\pgfpathclose%
\pgfusepath{fill}%
\end{pgfscope}%
\begin{pgfscope}%
\pgfpathrectangle{\pgfqpoint{1.254980in}{0.150000in}}{\pgfqpoint{5.490039in}{5.490039in}}%
\pgfusepath{clip}%
\pgfsetbuttcap%
\pgfsetroundjoin%
\definecolor{currentfill}{rgb}{0.223925,0.334994,0.548053}%
\pgfsetfillcolor{currentfill}%
\pgfsetfillopacity{0.700000}%
\pgfsetlinewidth{0.000000pt}%
\definecolor{currentstroke}{rgb}{0.000000,0.000000,0.000000}%
\pgfsetstrokecolor{currentstroke}%
\pgfsetdash{}{0pt}%
\pgfpathmoveto{\pgfqpoint{3.237129in}{2.851675in}}%
\pgfpathlineto{\pgfqpoint{3.250144in}{2.837032in}}%
\pgfpathlineto{\pgfqpoint{3.263157in}{2.822563in}}%
\pgfpathlineto{\pgfqpoint{3.276167in}{2.808268in}}%
\pgfpathlineto{\pgfqpoint{3.289175in}{2.794145in}}%
\pgfpathlineto{\pgfqpoint{3.296967in}{2.802118in}}%
\pgfpathlineto{\pgfqpoint{3.304753in}{2.810178in}}%
\pgfpathlineto{\pgfqpoint{3.312531in}{2.818326in}}%
\pgfpathlineto{\pgfqpoint{3.320303in}{2.826560in}}%
\pgfpathlineto{\pgfqpoint{3.307314in}{2.840590in}}%
\pgfpathlineto{\pgfqpoint{3.294323in}{2.854792in}}%
\pgfpathlineto{\pgfqpoint{3.281329in}{2.869167in}}%
\pgfpathlineto{\pgfqpoint{3.268333in}{2.883716in}}%
\pgfpathlineto{\pgfqpoint{3.260543in}{2.875569in}}%
\pgfpathlineto{\pgfqpoint{3.252745in}{2.867513in}}%
\pgfpathlineto{\pgfqpoint{3.244941in}{2.859548in}}%
\pgfpathlineto{\pgfqpoint{3.237129in}{2.851675in}}%
\pgfpathclose%
\pgfusepath{fill}%
\end{pgfscope}%
\begin{pgfscope}%
\pgfpathrectangle{\pgfqpoint{1.254980in}{0.150000in}}{\pgfqpoint{5.490039in}{5.490039in}}%
\pgfusepath{clip}%
\pgfsetbuttcap%
\pgfsetroundjoin%
\definecolor{currentfill}{rgb}{0.201239,0.383670,0.554294}%
\pgfsetfillcolor{currentfill}%
\pgfsetfillopacity{0.700000}%
\pgfsetlinewidth{0.000000pt}%
\definecolor{currentstroke}{rgb}{0.000000,0.000000,0.000000}%
\pgfsetstrokecolor{currentstroke}%
\pgfsetdash{}{0pt}%
\pgfpathmoveto{\pgfqpoint{3.132897in}{2.975229in}}%
\pgfpathlineto{\pgfqpoint{3.145938in}{2.959152in}}%
\pgfpathlineto{\pgfqpoint{3.158976in}{2.943257in}}%
\pgfpathlineto{\pgfqpoint{3.172009in}{2.927545in}}%
\pgfpathlineto{\pgfqpoint{3.185040in}{2.912014in}}%
\pgfpathlineto{\pgfqpoint{3.192871in}{2.919786in}}%
\pgfpathlineto{\pgfqpoint{3.200695in}{2.927654in}}%
\pgfpathlineto{\pgfqpoint{3.208512in}{2.935618in}}%
\pgfpathlineto{\pgfqpoint{3.216322in}{2.943677in}}%
\pgfpathlineto{\pgfqpoint{3.203312in}{2.959113in}}%
\pgfpathlineto{\pgfqpoint{3.190298in}{2.974730in}}%
\pgfpathlineto{\pgfqpoint{3.177281in}{2.990529in}}%
\pgfpathlineto{\pgfqpoint{3.164260in}{3.006511in}}%
\pgfpathlineto{\pgfqpoint{3.156431in}{2.998541in}}%
\pgfpathlineto{\pgfqpoint{3.148594in}{2.990671in}}%
\pgfpathlineto{\pgfqpoint{3.140749in}{2.982900in}}%
\pgfpathlineto{\pgfqpoint{3.132897in}{2.975229in}}%
\pgfpathclose%
\pgfusepath{fill}%
\end{pgfscope}%
\begin{pgfscope}%
\pgfpathrectangle{\pgfqpoint{1.254980in}{0.150000in}}{\pgfqpoint{5.490039in}{5.490039in}}%
\pgfusepath{clip}%
\pgfsetbuttcap%
\pgfsetroundjoin%
\definecolor{currentfill}{rgb}{0.273006,0.204520,0.501721}%
\pgfsetfillcolor{currentfill}%
\pgfsetfillopacity{0.700000}%
\pgfsetlinewidth{0.000000pt}%
\definecolor{currentstroke}{rgb}{0.000000,0.000000,0.000000}%
\pgfsetstrokecolor{currentstroke}%
\pgfsetdash{}{0pt}%
\pgfpathmoveto{\pgfqpoint{4.683435in}{2.545010in}}%
\pgfpathlineto{\pgfqpoint{4.696610in}{2.543115in}}%
\pgfpathlineto{\pgfqpoint{4.709794in}{2.541341in}}%
\pgfpathlineto{\pgfqpoint{4.722986in}{2.539689in}}%
\pgfpathlineto{\pgfqpoint{4.736187in}{2.538157in}}%
\pgfpathlineto{\pgfqpoint{4.743499in}{2.548215in}}%
\pgfpathlineto{\pgfqpoint{4.750807in}{2.558282in}}%
\pgfpathlineto{\pgfqpoint{4.758110in}{2.568359in}}%
\pgfpathlineto{\pgfqpoint{4.765409in}{2.578446in}}%
\pgfpathlineto{\pgfqpoint{4.752218in}{2.580048in}}%
\pgfpathlineto{\pgfqpoint{4.739036in}{2.581771in}}%
\pgfpathlineto{\pgfqpoint{4.725862in}{2.583615in}}%
\pgfpathlineto{\pgfqpoint{4.712697in}{2.585580in}}%
\pgfpathlineto{\pgfqpoint{4.705388in}{2.575417in}}%
\pgfpathlineto{\pgfqpoint{4.698074in}{2.565268in}}%
\pgfpathlineto{\pgfqpoint{4.690757in}{2.555132in}}%
\pgfpathlineto{\pgfqpoint{4.683435in}{2.545010in}}%
\pgfpathclose%
\pgfusepath{fill}%
\end{pgfscope}%
\begin{pgfscope}%
\pgfpathrectangle{\pgfqpoint{1.254980in}{0.150000in}}{\pgfqpoint{5.490039in}{5.490039in}}%
\pgfusepath{clip}%
\pgfsetbuttcap%
\pgfsetroundjoin%
\definecolor{currentfill}{rgb}{0.235526,0.309527,0.542944}%
\pgfsetfillcolor{currentfill}%
\pgfsetfillopacity{0.700000}%
\pgfsetlinewidth{0.000000pt}%
\definecolor{currentstroke}{rgb}{0.000000,0.000000,0.000000}%
\pgfsetstrokecolor{currentstroke}%
\pgfsetdash{}{0pt}%
\pgfpathmoveto{\pgfqpoint{3.289175in}{2.794145in}}%
\pgfpathlineto{\pgfqpoint{3.302180in}{2.780194in}}%
\pgfpathlineto{\pgfqpoint{3.315184in}{2.766413in}}%
\pgfpathlineto{\pgfqpoint{3.328186in}{2.752801in}}%
\pgfpathlineto{\pgfqpoint{3.341186in}{2.739358in}}%
\pgfpathlineto{\pgfqpoint{3.348960in}{2.747430in}}%
\pgfpathlineto{\pgfqpoint{3.356727in}{2.755585in}}%
\pgfpathlineto{\pgfqpoint{3.364488in}{2.763824in}}%
\pgfpathlineto{\pgfqpoint{3.372242in}{2.772145in}}%
\pgfpathlineto{\pgfqpoint{3.359260in}{2.785495in}}%
\pgfpathlineto{\pgfqpoint{3.346276in}{2.799014in}}%
\pgfpathlineto{\pgfqpoint{3.333291in}{2.812702in}}%
\pgfpathlineto{\pgfqpoint{3.320303in}{2.826560in}}%
\pgfpathlineto{\pgfqpoint{3.312531in}{2.818326in}}%
\pgfpathlineto{\pgfqpoint{3.304753in}{2.810178in}}%
\pgfpathlineto{\pgfqpoint{3.296967in}{2.802118in}}%
\pgfpathlineto{\pgfqpoint{3.289175in}{2.794145in}}%
\pgfpathclose%
\pgfusepath{fill}%
\end{pgfscope}%
\begin{pgfscope}%
\pgfpathrectangle{\pgfqpoint{1.254980in}{0.150000in}}{\pgfqpoint{5.490039in}{5.490039in}}%
\pgfusepath{clip}%
\pgfsetbuttcap%
\pgfsetroundjoin%
\definecolor{currentfill}{rgb}{0.188923,0.410910,0.556326}%
\pgfsetfillcolor{currentfill}%
\pgfsetfillopacity{0.700000}%
\pgfsetlinewidth{0.000000pt}%
\definecolor{currentstroke}{rgb}{0.000000,0.000000,0.000000}%
\pgfsetstrokecolor{currentstroke}%
\pgfsetdash{}{0pt}%
\pgfpathmoveto{\pgfqpoint{3.080692in}{3.041393in}}%
\pgfpathlineto{\pgfqpoint{3.093750in}{3.024571in}}%
\pgfpathlineto{\pgfqpoint{3.106803in}{3.007937in}}%
\pgfpathlineto{\pgfqpoint{3.119852in}{2.991490in}}%
\pgfpathlineto{\pgfqpoint{3.132897in}{2.975229in}}%
\pgfpathlineto{\pgfqpoint{3.140749in}{2.982900in}}%
\pgfpathlineto{\pgfqpoint{3.148594in}{2.990671in}}%
\pgfpathlineto{\pgfqpoint{3.156431in}{2.998541in}}%
\pgfpathlineto{\pgfqpoint{3.164260in}{3.006511in}}%
\pgfpathlineto{\pgfqpoint{3.151236in}{3.022677in}}%
\pgfpathlineto{\pgfqpoint{3.138208in}{3.039028in}}%
\pgfpathlineto{\pgfqpoint{3.125175in}{3.055566in}}%
\pgfpathlineto{\pgfqpoint{3.112138in}{3.072291in}}%
\pgfpathlineto{\pgfqpoint{3.104288in}{3.064411in}}%
\pgfpathlineto{\pgfqpoint{3.096430in}{3.056634in}}%
\pgfpathlineto{\pgfqpoint{3.088565in}{3.048961in}}%
\pgfpathlineto{\pgfqpoint{3.080692in}{3.041393in}}%
\pgfpathclose%
\pgfusepath{fill}%
\end{pgfscope}%
\begin{pgfscope}%
\pgfpathrectangle{\pgfqpoint{1.254980in}{0.150000in}}{\pgfqpoint{5.490039in}{5.490039in}}%
\pgfusepath{clip}%
\pgfsetbuttcap%
\pgfsetroundjoin%
\definecolor{currentfill}{rgb}{0.244972,0.287675,0.537260}%
\pgfsetfillcolor{currentfill}%
\pgfsetfillopacity{0.700000}%
\pgfsetlinewidth{0.000000pt}%
\definecolor{currentstroke}{rgb}{0.000000,0.000000,0.000000}%
\pgfsetstrokecolor{currentstroke}%
\pgfsetdash{}{0pt}%
\pgfpathmoveto{\pgfqpoint{3.341186in}{2.739358in}}%
\pgfpathlineto{\pgfqpoint{3.354185in}{2.726083in}}%
\pgfpathlineto{\pgfqpoint{3.367182in}{2.712974in}}%
\pgfpathlineto{\pgfqpoint{3.380178in}{2.700031in}}%
\pgfpathlineto{\pgfqpoint{3.393173in}{2.687252in}}%
\pgfpathlineto{\pgfqpoint{3.400929in}{2.695422in}}%
\pgfpathlineto{\pgfqpoint{3.408678in}{2.703672in}}%
\pgfpathlineto{\pgfqpoint{3.416421in}{2.712001in}}%
\pgfpathlineto{\pgfqpoint{3.424157in}{2.720409in}}%
\pgfpathlineto{\pgfqpoint{3.411180in}{2.733095in}}%
\pgfpathlineto{\pgfqpoint{3.398202in}{2.745946in}}%
\pgfpathlineto{\pgfqpoint{3.385222in}{2.758962in}}%
\pgfpathlineto{\pgfqpoint{3.372242in}{2.772145in}}%
\pgfpathlineto{\pgfqpoint{3.364488in}{2.763824in}}%
\pgfpathlineto{\pgfqpoint{3.356727in}{2.755585in}}%
\pgfpathlineto{\pgfqpoint{3.348960in}{2.747430in}}%
\pgfpathlineto{\pgfqpoint{3.341186in}{2.739358in}}%
\pgfpathclose%
\pgfusepath{fill}%
\end{pgfscope}%
\begin{pgfscope}%
\pgfpathrectangle{\pgfqpoint{1.254980in}{0.150000in}}{\pgfqpoint{5.490039in}{5.490039in}}%
\pgfusepath{clip}%
\pgfsetbuttcap%
\pgfsetroundjoin%
\definecolor{currentfill}{rgb}{0.177423,0.437527,0.557565}%
\pgfsetfillcolor{currentfill}%
\pgfsetfillopacity{0.700000}%
\pgfsetlinewidth{0.000000pt}%
\definecolor{currentstroke}{rgb}{0.000000,0.000000,0.000000}%
\pgfsetstrokecolor{currentstroke}%
\pgfsetdash{}{0pt}%
\pgfpathmoveto{\pgfqpoint{3.028413in}{3.110581in}}%
\pgfpathlineto{\pgfqpoint{3.041490in}{3.092996in}}%
\pgfpathlineto{\pgfqpoint{3.054562in}{3.075604in}}%
\pgfpathlineto{\pgfqpoint{3.067630in}{3.058403in}}%
\pgfpathlineto{\pgfqpoint{3.080692in}{3.041393in}}%
\pgfpathlineto{\pgfqpoint{3.088565in}{3.048961in}}%
\pgfpathlineto{\pgfqpoint{3.096430in}{3.056634in}}%
\pgfpathlineto{\pgfqpoint{3.104288in}{3.064411in}}%
\pgfpathlineto{\pgfqpoint{3.112138in}{3.072291in}}%
\pgfpathlineto{\pgfqpoint{3.099097in}{3.089205in}}%
\pgfpathlineto{\pgfqpoint{3.086051in}{3.106309in}}%
\pgfpathlineto{\pgfqpoint{3.073001in}{3.123605in}}%
\pgfpathlineto{\pgfqpoint{3.059945in}{3.141093in}}%
\pgfpathlineto{\pgfqpoint{3.052074in}{3.133303in}}%
\pgfpathlineto{\pgfqpoint{3.044195in}{3.125621in}}%
\pgfpathlineto{\pgfqpoint{3.036308in}{3.118047in}}%
\pgfpathlineto{\pgfqpoint{3.028413in}{3.110581in}}%
\pgfpathclose%
\pgfusepath{fill}%
\end{pgfscope}%
\begin{pgfscope}%
\pgfpathrectangle{\pgfqpoint{1.254980in}{0.150000in}}{\pgfqpoint{5.490039in}{5.490039in}}%
\pgfusepath{clip}%
\pgfsetbuttcap%
\pgfsetroundjoin%
\definecolor{currentfill}{rgb}{0.275191,0.194905,0.496005}%
\pgfsetfillcolor{currentfill}%
\pgfsetfillopacity{0.700000}%
\pgfsetlinewidth{0.000000pt}%
\definecolor{currentstroke}{rgb}{0.000000,0.000000,0.000000}%
\pgfsetstrokecolor{currentstroke}%
\pgfsetdash{}{0pt}%
\pgfpathmoveto{\pgfqpoint{3.631765in}{2.539130in}}%
\pgfpathlineto{\pgfqpoint{3.644745in}{2.529116in}}%
\pgfpathlineto{\pgfqpoint{3.657726in}{2.519251in}}%
\pgfpathlineto{\pgfqpoint{3.670709in}{2.509537in}}%
\pgfpathlineto{\pgfqpoint{3.683694in}{2.499971in}}%
\pgfpathlineto{\pgfqpoint{3.691345in}{2.508861in}}%
\pgfpathlineto{\pgfqpoint{3.698990in}{2.517809in}}%
\pgfpathlineto{\pgfqpoint{3.706629in}{2.526815in}}%
\pgfpathlineto{\pgfqpoint{3.714263in}{2.535878in}}%
\pgfpathlineto{\pgfqpoint{3.701293in}{2.545370in}}%
\pgfpathlineto{\pgfqpoint{3.688325in}{2.555011in}}%
\pgfpathlineto{\pgfqpoint{3.675358in}{2.564801in}}%
\pgfpathlineto{\pgfqpoint{3.662393in}{2.574742in}}%
\pgfpathlineto{\pgfqpoint{3.654744in}{2.565747in}}%
\pgfpathlineto{\pgfqpoint{3.647090in}{2.556813in}}%
\pgfpathlineto{\pgfqpoint{3.639430in}{2.547940in}}%
\pgfpathlineto{\pgfqpoint{3.631765in}{2.539130in}}%
\pgfpathclose%
\pgfusepath{fill}%
\end{pgfscope}%
\begin{pgfscope}%
\pgfpathrectangle{\pgfqpoint{1.254980in}{0.150000in}}{\pgfqpoint{5.490039in}{5.490039in}}%
\pgfusepath{clip}%
\pgfsetbuttcap%
\pgfsetroundjoin%
\definecolor{currentfill}{rgb}{0.276194,0.190074,0.493001}%
\pgfsetfillcolor{currentfill}%
\pgfsetfillopacity{0.700000}%
\pgfsetlinewidth{0.000000pt}%
\definecolor{currentstroke}{rgb}{0.000000,0.000000,0.000000}%
\pgfsetstrokecolor{currentstroke}%
\pgfsetdash{}{0pt}%
\pgfpathmoveto{\pgfqpoint{4.601444in}{2.513209in}}%
\pgfpathlineto{\pgfqpoint{4.614597in}{2.510881in}}%
\pgfpathlineto{\pgfqpoint{4.627758in}{2.508674in}}%
\pgfpathlineto{\pgfqpoint{4.640927in}{2.506591in}}%
\pgfpathlineto{\pgfqpoint{4.654104in}{2.504629in}}%
\pgfpathlineto{\pgfqpoint{4.661443in}{2.514709in}}%
\pgfpathlineto{\pgfqpoint{4.668778in}{2.524799in}}%
\pgfpathlineto{\pgfqpoint{4.676109in}{2.534899in}}%
\pgfpathlineto{\pgfqpoint{4.683435in}{2.545010in}}%
\pgfpathlineto{\pgfqpoint{4.670268in}{2.547026in}}%
\pgfpathlineto{\pgfqpoint{4.657109in}{2.549165in}}%
\pgfpathlineto{\pgfqpoint{4.643958in}{2.551425in}}%
\pgfpathlineto{\pgfqpoint{4.630815in}{2.553809in}}%
\pgfpathlineto{\pgfqpoint{4.623479in}{2.543637in}}%
\pgfpathlineto{\pgfqpoint{4.616138in}{2.533481in}}%
\pgfpathlineto{\pgfqpoint{4.608793in}{2.523338in}}%
\pgfpathlineto{\pgfqpoint{4.601444in}{2.513209in}}%
\pgfpathclose%
\pgfusepath{fill}%
\end{pgfscope}%
\begin{pgfscope}%
\pgfpathrectangle{\pgfqpoint{1.254980in}{0.150000in}}{\pgfqpoint{5.490039in}{5.490039in}}%
\pgfusepath{clip}%
\pgfsetbuttcap%
\pgfsetroundjoin%
\definecolor{currentfill}{rgb}{0.281412,0.155834,0.469201}%
\pgfsetfillcolor{currentfill}%
\pgfsetfillopacity{0.700000}%
\pgfsetlinewidth{0.000000pt}%
\definecolor{currentstroke}{rgb}{0.000000,0.000000,0.000000}%
\pgfsetstrokecolor{currentstroke}%
\pgfsetdash{}{0pt}%
\pgfpathmoveto{\pgfqpoint{4.302981in}{2.445571in}}%
\pgfpathlineto{\pgfqpoint{4.316051in}{2.441303in}}%
\pgfpathlineto{\pgfqpoint{4.329128in}{2.437163in}}%
\pgfpathlineto{\pgfqpoint{4.342211in}{2.433150in}}%
\pgfpathlineto{\pgfqpoint{4.355300in}{2.429266in}}%
\pgfpathlineto{\pgfqpoint{4.362732in}{2.439287in}}%
\pgfpathlineto{\pgfqpoint{4.370160in}{2.449327in}}%
\pgfpathlineto{\pgfqpoint{4.377583in}{2.459385in}}%
\pgfpathlineto{\pgfqpoint{4.385001in}{2.469463in}}%
\pgfpathlineto{\pgfqpoint{4.371922in}{2.473355in}}%
\pgfpathlineto{\pgfqpoint{4.358849in}{2.477374in}}%
\pgfpathlineto{\pgfqpoint{4.345783in}{2.481521in}}%
\pgfpathlineto{\pgfqpoint{4.332723in}{2.485796in}}%
\pgfpathlineto{\pgfqpoint{4.325294in}{2.475706in}}%
\pgfpathlineto{\pgfqpoint{4.317861in}{2.465638in}}%
\pgfpathlineto{\pgfqpoint{4.310423in}{2.455594in}}%
\pgfpathlineto{\pgfqpoint{4.302981in}{2.445571in}}%
\pgfpathclose%
\pgfusepath{fill}%
\end{pgfscope}%
\begin{pgfscope}%
\pgfpathrectangle{\pgfqpoint{1.254980in}{0.150000in}}{\pgfqpoint{5.490039in}{5.490039in}}%
\pgfusepath{clip}%
\pgfsetbuttcap%
\pgfsetroundjoin%
\definecolor{currentfill}{rgb}{0.281887,0.150881,0.465405}%
\pgfsetfillcolor{currentfill}%
\pgfsetfillopacity{0.700000}%
\pgfsetlinewidth{0.000000pt}%
\definecolor{currentstroke}{rgb}{0.000000,0.000000,0.000000}%
\pgfsetstrokecolor{currentstroke}%
\pgfsetdash{}{0pt}%
\pgfpathmoveto{\pgfqpoint{3.952377in}{2.441445in}}%
\pgfpathlineto{\pgfqpoint{3.965381in}{2.434441in}}%
\pgfpathlineto{\pgfqpoint{3.978389in}{2.427575in}}%
\pgfpathlineto{\pgfqpoint{3.991401in}{2.420846in}}%
\pgfpathlineto{\pgfqpoint{4.004417in}{2.414254in}}%
\pgfpathlineto{\pgfqpoint{4.011960in}{2.423838in}}%
\pgfpathlineto{\pgfqpoint{4.019498in}{2.433457in}}%
\pgfpathlineto{\pgfqpoint{4.027031in}{2.443114in}}%
\pgfpathlineto{\pgfqpoint{4.034559in}{2.452806in}}%
\pgfpathlineto{\pgfqpoint{4.021555in}{2.459358in}}%
\pgfpathlineto{\pgfqpoint{4.008555in}{2.466045in}}%
\pgfpathlineto{\pgfqpoint{3.995559in}{2.472870in}}%
\pgfpathlineto{\pgfqpoint{3.982567in}{2.479833in}}%
\pgfpathlineto{\pgfqpoint{3.975027in}{2.470176in}}%
\pgfpathlineto{\pgfqpoint{3.967482in}{2.460558in}}%
\pgfpathlineto{\pgfqpoint{3.959932in}{2.450981in}}%
\pgfpathlineto{\pgfqpoint{3.952377in}{2.441445in}}%
\pgfpathclose%
\pgfusepath{fill}%
\end{pgfscope}%
\begin{pgfscope}%
\pgfpathrectangle{\pgfqpoint{1.254980in}{0.150000in}}{\pgfqpoint{5.490039in}{5.490039in}}%
\pgfusepath{clip}%
\pgfsetbuttcap%
\pgfsetroundjoin%
\definecolor{currentfill}{rgb}{0.253935,0.265254,0.529983}%
\pgfsetfillcolor{currentfill}%
\pgfsetfillopacity{0.700000}%
\pgfsetlinewidth{0.000000pt}%
\definecolor{currentstroke}{rgb}{0.000000,0.000000,0.000000}%
\pgfsetstrokecolor{currentstroke}%
\pgfsetdash{}{0pt}%
\pgfpathmoveto{\pgfqpoint{3.393173in}{2.687252in}}%
\pgfpathlineto{\pgfqpoint{3.406167in}{2.674638in}}%
\pgfpathlineto{\pgfqpoint{3.419160in}{2.662187in}}%
\pgfpathlineto{\pgfqpoint{3.432152in}{2.649898in}}%
\pgfpathlineto{\pgfqpoint{3.445144in}{2.637770in}}%
\pgfpathlineto{\pgfqpoint{3.452882in}{2.646037in}}%
\pgfpathlineto{\pgfqpoint{3.460614in}{2.654381in}}%
\pgfpathlineto{\pgfqpoint{3.468340in}{2.662800in}}%
\pgfpathlineto{\pgfqpoint{3.476059in}{2.671293in}}%
\pgfpathlineto{\pgfqpoint{3.463084in}{2.683330in}}%
\pgfpathlineto{\pgfqpoint{3.450109in}{2.695527in}}%
\pgfpathlineto{\pgfqpoint{3.437134in}{2.707887in}}%
\pgfpathlineto{\pgfqpoint{3.424157in}{2.720409in}}%
\pgfpathlineto{\pgfqpoint{3.416421in}{2.712001in}}%
\pgfpathlineto{\pgfqpoint{3.408678in}{2.703672in}}%
\pgfpathlineto{\pgfqpoint{3.400929in}{2.695422in}}%
\pgfpathlineto{\pgfqpoint{3.393173in}{2.687252in}}%
\pgfpathclose%
\pgfusepath{fill}%
\end{pgfscope}%
\begin{pgfscope}%
\pgfpathrectangle{\pgfqpoint{1.254980in}{0.150000in}}{\pgfqpoint{5.490039in}{5.490039in}}%
\pgfusepath{clip}%
\pgfsetbuttcap%
\pgfsetroundjoin%
\definecolor{currentfill}{rgb}{0.280868,0.160771,0.472899}%
\pgfsetfillcolor{currentfill}%
\pgfsetfillopacity{0.700000}%
\pgfsetlinewidth{0.000000pt}%
\definecolor{currentstroke}{rgb}{0.000000,0.000000,0.000000}%
\pgfsetstrokecolor{currentstroke}%
\pgfsetdash{}{0pt}%
\pgfpathmoveto{\pgfqpoint{3.818098in}{2.465217in}}%
\pgfpathlineto{\pgfqpoint{3.831088in}{2.457034in}}%
\pgfpathlineto{\pgfqpoint{3.844082in}{2.448994in}}%
\pgfpathlineto{\pgfqpoint{3.857078in}{2.441095in}}%
\pgfpathlineto{\pgfqpoint{3.870077in}{2.433337in}}%
\pgfpathlineto{\pgfqpoint{3.877665in}{2.442648in}}%
\pgfpathlineto{\pgfqpoint{3.885248in}{2.452003in}}%
\pgfpathlineto{\pgfqpoint{3.892825in}{2.461403in}}%
\pgfpathlineto{\pgfqpoint{3.900397in}{2.470848in}}%
\pgfpathlineto{\pgfqpoint{3.887411in}{2.478549in}}%
\pgfpathlineto{\pgfqpoint{3.874428in}{2.486390in}}%
\pgfpathlineto{\pgfqpoint{3.861447in}{2.494374in}}%
\pgfpathlineto{\pgfqpoint{3.848470in}{2.502499in}}%
\pgfpathlineto{\pgfqpoint{3.840885in}{2.493106in}}%
\pgfpathlineto{\pgfqpoint{3.833295in}{2.483761in}}%
\pgfpathlineto{\pgfqpoint{3.825699in}{2.474464in}}%
\pgfpathlineto{\pgfqpoint{3.818098in}{2.465217in}}%
\pgfpathclose%
\pgfusepath{fill}%
\end{pgfscope}%
\begin{pgfscope}%
\pgfpathrectangle{\pgfqpoint{1.254980in}{0.150000in}}{\pgfqpoint{5.490039in}{5.490039in}}%
\pgfusepath{clip}%
\pgfsetbuttcap%
\pgfsetroundjoin%
\definecolor{currentfill}{rgb}{0.165117,0.467423,0.558141}%
\pgfsetfillcolor{currentfill}%
\pgfsetfillopacity{0.700000}%
\pgfsetlinewidth{0.000000pt}%
\definecolor{currentstroke}{rgb}{0.000000,0.000000,0.000000}%
\pgfsetstrokecolor{currentstroke}%
\pgfsetdash{}{0pt}%
\pgfpathmoveto{\pgfqpoint{2.976051in}{3.182875in}}%
\pgfpathlineto{\pgfqpoint{2.989150in}{3.164506in}}%
\pgfpathlineto{\pgfqpoint{3.002243in}{3.146335in}}%
\pgfpathlineto{\pgfqpoint{3.015331in}{3.128360in}}%
\pgfpathlineto{\pgfqpoint{3.028413in}{3.110581in}}%
\pgfpathlineto{\pgfqpoint{3.036308in}{3.118047in}}%
\pgfpathlineto{\pgfqpoint{3.044195in}{3.125621in}}%
\pgfpathlineto{\pgfqpoint{3.052074in}{3.133303in}}%
\pgfpathlineto{\pgfqpoint{3.059945in}{3.141093in}}%
\pgfpathlineto{\pgfqpoint{3.046885in}{3.158775in}}%
\pgfpathlineto{\pgfqpoint{3.033819in}{3.176651in}}%
\pgfpathlineto{\pgfqpoint{3.020748in}{3.194725in}}%
\pgfpathlineto{\pgfqpoint{3.007671in}{3.212996in}}%
\pgfpathlineto{\pgfqpoint{2.999778in}{3.205297in}}%
\pgfpathlineto{\pgfqpoint{2.991877in}{3.197711in}}%
\pgfpathlineto{\pgfqpoint{2.983968in}{3.190236in}}%
\pgfpathlineto{\pgfqpoint{2.976051in}{3.182875in}}%
\pgfpathclose%
\pgfusepath{fill}%
\end{pgfscope}%
\begin{pgfscope}%
\pgfpathrectangle{\pgfqpoint{1.254980in}{0.150000in}}{\pgfqpoint{5.490039in}{5.490039in}}%
\pgfusepath{clip}%
\pgfsetbuttcap%
\pgfsetroundjoin%
\definecolor{currentfill}{rgb}{0.199430,0.387607,0.554642}%
\pgfsetfillcolor{currentfill}%
\pgfsetfillopacity{0.700000}%
\pgfsetlinewidth{0.000000pt}%
\definecolor{currentstroke}{rgb}{0.000000,0.000000,0.000000}%
\pgfsetstrokecolor{currentstroke}%
\pgfsetdash{}{0pt}%
\pgfpathmoveto{\pgfqpoint{5.557031in}{2.937615in}}%
\pgfpathlineto{\pgfqpoint{5.570539in}{2.939399in}}%
\pgfpathlineto{\pgfqpoint{5.584059in}{2.941295in}}%
\pgfpathlineto{\pgfqpoint{5.597592in}{2.943303in}}%
\pgfpathlineto{\pgfqpoint{5.611136in}{2.945421in}}%
\pgfpathlineto{\pgfqpoint{5.618150in}{2.954435in}}%
\pgfpathlineto{\pgfqpoint{5.625160in}{2.963479in}}%
\pgfpathlineto{\pgfqpoint{5.632166in}{2.972555in}}%
\pgfpathlineto{\pgfqpoint{5.639169in}{2.981666in}}%
\pgfpathlineto{\pgfqpoint{5.625640in}{2.979779in}}%
\pgfpathlineto{\pgfqpoint{5.612124in}{2.978003in}}%
\pgfpathlineto{\pgfqpoint{5.598619in}{2.976338in}}%
\pgfpathlineto{\pgfqpoint{5.585127in}{2.974785in}}%
\pgfpathlineto{\pgfqpoint{5.578108in}{2.965437in}}%
\pgfpathlineto{\pgfqpoint{5.571086in}{2.956128in}}%
\pgfpathlineto{\pgfqpoint{5.564060in}{2.946854in}}%
\pgfpathlineto{\pgfqpoint{5.557031in}{2.937615in}}%
\pgfpathclose%
\pgfusepath{fill}%
\end{pgfscope}%
\begin{pgfscope}%
\pgfpathrectangle{\pgfqpoint{1.254980in}{0.150000in}}{\pgfqpoint{5.490039in}{5.490039in}}%
\pgfusepath{clip}%
\pgfsetbuttcap%
\pgfsetroundjoin%
\definecolor{currentfill}{rgb}{0.206756,0.371758,0.553117}%
\pgfsetfillcolor{currentfill}%
\pgfsetfillopacity{0.700000}%
\pgfsetlinewidth{0.000000pt}%
\definecolor{currentstroke}{rgb}{0.000000,0.000000,0.000000}%
\pgfsetstrokecolor{currentstroke}%
\pgfsetdash{}{0pt}%
\pgfpathmoveto{\pgfqpoint{5.474906in}{2.894066in}}%
\pgfpathlineto{\pgfqpoint{5.488382in}{2.895618in}}%
\pgfpathlineto{\pgfqpoint{5.501869in}{2.897283in}}%
\pgfpathlineto{\pgfqpoint{5.515368in}{2.899059in}}%
\pgfpathlineto{\pgfqpoint{5.528879in}{2.900947in}}%
\pgfpathlineto{\pgfqpoint{5.535922in}{2.910075in}}%
\pgfpathlineto{\pgfqpoint{5.542962in}{2.919227in}}%
\pgfpathlineto{\pgfqpoint{5.549998in}{2.928406in}}%
\pgfpathlineto{\pgfqpoint{5.557031in}{2.937615in}}%
\pgfpathlineto{\pgfqpoint{5.543535in}{2.935942in}}%
\pgfpathlineto{\pgfqpoint{5.530051in}{2.934380in}}%
\pgfpathlineto{\pgfqpoint{5.516578in}{2.932931in}}%
\pgfpathlineto{\pgfqpoint{5.503118in}{2.931594in}}%
\pgfpathlineto{\pgfqpoint{5.496070in}{2.922164in}}%
\pgfpathlineto{\pgfqpoint{5.489019in}{2.912768in}}%
\pgfpathlineto{\pgfqpoint{5.481965in}{2.903403in}}%
\pgfpathlineto{\pgfqpoint{5.474906in}{2.894066in}}%
\pgfpathclose%
\pgfusepath{fill}%
\end{pgfscope}%
\begin{pgfscope}%
\pgfpathrectangle{\pgfqpoint{1.254980in}{0.150000in}}{\pgfqpoint{5.490039in}{5.490039in}}%
\pgfusepath{clip}%
\pgfsetbuttcap%
\pgfsetroundjoin%
\definecolor{currentfill}{rgb}{0.282290,0.145912,0.461510}%
\pgfsetfillcolor{currentfill}%
\pgfsetfillopacity{0.700000}%
\pgfsetlinewidth{0.000000pt}%
\definecolor{currentstroke}{rgb}{0.000000,0.000000,0.000000}%
\pgfsetstrokecolor{currentstroke}%
\pgfsetdash{}{0pt}%
\pgfpathmoveto{\pgfqpoint{4.086619in}{2.427957in}}%
\pgfpathlineto{\pgfqpoint{4.099645in}{2.422082in}}%
\pgfpathlineto{\pgfqpoint{4.112677in}{2.416339in}}%
\pgfpathlineto{\pgfqpoint{4.125713in}{2.410731in}}%
\pgfpathlineto{\pgfqpoint{4.138754in}{2.405254in}}%
\pgfpathlineto{\pgfqpoint{4.146255in}{2.415045in}}%
\pgfpathlineto{\pgfqpoint{4.153751in}{2.424865in}}%
\pgfpathlineto{\pgfqpoint{4.161242in}{2.434713in}}%
\pgfpathlineto{\pgfqpoint{4.168728in}{2.444590in}}%
\pgfpathlineto{\pgfqpoint{4.155698in}{2.450041in}}%
\pgfpathlineto{\pgfqpoint{4.142673in}{2.455625in}}%
\pgfpathlineto{\pgfqpoint{4.129653in}{2.461342in}}%
\pgfpathlineto{\pgfqpoint{4.116637in}{2.467193in}}%
\pgfpathlineto{\pgfqpoint{4.109140in}{2.457335in}}%
\pgfpathlineto{\pgfqpoint{4.101638in}{2.447510in}}%
\pgfpathlineto{\pgfqpoint{4.094131in}{2.437717in}}%
\pgfpathlineto{\pgfqpoint{4.086619in}{2.427957in}}%
\pgfpathclose%
\pgfusepath{fill}%
\end{pgfscope}%
\begin{pgfscope}%
\pgfpathrectangle{\pgfqpoint{1.254980in}{0.150000in}}{\pgfqpoint{5.490039in}{5.490039in}}%
\pgfusepath{clip}%
\pgfsetbuttcap%
\pgfsetroundjoin%
\definecolor{currentfill}{rgb}{0.216210,0.351535,0.550627}%
\pgfsetfillcolor{currentfill}%
\pgfsetfillopacity{0.700000}%
\pgfsetlinewidth{0.000000pt}%
\definecolor{currentstroke}{rgb}{0.000000,0.000000,0.000000}%
\pgfsetstrokecolor{currentstroke}%
\pgfsetdash{}{0pt}%
\pgfpathmoveto{\pgfqpoint{5.392795in}{2.851082in}}%
\pgfpathlineto{\pgfqpoint{5.406238in}{2.852383in}}%
\pgfpathlineto{\pgfqpoint{5.419692in}{2.853797in}}%
\pgfpathlineto{\pgfqpoint{5.433158in}{2.855323in}}%
\pgfpathlineto{\pgfqpoint{5.446636in}{2.856962in}}%
\pgfpathlineto{\pgfqpoint{5.453709in}{2.866206in}}%
\pgfpathlineto{\pgfqpoint{5.460779in}{2.875469in}}%
\pgfpathlineto{\pgfqpoint{5.467845in}{2.884755in}}%
\pgfpathlineto{\pgfqpoint{5.474906in}{2.894066in}}%
\pgfpathlineto{\pgfqpoint{5.461443in}{2.892626in}}%
\pgfpathlineto{\pgfqpoint{5.447991in}{2.891298in}}%
\pgfpathlineto{\pgfqpoint{5.434550in}{2.890084in}}%
\pgfpathlineto{\pgfqpoint{5.421121in}{2.888981in}}%
\pgfpathlineto{\pgfqpoint{5.414045in}{2.879466in}}%
\pgfpathlineto{\pgfqpoint{5.406966in}{2.869979in}}%
\pgfpathlineto{\pgfqpoint{5.399882in}{2.860519in}}%
\pgfpathlineto{\pgfqpoint{5.392795in}{2.851082in}}%
\pgfpathclose%
\pgfusepath{fill}%
\end{pgfscope}%
\begin{pgfscope}%
\pgfpathrectangle{\pgfqpoint{1.254980in}{0.150000in}}{\pgfqpoint{5.490039in}{5.490039in}}%
\pgfusepath{clip}%
\pgfsetbuttcap%
\pgfsetroundjoin%
\definecolor{currentfill}{rgb}{0.223925,0.334994,0.548053}%
\pgfsetfillcolor{currentfill}%
\pgfsetfillopacity{0.700000}%
\pgfsetlinewidth{0.000000pt}%
\definecolor{currentstroke}{rgb}{0.000000,0.000000,0.000000}%
\pgfsetstrokecolor{currentstroke}%
\pgfsetdash{}{0pt}%
\pgfpathmoveto{\pgfqpoint{5.310696in}{2.808735in}}%
\pgfpathlineto{\pgfqpoint{5.324107in}{2.809765in}}%
\pgfpathlineto{\pgfqpoint{5.337529in}{2.810909in}}%
\pgfpathlineto{\pgfqpoint{5.350962in}{2.812167in}}%
\pgfpathlineto{\pgfqpoint{5.364407in}{2.813537in}}%
\pgfpathlineto{\pgfqpoint{5.371510in}{2.822897in}}%
\pgfpathlineto{\pgfqpoint{5.378609in}{2.832273in}}%
\pgfpathlineto{\pgfqpoint{5.385704in}{2.841667in}}%
\pgfpathlineto{\pgfqpoint{5.392795in}{2.851082in}}%
\pgfpathlineto{\pgfqpoint{5.379364in}{2.849894in}}%
\pgfpathlineto{\pgfqpoint{5.365943in}{2.848820in}}%
\pgfpathlineto{\pgfqpoint{5.352535in}{2.847858in}}%
\pgfpathlineto{\pgfqpoint{5.339137in}{2.847011in}}%
\pgfpathlineto{\pgfqpoint{5.332032in}{2.837407in}}%
\pgfpathlineto{\pgfqpoint{5.324924in}{2.827828in}}%
\pgfpathlineto{\pgfqpoint{5.317812in}{2.818271in}}%
\pgfpathlineto{\pgfqpoint{5.310696in}{2.808735in}}%
\pgfpathclose%
\pgfusepath{fill}%
\end{pgfscope}%
\begin{pgfscope}%
\pgfpathrectangle{\pgfqpoint{1.254980in}{0.150000in}}{\pgfqpoint{5.490039in}{5.490039in}}%
\pgfusepath{clip}%
\pgfsetbuttcap%
\pgfsetroundjoin%
\definecolor{currentfill}{rgb}{0.233603,0.313828,0.543914}%
\pgfsetfillcolor{currentfill}%
\pgfsetfillopacity{0.700000}%
\pgfsetlinewidth{0.000000pt}%
\definecolor{currentstroke}{rgb}{0.000000,0.000000,0.000000}%
\pgfsetstrokecolor{currentstroke}%
\pgfsetdash{}{0pt}%
\pgfpathmoveto{\pgfqpoint{5.228607in}{2.767105in}}%
\pgfpathlineto{\pgfqpoint{5.241987in}{2.767846in}}%
\pgfpathlineto{\pgfqpoint{5.255377in}{2.768701in}}%
\pgfpathlineto{\pgfqpoint{5.268778in}{2.769670in}}%
\pgfpathlineto{\pgfqpoint{5.282191in}{2.770754in}}%
\pgfpathlineto{\pgfqpoint{5.289323in}{2.780228in}}%
\pgfpathlineto{\pgfqpoint{5.296451in}{2.789715in}}%
\pgfpathlineto{\pgfqpoint{5.303576in}{2.799216in}}%
\pgfpathlineto{\pgfqpoint{5.310696in}{2.808735in}}%
\pgfpathlineto{\pgfqpoint{5.297296in}{2.807818in}}%
\pgfpathlineto{\pgfqpoint{5.283907in}{2.807016in}}%
\pgfpathlineto{\pgfqpoint{5.270529in}{2.806327in}}%
\pgfpathlineto{\pgfqpoint{5.257162in}{2.805753in}}%
\pgfpathlineto{\pgfqpoint{5.250029in}{2.796063in}}%
\pgfpathlineto{\pgfqpoint{5.242893in}{2.786392in}}%
\pgfpathlineto{\pgfqpoint{5.235752in}{2.776740in}}%
\pgfpathlineto{\pgfqpoint{5.228607in}{2.767105in}}%
\pgfpathclose%
\pgfusepath{fill}%
\end{pgfscope}%
\begin{pgfscope}%
\pgfpathrectangle{\pgfqpoint{1.254980in}{0.150000in}}{\pgfqpoint{5.490039in}{5.490039in}}%
\pgfusepath{clip}%
\pgfsetbuttcap%
\pgfsetroundjoin%
\definecolor{currentfill}{rgb}{0.278826,0.175490,0.483397}%
\pgfsetfillcolor{currentfill}%
\pgfsetfillopacity{0.700000}%
\pgfsetlinewidth{0.000000pt}%
\definecolor{currentstroke}{rgb}{0.000000,0.000000,0.000000}%
\pgfsetstrokecolor{currentstroke}%
\pgfsetdash{}{0pt}%
\pgfpathmoveto{\pgfqpoint{4.519430in}{2.483204in}}%
\pgfpathlineto{\pgfqpoint{4.532562in}{2.480421in}}%
\pgfpathlineto{\pgfqpoint{4.545702in}{2.477761in}}%
\pgfpathlineto{\pgfqpoint{4.558849in}{2.475225in}}%
\pgfpathlineto{\pgfqpoint{4.572004in}{2.472812in}}%
\pgfpathlineto{\pgfqpoint{4.579370in}{2.482895in}}%
\pgfpathlineto{\pgfqpoint{4.586733in}{2.492988in}}%
\pgfpathlineto{\pgfqpoint{4.594091in}{2.503093in}}%
\pgfpathlineto{\pgfqpoint{4.601444in}{2.513209in}}%
\pgfpathlineto{\pgfqpoint{4.588299in}{2.515661in}}%
\pgfpathlineto{\pgfqpoint{4.575162in}{2.518236in}}%
\pgfpathlineto{\pgfqpoint{4.562032in}{2.520934in}}%
\pgfpathlineto{\pgfqpoint{4.548910in}{2.523757in}}%
\pgfpathlineto{\pgfqpoint{4.541546in}{2.513596in}}%
\pgfpathlineto{\pgfqpoint{4.534179in}{2.503450in}}%
\pgfpathlineto{\pgfqpoint{4.526807in}{2.493320in}}%
\pgfpathlineto{\pgfqpoint{4.519430in}{2.483204in}}%
\pgfpathclose%
\pgfusepath{fill}%
\end{pgfscope}%
\begin{pgfscope}%
\pgfpathrectangle{\pgfqpoint{1.254980in}{0.150000in}}{\pgfqpoint{5.490039in}{5.490039in}}%
\pgfusepath{clip}%
\pgfsetbuttcap%
\pgfsetroundjoin%
\definecolor{currentfill}{rgb}{0.241237,0.296485,0.539709}%
\pgfsetfillcolor{currentfill}%
\pgfsetfillopacity{0.700000}%
\pgfsetlinewidth{0.000000pt}%
\definecolor{currentstroke}{rgb}{0.000000,0.000000,0.000000}%
\pgfsetstrokecolor{currentstroke}%
\pgfsetdash{}{0pt}%
\pgfpathmoveto{\pgfqpoint{5.146528in}{2.726285in}}%
\pgfpathlineto{\pgfqpoint{5.159877in}{2.726716in}}%
\pgfpathlineto{\pgfqpoint{5.173236in}{2.727263in}}%
\pgfpathlineto{\pgfqpoint{5.186606in}{2.727925in}}%
\pgfpathlineto{\pgfqpoint{5.199986in}{2.728701in}}%
\pgfpathlineto{\pgfqpoint{5.207148in}{2.738285in}}%
\pgfpathlineto{\pgfqpoint{5.214305in}{2.747880in}}%
\pgfpathlineto{\pgfqpoint{5.221458in}{2.757486in}}%
\pgfpathlineto{\pgfqpoint{5.228607in}{2.767105in}}%
\pgfpathlineto{\pgfqpoint{5.215239in}{2.766480in}}%
\pgfpathlineto{\pgfqpoint{5.201881in}{2.765969in}}%
\pgfpathlineto{\pgfqpoint{5.188533in}{2.765573in}}%
\pgfpathlineto{\pgfqpoint{5.175196in}{2.765292in}}%
\pgfpathlineto{\pgfqpoint{5.168035in}{2.755516in}}%
\pgfpathlineto{\pgfqpoint{5.160870in}{2.745757in}}%
\pgfpathlineto{\pgfqpoint{5.153701in}{2.736014in}}%
\pgfpathlineto{\pgfqpoint{5.146528in}{2.726285in}}%
\pgfpathclose%
\pgfusepath{fill}%
\end{pgfscope}%
\begin{pgfscope}%
\pgfpathrectangle{\pgfqpoint{1.254980in}{0.150000in}}{\pgfqpoint{5.490039in}{5.490039in}}%
\pgfusepath{clip}%
\pgfsetbuttcap%
\pgfsetroundjoin%
\definecolor{currentfill}{rgb}{0.190631,0.407061,0.556089}%
\pgfsetfillcolor{currentfill}%
\pgfsetfillopacity{0.700000}%
\pgfsetlinewidth{0.000000pt}%
\definecolor{currentstroke}{rgb}{0.000000,0.000000,0.000000}%
\pgfsetstrokecolor{currentstroke}%
\pgfsetdash{}{0pt}%
\pgfpathmoveto{\pgfqpoint{5.639169in}{2.981666in}}%
\pgfpathlineto{\pgfqpoint{5.652710in}{2.983664in}}%
\pgfpathlineto{\pgfqpoint{5.666264in}{2.985773in}}%
\pgfpathlineto{\pgfqpoint{5.679830in}{2.987992in}}%
\pgfpathlineto{\pgfqpoint{5.693408in}{2.990322in}}%
\pgfpathlineto{\pgfqpoint{5.700392in}{2.999228in}}%
\pgfpathlineto{\pgfqpoint{5.707372in}{3.008169in}}%
\pgfpathlineto{\pgfqpoint{5.714349in}{3.017148in}}%
\pgfpathlineto{\pgfqpoint{5.700783in}{3.015002in}}%
\pgfpathlineto{\pgfqpoint{5.687229in}{3.012967in}}%
\pgfpathlineto{\pgfqpoint{5.673688in}{3.011042in}}%
\pgfpathlineto{\pgfqpoint{5.660159in}{3.009229in}}%
\pgfpathlineto{\pgfqpoint{5.653166in}{3.000000in}}%
\pgfpathlineto{\pgfqpoint{5.646169in}{2.990813in}}%
\pgfpathlineto{\pgfqpoint{5.639169in}{2.981666in}}%
\pgfpathclose%
\pgfusepath{fill}%
\end{pgfscope}%
\begin{pgfscope}%
\pgfpathrectangle{\pgfqpoint{1.254980in}{0.150000in}}{\pgfqpoint{5.490039in}{5.490039in}}%
\pgfusepath{clip}%
\pgfsetbuttcap%
\pgfsetroundjoin%
\definecolor{currentfill}{rgb}{0.262138,0.242286,0.520837}%
\pgfsetfillcolor{currentfill}%
\pgfsetfillopacity{0.700000}%
\pgfsetlinewidth{0.000000pt}%
\definecolor{currentstroke}{rgb}{0.000000,0.000000,0.000000}%
\pgfsetstrokecolor{currentstroke}%
\pgfsetdash{}{0pt}%
\pgfpathmoveto{\pgfqpoint{3.445144in}{2.637770in}}%
\pgfpathlineto{\pgfqpoint{3.458135in}{2.625802in}}%
\pgfpathlineto{\pgfqpoint{3.471126in}{2.613994in}}%
\pgfpathlineto{\pgfqpoint{3.484116in}{2.602345in}}%
\pgfpathlineto{\pgfqpoint{3.497107in}{2.590854in}}%
\pgfpathlineto{\pgfqpoint{3.504828in}{2.599219in}}%
\pgfpathlineto{\pgfqpoint{3.512543in}{2.607656in}}%
\pgfpathlineto{\pgfqpoint{3.520252in}{2.616164in}}%
\pgfpathlineto{\pgfqpoint{3.527955in}{2.624743in}}%
\pgfpathlineto{\pgfqpoint{3.514981in}{2.636143in}}%
\pgfpathlineto{\pgfqpoint{3.502007in}{2.647701in}}%
\pgfpathlineto{\pgfqpoint{3.489033in}{2.659417in}}%
\pgfpathlineto{\pgfqpoint{3.476059in}{2.671293in}}%
\pgfpathlineto{\pgfqpoint{3.468340in}{2.662800in}}%
\pgfpathlineto{\pgfqpoint{3.460614in}{2.654381in}}%
\pgfpathlineto{\pgfqpoint{3.452882in}{2.646037in}}%
\pgfpathlineto{\pgfqpoint{3.445144in}{2.637770in}}%
\pgfpathclose%
\pgfusepath{fill}%
\end{pgfscope}%
\begin{pgfscope}%
\pgfpathrectangle{\pgfqpoint{1.254980in}{0.150000in}}{\pgfqpoint{5.490039in}{5.490039in}}%
\pgfusepath{clip}%
\pgfsetbuttcap%
\pgfsetroundjoin%
\definecolor{currentfill}{rgb}{0.248629,0.278775,0.534556}%
\pgfsetfillcolor{currentfill}%
\pgfsetfillopacity{0.700000}%
\pgfsetlinewidth{0.000000pt}%
\definecolor{currentstroke}{rgb}{0.000000,0.000000,0.000000}%
\pgfsetstrokecolor{currentstroke}%
\pgfsetdash{}{0pt}%
\pgfpathmoveto{\pgfqpoint{5.064456in}{2.686374in}}%
\pgfpathlineto{\pgfqpoint{5.077775in}{2.686476in}}%
\pgfpathlineto{\pgfqpoint{5.091104in}{2.686695in}}%
\pgfpathlineto{\pgfqpoint{5.104443in}{2.687029in}}%
\pgfpathlineto{\pgfqpoint{5.117793in}{2.687480in}}%
\pgfpathlineto{\pgfqpoint{5.124983in}{2.697168in}}%
\pgfpathlineto{\pgfqpoint{5.132169in}{2.706863in}}%
\pgfpathlineto{\pgfqpoint{5.139351in}{2.716569in}}%
\pgfpathlineto{\pgfqpoint{5.146528in}{2.726285in}}%
\pgfpathlineto{\pgfqpoint{5.133190in}{2.725969in}}%
\pgfpathlineto{\pgfqpoint{5.119862in}{2.725769in}}%
\pgfpathlineto{\pgfqpoint{5.106544in}{2.725685in}}%
\pgfpathlineto{\pgfqpoint{5.093237in}{2.725717in}}%
\pgfpathlineto{\pgfqpoint{5.086048in}{2.715860in}}%
\pgfpathlineto{\pgfqpoint{5.078855in}{2.706018in}}%
\pgfpathlineto{\pgfqpoint{5.071658in}{2.696190in}}%
\pgfpathlineto{\pgfqpoint{5.064456in}{2.686374in}}%
\pgfpathclose%
\pgfusepath{fill}%
\end{pgfscope}%
\begin{pgfscope}%
\pgfpathrectangle{\pgfqpoint{1.254980in}{0.150000in}}{\pgfqpoint{5.490039in}{5.490039in}}%
\pgfusepath{clip}%
\pgfsetbuttcap%
\pgfsetroundjoin%
\definecolor{currentfill}{rgb}{0.153364,0.497000,0.557724}%
\pgfsetfillcolor{currentfill}%
\pgfsetfillopacity{0.700000}%
\pgfsetlinewidth{0.000000pt}%
\definecolor{currentstroke}{rgb}{0.000000,0.000000,0.000000}%
\pgfsetstrokecolor{currentstroke}%
\pgfsetdash{}{0pt}%
\pgfpathmoveto{\pgfqpoint{2.923594in}{3.258357in}}%
\pgfpathlineto{\pgfqpoint{2.936717in}{3.239183in}}%
\pgfpathlineto{\pgfqpoint{2.949835in}{3.220212in}}%
\pgfpathlineto{\pgfqpoint{2.962946in}{3.201443in}}%
\pgfpathlineto{\pgfqpoint{2.976051in}{3.182875in}}%
\pgfpathlineto{\pgfqpoint{2.983968in}{3.190236in}}%
\pgfpathlineto{\pgfqpoint{2.991877in}{3.197711in}}%
\pgfpathlineto{\pgfqpoint{2.999778in}{3.205297in}}%
\pgfpathlineto{\pgfqpoint{3.007671in}{3.212996in}}%
\pgfpathlineto{\pgfqpoint{2.994588in}{3.231466in}}%
\pgfpathlineto{\pgfqpoint{2.981500in}{3.250136in}}%
\pgfpathlineto{\pgfqpoint{2.968405in}{3.269008in}}%
\pgfpathlineto{\pgfqpoint{2.955305in}{3.288084in}}%
\pgfpathlineto{\pgfqpoint{2.947389in}{3.280478in}}%
\pgfpathlineto{\pgfqpoint{2.939466in}{3.272988in}}%
\pgfpathlineto{\pgfqpoint{2.931534in}{3.265614in}}%
\pgfpathlineto{\pgfqpoint{2.923594in}{3.258357in}}%
\pgfpathclose%
\pgfusepath{fill}%
\end{pgfscope}%
\begin{pgfscope}%
\pgfpathrectangle{\pgfqpoint{1.254980in}{0.150000in}}{\pgfqpoint{5.490039in}{5.490039in}}%
\pgfusepath{clip}%
\pgfsetbuttcap%
\pgfsetroundjoin%
\definecolor{currentfill}{rgb}{0.282290,0.145912,0.461510}%
\pgfsetfillcolor{currentfill}%
\pgfsetfillopacity{0.700000}%
\pgfsetlinewidth{0.000000pt}%
\definecolor{currentstroke}{rgb}{0.000000,0.000000,0.000000}%
\pgfsetstrokecolor{currentstroke}%
\pgfsetdash{}{0pt}%
\pgfpathmoveto{\pgfqpoint{4.220902in}{2.424101in}}%
\pgfpathlineto{\pgfqpoint{4.233959in}{2.419307in}}%
\pgfpathlineto{\pgfqpoint{4.247022in}{2.414642in}}%
\pgfpathlineto{\pgfqpoint{4.260091in}{2.410107in}}%
\pgfpathlineto{\pgfqpoint{4.273166in}{2.405701in}}%
\pgfpathlineto{\pgfqpoint{4.280626in}{2.415636in}}%
\pgfpathlineto{\pgfqpoint{4.288082in}{2.425593in}}%
\pgfpathlineto{\pgfqpoint{4.295534in}{2.435571in}}%
\pgfpathlineto{\pgfqpoint{4.302981in}{2.445571in}}%
\pgfpathlineto{\pgfqpoint{4.289916in}{2.449969in}}%
\pgfpathlineto{\pgfqpoint{4.276858in}{2.454495in}}%
\pgfpathlineto{\pgfqpoint{4.263805in}{2.459151in}}%
\pgfpathlineto{\pgfqpoint{4.250759in}{2.463937in}}%
\pgfpathlineto{\pgfqpoint{4.243302in}{2.453940in}}%
\pgfpathlineto{\pgfqpoint{4.235840in}{2.443968in}}%
\pgfpathlineto{\pgfqpoint{4.228373in}{2.434022in}}%
\pgfpathlineto{\pgfqpoint{4.220902in}{2.424101in}}%
\pgfpathclose%
\pgfusepath{fill}%
\end{pgfscope}%
\begin{pgfscope}%
\pgfpathrectangle{\pgfqpoint{1.254980in}{0.150000in}}{\pgfqpoint{5.490039in}{5.490039in}}%
\pgfusepath{clip}%
\pgfsetbuttcap%
\pgfsetroundjoin%
\definecolor{currentfill}{rgb}{0.278012,0.180367,0.486697}%
\pgfsetfillcolor{currentfill}%
\pgfsetfillopacity{0.700000}%
\pgfsetlinewidth{0.000000pt}%
\definecolor{currentstroke}{rgb}{0.000000,0.000000,0.000000}%
\pgfsetstrokecolor{currentstroke}%
\pgfsetdash{}{0pt}%
\pgfpathmoveto{\pgfqpoint{3.683694in}{2.499971in}}%
\pgfpathlineto{\pgfqpoint{3.696680in}{2.490553in}}%
\pgfpathlineto{\pgfqpoint{3.709668in}{2.481283in}}%
\pgfpathlineto{\pgfqpoint{3.722658in}{2.472160in}}%
\pgfpathlineto{\pgfqpoint{3.735650in}{2.463182in}}%
\pgfpathlineto{\pgfqpoint{3.743286in}{2.472152in}}%
\pgfpathlineto{\pgfqpoint{3.750917in}{2.481176in}}%
\pgfpathlineto{\pgfqpoint{3.758543in}{2.490253in}}%
\pgfpathlineto{\pgfqpoint{3.766162in}{2.499384in}}%
\pgfpathlineto{\pgfqpoint{3.753184in}{2.508288in}}%
\pgfpathlineto{\pgfqpoint{3.740209in}{2.517338in}}%
\pgfpathlineto{\pgfqpoint{3.727235in}{2.526534in}}%
\pgfpathlineto{\pgfqpoint{3.714263in}{2.535878in}}%
\pgfpathlineto{\pgfqpoint{3.706629in}{2.526815in}}%
\pgfpathlineto{\pgfqpoint{3.698990in}{2.517809in}}%
\pgfpathlineto{\pgfqpoint{3.691345in}{2.508861in}}%
\pgfpathlineto{\pgfqpoint{3.683694in}{2.499971in}}%
\pgfpathclose%
\pgfusepath{fill}%
\end{pgfscope}%
\begin{pgfscope}%
\pgfpathrectangle{\pgfqpoint{1.254980in}{0.150000in}}{\pgfqpoint{5.490039in}{5.490039in}}%
\pgfusepath{clip}%
\pgfsetbuttcap%
\pgfsetroundjoin%
\definecolor{currentfill}{rgb}{0.255645,0.260703,0.528312}%
\pgfsetfillcolor{currentfill}%
\pgfsetfillopacity{0.700000}%
\pgfsetlinewidth{0.000000pt}%
\definecolor{currentstroke}{rgb}{0.000000,0.000000,0.000000}%
\pgfsetstrokecolor{currentstroke}%
\pgfsetdash{}{0pt}%
\pgfpathmoveto{\pgfqpoint{4.982390in}{2.647481in}}%
\pgfpathlineto{\pgfqpoint{4.995679in}{2.647235in}}%
\pgfpathlineto{\pgfqpoint{5.008979in}{2.647106in}}%
\pgfpathlineto{\pgfqpoint{5.022288in}{2.647094in}}%
\pgfpathlineto{\pgfqpoint{5.035608in}{2.647199in}}%
\pgfpathlineto{\pgfqpoint{5.042827in}{2.656982in}}%
\pgfpathlineto{\pgfqpoint{5.050041in}{2.666771in}}%
\pgfpathlineto{\pgfqpoint{5.057251in}{2.676568in}}%
\pgfpathlineto{\pgfqpoint{5.064456in}{2.686374in}}%
\pgfpathlineto{\pgfqpoint{5.051148in}{2.686387in}}%
\pgfpathlineto{\pgfqpoint{5.037849in}{2.686518in}}%
\pgfpathlineto{\pgfqpoint{5.024560in}{2.686765in}}%
\pgfpathlineto{\pgfqpoint{5.011282in}{2.687130in}}%
\pgfpathlineto{\pgfqpoint{5.004065in}{2.677199in}}%
\pgfpathlineto{\pgfqpoint{4.996844in}{2.667282in}}%
\pgfpathlineto{\pgfqpoint{4.989619in}{2.657376in}}%
\pgfpathlineto{\pgfqpoint{4.982390in}{2.647481in}}%
\pgfpathclose%
\pgfusepath{fill}%
\end{pgfscope}%
\begin{pgfscope}%
\pgfpathrectangle{\pgfqpoint{1.254980in}{0.150000in}}{\pgfqpoint{5.490039in}{5.490039in}}%
\pgfusepath{clip}%
\pgfsetbuttcap%
\pgfsetroundjoin%
\definecolor{currentfill}{rgb}{0.262138,0.242286,0.520837}%
\pgfsetfillcolor{currentfill}%
\pgfsetfillopacity{0.700000}%
\pgfsetlinewidth{0.000000pt}%
\definecolor{currentstroke}{rgb}{0.000000,0.000000,0.000000}%
\pgfsetstrokecolor{currentstroke}%
\pgfsetdash{}{0pt}%
\pgfpathmoveto{\pgfqpoint{4.900325in}{2.609726in}}%
\pgfpathlineto{\pgfqpoint{4.913587in}{2.609112in}}%
\pgfpathlineto{\pgfqpoint{4.926858in}{2.608616in}}%
\pgfpathlineto{\pgfqpoint{4.940139in}{2.608238in}}%
\pgfpathlineto{\pgfqpoint{4.953429in}{2.607977in}}%
\pgfpathlineto{\pgfqpoint{4.960676in}{2.617844in}}%
\pgfpathlineto{\pgfqpoint{4.967918in}{2.627716in}}%
\pgfpathlineto{\pgfqpoint{4.975156in}{2.637595in}}%
\pgfpathlineto{\pgfqpoint{4.982390in}{2.647481in}}%
\pgfpathlineto{\pgfqpoint{4.969110in}{2.647844in}}%
\pgfpathlineto{\pgfqpoint{4.955840in}{2.648325in}}%
\pgfpathlineto{\pgfqpoint{4.942579in}{2.648924in}}%
\pgfpathlineto{\pgfqpoint{4.929328in}{2.649640in}}%
\pgfpathlineto{\pgfqpoint{4.922083in}{2.639646in}}%
\pgfpathlineto{\pgfqpoint{4.914835in}{2.629663in}}%
\pgfpathlineto{\pgfqpoint{4.907582in}{2.619690in}}%
\pgfpathlineto{\pgfqpoint{4.900325in}{2.609726in}}%
\pgfpathclose%
\pgfusepath{fill}%
\end{pgfscope}%
\begin{pgfscope}%
\pgfpathrectangle{\pgfqpoint{1.254980in}{0.150000in}}{\pgfqpoint{5.490039in}{5.490039in}}%
\pgfusepath{clip}%
\pgfsetbuttcap%
\pgfsetroundjoin%
\definecolor{currentfill}{rgb}{0.280255,0.165693,0.476498}%
\pgfsetfillcolor{currentfill}%
\pgfsetfillopacity{0.700000}%
\pgfsetlinewidth{0.000000pt}%
\definecolor{currentstroke}{rgb}{0.000000,0.000000,0.000000}%
\pgfsetstrokecolor{currentstroke}%
\pgfsetdash{}{0pt}%
\pgfpathmoveto{\pgfqpoint{4.437385in}{2.455163in}}%
\pgfpathlineto{\pgfqpoint{4.450498in}{2.451904in}}%
\pgfpathlineto{\pgfqpoint{4.463618in}{2.448769in}}%
\pgfpathlineto{\pgfqpoint{4.476745in}{2.445760in}}%
\pgfpathlineto{\pgfqpoint{4.489879in}{2.442875in}}%
\pgfpathlineto{\pgfqpoint{4.497274in}{2.452938in}}%
\pgfpathlineto{\pgfqpoint{4.504664in}{2.463014in}}%
\pgfpathlineto{\pgfqpoint{4.512049in}{2.473103in}}%
\pgfpathlineto{\pgfqpoint{4.519430in}{2.483204in}}%
\pgfpathlineto{\pgfqpoint{4.506305in}{2.486112in}}%
\pgfpathlineto{\pgfqpoint{4.493188in}{2.489145in}}%
\pgfpathlineto{\pgfqpoint{4.480078in}{2.492302in}}%
\pgfpathlineto{\pgfqpoint{4.466975in}{2.495585in}}%
\pgfpathlineto{\pgfqpoint{4.459584in}{2.485454in}}%
\pgfpathlineto{\pgfqpoint{4.452189in}{2.475341in}}%
\pgfpathlineto{\pgfqpoint{4.444789in}{2.465244in}}%
\pgfpathlineto{\pgfqpoint{4.437385in}{2.455163in}}%
\pgfpathclose%
\pgfusepath{fill}%
\end{pgfscope}%
\begin{pgfscope}%
\pgfpathrectangle{\pgfqpoint{1.254980in}{0.150000in}}{\pgfqpoint{5.490039in}{5.490039in}}%
\pgfusepath{clip}%
\pgfsetbuttcap%
\pgfsetroundjoin%
\definecolor{currentfill}{rgb}{0.267968,0.223549,0.512008}%
\pgfsetfillcolor{currentfill}%
\pgfsetfillopacity{0.700000}%
\pgfsetlinewidth{0.000000pt}%
\definecolor{currentstroke}{rgb}{0.000000,0.000000,0.000000}%
\pgfsetstrokecolor{currentstroke}%
\pgfsetdash{}{0pt}%
\pgfpathmoveto{\pgfqpoint{3.497107in}{2.590854in}}%
\pgfpathlineto{\pgfqpoint{3.510098in}{2.579520in}}%
\pgfpathlineto{\pgfqpoint{3.523088in}{2.568343in}}%
\pgfpathlineto{\pgfqpoint{3.536079in}{2.557321in}}%
\pgfpathlineto{\pgfqpoint{3.549071in}{2.546454in}}%
\pgfpathlineto{\pgfqpoint{3.556776in}{2.554915in}}%
\pgfpathlineto{\pgfqpoint{3.564475in}{2.563445in}}%
\pgfpathlineto{\pgfqpoint{3.572168in}{2.572042in}}%
\pgfpathlineto{\pgfqpoint{3.579854in}{2.580705in}}%
\pgfpathlineto{\pgfqpoint{3.566879in}{2.591482in}}%
\pgfpathlineto{\pgfqpoint{3.553904in}{2.602413in}}%
\pgfpathlineto{\pgfqpoint{3.540929in}{2.613500in}}%
\pgfpathlineto{\pgfqpoint{3.527955in}{2.624743in}}%
\pgfpathlineto{\pgfqpoint{3.520252in}{2.616164in}}%
\pgfpathlineto{\pgfqpoint{3.512543in}{2.607656in}}%
\pgfpathlineto{\pgfqpoint{3.504828in}{2.599219in}}%
\pgfpathlineto{\pgfqpoint{3.497107in}{2.590854in}}%
\pgfpathclose%
\pgfusepath{fill}%
\end{pgfscope}%
\begin{pgfscope}%
\pgfpathrectangle{\pgfqpoint{1.254980in}{0.150000in}}{\pgfqpoint{5.490039in}{5.490039in}}%
\pgfusepath{clip}%
\pgfsetbuttcap%
\pgfsetroundjoin%
\definecolor{currentfill}{rgb}{0.266580,0.228262,0.514349}%
\pgfsetfillcolor{currentfill}%
\pgfsetfillopacity{0.700000}%
\pgfsetlinewidth{0.000000pt}%
\definecolor{currentstroke}{rgb}{0.000000,0.000000,0.000000}%
\pgfsetstrokecolor{currentstroke}%
\pgfsetdash{}{0pt}%
\pgfpathmoveto{\pgfqpoint{4.818259in}{2.573239in}}%
\pgfpathlineto{\pgfqpoint{4.831494in}{2.572236in}}%
\pgfpathlineto{\pgfqpoint{4.844738in}{2.571353in}}%
\pgfpathlineto{\pgfqpoint{4.857991in}{2.570589in}}%
\pgfpathlineto{\pgfqpoint{4.871253in}{2.569943in}}%
\pgfpathlineto{\pgfqpoint{4.878528in}{2.579880in}}%
\pgfpathlineto{\pgfqpoint{4.885798in}{2.589823in}}%
\pgfpathlineto{\pgfqpoint{4.893064in}{2.599771in}}%
\pgfpathlineto{\pgfqpoint{4.900325in}{2.609726in}}%
\pgfpathlineto{\pgfqpoint{4.887073in}{2.610459in}}%
\pgfpathlineto{\pgfqpoint{4.873830in}{2.611310in}}%
\pgfpathlineto{\pgfqpoint{4.860596in}{2.612280in}}%
\pgfpathlineto{\pgfqpoint{4.847371in}{2.613369in}}%
\pgfpathlineto{\pgfqpoint{4.840100in}{2.603322in}}%
\pgfpathlineto{\pgfqpoint{4.832824in}{2.593285in}}%
\pgfpathlineto{\pgfqpoint{4.825544in}{2.583257in}}%
\pgfpathlineto{\pgfqpoint{4.818259in}{2.573239in}}%
\pgfpathclose%
\pgfusepath{fill}%
\end{pgfscope}%
\begin{pgfscope}%
\pgfpathrectangle{\pgfqpoint{1.254980in}{0.150000in}}{\pgfqpoint{5.490039in}{5.490039in}}%
\pgfusepath{clip}%
\pgfsetbuttcap%
\pgfsetroundjoin%
\definecolor{currentfill}{rgb}{0.282623,0.140926,0.457517}%
\pgfsetfillcolor{currentfill}%
\pgfsetfillopacity{0.700000}%
\pgfsetlinewidth{0.000000pt}%
\definecolor{currentstroke}{rgb}{0.000000,0.000000,0.000000}%
\pgfsetstrokecolor{currentstroke}%
\pgfsetdash{}{0pt}%
\pgfpathmoveto{\pgfqpoint{4.004417in}{2.414254in}}%
\pgfpathlineto{\pgfqpoint{4.017437in}{2.407798in}}%
\pgfpathlineto{\pgfqpoint{4.030461in}{2.401478in}}%
\pgfpathlineto{\pgfqpoint{4.043490in}{2.395293in}}%
\pgfpathlineto{\pgfqpoint{4.056523in}{2.389243in}}%
\pgfpathlineto{\pgfqpoint{4.064055in}{2.398872in}}%
\pgfpathlineto{\pgfqpoint{4.071581in}{2.408535in}}%
\pgfpathlineto{\pgfqpoint{4.079102in}{2.418230in}}%
\pgfpathlineto{\pgfqpoint{4.086619in}{2.427957in}}%
\pgfpathlineto{\pgfqpoint{4.073597in}{2.433967in}}%
\pgfpathlineto{\pgfqpoint{4.060580in}{2.440112in}}%
\pgfpathlineto{\pgfqpoint{4.047567in}{2.446391in}}%
\pgfpathlineto{\pgfqpoint{4.034559in}{2.452806in}}%
\pgfpathlineto{\pgfqpoint{4.027031in}{2.443114in}}%
\pgfpathlineto{\pgfqpoint{4.019498in}{2.433457in}}%
\pgfpathlineto{\pgfqpoint{4.011960in}{2.423838in}}%
\pgfpathlineto{\pgfqpoint{4.004417in}{2.414254in}}%
\pgfpathclose%
\pgfusepath{fill}%
\end{pgfscope}%
\begin{pgfscope}%
\pgfpathrectangle{\pgfqpoint{1.254980in}{0.150000in}}{\pgfqpoint{5.490039in}{5.490039in}}%
\pgfusepath{clip}%
\pgfsetbuttcap%
\pgfsetroundjoin%
\definecolor{currentfill}{rgb}{0.281887,0.150881,0.465405}%
\pgfsetfillcolor{currentfill}%
\pgfsetfillopacity{0.700000}%
\pgfsetlinewidth{0.000000pt}%
\definecolor{currentstroke}{rgb}{0.000000,0.000000,0.000000}%
\pgfsetstrokecolor{currentstroke}%
\pgfsetdash{}{0pt}%
\pgfpathmoveto{\pgfqpoint{3.870077in}{2.433337in}}%
\pgfpathlineto{\pgfqpoint{3.883080in}{2.425720in}}%
\pgfpathlineto{\pgfqpoint{3.896085in}{2.418243in}}%
\pgfpathlineto{\pgfqpoint{3.909095in}{2.410905in}}%
\pgfpathlineto{\pgfqpoint{3.922107in}{2.403707in}}%
\pgfpathlineto{\pgfqpoint{3.929683in}{2.413080in}}%
\pgfpathlineto{\pgfqpoint{3.937253in}{2.422494in}}%
\pgfpathlineto{\pgfqpoint{3.944817in}{2.431949in}}%
\pgfpathlineto{\pgfqpoint{3.952377in}{2.441445in}}%
\pgfpathlineto{\pgfqpoint{3.939377in}{2.448587in}}%
\pgfpathlineto{\pgfqpoint{3.926380in}{2.455868in}}%
\pgfpathlineto{\pgfqpoint{3.913387in}{2.463288in}}%
\pgfpathlineto{\pgfqpoint{3.900397in}{2.470848in}}%
\pgfpathlineto{\pgfqpoint{3.892825in}{2.461403in}}%
\pgfpathlineto{\pgfqpoint{3.885248in}{2.452003in}}%
\pgfpathlineto{\pgfqpoint{3.877665in}{2.442648in}}%
\pgfpathlineto{\pgfqpoint{3.870077in}{2.433337in}}%
\pgfpathclose%
\pgfusepath{fill}%
\end{pgfscope}%
\begin{pgfscope}%
\pgfpathrectangle{\pgfqpoint{1.254980in}{0.150000in}}{\pgfqpoint{5.490039in}{5.490039in}}%
\pgfusepath{clip}%
\pgfsetbuttcap%
\pgfsetroundjoin%
\definecolor{currentfill}{rgb}{0.140536,0.530132,0.555659}%
\pgfsetfillcolor{currentfill}%
\pgfsetfillopacity{0.700000}%
\pgfsetlinewidth{0.000000pt}%
\definecolor{currentstroke}{rgb}{0.000000,0.000000,0.000000}%
\pgfsetstrokecolor{currentstroke}%
\pgfsetdash{}{0pt}%
\pgfpathmoveto{\pgfqpoint{2.871030in}{3.337118in}}%
\pgfpathlineto{\pgfqpoint{2.884181in}{3.317115in}}%
\pgfpathlineto{\pgfqpoint{2.897326in}{3.297322in}}%
\pgfpathlineto{\pgfqpoint{2.910463in}{3.277736in}}%
\pgfpathlineto{\pgfqpoint{2.923594in}{3.258357in}}%
\pgfpathlineto{\pgfqpoint{2.931534in}{3.265614in}}%
\pgfpathlineto{\pgfqpoint{2.939466in}{3.272988in}}%
\pgfpathlineto{\pgfqpoint{2.947389in}{3.280478in}}%
\pgfpathlineto{\pgfqpoint{2.955305in}{3.288084in}}%
\pgfpathlineto{\pgfqpoint{2.942197in}{3.307364in}}%
\pgfpathlineto{\pgfqpoint{2.929083in}{3.326850in}}%
\pgfpathlineto{\pgfqpoint{2.915963in}{3.346543in}}%
\pgfpathlineto{\pgfqpoint{2.902835in}{3.366446in}}%
\pgfpathlineto{\pgfqpoint{2.894896in}{3.358933in}}%
\pgfpathlineto{\pgfqpoint{2.886950in}{3.351541in}}%
\pgfpathlineto{\pgfqpoint{2.878994in}{3.344269in}}%
\pgfpathlineto{\pgfqpoint{2.871030in}{3.337118in}}%
\pgfpathclose%
\pgfusepath{fill}%
\end{pgfscope}%
\begin{pgfscope}%
\pgfpathrectangle{\pgfqpoint{1.254980in}{0.150000in}}{\pgfqpoint{5.490039in}{5.490039in}}%
\pgfusepath{clip}%
\pgfsetbuttcap%
\pgfsetroundjoin%
\definecolor{currentfill}{rgb}{0.271828,0.209303,0.504434}%
\pgfsetfillcolor{currentfill}%
\pgfsetfillopacity{0.700000}%
\pgfsetlinewidth{0.000000pt}%
\definecolor{currentstroke}{rgb}{0.000000,0.000000,0.000000}%
\pgfsetstrokecolor{currentstroke}%
\pgfsetdash{}{0pt}%
\pgfpathmoveto{\pgfqpoint{4.736187in}{2.538157in}}%
\pgfpathlineto{\pgfqpoint{4.749396in}{2.536746in}}%
\pgfpathlineto{\pgfqpoint{4.762614in}{2.535455in}}%
\pgfpathlineto{\pgfqpoint{4.775841in}{2.534284in}}%
\pgfpathlineto{\pgfqpoint{4.789076in}{2.533233in}}%
\pgfpathlineto{\pgfqpoint{4.796379in}{2.543226in}}%
\pgfpathlineto{\pgfqpoint{4.803677in}{2.553224in}}%
\pgfpathlineto{\pgfqpoint{4.810970in}{2.563228in}}%
\pgfpathlineto{\pgfqpoint{4.818259in}{2.573239in}}%
\pgfpathlineto{\pgfqpoint{4.805033in}{2.574361in}}%
\pgfpathlineto{\pgfqpoint{4.791816in}{2.575603in}}%
\pgfpathlineto{\pgfqpoint{4.778608in}{2.576964in}}%
\pgfpathlineto{\pgfqpoint{4.765409in}{2.578446in}}%
\pgfpathlineto{\pgfqpoint{4.758110in}{2.568359in}}%
\pgfpathlineto{\pgfqpoint{4.750807in}{2.558282in}}%
\pgfpathlineto{\pgfqpoint{4.743499in}{2.548215in}}%
\pgfpathlineto{\pgfqpoint{4.736187in}{2.538157in}}%
\pgfpathclose%
\pgfusepath{fill}%
\end{pgfscope}%
\begin{pgfscope}%
\pgfpathrectangle{\pgfqpoint{1.254980in}{0.150000in}}{\pgfqpoint{5.490039in}{5.490039in}}%
\pgfusepath{clip}%
\pgfsetbuttcap%
\pgfsetroundjoin%
\definecolor{currentfill}{rgb}{0.281412,0.155834,0.469201}%
\pgfsetfillcolor{currentfill}%
\pgfsetfillopacity{0.700000}%
\pgfsetlinewidth{0.000000pt}%
\definecolor{currentstroke}{rgb}{0.000000,0.000000,0.000000}%
\pgfsetstrokecolor{currentstroke}%
\pgfsetdash{}{0pt}%
\pgfpathmoveto{\pgfqpoint{4.355300in}{2.429266in}}%
\pgfpathlineto{\pgfqpoint{4.368396in}{2.425508in}}%
\pgfpathlineto{\pgfqpoint{4.381498in}{2.421878in}}%
\pgfpathlineto{\pgfqpoint{4.394607in}{2.418373in}}%
\pgfpathlineto{\pgfqpoint{4.407723in}{2.414995in}}%
\pgfpathlineto{\pgfqpoint{4.415145in}{2.425015in}}%
\pgfpathlineto{\pgfqpoint{4.422563in}{2.435049in}}%
\pgfpathlineto{\pgfqpoint{4.429976in}{2.445099in}}%
\pgfpathlineto{\pgfqpoint{4.437385in}{2.455163in}}%
\pgfpathlineto{\pgfqpoint{4.424279in}{2.458549in}}%
\pgfpathlineto{\pgfqpoint{4.411180in}{2.462060in}}%
\pgfpathlineto{\pgfqpoint{4.398087in}{2.465698in}}%
\pgfpathlineto{\pgfqpoint{4.385001in}{2.469463in}}%
\pgfpathlineto{\pgfqpoint{4.377583in}{2.459385in}}%
\pgfpathlineto{\pgfqpoint{4.370160in}{2.449327in}}%
\pgfpathlineto{\pgfqpoint{4.362732in}{2.439287in}}%
\pgfpathlineto{\pgfqpoint{4.355300in}{2.429266in}}%
\pgfpathclose%
\pgfusepath{fill}%
\end{pgfscope}%
\begin{pgfscope}%
\pgfpathrectangle{\pgfqpoint{1.254980in}{0.150000in}}{\pgfqpoint{5.490039in}{5.490039in}}%
\pgfusepath{clip}%
\pgfsetbuttcap%
\pgfsetroundjoin%
\definecolor{currentfill}{rgb}{0.282623,0.140926,0.457517}%
\pgfsetfillcolor{currentfill}%
\pgfsetfillopacity{0.700000}%
\pgfsetlinewidth{0.000000pt}%
\definecolor{currentstroke}{rgb}{0.000000,0.000000,0.000000}%
\pgfsetstrokecolor{currentstroke}%
\pgfsetdash{}{0pt}%
\pgfpathmoveto{\pgfqpoint{4.138754in}{2.405254in}}%
\pgfpathlineto{\pgfqpoint{4.151800in}{2.399911in}}%
\pgfpathlineto{\pgfqpoint{4.164852in}{2.394698in}}%
\pgfpathlineto{\pgfqpoint{4.177909in}{2.389618in}}%
\pgfpathlineto{\pgfqpoint{4.190971in}{2.384668in}}%
\pgfpathlineto{\pgfqpoint{4.198461in}{2.394489in}}%
\pgfpathlineto{\pgfqpoint{4.205946in}{2.404335in}}%
\pgfpathlineto{\pgfqpoint{4.213426in}{2.414206in}}%
\pgfpathlineto{\pgfqpoint{4.220902in}{2.424101in}}%
\pgfpathlineto{\pgfqpoint{4.207850in}{2.429027in}}%
\pgfpathlineto{\pgfqpoint{4.194804in}{2.434083in}}%
\pgfpathlineto{\pgfqpoint{4.181764in}{2.439270in}}%
\pgfpathlineto{\pgfqpoint{4.168728in}{2.444590in}}%
\pgfpathlineto{\pgfqpoint{4.161242in}{2.434713in}}%
\pgfpathlineto{\pgfqpoint{4.153751in}{2.424865in}}%
\pgfpathlineto{\pgfqpoint{4.146255in}{2.415045in}}%
\pgfpathlineto{\pgfqpoint{4.138754in}{2.405254in}}%
\pgfpathclose%
\pgfusepath{fill}%
\end{pgfscope}%
\begin{pgfscope}%
\pgfpathrectangle{\pgfqpoint{1.254980in}{0.150000in}}{\pgfqpoint{5.490039in}{5.490039in}}%
\pgfusepath{clip}%
\pgfsetbuttcap%
\pgfsetroundjoin%
\definecolor{currentfill}{rgb}{0.280255,0.165693,0.476498}%
\pgfsetfillcolor{currentfill}%
\pgfsetfillopacity{0.700000}%
\pgfsetlinewidth{0.000000pt}%
\definecolor{currentstroke}{rgb}{0.000000,0.000000,0.000000}%
\pgfsetstrokecolor{currentstroke}%
\pgfsetdash{}{0pt}%
\pgfpathmoveto{\pgfqpoint{3.735650in}{2.463182in}}%
\pgfpathlineto{\pgfqpoint{3.748644in}{2.454351in}}%
\pgfpathlineto{\pgfqpoint{3.761641in}{2.445664in}}%
\pgfpathlineto{\pgfqpoint{3.774640in}{2.437122in}}%
\pgfpathlineto{\pgfqpoint{3.787641in}{2.428723in}}%
\pgfpathlineto{\pgfqpoint{3.795263in}{2.437771in}}%
\pgfpathlineto{\pgfqpoint{3.802880in}{2.446870in}}%
\pgfpathlineto{\pgfqpoint{3.810492in}{2.456019in}}%
\pgfpathlineto{\pgfqpoint{3.818098in}{2.465217in}}%
\pgfpathlineto{\pgfqpoint{3.805110in}{2.473543in}}%
\pgfpathlineto{\pgfqpoint{3.792125in}{2.482012in}}%
\pgfpathlineto{\pgfqpoint{3.779143in}{2.490626in}}%
\pgfpathlineto{\pgfqpoint{3.766162in}{2.499384in}}%
\pgfpathlineto{\pgfqpoint{3.758543in}{2.490253in}}%
\pgfpathlineto{\pgfqpoint{3.750917in}{2.481176in}}%
\pgfpathlineto{\pgfqpoint{3.743286in}{2.472152in}}%
\pgfpathlineto{\pgfqpoint{3.735650in}{2.463182in}}%
\pgfpathclose%
\pgfusepath{fill}%
\end{pgfscope}%
\begin{pgfscope}%
\pgfpathrectangle{\pgfqpoint{1.254980in}{0.150000in}}{\pgfqpoint{5.490039in}{5.490039in}}%
\pgfusepath{clip}%
\pgfsetbuttcap%
\pgfsetroundjoin%
\definecolor{currentfill}{rgb}{0.273006,0.204520,0.501721}%
\pgfsetfillcolor{currentfill}%
\pgfsetfillopacity{0.700000}%
\pgfsetlinewidth{0.000000pt}%
\definecolor{currentstroke}{rgb}{0.000000,0.000000,0.000000}%
\pgfsetstrokecolor{currentstroke}%
\pgfsetdash{}{0pt}%
\pgfpathmoveto{\pgfqpoint{3.549071in}{2.546454in}}%
\pgfpathlineto{\pgfqpoint{3.562063in}{2.535741in}}%
\pgfpathlineto{\pgfqpoint{3.575056in}{2.525181in}}%
\pgfpathlineto{\pgfqpoint{3.588050in}{2.514773in}}%
\pgfpathlineto{\pgfqpoint{3.601044in}{2.504518in}}%
\pgfpathlineto{\pgfqpoint{3.608733in}{2.513076in}}%
\pgfpathlineto{\pgfqpoint{3.616416in}{2.521697in}}%
\pgfpathlineto{\pgfqpoint{3.624093in}{2.530382in}}%
\pgfpathlineto{\pgfqpoint{3.631765in}{2.539130in}}%
\pgfpathlineto{\pgfqpoint{3.618786in}{2.549296in}}%
\pgfpathlineto{\pgfqpoint{3.605808in}{2.559613in}}%
\pgfpathlineto{\pgfqpoint{3.592831in}{2.570082in}}%
\pgfpathlineto{\pgfqpoint{3.579854in}{2.580705in}}%
\pgfpathlineto{\pgfqpoint{3.572168in}{2.572042in}}%
\pgfpathlineto{\pgfqpoint{3.564475in}{2.563445in}}%
\pgfpathlineto{\pgfqpoint{3.556776in}{2.554915in}}%
\pgfpathlineto{\pgfqpoint{3.549071in}{2.546454in}}%
\pgfpathclose%
\pgfusepath{fill}%
\end{pgfscope}%
\begin{pgfscope}%
\pgfpathrectangle{\pgfqpoint{1.254980in}{0.150000in}}{\pgfqpoint{5.490039in}{5.490039in}}%
\pgfusepath{clip}%
\pgfsetbuttcap%
\pgfsetroundjoin%
\definecolor{currentfill}{rgb}{0.275191,0.194905,0.496005}%
\pgfsetfillcolor{currentfill}%
\pgfsetfillopacity{0.700000}%
\pgfsetlinewidth{0.000000pt}%
\definecolor{currentstroke}{rgb}{0.000000,0.000000,0.000000}%
\pgfsetstrokecolor{currentstroke}%
\pgfsetdash{}{0pt}%
\pgfpathmoveto{\pgfqpoint{4.654104in}{2.504629in}}%
\pgfpathlineto{\pgfqpoint{4.667289in}{2.502789in}}%
\pgfpathlineto{\pgfqpoint{4.680482in}{2.501070in}}%
\pgfpathlineto{\pgfqpoint{4.693684in}{2.499473in}}%
\pgfpathlineto{\pgfqpoint{4.706894in}{2.497996in}}%
\pgfpathlineto{\pgfqpoint{4.714224in}{2.508027in}}%
\pgfpathlineto{\pgfqpoint{4.721550in}{2.518064in}}%
\pgfpathlineto{\pgfqpoint{4.728870in}{2.528107in}}%
\pgfpathlineto{\pgfqpoint{4.736187in}{2.538157in}}%
\pgfpathlineto{\pgfqpoint{4.722986in}{2.539689in}}%
\pgfpathlineto{\pgfqpoint{4.709794in}{2.541341in}}%
\pgfpathlineto{\pgfqpoint{4.696610in}{2.543115in}}%
\pgfpathlineto{\pgfqpoint{4.683435in}{2.545010in}}%
\pgfpathlineto{\pgfqpoint{4.676109in}{2.534899in}}%
\pgfpathlineto{\pgfqpoint{4.668778in}{2.524799in}}%
\pgfpathlineto{\pgfqpoint{4.661443in}{2.514709in}}%
\pgfpathlineto{\pgfqpoint{4.654104in}{2.504629in}}%
\pgfpathclose%
\pgfusepath{fill}%
\end{pgfscope}%
\begin{pgfscope}%
\pgfpathrectangle{\pgfqpoint{1.254980in}{0.150000in}}{\pgfqpoint{5.490039in}{5.490039in}}%
\pgfusepath{clip}%
\pgfsetbuttcap%
\pgfsetroundjoin%
\definecolor{currentfill}{rgb}{0.214298,0.355619,0.551184}%
\pgfsetfillcolor{currentfill}%
\pgfsetfillopacity{0.700000}%
\pgfsetlinewidth{0.000000pt}%
\definecolor{currentstroke}{rgb}{0.000000,0.000000,0.000000}%
\pgfsetstrokecolor{currentstroke}%
\pgfsetdash{}{0pt}%
\pgfpathmoveto{\pgfqpoint{3.153639in}{2.881897in}}%
\pgfpathlineto{\pgfqpoint{3.166687in}{2.866435in}}%
\pgfpathlineto{\pgfqpoint{3.179731in}{2.851151in}}%
\pgfpathlineto{\pgfqpoint{3.192772in}{2.836045in}}%
\pgfpathlineto{\pgfqpoint{3.205810in}{2.821116in}}%
\pgfpathlineto{\pgfqpoint{3.213650in}{2.828615in}}%
\pgfpathlineto{\pgfqpoint{3.221484in}{2.836208in}}%
\pgfpathlineto{\pgfqpoint{3.229310in}{2.843895in}}%
\pgfpathlineto{\pgfqpoint{3.237129in}{2.851675in}}%
\pgfpathlineto{\pgfqpoint{3.224111in}{2.866494in}}%
\pgfpathlineto{\pgfqpoint{3.211090in}{2.881489in}}%
\pgfpathlineto{\pgfqpoint{3.198066in}{2.896662in}}%
\pgfpathlineto{\pgfqpoint{3.185040in}{2.912014in}}%
\pgfpathlineto{\pgfqpoint{3.177201in}{2.904338in}}%
\pgfpathlineto{\pgfqpoint{3.169354in}{2.896760in}}%
\pgfpathlineto{\pgfqpoint{3.161500in}{2.889279in}}%
\pgfpathlineto{\pgfqpoint{3.153639in}{2.881897in}}%
\pgfpathclose%
\pgfusepath{fill}%
\end{pgfscope}%
\begin{pgfscope}%
\pgfpathrectangle{\pgfqpoint{1.254980in}{0.150000in}}{\pgfqpoint{5.490039in}{5.490039in}}%
\pgfusepath{clip}%
\pgfsetbuttcap%
\pgfsetroundjoin%
\definecolor{currentfill}{rgb}{0.225863,0.330805,0.547314}%
\pgfsetfillcolor{currentfill}%
\pgfsetfillopacity{0.700000}%
\pgfsetlinewidth{0.000000pt}%
\definecolor{currentstroke}{rgb}{0.000000,0.000000,0.000000}%
\pgfsetstrokecolor{currentstroke}%
\pgfsetdash{}{0pt}%
\pgfpathmoveto{\pgfqpoint{3.205810in}{2.821116in}}%
\pgfpathlineto{\pgfqpoint{3.218845in}{2.806363in}}%
\pgfpathlineto{\pgfqpoint{3.231877in}{2.791784in}}%
\pgfpathlineto{\pgfqpoint{3.244907in}{2.777379in}}%
\pgfpathlineto{\pgfqpoint{3.257935in}{2.763146in}}%
\pgfpathlineto{\pgfqpoint{3.265755in}{2.770761in}}%
\pgfpathlineto{\pgfqpoint{3.273569in}{2.778466in}}%
\pgfpathlineto{\pgfqpoint{3.281375in}{2.786261in}}%
\pgfpathlineto{\pgfqpoint{3.289175in}{2.794145in}}%
\pgfpathlineto{\pgfqpoint{3.276167in}{2.808268in}}%
\pgfpathlineto{\pgfqpoint{3.263157in}{2.822563in}}%
\pgfpathlineto{\pgfqpoint{3.250144in}{2.837032in}}%
\pgfpathlineto{\pgfqpoint{3.237129in}{2.851675in}}%
\pgfpathlineto{\pgfqpoint{3.229310in}{2.843895in}}%
\pgfpathlineto{\pgfqpoint{3.221484in}{2.836208in}}%
\pgfpathlineto{\pgfqpoint{3.213650in}{2.828615in}}%
\pgfpathlineto{\pgfqpoint{3.205810in}{2.821116in}}%
\pgfpathclose%
\pgfusepath{fill}%
\end{pgfscope}%
\begin{pgfscope}%
\pgfpathrectangle{\pgfqpoint{1.254980in}{0.150000in}}{\pgfqpoint{5.490039in}{5.490039in}}%
\pgfusepath{clip}%
\pgfsetbuttcap%
\pgfsetroundjoin%
\definecolor{currentfill}{rgb}{0.201239,0.383670,0.554294}%
\pgfsetfillcolor{currentfill}%
\pgfsetfillopacity{0.700000}%
\pgfsetlinewidth{0.000000pt}%
\definecolor{currentstroke}{rgb}{0.000000,0.000000,0.000000}%
\pgfsetstrokecolor{currentstroke}%
\pgfsetdash{}{0pt}%
\pgfpathmoveto{\pgfqpoint{3.101413in}{2.945559in}}%
\pgfpathlineto{\pgfqpoint{3.114475in}{2.929370in}}%
\pgfpathlineto{\pgfqpoint{3.127533in}{2.913364in}}%
\pgfpathlineto{\pgfqpoint{3.140588in}{2.897540in}}%
\pgfpathlineto{\pgfqpoint{3.153639in}{2.881897in}}%
\pgfpathlineto{\pgfqpoint{3.161500in}{2.889279in}}%
\pgfpathlineto{\pgfqpoint{3.169354in}{2.896760in}}%
\pgfpathlineto{\pgfqpoint{3.177201in}{2.904338in}}%
\pgfpathlineto{\pgfqpoint{3.185040in}{2.912014in}}%
\pgfpathlineto{\pgfqpoint{3.172009in}{2.927545in}}%
\pgfpathlineto{\pgfqpoint{3.158976in}{2.943257in}}%
\pgfpathlineto{\pgfqpoint{3.145938in}{2.959152in}}%
\pgfpathlineto{\pgfqpoint{3.132897in}{2.975229in}}%
\pgfpathlineto{\pgfqpoint{3.125038in}{2.967659in}}%
\pgfpathlineto{\pgfqpoint{3.117170in}{2.960190in}}%
\pgfpathlineto{\pgfqpoint{3.109295in}{2.952823in}}%
\pgfpathlineto{\pgfqpoint{3.101413in}{2.945559in}}%
\pgfpathclose%
\pgfusepath{fill}%
\end{pgfscope}%
\begin{pgfscope}%
\pgfpathrectangle{\pgfqpoint{1.254980in}{0.150000in}}{\pgfqpoint{5.490039in}{5.490039in}}%
\pgfusepath{clip}%
\pgfsetbuttcap%
\pgfsetroundjoin%
\definecolor{currentfill}{rgb}{0.237441,0.305202,0.541921}%
\pgfsetfillcolor{currentfill}%
\pgfsetfillopacity{0.700000}%
\pgfsetlinewidth{0.000000pt}%
\definecolor{currentstroke}{rgb}{0.000000,0.000000,0.000000}%
\pgfsetstrokecolor{currentstroke}%
\pgfsetdash{}{0pt}%
\pgfpathmoveto{\pgfqpoint{3.257935in}{2.763146in}}%
\pgfpathlineto{\pgfqpoint{3.270960in}{2.749085in}}%
\pgfpathlineto{\pgfqpoint{3.283983in}{2.735195in}}%
\pgfpathlineto{\pgfqpoint{3.297004in}{2.721474in}}%
\pgfpathlineto{\pgfqpoint{3.310023in}{2.707922in}}%
\pgfpathlineto{\pgfqpoint{3.317824in}{2.715652in}}%
\pgfpathlineto{\pgfqpoint{3.325618in}{2.723469in}}%
\pgfpathlineto{\pgfqpoint{3.333406in}{2.731371in}}%
\pgfpathlineto{\pgfqpoint{3.341186in}{2.739358in}}%
\pgfpathlineto{\pgfqpoint{3.328186in}{2.752801in}}%
\pgfpathlineto{\pgfqpoint{3.315184in}{2.766413in}}%
\pgfpathlineto{\pgfqpoint{3.302180in}{2.780194in}}%
\pgfpathlineto{\pgfqpoint{3.289175in}{2.794145in}}%
\pgfpathlineto{\pgfqpoint{3.281375in}{2.786261in}}%
\pgfpathlineto{\pgfqpoint{3.273569in}{2.778466in}}%
\pgfpathlineto{\pgfqpoint{3.265755in}{2.770761in}}%
\pgfpathlineto{\pgfqpoint{3.257935in}{2.763146in}}%
\pgfpathclose%
\pgfusepath{fill}%
\end{pgfscope}%
\begin{pgfscope}%
\pgfpathrectangle{\pgfqpoint{1.254980in}{0.150000in}}{\pgfqpoint{5.490039in}{5.490039in}}%
\pgfusepath{clip}%
\pgfsetbuttcap%
\pgfsetroundjoin%
\definecolor{currentfill}{rgb}{0.188923,0.410910,0.556326}%
\pgfsetfillcolor{currentfill}%
\pgfsetfillopacity{0.700000}%
\pgfsetlinewidth{0.000000pt}%
\definecolor{currentstroke}{rgb}{0.000000,0.000000,0.000000}%
\pgfsetstrokecolor{currentstroke}%
\pgfsetdash{}{0pt}%
\pgfpathmoveto{\pgfqpoint{3.049121in}{3.012172in}}%
\pgfpathlineto{\pgfqpoint{3.062200in}{2.995238in}}%
\pgfpathlineto{\pgfqpoint{3.075276in}{2.978492in}}%
\pgfpathlineto{\pgfqpoint{3.088346in}{2.961932in}}%
\pgfpathlineto{\pgfqpoint{3.101413in}{2.945559in}}%
\pgfpathlineto{\pgfqpoint{3.109295in}{2.952823in}}%
\pgfpathlineto{\pgfqpoint{3.117170in}{2.960190in}}%
\pgfpathlineto{\pgfqpoint{3.125038in}{2.967659in}}%
\pgfpathlineto{\pgfqpoint{3.132897in}{2.975229in}}%
\pgfpathlineto{\pgfqpoint{3.119852in}{2.991490in}}%
\pgfpathlineto{\pgfqpoint{3.106803in}{3.007937in}}%
\pgfpathlineto{\pgfqpoint{3.093750in}{3.024571in}}%
\pgfpathlineto{\pgfqpoint{3.080692in}{3.041393in}}%
\pgfpathlineto{\pgfqpoint{3.072811in}{3.033929in}}%
\pgfpathlineto{\pgfqpoint{3.064922in}{3.026570in}}%
\pgfpathlineto{\pgfqpoint{3.057026in}{3.019318in}}%
\pgfpathlineto{\pgfqpoint{3.049121in}{3.012172in}}%
\pgfpathclose%
\pgfusepath{fill}%
\end{pgfscope}%
\begin{pgfscope}%
\pgfpathrectangle{\pgfqpoint{1.254980in}{0.150000in}}{\pgfqpoint{5.490039in}{5.490039in}}%
\pgfusepath{clip}%
\pgfsetbuttcap%
\pgfsetroundjoin%
\definecolor{currentfill}{rgb}{0.278012,0.180367,0.486697}%
\pgfsetfillcolor{currentfill}%
\pgfsetfillopacity{0.700000}%
\pgfsetlinewidth{0.000000pt}%
\definecolor{currentstroke}{rgb}{0.000000,0.000000,0.000000}%
\pgfsetstrokecolor{currentstroke}%
\pgfsetdash{}{0pt}%
\pgfpathmoveto{\pgfqpoint{4.572004in}{2.472812in}}%
\pgfpathlineto{\pgfqpoint{4.585166in}{2.470522in}}%
\pgfpathlineto{\pgfqpoint{4.598337in}{2.468355in}}%
\pgfpathlineto{\pgfqpoint{4.611515in}{2.466311in}}%
\pgfpathlineto{\pgfqpoint{4.624701in}{2.464388in}}%
\pgfpathlineto{\pgfqpoint{4.632059in}{2.474437in}}%
\pgfpathlineto{\pgfqpoint{4.639412in}{2.484494in}}%
\pgfpathlineto{\pgfqpoint{4.646760in}{2.494557in}}%
\pgfpathlineto{\pgfqpoint{4.654104in}{2.504629in}}%
\pgfpathlineto{\pgfqpoint{4.640927in}{2.506591in}}%
\pgfpathlineto{\pgfqpoint{4.627758in}{2.508674in}}%
\pgfpathlineto{\pgfqpoint{4.614597in}{2.510881in}}%
\pgfpathlineto{\pgfqpoint{4.601444in}{2.513209in}}%
\pgfpathlineto{\pgfqpoint{4.594091in}{2.503093in}}%
\pgfpathlineto{\pgfqpoint{4.586733in}{2.492988in}}%
\pgfpathlineto{\pgfqpoint{4.579370in}{2.482895in}}%
\pgfpathlineto{\pgfqpoint{4.572004in}{2.472812in}}%
\pgfpathclose%
\pgfusepath{fill}%
\end{pgfscope}%
\begin{pgfscope}%
\pgfpathrectangle{\pgfqpoint{1.254980in}{0.150000in}}{\pgfqpoint{5.490039in}{5.490039in}}%
\pgfusepath{clip}%
\pgfsetbuttcap%
\pgfsetroundjoin%
\definecolor{currentfill}{rgb}{0.282290,0.145912,0.461510}%
\pgfsetfillcolor{currentfill}%
\pgfsetfillopacity{0.700000}%
\pgfsetlinewidth{0.000000pt}%
\definecolor{currentstroke}{rgb}{0.000000,0.000000,0.000000}%
\pgfsetstrokecolor{currentstroke}%
\pgfsetdash{}{0pt}%
\pgfpathmoveto{\pgfqpoint{4.273166in}{2.405701in}}%
\pgfpathlineto{\pgfqpoint{4.286246in}{2.401423in}}%
\pgfpathlineto{\pgfqpoint{4.299333in}{2.397275in}}%
\pgfpathlineto{\pgfqpoint{4.312426in}{2.393254in}}%
\pgfpathlineto{\pgfqpoint{4.325525in}{2.389361in}}%
\pgfpathlineto{\pgfqpoint{4.332976in}{2.399310in}}%
\pgfpathlineto{\pgfqpoint{4.340422in}{2.409278in}}%
\pgfpathlineto{\pgfqpoint{4.347863in}{2.419263in}}%
\pgfpathlineto{\pgfqpoint{4.355300in}{2.429266in}}%
\pgfpathlineto{\pgfqpoint{4.342211in}{2.433150in}}%
\pgfpathlineto{\pgfqpoint{4.329128in}{2.437163in}}%
\pgfpathlineto{\pgfqpoint{4.316051in}{2.441303in}}%
\pgfpathlineto{\pgfqpoint{4.302981in}{2.445571in}}%
\pgfpathlineto{\pgfqpoint{4.295534in}{2.435571in}}%
\pgfpathlineto{\pgfqpoint{4.288082in}{2.425593in}}%
\pgfpathlineto{\pgfqpoint{4.280626in}{2.415636in}}%
\pgfpathlineto{\pgfqpoint{4.273166in}{2.405701in}}%
\pgfpathclose%
\pgfusepath{fill}%
\end{pgfscope}%
\begin{pgfscope}%
\pgfpathrectangle{\pgfqpoint{1.254980in}{0.150000in}}{\pgfqpoint{5.490039in}{5.490039in}}%
\pgfusepath{clip}%
\pgfsetbuttcap%
\pgfsetroundjoin%
\definecolor{currentfill}{rgb}{0.246811,0.283237,0.535941}%
\pgfsetfillcolor{currentfill}%
\pgfsetfillopacity{0.700000}%
\pgfsetlinewidth{0.000000pt}%
\definecolor{currentstroke}{rgb}{0.000000,0.000000,0.000000}%
\pgfsetstrokecolor{currentstroke}%
\pgfsetdash{}{0pt}%
\pgfpathmoveto{\pgfqpoint{3.310023in}{2.707922in}}%
\pgfpathlineto{\pgfqpoint{3.323040in}{2.694538in}}%
\pgfpathlineto{\pgfqpoint{3.336056in}{2.681320in}}%
\pgfpathlineto{\pgfqpoint{3.349071in}{2.668269in}}%
\pgfpathlineto{\pgfqpoint{3.362084in}{2.655382in}}%
\pgfpathlineto{\pgfqpoint{3.369866in}{2.663227in}}%
\pgfpathlineto{\pgfqpoint{3.377642in}{2.671154in}}%
\pgfpathlineto{\pgfqpoint{3.385411in}{2.679163in}}%
\pgfpathlineto{\pgfqpoint{3.393173in}{2.687252in}}%
\pgfpathlineto{\pgfqpoint{3.380178in}{2.700031in}}%
\pgfpathlineto{\pgfqpoint{3.367182in}{2.712974in}}%
\pgfpathlineto{\pgfqpoint{3.354185in}{2.726083in}}%
\pgfpathlineto{\pgfqpoint{3.341186in}{2.739358in}}%
\pgfpathlineto{\pgfqpoint{3.333406in}{2.731371in}}%
\pgfpathlineto{\pgfqpoint{3.325618in}{2.723469in}}%
\pgfpathlineto{\pgfqpoint{3.317824in}{2.715652in}}%
\pgfpathlineto{\pgfqpoint{3.310023in}{2.707922in}}%
\pgfpathclose%
\pgfusepath{fill}%
\end{pgfscope}%
\begin{pgfscope}%
\pgfpathrectangle{\pgfqpoint{1.254980in}{0.150000in}}{\pgfqpoint{5.490039in}{5.490039in}}%
\pgfusepath{clip}%
\pgfsetbuttcap%
\pgfsetroundjoin%
\definecolor{currentfill}{rgb}{0.282623,0.140926,0.457517}%
\pgfsetfillcolor{currentfill}%
\pgfsetfillopacity{0.700000}%
\pgfsetlinewidth{0.000000pt}%
\definecolor{currentstroke}{rgb}{0.000000,0.000000,0.000000}%
\pgfsetstrokecolor{currentstroke}%
\pgfsetdash{}{0pt}%
\pgfpathmoveto{\pgfqpoint{3.922107in}{2.403707in}}%
\pgfpathlineto{\pgfqpoint{3.935124in}{2.396646in}}%
\pgfpathlineto{\pgfqpoint{3.948144in}{2.389724in}}%
\pgfpathlineto{\pgfqpoint{3.961168in}{2.382938in}}%
\pgfpathlineto{\pgfqpoint{3.974196in}{2.376290in}}%
\pgfpathlineto{\pgfqpoint{3.981759in}{2.385725in}}%
\pgfpathlineto{\pgfqpoint{3.989316in}{2.395198in}}%
\pgfpathlineto{\pgfqpoint{3.996869in}{2.404708in}}%
\pgfpathlineto{\pgfqpoint{4.004417in}{2.414254in}}%
\pgfpathlineto{\pgfqpoint{3.991401in}{2.420846in}}%
\pgfpathlineto{\pgfqpoint{3.978389in}{2.427575in}}%
\pgfpathlineto{\pgfqpoint{3.965381in}{2.434441in}}%
\pgfpathlineto{\pgfqpoint{3.952377in}{2.441445in}}%
\pgfpathlineto{\pgfqpoint{3.944817in}{2.431949in}}%
\pgfpathlineto{\pgfqpoint{3.937253in}{2.422494in}}%
\pgfpathlineto{\pgfqpoint{3.929683in}{2.413080in}}%
\pgfpathlineto{\pgfqpoint{3.922107in}{2.403707in}}%
\pgfpathclose%
\pgfusepath{fill}%
\end{pgfscope}%
\begin{pgfscope}%
\pgfpathrectangle{\pgfqpoint{1.254980in}{0.150000in}}{\pgfqpoint{5.490039in}{5.490039in}}%
\pgfusepath{clip}%
\pgfsetbuttcap%
\pgfsetroundjoin%
\definecolor{currentfill}{rgb}{0.276194,0.190074,0.493001}%
\pgfsetfillcolor{currentfill}%
\pgfsetfillopacity{0.700000}%
\pgfsetlinewidth{0.000000pt}%
\definecolor{currentstroke}{rgb}{0.000000,0.000000,0.000000}%
\pgfsetstrokecolor{currentstroke}%
\pgfsetdash{}{0pt}%
\pgfpathmoveto{\pgfqpoint{3.601044in}{2.504518in}}%
\pgfpathlineto{\pgfqpoint{3.614040in}{2.494413in}}%
\pgfpathlineto{\pgfqpoint{3.627036in}{2.484459in}}%
\pgfpathlineto{\pgfqpoint{3.640035in}{2.474655in}}%
\pgfpathlineto{\pgfqpoint{3.653034in}{2.465000in}}%
\pgfpathlineto{\pgfqpoint{3.660708in}{2.473653in}}%
\pgfpathlineto{\pgfqpoint{3.668376in}{2.482366in}}%
\pgfpathlineto{\pgfqpoint{3.676038in}{2.491139in}}%
\pgfpathlineto{\pgfqpoint{3.683694in}{2.499971in}}%
\pgfpathlineto{\pgfqpoint{3.670709in}{2.509537in}}%
\pgfpathlineto{\pgfqpoint{3.657726in}{2.519251in}}%
\pgfpathlineto{\pgfqpoint{3.644745in}{2.529116in}}%
\pgfpathlineto{\pgfqpoint{3.631765in}{2.539130in}}%
\pgfpathlineto{\pgfqpoint{3.624093in}{2.530382in}}%
\pgfpathlineto{\pgfqpoint{3.616416in}{2.521697in}}%
\pgfpathlineto{\pgfqpoint{3.608733in}{2.513076in}}%
\pgfpathlineto{\pgfqpoint{3.601044in}{2.504518in}}%
\pgfpathclose%
\pgfusepath{fill}%
\end{pgfscope}%
\begin{pgfscope}%
\pgfpathrectangle{\pgfqpoint{1.254980in}{0.150000in}}{\pgfqpoint{5.490039in}{5.490039in}}%
\pgfusepath{clip}%
\pgfsetbuttcap%
\pgfsetroundjoin%
\definecolor{currentfill}{rgb}{0.177423,0.437527,0.557565}%
\pgfsetfillcolor{currentfill}%
\pgfsetfillopacity{0.700000}%
\pgfsetlinewidth{0.000000pt}%
\definecolor{currentstroke}{rgb}{0.000000,0.000000,0.000000}%
\pgfsetstrokecolor{currentstroke}%
\pgfsetdash{}{0pt}%
\pgfpathmoveto{\pgfqpoint{2.996753in}{3.081815in}}%
\pgfpathlineto{\pgfqpoint{3.009853in}{3.064116in}}%
\pgfpathlineto{\pgfqpoint{3.022947in}{3.046610in}}%
\pgfpathlineto{\pgfqpoint{3.036036in}{3.029296in}}%
\pgfpathlineto{\pgfqpoint{3.049121in}{3.012172in}}%
\pgfpathlineto{\pgfqpoint{3.057026in}{3.019318in}}%
\pgfpathlineto{\pgfqpoint{3.064922in}{3.026570in}}%
\pgfpathlineto{\pgfqpoint{3.072811in}{3.033929in}}%
\pgfpathlineto{\pgfqpoint{3.080692in}{3.041393in}}%
\pgfpathlineto{\pgfqpoint{3.067630in}{3.058403in}}%
\pgfpathlineto{\pgfqpoint{3.054562in}{3.075604in}}%
\pgfpathlineto{\pgfqpoint{3.041490in}{3.092996in}}%
\pgfpathlineto{\pgfqpoint{3.028413in}{3.110581in}}%
\pgfpathlineto{\pgfqpoint{3.020510in}{3.103224in}}%
\pgfpathlineto{\pgfqpoint{3.012599in}{3.095977in}}%
\pgfpathlineto{\pgfqpoint{3.004680in}{3.088841in}}%
\pgfpathlineto{\pgfqpoint{2.996753in}{3.081815in}}%
\pgfpathclose%
\pgfusepath{fill}%
\end{pgfscope}%
\begin{pgfscope}%
\pgfpathrectangle{\pgfqpoint{1.254980in}{0.150000in}}{\pgfqpoint{5.490039in}{5.490039in}}%
\pgfusepath{clip}%
\pgfsetbuttcap%
\pgfsetroundjoin%
\definecolor{currentfill}{rgb}{0.282884,0.135920,0.453427}%
\pgfsetfillcolor{currentfill}%
\pgfsetfillopacity{0.700000}%
\pgfsetlinewidth{0.000000pt}%
\definecolor{currentstroke}{rgb}{0.000000,0.000000,0.000000}%
\pgfsetstrokecolor{currentstroke}%
\pgfsetdash{}{0pt}%
\pgfpathmoveto{\pgfqpoint{4.056523in}{2.389243in}}%
\pgfpathlineto{\pgfqpoint{4.069561in}{2.383327in}}%
\pgfpathlineto{\pgfqpoint{4.082604in}{2.377544in}}%
\pgfpathlineto{\pgfqpoint{4.095651in}{2.371895in}}%
\pgfpathlineto{\pgfqpoint{4.108703in}{2.366378in}}%
\pgfpathlineto{\pgfqpoint{4.116223in}{2.376054in}}%
\pgfpathlineto{\pgfqpoint{4.123738in}{2.385759in}}%
\pgfpathlineto{\pgfqpoint{4.131249in}{2.395492in}}%
\pgfpathlineto{\pgfqpoint{4.138754in}{2.405254in}}%
\pgfpathlineto{\pgfqpoint{4.125713in}{2.410731in}}%
\pgfpathlineto{\pgfqpoint{4.112677in}{2.416339in}}%
\pgfpathlineto{\pgfqpoint{4.099645in}{2.422082in}}%
\pgfpathlineto{\pgfqpoint{4.086619in}{2.427957in}}%
\pgfpathlineto{\pgfqpoint{4.079102in}{2.418230in}}%
\pgfpathlineto{\pgfqpoint{4.071581in}{2.408535in}}%
\pgfpathlineto{\pgfqpoint{4.064055in}{2.398872in}}%
\pgfpathlineto{\pgfqpoint{4.056523in}{2.389243in}}%
\pgfpathclose%
\pgfusepath{fill}%
\end{pgfscope}%
\begin{pgfscope}%
\pgfpathrectangle{\pgfqpoint{1.254980in}{0.150000in}}{\pgfqpoint{5.490039in}{5.490039in}}%
\pgfusepath{clip}%
\pgfsetbuttcap%
\pgfsetroundjoin%
\definecolor{currentfill}{rgb}{0.281412,0.155834,0.469201}%
\pgfsetfillcolor{currentfill}%
\pgfsetfillopacity{0.700000}%
\pgfsetlinewidth{0.000000pt}%
\definecolor{currentstroke}{rgb}{0.000000,0.000000,0.000000}%
\pgfsetstrokecolor{currentstroke}%
\pgfsetdash{}{0pt}%
\pgfpathmoveto{\pgfqpoint{3.787641in}{2.428723in}}%
\pgfpathlineto{\pgfqpoint{3.800645in}{2.420467in}}%
\pgfpathlineto{\pgfqpoint{3.813651in}{2.412353in}}%
\pgfpathlineto{\pgfqpoint{3.826661in}{2.404382in}}%
\pgfpathlineto{\pgfqpoint{3.839673in}{2.396552in}}%
\pgfpathlineto{\pgfqpoint{3.847282in}{2.405679in}}%
\pgfpathlineto{\pgfqpoint{3.854886in}{2.414853in}}%
\pgfpathlineto{\pgfqpoint{3.862484in}{2.424072in}}%
\pgfpathlineto{\pgfqpoint{3.870077in}{2.433337in}}%
\pgfpathlineto{\pgfqpoint{3.857078in}{2.441095in}}%
\pgfpathlineto{\pgfqpoint{3.844082in}{2.448994in}}%
\pgfpathlineto{\pgfqpoint{3.831088in}{2.457034in}}%
\pgfpathlineto{\pgfqpoint{3.818098in}{2.465217in}}%
\pgfpathlineto{\pgfqpoint{3.810492in}{2.456019in}}%
\pgfpathlineto{\pgfqpoint{3.802880in}{2.446870in}}%
\pgfpathlineto{\pgfqpoint{3.795263in}{2.437771in}}%
\pgfpathlineto{\pgfqpoint{3.787641in}{2.428723in}}%
\pgfpathclose%
\pgfusepath{fill}%
\end{pgfscope}%
\begin{pgfscope}%
\pgfpathrectangle{\pgfqpoint{1.254980in}{0.150000in}}{\pgfqpoint{5.490039in}{5.490039in}}%
\pgfusepath{clip}%
\pgfsetbuttcap%
\pgfsetroundjoin%
\definecolor{currentfill}{rgb}{0.210503,0.363727,0.552206}%
\pgfsetfillcolor{currentfill}%
\pgfsetfillopacity{0.700000}%
\pgfsetlinewidth{0.000000pt}%
\definecolor{currentstroke}{rgb}{0.000000,0.000000,0.000000}%
\pgfsetstrokecolor{currentstroke}%
\pgfsetdash{}{0pt}%
\pgfpathmoveto{\pgfqpoint{5.446636in}{2.856962in}}%
\pgfpathlineto{\pgfqpoint{5.460125in}{2.858714in}}%
\pgfpathlineto{\pgfqpoint{5.473627in}{2.860578in}}%
\pgfpathlineto{\pgfqpoint{5.487140in}{2.862555in}}%
\pgfpathlineto{\pgfqpoint{5.500665in}{2.864643in}}%
\pgfpathlineto{\pgfqpoint{5.507725in}{2.873693in}}%
\pgfpathlineto{\pgfqpoint{5.514780in}{2.882758in}}%
\pgfpathlineto{\pgfqpoint{5.521831in}{2.891843in}}%
\pgfpathlineto{\pgfqpoint{5.528879in}{2.900947in}}%
\pgfpathlineto{\pgfqpoint{5.515368in}{2.899059in}}%
\pgfpathlineto{\pgfqpoint{5.501869in}{2.897283in}}%
\pgfpathlineto{\pgfqpoint{5.488382in}{2.895618in}}%
\pgfpathlineto{\pgfqpoint{5.474906in}{2.894066in}}%
\pgfpathlineto{\pgfqpoint{5.467845in}{2.884755in}}%
\pgfpathlineto{\pgfqpoint{5.460779in}{2.875469in}}%
\pgfpathlineto{\pgfqpoint{5.453709in}{2.866206in}}%
\pgfpathlineto{\pgfqpoint{5.446636in}{2.856962in}}%
\pgfpathclose%
\pgfusepath{fill}%
\end{pgfscope}%
\begin{pgfscope}%
\pgfpathrectangle{\pgfqpoint{1.254980in}{0.150000in}}{\pgfqpoint{5.490039in}{5.490039in}}%
\pgfusepath{clip}%
\pgfsetbuttcap%
\pgfsetroundjoin%
\definecolor{currentfill}{rgb}{0.201239,0.383670,0.554294}%
\pgfsetfillcolor{currentfill}%
\pgfsetfillopacity{0.700000}%
\pgfsetlinewidth{0.000000pt}%
\definecolor{currentstroke}{rgb}{0.000000,0.000000,0.000000}%
\pgfsetstrokecolor{currentstroke}%
\pgfsetdash{}{0pt}%
\pgfpathmoveto{\pgfqpoint{5.528879in}{2.900947in}}%
\pgfpathlineto{\pgfqpoint{5.542402in}{2.902948in}}%
\pgfpathlineto{\pgfqpoint{5.555937in}{2.905060in}}%
\pgfpathlineto{\pgfqpoint{5.569485in}{2.907283in}}%
\pgfpathlineto{\pgfqpoint{5.583045in}{2.909618in}}%
\pgfpathlineto{\pgfqpoint{5.590073in}{2.918536in}}%
\pgfpathlineto{\pgfqpoint{5.597098in}{2.927474in}}%
\pgfpathlineto{\pgfqpoint{5.604119in}{2.936435in}}%
\pgfpathlineto{\pgfqpoint{5.611136in}{2.945421in}}%
\pgfpathlineto{\pgfqpoint{5.597592in}{2.943303in}}%
\pgfpathlineto{\pgfqpoint{5.584059in}{2.941295in}}%
\pgfpathlineto{\pgfqpoint{5.570539in}{2.939399in}}%
\pgfpathlineto{\pgfqpoint{5.557031in}{2.937615in}}%
\pgfpathlineto{\pgfqpoint{5.549998in}{2.928406in}}%
\pgfpathlineto{\pgfqpoint{5.542962in}{2.919227in}}%
\pgfpathlineto{\pgfqpoint{5.535922in}{2.910075in}}%
\pgfpathlineto{\pgfqpoint{5.528879in}{2.900947in}}%
\pgfpathclose%
\pgfusepath{fill}%
\end{pgfscope}%
\begin{pgfscope}%
\pgfpathrectangle{\pgfqpoint{1.254980in}{0.150000in}}{\pgfqpoint{5.490039in}{5.490039in}}%
\pgfusepath{clip}%
\pgfsetbuttcap%
\pgfsetroundjoin%
\definecolor{currentfill}{rgb}{0.218130,0.347432,0.550038}%
\pgfsetfillcolor{currentfill}%
\pgfsetfillopacity{0.700000}%
\pgfsetlinewidth{0.000000pt}%
\definecolor{currentstroke}{rgb}{0.000000,0.000000,0.000000}%
\pgfsetstrokecolor{currentstroke}%
\pgfsetdash{}{0pt}%
\pgfpathmoveto{\pgfqpoint{5.364407in}{2.813537in}}%
\pgfpathlineto{\pgfqpoint{5.377863in}{2.815022in}}%
\pgfpathlineto{\pgfqpoint{5.391331in}{2.816619in}}%
\pgfpathlineto{\pgfqpoint{5.404810in}{2.818329in}}%
\pgfpathlineto{\pgfqpoint{5.418302in}{2.820152in}}%
\pgfpathlineto{\pgfqpoint{5.425391in}{2.829334in}}%
\pgfpathlineto{\pgfqpoint{5.432477in}{2.838528in}}%
\pgfpathlineto{\pgfqpoint{5.439558in}{2.847737in}}%
\pgfpathlineto{\pgfqpoint{5.446636in}{2.856962in}}%
\pgfpathlineto{\pgfqpoint{5.433158in}{2.855323in}}%
\pgfpathlineto{\pgfqpoint{5.419692in}{2.853797in}}%
\pgfpathlineto{\pgfqpoint{5.406238in}{2.852383in}}%
\pgfpathlineto{\pgfqpoint{5.392795in}{2.851082in}}%
\pgfpathlineto{\pgfqpoint{5.385704in}{2.841667in}}%
\pgfpathlineto{\pgfqpoint{5.378609in}{2.832273in}}%
\pgfpathlineto{\pgfqpoint{5.371510in}{2.822897in}}%
\pgfpathlineto{\pgfqpoint{5.364407in}{2.813537in}}%
\pgfpathclose%
\pgfusepath{fill}%
\end{pgfscope}%
\begin{pgfscope}%
\pgfpathrectangle{\pgfqpoint{1.254980in}{0.150000in}}{\pgfqpoint{5.490039in}{5.490039in}}%
\pgfusepath{clip}%
\pgfsetbuttcap%
\pgfsetroundjoin%
\definecolor{currentfill}{rgb}{0.194100,0.399323,0.555565}%
\pgfsetfillcolor{currentfill}%
\pgfsetfillopacity{0.700000}%
\pgfsetlinewidth{0.000000pt}%
\definecolor{currentstroke}{rgb}{0.000000,0.000000,0.000000}%
\pgfsetstrokecolor{currentstroke}%
\pgfsetdash{}{0pt}%
\pgfpathmoveto{\pgfqpoint{5.611136in}{2.945421in}}%
\pgfpathlineto{\pgfqpoint{5.624693in}{2.947651in}}%
\pgfpathlineto{\pgfqpoint{5.638263in}{2.949992in}}%
\pgfpathlineto{\pgfqpoint{5.651845in}{2.952444in}}%
\pgfpathlineto{\pgfqpoint{5.665439in}{2.955007in}}%
\pgfpathlineto{\pgfqpoint{5.672437in}{2.963794in}}%
\pgfpathlineto{\pgfqpoint{5.679431in}{2.972608in}}%
\pgfpathlineto{\pgfqpoint{5.686421in}{2.981450in}}%
\pgfpathlineto{\pgfqpoint{5.693408in}{2.990322in}}%
\pgfpathlineto{\pgfqpoint{5.679830in}{2.987992in}}%
\pgfpathlineto{\pgfqpoint{5.666264in}{2.985773in}}%
\pgfpathlineto{\pgfqpoint{5.652710in}{2.983664in}}%
\pgfpathlineto{\pgfqpoint{5.639169in}{2.981666in}}%
\pgfpathlineto{\pgfqpoint{5.632166in}{2.972555in}}%
\pgfpathlineto{\pgfqpoint{5.625160in}{2.963479in}}%
\pgfpathlineto{\pgfqpoint{5.618150in}{2.954435in}}%
\pgfpathlineto{\pgfqpoint{5.611136in}{2.945421in}}%
\pgfpathclose%
\pgfusepath{fill}%
\end{pgfscope}%
\begin{pgfscope}%
\pgfpathrectangle{\pgfqpoint{1.254980in}{0.150000in}}{\pgfqpoint{5.490039in}{5.490039in}}%
\pgfusepath{clip}%
\pgfsetbuttcap%
\pgfsetroundjoin%
\definecolor{currentfill}{rgb}{0.227802,0.326594,0.546532}%
\pgfsetfillcolor{currentfill}%
\pgfsetfillopacity{0.700000}%
\pgfsetlinewidth{0.000000pt}%
\definecolor{currentstroke}{rgb}{0.000000,0.000000,0.000000}%
\pgfsetstrokecolor{currentstroke}%
\pgfsetdash{}{0pt}%
\pgfpathmoveto{\pgfqpoint{5.282191in}{2.770754in}}%
\pgfpathlineto{\pgfqpoint{5.295614in}{2.771951in}}%
\pgfpathlineto{\pgfqpoint{5.309049in}{2.773262in}}%
\pgfpathlineto{\pgfqpoint{5.322495in}{2.774688in}}%
\pgfpathlineto{\pgfqpoint{5.335953in}{2.776226in}}%
\pgfpathlineto{\pgfqpoint{5.343073in}{2.785538in}}%
\pgfpathlineto{\pgfqpoint{5.350188in}{2.794860in}}%
\pgfpathlineto{\pgfqpoint{5.357300in}{2.804192in}}%
\pgfpathlineto{\pgfqpoint{5.364407in}{2.813537in}}%
\pgfpathlineto{\pgfqpoint{5.350962in}{2.812167in}}%
\pgfpathlineto{\pgfqpoint{5.337529in}{2.810909in}}%
\pgfpathlineto{\pgfqpoint{5.324107in}{2.809765in}}%
\pgfpathlineto{\pgfqpoint{5.310696in}{2.808735in}}%
\pgfpathlineto{\pgfqpoint{5.303576in}{2.799216in}}%
\pgfpathlineto{\pgfqpoint{5.296451in}{2.789715in}}%
\pgfpathlineto{\pgfqpoint{5.289323in}{2.780228in}}%
\pgfpathlineto{\pgfqpoint{5.282191in}{2.770754in}}%
\pgfpathclose%
\pgfusepath{fill}%
\end{pgfscope}%
\begin{pgfscope}%
\pgfpathrectangle{\pgfqpoint{1.254980in}{0.150000in}}{\pgfqpoint{5.490039in}{5.490039in}}%
\pgfusepath{clip}%
\pgfsetbuttcap%
\pgfsetroundjoin%
\definecolor{currentfill}{rgb}{0.255645,0.260703,0.528312}%
\pgfsetfillcolor{currentfill}%
\pgfsetfillopacity{0.700000}%
\pgfsetlinewidth{0.000000pt}%
\definecolor{currentstroke}{rgb}{0.000000,0.000000,0.000000}%
\pgfsetstrokecolor{currentstroke}%
\pgfsetdash{}{0pt}%
\pgfpathmoveto{\pgfqpoint{3.362084in}{2.655382in}}%
\pgfpathlineto{\pgfqpoint{3.375096in}{2.642660in}}%
\pgfpathlineto{\pgfqpoint{3.388107in}{2.630101in}}%
\pgfpathlineto{\pgfqpoint{3.401117in}{2.617704in}}%
\pgfpathlineto{\pgfqpoint{3.414126in}{2.605468in}}%
\pgfpathlineto{\pgfqpoint{3.421890in}{2.613427in}}%
\pgfpathlineto{\pgfqpoint{3.429648in}{2.621464in}}%
\pgfpathlineto{\pgfqpoint{3.437399in}{2.629578in}}%
\pgfpathlineto{\pgfqpoint{3.445144in}{2.637770in}}%
\pgfpathlineto{\pgfqpoint{3.432152in}{2.649898in}}%
\pgfpathlineto{\pgfqpoint{3.419160in}{2.662187in}}%
\pgfpathlineto{\pgfqpoint{3.406167in}{2.674638in}}%
\pgfpathlineto{\pgfqpoint{3.393173in}{2.687252in}}%
\pgfpathlineto{\pgfqpoint{3.385411in}{2.679163in}}%
\pgfpathlineto{\pgfqpoint{3.377642in}{2.671154in}}%
\pgfpathlineto{\pgfqpoint{3.369866in}{2.663227in}}%
\pgfpathlineto{\pgfqpoint{3.362084in}{2.655382in}}%
\pgfpathclose%
\pgfusepath{fill}%
\end{pgfscope}%
\begin{pgfscope}%
\pgfpathrectangle{\pgfqpoint{1.254980in}{0.150000in}}{\pgfqpoint{5.490039in}{5.490039in}}%
\pgfusepath{clip}%
\pgfsetbuttcap%
\pgfsetroundjoin%
\definecolor{currentfill}{rgb}{0.235526,0.309527,0.542944}%
\pgfsetfillcolor{currentfill}%
\pgfsetfillopacity{0.700000}%
\pgfsetlinewidth{0.000000pt}%
\definecolor{currentstroke}{rgb}{0.000000,0.000000,0.000000}%
\pgfsetstrokecolor{currentstroke}%
\pgfsetdash{}{0pt}%
\pgfpathmoveto{\pgfqpoint{5.199986in}{2.728701in}}%
\pgfpathlineto{\pgfqpoint{5.213378in}{2.729593in}}%
\pgfpathlineto{\pgfqpoint{5.226780in}{2.730599in}}%
\pgfpathlineto{\pgfqpoint{5.240194in}{2.731720in}}%
\pgfpathlineto{\pgfqpoint{5.253618in}{2.732955in}}%
\pgfpathlineto{\pgfqpoint{5.260768in}{2.742393in}}%
\pgfpathlineto{\pgfqpoint{5.267913in}{2.751838in}}%
\pgfpathlineto{\pgfqpoint{5.275054in}{2.761291in}}%
\pgfpathlineto{\pgfqpoint{5.282191in}{2.770754in}}%
\pgfpathlineto{\pgfqpoint{5.268778in}{2.769670in}}%
\pgfpathlineto{\pgfqpoint{5.255377in}{2.768701in}}%
\pgfpathlineto{\pgfqpoint{5.241987in}{2.767846in}}%
\pgfpathlineto{\pgfqpoint{5.228607in}{2.767105in}}%
\pgfpathlineto{\pgfqpoint{5.221458in}{2.757486in}}%
\pgfpathlineto{\pgfqpoint{5.214305in}{2.747880in}}%
\pgfpathlineto{\pgfqpoint{5.207148in}{2.738285in}}%
\pgfpathlineto{\pgfqpoint{5.199986in}{2.728701in}}%
\pgfpathclose%
\pgfusepath{fill}%
\end{pgfscope}%
\begin{pgfscope}%
\pgfpathrectangle{\pgfqpoint{1.254980in}{0.150000in}}{\pgfqpoint{5.490039in}{5.490039in}}%
\pgfusepath{clip}%
\pgfsetbuttcap%
\pgfsetroundjoin%
\definecolor{currentfill}{rgb}{0.280255,0.165693,0.476498}%
\pgfsetfillcolor{currentfill}%
\pgfsetfillopacity{0.700000}%
\pgfsetlinewidth{0.000000pt}%
\definecolor{currentstroke}{rgb}{0.000000,0.000000,0.000000}%
\pgfsetstrokecolor{currentstroke}%
\pgfsetdash{}{0pt}%
\pgfpathmoveto{\pgfqpoint{4.489879in}{2.442875in}}%
\pgfpathlineto{\pgfqpoint{4.503021in}{2.440115in}}%
\pgfpathlineto{\pgfqpoint{4.516170in}{2.437478in}}%
\pgfpathlineto{\pgfqpoint{4.529327in}{2.434965in}}%
\pgfpathlineto{\pgfqpoint{4.542491in}{2.432576in}}%
\pgfpathlineto{\pgfqpoint{4.549876in}{2.442622in}}%
\pgfpathlineto{\pgfqpoint{4.557256in}{2.452676in}}%
\pgfpathlineto{\pgfqpoint{4.564632in}{2.462739in}}%
\pgfpathlineto{\pgfqpoint{4.572004in}{2.472812in}}%
\pgfpathlineto{\pgfqpoint{4.558849in}{2.475225in}}%
\pgfpathlineto{\pgfqpoint{4.545702in}{2.477761in}}%
\pgfpathlineto{\pgfqpoint{4.532562in}{2.480421in}}%
\pgfpathlineto{\pgfqpoint{4.519430in}{2.483204in}}%
\pgfpathlineto{\pgfqpoint{4.512049in}{2.473103in}}%
\pgfpathlineto{\pgfqpoint{4.504664in}{2.463014in}}%
\pgfpathlineto{\pgfqpoint{4.497274in}{2.452938in}}%
\pgfpathlineto{\pgfqpoint{4.489879in}{2.442875in}}%
\pgfpathclose%
\pgfusepath{fill}%
\end{pgfscope}%
\begin{pgfscope}%
\pgfpathrectangle{\pgfqpoint{1.254980in}{0.150000in}}{\pgfqpoint{5.490039in}{5.490039in}}%
\pgfusepath{clip}%
\pgfsetbuttcap%
\pgfsetroundjoin%
\definecolor{currentfill}{rgb}{0.243113,0.292092,0.538516}%
\pgfsetfillcolor{currentfill}%
\pgfsetfillopacity{0.700000}%
\pgfsetlinewidth{0.000000pt}%
\definecolor{currentstroke}{rgb}{0.000000,0.000000,0.000000}%
\pgfsetstrokecolor{currentstroke}%
\pgfsetdash{}{0pt}%
\pgfpathmoveto{\pgfqpoint{5.117793in}{2.687480in}}%
\pgfpathlineto{\pgfqpoint{5.131153in}{2.688046in}}%
\pgfpathlineto{\pgfqpoint{5.144524in}{2.688728in}}%
\pgfpathlineto{\pgfqpoint{5.157905in}{2.689525in}}%
\pgfpathlineto{\pgfqpoint{5.171297in}{2.690437in}}%
\pgfpathlineto{\pgfqpoint{5.178476in}{2.699995in}}%
\pgfpathlineto{\pgfqpoint{5.185651in}{2.709558in}}%
\pgfpathlineto{\pgfqpoint{5.192821in}{2.719126in}}%
\pgfpathlineto{\pgfqpoint{5.199986in}{2.728701in}}%
\pgfpathlineto{\pgfqpoint{5.186606in}{2.727925in}}%
\pgfpathlineto{\pgfqpoint{5.173236in}{2.727263in}}%
\pgfpathlineto{\pgfqpoint{5.159877in}{2.726716in}}%
\pgfpathlineto{\pgfqpoint{5.146528in}{2.726285in}}%
\pgfpathlineto{\pgfqpoint{5.139351in}{2.716569in}}%
\pgfpathlineto{\pgfqpoint{5.132169in}{2.706863in}}%
\pgfpathlineto{\pgfqpoint{5.124983in}{2.697168in}}%
\pgfpathlineto{\pgfqpoint{5.117793in}{2.687480in}}%
\pgfpathclose%
\pgfusepath{fill}%
\end{pgfscope}%
\begin{pgfscope}%
\pgfpathrectangle{\pgfqpoint{1.254980in}{0.150000in}}{\pgfqpoint{5.490039in}{5.490039in}}%
\pgfusepath{clip}%
\pgfsetbuttcap%
\pgfsetroundjoin%
\definecolor{currentfill}{rgb}{0.165117,0.467423,0.558141}%
\pgfsetfillcolor{currentfill}%
\pgfsetfillopacity{0.700000}%
\pgfsetlinewidth{0.000000pt}%
\definecolor{currentstroke}{rgb}{0.000000,0.000000,0.000000}%
\pgfsetstrokecolor{currentstroke}%
\pgfsetdash{}{0pt}%
\pgfpathmoveto{\pgfqpoint{2.944298in}{3.154566in}}%
\pgfpathlineto{\pgfqpoint{2.957421in}{3.136082in}}%
\pgfpathlineto{\pgfqpoint{2.970537in}{3.117797in}}%
\pgfpathlineto{\pgfqpoint{2.983648in}{3.099708in}}%
\pgfpathlineto{\pgfqpoint{2.996753in}{3.081815in}}%
\pgfpathlineto{\pgfqpoint{3.004680in}{3.088841in}}%
\pgfpathlineto{\pgfqpoint{3.012599in}{3.095977in}}%
\pgfpathlineto{\pgfqpoint{3.020510in}{3.103224in}}%
\pgfpathlineto{\pgfqpoint{3.028413in}{3.110581in}}%
\pgfpathlineto{\pgfqpoint{3.015331in}{3.128360in}}%
\pgfpathlineto{\pgfqpoint{3.002243in}{3.146335in}}%
\pgfpathlineto{\pgfqpoint{2.989150in}{3.164506in}}%
\pgfpathlineto{\pgfqpoint{2.976051in}{3.182875in}}%
\pgfpathlineto{\pgfqpoint{2.968125in}{3.175626in}}%
\pgfpathlineto{\pgfqpoint{2.960191in}{3.168491in}}%
\pgfpathlineto{\pgfqpoint{2.952249in}{3.161471in}}%
\pgfpathlineto{\pgfqpoint{2.944298in}{3.154566in}}%
\pgfpathclose%
\pgfusepath{fill}%
\end{pgfscope}%
\begin{pgfscope}%
\pgfpathrectangle{\pgfqpoint{1.254980in}{0.150000in}}{\pgfqpoint{5.490039in}{5.490039in}}%
\pgfusepath{clip}%
\pgfsetbuttcap%
\pgfsetroundjoin%
\definecolor{currentfill}{rgb}{0.252194,0.269783,0.531579}%
\pgfsetfillcolor{currentfill}%
\pgfsetfillopacity{0.700000}%
\pgfsetlinewidth{0.000000pt}%
\definecolor{currentstroke}{rgb}{0.000000,0.000000,0.000000}%
\pgfsetstrokecolor{currentstroke}%
\pgfsetdash{}{0pt}%
\pgfpathmoveto{\pgfqpoint{5.035608in}{2.647199in}}%
\pgfpathlineto{\pgfqpoint{5.048938in}{2.647421in}}%
\pgfpathlineto{\pgfqpoint{5.062277in}{2.647758in}}%
\pgfpathlineto{\pgfqpoint{5.075628in}{2.648212in}}%
\pgfpathlineto{\pgfqpoint{5.088988in}{2.648782in}}%
\pgfpathlineto{\pgfqpoint{5.096196in}{2.658452in}}%
\pgfpathlineto{\pgfqpoint{5.103399in}{2.668123in}}%
\pgfpathlineto{\pgfqpoint{5.110598in}{2.677799in}}%
\pgfpathlineto{\pgfqpoint{5.117793in}{2.687480in}}%
\pgfpathlineto{\pgfqpoint{5.104443in}{2.687029in}}%
\pgfpathlineto{\pgfqpoint{5.091104in}{2.686695in}}%
\pgfpathlineto{\pgfqpoint{5.077775in}{2.686476in}}%
\pgfpathlineto{\pgfqpoint{5.064456in}{2.686374in}}%
\pgfpathlineto{\pgfqpoint{5.057251in}{2.676568in}}%
\pgfpathlineto{\pgfqpoint{5.050041in}{2.666771in}}%
\pgfpathlineto{\pgfqpoint{5.042827in}{2.656982in}}%
\pgfpathlineto{\pgfqpoint{5.035608in}{2.647199in}}%
\pgfpathclose%
\pgfusepath{fill}%
\end{pgfscope}%
\begin{pgfscope}%
\pgfpathrectangle{\pgfqpoint{1.254980in}{0.150000in}}{\pgfqpoint{5.490039in}{5.490039in}}%
\pgfusepath{clip}%
\pgfsetbuttcap%
\pgfsetroundjoin%
\definecolor{currentfill}{rgb}{0.185556,0.418570,0.556753}%
\pgfsetfillcolor{currentfill}%
\pgfsetfillopacity{0.700000}%
\pgfsetlinewidth{0.000000pt}%
\definecolor{currentstroke}{rgb}{0.000000,0.000000,0.000000}%
\pgfsetstrokecolor{currentstroke}%
\pgfsetdash{}{0pt}%
\pgfpathmoveto{\pgfqpoint{5.693408in}{2.990322in}}%
\pgfpathlineto{\pgfqpoint{5.706999in}{2.992763in}}%
\pgfpathlineto{\pgfqpoint{5.720603in}{2.995314in}}%
\pgfpathlineto{\pgfqpoint{5.734220in}{2.997975in}}%
\pgfpathlineto{\pgfqpoint{5.747850in}{3.000747in}}%
\pgfpathlineto{\pgfqpoint{5.754816in}{3.009410in}}%
\pgfpathlineto{\pgfqpoint{5.761779in}{3.018104in}}%
\pgfpathlineto{\pgfqpoint{5.768739in}{3.026832in}}%
\pgfpathlineto{\pgfqpoint{5.755122in}{3.024246in}}%
\pgfpathlineto{\pgfqpoint{5.741518in}{3.021770in}}%
\pgfpathlineto{\pgfqpoint{5.727927in}{3.019404in}}%
\pgfpathlineto{\pgfqpoint{5.714349in}{3.017148in}}%
\pgfpathlineto{\pgfqpoint{5.707372in}{3.008169in}}%
\pgfpathlineto{\pgfqpoint{5.700392in}{2.999228in}}%
\pgfpathlineto{\pgfqpoint{5.693408in}{2.990322in}}%
\pgfpathclose%
\pgfusepath{fill}%
\end{pgfscope}%
\begin{pgfscope}%
\pgfpathrectangle{\pgfqpoint{1.254980in}{0.150000in}}{\pgfqpoint{5.490039in}{5.490039in}}%
\pgfusepath{clip}%
\pgfsetbuttcap%
\pgfsetroundjoin%
\definecolor{currentfill}{rgb}{0.282884,0.135920,0.453427}%
\pgfsetfillcolor{currentfill}%
\pgfsetfillopacity{0.700000}%
\pgfsetlinewidth{0.000000pt}%
\definecolor{currentstroke}{rgb}{0.000000,0.000000,0.000000}%
\pgfsetstrokecolor{currentstroke}%
\pgfsetdash{}{0pt}%
\pgfpathmoveto{\pgfqpoint{4.190971in}{2.384668in}}%
\pgfpathlineto{\pgfqpoint{4.204039in}{2.379849in}}%
\pgfpathlineto{\pgfqpoint{4.217112in}{2.375159in}}%
\pgfpathlineto{\pgfqpoint{4.230191in}{2.370600in}}%
\pgfpathlineto{\pgfqpoint{4.243276in}{2.366170in}}%
\pgfpathlineto{\pgfqpoint{4.250756in}{2.376021in}}%
\pgfpathlineto{\pgfqpoint{4.258230in}{2.385893in}}%
\pgfpathlineto{\pgfqpoint{4.265700in}{2.395786in}}%
\pgfpathlineto{\pgfqpoint{4.273166in}{2.405701in}}%
\pgfpathlineto{\pgfqpoint{4.260091in}{2.410107in}}%
\pgfpathlineto{\pgfqpoint{4.247022in}{2.414642in}}%
\pgfpathlineto{\pgfqpoint{4.233959in}{2.419307in}}%
\pgfpathlineto{\pgfqpoint{4.220902in}{2.424101in}}%
\pgfpathlineto{\pgfqpoint{4.213426in}{2.414206in}}%
\pgfpathlineto{\pgfqpoint{4.205946in}{2.404335in}}%
\pgfpathlineto{\pgfqpoint{4.198461in}{2.394489in}}%
\pgfpathlineto{\pgfqpoint{4.190971in}{2.384668in}}%
\pgfpathclose%
\pgfusepath{fill}%
\end{pgfscope}%
\begin{pgfscope}%
\pgfpathrectangle{\pgfqpoint{1.254980in}{0.150000in}}{\pgfqpoint{5.490039in}{5.490039in}}%
\pgfusepath{clip}%
\pgfsetbuttcap%
\pgfsetroundjoin%
\definecolor{currentfill}{rgb}{0.258965,0.251537,0.524736}%
\pgfsetfillcolor{currentfill}%
\pgfsetfillopacity{0.700000}%
\pgfsetlinewidth{0.000000pt}%
\definecolor{currentstroke}{rgb}{0.000000,0.000000,0.000000}%
\pgfsetstrokecolor{currentstroke}%
\pgfsetdash{}{0pt}%
\pgfpathmoveto{\pgfqpoint{4.953429in}{2.607977in}}%
\pgfpathlineto{\pgfqpoint{4.966729in}{2.607834in}}%
\pgfpathlineto{\pgfqpoint{4.980039in}{2.607809in}}%
\pgfpathlineto{\pgfqpoint{4.993359in}{2.607900in}}%
\pgfpathlineto{\pgfqpoint{5.006689in}{2.608108in}}%
\pgfpathlineto{\pgfqpoint{5.013926in}{2.617877in}}%
\pgfpathlineto{\pgfqpoint{5.021157in}{2.627648in}}%
\pgfpathlineto{\pgfqpoint{5.028385in}{2.637422in}}%
\pgfpathlineto{\pgfqpoint{5.035608in}{2.647199in}}%
\pgfpathlineto{\pgfqpoint{5.022288in}{2.647094in}}%
\pgfpathlineto{\pgfqpoint{5.008979in}{2.647106in}}%
\pgfpathlineto{\pgfqpoint{4.995679in}{2.647235in}}%
\pgfpathlineto{\pgfqpoint{4.982390in}{2.647481in}}%
\pgfpathlineto{\pgfqpoint{4.975156in}{2.637595in}}%
\pgfpathlineto{\pgfqpoint{4.967918in}{2.627716in}}%
\pgfpathlineto{\pgfqpoint{4.960676in}{2.617844in}}%
\pgfpathlineto{\pgfqpoint{4.953429in}{2.607977in}}%
\pgfpathclose%
\pgfusepath{fill}%
\end{pgfscope}%
\begin{pgfscope}%
\pgfpathrectangle{\pgfqpoint{1.254980in}{0.150000in}}{\pgfqpoint{5.490039in}{5.490039in}}%
\pgfusepath{clip}%
\pgfsetbuttcap%
\pgfsetroundjoin%
\definecolor{currentfill}{rgb}{0.263663,0.237631,0.518762}%
\pgfsetfillcolor{currentfill}%
\pgfsetfillopacity{0.700000}%
\pgfsetlinewidth{0.000000pt}%
\definecolor{currentstroke}{rgb}{0.000000,0.000000,0.000000}%
\pgfsetstrokecolor{currentstroke}%
\pgfsetdash{}{0pt}%
\pgfpathmoveto{\pgfqpoint{3.414126in}{2.605468in}}%
\pgfpathlineto{\pgfqpoint{3.427135in}{2.593393in}}%
\pgfpathlineto{\pgfqpoint{3.440143in}{2.581478in}}%
\pgfpathlineto{\pgfqpoint{3.453151in}{2.569722in}}%
\pgfpathlineto{\pgfqpoint{3.466159in}{2.558124in}}%
\pgfpathlineto{\pgfqpoint{3.473905in}{2.566196in}}%
\pgfpathlineto{\pgfqpoint{3.481646in}{2.574342in}}%
\pgfpathlineto{\pgfqpoint{3.489379in}{2.582562in}}%
\pgfpathlineto{\pgfqpoint{3.497107in}{2.590854in}}%
\pgfpathlineto{\pgfqpoint{3.484116in}{2.602345in}}%
\pgfpathlineto{\pgfqpoint{3.471126in}{2.613994in}}%
\pgfpathlineto{\pgfqpoint{3.458135in}{2.625802in}}%
\pgfpathlineto{\pgfqpoint{3.445144in}{2.637770in}}%
\pgfpathlineto{\pgfqpoint{3.437399in}{2.629578in}}%
\pgfpathlineto{\pgfqpoint{3.429648in}{2.621464in}}%
\pgfpathlineto{\pgfqpoint{3.421890in}{2.613427in}}%
\pgfpathlineto{\pgfqpoint{3.414126in}{2.605468in}}%
\pgfpathclose%
\pgfusepath{fill}%
\end{pgfscope}%
\begin{pgfscope}%
\pgfpathrectangle{\pgfqpoint{1.254980in}{0.150000in}}{\pgfqpoint{5.490039in}{5.490039in}}%
\pgfusepath{clip}%
\pgfsetbuttcap%
\pgfsetroundjoin%
\definecolor{currentfill}{rgb}{0.278826,0.175490,0.483397}%
\pgfsetfillcolor{currentfill}%
\pgfsetfillopacity{0.700000}%
\pgfsetlinewidth{0.000000pt}%
\definecolor{currentstroke}{rgb}{0.000000,0.000000,0.000000}%
\pgfsetstrokecolor{currentstroke}%
\pgfsetdash{}{0pt}%
\pgfpathmoveto{\pgfqpoint{3.653034in}{2.465000in}}%
\pgfpathlineto{\pgfqpoint{3.666035in}{2.455492in}}%
\pgfpathlineto{\pgfqpoint{3.679038in}{2.446133in}}%
\pgfpathlineto{\pgfqpoint{3.692043in}{2.436920in}}%
\pgfpathlineto{\pgfqpoint{3.705049in}{2.427854in}}%
\pgfpathlineto{\pgfqpoint{3.712708in}{2.436603in}}%
\pgfpathlineto{\pgfqpoint{3.720361in}{2.445407in}}%
\pgfpathlineto{\pgfqpoint{3.728008in}{2.454267in}}%
\pgfpathlineto{\pgfqpoint{3.735650in}{2.463182in}}%
\pgfpathlineto{\pgfqpoint{3.722658in}{2.472160in}}%
\pgfpathlineto{\pgfqpoint{3.709668in}{2.481283in}}%
\pgfpathlineto{\pgfqpoint{3.696680in}{2.490553in}}%
\pgfpathlineto{\pgfqpoint{3.683694in}{2.499971in}}%
\pgfpathlineto{\pgfqpoint{3.676038in}{2.491139in}}%
\pgfpathlineto{\pgfqpoint{3.668376in}{2.482366in}}%
\pgfpathlineto{\pgfqpoint{3.660708in}{2.473653in}}%
\pgfpathlineto{\pgfqpoint{3.653034in}{2.465000in}}%
\pgfpathclose%
\pgfusepath{fill}%
\end{pgfscope}%
\begin{pgfscope}%
\pgfpathrectangle{\pgfqpoint{1.254980in}{0.150000in}}{\pgfqpoint{5.490039in}{5.490039in}}%
\pgfusepath{clip}%
\pgfsetbuttcap%
\pgfsetroundjoin%
\definecolor{currentfill}{rgb}{0.263663,0.237631,0.518762}%
\pgfsetfillcolor{currentfill}%
\pgfsetfillopacity{0.700000}%
\pgfsetlinewidth{0.000000pt}%
\definecolor{currentstroke}{rgb}{0.000000,0.000000,0.000000}%
\pgfsetstrokecolor{currentstroke}%
\pgfsetdash{}{0pt}%
\pgfpathmoveto{\pgfqpoint{4.871253in}{2.569943in}}%
\pgfpathlineto{\pgfqpoint{4.884525in}{2.569416in}}%
\pgfpathlineto{\pgfqpoint{4.897806in}{2.569007in}}%
\pgfpathlineto{\pgfqpoint{4.911097in}{2.568716in}}%
\pgfpathlineto{\pgfqpoint{4.924398in}{2.568542in}}%
\pgfpathlineto{\pgfqpoint{4.931662in}{2.578399in}}%
\pgfpathlineto{\pgfqpoint{4.938922in}{2.588256in}}%
\pgfpathlineto{\pgfqpoint{4.946178in}{2.598115in}}%
\pgfpathlineto{\pgfqpoint{4.953429in}{2.607977in}}%
\pgfpathlineto{\pgfqpoint{4.940139in}{2.608238in}}%
\pgfpathlineto{\pgfqpoint{4.926858in}{2.608616in}}%
\pgfpathlineto{\pgfqpoint{4.913587in}{2.609112in}}%
\pgfpathlineto{\pgfqpoint{4.900325in}{2.609726in}}%
\pgfpathlineto{\pgfqpoint{4.893064in}{2.599771in}}%
\pgfpathlineto{\pgfqpoint{4.885798in}{2.589823in}}%
\pgfpathlineto{\pgfqpoint{4.878528in}{2.579880in}}%
\pgfpathlineto{\pgfqpoint{4.871253in}{2.569943in}}%
\pgfpathclose%
\pgfusepath{fill}%
\end{pgfscope}%
\begin{pgfscope}%
\pgfpathrectangle{\pgfqpoint{1.254980in}{0.150000in}}{\pgfqpoint{5.490039in}{5.490039in}}%
\pgfusepath{clip}%
\pgfsetbuttcap%
\pgfsetroundjoin%
\definecolor{currentfill}{rgb}{0.281412,0.155834,0.469201}%
\pgfsetfillcolor{currentfill}%
\pgfsetfillopacity{0.700000}%
\pgfsetlinewidth{0.000000pt}%
\definecolor{currentstroke}{rgb}{0.000000,0.000000,0.000000}%
\pgfsetstrokecolor{currentstroke}%
\pgfsetdash{}{0pt}%
\pgfpathmoveto{\pgfqpoint{4.407723in}{2.414995in}}%
\pgfpathlineto{\pgfqpoint{4.420846in}{2.411743in}}%
\pgfpathlineto{\pgfqpoint{4.433976in}{2.408616in}}%
\pgfpathlineto{\pgfqpoint{4.447112in}{2.405614in}}%
\pgfpathlineto{\pgfqpoint{4.460256in}{2.402737in}}%
\pgfpathlineto{\pgfqpoint{4.467669in}{2.412755in}}%
\pgfpathlineto{\pgfqpoint{4.475077in}{2.422784in}}%
\pgfpathlineto{\pgfqpoint{4.482480in}{2.432824in}}%
\pgfpathlineto{\pgfqpoint{4.489879in}{2.442875in}}%
\pgfpathlineto{\pgfqpoint{4.476745in}{2.445760in}}%
\pgfpathlineto{\pgfqpoint{4.463618in}{2.448769in}}%
\pgfpathlineto{\pgfqpoint{4.450498in}{2.451904in}}%
\pgfpathlineto{\pgfqpoint{4.437385in}{2.455163in}}%
\pgfpathlineto{\pgfqpoint{4.429976in}{2.445099in}}%
\pgfpathlineto{\pgfqpoint{4.422563in}{2.435049in}}%
\pgfpathlineto{\pgfqpoint{4.415145in}{2.425015in}}%
\pgfpathlineto{\pgfqpoint{4.407723in}{2.414995in}}%
\pgfpathclose%
\pgfusepath{fill}%
\end{pgfscope}%
\begin{pgfscope}%
\pgfpathrectangle{\pgfqpoint{1.254980in}{0.150000in}}{\pgfqpoint{5.490039in}{5.490039in}}%
\pgfusepath{clip}%
\pgfsetbuttcap%
\pgfsetroundjoin%
\definecolor{currentfill}{rgb}{0.153364,0.497000,0.557724}%
\pgfsetfillcolor{currentfill}%
\pgfsetfillopacity{0.700000}%
\pgfsetlinewidth{0.000000pt}%
\definecolor{currentstroke}{rgb}{0.000000,0.000000,0.000000}%
\pgfsetstrokecolor{currentstroke}%
\pgfsetdash{}{0pt}%
\pgfpathmoveto{\pgfqpoint{2.891746in}{3.230510in}}%
\pgfpathlineto{\pgfqpoint{2.904894in}{3.211220in}}%
\pgfpathlineto{\pgfqpoint{2.918035in}{3.192133in}}%
\pgfpathlineto{\pgfqpoint{2.931170in}{3.173249in}}%
\pgfpathlineto{\pgfqpoint{2.944298in}{3.154566in}}%
\pgfpathlineto{\pgfqpoint{2.952249in}{3.161471in}}%
\pgfpathlineto{\pgfqpoint{2.960191in}{3.168491in}}%
\pgfpathlineto{\pgfqpoint{2.968125in}{3.175626in}}%
\pgfpathlineto{\pgfqpoint{2.976051in}{3.182875in}}%
\pgfpathlineto{\pgfqpoint{2.962946in}{3.201443in}}%
\pgfpathlineto{\pgfqpoint{2.949835in}{3.220212in}}%
\pgfpathlineto{\pgfqpoint{2.936717in}{3.239183in}}%
\pgfpathlineto{\pgfqpoint{2.923594in}{3.258357in}}%
\pgfpathlineto{\pgfqpoint{2.915645in}{3.251218in}}%
\pgfpathlineto{\pgfqpoint{2.907687in}{3.244196in}}%
\pgfpathlineto{\pgfqpoint{2.899721in}{3.237294in}}%
\pgfpathlineto{\pgfqpoint{2.891746in}{3.230510in}}%
\pgfpathclose%
\pgfusepath{fill}%
\end{pgfscope}%
\begin{pgfscope}%
\pgfpathrectangle{\pgfqpoint{1.254980in}{0.150000in}}{\pgfqpoint{5.490039in}{5.490039in}}%
\pgfusepath{clip}%
\pgfsetbuttcap%
\pgfsetroundjoin%
\definecolor{currentfill}{rgb}{0.269308,0.218818,0.509577}%
\pgfsetfillcolor{currentfill}%
\pgfsetfillopacity{0.700000}%
\pgfsetlinewidth{0.000000pt}%
\definecolor{currentstroke}{rgb}{0.000000,0.000000,0.000000}%
\pgfsetstrokecolor{currentstroke}%
\pgfsetdash{}{0pt}%
\pgfpathmoveto{\pgfqpoint{4.789076in}{2.533233in}}%
\pgfpathlineto{\pgfqpoint{4.802321in}{2.532302in}}%
\pgfpathlineto{\pgfqpoint{4.815575in}{2.531489in}}%
\pgfpathlineto{\pgfqpoint{4.828838in}{2.530796in}}%
\pgfpathlineto{\pgfqpoint{4.842110in}{2.530222in}}%
\pgfpathlineto{\pgfqpoint{4.849403in}{2.540150in}}%
\pgfpathlineto{\pgfqpoint{4.856691in}{2.550078in}}%
\pgfpathlineto{\pgfqpoint{4.863974in}{2.560009in}}%
\pgfpathlineto{\pgfqpoint{4.871253in}{2.569943in}}%
\pgfpathlineto{\pgfqpoint{4.857991in}{2.570589in}}%
\pgfpathlineto{\pgfqpoint{4.844738in}{2.571353in}}%
\pgfpathlineto{\pgfqpoint{4.831494in}{2.572236in}}%
\pgfpathlineto{\pgfqpoint{4.818259in}{2.573239in}}%
\pgfpathlineto{\pgfqpoint{4.810970in}{2.563228in}}%
\pgfpathlineto{\pgfqpoint{4.803677in}{2.553224in}}%
\pgfpathlineto{\pgfqpoint{4.796379in}{2.543226in}}%
\pgfpathlineto{\pgfqpoint{4.789076in}{2.533233in}}%
\pgfpathclose%
\pgfusepath{fill}%
\end{pgfscope}%
\begin{pgfscope}%
\pgfpathrectangle{\pgfqpoint{1.254980in}{0.150000in}}{\pgfqpoint{5.490039in}{5.490039in}}%
\pgfusepath{clip}%
\pgfsetbuttcap%
\pgfsetroundjoin%
\definecolor{currentfill}{rgb}{0.282884,0.135920,0.453427}%
\pgfsetfillcolor{currentfill}%
\pgfsetfillopacity{0.700000}%
\pgfsetlinewidth{0.000000pt}%
\definecolor{currentstroke}{rgb}{0.000000,0.000000,0.000000}%
\pgfsetstrokecolor{currentstroke}%
\pgfsetdash{}{0pt}%
\pgfpathmoveto{\pgfqpoint{3.974196in}{2.376290in}}%
\pgfpathlineto{\pgfqpoint{3.987228in}{2.369778in}}%
\pgfpathlineto{\pgfqpoint{4.000264in}{2.363401in}}%
\pgfpathlineto{\pgfqpoint{4.013305in}{2.357160in}}%
\pgfpathlineto{\pgfqpoint{4.026350in}{2.351053in}}%
\pgfpathlineto{\pgfqpoint{4.033900in}{2.360551in}}%
\pgfpathlineto{\pgfqpoint{4.041446in}{2.370082in}}%
\pgfpathlineto{\pgfqpoint{4.048987in}{2.379646in}}%
\pgfpathlineto{\pgfqpoint{4.056523in}{2.389243in}}%
\pgfpathlineto{\pgfqpoint{4.043490in}{2.395293in}}%
\pgfpathlineto{\pgfqpoint{4.030461in}{2.401478in}}%
\pgfpathlineto{\pgfqpoint{4.017437in}{2.407798in}}%
\pgfpathlineto{\pgfqpoint{4.004417in}{2.414254in}}%
\pgfpathlineto{\pgfqpoint{3.996869in}{2.404708in}}%
\pgfpathlineto{\pgfqpoint{3.989316in}{2.395198in}}%
\pgfpathlineto{\pgfqpoint{3.981759in}{2.385725in}}%
\pgfpathlineto{\pgfqpoint{3.974196in}{2.376290in}}%
\pgfpathclose%
\pgfusepath{fill}%
\end{pgfscope}%
\begin{pgfscope}%
\pgfpathrectangle{\pgfqpoint{1.254980in}{0.150000in}}{\pgfqpoint{5.490039in}{5.490039in}}%
\pgfusepath{clip}%
\pgfsetbuttcap%
\pgfsetroundjoin%
\definecolor{currentfill}{rgb}{0.269308,0.218818,0.509577}%
\pgfsetfillcolor{currentfill}%
\pgfsetfillopacity{0.700000}%
\pgfsetlinewidth{0.000000pt}%
\definecolor{currentstroke}{rgb}{0.000000,0.000000,0.000000}%
\pgfsetstrokecolor{currentstroke}%
\pgfsetdash{}{0pt}%
\pgfpathmoveto{\pgfqpoint{3.466159in}{2.558124in}}%
\pgfpathlineto{\pgfqpoint{3.479166in}{2.546683in}}%
\pgfpathlineto{\pgfqpoint{3.492174in}{2.535399in}}%
\pgfpathlineto{\pgfqpoint{3.505182in}{2.524270in}}%
\pgfpathlineto{\pgfqpoint{3.518190in}{2.513297in}}%
\pgfpathlineto{\pgfqpoint{3.525920in}{2.521481in}}%
\pgfpathlineto{\pgfqpoint{3.533643in}{2.529736in}}%
\pgfpathlineto{\pgfqpoint{3.541360in}{2.538060in}}%
\pgfpathlineto{\pgfqpoint{3.549071in}{2.546454in}}%
\pgfpathlineto{\pgfqpoint{3.536079in}{2.557321in}}%
\pgfpathlineto{\pgfqpoint{3.523088in}{2.568343in}}%
\pgfpathlineto{\pgfqpoint{3.510098in}{2.579520in}}%
\pgfpathlineto{\pgfqpoint{3.497107in}{2.590854in}}%
\pgfpathlineto{\pgfqpoint{3.489379in}{2.582562in}}%
\pgfpathlineto{\pgfqpoint{3.481646in}{2.574342in}}%
\pgfpathlineto{\pgfqpoint{3.473905in}{2.566196in}}%
\pgfpathlineto{\pgfqpoint{3.466159in}{2.558124in}}%
\pgfpathclose%
\pgfusepath{fill}%
\end{pgfscope}%
\begin{pgfscope}%
\pgfpathrectangle{\pgfqpoint{1.254980in}{0.150000in}}{\pgfqpoint{5.490039in}{5.490039in}}%
\pgfusepath{clip}%
\pgfsetbuttcap%
\pgfsetroundjoin%
\definecolor{currentfill}{rgb}{0.282623,0.140926,0.457517}%
\pgfsetfillcolor{currentfill}%
\pgfsetfillopacity{0.700000}%
\pgfsetlinewidth{0.000000pt}%
\definecolor{currentstroke}{rgb}{0.000000,0.000000,0.000000}%
\pgfsetstrokecolor{currentstroke}%
\pgfsetdash{}{0pt}%
\pgfpathmoveto{\pgfqpoint{3.839673in}{2.396552in}}%
\pgfpathlineto{\pgfqpoint{3.852689in}{2.388862in}}%
\pgfpathlineto{\pgfqpoint{3.865708in}{2.381312in}}%
\pgfpathlineto{\pgfqpoint{3.878730in}{2.373902in}}%
\pgfpathlineto{\pgfqpoint{3.891755in}{2.366631in}}%
\pgfpathlineto{\pgfqpoint{3.899351in}{2.375837in}}%
\pgfpathlineto{\pgfqpoint{3.906942in}{2.385085in}}%
\pgfpathlineto{\pgfqpoint{3.914527in}{2.394375in}}%
\pgfpathlineto{\pgfqpoint{3.922107in}{2.403707in}}%
\pgfpathlineto{\pgfqpoint{3.909095in}{2.410905in}}%
\pgfpathlineto{\pgfqpoint{3.896085in}{2.418243in}}%
\pgfpathlineto{\pgfqpoint{3.883080in}{2.425720in}}%
\pgfpathlineto{\pgfqpoint{3.870077in}{2.433337in}}%
\pgfpathlineto{\pgfqpoint{3.862484in}{2.424072in}}%
\pgfpathlineto{\pgfqpoint{3.854886in}{2.414853in}}%
\pgfpathlineto{\pgfqpoint{3.847282in}{2.405679in}}%
\pgfpathlineto{\pgfqpoint{3.839673in}{2.396552in}}%
\pgfpathclose%
\pgfusepath{fill}%
\end{pgfscope}%
\begin{pgfscope}%
\pgfpathrectangle{\pgfqpoint{1.254980in}{0.150000in}}{\pgfqpoint{5.490039in}{5.490039in}}%
\pgfusepath{clip}%
\pgfsetbuttcap%
\pgfsetroundjoin%
\definecolor{currentfill}{rgb}{0.273006,0.204520,0.501721}%
\pgfsetfillcolor{currentfill}%
\pgfsetfillopacity{0.700000}%
\pgfsetlinewidth{0.000000pt}%
\definecolor{currentstroke}{rgb}{0.000000,0.000000,0.000000}%
\pgfsetstrokecolor{currentstroke}%
\pgfsetdash{}{0pt}%
\pgfpathmoveto{\pgfqpoint{4.706894in}{2.497996in}}%
\pgfpathlineto{\pgfqpoint{4.720113in}{2.496640in}}%
\pgfpathlineto{\pgfqpoint{4.733341in}{2.495404in}}%
\pgfpathlineto{\pgfqpoint{4.746577in}{2.494289in}}%
\pgfpathlineto{\pgfqpoint{4.759823in}{2.493293in}}%
\pgfpathlineto{\pgfqpoint{4.767143in}{2.503275in}}%
\pgfpathlineto{\pgfqpoint{4.774459in}{2.513258in}}%
\pgfpathlineto{\pgfqpoint{4.781770in}{2.523244in}}%
\pgfpathlineto{\pgfqpoint{4.789076in}{2.533233in}}%
\pgfpathlineto{\pgfqpoint{4.775841in}{2.534284in}}%
\pgfpathlineto{\pgfqpoint{4.762614in}{2.535455in}}%
\pgfpathlineto{\pgfqpoint{4.749396in}{2.536746in}}%
\pgfpathlineto{\pgfqpoint{4.736187in}{2.538157in}}%
\pgfpathlineto{\pgfqpoint{4.728870in}{2.528107in}}%
\pgfpathlineto{\pgfqpoint{4.721550in}{2.518064in}}%
\pgfpathlineto{\pgfqpoint{4.714224in}{2.508027in}}%
\pgfpathlineto{\pgfqpoint{4.706894in}{2.497996in}}%
\pgfpathclose%
\pgfusepath{fill}%
\end{pgfscope}%
\begin{pgfscope}%
\pgfpathrectangle{\pgfqpoint{1.254980in}{0.150000in}}{\pgfqpoint{5.490039in}{5.490039in}}%
\pgfusepath{clip}%
\pgfsetbuttcap%
\pgfsetroundjoin%
\definecolor{currentfill}{rgb}{0.282290,0.145912,0.461510}%
\pgfsetfillcolor{currentfill}%
\pgfsetfillopacity{0.700000}%
\pgfsetlinewidth{0.000000pt}%
\definecolor{currentstroke}{rgb}{0.000000,0.000000,0.000000}%
\pgfsetstrokecolor{currentstroke}%
\pgfsetdash{}{0pt}%
\pgfpathmoveto{\pgfqpoint{4.325525in}{2.389361in}}%
\pgfpathlineto{\pgfqpoint{4.338631in}{2.385595in}}%
\pgfpathlineto{\pgfqpoint{4.351744in}{2.381956in}}%
\pgfpathlineto{\pgfqpoint{4.364863in}{2.378443in}}%
\pgfpathlineto{\pgfqpoint{4.377988in}{2.375057in}}%
\pgfpathlineto{\pgfqpoint{4.385429in}{2.385021in}}%
\pgfpathlineto{\pgfqpoint{4.392865in}{2.394998in}}%
\pgfpathlineto{\pgfqpoint{4.400296in}{2.404990in}}%
\pgfpathlineto{\pgfqpoint{4.407723in}{2.414995in}}%
\pgfpathlineto{\pgfqpoint{4.394607in}{2.418373in}}%
\pgfpathlineto{\pgfqpoint{4.381498in}{2.421878in}}%
\pgfpathlineto{\pgfqpoint{4.368396in}{2.425508in}}%
\pgfpathlineto{\pgfqpoint{4.355300in}{2.429266in}}%
\pgfpathlineto{\pgfqpoint{4.347863in}{2.419263in}}%
\pgfpathlineto{\pgfqpoint{4.340422in}{2.409278in}}%
\pgfpathlineto{\pgfqpoint{4.332976in}{2.399310in}}%
\pgfpathlineto{\pgfqpoint{4.325525in}{2.389361in}}%
\pgfpathclose%
\pgfusepath{fill}%
\end{pgfscope}%
\begin{pgfscope}%
\pgfpathrectangle{\pgfqpoint{1.254980in}{0.150000in}}{\pgfqpoint{5.490039in}{5.490039in}}%
\pgfusepath{clip}%
\pgfsetbuttcap%
\pgfsetroundjoin%
\definecolor{currentfill}{rgb}{0.283072,0.130895,0.449241}%
\pgfsetfillcolor{currentfill}%
\pgfsetfillopacity{0.700000}%
\pgfsetlinewidth{0.000000pt}%
\definecolor{currentstroke}{rgb}{0.000000,0.000000,0.000000}%
\pgfsetstrokecolor{currentstroke}%
\pgfsetdash{}{0pt}%
\pgfpathmoveto{\pgfqpoint{4.108703in}{2.366378in}}%
\pgfpathlineto{\pgfqpoint{4.121761in}{2.360994in}}%
\pgfpathlineto{\pgfqpoint{4.134823in}{2.355742in}}%
\pgfpathlineto{\pgfqpoint{4.147891in}{2.350621in}}%
\pgfpathlineto{\pgfqpoint{4.160964in}{2.345631in}}%
\pgfpathlineto{\pgfqpoint{4.168473in}{2.355353in}}%
\pgfpathlineto{\pgfqpoint{4.175977in}{2.365100in}}%
\pgfpathlineto{\pgfqpoint{4.183476in}{2.374871in}}%
\pgfpathlineto{\pgfqpoint{4.190971in}{2.384668in}}%
\pgfpathlineto{\pgfqpoint{4.177909in}{2.389618in}}%
\pgfpathlineto{\pgfqpoint{4.164852in}{2.394698in}}%
\pgfpathlineto{\pgfqpoint{4.151800in}{2.399911in}}%
\pgfpathlineto{\pgfqpoint{4.138754in}{2.405254in}}%
\pgfpathlineto{\pgfqpoint{4.131249in}{2.395492in}}%
\pgfpathlineto{\pgfqpoint{4.123738in}{2.385759in}}%
\pgfpathlineto{\pgfqpoint{4.116223in}{2.376054in}}%
\pgfpathlineto{\pgfqpoint{4.108703in}{2.366378in}}%
\pgfpathclose%
\pgfusepath{fill}%
\end{pgfscope}%
\begin{pgfscope}%
\pgfpathrectangle{\pgfqpoint{1.254980in}{0.150000in}}{\pgfqpoint{5.490039in}{5.490039in}}%
\pgfusepath{clip}%
\pgfsetbuttcap%
\pgfsetroundjoin%
\definecolor{currentfill}{rgb}{0.141935,0.526453,0.555991}%
\pgfsetfillcolor{currentfill}%
\pgfsetfillopacity{0.700000}%
\pgfsetlinewidth{0.000000pt}%
\definecolor{currentstroke}{rgb}{0.000000,0.000000,0.000000}%
\pgfsetstrokecolor{currentstroke}%
\pgfsetdash{}{0pt}%
\pgfpathmoveto{\pgfqpoint{2.839085in}{3.309737in}}%
\pgfpathlineto{\pgfqpoint{2.852261in}{3.289618in}}%
\pgfpathlineto{\pgfqpoint{2.865430in}{3.269708in}}%
\pgfpathlineto{\pgfqpoint{2.878591in}{3.250006in}}%
\pgfpathlineto{\pgfqpoint{2.891746in}{3.230510in}}%
\pgfpathlineto{\pgfqpoint{2.899721in}{3.237294in}}%
\pgfpathlineto{\pgfqpoint{2.907687in}{3.244196in}}%
\pgfpathlineto{\pgfqpoint{2.915645in}{3.251218in}}%
\pgfpathlineto{\pgfqpoint{2.923594in}{3.258357in}}%
\pgfpathlineto{\pgfqpoint{2.910463in}{3.277736in}}%
\pgfpathlineto{\pgfqpoint{2.897326in}{3.297322in}}%
\pgfpathlineto{\pgfqpoint{2.884181in}{3.317115in}}%
\pgfpathlineto{\pgfqpoint{2.871030in}{3.337118in}}%
\pgfpathlineto{\pgfqpoint{2.863057in}{3.330089in}}%
\pgfpathlineto{\pgfqpoint{2.855075in}{3.323182in}}%
\pgfpathlineto{\pgfqpoint{2.847084in}{3.316398in}}%
\pgfpathlineto{\pgfqpoint{2.839085in}{3.309737in}}%
\pgfpathclose%
\pgfusepath{fill}%
\end{pgfscope}%
\begin{pgfscope}%
\pgfpathrectangle{\pgfqpoint{1.254980in}{0.150000in}}{\pgfqpoint{5.490039in}{5.490039in}}%
\pgfusepath{clip}%
\pgfsetbuttcap%
\pgfsetroundjoin%
\definecolor{currentfill}{rgb}{0.280868,0.160771,0.472899}%
\pgfsetfillcolor{currentfill}%
\pgfsetfillopacity{0.700000}%
\pgfsetlinewidth{0.000000pt}%
\definecolor{currentstroke}{rgb}{0.000000,0.000000,0.000000}%
\pgfsetstrokecolor{currentstroke}%
\pgfsetdash{}{0pt}%
\pgfpathmoveto{\pgfqpoint{3.705049in}{2.427854in}}%
\pgfpathlineto{\pgfqpoint{3.718058in}{2.418933in}}%
\pgfpathlineto{\pgfqpoint{3.731068in}{2.410157in}}%
\pgfpathlineto{\pgfqpoint{3.744081in}{2.401526in}}%
\pgfpathlineto{\pgfqpoint{3.757097in}{2.393038in}}%
\pgfpathlineto{\pgfqpoint{3.764741in}{2.401882in}}%
\pgfpathlineto{\pgfqpoint{3.772380in}{2.410778in}}%
\pgfpathlineto{\pgfqpoint{3.780013in}{2.419725in}}%
\pgfpathlineto{\pgfqpoint{3.787641in}{2.428723in}}%
\pgfpathlineto{\pgfqpoint{3.774640in}{2.437122in}}%
\pgfpathlineto{\pgfqpoint{3.761641in}{2.445664in}}%
\pgfpathlineto{\pgfqpoint{3.748644in}{2.454351in}}%
\pgfpathlineto{\pgfqpoint{3.735650in}{2.463182in}}%
\pgfpathlineto{\pgfqpoint{3.728008in}{2.454267in}}%
\pgfpathlineto{\pgfqpoint{3.720361in}{2.445407in}}%
\pgfpathlineto{\pgfqpoint{3.712708in}{2.436603in}}%
\pgfpathlineto{\pgfqpoint{3.705049in}{2.427854in}}%
\pgfpathclose%
\pgfusepath{fill}%
\end{pgfscope}%
\begin{pgfscope}%
\pgfpathrectangle{\pgfqpoint{1.254980in}{0.150000in}}{\pgfqpoint{5.490039in}{5.490039in}}%
\pgfusepath{clip}%
\pgfsetbuttcap%
\pgfsetroundjoin%
\definecolor{currentfill}{rgb}{0.277134,0.185228,0.489898}%
\pgfsetfillcolor{currentfill}%
\pgfsetfillopacity{0.700000}%
\pgfsetlinewidth{0.000000pt}%
\definecolor{currentstroke}{rgb}{0.000000,0.000000,0.000000}%
\pgfsetstrokecolor{currentstroke}%
\pgfsetdash{}{0pt}%
\pgfpathmoveto{\pgfqpoint{4.624701in}{2.464388in}}%
\pgfpathlineto{\pgfqpoint{4.637896in}{2.462587in}}%
\pgfpathlineto{\pgfqpoint{4.651099in}{2.460908in}}%
\pgfpathlineto{\pgfqpoint{4.664310in}{2.459350in}}%
\pgfpathlineto{\pgfqpoint{4.677530in}{2.457912in}}%
\pgfpathlineto{\pgfqpoint{4.684878in}{2.467928in}}%
\pgfpathlineto{\pgfqpoint{4.692221in}{2.477947in}}%
\pgfpathlineto{\pgfqpoint{4.699560in}{2.487970in}}%
\pgfpathlineto{\pgfqpoint{4.706894in}{2.497996in}}%
\pgfpathlineto{\pgfqpoint{4.693684in}{2.499473in}}%
\pgfpathlineto{\pgfqpoint{4.680482in}{2.501070in}}%
\pgfpathlineto{\pgfqpoint{4.667289in}{2.502789in}}%
\pgfpathlineto{\pgfqpoint{4.654104in}{2.504629in}}%
\pgfpathlineto{\pgfqpoint{4.646760in}{2.494557in}}%
\pgfpathlineto{\pgfqpoint{4.639412in}{2.484494in}}%
\pgfpathlineto{\pgfqpoint{4.632059in}{2.474437in}}%
\pgfpathlineto{\pgfqpoint{4.624701in}{2.464388in}}%
\pgfpathclose%
\pgfusepath{fill}%
\end{pgfscope}%
\begin{pgfscope}%
\pgfpathrectangle{\pgfqpoint{1.254980in}{0.150000in}}{\pgfqpoint{5.490039in}{5.490039in}}%
\pgfusepath{clip}%
\pgfsetbuttcap%
\pgfsetroundjoin%
\definecolor{currentfill}{rgb}{0.274128,0.199721,0.498911}%
\pgfsetfillcolor{currentfill}%
\pgfsetfillopacity{0.700000}%
\pgfsetlinewidth{0.000000pt}%
\definecolor{currentstroke}{rgb}{0.000000,0.000000,0.000000}%
\pgfsetstrokecolor{currentstroke}%
\pgfsetdash{}{0pt}%
\pgfpathmoveto{\pgfqpoint{3.518190in}{2.513297in}}%
\pgfpathlineto{\pgfqpoint{3.531199in}{2.502477in}}%
\pgfpathlineto{\pgfqpoint{3.544208in}{2.491811in}}%
\pgfpathlineto{\pgfqpoint{3.557218in}{2.481298in}}%
\pgfpathlineto{\pgfqpoint{3.570228in}{2.470936in}}%
\pgfpathlineto{\pgfqpoint{3.577941in}{2.479233in}}%
\pgfpathlineto{\pgfqpoint{3.585648in}{2.487596in}}%
\pgfpathlineto{\pgfqpoint{3.593349in}{2.496024in}}%
\pgfpathlineto{\pgfqpoint{3.601044in}{2.504518in}}%
\pgfpathlineto{\pgfqpoint{3.588050in}{2.514773in}}%
\pgfpathlineto{\pgfqpoint{3.575056in}{2.525181in}}%
\pgfpathlineto{\pgfqpoint{3.562063in}{2.535741in}}%
\pgfpathlineto{\pgfqpoint{3.549071in}{2.546454in}}%
\pgfpathlineto{\pgfqpoint{3.541360in}{2.538060in}}%
\pgfpathlineto{\pgfqpoint{3.533643in}{2.529736in}}%
\pgfpathlineto{\pgfqpoint{3.525920in}{2.521481in}}%
\pgfpathlineto{\pgfqpoint{3.518190in}{2.513297in}}%
\pgfpathclose%
\pgfusepath{fill}%
\end{pgfscope}%
\begin{pgfscope}%
\pgfpathrectangle{\pgfqpoint{1.254980in}{0.150000in}}{\pgfqpoint{5.490039in}{5.490039in}}%
\pgfusepath{clip}%
\pgfsetbuttcap%
\pgfsetroundjoin%
\definecolor{currentfill}{rgb}{0.279574,0.170599,0.479997}%
\pgfsetfillcolor{currentfill}%
\pgfsetfillopacity{0.700000}%
\pgfsetlinewidth{0.000000pt}%
\definecolor{currentstroke}{rgb}{0.000000,0.000000,0.000000}%
\pgfsetstrokecolor{currentstroke}%
\pgfsetdash{}{0pt}%
\pgfpathmoveto{\pgfqpoint{4.542491in}{2.432576in}}%
\pgfpathlineto{\pgfqpoint{4.555663in}{2.430309in}}%
\pgfpathlineto{\pgfqpoint{4.568843in}{2.428166in}}%
\pgfpathlineto{\pgfqpoint{4.582031in}{2.426145in}}%
\pgfpathlineto{\pgfqpoint{4.595227in}{2.424246in}}%
\pgfpathlineto{\pgfqpoint{4.602603in}{2.434274in}}%
\pgfpathlineto{\pgfqpoint{4.609973in}{2.444307in}}%
\pgfpathlineto{\pgfqpoint{4.617340in}{2.454344in}}%
\pgfpathlineto{\pgfqpoint{4.624701in}{2.464388in}}%
\pgfpathlineto{\pgfqpoint{4.611515in}{2.466311in}}%
\pgfpathlineto{\pgfqpoint{4.598337in}{2.468355in}}%
\pgfpathlineto{\pgfqpoint{4.585166in}{2.470522in}}%
\pgfpathlineto{\pgfqpoint{4.572004in}{2.472812in}}%
\pgfpathlineto{\pgfqpoint{4.564632in}{2.462739in}}%
\pgfpathlineto{\pgfqpoint{4.557256in}{2.452676in}}%
\pgfpathlineto{\pgfqpoint{4.549876in}{2.442622in}}%
\pgfpathlineto{\pgfqpoint{4.542491in}{2.432576in}}%
\pgfpathclose%
\pgfusepath{fill}%
\end{pgfscope}%
\begin{pgfscope}%
\pgfpathrectangle{\pgfqpoint{1.254980in}{0.150000in}}{\pgfqpoint{5.490039in}{5.490039in}}%
\pgfusepath{clip}%
\pgfsetbuttcap%
\pgfsetroundjoin%
\definecolor{currentfill}{rgb}{0.282884,0.135920,0.453427}%
\pgfsetfillcolor{currentfill}%
\pgfsetfillopacity{0.700000}%
\pgfsetlinewidth{0.000000pt}%
\definecolor{currentstroke}{rgb}{0.000000,0.000000,0.000000}%
\pgfsetstrokecolor{currentstroke}%
\pgfsetdash{}{0pt}%
\pgfpathmoveto{\pgfqpoint{4.243276in}{2.366170in}}%
\pgfpathlineto{\pgfqpoint{4.256367in}{2.361868in}}%
\pgfpathlineto{\pgfqpoint{4.269464in}{2.357695in}}%
\pgfpathlineto{\pgfqpoint{4.282568in}{2.353650in}}%
\pgfpathlineto{\pgfqpoint{4.295677in}{2.349733in}}%
\pgfpathlineto{\pgfqpoint{4.303146in}{2.359614in}}%
\pgfpathlineto{\pgfqpoint{4.310611in}{2.369513in}}%
\pgfpathlineto{\pgfqpoint{4.318070in}{2.379428in}}%
\pgfpathlineto{\pgfqpoint{4.325525in}{2.389361in}}%
\pgfpathlineto{\pgfqpoint{4.312426in}{2.393254in}}%
\pgfpathlineto{\pgfqpoint{4.299333in}{2.397275in}}%
\pgfpathlineto{\pgfqpoint{4.286246in}{2.401423in}}%
\pgfpathlineto{\pgfqpoint{4.273166in}{2.405701in}}%
\pgfpathlineto{\pgfqpoint{4.265700in}{2.395786in}}%
\pgfpathlineto{\pgfqpoint{4.258230in}{2.385893in}}%
\pgfpathlineto{\pgfqpoint{4.250756in}{2.376021in}}%
\pgfpathlineto{\pgfqpoint{4.243276in}{2.366170in}}%
\pgfpathclose%
\pgfusepath{fill}%
\end{pgfscope}%
\begin{pgfscope}%
\pgfpathrectangle{\pgfqpoint{1.254980in}{0.150000in}}{\pgfqpoint{5.490039in}{5.490039in}}%
\pgfusepath{clip}%
\pgfsetbuttcap%
\pgfsetroundjoin%
\definecolor{currentfill}{rgb}{0.214298,0.355619,0.551184}%
\pgfsetfillcolor{currentfill}%
\pgfsetfillopacity{0.700000}%
\pgfsetlinewidth{0.000000pt}%
\definecolor{currentstroke}{rgb}{0.000000,0.000000,0.000000}%
\pgfsetstrokecolor{currentstroke}%
\pgfsetdash{}{0pt}%
\pgfpathmoveto{\pgfqpoint{3.122117in}{2.853370in}}%
\pgfpathlineto{\pgfqpoint{3.135186in}{2.837780in}}%
\pgfpathlineto{\pgfqpoint{3.148251in}{2.822369in}}%
\pgfpathlineto{\pgfqpoint{3.161313in}{2.807136in}}%
\pgfpathlineto{\pgfqpoint{3.174372in}{2.792080in}}%
\pgfpathlineto{\pgfqpoint{3.182243in}{2.799193in}}%
\pgfpathlineto{\pgfqpoint{3.190106in}{2.806404in}}%
\pgfpathlineto{\pgfqpoint{3.197962in}{2.813712in}}%
\pgfpathlineto{\pgfqpoint{3.205810in}{2.821116in}}%
\pgfpathlineto{\pgfqpoint{3.192772in}{2.836045in}}%
\pgfpathlineto{\pgfqpoint{3.179731in}{2.851151in}}%
\pgfpathlineto{\pgfqpoint{3.166687in}{2.866435in}}%
\pgfpathlineto{\pgfqpoint{3.153639in}{2.881897in}}%
\pgfpathlineto{\pgfqpoint{3.145770in}{2.874615in}}%
\pgfpathlineto{\pgfqpoint{3.137893in}{2.867432in}}%
\pgfpathlineto{\pgfqpoint{3.130009in}{2.860350in}}%
\pgfpathlineto{\pgfqpoint{3.122117in}{2.853370in}}%
\pgfpathclose%
\pgfusepath{fill}%
\end{pgfscope}%
\begin{pgfscope}%
\pgfpathrectangle{\pgfqpoint{1.254980in}{0.150000in}}{\pgfqpoint{5.490039in}{5.490039in}}%
\pgfusepath{clip}%
\pgfsetbuttcap%
\pgfsetroundjoin%
\definecolor{currentfill}{rgb}{0.225863,0.330805,0.547314}%
\pgfsetfillcolor{currentfill}%
\pgfsetfillopacity{0.700000}%
\pgfsetlinewidth{0.000000pt}%
\definecolor{currentstroke}{rgb}{0.000000,0.000000,0.000000}%
\pgfsetstrokecolor{currentstroke}%
\pgfsetdash{}{0pt}%
\pgfpathmoveto{\pgfqpoint{3.174372in}{2.792080in}}%
\pgfpathlineto{\pgfqpoint{3.187428in}{2.777200in}}%
\pgfpathlineto{\pgfqpoint{3.200481in}{2.762494in}}%
\pgfpathlineto{\pgfqpoint{3.213532in}{2.747963in}}%
\pgfpathlineto{\pgfqpoint{3.226579in}{2.733604in}}%
\pgfpathlineto{\pgfqpoint{3.234429in}{2.740850in}}%
\pgfpathlineto{\pgfqpoint{3.242272in}{2.748190in}}%
\pgfpathlineto{\pgfqpoint{3.250107in}{2.755622in}}%
\pgfpathlineto{\pgfqpoint{3.257935in}{2.763146in}}%
\pgfpathlineto{\pgfqpoint{3.244907in}{2.777379in}}%
\pgfpathlineto{\pgfqpoint{3.231877in}{2.791784in}}%
\pgfpathlineto{\pgfqpoint{3.218845in}{2.806363in}}%
\pgfpathlineto{\pgfqpoint{3.205810in}{2.821116in}}%
\pgfpathlineto{\pgfqpoint{3.197962in}{2.813712in}}%
\pgfpathlineto{\pgfqpoint{3.190106in}{2.806404in}}%
\pgfpathlineto{\pgfqpoint{3.182243in}{2.799193in}}%
\pgfpathlineto{\pgfqpoint{3.174372in}{2.792080in}}%
\pgfpathclose%
\pgfusepath{fill}%
\end{pgfscope}%
\begin{pgfscope}%
\pgfpathrectangle{\pgfqpoint{1.254980in}{0.150000in}}{\pgfqpoint{5.490039in}{5.490039in}}%
\pgfusepath{clip}%
\pgfsetbuttcap%
\pgfsetroundjoin%
\definecolor{currentfill}{rgb}{0.203063,0.379716,0.553925}%
\pgfsetfillcolor{currentfill}%
\pgfsetfillopacity{0.700000}%
\pgfsetlinewidth{0.000000pt}%
\definecolor{currentstroke}{rgb}{0.000000,0.000000,0.000000}%
\pgfsetstrokecolor{currentstroke}%
\pgfsetdash{}{0pt}%
\pgfpathmoveto{\pgfqpoint{3.069803in}{2.917543in}}%
\pgfpathlineto{\pgfqpoint{3.082887in}{2.901226in}}%
\pgfpathlineto{\pgfqpoint{3.095968in}{2.885092in}}%
\pgfpathlineto{\pgfqpoint{3.109044in}{2.869140in}}%
\pgfpathlineto{\pgfqpoint{3.122117in}{2.853370in}}%
\pgfpathlineto{\pgfqpoint{3.130009in}{2.860350in}}%
\pgfpathlineto{\pgfqpoint{3.137893in}{2.867432in}}%
\pgfpathlineto{\pgfqpoint{3.145770in}{2.874615in}}%
\pgfpathlineto{\pgfqpoint{3.153639in}{2.881897in}}%
\pgfpathlineto{\pgfqpoint{3.140588in}{2.897540in}}%
\pgfpathlineto{\pgfqpoint{3.127533in}{2.913364in}}%
\pgfpathlineto{\pgfqpoint{3.114475in}{2.929370in}}%
\pgfpathlineto{\pgfqpoint{3.101413in}{2.945559in}}%
\pgfpathlineto{\pgfqpoint{3.093522in}{2.938398in}}%
\pgfpathlineto{\pgfqpoint{3.085624in}{2.931341in}}%
\pgfpathlineto{\pgfqpoint{3.077717in}{2.924389in}}%
\pgfpathlineto{\pgfqpoint{3.069803in}{2.917543in}}%
\pgfpathclose%
\pgfusepath{fill}%
\end{pgfscope}%
\begin{pgfscope}%
\pgfpathrectangle{\pgfqpoint{1.254980in}{0.150000in}}{\pgfqpoint{5.490039in}{5.490039in}}%
\pgfusepath{clip}%
\pgfsetbuttcap%
\pgfsetroundjoin%
\definecolor{currentfill}{rgb}{0.237441,0.305202,0.541921}%
\pgfsetfillcolor{currentfill}%
\pgfsetfillopacity{0.700000}%
\pgfsetlinewidth{0.000000pt}%
\definecolor{currentstroke}{rgb}{0.000000,0.000000,0.000000}%
\pgfsetstrokecolor{currentstroke}%
\pgfsetdash{}{0pt}%
\pgfpathmoveto{\pgfqpoint{3.226579in}{2.733604in}}%
\pgfpathlineto{\pgfqpoint{3.239625in}{2.719418in}}%
\pgfpathlineto{\pgfqpoint{3.252668in}{2.705402in}}%
\pgfpathlineto{\pgfqpoint{3.265708in}{2.691556in}}%
\pgfpathlineto{\pgfqpoint{3.278747in}{2.677878in}}%
\pgfpathlineto{\pgfqpoint{3.286577in}{2.685256in}}%
\pgfpathlineto{\pgfqpoint{3.294399in}{2.692723in}}%
\pgfpathlineto{\pgfqpoint{3.302215in}{2.700279in}}%
\pgfpathlineto{\pgfqpoint{3.310023in}{2.707922in}}%
\pgfpathlineto{\pgfqpoint{3.297004in}{2.721474in}}%
\pgfpathlineto{\pgfqpoint{3.283983in}{2.735195in}}%
\pgfpathlineto{\pgfqpoint{3.270960in}{2.749085in}}%
\pgfpathlineto{\pgfqpoint{3.257935in}{2.763146in}}%
\pgfpathlineto{\pgfqpoint{3.250107in}{2.755622in}}%
\pgfpathlineto{\pgfqpoint{3.242272in}{2.748190in}}%
\pgfpathlineto{\pgfqpoint{3.234429in}{2.740850in}}%
\pgfpathlineto{\pgfqpoint{3.226579in}{2.733604in}}%
\pgfpathclose%
\pgfusepath{fill}%
\end{pgfscope}%
\begin{pgfscope}%
\pgfpathrectangle{\pgfqpoint{1.254980in}{0.150000in}}{\pgfqpoint{5.490039in}{5.490039in}}%
\pgfusepath{clip}%
\pgfsetbuttcap%
\pgfsetroundjoin%
\definecolor{currentfill}{rgb}{0.282884,0.135920,0.453427}%
\pgfsetfillcolor{currentfill}%
\pgfsetfillopacity{0.700000}%
\pgfsetlinewidth{0.000000pt}%
\definecolor{currentstroke}{rgb}{0.000000,0.000000,0.000000}%
\pgfsetstrokecolor{currentstroke}%
\pgfsetdash{}{0pt}%
\pgfpathmoveto{\pgfqpoint{3.891755in}{2.366631in}}%
\pgfpathlineto{\pgfqpoint{3.904784in}{2.359498in}}%
\pgfpathlineto{\pgfqpoint{3.917817in}{2.352504in}}%
\pgfpathlineto{\pgfqpoint{3.930854in}{2.345646in}}%
\pgfpathlineto{\pgfqpoint{3.943894in}{2.338925in}}%
\pgfpathlineto{\pgfqpoint{3.951477in}{2.348209in}}%
\pgfpathlineto{\pgfqpoint{3.959055in}{2.357532in}}%
\pgfpathlineto{\pgfqpoint{3.966628in}{2.366892in}}%
\pgfpathlineto{\pgfqpoint{3.974196in}{2.376290in}}%
\pgfpathlineto{\pgfqpoint{3.961168in}{2.382938in}}%
\pgfpathlineto{\pgfqpoint{3.948144in}{2.389724in}}%
\pgfpathlineto{\pgfqpoint{3.935124in}{2.396646in}}%
\pgfpathlineto{\pgfqpoint{3.922107in}{2.403707in}}%
\pgfpathlineto{\pgfqpoint{3.914527in}{2.394375in}}%
\pgfpathlineto{\pgfqpoint{3.906942in}{2.385085in}}%
\pgfpathlineto{\pgfqpoint{3.899351in}{2.375837in}}%
\pgfpathlineto{\pgfqpoint{3.891755in}{2.366631in}}%
\pgfpathclose%
\pgfusepath{fill}%
\end{pgfscope}%
\begin{pgfscope}%
\pgfpathrectangle{\pgfqpoint{1.254980in}{0.150000in}}{\pgfqpoint{5.490039in}{5.490039in}}%
\pgfusepath{clip}%
\pgfsetbuttcap%
\pgfsetroundjoin%
\definecolor{currentfill}{rgb}{0.190631,0.407061,0.556089}%
\pgfsetfillcolor{currentfill}%
\pgfsetfillopacity{0.700000}%
\pgfsetlinewidth{0.000000pt}%
\definecolor{currentstroke}{rgb}{0.000000,0.000000,0.000000}%
\pgfsetstrokecolor{currentstroke}%
\pgfsetdash{}{0pt}%
\pgfpathmoveto{\pgfqpoint{3.017420in}{2.984674in}}%
\pgfpathlineto{\pgfqpoint{3.030523in}{2.967610in}}%
\pgfpathlineto{\pgfqpoint{3.043621in}{2.950734in}}%
\pgfpathlineto{\pgfqpoint{3.056714in}{2.934046in}}%
\pgfpathlineto{\pgfqpoint{3.069803in}{2.917543in}}%
\pgfpathlineto{\pgfqpoint{3.077717in}{2.924389in}}%
\pgfpathlineto{\pgfqpoint{3.085624in}{2.931341in}}%
\pgfpathlineto{\pgfqpoint{3.093522in}{2.938398in}}%
\pgfpathlineto{\pgfqpoint{3.101413in}{2.945559in}}%
\pgfpathlineto{\pgfqpoint{3.088346in}{2.961932in}}%
\pgfpathlineto{\pgfqpoint{3.075276in}{2.978492in}}%
\pgfpathlineto{\pgfqpoint{3.062200in}{2.995238in}}%
\pgfpathlineto{\pgfqpoint{3.049121in}{3.012172in}}%
\pgfpathlineto{\pgfqpoint{3.041208in}{3.005134in}}%
\pgfpathlineto{\pgfqpoint{3.033287in}{2.998205in}}%
\pgfpathlineto{\pgfqpoint{3.025358in}{2.991384in}}%
\pgfpathlineto{\pgfqpoint{3.017420in}{2.984674in}}%
\pgfpathclose%
\pgfusepath{fill}%
\end{pgfscope}%
\begin{pgfscope}%
\pgfpathrectangle{\pgfqpoint{1.254980in}{0.150000in}}{\pgfqpoint{5.490039in}{5.490039in}}%
\pgfusepath{clip}%
\pgfsetbuttcap%
\pgfsetroundjoin%
\definecolor{currentfill}{rgb}{0.283187,0.125848,0.444960}%
\pgfsetfillcolor{currentfill}%
\pgfsetfillopacity{0.700000}%
\pgfsetlinewidth{0.000000pt}%
\definecolor{currentstroke}{rgb}{0.000000,0.000000,0.000000}%
\pgfsetstrokecolor{currentstroke}%
\pgfsetdash{}{0pt}%
\pgfpathmoveto{\pgfqpoint{4.026350in}{2.351053in}}%
\pgfpathlineto{\pgfqpoint{4.039399in}{2.345081in}}%
\pgfpathlineto{\pgfqpoint{4.052453in}{2.339242in}}%
\pgfpathlineto{\pgfqpoint{4.065512in}{2.333537in}}%
\pgfpathlineto{\pgfqpoint{4.078575in}{2.327964in}}%
\pgfpathlineto{\pgfqpoint{4.086115in}{2.337524in}}%
\pgfpathlineto{\pgfqpoint{4.093649in}{2.347113in}}%
\pgfpathlineto{\pgfqpoint{4.101179in}{2.356731in}}%
\pgfpathlineto{\pgfqpoint{4.108703in}{2.366378in}}%
\pgfpathlineto{\pgfqpoint{4.095651in}{2.371895in}}%
\pgfpathlineto{\pgfqpoint{4.082604in}{2.377544in}}%
\pgfpathlineto{\pgfqpoint{4.069561in}{2.383327in}}%
\pgfpathlineto{\pgfqpoint{4.056523in}{2.389243in}}%
\pgfpathlineto{\pgfqpoint{4.048987in}{2.379646in}}%
\pgfpathlineto{\pgfqpoint{4.041446in}{2.370082in}}%
\pgfpathlineto{\pgfqpoint{4.033900in}{2.360551in}}%
\pgfpathlineto{\pgfqpoint{4.026350in}{2.351053in}}%
\pgfpathclose%
\pgfusepath{fill}%
\end{pgfscope}%
\begin{pgfscope}%
\pgfpathrectangle{\pgfqpoint{1.254980in}{0.150000in}}{\pgfqpoint{5.490039in}{5.490039in}}%
\pgfusepath{clip}%
\pgfsetbuttcap%
\pgfsetroundjoin%
\definecolor{currentfill}{rgb}{0.277134,0.185228,0.489898}%
\pgfsetfillcolor{currentfill}%
\pgfsetfillopacity{0.700000}%
\pgfsetlinewidth{0.000000pt}%
\definecolor{currentstroke}{rgb}{0.000000,0.000000,0.000000}%
\pgfsetstrokecolor{currentstroke}%
\pgfsetdash{}{0pt}%
\pgfpathmoveto{\pgfqpoint{3.570228in}{2.470936in}}%
\pgfpathlineto{\pgfqpoint{3.583240in}{2.460726in}}%
\pgfpathlineto{\pgfqpoint{3.596252in}{2.450666in}}%
\pgfpathlineto{\pgfqpoint{3.609266in}{2.440756in}}%
\pgfpathlineto{\pgfqpoint{3.622281in}{2.430995in}}%
\pgfpathlineto{\pgfqpoint{3.629978in}{2.439404in}}%
\pgfpathlineto{\pgfqpoint{3.637670in}{2.447874in}}%
\pgfpathlineto{\pgfqpoint{3.645355in}{2.456406in}}%
\pgfpathlineto{\pgfqpoint{3.653034in}{2.465000in}}%
\pgfpathlineto{\pgfqpoint{3.640035in}{2.474655in}}%
\pgfpathlineto{\pgfqpoint{3.627036in}{2.484459in}}%
\pgfpathlineto{\pgfqpoint{3.614040in}{2.494413in}}%
\pgfpathlineto{\pgfqpoint{3.601044in}{2.504518in}}%
\pgfpathlineto{\pgfqpoint{3.593349in}{2.496024in}}%
\pgfpathlineto{\pgfqpoint{3.585648in}{2.487596in}}%
\pgfpathlineto{\pgfqpoint{3.577941in}{2.479233in}}%
\pgfpathlineto{\pgfqpoint{3.570228in}{2.470936in}}%
\pgfpathclose%
\pgfusepath{fill}%
\end{pgfscope}%
\begin{pgfscope}%
\pgfpathrectangle{\pgfqpoint{1.254980in}{0.150000in}}{\pgfqpoint{5.490039in}{5.490039in}}%
\pgfusepath{clip}%
\pgfsetbuttcap%
\pgfsetroundjoin%
\definecolor{currentfill}{rgb}{0.248629,0.278775,0.534556}%
\pgfsetfillcolor{currentfill}%
\pgfsetfillopacity{0.700000}%
\pgfsetlinewidth{0.000000pt}%
\definecolor{currentstroke}{rgb}{0.000000,0.000000,0.000000}%
\pgfsetstrokecolor{currentstroke}%
\pgfsetdash{}{0pt}%
\pgfpathmoveto{\pgfqpoint{3.278747in}{2.677878in}}%
\pgfpathlineto{\pgfqpoint{3.291784in}{2.664369in}}%
\pgfpathlineto{\pgfqpoint{3.304819in}{2.651027in}}%
\pgfpathlineto{\pgfqpoint{3.317853in}{2.637851in}}%
\pgfpathlineto{\pgfqpoint{3.330885in}{2.624840in}}%
\pgfpathlineto{\pgfqpoint{3.338695in}{2.632348in}}%
\pgfpathlineto{\pgfqpoint{3.346498in}{2.639942in}}%
\pgfpathlineto{\pgfqpoint{3.354294in}{2.647620in}}%
\pgfpathlineto{\pgfqpoint{3.362084in}{2.655382in}}%
\pgfpathlineto{\pgfqpoint{3.349071in}{2.668269in}}%
\pgfpathlineto{\pgfqpoint{3.336056in}{2.681320in}}%
\pgfpathlineto{\pgfqpoint{3.323040in}{2.694538in}}%
\pgfpathlineto{\pgfqpoint{3.310023in}{2.707922in}}%
\pgfpathlineto{\pgfqpoint{3.302215in}{2.700279in}}%
\pgfpathlineto{\pgfqpoint{3.294399in}{2.692723in}}%
\pgfpathlineto{\pgfqpoint{3.286577in}{2.685256in}}%
\pgfpathlineto{\pgfqpoint{3.278747in}{2.677878in}}%
\pgfpathclose%
\pgfusepath{fill}%
\end{pgfscope}%
\begin{pgfscope}%
\pgfpathrectangle{\pgfqpoint{1.254980in}{0.150000in}}{\pgfqpoint{5.490039in}{5.490039in}}%
\pgfusepath{clip}%
\pgfsetbuttcap%
\pgfsetroundjoin%
\definecolor{currentfill}{rgb}{0.221989,0.339161,0.548752}%
\pgfsetfillcolor{currentfill}%
\pgfsetfillopacity{0.700000}%
\pgfsetlinewidth{0.000000pt}%
\definecolor{currentstroke}{rgb}{0.000000,0.000000,0.000000}%
\pgfsetstrokecolor{currentstroke}%
\pgfsetdash{}{0pt}%
\pgfpathmoveto{\pgfqpoint{5.335953in}{2.776226in}}%
\pgfpathlineto{\pgfqpoint{5.349422in}{2.777878in}}%
\pgfpathlineto{\pgfqpoint{5.362902in}{2.779644in}}%
\pgfpathlineto{\pgfqpoint{5.376395in}{2.781522in}}%
\pgfpathlineto{\pgfqpoint{5.389899in}{2.783514in}}%
\pgfpathlineto{\pgfqpoint{5.397006in}{2.792664in}}%
\pgfpathlineto{\pgfqpoint{5.404109in}{2.801819in}}%
\pgfpathlineto{\pgfqpoint{5.411207in}{2.810981in}}%
\pgfpathlineto{\pgfqpoint{5.418302in}{2.820152in}}%
\pgfpathlineto{\pgfqpoint{5.404810in}{2.818329in}}%
\pgfpathlineto{\pgfqpoint{5.391331in}{2.816619in}}%
\pgfpathlineto{\pgfqpoint{5.377863in}{2.815022in}}%
\pgfpathlineto{\pgfqpoint{5.364407in}{2.813537in}}%
\pgfpathlineto{\pgfqpoint{5.357300in}{2.804192in}}%
\pgfpathlineto{\pgfqpoint{5.350188in}{2.794860in}}%
\pgfpathlineto{\pgfqpoint{5.343073in}{2.785538in}}%
\pgfpathlineto{\pgfqpoint{5.335953in}{2.776226in}}%
\pgfpathclose%
\pgfusepath{fill}%
\end{pgfscope}%
\begin{pgfscope}%
\pgfpathrectangle{\pgfqpoint{1.254980in}{0.150000in}}{\pgfqpoint{5.490039in}{5.490039in}}%
\pgfusepath{clip}%
\pgfsetbuttcap%
\pgfsetroundjoin%
\definecolor{currentfill}{rgb}{0.212395,0.359683,0.551710}%
\pgfsetfillcolor{currentfill}%
\pgfsetfillopacity{0.700000}%
\pgfsetlinewidth{0.000000pt}%
\definecolor{currentstroke}{rgb}{0.000000,0.000000,0.000000}%
\pgfsetstrokecolor{currentstroke}%
\pgfsetdash{}{0pt}%
\pgfpathmoveto{\pgfqpoint{5.418302in}{2.820152in}}%
\pgfpathlineto{\pgfqpoint{5.431805in}{2.822088in}}%
\pgfpathlineto{\pgfqpoint{5.445320in}{2.824137in}}%
\pgfpathlineto{\pgfqpoint{5.458847in}{2.826298in}}%
\pgfpathlineto{\pgfqpoint{5.472386in}{2.828571in}}%
\pgfpathlineto{\pgfqpoint{5.479462in}{2.837574in}}%
\pgfpathlineto{\pgfqpoint{5.486534in}{2.846586in}}%
\pgfpathlineto{\pgfqpoint{5.493602in}{2.855608in}}%
\pgfpathlineto{\pgfqpoint{5.500665in}{2.864643in}}%
\pgfpathlineto{\pgfqpoint{5.487140in}{2.862555in}}%
\pgfpathlineto{\pgfqpoint{5.473627in}{2.860578in}}%
\pgfpathlineto{\pgfqpoint{5.460125in}{2.858714in}}%
\pgfpathlineto{\pgfqpoint{5.446636in}{2.856962in}}%
\pgfpathlineto{\pgfqpoint{5.439558in}{2.847737in}}%
\pgfpathlineto{\pgfqpoint{5.432477in}{2.838528in}}%
\pgfpathlineto{\pgfqpoint{5.425391in}{2.829334in}}%
\pgfpathlineto{\pgfqpoint{5.418302in}{2.820152in}}%
\pgfpathclose%
\pgfusepath{fill}%
\end{pgfscope}%
\begin{pgfscope}%
\pgfpathrectangle{\pgfqpoint{1.254980in}{0.150000in}}{\pgfqpoint{5.490039in}{5.490039in}}%
\pgfusepath{clip}%
\pgfsetbuttcap%
\pgfsetroundjoin%
\definecolor{currentfill}{rgb}{0.280868,0.160771,0.472899}%
\pgfsetfillcolor{currentfill}%
\pgfsetfillopacity{0.700000}%
\pgfsetlinewidth{0.000000pt}%
\definecolor{currentstroke}{rgb}{0.000000,0.000000,0.000000}%
\pgfsetstrokecolor{currentstroke}%
\pgfsetdash{}{0pt}%
\pgfpathmoveto{\pgfqpoint{4.460256in}{2.402737in}}%
\pgfpathlineto{\pgfqpoint{4.473408in}{2.399984in}}%
\pgfpathlineto{\pgfqpoint{4.486566in}{2.397355in}}%
\pgfpathlineto{\pgfqpoint{4.499733in}{2.394850in}}%
\pgfpathlineto{\pgfqpoint{4.512907in}{2.392468in}}%
\pgfpathlineto{\pgfqpoint{4.520310in}{2.402484in}}%
\pgfpathlineto{\pgfqpoint{4.527708in}{2.412508in}}%
\pgfpathlineto{\pgfqpoint{4.535102in}{2.422538in}}%
\pgfpathlineto{\pgfqpoint{4.542491in}{2.432576in}}%
\pgfpathlineto{\pgfqpoint{4.529327in}{2.434965in}}%
\pgfpathlineto{\pgfqpoint{4.516170in}{2.437478in}}%
\pgfpathlineto{\pgfqpoint{4.503021in}{2.440115in}}%
\pgfpathlineto{\pgfqpoint{4.489879in}{2.442875in}}%
\pgfpathlineto{\pgfqpoint{4.482480in}{2.432824in}}%
\pgfpathlineto{\pgfqpoint{4.475077in}{2.422784in}}%
\pgfpathlineto{\pgfqpoint{4.467669in}{2.412755in}}%
\pgfpathlineto{\pgfqpoint{4.460256in}{2.402737in}}%
\pgfpathclose%
\pgfusepath{fill}%
\end{pgfscope}%
\begin{pgfscope}%
\pgfpathrectangle{\pgfqpoint{1.254980in}{0.150000in}}{\pgfqpoint{5.490039in}{5.490039in}}%
\pgfusepath{clip}%
\pgfsetbuttcap%
\pgfsetroundjoin%
\definecolor{currentfill}{rgb}{0.229739,0.322361,0.545706}%
\pgfsetfillcolor{currentfill}%
\pgfsetfillopacity{0.700000}%
\pgfsetlinewidth{0.000000pt}%
\definecolor{currentstroke}{rgb}{0.000000,0.000000,0.000000}%
\pgfsetstrokecolor{currentstroke}%
\pgfsetdash{}{0pt}%
\pgfpathmoveto{\pgfqpoint{5.253618in}{2.732955in}}%
\pgfpathlineto{\pgfqpoint{5.267054in}{2.734304in}}%
\pgfpathlineto{\pgfqpoint{5.280501in}{2.735767in}}%
\pgfpathlineto{\pgfqpoint{5.293959in}{2.737344in}}%
\pgfpathlineto{\pgfqpoint{5.307429in}{2.739035in}}%
\pgfpathlineto{\pgfqpoint{5.314567in}{2.748328in}}%
\pgfpathlineto{\pgfqpoint{5.321700in}{2.757622in}}%
\pgfpathlineto{\pgfqpoint{5.328828in}{2.766921in}}%
\pgfpathlineto{\pgfqpoint{5.335953in}{2.776226in}}%
\pgfpathlineto{\pgfqpoint{5.322495in}{2.774688in}}%
\pgfpathlineto{\pgfqpoint{5.309049in}{2.773262in}}%
\pgfpathlineto{\pgfqpoint{5.295614in}{2.771951in}}%
\pgfpathlineto{\pgfqpoint{5.282191in}{2.770754in}}%
\pgfpathlineto{\pgfqpoint{5.275054in}{2.761291in}}%
\pgfpathlineto{\pgfqpoint{5.267913in}{2.751838in}}%
\pgfpathlineto{\pgfqpoint{5.260768in}{2.742393in}}%
\pgfpathlineto{\pgfqpoint{5.253618in}{2.732955in}}%
\pgfpathclose%
\pgfusepath{fill}%
\end{pgfscope}%
\begin{pgfscope}%
\pgfpathrectangle{\pgfqpoint{1.254980in}{0.150000in}}{\pgfqpoint{5.490039in}{5.490039in}}%
\pgfusepath{clip}%
\pgfsetbuttcap%
\pgfsetroundjoin%
\definecolor{currentfill}{rgb}{0.204903,0.375746,0.553533}%
\pgfsetfillcolor{currentfill}%
\pgfsetfillopacity{0.700000}%
\pgfsetlinewidth{0.000000pt}%
\definecolor{currentstroke}{rgb}{0.000000,0.000000,0.000000}%
\pgfsetstrokecolor{currentstroke}%
\pgfsetdash{}{0pt}%
\pgfpathmoveto{\pgfqpoint{5.500665in}{2.864643in}}%
\pgfpathlineto{\pgfqpoint{5.514203in}{2.866844in}}%
\pgfpathlineto{\pgfqpoint{5.527753in}{2.869157in}}%
\pgfpathlineto{\pgfqpoint{5.541315in}{2.871581in}}%
\pgfpathlineto{\pgfqpoint{5.554889in}{2.874117in}}%
\pgfpathlineto{\pgfqpoint{5.561935in}{2.882972in}}%
\pgfpathlineto{\pgfqpoint{5.568975in}{2.891839in}}%
\pgfpathlineto{\pgfqpoint{5.576012in}{2.900720in}}%
\pgfpathlineto{\pgfqpoint{5.583045in}{2.909618in}}%
\pgfpathlineto{\pgfqpoint{5.569485in}{2.907283in}}%
\pgfpathlineto{\pgfqpoint{5.555937in}{2.905060in}}%
\pgfpathlineto{\pgfqpoint{5.542402in}{2.902948in}}%
\pgfpathlineto{\pgfqpoint{5.528879in}{2.900947in}}%
\pgfpathlineto{\pgfqpoint{5.521831in}{2.891843in}}%
\pgfpathlineto{\pgfqpoint{5.514780in}{2.882758in}}%
\pgfpathlineto{\pgfqpoint{5.507725in}{2.873693in}}%
\pgfpathlineto{\pgfqpoint{5.500665in}{2.864643in}}%
\pgfpathclose%
\pgfusepath{fill}%
\end{pgfscope}%
\begin{pgfscope}%
\pgfpathrectangle{\pgfqpoint{1.254980in}{0.150000in}}{\pgfqpoint{5.490039in}{5.490039in}}%
\pgfusepath{clip}%
\pgfsetbuttcap%
\pgfsetroundjoin%
\definecolor{currentfill}{rgb}{0.282290,0.145912,0.461510}%
\pgfsetfillcolor{currentfill}%
\pgfsetfillopacity{0.700000}%
\pgfsetlinewidth{0.000000pt}%
\definecolor{currentstroke}{rgb}{0.000000,0.000000,0.000000}%
\pgfsetstrokecolor{currentstroke}%
\pgfsetdash{}{0pt}%
\pgfpathmoveto{\pgfqpoint{3.757097in}{2.393038in}}%
\pgfpathlineto{\pgfqpoint{3.770115in}{2.384694in}}%
\pgfpathlineto{\pgfqpoint{3.783135in}{2.376492in}}%
\pgfpathlineto{\pgfqpoint{3.796158in}{2.368432in}}%
\pgfpathlineto{\pgfqpoint{3.809184in}{2.360513in}}%
\pgfpathlineto{\pgfqpoint{3.816815in}{2.369451in}}%
\pgfpathlineto{\pgfqpoint{3.824440in}{2.378437in}}%
\pgfpathlineto{\pgfqpoint{3.832059in}{2.387471in}}%
\pgfpathlineto{\pgfqpoint{3.839673in}{2.396552in}}%
\pgfpathlineto{\pgfqpoint{3.826661in}{2.404382in}}%
\pgfpathlineto{\pgfqpoint{3.813651in}{2.412353in}}%
\pgfpathlineto{\pgfqpoint{3.800645in}{2.420467in}}%
\pgfpathlineto{\pgfqpoint{3.787641in}{2.428723in}}%
\pgfpathlineto{\pgfqpoint{3.780013in}{2.419725in}}%
\pgfpathlineto{\pgfqpoint{3.772380in}{2.410778in}}%
\pgfpathlineto{\pgfqpoint{3.764741in}{2.401882in}}%
\pgfpathlineto{\pgfqpoint{3.757097in}{2.393038in}}%
\pgfpathclose%
\pgfusepath{fill}%
\end{pgfscope}%
\begin{pgfscope}%
\pgfpathrectangle{\pgfqpoint{1.254980in}{0.150000in}}{\pgfqpoint{5.490039in}{5.490039in}}%
\pgfusepath{clip}%
\pgfsetbuttcap%
\pgfsetroundjoin%
\definecolor{currentfill}{rgb}{0.239346,0.300855,0.540844}%
\pgfsetfillcolor{currentfill}%
\pgfsetfillopacity{0.700000}%
\pgfsetlinewidth{0.000000pt}%
\definecolor{currentstroke}{rgb}{0.000000,0.000000,0.000000}%
\pgfsetstrokecolor{currentstroke}%
\pgfsetdash{}{0pt}%
\pgfpathmoveto{\pgfqpoint{5.171297in}{2.690437in}}%
\pgfpathlineto{\pgfqpoint{5.184700in}{2.691464in}}%
\pgfpathlineto{\pgfqpoint{5.198114in}{2.692606in}}%
\pgfpathlineto{\pgfqpoint{5.211539in}{2.693863in}}%
\pgfpathlineto{\pgfqpoint{5.224975in}{2.695234in}}%
\pgfpathlineto{\pgfqpoint{5.232143in}{2.704662in}}%
\pgfpathlineto{\pgfqpoint{5.239306in}{2.714091in}}%
\pgfpathlineto{\pgfqpoint{5.246464in}{2.723521in}}%
\pgfpathlineto{\pgfqpoint{5.253618in}{2.732955in}}%
\pgfpathlineto{\pgfqpoint{5.240194in}{2.731720in}}%
\pgfpathlineto{\pgfqpoint{5.226780in}{2.730599in}}%
\pgfpathlineto{\pgfqpoint{5.213378in}{2.729593in}}%
\pgfpathlineto{\pgfqpoint{5.199986in}{2.728701in}}%
\pgfpathlineto{\pgfqpoint{5.192821in}{2.719126in}}%
\pgfpathlineto{\pgfqpoint{5.185651in}{2.709558in}}%
\pgfpathlineto{\pgfqpoint{5.178476in}{2.699995in}}%
\pgfpathlineto{\pgfqpoint{5.171297in}{2.690437in}}%
\pgfpathclose%
\pgfusepath{fill}%
\end{pgfscope}%
\begin{pgfscope}%
\pgfpathrectangle{\pgfqpoint{1.254980in}{0.150000in}}{\pgfqpoint{5.490039in}{5.490039in}}%
\pgfusepath{clip}%
\pgfsetbuttcap%
\pgfsetroundjoin%
\definecolor{currentfill}{rgb}{0.195860,0.395433,0.555276}%
\pgfsetfillcolor{currentfill}%
\pgfsetfillopacity{0.700000}%
\pgfsetlinewidth{0.000000pt}%
\definecolor{currentstroke}{rgb}{0.000000,0.000000,0.000000}%
\pgfsetstrokecolor{currentstroke}%
\pgfsetdash{}{0pt}%
\pgfpathmoveto{\pgfqpoint{5.583045in}{2.909618in}}%
\pgfpathlineto{\pgfqpoint{5.596617in}{2.912065in}}%
\pgfpathlineto{\pgfqpoint{5.610202in}{2.914623in}}%
\pgfpathlineto{\pgfqpoint{5.623800in}{2.917292in}}%
\pgfpathlineto{\pgfqpoint{5.637410in}{2.920072in}}%
\pgfpathlineto{\pgfqpoint{5.644423in}{2.928778in}}%
\pgfpathlineto{\pgfqpoint{5.651433in}{2.937501in}}%
\pgfpathlineto{\pgfqpoint{5.658438in}{2.946243in}}%
\pgfpathlineto{\pgfqpoint{5.665439in}{2.955007in}}%
\pgfpathlineto{\pgfqpoint{5.651845in}{2.952444in}}%
\pgfpathlineto{\pgfqpoint{5.638263in}{2.949992in}}%
\pgfpathlineto{\pgfqpoint{5.624693in}{2.947651in}}%
\pgfpathlineto{\pgfqpoint{5.611136in}{2.945421in}}%
\pgfpathlineto{\pgfqpoint{5.604119in}{2.936435in}}%
\pgfpathlineto{\pgfqpoint{5.597098in}{2.927474in}}%
\pgfpathlineto{\pgfqpoint{5.590073in}{2.918536in}}%
\pgfpathlineto{\pgfqpoint{5.583045in}{2.909618in}}%
\pgfpathclose%
\pgfusepath{fill}%
\end{pgfscope}%
\begin{pgfscope}%
\pgfpathrectangle{\pgfqpoint{1.254980in}{0.150000in}}{\pgfqpoint{5.490039in}{5.490039in}}%
\pgfusepath{clip}%
\pgfsetbuttcap%
\pgfsetroundjoin%
\definecolor{currentfill}{rgb}{0.187231,0.414746,0.556547}%
\pgfsetfillcolor{currentfill}%
\pgfsetfillopacity{0.700000}%
\pgfsetlinewidth{0.000000pt}%
\definecolor{currentstroke}{rgb}{0.000000,0.000000,0.000000}%
\pgfsetstrokecolor{currentstroke}%
\pgfsetdash{}{0pt}%
\pgfpathmoveto{\pgfqpoint{5.665439in}{2.955007in}}%
\pgfpathlineto{\pgfqpoint{5.679047in}{2.957680in}}%
\pgfpathlineto{\pgfqpoint{5.692667in}{2.960465in}}%
\pgfpathlineto{\pgfqpoint{5.706300in}{2.963360in}}%
\pgfpathlineto{\pgfqpoint{5.719946in}{2.966365in}}%
\pgfpathlineto{\pgfqpoint{5.726928in}{2.974925in}}%
\pgfpathlineto{\pgfqpoint{5.733906in}{2.983507in}}%
\pgfpathlineto{\pgfqpoint{5.740879in}{2.992113in}}%
\pgfpathlineto{\pgfqpoint{5.747850in}{3.000747in}}%
\pgfpathlineto{\pgfqpoint{5.734220in}{2.997975in}}%
\pgfpathlineto{\pgfqpoint{5.720603in}{2.995314in}}%
\pgfpathlineto{\pgfqpoint{5.706999in}{2.992763in}}%
\pgfpathlineto{\pgfqpoint{5.693408in}{2.990322in}}%
\pgfpathlineto{\pgfqpoint{5.686421in}{2.981450in}}%
\pgfpathlineto{\pgfqpoint{5.679431in}{2.972608in}}%
\pgfpathlineto{\pgfqpoint{5.672437in}{2.963794in}}%
\pgfpathlineto{\pgfqpoint{5.665439in}{2.955007in}}%
\pgfpathclose%
\pgfusepath{fill}%
\end{pgfscope}%
\begin{pgfscope}%
\pgfpathrectangle{\pgfqpoint{1.254980in}{0.150000in}}{\pgfqpoint{5.490039in}{5.490039in}}%
\pgfusepath{clip}%
\pgfsetbuttcap%
\pgfsetroundjoin%
\definecolor{currentfill}{rgb}{0.246811,0.283237,0.535941}%
\pgfsetfillcolor{currentfill}%
\pgfsetfillopacity{0.700000}%
\pgfsetlinewidth{0.000000pt}%
\definecolor{currentstroke}{rgb}{0.000000,0.000000,0.000000}%
\pgfsetstrokecolor{currentstroke}%
\pgfsetdash{}{0pt}%
\pgfpathmoveto{\pgfqpoint{5.088988in}{2.648782in}}%
\pgfpathlineto{\pgfqpoint{5.102359in}{2.649468in}}%
\pgfpathlineto{\pgfqpoint{5.115741in}{2.650269in}}%
\pgfpathlineto{\pgfqpoint{5.129133in}{2.651186in}}%
\pgfpathlineto{\pgfqpoint{5.142537in}{2.652218in}}%
\pgfpathlineto{\pgfqpoint{5.149734in}{2.661773in}}%
\pgfpathlineto{\pgfqpoint{5.156926in}{2.671328in}}%
\pgfpathlineto{\pgfqpoint{5.164114in}{2.680882in}}%
\pgfpathlineto{\pgfqpoint{5.171297in}{2.690437in}}%
\pgfpathlineto{\pgfqpoint{5.157905in}{2.689525in}}%
\pgfpathlineto{\pgfqpoint{5.144524in}{2.688728in}}%
\pgfpathlineto{\pgfqpoint{5.131153in}{2.688046in}}%
\pgfpathlineto{\pgfqpoint{5.117793in}{2.687480in}}%
\pgfpathlineto{\pgfqpoint{5.110598in}{2.677799in}}%
\pgfpathlineto{\pgfqpoint{5.103399in}{2.668123in}}%
\pgfpathlineto{\pgfqpoint{5.096196in}{2.658452in}}%
\pgfpathlineto{\pgfqpoint{5.088988in}{2.648782in}}%
\pgfpathclose%
\pgfusepath{fill}%
\end{pgfscope}%
\begin{pgfscope}%
\pgfpathrectangle{\pgfqpoint{1.254980in}{0.150000in}}{\pgfqpoint{5.490039in}{5.490039in}}%
\pgfusepath{clip}%
\pgfsetbuttcap%
\pgfsetroundjoin%
\definecolor{currentfill}{rgb}{0.177423,0.437527,0.557565}%
\pgfsetfillcolor{currentfill}%
\pgfsetfillopacity{0.700000}%
\pgfsetlinewidth{0.000000pt}%
\definecolor{currentstroke}{rgb}{0.000000,0.000000,0.000000}%
\pgfsetstrokecolor{currentstroke}%
\pgfsetdash{}{0pt}%
\pgfpathmoveto{\pgfqpoint{2.964959in}{3.054837in}}%
\pgfpathlineto{\pgfqpoint{2.978082in}{3.037007in}}%
\pgfpathlineto{\pgfqpoint{2.991200in}{3.019371in}}%
\pgfpathlineto{\pgfqpoint{3.004313in}{3.001927in}}%
\pgfpathlineto{\pgfqpoint{3.017420in}{2.984674in}}%
\pgfpathlineto{\pgfqpoint{3.025358in}{2.991384in}}%
\pgfpathlineto{\pgfqpoint{3.033287in}{2.998205in}}%
\pgfpathlineto{\pgfqpoint{3.041208in}{3.005134in}}%
\pgfpathlineto{\pgfqpoint{3.049121in}{3.012172in}}%
\pgfpathlineto{\pgfqpoint{3.036036in}{3.029296in}}%
\pgfpathlineto{\pgfqpoint{3.022947in}{3.046610in}}%
\pgfpathlineto{\pgfqpoint{3.009853in}{3.064116in}}%
\pgfpathlineto{\pgfqpoint{2.996753in}{3.081815in}}%
\pgfpathlineto{\pgfqpoint{2.988817in}{3.074901in}}%
\pgfpathlineto{\pgfqpoint{2.980873in}{3.068099in}}%
\pgfpathlineto{\pgfqpoint{2.972920in}{3.061411in}}%
\pgfpathlineto{\pgfqpoint{2.964959in}{3.054837in}}%
\pgfpathclose%
\pgfusepath{fill}%
\end{pgfscope}%
\begin{pgfscope}%
\pgfpathrectangle{\pgfqpoint{1.254980in}{0.150000in}}{\pgfqpoint{5.490039in}{5.490039in}}%
\pgfusepath{clip}%
\pgfsetbuttcap%
\pgfsetroundjoin%
\definecolor{currentfill}{rgb}{0.253935,0.265254,0.529983}%
\pgfsetfillcolor{currentfill}%
\pgfsetfillopacity{0.700000}%
\pgfsetlinewidth{0.000000pt}%
\definecolor{currentstroke}{rgb}{0.000000,0.000000,0.000000}%
\pgfsetstrokecolor{currentstroke}%
\pgfsetdash{}{0pt}%
\pgfpathmoveto{\pgfqpoint{5.006689in}{2.608108in}}%
\pgfpathlineto{\pgfqpoint{5.020029in}{2.608433in}}%
\pgfpathlineto{\pgfqpoint{5.033380in}{2.608874in}}%
\pgfpathlineto{\pgfqpoint{5.046740in}{2.609432in}}%
\pgfpathlineto{\pgfqpoint{5.060112in}{2.610105in}}%
\pgfpathlineto{\pgfqpoint{5.067338in}{2.619777in}}%
\pgfpathlineto{\pgfqpoint{5.074559in}{2.629446in}}%
\pgfpathlineto{\pgfqpoint{5.081776in}{2.639114in}}%
\pgfpathlineto{\pgfqpoint{5.088988in}{2.648782in}}%
\pgfpathlineto{\pgfqpoint{5.075628in}{2.648212in}}%
\pgfpathlineto{\pgfqpoint{5.062277in}{2.647758in}}%
\pgfpathlineto{\pgfqpoint{5.048938in}{2.647421in}}%
\pgfpathlineto{\pgfqpoint{5.035608in}{2.647199in}}%
\pgfpathlineto{\pgfqpoint{5.028385in}{2.637422in}}%
\pgfpathlineto{\pgfqpoint{5.021157in}{2.627648in}}%
\pgfpathlineto{\pgfqpoint{5.013926in}{2.617877in}}%
\pgfpathlineto{\pgfqpoint{5.006689in}{2.608108in}}%
\pgfpathclose%
\pgfusepath{fill}%
\end{pgfscope}%
\begin{pgfscope}%
\pgfpathrectangle{\pgfqpoint{1.254980in}{0.150000in}}{\pgfqpoint{5.490039in}{5.490039in}}%
\pgfusepath{clip}%
\pgfsetbuttcap%
\pgfsetroundjoin%
\definecolor{currentfill}{rgb}{0.257322,0.256130,0.526563}%
\pgfsetfillcolor{currentfill}%
\pgfsetfillopacity{0.700000}%
\pgfsetlinewidth{0.000000pt}%
\definecolor{currentstroke}{rgb}{0.000000,0.000000,0.000000}%
\pgfsetstrokecolor{currentstroke}%
\pgfsetdash{}{0pt}%
\pgfpathmoveto{\pgfqpoint{3.330885in}{2.624840in}}%
\pgfpathlineto{\pgfqpoint{3.343916in}{2.611993in}}%
\pgfpathlineto{\pgfqpoint{3.356946in}{2.599310in}}%
\pgfpathlineto{\pgfqpoint{3.369975in}{2.586789in}}%
\pgfpathlineto{\pgfqpoint{3.383002in}{2.574430in}}%
\pgfpathlineto{\pgfqpoint{3.390793in}{2.582068in}}%
\pgfpathlineto{\pgfqpoint{3.398578in}{2.589788in}}%
\pgfpathlineto{\pgfqpoint{3.406355in}{2.597588in}}%
\pgfpathlineto{\pgfqpoint{3.414126in}{2.605468in}}%
\pgfpathlineto{\pgfqpoint{3.401117in}{2.617704in}}%
\pgfpathlineto{\pgfqpoint{3.388107in}{2.630101in}}%
\pgfpathlineto{\pgfqpoint{3.375096in}{2.642660in}}%
\pgfpathlineto{\pgfqpoint{3.362084in}{2.655382in}}%
\pgfpathlineto{\pgfqpoint{3.354294in}{2.647620in}}%
\pgfpathlineto{\pgfqpoint{3.346498in}{2.639942in}}%
\pgfpathlineto{\pgfqpoint{3.338695in}{2.632348in}}%
\pgfpathlineto{\pgfqpoint{3.330885in}{2.624840in}}%
\pgfpathclose%
\pgfusepath{fill}%
\end{pgfscope}%
\begin{pgfscope}%
\pgfpathrectangle{\pgfqpoint{1.254980in}{0.150000in}}{\pgfqpoint{5.490039in}{5.490039in}}%
\pgfusepath{clip}%
\pgfsetbuttcap%
\pgfsetroundjoin%
\definecolor{currentfill}{rgb}{0.283072,0.130895,0.449241}%
\pgfsetfillcolor{currentfill}%
\pgfsetfillopacity{0.700000}%
\pgfsetlinewidth{0.000000pt}%
\definecolor{currentstroke}{rgb}{0.000000,0.000000,0.000000}%
\pgfsetstrokecolor{currentstroke}%
\pgfsetdash{}{0pt}%
\pgfpathmoveto{\pgfqpoint{4.160964in}{2.345631in}}%
\pgfpathlineto{\pgfqpoint{4.174043in}{2.340772in}}%
\pgfpathlineto{\pgfqpoint{4.187127in}{2.336042in}}%
\pgfpathlineto{\pgfqpoint{4.200217in}{2.331443in}}%
\pgfpathlineto{\pgfqpoint{4.213312in}{2.326973in}}%
\pgfpathlineto{\pgfqpoint{4.220810in}{2.336741in}}%
\pgfpathlineto{\pgfqpoint{4.228304in}{2.346529in}}%
\pgfpathlineto{\pgfqpoint{4.235792in}{2.356339in}}%
\pgfpathlineto{\pgfqpoint{4.243276in}{2.366170in}}%
\pgfpathlineto{\pgfqpoint{4.230191in}{2.370600in}}%
\pgfpathlineto{\pgfqpoint{4.217112in}{2.375159in}}%
\pgfpathlineto{\pgfqpoint{4.204039in}{2.379849in}}%
\pgfpathlineto{\pgfqpoint{4.190971in}{2.384668in}}%
\pgfpathlineto{\pgfqpoint{4.183476in}{2.374871in}}%
\pgfpathlineto{\pgfqpoint{4.175977in}{2.365100in}}%
\pgfpathlineto{\pgfqpoint{4.168473in}{2.355353in}}%
\pgfpathlineto{\pgfqpoint{4.160964in}{2.345631in}}%
\pgfpathclose%
\pgfusepath{fill}%
\end{pgfscope}%
\begin{pgfscope}%
\pgfpathrectangle{\pgfqpoint{1.254980in}{0.150000in}}{\pgfqpoint{5.490039in}{5.490039in}}%
\pgfusepath{clip}%
\pgfsetbuttcap%
\pgfsetroundjoin%
\definecolor{currentfill}{rgb}{0.260571,0.246922,0.522828}%
\pgfsetfillcolor{currentfill}%
\pgfsetfillopacity{0.700000}%
\pgfsetlinewidth{0.000000pt}%
\definecolor{currentstroke}{rgb}{0.000000,0.000000,0.000000}%
\pgfsetstrokecolor{currentstroke}%
\pgfsetdash{}{0pt}%
\pgfpathmoveto{\pgfqpoint{4.924398in}{2.568542in}}%
\pgfpathlineto{\pgfqpoint{4.937708in}{2.568487in}}%
\pgfpathlineto{\pgfqpoint{4.951028in}{2.568548in}}%
\pgfpathlineto{\pgfqpoint{4.964358in}{2.568727in}}%
\pgfpathlineto{\pgfqpoint{4.977698in}{2.569023in}}%
\pgfpathlineto{\pgfqpoint{4.984953in}{2.578798in}}%
\pgfpathlineto{\pgfqpoint{4.992203in}{2.588569in}}%
\pgfpathlineto{\pgfqpoint{4.999448in}{2.598339in}}%
\pgfpathlineto{\pgfqpoint{5.006689in}{2.608108in}}%
\pgfpathlineto{\pgfqpoint{4.993359in}{2.607900in}}%
\pgfpathlineto{\pgfqpoint{4.980039in}{2.607809in}}%
\pgfpathlineto{\pgfqpoint{4.966729in}{2.607834in}}%
\pgfpathlineto{\pgfqpoint{4.953429in}{2.607977in}}%
\pgfpathlineto{\pgfqpoint{4.946178in}{2.598115in}}%
\pgfpathlineto{\pgfqpoint{4.938922in}{2.588256in}}%
\pgfpathlineto{\pgfqpoint{4.931662in}{2.578399in}}%
\pgfpathlineto{\pgfqpoint{4.924398in}{2.568542in}}%
\pgfpathclose%
\pgfusepath{fill}%
\end{pgfscope}%
\begin{pgfscope}%
\pgfpathrectangle{\pgfqpoint{1.254980in}{0.150000in}}{\pgfqpoint{5.490039in}{5.490039in}}%
\pgfusepath{clip}%
\pgfsetbuttcap%
\pgfsetroundjoin%
\definecolor{currentfill}{rgb}{0.180629,0.429975,0.557282}%
\pgfsetfillcolor{currentfill}%
\pgfsetfillopacity{0.700000}%
\pgfsetlinewidth{0.000000pt}%
\definecolor{currentstroke}{rgb}{0.000000,0.000000,0.000000}%
\pgfsetstrokecolor{currentstroke}%
\pgfsetdash{}{0pt}%
\pgfpathmoveto{\pgfqpoint{5.747850in}{3.000747in}}%
\pgfpathlineto{\pgfqpoint{5.761492in}{3.003629in}}%
\pgfpathlineto{\pgfqpoint{5.775148in}{3.006621in}}%
\pgfpathlineto{\pgfqpoint{5.788817in}{3.009723in}}%
\pgfpathlineto{\pgfqpoint{5.802499in}{3.012935in}}%
\pgfpathlineto{\pgfqpoint{5.809448in}{3.021353in}}%
\pgfpathlineto{\pgfqpoint{5.816394in}{3.029800in}}%
\pgfpathlineto{\pgfqpoint{5.823336in}{3.038277in}}%
\pgfpathlineto{\pgfqpoint{5.809667in}{3.035251in}}%
\pgfpathlineto{\pgfqpoint{5.796011in}{3.032335in}}%
\pgfpathlineto{\pgfqpoint{5.782369in}{3.029529in}}%
\pgfpathlineto{\pgfqpoint{5.768739in}{3.026832in}}%
\pgfpathlineto{\pgfqpoint{5.761779in}{3.018104in}}%
\pgfpathlineto{\pgfqpoint{5.754816in}{3.009410in}}%
\pgfpathlineto{\pgfqpoint{5.747850in}{3.000747in}}%
\pgfpathclose%
\pgfusepath{fill}%
\end{pgfscope}%
\begin{pgfscope}%
\pgfpathrectangle{\pgfqpoint{1.254980in}{0.150000in}}{\pgfqpoint{5.490039in}{5.490039in}}%
\pgfusepath{clip}%
\pgfsetbuttcap%
\pgfsetroundjoin%
\definecolor{currentfill}{rgb}{0.166617,0.463708,0.558119}%
\pgfsetfillcolor{currentfill}%
\pgfsetfillopacity{0.700000}%
\pgfsetlinewidth{0.000000pt}%
\definecolor{currentstroke}{rgb}{0.000000,0.000000,0.000000}%
\pgfsetstrokecolor{currentstroke}%
\pgfsetdash{}{0pt}%
\pgfpathmoveto{\pgfqpoint{2.912408in}{3.128113in}}%
\pgfpathlineto{\pgfqpoint{2.925555in}{3.109497in}}%
\pgfpathlineto{\pgfqpoint{2.938696in}{3.091080in}}%
\pgfpathlineto{\pgfqpoint{2.951830in}{3.072860in}}%
\pgfpathlineto{\pgfqpoint{2.964959in}{3.054837in}}%
\pgfpathlineto{\pgfqpoint{2.972920in}{3.061411in}}%
\pgfpathlineto{\pgfqpoint{2.980873in}{3.068099in}}%
\pgfpathlineto{\pgfqpoint{2.988817in}{3.074901in}}%
\pgfpathlineto{\pgfqpoint{2.996753in}{3.081815in}}%
\pgfpathlineto{\pgfqpoint{2.983648in}{3.099708in}}%
\pgfpathlineto{\pgfqpoint{2.970537in}{3.117797in}}%
\pgfpathlineto{\pgfqpoint{2.957421in}{3.136082in}}%
\pgfpathlineto{\pgfqpoint{2.944298in}{3.154566in}}%
\pgfpathlineto{\pgfqpoint{2.936339in}{3.147777in}}%
\pgfpathlineto{\pgfqpoint{2.928371in}{3.141104in}}%
\pgfpathlineto{\pgfqpoint{2.920394in}{3.134549in}}%
\pgfpathlineto{\pgfqpoint{2.912408in}{3.128113in}}%
\pgfpathclose%
\pgfusepath{fill}%
\end{pgfscope}%
\begin{pgfscope}%
\pgfpathrectangle{\pgfqpoint{1.254980in}{0.150000in}}{\pgfqpoint{5.490039in}{5.490039in}}%
\pgfusepath{clip}%
\pgfsetbuttcap%
\pgfsetroundjoin%
\definecolor{currentfill}{rgb}{0.282290,0.145912,0.461510}%
\pgfsetfillcolor{currentfill}%
\pgfsetfillopacity{0.700000}%
\pgfsetlinewidth{0.000000pt}%
\definecolor{currentstroke}{rgb}{0.000000,0.000000,0.000000}%
\pgfsetstrokecolor{currentstroke}%
\pgfsetdash{}{0pt}%
\pgfpathmoveto{\pgfqpoint{4.377988in}{2.375057in}}%
\pgfpathlineto{\pgfqpoint{4.391121in}{2.371796in}}%
\pgfpathlineto{\pgfqpoint{4.404260in}{2.368661in}}%
\pgfpathlineto{\pgfqpoint{4.417407in}{2.365651in}}%
\pgfpathlineto{\pgfqpoint{4.430560in}{2.362765in}}%
\pgfpathlineto{\pgfqpoint{4.437991in}{2.372743in}}%
\pgfpathlineto{\pgfqpoint{4.445418in}{2.382731in}}%
\pgfpathlineto{\pgfqpoint{4.452839in}{2.392729in}}%
\pgfpathlineto{\pgfqpoint{4.460256in}{2.402737in}}%
\pgfpathlineto{\pgfqpoint{4.447112in}{2.405614in}}%
\pgfpathlineto{\pgfqpoint{4.433976in}{2.408616in}}%
\pgfpathlineto{\pgfqpoint{4.420846in}{2.411743in}}%
\pgfpathlineto{\pgfqpoint{4.407723in}{2.414995in}}%
\pgfpathlineto{\pgfqpoint{4.400296in}{2.404990in}}%
\pgfpathlineto{\pgfqpoint{4.392865in}{2.394998in}}%
\pgfpathlineto{\pgfqpoint{4.385429in}{2.385021in}}%
\pgfpathlineto{\pgfqpoint{4.377988in}{2.375057in}}%
\pgfpathclose%
\pgfusepath{fill}%
\end{pgfscope}%
\begin{pgfscope}%
\pgfpathrectangle{\pgfqpoint{1.254980in}{0.150000in}}{\pgfqpoint{5.490039in}{5.490039in}}%
\pgfusepath{clip}%
\pgfsetbuttcap%
\pgfsetroundjoin%
\definecolor{currentfill}{rgb}{0.266580,0.228262,0.514349}%
\pgfsetfillcolor{currentfill}%
\pgfsetfillopacity{0.700000}%
\pgfsetlinewidth{0.000000pt}%
\definecolor{currentstroke}{rgb}{0.000000,0.000000,0.000000}%
\pgfsetstrokecolor{currentstroke}%
\pgfsetdash{}{0pt}%
\pgfpathmoveto{\pgfqpoint{4.842110in}{2.530222in}}%
\pgfpathlineto{\pgfqpoint{4.855392in}{2.529766in}}%
\pgfpathlineto{\pgfqpoint{4.868683in}{2.529428in}}%
\pgfpathlineto{\pgfqpoint{4.881983in}{2.529209in}}%
\pgfpathlineto{\pgfqpoint{4.895294in}{2.529107in}}%
\pgfpathlineto{\pgfqpoint{4.902576in}{2.538969in}}%
\pgfpathlineto{\pgfqpoint{4.909855in}{2.548829in}}%
\pgfpathlineto{\pgfqpoint{4.917128in}{2.558686in}}%
\pgfpathlineto{\pgfqpoint{4.924398in}{2.568542in}}%
\pgfpathlineto{\pgfqpoint{4.911097in}{2.568716in}}%
\pgfpathlineto{\pgfqpoint{4.897806in}{2.569007in}}%
\pgfpathlineto{\pgfqpoint{4.884525in}{2.569416in}}%
\pgfpathlineto{\pgfqpoint{4.871253in}{2.569943in}}%
\pgfpathlineto{\pgfqpoint{4.863974in}{2.560009in}}%
\pgfpathlineto{\pgfqpoint{4.856691in}{2.550078in}}%
\pgfpathlineto{\pgfqpoint{4.849403in}{2.540150in}}%
\pgfpathlineto{\pgfqpoint{4.842110in}{2.530222in}}%
\pgfpathclose%
\pgfusepath{fill}%
\end{pgfscope}%
\begin{pgfscope}%
\pgfpathrectangle{\pgfqpoint{1.254980in}{0.150000in}}{\pgfqpoint{5.490039in}{5.490039in}}%
\pgfusepath{clip}%
\pgfsetbuttcap%
\pgfsetroundjoin%
\definecolor{currentfill}{rgb}{0.280255,0.165693,0.476498}%
\pgfsetfillcolor{currentfill}%
\pgfsetfillopacity{0.700000}%
\pgfsetlinewidth{0.000000pt}%
\definecolor{currentstroke}{rgb}{0.000000,0.000000,0.000000}%
\pgfsetstrokecolor{currentstroke}%
\pgfsetdash{}{0pt}%
\pgfpathmoveto{\pgfqpoint{3.622281in}{2.430995in}}%
\pgfpathlineto{\pgfqpoint{3.635298in}{2.421383in}}%
\pgfpathlineto{\pgfqpoint{3.648316in}{2.411918in}}%
\pgfpathlineto{\pgfqpoint{3.661336in}{2.402600in}}%
\pgfpathlineto{\pgfqpoint{3.674357in}{2.393429in}}%
\pgfpathlineto{\pgfqpoint{3.682039in}{2.401948in}}%
\pgfpathlineto{\pgfqpoint{3.689715in}{2.410526in}}%
\pgfpathlineto{\pgfqpoint{3.697385in}{2.419162in}}%
\pgfpathlineto{\pgfqpoint{3.705049in}{2.427854in}}%
\pgfpathlineto{\pgfqpoint{3.692043in}{2.436920in}}%
\pgfpathlineto{\pgfqpoint{3.679038in}{2.446133in}}%
\pgfpathlineto{\pgfqpoint{3.666035in}{2.455492in}}%
\pgfpathlineto{\pgfqpoint{3.653034in}{2.465000in}}%
\pgfpathlineto{\pgfqpoint{3.645355in}{2.456406in}}%
\pgfpathlineto{\pgfqpoint{3.637670in}{2.447874in}}%
\pgfpathlineto{\pgfqpoint{3.629978in}{2.439404in}}%
\pgfpathlineto{\pgfqpoint{3.622281in}{2.430995in}}%
\pgfpathclose%
\pgfusepath{fill}%
\end{pgfscope}%
\begin{pgfscope}%
\pgfpathrectangle{\pgfqpoint{1.254980in}{0.150000in}}{\pgfqpoint{5.490039in}{5.490039in}}%
\pgfusepath{clip}%
\pgfsetbuttcap%
\pgfsetroundjoin%
\definecolor{currentfill}{rgb}{0.263663,0.237631,0.518762}%
\pgfsetfillcolor{currentfill}%
\pgfsetfillopacity{0.700000}%
\pgfsetlinewidth{0.000000pt}%
\definecolor{currentstroke}{rgb}{0.000000,0.000000,0.000000}%
\pgfsetstrokecolor{currentstroke}%
\pgfsetdash{}{0pt}%
\pgfpathmoveto{\pgfqpoint{3.383002in}{2.574430in}}%
\pgfpathlineto{\pgfqpoint{3.396029in}{2.562231in}}%
\pgfpathlineto{\pgfqpoint{3.409056in}{2.550193in}}%
\pgfpathlineto{\pgfqpoint{3.422082in}{2.538313in}}%
\pgfpathlineto{\pgfqpoint{3.435107in}{2.526592in}}%
\pgfpathlineto{\pgfqpoint{3.442880in}{2.534360in}}%
\pgfpathlineto{\pgfqpoint{3.450646in}{2.542205in}}%
\pgfpathlineto{\pgfqpoint{3.458406in}{2.550127in}}%
\pgfpathlineto{\pgfqpoint{3.466159in}{2.558124in}}%
\pgfpathlineto{\pgfqpoint{3.453151in}{2.569722in}}%
\pgfpathlineto{\pgfqpoint{3.440143in}{2.581478in}}%
\pgfpathlineto{\pgfqpoint{3.427135in}{2.593393in}}%
\pgfpathlineto{\pgfqpoint{3.414126in}{2.605468in}}%
\pgfpathlineto{\pgfqpoint{3.406355in}{2.597588in}}%
\pgfpathlineto{\pgfqpoint{3.398578in}{2.589788in}}%
\pgfpathlineto{\pgfqpoint{3.390793in}{2.582068in}}%
\pgfpathlineto{\pgfqpoint{3.383002in}{2.574430in}}%
\pgfpathclose%
\pgfusepath{fill}%
\end{pgfscope}%
\begin{pgfscope}%
\pgfpathrectangle{\pgfqpoint{1.254980in}{0.150000in}}{\pgfqpoint{5.490039in}{5.490039in}}%
\pgfusepath{clip}%
\pgfsetbuttcap%
\pgfsetroundjoin%
\definecolor{currentfill}{rgb}{0.271828,0.209303,0.504434}%
\pgfsetfillcolor{currentfill}%
\pgfsetfillopacity{0.700000}%
\pgfsetlinewidth{0.000000pt}%
\definecolor{currentstroke}{rgb}{0.000000,0.000000,0.000000}%
\pgfsetstrokecolor{currentstroke}%
\pgfsetdash{}{0pt}%
\pgfpathmoveto{\pgfqpoint{4.759823in}{2.493293in}}%
\pgfpathlineto{\pgfqpoint{4.773077in}{2.492417in}}%
\pgfpathlineto{\pgfqpoint{4.786340in}{2.491660in}}%
\pgfpathlineto{\pgfqpoint{4.799613in}{2.491023in}}%
\pgfpathlineto{\pgfqpoint{4.812894in}{2.490504in}}%
\pgfpathlineto{\pgfqpoint{4.820205in}{2.500436in}}%
\pgfpathlineto{\pgfqpoint{4.827511in}{2.510366in}}%
\pgfpathlineto{\pgfqpoint{4.834813in}{2.520294in}}%
\pgfpathlineto{\pgfqpoint{4.842110in}{2.530222in}}%
\pgfpathlineto{\pgfqpoint{4.828838in}{2.530796in}}%
\pgfpathlineto{\pgfqpoint{4.815575in}{2.531489in}}%
\pgfpathlineto{\pgfqpoint{4.802321in}{2.532302in}}%
\pgfpathlineto{\pgfqpoint{4.789076in}{2.533233in}}%
\pgfpathlineto{\pgfqpoint{4.781770in}{2.523244in}}%
\pgfpathlineto{\pgfqpoint{4.774459in}{2.513258in}}%
\pgfpathlineto{\pgfqpoint{4.767143in}{2.503275in}}%
\pgfpathlineto{\pgfqpoint{4.759823in}{2.493293in}}%
\pgfpathclose%
\pgfusepath{fill}%
\end{pgfscope}%
\begin{pgfscope}%
\pgfpathrectangle{\pgfqpoint{1.254980in}{0.150000in}}{\pgfqpoint{5.490039in}{5.490039in}}%
\pgfusepath{clip}%
\pgfsetbuttcap%
\pgfsetroundjoin%
\definecolor{currentfill}{rgb}{0.283187,0.125848,0.444960}%
\pgfsetfillcolor{currentfill}%
\pgfsetfillopacity{0.700000}%
\pgfsetlinewidth{0.000000pt}%
\definecolor{currentstroke}{rgb}{0.000000,0.000000,0.000000}%
\pgfsetstrokecolor{currentstroke}%
\pgfsetdash{}{0pt}%
\pgfpathmoveto{\pgfqpoint{3.943894in}{2.338925in}}%
\pgfpathlineto{\pgfqpoint{3.956938in}{2.332341in}}%
\pgfpathlineto{\pgfqpoint{3.969987in}{2.325892in}}%
\pgfpathlineto{\pgfqpoint{3.983039in}{2.319579in}}%
\pgfpathlineto{\pgfqpoint{3.996096in}{2.313401in}}%
\pgfpathlineto{\pgfqpoint{4.003667in}{2.322762in}}%
\pgfpathlineto{\pgfqpoint{4.011233in}{2.332159in}}%
\pgfpathlineto{\pgfqpoint{4.018794in}{2.341589in}}%
\pgfpathlineto{\pgfqpoint{4.026350in}{2.351053in}}%
\pgfpathlineto{\pgfqpoint{4.013305in}{2.357160in}}%
\pgfpathlineto{\pgfqpoint{4.000264in}{2.363401in}}%
\pgfpathlineto{\pgfqpoint{3.987228in}{2.369778in}}%
\pgfpathlineto{\pgfqpoint{3.974196in}{2.376290in}}%
\pgfpathlineto{\pgfqpoint{3.966628in}{2.366892in}}%
\pgfpathlineto{\pgfqpoint{3.959055in}{2.357532in}}%
\pgfpathlineto{\pgfqpoint{3.951477in}{2.348209in}}%
\pgfpathlineto{\pgfqpoint{3.943894in}{2.338925in}}%
\pgfpathclose%
\pgfusepath{fill}%
\end{pgfscope}%
\begin{pgfscope}%
\pgfpathrectangle{\pgfqpoint{1.254980in}{0.150000in}}{\pgfqpoint{5.490039in}{5.490039in}}%
\pgfusepath{clip}%
\pgfsetbuttcap%
\pgfsetroundjoin%
\definecolor{currentfill}{rgb}{0.153364,0.497000,0.557724}%
\pgfsetfillcolor{currentfill}%
\pgfsetfillopacity{0.700000}%
\pgfsetlinewidth{0.000000pt}%
\definecolor{currentstroke}{rgb}{0.000000,0.000000,0.000000}%
\pgfsetstrokecolor{currentstroke}%
\pgfsetdash{}{0pt}%
\pgfpathmoveto{\pgfqpoint{2.859757in}{3.204587in}}%
\pgfpathlineto{\pgfqpoint{2.872930in}{3.185164in}}%
\pgfpathlineto{\pgfqpoint{2.886096in}{3.165945in}}%
\pgfpathlineto{\pgfqpoint{2.899255in}{3.146928in}}%
\pgfpathlineto{\pgfqpoint{2.912408in}{3.128113in}}%
\pgfpathlineto{\pgfqpoint{2.920394in}{3.134549in}}%
\pgfpathlineto{\pgfqpoint{2.928371in}{3.141104in}}%
\pgfpathlineto{\pgfqpoint{2.936339in}{3.147777in}}%
\pgfpathlineto{\pgfqpoint{2.944298in}{3.154566in}}%
\pgfpathlineto{\pgfqpoint{2.931170in}{3.173249in}}%
\pgfpathlineto{\pgfqpoint{2.918035in}{3.192133in}}%
\pgfpathlineto{\pgfqpoint{2.904894in}{3.211220in}}%
\pgfpathlineto{\pgfqpoint{2.891746in}{3.230510in}}%
\pgfpathlineto{\pgfqpoint{2.883762in}{3.223847in}}%
\pgfpathlineto{\pgfqpoint{2.875769in}{3.217305in}}%
\pgfpathlineto{\pgfqpoint{2.867768in}{3.210885in}}%
\pgfpathlineto{\pgfqpoint{2.859757in}{3.204587in}}%
\pgfpathclose%
\pgfusepath{fill}%
\end{pgfscope}%
\begin{pgfscope}%
\pgfpathrectangle{\pgfqpoint{1.254980in}{0.150000in}}{\pgfqpoint{5.490039in}{5.490039in}}%
\pgfusepath{clip}%
\pgfsetbuttcap%
\pgfsetroundjoin%
\definecolor{currentfill}{rgb}{0.282884,0.135920,0.453427}%
\pgfsetfillcolor{currentfill}%
\pgfsetfillopacity{0.700000}%
\pgfsetlinewidth{0.000000pt}%
\definecolor{currentstroke}{rgb}{0.000000,0.000000,0.000000}%
\pgfsetstrokecolor{currentstroke}%
\pgfsetdash{}{0pt}%
\pgfpathmoveto{\pgfqpoint{3.809184in}{2.360513in}}%
\pgfpathlineto{\pgfqpoint{3.822213in}{2.352735in}}%
\pgfpathlineto{\pgfqpoint{3.835246in}{2.345097in}}%
\pgfpathlineto{\pgfqpoint{3.848281in}{2.337599in}}%
\pgfpathlineto{\pgfqpoint{3.861320in}{2.330239in}}%
\pgfpathlineto{\pgfqpoint{3.868936in}{2.339272in}}%
\pgfpathlineto{\pgfqpoint{3.876548in}{2.348348in}}%
\pgfpathlineto{\pgfqpoint{3.884154in}{2.357468in}}%
\pgfpathlineto{\pgfqpoint{3.891755in}{2.366631in}}%
\pgfpathlineto{\pgfqpoint{3.878730in}{2.373902in}}%
\pgfpathlineto{\pgfqpoint{3.865708in}{2.381312in}}%
\pgfpathlineto{\pgfqpoint{3.852689in}{2.388862in}}%
\pgfpathlineto{\pgfqpoint{3.839673in}{2.396552in}}%
\pgfpathlineto{\pgfqpoint{3.832059in}{2.387471in}}%
\pgfpathlineto{\pgfqpoint{3.824440in}{2.378437in}}%
\pgfpathlineto{\pgfqpoint{3.816815in}{2.369451in}}%
\pgfpathlineto{\pgfqpoint{3.809184in}{2.360513in}}%
\pgfpathclose%
\pgfusepath{fill}%
\end{pgfscope}%
\begin{pgfscope}%
\pgfpathrectangle{\pgfqpoint{1.254980in}{0.150000in}}{\pgfqpoint{5.490039in}{5.490039in}}%
\pgfusepath{clip}%
\pgfsetbuttcap%
\pgfsetroundjoin%
\definecolor{currentfill}{rgb}{0.275191,0.194905,0.496005}%
\pgfsetfillcolor{currentfill}%
\pgfsetfillopacity{0.700000}%
\pgfsetlinewidth{0.000000pt}%
\definecolor{currentstroke}{rgb}{0.000000,0.000000,0.000000}%
\pgfsetstrokecolor{currentstroke}%
\pgfsetdash{}{0pt}%
\pgfpathmoveto{\pgfqpoint{4.677530in}{2.457912in}}%
\pgfpathlineto{\pgfqpoint{4.690759in}{2.456596in}}%
\pgfpathlineto{\pgfqpoint{4.703996in}{2.455400in}}%
\pgfpathlineto{\pgfqpoint{4.717241in}{2.454324in}}%
\pgfpathlineto{\pgfqpoint{4.730496in}{2.453368in}}%
\pgfpathlineto{\pgfqpoint{4.737834in}{2.463350in}}%
\pgfpathlineto{\pgfqpoint{4.745168in}{2.473332in}}%
\pgfpathlineto{\pgfqpoint{4.752498in}{2.483312in}}%
\pgfpathlineto{\pgfqpoint{4.759823in}{2.493293in}}%
\pgfpathlineto{\pgfqpoint{4.746577in}{2.494289in}}%
\pgfpathlineto{\pgfqpoint{4.733341in}{2.495404in}}%
\pgfpathlineto{\pgfqpoint{4.720113in}{2.496640in}}%
\pgfpathlineto{\pgfqpoint{4.706894in}{2.497996in}}%
\pgfpathlineto{\pgfqpoint{4.699560in}{2.487970in}}%
\pgfpathlineto{\pgfqpoint{4.692221in}{2.477947in}}%
\pgfpathlineto{\pgfqpoint{4.684878in}{2.467928in}}%
\pgfpathlineto{\pgfqpoint{4.677530in}{2.457912in}}%
\pgfpathclose%
\pgfusepath{fill}%
\end{pgfscope}%
\begin{pgfscope}%
\pgfpathrectangle{\pgfqpoint{1.254980in}{0.150000in}}{\pgfqpoint{5.490039in}{5.490039in}}%
\pgfusepath{clip}%
\pgfsetbuttcap%
\pgfsetroundjoin%
\definecolor{currentfill}{rgb}{0.282884,0.135920,0.453427}%
\pgfsetfillcolor{currentfill}%
\pgfsetfillopacity{0.700000}%
\pgfsetlinewidth{0.000000pt}%
\definecolor{currentstroke}{rgb}{0.000000,0.000000,0.000000}%
\pgfsetstrokecolor{currentstroke}%
\pgfsetdash{}{0pt}%
\pgfpathmoveto{\pgfqpoint{4.295677in}{2.349733in}}%
\pgfpathlineto{\pgfqpoint{4.308793in}{2.345943in}}%
\pgfpathlineto{\pgfqpoint{4.321915in}{2.342280in}}%
\pgfpathlineto{\pgfqpoint{4.335044in}{2.338743in}}%
\pgfpathlineto{\pgfqpoint{4.348180in}{2.335333in}}%
\pgfpathlineto{\pgfqpoint{4.355639in}{2.345244in}}%
\pgfpathlineto{\pgfqpoint{4.363093in}{2.355168in}}%
\pgfpathlineto{\pgfqpoint{4.370543in}{2.365106in}}%
\pgfpathlineto{\pgfqpoint{4.377988in}{2.375057in}}%
\pgfpathlineto{\pgfqpoint{4.364863in}{2.378443in}}%
\pgfpathlineto{\pgfqpoint{4.351744in}{2.381956in}}%
\pgfpathlineto{\pgfqpoint{4.338631in}{2.385595in}}%
\pgfpathlineto{\pgfqpoint{4.325525in}{2.389361in}}%
\pgfpathlineto{\pgfqpoint{4.318070in}{2.379428in}}%
\pgfpathlineto{\pgfqpoint{4.310611in}{2.369513in}}%
\pgfpathlineto{\pgfqpoint{4.303146in}{2.359614in}}%
\pgfpathlineto{\pgfqpoint{4.295677in}{2.349733in}}%
\pgfpathclose%
\pgfusepath{fill}%
\end{pgfscope}%
\begin{pgfscope}%
\pgfpathrectangle{\pgfqpoint{1.254980in}{0.150000in}}{\pgfqpoint{5.490039in}{5.490039in}}%
\pgfusepath{clip}%
\pgfsetbuttcap%
\pgfsetroundjoin%
\definecolor{currentfill}{rgb}{0.270595,0.214069,0.507052}%
\pgfsetfillcolor{currentfill}%
\pgfsetfillopacity{0.700000}%
\pgfsetlinewidth{0.000000pt}%
\definecolor{currentstroke}{rgb}{0.000000,0.000000,0.000000}%
\pgfsetstrokecolor{currentstroke}%
\pgfsetdash{}{0pt}%
\pgfpathmoveto{\pgfqpoint{3.435107in}{2.526592in}}%
\pgfpathlineto{\pgfqpoint{3.448133in}{2.515029in}}%
\pgfpathlineto{\pgfqpoint{3.461158in}{2.503622in}}%
\pgfpathlineto{\pgfqpoint{3.474183in}{2.492370in}}%
\pgfpathlineto{\pgfqpoint{3.487208in}{2.481274in}}%
\pgfpathlineto{\pgfqpoint{3.494963in}{2.489171in}}%
\pgfpathlineto{\pgfqpoint{3.502712in}{2.497141in}}%
\pgfpathlineto{\pgfqpoint{3.510454in}{2.505183in}}%
\pgfpathlineto{\pgfqpoint{3.518190in}{2.513297in}}%
\pgfpathlineto{\pgfqpoint{3.505182in}{2.524270in}}%
\pgfpathlineto{\pgfqpoint{3.492174in}{2.535399in}}%
\pgfpathlineto{\pgfqpoint{3.479166in}{2.546683in}}%
\pgfpathlineto{\pgfqpoint{3.466159in}{2.558124in}}%
\pgfpathlineto{\pgfqpoint{3.458406in}{2.550127in}}%
\pgfpathlineto{\pgfqpoint{3.450646in}{2.542205in}}%
\pgfpathlineto{\pgfqpoint{3.442880in}{2.534360in}}%
\pgfpathlineto{\pgfqpoint{3.435107in}{2.526592in}}%
\pgfpathclose%
\pgfusepath{fill}%
\end{pgfscope}%
\begin{pgfscope}%
\pgfpathrectangle{\pgfqpoint{1.254980in}{0.150000in}}{\pgfqpoint{5.490039in}{5.490039in}}%
\pgfusepath{clip}%
\pgfsetbuttcap%
\pgfsetroundjoin%
\definecolor{currentfill}{rgb}{0.283229,0.120777,0.440584}%
\pgfsetfillcolor{currentfill}%
\pgfsetfillopacity{0.700000}%
\pgfsetlinewidth{0.000000pt}%
\definecolor{currentstroke}{rgb}{0.000000,0.000000,0.000000}%
\pgfsetstrokecolor{currentstroke}%
\pgfsetdash{}{0pt}%
\pgfpathmoveto{\pgfqpoint{4.078575in}{2.327964in}}%
\pgfpathlineto{\pgfqpoint{4.091644in}{2.322524in}}%
\pgfpathlineto{\pgfqpoint{4.104718in}{2.317216in}}%
\pgfpathlineto{\pgfqpoint{4.117796in}{2.312040in}}%
\pgfpathlineto{\pgfqpoint{4.130880in}{2.306994in}}%
\pgfpathlineto{\pgfqpoint{4.138409in}{2.316615in}}%
\pgfpathlineto{\pgfqpoint{4.145932in}{2.326262in}}%
\pgfpathlineto{\pgfqpoint{4.153450in}{2.335934in}}%
\pgfpathlineto{\pgfqpoint{4.160964in}{2.345631in}}%
\pgfpathlineto{\pgfqpoint{4.147891in}{2.350621in}}%
\pgfpathlineto{\pgfqpoint{4.134823in}{2.355742in}}%
\pgfpathlineto{\pgfqpoint{4.121761in}{2.360994in}}%
\pgfpathlineto{\pgfqpoint{4.108703in}{2.366378in}}%
\pgfpathlineto{\pgfqpoint{4.101179in}{2.356731in}}%
\pgfpathlineto{\pgfqpoint{4.093649in}{2.347113in}}%
\pgfpathlineto{\pgfqpoint{4.086115in}{2.337524in}}%
\pgfpathlineto{\pgfqpoint{4.078575in}{2.327964in}}%
\pgfpathclose%
\pgfusepath{fill}%
\end{pgfscope}%
\begin{pgfscope}%
\pgfpathrectangle{\pgfqpoint{1.254980in}{0.150000in}}{\pgfqpoint{5.490039in}{5.490039in}}%
\pgfusepath{clip}%
\pgfsetbuttcap%
\pgfsetroundjoin%
\definecolor{currentfill}{rgb}{0.278826,0.175490,0.483397}%
\pgfsetfillcolor{currentfill}%
\pgfsetfillopacity{0.700000}%
\pgfsetlinewidth{0.000000pt}%
\definecolor{currentstroke}{rgb}{0.000000,0.000000,0.000000}%
\pgfsetstrokecolor{currentstroke}%
\pgfsetdash{}{0pt}%
\pgfpathmoveto{\pgfqpoint{4.595227in}{2.424246in}}%
\pgfpathlineto{\pgfqpoint{4.608431in}{2.422468in}}%
\pgfpathlineto{\pgfqpoint{4.621644in}{2.420813in}}%
\pgfpathlineto{\pgfqpoint{4.634864in}{2.419278in}}%
\pgfpathlineto{\pgfqpoint{4.648093in}{2.417865in}}%
\pgfpathlineto{\pgfqpoint{4.655459in}{2.427876in}}%
\pgfpathlineto{\pgfqpoint{4.662821in}{2.437887in}}%
\pgfpathlineto{\pgfqpoint{4.670178in}{2.447899in}}%
\pgfpathlineto{\pgfqpoint{4.677530in}{2.457912in}}%
\pgfpathlineto{\pgfqpoint{4.664310in}{2.459350in}}%
\pgfpathlineto{\pgfqpoint{4.651099in}{2.460908in}}%
\pgfpathlineto{\pgfqpoint{4.637896in}{2.462587in}}%
\pgfpathlineto{\pgfqpoint{4.624701in}{2.464388in}}%
\pgfpathlineto{\pgfqpoint{4.617340in}{2.454344in}}%
\pgfpathlineto{\pgfqpoint{4.609973in}{2.444307in}}%
\pgfpathlineto{\pgfqpoint{4.602603in}{2.434274in}}%
\pgfpathlineto{\pgfqpoint{4.595227in}{2.424246in}}%
\pgfpathclose%
\pgfusepath{fill}%
\end{pgfscope}%
\begin{pgfscope}%
\pgfpathrectangle{\pgfqpoint{1.254980in}{0.150000in}}{\pgfqpoint{5.490039in}{5.490039in}}%
\pgfusepath{clip}%
\pgfsetbuttcap%
\pgfsetroundjoin%
\definecolor{currentfill}{rgb}{0.281412,0.155834,0.469201}%
\pgfsetfillcolor{currentfill}%
\pgfsetfillopacity{0.700000}%
\pgfsetlinewidth{0.000000pt}%
\definecolor{currentstroke}{rgb}{0.000000,0.000000,0.000000}%
\pgfsetstrokecolor{currentstroke}%
\pgfsetdash{}{0pt}%
\pgfpathmoveto{\pgfqpoint{3.674357in}{2.393429in}}%
\pgfpathlineto{\pgfqpoint{3.687381in}{2.384403in}}%
\pgfpathlineto{\pgfqpoint{3.700406in}{2.375523in}}%
\pgfpathlineto{\pgfqpoint{3.713434in}{2.366787in}}%
\pgfpathlineto{\pgfqpoint{3.726464in}{2.358195in}}%
\pgfpathlineto{\pgfqpoint{3.734130in}{2.366825in}}%
\pgfpathlineto{\pgfqpoint{3.741791in}{2.375509in}}%
\pgfpathlineto{\pgfqpoint{3.749447in}{2.384247in}}%
\pgfpathlineto{\pgfqpoint{3.757097in}{2.393038in}}%
\pgfpathlineto{\pgfqpoint{3.744081in}{2.401526in}}%
\pgfpathlineto{\pgfqpoint{3.731068in}{2.410157in}}%
\pgfpathlineto{\pgfqpoint{3.718058in}{2.418933in}}%
\pgfpathlineto{\pgfqpoint{3.705049in}{2.427854in}}%
\pgfpathlineto{\pgfqpoint{3.697385in}{2.419162in}}%
\pgfpathlineto{\pgfqpoint{3.689715in}{2.410526in}}%
\pgfpathlineto{\pgfqpoint{3.682039in}{2.401948in}}%
\pgfpathlineto{\pgfqpoint{3.674357in}{2.393429in}}%
\pgfpathclose%
\pgfusepath{fill}%
\end{pgfscope}%
\begin{pgfscope}%
\pgfpathrectangle{\pgfqpoint{1.254980in}{0.150000in}}{\pgfqpoint{5.490039in}{5.490039in}}%
\pgfusepath{clip}%
\pgfsetbuttcap%
\pgfsetroundjoin%
\definecolor{currentfill}{rgb}{0.141935,0.526453,0.555991}%
\pgfsetfillcolor{currentfill}%
\pgfsetfillopacity{0.700000}%
\pgfsetlinewidth{0.000000pt}%
\definecolor{currentstroke}{rgb}{0.000000,0.000000,0.000000}%
\pgfsetstrokecolor{currentstroke}%
\pgfsetdash{}{0pt}%
\pgfpathmoveto{\pgfqpoint{2.806993in}{3.284349in}}%
\pgfpathlineto{\pgfqpoint{2.820195in}{3.264095in}}%
\pgfpathlineto{\pgfqpoint{2.833390in}{3.244051in}}%
\pgfpathlineto{\pgfqpoint{2.846577in}{3.224216in}}%
\pgfpathlineto{\pgfqpoint{2.859757in}{3.204587in}}%
\pgfpathlineto{\pgfqpoint{2.867768in}{3.210885in}}%
\pgfpathlineto{\pgfqpoint{2.875769in}{3.217305in}}%
\pgfpathlineto{\pgfqpoint{2.883762in}{3.223847in}}%
\pgfpathlineto{\pgfqpoint{2.891746in}{3.230510in}}%
\pgfpathlineto{\pgfqpoint{2.878591in}{3.250006in}}%
\pgfpathlineto{\pgfqpoint{2.865430in}{3.269708in}}%
\pgfpathlineto{\pgfqpoint{2.852261in}{3.289618in}}%
\pgfpathlineto{\pgfqpoint{2.839085in}{3.309737in}}%
\pgfpathlineto{\pgfqpoint{2.831076in}{3.303202in}}%
\pgfpathlineto{\pgfqpoint{2.823057in}{3.296791in}}%
\pgfpathlineto{\pgfqpoint{2.815030in}{3.290506in}}%
\pgfpathlineto{\pgfqpoint{2.806993in}{3.284349in}}%
\pgfpathclose%
\pgfusepath{fill}%
\end{pgfscope}%
\begin{pgfscope}%
\pgfpathrectangle{\pgfqpoint{1.254980in}{0.150000in}}{\pgfqpoint{5.490039in}{5.490039in}}%
\pgfusepath{clip}%
\pgfsetbuttcap%
\pgfsetroundjoin%
\definecolor{currentfill}{rgb}{0.275191,0.194905,0.496005}%
\pgfsetfillcolor{currentfill}%
\pgfsetfillopacity{0.700000}%
\pgfsetlinewidth{0.000000pt}%
\definecolor{currentstroke}{rgb}{0.000000,0.000000,0.000000}%
\pgfsetstrokecolor{currentstroke}%
\pgfsetdash{}{0pt}%
\pgfpathmoveto{\pgfqpoint{3.487208in}{2.481274in}}%
\pgfpathlineto{\pgfqpoint{3.500234in}{2.470333in}}%
\pgfpathlineto{\pgfqpoint{3.513260in}{2.459544in}}%
\pgfpathlineto{\pgfqpoint{3.526287in}{2.448909in}}%
\pgfpathlineto{\pgfqpoint{3.539314in}{2.438426in}}%
\pgfpathlineto{\pgfqpoint{3.547052in}{2.446450in}}%
\pgfpathlineto{\pgfqpoint{3.554784in}{2.454544in}}%
\pgfpathlineto{\pgfqpoint{3.562509in}{2.462706in}}%
\pgfpathlineto{\pgfqpoint{3.570228in}{2.470936in}}%
\pgfpathlineto{\pgfqpoint{3.557218in}{2.481298in}}%
\pgfpathlineto{\pgfqpoint{3.544208in}{2.491811in}}%
\pgfpathlineto{\pgfqpoint{3.531199in}{2.502477in}}%
\pgfpathlineto{\pgfqpoint{3.518190in}{2.513297in}}%
\pgfpathlineto{\pgfqpoint{3.510454in}{2.505183in}}%
\pgfpathlineto{\pgfqpoint{3.502712in}{2.497141in}}%
\pgfpathlineto{\pgfqpoint{3.494963in}{2.489171in}}%
\pgfpathlineto{\pgfqpoint{3.487208in}{2.481274in}}%
\pgfpathclose%
\pgfusepath{fill}%
\end{pgfscope}%
\begin{pgfscope}%
\pgfpathrectangle{\pgfqpoint{1.254980in}{0.150000in}}{\pgfqpoint{5.490039in}{5.490039in}}%
\pgfusepath{clip}%
\pgfsetbuttcap%
\pgfsetroundjoin%
\definecolor{currentfill}{rgb}{0.280868,0.160771,0.472899}%
\pgfsetfillcolor{currentfill}%
\pgfsetfillopacity{0.700000}%
\pgfsetlinewidth{0.000000pt}%
\definecolor{currentstroke}{rgb}{0.000000,0.000000,0.000000}%
\pgfsetstrokecolor{currentstroke}%
\pgfsetdash{}{0pt}%
\pgfpathmoveto{\pgfqpoint{4.512907in}{2.392468in}}%
\pgfpathlineto{\pgfqpoint{4.526088in}{2.390209in}}%
\pgfpathlineto{\pgfqpoint{4.539277in}{2.388074in}}%
\pgfpathlineto{\pgfqpoint{4.552475in}{2.386060in}}%
\pgfpathlineto{\pgfqpoint{4.565680in}{2.384169in}}%
\pgfpathlineto{\pgfqpoint{4.573074in}{2.394184in}}%
\pgfpathlineto{\pgfqpoint{4.580463in}{2.404201in}}%
\pgfpathlineto{\pgfqpoint{4.587847in}{2.414221in}}%
\pgfpathlineto{\pgfqpoint{4.595227in}{2.424246in}}%
\pgfpathlineto{\pgfqpoint{4.582031in}{2.426145in}}%
\pgfpathlineto{\pgfqpoint{4.568843in}{2.428166in}}%
\pgfpathlineto{\pgfqpoint{4.555663in}{2.430309in}}%
\pgfpathlineto{\pgfqpoint{4.542491in}{2.432576in}}%
\pgfpathlineto{\pgfqpoint{4.535102in}{2.422538in}}%
\pgfpathlineto{\pgfqpoint{4.527708in}{2.412508in}}%
\pgfpathlineto{\pgfqpoint{4.520310in}{2.402484in}}%
\pgfpathlineto{\pgfqpoint{4.512907in}{2.392468in}}%
\pgfpathclose%
\pgfusepath{fill}%
\end{pgfscope}%
\begin{pgfscope}%
\pgfpathrectangle{\pgfqpoint{1.254980in}{0.150000in}}{\pgfqpoint{5.490039in}{5.490039in}}%
\pgfusepath{clip}%
\pgfsetbuttcap%
\pgfsetroundjoin%
\definecolor{currentfill}{rgb}{0.283187,0.125848,0.444960}%
\pgfsetfillcolor{currentfill}%
\pgfsetfillopacity{0.700000}%
\pgfsetlinewidth{0.000000pt}%
\definecolor{currentstroke}{rgb}{0.000000,0.000000,0.000000}%
\pgfsetstrokecolor{currentstroke}%
\pgfsetdash{}{0pt}%
\pgfpathmoveto{\pgfqpoint{4.213312in}{2.326973in}}%
\pgfpathlineto{\pgfqpoint{4.226413in}{2.322631in}}%
\pgfpathlineto{\pgfqpoint{4.239521in}{2.318419in}}%
\pgfpathlineto{\pgfqpoint{4.252634in}{2.314334in}}%
\pgfpathlineto{\pgfqpoint{4.265754in}{2.310377in}}%
\pgfpathlineto{\pgfqpoint{4.273242in}{2.320190in}}%
\pgfpathlineto{\pgfqpoint{4.280725in}{2.330021in}}%
\pgfpathlineto{\pgfqpoint{4.288203in}{2.339868in}}%
\pgfpathlineto{\pgfqpoint{4.295677in}{2.349733in}}%
\pgfpathlineto{\pgfqpoint{4.282568in}{2.353650in}}%
\pgfpathlineto{\pgfqpoint{4.269464in}{2.357695in}}%
\pgfpathlineto{\pgfqpoint{4.256367in}{2.361868in}}%
\pgfpathlineto{\pgfqpoint{4.243276in}{2.366170in}}%
\pgfpathlineto{\pgfqpoint{4.235792in}{2.356339in}}%
\pgfpathlineto{\pgfqpoint{4.228304in}{2.346529in}}%
\pgfpathlineto{\pgfqpoint{4.220810in}{2.336741in}}%
\pgfpathlineto{\pgfqpoint{4.213312in}{2.326973in}}%
\pgfpathclose%
\pgfusepath{fill}%
\end{pgfscope}%
\begin{pgfscope}%
\pgfpathrectangle{\pgfqpoint{1.254980in}{0.150000in}}{\pgfqpoint{5.490039in}{5.490039in}}%
\pgfusepath{clip}%
\pgfsetbuttcap%
\pgfsetroundjoin%
\definecolor{currentfill}{rgb}{0.283187,0.125848,0.444960}%
\pgfsetfillcolor{currentfill}%
\pgfsetfillopacity{0.700000}%
\pgfsetlinewidth{0.000000pt}%
\definecolor{currentstroke}{rgb}{0.000000,0.000000,0.000000}%
\pgfsetstrokecolor{currentstroke}%
\pgfsetdash{}{0pt}%
\pgfpathmoveto{\pgfqpoint{3.861320in}{2.330239in}}%
\pgfpathlineto{\pgfqpoint{3.874362in}{2.323019in}}%
\pgfpathlineto{\pgfqpoint{3.887407in}{2.315936in}}%
\pgfpathlineto{\pgfqpoint{3.900456in}{2.308991in}}%
\pgfpathlineto{\pgfqpoint{3.913509in}{2.302182in}}%
\pgfpathlineto{\pgfqpoint{3.921113in}{2.311308in}}%
\pgfpathlineto{\pgfqpoint{3.928712in}{2.320474in}}%
\pgfpathlineto{\pgfqpoint{3.936306in}{2.329680in}}%
\pgfpathlineto{\pgfqpoint{3.943894in}{2.338925in}}%
\pgfpathlineto{\pgfqpoint{3.930854in}{2.345646in}}%
\pgfpathlineto{\pgfqpoint{3.917817in}{2.352504in}}%
\pgfpathlineto{\pgfqpoint{3.904784in}{2.359498in}}%
\pgfpathlineto{\pgfqpoint{3.891755in}{2.366631in}}%
\pgfpathlineto{\pgfqpoint{3.884154in}{2.357468in}}%
\pgfpathlineto{\pgfqpoint{3.876548in}{2.348348in}}%
\pgfpathlineto{\pgfqpoint{3.868936in}{2.339272in}}%
\pgfpathlineto{\pgfqpoint{3.861320in}{2.330239in}}%
\pgfpathclose%
\pgfusepath{fill}%
\end{pgfscope}%
\begin{pgfscope}%
\pgfpathrectangle{\pgfqpoint{1.254980in}{0.150000in}}{\pgfqpoint{5.490039in}{5.490039in}}%
\pgfusepath{clip}%
\pgfsetbuttcap%
\pgfsetroundjoin%
\definecolor{currentfill}{rgb}{0.283229,0.120777,0.440584}%
\pgfsetfillcolor{currentfill}%
\pgfsetfillopacity{0.700000}%
\pgfsetlinewidth{0.000000pt}%
\definecolor{currentstroke}{rgb}{0.000000,0.000000,0.000000}%
\pgfsetstrokecolor{currentstroke}%
\pgfsetdash{}{0pt}%
\pgfpathmoveto{\pgfqpoint{3.996096in}{2.313401in}}%
\pgfpathlineto{\pgfqpoint{4.009157in}{2.307357in}}%
\pgfpathlineto{\pgfqpoint{4.022223in}{2.301446in}}%
\pgfpathlineto{\pgfqpoint{4.035293in}{2.295669in}}%
\pgfpathlineto{\pgfqpoint{4.048369in}{2.290025in}}%
\pgfpathlineto{\pgfqpoint{4.055928in}{2.299465in}}%
\pgfpathlineto{\pgfqpoint{4.063482in}{2.308934in}}%
\pgfpathlineto{\pgfqpoint{4.071031in}{2.318434in}}%
\pgfpathlineto{\pgfqpoint{4.078575in}{2.327964in}}%
\pgfpathlineto{\pgfqpoint{4.065512in}{2.333537in}}%
\pgfpathlineto{\pgfqpoint{4.052453in}{2.339242in}}%
\pgfpathlineto{\pgfqpoint{4.039399in}{2.345081in}}%
\pgfpathlineto{\pgfqpoint{4.026350in}{2.351053in}}%
\pgfpathlineto{\pgfqpoint{4.018794in}{2.341589in}}%
\pgfpathlineto{\pgfqpoint{4.011233in}{2.332159in}}%
\pgfpathlineto{\pgfqpoint{4.003667in}{2.322762in}}%
\pgfpathlineto{\pgfqpoint{3.996096in}{2.313401in}}%
\pgfpathclose%
\pgfusepath{fill}%
\end{pgfscope}%
\begin{pgfscope}%
\pgfpathrectangle{\pgfqpoint{1.254980in}{0.150000in}}{\pgfqpoint{5.490039in}{5.490039in}}%
\pgfusepath{clip}%
\pgfsetbuttcap%
\pgfsetroundjoin%
\definecolor{currentfill}{rgb}{0.281887,0.150881,0.465405}%
\pgfsetfillcolor{currentfill}%
\pgfsetfillopacity{0.700000}%
\pgfsetlinewidth{0.000000pt}%
\definecolor{currentstroke}{rgb}{0.000000,0.000000,0.000000}%
\pgfsetstrokecolor{currentstroke}%
\pgfsetdash{}{0pt}%
\pgfpathmoveto{\pgfqpoint{4.430560in}{2.362765in}}%
\pgfpathlineto{\pgfqpoint{4.443721in}{2.360004in}}%
\pgfpathlineto{\pgfqpoint{4.456890in}{2.357368in}}%
\pgfpathlineto{\pgfqpoint{4.470065in}{2.354855in}}%
\pgfpathlineto{\pgfqpoint{4.483248in}{2.352465in}}%
\pgfpathlineto{\pgfqpoint{4.490670in}{2.362457in}}%
\pgfpathlineto{\pgfqpoint{4.498087in}{2.372455in}}%
\pgfpathlineto{\pgfqpoint{4.505499in}{2.382458in}}%
\pgfpathlineto{\pgfqpoint{4.512907in}{2.392468in}}%
\pgfpathlineto{\pgfqpoint{4.499733in}{2.394850in}}%
\pgfpathlineto{\pgfqpoint{4.486566in}{2.397355in}}%
\pgfpathlineto{\pgfqpoint{4.473408in}{2.399984in}}%
\pgfpathlineto{\pgfqpoint{4.460256in}{2.402737in}}%
\pgfpathlineto{\pgfqpoint{4.452839in}{2.392729in}}%
\pgfpathlineto{\pgfqpoint{4.445418in}{2.382731in}}%
\pgfpathlineto{\pgfqpoint{4.437991in}{2.372743in}}%
\pgfpathlineto{\pgfqpoint{4.430560in}{2.362765in}}%
\pgfpathclose%
\pgfusepath{fill}%
\end{pgfscope}%
\begin{pgfscope}%
\pgfpathrectangle{\pgfqpoint{1.254980in}{0.150000in}}{\pgfqpoint{5.490039in}{5.490039in}}%
\pgfusepath{clip}%
\pgfsetbuttcap%
\pgfsetroundjoin%
\definecolor{currentfill}{rgb}{0.216210,0.351535,0.550627}%
\pgfsetfillcolor{currentfill}%
\pgfsetfillopacity{0.700000}%
\pgfsetlinewidth{0.000000pt}%
\definecolor{currentstroke}{rgb}{0.000000,0.000000,0.000000}%
\pgfsetstrokecolor{currentstroke}%
\pgfsetdash{}{0pt}%
\pgfpathmoveto{\pgfqpoint{3.090468in}{2.826482in}}%
\pgfpathlineto{\pgfqpoint{3.103559in}{2.810748in}}%
\pgfpathlineto{\pgfqpoint{3.116647in}{2.795193in}}%
\pgfpathlineto{\pgfqpoint{3.129731in}{2.779817in}}%
\pgfpathlineto{\pgfqpoint{3.142812in}{2.764617in}}%
\pgfpathlineto{\pgfqpoint{3.150714in}{2.771332in}}%
\pgfpathlineto{\pgfqpoint{3.158608in}{2.778148in}}%
\pgfpathlineto{\pgfqpoint{3.166494in}{2.785064in}}%
\pgfpathlineto{\pgfqpoint{3.174372in}{2.792080in}}%
\pgfpathlineto{\pgfqpoint{3.161313in}{2.807136in}}%
\pgfpathlineto{\pgfqpoint{3.148251in}{2.822369in}}%
\pgfpathlineto{\pgfqpoint{3.135186in}{2.837780in}}%
\pgfpathlineto{\pgfqpoint{3.122117in}{2.853370in}}%
\pgfpathlineto{\pgfqpoint{3.114216in}{2.846492in}}%
\pgfpathlineto{\pgfqpoint{3.106308in}{2.839717in}}%
\pgfpathlineto{\pgfqpoint{3.098392in}{2.833047in}}%
\pgfpathlineto{\pgfqpoint{3.090468in}{2.826482in}}%
\pgfpathclose%
\pgfusepath{fill}%
\end{pgfscope}%
\begin{pgfscope}%
\pgfpathrectangle{\pgfqpoint{1.254980in}{0.150000in}}{\pgfqpoint{5.490039in}{5.490039in}}%
\pgfusepath{clip}%
\pgfsetbuttcap%
\pgfsetroundjoin%
\definecolor{currentfill}{rgb}{0.227802,0.326594,0.546532}%
\pgfsetfillcolor{currentfill}%
\pgfsetfillopacity{0.700000}%
\pgfsetlinewidth{0.000000pt}%
\definecolor{currentstroke}{rgb}{0.000000,0.000000,0.000000}%
\pgfsetstrokecolor{currentstroke}%
\pgfsetdash{}{0pt}%
\pgfpathmoveto{\pgfqpoint{3.142812in}{2.764617in}}%
\pgfpathlineto{\pgfqpoint{3.155889in}{2.749594in}}%
\pgfpathlineto{\pgfqpoint{3.168964in}{2.734746in}}%
\pgfpathlineto{\pgfqpoint{3.182036in}{2.720072in}}%
\pgfpathlineto{\pgfqpoint{3.195105in}{2.705571in}}%
\pgfpathlineto{\pgfqpoint{3.202985in}{2.712435in}}%
\pgfpathlineto{\pgfqpoint{3.210857in}{2.719396in}}%
\pgfpathlineto{\pgfqpoint{3.218722in}{2.726452in}}%
\pgfpathlineto{\pgfqpoint{3.226579in}{2.733604in}}%
\pgfpathlineto{\pgfqpoint{3.213532in}{2.747963in}}%
\pgfpathlineto{\pgfqpoint{3.200481in}{2.762494in}}%
\pgfpathlineto{\pgfqpoint{3.187428in}{2.777200in}}%
\pgfpathlineto{\pgfqpoint{3.174372in}{2.792080in}}%
\pgfpathlineto{\pgfqpoint{3.166494in}{2.785064in}}%
\pgfpathlineto{\pgfqpoint{3.158608in}{2.778148in}}%
\pgfpathlineto{\pgfqpoint{3.150714in}{2.771332in}}%
\pgfpathlineto{\pgfqpoint{3.142812in}{2.764617in}}%
\pgfpathclose%
\pgfusepath{fill}%
\end{pgfscope}%
\begin{pgfscope}%
\pgfpathrectangle{\pgfqpoint{1.254980in}{0.150000in}}{\pgfqpoint{5.490039in}{5.490039in}}%
\pgfusepath{clip}%
\pgfsetbuttcap%
\pgfsetroundjoin%
\definecolor{currentfill}{rgb}{0.223925,0.334994,0.548053}%
\pgfsetfillcolor{currentfill}%
\pgfsetfillopacity{0.700000}%
\pgfsetlinewidth{0.000000pt}%
\definecolor{currentstroke}{rgb}{0.000000,0.000000,0.000000}%
\pgfsetstrokecolor{currentstroke}%
\pgfsetdash{}{0pt}%
\pgfpathmoveto{\pgfqpoint{5.307429in}{2.739035in}}%
\pgfpathlineto{\pgfqpoint{5.320911in}{2.740840in}}%
\pgfpathlineto{\pgfqpoint{5.334404in}{2.742758in}}%
\pgfpathlineto{\pgfqpoint{5.347909in}{2.744789in}}%
\pgfpathlineto{\pgfqpoint{5.361425in}{2.746934in}}%
\pgfpathlineto{\pgfqpoint{5.368551in}{2.756079in}}%
\pgfpathlineto{\pgfqpoint{5.375671in}{2.765223in}}%
\pgfpathlineto{\pgfqpoint{5.382788in}{2.774368in}}%
\pgfpathlineto{\pgfqpoint{5.389899in}{2.783514in}}%
\pgfpathlineto{\pgfqpoint{5.376395in}{2.781522in}}%
\pgfpathlineto{\pgfqpoint{5.362902in}{2.779644in}}%
\pgfpathlineto{\pgfqpoint{5.349422in}{2.777878in}}%
\pgfpathlineto{\pgfqpoint{5.335953in}{2.776226in}}%
\pgfpathlineto{\pgfqpoint{5.328828in}{2.766921in}}%
\pgfpathlineto{\pgfqpoint{5.321700in}{2.757622in}}%
\pgfpathlineto{\pgfqpoint{5.314567in}{2.748328in}}%
\pgfpathlineto{\pgfqpoint{5.307429in}{2.739035in}}%
\pgfpathclose%
\pgfusepath{fill}%
\end{pgfscope}%
\begin{pgfscope}%
\pgfpathrectangle{\pgfqpoint{1.254980in}{0.150000in}}{\pgfqpoint{5.490039in}{5.490039in}}%
\pgfusepath{clip}%
\pgfsetbuttcap%
\pgfsetroundjoin%
\definecolor{currentfill}{rgb}{0.233603,0.313828,0.543914}%
\pgfsetfillcolor{currentfill}%
\pgfsetfillopacity{0.700000}%
\pgfsetlinewidth{0.000000pt}%
\definecolor{currentstroke}{rgb}{0.000000,0.000000,0.000000}%
\pgfsetstrokecolor{currentstroke}%
\pgfsetdash{}{0pt}%
\pgfpathmoveto{\pgfqpoint{5.224975in}{2.695234in}}%
\pgfpathlineto{\pgfqpoint{5.238423in}{2.696719in}}%
\pgfpathlineto{\pgfqpoint{5.251881in}{2.698319in}}%
\pgfpathlineto{\pgfqpoint{5.265352in}{2.700032in}}%
\pgfpathlineto{\pgfqpoint{5.278833in}{2.701860in}}%
\pgfpathlineto{\pgfqpoint{5.285989in}{2.711158in}}%
\pgfpathlineto{\pgfqpoint{5.293140in}{2.720452in}}%
\pgfpathlineto{\pgfqpoint{5.300287in}{2.729744in}}%
\pgfpathlineto{\pgfqpoint{5.307429in}{2.739035in}}%
\pgfpathlineto{\pgfqpoint{5.293959in}{2.737344in}}%
\pgfpathlineto{\pgfqpoint{5.280501in}{2.735767in}}%
\pgfpathlineto{\pgfqpoint{5.267054in}{2.734304in}}%
\pgfpathlineto{\pgfqpoint{5.253618in}{2.732955in}}%
\pgfpathlineto{\pgfqpoint{5.246464in}{2.723521in}}%
\pgfpathlineto{\pgfqpoint{5.239306in}{2.714091in}}%
\pgfpathlineto{\pgfqpoint{5.232143in}{2.704662in}}%
\pgfpathlineto{\pgfqpoint{5.224975in}{2.695234in}}%
\pgfpathclose%
\pgfusepath{fill}%
\end{pgfscope}%
\begin{pgfscope}%
\pgfpathrectangle{\pgfqpoint{1.254980in}{0.150000in}}{\pgfqpoint{5.490039in}{5.490039in}}%
\pgfusepath{clip}%
\pgfsetbuttcap%
\pgfsetroundjoin%
\definecolor{currentfill}{rgb}{0.203063,0.379716,0.553925}%
\pgfsetfillcolor{currentfill}%
\pgfsetfillopacity{0.700000}%
\pgfsetlinewidth{0.000000pt}%
\definecolor{currentstroke}{rgb}{0.000000,0.000000,0.000000}%
\pgfsetstrokecolor{currentstroke}%
\pgfsetdash{}{0pt}%
\pgfpathmoveto{\pgfqpoint{3.038062in}{2.891235in}}%
\pgfpathlineto{\pgfqpoint{3.051170in}{2.874772in}}%
\pgfpathlineto{\pgfqpoint{3.064273in}{2.858493in}}%
\pgfpathlineto{\pgfqpoint{3.077372in}{2.842397in}}%
\pgfpathlineto{\pgfqpoint{3.090468in}{2.826482in}}%
\pgfpathlineto{\pgfqpoint{3.098392in}{2.833047in}}%
\pgfpathlineto{\pgfqpoint{3.106308in}{2.839717in}}%
\pgfpathlineto{\pgfqpoint{3.114216in}{2.846492in}}%
\pgfpathlineto{\pgfqpoint{3.122117in}{2.853370in}}%
\pgfpathlineto{\pgfqpoint{3.109044in}{2.869140in}}%
\pgfpathlineto{\pgfqpoint{3.095968in}{2.885092in}}%
\pgfpathlineto{\pgfqpoint{3.082887in}{2.901226in}}%
\pgfpathlineto{\pgfqpoint{3.069803in}{2.917543in}}%
\pgfpathlineto{\pgfqpoint{3.061880in}{2.910804in}}%
\pgfpathlineto{\pgfqpoint{3.053949in}{2.904172in}}%
\pgfpathlineto{\pgfqpoint{3.046010in}{2.897649in}}%
\pgfpathlineto{\pgfqpoint{3.038062in}{2.891235in}}%
\pgfpathclose%
\pgfusepath{fill}%
\end{pgfscope}%
\begin{pgfscope}%
\pgfpathrectangle{\pgfqpoint{1.254980in}{0.150000in}}{\pgfqpoint{5.490039in}{5.490039in}}%
\pgfusepath{clip}%
\pgfsetbuttcap%
\pgfsetroundjoin%
\definecolor{currentfill}{rgb}{0.216210,0.351535,0.550627}%
\pgfsetfillcolor{currentfill}%
\pgfsetfillopacity{0.700000}%
\pgfsetlinewidth{0.000000pt}%
\definecolor{currentstroke}{rgb}{0.000000,0.000000,0.000000}%
\pgfsetstrokecolor{currentstroke}%
\pgfsetdash{}{0pt}%
\pgfpathmoveto{\pgfqpoint{5.389899in}{2.783514in}}%
\pgfpathlineto{\pgfqpoint{5.403415in}{2.785618in}}%
\pgfpathlineto{\pgfqpoint{5.416944in}{2.787836in}}%
\pgfpathlineto{\pgfqpoint{5.430484in}{2.790166in}}%
\pgfpathlineto{\pgfqpoint{5.444036in}{2.792608in}}%
\pgfpathlineto{\pgfqpoint{5.451130in}{2.801595in}}%
\pgfpathlineto{\pgfqpoint{5.458220in}{2.810583in}}%
\pgfpathlineto{\pgfqpoint{5.465305in}{2.819575in}}%
\pgfpathlineto{\pgfqpoint{5.472386in}{2.828571in}}%
\pgfpathlineto{\pgfqpoint{5.458847in}{2.826298in}}%
\pgfpathlineto{\pgfqpoint{5.445320in}{2.824137in}}%
\pgfpathlineto{\pgfqpoint{5.431805in}{2.822088in}}%
\pgfpathlineto{\pgfqpoint{5.418302in}{2.820152in}}%
\pgfpathlineto{\pgfqpoint{5.411207in}{2.810981in}}%
\pgfpathlineto{\pgfqpoint{5.404109in}{2.801819in}}%
\pgfpathlineto{\pgfqpoint{5.397006in}{2.792664in}}%
\pgfpathlineto{\pgfqpoint{5.389899in}{2.783514in}}%
\pgfpathclose%
\pgfusepath{fill}%
\end{pgfscope}%
\begin{pgfscope}%
\pgfpathrectangle{\pgfqpoint{1.254980in}{0.150000in}}{\pgfqpoint{5.490039in}{5.490039in}}%
\pgfusepath{clip}%
\pgfsetbuttcap%
\pgfsetroundjoin%
\definecolor{currentfill}{rgb}{0.241237,0.296485,0.539709}%
\pgfsetfillcolor{currentfill}%
\pgfsetfillopacity{0.700000}%
\pgfsetlinewidth{0.000000pt}%
\definecolor{currentstroke}{rgb}{0.000000,0.000000,0.000000}%
\pgfsetstrokecolor{currentstroke}%
\pgfsetdash{}{0pt}%
\pgfpathmoveto{\pgfqpoint{5.142537in}{2.652218in}}%
\pgfpathlineto{\pgfqpoint{5.155951in}{2.653365in}}%
\pgfpathlineto{\pgfqpoint{5.169376in}{2.654627in}}%
\pgfpathlineto{\pgfqpoint{5.182812in}{2.656004in}}%
\pgfpathlineto{\pgfqpoint{5.196259in}{2.657496in}}%
\pgfpathlineto{\pgfqpoint{5.203445in}{2.666937in}}%
\pgfpathlineto{\pgfqpoint{5.210627in}{2.676373in}}%
\pgfpathlineto{\pgfqpoint{5.217803in}{2.685804in}}%
\pgfpathlineto{\pgfqpoint{5.224975in}{2.695234in}}%
\pgfpathlineto{\pgfqpoint{5.211539in}{2.693863in}}%
\pgfpathlineto{\pgfqpoint{5.198114in}{2.692606in}}%
\pgfpathlineto{\pgfqpoint{5.184700in}{2.691464in}}%
\pgfpathlineto{\pgfqpoint{5.171297in}{2.690437in}}%
\pgfpathlineto{\pgfqpoint{5.164114in}{2.680882in}}%
\pgfpathlineto{\pgfqpoint{5.156926in}{2.671328in}}%
\pgfpathlineto{\pgfqpoint{5.149734in}{2.661773in}}%
\pgfpathlineto{\pgfqpoint{5.142537in}{2.652218in}}%
\pgfpathclose%
\pgfusepath{fill}%
\end{pgfscope}%
\begin{pgfscope}%
\pgfpathrectangle{\pgfqpoint{1.254980in}{0.150000in}}{\pgfqpoint{5.490039in}{5.490039in}}%
\pgfusepath{clip}%
\pgfsetbuttcap%
\pgfsetroundjoin%
\definecolor{currentfill}{rgb}{0.239346,0.300855,0.540844}%
\pgfsetfillcolor{currentfill}%
\pgfsetfillopacity{0.700000}%
\pgfsetlinewidth{0.000000pt}%
\definecolor{currentstroke}{rgb}{0.000000,0.000000,0.000000}%
\pgfsetstrokecolor{currentstroke}%
\pgfsetdash{}{0pt}%
\pgfpathmoveto{\pgfqpoint{3.195105in}{2.705571in}}%
\pgfpathlineto{\pgfqpoint{3.208171in}{2.691242in}}%
\pgfpathlineto{\pgfqpoint{3.221235in}{2.677084in}}%
\pgfpathlineto{\pgfqpoint{3.234296in}{2.663097in}}%
\pgfpathlineto{\pgfqpoint{3.247356in}{2.649278in}}%
\pgfpathlineto{\pgfqpoint{3.255215in}{2.656290in}}%
\pgfpathlineto{\pgfqpoint{3.263066in}{2.663394in}}%
\pgfpathlineto{\pgfqpoint{3.270910in}{2.670591in}}%
\pgfpathlineto{\pgfqpoint{3.278747in}{2.677878in}}%
\pgfpathlineto{\pgfqpoint{3.265708in}{2.691556in}}%
\pgfpathlineto{\pgfqpoint{3.252668in}{2.705402in}}%
\pgfpathlineto{\pgfqpoint{3.239625in}{2.719418in}}%
\pgfpathlineto{\pgfqpoint{3.226579in}{2.733604in}}%
\pgfpathlineto{\pgfqpoint{3.218722in}{2.726452in}}%
\pgfpathlineto{\pgfqpoint{3.210857in}{2.719396in}}%
\pgfpathlineto{\pgfqpoint{3.202985in}{2.712435in}}%
\pgfpathlineto{\pgfqpoint{3.195105in}{2.705571in}}%
\pgfpathclose%
\pgfusepath{fill}%
\end{pgfscope}%
\begin{pgfscope}%
\pgfpathrectangle{\pgfqpoint{1.254980in}{0.150000in}}{\pgfqpoint{5.490039in}{5.490039in}}%
\pgfusepath{clip}%
\pgfsetbuttcap%
\pgfsetroundjoin%
\definecolor{currentfill}{rgb}{0.206756,0.371758,0.553117}%
\pgfsetfillcolor{currentfill}%
\pgfsetfillopacity{0.700000}%
\pgfsetlinewidth{0.000000pt}%
\definecolor{currentstroke}{rgb}{0.000000,0.000000,0.000000}%
\pgfsetstrokecolor{currentstroke}%
\pgfsetdash{}{0pt}%
\pgfpathmoveto{\pgfqpoint{5.472386in}{2.828571in}}%
\pgfpathlineto{\pgfqpoint{5.485937in}{2.830957in}}%
\pgfpathlineto{\pgfqpoint{5.499501in}{2.833454in}}%
\pgfpathlineto{\pgfqpoint{5.513077in}{2.836064in}}%
\pgfpathlineto{\pgfqpoint{5.526666in}{2.838786in}}%
\pgfpathlineto{\pgfqpoint{5.533728in}{2.847610in}}%
\pgfpathlineto{\pgfqpoint{5.540787in}{2.856438in}}%
\pgfpathlineto{\pgfqpoint{5.547840in}{2.865273in}}%
\pgfpathlineto{\pgfqpoint{5.554889in}{2.874117in}}%
\pgfpathlineto{\pgfqpoint{5.541315in}{2.871581in}}%
\pgfpathlineto{\pgfqpoint{5.527753in}{2.869157in}}%
\pgfpathlineto{\pgfqpoint{5.514203in}{2.866844in}}%
\pgfpathlineto{\pgfqpoint{5.500665in}{2.864643in}}%
\pgfpathlineto{\pgfqpoint{5.493602in}{2.855608in}}%
\pgfpathlineto{\pgfqpoint{5.486534in}{2.846586in}}%
\pgfpathlineto{\pgfqpoint{5.479462in}{2.837574in}}%
\pgfpathlineto{\pgfqpoint{5.472386in}{2.828571in}}%
\pgfpathclose%
\pgfusepath{fill}%
\end{pgfscope}%
\begin{pgfscope}%
\pgfpathrectangle{\pgfqpoint{1.254980in}{0.150000in}}{\pgfqpoint{5.490039in}{5.490039in}}%
\pgfusepath{clip}%
\pgfsetbuttcap%
\pgfsetroundjoin%
\definecolor{currentfill}{rgb}{0.278012,0.180367,0.486697}%
\pgfsetfillcolor{currentfill}%
\pgfsetfillopacity{0.700000}%
\pgfsetlinewidth{0.000000pt}%
\definecolor{currentstroke}{rgb}{0.000000,0.000000,0.000000}%
\pgfsetstrokecolor{currentstroke}%
\pgfsetdash{}{0pt}%
\pgfpathmoveto{\pgfqpoint{3.539314in}{2.438426in}}%
\pgfpathlineto{\pgfqpoint{3.552342in}{2.428094in}}%
\pgfpathlineto{\pgfqpoint{3.565371in}{2.417912in}}%
\pgfpathlineto{\pgfqpoint{3.578401in}{2.407881in}}%
\pgfpathlineto{\pgfqpoint{3.591433in}{2.397999in}}%
\pgfpathlineto{\pgfqpoint{3.599154in}{2.406151in}}%
\pgfpathlineto{\pgfqpoint{3.606869in}{2.414368in}}%
\pgfpathlineto{\pgfqpoint{3.614578in}{2.422650in}}%
\pgfpathlineto{\pgfqpoint{3.622281in}{2.430995in}}%
\pgfpathlineto{\pgfqpoint{3.609266in}{2.440756in}}%
\pgfpathlineto{\pgfqpoint{3.596252in}{2.450666in}}%
\pgfpathlineto{\pgfqpoint{3.583240in}{2.460726in}}%
\pgfpathlineto{\pgfqpoint{3.570228in}{2.470936in}}%
\pgfpathlineto{\pgfqpoint{3.562509in}{2.462706in}}%
\pgfpathlineto{\pgfqpoint{3.554784in}{2.454544in}}%
\pgfpathlineto{\pgfqpoint{3.547052in}{2.446450in}}%
\pgfpathlineto{\pgfqpoint{3.539314in}{2.438426in}}%
\pgfpathclose%
\pgfusepath{fill}%
\end{pgfscope}%
\begin{pgfscope}%
\pgfpathrectangle{\pgfqpoint{1.254980in}{0.150000in}}{\pgfqpoint{5.490039in}{5.490039in}}%
\pgfusepath{clip}%
\pgfsetbuttcap%
\pgfsetroundjoin%
\definecolor{currentfill}{rgb}{0.250425,0.274290,0.533103}%
\pgfsetfillcolor{currentfill}%
\pgfsetfillopacity{0.700000}%
\pgfsetlinewidth{0.000000pt}%
\definecolor{currentstroke}{rgb}{0.000000,0.000000,0.000000}%
\pgfsetstrokecolor{currentstroke}%
\pgfsetdash{}{0pt}%
\pgfpathmoveto{\pgfqpoint{5.060112in}{2.610105in}}%
\pgfpathlineto{\pgfqpoint{5.073493in}{2.610895in}}%
\pgfpathlineto{\pgfqpoint{5.086886in}{2.611800in}}%
\pgfpathlineto{\pgfqpoint{5.100289in}{2.612821in}}%
\pgfpathlineto{\pgfqpoint{5.113702in}{2.613958in}}%
\pgfpathlineto{\pgfqpoint{5.120918in}{2.623531in}}%
\pgfpathlineto{\pgfqpoint{5.128129in}{2.633098in}}%
\pgfpathlineto{\pgfqpoint{5.135335in}{2.642660in}}%
\pgfpathlineto{\pgfqpoint{5.142537in}{2.652218in}}%
\pgfpathlineto{\pgfqpoint{5.129133in}{2.651186in}}%
\pgfpathlineto{\pgfqpoint{5.115741in}{2.650269in}}%
\pgfpathlineto{\pgfqpoint{5.102359in}{2.649468in}}%
\pgfpathlineto{\pgfqpoint{5.088988in}{2.648782in}}%
\pgfpathlineto{\pgfqpoint{5.081776in}{2.639114in}}%
\pgfpathlineto{\pgfqpoint{5.074559in}{2.629446in}}%
\pgfpathlineto{\pgfqpoint{5.067338in}{2.619777in}}%
\pgfpathlineto{\pgfqpoint{5.060112in}{2.610105in}}%
\pgfpathclose%
\pgfusepath{fill}%
\end{pgfscope}%
\begin{pgfscope}%
\pgfpathrectangle{\pgfqpoint{1.254980in}{0.150000in}}{\pgfqpoint{5.490039in}{5.490039in}}%
\pgfusepath{clip}%
\pgfsetbuttcap%
\pgfsetroundjoin%
\definecolor{currentfill}{rgb}{0.282623,0.140926,0.457517}%
\pgfsetfillcolor{currentfill}%
\pgfsetfillopacity{0.700000}%
\pgfsetlinewidth{0.000000pt}%
\definecolor{currentstroke}{rgb}{0.000000,0.000000,0.000000}%
\pgfsetstrokecolor{currentstroke}%
\pgfsetdash{}{0pt}%
\pgfpathmoveto{\pgfqpoint{3.726464in}{2.358195in}}%
\pgfpathlineto{\pgfqpoint{3.739496in}{2.349746in}}%
\pgfpathlineto{\pgfqpoint{3.752531in}{2.341439in}}%
\pgfpathlineto{\pgfqpoint{3.765568in}{2.333275in}}%
\pgfpathlineto{\pgfqpoint{3.778608in}{2.325252in}}%
\pgfpathlineto{\pgfqpoint{3.786260in}{2.333992in}}%
\pgfpathlineto{\pgfqpoint{3.793907in}{2.342783in}}%
\pgfpathlineto{\pgfqpoint{3.801549in}{2.351623in}}%
\pgfpathlineto{\pgfqpoint{3.809184in}{2.360513in}}%
\pgfpathlineto{\pgfqpoint{3.796158in}{2.368432in}}%
\pgfpathlineto{\pgfqpoint{3.783135in}{2.376492in}}%
\pgfpathlineto{\pgfqpoint{3.770115in}{2.384694in}}%
\pgfpathlineto{\pgfqpoint{3.757097in}{2.393038in}}%
\pgfpathlineto{\pgfqpoint{3.749447in}{2.384247in}}%
\pgfpathlineto{\pgfqpoint{3.741791in}{2.375509in}}%
\pgfpathlineto{\pgfqpoint{3.734130in}{2.366825in}}%
\pgfpathlineto{\pgfqpoint{3.726464in}{2.358195in}}%
\pgfpathclose%
\pgfusepath{fill}%
\end{pgfscope}%
\begin{pgfscope}%
\pgfpathrectangle{\pgfqpoint{1.254980in}{0.150000in}}{\pgfqpoint{5.490039in}{5.490039in}}%
\pgfusepath{clip}%
\pgfsetbuttcap%
\pgfsetroundjoin%
\definecolor{currentfill}{rgb}{0.197636,0.391528,0.554969}%
\pgfsetfillcolor{currentfill}%
\pgfsetfillopacity{0.700000}%
\pgfsetlinewidth{0.000000pt}%
\definecolor{currentstroke}{rgb}{0.000000,0.000000,0.000000}%
\pgfsetstrokecolor{currentstroke}%
\pgfsetdash{}{0pt}%
\pgfpathmoveto{\pgfqpoint{5.554889in}{2.874117in}}%
\pgfpathlineto{\pgfqpoint{5.568477in}{2.876765in}}%
\pgfpathlineto{\pgfqpoint{5.582076in}{2.879524in}}%
\pgfpathlineto{\pgfqpoint{5.595689in}{2.882395in}}%
\pgfpathlineto{\pgfqpoint{5.609314in}{2.885377in}}%
\pgfpathlineto{\pgfqpoint{5.616344in}{2.894036in}}%
\pgfpathlineto{\pgfqpoint{5.623370in}{2.902703in}}%
\pgfpathlineto{\pgfqpoint{5.630392in}{2.911381in}}%
\pgfpathlineto{\pgfqpoint{5.637410in}{2.920072in}}%
\pgfpathlineto{\pgfqpoint{5.623800in}{2.917292in}}%
\pgfpathlineto{\pgfqpoint{5.610202in}{2.914623in}}%
\pgfpathlineto{\pgfqpoint{5.596617in}{2.912065in}}%
\pgfpathlineto{\pgfqpoint{5.583045in}{2.909618in}}%
\pgfpathlineto{\pgfqpoint{5.576012in}{2.900720in}}%
\pgfpathlineto{\pgfqpoint{5.568975in}{2.891839in}}%
\pgfpathlineto{\pgfqpoint{5.561935in}{2.882972in}}%
\pgfpathlineto{\pgfqpoint{5.554889in}{2.874117in}}%
\pgfpathclose%
\pgfusepath{fill}%
\end{pgfscope}%
\begin{pgfscope}%
\pgfpathrectangle{\pgfqpoint{1.254980in}{0.150000in}}{\pgfqpoint{5.490039in}{5.490039in}}%
\pgfusepath{clip}%
\pgfsetbuttcap%
\pgfsetroundjoin%
\definecolor{currentfill}{rgb}{0.257322,0.256130,0.526563}%
\pgfsetfillcolor{currentfill}%
\pgfsetfillopacity{0.700000}%
\pgfsetlinewidth{0.000000pt}%
\definecolor{currentstroke}{rgb}{0.000000,0.000000,0.000000}%
\pgfsetstrokecolor{currentstroke}%
\pgfsetdash{}{0pt}%
\pgfpathmoveto{\pgfqpoint{4.977698in}{2.569023in}}%
\pgfpathlineto{\pgfqpoint{4.991049in}{2.569436in}}%
\pgfpathlineto{\pgfqpoint{5.004409in}{2.569965in}}%
\pgfpathlineto{\pgfqpoint{5.017780in}{2.570611in}}%
\pgfpathlineto{\pgfqpoint{5.031161in}{2.571372in}}%
\pgfpathlineto{\pgfqpoint{5.038406in}{2.581065in}}%
\pgfpathlineto{\pgfqpoint{5.045646in}{2.590750in}}%
\pgfpathlineto{\pgfqpoint{5.052881in}{2.600430in}}%
\pgfpathlineto{\pgfqpoint{5.060112in}{2.610105in}}%
\pgfpathlineto{\pgfqpoint{5.046740in}{2.609432in}}%
\pgfpathlineto{\pgfqpoint{5.033380in}{2.608874in}}%
\pgfpathlineto{\pgfqpoint{5.020029in}{2.608433in}}%
\pgfpathlineto{\pgfqpoint{5.006689in}{2.608108in}}%
\pgfpathlineto{\pgfqpoint{4.999448in}{2.598339in}}%
\pgfpathlineto{\pgfqpoint{4.992203in}{2.588569in}}%
\pgfpathlineto{\pgfqpoint{4.984953in}{2.578798in}}%
\pgfpathlineto{\pgfqpoint{4.977698in}{2.569023in}}%
\pgfpathclose%
\pgfusepath{fill}%
\end{pgfscope}%
\begin{pgfscope}%
\pgfpathrectangle{\pgfqpoint{1.254980in}{0.150000in}}{\pgfqpoint{5.490039in}{5.490039in}}%
\pgfusepath{clip}%
\pgfsetbuttcap%
\pgfsetroundjoin%
\definecolor{currentfill}{rgb}{0.188923,0.410910,0.556326}%
\pgfsetfillcolor{currentfill}%
\pgfsetfillopacity{0.700000}%
\pgfsetlinewidth{0.000000pt}%
\definecolor{currentstroke}{rgb}{0.000000,0.000000,0.000000}%
\pgfsetstrokecolor{currentstroke}%
\pgfsetdash{}{0pt}%
\pgfpathmoveto{\pgfqpoint{5.637410in}{2.920072in}}%
\pgfpathlineto{\pgfqpoint{5.651033in}{2.922963in}}%
\pgfpathlineto{\pgfqpoint{5.664669in}{2.925965in}}%
\pgfpathlineto{\pgfqpoint{5.678318in}{2.929078in}}%
\pgfpathlineto{\pgfqpoint{5.691980in}{2.932302in}}%
\pgfpathlineto{\pgfqpoint{5.698978in}{2.940796in}}%
\pgfpathlineto{\pgfqpoint{5.705972in}{2.949303in}}%
\pgfpathlineto{\pgfqpoint{5.712961in}{2.957825in}}%
\pgfpathlineto{\pgfqpoint{5.719946in}{2.966365in}}%
\pgfpathlineto{\pgfqpoint{5.706300in}{2.963360in}}%
\pgfpathlineto{\pgfqpoint{5.692667in}{2.960465in}}%
\pgfpathlineto{\pgfqpoint{5.679047in}{2.957680in}}%
\pgfpathlineto{\pgfqpoint{5.665439in}{2.955007in}}%
\pgfpathlineto{\pgfqpoint{5.658438in}{2.946243in}}%
\pgfpathlineto{\pgfqpoint{5.651433in}{2.937501in}}%
\pgfpathlineto{\pgfqpoint{5.644423in}{2.928778in}}%
\pgfpathlineto{\pgfqpoint{5.637410in}{2.920072in}}%
\pgfpathclose%
\pgfusepath{fill}%
\end{pgfscope}%
\begin{pgfscope}%
\pgfpathrectangle{\pgfqpoint{1.254980in}{0.150000in}}{\pgfqpoint{5.490039in}{5.490039in}}%
\pgfusepath{clip}%
\pgfsetbuttcap%
\pgfsetroundjoin%
\definecolor{currentfill}{rgb}{0.190631,0.407061,0.556089}%
\pgfsetfillcolor{currentfill}%
\pgfsetfillopacity{0.700000}%
\pgfsetlinewidth{0.000000pt}%
\definecolor{currentstroke}{rgb}{0.000000,0.000000,0.000000}%
\pgfsetstrokecolor{currentstroke}%
\pgfsetdash{}{0pt}%
\pgfpathmoveto{\pgfqpoint{2.985585in}{2.958948in}}%
\pgfpathlineto{\pgfqpoint{2.998712in}{2.941738in}}%
\pgfpathlineto{\pgfqpoint{3.011833in}{2.924717in}}%
\pgfpathlineto{\pgfqpoint{3.024950in}{2.907883in}}%
\pgfpathlineto{\pgfqpoint{3.038062in}{2.891235in}}%
\pgfpathlineto{\pgfqpoint{3.046010in}{2.897649in}}%
\pgfpathlineto{\pgfqpoint{3.053949in}{2.904172in}}%
\pgfpathlineto{\pgfqpoint{3.061880in}{2.910804in}}%
\pgfpathlineto{\pgfqpoint{3.069803in}{2.917543in}}%
\pgfpathlineto{\pgfqpoint{3.056714in}{2.934046in}}%
\pgfpathlineto{\pgfqpoint{3.043621in}{2.950734in}}%
\pgfpathlineto{\pgfqpoint{3.030523in}{2.967610in}}%
\pgfpathlineto{\pgfqpoint{3.017420in}{2.984674in}}%
\pgfpathlineto{\pgfqpoint{3.009474in}{2.978074in}}%
\pgfpathlineto{\pgfqpoint{3.001520in}{2.971586in}}%
\pgfpathlineto{\pgfqpoint{2.993557in}{2.965210in}}%
\pgfpathlineto{\pgfqpoint{2.985585in}{2.958948in}}%
\pgfpathclose%
\pgfusepath{fill}%
\end{pgfscope}%
\begin{pgfscope}%
\pgfpathrectangle{\pgfqpoint{1.254980in}{0.150000in}}{\pgfqpoint{5.490039in}{5.490039in}}%
\pgfusepath{clip}%
\pgfsetbuttcap%
\pgfsetroundjoin%
\definecolor{currentfill}{rgb}{0.248629,0.278775,0.534556}%
\pgfsetfillcolor{currentfill}%
\pgfsetfillopacity{0.700000}%
\pgfsetlinewidth{0.000000pt}%
\definecolor{currentstroke}{rgb}{0.000000,0.000000,0.000000}%
\pgfsetstrokecolor{currentstroke}%
\pgfsetdash{}{0pt}%
\pgfpathmoveto{\pgfqpoint{3.247356in}{2.649278in}}%
\pgfpathlineto{\pgfqpoint{3.260413in}{2.635628in}}%
\pgfpathlineto{\pgfqpoint{3.273468in}{2.622144in}}%
\pgfpathlineto{\pgfqpoint{3.286522in}{2.608827in}}%
\pgfpathlineto{\pgfqpoint{3.299574in}{2.595676in}}%
\pgfpathlineto{\pgfqpoint{3.307413in}{2.602834in}}%
\pgfpathlineto{\pgfqpoint{3.315244in}{2.610082in}}%
\pgfpathlineto{\pgfqpoint{3.323068in}{2.617418in}}%
\pgfpathlineto{\pgfqpoint{3.330885in}{2.624840in}}%
\pgfpathlineto{\pgfqpoint{3.317853in}{2.637851in}}%
\pgfpathlineto{\pgfqpoint{3.304819in}{2.651027in}}%
\pgfpathlineto{\pgfqpoint{3.291784in}{2.664369in}}%
\pgfpathlineto{\pgfqpoint{3.278747in}{2.677878in}}%
\pgfpathlineto{\pgfqpoint{3.270910in}{2.670591in}}%
\pgfpathlineto{\pgfqpoint{3.263066in}{2.663394in}}%
\pgfpathlineto{\pgfqpoint{3.255215in}{2.656290in}}%
\pgfpathlineto{\pgfqpoint{3.247356in}{2.649278in}}%
\pgfpathclose%
\pgfusepath{fill}%
\end{pgfscope}%
\begin{pgfscope}%
\pgfpathrectangle{\pgfqpoint{1.254980in}{0.150000in}}{\pgfqpoint{5.490039in}{5.490039in}}%
\pgfusepath{clip}%
\pgfsetbuttcap%
\pgfsetroundjoin%
\definecolor{currentfill}{rgb}{0.180629,0.429975,0.557282}%
\pgfsetfillcolor{currentfill}%
\pgfsetfillopacity{0.700000}%
\pgfsetlinewidth{0.000000pt}%
\definecolor{currentstroke}{rgb}{0.000000,0.000000,0.000000}%
\pgfsetstrokecolor{currentstroke}%
\pgfsetdash{}{0pt}%
\pgfpathmoveto{\pgfqpoint{5.719946in}{2.966365in}}%
\pgfpathlineto{\pgfqpoint{5.733606in}{2.969481in}}%
\pgfpathlineto{\pgfqpoint{5.747278in}{2.972707in}}%
\pgfpathlineto{\pgfqpoint{5.760964in}{2.976044in}}%
\pgfpathlineto{\pgfqpoint{5.774663in}{2.979490in}}%
\pgfpathlineto{\pgfqpoint{5.781628in}{2.987822in}}%
\pgfpathlineto{\pgfqpoint{5.788589in}{2.996171in}}%
\pgfpathlineto{\pgfqpoint{5.795546in}{3.004541in}}%
\pgfpathlineto{\pgfqpoint{5.802499in}{3.012935in}}%
\pgfpathlineto{\pgfqpoint{5.788817in}{3.009723in}}%
\pgfpathlineto{\pgfqpoint{5.775148in}{3.006621in}}%
\pgfpathlineto{\pgfqpoint{5.761492in}{3.003629in}}%
\pgfpathlineto{\pgfqpoint{5.747850in}{3.000747in}}%
\pgfpathlineto{\pgfqpoint{5.740879in}{2.992113in}}%
\pgfpathlineto{\pgfqpoint{5.733906in}{2.983507in}}%
\pgfpathlineto{\pgfqpoint{5.726928in}{2.974925in}}%
\pgfpathlineto{\pgfqpoint{5.719946in}{2.966365in}}%
\pgfpathclose%
\pgfusepath{fill}%
\end{pgfscope}%
\begin{pgfscope}%
\pgfpathrectangle{\pgfqpoint{1.254980in}{0.150000in}}{\pgfqpoint{5.490039in}{5.490039in}}%
\pgfusepath{clip}%
\pgfsetbuttcap%
\pgfsetroundjoin%
\definecolor{currentfill}{rgb}{0.263663,0.237631,0.518762}%
\pgfsetfillcolor{currentfill}%
\pgfsetfillopacity{0.700000}%
\pgfsetlinewidth{0.000000pt}%
\definecolor{currentstroke}{rgb}{0.000000,0.000000,0.000000}%
\pgfsetstrokecolor{currentstroke}%
\pgfsetdash{}{0pt}%
\pgfpathmoveto{\pgfqpoint{4.895294in}{2.529107in}}%
\pgfpathlineto{\pgfqpoint{4.908614in}{2.529123in}}%
\pgfpathlineto{\pgfqpoint{4.921944in}{2.529257in}}%
\pgfpathlineto{\pgfqpoint{4.935283in}{2.529508in}}%
\pgfpathlineto{\pgfqpoint{4.948633in}{2.529876in}}%
\pgfpathlineto{\pgfqpoint{4.955907in}{2.539672in}}%
\pgfpathlineto{\pgfqpoint{4.963175in}{2.549461in}}%
\pgfpathlineto{\pgfqpoint{4.970439in}{2.559245in}}%
\pgfpathlineto{\pgfqpoint{4.977698in}{2.569023in}}%
\pgfpathlineto{\pgfqpoint{4.964358in}{2.568727in}}%
\pgfpathlineto{\pgfqpoint{4.951028in}{2.568548in}}%
\pgfpathlineto{\pgfqpoint{4.937708in}{2.568487in}}%
\pgfpathlineto{\pgfqpoint{4.924398in}{2.568542in}}%
\pgfpathlineto{\pgfqpoint{4.917128in}{2.558686in}}%
\pgfpathlineto{\pgfqpoint{4.909855in}{2.548829in}}%
\pgfpathlineto{\pgfqpoint{4.902576in}{2.538969in}}%
\pgfpathlineto{\pgfqpoint{4.895294in}{2.529107in}}%
\pgfpathclose%
\pgfusepath{fill}%
\end{pgfscope}%
\begin{pgfscope}%
\pgfpathrectangle{\pgfqpoint{1.254980in}{0.150000in}}{\pgfqpoint{5.490039in}{5.490039in}}%
\pgfusepath{clip}%
\pgfsetbuttcap%
\pgfsetroundjoin%
\definecolor{currentfill}{rgb}{0.283229,0.120777,0.440584}%
\pgfsetfillcolor{currentfill}%
\pgfsetfillopacity{0.700000}%
\pgfsetlinewidth{0.000000pt}%
\definecolor{currentstroke}{rgb}{0.000000,0.000000,0.000000}%
\pgfsetstrokecolor{currentstroke}%
\pgfsetdash{}{0pt}%
\pgfpathmoveto{\pgfqpoint{4.130880in}{2.306994in}}%
\pgfpathlineto{\pgfqpoint{4.143970in}{2.302079in}}%
\pgfpathlineto{\pgfqpoint{4.157065in}{2.297294in}}%
\pgfpathlineto{\pgfqpoint{4.170165in}{2.292639in}}%
\pgfpathlineto{\pgfqpoint{4.183271in}{2.288113in}}%
\pgfpathlineto{\pgfqpoint{4.190789in}{2.297796in}}%
\pgfpathlineto{\pgfqpoint{4.198301in}{2.307500in}}%
\pgfpathlineto{\pgfqpoint{4.205809in}{2.317226in}}%
\pgfpathlineto{\pgfqpoint{4.213312in}{2.326973in}}%
\pgfpathlineto{\pgfqpoint{4.200217in}{2.331443in}}%
\pgfpathlineto{\pgfqpoint{4.187127in}{2.336042in}}%
\pgfpathlineto{\pgfqpoint{4.174043in}{2.340772in}}%
\pgfpathlineto{\pgfqpoint{4.160964in}{2.345631in}}%
\pgfpathlineto{\pgfqpoint{4.153450in}{2.335934in}}%
\pgfpathlineto{\pgfqpoint{4.145932in}{2.326262in}}%
\pgfpathlineto{\pgfqpoint{4.138409in}{2.316615in}}%
\pgfpathlineto{\pgfqpoint{4.130880in}{2.306994in}}%
\pgfpathclose%
\pgfusepath{fill}%
\end{pgfscope}%
\begin{pgfscope}%
\pgfpathrectangle{\pgfqpoint{1.254980in}{0.150000in}}{\pgfqpoint{5.490039in}{5.490039in}}%
\pgfusepath{clip}%
\pgfsetbuttcap%
\pgfsetroundjoin%
\definecolor{currentfill}{rgb}{0.282884,0.135920,0.453427}%
\pgfsetfillcolor{currentfill}%
\pgfsetfillopacity{0.700000}%
\pgfsetlinewidth{0.000000pt}%
\definecolor{currentstroke}{rgb}{0.000000,0.000000,0.000000}%
\pgfsetstrokecolor{currentstroke}%
\pgfsetdash{}{0pt}%
\pgfpathmoveto{\pgfqpoint{4.348180in}{2.335333in}}%
\pgfpathlineto{\pgfqpoint{4.361322in}{2.332049in}}%
\pgfpathlineto{\pgfqpoint{4.374471in}{2.328890in}}%
\pgfpathlineto{\pgfqpoint{4.387627in}{2.325856in}}%
\pgfpathlineto{\pgfqpoint{4.400791in}{2.322947in}}%
\pgfpathlineto{\pgfqpoint{4.408240in}{2.332888in}}%
\pgfpathlineto{\pgfqpoint{4.415685in}{2.342838in}}%
\pgfpathlineto{\pgfqpoint{4.423125in}{2.352797in}}%
\pgfpathlineto{\pgfqpoint{4.430560in}{2.362765in}}%
\pgfpathlineto{\pgfqpoint{4.417407in}{2.365651in}}%
\pgfpathlineto{\pgfqpoint{4.404260in}{2.368661in}}%
\pgfpathlineto{\pgfqpoint{4.391121in}{2.371796in}}%
\pgfpathlineto{\pgfqpoint{4.377988in}{2.375057in}}%
\pgfpathlineto{\pgfqpoint{4.370543in}{2.365106in}}%
\pgfpathlineto{\pgfqpoint{4.363093in}{2.355168in}}%
\pgfpathlineto{\pgfqpoint{4.355639in}{2.345244in}}%
\pgfpathlineto{\pgfqpoint{4.348180in}{2.335333in}}%
\pgfpathclose%
\pgfusepath{fill}%
\end{pgfscope}%
\begin{pgfscope}%
\pgfpathrectangle{\pgfqpoint{1.254980in}{0.150000in}}{\pgfqpoint{5.490039in}{5.490039in}}%
\pgfusepath{clip}%
\pgfsetbuttcap%
\pgfsetroundjoin%
\definecolor{currentfill}{rgb}{0.269308,0.218818,0.509577}%
\pgfsetfillcolor{currentfill}%
\pgfsetfillopacity{0.700000}%
\pgfsetlinewidth{0.000000pt}%
\definecolor{currentstroke}{rgb}{0.000000,0.000000,0.000000}%
\pgfsetstrokecolor{currentstroke}%
\pgfsetdash{}{0pt}%
\pgfpathmoveto{\pgfqpoint{4.812894in}{2.490504in}}%
\pgfpathlineto{\pgfqpoint{4.826185in}{2.490104in}}%
\pgfpathlineto{\pgfqpoint{4.839486in}{2.489822in}}%
\pgfpathlineto{\pgfqpoint{4.852796in}{2.489659in}}%
\pgfpathlineto{\pgfqpoint{4.866116in}{2.489613in}}%
\pgfpathlineto{\pgfqpoint{4.873417in}{2.499495in}}%
\pgfpathlineto{\pgfqpoint{4.880714in}{2.509371in}}%
\pgfpathlineto{\pgfqpoint{4.888006in}{2.519241in}}%
\pgfpathlineto{\pgfqpoint{4.895294in}{2.529107in}}%
\pgfpathlineto{\pgfqpoint{4.881983in}{2.529209in}}%
\pgfpathlineto{\pgfqpoint{4.868683in}{2.529428in}}%
\pgfpathlineto{\pgfqpoint{4.855392in}{2.529766in}}%
\pgfpathlineto{\pgfqpoint{4.842110in}{2.530222in}}%
\pgfpathlineto{\pgfqpoint{4.834813in}{2.520294in}}%
\pgfpathlineto{\pgfqpoint{4.827511in}{2.510366in}}%
\pgfpathlineto{\pgfqpoint{4.820205in}{2.500436in}}%
\pgfpathlineto{\pgfqpoint{4.812894in}{2.490504in}}%
\pgfpathclose%
\pgfusepath{fill}%
\end{pgfscope}%
\begin{pgfscope}%
\pgfpathrectangle{\pgfqpoint{1.254980in}{0.150000in}}{\pgfqpoint{5.490039in}{5.490039in}}%
\pgfusepath{clip}%
\pgfsetbuttcap%
\pgfsetroundjoin%
\definecolor{currentfill}{rgb}{0.177423,0.437527,0.557565}%
\pgfsetfillcolor{currentfill}%
\pgfsetfillopacity{0.700000}%
\pgfsetlinewidth{0.000000pt}%
\definecolor{currentstroke}{rgb}{0.000000,0.000000,0.000000}%
\pgfsetstrokecolor{currentstroke}%
\pgfsetdash{}{0pt}%
\pgfpathmoveto{\pgfqpoint{2.933027in}{3.029699in}}%
\pgfpathlineto{\pgfqpoint{2.946175in}{3.011722in}}%
\pgfpathlineto{\pgfqpoint{2.959317in}{2.993939in}}%
\pgfpathlineto{\pgfqpoint{2.972454in}{2.976348in}}%
\pgfpathlineto{\pgfqpoint{2.985585in}{2.958948in}}%
\pgfpathlineto{\pgfqpoint{2.993557in}{2.965210in}}%
\pgfpathlineto{\pgfqpoint{3.001520in}{2.971586in}}%
\pgfpathlineto{\pgfqpoint{3.009474in}{2.978074in}}%
\pgfpathlineto{\pgfqpoint{3.017420in}{2.984674in}}%
\pgfpathlineto{\pgfqpoint{3.004313in}{3.001927in}}%
\pgfpathlineto{\pgfqpoint{2.991200in}{3.019371in}}%
\pgfpathlineto{\pgfqpoint{2.978082in}{3.037007in}}%
\pgfpathlineto{\pgfqpoint{2.964959in}{3.054837in}}%
\pgfpathlineto{\pgfqpoint{2.956989in}{3.048377in}}%
\pgfpathlineto{\pgfqpoint{2.949011in}{3.042034in}}%
\pgfpathlineto{\pgfqpoint{2.941023in}{3.035808in}}%
\pgfpathlineto{\pgfqpoint{2.933027in}{3.029699in}}%
\pgfpathclose%
\pgfusepath{fill}%
\end{pgfscope}%
\begin{pgfscope}%
\pgfpathrectangle{\pgfqpoint{1.254980in}{0.150000in}}{\pgfqpoint{5.490039in}{5.490039in}}%
\pgfusepath{clip}%
\pgfsetbuttcap%
\pgfsetroundjoin%
\definecolor{currentfill}{rgb}{0.257322,0.256130,0.526563}%
\pgfsetfillcolor{currentfill}%
\pgfsetfillopacity{0.700000}%
\pgfsetlinewidth{0.000000pt}%
\definecolor{currentstroke}{rgb}{0.000000,0.000000,0.000000}%
\pgfsetstrokecolor{currentstroke}%
\pgfsetdash{}{0pt}%
\pgfpathmoveto{\pgfqpoint{3.299574in}{2.595676in}}%
\pgfpathlineto{\pgfqpoint{3.312625in}{2.582688in}}%
\pgfpathlineto{\pgfqpoint{3.325674in}{2.569865in}}%
\pgfpathlineto{\pgfqpoint{3.338722in}{2.557204in}}%
\pgfpathlineto{\pgfqpoint{3.351769in}{2.544705in}}%
\pgfpathlineto{\pgfqpoint{3.359588in}{2.552010in}}%
\pgfpathlineto{\pgfqpoint{3.367400in}{2.559400in}}%
\pgfpathlineto{\pgfqpoint{3.375204in}{2.566873in}}%
\pgfpathlineto{\pgfqpoint{3.383002in}{2.574430in}}%
\pgfpathlineto{\pgfqpoint{3.369975in}{2.586789in}}%
\pgfpathlineto{\pgfqpoint{3.356946in}{2.599310in}}%
\pgfpathlineto{\pgfqpoint{3.343916in}{2.611993in}}%
\pgfpathlineto{\pgfqpoint{3.330885in}{2.624840in}}%
\pgfpathlineto{\pgfqpoint{3.323068in}{2.617418in}}%
\pgfpathlineto{\pgfqpoint{3.315244in}{2.610082in}}%
\pgfpathlineto{\pgfqpoint{3.307413in}{2.602834in}}%
\pgfpathlineto{\pgfqpoint{3.299574in}{2.595676in}}%
\pgfpathclose%
\pgfusepath{fill}%
\end{pgfscope}%
\begin{pgfscope}%
\pgfpathrectangle{\pgfqpoint{1.254980in}{0.150000in}}{\pgfqpoint{5.490039in}{5.490039in}}%
\pgfusepath{clip}%
\pgfsetbuttcap%
\pgfsetroundjoin%
\definecolor{currentfill}{rgb}{0.174274,0.445044,0.557792}%
\pgfsetfillcolor{currentfill}%
\pgfsetfillopacity{0.700000}%
\pgfsetlinewidth{0.000000pt}%
\definecolor{currentstroke}{rgb}{0.000000,0.000000,0.000000}%
\pgfsetstrokecolor{currentstroke}%
\pgfsetdash{}{0pt}%
\pgfpathmoveto{\pgfqpoint{5.802499in}{3.012935in}}%
\pgfpathlineto{\pgfqpoint{5.816195in}{3.016257in}}%
\pgfpathlineto{\pgfqpoint{5.829904in}{3.019688in}}%
\pgfpathlineto{\pgfqpoint{5.843626in}{3.023230in}}%
\pgfpathlineto{\pgfqpoint{5.857363in}{3.026881in}}%
\pgfpathlineto{\pgfqpoint{5.864294in}{3.035054in}}%
\pgfpathlineto{\pgfqpoint{5.871222in}{3.043252in}}%
\pgfpathlineto{\pgfqpoint{5.878146in}{3.051476in}}%
\pgfpathlineto{\pgfqpoint{5.864424in}{3.048012in}}%
\pgfpathlineto{\pgfqpoint{5.850714in}{3.044657in}}%
\pgfpathlineto{\pgfqpoint{5.837019in}{3.041412in}}%
\pgfpathlineto{\pgfqpoint{5.823336in}{3.038277in}}%
\pgfpathlineto{\pgfqpoint{5.816394in}{3.029800in}}%
\pgfpathlineto{\pgfqpoint{5.809448in}{3.021353in}}%
\pgfpathlineto{\pgfqpoint{5.802499in}{3.012935in}}%
\pgfpathclose%
\pgfusepath{fill}%
\end{pgfscope}%
\begin{pgfscope}%
\pgfpathrectangle{\pgfqpoint{1.254980in}{0.150000in}}{\pgfqpoint{5.490039in}{5.490039in}}%
\pgfusepath{clip}%
\pgfsetbuttcap%
\pgfsetroundjoin%
\definecolor{currentfill}{rgb}{0.274128,0.199721,0.498911}%
\pgfsetfillcolor{currentfill}%
\pgfsetfillopacity{0.700000}%
\pgfsetlinewidth{0.000000pt}%
\definecolor{currentstroke}{rgb}{0.000000,0.000000,0.000000}%
\pgfsetstrokecolor{currentstroke}%
\pgfsetdash{}{0pt}%
\pgfpathmoveto{\pgfqpoint{4.730496in}{2.453368in}}%
\pgfpathlineto{\pgfqpoint{4.743760in}{2.452532in}}%
\pgfpathlineto{\pgfqpoint{4.757032in}{2.451815in}}%
\pgfpathlineto{\pgfqpoint{4.770314in}{2.451218in}}%
\pgfpathlineto{\pgfqpoint{4.783605in}{2.450739in}}%
\pgfpathlineto{\pgfqpoint{4.790934in}{2.460687in}}%
\pgfpathlineto{\pgfqpoint{4.798259in}{2.470630in}}%
\pgfpathlineto{\pgfqpoint{4.805579in}{2.480569in}}%
\pgfpathlineto{\pgfqpoint{4.812894in}{2.490504in}}%
\pgfpathlineto{\pgfqpoint{4.799613in}{2.491023in}}%
\pgfpathlineto{\pgfqpoint{4.786340in}{2.491660in}}%
\pgfpathlineto{\pgfqpoint{4.773077in}{2.492417in}}%
\pgfpathlineto{\pgfqpoint{4.759823in}{2.493293in}}%
\pgfpathlineto{\pgfqpoint{4.752498in}{2.483312in}}%
\pgfpathlineto{\pgfqpoint{4.745168in}{2.473332in}}%
\pgfpathlineto{\pgfqpoint{4.737834in}{2.463350in}}%
\pgfpathlineto{\pgfqpoint{4.730496in}{2.453368in}}%
\pgfpathclose%
\pgfusepath{fill}%
\end{pgfscope}%
\begin{pgfscope}%
\pgfpathrectangle{\pgfqpoint{1.254980in}{0.150000in}}{\pgfqpoint{5.490039in}{5.490039in}}%
\pgfusepath{clip}%
\pgfsetbuttcap%
\pgfsetroundjoin%
\definecolor{currentfill}{rgb}{0.280255,0.165693,0.476498}%
\pgfsetfillcolor{currentfill}%
\pgfsetfillopacity{0.700000}%
\pgfsetlinewidth{0.000000pt}%
\definecolor{currentstroke}{rgb}{0.000000,0.000000,0.000000}%
\pgfsetstrokecolor{currentstroke}%
\pgfsetdash{}{0pt}%
\pgfpathmoveto{\pgfqpoint{3.591433in}{2.397999in}}%
\pgfpathlineto{\pgfqpoint{3.604465in}{2.388265in}}%
\pgfpathlineto{\pgfqpoint{3.617499in}{2.378679in}}%
\pgfpathlineto{\pgfqpoint{3.630535in}{2.369240in}}%
\pgfpathlineto{\pgfqpoint{3.643572in}{2.359948in}}%
\pgfpathlineto{\pgfqpoint{3.651277in}{2.368227in}}%
\pgfpathlineto{\pgfqpoint{3.658976in}{2.376568in}}%
\pgfpathlineto{\pgfqpoint{3.666670in}{2.384969in}}%
\pgfpathlineto{\pgfqpoint{3.674357in}{2.393429in}}%
\pgfpathlineto{\pgfqpoint{3.661336in}{2.402600in}}%
\pgfpathlineto{\pgfqpoint{3.648316in}{2.411918in}}%
\pgfpathlineto{\pgfqpoint{3.635298in}{2.421383in}}%
\pgfpathlineto{\pgfqpoint{3.622281in}{2.430995in}}%
\pgfpathlineto{\pgfqpoint{3.614578in}{2.422650in}}%
\pgfpathlineto{\pgfqpoint{3.606869in}{2.414368in}}%
\pgfpathlineto{\pgfqpoint{3.599154in}{2.406151in}}%
\pgfpathlineto{\pgfqpoint{3.591433in}{2.397999in}}%
\pgfpathclose%
\pgfusepath{fill}%
\end{pgfscope}%
\begin{pgfscope}%
\pgfpathrectangle{\pgfqpoint{1.254980in}{0.150000in}}{\pgfqpoint{5.490039in}{5.490039in}}%
\pgfusepath{clip}%
\pgfsetbuttcap%
\pgfsetroundjoin%
\definecolor{currentfill}{rgb}{0.166617,0.463708,0.558119}%
\pgfsetfillcolor{currentfill}%
\pgfsetfillopacity{0.700000}%
\pgfsetlinewidth{0.000000pt}%
\definecolor{currentstroke}{rgb}{0.000000,0.000000,0.000000}%
\pgfsetstrokecolor{currentstroke}%
\pgfsetdash{}{0pt}%
\pgfpathmoveto{\pgfqpoint{2.880375in}{3.103568in}}%
\pgfpathlineto{\pgfqpoint{2.893548in}{3.084804in}}%
\pgfpathlineto{\pgfqpoint{2.906713in}{3.066239in}}%
\pgfpathlineto{\pgfqpoint{2.919873in}{3.047871in}}%
\pgfpathlineto{\pgfqpoint{2.933027in}{3.029699in}}%
\pgfpathlineto{\pgfqpoint{2.941023in}{3.035808in}}%
\pgfpathlineto{\pgfqpoint{2.949011in}{3.042034in}}%
\pgfpathlineto{\pgfqpoint{2.956989in}{3.048377in}}%
\pgfpathlineto{\pgfqpoint{2.964959in}{3.054837in}}%
\pgfpathlineto{\pgfqpoint{2.951830in}{3.072860in}}%
\pgfpathlineto{\pgfqpoint{2.938696in}{3.091080in}}%
\pgfpathlineto{\pgfqpoint{2.925555in}{3.109497in}}%
\pgfpathlineto{\pgfqpoint{2.912408in}{3.128113in}}%
\pgfpathlineto{\pgfqpoint{2.904414in}{3.121795in}}%
\pgfpathlineto{\pgfqpoint{2.896410in}{3.115598in}}%
\pgfpathlineto{\pgfqpoint{2.888397in}{3.109522in}}%
\pgfpathlineto{\pgfqpoint{2.880375in}{3.103568in}}%
\pgfpathclose%
\pgfusepath{fill}%
\end{pgfscope}%
\begin{pgfscope}%
\pgfpathrectangle{\pgfqpoint{1.254980in}{0.150000in}}{\pgfqpoint{5.490039in}{5.490039in}}%
\pgfusepath{clip}%
\pgfsetbuttcap%
\pgfsetroundjoin%
\definecolor{currentfill}{rgb}{0.265145,0.232956,0.516599}%
\pgfsetfillcolor{currentfill}%
\pgfsetfillopacity{0.700000}%
\pgfsetlinewidth{0.000000pt}%
\definecolor{currentstroke}{rgb}{0.000000,0.000000,0.000000}%
\pgfsetstrokecolor{currentstroke}%
\pgfsetdash{}{0pt}%
\pgfpathmoveto{\pgfqpoint{3.351769in}{2.544705in}}%
\pgfpathlineto{\pgfqpoint{3.364815in}{2.532367in}}%
\pgfpathlineto{\pgfqpoint{3.377860in}{2.520189in}}%
\pgfpathlineto{\pgfqpoint{3.390905in}{2.508170in}}%
\pgfpathlineto{\pgfqpoint{3.403949in}{2.496309in}}%
\pgfpathlineto{\pgfqpoint{3.411749in}{2.503760in}}%
\pgfpathlineto{\pgfqpoint{3.419542in}{2.511291in}}%
\pgfpathlineto{\pgfqpoint{3.427328in}{2.518902in}}%
\pgfpathlineto{\pgfqpoint{3.435107in}{2.526592in}}%
\pgfpathlineto{\pgfqpoint{3.422082in}{2.538313in}}%
\pgfpathlineto{\pgfqpoint{3.409056in}{2.550193in}}%
\pgfpathlineto{\pgfqpoint{3.396029in}{2.562231in}}%
\pgfpathlineto{\pgfqpoint{3.383002in}{2.574430in}}%
\pgfpathlineto{\pgfqpoint{3.375204in}{2.566873in}}%
\pgfpathlineto{\pgfqpoint{3.367400in}{2.559400in}}%
\pgfpathlineto{\pgfqpoint{3.359588in}{2.552010in}}%
\pgfpathlineto{\pgfqpoint{3.351769in}{2.544705in}}%
\pgfpathclose%
\pgfusepath{fill}%
\end{pgfscope}%
\begin{pgfscope}%
\pgfpathrectangle{\pgfqpoint{1.254980in}{0.150000in}}{\pgfqpoint{5.490039in}{5.490039in}}%
\pgfusepath{clip}%
\pgfsetbuttcap%
\pgfsetroundjoin%
\definecolor{currentfill}{rgb}{0.277134,0.185228,0.489898}%
\pgfsetfillcolor{currentfill}%
\pgfsetfillopacity{0.700000}%
\pgfsetlinewidth{0.000000pt}%
\definecolor{currentstroke}{rgb}{0.000000,0.000000,0.000000}%
\pgfsetstrokecolor{currentstroke}%
\pgfsetdash{}{0pt}%
\pgfpathmoveto{\pgfqpoint{4.648093in}{2.417865in}}%
\pgfpathlineto{\pgfqpoint{4.661331in}{2.416573in}}%
\pgfpathlineto{\pgfqpoint{4.674577in}{2.415401in}}%
\pgfpathlineto{\pgfqpoint{4.687832in}{2.414349in}}%
\pgfpathlineto{\pgfqpoint{4.701096in}{2.413417in}}%
\pgfpathlineto{\pgfqpoint{4.708453in}{2.423409in}}%
\pgfpathlineto{\pgfqpoint{4.715805in}{2.433398in}}%
\pgfpathlineto{\pgfqpoint{4.723153in}{2.443384in}}%
\pgfpathlineto{\pgfqpoint{4.730496in}{2.453368in}}%
\pgfpathlineto{\pgfqpoint{4.717241in}{2.454324in}}%
\pgfpathlineto{\pgfqpoint{4.703996in}{2.455400in}}%
\pgfpathlineto{\pgfqpoint{4.690759in}{2.456596in}}%
\pgfpathlineto{\pgfqpoint{4.677530in}{2.457912in}}%
\pgfpathlineto{\pgfqpoint{4.670178in}{2.447899in}}%
\pgfpathlineto{\pgfqpoint{4.662821in}{2.437887in}}%
\pgfpathlineto{\pgfqpoint{4.655459in}{2.427876in}}%
\pgfpathlineto{\pgfqpoint{4.648093in}{2.417865in}}%
\pgfpathclose%
\pgfusepath{fill}%
\end{pgfscope}%
\begin{pgfscope}%
\pgfpathrectangle{\pgfqpoint{1.254980in}{0.150000in}}{\pgfqpoint{5.490039in}{5.490039in}}%
\pgfusepath{clip}%
\pgfsetbuttcap%
\pgfsetroundjoin%
\definecolor{currentfill}{rgb}{0.283229,0.120777,0.440584}%
\pgfsetfillcolor{currentfill}%
\pgfsetfillopacity{0.700000}%
\pgfsetlinewidth{0.000000pt}%
\definecolor{currentstroke}{rgb}{0.000000,0.000000,0.000000}%
\pgfsetstrokecolor{currentstroke}%
\pgfsetdash{}{0pt}%
\pgfpathmoveto{\pgfqpoint{3.913509in}{2.302182in}}%
\pgfpathlineto{\pgfqpoint{3.926566in}{2.295510in}}%
\pgfpathlineto{\pgfqpoint{3.939627in}{2.288974in}}%
\pgfpathlineto{\pgfqpoint{3.952692in}{2.282573in}}%
\pgfpathlineto{\pgfqpoint{3.965761in}{2.276307in}}%
\pgfpathlineto{\pgfqpoint{3.973353in}{2.285527in}}%
\pgfpathlineto{\pgfqpoint{3.980939in}{2.294782in}}%
\pgfpathlineto{\pgfqpoint{3.988520in}{2.304074in}}%
\pgfpathlineto{\pgfqpoint{3.996096in}{2.313401in}}%
\pgfpathlineto{\pgfqpoint{3.983039in}{2.319579in}}%
\pgfpathlineto{\pgfqpoint{3.969987in}{2.325892in}}%
\pgfpathlineto{\pgfqpoint{3.956938in}{2.332341in}}%
\pgfpathlineto{\pgfqpoint{3.943894in}{2.338925in}}%
\pgfpathlineto{\pgfqpoint{3.936306in}{2.329680in}}%
\pgfpathlineto{\pgfqpoint{3.928712in}{2.320474in}}%
\pgfpathlineto{\pgfqpoint{3.921113in}{2.311308in}}%
\pgfpathlineto{\pgfqpoint{3.913509in}{2.302182in}}%
\pgfpathclose%
\pgfusepath{fill}%
\end{pgfscope}%
\begin{pgfscope}%
\pgfpathrectangle{\pgfqpoint{1.254980in}{0.150000in}}{\pgfqpoint{5.490039in}{5.490039in}}%
\pgfusepath{clip}%
\pgfsetbuttcap%
\pgfsetroundjoin%
\definecolor{currentfill}{rgb}{0.283187,0.125848,0.444960}%
\pgfsetfillcolor{currentfill}%
\pgfsetfillopacity{0.700000}%
\pgfsetlinewidth{0.000000pt}%
\definecolor{currentstroke}{rgb}{0.000000,0.000000,0.000000}%
\pgfsetstrokecolor{currentstroke}%
\pgfsetdash{}{0pt}%
\pgfpathmoveto{\pgfqpoint{4.265754in}{2.310377in}}%
\pgfpathlineto{\pgfqpoint{4.278880in}{2.306547in}}%
\pgfpathlineto{\pgfqpoint{4.292012in}{2.302845in}}%
\pgfpathlineto{\pgfqpoint{4.305151in}{2.299269in}}%
\pgfpathlineto{\pgfqpoint{4.318296in}{2.295819in}}%
\pgfpathlineto{\pgfqpoint{4.325774in}{2.305678in}}%
\pgfpathlineto{\pgfqpoint{4.333247in}{2.315550in}}%
\pgfpathlineto{\pgfqpoint{4.340716in}{2.325435in}}%
\pgfpathlineto{\pgfqpoint{4.348180in}{2.335333in}}%
\pgfpathlineto{\pgfqpoint{4.335044in}{2.338743in}}%
\pgfpathlineto{\pgfqpoint{4.321915in}{2.342280in}}%
\pgfpathlineto{\pgfqpoint{4.308793in}{2.345943in}}%
\pgfpathlineto{\pgfqpoint{4.295677in}{2.349733in}}%
\pgfpathlineto{\pgfqpoint{4.288203in}{2.339868in}}%
\pgfpathlineto{\pgfqpoint{4.280725in}{2.330021in}}%
\pgfpathlineto{\pgfqpoint{4.273242in}{2.320190in}}%
\pgfpathlineto{\pgfqpoint{4.265754in}{2.310377in}}%
\pgfpathclose%
\pgfusepath{fill}%
\end{pgfscope}%
\begin{pgfscope}%
\pgfpathrectangle{\pgfqpoint{1.254980in}{0.150000in}}{\pgfqpoint{5.490039in}{5.490039in}}%
\pgfusepath{clip}%
\pgfsetbuttcap%
\pgfsetroundjoin%
\definecolor{currentfill}{rgb}{0.283072,0.130895,0.449241}%
\pgfsetfillcolor{currentfill}%
\pgfsetfillopacity{0.700000}%
\pgfsetlinewidth{0.000000pt}%
\definecolor{currentstroke}{rgb}{0.000000,0.000000,0.000000}%
\pgfsetstrokecolor{currentstroke}%
\pgfsetdash{}{0pt}%
\pgfpathmoveto{\pgfqpoint{3.778608in}{2.325252in}}%
\pgfpathlineto{\pgfqpoint{3.791651in}{2.317370in}}%
\pgfpathlineto{\pgfqpoint{3.804697in}{2.309628in}}%
\pgfpathlineto{\pgfqpoint{3.817746in}{2.302025in}}%
\pgfpathlineto{\pgfqpoint{3.830798in}{2.294562in}}%
\pgfpathlineto{\pgfqpoint{3.838437in}{2.303413in}}%
\pgfpathlineto{\pgfqpoint{3.846070in}{2.312309in}}%
\pgfpathlineto{\pgfqpoint{3.853697in}{2.321252in}}%
\pgfpathlineto{\pgfqpoint{3.861320in}{2.330239in}}%
\pgfpathlineto{\pgfqpoint{3.848281in}{2.337599in}}%
\pgfpathlineto{\pgfqpoint{3.835246in}{2.345097in}}%
\pgfpathlineto{\pgfqpoint{3.822213in}{2.352735in}}%
\pgfpathlineto{\pgfqpoint{3.809184in}{2.360513in}}%
\pgfpathlineto{\pgfqpoint{3.801549in}{2.351623in}}%
\pgfpathlineto{\pgfqpoint{3.793907in}{2.342783in}}%
\pgfpathlineto{\pgfqpoint{3.786260in}{2.333992in}}%
\pgfpathlineto{\pgfqpoint{3.778608in}{2.325252in}}%
\pgfpathclose%
\pgfusepath{fill}%
\end{pgfscope}%
\begin{pgfscope}%
\pgfpathrectangle{\pgfqpoint{1.254980in}{0.150000in}}{\pgfqpoint{5.490039in}{5.490039in}}%
\pgfusepath{clip}%
\pgfsetbuttcap%
\pgfsetroundjoin%
\definecolor{currentfill}{rgb}{0.283197,0.115680,0.436115}%
\pgfsetfillcolor{currentfill}%
\pgfsetfillopacity{0.700000}%
\pgfsetlinewidth{0.000000pt}%
\definecolor{currentstroke}{rgb}{0.000000,0.000000,0.000000}%
\pgfsetstrokecolor{currentstroke}%
\pgfsetdash{}{0pt}%
\pgfpathmoveto{\pgfqpoint{4.048369in}{2.290025in}}%
\pgfpathlineto{\pgfqpoint{4.061449in}{2.284514in}}%
\pgfpathlineto{\pgfqpoint{4.074534in}{2.279134in}}%
\pgfpathlineto{\pgfqpoint{4.087624in}{2.273886in}}%
\pgfpathlineto{\pgfqpoint{4.100719in}{2.268769in}}%
\pgfpathlineto{\pgfqpoint{4.108267in}{2.278286in}}%
\pgfpathlineto{\pgfqpoint{4.115810in}{2.287829in}}%
\pgfpathlineto{\pgfqpoint{4.123347in}{2.297398in}}%
\pgfpathlineto{\pgfqpoint{4.130880in}{2.306994in}}%
\pgfpathlineto{\pgfqpoint{4.117796in}{2.312040in}}%
\pgfpathlineto{\pgfqpoint{4.104718in}{2.317216in}}%
\pgfpathlineto{\pgfqpoint{4.091644in}{2.322524in}}%
\pgfpathlineto{\pgfqpoint{4.078575in}{2.327964in}}%
\pgfpathlineto{\pgfqpoint{4.071031in}{2.318434in}}%
\pgfpathlineto{\pgfqpoint{4.063482in}{2.308934in}}%
\pgfpathlineto{\pgfqpoint{4.055928in}{2.299465in}}%
\pgfpathlineto{\pgfqpoint{4.048369in}{2.290025in}}%
\pgfpathclose%
\pgfusepath{fill}%
\end{pgfscope}%
\begin{pgfscope}%
\pgfpathrectangle{\pgfqpoint{1.254980in}{0.150000in}}{\pgfqpoint{5.490039in}{5.490039in}}%
\pgfusepath{clip}%
\pgfsetbuttcap%
\pgfsetroundjoin%
\definecolor{currentfill}{rgb}{0.279574,0.170599,0.479997}%
\pgfsetfillcolor{currentfill}%
\pgfsetfillopacity{0.700000}%
\pgfsetlinewidth{0.000000pt}%
\definecolor{currentstroke}{rgb}{0.000000,0.000000,0.000000}%
\pgfsetstrokecolor{currentstroke}%
\pgfsetdash{}{0pt}%
\pgfpathmoveto{\pgfqpoint{4.565680in}{2.384169in}}%
\pgfpathlineto{\pgfqpoint{4.578893in}{2.382400in}}%
\pgfpathlineto{\pgfqpoint{4.592115in}{2.380753in}}%
\pgfpathlineto{\pgfqpoint{4.605344in}{2.379227in}}%
\pgfpathlineto{\pgfqpoint{4.618583in}{2.377822in}}%
\pgfpathlineto{\pgfqpoint{4.625967in}{2.387834in}}%
\pgfpathlineto{\pgfqpoint{4.633347in}{2.397845in}}%
\pgfpathlineto{\pgfqpoint{4.640722in}{2.407855in}}%
\pgfpathlineto{\pgfqpoint{4.648093in}{2.417865in}}%
\pgfpathlineto{\pgfqpoint{4.634864in}{2.419278in}}%
\pgfpathlineto{\pgfqpoint{4.621644in}{2.420813in}}%
\pgfpathlineto{\pgfqpoint{4.608431in}{2.422468in}}%
\pgfpathlineto{\pgfqpoint{4.595227in}{2.424246in}}%
\pgfpathlineto{\pgfqpoint{4.587847in}{2.414221in}}%
\pgfpathlineto{\pgfqpoint{4.580463in}{2.404201in}}%
\pgfpathlineto{\pgfqpoint{4.573074in}{2.394184in}}%
\pgfpathlineto{\pgfqpoint{4.565680in}{2.384169in}}%
\pgfpathclose%
\pgfusepath{fill}%
\end{pgfscope}%
\begin{pgfscope}%
\pgfpathrectangle{\pgfqpoint{1.254980in}{0.150000in}}{\pgfqpoint{5.490039in}{5.490039in}}%
\pgfusepath{clip}%
\pgfsetbuttcap%
\pgfsetroundjoin%
\definecolor{currentfill}{rgb}{0.270595,0.214069,0.507052}%
\pgfsetfillcolor{currentfill}%
\pgfsetfillopacity{0.700000}%
\pgfsetlinewidth{0.000000pt}%
\definecolor{currentstroke}{rgb}{0.000000,0.000000,0.000000}%
\pgfsetstrokecolor{currentstroke}%
\pgfsetdash{}{0pt}%
\pgfpathmoveto{\pgfqpoint{3.403949in}{2.496309in}}%
\pgfpathlineto{\pgfqpoint{3.416993in}{2.484607in}}%
\pgfpathlineto{\pgfqpoint{3.430036in}{2.473061in}}%
\pgfpathlineto{\pgfqpoint{3.443080in}{2.461671in}}%
\pgfpathlineto{\pgfqpoint{3.456123in}{2.450436in}}%
\pgfpathlineto{\pgfqpoint{3.463904in}{2.458032in}}%
\pgfpathlineto{\pgfqpoint{3.471679in}{2.465704in}}%
\pgfpathlineto{\pgfqpoint{3.479447in}{2.473452in}}%
\pgfpathlineto{\pgfqpoint{3.487208in}{2.481274in}}%
\pgfpathlineto{\pgfqpoint{3.474183in}{2.492370in}}%
\pgfpathlineto{\pgfqpoint{3.461158in}{2.503622in}}%
\pgfpathlineto{\pgfqpoint{3.448133in}{2.515029in}}%
\pgfpathlineto{\pgfqpoint{3.435107in}{2.526592in}}%
\pgfpathlineto{\pgfqpoint{3.427328in}{2.518902in}}%
\pgfpathlineto{\pgfqpoint{3.419542in}{2.511291in}}%
\pgfpathlineto{\pgfqpoint{3.411749in}{2.503760in}}%
\pgfpathlineto{\pgfqpoint{3.403949in}{2.496309in}}%
\pgfpathclose%
\pgfusepath{fill}%
\end{pgfscope}%
\begin{pgfscope}%
\pgfpathrectangle{\pgfqpoint{1.254980in}{0.150000in}}{\pgfqpoint{5.490039in}{5.490039in}}%
\pgfusepath{clip}%
\pgfsetbuttcap%
\pgfsetroundjoin%
\definecolor{currentfill}{rgb}{0.153364,0.497000,0.557724}%
\pgfsetfillcolor{currentfill}%
\pgfsetfillopacity{0.700000}%
\pgfsetlinewidth{0.000000pt}%
\definecolor{currentstroke}{rgb}{0.000000,0.000000,0.000000}%
\pgfsetstrokecolor{currentstroke}%
\pgfsetdash{}{0pt}%
\pgfpathmoveto{\pgfqpoint{2.827620in}{3.180641in}}%
\pgfpathlineto{\pgfqpoint{2.840819in}{3.161067in}}%
\pgfpathlineto{\pgfqpoint{2.854012in}{3.141699in}}%
\pgfpathlineto{\pgfqpoint{2.867197in}{3.122533in}}%
\pgfpathlineto{\pgfqpoint{2.880375in}{3.103568in}}%
\pgfpathlineto{\pgfqpoint{2.888397in}{3.109522in}}%
\pgfpathlineto{\pgfqpoint{2.896410in}{3.115598in}}%
\pgfpathlineto{\pgfqpoint{2.904414in}{3.121795in}}%
\pgfpathlineto{\pgfqpoint{2.912408in}{3.128113in}}%
\pgfpathlineto{\pgfqpoint{2.899255in}{3.146928in}}%
\pgfpathlineto{\pgfqpoint{2.886096in}{3.165945in}}%
\pgfpathlineto{\pgfqpoint{2.872930in}{3.185164in}}%
\pgfpathlineto{\pgfqpoint{2.859757in}{3.204587in}}%
\pgfpathlineto{\pgfqpoint{2.851737in}{3.198413in}}%
\pgfpathlineto{\pgfqpoint{2.843707in}{3.192363in}}%
\pgfpathlineto{\pgfqpoint{2.835669in}{3.186439in}}%
\pgfpathlineto{\pgfqpoint{2.827620in}{3.180641in}}%
\pgfpathclose%
\pgfusepath{fill}%
\end{pgfscope}%
\begin{pgfscope}%
\pgfpathrectangle{\pgfqpoint{1.254980in}{0.150000in}}{\pgfqpoint{5.490039in}{5.490039in}}%
\pgfusepath{clip}%
\pgfsetbuttcap%
\pgfsetroundjoin%
\definecolor{currentfill}{rgb}{0.281887,0.150881,0.465405}%
\pgfsetfillcolor{currentfill}%
\pgfsetfillopacity{0.700000}%
\pgfsetlinewidth{0.000000pt}%
\definecolor{currentstroke}{rgb}{0.000000,0.000000,0.000000}%
\pgfsetstrokecolor{currentstroke}%
\pgfsetdash{}{0pt}%
\pgfpathmoveto{\pgfqpoint{3.643572in}{2.359948in}}%
\pgfpathlineto{\pgfqpoint{3.656610in}{2.350802in}}%
\pgfpathlineto{\pgfqpoint{3.669651in}{2.341800in}}%
\pgfpathlineto{\pgfqpoint{3.682694in}{2.332944in}}%
\pgfpathlineto{\pgfqpoint{3.695739in}{2.324231in}}%
\pgfpathlineto{\pgfqpoint{3.703429in}{2.332637in}}%
\pgfpathlineto{\pgfqpoint{3.711113in}{2.341100in}}%
\pgfpathlineto{\pgfqpoint{3.718791in}{2.349619in}}%
\pgfpathlineto{\pgfqpoint{3.726464in}{2.358195in}}%
\pgfpathlineto{\pgfqpoint{3.713434in}{2.366787in}}%
\pgfpathlineto{\pgfqpoint{3.700406in}{2.375523in}}%
\pgfpathlineto{\pgfqpoint{3.687381in}{2.384403in}}%
\pgfpathlineto{\pgfqpoint{3.674357in}{2.393429in}}%
\pgfpathlineto{\pgfqpoint{3.666670in}{2.384969in}}%
\pgfpathlineto{\pgfqpoint{3.658976in}{2.376568in}}%
\pgfpathlineto{\pgfqpoint{3.651277in}{2.368227in}}%
\pgfpathlineto{\pgfqpoint{3.643572in}{2.359948in}}%
\pgfpathclose%
\pgfusepath{fill}%
\end{pgfscope}%
\begin{pgfscope}%
\pgfpathrectangle{\pgfqpoint{1.254980in}{0.150000in}}{\pgfqpoint{5.490039in}{5.490039in}}%
\pgfusepath{clip}%
\pgfsetbuttcap%
\pgfsetroundjoin%
\definecolor{currentfill}{rgb}{0.281412,0.155834,0.469201}%
\pgfsetfillcolor{currentfill}%
\pgfsetfillopacity{0.700000}%
\pgfsetlinewidth{0.000000pt}%
\definecolor{currentstroke}{rgb}{0.000000,0.000000,0.000000}%
\pgfsetstrokecolor{currentstroke}%
\pgfsetdash{}{0pt}%
\pgfpathmoveto{\pgfqpoint{4.483248in}{2.352465in}}%
\pgfpathlineto{\pgfqpoint{4.496439in}{2.350199in}}%
\pgfpathlineto{\pgfqpoint{4.509638in}{2.348055in}}%
\pgfpathlineto{\pgfqpoint{4.522844in}{2.346034in}}%
\pgfpathlineto{\pgfqpoint{4.536059in}{2.344136in}}%
\pgfpathlineto{\pgfqpoint{4.543471in}{2.354141in}}%
\pgfpathlineto{\pgfqpoint{4.550879in}{2.364148in}}%
\pgfpathlineto{\pgfqpoint{4.558281in}{2.374158in}}%
\pgfpathlineto{\pgfqpoint{4.565680in}{2.384169in}}%
\pgfpathlineto{\pgfqpoint{4.552475in}{2.386060in}}%
\pgfpathlineto{\pgfqpoint{4.539277in}{2.388074in}}%
\pgfpathlineto{\pgfqpoint{4.526088in}{2.390209in}}%
\pgfpathlineto{\pgfqpoint{4.512907in}{2.392468in}}%
\pgfpathlineto{\pgfqpoint{4.505499in}{2.382458in}}%
\pgfpathlineto{\pgfqpoint{4.498087in}{2.372455in}}%
\pgfpathlineto{\pgfqpoint{4.490670in}{2.362457in}}%
\pgfpathlineto{\pgfqpoint{4.483248in}{2.352465in}}%
\pgfpathclose%
\pgfusepath{fill}%
\end{pgfscope}%
\begin{pgfscope}%
\pgfpathrectangle{\pgfqpoint{1.254980in}{0.150000in}}{\pgfqpoint{5.490039in}{5.490039in}}%
\pgfusepath{clip}%
\pgfsetbuttcap%
\pgfsetroundjoin%
\definecolor{currentfill}{rgb}{0.283229,0.120777,0.440584}%
\pgfsetfillcolor{currentfill}%
\pgfsetfillopacity{0.700000}%
\pgfsetlinewidth{0.000000pt}%
\definecolor{currentstroke}{rgb}{0.000000,0.000000,0.000000}%
\pgfsetstrokecolor{currentstroke}%
\pgfsetdash{}{0pt}%
\pgfpathmoveto{\pgfqpoint{4.183271in}{2.288113in}}%
\pgfpathlineto{\pgfqpoint{4.196383in}{2.283717in}}%
\pgfpathlineto{\pgfqpoint{4.209501in}{2.279448in}}%
\pgfpathlineto{\pgfqpoint{4.222625in}{2.275308in}}%
\pgfpathlineto{\pgfqpoint{4.235755in}{2.271296in}}%
\pgfpathlineto{\pgfqpoint{4.243262in}{2.281040in}}%
\pgfpathlineto{\pgfqpoint{4.250764in}{2.290802in}}%
\pgfpathlineto{\pgfqpoint{4.258261in}{2.300581in}}%
\pgfpathlineto{\pgfqpoint{4.265754in}{2.310377in}}%
\pgfpathlineto{\pgfqpoint{4.252634in}{2.314334in}}%
\pgfpathlineto{\pgfqpoint{4.239521in}{2.318419in}}%
\pgfpathlineto{\pgfqpoint{4.226413in}{2.322631in}}%
\pgfpathlineto{\pgfqpoint{4.213312in}{2.326973in}}%
\pgfpathlineto{\pgfqpoint{4.205809in}{2.317226in}}%
\pgfpathlineto{\pgfqpoint{4.198301in}{2.307500in}}%
\pgfpathlineto{\pgfqpoint{4.190789in}{2.297796in}}%
\pgfpathlineto{\pgfqpoint{4.183271in}{2.288113in}}%
\pgfpathclose%
\pgfusepath{fill}%
\end{pgfscope}%
\begin{pgfscope}%
\pgfpathrectangle{\pgfqpoint{1.254980in}{0.150000in}}{\pgfqpoint{5.490039in}{5.490039in}}%
\pgfusepath{clip}%
\pgfsetbuttcap%
\pgfsetroundjoin%
\definecolor{currentfill}{rgb}{0.275191,0.194905,0.496005}%
\pgfsetfillcolor{currentfill}%
\pgfsetfillopacity{0.700000}%
\pgfsetlinewidth{0.000000pt}%
\definecolor{currentstroke}{rgb}{0.000000,0.000000,0.000000}%
\pgfsetstrokecolor{currentstroke}%
\pgfsetdash{}{0pt}%
\pgfpathmoveto{\pgfqpoint{3.456123in}{2.450436in}}%
\pgfpathlineto{\pgfqpoint{3.469167in}{2.439356in}}%
\pgfpathlineto{\pgfqpoint{3.482210in}{2.428430in}}%
\pgfpathlineto{\pgfqpoint{3.495254in}{2.417656in}}%
\pgfpathlineto{\pgfqpoint{3.508299in}{2.407035in}}%
\pgfpathlineto{\pgfqpoint{3.516062in}{2.414775in}}%
\pgfpathlineto{\pgfqpoint{3.523819in}{2.422587in}}%
\pgfpathlineto{\pgfqpoint{3.531570in}{2.430471in}}%
\pgfpathlineto{\pgfqpoint{3.539314in}{2.438426in}}%
\pgfpathlineto{\pgfqpoint{3.526287in}{2.448909in}}%
\pgfpathlineto{\pgfqpoint{3.513260in}{2.459544in}}%
\pgfpathlineto{\pgfqpoint{3.500234in}{2.470333in}}%
\pgfpathlineto{\pgfqpoint{3.487208in}{2.481274in}}%
\pgfpathlineto{\pgfqpoint{3.479447in}{2.473452in}}%
\pgfpathlineto{\pgfqpoint{3.471679in}{2.465704in}}%
\pgfpathlineto{\pgfqpoint{3.463904in}{2.458032in}}%
\pgfpathlineto{\pgfqpoint{3.456123in}{2.450436in}}%
\pgfpathclose%
\pgfusepath{fill}%
\end{pgfscope}%
\begin{pgfscope}%
\pgfpathrectangle{\pgfqpoint{1.254980in}{0.150000in}}{\pgfqpoint{5.490039in}{5.490039in}}%
\pgfusepath{clip}%
\pgfsetbuttcap%
\pgfsetroundjoin%
\definecolor{currentfill}{rgb}{0.141935,0.526453,0.555991}%
\pgfsetfillcolor{currentfill}%
\pgfsetfillopacity{0.700000}%
\pgfsetlinewidth{0.000000pt}%
\definecolor{currentstroke}{rgb}{0.000000,0.000000,0.000000}%
\pgfsetstrokecolor{currentstroke}%
\pgfsetdash{}{0pt}%
\pgfpathmoveto{\pgfqpoint{2.774750in}{3.261006in}}%
\pgfpathlineto{\pgfqpoint{2.787979in}{3.240601in}}%
\pgfpathlineto{\pgfqpoint{2.801200in}{3.220406in}}%
\pgfpathlineto{\pgfqpoint{2.814414in}{3.200420in}}%
\pgfpathlineto{\pgfqpoint{2.827620in}{3.180641in}}%
\pgfpathlineto{\pgfqpoint{2.835669in}{3.186439in}}%
\pgfpathlineto{\pgfqpoint{2.843707in}{3.192363in}}%
\pgfpathlineto{\pgfqpoint{2.851737in}{3.198413in}}%
\pgfpathlineto{\pgfqpoint{2.859757in}{3.204587in}}%
\pgfpathlineto{\pgfqpoint{2.846577in}{3.224216in}}%
\pgfpathlineto{\pgfqpoint{2.833390in}{3.244051in}}%
\pgfpathlineto{\pgfqpoint{2.820195in}{3.264095in}}%
\pgfpathlineto{\pgfqpoint{2.806993in}{3.284349in}}%
\pgfpathlineto{\pgfqpoint{2.798947in}{3.278319in}}%
\pgfpathlineto{\pgfqpoint{2.790891in}{3.272418in}}%
\pgfpathlineto{\pgfqpoint{2.782825in}{3.266647in}}%
\pgfpathlineto{\pgfqpoint{2.774750in}{3.261006in}}%
\pgfpathclose%
\pgfusepath{fill}%
\end{pgfscope}%
\begin{pgfscope}%
\pgfpathrectangle{\pgfqpoint{1.254980in}{0.150000in}}{\pgfqpoint{5.490039in}{5.490039in}}%
\pgfusepath{clip}%
\pgfsetbuttcap%
\pgfsetroundjoin%
\definecolor{currentfill}{rgb}{0.282623,0.140926,0.457517}%
\pgfsetfillcolor{currentfill}%
\pgfsetfillopacity{0.700000}%
\pgfsetlinewidth{0.000000pt}%
\definecolor{currentstroke}{rgb}{0.000000,0.000000,0.000000}%
\pgfsetstrokecolor{currentstroke}%
\pgfsetdash{}{0pt}%
\pgfpathmoveto{\pgfqpoint{4.400791in}{2.322947in}}%
\pgfpathlineto{\pgfqpoint{4.413961in}{2.320162in}}%
\pgfpathlineto{\pgfqpoint{4.427139in}{2.317502in}}%
\pgfpathlineto{\pgfqpoint{4.440324in}{2.314966in}}%
\pgfpathlineto{\pgfqpoint{4.453516in}{2.312553in}}%
\pgfpathlineto{\pgfqpoint{4.460956in}{2.322523in}}%
\pgfpathlineto{\pgfqpoint{4.468392in}{2.332498in}}%
\pgfpathlineto{\pgfqpoint{4.475822in}{2.342479in}}%
\pgfpathlineto{\pgfqpoint{4.483248in}{2.352465in}}%
\pgfpathlineto{\pgfqpoint{4.470065in}{2.354855in}}%
\pgfpathlineto{\pgfqpoint{4.456890in}{2.357368in}}%
\pgfpathlineto{\pgfqpoint{4.443721in}{2.360004in}}%
\pgfpathlineto{\pgfqpoint{4.430560in}{2.362765in}}%
\pgfpathlineto{\pgfqpoint{4.423125in}{2.352797in}}%
\pgfpathlineto{\pgfqpoint{4.415685in}{2.342838in}}%
\pgfpathlineto{\pgfqpoint{4.408240in}{2.332888in}}%
\pgfpathlineto{\pgfqpoint{4.400791in}{2.322947in}}%
\pgfpathclose%
\pgfusepath{fill}%
\end{pgfscope}%
\begin{pgfscope}%
\pgfpathrectangle{\pgfqpoint{1.254980in}{0.150000in}}{\pgfqpoint{5.490039in}{5.490039in}}%
\pgfusepath{clip}%
\pgfsetbuttcap%
\pgfsetroundjoin%
\definecolor{currentfill}{rgb}{0.283229,0.120777,0.440584}%
\pgfsetfillcolor{currentfill}%
\pgfsetfillopacity{0.700000}%
\pgfsetlinewidth{0.000000pt}%
\definecolor{currentstroke}{rgb}{0.000000,0.000000,0.000000}%
\pgfsetstrokecolor{currentstroke}%
\pgfsetdash{}{0pt}%
\pgfpathmoveto{\pgfqpoint{3.830798in}{2.294562in}}%
\pgfpathlineto{\pgfqpoint{3.843853in}{2.287238in}}%
\pgfpathlineto{\pgfqpoint{3.856912in}{2.280051in}}%
\pgfpathlineto{\pgfqpoint{3.869975in}{2.273003in}}%
\pgfpathlineto{\pgfqpoint{3.883041in}{2.266091in}}%
\pgfpathlineto{\pgfqpoint{3.890666in}{2.275050in}}%
\pgfpathlineto{\pgfqpoint{3.898286in}{2.284053in}}%
\pgfpathlineto{\pgfqpoint{3.905900in}{2.293097in}}%
\pgfpathlineto{\pgfqpoint{3.913509in}{2.302182in}}%
\pgfpathlineto{\pgfqpoint{3.900456in}{2.308991in}}%
\pgfpathlineto{\pgfqpoint{3.887407in}{2.315936in}}%
\pgfpathlineto{\pgfqpoint{3.874362in}{2.323019in}}%
\pgfpathlineto{\pgfqpoint{3.861320in}{2.330239in}}%
\pgfpathlineto{\pgfqpoint{3.853697in}{2.321252in}}%
\pgfpathlineto{\pgfqpoint{3.846070in}{2.312309in}}%
\pgfpathlineto{\pgfqpoint{3.838437in}{2.303413in}}%
\pgfpathlineto{\pgfqpoint{3.830798in}{2.294562in}}%
\pgfpathclose%
\pgfusepath{fill}%
\end{pgfscope}%
\begin{pgfscope}%
\pgfpathrectangle{\pgfqpoint{1.254980in}{0.150000in}}{\pgfqpoint{5.490039in}{5.490039in}}%
\pgfusepath{clip}%
\pgfsetbuttcap%
\pgfsetroundjoin%
\definecolor{currentfill}{rgb}{0.283091,0.110553,0.431554}%
\pgfsetfillcolor{currentfill}%
\pgfsetfillopacity{0.700000}%
\pgfsetlinewidth{0.000000pt}%
\definecolor{currentstroke}{rgb}{0.000000,0.000000,0.000000}%
\pgfsetstrokecolor{currentstroke}%
\pgfsetdash{}{0pt}%
\pgfpathmoveto{\pgfqpoint{3.965761in}{2.276307in}}%
\pgfpathlineto{\pgfqpoint{3.978835in}{2.270175in}}%
\pgfpathlineto{\pgfqpoint{3.991912in}{2.264178in}}%
\pgfpathlineto{\pgfqpoint{4.004995in}{2.258314in}}%
\pgfpathlineto{\pgfqpoint{4.018082in}{2.252582in}}%
\pgfpathlineto{\pgfqpoint{4.025661in}{2.261895in}}%
\pgfpathlineto{\pgfqpoint{4.033235in}{2.271240in}}%
\pgfpathlineto{\pgfqpoint{4.040804in}{2.280617in}}%
\pgfpathlineto{\pgfqpoint{4.048369in}{2.290025in}}%
\pgfpathlineto{\pgfqpoint{4.035293in}{2.295669in}}%
\pgfpathlineto{\pgfqpoint{4.022223in}{2.301446in}}%
\pgfpathlineto{\pgfqpoint{4.009157in}{2.307357in}}%
\pgfpathlineto{\pgfqpoint{3.996096in}{2.313401in}}%
\pgfpathlineto{\pgfqpoint{3.988520in}{2.304074in}}%
\pgfpathlineto{\pgfqpoint{3.980939in}{2.294782in}}%
\pgfpathlineto{\pgfqpoint{3.973353in}{2.285527in}}%
\pgfpathlineto{\pgfqpoint{3.965761in}{2.276307in}}%
\pgfpathclose%
\pgfusepath{fill}%
\end{pgfscope}%
\begin{pgfscope}%
\pgfpathrectangle{\pgfqpoint{1.254980in}{0.150000in}}{\pgfqpoint{5.490039in}{5.490039in}}%
\pgfusepath{clip}%
\pgfsetbuttcap%
\pgfsetroundjoin%
\definecolor{currentfill}{rgb}{0.235526,0.309527,0.542944}%
\pgfsetfillcolor{currentfill}%
\pgfsetfillopacity{0.700000}%
\pgfsetlinewidth{0.000000pt}%
\definecolor{currentstroke}{rgb}{0.000000,0.000000,0.000000}%
\pgfsetstrokecolor{currentstroke}%
\pgfsetdash{}{0pt}%
\pgfpathmoveto{\pgfqpoint{5.196259in}{2.657496in}}%
\pgfpathlineto{\pgfqpoint{5.209718in}{2.659102in}}%
\pgfpathlineto{\pgfqpoint{5.223188in}{2.660822in}}%
\pgfpathlineto{\pgfqpoint{5.236669in}{2.662657in}}%
\pgfpathlineto{\pgfqpoint{5.250162in}{2.664605in}}%
\pgfpathlineto{\pgfqpoint{5.257337in}{2.673931in}}%
\pgfpathlineto{\pgfqpoint{5.264507in}{2.683248in}}%
\pgfpathlineto{\pgfqpoint{5.271673in}{2.692557in}}%
\pgfpathlineto{\pgfqpoint{5.278833in}{2.701860in}}%
\pgfpathlineto{\pgfqpoint{5.265352in}{2.700032in}}%
\pgfpathlineto{\pgfqpoint{5.251881in}{2.698319in}}%
\pgfpathlineto{\pgfqpoint{5.238423in}{2.696719in}}%
\pgfpathlineto{\pgfqpoint{5.224975in}{2.695234in}}%
\pgfpathlineto{\pgfqpoint{5.217803in}{2.685804in}}%
\pgfpathlineto{\pgfqpoint{5.210627in}{2.676373in}}%
\pgfpathlineto{\pgfqpoint{5.203445in}{2.666937in}}%
\pgfpathlineto{\pgfqpoint{5.196259in}{2.657496in}}%
\pgfpathclose%
\pgfusepath{fill}%
\end{pgfscope}%
\begin{pgfscope}%
\pgfpathrectangle{\pgfqpoint{1.254980in}{0.150000in}}{\pgfqpoint{5.490039in}{5.490039in}}%
\pgfusepath{clip}%
\pgfsetbuttcap%
\pgfsetroundjoin%
\definecolor{currentfill}{rgb}{0.244972,0.287675,0.537260}%
\pgfsetfillcolor{currentfill}%
\pgfsetfillopacity{0.700000}%
\pgfsetlinewidth{0.000000pt}%
\definecolor{currentstroke}{rgb}{0.000000,0.000000,0.000000}%
\pgfsetstrokecolor{currentstroke}%
\pgfsetdash{}{0pt}%
\pgfpathmoveto{\pgfqpoint{5.113702in}{2.613958in}}%
\pgfpathlineto{\pgfqpoint{5.127127in}{2.615209in}}%
\pgfpathlineto{\pgfqpoint{5.140563in}{2.616576in}}%
\pgfpathlineto{\pgfqpoint{5.154010in}{2.618057in}}%
\pgfpathlineto{\pgfqpoint{5.167468in}{2.619654in}}%
\pgfpathlineto{\pgfqpoint{5.174673in}{2.629128in}}%
\pgfpathlineto{\pgfqpoint{5.181873in}{2.638593in}}%
\pgfpathlineto{\pgfqpoint{5.189069in}{2.648048in}}%
\pgfpathlineto{\pgfqpoint{5.196259in}{2.657496in}}%
\pgfpathlineto{\pgfqpoint{5.182812in}{2.656004in}}%
\pgfpathlineto{\pgfqpoint{5.169376in}{2.654627in}}%
\pgfpathlineto{\pgfqpoint{5.155951in}{2.653365in}}%
\pgfpathlineto{\pgfqpoint{5.142537in}{2.652218in}}%
\pgfpathlineto{\pgfqpoint{5.135335in}{2.642660in}}%
\pgfpathlineto{\pgfqpoint{5.128129in}{2.633098in}}%
\pgfpathlineto{\pgfqpoint{5.120918in}{2.623531in}}%
\pgfpathlineto{\pgfqpoint{5.113702in}{2.613958in}}%
\pgfpathclose%
\pgfusepath{fill}%
\end{pgfscope}%
\begin{pgfscope}%
\pgfpathrectangle{\pgfqpoint{1.254980in}{0.150000in}}{\pgfqpoint{5.490039in}{5.490039in}}%
\pgfusepath{clip}%
\pgfsetbuttcap%
\pgfsetroundjoin%
\definecolor{currentfill}{rgb}{0.227802,0.326594,0.546532}%
\pgfsetfillcolor{currentfill}%
\pgfsetfillopacity{0.700000}%
\pgfsetlinewidth{0.000000pt}%
\definecolor{currentstroke}{rgb}{0.000000,0.000000,0.000000}%
\pgfsetstrokecolor{currentstroke}%
\pgfsetdash{}{0pt}%
\pgfpathmoveto{\pgfqpoint{5.278833in}{2.701860in}}%
\pgfpathlineto{\pgfqpoint{5.292327in}{2.703801in}}%
\pgfpathlineto{\pgfqpoint{5.305832in}{2.705856in}}%
\pgfpathlineto{\pgfqpoint{5.319348in}{2.708025in}}%
\pgfpathlineto{\pgfqpoint{5.332877in}{2.710307in}}%
\pgfpathlineto{\pgfqpoint{5.340021in}{2.719473in}}%
\pgfpathlineto{\pgfqpoint{5.347161in}{2.728632in}}%
\pgfpathlineto{\pgfqpoint{5.354295in}{2.737785in}}%
\pgfpathlineto{\pgfqpoint{5.361425in}{2.746934in}}%
\pgfpathlineto{\pgfqpoint{5.347909in}{2.744789in}}%
\pgfpathlineto{\pgfqpoint{5.334404in}{2.742758in}}%
\pgfpathlineto{\pgfqpoint{5.320911in}{2.740840in}}%
\pgfpathlineto{\pgfqpoint{5.307429in}{2.739035in}}%
\pgfpathlineto{\pgfqpoint{5.300287in}{2.729744in}}%
\pgfpathlineto{\pgfqpoint{5.293140in}{2.720452in}}%
\pgfpathlineto{\pgfqpoint{5.285989in}{2.711158in}}%
\pgfpathlineto{\pgfqpoint{5.278833in}{2.701860in}}%
\pgfpathclose%
\pgfusepath{fill}%
\end{pgfscope}%
\begin{pgfscope}%
\pgfpathrectangle{\pgfqpoint{1.254980in}{0.150000in}}{\pgfqpoint{5.490039in}{5.490039in}}%
\pgfusepath{clip}%
\pgfsetbuttcap%
\pgfsetroundjoin%
\definecolor{currentfill}{rgb}{0.252194,0.269783,0.531579}%
\pgfsetfillcolor{currentfill}%
\pgfsetfillopacity{0.700000}%
\pgfsetlinewidth{0.000000pt}%
\definecolor{currentstroke}{rgb}{0.000000,0.000000,0.000000}%
\pgfsetstrokecolor{currentstroke}%
\pgfsetdash{}{0pt}%
\pgfpathmoveto{\pgfqpoint{5.031161in}{2.571372in}}%
\pgfpathlineto{\pgfqpoint{5.044553in}{2.572250in}}%
\pgfpathlineto{\pgfqpoint{5.057956in}{2.573244in}}%
\pgfpathlineto{\pgfqpoint{5.071369in}{2.574354in}}%
\pgfpathlineto{\pgfqpoint{5.084793in}{2.575579in}}%
\pgfpathlineto{\pgfqpoint{5.092027in}{2.585188in}}%
\pgfpathlineto{\pgfqpoint{5.099257in}{2.594787in}}%
\pgfpathlineto{\pgfqpoint{5.106482in}{2.604377in}}%
\pgfpathlineto{\pgfqpoint{5.113702in}{2.613958in}}%
\pgfpathlineto{\pgfqpoint{5.100289in}{2.612821in}}%
\pgfpathlineto{\pgfqpoint{5.086886in}{2.611800in}}%
\pgfpathlineto{\pgfqpoint{5.073493in}{2.610895in}}%
\pgfpathlineto{\pgfqpoint{5.060112in}{2.610105in}}%
\pgfpathlineto{\pgfqpoint{5.052881in}{2.600430in}}%
\pgfpathlineto{\pgfqpoint{5.045646in}{2.590750in}}%
\pgfpathlineto{\pgfqpoint{5.038406in}{2.581065in}}%
\pgfpathlineto{\pgfqpoint{5.031161in}{2.571372in}}%
\pgfpathclose%
\pgfusepath{fill}%
\end{pgfscope}%
\begin{pgfscope}%
\pgfpathrectangle{\pgfqpoint{1.254980in}{0.150000in}}{\pgfqpoint{5.490039in}{5.490039in}}%
\pgfusepath{clip}%
\pgfsetbuttcap%
\pgfsetroundjoin%
\definecolor{currentfill}{rgb}{0.218130,0.347432,0.550038}%
\pgfsetfillcolor{currentfill}%
\pgfsetfillopacity{0.700000}%
\pgfsetlinewidth{0.000000pt}%
\definecolor{currentstroke}{rgb}{0.000000,0.000000,0.000000}%
\pgfsetstrokecolor{currentstroke}%
\pgfsetdash{}{0pt}%
\pgfpathmoveto{\pgfqpoint{5.361425in}{2.746934in}}%
\pgfpathlineto{\pgfqpoint{5.374954in}{2.749191in}}%
\pgfpathlineto{\pgfqpoint{5.388495in}{2.751562in}}%
\pgfpathlineto{\pgfqpoint{5.402048in}{2.754045in}}%
\pgfpathlineto{\pgfqpoint{5.415613in}{2.756642in}}%
\pgfpathlineto{\pgfqpoint{5.422726in}{2.765640in}}%
\pgfpathlineto{\pgfqpoint{5.429834in}{2.774632in}}%
\pgfpathlineto{\pgfqpoint{5.436938in}{2.783621in}}%
\pgfpathlineto{\pgfqpoint{5.444036in}{2.792608in}}%
\pgfpathlineto{\pgfqpoint{5.430484in}{2.790166in}}%
\pgfpathlineto{\pgfqpoint{5.416944in}{2.787836in}}%
\pgfpathlineto{\pgfqpoint{5.403415in}{2.785618in}}%
\pgfpathlineto{\pgfqpoint{5.389899in}{2.783514in}}%
\pgfpathlineto{\pgfqpoint{5.382788in}{2.774368in}}%
\pgfpathlineto{\pgfqpoint{5.375671in}{2.765223in}}%
\pgfpathlineto{\pgfqpoint{5.368551in}{2.756079in}}%
\pgfpathlineto{\pgfqpoint{5.361425in}{2.746934in}}%
\pgfpathclose%
\pgfusepath{fill}%
\end{pgfscope}%
\begin{pgfscope}%
\pgfpathrectangle{\pgfqpoint{1.254980in}{0.150000in}}{\pgfqpoint{5.490039in}{5.490039in}}%
\pgfusepath{clip}%
\pgfsetbuttcap%
\pgfsetroundjoin%
\definecolor{currentfill}{rgb}{0.208623,0.367752,0.552675}%
\pgfsetfillcolor{currentfill}%
\pgfsetfillopacity{0.700000}%
\pgfsetlinewidth{0.000000pt}%
\definecolor{currentstroke}{rgb}{0.000000,0.000000,0.000000}%
\pgfsetstrokecolor{currentstroke}%
\pgfsetdash{}{0pt}%
\pgfpathmoveto{\pgfqpoint{5.444036in}{2.792608in}}%
\pgfpathlineto{\pgfqpoint{5.457601in}{2.795163in}}%
\pgfpathlineto{\pgfqpoint{5.471178in}{2.797831in}}%
\pgfpathlineto{\pgfqpoint{5.484768in}{2.800611in}}%
\pgfpathlineto{\pgfqpoint{5.498370in}{2.803503in}}%
\pgfpathlineto{\pgfqpoint{5.505451in}{2.812325in}}%
\pgfpathlineto{\pgfqpoint{5.512527in}{2.821146in}}%
\pgfpathlineto{\pgfqpoint{5.519599in}{2.829965in}}%
\pgfpathlineto{\pgfqpoint{5.526666in}{2.838786in}}%
\pgfpathlineto{\pgfqpoint{5.513077in}{2.836064in}}%
\pgfpathlineto{\pgfqpoint{5.499501in}{2.833454in}}%
\pgfpathlineto{\pgfqpoint{5.485937in}{2.830957in}}%
\pgfpathlineto{\pgfqpoint{5.472386in}{2.828571in}}%
\pgfpathlineto{\pgfqpoint{5.465305in}{2.819575in}}%
\pgfpathlineto{\pgfqpoint{5.458220in}{2.810583in}}%
\pgfpathlineto{\pgfqpoint{5.451130in}{2.801595in}}%
\pgfpathlineto{\pgfqpoint{5.444036in}{2.792608in}}%
\pgfpathclose%
\pgfusepath{fill}%
\end{pgfscope}%
\begin{pgfscope}%
\pgfpathrectangle{\pgfqpoint{1.254980in}{0.150000in}}{\pgfqpoint{5.490039in}{5.490039in}}%
\pgfusepath{clip}%
\pgfsetbuttcap%
\pgfsetroundjoin%
\definecolor{currentfill}{rgb}{0.260571,0.246922,0.522828}%
\pgfsetfillcolor{currentfill}%
\pgfsetfillopacity{0.700000}%
\pgfsetlinewidth{0.000000pt}%
\definecolor{currentstroke}{rgb}{0.000000,0.000000,0.000000}%
\pgfsetstrokecolor{currentstroke}%
\pgfsetdash{}{0pt}%
\pgfpathmoveto{\pgfqpoint{4.948633in}{2.529876in}}%
\pgfpathlineto{\pgfqpoint{4.961994in}{2.530361in}}%
\pgfpathlineto{\pgfqpoint{4.975364in}{2.530962in}}%
\pgfpathlineto{\pgfqpoint{4.988744in}{2.531680in}}%
\pgfpathlineto{\pgfqpoint{5.002136in}{2.532515in}}%
\pgfpathlineto{\pgfqpoint{5.009399in}{2.542244in}}%
\pgfpathlineto{\pgfqpoint{5.016658in}{2.551963in}}%
\pgfpathlineto{\pgfqpoint{5.023912in}{2.561672in}}%
\pgfpathlineto{\pgfqpoint{5.031161in}{2.571372in}}%
\pgfpathlineto{\pgfqpoint{5.017780in}{2.570611in}}%
\pgfpathlineto{\pgfqpoint{5.004409in}{2.569965in}}%
\pgfpathlineto{\pgfqpoint{4.991049in}{2.569436in}}%
\pgfpathlineto{\pgfqpoint{4.977698in}{2.569023in}}%
\pgfpathlineto{\pgfqpoint{4.970439in}{2.559245in}}%
\pgfpathlineto{\pgfqpoint{4.963175in}{2.549461in}}%
\pgfpathlineto{\pgfqpoint{4.955907in}{2.539672in}}%
\pgfpathlineto{\pgfqpoint{4.948633in}{2.529876in}}%
\pgfpathclose%
\pgfusepath{fill}%
\end{pgfscope}%
\begin{pgfscope}%
\pgfpathrectangle{\pgfqpoint{1.254980in}{0.150000in}}{\pgfqpoint{5.490039in}{5.490039in}}%
\pgfusepath{clip}%
\pgfsetbuttcap%
\pgfsetroundjoin%
\definecolor{currentfill}{rgb}{0.199430,0.387607,0.554642}%
\pgfsetfillcolor{currentfill}%
\pgfsetfillopacity{0.700000}%
\pgfsetlinewidth{0.000000pt}%
\definecolor{currentstroke}{rgb}{0.000000,0.000000,0.000000}%
\pgfsetstrokecolor{currentstroke}%
\pgfsetdash{}{0pt}%
\pgfpathmoveto{\pgfqpoint{5.526666in}{2.838786in}}%
\pgfpathlineto{\pgfqpoint{5.540267in}{2.841620in}}%
\pgfpathlineto{\pgfqpoint{5.553881in}{2.844565in}}%
\pgfpathlineto{\pgfqpoint{5.567508in}{2.847622in}}%
\pgfpathlineto{\pgfqpoint{5.581148in}{2.850791in}}%
\pgfpathlineto{\pgfqpoint{5.588196in}{2.859434in}}%
\pgfpathlineto{\pgfqpoint{5.595240in}{2.868078in}}%
\pgfpathlineto{\pgfqpoint{5.602279in}{2.876725in}}%
\pgfpathlineto{\pgfqpoint{5.609314in}{2.885377in}}%
\pgfpathlineto{\pgfqpoint{5.595689in}{2.882395in}}%
\pgfpathlineto{\pgfqpoint{5.582076in}{2.879524in}}%
\pgfpathlineto{\pgfqpoint{5.568477in}{2.876765in}}%
\pgfpathlineto{\pgfqpoint{5.554889in}{2.874117in}}%
\pgfpathlineto{\pgfqpoint{5.547840in}{2.865273in}}%
\pgfpathlineto{\pgfqpoint{5.540787in}{2.856438in}}%
\pgfpathlineto{\pgfqpoint{5.533728in}{2.847610in}}%
\pgfpathlineto{\pgfqpoint{5.526666in}{2.838786in}}%
\pgfpathclose%
\pgfusepath{fill}%
\end{pgfscope}%
\begin{pgfscope}%
\pgfpathrectangle{\pgfqpoint{1.254980in}{0.150000in}}{\pgfqpoint{5.490039in}{5.490039in}}%
\pgfusepath{clip}%
\pgfsetbuttcap%
\pgfsetroundjoin%
\definecolor{currentfill}{rgb}{0.282884,0.135920,0.453427}%
\pgfsetfillcolor{currentfill}%
\pgfsetfillopacity{0.700000}%
\pgfsetlinewidth{0.000000pt}%
\definecolor{currentstroke}{rgb}{0.000000,0.000000,0.000000}%
\pgfsetstrokecolor{currentstroke}%
\pgfsetdash{}{0pt}%
\pgfpathmoveto{\pgfqpoint{3.695739in}{2.324231in}}%
\pgfpathlineto{\pgfqpoint{3.708786in}{2.315662in}}%
\pgfpathlineto{\pgfqpoint{3.721836in}{2.307235in}}%
\pgfpathlineto{\pgfqpoint{3.734888in}{2.298951in}}%
\pgfpathlineto{\pgfqpoint{3.747942in}{2.290808in}}%
\pgfpathlineto{\pgfqpoint{3.755617in}{2.299340in}}%
\pgfpathlineto{\pgfqpoint{3.763286in}{2.307925in}}%
\pgfpathlineto{\pgfqpoint{3.770950in}{2.316562in}}%
\pgfpathlineto{\pgfqpoint{3.778608in}{2.325252in}}%
\pgfpathlineto{\pgfqpoint{3.765568in}{2.333275in}}%
\pgfpathlineto{\pgfqpoint{3.752531in}{2.341439in}}%
\pgfpathlineto{\pgfqpoint{3.739496in}{2.349746in}}%
\pgfpathlineto{\pgfqpoint{3.726464in}{2.358195in}}%
\pgfpathlineto{\pgfqpoint{3.718791in}{2.349619in}}%
\pgfpathlineto{\pgfqpoint{3.711113in}{2.341100in}}%
\pgfpathlineto{\pgfqpoint{3.703429in}{2.332637in}}%
\pgfpathlineto{\pgfqpoint{3.695739in}{2.324231in}}%
\pgfpathclose%
\pgfusepath{fill}%
\end{pgfscope}%
\begin{pgfscope}%
\pgfpathrectangle{\pgfqpoint{1.254980in}{0.150000in}}{\pgfqpoint{5.490039in}{5.490039in}}%
\pgfusepath{clip}%
\pgfsetbuttcap%
\pgfsetroundjoin%
\definecolor{currentfill}{rgb}{0.266580,0.228262,0.514349}%
\pgfsetfillcolor{currentfill}%
\pgfsetfillopacity{0.700000}%
\pgfsetlinewidth{0.000000pt}%
\definecolor{currentstroke}{rgb}{0.000000,0.000000,0.000000}%
\pgfsetstrokecolor{currentstroke}%
\pgfsetdash{}{0pt}%
\pgfpathmoveto{\pgfqpoint{4.866116in}{2.489613in}}%
\pgfpathlineto{\pgfqpoint{4.879445in}{2.489685in}}%
\pgfpathlineto{\pgfqpoint{4.892785in}{2.489875in}}%
\pgfpathlineto{\pgfqpoint{4.906134in}{2.490183in}}%
\pgfpathlineto{\pgfqpoint{4.919493in}{2.490607in}}%
\pgfpathlineto{\pgfqpoint{4.926786in}{2.500439in}}%
\pgfpathlineto{\pgfqpoint{4.934073in}{2.510260in}}%
\pgfpathlineto{\pgfqpoint{4.941356in}{2.520072in}}%
\pgfpathlineto{\pgfqpoint{4.948633in}{2.529876in}}%
\pgfpathlineto{\pgfqpoint{4.935283in}{2.529508in}}%
\pgfpathlineto{\pgfqpoint{4.921944in}{2.529257in}}%
\pgfpathlineto{\pgfqpoint{4.908614in}{2.529123in}}%
\pgfpathlineto{\pgfqpoint{4.895294in}{2.529107in}}%
\pgfpathlineto{\pgfqpoint{4.888006in}{2.519241in}}%
\pgfpathlineto{\pgfqpoint{4.880714in}{2.509371in}}%
\pgfpathlineto{\pgfqpoint{4.873417in}{2.499495in}}%
\pgfpathlineto{\pgfqpoint{4.866116in}{2.489613in}}%
\pgfpathclose%
\pgfusepath{fill}%
\end{pgfscope}%
\begin{pgfscope}%
\pgfpathrectangle{\pgfqpoint{1.254980in}{0.150000in}}{\pgfqpoint{5.490039in}{5.490039in}}%
\pgfusepath{clip}%
\pgfsetbuttcap%
\pgfsetroundjoin%
\definecolor{currentfill}{rgb}{0.278826,0.175490,0.483397}%
\pgfsetfillcolor{currentfill}%
\pgfsetfillopacity{0.700000}%
\pgfsetlinewidth{0.000000pt}%
\definecolor{currentstroke}{rgb}{0.000000,0.000000,0.000000}%
\pgfsetstrokecolor{currentstroke}%
\pgfsetdash{}{0pt}%
\pgfpathmoveto{\pgfqpoint{3.508299in}{2.407035in}}%
\pgfpathlineto{\pgfqpoint{3.521344in}{2.396565in}}%
\pgfpathlineto{\pgfqpoint{3.534390in}{2.386246in}}%
\pgfpathlineto{\pgfqpoint{3.547437in}{2.376077in}}%
\pgfpathlineto{\pgfqpoint{3.560485in}{2.366058in}}%
\pgfpathlineto{\pgfqpoint{3.568232in}{2.373941in}}%
\pgfpathlineto{\pgfqpoint{3.575971in}{2.381893in}}%
\pgfpathlineto{\pgfqpoint{3.583705in}{2.389912in}}%
\pgfpathlineto{\pgfqpoint{3.591433in}{2.397999in}}%
\pgfpathlineto{\pgfqpoint{3.578401in}{2.407881in}}%
\pgfpathlineto{\pgfqpoint{3.565371in}{2.417912in}}%
\pgfpathlineto{\pgfqpoint{3.552342in}{2.428094in}}%
\pgfpathlineto{\pgfqpoint{3.539314in}{2.438426in}}%
\pgfpathlineto{\pgfqpoint{3.531570in}{2.430471in}}%
\pgfpathlineto{\pgfqpoint{3.523819in}{2.422587in}}%
\pgfpathlineto{\pgfqpoint{3.516062in}{2.414775in}}%
\pgfpathlineto{\pgfqpoint{3.508299in}{2.407035in}}%
\pgfpathclose%
\pgfusepath{fill}%
\end{pgfscope}%
\begin{pgfscope}%
\pgfpathrectangle{\pgfqpoint{1.254980in}{0.150000in}}{\pgfqpoint{5.490039in}{5.490039in}}%
\pgfusepath{clip}%
\pgfsetbuttcap%
\pgfsetroundjoin%
\definecolor{currentfill}{rgb}{0.190631,0.407061,0.556089}%
\pgfsetfillcolor{currentfill}%
\pgfsetfillopacity{0.700000}%
\pgfsetlinewidth{0.000000pt}%
\definecolor{currentstroke}{rgb}{0.000000,0.000000,0.000000}%
\pgfsetstrokecolor{currentstroke}%
\pgfsetdash{}{0pt}%
\pgfpathmoveto{\pgfqpoint{5.609314in}{2.885377in}}%
\pgfpathlineto{\pgfqpoint{5.622952in}{2.888471in}}%
\pgfpathlineto{\pgfqpoint{5.636603in}{2.891675in}}%
\pgfpathlineto{\pgfqpoint{5.650268in}{2.894991in}}%
\pgfpathlineto{\pgfqpoint{5.663945in}{2.898418in}}%
\pgfpathlineto{\pgfqpoint{5.670961in}{2.906880in}}%
\pgfpathlineto{\pgfqpoint{5.677972in}{2.915346in}}%
\pgfpathlineto{\pgfqpoint{5.684978in}{2.923820in}}%
\pgfpathlineto{\pgfqpoint{5.691980in}{2.932302in}}%
\pgfpathlineto{\pgfqpoint{5.678318in}{2.929078in}}%
\pgfpathlineto{\pgfqpoint{5.664669in}{2.925965in}}%
\pgfpathlineto{\pgfqpoint{5.651033in}{2.922963in}}%
\pgfpathlineto{\pgfqpoint{5.637410in}{2.920072in}}%
\pgfpathlineto{\pgfqpoint{5.630392in}{2.911381in}}%
\pgfpathlineto{\pgfqpoint{5.623370in}{2.902703in}}%
\pgfpathlineto{\pgfqpoint{5.616344in}{2.894036in}}%
\pgfpathlineto{\pgfqpoint{5.609314in}{2.885377in}}%
\pgfpathclose%
\pgfusepath{fill}%
\end{pgfscope}%
\begin{pgfscope}%
\pgfpathrectangle{\pgfqpoint{1.254980in}{0.150000in}}{\pgfqpoint{5.490039in}{5.490039in}}%
\pgfusepath{clip}%
\pgfsetbuttcap%
\pgfsetroundjoin%
\definecolor{currentfill}{rgb}{0.227802,0.326594,0.546532}%
\pgfsetfillcolor{currentfill}%
\pgfsetfillopacity{0.700000}%
\pgfsetlinewidth{0.000000pt}%
\definecolor{currentstroke}{rgb}{0.000000,0.000000,0.000000}%
\pgfsetstrokecolor{currentstroke}%
\pgfsetdash{}{0pt}%
\pgfpathmoveto{\pgfqpoint{3.111124in}{2.738790in}}%
\pgfpathlineto{\pgfqpoint{3.124224in}{2.723607in}}%
\pgfpathlineto{\pgfqpoint{3.137321in}{2.708600in}}%
\pgfpathlineto{\pgfqpoint{3.150415in}{2.693767in}}%
\pgfpathlineto{\pgfqpoint{3.163506in}{2.679107in}}%
\pgfpathlineto{\pgfqpoint{3.171418in}{2.685572in}}%
\pgfpathlineto{\pgfqpoint{3.179321in}{2.692139in}}%
\pgfpathlineto{\pgfqpoint{3.187217in}{2.698806in}}%
\pgfpathlineto{\pgfqpoint{3.195105in}{2.705571in}}%
\pgfpathlineto{\pgfqpoint{3.182036in}{2.720072in}}%
\pgfpathlineto{\pgfqpoint{3.168964in}{2.734746in}}%
\pgfpathlineto{\pgfqpoint{3.155889in}{2.749594in}}%
\pgfpathlineto{\pgfqpoint{3.142812in}{2.764617in}}%
\pgfpathlineto{\pgfqpoint{3.134902in}{2.758005in}}%
\pgfpathlineto{\pgfqpoint{3.126984in}{2.751495in}}%
\pgfpathlineto{\pgfqpoint{3.119058in}{2.745090in}}%
\pgfpathlineto{\pgfqpoint{3.111124in}{2.738790in}}%
\pgfpathclose%
\pgfusepath{fill}%
\end{pgfscope}%
\begin{pgfscope}%
\pgfpathrectangle{\pgfqpoint{1.254980in}{0.150000in}}{\pgfqpoint{5.490039in}{5.490039in}}%
\pgfusepath{clip}%
\pgfsetbuttcap%
\pgfsetroundjoin%
\definecolor{currentfill}{rgb}{0.283091,0.110553,0.431554}%
\pgfsetfillcolor{currentfill}%
\pgfsetfillopacity{0.700000}%
\pgfsetlinewidth{0.000000pt}%
\definecolor{currentstroke}{rgb}{0.000000,0.000000,0.000000}%
\pgfsetstrokecolor{currentstroke}%
\pgfsetdash{}{0pt}%
\pgfpathmoveto{\pgfqpoint{4.100719in}{2.268769in}}%
\pgfpathlineto{\pgfqpoint{4.113819in}{2.263783in}}%
\pgfpathlineto{\pgfqpoint{4.126925in}{2.258927in}}%
\pgfpathlineto{\pgfqpoint{4.140036in}{2.254200in}}%
\pgfpathlineto{\pgfqpoint{4.153153in}{2.249604in}}%
\pgfpathlineto{\pgfqpoint{4.160690in}{2.259197in}}%
\pgfpathlineto{\pgfqpoint{4.168222in}{2.268814in}}%
\pgfpathlineto{\pgfqpoint{4.175749in}{2.278452in}}%
\pgfpathlineto{\pgfqpoint{4.183271in}{2.288113in}}%
\pgfpathlineto{\pgfqpoint{4.170165in}{2.292639in}}%
\pgfpathlineto{\pgfqpoint{4.157065in}{2.297294in}}%
\pgfpathlineto{\pgfqpoint{4.143970in}{2.302079in}}%
\pgfpathlineto{\pgfqpoint{4.130880in}{2.306994in}}%
\pgfpathlineto{\pgfqpoint{4.123347in}{2.297398in}}%
\pgfpathlineto{\pgfqpoint{4.115810in}{2.287829in}}%
\pgfpathlineto{\pgfqpoint{4.108267in}{2.278286in}}%
\pgfpathlineto{\pgfqpoint{4.100719in}{2.268769in}}%
\pgfpathclose%
\pgfusepath{fill}%
\end{pgfscope}%
\begin{pgfscope}%
\pgfpathrectangle{\pgfqpoint{1.254980in}{0.150000in}}{\pgfqpoint{5.490039in}{5.490039in}}%
\pgfusepath{clip}%
\pgfsetbuttcap%
\pgfsetroundjoin%
\definecolor{currentfill}{rgb}{0.216210,0.351535,0.550627}%
\pgfsetfillcolor{currentfill}%
\pgfsetfillopacity{0.700000}%
\pgfsetlinewidth{0.000000pt}%
\definecolor{currentstroke}{rgb}{0.000000,0.000000,0.000000}%
\pgfsetstrokecolor{currentstroke}%
\pgfsetdash{}{0pt}%
\pgfpathmoveto{\pgfqpoint{3.058687in}{2.801296in}}%
\pgfpathlineto{\pgfqpoint{3.071802in}{2.785401in}}%
\pgfpathlineto{\pgfqpoint{3.084913in}{2.769686in}}%
\pgfpathlineto{\pgfqpoint{3.098020in}{2.754149in}}%
\pgfpathlineto{\pgfqpoint{3.111124in}{2.738790in}}%
\pgfpathlineto{\pgfqpoint{3.119058in}{2.745090in}}%
\pgfpathlineto{\pgfqpoint{3.126984in}{2.751495in}}%
\pgfpathlineto{\pgfqpoint{3.134902in}{2.758005in}}%
\pgfpathlineto{\pgfqpoint{3.142812in}{2.764617in}}%
\pgfpathlineto{\pgfqpoint{3.129731in}{2.779817in}}%
\pgfpathlineto{\pgfqpoint{3.116647in}{2.795193in}}%
\pgfpathlineto{\pgfqpoint{3.103559in}{2.810748in}}%
\pgfpathlineto{\pgfqpoint{3.090468in}{2.826482in}}%
\pgfpathlineto{\pgfqpoint{3.082535in}{2.820024in}}%
\pgfpathlineto{\pgfqpoint{3.074594in}{2.813672in}}%
\pgfpathlineto{\pgfqpoint{3.066645in}{2.807430in}}%
\pgfpathlineto{\pgfqpoint{3.058687in}{2.801296in}}%
\pgfpathclose%
\pgfusepath{fill}%
\end{pgfscope}%
\begin{pgfscope}%
\pgfpathrectangle{\pgfqpoint{1.254980in}{0.150000in}}{\pgfqpoint{5.490039in}{5.490039in}}%
\pgfusepath{clip}%
\pgfsetbuttcap%
\pgfsetroundjoin%
\definecolor{currentfill}{rgb}{0.183898,0.422383,0.556944}%
\pgfsetfillcolor{currentfill}%
\pgfsetfillopacity{0.700000}%
\pgfsetlinewidth{0.000000pt}%
\definecolor{currentstroke}{rgb}{0.000000,0.000000,0.000000}%
\pgfsetstrokecolor{currentstroke}%
\pgfsetdash{}{0pt}%
\pgfpathmoveto{\pgfqpoint{5.691980in}{2.932302in}}%
\pgfpathlineto{\pgfqpoint{5.705655in}{2.935637in}}%
\pgfpathlineto{\pgfqpoint{5.719344in}{2.939082in}}%
\pgfpathlineto{\pgfqpoint{5.733046in}{2.942638in}}%
\pgfpathlineto{\pgfqpoint{5.746762in}{2.946304in}}%
\pgfpathlineto{\pgfqpoint{5.753744in}{2.954584in}}%
\pgfpathlineto{\pgfqpoint{5.760721in}{2.962874in}}%
\pgfpathlineto{\pgfqpoint{5.767694in}{2.971175in}}%
\pgfpathlineto{\pgfqpoint{5.774663in}{2.979490in}}%
\pgfpathlineto{\pgfqpoint{5.760964in}{2.976044in}}%
\pgfpathlineto{\pgfqpoint{5.747278in}{2.972707in}}%
\pgfpathlineto{\pgfqpoint{5.733606in}{2.969481in}}%
\pgfpathlineto{\pgfqpoint{5.719946in}{2.966365in}}%
\pgfpathlineto{\pgfqpoint{5.712961in}{2.957825in}}%
\pgfpathlineto{\pgfqpoint{5.705972in}{2.949303in}}%
\pgfpathlineto{\pgfqpoint{5.698978in}{2.940796in}}%
\pgfpathlineto{\pgfqpoint{5.691980in}{2.932302in}}%
\pgfpathclose%
\pgfusepath{fill}%
\end{pgfscope}%
\begin{pgfscope}%
\pgfpathrectangle{\pgfqpoint{1.254980in}{0.150000in}}{\pgfqpoint{5.490039in}{5.490039in}}%
\pgfusepath{clip}%
\pgfsetbuttcap%
\pgfsetroundjoin%
\definecolor{currentfill}{rgb}{0.239346,0.300855,0.540844}%
\pgfsetfillcolor{currentfill}%
\pgfsetfillopacity{0.700000}%
\pgfsetlinewidth{0.000000pt}%
\definecolor{currentstroke}{rgb}{0.000000,0.000000,0.000000}%
\pgfsetstrokecolor{currentstroke}%
\pgfsetdash{}{0pt}%
\pgfpathmoveto{\pgfqpoint{3.163506in}{2.679107in}}%
\pgfpathlineto{\pgfqpoint{3.176594in}{2.664619in}}%
\pgfpathlineto{\pgfqpoint{3.189680in}{2.650303in}}%
\pgfpathlineto{\pgfqpoint{3.202763in}{2.636157in}}%
\pgfpathlineto{\pgfqpoint{3.215844in}{2.622181in}}%
\pgfpathlineto{\pgfqpoint{3.223733in}{2.628810in}}%
\pgfpathlineto{\pgfqpoint{3.231615in}{2.635537in}}%
\pgfpathlineto{\pgfqpoint{3.239489in}{2.642360in}}%
\pgfpathlineto{\pgfqpoint{3.247356in}{2.649278in}}%
\pgfpathlineto{\pgfqpoint{3.234296in}{2.663097in}}%
\pgfpathlineto{\pgfqpoint{3.221235in}{2.677084in}}%
\pgfpathlineto{\pgfqpoint{3.208171in}{2.691242in}}%
\pgfpathlineto{\pgfqpoint{3.195105in}{2.705571in}}%
\pgfpathlineto{\pgfqpoint{3.187217in}{2.698806in}}%
\pgfpathlineto{\pgfqpoint{3.179321in}{2.692139in}}%
\pgfpathlineto{\pgfqpoint{3.171418in}{2.685572in}}%
\pgfpathlineto{\pgfqpoint{3.163506in}{2.679107in}}%
\pgfpathclose%
\pgfusepath{fill}%
\end{pgfscope}%
\begin{pgfscope}%
\pgfpathrectangle{\pgfqpoint{1.254980in}{0.150000in}}{\pgfqpoint{5.490039in}{5.490039in}}%
\pgfusepath{clip}%
\pgfsetbuttcap%
\pgfsetroundjoin%
\definecolor{currentfill}{rgb}{0.283072,0.130895,0.449241}%
\pgfsetfillcolor{currentfill}%
\pgfsetfillopacity{0.700000}%
\pgfsetlinewidth{0.000000pt}%
\definecolor{currentstroke}{rgb}{0.000000,0.000000,0.000000}%
\pgfsetstrokecolor{currentstroke}%
\pgfsetdash{}{0pt}%
\pgfpathmoveto{\pgfqpoint{4.318296in}{2.295819in}}%
\pgfpathlineto{\pgfqpoint{4.331448in}{2.292495in}}%
\pgfpathlineto{\pgfqpoint{4.344607in}{2.289297in}}%
\pgfpathlineto{\pgfqpoint{4.357773in}{2.286224in}}%
\pgfpathlineto{\pgfqpoint{4.370946in}{2.283275in}}%
\pgfpathlineto{\pgfqpoint{4.378414in}{2.293179in}}%
\pgfpathlineto{\pgfqpoint{4.385878in}{2.303093in}}%
\pgfpathlineto{\pgfqpoint{4.393336in}{2.313015in}}%
\pgfpathlineto{\pgfqpoint{4.400791in}{2.322947in}}%
\pgfpathlineto{\pgfqpoint{4.387627in}{2.325856in}}%
\pgfpathlineto{\pgfqpoint{4.374471in}{2.328890in}}%
\pgfpathlineto{\pgfqpoint{4.361322in}{2.332049in}}%
\pgfpathlineto{\pgfqpoint{4.348180in}{2.335333in}}%
\pgfpathlineto{\pgfqpoint{4.340716in}{2.325435in}}%
\pgfpathlineto{\pgfqpoint{4.333247in}{2.315550in}}%
\pgfpathlineto{\pgfqpoint{4.325774in}{2.305678in}}%
\pgfpathlineto{\pgfqpoint{4.318296in}{2.295819in}}%
\pgfpathclose%
\pgfusepath{fill}%
\end{pgfscope}%
\begin{pgfscope}%
\pgfpathrectangle{\pgfqpoint{1.254980in}{0.150000in}}{\pgfqpoint{5.490039in}{5.490039in}}%
\pgfusepath{clip}%
\pgfsetbuttcap%
\pgfsetroundjoin%
\definecolor{currentfill}{rgb}{0.271828,0.209303,0.504434}%
\pgfsetfillcolor{currentfill}%
\pgfsetfillopacity{0.700000}%
\pgfsetlinewidth{0.000000pt}%
\definecolor{currentstroke}{rgb}{0.000000,0.000000,0.000000}%
\pgfsetstrokecolor{currentstroke}%
\pgfsetdash{}{0pt}%
\pgfpathmoveto{\pgfqpoint{4.783605in}{2.450739in}}%
\pgfpathlineto{\pgfqpoint{4.796905in}{2.450379in}}%
\pgfpathlineto{\pgfqpoint{4.810215in}{2.450138in}}%
\pgfpathlineto{\pgfqpoint{4.823534in}{2.450014in}}%
\pgfpathlineto{\pgfqpoint{4.836863in}{2.450009in}}%
\pgfpathlineto{\pgfqpoint{4.844184in}{2.459923in}}%
\pgfpathlineto{\pgfqpoint{4.851499in}{2.469828in}}%
\pgfpathlineto{\pgfqpoint{4.858810in}{2.479724in}}%
\pgfpathlineto{\pgfqpoint{4.866116in}{2.489613in}}%
\pgfpathlineto{\pgfqpoint{4.852796in}{2.489659in}}%
\pgfpathlineto{\pgfqpoint{4.839486in}{2.489822in}}%
\pgfpathlineto{\pgfqpoint{4.826185in}{2.490104in}}%
\pgfpathlineto{\pgfqpoint{4.812894in}{2.490504in}}%
\pgfpathlineto{\pgfqpoint{4.805579in}{2.480569in}}%
\pgfpathlineto{\pgfqpoint{4.798259in}{2.470630in}}%
\pgfpathlineto{\pgfqpoint{4.790934in}{2.460687in}}%
\pgfpathlineto{\pgfqpoint{4.783605in}{2.450739in}}%
\pgfpathclose%
\pgfusepath{fill}%
\end{pgfscope}%
\begin{pgfscope}%
\pgfpathrectangle{\pgfqpoint{1.254980in}{0.150000in}}{\pgfqpoint{5.490039in}{5.490039in}}%
\pgfusepath{clip}%
\pgfsetbuttcap%
\pgfsetroundjoin%
\definecolor{currentfill}{rgb}{0.203063,0.379716,0.553925}%
\pgfsetfillcolor{currentfill}%
\pgfsetfillopacity{0.700000}%
\pgfsetlinewidth{0.000000pt}%
\definecolor{currentstroke}{rgb}{0.000000,0.000000,0.000000}%
\pgfsetstrokecolor{currentstroke}%
\pgfsetdash{}{0pt}%
\pgfpathmoveto{\pgfqpoint{3.006186in}{2.866695in}}%
\pgfpathlineto{\pgfqpoint{3.019318in}{2.850070in}}%
\pgfpathlineto{\pgfqpoint{3.032445in}{2.833629in}}%
\pgfpathlineto{\pgfqpoint{3.045568in}{2.817372in}}%
\pgfpathlineto{\pgfqpoint{3.058687in}{2.801296in}}%
\pgfpathlineto{\pgfqpoint{3.066645in}{2.807430in}}%
\pgfpathlineto{\pgfqpoint{3.074594in}{2.813672in}}%
\pgfpathlineto{\pgfqpoint{3.082535in}{2.820024in}}%
\pgfpathlineto{\pgfqpoint{3.090468in}{2.826482in}}%
\pgfpathlineto{\pgfqpoint{3.077372in}{2.842397in}}%
\pgfpathlineto{\pgfqpoint{3.064273in}{2.858493in}}%
\pgfpathlineto{\pgfqpoint{3.051170in}{2.874772in}}%
\pgfpathlineto{\pgfqpoint{3.038062in}{2.891235in}}%
\pgfpathlineto{\pgfqpoint{3.030106in}{2.884932in}}%
\pgfpathlineto{\pgfqpoint{3.022141in}{2.878740in}}%
\pgfpathlineto{\pgfqpoint{3.014168in}{2.872660in}}%
\pgfpathlineto{\pgfqpoint{3.006186in}{2.866695in}}%
\pgfpathclose%
\pgfusepath{fill}%
\end{pgfscope}%
\begin{pgfscope}%
\pgfpathrectangle{\pgfqpoint{1.254980in}{0.150000in}}{\pgfqpoint{5.490039in}{5.490039in}}%
\pgfusepath{clip}%
\pgfsetbuttcap%
\pgfsetroundjoin%
\definecolor{currentfill}{rgb}{0.175841,0.441290,0.557685}%
\pgfsetfillcolor{currentfill}%
\pgfsetfillopacity{0.700000}%
\pgfsetlinewidth{0.000000pt}%
\definecolor{currentstroke}{rgb}{0.000000,0.000000,0.000000}%
\pgfsetstrokecolor{currentstroke}%
\pgfsetdash{}{0pt}%
\pgfpathmoveto{\pgfqpoint{5.774663in}{2.979490in}}%
\pgfpathlineto{\pgfqpoint{5.788376in}{2.983047in}}%
\pgfpathlineto{\pgfqpoint{5.802102in}{2.986715in}}%
\pgfpathlineto{\pgfqpoint{5.815842in}{2.990492in}}%
\pgfpathlineto{\pgfqpoint{5.829596in}{2.994379in}}%
\pgfpathlineto{\pgfqpoint{5.836544in}{3.002480in}}%
\pgfpathlineto{\pgfqpoint{5.843488in}{3.010596in}}%
\pgfpathlineto{\pgfqpoint{5.850427in}{3.018729in}}%
\pgfpathlineto{\pgfqpoint{5.857363in}{3.026881in}}%
\pgfpathlineto{\pgfqpoint{5.843626in}{3.023230in}}%
\pgfpathlineto{\pgfqpoint{5.829904in}{3.019688in}}%
\pgfpathlineto{\pgfqpoint{5.816195in}{3.016257in}}%
\pgfpathlineto{\pgfqpoint{5.802499in}{3.012935in}}%
\pgfpathlineto{\pgfqpoint{5.795546in}{3.004541in}}%
\pgfpathlineto{\pgfqpoint{5.788589in}{2.996171in}}%
\pgfpathlineto{\pgfqpoint{5.781628in}{2.987822in}}%
\pgfpathlineto{\pgfqpoint{5.774663in}{2.979490in}}%
\pgfpathclose%
\pgfusepath{fill}%
\end{pgfscope}%
\begin{pgfscope}%
\pgfpathrectangle{\pgfqpoint{1.254980in}{0.150000in}}{\pgfqpoint{5.490039in}{5.490039in}}%
\pgfusepath{clip}%
\pgfsetbuttcap%
\pgfsetroundjoin%
\definecolor{currentfill}{rgb}{0.248629,0.278775,0.534556}%
\pgfsetfillcolor{currentfill}%
\pgfsetfillopacity{0.700000}%
\pgfsetlinewidth{0.000000pt}%
\definecolor{currentstroke}{rgb}{0.000000,0.000000,0.000000}%
\pgfsetstrokecolor{currentstroke}%
\pgfsetdash{}{0pt}%
\pgfpathmoveto{\pgfqpoint{3.215844in}{2.622181in}}%
\pgfpathlineto{\pgfqpoint{3.228922in}{2.608372in}}%
\pgfpathlineto{\pgfqpoint{3.241999in}{2.594732in}}%
\pgfpathlineto{\pgfqpoint{3.255073in}{2.581258in}}%
\pgfpathlineto{\pgfqpoint{3.268146in}{2.567949in}}%
\pgfpathlineto{\pgfqpoint{3.276014in}{2.574742in}}%
\pgfpathlineto{\pgfqpoint{3.283875in}{2.581628in}}%
\pgfpathlineto{\pgfqpoint{3.291728in}{2.588607in}}%
\pgfpathlineto{\pgfqpoint{3.299574in}{2.595676in}}%
\pgfpathlineto{\pgfqpoint{3.286522in}{2.608827in}}%
\pgfpathlineto{\pgfqpoint{3.273468in}{2.622144in}}%
\pgfpathlineto{\pgfqpoint{3.260413in}{2.635628in}}%
\pgfpathlineto{\pgfqpoint{3.247356in}{2.649278in}}%
\pgfpathlineto{\pgfqpoint{3.239489in}{2.642360in}}%
\pgfpathlineto{\pgfqpoint{3.231615in}{2.635537in}}%
\pgfpathlineto{\pgfqpoint{3.223733in}{2.628810in}}%
\pgfpathlineto{\pgfqpoint{3.215844in}{2.622181in}}%
\pgfpathclose%
\pgfusepath{fill}%
\end{pgfscope}%
\begin{pgfscope}%
\pgfpathrectangle{\pgfqpoint{1.254980in}{0.150000in}}{\pgfqpoint{5.490039in}{5.490039in}}%
\pgfusepath{clip}%
\pgfsetbuttcap%
\pgfsetroundjoin%
\definecolor{currentfill}{rgb}{0.275191,0.194905,0.496005}%
\pgfsetfillcolor{currentfill}%
\pgfsetfillopacity{0.700000}%
\pgfsetlinewidth{0.000000pt}%
\definecolor{currentstroke}{rgb}{0.000000,0.000000,0.000000}%
\pgfsetstrokecolor{currentstroke}%
\pgfsetdash{}{0pt}%
\pgfpathmoveto{\pgfqpoint{4.701096in}{2.413417in}}%
\pgfpathlineto{\pgfqpoint{4.714368in}{2.412605in}}%
\pgfpathlineto{\pgfqpoint{4.727650in}{2.411913in}}%
\pgfpathlineto{\pgfqpoint{4.740941in}{2.411339in}}%
\pgfpathlineto{\pgfqpoint{4.754241in}{2.410885in}}%
\pgfpathlineto{\pgfqpoint{4.761589in}{2.420859in}}%
\pgfpathlineto{\pgfqpoint{4.768932in}{2.430825in}}%
\pgfpathlineto{\pgfqpoint{4.776271in}{2.440785in}}%
\pgfpathlineto{\pgfqpoint{4.783605in}{2.450739in}}%
\pgfpathlineto{\pgfqpoint{4.770314in}{2.451218in}}%
\pgfpathlineto{\pgfqpoint{4.757032in}{2.451815in}}%
\pgfpathlineto{\pgfqpoint{4.743760in}{2.452532in}}%
\pgfpathlineto{\pgfqpoint{4.730496in}{2.453368in}}%
\pgfpathlineto{\pgfqpoint{4.723153in}{2.443384in}}%
\pgfpathlineto{\pgfqpoint{4.715805in}{2.433398in}}%
\pgfpathlineto{\pgfqpoint{4.708453in}{2.423409in}}%
\pgfpathlineto{\pgfqpoint{4.701096in}{2.413417in}}%
\pgfpathclose%
\pgfusepath{fill}%
\end{pgfscope}%
\begin{pgfscope}%
\pgfpathrectangle{\pgfqpoint{1.254980in}{0.150000in}}{\pgfqpoint{5.490039in}{5.490039in}}%
\pgfusepath{clip}%
\pgfsetbuttcap%
\pgfsetroundjoin%
\definecolor{currentfill}{rgb}{0.190631,0.407061,0.556089}%
\pgfsetfillcolor{currentfill}%
\pgfsetfillopacity{0.700000}%
\pgfsetlinewidth{0.000000pt}%
\definecolor{currentstroke}{rgb}{0.000000,0.000000,0.000000}%
\pgfsetstrokecolor{currentstroke}%
\pgfsetdash{}{0pt}%
\pgfpathmoveto{\pgfqpoint{2.953611in}{2.935059in}}%
\pgfpathlineto{\pgfqpoint{2.966762in}{2.917685in}}%
\pgfpathlineto{\pgfqpoint{2.979908in}{2.900501in}}%
\pgfpathlineto{\pgfqpoint{2.993050in}{2.883505in}}%
\pgfpathlineto{\pgfqpoint{3.006186in}{2.866695in}}%
\pgfpathlineto{\pgfqpoint{3.014168in}{2.872660in}}%
\pgfpathlineto{\pgfqpoint{3.022141in}{2.878740in}}%
\pgfpathlineto{\pgfqpoint{3.030106in}{2.884932in}}%
\pgfpathlineto{\pgfqpoint{3.038062in}{2.891235in}}%
\pgfpathlineto{\pgfqpoint{3.024950in}{2.907883in}}%
\pgfpathlineto{\pgfqpoint{3.011833in}{2.924717in}}%
\pgfpathlineto{\pgfqpoint{2.998712in}{2.941738in}}%
\pgfpathlineto{\pgfqpoint{2.985585in}{2.958948in}}%
\pgfpathlineto{\pgfqpoint{2.977605in}{2.952801in}}%
\pgfpathlineto{\pgfqpoint{2.969616in}{2.946770in}}%
\pgfpathlineto{\pgfqpoint{2.961618in}{2.940855in}}%
\pgfpathlineto{\pgfqpoint{2.953611in}{2.935059in}}%
\pgfpathclose%
\pgfusepath{fill}%
\end{pgfscope}%
\begin{pgfscope}%
\pgfpathrectangle{\pgfqpoint{1.254980in}{0.150000in}}{\pgfqpoint{5.490039in}{5.490039in}}%
\pgfusepath{clip}%
\pgfsetbuttcap%
\pgfsetroundjoin%
\definecolor{currentfill}{rgb}{0.257322,0.256130,0.526563}%
\pgfsetfillcolor{currentfill}%
\pgfsetfillopacity{0.700000}%
\pgfsetlinewidth{0.000000pt}%
\definecolor{currentstroke}{rgb}{0.000000,0.000000,0.000000}%
\pgfsetstrokecolor{currentstroke}%
\pgfsetdash{}{0pt}%
\pgfpathmoveto{\pgfqpoint{3.268146in}{2.567949in}}%
\pgfpathlineto{\pgfqpoint{3.281217in}{2.554805in}}%
\pgfpathlineto{\pgfqpoint{3.294287in}{2.541825in}}%
\pgfpathlineto{\pgfqpoint{3.307355in}{2.529007in}}%
\pgfpathlineto{\pgfqpoint{3.320422in}{2.516352in}}%
\pgfpathlineto{\pgfqpoint{3.328270in}{2.523308in}}%
\pgfpathlineto{\pgfqpoint{3.336110in}{2.530353in}}%
\pgfpathlineto{\pgfqpoint{3.343943in}{2.537485in}}%
\pgfpathlineto{\pgfqpoint{3.351769in}{2.544705in}}%
\pgfpathlineto{\pgfqpoint{3.338722in}{2.557204in}}%
\pgfpathlineto{\pgfqpoint{3.325674in}{2.569865in}}%
\pgfpathlineto{\pgfqpoint{3.312625in}{2.582688in}}%
\pgfpathlineto{\pgfqpoint{3.299574in}{2.595676in}}%
\pgfpathlineto{\pgfqpoint{3.291728in}{2.588607in}}%
\pgfpathlineto{\pgfqpoint{3.283875in}{2.581628in}}%
\pgfpathlineto{\pgfqpoint{3.276014in}{2.574742in}}%
\pgfpathlineto{\pgfqpoint{3.268146in}{2.567949in}}%
\pgfpathclose%
\pgfusepath{fill}%
\end{pgfscope}%
\begin{pgfscope}%
\pgfpathrectangle{\pgfqpoint{1.254980in}{0.150000in}}{\pgfqpoint{5.490039in}{5.490039in}}%
\pgfusepath{clip}%
\pgfsetbuttcap%
\pgfsetroundjoin%
\definecolor{currentfill}{rgb}{0.278826,0.175490,0.483397}%
\pgfsetfillcolor{currentfill}%
\pgfsetfillopacity{0.700000}%
\pgfsetlinewidth{0.000000pt}%
\definecolor{currentstroke}{rgb}{0.000000,0.000000,0.000000}%
\pgfsetstrokecolor{currentstroke}%
\pgfsetdash{}{0pt}%
\pgfpathmoveto{\pgfqpoint{4.618583in}{2.377822in}}%
\pgfpathlineto{\pgfqpoint{4.631829in}{2.376537in}}%
\pgfpathlineto{\pgfqpoint{4.645084in}{2.375374in}}%
\pgfpathlineto{\pgfqpoint{4.658348in}{2.374330in}}%
\pgfpathlineto{\pgfqpoint{4.671621in}{2.373407in}}%
\pgfpathlineto{\pgfqpoint{4.678997in}{2.383417in}}%
\pgfpathlineto{\pgfqpoint{4.686368in}{2.393421in}}%
\pgfpathlineto{\pgfqpoint{4.693734in}{2.403421in}}%
\pgfpathlineto{\pgfqpoint{4.701096in}{2.413417in}}%
\pgfpathlineto{\pgfqpoint{4.687832in}{2.414349in}}%
\pgfpathlineto{\pgfqpoint{4.674577in}{2.415401in}}%
\pgfpathlineto{\pgfqpoint{4.661331in}{2.416573in}}%
\pgfpathlineto{\pgfqpoint{4.648093in}{2.417865in}}%
\pgfpathlineto{\pgfqpoint{4.640722in}{2.407855in}}%
\pgfpathlineto{\pgfqpoint{4.633347in}{2.397845in}}%
\pgfpathlineto{\pgfqpoint{4.625967in}{2.387834in}}%
\pgfpathlineto{\pgfqpoint{4.618583in}{2.377822in}}%
\pgfpathclose%
\pgfusepath{fill}%
\end{pgfscope}%
\begin{pgfscope}%
\pgfpathrectangle{\pgfqpoint{1.254980in}{0.150000in}}{\pgfqpoint{5.490039in}{5.490039in}}%
\pgfusepath{clip}%
\pgfsetbuttcap%
\pgfsetroundjoin%
\definecolor{currentfill}{rgb}{0.168126,0.459988,0.558082}%
\pgfsetfillcolor{currentfill}%
\pgfsetfillopacity{0.700000}%
\pgfsetlinewidth{0.000000pt}%
\definecolor{currentstroke}{rgb}{0.000000,0.000000,0.000000}%
\pgfsetstrokecolor{currentstroke}%
\pgfsetdash{}{0pt}%
\pgfpathmoveto{\pgfqpoint{5.857363in}{3.026881in}}%
\pgfpathlineto{\pgfqpoint{5.871113in}{3.030642in}}%
\pgfpathlineto{\pgfqpoint{5.884877in}{3.034512in}}%
\pgfpathlineto{\pgfqpoint{5.898655in}{3.038492in}}%
\pgfpathlineto{\pgfqpoint{5.912447in}{3.042582in}}%
\pgfpathlineto{\pgfqpoint{5.919360in}{3.050509in}}%
\pgfpathlineto{\pgfqpoint{5.926270in}{3.058456in}}%
\pgfpathlineto{\pgfqpoint{5.933175in}{3.066427in}}%
\pgfpathlineto{\pgfqpoint{5.919397in}{3.062525in}}%
\pgfpathlineto{\pgfqpoint{5.905633in}{3.058733in}}%
\pgfpathlineto{\pgfqpoint{5.891883in}{3.055050in}}%
\pgfpathlineto{\pgfqpoint{5.878146in}{3.051476in}}%
\pgfpathlineto{\pgfqpoint{5.871222in}{3.043252in}}%
\pgfpathlineto{\pgfqpoint{5.864294in}{3.035054in}}%
\pgfpathlineto{\pgfqpoint{5.857363in}{3.026881in}}%
\pgfpathclose%
\pgfusepath{fill}%
\end{pgfscope}%
\begin{pgfscope}%
\pgfpathrectangle{\pgfqpoint{1.254980in}{0.150000in}}{\pgfqpoint{5.490039in}{5.490039in}}%
\pgfusepath{clip}%
\pgfsetbuttcap%
\pgfsetroundjoin%
\definecolor{currentfill}{rgb}{0.177423,0.437527,0.557565}%
\pgfsetfillcolor{currentfill}%
\pgfsetfillopacity{0.700000}%
\pgfsetlinewidth{0.000000pt}%
\definecolor{currentstroke}{rgb}{0.000000,0.000000,0.000000}%
\pgfsetstrokecolor{currentstroke}%
\pgfsetdash{}{0pt}%
\pgfpathmoveto{\pgfqpoint{2.900950in}{3.006465in}}%
\pgfpathlineto{\pgfqpoint{2.914124in}{2.988324in}}%
\pgfpathlineto{\pgfqpoint{2.927292in}{2.970377in}}%
\pgfpathlineto{\pgfqpoint{2.940454in}{2.952622in}}%
\pgfpathlineto{\pgfqpoint{2.953611in}{2.935059in}}%
\pgfpathlineto{\pgfqpoint{2.961618in}{2.940855in}}%
\pgfpathlineto{\pgfqpoint{2.969616in}{2.946770in}}%
\pgfpathlineto{\pgfqpoint{2.977605in}{2.952801in}}%
\pgfpathlineto{\pgfqpoint{2.985585in}{2.958948in}}%
\pgfpathlineto{\pgfqpoint{2.972454in}{2.976348in}}%
\pgfpathlineto{\pgfqpoint{2.959317in}{2.993939in}}%
\pgfpathlineto{\pgfqpoint{2.946175in}{3.011722in}}%
\pgfpathlineto{\pgfqpoint{2.933027in}{3.029699in}}%
\pgfpathlineto{\pgfqpoint{2.925021in}{3.023710in}}%
\pgfpathlineto{\pgfqpoint{2.917007in}{3.017840in}}%
\pgfpathlineto{\pgfqpoint{2.908983in}{3.012091in}}%
\pgfpathlineto{\pgfqpoint{2.900950in}{3.006465in}}%
\pgfpathclose%
\pgfusepath{fill}%
\end{pgfscope}%
\begin{pgfscope}%
\pgfpathrectangle{\pgfqpoint{1.254980in}{0.150000in}}{\pgfqpoint{5.490039in}{5.490039in}}%
\pgfusepath{clip}%
\pgfsetbuttcap%
\pgfsetroundjoin%
\definecolor{currentfill}{rgb}{0.280868,0.160771,0.472899}%
\pgfsetfillcolor{currentfill}%
\pgfsetfillopacity{0.700000}%
\pgfsetlinewidth{0.000000pt}%
\definecolor{currentstroke}{rgb}{0.000000,0.000000,0.000000}%
\pgfsetstrokecolor{currentstroke}%
\pgfsetdash{}{0pt}%
\pgfpathmoveto{\pgfqpoint{3.560485in}{2.366058in}}%
\pgfpathlineto{\pgfqpoint{3.573534in}{2.356187in}}%
\pgfpathlineto{\pgfqpoint{3.586585in}{2.346464in}}%
\pgfpathlineto{\pgfqpoint{3.599636in}{2.336888in}}%
\pgfpathlineto{\pgfqpoint{3.612690in}{2.327459in}}%
\pgfpathlineto{\pgfqpoint{3.620419in}{2.335485in}}%
\pgfpathlineto{\pgfqpoint{3.628143in}{2.343576in}}%
\pgfpathlineto{\pgfqpoint{3.635860in}{2.351731in}}%
\pgfpathlineto{\pgfqpoint{3.643572in}{2.359948in}}%
\pgfpathlineto{\pgfqpoint{3.630535in}{2.369240in}}%
\pgfpathlineto{\pgfqpoint{3.617499in}{2.378679in}}%
\pgfpathlineto{\pgfqpoint{3.604465in}{2.388265in}}%
\pgfpathlineto{\pgfqpoint{3.591433in}{2.397999in}}%
\pgfpathlineto{\pgfqpoint{3.583705in}{2.389912in}}%
\pgfpathlineto{\pgfqpoint{3.575971in}{2.381893in}}%
\pgfpathlineto{\pgfqpoint{3.568232in}{2.373941in}}%
\pgfpathlineto{\pgfqpoint{3.560485in}{2.366058in}}%
\pgfpathclose%
\pgfusepath{fill}%
\end{pgfscope}%
\begin{pgfscope}%
\pgfpathrectangle{\pgfqpoint{1.254980in}{0.150000in}}{\pgfqpoint{5.490039in}{5.490039in}}%
\pgfusepath{clip}%
\pgfsetbuttcap%
\pgfsetroundjoin%
\definecolor{currentfill}{rgb}{0.283091,0.110553,0.431554}%
\pgfsetfillcolor{currentfill}%
\pgfsetfillopacity{0.700000}%
\pgfsetlinewidth{0.000000pt}%
\definecolor{currentstroke}{rgb}{0.000000,0.000000,0.000000}%
\pgfsetstrokecolor{currentstroke}%
\pgfsetdash{}{0pt}%
\pgfpathmoveto{\pgfqpoint{3.883041in}{2.266091in}}%
\pgfpathlineto{\pgfqpoint{3.896111in}{2.259315in}}%
\pgfpathlineto{\pgfqpoint{3.909184in}{2.252675in}}%
\pgfpathlineto{\pgfqpoint{3.922262in}{2.246171in}}%
\pgfpathlineto{\pgfqpoint{3.935343in}{2.239802in}}%
\pgfpathlineto{\pgfqpoint{3.942956in}{2.248871in}}%
\pgfpathlineto{\pgfqpoint{3.950563in}{2.257979in}}%
\pgfpathlineto{\pgfqpoint{3.958165in}{2.267124in}}%
\pgfpathlineto{\pgfqpoint{3.965761in}{2.276307in}}%
\pgfpathlineto{\pgfqpoint{3.952692in}{2.282573in}}%
\pgfpathlineto{\pgfqpoint{3.939627in}{2.288974in}}%
\pgfpathlineto{\pgfqpoint{3.926566in}{2.295510in}}%
\pgfpathlineto{\pgfqpoint{3.913509in}{2.302182in}}%
\pgfpathlineto{\pgfqpoint{3.905900in}{2.293097in}}%
\pgfpathlineto{\pgfqpoint{3.898286in}{2.284053in}}%
\pgfpathlineto{\pgfqpoint{3.890666in}{2.275050in}}%
\pgfpathlineto{\pgfqpoint{3.883041in}{2.266091in}}%
\pgfpathclose%
\pgfusepath{fill}%
\end{pgfscope}%
\begin{pgfscope}%
\pgfpathrectangle{\pgfqpoint{1.254980in}{0.150000in}}{\pgfqpoint{5.490039in}{5.490039in}}%
\pgfusepath{clip}%
\pgfsetbuttcap%
\pgfsetroundjoin%
\definecolor{currentfill}{rgb}{0.283229,0.120777,0.440584}%
\pgfsetfillcolor{currentfill}%
\pgfsetfillopacity{0.700000}%
\pgfsetlinewidth{0.000000pt}%
\definecolor{currentstroke}{rgb}{0.000000,0.000000,0.000000}%
\pgfsetstrokecolor{currentstroke}%
\pgfsetdash{}{0pt}%
\pgfpathmoveto{\pgfqpoint{4.235755in}{2.271296in}}%
\pgfpathlineto{\pgfqpoint{4.248891in}{2.267411in}}%
\pgfpathlineto{\pgfqpoint{4.262033in}{2.263653in}}%
\pgfpathlineto{\pgfqpoint{4.275182in}{2.260022in}}%
\pgfpathlineto{\pgfqpoint{4.288338in}{2.256517in}}%
\pgfpathlineto{\pgfqpoint{4.295834in}{2.266322in}}%
\pgfpathlineto{\pgfqpoint{4.303326in}{2.276141in}}%
\pgfpathlineto{\pgfqpoint{4.310814in}{2.285973in}}%
\pgfpathlineto{\pgfqpoint{4.318296in}{2.295819in}}%
\pgfpathlineto{\pgfqpoint{4.305151in}{2.299269in}}%
\pgfpathlineto{\pgfqpoint{4.292012in}{2.302845in}}%
\pgfpathlineto{\pgfqpoint{4.278880in}{2.306547in}}%
\pgfpathlineto{\pgfqpoint{4.265754in}{2.310377in}}%
\pgfpathlineto{\pgfqpoint{4.258261in}{2.300581in}}%
\pgfpathlineto{\pgfqpoint{4.250764in}{2.290802in}}%
\pgfpathlineto{\pgfqpoint{4.243262in}{2.281040in}}%
\pgfpathlineto{\pgfqpoint{4.235755in}{2.271296in}}%
\pgfpathclose%
\pgfusepath{fill}%
\end{pgfscope}%
\begin{pgfscope}%
\pgfpathrectangle{\pgfqpoint{1.254980in}{0.150000in}}{\pgfqpoint{5.490039in}{5.490039in}}%
\pgfusepath{clip}%
\pgfsetbuttcap%
\pgfsetroundjoin%
\definecolor{currentfill}{rgb}{0.265145,0.232956,0.516599}%
\pgfsetfillcolor{currentfill}%
\pgfsetfillopacity{0.700000}%
\pgfsetlinewidth{0.000000pt}%
\definecolor{currentstroke}{rgb}{0.000000,0.000000,0.000000}%
\pgfsetstrokecolor{currentstroke}%
\pgfsetdash{}{0pt}%
\pgfpathmoveto{\pgfqpoint{3.320422in}{2.516352in}}%
\pgfpathlineto{\pgfqpoint{3.333488in}{2.503858in}}%
\pgfpathlineto{\pgfqpoint{3.346553in}{2.491524in}}%
\pgfpathlineto{\pgfqpoint{3.359617in}{2.479350in}}%
\pgfpathlineto{\pgfqpoint{3.372681in}{2.467334in}}%
\pgfpathlineto{\pgfqpoint{3.380508in}{2.474452in}}%
\pgfpathlineto{\pgfqpoint{3.388329in}{2.481654in}}%
\pgfpathlineto{\pgfqpoint{3.396142in}{2.488940in}}%
\pgfpathlineto{\pgfqpoint{3.403949in}{2.496309in}}%
\pgfpathlineto{\pgfqpoint{3.390905in}{2.508170in}}%
\pgfpathlineto{\pgfqpoint{3.377860in}{2.520189in}}%
\pgfpathlineto{\pgfqpoint{3.364815in}{2.532367in}}%
\pgfpathlineto{\pgfqpoint{3.351769in}{2.544705in}}%
\pgfpathlineto{\pgfqpoint{3.343943in}{2.537485in}}%
\pgfpathlineto{\pgfqpoint{3.336110in}{2.530353in}}%
\pgfpathlineto{\pgfqpoint{3.328270in}{2.523308in}}%
\pgfpathlineto{\pgfqpoint{3.320422in}{2.516352in}}%
\pgfpathclose%
\pgfusepath{fill}%
\end{pgfscope}%
\begin{pgfscope}%
\pgfpathrectangle{\pgfqpoint{1.254980in}{0.150000in}}{\pgfqpoint{5.490039in}{5.490039in}}%
\pgfusepath{clip}%
\pgfsetbuttcap%
\pgfsetroundjoin%
\definecolor{currentfill}{rgb}{0.283187,0.125848,0.444960}%
\pgfsetfillcolor{currentfill}%
\pgfsetfillopacity{0.700000}%
\pgfsetlinewidth{0.000000pt}%
\definecolor{currentstroke}{rgb}{0.000000,0.000000,0.000000}%
\pgfsetstrokecolor{currentstroke}%
\pgfsetdash{}{0pt}%
\pgfpathmoveto{\pgfqpoint{3.747942in}{2.290808in}}%
\pgfpathlineto{\pgfqpoint{3.760999in}{2.282806in}}%
\pgfpathlineto{\pgfqpoint{3.774060in}{2.274944in}}%
\pgfpathlineto{\pgfqpoint{3.787123in}{2.267222in}}%
\pgfpathlineto{\pgfqpoint{3.800189in}{2.259639in}}%
\pgfpathlineto{\pgfqpoint{3.807849in}{2.268297in}}%
\pgfpathlineto{\pgfqpoint{3.815504in}{2.277004in}}%
\pgfpathlineto{\pgfqpoint{3.823154in}{2.285759in}}%
\pgfpathlineto{\pgfqpoint{3.830798in}{2.294562in}}%
\pgfpathlineto{\pgfqpoint{3.817746in}{2.302025in}}%
\pgfpathlineto{\pgfqpoint{3.804697in}{2.309628in}}%
\pgfpathlineto{\pgfqpoint{3.791651in}{2.317370in}}%
\pgfpathlineto{\pgfqpoint{3.778608in}{2.325252in}}%
\pgfpathlineto{\pgfqpoint{3.770950in}{2.316562in}}%
\pgfpathlineto{\pgfqpoint{3.763286in}{2.307925in}}%
\pgfpathlineto{\pgfqpoint{3.755617in}{2.299340in}}%
\pgfpathlineto{\pgfqpoint{3.747942in}{2.290808in}}%
\pgfpathclose%
\pgfusepath{fill}%
\end{pgfscope}%
\begin{pgfscope}%
\pgfpathrectangle{\pgfqpoint{1.254980in}{0.150000in}}{\pgfqpoint{5.490039in}{5.490039in}}%
\pgfusepath{clip}%
\pgfsetbuttcap%
\pgfsetroundjoin%
\definecolor{currentfill}{rgb}{0.280868,0.160771,0.472899}%
\pgfsetfillcolor{currentfill}%
\pgfsetfillopacity{0.700000}%
\pgfsetlinewidth{0.000000pt}%
\definecolor{currentstroke}{rgb}{0.000000,0.000000,0.000000}%
\pgfsetstrokecolor{currentstroke}%
\pgfsetdash{}{0pt}%
\pgfpathmoveto{\pgfqpoint{4.536059in}{2.344136in}}%
\pgfpathlineto{\pgfqpoint{4.549281in}{2.342359in}}%
\pgfpathlineto{\pgfqpoint{4.562512in}{2.340704in}}%
\pgfpathlineto{\pgfqpoint{4.575750in}{2.339170in}}%
\pgfpathlineto{\pgfqpoint{4.588997in}{2.337758in}}%
\pgfpathlineto{\pgfqpoint{4.596401in}{2.347777in}}%
\pgfpathlineto{\pgfqpoint{4.603799in}{2.357793in}}%
\pgfpathlineto{\pgfqpoint{4.611193in}{2.367808in}}%
\pgfpathlineto{\pgfqpoint{4.618583in}{2.377822in}}%
\pgfpathlineto{\pgfqpoint{4.605344in}{2.379227in}}%
\pgfpathlineto{\pgfqpoint{4.592115in}{2.380753in}}%
\pgfpathlineto{\pgfqpoint{4.578893in}{2.382400in}}%
\pgfpathlineto{\pgfqpoint{4.565680in}{2.384169in}}%
\pgfpathlineto{\pgfqpoint{4.558281in}{2.374158in}}%
\pgfpathlineto{\pgfqpoint{4.550879in}{2.364148in}}%
\pgfpathlineto{\pgfqpoint{4.543471in}{2.354141in}}%
\pgfpathlineto{\pgfqpoint{4.536059in}{2.344136in}}%
\pgfpathclose%
\pgfusepath{fill}%
\end{pgfscope}%
\begin{pgfscope}%
\pgfpathrectangle{\pgfqpoint{1.254980in}{0.150000in}}{\pgfqpoint{5.490039in}{5.490039in}}%
\pgfusepath{clip}%
\pgfsetbuttcap%
\pgfsetroundjoin%
\definecolor{currentfill}{rgb}{0.282910,0.105393,0.426902}%
\pgfsetfillcolor{currentfill}%
\pgfsetfillopacity{0.700000}%
\pgfsetlinewidth{0.000000pt}%
\definecolor{currentstroke}{rgb}{0.000000,0.000000,0.000000}%
\pgfsetstrokecolor{currentstroke}%
\pgfsetdash{}{0pt}%
\pgfpathmoveto{\pgfqpoint{4.018082in}{2.252582in}}%
\pgfpathlineto{\pgfqpoint{4.031173in}{2.246983in}}%
\pgfpathlineto{\pgfqpoint{4.044270in}{2.241517in}}%
\pgfpathlineto{\pgfqpoint{4.057371in}{2.236181in}}%
\pgfpathlineto{\pgfqpoint{4.070478in}{2.230977in}}%
\pgfpathlineto{\pgfqpoint{4.078046in}{2.240383in}}%
\pgfpathlineto{\pgfqpoint{4.085608in}{2.249817in}}%
\pgfpathlineto{\pgfqpoint{4.093166in}{2.259280in}}%
\pgfpathlineto{\pgfqpoint{4.100719in}{2.268769in}}%
\pgfpathlineto{\pgfqpoint{4.087624in}{2.273886in}}%
\pgfpathlineto{\pgfqpoint{4.074534in}{2.279134in}}%
\pgfpathlineto{\pgfqpoint{4.061449in}{2.284514in}}%
\pgfpathlineto{\pgfqpoint{4.048369in}{2.290025in}}%
\pgfpathlineto{\pgfqpoint{4.040804in}{2.280617in}}%
\pgfpathlineto{\pgfqpoint{4.033235in}{2.271240in}}%
\pgfpathlineto{\pgfqpoint{4.025661in}{2.261895in}}%
\pgfpathlineto{\pgfqpoint{4.018082in}{2.252582in}}%
\pgfpathclose%
\pgfusepath{fill}%
\end{pgfscope}%
\begin{pgfscope}%
\pgfpathrectangle{\pgfqpoint{1.254980in}{0.150000in}}{\pgfqpoint{5.490039in}{5.490039in}}%
\pgfusepath{clip}%
\pgfsetbuttcap%
\pgfsetroundjoin%
\definecolor{currentfill}{rgb}{0.165117,0.467423,0.558141}%
\pgfsetfillcolor{currentfill}%
\pgfsetfillopacity{0.700000}%
\pgfsetlinewidth{0.000000pt}%
\definecolor{currentstroke}{rgb}{0.000000,0.000000,0.000000}%
\pgfsetstrokecolor{currentstroke}%
\pgfsetdash{}{0pt}%
\pgfpathmoveto{\pgfqpoint{2.848194in}{3.080995in}}%
\pgfpathlineto{\pgfqpoint{2.861393in}{3.062065in}}%
\pgfpathlineto{\pgfqpoint{2.874585in}{3.043334in}}%
\pgfpathlineto{\pgfqpoint{2.887771in}{3.024801in}}%
\pgfpathlineto{\pgfqpoint{2.900950in}{3.006465in}}%
\pgfpathlineto{\pgfqpoint{2.908983in}{3.012091in}}%
\pgfpathlineto{\pgfqpoint{2.917007in}{3.017840in}}%
\pgfpathlineto{\pgfqpoint{2.925021in}{3.023710in}}%
\pgfpathlineto{\pgfqpoint{2.933027in}{3.029699in}}%
\pgfpathlineto{\pgfqpoint{2.919873in}{3.047871in}}%
\pgfpathlineto{\pgfqpoint{2.906713in}{3.066239in}}%
\pgfpathlineto{\pgfqpoint{2.893548in}{3.084804in}}%
\pgfpathlineto{\pgfqpoint{2.880375in}{3.103568in}}%
\pgfpathlineto{\pgfqpoint{2.872344in}{3.097738in}}%
\pgfpathlineto{\pgfqpoint{2.864304in}{3.092031in}}%
\pgfpathlineto{\pgfqpoint{2.856254in}{3.086450in}}%
\pgfpathlineto{\pgfqpoint{2.848194in}{3.080995in}}%
\pgfpathclose%
\pgfusepath{fill}%
\end{pgfscope}%
\begin{pgfscope}%
\pgfpathrectangle{\pgfqpoint{1.254980in}{0.150000in}}{\pgfqpoint{5.490039in}{5.490039in}}%
\pgfusepath{clip}%
\pgfsetbuttcap%
\pgfsetroundjoin%
\definecolor{currentfill}{rgb}{0.282290,0.145912,0.461510}%
\pgfsetfillcolor{currentfill}%
\pgfsetfillopacity{0.700000}%
\pgfsetlinewidth{0.000000pt}%
\definecolor{currentstroke}{rgb}{0.000000,0.000000,0.000000}%
\pgfsetstrokecolor{currentstroke}%
\pgfsetdash{}{0pt}%
\pgfpathmoveto{\pgfqpoint{4.453516in}{2.312553in}}%
\pgfpathlineto{\pgfqpoint{4.466716in}{2.310263in}}%
\pgfpathlineto{\pgfqpoint{4.479924in}{2.308096in}}%
\pgfpathlineto{\pgfqpoint{4.493140in}{2.306052in}}%
\pgfpathlineto{\pgfqpoint{4.506363in}{2.304130in}}%
\pgfpathlineto{\pgfqpoint{4.513794in}{2.314129in}}%
\pgfpathlineto{\pgfqpoint{4.521220in}{2.324130in}}%
\pgfpathlineto{\pgfqpoint{4.528642in}{2.334132in}}%
\pgfpathlineto{\pgfqpoint{4.536059in}{2.344136in}}%
\pgfpathlineto{\pgfqpoint{4.522844in}{2.346034in}}%
\pgfpathlineto{\pgfqpoint{4.509638in}{2.348055in}}%
\pgfpathlineto{\pgfqpoint{4.496439in}{2.350199in}}%
\pgfpathlineto{\pgfqpoint{4.483248in}{2.352465in}}%
\pgfpathlineto{\pgfqpoint{4.475822in}{2.342479in}}%
\pgfpathlineto{\pgfqpoint{4.468392in}{2.332498in}}%
\pgfpathlineto{\pgfqpoint{4.460956in}{2.322523in}}%
\pgfpathlineto{\pgfqpoint{4.453516in}{2.312553in}}%
\pgfpathclose%
\pgfusepath{fill}%
\end{pgfscope}%
\begin{pgfscope}%
\pgfpathrectangle{\pgfqpoint{1.254980in}{0.150000in}}{\pgfqpoint{5.490039in}{5.490039in}}%
\pgfusepath{clip}%
\pgfsetbuttcap%
\pgfsetroundjoin%
\definecolor{currentfill}{rgb}{0.271828,0.209303,0.504434}%
\pgfsetfillcolor{currentfill}%
\pgfsetfillopacity{0.700000}%
\pgfsetlinewidth{0.000000pt}%
\definecolor{currentstroke}{rgb}{0.000000,0.000000,0.000000}%
\pgfsetstrokecolor{currentstroke}%
\pgfsetdash{}{0pt}%
\pgfpathmoveto{\pgfqpoint{3.372681in}{2.467334in}}%
\pgfpathlineto{\pgfqpoint{3.385743in}{2.455476in}}%
\pgfpathlineto{\pgfqpoint{3.398806in}{2.443775in}}%
\pgfpathlineto{\pgfqpoint{3.411868in}{2.432231in}}%
\pgfpathlineto{\pgfqpoint{3.424930in}{2.420841in}}%
\pgfpathlineto{\pgfqpoint{3.432739in}{2.428120in}}%
\pgfpathlineto{\pgfqpoint{3.440540in}{2.435479in}}%
\pgfpathlineto{\pgfqpoint{3.448335in}{2.442919in}}%
\pgfpathlineto{\pgfqpoint{3.456123in}{2.450436in}}%
\pgfpathlineto{\pgfqpoint{3.443080in}{2.461671in}}%
\pgfpathlineto{\pgfqpoint{3.430036in}{2.473061in}}%
\pgfpathlineto{\pgfqpoint{3.416993in}{2.484607in}}%
\pgfpathlineto{\pgfqpoint{3.403949in}{2.496309in}}%
\pgfpathlineto{\pgfqpoint{3.396142in}{2.488940in}}%
\pgfpathlineto{\pgfqpoint{3.388329in}{2.481654in}}%
\pgfpathlineto{\pgfqpoint{3.380508in}{2.474452in}}%
\pgfpathlineto{\pgfqpoint{3.372681in}{2.467334in}}%
\pgfpathclose%
\pgfusepath{fill}%
\end{pgfscope}%
\begin{pgfscope}%
\pgfpathrectangle{\pgfqpoint{1.254980in}{0.150000in}}{\pgfqpoint{5.490039in}{5.490039in}}%
\pgfusepath{clip}%
\pgfsetbuttcap%
\pgfsetroundjoin%
\definecolor{currentfill}{rgb}{0.283091,0.110553,0.431554}%
\pgfsetfillcolor{currentfill}%
\pgfsetfillopacity{0.700000}%
\pgfsetlinewidth{0.000000pt}%
\definecolor{currentstroke}{rgb}{0.000000,0.000000,0.000000}%
\pgfsetstrokecolor{currentstroke}%
\pgfsetdash{}{0pt}%
\pgfpathmoveto{\pgfqpoint{4.153153in}{2.249604in}}%
\pgfpathlineto{\pgfqpoint{4.166276in}{2.245136in}}%
\pgfpathlineto{\pgfqpoint{4.179404in}{2.240796in}}%
\pgfpathlineto{\pgfqpoint{4.192539in}{2.236585in}}%
\pgfpathlineto{\pgfqpoint{4.205679in}{2.232502in}}%
\pgfpathlineto{\pgfqpoint{4.213205in}{2.242173in}}%
\pgfpathlineto{\pgfqpoint{4.220727in}{2.251862in}}%
\pgfpathlineto{\pgfqpoint{4.228243in}{2.261570in}}%
\pgfpathlineto{\pgfqpoint{4.235755in}{2.271296in}}%
\pgfpathlineto{\pgfqpoint{4.222625in}{2.275308in}}%
\pgfpathlineto{\pgfqpoint{4.209501in}{2.279448in}}%
\pgfpathlineto{\pgfqpoint{4.196383in}{2.283717in}}%
\pgfpathlineto{\pgfqpoint{4.183271in}{2.288113in}}%
\pgfpathlineto{\pgfqpoint{4.175749in}{2.278452in}}%
\pgfpathlineto{\pgfqpoint{4.168222in}{2.268814in}}%
\pgfpathlineto{\pgfqpoint{4.160690in}{2.259197in}}%
\pgfpathlineto{\pgfqpoint{4.153153in}{2.249604in}}%
\pgfpathclose%
\pgfusepath{fill}%
\end{pgfscope}%
\begin{pgfscope}%
\pgfpathrectangle{\pgfqpoint{1.254980in}{0.150000in}}{\pgfqpoint{5.490039in}{5.490039in}}%
\pgfusepath{clip}%
\pgfsetbuttcap%
\pgfsetroundjoin%
\definecolor{currentfill}{rgb}{0.282290,0.145912,0.461510}%
\pgfsetfillcolor{currentfill}%
\pgfsetfillopacity{0.700000}%
\pgfsetlinewidth{0.000000pt}%
\definecolor{currentstroke}{rgb}{0.000000,0.000000,0.000000}%
\pgfsetstrokecolor{currentstroke}%
\pgfsetdash{}{0pt}%
\pgfpathmoveto{\pgfqpoint{3.612690in}{2.327459in}}%
\pgfpathlineto{\pgfqpoint{3.625745in}{2.318176in}}%
\pgfpathlineto{\pgfqpoint{3.638801in}{2.309039in}}%
\pgfpathlineto{\pgfqpoint{3.651860in}{2.300046in}}%
\pgfpathlineto{\pgfqpoint{3.664920in}{2.291197in}}%
\pgfpathlineto{\pgfqpoint{3.672634in}{2.299365in}}%
\pgfpathlineto{\pgfqpoint{3.680341in}{2.307594in}}%
\pgfpathlineto{\pgfqpoint{3.688043in}{2.315883in}}%
\pgfpathlineto{\pgfqpoint{3.695739in}{2.324231in}}%
\pgfpathlineto{\pgfqpoint{3.682694in}{2.332944in}}%
\pgfpathlineto{\pgfqpoint{3.669651in}{2.341800in}}%
\pgfpathlineto{\pgfqpoint{3.656610in}{2.350802in}}%
\pgfpathlineto{\pgfqpoint{3.643572in}{2.359948in}}%
\pgfpathlineto{\pgfqpoint{3.635860in}{2.351731in}}%
\pgfpathlineto{\pgfqpoint{3.628143in}{2.343576in}}%
\pgfpathlineto{\pgfqpoint{3.620419in}{2.335485in}}%
\pgfpathlineto{\pgfqpoint{3.612690in}{2.327459in}}%
\pgfpathclose%
\pgfusepath{fill}%
\end{pgfscope}%
\begin{pgfscope}%
\pgfpathrectangle{\pgfqpoint{1.254980in}{0.150000in}}{\pgfqpoint{5.490039in}{5.490039in}}%
\pgfusepath{clip}%
\pgfsetbuttcap%
\pgfsetroundjoin%
\definecolor{currentfill}{rgb}{0.153364,0.497000,0.557724}%
\pgfsetfillcolor{currentfill}%
\pgfsetfillopacity{0.700000}%
\pgfsetlinewidth{0.000000pt}%
\definecolor{currentstroke}{rgb}{0.000000,0.000000,0.000000}%
\pgfsetstrokecolor{currentstroke}%
\pgfsetdash{}{0pt}%
\pgfpathmoveto{\pgfqpoint{2.795331in}{3.158734in}}%
\pgfpathlineto{\pgfqpoint{2.808557in}{3.138994in}}%
\pgfpathlineto{\pgfqpoint{2.821777in}{3.119458in}}%
\pgfpathlineto{\pgfqpoint{2.834989in}{3.100126in}}%
\pgfpathlineto{\pgfqpoint{2.848194in}{3.080995in}}%
\pgfpathlineto{\pgfqpoint{2.856254in}{3.086450in}}%
\pgfpathlineto{\pgfqpoint{2.864304in}{3.092031in}}%
\pgfpathlineto{\pgfqpoint{2.872344in}{3.097738in}}%
\pgfpathlineto{\pgfqpoint{2.880375in}{3.103568in}}%
\pgfpathlineto{\pgfqpoint{2.867197in}{3.122533in}}%
\pgfpathlineto{\pgfqpoint{2.854012in}{3.141699in}}%
\pgfpathlineto{\pgfqpoint{2.840819in}{3.161067in}}%
\pgfpathlineto{\pgfqpoint{2.827620in}{3.180641in}}%
\pgfpathlineto{\pgfqpoint{2.819563in}{3.174970in}}%
\pgfpathlineto{\pgfqpoint{2.811495in}{3.169428in}}%
\pgfpathlineto{\pgfqpoint{2.803418in}{3.164016in}}%
\pgfpathlineto{\pgfqpoint{2.795331in}{3.158734in}}%
\pgfpathclose%
\pgfusepath{fill}%
\end{pgfscope}%
\begin{pgfscope}%
\pgfpathrectangle{\pgfqpoint{1.254980in}{0.150000in}}{\pgfqpoint{5.490039in}{5.490039in}}%
\pgfusepath{clip}%
\pgfsetbuttcap%
\pgfsetroundjoin%
\definecolor{currentfill}{rgb}{0.283072,0.130895,0.449241}%
\pgfsetfillcolor{currentfill}%
\pgfsetfillopacity{0.700000}%
\pgfsetlinewidth{0.000000pt}%
\definecolor{currentstroke}{rgb}{0.000000,0.000000,0.000000}%
\pgfsetstrokecolor{currentstroke}%
\pgfsetdash{}{0pt}%
\pgfpathmoveto{\pgfqpoint{4.370946in}{2.283275in}}%
\pgfpathlineto{\pgfqpoint{4.384126in}{2.280452in}}%
\pgfpathlineto{\pgfqpoint{4.397313in}{2.277752in}}%
\pgfpathlineto{\pgfqpoint{4.410507in}{2.275176in}}%
\pgfpathlineto{\pgfqpoint{4.423709in}{2.272724in}}%
\pgfpathlineto{\pgfqpoint{4.431168in}{2.282673in}}%
\pgfpathlineto{\pgfqpoint{4.438622in}{2.292628in}}%
\pgfpathlineto{\pgfqpoint{4.446071in}{2.302588in}}%
\pgfpathlineto{\pgfqpoint{4.453516in}{2.312553in}}%
\pgfpathlineto{\pgfqpoint{4.440324in}{2.314966in}}%
\pgfpathlineto{\pgfqpoint{4.427139in}{2.317502in}}%
\pgfpathlineto{\pgfqpoint{4.413961in}{2.320162in}}%
\pgfpathlineto{\pgfqpoint{4.400791in}{2.322947in}}%
\pgfpathlineto{\pgfqpoint{4.393336in}{2.313015in}}%
\pgfpathlineto{\pgfqpoint{4.385878in}{2.303093in}}%
\pgfpathlineto{\pgfqpoint{4.378414in}{2.293179in}}%
\pgfpathlineto{\pgfqpoint{4.370946in}{2.283275in}}%
\pgfpathclose%
\pgfusepath{fill}%
\end{pgfscope}%
\begin{pgfscope}%
\pgfpathrectangle{\pgfqpoint{1.254980in}{0.150000in}}{\pgfqpoint{5.490039in}{5.490039in}}%
\pgfusepath{clip}%
\pgfsetbuttcap%
\pgfsetroundjoin%
\definecolor{currentfill}{rgb}{0.276194,0.190074,0.493001}%
\pgfsetfillcolor{currentfill}%
\pgfsetfillopacity{0.700000}%
\pgfsetlinewidth{0.000000pt}%
\definecolor{currentstroke}{rgb}{0.000000,0.000000,0.000000}%
\pgfsetstrokecolor{currentstroke}%
\pgfsetdash{}{0pt}%
\pgfpathmoveto{\pgfqpoint{3.424930in}{2.420841in}}%
\pgfpathlineto{\pgfqpoint{3.437992in}{2.409607in}}%
\pgfpathlineto{\pgfqpoint{3.451054in}{2.398526in}}%
\pgfpathlineto{\pgfqpoint{3.464117in}{2.387599in}}%
\pgfpathlineto{\pgfqpoint{3.477179in}{2.376823in}}%
\pgfpathlineto{\pgfqpoint{3.484969in}{2.384262in}}%
\pgfpathlineto{\pgfqpoint{3.492752in}{2.391778in}}%
\pgfpathlineto{\pgfqpoint{3.500529in}{2.399369in}}%
\pgfpathlineto{\pgfqpoint{3.508299in}{2.407035in}}%
\pgfpathlineto{\pgfqpoint{3.495254in}{2.417656in}}%
\pgfpathlineto{\pgfqpoint{3.482210in}{2.428430in}}%
\pgfpathlineto{\pgfqpoint{3.469167in}{2.439356in}}%
\pgfpathlineto{\pgfqpoint{3.456123in}{2.450436in}}%
\pgfpathlineto{\pgfqpoint{3.448335in}{2.442919in}}%
\pgfpathlineto{\pgfqpoint{3.440540in}{2.435479in}}%
\pgfpathlineto{\pgfqpoint{3.432739in}{2.428120in}}%
\pgfpathlineto{\pgfqpoint{3.424930in}{2.420841in}}%
\pgfpathclose%
\pgfusepath{fill}%
\end{pgfscope}%
\begin{pgfscope}%
\pgfpathrectangle{\pgfqpoint{1.254980in}{0.150000in}}{\pgfqpoint{5.490039in}{5.490039in}}%
\pgfusepath{clip}%
\pgfsetbuttcap%
\pgfsetroundjoin%
\definecolor{currentfill}{rgb}{0.282910,0.105393,0.426902}%
\pgfsetfillcolor{currentfill}%
\pgfsetfillopacity{0.700000}%
\pgfsetlinewidth{0.000000pt}%
\definecolor{currentstroke}{rgb}{0.000000,0.000000,0.000000}%
\pgfsetstrokecolor{currentstroke}%
\pgfsetdash{}{0pt}%
\pgfpathmoveto{\pgfqpoint{3.935343in}{2.239802in}}%
\pgfpathlineto{\pgfqpoint{3.948429in}{2.233567in}}%
\pgfpathlineto{\pgfqpoint{3.961519in}{2.227467in}}%
\pgfpathlineto{\pgfqpoint{3.974614in}{2.221499in}}%
\pgfpathlineto{\pgfqpoint{3.987713in}{2.215665in}}%
\pgfpathlineto{\pgfqpoint{3.995313in}{2.224843in}}%
\pgfpathlineto{\pgfqpoint{4.002908in}{2.234056in}}%
\pgfpathlineto{\pgfqpoint{4.010497in}{2.243302in}}%
\pgfpathlineto{\pgfqpoint{4.018082in}{2.252582in}}%
\pgfpathlineto{\pgfqpoint{4.004995in}{2.258314in}}%
\pgfpathlineto{\pgfqpoint{3.991912in}{2.264178in}}%
\pgfpathlineto{\pgfqpoint{3.978835in}{2.270175in}}%
\pgfpathlineto{\pgfqpoint{3.965761in}{2.276307in}}%
\pgfpathlineto{\pgfqpoint{3.958165in}{2.267124in}}%
\pgfpathlineto{\pgfqpoint{3.950563in}{2.257979in}}%
\pgfpathlineto{\pgfqpoint{3.942956in}{2.248871in}}%
\pgfpathlineto{\pgfqpoint{3.935343in}{2.239802in}}%
\pgfpathclose%
\pgfusepath{fill}%
\end{pgfscope}%
\begin{pgfscope}%
\pgfpathrectangle{\pgfqpoint{1.254980in}{0.150000in}}{\pgfqpoint{5.490039in}{5.490039in}}%
\pgfusepath{clip}%
\pgfsetbuttcap%
\pgfsetroundjoin%
\definecolor{currentfill}{rgb}{0.283197,0.115680,0.436115}%
\pgfsetfillcolor{currentfill}%
\pgfsetfillopacity{0.700000}%
\pgfsetlinewidth{0.000000pt}%
\definecolor{currentstroke}{rgb}{0.000000,0.000000,0.000000}%
\pgfsetstrokecolor{currentstroke}%
\pgfsetdash{}{0pt}%
\pgfpathmoveto{\pgfqpoint{3.800189in}{2.259639in}}%
\pgfpathlineto{\pgfqpoint{3.813258in}{2.252195in}}%
\pgfpathlineto{\pgfqpoint{3.826331in}{2.244889in}}%
\pgfpathlineto{\pgfqpoint{3.839407in}{2.237721in}}%
\pgfpathlineto{\pgfqpoint{3.852486in}{2.230690in}}%
\pgfpathlineto{\pgfqpoint{3.860133in}{2.239473in}}%
\pgfpathlineto{\pgfqpoint{3.867774in}{2.248301in}}%
\pgfpathlineto{\pgfqpoint{3.875410in}{2.257174in}}%
\pgfpathlineto{\pgfqpoint{3.883041in}{2.266091in}}%
\pgfpathlineto{\pgfqpoint{3.869975in}{2.273003in}}%
\pgfpathlineto{\pgfqpoint{3.856912in}{2.280051in}}%
\pgfpathlineto{\pgfqpoint{3.843853in}{2.287238in}}%
\pgfpathlineto{\pgfqpoint{3.830798in}{2.294562in}}%
\pgfpathlineto{\pgfqpoint{3.823154in}{2.285759in}}%
\pgfpathlineto{\pgfqpoint{3.815504in}{2.277004in}}%
\pgfpathlineto{\pgfqpoint{3.807849in}{2.268297in}}%
\pgfpathlineto{\pgfqpoint{3.800189in}{2.259639in}}%
\pgfpathclose%
\pgfusepath{fill}%
\end{pgfscope}%
\begin{pgfscope}%
\pgfpathrectangle{\pgfqpoint{1.254980in}{0.150000in}}{\pgfqpoint{5.490039in}{5.490039in}}%
\pgfusepath{clip}%
\pgfsetbuttcap%
\pgfsetroundjoin%
\definecolor{currentfill}{rgb}{0.248629,0.278775,0.534556}%
\pgfsetfillcolor{currentfill}%
\pgfsetfillopacity{0.700000}%
\pgfsetlinewidth{0.000000pt}%
\definecolor{currentstroke}{rgb}{0.000000,0.000000,0.000000}%
\pgfsetstrokecolor{currentstroke}%
\pgfsetdash{}{0pt}%
\pgfpathmoveto{\pgfqpoint{5.084793in}{2.575579in}}%
\pgfpathlineto{\pgfqpoint{5.098228in}{2.576919in}}%
\pgfpathlineto{\pgfqpoint{5.111674in}{2.578374in}}%
\pgfpathlineto{\pgfqpoint{5.125131in}{2.579945in}}%
\pgfpathlineto{\pgfqpoint{5.138599in}{2.581631in}}%
\pgfpathlineto{\pgfqpoint{5.145824in}{2.591157in}}%
\pgfpathlineto{\pgfqpoint{5.153043in}{2.600669in}}%
\pgfpathlineto{\pgfqpoint{5.160258in}{2.610168in}}%
\pgfpathlineto{\pgfqpoint{5.167468in}{2.619654in}}%
\pgfpathlineto{\pgfqpoint{5.154010in}{2.618057in}}%
\pgfpathlineto{\pgfqpoint{5.140563in}{2.616576in}}%
\pgfpathlineto{\pgfqpoint{5.127127in}{2.615209in}}%
\pgfpathlineto{\pgfqpoint{5.113702in}{2.613958in}}%
\pgfpathlineto{\pgfqpoint{5.106482in}{2.604377in}}%
\pgfpathlineto{\pgfqpoint{5.099257in}{2.594787in}}%
\pgfpathlineto{\pgfqpoint{5.092027in}{2.585188in}}%
\pgfpathlineto{\pgfqpoint{5.084793in}{2.575579in}}%
\pgfpathclose%
\pgfusepath{fill}%
\end{pgfscope}%
\begin{pgfscope}%
\pgfpathrectangle{\pgfqpoint{1.254980in}{0.150000in}}{\pgfqpoint{5.490039in}{5.490039in}}%
\pgfusepath{clip}%
\pgfsetbuttcap%
\pgfsetroundjoin%
\definecolor{currentfill}{rgb}{0.239346,0.300855,0.540844}%
\pgfsetfillcolor{currentfill}%
\pgfsetfillopacity{0.700000}%
\pgfsetlinewidth{0.000000pt}%
\definecolor{currentstroke}{rgb}{0.000000,0.000000,0.000000}%
\pgfsetstrokecolor{currentstroke}%
\pgfsetdash{}{0pt}%
\pgfpathmoveto{\pgfqpoint{5.167468in}{2.619654in}}%
\pgfpathlineto{\pgfqpoint{5.180937in}{2.621365in}}%
\pgfpathlineto{\pgfqpoint{5.194418in}{2.623190in}}%
\pgfpathlineto{\pgfqpoint{5.207910in}{2.625130in}}%
\pgfpathlineto{\pgfqpoint{5.221414in}{2.627184in}}%
\pgfpathlineto{\pgfqpoint{5.228608in}{2.636560in}}%
\pgfpathlineto{\pgfqpoint{5.235798in}{2.645921in}}%
\pgfpathlineto{\pgfqpoint{5.242982in}{2.655269in}}%
\pgfpathlineto{\pgfqpoint{5.250162in}{2.664605in}}%
\pgfpathlineto{\pgfqpoint{5.236669in}{2.662657in}}%
\pgfpathlineto{\pgfqpoint{5.223188in}{2.660822in}}%
\pgfpathlineto{\pgfqpoint{5.209718in}{2.659102in}}%
\pgfpathlineto{\pgfqpoint{5.196259in}{2.657496in}}%
\pgfpathlineto{\pgfqpoint{5.189069in}{2.648048in}}%
\pgfpathlineto{\pgfqpoint{5.181873in}{2.638593in}}%
\pgfpathlineto{\pgfqpoint{5.174673in}{2.629128in}}%
\pgfpathlineto{\pgfqpoint{5.167468in}{2.619654in}}%
\pgfpathclose%
\pgfusepath{fill}%
\end{pgfscope}%
\begin{pgfscope}%
\pgfpathrectangle{\pgfqpoint{1.254980in}{0.150000in}}{\pgfqpoint{5.490039in}{5.490039in}}%
\pgfusepath{clip}%
\pgfsetbuttcap%
\pgfsetroundjoin%
\definecolor{currentfill}{rgb}{0.255645,0.260703,0.528312}%
\pgfsetfillcolor{currentfill}%
\pgfsetfillopacity{0.700000}%
\pgfsetlinewidth{0.000000pt}%
\definecolor{currentstroke}{rgb}{0.000000,0.000000,0.000000}%
\pgfsetstrokecolor{currentstroke}%
\pgfsetdash{}{0pt}%
\pgfpathmoveto{\pgfqpoint{5.002136in}{2.532515in}}%
\pgfpathlineto{\pgfqpoint{5.015537in}{2.533465in}}%
\pgfpathlineto{\pgfqpoint{5.028949in}{2.534532in}}%
\pgfpathlineto{\pgfqpoint{5.042372in}{2.535714in}}%
\pgfpathlineto{\pgfqpoint{5.055806in}{2.537012in}}%
\pgfpathlineto{\pgfqpoint{5.063060in}{2.546675in}}%
\pgfpathlineto{\pgfqpoint{5.070309in}{2.556323in}}%
\pgfpathlineto{\pgfqpoint{5.077553in}{2.565957in}}%
\pgfpathlineto{\pgfqpoint{5.084793in}{2.575579in}}%
\pgfpathlineto{\pgfqpoint{5.071369in}{2.574354in}}%
\pgfpathlineto{\pgfqpoint{5.057956in}{2.573244in}}%
\pgfpathlineto{\pgfqpoint{5.044553in}{2.572250in}}%
\pgfpathlineto{\pgfqpoint{5.031161in}{2.571372in}}%
\pgfpathlineto{\pgfqpoint{5.023912in}{2.561672in}}%
\pgfpathlineto{\pgfqpoint{5.016658in}{2.551963in}}%
\pgfpathlineto{\pgfqpoint{5.009399in}{2.542244in}}%
\pgfpathlineto{\pgfqpoint{5.002136in}{2.532515in}}%
\pgfpathclose%
\pgfusepath{fill}%
\end{pgfscope}%
\begin{pgfscope}%
\pgfpathrectangle{\pgfqpoint{1.254980in}{0.150000in}}{\pgfqpoint{5.490039in}{5.490039in}}%
\pgfusepath{clip}%
\pgfsetbuttcap%
\pgfsetroundjoin%
\definecolor{currentfill}{rgb}{0.229739,0.322361,0.545706}%
\pgfsetfillcolor{currentfill}%
\pgfsetfillopacity{0.700000}%
\pgfsetlinewidth{0.000000pt}%
\definecolor{currentstroke}{rgb}{0.000000,0.000000,0.000000}%
\pgfsetstrokecolor{currentstroke}%
\pgfsetdash{}{0pt}%
\pgfpathmoveto{\pgfqpoint{5.250162in}{2.664605in}}%
\pgfpathlineto{\pgfqpoint{5.263667in}{2.666668in}}%
\pgfpathlineto{\pgfqpoint{5.277183in}{2.668844in}}%
\pgfpathlineto{\pgfqpoint{5.290711in}{2.671135in}}%
\pgfpathlineto{\pgfqpoint{5.304251in}{2.673538in}}%
\pgfpathlineto{\pgfqpoint{5.311415in}{2.682749in}}%
\pgfpathlineto{\pgfqpoint{5.318574in}{2.691946in}}%
\pgfpathlineto{\pgfqpoint{5.325728in}{2.701132in}}%
\pgfpathlineto{\pgfqpoint{5.332877in}{2.710307in}}%
\pgfpathlineto{\pgfqpoint{5.319348in}{2.708025in}}%
\pgfpathlineto{\pgfqpoint{5.305832in}{2.705856in}}%
\pgfpathlineto{\pgfqpoint{5.292327in}{2.703801in}}%
\pgfpathlineto{\pgfqpoint{5.278833in}{2.701860in}}%
\pgfpathlineto{\pgfqpoint{5.271673in}{2.692557in}}%
\pgfpathlineto{\pgfqpoint{5.264507in}{2.683248in}}%
\pgfpathlineto{\pgfqpoint{5.257337in}{2.673931in}}%
\pgfpathlineto{\pgfqpoint{5.250162in}{2.664605in}}%
\pgfpathclose%
\pgfusepath{fill}%
\end{pgfscope}%
\begin{pgfscope}%
\pgfpathrectangle{\pgfqpoint{1.254980in}{0.150000in}}{\pgfqpoint{5.490039in}{5.490039in}}%
\pgfusepath{clip}%
\pgfsetbuttcap%
\pgfsetroundjoin%
\definecolor{currentfill}{rgb}{0.262138,0.242286,0.520837}%
\pgfsetfillcolor{currentfill}%
\pgfsetfillopacity{0.700000}%
\pgfsetlinewidth{0.000000pt}%
\definecolor{currentstroke}{rgb}{0.000000,0.000000,0.000000}%
\pgfsetstrokecolor{currentstroke}%
\pgfsetdash{}{0pt}%
\pgfpathmoveto{\pgfqpoint{4.919493in}{2.490607in}}%
\pgfpathlineto{\pgfqpoint{4.932863in}{2.491148in}}%
\pgfpathlineto{\pgfqpoint{4.946243in}{2.491807in}}%
\pgfpathlineto{\pgfqpoint{4.959633in}{2.492581in}}%
\pgfpathlineto{\pgfqpoint{4.973033in}{2.493473in}}%
\pgfpathlineto{\pgfqpoint{4.980316in}{2.503253in}}%
\pgfpathlineto{\pgfqpoint{4.987594in}{2.513020in}}%
\pgfpathlineto{\pgfqpoint{4.994867in}{2.522774in}}%
\pgfpathlineto{\pgfqpoint{5.002136in}{2.532515in}}%
\pgfpathlineto{\pgfqpoint{4.988744in}{2.531680in}}%
\pgfpathlineto{\pgfqpoint{4.975364in}{2.530962in}}%
\pgfpathlineto{\pgfqpoint{4.961994in}{2.530361in}}%
\pgfpathlineto{\pgfqpoint{4.948633in}{2.529876in}}%
\pgfpathlineto{\pgfqpoint{4.941356in}{2.520072in}}%
\pgfpathlineto{\pgfqpoint{4.934073in}{2.510260in}}%
\pgfpathlineto{\pgfqpoint{4.926786in}{2.500439in}}%
\pgfpathlineto{\pgfqpoint{4.919493in}{2.490607in}}%
\pgfpathclose%
\pgfusepath{fill}%
\end{pgfscope}%
\begin{pgfscope}%
\pgfpathrectangle{\pgfqpoint{1.254980in}{0.150000in}}{\pgfqpoint{5.490039in}{5.490039in}}%
\pgfusepath{clip}%
\pgfsetbuttcap%
\pgfsetroundjoin%
\definecolor{currentfill}{rgb}{0.220057,0.343307,0.549413}%
\pgfsetfillcolor{currentfill}%
\pgfsetfillopacity{0.700000}%
\pgfsetlinewidth{0.000000pt}%
\definecolor{currentstroke}{rgb}{0.000000,0.000000,0.000000}%
\pgfsetstrokecolor{currentstroke}%
\pgfsetdash{}{0pt}%
\pgfpathmoveto{\pgfqpoint{5.332877in}{2.710307in}}%
\pgfpathlineto{\pgfqpoint{5.346418in}{2.712702in}}%
\pgfpathlineto{\pgfqpoint{5.359971in}{2.715211in}}%
\pgfpathlineto{\pgfqpoint{5.373536in}{2.717832in}}%
\pgfpathlineto{\pgfqpoint{5.387113in}{2.720566in}}%
\pgfpathlineto{\pgfqpoint{5.394245in}{2.729601in}}%
\pgfpathlineto{\pgfqpoint{5.401373in}{2.738624in}}%
\pgfpathlineto{\pgfqpoint{5.408495in}{2.747637in}}%
\pgfpathlineto{\pgfqpoint{5.415613in}{2.756642in}}%
\pgfpathlineto{\pgfqpoint{5.402048in}{2.754045in}}%
\pgfpathlineto{\pgfqpoint{5.388495in}{2.751562in}}%
\pgfpathlineto{\pgfqpoint{5.374954in}{2.749191in}}%
\pgfpathlineto{\pgfqpoint{5.361425in}{2.746934in}}%
\pgfpathlineto{\pgfqpoint{5.354295in}{2.737785in}}%
\pgfpathlineto{\pgfqpoint{5.347161in}{2.728632in}}%
\pgfpathlineto{\pgfqpoint{5.340021in}{2.719473in}}%
\pgfpathlineto{\pgfqpoint{5.332877in}{2.710307in}}%
\pgfpathclose%
\pgfusepath{fill}%
\end{pgfscope}%
\begin{pgfscope}%
\pgfpathrectangle{\pgfqpoint{1.254980in}{0.150000in}}{\pgfqpoint{5.490039in}{5.490039in}}%
\pgfusepath{clip}%
\pgfsetbuttcap%
\pgfsetroundjoin%
\definecolor{currentfill}{rgb}{0.212395,0.359683,0.551710}%
\pgfsetfillcolor{currentfill}%
\pgfsetfillopacity{0.700000}%
\pgfsetlinewidth{0.000000pt}%
\definecolor{currentstroke}{rgb}{0.000000,0.000000,0.000000}%
\pgfsetstrokecolor{currentstroke}%
\pgfsetdash{}{0pt}%
\pgfpathmoveto{\pgfqpoint{5.415613in}{2.756642in}}%
\pgfpathlineto{\pgfqpoint{5.429190in}{2.759351in}}%
\pgfpathlineto{\pgfqpoint{5.442780in}{2.762172in}}%
\pgfpathlineto{\pgfqpoint{5.456383in}{2.765106in}}%
\pgfpathlineto{\pgfqpoint{5.469998in}{2.768153in}}%
\pgfpathlineto{\pgfqpoint{5.477098in}{2.777002in}}%
\pgfpathlineto{\pgfqpoint{5.484194in}{2.785843in}}%
\pgfpathlineto{\pgfqpoint{5.491284in}{2.794676in}}%
\pgfpathlineto{\pgfqpoint{5.498370in}{2.803503in}}%
\pgfpathlineto{\pgfqpoint{5.484768in}{2.800611in}}%
\pgfpathlineto{\pgfqpoint{5.471178in}{2.797831in}}%
\pgfpathlineto{\pgfqpoint{5.457601in}{2.795163in}}%
\pgfpathlineto{\pgfqpoint{5.444036in}{2.792608in}}%
\pgfpathlineto{\pgfqpoint{5.436938in}{2.783621in}}%
\pgfpathlineto{\pgfqpoint{5.429834in}{2.774632in}}%
\pgfpathlineto{\pgfqpoint{5.422726in}{2.765640in}}%
\pgfpathlineto{\pgfqpoint{5.415613in}{2.756642in}}%
\pgfpathclose%
\pgfusepath{fill}%
\end{pgfscope}%
\begin{pgfscope}%
\pgfpathrectangle{\pgfqpoint{1.254980in}{0.150000in}}{\pgfqpoint{5.490039in}{5.490039in}}%
\pgfusepath{clip}%
\pgfsetbuttcap%
\pgfsetroundjoin%
\definecolor{currentfill}{rgb}{0.269308,0.218818,0.509577}%
\pgfsetfillcolor{currentfill}%
\pgfsetfillopacity{0.700000}%
\pgfsetlinewidth{0.000000pt}%
\definecolor{currentstroke}{rgb}{0.000000,0.000000,0.000000}%
\pgfsetstrokecolor{currentstroke}%
\pgfsetdash{}{0pt}%
\pgfpathmoveto{\pgfqpoint{4.836863in}{2.450009in}}%
\pgfpathlineto{\pgfqpoint{4.850202in}{2.450122in}}%
\pgfpathlineto{\pgfqpoint{4.863550in}{2.450353in}}%
\pgfpathlineto{\pgfqpoint{4.876909in}{2.450700in}}%
\pgfpathlineto{\pgfqpoint{4.890277in}{2.451166in}}%
\pgfpathlineto{\pgfqpoint{4.897589in}{2.461045in}}%
\pgfpathlineto{\pgfqpoint{4.904895in}{2.470911in}}%
\pgfpathlineto{\pgfqpoint{4.912197in}{2.480765in}}%
\pgfpathlineto{\pgfqpoint{4.919493in}{2.490607in}}%
\pgfpathlineto{\pgfqpoint{4.906134in}{2.490183in}}%
\pgfpathlineto{\pgfqpoint{4.892785in}{2.489875in}}%
\pgfpathlineto{\pgfqpoint{4.879445in}{2.489685in}}%
\pgfpathlineto{\pgfqpoint{4.866116in}{2.489613in}}%
\pgfpathlineto{\pgfqpoint{4.858810in}{2.479724in}}%
\pgfpathlineto{\pgfqpoint{4.851499in}{2.469828in}}%
\pgfpathlineto{\pgfqpoint{4.844184in}{2.459923in}}%
\pgfpathlineto{\pgfqpoint{4.836863in}{2.450009in}}%
\pgfpathclose%
\pgfusepath{fill}%
\end{pgfscope}%
\begin{pgfscope}%
\pgfpathrectangle{\pgfqpoint{1.254980in}{0.150000in}}{\pgfqpoint{5.490039in}{5.490039in}}%
\pgfusepath{clip}%
\pgfsetbuttcap%
\pgfsetroundjoin%
\definecolor{currentfill}{rgb}{0.140536,0.530132,0.555659}%
\pgfsetfillcolor{currentfill}%
\pgfsetfillopacity{0.700000}%
\pgfsetlinewidth{0.000000pt}%
\definecolor{currentstroke}{rgb}{0.000000,0.000000,0.000000}%
\pgfsetstrokecolor{currentstroke}%
\pgfsetdash{}{0pt}%
\pgfpathmoveto{\pgfqpoint{2.742348in}{3.239773in}}%
\pgfpathlineto{\pgfqpoint{2.755606in}{3.219199in}}%
\pgfpathlineto{\pgfqpoint{2.768855in}{3.198835in}}%
\pgfpathlineto{\pgfqpoint{2.782097in}{3.178681in}}%
\pgfpathlineto{\pgfqpoint{2.795331in}{3.158734in}}%
\pgfpathlineto{\pgfqpoint{2.803418in}{3.164016in}}%
\pgfpathlineto{\pgfqpoint{2.811495in}{3.169428in}}%
\pgfpathlineto{\pgfqpoint{2.819563in}{3.174970in}}%
\pgfpathlineto{\pgfqpoint{2.827620in}{3.180641in}}%
\pgfpathlineto{\pgfqpoint{2.814414in}{3.200420in}}%
\pgfpathlineto{\pgfqpoint{2.801200in}{3.220406in}}%
\pgfpathlineto{\pgfqpoint{2.787979in}{3.240601in}}%
\pgfpathlineto{\pgfqpoint{2.774750in}{3.261006in}}%
\pgfpathlineto{\pgfqpoint{2.766664in}{3.255497in}}%
\pgfpathlineto{\pgfqpoint{2.758569in}{3.250121in}}%
\pgfpathlineto{\pgfqpoint{2.750464in}{3.244879in}}%
\pgfpathlineto{\pgfqpoint{2.742348in}{3.239773in}}%
\pgfpathclose%
\pgfusepath{fill}%
\end{pgfscope}%
\begin{pgfscope}%
\pgfpathrectangle{\pgfqpoint{1.254980in}{0.150000in}}{\pgfqpoint{5.490039in}{5.490039in}}%
\pgfusepath{clip}%
\pgfsetbuttcap%
\pgfsetroundjoin%
\definecolor{currentfill}{rgb}{0.203063,0.379716,0.553925}%
\pgfsetfillcolor{currentfill}%
\pgfsetfillopacity{0.700000}%
\pgfsetlinewidth{0.000000pt}%
\definecolor{currentstroke}{rgb}{0.000000,0.000000,0.000000}%
\pgfsetstrokecolor{currentstroke}%
\pgfsetdash{}{0pt}%
\pgfpathmoveto{\pgfqpoint{5.498370in}{2.803503in}}%
\pgfpathlineto{\pgfqpoint{5.511985in}{2.806506in}}%
\pgfpathlineto{\pgfqpoint{5.525612in}{2.809622in}}%
\pgfpathlineto{\pgfqpoint{5.539253in}{2.812850in}}%
\pgfpathlineto{\pgfqpoint{5.552906in}{2.816190in}}%
\pgfpathlineto{\pgfqpoint{5.559974in}{2.824848in}}%
\pgfpathlineto{\pgfqpoint{5.567037in}{2.833500in}}%
\pgfpathlineto{\pgfqpoint{5.574094in}{2.842147in}}%
\pgfpathlineto{\pgfqpoint{5.581148in}{2.850791in}}%
\pgfpathlineto{\pgfqpoint{5.567508in}{2.847622in}}%
\pgfpathlineto{\pgfqpoint{5.553881in}{2.844565in}}%
\pgfpathlineto{\pgfqpoint{5.540267in}{2.841620in}}%
\pgfpathlineto{\pgfqpoint{5.526666in}{2.838786in}}%
\pgfpathlineto{\pgfqpoint{5.519599in}{2.829965in}}%
\pgfpathlineto{\pgfqpoint{5.512527in}{2.821146in}}%
\pgfpathlineto{\pgfqpoint{5.505451in}{2.812325in}}%
\pgfpathlineto{\pgfqpoint{5.498370in}{2.803503in}}%
\pgfpathclose%
\pgfusepath{fill}%
\end{pgfscope}%
\begin{pgfscope}%
\pgfpathrectangle{\pgfqpoint{1.254980in}{0.150000in}}{\pgfqpoint{5.490039in}{5.490039in}}%
\pgfusepath{clip}%
\pgfsetbuttcap%
\pgfsetroundjoin%
\definecolor{currentfill}{rgb}{0.282910,0.105393,0.426902}%
\pgfsetfillcolor{currentfill}%
\pgfsetfillopacity{0.700000}%
\pgfsetlinewidth{0.000000pt}%
\definecolor{currentstroke}{rgb}{0.000000,0.000000,0.000000}%
\pgfsetstrokecolor{currentstroke}%
\pgfsetdash{}{0pt}%
\pgfpathmoveto{\pgfqpoint{4.070478in}{2.230977in}}%
\pgfpathlineto{\pgfqpoint{4.083590in}{2.225904in}}%
\pgfpathlineto{\pgfqpoint{4.096707in}{2.220961in}}%
\pgfpathlineto{\pgfqpoint{4.109829in}{2.216148in}}%
\pgfpathlineto{\pgfqpoint{4.122957in}{2.211464in}}%
\pgfpathlineto{\pgfqpoint{4.130514in}{2.220963in}}%
\pgfpathlineto{\pgfqpoint{4.138065in}{2.230486in}}%
\pgfpathlineto{\pgfqpoint{4.145612in}{2.240033in}}%
\pgfpathlineto{\pgfqpoint{4.153153in}{2.249604in}}%
\pgfpathlineto{\pgfqpoint{4.140036in}{2.254200in}}%
\pgfpathlineto{\pgfqpoint{4.126925in}{2.258927in}}%
\pgfpathlineto{\pgfqpoint{4.113819in}{2.263783in}}%
\pgfpathlineto{\pgfqpoint{4.100719in}{2.268769in}}%
\pgfpathlineto{\pgfqpoint{4.093166in}{2.259280in}}%
\pgfpathlineto{\pgfqpoint{4.085608in}{2.249817in}}%
\pgfpathlineto{\pgfqpoint{4.078046in}{2.240383in}}%
\pgfpathlineto{\pgfqpoint{4.070478in}{2.230977in}}%
\pgfpathclose%
\pgfusepath{fill}%
\end{pgfscope}%
\begin{pgfscope}%
\pgfpathrectangle{\pgfqpoint{1.254980in}{0.150000in}}{\pgfqpoint{5.490039in}{5.490039in}}%
\pgfusepath{clip}%
\pgfsetbuttcap%
\pgfsetroundjoin%
\definecolor{currentfill}{rgb}{0.274128,0.199721,0.498911}%
\pgfsetfillcolor{currentfill}%
\pgfsetfillopacity{0.700000}%
\pgfsetlinewidth{0.000000pt}%
\definecolor{currentstroke}{rgb}{0.000000,0.000000,0.000000}%
\pgfsetstrokecolor{currentstroke}%
\pgfsetdash{}{0pt}%
\pgfpathmoveto{\pgfqpoint{4.754241in}{2.410885in}}%
\pgfpathlineto{\pgfqpoint{4.767550in}{2.410550in}}%
\pgfpathlineto{\pgfqpoint{4.780869in}{2.410333in}}%
\pgfpathlineto{\pgfqpoint{4.794197in}{2.410234in}}%
\pgfpathlineto{\pgfqpoint{4.807535in}{2.410254in}}%
\pgfpathlineto{\pgfqpoint{4.814874in}{2.420209in}}%
\pgfpathlineto{\pgfqpoint{4.822209in}{2.430153in}}%
\pgfpathlineto{\pgfqpoint{4.829538in}{2.440086in}}%
\pgfpathlineto{\pgfqpoint{4.836863in}{2.450009in}}%
\pgfpathlineto{\pgfqpoint{4.823534in}{2.450014in}}%
\pgfpathlineto{\pgfqpoint{4.810215in}{2.450138in}}%
\pgfpathlineto{\pgfqpoint{4.796905in}{2.450379in}}%
\pgfpathlineto{\pgfqpoint{4.783605in}{2.450739in}}%
\pgfpathlineto{\pgfqpoint{4.776271in}{2.440785in}}%
\pgfpathlineto{\pgfqpoint{4.768932in}{2.430825in}}%
\pgfpathlineto{\pgfqpoint{4.761589in}{2.420859in}}%
\pgfpathlineto{\pgfqpoint{4.754241in}{2.410885in}}%
\pgfpathclose%
\pgfusepath{fill}%
\end{pgfscope}%
\begin{pgfscope}%
\pgfpathrectangle{\pgfqpoint{1.254980in}{0.150000in}}{\pgfqpoint{5.490039in}{5.490039in}}%
\pgfusepath{clip}%
\pgfsetbuttcap%
\pgfsetroundjoin%
\definecolor{currentfill}{rgb}{0.283229,0.120777,0.440584}%
\pgfsetfillcolor{currentfill}%
\pgfsetfillopacity{0.700000}%
\pgfsetlinewidth{0.000000pt}%
\definecolor{currentstroke}{rgb}{0.000000,0.000000,0.000000}%
\pgfsetstrokecolor{currentstroke}%
\pgfsetdash{}{0pt}%
\pgfpathmoveto{\pgfqpoint{4.288338in}{2.256517in}}%
\pgfpathlineto{\pgfqpoint{4.301499in}{2.253138in}}%
\pgfpathlineto{\pgfqpoint{4.314668in}{2.249885in}}%
\pgfpathlineto{\pgfqpoint{4.327844in}{2.246757in}}%
\pgfpathlineto{\pgfqpoint{4.341026in}{2.243754in}}%
\pgfpathlineto{\pgfqpoint{4.348513in}{2.253619in}}%
\pgfpathlineto{\pgfqpoint{4.355995in}{2.263495in}}%
\pgfpathlineto{\pgfqpoint{4.363473in}{2.273381in}}%
\pgfpathlineto{\pgfqpoint{4.370946in}{2.283275in}}%
\pgfpathlineto{\pgfqpoint{4.357773in}{2.286224in}}%
\pgfpathlineto{\pgfqpoint{4.344607in}{2.289297in}}%
\pgfpathlineto{\pgfqpoint{4.331448in}{2.292495in}}%
\pgfpathlineto{\pgfqpoint{4.318296in}{2.295819in}}%
\pgfpathlineto{\pgfqpoint{4.310814in}{2.285973in}}%
\pgfpathlineto{\pgfqpoint{4.303326in}{2.276141in}}%
\pgfpathlineto{\pgfqpoint{4.295834in}{2.266322in}}%
\pgfpathlineto{\pgfqpoint{4.288338in}{2.256517in}}%
\pgfpathclose%
\pgfusepath{fill}%
\end{pgfscope}%
\begin{pgfscope}%
\pgfpathrectangle{\pgfqpoint{1.254980in}{0.150000in}}{\pgfqpoint{5.490039in}{5.490039in}}%
\pgfusepath{clip}%
\pgfsetbuttcap%
\pgfsetroundjoin%
\definecolor{currentfill}{rgb}{0.283072,0.130895,0.449241}%
\pgfsetfillcolor{currentfill}%
\pgfsetfillopacity{0.700000}%
\pgfsetlinewidth{0.000000pt}%
\definecolor{currentstroke}{rgb}{0.000000,0.000000,0.000000}%
\pgfsetstrokecolor{currentstroke}%
\pgfsetdash{}{0pt}%
\pgfpathmoveto{\pgfqpoint{3.664920in}{2.291197in}}%
\pgfpathlineto{\pgfqpoint{3.677983in}{2.282491in}}%
\pgfpathlineto{\pgfqpoint{3.691048in}{2.273928in}}%
\pgfpathlineto{\pgfqpoint{3.704115in}{2.265508in}}%
\pgfpathlineto{\pgfqpoint{3.717184in}{2.257229in}}%
\pgfpathlineto{\pgfqpoint{3.724882in}{2.265540in}}%
\pgfpathlineto{\pgfqpoint{3.732575in}{2.273907in}}%
\pgfpathlineto{\pgfqpoint{3.740261in}{2.282330in}}%
\pgfpathlineto{\pgfqpoint{3.747942in}{2.290808in}}%
\pgfpathlineto{\pgfqpoint{3.734888in}{2.298951in}}%
\pgfpathlineto{\pgfqpoint{3.721836in}{2.307235in}}%
\pgfpathlineto{\pgfqpoint{3.708786in}{2.315662in}}%
\pgfpathlineto{\pgfqpoint{3.695739in}{2.324231in}}%
\pgfpathlineto{\pgfqpoint{3.688043in}{2.315883in}}%
\pgfpathlineto{\pgfqpoint{3.680341in}{2.307594in}}%
\pgfpathlineto{\pgfqpoint{3.672634in}{2.299365in}}%
\pgfpathlineto{\pgfqpoint{3.664920in}{2.291197in}}%
\pgfpathclose%
\pgfusepath{fill}%
\end{pgfscope}%
\begin{pgfscope}%
\pgfpathrectangle{\pgfqpoint{1.254980in}{0.150000in}}{\pgfqpoint{5.490039in}{5.490039in}}%
\pgfusepath{clip}%
\pgfsetbuttcap%
\pgfsetroundjoin%
\definecolor{currentfill}{rgb}{0.194100,0.399323,0.555565}%
\pgfsetfillcolor{currentfill}%
\pgfsetfillopacity{0.700000}%
\pgfsetlinewidth{0.000000pt}%
\definecolor{currentstroke}{rgb}{0.000000,0.000000,0.000000}%
\pgfsetstrokecolor{currentstroke}%
\pgfsetdash{}{0pt}%
\pgfpathmoveto{\pgfqpoint{5.581148in}{2.850791in}}%
\pgfpathlineto{\pgfqpoint{5.594800in}{2.854071in}}%
\pgfpathlineto{\pgfqpoint{5.608466in}{2.857463in}}%
\pgfpathlineto{\pgfqpoint{5.622145in}{2.860966in}}%
\pgfpathlineto{\pgfqpoint{5.635837in}{2.864580in}}%
\pgfpathlineto{\pgfqpoint{5.642871in}{2.873042in}}%
\pgfpathlineto{\pgfqpoint{5.649901in}{2.881501in}}%
\pgfpathlineto{\pgfqpoint{5.656925in}{2.889959in}}%
\pgfpathlineto{\pgfqpoint{5.663945in}{2.898418in}}%
\pgfpathlineto{\pgfqpoint{5.650268in}{2.894991in}}%
\pgfpathlineto{\pgfqpoint{5.636603in}{2.891675in}}%
\pgfpathlineto{\pgfqpoint{5.622952in}{2.888471in}}%
\pgfpathlineto{\pgfqpoint{5.609314in}{2.885377in}}%
\pgfpathlineto{\pgfqpoint{5.602279in}{2.876725in}}%
\pgfpathlineto{\pgfqpoint{5.595240in}{2.868078in}}%
\pgfpathlineto{\pgfqpoint{5.588196in}{2.859434in}}%
\pgfpathlineto{\pgfqpoint{5.581148in}{2.850791in}}%
\pgfpathclose%
\pgfusepath{fill}%
\end{pgfscope}%
\begin{pgfscope}%
\pgfpathrectangle{\pgfqpoint{1.254980in}{0.150000in}}{\pgfqpoint{5.490039in}{5.490039in}}%
\pgfusepath{clip}%
\pgfsetbuttcap%
\pgfsetroundjoin%
\definecolor{currentfill}{rgb}{0.278826,0.175490,0.483397}%
\pgfsetfillcolor{currentfill}%
\pgfsetfillopacity{0.700000}%
\pgfsetlinewidth{0.000000pt}%
\definecolor{currentstroke}{rgb}{0.000000,0.000000,0.000000}%
\pgfsetstrokecolor{currentstroke}%
\pgfsetdash{}{0pt}%
\pgfpathmoveto{\pgfqpoint{3.477179in}{2.376823in}}%
\pgfpathlineto{\pgfqpoint{3.490242in}{2.366200in}}%
\pgfpathlineto{\pgfqpoint{3.503306in}{2.355727in}}%
\pgfpathlineto{\pgfqpoint{3.516371in}{2.345405in}}%
\pgfpathlineto{\pgfqpoint{3.529436in}{2.335232in}}%
\pgfpathlineto{\pgfqpoint{3.537208in}{2.342830in}}%
\pgfpathlineto{\pgfqpoint{3.544974in}{2.350502in}}%
\pgfpathlineto{\pgfqpoint{3.552733in}{2.358244in}}%
\pgfpathlineto{\pgfqpoint{3.560485in}{2.366058in}}%
\pgfpathlineto{\pgfqpoint{3.547437in}{2.376077in}}%
\pgfpathlineto{\pgfqpoint{3.534390in}{2.386246in}}%
\pgfpathlineto{\pgfqpoint{3.521344in}{2.396565in}}%
\pgfpathlineto{\pgfqpoint{3.508299in}{2.407035in}}%
\pgfpathlineto{\pgfqpoint{3.500529in}{2.399369in}}%
\pgfpathlineto{\pgfqpoint{3.492752in}{2.391778in}}%
\pgfpathlineto{\pgfqpoint{3.484969in}{2.384262in}}%
\pgfpathlineto{\pgfqpoint{3.477179in}{2.376823in}}%
\pgfpathclose%
\pgfusepath{fill}%
\end{pgfscope}%
\begin{pgfscope}%
\pgfpathrectangle{\pgfqpoint{1.254980in}{0.150000in}}{\pgfqpoint{5.490039in}{5.490039in}}%
\pgfusepath{clip}%
\pgfsetbuttcap%
\pgfsetroundjoin%
\definecolor{currentfill}{rgb}{0.185556,0.418570,0.556753}%
\pgfsetfillcolor{currentfill}%
\pgfsetfillopacity{0.700000}%
\pgfsetlinewidth{0.000000pt}%
\definecolor{currentstroke}{rgb}{0.000000,0.000000,0.000000}%
\pgfsetstrokecolor{currentstroke}%
\pgfsetdash{}{0pt}%
\pgfpathmoveto{\pgfqpoint{5.663945in}{2.898418in}}%
\pgfpathlineto{\pgfqpoint{5.677636in}{2.901956in}}%
\pgfpathlineto{\pgfqpoint{5.691340in}{2.905604in}}%
\pgfpathlineto{\pgfqpoint{5.705058in}{2.909364in}}%
\pgfpathlineto{\pgfqpoint{5.718789in}{2.913234in}}%
\pgfpathlineto{\pgfqpoint{5.725789in}{2.921498in}}%
\pgfpathlineto{\pgfqpoint{5.732784in}{2.929763in}}%
\pgfpathlineto{\pgfqpoint{5.739775in}{2.938031in}}%
\pgfpathlineto{\pgfqpoint{5.746762in}{2.946304in}}%
\pgfpathlineto{\pgfqpoint{5.733046in}{2.942638in}}%
\pgfpathlineto{\pgfqpoint{5.719344in}{2.939082in}}%
\pgfpathlineto{\pgfqpoint{5.705655in}{2.935637in}}%
\pgfpathlineto{\pgfqpoint{5.691980in}{2.932302in}}%
\pgfpathlineto{\pgfqpoint{5.684978in}{2.923820in}}%
\pgfpathlineto{\pgfqpoint{5.677972in}{2.915346in}}%
\pgfpathlineto{\pgfqpoint{5.670961in}{2.906880in}}%
\pgfpathlineto{\pgfqpoint{5.663945in}{2.898418in}}%
\pgfpathclose%
\pgfusepath{fill}%
\end{pgfscope}%
\begin{pgfscope}%
\pgfpathrectangle{\pgfqpoint{1.254980in}{0.150000in}}{\pgfqpoint{5.490039in}{5.490039in}}%
\pgfusepath{clip}%
\pgfsetbuttcap%
\pgfsetroundjoin%
\definecolor{currentfill}{rgb}{0.277134,0.185228,0.489898}%
\pgfsetfillcolor{currentfill}%
\pgfsetfillopacity{0.700000}%
\pgfsetlinewidth{0.000000pt}%
\definecolor{currentstroke}{rgb}{0.000000,0.000000,0.000000}%
\pgfsetstrokecolor{currentstroke}%
\pgfsetdash{}{0pt}%
\pgfpathmoveto{\pgfqpoint{4.671621in}{2.373407in}}%
\pgfpathlineto{\pgfqpoint{4.684902in}{2.372604in}}%
\pgfpathlineto{\pgfqpoint{4.698193in}{2.371920in}}%
\pgfpathlineto{\pgfqpoint{4.711493in}{2.371355in}}%
\pgfpathlineto{\pgfqpoint{4.724801in}{2.370910in}}%
\pgfpathlineto{\pgfqpoint{4.732168in}{2.380916in}}%
\pgfpathlineto{\pgfqpoint{4.739531in}{2.390914in}}%
\pgfpathlineto{\pgfqpoint{4.746888in}{2.400904in}}%
\pgfpathlineto{\pgfqpoint{4.754241in}{2.410885in}}%
\pgfpathlineto{\pgfqpoint{4.740941in}{2.411339in}}%
\pgfpathlineto{\pgfqpoint{4.727650in}{2.411913in}}%
\pgfpathlineto{\pgfqpoint{4.714368in}{2.412605in}}%
\pgfpathlineto{\pgfqpoint{4.701096in}{2.413417in}}%
\pgfpathlineto{\pgfqpoint{4.693734in}{2.403421in}}%
\pgfpathlineto{\pgfqpoint{4.686368in}{2.393421in}}%
\pgfpathlineto{\pgfqpoint{4.678997in}{2.383417in}}%
\pgfpathlineto{\pgfqpoint{4.671621in}{2.373407in}}%
\pgfpathclose%
\pgfusepath{fill}%
\end{pgfscope}%
\begin{pgfscope}%
\pgfpathrectangle{\pgfqpoint{1.254980in}{0.150000in}}{\pgfqpoint{5.490039in}{5.490039in}}%
\pgfusepath{clip}%
\pgfsetbuttcap%
\pgfsetroundjoin%
\definecolor{currentfill}{rgb}{0.175841,0.441290,0.557685}%
\pgfsetfillcolor{currentfill}%
\pgfsetfillopacity{0.700000}%
\pgfsetlinewidth{0.000000pt}%
\definecolor{currentstroke}{rgb}{0.000000,0.000000,0.000000}%
\pgfsetstrokecolor{currentstroke}%
\pgfsetdash{}{0pt}%
\pgfpathmoveto{\pgfqpoint{5.746762in}{2.946304in}}%
\pgfpathlineto{\pgfqpoint{5.760491in}{2.950081in}}%
\pgfpathlineto{\pgfqpoint{5.774234in}{2.953968in}}%
\pgfpathlineto{\pgfqpoint{5.787990in}{2.957965in}}%
\pgfpathlineto{\pgfqpoint{5.801761in}{2.962073in}}%
\pgfpathlineto{\pgfqpoint{5.808726in}{2.970139in}}%
\pgfpathlineto{\pgfqpoint{5.815687in}{2.978210in}}%
\pgfpathlineto{\pgfqpoint{5.822644in}{2.986290in}}%
\pgfpathlineto{\pgfqpoint{5.829596in}{2.994379in}}%
\pgfpathlineto{\pgfqpoint{5.815842in}{2.990492in}}%
\pgfpathlineto{\pgfqpoint{5.802102in}{2.986715in}}%
\pgfpathlineto{\pgfqpoint{5.788376in}{2.983047in}}%
\pgfpathlineto{\pgfqpoint{5.774663in}{2.979490in}}%
\pgfpathlineto{\pgfqpoint{5.767694in}{2.971175in}}%
\pgfpathlineto{\pgfqpoint{5.760721in}{2.962874in}}%
\pgfpathlineto{\pgfqpoint{5.753744in}{2.954584in}}%
\pgfpathlineto{\pgfqpoint{5.746762in}{2.946304in}}%
\pgfpathclose%
\pgfusepath{fill}%
\end{pgfscope}%
\begin{pgfscope}%
\pgfpathrectangle{\pgfqpoint{1.254980in}{0.150000in}}{\pgfqpoint{5.490039in}{5.490039in}}%
\pgfusepath{clip}%
\pgfsetbuttcap%
\pgfsetroundjoin%
\definecolor{currentfill}{rgb}{0.227802,0.326594,0.546532}%
\pgfsetfillcolor{currentfill}%
\pgfsetfillopacity{0.700000}%
\pgfsetlinewidth{0.000000pt}%
\definecolor{currentstroke}{rgb}{0.000000,0.000000,0.000000}%
\pgfsetstrokecolor{currentstroke}%
\pgfsetdash{}{0pt}%
\pgfpathmoveto{\pgfqpoint{3.079303in}{2.714668in}}%
\pgfpathlineto{\pgfqpoint{3.092427in}{2.699308in}}%
\pgfpathlineto{\pgfqpoint{3.105548in}{2.684125in}}%
\pgfpathlineto{\pgfqpoint{3.118665in}{2.669116in}}%
\pgfpathlineto{\pgfqpoint{3.131779in}{2.654281in}}%
\pgfpathlineto{\pgfqpoint{3.139723in}{2.660330in}}%
\pgfpathlineto{\pgfqpoint{3.147659in}{2.666484in}}%
\pgfpathlineto{\pgfqpoint{3.155587in}{2.672744in}}%
\pgfpathlineto{\pgfqpoint{3.163506in}{2.679107in}}%
\pgfpathlineto{\pgfqpoint{3.150415in}{2.693767in}}%
\pgfpathlineto{\pgfqpoint{3.137321in}{2.708600in}}%
\pgfpathlineto{\pgfqpoint{3.124224in}{2.723607in}}%
\pgfpathlineto{\pgfqpoint{3.111124in}{2.738790in}}%
\pgfpathlineto{\pgfqpoint{3.103181in}{2.732597in}}%
\pgfpathlineto{\pgfqpoint{3.095230in}{2.726511in}}%
\pgfpathlineto{\pgfqpoint{3.087271in}{2.720534in}}%
\pgfpathlineto{\pgfqpoint{3.079303in}{2.714668in}}%
\pgfpathclose%
\pgfusepath{fill}%
\end{pgfscope}%
\begin{pgfscope}%
\pgfpathrectangle{\pgfqpoint{1.254980in}{0.150000in}}{\pgfqpoint{5.490039in}{5.490039in}}%
\pgfusepath{clip}%
\pgfsetbuttcap%
\pgfsetroundjoin%
\definecolor{currentfill}{rgb}{0.214298,0.355619,0.551184}%
\pgfsetfillcolor{currentfill}%
\pgfsetfillopacity{0.700000}%
\pgfsetlinewidth{0.000000pt}%
\definecolor{currentstroke}{rgb}{0.000000,0.000000,0.000000}%
\pgfsetstrokecolor{currentstroke}%
\pgfsetdash{}{0pt}%
\pgfpathmoveto{\pgfqpoint{3.026770in}{2.777882in}}%
\pgfpathlineto{\pgfqpoint{3.039909in}{2.761809in}}%
\pgfpathlineto{\pgfqpoint{3.053044in}{2.745917in}}%
\pgfpathlineto{\pgfqpoint{3.066176in}{2.730203in}}%
\pgfpathlineto{\pgfqpoint{3.079303in}{2.714668in}}%
\pgfpathlineto{\pgfqpoint{3.087271in}{2.720534in}}%
\pgfpathlineto{\pgfqpoint{3.095230in}{2.726511in}}%
\pgfpathlineto{\pgfqpoint{3.103181in}{2.732597in}}%
\pgfpathlineto{\pgfqpoint{3.111124in}{2.738790in}}%
\pgfpathlineto{\pgfqpoint{3.098020in}{2.754149in}}%
\pgfpathlineto{\pgfqpoint{3.084913in}{2.769686in}}%
\pgfpathlineto{\pgfqpoint{3.071802in}{2.785401in}}%
\pgfpathlineto{\pgfqpoint{3.058687in}{2.801296in}}%
\pgfpathlineto{\pgfqpoint{3.050721in}{2.795274in}}%
\pgfpathlineto{\pgfqpoint{3.042746in}{2.789363in}}%
\pgfpathlineto{\pgfqpoint{3.034762in}{2.783565in}}%
\pgfpathlineto{\pgfqpoint{3.026770in}{2.777882in}}%
\pgfpathclose%
\pgfusepath{fill}%
\end{pgfscope}%
\begin{pgfscope}%
\pgfpathrectangle{\pgfqpoint{1.254980in}{0.150000in}}{\pgfqpoint{5.490039in}{5.490039in}}%
\pgfusepath{clip}%
\pgfsetbuttcap%
\pgfsetroundjoin%
\definecolor{currentfill}{rgb}{0.168126,0.459988,0.558082}%
\pgfsetfillcolor{currentfill}%
\pgfsetfillopacity{0.700000}%
\pgfsetlinewidth{0.000000pt}%
\definecolor{currentstroke}{rgb}{0.000000,0.000000,0.000000}%
\pgfsetstrokecolor{currentstroke}%
\pgfsetdash{}{0pt}%
\pgfpathmoveto{\pgfqpoint{5.829596in}{2.994379in}}%
\pgfpathlineto{\pgfqpoint{5.843363in}{2.998376in}}%
\pgfpathlineto{\pgfqpoint{5.857145in}{3.002483in}}%
\pgfpathlineto{\pgfqpoint{5.870941in}{3.006700in}}%
\pgfpathlineto{\pgfqpoint{5.884750in}{3.011027in}}%
\pgfpathlineto{\pgfqpoint{5.891681in}{3.018897in}}%
\pgfpathlineto{\pgfqpoint{5.898607in}{3.026778in}}%
\pgfpathlineto{\pgfqpoint{5.905529in}{3.034672in}}%
\pgfpathlineto{\pgfqpoint{5.912447in}{3.042582in}}%
\pgfpathlineto{\pgfqpoint{5.898655in}{3.038492in}}%
\pgfpathlineto{\pgfqpoint{5.884877in}{3.034512in}}%
\pgfpathlineto{\pgfqpoint{5.871113in}{3.030642in}}%
\pgfpathlineto{\pgfqpoint{5.857363in}{3.026881in}}%
\pgfpathlineto{\pgfqpoint{5.850427in}{3.018729in}}%
\pgfpathlineto{\pgfqpoint{5.843488in}{3.010596in}}%
\pgfpathlineto{\pgfqpoint{5.836544in}{3.002480in}}%
\pgfpathlineto{\pgfqpoint{5.829596in}{2.994379in}}%
\pgfpathclose%
\pgfusepath{fill}%
\end{pgfscope}%
\begin{pgfscope}%
\pgfpathrectangle{\pgfqpoint{1.254980in}{0.150000in}}{\pgfqpoint{5.490039in}{5.490039in}}%
\pgfusepath{clip}%
\pgfsetbuttcap%
\pgfsetroundjoin%
\definecolor{currentfill}{rgb}{0.239346,0.300855,0.540844}%
\pgfsetfillcolor{currentfill}%
\pgfsetfillopacity{0.700000}%
\pgfsetlinewidth{0.000000pt}%
\definecolor{currentstroke}{rgb}{0.000000,0.000000,0.000000}%
\pgfsetstrokecolor{currentstroke}%
\pgfsetdash{}{0pt}%
\pgfpathmoveto{\pgfqpoint{3.131779in}{2.654281in}}%
\pgfpathlineto{\pgfqpoint{3.144890in}{2.639618in}}%
\pgfpathlineto{\pgfqpoint{3.157998in}{2.625127in}}%
\pgfpathlineto{\pgfqpoint{3.171104in}{2.610806in}}%
\pgfpathlineto{\pgfqpoint{3.184207in}{2.596655in}}%
\pgfpathlineto{\pgfqpoint{3.192128in}{2.602885in}}%
\pgfpathlineto{\pgfqpoint{3.200041in}{2.609217in}}%
\pgfpathlineto{\pgfqpoint{3.207947in}{2.615649in}}%
\pgfpathlineto{\pgfqpoint{3.215844in}{2.622181in}}%
\pgfpathlineto{\pgfqpoint{3.202763in}{2.636157in}}%
\pgfpathlineto{\pgfqpoint{3.189680in}{2.650303in}}%
\pgfpathlineto{\pgfqpoint{3.176594in}{2.664619in}}%
\pgfpathlineto{\pgfqpoint{3.163506in}{2.679107in}}%
\pgfpathlineto{\pgfqpoint{3.155587in}{2.672744in}}%
\pgfpathlineto{\pgfqpoint{3.147659in}{2.666484in}}%
\pgfpathlineto{\pgfqpoint{3.139723in}{2.660330in}}%
\pgfpathlineto{\pgfqpoint{3.131779in}{2.654281in}}%
\pgfpathclose%
\pgfusepath{fill}%
\end{pgfscope}%
\begin{pgfscope}%
\pgfpathrectangle{\pgfqpoint{1.254980in}{0.150000in}}{\pgfqpoint{5.490039in}{5.490039in}}%
\pgfusepath{clip}%
\pgfsetbuttcap%
\pgfsetroundjoin%
\definecolor{currentfill}{rgb}{0.280255,0.165693,0.476498}%
\pgfsetfillcolor{currentfill}%
\pgfsetfillopacity{0.700000}%
\pgfsetlinewidth{0.000000pt}%
\definecolor{currentstroke}{rgb}{0.000000,0.000000,0.000000}%
\pgfsetstrokecolor{currentstroke}%
\pgfsetdash{}{0pt}%
\pgfpathmoveto{\pgfqpoint{4.588997in}{2.337758in}}%
\pgfpathlineto{\pgfqpoint{4.602253in}{2.336466in}}%
\pgfpathlineto{\pgfqpoint{4.615517in}{2.335296in}}%
\pgfpathlineto{\pgfqpoint{4.628790in}{2.334245in}}%
\pgfpathlineto{\pgfqpoint{4.642071in}{2.333314in}}%
\pgfpathlineto{\pgfqpoint{4.649466in}{2.343346in}}%
\pgfpathlineto{\pgfqpoint{4.656855in}{2.353372in}}%
\pgfpathlineto{\pgfqpoint{4.664241in}{2.363392in}}%
\pgfpathlineto{\pgfqpoint{4.671621in}{2.373407in}}%
\pgfpathlineto{\pgfqpoint{4.658348in}{2.374330in}}%
\pgfpathlineto{\pgfqpoint{4.645084in}{2.375374in}}%
\pgfpathlineto{\pgfqpoint{4.631829in}{2.376537in}}%
\pgfpathlineto{\pgfqpoint{4.618583in}{2.377822in}}%
\pgfpathlineto{\pgfqpoint{4.611193in}{2.367808in}}%
\pgfpathlineto{\pgfqpoint{4.603799in}{2.357793in}}%
\pgfpathlineto{\pgfqpoint{4.596401in}{2.347777in}}%
\pgfpathlineto{\pgfqpoint{4.588997in}{2.337758in}}%
\pgfpathclose%
\pgfusepath{fill}%
\end{pgfscope}%
\begin{pgfscope}%
\pgfpathrectangle{\pgfqpoint{1.254980in}{0.150000in}}{\pgfqpoint{5.490039in}{5.490039in}}%
\pgfusepath{clip}%
\pgfsetbuttcap%
\pgfsetroundjoin%
\definecolor{currentfill}{rgb}{0.201239,0.383670,0.554294}%
\pgfsetfillcolor{currentfill}%
\pgfsetfillopacity{0.700000}%
\pgfsetlinewidth{0.000000pt}%
\definecolor{currentstroke}{rgb}{0.000000,0.000000,0.000000}%
\pgfsetstrokecolor{currentstroke}%
\pgfsetdash{}{0pt}%
\pgfpathmoveto{\pgfqpoint{2.974169in}{2.843993in}}%
\pgfpathlineto{\pgfqpoint{2.987326in}{2.827190in}}%
\pgfpathlineto{\pgfqpoint{3.000479in}{2.810571in}}%
\pgfpathlineto{\pgfqpoint{3.013626in}{2.794135in}}%
\pgfpathlineto{\pgfqpoint{3.026770in}{2.777882in}}%
\pgfpathlineto{\pgfqpoint{3.034762in}{2.783565in}}%
\pgfpathlineto{\pgfqpoint{3.042746in}{2.789363in}}%
\pgfpathlineto{\pgfqpoint{3.050721in}{2.795274in}}%
\pgfpathlineto{\pgfqpoint{3.058687in}{2.801296in}}%
\pgfpathlineto{\pgfqpoint{3.045568in}{2.817372in}}%
\pgfpathlineto{\pgfqpoint{3.032445in}{2.833629in}}%
\pgfpathlineto{\pgfqpoint{3.019318in}{2.850070in}}%
\pgfpathlineto{\pgfqpoint{3.006186in}{2.866695in}}%
\pgfpathlineto{\pgfqpoint{2.998195in}{2.860844in}}%
\pgfpathlineto{\pgfqpoint{2.990196in}{2.855109in}}%
\pgfpathlineto{\pgfqpoint{2.982187in}{2.849492in}}%
\pgfpathlineto{\pgfqpoint{2.974169in}{2.843993in}}%
\pgfpathclose%
\pgfusepath{fill}%
\end{pgfscope}%
\begin{pgfscope}%
\pgfpathrectangle{\pgfqpoint{1.254980in}{0.150000in}}{\pgfqpoint{5.490039in}{5.490039in}}%
\pgfusepath{clip}%
\pgfsetbuttcap%
\pgfsetroundjoin%
\definecolor{currentfill}{rgb}{0.248629,0.278775,0.534556}%
\pgfsetfillcolor{currentfill}%
\pgfsetfillopacity{0.700000}%
\pgfsetlinewidth{0.000000pt}%
\definecolor{currentstroke}{rgb}{0.000000,0.000000,0.000000}%
\pgfsetstrokecolor{currentstroke}%
\pgfsetdash{}{0pt}%
\pgfpathmoveto{\pgfqpoint{3.184207in}{2.596655in}}%
\pgfpathlineto{\pgfqpoint{3.197308in}{2.582673in}}%
\pgfpathlineto{\pgfqpoint{3.210406in}{2.568859in}}%
\pgfpathlineto{\pgfqpoint{3.223502in}{2.555211in}}%
\pgfpathlineto{\pgfqpoint{3.236597in}{2.541729in}}%
\pgfpathlineto{\pgfqpoint{3.244496in}{2.548138in}}%
\pgfpathlineto{\pgfqpoint{3.252387in}{2.554646in}}%
\pgfpathlineto{\pgfqpoint{3.260270in}{2.561250in}}%
\pgfpathlineto{\pgfqpoint{3.268146in}{2.567949in}}%
\pgfpathlineto{\pgfqpoint{3.255073in}{2.581258in}}%
\pgfpathlineto{\pgfqpoint{3.241999in}{2.594732in}}%
\pgfpathlineto{\pgfqpoint{3.228922in}{2.608372in}}%
\pgfpathlineto{\pgfqpoint{3.215844in}{2.622181in}}%
\pgfpathlineto{\pgfqpoint{3.207947in}{2.615649in}}%
\pgfpathlineto{\pgfqpoint{3.200041in}{2.609217in}}%
\pgfpathlineto{\pgfqpoint{3.192128in}{2.602885in}}%
\pgfpathlineto{\pgfqpoint{3.184207in}{2.596655in}}%
\pgfpathclose%
\pgfusepath{fill}%
\end{pgfscope}%
\begin{pgfscope}%
\pgfpathrectangle{\pgfqpoint{1.254980in}{0.150000in}}{\pgfqpoint{5.490039in}{5.490039in}}%
\pgfusepath{clip}%
\pgfsetbuttcap%
\pgfsetroundjoin%
\definecolor{currentfill}{rgb}{0.283091,0.110553,0.431554}%
\pgfsetfillcolor{currentfill}%
\pgfsetfillopacity{0.700000}%
\pgfsetlinewidth{0.000000pt}%
\definecolor{currentstroke}{rgb}{0.000000,0.000000,0.000000}%
\pgfsetstrokecolor{currentstroke}%
\pgfsetdash{}{0pt}%
\pgfpathmoveto{\pgfqpoint{4.205679in}{2.232502in}}%
\pgfpathlineto{\pgfqpoint{4.218825in}{2.228546in}}%
\pgfpathlineto{\pgfqpoint{4.231978in}{2.224718in}}%
\pgfpathlineto{\pgfqpoint{4.245137in}{2.221016in}}%
\pgfpathlineto{\pgfqpoint{4.258302in}{2.217440in}}%
\pgfpathlineto{\pgfqpoint{4.265818in}{2.227187in}}%
\pgfpathlineto{\pgfqpoint{4.273329in}{2.236950in}}%
\pgfpathlineto{\pgfqpoint{4.280836in}{2.246726in}}%
\pgfpathlineto{\pgfqpoint{4.288338in}{2.256517in}}%
\pgfpathlineto{\pgfqpoint{4.275182in}{2.260022in}}%
\pgfpathlineto{\pgfqpoint{4.262033in}{2.263653in}}%
\pgfpathlineto{\pgfqpoint{4.248891in}{2.267411in}}%
\pgfpathlineto{\pgfqpoint{4.235755in}{2.271296in}}%
\pgfpathlineto{\pgfqpoint{4.228243in}{2.261570in}}%
\pgfpathlineto{\pgfqpoint{4.220727in}{2.251862in}}%
\pgfpathlineto{\pgfqpoint{4.213205in}{2.242173in}}%
\pgfpathlineto{\pgfqpoint{4.205679in}{2.232502in}}%
\pgfpathclose%
\pgfusepath{fill}%
\end{pgfscope}%
\begin{pgfscope}%
\pgfpathrectangle{\pgfqpoint{1.254980in}{0.150000in}}{\pgfqpoint{5.490039in}{5.490039in}}%
\pgfusepath{clip}%
\pgfsetbuttcap%
\pgfsetroundjoin%
\definecolor{currentfill}{rgb}{0.282910,0.105393,0.426902}%
\pgfsetfillcolor{currentfill}%
\pgfsetfillopacity{0.700000}%
\pgfsetlinewidth{0.000000pt}%
\definecolor{currentstroke}{rgb}{0.000000,0.000000,0.000000}%
\pgfsetstrokecolor{currentstroke}%
\pgfsetdash{}{0pt}%
\pgfpathmoveto{\pgfqpoint{3.852486in}{2.230690in}}%
\pgfpathlineto{\pgfqpoint{3.865569in}{2.223795in}}%
\pgfpathlineto{\pgfqpoint{3.878656in}{2.217036in}}%
\pgfpathlineto{\pgfqpoint{3.891747in}{2.210413in}}%
\pgfpathlineto{\pgfqpoint{3.904841in}{2.203925in}}%
\pgfpathlineto{\pgfqpoint{3.912475in}{2.212833in}}%
\pgfpathlineto{\pgfqpoint{3.920103in}{2.221782in}}%
\pgfpathlineto{\pgfqpoint{3.927726in}{2.230772in}}%
\pgfpathlineto{\pgfqpoint{3.935343in}{2.239802in}}%
\pgfpathlineto{\pgfqpoint{3.922262in}{2.246171in}}%
\pgfpathlineto{\pgfqpoint{3.909184in}{2.252675in}}%
\pgfpathlineto{\pgfqpoint{3.896111in}{2.259315in}}%
\pgfpathlineto{\pgfqpoint{3.883041in}{2.266091in}}%
\pgfpathlineto{\pgfqpoint{3.875410in}{2.257174in}}%
\pgfpathlineto{\pgfqpoint{3.867774in}{2.248301in}}%
\pgfpathlineto{\pgfqpoint{3.860133in}{2.239473in}}%
\pgfpathlineto{\pgfqpoint{3.852486in}{2.230690in}}%
\pgfpathclose%
\pgfusepath{fill}%
\end{pgfscope}%
\begin{pgfscope}%
\pgfpathrectangle{\pgfqpoint{1.254980in}{0.150000in}}{\pgfqpoint{5.490039in}{5.490039in}}%
\pgfusepath{clip}%
\pgfsetbuttcap%
\pgfsetroundjoin%
\definecolor{currentfill}{rgb}{0.188923,0.410910,0.556326}%
\pgfsetfillcolor{currentfill}%
\pgfsetfillopacity{0.700000}%
\pgfsetlinewidth{0.000000pt}%
\definecolor{currentstroke}{rgb}{0.000000,0.000000,0.000000}%
\pgfsetstrokecolor{currentstroke}%
\pgfsetdash{}{0pt}%
\pgfpathmoveto{\pgfqpoint{2.921491in}{2.913076in}}%
\pgfpathlineto{\pgfqpoint{2.934668in}{2.895522in}}%
\pgfpathlineto{\pgfqpoint{2.947840in}{2.878158in}}%
\pgfpathlineto{\pgfqpoint{2.961007in}{2.860982in}}%
\pgfpathlineto{\pgfqpoint{2.974169in}{2.843993in}}%
\pgfpathlineto{\pgfqpoint{2.982187in}{2.849492in}}%
\pgfpathlineto{\pgfqpoint{2.990196in}{2.855109in}}%
\pgfpathlineto{\pgfqpoint{2.998195in}{2.860844in}}%
\pgfpathlineto{\pgfqpoint{3.006186in}{2.866695in}}%
\pgfpathlineto{\pgfqpoint{2.993050in}{2.883505in}}%
\pgfpathlineto{\pgfqpoint{2.979908in}{2.900501in}}%
\pgfpathlineto{\pgfqpoint{2.966762in}{2.917685in}}%
\pgfpathlineto{\pgfqpoint{2.953611in}{2.935059in}}%
\pgfpathlineto{\pgfqpoint{2.945595in}{2.929381in}}%
\pgfpathlineto{\pgfqpoint{2.937569in}{2.923824in}}%
\pgfpathlineto{\pgfqpoint{2.929535in}{2.918388in}}%
\pgfpathlineto{\pgfqpoint{2.921491in}{2.913076in}}%
\pgfpathclose%
\pgfusepath{fill}%
\end{pgfscope}%
\begin{pgfscope}%
\pgfpathrectangle{\pgfqpoint{1.254980in}{0.150000in}}{\pgfqpoint{5.490039in}{5.490039in}}%
\pgfusepath{clip}%
\pgfsetbuttcap%
\pgfsetroundjoin%
\definecolor{currentfill}{rgb}{0.281887,0.150881,0.465405}%
\pgfsetfillcolor{currentfill}%
\pgfsetfillopacity{0.700000}%
\pgfsetlinewidth{0.000000pt}%
\definecolor{currentstroke}{rgb}{0.000000,0.000000,0.000000}%
\pgfsetstrokecolor{currentstroke}%
\pgfsetdash{}{0pt}%
\pgfpathmoveto{\pgfqpoint{4.506363in}{2.304130in}}%
\pgfpathlineto{\pgfqpoint{4.519594in}{2.302330in}}%
\pgfpathlineto{\pgfqpoint{4.532834in}{2.300652in}}%
\pgfpathlineto{\pgfqpoint{4.546082in}{2.299095in}}%
\pgfpathlineto{\pgfqpoint{4.559337in}{2.297659in}}%
\pgfpathlineto{\pgfqpoint{4.566759in}{2.307688in}}%
\pgfpathlineto{\pgfqpoint{4.574177in}{2.317714in}}%
\pgfpathlineto{\pgfqpoint{4.581589in}{2.327737in}}%
\pgfpathlineto{\pgfqpoint{4.588997in}{2.337758in}}%
\pgfpathlineto{\pgfqpoint{4.575750in}{2.339170in}}%
\pgfpathlineto{\pgfqpoint{4.562512in}{2.340704in}}%
\pgfpathlineto{\pgfqpoint{4.549281in}{2.342359in}}%
\pgfpathlineto{\pgfqpoint{4.536059in}{2.344136in}}%
\pgfpathlineto{\pgfqpoint{4.528642in}{2.334132in}}%
\pgfpathlineto{\pgfqpoint{4.521220in}{2.324130in}}%
\pgfpathlineto{\pgfqpoint{4.513794in}{2.314129in}}%
\pgfpathlineto{\pgfqpoint{4.506363in}{2.304130in}}%
\pgfpathclose%
\pgfusepath{fill}%
\end{pgfscope}%
\begin{pgfscope}%
\pgfpathrectangle{\pgfqpoint{1.254980in}{0.150000in}}{\pgfqpoint{5.490039in}{5.490039in}}%
\pgfusepath{clip}%
\pgfsetbuttcap%
\pgfsetroundjoin%
\definecolor{currentfill}{rgb}{0.257322,0.256130,0.526563}%
\pgfsetfillcolor{currentfill}%
\pgfsetfillopacity{0.700000}%
\pgfsetlinewidth{0.000000pt}%
\definecolor{currentstroke}{rgb}{0.000000,0.000000,0.000000}%
\pgfsetstrokecolor{currentstroke}%
\pgfsetdash{}{0pt}%
\pgfpathmoveto{\pgfqpoint{3.236597in}{2.541729in}}%
\pgfpathlineto{\pgfqpoint{3.249689in}{2.528411in}}%
\pgfpathlineto{\pgfqpoint{3.262780in}{2.515258in}}%
\pgfpathlineto{\pgfqpoint{3.275870in}{2.502268in}}%
\pgfpathlineto{\pgfqpoint{3.288957in}{2.489440in}}%
\pgfpathlineto{\pgfqpoint{3.296835in}{2.496029in}}%
\pgfpathlineto{\pgfqpoint{3.304705in}{2.502712in}}%
\pgfpathlineto{\pgfqpoint{3.312567in}{2.509486in}}%
\pgfpathlineto{\pgfqpoint{3.320422in}{2.516352in}}%
\pgfpathlineto{\pgfqpoint{3.307355in}{2.529007in}}%
\pgfpathlineto{\pgfqpoint{3.294287in}{2.541825in}}%
\pgfpathlineto{\pgfqpoint{3.281217in}{2.554805in}}%
\pgfpathlineto{\pgfqpoint{3.268146in}{2.567949in}}%
\pgfpathlineto{\pgfqpoint{3.260270in}{2.561250in}}%
\pgfpathlineto{\pgfqpoint{3.252387in}{2.554646in}}%
\pgfpathlineto{\pgfqpoint{3.244496in}{2.548138in}}%
\pgfpathlineto{\pgfqpoint{3.236597in}{2.541729in}}%
\pgfpathclose%
\pgfusepath{fill}%
\end{pgfscope}%
\begin{pgfscope}%
\pgfpathrectangle{\pgfqpoint{1.254980in}{0.150000in}}{\pgfqpoint{5.490039in}{5.490039in}}%
\pgfusepath{clip}%
\pgfsetbuttcap%
\pgfsetroundjoin%
\definecolor{currentfill}{rgb}{0.160665,0.478540,0.558115}%
\pgfsetfillcolor{currentfill}%
\pgfsetfillopacity{0.700000}%
\pgfsetlinewidth{0.000000pt}%
\definecolor{currentstroke}{rgb}{0.000000,0.000000,0.000000}%
\pgfsetstrokecolor{currentstroke}%
\pgfsetdash{}{0pt}%
\pgfpathmoveto{\pgfqpoint{5.912447in}{3.042582in}}%
\pgfpathlineto{\pgfqpoint{5.926253in}{3.046781in}}%
\pgfpathlineto{\pgfqpoint{5.940073in}{3.051089in}}%
\pgfpathlineto{\pgfqpoint{5.953908in}{3.055507in}}%
\pgfpathlineto{\pgfqpoint{5.967757in}{3.060034in}}%
\pgfpathlineto{\pgfqpoint{5.974652in}{3.067714in}}%
\pgfpathlineto{\pgfqpoint{5.981542in}{3.075410in}}%
\pgfpathlineto{\pgfqpoint{5.988429in}{3.083125in}}%
\pgfpathlineto{\pgfqpoint{5.974594in}{3.078787in}}%
\pgfpathlineto{\pgfqpoint{5.960774in}{3.074557in}}%
\pgfpathlineto{\pgfqpoint{5.946968in}{3.070437in}}%
\pgfpathlineto{\pgfqpoint{5.933175in}{3.066427in}}%
\pgfpathlineto{\pgfqpoint{5.926270in}{3.058456in}}%
\pgfpathlineto{\pgfqpoint{5.919360in}{3.050509in}}%
\pgfpathlineto{\pgfqpoint{5.912447in}{3.042582in}}%
\pgfpathclose%
\pgfusepath{fill}%
\end{pgfscope}%
\begin{pgfscope}%
\pgfpathrectangle{\pgfqpoint{1.254980in}{0.150000in}}{\pgfqpoint{5.490039in}{5.490039in}}%
\pgfusepath{clip}%
\pgfsetbuttcap%
\pgfsetroundjoin%
\definecolor{currentfill}{rgb}{0.281412,0.155834,0.469201}%
\pgfsetfillcolor{currentfill}%
\pgfsetfillopacity{0.700000}%
\pgfsetlinewidth{0.000000pt}%
\definecolor{currentstroke}{rgb}{0.000000,0.000000,0.000000}%
\pgfsetstrokecolor{currentstroke}%
\pgfsetdash{}{0pt}%
\pgfpathmoveto{\pgfqpoint{3.529436in}{2.335232in}}%
\pgfpathlineto{\pgfqpoint{3.542502in}{2.325208in}}%
\pgfpathlineto{\pgfqpoint{3.555570in}{2.315332in}}%
\pgfpathlineto{\pgfqpoint{3.568639in}{2.305603in}}%
\pgfpathlineto{\pgfqpoint{3.581709in}{2.296021in}}%
\pgfpathlineto{\pgfqpoint{3.589463in}{2.303779in}}%
\pgfpathlineto{\pgfqpoint{3.597212in}{2.311605in}}%
\pgfpathlineto{\pgfqpoint{3.604954in}{2.319499in}}%
\pgfpathlineto{\pgfqpoint{3.612690in}{2.327459in}}%
\pgfpathlineto{\pgfqpoint{3.599636in}{2.336888in}}%
\pgfpathlineto{\pgfqpoint{3.586585in}{2.346464in}}%
\pgfpathlineto{\pgfqpoint{3.573534in}{2.356187in}}%
\pgfpathlineto{\pgfqpoint{3.560485in}{2.366058in}}%
\pgfpathlineto{\pgfqpoint{3.552733in}{2.358244in}}%
\pgfpathlineto{\pgfqpoint{3.544974in}{2.350502in}}%
\pgfpathlineto{\pgfqpoint{3.537208in}{2.342830in}}%
\pgfpathlineto{\pgfqpoint{3.529436in}{2.335232in}}%
\pgfpathclose%
\pgfusepath{fill}%
\end{pgfscope}%
\begin{pgfscope}%
\pgfpathrectangle{\pgfqpoint{1.254980in}{0.150000in}}{\pgfqpoint{5.490039in}{5.490039in}}%
\pgfusepath{clip}%
\pgfsetbuttcap%
\pgfsetroundjoin%
\definecolor{currentfill}{rgb}{0.282656,0.100196,0.422160}%
\pgfsetfillcolor{currentfill}%
\pgfsetfillopacity{0.700000}%
\pgfsetlinewidth{0.000000pt}%
\definecolor{currentstroke}{rgb}{0.000000,0.000000,0.000000}%
\pgfsetstrokecolor{currentstroke}%
\pgfsetdash{}{0pt}%
\pgfpathmoveto{\pgfqpoint{3.987713in}{2.215665in}}%
\pgfpathlineto{\pgfqpoint{4.000817in}{2.209963in}}%
\pgfpathlineto{\pgfqpoint{4.013925in}{2.204394in}}%
\pgfpathlineto{\pgfqpoint{4.027038in}{2.198956in}}%
\pgfpathlineto{\pgfqpoint{4.040157in}{2.193649in}}%
\pgfpathlineto{\pgfqpoint{4.047745in}{2.202936in}}%
\pgfpathlineto{\pgfqpoint{4.055327in}{2.212253in}}%
\pgfpathlineto{\pgfqpoint{4.062905in}{2.221600in}}%
\pgfpathlineto{\pgfqpoint{4.070478in}{2.230977in}}%
\pgfpathlineto{\pgfqpoint{4.057371in}{2.236181in}}%
\pgfpathlineto{\pgfqpoint{4.044270in}{2.241517in}}%
\pgfpathlineto{\pgfqpoint{4.031173in}{2.246983in}}%
\pgfpathlineto{\pgfqpoint{4.018082in}{2.252582in}}%
\pgfpathlineto{\pgfqpoint{4.010497in}{2.243302in}}%
\pgfpathlineto{\pgfqpoint{4.002908in}{2.234056in}}%
\pgfpathlineto{\pgfqpoint{3.995313in}{2.224843in}}%
\pgfpathlineto{\pgfqpoint{3.987713in}{2.215665in}}%
\pgfpathclose%
\pgfusepath{fill}%
\end{pgfscope}%
\begin{pgfscope}%
\pgfpathrectangle{\pgfqpoint{1.254980in}{0.150000in}}{\pgfqpoint{5.490039in}{5.490039in}}%
\pgfusepath{clip}%
\pgfsetbuttcap%
\pgfsetroundjoin%
\definecolor{currentfill}{rgb}{0.283229,0.120777,0.440584}%
\pgfsetfillcolor{currentfill}%
\pgfsetfillopacity{0.700000}%
\pgfsetlinewidth{0.000000pt}%
\definecolor{currentstroke}{rgb}{0.000000,0.000000,0.000000}%
\pgfsetstrokecolor{currentstroke}%
\pgfsetdash{}{0pt}%
\pgfpathmoveto{\pgfqpoint{3.717184in}{2.257229in}}%
\pgfpathlineto{\pgfqpoint{3.730256in}{2.249091in}}%
\pgfpathlineto{\pgfqpoint{3.743331in}{2.241094in}}%
\pgfpathlineto{\pgfqpoint{3.756409in}{2.233236in}}%
\pgfpathlineto{\pgfqpoint{3.769489in}{2.225518in}}%
\pgfpathlineto{\pgfqpoint{3.777173in}{2.233970in}}%
\pgfpathlineto{\pgfqpoint{3.784850in}{2.242475in}}%
\pgfpathlineto{\pgfqpoint{3.792522in}{2.251032in}}%
\pgfpathlineto{\pgfqpoint{3.800189in}{2.259639in}}%
\pgfpathlineto{\pgfqpoint{3.787123in}{2.267222in}}%
\pgfpathlineto{\pgfqpoint{3.774060in}{2.274944in}}%
\pgfpathlineto{\pgfqpoint{3.760999in}{2.282806in}}%
\pgfpathlineto{\pgfqpoint{3.747942in}{2.290808in}}%
\pgfpathlineto{\pgfqpoint{3.740261in}{2.282330in}}%
\pgfpathlineto{\pgfqpoint{3.732575in}{2.273907in}}%
\pgfpathlineto{\pgfqpoint{3.724882in}{2.265540in}}%
\pgfpathlineto{\pgfqpoint{3.717184in}{2.257229in}}%
\pgfpathclose%
\pgfusepath{fill}%
\end{pgfscope}%
\begin{pgfscope}%
\pgfpathrectangle{\pgfqpoint{1.254980in}{0.150000in}}{\pgfqpoint{5.490039in}{5.490039in}}%
\pgfusepath{clip}%
\pgfsetbuttcap%
\pgfsetroundjoin%
\definecolor{currentfill}{rgb}{0.177423,0.437527,0.557565}%
\pgfsetfillcolor{currentfill}%
\pgfsetfillopacity{0.700000}%
\pgfsetlinewidth{0.000000pt}%
\definecolor{currentstroke}{rgb}{0.000000,0.000000,0.000000}%
\pgfsetstrokecolor{currentstroke}%
\pgfsetdash{}{0pt}%
\pgfpathmoveto{\pgfqpoint{2.868724in}{2.985206in}}%
\pgfpathlineto{\pgfqpoint{2.881925in}{2.966883in}}%
\pgfpathlineto{\pgfqpoint{2.895119in}{2.948755in}}%
\pgfpathlineto{\pgfqpoint{2.908308in}{2.930819in}}%
\pgfpathlineto{\pgfqpoint{2.921491in}{2.913076in}}%
\pgfpathlineto{\pgfqpoint{2.929535in}{2.918388in}}%
\pgfpathlineto{\pgfqpoint{2.937569in}{2.923824in}}%
\pgfpathlineto{\pgfqpoint{2.945595in}{2.929381in}}%
\pgfpathlineto{\pgfqpoint{2.953611in}{2.935059in}}%
\pgfpathlineto{\pgfqpoint{2.940454in}{2.952622in}}%
\pgfpathlineto{\pgfqpoint{2.927292in}{2.970377in}}%
\pgfpathlineto{\pgfqpoint{2.914124in}{2.988324in}}%
\pgfpathlineto{\pgfqpoint{2.900950in}{3.006465in}}%
\pgfpathlineto{\pgfqpoint{2.892908in}{3.000962in}}%
\pgfpathlineto{\pgfqpoint{2.884856in}{2.995584in}}%
\pgfpathlineto{\pgfqpoint{2.876795in}{2.990331in}}%
\pgfpathlineto{\pgfqpoint{2.868724in}{2.985206in}}%
\pgfpathclose%
\pgfusepath{fill}%
\end{pgfscope}%
\begin{pgfscope}%
\pgfpathrectangle{\pgfqpoint{1.254980in}{0.150000in}}{\pgfqpoint{5.490039in}{5.490039in}}%
\pgfusepath{clip}%
\pgfsetbuttcap%
\pgfsetroundjoin%
\definecolor{currentfill}{rgb}{0.265145,0.232956,0.516599}%
\pgfsetfillcolor{currentfill}%
\pgfsetfillopacity{0.700000}%
\pgfsetlinewidth{0.000000pt}%
\definecolor{currentstroke}{rgb}{0.000000,0.000000,0.000000}%
\pgfsetstrokecolor{currentstroke}%
\pgfsetdash{}{0pt}%
\pgfpathmoveto{\pgfqpoint{3.288957in}{2.489440in}}%
\pgfpathlineto{\pgfqpoint{3.302044in}{2.476774in}}%
\pgfpathlineto{\pgfqpoint{3.315130in}{2.464268in}}%
\pgfpathlineto{\pgfqpoint{3.328214in}{2.451922in}}%
\pgfpathlineto{\pgfqpoint{3.341298in}{2.439735in}}%
\pgfpathlineto{\pgfqpoint{3.349154in}{2.446501in}}%
\pgfpathlineto{\pgfqpoint{3.357004in}{2.453358in}}%
\pgfpathlineto{\pgfqpoint{3.364846in}{2.460302in}}%
\pgfpathlineto{\pgfqpoint{3.372681in}{2.467334in}}%
\pgfpathlineto{\pgfqpoint{3.359617in}{2.479350in}}%
\pgfpathlineto{\pgfqpoint{3.346553in}{2.491524in}}%
\pgfpathlineto{\pgfqpoint{3.333488in}{2.503858in}}%
\pgfpathlineto{\pgfqpoint{3.320422in}{2.516352in}}%
\pgfpathlineto{\pgfqpoint{3.312567in}{2.509486in}}%
\pgfpathlineto{\pgfqpoint{3.304705in}{2.502712in}}%
\pgfpathlineto{\pgfqpoint{3.296835in}{2.496029in}}%
\pgfpathlineto{\pgfqpoint{3.288957in}{2.489440in}}%
\pgfpathclose%
\pgfusepath{fill}%
\end{pgfscope}%
\begin{pgfscope}%
\pgfpathrectangle{\pgfqpoint{1.254980in}{0.150000in}}{\pgfqpoint{5.490039in}{5.490039in}}%
\pgfusepath{clip}%
\pgfsetbuttcap%
\pgfsetroundjoin%
\definecolor{currentfill}{rgb}{0.282884,0.135920,0.453427}%
\pgfsetfillcolor{currentfill}%
\pgfsetfillopacity{0.700000}%
\pgfsetlinewidth{0.000000pt}%
\definecolor{currentstroke}{rgb}{0.000000,0.000000,0.000000}%
\pgfsetstrokecolor{currentstroke}%
\pgfsetdash{}{0pt}%
\pgfpathmoveto{\pgfqpoint{4.423709in}{2.272724in}}%
\pgfpathlineto{\pgfqpoint{4.436918in}{2.270395in}}%
\pgfpathlineto{\pgfqpoint{4.450135in}{2.268189in}}%
\pgfpathlineto{\pgfqpoint{4.463360in}{2.266106in}}%
\pgfpathlineto{\pgfqpoint{4.476592in}{2.264145in}}%
\pgfpathlineto{\pgfqpoint{4.484042in}{2.274139in}}%
\pgfpathlineto{\pgfqpoint{4.491487in}{2.284135in}}%
\pgfpathlineto{\pgfqpoint{4.498927in}{2.294132in}}%
\pgfpathlineto{\pgfqpoint{4.506363in}{2.304130in}}%
\pgfpathlineto{\pgfqpoint{4.493140in}{2.306052in}}%
\pgfpathlineto{\pgfqpoint{4.479924in}{2.308096in}}%
\pgfpathlineto{\pgfqpoint{4.466716in}{2.310263in}}%
\pgfpathlineto{\pgfqpoint{4.453516in}{2.312553in}}%
\pgfpathlineto{\pgfqpoint{4.446071in}{2.302588in}}%
\pgfpathlineto{\pgfqpoint{4.438622in}{2.292628in}}%
\pgfpathlineto{\pgfqpoint{4.431168in}{2.282673in}}%
\pgfpathlineto{\pgfqpoint{4.423709in}{2.272724in}}%
\pgfpathclose%
\pgfusepath{fill}%
\end{pgfscope}%
\begin{pgfscope}%
\pgfpathrectangle{\pgfqpoint{1.254980in}{0.150000in}}{\pgfqpoint{5.490039in}{5.490039in}}%
\pgfusepath{clip}%
\pgfsetbuttcap%
\pgfsetroundjoin%
\definecolor{currentfill}{rgb}{0.282656,0.100196,0.422160}%
\pgfsetfillcolor{currentfill}%
\pgfsetfillopacity{0.700000}%
\pgfsetlinewidth{0.000000pt}%
\definecolor{currentstroke}{rgb}{0.000000,0.000000,0.000000}%
\pgfsetstrokecolor{currentstroke}%
\pgfsetdash{}{0pt}%
\pgfpathmoveto{\pgfqpoint{4.122957in}{2.211464in}}%
\pgfpathlineto{\pgfqpoint{4.136090in}{2.206910in}}%
\pgfpathlineto{\pgfqpoint{4.149230in}{2.202484in}}%
\pgfpathlineto{\pgfqpoint{4.162375in}{2.198186in}}%
\pgfpathlineto{\pgfqpoint{4.175525in}{2.194016in}}%
\pgfpathlineto{\pgfqpoint{4.183071in}{2.203607in}}%
\pgfpathlineto{\pgfqpoint{4.190612in}{2.213219in}}%
\pgfpathlineto{\pgfqpoint{4.198148in}{2.222851in}}%
\pgfpathlineto{\pgfqpoint{4.205679in}{2.232502in}}%
\pgfpathlineto{\pgfqpoint{4.192539in}{2.236585in}}%
\pgfpathlineto{\pgfqpoint{4.179404in}{2.240796in}}%
\pgfpathlineto{\pgfqpoint{4.166276in}{2.245136in}}%
\pgfpathlineto{\pgfqpoint{4.153153in}{2.249604in}}%
\pgfpathlineto{\pgfqpoint{4.145612in}{2.240033in}}%
\pgfpathlineto{\pgfqpoint{4.138065in}{2.230486in}}%
\pgfpathlineto{\pgfqpoint{4.130514in}{2.220963in}}%
\pgfpathlineto{\pgfqpoint{4.122957in}{2.211464in}}%
\pgfpathclose%
\pgfusepath{fill}%
\end{pgfscope}%
\begin{pgfscope}%
\pgfpathrectangle{\pgfqpoint{1.254980in}{0.150000in}}{\pgfqpoint{5.490039in}{5.490039in}}%
\pgfusepath{clip}%
\pgfsetbuttcap%
\pgfsetroundjoin%
\definecolor{currentfill}{rgb}{0.165117,0.467423,0.558141}%
\pgfsetfillcolor{currentfill}%
\pgfsetfillopacity{0.700000}%
\pgfsetlinewidth{0.000000pt}%
\definecolor{currentstroke}{rgb}{0.000000,0.000000,0.000000}%
\pgfsetstrokecolor{currentstroke}%
\pgfsetdash{}{0pt}%
\pgfpathmoveto{\pgfqpoint{2.815858in}{3.060466in}}%
\pgfpathlineto{\pgfqpoint{2.829085in}{3.041353in}}%
\pgfpathlineto{\pgfqpoint{2.842304in}{3.022439in}}%
\pgfpathlineto{\pgfqpoint{2.855517in}{3.003724in}}%
\pgfpathlineto{\pgfqpoint{2.868724in}{2.985206in}}%
\pgfpathlineto{\pgfqpoint{2.876795in}{2.990331in}}%
\pgfpathlineto{\pgfqpoint{2.884856in}{2.995584in}}%
\pgfpathlineto{\pgfqpoint{2.892908in}{3.000962in}}%
\pgfpathlineto{\pgfqpoint{2.900950in}{3.006465in}}%
\pgfpathlineto{\pgfqpoint{2.887771in}{3.024801in}}%
\pgfpathlineto{\pgfqpoint{2.874585in}{3.043334in}}%
\pgfpathlineto{\pgfqpoint{2.861393in}{3.062065in}}%
\pgfpathlineto{\pgfqpoint{2.848194in}{3.080995in}}%
\pgfpathlineto{\pgfqpoint{2.840125in}{3.075668in}}%
\pgfpathlineto{\pgfqpoint{2.832046in}{3.070470in}}%
\pgfpathlineto{\pgfqpoint{2.823957in}{3.065402in}}%
\pgfpathlineto{\pgfqpoint{2.815858in}{3.060466in}}%
\pgfpathclose%
\pgfusepath{fill}%
\end{pgfscope}%
\begin{pgfscope}%
\pgfpathrectangle{\pgfqpoint{1.254980in}{0.150000in}}{\pgfqpoint{5.490039in}{5.490039in}}%
\pgfusepath{clip}%
\pgfsetbuttcap%
\pgfsetroundjoin%
\definecolor{currentfill}{rgb}{0.271828,0.209303,0.504434}%
\pgfsetfillcolor{currentfill}%
\pgfsetfillopacity{0.700000}%
\pgfsetlinewidth{0.000000pt}%
\definecolor{currentstroke}{rgb}{0.000000,0.000000,0.000000}%
\pgfsetstrokecolor{currentstroke}%
\pgfsetdash{}{0pt}%
\pgfpathmoveto{\pgfqpoint{3.341298in}{2.439735in}}%
\pgfpathlineto{\pgfqpoint{3.354381in}{2.427705in}}%
\pgfpathlineto{\pgfqpoint{3.367463in}{2.415833in}}%
\pgfpathlineto{\pgfqpoint{3.380545in}{2.404117in}}%
\pgfpathlineto{\pgfqpoint{3.393626in}{2.392557in}}%
\pgfpathlineto{\pgfqpoint{3.401463in}{2.399501in}}%
\pgfpathlineto{\pgfqpoint{3.409292in}{2.406531in}}%
\pgfpathlineto{\pgfqpoint{3.417115in}{2.413645in}}%
\pgfpathlineto{\pgfqpoint{3.424930in}{2.420841in}}%
\pgfpathlineto{\pgfqpoint{3.411868in}{2.432231in}}%
\pgfpathlineto{\pgfqpoint{3.398806in}{2.443775in}}%
\pgfpathlineto{\pgfqpoint{3.385743in}{2.455476in}}%
\pgfpathlineto{\pgfqpoint{3.372681in}{2.467334in}}%
\pgfpathlineto{\pgfqpoint{3.364846in}{2.460302in}}%
\pgfpathlineto{\pgfqpoint{3.357004in}{2.453358in}}%
\pgfpathlineto{\pgfqpoint{3.349154in}{2.446501in}}%
\pgfpathlineto{\pgfqpoint{3.341298in}{2.439735in}}%
\pgfpathclose%
\pgfusepath{fill}%
\end{pgfscope}%
\begin{pgfscope}%
\pgfpathrectangle{\pgfqpoint{1.254980in}{0.150000in}}{\pgfqpoint{5.490039in}{5.490039in}}%
\pgfusepath{clip}%
\pgfsetbuttcap%
\pgfsetroundjoin%
\definecolor{currentfill}{rgb}{0.282623,0.140926,0.457517}%
\pgfsetfillcolor{currentfill}%
\pgfsetfillopacity{0.700000}%
\pgfsetlinewidth{0.000000pt}%
\definecolor{currentstroke}{rgb}{0.000000,0.000000,0.000000}%
\pgfsetstrokecolor{currentstroke}%
\pgfsetdash{}{0pt}%
\pgfpathmoveto{\pgfqpoint{3.581709in}{2.296021in}}%
\pgfpathlineto{\pgfqpoint{3.594780in}{2.286586in}}%
\pgfpathlineto{\pgfqpoint{3.607853in}{2.277296in}}%
\pgfpathlineto{\pgfqpoint{3.620928in}{2.268150in}}%
\pgfpathlineto{\pgfqpoint{3.634005in}{2.259149in}}%
\pgfpathlineto{\pgfqpoint{3.641743in}{2.267065in}}%
\pgfpathlineto{\pgfqpoint{3.649475in}{2.275046in}}%
\pgfpathlineto{\pgfqpoint{3.657200in}{2.283090in}}%
\pgfpathlineto{\pgfqpoint{3.664920in}{2.291197in}}%
\pgfpathlineto{\pgfqpoint{3.651860in}{2.300046in}}%
\pgfpathlineto{\pgfqpoint{3.638801in}{2.309039in}}%
\pgfpathlineto{\pgfqpoint{3.625745in}{2.318176in}}%
\pgfpathlineto{\pgfqpoint{3.612690in}{2.327459in}}%
\pgfpathlineto{\pgfqpoint{3.604954in}{2.319499in}}%
\pgfpathlineto{\pgfqpoint{3.597212in}{2.311605in}}%
\pgfpathlineto{\pgfqpoint{3.589463in}{2.303779in}}%
\pgfpathlineto{\pgfqpoint{3.581709in}{2.296021in}}%
\pgfpathclose%
\pgfusepath{fill}%
\end{pgfscope}%
\begin{pgfscope}%
\pgfpathrectangle{\pgfqpoint{1.254980in}{0.150000in}}{\pgfqpoint{5.490039in}{5.490039in}}%
\pgfusepath{clip}%
\pgfsetbuttcap%
\pgfsetroundjoin%
\definecolor{currentfill}{rgb}{0.283229,0.120777,0.440584}%
\pgfsetfillcolor{currentfill}%
\pgfsetfillopacity{0.700000}%
\pgfsetlinewidth{0.000000pt}%
\definecolor{currentstroke}{rgb}{0.000000,0.000000,0.000000}%
\pgfsetstrokecolor{currentstroke}%
\pgfsetdash{}{0pt}%
\pgfpathmoveto{\pgfqpoint{4.341026in}{2.243754in}}%
\pgfpathlineto{\pgfqpoint{4.354215in}{2.240875in}}%
\pgfpathlineto{\pgfqpoint{4.367412in}{2.238120in}}%
\pgfpathlineto{\pgfqpoint{4.380616in}{2.235490in}}%
\pgfpathlineto{\pgfqpoint{4.393827in}{2.232983in}}%
\pgfpathlineto{\pgfqpoint{4.401304in}{2.242910in}}%
\pgfpathlineto{\pgfqpoint{4.408777in}{2.252842in}}%
\pgfpathlineto{\pgfqpoint{4.416246in}{2.262780in}}%
\pgfpathlineto{\pgfqpoint{4.423709in}{2.272724in}}%
\pgfpathlineto{\pgfqpoint{4.410507in}{2.275176in}}%
\pgfpathlineto{\pgfqpoint{4.397313in}{2.277752in}}%
\pgfpathlineto{\pgfqpoint{4.384126in}{2.280452in}}%
\pgfpathlineto{\pgfqpoint{4.370946in}{2.283275in}}%
\pgfpathlineto{\pgfqpoint{4.363473in}{2.273381in}}%
\pgfpathlineto{\pgfqpoint{4.355995in}{2.263495in}}%
\pgfpathlineto{\pgfqpoint{4.348513in}{2.253619in}}%
\pgfpathlineto{\pgfqpoint{4.341026in}{2.243754in}}%
\pgfpathclose%
\pgfusepath{fill}%
\end{pgfscope}%
\begin{pgfscope}%
\pgfpathrectangle{\pgfqpoint{1.254980in}{0.150000in}}{\pgfqpoint{5.490039in}{5.490039in}}%
\pgfusepath{clip}%
\pgfsetbuttcap%
\pgfsetroundjoin%
\definecolor{currentfill}{rgb}{0.250425,0.274290,0.533103}%
\pgfsetfillcolor{currentfill}%
\pgfsetfillopacity{0.700000}%
\pgfsetlinewidth{0.000000pt}%
\definecolor{currentstroke}{rgb}{0.000000,0.000000,0.000000}%
\pgfsetstrokecolor{currentstroke}%
\pgfsetdash{}{0pt}%
\pgfpathmoveto{\pgfqpoint{5.055806in}{2.537012in}}%
\pgfpathlineto{\pgfqpoint{5.069251in}{2.538425in}}%
\pgfpathlineto{\pgfqpoint{5.082707in}{2.539954in}}%
\pgfpathlineto{\pgfqpoint{5.096173in}{2.541598in}}%
\pgfpathlineto{\pgfqpoint{5.109651in}{2.543357in}}%
\pgfpathlineto{\pgfqpoint{5.116896in}{2.552952in}}%
\pgfpathlineto{\pgfqpoint{5.124135in}{2.562529in}}%
\pgfpathlineto{\pgfqpoint{5.131370in}{2.572088in}}%
\pgfpathlineto{\pgfqpoint{5.138599in}{2.581631in}}%
\pgfpathlineto{\pgfqpoint{5.125131in}{2.579945in}}%
\pgfpathlineto{\pgfqpoint{5.111674in}{2.578374in}}%
\pgfpathlineto{\pgfqpoint{5.098228in}{2.576919in}}%
\pgfpathlineto{\pgfqpoint{5.084793in}{2.575579in}}%
\pgfpathlineto{\pgfqpoint{5.077553in}{2.565957in}}%
\pgfpathlineto{\pgfqpoint{5.070309in}{2.556323in}}%
\pgfpathlineto{\pgfqpoint{5.063060in}{2.546675in}}%
\pgfpathlineto{\pgfqpoint{5.055806in}{2.537012in}}%
\pgfpathclose%
\pgfusepath{fill}%
\end{pgfscope}%
\begin{pgfscope}%
\pgfpathrectangle{\pgfqpoint{1.254980in}{0.150000in}}{\pgfqpoint{5.490039in}{5.490039in}}%
\pgfusepath{clip}%
\pgfsetbuttcap%
\pgfsetroundjoin%
\definecolor{currentfill}{rgb}{0.258965,0.251537,0.524736}%
\pgfsetfillcolor{currentfill}%
\pgfsetfillopacity{0.700000}%
\pgfsetlinewidth{0.000000pt}%
\definecolor{currentstroke}{rgb}{0.000000,0.000000,0.000000}%
\pgfsetstrokecolor{currentstroke}%
\pgfsetdash{}{0pt}%
\pgfpathmoveto{\pgfqpoint{4.973033in}{2.493473in}}%
\pgfpathlineto{\pgfqpoint{4.986444in}{2.494480in}}%
\pgfpathlineto{\pgfqpoint{4.999866in}{2.495604in}}%
\pgfpathlineto{\pgfqpoint{5.013298in}{2.496843in}}%
\pgfpathlineto{\pgfqpoint{5.026741in}{2.498198in}}%
\pgfpathlineto{\pgfqpoint{5.034015in}{2.507928in}}%
\pgfpathlineto{\pgfqpoint{5.041284in}{2.517639in}}%
\pgfpathlineto{\pgfqpoint{5.048547in}{2.527334in}}%
\pgfpathlineto{\pgfqpoint{5.055806in}{2.537012in}}%
\pgfpathlineto{\pgfqpoint{5.042372in}{2.535714in}}%
\pgfpathlineto{\pgfqpoint{5.028949in}{2.534532in}}%
\pgfpathlineto{\pgfqpoint{5.015537in}{2.533465in}}%
\pgfpathlineto{\pgfqpoint{5.002136in}{2.532515in}}%
\pgfpathlineto{\pgfqpoint{4.994867in}{2.522774in}}%
\pgfpathlineto{\pgfqpoint{4.987594in}{2.513020in}}%
\pgfpathlineto{\pgfqpoint{4.980316in}{2.503253in}}%
\pgfpathlineto{\pgfqpoint{4.973033in}{2.493473in}}%
\pgfpathclose%
\pgfusepath{fill}%
\end{pgfscope}%
\begin{pgfscope}%
\pgfpathrectangle{\pgfqpoint{1.254980in}{0.150000in}}{\pgfqpoint{5.490039in}{5.490039in}}%
\pgfusepath{clip}%
\pgfsetbuttcap%
\pgfsetroundjoin%
\definecolor{currentfill}{rgb}{0.243113,0.292092,0.538516}%
\pgfsetfillcolor{currentfill}%
\pgfsetfillopacity{0.700000}%
\pgfsetlinewidth{0.000000pt}%
\definecolor{currentstroke}{rgb}{0.000000,0.000000,0.000000}%
\pgfsetstrokecolor{currentstroke}%
\pgfsetdash{}{0pt}%
\pgfpathmoveto{\pgfqpoint{5.138599in}{2.581631in}}%
\pgfpathlineto{\pgfqpoint{5.152079in}{2.583431in}}%
\pgfpathlineto{\pgfqpoint{5.165569in}{2.585346in}}%
\pgfpathlineto{\pgfqpoint{5.179072in}{2.587375in}}%
\pgfpathlineto{\pgfqpoint{5.192586in}{2.589519in}}%
\pgfpathlineto{\pgfqpoint{5.199800in}{2.598962in}}%
\pgfpathlineto{\pgfqpoint{5.207010in}{2.608387in}}%
\pgfpathlineto{\pgfqpoint{5.214214in}{2.617794in}}%
\pgfpathlineto{\pgfqpoint{5.221414in}{2.627184in}}%
\pgfpathlineto{\pgfqpoint{5.207910in}{2.625130in}}%
\pgfpathlineto{\pgfqpoint{5.194418in}{2.623190in}}%
\pgfpathlineto{\pgfqpoint{5.180937in}{2.621365in}}%
\pgfpathlineto{\pgfqpoint{5.167468in}{2.619654in}}%
\pgfpathlineto{\pgfqpoint{5.160258in}{2.610168in}}%
\pgfpathlineto{\pgfqpoint{5.153043in}{2.600669in}}%
\pgfpathlineto{\pgfqpoint{5.145824in}{2.591157in}}%
\pgfpathlineto{\pgfqpoint{5.138599in}{2.581631in}}%
\pgfpathclose%
\pgfusepath{fill}%
\end{pgfscope}%
\begin{pgfscope}%
\pgfpathrectangle{\pgfqpoint{1.254980in}{0.150000in}}{\pgfqpoint{5.490039in}{5.490039in}}%
\pgfusepath{clip}%
\pgfsetbuttcap%
\pgfsetroundjoin%
\definecolor{currentfill}{rgb}{0.282656,0.100196,0.422160}%
\pgfsetfillcolor{currentfill}%
\pgfsetfillopacity{0.700000}%
\pgfsetlinewidth{0.000000pt}%
\definecolor{currentstroke}{rgb}{0.000000,0.000000,0.000000}%
\pgfsetstrokecolor{currentstroke}%
\pgfsetdash{}{0pt}%
\pgfpathmoveto{\pgfqpoint{3.904841in}{2.203925in}}%
\pgfpathlineto{\pgfqpoint{3.917940in}{2.197571in}}%
\pgfpathlineto{\pgfqpoint{3.931043in}{2.191352in}}%
\pgfpathlineto{\pgfqpoint{3.944150in}{2.185265in}}%
\pgfpathlineto{\pgfqpoint{3.957261in}{2.179312in}}%
\pgfpathlineto{\pgfqpoint{3.964882in}{2.188345in}}%
\pgfpathlineto{\pgfqpoint{3.972498in}{2.197415in}}%
\pgfpathlineto{\pgfqpoint{3.980108in}{2.206522in}}%
\pgfpathlineto{\pgfqpoint{3.987713in}{2.215665in}}%
\pgfpathlineto{\pgfqpoint{3.974614in}{2.221499in}}%
\pgfpathlineto{\pgfqpoint{3.961519in}{2.227467in}}%
\pgfpathlineto{\pgfqpoint{3.948429in}{2.233567in}}%
\pgfpathlineto{\pgfqpoint{3.935343in}{2.239802in}}%
\pgfpathlineto{\pgfqpoint{3.927726in}{2.230772in}}%
\pgfpathlineto{\pgfqpoint{3.920103in}{2.221782in}}%
\pgfpathlineto{\pgfqpoint{3.912475in}{2.212833in}}%
\pgfpathlineto{\pgfqpoint{3.904841in}{2.203925in}}%
\pgfpathclose%
\pgfusepath{fill}%
\end{pgfscope}%
\begin{pgfscope}%
\pgfpathrectangle{\pgfqpoint{1.254980in}{0.150000in}}{\pgfqpoint{5.490039in}{5.490039in}}%
\pgfusepath{clip}%
\pgfsetbuttcap%
\pgfsetroundjoin%
\definecolor{currentfill}{rgb}{0.265145,0.232956,0.516599}%
\pgfsetfillcolor{currentfill}%
\pgfsetfillopacity{0.700000}%
\pgfsetlinewidth{0.000000pt}%
\definecolor{currentstroke}{rgb}{0.000000,0.000000,0.000000}%
\pgfsetstrokecolor{currentstroke}%
\pgfsetdash{}{0pt}%
\pgfpathmoveto{\pgfqpoint{4.890277in}{2.451166in}}%
\pgfpathlineto{\pgfqpoint{4.903656in}{2.451748in}}%
\pgfpathlineto{\pgfqpoint{4.917045in}{2.452447in}}%
\pgfpathlineto{\pgfqpoint{4.930444in}{2.453263in}}%
\pgfpathlineto{\pgfqpoint{4.943853in}{2.454195in}}%
\pgfpathlineto{\pgfqpoint{4.951155in}{2.464039in}}%
\pgfpathlineto{\pgfqpoint{4.958453in}{2.473866in}}%
\pgfpathlineto{\pgfqpoint{4.965745in}{2.483677in}}%
\pgfpathlineto{\pgfqpoint{4.973033in}{2.493473in}}%
\pgfpathlineto{\pgfqpoint{4.959633in}{2.492581in}}%
\pgfpathlineto{\pgfqpoint{4.946243in}{2.491807in}}%
\pgfpathlineto{\pgfqpoint{4.932863in}{2.491148in}}%
\pgfpathlineto{\pgfqpoint{4.919493in}{2.490607in}}%
\pgfpathlineto{\pgfqpoint{4.912197in}{2.480765in}}%
\pgfpathlineto{\pgfqpoint{4.904895in}{2.470911in}}%
\pgfpathlineto{\pgfqpoint{4.897589in}{2.461045in}}%
\pgfpathlineto{\pgfqpoint{4.890277in}{2.451166in}}%
\pgfpathclose%
\pgfusepath{fill}%
\end{pgfscope}%
\begin{pgfscope}%
\pgfpathrectangle{\pgfqpoint{1.254980in}{0.150000in}}{\pgfqpoint{5.490039in}{5.490039in}}%
\pgfusepath{clip}%
\pgfsetbuttcap%
\pgfsetroundjoin%
\definecolor{currentfill}{rgb}{0.233603,0.313828,0.543914}%
\pgfsetfillcolor{currentfill}%
\pgfsetfillopacity{0.700000}%
\pgfsetlinewidth{0.000000pt}%
\definecolor{currentstroke}{rgb}{0.000000,0.000000,0.000000}%
\pgfsetstrokecolor{currentstroke}%
\pgfsetdash{}{0pt}%
\pgfpathmoveto{\pgfqpoint{5.221414in}{2.627184in}}%
\pgfpathlineto{\pgfqpoint{5.234929in}{2.629353in}}%
\pgfpathlineto{\pgfqpoint{5.248456in}{2.631635in}}%
\pgfpathlineto{\pgfqpoint{5.261995in}{2.634031in}}%
\pgfpathlineto{\pgfqpoint{5.275546in}{2.636540in}}%
\pgfpathlineto{\pgfqpoint{5.282730in}{2.645816in}}%
\pgfpathlineto{\pgfqpoint{5.289909in}{2.655073in}}%
\pgfpathlineto{\pgfqpoint{5.297083in}{2.664313in}}%
\pgfpathlineto{\pgfqpoint{5.304251in}{2.673538in}}%
\pgfpathlineto{\pgfqpoint{5.290711in}{2.671135in}}%
\pgfpathlineto{\pgfqpoint{5.277183in}{2.668844in}}%
\pgfpathlineto{\pgfqpoint{5.263667in}{2.666668in}}%
\pgfpathlineto{\pgfqpoint{5.250162in}{2.664605in}}%
\pgfpathlineto{\pgfqpoint{5.242982in}{2.655269in}}%
\pgfpathlineto{\pgfqpoint{5.235798in}{2.645921in}}%
\pgfpathlineto{\pgfqpoint{5.228608in}{2.636560in}}%
\pgfpathlineto{\pgfqpoint{5.221414in}{2.627184in}}%
\pgfpathclose%
\pgfusepath{fill}%
\end{pgfscope}%
\begin{pgfscope}%
\pgfpathrectangle{\pgfqpoint{1.254980in}{0.150000in}}{\pgfqpoint{5.490039in}{5.490039in}}%
\pgfusepath{clip}%
\pgfsetbuttcap%
\pgfsetroundjoin%
\definecolor{currentfill}{rgb}{0.283091,0.110553,0.431554}%
\pgfsetfillcolor{currentfill}%
\pgfsetfillopacity{0.700000}%
\pgfsetlinewidth{0.000000pt}%
\definecolor{currentstroke}{rgb}{0.000000,0.000000,0.000000}%
\pgfsetstrokecolor{currentstroke}%
\pgfsetdash{}{0pt}%
\pgfpathmoveto{\pgfqpoint{3.769489in}{2.225518in}}%
\pgfpathlineto{\pgfqpoint{3.782573in}{2.217939in}}%
\pgfpathlineto{\pgfqpoint{3.795660in}{2.210498in}}%
\pgfpathlineto{\pgfqpoint{3.808750in}{2.203194in}}%
\pgfpathlineto{\pgfqpoint{3.821843in}{2.196028in}}%
\pgfpathlineto{\pgfqpoint{3.829512in}{2.204621in}}%
\pgfpathlineto{\pgfqpoint{3.837176in}{2.213263in}}%
\pgfpathlineto{\pgfqpoint{3.844834in}{2.221953in}}%
\pgfpathlineto{\pgfqpoint{3.852486in}{2.230690in}}%
\pgfpathlineto{\pgfqpoint{3.839407in}{2.237721in}}%
\pgfpathlineto{\pgfqpoint{3.826331in}{2.244889in}}%
\pgfpathlineto{\pgfqpoint{3.813258in}{2.252195in}}%
\pgfpathlineto{\pgfqpoint{3.800189in}{2.259639in}}%
\pgfpathlineto{\pgfqpoint{3.792522in}{2.251032in}}%
\pgfpathlineto{\pgfqpoint{3.784850in}{2.242475in}}%
\pgfpathlineto{\pgfqpoint{3.777173in}{2.233970in}}%
\pgfpathlineto{\pgfqpoint{3.769489in}{2.225518in}}%
\pgfpathclose%
\pgfusepath{fill}%
\end{pgfscope}%
\begin{pgfscope}%
\pgfpathrectangle{\pgfqpoint{1.254980in}{0.150000in}}{\pgfqpoint{5.490039in}{5.490039in}}%
\pgfusepath{clip}%
\pgfsetbuttcap%
\pgfsetroundjoin%
\definecolor{currentfill}{rgb}{0.151918,0.500685,0.557587}%
\pgfsetfillcolor{currentfill}%
\pgfsetfillopacity{0.700000}%
\pgfsetlinewidth{0.000000pt}%
\definecolor{currentstroke}{rgb}{0.000000,0.000000,0.000000}%
\pgfsetstrokecolor{currentstroke}%
\pgfsetdash{}{0pt}%
\pgfpathmoveto{\pgfqpoint{2.762882in}{3.138941in}}%
\pgfpathlineto{\pgfqpoint{2.776137in}{3.119016in}}%
\pgfpathlineto{\pgfqpoint{2.789384in}{3.099296in}}%
\pgfpathlineto{\pgfqpoint{2.802625in}{3.079780in}}%
\pgfpathlineto{\pgfqpoint{2.815858in}{3.060466in}}%
\pgfpathlineto{\pgfqpoint{2.823957in}{3.065402in}}%
\pgfpathlineto{\pgfqpoint{2.832046in}{3.070470in}}%
\pgfpathlineto{\pgfqpoint{2.840125in}{3.075668in}}%
\pgfpathlineto{\pgfqpoint{2.848194in}{3.080995in}}%
\pgfpathlineto{\pgfqpoint{2.834989in}{3.100126in}}%
\pgfpathlineto{\pgfqpoint{2.821777in}{3.119458in}}%
\pgfpathlineto{\pgfqpoint{2.808557in}{3.138994in}}%
\pgfpathlineto{\pgfqpoint{2.795331in}{3.158734in}}%
\pgfpathlineto{\pgfqpoint{2.787234in}{3.153585in}}%
\pgfpathlineto{\pgfqpoint{2.779126in}{3.148569in}}%
\pgfpathlineto{\pgfqpoint{2.771009in}{3.143687in}}%
\pgfpathlineto{\pgfqpoint{2.762882in}{3.138941in}}%
\pgfpathclose%
\pgfusepath{fill}%
\end{pgfscope}%
\begin{pgfscope}%
\pgfpathrectangle{\pgfqpoint{1.254980in}{0.150000in}}{\pgfqpoint{5.490039in}{5.490039in}}%
\pgfusepath{clip}%
\pgfsetbuttcap%
\pgfsetroundjoin%
\definecolor{currentfill}{rgb}{0.270595,0.214069,0.507052}%
\pgfsetfillcolor{currentfill}%
\pgfsetfillopacity{0.700000}%
\pgfsetlinewidth{0.000000pt}%
\definecolor{currentstroke}{rgb}{0.000000,0.000000,0.000000}%
\pgfsetstrokecolor{currentstroke}%
\pgfsetdash{}{0pt}%
\pgfpathmoveto{\pgfqpoint{4.807535in}{2.410254in}}%
\pgfpathlineto{\pgfqpoint{4.820882in}{2.410392in}}%
\pgfpathlineto{\pgfqpoint{4.834240in}{2.410647in}}%
\pgfpathlineto{\pgfqpoint{4.847607in}{2.411020in}}%
\pgfpathlineto{\pgfqpoint{4.860984in}{2.411510in}}%
\pgfpathlineto{\pgfqpoint{4.868315in}{2.421446in}}%
\pgfpathlineto{\pgfqpoint{4.875640in}{2.431367in}}%
\pgfpathlineto{\pgfqpoint{4.882961in}{2.441273in}}%
\pgfpathlineto{\pgfqpoint{4.890277in}{2.451166in}}%
\pgfpathlineto{\pgfqpoint{4.876909in}{2.450700in}}%
\pgfpathlineto{\pgfqpoint{4.863550in}{2.450353in}}%
\pgfpathlineto{\pgfqpoint{4.850202in}{2.450122in}}%
\pgfpathlineto{\pgfqpoint{4.836863in}{2.450009in}}%
\pgfpathlineto{\pgfqpoint{4.829538in}{2.440086in}}%
\pgfpathlineto{\pgfqpoint{4.822209in}{2.430153in}}%
\pgfpathlineto{\pgfqpoint{4.814874in}{2.420209in}}%
\pgfpathlineto{\pgfqpoint{4.807535in}{2.410254in}}%
\pgfpathclose%
\pgfusepath{fill}%
\end{pgfscope}%
\begin{pgfscope}%
\pgfpathrectangle{\pgfqpoint{1.254980in}{0.150000in}}{\pgfqpoint{5.490039in}{5.490039in}}%
\pgfusepath{clip}%
\pgfsetbuttcap%
\pgfsetroundjoin%
\definecolor{currentfill}{rgb}{0.223925,0.334994,0.548053}%
\pgfsetfillcolor{currentfill}%
\pgfsetfillopacity{0.700000}%
\pgfsetlinewidth{0.000000pt}%
\definecolor{currentstroke}{rgb}{0.000000,0.000000,0.000000}%
\pgfsetstrokecolor{currentstroke}%
\pgfsetdash{}{0pt}%
\pgfpathmoveto{\pgfqpoint{5.304251in}{2.673538in}}%
\pgfpathlineto{\pgfqpoint{5.317804in}{2.676055in}}%
\pgfpathlineto{\pgfqpoint{5.331368in}{2.678686in}}%
\pgfpathlineto{\pgfqpoint{5.344944in}{2.681430in}}%
\pgfpathlineto{\pgfqpoint{5.358533in}{2.684286in}}%
\pgfpathlineto{\pgfqpoint{5.365686in}{2.693381in}}%
\pgfpathlineto{\pgfqpoint{5.372833in}{2.702458in}}%
\pgfpathlineto{\pgfqpoint{5.379975in}{2.711519in}}%
\pgfpathlineto{\pgfqpoint{5.387113in}{2.720566in}}%
\pgfpathlineto{\pgfqpoint{5.373536in}{2.717832in}}%
\pgfpathlineto{\pgfqpoint{5.359971in}{2.715211in}}%
\pgfpathlineto{\pgfqpoint{5.346418in}{2.712702in}}%
\pgfpathlineto{\pgfqpoint{5.332877in}{2.710307in}}%
\pgfpathlineto{\pgfqpoint{5.325728in}{2.701132in}}%
\pgfpathlineto{\pgfqpoint{5.318574in}{2.691946in}}%
\pgfpathlineto{\pgfqpoint{5.311415in}{2.682749in}}%
\pgfpathlineto{\pgfqpoint{5.304251in}{2.673538in}}%
\pgfpathclose%
\pgfusepath{fill}%
\end{pgfscope}%
\begin{pgfscope}%
\pgfpathrectangle{\pgfqpoint{1.254980in}{0.150000in}}{\pgfqpoint{5.490039in}{5.490039in}}%
\pgfusepath{clip}%
\pgfsetbuttcap%
\pgfsetroundjoin%
\definecolor{currentfill}{rgb}{0.276194,0.190074,0.493001}%
\pgfsetfillcolor{currentfill}%
\pgfsetfillopacity{0.700000}%
\pgfsetlinewidth{0.000000pt}%
\definecolor{currentstroke}{rgb}{0.000000,0.000000,0.000000}%
\pgfsetstrokecolor{currentstroke}%
\pgfsetdash{}{0pt}%
\pgfpathmoveto{\pgfqpoint{3.393626in}{2.392557in}}%
\pgfpathlineto{\pgfqpoint{3.406707in}{2.381152in}}%
\pgfpathlineto{\pgfqpoint{3.419789in}{2.369901in}}%
\pgfpathlineto{\pgfqpoint{3.432870in}{2.358803in}}%
\pgfpathlineto{\pgfqpoint{3.445951in}{2.347858in}}%
\pgfpathlineto{\pgfqpoint{3.453769in}{2.354978in}}%
\pgfpathlineto{\pgfqpoint{3.461579in}{2.362180in}}%
\pgfpathlineto{\pgfqpoint{3.469383in}{2.369462in}}%
\pgfpathlineto{\pgfqpoint{3.477179in}{2.376823in}}%
\pgfpathlineto{\pgfqpoint{3.464117in}{2.387599in}}%
\pgfpathlineto{\pgfqpoint{3.451054in}{2.398526in}}%
\pgfpathlineto{\pgfqpoint{3.437992in}{2.409607in}}%
\pgfpathlineto{\pgfqpoint{3.424930in}{2.420841in}}%
\pgfpathlineto{\pgfqpoint{3.417115in}{2.413645in}}%
\pgfpathlineto{\pgfqpoint{3.409292in}{2.406531in}}%
\pgfpathlineto{\pgfqpoint{3.401463in}{2.399501in}}%
\pgfpathlineto{\pgfqpoint{3.393626in}{2.392557in}}%
\pgfpathclose%
\pgfusepath{fill}%
\end{pgfscope}%
\begin{pgfscope}%
\pgfpathrectangle{\pgfqpoint{1.254980in}{0.150000in}}{\pgfqpoint{5.490039in}{5.490039in}}%
\pgfusepath{clip}%
\pgfsetbuttcap%
\pgfsetroundjoin%
\definecolor{currentfill}{rgb}{0.214298,0.355619,0.551184}%
\pgfsetfillcolor{currentfill}%
\pgfsetfillopacity{0.700000}%
\pgfsetlinewidth{0.000000pt}%
\definecolor{currentstroke}{rgb}{0.000000,0.000000,0.000000}%
\pgfsetstrokecolor{currentstroke}%
\pgfsetdash{}{0pt}%
\pgfpathmoveto{\pgfqpoint{5.387113in}{2.720566in}}%
\pgfpathlineto{\pgfqpoint{5.400702in}{2.723414in}}%
\pgfpathlineto{\pgfqpoint{5.414305in}{2.726374in}}%
\pgfpathlineto{\pgfqpoint{5.427919in}{2.729447in}}%
\pgfpathlineto{\pgfqpoint{5.441547in}{2.732632in}}%
\pgfpathlineto{\pgfqpoint{5.448667in}{2.741534in}}%
\pgfpathlineto{\pgfqpoint{5.455782in}{2.750420in}}%
\pgfpathlineto{\pgfqpoint{5.462893in}{2.759293in}}%
\pgfpathlineto{\pgfqpoint{5.469998in}{2.768153in}}%
\pgfpathlineto{\pgfqpoint{5.456383in}{2.765106in}}%
\pgfpathlineto{\pgfqpoint{5.442780in}{2.762172in}}%
\pgfpathlineto{\pgfqpoint{5.429190in}{2.759351in}}%
\pgfpathlineto{\pgfqpoint{5.415613in}{2.756642in}}%
\pgfpathlineto{\pgfqpoint{5.408495in}{2.747637in}}%
\pgfpathlineto{\pgfqpoint{5.401373in}{2.738624in}}%
\pgfpathlineto{\pgfqpoint{5.394245in}{2.729601in}}%
\pgfpathlineto{\pgfqpoint{5.387113in}{2.720566in}}%
\pgfpathclose%
\pgfusepath{fill}%
\end{pgfscope}%
\begin{pgfscope}%
\pgfpathrectangle{\pgfqpoint{1.254980in}{0.150000in}}{\pgfqpoint{5.490039in}{5.490039in}}%
\pgfusepath{clip}%
\pgfsetbuttcap%
\pgfsetroundjoin%
\definecolor{currentfill}{rgb}{0.275191,0.194905,0.496005}%
\pgfsetfillcolor{currentfill}%
\pgfsetfillopacity{0.700000}%
\pgfsetlinewidth{0.000000pt}%
\definecolor{currentstroke}{rgb}{0.000000,0.000000,0.000000}%
\pgfsetstrokecolor{currentstroke}%
\pgfsetdash{}{0pt}%
\pgfpathmoveto{\pgfqpoint{4.724801in}{2.370910in}}%
\pgfpathlineto{\pgfqpoint{4.738119in}{2.370583in}}%
\pgfpathlineto{\pgfqpoint{4.751447in}{2.370375in}}%
\pgfpathlineto{\pgfqpoint{4.764784in}{2.370286in}}%
\pgfpathlineto{\pgfqpoint{4.778130in}{2.370314in}}%
\pgfpathlineto{\pgfqpoint{4.785488in}{2.380318in}}%
\pgfpathlineto{\pgfqpoint{4.792842in}{2.390309in}}%
\pgfpathlineto{\pgfqpoint{4.800191in}{2.400287in}}%
\pgfpathlineto{\pgfqpoint{4.807535in}{2.410254in}}%
\pgfpathlineto{\pgfqpoint{4.794197in}{2.410234in}}%
\pgfpathlineto{\pgfqpoint{4.780869in}{2.410333in}}%
\pgfpathlineto{\pgfqpoint{4.767550in}{2.410550in}}%
\pgfpathlineto{\pgfqpoint{4.754241in}{2.410885in}}%
\pgfpathlineto{\pgfqpoint{4.746888in}{2.400904in}}%
\pgfpathlineto{\pgfqpoint{4.739531in}{2.390914in}}%
\pgfpathlineto{\pgfqpoint{4.732168in}{2.380916in}}%
\pgfpathlineto{\pgfqpoint{4.724801in}{2.370910in}}%
\pgfpathclose%
\pgfusepath{fill}%
\end{pgfscope}%
\begin{pgfscope}%
\pgfpathrectangle{\pgfqpoint{1.254980in}{0.150000in}}{\pgfqpoint{5.490039in}{5.490039in}}%
\pgfusepath{clip}%
\pgfsetbuttcap%
\pgfsetroundjoin%
\definecolor{currentfill}{rgb}{0.283091,0.110553,0.431554}%
\pgfsetfillcolor{currentfill}%
\pgfsetfillopacity{0.700000}%
\pgfsetlinewidth{0.000000pt}%
\definecolor{currentstroke}{rgb}{0.000000,0.000000,0.000000}%
\pgfsetstrokecolor{currentstroke}%
\pgfsetdash{}{0pt}%
\pgfpathmoveto{\pgfqpoint{4.258302in}{2.217440in}}%
\pgfpathlineto{\pgfqpoint{4.271474in}{2.213990in}}%
\pgfpathlineto{\pgfqpoint{4.284653in}{2.210666in}}%
\pgfpathlineto{\pgfqpoint{4.297838in}{2.207467in}}%
\pgfpathlineto{\pgfqpoint{4.311030in}{2.204393in}}%
\pgfpathlineto{\pgfqpoint{4.318536in}{2.214217in}}%
\pgfpathlineto{\pgfqpoint{4.326037in}{2.224052in}}%
\pgfpathlineto{\pgfqpoint{4.333534in}{2.233898in}}%
\pgfpathlineto{\pgfqpoint{4.341026in}{2.243754in}}%
\pgfpathlineto{\pgfqpoint{4.327844in}{2.246757in}}%
\pgfpathlineto{\pgfqpoint{4.314668in}{2.249885in}}%
\pgfpathlineto{\pgfqpoint{4.301499in}{2.253138in}}%
\pgfpathlineto{\pgfqpoint{4.288338in}{2.256517in}}%
\pgfpathlineto{\pgfqpoint{4.280836in}{2.246726in}}%
\pgfpathlineto{\pgfqpoint{4.273329in}{2.236950in}}%
\pgfpathlineto{\pgfqpoint{4.265818in}{2.227187in}}%
\pgfpathlineto{\pgfqpoint{4.258302in}{2.217440in}}%
\pgfpathclose%
\pgfusepath{fill}%
\end{pgfscope}%
\begin{pgfscope}%
\pgfpathrectangle{\pgfqpoint{1.254980in}{0.150000in}}{\pgfqpoint{5.490039in}{5.490039in}}%
\pgfusepath{clip}%
\pgfsetbuttcap%
\pgfsetroundjoin%
\definecolor{currentfill}{rgb}{0.282327,0.094955,0.417331}%
\pgfsetfillcolor{currentfill}%
\pgfsetfillopacity{0.700000}%
\pgfsetlinewidth{0.000000pt}%
\definecolor{currentstroke}{rgb}{0.000000,0.000000,0.000000}%
\pgfsetstrokecolor{currentstroke}%
\pgfsetdash{}{0pt}%
\pgfpathmoveto{\pgfqpoint{4.040157in}{2.193649in}}%
\pgfpathlineto{\pgfqpoint{4.053280in}{2.188473in}}%
\pgfpathlineto{\pgfqpoint{4.066408in}{2.183427in}}%
\pgfpathlineto{\pgfqpoint{4.079542in}{2.178511in}}%
\pgfpathlineto{\pgfqpoint{4.092681in}{2.173725in}}%
\pgfpathlineto{\pgfqpoint{4.100258in}{2.183121in}}%
\pgfpathlineto{\pgfqpoint{4.107829in}{2.192543in}}%
\pgfpathlineto{\pgfqpoint{4.115396in}{2.201991in}}%
\pgfpathlineto{\pgfqpoint{4.122957in}{2.211464in}}%
\pgfpathlineto{\pgfqpoint{4.109829in}{2.216148in}}%
\pgfpathlineto{\pgfqpoint{4.096707in}{2.220961in}}%
\pgfpathlineto{\pgfqpoint{4.083590in}{2.225904in}}%
\pgfpathlineto{\pgfqpoint{4.070478in}{2.230977in}}%
\pgfpathlineto{\pgfqpoint{4.062905in}{2.221600in}}%
\pgfpathlineto{\pgfqpoint{4.055327in}{2.212253in}}%
\pgfpathlineto{\pgfqpoint{4.047745in}{2.202936in}}%
\pgfpathlineto{\pgfqpoint{4.040157in}{2.193649in}}%
\pgfpathclose%
\pgfusepath{fill}%
\end{pgfscope}%
\begin{pgfscope}%
\pgfpathrectangle{\pgfqpoint{1.254980in}{0.150000in}}{\pgfqpoint{5.490039in}{5.490039in}}%
\pgfusepath{clip}%
\pgfsetbuttcap%
\pgfsetroundjoin%
\definecolor{currentfill}{rgb}{0.204903,0.375746,0.553533}%
\pgfsetfillcolor{currentfill}%
\pgfsetfillopacity{0.700000}%
\pgfsetlinewidth{0.000000pt}%
\definecolor{currentstroke}{rgb}{0.000000,0.000000,0.000000}%
\pgfsetstrokecolor{currentstroke}%
\pgfsetdash{}{0pt}%
\pgfpathmoveto{\pgfqpoint{5.469998in}{2.768153in}}%
\pgfpathlineto{\pgfqpoint{5.483626in}{2.771312in}}%
\pgfpathlineto{\pgfqpoint{5.497266in}{2.774582in}}%
\pgfpathlineto{\pgfqpoint{5.510920in}{2.777965in}}%
\pgfpathlineto{\pgfqpoint{5.524586in}{2.781460in}}%
\pgfpathlineto{\pgfqpoint{5.531674in}{2.790161in}}%
\pgfpathlineto{\pgfqpoint{5.538756in}{2.798848in}}%
\pgfpathlineto{\pgfqpoint{5.545834in}{2.807524in}}%
\pgfpathlineto{\pgfqpoint{5.552906in}{2.816190in}}%
\pgfpathlineto{\pgfqpoint{5.539253in}{2.812850in}}%
\pgfpathlineto{\pgfqpoint{5.525612in}{2.809622in}}%
\pgfpathlineto{\pgfqpoint{5.511985in}{2.806506in}}%
\pgfpathlineto{\pgfqpoint{5.498370in}{2.803503in}}%
\pgfpathlineto{\pgfqpoint{5.491284in}{2.794676in}}%
\pgfpathlineto{\pgfqpoint{5.484194in}{2.785843in}}%
\pgfpathlineto{\pgfqpoint{5.477098in}{2.777002in}}%
\pgfpathlineto{\pgfqpoint{5.469998in}{2.768153in}}%
\pgfpathclose%
\pgfusepath{fill}%
\end{pgfscope}%
\begin{pgfscope}%
\pgfpathrectangle{\pgfqpoint{1.254980in}{0.150000in}}{\pgfqpoint{5.490039in}{5.490039in}}%
\pgfusepath{clip}%
\pgfsetbuttcap%
\pgfsetroundjoin%
\definecolor{currentfill}{rgb}{0.278826,0.175490,0.483397}%
\pgfsetfillcolor{currentfill}%
\pgfsetfillopacity{0.700000}%
\pgfsetlinewidth{0.000000pt}%
\definecolor{currentstroke}{rgb}{0.000000,0.000000,0.000000}%
\pgfsetstrokecolor{currentstroke}%
\pgfsetdash{}{0pt}%
\pgfpathmoveto{\pgfqpoint{4.642071in}{2.333314in}}%
\pgfpathlineto{\pgfqpoint{4.655361in}{2.332504in}}%
\pgfpathlineto{\pgfqpoint{4.668661in}{2.331813in}}%
\pgfpathlineto{\pgfqpoint{4.681969in}{2.331241in}}%
\pgfpathlineto{\pgfqpoint{4.695286in}{2.330789in}}%
\pgfpathlineto{\pgfqpoint{4.702672in}{2.340834in}}%
\pgfpathlineto{\pgfqpoint{4.710053in}{2.350869in}}%
\pgfpathlineto{\pgfqpoint{4.717430in}{2.360894in}}%
\pgfpathlineto{\pgfqpoint{4.724801in}{2.370910in}}%
\pgfpathlineto{\pgfqpoint{4.711493in}{2.371355in}}%
\pgfpathlineto{\pgfqpoint{4.698193in}{2.371920in}}%
\pgfpathlineto{\pgfqpoint{4.684902in}{2.372604in}}%
\pgfpathlineto{\pgfqpoint{4.671621in}{2.373407in}}%
\pgfpathlineto{\pgfqpoint{4.664241in}{2.363392in}}%
\pgfpathlineto{\pgfqpoint{4.656855in}{2.353372in}}%
\pgfpathlineto{\pgfqpoint{4.649466in}{2.343346in}}%
\pgfpathlineto{\pgfqpoint{4.642071in}{2.333314in}}%
\pgfpathclose%
\pgfusepath{fill}%
\end{pgfscope}%
\begin{pgfscope}%
\pgfpathrectangle{\pgfqpoint{1.254980in}{0.150000in}}{\pgfqpoint{5.490039in}{5.490039in}}%
\pgfusepath{clip}%
\pgfsetbuttcap%
\pgfsetroundjoin%
\definecolor{currentfill}{rgb}{0.195860,0.395433,0.555276}%
\pgfsetfillcolor{currentfill}%
\pgfsetfillopacity{0.700000}%
\pgfsetlinewidth{0.000000pt}%
\definecolor{currentstroke}{rgb}{0.000000,0.000000,0.000000}%
\pgfsetstrokecolor{currentstroke}%
\pgfsetdash{}{0pt}%
\pgfpathmoveto{\pgfqpoint{5.552906in}{2.816190in}}%
\pgfpathlineto{\pgfqpoint{5.566573in}{2.819641in}}%
\pgfpathlineto{\pgfqpoint{5.580252in}{2.823204in}}%
\pgfpathlineto{\pgfqpoint{5.593945in}{2.826879in}}%
\pgfpathlineto{\pgfqpoint{5.607651in}{2.830665in}}%
\pgfpathlineto{\pgfqpoint{5.614705in}{2.839158in}}%
\pgfpathlineto{\pgfqpoint{5.621754in}{2.847640in}}%
\pgfpathlineto{\pgfqpoint{5.628798in}{2.856113in}}%
\pgfpathlineto{\pgfqpoint{5.635837in}{2.864580in}}%
\pgfpathlineto{\pgfqpoint{5.622145in}{2.860966in}}%
\pgfpathlineto{\pgfqpoint{5.608466in}{2.857463in}}%
\pgfpathlineto{\pgfqpoint{5.594800in}{2.854071in}}%
\pgfpathlineto{\pgfqpoint{5.581148in}{2.850791in}}%
\pgfpathlineto{\pgfqpoint{5.574094in}{2.842147in}}%
\pgfpathlineto{\pgfqpoint{5.567037in}{2.833500in}}%
\pgfpathlineto{\pgfqpoint{5.559974in}{2.824848in}}%
\pgfpathlineto{\pgfqpoint{5.552906in}{2.816190in}}%
\pgfpathclose%
\pgfusepath{fill}%
\end{pgfscope}%
\begin{pgfscope}%
\pgfpathrectangle{\pgfqpoint{1.254980in}{0.150000in}}{\pgfqpoint{5.490039in}{5.490039in}}%
\pgfusepath{clip}%
\pgfsetbuttcap%
\pgfsetroundjoin%
\definecolor{currentfill}{rgb}{0.283187,0.125848,0.444960}%
\pgfsetfillcolor{currentfill}%
\pgfsetfillopacity{0.700000}%
\pgfsetlinewidth{0.000000pt}%
\definecolor{currentstroke}{rgb}{0.000000,0.000000,0.000000}%
\pgfsetstrokecolor{currentstroke}%
\pgfsetdash{}{0pt}%
\pgfpathmoveto{\pgfqpoint{3.634005in}{2.259149in}}%
\pgfpathlineto{\pgfqpoint{3.647083in}{2.250291in}}%
\pgfpathlineto{\pgfqpoint{3.660164in}{2.241577in}}%
\pgfpathlineto{\pgfqpoint{3.673247in}{2.233004in}}%
\pgfpathlineto{\pgfqpoint{3.686332in}{2.224574in}}%
\pgfpathlineto{\pgfqpoint{3.694054in}{2.232647in}}%
\pgfpathlineto{\pgfqpoint{3.701770in}{2.240782in}}%
\pgfpathlineto{\pgfqpoint{3.709480in}{2.248976in}}%
\pgfpathlineto{\pgfqpoint{3.717184in}{2.257229in}}%
\pgfpathlineto{\pgfqpoint{3.704115in}{2.265508in}}%
\pgfpathlineto{\pgfqpoint{3.691048in}{2.273928in}}%
\pgfpathlineto{\pgfqpoint{3.677983in}{2.282491in}}%
\pgfpathlineto{\pgfqpoint{3.664920in}{2.291197in}}%
\pgfpathlineto{\pgfqpoint{3.657200in}{2.283090in}}%
\pgfpathlineto{\pgfqpoint{3.649475in}{2.275046in}}%
\pgfpathlineto{\pgfqpoint{3.641743in}{2.267065in}}%
\pgfpathlineto{\pgfqpoint{3.634005in}{2.259149in}}%
\pgfpathclose%
\pgfusepath{fill}%
\end{pgfscope}%
\begin{pgfscope}%
\pgfpathrectangle{\pgfqpoint{1.254980in}{0.150000in}}{\pgfqpoint{5.490039in}{5.490039in}}%
\pgfusepath{clip}%
\pgfsetbuttcap%
\pgfsetroundjoin%
\definecolor{currentfill}{rgb}{0.187231,0.414746,0.556547}%
\pgfsetfillcolor{currentfill}%
\pgfsetfillopacity{0.700000}%
\pgfsetlinewidth{0.000000pt}%
\definecolor{currentstroke}{rgb}{0.000000,0.000000,0.000000}%
\pgfsetstrokecolor{currentstroke}%
\pgfsetdash{}{0pt}%
\pgfpathmoveto{\pgfqpoint{5.635837in}{2.864580in}}%
\pgfpathlineto{\pgfqpoint{5.649542in}{2.868306in}}%
\pgfpathlineto{\pgfqpoint{5.663261in}{2.872142in}}%
\pgfpathlineto{\pgfqpoint{5.676994in}{2.876090in}}%
\pgfpathlineto{\pgfqpoint{5.690740in}{2.880149in}}%
\pgfpathlineto{\pgfqpoint{5.697760in}{2.888428in}}%
\pgfpathlineto{\pgfqpoint{5.704774in}{2.896701in}}%
\pgfpathlineto{\pgfqpoint{5.711784in}{2.904969in}}%
\pgfpathlineto{\pgfqpoint{5.718789in}{2.913234in}}%
\pgfpathlineto{\pgfqpoint{5.705058in}{2.909364in}}%
\pgfpathlineto{\pgfqpoint{5.691340in}{2.905604in}}%
\pgfpathlineto{\pgfqpoint{5.677636in}{2.901956in}}%
\pgfpathlineto{\pgfqpoint{5.663945in}{2.898418in}}%
\pgfpathlineto{\pgfqpoint{5.656925in}{2.889959in}}%
\pgfpathlineto{\pgfqpoint{5.649901in}{2.881501in}}%
\pgfpathlineto{\pgfqpoint{5.642871in}{2.873042in}}%
\pgfpathlineto{\pgfqpoint{5.635837in}{2.864580in}}%
\pgfpathclose%
\pgfusepath{fill}%
\end{pgfscope}%
\begin{pgfscope}%
\pgfpathrectangle{\pgfqpoint{1.254980in}{0.150000in}}{\pgfqpoint{5.490039in}{5.490039in}}%
\pgfusepath{clip}%
\pgfsetbuttcap%
\pgfsetroundjoin%
\definecolor{currentfill}{rgb}{0.281412,0.155834,0.469201}%
\pgfsetfillcolor{currentfill}%
\pgfsetfillopacity{0.700000}%
\pgfsetlinewidth{0.000000pt}%
\definecolor{currentstroke}{rgb}{0.000000,0.000000,0.000000}%
\pgfsetstrokecolor{currentstroke}%
\pgfsetdash{}{0pt}%
\pgfpathmoveto{\pgfqpoint{4.559337in}{2.297659in}}%
\pgfpathlineto{\pgfqpoint{4.572602in}{2.296345in}}%
\pgfpathlineto{\pgfqpoint{4.585874in}{2.295151in}}%
\pgfpathlineto{\pgfqpoint{4.599156in}{2.294077in}}%
\pgfpathlineto{\pgfqpoint{4.612446in}{2.293124in}}%
\pgfpathlineto{\pgfqpoint{4.619859in}{2.303181in}}%
\pgfpathlineto{\pgfqpoint{4.627268in}{2.313232in}}%
\pgfpathlineto{\pgfqpoint{4.634672in}{2.323276in}}%
\pgfpathlineto{\pgfqpoint{4.642071in}{2.333314in}}%
\pgfpathlineto{\pgfqpoint{4.628790in}{2.334245in}}%
\pgfpathlineto{\pgfqpoint{4.615517in}{2.335296in}}%
\pgfpathlineto{\pgfqpoint{4.602253in}{2.336466in}}%
\pgfpathlineto{\pgfqpoint{4.588997in}{2.337758in}}%
\pgfpathlineto{\pgfqpoint{4.581589in}{2.327737in}}%
\pgfpathlineto{\pgfqpoint{4.574177in}{2.317714in}}%
\pgfpathlineto{\pgfqpoint{4.566759in}{2.307688in}}%
\pgfpathlineto{\pgfqpoint{4.559337in}{2.297659in}}%
\pgfpathclose%
\pgfusepath{fill}%
\end{pgfscope}%
\begin{pgfscope}%
\pgfpathrectangle{\pgfqpoint{1.254980in}{0.150000in}}{\pgfqpoint{5.490039in}{5.490039in}}%
\pgfusepath{clip}%
\pgfsetbuttcap%
\pgfsetroundjoin%
\definecolor{currentfill}{rgb}{0.139147,0.533812,0.555298}%
\pgfsetfillcolor{currentfill}%
\pgfsetfillopacity{0.700000}%
\pgfsetlinewidth{0.000000pt}%
\definecolor{currentstroke}{rgb}{0.000000,0.000000,0.000000}%
\pgfsetstrokecolor{currentstroke}%
\pgfsetdash{}{0pt}%
\pgfpathmoveto{\pgfqpoint{2.709782in}{3.220722in}}%
\pgfpathlineto{\pgfqpoint{2.723069in}{3.199962in}}%
\pgfpathlineto{\pgfqpoint{2.736348in}{3.179412in}}%
\pgfpathlineto{\pgfqpoint{2.749619in}{3.159073in}}%
\pgfpathlineto{\pgfqpoint{2.762882in}{3.138941in}}%
\pgfpathlineto{\pgfqpoint{2.771009in}{3.143687in}}%
\pgfpathlineto{\pgfqpoint{2.779126in}{3.148569in}}%
\pgfpathlineto{\pgfqpoint{2.787234in}{3.153585in}}%
\pgfpathlineto{\pgfqpoint{2.795331in}{3.158734in}}%
\pgfpathlineto{\pgfqpoint{2.782097in}{3.178681in}}%
\pgfpathlineto{\pgfqpoint{2.768855in}{3.198835in}}%
\pgfpathlineto{\pgfqpoint{2.755606in}{3.219199in}}%
\pgfpathlineto{\pgfqpoint{2.742348in}{3.239773in}}%
\pgfpathlineto{\pgfqpoint{2.734222in}{3.234802in}}%
\pgfpathlineto{\pgfqpoint{2.726086in}{3.229969in}}%
\pgfpathlineto{\pgfqpoint{2.717940in}{3.225276in}}%
\pgfpathlineto{\pgfqpoint{2.709782in}{3.220722in}}%
\pgfpathclose%
\pgfusepath{fill}%
\end{pgfscope}%
\begin{pgfscope}%
\pgfpathrectangle{\pgfqpoint{1.254980in}{0.150000in}}{\pgfqpoint{5.490039in}{5.490039in}}%
\pgfusepath{clip}%
\pgfsetbuttcap%
\pgfsetroundjoin%
\definecolor{currentfill}{rgb}{0.279574,0.170599,0.479997}%
\pgfsetfillcolor{currentfill}%
\pgfsetfillopacity{0.700000}%
\pgfsetlinewidth{0.000000pt}%
\definecolor{currentstroke}{rgb}{0.000000,0.000000,0.000000}%
\pgfsetstrokecolor{currentstroke}%
\pgfsetdash{}{0pt}%
\pgfpathmoveto{\pgfqpoint{3.445951in}{2.347858in}}%
\pgfpathlineto{\pgfqpoint{3.459033in}{2.337064in}}%
\pgfpathlineto{\pgfqpoint{3.472116in}{2.326422in}}%
\pgfpathlineto{\pgfqpoint{3.485198in}{2.315930in}}%
\pgfpathlineto{\pgfqpoint{3.498282in}{2.305588in}}%
\pgfpathlineto{\pgfqpoint{3.506080in}{2.312884in}}%
\pgfpathlineto{\pgfqpoint{3.513872in}{2.320257in}}%
\pgfpathlineto{\pgfqpoint{3.521658in}{2.327707in}}%
\pgfpathlineto{\pgfqpoint{3.529436in}{2.335232in}}%
\pgfpathlineto{\pgfqpoint{3.516371in}{2.345405in}}%
\pgfpathlineto{\pgfqpoint{3.503306in}{2.355727in}}%
\pgfpathlineto{\pgfqpoint{3.490242in}{2.366200in}}%
\pgfpathlineto{\pgfqpoint{3.477179in}{2.376823in}}%
\pgfpathlineto{\pgfqpoint{3.469383in}{2.369462in}}%
\pgfpathlineto{\pgfqpoint{3.461579in}{2.362180in}}%
\pgfpathlineto{\pgfqpoint{3.453769in}{2.354978in}}%
\pgfpathlineto{\pgfqpoint{3.445951in}{2.347858in}}%
\pgfpathclose%
\pgfusepath{fill}%
\end{pgfscope}%
\begin{pgfscope}%
\pgfpathrectangle{\pgfqpoint{1.254980in}{0.150000in}}{\pgfqpoint{5.490039in}{5.490039in}}%
\pgfusepath{clip}%
\pgfsetbuttcap%
\pgfsetroundjoin%
\definecolor{currentfill}{rgb}{0.177423,0.437527,0.557565}%
\pgfsetfillcolor{currentfill}%
\pgfsetfillopacity{0.700000}%
\pgfsetlinewidth{0.000000pt}%
\definecolor{currentstroke}{rgb}{0.000000,0.000000,0.000000}%
\pgfsetstrokecolor{currentstroke}%
\pgfsetdash{}{0pt}%
\pgfpathmoveto{\pgfqpoint{5.718789in}{2.913234in}}%
\pgfpathlineto{\pgfqpoint{5.732534in}{2.917215in}}%
\pgfpathlineto{\pgfqpoint{5.746292in}{2.921307in}}%
\pgfpathlineto{\pgfqpoint{5.760065in}{2.925509in}}%
\pgfpathlineto{\pgfqpoint{5.773851in}{2.929822in}}%
\pgfpathlineto{\pgfqpoint{5.780836in}{2.937887in}}%
\pgfpathlineto{\pgfqpoint{5.787815in}{2.945949in}}%
\pgfpathlineto{\pgfqpoint{5.794790in}{2.954011in}}%
\pgfpathlineto{\pgfqpoint{5.801761in}{2.962073in}}%
\pgfpathlineto{\pgfqpoint{5.787990in}{2.957965in}}%
\pgfpathlineto{\pgfqpoint{5.774234in}{2.953968in}}%
\pgfpathlineto{\pgfqpoint{5.760491in}{2.950081in}}%
\pgfpathlineto{\pgfqpoint{5.746762in}{2.946304in}}%
\pgfpathlineto{\pgfqpoint{5.739775in}{2.938031in}}%
\pgfpathlineto{\pgfqpoint{5.732784in}{2.929763in}}%
\pgfpathlineto{\pgfqpoint{5.725789in}{2.921498in}}%
\pgfpathlineto{\pgfqpoint{5.718789in}{2.913234in}}%
\pgfpathclose%
\pgfusepath{fill}%
\end{pgfscope}%
\begin{pgfscope}%
\pgfpathrectangle{\pgfqpoint{1.254980in}{0.150000in}}{\pgfqpoint{5.490039in}{5.490039in}}%
\pgfusepath{clip}%
\pgfsetbuttcap%
\pgfsetroundjoin%
\definecolor{currentfill}{rgb}{0.169646,0.456262,0.558030}%
\pgfsetfillcolor{currentfill}%
\pgfsetfillopacity{0.700000}%
\pgfsetlinewidth{0.000000pt}%
\definecolor{currentstroke}{rgb}{0.000000,0.000000,0.000000}%
\pgfsetstrokecolor{currentstroke}%
\pgfsetdash{}{0pt}%
\pgfpathmoveto{\pgfqpoint{5.801761in}{2.962073in}}%
\pgfpathlineto{\pgfqpoint{5.815545in}{2.966291in}}%
\pgfpathlineto{\pgfqpoint{5.829343in}{2.970619in}}%
\pgfpathlineto{\pgfqpoint{5.843156in}{2.975058in}}%
\pgfpathlineto{\pgfqpoint{5.856983in}{2.979606in}}%
\pgfpathlineto{\pgfqpoint{5.863932in}{2.987457in}}%
\pgfpathlineto{\pgfqpoint{5.870876in}{2.995309in}}%
\pgfpathlineto{\pgfqpoint{5.877815in}{3.003165in}}%
\pgfpathlineto{\pgfqpoint{5.884750in}{3.011027in}}%
\pgfpathlineto{\pgfqpoint{5.870941in}{3.006700in}}%
\pgfpathlineto{\pgfqpoint{5.857145in}{3.002483in}}%
\pgfpathlineto{\pgfqpoint{5.843363in}{2.998376in}}%
\pgfpathlineto{\pgfqpoint{5.829596in}{2.994379in}}%
\pgfpathlineto{\pgfqpoint{5.822644in}{2.986290in}}%
\pgfpathlineto{\pgfqpoint{5.815687in}{2.978210in}}%
\pgfpathlineto{\pgfqpoint{5.808726in}{2.970139in}}%
\pgfpathlineto{\pgfqpoint{5.801761in}{2.962073in}}%
\pgfpathclose%
\pgfusepath{fill}%
\end{pgfscope}%
\begin{pgfscope}%
\pgfpathrectangle{\pgfqpoint{1.254980in}{0.150000in}}{\pgfqpoint{5.490039in}{5.490039in}}%
\pgfusepath{clip}%
\pgfsetbuttcap%
\pgfsetroundjoin%
\definecolor{currentfill}{rgb}{0.282656,0.100196,0.422160}%
\pgfsetfillcolor{currentfill}%
\pgfsetfillopacity{0.700000}%
\pgfsetlinewidth{0.000000pt}%
\definecolor{currentstroke}{rgb}{0.000000,0.000000,0.000000}%
\pgfsetstrokecolor{currentstroke}%
\pgfsetdash{}{0pt}%
\pgfpathmoveto{\pgfqpoint{4.175525in}{2.194016in}}%
\pgfpathlineto{\pgfqpoint{4.188682in}{2.189974in}}%
\pgfpathlineto{\pgfqpoint{4.201845in}{2.186058in}}%
\pgfpathlineto{\pgfqpoint{4.215014in}{2.182270in}}%
\pgfpathlineto{\pgfqpoint{4.228190in}{2.178608in}}%
\pgfpathlineto{\pgfqpoint{4.235725in}{2.188291in}}%
\pgfpathlineto{\pgfqpoint{4.243256in}{2.197992in}}%
\pgfpathlineto{\pgfqpoint{4.250781in}{2.207708in}}%
\pgfpathlineto{\pgfqpoint{4.258302in}{2.217440in}}%
\pgfpathlineto{\pgfqpoint{4.245137in}{2.221016in}}%
\pgfpathlineto{\pgfqpoint{4.231978in}{2.224718in}}%
\pgfpathlineto{\pgfqpoint{4.218825in}{2.228546in}}%
\pgfpathlineto{\pgfqpoint{4.205679in}{2.232502in}}%
\pgfpathlineto{\pgfqpoint{4.198148in}{2.222851in}}%
\pgfpathlineto{\pgfqpoint{4.190612in}{2.213219in}}%
\pgfpathlineto{\pgfqpoint{4.183071in}{2.203607in}}%
\pgfpathlineto{\pgfqpoint{4.175525in}{2.194016in}}%
\pgfpathclose%
\pgfusepath{fill}%
\end{pgfscope}%
\begin{pgfscope}%
\pgfpathrectangle{\pgfqpoint{1.254980in}{0.150000in}}{\pgfqpoint{5.490039in}{5.490039in}}%
\pgfusepath{clip}%
\pgfsetbuttcap%
\pgfsetroundjoin%
\definecolor{currentfill}{rgb}{0.282623,0.140926,0.457517}%
\pgfsetfillcolor{currentfill}%
\pgfsetfillopacity{0.700000}%
\pgfsetlinewidth{0.000000pt}%
\definecolor{currentstroke}{rgb}{0.000000,0.000000,0.000000}%
\pgfsetstrokecolor{currentstroke}%
\pgfsetdash{}{0pt}%
\pgfpathmoveto{\pgfqpoint{4.476592in}{2.264145in}}%
\pgfpathlineto{\pgfqpoint{4.489833in}{2.262306in}}%
\pgfpathlineto{\pgfqpoint{4.503081in}{2.260589in}}%
\pgfpathlineto{\pgfqpoint{4.516337in}{2.258994in}}%
\pgfpathlineto{\pgfqpoint{4.529602in}{2.257520in}}%
\pgfpathlineto{\pgfqpoint{4.537043in}{2.267558in}}%
\pgfpathlineto{\pgfqpoint{4.544479in}{2.277595in}}%
\pgfpathlineto{\pgfqpoint{4.551911in}{2.287628in}}%
\pgfpathlineto{\pgfqpoint{4.559337in}{2.297659in}}%
\pgfpathlineto{\pgfqpoint{4.546082in}{2.299095in}}%
\pgfpathlineto{\pgfqpoint{4.532834in}{2.300652in}}%
\pgfpathlineto{\pgfqpoint{4.519594in}{2.302330in}}%
\pgfpathlineto{\pgfqpoint{4.506363in}{2.304130in}}%
\pgfpathlineto{\pgfqpoint{4.498927in}{2.294132in}}%
\pgfpathlineto{\pgfqpoint{4.491487in}{2.284135in}}%
\pgfpathlineto{\pgfqpoint{4.484042in}{2.274139in}}%
\pgfpathlineto{\pgfqpoint{4.476592in}{2.264145in}}%
\pgfpathclose%
\pgfusepath{fill}%
\end{pgfscope}%
\begin{pgfscope}%
\pgfpathrectangle{\pgfqpoint{1.254980in}{0.150000in}}{\pgfqpoint{5.490039in}{5.490039in}}%
\pgfusepath{clip}%
\pgfsetbuttcap%
\pgfsetroundjoin%
\definecolor{currentfill}{rgb}{0.162142,0.474838,0.558140}%
\pgfsetfillcolor{currentfill}%
\pgfsetfillopacity{0.700000}%
\pgfsetlinewidth{0.000000pt}%
\definecolor{currentstroke}{rgb}{0.000000,0.000000,0.000000}%
\pgfsetstrokecolor{currentstroke}%
\pgfsetdash{}{0pt}%
\pgfpathmoveto{\pgfqpoint{5.884750in}{3.011027in}}%
\pgfpathlineto{\pgfqpoint{5.898574in}{3.015463in}}%
\pgfpathlineto{\pgfqpoint{5.912413in}{3.020010in}}%
\pgfpathlineto{\pgfqpoint{5.926265in}{3.024666in}}%
\pgfpathlineto{\pgfqpoint{5.940133in}{3.029431in}}%
\pgfpathlineto{\pgfqpoint{5.947045in}{3.037069in}}%
\pgfpathlineto{\pgfqpoint{5.953954in}{3.044715in}}%
\pgfpathlineto{\pgfqpoint{5.960857in}{3.052369in}}%
\pgfpathlineto{\pgfqpoint{5.967757in}{3.060034in}}%
\pgfpathlineto{\pgfqpoint{5.953908in}{3.055507in}}%
\pgfpathlineto{\pgfqpoint{5.940073in}{3.051089in}}%
\pgfpathlineto{\pgfqpoint{5.926253in}{3.046781in}}%
\pgfpathlineto{\pgfqpoint{5.912447in}{3.042582in}}%
\pgfpathlineto{\pgfqpoint{5.905529in}{3.034672in}}%
\pgfpathlineto{\pgfqpoint{5.898607in}{3.026778in}}%
\pgfpathlineto{\pgfqpoint{5.891681in}{3.018897in}}%
\pgfpathlineto{\pgfqpoint{5.884750in}{3.011027in}}%
\pgfpathclose%
\pgfusepath{fill}%
\end{pgfscope}%
\begin{pgfscope}%
\pgfpathrectangle{\pgfqpoint{1.254980in}{0.150000in}}{\pgfqpoint{5.490039in}{5.490039in}}%
\pgfusepath{clip}%
\pgfsetbuttcap%
\pgfsetroundjoin%
\definecolor{currentfill}{rgb}{0.225863,0.330805,0.547314}%
\pgfsetfillcolor{currentfill}%
\pgfsetfillopacity{0.700000}%
\pgfsetlinewidth{0.000000pt}%
\definecolor{currentstroke}{rgb}{0.000000,0.000000,0.000000}%
\pgfsetstrokecolor{currentstroke}%
\pgfsetdash{}{0pt}%
\pgfpathmoveto{\pgfqpoint{3.047344in}{2.692329in}}%
\pgfpathlineto{\pgfqpoint{3.060493in}{2.676777in}}%
\pgfpathlineto{\pgfqpoint{3.073638in}{2.661401in}}%
\pgfpathlineto{\pgfqpoint{3.086780in}{2.646200in}}%
\pgfpathlineto{\pgfqpoint{3.099918in}{2.631172in}}%
\pgfpathlineto{\pgfqpoint{3.107896in}{2.636784in}}%
\pgfpathlineto{\pgfqpoint{3.115866in}{2.642507in}}%
\pgfpathlineto{\pgfqpoint{3.123826in}{2.648339in}}%
\pgfpathlineto{\pgfqpoint{3.131779in}{2.654281in}}%
\pgfpathlineto{\pgfqpoint{3.118665in}{2.669116in}}%
\pgfpathlineto{\pgfqpoint{3.105548in}{2.684125in}}%
\pgfpathlineto{\pgfqpoint{3.092427in}{2.699308in}}%
\pgfpathlineto{\pgfqpoint{3.079303in}{2.714668in}}%
\pgfpathlineto{\pgfqpoint{3.071327in}{2.708912in}}%
\pgfpathlineto{\pgfqpoint{3.063341in}{2.703270in}}%
\pgfpathlineto{\pgfqpoint{3.055347in}{2.697742in}}%
\pgfpathlineto{\pgfqpoint{3.047344in}{2.692329in}}%
\pgfpathclose%
\pgfusepath{fill}%
\end{pgfscope}%
\begin{pgfscope}%
\pgfpathrectangle{\pgfqpoint{1.254980in}{0.150000in}}{\pgfqpoint{5.490039in}{5.490039in}}%
\pgfusepath{clip}%
\pgfsetbuttcap%
\pgfsetroundjoin%
\definecolor{currentfill}{rgb}{0.282656,0.100196,0.422160}%
\pgfsetfillcolor{currentfill}%
\pgfsetfillopacity{0.700000}%
\pgfsetlinewidth{0.000000pt}%
\definecolor{currentstroke}{rgb}{0.000000,0.000000,0.000000}%
\pgfsetstrokecolor{currentstroke}%
\pgfsetdash{}{0pt}%
\pgfpathmoveto{\pgfqpoint{3.821843in}{2.196028in}}%
\pgfpathlineto{\pgfqpoint{3.834940in}{2.188998in}}%
\pgfpathlineto{\pgfqpoint{3.848040in}{2.182104in}}%
\pgfpathlineto{\pgfqpoint{3.861145in}{2.175346in}}%
\pgfpathlineto{\pgfqpoint{3.874253in}{2.168723in}}%
\pgfpathlineto{\pgfqpoint{3.881908in}{2.177457in}}%
\pgfpathlineto{\pgfqpoint{3.889558in}{2.186236in}}%
\pgfpathlineto{\pgfqpoint{3.897202in}{2.195059in}}%
\pgfpathlineto{\pgfqpoint{3.904841in}{2.203925in}}%
\pgfpathlineto{\pgfqpoint{3.891747in}{2.210413in}}%
\pgfpathlineto{\pgfqpoint{3.878656in}{2.217036in}}%
\pgfpathlineto{\pgfqpoint{3.865569in}{2.223795in}}%
\pgfpathlineto{\pgfqpoint{3.852486in}{2.230690in}}%
\pgfpathlineto{\pgfqpoint{3.844834in}{2.221953in}}%
\pgfpathlineto{\pgfqpoint{3.837176in}{2.213263in}}%
\pgfpathlineto{\pgfqpoint{3.829512in}{2.204621in}}%
\pgfpathlineto{\pgfqpoint{3.821843in}{2.196028in}}%
\pgfpathclose%
\pgfusepath{fill}%
\end{pgfscope}%
\begin{pgfscope}%
\pgfpathrectangle{\pgfqpoint{1.254980in}{0.150000in}}{\pgfqpoint{5.490039in}{5.490039in}}%
\pgfusepath{clip}%
\pgfsetbuttcap%
\pgfsetroundjoin%
\definecolor{currentfill}{rgb}{0.237441,0.305202,0.541921}%
\pgfsetfillcolor{currentfill}%
\pgfsetfillopacity{0.700000}%
\pgfsetlinewidth{0.000000pt}%
\definecolor{currentstroke}{rgb}{0.000000,0.000000,0.000000}%
\pgfsetstrokecolor{currentstroke}%
\pgfsetdash{}{0pt}%
\pgfpathmoveto{\pgfqpoint{3.099918in}{2.631172in}}%
\pgfpathlineto{\pgfqpoint{3.113053in}{2.616318in}}%
\pgfpathlineto{\pgfqpoint{3.126185in}{2.601635in}}%
\pgfpathlineto{\pgfqpoint{3.139314in}{2.587123in}}%
\pgfpathlineto{\pgfqpoint{3.152440in}{2.572781in}}%
\pgfpathlineto{\pgfqpoint{3.160395in}{2.578590in}}%
\pgfpathlineto{\pgfqpoint{3.168340in}{2.584507in}}%
\pgfpathlineto{\pgfqpoint{3.176278in}{2.590529in}}%
\pgfpathlineto{\pgfqpoint{3.184207in}{2.596655in}}%
\pgfpathlineto{\pgfqpoint{3.171104in}{2.610806in}}%
\pgfpathlineto{\pgfqpoint{3.157998in}{2.625127in}}%
\pgfpathlineto{\pgfqpoint{3.144890in}{2.639618in}}%
\pgfpathlineto{\pgfqpoint{3.131779in}{2.654281in}}%
\pgfpathlineto{\pgfqpoint{3.123826in}{2.648339in}}%
\pgfpathlineto{\pgfqpoint{3.115866in}{2.642507in}}%
\pgfpathlineto{\pgfqpoint{3.107896in}{2.636784in}}%
\pgfpathlineto{\pgfqpoint{3.099918in}{2.631172in}}%
\pgfpathclose%
\pgfusepath{fill}%
\end{pgfscope}%
\begin{pgfscope}%
\pgfpathrectangle{\pgfqpoint{1.254980in}{0.150000in}}{\pgfqpoint{5.490039in}{5.490039in}}%
\pgfusepath{clip}%
\pgfsetbuttcap%
\pgfsetroundjoin%
\definecolor{currentfill}{rgb}{0.214298,0.355619,0.551184}%
\pgfsetfillcolor{currentfill}%
\pgfsetfillopacity{0.700000}%
\pgfsetlinewidth{0.000000pt}%
\definecolor{currentstroke}{rgb}{0.000000,0.000000,0.000000}%
\pgfsetstrokecolor{currentstroke}%
\pgfsetdash{}{0pt}%
\pgfpathmoveto{\pgfqpoint{2.994710in}{2.756319in}}%
\pgfpathlineto{\pgfqpoint{3.007875in}{2.740052in}}%
\pgfpathlineto{\pgfqpoint{3.021036in}{2.723966in}}%
\pgfpathlineto{\pgfqpoint{3.034192in}{2.708059in}}%
\pgfpathlineto{\pgfqpoint{3.047344in}{2.692329in}}%
\pgfpathlineto{\pgfqpoint{3.055347in}{2.697742in}}%
\pgfpathlineto{\pgfqpoint{3.063341in}{2.703270in}}%
\pgfpathlineto{\pgfqpoint{3.071327in}{2.708912in}}%
\pgfpathlineto{\pgfqpoint{3.079303in}{2.714668in}}%
\pgfpathlineto{\pgfqpoint{3.066176in}{2.730203in}}%
\pgfpathlineto{\pgfqpoint{3.053044in}{2.745917in}}%
\pgfpathlineto{\pgfqpoint{3.039909in}{2.761809in}}%
\pgfpathlineto{\pgfqpoint{3.026770in}{2.777882in}}%
\pgfpathlineto{\pgfqpoint{3.018769in}{2.772314in}}%
\pgfpathlineto{\pgfqpoint{3.010758in}{2.766864in}}%
\pgfpathlineto{\pgfqpoint{3.002739in}{2.761531in}}%
\pgfpathlineto{\pgfqpoint{2.994710in}{2.756319in}}%
\pgfpathclose%
\pgfusepath{fill}%
\end{pgfscope}%
\begin{pgfscope}%
\pgfpathrectangle{\pgfqpoint{1.254980in}{0.150000in}}{\pgfqpoint{5.490039in}{5.490039in}}%
\pgfusepath{clip}%
\pgfsetbuttcap%
\pgfsetroundjoin%
\definecolor{currentfill}{rgb}{0.282327,0.094955,0.417331}%
\pgfsetfillcolor{currentfill}%
\pgfsetfillopacity{0.700000}%
\pgfsetlinewidth{0.000000pt}%
\definecolor{currentstroke}{rgb}{0.000000,0.000000,0.000000}%
\pgfsetstrokecolor{currentstroke}%
\pgfsetdash{}{0pt}%
\pgfpathmoveto{\pgfqpoint{3.957261in}{2.179312in}}%
\pgfpathlineto{\pgfqpoint{3.970377in}{2.173492in}}%
\pgfpathlineto{\pgfqpoint{3.983498in}{2.167804in}}%
\pgfpathlineto{\pgfqpoint{3.996623in}{2.162247in}}%
\pgfpathlineto{\pgfqpoint{4.009753in}{2.156822in}}%
\pgfpathlineto{\pgfqpoint{4.017362in}{2.165979in}}%
\pgfpathlineto{\pgfqpoint{4.024965in}{2.175170in}}%
\pgfpathlineto{\pgfqpoint{4.032564in}{2.184393in}}%
\pgfpathlineto{\pgfqpoint{4.040157in}{2.193649in}}%
\pgfpathlineto{\pgfqpoint{4.027038in}{2.198956in}}%
\pgfpathlineto{\pgfqpoint{4.013925in}{2.204394in}}%
\pgfpathlineto{\pgfqpoint{4.000817in}{2.209963in}}%
\pgfpathlineto{\pgfqpoint{3.987713in}{2.215665in}}%
\pgfpathlineto{\pgfqpoint{3.980108in}{2.206522in}}%
\pgfpathlineto{\pgfqpoint{3.972498in}{2.197415in}}%
\pgfpathlineto{\pgfqpoint{3.964882in}{2.188345in}}%
\pgfpathlineto{\pgfqpoint{3.957261in}{2.179312in}}%
\pgfpathclose%
\pgfusepath{fill}%
\end{pgfscope}%
\begin{pgfscope}%
\pgfpathrectangle{\pgfqpoint{1.254980in}{0.150000in}}{\pgfqpoint{5.490039in}{5.490039in}}%
\pgfusepath{clip}%
\pgfsetbuttcap%
\pgfsetroundjoin%
\definecolor{currentfill}{rgb}{0.248629,0.278775,0.534556}%
\pgfsetfillcolor{currentfill}%
\pgfsetfillopacity{0.700000}%
\pgfsetlinewidth{0.000000pt}%
\definecolor{currentstroke}{rgb}{0.000000,0.000000,0.000000}%
\pgfsetstrokecolor{currentstroke}%
\pgfsetdash{}{0pt}%
\pgfpathmoveto{\pgfqpoint{3.152440in}{2.572781in}}%
\pgfpathlineto{\pgfqpoint{3.165564in}{2.558608in}}%
\pgfpathlineto{\pgfqpoint{3.178686in}{2.544603in}}%
\pgfpathlineto{\pgfqpoint{3.191805in}{2.530765in}}%
\pgfpathlineto{\pgfqpoint{3.204922in}{2.517093in}}%
\pgfpathlineto{\pgfqpoint{3.212852in}{2.523099in}}%
\pgfpathlineto{\pgfqpoint{3.220775in}{2.529207in}}%
\pgfpathlineto{\pgfqpoint{3.228690in}{2.535418in}}%
\pgfpathlineto{\pgfqpoint{3.236597in}{2.541729in}}%
\pgfpathlineto{\pgfqpoint{3.223502in}{2.555211in}}%
\pgfpathlineto{\pgfqpoint{3.210406in}{2.568859in}}%
\pgfpathlineto{\pgfqpoint{3.197308in}{2.582673in}}%
\pgfpathlineto{\pgfqpoint{3.184207in}{2.596655in}}%
\pgfpathlineto{\pgfqpoint{3.176278in}{2.590529in}}%
\pgfpathlineto{\pgfqpoint{3.168340in}{2.584507in}}%
\pgfpathlineto{\pgfqpoint{3.160395in}{2.578590in}}%
\pgfpathlineto{\pgfqpoint{3.152440in}{2.572781in}}%
\pgfpathclose%
\pgfusepath{fill}%
\end{pgfscope}%
\begin{pgfscope}%
\pgfpathrectangle{\pgfqpoint{1.254980in}{0.150000in}}{\pgfqpoint{5.490039in}{5.490039in}}%
\pgfusepath{clip}%
\pgfsetbuttcap%
\pgfsetroundjoin%
\definecolor{currentfill}{rgb}{0.201239,0.383670,0.554294}%
\pgfsetfillcolor{currentfill}%
\pgfsetfillopacity{0.700000}%
\pgfsetlinewidth{0.000000pt}%
\definecolor{currentstroke}{rgb}{0.000000,0.000000,0.000000}%
\pgfsetstrokecolor{currentstroke}%
\pgfsetdash{}{0pt}%
\pgfpathmoveto{\pgfqpoint{2.942005in}{2.823211in}}%
\pgfpathlineto{\pgfqpoint{2.955189in}{2.806212in}}%
\pgfpathlineto{\pgfqpoint{2.968367in}{2.789398in}}%
\pgfpathlineto{\pgfqpoint{2.981541in}{2.772767in}}%
\pgfpathlineto{\pgfqpoint{2.994710in}{2.756319in}}%
\pgfpathlineto{\pgfqpoint{3.002739in}{2.761531in}}%
\pgfpathlineto{\pgfqpoint{3.010758in}{2.766864in}}%
\pgfpathlineto{\pgfqpoint{3.018769in}{2.772314in}}%
\pgfpathlineto{\pgfqpoint{3.026770in}{2.777882in}}%
\pgfpathlineto{\pgfqpoint{3.013626in}{2.794135in}}%
\pgfpathlineto{\pgfqpoint{3.000479in}{2.810571in}}%
\pgfpathlineto{\pgfqpoint{2.987326in}{2.827190in}}%
\pgfpathlineto{\pgfqpoint{2.974169in}{2.843993in}}%
\pgfpathlineto{\pgfqpoint{2.966142in}{2.838615in}}%
\pgfpathlineto{\pgfqpoint{2.958106in}{2.833357in}}%
\pgfpathlineto{\pgfqpoint{2.950060in}{2.828222in}}%
\pgfpathlineto{\pgfqpoint{2.942005in}{2.823211in}}%
\pgfpathclose%
\pgfusepath{fill}%
\end{pgfscope}%
\begin{pgfscope}%
\pgfpathrectangle{\pgfqpoint{1.254980in}{0.150000in}}{\pgfqpoint{5.490039in}{5.490039in}}%
\pgfusepath{clip}%
\pgfsetbuttcap%
\pgfsetroundjoin%
\definecolor{currentfill}{rgb}{0.281412,0.155834,0.469201}%
\pgfsetfillcolor{currentfill}%
\pgfsetfillopacity{0.700000}%
\pgfsetlinewidth{0.000000pt}%
\definecolor{currentstroke}{rgb}{0.000000,0.000000,0.000000}%
\pgfsetstrokecolor{currentstroke}%
\pgfsetdash{}{0pt}%
\pgfpathmoveto{\pgfqpoint{3.498282in}{2.305588in}}%
\pgfpathlineto{\pgfqpoint{3.511366in}{2.295394in}}%
\pgfpathlineto{\pgfqpoint{3.524451in}{2.285349in}}%
\pgfpathlineto{\pgfqpoint{3.537538in}{2.275451in}}%
\pgfpathlineto{\pgfqpoint{3.550625in}{2.265701in}}%
\pgfpathlineto{\pgfqpoint{3.558406in}{2.273172in}}%
\pgfpathlineto{\pgfqpoint{3.566180in}{2.280717in}}%
\pgfpathlineto{\pgfqpoint{3.573948in}{2.288334in}}%
\pgfpathlineto{\pgfqpoint{3.581709in}{2.296021in}}%
\pgfpathlineto{\pgfqpoint{3.568639in}{2.305603in}}%
\pgfpathlineto{\pgfqpoint{3.555570in}{2.315332in}}%
\pgfpathlineto{\pgfqpoint{3.542502in}{2.325208in}}%
\pgfpathlineto{\pgfqpoint{3.529436in}{2.335232in}}%
\pgfpathlineto{\pgfqpoint{3.521658in}{2.327707in}}%
\pgfpathlineto{\pgfqpoint{3.513872in}{2.320257in}}%
\pgfpathlineto{\pgfqpoint{3.506080in}{2.312884in}}%
\pgfpathlineto{\pgfqpoint{3.498282in}{2.305588in}}%
\pgfpathclose%
\pgfusepath{fill}%
\end{pgfscope}%
\begin{pgfscope}%
\pgfpathrectangle{\pgfqpoint{1.254980in}{0.150000in}}{\pgfqpoint{5.490039in}{5.490039in}}%
\pgfusepath{clip}%
\pgfsetbuttcap%
\pgfsetroundjoin%
\definecolor{currentfill}{rgb}{0.283187,0.125848,0.444960}%
\pgfsetfillcolor{currentfill}%
\pgfsetfillopacity{0.700000}%
\pgfsetlinewidth{0.000000pt}%
\definecolor{currentstroke}{rgb}{0.000000,0.000000,0.000000}%
\pgfsetstrokecolor{currentstroke}%
\pgfsetdash{}{0pt}%
\pgfpathmoveto{\pgfqpoint{4.393827in}{2.232983in}}%
\pgfpathlineto{\pgfqpoint{4.407045in}{2.230599in}}%
\pgfpathlineto{\pgfqpoint{4.420271in}{2.228339in}}%
\pgfpathlineto{\pgfqpoint{4.433505in}{2.226201in}}%
\pgfpathlineto{\pgfqpoint{4.446746in}{2.224185in}}%
\pgfpathlineto{\pgfqpoint{4.454215in}{2.234173in}}%
\pgfpathlineto{\pgfqpoint{4.461679in}{2.244162in}}%
\pgfpathlineto{\pgfqpoint{4.469138in}{2.254153in}}%
\pgfpathlineto{\pgfqpoint{4.476592in}{2.264145in}}%
\pgfpathlineto{\pgfqpoint{4.463360in}{2.266106in}}%
\pgfpathlineto{\pgfqpoint{4.450135in}{2.268189in}}%
\pgfpathlineto{\pgfqpoint{4.436918in}{2.270395in}}%
\pgfpathlineto{\pgfqpoint{4.423709in}{2.272724in}}%
\pgfpathlineto{\pgfqpoint{4.416246in}{2.262780in}}%
\pgfpathlineto{\pgfqpoint{4.408777in}{2.252842in}}%
\pgfpathlineto{\pgfqpoint{4.401304in}{2.242910in}}%
\pgfpathlineto{\pgfqpoint{4.393827in}{2.232983in}}%
\pgfpathclose%
\pgfusepath{fill}%
\end{pgfscope}%
\begin{pgfscope}%
\pgfpathrectangle{\pgfqpoint{1.254980in}{0.150000in}}{\pgfqpoint{5.490039in}{5.490039in}}%
\pgfusepath{clip}%
\pgfsetbuttcap%
\pgfsetroundjoin%
\definecolor{currentfill}{rgb}{0.283197,0.115680,0.436115}%
\pgfsetfillcolor{currentfill}%
\pgfsetfillopacity{0.700000}%
\pgfsetlinewidth{0.000000pt}%
\definecolor{currentstroke}{rgb}{0.000000,0.000000,0.000000}%
\pgfsetstrokecolor{currentstroke}%
\pgfsetdash{}{0pt}%
\pgfpathmoveto{\pgfqpoint{3.686332in}{2.224574in}}%
\pgfpathlineto{\pgfqpoint{3.699419in}{2.216284in}}%
\pgfpathlineto{\pgfqpoint{3.712510in}{2.208135in}}%
\pgfpathlineto{\pgfqpoint{3.725602in}{2.200126in}}%
\pgfpathlineto{\pgfqpoint{3.738698in}{2.192257in}}%
\pgfpathlineto{\pgfqpoint{3.746405in}{2.200488in}}%
\pgfpathlineto{\pgfqpoint{3.754105in}{2.208776in}}%
\pgfpathlineto{\pgfqpoint{3.761800in}{2.217120in}}%
\pgfpathlineto{\pgfqpoint{3.769489in}{2.225518in}}%
\pgfpathlineto{\pgfqpoint{3.756409in}{2.233236in}}%
\pgfpathlineto{\pgfqpoint{3.743331in}{2.241094in}}%
\pgfpathlineto{\pgfqpoint{3.730256in}{2.249091in}}%
\pgfpathlineto{\pgfqpoint{3.717184in}{2.257229in}}%
\pgfpathlineto{\pgfqpoint{3.709480in}{2.248976in}}%
\pgfpathlineto{\pgfqpoint{3.701770in}{2.240782in}}%
\pgfpathlineto{\pgfqpoint{3.694054in}{2.232647in}}%
\pgfpathlineto{\pgfqpoint{3.686332in}{2.224574in}}%
\pgfpathclose%
\pgfusepath{fill}%
\end{pgfscope}%
\begin{pgfscope}%
\pgfpathrectangle{\pgfqpoint{1.254980in}{0.150000in}}{\pgfqpoint{5.490039in}{5.490039in}}%
\pgfusepath{clip}%
\pgfsetbuttcap%
\pgfsetroundjoin%
\definecolor{currentfill}{rgb}{0.154815,0.493313,0.557840}%
\pgfsetfillcolor{currentfill}%
\pgfsetfillopacity{0.700000}%
\pgfsetlinewidth{0.000000pt}%
\definecolor{currentstroke}{rgb}{0.000000,0.000000,0.000000}%
\pgfsetstrokecolor{currentstroke}%
\pgfsetdash{}{0pt}%
\pgfpathmoveto{\pgfqpoint{5.967757in}{3.060034in}}%
\pgfpathlineto{\pgfqpoint{5.981620in}{3.064671in}}%
\pgfpathlineto{\pgfqpoint{5.995498in}{3.069417in}}%
\pgfpathlineto{\pgfqpoint{6.009391in}{3.074272in}}%
\pgfpathlineto{\pgfqpoint{6.023299in}{3.079237in}}%
\pgfpathlineto{\pgfqpoint{6.030175in}{3.086667in}}%
\pgfpathlineto{\pgfqpoint{6.037047in}{3.094111in}}%
\pgfpathlineto{\pgfqpoint{6.043914in}{3.101569in}}%
\pgfpathlineto{\pgfqpoint{6.030021in}{3.096794in}}%
\pgfpathlineto{\pgfqpoint{6.016142in}{3.092129in}}%
\pgfpathlineto{\pgfqpoint{6.002279in}{3.087572in}}%
\pgfpathlineto{\pgfqpoint{5.988429in}{3.083125in}}%
\pgfpathlineto{\pgfqpoint{5.981542in}{3.075410in}}%
\pgfpathlineto{\pgfqpoint{5.974652in}{3.067714in}}%
\pgfpathlineto{\pgfqpoint{5.967757in}{3.060034in}}%
\pgfpathclose%
\pgfusepath{fill}%
\end{pgfscope}%
\begin{pgfscope}%
\pgfpathrectangle{\pgfqpoint{1.254980in}{0.150000in}}{\pgfqpoint{5.490039in}{5.490039in}}%
\pgfusepath{clip}%
\pgfsetbuttcap%
\pgfsetroundjoin%
\definecolor{currentfill}{rgb}{0.257322,0.256130,0.526563}%
\pgfsetfillcolor{currentfill}%
\pgfsetfillopacity{0.700000}%
\pgfsetlinewidth{0.000000pt}%
\definecolor{currentstroke}{rgb}{0.000000,0.000000,0.000000}%
\pgfsetstrokecolor{currentstroke}%
\pgfsetdash{}{0pt}%
\pgfpathmoveto{\pgfqpoint{3.204922in}{2.517093in}}%
\pgfpathlineto{\pgfqpoint{3.218036in}{2.503586in}}%
\pgfpathlineto{\pgfqpoint{3.231149in}{2.490244in}}%
\pgfpathlineto{\pgfqpoint{3.244261in}{2.477064in}}%
\pgfpathlineto{\pgfqpoint{3.257370in}{2.464048in}}%
\pgfpathlineto{\pgfqpoint{3.265279in}{2.470249in}}%
\pgfpathlineto{\pgfqpoint{3.273180in}{2.476549in}}%
\pgfpathlineto{\pgfqpoint{3.281072in}{2.482947in}}%
\pgfpathlineto{\pgfqpoint{3.288957in}{2.489440in}}%
\pgfpathlineto{\pgfqpoint{3.275870in}{2.502268in}}%
\pgfpathlineto{\pgfqpoint{3.262780in}{2.515258in}}%
\pgfpathlineto{\pgfqpoint{3.249689in}{2.528411in}}%
\pgfpathlineto{\pgfqpoint{3.236597in}{2.541729in}}%
\pgfpathlineto{\pgfqpoint{3.228690in}{2.535418in}}%
\pgfpathlineto{\pgfqpoint{3.220775in}{2.529207in}}%
\pgfpathlineto{\pgfqpoint{3.212852in}{2.523099in}}%
\pgfpathlineto{\pgfqpoint{3.204922in}{2.517093in}}%
\pgfpathclose%
\pgfusepath{fill}%
\end{pgfscope}%
\begin{pgfscope}%
\pgfpathrectangle{\pgfqpoint{1.254980in}{0.150000in}}{\pgfqpoint{5.490039in}{5.490039in}}%
\pgfusepath{clip}%
\pgfsetbuttcap%
\pgfsetroundjoin%
\definecolor{currentfill}{rgb}{0.188923,0.410910,0.556326}%
\pgfsetfillcolor{currentfill}%
\pgfsetfillopacity{0.700000}%
\pgfsetlinewidth{0.000000pt}%
\definecolor{currentstroke}{rgb}{0.000000,0.000000,0.000000}%
\pgfsetstrokecolor{currentstroke}%
\pgfsetdash{}{0pt}%
\pgfpathmoveto{\pgfqpoint{2.889219in}{2.893080in}}%
\pgfpathlineto{\pgfqpoint{2.902424in}{2.875330in}}%
\pgfpathlineto{\pgfqpoint{2.915623in}{2.857769in}}%
\pgfpathlineto{\pgfqpoint{2.928817in}{2.840397in}}%
\pgfpathlineto{\pgfqpoint{2.942005in}{2.823211in}}%
\pgfpathlineto{\pgfqpoint{2.950060in}{2.828222in}}%
\pgfpathlineto{\pgfqpoint{2.958106in}{2.833357in}}%
\pgfpathlineto{\pgfqpoint{2.966142in}{2.838615in}}%
\pgfpathlineto{\pgfqpoint{2.974169in}{2.843993in}}%
\pgfpathlineto{\pgfqpoint{2.961007in}{2.860982in}}%
\pgfpathlineto{\pgfqpoint{2.947840in}{2.878158in}}%
\pgfpathlineto{\pgfqpoint{2.934668in}{2.895522in}}%
\pgfpathlineto{\pgfqpoint{2.921491in}{2.913076in}}%
\pgfpathlineto{\pgfqpoint{2.913437in}{2.907887in}}%
\pgfpathlineto{\pgfqpoint{2.905375in}{2.902824in}}%
\pgfpathlineto{\pgfqpoint{2.897302in}{2.897888in}}%
\pgfpathlineto{\pgfqpoint{2.889219in}{2.893080in}}%
\pgfpathclose%
\pgfusepath{fill}%
\end{pgfscope}%
\begin{pgfscope}%
\pgfpathrectangle{\pgfqpoint{1.254980in}{0.150000in}}{\pgfqpoint{5.490039in}{5.490039in}}%
\pgfusepath{clip}%
\pgfsetbuttcap%
\pgfsetroundjoin%
\definecolor{currentfill}{rgb}{0.282327,0.094955,0.417331}%
\pgfsetfillcolor{currentfill}%
\pgfsetfillopacity{0.700000}%
\pgfsetlinewidth{0.000000pt}%
\definecolor{currentstroke}{rgb}{0.000000,0.000000,0.000000}%
\pgfsetstrokecolor{currentstroke}%
\pgfsetdash{}{0pt}%
\pgfpathmoveto{\pgfqpoint{4.092681in}{2.173725in}}%
\pgfpathlineto{\pgfqpoint{4.105826in}{2.169068in}}%
\pgfpathlineto{\pgfqpoint{4.118976in}{2.164540in}}%
\pgfpathlineto{\pgfqpoint{4.132132in}{2.160140in}}%
\pgfpathlineto{\pgfqpoint{4.145293in}{2.155867in}}%
\pgfpathlineto{\pgfqpoint{4.152859in}{2.165371in}}%
\pgfpathlineto{\pgfqpoint{4.160419in}{2.174897in}}%
\pgfpathlineto{\pgfqpoint{4.167975in}{2.184446in}}%
\pgfpathlineto{\pgfqpoint{4.175525in}{2.194016in}}%
\pgfpathlineto{\pgfqpoint{4.162375in}{2.198186in}}%
\pgfpathlineto{\pgfqpoint{4.149230in}{2.202484in}}%
\pgfpathlineto{\pgfqpoint{4.136090in}{2.206910in}}%
\pgfpathlineto{\pgfqpoint{4.122957in}{2.211464in}}%
\pgfpathlineto{\pgfqpoint{4.115396in}{2.201991in}}%
\pgfpathlineto{\pgfqpoint{4.107829in}{2.192543in}}%
\pgfpathlineto{\pgfqpoint{4.100258in}{2.183121in}}%
\pgfpathlineto{\pgfqpoint{4.092681in}{2.173725in}}%
\pgfpathclose%
\pgfusepath{fill}%
\end{pgfscope}%
\begin{pgfscope}%
\pgfpathrectangle{\pgfqpoint{1.254980in}{0.150000in}}{\pgfqpoint{5.490039in}{5.490039in}}%
\pgfusepath{clip}%
\pgfsetbuttcap%
\pgfsetroundjoin%
\definecolor{currentfill}{rgb}{0.265145,0.232956,0.516599}%
\pgfsetfillcolor{currentfill}%
\pgfsetfillopacity{0.700000}%
\pgfsetlinewidth{0.000000pt}%
\definecolor{currentstroke}{rgb}{0.000000,0.000000,0.000000}%
\pgfsetstrokecolor{currentstroke}%
\pgfsetdash{}{0pt}%
\pgfpathmoveto{\pgfqpoint{3.257370in}{2.464048in}}%
\pgfpathlineto{\pgfqpoint{3.270479in}{2.451193in}}%
\pgfpathlineto{\pgfqpoint{3.283586in}{2.438498in}}%
\pgfpathlineto{\pgfqpoint{3.296691in}{2.425964in}}%
\pgfpathlineto{\pgfqpoint{3.309796in}{2.413589in}}%
\pgfpathlineto{\pgfqpoint{3.317683in}{2.419984in}}%
\pgfpathlineto{\pgfqpoint{3.325562in}{2.426475in}}%
\pgfpathlineto{\pgfqpoint{3.333434in}{2.433059in}}%
\pgfpathlineto{\pgfqpoint{3.341298in}{2.439735in}}%
\pgfpathlineto{\pgfqpoint{3.328214in}{2.451922in}}%
\pgfpathlineto{\pgfqpoint{3.315130in}{2.464268in}}%
\pgfpathlineto{\pgfqpoint{3.302044in}{2.476774in}}%
\pgfpathlineto{\pgfqpoint{3.288957in}{2.489440in}}%
\pgfpathlineto{\pgfqpoint{3.281072in}{2.482947in}}%
\pgfpathlineto{\pgfqpoint{3.273180in}{2.476549in}}%
\pgfpathlineto{\pgfqpoint{3.265279in}{2.470249in}}%
\pgfpathlineto{\pgfqpoint{3.257370in}{2.464048in}}%
\pgfpathclose%
\pgfusepath{fill}%
\end{pgfscope}%
\begin{pgfscope}%
\pgfpathrectangle{\pgfqpoint{1.254980in}{0.150000in}}{\pgfqpoint{5.490039in}{5.490039in}}%
\pgfusepath{clip}%
\pgfsetbuttcap%
\pgfsetroundjoin%
\definecolor{currentfill}{rgb}{0.283197,0.115680,0.436115}%
\pgfsetfillcolor{currentfill}%
\pgfsetfillopacity{0.700000}%
\pgfsetlinewidth{0.000000pt}%
\definecolor{currentstroke}{rgb}{0.000000,0.000000,0.000000}%
\pgfsetstrokecolor{currentstroke}%
\pgfsetdash{}{0pt}%
\pgfpathmoveto{\pgfqpoint{4.311030in}{2.204393in}}%
\pgfpathlineto{\pgfqpoint{4.324229in}{2.201444in}}%
\pgfpathlineto{\pgfqpoint{4.337435in}{2.198619in}}%
\pgfpathlineto{\pgfqpoint{4.350648in}{2.195918in}}%
\pgfpathlineto{\pgfqpoint{4.363868in}{2.193341in}}%
\pgfpathlineto{\pgfqpoint{4.371365in}{2.203241in}}%
\pgfpathlineto{\pgfqpoint{4.378857in}{2.213148in}}%
\pgfpathlineto{\pgfqpoint{4.386344in}{2.223063in}}%
\pgfpathlineto{\pgfqpoint{4.393827in}{2.232983in}}%
\pgfpathlineto{\pgfqpoint{4.380616in}{2.235490in}}%
\pgfpathlineto{\pgfqpoint{4.367412in}{2.238120in}}%
\pgfpathlineto{\pgfqpoint{4.354215in}{2.240875in}}%
\pgfpathlineto{\pgfqpoint{4.341026in}{2.243754in}}%
\pgfpathlineto{\pgfqpoint{4.333534in}{2.233898in}}%
\pgfpathlineto{\pgfqpoint{4.326037in}{2.224052in}}%
\pgfpathlineto{\pgfqpoint{4.318536in}{2.214217in}}%
\pgfpathlineto{\pgfqpoint{4.311030in}{2.204393in}}%
\pgfpathclose%
\pgfusepath{fill}%
\end{pgfscope}%
\begin{pgfscope}%
\pgfpathrectangle{\pgfqpoint{1.254980in}{0.150000in}}{\pgfqpoint{5.490039in}{5.490039in}}%
\pgfusepath{clip}%
\pgfsetbuttcap%
\pgfsetroundjoin%
\definecolor{currentfill}{rgb}{0.175841,0.441290,0.557685}%
\pgfsetfillcolor{currentfill}%
\pgfsetfillopacity{0.700000}%
\pgfsetlinewidth{0.000000pt}%
\definecolor{currentstroke}{rgb}{0.000000,0.000000,0.000000}%
\pgfsetstrokecolor{currentstroke}%
\pgfsetdash{}{0pt}%
\pgfpathmoveto{\pgfqpoint{2.836342in}{2.966003in}}%
\pgfpathlineto{\pgfqpoint{2.849570in}{2.947482in}}%
\pgfpathlineto{\pgfqpoint{2.862793in}{2.929155in}}%
\pgfpathlineto{\pgfqpoint{2.876009in}{2.911021in}}%
\pgfpathlineto{\pgfqpoint{2.889219in}{2.893080in}}%
\pgfpathlineto{\pgfqpoint{2.897302in}{2.897888in}}%
\pgfpathlineto{\pgfqpoint{2.905375in}{2.902824in}}%
\pgfpathlineto{\pgfqpoint{2.913437in}{2.907887in}}%
\pgfpathlineto{\pgfqpoint{2.921491in}{2.913076in}}%
\pgfpathlineto{\pgfqpoint{2.908308in}{2.930819in}}%
\pgfpathlineto{\pgfqpoint{2.895119in}{2.948755in}}%
\pgfpathlineto{\pgfqpoint{2.881925in}{2.966883in}}%
\pgfpathlineto{\pgfqpoint{2.868724in}{2.985206in}}%
\pgfpathlineto{\pgfqpoint{2.860644in}{2.980209in}}%
\pgfpathlineto{\pgfqpoint{2.852553in}{2.975342in}}%
\pgfpathlineto{\pgfqpoint{2.844452in}{2.970606in}}%
\pgfpathlineto{\pgfqpoint{2.836342in}{2.966003in}}%
\pgfpathclose%
\pgfusepath{fill}%
\end{pgfscope}%
\begin{pgfscope}%
\pgfpathrectangle{\pgfqpoint{1.254980in}{0.150000in}}{\pgfqpoint{5.490039in}{5.490039in}}%
\pgfusepath{clip}%
\pgfsetbuttcap%
\pgfsetroundjoin%
\definecolor{currentfill}{rgb}{0.282623,0.140926,0.457517}%
\pgfsetfillcolor{currentfill}%
\pgfsetfillopacity{0.700000}%
\pgfsetlinewidth{0.000000pt}%
\definecolor{currentstroke}{rgb}{0.000000,0.000000,0.000000}%
\pgfsetstrokecolor{currentstroke}%
\pgfsetdash{}{0pt}%
\pgfpathmoveto{\pgfqpoint{3.550625in}{2.265701in}}%
\pgfpathlineto{\pgfqpoint{3.563714in}{2.256097in}}%
\pgfpathlineto{\pgfqpoint{3.576804in}{2.246638in}}%
\pgfpathlineto{\pgfqpoint{3.589896in}{2.237324in}}%
\pgfpathlineto{\pgfqpoint{3.602989in}{2.228155in}}%
\pgfpathlineto{\pgfqpoint{3.610753in}{2.235800in}}%
\pgfpathlineto{\pgfqpoint{3.618510in}{2.243516in}}%
\pgfpathlineto{\pgfqpoint{3.626260in}{2.251299in}}%
\pgfpathlineto{\pgfqpoint{3.634005in}{2.259149in}}%
\pgfpathlineto{\pgfqpoint{3.620928in}{2.268150in}}%
\pgfpathlineto{\pgfqpoint{3.607853in}{2.277296in}}%
\pgfpathlineto{\pgfqpoint{3.594780in}{2.286586in}}%
\pgfpathlineto{\pgfqpoint{3.581709in}{2.296021in}}%
\pgfpathlineto{\pgfqpoint{3.573948in}{2.288334in}}%
\pgfpathlineto{\pgfqpoint{3.566180in}{2.280717in}}%
\pgfpathlineto{\pgfqpoint{3.558406in}{2.273172in}}%
\pgfpathlineto{\pgfqpoint{3.550625in}{2.265701in}}%
\pgfpathclose%
\pgfusepath{fill}%
\end{pgfscope}%
\begin{pgfscope}%
\pgfpathrectangle{\pgfqpoint{1.254980in}{0.150000in}}{\pgfqpoint{5.490039in}{5.490039in}}%
\pgfusepath{clip}%
\pgfsetbuttcap%
\pgfsetroundjoin%
\definecolor{currentfill}{rgb}{0.260571,0.246922,0.522828}%
\pgfsetfillcolor{currentfill}%
\pgfsetfillopacity{0.700000}%
\pgfsetlinewidth{0.000000pt}%
\definecolor{currentstroke}{rgb}{0.000000,0.000000,0.000000}%
\pgfsetstrokecolor{currentstroke}%
\pgfsetdash{}{0pt}%
\pgfpathmoveto{\pgfqpoint{4.943853in}{2.454195in}}%
\pgfpathlineto{\pgfqpoint{4.957273in}{2.455244in}}%
\pgfpathlineto{\pgfqpoint{4.970704in}{2.456409in}}%
\pgfpathlineto{\pgfqpoint{4.984145in}{2.457689in}}%
\pgfpathlineto{\pgfqpoint{4.997597in}{2.459086in}}%
\pgfpathlineto{\pgfqpoint{5.004890in}{2.468894in}}%
\pgfpathlineto{\pgfqpoint{5.012179in}{2.478682in}}%
\pgfpathlineto{\pgfqpoint{5.019463in}{2.488450in}}%
\pgfpathlineto{\pgfqpoint{5.026741in}{2.498198in}}%
\pgfpathlineto{\pgfqpoint{5.013298in}{2.496843in}}%
\pgfpathlineto{\pgfqpoint{4.999866in}{2.495604in}}%
\pgfpathlineto{\pgfqpoint{4.986444in}{2.494480in}}%
\pgfpathlineto{\pgfqpoint{4.973033in}{2.493473in}}%
\pgfpathlineto{\pgfqpoint{4.965745in}{2.483677in}}%
\pgfpathlineto{\pgfqpoint{4.958453in}{2.473866in}}%
\pgfpathlineto{\pgfqpoint{4.951155in}{2.464039in}}%
\pgfpathlineto{\pgfqpoint{4.943853in}{2.454195in}}%
\pgfpathclose%
\pgfusepath{fill}%
\end{pgfscope}%
\begin{pgfscope}%
\pgfpathrectangle{\pgfqpoint{1.254980in}{0.150000in}}{\pgfqpoint{5.490039in}{5.490039in}}%
\pgfusepath{clip}%
\pgfsetbuttcap%
\pgfsetroundjoin%
\definecolor{currentfill}{rgb}{0.253935,0.265254,0.529983}%
\pgfsetfillcolor{currentfill}%
\pgfsetfillopacity{0.700000}%
\pgfsetlinewidth{0.000000pt}%
\definecolor{currentstroke}{rgb}{0.000000,0.000000,0.000000}%
\pgfsetstrokecolor{currentstroke}%
\pgfsetdash{}{0pt}%
\pgfpathmoveto{\pgfqpoint{5.026741in}{2.498198in}}%
\pgfpathlineto{\pgfqpoint{5.040195in}{2.499669in}}%
\pgfpathlineto{\pgfqpoint{5.053660in}{2.501255in}}%
\pgfpathlineto{\pgfqpoint{5.067136in}{2.502957in}}%
\pgfpathlineto{\pgfqpoint{5.080624in}{2.504773in}}%
\pgfpathlineto{\pgfqpoint{5.087888in}{2.514451in}}%
\pgfpathlineto{\pgfqpoint{5.095148in}{2.524108in}}%
\pgfpathlineto{\pgfqpoint{5.102402in}{2.533743in}}%
\pgfpathlineto{\pgfqpoint{5.109651in}{2.543357in}}%
\pgfpathlineto{\pgfqpoint{5.096173in}{2.541598in}}%
\pgfpathlineto{\pgfqpoint{5.082707in}{2.539954in}}%
\pgfpathlineto{\pgfqpoint{5.069251in}{2.538425in}}%
\pgfpathlineto{\pgfqpoint{5.055806in}{2.537012in}}%
\pgfpathlineto{\pgfqpoint{5.048547in}{2.527334in}}%
\pgfpathlineto{\pgfqpoint{5.041284in}{2.517639in}}%
\pgfpathlineto{\pgfqpoint{5.034015in}{2.507928in}}%
\pgfpathlineto{\pgfqpoint{5.026741in}{2.498198in}}%
\pgfpathclose%
\pgfusepath{fill}%
\end{pgfscope}%
\begin{pgfscope}%
\pgfpathrectangle{\pgfqpoint{1.254980in}{0.150000in}}{\pgfqpoint{5.490039in}{5.490039in}}%
\pgfusepath{clip}%
\pgfsetbuttcap%
\pgfsetroundjoin%
\definecolor{currentfill}{rgb}{0.267968,0.223549,0.512008}%
\pgfsetfillcolor{currentfill}%
\pgfsetfillopacity{0.700000}%
\pgfsetlinewidth{0.000000pt}%
\definecolor{currentstroke}{rgb}{0.000000,0.000000,0.000000}%
\pgfsetstrokecolor{currentstroke}%
\pgfsetdash{}{0pt}%
\pgfpathmoveto{\pgfqpoint{4.860984in}{2.411510in}}%
\pgfpathlineto{\pgfqpoint{4.874371in}{2.412117in}}%
\pgfpathlineto{\pgfqpoint{4.887769in}{2.412841in}}%
\pgfpathlineto{\pgfqpoint{4.901176in}{2.413683in}}%
\pgfpathlineto{\pgfqpoint{4.914595in}{2.414640in}}%
\pgfpathlineto{\pgfqpoint{4.921917in}{2.424557in}}%
\pgfpathlineto{\pgfqpoint{4.929234in}{2.434455in}}%
\pgfpathlineto{\pgfqpoint{4.936546in}{2.444334in}}%
\pgfpathlineto{\pgfqpoint{4.943853in}{2.454195in}}%
\pgfpathlineto{\pgfqpoint{4.930444in}{2.453263in}}%
\pgfpathlineto{\pgfqpoint{4.917045in}{2.452447in}}%
\pgfpathlineto{\pgfqpoint{4.903656in}{2.451748in}}%
\pgfpathlineto{\pgfqpoint{4.890277in}{2.451166in}}%
\pgfpathlineto{\pgfqpoint{4.882961in}{2.441273in}}%
\pgfpathlineto{\pgfqpoint{4.875640in}{2.431367in}}%
\pgfpathlineto{\pgfqpoint{4.868315in}{2.421446in}}%
\pgfpathlineto{\pgfqpoint{4.860984in}{2.411510in}}%
\pgfpathclose%
\pgfusepath{fill}%
\end{pgfscope}%
\begin{pgfscope}%
\pgfpathrectangle{\pgfqpoint{1.254980in}{0.150000in}}{\pgfqpoint{5.490039in}{5.490039in}}%
\pgfusepath{clip}%
\pgfsetbuttcap%
\pgfsetroundjoin%
\definecolor{currentfill}{rgb}{0.244972,0.287675,0.537260}%
\pgfsetfillcolor{currentfill}%
\pgfsetfillopacity{0.700000}%
\pgfsetlinewidth{0.000000pt}%
\definecolor{currentstroke}{rgb}{0.000000,0.000000,0.000000}%
\pgfsetstrokecolor{currentstroke}%
\pgfsetdash{}{0pt}%
\pgfpathmoveto{\pgfqpoint{5.109651in}{2.543357in}}%
\pgfpathlineto{\pgfqpoint{5.123141in}{2.545231in}}%
\pgfpathlineto{\pgfqpoint{5.136641in}{2.547220in}}%
\pgfpathlineto{\pgfqpoint{5.150154in}{2.549323in}}%
\pgfpathlineto{\pgfqpoint{5.163677in}{2.551541in}}%
\pgfpathlineto{\pgfqpoint{5.170912in}{2.561068in}}%
\pgfpathlineto{\pgfqpoint{5.178142in}{2.570573in}}%
\pgfpathlineto{\pgfqpoint{5.185366in}{2.580056in}}%
\pgfpathlineto{\pgfqpoint{5.192586in}{2.589519in}}%
\pgfpathlineto{\pgfqpoint{5.179072in}{2.587375in}}%
\pgfpathlineto{\pgfqpoint{5.165569in}{2.585346in}}%
\pgfpathlineto{\pgfqpoint{5.152079in}{2.583431in}}%
\pgfpathlineto{\pgfqpoint{5.138599in}{2.581631in}}%
\pgfpathlineto{\pgfqpoint{5.131370in}{2.572088in}}%
\pgfpathlineto{\pgfqpoint{5.124135in}{2.562529in}}%
\pgfpathlineto{\pgfqpoint{5.116896in}{2.552952in}}%
\pgfpathlineto{\pgfqpoint{5.109651in}{2.543357in}}%
\pgfpathclose%
\pgfusepath{fill}%
\end{pgfscope}%
\begin{pgfscope}%
\pgfpathrectangle{\pgfqpoint{1.254980in}{0.150000in}}{\pgfqpoint{5.490039in}{5.490039in}}%
\pgfusepath{clip}%
\pgfsetbuttcap%
\pgfsetroundjoin%
\definecolor{currentfill}{rgb}{0.271828,0.209303,0.504434}%
\pgfsetfillcolor{currentfill}%
\pgfsetfillopacity{0.700000}%
\pgfsetlinewidth{0.000000pt}%
\definecolor{currentstroke}{rgb}{0.000000,0.000000,0.000000}%
\pgfsetstrokecolor{currentstroke}%
\pgfsetdash{}{0pt}%
\pgfpathmoveto{\pgfqpoint{3.309796in}{2.413589in}}%
\pgfpathlineto{\pgfqpoint{3.322900in}{2.401371in}}%
\pgfpathlineto{\pgfqpoint{3.336003in}{2.389312in}}%
\pgfpathlineto{\pgfqpoint{3.349105in}{2.377409in}}%
\pgfpathlineto{\pgfqpoint{3.362207in}{2.365661in}}%
\pgfpathlineto{\pgfqpoint{3.370073in}{2.372251in}}%
\pgfpathlineto{\pgfqpoint{3.377931in}{2.378931in}}%
\pgfpathlineto{\pgfqpoint{3.385782in}{2.385700in}}%
\pgfpathlineto{\pgfqpoint{3.393626in}{2.392557in}}%
\pgfpathlineto{\pgfqpoint{3.380545in}{2.404117in}}%
\pgfpathlineto{\pgfqpoint{3.367463in}{2.415833in}}%
\pgfpathlineto{\pgfqpoint{3.354381in}{2.427705in}}%
\pgfpathlineto{\pgfqpoint{3.341298in}{2.439735in}}%
\pgfpathlineto{\pgfqpoint{3.333434in}{2.433059in}}%
\pgfpathlineto{\pgfqpoint{3.325562in}{2.426475in}}%
\pgfpathlineto{\pgfqpoint{3.317683in}{2.419984in}}%
\pgfpathlineto{\pgfqpoint{3.309796in}{2.413589in}}%
\pgfpathclose%
\pgfusepath{fill}%
\end{pgfscope}%
\begin{pgfscope}%
\pgfpathrectangle{\pgfqpoint{1.254980in}{0.150000in}}{\pgfqpoint{5.490039in}{5.490039in}}%
\pgfusepath{clip}%
\pgfsetbuttcap%
\pgfsetroundjoin%
\definecolor{currentfill}{rgb}{0.273006,0.204520,0.501721}%
\pgfsetfillcolor{currentfill}%
\pgfsetfillopacity{0.700000}%
\pgfsetlinewidth{0.000000pt}%
\definecolor{currentstroke}{rgb}{0.000000,0.000000,0.000000}%
\pgfsetstrokecolor{currentstroke}%
\pgfsetdash{}{0pt}%
\pgfpathmoveto{\pgfqpoint{4.778130in}{2.370314in}}%
\pgfpathlineto{\pgfqpoint{4.791486in}{2.370461in}}%
\pgfpathlineto{\pgfqpoint{4.804852in}{2.370725in}}%
\pgfpathlineto{\pgfqpoint{4.818228in}{2.371107in}}%
\pgfpathlineto{\pgfqpoint{4.831613in}{2.371607in}}%
\pgfpathlineto{\pgfqpoint{4.838963in}{2.381607in}}%
\pgfpathlineto{\pgfqpoint{4.846308in}{2.391591in}}%
\pgfpathlineto{\pgfqpoint{4.853649in}{2.401558in}}%
\pgfpathlineto{\pgfqpoint{4.860984in}{2.411510in}}%
\pgfpathlineto{\pgfqpoint{4.847607in}{2.411020in}}%
\pgfpathlineto{\pgfqpoint{4.834240in}{2.410647in}}%
\pgfpathlineto{\pgfqpoint{4.820882in}{2.410392in}}%
\pgfpathlineto{\pgfqpoint{4.807535in}{2.410254in}}%
\pgfpathlineto{\pgfqpoint{4.800191in}{2.400287in}}%
\pgfpathlineto{\pgfqpoint{4.792842in}{2.390309in}}%
\pgfpathlineto{\pgfqpoint{4.785488in}{2.380318in}}%
\pgfpathlineto{\pgfqpoint{4.778130in}{2.370314in}}%
\pgfpathclose%
\pgfusepath{fill}%
\end{pgfscope}%
\begin{pgfscope}%
\pgfpathrectangle{\pgfqpoint{1.254980in}{0.150000in}}{\pgfqpoint{5.490039in}{5.490039in}}%
\pgfusepath{clip}%
\pgfsetbuttcap%
\pgfsetroundjoin%
\definecolor{currentfill}{rgb}{0.235526,0.309527,0.542944}%
\pgfsetfillcolor{currentfill}%
\pgfsetfillopacity{0.700000}%
\pgfsetlinewidth{0.000000pt}%
\definecolor{currentstroke}{rgb}{0.000000,0.000000,0.000000}%
\pgfsetstrokecolor{currentstroke}%
\pgfsetdash{}{0pt}%
\pgfpathmoveto{\pgfqpoint{5.192586in}{2.589519in}}%
\pgfpathlineto{\pgfqpoint{5.206112in}{2.591777in}}%
\pgfpathlineto{\pgfqpoint{5.219649in}{2.594149in}}%
\pgfpathlineto{\pgfqpoint{5.233198in}{2.596635in}}%
\pgfpathlineto{\pgfqpoint{5.246760in}{2.599235in}}%
\pgfpathlineto{\pgfqpoint{5.253964in}{2.608594in}}%
\pgfpathlineto{\pgfqpoint{5.261163in}{2.617931in}}%
\pgfpathlineto{\pgfqpoint{5.268357in}{2.627246in}}%
\pgfpathlineto{\pgfqpoint{5.275546in}{2.636540in}}%
\pgfpathlineto{\pgfqpoint{5.261995in}{2.634031in}}%
\pgfpathlineto{\pgfqpoint{5.248456in}{2.631635in}}%
\pgfpathlineto{\pgfqpoint{5.234929in}{2.629353in}}%
\pgfpathlineto{\pgfqpoint{5.221414in}{2.627184in}}%
\pgfpathlineto{\pgfqpoint{5.214214in}{2.617794in}}%
\pgfpathlineto{\pgfqpoint{5.207010in}{2.608387in}}%
\pgfpathlineto{\pgfqpoint{5.199800in}{2.598962in}}%
\pgfpathlineto{\pgfqpoint{5.192586in}{2.589519in}}%
\pgfpathclose%
\pgfusepath{fill}%
\end{pgfscope}%
\begin{pgfscope}%
\pgfpathrectangle{\pgfqpoint{1.254980in}{0.150000in}}{\pgfqpoint{5.490039in}{5.490039in}}%
\pgfusepath{clip}%
\pgfsetbuttcap%
\pgfsetroundjoin%
\definecolor{currentfill}{rgb}{0.282327,0.094955,0.417331}%
\pgfsetfillcolor{currentfill}%
\pgfsetfillopacity{0.700000}%
\pgfsetlinewidth{0.000000pt}%
\definecolor{currentstroke}{rgb}{0.000000,0.000000,0.000000}%
\pgfsetstrokecolor{currentstroke}%
\pgfsetdash{}{0pt}%
\pgfpathmoveto{\pgfqpoint{3.874253in}{2.168723in}}%
\pgfpathlineto{\pgfqpoint{3.887364in}{2.162235in}}%
\pgfpathlineto{\pgfqpoint{3.900480in}{2.155881in}}%
\pgfpathlineto{\pgfqpoint{3.913600in}{2.149660in}}%
\pgfpathlineto{\pgfqpoint{3.926725in}{2.143573in}}%
\pgfpathlineto{\pgfqpoint{3.934367in}{2.152447in}}%
\pgfpathlineto{\pgfqpoint{3.942004in}{2.161363in}}%
\pgfpathlineto{\pgfqpoint{3.949635in}{2.170318in}}%
\pgfpathlineto{\pgfqpoint{3.957261in}{2.179312in}}%
\pgfpathlineto{\pgfqpoint{3.944150in}{2.185265in}}%
\pgfpathlineto{\pgfqpoint{3.931043in}{2.191352in}}%
\pgfpathlineto{\pgfqpoint{3.917940in}{2.197571in}}%
\pgfpathlineto{\pgfqpoint{3.904841in}{2.203925in}}%
\pgfpathlineto{\pgfqpoint{3.897202in}{2.195059in}}%
\pgfpathlineto{\pgfqpoint{3.889558in}{2.186236in}}%
\pgfpathlineto{\pgfqpoint{3.881908in}{2.177457in}}%
\pgfpathlineto{\pgfqpoint{3.874253in}{2.168723in}}%
\pgfpathclose%
\pgfusepath{fill}%
\end{pgfscope}%
\begin{pgfscope}%
\pgfpathrectangle{\pgfqpoint{1.254980in}{0.150000in}}{\pgfqpoint{5.490039in}{5.490039in}}%
\pgfusepath{clip}%
\pgfsetbuttcap%
\pgfsetroundjoin%
\definecolor{currentfill}{rgb}{0.163625,0.471133,0.558148}%
\pgfsetfillcolor{currentfill}%
\pgfsetfillopacity{0.700000}%
\pgfsetlinewidth{0.000000pt}%
\definecolor{currentstroke}{rgb}{0.000000,0.000000,0.000000}%
\pgfsetstrokecolor{currentstroke}%
\pgfsetdash{}{0pt}%
\pgfpathmoveto{\pgfqpoint{2.783361in}{3.042062in}}%
\pgfpathlineto{\pgfqpoint{2.796616in}{3.022749in}}%
\pgfpathlineto{\pgfqpoint{2.809865in}{3.003635in}}%
\pgfpathlineto{\pgfqpoint{2.823107in}{2.984720in}}%
\pgfpathlineto{\pgfqpoint{2.836342in}{2.966003in}}%
\pgfpathlineto{\pgfqpoint{2.844452in}{2.970606in}}%
\pgfpathlineto{\pgfqpoint{2.852553in}{2.975342in}}%
\pgfpathlineto{\pgfqpoint{2.860644in}{2.980209in}}%
\pgfpathlineto{\pgfqpoint{2.868724in}{2.985206in}}%
\pgfpathlineto{\pgfqpoint{2.855517in}{3.003724in}}%
\pgfpathlineto{\pgfqpoint{2.842304in}{3.022439in}}%
\pgfpathlineto{\pgfqpoint{2.829085in}{3.041353in}}%
\pgfpathlineto{\pgfqpoint{2.815858in}{3.060466in}}%
\pgfpathlineto{\pgfqpoint{2.807749in}{3.055662in}}%
\pgfpathlineto{\pgfqpoint{2.799630in}{3.050993in}}%
\pgfpathlineto{\pgfqpoint{2.791501in}{3.046459in}}%
\pgfpathlineto{\pgfqpoint{2.783361in}{3.042062in}}%
\pgfpathclose%
\pgfusepath{fill}%
\end{pgfscope}%
\begin{pgfscope}%
\pgfpathrectangle{\pgfqpoint{1.254980in}{0.150000in}}{\pgfqpoint{5.490039in}{5.490039in}}%
\pgfusepath{clip}%
\pgfsetbuttcap%
\pgfsetroundjoin%
\definecolor{currentfill}{rgb}{0.225863,0.330805,0.547314}%
\pgfsetfillcolor{currentfill}%
\pgfsetfillopacity{0.700000}%
\pgfsetlinewidth{0.000000pt}%
\definecolor{currentstroke}{rgb}{0.000000,0.000000,0.000000}%
\pgfsetstrokecolor{currentstroke}%
\pgfsetdash{}{0pt}%
\pgfpathmoveto{\pgfqpoint{5.275546in}{2.636540in}}%
\pgfpathlineto{\pgfqpoint{5.289109in}{2.639164in}}%
\pgfpathlineto{\pgfqpoint{5.302684in}{2.641901in}}%
\pgfpathlineto{\pgfqpoint{5.316272in}{2.644751in}}%
\pgfpathlineto{\pgfqpoint{5.329872in}{2.647714in}}%
\pgfpathlineto{\pgfqpoint{5.337045in}{2.656889in}}%
\pgfpathlineto{\pgfqpoint{5.344213in}{2.666042in}}%
\pgfpathlineto{\pgfqpoint{5.351375in}{2.675174in}}%
\pgfpathlineto{\pgfqpoint{5.358533in}{2.684286in}}%
\pgfpathlineto{\pgfqpoint{5.344944in}{2.681430in}}%
\pgfpathlineto{\pgfqpoint{5.331368in}{2.678686in}}%
\pgfpathlineto{\pgfqpoint{5.317804in}{2.676055in}}%
\pgfpathlineto{\pgfqpoint{5.304251in}{2.673538in}}%
\pgfpathlineto{\pgfqpoint{5.297083in}{2.664313in}}%
\pgfpathlineto{\pgfqpoint{5.289909in}{2.655073in}}%
\pgfpathlineto{\pgfqpoint{5.282730in}{2.645816in}}%
\pgfpathlineto{\pgfqpoint{5.275546in}{2.636540in}}%
\pgfpathclose%
\pgfusepath{fill}%
\end{pgfscope}%
\begin{pgfscope}%
\pgfpathrectangle{\pgfqpoint{1.254980in}{0.150000in}}{\pgfqpoint{5.490039in}{5.490039in}}%
\pgfusepath{clip}%
\pgfsetbuttcap%
\pgfsetroundjoin%
\definecolor{currentfill}{rgb}{0.277134,0.185228,0.489898}%
\pgfsetfillcolor{currentfill}%
\pgfsetfillopacity{0.700000}%
\pgfsetlinewidth{0.000000pt}%
\definecolor{currentstroke}{rgb}{0.000000,0.000000,0.000000}%
\pgfsetstrokecolor{currentstroke}%
\pgfsetdash{}{0pt}%
\pgfpathmoveto{\pgfqpoint{4.695286in}{2.330789in}}%
\pgfpathlineto{\pgfqpoint{4.708613in}{2.330455in}}%
\pgfpathlineto{\pgfqpoint{4.721948in}{2.330240in}}%
\pgfpathlineto{\pgfqpoint{4.735294in}{2.330144in}}%
\pgfpathlineto{\pgfqpoint{4.748648in}{2.330166in}}%
\pgfpathlineto{\pgfqpoint{4.756026in}{2.340224in}}%
\pgfpathlineto{\pgfqpoint{4.763399in}{2.350267in}}%
\pgfpathlineto{\pgfqpoint{4.770767in}{2.360297in}}%
\pgfpathlineto{\pgfqpoint{4.778130in}{2.370314in}}%
\pgfpathlineto{\pgfqpoint{4.764784in}{2.370286in}}%
\pgfpathlineto{\pgfqpoint{4.751447in}{2.370375in}}%
\pgfpathlineto{\pgfqpoint{4.738119in}{2.370583in}}%
\pgfpathlineto{\pgfqpoint{4.724801in}{2.370910in}}%
\pgfpathlineto{\pgfqpoint{4.717430in}{2.360894in}}%
\pgfpathlineto{\pgfqpoint{4.710053in}{2.350869in}}%
\pgfpathlineto{\pgfqpoint{4.702672in}{2.340834in}}%
\pgfpathlineto{\pgfqpoint{4.695286in}{2.330789in}}%
\pgfpathclose%
\pgfusepath{fill}%
\end{pgfscope}%
\begin{pgfscope}%
\pgfpathrectangle{\pgfqpoint{1.254980in}{0.150000in}}{\pgfqpoint{5.490039in}{5.490039in}}%
\pgfusepath{clip}%
\pgfsetbuttcap%
\pgfsetroundjoin%
\definecolor{currentfill}{rgb}{0.282910,0.105393,0.426902}%
\pgfsetfillcolor{currentfill}%
\pgfsetfillopacity{0.700000}%
\pgfsetlinewidth{0.000000pt}%
\definecolor{currentstroke}{rgb}{0.000000,0.000000,0.000000}%
\pgfsetstrokecolor{currentstroke}%
\pgfsetdash{}{0pt}%
\pgfpathmoveto{\pgfqpoint{3.738698in}{2.192257in}}%
\pgfpathlineto{\pgfqpoint{3.751796in}{2.184526in}}%
\pgfpathlineto{\pgfqpoint{3.764898in}{2.176934in}}%
\pgfpathlineto{\pgfqpoint{3.778002in}{2.169479in}}%
\pgfpathlineto{\pgfqpoint{3.791110in}{2.162162in}}%
\pgfpathlineto{\pgfqpoint{3.798802in}{2.170550in}}%
\pgfpathlineto{\pgfqpoint{3.806488in}{2.178991in}}%
\pgfpathlineto{\pgfqpoint{3.814168in}{2.187484in}}%
\pgfpathlineto{\pgfqpoint{3.821843in}{2.196028in}}%
\pgfpathlineto{\pgfqpoint{3.808750in}{2.203194in}}%
\pgfpathlineto{\pgfqpoint{3.795660in}{2.210498in}}%
\pgfpathlineto{\pgfqpoint{3.782573in}{2.217939in}}%
\pgfpathlineto{\pgfqpoint{3.769489in}{2.225518in}}%
\pgfpathlineto{\pgfqpoint{3.761800in}{2.217120in}}%
\pgfpathlineto{\pgfqpoint{3.754105in}{2.208776in}}%
\pgfpathlineto{\pgfqpoint{3.746405in}{2.200488in}}%
\pgfpathlineto{\pgfqpoint{3.738698in}{2.192257in}}%
\pgfpathclose%
\pgfusepath{fill}%
\end{pgfscope}%
\begin{pgfscope}%
\pgfpathrectangle{\pgfqpoint{1.254980in}{0.150000in}}{\pgfqpoint{5.490039in}{5.490039in}}%
\pgfusepath{clip}%
\pgfsetbuttcap%
\pgfsetroundjoin%
\definecolor{currentfill}{rgb}{0.282910,0.105393,0.426902}%
\pgfsetfillcolor{currentfill}%
\pgfsetfillopacity{0.700000}%
\pgfsetlinewidth{0.000000pt}%
\definecolor{currentstroke}{rgb}{0.000000,0.000000,0.000000}%
\pgfsetstrokecolor{currentstroke}%
\pgfsetdash{}{0pt}%
\pgfpathmoveto{\pgfqpoint{4.228190in}{2.178608in}}%
\pgfpathlineto{\pgfqpoint{4.241372in}{2.175072in}}%
\pgfpathlineto{\pgfqpoint{4.254560in}{2.171661in}}%
\pgfpathlineto{\pgfqpoint{4.267755in}{2.168376in}}%
\pgfpathlineto{\pgfqpoint{4.280957in}{2.165216in}}%
\pgfpathlineto{\pgfqpoint{4.288482in}{2.174992in}}%
\pgfpathlineto{\pgfqpoint{4.296003in}{2.184780in}}%
\pgfpathlineto{\pgfqpoint{4.303519in}{2.194581in}}%
\pgfpathlineto{\pgfqpoint{4.311030in}{2.204393in}}%
\pgfpathlineto{\pgfqpoint{4.297838in}{2.207467in}}%
\pgfpathlineto{\pgfqpoint{4.284653in}{2.210666in}}%
\pgfpathlineto{\pgfqpoint{4.271474in}{2.213990in}}%
\pgfpathlineto{\pgfqpoint{4.258302in}{2.217440in}}%
\pgfpathlineto{\pgfqpoint{4.250781in}{2.207708in}}%
\pgfpathlineto{\pgfqpoint{4.243256in}{2.197992in}}%
\pgfpathlineto{\pgfqpoint{4.235725in}{2.188291in}}%
\pgfpathlineto{\pgfqpoint{4.228190in}{2.178608in}}%
\pgfpathclose%
\pgfusepath{fill}%
\end{pgfscope}%
\begin{pgfscope}%
\pgfpathrectangle{\pgfqpoint{1.254980in}{0.150000in}}{\pgfqpoint{5.490039in}{5.490039in}}%
\pgfusepath{clip}%
\pgfsetbuttcap%
\pgfsetroundjoin%
\definecolor{currentfill}{rgb}{0.216210,0.351535,0.550627}%
\pgfsetfillcolor{currentfill}%
\pgfsetfillopacity{0.700000}%
\pgfsetlinewidth{0.000000pt}%
\definecolor{currentstroke}{rgb}{0.000000,0.000000,0.000000}%
\pgfsetstrokecolor{currentstroke}%
\pgfsetdash{}{0pt}%
\pgfpathmoveto{\pgfqpoint{5.358533in}{2.684286in}}%
\pgfpathlineto{\pgfqpoint{5.372134in}{2.687256in}}%
\pgfpathlineto{\pgfqpoint{5.385748in}{2.690339in}}%
\pgfpathlineto{\pgfqpoint{5.399374in}{2.693535in}}%
\pgfpathlineto{\pgfqpoint{5.413013in}{2.696843in}}%
\pgfpathlineto{\pgfqpoint{5.420154in}{2.705820in}}%
\pgfpathlineto{\pgfqpoint{5.427290in}{2.714777in}}%
\pgfpathlineto{\pgfqpoint{5.434421in}{2.723713in}}%
\pgfpathlineto{\pgfqpoint{5.441547in}{2.732632in}}%
\pgfpathlineto{\pgfqpoint{5.427919in}{2.729447in}}%
\pgfpathlineto{\pgfqpoint{5.414305in}{2.726374in}}%
\pgfpathlineto{\pgfqpoint{5.400702in}{2.723414in}}%
\pgfpathlineto{\pgfqpoint{5.387113in}{2.720566in}}%
\pgfpathlineto{\pgfqpoint{5.379975in}{2.711519in}}%
\pgfpathlineto{\pgfqpoint{5.372833in}{2.702458in}}%
\pgfpathlineto{\pgfqpoint{5.365686in}{2.693381in}}%
\pgfpathlineto{\pgfqpoint{5.358533in}{2.684286in}}%
\pgfpathclose%
\pgfusepath{fill}%
\end{pgfscope}%
\begin{pgfscope}%
\pgfpathrectangle{\pgfqpoint{1.254980in}{0.150000in}}{\pgfqpoint{5.490039in}{5.490039in}}%
\pgfusepath{clip}%
\pgfsetbuttcap%
\pgfsetroundjoin%
\definecolor{currentfill}{rgb}{0.281924,0.089666,0.412415}%
\pgfsetfillcolor{currentfill}%
\pgfsetfillopacity{0.700000}%
\pgfsetlinewidth{0.000000pt}%
\definecolor{currentstroke}{rgb}{0.000000,0.000000,0.000000}%
\pgfsetstrokecolor{currentstroke}%
\pgfsetdash{}{0pt}%
\pgfpathmoveto{\pgfqpoint{4.009753in}{2.156822in}}%
\pgfpathlineto{\pgfqpoint{4.022888in}{2.151528in}}%
\pgfpathlineto{\pgfqpoint{4.036029in}{2.146364in}}%
\pgfpathlineto{\pgfqpoint{4.049174in}{2.141330in}}%
\pgfpathlineto{\pgfqpoint{4.062324in}{2.136425in}}%
\pgfpathlineto{\pgfqpoint{4.069921in}{2.145707in}}%
\pgfpathlineto{\pgfqpoint{4.077513in}{2.155018in}}%
\pgfpathlineto{\pgfqpoint{4.085100in}{2.164357in}}%
\pgfpathlineto{\pgfqpoint{4.092681in}{2.173725in}}%
\pgfpathlineto{\pgfqpoint{4.079542in}{2.178511in}}%
\pgfpathlineto{\pgfqpoint{4.066408in}{2.183427in}}%
\pgfpathlineto{\pgfqpoint{4.053280in}{2.188473in}}%
\pgfpathlineto{\pgfqpoint{4.040157in}{2.193649in}}%
\pgfpathlineto{\pgfqpoint{4.032564in}{2.184393in}}%
\pgfpathlineto{\pgfqpoint{4.024965in}{2.175170in}}%
\pgfpathlineto{\pgfqpoint{4.017362in}{2.165979in}}%
\pgfpathlineto{\pgfqpoint{4.009753in}{2.156822in}}%
\pgfpathclose%
\pgfusepath{fill}%
\end{pgfscope}%
\begin{pgfscope}%
\pgfpathrectangle{\pgfqpoint{1.254980in}{0.150000in}}{\pgfqpoint{5.490039in}{5.490039in}}%
\pgfusepath{clip}%
\pgfsetbuttcap%
\pgfsetroundjoin%
\definecolor{currentfill}{rgb}{0.280255,0.165693,0.476498}%
\pgfsetfillcolor{currentfill}%
\pgfsetfillopacity{0.700000}%
\pgfsetlinewidth{0.000000pt}%
\definecolor{currentstroke}{rgb}{0.000000,0.000000,0.000000}%
\pgfsetstrokecolor{currentstroke}%
\pgfsetdash{}{0pt}%
\pgfpathmoveto{\pgfqpoint{4.612446in}{2.293124in}}%
\pgfpathlineto{\pgfqpoint{4.625745in}{2.292290in}}%
\pgfpathlineto{\pgfqpoint{4.639052in}{2.291577in}}%
\pgfpathlineto{\pgfqpoint{4.652369in}{2.290982in}}%
\pgfpathlineto{\pgfqpoint{4.665694in}{2.290507in}}%
\pgfpathlineto{\pgfqpoint{4.673099in}{2.300593in}}%
\pgfpathlineto{\pgfqpoint{4.680500in}{2.310669in}}%
\pgfpathlineto{\pgfqpoint{4.687895in}{2.320734in}}%
\pgfpathlineto{\pgfqpoint{4.695286in}{2.330789in}}%
\pgfpathlineto{\pgfqpoint{4.681969in}{2.331241in}}%
\pgfpathlineto{\pgfqpoint{4.668661in}{2.331813in}}%
\pgfpathlineto{\pgfqpoint{4.655361in}{2.332504in}}%
\pgfpathlineto{\pgfqpoint{4.642071in}{2.333314in}}%
\pgfpathlineto{\pgfqpoint{4.634672in}{2.323276in}}%
\pgfpathlineto{\pgfqpoint{4.627268in}{2.313232in}}%
\pgfpathlineto{\pgfqpoint{4.619859in}{2.303181in}}%
\pgfpathlineto{\pgfqpoint{4.612446in}{2.293124in}}%
\pgfpathclose%
\pgfusepath{fill}%
\end{pgfscope}%
\begin{pgfscope}%
\pgfpathrectangle{\pgfqpoint{1.254980in}{0.150000in}}{\pgfqpoint{5.490039in}{5.490039in}}%
\pgfusepath{clip}%
\pgfsetbuttcap%
\pgfsetroundjoin%
\definecolor{currentfill}{rgb}{0.206756,0.371758,0.553117}%
\pgfsetfillcolor{currentfill}%
\pgfsetfillopacity{0.700000}%
\pgfsetlinewidth{0.000000pt}%
\definecolor{currentstroke}{rgb}{0.000000,0.000000,0.000000}%
\pgfsetstrokecolor{currentstroke}%
\pgfsetdash{}{0pt}%
\pgfpathmoveto{\pgfqpoint{5.441547in}{2.732632in}}%
\pgfpathlineto{\pgfqpoint{5.455187in}{2.735930in}}%
\pgfpathlineto{\pgfqpoint{5.468840in}{2.739340in}}%
\pgfpathlineto{\pgfqpoint{5.482506in}{2.742862in}}%
\pgfpathlineto{\pgfqpoint{5.496185in}{2.746497in}}%
\pgfpathlineto{\pgfqpoint{5.503293in}{2.755265in}}%
\pgfpathlineto{\pgfqpoint{5.510396in}{2.764014in}}%
\pgfpathlineto{\pgfqpoint{5.517494in}{2.772745in}}%
\pgfpathlineto{\pgfqpoint{5.524586in}{2.781460in}}%
\pgfpathlineto{\pgfqpoint{5.510920in}{2.777965in}}%
\pgfpathlineto{\pgfqpoint{5.497266in}{2.774582in}}%
\pgfpathlineto{\pgfqpoint{5.483626in}{2.771312in}}%
\pgfpathlineto{\pgfqpoint{5.469998in}{2.768153in}}%
\pgfpathlineto{\pgfqpoint{5.462893in}{2.759293in}}%
\pgfpathlineto{\pgfqpoint{5.455782in}{2.750420in}}%
\pgfpathlineto{\pgfqpoint{5.448667in}{2.741534in}}%
\pgfpathlineto{\pgfqpoint{5.441547in}{2.732632in}}%
\pgfpathclose%
\pgfusepath{fill}%
\end{pgfscope}%
\begin{pgfscope}%
\pgfpathrectangle{\pgfqpoint{1.254980in}{0.150000in}}{\pgfqpoint{5.490039in}{5.490039in}}%
\pgfusepath{clip}%
\pgfsetbuttcap%
\pgfsetroundjoin%
\definecolor{currentfill}{rgb}{0.276194,0.190074,0.493001}%
\pgfsetfillcolor{currentfill}%
\pgfsetfillopacity{0.700000}%
\pgfsetlinewidth{0.000000pt}%
\definecolor{currentstroke}{rgb}{0.000000,0.000000,0.000000}%
\pgfsetstrokecolor{currentstroke}%
\pgfsetdash{}{0pt}%
\pgfpathmoveto{\pgfqpoint{3.362207in}{2.365661in}}%
\pgfpathlineto{\pgfqpoint{3.375308in}{2.354069in}}%
\pgfpathlineto{\pgfqpoint{3.388409in}{2.342631in}}%
\pgfpathlineto{\pgfqpoint{3.401511in}{2.331347in}}%
\pgfpathlineto{\pgfqpoint{3.414612in}{2.320215in}}%
\pgfpathlineto{\pgfqpoint{3.422457in}{2.326997in}}%
\pgfpathlineto{\pgfqpoint{3.430296in}{2.333866in}}%
\pgfpathlineto{\pgfqpoint{3.438127in}{2.340820in}}%
\pgfpathlineto{\pgfqpoint{3.445951in}{2.347858in}}%
\pgfpathlineto{\pgfqpoint{3.432870in}{2.358803in}}%
\pgfpathlineto{\pgfqpoint{3.419789in}{2.369901in}}%
\pgfpathlineto{\pgfqpoint{3.406707in}{2.381152in}}%
\pgfpathlineto{\pgfqpoint{3.393626in}{2.392557in}}%
\pgfpathlineto{\pgfqpoint{3.385782in}{2.385700in}}%
\pgfpathlineto{\pgfqpoint{3.377931in}{2.378931in}}%
\pgfpathlineto{\pgfqpoint{3.370073in}{2.372251in}}%
\pgfpathlineto{\pgfqpoint{3.362207in}{2.365661in}}%
\pgfpathclose%
\pgfusepath{fill}%
\end{pgfscope}%
\begin{pgfscope}%
\pgfpathrectangle{\pgfqpoint{1.254980in}{0.150000in}}{\pgfqpoint{5.490039in}{5.490039in}}%
\pgfusepath{clip}%
\pgfsetbuttcap%
\pgfsetroundjoin%
\definecolor{currentfill}{rgb}{0.197636,0.391528,0.554969}%
\pgfsetfillcolor{currentfill}%
\pgfsetfillopacity{0.700000}%
\pgfsetlinewidth{0.000000pt}%
\definecolor{currentstroke}{rgb}{0.000000,0.000000,0.000000}%
\pgfsetstrokecolor{currentstroke}%
\pgfsetdash{}{0pt}%
\pgfpathmoveto{\pgfqpoint{5.524586in}{2.781460in}}%
\pgfpathlineto{\pgfqpoint{5.538266in}{2.785067in}}%
\pgfpathlineto{\pgfqpoint{5.551959in}{2.788786in}}%
\pgfpathlineto{\pgfqpoint{5.565665in}{2.792617in}}%
\pgfpathlineto{\pgfqpoint{5.579384in}{2.796559in}}%
\pgfpathlineto{\pgfqpoint{5.586459in}{2.805110in}}%
\pgfpathlineto{\pgfqpoint{5.593528in}{2.813643in}}%
\pgfpathlineto{\pgfqpoint{5.600592in}{2.822161in}}%
\pgfpathlineto{\pgfqpoint{5.607651in}{2.830665in}}%
\pgfpathlineto{\pgfqpoint{5.593945in}{2.826879in}}%
\pgfpathlineto{\pgfqpoint{5.580252in}{2.823204in}}%
\pgfpathlineto{\pgfqpoint{5.566573in}{2.819641in}}%
\pgfpathlineto{\pgfqpoint{5.552906in}{2.816190in}}%
\pgfpathlineto{\pgfqpoint{5.545834in}{2.807524in}}%
\pgfpathlineto{\pgfqpoint{5.538756in}{2.798848in}}%
\pgfpathlineto{\pgfqpoint{5.531674in}{2.790161in}}%
\pgfpathlineto{\pgfqpoint{5.524586in}{2.781460in}}%
\pgfpathclose%
\pgfusepath{fill}%
\end{pgfscope}%
\begin{pgfscope}%
\pgfpathrectangle{\pgfqpoint{1.254980in}{0.150000in}}{\pgfqpoint{5.490039in}{5.490039in}}%
\pgfusepath{clip}%
\pgfsetbuttcap%
\pgfsetroundjoin%
\definecolor{currentfill}{rgb}{0.281887,0.150881,0.465405}%
\pgfsetfillcolor{currentfill}%
\pgfsetfillopacity{0.700000}%
\pgfsetlinewidth{0.000000pt}%
\definecolor{currentstroke}{rgb}{0.000000,0.000000,0.000000}%
\pgfsetstrokecolor{currentstroke}%
\pgfsetdash{}{0pt}%
\pgfpathmoveto{\pgfqpoint{4.529602in}{2.257520in}}%
\pgfpathlineto{\pgfqpoint{4.542875in}{2.256166in}}%
\pgfpathlineto{\pgfqpoint{4.556156in}{2.254934in}}%
\pgfpathlineto{\pgfqpoint{4.569446in}{2.253821in}}%
\pgfpathlineto{\pgfqpoint{4.582744in}{2.252829in}}%
\pgfpathlineto{\pgfqpoint{4.590177in}{2.262913in}}%
\pgfpathlineto{\pgfqpoint{4.597605in}{2.272990in}}%
\pgfpathlineto{\pgfqpoint{4.605028in}{2.283060in}}%
\pgfpathlineto{\pgfqpoint{4.612446in}{2.293124in}}%
\pgfpathlineto{\pgfqpoint{4.599156in}{2.294077in}}%
\pgfpathlineto{\pgfqpoint{4.585874in}{2.295151in}}%
\pgfpathlineto{\pgfqpoint{4.572602in}{2.296345in}}%
\pgfpathlineto{\pgfqpoint{4.559337in}{2.297659in}}%
\pgfpathlineto{\pgfqpoint{4.551911in}{2.287628in}}%
\pgfpathlineto{\pgfqpoint{4.544479in}{2.277595in}}%
\pgfpathlineto{\pgfqpoint{4.537043in}{2.267558in}}%
\pgfpathlineto{\pgfqpoint{4.529602in}{2.257520in}}%
\pgfpathclose%
\pgfusepath{fill}%
\end{pgfscope}%
\begin{pgfscope}%
\pgfpathrectangle{\pgfqpoint{1.254980in}{0.150000in}}{\pgfqpoint{5.490039in}{5.490039in}}%
\pgfusepath{clip}%
\pgfsetbuttcap%
\pgfsetroundjoin%
\definecolor{currentfill}{rgb}{0.150476,0.504369,0.557430}%
\pgfsetfillcolor{currentfill}%
\pgfsetfillopacity{0.700000}%
\pgfsetlinewidth{0.000000pt}%
\definecolor{currentstroke}{rgb}{0.000000,0.000000,0.000000}%
\pgfsetstrokecolor{currentstroke}%
\pgfsetdash{}{0pt}%
\pgfpathmoveto{\pgfqpoint{2.730266in}{3.121343in}}%
\pgfpathlineto{\pgfqpoint{2.743551in}{3.101216in}}%
\pgfpathlineto{\pgfqpoint{2.756829in}{3.081294in}}%
\pgfpathlineto{\pgfqpoint{2.770098in}{3.061577in}}%
\pgfpathlineto{\pgfqpoint{2.783361in}{3.042062in}}%
\pgfpathlineto{\pgfqpoint{2.791501in}{3.046459in}}%
\pgfpathlineto{\pgfqpoint{2.799630in}{3.050993in}}%
\pgfpathlineto{\pgfqpoint{2.807749in}{3.055662in}}%
\pgfpathlineto{\pgfqpoint{2.815858in}{3.060466in}}%
\pgfpathlineto{\pgfqpoint{2.802625in}{3.079780in}}%
\pgfpathlineto{\pgfqpoint{2.789384in}{3.099296in}}%
\pgfpathlineto{\pgfqpoint{2.776137in}{3.119016in}}%
\pgfpathlineto{\pgfqpoint{2.762882in}{3.138941in}}%
\pgfpathlineto{\pgfqpoint{2.754744in}{3.134332in}}%
\pgfpathlineto{\pgfqpoint{2.746595in}{3.129862in}}%
\pgfpathlineto{\pgfqpoint{2.738436in}{3.125532in}}%
\pgfpathlineto{\pgfqpoint{2.730266in}{3.121343in}}%
\pgfpathclose%
\pgfusepath{fill}%
\end{pgfscope}%
\begin{pgfscope}%
\pgfpathrectangle{\pgfqpoint{1.254980in}{0.150000in}}{\pgfqpoint{5.490039in}{5.490039in}}%
\pgfusepath{clip}%
\pgfsetbuttcap%
\pgfsetroundjoin%
\definecolor{currentfill}{rgb}{0.187231,0.414746,0.556547}%
\pgfsetfillcolor{currentfill}%
\pgfsetfillopacity{0.700000}%
\pgfsetlinewidth{0.000000pt}%
\definecolor{currentstroke}{rgb}{0.000000,0.000000,0.000000}%
\pgfsetstrokecolor{currentstroke}%
\pgfsetdash{}{0pt}%
\pgfpathmoveto{\pgfqpoint{5.607651in}{2.830665in}}%
\pgfpathlineto{\pgfqpoint{5.621371in}{2.834563in}}%
\pgfpathlineto{\pgfqpoint{5.635104in}{2.838572in}}%
\pgfpathlineto{\pgfqpoint{5.648851in}{2.842692in}}%
\pgfpathlineto{\pgfqpoint{5.662611in}{2.846924in}}%
\pgfpathlineto{\pgfqpoint{5.669651in}{2.855250in}}%
\pgfpathlineto{\pgfqpoint{5.676686in}{2.863561in}}%
\pgfpathlineto{\pgfqpoint{5.683715in}{2.871860in}}%
\pgfpathlineto{\pgfqpoint{5.690740in}{2.880149in}}%
\pgfpathlineto{\pgfqpoint{5.676994in}{2.876090in}}%
\pgfpathlineto{\pgfqpoint{5.663261in}{2.872142in}}%
\pgfpathlineto{\pgfqpoint{5.649542in}{2.868306in}}%
\pgfpathlineto{\pgfqpoint{5.635837in}{2.864580in}}%
\pgfpathlineto{\pgfqpoint{5.628798in}{2.856113in}}%
\pgfpathlineto{\pgfqpoint{5.621754in}{2.847640in}}%
\pgfpathlineto{\pgfqpoint{5.614705in}{2.839158in}}%
\pgfpathlineto{\pgfqpoint{5.607651in}{2.830665in}}%
\pgfpathclose%
\pgfusepath{fill}%
\end{pgfscope}%
\begin{pgfscope}%
\pgfpathrectangle{\pgfqpoint{1.254980in}{0.150000in}}{\pgfqpoint{5.490039in}{5.490039in}}%
\pgfusepath{clip}%
\pgfsetbuttcap%
\pgfsetroundjoin%
\definecolor{currentfill}{rgb}{0.283187,0.125848,0.444960}%
\pgfsetfillcolor{currentfill}%
\pgfsetfillopacity{0.700000}%
\pgfsetlinewidth{0.000000pt}%
\definecolor{currentstroke}{rgb}{0.000000,0.000000,0.000000}%
\pgfsetstrokecolor{currentstroke}%
\pgfsetdash{}{0pt}%
\pgfpathmoveto{\pgfqpoint{3.602989in}{2.228155in}}%
\pgfpathlineto{\pgfqpoint{3.616085in}{2.219129in}}%
\pgfpathlineto{\pgfqpoint{3.629182in}{2.210246in}}%
\pgfpathlineto{\pgfqpoint{3.642281in}{2.201506in}}%
\pgfpathlineto{\pgfqpoint{3.655382in}{2.192908in}}%
\pgfpathlineto{\pgfqpoint{3.663129in}{2.200727in}}%
\pgfpathlineto{\pgfqpoint{3.670869in}{2.208612in}}%
\pgfpathlineto{\pgfqpoint{3.678604in}{2.216561in}}%
\pgfpathlineto{\pgfqpoint{3.686332in}{2.224574in}}%
\pgfpathlineto{\pgfqpoint{3.673247in}{2.233004in}}%
\pgfpathlineto{\pgfqpoint{3.660164in}{2.241577in}}%
\pgfpathlineto{\pgfqpoint{3.647083in}{2.250291in}}%
\pgfpathlineto{\pgfqpoint{3.634005in}{2.259149in}}%
\pgfpathlineto{\pgfqpoint{3.626260in}{2.251299in}}%
\pgfpathlineto{\pgfqpoint{3.618510in}{2.243516in}}%
\pgfpathlineto{\pgfqpoint{3.610753in}{2.235800in}}%
\pgfpathlineto{\pgfqpoint{3.602989in}{2.228155in}}%
\pgfpathclose%
\pgfusepath{fill}%
\end{pgfscope}%
\begin{pgfscope}%
\pgfpathrectangle{\pgfqpoint{1.254980in}{0.150000in}}{\pgfqpoint{5.490039in}{5.490039in}}%
\pgfusepath{clip}%
\pgfsetbuttcap%
\pgfsetroundjoin%
\definecolor{currentfill}{rgb}{0.179019,0.433756,0.557430}%
\pgfsetfillcolor{currentfill}%
\pgfsetfillopacity{0.700000}%
\pgfsetlinewidth{0.000000pt}%
\definecolor{currentstroke}{rgb}{0.000000,0.000000,0.000000}%
\pgfsetstrokecolor{currentstroke}%
\pgfsetdash{}{0pt}%
\pgfpathmoveto{\pgfqpoint{5.690740in}{2.880149in}}%
\pgfpathlineto{\pgfqpoint{5.704500in}{2.884319in}}%
\pgfpathlineto{\pgfqpoint{5.718274in}{2.888599in}}%
\pgfpathlineto{\pgfqpoint{5.732061in}{2.892991in}}%
\pgfpathlineto{\pgfqpoint{5.745863in}{2.897493in}}%
\pgfpathlineto{\pgfqpoint{5.752868in}{2.905590in}}%
\pgfpathlineto{\pgfqpoint{5.759867in}{2.913675in}}%
\pgfpathlineto{\pgfqpoint{5.766862in}{2.921752in}}%
\pgfpathlineto{\pgfqpoint{5.773851in}{2.929822in}}%
\pgfpathlineto{\pgfqpoint{5.760065in}{2.925509in}}%
\pgfpathlineto{\pgfqpoint{5.746292in}{2.921307in}}%
\pgfpathlineto{\pgfqpoint{5.732534in}{2.917215in}}%
\pgfpathlineto{\pgfqpoint{5.718789in}{2.913234in}}%
\pgfpathlineto{\pgfqpoint{5.711784in}{2.904969in}}%
\pgfpathlineto{\pgfqpoint{5.704774in}{2.896701in}}%
\pgfpathlineto{\pgfqpoint{5.697760in}{2.888428in}}%
\pgfpathlineto{\pgfqpoint{5.690740in}{2.880149in}}%
\pgfpathclose%
\pgfusepath{fill}%
\end{pgfscope}%
\begin{pgfscope}%
\pgfpathrectangle{\pgfqpoint{1.254980in}{0.150000in}}{\pgfqpoint{5.490039in}{5.490039in}}%
\pgfusepath{clip}%
\pgfsetbuttcap%
\pgfsetroundjoin%
\definecolor{currentfill}{rgb}{0.282327,0.094955,0.417331}%
\pgfsetfillcolor{currentfill}%
\pgfsetfillopacity{0.700000}%
\pgfsetlinewidth{0.000000pt}%
\definecolor{currentstroke}{rgb}{0.000000,0.000000,0.000000}%
\pgfsetstrokecolor{currentstroke}%
\pgfsetdash{}{0pt}%
\pgfpathmoveto{\pgfqpoint{4.145293in}{2.155867in}}%
\pgfpathlineto{\pgfqpoint{4.158461in}{2.151723in}}%
\pgfpathlineto{\pgfqpoint{4.171634in}{2.147705in}}%
\pgfpathlineto{\pgfqpoint{4.184814in}{2.143814in}}%
\pgfpathlineto{\pgfqpoint{4.197999in}{2.140050in}}%
\pgfpathlineto{\pgfqpoint{4.205554in}{2.149662in}}%
\pgfpathlineto{\pgfqpoint{4.213104in}{2.159292in}}%
\pgfpathlineto{\pgfqpoint{4.220650in}{2.168941in}}%
\pgfpathlineto{\pgfqpoint{4.228190in}{2.178608in}}%
\pgfpathlineto{\pgfqpoint{4.215014in}{2.182270in}}%
\pgfpathlineto{\pgfqpoint{4.201845in}{2.186058in}}%
\pgfpathlineto{\pgfqpoint{4.188682in}{2.189974in}}%
\pgfpathlineto{\pgfqpoint{4.175525in}{2.194016in}}%
\pgfpathlineto{\pgfqpoint{4.167975in}{2.184446in}}%
\pgfpathlineto{\pgfqpoint{4.160419in}{2.174897in}}%
\pgfpathlineto{\pgfqpoint{4.152859in}{2.165371in}}%
\pgfpathlineto{\pgfqpoint{4.145293in}{2.155867in}}%
\pgfpathclose%
\pgfusepath{fill}%
\end{pgfscope}%
\begin{pgfscope}%
\pgfpathrectangle{\pgfqpoint{1.254980in}{0.150000in}}{\pgfqpoint{5.490039in}{5.490039in}}%
\pgfusepath{clip}%
\pgfsetbuttcap%
\pgfsetroundjoin%
\definecolor{currentfill}{rgb}{0.283072,0.130895,0.449241}%
\pgfsetfillcolor{currentfill}%
\pgfsetfillopacity{0.700000}%
\pgfsetlinewidth{0.000000pt}%
\definecolor{currentstroke}{rgb}{0.000000,0.000000,0.000000}%
\pgfsetstrokecolor{currentstroke}%
\pgfsetdash{}{0pt}%
\pgfpathmoveto{\pgfqpoint{4.446746in}{2.224185in}}%
\pgfpathlineto{\pgfqpoint{4.459995in}{2.222292in}}%
\pgfpathlineto{\pgfqpoint{4.473252in}{2.220520in}}%
\pgfpathlineto{\pgfqpoint{4.486518in}{2.218870in}}%
\pgfpathlineto{\pgfqpoint{4.499791in}{2.217341in}}%
\pgfpathlineto{\pgfqpoint{4.507251in}{2.227389in}}%
\pgfpathlineto{\pgfqpoint{4.514706in}{2.237435in}}%
\pgfpathlineto{\pgfqpoint{4.522156in}{2.247478in}}%
\pgfpathlineto{\pgfqpoint{4.529602in}{2.257520in}}%
\pgfpathlineto{\pgfqpoint{4.516337in}{2.258994in}}%
\pgfpathlineto{\pgfqpoint{4.503081in}{2.260589in}}%
\pgfpathlineto{\pgfqpoint{4.489833in}{2.262306in}}%
\pgfpathlineto{\pgfqpoint{4.476592in}{2.264145in}}%
\pgfpathlineto{\pgfqpoint{4.469138in}{2.254153in}}%
\pgfpathlineto{\pgfqpoint{4.461679in}{2.244162in}}%
\pgfpathlineto{\pgfqpoint{4.454215in}{2.234173in}}%
\pgfpathlineto{\pgfqpoint{4.446746in}{2.224185in}}%
\pgfpathclose%
\pgfusepath{fill}%
\end{pgfscope}%
\begin{pgfscope}%
\pgfpathrectangle{\pgfqpoint{1.254980in}{0.150000in}}{\pgfqpoint{5.490039in}{5.490039in}}%
\pgfusepath{clip}%
\pgfsetbuttcap%
\pgfsetroundjoin%
\definecolor{currentfill}{rgb}{0.279574,0.170599,0.479997}%
\pgfsetfillcolor{currentfill}%
\pgfsetfillopacity{0.700000}%
\pgfsetlinewidth{0.000000pt}%
\definecolor{currentstroke}{rgb}{0.000000,0.000000,0.000000}%
\pgfsetstrokecolor{currentstroke}%
\pgfsetdash{}{0pt}%
\pgfpathmoveto{\pgfqpoint{3.414612in}{2.320215in}}%
\pgfpathlineto{\pgfqpoint{3.427713in}{2.309236in}}%
\pgfpathlineto{\pgfqpoint{3.440814in}{2.298407in}}%
\pgfpathlineto{\pgfqpoint{3.453916in}{2.287729in}}%
\pgfpathlineto{\pgfqpoint{3.467019in}{2.277201in}}%
\pgfpathlineto{\pgfqpoint{3.474845in}{2.284176in}}%
\pgfpathlineto{\pgfqpoint{3.482664in}{2.291232in}}%
\pgfpathlineto{\pgfqpoint{3.490476in}{2.298370in}}%
\pgfpathlineto{\pgfqpoint{3.498282in}{2.305588in}}%
\pgfpathlineto{\pgfqpoint{3.485198in}{2.315930in}}%
\pgfpathlineto{\pgfqpoint{3.472116in}{2.326422in}}%
\pgfpathlineto{\pgfqpoint{3.459033in}{2.337064in}}%
\pgfpathlineto{\pgfqpoint{3.445951in}{2.347858in}}%
\pgfpathlineto{\pgfqpoint{3.438127in}{2.340820in}}%
\pgfpathlineto{\pgfqpoint{3.430296in}{2.333866in}}%
\pgfpathlineto{\pgfqpoint{3.422457in}{2.326997in}}%
\pgfpathlineto{\pgfqpoint{3.414612in}{2.320215in}}%
\pgfpathclose%
\pgfusepath{fill}%
\end{pgfscope}%
\begin{pgfscope}%
\pgfpathrectangle{\pgfqpoint{1.254980in}{0.150000in}}{\pgfqpoint{5.490039in}{5.490039in}}%
\pgfusepath{clip}%
\pgfsetbuttcap%
\pgfsetroundjoin%
\definecolor{currentfill}{rgb}{0.171176,0.452530,0.557965}%
\pgfsetfillcolor{currentfill}%
\pgfsetfillopacity{0.700000}%
\pgfsetlinewidth{0.000000pt}%
\definecolor{currentstroke}{rgb}{0.000000,0.000000,0.000000}%
\pgfsetstrokecolor{currentstroke}%
\pgfsetdash{}{0pt}%
\pgfpathmoveto{\pgfqpoint{5.773851in}{2.929822in}}%
\pgfpathlineto{\pgfqpoint{5.787652in}{2.934246in}}%
\pgfpathlineto{\pgfqpoint{5.801466in}{2.938779in}}%
\pgfpathlineto{\pgfqpoint{5.815295in}{2.943424in}}%
\pgfpathlineto{\pgfqpoint{5.829138in}{2.948179in}}%
\pgfpathlineto{\pgfqpoint{5.836107in}{2.956044in}}%
\pgfpathlineto{\pgfqpoint{5.843070in}{2.963902in}}%
\pgfpathlineto{\pgfqpoint{5.850029in}{2.971755in}}%
\pgfpathlineto{\pgfqpoint{5.856983in}{2.979606in}}%
\pgfpathlineto{\pgfqpoint{5.843156in}{2.975058in}}%
\pgfpathlineto{\pgfqpoint{5.829343in}{2.970619in}}%
\pgfpathlineto{\pgfqpoint{5.815545in}{2.966291in}}%
\pgfpathlineto{\pgfqpoint{5.801761in}{2.962073in}}%
\pgfpathlineto{\pgfqpoint{5.794790in}{2.954011in}}%
\pgfpathlineto{\pgfqpoint{5.787815in}{2.945949in}}%
\pgfpathlineto{\pgfqpoint{5.780836in}{2.937887in}}%
\pgfpathlineto{\pgfqpoint{5.773851in}{2.929822in}}%
\pgfpathclose%
\pgfusepath{fill}%
\end{pgfscope}%
\begin{pgfscope}%
\pgfpathrectangle{\pgfqpoint{1.254980in}{0.150000in}}{\pgfqpoint{5.490039in}{5.490039in}}%
\pgfusepath{clip}%
\pgfsetbuttcap%
\pgfsetroundjoin%
\definecolor{currentfill}{rgb}{0.163625,0.471133,0.558148}%
\pgfsetfillcolor{currentfill}%
\pgfsetfillopacity{0.700000}%
\pgfsetlinewidth{0.000000pt}%
\definecolor{currentstroke}{rgb}{0.000000,0.000000,0.000000}%
\pgfsetstrokecolor{currentstroke}%
\pgfsetdash{}{0pt}%
\pgfpathmoveto{\pgfqpoint{5.856983in}{2.979606in}}%
\pgfpathlineto{\pgfqpoint{5.870824in}{2.984265in}}%
\pgfpathlineto{\pgfqpoint{5.884680in}{2.989034in}}%
\pgfpathlineto{\pgfqpoint{5.898550in}{2.993912in}}%
\pgfpathlineto{\pgfqpoint{5.912435in}{2.998901in}}%
\pgfpathlineto{\pgfqpoint{5.919366in}{3.006535in}}%
\pgfpathlineto{\pgfqpoint{5.926293in}{3.014166in}}%
\pgfpathlineto{\pgfqpoint{5.933215in}{3.021798in}}%
\pgfpathlineto{\pgfqpoint{5.940133in}{3.029431in}}%
\pgfpathlineto{\pgfqpoint{5.926265in}{3.024666in}}%
\pgfpathlineto{\pgfqpoint{5.912413in}{3.020010in}}%
\pgfpathlineto{\pgfqpoint{5.898574in}{3.015463in}}%
\pgfpathlineto{\pgfqpoint{5.884750in}{3.011027in}}%
\pgfpathlineto{\pgfqpoint{5.877815in}{3.003165in}}%
\pgfpathlineto{\pgfqpoint{5.870876in}{2.995309in}}%
\pgfpathlineto{\pgfqpoint{5.863932in}{2.987457in}}%
\pgfpathlineto{\pgfqpoint{5.856983in}{2.979606in}}%
\pgfpathclose%
\pgfusepath{fill}%
\end{pgfscope}%
\begin{pgfscope}%
\pgfpathrectangle{\pgfqpoint{1.254980in}{0.150000in}}{\pgfqpoint{5.490039in}{5.490039in}}%
\pgfusepath{clip}%
\pgfsetbuttcap%
\pgfsetroundjoin%
\definecolor{currentfill}{rgb}{0.282327,0.094955,0.417331}%
\pgfsetfillcolor{currentfill}%
\pgfsetfillopacity{0.700000}%
\pgfsetlinewidth{0.000000pt}%
\definecolor{currentstroke}{rgb}{0.000000,0.000000,0.000000}%
\pgfsetstrokecolor{currentstroke}%
\pgfsetdash{}{0pt}%
\pgfpathmoveto{\pgfqpoint{3.791110in}{2.162162in}}%
\pgfpathlineto{\pgfqpoint{3.804221in}{2.154981in}}%
\pgfpathlineto{\pgfqpoint{3.817336in}{2.147937in}}%
\pgfpathlineto{\pgfqpoint{3.830454in}{2.141028in}}%
\pgfpathlineto{\pgfqpoint{3.843575in}{2.134255in}}%
\pgfpathlineto{\pgfqpoint{3.851253in}{2.142800in}}%
\pgfpathlineto{\pgfqpoint{3.858925in}{2.151393in}}%
\pgfpathlineto{\pgfqpoint{3.866592in}{2.160035in}}%
\pgfpathlineto{\pgfqpoint{3.874253in}{2.168723in}}%
\pgfpathlineto{\pgfqpoint{3.861145in}{2.175346in}}%
\pgfpathlineto{\pgfqpoint{3.848040in}{2.182104in}}%
\pgfpathlineto{\pgfqpoint{3.834940in}{2.188998in}}%
\pgfpathlineto{\pgfqpoint{3.821843in}{2.196028in}}%
\pgfpathlineto{\pgfqpoint{3.814168in}{2.187484in}}%
\pgfpathlineto{\pgfqpoint{3.806488in}{2.178991in}}%
\pgfpathlineto{\pgfqpoint{3.798802in}{2.170550in}}%
\pgfpathlineto{\pgfqpoint{3.791110in}{2.162162in}}%
\pgfpathclose%
\pgfusepath{fill}%
\end{pgfscope}%
\begin{pgfscope}%
\pgfpathrectangle{\pgfqpoint{1.254980in}{0.150000in}}{\pgfqpoint{5.490039in}{5.490039in}}%
\pgfusepath{clip}%
\pgfsetbuttcap%
\pgfsetroundjoin%
\definecolor{currentfill}{rgb}{0.281446,0.084320,0.407414}%
\pgfsetfillcolor{currentfill}%
\pgfsetfillopacity{0.700000}%
\pgfsetlinewidth{0.000000pt}%
\definecolor{currentstroke}{rgb}{0.000000,0.000000,0.000000}%
\pgfsetstrokecolor{currentstroke}%
\pgfsetdash{}{0pt}%
\pgfpathmoveto{\pgfqpoint{3.926725in}{2.143573in}}%
\pgfpathlineto{\pgfqpoint{3.939853in}{2.137618in}}%
\pgfpathlineto{\pgfqpoint{3.952987in}{2.131795in}}%
\pgfpathlineto{\pgfqpoint{3.966124in}{2.126104in}}%
\pgfpathlineto{\pgfqpoint{3.979267in}{2.120545in}}%
\pgfpathlineto{\pgfqpoint{3.986896in}{2.129560in}}%
\pgfpathlineto{\pgfqpoint{3.994521in}{2.138612in}}%
\pgfpathlineto{\pgfqpoint{4.002140in}{2.147699in}}%
\pgfpathlineto{\pgfqpoint{4.009753in}{2.156822in}}%
\pgfpathlineto{\pgfqpoint{3.996623in}{2.162247in}}%
\pgfpathlineto{\pgfqpoint{3.983498in}{2.167804in}}%
\pgfpathlineto{\pgfqpoint{3.970377in}{2.173492in}}%
\pgfpathlineto{\pgfqpoint{3.957261in}{2.179312in}}%
\pgfpathlineto{\pgfqpoint{3.949635in}{2.170318in}}%
\pgfpathlineto{\pgfqpoint{3.942004in}{2.161363in}}%
\pgfpathlineto{\pgfqpoint{3.934367in}{2.152447in}}%
\pgfpathlineto{\pgfqpoint{3.926725in}{2.143573in}}%
\pgfpathclose%
\pgfusepath{fill}%
\end{pgfscope}%
\begin{pgfscope}%
\pgfpathrectangle{\pgfqpoint{1.254980in}{0.150000in}}{\pgfqpoint{5.490039in}{5.490039in}}%
\pgfusepath{clip}%
\pgfsetbuttcap%
\pgfsetroundjoin%
\definecolor{currentfill}{rgb}{0.137770,0.537492,0.554906}%
\pgfsetfillcolor{currentfill}%
\pgfsetfillopacity{0.700000}%
\pgfsetlinewidth{0.000000pt}%
\definecolor{currentstroke}{rgb}{0.000000,0.000000,0.000000}%
\pgfsetstrokecolor{currentstroke}%
\pgfsetdash{}{0pt}%
\pgfpathmoveto{\pgfqpoint{2.677045in}{3.203937in}}%
\pgfpathlineto{\pgfqpoint{2.690362in}{3.182973in}}%
\pgfpathlineto{\pgfqpoint{2.703672in}{3.162220in}}%
\pgfpathlineto{\pgfqpoint{2.716973in}{3.141677in}}%
\pgfpathlineto{\pgfqpoint{2.730266in}{3.121343in}}%
\pgfpathlineto{\pgfqpoint{2.738436in}{3.125532in}}%
\pgfpathlineto{\pgfqpoint{2.746595in}{3.129862in}}%
\pgfpathlineto{\pgfqpoint{2.754744in}{3.134332in}}%
\pgfpathlineto{\pgfqpoint{2.762882in}{3.138941in}}%
\pgfpathlineto{\pgfqpoint{2.749619in}{3.159073in}}%
\pgfpathlineto{\pgfqpoint{2.736348in}{3.179412in}}%
\pgfpathlineto{\pgfqpoint{2.723069in}{3.199962in}}%
\pgfpathlineto{\pgfqpoint{2.709782in}{3.220722in}}%
\pgfpathlineto{\pgfqpoint{2.701614in}{3.216310in}}%
\pgfpathlineto{\pgfqpoint{2.693435in}{3.212041in}}%
\pgfpathlineto{\pgfqpoint{2.685246in}{3.207916in}}%
\pgfpathlineto{\pgfqpoint{2.677045in}{3.203937in}}%
\pgfpathclose%
\pgfusepath{fill}%
\end{pgfscope}%
\begin{pgfscope}%
\pgfpathrectangle{\pgfqpoint{1.254980in}{0.150000in}}{\pgfqpoint{5.490039in}{5.490039in}}%
\pgfusepath{clip}%
\pgfsetbuttcap%
\pgfsetroundjoin%
\definecolor{currentfill}{rgb}{0.283229,0.120777,0.440584}%
\pgfsetfillcolor{currentfill}%
\pgfsetfillopacity{0.700000}%
\pgfsetlinewidth{0.000000pt}%
\definecolor{currentstroke}{rgb}{0.000000,0.000000,0.000000}%
\pgfsetstrokecolor{currentstroke}%
\pgfsetdash{}{0pt}%
\pgfpathmoveto{\pgfqpoint{4.363868in}{2.193341in}}%
\pgfpathlineto{\pgfqpoint{4.377096in}{2.190887in}}%
\pgfpathlineto{\pgfqpoint{4.390331in}{2.188556in}}%
\pgfpathlineto{\pgfqpoint{4.403574in}{2.186347in}}%
\pgfpathlineto{\pgfqpoint{4.416824in}{2.184261in}}%
\pgfpathlineto{\pgfqpoint{4.424312in}{2.194238in}}%
\pgfpathlineto{\pgfqpoint{4.431795in}{2.204218in}}%
\pgfpathlineto{\pgfqpoint{4.439273in}{2.214200in}}%
\pgfpathlineto{\pgfqpoint{4.446746in}{2.224185in}}%
\pgfpathlineto{\pgfqpoint{4.433505in}{2.226201in}}%
\pgfpathlineto{\pgfqpoint{4.420271in}{2.228339in}}%
\pgfpathlineto{\pgfqpoint{4.407045in}{2.230599in}}%
\pgfpathlineto{\pgfqpoint{4.393827in}{2.232983in}}%
\pgfpathlineto{\pgfqpoint{4.386344in}{2.223063in}}%
\pgfpathlineto{\pgfqpoint{4.378857in}{2.213148in}}%
\pgfpathlineto{\pgfqpoint{4.371365in}{2.203241in}}%
\pgfpathlineto{\pgfqpoint{4.363868in}{2.193341in}}%
\pgfpathclose%
\pgfusepath{fill}%
\end{pgfscope}%
\begin{pgfscope}%
\pgfpathrectangle{\pgfqpoint{1.254980in}{0.150000in}}{\pgfqpoint{5.490039in}{5.490039in}}%
\pgfusepath{clip}%
\pgfsetbuttcap%
\pgfsetroundjoin%
\definecolor{currentfill}{rgb}{0.154815,0.493313,0.557840}%
\pgfsetfillcolor{currentfill}%
\pgfsetfillopacity{0.700000}%
\pgfsetlinewidth{0.000000pt}%
\definecolor{currentstroke}{rgb}{0.000000,0.000000,0.000000}%
\pgfsetstrokecolor{currentstroke}%
\pgfsetdash{}{0pt}%
\pgfpathmoveto{\pgfqpoint{5.940133in}{3.029431in}}%
\pgfpathlineto{\pgfqpoint{5.954015in}{3.034307in}}%
\pgfpathlineto{\pgfqpoint{5.967911in}{3.039292in}}%
\pgfpathlineto{\pgfqpoint{5.981823in}{3.044386in}}%
\pgfpathlineto{\pgfqpoint{5.995749in}{3.049591in}}%
\pgfpathlineto{\pgfqpoint{6.002644in}{3.056995in}}%
\pgfpathlineto{\pgfqpoint{6.009533in}{3.064403in}}%
\pgfpathlineto{\pgfqpoint{6.016418in}{3.071816in}}%
\pgfpathlineto{\pgfqpoint{6.023299in}{3.079237in}}%
\pgfpathlineto{\pgfqpoint{6.009391in}{3.074272in}}%
\pgfpathlineto{\pgfqpoint{5.995498in}{3.069417in}}%
\pgfpathlineto{\pgfqpoint{5.981620in}{3.064671in}}%
\pgfpathlineto{\pgfqpoint{5.967757in}{3.060034in}}%
\pgfpathlineto{\pgfqpoint{5.960857in}{3.052369in}}%
\pgfpathlineto{\pgfqpoint{5.953954in}{3.044715in}}%
\pgfpathlineto{\pgfqpoint{5.947045in}{3.037069in}}%
\pgfpathlineto{\pgfqpoint{5.940133in}{3.029431in}}%
\pgfpathclose%
\pgfusepath{fill}%
\end{pgfscope}%
\begin{pgfscope}%
\pgfpathrectangle{\pgfqpoint{1.254980in}{0.150000in}}{\pgfqpoint{5.490039in}{5.490039in}}%
\pgfusepath{clip}%
\pgfsetbuttcap%
\pgfsetroundjoin%
\definecolor{currentfill}{rgb}{0.283091,0.110553,0.431554}%
\pgfsetfillcolor{currentfill}%
\pgfsetfillopacity{0.700000}%
\pgfsetlinewidth{0.000000pt}%
\definecolor{currentstroke}{rgb}{0.000000,0.000000,0.000000}%
\pgfsetstrokecolor{currentstroke}%
\pgfsetdash{}{0pt}%
\pgfpathmoveto{\pgfqpoint{3.655382in}{2.192908in}}%
\pgfpathlineto{\pgfqpoint{3.668486in}{2.184450in}}%
\pgfpathlineto{\pgfqpoint{3.681592in}{2.176134in}}%
\pgfpathlineto{\pgfqpoint{3.694700in}{2.167958in}}%
\pgfpathlineto{\pgfqpoint{3.707811in}{2.159921in}}%
\pgfpathlineto{\pgfqpoint{3.715542in}{2.167914in}}%
\pgfpathlineto{\pgfqpoint{3.723267in}{2.175969in}}%
\pgfpathlineto{\pgfqpoint{3.730985in}{2.184083in}}%
\pgfpathlineto{\pgfqpoint{3.738698in}{2.192257in}}%
\pgfpathlineto{\pgfqpoint{3.725602in}{2.200126in}}%
\pgfpathlineto{\pgfqpoint{3.712510in}{2.208135in}}%
\pgfpathlineto{\pgfqpoint{3.699419in}{2.216284in}}%
\pgfpathlineto{\pgfqpoint{3.686332in}{2.224574in}}%
\pgfpathlineto{\pgfqpoint{3.678604in}{2.216561in}}%
\pgfpathlineto{\pgfqpoint{3.670869in}{2.208612in}}%
\pgfpathlineto{\pgfqpoint{3.663129in}{2.200727in}}%
\pgfpathlineto{\pgfqpoint{3.655382in}{2.192908in}}%
\pgfpathclose%
\pgfusepath{fill}%
\end{pgfscope}%
\begin{pgfscope}%
\pgfpathrectangle{\pgfqpoint{1.254980in}{0.150000in}}{\pgfqpoint{5.490039in}{5.490039in}}%
\pgfusepath{clip}%
\pgfsetbuttcap%
\pgfsetroundjoin%
\definecolor{currentfill}{rgb}{0.225863,0.330805,0.547314}%
\pgfsetfillcolor{currentfill}%
\pgfsetfillopacity{0.700000}%
\pgfsetlinewidth{0.000000pt}%
\definecolor{currentstroke}{rgb}{0.000000,0.000000,0.000000}%
\pgfsetstrokecolor{currentstroke}%
\pgfsetdash{}{0pt}%
\pgfpathmoveto{\pgfqpoint{3.015242in}{2.671865in}}%
\pgfpathlineto{\pgfqpoint{3.028416in}{2.656103in}}%
\pgfpathlineto{\pgfqpoint{3.041587in}{2.640517in}}%
\pgfpathlineto{\pgfqpoint{3.054754in}{2.625106in}}%
\pgfpathlineto{\pgfqpoint{3.067917in}{2.609870in}}%
\pgfpathlineto{\pgfqpoint{3.075931in}{2.615021in}}%
\pgfpathlineto{\pgfqpoint{3.083935in}{2.620289in}}%
\pgfpathlineto{\pgfqpoint{3.091931in}{2.625674in}}%
\pgfpathlineto{\pgfqpoint{3.099918in}{2.631172in}}%
\pgfpathlineto{\pgfqpoint{3.086780in}{2.646200in}}%
\pgfpathlineto{\pgfqpoint{3.073638in}{2.661401in}}%
\pgfpathlineto{\pgfqpoint{3.060493in}{2.676777in}}%
\pgfpathlineto{\pgfqpoint{3.047344in}{2.692329in}}%
\pgfpathlineto{\pgfqpoint{3.039332in}{2.687034in}}%
\pgfpathlineto{\pgfqpoint{3.031311in}{2.681857in}}%
\pgfpathlineto{\pgfqpoint{3.023281in}{2.676800in}}%
\pgfpathlineto{\pgfqpoint{3.015242in}{2.671865in}}%
\pgfpathclose%
\pgfusepath{fill}%
\end{pgfscope}%
\begin{pgfscope}%
\pgfpathrectangle{\pgfqpoint{1.254980in}{0.150000in}}{\pgfqpoint{5.490039in}{5.490039in}}%
\pgfusepath{clip}%
\pgfsetbuttcap%
\pgfsetroundjoin%
\definecolor{currentfill}{rgb}{0.237441,0.305202,0.541921}%
\pgfsetfillcolor{currentfill}%
\pgfsetfillopacity{0.700000}%
\pgfsetlinewidth{0.000000pt}%
\definecolor{currentstroke}{rgb}{0.000000,0.000000,0.000000}%
\pgfsetstrokecolor{currentstroke}%
\pgfsetdash{}{0pt}%
\pgfpathmoveto{\pgfqpoint{3.067917in}{2.609870in}}%
\pgfpathlineto{\pgfqpoint{3.081077in}{2.594807in}}%
\pgfpathlineto{\pgfqpoint{3.094234in}{2.579916in}}%
\pgfpathlineto{\pgfqpoint{3.107388in}{2.565196in}}%
\pgfpathlineto{\pgfqpoint{3.120539in}{2.550646in}}%
\pgfpathlineto{\pgfqpoint{3.128527in}{2.556012in}}%
\pgfpathlineto{\pgfqpoint{3.136507in}{2.561490in}}%
\pgfpathlineto{\pgfqpoint{3.144478in}{2.567081in}}%
\pgfpathlineto{\pgfqpoint{3.152440in}{2.572781in}}%
\pgfpathlineto{\pgfqpoint{3.139314in}{2.587123in}}%
\pgfpathlineto{\pgfqpoint{3.126185in}{2.601635in}}%
\pgfpathlineto{\pgfqpoint{3.113053in}{2.616318in}}%
\pgfpathlineto{\pgfqpoint{3.099918in}{2.631172in}}%
\pgfpathlineto{\pgfqpoint{3.091931in}{2.625674in}}%
\pgfpathlineto{\pgfqpoint{3.083935in}{2.620289in}}%
\pgfpathlineto{\pgfqpoint{3.075931in}{2.615021in}}%
\pgfpathlineto{\pgfqpoint{3.067917in}{2.609870in}}%
\pgfpathclose%
\pgfusepath{fill}%
\end{pgfscope}%
\begin{pgfscope}%
\pgfpathrectangle{\pgfqpoint{1.254980in}{0.150000in}}{\pgfqpoint{5.490039in}{5.490039in}}%
\pgfusepath{clip}%
\pgfsetbuttcap%
\pgfsetroundjoin%
\definecolor{currentfill}{rgb}{0.281887,0.150881,0.465405}%
\pgfsetfillcolor{currentfill}%
\pgfsetfillopacity{0.700000}%
\pgfsetlinewidth{0.000000pt}%
\definecolor{currentstroke}{rgb}{0.000000,0.000000,0.000000}%
\pgfsetstrokecolor{currentstroke}%
\pgfsetdash{}{0pt}%
\pgfpathmoveto{\pgfqpoint{3.467019in}{2.277201in}}%
\pgfpathlineto{\pgfqpoint{3.480122in}{2.266823in}}%
\pgfpathlineto{\pgfqpoint{3.493225in}{2.256592in}}%
\pgfpathlineto{\pgfqpoint{3.506330in}{2.246510in}}%
\pgfpathlineto{\pgfqpoint{3.519436in}{2.236574in}}%
\pgfpathlineto{\pgfqpoint{3.527243in}{2.243739in}}%
\pgfpathlineto{\pgfqpoint{3.535044in}{2.250983in}}%
\pgfpathlineto{\pgfqpoint{3.542838in}{2.258304in}}%
\pgfpathlineto{\pgfqpoint{3.550625in}{2.265701in}}%
\pgfpathlineto{\pgfqpoint{3.537538in}{2.275451in}}%
\pgfpathlineto{\pgfqpoint{3.524451in}{2.285349in}}%
\pgfpathlineto{\pgfqpoint{3.511366in}{2.295394in}}%
\pgfpathlineto{\pgfqpoint{3.498282in}{2.305588in}}%
\pgfpathlineto{\pgfqpoint{3.490476in}{2.298370in}}%
\pgfpathlineto{\pgfqpoint{3.482664in}{2.291232in}}%
\pgfpathlineto{\pgfqpoint{3.474845in}{2.284176in}}%
\pgfpathlineto{\pgfqpoint{3.467019in}{2.277201in}}%
\pgfpathclose%
\pgfusepath{fill}%
\end{pgfscope}%
\begin{pgfscope}%
\pgfpathrectangle{\pgfqpoint{1.254980in}{0.150000in}}{\pgfqpoint{5.490039in}{5.490039in}}%
\pgfusepath{clip}%
\pgfsetbuttcap%
\pgfsetroundjoin%
\definecolor{currentfill}{rgb}{0.281924,0.089666,0.412415}%
\pgfsetfillcolor{currentfill}%
\pgfsetfillopacity{0.700000}%
\pgfsetlinewidth{0.000000pt}%
\definecolor{currentstroke}{rgb}{0.000000,0.000000,0.000000}%
\pgfsetstrokecolor{currentstroke}%
\pgfsetdash{}{0pt}%
\pgfpathmoveto{\pgfqpoint{4.062324in}{2.136425in}}%
\pgfpathlineto{\pgfqpoint{4.075480in}{2.131650in}}%
\pgfpathlineto{\pgfqpoint{4.088642in}{2.127003in}}%
\pgfpathlineto{\pgfqpoint{4.101808in}{2.122485in}}%
\pgfpathlineto{\pgfqpoint{4.114981in}{2.118095in}}%
\pgfpathlineto{\pgfqpoint{4.122567in}{2.127500in}}%
\pgfpathlineto{\pgfqpoint{4.130147in}{2.136931in}}%
\pgfpathlineto{\pgfqpoint{4.137723in}{2.146387in}}%
\pgfpathlineto{\pgfqpoint{4.145293in}{2.155867in}}%
\pgfpathlineto{\pgfqpoint{4.132132in}{2.160140in}}%
\pgfpathlineto{\pgfqpoint{4.118976in}{2.164540in}}%
\pgfpathlineto{\pgfqpoint{4.105826in}{2.169068in}}%
\pgfpathlineto{\pgfqpoint{4.092681in}{2.173725in}}%
\pgfpathlineto{\pgfqpoint{4.085100in}{2.164357in}}%
\pgfpathlineto{\pgfqpoint{4.077513in}{2.155018in}}%
\pgfpathlineto{\pgfqpoint{4.069921in}{2.145707in}}%
\pgfpathlineto{\pgfqpoint{4.062324in}{2.136425in}}%
\pgfpathclose%
\pgfusepath{fill}%
\end{pgfscope}%
\begin{pgfscope}%
\pgfpathrectangle{\pgfqpoint{1.254980in}{0.150000in}}{\pgfqpoint{5.490039in}{5.490039in}}%
\pgfusepath{clip}%
\pgfsetbuttcap%
\pgfsetroundjoin%
\definecolor{currentfill}{rgb}{0.212395,0.359683,0.551710}%
\pgfsetfillcolor{currentfill}%
\pgfsetfillopacity{0.700000}%
\pgfsetlinewidth{0.000000pt}%
\definecolor{currentstroke}{rgb}{0.000000,0.000000,0.000000}%
\pgfsetstrokecolor{currentstroke}%
\pgfsetdash{}{0pt}%
\pgfpathmoveto{\pgfqpoint{2.962502in}{2.736698in}}%
\pgfpathlineto{\pgfqpoint{2.975694in}{2.720220in}}%
\pgfpathlineto{\pgfqpoint{2.988881in}{2.703922in}}%
\pgfpathlineto{\pgfqpoint{3.002063in}{2.687804in}}%
\pgfpathlineto{\pgfqpoint{3.015242in}{2.671865in}}%
\pgfpathlineto{\pgfqpoint{3.023281in}{2.676800in}}%
\pgfpathlineto{\pgfqpoint{3.031311in}{2.681857in}}%
\pgfpathlineto{\pgfqpoint{3.039332in}{2.687034in}}%
\pgfpathlineto{\pgfqpoint{3.047344in}{2.692329in}}%
\pgfpathlineto{\pgfqpoint{3.034192in}{2.708059in}}%
\pgfpathlineto{\pgfqpoint{3.021036in}{2.723966in}}%
\pgfpathlineto{\pgfqpoint{3.007875in}{2.740052in}}%
\pgfpathlineto{\pgfqpoint{2.994710in}{2.756319in}}%
\pgfpathlineto{\pgfqpoint{2.986672in}{2.751228in}}%
\pgfpathlineto{\pgfqpoint{2.978625in}{2.746260in}}%
\pgfpathlineto{\pgfqpoint{2.970568in}{2.741416in}}%
\pgfpathlineto{\pgfqpoint{2.962502in}{2.736698in}}%
\pgfpathclose%
\pgfusepath{fill}%
\end{pgfscope}%
\begin{pgfscope}%
\pgfpathrectangle{\pgfqpoint{1.254980in}{0.150000in}}{\pgfqpoint{5.490039in}{5.490039in}}%
\pgfusepath{clip}%
\pgfsetbuttcap%
\pgfsetroundjoin%
\definecolor{currentfill}{rgb}{0.248629,0.278775,0.534556}%
\pgfsetfillcolor{currentfill}%
\pgfsetfillopacity{0.700000}%
\pgfsetlinewidth{0.000000pt}%
\definecolor{currentstroke}{rgb}{0.000000,0.000000,0.000000}%
\pgfsetstrokecolor{currentstroke}%
\pgfsetdash{}{0pt}%
\pgfpathmoveto{\pgfqpoint{3.120539in}{2.550646in}}%
\pgfpathlineto{\pgfqpoint{3.133687in}{2.536266in}}%
\pgfpathlineto{\pgfqpoint{3.146832in}{2.522053in}}%
\pgfpathlineto{\pgfqpoint{3.159975in}{2.508008in}}%
\pgfpathlineto{\pgfqpoint{3.173115in}{2.494130in}}%
\pgfpathlineto{\pgfqpoint{3.181079in}{2.499709in}}%
\pgfpathlineto{\pgfqpoint{3.189035in}{2.505397in}}%
\pgfpathlineto{\pgfqpoint{3.196982in}{2.511192in}}%
\pgfpathlineto{\pgfqpoint{3.204922in}{2.517093in}}%
\pgfpathlineto{\pgfqpoint{3.191805in}{2.530765in}}%
\pgfpathlineto{\pgfqpoint{3.178686in}{2.544603in}}%
\pgfpathlineto{\pgfqpoint{3.165564in}{2.558608in}}%
\pgfpathlineto{\pgfqpoint{3.152440in}{2.572781in}}%
\pgfpathlineto{\pgfqpoint{3.144478in}{2.567081in}}%
\pgfpathlineto{\pgfqpoint{3.136507in}{2.561490in}}%
\pgfpathlineto{\pgfqpoint{3.128527in}{2.556012in}}%
\pgfpathlineto{\pgfqpoint{3.120539in}{2.550646in}}%
\pgfpathclose%
\pgfusepath{fill}%
\end{pgfscope}%
\begin{pgfscope}%
\pgfpathrectangle{\pgfqpoint{1.254980in}{0.150000in}}{\pgfqpoint{5.490039in}{5.490039in}}%
\pgfusepath{clip}%
\pgfsetbuttcap%
\pgfsetroundjoin%
\definecolor{currentfill}{rgb}{0.282910,0.105393,0.426902}%
\pgfsetfillcolor{currentfill}%
\pgfsetfillopacity{0.700000}%
\pgfsetlinewidth{0.000000pt}%
\definecolor{currentstroke}{rgb}{0.000000,0.000000,0.000000}%
\pgfsetstrokecolor{currentstroke}%
\pgfsetdash{}{0pt}%
\pgfpathmoveto{\pgfqpoint{4.280957in}{2.165216in}}%
\pgfpathlineto{\pgfqpoint{4.294166in}{2.162180in}}%
\pgfpathlineto{\pgfqpoint{4.307381in}{2.159269in}}%
\pgfpathlineto{\pgfqpoint{4.320604in}{2.156482in}}%
\pgfpathlineto{\pgfqpoint{4.333833in}{2.153818in}}%
\pgfpathlineto{\pgfqpoint{4.341349in}{2.163686in}}%
\pgfpathlineto{\pgfqpoint{4.348860in}{2.173563in}}%
\pgfpathlineto{\pgfqpoint{4.356367in}{2.183448in}}%
\pgfpathlineto{\pgfqpoint{4.363868in}{2.193341in}}%
\pgfpathlineto{\pgfqpoint{4.350648in}{2.195918in}}%
\pgfpathlineto{\pgfqpoint{4.337435in}{2.198619in}}%
\pgfpathlineto{\pgfqpoint{4.324229in}{2.201444in}}%
\pgfpathlineto{\pgfqpoint{4.311030in}{2.204393in}}%
\pgfpathlineto{\pgfqpoint{4.303519in}{2.194581in}}%
\pgfpathlineto{\pgfqpoint{4.296003in}{2.184780in}}%
\pgfpathlineto{\pgfqpoint{4.288482in}{2.174992in}}%
\pgfpathlineto{\pgfqpoint{4.280957in}{2.165216in}}%
\pgfpathclose%
\pgfusepath{fill}%
\end{pgfscope}%
\begin{pgfscope}%
\pgfpathrectangle{\pgfqpoint{1.254980in}{0.150000in}}{\pgfqpoint{5.490039in}{5.490039in}}%
\pgfusepath{clip}%
\pgfsetbuttcap%
\pgfsetroundjoin%
\definecolor{currentfill}{rgb}{0.147607,0.511733,0.557049}%
\pgfsetfillcolor{currentfill}%
\pgfsetfillopacity{0.700000}%
\pgfsetlinewidth{0.000000pt}%
\definecolor{currentstroke}{rgb}{0.000000,0.000000,0.000000}%
\pgfsetstrokecolor{currentstroke}%
\pgfsetdash{}{0pt}%
\pgfpathmoveto{\pgfqpoint{6.023299in}{3.079237in}}%
\pgfpathlineto{\pgfqpoint{6.037221in}{3.084310in}}%
\pgfpathlineto{\pgfqpoint{6.051159in}{3.089493in}}%
\pgfpathlineto{\pgfqpoint{6.065112in}{3.094786in}}%
\pgfpathlineto{\pgfqpoint{6.079080in}{3.100187in}}%
\pgfpathlineto{\pgfqpoint{6.085936in}{3.107368in}}%
\pgfpathlineto{\pgfqpoint{6.092788in}{3.114557in}}%
\pgfpathlineto{\pgfqpoint{6.099636in}{3.121757in}}%
\pgfpathlineto{\pgfqpoint{6.085683in}{3.116547in}}%
\pgfpathlineto{\pgfqpoint{6.071745in}{3.111445in}}%
\pgfpathlineto{\pgfqpoint{6.057822in}{3.106452in}}%
\pgfpathlineto{\pgfqpoint{6.043914in}{3.101569in}}%
\pgfpathlineto{\pgfqpoint{6.037047in}{3.094111in}}%
\pgfpathlineto{\pgfqpoint{6.030175in}{3.086667in}}%
\pgfpathlineto{\pgfqpoint{6.023299in}{3.079237in}}%
\pgfpathclose%
\pgfusepath{fill}%
\end{pgfscope}%
\begin{pgfscope}%
\pgfpathrectangle{\pgfqpoint{1.254980in}{0.150000in}}{\pgfqpoint{5.490039in}{5.490039in}}%
\pgfusepath{clip}%
\pgfsetbuttcap%
\pgfsetroundjoin%
\definecolor{currentfill}{rgb}{0.199430,0.387607,0.554642}%
\pgfsetfillcolor{currentfill}%
\pgfsetfillopacity{0.700000}%
\pgfsetlinewidth{0.000000pt}%
\definecolor{currentstroke}{rgb}{0.000000,0.000000,0.000000}%
\pgfsetstrokecolor{currentstroke}%
\pgfsetdash{}{0pt}%
\pgfpathmoveto{\pgfqpoint{2.909688in}{2.804439in}}%
\pgfpathlineto{\pgfqpoint{2.922899in}{2.787227in}}%
\pgfpathlineto{\pgfqpoint{2.936105in}{2.770200in}}%
\pgfpathlineto{\pgfqpoint{2.949306in}{2.753358in}}%
\pgfpathlineto{\pgfqpoint{2.962502in}{2.736698in}}%
\pgfpathlineto{\pgfqpoint{2.970568in}{2.741416in}}%
\pgfpathlineto{\pgfqpoint{2.978625in}{2.746260in}}%
\pgfpathlineto{\pgfqpoint{2.986672in}{2.751228in}}%
\pgfpathlineto{\pgfqpoint{2.994710in}{2.756319in}}%
\pgfpathlineto{\pgfqpoint{2.981541in}{2.772767in}}%
\pgfpathlineto{\pgfqpoint{2.968367in}{2.789398in}}%
\pgfpathlineto{\pgfqpoint{2.955189in}{2.806212in}}%
\pgfpathlineto{\pgfqpoint{2.942005in}{2.823211in}}%
\pgfpathlineto{\pgfqpoint{2.933941in}{2.818326in}}%
\pgfpathlineto{\pgfqpoint{2.925866in}{2.813568in}}%
\pgfpathlineto{\pgfqpoint{2.917782in}{2.808939in}}%
\pgfpathlineto{\pgfqpoint{2.909688in}{2.804439in}}%
\pgfpathclose%
\pgfusepath{fill}%
\end{pgfscope}%
\begin{pgfscope}%
\pgfpathrectangle{\pgfqpoint{1.254980in}{0.150000in}}{\pgfqpoint{5.490039in}{5.490039in}}%
\pgfusepath{clip}%
\pgfsetbuttcap%
\pgfsetroundjoin%
\definecolor{currentfill}{rgb}{0.257322,0.256130,0.526563}%
\pgfsetfillcolor{currentfill}%
\pgfsetfillopacity{0.700000}%
\pgfsetlinewidth{0.000000pt}%
\definecolor{currentstroke}{rgb}{0.000000,0.000000,0.000000}%
\pgfsetstrokecolor{currentstroke}%
\pgfsetdash{}{0pt}%
\pgfpathmoveto{\pgfqpoint{3.173115in}{2.494130in}}%
\pgfpathlineto{\pgfqpoint{3.186253in}{2.480417in}}%
\pgfpathlineto{\pgfqpoint{3.199390in}{2.466868in}}%
\pgfpathlineto{\pgfqpoint{3.212524in}{2.453483in}}%
\pgfpathlineto{\pgfqpoint{3.225656in}{2.440261in}}%
\pgfpathlineto{\pgfqpoint{3.233597in}{2.446052in}}%
\pgfpathlineto{\pgfqpoint{3.241530in}{2.451948in}}%
\pgfpathlineto{\pgfqpoint{3.249454in}{2.457947in}}%
\pgfpathlineto{\pgfqpoint{3.257370in}{2.464048in}}%
\pgfpathlineto{\pgfqpoint{3.244261in}{2.477064in}}%
\pgfpathlineto{\pgfqpoint{3.231149in}{2.490244in}}%
\pgfpathlineto{\pgfqpoint{3.218036in}{2.503586in}}%
\pgfpathlineto{\pgfqpoint{3.204922in}{2.517093in}}%
\pgfpathlineto{\pgfqpoint{3.196982in}{2.511192in}}%
\pgfpathlineto{\pgfqpoint{3.189035in}{2.505397in}}%
\pgfpathlineto{\pgfqpoint{3.181079in}{2.499709in}}%
\pgfpathlineto{\pgfqpoint{3.173115in}{2.494130in}}%
\pgfpathclose%
\pgfusepath{fill}%
\end{pgfscope}%
\begin{pgfscope}%
\pgfpathrectangle{\pgfqpoint{1.254980in}{0.150000in}}{\pgfqpoint{5.490039in}{5.490039in}}%
\pgfusepath{clip}%
\pgfsetbuttcap%
\pgfsetroundjoin%
\definecolor{currentfill}{rgb}{0.263663,0.237631,0.518762}%
\pgfsetfillcolor{currentfill}%
\pgfsetfillopacity{0.700000}%
\pgfsetlinewidth{0.000000pt}%
\definecolor{currentstroke}{rgb}{0.000000,0.000000,0.000000}%
\pgfsetstrokecolor{currentstroke}%
\pgfsetdash{}{0pt}%
\pgfpathmoveto{\pgfqpoint{4.914595in}{2.414640in}}%
\pgfpathlineto{\pgfqpoint{4.928023in}{2.415714in}}%
\pgfpathlineto{\pgfqpoint{4.941462in}{2.416905in}}%
\pgfpathlineto{\pgfqpoint{4.954912in}{2.418211in}}%
\pgfpathlineto{\pgfqpoint{4.968373in}{2.419633in}}%
\pgfpathlineto{\pgfqpoint{4.975686in}{2.429530in}}%
\pgfpathlineto{\pgfqpoint{4.982995in}{2.439404in}}%
\pgfpathlineto{\pgfqpoint{4.990298in}{2.449256in}}%
\pgfpathlineto{\pgfqpoint{4.997597in}{2.459086in}}%
\pgfpathlineto{\pgfqpoint{4.984145in}{2.457689in}}%
\pgfpathlineto{\pgfqpoint{4.970704in}{2.456409in}}%
\pgfpathlineto{\pgfqpoint{4.957273in}{2.455244in}}%
\pgfpathlineto{\pgfqpoint{4.943853in}{2.454195in}}%
\pgfpathlineto{\pgfqpoint{4.936546in}{2.444334in}}%
\pgfpathlineto{\pgfqpoint{4.929234in}{2.434455in}}%
\pgfpathlineto{\pgfqpoint{4.921917in}{2.424557in}}%
\pgfpathlineto{\pgfqpoint{4.914595in}{2.414640in}}%
\pgfpathclose%
\pgfusepath{fill}%
\end{pgfscope}%
\begin{pgfscope}%
\pgfpathrectangle{\pgfqpoint{1.254980in}{0.150000in}}{\pgfqpoint{5.490039in}{5.490039in}}%
\pgfusepath{clip}%
\pgfsetbuttcap%
\pgfsetroundjoin%
\definecolor{currentfill}{rgb}{0.255645,0.260703,0.528312}%
\pgfsetfillcolor{currentfill}%
\pgfsetfillopacity{0.700000}%
\pgfsetlinewidth{0.000000pt}%
\definecolor{currentstroke}{rgb}{0.000000,0.000000,0.000000}%
\pgfsetstrokecolor{currentstroke}%
\pgfsetdash{}{0pt}%
\pgfpathmoveto{\pgfqpoint{4.997597in}{2.459086in}}%
\pgfpathlineto{\pgfqpoint{5.011060in}{2.460598in}}%
\pgfpathlineto{\pgfqpoint{5.024534in}{2.462226in}}%
\pgfpathlineto{\pgfqpoint{5.038019in}{2.463969in}}%
\pgfpathlineto{\pgfqpoint{5.051515in}{2.465828in}}%
\pgfpathlineto{\pgfqpoint{5.058800in}{2.475601in}}%
\pgfpathlineto{\pgfqpoint{5.066079in}{2.485349in}}%
\pgfpathlineto{\pgfqpoint{5.073354in}{2.495073in}}%
\pgfpathlineto{\pgfqpoint{5.080624in}{2.504773in}}%
\pgfpathlineto{\pgfqpoint{5.067136in}{2.502957in}}%
\pgfpathlineto{\pgfqpoint{5.053660in}{2.501255in}}%
\pgfpathlineto{\pgfqpoint{5.040195in}{2.499669in}}%
\pgfpathlineto{\pgfqpoint{5.026741in}{2.498198in}}%
\pgfpathlineto{\pgfqpoint{5.019463in}{2.488450in}}%
\pgfpathlineto{\pgfqpoint{5.012179in}{2.478682in}}%
\pgfpathlineto{\pgfqpoint{5.004890in}{2.468894in}}%
\pgfpathlineto{\pgfqpoint{4.997597in}{2.459086in}}%
\pgfpathclose%
\pgfusepath{fill}%
\end{pgfscope}%
\begin{pgfscope}%
\pgfpathrectangle{\pgfqpoint{1.254980in}{0.150000in}}{\pgfqpoint{5.490039in}{5.490039in}}%
\pgfusepath{clip}%
\pgfsetbuttcap%
\pgfsetroundjoin%
\definecolor{currentfill}{rgb}{0.270595,0.214069,0.507052}%
\pgfsetfillcolor{currentfill}%
\pgfsetfillopacity{0.700000}%
\pgfsetlinewidth{0.000000pt}%
\definecolor{currentstroke}{rgb}{0.000000,0.000000,0.000000}%
\pgfsetstrokecolor{currentstroke}%
\pgfsetdash{}{0pt}%
\pgfpathmoveto{\pgfqpoint{4.831613in}{2.371607in}}%
\pgfpathlineto{\pgfqpoint{4.845009in}{2.372223in}}%
\pgfpathlineto{\pgfqpoint{4.858415in}{2.372957in}}%
\pgfpathlineto{\pgfqpoint{4.871831in}{2.373808in}}%
\pgfpathlineto{\pgfqpoint{4.885257in}{2.374775in}}%
\pgfpathlineto{\pgfqpoint{4.892599in}{2.384772in}}%
\pgfpathlineto{\pgfqpoint{4.899936in}{2.394748in}}%
\pgfpathlineto{\pgfqpoint{4.907268in}{2.404704in}}%
\pgfpathlineto{\pgfqpoint{4.914595in}{2.414640in}}%
\pgfpathlineto{\pgfqpoint{4.901176in}{2.413683in}}%
\pgfpathlineto{\pgfqpoint{4.887769in}{2.412841in}}%
\pgfpathlineto{\pgfqpoint{4.874371in}{2.412117in}}%
\pgfpathlineto{\pgfqpoint{4.860984in}{2.411510in}}%
\pgfpathlineto{\pgfqpoint{4.853649in}{2.401558in}}%
\pgfpathlineto{\pgfqpoint{4.846308in}{2.391591in}}%
\pgfpathlineto{\pgfqpoint{4.838963in}{2.381607in}}%
\pgfpathlineto{\pgfqpoint{4.831613in}{2.371607in}}%
\pgfpathclose%
\pgfusepath{fill}%
\end{pgfscope}%
\begin{pgfscope}%
\pgfpathrectangle{\pgfqpoint{1.254980in}{0.150000in}}{\pgfqpoint{5.490039in}{5.490039in}}%
\pgfusepath{clip}%
\pgfsetbuttcap%
\pgfsetroundjoin%
\definecolor{currentfill}{rgb}{0.187231,0.414746,0.556547}%
\pgfsetfillcolor{currentfill}%
\pgfsetfillopacity{0.700000}%
\pgfsetlinewidth{0.000000pt}%
\definecolor{currentstroke}{rgb}{0.000000,0.000000,0.000000}%
\pgfsetstrokecolor{currentstroke}%
\pgfsetdash{}{0pt}%
\pgfpathmoveto{\pgfqpoint{2.856790in}{2.875163in}}%
\pgfpathlineto{\pgfqpoint{2.870023in}{2.857198in}}%
\pgfpathlineto{\pgfqpoint{2.883250in}{2.839424in}}%
\pgfpathlineto{\pgfqpoint{2.896472in}{2.821838in}}%
\pgfpathlineto{\pgfqpoint{2.909688in}{2.804439in}}%
\pgfpathlineto{\pgfqpoint{2.917782in}{2.808939in}}%
\pgfpathlineto{\pgfqpoint{2.925866in}{2.813568in}}%
\pgfpathlineto{\pgfqpoint{2.933941in}{2.818326in}}%
\pgfpathlineto{\pgfqpoint{2.942005in}{2.823211in}}%
\pgfpathlineto{\pgfqpoint{2.928817in}{2.840397in}}%
\pgfpathlineto{\pgfqpoint{2.915623in}{2.857769in}}%
\pgfpathlineto{\pgfqpoint{2.902424in}{2.875330in}}%
\pgfpathlineto{\pgfqpoint{2.889219in}{2.893080in}}%
\pgfpathlineto{\pgfqpoint{2.881127in}{2.888402in}}%
\pgfpathlineto{\pgfqpoint{2.873025in}{2.883856in}}%
\pgfpathlineto{\pgfqpoint{2.864912in}{2.879442in}}%
\pgfpathlineto{\pgfqpoint{2.856790in}{2.875163in}}%
\pgfpathclose%
\pgfusepath{fill}%
\end{pgfscope}%
\begin{pgfscope}%
\pgfpathrectangle{\pgfqpoint{1.254980in}{0.150000in}}{\pgfqpoint{5.490039in}{5.490039in}}%
\pgfusepath{clip}%
\pgfsetbuttcap%
\pgfsetroundjoin%
\definecolor{currentfill}{rgb}{0.248629,0.278775,0.534556}%
\pgfsetfillcolor{currentfill}%
\pgfsetfillopacity{0.700000}%
\pgfsetlinewidth{0.000000pt}%
\definecolor{currentstroke}{rgb}{0.000000,0.000000,0.000000}%
\pgfsetstrokecolor{currentstroke}%
\pgfsetdash{}{0pt}%
\pgfpathmoveto{\pgfqpoint{5.080624in}{2.504773in}}%
\pgfpathlineto{\pgfqpoint{5.094122in}{2.506705in}}%
\pgfpathlineto{\pgfqpoint{5.107632in}{2.508752in}}%
\pgfpathlineto{\pgfqpoint{5.121154in}{2.510913in}}%
\pgfpathlineto{\pgfqpoint{5.134687in}{2.513189in}}%
\pgfpathlineto{\pgfqpoint{5.141942in}{2.522815in}}%
\pgfpathlineto{\pgfqpoint{5.149192in}{2.532415in}}%
\pgfpathlineto{\pgfqpoint{5.156437in}{2.541990in}}%
\pgfpathlineto{\pgfqpoint{5.163677in}{2.551541in}}%
\pgfpathlineto{\pgfqpoint{5.150154in}{2.549323in}}%
\pgfpathlineto{\pgfqpoint{5.136641in}{2.547220in}}%
\pgfpathlineto{\pgfqpoint{5.123141in}{2.545231in}}%
\pgfpathlineto{\pgfqpoint{5.109651in}{2.543357in}}%
\pgfpathlineto{\pgfqpoint{5.102402in}{2.533743in}}%
\pgfpathlineto{\pgfqpoint{5.095148in}{2.524108in}}%
\pgfpathlineto{\pgfqpoint{5.087888in}{2.514451in}}%
\pgfpathlineto{\pgfqpoint{5.080624in}{2.504773in}}%
\pgfpathclose%
\pgfusepath{fill}%
\end{pgfscope}%
\begin{pgfscope}%
\pgfpathrectangle{\pgfqpoint{1.254980in}{0.150000in}}{\pgfqpoint{5.490039in}{5.490039in}}%
\pgfusepath{clip}%
\pgfsetbuttcap%
\pgfsetroundjoin%
\definecolor{currentfill}{rgb}{0.275191,0.194905,0.496005}%
\pgfsetfillcolor{currentfill}%
\pgfsetfillopacity{0.700000}%
\pgfsetlinewidth{0.000000pt}%
\definecolor{currentstroke}{rgb}{0.000000,0.000000,0.000000}%
\pgfsetstrokecolor{currentstroke}%
\pgfsetdash{}{0pt}%
\pgfpathmoveto{\pgfqpoint{4.748648in}{2.330166in}}%
\pgfpathlineto{\pgfqpoint{4.762013in}{2.330306in}}%
\pgfpathlineto{\pgfqpoint{4.775387in}{2.330564in}}%
\pgfpathlineto{\pgfqpoint{4.788771in}{2.330939in}}%
\pgfpathlineto{\pgfqpoint{4.802164in}{2.331432in}}%
\pgfpathlineto{\pgfqpoint{4.809534in}{2.341502in}}%
\pgfpathlineto{\pgfqpoint{4.816899in}{2.351554in}}%
\pgfpathlineto{\pgfqpoint{4.824258in}{2.361589in}}%
\pgfpathlineto{\pgfqpoint{4.831613in}{2.371607in}}%
\pgfpathlineto{\pgfqpoint{4.818228in}{2.371107in}}%
\pgfpathlineto{\pgfqpoint{4.804852in}{2.370725in}}%
\pgfpathlineto{\pgfqpoint{4.791486in}{2.370461in}}%
\pgfpathlineto{\pgfqpoint{4.778130in}{2.370314in}}%
\pgfpathlineto{\pgfqpoint{4.770767in}{2.360297in}}%
\pgfpathlineto{\pgfqpoint{4.763399in}{2.350267in}}%
\pgfpathlineto{\pgfqpoint{4.756026in}{2.340224in}}%
\pgfpathlineto{\pgfqpoint{4.748648in}{2.330166in}}%
\pgfpathclose%
\pgfusepath{fill}%
\end{pgfscope}%
\begin{pgfscope}%
\pgfpathrectangle{\pgfqpoint{1.254980in}{0.150000in}}{\pgfqpoint{5.490039in}{5.490039in}}%
\pgfusepath{clip}%
\pgfsetbuttcap%
\pgfsetroundjoin%
\definecolor{currentfill}{rgb}{0.265145,0.232956,0.516599}%
\pgfsetfillcolor{currentfill}%
\pgfsetfillopacity{0.700000}%
\pgfsetlinewidth{0.000000pt}%
\definecolor{currentstroke}{rgb}{0.000000,0.000000,0.000000}%
\pgfsetstrokecolor{currentstroke}%
\pgfsetdash{}{0pt}%
\pgfpathmoveto{\pgfqpoint{3.225656in}{2.440261in}}%
\pgfpathlineto{\pgfqpoint{3.238787in}{2.427201in}}%
\pgfpathlineto{\pgfqpoint{3.251916in}{2.414302in}}%
\pgfpathlineto{\pgfqpoint{3.265044in}{2.401562in}}%
\pgfpathlineto{\pgfqpoint{3.278171in}{2.388983in}}%
\pgfpathlineto{\pgfqpoint{3.286089in}{2.394985in}}%
\pgfpathlineto{\pgfqpoint{3.293999in}{2.401087in}}%
\pgfpathlineto{\pgfqpoint{3.301901in}{2.407289in}}%
\pgfpathlineto{\pgfqpoint{3.309796in}{2.413589in}}%
\pgfpathlineto{\pgfqpoint{3.296691in}{2.425964in}}%
\pgfpathlineto{\pgfqpoint{3.283586in}{2.438498in}}%
\pgfpathlineto{\pgfqpoint{3.270479in}{2.451193in}}%
\pgfpathlineto{\pgfqpoint{3.257370in}{2.464048in}}%
\pgfpathlineto{\pgfqpoint{3.249454in}{2.457947in}}%
\pgfpathlineto{\pgfqpoint{3.241530in}{2.451948in}}%
\pgfpathlineto{\pgfqpoint{3.233597in}{2.446052in}}%
\pgfpathlineto{\pgfqpoint{3.225656in}{2.440261in}}%
\pgfpathclose%
\pgfusepath{fill}%
\end{pgfscope}%
\begin{pgfscope}%
\pgfpathrectangle{\pgfqpoint{1.254980in}{0.150000in}}{\pgfqpoint{5.490039in}{5.490039in}}%
\pgfusepath{clip}%
\pgfsetbuttcap%
\pgfsetroundjoin%
\definecolor{currentfill}{rgb}{0.239346,0.300855,0.540844}%
\pgfsetfillcolor{currentfill}%
\pgfsetfillopacity{0.700000}%
\pgfsetlinewidth{0.000000pt}%
\definecolor{currentstroke}{rgb}{0.000000,0.000000,0.000000}%
\pgfsetstrokecolor{currentstroke}%
\pgfsetdash{}{0pt}%
\pgfpathmoveto{\pgfqpoint{5.163677in}{2.551541in}}%
\pgfpathlineto{\pgfqpoint{5.177213in}{2.553873in}}%
\pgfpathlineto{\pgfqpoint{5.190760in}{2.556319in}}%
\pgfpathlineto{\pgfqpoint{5.204319in}{2.558880in}}%
\pgfpathlineto{\pgfqpoint{5.217890in}{2.561554in}}%
\pgfpathlineto{\pgfqpoint{5.225115in}{2.571013in}}%
\pgfpathlineto{\pgfqpoint{5.232335in}{2.580446in}}%
\pgfpathlineto{\pgfqpoint{5.239550in}{2.589853in}}%
\pgfpathlineto{\pgfqpoint{5.246760in}{2.599235in}}%
\pgfpathlineto{\pgfqpoint{5.233198in}{2.596635in}}%
\pgfpathlineto{\pgfqpoint{5.219649in}{2.594149in}}%
\pgfpathlineto{\pgfqpoint{5.206112in}{2.591777in}}%
\pgfpathlineto{\pgfqpoint{5.192586in}{2.589519in}}%
\pgfpathlineto{\pgfqpoint{5.185366in}{2.580056in}}%
\pgfpathlineto{\pgfqpoint{5.178142in}{2.570573in}}%
\pgfpathlineto{\pgfqpoint{5.170912in}{2.561068in}}%
\pgfpathlineto{\pgfqpoint{5.163677in}{2.551541in}}%
\pgfpathclose%
\pgfusepath{fill}%
\end{pgfscope}%
\begin{pgfscope}%
\pgfpathrectangle{\pgfqpoint{1.254980in}{0.150000in}}{\pgfqpoint{5.490039in}{5.490039in}}%
\pgfusepath{clip}%
\pgfsetbuttcap%
\pgfsetroundjoin%
\definecolor{currentfill}{rgb}{0.278826,0.175490,0.483397}%
\pgfsetfillcolor{currentfill}%
\pgfsetfillopacity{0.700000}%
\pgfsetlinewidth{0.000000pt}%
\definecolor{currentstroke}{rgb}{0.000000,0.000000,0.000000}%
\pgfsetstrokecolor{currentstroke}%
\pgfsetdash{}{0pt}%
\pgfpathmoveto{\pgfqpoint{4.665694in}{2.290507in}}%
\pgfpathlineto{\pgfqpoint{4.679029in}{2.290151in}}%
\pgfpathlineto{\pgfqpoint{4.692373in}{2.289913in}}%
\pgfpathlineto{\pgfqpoint{4.705727in}{2.289794in}}%
\pgfpathlineto{\pgfqpoint{4.719089in}{2.289794in}}%
\pgfpathlineto{\pgfqpoint{4.726486in}{2.299908in}}%
\pgfpathlineto{\pgfqpoint{4.733879in}{2.310008in}}%
\pgfpathlineto{\pgfqpoint{4.741266in}{2.320094in}}%
\pgfpathlineto{\pgfqpoint{4.748648in}{2.330166in}}%
\pgfpathlineto{\pgfqpoint{4.735294in}{2.330144in}}%
\pgfpathlineto{\pgfqpoint{4.721948in}{2.330240in}}%
\pgfpathlineto{\pgfqpoint{4.708613in}{2.330455in}}%
\pgfpathlineto{\pgfqpoint{4.695286in}{2.330789in}}%
\pgfpathlineto{\pgfqpoint{4.687895in}{2.320734in}}%
\pgfpathlineto{\pgfqpoint{4.680500in}{2.310669in}}%
\pgfpathlineto{\pgfqpoint{4.673099in}{2.300593in}}%
\pgfpathlineto{\pgfqpoint{4.665694in}{2.290507in}}%
\pgfpathclose%
\pgfusepath{fill}%
\end{pgfscope}%
\begin{pgfscope}%
\pgfpathrectangle{\pgfqpoint{1.254980in}{0.150000in}}{\pgfqpoint{5.490039in}{5.490039in}}%
\pgfusepath{clip}%
\pgfsetbuttcap%
\pgfsetroundjoin%
\definecolor{currentfill}{rgb}{0.229739,0.322361,0.545706}%
\pgfsetfillcolor{currentfill}%
\pgfsetfillopacity{0.700000}%
\pgfsetlinewidth{0.000000pt}%
\definecolor{currentstroke}{rgb}{0.000000,0.000000,0.000000}%
\pgfsetstrokecolor{currentstroke}%
\pgfsetdash{}{0pt}%
\pgfpathmoveto{\pgfqpoint{5.246760in}{2.599235in}}%
\pgfpathlineto{\pgfqpoint{5.260333in}{2.601949in}}%
\pgfpathlineto{\pgfqpoint{5.273919in}{2.604777in}}%
\pgfpathlineto{\pgfqpoint{5.287516in}{2.607718in}}%
\pgfpathlineto{\pgfqpoint{5.301126in}{2.610772in}}%
\pgfpathlineto{\pgfqpoint{5.308321in}{2.620046in}}%
\pgfpathlineto{\pgfqpoint{5.315510in}{2.629294in}}%
\pgfpathlineto{\pgfqpoint{5.322693in}{2.638516in}}%
\pgfpathlineto{\pgfqpoint{5.329872in}{2.647714in}}%
\pgfpathlineto{\pgfqpoint{5.316272in}{2.644751in}}%
\pgfpathlineto{\pgfqpoint{5.302684in}{2.641901in}}%
\pgfpathlineto{\pgfqpoint{5.289109in}{2.639164in}}%
\pgfpathlineto{\pgfqpoint{5.275546in}{2.636540in}}%
\pgfpathlineto{\pgfqpoint{5.268357in}{2.627246in}}%
\pgfpathlineto{\pgfqpoint{5.261163in}{2.617931in}}%
\pgfpathlineto{\pgfqpoint{5.253964in}{2.608594in}}%
\pgfpathlineto{\pgfqpoint{5.246760in}{2.599235in}}%
\pgfpathclose%
\pgfusepath{fill}%
\end{pgfscope}%
\begin{pgfscope}%
\pgfpathrectangle{\pgfqpoint{1.254980in}{0.150000in}}{\pgfqpoint{5.490039in}{5.490039in}}%
\pgfusepath{clip}%
\pgfsetbuttcap%
\pgfsetroundjoin%
\definecolor{currentfill}{rgb}{0.281924,0.089666,0.412415}%
\pgfsetfillcolor{currentfill}%
\pgfsetfillopacity{0.700000}%
\pgfsetlinewidth{0.000000pt}%
\definecolor{currentstroke}{rgb}{0.000000,0.000000,0.000000}%
\pgfsetstrokecolor{currentstroke}%
\pgfsetdash{}{0pt}%
\pgfpathmoveto{\pgfqpoint{3.843575in}{2.134255in}}%
\pgfpathlineto{\pgfqpoint{3.856701in}{2.127616in}}%
\pgfpathlineto{\pgfqpoint{3.869830in}{2.121111in}}%
\pgfpathlineto{\pgfqpoint{3.882964in}{2.114740in}}%
\pgfpathlineto{\pgfqpoint{3.896101in}{2.108502in}}%
\pgfpathlineto{\pgfqpoint{3.903765in}{2.117204in}}%
\pgfpathlineto{\pgfqpoint{3.911424in}{2.125950in}}%
\pgfpathlineto{\pgfqpoint{3.919077in}{2.134740in}}%
\pgfpathlineto{\pgfqpoint{3.926725in}{2.143573in}}%
\pgfpathlineto{\pgfqpoint{3.913600in}{2.149660in}}%
\pgfpathlineto{\pgfqpoint{3.900480in}{2.155881in}}%
\pgfpathlineto{\pgfqpoint{3.887364in}{2.162235in}}%
\pgfpathlineto{\pgfqpoint{3.874253in}{2.168723in}}%
\pgfpathlineto{\pgfqpoint{3.866592in}{2.160035in}}%
\pgfpathlineto{\pgfqpoint{3.858925in}{2.151393in}}%
\pgfpathlineto{\pgfqpoint{3.851253in}{2.142800in}}%
\pgfpathlineto{\pgfqpoint{3.843575in}{2.134255in}}%
\pgfpathclose%
\pgfusepath{fill}%
\end{pgfscope}%
\begin{pgfscope}%
\pgfpathrectangle{\pgfqpoint{1.254980in}{0.150000in}}{\pgfqpoint{5.490039in}{5.490039in}}%
\pgfusepath{clip}%
\pgfsetbuttcap%
\pgfsetroundjoin%
\definecolor{currentfill}{rgb}{0.282884,0.135920,0.453427}%
\pgfsetfillcolor{currentfill}%
\pgfsetfillopacity{0.700000}%
\pgfsetlinewidth{0.000000pt}%
\definecolor{currentstroke}{rgb}{0.000000,0.000000,0.000000}%
\pgfsetstrokecolor{currentstroke}%
\pgfsetdash{}{0pt}%
\pgfpathmoveto{\pgfqpoint{3.519436in}{2.236574in}}%
\pgfpathlineto{\pgfqpoint{3.532543in}{2.226785in}}%
\pgfpathlineto{\pgfqpoint{3.545651in}{2.217142in}}%
\pgfpathlineto{\pgfqpoint{3.558760in}{2.207643in}}%
\pgfpathlineto{\pgfqpoint{3.571871in}{2.198290in}}%
\pgfpathlineto{\pgfqpoint{3.579661in}{2.205646in}}%
\pgfpathlineto{\pgfqpoint{3.587443in}{2.213076in}}%
\pgfpathlineto{\pgfqpoint{3.595220in}{2.220579in}}%
\pgfpathlineto{\pgfqpoint{3.602989in}{2.228155in}}%
\pgfpathlineto{\pgfqpoint{3.589896in}{2.237324in}}%
\pgfpathlineto{\pgfqpoint{3.576804in}{2.246638in}}%
\pgfpathlineto{\pgfqpoint{3.563714in}{2.256097in}}%
\pgfpathlineto{\pgfqpoint{3.550625in}{2.265701in}}%
\pgfpathlineto{\pgfqpoint{3.542838in}{2.258304in}}%
\pgfpathlineto{\pgfqpoint{3.535044in}{2.250983in}}%
\pgfpathlineto{\pgfqpoint{3.527243in}{2.243739in}}%
\pgfpathlineto{\pgfqpoint{3.519436in}{2.236574in}}%
\pgfpathclose%
\pgfusepath{fill}%
\end{pgfscope}%
\begin{pgfscope}%
\pgfpathrectangle{\pgfqpoint{1.254980in}{0.150000in}}{\pgfqpoint{5.490039in}{5.490039in}}%
\pgfusepath{clip}%
\pgfsetbuttcap%
\pgfsetroundjoin%
\definecolor{currentfill}{rgb}{0.282327,0.094955,0.417331}%
\pgfsetfillcolor{currentfill}%
\pgfsetfillopacity{0.700000}%
\pgfsetlinewidth{0.000000pt}%
\definecolor{currentstroke}{rgb}{0.000000,0.000000,0.000000}%
\pgfsetstrokecolor{currentstroke}%
\pgfsetdash{}{0pt}%
\pgfpathmoveto{\pgfqpoint{4.197999in}{2.140050in}}%
\pgfpathlineto{\pgfqpoint{4.211191in}{2.136412in}}%
\pgfpathlineto{\pgfqpoint{4.224390in}{2.132899in}}%
\pgfpathlineto{\pgfqpoint{4.237595in}{2.129512in}}%
\pgfpathlineto{\pgfqpoint{4.250807in}{2.126250in}}%
\pgfpathlineto{\pgfqpoint{4.258352in}{2.135969in}}%
\pgfpathlineto{\pgfqpoint{4.265892in}{2.145704in}}%
\pgfpathlineto{\pgfqpoint{4.273427in}{2.155453in}}%
\pgfpathlineto{\pgfqpoint{4.280957in}{2.165216in}}%
\pgfpathlineto{\pgfqpoint{4.267755in}{2.168376in}}%
\pgfpathlineto{\pgfqpoint{4.254560in}{2.171661in}}%
\pgfpathlineto{\pgfqpoint{4.241372in}{2.175072in}}%
\pgfpathlineto{\pgfqpoint{4.228190in}{2.178608in}}%
\pgfpathlineto{\pgfqpoint{4.220650in}{2.168941in}}%
\pgfpathlineto{\pgfqpoint{4.213104in}{2.159292in}}%
\pgfpathlineto{\pgfqpoint{4.205554in}{2.149662in}}%
\pgfpathlineto{\pgfqpoint{4.197999in}{2.140050in}}%
\pgfpathclose%
\pgfusepath{fill}%
\end{pgfscope}%
\begin{pgfscope}%
\pgfpathrectangle{\pgfqpoint{1.254980in}{0.150000in}}{\pgfqpoint{5.490039in}{5.490039in}}%
\pgfusepath{clip}%
\pgfsetbuttcap%
\pgfsetroundjoin%
\definecolor{currentfill}{rgb}{0.174274,0.445044,0.557792}%
\pgfsetfillcolor{currentfill}%
\pgfsetfillopacity{0.700000}%
\pgfsetlinewidth{0.000000pt}%
\definecolor{currentstroke}{rgb}{0.000000,0.000000,0.000000}%
\pgfsetstrokecolor{currentstroke}%
\pgfsetdash{}{0pt}%
\pgfpathmoveto{\pgfqpoint{2.803796in}{2.948948in}}%
\pgfpathlineto{\pgfqpoint{2.817054in}{2.930210in}}%
\pgfpathlineto{\pgfqpoint{2.830306in}{2.911668in}}%
\pgfpathlineto{\pgfqpoint{2.843551in}{2.893319in}}%
\pgfpathlineto{\pgfqpoint{2.856790in}{2.875163in}}%
\pgfpathlineto{\pgfqpoint{2.864912in}{2.879442in}}%
\pgfpathlineto{\pgfqpoint{2.873025in}{2.883856in}}%
\pgfpathlineto{\pgfqpoint{2.881127in}{2.888402in}}%
\pgfpathlineto{\pgfqpoint{2.889219in}{2.893080in}}%
\pgfpathlineto{\pgfqpoint{2.876009in}{2.911021in}}%
\pgfpathlineto{\pgfqpoint{2.862793in}{2.929155in}}%
\pgfpathlineto{\pgfqpoint{2.849570in}{2.947482in}}%
\pgfpathlineto{\pgfqpoint{2.836342in}{2.966003in}}%
\pgfpathlineto{\pgfqpoint{2.828221in}{2.961534in}}%
\pgfpathlineto{\pgfqpoint{2.820090in}{2.957201in}}%
\pgfpathlineto{\pgfqpoint{2.811948in}{2.953005in}}%
\pgfpathlineto{\pgfqpoint{2.803796in}{2.948948in}}%
\pgfpathclose%
\pgfusepath{fill}%
\end{pgfscope}%
\begin{pgfscope}%
\pgfpathrectangle{\pgfqpoint{1.254980in}{0.150000in}}{\pgfqpoint{5.490039in}{5.490039in}}%
\pgfusepath{clip}%
\pgfsetbuttcap%
\pgfsetroundjoin%
\definecolor{currentfill}{rgb}{0.281412,0.155834,0.469201}%
\pgfsetfillcolor{currentfill}%
\pgfsetfillopacity{0.700000}%
\pgfsetlinewidth{0.000000pt}%
\definecolor{currentstroke}{rgb}{0.000000,0.000000,0.000000}%
\pgfsetstrokecolor{currentstroke}%
\pgfsetdash{}{0pt}%
\pgfpathmoveto{\pgfqpoint{4.582744in}{2.252829in}}%
\pgfpathlineto{\pgfqpoint{4.596052in}{2.251957in}}%
\pgfpathlineto{\pgfqpoint{4.609367in}{2.251205in}}%
\pgfpathlineto{\pgfqpoint{4.622692in}{2.250572in}}%
\pgfpathlineto{\pgfqpoint{4.636026in}{2.250058in}}%
\pgfpathlineto{\pgfqpoint{4.643450in}{2.260186in}}%
\pgfpathlineto{\pgfqpoint{4.650870in}{2.270304in}}%
\pgfpathlineto{\pgfqpoint{4.658284in}{2.280411in}}%
\pgfpathlineto{\pgfqpoint{4.665694in}{2.290507in}}%
\pgfpathlineto{\pgfqpoint{4.652369in}{2.290982in}}%
\pgfpathlineto{\pgfqpoint{4.639052in}{2.291577in}}%
\pgfpathlineto{\pgfqpoint{4.625745in}{2.292290in}}%
\pgfpathlineto{\pgfqpoint{4.612446in}{2.293124in}}%
\pgfpathlineto{\pgfqpoint{4.605028in}{2.283060in}}%
\pgfpathlineto{\pgfqpoint{4.597605in}{2.272990in}}%
\pgfpathlineto{\pgfqpoint{4.590177in}{2.262913in}}%
\pgfpathlineto{\pgfqpoint{4.582744in}{2.252829in}}%
\pgfpathclose%
\pgfusepath{fill}%
\end{pgfscope}%
\begin{pgfscope}%
\pgfpathrectangle{\pgfqpoint{1.254980in}{0.150000in}}{\pgfqpoint{5.490039in}{5.490039in}}%
\pgfusepath{clip}%
\pgfsetbuttcap%
\pgfsetroundjoin%
\definecolor{currentfill}{rgb}{0.218130,0.347432,0.550038}%
\pgfsetfillcolor{currentfill}%
\pgfsetfillopacity{0.700000}%
\pgfsetlinewidth{0.000000pt}%
\definecolor{currentstroke}{rgb}{0.000000,0.000000,0.000000}%
\pgfsetstrokecolor{currentstroke}%
\pgfsetdash{}{0pt}%
\pgfpathmoveto{\pgfqpoint{5.329872in}{2.647714in}}%
\pgfpathlineto{\pgfqpoint{5.343484in}{2.650791in}}%
\pgfpathlineto{\pgfqpoint{5.357108in}{2.653981in}}%
\pgfpathlineto{\pgfqpoint{5.370746in}{2.657284in}}%
\pgfpathlineto{\pgfqpoint{5.384396in}{2.660699in}}%
\pgfpathlineto{\pgfqpoint{5.391558in}{2.669773in}}%
\pgfpathlineto{\pgfqpoint{5.398715in}{2.678821in}}%
\pgfpathlineto{\pgfqpoint{5.405867in}{2.687844in}}%
\pgfpathlineto{\pgfqpoint{5.413013in}{2.696843in}}%
\pgfpathlineto{\pgfqpoint{5.399374in}{2.693535in}}%
\pgfpathlineto{\pgfqpoint{5.385748in}{2.690339in}}%
\pgfpathlineto{\pgfqpoint{5.372134in}{2.687256in}}%
\pgfpathlineto{\pgfqpoint{5.358533in}{2.684286in}}%
\pgfpathlineto{\pgfqpoint{5.351375in}{2.675174in}}%
\pgfpathlineto{\pgfqpoint{5.344213in}{2.666042in}}%
\pgfpathlineto{\pgfqpoint{5.337045in}{2.656889in}}%
\pgfpathlineto{\pgfqpoint{5.329872in}{2.647714in}}%
\pgfpathclose%
\pgfusepath{fill}%
\end{pgfscope}%
\begin{pgfscope}%
\pgfpathrectangle{\pgfqpoint{1.254980in}{0.150000in}}{\pgfqpoint{5.490039in}{5.490039in}}%
\pgfusepath{clip}%
\pgfsetbuttcap%
\pgfsetroundjoin%
\definecolor{currentfill}{rgb}{0.282656,0.100196,0.422160}%
\pgfsetfillcolor{currentfill}%
\pgfsetfillopacity{0.700000}%
\pgfsetlinewidth{0.000000pt}%
\definecolor{currentstroke}{rgb}{0.000000,0.000000,0.000000}%
\pgfsetstrokecolor{currentstroke}%
\pgfsetdash{}{0pt}%
\pgfpathmoveto{\pgfqpoint{3.707811in}{2.159921in}}%
\pgfpathlineto{\pgfqpoint{3.720925in}{2.152023in}}%
\pgfpathlineto{\pgfqpoint{3.734042in}{2.144264in}}%
\pgfpathlineto{\pgfqpoint{3.747161in}{2.136642in}}%
\pgfpathlineto{\pgfqpoint{3.760284in}{2.129158in}}%
\pgfpathlineto{\pgfqpoint{3.767999in}{2.137324in}}%
\pgfpathlineto{\pgfqpoint{3.775709in}{2.145548in}}%
\pgfpathlineto{\pgfqpoint{3.783412in}{2.153827in}}%
\pgfpathlineto{\pgfqpoint{3.791110in}{2.162162in}}%
\pgfpathlineto{\pgfqpoint{3.778002in}{2.169479in}}%
\pgfpathlineto{\pgfqpoint{3.764898in}{2.176934in}}%
\pgfpathlineto{\pgfqpoint{3.751796in}{2.184526in}}%
\pgfpathlineto{\pgfqpoint{3.738698in}{2.192257in}}%
\pgfpathlineto{\pgfqpoint{3.730985in}{2.184083in}}%
\pgfpathlineto{\pgfqpoint{3.723267in}{2.175969in}}%
\pgfpathlineto{\pgfqpoint{3.715542in}{2.167914in}}%
\pgfpathlineto{\pgfqpoint{3.707811in}{2.159921in}}%
\pgfpathclose%
\pgfusepath{fill}%
\end{pgfscope}%
\begin{pgfscope}%
\pgfpathrectangle{\pgfqpoint{1.254980in}{0.150000in}}{\pgfqpoint{5.490039in}{5.490039in}}%
\pgfusepath{clip}%
\pgfsetbuttcap%
\pgfsetroundjoin%
\definecolor{currentfill}{rgb}{0.281446,0.084320,0.407414}%
\pgfsetfillcolor{currentfill}%
\pgfsetfillopacity{0.700000}%
\pgfsetlinewidth{0.000000pt}%
\definecolor{currentstroke}{rgb}{0.000000,0.000000,0.000000}%
\pgfsetstrokecolor{currentstroke}%
\pgfsetdash{}{0pt}%
\pgfpathmoveto{\pgfqpoint{3.979267in}{2.120545in}}%
\pgfpathlineto{\pgfqpoint{3.992414in}{2.115116in}}%
\pgfpathlineto{\pgfqpoint{4.005566in}{2.109818in}}%
\pgfpathlineto{\pgfqpoint{4.018723in}{2.104650in}}%
\pgfpathlineto{\pgfqpoint{4.031886in}{2.099612in}}%
\pgfpathlineto{\pgfqpoint{4.039503in}{2.108766in}}%
\pgfpathlineto{\pgfqpoint{4.047115in}{2.117954in}}%
\pgfpathlineto{\pgfqpoint{4.054722in}{2.127174in}}%
\pgfpathlineto{\pgfqpoint{4.062324in}{2.136425in}}%
\pgfpathlineto{\pgfqpoint{4.049174in}{2.141330in}}%
\pgfpathlineto{\pgfqpoint{4.036029in}{2.146364in}}%
\pgfpathlineto{\pgfqpoint{4.022888in}{2.151528in}}%
\pgfpathlineto{\pgfqpoint{4.009753in}{2.156822in}}%
\pgfpathlineto{\pgfqpoint{4.002140in}{2.147699in}}%
\pgfpathlineto{\pgfqpoint{3.994521in}{2.138612in}}%
\pgfpathlineto{\pgfqpoint{3.986896in}{2.129560in}}%
\pgfpathlineto{\pgfqpoint{3.979267in}{2.120545in}}%
\pgfpathclose%
\pgfusepath{fill}%
\end{pgfscope}%
\begin{pgfscope}%
\pgfpathrectangle{\pgfqpoint{1.254980in}{0.150000in}}{\pgfqpoint{5.490039in}{5.490039in}}%
\pgfusepath{clip}%
\pgfsetbuttcap%
\pgfsetroundjoin%
\definecolor{currentfill}{rgb}{0.271828,0.209303,0.504434}%
\pgfsetfillcolor{currentfill}%
\pgfsetfillopacity{0.700000}%
\pgfsetlinewidth{0.000000pt}%
\definecolor{currentstroke}{rgb}{0.000000,0.000000,0.000000}%
\pgfsetstrokecolor{currentstroke}%
\pgfsetdash{}{0pt}%
\pgfpathmoveto{\pgfqpoint{3.278171in}{2.388983in}}%
\pgfpathlineto{\pgfqpoint{3.291296in}{2.376561in}}%
\pgfpathlineto{\pgfqpoint{3.304421in}{2.364298in}}%
\pgfpathlineto{\pgfqpoint{3.317545in}{2.352191in}}%
\pgfpathlineto{\pgfqpoint{3.330668in}{2.340240in}}%
\pgfpathlineto{\pgfqpoint{3.338564in}{2.346452in}}%
\pgfpathlineto{\pgfqpoint{3.346453in}{2.352761in}}%
\pgfpathlineto{\pgfqpoint{3.354334in}{2.359164in}}%
\pgfpathlineto{\pgfqpoint{3.362207in}{2.365661in}}%
\pgfpathlineto{\pgfqpoint{3.349105in}{2.377409in}}%
\pgfpathlineto{\pgfqpoint{3.336003in}{2.389312in}}%
\pgfpathlineto{\pgfqpoint{3.322900in}{2.401371in}}%
\pgfpathlineto{\pgfqpoint{3.309796in}{2.413589in}}%
\pgfpathlineto{\pgfqpoint{3.301901in}{2.407289in}}%
\pgfpathlineto{\pgfqpoint{3.293999in}{2.401087in}}%
\pgfpathlineto{\pgfqpoint{3.286089in}{2.394985in}}%
\pgfpathlineto{\pgfqpoint{3.278171in}{2.388983in}}%
\pgfpathclose%
\pgfusepath{fill}%
\end{pgfscope}%
\begin{pgfscope}%
\pgfpathrectangle{\pgfqpoint{1.254980in}{0.150000in}}{\pgfqpoint{5.490039in}{5.490039in}}%
\pgfusepath{clip}%
\pgfsetbuttcap%
\pgfsetroundjoin%
\definecolor{currentfill}{rgb}{0.208623,0.367752,0.552675}%
\pgfsetfillcolor{currentfill}%
\pgfsetfillopacity{0.700000}%
\pgfsetlinewidth{0.000000pt}%
\definecolor{currentstroke}{rgb}{0.000000,0.000000,0.000000}%
\pgfsetstrokecolor{currentstroke}%
\pgfsetdash{}{0pt}%
\pgfpathmoveto{\pgfqpoint{5.413013in}{2.696843in}}%
\pgfpathlineto{\pgfqpoint{5.426665in}{2.700264in}}%
\pgfpathlineto{\pgfqpoint{5.440330in}{2.703798in}}%
\pgfpathlineto{\pgfqpoint{5.454008in}{2.707444in}}%
\pgfpathlineto{\pgfqpoint{5.467698in}{2.711203in}}%
\pgfpathlineto{\pgfqpoint{5.474828in}{2.720062in}}%
\pgfpathlineto{\pgfqpoint{5.481952in}{2.728897in}}%
\pgfpathlineto{\pgfqpoint{5.489071in}{2.737708in}}%
\pgfpathlineto{\pgfqpoint{5.496185in}{2.746497in}}%
\pgfpathlineto{\pgfqpoint{5.482506in}{2.742862in}}%
\pgfpathlineto{\pgfqpoint{5.468840in}{2.739340in}}%
\pgfpathlineto{\pgfqpoint{5.455187in}{2.735930in}}%
\pgfpathlineto{\pgfqpoint{5.441547in}{2.732632in}}%
\pgfpathlineto{\pgfqpoint{5.434421in}{2.723713in}}%
\pgfpathlineto{\pgfqpoint{5.427290in}{2.714777in}}%
\pgfpathlineto{\pgfqpoint{5.420154in}{2.705820in}}%
\pgfpathlineto{\pgfqpoint{5.413013in}{2.696843in}}%
\pgfpathclose%
\pgfusepath{fill}%
\end{pgfscope}%
\begin{pgfscope}%
\pgfpathrectangle{\pgfqpoint{1.254980in}{0.150000in}}{\pgfqpoint{5.490039in}{5.490039in}}%
\pgfusepath{clip}%
\pgfsetbuttcap%
\pgfsetroundjoin%
\definecolor{currentfill}{rgb}{0.282623,0.140926,0.457517}%
\pgfsetfillcolor{currentfill}%
\pgfsetfillopacity{0.700000}%
\pgfsetlinewidth{0.000000pt}%
\definecolor{currentstroke}{rgb}{0.000000,0.000000,0.000000}%
\pgfsetstrokecolor{currentstroke}%
\pgfsetdash{}{0pt}%
\pgfpathmoveto{\pgfqpoint{4.499791in}{2.217341in}}%
\pgfpathlineto{\pgfqpoint{4.513072in}{2.215934in}}%
\pgfpathlineto{\pgfqpoint{4.526362in}{2.214646in}}%
\pgfpathlineto{\pgfqpoint{4.539660in}{2.213480in}}%
\pgfpathlineto{\pgfqpoint{4.552967in}{2.212433in}}%
\pgfpathlineto{\pgfqpoint{4.560419in}{2.222541in}}%
\pgfpathlineto{\pgfqpoint{4.567865in}{2.232644in}}%
\pgfpathlineto{\pgfqpoint{4.575307in}{2.242740in}}%
\pgfpathlineto{\pgfqpoint{4.582744in}{2.252829in}}%
\pgfpathlineto{\pgfqpoint{4.569446in}{2.253821in}}%
\pgfpathlineto{\pgfqpoint{4.556156in}{2.254934in}}%
\pgfpathlineto{\pgfqpoint{4.542875in}{2.256166in}}%
\pgfpathlineto{\pgfqpoint{4.529602in}{2.257520in}}%
\pgfpathlineto{\pgfqpoint{4.522156in}{2.247478in}}%
\pgfpathlineto{\pgfqpoint{4.514706in}{2.237435in}}%
\pgfpathlineto{\pgfqpoint{4.507251in}{2.227389in}}%
\pgfpathlineto{\pgfqpoint{4.499791in}{2.217341in}}%
\pgfpathclose%
\pgfusepath{fill}%
\end{pgfscope}%
\begin{pgfscope}%
\pgfpathrectangle{\pgfqpoint{1.254980in}{0.150000in}}{\pgfqpoint{5.490039in}{5.490039in}}%
\pgfusepath{clip}%
\pgfsetbuttcap%
\pgfsetroundjoin%
\definecolor{currentfill}{rgb}{0.199430,0.387607,0.554642}%
\pgfsetfillcolor{currentfill}%
\pgfsetfillopacity{0.700000}%
\pgfsetlinewidth{0.000000pt}%
\definecolor{currentstroke}{rgb}{0.000000,0.000000,0.000000}%
\pgfsetstrokecolor{currentstroke}%
\pgfsetdash{}{0pt}%
\pgfpathmoveto{\pgfqpoint{5.496185in}{2.746497in}}%
\pgfpathlineto{\pgfqpoint{5.509877in}{2.750244in}}%
\pgfpathlineto{\pgfqpoint{5.523582in}{2.754103in}}%
\pgfpathlineto{\pgfqpoint{5.537301in}{2.758074in}}%
\pgfpathlineto{\pgfqpoint{5.551033in}{2.762156in}}%
\pgfpathlineto{\pgfqpoint{5.558129in}{2.770790in}}%
\pgfpathlineto{\pgfqpoint{5.565219in}{2.779401in}}%
\pgfpathlineto{\pgfqpoint{5.572304in}{2.787990in}}%
\pgfpathlineto{\pgfqpoint{5.579384in}{2.796559in}}%
\pgfpathlineto{\pgfqpoint{5.565665in}{2.792617in}}%
\pgfpathlineto{\pgfqpoint{5.551959in}{2.788786in}}%
\pgfpathlineto{\pgfqpoint{5.538266in}{2.785067in}}%
\pgfpathlineto{\pgfqpoint{5.524586in}{2.781460in}}%
\pgfpathlineto{\pgfqpoint{5.517494in}{2.772745in}}%
\pgfpathlineto{\pgfqpoint{5.510396in}{2.764014in}}%
\pgfpathlineto{\pgfqpoint{5.503293in}{2.755265in}}%
\pgfpathlineto{\pgfqpoint{5.496185in}{2.746497in}}%
\pgfpathclose%
\pgfusepath{fill}%
\end{pgfscope}%
\begin{pgfscope}%
\pgfpathrectangle{\pgfqpoint{1.254980in}{0.150000in}}{\pgfqpoint{5.490039in}{5.490039in}}%
\pgfusepath{clip}%
\pgfsetbuttcap%
\pgfsetroundjoin%
\definecolor{currentfill}{rgb}{0.162142,0.474838,0.558140}%
\pgfsetfillcolor{currentfill}%
\pgfsetfillopacity{0.700000}%
\pgfsetlinewidth{0.000000pt}%
\definecolor{currentstroke}{rgb}{0.000000,0.000000,0.000000}%
\pgfsetstrokecolor{currentstroke}%
\pgfsetdash{}{0pt}%
\pgfpathmoveto{\pgfqpoint{2.750695in}{3.025876in}}%
\pgfpathlineto{\pgfqpoint{2.763981in}{3.006345in}}%
\pgfpathlineto{\pgfqpoint{2.777260in}{2.987014in}}%
\pgfpathlineto{\pgfqpoint{2.790531in}{2.967882in}}%
\pgfpathlineto{\pgfqpoint{2.803796in}{2.948948in}}%
\pgfpathlineto{\pgfqpoint{2.811948in}{2.953005in}}%
\pgfpathlineto{\pgfqpoint{2.820090in}{2.957201in}}%
\pgfpathlineto{\pgfqpoint{2.828221in}{2.961534in}}%
\pgfpathlineto{\pgfqpoint{2.836342in}{2.966003in}}%
\pgfpathlineto{\pgfqpoint{2.823107in}{2.984720in}}%
\pgfpathlineto{\pgfqpoint{2.809865in}{3.003635in}}%
\pgfpathlineto{\pgfqpoint{2.796616in}{3.022749in}}%
\pgfpathlineto{\pgfqpoint{2.783361in}{3.042062in}}%
\pgfpathlineto{\pgfqpoint{2.775211in}{3.037804in}}%
\pgfpathlineto{\pgfqpoint{2.767050in}{3.033686in}}%
\pgfpathlineto{\pgfqpoint{2.758878in}{3.029709in}}%
\pgfpathlineto{\pgfqpoint{2.750695in}{3.025876in}}%
\pgfpathclose%
\pgfusepath{fill}%
\end{pgfscope}%
\begin{pgfscope}%
\pgfpathrectangle{\pgfqpoint{1.254980in}{0.150000in}}{\pgfqpoint{5.490039in}{5.490039in}}%
\pgfusepath{clip}%
\pgfsetbuttcap%
\pgfsetroundjoin%
\definecolor{currentfill}{rgb}{0.188923,0.410910,0.556326}%
\pgfsetfillcolor{currentfill}%
\pgfsetfillopacity{0.700000}%
\pgfsetlinewidth{0.000000pt}%
\definecolor{currentstroke}{rgb}{0.000000,0.000000,0.000000}%
\pgfsetstrokecolor{currentstroke}%
\pgfsetdash{}{0pt}%
\pgfpathmoveto{\pgfqpoint{5.579384in}{2.796559in}}%
\pgfpathlineto{\pgfqpoint{5.593117in}{2.800613in}}%
\pgfpathlineto{\pgfqpoint{5.606864in}{2.804779in}}%
\pgfpathlineto{\pgfqpoint{5.620624in}{2.809057in}}%
\pgfpathlineto{\pgfqpoint{5.634398in}{2.813445in}}%
\pgfpathlineto{\pgfqpoint{5.641459in}{2.821845in}}%
\pgfpathlineto{\pgfqpoint{5.648515in}{2.830223in}}%
\pgfpathlineto{\pgfqpoint{5.655566in}{2.838583in}}%
\pgfpathlineto{\pgfqpoint{5.662611in}{2.846924in}}%
\pgfpathlineto{\pgfqpoint{5.648851in}{2.842692in}}%
\pgfpathlineto{\pgfqpoint{5.635104in}{2.838572in}}%
\pgfpathlineto{\pgfqpoint{5.621371in}{2.834563in}}%
\pgfpathlineto{\pgfqpoint{5.607651in}{2.830665in}}%
\pgfpathlineto{\pgfqpoint{5.600592in}{2.822161in}}%
\pgfpathlineto{\pgfqpoint{5.593528in}{2.813643in}}%
\pgfpathlineto{\pgfqpoint{5.586459in}{2.805110in}}%
\pgfpathlineto{\pgfqpoint{5.579384in}{2.796559in}}%
\pgfpathclose%
\pgfusepath{fill}%
\end{pgfscope}%
\begin{pgfscope}%
\pgfpathrectangle{\pgfqpoint{1.254980in}{0.150000in}}{\pgfqpoint{5.490039in}{5.490039in}}%
\pgfusepath{clip}%
\pgfsetbuttcap%
\pgfsetroundjoin%
\definecolor{currentfill}{rgb}{0.276194,0.190074,0.493001}%
\pgfsetfillcolor{currentfill}%
\pgfsetfillopacity{0.700000}%
\pgfsetlinewidth{0.000000pt}%
\definecolor{currentstroke}{rgb}{0.000000,0.000000,0.000000}%
\pgfsetstrokecolor{currentstroke}%
\pgfsetdash{}{0pt}%
\pgfpathmoveto{\pgfqpoint{3.330668in}{2.340240in}}%
\pgfpathlineto{\pgfqpoint{3.343790in}{2.328444in}}%
\pgfpathlineto{\pgfqpoint{3.356912in}{2.316803in}}%
\pgfpathlineto{\pgfqpoint{3.370034in}{2.305316in}}%
\pgfpathlineto{\pgfqpoint{3.383156in}{2.293982in}}%
\pgfpathlineto{\pgfqpoint{3.391031in}{2.300403in}}%
\pgfpathlineto{\pgfqpoint{3.398898in}{2.306917in}}%
\pgfpathlineto{\pgfqpoint{3.406759in}{2.313521in}}%
\pgfpathlineto{\pgfqpoint{3.414612in}{2.320215in}}%
\pgfpathlineto{\pgfqpoint{3.401511in}{2.331347in}}%
\pgfpathlineto{\pgfqpoint{3.388409in}{2.342631in}}%
\pgfpathlineto{\pgfqpoint{3.375308in}{2.354069in}}%
\pgfpathlineto{\pgfqpoint{3.362207in}{2.365661in}}%
\pgfpathlineto{\pgfqpoint{3.354334in}{2.359164in}}%
\pgfpathlineto{\pgfqpoint{3.346453in}{2.352761in}}%
\pgfpathlineto{\pgfqpoint{3.338564in}{2.346452in}}%
\pgfpathlineto{\pgfqpoint{3.330668in}{2.340240in}}%
\pgfpathclose%
\pgfusepath{fill}%
\end{pgfscope}%
\begin{pgfscope}%
\pgfpathrectangle{\pgfqpoint{1.254980in}{0.150000in}}{\pgfqpoint{5.490039in}{5.490039in}}%
\pgfusepath{clip}%
\pgfsetbuttcap%
\pgfsetroundjoin%
\definecolor{currentfill}{rgb}{0.283187,0.125848,0.444960}%
\pgfsetfillcolor{currentfill}%
\pgfsetfillopacity{0.700000}%
\pgfsetlinewidth{0.000000pt}%
\definecolor{currentstroke}{rgb}{0.000000,0.000000,0.000000}%
\pgfsetstrokecolor{currentstroke}%
\pgfsetdash{}{0pt}%
\pgfpathmoveto{\pgfqpoint{4.416824in}{2.184261in}}%
\pgfpathlineto{\pgfqpoint{4.430082in}{2.182298in}}%
\pgfpathlineto{\pgfqpoint{4.443348in}{2.180456in}}%
\pgfpathlineto{\pgfqpoint{4.456622in}{2.178735in}}%
\pgfpathlineto{\pgfqpoint{4.469904in}{2.177136in}}%
\pgfpathlineto{\pgfqpoint{4.477382in}{2.187189in}}%
\pgfpathlineto{\pgfqpoint{4.484857in}{2.197241in}}%
\pgfpathlineto{\pgfqpoint{4.492326in}{2.207292in}}%
\pgfpathlineto{\pgfqpoint{4.499791in}{2.217341in}}%
\pgfpathlineto{\pgfqpoint{4.486518in}{2.218870in}}%
\pgfpathlineto{\pgfqpoint{4.473252in}{2.220520in}}%
\pgfpathlineto{\pgfqpoint{4.459995in}{2.222292in}}%
\pgfpathlineto{\pgfqpoint{4.446746in}{2.224185in}}%
\pgfpathlineto{\pgfqpoint{4.439273in}{2.214200in}}%
\pgfpathlineto{\pgfqpoint{4.431795in}{2.204218in}}%
\pgfpathlineto{\pgfqpoint{4.424312in}{2.194238in}}%
\pgfpathlineto{\pgfqpoint{4.416824in}{2.184261in}}%
\pgfpathclose%
\pgfusepath{fill}%
\end{pgfscope}%
\begin{pgfscope}%
\pgfpathrectangle{\pgfqpoint{1.254980in}{0.150000in}}{\pgfqpoint{5.490039in}{5.490039in}}%
\pgfusepath{clip}%
\pgfsetbuttcap%
\pgfsetroundjoin%
\definecolor{currentfill}{rgb}{0.281446,0.084320,0.407414}%
\pgfsetfillcolor{currentfill}%
\pgfsetfillopacity{0.700000}%
\pgfsetlinewidth{0.000000pt}%
\definecolor{currentstroke}{rgb}{0.000000,0.000000,0.000000}%
\pgfsetstrokecolor{currentstroke}%
\pgfsetdash{}{0pt}%
\pgfpathmoveto{\pgfqpoint{4.114981in}{2.118095in}}%
\pgfpathlineto{\pgfqpoint{4.128159in}{2.113832in}}%
\pgfpathlineto{\pgfqpoint{4.141344in}{2.109696in}}%
\pgfpathlineto{\pgfqpoint{4.154534in}{2.105688in}}%
\pgfpathlineto{\pgfqpoint{4.167730in}{2.101805in}}%
\pgfpathlineto{\pgfqpoint{4.175305in}{2.111335in}}%
\pgfpathlineto{\pgfqpoint{4.182875in}{2.120886in}}%
\pgfpathlineto{\pgfqpoint{4.190439in}{2.130458in}}%
\pgfpathlineto{\pgfqpoint{4.197999in}{2.140050in}}%
\pgfpathlineto{\pgfqpoint{4.184814in}{2.143814in}}%
\pgfpathlineto{\pgfqpoint{4.171634in}{2.147705in}}%
\pgfpathlineto{\pgfqpoint{4.158461in}{2.151723in}}%
\pgfpathlineto{\pgfqpoint{4.145293in}{2.155867in}}%
\pgfpathlineto{\pgfqpoint{4.137723in}{2.146387in}}%
\pgfpathlineto{\pgfqpoint{4.130147in}{2.136931in}}%
\pgfpathlineto{\pgfqpoint{4.122567in}{2.127500in}}%
\pgfpathlineto{\pgfqpoint{4.114981in}{2.118095in}}%
\pgfpathclose%
\pgfusepath{fill}%
\end{pgfscope}%
\begin{pgfscope}%
\pgfpathrectangle{\pgfqpoint{1.254980in}{0.150000in}}{\pgfqpoint{5.490039in}{5.490039in}}%
\pgfusepath{clip}%
\pgfsetbuttcap%
\pgfsetroundjoin%
\definecolor{currentfill}{rgb}{0.283229,0.120777,0.440584}%
\pgfsetfillcolor{currentfill}%
\pgfsetfillopacity{0.700000}%
\pgfsetlinewidth{0.000000pt}%
\definecolor{currentstroke}{rgb}{0.000000,0.000000,0.000000}%
\pgfsetstrokecolor{currentstroke}%
\pgfsetdash{}{0pt}%
\pgfpathmoveto{\pgfqpoint{3.571871in}{2.198290in}}%
\pgfpathlineto{\pgfqpoint{3.584984in}{2.189080in}}%
\pgfpathlineto{\pgfqpoint{3.598098in}{2.180013in}}%
\pgfpathlineto{\pgfqpoint{3.611214in}{2.171089in}}%
\pgfpathlineto{\pgfqpoint{3.624332in}{2.162307in}}%
\pgfpathlineto{\pgfqpoint{3.632104in}{2.169853in}}%
\pgfpathlineto{\pgfqpoint{3.639870in}{2.177469in}}%
\pgfpathlineto{\pgfqpoint{3.647629in}{2.185154in}}%
\pgfpathlineto{\pgfqpoint{3.655382in}{2.192908in}}%
\pgfpathlineto{\pgfqpoint{3.642281in}{2.201506in}}%
\pgfpathlineto{\pgfqpoint{3.629182in}{2.210246in}}%
\pgfpathlineto{\pgfqpoint{3.616085in}{2.219129in}}%
\pgfpathlineto{\pgfqpoint{3.602989in}{2.228155in}}%
\pgfpathlineto{\pgfqpoint{3.595220in}{2.220579in}}%
\pgfpathlineto{\pgfqpoint{3.587443in}{2.213076in}}%
\pgfpathlineto{\pgfqpoint{3.579661in}{2.205646in}}%
\pgfpathlineto{\pgfqpoint{3.571871in}{2.198290in}}%
\pgfpathclose%
\pgfusepath{fill}%
\end{pgfscope}%
\begin{pgfscope}%
\pgfpathrectangle{\pgfqpoint{1.254980in}{0.150000in}}{\pgfqpoint{5.490039in}{5.490039in}}%
\pgfusepath{clip}%
\pgfsetbuttcap%
\pgfsetroundjoin%
\definecolor{currentfill}{rgb}{0.180629,0.429975,0.557282}%
\pgfsetfillcolor{currentfill}%
\pgfsetfillopacity{0.700000}%
\pgfsetlinewidth{0.000000pt}%
\definecolor{currentstroke}{rgb}{0.000000,0.000000,0.000000}%
\pgfsetstrokecolor{currentstroke}%
\pgfsetdash{}{0pt}%
\pgfpathmoveto{\pgfqpoint{5.662611in}{2.846924in}}%
\pgfpathlineto{\pgfqpoint{5.676385in}{2.851267in}}%
\pgfpathlineto{\pgfqpoint{5.690173in}{2.855721in}}%
\pgfpathlineto{\pgfqpoint{5.703975in}{2.860286in}}%
\pgfpathlineto{\pgfqpoint{5.717792in}{2.864963in}}%
\pgfpathlineto{\pgfqpoint{5.724817in}{2.873121in}}%
\pgfpathlineto{\pgfqpoint{5.731838in}{2.881261in}}%
\pgfpathlineto{\pgfqpoint{5.738853in}{2.889384in}}%
\pgfpathlineto{\pgfqpoint{5.745863in}{2.897493in}}%
\pgfpathlineto{\pgfqpoint{5.732061in}{2.892991in}}%
\pgfpathlineto{\pgfqpoint{5.718274in}{2.888599in}}%
\pgfpathlineto{\pgfqpoint{5.704500in}{2.884319in}}%
\pgfpathlineto{\pgfqpoint{5.690740in}{2.880149in}}%
\pgfpathlineto{\pgfqpoint{5.683715in}{2.871860in}}%
\pgfpathlineto{\pgfqpoint{5.676686in}{2.863561in}}%
\pgfpathlineto{\pgfqpoint{5.669651in}{2.855250in}}%
\pgfpathlineto{\pgfqpoint{5.662611in}{2.846924in}}%
\pgfpathclose%
\pgfusepath{fill}%
\end{pgfscope}%
\begin{pgfscope}%
\pgfpathrectangle{\pgfqpoint{1.254980in}{0.150000in}}{\pgfqpoint{5.490039in}{5.490039in}}%
\pgfusepath{clip}%
\pgfsetbuttcap%
\pgfsetroundjoin%
\definecolor{currentfill}{rgb}{0.172719,0.448791,0.557885}%
\pgfsetfillcolor{currentfill}%
\pgfsetfillopacity{0.700000}%
\pgfsetlinewidth{0.000000pt}%
\definecolor{currentstroke}{rgb}{0.000000,0.000000,0.000000}%
\pgfsetstrokecolor{currentstroke}%
\pgfsetdash{}{0pt}%
\pgfpathmoveto{\pgfqpoint{5.745863in}{2.897493in}}%
\pgfpathlineto{\pgfqpoint{5.759679in}{2.902107in}}%
\pgfpathlineto{\pgfqpoint{5.773509in}{2.906831in}}%
\pgfpathlineto{\pgfqpoint{5.787353in}{2.911665in}}%
\pgfpathlineto{\pgfqpoint{5.801212in}{2.916611in}}%
\pgfpathlineto{\pgfqpoint{5.808201in}{2.924523in}}%
\pgfpathlineto{\pgfqpoint{5.815186in}{2.932420in}}%
\pgfpathlineto{\pgfqpoint{5.822165in}{2.940305in}}%
\pgfpathlineto{\pgfqpoint{5.829138in}{2.948179in}}%
\pgfpathlineto{\pgfqpoint{5.815295in}{2.943424in}}%
\pgfpathlineto{\pgfqpoint{5.801466in}{2.938779in}}%
\pgfpathlineto{\pgfqpoint{5.787652in}{2.934246in}}%
\pgfpathlineto{\pgfqpoint{5.773851in}{2.929822in}}%
\pgfpathlineto{\pgfqpoint{5.766862in}{2.921752in}}%
\pgfpathlineto{\pgfqpoint{5.759867in}{2.913675in}}%
\pgfpathlineto{\pgfqpoint{5.752868in}{2.905590in}}%
\pgfpathlineto{\pgfqpoint{5.745863in}{2.897493in}}%
\pgfpathclose%
\pgfusepath{fill}%
\end{pgfscope}%
\begin{pgfscope}%
\pgfpathrectangle{\pgfqpoint{1.254980in}{0.150000in}}{\pgfqpoint{5.490039in}{5.490039in}}%
\pgfusepath{clip}%
\pgfsetbuttcap%
\pgfsetroundjoin%
\definecolor{currentfill}{rgb}{0.283091,0.110553,0.431554}%
\pgfsetfillcolor{currentfill}%
\pgfsetfillopacity{0.700000}%
\pgfsetlinewidth{0.000000pt}%
\definecolor{currentstroke}{rgb}{0.000000,0.000000,0.000000}%
\pgfsetstrokecolor{currentstroke}%
\pgfsetdash{}{0pt}%
\pgfpathmoveto{\pgfqpoint{4.333833in}{2.153818in}}%
\pgfpathlineto{\pgfqpoint{4.347070in}{2.151278in}}%
\pgfpathlineto{\pgfqpoint{4.360315in}{2.148860in}}%
\pgfpathlineto{\pgfqpoint{4.373566in}{2.146566in}}%
\pgfpathlineto{\pgfqpoint{4.386825in}{2.144394in}}%
\pgfpathlineto{\pgfqpoint{4.394332in}{2.154354in}}%
\pgfpathlineto{\pgfqpoint{4.401834in}{2.164319in}}%
\pgfpathlineto{\pgfqpoint{4.409332in}{2.174288in}}%
\pgfpathlineto{\pgfqpoint{4.416824in}{2.184261in}}%
\pgfpathlineto{\pgfqpoint{4.403574in}{2.186347in}}%
\pgfpathlineto{\pgfqpoint{4.390331in}{2.188556in}}%
\pgfpathlineto{\pgfqpoint{4.377096in}{2.190887in}}%
\pgfpathlineto{\pgfqpoint{4.363868in}{2.193341in}}%
\pgfpathlineto{\pgfqpoint{4.356367in}{2.183448in}}%
\pgfpathlineto{\pgfqpoint{4.348860in}{2.173563in}}%
\pgfpathlineto{\pgfqpoint{4.341349in}{2.163686in}}%
\pgfpathlineto{\pgfqpoint{4.333833in}{2.153818in}}%
\pgfpathclose%
\pgfusepath{fill}%
\end{pgfscope}%
\begin{pgfscope}%
\pgfpathrectangle{\pgfqpoint{1.254980in}{0.150000in}}{\pgfqpoint{5.490039in}{5.490039in}}%
\pgfusepath{clip}%
\pgfsetbuttcap%
\pgfsetroundjoin%
\definecolor{currentfill}{rgb}{0.149039,0.508051,0.557250}%
\pgfsetfillcolor{currentfill}%
\pgfsetfillopacity{0.700000}%
\pgfsetlinewidth{0.000000pt}%
\definecolor{currentstroke}{rgb}{0.000000,0.000000,0.000000}%
\pgfsetstrokecolor{currentstroke}%
\pgfsetdash{}{0pt}%
\pgfpathmoveto{\pgfqpoint{2.697477in}{3.106033in}}%
\pgfpathlineto{\pgfqpoint{2.710793in}{3.085686in}}%
\pgfpathlineto{\pgfqpoint{2.724102in}{3.065545in}}%
\pgfpathlineto{\pgfqpoint{2.737402in}{3.045609in}}%
\pgfpathlineto{\pgfqpoint{2.750695in}{3.025876in}}%
\pgfpathlineto{\pgfqpoint{2.758878in}{3.029709in}}%
\pgfpathlineto{\pgfqpoint{2.767050in}{3.033686in}}%
\pgfpathlineto{\pgfqpoint{2.775211in}{3.037804in}}%
\pgfpathlineto{\pgfqpoint{2.783361in}{3.042062in}}%
\pgfpathlineto{\pgfqpoint{2.770098in}{3.061577in}}%
\pgfpathlineto{\pgfqpoint{2.756829in}{3.081294in}}%
\pgfpathlineto{\pgfqpoint{2.743551in}{3.101216in}}%
\pgfpathlineto{\pgfqpoint{2.730266in}{3.121343in}}%
\pgfpathlineto{\pgfqpoint{2.722085in}{3.117297in}}%
\pgfpathlineto{\pgfqpoint{2.713894in}{3.113396in}}%
\pgfpathlineto{\pgfqpoint{2.705691in}{3.109641in}}%
\pgfpathlineto{\pgfqpoint{2.697477in}{3.106033in}}%
\pgfpathclose%
\pgfusepath{fill}%
\end{pgfscope}%
\begin{pgfscope}%
\pgfpathrectangle{\pgfqpoint{1.254980in}{0.150000in}}{\pgfqpoint{5.490039in}{5.490039in}}%
\pgfusepath{clip}%
\pgfsetbuttcap%
\pgfsetroundjoin%
\definecolor{currentfill}{rgb}{0.280894,0.078907,0.402329}%
\pgfsetfillcolor{currentfill}%
\pgfsetfillopacity{0.700000}%
\pgfsetlinewidth{0.000000pt}%
\definecolor{currentstroke}{rgb}{0.000000,0.000000,0.000000}%
\pgfsetstrokecolor{currentstroke}%
\pgfsetdash{}{0pt}%
\pgfpathmoveto{\pgfqpoint{3.896101in}{2.108502in}}%
\pgfpathlineto{\pgfqpoint{3.909243in}{2.102397in}}%
\pgfpathlineto{\pgfqpoint{3.922389in}{2.096424in}}%
\pgfpathlineto{\pgfqpoint{3.935540in}{2.090584in}}%
\pgfpathlineto{\pgfqpoint{3.948695in}{2.084874in}}%
\pgfpathlineto{\pgfqpoint{3.956346in}{2.093731in}}%
\pgfpathlineto{\pgfqpoint{3.963992in}{2.102630in}}%
\pgfpathlineto{\pgfqpoint{3.971632in}{2.111568in}}%
\pgfpathlineto{\pgfqpoint{3.979267in}{2.120545in}}%
\pgfpathlineto{\pgfqpoint{3.966124in}{2.126104in}}%
\pgfpathlineto{\pgfqpoint{3.952987in}{2.131795in}}%
\pgfpathlineto{\pgfqpoint{3.939853in}{2.137618in}}%
\pgfpathlineto{\pgfqpoint{3.926725in}{2.143573in}}%
\pgfpathlineto{\pgfqpoint{3.919077in}{2.134740in}}%
\pgfpathlineto{\pgfqpoint{3.911424in}{2.125950in}}%
\pgfpathlineto{\pgfqpoint{3.903765in}{2.117204in}}%
\pgfpathlineto{\pgfqpoint{3.896101in}{2.108502in}}%
\pgfpathclose%
\pgfusepath{fill}%
\end{pgfscope}%
\begin{pgfscope}%
\pgfpathrectangle{\pgfqpoint{1.254980in}{0.150000in}}{\pgfqpoint{5.490039in}{5.490039in}}%
\pgfusepath{clip}%
\pgfsetbuttcap%
\pgfsetroundjoin%
\definecolor{currentfill}{rgb}{0.279574,0.170599,0.479997}%
\pgfsetfillcolor{currentfill}%
\pgfsetfillopacity{0.700000}%
\pgfsetlinewidth{0.000000pt}%
\definecolor{currentstroke}{rgb}{0.000000,0.000000,0.000000}%
\pgfsetstrokecolor{currentstroke}%
\pgfsetdash{}{0pt}%
\pgfpathmoveto{\pgfqpoint{3.383156in}{2.293982in}}%
\pgfpathlineto{\pgfqpoint{3.396277in}{2.282800in}}%
\pgfpathlineto{\pgfqpoint{3.409399in}{2.271769in}}%
\pgfpathlineto{\pgfqpoint{3.422520in}{2.260889in}}%
\pgfpathlineto{\pgfqpoint{3.435642in}{2.250160in}}%
\pgfpathlineto{\pgfqpoint{3.443497in}{2.256789in}}%
\pgfpathlineto{\pgfqpoint{3.451345in}{2.263507in}}%
\pgfpathlineto{\pgfqpoint{3.459185in}{2.270311in}}%
\pgfpathlineto{\pgfqpoint{3.467019in}{2.277201in}}%
\pgfpathlineto{\pgfqpoint{3.453916in}{2.287729in}}%
\pgfpathlineto{\pgfqpoint{3.440814in}{2.298407in}}%
\pgfpathlineto{\pgfqpoint{3.427713in}{2.309236in}}%
\pgfpathlineto{\pgfqpoint{3.414612in}{2.320215in}}%
\pgfpathlineto{\pgfqpoint{3.406759in}{2.313521in}}%
\pgfpathlineto{\pgfqpoint{3.398898in}{2.306917in}}%
\pgfpathlineto{\pgfqpoint{3.391031in}{2.300403in}}%
\pgfpathlineto{\pgfqpoint{3.383156in}{2.293982in}}%
\pgfpathclose%
\pgfusepath{fill}%
\end{pgfscope}%
\begin{pgfscope}%
\pgfpathrectangle{\pgfqpoint{1.254980in}{0.150000in}}{\pgfqpoint{5.490039in}{5.490039in}}%
\pgfusepath{clip}%
\pgfsetbuttcap%
\pgfsetroundjoin%
\definecolor{currentfill}{rgb}{0.281924,0.089666,0.412415}%
\pgfsetfillcolor{currentfill}%
\pgfsetfillopacity{0.700000}%
\pgfsetlinewidth{0.000000pt}%
\definecolor{currentstroke}{rgb}{0.000000,0.000000,0.000000}%
\pgfsetstrokecolor{currentstroke}%
\pgfsetdash{}{0pt}%
\pgfpathmoveto{\pgfqpoint{3.760284in}{2.129158in}}%
\pgfpathlineto{\pgfqpoint{3.773410in}{2.121811in}}%
\pgfpathlineto{\pgfqpoint{3.786539in}{2.114600in}}%
\pgfpathlineto{\pgfqpoint{3.799672in}{2.107524in}}%
\pgfpathlineto{\pgfqpoint{3.812808in}{2.100584in}}%
\pgfpathlineto{\pgfqpoint{3.820508in}{2.108923in}}%
\pgfpathlineto{\pgfqpoint{3.828203in}{2.117315in}}%
\pgfpathlineto{\pgfqpoint{3.835892in}{2.125759in}}%
\pgfpathlineto{\pgfqpoint{3.843575in}{2.134255in}}%
\pgfpathlineto{\pgfqpoint{3.830454in}{2.141028in}}%
\pgfpathlineto{\pgfqpoint{3.817336in}{2.147937in}}%
\pgfpathlineto{\pgfqpoint{3.804221in}{2.154981in}}%
\pgfpathlineto{\pgfqpoint{3.791110in}{2.162162in}}%
\pgfpathlineto{\pgfqpoint{3.783412in}{2.153827in}}%
\pgfpathlineto{\pgfqpoint{3.775709in}{2.145548in}}%
\pgfpathlineto{\pgfqpoint{3.767999in}{2.137324in}}%
\pgfpathlineto{\pgfqpoint{3.760284in}{2.129158in}}%
\pgfpathclose%
\pgfusepath{fill}%
\end{pgfscope}%
\begin{pgfscope}%
\pgfpathrectangle{\pgfqpoint{1.254980in}{0.150000in}}{\pgfqpoint{5.490039in}{5.490039in}}%
\pgfusepath{clip}%
\pgfsetbuttcap%
\pgfsetroundjoin%
\definecolor{currentfill}{rgb}{0.163625,0.471133,0.558148}%
\pgfsetfillcolor{currentfill}%
\pgfsetfillopacity{0.700000}%
\pgfsetlinewidth{0.000000pt}%
\definecolor{currentstroke}{rgb}{0.000000,0.000000,0.000000}%
\pgfsetstrokecolor{currentstroke}%
\pgfsetdash{}{0pt}%
\pgfpathmoveto{\pgfqpoint{5.829138in}{2.948179in}}%
\pgfpathlineto{\pgfqpoint{5.842996in}{2.953044in}}%
\pgfpathlineto{\pgfqpoint{5.856868in}{2.958019in}}%
\pgfpathlineto{\pgfqpoint{5.870755in}{2.963105in}}%
\pgfpathlineto{\pgfqpoint{5.884656in}{2.968301in}}%
\pgfpathlineto{\pgfqpoint{5.891609in}{2.975965in}}%
\pgfpathlineto{\pgfqpoint{5.898556in}{2.983618in}}%
\pgfpathlineto{\pgfqpoint{5.905498in}{2.991263in}}%
\pgfpathlineto{\pgfqpoint{5.912435in}{2.998901in}}%
\pgfpathlineto{\pgfqpoint{5.898550in}{2.993912in}}%
\pgfpathlineto{\pgfqpoint{5.884680in}{2.989034in}}%
\pgfpathlineto{\pgfqpoint{5.870824in}{2.984265in}}%
\pgfpathlineto{\pgfqpoint{5.856983in}{2.979606in}}%
\pgfpathlineto{\pgfqpoint{5.850029in}{2.971755in}}%
\pgfpathlineto{\pgfqpoint{5.843070in}{2.963902in}}%
\pgfpathlineto{\pgfqpoint{5.836107in}{2.956044in}}%
\pgfpathlineto{\pgfqpoint{5.829138in}{2.948179in}}%
\pgfpathclose%
\pgfusepath{fill}%
\end{pgfscope}%
\begin{pgfscope}%
\pgfpathrectangle{\pgfqpoint{1.254980in}{0.150000in}}{\pgfqpoint{5.490039in}{5.490039in}}%
\pgfusepath{clip}%
\pgfsetbuttcap%
\pgfsetroundjoin%
\definecolor{currentfill}{rgb}{0.156270,0.489624,0.557936}%
\pgfsetfillcolor{currentfill}%
\pgfsetfillopacity{0.700000}%
\pgfsetlinewidth{0.000000pt}%
\definecolor{currentstroke}{rgb}{0.000000,0.000000,0.000000}%
\pgfsetstrokecolor{currentstroke}%
\pgfsetdash{}{0pt}%
\pgfpathmoveto{\pgfqpoint{5.912435in}{2.998901in}}%
\pgfpathlineto{\pgfqpoint{5.926334in}{3.004000in}}%
\pgfpathlineto{\pgfqpoint{5.940248in}{3.009208in}}%
\pgfpathlineto{\pgfqpoint{5.954178in}{3.014527in}}%
\pgfpathlineto{\pgfqpoint{5.968122in}{3.019956in}}%
\pgfpathlineto{\pgfqpoint{5.975036in}{3.027371in}}%
\pgfpathlineto{\pgfqpoint{5.981946in}{3.034781in}}%
\pgfpathlineto{\pgfqpoint{5.988850in}{3.042187in}}%
\pgfpathlineto{\pgfqpoint{5.995749in}{3.049591in}}%
\pgfpathlineto{\pgfqpoint{5.981823in}{3.044386in}}%
\pgfpathlineto{\pgfqpoint{5.967911in}{3.039292in}}%
\pgfpathlineto{\pgfqpoint{5.954015in}{3.034307in}}%
\pgfpathlineto{\pgfqpoint{5.940133in}{3.029431in}}%
\pgfpathlineto{\pgfqpoint{5.933215in}{3.021798in}}%
\pgfpathlineto{\pgfqpoint{5.926293in}{3.014166in}}%
\pgfpathlineto{\pgfqpoint{5.919366in}{3.006535in}}%
\pgfpathlineto{\pgfqpoint{5.912435in}{2.998901in}}%
\pgfpathclose%
\pgfusepath{fill}%
\end{pgfscope}%
\begin{pgfscope}%
\pgfpathrectangle{\pgfqpoint{1.254980in}{0.150000in}}{\pgfqpoint{5.490039in}{5.490039in}}%
\pgfusepath{clip}%
\pgfsetbuttcap%
\pgfsetroundjoin%
\definecolor{currentfill}{rgb}{0.280894,0.078907,0.402329}%
\pgfsetfillcolor{currentfill}%
\pgfsetfillopacity{0.700000}%
\pgfsetlinewidth{0.000000pt}%
\definecolor{currentstroke}{rgb}{0.000000,0.000000,0.000000}%
\pgfsetstrokecolor{currentstroke}%
\pgfsetdash{}{0pt}%
\pgfpathmoveto{\pgfqpoint{4.031886in}{2.099612in}}%
\pgfpathlineto{\pgfqpoint{4.045053in}{2.094702in}}%
\pgfpathlineto{\pgfqpoint{4.058226in}{2.089922in}}%
\pgfpathlineto{\pgfqpoint{4.071404in}{2.085270in}}%
\pgfpathlineto{\pgfqpoint{4.084588in}{2.080746in}}%
\pgfpathlineto{\pgfqpoint{4.092194in}{2.090040in}}%
\pgfpathlineto{\pgfqpoint{4.099795in}{2.099364in}}%
\pgfpathlineto{\pgfqpoint{4.107390in}{2.108716in}}%
\pgfpathlineto{\pgfqpoint{4.114981in}{2.118095in}}%
\pgfpathlineto{\pgfqpoint{4.101808in}{2.122485in}}%
\pgfpathlineto{\pgfqpoint{4.088642in}{2.127003in}}%
\pgfpathlineto{\pgfqpoint{4.075480in}{2.131650in}}%
\pgfpathlineto{\pgfqpoint{4.062324in}{2.136425in}}%
\pgfpathlineto{\pgfqpoint{4.054722in}{2.127174in}}%
\pgfpathlineto{\pgfqpoint{4.047115in}{2.117954in}}%
\pgfpathlineto{\pgfqpoint{4.039503in}{2.108766in}}%
\pgfpathlineto{\pgfqpoint{4.031886in}{2.099612in}}%
\pgfpathclose%
\pgfusepath{fill}%
\end{pgfscope}%
\begin{pgfscope}%
\pgfpathrectangle{\pgfqpoint{1.254980in}{0.150000in}}{\pgfqpoint{5.490039in}{5.490039in}}%
\pgfusepath{clip}%
\pgfsetbuttcap%
\pgfsetroundjoin%
\definecolor{currentfill}{rgb}{0.282327,0.094955,0.417331}%
\pgfsetfillcolor{currentfill}%
\pgfsetfillopacity{0.700000}%
\pgfsetlinewidth{0.000000pt}%
\definecolor{currentstroke}{rgb}{0.000000,0.000000,0.000000}%
\pgfsetstrokecolor{currentstroke}%
\pgfsetdash{}{0pt}%
\pgfpathmoveto{\pgfqpoint{4.250807in}{2.126250in}}%
\pgfpathlineto{\pgfqpoint{4.264025in}{2.123112in}}%
\pgfpathlineto{\pgfqpoint{4.277250in}{2.120099in}}%
\pgfpathlineto{\pgfqpoint{4.290482in}{2.117209in}}%
\pgfpathlineto{\pgfqpoint{4.303721in}{2.114444in}}%
\pgfpathlineto{\pgfqpoint{4.311257in}{2.124271in}}%
\pgfpathlineto{\pgfqpoint{4.318787in}{2.134110in}}%
\pgfpathlineto{\pgfqpoint{4.326313in}{2.143959in}}%
\pgfpathlineto{\pgfqpoint{4.333833in}{2.153818in}}%
\pgfpathlineto{\pgfqpoint{4.320604in}{2.156482in}}%
\pgfpathlineto{\pgfqpoint{4.307381in}{2.159269in}}%
\pgfpathlineto{\pgfqpoint{4.294166in}{2.162180in}}%
\pgfpathlineto{\pgfqpoint{4.280957in}{2.165216in}}%
\pgfpathlineto{\pgfqpoint{4.273427in}{2.155453in}}%
\pgfpathlineto{\pgfqpoint{4.265892in}{2.145704in}}%
\pgfpathlineto{\pgfqpoint{4.258352in}{2.135969in}}%
\pgfpathlineto{\pgfqpoint{4.250807in}{2.126250in}}%
\pgfpathclose%
\pgfusepath{fill}%
\end{pgfscope}%
\begin{pgfscope}%
\pgfpathrectangle{\pgfqpoint{1.254980in}{0.150000in}}{\pgfqpoint{5.490039in}{5.490039in}}%
\pgfusepath{clip}%
\pgfsetbuttcap%
\pgfsetroundjoin%
\definecolor{currentfill}{rgb}{0.147607,0.511733,0.557049}%
\pgfsetfillcolor{currentfill}%
\pgfsetfillopacity{0.700000}%
\pgfsetlinewidth{0.000000pt}%
\definecolor{currentstroke}{rgb}{0.000000,0.000000,0.000000}%
\pgfsetstrokecolor{currentstroke}%
\pgfsetdash{}{0pt}%
\pgfpathmoveto{\pgfqpoint{5.995749in}{3.049591in}}%
\pgfpathlineto{\pgfqpoint{6.009691in}{3.054905in}}%
\pgfpathlineto{\pgfqpoint{6.023647in}{3.060328in}}%
\pgfpathlineto{\pgfqpoint{6.037619in}{3.065861in}}%
\pgfpathlineto{\pgfqpoint{6.051606in}{3.071504in}}%
\pgfpathlineto{\pgfqpoint{6.058482in}{3.078674in}}%
\pgfpathlineto{\pgfqpoint{6.065352in}{3.085842in}}%
\pgfpathlineto{\pgfqpoint{6.072218in}{3.093013in}}%
\pgfpathlineto{\pgfqpoint{6.079080in}{3.100187in}}%
\pgfpathlineto{\pgfqpoint{6.065112in}{3.094786in}}%
\pgfpathlineto{\pgfqpoint{6.051159in}{3.089493in}}%
\pgfpathlineto{\pgfqpoint{6.037221in}{3.084310in}}%
\pgfpathlineto{\pgfqpoint{6.023299in}{3.079237in}}%
\pgfpathlineto{\pgfqpoint{6.016418in}{3.071816in}}%
\pgfpathlineto{\pgfqpoint{6.009533in}{3.064403in}}%
\pgfpathlineto{\pgfqpoint{6.002644in}{3.056995in}}%
\pgfpathlineto{\pgfqpoint{5.995749in}{3.049591in}}%
\pgfpathclose%
\pgfusepath{fill}%
\end{pgfscope}%
\begin{pgfscope}%
\pgfpathrectangle{\pgfqpoint{1.254980in}{0.150000in}}{\pgfqpoint{5.490039in}{5.490039in}}%
\pgfusepath{clip}%
\pgfsetbuttcap%
\pgfsetroundjoin%
\definecolor{currentfill}{rgb}{0.283091,0.110553,0.431554}%
\pgfsetfillcolor{currentfill}%
\pgfsetfillopacity{0.700000}%
\pgfsetlinewidth{0.000000pt}%
\definecolor{currentstroke}{rgb}{0.000000,0.000000,0.000000}%
\pgfsetstrokecolor{currentstroke}%
\pgfsetdash{}{0pt}%
\pgfpathmoveto{\pgfqpoint{3.624332in}{2.162307in}}%
\pgfpathlineto{\pgfqpoint{3.637453in}{2.153666in}}%
\pgfpathlineto{\pgfqpoint{3.650575in}{2.145166in}}%
\pgfpathlineto{\pgfqpoint{3.663700in}{2.136806in}}%
\pgfpathlineto{\pgfqpoint{3.676827in}{2.128586in}}%
\pgfpathlineto{\pgfqpoint{3.684583in}{2.136322in}}%
\pgfpathlineto{\pgfqpoint{3.692332in}{2.144124in}}%
\pgfpathlineto{\pgfqpoint{3.700075in}{2.151990in}}%
\pgfpathlineto{\pgfqpoint{3.707811in}{2.159921in}}%
\pgfpathlineto{\pgfqpoint{3.694700in}{2.167958in}}%
\pgfpathlineto{\pgfqpoint{3.681592in}{2.176134in}}%
\pgfpathlineto{\pgfqpoint{3.668486in}{2.184450in}}%
\pgfpathlineto{\pgfqpoint{3.655382in}{2.192908in}}%
\pgfpathlineto{\pgfqpoint{3.647629in}{2.185154in}}%
\pgfpathlineto{\pgfqpoint{3.639870in}{2.177469in}}%
\pgfpathlineto{\pgfqpoint{3.632104in}{2.169853in}}%
\pgfpathlineto{\pgfqpoint{3.624332in}{2.162307in}}%
\pgfpathclose%
\pgfusepath{fill}%
\end{pgfscope}%
\begin{pgfscope}%
\pgfpathrectangle{\pgfqpoint{1.254980in}{0.150000in}}{\pgfqpoint{5.490039in}{5.490039in}}%
\pgfusepath{clip}%
\pgfsetbuttcap%
\pgfsetroundjoin%
\definecolor{currentfill}{rgb}{0.136408,0.541173,0.554483}%
\pgfsetfillcolor{currentfill}%
\pgfsetfillopacity{0.700000}%
\pgfsetlinewidth{0.000000pt}%
\definecolor{currentstroke}{rgb}{0.000000,0.000000,0.000000}%
\pgfsetstrokecolor{currentstroke}%
\pgfsetdash{}{0pt}%
\pgfpathmoveto{\pgfqpoint{2.644128in}{3.189512in}}%
\pgfpathlineto{\pgfqpoint{2.657478in}{3.168325in}}%
\pgfpathlineto{\pgfqpoint{2.670819in}{3.147351in}}%
\pgfpathlineto{\pgfqpoint{2.684152in}{3.126588in}}%
\pgfpathlineto{\pgfqpoint{2.697477in}{3.106033in}}%
\pgfpathlineto{\pgfqpoint{2.705691in}{3.109641in}}%
\pgfpathlineto{\pgfqpoint{2.713894in}{3.113396in}}%
\pgfpathlineto{\pgfqpoint{2.722085in}{3.117297in}}%
\pgfpathlineto{\pgfqpoint{2.730266in}{3.121343in}}%
\pgfpathlineto{\pgfqpoint{2.716973in}{3.141677in}}%
\pgfpathlineto{\pgfqpoint{2.703672in}{3.162220in}}%
\pgfpathlineto{\pgfqpoint{2.690362in}{3.182973in}}%
\pgfpathlineto{\pgfqpoint{2.677045in}{3.203937in}}%
\pgfpathlineto{\pgfqpoint{2.668833in}{3.200106in}}%
\pgfpathlineto{\pgfqpoint{2.660609in}{3.196423in}}%
\pgfpathlineto{\pgfqpoint{2.652374in}{3.192891in}}%
\pgfpathlineto{\pgfqpoint{2.644128in}{3.189512in}}%
\pgfpathclose%
\pgfusepath{fill}%
\end{pgfscope}%
\begin{pgfscope}%
\pgfpathrectangle{\pgfqpoint{1.254980in}{0.150000in}}{\pgfqpoint{5.490039in}{5.490039in}}%
\pgfusepath{clip}%
\pgfsetbuttcap%
\pgfsetroundjoin%
\definecolor{currentfill}{rgb}{0.141935,0.526453,0.555991}%
\pgfsetfillcolor{currentfill}%
\pgfsetfillopacity{0.700000}%
\pgfsetlinewidth{0.000000pt}%
\definecolor{currentstroke}{rgb}{0.000000,0.000000,0.000000}%
\pgfsetstrokecolor{currentstroke}%
\pgfsetdash{}{0pt}%
\pgfpathmoveto{\pgfqpoint{6.079080in}{3.100187in}}%
\pgfpathlineto{\pgfqpoint{6.093063in}{3.105698in}}%
\pgfpathlineto{\pgfqpoint{6.107061in}{3.111318in}}%
\pgfpathlineto{\pgfqpoint{6.121075in}{3.117047in}}%
\pgfpathlineto{\pgfqpoint{6.127917in}{3.124039in}}%
\pgfpathlineto{\pgfqpoint{6.134754in}{3.131036in}}%
\pgfpathlineto{\pgfqpoint{6.141586in}{3.138042in}}%
\pgfpathlineto{\pgfqpoint{6.127587in}{3.132505in}}%
\pgfpathlineto{\pgfqpoint{6.113604in}{3.127076in}}%
\pgfpathlineto{\pgfqpoint{6.099636in}{3.121757in}}%
\pgfpathlineto{\pgfqpoint{6.092788in}{3.114557in}}%
\pgfpathlineto{\pgfqpoint{6.085936in}{3.107368in}}%
\pgfpathlineto{\pgfqpoint{6.079080in}{3.100187in}}%
\pgfpathclose%
\pgfusepath{fill}%
\end{pgfscope}%
\begin{pgfscope}%
\pgfpathrectangle{\pgfqpoint{1.254980in}{0.150000in}}{\pgfqpoint{5.490039in}{5.490039in}}%
\pgfusepath{clip}%
\pgfsetbuttcap%
\pgfsetroundjoin%
\definecolor{currentfill}{rgb}{0.281887,0.150881,0.465405}%
\pgfsetfillcolor{currentfill}%
\pgfsetfillopacity{0.700000}%
\pgfsetlinewidth{0.000000pt}%
\definecolor{currentstroke}{rgb}{0.000000,0.000000,0.000000}%
\pgfsetstrokecolor{currentstroke}%
\pgfsetdash{}{0pt}%
\pgfpathmoveto{\pgfqpoint{3.435642in}{2.250160in}}%
\pgfpathlineto{\pgfqpoint{3.448765in}{2.239579in}}%
\pgfpathlineto{\pgfqpoint{3.461888in}{2.229147in}}%
\pgfpathlineto{\pgfqpoint{3.475012in}{2.218863in}}%
\pgfpathlineto{\pgfqpoint{3.488137in}{2.208727in}}%
\pgfpathlineto{\pgfqpoint{3.495972in}{2.215564in}}%
\pgfpathlineto{\pgfqpoint{3.503800in}{2.222485in}}%
\pgfpathlineto{\pgfqpoint{3.511621in}{2.229489in}}%
\pgfpathlineto{\pgfqpoint{3.519436in}{2.236574in}}%
\pgfpathlineto{\pgfqpoint{3.506330in}{2.246510in}}%
\pgfpathlineto{\pgfqpoint{3.493225in}{2.256592in}}%
\pgfpathlineto{\pgfqpoint{3.480122in}{2.266823in}}%
\pgfpathlineto{\pgfqpoint{3.467019in}{2.277201in}}%
\pgfpathlineto{\pgfqpoint{3.459185in}{2.270311in}}%
\pgfpathlineto{\pgfqpoint{3.451345in}{2.263507in}}%
\pgfpathlineto{\pgfqpoint{3.443497in}{2.256789in}}%
\pgfpathlineto{\pgfqpoint{3.435642in}{2.250160in}}%
\pgfpathclose%
\pgfusepath{fill}%
\end{pgfscope}%
\begin{pgfscope}%
\pgfpathrectangle{\pgfqpoint{1.254980in}{0.150000in}}{\pgfqpoint{5.490039in}{5.490039in}}%
\pgfusepath{clip}%
\pgfsetbuttcap%
\pgfsetroundjoin%
\definecolor{currentfill}{rgb}{0.266580,0.228262,0.514349}%
\pgfsetfillcolor{currentfill}%
\pgfsetfillopacity{0.700000}%
\pgfsetlinewidth{0.000000pt}%
\definecolor{currentstroke}{rgb}{0.000000,0.000000,0.000000}%
\pgfsetstrokecolor{currentstroke}%
\pgfsetdash{}{0pt}%
\pgfpathmoveto{\pgfqpoint{4.885257in}{2.374775in}}%
\pgfpathlineto{\pgfqpoint{4.898694in}{2.375858in}}%
\pgfpathlineto{\pgfqpoint{4.912141in}{2.377058in}}%
\pgfpathlineto{\pgfqpoint{4.925599in}{2.378374in}}%
\pgfpathlineto{\pgfqpoint{4.939068in}{2.379806in}}%
\pgfpathlineto{\pgfqpoint{4.946402in}{2.389800in}}%
\pgfpathlineto{\pgfqpoint{4.953730in}{2.399768in}}%
\pgfpathlineto{\pgfqpoint{4.961054in}{2.409713in}}%
\pgfpathlineto{\pgfqpoint{4.968373in}{2.419633in}}%
\pgfpathlineto{\pgfqpoint{4.954912in}{2.418211in}}%
\pgfpathlineto{\pgfqpoint{4.941462in}{2.416905in}}%
\pgfpathlineto{\pgfqpoint{4.928023in}{2.415714in}}%
\pgfpathlineto{\pgfqpoint{4.914595in}{2.414640in}}%
\pgfpathlineto{\pgfqpoint{4.907268in}{2.404704in}}%
\pgfpathlineto{\pgfqpoint{4.899936in}{2.394748in}}%
\pgfpathlineto{\pgfqpoint{4.892599in}{2.384772in}}%
\pgfpathlineto{\pgfqpoint{4.885257in}{2.374775in}}%
\pgfpathclose%
\pgfusepath{fill}%
\end{pgfscope}%
\begin{pgfscope}%
\pgfpathrectangle{\pgfqpoint{1.254980in}{0.150000in}}{\pgfqpoint{5.490039in}{5.490039in}}%
\pgfusepath{clip}%
\pgfsetbuttcap%
\pgfsetroundjoin%
\definecolor{currentfill}{rgb}{0.271828,0.209303,0.504434}%
\pgfsetfillcolor{currentfill}%
\pgfsetfillopacity{0.700000}%
\pgfsetlinewidth{0.000000pt}%
\definecolor{currentstroke}{rgb}{0.000000,0.000000,0.000000}%
\pgfsetstrokecolor{currentstroke}%
\pgfsetdash{}{0pt}%
\pgfpathmoveto{\pgfqpoint{4.802164in}{2.331432in}}%
\pgfpathlineto{\pgfqpoint{4.815568in}{2.332042in}}%
\pgfpathlineto{\pgfqpoint{4.828982in}{2.332769in}}%
\pgfpathlineto{\pgfqpoint{4.842406in}{2.333613in}}%
\pgfpathlineto{\pgfqpoint{4.855840in}{2.334574in}}%
\pgfpathlineto{\pgfqpoint{4.863202in}{2.344657in}}%
\pgfpathlineto{\pgfqpoint{4.870559in}{2.354718in}}%
\pgfpathlineto{\pgfqpoint{4.877910in}{2.364757in}}%
\pgfpathlineto{\pgfqpoint{4.885257in}{2.374775in}}%
\pgfpathlineto{\pgfqpoint{4.871831in}{2.373808in}}%
\pgfpathlineto{\pgfqpoint{4.858415in}{2.372957in}}%
\pgfpathlineto{\pgfqpoint{4.845009in}{2.372223in}}%
\pgfpathlineto{\pgfqpoint{4.831613in}{2.371607in}}%
\pgfpathlineto{\pgfqpoint{4.824258in}{2.361589in}}%
\pgfpathlineto{\pgfqpoint{4.816899in}{2.351554in}}%
\pgfpathlineto{\pgfqpoint{4.809534in}{2.341502in}}%
\pgfpathlineto{\pgfqpoint{4.802164in}{2.331432in}}%
\pgfpathclose%
\pgfusepath{fill}%
\end{pgfscope}%
\begin{pgfscope}%
\pgfpathrectangle{\pgfqpoint{1.254980in}{0.150000in}}{\pgfqpoint{5.490039in}{5.490039in}}%
\pgfusepath{clip}%
\pgfsetbuttcap%
\pgfsetroundjoin%
\definecolor{currentfill}{rgb}{0.258965,0.251537,0.524736}%
\pgfsetfillcolor{currentfill}%
\pgfsetfillopacity{0.700000}%
\pgfsetlinewidth{0.000000pt}%
\definecolor{currentstroke}{rgb}{0.000000,0.000000,0.000000}%
\pgfsetstrokecolor{currentstroke}%
\pgfsetdash{}{0pt}%
\pgfpathmoveto{\pgfqpoint{4.968373in}{2.419633in}}%
\pgfpathlineto{\pgfqpoint{4.981844in}{2.421171in}}%
\pgfpathlineto{\pgfqpoint{4.995326in}{2.422825in}}%
\pgfpathlineto{\pgfqpoint{5.008820in}{2.424594in}}%
\pgfpathlineto{\pgfqpoint{5.022324in}{2.426479in}}%
\pgfpathlineto{\pgfqpoint{5.029630in}{2.436356in}}%
\pgfpathlineto{\pgfqpoint{5.036930in}{2.446206in}}%
\pgfpathlineto{\pgfqpoint{5.044225in}{2.456030in}}%
\pgfpathlineto{\pgfqpoint{5.051515in}{2.465828in}}%
\pgfpathlineto{\pgfqpoint{5.038019in}{2.463969in}}%
\pgfpathlineto{\pgfqpoint{5.024534in}{2.462226in}}%
\pgfpathlineto{\pgfqpoint{5.011060in}{2.460598in}}%
\pgfpathlineto{\pgfqpoint{4.997597in}{2.459086in}}%
\pgfpathlineto{\pgfqpoint{4.990298in}{2.449256in}}%
\pgfpathlineto{\pgfqpoint{4.982995in}{2.439404in}}%
\pgfpathlineto{\pgfqpoint{4.975686in}{2.429530in}}%
\pgfpathlineto{\pgfqpoint{4.968373in}{2.419633in}}%
\pgfpathclose%
\pgfusepath{fill}%
\end{pgfscope}%
\begin{pgfscope}%
\pgfpathrectangle{\pgfqpoint{1.254980in}{0.150000in}}{\pgfqpoint{5.490039in}{5.490039in}}%
\pgfusepath{clip}%
\pgfsetbuttcap%
\pgfsetroundjoin%
\definecolor{currentfill}{rgb}{0.277134,0.185228,0.489898}%
\pgfsetfillcolor{currentfill}%
\pgfsetfillopacity{0.700000}%
\pgfsetlinewidth{0.000000pt}%
\definecolor{currentstroke}{rgb}{0.000000,0.000000,0.000000}%
\pgfsetstrokecolor{currentstroke}%
\pgfsetdash{}{0pt}%
\pgfpathmoveto{\pgfqpoint{4.719089in}{2.289794in}}%
\pgfpathlineto{\pgfqpoint{4.732462in}{2.289911in}}%
\pgfpathlineto{\pgfqpoint{4.745844in}{2.290147in}}%
\pgfpathlineto{\pgfqpoint{4.759236in}{2.290500in}}%
\pgfpathlineto{\pgfqpoint{4.772637in}{2.290970in}}%
\pgfpathlineto{\pgfqpoint{4.780026in}{2.301113in}}%
\pgfpathlineto{\pgfqpoint{4.787411in}{2.311238in}}%
\pgfpathlineto{\pgfqpoint{4.794790in}{2.321344in}}%
\pgfpathlineto{\pgfqpoint{4.802164in}{2.331432in}}%
\pgfpathlineto{\pgfqpoint{4.788771in}{2.330939in}}%
\pgfpathlineto{\pgfqpoint{4.775387in}{2.330564in}}%
\pgfpathlineto{\pgfqpoint{4.762013in}{2.330306in}}%
\pgfpathlineto{\pgfqpoint{4.748648in}{2.330166in}}%
\pgfpathlineto{\pgfqpoint{4.741266in}{2.320094in}}%
\pgfpathlineto{\pgfqpoint{4.733879in}{2.310008in}}%
\pgfpathlineto{\pgfqpoint{4.726486in}{2.299908in}}%
\pgfpathlineto{\pgfqpoint{4.719089in}{2.289794in}}%
\pgfpathclose%
\pgfusepath{fill}%
\end{pgfscope}%
\begin{pgfscope}%
\pgfpathrectangle{\pgfqpoint{1.254980in}{0.150000in}}{\pgfqpoint{5.490039in}{5.490039in}}%
\pgfusepath{clip}%
\pgfsetbuttcap%
\pgfsetroundjoin%
\definecolor{currentfill}{rgb}{0.250425,0.274290,0.533103}%
\pgfsetfillcolor{currentfill}%
\pgfsetfillopacity{0.700000}%
\pgfsetlinewidth{0.000000pt}%
\definecolor{currentstroke}{rgb}{0.000000,0.000000,0.000000}%
\pgfsetstrokecolor{currentstroke}%
\pgfsetdash{}{0pt}%
\pgfpathmoveto{\pgfqpoint{5.051515in}{2.465828in}}%
\pgfpathlineto{\pgfqpoint{5.065022in}{2.467802in}}%
\pgfpathlineto{\pgfqpoint{5.078541in}{2.469890in}}%
\pgfpathlineto{\pgfqpoint{5.092071in}{2.472093in}}%
\pgfpathlineto{\pgfqpoint{5.105613in}{2.474411in}}%
\pgfpathlineto{\pgfqpoint{5.112889in}{2.484148in}}%
\pgfpathlineto{\pgfqpoint{5.120160in}{2.493856in}}%
\pgfpathlineto{\pgfqpoint{5.127426in}{2.503536in}}%
\pgfpathlineto{\pgfqpoint{5.134687in}{2.513189in}}%
\pgfpathlineto{\pgfqpoint{5.121154in}{2.510913in}}%
\pgfpathlineto{\pgfqpoint{5.107632in}{2.508752in}}%
\pgfpathlineto{\pgfqpoint{5.094122in}{2.506705in}}%
\pgfpathlineto{\pgfqpoint{5.080624in}{2.504773in}}%
\pgfpathlineto{\pgfqpoint{5.073354in}{2.495073in}}%
\pgfpathlineto{\pgfqpoint{5.066079in}{2.485349in}}%
\pgfpathlineto{\pgfqpoint{5.058800in}{2.475601in}}%
\pgfpathlineto{\pgfqpoint{5.051515in}{2.465828in}}%
\pgfpathclose%
\pgfusepath{fill}%
\end{pgfscope}%
\begin{pgfscope}%
\pgfpathrectangle{\pgfqpoint{1.254980in}{0.150000in}}{\pgfqpoint{5.490039in}{5.490039in}}%
\pgfusepath{clip}%
\pgfsetbuttcap%
\pgfsetroundjoin%
\definecolor{currentfill}{rgb}{0.235526,0.309527,0.542944}%
\pgfsetfillcolor{currentfill}%
\pgfsetfillopacity{0.700000}%
\pgfsetlinewidth{0.000000pt}%
\definecolor{currentstroke}{rgb}{0.000000,0.000000,0.000000}%
\pgfsetstrokecolor{currentstroke}%
\pgfsetdash{}{0pt}%
\pgfpathmoveto{\pgfqpoint{3.035771in}{2.590472in}}%
\pgfpathlineto{\pgfqpoint{3.048957in}{2.575183in}}%
\pgfpathlineto{\pgfqpoint{3.062140in}{2.560067in}}%
\pgfpathlineto{\pgfqpoint{3.075319in}{2.545122in}}%
\pgfpathlineto{\pgfqpoint{3.088496in}{2.530348in}}%
\pgfpathlineto{\pgfqpoint{3.096520in}{2.535244in}}%
\pgfpathlineto{\pgfqpoint{3.104535in}{2.540261in}}%
\pgfpathlineto{\pgfqpoint{3.112541in}{2.545395in}}%
\pgfpathlineto{\pgfqpoint{3.120539in}{2.550646in}}%
\pgfpathlineto{\pgfqpoint{3.107388in}{2.565196in}}%
\pgfpathlineto{\pgfqpoint{3.094234in}{2.579916in}}%
\pgfpathlineto{\pgfqpoint{3.081077in}{2.594807in}}%
\pgfpathlineto{\pgfqpoint{3.067917in}{2.609870in}}%
\pgfpathlineto{\pgfqpoint{3.059895in}{2.604838in}}%
\pgfpathlineto{\pgfqpoint{3.051863in}{2.599926in}}%
\pgfpathlineto{\pgfqpoint{3.043822in}{2.595137in}}%
\pgfpathlineto{\pgfqpoint{3.035771in}{2.590472in}}%
\pgfpathclose%
\pgfusepath{fill}%
\end{pgfscope}%
\begin{pgfscope}%
\pgfpathrectangle{\pgfqpoint{1.254980in}{0.150000in}}{\pgfqpoint{5.490039in}{5.490039in}}%
\pgfusepath{clip}%
\pgfsetbuttcap%
\pgfsetroundjoin%
\definecolor{currentfill}{rgb}{0.223925,0.334994,0.548053}%
\pgfsetfillcolor{currentfill}%
\pgfsetfillopacity{0.700000}%
\pgfsetlinewidth{0.000000pt}%
\definecolor{currentstroke}{rgb}{0.000000,0.000000,0.000000}%
\pgfsetstrokecolor{currentstroke}%
\pgfsetdash{}{0pt}%
\pgfpathmoveto{\pgfqpoint{2.982989in}{2.653373in}}%
\pgfpathlineto{\pgfqpoint{2.996190in}{2.637383in}}%
\pgfpathlineto{\pgfqpoint{3.009388in}{2.621571in}}%
\pgfpathlineto{\pgfqpoint{3.022581in}{2.605934in}}%
\pgfpathlineto{\pgfqpoint{3.035771in}{2.590472in}}%
\pgfpathlineto{\pgfqpoint{3.043822in}{2.595137in}}%
\pgfpathlineto{\pgfqpoint{3.051863in}{2.599926in}}%
\pgfpathlineto{\pgfqpoint{3.059895in}{2.604838in}}%
\pgfpathlineto{\pgfqpoint{3.067917in}{2.609870in}}%
\pgfpathlineto{\pgfqpoint{3.054754in}{2.625106in}}%
\pgfpathlineto{\pgfqpoint{3.041587in}{2.640517in}}%
\pgfpathlineto{\pgfqpoint{3.028416in}{2.656103in}}%
\pgfpathlineto{\pgfqpoint{3.015242in}{2.671865in}}%
\pgfpathlineto{\pgfqpoint{3.007193in}{2.667053in}}%
\pgfpathlineto{\pgfqpoint{2.999134in}{2.662366in}}%
\pgfpathlineto{\pgfqpoint{2.991066in}{2.657805in}}%
\pgfpathlineto{\pgfqpoint{2.982989in}{2.653373in}}%
\pgfpathclose%
\pgfusepath{fill}%
\end{pgfscope}%
\begin{pgfscope}%
\pgfpathrectangle{\pgfqpoint{1.254980in}{0.150000in}}{\pgfqpoint{5.490039in}{5.490039in}}%
\pgfusepath{clip}%
\pgfsetbuttcap%
\pgfsetroundjoin%
\definecolor{currentfill}{rgb}{0.280255,0.165693,0.476498}%
\pgfsetfillcolor{currentfill}%
\pgfsetfillopacity{0.700000}%
\pgfsetlinewidth{0.000000pt}%
\definecolor{currentstroke}{rgb}{0.000000,0.000000,0.000000}%
\pgfsetstrokecolor{currentstroke}%
\pgfsetdash{}{0pt}%
\pgfpathmoveto{\pgfqpoint{4.636026in}{2.250058in}}%
\pgfpathlineto{\pgfqpoint{4.649369in}{2.249664in}}%
\pgfpathlineto{\pgfqpoint{4.662721in}{2.249388in}}%
\pgfpathlineto{\pgfqpoint{4.676082in}{2.249230in}}%
\pgfpathlineto{\pgfqpoint{4.689453in}{2.249191in}}%
\pgfpathlineto{\pgfqpoint{4.696869in}{2.259364in}}%
\pgfpathlineto{\pgfqpoint{4.704281in}{2.269521in}}%
\pgfpathlineto{\pgfqpoint{4.711688in}{2.279665in}}%
\pgfpathlineto{\pgfqpoint{4.719089in}{2.289794in}}%
\pgfpathlineto{\pgfqpoint{4.705727in}{2.289794in}}%
\pgfpathlineto{\pgfqpoint{4.692373in}{2.289913in}}%
\pgfpathlineto{\pgfqpoint{4.679029in}{2.290151in}}%
\pgfpathlineto{\pgfqpoint{4.665694in}{2.290507in}}%
\pgfpathlineto{\pgfqpoint{4.658284in}{2.280411in}}%
\pgfpathlineto{\pgfqpoint{4.650870in}{2.270304in}}%
\pgfpathlineto{\pgfqpoint{4.643450in}{2.260186in}}%
\pgfpathlineto{\pgfqpoint{4.636026in}{2.250058in}}%
\pgfpathclose%
\pgfusepath{fill}%
\end{pgfscope}%
\begin{pgfscope}%
\pgfpathrectangle{\pgfqpoint{1.254980in}{0.150000in}}{\pgfqpoint{5.490039in}{5.490039in}}%
\pgfusepath{clip}%
\pgfsetbuttcap%
\pgfsetroundjoin%
\definecolor{currentfill}{rgb}{0.241237,0.296485,0.539709}%
\pgfsetfillcolor{currentfill}%
\pgfsetfillopacity{0.700000}%
\pgfsetlinewidth{0.000000pt}%
\definecolor{currentstroke}{rgb}{0.000000,0.000000,0.000000}%
\pgfsetstrokecolor{currentstroke}%
\pgfsetdash{}{0pt}%
\pgfpathmoveto{\pgfqpoint{5.134687in}{2.513189in}}%
\pgfpathlineto{\pgfqpoint{5.148231in}{2.515579in}}%
\pgfpathlineto{\pgfqpoint{5.161788in}{2.518084in}}%
\pgfpathlineto{\pgfqpoint{5.175356in}{2.520702in}}%
\pgfpathlineto{\pgfqpoint{5.188937in}{2.523435in}}%
\pgfpathlineto{\pgfqpoint{5.196183in}{2.533009in}}%
\pgfpathlineto{\pgfqpoint{5.203424in}{2.542553in}}%
\pgfpathlineto{\pgfqpoint{5.210660in}{2.552067in}}%
\pgfpathlineto{\pgfqpoint{5.217890in}{2.561554in}}%
\pgfpathlineto{\pgfqpoint{5.204319in}{2.558880in}}%
\pgfpathlineto{\pgfqpoint{5.190760in}{2.556319in}}%
\pgfpathlineto{\pgfqpoint{5.177213in}{2.553873in}}%
\pgfpathlineto{\pgfqpoint{5.163677in}{2.551541in}}%
\pgfpathlineto{\pgfqpoint{5.156437in}{2.541990in}}%
\pgfpathlineto{\pgfqpoint{5.149192in}{2.532415in}}%
\pgfpathlineto{\pgfqpoint{5.141942in}{2.522815in}}%
\pgfpathlineto{\pgfqpoint{5.134687in}{2.513189in}}%
\pgfpathclose%
\pgfusepath{fill}%
\end{pgfscope}%
\begin{pgfscope}%
\pgfpathrectangle{\pgfqpoint{1.254980in}{0.150000in}}{\pgfqpoint{5.490039in}{5.490039in}}%
\pgfusepath{clip}%
\pgfsetbuttcap%
\pgfsetroundjoin%
\definecolor{currentfill}{rgb}{0.246811,0.283237,0.535941}%
\pgfsetfillcolor{currentfill}%
\pgfsetfillopacity{0.700000}%
\pgfsetlinewidth{0.000000pt}%
\definecolor{currentstroke}{rgb}{0.000000,0.000000,0.000000}%
\pgfsetstrokecolor{currentstroke}%
\pgfsetdash{}{0pt}%
\pgfpathmoveto{\pgfqpoint{3.088496in}{2.530348in}}%
\pgfpathlineto{\pgfqpoint{3.101669in}{2.515743in}}%
\pgfpathlineto{\pgfqpoint{3.114839in}{2.501307in}}%
\pgfpathlineto{\pgfqpoint{3.128007in}{2.487038in}}%
\pgfpathlineto{\pgfqpoint{3.141172in}{2.472936in}}%
\pgfpathlineto{\pgfqpoint{3.149171in}{2.478063in}}%
\pgfpathlineto{\pgfqpoint{3.157161in}{2.483305in}}%
\pgfpathlineto{\pgfqpoint{3.165142in}{2.488662in}}%
\pgfpathlineto{\pgfqpoint{3.173115in}{2.494130in}}%
\pgfpathlineto{\pgfqpoint{3.159975in}{2.508008in}}%
\pgfpathlineto{\pgfqpoint{3.146832in}{2.522053in}}%
\pgfpathlineto{\pgfqpoint{3.133687in}{2.536266in}}%
\pgfpathlineto{\pgfqpoint{3.120539in}{2.550646in}}%
\pgfpathlineto{\pgfqpoint{3.112541in}{2.545395in}}%
\pgfpathlineto{\pgfqpoint{3.104535in}{2.540261in}}%
\pgfpathlineto{\pgfqpoint{3.096520in}{2.535244in}}%
\pgfpathlineto{\pgfqpoint{3.088496in}{2.530348in}}%
\pgfpathclose%
\pgfusepath{fill}%
\end{pgfscope}%
\begin{pgfscope}%
\pgfpathrectangle{\pgfqpoint{1.254980in}{0.150000in}}{\pgfqpoint{5.490039in}{5.490039in}}%
\pgfusepath{clip}%
\pgfsetbuttcap%
\pgfsetroundjoin%
\definecolor{currentfill}{rgb}{0.210503,0.363727,0.552206}%
\pgfsetfillcolor{currentfill}%
\pgfsetfillopacity{0.700000}%
\pgfsetlinewidth{0.000000pt}%
\definecolor{currentstroke}{rgb}{0.000000,0.000000,0.000000}%
\pgfsetstrokecolor{currentstroke}%
\pgfsetdash{}{0pt}%
\pgfpathmoveto{\pgfqpoint{2.930139in}{2.719117in}}%
\pgfpathlineto{\pgfqpoint{2.943358in}{2.702411in}}%
\pgfpathlineto{\pgfqpoint{2.956573in}{2.685885in}}%
\pgfpathlineto{\pgfqpoint{2.969783in}{2.669539in}}%
\pgfpathlineto{\pgfqpoint{2.982989in}{2.653373in}}%
\pgfpathlineto{\pgfqpoint{2.991066in}{2.657805in}}%
\pgfpathlineto{\pgfqpoint{2.999134in}{2.662366in}}%
\pgfpathlineto{\pgfqpoint{3.007193in}{2.667053in}}%
\pgfpathlineto{\pgfqpoint{3.015242in}{2.671865in}}%
\pgfpathlineto{\pgfqpoint{3.002063in}{2.687804in}}%
\pgfpathlineto{\pgfqpoint{2.988881in}{2.703922in}}%
\pgfpathlineto{\pgfqpoint{2.975694in}{2.720220in}}%
\pgfpathlineto{\pgfqpoint{2.962502in}{2.736698in}}%
\pgfpathlineto{\pgfqpoint{2.954426in}{2.732107in}}%
\pgfpathlineto{\pgfqpoint{2.946340in}{2.727646in}}%
\pgfpathlineto{\pgfqpoint{2.938245in}{2.723315in}}%
\pgfpathlineto{\pgfqpoint{2.930139in}{2.719117in}}%
\pgfpathclose%
\pgfusepath{fill}%
\end{pgfscope}%
\begin{pgfscope}%
\pgfpathrectangle{\pgfqpoint{1.254980in}{0.150000in}}{\pgfqpoint{5.490039in}{5.490039in}}%
\pgfusepath{clip}%
\pgfsetbuttcap%
\pgfsetroundjoin%
\definecolor{currentfill}{rgb}{0.231674,0.318106,0.544834}%
\pgfsetfillcolor{currentfill}%
\pgfsetfillopacity{0.700000}%
\pgfsetlinewidth{0.000000pt}%
\definecolor{currentstroke}{rgb}{0.000000,0.000000,0.000000}%
\pgfsetstrokecolor{currentstroke}%
\pgfsetdash{}{0pt}%
\pgfpathmoveto{\pgfqpoint{5.217890in}{2.561554in}}%
\pgfpathlineto{\pgfqpoint{5.231473in}{2.564342in}}%
\pgfpathlineto{\pgfqpoint{5.245069in}{2.567244in}}%
\pgfpathlineto{\pgfqpoint{5.258676in}{2.570260in}}%
\pgfpathlineto{\pgfqpoint{5.272296in}{2.573389in}}%
\pgfpathlineto{\pgfqpoint{5.279512in}{2.582780in}}%
\pgfpathlineto{\pgfqpoint{5.286722in}{2.592140in}}%
\pgfpathlineto{\pgfqpoint{5.293927in}{2.601470in}}%
\pgfpathlineto{\pgfqpoint{5.301126in}{2.610772in}}%
\pgfpathlineto{\pgfqpoint{5.287516in}{2.607718in}}%
\pgfpathlineto{\pgfqpoint{5.273919in}{2.604777in}}%
\pgfpathlineto{\pgfqpoint{5.260333in}{2.601949in}}%
\pgfpathlineto{\pgfqpoint{5.246760in}{2.599235in}}%
\pgfpathlineto{\pgfqpoint{5.239550in}{2.589853in}}%
\pgfpathlineto{\pgfqpoint{5.232335in}{2.580446in}}%
\pgfpathlineto{\pgfqpoint{5.225115in}{2.571013in}}%
\pgfpathlineto{\pgfqpoint{5.217890in}{2.561554in}}%
\pgfpathclose%
\pgfusepath{fill}%
\end{pgfscope}%
\begin{pgfscope}%
\pgfpathrectangle{\pgfqpoint{1.254980in}{0.150000in}}{\pgfqpoint{5.490039in}{5.490039in}}%
\pgfusepath{clip}%
\pgfsetbuttcap%
\pgfsetroundjoin%
\definecolor{currentfill}{rgb}{0.281887,0.150881,0.465405}%
\pgfsetfillcolor{currentfill}%
\pgfsetfillopacity{0.700000}%
\pgfsetlinewidth{0.000000pt}%
\definecolor{currentstroke}{rgb}{0.000000,0.000000,0.000000}%
\pgfsetstrokecolor{currentstroke}%
\pgfsetdash{}{0pt}%
\pgfpathmoveto{\pgfqpoint{4.552967in}{2.212433in}}%
\pgfpathlineto{\pgfqpoint{4.566282in}{2.211507in}}%
\pgfpathlineto{\pgfqpoint{4.579606in}{2.210700in}}%
\pgfpathlineto{\pgfqpoint{4.592939in}{2.210013in}}%
\pgfpathlineto{\pgfqpoint{4.606281in}{2.209445in}}%
\pgfpathlineto{\pgfqpoint{4.613725in}{2.219613in}}%
\pgfpathlineto{\pgfqpoint{4.621163in}{2.229772in}}%
\pgfpathlineto{\pgfqpoint{4.628597in}{2.239920in}}%
\pgfpathlineto{\pgfqpoint{4.636026in}{2.250058in}}%
\pgfpathlineto{\pgfqpoint{4.622692in}{2.250572in}}%
\pgfpathlineto{\pgfqpoint{4.609367in}{2.251205in}}%
\pgfpathlineto{\pgfqpoint{4.596052in}{2.251957in}}%
\pgfpathlineto{\pgfqpoint{4.582744in}{2.252829in}}%
\pgfpathlineto{\pgfqpoint{4.575307in}{2.242740in}}%
\pgfpathlineto{\pgfqpoint{4.567865in}{2.232644in}}%
\pgfpathlineto{\pgfqpoint{4.560419in}{2.222541in}}%
\pgfpathlineto{\pgfqpoint{4.552967in}{2.212433in}}%
\pgfpathclose%
\pgfusepath{fill}%
\end{pgfscope}%
\begin{pgfscope}%
\pgfpathrectangle{\pgfqpoint{1.254980in}{0.150000in}}{\pgfqpoint{5.490039in}{5.490039in}}%
\pgfusepath{clip}%
\pgfsetbuttcap%
\pgfsetroundjoin%
\definecolor{currentfill}{rgb}{0.281446,0.084320,0.407414}%
\pgfsetfillcolor{currentfill}%
\pgfsetfillopacity{0.700000}%
\pgfsetlinewidth{0.000000pt}%
\definecolor{currentstroke}{rgb}{0.000000,0.000000,0.000000}%
\pgfsetstrokecolor{currentstroke}%
\pgfsetdash{}{0pt}%
\pgfpathmoveto{\pgfqpoint{4.167730in}{2.101805in}}%
\pgfpathlineto{\pgfqpoint{4.180932in}{2.098049in}}%
\pgfpathlineto{\pgfqpoint{4.194141in}{2.094419in}}%
\pgfpathlineto{\pgfqpoint{4.207356in}{2.090914in}}%
\pgfpathlineto{\pgfqpoint{4.220578in}{2.087534in}}%
\pgfpathlineto{\pgfqpoint{4.228143in}{2.097187in}}%
\pgfpathlineto{\pgfqpoint{4.235702in}{2.106858in}}%
\pgfpathlineto{\pgfqpoint{4.243257in}{2.116545in}}%
\pgfpathlineto{\pgfqpoint{4.250807in}{2.126250in}}%
\pgfpathlineto{\pgfqpoint{4.237595in}{2.129512in}}%
\pgfpathlineto{\pgfqpoint{4.224390in}{2.132899in}}%
\pgfpathlineto{\pgfqpoint{4.211191in}{2.136412in}}%
\pgfpathlineto{\pgfqpoint{4.197999in}{2.140050in}}%
\pgfpathlineto{\pgfqpoint{4.190439in}{2.130458in}}%
\pgfpathlineto{\pgfqpoint{4.182875in}{2.120886in}}%
\pgfpathlineto{\pgfqpoint{4.175305in}{2.111335in}}%
\pgfpathlineto{\pgfqpoint{4.167730in}{2.101805in}}%
\pgfpathclose%
\pgfusepath{fill}%
\end{pgfscope}%
\begin{pgfscope}%
\pgfpathrectangle{\pgfqpoint{1.254980in}{0.150000in}}{\pgfqpoint{5.490039in}{5.490039in}}%
\pgfusepath{clip}%
\pgfsetbuttcap%
\pgfsetroundjoin%
\definecolor{currentfill}{rgb}{0.255645,0.260703,0.528312}%
\pgfsetfillcolor{currentfill}%
\pgfsetfillopacity{0.700000}%
\pgfsetlinewidth{0.000000pt}%
\definecolor{currentstroke}{rgb}{0.000000,0.000000,0.000000}%
\pgfsetstrokecolor{currentstroke}%
\pgfsetdash{}{0pt}%
\pgfpathmoveto{\pgfqpoint{3.141172in}{2.472936in}}%
\pgfpathlineto{\pgfqpoint{3.154335in}{2.459000in}}%
\pgfpathlineto{\pgfqpoint{3.167495in}{2.445229in}}%
\pgfpathlineto{\pgfqpoint{3.180653in}{2.431622in}}%
\pgfpathlineto{\pgfqpoint{3.193809in}{2.418177in}}%
\pgfpathlineto{\pgfqpoint{3.201784in}{2.423533in}}%
\pgfpathlineto{\pgfqpoint{3.209750in}{2.429000in}}%
\pgfpathlineto{\pgfqpoint{3.217707in}{2.434577in}}%
\pgfpathlineto{\pgfqpoint{3.225656in}{2.440261in}}%
\pgfpathlineto{\pgfqpoint{3.212524in}{2.453483in}}%
\pgfpathlineto{\pgfqpoint{3.199390in}{2.466868in}}%
\pgfpathlineto{\pgfqpoint{3.186253in}{2.480417in}}%
\pgfpathlineto{\pgfqpoint{3.173115in}{2.494130in}}%
\pgfpathlineto{\pgfqpoint{3.165142in}{2.488662in}}%
\pgfpathlineto{\pgfqpoint{3.157161in}{2.483305in}}%
\pgfpathlineto{\pgfqpoint{3.149171in}{2.478063in}}%
\pgfpathlineto{\pgfqpoint{3.141172in}{2.472936in}}%
\pgfpathclose%
\pgfusepath{fill}%
\end{pgfscope}%
\begin{pgfscope}%
\pgfpathrectangle{\pgfqpoint{1.254980in}{0.150000in}}{\pgfqpoint{5.490039in}{5.490039in}}%
\pgfusepath{clip}%
\pgfsetbuttcap%
\pgfsetroundjoin%
\definecolor{currentfill}{rgb}{0.281446,0.084320,0.407414}%
\pgfsetfillcolor{currentfill}%
\pgfsetfillopacity{0.700000}%
\pgfsetlinewidth{0.000000pt}%
\definecolor{currentstroke}{rgb}{0.000000,0.000000,0.000000}%
\pgfsetstrokecolor{currentstroke}%
\pgfsetdash{}{0pt}%
\pgfpathmoveto{\pgfqpoint{3.812808in}{2.100584in}}%
\pgfpathlineto{\pgfqpoint{3.825947in}{2.093779in}}%
\pgfpathlineto{\pgfqpoint{3.839091in}{2.087108in}}%
\pgfpathlineto{\pgfqpoint{3.852238in}{2.080571in}}%
\pgfpathlineto{\pgfqpoint{3.865389in}{2.074167in}}%
\pgfpathlineto{\pgfqpoint{3.873076in}{2.082678in}}%
\pgfpathlineto{\pgfqpoint{3.880757in}{2.091238in}}%
\pgfpathlineto{\pgfqpoint{3.888432in}{2.099847in}}%
\pgfpathlineto{\pgfqpoint{3.896101in}{2.108502in}}%
\pgfpathlineto{\pgfqpoint{3.882964in}{2.114740in}}%
\pgfpathlineto{\pgfqpoint{3.869830in}{2.121111in}}%
\pgfpathlineto{\pgfqpoint{3.856701in}{2.127616in}}%
\pgfpathlineto{\pgfqpoint{3.843575in}{2.134255in}}%
\pgfpathlineto{\pgfqpoint{3.835892in}{2.125759in}}%
\pgfpathlineto{\pgfqpoint{3.828203in}{2.117315in}}%
\pgfpathlineto{\pgfqpoint{3.820508in}{2.108923in}}%
\pgfpathlineto{\pgfqpoint{3.812808in}{2.100584in}}%
\pgfpathclose%
\pgfusepath{fill}%
\end{pgfscope}%
\begin{pgfscope}%
\pgfpathrectangle{\pgfqpoint{1.254980in}{0.150000in}}{\pgfqpoint{5.490039in}{5.490039in}}%
\pgfusepath{clip}%
\pgfsetbuttcap%
\pgfsetroundjoin%
\definecolor{currentfill}{rgb}{0.197636,0.391528,0.554969}%
\pgfsetfillcolor{currentfill}%
\pgfsetfillopacity{0.700000}%
\pgfsetlinewidth{0.000000pt}%
\definecolor{currentstroke}{rgb}{0.000000,0.000000,0.000000}%
\pgfsetstrokecolor{currentstroke}%
\pgfsetdash{}{0pt}%
\pgfpathmoveto{\pgfqpoint{2.877211in}{2.787777in}}%
\pgfpathlineto{\pgfqpoint{2.890451in}{2.770335in}}%
\pgfpathlineto{\pgfqpoint{2.903685in}{2.753078in}}%
\pgfpathlineto{\pgfqpoint{2.916915in}{2.736006in}}%
\pgfpathlineto{\pgfqpoint{2.930139in}{2.719117in}}%
\pgfpathlineto{\pgfqpoint{2.938245in}{2.723315in}}%
\pgfpathlineto{\pgfqpoint{2.946340in}{2.727646in}}%
\pgfpathlineto{\pgfqpoint{2.954426in}{2.732107in}}%
\pgfpathlineto{\pgfqpoint{2.962502in}{2.736698in}}%
\pgfpathlineto{\pgfqpoint{2.949306in}{2.753358in}}%
\pgfpathlineto{\pgfqpoint{2.936105in}{2.770200in}}%
\pgfpathlineto{\pgfqpoint{2.922899in}{2.787227in}}%
\pgfpathlineto{\pgfqpoint{2.909688in}{2.804439in}}%
\pgfpathlineto{\pgfqpoint{2.901584in}{2.800072in}}%
\pgfpathlineto{\pgfqpoint{2.893470in}{2.795838in}}%
\pgfpathlineto{\pgfqpoint{2.885346in}{2.791739in}}%
\pgfpathlineto{\pgfqpoint{2.877211in}{2.787777in}}%
\pgfpathclose%
\pgfusepath{fill}%
\end{pgfscope}%
\begin{pgfscope}%
\pgfpathrectangle{\pgfqpoint{1.254980in}{0.150000in}}{\pgfqpoint{5.490039in}{5.490039in}}%
\pgfusepath{clip}%
\pgfsetbuttcap%
\pgfsetroundjoin%
\definecolor{currentfill}{rgb}{0.221989,0.339161,0.548752}%
\pgfsetfillcolor{currentfill}%
\pgfsetfillopacity{0.700000}%
\pgfsetlinewidth{0.000000pt}%
\definecolor{currentstroke}{rgb}{0.000000,0.000000,0.000000}%
\pgfsetstrokecolor{currentstroke}%
\pgfsetdash{}{0pt}%
\pgfpathmoveto{\pgfqpoint{5.301126in}{2.610772in}}%
\pgfpathlineto{\pgfqpoint{5.314749in}{2.613940in}}%
\pgfpathlineto{\pgfqpoint{5.328384in}{2.617221in}}%
\pgfpathlineto{\pgfqpoint{5.342032in}{2.620615in}}%
\pgfpathlineto{\pgfqpoint{5.355692in}{2.624122in}}%
\pgfpathlineto{\pgfqpoint{5.362876in}{2.633311in}}%
\pgfpathlineto{\pgfqpoint{5.370055in}{2.642469in}}%
\pgfpathlineto{\pgfqpoint{5.377228in}{2.651599in}}%
\pgfpathlineto{\pgfqpoint{5.384396in}{2.660699in}}%
\pgfpathlineto{\pgfqpoint{5.370746in}{2.657284in}}%
\pgfpathlineto{\pgfqpoint{5.357108in}{2.653981in}}%
\pgfpathlineto{\pgfqpoint{5.343484in}{2.650791in}}%
\pgfpathlineto{\pgfqpoint{5.329872in}{2.647714in}}%
\pgfpathlineto{\pgfqpoint{5.322693in}{2.638516in}}%
\pgfpathlineto{\pgfqpoint{5.315510in}{2.629294in}}%
\pgfpathlineto{\pgfqpoint{5.308321in}{2.620046in}}%
\pgfpathlineto{\pgfqpoint{5.301126in}{2.610772in}}%
\pgfpathclose%
\pgfusepath{fill}%
\end{pgfscope}%
\begin{pgfscope}%
\pgfpathrectangle{\pgfqpoint{1.254980in}{0.150000in}}{\pgfqpoint{5.490039in}{5.490039in}}%
\pgfusepath{clip}%
\pgfsetbuttcap%
\pgfsetroundjoin%
\definecolor{currentfill}{rgb}{0.280267,0.073417,0.397163}%
\pgfsetfillcolor{currentfill}%
\pgfsetfillopacity{0.700000}%
\pgfsetlinewidth{0.000000pt}%
\definecolor{currentstroke}{rgb}{0.000000,0.000000,0.000000}%
\pgfsetstrokecolor{currentstroke}%
\pgfsetdash{}{0pt}%
\pgfpathmoveto{\pgfqpoint{3.948695in}{2.084874in}}%
\pgfpathlineto{\pgfqpoint{3.961855in}{2.079295in}}%
\pgfpathlineto{\pgfqpoint{3.975019in}{2.073847in}}%
\pgfpathlineto{\pgfqpoint{3.988189in}{2.068529in}}%
\pgfpathlineto{\pgfqpoint{4.001363in}{2.063341in}}%
\pgfpathlineto{\pgfqpoint{4.009002in}{2.072354in}}%
\pgfpathlineto{\pgfqpoint{4.016635in}{2.081405in}}%
\pgfpathlineto{\pgfqpoint{4.024263in}{2.090491in}}%
\pgfpathlineto{\pgfqpoint{4.031886in}{2.099612in}}%
\pgfpathlineto{\pgfqpoint{4.018723in}{2.104650in}}%
\pgfpathlineto{\pgfqpoint{4.005566in}{2.109818in}}%
\pgfpathlineto{\pgfqpoint{3.992414in}{2.115116in}}%
\pgfpathlineto{\pgfqpoint{3.979267in}{2.120545in}}%
\pgfpathlineto{\pgfqpoint{3.971632in}{2.111568in}}%
\pgfpathlineto{\pgfqpoint{3.963992in}{2.102630in}}%
\pgfpathlineto{\pgfqpoint{3.956346in}{2.093731in}}%
\pgfpathlineto{\pgfqpoint{3.948695in}{2.084874in}}%
\pgfpathclose%
\pgfusepath{fill}%
\end{pgfscope}%
\begin{pgfscope}%
\pgfpathrectangle{\pgfqpoint{1.254980in}{0.150000in}}{\pgfqpoint{5.490039in}{5.490039in}}%
\pgfusepath{clip}%
\pgfsetbuttcap%
\pgfsetroundjoin%
\definecolor{currentfill}{rgb}{0.282884,0.135920,0.453427}%
\pgfsetfillcolor{currentfill}%
\pgfsetfillopacity{0.700000}%
\pgfsetlinewidth{0.000000pt}%
\definecolor{currentstroke}{rgb}{0.000000,0.000000,0.000000}%
\pgfsetstrokecolor{currentstroke}%
\pgfsetdash{}{0pt}%
\pgfpathmoveto{\pgfqpoint{3.488137in}{2.208727in}}%
\pgfpathlineto{\pgfqpoint{3.501262in}{2.198737in}}%
\pgfpathlineto{\pgfqpoint{3.514389in}{2.188892in}}%
\pgfpathlineto{\pgfqpoint{3.527517in}{2.179194in}}%
\pgfpathlineto{\pgfqpoint{3.540646in}{2.169639in}}%
\pgfpathlineto{\pgfqpoint{3.548463in}{2.176683in}}%
\pgfpathlineto{\pgfqpoint{3.556272in}{2.183807in}}%
\pgfpathlineto{\pgfqpoint{3.564075in}{2.191010in}}%
\pgfpathlineto{\pgfqpoint{3.571871in}{2.198290in}}%
\pgfpathlineto{\pgfqpoint{3.558760in}{2.207643in}}%
\pgfpathlineto{\pgfqpoint{3.545651in}{2.217142in}}%
\pgfpathlineto{\pgfqpoint{3.532543in}{2.226785in}}%
\pgfpathlineto{\pgfqpoint{3.519436in}{2.236574in}}%
\pgfpathlineto{\pgfqpoint{3.511621in}{2.229489in}}%
\pgfpathlineto{\pgfqpoint{3.503800in}{2.222485in}}%
\pgfpathlineto{\pgfqpoint{3.495972in}{2.215564in}}%
\pgfpathlineto{\pgfqpoint{3.488137in}{2.208727in}}%
\pgfpathclose%
\pgfusepath{fill}%
\end{pgfscope}%
\begin{pgfscope}%
\pgfpathrectangle{\pgfqpoint{1.254980in}{0.150000in}}{\pgfqpoint{5.490039in}{5.490039in}}%
\pgfusepath{clip}%
\pgfsetbuttcap%
\pgfsetroundjoin%
\definecolor{currentfill}{rgb}{0.283072,0.130895,0.449241}%
\pgfsetfillcolor{currentfill}%
\pgfsetfillopacity{0.700000}%
\pgfsetlinewidth{0.000000pt}%
\definecolor{currentstroke}{rgb}{0.000000,0.000000,0.000000}%
\pgfsetstrokecolor{currentstroke}%
\pgfsetdash{}{0pt}%
\pgfpathmoveto{\pgfqpoint{4.469904in}{2.177136in}}%
\pgfpathlineto{\pgfqpoint{4.483193in}{2.175658in}}%
\pgfpathlineto{\pgfqpoint{4.496492in}{2.174301in}}%
\pgfpathlineto{\pgfqpoint{4.509798in}{2.173064in}}%
\pgfpathlineto{\pgfqpoint{4.523113in}{2.171947in}}%
\pgfpathlineto{\pgfqpoint{4.530584in}{2.182076in}}%
\pgfpathlineto{\pgfqpoint{4.538050in}{2.192201in}}%
\pgfpathlineto{\pgfqpoint{4.545511in}{2.202320in}}%
\pgfpathlineto{\pgfqpoint{4.552967in}{2.212433in}}%
\pgfpathlineto{\pgfqpoint{4.539660in}{2.213480in}}%
\pgfpathlineto{\pgfqpoint{4.526362in}{2.214646in}}%
\pgfpathlineto{\pgfqpoint{4.513072in}{2.215934in}}%
\pgfpathlineto{\pgfqpoint{4.499791in}{2.217341in}}%
\pgfpathlineto{\pgfqpoint{4.492326in}{2.207292in}}%
\pgfpathlineto{\pgfqpoint{4.484857in}{2.197241in}}%
\pgfpathlineto{\pgfqpoint{4.477382in}{2.187189in}}%
\pgfpathlineto{\pgfqpoint{4.469904in}{2.177136in}}%
\pgfpathclose%
\pgfusepath{fill}%
\end{pgfscope}%
\begin{pgfscope}%
\pgfpathrectangle{\pgfqpoint{1.254980in}{0.150000in}}{\pgfqpoint{5.490039in}{5.490039in}}%
\pgfusepath{clip}%
\pgfsetbuttcap%
\pgfsetroundjoin%
\definecolor{currentfill}{rgb}{0.263663,0.237631,0.518762}%
\pgfsetfillcolor{currentfill}%
\pgfsetfillopacity{0.700000}%
\pgfsetlinewidth{0.000000pt}%
\definecolor{currentstroke}{rgb}{0.000000,0.000000,0.000000}%
\pgfsetstrokecolor{currentstroke}%
\pgfsetdash{}{0pt}%
\pgfpathmoveto{\pgfqpoint{3.193809in}{2.418177in}}%
\pgfpathlineto{\pgfqpoint{3.206964in}{2.404895in}}%
\pgfpathlineto{\pgfqpoint{3.220117in}{2.391774in}}%
\pgfpathlineto{\pgfqpoint{3.233268in}{2.378814in}}%
\pgfpathlineto{\pgfqpoint{3.246417in}{2.366013in}}%
\pgfpathlineto{\pgfqpoint{3.254368in}{2.371596in}}%
\pgfpathlineto{\pgfqpoint{3.262310in}{2.377287in}}%
\pgfpathlineto{\pgfqpoint{3.270245in}{2.383083in}}%
\pgfpathlineto{\pgfqpoint{3.278171in}{2.388983in}}%
\pgfpathlineto{\pgfqpoint{3.265044in}{2.401562in}}%
\pgfpathlineto{\pgfqpoint{3.251916in}{2.414302in}}%
\pgfpathlineto{\pgfqpoint{3.238787in}{2.427201in}}%
\pgfpathlineto{\pgfqpoint{3.225656in}{2.440261in}}%
\pgfpathlineto{\pgfqpoint{3.217707in}{2.434577in}}%
\pgfpathlineto{\pgfqpoint{3.209750in}{2.429000in}}%
\pgfpathlineto{\pgfqpoint{3.201784in}{2.423533in}}%
\pgfpathlineto{\pgfqpoint{3.193809in}{2.418177in}}%
\pgfpathclose%
\pgfusepath{fill}%
\end{pgfscope}%
\begin{pgfscope}%
\pgfpathrectangle{\pgfqpoint{1.254980in}{0.150000in}}{\pgfqpoint{5.490039in}{5.490039in}}%
\pgfusepath{clip}%
\pgfsetbuttcap%
\pgfsetroundjoin%
\definecolor{currentfill}{rgb}{0.210503,0.363727,0.552206}%
\pgfsetfillcolor{currentfill}%
\pgfsetfillopacity{0.700000}%
\pgfsetlinewidth{0.000000pt}%
\definecolor{currentstroke}{rgb}{0.000000,0.000000,0.000000}%
\pgfsetstrokecolor{currentstroke}%
\pgfsetdash{}{0pt}%
\pgfpathmoveto{\pgfqpoint{5.384396in}{2.660699in}}%
\pgfpathlineto{\pgfqpoint{5.398059in}{2.664228in}}%
\pgfpathlineto{\pgfqpoint{5.411735in}{2.667869in}}%
\pgfpathlineto{\pgfqpoint{5.425423in}{2.671623in}}%
\pgfpathlineto{\pgfqpoint{5.439125in}{2.675490in}}%
\pgfpathlineto{\pgfqpoint{5.446277in}{2.684462in}}%
\pgfpathlineto{\pgfqpoint{5.453423in}{2.693404in}}%
\pgfpathlineto{\pgfqpoint{5.460563in}{2.702317in}}%
\pgfpathlineto{\pgfqpoint{5.467698in}{2.711203in}}%
\pgfpathlineto{\pgfqpoint{5.454008in}{2.707444in}}%
\pgfpathlineto{\pgfqpoint{5.440330in}{2.703798in}}%
\pgfpathlineto{\pgfqpoint{5.426665in}{2.700264in}}%
\pgfpathlineto{\pgfqpoint{5.413013in}{2.696843in}}%
\pgfpathlineto{\pgfqpoint{5.405867in}{2.687844in}}%
\pgfpathlineto{\pgfqpoint{5.398715in}{2.678821in}}%
\pgfpathlineto{\pgfqpoint{5.391558in}{2.669773in}}%
\pgfpathlineto{\pgfqpoint{5.384396in}{2.660699in}}%
\pgfpathclose%
\pgfusepath{fill}%
\end{pgfscope}%
\begin{pgfscope}%
\pgfpathrectangle{\pgfqpoint{1.254980in}{0.150000in}}{\pgfqpoint{5.490039in}{5.490039in}}%
\pgfusepath{clip}%
\pgfsetbuttcap%
\pgfsetroundjoin%
\definecolor{currentfill}{rgb}{0.282327,0.094955,0.417331}%
\pgfsetfillcolor{currentfill}%
\pgfsetfillopacity{0.700000}%
\pgfsetlinewidth{0.000000pt}%
\definecolor{currentstroke}{rgb}{0.000000,0.000000,0.000000}%
\pgfsetstrokecolor{currentstroke}%
\pgfsetdash{}{0pt}%
\pgfpathmoveto{\pgfqpoint{3.676827in}{2.128586in}}%
\pgfpathlineto{\pgfqpoint{3.689957in}{2.120506in}}%
\pgfpathlineto{\pgfqpoint{3.703089in}{2.112563in}}%
\pgfpathlineto{\pgfqpoint{3.716225in}{2.104759in}}%
\pgfpathlineto{\pgfqpoint{3.729363in}{2.097092in}}%
\pgfpathlineto{\pgfqpoint{3.737102in}{2.105016in}}%
\pgfpathlineto{\pgfqpoint{3.744836in}{2.113003in}}%
\pgfpathlineto{\pgfqpoint{3.752563in}{2.121051in}}%
\pgfpathlineto{\pgfqpoint{3.760284in}{2.129158in}}%
\pgfpathlineto{\pgfqpoint{3.747161in}{2.136642in}}%
\pgfpathlineto{\pgfqpoint{3.734042in}{2.144264in}}%
\pgfpathlineto{\pgfqpoint{3.720925in}{2.152023in}}%
\pgfpathlineto{\pgfqpoint{3.707811in}{2.159921in}}%
\pgfpathlineto{\pgfqpoint{3.700075in}{2.151990in}}%
\pgfpathlineto{\pgfqpoint{3.692332in}{2.144124in}}%
\pgfpathlineto{\pgfqpoint{3.684583in}{2.136322in}}%
\pgfpathlineto{\pgfqpoint{3.676827in}{2.128586in}}%
\pgfpathclose%
\pgfusepath{fill}%
\end{pgfscope}%
\begin{pgfscope}%
\pgfpathrectangle{\pgfqpoint{1.254980in}{0.150000in}}{\pgfqpoint{5.490039in}{5.490039in}}%
\pgfusepath{clip}%
\pgfsetbuttcap%
\pgfsetroundjoin%
\definecolor{currentfill}{rgb}{0.183898,0.422383,0.556944}%
\pgfsetfillcolor{currentfill}%
\pgfsetfillopacity{0.700000}%
\pgfsetlinewidth{0.000000pt}%
\definecolor{currentstroke}{rgb}{0.000000,0.000000,0.000000}%
\pgfsetstrokecolor{currentstroke}%
\pgfsetdash{}{0pt}%
\pgfpathmoveto{\pgfqpoint{2.824195in}{2.859425in}}%
\pgfpathlineto{\pgfqpoint{2.837458in}{2.841229in}}%
\pgfpathlineto{\pgfqpoint{2.850715in}{2.823223in}}%
\pgfpathlineto{\pgfqpoint{2.863966in}{2.805406in}}%
\pgfpathlineto{\pgfqpoint{2.877211in}{2.787777in}}%
\pgfpathlineto{\pgfqpoint{2.885346in}{2.791739in}}%
\pgfpathlineto{\pgfqpoint{2.893470in}{2.795838in}}%
\pgfpathlineto{\pgfqpoint{2.901584in}{2.800072in}}%
\pgfpathlineto{\pgfqpoint{2.909688in}{2.804439in}}%
\pgfpathlineto{\pgfqpoint{2.896472in}{2.821838in}}%
\pgfpathlineto{\pgfqpoint{2.883250in}{2.839424in}}%
\pgfpathlineto{\pgfqpoint{2.870023in}{2.857198in}}%
\pgfpathlineto{\pgfqpoint{2.856790in}{2.875163in}}%
\pgfpathlineto{\pgfqpoint{2.848657in}{2.871020in}}%
\pgfpathlineto{\pgfqpoint{2.840514in}{2.867015in}}%
\pgfpathlineto{\pgfqpoint{2.832360in}{2.863150in}}%
\pgfpathlineto{\pgfqpoint{2.824195in}{2.859425in}}%
\pgfpathclose%
\pgfusepath{fill}%
\end{pgfscope}%
\begin{pgfscope}%
\pgfpathrectangle{\pgfqpoint{1.254980in}{0.150000in}}{\pgfqpoint{5.490039in}{5.490039in}}%
\pgfusepath{clip}%
\pgfsetbuttcap%
\pgfsetroundjoin%
\definecolor{currentfill}{rgb}{0.201239,0.383670,0.554294}%
\pgfsetfillcolor{currentfill}%
\pgfsetfillopacity{0.700000}%
\pgfsetlinewidth{0.000000pt}%
\definecolor{currentstroke}{rgb}{0.000000,0.000000,0.000000}%
\pgfsetstrokecolor{currentstroke}%
\pgfsetdash{}{0pt}%
\pgfpathmoveto{\pgfqpoint{5.467698in}{2.711203in}}%
\pgfpathlineto{\pgfqpoint{5.481402in}{2.715073in}}%
\pgfpathlineto{\pgfqpoint{5.495119in}{2.719057in}}%
\pgfpathlineto{\pgfqpoint{5.508850in}{2.723152in}}%
\pgfpathlineto{\pgfqpoint{5.522594in}{2.727359in}}%
\pgfpathlineto{\pgfqpoint{5.529712in}{2.736101in}}%
\pgfpathlineto{\pgfqpoint{5.536824in}{2.744813in}}%
\pgfpathlineto{\pgfqpoint{5.543931in}{2.753498in}}%
\pgfpathlineto{\pgfqpoint{5.551033in}{2.762156in}}%
\pgfpathlineto{\pgfqpoint{5.537301in}{2.758074in}}%
\pgfpathlineto{\pgfqpoint{5.523582in}{2.754103in}}%
\pgfpathlineto{\pgfqpoint{5.509877in}{2.750244in}}%
\pgfpathlineto{\pgfqpoint{5.496185in}{2.746497in}}%
\pgfpathlineto{\pgfqpoint{5.489071in}{2.737708in}}%
\pgfpathlineto{\pgfqpoint{5.481952in}{2.728897in}}%
\pgfpathlineto{\pgfqpoint{5.474828in}{2.720062in}}%
\pgfpathlineto{\pgfqpoint{5.467698in}{2.711203in}}%
\pgfpathclose%
\pgfusepath{fill}%
\end{pgfscope}%
\begin{pgfscope}%
\pgfpathrectangle{\pgfqpoint{1.254980in}{0.150000in}}{\pgfqpoint{5.490039in}{5.490039in}}%
\pgfusepath{clip}%
\pgfsetbuttcap%
\pgfsetroundjoin%
\definecolor{currentfill}{rgb}{0.283197,0.115680,0.436115}%
\pgfsetfillcolor{currentfill}%
\pgfsetfillopacity{0.700000}%
\pgfsetlinewidth{0.000000pt}%
\definecolor{currentstroke}{rgb}{0.000000,0.000000,0.000000}%
\pgfsetstrokecolor{currentstroke}%
\pgfsetdash{}{0pt}%
\pgfpathmoveto{\pgfqpoint{4.386825in}{2.144394in}}%
\pgfpathlineto{\pgfqpoint{4.400092in}{2.142344in}}%
\pgfpathlineto{\pgfqpoint{4.413367in}{2.140416in}}%
\pgfpathlineto{\pgfqpoint{4.426649in}{2.138610in}}%
\pgfpathlineto{\pgfqpoint{4.439940in}{2.136924in}}%
\pgfpathlineto{\pgfqpoint{4.447438in}{2.146977in}}%
\pgfpathlineto{\pgfqpoint{4.454931in}{2.157029in}}%
\pgfpathlineto{\pgfqpoint{4.462420in}{2.167083in}}%
\pgfpathlineto{\pgfqpoint{4.469904in}{2.177136in}}%
\pgfpathlineto{\pgfqpoint{4.456622in}{2.178735in}}%
\pgfpathlineto{\pgfqpoint{4.443348in}{2.180456in}}%
\pgfpathlineto{\pgfqpoint{4.430082in}{2.182298in}}%
\pgfpathlineto{\pgfqpoint{4.416824in}{2.184261in}}%
\pgfpathlineto{\pgfqpoint{4.409332in}{2.174288in}}%
\pgfpathlineto{\pgfqpoint{4.401834in}{2.164319in}}%
\pgfpathlineto{\pgfqpoint{4.394332in}{2.154354in}}%
\pgfpathlineto{\pgfqpoint{4.386825in}{2.144394in}}%
\pgfpathclose%
\pgfusepath{fill}%
\end{pgfscope}%
\begin{pgfscope}%
\pgfpathrectangle{\pgfqpoint{1.254980in}{0.150000in}}{\pgfqpoint{5.490039in}{5.490039in}}%
\pgfusepath{clip}%
\pgfsetbuttcap%
\pgfsetroundjoin%
\definecolor{currentfill}{rgb}{0.270595,0.214069,0.507052}%
\pgfsetfillcolor{currentfill}%
\pgfsetfillopacity{0.700000}%
\pgfsetlinewidth{0.000000pt}%
\definecolor{currentstroke}{rgb}{0.000000,0.000000,0.000000}%
\pgfsetstrokecolor{currentstroke}%
\pgfsetdash{}{0pt}%
\pgfpathmoveto{\pgfqpoint{3.246417in}{2.366013in}}%
\pgfpathlineto{\pgfqpoint{3.259566in}{2.353371in}}%
\pgfpathlineto{\pgfqpoint{3.272713in}{2.340887in}}%
\pgfpathlineto{\pgfqpoint{3.285859in}{2.328560in}}%
\pgfpathlineto{\pgfqpoint{3.299004in}{2.316389in}}%
\pgfpathlineto{\pgfqpoint{3.306932in}{2.322199in}}%
\pgfpathlineto{\pgfqpoint{3.314852in}{2.328112in}}%
\pgfpathlineto{\pgfqpoint{3.322764in}{2.334126in}}%
\pgfpathlineto{\pgfqpoint{3.330668in}{2.340240in}}%
\pgfpathlineto{\pgfqpoint{3.317545in}{2.352191in}}%
\pgfpathlineto{\pgfqpoint{3.304421in}{2.364298in}}%
\pgfpathlineto{\pgfqpoint{3.291296in}{2.376561in}}%
\pgfpathlineto{\pgfqpoint{3.278171in}{2.388983in}}%
\pgfpathlineto{\pgfqpoint{3.270245in}{2.383083in}}%
\pgfpathlineto{\pgfqpoint{3.262310in}{2.377287in}}%
\pgfpathlineto{\pgfqpoint{3.254368in}{2.371596in}}%
\pgfpathlineto{\pgfqpoint{3.246417in}{2.366013in}}%
\pgfpathclose%
\pgfusepath{fill}%
\end{pgfscope}%
\begin{pgfscope}%
\pgfpathrectangle{\pgfqpoint{1.254980in}{0.150000in}}{\pgfqpoint{5.490039in}{5.490039in}}%
\pgfusepath{clip}%
\pgfsetbuttcap%
\pgfsetroundjoin%
\definecolor{currentfill}{rgb}{0.280894,0.078907,0.402329}%
\pgfsetfillcolor{currentfill}%
\pgfsetfillopacity{0.700000}%
\pgfsetlinewidth{0.000000pt}%
\definecolor{currentstroke}{rgb}{0.000000,0.000000,0.000000}%
\pgfsetstrokecolor{currentstroke}%
\pgfsetdash{}{0pt}%
\pgfpathmoveto{\pgfqpoint{4.084588in}{2.080746in}}%
\pgfpathlineto{\pgfqpoint{4.097777in}{2.076349in}}%
\pgfpathlineto{\pgfqpoint{4.110973in}{2.072080in}}%
\pgfpathlineto{\pgfqpoint{4.124174in}{2.067937in}}%
\pgfpathlineto{\pgfqpoint{4.137381in}{2.063921in}}%
\pgfpathlineto{\pgfqpoint{4.144976in}{2.073356in}}%
\pgfpathlineto{\pgfqpoint{4.152565in}{2.082815in}}%
\pgfpathlineto{\pgfqpoint{4.160150in}{2.092299in}}%
\pgfpathlineto{\pgfqpoint{4.167730in}{2.101805in}}%
\pgfpathlineto{\pgfqpoint{4.154534in}{2.105688in}}%
\pgfpathlineto{\pgfqpoint{4.141344in}{2.109696in}}%
\pgfpathlineto{\pgfqpoint{4.128159in}{2.113832in}}%
\pgfpathlineto{\pgfqpoint{4.114981in}{2.118095in}}%
\pgfpathlineto{\pgfqpoint{4.107390in}{2.108716in}}%
\pgfpathlineto{\pgfqpoint{4.099795in}{2.099364in}}%
\pgfpathlineto{\pgfqpoint{4.092194in}{2.090040in}}%
\pgfpathlineto{\pgfqpoint{4.084588in}{2.080746in}}%
\pgfpathclose%
\pgfusepath{fill}%
\end{pgfscope}%
\begin{pgfscope}%
\pgfpathrectangle{\pgfqpoint{1.254980in}{0.150000in}}{\pgfqpoint{5.490039in}{5.490039in}}%
\pgfusepath{clip}%
\pgfsetbuttcap%
\pgfsetroundjoin%
\definecolor{currentfill}{rgb}{0.190631,0.407061,0.556089}%
\pgfsetfillcolor{currentfill}%
\pgfsetfillopacity{0.700000}%
\pgfsetlinewidth{0.000000pt}%
\definecolor{currentstroke}{rgb}{0.000000,0.000000,0.000000}%
\pgfsetstrokecolor{currentstroke}%
\pgfsetdash{}{0pt}%
\pgfpathmoveto{\pgfqpoint{5.551033in}{2.762156in}}%
\pgfpathlineto{\pgfqpoint{5.564778in}{2.766351in}}%
\pgfpathlineto{\pgfqpoint{5.578538in}{2.770658in}}%
\pgfpathlineto{\pgfqpoint{5.592311in}{2.775076in}}%
\pgfpathlineto{\pgfqpoint{5.606097in}{2.779606in}}%
\pgfpathlineto{\pgfqpoint{5.613181in}{2.788105in}}%
\pgfpathlineto{\pgfqpoint{5.620259in}{2.796577in}}%
\pgfpathlineto{\pgfqpoint{5.627331in}{2.805023in}}%
\pgfpathlineto{\pgfqpoint{5.634398in}{2.813445in}}%
\pgfpathlineto{\pgfqpoint{5.620624in}{2.809057in}}%
\pgfpathlineto{\pgfqpoint{5.606864in}{2.804779in}}%
\pgfpathlineto{\pgfqpoint{5.593117in}{2.800613in}}%
\pgfpathlineto{\pgfqpoint{5.579384in}{2.796559in}}%
\pgfpathlineto{\pgfqpoint{5.572304in}{2.787990in}}%
\pgfpathlineto{\pgfqpoint{5.565219in}{2.779401in}}%
\pgfpathlineto{\pgfqpoint{5.558129in}{2.770790in}}%
\pgfpathlineto{\pgfqpoint{5.551033in}{2.762156in}}%
\pgfpathclose%
\pgfusepath{fill}%
\end{pgfscope}%
\begin{pgfscope}%
\pgfpathrectangle{\pgfqpoint{1.254980in}{0.150000in}}{\pgfqpoint{5.490039in}{5.490039in}}%
\pgfusepath{clip}%
\pgfsetbuttcap%
\pgfsetroundjoin%
\definecolor{currentfill}{rgb}{0.172719,0.448791,0.557885}%
\pgfsetfillcolor{currentfill}%
\pgfsetfillopacity{0.700000}%
\pgfsetlinewidth{0.000000pt}%
\definecolor{currentstroke}{rgb}{0.000000,0.000000,0.000000}%
\pgfsetstrokecolor{currentstroke}%
\pgfsetdash{}{0pt}%
\pgfpathmoveto{\pgfqpoint{2.771080in}{2.934142in}}%
\pgfpathlineto{\pgfqpoint{2.784368in}{2.915171in}}%
\pgfpathlineto{\pgfqpoint{2.797650in}{2.896395in}}%
\pgfpathlineto{\pgfqpoint{2.810926in}{2.877814in}}%
\pgfpathlineto{\pgfqpoint{2.824195in}{2.859425in}}%
\pgfpathlineto{\pgfqpoint{2.832360in}{2.863150in}}%
\pgfpathlineto{\pgfqpoint{2.840514in}{2.867015in}}%
\pgfpathlineto{\pgfqpoint{2.848657in}{2.871020in}}%
\pgfpathlineto{\pgfqpoint{2.856790in}{2.875163in}}%
\pgfpathlineto{\pgfqpoint{2.843551in}{2.893319in}}%
\pgfpathlineto{\pgfqpoint{2.830306in}{2.911668in}}%
\pgfpathlineto{\pgfqpoint{2.817054in}{2.930210in}}%
\pgfpathlineto{\pgfqpoint{2.803796in}{2.948948in}}%
\pgfpathlineto{\pgfqpoint{2.795633in}{2.945032in}}%
\pgfpathlineto{\pgfqpoint{2.787460in}{2.941257in}}%
\pgfpathlineto{\pgfqpoint{2.779275in}{2.937627in}}%
\pgfpathlineto{\pgfqpoint{2.771080in}{2.934142in}}%
\pgfpathclose%
\pgfusepath{fill}%
\end{pgfscope}%
\begin{pgfscope}%
\pgfpathrectangle{\pgfqpoint{1.254980in}{0.150000in}}{\pgfqpoint{5.490039in}{5.490039in}}%
\pgfusepath{clip}%
\pgfsetbuttcap%
\pgfsetroundjoin%
\definecolor{currentfill}{rgb}{0.182256,0.426184,0.557120}%
\pgfsetfillcolor{currentfill}%
\pgfsetfillopacity{0.700000}%
\pgfsetlinewidth{0.000000pt}%
\definecolor{currentstroke}{rgb}{0.000000,0.000000,0.000000}%
\pgfsetstrokecolor{currentstroke}%
\pgfsetdash{}{0pt}%
\pgfpathmoveto{\pgfqpoint{5.634398in}{2.813445in}}%
\pgfpathlineto{\pgfqpoint{5.648186in}{2.817946in}}%
\pgfpathlineto{\pgfqpoint{5.661987in}{2.822557in}}%
\pgfpathlineto{\pgfqpoint{5.675803in}{2.827280in}}%
\pgfpathlineto{\pgfqpoint{5.689633in}{2.832115in}}%
\pgfpathlineto{\pgfqpoint{5.696681in}{2.840362in}}%
\pgfpathlineto{\pgfqpoint{5.703724in}{2.848585in}}%
\pgfpathlineto{\pgfqpoint{5.710760in}{2.856785in}}%
\pgfpathlineto{\pgfqpoint{5.717792in}{2.864963in}}%
\pgfpathlineto{\pgfqpoint{5.703975in}{2.860286in}}%
\pgfpathlineto{\pgfqpoint{5.690173in}{2.855721in}}%
\pgfpathlineto{\pgfqpoint{5.676385in}{2.851267in}}%
\pgfpathlineto{\pgfqpoint{5.662611in}{2.846924in}}%
\pgfpathlineto{\pgfqpoint{5.655566in}{2.838583in}}%
\pgfpathlineto{\pgfqpoint{5.648515in}{2.830223in}}%
\pgfpathlineto{\pgfqpoint{5.641459in}{2.821845in}}%
\pgfpathlineto{\pgfqpoint{5.634398in}{2.813445in}}%
\pgfpathclose%
\pgfusepath{fill}%
\end{pgfscope}%
\begin{pgfscope}%
\pgfpathrectangle{\pgfqpoint{1.254980in}{0.150000in}}{\pgfqpoint{5.490039in}{5.490039in}}%
\pgfusepath{clip}%
\pgfsetbuttcap%
\pgfsetroundjoin%
\definecolor{currentfill}{rgb}{0.283229,0.120777,0.440584}%
\pgfsetfillcolor{currentfill}%
\pgfsetfillopacity{0.700000}%
\pgfsetlinewidth{0.000000pt}%
\definecolor{currentstroke}{rgb}{0.000000,0.000000,0.000000}%
\pgfsetstrokecolor{currentstroke}%
\pgfsetdash{}{0pt}%
\pgfpathmoveto{\pgfqpoint{3.540646in}{2.169639in}}%
\pgfpathlineto{\pgfqpoint{3.553777in}{2.160229in}}%
\pgfpathlineto{\pgfqpoint{3.566909in}{2.150962in}}%
\pgfpathlineto{\pgfqpoint{3.580043in}{2.141838in}}%
\pgfpathlineto{\pgfqpoint{3.593179in}{2.132856in}}%
\pgfpathlineto{\pgfqpoint{3.600977in}{2.140106in}}%
\pgfpathlineto{\pgfqpoint{3.608769in}{2.147432in}}%
\pgfpathlineto{\pgfqpoint{3.616554in}{2.154833in}}%
\pgfpathlineto{\pgfqpoint{3.624332in}{2.162307in}}%
\pgfpathlineto{\pgfqpoint{3.611214in}{2.171089in}}%
\pgfpathlineto{\pgfqpoint{3.598098in}{2.180013in}}%
\pgfpathlineto{\pgfqpoint{3.584984in}{2.189080in}}%
\pgfpathlineto{\pgfqpoint{3.571871in}{2.198290in}}%
\pgfpathlineto{\pgfqpoint{3.564075in}{2.191010in}}%
\pgfpathlineto{\pgfqpoint{3.556272in}{2.183807in}}%
\pgfpathlineto{\pgfqpoint{3.548463in}{2.176683in}}%
\pgfpathlineto{\pgfqpoint{3.540646in}{2.169639in}}%
\pgfpathclose%
\pgfusepath{fill}%
\end{pgfscope}%
\begin{pgfscope}%
\pgfpathrectangle{\pgfqpoint{1.254980in}{0.150000in}}{\pgfqpoint{5.490039in}{5.490039in}}%
\pgfusepath{clip}%
\pgfsetbuttcap%
\pgfsetroundjoin%
\definecolor{currentfill}{rgb}{0.282656,0.100196,0.422160}%
\pgfsetfillcolor{currentfill}%
\pgfsetfillopacity{0.700000}%
\pgfsetlinewidth{0.000000pt}%
\definecolor{currentstroke}{rgb}{0.000000,0.000000,0.000000}%
\pgfsetstrokecolor{currentstroke}%
\pgfsetdash{}{0pt}%
\pgfpathmoveto{\pgfqpoint{4.303721in}{2.114444in}}%
\pgfpathlineto{\pgfqpoint{4.316968in}{2.111802in}}%
\pgfpathlineto{\pgfqpoint{4.330221in}{2.109282in}}%
\pgfpathlineto{\pgfqpoint{4.343482in}{2.106886in}}%
\pgfpathlineto{\pgfqpoint{4.356750in}{2.104612in}}%
\pgfpathlineto{\pgfqpoint{4.364276in}{2.114548in}}%
\pgfpathlineto{\pgfqpoint{4.371797in}{2.124490in}}%
\pgfpathlineto{\pgfqpoint{4.379314in}{2.134439in}}%
\pgfpathlineto{\pgfqpoint{4.386825in}{2.144394in}}%
\pgfpathlineto{\pgfqpoint{4.373566in}{2.146566in}}%
\pgfpathlineto{\pgfqpoint{4.360315in}{2.148860in}}%
\pgfpathlineto{\pgfqpoint{4.347070in}{2.151278in}}%
\pgfpathlineto{\pgfqpoint{4.333833in}{2.153818in}}%
\pgfpathlineto{\pgfqpoint{4.326313in}{2.143959in}}%
\pgfpathlineto{\pgfqpoint{4.318787in}{2.134110in}}%
\pgfpathlineto{\pgfqpoint{4.311257in}{2.124271in}}%
\pgfpathlineto{\pgfqpoint{4.303721in}{2.114444in}}%
\pgfpathclose%
\pgfusepath{fill}%
\end{pgfscope}%
\begin{pgfscope}%
\pgfpathrectangle{\pgfqpoint{1.254980in}{0.150000in}}{\pgfqpoint{5.490039in}{5.490039in}}%
\pgfusepath{clip}%
\pgfsetbuttcap%
\pgfsetroundjoin%
\definecolor{currentfill}{rgb}{0.276194,0.190074,0.493001}%
\pgfsetfillcolor{currentfill}%
\pgfsetfillopacity{0.700000}%
\pgfsetlinewidth{0.000000pt}%
\definecolor{currentstroke}{rgb}{0.000000,0.000000,0.000000}%
\pgfsetstrokecolor{currentstroke}%
\pgfsetdash{}{0pt}%
\pgfpathmoveto{\pgfqpoint{3.299004in}{2.316389in}}%
\pgfpathlineto{\pgfqpoint{3.312148in}{2.304374in}}%
\pgfpathlineto{\pgfqpoint{3.325292in}{2.292513in}}%
\pgfpathlineto{\pgfqpoint{3.338436in}{2.280807in}}%
\pgfpathlineto{\pgfqpoint{3.351578in}{2.269253in}}%
\pgfpathlineto{\pgfqpoint{3.359484in}{2.275289in}}%
\pgfpathlineto{\pgfqpoint{3.367382in}{2.281423in}}%
\pgfpathlineto{\pgfqpoint{3.375273in}{2.287655in}}%
\pgfpathlineto{\pgfqpoint{3.383156in}{2.293982in}}%
\pgfpathlineto{\pgfqpoint{3.370034in}{2.305316in}}%
\pgfpathlineto{\pgfqpoint{3.356912in}{2.316803in}}%
\pgfpathlineto{\pgfqpoint{3.343790in}{2.328444in}}%
\pgfpathlineto{\pgfqpoint{3.330668in}{2.340240in}}%
\pgfpathlineto{\pgfqpoint{3.322764in}{2.334126in}}%
\pgfpathlineto{\pgfqpoint{3.314852in}{2.328112in}}%
\pgfpathlineto{\pgfqpoint{3.306932in}{2.322199in}}%
\pgfpathlineto{\pgfqpoint{3.299004in}{2.316389in}}%
\pgfpathclose%
\pgfusepath{fill}%
\end{pgfscope}%
\begin{pgfscope}%
\pgfpathrectangle{\pgfqpoint{1.254980in}{0.150000in}}{\pgfqpoint{5.490039in}{5.490039in}}%
\pgfusepath{clip}%
\pgfsetbuttcap%
\pgfsetroundjoin%
\definecolor{currentfill}{rgb}{0.172719,0.448791,0.557885}%
\pgfsetfillcolor{currentfill}%
\pgfsetfillopacity{0.700000}%
\pgfsetlinewidth{0.000000pt}%
\definecolor{currentstroke}{rgb}{0.000000,0.000000,0.000000}%
\pgfsetstrokecolor{currentstroke}%
\pgfsetdash{}{0pt}%
\pgfpathmoveto{\pgfqpoint{5.717792in}{2.864963in}}%
\pgfpathlineto{\pgfqpoint{5.731622in}{2.869750in}}%
\pgfpathlineto{\pgfqpoint{5.745467in}{2.874648in}}%
\pgfpathlineto{\pgfqpoint{5.759326in}{2.879658in}}%
\pgfpathlineto{\pgfqpoint{5.773199in}{2.884778in}}%
\pgfpathlineto{\pgfqpoint{5.780211in}{2.892767in}}%
\pgfpathlineto{\pgfqpoint{5.787217in}{2.900735in}}%
\pgfpathlineto{\pgfqpoint{5.794217in}{2.908682in}}%
\pgfpathlineto{\pgfqpoint{5.801212in}{2.916611in}}%
\pgfpathlineto{\pgfqpoint{5.787353in}{2.911665in}}%
\pgfpathlineto{\pgfqpoint{5.773509in}{2.906831in}}%
\pgfpathlineto{\pgfqpoint{5.759679in}{2.902107in}}%
\pgfpathlineto{\pgfqpoint{5.745863in}{2.897493in}}%
\pgfpathlineto{\pgfqpoint{5.738853in}{2.889384in}}%
\pgfpathlineto{\pgfqpoint{5.731838in}{2.881261in}}%
\pgfpathlineto{\pgfqpoint{5.724817in}{2.873121in}}%
\pgfpathlineto{\pgfqpoint{5.717792in}{2.864963in}}%
\pgfpathclose%
\pgfusepath{fill}%
\end{pgfscope}%
\begin{pgfscope}%
\pgfpathrectangle{\pgfqpoint{1.254980in}{0.150000in}}{\pgfqpoint{5.490039in}{5.490039in}}%
\pgfusepath{clip}%
\pgfsetbuttcap%
\pgfsetroundjoin%
\definecolor{currentfill}{rgb}{0.159194,0.482237,0.558073}%
\pgfsetfillcolor{currentfill}%
\pgfsetfillopacity{0.700000}%
\pgfsetlinewidth{0.000000pt}%
\definecolor{currentstroke}{rgb}{0.000000,0.000000,0.000000}%
\pgfsetstrokecolor{currentstroke}%
\pgfsetdash{}{0pt}%
\pgfpathmoveto{\pgfqpoint{2.717854in}{3.012009in}}%
\pgfpathlineto{\pgfqpoint{2.731171in}{2.992242in}}%
\pgfpathlineto{\pgfqpoint{2.744481in}{2.972677in}}%
\pgfpathlineto{\pgfqpoint{2.757784in}{2.953310in}}%
\pgfpathlineto{\pgfqpoint{2.771080in}{2.934142in}}%
\pgfpathlineto{\pgfqpoint{2.779275in}{2.937627in}}%
\pgfpathlineto{\pgfqpoint{2.787460in}{2.941257in}}%
\pgfpathlineto{\pgfqpoint{2.795633in}{2.945032in}}%
\pgfpathlineto{\pgfqpoint{2.803796in}{2.948948in}}%
\pgfpathlineto{\pgfqpoint{2.790531in}{2.967882in}}%
\pgfpathlineto{\pgfqpoint{2.777260in}{2.987014in}}%
\pgfpathlineto{\pgfqpoint{2.763981in}{3.006345in}}%
\pgfpathlineto{\pgfqpoint{2.750695in}{3.025876in}}%
\pgfpathlineto{\pgfqpoint{2.742502in}{3.022188in}}%
\pgfpathlineto{\pgfqpoint{2.734297in}{3.018646in}}%
\pgfpathlineto{\pgfqpoint{2.726081in}{3.015253in}}%
\pgfpathlineto{\pgfqpoint{2.717854in}{3.012009in}}%
\pgfpathclose%
\pgfusepath{fill}%
\end{pgfscope}%
\begin{pgfscope}%
\pgfpathrectangle{\pgfqpoint{1.254980in}{0.150000in}}{\pgfqpoint{5.490039in}{5.490039in}}%
\pgfusepath{clip}%
\pgfsetbuttcap%
\pgfsetroundjoin%
\definecolor{currentfill}{rgb}{0.280267,0.073417,0.397163}%
\pgfsetfillcolor{currentfill}%
\pgfsetfillopacity{0.700000}%
\pgfsetlinewidth{0.000000pt}%
\definecolor{currentstroke}{rgb}{0.000000,0.000000,0.000000}%
\pgfsetstrokecolor{currentstroke}%
\pgfsetdash{}{0pt}%
\pgfpathmoveto{\pgfqpoint{3.865389in}{2.074167in}}%
\pgfpathlineto{\pgfqpoint{3.878545in}{2.067896in}}%
\pgfpathlineto{\pgfqpoint{3.891704in}{2.061757in}}%
\pgfpathlineto{\pgfqpoint{3.904868in}{2.055750in}}%
\pgfpathlineto{\pgfqpoint{3.918036in}{2.049875in}}%
\pgfpathlineto{\pgfqpoint{3.925709in}{2.058558in}}%
\pgfpathlineto{\pgfqpoint{3.933377in}{2.067286in}}%
\pgfpathlineto{\pgfqpoint{3.941039in}{2.076058in}}%
\pgfpathlineto{\pgfqpoint{3.948695in}{2.084874in}}%
\pgfpathlineto{\pgfqpoint{3.935540in}{2.090584in}}%
\pgfpathlineto{\pgfqpoint{3.922389in}{2.096424in}}%
\pgfpathlineto{\pgfqpoint{3.909243in}{2.102397in}}%
\pgfpathlineto{\pgfqpoint{3.896101in}{2.108502in}}%
\pgfpathlineto{\pgfqpoint{3.888432in}{2.099847in}}%
\pgfpathlineto{\pgfqpoint{3.880757in}{2.091238in}}%
\pgfpathlineto{\pgfqpoint{3.873076in}{2.082678in}}%
\pgfpathlineto{\pgfqpoint{3.865389in}{2.074167in}}%
\pgfpathclose%
\pgfusepath{fill}%
\end{pgfscope}%
\begin{pgfscope}%
\pgfpathrectangle{\pgfqpoint{1.254980in}{0.150000in}}{\pgfqpoint{5.490039in}{5.490039in}}%
\pgfusepath{clip}%
\pgfsetbuttcap%
\pgfsetroundjoin%
\definecolor{currentfill}{rgb}{0.281446,0.084320,0.407414}%
\pgfsetfillcolor{currentfill}%
\pgfsetfillopacity{0.700000}%
\pgfsetlinewidth{0.000000pt}%
\definecolor{currentstroke}{rgb}{0.000000,0.000000,0.000000}%
\pgfsetstrokecolor{currentstroke}%
\pgfsetdash{}{0pt}%
\pgfpathmoveto{\pgfqpoint{3.729363in}{2.097092in}}%
\pgfpathlineto{\pgfqpoint{3.742504in}{2.089562in}}%
\pgfpathlineto{\pgfqpoint{3.755648in}{2.082168in}}%
\pgfpathlineto{\pgfqpoint{3.768796in}{2.074910in}}%
\pgfpathlineto{\pgfqpoint{3.781947in}{2.067788in}}%
\pgfpathlineto{\pgfqpoint{3.789671in}{2.075901in}}%
\pgfpathlineto{\pgfqpoint{3.797389in}{2.084072in}}%
\pgfpathlineto{\pgfqpoint{3.805101in}{2.092300in}}%
\pgfpathlineto{\pgfqpoint{3.812808in}{2.100584in}}%
\pgfpathlineto{\pgfqpoint{3.799672in}{2.107524in}}%
\pgfpathlineto{\pgfqpoint{3.786539in}{2.114600in}}%
\pgfpathlineto{\pgfqpoint{3.773410in}{2.121811in}}%
\pgfpathlineto{\pgfqpoint{3.760284in}{2.129158in}}%
\pgfpathlineto{\pgfqpoint{3.752563in}{2.121051in}}%
\pgfpathlineto{\pgfqpoint{3.744836in}{2.113003in}}%
\pgfpathlineto{\pgfqpoint{3.737102in}{2.105016in}}%
\pgfpathlineto{\pgfqpoint{3.729363in}{2.097092in}}%
\pgfpathclose%
\pgfusepath{fill}%
\end{pgfscope}%
\begin{pgfscope}%
\pgfpathrectangle{\pgfqpoint{1.254980in}{0.150000in}}{\pgfqpoint{5.490039in}{5.490039in}}%
\pgfusepath{clip}%
\pgfsetbuttcap%
\pgfsetroundjoin%
\definecolor{currentfill}{rgb}{0.165117,0.467423,0.558141}%
\pgfsetfillcolor{currentfill}%
\pgfsetfillopacity{0.700000}%
\pgfsetlinewidth{0.000000pt}%
\definecolor{currentstroke}{rgb}{0.000000,0.000000,0.000000}%
\pgfsetstrokecolor{currentstroke}%
\pgfsetdash{}{0pt}%
\pgfpathmoveto{\pgfqpoint{5.801212in}{2.916611in}}%
\pgfpathlineto{\pgfqpoint{5.815085in}{2.921667in}}%
\pgfpathlineto{\pgfqpoint{5.828973in}{2.926833in}}%
\pgfpathlineto{\pgfqpoint{5.842876in}{2.932111in}}%
\pgfpathlineto{\pgfqpoint{5.856793in}{2.937499in}}%
\pgfpathlineto{\pgfqpoint{5.863767in}{2.945225in}}%
\pgfpathlineto{\pgfqpoint{5.870736in}{2.952933in}}%
\pgfpathlineto{\pgfqpoint{5.877699in}{2.960625in}}%
\pgfpathlineto{\pgfqpoint{5.884656in}{2.968301in}}%
\pgfpathlineto{\pgfqpoint{5.870755in}{2.963105in}}%
\pgfpathlineto{\pgfqpoint{5.856868in}{2.958019in}}%
\pgfpathlineto{\pgfqpoint{5.842996in}{2.953044in}}%
\pgfpathlineto{\pgfqpoint{5.829138in}{2.948179in}}%
\pgfpathlineto{\pgfqpoint{5.822165in}{2.940305in}}%
\pgfpathlineto{\pgfqpoint{5.815186in}{2.932420in}}%
\pgfpathlineto{\pgfqpoint{5.808201in}{2.924523in}}%
\pgfpathlineto{\pgfqpoint{5.801212in}{2.916611in}}%
\pgfpathclose%
\pgfusepath{fill}%
\end{pgfscope}%
\begin{pgfscope}%
\pgfpathrectangle{\pgfqpoint{1.254980in}{0.150000in}}{\pgfqpoint{5.490039in}{5.490039in}}%
\pgfusepath{clip}%
\pgfsetbuttcap%
\pgfsetroundjoin%
\definecolor{currentfill}{rgb}{0.280267,0.073417,0.397163}%
\pgfsetfillcolor{currentfill}%
\pgfsetfillopacity{0.700000}%
\pgfsetlinewidth{0.000000pt}%
\definecolor{currentstroke}{rgb}{0.000000,0.000000,0.000000}%
\pgfsetstrokecolor{currentstroke}%
\pgfsetdash{}{0pt}%
\pgfpathmoveto{\pgfqpoint{4.001363in}{2.063341in}}%
\pgfpathlineto{\pgfqpoint{4.014543in}{2.058282in}}%
\pgfpathlineto{\pgfqpoint{4.027727in}{2.053352in}}%
\pgfpathlineto{\pgfqpoint{4.040917in}{2.048550in}}%
\pgfpathlineto{\pgfqpoint{4.054113in}{2.043876in}}%
\pgfpathlineto{\pgfqpoint{4.061739in}{2.053045in}}%
\pgfpathlineto{\pgfqpoint{4.069361in}{2.062247in}}%
\pgfpathlineto{\pgfqpoint{4.076977in}{2.071481in}}%
\pgfpathlineto{\pgfqpoint{4.084588in}{2.080746in}}%
\pgfpathlineto{\pgfqpoint{4.071404in}{2.085270in}}%
\pgfpathlineto{\pgfqpoint{4.058226in}{2.089922in}}%
\pgfpathlineto{\pgfqpoint{4.045053in}{2.094702in}}%
\pgfpathlineto{\pgfqpoint{4.031886in}{2.099612in}}%
\pgfpathlineto{\pgfqpoint{4.024263in}{2.090491in}}%
\pgfpathlineto{\pgfqpoint{4.016635in}{2.081405in}}%
\pgfpathlineto{\pgfqpoint{4.009002in}{2.072354in}}%
\pgfpathlineto{\pgfqpoint{4.001363in}{2.063341in}}%
\pgfpathclose%
\pgfusepath{fill}%
\end{pgfscope}%
\begin{pgfscope}%
\pgfpathrectangle{\pgfqpoint{1.254980in}{0.150000in}}{\pgfqpoint{5.490039in}{5.490039in}}%
\pgfusepath{clip}%
\pgfsetbuttcap%
\pgfsetroundjoin%
\definecolor{currentfill}{rgb}{0.279574,0.170599,0.479997}%
\pgfsetfillcolor{currentfill}%
\pgfsetfillopacity{0.700000}%
\pgfsetlinewidth{0.000000pt}%
\definecolor{currentstroke}{rgb}{0.000000,0.000000,0.000000}%
\pgfsetstrokecolor{currentstroke}%
\pgfsetdash{}{0pt}%
\pgfpathmoveto{\pgfqpoint{3.351578in}{2.269253in}}%
\pgfpathlineto{\pgfqpoint{3.364721in}{2.257852in}}%
\pgfpathlineto{\pgfqpoint{3.377864in}{2.246603in}}%
\pgfpathlineto{\pgfqpoint{3.391006in}{2.235505in}}%
\pgfpathlineto{\pgfqpoint{3.404149in}{2.224557in}}%
\pgfpathlineto{\pgfqpoint{3.412034in}{2.230817in}}%
\pgfpathlineto{\pgfqpoint{3.419911in}{2.237172in}}%
\pgfpathlineto{\pgfqpoint{3.427780in}{2.243620in}}%
\pgfpathlineto{\pgfqpoint{3.435642in}{2.250160in}}%
\pgfpathlineto{\pgfqpoint{3.422520in}{2.260889in}}%
\pgfpathlineto{\pgfqpoint{3.409399in}{2.271769in}}%
\pgfpathlineto{\pgfqpoint{3.396277in}{2.282800in}}%
\pgfpathlineto{\pgfqpoint{3.383156in}{2.293982in}}%
\pgfpathlineto{\pgfqpoint{3.375273in}{2.287655in}}%
\pgfpathlineto{\pgfqpoint{3.367382in}{2.281423in}}%
\pgfpathlineto{\pgfqpoint{3.359484in}{2.275289in}}%
\pgfpathlineto{\pgfqpoint{3.351578in}{2.269253in}}%
\pgfpathclose%
\pgfusepath{fill}%
\end{pgfscope}%
\begin{pgfscope}%
\pgfpathrectangle{\pgfqpoint{1.254980in}{0.150000in}}{\pgfqpoint{5.490039in}{5.490039in}}%
\pgfusepath{clip}%
\pgfsetbuttcap%
\pgfsetroundjoin%
\definecolor{currentfill}{rgb}{0.281924,0.089666,0.412415}%
\pgfsetfillcolor{currentfill}%
\pgfsetfillopacity{0.700000}%
\pgfsetlinewidth{0.000000pt}%
\definecolor{currentstroke}{rgb}{0.000000,0.000000,0.000000}%
\pgfsetstrokecolor{currentstroke}%
\pgfsetdash{}{0pt}%
\pgfpathmoveto{\pgfqpoint{4.220578in}{2.087534in}}%
\pgfpathlineto{\pgfqpoint{4.233806in}{2.084278in}}%
\pgfpathlineto{\pgfqpoint{4.247041in}{2.081147in}}%
\pgfpathlineto{\pgfqpoint{4.260283in}{2.078140in}}%
\pgfpathlineto{\pgfqpoint{4.273532in}{2.075257in}}%
\pgfpathlineto{\pgfqpoint{4.281086in}{2.085034in}}%
\pgfpathlineto{\pgfqpoint{4.288636in}{2.094824in}}%
\pgfpathlineto{\pgfqpoint{4.296181in}{2.104628in}}%
\pgfpathlineto{\pgfqpoint{4.303721in}{2.114444in}}%
\pgfpathlineto{\pgfqpoint{4.290482in}{2.117209in}}%
\pgfpathlineto{\pgfqpoint{4.277250in}{2.120099in}}%
\pgfpathlineto{\pgfqpoint{4.264025in}{2.123112in}}%
\pgfpathlineto{\pgfqpoint{4.250807in}{2.126250in}}%
\pgfpathlineto{\pgfqpoint{4.243257in}{2.116545in}}%
\pgfpathlineto{\pgfqpoint{4.235702in}{2.106858in}}%
\pgfpathlineto{\pgfqpoint{4.228143in}{2.097187in}}%
\pgfpathlineto{\pgfqpoint{4.220578in}{2.087534in}}%
\pgfpathclose%
\pgfusepath{fill}%
\end{pgfscope}%
\begin{pgfscope}%
\pgfpathrectangle{\pgfqpoint{1.254980in}{0.150000in}}{\pgfqpoint{5.490039in}{5.490039in}}%
\pgfusepath{clip}%
\pgfsetbuttcap%
\pgfsetroundjoin%
\definecolor{currentfill}{rgb}{0.156270,0.489624,0.557936}%
\pgfsetfillcolor{currentfill}%
\pgfsetfillopacity{0.700000}%
\pgfsetlinewidth{0.000000pt}%
\definecolor{currentstroke}{rgb}{0.000000,0.000000,0.000000}%
\pgfsetstrokecolor{currentstroke}%
\pgfsetdash{}{0pt}%
\pgfpathmoveto{\pgfqpoint{5.884656in}{2.968301in}}%
\pgfpathlineto{\pgfqpoint{5.898573in}{2.973608in}}%
\pgfpathlineto{\pgfqpoint{5.912504in}{2.979025in}}%
\pgfpathlineto{\pgfqpoint{5.926450in}{2.984552in}}%
\pgfpathlineto{\pgfqpoint{5.940412in}{2.990189in}}%
\pgfpathlineto{\pgfqpoint{5.947347in}{2.997650in}}%
\pgfpathlineto{\pgfqpoint{5.954277in}{3.005097in}}%
\pgfpathlineto{\pgfqpoint{5.961202in}{3.012532in}}%
\pgfpathlineto{\pgfqpoint{5.968122in}{3.019956in}}%
\pgfpathlineto{\pgfqpoint{5.954178in}{3.014527in}}%
\pgfpathlineto{\pgfqpoint{5.940248in}{3.009208in}}%
\pgfpathlineto{\pgfqpoint{5.926334in}{3.004000in}}%
\pgfpathlineto{\pgfqpoint{5.912435in}{2.998901in}}%
\pgfpathlineto{\pgfqpoint{5.905498in}{2.991263in}}%
\pgfpathlineto{\pgfqpoint{5.898556in}{2.983618in}}%
\pgfpathlineto{\pgfqpoint{5.891609in}{2.975965in}}%
\pgfpathlineto{\pgfqpoint{5.884656in}{2.968301in}}%
\pgfpathclose%
\pgfusepath{fill}%
\end{pgfscope}%
\begin{pgfscope}%
\pgfpathrectangle{\pgfqpoint{1.254980in}{0.150000in}}{\pgfqpoint{5.490039in}{5.490039in}}%
\pgfusepath{clip}%
\pgfsetbuttcap%
\pgfsetroundjoin%
\definecolor{currentfill}{rgb}{0.141935,0.526453,0.555991}%
\pgfsetfillcolor{currentfill}%
\pgfsetfillopacity{0.700000}%
\pgfsetlinewidth{0.000000pt}%
\definecolor{currentstroke}{rgb}{0.000000,0.000000,0.000000}%
\pgfsetstrokecolor{currentstroke}%
\pgfsetdash{}{0pt}%
\pgfpathmoveto{\pgfqpoint{6.051606in}{3.071504in}}%
\pgfpathlineto{\pgfqpoint{6.065608in}{3.077256in}}%
\pgfpathlineto{\pgfqpoint{6.079626in}{3.083118in}}%
\pgfpathlineto{\pgfqpoint{6.093660in}{3.089089in}}%
\pgfpathlineto{\pgfqpoint{6.100521in}{3.096082in}}%
\pgfpathlineto{\pgfqpoint{6.107378in}{3.103071in}}%
\pgfpathlineto{\pgfqpoint{6.114229in}{3.110058in}}%
\pgfpathlineto{\pgfqpoint{6.121075in}{3.117047in}}%
\pgfpathlineto{\pgfqpoint{6.107061in}{3.111318in}}%
\pgfpathlineto{\pgfqpoint{6.093063in}{3.105698in}}%
\pgfpathlineto{\pgfqpoint{6.079080in}{3.100187in}}%
\pgfpathlineto{\pgfqpoint{6.072218in}{3.093013in}}%
\pgfpathlineto{\pgfqpoint{6.065352in}{3.085842in}}%
\pgfpathlineto{\pgfqpoint{6.058482in}{3.078674in}}%
\pgfpathlineto{\pgfqpoint{6.051606in}{3.071504in}}%
\pgfpathclose%
\pgfusepath{fill}%
\end{pgfscope}%
\begin{pgfscope}%
\pgfpathrectangle{\pgfqpoint{1.254980in}{0.150000in}}{\pgfqpoint{5.490039in}{5.490039in}}%
\pgfusepath{clip}%
\pgfsetbuttcap%
\pgfsetroundjoin%
\definecolor{currentfill}{rgb}{0.269308,0.218818,0.509577}%
\pgfsetfillcolor{currentfill}%
\pgfsetfillopacity{0.700000}%
\pgfsetlinewidth{0.000000pt}%
\definecolor{currentstroke}{rgb}{0.000000,0.000000,0.000000}%
\pgfsetstrokecolor{currentstroke}%
\pgfsetdash{}{0pt}%
\pgfpathmoveto{\pgfqpoint{4.855840in}{2.334574in}}%
\pgfpathlineto{\pgfqpoint{4.869285in}{2.335652in}}%
\pgfpathlineto{\pgfqpoint{4.882740in}{2.336845in}}%
\pgfpathlineto{\pgfqpoint{4.896206in}{2.338155in}}%
\pgfpathlineto{\pgfqpoint{4.909683in}{2.339581in}}%
\pgfpathlineto{\pgfqpoint{4.917036in}{2.349676in}}%
\pgfpathlineto{\pgfqpoint{4.924385in}{2.359745in}}%
\pgfpathlineto{\pgfqpoint{4.931729in}{2.369788in}}%
\pgfpathlineto{\pgfqpoint{4.939068in}{2.379806in}}%
\pgfpathlineto{\pgfqpoint{4.925599in}{2.378374in}}%
\pgfpathlineto{\pgfqpoint{4.912141in}{2.377058in}}%
\pgfpathlineto{\pgfqpoint{4.898694in}{2.375858in}}%
\pgfpathlineto{\pgfqpoint{4.885257in}{2.374775in}}%
\pgfpathlineto{\pgfqpoint{4.877910in}{2.364757in}}%
\pgfpathlineto{\pgfqpoint{4.870559in}{2.354718in}}%
\pgfpathlineto{\pgfqpoint{4.863202in}{2.344657in}}%
\pgfpathlineto{\pgfqpoint{4.855840in}{2.334574in}}%
\pgfpathclose%
\pgfusepath{fill}%
\end{pgfscope}%
\begin{pgfscope}%
\pgfpathrectangle{\pgfqpoint{1.254980in}{0.150000in}}{\pgfqpoint{5.490039in}{5.490039in}}%
\pgfusepath{clip}%
\pgfsetbuttcap%
\pgfsetroundjoin%
\definecolor{currentfill}{rgb}{0.274128,0.199721,0.498911}%
\pgfsetfillcolor{currentfill}%
\pgfsetfillopacity{0.700000}%
\pgfsetlinewidth{0.000000pt}%
\definecolor{currentstroke}{rgb}{0.000000,0.000000,0.000000}%
\pgfsetstrokecolor{currentstroke}%
\pgfsetdash{}{0pt}%
\pgfpathmoveto{\pgfqpoint{4.772637in}{2.290970in}}%
\pgfpathlineto{\pgfqpoint{4.786049in}{2.291558in}}%
\pgfpathlineto{\pgfqpoint{4.799470in}{2.292263in}}%
\pgfpathlineto{\pgfqpoint{4.812902in}{2.293085in}}%
\pgfpathlineto{\pgfqpoint{4.826344in}{2.294023in}}%
\pgfpathlineto{\pgfqpoint{4.833726in}{2.304194in}}%
\pgfpathlineto{\pgfqpoint{4.841102in}{2.314343in}}%
\pgfpathlineto{\pgfqpoint{4.848474in}{2.324470in}}%
\pgfpathlineto{\pgfqpoint{4.855840in}{2.334574in}}%
\pgfpathlineto{\pgfqpoint{4.842406in}{2.333613in}}%
\pgfpathlineto{\pgfqpoint{4.828982in}{2.332769in}}%
\pgfpathlineto{\pgfqpoint{4.815568in}{2.332042in}}%
\pgfpathlineto{\pgfqpoint{4.802164in}{2.331432in}}%
\pgfpathlineto{\pgfqpoint{4.794790in}{2.321344in}}%
\pgfpathlineto{\pgfqpoint{4.787411in}{2.311238in}}%
\pgfpathlineto{\pgfqpoint{4.780026in}{2.301113in}}%
\pgfpathlineto{\pgfqpoint{4.772637in}{2.290970in}}%
\pgfpathclose%
\pgfusepath{fill}%
\end{pgfscope}%
\begin{pgfscope}%
\pgfpathrectangle{\pgfqpoint{1.254980in}{0.150000in}}{\pgfqpoint{5.490039in}{5.490039in}}%
\pgfusepath{clip}%
\pgfsetbuttcap%
\pgfsetroundjoin%
\definecolor{currentfill}{rgb}{0.262138,0.242286,0.520837}%
\pgfsetfillcolor{currentfill}%
\pgfsetfillopacity{0.700000}%
\pgfsetlinewidth{0.000000pt}%
\definecolor{currentstroke}{rgb}{0.000000,0.000000,0.000000}%
\pgfsetstrokecolor{currentstroke}%
\pgfsetdash{}{0pt}%
\pgfpathmoveto{\pgfqpoint{4.939068in}{2.379806in}}%
\pgfpathlineto{\pgfqpoint{4.952547in}{2.381354in}}%
\pgfpathlineto{\pgfqpoint{4.966038in}{2.383018in}}%
\pgfpathlineto{\pgfqpoint{4.979539in}{2.384797in}}%
\pgfpathlineto{\pgfqpoint{4.993052in}{2.386692in}}%
\pgfpathlineto{\pgfqpoint{5.000378in}{2.396681in}}%
\pgfpathlineto{\pgfqpoint{5.007698in}{2.406642in}}%
\pgfpathlineto{\pgfqpoint{5.015014in}{2.416574in}}%
\pgfpathlineto{\pgfqpoint{5.022324in}{2.426479in}}%
\pgfpathlineto{\pgfqpoint{5.008820in}{2.424594in}}%
\pgfpathlineto{\pgfqpoint{4.995326in}{2.422825in}}%
\pgfpathlineto{\pgfqpoint{4.981844in}{2.421171in}}%
\pgfpathlineto{\pgfqpoint{4.968373in}{2.419633in}}%
\pgfpathlineto{\pgfqpoint{4.961054in}{2.409713in}}%
\pgfpathlineto{\pgfqpoint{4.953730in}{2.399768in}}%
\pgfpathlineto{\pgfqpoint{4.946402in}{2.389800in}}%
\pgfpathlineto{\pgfqpoint{4.939068in}{2.379806in}}%
\pgfpathclose%
\pgfusepath{fill}%
\end{pgfscope}%
\begin{pgfscope}%
\pgfpathrectangle{\pgfqpoint{1.254980in}{0.150000in}}{\pgfqpoint{5.490039in}{5.490039in}}%
\pgfusepath{clip}%
\pgfsetbuttcap%
\pgfsetroundjoin%
\definecolor{currentfill}{rgb}{0.278012,0.180367,0.486697}%
\pgfsetfillcolor{currentfill}%
\pgfsetfillopacity{0.700000}%
\pgfsetlinewidth{0.000000pt}%
\definecolor{currentstroke}{rgb}{0.000000,0.000000,0.000000}%
\pgfsetstrokecolor{currentstroke}%
\pgfsetdash{}{0pt}%
\pgfpathmoveto{\pgfqpoint{4.689453in}{2.249191in}}%
\pgfpathlineto{\pgfqpoint{4.702833in}{2.249271in}}%
\pgfpathlineto{\pgfqpoint{4.716223in}{2.249468in}}%
\pgfpathlineto{\pgfqpoint{4.729623in}{2.249782in}}%
\pgfpathlineto{\pgfqpoint{4.743032in}{2.250215in}}%
\pgfpathlineto{\pgfqpoint{4.750441in}{2.260431in}}%
\pgfpathlineto{\pgfqpoint{4.757844in}{2.270629in}}%
\pgfpathlineto{\pgfqpoint{4.765243in}{2.280809in}}%
\pgfpathlineto{\pgfqpoint{4.772637in}{2.290970in}}%
\pgfpathlineto{\pgfqpoint{4.759236in}{2.290500in}}%
\pgfpathlineto{\pgfqpoint{4.745844in}{2.290147in}}%
\pgfpathlineto{\pgfqpoint{4.732462in}{2.289911in}}%
\pgfpathlineto{\pgfqpoint{4.719089in}{2.289794in}}%
\pgfpathlineto{\pgfqpoint{4.711688in}{2.279665in}}%
\pgfpathlineto{\pgfqpoint{4.704281in}{2.269521in}}%
\pgfpathlineto{\pgfqpoint{4.696869in}{2.259364in}}%
\pgfpathlineto{\pgfqpoint{4.689453in}{2.249191in}}%
\pgfpathclose%
\pgfusepath{fill}%
\end{pgfscope}%
\begin{pgfscope}%
\pgfpathrectangle{\pgfqpoint{1.254980in}{0.150000in}}{\pgfqpoint{5.490039in}{5.490039in}}%
\pgfusepath{clip}%
\pgfsetbuttcap%
\pgfsetroundjoin%
\definecolor{currentfill}{rgb}{0.146180,0.515413,0.556823}%
\pgfsetfillcolor{currentfill}%
\pgfsetfillopacity{0.700000}%
\pgfsetlinewidth{0.000000pt}%
\definecolor{currentstroke}{rgb}{0.000000,0.000000,0.000000}%
\pgfsetstrokecolor{currentstroke}%
\pgfsetdash{}{0pt}%
\pgfpathmoveto{\pgfqpoint{2.664506in}{3.093114in}}%
\pgfpathlineto{\pgfqpoint{2.677855in}{3.072530in}}%
\pgfpathlineto{\pgfqpoint{2.691196in}{3.052151in}}%
\pgfpathlineto{\pgfqpoint{2.704529in}{3.031979in}}%
\pgfpathlineto{\pgfqpoint{2.717854in}{3.012009in}}%
\pgfpathlineto{\pgfqpoint{2.726081in}{3.015253in}}%
\pgfpathlineto{\pgfqpoint{2.734297in}{3.018646in}}%
\pgfpathlineto{\pgfqpoint{2.742502in}{3.022188in}}%
\pgfpathlineto{\pgfqpoint{2.750695in}{3.025876in}}%
\pgfpathlineto{\pgfqpoint{2.737402in}{3.045609in}}%
\pgfpathlineto{\pgfqpoint{2.724102in}{3.065545in}}%
\pgfpathlineto{\pgfqpoint{2.710793in}{3.085686in}}%
\pgfpathlineto{\pgfqpoint{2.697477in}{3.106033in}}%
\pgfpathlineto{\pgfqpoint{2.689251in}{3.102575in}}%
\pgfpathlineto{\pgfqpoint{2.681014in}{3.099268in}}%
\pgfpathlineto{\pgfqpoint{2.672766in}{3.096114in}}%
\pgfpathlineto{\pgfqpoint{2.664506in}{3.093114in}}%
\pgfpathclose%
\pgfusepath{fill}%
\end{pgfscope}%
\begin{pgfscope}%
\pgfpathrectangle{\pgfqpoint{1.254980in}{0.150000in}}{\pgfqpoint{5.490039in}{5.490039in}}%
\pgfusepath{clip}%
\pgfsetbuttcap%
\pgfsetroundjoin%
\definecolor{currentfill}{rgb}{0.149039,0.508051,0.557250}%
\pgfsetfillcolor{currentfill}%
\pgfsetfillopacity{0.700000}%
\pgfsetlinewidth{0.000000pt}%
\definecolor{currentstroke}{rgb}{0.000000,0.000000,0.000000}%
\pgfsetstrokecolor{currentstroke}%
\pgfsetdash{}{0pt}%
\pgfpathmoveto{\pgfqpoint{5.968122in}{3.019956in}}%
\pgfpathlineto{\pgfqpoint{5.982081in}{3.025494in}}%
\pgfpathlineto{\pgfqpoint{5.996056in}{3.031142in}}%
\pgfpathlineto{\pgfqpoint{6.010046in}{3.036901in}}%
\pgfpathlineto{\pgfqpoint{6.024051in}{3.042769in}}%
\pgfpathlineto{\pgfqpoint{6.030948in}{3.049965in}}%
\pgfpathlineto{\pgfqpoint{6.037839in}{3.057152in}}%
\pgfpathlineto{\pgfqpoint{6.044725in}{3.064331in}}%
\pgfpathlineto{\pgfqpoint{6.051606in}{3.071504in}}%
\pgfpathlineto{\pgfqpoint{6.037619in}{3.065861in}}%
\pgfpathlineto{\pgfqpoint{6.023647in}{3.060328in}}%
\pgfpathlineto{\pgfqpoint{6.009691in}{3.054905in}}%
\pgfpathlineto{\pgfqpoint{5.995749in}{3.049591in}}%
\pgfpathlineto{\pgfqpoint{5.988850in}{3.042187in}}%
\pgfpathlineto{\pgfqpoint{5.981946in}{3.034781in}}%
\pgfpathlineto{\pgfqpoint{5.975036in}{3.027371in}}%
\pgfpathlineto{\pgfqpoint{5.968122in}{3.019956in}}%
\pgfpathclose%
\pgfusepath{fill}%
\end{pgfscope}%
\begin{pgfscope}%
\pgfpathrectangle{\pgfqpoint{1.254980in}{0.150000in}}{\pgfqpoint{5.490039in}{5.490039in}}%
\pgfusepath{clip}%
\pgfsetbuttcap%
\pgfsetroundjoin%
\definecolor{currentfill}{rgb}{0.282910,0.105393,0.426902}%
\pgfsetfillcolor{currentfill}%
\pgfsetfillopacity{0.700000}%
\pgfsetlinewidth{0.000000pt}%
\definecolor{currentstroke}{rgb}{0.000000,0.000000,0.000000}%
\pgfsetstrokecolor{currentstroke}%
\pgfsetdash{}{0pt}%
\pgfpathmoveto{\pgfqpoint{3.593179in}{2.132856in}}%
\pgfpathlineto{\pgfqpoint{3.606316in}{2.124016in}}%
\pgfpathlineto{\pgfqpoint{3.619456in}{2.115316in}}%
\pgfpathlineto{\pgfqpoint{3.632598in}{2.106757in}}%
\pgfpathlineto{\pgfqpoint{3.645742in}{2.098338in}}%
\pgfpathlineto{\pgfqpoint{3.653523in}{2.105793in}}%
\pgfpathlineto{\pgfqpoint{3.661297in}{2.113321in}}%
\pgfpathlineto{\pgfqpoint{3.669065in}{2.120919in}}%
\pgfpathlineto{\pgfqpoint{3.676827in}{2.128586in}}%
\pgfpathlineto{\pgfqpoint{3.663700in}{2.136806in}}%
\pgfpathlineto{\pgfqpoint{3.650575in}{2.145166in}}%
\pgfpathlineto{\pgfqpoint{3.637453in}{2.153666in}}%
\pgfpathlineto{\pgfqpoint{3.624332in}{2.162307in}}%
\pgfpathlineto{\pgfqpoint{3.616554in}{2.154833in}}%
\pgfpathlineto{\pgfqpoint{3.608769in}{2.147432in}}%
\pgfpathlineto{\pgfqpoint{3.600977in}{2.140106in}}%
\pgfpathlineto{\pgfqpoint{3.593179in}{2.132856in}}%
\pgfpathclose%
\pgfusepath{fill}%
\end{pgfscope}%
\begin{pgfscope}%
\pgfpathrectangle{\pgfqpoint{1.254980in}{0.150000in}}{\pgfqpoint{5.490039in}{5.490039in}}%
\pgfusepath{clip}%
\pgfsetbuttcap%
\pgfsetroundjoin%
\definecolor{currentfill}{rgb}{0.253935,0.265254,0.529983}%
\pgfsetfillcolor{currentfill}%
\pgfsetfillopacity{0.700000}%
\pgfsetlinewidth{0.000000pt}%
\definecolor{currentstroke}{rgb}{0.000000,0.000000,0.000000}%
\pgfsetstrokecolor{currentstroke}%
\pgfsetdash{}{0pt}%
\pgfpathmoveto{\pgfqpoint{5.022324in}{2.426479in}}%
\pgfpathlineto{\pgfqpoint{5.035840in}{2.428478in}}%
\pgfpathlineto{\pgfqpoint{5.049367in}{2.430593in}}%
\pgfpathlineto{\pgfqpoint{5.062906in}{2.432822in}}%
\pgfpathlineto{\pgfqpoint{5.076456in}{2.435167in}}%
\pgfpathlineto{\pgfqpoint{5.083753in}{2.445024in}}%
\pgfpathlineto{\pgfqpoint{5.091045in}{2.454850in}}%
\pgfpathlineto{\pgfqpoint{5.098332in}{2.464646in}}%
\pgfpathlineto{\pgfqpoint{5.105613in}{2.474411in}}%
\pgfpathlineto{\pgfqpoint{5.092071in}{2.472093in}}%
\pgfpathlineto{\pgfqpoint{5.078541in}{2.469890in}}%
\pgfpathlineto{\pgfqpoint{5.065022in}{2.467802in}}%
\pgfpathlineto{\pgfqpoint{5.051515in}{2.465828in}}%
\pgfpathlineto{\pgfqpoint{5.044225in}{2.456030in}}%
\pgfpathlineto{\pgfqpoint{5.036930in}{2.446206in}}%
\pgfpathlineto{\pgfqpoint{5.029630in}{2.436356in}}%
\pgfpathlineto{\pgfqpoint{5.022324in}{2.426479in}}%
\pgfpathclose%
\pgfusepath{fill}%
\end{pgfscope}%
\begin{pgfscope}%
\pgfpathrectangle{\pgfqpoint{1.254980in}{0.150000in}}{\pgfqpoint{5.490039in}{5.490039in}}%
\pgfusepath{clip}%
\pgfsetbuttcap%
\pgfsetroundjoin%
\definecolor{currentfill}{rgb}{0.281412,0.155834,0.469201}%
\pgfsetfillcolor{currentfill}%
\pgfsetfillopacity{0.700000}%
\pgfsetlinewidth{0.000000pt}%
\definecolor{currentstroke}{rgb}{0.000000,0.000000,0.000000}%
\pgfsetstrokecolor{currentstroke}%
\pgfsetdash{}{0pt}%
\pgfpathmoveto{\pgfqpoint{4.606281in}{2.209445in}}%
\pgfpathlineto{\pgfqpoint{4.619632in}{2.208996in}}%
\pgfpathlineto{\pgfqpoint{4.632992in}{2.208666in}}%
\pgfpathlineto{\pgfqpoint{4.646361in}{2.208454in}}%
\pgfpathlineto{\pgfqpoint{4.659739in}{2.208361in}}%
\pgfpathlineto{\pgfqpoint{4.667175in}{2.218590in}}%
\pgfpathlineto{\pgfqpoint{4.674606in}{2.228804in}}%
\pgfpathlineto{\pgfqpoint{4.682032in}{2.239005in}}%
\pgfpathlineto{\pgfqpoint{4.689453in}{2.249191in}}%
\pgfpathlineto{\pgfqpoint{4.676082in}{2.249230in}}%
\pgfpathlineto{\pgfqpoint{4.662721in}{2.249388in}}%
\pgfpathlineto{\pgfqpoint{4.649369in}{2.249664in}}%
\pgfpathlineto{\pgfqpoint{4.636026in}{2.250058in}}%
\pgfpathlineto{\pgfqpoint{4.628597in}{2.239920in}}%
\pgfpathlineto{\pgfqpoint{4.621163in}{2.229772in}}%
\pgfpathlineto{\pgfqpoint{4.613725in}{2.219613in}}%
\pgfpathlineto{\pgfqpoint{4.606281in}{2.209445in}}%
\pgfpathclose%
\pgfusepath{fill}%
\end{pgfscope}%
\begin{pgfscope}%
\pgfpathrectangle{\pgfqpoint{1.254980in}{0.150000in}}{\pgfqpoint{5.490039in}{5.490039in}}%
\pgfusepath{clip}%
\pgfsetbuttcap%
\pgfsetroundjoin%
\definecolor{currentfill}{rgb}{0.244972,0.287675,0.537260}%
\pgfsetfillcolor{currentfill}%
\pgfsetfillopacity{0.700000}%
\pgfsetlinewidth{0.000000pt}%
\definecolor{currentstroke}{rgb}{0.000000,0.000000,0.000000}%
\pgfsetstrokecolor{currentstroke}%
\pgfsetdash{}{0pt}%
\pgfpathmoveto{\pgfqpoint{5.105613in}{2.474411in}}%
\pgfpathlineto{\pgfqpoint{5.119167in}{2.476844in}}%
\pgfpathlineto{\pgfqpoint{5.132732in}{2.479391in}}%
\pgfpathlineto{\pgfqpoint{5.146309in}{2.482053in}}%
\pgfpathlineto{\pgfqpoint{5.159898in}{2.484828in}}%
\pgfpathlineto{\pgfqpoint{5.167166in}{2.494528in}}%
\pgfpathlineto{\pgfqpoint{5.174428in}{2.504196in}}%
\pgfpathlineto{\pgfqpoint{5.181685in}{2.513831in}}%
\pgfpathlineto{\pgfqpoint{5.188937in}{2.523435in}}%
\pgfpathlineto{\pgfqpoint{5.175356in}{2.520702in}}%
\pgfpathlineto{\pgfqpoint{5.161788in}{2.518084in}}%
\pgfpathlineto{\pgfqpoint{5.148231in}{2.515579in}}%
\pgfpathlineto{\pgfqpoint{5.134687in}{2.513189in}}%
\pgfpathlineto{\pgfqpoint{5.127426in}{2.503536in}}%
\pgfpathlineto{\pgfqpoint{5.120160in}{2.493856in}}%
\pgfpathlineto{\pgfqpoint{5.112889in}{2.484148in}}%
\pgfpathlineto{\pgfqpoint{5.105613in}{2.474411in}}%
\pgfpathclose%
\pgfusepath{fill}%
\end{pgfscope}%
\begin{pgfscope}%
\pgfpathrectangle{\pgfqpoint{1.254980in}{0.150000in}}{\pgfqpoint{5.490039in}{5.490039in}}%
\pgfusepath{clip}%
\pgfsetbuttcap%
\pgfsetroundjoin%
\definecolor{currentfill}{rgb}{0.282623,0.140926,0.457517}%
\pgfsetfillcolor{currentfill}%
\pgfsetfillopacity{0.700000}%
\pgfsetlinewidth{0.000000pt}%
\definecolor{currentstroke}{rgb}{0.000000,0.000000,0.000000}%
\pgfsetstrokecolor{currentstroke}%
\pgfsetdash{}{0pt}%
\pgfpathmoveto{\pgfqpoint{4.523113in}{2.171947in}}%
\pgfpathlineto{\pgfqpoint{4.536437in}{2.170951in}}%
\pgfpathlineto{\pgfqpoint{4.549769in}{2.170074in}}%
\pgfpathlineto{\pgfqpoint{4.563110in}{2.169317in}}%
\pgfpathlineto{\pgfqpoint{4.576459in}{2.168679in}}%
\pgfpathlineto{\pgfqpoint{4.583922in}{2.178884in}}%
\pgfpathlineto{\pgfqpoint{4.591380in}{2.189080in}}%
\pgfpathlineto{\pgfqpoint{4.598833in}{2.199267in}}%
\pgfpathlineto{\pgfqpoint{4.606281in}{2.209445in}}%
\pgfpathlineto{\pgfqpoint{4.592939in}{2.210013in}}%
\pgfpathlineto{\pgfqpoint{4.579606in}{2.210700in}}%
\pgfpathlineto{\pgfqpoint{4.566282in}{2.211507in}}%
\pgfpathlineto{\pgfqpoint{4.552967in}{2.212433in}}%
\pgfpathlineto{\pgfqpoint{4.545511in}{2.202320in}}%
\pgfpathlineto{\pgfqpoint{4.538050in}{2.192201in}}%
\pgfpathlineto{\pgfqpoint{4.530584in}{2.182076in}}%
\pgfpathlineto{\pgfqpoint{4.523113in}{2.171947in}}%
\pgfpathclose%
\pgfusepath{fill}%
\end{pgfscope}%
\begin{pgfscope}%
\pgfpathrectangle{\pgfqpoint{1.254980in}{0.150000in}}{\pgfqpoint{5.490039in}{5.490039in}}%
\pgfusepath{clip}%
\pgfsetbuttcap%
\pgfsetroundjoin%
\definecolor{currentfill}{rgb}{0.281887,0.150881,0.465405}%
\pgfsetfillcolor{currentfill}%
\pgfsetfillopacity{0.700000}%
\pgfsetlinewidth{0.000000pt}%
\definecolor{currentstroke}{rgb}{0.000000,0.000000,0.000000}%
\pgfsetstrokecolor{currentstroke}%
\pgfsetdash{}{0pt}%
\pgfpathmoveto{\pgfqpoint{3.404149in}{2.224557in}}%
\pgfpathlineto{\pgfqpoint{3.417292in}{2.213758in}}%
\pgfpathlineto{\pgfqpoint{3.430435in}{2.203109in}}%
\pgfpathlineto{\pgfqpoint{3.443579in}{2.192607in}}%
\pgfpathlineto{\pgfqpoint{3.456724in}{2.182253in}}%
\pgfpathlineto{\pgfqpoint{3.464588in}{2.188737in}}%
\pgfpathlineto{\pgfqpoint{3.472445in}{2.195312in}}%
\pgfpathlineto{\pgfqpoint{3.480294in}{2.201975in}}%
\pgfpathlineto{\pgfqpoint{3.488137in}{2.208727in}}%
\pgfpathlineto{\pgfqpoint{3.475012in}{2.218863in}}%
\pgfpathlineto{\pgfqpoint{3.461888in}{2.229147in}}%
\pgfpathlineto{\pgfqpoint{3.448765in}{2.239579in}}%
\pgfpathlineto{\pgfqpoint{3.435642in}{2.250160in}}%
\pgfpathlineto{\pgfqpoint{3.427780in}{2.243620in}}%
\pgfpathlineto{\pgfqpoint{3.419911in}{2.237172in}}%
\pgfpathlineto{\pgfqpoint{3.412034in}{2.230817in}}%
\pgfpathlineto{\pgfqpoint{3.404149in}{2.224557in}}%
\pgfpathclose%
\pgfusepath{fill}%
\end{pgfscope}%
\begin{pgfscope}%
\pgfpathrectangle{\pgfqpoint{1.254980in}{0.150000in}}{\pgfqpoint{5.490039in}{5.490039in}}%
\pgfusepath{clip}%
\pgfsetbuttcap%
\pgfsetroundjoin%
\definecolor{currentfill}{rgb}{0.235526,0.309527,0.542944}%
\pgfsetfillcolor{currentfill}%
\pgfsetfillopacity{0.700000}%
\pgfsetlinewidth{0.000000pt}%
\definecolor{currentstroke}{rgb}{0.000000,0.000000,0.000000}%
\pgfsetstrokecolor{currentstroke}%
\pgfsetdash{}{0pt}%
\pgfpathmoveto{\pgfqpoint{5.188937in}{2.523435in}}%
\pgfpathlineto{\pgfqpoint{5.202529in}{2.526282in}}%
\pgfpathlineto{\pgfqpoint{5.216133in}{2.529243in}}%
\pgfpathlineto{\pgfqpoint{5.229750in}{2.532318in}}%
\pgfpathlineto{\pgfqpoint{5.243379in}{2.535506in}}%
\pgfpathlineto{\pgfqpoint{5.250617in}{2.545027in}}%
\pgfpathlineto{\pgfqpoint{5.257849in}{2.554514in}}%
\pgfpathlineto{\pgfqpoint{5.265075in}{2.563968in}}%
\pgfpathlineto{\pgfqpoint{5.272296in}{2.573389in}}%
\pgfpathlineto{\pgfqpoint{5.258676in}{2.570260in}}%
\pgfpathlineto{\pgfqpoint{5.245069in}{2.567244in}}%
\pgfpathlineto{\pgfqpoint{5.231473in}{2.564342in}}%
\pgfpathlineto{\pgfqpoint{5.217890in}{2.561554in}}%
\pgfpathlineto{\pgfqpoint{5.210660in}{2.552067in}}%
\pgfpathlineto{\pgfqpoint{5.203424in}{2.542553in}}%
\pgfpathlineto{\pgfqpoint{5.196183in}{2.533009in}}%
\pgfpathlineto{\pgfqpoint{5.188937in}{2.523435in}}%
\pgfpathclose%
\pgfusepath{fill}%
\end{pgfscope}%
\begin{pgfscope}%
\pgfpathrectangle{\pgfqpoint{1.254980in}{0.150000in}}{\pgfqpoint{5.490039in}{5.490039in}}%
\pgfusepath{clip}%
\pgfsetbuttcap%
\pgfsetroundjoin%
\definecolor{currentfill}{rgb}{0.280894,0.078907,0.402329}%
\pgfsetfillcolor{currentfill}%
\pgfsetfillopacity{0.700000}%
\pgfsetlinewidth{0.000000pt}%
\definecolor{currentstroke}{rgb}{0.000000,0.000000,0.000000}%
\pgfsetstrokecolor{currentstroke}%
\pgfsetdash{}{0pt}%
\pgfpathmoveto{\pgfqpoint{4.137381in}{2.063921in}}%
\pgfpathlineto{\pgfqpoint{4.150594in}{2.060031in}}%
\pgfpathlineto{\pgfqpoint{4.163813in}{2.056267in}}%
\pgfpathlineto{\pgfqpoint{4.177038in}{2.052629in}}%
\pgfpathlineto{\pgfqpoint{4.190270in}{2.049115in}}%
\pgfpathlineto{\pgfqpoint{4.197855in}{2.058689in}}%
\pgfpathlineto{\pgfqpoint{4.205434in}{2.068284in}}%
\pgfpathlineto{\pgfqpoint{4.213009in}{2.077899in}}%
\pgfpathlineto{\pgfqpoint{4.220578in}{2.087534in}}%
\pgfpathlineto{\pgfqpoint{4.207356in}{2.090914in}}%
\pgfpathlineto{\pgfqpoint{4.194141in}{2.094419in}}%
\pgfpathlineto{\pgfqpoint{4.180932in}{2.098049in}}%
\pgfpathlineto{\pgfqpoint{4.167730in}{2.101805in}}%
\pgfpathlineto{\pgfqpoint{4.160150in}{2.092299in}}%
\pgfpathlineto{\pgfqpoint{4.152565in}{2.082815in}}%
\pgfpathlineto{\pgfqpoint{4.144976in}{2.073356in}}%
\pgfpathlineto{\pgfqpoint{4.137381in}{2.063921in}}%
\pgfpathclose%
\pgfusepath{fill}%
\end{pgfscope}%
\begin{pgfscope}%
\pgfpathrectangle{\pgfqpoint{1.254980in}{0.150000in}}{\pgfqpoint{5.490039in}{5.490039in}}%
\pgfusepath{clip}%
\pgfsetbuttcap%
\pgfsetroundjoin%
\definecolor{currentfill}{rgb}{0.223925,0.334994,0.548053}%
\pgfsetfillcolor{currentfill}%
\pgfsetfillopacity{0.700000}%
\pgfsetlinewidth{0.000000pt}%
\definecolor{currentstroke}{rgb}{0.000000,0.000000,0.000000}%
\pgfsetstrokecolor{currentstroke}%
\pgfsetdash{}{0pt}%
\pgfpathmoveto{\pgfqpoint{5.272296in}{2.573389in}}%
\pgfpathlineto{\pgfqpoint{5.285928in}{2.576632in}}%
\pgfpathlineto{\pgfqpoint{5.299573in}{2.579988in}}%
\pgfpathlineto{\pgfqpoint{5.313231in}{2.583458in}}%
\pgfpathlineto{\pgfqpoint{5.326901in}{2.587040in}}%
\pgfpathlineto{\pgfqpoint{5.334107in}{2.596362in}}%
\pgfpathlineto{\pgfqpoint{5.341308in}{2.605648in}}%
\pgfpathlineto{\pgfqpoint{5.348503in}{2.614901in}}%
\pgfpathlineto{\pgfqpoint{5.355692in}{2.624122in}}%
\pgfpathlineto{\pgfqpoint{5.342032in}{2.620615in}}%
\pgfpathlineto{\pgfqpoint{5.328384in}{2.617221in}}%
\pgfpathlineto{\pgfqpoint{5.314749in}{2.613940in}}%
\pgfpathlineto{\pgfqpoint{5.301126in}{2.610772in}}%
\pgfpathlineto{\pgfqpoint{5.293927in}{2.601470in}}%
\pgfpathlineto{\pgfqpoint{5.286722in}{2.592140in}}%
\pgfpathlineto{\pgfqpoint{5.279512in}{2.582780in}}%
\pgfpathlineto{\pgfqpoint{5.272296in}{2.573389in}}%
\pgfpathclose%
\pgfusepath{fill}%
\end{pgfscope}%
\begin{pgfscope}%
\pgfpathrectangle{\pgfqpoint{1.254980in}{0.150000in}}{\pgfqpoint{5.490039in}{5.490039in}}%
\pgfusepath{clip}%
\pgfsetbuttcap%
\pgfsetroundjoin%
\definecolor{currentfill}{rgb}{0.283229,0.120777,0.440584}%
\pgfsetfillcolor{currentfill}%
\pgfsetfillopacity{0.700000}%
\pgfsetlinewidth{0.000000pt}%
\definecolor{currentstroke}{rgb}{0.000000,0.000000,0.000000}%
\pgfsetstrokecolor{currentstroke}%
\pgfsetdash{}{0pt}%
\pgfpathmoveto{\pgfqpoint{4.439940in}{2.136924in}}%
\pgfpathlineto{\pgfqpoint{4.453238in}{2.135360in}}%
\pgfpathlineto{\pgfqpoint{4.466545in}{2.133917in}}%
\pgfpathlineto{\pgfqpoint{4.479860in}{2.132594in}}%
\pgfpathlineto{\pgfqpoint{4.493183in}{2.131392in}}%
\pgfpathlineto{\pgfqpoint{4.500673in}{2.141536in}}%
\pgfpathlineto{\pgfqpoint{4.508158in}{2.151677in}}%
\pgfpathlineto{\pgfqpoint{4.515638in}{2.161814in}}%
\pgfpathlineto{\pgfqpoint{4.523113in}{2.171947in}}%
\pgfpathlineto{\pgfqpoint{4.509798in}{2.173064in}}%
\pgfpathlineto{\pgfqpoint{4.496492in}{2.174301in}}%
\pgfpathlineto{\pgfqpoint{4.483193in}{2.175658in}}%
\pgfpathlineto{\pgfqpoint{4.469904in}{2.177136in}}%
\pgfpathlineto{\pgfqpoint{4.462420in}{2.167083in}}%
\pgfpathlineto{\pgfqpoint{4.454931in}{2.157029in}}%
\pgfpathlineto{\pgfqpoint{4.447438in}{2.146977in}}%
\pgfpathlineto{\pgfqpoint{4.439940in}{2.136924in}}%
\pgfpathclose%
\pgfusepath{fill}%
\end{pgfscope}%
\begin{pgfscope}%
\pgfpathrectangle{\pgfqpoint{1.254980in}{0.150000in}}{\pgfqpoint{5.490039in}{5.490039in}}%
\pgfusepath{clip}%
\pgfsetbuttcap%
\pgfsetroundjoin%
\definecolor{currentfill}{rgb}{0.280894,0.078907,0.402329}%
\pgfsetfillcolor{currentfill}%
\pgfsetfillopacity{0.700000}%
\pgfsetlinewidth{0.000000pt}%
\definecolor{currentstroke}{rgb}{0.000000,0.000000,0.000000}%
\pgfsetstrokecolor{currentstroke}%
\pgfsetdash{}{0pt}%
\pgfpathmoveto{\pgfqpoint{3.781947in}{2.067788in}}%
\pgfpathlineto{\pgfqpoint{3.795101in}{2.060800in}}%
\pgfpathlineto{\pgfqpoint{3.808259in}{2.053947in}}%
\pgfpathlineto{\pgfqpoint{3.821421in}{2.047228in}}%
\pgfpathlineto{\pgfqpoint{3.834587in}{2.040642in}}%
\pgfpathlineto{\pgfqpoint{3.842296in}{2.048943in}}%
\pgfpathlineto{\pgfqpoint{3.850000in}{2.057298in}}%
\pgfpathlineto{\pgfqpoint{3.857697in}{2.065706in}}%
\pgfpathlineto{\pgfqpoint{3.865389in}{2.074167in}}%
\pgfpathlineto{\pgfqpoint{3.852238in}{2.080571in}}%
\pgfpathlineto{\pgfqpoint{3.839091in}{2.087108in}}%
\pgfpathlineto{\pgfqpoint{3.825947in}{2.093779in}}%
\pgfpathlineto{\pgfqpoint{3.812808in}{2.100584in}}%
\pgfpathlineto{\pgfqpoint{3.805101in}{2.092300in}}%
\pgfpathlineto{\pgfqpoint{3.797389in}{2.084072in}}%
\pgfpathlineto{\pgfqpoint{3.789671in}{2.075901in}}%
\pgfpathlineto{\pgfqpoint{3.781947in}{2.067788in}}%
\pgfpathclose%
\pgfusepath{fill}%
\end{pgfscope}%
\begin{pgfscope}%
\pgfpathrectangle{\pgfqpoint{1.254980in}{0.150000in}}{\pgfqpoint{5.490039in}{5.490039in}}%
\pgfusepath{clip}%
\pgfsetbuttcap%
\pgfsetroundjoin%
\definecolor{currentfill}{rgb}{0.133743,0.548535,0.553541}%
\pgfsetfillcolor{currentfill}%
\pgfsetfillopacity{0.700000}%
\pgfsetlinewidth{0.000000pt}%
\definecolor{currentstroke}{rgb}{0.000000,0.000000,0.000000}%
\pgfsetstrokecolor{currentstroke}%
\pgfsetdash{}{0pt}%
\pgfpathmoveto{\pgfqpoint{2.611023in}{3.177549in}}%
\pgfpathlineto{\pgfqpoint{2.624407in}{3.156123in}}%
\pgfpathlineto{\pgfqpoint{2.637782in}{3.134909in}}%
\pgfpathlineto{\pgfqpoint{2.651148in}{3.113907in}}%
\pgfpathlineto{\pgfqpoint{2.664506in}{3.093114in}}%
\pgfpathlineto{\pgfqpoint{2.672766in}{3.096114in}}%
\pgfpathlineto{\pgfqpoint{2.681014in}{3.099268in}}%
\pgfpathlineto{\pgfqpoint{2.689251in}{3.102575in}}%
\pgfpathlineto{\pgfqpoint{2.697477in}{3.106033in}}%
\pgfpathlineto{\pgfqpoint{2.684152in}{3.126588in}}%
\pgfpathlineto{\pgfqpoint{2.670819in}{3.147351in}}%
\pgfpathlineto{\pgfqpoint{2.657478in}{3.168325in}}%
\pgfpathlineto{\pgfqpoint{2.644128in}{3.189512in}}%
\pgfpathlineto{\pgfqpoint{2.635869in}{3.186286in}}%
\pgfpathlineto{\pgfqpoint{2.627599in}{3.183215in}}%
\pgfpathlineto{\pgfqpoint{2.619317in}{3.180302in}}%
\pgfpathlineto{\pgfqpoint{2.611023in}{3.177549in}}%
\pgfpathclose%
\pgfusepath{fill}%
\end{pgfscope}%
\begin{pgfscope}%
\pgfpathrectangle{\pgfqpoint{1.254980in}{0.150000in}}{\pgfqpoint{5.490039in}{5.490039in}}%
\pgfusepath{clip}%
\pgfsetbuttcap%
\pgfsetroundjoin%
\definecolor{currentfill}{rgb}{0.279566,0.067836,0.391917}%
\pgfsetfillcolor{currentfill}%
\pgfsetfillopacity{0.700000}%
\pgfsetlinewidth{0.000000pt}%
\definecolor{currentstroke}{rgb}{0.000000,0.000000,0.000000}%
\pgfsetstrokecolor{currentstroke}%
\pgfsetdash{}{0pt}%
\pgfpathmoveto{\pgfqpoint{3.918036in}{2.049875in}}%
\pgfpathlineto{\pgfqpoint{3.931209in}{2.044130in}}%
\pgfpathlineto{\pgfqpoint{3.944386in}{2.038517in}}%
\pgfpathlineto{\pgfqpoint{3.957569in}{2.033033in}}%
\pgfpathlineto{\pgfqpoint{3.970755in}{2.027679in}}%
\pgfpathlineto{\pgfqpoint{3.978415in}{2.036534in}}%
\pgfpathlineto{\pgfqpoint{3.986070in}{2.045430in}}%
\pgfpathlineto{\pgfqpoint{3.993719in}{2.054366in}}%
\pgfpathlineto{\pgfqpoint{4.001363in}{2.063341in}}%
\pgfpathlineto{\pgfqpoint{3.988189in}{2.068529in}}%
\pgfpathlineto{\pgfqpoint{3.975019in}{2.073847in}}%
\pgfpathlineto{\pgfqpoint{3.961855in}{2.079295in}}%
\pgfpathlineto{\pgfqpoint{3.948695in}{2.084874in}}%
\pgfpathlineto{\pgfqpoint{3.941039in}{2.076058in}}%
\pgfpathlineto{\pgfqpoint{3.933377in}{2.067286in}}%
\pgfpathlineto{\pgfqpoint{3.925709in}{2.058558in}}%
\pgfpathlineto{\pgfqpoint{3.918036in}{2.049875in}}%
\pgfpathclose%
\pgfusepath{fill}%
\end{pgfscope}%
\begin{pgfscope}%
\pgfpathrectangle{\pgfqpoint{1.254980in}{0.150000in}}{\pgfqpoint{5.490039in}{5.490039in}}%
\pgfusepath{clip}%
\pgfsetbuttcap%
\pgfsetroundjoin%
\definecolor{currentfill}{rgb}{0.214298,0.355619,0.551184}%
\pgfsetfillcolor{currentfill}%
\pgfsetfillopacity{0.700000}%
\pgfsetlinewidth{0.000000pt}%
\definecolor{currentstroke}{rgb}{0.000000,0.000000,0.000000}%
\pgfsetstrokecolor{currentstroke}%
\pgfsetdash{}{0pt}%
\pgfpathmoveto{\pgfqpoint{5.355692in}{2.624122in}}%
\pgfpathlineto{\pgfqpoint{5.369366in}{2.627742in}}%
\pgfpathlineto{\pgfqpoint{5.383052in}{2.631475in}}%
\pgfpathlineto{\pgfqpoint{5.396751in}{2.635321in}}%
\pgfpathlineto{\pgfqpoint{5.410464in}{2.639279in}}%
\pgfpathlineto{\pgfqpoint{5.417637in}{2.648383in}}%
\pgfpathlineto{\pgfqpoint{5.424806in}{2.657451in}}%
\pgfpathlineto{\pgfqpoint{5.431968in}{2.666487in}}%
\pgfpathlineto{\pgfqpoint{5.439125in}{2.675490in}}%
\pgfpathlineto{\pgfqpoint{5.425423in}{2.671623in}}%
\pgfpathlineto{\pgfqpoint{5.411735in}{2.667869in}}%
\pgfpathlineto{\pgfqpoint{5.398059in}{2.664228in}}%
\pgfpathlineto{\pgfqpoint{5.384396in}{2.660699in}}%
\pgfpathlineto{\pgfqpoint{5.377228in}{2.651599in}}%
\pgfpathlineto{\pgfqpoint{5.370055in}{2.642469in}}%
\pgfpathlineto{\pgfqpoint{5.362876in}{2.633311in}}%
\pgfpathlineto{\pgfqpoint{5.355692in}{2.624122in}}%
\pgfpathclose%
\pgfusepath{fill}%
\end{pgfscope}%
\begin{pgfscope}%
\pgfpathrectangle{\pgfqpoint{1.254980in}{0.150000in}}{\pgfqpoint{5.490039in}{5.490039in}}%
\pgfusepath{clip}%
\pgfsetbuttcap%
\pgfsetroundjoin%
\definecolor{currentfill}{rgb}{0.233603,0.313828,0.543914}%
\pgfsetfillcolor{currentfill}%
\pgfsetfillopacity{0.700000}%
\pgfsetlinewidth{0.000000pt}%
\definecolor{currentstroke}{rgb}{0.000000,0.000000,0.000000}%
\pgfsetstrokecolor{currentstroke}%
\pgfsetdash{}{0pt}%
\pgfpathmoveto{\pgfqpoint{3.003473in}{2.573086in}}%
\pgfpathlineto{\pgfqpoint{3.016686in}{2.557555in}}%
\pgfpathlineto{\pgfqpoint{3.029896in}{2.542196in}}%
\pgfpathlineto{\pgfqpoint{3.043102in}{2.527009in}}%
\pgfpathlineto{\pgfqpoint{3.056305in}{2.511994in}}%
\pgfpathlineto{\pgfqpoint{3.064367in}{2.516394in}}%
\pgfpathlineto{\pgfqpoint{3.072419in}{2.520920in}}%
\pgfpathlineto{\pgfqpoint{3.080462in}{2.525572in}}%
\pgfpathlineto{\pgfqpoint{3.088496in}{2.530348in}}%
\pgfpathlineto{\pgfqpoint{3.075319in}{2.545122in}}%
\pgfpathlineto{\pgfqpoint{3.062140in}{2.560067in}}%
\pgfpathlineto{\pgfqpoint{3.048957in}{2.575183in}}%
\pgfpathlineto{\pgfqpoint{3.035771in}{2.590472in}}%
\pgfpathlineto{\pgfqpoint{3.027711in}{2.585932in}}%
\pgfpathlineto{\pgfqpoint{3.019641in}{2.581520in}}%
\pgfpathlineto{\pgfqpoint{3.011562in}{2.577238in}}%
\pgfpathlineto{\pgfqpoint{3.003473in}{2.573086in}}%
\pgfpathclose%
\pgfusepath{fill}%
\end{pgfscope}%
\begin{pgfscope}%
\pgfpathrectangle{\pgfqpoint{1.254980in}{0.150000in}}{\pgfqpoint{5.490039in}{5.490039in}}%
\pgfusepath{clip}%
\pgfsetbuttcap%
\pgfsetroundjoin%
\definecolor{currentfill}{rgb}{0.220057,0.343307,0.549413}%
\pgfsetfillcolor{currentfill}%
\pgfsetfillopacity{0.700000}%
\pgfsetlinewidth{0.000000pt}%
\definecolor{currentstroke}{rgb}{0.000000,0.000000,0.000000}%
\pgfsetstrokecolor{currentstroke}%
\pgfsetdash{}{0pt}%
\pgfpathmoveto{\pgfqpoint{2.950579in}{2.636961in}}%
\pgfpathlineto{\pgfqpoint{2.963808in}{2.620728in}}%
\pgfpathlineto{\pgfqpoint{2.977034in}{2.604672in}}%
\pgfpathlineto{\pgfqpoint{2.990255in}{2.588791in}}%
\pgfpathlineto{\pgfqpoint{3.003473in}{2.573086in}}%
\pgfpathlineto{\pgfqpoint{3.011562in}{2.577238in}}%
\pgfpathlineto{\pgfqpoint{3.019641in}{2.581520in}}%
\pgfpathlineto{\pgfqpoint{3.027711in}{2.585932in}}%
\pgfpathlineto{\pgfqpoint{3.035771in}{2.590472in}}%
\pgfpathlineto{\pgfqpoint{3.022581in}{2.605934in}}%
\pgfpathlineto{\pgfqpoint{3.009388in}{2.621571in}}%
\pgfpathlineto{\pgfqpoint{2.996190in}{2.637383in}}%
\pgfpathlineto{\pgfqpoint{2.982989in}{2.653373in}}%
\pgfpathlineto{\pgfqpoint{2.974901in}{2.649070in}}%
\pgfpathlineto{\pgfqpoint{2.966804in}{2.644900in}}%
\pgfpathlineto{\pgfqpoint{2.958696in}{2.640863in}}%
\pgfpathlineto{\pgfqpoint{2.950579in}{2.636961in}}%
\pgfpathclose%
\pgfusepath{fill}%
\end{pgfscope}%
\begin{pgfscope}%
\pgfpathrectangle{\pgfqpoint{1.254980in}{0.150000in}}{\pgfqpoint{5.490039in}{5.490039in}}%
\pgfusepath{clip}%
\pgfsetbuttcap%
\pgfsetroundjoin%
\definecolor{currentfill}{rgb}{0.244972,0.287675,0.537260}%
\pgfsetfillcolor{currentfill}%
\pgfsetfillopacity{0.700000}%
\pgfsetlinewidth{0.000000pt}%
\definecolor{currentstroke}{rgb}{0.000000,0.000000,0.000000}%
\pgfsetstrokecolor{currentstroke}%
\pgfsetdash{}{0pt}%
\pgfpathmoveto{\pgfqpoint{3.056305in}{2.511994in}}%
\pgfpathlineto{\pgfqpoint{3.069505in}{2.497148in}}%
\pgfpathlineto{\pgfqpoint{3.082701in}{2.482471in}}%
\pgfpathlineto{\pgfqpoint{3.095895in}{2.467962in}}%
\pgfpathlineto{\pgfqpoint{3.109086in}{2.453620in}}%
\pgfpathlineto{\pgfqpoint{3.117121in}{2.458267in}}%
\pgfpathlineto{\pgfqpoint{3.125147in}{2.463036in}}%
\pgfpathlineto{\pgfqpoint{3.133164in}{2.467927in}}%
\pgfpathlineto{\pgfqpoint{3.141172in}{2.472936in}}%
\pgfpathlineto{\pgfqpoint{3.128007in}{2.487038in}}%
\pgfpathlineto{\pgfqpoint{3.114839in}{2.501307in}}%
\pgfpathlineto{\pgfqpoint{3.101669in}{2.515743in}}%
\pgfpathlineto{\pgfqpoint{3.088496in}{2.530348in}}%
\pgfpathlineto{\pgfqpoint{3.080462in}{2.525572in}}%
\pgfpathlineto{\pgfqpoint{3.072419in}{2.520920in}}%
\pgfpathlineto{\pgfqpoint{3.064367in}{2.516394in}}%
\pgfpathlineto{\pgfqpoint{3.056305in}{2.511994in}}%
\pgfpathclose%
\pgfusepath{fill}%
\end{pgfscope}%
\begin{pgfscope}%
\pgfpathrectangle{\pgfqpoint{1.254980in}{0.150000in}}{\pgfqpoint{5.490039in}{5.490039in}}%
\pgfusepath{clip}%
\pgfsetbuttcap%
\pgfsetroundjoin%
\definecolor{currentfill}{rgb}{0.282910,0.105393,0.426902}%
\pgfsetfillcolor{currentfill}%
\pgfsetfillopacity{0.700000}%
\pgfsetlinewidth{0.000000pt}%
\definecolor{currentstroke}{rgb}{0.000000,0.000000,0.000000}%
\pgfsetstrokecolor{currentstroke}%
\pgfsetdash{}{0pt}%
\pgfpathmoveto{\pgfqpoint{4.356750in}{2.104612in}}%
\pgfpathlineto{\pgfqpoint{4.370026in}{2.102460in}}%
\pgfpathlineto{\pgfqpoint{4.383309in}{2.100431in}}%
\pgfpathlineto{\pgfqpoint{4.396601in}{2.098522in}}%
\pgfpathlineto{\pgfqpoint{4.409900in}{2.096735in}}%
\pgfpathlineto{\pgfqpoint{4.417417in}{2.106779in}}%
\pgfpathlineto{\pgfqpoint{4.424929in}{2.116825in}}%
\pgfpathlineto{\pgfqpoint{4.432437in}{2.126874in}}%
\pgfpathlineto{\pgfqpoint{4.439940in}{2.136924in}}%
\pgfpathlineto{\pgfqpoint{4.426649in}{2.138610in}}%
\pgfpathlineto{\pgfqpoint{4.413367in}{2.140416in}}%
\pgfpathlineto{\pgfqpoint{4.400092in}{2.142344in}}%
\pgfpathlineto{\pgfqpoint{4.386825in}{2.144394in}}%
\pgfpathlineto{\pgfqpoint{4.379314in}{2.134439in}}%
\pgfpathlineto{\pgfqpoint{4.371797in}{2.124490in}}%
\pgfpathlineto{\pgfqpoint{4.364276in}{2.114548in}}%
\pgfpathlineto{\pgfqpoint{4.356750in}{2.104612in}}%
\pgfpathclose%
\pgfusepath{fill}%
\end{pgfscope}%
\begin{pgfscope}%
\pgfpathrectangle{\pgfqpoint{1.254980in}{0.150000in}}{\pgfqpoint{5.490039in}{5.490039in}}%
\pgfusepath{clip}%
\pgfsetbuttcap%
\pgfsetroundjoin%
\definecolor{currentfill}{rgb}{0.208623,0.367752,0.552675}%
\pgfsetfillcolor{currentfill}%
\pgfsetfillopacity{0.700000}%
\pgfsetlinewidth{0.000000pt}%
\definecolor{currentstroke}{rgb}{0.000000,0.000000,0.000000}%
\pgfsetstrokecolor{currentstroke}%
\pgfsetdash{}{0pt}%
\pgfpathmoveto{\pgfqpoint{2.897613in}{2.703687in}}%
\pgfpathlineto{\pgfqpoint{2.910862in}{2.686734in}}%
\pgfpathlineto{\pgfqpoint{2.924106in}{2.669963in}}%
\pgfpathlineto{\pgfqpoint{2.937344in}{2.653373in}}%
\pgfpathlineto{\pgfqpoint{2.950579in}{2.636961in}}%
\pgfpathlineto{\pgfqpoint{2.958696in}{2.640863in}}%
\pgfpathlineto{\pgfqpoint{2.966804in}{2.644900in}}%
\pgfpathlineto{\pgfqpoint{2.974901in}{2.649070in}}%
\pgfpathlineto{\pgfqpoint{2.982989in}{2.653373in}}%
\pgfpathlineto{\pgfqpoint{2.969783in}{2.669539in}}%
\pgfpathlineto{\pgfqpoint{2.956573in}{2.685885in}}%
\pgfpathlineto{\pgfqpoint{2.943358in}{2.702411in}}%
\pgfpathlineto{\pgfqpoint{2.930139in}{2.719117in}}%
\pgfpathlineto{\pgfqpoint{2.922023in}{2.715054in}}%
\pgfpathlineto{\pgfqpoint{2.913897in}{2.711126in}}%
\pgfpathlineto{\pgfqpoint{2.905760in}{2.707336in}}%
\pgfpathlineto{\pgfqpoint{2.897613in}{2.703687in}}%
\pgfpathclose%
\pgfusepath{fill}%
\end{pgfscope}%
\begin{pgfscope}%
\pgfpathrectangle{\pgfqpoint{1.254980in}{0.150000in}}{\pgfqpoint{5.490039in}{5.490039in}}%
\pgfusepath{clip}%
\pgfsetbuttcap%
\pgfsetroundjoin%
\definecolor{currentfill}{rgb}{0.282327,0.094955,0.417331}%
\pgfsetfillcolor{currentfill}%
\pgfsetfillopacity{0.700000}%
\pgfsetlinewidth{0.000000pt}%
\definecolor{currentstroke}{rgb}{0.000000,0.000000,0.000000}%
\pgfsetstrokecolor{currentstroke}%
\pgfsetdash{}{0pt}%
\pgfpathmoveto{\pgfqpoint{3.645742in}{2.098338in}}%
\pgfpathlineto{\pgfqpoint{3.658888in}{2.090058in}}%
\pgfpathlineto{\pgfqpoint{3.672037in}{2.081916in}}%
\pgfpathlineto{\pgfqpoint{3.685189in}{2.073913in}}%
\pgfpathlineto{\pgfqpoint{3.698343in}{2.066047in}}%
\pgfpathlineto{\pgfqpoint{3.706108in}{2.073707in}}%
\pgfpathlineto{\pgfqpoint{3.713866in}{2.081436in}}%
\pgfpathlineto{\pgfqpoint{3.721617in}{2.089231in}}%
\pgfpathlineto{\pgfqpoint{3.729363in}{2.097092in}}%
\pgfpathlineto{\pgfqpoint{3.716225in}{2.104759in}}%
\pgfpathlineto{\pgfqpoint{3.703089in}{2.112563in}}%
\pgfpathlineto{\pgfqpoint{3.689957in}{2.120506in}}%
\pgfpathlineto{\pgfqpoint{3.676827in}{2.128586in}}%
\pgfpathlineto{\pgfqpoint{3.669065in}{2.120919in}}%
\pgfpathlineto{\pgfqpoint{3.661297in}{2.113321in}}%
\pgfpathlineto{\pgfqpoint{3.653523in}{2.105793in}}%
\pgfpathlineto{\pgfqpoint{3.645742in}{2.098338in}}%
\pgfpathclose%
\pgfusepath{fill}%
\end{pgfscope}%
\begin{pgfscope}%
\pgfpathrectangle{\pgfqpoint{1.254980in}{0.150000in}}{\pgfqpoint{5.490039in}{5.490039in}}%
\pgfusepath{clip}%
\pgfsetbuttcap%
\pgfsetroundjoin%
\definecolor{currentfill}{rgb}{0.203063,0.379716,0.553925}%
\pgfsetfillcolor{currentfill}%
\pgfsetfillopacity{0.700000}%
\pgfsetlinewidth{0.000000pt}%
\definecolor{currentstroke}{rgb}{0.000000,0.000000,0.000000}%
\pgfsetstrokecolor{currentstroke}%
\pgfsetdash{}{0pt}%
\pgfpathmoveto{\pgfqpoint{5.439125in}{2.675490in}}%
\pgfpathlineto{\pgfqpoint{5.452840in}{2.679469in}}%
\pgfpathlineto{\pgfqpoint{5.466569in}{2.683560in}}%
\pgfpathlineto{\pgfqpoint{5.480311in}{2.687764in}}%
\pgfpathlineto{\pgfqpoint{5.494066in}{2.692080in}}%
\pgfpathlineto{\pgfqpoint{5.501206in}{2.700950in}}%
\pgfpathlineto{\pgfqpoint{5.508341in}{2.709785in}}%
\pgfpathlineto{\pgfqpoint{5.515470in}{2.718588in}}%
\pgfpathlineto{\pgfqpoint{5.522594in}{2.727359in}}%
\pgfpathlineto{\pgfqpoint{5.508850in}{2.723152in}}%
\pgfpathlineto{\pgfqpoint{5.495119in}{2.719057in}}%
\pgfpathlineto{\pgfqpoint{5.481402in}{2.715073in}}%
\pgfpathlineto{\pgfqpoint{5.467698in}{2.711203in}}%
\pgfpathlineto{\pgfqpoint{5.460563in}{2.702317in}}%
\pgfpathlineto{\pgfqpoint{5.453423in}{2.693404in}}%
\pgfpathlineto{\pgfqpoint{5.446277in}{2.684462in}}%
\pgfpathlineto{\pgfqpoint{5.439125in}{2.675490in}}%
\pgfpathclose%
\pgfusepath{fill}%
\end{pgfscope}%
\begin{pgfscope}%
\pgfpathrectangle{\pgfqpoint{1.254980in}{0.150000in}}{\pgfqpoint{5.490039in}{5.490039in}}%
\pgfusepath{clip}%
\pgfsetbuttcap%
\pgfsetroundjoin%
\definecolor{currentfill}{rgb}{0.253935,0.265254,0.529983}%
\pgfsetfillcolor{currentfill}%
\pgfsetfillopacity{0.700000}%
\pgfsetlinewidth{0.000000pt}%
\definecolor{currentstroke}{rgb}{0.000000,0.000000,0.000000}%
\pgfsetstrokecolor{currentstroke}%
\pgfsetdash{}{0pt}%
\pgfpathmoveto{\pgfqpoint{3.109086in}{2.453620in}}%
\pgfpathlineto{\pgfqpoint{3.122274in}{2.439444in}}%
\pgfpathlineto{\pgfqpoint{3.135460in}{2.425433in}}%
\pgfpathlineto{\pgfqpoint{3.148643in}{2.411586in}}%
\pgfpathlineto{\pgfqpoint{3.161824in}{2.397903in}}%
\pgfpathlineto{\pgfqpoint{3.169834in}{2.402796in}}%
\pgfpathlineto{\pgfqpoint{3.177835in}{2.407807in}}%
\pgfpathlineto{\pgfqpoint{3.185826in}{2.412935in}}%
\pgfpathlineto{\pgfqpoint{3.193809in}{2.418177in}}%
\pgfpathlineto{\pgfqpoint{3.180653in}{2.431622in}}%
\pgfpathlineto{\pgfqpoint{3.167495in}{2.445229in}}%
\pgfpathlineto{\pgfqpoint{3.154335in}{2.459000in}}%
\pgfpathlineto{\pgfqpoint{3.141172in}{2.472936in}}%
\pgfpathlineto{\pgfqpoint{3.133164in}{2.467927in}}%
\pgfpathlineto{\pgfqpoint{3.125147in}{2.463036in}}%
\pgfpathlineto{\pgfqpoint{3.117121in}{2.458267in}}%
\pgfpathlineto{\pgfqpoint{3.109086in}{2.453620in}}%
\pgfpathclose%
\pgfusepath{fill}%
\end{pgfscope}%
\begin{pgfscope}%
\pgfpathrectangle{\pgfqpoint{1.254980in}{0.150000in}}{\pgfqpoint{5.490039in}{5.490039in}}%
\pgfusepath{clip}%
\pgfsetbuttcap%
\pgfsetroundjoin%
\definecolor{currentfill}{rgb}{0.282884,0.135920,0.453427}%
\pgfsetfillcolor{currentfill}%
\pgfsetfillopacity{0.700000}%
\pgfsetlinewidth{0.000000pt}%
\definecolor{currentstroke}{rgb}{0.000000,0.000000,0.000000}%
\pgfsetstrokecolor{currentstroke}%
\pgfsetdash{}{0pt}%
\pgfpathmoveto{\pgfqpoint{3.456724in}{2.182253in}}%
\pgfpathlineto{\pgfqpoint{3.469869in}{2.172046in}}%
\pgfpathlineto{\pgfqpoint{3.483015in}{2.161985in}}%
\pgfpathlineto{\pgfqpoint{3.496162in}{2.152069in}}%
\pgfpathlineto{\pgfqpoint{3.509311in}{2.142298in}}%
\pgfpathlineto{\pgfqpoint{3.517155in}{2.149005in}}%
\pgfpathlineto{\pgfqpoint{3.524992in}{2.155799in}}%
\pgfpathlineto{\pgfqpoint{3.532823in}{2.162677in}}%
\pgfpathlineto{\pgfqpoint{3.540646in}{2.169639in}}%
\pgfpathlineto{\pgfqpoint{3.527517in}{2.179194in}}%
\pgfpathlineto{\pgfqpoint{3.514389in}{2.188892in}}%
\pgfpathlineto{\pgfqpoint{3.501262in}{2.198737in}}%
\pgfpathlineto{\pgfqpoint{3.488137in}{2.208727in}}%
\pgfpathlineto{\pgfqpoint{3.480294in}{2.201975in}}%
\pgfpathlineto{\pgfqpoint{3.472445in}{2.195312in}}%
\pgfpathlineto{\pgfqpoint{3.464588in}{2.188737in}}%
\pgfpathlineto{\pgfqpoint{3.456724in}{2.182253in}}%
\pgfpathclose%
\pgfusepath{fill}%
\end{pgfscope}%
\begin{pgfscope}%
\pgfpathrectangle{\pgfqpoint{1.254980in}{0.150000in}}{\pgfqpoint{5.490039in}{5.490039in}}%
\pgfusepath{clip}%
\pgfsetbuttcap%
\pgfsetroundjoin%
\definecolor{currentfill}{rgb}{0.280267,0.073417,0.397163}%
\pgfsetfillcolor{currentfill}%
\pgfsetfillopacity{0.700000}%
\pgfsetlinewidth{0.000000pt}%
\definecolor{currentstroke}{rgb}{0.000000,0.000000,0.000000}%
\pgfsetstrokecolor{currentstroke}%
\pgfsetdash{}{0pt}%
\pgfpathmoveto{\pgfqpoint{4.054113in}{2.043876in}}%
\pgfpathlineto{\pgfqpoint{4.067313in}{2.039330in}}%
\pgfpathlineto{\pgfqpoint{4.080520in}{2.034911in}}%
\pgfpathlineto{\pgfqpoint{4.093732in}{2.030619in}}%
\pgfpathlineto{\pgfqpoint{4.106950in}{2.026454in}}%
\pgfpathlineto{\pgfqpoint{4.114565in}{2.035778in}}%
\pgfpathlineto{\pgfqpoint{4.122176in}{2.045132in}}%
\pgfpathlineto{\pgfqpoint{4.129781in}{2.054513in}}%
\pgfpathlineto{\pgfqpoint{4.137381in}{2.063921in}}%
\pgfpathlineto{\pgfqpoint{4.124174in}{2.067937in}}%
\pgfpathlineto{\pgfqpoint{4.110973in}{2.072080in}}%
\pgfpathlineto{\pgfqpoint{4.097777in}{2.076349in}}%
\pgfpathlineto{\pgfqpoint{4.084588in}{2.080746in}}%
\pgfpathlineto{\pgfqpoint{4.076977in}{2.071481in}}%
\pgfpathlineto{\pgfqpoint{4.069361in}{2.062247in}}%
\pgfpathlineto{\pgfqpoint{4.061739in}{2.053045in}}%
\pgfpathlineto{\pgfqpoint{4.054113in}{2.043876in}}%
\pgfpathclose%
\pgfusepath{fill}%
\end{pgfscope}%
\begin{pgfscope}%
\pgfpathrectangle{\pgfqpoint{1.254980in}{0.150000in}}{\pgfqpoint{5.490039in}{5.490039in}}%
\pgfusepath{clip}%
\pgfsetbuttcap%
\pgfsetroundjoin%
\definecolor{currentfill}{rgb}{0.194100,0.399323,0.555565}%
\pgfsetfillcolor{currentfill}%
\pgfsetfillopacity{0.700000}%
\pgfsetlinewidth{0.000000pt}%
\definecolor{currentstroke}{rgb}{0.000000,0.000000,0.000000}%
\pgfsetstrokecolor{currentstroke}%
\pgfsetdash{}{0pt}%
\pgfpathmoveto{\pgfqpoint{2.844566in}{2.773334in}}%
\pgfpathlineto{\pgfqpoint{2.857836in}{2.755644in}}%
\pgfpathlineto{\pgfqpoint{2.871101in}{2.738140in}}%
\pgfpathlineto{\pgfqpoint{2.884360in}{2.720822in}}%
\pgfpathlineto{\pgfqpoint{2.897613in}{2.703687in}}%
\pgfpathlineto{\pgfqpoint{2.905760in}{2.707336in}}%
\pgfpathlineto{\pgfqpoint{2.913897in}{2.711126in}}%
\pgfpathlineto{\pgfqpoint{2.922023in}{2.715054in}}%
\pgfpathlineto{\pgfqpoint{2.930139in}{2.719117in}}%
\pgfpathlineto{\pgfqpoint{2.916915in}{2.736006in}}%
\pgfpathlineto{\pgfqpoint{2.903685in}{2.753078in}}%
\pgfpathlineto{\pgfqpoint{2.890451in}{2.770335in}}%
\pgfpathlineto{\pgfqpoint{2.877211in}{2.787777in}}%
\pgfpathlineto{\pgfqpoint{2.869066in}{2.783953in}}%
\pgfpathlineto{\pgfqpoint{2.860910in}{2.780270in}}%
\pgfpathlineto{\pgfqpoint{2.852744in}{2.776730in}}%
\pgfpathlineto{\pgfqpoint{2.844566in}{2.773334in}}%
\pgfpathclose%
\pgfusepath{fill}%
\end{pgfscope}%
\begin{pgfscope}%
\pgfpathrectangle{\pgfqpoint{1.254980in}{0.150000in}}{\pgfqpoint{5.490039in}{5.490039in}}%
\pgfusepath{clip}%
\pgfsetbuttcap%
\pgfsetroundjoin%
\definecolor{currentfill}{rgb}{0.194100,0.399323,0.555565}%
\pgfsetfillcolor{currentfill}%
\pgfsetfillopacity{0.700000}%
\pgfsetlinewidth{0.000000pt}%
\definecolor{currentstroke}{rgb}{0.000000,0.000000,0.000000}%
\pgfsetstrokecolor{currentstroke}%
\pgfsetdash{}{0pt}%
\pgfpathmoveto{\pgfqpoint{5.522594in}{2.727359in}}%
\pgfpathlineto{\pgfqpoint{5.536352in}{2.731679in}}%
\pgfpathlineto{\pgfqpoint{5.550123in}{2.736111in}}%
\pgfpathlineto{\pgfqpoint{5.563908in}{2.740654in}}%
\pgfpathlineto{\pgfqpoint{5.577707in}{2.745310in}}%
\pgfpathlineto{\pgfqpoint{5.584813in}{2.753932in}}%
\pgfpathlineto{\pgfqpoint{5.591914in}{2.762521in}}%
\pgfpathlineto{\pgfqpoint{5.599008in}{2.771079in}}%
\pgfpathlineto{\pgfqpoint{5.606097in}{2.779606in}}%
\pgfpathlineto{\pgfqpoint{5.592311in}{2.775076in}}%
\pgfpathlineto{\pgfqpoint{5.578538in}{2.770658in}}%
\pgfpathlineto{\pgfqpoint{5.564778in}{2.766351in}}%
\pgfpathlineto{\pgfqpoint{5.551033in}{2.762156in}}%
\pgfpathlineto{\pgfqpoint{5.543931in}{2.753498in}}%
\pgfpathlineto{\pgfqpoint{5.536824in}{2.744813in}}%
\pgfpathlineto{\pgfqpoint{5.529712in}{2.736101in}}%
\pgfpathlineto{\pgfqpoint{5.522594in}{2.727359in}}%
\pgfpathclose%
\pgfusepath{fill}%
\end{pgfscope}%
\begin{pgfscope}%
\pgfpathrectangle{\pgfqpoint{1.254980in}{0.150000in}}{\pgfqpoint{5.490039in}{5.490039in}}%
\pgfusepath{clip}%
\pgfsetbuttcap%
\pgfsetroundjoin%
\definecolor{currentfill}{rgb}{0.263663,0.237631,0.518762}%
\pgfsetfillcolor{currentfill}%
\pgfsetfillopacity{0.700000}%
\pgfsetlinewidth{0.000000pt}%
\definecolor{currentstroke}{rgb}{0.000000,0.000000,0.000000}%
\pgfsetstrokecolor{currentstroke}%
\pgfsetdash{}{0pt}%
\pgfpathmoveto{\pgfqpoint{3.161824in}{2.397903in}}%
\pgfpathlineto{\pgfqpoint{3.175003in}{2.384382in}}%
\pgfpathlineto{\pgfqpoint{3.188180in}{2.371023in}}%
\pgfpathlineto{\pgfqpoint{3.201356in}{2.357825in}}%
\pgfpathlineto{\pgfqpoint{3.214529in}{2.344786in}}%
\pgfpathlineto{\pgfqpoint{3.222514in}{2.349923in}}%
\pgfpathlineto{\pgfqpoint{3.230490in}{2.355175in}}%
\pgfpathlineto{\pgfqpoint{3.238458in}{2.360539in}}%
\pgfpathlineto{\pgfqpoint{3.246417in}{2.366013in}}%
\pgfpathlineto{\pgfqpoint{3.233268in}{2.378814in}}%
\pgfpathlineto{\pgfqpoint{3.220117in}{2.391774in}}%
\pgfpathlineto{\pgfqpoint{3.206964in}{2.404895in}}%
\pgfpathlineto{\pgfqpoint{3.193809in}{2.418177in}}%
\pgfpathlineto{\pgfqpoint{3.185826in}{2.412935in}}%
\pgfpathlineto{\pgfqpoint{3.177835in}{2.407807in}}%
\pgfpathlineto{\pgfqpoint{3.169834in}{2.402796in}}%
\pgfpathlineto{\pgfqpoint{3.161824in}{2.397903in}}%
\pgfpathclose%
\pgfusepath{fill}%
\end{pgfscope}%
\begin{pgfscope}%
\pgfpathrectangle{\pgfqpoint{1.254980in}{0.150000in}}{\pgfqpoint{5.490039in}{5.490039in}}%
\pgfusepath{clip}%
\pgfsetbuttcap%
\pgfsetroundjoin%
\definecolor{currentfill}{rgb}{0.282327,0.094955,0.417331}%
\pgfsetfillcolor{currentfill}%
\pgfsetfillopacity{0.700000}%
\pgfsetlinewidth{0.000000pt}%
\definecolor{currentstroke}{rgb}{0.000000,0.000000,0.000000}%
\pgfsetstrokecolor{currentstroke}%
\pgfsetdash{}{0pt}%
\pgfpathmoveto{\pgfqpoint{4.273532in}{2.075257in}}%
\pgfpathlineto{\pgfqpoint{4.286787in}{2.072497in}}%
\pgfpathlineto{\pgfqpoint{4.300050in}{2.069860in}}%
\pgfpathlineto{\pgfqpoint{4.313320in}{2.067346in}}%
\pgfpathlineto{\pgfqpoint{4.326598in}{2.064954in}}%
\pgfpathlineto{\pgfqpoint{4.334143in}{2.074855in}}%
\pgfpathlineto{\pgfqpoint{4.341684in}{2.084765in}}%
\pgfpathlineto{\pgfqpoint{4.349219in}{2.094684in}}%
\pgfpathlineto{\pgfqpoint{4.356750in}{2.104612in}}%
\pgfpathlineto{\pgfqpoint{4.343482in}{2.106886in}}%
\pgfpathlineto{\pgfqpoint{4.330221in}{2.109282in}}%
\pgfpathlineto{\pgfqpoint{4.316968in}{2.111802in}}%
\pgfpathlineto{\pgfqpoint{4.303721in}{2.114444in}}%
\pgfpathlineto{\pgfqpoint{4.296181in}{2.104628in}}%
\pgfpathlineto{\pgfqpoint{4.288636in}{2.094824in}}%
\pgfpathlineto{\pgfqpoint{4.281086in}{2.085034in}}%
\pgfpathlineto{\pgfqpoint{4.273532in}{2.075257in}}%
\pgfpathclose%
\pgfusepath{fill}%
\end{pgfscope}%
\begin{pgfscope}%
\pgfpathrectangle{\pgfqpoint{1.254980in}{0.150000in}}{\pgfqpoint{5.490039in}{5.490039in}}%
\pgfusepath{clip}%
\pgfsetbuttcap%
\pgfsetroundjoin%
\definecolor{currentfill}{rgb}{0.183898,0.422383,0.556944}%
\pgfsetfillcolor{currentfill}%
\pgfsetfillopacity{0.700000}%
\pgfsetlinewidth{0.000000pt}%
\definecolor{currentstroke}{rgb}{0.000000,0.000000,0.000000}%
\pgfsetstrokecolor{currentstroke}%
\pgfsetdash{}{0pt}%
\pgfpathmoveto{\pgfqpoint{5.606097in}{2.779606in}}%
\pgfpathlineto{\pgfqpoint{5.619898in}{2.784248in}}%
\pgfpathlineto{\pgfqpoint{5.633713in}{2.789002in}}%
\pgfpathlineto{\pgfqpoint{5.647542in}{2.793867in}}%
\pgfpathlineto{\pgfqpoint{5.661385in}{2.798843in}}%
\pgfpathlineto{\pgfqpoint{5.668455in}{2.807206in}}%
\pgfpathlineto{\pgfqpoint{5.675520in}{2.815538in}}%
\pgfpathlineto{\pgfqpoint{5.682580in}{2.823841in}}%
\pgfpathlineto{\pgfqpoint{5.689633in}{2.832115in}}%
\pgfpathlineto{\pgfqpoint{5.675803in}{2.827280in}}%
\pgfpathlineto{\pgfqpoint{5.661987in}{2.822557in}}%
\pgfpathlineto{\pgfqpoint{5.648186in}{2.817946in}}%
\pgfpathlineto{\pgfqpoint{5.634398in}{2.813445in}}%
\pgfpathlineto{\pgfqpoint{5.627331in}{2.805023in}}%
\pgfpathlineto{\pgfqpoint{5.620259in}{2.796577in}}%
\pgfpathlineto{\pgfqpoint{5.613181in}{2.788105in}}%
\pgfpathlineto{\pgfqpoint{5.606097in}{2.779606in}}%
\pgfpathclose%
\pgfusepath{fill}%
\end{pgfscope}%
\begin{pgfscope}%
\pgfpathrectangle{\pgfqpoint{1.254980in}{0.150000in}}{\pgfqpoint{5.490039in}{5.490039in}}%
\pgfusepath{clip}%
\pgfsetbuttcap%
\pgfsetroundjoin%
\definecolor{currentfill}{rgb}{0.182256,0.426184,0.557120}%
\pgfsetfillcolor{currentfill}%
\pgfsetfillopacity{0.700000}%
\pgfsetlinewidth{0.000000pt}%
\definecolor{currentstroke}{rgb}{0.000000,0.000000,0.000000}%
\pgfsetstrokecolor{currentstroke}%
\pgfsetdash{}{0pt}%
\pgfpathmoveto{\pgfqpoint{2.791427in}{2.845977in}}%
\pgfpathlineto{\pgfqpoint{2.804721in}{2.827531in}}%
\pgfpathlineto{\pgfqpoint{2.818009in}{2.809276in}}%
\pgfpathlineto{\pgfqpoint{2.831291in}{2.791211in}}%
\pgfpathlineto{\pgfqpoint{2.844566in}{2.773334in}}%
\pgfpathlineto{\pgfqpoint{2.852744in}{2.776730in}}%
\pgfpathlineto{\pgfqpoint{2.860910in}{2.780270in}}%
\pgfpathlineto{\pgfqpoint{2.869066in}{2.783953in}}%
\pgfpathlineto{\pgfqpoint{2.877211in}{2.787777in}}%
\pgfpathlineto{\pgfqpoint{2.863966in}{2.805406in}}%
\pgfpathlineto{\pgfqpoint{2.850715in}{2.823223in}}%
\pgfpathlineto{\pgfqpoint{2.837458in}{2.841229in}}%
\pgfpathlineto{\pgfqpoint{2.824195in}{2.859425in}}%
\pgfpathlineto{\pgfqpoint{2.816020in}{2.855844in}}%
\pgfpathlineto{\pgfqpoint{2.807833in}{2.852408in}}%
\pgfpathlineto{\pgfqpoint{2.799636in}{2.849118in}}%
\pgfpathlineto{\pgfqpoint{2.791427in}{2.845977in}}%
\pgfpathclose%
\pgfusepath{fill}%
\end{pgfscope}%
\begin{pgfscope}%
\pgfpathrectangle{\pgfqpoint{1.254980in}{0.150000in}}{\pgfqpoint{5.490039in}{5.490039in}}%
\pgfusepath{clip}%
\pgfsetbuttcap%
\pgfsetroundjoin%
\definecolor{currentfill}{rgb}{0.270595,0.214069,0.507052}%
\pgfsetfillcolor{currentfill}%
\pgfsetfillopacity{0.700000}%
\pgfsetlinewidth{0.000000pt}%
\definecolor{currentstroke}{rgb}{0.000000,0.000000,0.000000}%
\pgfsetstrokecolor{currentstroke}%
\pgfsetdash{}{0pt}%
\pgfpathmoveto{\pgfqpoint{3.214529in}{2.344786in}}%
\pgfpathlineto{\pgfqpoint{3.227702in}{2.331907in}}%
\pgfpathlineto{\pgfqpoint{3.240872in}{2.319185in}}%
\pgfpathlineto{\pgfqpoint{3.254042in}{2.306621in}}%
\pgfpathlineto{\pgfqpoint{3.267210in}{2.294214in}}%
\pgfpathlineto{\pgfqpoint{3.275171in}{2.299594in}}%
\pgfpathlineto{\pgfqpoint{3.283124in}{2.305085in}}%
\pgfpathlineto{\pgfqpoint{3.291068in}{2.310684in}}%
\pgfpathlineto{\pgfqpoint{3.299004in}{2.316389in}}%
\pgfpathlineto{\pgfqpoint{3.285859in}{2.328560in}}%
\pgfpathlineto{\pgfqpoint{3.272713in}{2.340887in}}%
\pgfpathlineto{\pgfqpoint{3.259566in}{2.353371in}}%
\pgfpathlineto{\pgfqpoint{3.246417in}{2.366013in}}%
\pgfpathlineto{\pgfqpoint{3.238458in}{2.360539in}}%
\pgfpathlineto{\pgfqpoint{3.230490in}{2.355175in}}%
\pgfpathlineto{\pgfqpoint{3.222514in}{2.349923in}}%
\pgfpathlineto{\pgfqpoint{3.214529in}{2.344786in}}%
\pgfpathclose%
\pgfusepath{fill}%
\end{pgfscope}%
\begin{pgfscope}%
\pgfpathrectangle{\pgfqpoint{1.254980in}{0.150000in}}{\pgfqpoint{5.490039in}{5.490039in}}%
\pgfusepath{clip}%
\pgfsetbuttcap%
\pgfsetroundjoin%
\definecolor{currentfill}{rgb}{0.174274,0.445044,0.557792}%
\pgfsetfillcolor{currentfill}%
\pgfsetfillopacity{0.700000}%
\pgfsetlinewidth{0.000000pt}%
\definecolor{currentstroke}{rgb}{0.000000,0.000000,0.000000}%
\pgfsetstrokecolor{currentstroke}%
\pgfsetdash{}{0pt}%
\pgfpathmoveto{\pgfqpoint{5.689633in}{2.832115in}}%
\pgfpathlineto{\pgfqpoint{5.703478in}{2.837060in}}%
\pgfpathlineto{\pgfqpoint{5.717336in}{2.842117in}}%
\pgfpathlineto{\pgfqpoint{5.731209in}{2.847286in}}%
\pgfpathlineto{\pgfqpoint{5.745097in}{2.852565in}}%
\pgfpathlineto{\pgfqpoint{5.752131in}{2.860660in}}%
\pgfpathlineto{\pgfqpoint{5.759160in}{2.868726in}}%
\pgfpathlineto{\pgfqpoint{5.766182in}{2.876765in}}%
\pgfpathlineto{\pgfqpoint{5.773199in}{2.884778in}}%
\pgfpathlineto{\pgfqpoint{5.759326in}{2.879658in}}%
\pgfpathlineto{\pgfqpoint{5.745467in}{2.874648in}}%
\pgfpathlineto{\pgfqpoint{5.731622in}{2.869750in}}%
\pgfpathlineto{\pgfqpoint{5.717792in}{2.864963in}}%
\pgfpathlineto{\pgfqpoint{5.710760in}{2.856785in}}%
\pgfpathlineto{\pgfqpoint{5.703724in}{2.848585in}}%
\pgfpathlineto{\pgfqpoint{5.696681in}{2.840362in}}%
\pgfpathlineto{\pgfqpoint{5.689633in}{2.832115in}}%
\pgfpathclose%
\pgfusepath{fill}%
\end{pgfscope}%
\begin{pgfscope}%
\pgfpathrectangle{\pgfqpoint{1.254980in}{0.150000in}}{\pgfqpoint{5.490039in}{5.490039in}}%
\pgfusepath{clip}%
\pgfsetbuttcap%
\pgfsetroundjoin%
\definecolor{currentfill}{rgb}{0.283229,0.120777,0.440584}%
\pgfsetfillcolor{currentfill}%
\pgfsetfillopacity{0.700000}%
\pgfsetlinewidth{0.000000pt}%
\definecolor{currentstroke}{rgb}{0.000000,0.000000,0.000000}%
\pgfsetstrokecolor{currentstroke}%
\pgfsetdash{}{0pt}%
\pgfpathmoveto{\pgfqpoint{3.509311in}{2.142298in}}%
\pgfpathlineto{\pgfqpoint{3.522460in}{2.132671in}}%
\pgfpathlineto{\pgfqpoint{3.535611in}{2.123188in}}%
\pgfpathlineto{\pgfqpoint{3.548763in}{2.113848in}}%
\pgfpathlineto{\pgfqpoint{3.561918in}{2.104650in}}%
\pgfpathlineto{\pgfqpoint{3.569743in}{2.111579in}}%
\pgfpathlineto{\pgfqpoint{3.577562in}{2.118591in}}%
\pgfpathlineto{\pgfqpoint{3.585374in}{2.125684in}}%
\pgfpathlineto{\pgfqpoint{3.593179in}{2.132856in}}%
\pgfpathlineto{\pgfqpoint{3.580043in}{2.141838in}}%
\pgfpathlineto{\pgfqpoint{3.566909in}{2.150962in}}%
\pgfpathlineto{\pgfqpoint{3.553777in}{2.160229in}}%
\pgfpathlineto{\pgfqpoint{3.540646in}{2.169639in}}%
\pgfpathlineto{\pgfqpoint{3.532823in}{2.162677in}}%
\pgfpathlineto{\pgfqpoint{3.524992in}{2.155799in}}%
\pgfpathlineto{\pgfqpoint{3.517155in}{2.149005in}}%
\pgfpathlineto{\pgfqpoint{3.509311in}{2.142298in}}%
\pgfpathclose%
\pgfusepath{fill}%
\end{pgfscope}%
\begin{pgfscope}%
\pgfpathrectangle{\pgfqpoint{1.254980in}{0.150000in}}{\pgfqpoint{5.490039in}{5.490039in}}%
\pgfusepath{clip}%
\pgfsetbuttcap%
\pgfsetroundjoin%
\definecolor{currentfill}{rgb}{0.279566,0.067836,0.391917}%
\pgfsetfillcolor{currentfill}%
\pgfsetfillopacity{0.700000}%
\pgfsetlinewidth{0.000000pt}%
\definecolor{currentstroke}{rgb}{0.000000,0.000000,0.000000}%
\pgfsetstrokecolor{currentstroke}%
\pgfsetdash{}{0pt}%
\pgfpathmoveto{\pgfqpoint{3.834587in}{2.040642in}}%
\pgfpathlineto{\pgfqpoint{3.847756in}{2.034188in}}%
\pgfpathlineto{\pgfqpoint{3.860929in}{2.027868in}}%
\pgfpathlineto{\pgfqpoint{3.874107in}{2.021679in}}%
\pgfpathlineto{\pgfqpoint{3.887289in}{2.015622in}}%
\pgfpathlineto{\pgfqpoint{3.894984in}{2.024111in}}%
\pgfpathlineto{\pgfqpoint{3.902674in}{2.032650in}}%
\pgfpathlineto{\pgfqpoint{3.910358in}{2.041238in}}%
\pgfpathlineto{\pgfqpoint{3.918036in}{2.049875in}}%
\pgfpathlineto{\pgfqpoint{3.904868in}{2.055750in}}%
\pgfpathlineto{\pgfqpoint{3.891704in}{2.061757in}}%
\pgfpathlineto{\pgfqpoint{3.878545in}{2.067896in}}%
\pgfpathlineto{\pgfqpoint{3.865389in}{2.074167in}}%
\pgfpathlineto{\pgfqpoint{3.857697in}{2.065706in}}%
\pgfpathlineto{\pgfqpoint{3.850000in}{2.057298in}}%
\pgfpathlineto{\pgfqpoint{3.842296in}{2.048943in}}%
\pgfpathlineto{\pgfqpoint{3.834587in}{2.040642in}}%
\pgfpathclose%
\pgfusepath{fill}%
\end{pgfscope}%
\begin{pgfscope}%
\pgfpathrectangle{\pgfqpoint{1.254980in}{0.150000in}}{\pgfqpoint{5.490039in}{5.490039in}}%
\pgfusepath{clip}%
\pgfsetbuttcap%
\pgfsetroundjoin%
\definecolor{currentfill}{rgb}{0.169646,0.456262,0.558030}%
\pgfsetfillcolor{currentfill}%
\pgfsetfillopacity{0.700000}%
\pgfsetlinewidth{0.000000pt}%
\definecolor{currentstroke}{rgb}{0.000000,0.000000,0.000000}%
\pgfsetstrokecolor{currentstroke}%
\pgfsetdash{}{0pt}%
\pgfpathmoveto{\pgfqpoint{2.738185in}{2.921697in}}%
\pgfpathlineto{\pgfqpoint{2.751506in}{2.902474in}}%
\pgfpathlineto{\pgfqpoint{2.764820in}{2.883447in}}%
\pgfpathlineto{\pgfqpoint{2.778127in}{2.864616in}}%
\pgfpathlineto{\pgfqpoint{2.791427in}{2.845977in}}%
\pgfpathlineto{\pgfqpoint{2.799636in}{2.849118in}}%
\pgfpathlineto{\pgfqpoint{2.807833in}{2.852408in}}%
\pgfpathlineto{\pgfqpoint{2.816020in}{2.855844in}}%
\pgfpathlineto{\pgfqpoint{2.824195in}{2.859425in}}%
\pgfpathlineto{\pgfqpoint{2.810926in}{2.877814in}}%
\pgfpathlineto{\pgfqpoint{2.797650in}{2.896395in}}%
\pgfpathlineto{\pgfqpoint{2.784368in}{2.915171in}}%
\pgfpathlineto{\pgfqpoint{2.771080in}{2.934142in}}%
\pgfpathlineto{\pgfqpoint{2.762873in}{2.930805in}}%
\pgfpathlineto{\pgfqpoint{2.754655in}{2.927617in}}%
\pgfpathlineto{\pgfqpoint{2.746426in}{2.924580in}}%
\pgfpathlineto{\pgfqpoint{2.738185in}{2.921697in}}%
\pgfpathclose%
\pgfusepath{fill}%
\end{pgfscope}%
\begin{pgfscope}%
\pgfpathrectangle{\pgfqpoint{1.254980in}{0.150000in}}{\pgfqpoint{5.490039in}{5.490039in}}%
\pgfusepath{clip}%
\pgfsetbuttcap%
\pgfsetroundjoin%
\definecolor{currentfill}{rgb}{0.275191,0.194905,0.496005}%
\pgfsetfillcolor{currentfill}%
\pgfsetfillopacity{0.700000}%
\pgfsetlinewidth{0.000000pt}%
\definecolor{currentstroke}{rgb}{0.000000,0.000000,0.000000}%
\pgfsetstrokecolor{currentstroke}%
\pgfsetdash{}{0pt}%
\pgfpathmoveto{\pgfqpoint{3.267210in}{2.294214in}}%
\pgfpathlineto{\pgfqpoint{3.280378in}{2.281962in}}%
\pgfpathlineto{\pgfqpoint{3.293544in}{2.269866in}}%
\pgfpathlineto{\pgfqpoint{3.306710in}{2.257923in}}%
\pgfpathlineto{\pgfqpoint{3.319875in}{2.246134in}}%
\pgfpathlineto{\pgfqpoint{3.327813in}{2.251757in}}%
\pgfpathlineto{\pgfqpoint{3.335743in}{2.257486in}}%
\pgfpathlineto{\pgfqpoint{3.343665in}{2.263318in}}%
\pgfpathlineto{\pgfqpoint{3.351578in}{2.269253in}}%
\pgfpathlineto{\pgfqpoint{3.338436in}{2.280807in}}%
\pgfpathlineto{\pgfqpoint{3.325292in}{2.292513in}}%
\pgfpathlineto{\pgfqpoint{3.312148in}{2.304374in}}%
\pgfpathlineto{\pgfqpoint{3.299004in}{2.316389in}}%
\pgfpathlineto{\pgfqpoint{3.291068in}{2.310684in}}%
\pgfpathlineto{\pgfqpoint{3.283124in}{2.305085in}}%
\pgfpathlineto{\pgfqpoint{3.275171in}{2.299594in}}%
\pgfpathlineto{\pgfqpoint{3.267210in}{2.294214in}}%
\pgfpathclose%
\pgfusepath{fill}%
\end{pgfscope}%
\begin{pgfscope}%
\pgfpathrectangle{\pgfqpoint{1.254980in}{0.150000in}}{\pgfqpoint{5.490039in}{5.490039in}}%
\pgfusepath{clip}%
\pgfsetbuttcap%
\pgfsetroundjoin%
\definecolor{currentfill}{rgb}{0.280894,0.078907,0.402329}%
\pgfsetfillcolor{currentfill}%
\pgfsetfillopacity{0.700000}%
\pgfsetlinewidth{0.000000pt}%
\definecolor{currentstroke}{rgb}{0.000000,0.000000,0.000000}%
\pgfsetstrokecolor{currentstroke}%
\pgfsetdash{}{0pt}%
\pgfpathmoveto{\pgfqpoint{4.190270in}{2.049115in}}%
\pgfpathlineto{\pgfqpoint{4.203509in}{2.045726in}}%
\pgfpathlineto{\pgfqpoint{4.216754in}{2.042461in}}%
\pgfpathlineto{\pgfqpoint{4.230005in}{2.039321in}}%
\pgfpathlineto{\pgfqpoint{4.243264in}{2.036304in}}%
\pgfpathlineto{\pgfqpoint{4.250838in}{2.046017in}}%
\pgfpathlineto{\pgfqpoint{4.258408in}{2.055748in}}%
\pgfpathlineto{\pgfqpoint{4.265972in}{2.065495in}}%
\pgfpathlineto{\pgfqpoint{4.273532in}{2.075257in}}%
\pgfpathlineto{\pgfqpoint{4.260283in}{2.078140in}}%
\pgfpathlineto{\pgfqpoint{4.247041in}{2.081147in}}%
\pgfpathlineto{\pgfqpoint{4.233806in}{2.084278in}}%
\pgfpathlineto{\pgfqpoint{4.220578in}{2.087534in}}%
\pgfpathlineto{\pgfqpoint{4.213009in}{2.077899in}}%
\pgfpathlineto{\pgfqpoint{4.205434in}{2.068284in}}%
\pgfpathlineto{\pgfqpoint{4.197855in}{2.058689in}}%
\pgfpathlineto{\pgfqpoint{4.190270in}{2.049115in}}%
\pgfpathclose%
\pgfusepath{fill}%
\end{pgfscope}%
\begin{pgfscope}%
\pgfpathrectangle{\pgfqpoint{1.254980in}{0.150000in}}{\pgfqpoint{5.490039in}{5.490039in}}%
\pgfusepath{clip}%
\pgfsetbuttcap%
\pgfsetroundjoin%
\definecolor{currentfill}{rgb}{0.281446,0.084320,0.407414}%
\pgfsetfillcolor{currentfill}%
\pgfsetfillopacity{0.700000}%
\pgfsetlinewidth{0.000000pt}%
\definecolor{currentstroke}{rgb}{0.000000,0.000000,0.000000}%
\pgfsetstrokecolor{currentstroke}%
\pgfsetdash{}{0pt}%
\pgfpathmoveto{\pgfqpoint{3.698343in}{2.066047in}}%
\pgfpathlineto{\pgfqpoint{3.711501in}{2.058318in}}%
\pgfpathlineto{\pgfqpoint{3.724661in}{2.050726in}}%
\pgfpathlineto{\pgfqpoint{3.737824in}{2.043270in}}%
\pgfpathlineto{\pgfqpoint{3.750990in}{2.035949in}}%
\pgfpathlineto{\pgfqpoint{3.758739in}{2.043814in}}%
\pgfpathlineto{\pgfqpoint{3.766481in}{2.051743in}}%
\pgfpathlineto{\pgfqpoint{3.774217in}{2.059735in}}%
\pgfpathlineto{\pgfqpoint{3.781947in}{2.067788in}}%
\pgfpathlineto{\pgfqpoint{3.768796in}{2.074910in}}%
\pgfpathlineto{\pgfqpoint{3.755648in}{2.082168in}}%
\pgfpathlineto{\pgfqpoint{3.742504in}{2.089562in}}%
\pgfpathlineto{\pgfqpoint{3.729363in}{2.097092in}}%
\pgfpathlineto{\pgfqpoint{3.721617in}{2.089231in}}%
\pgfpathlineto{\pgfqpoint{3.713866in}{2.081436in}}%
\pgfpathlineto{\pgfqpoint{3.706108in}{2.073707in}}%
\pgfpathlineto{\pgfqpoint{3.698343in}{2.066047in}}%
\pgfpathclose%
\pgfusepath{fill}%
\end{pgfscope}%
\begin{pgfscope}%
\pgfpathrectangle{\pgfqpoint{1.254980in}{0.150000in}}{\pgfqpoint{5.490039in}{5.490039in}}%
\pgfusepath{clip}%
\pgfsetbuttcap%
\pgfsetroundjoin%
\definecolor{currentfill}{rgb}{0.166617,0.463708,0.558119}%
\pgfsetfillcolor{currentfill}%
\pgfsetfillopacity{0.700000}%
\pgfsetlinewidth{0.000000pt}%
\definecolor{currentstroke}{rgb}{0.000000,0.000000,0.000000}%
\pgfsetstrokecolor{currentstroke}%
\pgfsetdash{}{0pt}%
\pgfpathmoveto{\pgfqpoint{5.773199in}{2.884778in}}%
\pgfpathlineto{\pgfqpoint{5.787088in}{2.890009in}}%
\pgfpathlineto{\pgfqpoint{5.800990in}{2.895351in}}%
\pgfpathlineto{\pgfqpoint{5.814908in}{2.900804in}}%
\pgfpathlineto{\pgfqpoint{5.828841in}{2.906368in}}%
\pgfpathlineto{\pgfqpoint{5.835837in}{2.914188in}}%
\pgfpathlineto{\pgfqpoint{5.842828in}{2.921982in}}%
\pgfpathlineto{\pgfqpoint{5.849814in}{2.929752in}}%
\pgfpathlineto{\pgfqpoint{5.856793in}{2.937499in}}%
\pgfpathlineto{\pgfqpoint{5.842876in}{2.932111in}}%
\pgfpathlineto{\pgfqpoint{5.828973in}{2.926833in}}%
\pgfpathlineto{\pgfqpoint{5.815085in}{2.921667in}}%
\pgfpathlineto{\pgfqpoint{5.801212in}{2.916611in}}%
\pgfpathlineto{\pgfqpoint{5.794217in}{2.908682in}}%
\pgfpathlineto{\pgfqpoint{5.787217in}{2.900735in}}%
\pgfpathlineto{\pgfqpoint{5.780211in}{2.892767in}}%
\pgfpathlineto{\pgfqpoint{5.773199in}{2.884778in}}%
\pgfpathclose%
\pgfusepath{fill}%
\end{pgfscope}%
\begin{pgfscope}%
\pgfpathrectangle{\pgfqpoint{1.254980in}{0.150000in}}{\pgfqpoint{5.490039in}{5.490039in}}%
\pgfusepath{clip}%
\pgfsetbuttcap%
\pgfsetroundjoin%
\definecolor{currentfill}{rgb}{0.279566,0.067836,0.391917}%
\pgfsetfillcolor{currentfill}%
\pgfsetfillopacity{0.700000}%
\pgfsetlinewidth{0.000000pt}%
\definecolor{currentstroke}{rgb}{0.000000,0.000000,0.000000}%
\pgfsetstrokecolor{currentstroke}%
\pgfsetdash{}{0pt}%
\pgfpathmoveto{\pgfqpoint{3.970755in}{2.027679in}}%
\pgfpathlineto{\pgfqpoint{3.983947in}{2.022454in}}%
\pgfpathlineto{\pgfqpoint{3.997144in}{2.017359in}}%
\pgfpathlineto{\pgfqpoint{4.010346in}{2.012391in}}%
\pgfpathlineto{\pgfqpoint{4.023554in}{2.007552in}}%
\pgfpathlineto{\pgfqpoint{4.031201in}{2.016578in}}%
\pgfpathlineto{\pgfqpoint{4.038844in}{2.025642in}}%
\pgfpathlineto{\pgfqpoint{4.046481in}{2.034741in}}%
\pgfpathlineto{\pgfqpoint{4.054113in}{2.043876in}}%
\pgfpathlineto{\pgfqpoint{4.040917in}{2.048550in}}%
\pgfpathlineto{\pgfqpoint{4.027727in}{2.053352in}}%
\pgfpathlineto{\pgfqpoint{4.014543in}{2.058282in}}%
\pgfpathlineto{\pgfqpoint{4.001363in}{2.063341in}}%
\pgfpathlineto{\pgfqpoint{3.993719in}{2.054366in}}%
\pgfpathlineto{\pgfqpoint{3.986070in}{2.045430in}}%
\pgfpathlineto{\pgfqpoint{3.978415in}{2.036534in}}%
\pgfpathlineto{\pgfqpoint{3.970755in}{2.027679in}}%
\pgfpathclose%
\pgfusepath{fill}%
\end{pgfscope}%
\begin{pgfscope}%
\pgfpathrectangle{\pgfqpoint{1.254980in}{0.150000in}}{\pgfqpoint{5.490039in}{5.490039in}}%
\pgfusepath{clip}%
\pgfsetbuttcap%
\pgfsetroundjoin%
\definecolor{currentfill}{rgb}{0.276194,0.190074,0.493001}%
\pgfsetfillcolor{currentfill}%
\pgfsetfillopacity{0.700000}%
\pgfsetlinewidth{0.000000pt}%
\definecolor{currentstroke}{rgb}{0.000000,0.000000,0.000000}%
\pgfsetstrokecolor{currentstroke}%
\pgfsetdash{}{0pt}%
\pgfpathmoveto{\pgfqpoint{4.743032in}{2.250215in}}%
\pgfpathlineto{\pgfqpoint{4.756451in}{2.250764in}}%
\pgfpathlineto{\pgfqpoint{4.769880in}{2.251431in}}%
\pgfpathlineto{\pgfqpoint{4.783319in}{2.252215in}}%
\pgfpathlineto{\pgfqpoint{4.796769in}{2.253115in}}%
\pgfpathlineto{\pgfqpoint{4.804170in}{2.263376in}}%
\pgfpathlineto{\pgfqpoint{4.811566in}{2.273614in}}%
\pgfpathlineto{\pgfqpoint{4.818958in}{2.283830in}}%
\pgfpathlineto{\pgfqpoint{4.826344in}{2.294023in}}%
\pgfpathlineto{\pgfqpoint{4.812902in}{2.293085in}}%
\pgfpathlineto{\pgfqpoint{4.799470in}{2.292263in}}%
\pgfpathlineto{\pgfqpoint{4.786049in}{2.291558in}}%
\pgfpathlineto{\pgfqpoint{4.772637in}{2.290970in}}%
\pgfpathlineto{\pgfqpoint{4.765243in}{2.280809in}}%
\pgfpathlineto{\pgfqpoint{4.757844in}{2.270629in}}%
\pgfpathlineto{\pgfqpoint{4.750441in}{2.260431in}}%
\pgfpathlineto{\pgfqpoint{4.743032in}{2.250215in}}%
\pgfpathclose%
\pgfusepath{fill}%
\end{pgfscope}%
\begin{pgfscope}%
\pgfpathrectangle{\pgfqpoint{1.254980in}{0.150000in}}{\pgfqpoint{5.490039in}{5.490039in}}%
\pgfusepath{clip}%
\pgfsetbuttcap%
\pgfsetroundjoin%
\definecolor{currentfill}{rgb}{0.270595,0.214069,0.507052}%
\pgfsetfillcolor{currentfill}%
\pgfsetfillopacity{0.700000}%
\pgfsetlinewidth{0.000000pt}%
\definecolor{currentstroke}{rgb}{0.000000,0.000000,0.000000}%
\pgfsetstrokecolor{currentstroke}%
\pgfsetdash{}{0pt}%
\pgfpathmoveto{\pgfqpoint{4.826344in}{2.294023in}}%
\pgfpathlineto{\pgfqpoint{4.839796in}{2.295079in}}%
\pgfpathlineto{\pgfqpoint{4.853259in}{2.296250in}}%
\pgfpathlineto{\pgfqpoint{4.866733in}{2.297538in}}%
\pgfpathlineto{\pgfqpoint{4.880216in}{2.298942in}}%
\pgfpathlineto{\pgfqpoint{4.887591in}{2.309141in}}%
\pgfpathlineto{\pgfqpoint{4.894960in}{2.319314in}}%
\pgfpathlineto{\pgfqpoint{4.902324in}{2.329461in}}%
\pgfpathlineto{\pgfqpoint{4.909683in}{2.339581in}}%
\pgfpathlineto{\pgfqpoint{4.896206in}{2.338155in}}%
\pgfpathlineto{\pgfqpoint{4.882740in}{2.336845in}}%
\pgfpathlineto{\pgfqpoint{4.869285in}{2.335652in}}%
\pgfpathlineto{\pgfqpoint{4.855840in}{2.334574in}}%
\pgfpathlineto{\pgfqpoint{4.848474in}{2.324470in}}%
\pgfpathlineto{\pgfqpoint{4.841102in}{2.314343in}}%
\pgfpathlineto{\pgfqpoint{4.833726in}{2.304194in}}%
\pgfpathlineto{\pgfqpoint{4.826344in}{2.294023in}}%
\pgfpathclose%
\pgfusepath{fill}%
\end{pgfscope}%
\begin{pgfscope}%
\pgfpathrectangle{\pgfqpoint{1.254980in}{0.150000in}}{\pgfqpoint{5.490039in}{5.490039in}}%
\pgfusepath{clip}%
\pgfsetbuttcap%
\pgfsetroundjoin%
\definecolor{currentfill}{rgb}{0.265145,0.232956,0.516599}%
\pgfsetfillcolor{currentfill}%
\pgfsetfillopacity{0.700000}%
\pgfsetlinewidth{0.000000pt}%
\definecolor{currentstroke}{rgb}{0.000000,0.000000,0.000000}%
\pgfsetstrokecolor{currentstroke}%
\pgfsetdash{}{0pt}%
\pgfpathmoveto{\pgfqpoint{4.909683in}{2.339581in}}%
\pgfpathlineto{\pgfqpoint{4.923170in}{2.341123in}}%
\pgfpathlineto{\pgfqpoint{4.936668in}{2.342781in}}%
\pgfpathlineto{\pgfqpoint{4.950177in}{2.344554in}}%
\pgfpathlineto{\pgfqpoint{4.963697in}{2.346443in}}%
\pgfpathlineto{\pgfqpoint{4.971044in}{2.356549in}}%
\pgfpathlineto{\pgfqpoint{4.978385in}{2.366626in}}%
\pgfpathlineto{\pgfqpoint{4.985721in}{2.376673in}}%
\pgfpathlineto{\pgfqpoint{4.993052in}{2.386692in}}%
\pgfpathlineto{\pgfqpoint{4.979539in}{2.384797in}}%
\pgfpathlineto{\pgfqpoint{4.966038in}{2.383018in}}%
\pgfpathlineto{\pgfqpoint{4.952547in}{2.381354in}}%
\pgfpathlineto{\pgfqpoint{4.939068in}{2.379806in}}%
\pgfpathlineto{\pgfqpoint{4.931729in}{2.369788in}}%
\pgfpathlineto{\pgfqpoint{4.924385in}{2.359745in}}%
\pgfpathlineto{\pgfqpoint{4.917036in}{2.349676in}}%
\pgfpathlineto{\pgfqpoint{4.909683in}{2.339581in}}%
\pgfpathclose%
\pgfusepath{fill}%
\end{pgfscope}%
\begin{pgfscope}%
\pgfpathrectangle{\pgfqpoint{1.254980in}{0.150000in}}{\pgfqpoint{5.490039in}{5.490039in}}%
\pgfusepath{clip}%
\pgfsetbuttcap%
\pgfsetroundjoin%
\definecolor{currentfill}{rgb}{0.279574,0.170599,0.479997}%
\pgfsetfillcolor{currentfill}%
\pgfsetfillopacity{0.700000}%
\pgfsetlinewidth{0.000000pt}%
\definecolor{currentstroke}{rgb}{0.000000,0.000000,0.000000}%
\pgfsetstrokecolor{currentstroke}%
\pgfsetdash{}{0pt}%
\pgfpathmoveto{\pgfqpoint{4.659739in}{2.208361in}}%
\pgfpathlineto{\pgfqpoint{4.673127in}{2.208386in}}%
\pgfpathlineto{\pgfqpoint{4.686525in}{2.208529in}}%
\pgfpathlineto{\pgfqpoint{4.699932in}{2.208790in}}%
\pgfpathlineto{\pgfqpoint{4.713348in}{2.209168in}}%
\pgfpathlineto{\pgfqpoint{4.720776in}{2.219456in}}%
\pgfpathlineto{\pgfqpoint{4.728200in}{2.229727in}}%
\pgfpathlineto{\pgfqpoint{4.735618in}{2.239980in}}%
\pgfpathlineto{\pgfqpoint{4.743032in}{2.250215in}}%
\pgfpathlineto{\pgfqpoint{4.729623in}{2.249782in}}%
\pgfpathlineto{\pgfqpoint{4.716223in}{2.249468in}}%
\pgfpathlineto{\pgfqpoint{4.702833in}{2.249271in}}%
\pgfpathlineto{\pgfqpoint{4.689453in}{2.249191in}}%
\pgfpathlineto{\pgfqpoint{4.682032in}{2.239005in}}%
\pgfpathlineto{\pgfqpoint{4.674606in}{2.228804in}}%
\pgfpathlineto{\pgfqpoint{4.667175in}{2.218590in}}%
\pgfpathlineto{\pgfqpoint{4.659739in}{2.208361in}}%
\pgfpathclose%
\pgfusepath{fill}%
\end{pgfscope}%
\begin{pgfscope}%
\pgfpathrectangle{\pgfqpoint{1.254980in}{0.150000in}}{\pgfqpoint{5.490039in}{5.490039in}}%
\pgfusepath{clip}%
\pgfsetbuttcap%
\pgfsetroundjoin%
\definecolor{currentfill}{rgb}{0.257322,0.256130,0.526563}%
\pgfsetfillcolor{currentfill}%
\pgfsetfillopacity{0.700000}%
\pgfsetlinewidth{0.000000pt}%
\definecolor{currentstroke}{rgb}{0.000000,0.000000,0.000000}%
\pgfsetstrokecolor{currentstroke}%
\pgfsetdash{}{0pt}%
\pgfpathmoveto{\pgfqpoint{4.993052in}{2.386692in}}%
\pgfpathlineto{\pgfqpoint{5.006576in}{2.388701in}}%
\pgfpathlineto{\pgfqpoint{5.020111in}{2.390826in}}%
\pgfpathlineto{\pgfqpoint{5.033657in}{2.393066in}}%
\pgfpathlineto{\pgfqpoint{5.047215in}{2.395421in}}%
\pgfpathlineto{\pgfqpoint{5.054533in}{2.405406in}}%
\pgfpathlineto{\pgfqpoint{5.061846in}{2.415358in}}%
\pgfpathlineto{\pgfqpoint{5.069154in}{2.425278in}}%
\pgfpathlineto{\pgfqpoint{5.076456in}{2.435167in}}%
\pgfpathlineto{\pgfqpoint{5.062906in}{2.432822in}}%
\pgfpathlineto{\pgfqpoint{5.049367in}{2.430593in}}%
\pgfpathlineto{\pgfqpoint{5.035840in}{2.428478in}}%
\pgfpathlineto{\pgfqpoint{5.022324in}{2.426479in}}%
\pgfpathlineto{\pgfqpoint{5.015014in}{2.416574in}}%
\pgfpathlineto{\pgfqpoint{5.007698in}{2.406642in}}%
\pgfpathlineto{\pgfqpoint{5.000378in}{2.396681in}}%
\pgfpathlineto{\pgfqpoint{4.993052in}{2.386692in}}%
\pgfpathclose%
\pgfusepath{fill}%
\end{pgfscope}%
\begin{pgfscope}%
\pgfpathrectangle{\pgfqpoint{1.254980in}{0.150000in}}{\pgfqpoint{5.490039in}{5.490039in}}%
\pgfusepath{clip}%
\pgfsetbuttcap%
\pgfsetroundjoin%
\definecolor{currentfill}{rgb}{0.281887,0.150881,0.465405}%
\pgfsetfillcolor{currentfill}%
\pgfsetfillopacity{0.700000}%
\pgfsetlinewidth{0.000000pt}%
\definecolor{currentstroke}{rgb}{0.000000,0.000000,0.000000}%
\pgfsetstrokecolor{currentstroke}%
\pgfsetdash{}{0pt}%
\pgfpathmoveto{\pgfqpoint{4.576459in}{2.168679in}}%
\pgfpathlineto{\pgfqpoint{4.589818in}{2.168160in}}%
\pgfpathlineto{\pgfqpoint{4.603186in}{2.167759in}}%
\pgfpathlineto{\pgfqpoint{4.616562in}{2.167478in}}%
\pgfpathlineto{\pgfqpoint{4.629948in}{2.167314in}}%
\pgfpathlineto{\pgfqpoint{4.637403in}{2.177595in}}%
\pgfpathlineto{\pgfqpoint{4.644854in}{2.187864in}}%
\pgfpathlineto{\pgfqpoint{4.652299in}{2.198119in}}%
\pgfpathlineto{\pgfqpoint{4.659739in}{2.208361in}}%
\pgfpathlineto{\pgfqpoint{4.646361in}{2.208454in}}%
\pgfpathlineto{\pgfqpoint{4.632992in}{2.208666in}}%
\pgfpathlineto{\pgfqpoint{4.619632in}{2.208996in}}%
\pgfpathlineto{\pgfqpoint{4.606281in}{2.209445in}}%
\pgfpathlineto{\pgfqpoint{4.598833in}{2.199267in}}%
\pgfpathlineto{\pgfqpoint{4.591380in}{2.189080in}}%
\pgfpathlineto{\pgfqpoint{4.583922in}{2.178884in}}%
\pgfpathlineto{\pgfqpoint{4.576459in}{2.168679in}}%
\pgfpathclose%
\pgfusepath{fill}%
\end{pgfscope}%
\begin{pgfscope}%
\pgfpathrectangle{\pgfqpoint{1.254980in}{0.150000in}}{\pgfqpoint{5.490039in}{5.490039in}}%
\pgfusepath{clip}%
\pgfsetbuttcap%
\pgfsetroundjoin%
\definecolor{currentfill}{rgb}{0.157729,0.485932,0.558013}%
\pgfsetfillcolor{currentfill}%
\pgfsetfillopacity{0.700000}%
\pgfsetlinewidth{0.000000pt}%
\definecolor{currentstroke}{rgb}{0.000000,0.000000,0.000000}%
\pgfsetstrokecolor{currentstroke}%
\pgfsetdash{}{0pt}%
\pgfpathmoveto{\pgfqpoint{5.856793in}{2.937499in}}%
\pgfpathlineto{\pgfqpoint{5.870726in}{2.942997in}}%
\pgfpathlineto{\pgfqpoint{5.884673in}{2.948606in}}%
\pgfpathlineto{\pgfqpoint{5.898635in}{2.954326in}}%
\pgfpathlineto{\pgfqpoint{5.912613in}{2.960156in}}%
\pgfpathlineto{\pgfqpoint{5.919571in}{2.967696in}}%
\pgfpathlineto{\pgfqpoint{5.926524in}{2.975214in}}%
\pgfpathlineto{\pgfqpoint{5.933470in}{2.982711in}}%
\pgfpathlineto{\pgfqpoint{5.940412in}{2.990189in}}%
\pgfpathlineto{\pgfqpoint{5.926450in}{2.984552in}}%
\pgfpathlineto{\pgfqpoint{5.912504in}{2.979025in}}%
\pgfpathlineto{\pgfqpoint{5.898573in}{2.973608in}}%
\pgfpathlineto{\pgfqpoint{5.884656in}{2.968301in}}%
\pgfpathlineto{\pgfqpoint{5.877699in}{2.960625in}}%
\pgfpathlineto{\pgfqpoint{5.870736in}{2.952933in}}%
\pgfpathlineto{\pgfqpoint{5.863767in}{2.945225in}}%
\pgfpathlineto{\pgfqpoint{5.856793in}{2.937499in}}%
\pgfpathclose%
\pgfusepath{fill}%
\end{pgfscope}%
\begin{pgfscope}%
\pgfpathrectangle{\pgfqpoint{1.254980in}{0.150000in}}{\pgfqpoint{5.490039in}{5.490039in}}%
\pgfusepath{clip}%
\pgfsetbuttcap%
\pgfsetroundjoin%
\definecolor{currentfill}{rgb}{0.246811,0.283237,0.535941}%
\pgfsetfillcolor{currentfill}%
\pgfsetfillopacity{0.700000}%
\pgfsetlinewidth{0.000000pt}%
\definecolor{currentstroke}{rgb}{0.000000,0.000000,0.000000}%
\pgfsetstrokecolor{currentstroke}%
\pgfsetdash{}{0pt}%
\pgfpathmoveto{\pgfqpoint{5.076456in}{2.435167in}}%
\pgfpathlineto{\pgfqpoint{5.090018in}{2.437626in}}%
\pgfpathlineto{\pgfqpoint{5.103591in}{2.440199in}}%
\pgfpathlineto{\pgfqpoint{5.117177in}{2.442887in}}%
\pgfpathlineto{\pgfqpoint{5.130774in}{2.445689in}}%
\pgfpathlineto{\pgfqpoint{5.138063in}{2.455526in}}%
\pgfpathlineto{\pgfqpoint{5.145347in}{2.465328in}}%
\pgfpathlineto{\pgfqpoint{5.152625in}{2.475095in}}%
\pgfpathlineto{\pgfqpoint{5.159898in}{2.484828in}}%
\pgfpathlineto{\pgfqpoint{5.146309in}{2.482053in}}%
\pgfpathlineto{\pgfqpoint{5.132732in}{2.479391in}}%
\pgfpathlineto{\pgfqpoint{5.119167in}{2.476844in}}%
\pgfpathlineto{\pgfqpoint{5.105613in}{2.474411in}}%
\pgfpathlineto{\pgfqpoint{5.098332in}{2.464646in}}%
\pgfpathlineto{\pgfqpoint{5.091045in}{2.454850in}}%
\pgfpathlineto{\pgfqpoint{5.083753in}{2.445024in}}%
\pgfpathlineto{\pgfqpoint{5.076456in}{2.435167in}}%
\pgfpathclose%
\pgfusepath{fill}%
\end{pgfscope}%
\begin{pgfscope}%
\pgfpathrectangle{\pgfqpoint{1.254980in}{0.150000in}}{\pgfqpoint{5.490039in}{5.490039in}}%
\pgfusepath{clip}%
\pgfsetbuttcap%
\pgfsetroundjoin%
\definecolor{currentfill}{rgb}{0.156270,0.489624,0.557936}%
\pgfsetfillcolor{currentfill}%
\pgfsetfillopacity{0.700000}%
\pgfsetlinewidth{0.000000pt}%
\definecolor{currentstroke}{rgb}{0.000000,0.000000,0.000000}%
\pgfsetstrokecolor{currentstroke}%
\pgfsetdash{}{0pt}%
\pgfpathmoveto{\pgfqpoint{2.684828in}{3.000575in}}%
\pgfpathlineto{\pgfqpoint{2.698178in}{2.980554in}}%
\pgfpathlineto{\pgfqpoint{2.711521in}{2.960735in}}%
\pgfpathlineto{\pgfqpoint{2.724857in}{2.941117in}}%
\pgfpathlineto{\pgfqpoint{2.738185in}{2.921697in}}%
\pgfpathlineto{\pgfqpoint{2.746426in}{2.924580in}}%
\pgfpathlineto{\pgfqpoint{2.754655in}{2.927617in}}%
\pgfpathlineto{\pgfqpoint{2.762873in}{2.930805in}}%
\pgfpathlineto{\pgfqpoint{2.771080in}{2.934142in}}%
\pgfpathlineto{\pgfqpoint{2.757784in}{2.953310in}}%
\pgfpathlineto{\pgfqpoint{2.744481in}{2.972677in}}%
\pgfpathlineto{\pgfqpoint{2.731171in}{2.992242in}}%
\pgfpathlineto{\pgfqpoint{2.717854in}{3.012009in}}%
\pgfpathlineto{\pgfqpoint{2.709615in}{3.008918in}}%
\pgfpathlineto{\pgfqpoint{2.701364in}{3.005981in}}%
\pgfpathlineto{\pgfqpoint{2.693102in}{3.003199in}}%
\pgfpathlineto{\pgfqpoint{2.684828in}{3.000575in}}%
\pgfpathclose%
\pgfusepath{fill}%
\end{pgfscope}%
\begin{pgfscope}%
\pgfpathrectangle{\pgfqpoint{1.254980in}{0.150000in}}{\pgfqpoint{5.490039in}{5.490039in}}%
\pgfusepath{clip}%
\pgfsetbuttcap%
\pgfsetroundjoin%
\definecolor{currentfill}{rgb}{0.278826,0.175490,0.483397}%
\pgfsetfillcolor{currentfill}%
\pgfsetfillopacity{0.700000}%
\pgfsetlinewidth{0.000000pt}%
\definecolor{currentstroke}{rgb}{0.000000,0.000000,0.000000}%
\pgfsetstrokecolor{currentstroke}%
\pgfsetdash{}{0pt}%
\pgfpathmoveto{\pgfqpoint{3.319875in}{2.246134in}}%
\pgfpathlineto{\pgfqpoint{3.333040in}{2.234498in}}%
\pgfpathlineto{\pgfqpoint{3.346205in}{2.223014in}}%
\pgfpathlineto{\pgfqpoint{3.359369in}{2.211680in}}%
\pgfpathlineto{\pgfqpoint{3.372533in}{2.200498in}}%
\pgfpathlineto{\pgfqpoint{3.380449in}{2.206362in}}%
\pgfpathlineto{\pgfqpoint{3.388357in}{2.212327in}}%
\pgfpathlineto{\pgfqpoint{3.396257in}{2.218393in}}%
\pgfpathlineto{\pgfqpoint{3.404149in}{2.224557in}}%
\pgfpathlineto{\pgfqpoint{3.391006in}{2.235505in}}%
\pgfpathlineto{\pgfqpoint{3.377864in}{2.246603in}}%
\pgfpathlineto{\pgfqpoint{3.364721in}{2.257852in}}%
\pgfpathlineto{\pgfqpoint{3.351578in}{2.269253in}}%
\pgfpathlineto{\pgfqpoint{3.343665in}{2.263318in}}%
\pgfpathlineto{\pgfqpoint{3.335743in}{2.257486in}}%
\pgfpathlineto{\pgfqpoint{3.327813in}{2.251757in}}%
\pgfpathlineto{\pgfqpoint{3.319875in}{2.246134in}}%
\pgfpathclose%
\pgfusepath{fill}%
\end{pgfscope}%
\begin{pgfscope}%
\pgfpathrectangle{\pgfqpoint{1.254980in}{0.150000in}}{\pgfqpoint{5.490039in}{5.490039in}}%
\pgfusepath{clip}%
\pgfsetbuttcap%
\pgfsetroundjoin%
\definecolor{currentfill}{rgb}{0.141935,0.526453,0.555991}%
\pgfsetfillcolor{currentfill}%
\pgfsetfillopacity{0.700000}%
\pgfsetlinewidth{0.000000pt}%
\definecolor{currentstroke}{rgb}{0.000000,0.000000,0.000000}%
\pgfsetstrokecolor{currentstroke}%
\pgfsetdash{}{0pt}%
\pgfpathmoveto{\pgfqpoint{6.024051in}{3.042769in}}%
\pgfpathlineto{\pgfqpoint{6.038072in}{3.048747in}}%
\pgfpathlineto{\pgfqpoint{6.052108in}{3.054835in}}%
\pgfpathlineto{\pgfqpoint{6.066161in}{3.061033in}}%
\pgfpathlineto{\pgfqpoint{6.073044in}{3.068064in}}%
\pgfpathlineto{\pgfqpoint{6.079921in}{3.075083in}}%
\pgfpathlineto{\pgfqpoint{6.086793in}{3.082090in}}%
\pgfpathlineto{\pgfqpoint{6.093660in}{3.089089in}}%
\pgfpathlineto{\pgfqpoint{6.079626in}{3.083118in}}%
\pgfpathlineto{\pgfqpoint{6.065608in}{3.077256in}}%
\pgfpathlineto{\pgfqpoint{6.051606in}{3.071504in}}%
\pgfpathlineto{\pgfqpoint{6.044725in}{3.064331in}}%
\pgfpathlineto{\pgfqpoint{6.037839in}{3.057152in}}%
\pgfpathlineto{\pgfqpoint{6.030948in}{3.049965in}}%
\pgfpathlineto{\pgfqpoint{6.024051in}{3.042769in}}%
\pgfpathclose%
\pgfusepath{fill}%
\end{pgfscope}%
\begin{pgfscope}%
\pgfpathrectangle{\pgfqpoint{1.254980in}{0.150000in}}{\pgfqpoint{5.490039in}{5.490039in}}%
\pgfusepath{clip}%
\pgfsetbuttcap%
\pgfsetroundjoin%
\definecolor{currentfill}{rgb}{0.283072,0.130895,0.449241}%
\pgfsetfillcolor{currentfill}%
\pgfsetfillopacity{0.700000}%
\pgfsetlinewidth{0.000000pt}%
\definecolor{currentstroke}{rgb}{0.000000,0.000000,0.000000}%
\pgfsetstrokecolor{currentstroke}%
\pgfsetdash{}{0pt}%
\pgfpathmoveto{\pgfqpoint{4.493183in}{2.131392in}}%
\pgfpathlineto{\pgfqpoint{4.506514in}{2.130309in}}%
\pgfpathlineto{\pgfqpoint{4.519854in}{2.129346in}}%
\pgfpathlineto{\pgfqpoint{4.533203in}{2.128503in}}%
\pgfpathlineto{\pgfqpoint{4.546561in}{2.127779in}}%
\pgfpathlineto{\pgfqpoint{4.554043in}{2.138015in}}%
\pgfpathlineto{\pgfqpoint{4.561520in}{2.148244in}}%
\pgfpathlineto{\pgfqpoint{4.568992in}{2.158465in}}%
\pgfpathlineto{\pgfqpoint{4.576459in}{2.168679in}}%
\pgfpathlineto{\pgfqpoint{4.563110in}{2.169317in}}%
\pgfpathlineto{\pgfqpoint{4.549769in}{2.170074in}}%
\pgfpathlineto{\pgfqpoint{4.536437in}{2.170951in}}%
\pgfpathlineto{\pgfqpoint{4.523113in}{2.171947in}}%
\pgfpathlineto{\pgfqpoint{4.515638in}{2.161814in}}%
\pgfpathlineto{\pgfqpoint{4.508158in}{2.151677in}}%
\pgfpathlineto{\pgfqpoint{4.500673in}{2.141536in}}%
\pgfpathlineto{\pgfqpoint{4.493183in}{2.131392in}}%
\pgfpathclose%
\pgfusepath{fill}%
\end{pgfscope}%
\begin{pgfscope}%
\pgfpathrectangle{\pgfqpoint{1.254980in}{0.150000in}}{\pgfqpoint{5.490039in}{5.490039in}}%
\pgfusepath{clip}%
\pgfsetbuttcap%
\pgfsetroundjoin%
\definecolor{currentfill}{rgb}{0.149039,0.508051,0.557250}%
\pgfsetfillcolor{currentfill}%
\pgfsetfillopacity{0.700000}%
\pgfsetlinewidth{0.000000pt}%
\definecolor{currentstroke}{rgb}{0.000000,0.000000,0.000000}%
\pgfsetstrokecolor{currentstroke}%
\pgfsetdash{}{0pt}%
\pgfpathmoveto{\pgfqpoint{5.940412in}{2.990189in}}%
\pgfpathlineto{\pgfqpoint{5.954388in}{2.995936in}}%
\pgfpathlineto{\pgfqpoint{5.968380in}{3.001794in}}%
\pgfpathlineto{\pgfqpoint{5.982387in}{3.007762in}}%
\pgfpathlineto{\pgfqpoint{5.996410in}{3.013840in}}%
\pgfpathlineto{\pgfqpoint{6.003329in}{3.021098in}}%
\pgfpathlineto{\pgfqpoint{6.010242in}{3.028337in}}%
\pgfpathlineto{\pgfqpoint{6.017149in}{3.035560in}}%
\pgfpathlineto{\pgfqpoint{6.024051in}{3.042769in}}%
\pgfpathlineto{\pgfqpoint{6.010046in}{3.036901in}}%
\pgfpathlineto{\pgfqpoint{5.996056in}{3.031142in}}%
\pgfpathlineto{\pgfqpoint{5.982081in}{3.025494in}}%
\pgfpathlineto{\pgfqpoint{5.968122in}{3.019956in}}%
\pgfpathlineto{\pgfqpoint{5.961202in}{3.012532in}}%
\pgfpathlineto{\pgfqpoint{5.954277in}{3.005097in}}%
\pgfpathlineto{\pgfqpoint{5.947347in}{2.997650in}}%
\pgfpathlineto{\pgfqpoint{5.940412in}{2.990189in}}%
\pgfpathclose%
\pgfusepath{fill}%
\end{pgfscope}%
\begin{pgfscope}%
\pgfpathrectangle{\pgfqpoint{1.254980in}{0.150000in}}{\pgfqpoint{5.490039in}{5.490039in}}%
\pgfusepath{clip}%
\pgfsetbuttcap%
\pgfsetroundjoin%
\definecolor{currentfill}{rgb}{0.237441,0.305202,0.541921}%
\pgfsetfillcolor{currentfill}%
\pgfsetfillopacity{0.700000}%
\pgfsetlinewidth{0.000000pt}%
\definecolor{currentstroke}{rgb}{0.000000,0.000000,0.000000}%
\pgfsetstrokecolor{currentstroke}%
\pgfsetdash{}{0pt}%
\pgfpathmoveto{\pgfqpoint{5.159898in}{2.484828in}}%
\pgfpathlineto{\pgfqpoint{5.173499in}{2.487718in}}%
\pgfpathlineto{\pgfqpoint{5.187112in}{2.490721in}}%
\pgfpathlineto{\pgfqpoint{5.200738in}{2.493839in}}%
\pgfpathlineto{\pgfqpoint{5.214376in}{2.497070in}}%
\pgfpathlineto{\pgfqpoint{5.221635in}{2.506734in}}%
\pgfpathlineto{\pgfqpoint{5.228888in}{2.516360in}}%
\pgfpathlineto{\pgfqpoint{5.236137in}{2.525951in}}%
\pgfpathlineto{\pgfqpoint{5.243379in}{2.535506in}}%
\pgfpathlineto{\pgfqpoint{5.229750in}{2.532318in}}%
\pgfpathlineto{\pgfqpoint{5.216133in}{2.529243in}}%
\pgfpathlineto{\pgfqpoint{5.202529in}{2.526282in}}%
\pgfpathlineto{\pgfqpoint{5.188937in}{2.523435in}}%
\pgfpathlineto{\pgfqpoint{5.181685in}{2.513831in}}%
\pgfpathlineto{\pgfqpoint{5.174428in}{2.504196in}}%
\pgfpathlineto{\pgfqpoint{5.167166in}{2.494528in}}%
\pgfpathlineto{\pgfqpoint{5.159898in}{2.484828in}}%
\pgfpathclose%
\pgfusepath{fill}%
\end{pgfscope}%
\begin{pgfscope}%
\pgfpathrectangle{\pgfqpoint{1.254980in}{0.150000in}}{\pgfqpoint{5.490039in}{5.490039in}}%
\pgfusepath{clip}%
\pgfsetbuttcap%
\pgfsetroundjoin%
\definecolor{currentfill}{rgb}{0.282910,0.105393,0.426902}%
\pgfsetfillcolor{currentfill}%
\pgfsetfillopacity{0.700000}%
\pgfsetlinewidth{0.000000pt}%
\definecolor{currentstroke}{rgb}{0.000000,0.000000,0.000000}%
\pgfsetstrokecolor{currentstroke}%
\pgfsetdash{}{0pt}%
\pgfpathmoveto{\pgfqpoint{3.561918in}{2.104650in}}%
\pgfpathlineto{\pgfqpoint{3.575073in}{2.095593in}}%
\pgfpathlineto{\pgfqpoint{3.588231in}{2.086678in}}%
\pgfpathlineto{\pgfqpoint{3.601391in}{2.077903in}}%
\pgfpathlineto{\pgfqpoint{3.614552in}{2.069269in}}%
\pgfpathlineto{\pgfqpoint{3.622360in}{2.076420in}}%
\pgfpathlineto{\pgfqpoint{3.630160in}{2.083650in}}%
\pgfpathlineto{\pgfqpoint{3.637954in}{2.090956in}}%
\pgfpathlineto{\pgfqpoint{3.645742in}{2.098338in}}%
\pgfpathlineto{\pgfqpoint{3.632598in}{2.106757in}}%
\pgfpathlineto{\pgfqpoint{3.619456in}{2.115316in}}%
\pgfpathlineto{\pgfqpoint{3.606316in}{2.124016in}}%
\pgfpathlineto{\pgfqpoint{3.593179in}{2.132856in}}%
\pgfpathlineto{\pgfqpoint{3.585374in}{2.125684in}}%
\pgfpathlineto{\pgfqpoint{3.577562in}{2.118591in}}%
\pgfpathlineto{\pgfqpoint{3.569743in}{2.111579in}}%
\pgfpathlineto{\pgfqpoint{3.561918in}{2.104650in}}%
\pgfpathclose%
\pgfusepath{fill}%
\end{pgfscope}%
\begin{pgfscope}%
\pgfpathrectangle{\pgfqpoint{1.254980in}{0.150000in}}{\pgfqpoint{5.490039in}{5.490039in}}%
\pgfusepath{clip}%
\pgfsetbuttcap%
\pgfsetroundjoin%
\definecolor{currentfill}{rgb}{0.280267,0.073417,0.397163}%
\pgfsetfillcolor{currentfill}%
\pgfsetfillopacity{0.700000}%
\pgfsetlinewidth{0.000000pt}%
\definecolor{currentstroke}{rgb}{0.000000,0.000000,0.000000}%
\pgfsetstrokecolor{currentstroke}%
\pgfsetdash{}{0pt}%
\pgfpathmoveto{\pgfqpoint{4.106950in}{2.026454in}}%
\pgfpathlineto{\pgfqpoint{4.120174in}{2.022414in}}%
\pgfpathlineto{\pgfqpoint{4.133404in}{2.018501in}}%
\pgfpathlineto{\pgfqpoint{4.146640in}{2.014713in}}%
\pgfpathlineto{\pgfqpoint{4.159883in}{2.011050in}}%
\pgfpathlineto{\pgfqpoint{4.167487in}{2.020529in}}%
\pgfpathlineto{\pgfqpoint{4.175087in}{2.030034in}}%
\pgfpathlineto{\pgfqpoint{4.182681in}{2.039563in}}%
\pgfpathlineto{\pgfqpoint{4.190270in}{2.049115in}}%
\pgfpathlineto{\pgfqpoint{4.177038in}{2.052629in}}%
\pgfpathlineto{\pgfqpoint{4.163813in}{2.056267in}}%
\pgfpathlineto{\pgfqpoint{4.150594in}{2.060031in}}%
\pgfpathlineto{\pgfqpoint{4.137381in}{2.063921in}}%
\pgfpathlineto{\pgfqpoint{4.129781in}{2.054513in}}%
\pgfpathlineto{\pgfqpoint{4.122176in}{2.045132in}}%
\pgfpathlineto{\pgfqpoint{4.114565in}{2.035778in}}%
\pgfpathlineto{\pgfqpoint{4.106950in}{2.026454in}}%
\pgfpathclose%
\pgfusepath{fill}%
\end{pgfscope}%
\begin{pgfscope}%
\pgfpathrectangle{\pgfqpoint{1.254980in}{0.150000in}}{\pgfqpoint{5.490039in}{5.490039in}}%
\pgfusepath{clip}%
\pgfsetbuttcap%
\pgfsetroundjoin%
\definecolor{currentfill}{rgb}{0.283197,0.115680,0.436115}%
\pgfsetfillcolor{currentfill}%
\pgfsetfillopacity{0.700000}%
\pgfsetlinewidth{0.000000pt}%
\definecolor{currentstroke}{rgb}{0.000000,0.000000,0.000000}%
\pgfsetstrokecolor{currentstroke}%
\pgfsetdash{}{0pt}%
\pgfpathmoveto{\pgfqpoint{4.409900in}{2.096735in}}%
\pgfpathlineto{\pgfqpoint{4.423206in}{2.095069in}}%
\pgfpathlineto{\pgfqpoint{4.436521in}{2.093524in}}%
\pgfpathlineto{\pgfqpoint{4.449844in}{2.092100in}}%
\pgfpathlineto{\pgfqpoint{4.463176in}{2.090795in}}%
\pgfpathlineto{\pgfqpoint{4.470685in}{2.100946in}}%
\pgfpathlineto{\pgfqpoint{4.478189in}{2.111096in}}%
\pgfpathlineto{\pgfqpoint{4.485688in}{2.121245in}}%
\pgfpathlineto{\pgfqpoint{4.493183in}{2.131392in}}%
\pgfpathlineto{\pgfqpoint{4.479860in}{2.132594in}}%
\pgfpathlineto{\pgfqpoint{4.466545in}{2.133917in}}%
\pgfpathlineto{\pgfqpoint{4.453238in}{2.135360in}}%
\pgfpathlineto{\pgfqpoint{4.439940in}{2.136924in}}%
\pgfpathlineto{\pgfqpoint{4.432437in}{2.126874in}}%
\pgfpathlineto{\pgfqpoint{4.424929in}{2.116825in}}%
\pgfpathlineto{\pgfqpoint{4.417417in}{2.106779in}}%
\pgfpathlineto{\pgfqpoint{4.409900in}{2.096735in}}%
\pgfpathclose%
\pgfusepath{fill}%
\end{pgfscope}%
\begin{pgfscope}%
\pgfpathrectangle{\pgfqpoint{1.254980in}{0.150000in}}{\pgfqpoint{5.490039in}{5.490039in}}%
\pgfusepath{clip}%
\pgfsetbuttcap%
\pgfsetroundjoin%
\definecolor{currentfill}{rgb}{0.227802,0.326594,0.546532}%
\pgfsetfillcolor{currentfill}%
\pgfsetfillopacity{0.700000}%
\pgfsetlinewidth{0.000000pt}%
\definecolor{currentstroke}{rgb}{0.000000,0.000000,0.000000}%
\pgfsetstrokecolor{currentstroke}%
\pgfsetdash{}{0pt}%
\pgfpathmoveto{\pgfqpoint{5.243379in}{2.535506in}}%
\pgfpathlineto{\pgfqpoint{5.257021in}{2.538808in}}%
\pgfpathlineto{\pgfqpoint{5.270675in}{2.542224in}}%
\pgfpathlineto{\pgfqpoint{5.284342in}{2.545752in}}%
\pgfpathlineto{\pgfqpoint{5.298021in}{2.549395in}}%
\pgfpathlineto{\pgfqpoint{5.305250in}{2.558862in}}%
\pgfpathlineto{\pgfqpoint{5.312472in}{2.568292in}}%
\pgfpathlineto{\pgfqpoint{5.319690in}{2.577684in}}%
\pgfpathlineto{\pgfqpoint{5.326901in}{2.587040in}}%
\pgfpathlineto{\pgfqpoint{5.313231in}{2.583458in}}%
\pgfpathlineto{\pgfqpoint{5.299573in}{2.579988in}}%
\pgfpathlineto{\pgfqpoint{5.285928in}{2.576632in}}%
\pgfpathlineto{\pgfqpoint{5.272296in}{2.573389in}}%
\pgfpathlineto{\pgfqpoint{5.265075in}{2.563968in}}%
\pgfpathlineto{\pgfqpoint{5.257849in}{2.554514in}}%
\pgfpathlineto{\pgfqpoint{5.250617in}{2.545027in}}%
\pgfpathlineto{\pgfqpoint{5.243379in}{2.535506in}}%
\pgfpathclose%
\pgfusepath{fill}%
\end{pgfscope}%
\begin{pgfscope}%
\pgfpathrectangle{\pgfqpoint{1.254980in}{0.150000in}}{\pgfqpoint{5.490039in}{5.490039in}}%
\pgfusepath{clip}%
\pgfsetbuttcap%
\pgfsetroundjoin%
\definecolor{currentfill}{rgb}{0.281412,0.155834,0.469201}%
\pgfsetfillcolor{currentfill}%
\pgfsetfillopacity{0.700000}%
\pgfsetlinewidth{0.000000pt}%
\definecolor{currentstroke}{rgb}{0.000000,0.000000,0.000000}%
\pgfsetstrokecolor{currentstroke}%
\pgfsetdash{}{0pt}%
\pgfpathmoveto{\pgfqpoint{3.372533in}{2.200498in}}%
\pgfpathlineto{\pgfqpoint{3.385698in}{2.189465in}}%
\pgfpathlineto{\pgfqpoint{3.398862in}{2.178581in}}%
\pgfpathlineto{\pgfqpoint{3.412027in}{2.167845in}}%
\pgfpathlineto{\pgfqpoint{3.425192in}{2.157258in}}%
\pgfpathlineto{\pgfqpoint{3.433086in}{2.163362in}}%
\pgfpathlineto{\pgfqpoint{3.440973in}{2.169564in}}%
\pgfpathlineto{\pgfqpoint{3.448852in}{2.175861in}}%
\pgfpathlineto{\pgfqpoint{3.456724in}{2.182253in}}%
\pgfpathlineto{\pgfqpoint{3.443579in}{2.192607in}}%
\pgfpathlineto{\pgfqpoint{3.430435in}{2.203109in}}%
\pgfpathlineto{\pgfqpoint{3.417292in}{2.213758in}}%
\pgfpathlineto{\pgfqpoint{3.404149in}{2.224557in}}%
\pgfpathlineto{\pgfqpoint{3.396257in}{2.218393in}}%
\pgfpathlineto{\pgfqpoint{3.388357in}{2.212327in}}%
\pgfpathlineto{\pgfqpoint{3.380449in}{2.206362in}}%
\pgfpathlineto{\pgfqpoint{3.372533in}{2.200498in}}%
\pgfpathclose%
\pgfusepath{fill}%
\end{pgfscope}%
\begin{pgfscope}%
\pgfpathrectangle{\pgfqpoint{1.254980in}{0.150000in}}{\pgfqpoint{5.490039in}{5.490039in}}%
\pgfusepath{clip}%
\pgfsetbuttcap%
\pgfsetroundjoin%
\definecolor{currentfill}{rgb}{0.216210,0.351535,0.550627}%
\pgfsetfillcolor{currentfill}%
\pgfsetfillopacity{0.700000}%
\pgfsetlinewidth{0.000000pt}%
\definecolor{currentstroke}{rgb}{0.000000,0.000000,0.000000}%
\pgfsetstrokecolor{currentstroke}%
\pgfsetdash{}{0pt}%
\pgfpathmoveto{\pgfqpoint{5.326901in}{2.587040in}}%
\pgfpathlineto{\pgfqpoint{5.340584in}{2.590736in}}%
\pgfpathlineto{\pgfqpoint{5.354280in}{2.594545in}}%
\pgfpathlineto{\pgfqpoint{5.367990in}{2.598467in}}%
\pgfpathlineto{\pgfqpoint{5.381712in}{2.602502in}}%
\pgfpathlineto{\pgfqpoint{5.388908in}{2.611753in}}%
\pgfpathlineto{\pgfqpoint{5.396099in}{2.620966in}}%
\pgfpathlineto{\pgfqpoint{5.403284in}{2.630141in}}%
\pgfpathlineto{\pgfqpoint{5.410464in}{2.639279in}}%
\pgfpathlineto{\pgfqpoint{5.396751in}{2.635321in}}%
\pgfpathlineto{\pgfqpoint{5.383052in}{2.631475in}}%
\pgfpathlineto{\pgfqpoint{5.369366in}{2.627742in}}%
\pgfpathlineto{\pgfqpoint{5.355692in}{2.624122in}}%
\pgfpathlineto{\pgfqpoint{5.348503in}{2.614901in}}%
\pgfpathlineto{\pgfqpoint{5.341308in}{2.605648in}}%
\pgfpathlineto{\pgfqpoint{5.334107in}{2.596362in}}%
\pgfpathlineto{\pgfqpoint{5.326901in}{2.587040in}}%
\pgfpathclose%
\pgfusepath{fill}%
\end{pgfscope}%
\begin{pgfscope}%
\pgfpathrectangle{\pgfqpoint{1.254980in}{0.150000in}}{\pgfqpoint{5.490039in}{5.490039in}}%
\pgfusepath{clip}%
\pgfsetbuttcap%
\pgfsetroundjoin%
\definecolor{currentfill}{rgb}{0.144759,0.519093,0.556572}%
\pgfsetfillcolor{currentfill}%
\pgfsetfillopacity{0.700000}%
\pgfsetlinewidth{0.000000pt}%
\definecolor{currentstroke}{rgb}{0.000000,0.000000,0.000000}%
\pgfsetstrokecolor{currentstroke}%
\pgfsetdash{}{0pt}%
\pgfpathmoveto{\pgfqpoint{2.631344in}{3.082699in}}%
\pgfpathlineto{\pgfqpoint{2.644728in}{3.061859in}}%
\pgfpathlineto{\pgfqpoint{2.658102in}{3.041225in}}%
\pgfpathlineto{\pgfqpoint{2.671469in}{3.020798in}}%
\pgfpathlineto{\pgfqpoint{2.684828in}{3.000575in}}%
\pgfpathlineto{\pgfqpoint{2.693102in}{3.003199in}}%
\pgfpathlineto{\pgfqpoint{2.701364in}{3.005981in}}%
\pgfpathlineto{\pgfqpoint{2.709615in}{3.008918in}}%
\pgfpathlineto{\pgfqpoint{2.717854in}{3.012009in}}%
\pgfpathlineto{\pgfqpoint{2.704529in}{3.031979in}}%
\pgfpathlineto{\pgfqpoint{2.691196in}{3.052151in}}%
\pgfpathlineto{\pgfqpoint{2.677855in}{3.072530in}}%
\pgfpathlineto{\pgfqpoint{2.664506in}{3.093114in}}%
\pgfpathlineto{\pgfqpoint{2.656233in}{3.090271in}}%
\pgfpathlineto{\pgfqpoint{2.647949in}{3.087586in}}%
\pgfpathlineto{\pgfqpoint{2.639653in}{3.085062in}}%
\pgfpathlineto{\pgfqpoint{2.631344in}{3.082699in}}%
\pgfpathclose%
\pgfusepath{fill}%
\end{pgfscope}%
\begin{pgfscope}%
\pgfpathrectangle{\pgfqpoint{1.254980in}{0.150000in}}{\pgfqpoint{5.490039in}{5.490039in}}%
\pgfusepath{clip}%
\pgfsetbuttcap%
\pgfsetroundjoin%
\definecolor{currentfill}{rgb}{0.278791,0.062145,0.386592}%
\pgfsetfillcolor{currentfill}%
\pgfsetfillopacity{0.700000}%
\pgfsetlinewidth{0.000000pt}%
\definecolor{currentstroke}{rgb}{0.000000,0.000000,0.000000}%
\pgfsetstrokecolor{currentstroke}%
\pgfsetdash{}{0pt}%
\pgfpathmoveto{\pgfqpoint{3.887289in}{2.015622in}}%
\pgfpathlineto{\pgfqpoint{3.900475in}{2.009696in}}%
\pgfpathlineto{\pgfqpoint{3.913666in}{2.003900in}}%
\pgfpathlineto{\pgfqpoint{3.926861in}{1.998235in}}%
\pgfpathlineto{\pgfqpoint{3.940061in}{1.992700in}}%
\pgfpathlineto{\pgfqpoint{3.947743in}{2.001376in}}%
\pgfpathlineto{\pgfqpoint{3.955419in}{2.010099in}}%
\pgfpathlineto{\pgfqpoint{3.963090in}{2.018867in}}%
\pgfpathlineto{\pgfqpoint{3.970755in}{2.027679in}}%
\pgfpathlineto{\pgfqpoint{3.957569in}{2.033033in}}%
\pgfpathlineto{\pgfqpoint{3.944386in}{2.038517in}}%
\pgfpathlineto{\pgfqpoint{3.931209in}{2.044130in}}%
\pgfpathlineto{\pgfqpoint{3.918036in}{2.049875in}}%
\pgfpathlineto{\pgfqpoint{3.910358in}{2.041238in}}%
\pgfpathlineto{\pgfqpoint{3.902674in}{2.032650in}}%
\pgfpathlineto{\pgfqpoint{3.894984in}{2.024111in}}%
\pgfpathlineto{\pgfqpoint{3.887289in}{2.015622in}}%
\pgfpathclose%
\pgfusepath{fill}%
\end{pgfscope}%
\begin{pgfscope}%
\pgfpathrectangle{\pgfqpoint{1.254980in}{0.150000in}}{\pgfqpoint{5.490039in}{5.490039in}}%
\pgfusepath{clip}%
\pgfsetbuttcap%
\pgfsetroundjoin%
\definecolor{currentfill}{rgb}{0.280267,0.073417,0.397163}%
\pgfsetfillcolor{currentfill}%
\pgfsetfillopacity{0.700000}%
\pgfsetlinewidth{0.000000pt}%
\definecolor{currentstroke}{rgb}{0.000000,0.000000,0.000000}%
\pgfsetstrokecolor{currentstroke}%
\pgfsetdash{}{0pt}%
\pgfpathmoveto{\pgfqpoint{3.750990in}{2.035949in}}%
\pgfpathlineto{\pgfqpoint{3.764160in}{2.028763in}}%
\pgfpathlineto{\pgfqpoint{3.777333in}{2.021712in}}%
\pgfpathlineto{\pgfqpoint{3.790510in}{2.014794in}}%
\pgfpathlineto{\pgfqpoint{3.803690in}{2.008010in}}%
\pgfpathlineto{\pgfqpoint{3.811423in}{2.016079in}}%
\pgfpathlineto{\pgfqpoint{3.819150in}{2.024208in}}%
\pgfpathlineto{\pgfqpoint{3.826871in}{2.032396in}}%
\pgfpathlineto{\pgfqpoint{3.834587in}{2.040642in}}%
\pgfpathlineto{\pgfqpoint{3.821421in}{2.047228in}}%
\pgfpathlineto{\pgfqpoint{3.808259in}{2.053947in}}%
\pgfpathlineto{\pgfqpoint{3.795101in}{2.060800in}}%
\pgfpathlineto{\pgfqpoint{3.781947in}{2.067788in}}%
\pgfpathlineto{\pgfqpoint{3.774217in}{2.059735in}}%
\pgfpathlineto{\pgfqpoint{3.766481in}{2.051743in}}%
\pgfpathlineto{\pgfqpoint{3.758739in}{2.043814in}}%
\pgfpathlineto{\pgfqpoint{3.750990in}{2.035949in}}%
\pgfpathclose%
\pgfusepath{fill}%
\end{pgfscope}%
\begin{pgfscope}%
\pgfpathrectangle{\pgfqpoint{1.254980in}{0.150000in}}{\pgfqpoint{5.490039in}{5.490039in}}%
\pgfusepath{clip}%
\pgfsetbuttcap%
\pgfsetroundjoin%
\definecolor{currentfill}{rgb}{0.282656,0.100196,0.422160}%
\pgfsetfillcolor{currentfill}%
\pgfsetfillopacity{0.700000}%
\pgfsetlinewidth{0.000000pt}%
\definecolor{currentstroke}{rgb}{0.000000,0.000000,0.000000}%
\pgfsetstrokecolor{currentstroke}%
\pgfsetdash{}{0pt}%
\pgfpathmoveto{\pgfqpoint{4.326598in}{2.064954in}}%
\pgfpathlineto{\pgfqpoint{4.339882in}{2.062685in}}%
\pgfpathlineto{\pgfqpoint{4.353175in}{2.060538in}}%
\pgfpathlineto{\pgfqpoint{4.366475in}{2.058512in}}%
\pgfpathlineto{\pgfqpoint{4.379782in}{2.056607in}}%
\pgfpathlineto{\pgfqpoint{4.387319in}{2.066631in}}%
\pgfpathlineto{\pgfqpoint{4.394851in}{2.076661in}}%
\pgfpathlineto{\pgfqpoint{4.402377in}{2.086696in}}%
\pgfpathlineto{\pgfqpoint{4.409900in}{2.096735in}}%
\pgfpathlineto{\pgfqpoint{4.396601in}{2.098522in}}%
\pgfpathlineto{\pgfqpoint{4.383309in}{2.100431in}}%
\pgfpathlineto{\pgfqpoint{4.370026in}{2.102460in}}%
\pgfpathlineto{\pgfqpoint{4.356750in}{2.104612in}}%
\pgfpathlineto{\pgfqpoint{4.349219in}{2.094684in}}%
\pgfpathlineto{\pgfqpoint{4.341684in}{2.084765in}}%
\pgfpathlineto{\pgfqpoint{4.334143in}{2.074855in}}%
\pgfpathlineto{\pgfqpoint{4.326598in}{2.064954in}}%
\pgfpathclose%
\pgfusepath{fill}%
\end{pgfscope}%
\begin{pgfscope}%
\pgfpathrectangle{\pgfqpoint{1.254980in}{0.150000in}}{\pgfqpoint{5.490039in}{5.490039in}}%
\pgfusepath{clip}%
\pgfsetbuttcap%
\pgfsetroundjoin%
\definecolor{currentfill}{rgb}{0.204903,0.375746,0.553533}%
\pgfsetfillcolor{currentfill}%
\pgfsetfillopacity{0.700000}%
\pgfsetlinewidth{0.000000pt}%
\definecolor{currentstroke}{rgb}{0.000000,0.000000,0.000000}%
\pgfsetstrokecolor{currentstroke}%
\pgfsetdash{}{0pt}%
\pgfpathmoveto{\pgfqpoint{5.410464in}{2.639279in}}%
\pgfpathlineto{\pgfqpoint{5.424189in}{2.643351in}}%
\pgfpathlineto{\pgfqpoint{5.437928in}{2.647535in}}%
\pgfpathlineto{\pgfqpoint{5.451681in}{2.651831in}}%
\pgfpathlineto{\pgfqpoint{5.465446in}{2.656240in}}%
\pgfpathlineto{\pgfqpoint{5.472610in}{2.665257in}}%
\pgfpathlineto{\pgfqpoint{5.479768in}{2.674235in}}%
\pgfpathlineto{\pgfqpoint{5.486920in}{2.683176in}}%
\pgfpathlineto{\pgfqpoint{5.494066in}{2.692080in}}%
\pgfpathlineto{\pgfqpoint{5.480311in}{2.687764in}}%
\pgfpathlineto{\pgfqpoint{5.466569in}{2.683560in}}%
\pgfpathlineto{\pgfqpoint{5.452840in}{2.679469in}}%
\pgfpathlineto{\pgfqpoint{5.439125in}{2.675490in}}%
\pgfpathlineto{\pgfqpoint{5.431968in}{2.666487in}}%
\pgfpathlineto{\pgfqpoint{5.424806in}{2.657451in}}%
\pgfpathlineto{\pgfqpoint{5.417637in}{2.648383in}}%
\pgfpathlineto{\pgfqpoint{5.410464in}{2.639279in}}%
\pgfpathclose%
\pgfusepath{fill}%
\end{pgfscope}%
\begin{pgfscope}%
\pgfpathrectangle{\pgfqpoint{1.254980in}{0.150000in}}{\pgfqpoint{5.490039in}{5.490039in}}%
\pgfusepath{clip}%
\pgfsetbuttcap%
\pgfsetroundjoin%
\definecolor{currentfill}{rgb}{0.278791,0.062145,0.386592}%
\pgfsetfillcolor{currentfill}%
\pgfsetfillopacity{0.700000}%
\pgfsetlinewidth{0.000000pt}%
\definecolor{currentstroke}{rgb}{0.000000,0.000000,0.000000}%
\pgfsetstrokecolor{currentstroke}%
\pgfsetdash{}{0pt}%
\pgfpathmoveto{\pgfqpoint{4.023554in}{2.007552in}}%
\pgfpathlineto{\pgfqpoint{4.036766in}{2.002840in}}%
\pgfpathlineto{\pgfqpoint{4.049984in}{1.998256in}}%
\pgfpathlineto{\pgfqpoint{4.063208in}{1.993799in}}%
\pgfpathlineto{\pgfqpoint{4.076437in}{1.989468in}}%
\pgfpathlineto{\pgfqpoint{4.084073in}{1.998665in}}%
\pgfpathlineto{\pgfqpoint{4.091704in}{2.007896in}}%
\pgfpathlineto{\pgfqpoint{4.099330in}{2.017159in}}%
\pgfpathlineto{\pgfqpoint{4.106950in}{2.026454in}}%
\pgfpathlineto{\pgfqpoint{4.093732in}{2.030619in}}%
\pgfpathlineto{\pgfqpoint{4.080520in}{2.034911in}}%
\pgfpathlineto{\pgfqpoint{4.067313in}{2.039330in}}%
\pgfpathlineto{\pgfqpoint{4.054113in}{2.043876in}}%
\pgfpathlineto{\pgfqpoint{4.046481in}{2.034741in}}%
\pgfpathlineto{\pgfqpoint{4.038844in}{2.025642in}}%
\pgfpathlineto{\pgfqpoint{4.031201in}{2.016578in}}%
\pgfpathlineto{\pgfqpoint{4.023554in}{2.007552in}}%
\pgfpathclose%
\pgfusepath{fill}%
\end{pgfscope}%
\begin{pgfscope}%
\pgfpathrectangle{\pgfqpoint{1.254980in}{0.150000in}}{\pgfqpoint{5.490039in}{5.490039in}}%
\pgfusepath{clip}%
\pgfsetbuttcap%
\pgfsetroundjoin%
\definecolor{currentfill}{rgb}{0.282327,0.094955,0.417331}%
\pgfsetfillcolor{currentfill}%
\pgfsetfillopacity{0.700000}%
\pgfsetlinewidth{0.000000pt}%
\definecolor{currentstroke}{rgb}{0.000000,0.000000,0.000000}%
\pgfsetstrokecolor{currentstroke}%
\pgfsetdash{}{0pt}%
\pgfpathmoveto{\pgfqpoint{3.614552in}{2.069269in}}%
\pgfpathlineto{\pgfqpoint{3.627716in}{2.060773in}}%
\pgfpathlineto{\pgfqpoint{3.640882in}{2.052417in}}%
\pgfpathlineto{\pgfqpoint{3.654051in}{2.044198in}}%
\pgfpathlineto{\pgfqpoint{3.667222in}{2.036118in}}%
\pgfpathlineto{\pgfqpoint{3.675012in}{2.043490in}}%
\pgfpathlineto{\pgfqpoint{3.682796in}{2.050937in}}%
\pgfpathlineto{\pgfqpoint{3.690573in}{2.058456in}}%
\pgfpathlineto{\pgfqpoint{3.698343in}{2.066047in}}%
\pgfpathlineto{\pgfqpoint{3.685189in}{2.073913in}}%
\pgfpathlineto{\pgfqpoint{3.672037in}{2.081916in}}%
\pgfpathlineto{\pgfqpoint{3.658888in}{2.090058in}}%
\pgfpathlineto{\pgfqpoint{3.645742in}{2.098338in}}%
\pgfpathlineto{\pgfqpoint{3.637954in}{2.090956in}}%
\pgfpathlineto{\pgfqpoint{3.630160in}{2.083650in}}%
\pgfpathlineto{\pgfqpoint{3.622360in}{2.076420in}}%
\pgfpathlineto{\pgfqpoint{3.614552in}{2.069269in}}%
\pgfpathclose%
\pgfusepath{fill}%
\end{pgfscope}%
\begin{pgfscope}%
\pgfpathrectangle{\pgfqpoint{1.254980in}{0.150000in}}{\pgfqpoint{5.490039in}{5.490039in}}%
\pgfusepath{clip}%
\pgfsetbuttcap%
\pgfsetroundjoin%
\definecolor{currentfill}{rgb}{0.195860,0.395433,0.555276}%
\pgfsetfillcolor{currentfill}%
\pgfsetfillopacity{0.700000}%
\pgfsetlinewidth{0.000000pt}%
\definecolor{currentstroke}{rgb}{0.000000,0.000000,0.000000}%
\pgfsetstrokecolor{currentstroke}%
\pgfsetdash{}{0pt}%
\pgfpathmoveto{\pgfqpoint{5.494066in}{2.692080in}}%
\pgfpathlineto{\pgfqpoint{5.507835in}{2.696509in}}%
\pgfpathlineto{\pgfqpoint{5.521617in}{2.701050in}}%
\pgfpathlineto{\pgfqpoint{5.535414in}{2.705703in}}%
\pgfpathlineto{\pgfqpoint{5.549224in}{2.710468in}}%
\pgfpathlineto{\pgfqpoint{5.556354in}{2.719234in}}%
\pgfpathlineto{\pgfqpoint{5.563477in}{2.727962in}}%
\pgfpathlineto{\pgfqpoint{5.570595in}{2.736654in}}%
\pgfpathlineto{\pgfqpoint{5.577707in}{2.745310in}}%
\pgfpathlineto{\pgfqpoint{5.563908in}{2.740654in}}%
\pgfpathlineto{\pgfqpoint{5.550123in}{2.736111in}}%
\pgfpathlineto{\pgfqpoint{5.536352in}{2.731679in}}%
\pgfpathlineto{\pgfqpoint{5.522594in}{2.727359in}}%
\pgfpathlineto{\pgfqpoint{5.515470in}{2.718588in}}%
\pgfpathlineto{\pgfqpoint{5.508341in}{2.709785in}}%
\pgfpathlineto{\pgfqpoint{5.501206in}{2.700950in}}%
\pgfpathlineto{\pgfqpoint{5.494066in}{2.692080in}}%
\pgfpathclose%
\pgfusepath{fill}%
\end{pgfscope}%
\begin{pgfscope}%
\pgfpathrectangle{\pgfqpoint{1.254980in}{0.150000in}}{\pgfqpoint{5.490039in}{5.490039in}}%
\pgfusepath{clip}%
\pgfsetbuttcap%
\pgfsetroundjoin%
\definecolor{currentfill}{rgb}{0.281446,0.084320,0.407414}%
\pgfsetfillcolor{currentfill}%
\pgfsetfillopacity{0.700000}%
\pgfsetlinewidth{0.000000pt}%
\definecolor{currentstroke}{rgb}{0.000000,0.000000,0.000000}%
\pgfsetstrokecolor{currentstroke}%
\pgfsetdash{}{0pt}%
\pgfpathmoveto{\pgfqpoint{4.243264in}{2.036304in}}%
\pgfpathlineto{\pgfqpoint{4.256529in}{2.033411in}}%
\pgfpathlineto{\pgfqpoint{4.269801in}{2.030640in}}%
\pgfpathlineto{\pgfqpoint{4.283081in}{2.027993in}}%
\pgfpathlineto{\pgfqpoint{4.296367in}{2.025468in}}%
\pgfpathlineto{\pgfqpoint{4.303932in}{2.035321in}}%
\pgfpathlineto{\pgfqpoint{4.311492in}{2.045186in}}%
\pgfpathlineto{\pgfqpoint{4.319047in}{2.055065in}}%
\pgfpathlineto{\pgfqpoint{4.326598in}{2.064954in}}%
\pgfpathlineto{\pgfqpoint{4.313320in}{2.067346in}}%
\pgfpathlineto{\pgfqpoint{4.300050in}{2.069860in}}%
\pgfpathlineto{\pgfqpoint{4.286787in}{2.072497in}}%
\pgfpathlineto{\pgfqpoint{4.273532in}{2.075257in}}%
\pgfpathlineto{\pgfqpoint{4.265972in}{2.065495in}}%
\pgfpathlineto{\pgfqpoint{4.258408in}{2.055748in}}%
\pgfpathlineto{\pgfqpoint{4.250838in}{2.046017in}}%
\pgfpathlineto{\pgfqpoint{4.243264in}{2.036304in}}%
\pgfpathclose%
\pgfusepath{fill}%
\end{pgfscope}%
\begin{pgfscope}%
\pgfpathrectangle{\pgfqpoint{1.254980in}{0.150000in}}{\pgfqpoint{5.490039in}{5.490039in}}%
\pgfusepath{clip}%
\pgfsetbuttcap%
\pgfsetroundjoin%
\definecolor{currentfill}{rgb}{0.282884,0.135920,0.453427}%
\pgfsetfillcolor{currentfill}%
\pgfsetfillopacity{0.700000}%
\pgfsetlinewidth{0.000000pt}%
\definecolor{currentstroke}{rgb}{0.000000,0.000000,0.000000}%
\pgfsetstrokecolor{currentstroke}%
\pgfsetdash{}{0pt}%
\pgfpathmoveto{\pgfqpoint{3.425192in}{2.157258in}}%
\pgfpathlineto{\pgfqpoint{3.438358in}{2.146817in}}%
\pgfpathlineto{\pgfqpoint{3.451525in}{2.136522in}}%
\pgfpathlineto{\pgfqpoint{3.464692in}{2.126373in}}%
\pgfpathlineto{\pgfqpoint{3.477860in}{2.116369in}}%
\pgfpathlineto{\pgfqpoint{3.485734in}{2.122713in}}%
\pgfpathlineto{\pgfqpoint{3.493600in}{2.129150in}}%
\pgfpathlineto{\pgfqpoint{3.501459in}{2.135679in}}%
\pgfpathlineto{\pgfqpoint{3.509311in}{2.142298in}}%
\pgfpathlineto{\pgfqpoint{3.496162in}{2.152069in}}%
\pgfpathlineto{\pgfqpoint{3.483015in}{2.161985in}}%
\pgfpathlineto{\pgfqpoint{3.469869in}{2.172046in}}%
\pgfpathlineto{\pgfqpoint{3.456724in}{2.182253in}}%
\pgfpathlineto{\pgfqpoint{3.448852in}{2.175861in}}%
\pgfpathlineto{\pgfqpoint{3.440973in}{2.169564in}}%
\pgfpathlineto{\pgfqpoint{3.433086in}{2.163362in}}%
\pgfpathlineto{\pgfqpoint{3.425192in}{2.157258in}}%
\pgfpathclose%
\pgfusepath{fill}%
\end{pgfscope}%
\begin{pgfscope}%
\pgfpathrectangle{\pgfqpoint{1.254980in}{0.150000in}}{\pgfqpoint{5.490039in}{5.490039in}}%
\pgfusepath{clip}%
\pgfsetbuttcap%
\pgfsetroundjoin%
\definecolor{currentfill}{rgb}{0.229739,0.322361,0.545706}%
\pgfsetfillcolor{currentfill}%
\pgfsetfillopacity{0.700000}%
\pgfsetlinewidth{0.000000pt}%
\definecolor{currentstroke}{rgb}{0.000000,0.000000,0.000000}%
\pgfsetstrokecolor{currentstroke}%
\pgfsetdash{}{0pt}%
\pgfpathmoveto{\pgfqpoint{2.971015in}{2.557831in}}%
\pgfpathlineto{\pgfqpoint{2.984257in}{2.542040in}}%
\pgfpathlineto{\pgfqpoint{2.997495in}{2.526421in}}%
\pgfpathlineto{\pgfqpoint{3.010729in}{2.510976in}}%
\pgfpathlineto{\pgfqpoint{3.023960in}{2.495701in}}%
\pgfpathlineto{\pgfqpoint{3.032061in}{2.499574in}}%
\pgfpathlineto{\pgfqpoint{3.040152in}{2.503582in}}%
\pgfpathlineto{\pgfqpoint{3.048234in}{2.507722in}}%
\pgfpathlineto{\pgfqpoint{3.056305in}{2.511994in}}%
\pgfpathlineto{\pgfqpoint{3.043102in}{2.527009in}}%
\pgfpathlineto{\pgfqpoint{3.029896in}{2.542196in}}%
\pgfpathlineto{\pgfqpoint{3.016686in}{2.557555in}}%
\pgfpathlineto{\pgfqpoint{3.003473in}{2.573086in}}%
\pgfpathlineto{\pgfqpoint{2.995373in}{2.569068in}}%
\pgfpathlineto{\pgfqpoint{2.987264in}{2.565184in}}%
\pgfpathlineto{\pgfqpoint{2.979145in}{2.561438in}}%
\pgfpathlineto{\pgfqpoint{2.971015in}{2.557831in}}%
\pgfpathclose%
\pgfusepath{fill}%
\end{pgfscope}%
\begin{pgfscope}%
\pgfpathrectangle{\pgfqpoint{1.254980in}{0.150000in}}{\pgfqpoint{5.490039in}{5.490039in}}%
\pgfusepath{clip}%
\pgfsetbuttcap%
\pgfsetroundjoin%
\definecolor{currentfill}{rgb}{0.241237,0.296485,0.539709}%
\pgfsetfillcolor{currentfill}%
\pgfsetfillopacity{0.700000}%
\pgfsetlinewidth{0.000000pt}%
\definecolor{currentstroke}{rgb}{0.000000,0.000000,0.000000}%
\pgfsetstrokecolor{currentstroke}%
\pgfsetdash{}{0pt}%
\pgfpathmoveto{\pgfqpoint{3.023960in}{2.495701in}}%
\pgfpathlineto{\pgfqpoint{3.037188in}{2.480597in}}%
\pgfpathlineto{\pgfqpoint{3.050412in}{2.465662in}}%
\pgfpathlineto{\pgfqpoint{3.063633in}{2.450895in}}%
\pgfpathlineto{\pgfqpoint{3.076850in}{2.436296in}}%
\pgfpathlineto{\pgfqpoint{3.084924in}{2.440433in}}%
\pgfpathlineto{\pgfqpoint{3.092987in}{2.444701in}}%
\pgfpathlineto{\pgfqpoint{3.101041in}{2.449097in}}%
\pgfpathlineto{\pgfqpoint{3.109086in}{2.453620in}}%
\pgfpathlineto{\pgfqpoint{3.095895in}{2.467962in}}%
\pgfpathlineto{\pgfqpoint{3.082701in}{2.482471in}}%
\pgfpathlineto{\pgfqpoint{3.069505in}{2.497148in}}%
\pgfpathlineto{\pgfqpoint{3.056305in}{2.511994in}}%
\pgfpathlineto{\pgfqpoint{3.048234in}{2.507722in}}%
\pgfpathlineto{\pgfqpoint{3.040152in}{2.503582in}}%
\pgfpathlineto{\pgfqpoint{3.032061in}{2.499574in}}%
\pgfpathlineto{\pgfqpoint{3.023960in}{2.495701in}}%
\pgfpathclose%
\pgfusepath{fill}%
\end{pgfscope}%
\begin{pgfscope}%
\pgfpathrectangle{\pgfqpoint{1.254980in}{0.150000in}}{\pgfqpoint{5.490039in}{5.490039in}}%
\pgfusepath{clip}%
\pgfsetbuttcap%
\pgfsetroundjoin%
\definecolor{currentfill}{rgb}{0.131172,0.555899,0.552459}%
\pgfsetfillcolor{currentfill}%
\pgfsetfillopacity{0.700000}%
\pgfsetlinewidth{0.000000pt}%
\definecolor{currentstroke}{rgb}{0.000000,0.000000,0.000000}%
\pgfsetstrokecolor{currentstroke}%
\pgfsetdash{}{0pt}%
\pgfpathmoveto{\pgfqpoint{2.577722in}{3.168162in}}%
\pgfpathlineto{\pgfqpoint{2.591141in}{3.146478in}}%
\pgfpathlineto{\pgfqpoint{2.604551in}{3.125007in}}%
\pgfpathlineto{\pgfqpoint{2.617952in}{3.103748in}}%
\pgfpathlineto{\pgfqpoint{2.631344in}{3.082699in}}%
\pgfpathlineto{\pgfqpoint{2.639653in}{3.085062in}}%
\pgfpathlineto{\pgfqpoint{2.647949in}{3.087586in}}%
\pgfpathlineto{\pgfqpoint{2.656233in}{3.090271in}}%
\pgfpathlineto{\pgfqpoint{2.664506in}{3.093114in}}%
\pgfpathlineto{\pgfqpoint{2.651148in}{3.113907in}}%
\pgfpathlineto{\pgfqpoint{2.637782in}{3.134909in}}%
\pgfpathlineto{\pgfqpoint{2.624407in}{3.156123in}}%
\pgfpathlineto{\pgfqpoint{2.611023in}{3.177549in}}%
\pgfpathlineto{\pgfqpoint{2.602717in}{3.174956in}}%
\pgfpathlineto{\pgfqpoint{2.594398in}{3.172526in}}%
\pgfpathlineto{\pgfqpoint{2.586066in}{3.170261in}}%
\pgfpathlineto{\pgfqpoint{2.577722in}{3.168162in}}%
\pgfpathclose%
\pgfusepath{fill}%
\end{pgfscope}%
\begin{pgfscope}%
\pgfpathrectangle{\pgfqpoint{1.254980in}{0.150000in}}{\pgfqpoint{5.490039in}{5.490039in}}%
\pgfusepath{clip}%
\pgfsetbuttcap%
\pgfsetroundjoin%
\definecolor{currentfill}{rgb}{0.218130,0.347432,0.550038}%
\pgfsetfillcolor{currentfill}%
\pgfsetfillopacity{0.700000}%
\pgfsetlinewidth{0.000000pt}%
\definecolor{currentstroke}{rgb}{0.000000,0.000000,0.000000}%
\pgfsetstrokecolor{currentstroke}%
\pgfsetdash{}{0pt}%
\pgfpathmoveto{\pgfqpoint{2.918004in}{2.622750in}}%
\pgfpathlineto{\pgfqpoint{2.931263in}{2.606255in}}%
\pgfpathlineto{\pgfqpoint{2.944518in}{2.589937in}}%
\pgfpathlineto{\pgfqpoint{2.957769in}{2.573796in}}%
\pgfpathlineto{\pgfqpoint{2.971015in}{2.557831in}}%
\pgfpathlineto{\pgfqpoint{2.979145in}{2.561438in}}%
\pgfpathlineto{\pgfqpoint{2.987264in}{2.565184in}}%
\pgfpathlineto{\pgfqpoint{2.995373in}{2.569068in}}%
\pgfpathlineto{\pgfqpoint{3.003473in}{2.573086in}}%
\pgfpathlineto{\pgfqpoint{2.990255in}{2.588791in}}%
\pgfpathlineto{\pgfqpoint{2.977034in}{2.604672in}}%
\pgfpathlineto{\pgfqpoint{2.963808in}{2.620728in}}%
\pgfpathlineto{\pgfqpoint{2.950579in}{2.636961in}}%
\pgfpathlineto{\pgfqpoint{2.942451in}{2.633197in}}%
\pgfpathlineto{\pgfqpoint{2.934312in}{2.629573in}}%
\pgfpathlineto{\pgfqpoint{2.926163in}{2.626089in}}%
\pgfpathlineto{\pgfqpoint{2.918004in}{2.622750in}}%
\pgfpathclose%
\pgfusepath{fill}%
\end{pgfscope}%
\begin{pgfscope}%
\pgfpathrectangle{\pgfqpoint{1.254980in}{0.150000in}}{\pgfqpoint{5.490039in}{5.490039in}}%
\pgfusepath{clip}%
\pgfsetbuttcap%
\pgfsetroundjoin%
\definecolor{currentfill}{rgb}{0.185556,0.418570,0.556753}%
\pgfsetfillcolor{currentfill}%
\pgfsetfillopacity{0.700000}%
\pgfsetlinewidth{0.000000pt}%
\definecolor{currentstroke}{rgb}{0.000000,0.000000,0.000000}%
\pgfsetstrokecolor{currentstroke}%
\pgfsetdash{}{0pt}%
\pgfpathmoveto{\pgfqpoint{5.577707in}{2.745310in}}%
\pgfpathlineto{\pgfqpoint{5.591520in}{2.750077in}}%
\pgfpathlineto{\pgfqpoint{5.605347in}{2.754957in}}%
\pgfpathlineto{\pgfqpoint{5.619188in}{2.759948in}}%
\pgfpathlineto{\pgfqpoint{5.633043in}{2.765051in}}%
\pgfpathlineto{\pgfqpoint{5.640137in}{2.773553in}}%
\pgfpathlineto{\pgfqpoint{5.647226in}{2.782018in}}%
\pgfpathlineto{\pgfqpoint{5.654308in}{2.790448in}}%
\pgfpathlineto{\pgfqpoint{5.661385in}{2.798843in}}%
\pgfpathlineto{\pgfqpoint{5.647542in}{2.793867in}}%
\pgfpathlineto{\pgfqpoint{5.633713in}{2.789002in}}%
\pgfpathlineto{\pgfqpoint{5.619898in}{2.784248in}}%
\pgfpathlineto{\pgfqpoint{5.606097in}{2.779606in}}%
\pgfpathlineto{\pgfqpoint{5.599008in}{2.771079in}}%
\pgfpathlineto{\pgfqpoint{5.591914in}{2.762521in}}%
\pgfpathlineto{\pgfqpoint{5.584813in}{2.753932in}}%
\pgfpathlineto{\pgfqpoint{5.577707in}{2.745310in}}%
\pgfpathclose%
\pgfusepath{fill}%
\end{pgfscope}%
\begin{pgfscope}%
\pgfpathrectangle{\pgfqpoint{1.254980in}{0.150000in}}{\pgfqpoint{5.490039in}{5.490039in}}%
\pgfusepath{clip}%
\pgfsetbuttcap%
\pgfsetroundjoin%
\definecolor{currentfill}{rgb}{0.252194,0.269783,0.531579}%
\pgfsetfillcolor{currentfill}%
\pgfsetfillopacity{0.700000}%
\pgfsetlinewidth{0.000000pt}%
\definecolor{currentstroke}{rgb}{0.000000,0.000000,0.000000}%
\pgfsetstrokecolor{currentstroke}%
\pgfsetdash{}{0pt}%
\pgfpathmoveto{\pgfqpoint{3.076850in}{2.436296in}}%
\pgfpathlineto{\pgfqpoint{3.090065in}{2.421863in}}%
\pgfpathlineto{\pgfqpoint{3.103278in}{2.407596in}}%
\pgfpathlineto{\pgfqpoint{3.116487in}{2.393493in}}%
\pgfpathlineto{\pgfqpoint{3.129694in}{2.379554in}}%
\pgfpathlineto{\pgfqpoint{3.137741in}{2.383954in}}%
\pgfpathlineto{\pgfqpoint{3.145778in}{2.388480in}}%
\pgfpathlineto{\pgfqpoint{3.153806in}{2.393131in}}%
\pgfpathlineto{\pgfqpoint{3.161824in}{2.397903in}}%
\pgfpathlineto{\pgfqpoint{3.148643in}{2.411586in}}%
\pgfpathlineto{\pgfqpoint{3.135460in}{2.425433in}}%
\pgfpathlineto{\pgfqpoint{3.122274in}{2.439444in}}%
\pgfpathlineto{\pgfqpoint{3.109086in}{2.453620in}}%
\pgfpathlineto{\pgfqpoint{3.101041in}{2.449097in}}%
\pgfpathlineto{\pgfqpoint{3.092987in}{2.444701in}}%
\pgfpathlineto{\pgfqpoint{3.084924in}{2.440433in}}%
\pgfpathlineto{\pgfqpoint{3.076850in}{2.436296in}}%
\pgfpathclose%
\pgfusepath{fill}%
\end{pgfscope}%
\begin{pgfscope}%
\pgfpathrectangle{\pgfqpoint{1.254980in}{0.150000in}}{\pgfqpoint{5.490039in}{5.490039in}}%
\pgfusepath{clip}%
\pgfsetbuttcap%
\pgfsetroundjoin%
\definecolor{currentfill}{rgb}{0.204903,0.375746,0.553533}%
\pgfsetfillcolor{currentfill}%
\pgfsetfillopacity{0.700000}%
\pgfsetlinewidth{0.000000pt}%
\definecolor{currentstroke}{rgb}{0.000000,0.000000,0.000000}%
\pgfsetstrokecolor{currentstroke}%
\pgfsetdash{}{0pt}%
\pgfpathmoveto{\pgfqpoint{2.864918in}{2.690526in}}%
\pgfpathlineto{\pgfqpoint{2.878197in}{2.673310in}}%
\pgfpathlineto{\pgfqpoint{2.891471in}{2.656276in}}%
\pgfpathlineto{\pgfqpoint{2.904740in}{2.639423in}}%
\pgfpathlineto{\pgfqpoint{2.918004in}{2.622750in}}%
\pgfpathlineto{\pgfqpoint{2.926163in}{2.626089in}}%
\pgfpathlineto{\pgfqpoint{2.934312in}{2.629573in}}%
\pgfpathlineto{\pgfqpoint{2.942451in}{2.633197in}}%
\pgfpathlineto{\pgfqpoint{2.950579in}{2.636961in}}%
\pgfpathlineto{\pgfqpoint{2.937344in}{2.653373in}}%
\pgfpathlineto{\pgfqpoint{2.924106in}{2.669963in}}%
\pgfpathlineto{\pgfqpoint{2.910862in}{2.686734in}}%
\pgfpathlineto{\pgfqpoint{2.897613in}{2.703687in}}%
\pgfpathlineto{\pgfqpoint{2.889456in}{2.700179in}}%
\pgfpathlineto{\pgfqpoint{2.881287in}{2.696815in}}%
\pgfpathlineto{\pgfqpoint{2.873108in}{2.693596in}}%
\pgfpathlineto{\pgfqpoint{2.864918in}{2.690526in}}%
\pgfpathclose%
\pgfusepath{fill}%
\end{pgfscope}%
\begin{pgfscope}%
\pgfpathrectangle{\pgfqpoint{1.254980in}{0.150000in}}{\pgfqpoint{5.490039in}{5.490039in}}%
\pgfusepath{clip}%
\pgfsetbuttcap%
\pgfsetroundjoin%
\definecolor{currentfill}{rgb}{0.262138,0.242286,0.520837}%
\pgfsetfillcolor{currentfill}%
\pgfsetfillopacity{0.700000}%
\pgfsetlinewidth{0.000000pt}%
\definecolor{currentstroke}{rgb}{0.000000,0.000000,0.000000}%
\pgfsetstrokecolor{currentstroke}%
\pgfsetdash{}{0pt}%
\pgfpathmoveto{\pgfqpoint{3.129694in}{2.379554in}}%
\pgfpathlineto{\pgfqpoint{3.142899in}{2.365778in}}%
\pgfpathlineto{\pgfqpoint{3.156102in}{2.352164in}}%
\pgfpathlineto{\pgfqpoint{3.169303in}{2.338710in}}%
\pgfpathlineto{\pgfqpoint{3.182502in}{2.325417in}}%
\pgfpathlineto{\pgfqpoint{3.190522in}{2.330079in}}%
\pgfpathlineto{\pgfqpoint{3.198533in}{2.334862in}}%
\pgfpathlineto{\pgfqpoint{3.206536in}{2.339765in}}%
\pgfpathlineto{\pgfqpoint{3.214529in}{2.344786in}}%
\pgfpathlineto{\pgfqpoint{3.201356in}{2.357825in}}%
\pgfpathlineto{\pgfqpoint{3.188180in}{2.371023in}}%
\pgfpathlineto{\pgfqpoint{3.175003in}{2.384382in}}%
\pgfpathlineto{\pgfqpoint{3.161824in}{2.397903in}}%
\pgfpathlineto{\pgfqpoint{3.153806in}{2.393131in}}%
\pgfpathlineto{\pgfqpoint{3.145778in}{2.388480in}}%
\pgfpathlineto{\pgfqpoint{3.137741in}{2.383954in}}%
\pgfpathlineto{\pgfqpoint{3.129694in}{2.379554in}}%
\pgfpathclose%
\pgfusepath{fill}%
\end{pgfscope}%
\begin{pgfscope}%
\pgfpathrectangle{\pgfqpoint{1.254980in}{0.150000in}}{\pgfqpoint{5.490039in}{5.490039in}}%
\pgfusepath{clip}%
\pgfsetbuttcap%
\pgfsetroundjoin%
\definecolor{currentfill}{rgb}{0.175841,0.441290,0.557685}%
\pgfsetfillcolor{currentfill}%
\pgfsetfillopacity{0.700000}%
\pgfsetlinewidth{0.000000pt}%
\definecolor{currentstroke}{rgb}{0.000000,0.000000,0.000000}%
\pgfsetstrokecolor{currentstroke}%
\pgfsetdash{}{0pt}%
\pgfpathmoveto{\pgfqpoint{5.661385in}{2.798843in}}%
\pgfpathlineto{\pgfqpoint{5.675242in}{2.803931in}}%
\pgfpathlineto{\pgfqpoint{5.689114in}{2.809131in}}%
\pgfpathlineto{\pgfqpoint{5.703000in}{2.814442in}}%
\pgfpathlineto{\pgfqpoint{5.716901in}{2.819865in}}%
\pgfpathlineto{\pgfqpoint{5.723959in}{2.828091in}}%
\pgfpathlineto{\pgfqpoint{5.731011in}{2.836282in}}%
\pgfpathlineto{\pgfqpoint{5.738057in}{2.844440in}}%
\pgfpathlineto{\pgfqpoint{5.745097in}{2.852565in}}%
\pgfpathlineto{\pgfqpoint{5.731209in}{2.847286in}}%
\pgfpathlineto{\pgfqpoint{5.717336in}{2.842117in}}%
\pgfpathlineto{\pgfqpoint{5.703478in}{2.837060in}}%
\pgfpathlineto{\pgfqpoint{5.689633in}{2.832115in}}%
\pgfpathlineto{\pgfqpoint{5.682580in}{2.823841in}}%
\pgfpathlineto{\pgfqpoint{5.675520in}{2.815538in}}%
\pgfpathlineto{\pgfqpoint{5.668455in}{2.807206in}}%
\pgfpathlineto{\pgfqpoint{5.661385in}{2.798843in}}%
\pgfpathclose%
\pgfusepath{fill}%
\end{pgfscope}%
\begin{pgfscope}%
\pgfpathrectangle{\pgfqpoint{1.254980in}{0.150000in}}{\pgfqpoint{5.490039in}{5.490039in}}%
\pgfusepath{clip}%
\pgfsetbuttcap%
\pgfsetroundjoin%
\definecolor{currentfill}{rgb}{0.192357,0.403199,0.555836}%
\pgfsetfillcolor{currentfill}%
\pgfsetfillopacity{0.700000}%
\pgfsetlinewidth{0.000000pt}%
\definecolor{currentstroke}{rgb}{0.000000,0.000000,0.000000}%
\pgfsetstrokecolor{currentstroke}%
\pgfsetdash{}{0pt}%
\pgfpathmoveto{\pgfqpoint{2.811746in}{2.761231in}}%
\pgfpathlineto{\pgfqpoint{2.825048in}{2.743276in}}%
\pgfpathlineto{\pgfqpoint{2.838343in}{2.725507in}}%
\pgfpathlineto{\pgfqpoint{2.851633in}{2.707924in}}%
\pgfpathlineto{\pgfqpoint{2.864918in}{2.690526in}}%
\pgfpathlineto{\pgfqpoint{2.873108in}{2.693596in}}%
\pgfpathlineto{\pgfqpoint{2.881287in}{2.696815in}}%
\pgfpathlineto{\pgfqpoint{2.889456in}{2.700179in}}%
\pgfpathlineto{\pgfqpoint{2.897613in}{2.703687in}}%
\pgfpathlineto{\pgfqpoint{2.884360in}{2.720822in}}%
\pgfpathlineto{\pgfqpoint{2.871101in}{2.738140in}}%
\pgfpathlineto{\pgfqpoint{2.857836in}{2.755644in}}%
\pgfpathlineto{\pgfqpoint{2.844566in}{2.773334in}}%
\pgfpathlineto{\pgfqpoint{2.836378in}{2.770084in}}%
\pgfpathlineto{\pgfqpoint{2.828179in}{2.766982in}}%
\pgfpathlineto{\pgfqpoint{2.819968in}{2.764030in}}%
\pgfpathlineto{\pgfqpoint{2.811746in}{2.761231in}}%
\pgfpathclose%
\pgfusepath{fill}%
\end{pgfscope}%
\begin{pgfscope}%
\pgfpathrectangle{\pgfqpoint{1.254980in}{0.150000in}}{\pgfqpoint{5.490039in}{5.490039in}}%
\pgfusepath{clip}%
\pgfsetbuttcap%
\pgfsetroundjoin%
\definecolor{currentfill}{rgb}{0.280267,0.073417,0.397163}%
\pgfsetfillcolor{currentfill}%
\pgfsetfillopacity{0.700000}%
\pgfsetlinewidth{0.000000pt}%
\definecolor{currentstroke}{rgb}{0.000000,0.000000,0.000000}%
\pgfsetstrokecolor{currentstroke}%
\pgfsetdash{}{0pt}%
\pgfpathmoveto{\pgfqpoint{4.159883in}{2.011050in}}%
\pgfpathlineto{\pgfqpoint{4.173131in}{2.007511in}}%
\pgfpathlineto{\pgfqpoint{4.186387in}{2.004097in}}%
\pgfpathlineto{\pgfqpoint{4.199648in}{2.000808in}}%
\pgfpathlineto{\pgfqpoint{4.212917in}{1.997642in}}%
\pgfpathlineto{\pgfqpoint{4.220511in}{2.007277in}}%
\pgfpathlineto{\pgfqpoint{4.228100in}{2.016933in}}%
\pgfpathlineto{\pgfqpoint{4.235684in}{2.026609in}}%
\pgfpathlineto{\pgfqpoint{4.243264in}{2.036304in}}%
\pgfpathlineto{\pgfqpoint{4.230005in}{2.039321in}}%
\pgfpathlineto{\pgfqpoint{4.216754in}{2.042461in}}%
\pgfpathlineto{\pgfqpoint{4.203509in}{2.045726in}}%
\pgfpathlineto{\pgfqpoint{4.190270in}{2.049115in}}%
\pgfpathlineto{\pgfqpoint{4.182681in}{2.039563in}}%
\pgfpathlineto{\pgfqpoint{4.175087in}{2.030034in}}%
\pgfpathlineto{\pgfqpoint{4.167487in}{2.020529in}}%
\pgfpathlineto{\pgfqpoint{4.159883in}{2.011050in}}%
\pgfpathclose%
\pgfusepath{fill}%
\end{pgfscope}%
\begin{pgfscope}%
\pgfpathrectangle{\pgfqpoint{1.254980in}{0.150000in}}{\pgfqpoint{5.490039in}{5.490039in}}%
\pgfusepath{clip}%
\pgfsetbuttcap%
\pgfsetroundjoin%
\definecolor{currentfill}{rgb}{0.279566,0.067836,0.391917}%
\pgfsetfillcolor{currentfill}%
\pgfsetfillopacity{0.700000}%
\pgfsetlinewidth{0.000000pt}%
\definecolor{currentstroke}{rgb}{0.000000,0.000000,0.000000}%
\pgfsetstrokecolor{currentstroke}%
\pgfsetdash{}{0pt}%
\pgfpathmoveto{\pgfqpoint{3.803690in}{2.008010in}}%
\pgfpathlineto{\pgfqpoint{3.816874in}{2.001359in}}%
\pgfpathlineto{\pgfqpoint{3.830062in}{1.994841in}}%
\pgfpathlineto{\pgfqpoint{3.843254in}{1.988454in}}%
\pgfpathlineto{\pgfqpoint{3.856450in}{1.982199in}}%
\pgfpathlineto{\pgfqpoint{3.864168in}{1.990472in}}%
\pgfpathlineto{\pgfqpoint{3.871881in}{1.998801in}}%
\pgfpathlineto{\pgfqpoint{3.879588in}{2.007185in}}%
\pgfpathlineto{\pgfqpoint{3.887289in}{2.015622in}}%
\pgfpathlineto{\pgfqpoint{3.874107in}{2.021679in}}%
\pgfpathlineto{\pgfqpoint{3.860929in}{2.027868in}}%
\pgfpathlineto{\pgfqpoint{3.847756in}{2.034188in}}%
\pgfpathlineto{\pgfqpoint{3.834587in}{2.040642in}}%
\pgfpathlineto{\pgfqpoint{3.826871in}{2.032396in}}%
\pgfpathlineto{\pgfqpoint{3.819150in}{2.024208in}}%
\pgfpathlineto{\pgfqpoint{3.811423in}{2.016079in}}%
\pgfpathlineto{\pgfqpoint{3.803690in}{2.008010in}}%
\pgfpathclose%
\pgfusepath{fill}%
\end{pgfscope}%
\begin{pgfscope}%
\pgfpathrectangle{\pgfqpoint{1.254980in}{0.150000in}}{\pgfqpoint{5.490039in}{5.490039in}}%
\pgfusepath{clip}%
\pgfsetbuttcap%
\pgfsetroundjoin%
\definecolor{currentfill}{rgb}{0.278012,0.180367,0.486697}%
\pgfsetfillcolor{currentfill}%
\pgfsetfillopacity{0.700000}%
\pgfsetlinewidth{0.000000pt}%
\definecolor{currentstroke}{rgb}{0.000000,0.000000,0.000000}%
\pgfsetstrokecolor{currentstroke}%
\pgfsetdash{}{0pt}%
\pgfpathmoveto{\pgfqpoint{4.713348in}{2.209168in}}%
\pgfpathlineto{\pgfqpoint{4.726775in}{2.209663in}}%
\pgfpathlineto{\pgfqpoint{4.740211in}{2.210276in}}%
\pgfpathlineto{\pgfqpoint{4.753658in}{2.211006in}}%
\pgfpathlineto{\pgfqpoint{4.767114in}{2.211852in}}%
\pgfpathlineto{\pgfqpoint{4.774535in}{2.222201in}}%
\pgfpathlineto{\pgfqpoint{4.781951in}{2.232528in}}%
\pgfpathlineto{\pgfqpoint{4.789362in}{2.242833in}}%
\pgfpathlineto{\pgfqpoint{4.796769in}{2.253115in}}%
\pgfpathlineto{\pgfqpoint{4.783319in}{2.252215in}}%
\pgfpathlineto{\pgfqpoint{4.769880in}{2.251431in}}%
\pgfpathlineto{\pgfqpoint{4.756451in}{2.250764in}}%
\pgfpathlineto{\pgfqpoint{4.743032in}{2.250215in}}%
\pgfpathlineto{\pgfqpoint{4.735618in}{2.239980in}}%
\pgfpathlineto{\pgfqpoint{4.728200in}{2.229727in}}%
\pgfpathlineto{\pgfqpoint{4.720776in}{2.219456in}}%
\pgfpathlineto{\pgfqpoint{4.713348in}{2.209168in}}%
\pgfpathclose%
\pgfusepath{fill}%
\end{pgfscope}%
\begin{pgfscope}%
\pgfpathrectangle{\pgfqpoint{1.254980in}{0.150000in}}{\pgfqpoint{5.490039in}{5.490039in}}%
\pgfusepath{clip}%
\pgfsetbuttcap%
\pgfsetroundjoin%
\definecolor{currentfill}{rgb}{0.273006,0.204520,0.501721}%
\pgfsetfillcolor{currentfill}%
\pgfsetfillopacity{0.700000}%
\pgfsetlinewidth{0.000000pt}%
\definecolor{currentstroke}{rgb}{0.000000,0.000000,0.000000}%
\pgfsetstrokecolor{currentstroke}%
\pgfsetdash{}{0pt}%
\pgfpathmoveto{\pgfqpoint{4.796769in}{2.253115in}}%
\pgfpathlineto{\pgfqpoint{4.810228in}{2.254132in}}%
\pgfpathlineto{\pgfqpoint{4.823698in}{2.255266in}}%
\pgfpathlineto{\pgfqpoint{4.837179in}{2.256516in}}%
\pgfpathlineto{\pgfqpoint{4.850670in}{2.257882in}}%
\pgfpathlineto{\pgfqpoint{4.858064in}{2.268186in}}%
\pgfpathlineto{\pgfqpoint{4.865453in}{2.278465in}}%
\pgfpathlineto{\pgfqpoint{4.872837in}{2.288716in}}%
\pgfpathlineto{\pgfqpoint{4.880216in}{2.298942in}}%
\pgfpathlineto{\pgfqpoint{4.866733in}{2.297538in}}%
\pgfpathlineto{\pgfqpoint{4.853259in}{2.296250in}}%
\pgfpathlineto{\pgfqpoint{4.839796in}{2.295079in}}%
\pgfpathlineto{\pgfqpoint{4.826344in}{2.294023in}}%
\pgfpathlineto{\pgfqpoint{4.818958in}{2.283830in}}%
\pgfpathlineto{\pgfqpoint{4.811566in}{2.273614in}}%
\pgfpathlineto{\pgfqpoint{4.804170in}{2.263376in}}%
\pgfpathlineto{\pgfqpoint{4.796769in}{2.253115in}}%
\pgfpathclose%
\pgfusepath{fill}%
\end{pgfscope}%
\begin{pgfscope}%
\pgfpathrectangle{\pgfqpoint{1.254980in}{0.150000in}}{\pgfqpoint{5.490039in}{5.490039in}}%
\pgfusepath{clip}%
\pgfsetbuttcap%
\pgfsetroundjoin%
\definecolor{currentfill}{rgb}{0.280868,0.160771,0.472899}%
\pgfsetfillcolor{currentfill}%
\pgfsetfillopacity{0.700000}%
\pgfsetlinewidth{0.000000pt}%
\definecolor{currentstroke}{rgb}{0.000000,0.000000,0.000000}%
\pgfsetstrokecolor{currentstroke}%
\pgfsetdash{}{0pt}%
\pgfpathmoveto{\pgfqpoint{4.629948in}{2.167314in}}%
\pgfpathlineto{\pgfqpoint{4.643344in}{2.167269in}}%
\pgfpathlineto{\pgfqpoint{4.656748in}{2.167342in}}%
\pgfpathlineto{\pgfqpoint{4.670163in}{2.167533in}}%
\pgfpathlineto{\pgfqpoint{4.683587in}{2.167841in}}%
\pgfpathlineto{\pgfqpoint{4.691034in}{2.178198in}}%
\pgfpathlineto{\pgfqpoint{4.698477in}{2.188538in}}%
\pgfpathlineto{\pgfqpoint{4.705915in}{2.198862in}}%
\pgfpathlineto{\pgfqpoint{4.713348in}{2.209168in}}%
\pgfpathlineto{\pgfqpoint{4.699932in}{2.208790in}}%
\pgfpathlineto{\pgfqpoint{4.686525in}{2.208529in}}%
\pgfpathlineto{\pgfqpoint{4.673127in}{2.208386in}}%
\pgfpathlineto{\pgfqpoint{4.659739in}{2.208361in}}%
\pgfpathlineto{\pgfqpoint{4.652299in}{2.198119in}}%
\pgfpathlineto{\pgfqpoint{4.644854in}{2.187864in}}%
\pgfpathlineto{\pgfqpoint{4.637403in}{2.177595in}}%
\pgfpathlineto{\pgfqpoint{4.629948in}{2.167314in}}%
\pgfpathclose%
\pgfusepath{fill}%
\end{pgfscope}%
\begin{pgfscope}%
\pgfpathrectangle{\pgfqpoint{1.254980in}{0.150000in}}{\pgfqpoint{5.490039in}{5.490039in}}%
\pgfusepath{clip}%
\pgfsetbuttcap%
\pgfsetroundjoin%
\definecolor{currentfill}{rgb}{0.266580,0.228262,0.514349}%
\pgfsetfillcolor{currentfill}%
\pgfsetfillopacity{0.700000}%
\pgfsetlinewidth{0.000000pt}%
\definecolor{currentstroke}{rgb}{0.000000,0.000000,0.000000}%
\pgfsetstrokecolor{currentstroke}%
\pgfsetdash{}{0pt}%
\pgfpathmoveto{\pgfqpoint{4.880216in}{2.298942in}}%
\pgfpathlineto{\pgfqpoint{4.893711in}{2.300462in}}%
\pgfpathlineto{\pgfqpoint{4.907217in}{2.302097in}}%
\pgfpathlineto{\pgfqpoint{4.920733in}{2.303849in}}%
\pgfpathlineto{\pgfqpoint{4.934260in}{2.305715in}}%
\pgfpathlineto{\pgfqpoint{4.941627in}{2.315943in}}%
\pgfpathlineto{\pgfqpoint{4.948989in}{2.326139in}}%
\pgfpathlineto{\pgfqpoint{4.956346in}{2.336306in}}%
\pgfpathlineto{\pgfqpoint{4.963697in}{2.346443in}}%
\pgfpathlineto{\pgfqpoint{4.950177in}{2.344554in}}%
\pgfpathlineto{\pgfqpoint{4.936668in}{2.342781in}}%
\pgfpathlineto{\pgfqpoint{4.923170in}{2.341123in}}%
\pgfpathlineto{\pgfqpoint{4.909683in}{2.339581in}}%
\pgfpathlineto{\pgfqpoint{4.902324in}{2.329461in}}%
\pgfpathlineto{\pgfqpoint{4.894960in}{2.319314in}}%
\pgfpathlineto{\pgfqpoint{4.887591in}{2.309141in}}%
\pgfpathlineto{\pgfqpoint{4.880216in}{2.298942in}}%
\pgfpathclose%
\pgfusepath{fill}%
\end{pgfscope}%
\begin{pgfscope}%
\pgfpathrectangle{\pgfqpoint{1.254980in}{0.150000in}}{\pgfqpoint{5.490039in}{5.490039in}}%
\pgfusepath{clip}%
\pgfsetbuttcap%
\pgfsetroundjoin%
\definecolor{currentfill}{rgb}{0.269308,0.218818,0.509577}%
\pgfsetfillcolor{currentfill}%
\pgfsetfillopacity{0.700000}%
\pgfsetlinewidth{0.000000pt}%
\definecolor{currentstroke}{rgb}{0.000000,0.000000,0.000000}%
\pgfsetstrokecolor{currentstroke}%
\pgfsetdash{}{0pt}%
\pgfpathmoveto{\pgfqpoint{3.182502in}{2.325417in}}%
\pgfpathlineto{\pgfqpoint{3.195699in}{2.312283in}}%
\pgfpathlineto{\pgfqpoint{3.208894in}{2.299308in}}%
\pgfpathlineto{\pgfqpoint{3.222088in}{2.286491in}}%
\pgfpathlineto{\pgfqpoint{3.235281in}{2.273830in}}%
\pgfpathlineto{\pgfqpoint{3.243276in}{2.278751in}}%
\pgfpathlineto{\pgfqpoint{3.251263in}{2.283790in}}%
\pgfpathlineto{\pgfqpoint{3.259241in}{2.288945in}}%
\pgfpathlineto{\pgfqpoint{3.267210in}{2.294214in}}%
\pgfpathlineto{\pgfqpoint{3.254042in}{2.306621in}}%
\pgfpathlineto{\pgfqpoint{3.240872in}{2.319185in}}%
\pgfpathlineto{\pgfqpoint{3.227702in}{2.331907in}}%
\pgfpathlineto{\pgfqpoint{3.214529in}{2.344786in}}%
\pgfpathlineto{\pgfqpoint{3.206536in}{2.339765in}}%
\pgfpathlineto{\pgfqpoint{3.198533in}{2.334862in}}%
\pgfpathlineto{\pgfqpoint{3.190522in}{2.330079in}}%
\pgfpathlineto{\pgfqpoint{3.182502in}{2.325417in}}%
\pgfpathclose%
\pgfusepath{fill}%
\end{pgfscope}%
\begin{pgfscope}%
\pgfpathrectangle{\pgfqpoint{1.254980in}{0.150000in}}{\pgfqpoint{5.490039in}{5.490039in}}%
\pgfusepath{clip}%
\pgfsetbuttcap%
\pgfsetroundjoin%
\definecolor{currentfill}{rgb}{0.278791,0.062145,0.386592}%
\pgfsetfillcolor{currentfill}%
\pgfsetfillopacity{0.700000}%
\pgfsetlinewidth{0.000000pt}%
\definecolor{currentstroke}{rgb}{0.000000,0.000000,0.000000}%
\pgfsetstrokecolor{currentstroke}%
\pgfsetdash{}{0pt}%
\pgfpathmoveto{\pgfqpoint{3.940061in}{1.992700in}}%
\pgfpathlineto{\pgfqpoint{3.953265in}{1.987294in}}%
\pgfpathlineto{\pgfqpoint{3.966475in}{1.982016in}}%
\pgfpathlineto{\pgfqpoint{3.979689in}{1.976868in}}%
\pgfpathlineto{\pgfqpoint{3.992909in}{1.971847in}}%
\pgfpathlineto{\pgfqpoint{4.000578in}{1.980711in}}%
\pgfpathlineto{\pgfqpoint{4.008242in}{1.989617in}}%
\pgfpathlineto{\pgfqpoint{4.015900in}{1.998565in}}%
\pgfpathlineto{\pgfqpoint{4.023554in}{2.007552in}}%
\pgfpathlineto{\pgfqpoint{4.010346in}{2.012391in}}%
\pgfpathlineto{\pgfqpoint{3.997144in}{2.017359in}}%
\pgfpathlineto{\pgfqpoint{3.983947in}{2.022454in}}%
\pgfpathlineto{\pgfqpoint{3.970755in}{2.027679in}}%
\pgfpathlineto{\pgfqpoint{3.963090in}{2.018867in}}%
\pgfpathlineto{\pgfqpoint{3.955419in}{2.010099in}}%
\pgfpathlineto{\pgfqpoint{3.947743in}{2.001376in}}%
\pgfpathlineto{\pgfqpoint{3.940061in}{1.992700in}}%
\pgfpathclose%
\pgfusepath{fill}%
\end{pgfscope}%
\begin{pgfscope}%
\pgfpathrectangle{\pgfqpoint{1.254980in}{0.150000in}}{\pgfqpoint{5.490039in}{5.490039in}}%
\pgfusepath{clip}%
\pgfsetbuttcap%
\pgfsetroundjoin%
\definecolor{currentfill}{rgb}{0.283229,0.120777,0.440584}%
\pgfsetfillcolor{currentfill}%
\pgfsetfillopacity{0.700000}%
\pgfsetlinewidth{0.000000pt}%
\definecolor{currentstroke}{rgb}{0.000000,0.000000,0.000000}%
\pgfsetstrokecolor{currentstroke}%
\pgfsetdash{}{0pt}%
\pgfpathmoveto{\pgfqpoint{3.477860in}{2.116369in}}%
\pgfpathlineto{\pgfqpoint{3.491029in}{2.106510in}}%
\pgfpathlineto{\pgfqpoint{3.504200in}{2.096794in}}%
\pgfpathlineto{\pgfqpoint{3.517372in}{2.087221in}}%
\pgfpathlineto{\pgfqpoint{3.530545in}{2.077791in}}%
\pgfpathlineto{\pgfqpoint{3.538399in}{2.084373in}}%
\pgfpathlineto{\pgfqpoint{3.546245in}{2.091045in}}%
\pgfpathlineto{\pgfqpoint{3.554085in}{2.097804in}}%
\pgfpathlineto{\pgfqpoint{3.561918in}{2.104650in}}%
\pgfpathlineto{\pgfqpoint{3.548763in}{2.113848in}}%
\pgfpathlineto{\pgfqpoint{3.535611in}{2.123188in}}%
\pgfpathlineto{\pgfqpoint{3.522460in}{2.132671in}}%
\pgfpathlineto{\pgfqpoint{3.509311in}{2.142298in}}%
\pgfpathlineto{\pgfqpoint{3.501459in}{2.135679in}}%
\pgfpathlineto{\pgfqpoint{3.493600in}{2.129150in}}%
\pgfpathlineto{\pgfqpoint{3.485734in}{2.122713in}}%
\pgfpathlineto{\pgfqpoint{3.477860in}{2.116369in}}%
\pgfpathclose%
\pgfusepath{fill}%
\end{pgfscope}%
\begin{pgfscope}%
\pgfpathrectangle{\pgfqpoint{1.254980in}{0.150000in}}{\pgfqpoint{5.490039in}{5.490039in}}%
\pgfusepath{clip}%
\pgfsetbuttcap%
\pgfsetroundjoin%
\definecolor{currentfill}{rgb}{0.258965,0.251537,0.524736}%
\pgfsetfillcolor{currentfill}%
\pgfsetfillopacity{0.700000}%
\pgfsetlinewidth{0.000000pt}%
\definecolor{currentstroke}{rgb}{0.000000,0.000000,0.000000}%
\pgfsetstrokecolor{currentstroke}%
\pgfsetdash{}{0pt}%
\pgfpathmoveto{\pgfqpoint{4.963697in}{2.346443in}}%
\pgfpathlineto{\pgfqpoint{4.977229in}{2.348446in}}%
\pgfpathlineto{\pgfqpoint{4.990771in}{2.350565in}}%
\pgfpathlineto{\pgfqpoint{5.004325in}{2.352800in}}%
\pgfpathlineto{\pgfqpoint{5.017890in}{2.355149in}}%
\pgfpathlineto{\pgfqpoint{5.025230in}{2.365267in}}%
\pgfpathlineto{\pgfqpoint{5.032563in}{2.375352in}}%
\pgfpathlineto{\pgfqpoint{5.039892in}{2.385403in}}%
\pgfpathlineto{\pgfqpoint{5.047215in}{2.395421in}}%
\pgfpathlineto{\pgfqpoint{5.033657in}{2.393066in}}%
\pgfpathlineto{\pgfqpoint{5.020111in}{2.390826in}}%
\pgfpathlineto{\pgfqpoint{5.006576in}{2.388701in}}%
\pgfpathlineto{\pgfqpoint{4.993052in}{2.386692in}}%
\pgfpathlineto{\pgfqpoint{4.985721in}{2.376673in}}%
\pgfpathlineto{\pgfqpoint{4.978385in}{2.366626in}}%
\pgfpathlineto{\pgfqpoint{4.971044in}{2.356549in}}%
\pgfpathlineto{\pgfqpoint{4.963697in}{2.346443in}}%
\pgfpathclose%
\pgfusepath{fill}%
\end{pgfscope}%
\begin{pgfscope}%
\pgfpathrectangle{\pgfqpoint{1.254980in}{0.150000in}}{\pgfqpoint{5.490039in}{5.490039in}}%
\pgfusepath{clip}%
\pgfsetbuttcap%
\pgfsetroundjoin%
\definecolor{currentfill}{rgb}{0.282623,0.140926,0.457517}%
\pgfsetfillcolor{currentfill}%
\pgfsetfillopacity{0.700000}%
\pgfsetlinewidth{0.000000pt}%
\definecolor{currentstroke}{rgb}{0.000000,0.000000,0.000000}%
\pgfsetstrokecolor{currentstroke}%
\pgfsetdash{}{0pt}%
\pgfpathmoveto{\pgfqpoint{4.546561in}{2.127779in}}%
\pgfpathlineto{\pgfqpoint{4.559927in}{2.127174in}}%
\pgfpathlineto{\pgfqpoint{4.573302in}{2.126688in}}%
\pgfpathlineto{\pgfqpoint{4.586687in}{2.126320in}}%
\pgfpathlineto{\pgfqpoint{4.600080in}{2.126071in}}%
\pgfpathlineto{\pgfqpoint{4.607554in}{2.136399in}}%
\pgfpathlineto{\pgfqpoint{4.615024in}{2.146716in}}%
\pgfpathlineto{\pgfqpoint{4.622489in}{2.157021in}}%
\pgfpathlineto{\pgfqpoint{4.629948in}{2.167314in}}%
\pgfpathlineto{\pgfqpoint{4.616562in}{2.167478in}}%
\pgfpathlineto{\pgfqpoint{4.603186in}{2.167759in}}%
\pgfpathlineto{\pgfqpoint{4.589818in}{2.168160in}}%
\pgfpathlineto{\pgfqpoint{4.576459in}{2.168679in}}%
\pgfpathlineto{\pgfqpoint{4.568992in}{2.158465in}}%
\pgfpathlineto{\pgfqpoint{4.561520in}{2.148244in}}%
\pgfpathlineto{\pgfqpoint{4.554043in}{2.138015in}}%
\pgfpathlineto{\pgfqpoint{4.546561in}{2.127779in}}%
\pgfpathclose%
\pgfusepath{fill}%
\end{pgfscope}%
\begin{pgfscope}%
\pgfpathrectangle{\pgfqpoint{1.254980in}{0.150000in}}{\pgfqpoint{5.490039in}{5.490039in}}%
\pgfusepath{clip}%
\pgfsetbuttcap%
\pgfsetroundjoin%
\definecolor{currentfill}{rgb}{0.166617,0.463708,0.558119}%
\pgfsetfillcolor{currentfill}%
\pgfsetfillopacity{0.700000}%
\pgfsetlinewidth{0.000000pt}%
\definecolor{currentstroke}{rgb}{0.000000,0.000000,0.000000}%
\pgfsetstrokecolor{currentstroke}%
\pgfsetdash{}{0pt}%
\pgfpathmoveto{\pgfqpoint{5.745097in}{2.852565in}}%
\pgfpathlineto{\pgfqpoint{5.758999in}{2.857956in}}%
\pgfpathlineto{\pgfqpoint{5.772916in}{2.863457in}}%
\pgfpathlineto{\pgfqpoint{5.786848in}{2.869070in}}%
\pgfpathlineto{\pgfqpoint{5.800795in}{2.874794in}}%
\pgfpathlineto{\pgfqpoint{5.807815in}{2.882735in}}%
\pgfpathlineto{\pgfqpoint{5.814830in}{2.890644in}}%
\pgfpathlineto{\pgfqpoint{5.821838in}{2.898521in}}%
\pgfpathlineto{\pgfqpoint{5.828841in}{2.906368in}}%
\pgfpathlineto{\pgfqpoint{5.814908in}{2.900804in}}%
\pgfpathlineto{\pgfqpoint{5.800990in}{2.895351in}}%
\pgfpathlineto{\pgfqpoint{5.787088in}{2.890009in}}%
\pgfpathlineto{\pgfqpoint{5.773199in}{2.884778in}}%
\pgfpathlineto{\pgfqpoint{5.766182in}{2.876765in}}%
\pgfpathlineto{\pgfqpoint{5.759160in}{2.868726in}}%
\pgfpathlineto{\pgfqpoint{5.752131in}{2.860660in}}%
\pgfpathlineto{\pgfqpoint{5.745097in}{2.852565in}}%
\pgfpathclose%
\pgfusepath{fill}%
\end{pgfscope}%
\begin{pgfscope}%
\pgfpathrectangle{\pgfqpoint{1.254980in}{0.150000in}}{\pgfqpoint{5.490039in}{5.490039in}}%
\pgfusepath{clip}%
\pgfsetbuttcap%
\pgfsetroundjoin%
\definecolor{currentfill}{rgb}{0.280894,0.078907,0.402329}%
\pgfsetfillcolor{currentfill}%
\pgfsetfillopacity{0.700000}%
\pgfsetlinewidth{0.000000pt}%
\definecolor{currentstroke}{rgb}{0.000000,0.000000,0.000000}%
\pgfsetstrokecolor{currentstroke}%
\pgfsetdash{}{0pt}%
\pgfpathmoveto{\pgfqpoint{3.667222in}{2.036118in}}%
\pgfpathlineto{\pgfqpoint{3.680396in}{2.028174in}}%
\pgfpathlineto{\pgfqpoint{3.693573in}{2.020367in}}%
\pgfpathlineto{\pgfqpoint{3.706752in}{2.012696in}}%
\pgfpathlineto{\pgfqpoint{3.719935in}{2.005161in}}%
\pgfpathlineto{\pgfqpoint{3.727708in}{2.012754in}}%
\pgfpathlineto{\pgfqpoint{3.735475in}{2.020417in}}%
\pgfpathlineto{\pgfqpoint{3.743236in}{2.028150in}}%
\pgfpathlineto{\pgfqpoint{3.750990in}{2.035949in}}%
\pgfpathlineto{\pgfqpoint{3.737824in}{2.043270in}}%
\pgfpathlineto{\pgfqpoint{3.724661in}{2.050726in}}%
\pgfpathlineto{\pgfqpoint{3.711501in}{2.058318in}}%
\pgfpathlineto{\pgfqpoint{3.698343in}{2.066047in}}%
\pgfpathlineto{\pgfqpoint{3.690573in}{2.058456in}}%
\pgfpathlineto{\pgfqpoint{3.682796in}{2.050937in}}%
\pgfpathlineto{\pgfqpoint{3.675012in}{2.043490in}}%
\pgfpathlineto{\pgfqpoint{3.667222in}{2.036118in}}%
\pgfpathclose%
\pgfusepath{fill}%
\end{pgfscope}%
\begin{pgfscope}%
\pgfpathrectangle{\pgfqpoint{1.254980in}{0.150000in}}{\pgfqpoint{5.490039in}{5.490039in}}%
\pgfusepath{clip}%
\pgfsetbuttcap%
\pgfsetroundjoin%
\definecolor{currentfill}{rgb}{0.179019,0.433756,0.557430}%
\pgfsetfillcolor{currentfill}%
\pgfsetfillopacity{0.700000}%
\pgfsetlinewidth{0.000000pt}%
\definecolor{currentstroke}{rgb}{0.000000,0.000000,0.000000}%
\pgfsetstrokecolor{currentstroke}%
\pgfsetdash{}{0pt}%
\pgfpathmoveto{\pgfqpoint{2.758478in}{2.834940in}}%
\pgfpathlineto{\pgfqpoint{2.771805in}{2.816227in}}%
\pgfpathlineto{\pgfqpoint{2.785125in}{2.797705in}}%
\pgfpathlineto{\pgfqpoint{2.798439in}{2.779373in}}%
\pgfpathlineto{\pgfqpoint{2.811746in}{2.761231in}}%
\pgfpathlineto{\pgfqpoint{2.819968in}{2.764030in}}%
\pgfpathlineto{\pgfqpoint{2.828179in}{2.766982in}}%
\pgfpathlineto{\pgfqpoint{2.836378in}{2.770084in}}%
\pgfpathlineto{\pgfqpoint{2.844566in}{2.773334in}}%
\pgfpathlineto{\pgfqpoint{2.831291in}{2.791211in}}%
\pgfpathlineto{\pgfqpoint{2.818009in}{2.809276in}}%
\pgfpathlineto{\pgfqpoint{2.804721in}{2.827531in}}%
\pgfpathlineto{\pgfqpoint{2.791427in}{2.845977in}}%
\pgfpathlineto{\pgfqpoint{2.783207in}{2.842987in}}%
\pgfpathlineto{\pgfqpoint{2.774976in}{2.840149in}}%
\pgfpathlineto{\pgfqpoint{2.766733in}{2.837466in}}%
\pgfpathlineto{\pgfqpoint{2.758478in}{2.834940in}}%
\pgfpathclose%
\pgfusepath{fill}%
\end{pgfscope}%
\begin{pgfscope}%
\pgfpathrectangle{\pgfqpoint{1.254980in}{0.150000in}}{\pgfqpoint{5.490039in}{5.490039in}}%
\pgfusepath{clip}%
\pgfsetbuttcap%
\pgfsetroundjoin%
\definecolor{currentfill}{rgb}{0.250425,0.274290,0.533103}%
\pgfsetfillcolor{currentfill}%
\pgfsetfillopacity{0.700000}%
\pgfsetlinewidth{0.000000pt}%
\definecolor{currentstroke}{rgb}{0.000000,0.000000,0.000000}%
\pgfsetstrokecolor{currentstroke}%
\pgfsetdash{}{0pt}%
\pgfpathmoveto{\pgfqpoint{5.047215in}{2.395421in}}%
\pgfpathlineto{\pgfqpoint{5.060785in}{2.397890in}}%
\pgfpathlineto{\pgfqpoint{5.074366in}{2.400474in}}%
\pgfpathlineto{\pgfqpoint{5.087959in}{2.403173in}}%
\pgfpathlineto{\pgfqpoint{5.101564in}{2.405985in}}%
\pgfpathlineto{\pgfqpoint{5.108875in}{2.415966in}}%
\pgfpathlineto{\pgfqpoint{5.116180in}{2.425910in}}%
\pgfpathlineto{\pgfqpoint{5.123480in}{2.435817in}}%
\pgfpathlineto{\pgfqpoint{5.130774in}{2.445689in}}%
\pgfpathlineto{\pgfqpoint{5.117177in}{2.442887in}}%
\pgfpathlineto{\pgfqpoint{5.103591in}{2.440199in}}%
\pgfpathlineto{\pgfqpoint{5.090018in}{2.437626in}}%
\pgfpathlineto{\pgfqpoint{5.076456in}{2.435167in}}%
\pgfpathlineto{\pgfqpoint{5.069154in}{2.425278in}}%
\pgfpathlineto{\pgfqpoint{5.061846in}{2.415358in}}%
\pgfpathlineto{\pgfqpoint{5.054533in}{2.405406in}}%
\pgfpathlineto{\pgfqpoint{5.047215in}{2.395421in}}%
\pgfpathclose%
\pgfusepath{fill}%
\end{pgfscope}%
\begin{pgfscope}%
\pgfpathrectangle{\pgfqpoint{1.254980in}{0.150000in}}{\pgfqpoint{5.490039in}{5.490039in}}%
\pgfusepath{clip}%
\pgfsetbuttcap%
\pgfsetroundjoin%
\definecolor{currentfill}{rgb}{0.283229,0.120777,0.440584}%
\pgfsetfillcolor{currentfill}%
\pgfsetfillopacity{0.700000}%
\pgfsetlinewidth{0.000000pt}%
\definecolor{currentstroke}{rgb}{0.000000,0.000000,0.000000}%
\pgfsetstrokecolor{currentstroke}%
\pgfsetdash{}{0pt}%
\pgfpathmoveto{\pgfqpoint{4.463176in}{2.090795in}}%
\pgfpathlineto{\pgfqpoint{4.476515in}{2.089611in}}%
\pgfpathlineto{\pgfqpoint{4.489864in}{2.088547in}}%
\pgfpathlineto{\pgfqpoint{4.503220in}{2.087601in}}%
\pgfpathlineto{\pgfqpoint{4.516585in}{2.086776in}}%
\pgfpathlineto{\pgfqpoint{4.524086in}{2.097034in}}%
\pgfpathlineto{\pgfqpoint{4.531583in}{2.107288in}}%
\pgfpathlineto{\pgfqpoint{4.539074in}{2.117537in}}%
\pgfpathlineto{\pgfqpoint{4.546561in}{2.127779in}}%
\pgfpathlineto{\pgfqpoint{4.533203in}{2.128503in}}%
\pgfpathlineto{\pgfqpoint{4.519854in}{2.129346in}}%
\pgfpathlineto{\pgfqpoint{4.506514in}{2.130309in}}%
\pgfpathlineto{\pgfqpoint{4.493183in}{2.131392in}}%
\pgfpathlineto{\pgfqpoint{4.485688in}{2.121245in}}%
\pgfpathlineto{\pgfqpoint{4.478189in}{2.111096in}}%
\pgfpathlineto{\pgfqpoint{4.470685in}{2.100946in}}%
\pgfpathlineto{\pgfqpoint{4.463176in}{2.090795in}}%
\pgfpathclose%
\pgfusepath{fill}%
\end{pgfscope}%
\begin{pgfscope}%
\pgfpathrectangle{\pgfqpoint{1.254980in}{0.150000in}}{\pgfqpoint{5.490039in}{5.490039in}}%
\pgfusepath{clip}%
\pgfsetbuttcap%
\pgfsetroundjoin%
\definecolor{currentfill}{rgb}{0.157729,0.485932,0.558013}%
\pgfsetfillcolor{currentfill}%
\pgfsetfillopacity{0.700000}%
\pgfsetlinewidth{0.000000pt}%
\definecolor{currentstroke}{rgb}{0.000000,0.000000,0.000000}%
\pgfsetstrokecolor{currentstroke}%
\pgfsetdash{}{0pt}%
\pgfpathmoveto{\pgfqpoint{5.828841in}{2.906368in}}%
\pgfpathlineto{\pgfqpoint{5.842788in}{2.912043in}}%
\pgfpathlineto{\pgfqpoint{5.856751in}{2.917829in}}%
\pgfpathlineto{\pgfqpoint{5.870728in}{2.923725in}}%
\pgfpathlineto{\pgfqpoint{5.884721in}{2.929733in}}%
\pgfpathlineto{\pgfqpoint{5.891703in}{2.937382in}}%
\pgfpathlineto{\pgfqpoint{5.898679in}{2.945001in}}%
\pgfpathlineto{\pgfqpoint{5.905649in}{2.952592in}}%
\pgfpathlineto{\pgfqpoint{5.912613in}{2.960156in}}%
\pgfpathlineto{\pgfqpoint{5.898635in}{2.954326in}}%
\pgfpathlineto{\pgfqpoint{5.884673in}{2.948606in}}%
\pgfpathlineto{\pgfqpoint{5.870726in}{2.942997in}}%
\pgfpathlineto{\pgfqpoint{5.856793in}{2.937499in}}%
\pgfpathlineto{\pgfqpoint{5.849814in}{2.929752in}}%
\pgfpathlineto{\pgfqpoint{5.842828in}{2.921982in}}%
\pgfpathlineto{\pgfqpoint{5.835837in}{2.914188in}}%
\pgfpathlineto{\pgfqpoint{5.828841in}{2.906368in}}%
\pgfpathclose%
\pgfusepath{fill}%
\end{pgfscope}%
\begin{pgfscope}%
\pgfpathrectangle{\pgfqpoint{1.254980in}{0.150000in}}{\pgfqpoint{5.490039in}{5.490039in}}%
\pgfusepath{clip}%
\pgfsetbuttcap%
\pgfsetroundjoin%
\definecolor{currentfill}{rgb}{0.274128,0.199721,0.498911}%
\pgfsetfillcolor{currentfill}%
\pgfsetfillopacity{0.700000}%
\pgfsetlinewidth{0.000000pt}%
\definecolor{currentstroke}{rgb}{0.000000,0.000000,0.000000}%
\pgfsetstrokecolor{currentstroke}%
\pgfsetdash{}{0pt}%
\pgfpathmoveto{\pgfqpoint{3.235281in}{2.273830in}}%
\pgfpathlineto{\pgfqpoint{3.248472in}{2.261325in}}%
\pgfpathlineto{\pgfqpoint{3.261663in}{2.248976in}}%
\pgfpathlineto{\pgfqpoint{3.274852in}{2.236781in}}%
\pgfpathlineto{\pgfqpoint{3.288041in}{2.224739in}}%
\pgfpathlineto{\pgfqpoint{3.296012in}{2.229920in}}%
\pgfpathlineto{\pgfqpoint{3.303975in}{2.235214in}}%
\pgfpathlineto{\pgfqpoint{3.311929in}{2.240619in}}%
\pgfpathlineto{\pgfqpoint{3.319875in}{2.246134in}}%
\pgfpathlineto{\pgfqpoint{3.306710in}{2.257923in}}%
\pgfpathlineto{\pgfqpoint{3.293544in}{2.269866in}}%
\pgfpathlineto{\pgfqpoint{3.280378in}{2.281962in}}%
\pgfpathlineto{\pgfqpoint{3.267210in}{2.294214in}}%
\pgfpathlineto{\pgfqpoint{3.259241in}{2.288945in}}%
\pgfpathlineto{\pgfqpoint{3.251263in}{2.283790in}}%
\pgfpathlineto{\pgfqpoint{3.243276in}{2.278751in}}%
\pgfpathlineto{\pgfqpoint{3.235281in}{2.273830in}}%
\pgfpathclose%
\pgfusepath{fill}%
\end{pgfscope}%
\begin{pgfscope}%
\pgfpathrectangle{\pgfqpoint{1.254980in}{0.150000in}}{\pgfqpoint{5.490039in}{5.490039in}}%
\pgfusepath{clip}%
\pgfsetbuttcap%
\pgfsetroundjoin%
\definecolor{currentfill}{rgb}{0.241237,0.296485,0.539709}%
\pgfsetfillcolor{currentfill}%
\pgfsetfillopacity{0.700000}%
\pgfsetlinewidth{0.000000pt}%
\definecolor{currentstroke}{rgb}{0.000000,0.000000,0.000000}%
\pgfsetstrokecolor{currentstroke}%
\pgfsetdash{}{0pt}%
\pgfpathmoveto{\pgfqpoint{5.130774in}{2.445689in}}%
\pgfpathlineto{\pgfqpoint{5.144383in}{2.448606in}}%
\pgfpathlineto{\pgfqpoint{5.158005in}{2.451636in}}%
\pgfpathlineto{\pgfqpoint{5.171638in}{2.454781in}}%
\pgfpathlineto{\pgfqpoint{5.185284in}{2.458039in}}%
\pgfpathlineto{\pgfqpoint{5.192565in}{2.467855in}}%
\pgfpathlineto{\pgfqpoint{5.199841in}{2.477632in}}%
\pgfpathlineto{\pgfqpoint{5.207111in}{2.487370in}}%
\pgfpathlineto{\pgfqpoint{5.214376in}{2.497070in}}%
\pgfpathlineto{\pgfqpoint{5.200738in}{2.493839in}}%
\pgfpathlineto{\pgfqpoint{5.187112in}{2.490721in}}%
\pgfpathlineto{\pgfqpoint{5.173499in}{2.487718in}}%
\pgfpathlineto{\pgfqpoint{5.159898in}{2.484828in}}%
\pgfpathlineto{\pgfqpoint{5.152625in}{2.475095in}}%
\pgfpathlineto{\pgfqpoint{5.145347in}{2.465328in}}%
\pgfpathlineto{\pgfqpoint{5.138063in}{2.455526in}}%
\pgfpathlineto{\pgfqpoint{5.130774in}{2.445689in}}%
\pgfpathclose%
\pgfusepath{fill}%
\end{pgfscope}%
\begin{pgfscope}%
\pgfpathrectangle{\pgfqpoint{1.254980in}{0.150000in}}{\pgfqpoint{5.490039in}{5.490039in}}%
\pgfusepath{clip}%
\pgfsetbuttcap%
\pgfsetroundjoin%
\definecolor{currentfill}{rgb}{0.278791,0.062145,0.386592}%
\pgfsetfillcolor{currentfill}%
\pgfsetfillopacity{0.700000}%
\pgfsetlinewidth{0.000000pt}%
\definecolor{currentstroke}{rgb}{0.000000,0.000000,0.000000}%
\pgfsetstrokecolor{currentstroke}%
\pgfsetdash{}{0pt}%
\pgfpathmoveto{\pgfqpoint{4.076437in}{1.989468in}}%
\pgfpathlineto{\pgfqpoint{4.089673in}{1.985263in}}%
\pgfpathlineto{\pgfqpoint{4.102914in}{1.981185in}}%
\pgfpathlineto{\pgfqpoint{4.116161in}{1.977231in}}%
\pgfpathlineto{\pgfqpoint{4.129414in}{1.973403in}}%
\pgfpathlineto{\pgfqpoint{4.137039in}{1.982772in}}%
\pgfpathlineto{\pgfqpoint{4.144659in}{1.992170in}}%
\pgfpathlineto{\pgfqpoint{4.152273in}{2.001596in}}%
\pgfpathlineto{\pgfqpoint{4.159883in}{2.011050in}}%
\pgfpathlineto{\pgfqpoint{4.146640in}{2.014713in}}%
\pgfpathlineto{\pgfqpoint{4.133404in}{2.018501in}}%
\pgfpathlineto{\pgfqpoint{4.120174in}{2.022414in}}%
\pgfpathlineto{\pgfqpoint{4.106950in}{2.026454in}}%
\pgfpathlineto{\pgfqpoint{4.099330in}{2.017159in}}%
\pgfpathlineto{\pgfqpoint{4.091704in}{2.007896in}}%
\pgfpathlineto{\pgfqpoint{4.084073in}{1.998665in}}%
\pgfpathlineto{\pgfqpoint{4.076437in}{1.989468in}}%
\pgfpathclose%
\pgfusepath{fill}%
\end{pgfscope}%
\begin{pgfscope}%
\pgfpathrectangle{\pgfqpoint{1.254980in}{0.150000in}}{\pgfqpoint{5.490039in}{5.490039in}}%
\pgfusepath{clip}%
\pgfsetbuttcap%
\pgfsetroundjoin%
\definecolor{currentfill}{rgb}{0.282910,0.105393,0.426902}%
\pgfsetfillcolor{currentfill}%
\pgfsetfillopacity{0.700000}%
\pgfsetlinewidth{0.000000pt}%
\definecolor{currentstroke}{rgb}{0.000000,0.000000,0.000000}%
\pgfsetstrokecolor{currentstroke}%
\pgfsetdash{}{0pt}%
\pgfpathmoveto{\pgfqpoint{4.379782in}{2.056607in}}%
\pgfpathlineto{\pgfqpoint{4.393098in}{2.054824in}}%
\pgfpathlineto{\pgfqpoint{4.406421in}{2.053161in}}%
\pgfpathlineto{\pgfqpoint{4.419752in}{2.051619in}}%
\pgfpathlineto{\pgfqpoint{4.433092in}{2.050197in}}%
\pgfpathlineto{\pgfqpoint{4.440620in}{2.060344in}}%
\pgfpathlineto{\pgfqpoint{4.448143in}{2.070494in}}%
\pgfpathlineto{\pgfqpoint{4.455662in}{2.080644in}}%
\pgfpathlineto{\pgfqpoint{4.463176in}{2.090795in}}%
\pgfpathlineto{\pgfqpoint{4.449844in}{2.092100in}}%
\pgfpathlineto{\pgfqpoint{4.436521in}{2.093524in}}%
\pgfpathlineto{\pgfqpoint{4.423206in}{2.095069in}}%
\pgfpathlineto{\pgfqpoint{4.409900in}{2.096735in}}%
\pgfpathlineto{\pgfqpoint{4.402377in}{2.086696in}}%
\pgfpathlineto{\pgfqpoint{4.394851in}{2.076661in}}%
\pgfpathlineto{\pgfqpoint{4.387319in}{2.066631in}}%
\pgfpathlineto{\pgfqpoint{4.379782in}{2.056607in}}%
\pgfpathclose%
\pgfusepath{fill}%
\end{pgfscope}%
\begin{pgfscope}%
\pgfpathrectangle{\pgfqpoint{1.254980in}{0.150000in}}{\pgfqpoint{5.490039in}{5.490039in}}%
\pgfusepath{clip}%
\pgfsetbuttcap%
\pgfsetroundjoin%
\definecolor{currentfill}{rgb}{0.143343,0.522773,0.556295}%
\pgfsetfillcolor{currentfill}%
\pgfsetfillopacity{0.700000}%
\pgfsetlinewidth{0.000000pt}%
\definecolor{currentstroke}{rgb}{0.000000,0.000000,0.000000}%
\pgfsetstrokecolor{currentstroke}%
\pgfsetdash{}{0pt}%
\pgfpathmoveto{\pgfqpoint{5.996410in}{3.013840in}}%
\pgfpathlineto{\pgfqpoint{6.010448in}{3.020029in}}%
\pgfpathlineto{\pgfqpoint{6.024503in}{3.026327in}}%
\pgfpathlineto{\pgfqpoint{6.038573in}{3.032736in}}%
\pgfpathlineto{\pgfqpoint{6.045478in}{3.039841in}}%
\pgfpathlineto{\pgfqpoint{6.052378in}{3.046924in}}%
\pgfpathlineto{\pgfqpoint{6.059272in}{3.053987in}}%
\pgfpathlineto{\pgfqpoint{6.066161in}{3.061033in}}%
\pgfpathlineto{\pgfqpoint{6.052108in}{3.054835in}}%
\pgfpathlineto{\pgfqpoint{6.038072in}{3.048747in}}%
\pgfpathlineto{\pgfqpoint{6.024051in}{3.042769in}}%
\pgfpathlineto{\pgfqpoint{6.017149in}{3.035560in}}%
\pgfpathlineto{\pgfqpoint{6.010242in}{3.028337in}}%
\pgfpathlineto{\pgfqpoint{6.003329in}{3.021098in}}%
\pgfpathlineto{\pgfqpoint{5.996410in}{3.013840in}}%
\pgfpathclose%
\pgfusepath{fill}%
\end{pgfscope}%
\begin{pgfscope}%
\pgfpathrectangle{\pgfqpoint{1.254980in}{0.150000in}}{\pgfqpoint{5.490039in}{5.490039in}}%
\pgfusepath{clip}%
\pgfsetbuttcap%
\pgfsetroundjoin%
\definecolor{currentfill}{rgb}{0.166617,0.463708,0.558119}%
\pgfsetfillcolor{currentfill}%
\pgfsetfillopacity{0.700000}%
\pgfsetlinewidth{0.000000pt}%
\definecolor{currentstroke}{rgb}{0.000000,0.000000,0.000000}%
\pgfsetstrokecolor{currentstroke}%
\pgfsetdash{}{0pt}%
\pgfpathmoveto{\pgfqpoint{2.705103in}{2.911733in}}%
\pgfpathlineto{\pgfqpoint{2.718457in}{2.892241in}}%
\pgfpathlineto{\pgfqpoint{2.731805in}{2.872946in}}%
\pgfpathlineto{\pgfqpoint{2.745145in}{2.853846in}}%
\pgfpathlineto{\pgfqpoint{2.758478in}{2.834940in}}%
\pgfpathlineto{\pgfqpoint{2.766733in}{2.837466in}}%
\pgfpathlineto{\pgfqpoint{2.774976in}{2.840149in}}%
\pgfpathlineto{\pgfqpoint{2.783207in}{2.842987in}}%
\pgfpathlineto{\pgfqpoint{2.791427in}{2.845977in}}%
\pgfpathlineto{\pgfqpoint{2.778127in}{2.864616in}}%
\pgfpathlineto{\pgfqpoint{2.764820in}{2.883447in}}%
\pgfpathlineto{\pgfqpoint{2.751506in}{2.902474in}}%
\pgfpathlineto{\pgfqpoint{2.738185in}{2.921697in}}%
\pgfpathlineto{\pgfqpoint{2.729932in}{2.918968in}}%
\pgfpathlineto{\pgfqpoint{2.721668in}{2.916397in}}%
\pgfpathlineto{\pgfqpoint{2.713391in}{2.913984in}}%
\pgfpathlineto{\pgfqpoint{2.705103in}{2.911733in}}%
\pgfpathclose%
\pgfusepath{fill}%
\end{pgfscope}%
\begin{pgfscope}%
\pgfpathrectangle{\pgfqpoint{1.254980in}{0.150000in}}{\pgfqpoint{5.490039in}{5.490039in}}%
\pgfusepath{clip}%
\pgfsetbuttcap%
\pgfsetroundjoin%
\definecolor{currentfill}{rgb}{0.229739,0.322361,0.545706}%
\pgfsetfillcolor{currentfill}%
\pgfsetfillopacity{0.700000}%
\pgfsetlinewidth{0.000000pt}%
\definecolor{currentstroke}{rgb}{0.000000,0.000000,0.000000}%
\pgfsetstrokecolor{currentstroke}%
\pgfsetdash{}{0pt}%
\pgfpathmoveto{\pgfqpoint{5.214376in}{2.497070in}}%
\pgfpathlineto{\pgfqpoint{5.228026in}{2.500415in}}%
\pgfpathlineto{\pgfqpoint{5.241689in}{2.503874in}}%
\pgfpathlineto{\pgfqpoint{5.255364in}{2.507446in}}%
\pgfpathlineto{\pgfqpoint{5.269052in}{2.511132in}}%
\pgfpathlineto{\pgfqpoint{5.276303in}{2.520758in}}%
\pgfpathlineto{\pgfqpoint{5.283548in}{2.530343in}}%
\pgfpathlineto{\pgfqpoint{5.290788in}{2.539889in}}%
\pgfpathlineto{\pgfqpoint{5.298021in}{2.549395in}}%
\pgfpathlineto{\pgfqpoint{5.284342in}{2.545752in}}%
\pgfpathlineto{\pgfqpoint{5.270675in}{2.542224in}}%
\pgfpathlineto{\pgfqpoint{5.257021in}{2.538808in}}%
\pgfpathlineto{\pgfqpoint{5.243379in}{2.535506in}}%
\pgfpathlineto{\pgfqpoint{5.236137in}{2.525951in}}%
\pgfpathlineto{\pgfqpoint{5.228888in}{2.516360in}}%
\pgfpathlineto{\pgfqpoint{5.221635in}{2.506734in}}%
\pgfpathlineto{\pgfqpoint{5.214376in}{2.497070in}}%
\pgfpathclose%
\pgfusepath{fill}%
\end{pgfscope}%
\begin{pgfscope}%
\pgfpathrectangle{\pgfqpoint{1.254980in}{0.150000in}}{\pgfqpoint{5.490039in}{5.490039in}}%
\pgfusepath{clip}%
\pgfsetbuttcap%
\pgfsetroundjoin%
\definecolor{currentfill}{rgb}{0.150476,0.504369,0.557430}%
\pgfsetfillcolor{currentfill}%
\pgfsetfillopacity{0.700000}%
\pgfsetlinewidth{0.000000pt}%
\definecolor{currentstroke}{rgb}{0.000000,0.000000,0.000000}%
\pgfsetstrokecolor{currentstroke}%
\pgfsetdash{}{0pt}%
\pgfpathmoveto{\pgfqpoint{5.912613in}{2.960156in}}%
\pgfpathlineto{\pgfqpoint{5.926606in}{2.966097in}}%
\pgfpathlineto{\pgfqpoint{5.940614in}{2.972148in}}%
\pgfpathlineto{\pgfqpoint{5.954638in}{2.978310in}}%
\pgfpathlineto{\pgfqpoint{5.968677in}{2.984582in}}%
\pgfpathlineto{\pgfqpoint{5.975619in}{2.991935in}}%
\pgfpathlineto{\pgfqpoint{5.982555in}{2.999260in}}%
\pgfpathlineto{\pgfqpoint{5.989486in}{3.006562in}}%
\pgfpathlineto{\pgfqpoint{5.996410in}{3.013840in}}%
\pgfpathlineto{\pgfqpoint{5.982387in}{3.007762in}}%
\pgfpathlineto{\pgfqpoint{5.968380in}{3.001794in}}%
\pgfpathlineto{\pgfqpoint{5.954388in}{2.995936in}}%
\pgfpathlineto{\pgfqpoint{5.940412in}{2.990189in}}%
\pgfpathlineto{\pgfqpoint{5.933470in}{2.982711in}}%
\pgfpathlineto{\pgfqpoint{5.926524in}{2.975214in}}%
\pgfpathlineto{\pgfqpoint{5.919571in}{2.967696in}}%
\pgfpathlineto{\pgfqpoint{5.912613in}{2.960156in}}%
\pgfpathclose%
\pgfusepath{fill}%
\end{pgfscope}%
\begin{pgfscope}%
\pgfpathrectangle{\pgfqpoint{1.254980in}{0.150000in}}{\pgfqpoint{5.490039in}{5.490039in}}%
\pgfusepath{clip}%
\pgfsetbuttcap%
\pgfsetroundjoin%
\definecolor{currentfill}{rgb}{0.282910,0.105393,0.426902}%
\pgfsetfillcolor{currentfill}%
\pgfsetfillopacity{0.700000}%
\pgfsetlinewidth{0.000000pt}%
\definecolor{currentstroke}{rgb}{0.000000,0.000000,0.000000}%
\pgfsetstrokecolor{currentstroke}%
\pgfsetdash{}{0pt}%
\pgfpathmoveto{\pgfqpoint{3.530545in}{2.077791in}}%
\pgfpathlineto{\pgfqpoint{3.543720in}{2.068502in}}%
\pgfpathlineto{\pgfqpoint{3.556896in}{2.059355in}}%
\pgfpathlineto{\pgfqpoint{3.570074in}{2.050349in}}%
\pgfpathlineto{\pgfqpoint{3.583254in}{2.041482in}}%
\pgfpathlineto{\pgfqpoint{3.591089in}{2.048303in}}%
\pgfpathlineto{\pgfqpoint{3.598917in}{2.055208in}}%
\pgfpathlineto{\pgfqpoint{3.606738in}{2.062198in}}%
\pgfpathlineto{\pgfqpoint{3.614552in}{2.069269in}}%
\pgfpathlineto{\pgfqpoint{3.601391in}{2.077903in}}%
\pgfpathlineto{\pgfqpoint{3.588231in}{2.086678in}}%
\pgfpathlineto{\pgfqpoint{3.575073in}{2.095593in}}%
\pgfpathlineto{\pgfqpoint{3.561918in}{2.104650in}}%
\pgfpathlineto{\pgfqpoint{3.554085in}{2.097804in}}%
\pgfpathlineto{\pgfqpoint{3.546245in}{2.091045in}}%
\pgfpathlineto{\pgfqpoint{3.538399in}{2.084373in}}%
\pgfpathlineto{\pgfqpoint{3.530545in}{2.077791in}}%
\pgfpathclose%
\pgfusepath{fill}%
\end{pgfscope}%
\begin{pgfscope}%
\pgfpathrectangle{\pgfqpoint{1.254980in}{0.150000in}}{\pgfqpoint{5.490039in}{5.490039in}}%
\pgfusepath{clip}%
\pgfsetbuttcap%
\pgfsetroundjoin%
\definecolor{currentfill}{rgb}{0.278826,0.175490,0.483397}%
\pgfsetfillcolor{currentfill}%
\pgfsetfillopacity{0.700000}%
\pgfsetlinewidth{0.000000pt}%
\definecolor{currentstroke}{rgb}{0.000000,0.000000,0.000000}%
\pgfsetstrokecolor{currentstroke}%
\pgfsetdash{}{0pt}%
\pgfpathmoveto{\pgfqpoint{3.288041in}{2.224739in}}%
\pgfpathlineto{\pgfqpoint{3.301229in}{2.212851in}}%
\pgfpathlineto{\pgfqpoint{3.314417in}{2.201115in}}%
\pgfpathlineto{\pgfqpoint{3.327604in}{2.189530in}}%
\pgfpathlineto{\pgfqpoint{3.340790in}{2.178097in}}%
\pgfpathlineto{\pgfqpoint{3.348738in}{2.183535in}}%
\pgfpathlineto{\pgfqpoint{3.356678in}{2.189082in}}%
\pgfpathlineto{\pgfqpoint{3.364610in}{2.194737in}}%
\pgfpathlineto{\pgfqpoint{3.372533in}{2.200498in}}%
\pgfpathlineto{\pgfqpoint{3.359369in}{2.211680in}}%
\pgfpathlineto{\pgfqpoint{3.346205in}{2.223014in}}%
\pgfpathlineto{\pgfqpoint{3.333040in}{2.234498in}}%
\pgfpathlineto{\pgfqpoint{3.319875in}{2.246134in}}%
\pgfpathlineto{\pgfqpoint{3.311929in}{2.240619in}}%
\pgfpathlineto{\pgfqpoint{3.303975in}{2.235214in}}%
\pgfpathlineto{\pgfqpoint{3.296012in}{2.229920in}}%
\pgfpathlineto{\pgfqpoint{3.288041in}{2.224739in}}%
\pgfpathclose%
\pgfusepath{fill}%
\end{pgfscope}%
\begin{pgfscope}%
\pgfpathrectangle{\pgfqpoint{1.254980in}{0.150000in}}{\pgfqpoint{5.490039in}{5.490039in}}%
\pgfusepath{clip}%
\pgfsetbuttcap%
\pgfsetroundjoin%
\definecolor{currentfill}{rgb}{0.218130,0.347432,0.550038}%
\pgfsetfillcolor{currentfill}%
\pgfsetfillopacity{0.700000}%
\pgfsetlinewidth{0.000000pt}%
\definecolor{currentstroke}{rgb}{0.000000,0.000000,0.000000}%
\pgfsetstrokecolor{currentstroke}%
\pgfsetdash{}{0pt}%
\pgfpathmoveto{\pgfqpoint{5.298021in}{2.549395in}}%
\pgfpathlineto{\pgfqpoint{5.311714in}{2.553150in}}%
\pgfpathlineto{\pgfqpoint{5.325419in}{2.557019in}}%
\pgfpathlineto{\pgfqpoint{5.339137in}{2.561000in}}%
\pgfpathlineto{\pgfqpoint{5.352869in}{2.565095in}}%
\pgfpathlineto{\pgfqpoint{5.360088in}{2.574509in}}%
\pgfpathlineto{\pgfqpoint{5.367302in}{2.583880in}}%
\pgfpathlineto{\pgfqpoint{5.374510in}{2.593211in}}%
\pgfpathlineto{\pgfqpoint{5.381712in}{2.602502in}}%
\pgfpathlineto{\pgfqpoint{5.367990in}{2.598467in}}%
\pgfpathlineto{\pgfqpoint{5.354280in}{2.594545in}}%
\pgfpathlineto{\pgfqpoint{5.340584in}{2.590736in}}%
\pgfpathlineto{\pgfqpoint{5.326901in}{2.587040in}}%
\pgfpathlineto{\pgfqpoint{5.319690in}{2.577684in}}%
\pgfpathlineto{\pgfqpoint{5.312472in}{2.568292in}}%
\pgfpathlineto{\pgfqpoint{5.305250in}{2.558862in}}%
\pgfpathlineto{\pgfqpoint{5.298021in}{2.549395in}}%
\pgfpathclose%
\pgfusepath{fill}%
\end{pgfscope}%
\begin{pgfscope}%
\pgfpathrectangle{\pgfqpoint{1.254980in}{0.150000in}}{\pgfqpoint{5.490039in}{5.490039in}}%
\pgfusepath{clip}%
\pgfsetbuttcap%
\pgfsetroundjoin%
\definecolor{currentfill}{rgb}{0.281924,0.089666,0.412415}%
\pgfsetfillcolor{currentfill}%
\pgfsetfillopacity{0.700000}%
\pgfsetlinewidth{0.000000pt}%
\definecolor{currentstroke}{rgb}{0.000000,0.000000,0.000000}%
\pgfsetstrokecolor{currentstroke}%
\pgfsetdash{}{0pt}%
\pgfpathmoveto{\pgfqpoint{4.296367in}{2.025468in}}%
\pgfpathlineto{\pgfqpoint{4.309661in}{2.023065in}}%
\pgfpathlineto{\pgfqpoint{4.322963in}{2.020784in}}%
\pgfpathlineto{\pgfqpoint{4.336271in}{2.018625in}}%
\pgfpathlineto{\pgfqpoint{4.349588in}{2.016587in}}%
\pgfpathlineto{\pgfqpoint{4.357144in}{2.026579in}}%
\pgfpathlineto{\pgfqpoint{4.364695in}{2.036580in}}%
\pgfpathlineto{\pgfqpoint{4.372241in}{2.046590in}}%
\pgfpathlineto{\pgfqpoint{4.379782in}{2.056607in}}%
\pgfpathlineto{\pgfqpoint{4.366475in}{2.058512in}}%
\pgfpathlineto{\pgfqpoint{4.353175in}{2.060538in}}%
\pgfpathlineto{\pgfqpoint{4.339882in}{2.062685in}}%
\pgfpathlineto{\pgfqpoint{4.326598in}{2.064954in}}%
\pgfpathlineto{\pgfqpoint{4.319047in}{2.055065in}}%
\pgfpathlineto{\pgfqpoint{4.311492in}{2.045186in}}%
\pgfpathlineto{\pgfqpoint{4.303932in}{2.035321in}}%
\pgfpathlineto{\pgfqpoint{4.296367in}{2.025468in}}%
\pgfpathclose%
\pgfusepath{fill}%
\end{pgfscope}%
\begin{pgfscope}%
\pgfpathrectangle{\pgfqpoint{1.254980in}{0.150000in}}{\pgfqpoint{5.490039in}{5.490039in}}%
\pgfusepath{clip}%
\pgfsetbuttcap%
\pgfsetroundjoin%
\definecolor{currentfill}{rgb}{0.278791,0.062145,0.386592}%
\pgfsetfillcolor{currentfill}%
\pgfsetfillopacity{0.700000}%
\pgfsetlinewidth{0.000000pt}%
\definecolor{currentstroke}{rgb}{0.000000,0.000000,0.000000}%
\pgfsetstrokecolor{currentstroke}%
\pgfsetdash{}{0pt}%
\pgfpathmoveto{\pgfqpoint{3.856450in}{1.982199in}}%
\pgfpathlineto{\pgfqpoint{3.869650in}{1.976076in}}%
\pgfpathlineto{\pgfqpoint{3.882855in}{1.970083in}}%
\pgfpathlineto{\pgfqpoint{3.896063in}{1.964220in}}%
\pgfpathlineto{\pgfqpoint{3.909277in}{1.958487in}}%
\pgfpathlineto{\pgfqpoint{3.916981in}{1.966963in}}%
\pgfpathlineto{\pgfqpoint{3.924680in}{1.975492in}}%
\pgfpathlineto{\pgfqpoint{3.932373in}{1.984071in}}%
\pgfpathlineto{\pgfqpoint{3.940061in}{1.992700in}}%
\pgfpathlineto{\pgfqpoint{3.926861in}{1.998235in}}%
\pgfpathlineto{\pgfqpoint{3.913666in}{2.003900in}}%
\pgfpathlineto{\pgfqpoint{3.900475in}{2.009696in}}%
\pgfpathlineto{\pgfqpoint{3.887289in}{2.015622in}}%
\pgfpathlineto{\pgfqpoint{3.879588in}{2.007185in}}%
\pgfpathlineto{\pgfqpoint{3.871881in}{1.998801in}}%
\pgfpathlineto{\pgfqpoint{3.864168in}{1.990472in}}%
\pgfpathlineto{\pgfqpoint{3.856450in}{1.982199in}}%
\pgfpathclose%
\pgfusepath{fill}%
\end{pgfscope}%
\begin{pgfscope}%
\pgfpathrectangle{\pgfqpoint{1.254980in}{0.150000in}}{\pgfqpoint{5.490039in}{5.490039in}}%
\pgfusepath{clip}%
\pgfsetbuttcap%
\pgfsetroundjoin%
\definecolor{currentfill}{rgb}{0.280267,0.073417,0.397163}%
\pgfsetfillcolor{currentfill}%
\pgfsetfillopacity{0.700000}%
\pgfsetlinewidth{0.000000pt}%
\definecolor{currentstroke}{rgb}{0.000000,0.000000,0.000000}%
\pgfsetstrokecolor{currentstroke}%
\pgfsetdash{}{0pt}%
\pgfpathmoveto{\pgfqpoint{3.719935in}{2.005161in}}%
\pgfpathlineto{\pgfqpoint{3.733121in}{1.997761in}}%
\pgfpathlineto{\pgfqpoint{3.746310in}{1.990495in}}%
\pgfpathlineto{\pgfqpoint{3.759502in}{1.983363in}}%
\pgfpathlineto{\pgfqpoint{3.772698in}{1.976365in}}%
\pgfpathlineto{\pgfqpoint{3.780455in}{1.984179in}}%
\pgfpathlineto{\pgfqpoint{3.788206in}{1.992058in}}%
\pgfpathlineto{\pgfqpoint{3.795951in}{2.000003in}}%
\pgfpathlineto{\pgfqpoint{3.803690in}{2.008010in}}%
\pgfpathlineto{\pgfqpoint{3.790510in}{2.014794in}}%
\pgfpathlineto{\pgfqpoint{3.777333in}{2.021712in}}%
\pgfpathlineto{\pgfqpoint{3.764160in}{2.028763in}}%
\pgfpathlineto{\pgfqpoint{3.750990in}{2.035949in}}%
\pgfpathlineto{\pgfqpoint{3.743236in}{2.028150in}}%
\pgfpathlineto{\pgfqpoint{3.735475in}{2.020417in}}%
\pgfpathlineto{\pgfqpoint{3.727708in}{2.012754in}}%
\pgfpathlineto{\pgfqpoint{3.719935in}{2.005161in}}%
\pgfpathclose%
\pgfusepath{fill}%
\end{pgfscope}%
\begin{pgfscope}%
\pgfpathrectangle{\pgfqpoint{1.254980in}{0.150000in}}{\pgfqpoint{5.490039in}{5.490039in}}%
\pgfusepath{clip}%
\pgfsetbuttcap%
\pgfsetroundjoin%
\definecolor{currentfill}{rgb}{0.153364,0.497000,0.557724}%
\pgfsetfillcolor{currentfill}%
\pgfsetfillopacity{0.700000}%
\pgfsetlinewidth{0.000000pt}%
\definecolor{currentstroke}{rgb}{0.000000,0.000000,0.000000}%
\pgfsetstrokecolor{currentstroke}%
\pgfsetdash{}{0pt}%
\pgfpathmoveto{\pgfqpoint{2.651608in}{2.991694in}}%
\pgfpathlineto{\pgfqpoint{2.664994in}{2.971402in}}%
\pgfpathlineto{\pgfqpoint{2.678371in}{2.951312in}}%
\pgfpathlineto{\pgfqpoint{2.691741in}{2.931423in}}%
\pgfpathlineto{\pgfqpoint{2.705103in}{2.911733in}}%
\pgfpathlineto{\pgfqpoint{2.713391in}{2.913984in}}%
\pgfpathlineto{\pgfqpoint{2.721668in}{2.916397in}}%
\pgfpathlineto{\pgfqpoint{2.729932in}{2.918968in}}%
\pgfpathlineto{\pgfqpoint{2.738185in}{2.921697in}}%
\pgfpathlineto{\pgfqpoint{2.724857in}{2.941117in}}%
\pgfpathlineto{\pgfqpoint{2.711521in}{2.960735in}}%
\pgfpathlineto{\pgfqpoint{2.698178in}{2.980554in}}%
\pgfpathlineto{\pgfqpoint{2.684828in}{3.000575in}}%
\pgfpathlineto{\pgfqpoint{2.676541in}{2.998110in}}%
\pgfpathlineto{\pgfqpoint{2.668243in}{2.995807in}}%
\pgfpathlineto{\pgfqpoint{2.659932in}{2.993668in}}%
\pgfpathlineto{\pgfqpoint{2.651608in}{2.991694in}}%
\pgfpathclose%
\pgfusepath{fill}%
\end{pgfscope}%
\begin{pgfscope}%
\pgfpathrectangle{\pgfqpoint{1.254980in}{0.150000in}}{\pgfqpoint{5.490039in}{5.490039in}}%
\pgfusepath{clip}%
\pgfsetbuttcap%
\pgfsetroundjoin%
\definecolor{currentfill}{rgb}{0.208623,0.367752,0.552675}%
\pgfsetfillcolor{currentfill}%
\pgfsetfillopacity{0.700000}%
\pgfsetlinewidth{0.000000pt}%
\definecolor{currentstroke}{rgb}{0.000000,0.000000,0.000000}%
\pgfsetstrokecolor{currentstroke}%
\pgfsetdash{}{0pt}%
\pgfpathmoveto{\pgfqpoint{5.381712in}{2.602502in}}%
\pgfpathlineto{\pgfqpoint{5.395447in}{2.606649in}}%
\pgfpathlineto{\pgfqpoint{5.409196in}{2.610909in}}%
\pgfpathlineto{\pgfqpoint{5.422958in}{2.615282in}}%
\pgfpathlineto{\pgfqpoint{5.436734in}{2.619768in}}%
\pgfpathlineto{\pgfqpoint{5.443921in}{2.628949in}}%
\pgfpathlineto{\pgfqpoint{5.451102in}{2.638087in}}%
\pgfpathlineto{\pgfqpoint{5.458277in}{2.647184in}}%
\pgfpathlineto{\pgfqpoint{5.465446in}{2.656240in}}%
\pgfpathlineto{\pgfqpoint{5.451681in}{2.651831in}}%
\pgfpathlineto{\pgfqpoint{5.437928in}{2.647535in}}%
\pgfpathlineto{\pgfqpoint{5.424189in}{2.643351in}}%
\pgfpathlineto{\pgfqpoint{5.410464in}{2.639279in}}%
\pgfpathlineto{\pgfqpoint{5.403284in}{2.630141in}}%
\pgfpathlineto{\pgfqpoint{5.396099in}{2.620966in}}%
\pgfpathlineto{\pgfqpoint{5.388908in}{2.611753in}}%
\pgfpathlineto{\pgfqpoint{5.381712in}{2.602502in}}%
\pgfpathclose%
\pgfusepath{fill}%
\end{pgfscope}%
\begin{pgfscope}%
\pgfpathrectangle{\pgfqpoint{1.254980in}{0.150000in}}{\pgfqpoint{5.490039in}{5.490039in}}%
\pgfusepath{clip}%
\pgfsetbuttcap%
\pgfsetroundjoin%
\definecolor{currentfill}{rgb}{0.281412,0.155834,0.469201}%
\pgfsetfillcolor{currentfill}%
\pgfsetfillopacity{0.700000}%
\pgfsetlinewidth{0.000000pt}%
\definecolor{currentstroke}{rgb}{0.000000,0.000000,0.000000}%
\pgfsetstrokecolor{currentstroke}%
\pgfsetdash{}{0pt}%
\pgfpathmoveto{\pgfqpoint{3.340790in}{2.178097in}}%
\pgfpathlineto{\pgfqpoint{3.353977in}{2.166813in}}%
\pgfpathlineto{\pgfqpoint{3.367164in}{2.155678in}}%
\pgfpathlineto{\pgfqpoint{3.380351in}{2.144692in}}%
\pgfpathlineto{\pgfqpoint{3.393537in}{2.133854in}}%
\pgfpathlineto{\pgfqpoint{3.401463in}{2.139549in}}%
\pgfpathlineto{\pgfqpoint{3.409381in}{2.145349in}}%
\pgfpathlineto{\pgfqpoint{3.417290in}{2.151253in}}%
\pgfpathlineto{\pgfqpoint{3.425192in}{2.157258in}}%
\pgfpathlineto{\pgfqpoint{3.412027in}{2.167845in}}%
\pgfpathlineto{\pgfqpoint{3.398862in}{2.178581in}}%
\pgfpathlineto{\pgfqpoint{3.385698in}{2.189465in}}%
\pgfpathlineto{\pgfqpoint{3.372533in}{2.200498in}}%
\pgfpathlineto{\pgfqpoint{3.364610in}{2.194737in}}%
\pgfpathlineto{\pgfqpoint{3.356678in}{2.189082in}}%
\pgfpathlineto{\pgfqpoint{3.348738in}{2.183535in}}%
\pgfpathlineto{\pgfqpoint{3.340790in}{2.178097in}}%
\pgfpathclose%
\pgfusepath{fill}%
\end{pgfscope}%
\begin{pgfscope}%
\pgfpathrectangle{\pgfqpoint{1.254980in}{0.150000in}}{\pgfqpoint{5.490039in}{5.490039in}}%
\pgfusepath{clip}%
\pgfsetbuttcap%
\pgfsetroundjoin%
\definecolor{currentfill}{rgb}{0.277941,0.056324,0.381191}%
\pgfsetfillcolor{currentfill}%
\pgfsetfillopacity{0.700000}%
\pgfsetlinewidth{0.000000pt}%
\definecolor{currentstroke}{rgb}{0.000000,0.000000,0.000000}%
\pgfsetstrokecolor{currentstroke}%
\pgfsetdash{}{0pt}%
\pgfpathmoveto{\pgfqpoint{3.992909in}{1.971847in}}%
\pgfpathlineto{\pgfqpoint{4.006134in}{1.966954in}}%
\pgfpathlineto{\pgfqpoint{4.019364in}{1.962189in}}%
\pgfpathlineto{\pgfqpoint{4.032600in}{1.957550in}}%
\pgfpathlineto{\pgfqpoint{4.045841in}{1.953038in}}%
\pgfpathlineto{\pgfqpoint{4.053498in}{1.962089in}}%
\pgfpathlineto{\pgfqpoint{4.061150in}{1.971179in}}%
\pgfpathlineto{\pgfqpoint{4.068796in}{1.980305in}}%
\pgfpathlineto{\pgfqpoint{4.076437in}{1.989468in}}%
\pgfpathlineto{\pgfqpoint{4.063208in}{1.993799in}}%
\pgfpathlineto{\pgfqpoint{4.049984in}{1.998256in}}%
\pgfpathlineto{\pgfqpoint{4.036766in}{2.002840in}}%
\pgfpathlineto{\pgfqpoint{4.023554in}{2.007552in}}%
\pgfpathlineto{\pgfqpoint{4.015900in}{1.998565in}}%
\pgfpathlineto{\pgfqpoint{4.008242in}{1.989617in}}%
\pgfpathlineto{\pgfqpoint{4.000578in}{1.980711in}}%
\pgfpathlineto{\pgfqpoint{3.992909in}{1.971847in}}%
\pgfpathclose%
\pgfusepath{fill}%
\end{pgfscope}%
\begin{pgfscope}%
\pgfpathrectangle{\pgfqpoint{1.254980in}{0.150000in}}{\pgfqpoint{5.490039in}{5.490039in}}%
\pgfusepath{clip}%
\pgfsetbuttcap%
\pgfsetroundjoin%
\definecolor{currentfill}{rgb}{0.280267,0.073417,0.397163}%
\pgfsetfillcolor{currentfill}%
\pgfsetfillopacity{0.700000}%
\pgfsetlinewidth{0.000000pt}%
\definecolor{currentstroke}{rgb}{0.000000,0.000000,0.000000}%
\pgfsetstrokecolor{currentstroke}%
\pgfsetdash{}{0pt}%
\pgfpathmoveto{\pgfqpoint{4.212917in}{1.997642in}}%
\pgfpathlineto{\pgfqpoint{4.226192in}{1.994599in}}%
\pgfpathlineto{\pgfqpoint{4.239474in}{1.991679in}}%
\pgfpathlineto{\pgfqpoint{4.252763in}{1.988882in}}%
\pgfpathlineto{\pgfqpoint{4.266059in}{1.986208in}}%
\pgfpathlineto{\pgfqpoint{4.273643in}{1.995998in}}%
\pgfpathlineto{\pgfqpoint{4.281223in}{2.005806in}}%
\pgfpathlineto{\pgfqpoint{4.288798in}{2.015629in}}%
\pgfpathlineto{\pgfqpoint{4.296367in}{2.025468in}}%
\pgfpathlineto{\pgfqpoint{4.283081in}{2.027993in}}%
\pgfpathlineto{\pgfqpoint{4.269801in}{2.030640in}}%
\pgfpathlineto{\pgfqpoint{4.256529in}{2.033411in}}%
\pgfpathlineto{\pgfqpoint{4.243264in}{2.036304in}}%
\pgfpathlineto{\pgfqpoint{4.235684in}{2.026609in}}%
\pgfpathlineto{\pgfqpoint{4.228100in}{2.016933in}}%
\pgfpathlineto{\pgfqpoint{4.220511in}{2.007277in}}%
\pgfpathlineto{\pgfqpoint{4.212917in}{1.997642in}}%
\pgfpathclose%
\pgfusepath{fill}%
\end{pgfscope}%
\begin{pgfscope}%
\pgfpathrectangle{\pgfqpoint{1.254980in}{0.150000in}}{\pgfqpoint{5.490039in}{5.490039in}}%
\pgfusepath{clip}%
\pgfsetbuttcap%
\pgfsetroundjoin%
\definecolor{currentfill}{rgb}{0.197636,0.391528,0.554969}%
\pgfsetfillcolor{currentfill}%
\pgfsetfillopacity{0.700000}%
\pgfsetlinewidth{0.000000pt}%
\definecolor{currentstroke}{rgb}{0.000000,0.000000,0.000000}%
\pgfsetstrokecolor{currentstroke}%
\pgfsetdash{}{0pt}%
\pgfpathmoveto{\pgfqpoint{5.465446in}{2.656240in}}%
\pgfpathlineto{\pgfqpoint{5.479226in}{2.660761in}}%
\pgfpathlineto{\pgfqpoint{5.493019in}{2.665395in}}%
\pgfpathlineto{\pgfqpoint{5.506826in}{2.670141in}}%
\pgfpathlineto{\pgfqpoint{5.520647in}{2.675000in}}%
\pgfpathlineto{\pgfqpoint{5.527800in}{2.683929in}}%
\pgfpathlineto{\pgfqpoint{5.534948in}{2.692817in}}%
\pgfpathlineto{\pgfqpoint{5.542089in}{2.701662in}}%
\pgfpathlineto{\pgfqpoint{5.549224in}{2.710468in}}%
\pgfpathlineto{\pgfqpoint{5.535414in}{2.705703in}}%
\pgfpathlineto{\pgfqpoint{5.521617in}{2.701050in}}%
\pgfpathlineto{\pgfqpoint{5.507835in}{2.696509in}}%
\pgfpathlineto{\pgfqpoint{5.494066in}{2.692080in}}%
\pgfpathlineto{\pgfqpoint{5.486920in}{2.683176in}}%
\pgfpathlineto{\pgfqpoint{5.479768in}{2.674235in}}%
\pgfpathlineto{\pgfqpoint{5.472610in}{2.665257in}}%
\pgfpathlineto{\pgfqpoint{5.465446in}{2.656240in}}%
\pgfpathclose%
\pgfusepath{fill}%
\end{pgfscope}%
\begin{pgfscope}%
\pgfpathrectangle{\pgfqpoint{1.254980in}{0.150000in}}{\pgfqpoint{5.490039in}{5.490039in}}%
\pgfusepath{clip}%
\pgfsetbuttcap%
\pgfsetroundjoin%
\definecolor{currentfill}{rgb}{0.281924,0.089666,0.412415}%
\pgfsetfillcolor{currentfill}%
\pgfsetfillopacity{0.700000}%
\pgfsetlinewidth{0.000000pt}%
\definecolor{currentstroke}{rgb}{0.000000,0.000000,0.000000}%
\pgfsetstrokecolor{currentstroke}%
\pgfsetdash{}{0pt}%
\pgfpathmoveto{\pgfqpoint{3.583254in}{2.041482in}}%
\pgfpathlineto{\pgfqpoint{3.596437in}{2.032755in}}%
\pgfpathlineto{\pgfqpoint{3.609621in}{2.024167in}}%
\pgfpathlineto{\pgfqpoint{3.622807in}{2.015718in}}%
\pgfpathlineto{\pgfqpoint{3.635996in}{2.007406in}}%
\pgfpathlineto{\pgfqpoint{3.643813in}{2.014464in}}%
\pgfpathlineto{\pgfqpoint{3.651623in}{2.021603in}}%
\pgfpathlineto{\pgfqpoint{3.659426in}{2.028821in}}%
\pgfpathlineto{\pgfqpoint{3.667222in}{2.036118in}}%
\pgfpathlineto{\pgfqpoint{3.654051in}{2.044198in}}%
\pgfpathlineto{\pgfqpoint{3.640882in}{2.052417in}}%
\pgfpathlineto{\pgfqpoint{3.627716in}{2.060773in}}%
\pgfpathlineto{\pgfqpoint{3.614552in}{2.069269in}}%
\pgfpathlineto{\pgfqpoint{3.606738in}{2.062198in}}%
\pgfpathlineto{\pgfqpoint{3.598917in}{2.055208in}}%
\pgfpathlineto{\pgfqpoint{3.591089in}{2.048303in}}%
\pgfpathlineto{\pgfqpoint{3.583254in}{2.041482in}}%
\pgfpathclose%
\pgfusepath{fill}%
\end{pgfscope}%
\begin{pgfscope}%
\pgfpathrectangle{\pgfqpoint{1.254980in}{0.150000in}}{\pgfqpoint{5.490039in}{5.490039in}}%
\pgfusepath{clip}%
\pgfsetbuttcap%
\pgfsetroundjoin%
\definecolor{currentfill}{rgb}{0.187231,0.414746,0.556547}%
\pgfsetfillcolor{currentfill}%
\pgfsetfillopacity{0.700000}%
\pgfsetlinewidth{0.000000pt}%
\definecolor{currentstroke}{rgb}{0.000000,0.000000,0.000000}%
\pgfsetstrokecolor{currentstroke}%
\pgfsetdash{}{0pt}%
\pgfpathmoveto{\pgfqpoint{5.549224in}{2.710468in}}%
\pgfpathlineto{\pgfqpoint{5.563048in}{2.715345in}}%
\pgfpathlineto{\pgfqpoint{5.576887in}{2.720334in}}%
\pgfpathlineto{\pgfqpoint{5.590739in}{2.725435in}}%
\pgfpathlineto{\pgfqpoint{5.604606in}{2.730648in}}%
\pgfpathlineto{\pgfqpoint{5.611724in}{2.739310in}}%
\pgfpathlineto{\pgfqpoint{5.618837in}{2.747931in}}%
\pgfpathlineto{\pgfqpoint{5.625943in}{2.756511in}}%
\pgfpathlineto{\pgfqpoint{5.633043in}{2.765051in}}%
\pgfpathlineto{\pgfqpoint{5.619188in}{2.759948in}}%
\pgfpathlineto{\pgfqpoint{5.605347in}{2.754957in}}%
\pgfpathlineto{\pgfqpoint{5.591520in}{2.750077in}}%
\pgfpathlineto{\pgfqpoint{5.577707in}{2.745310in}}%
\pgfpathlineto{\pgfqpoint{5.570595in}{2.736654in}}%
\pgfpathlineto{\pgfqpoint{5.563477in}{2.727962in}}%
\pgfpathlineto{\pgfqpoint{5.556354in}{2.719234in}}%
\pgfpathlineto{\pgfqpoint{5.549224in}{2.710468in}}%
\pgfpathclose%
\pgfusepath{fill}%
\end{pgfscope}%
\begin{pgfscope}%
\pgfpathrectangle{\pgfqpoint{1.254980in}{0.150000in}}{\pgfqpoint{5.490039in}{5.490039in}}%
\pgfusepath{clip}%
\pgfsetbuttcap%
\pgfsetroundjoin%
\definecolor{currentfill}{rgb}{0.140536,0.530132,0.555659}%
\pgfsetfillcolor{currentfill}%
\pgfsetfillopacity{0.700000}%
\pgfsetlinewidth{0.000000pt}%
\definecolor{currentstroke}{rgb}{0.000000,0.000000,0.000000}%
\pgfsetstrokecolor{currentstroke}%
\pgfsetdash{}{0pt}%
\pgfpathmoveto{\pgfqpoint{2.597983in}{3.074911in}}%
\pgfpathlineto{\pgfqpoint{2.611403in}{3.053797in}}%
\pgfpathlineto{\pgfqpoint{2.624813in}{3.032890in}}%
\pgfpathlineto{\pgfqpoint{2.638215in}{3.012190in}}%
\pgfpathlineto{\pgfqpoint{2.651608in}{2.991694in}}%
\pgfpathlineto{\pgfqpoint{2.659932in}{2.993668in}}%
\pgfpathlineto{\pgfqpoint{2.668243in}{2.995807in}}%
\pgfpathlineto{\pgfqpoint{2.676541in}{2.998110in}}%
\pgfpathlineto{\pgfqpoint{2.684828in}{3.000575in}}%
\pgfpathlineto{\pgfqpoint{2.671469in}{3.020798in}}%
\pgfpathlineto{\pgfqpoint{2.658102in}{3.041225in}}%
\pgfpathlineto{\pgfqpoint{2.644728in}{3.061859in}}%
\pgfpathlineto{\pgfqpoint{2.631344in}{3.082699in}}%
\pgfpathlineto{\pgfqpoint{2.623023in}{3.080501in}}%
\pgfpathlineto{\pgfqpoint{2.614689in}{3.078468in}}%
\pgfpathlineto{\pgfqpoint{2.606343in}{3.076605in}}%
\pgfpathlineto{\pgfqpoint{2.597983in}{3.074911in}}%
\pgfpathclose%
\pgfusepath{fill}%
\end{pgfscope}%
\begin{pgfscope}%
\pgfpathrectangle{\pgfqpoint{1.254980in}{0.150000in}}{\pgfqpoint{5.490039in}{5.490039in}}%
\pgfusepath{clip}%
\pgfsetbuttcap%
\pgfsetroundjoin%
\definecolor{currentfill}{rgb}{0.282623,0.140926,0.457517}%
\pgfsetfillcolor{currentfill}%
\pgfsetfillopacity{0.700000}%
\pgfsetlinewidth{0.000000pt}%
\definecolor{currentstroke}{rgb}{0.000000,0.000000,0.000000}%
\pgfsetstrokecolor{currentstroke}%
\pgfsetdash{}{0pt}%
\pgfpathmoveto{\pgfqpoint{3.393537in}{2.133854in}}%
\pgfpathlineto{\pgfqpoint{3.406725in}{2.123163in}}%
\pgfpathlineto{\pgfqpoint{3.419913in}{2.112618in}}%
\pgfpathlineto{\pgfqpoint{3.433101in}{2.102220in}}%
\pgfpathlineto{\pgfqpoint{3.446290in}{2.091967in}}%
\pgfpathlineto{\pgfqpoint{3.454194in}{2.097918in}}%
\pgfpathlineto{\pgfqpoint{3.462090in}{2.103970in}}%
\pgfpathlineto{\pgfqpoint{3.469979in}{2.110121in}}%
\pgfpathlineto{\pgfqpoint{3.477860in}{2.116369in}}%
\pgfpathlineto{\pgfqpoint{3.464692in}{2.126373in}}%
\pgfpathlineto{\pgfqpoint{3.451525in}{2.136522in}}%
\pgfpathlineto{\pgfqpoint{3.438358in}{2.146817in}}%
\pgfpathlineto{\pgfqpoint{3.425192in}{2.157258in}}%
\pgfpathlineto{\pgfqpoint{3.417290in}{2.151253in}}%
\pgfpathlineto{\pgfqpoint{3.409381in}{2.145349in}}%
\pgfpathlineto{\pgfqpoint{3.401463in}{2.139549in}}%
\pgfpathlineto{\pgfqpoint{3.393537in}{2.133854in}}%
\pgfpathclose%
\pgfusepath{fill}%
\end{pgfscope}%
\begin{pgfscope}%
\pgfpathrectangle{\pgfqpoint{1.254980in}{0.150000in}}{\pgfqpoint{5.490039in}{5.490039in}}%
\pgfusepath{clip}%
\pgfsetbuttcap%
\pgfsetroundjoin%
\definecolor{currentfill}{rgb}{0.279574,0.170599,0.479997}%
\pgfsetfillcolor{currentfill}%
\pgfsetfillopacity{0.700000}%
\pgfsetlinewidth{0.000000pt}%
\definecolor{currentstroke}{rgb}{0.000000,0.000000,0.000000}%
\pgfsetstrokecolor{currentstroke}%
\pgfsetdash{}{0pt}%
\pgfpathmoveto{\pgfqpoint{4.683587in}{2.167841in}}%
\pgfpathlineto{\pgfqpoint{4.697020in}{2.168267in}}%
\pgfpathlineto{\pgfqpoint{4.710464in}{2.168809in}}%
\pgfpathlineto{\pgfqpoint{4.723917in}{2.169469in}}%
\pgfpathlineto{\pgfqpoint{4.737380in}{2.170246in}}%
\pgfpathlineto{\pgfqpoint{4.744821in}{2.180679in}}%
\pgfpathlineto{\pgfqpoint{4.752257in}{2.191091in}}%
\pgfpathlineto{\pgfqpoint{4.759688in}{2.201482in}}%
\pgfpathlineto{\pgfqpoint{4.767114in}{2.211852in}}%
\pgfpathlineto{\pgfqpoint{4.753658in}{2.211006in}}%
\pgfpathlineto{\pgfqpoint{4.740211in}{2.210276in}}%
\pgfpathlineto{\pgfqpoint{4.726775in}{2.209663in}}%
\pgfpathlineto{\pgfqpoint{4.713348in}{2.209168in}}%
\pgfpathlineto{\pgfqpoint{4.705915in}{2.198862in}}%
\pgfpathlineto{\pgfqpoint{4.698477in}{2.188538in}}%
\pgfpathlineto{\pgfqpoint{4.691034in}{2.178198in}}%
\pgfpathlineto{\pgfqpoint{4.683587in}{2.167841in}}%
\pgfpathclose%
\pgfusepath{fill}%
\end{pgfscope}%
\begin{pgfscope}%
\pgfpathrectangle{\pgfqpoint{1.254980in}{0.150000in}}{\pgfqpoint{5.490039in}{5.490039in}}%
\pgfusepath{clip}%
\pgfsetbuttcap%
\pgfsetroundjoin%
\definecolor{currentfill}{rgb}{0.275191,0.194905,0.496005}%
\pgfsetfillcolor{currentfill}%
\pgfsetfillopacity{0.700000}%
\pgfsetlinewidth{0.000000pt}%
\definecolor{currentstroke}{rgb}{0.000000,0.000000,0.000000}%
\pgfsetstrokecolor{currentstroke}%
\pgfsetdash{}{0pt}%
\pgfpathmoveto{\pgfqpoint{4.767114in}{2.211852in}}%
\pgfpathlineto{\pgfqpoint{4.780581in}{2.212816in}}%
\pgfpathlineto{\pgfqpoint{4.794058in}{2.213895in}}%
\pgfpathlineto{\pgfqpoint{4.807545in}{2.215091in}}%
\pgfpathlineto{\pgfqpoint{4.821043in}{2.216403in}}%
\pgfpathlineto{\pgfqpoint{4.828457in}{2.226812in}}%
\pgfpathlineto{\pgfqpoint{4.835866in}{2.237194in}}%
\pgfpathlineto{\pgfqpoint{4.843271in}{2.247551in}}%
\pgfpathlineto{\pgfqpoint{4.850670in}{2.257882in}}%
\pgfpathlineto{\pgfqpoint{4.837179in}{2.256516in}}%
\pgfpathlineto{\pgfqpoint{4.823698in}{2.255266in}}%
\pgfpathlineto{\pgfqpoint{4.810228in}{2.254132in}}%
\pgfpathlineto{\pgfqpoint{4.796769in}{2.253115in}}%
\pgfpathlineto{\pgfqpoint{4.789362in}{2.242833in}}%
\pgfpathlineto{\pgfqpoint{4.781951in}{2.232528in}}%
\pgfpathlineto{\pgfqpoint{4.774535in}{2.222201in}}%
\pgfpathlineto{\pgfqpoint{4.767114in}{2.211852in}}%
\pgfpathclose%
\pgfusepath{fill}%
\end{pgfscope}%
\begin{pgfscope}%
\pgfpathrectangle{\pgfqpoint{1.254980in}{0.150000in}}{\pgfqpoint{5.490039in}{5.490039in}}%
\pgfusepath{clip}%
\pgfsetbuttcap%
\pgfsetroundjoin%
\definecolor{currentfill}{rgb}{0.281887,0.150881,0.465405}%
\pgfsetfillcolor{currentfill}%
\pgfsetfillopacity{0.700000}%
\pgfsetlinewidth{0.000000pt}%
\definecolor{currentstroke}{rgb}{0.000000,0.000000,0.000000}%
\pgfsetstrokecolor{currentstroke}%
\pgfsetdash{}{0pt}%
\pgfpathmoveto{\pgfqpoint{4.600080in}{2.126071in}}%
\pgfpathlineto{\pgfqpoint{4.613483in}{2.125940in}}%
\pgfpathlineto{\pgfqpoint{4.626895in}{2.125927in}}%
\pgfpathlineto{\pgfqpoint{4.640316in}{2.126032in}}%
\pgfpathlineto{\pgfqpoint{4.653747in}{2.126254in}}%
\pgfpathlineto{\pgfqpoint{4.661214in}{2.136674in}}%
\pgfpathlineto{\pgfqpoint{4.668677in}{2.147079in}}%
\pgfpathlineto{\pgfqpoint{4.676134in}{2.157468in}}%
\pgfpathlineto{\pgfqpoint{4.683587in}{2.167841in}}%
\pgfpathlineto{\pgfqpoint{4.670163in}{2.167533in}}%
\pgfpathlineto{\pgfqpoint{4.656748in}{2.167342in}}%
\pgfpathlineto{\pgfqpoint{4.643344in}{2.167269in}}%
\pgfpathlineto{\pgfqpoint{4.629948in}{2.167314in}}%
\pgfpathlineto{\pgfqpoint{4.622489in}{2.157021in}}%
\pgfpathlineto{\pgfqpoint{4.615024in}{2.146716in}}%
\pgfpathlineto{\pgfqpoint{4.607554in}{2.136399in}}%
\pgfpathlineto{\pgfqpoint{4.600080in}{2.126071in}}%
\pgfpathclose%
\pgfusepath{fill}%
\end{pgfscope}%
\begin{pgfscope}%
\pgfpathrectangle{\pgfqpoint{1.254980in}{0.150000in}}{\pgfqpoint{5.490039in}{5.490039in}}%
\pgfusepath{clip}%
\pgfsetbuttcap%
\pgfsetroundjoin%
\definecolor{currentfill}{rgb}{0.269308,0.218818,0.509577}%
\pgfsetfillcolor{currentfill}%
\pgfsetfillopacity{0.700000}%
\pgfsetlinewidth{0.000000pt}%
\definecolor{currentstroke}{rgb}{0.000000,0.000000,0.000000}%
\pgfsetstrokecolor{currentstroke}%
\pgfsetdash{}{0pt}%
\pgfpathmoveto{\pgfqpoint{4.850670in}{2.257882in}}%
\pgfpathlineto{\pgfqpoint{4.864171in}{2.259364in}}%
\pgfpathlineto{\pgfqpoint{4.877684in}{2.260961in}}%
\pgfpathlineto{\pgfqpoint{4.891207in}{2.262675in}}%
\pgfpathlineto{\pgfqpoint{4.904741in}{2.264504in}}%
\pgfpathlineto{\pgfqpoint{4.912129in}{2.274852in}}%
\pgfpathlineto{\pgfqpoint{4.919511in}{2.285170in}}%
\pgfpathlineto{\pgfqpoint{4.926888in}{2.295458in}}%
\pgfpathlineto{\pgfqpoint{4.934260in}{2.305715in}}%
\pgfpathlineto{\pgfqpoint{4.920733in}{2.303849in}}%
\pgfpathlineto{\pgfqpoint{4.907217in}{2.302097in}}%
\pgfpathlineto{\pgfqpoint{4.893711in}{2.300462in}}%
\pgfpathlineto{\pgfqpoint{4.880216in}{2.298942in}}%
\pgfpathlineto{\pgfqpoint{4.872837in}{2.288716in}}%
\pgfpathlineto{\pgfqpoint{4.865453in}{2.278465in}}%
\pgfpathlineto{\pgfqpoint{4.858064in}{2.268186in}}%
\pgfpathlineto{\pgfqpoint{4.850670in}{2.257882in}}%
\pgfpathclose%
\pgfusepath{fill}%
\end{pgfscope}%
\begin{pgfscope}%
\pgfpathrectangle{\pgfqpoint{1.254980in}{0.150000in}}{\pgfqpoint{5.490039in}{5.490039in}}%
\pgfusepath{clip}%
\pgfsetbuttcap%
\pgfsetroundjoin%
\definecolor{currentfill}{rgb}{0.278791,0.062145,0.386592}%
\pgfsetfillcolor{currentfill}%
\pgfsetfillopacity{0.700000}%
\pgfsetlinewidth{0.000000pt}%
\definecolor{currentstroke}{rgb}{0.000000,0.000000,0.000000}%
\pgfsetstrokecolor{currentstroke}%
\pgfsetdash{}{0pt}%
\pgfpathmoveto{\pgfqpoint{4.129414in}{1.973403in}}%
\pgfpathlineto{\pgfqpoint{4.142673in}{1.969700in}}%
\pgfpathlineto{\pgfqpoint{4.155939in}{1.966121in}}%
\pgfpathlineto{\pgfqpoint{4.169211in}{1.962666in}}%
\pgfpathlineto{\pgfqpoint{4.182490in}{1.959334in}}%
\pgfpathlineto{\pgfqpoint{4.190104in}{1.968874in}}%
\pgfpathlineto{\pgfqpoint{4.197713in}{1.978439in}}%
\pgfpathlineto{\pgfqpoint{4.205317in}{1.988029in}}%
\pgfpathlineto{\pgfqpoint{4.212917in}{1.997642in}}%
\pgfpathlineto{\pgfqpoint{4.199648in}{2.000808in}}%
\pgfpathlineto{\pgfqpoint{4.186387in}{2.004097in}}%
\pgfpathlineto{\pgfqpoint{4.173131in}{2.007511in}}%
\pgfpathlineto{\pgfqpoint{4.159883in}{2.011050in}}%
\pgfpathlineto{\pgfqpoint{4.152273in}{2.001596in}}%
\pgfpathlineto{\pgfqpoint{4.144659in}{1.992170in}}%
\pgfpathlineto{\pgfqpoint{4.137039in}{1.982772in}}%
\pgfpathlineto{\pgfqpoint{4.129414in}{1.973403in}}%
\pgfpathclose%
\pgfusepath{fill}%
\end{pgfscope}%
\begin{pgfscope}%
\pgfpathrectangle{\pgfqpoint{1.254980in}{0.150000in}}{\pgfqpoint{5.490039in}{5.490039in}}%
\pgfusepath{clip}%
\pgfsetbuttcap%
\pgfsetroundjoin%
\definecolor{currentfill}{rgb}{0.177423,0.437527,0.557565}%
\pgfsetfillcolor{currentfill}%
\pgfsetfillopacity{0.700000}%
\pgfsetlinewidth{0.000000pt}%
\definecolor{currentstroke}{rgb}{0.000000,0.000000,0.000000}%
\pgfsetstrokecolor{currentstroke}%
\pgfsetdash{}{0pt}%
\pgfpathmoveto{\pgfqpoint{5.633043in}{2.765051in}}%
\pgfpathlineto{\pgfqpoint{5.646913in}{2.770265in}}%
\pgfpathlineto{\pgfqpoint{5.660797in}{2.775592in}}%
\pgfpathlineto{\pgfqpoint{5.674696in}{2.781030in}}%
\pgfpathlineto{\pgfqpoint{5.688609in}{2.786579in}}%
\pgfpathlineto{\pgfqpoint{5.695691in}{2.794961in}}%
\pgfpathlineto{\pgfqpoint{5.702767in}{2.803301in}}%
\pgfpathlineto{\pgfqpoint{5.709837in}{2.811602in}}%
\pgfpathlineto{\pgfqpoint{5.716901in}{2.819865in}}%
\pgfpathlineto{\pgfqpoint{5.703000in}{2.814442in}}%
\pgfpathlineto{\pgfqpoint{5.689114in}{2.809131in}}%
\pgfpathlineto{\pgfqpoint{5.675242in}{2.803931in}}%
\pgfpathlineto{\pgfqpoint{5.661385in}{2.798843in}}%
\pgfpathlineto{\pgfqpoint{5.654308in}{2.790448in}}%
\pgfpathlineto{\pgfqpoint{5.647226in}{2.782018in}}%
\pgfpathlineto{\pgfqpoint{5.640137in}{2.773553in}}%
\pgfpathlineto{\pgfqpoint{5.633043in}{2.765051in}}%
\pgfpathclose%
\pgfusepath{fill}%
\end{pgfscope}%
\begin{pgfscope}%
\pgfpathrectangle{\pgfqpoint{1.254980in}{0.150000in}}{\pgfqpoint{5.490039in}{5.490039in}}%
\pgfusepath{clip}%
\pgfsetbuttcap%
\pgfsetroundjoin%
\definecolor{currentfill}{rgb}{0.283072,0.130895,0.449241}%
\pgfsetfillcolor{currentfill}%
\pgfsetfillopacity{0.700000}%
\pgfsetlinewidth{0.000000pt}%
\definecolor{currentstroke}{rgb}{0.000000,0.000000,0.000000}%
\pgfsetstrokecolor{currentstroke}%
\pgfsetdash{}{0pt}%
\pgfpathmoveto{\pgfqpoint{4.516585in}{2.086776in}}%
\pgfpathlineto{\pgfqpoint{4.529959in}{2.086069in}}%
\pgfpathlineto{\pgfqpoint{4.543342in}{2.085481in}}%
\pgfpathlineto{\pgfqpoint{4.556734in}{2.085012in}}%
\pgfpathlineto{\pgfqpoint{4.570135in}{2.084661in}}%
\pgfpathlineto{\pgfqpoint{4.577628in}{2.095027in}}%
\pgfpathlineto{\pgfqpoint{4.585117in}{2.105385in}}%
\pgfpathlineto{\pgfqpoint{4.592601in}{2.115733in}}%
\pgfpathlineto{\pgfqpoint{4.600080in}{2.126071in}}%
\pgfpathlineto{\pgfqpoint{4.586687in}{2.126320in}}%
\pgfpathlineto{\pgfqpoint{4.573302in}{2.126688in}}%
\pgfpathlineto{\pgfqpoint{4.559927in}{2.127174in}}%
\pgfpathlineto{\pgfqpoint{4.546561in}{2.127779in}}%
\pgfpathlineto{\pgfqpoint{4.539074in}{2.117537in}}%
\pgfpathlineto{\pgfqpoint{4.531583in}{2.107288in}}%
\pgfpathlineto{\pgfqpoint{4.524086in}{2.097034in}}%
\pgfpathlineto{\pgfqpoint{4.516585in}{2.086776in}}%
\pgfpathclose%
\pgfusepath{fill}%
\end{pgfscope}%
\begin{pgfscope}%
\pgfpathrectangle{\pgfqpoint{1.254980in}{0.150000in}}{\pgfqpoint{5.490039in}{5.490039in}}%
\pgfusepath{clip}%
\pgfsetbuttcap%
\pgfsetroundjoin%
\definecolor{currentfill}{rgb}{0.262138,0.242286,0.520837}%
\pgfsetfillcolor{currentfill}%
\pgfsetfillopacity{0.700000}%
\pgfsetlinewidth{0.000000pt}%
\definecolor{currentstroke}{rgb}{0.000000,0.000000,0.000000}%
\pgfsetstrokecolor{currentstroke}%
\pgfsetdash{}{0pt}%
\pgfpathmoveto{\pgfqpoint{4.934260in}{2.305715in}}%
\pgfpathlineto{\pgfqpoint{4.947799in}{2.307698in}}%
\pgfpathlineto{\pgfqpoint{4.961349in}{2.309795in}}%
\pgfpathlineto{\pgfqpoint{4.974910in}{2.312007in}}%
\pgfpathlineto{\pgfqpoint{4.988482in}{2.314334in}}%
\pgfpathlineto{\pgfqpoint{4.995842in}{2.324589in}}%
\pgfpathlineto{\pgfqpoint{5.003197in}{2.334810in}}%
\pgfpathlineto{\pgfqpoint{5.010546in}{2.344996in}}%
\pgfpathlineto{\pgfqpoint{5.017890in}{2.355149in}}%
\pgfpathlineto{\pgfqpoint{5.004325in}{2.352800in}}%
\pgfpathlineto{\pgfqpoint{4.990771in}{2.350565in}}%
\pgfpathlineto{\pgfqpoint{4.977229in}{2.348446in}}%
\pgfpathlineto{\pgfqpoint{4.963697in}{2.346443in}}%
\pgfpathlineto{\pgfqpoint{4.956346in}{2.336306in}}%
\pgfpathlineto{\pgfqpoint{4.948989in}{2.326139in}}%
\pgfpathlineto{\pgfqpoint{4.941627in}{2.315943in}}%
\pgfpathlineto{\pgfqpoint{4.934260in}{2.305715in}}%
\pgfpathclose%
\pgfusepath{fill}%
\end{pgfscope}%
\begin{pgfscope}%
\pgfpathrectangle{\pgfqpoint{1.254980in}{0.150000in}}{\pgfqpoint{5.490039in}{5.490039in}}%
\pgfusepath{clip}%
\pgfsetbuttcap%
\pgfsetroundjoin%
\definecolor{currentfill}{rgb}{0.278791,0.062145,0.386592}%
\pgfsetfillcolor{currentfill}%
\pgfsetfillopacity{0.700000}%
\pgfsetlinewidth{0.000000pt}%
\definecolor{currentstroke}{rgb}{0.000000,0.000000,0.000000}%
\pgfsetstrokecolor{currentstroke}%
\pgfsetdash{}{0pt}%
\pgfpathmoveto{\pgfqpoint{3.772698in}{1.976365in}}%
\pgfpathlineto{\pgfqpoint{3.785897in}{1.969500in}}%
\pgfpathlineto{\pgfqpoint{3.799100in}{1.962768in}}%
\pgfpathlineto{\pgfqpoint{3.812307in}{1.956168in}}%
\pgfpathlineto{\pgfqpoint{3.825517in}{1.949699in}}%
\pgfpathlineto{\pgfqpoint{3.833260in}{1.957732in}}%
\pgfpathlineto{\pgfqpoint{3.840996in}{1.965828in}}%
\pgfpathlineto{\pgfqpoint{3.848726in}{1.973984in}}%
\pgfpathlineto{\pgfqpoint{3.856450in}{1.982199in}}%
\pgfpathlineto{\pgfqpoint{3.843254in}{1.988454in}}%
\pgfpathlineto{\pgfqpoint{3.830062in}{1.994841in}}%
\pgfpathlineto{\pgfqpoint{3.816874in}{2.001359in}}%
\pgfpathlineto{\pgfqpoint{3.803690in}{2.008010in}}%
\pgfpathlineto{\pgfqpoint{3.795951in}{2.000003in}}%
\pgfpathlineto{\pgfqpoint{3.788206in}{1.992058in}}%
\pgfpathlineto{\pgfqpoint{3.780455in}{1.984179in}}%
\pgfpathlineto{\pgfqpoint{3.772698in}{1.976365in}}%
\pgfpathclose%
\pgfusepath{fill}%
\end{pgfscope}%
\begin{pgfscope}%
\pgfpathrectangle{\pgfqpoint{1.254980in}{0.150000in}}{\pgfqpoint{5.490039in}{5.490039in}}%
\pgfusepath{clip}%
\pgfsetbuttcap%
\pgfsetroundjoin%
\definecolor{currentfill}{rgb}{0.239346,0.300855,0.540844}%
\pgfsetfillcolor{currentfill}%
\pgfsetfillopacity{0.700000}%
\pgfsetlinewidth{0.000000pt}%
\definecolor{currentstroke}{rgb}{0.000000,0.000000,0.000000}%
\pgfsetstrokecolor{currentstroke}%
\pgfsetdash{}{0pt}%
\pgfpathmoveto{\pgfqpoint{2.991454in}{2.481598in}}%
\pgfpathlineto{\pgfqpoint{3.004710in}{2.466218in}}%
\pgfpathlineto{\pgfqpoint{3.017963in}{2.451008in}}%
\pgfpathlineto{\pgfqpoint{3.031212in}{2.435967in}}%
\pgfpathlineto{\pgfqpoint{3.044458in}{2.421093in}}%
\pgfpathlineto{\pgfqpoint{3.052571in}{2.424688in}}%
\pgfpathlineto{\pgfqpoint{3.060674in}{2.428421in}}%
\pgfpathlineto{\pgfqpoint{3.068767in}{2.432291in}}%
\pgfpathlineto{\pgfqpoint{3.076850in}{2.436296in}}%
\pgfpathlineto{\pgfqpoint{3.063633in}{2.450895in}}%
\pgfpathlineto{\pgfqpoint{3.050412in}{2.465662in}}%
\pgfpathlineto{\pgfqpoint{3.037188in}{2.480597in}}%
\pgfpathlineto{\pgfqpoint{3.023960in}{2.495701in}}%
\pgfpathlineto{\pgfqpoint{3.015849in}{2.491965in}}%
\pgfpathlineto{\pgfqpoint{3.007728in}{2.488368in}}%
\pgfpathlineto{\pgfqpoint{2.999596in}{2.484911in}}%
\pgfpathlineto{\pgfqpoint{2.991454in}{2.481598in}}%
\pgfpathclose%
\pgfusepath{fill}%
\end{pgfscope}%
\begin{pgfscope}%
\pgfpathrectangle{\pgfqpoint{1.254980in}{0.150000in}}{\pgfqpoint{5.490039in}{5.490039in}}%
\pgfusepath{clip}%
\pgfsetbuttcap%
\pgfsetroundjoin%
\definecolor{currentfill}{rgb}{0.277941,0.056324,0.381191}%
\pgfsetfillcolor{currentfill}%
\pgfsetfillopacity{0.700000}%
\pgfsetlinewidth{0.000000pt}%
\definecolor{currentstroke}{rgb}{0.000000,0.000000,0.000000}%
\pgfsetstrokecolor{currentstroke}%
\pgfsetdash{}{0pt}%
\pgfpathmoveto{\pgfqpoint{3.909277in}{1.958487in}}%
\pgfpathlineto{\pgfqpoint{3.922495in}{1.952883in}}%
\pgfpathlineto{\pgfqpoint{3.935717in}{1.947409in}}%
\pgfpathlineto{\pgfqpoint{3.948945in}{1.942063in}}%
\pgfpathlineto{\pgfqpoint{3.962177in}{1.936845in}}%
\pgfpathlineto{\pgfqpoint{3.969869in}{1.945525in}}%
\pgfpathlineto{\pgfqpoint{3.977554in}{1.954253in}}%
\pgfpathlineto{\pgfqpoint{3.985234in}{1.963027in}}%
\pgfpathlineto{\pgfqpoint{3.992909in}{1.971847in}}%
\pgfpathlineto{\pgfqpoint{3.979689in}{1.976868in}}%
\pgfpathlineto{\pgfqpoint{3.966475in}{1.982016in}}%
\pgfpathlineto{\pgfqpoint{3.953265in}{1.987294in}}%
\pgfpathlineto{\pgfqpoint{3.940061in}{1.992700in}}%
\pgfpathlineto{\pgfqpoint{3.932373in}{1.984071in}}%
\pgfpathlineto{\pgfqpoint{3.924680in}{1.975492in}}%
\pgfpathlineto{\pgfqpoint{3.916981in}{1.966963in}}%
\pgfpathlineto{\pgfqpoint{3.909277in}{1.958487in}}%
\pgfpathclose%
\pgfusepath{fill}%
\end{pgfscope}%
\begin{pgfscope}%
\pgfpathrectangle{\pgfqpoint{1.254980in}{0.150000in}}{\pgfqpoint{5.490039in}{5.490039in}}%
\pgfusepath{clip}%
\pgfsetbuttcap%
\pgfsetroundjoin%
\definecolor{currentfill}{rgb}{0.227802,0.326594,0.546532}%
\pgfsetfillcolor{currentfill}%
\pgfsetfillopacity{0.700000}%
\pgfsetlinewidth{0.000000pt}%
\definecolor{currentstroke}{rgb}{0.000000,0.000000,0.000000}%
\pgfsetstrokecolor{currentstroke}%
\pgfsetdash{}{0pt}%
\pgfpathmoveto{\pgfqpoint{2.938390in}{2.544834in}}%
\pgfpathlineto{\pgfqpoint{2.951662in}{2.528766in}}%
\pgfpathlineto{\pgfqpoint{2.964930in}{2.512871in}}%
\pgfpathlineto{\pgfqpoint{2.978194in}{2.497148in}}%
\pgfpathlineto{\pgfqpoint{2.991454in}{2.481598in}}%
\pgfpathlineto{\pgfqpoint{2.999596in}{2.484911in}}%
\pgfpathlineto{\pgfqpoint{3.007728in}{2.488368in}}%
\pgfpathlineto{\pgfqpoint{3.015849in}{2.491965in}}%
\pgfpathlineto{\pgfqpoint{3.023960in}{2.495701in}}%
\pgfpathlineto{\pgfqpoint{3.010729in}{2.510976in}}%
\pgfpathlineto{\pgfqpoint{2.997495in}{2.526421in}}%
\pgfpathlineto{\pgfqpoint{2.984257in}{2.542040in}}%
\pgfpathlineto{\pgfqpoint{2.971015in}{2.557831in}}%
\pgfpathlineto{\pgfqpoint{2.962874in}{2.554365in}}%
\pgfpathlineto{\pgfqpoint{2.954724in}{2.551042in}}%
\pgfpathlineto{\pgfqpoint{2.946562in}{2.547864in}}%
\pgfpathlineto{\pgfqpoint{2.938390in}{2.544834in}}%
\pgfpathclose%
\pgfusepath{fill}%
\end{pgfscope}%
\begin{pgfscope}%
\pgfpathrectangle{\pgfqpoint{1.254980in}{0.150000in}}{\pgfqpoint{5.490039in}{5.490039in}}%
\pgfusepath{clip}%
\pgfsetbuttcap%
\pgfsetroundjoin%
\definecolor{currentfill}{rgb}{0.253935,0.265254,0.529983}%
\pgfsetfillcolor{currentfill}%
\pgfsetfillopacity{0.700000}%
\pgfsetlinewidth{0.000000pt}%
\definecolor{currentstroke}{rgb}{0.000000,0.000000,0.000000}%
\pgfsetstrokecolor{currentstroke}%
\pgfsetdash{}{0pt}%
\pgfpathmoveto{\pgfqpoint{5.017890in}{2.355149in}}%
\pgfpathlineto{\pgfqpoint{5.031467in}{2.357612in}}%
\pgfpathlineto{\pgfqpoint{5.045056in}{2.360191in}}%
\pgfpathlineto{\pgfqpoint{5.058657in}{2.362884in}}%
\pgfpathlineto{\pgfqpoint{5.072269in}{2.365691in}}%
\pgfpathlineto{\pgfqpoint{5.079601in}{2.375821in}}%
\pgfpathlineto{\pgfqpoint{5.086927in}{2.385913in}}%
\pgfpathlineto{\pgfqpoint{5.094248in}{2.395968in}}%
\pgfpathlineto{\pgfqpoint{5.101564in}{2.405985in}}%
\pgfpathlineto{\pgfqpoint{5.087959in}{2.403173in}}%
\pgfpathlineto{\pgfqpoint{5.074366in}{2.400474in}}%
\pgfpathlineto{\pgfqpoint{5.060785in}{2.397890in}}%
\pgfpathlineto{\pgfqpoint{5.047215in}{2.395421in}}%
\pgfpathlineto{\pgfqpoint{5.039892in}{2.385403in}}%
\pgfpathlineto{\pgfqpoint{5.032563in}{2.375352in}}%
\pgfpathlineto{\pgfqpoint{5.025230in}{2.365267in}}%
\pgfpathlineto{\pgfqpoint{5.017890in}{2.355149in}}%
\pgfpathclose%
\pgfusepath{fill}%
\end{pgfscope}%
\begin{pgfscope}%
\pgfpathrectangle{\pgfqpoint{1.254980in}{0.150000in}}{\pgfqpoint{5.490039in}{5.490039in}}%
\pgfusepath{clip}%
\pgfsetbuttcap%
\pgfsetroundjoin%
\definecolor{currentfill}{rgb}{0.283091,0.110553,0.431554}%
\pgfsetfillcolor{currentfill}%
\pgfsetfillopacity{0.700000}%
\pgfsetlinewidth{0.000000pt}%
\definecolor{currentstroke}{rgb}{0.000000,0.000000,0.000000}%
\pgfsetstrokecolor{currentstroke}%
\pgfsetdash{}{0pt}%
\pgfpathmoveto{\pgfqpoint{4.433092in}{2.050197in}}%
\pgfpathlineto{\pgfqpoint{4.446440in}{2.048895in}}%
\pgfpathlineto{\pgfqpoint{4.459796in}{2.047713in}}%
\pgfpathlineto{\pgfqpoint{4.473160in}{2.046650in}}%
\pgfpathlineto{\pgfqpoint{4.486533in}{2.045707in}}%
\pgfpathlineto{\pgfqpoint{4.494054in}{2.055978in}}%
\pgfpathlineto{\pgfqpoint{4.501569in}{2.066247in}}%
\pgfpathlineto{\pgfqpoint{4.509080in}{2.076513in}}%
\pgfpathlineto{\pgfqpoint{4.516585in}{2.086776in}}%
\pgfpathlineto{\pgfqpoint{4.503220in}{2.087601in}}%
\pgfpathlineto{\pgfqpoint{4.489864in}{2.088547in}}%
\pgfpathlineto{\pgfqpoint{4.476515in}{2.089611in}}%
\pgfpathlineto{\pgfqpoint{4.463176in}{2.090795in}}%
\pgfpathlineto{\pgfqpoint{4.455662in}{2.080644in}}%
\pgfpathlineto{\pgfqpoint{4.448143in}{2.070494in}}%
\pgfpathlineto{\pgfqpoint{4.440620in}{2.060344in}}%
\pgfpathlineto{\pgfqpoint{4.433092in}{2.050197in}}%
\pgfpathclose%
\pgfusepath{fill}%
\end{pgfscope}%
\begin{pgfscope}%
\pgfpathrectangle{\pgfqpoint{1.254980in}{0.150000in}}{\pgfqpoint{5.490039in}{5.490039in}}%
\pgfusepath{clip}%
\pgfsetbuttcap%
\pgfsetroundjoin%
\definecolor{currentfill}{rgb}{0.250425,0.274290,0.533103}%
\pgfsetfillcolor{currentfill}%
\pgfsetfillopacity{0.700000}%
\pgfsetlinewidth{0.000000pt}%
\definecolor{currentstroke}{rgb}{0.000000,0.000000,0.000000}%
\pgfsetstrokecolor{currentstroke}%
\pgfsetdash{}{0pt}%
\pgfpathmoveto{\pgfqpoint{3.044458in}{2.421093in}}%
\pgfpathlineto{\pgfqpoint{3.057701in}{2.406386in}}%
\pgfpathlineto{\pgfqpoint{3.070941in}{2.391845in}}%
\pgfpathlineto{\pgfqpoint{3.084179in}{2.377469in}}%
\pgfpathlineto{\pgfqpoint{3.097413in}{2.363257in}}%
\pgfpathlineto{\pgfqpoint{3.105498in}{2.367132in}}%
\pgfpathlineto{\pgfqpoint{3.113573in}{2.371141in}}%
\pgfpathlineto{\pgfqpoint{3.121639in}{2.375282in}}%
\pgfpathlineto{\pgfqpoint{3.129694in}{2.379554in}}%
\pgfpathlineto{\pgfqpoint{3.116487in}{2.393493in}}%
\pgfpathlineto{\pgfqpoint{3.103278in}{2.407596in}}%
\pgfpathlineto{\pgfqpoint{3.090065in}{2.421863in}}%
\pgfpathlineto{\pgfqpoint{3.076850in}{2.436296in}}%
\pgfpathlineto{\pgfqpoint{3.068767in}{2.432291in}}%
\pgfpathlineto{\pgfqpoint{3.060674in}{2.428421in}}%
\pgfpathlineto{\pgfqpoint{3.052571in}{2.424688in}}%
\pgfpathlineto{\pgfqpoint{3.044458in}{2.421093in}}%
\pgfpathclose%
\pgfusepath{fill}%
\end{pgfscope}%
\begin{pgfscope}%
\pgfpathrectangle{\pgfqpoint{1.254980in}{0.150000in}}{\pgfqpoint{5.490039in}{5.490039in}}%
\pgfusepath{clip}%
\pgfsetbuttcap%
\pgfsetroundjoin%
\definecolor{currentfill}{rgb}{0.214298,0.355619,0.551184}%
\pgfsetfillcolor{currentfill}%
\pgfsetfillopacity{0.700000}%
\pgfsetlinewidth{0.000000pt}%
\definecolor{currentstroke}{rgb}{0.000000,0.000000,0.000000}%
\pgfsetstrokecolor{currentstroke}%
\pgfsetdash{}{0pt}%
\pgfpathmoveto{\pgfqpoint{2.885257in}{2.610867in}}%
\pgfpathlineto{\pgfqpoint{2.898547in}{2.594093in}}%
\pgfpathlineto{\pgfqpoint{2.911833in}{2.577497in}}%
\pgfpathlineto{\pgfqpoint{2.925113in}{2.561077in}}%
\pgfpathlineto{\pgfqpoint{2.938390in}{2.544834in}}%
\pgfpathlineto{\pgfqpoint{2.946562in}{2.547864in}}%
\pgfpathlineto{\pgfqpoint{2.954724in}{2.551042in}}%
\pgfpathlineto{\pgfqpoint{2.962874in}{2.554365in}}%
\pgfpathlineto{\pgfqpoint{2.971015in}{2.557831in}}%
\pgfpathlineto{\pgfqpoint{2.957769in}{2.573796in}}%
\pgfpathlineto{\pgfqpoint{2.944518in}{2.589937in}}%
\pgfpathlineto{\pgfqpoint{2.931263in}{2.606255in}}%
\pgfpathlineto{\pgfqpoint{2.918004in}{2.622750in}}%
\pgfpathlineto{\pgfqpoint{2.909834in}{2.619555in}}%
\pgfpathlineto{\pgfqpoint{2.901652in}{2.616509in}}%
\pgfpathlineto{\pgfqpoint{2.893460in}{2.613612in}}%
\pgfpathlineto{\pgfqpoint{2.885257in}{2.610867in}}%
\pgfpathclose%
\pgfusepath{fill}%
\end{pgfscope}%
\begin{pgfscope}%
\pgfpathrectangle{\pgfqpoint{1.254980in}{0.150000in}}{\pgfqpoint{5.490039in}{5.490039in}}%
\pgfusepath{clip}%
\pgfsetbuttcap%
\pgfsetroundjoin%
\definecolor{currentfill}{rgb}{0.168126,0.459988,0.558082}%
\pgfsetfillcolor{currentfill}%
\pgfsetfillopacity{0.700000}%
\pgfsetlinewidth{0.000000pt}%
\definecolor{currentstroke}{rgb}{0.000000,0.000000,0.000000}%
\pgfsetstrokecolor{currentstroke}%
\pgfsetdash{}{0pt}%
\pgfpathmoveto{\pgfqpoint{5.716901in}{2.819865in}}%
\pgfpathlineto{\pgfqpoint{5.730816in}{2.825399in}}%
\pgfpathlineto{\pgfqpoint{5.744747in}{2.831044in}}%
\pgfpathlineto{\pgfqpoint{5.758692in}{2.836801in}}%
\pgfpathlineto{\pgfqpoint{5.772652in}{2.842669in}}%
\pgfpathlineto{\pgfqpoint{5.779697in}{2.850757in}}%
\pgfpathlineto{\pgfqpoint{5.786736in}{2.858807in}}%
\pgfpathlineto{\pgfqpoint{5.793768in}{2.866819in}}%
\pgfpathlineto{\pgfqpoint{5.800795in}{2.874794in}}%
\pgfpathlineto{\pgfqpoint{5.786848in}{2.869070in}}%
\pgfpathlineto{\pgfqpoint{5.772916in}{2.863457in}}%
\pgfpathlineto{\pgfqpoint{5.758999in}{2.857956in}}%
\pgfpathlineto{\pgfqpoint{5.745097in}{2.852565in}}%
\pgfpathlineto{\pgfqpoint{5.738057in}{2.844440in}}%
\pgfpathlineto{\pgfqpoint{5.731011in}{2.836282in}}%
\pgfpathlineto{\pgfqpoint{5.723959in}{2.828091in}}%
\pgfpathlineto{\pgfqpoint{5.716901in}{2.819865in}}%
\pgfpathclose%
\pgfusepath{fill}%
\end{pgfscope}%
\begin{pgfscope}%
\pgfpathrectangle{\pgfqpoint{1.254980in}{0.150000in}}{\pgfqpoint{5.490039in}{5.490039in}}%
\pgfusepath{clip}%
\pgfsetbuttcap%
\pgfsetroundjoin%
\definecolor{currentfill}{rgb}{0.128729,0.563265,0.551229}%
\pgfsetfillcolor{currentfill}%
\pgfsetfillopacity{0.700000}%
\pgfsetlinewidth{0.000000pt}%
\definecolor{currentstroke}{rgb}{0.000000,0.000000,0.000000}%
\pgfsetstrokecolor{currentstroke}%
\pgfsetdash{}{0pt}%
\pgfpathmoveto{\pgfqpoint{2.544215in}{3.161476in}}%
\pgfpathlineto{\pgfqpoint{2.557671in}{3.139516in}}%
\pgfpathlineto{\pgfqpoint{2.571118in}{3.117769in}}%
\pgfpathlineto{\pgfqpoint{2.584555in}{3.096235in}}%
\pgfpathlineto{\pgfqpoint{2.597983in}{3.074911in}}%
\pgfpathlineto{\pgfqpoint{2.606343in}{3.076605in}}%
\pgfpathlineto{\pgfqpoint{2.614689in}{3.078468in}}%
\pgfpathlineto{\pgfqpoint{2.623023in}{3.080501in}}%
\pgfpathlineto{\pgfqpoint{2.631344in}{3.082699in}}%
\pgfpathlineto{\pgfqpoint{2.617952in}{3.103748in}}%
\pgfpathlineto{\pgfqpoint{2.604551in}{3.125007in}}%
\pgfpathlineto{\pgfqpoint{2.591141in}{3.146478in}}%
\pgfpathlineto{\pgfqpoint{2.577722in}{3.168162in}}%
\pgfpathlineto{\pgfqpoint{2.569365in}{3.166232in}}%
\pgfpathlineto{\pgfqpoint{2.560995in}{3.164473in}}%
\pgfpathlineto{\pgfqpoint{2.552612in}{3.162887in}}%
\pgfpathlineto{\pgfqpoint{2.544215in}{3.161476in}}%
\pgfpathclose%
\pgfusepath{fill}%
\end{pgfscope}%
\begin{pgfscope}%
\pgfpathrectangle{\pgfqpoint{1.254980in}{0.150000in}}{\pgfqpoint{5.490039in}{5.490039in}}%
\pgfusepath{clip}%
\pgfsetbuttcap%
\pgfsetroundjoin%
\definecolor{currentfill}{rgb}{0.243113,0.292092,0.538516}%
\pgfsetfillcolor{currentfill}%
\pgfsetfillopacity{0.700000}%
\pgfsetlinewidth{0.000000pt}%
\definecolor{currentstroke}{rgb}{0.000000,0.000000,0.000000}%
\pgfsetstrokecolor{currentstroke}%
\pgfsetdash{}{0pt}%
\pgfpathmoveto{\pgfqpoint{5.101564in}{2.405985in}}%
\pgfpathlineto{\pgfqpoint{5.115181in}{2.408912in}}%
\pgfpathlineto{\pgfqpoint{5.128810in}{2.411954in}}%
\pgfpathlineto{\pgfqpoint{5.142452in}{2.415109in}}%
\pgfpathlineto{\pgfqpoint{5.156105in}{2.418378in}}%
\pgfpathlineto{\pgfqpoint{5.163408in}{2.428354in}}%
\pgfpathlineto{\pgfqpoint{5.170706in}{2.438289in}}%
\pgfpathlineto{\pgfqpoint{5.177998in}{2.448184in}}%
\pgfpathlineto{\pgfqpoint{5.185284in}{2.458039in}}%
\pgfpathlineto{\pgfqpoint{5.171638in}{2.454781in}}%
\pgfpathlineto{\pgfqpoint{5.158005in}{2.451636in}}%
\pgfpathlineto{\pgfqpoint{5.144383in}{2.448606in}}%
\pgfpathlineto{\pgfqpoint{5.130774in}{2.445689in}}%
\pgfpathlineto{\pgfqpoint{5.123480in}{2.435817in}}%
\pgfpathlineto{\pgfqpoint{5.116180in}{2.425910in}}%
\pgfpathlineto{\pgfqpoint{5.108875in}{2.415966in}}%
\pgfpathlineto{\pgfqpoint{5.101564in}{2.405985in}}%
\pgfpathclose%
\pgfusepath{fill}%
\end{pgfscope}%
\begin{pgfscope}%
\pgfpathrectangle{\pgfqpoint{1.254980in}{0.150000in}}{\pgfqpoint{5.490039in}{5.490039in}}%
\pgfusepath{clip}%
\pgfsetbuttcap%
\pgfsetroundjoin%
\definecolor{currentfill}{rgb}{0.258965,0.251537,0.524736}%
\pgfsetfillcolor{currentfill}%
\pgfsetfillopacity{0.700000}%
\pgfsetlinewidth{0.000000pt}%
\definecolor{currentstroke}{rgb}{0.000000,0.000000,0.000000}%
\pgfsetstrokecolor{currentstroke}%
\pgfsetdash{}{0pt}%
\pgfpathmoveto{\pgfqpoint{3.097413in}{2.363257in}}%
\pgfpathlineto{\pgfqpoint{3.110645in}{2.349208in}}%
\pgfpathlineto{\pgfqpoint{3.123875in}{2.335322in}}%
\pgfpathlineto{\pgfqpoint{3.137102in}{2.321596in}}%
\pgfpathlineto{\pgfqpoint{3.150327in}{2.308032in}}%
\pgfpathlineto{\pgfqpoint{3.158385in}{2.312185in}}%
\pgfpathlineto{\pgfqpoint{3.166433in}{2.316468in}}%
\pgfpathlineto{\pgfqpoint{3.174472in}{2.320880in}}%
\pgfpathlineto{\pgfqpoint{3.182502in}{2.325417in}}%
\pgfpathlineto{\pgfqpoint{3.169303in}{2.338710in}}%
\pgfpathlineto{\pgfqpoint{3.156102in}{2.352164in}}%
\pgfpathlineto{\pgfqpoint{3.142899in}{2.365778in}}%
\pgfpathlineto{\pgfqpoint{3.129694in}{2.379554in}}%
\pgfpathlineto{\pgfqpoint{3.121639in}{2.375282in}}%
\pgfpathlineto{\pgfqpoint{3.113573in}{2.371141in}}%
\pgfpathlineto{\pgfqpoint{3.105498in}{2.367132in}}%
\pgfpathlineto{\pgfqpoint{3.097413in}{2.363257in}}%
\pgfpathclose%
\pgfusepath{fill}%
\end{pgfscope}%
\begin{pgfscope}%
\pgfpathrectangle{\pgfqpoint{1.254980in}{0.150000in}}{\pgfqpoint{5.490039in}{5.490039in}}%
\pgfusepath{clip}%
\pgfsetbuttcap%
\pgfsetroundjoin%
\definecolor{currentfill}{rgb}{0.280894,0.078907,0.402329}%
\pgfsetfillcolor{currentfill}%
\pgfsetfillopacity{0.700000}%
\pgfsetlinewidth{0.000000pt}%
\definecolor{currentstroke}{rgb}{0.000000,0.000000,0.000000}%
\pgfsetstrokecolor{currentstroke}%
\pgfsetdash{}{0pt}%
\pgfpathmoveto{\pgfqpoint{3.635996in}{2.007406in}}%
\pgfpathlineto{\pgfqpoint{3.649188in}{1.999232in}}%
\pgfpathlineto{\pgfqpoint{3.662381in}{1.991194in}}%
\pgfpathlineto{\pgfqpoint{3.675578in}{1.983292in}}%
\pgfpathlineto{\pgfqpoint{3.688778in}{1.975526in}}%
\pgfpathlineto{\pgfqpoint{3.696577in}{1.982821in}}%
\pgfpathlineto{\pgfqpoint{3.704369in}{1.990193in}}%
\pgfpathlineto{\pgfqpoint{3.712155in}{1.997640in}}%
\pgfpathlineto{\pgfqpoint{3.719935in}{2.005161in}}%
\pgfpathlineto{\pgfqpoint{3.706752in}{2.012696in}}%
\pgfpathlineto{\pgfqpoint{3.693573in}{2.020367in}}%
\pgfpathlineto{\pgfqpoint{3.680396in}{2.028174in}}%
\pgfpathlineto{\pgfqpoint{3.667222in}{2.036118in}}%
\pgfpathlineto{\pgfqpoint{3.659426in}{2.028821in}}%
\pgfpathlineto{\pgfqpoint{3.651623in}{2.021603in}}%
\pgfpathlineto{\pgfqpoint{3.643813in}{2.014464in}}%
\pgfpathlineto{\pgfqpoint{3.635996in}{2.007406in}}%
\pgfpathclose%
\pgfusepath{fill}%
\end{pgfscope}%
\begin{pgfscope}%
\pgfpathrectangle{\pgfqpoint{1.254980in}{0.150000in}}{\pgfqpoint{5.490039in}{5.490039in}}%
\pgfusepath{clip}%
\pgfsetbuttcap%
\pgfsetroundjoin%
\definecolor{currentfill}{rgb}{0.201239,0.383670,0.554294}%
\pgfsetfillcolor{currentfill}%
\pgfsetfillopacity{0.700000}%
\pgfsetlinewidth{0.000000pt}%
\definecolor{currentstroke}{rgb}{0.000000,0.000000,0.000000}%
\pgfsetstrokecolor{currentstroke}%
\pgfsetdash{}{0pt}%
\pgfpathmoveto{\pgfqpoint{2.832044in}{2.679764in}}%
\pgfpathlineto{\pgfqpoint{2.845355in}{2.662267in}}%
\pgfpathlineto{\pgfqpoint{2.858661in}{2.644953in}}%
\pgfpathlineto{\pgfqpoint{2.871961in}{2.627819in}}%
\pgfpathlineto{\pgfqpoint{2.885257in}{2.610867in}}%
\pgfpathlineto{\pgfqpoint{2.893460in}{2.613612in}}%
\pgfpathlineto{\pgfqpoint{2.901652in}{2.616509in}}%
\pgfpathlineto{\pgfqpoint{2.909834in}{2.619555in}}%
\pgfpathlineto{\pgfqpoint{2.918004in}{2.622750in}}%
\pgfpathlineto{\pgfqpoint{2.904740in}{2.639423in}}%
\pgfpathlineto{\pgfqpoint{2.891471in}{2.656276in}}%
\pgfpathlineto{\pgfqpoint{2.878197in}{2.673310in}}%
\pgfpathlineto{\pgfqpoint{2.864918in}{2.690526in}}%
\pgfpathlineto{\pgfqpoint{2.856716in}{2.687605in}}%
\pgfpathlineto{\pgfqpoint{2.848504in}{2.684836in}}%
\pgfpathlineto{\pgfqpoint{2.840280in}{2.682222in}}%
\pgfpathlineto{\pgfqpoint{2.832044in}{2.679764in}}%
\pgfpathclose%
\pgfusepath{fill}%
\end{pgfscope}%
\begin{pgfscope}%
\pgfpathrectangle{\pgfqpoint{1.254980in}{0.150000in}}{\pgfqpoint{5.490039in}{5.490039in}}%
\pgfusepath{clip}%
\pgfsetbuttcap%
\pgfsetroundjoin%
\definecolor{currentfill}{rgb}{0.283229,0.120777,0.440584}%
\pgfsetfillcolor{currentfill}%
\pgfsetfillopacity{0.700000}%
\pgfsetlinewidth{0.000000pt}%
\definecolor{currentstroke}{rgb}{0.000000,0.000000,0.000000}%
\pgfsetstrokecolor{currentstroke}%
\pgfsetdash{}{0pt}%
\pgfpathmoveto{\pgfqpoint{3.446290in}{2.091967in}}%
\pgfpathlineto{\pgfqpoint{3.459480in}{2.081858in}}%
\pgfpathlineto{\pgfqpoint{3.472671in}{2.071893in}}%
\pgfpathlineto{\pgfqpoint{3.485863in}{2.062071in}}%
\pgfpathlineto{\pgfqpoint{3.499057in}{2.052392in}}%
\pgfpathlineto{\pgfqpoint{3.506940in}{2.058598in}}%
\pgfpathlineto{\pgfqpoint{3.514816in}{2.064902in}}%
\pgfpathlineto{\pgfqpoint{3.522684in}{2.071300in}}%
\pgfpathlineto{\pgfqpoint{3.530545in}{2.077791in}}%
\pgfpathlineto{\pgfqpoint{3.517372in}{2.087221in}}%
\pgfpathlineto{\pgfqpoint{3.504200in}{2.096794in}}%
\pgfpathlineto{\pgfqpoint{3.491029in}{2.106510in}}%
\pgfpathlineto{\pgfqpoint{3.477860in}{2.116369in}}%
\pgfpathlineto{\pgfqpoint{3.469979in}{2.110121in}}%
\pgfpathlineto{\pgfqpoint{3.462090in}{2.103970in}}%
\pgfpathlineto{\pgfqpoint{3.454194in}{2.097918in}}%
\pgfpathlineto{\pgfqpoint{3.446290in}{2.091967in}}%
\pgfpathclose%
\pgfusepath{fill}%
\end{pgfscope}%
\begin{pgfscope}%
\pgfpathrectangle{\pgfqpoint{1.254980in}{0.150000in}}{\pgfqpoint{5.490039in}{5.490039in}}%
\pgfusepath{clip}%
\pgfsetbuttcap%
\pgfsetroundjoin%
\definecolor{currentfill}{rgb}{0.282327,0.094955,0.417331}%
\pgfsetfillcolor{currentfill}%
\pgfsetfillopacity{0.700000}%
\pgfsetlinewidth{0.000000pt}%
\definecolor{currentstroke}{rgb}{0.000000,0.000000,0.000000}%
\pgfsetstrokecolor{currentstroke}%
\pgfsetdash{}{0pt}%
\pgfpathmoveto{\pgfqpoint{4.349588in}{2.016587in}}%
\pgfpathlineto{\pgfqpoint{4.362912in}{2.014670in}}%
\pgfpathlineto{\pgfqpoint{4.376244in}{2.012874in}}%
\pgfpathlineto{\pgfqpoint{4.389583in}{2.011198in}}%
\pgfpathlineto{\pgfqpoint{4.402931in}{2.009643in}}%
\pgfpathlineto{\pgfqpoint{4.410479in}{2.019774in}}%
\pgfpathlineto{\pgfqpoint{4.418021in}{2.029911in}}%
\pgfpathlineto{\pgfqpoint{4.425559in}{2.040052in}}%
\pgfpathlineto{\pgfqpoint{4.433092in}{2.050197in}}%
\pgfpathlineto{\pgfqpoint{4.419752in}{2.051619in}}%
\pgfpathlineto{\pgfqpoint{4.406421in}{2.053161in}}%
\pgfpathlineto{\pgfqpoint{4.393098in}{2.054824in}}%
\pgfpathlineto{\pgfqpoint{4.379782in}{2.056607in}}%
\pgfpathlineto{\pgfqpoint{4.372241in}{2.046590in}}%
\pgfpathlineto{\pgfqpoint{4.364695in}{2.036580in}}%
\pgfpathlineto{\pgfqpoint{4.357144in}{2.026579in}}%
\pgfpathlineto{\pgfqpoint{4.349588in}{2.016587in}}%
\pgfpathclose%
\pgfusepath{fill}%
\end{pgfscope}%
\begin{pgfscope}%
\pgfpathrectangle{\pgfqpoint{1.254980in}{0.150000in}}{\pgfqpoint{5.490039in}{5.490039in}}%
\pgfusepath{clip}%
\pgfsetbuttcap%
\pgfsetroundjoin%
\definecolor{currentfill}{rgb}{0.159194,0.482237,0.558073}%
\pgfsetfillcolor{currentfill}%
\pgfsetfillopacity{0.700000}%
\pgfsetlinewidth{0.000000pt}%
\definecolor{currentstroke}{rgb}{0.000000,0.000000,0.000000}%
\pgfsetstrokecolor{currentstroke}%
\pgfsetdash{}{0pt}%
\pgfpathmoveto{\pgfqpoint{5.800795in}{2.874794in}}%
\pgfpathlineto{\pgfqpoint{5.814757in}{2.880629in}}%
\pgfpathlineto{\pgfqpoint{5.828734in}{2.886576in}}%
\pgfpathlineto{\pgfqpoint{5.842726in}{2.892633in}}%
\pgfpathlineto{\pgfqpoint{5.856733in}{2.898801in}}%
\pgfpathlineto{\pgfqpoint{5.863740in}{2.906588in}}%
\pgfpathlineto{\pgfqpoint{5.870740in}{2.914337in}}%
\pgfpathlineto{\pgfqpoint{5.877734in}{2.922052in}}%
\pgfpathlineto{\pgfqpoint{5.884721in}{2.929733in}}%
\pgfpathlineto{\pgfqpoint{5.870728in}{2.923725in}}%
\pgfpathlineto{\pgfqpoint{5.856751in}{2.917829in}}%
\pgfpathlineto{\pgfqpoint{5.842788in}{2.912043in}}%
\pgfpathlineto{\pgfqpoint{5.828841in}{2.906368in}}%
\pgfpathlineto{\pgfqpoint{5.821838in}{2.898521in}}%
\pgfpathlineto{\pgfqpoint{5.814830in}{2.890644in}}%
\pgfpathlineto{\pgfqpoint{5.807815in}{2.882735in}}%
\pgfpathlineto{\pgfqpoint{5.800795in}{2.874794in}}%
\pgfpathclose%
\pgfusepath{fill}%
\end{pgfscope}%
\begin{pgfscope}%
\pgfpathrectangle{\pgfqpoint{1.254980in}{0.150000in}}{\pgfqpoint{5.490039in}{5.490039in}}%
\pgfusepath{clip}%
\pgfsetbuttcap%
\pgfsetroundjoin%
\definecolor{currentfill}{rgb}{0.277941,0.056324,0.381191}%
\pgfsetfillcolor{currentfill}%
\pgfsetfillopacity{0.700000}%
\pgfsetlinewidth{0.000000pt}%
\definecolor{currentstroke}{rgb}{0.000000,0.000000,0.000000}%
\pgfsetstrokecolor{currentstroke}%
\pgfsetdash{}{0pt}%
\pgfpathmoveto{\pgfqpoint{4.045841in}{1.953038in}}%
\pgfpathlineto{\pgfqpoint{4.059088in}{1.948653in}}%
\pgfpathlineto{\pgfqpoint{4.072340in}{1.944393in}}%
\pgfpathlineto{\pgfqpoint{4.085599in}{1.940258in}}%
\pgfpathlineto{\pgfqpoint{4.098863in}{1.936249in}}%
\pgfpathlineto{\pgfqpoint{4.106508in}{1.945487in}}%
\pgfpathlineto{\pgfqpoint{4.114149in}{1.954760in}}%
\pgfpathlineto{\pgfqpoint{4.121784in}{1.964065in}}%
\pgfpathlineto{\pgfqpoint{4.129414in}{1.973403in}}%
\pgfpathlineto{\pgfqpoint{4.116161in}{1.977231in}}%
\pgfpathlineto{\pgfqpoint{4.102914in}{1.981185in}}%
\pgfpathlineto{\pgfqpoint{4.089673in}{1.985263in}}%
\pgfpathlineto{\pgfqpoint{4.076437in}{1.989468in}}%
\pgfpathlineto{\pgfqpoint{4.068796in}{1.980305in}}%
\pgfpathlineto{\pgfqpoint{4.061150in}{1.971179in}}%
\pgfpathlineto{\pgfqpoint{4.053498in}{1.962089in}}%
\pgfpathlineto{\pgfqpoint{4.045841in}{1.953038in}}%
\pgfpathclose%
\pgfusepath{fill}%
\end{pgfscope}%
\begin{pgfscope}%
\pgfpathrectangle{\pgfqpoint{1.254980in}{0.150000in}}{\pgfqpoint{5.490039in}{5.490039in}}%
\pgfusepath{clip}%
\pgfsetbuttcap%
\pgfsetroundjoin%
\definecolor{currentfill}{rgb}{0.233603,0.313828,0.543914}%
\pgfsetfillcolor{currentfill}%
\pgfsetfillopacity{0.700000}%
\pgfsetlinewidth{0.000000pt}%
\definecolor{currentstroke}{rgb}{0.000000,0.000000,0.000000}%
\pgfsetstrokecolor{currentstroke}%
\pgfsetdash{}{0pt}%
\pgfpathmoveto{\pgfqpoint{5.185284in}{2.458039in}}%
\pgfpathlineto{\pgfqpoint{5.198943in}{2.461411in}}%
\pgfpathlineto{\pgfqpoint{5.212614in}{2.464897in}}%
\pgfpathlineto{\pgfqpoint{5.226297in}{2.468496in}}%
\pgfpathlineto{\pgfqpoint{5.239993in}{2.472209in}}%
\pgfpathlineto{\pgfqpoint{5.247266in}{2.482004in}}%
\pgfpathlineto{\pgfqpoint{5.254534in}{2.491755in}}%
\pgfpathlineto{\pgfqpoint{5.261796in}{2.501465in}}%
\pgfpathlineto{\pgfqpoint{5.269052in}{2.511132in}}%
\pgfpathlineto{\pgfqpoint{5.255364in}{2.507446in}}%
\pgfpathlineto{\pgfqpoint{5.241689in}{2.503874in}}%
\pgfpathlineto{\pgfqpoint{5.228026in}{2.500415in}}%
\pgfpathlineto{\pgfqpoint{5.214376in}{2.497070in}}%
\pgfpathlineto{\pgfqpoint{5.207111in}{2.487370in}}%
\pgfpathlineto{\pgfqpoint{5.199841in}{2.477632in}}%
\pgfpathlineto{\pgfqpoint{5.192565in}{2.467855in}}%
\pgfpathlineto{\pgfqpoint{5.185284in}{2.458039in}}%
\pgfpathclose%
\pgfusepath{fill}%
\end{pgfscope}%
\begin{pgfscope}%
\pgfpathrectangle{\pgfqpoint{1.254980in}{0.150000in}}{\pgfqpoint{5.490039in}{5.490039in}}%
\pgfusepath{clip}%
\pgfsetbuttcap%
\pgfsetroundjoin%
\definecolor{currentfill}{rgb}{0.266580,0.228262,0.514349}%
\pgfsetfillcolor{currentfill}%
\pgfsetfillopacity{0.700000}%
\pgfsetlinewidth{0.000000pt}%
\definecolor{currentstroke}{rgb}{0.000000,0.000000,0.000000}%
\pgfsetstrokecolor{currentstroke}%
\pgfsetdash{}{0pt}%
\pgfpathmoveto{\pgfqpoint{3.150327in}{2.308032in}}%
\pgfpathlineto{\pgfqpoint{3.163550in}{2.294627in}}%
\pgfpathlineto{\pgfqpoint{3.176772in}{2.281381in}}%
\pgfpathlineto{\pgfqpoint{3.189991in}{2.268293in}}%
\pgfpathlineto{\pgfqpoint{3.203210in}{2.255362in}}%
\pgfpathlineto{\pgfqpoint{3.211241in}{2.259792in}}%
\pgfpathlineto{\pgfqpoint{3.219263in}{2.264348in}}%
\pgfpathlineto{\pgfqpoint{3.227277in}{2.269028in}}%
\pgfpathlineto{\pgfqpoint{3.235281in}{2.273830in}}%
\pgfpathlineto{\pgfqpoint{3.222088in}{2.286491in}}%
\pgfpathlineto{\pgfqpoint{3.208894in}{2.299308in}}%
\pgfpathlineto{\pgfqpoint{3.195699in}{2.312283in}}%
\pgfpathlineto{\pgfqpoint{3.182502in}{2.325417in}}%
\pgfpathlineto{\pgfqpoint{3.174472in}{2.320880in}}%
\pgfpathlineto{\pgfqpoint{3.166433in}{2.316468in}}%
\pgfpathlineto{\pgfqpoint{3.158385in}{2.312185in}}%
\pgfpathlineto{\pgfqpoint{3.150327in}{2.308032in}}%
\pgfpathclose%
\pgfusepath{fill}%
\end{pgfscope}%
\begin{pgfscope}%
\pgfpathrectangle{\pgfqpoint{1.254980in}{0.150000in}}{\pgfqpoint{5.490039in}{5.490039in}}%
\pgfusepath{clip}%
\pgfsetbuttcap%
\pgfsetroundjoin%
\definecolor{currentfill}{rgb}{0.187231,0.414746,0.556547}%
\pgfsetfillcolor{currentfill}%
\pgfsetfillopacity{0.700000}%
\pgfsetlinewidth{0.000000pt}%
\definecolor{currentstroke}{rgb}{0.000000,0.000000,0.000000}%
\pgfsetstrokecolor{currentstroke}%
\pgfsetdash{}{0pt}%
\pgfpathmoveto{\pgfqpoint{2.778742in}{2.751598in}}%
\pgfpathlineto{\pgfqpoint{2.792076in}{2.733360in}}%
\pgfpathlineto{\pgfqpoint{2.805405in}{2.715309in}}%
\pgfpathlineto{\pgfqpoint{2.818727in}{2.697444in}}%
\pgfpathlineto{\pgfqpoint{2.832044in}{2.679764in}}%
\pgfpathlineto{\pgfqpoint{2.840280in}{2.682222in}}%
\pgfpathlineto{\pgfqpoint{2.848504in}{2.684836in}}%
\pgfpathlineto{\pgfqpoint{2.856716in}{2.687605in}}%
\pgfpathlineto{\pgfqpoint{2.864918in}{2.690526in}}%
\pgfpathlineto{\pgfqpoint{2.851633in}{2.707924in}}%
\pgfpathlineto{\pgfqpoint{2.838343in}{2.725507in}}%
\pgfpathlineto{\pgfqpoint{2.825048in}{2.743276in}}%
\pgfpathlineto{\pgfqpoint{2.811746in}{2.761231in}}%
\pgfpathlineto{\pgfqpoint{2.803513in}{2.758585in}}%
\pgfpathlineto{\pgfqpoint{2.795268in}{2.756097in}}%
\pgfpathlineto{\pgfqpoint{2.787011in}{2.753767in}}%
\pgfpathlineto{\pgfqpoint{2.778742in}{2.751598in}}%
\pgfpathclose%
\pgfusepath{fill}%
\end{pgfscope}%
\begin{pgfscope}%
\pgfpathrectangle{\pgfqpoint{1.254980in}{0.150000in}}{\pgfqpoint{5.490039in}{5.490039in}}%
\pgfusepath{clip}%
\pgfsetbuttcap%
\pgfsetroundjoin%
\definecolor{currentfill}{rgb}{0.143343,0.522773,0.556295}%
\pgfsetfillcolor{currentfill}%
\pgfsetfillopacity{0.700000}%
\pgfsetlinewidth{0.000000pt}%
\definecolor{currentstroke}{rgb}{0.000000,0.000000,0.000000}%
\pgfsetstrokecolor{currentstroke}%
\pgfsetdash{}{0pt}%
\pgfpathmoveto{\pgfqpoint{5.968677in}{2.984582in}}%
\pgfpathlineto{\pgfqpoint{5.982732in}{2.990965in}}%
\pgfpathlineto{\pgfqpoint{5.996803in}{2.997459in}}%
\pgfpathlineto{\pgfqpoint{6.010890in}{3.004063in}}%
\pgfpathlineto{\pgfqpoint{6.017820in}{3.011274in}}%
\pgfpathlineto{\pgfqpoint{6.024743in}{3.018455in}}%
\pgfpathlineto{\pgfqpoint{6.031661in}{3.025608in}}%
\pgfpathlineto{\pgfqpoint{6.038573in}{3.032736in}}%
\pgfpathlineto{\pgfqpoint{6.024503in}{3.026327in}}%
\pgfpathlineto{\pgfqpoint{6.010448in}{3.020029in}}%
\pgfpathlineto{\pgfqpoint{5.996410in}{3.013840in}}%
\pgfpathlineto{\pgfqpoint{5.989486in}{3.006562in}}%
\pgfpathlineto{\pgfqpoint{5.982555in}{2.999260in}}%
\pgfpathlineto{\pgfqpoint{5.975619in}{2.991935in}}%
\pgfpathlineto{\pgfqpoint{5.968677in}{2.984582in}}%
\pgfpathclose%
\pgfusepath{fill}%
\end{pgfscope}%
\begin{pgfscope}%
\pgfpathrectangle{\pgfqpoint{1.254980in}{0.150000in}}{\pgfqpoint{5.490039in}{5.490039in}}%
\pgfusepath{clip}%
\pgfsetbuttcap%
\pgfsetroundjoin%
\definecolor{currentfill}{rgb}{0.150476,0.504369,0.557430}%
\pgfsetfillcolor{currentfill}%
\pgfsetfillopacity{0.700000}%
\pgfsetlinewidth{0.000000pt}%
\definecolor{currentstroke}{rgb}{0.000000,0.000000,0.000000}%
\pgfsetstrokecolor{currentstroke}%
\pgfsetdash{}{0pt}%
\pgfpathmoveto{\pgfqpoint{5.884721in}{2.929733in}}%
\pgfpathlineto{\pgfqpoint{5.898730in}{2.935851in}}%
\pgfpathlineto{\pgfqpoint{5.912754in}{2.942080in}}%
\pgfpathlineto{\pgfqpoint{5.926793in}{2.948419in}}%
\pgfpathlineto{\pgfqpoint{5.940848in}{2.954870in}}%
\pgfpathlineto{\pgfqpoint{5.947815in}{2.962347in}}%
\pgfpathlineto{\pgfqpoint{5.954775in}{2.969791in}}%
\pgfpathlineto{\pgfqpoint{5.961729in}{2.977202in}}%
\pgfpathlineto{\pgfqpoint{5.968677in}{2.984582in}}%
\pgfpathlineto{\pgfqpoint{5.954638in}{2.978310in}}%
\pgfpathlineto{\pgfqpoint{5.940614in}{2.972148in}}%
\pgfpathlineto{\pgfqpoint{5.926606in}{2.966097in}}%
\pgfpathlineto{\pgfqpoint{5.912613in}{2.960156in}}%
\pgfpathlineto{\pgfqpoint{5.905649in}{2.952592in}}%
\pgfpathlineto{\pgfqpoint{5.898679in}{2.945001in}}%
\pgfpathlineto{\pgfqpoint{5.891703in}{2.937382in}}%
\pgfpathlineto{\pgfqpoint{5.884721in}{2.929733in}}%
\pgfpathclose%
\pgfusepath{fill}%
\end{pgfscope}%
\begin{pgfscope}%
\pgfpathrectangle{\pgfqpoint{1.254980in}{0.150000in}}{\pgfqpoint{5.490039in}{5.490039in}}%
\pgfusepath{clip}%
\pgfsetbuttcap%
\pgfsetroundjoin%
\definecolor{currentfill}{rgb}{0.221989,0.339161,0.548752}%
\pgfsetfillcolor{currentfill}%
\pgfsetfillopacity{0.700000}%
\pgfsetlinewidth{0.000000pt}%
\definecolor{currentstroke}{rgb}{0.000000,0.000000,0.000000}%
\pgfsetstrokecolor{currentstroke}%
\pgfsetdash{}{0pt}%
\pgfpathmoveto{\pgfqpoint{5.269052in}{2.511132in}}%
\pgfpathlineto{\pgfqpoint{5.282753in}{2.514931in}}%
\pgfpathlineto{\pgfqpoint{5.296467in}{2.518843in}}%
\pgfpathlineto{\pgfqpoint{5.310194in}{2.522869in}}%
\pgfpathlineto{\pgfqpoint{5.323934in}{2.527007in}}%
\pgfpathlineto{\pgfqpoint{5.331177in}{2.536595in}}%
\pgfpathlineto{\pgfqpoint{5.338413in}{2.546139in}}%
\pgfpathlineto{\pgfqpoint{5.345644in}{2.555639in}}%
\pgfpathlineto{\pgfqpoint{5.352869in}{2.565095in}}%
\pgfpathlineto{\pgfqpoint{5.339137in}{2.561000in}}%
\pgfpathlineto{\pgfqpoint{5.325419in}{2.557019in}}%
\pgfpathlineto{\pgfqpoint{5.311714in}{2.553150in}}%
\pgfpathlineto{\pgfqpoint{5.298021in}{2.549395in}}%
\pgfpathlineto{\pgfqpoint{5.290788in}{2.539889in}}%
\pgfpathlineto{\pgfqpoint{5.283548in}{2.530343in}}%
\pgfpathlineto{\pgfqpoint{5.276303in}{2.520758in}}%
\pgfpathlineto{\pgfqpoint{5.269052in}{2.511132in}}%
\pgfpathclose%
\pgfusepath{fill}%
\end{pgfscope}%
\begin{pgfscope}%
\pgfpathrectangle{\pgfqpoint{1.254980in}{0.150000in}}{\pgfqpoint{5.490039in}{5.490039in}}%
\pgfusepath{clip}%
\pgfsetbuttcap%
\pgfsetroundjoin%
\definecolor{currentfill}{rgb}{0.280894,0.078907,0.402329}%
\pgfsetfillcolor{currentfill}%
\pgfsetfillopacity{0.700000}%
\pgfsetlinewidth{0.000000pt}%
\definecolor{currentstroke}{rgb}{0.000000,0.000000,0.000000}%
\pgfsetstrokecolor{currentstroke}%
\pgfsetdash{}{0pt}%
\pgfpathmoveto{\pgfqpoint{4.266059in}{1.986208in}}%
\pgfpathlineto{\pgfqpoint{4.279362in}{1.983656in}}%
\pgfpathlineto{\pgfqpoint{4.292672in}{1.981226in}}%
\pgfpathlineto{\pgfqpoint{4.305990in}{1.978917in}}%
\pgfpathlineto{\pgfqpoint{4.319315in}{1.976730in}}%
\pgfpathlineto{\pgfqpoint{4.326891in}{1.986675in}}%
\pgfpathlineto{\pgfqpoint{4.334461in}{1.996634in}}%
\pgfpathlineto{\pgfqpoint{4.342027in}{2.006605in}}%
\pgfpathlineto{\pgfqpoint{4.349588in}{2.016587in}}%
\pgfpathlineto{\pgfqpoint{4.336271in}{2.018625in}}%
\pgfpathlineto{\pgfqpoint{4.322963in}{2.020784in}}%
\pgfpathlineto{\pgfqpoint{4.309661in}{2.023065in}}%
\pgfpathlineto{\pgfqpoint{4.296367in}{2.025468in}}%
\pgfpathlineto{\pgfqpoint{4.288798in}{2.015629in}}%
\pgfpathlineto{\pgfqpoint{4.281223in}{2.005806in}}%
\pgfpathlineto{\pgfqpoint{4.273643in}{1.995998in}}%
\pgfpathlineto{\pgfqpoint{4.266059in}{1.986208in}}%
\pgfpathclose%
\pgfusepath{fill}%
\end{pgfscope}%
\begin{pgfscope}%
\pgfpathrectangle{\pgfqpoint{1.254980in}{0.150000in}}{\pgfqpoint{5.490039in}{5.490039in}}%
\pgfusepath{clip}%
\pgfsetbuttcap%
\pgfsetroundjoin%
\definecolor{currentfill}{rgb}{0.273006,0.204520,0.501721}%
\pgfsetfillcolor{currentfill}%
\pgfsetfillopacity{0.700000}%
\pgfsetlinewidth{0.000000pt}%
\definecolor{currentstroke}{rgb}{0.000000,0.000000,0.000000}%
\pgfsetstrokecolor{currentstroke}%
\pgfsetdash{}{0pt}%
\pgfpathmoveto{\pgfqpoint{3.203210in}{2.255362in}}%
\pgfpathlineto{\pgfqpoint{3.216426in}{2.242587in}}%
\pgfpathlineto{\pgfqpoint{3.229642in}{2.229968in}}%
\pgfpathlineto{\pgfqpoint{3.242856in}{2.217504in}}%
\pgfpathlineto{\pgfqpoint{3.256069in}{2.205193in}}%
\pgfpathlineto{\pgfqpoint{3.264075in}{2.209899in}}%
\pgfpathlineto{\pgfqpoint{3.272073in}{2.214727in}}%
\pgfpathlineto{\pgfqpoint{3.280061in}{2.219675in}}%
\pgfpathlineto{\pgfqpoint{3.288041in}{2.224739in}}%
\pgfpathlineto{\pgfqpoint{3.274852in}{2.236781in}}%
\pgfpathlineto{\pgfqpoint{3.261663in}{2.248976in}}%
\pgfpathlineto{\pgfqpoint{3.248472in}{2.261325in}}%
\pgfpathlineto{\pgfqpoint{3.235281in}{2.273830in}}%
\pgfpathlineto{\pgfqpoint{3.227277in}{2.269028in}}%
\pgfpathlineto{\pgfqpoint{3.219263in}{2.264348in}}%
\pgfpathlineto{\pgfqpoint{3.211241in}{2.259792in}}%
\pgfpathlineto{\pgfqpoint{3.203210in}{2.255362in}}%
\pgfpathclose%
\pgfusepath{fill}%
\end{pgfscope}%
\begin{pgfscope}%
\pgfpathrectangle{\pgfqpoint{1.254980in}{0.150000in}}{\pgfqpoint{5.490039in}{5.490039in}}%
\pgfusepath{clip}%
\pgfsetbuttcap%
\pgfsetroundjoin%
\definecolor{currentfill}{rgb}{0.175841,0.441290,0.557685}%
\pgfsetfillcolor{currentfill}%
\pgfsetfillopacity{0.700000}%
\pgfsetlinewidth{0.000000pt}%
\definecolor{currentstroke}{rgb}{0.000000,0.000000,0.000000}%
\pgfsetstrokecolor{currentstroke}%
\pgfsetdash{}{0pt}%
\pgfpathmoveto{\pgfqpoint{2.725339in}{2.826444in}}%
\pgfpathlineto{\pgfqpoint{2.738700in}{2.807446in}}%
\pgfpathlineto{\pgfqpoint{2.752054in}{2.788639in}}%
\pgfpathlineto{\pgfqpoint{2.765401in}{2.770024in}}%
\pgfpathlineto{\pgfqpoint{2.778742in}{2.751598in}}%
\pgfpathlineto{\pgfqpoint{2.787011in}{2.753767in}}%
\pgfpathlineto{\pgfqpoint{2.795268in}{2.756097in}}%
\pgfpathlineto{\pgfqpoint{2.803513in}{2.758585in}}%
\pgfpathlineto{\pgfqpoint{2.811746in}{2.761231in}}%
\pgfpathlineto{\pgfqpoint{2.798439in}{2.779373in}}%
\pgfpathlineto{\pgfqpoint{2.785125in}{2.797705in}}%
\pgfpathlineto{\pgfqpoint{2.771805in}{2.816227in}}%
\pgfpathlineto{\pgfqpoint{2.758478in}{2.834940in}}%
\pgfpathlineto{\pgfqpoint{2.750212in}{2.832572in}}%
\pgfpathlineto{\pgfqpoint{2.741933in}{2.830366in}}%
\pgfpathlineto{\pgfqpoint{2.733642in}{2.828322in}}%
\pgfpathlineto{\pgfqpoint{2.725339in}{2.826444in}}%
\pgfpathclose%
\pgfusepath{fill}%
\end{pgfscope}%
\begin{pgfscope}%
\pgfpathrectangle{\pgfqpoint{1.254980in}{0.150000in}}{\pgfqpoint{5.490039in}{5.490039in}}%
\pgfusepath{clip}%
\pgfsetbuttcap%
\pgfsetroundjoin%
\definecolor{currentfill}{rgb}{0.282910,0.105393,0.426902}%
\pgfsetfillcolor{currentfill}%
\pgfsetfillopacity{0.700000}%
\pgfsetlinewidth{0.000000pt}%
\definecolor{currentstroke}{rgb}{0.000000,0.000000,0.000000}%
\pgfsetstrokecolor{currentstroke}%
\pgfsetdash{}{0pt}%
\pgfpathmoveto{\pgfqpoint{3.499057in}{2.052392in}}%
\pgfpathlineto{\pgfqpoint{3.512251in}{2.042855in}}%
\pgfpathlineto{\pgfqpoint{3.525447in}{2.033460in}}%
\pgfpathlineto{\pgfqpoint{3.538645in}{2.024205in}}%
\pgfpathlineto{\pgfqpoint{3.551844in}{2.015091in}}%
\pgfpathlineto{\pgfqpoint{3.559708in}{2.021551in}}%
\pgfpathlineto{\pgfqpoint{3.567564in}{2.028105in}}%
\pgfpathlineto{\pgfqpoint{3.575413in}{2.034749in}}%
\pgfpathlineto{\pgfqpoint{3.583254in}{2.041482in}}%
\pgfpathlineto{\pgfqpoint{3.570074in}{2.050349in}}%
\pgfpathlineto{\pgfqpoint{3.556896in}{2.059355in}}%
\pgfpathlineto{\pgfqpoint{3.543720in}{2.068502in}}%
\pgfpathlineto{\pgfqpoint{3.530545in}{2.077791in}}%
\pgfpathlineto{\pgfqpoint{3.522684in}{2.071300in}}%
\pgfpathlineto{\pgfqpoint{3.514816in}{2.064902in}}%
\pgfpathlineto{\pgfqpoint{3.506940in}{2.058598in}}%
\pgfpathlineto{\pgfqpoint{3.499057in}{2.052392in}}%
\pgfpathclose%
\pgfusepath{fill}%
\end{pgfscope}%
\begin{pgfscope}%
\pgfpathrectangle{\pgfqpoint{1.254980in}{0.150000in}}{\pgfqpoint{5.490039in}{5.490039in}}%
\pgfusepath{clip}%
\pgfsetbuttcap%
\pgfsetroundjoin%
\definecolor{currentfill}{rgb}{0.210503,0.363727,0.552206}%
\pgfsetfillcolor{currentfill}%
\pgfsetfillopacity{0.700000}%
\pgfsetlinewidth{0.000000pt}%
\definecolor{currentstroke}{rgb}{0.000000,0.000000,0.000000}%
\pgfsetstrokecolor{currentstroke}%
\pgfsetdash{}{0pt}%
\pgfpathmoveto{\pgfqpoint{5.352869in}{2.565095in}}%
\pgfpathlineto{\pgfqpoint{5.366614in}{2.569303in}}%
\pgfpathlineto{\pgfqpoint{5.380372in}{2.573623in}}%
\pgfpathlineto{\pgfqpoint{5.394143in}{2.578056in}}%
\pgfpathlineto{\pgfqpoint{5.407928in}{2.582603in}}%
\pgfpathlineto{\pgfqpoint{5.415139in}{2.591962in}}%
\pgfpathlineto{\pgfqpoint{5.422343in}{2.601275in}}%
\pgfpathlineto{\pgfqpoint{5.429542in}{2.610544in}}%
\pgfpathlineto{\pgfqpoint{5.436734in}{2.619768in}}%
\pgfpathlineto{\pgfqpoint{5.422958in}{2.615282in}}%
\pgfpathlineto{\pgfqpoint{5.409196in}{2.610909in}}%
\pgfpathlineto{\pgfqpoint{5.395447in}{2.606649in}}%
\pgfpathlineto{\pgfqpoint{5.381712in}{2.602502in}}%
\pgfpathlineto{\pgfqpoint{5.374510in}{2.593211in}}%
\pgfpathlineto{\pgfqpoint{5.367302in}{2.583880in}}%
\pgfpathlineto{\pgfqpoint{5.360088in}{2.574509in}}%
\pgfpathlineto{\pgfqpoint{5.352869in}{2.565095in}}%
\pgfpathclose%
\pgfusepath{fill}%
\end{pgfscope}%
\begin{pgfscope}%
\pgfpathrectangle{\pgfqpoint{1.254980in}{0.150000in}}{\pgfqpoint{5.490039in}{5.490039in}}%
\pgfusepath{clip}%
\pgfsetbuttcap%
\pgfsetroundjoin%
\definecolor{currentfill}{rgb}{0.277941,0.056324,0.381191}%
\pgfsetfillcolor{currentfill}%
\pgfsetfillopacity{0.700000}%
\pgfsetlinewidth{0.000000pt}%
\definecolor{currentstroke}{rgb}{0.000000,0.000000,0.000000}%
\pgfsetstrokecolor{currentstroke}%
\pgfsetdash{}{0pt}%
\pgfpathmoveto{\pgfqpoint{3.825517in}{1.949699in}}%
\pgfpathlineto{\pgfqpoint{3.838732in}{1.943362in}}%
\pgfpathlineto{\pgfqpoint{3.851951in}{1.937156in}}%
\pgfpathlineto{\pgfqpoint{3.865174in}{1.931079in}}%
\pgfpathlineto{\pgfqpoint{3.878401in}{1.925133in}}%
\pgfpathlineto{\pgfqpoint{3.886129in}{1.933386in}}%
\pgfpathlineto{\pgfqpoint{3.893851in}{1.941696in}}%
\pgfpathlineto{\pgfqpoint{3.901567in}{1.950064in}}%
\pgfpathlineto{\pgfqpoint{3.909277in}{1.958487in}}%
\pgfpathlineto{\pgfqpoint{3.896063in}{1.964220in}}%
\pgfpathlineto{\pgfqpoint{3.882855in}{1.970083in}}%
\pgfpathlineto{\pgfqpoint{3.869650in}{1.976076in}}%
\pgfpathlineto{\pgfqpoint{3.856450in}{1.982199in}}%
\pgfpathlineto{\pgfqpoint{3.848726in}{1.973984in}}%
\pgfpathlineto{\pgfqpoint{3.840996in}{1.965828in}}%
\pgfpathlineto{\pgfqpoint{3.833260in}{1.957732in}}%
\pgfpathlineto{\pgfqpoint{3.825517in}{1.949699in}}%
\pgfpathclose%
\pgfusepath{fill}%
\end{pgfscope}%
\begin{pgfscope}%
\pgfpathrectangle{\pgfqpoint{1.254980in}{0.150000in}}{\pgfqpoint{5.490039in}{5.490039in}}%
\pgfusepath{clip}%
\pgfsetbuttcap%
\pgfsetroundjoin%
\definecolor{currentfill}{rgb}{0.278012,0.180367,0.486697}%
\pgfsetfillcolor{currentfill}%
\pgfsetfillopacity{0.700000}%
\pgfsetlinewidth{0.000000pt}%
\definecolor{currentstroke}{rgb}{0.000000,0.000000,0.000000}%
\pgfsetstrokecolor{currentstroke}%
\pgfsetdash{}{0pt}%
\pgfpathmoveto{\pgfqpoint{3.256069in}{2.205193in}}%
\pgfpathlineto{\pgfqpoint{3.269282in}{2.193036in}}%
\pgfpathlineto{\pgfqpoint{3.282493in}{2.181032in}}%
\pgfpathlineto{\pgfqpoint{3.295704in}{2.169179in}}%
\pgfpathlineto{\pgfqpoint{3.308915in}{2.157477in}}%
\pgfpathlineto{\pgfqpoint{3.316896in}{2.162458in}}%
\pgfpathlineto{\pgfqpoint{3.324870in}{2.167556in}}%
\pgfpathlineto{\pgfqpoint{3.332834in}{2.172770in}}%
\pgfpathlineto{\pgfqpoint{3.340790in}{2.178097in}}%
\pgfpathlineto{\pgfqpoint{3.327604in}{2.189530in}}%
\pgfpathlineto{\pgfqpoint{3.314417in}{2.201115in}}%
\pgfpathlineto{\pgfqpoint{3.301229in}{2.212851in}}%
\pgfpathlineto{\pgfqpoint{3.288041in}{2.224739in}}%
\pgfpathlineto{\pgfqpoint{3.280061in}{2.219675in}}%
\pgfpathlineto{\pgfqpoint{3.272073in}{2.214727in}}%
\pgfpathlineto{\pgfqpoint{3.264075in}{2.209899in}}%
\pgfpathlineto{\pgfqpoint{3.256069in}{2.205193in}}%
\pgfpathclose%
\pgfusepath{fill}%
\end{pgfscope}%
\begin{pgfscope}%
\pgfpathrectangle{\pgfqpoint{1.254980in}{0.150000in}}{\pgfqpoint{5.490039in}{5.490039in}}%
\pgfusepath{clip}%
\pgfsetbuttcap%
\pgfsetroundjoin%
\definecolor{currentfill}{rgb}{0.279566,0.067836,0.391917}%
\pgfsetfillcolor{currentfill}%
\pgfsetfillopacity{0.700000}%
\pgfsetlinewidth{0.000000pt}%
\definecolor{currentstroke}{rgb}{0.000000,0.000000,0.000000}%
\pgfsetstrokecolor{currentstroke}%
\pgfsetdash{}{0pt}%
\pgfpathmoveto{\pgfqpoint{3.688778in}{1.975526in}}%
\pgfpathlineto{\pgfqpoint{3.701980in}{1.967896in}}%
\pgfpathlineto{\pgfqpoint{3.715185in}{1.960400in}}%
\pgfpathlineto{\pgfqpoint{3.728394in}{1.953038in}}%
\pgfpathlineto{\pgfqpoint{3.741606in}{1.945810in}}%
\pgfpathlineto{\pgfqpoint{3.749388in}{1.953341in}}%
\pgfpathlineto{\pgfqpoint{3.757164in}{1.960945in}}%
\pgfpathlineto{\pgfqpoint{3.764934in}{1.968620in}}%
\pgfpathlineto{\pgfqpoint{3.772698in}{1.976365in}}%
\pgfpathlineto{\pgfqpoint{3.759502in}{1.983363in}}%
\pgfpathlineto{\pgfqpoint{3.746310in}{1.990495in}}%
\pgfpathlineto{\pgfqpoint{3.733121in}{1.997761in}}%
\pgfpathlineto{\pgfqpoint{3.719935in}{2.005161in}}%
\pgfpathlineto{\pgfqpoint{3.712155in}{1.997640in}}%
\pgfpathlineto{\pgfqpoint{3.704369in}{1.990193in}}%
\pgfpathlineto{\pgfqpoint{3.696577in}{1.982821in}}%
\pgfpathlineto{\pgfqpoint{3.688778in}{1.975526in}}%
\pgfpathclose%
\pgfusepath{fill}%
\end{pgfscope}%
\begin{pgfscope}%
\pgfpathrectangle{\pgfqpoint{1.254980in}{0.150000in}}{\pgfqpoint{5.490039in}{5.490039in}}%
\pgfusepath{clip}%
\pgfsetbuttcap%
\pgfsetroundjoin%
\definecolor{currentfill}{rgb}{0.279566,0.067836,0.391917}%
\pgfsetfillcolor{currentfill}%
\pgfsetfillopacity{0.700000}%
\pgfsetlinewidth{0.000000pt}%
\definecolor{currentstroke}{rgb}{0.000000,0.000000,0.000000}%
\pgfsetstrokecolor{currentstroke}%
\pgfsetdash{}{0pt}%
\pgfpathmoveto{\pgfqpoint{4.182490in}{1.959334in}}%
\pgfpathlineto{\pgfqpoint{4.195775in}{1.956127in}}%
\pgfpathlineto{\pgfqpoint{4.209067in}{1.953042in}}%
\pgfpathlineto{\pgfqpoint{4.222365in}{1.950080in}}%
\pgfpathlineto{\pgfqpoint{4.235671in}{1.947241in}}%
\pgfpathlineto{\pgfqpoint{4.243275in}{1.956951in}}%
\pgfpathlineto{\pgfqpoint{4.250875in}{1.966683in}}%
\pgfpathlineto{\pgfqpoint{4.258469in}{1.976436in}}%
\pgfpathlineto{\pgfqpoint{4.266059in}{1.986208in}}%
\pgfpathlineto{\pgfqpoint{4.252763in}{1.988882in}}%
\pgfpathlineto{\pgfqpoint{4.239474in}{1.991679in}}%
\pgfpathlineto{\pgfqpoint{4.226192in}{1.994599in}}%
\pgfpathlineto{\pgfqpoint{4.212917in}{1.997642in}}%
\pgfpathlineto{\pgfqpoint{4.205317in}{1.988029in}}%
\pgfpathlineto{\pgfqpoint{4.197713in}{1.978439in}}%
\pgfpathlineto{\pgfqpoint{4.190104in}{1.968874in}}%
\pgfpathlineto{\pgfqpoint{4.182490in}{1.959334in}}%
\pgfpathclose%
\pgfusepath{fill}%
\end{pgfscope}%
\begin{pgfscope}%
\pgfpathrectangle{\pgfqpoint{1.254980in}{0.150000in}}{\pgfqpoint{5.490039in}{5.490039in}}%
\pgfusepath{clip}%
\pgfsetbuttcap%
\pgfsetroundjoin%
\definecolor{currentfill}{rgb}{0.277018,0.050344,0.375715}%
\pgfsetfillcolor{currentfill}%
\pgfsetfillopacity{0.700000}%
\pgfsetlinewidth{0.000000pt}%
\definecolor{currentstroke}{rgb}{0.000000,0.000000,0.000000}%
\pgfsetstrokecolor{currentstroke}%
\pgfsetdash{}{0pt}%
\pgfpathmoveto{\pgfqpoint{3.962177in}{1.936845in}}%
\pgfpathlineto{\pgfqpoint{3.975415in}{1.931755in}}%
\pgfpathlineto{\pgfqpoint{3.988658in}{1.926793in}}%
\pgfpathlineto{\pgfqpoint{4.001906in}{1.921957in}}%
\pgfpathlineto{\pgfqpoint{4.015159in}{1.917248in}}%
\pgfpathlineto{\pgfqpoint{4.022838in}{1.926131in}}%
\pgfpathlineto{\pgfqpoint{4.030511in}{1.935058in}}%
\pgfpathlineto{\pgfqpoint{4.038178in}{1.944028in}}%
\pgfpathlineto{\pgfqpoint{4.045841in}{1.953038in}}%
\pgfpathlineto{\pgfqpoint{4.032600in}{1.957550in}}%
\pgfpathlineto{\pgfqpoint{4.019364in}{1.962189in}}%
\pgfpathlineto{\pgfqpoint{4.006134in}{1.966954in}}%
\pgfpathlineto{\pgfqpoint{3.992909in}{1.971847in}}%
\pgfpathlineto{\pgfqpoint{3.985234in}{1.963027in}}%
\pgfpathlineto{\pgfqpoint{3.977554in}{1.954253in}}%
\pgfpathlineto{\pgfqpoint{3.969869in}{1.945525in}}%
\pgfpathlineto{\pgfqpoint{3.962177in}{1.936845in}}%
\pgfpathclose%
\pgfusepath{fill}%
\end{pgfscope}%
\begin{pgfscope}%
\pgfpathrectangle{\pgfqpoint{1.254980in}{0.150000in}}{\pgfqpoint{5.490039in}{5.490039in}}%
\pgfusepath{clip}%
\pgfsetbuttcap%
\pgfsetroundjoin%
\definecolor{currentfill}{rgb}{0.162142,0.474838,0.558140}%
\pgfsetfillcolor{currentfill}%
\pgfsetfillopacity{0.700000}%
\pgfsetlinewidth{0.000000pt}%
\definecolor{currentstroke}{rgb}{0.000000,0.000000,0.000000}%
\pgfsetstrokecolor{currentstroke}%
\pgfsetdash{}{0pt}%
\pgfpathmoveto{\pgfqpoint{2.671824in}{2.904384in}}%
\pgfpathlineto{\pgfqpoint{2.685214in}{2.884605in}}%
\pgfpathlineto{\pgfqpoint{2.698596in}{2.865023in}}%
\pgfpathlineto{\pgfqpoint{2.711971in}{2.845636in}}%
\pgfpathlineto{\pgfqpoint{2.725339in}{2.826444in}}%
\pgfpathlineto{\pgfqpoint{2.733642in}{2.828322in}}%
\pgfpathlineto{\pgfqpoint{2.741933in}{2.830366in}}%
\pgfpathlineto{\pgfqpoint{2.750212in}{2.832572in}}%
\pgfpathlineto{\pgfqpoint{2.758478in}{2.834940in}}%
\pgfpathlineto{\pgfqpoint{2.745145in}{2.853846in}}%
\pgfpathlineto{\pgfqpoint{2.731805in}{2.872946in}}%
\pgfpathlineto{\pgfqpoint{2.718457in}{2.892241in}}%
\pgfpathlineto{\pgfqpoint{2.705103in}{2.911733in}}%
\pgfpathlineto{\pgfqpoint{2.696802in}{2.909646in}}%
\pgfpathlineto{\pgfqpoint{2.688489in}{2.907723in}}%
\pgfpathlineto{\pgfqpoint{2.680163in}{2.905969in}}%
\pgfpathlineto{\pgfqpoint{2.671824in}{2.904384in}}%
\pgfpathclose%
\pgfusepath{fill}%
\end{pgfscope}%
\begin{pgfscope}%
\pgfpathrectangle{\pgfqpoint{1.254980in}{0.150000in}}{\pgfqpoint{5.490039in}{5.490039in}}%
\pgfusepath{clip}%
\pgfsetbuttcap%
\pgfsetroundjoin%
\definecolor{currentfill}{rgb}{0.199430,0.387607,0.554642}%
\pgfsetfillcolor{currentfill}%
\pgfsetfillopacity{0.700000}%
\pgfsetlinewidth{0.000000pt}%
\definecolor{currentstroke}{rgb}{0.000000,0.000000,0.000000}%
\pgfsetstrokecolor{currentstroke}%
\pgfsetdash{}{0pt}%
\pgfpathmoveto{\pgfqpoint{5.436734in}{2.619768in}}%
\pgfpathlineto{\pgfqpoint{5.450524in}{2.624366in}}%
\pgfpathlineto{\pgfqpoint{5.464327in}{2.629077in}}%
\pgfpathlineto{\pgfqpoint{5.478144in}{2.633900in}}%
\pgfpathlineto{\pgfqpoint{5.491975in}{2.638836in}}%
\pgfpathlineto{\pgfqpoint{5.499152in}{2.647946in}}%
\pgfpathlineto{\pgfqpoint{5.506323in}{2.657009in}}%
\pgfpathlineto{\pgfqpoint{5.513488in}{2.666027in}}%
\pgfpathlineto{\pgfqpoint{5.520647in}{2.675000in}}%
\pgfpathlineto{\pgfqpoint{5.506826in}{2.670141in}}%
\pgfpathlineto{\pgfqpoint{5.493019in}{2.665395in}}%
\pgfpathlineto{\pgfqpoint{5.479226in}{2.660761in}}%
\pgfpathlineto{\pgfqpoint{5.465446in}{2.656240in}}%
\pgfpathlineto{\pgfqpoint{5.458277in}{2.647184in}}%
\pgfpathlineto{\pgfqpoint{5.451102in}{2.638087in}}%
\pgfpathlineto{\pgfqpoint{5.443921in}{2.628949in}}%
\pgfpathlineto{\pgfqpoint{5.436734in}{2.619768in}}%
\pgfpathclose%
\pgfusepath{fill}%
\end{pgfscope}%
\begin{pgfscope}%
\pgfpathrectangle{\pgfqpoint{1.254980in}{0.150000in}}{\pgfqpoint{5.490039in}{5.490039in}}%
\pgfusepath{clip}%
\pgfsetbuttcap%
\pgfsetroundjoin%
\definecolor{currentfill}{rgb}{0.280868,0.160771,0.472899}%
\pgfsetfillcolor{currentfill}%
\pgfsetfillopacity{0.700000}%
\pgfsetlinewidth{0.000000pt}%
\definecolor{currentstroke}{rgb}{0.000000,0.000000,0.000000}%
\pgfsetstrokecolor{currentstroke}%
\pgfsetdash{}{0pt}%
\pgfpathmoveto{\pgfqpoint{4.653747in}{2.126254in}}%
\pgfpathlineto{\pgfqpoint{4.667188in}{2.126594in}}%
\pgfpathlineto{\pgfqpoint{4.680638in}{2.127051in}}%
\pgfpathlineto{\pgfqpoint{4.694098in}{2.127625in}}%
\pgfpathlineto{\pgfqpoint{4.707568in}{2.128315in}}%
\pgfpathlineto{\pgfqpoint{4.715029in}{2.138827in}}%
\pgfpathlineto{\pgfqpoint{4.722484in}{2.149319in}}%
\pgfpathlineto{\pgfqpoint{4.729935in}{2.159793in}}%
\pgfpathlineto{\pgfqpoint{4.737380in}{2.170246in}}%
\pgfpathlineto{\pgfqpoint{4.723917in}{2.169469in}}%
\pgfpathlineto{\pgfqpoint{4.710464in}{2.168809in}}%
\pgfpathlineto{\pgfqpoint{4.697020in}{2.168267in}}%
\pgfpathlineto{\pgfqpoint{4.683587in}{2.167841in}}%
\pgfpathlineto{\pgfqpoint{4.676134in}{2.157468in}}%
\pgfpathlineto{\pgfqpoint{4.668677in}{2.147079in}}%
\pgfpathlineto{\pgfqpoint{4.661214in}{2.136674in}}%
\pgfpathlineto{\pgfqpoint{4.653747in}{2.126254in}}%
\pgfpathclose%
\pgfusepath{fill}%
\end{pgfscope}%
\begin{pgfscope}%
\pgfpathrectangle{\pgfqpoint{1.254980in}{0.150000in}}{\pgfqpoint{5.490039in}{5.490039in}}%
\pgfusepath{clip}%
\pgfsetbuttcap%
\pgfsetroundjoin%
\definecolor{currentfill}{rgb}{0.277134,0.185228,0.489898}%
\pgfsetfillcolor{currentfill}%
\pgfsetfillopacity{0.700000}%
\pgfsetlinewidth{0.000000pt}%
\definecolor{currentstroke}{rgb}{0.000000,0.000000,0.000000}%
\pgfsetstrokecolor{currentstroke}%
\pgfsetdash{}{0pt}%
\pgfpathmoveto{\pgfqpoint{4.737380in}{2.170246in}}%
\pgfpathlineto{\pgfqpoint{4.750854in}{2.171139in}}%
\pgfpathlineto{\pgfqpoint{4.764338in}{2.172149in}}%
\pgfpathlineto{\pgfqpoint{4.777832in}{2.173275in}}%
\pgfpathlineto{\pgfqpoint{4.791336in}{2.174517in}}%
\pgfpathlineto{\pgfqpoint{4.798770in}{2.185025in}}%
\pgfpathlineto{\pgfqpoint{4.806199in}{2.195510in}}%
\pgfpathlineto{\pgfqpoint{4.813624in}{2.205969in}}%
\pgfpathlineto{\pgfqpoint{4.821043in}{2.216403in}}%
\pgfpathlineto{\pgfqpoint{4.807545in}{2.215091in}}%
\pgfpathlineto{\pgfqpoint{4.794058in}{2.213895in}}%
\pgfpathlineto{\pgfqpoint{4.780581in}{2.212816in}}%
\pgfpathlineto{\pgfqpoint{4.767114in}{2.211852in}}%
\pgfpathlineto{\pgfqpoint{4.759688in}{2.201482in}}%
\pgfpathlineto{\pgfqpoint{4.752257in}{2.191091in}}%
\pgfpathlineto{\pgfqpoint{4.744821in}{2.180679in}}%
\pgfpathlineto{\pgfqpoint{4.737380in}{2.170246in}}%
\pgfpathclose%
\pgfusepath{fill}%
\end{pgfscope}%
\begin{pgfscope}%
\pgfpathrectangle{\pgfqpoint{1.254980in}{0.150000in}}{\pgfqpoint{5.490039in}{5.490039in}}%
\pgfusepath{clip}%
\pgfsetbuttcap%
\pgfsetroundjoin%
\definecolor{currentfill}{rgb}{0.280868,0.160771,0.472899}%
\pgfsetfillcolor{currentfill}%
\pgfsetfillopacity{0.700000}%
\pgfsetlinewidth{0.000000pt}%
\definecolor{currentstroke}{rgb}{0.000000,0.000000,0.000000}%
\pgfsetstrokecolor{currentstroke}%
\pgfsetdash{}{0pt}%
\pgfpathmoveto{\pgfqpoint{3.308915in}{2.157477in}}%
\pgfpathlineto{\pgfqpoint{3.322125in}{2.145925in}}%
\pgfpathlineto{\pgfqpoint{3.335335in}{2.134523in}}%
\pgfpathlineto{\pgfqpoint{3.348544in}{2.123270in}}%
\pgfpathlineto{\pgfqpoint{3.361754in}{2.112165in}}%
\pgfpathlineto{\pgfqpoint{3.369712in}{2.117419in}}%
\pgfpathlineto{\pgfqpoint{3.377662in}{2.122787in}}%
\pgfpathlineto{\pgfqpoint{3.385604in}{2.128266in}}%
\pgfpathlineto{\pgfqpoint{3.393537in}{2.133854in}}%
\pgfpathlineto{\pgfqpoint{3.380351in}{2.144692in}}%
\pgfpathlineto{\pgfqpoint{3.367164in}{2.155678in}}%
\pgfpathlineto{\pgfqpoint{3.353977in}{2.166813in}}%
\pgfpathlineto{\pgfqpoint{3.340790in}{2.178097in}}%
\pgfpathlineto{\pgfqpoint{3.332834in}{2.172770in}}%
\pgfpathlineto{\pgfqpoint{3.324870in}{2.167556in}}%
\pgfpathlineto{\pgfqpoint{3.316896in}{2.162458in}}%
\pgfpathlineto{\pgfqpoint{3.308915in}{2.157477in}}%
\pgfpathclose%
\pgfusepath{fill}%
\end{pgfscope}%
\begin{pgfscope}%
\pgfpathrectangle{\pgfqpoint{1.254980in}{0.150000in}}{\pgfqpoint{5.490039in}{5.490039in}}%
\pgfusepath{clip}%
\pgfsetbuttcap%
\pgfsetroundjoin%
\definecolor{currentfill}{rgb}{0.188923,0.410910,0.556326}%
\pgfsetfillcolor{currentfill}%
\pgfsetfillopacity{0.700000}%
\pgfsetlinewidth{0.000000pt}%
\definecolor{currentstroke}{rgb}{0.000000,0.000000,0.000000}%
\pgfsetstrokecolor{currentstroke}%
\pgfsetdash{}{0pt}%
\pgfpathmoveto{\pgfqpoint{5.520647in}{2.675000in}}%
\pgfpathlineto{\pgfqpoint{5.534482in}{2.679970in}}%
\pgfpathlineto{\pgfqpoint{5.548331in}{2.685053in}}%
\pgfpathlineto{\pgfqpoint{5.562194in}{2.690248in}}%
\pgfpathlineto{\pgfqpoint{5.576072in}{2.695556in}}%
\pgfpathlineto{\pgfqpoint{5.583215in}{2.704397in}}%
\pgfpathlineto{\pgfqpoint{5.590351in}{2.713193in}}%
\pgfpathlineto{\pgfqpoint{5.597482in}{2.721943in}}%
\pgfpathlineto{\pgfqpoint{5.604606in}{2.730648in}}%
\pgfpathlineto{\pgfqpoint{5.590739in}{2.725435in}}%
\pgfpathlineto{\pgfqpoint{5.576887in}{2.720334in}}%
\pgfpathlineto{\pgfqpoint{5.563048in}{2.715345in}}%
\pgfpathlineto{\pgfqpoint{5.549224in}{2.710468in}}%
\pgfpathlineto{\pgfqpoint{5.542089in}{2.701662in}}%
\pgfpathlineto{\pgfqpoint{5.534948in}{2.692817in}}%
\pgfpathlineto{\pgfqpoint{5.527800in}{2.683929in}}%
\pgfpathlineto{\pgfqpoint{5.520647in}{2.675000in}}%
\pgfpathclose%
\pgfusepath{fill}%
\end{pgfscope}%
\begin{pgfscope}%
\pgfpathrectangle{\pgfqpoint{1.254980in}{0.150000in}}{\pgfqpoint{5.490039in}{5.490039in}}%
\pgfusepath{clip}%
\pgfsetbuttcap%
\pgfsetroundjoin%
\definecolor{currentfill}{rgb}{0.282623,0.140926,0.457517}%
\pgfsetfillcolor{currentfill}%
\pgfsetfillopacity{0.700000}%
\pgfsetlinewidth{0.000000pt}%
\definecolor{currentstroke}{rgb}{0.000000,0.000000,0.000000}%
\pgfsetstrokecolor{currentstroke}%
\pgfsetdash{}{0pt}%
\pgfpathmoveto{\pgfqpoint{4.570135in}{2.084661in}}%
\pgfpathlineto{\pgfqpoint{4.583545in}{2.084428in}}%
\pgfpathlineto{\pgfqpoint{4.596964in}{2.084313in}}%
\pgfpathlineto{\pgfqpoint{4.610392in}{2.084316in}}%
\pgfpathlineto{\pgfqpoint{4.623830in}{2.084437in}}%
\pgfpathlineto{\pgfqpoint{4.631317in}{2.094911in}}%
\pgfpathlineto{\pgfqpoint{4.638798in}{2.105372in}}%
\pgfpathlineto{\pgfqpoint{4.646275in}{2.115820in}}%
\pgfpathlineto{\pgfqpoint{4.653747in}{2.126254in}}%
\pgfpathlineto{\pgfqpoint{4.640316in}{2.126032in}}%
\pgfpathlineto{\pgfqpoint{4.626895in}{2.125927in}}%
\pgfpathlineto{\pgfqpoint{4.613483in}{2.125940in}}%
\pgfpathlineto{\pgfqpoint{4.600080in}{2.126071in}}%
\pgfpathlineto{\pgfqpoint{4.592601in}{2.115733in}}%
\pgfpathlineto{\pgfqpoint{4.585117in}{2.105385in}}%
\pgfpathlineto{\pgfqpoint{4.577628in}{2.095027in}}%
\pgfpathlineto{\pgfqpoint{4.570135in}{2.084661in}}%
\pgfpathclose%
\pgfusepath{fill}%
\end{pgfscope}%
\begin{pgfscope}%
\pgfpathrectangle{\pgfqpoint{1.254980in}{0.150000in}}{\pgfqpoint{5.490039in}{5.490039in}}%
\pgfusepath{clip}%
\pgfsetbuttcap%
\pgfsetroundjoin%
\definecolor{currentfill}{rgb}{0.271828,0.209303,0.504434}%
\pgfsetfillcolor{currentfill}%
\pgfsetfillopacity{0.700000}%
\pgfsetlinewidth{0.000000pt}%
\definecolor{currentstroke}{rgb}{0.000000,0.000000,0.000000}%
\pgfsetstrokecolor{currentstroke}%
\pgfsetdash{}{0pt}%
\pgfpathmoveto{\pgfqpoint{4.821043in}{2.216403in}}%
\pgfpathlineto{\pgfqpoint{4.834551in}{2.217831in}}%
\pgfpathlineto{\pgfqpoint{4.848070in}{2.219375in}}%
\pgfpathlineto{\pgfqpoint{4.861600in}{2.221034in}}%
\pgfpathlineto{\pgfqpoint{4.875141in}{2.222809in}}%
\pgfpathlineto{\pgfqpoint{4.882549in}{2.233278in}}%
\pgfpathlineto{\pgfqpoint{4.889951in}{2.243716in}}%
\pgfpathlineto{\pgfqpoint{4.897349in}{2.254125in}}%
\pgfpathlineto{\pgfqpoint{4.904741in}{2.264504in}}%
\pgfpathlineto{\pgfqpoint{4.891207in}{2.262675in}}%
\pgfpathlineto{\pgfqpoint{4.877684in}{2.260961in}}%
\pgfpathlineto{\pgfqpoint{4.864171in}{2.259364in}}%
\pgfpathlineto{\pgfqpoint{4.850670in}{2.257882in}}%
\pgfpathlineto{\pgfqpoint{4.843271in}{2.247551in}}%
\pgfpathlineto{\pgfqpoint{4.835866in}{2.237194in}}%
\pgfpathlineto{\pgfqpoint{4.828457in}{2.226812in}}%
\pgfpathlineto{\pgfqpoint{4.821043in}{2.216403in}}%
\pgfpathclose%
\pgfusepath{fill}%
\end{pgfscope}%
\begin{pgfscope}%
\pgfpathrectangle{\pgfqpoint{1.254980in}{0.150000in}}{\pgfqpoint{5.490039in}{5.490039in}}%
\pgfusepath{clip}%
\pgfsetbuttcap%
\pgfsetroundjoin%
\definecolor{currentfill}{rgb}{0.281924,0.089666,0.412415}%
\pgfsetfillcolor{currentfill}%
\pgfsetfillopacity{0.700000}%
\pgfsetlinewidth{0.000000pt}%
\definecolor{currentstroke}{rgb}{0.000000,0.000000,0.000000}%
\pgfsetstrokecolor{currentstroke}%
\pgfsetdash{}{0pt}%
\pgfpathmoveto{\pgfqpoint{3.551844in}{2.015091in}}%
\pgfpathlineto{\pgfqpoint{3.565045in}{2.006116in}}%
\pgfpathlineto{\pgfqpoint{3.578249in}{1.997281in}}%
\pgfpathlineto{\pgfqpoint{3.591454in}{1.988584in}}%
\pgfpathlineto{\pgfqpoint{3.604661in}{1.980025in}}%
\pgfpathlineto{\pgfqpoint{3.612505in}{1.986739in}}%
\pgfpathlineto{\pgfqpoint{3.620342in}{1.993542in}}%
\pgfpathlineto{\pgfqpoint{3.628173in}{2.000431in}}%
\pgfpathlineto{\pgfqpoint{3.635996in}{2.007406in}}%
\pgfpathlineto{\pgfqpoint{3.622807in}{2.015718in}}%
\pgfpathlineto{\pgfqpoint{3.609621in}{2.024167in}}%
\pgfpathlineto{\pgfqpoint{3.596437in}{2.032755in}}%
\pgfpathlineto{\pgfqpoint{3.583254in}{2.041482in}}%
\pgfpathlineto{\pgfqpoint{3.575413in}{2.034749in}}%
\pgfpathlineto{\pgfqpoint{3.567564in}{2.028105in}}%
\pgfpathlineto{\pgfqpoint{3.559708in}{2.021551in}}%
\pgfpathlineto{\pgfqpoint{3.551844in}{2.015091in}}%
\pgfpathclose%
\pgfusepath{fill}%
\end{pgfscope}%
\begin{pgfscope}%
\pgfpathrectangle{\pgfqpoint{1.254980in}{0.150000in}}{\pgfqpoint{5.490039in}{5.490039in}}%
\pgfusepath{clip}%
\pgfsetbuttcap%
\pgfsetroundjoin%
\definecolor{currentfill}{rgb}{0.283229,0.120777,0.440584}%
\pgfsetfillcolor{currentfill}%
\pgfsetfillopacity{0.700000}%
\pgfsetlinewidth{0.000000pt}%
\definecolor{currentstroke}{rgb}{0.000000,0.000000,0.000000}%
\pgfsetstrokecolor{currentstroke}%
\pgfsetdash{}{0pt}%
\pgfpathmoveto{\pgfqpoint{4.486533in}{2.045707in}}%
\pgfpathlineto{\pgfqpoint{4.499915in}{2.044882in}}%
\pgfpathlineto{\pgfqpoint{4.513305in}{2.044177in}}%
\pgfpathlineto{\pgfqpoint{4.526704in}{2.043590in}}%
\pgfpathlineto{\pgfqpoint{4.540113in}{2.043121in}}%
\pgfpathlineto{\pgfqpoint{4.547625in}{2.053516in}}%
\pgfpathlineto{\pgfqpoint{4.555133in}{2.063905in}}%
\pgfpathlineto{\pgfqpoint{4.562636in}{2.074286in}}%
\pgfpathlineto{\pgfqpoint{4.570135in}{2.084661in}}%
\pgfpathlineto{\pgfqpoint{4.556734in}{2.085012in}}%
\pgfpathlineto{\pgfqpoint{4.543342in}{2.085481in}}%
\pgfpathlineto{\pgfqpoint{4.529959in}{2.086069in}}%
\pgfpathlineto{\pgfqpoint{4.516585in}{2.086776in}}%
\pgfpathlineto{\pgfqpoint{4.509080in}{2.076513in}}%
\pgfpathlineto{\pgfqpoint{4.501569in}{2.066247in}}%
\pgfpathlineto{\pgfqpoint{4.494054in}{2.055978in}}%
\pgfpathlineto{\pgfqpoint{4.486533in}{2.045707in}}%
\pgfpathclose%
\pgfusepath{fill}%
\end{pgfscope}%
\begin{pgfscope}%
\pgfpathrectangle{\pgfqpoint{1.254980in}{0.150000in}}{\pgfqpoint{5.490039in}{5.490039in}}%
\pgfusepath{clip}%
\pgfsetbuttcap%
\pgfsetroundjoin%
\definecolor{currentfill}{rgb}{0.265145,0.232956,0.516599}%
\pgfsetfillcolor{currentfill}%
\pgfsetfillopacity{0.700000}%
\pgfsetlinewidth{0.000000pt}%
\definecolor{currentstroke}{rgb}{0.000000,0.000000,0.000000}%
\pgfsetstrokecolor{currentstroke}%
\pgfsetdash{}{0pt}%
\pgfpathmoveto{\pgfqpoint{4.904741in}{2.264504in}}%
\pgfpathlineto{\pgfqpoint{4.918287in}{2.266448in}}%
\pgfpathlineto{\pgfqpoint{4.931843in}{2.268507in}}%
\pgfpathlineto{\pgfqpoint{4.945411in}{2.270682in}}%
\pgfpathlineto{\pgfqpoint{4.958990in}{2.272971in}}%
\pgfpathlineto{\pgfqpoint{4.966371in}{2.283364in}}%
\pgfpathlineto{\pgfqpoint{4.973746in}{2.293722in}}%
\pgfpathlineto{\pgfqpoint{4.981117in}{2.304045in}}%
\pgfpathlineto{\pgfqpoint{4.988482in}{2.314334in}}%
\pgfpathlineto{\pgfqpoint{4.974910in}{2.312007in}}%
\pgfpathlineto{\pgfqpoint{4.961349in}{2.309795in}}%
\pgfpathlineto{\pgfqpoint{4.947799in}{2.307698in}}%
\pgfpathlineto{\pgfqpoint{4.934260in}{2.305715in}}%
\pgfpathlineto{\pgfqpoint{4.926888in}{2.295458in}}%
\pgfpathlineto{\pgfqpoint{4.919511in}{2.285170in}}%
\pgfpathlineto{\pgfqpoint{4.912129in}{2.274852in}}%
\pgfpathlineto{\pgfqpoint{4.904741in}{2.264504in}}%
\pgfpathclose%
\pgfusepath{fill}%
\end{pgfscope}%
\begin{pgfscope}%
\pgfpathrectangle{\pgfqpoint{1.254980in}{0.150000in}}{\pgfqpoint{5.490039in}{5.490039in}}%
\pgfusepath{clip}%
\pgfsetbuttcap%
\pgfsetroundjoin%
\definecolor{currentfill}{rgb}{0.150476,0.504369,0.557430}%
\pgfsetfillcolor{currentfill}%
\pgfsetfillopacity{0.700000}%
\pgfsetlinewidth{0.000000pt}%
\definecolor{currentstroke}{rgb}{0.000000,0.000000,0.000000}%
\pgfsetstrokecolor{currentstroke}%
\pgfsetdash{}{0pt}%
\pgfpathmoveto{\pgfqpoint{2.618186in}{2.985501in}}%
\pgfpathlineto{\pgfqpoint{2.631608in}{2.964919in}}%
\pgfpathlineto{\pgfqpoint{2.645022in}{2.944540in}}%
\pgfpathlineto{\pgfqpoint{2.658427in}{2.924362in}}%
\pgfpathlineto{\pgfqpoint{2.671824in}{2.904384in}}%
\pgfpathlineto{\pgfqpoint{2.680163in}{2.905969in}}%
\pgfpathlineto{\pgfqpoint{2.688489in}{2.907723in}}%
\pgfpathlineto{\pgfqpoint{2.696802in}{2.909646in}}%
\pgfpathlineto{\pgfqpoint{2.705103in}{2.911733in}}%
\pgfpathlineto{\pgfqpoint{2.691741in}{2.931423in}}%
\pgfpathlineto{\pgfqpoint{2.678371in}{2.951312in}}%
\pgfpathlineto{\pgfqpoint{2.664994in}{2.971402in}}%
\pgfpathlineto{\pgfqpoint{2.651608in}{2.991694in}}%
\pgfpathlineto{\pgfqpoint{2.643272in}{2.989889in}}%
\pgfpathlineto{\pgfqpoint{2.634923in}{2.988253in}}%
\pgfpathlineto{\pgfqpoint{2.626561in}{2.986790in}}%
\pgfpathlineto{\pgfqpoint{2.618186in}{2.985501in}}%
\pgfpathclose%
\pgfusepath{fill}%
\end{pgfscope}%
\begin{pgfscope}%
\pgfpathrectangle{\pgfqpoint{1.254980in}{0.150000in}}{\pgfqpoint{5.490039in}{5.490039in}}%
\pgfusepath{clip}%
\pgfsetbuttcap%
\pgfsetroundjoin%
\definecolor{currentfill}{rgb}{0.277941,0.056324,0.381191}%
\pgfsetfillcolor{currentfill}%
\pgfsetfillopacity{0.700000}%
\pgfsetlinewidth{0.000000pt}%
\definecolor{currentstroke}{rgb}{0.000000,0.000000,0.000000}%
\pgfsetstrokecolor{currentstroke}%
\pgfsetdash{}{0pt}%
\pgfpathmoveto{\pgfqpoint{4.098863in}{1.936249in}}%
\pgfpathlineto{\pgfqpoint{4.112133in}{1.932365in}}%
\pgfpathlineto{\pgfqpoint{4.125410in}{1.928605in}}%
\pgfpathlineto{\pgfqpoint{4.138692in}{1.924969in}}%
\pgfpathlineto{\pgfqpoint{4.151981in}{1.921456in}}%
\pgfpathlineto{\pgfqpoint{4.159616in}{1.930881in}}%
\pgfpathlineto{\pgfqpoint{4.167246in}{1.940337in}}%
\pgfpathlineto{\pgfqpoint{4.174870in}{1.949821in}}%
\pgfpathlineto{\pgfqpoint{4.182490in}{1.959334in}}%
\pgfpathlineto{\pgfqpoint{4.169211in}{1.962666in}}%
\pgfpathlineto{\pgfqpoint{4.155939in}{1.966121in}}%
\pgfpathlineto{\pgfqpoint{4.142673in}{1.969700in}}%
\pgfpathlineto{\pgfqpoint{4.129414in}{1.973403in}}%
\pgfpathlineto{\pgfqpoint{4.121784in}{1.964065in}}%
\pgfpathlineto{\pgfqpoint{4.114149in}{1.954760in}}%
\pgfpathlineto{\pgfqpoint{4.106508in}{1.945487in}}%
\pgfpathlineto{\pgfqpoint{4.098863in}{1.936249in}}%
\pgfpathclose%
\pgfusepath{fill}%
\end{pgfscope}%
\begin{pgfscope}%
\pgfpathrectangle{\pgfqpoint{1.254980in}{0.150000in}}{\pgfqpoint{5.490039in}{5.490039in}}%
\pgfusepath{clip}%
\pgfsetbuttcap%
\pgfsetroundjoin%
\definecolor{currentfill}{rgb}{0.179019,0.433756,0.557430}%
\pgfsetfillcolor{currentfill}%
\pgfsetfillopacity{0.700000}%
\pgfsetlinewidth{0.000000pt}%
\definecolor{currentstroke}{rgb}{0.000000,0.000000,0.000000}%
\pgfsetstrokecolor{currentstroke}%
\pgfsetdash{}{0pt}%
\pgfpathmoveto{\pgfqpoint{5.604606in}{2.730648in}}%
\pgfpathlineto{\pgfqpoint{5.618487in}{2.735973in}}%
\pgfpathlineto{\pgfqpoint{5.632383in}{2.741410in}}%
\pgfpathlineto{\pgfqpoint{5.646293in}{2.746959in}}%
\pgfpathlineto{\pgfqpoint{5.660218in}{2.752620in}}%
\pgfpathlineto{\pgfqpoint{5.667325in}{2.761177in}}%
\pgfpathlineto{\pgfqpoint{5.674426in}{2.769689in}}%
\pgfpathlineto{\pgfqpoint{5.681520in}{2.778156in}}%
\pgfpathlineto{\pgfqpoint{5.688609in}{2.786579in}}%
\pgfpathlineto{\pgfqpoint{5.674696in}{2.781030in}}%
\pgfpathlineto{\pgfqpoint{5.660797in}{2.775592in}}%
\pgfpathlineto{\pgfqpoint{5.646913in}{2.770265in}}%
\pgfpathlineto{\pgfqpoint{5.633043in}{2.765051in}}%
\pgfpathlineto{\pgfqpoint{5.625943in}{2.756511in}}%
\pgfpathlineto{\pgfqpoint{5.618837in}{2.747931in}}%
\pgfpathlineto{\pgfqpoint{5.611724in}{2.739310in}}%
\pgfpathlineto{\pgfqpoint{5.604606in}{2.730648in}}%
\pgfpathclose%
\pgfusepath{fill}%
\end{pgfscope}%
\begin{pgfscope}%
\pgfpathrectangle{\pgfqpoint{1.254980in}{0.150000in}}{\pgfqpoint{5.490039in}{5.490039in}}%
\pgfusepath{clip}%
\pgfsetbuttcap%
\pgfsetroundjoin%
\definecolor{currentfill}{rgb}{0.257322,0.256130,0.526563}%
\pgfsetfillcolor{currentfill}%
\pgfsetfillopacity{0.700000}%
\pgfsetlinewidth{0.000000pt}%
\definecolor{currentstroke}{rgb}{0.000000,0.000000,0.000000}%
\pgfsetstrokecolor{currentstroke}%
\pgfsetdash{}{0pt}%
\pgfpathmoveto{\pgfqpoint{4.988482in}{2.314334in}}%
\pgfpathlineto{\pgfqpoint{5.002066in}{2.316777in}}%
\pgfpathlineto{\pgfqpoint{5.015661in}{2.319333in}}%
\pgfpathlineto{\pgfqpoint{5.029269in}{2.322005in}}%
\pgfpathlineto{\pgfqpoint{5.042888in}{2.324791in}}%
\pgfpathlineto{\pgfqpoint{5.050241in}{2.335073in}}%
\pgfpathlineto{\pgfqpoint{5.057589in}{2.345317in}}%
\pgfpathlineto{\pgfqpoint{5.064932in}{2.355523in}}%
\pgfpathlineto{\pgfqpoint{5.072269in}{2.365691in}}%
\pgfpathlineto{\pgfqpoint{5.058657in}{2.362884in}}%
\pgfpathlineto{\pgfqpoint{5.045056in}{2.360191in}}%
\pgfpathlineto{\pgfqpoint{5.031467in}{2.357612in}}%
\pgfpathlineto{\pgfqpoint{5.017890in}{2.355149in}}%
\pgfpathlineto{\pgfqpoint{5.010546in}{2.344996in}}%
\pgfpathlineto{\pgfqpoint{5.003197in}{2.334810in}}%
\pgfpathlineto{\pgfqpoint{4.995842in}{2.324589in}}%
\pgfpathlineto{\pgfqpoint{4.988482in}{2.314334in}}%
\pgfpathclose%
\pgfusepath{fill}%
\end{pgfscope}%
\begin{pgfscope}%
\pgfpathrectangle{\pgfqpoint{1.254980in}{0.150000in}}{\pgfqpoint{5.490039in}{5.490039in}}%
\pgfusepath{clip}%
\pgfsetbuttcap%
\pgfsetroundjoin%
\definecolor{currentfill}{rgb}{0.282910,0.105393,0.426902}%
\pgfsetfillcolor{currentfill}%
\pgfsetfillopacity{0.700000}%
\pgfsetlinewidth{0.000000pt}%
\definecolor{currentstroke}{rgb}{0.000000,0.000000,0.000000}%
\pgfsetstrokecolor{currentstroke}%
\pgfsetdash{}{0pt}%
\pgfpathmoveto{\pgfqpoint{4.402931in}{2.009643in}}%
\pgfpathlineto{\pgfqpoint{4.416287in}{2.008207in}}%
\pgfpathlineto{\pgfqpoint{4.429651in}{2.006892in}}%
\pgfpathlineto{\pgfqpoint{4.443024in}{2.005695in}}%
\pgfpathlineto{\pgfqpoint{4.456404in}{2.004618in}}%
\pgfpathlineto{\pgfqpoint{4.463944in}{2.014889in}}%
\pgfpathlineto{\pgfqpoint{4.471478in}{2.025162in}}%
\pgfpathlineto{\pgfqpoint{4.479008in}{2.035434in}}%
\pgfpathlineto{\pgfqpoint{4.486533in}{2.045707in}}%
\pgfpathlineto{\pgfqpoint{4.473160in}{2.046650in}}%
\pgfpathlineto{\pgfqpoint{4.459796in}{2.047713in}}%
\pgfpathlineto{\pgfqpoint{4.446440in}{2.048895in}}%
\pgfpathlineto{\pgfqpoint{4.433092in}{2.050197in}}%
\pgfpathlineto{\pgfqpoint{4.425559in}{2.040052in}}%
\pgfpathlineto{\pgfqpoint{4.418021in}{2.029911in}}%
\pgfpathlineto{\pgfqpoint{4.410479in}{2.019774in}}%
\pgfpathlineto{\pgfqpoint{4.402931in}{2.009643in}}%
\pgfpathclose%
\pgfusepath{fill}%
\end{pgfscope}%
\begin{pgfscope}%
\pgfpathrectangle{\pgfqpoint{1.254980in}{0.150000in}}{\pgfqpoint{5.490039in}{5.490039in}}%
\pgfusepath{clip}%
\pgfsetbuttcap%
\pgfsetroundjoin%
\definecolor{currentfill}{rgb}{0.282623,0.140926,0.457517}%
\pgfsetfillcolor{currentfill}%
\pgfsetfillopacity{0.700000}%
\pgfsetlinewidth{0.000000pt}%
\definecolor{currentstroke}{rgb}{0.000000,0.000000,0.000000}%
\pgfsetstrokecolor{currentstroke}%
\pgfsetdash{}{0pt}%
\pgfpathmoveto{\pgfqpoint{3.361754in}{2.112165in}}%
\pgfpathlineto{\pgfqpoint{3.374964in}{2.101208in}}%
\pgfpathlineto{\pgfqpoint{3.388174in}{2.090397in}}%
\pgfpathlineto{\pgfqpoint{3.401385in}{2.079732in}}%
\pgfpathlineto{\pgfqpoint{3.414596in}{2.069213in}}%
\pgfpathlineto{\pgfqpoint{3.422531in}{2.074739in}}%
\pgfpathlineto{\pgfqpoint{3.430459in}{2.080375in}}%
\pgfpathlineto{\pgfqpoint{3.438378in}{2.086118in}}%
\pgfpathlineto{\pgfqpoint{3.446290in}{2.091967in}}%
\pgfpathlineto{\pgfqpoint{3.433101in}{2.102220in}}%
\pgfpathlineto{\pgfqpoint{3.419913in}{2.112618in}}%
\pgfpathlineto{\pgfqpoint{3.406725in}{2.123163in}}%
\pgfpathlineto{\pgfqpoint{3.393537in}{2.133854in}}%
\pgfpathlineto{\pgfqpoint{3.385604in}{2.128266in}}%
\pgfpathlineto{\pgfqpoint{3.377662in}{2.122787in}}%
\pgfpathlineto{\pgfqpoint{3.369712in}{2.117419in}}%
\pgfpathlineto{\pgfqpoint{3.361754in}{2.112165in}}%
\pgfpathclose%
\pgfusepath{fill}%
\end{pgfscope}%
\begin{pgfscope}%
\pgfpathrectangle{\pgfqpoint{1.254980in}{0.150000in}}{\pgfqpoint{5.490039in}{5.490039in}}%
\pgfusepath{clip}%
\pgfsetbuttcap%
\pgfsetroundjoin%
\definecolor{currentfill}{rgb}{0.277018,0.050344,0.375715}%
\pgfsetfillcolor{currentfill}%
\pgfsetfillopacity{0.700000}%
\pgfsetlinewidth{0.000000pt}%
\definecolor{currentstroke}{rgb}{0.000000,0.000000,0.000000}%
\pgfsetstrokecolor{currentstroke}%
\pgfsetdash{}{0pt}%
\pgfpathmoveto{\pgfqpoint{3.878401in}{1.925133in}}%
\pgfpathlineto{\pgfqpoint{3.891633in}{1.919316in}}%
\pgfpathlineto{\pgfqpoint{3.904870in}{1.913628in}}%
\pgfpathlineto{\pgfqpoint{3.918111in}{1.908069in}}%
\pgfpathlineto{\pgfqpoint{3.931357in}{1.902638in}}%
\pgfpathlineto{\pgfqpoint{3.939070in}{1.911110in}}%
\pgfpathlineto{\pgfqpoint{3.946778in}{1.919636in}}%
\pgfpathlineto{\pgfqpoint{3.954481in}{1.928215in}}%
\pgfpathlineto{\pgfqpoint{3.962177in}{1.936845in}}%
\pgfpathlineto{\pgfqpoint{3.948945in}{1.942063in}}%
\pgfpathlineto{\pgfqpoint{3.935717in}{1.947409in}}%
\pgfpathlineto{\pgfqpoint{3.922495in}{1.952883in}}%
\pgfpathlineto{\pgfqpoint{3.909277in}{1.958487in}}%
\pgfpathlineto{\pgfqpoint{3.901567in}{1.950064in}}%
\pgfpathlineto{\pgfqpoint{3.893851in}{1.941696in}}%
\pgfpathlineto{\pgfqpoint{3.886129in}{1.933386in}}%
\pgfpathlineto{\pgfqpoint{3.878401in}{1.925133in}}%
\pgfpathclose%
\pgfusepath{fill}%
\end{pgfscope}%
\begin{pgfscope}%
\pgfpathrectangle{\pgfqpoint{1.254980in}{0.150000in}}{\pgfqpoint{5.490039in}{5.490039in}}%
\pgfusepath{clip}%
\pgfsetbuttcap%
\pgfsetroundjoin%
\definecolor{currentfill}{rgb}{0.278791,0.062145,0.386592}%
\pgfsetfillcolor{currentfill}%
\pgfsetfillopacity{0.700000}%
\pgfsetlinewidth{0.000000pt}%
\definecolor{currentstroke}{rgb}{0.000000,0.000000,0.000000}%
\pgfsetstrokecolor{currentstroke}%
\pgfsetdash{}{0pt}%
\pgfpathmoveto{\pgfqpoint{3.741606in}{1.945810in}}%
\pgfpathlineto{\pgfqpoint{3.754821in}{1.938715in}}%
\pgfpathlineto{\pgfqpoint{3.768040in}{1.931753in}}%
\pgfpathlineto{\pgfqpoint{3.781262in}{1.924923in}}%
\pgfpathlineto{\pgfqpoint{3.794488in}{1.918224in}}%
\pgfpathlineto{\pgfqpoint{3.802255in}{1.925991in}}%
\pgfpathlineto{\pgfqpoint{3.810015in}{1.933827in}}%
\pgfpathlineto{\pgfqpoint{3.817769in}{1.941730in}}%
\pgfpathlineto{\pgfqpoint{3.825517in}{1.949699in}}%
\pgfpathlineto{\pgfqpoint{3.812307in}{1.956168in}}%
\pgfpathlineto{\pgfqpoint{3.799100in}{1.962768in}}%
\pgfpathlineto{\pgfqpoint{3.785897in}{1.969500in}}%
\pgfpathlineto{\pgfqpoint{3.772698in}{1.976365in}}%
\pgfpathlineto{\pgfqpoint{3.764934in}{1.968620in}}%
\pgfpathlineto{\pgfqpoint{3.757164in}{1.960945in}}%
\pgfpathlineto{\pgfqpoint{3.749388in}{1.953341in}}%
\pgfpathlineto{\pgfqpoint{3.741606in}{1.945810in}}%
\pgfpathclose%
\pgfusepath{fill}%
\end{pgfscope}%
\begin{pgfscope}%
\pgfpathrectangle{\pgfqpoint{1.254980in}{0.150000in}}{\pgfqpoint{5.490039in}{5.490039in}}%
\pgfusepath{clip}%
\pgfsetbuttcap%
\pgfsetroundjoin%
\definecolor{currentfill}{rgb}{0.246811,0.283237,0.535941}%
\pgfsetfillcolor{currentfill}%
\pgfsetfillopacity{0.700000}%
\pgfsetlinewidth{0.000000pt}%
\definecolor{currentstroke}{rgb}{0.000000,0.000000,0.000000}%
\pgfsetstrokecolor{currentstroke}%
\pgfsetdash{}{0pt}%
\pgfpathmoveto{\pgfqpoint{5.072269in}{2.365691in}}%
\pgfpathlineto{\pgfqpoint{5.085893in}{2.368613in}}%
\pgfpathlineto{\pgfqpoint{5.099529in}{2.371649in}}%
\pgfpathlineto{\pgfqpoint{5.113178in}{2.374799in}}%
\pgfpathlineto{\pgfqpoint{5.126838in}{2.378063in}}%
\pgfpathlineto{\pgfqpoint{5.134163in}{2.388204in}}%
\pgfpathlineto{\pgfqpoint{5.141483in}{2.398303in}}%
\pgfpathlineto{\pgfqpoint{5.148797in}{2.408361in}}%
\pgfpathlineto{\pgfqpoint{5.156105in}{2.418378in}}%
\pgfpathlineto{\pgfqpoint{5.142452in}{2.415109in}}%
\pgfpathlineto{\pgfqpoint{5.128810in}{2.411954in}}%
\pgfpathlineto{\pgfqpoint{5.115181in}{2.408912in}}%
\pgfpathlineto{\pgfqpoint{5.101564in}{2.405985in}}%
\pgfpathlineto{\pgfqpoint{5.094248in}{2.395968in}}%
\pgfpathlineto{\pgfqpoint{5.086927in}{2.385913in}}%
\pgfpathlineto{\pgfqpoint{5.079601in}{2.375821in}}%
\pgfpathlineto{\pgfqpoint{5.072269in}{2.365691in}}%
\pgfpathclose%
\pgfusepath{fill}%
\end{pgfscope}%
\begin{pgfscope}%
\pgfpathrectangle{\pgfqpoint{1.254980in}{0.150000in}}{\pgfqpoint{5.490039in}{5.490039in}}%
\pgfusepath{clip}%
\pgfsetbuttcap%
\pgfsetroundjoin%
\definecolor{currentfill}{rgb}{0.169646,0.456262,0.558030}%
\pgfsetfillcolor{currentfill}%
\pgfsetfillopacity{0.700000}%
\pgfsetlinewidth{0.000000pt}%
\definecolor{currentstroke}{rgb}{0.000000,0.000000,0.000000}%
\pgfsetstrokecolor{currentstroke}%
\pgfsetdash{}{0pt}%
\pgfpathmoveto{\pgfqpoint{5.688609in}{2.786579in}}%
\pgfpathlineto{\pgfqpoint{5.702537in}{2.792241in}}%
\pgfpathlineto{\pgfqpoint{5.716480in}{2.798014in}}%
\pgfpathlineto{\pgfqpoint{5.730438in}{2.803898in}}%
\pgfpathlineto{\pgfqpoint{5.744411in}{2.809894in}}%
\pgfpathlineto{\pgfqpoint{5.751480in}{2.818154in}}%
\pgfpathlineto{\pgfqpoint{5.758544in}{2.826369in}}%
\pgfpathlineto{\pgfqpoint{5.765601in}{2.834540in}}%
\pgfpathlineto{\pgfqpoint{5.772652in}{2.842669in}}%
\pgfpathlineto{\pgfqpoint{5.758692in}{2.836801in}}%
\pgfpathlineto{\pgfqpoint{5.744747in}{2.831044in}}%
\pgfpathlineto{\pgfqpoint{5.730816in}{2.825399in}}%
\pgfpathlineto{\pgfqpoint{5.716901in}{2.819865in}}%
\pgfpathlineto{\pgfqpoint{5.709837in}{2.811602in}}%
\pgfpathlineto{\pgfqpoint{5.702767in}{2.803301in}}%
\pgfpathlineto{\pgfqpoint{5.695691in}{2.794961in}}%
\pgfpathlineto{\pgfqpoint{5.688609in}{2.786579in}}%
\pgfpathclose%
\pgfusepath{fill}%
\end{pgfscope}%
\begin{pgfscope}%
\pgfpathrectangle{\pgfqpoint{1.254980in}{0.150000in}}{\pgfqpoint{5.490039in}{5.490039in}}%
\pgfusepath{clip}%
\pgfsetbuttcap%
\pgfsetroundjoin%
\definecolor{currentfill}{rgb}{0.281446,0.084320,0.407414}%
\pgfsetfillcolor{currentfill}%
\pgfsetfillopacity{0.700000}%
\pgfsetlinewidth{0.000000pt}%
\definecolor{currentstroke}{rgb}{0.000000,0.000000,0.000000}%
\pgfsetstrokecolor{currentstroke}%
\pgfsetdash{}{0pt}%
\pgfpathmoveto{\pgfqpoint{4.319315in}{1.976730in}}%
\pgfpathlineto{\pgfqpoint{4.332648in}{1.974664in}}%
\pgfpathlineto{\pgfqpoint{4.345989in}{1.972718in}}%
\pgfpathlineto{\pgfqpoint{4.359337in}{1.970893in}}%
\pgfpathlineto{\pgfqpoint{4.372693in}{1.969189in}}%
\pgfpathlineto{\pgfqpoint{4.380260in}{1.979289in}}%
\pgfpathlineto{\pgfqpoint{4.387822in}{1.989399in}}%
\pgfpathlineto{\pgfqpoint{4.395379in}{1.999517in}}%
\pgfpathlineto{\pgfqpoint{4.402931in}{2.009643in}}%
\pgfpathlineto{\pgfqpoint{4.389583in}{2.011198in}}%
\pgfpathlineto{\pgfqpoint{4.376244in}{2.012874in}}%
\pgfpathlineto{\pgfqpoint{4.362912in}{2.014670in}}%
\pgfpathlineto{\pgfqpoint{4.349588in}{2.016587in}}%
\pgfpathlineto{\pgfqpoint{4.342027in}{2.006605in}}%
\pgfpathlineto{\pgfqpoint{4.334461in}{1.996634in}}%
\pgfpathlineto{\pgfqpoint{4.326891in}{1.986675in}}%
\pgfpathlineto{\pgfqpoint{4.319315in}{1.976730in}}%
\pgfpathclose%
\pgfusepath{fill}%
\end{pgfscope}%
\begin{pgfscope}%
\pgfpathrectangle{\pgfqpoint{1.254980in}{0.150000in}}{\pgfqpoint{5.490039in}{5.490039in}}%
\pgfusepath{clip}%
\pgfsetbuttcap%
\pgfsetroundjoin%
\definecolor{currentfill}{rgb}{0.137770,0.537492,0.554906}%
\pgfsetfillcolor{currentfill}%
\pgfsetfillopacity{0.700000}%
\pgfsetlinewidth{0.000000pt}%
\definecolor{currentstroke}{rgb}{0.000000,0.000000,0.000000}%
\pgfsetstrokecolor{currentstroke}%
\pgfsetdash{}{0pt}%
\pgfpathmoveto{\pgfqpoint{2.564413in}{3.069884in}}%
\pgfpathlineto{\pgfqpoint{2.577870in}{3.048477in}}%
\pgfpathlineto{\pgfqpoint{2.591317in}{3.027279in}}%
\pgfpathlineto{\pgfqpoint{2.604756in}{3.006287in}}%
\pgfpathlineto{\pgfqpoint{2.618186in}{2.985501in}}%
\pgfpathlineto{\pgfqpoint{2.626561in}{2.986790in}}%
\pgfpathlineto{\pgfqpoint{2.634923in}{2.988253in}}%
\pgfpathlineto{\pgfqpoint{2.643272in}{2.989889in}}%
\pgfpathlineto{\pgfqpoint{2.651608in}{2.991694in}}%
\pgfpathlineto{\pgfqpoint{2.638215in}{3.012190in}}%
\pgfpathlineto{\pgfqpoint{2.624813in}{3.032890in}}%
\pgfpathlineto{\pgfqpoint{2.611403in}{3.053797in}}%
\pgfpathlineto{\pgfqpoint{2.597983in}{3.074911in}}%
\pgfpathlineto{\pgfqpoint{2.589611in}{3.073390in}}%
\pgfpathlineto{\pgfqpoint{2.581225in}{3.072044in}}%
\pgfpathlineto{\pgfqpoint{2.572826in}{3.070874in}}%
\pgfpathlineto{\pgfqpoint{2.564413in}{3.069884in}}%
\pgfpathclose%
\pgfusepath{fill}%
\end{pgfscope}%
\begin{pgfscope}%
\pgfpathrectangle{\pgfqpoint{1.254980in}{0.150000in}}{\pgfqpoint{5.490039in}{5.490039in}}%
\pgfusepath{clip}%
\pgfsetbuttcap%
\pgfsetroundjoin%
\definecolor{currentfill}{rgb}{0.235526,0.309527,0.542944}%
\pgfsetfillcolor{currentfill}%
\pgfsetfillopacity{0.700000}%
\pgfsetlinewidth{0.000000pt}%
\definecolor{currentstroke}{rgb}{0.000000,0.000000,0.000000}%
\pgfsetstrokecolor{currentstroke}%
\pgfsetdash{}{0pt}%
\pgfpathmoveto{\pgfqpoint{5.156105in}{2.418378in}}%
\pgfpathlineto{\pgfqpoint{5.169771in}{2.421761in}}%
\pgfpathlineto{\pgfqpoint{5.183450in}{2.425258in}}%
\pgfpathlineto{\pgfqpoint{5.197141in}{2.428869in}}%
\pgfpathlineto{\pgfqpoint{5.210844in}{2.432593in}}%
\pgfpathlineto{\pgfqpoint{5.218140in}{2.442564in}}%
\pgfpathlineto{\pgfqpoint{5.225430in}{2.452490in}}%
\pgfpathlineto{\pgfqpoint{5.232714in}{2.462371in}}%
\pgfpathlineto{\pgfqpoint{5.239993in}{2.472209in}}%
\pgfpathlineto{\pgfqpoint{5.226297in}{2.468496in}}%
\pgfpathlineto{\pgfqpoint{5.212614in}{2.464897in}}%
\pgfpathlineto{\pgfqpoint{5.198943in}{2.461411in}}%
\pgfpathlineto{\pgfqpoint{5.185284in}{2.458039in}}%
\pgfpathlineto{\pgfqpoint{5.177998in}{2.448184in}}%
\pgfpathlineto{\pgfqpoint{5.170706in}{2.438289in}}%
\pgfpathlineto{\pgfqpoint{5.163408in}{2.428354in}}%
\pgfpathlineto{\pgfqpoint{5.156105in}{2.418378in}}%
\pgfpathclose%
\pgfusepath{fill}%
\end{pgfscope}%
\begin{pgfscope}%
\pgfpathrectangle{\pgfqpoint{1.254980in}{0.150000in}}{\pgfqpoint{5.490039in}{5.490039in}}%
\pgfusepath{clip}%
\pgfsetbuttcap%
\pgfsetroundjoin%
\definecolor{currentfill}{rgb}{0.277018,0.050344,0.375715}%
\pgfsetfillcolor{currentfill}%
\pgfsetfillopacity{0.700000}%
\pgfsetlinewidth{0.000000pt}%
\definecolor{currentstroke}{rgb}{0.000000,0.000000,0.000000}%
\pgfsetstrokecolor{currentstroke}%
\pgfsetdash{}{0pt}%
\pgfpathmoveto{\pgfqpoint{4.015159in}{1.917248in}}%
\pgfpathlineto{\pgfqpoint{4.028418in}{1.912665in}}%
\pgfpathlineto{\pgfqpoint{4.041682in}{1.908208in}}%
\pgfpathlineto{\pgfqpoint{4.054952in}{1.903877in}}%
\pgfpathlineto{\pgfqpoint{4.068228in}{1.899671in}}%
\pgfpathlineto{\pgfqpoint{4.075895in}{1.908757in}}%
\pgfpathlineto{\pgfqpoint{4.083556in}{1.917883in}}%
\pgfpathlineto{\pgfqpoint{4.091212in}{1.927047in}}%
\pgfpathlineto{\pgfqpoint{4.098863in}{1.936249in}}%
\pgfpathlineto{\pgfqpoint{4.085599in}{1.940258in}}%
\pgfpathlineto{\pgfqpoint{4.072340in}{1.944393in}}%
\pgfpathlineto{\pgfqpoint{4.059088in}{1.948653in}}%
\pgfpathlineto{\pgfqpoint{4.045841in}{1.953038in}}%
\pgfpathlineto{\pgfqpoint{4.038178in}{1.944028in}}%
\pgfpathlineto{\pgfqpoint{4.030511in}{1.935058in}}%
\pgfpathlineto{\pgfqpoint{4.022838in}{1.926131in}}%
\pgfpathlineto{\pgfqpoint{4.015159in}{1.917248in}}%
\pgfpathclose%
\pgfusepath{fill}%
\end{pgfscope}%
\begin{pgfscope}%
\pgfpathrectangle{\pgfqpoint{1.254980in}{0.150000in}}{\pgfqpoint{5.490039in}{5.490039in}}%
\pgfusepath{clip}%
\pgfsetbuttcap%
\pgfsetroundjoin%
\definecolor{currentfill}{rgb}{0.280894,0.078907,0.402329}%
\pgfsetfillcolor{currentfill}%
\pgfsetfillopacity{0.700000}%
\pgfsetlinewidth{0.000000pt}%
\definecolor{currentstroke}{rgb}{0.000000,0.000000,0.000000}%
\pgfsetstrokecolor{currentstroke}%
\pgfsetdash{}{0pt}%
\pgfpathmoveto{\pgfqpoint{3.604661in}{1.980025in}}%
\pgfpathlineto{\pgfqpoint{3.617871in}{1.971603in}}%
\pgfpathlineto{\pgfqpoint{3.631083in}{1.963318in}}%
\pgfpathlineto{\pgfqpoint{3.644297in}{1.955170in}}%
\pgfpathlineto{\pgfqpoint{3.657514in}{1.947157in}}%
\pgfpathlineto{\pgfqpoint{3.665340in}{1.954124in}}%
\pgfpathlineto{\pgfqpoint{3.673159in}{1.961176in}}%
\pgfpathlineto{\pgfqpoint{3.680972in}{1.968311in}}%
\pgfpathlineto{\pgfqpoint{3.688778in}{1.975526in}}%
\pgfpathlineto{\pgfqpoint{3.675578in}{1.983292in}}%
\pgfpathlineto{\pgfqpoint{3.662381in}{1.991194in}}%
\pgfpathlineto{\pgfqpoint{3.649188in}{1.999232in}}%
\pgfpathlineto{\pgfqpoint{3.635996in}{2.007406in}}%
\pgfpathlineto{\pgfqpoint{3.628173in}{2.000431in}}%
\pgfpathlineto{\pgfqpoint{3.620342in}{1.993542in}}%
\pgfpathlineto{\pgfqpoint{3.612505in}{1.986739in}}%
\pgfpathlineto{\pgfqpoint{3.604661in}{1.980025in}}%
\pgfpathclose%
\pgfusepath{fill}%
\end{pgfscope}%
\begin{pgfscope}%
\pgfpathrectangle{\pgfqpoint{1.254980in}{0.150000in}}{\pgfqpoint{5.490039in}{5.490039in}}%
\pgfusepath{clip}%
\pgfsetbuttcap%
\pgfsetroundjoin%
\definecolor{currentfill}{rgb}{0.160665,0.478540,0.558115}%
\pgfsetfillcolor{currentfill}%
\pgfsetfillopacity{0.700000}%
\pgfsetlinewidth{0.000000pt}%
\definecolor{currentstroke}{rgb}{0.000000,0.000000,0.000000}%
\pgfsetstrokecolor{currentstroke}%
\pgfsetdash{}{0pt}%
\pgfpathmoveto{\pgfqpoint{5.772652in}{2.842669in}}%
\pgfpathlineto{\pgfqpoint{5.786628in}{2.848648in}}%
\pgfpathlineto{\pgfqpoint{5.800618in}{2.854739in}}%
\pgfpathlineto{\pgfqpoint{5.814624in}{2.860942in}}%
\pgfpathlineto{\pgfqpoint{5.828645in}{2.867255in}}%
\pgfpathlineto{\pgfqpoint{5.835677in}{2.875205in}}%
\pgfpathlineto{\pgfqpoint{5.842702in}{2.883112in}}%
\pgfpathlineto{\pgfqpoint{5.849721in}{2.890977in}}%
\pgfpathlineto{\pgfqpoint{5.856733in}{2.898801in}}%
\pgfpathlineto{\pgfqpoint{5.842726in}{2.892633in}}%
\pgfpathlineto{\pgfqpoint{5.828734in}{2.886576in}}%
\pgfpathlineto{\pgfqpoint{5.814757in}{2.880629in}}%
\pgfpathlineto{\pgfqpoint{5.800795in}{2.874794in}}%
\pgfpathlineto{\pgfqpoint{5.793768in}{2.866819in}}%
\pgfpathlineto{\pgfqpoint{5.786736in}{2.858807in}}%
\pgfpathlineto{\pgfqpoint{5.779697in}{2.850757in}}%
\pgfpathlineto{\pgfqpoint{5.772652in}{2.842669in}}%
\pgfpathclose%
\pgfusepath{fill}%
\end{pgfscope}%
\begin{pgfscope}%
\pgfpathrectangle{\pgfqpoint{1.254980in}{0.150000in}}{\pgfqpoint{5.490039in}{5.490039in}}%
\pgfusepath{clip}%
\pgfsetbuttcap%
\pgfsetroundjoin%
\definecolor{currentfill}{rgb}{0.223925,0.334994,0.548053}%
\pgfsetfillcolor{currentfill}%
\pgfsetfillopacity{0.700000}%
\pgfsetlinewidth{0.000000pt}%
\definecolor{currentstroke}{rgb}{0.000000,0.000000,0.000000}%
\pgfsetstrokecolor{currentstroke}%
\pgfsetdash{}{0pt}%
\pgfpathmoveto{\pgfqpoint{5.239993in}{2.472209in}}%
\pgfpathlineto{\pgfqpoint{5.253702in}{2.476036in}}%
\pgfpathlineto{\pgfqpoint{5.267424in}{2.479976in}}%
\pgfpathlineto{\pgfqpoint{5.281159in}{2.484029in}}%
\pgfpathlineto{\pgfqpoint{5.294907in}{2.488195in}}%
\pgfpathlineto{\pgfqpoint{5.302173in}{2.497968in}}%
\pgfpathlineto{\pgfqpoint{5.309432in}{2.507694in}}%
\pgfpathlineto{\pgfqpoint{5.316686in}{2.517373in}}%
\pgfpathlineto{\pgfqpoint{5.323934in}{2.527007in}}%
\pgfpathlineto{\pgfqpoint{5.310194in}{2.522869in}}%
\pgfpathlineto{\pgfqpoint{5.296467in}{2.518843in}}%
\pgfpathlineto{\pgfqpoint{5.282753in}{2.514931in}}%
\pgfpathlineto{\pgfqpoint{5.269052in}{2.511132in}}%
\pgfpathlineto{\pgfqpoint{5.261796in}{2.501465in}}%
\pgfpathlineto{\pgfqpoint{5.254534in}{2.491755in}}%
\pgfpathlineto{\pgfqpoint{5.247266in}{2.482004in}}%
\pgfpathlineto{\pgfqpoint{5.239993in}{2.472209in}}%
\pgfpathclose%
\pgfusepath{fill}%
\end{pgfscope}%
\begin{pgfscope}%
\pgfpathrectangle{\pgfqpoint{1.254980in}{0.150000in}}{\pgfqpoint{5.490039in}{5.490039in}}%
\pgfusepath{clip}%
\pgfsetbuttcap%
\pgfsetroundjoin%
\definecolor{currentfill}{rgb}{0.280267,0.073417,0.397163}%
\pgfsetfillcolor{currentfill}%
\pgfsetfillopacity{0.700000}%
\pgfsetlinewidth{0.000000pt}%
\definecolor{currentstroke}{rgb}{0.000000,0.000000,0.000000}%
\pgfsetstrokecolor{currentstroke}%
\pgfsetdash{}{0pt}%
\pgfpathmoveto{\pgfqpoint{4.235671in}{1.947241in}}%
\pgfpathlineto{\pgfqpoint{4.248984in}{1.944523in}}%
\pgfpathlineto{\pgfqpoint{4.262304in}{1.941928in}}%
\pgfpathlineto{\pgfqpoint{4.275631in}{1.939454in}}%
\pgfpathlineto{\pgfqpoint{4.288965in}{1.937102in}}%
\pgfpathlineto{\pgfqpoint{4.296560in}{1.946984in}}%
\pgfpathlineto{\pgfqpoint{4.304150in}{1.956883in}}%
\pgfpathlineto{\pgfqpoint{4.311735in}{1.966799in}}%
\pgfpathlineto{\pgfqpoint{4.319315in}{1.976730in}}%
\pgfpathlineto{\pgfqpoint{4.305990in}{1.978917in}}%
\pgfpathlineto{\pgfqpoint{4.292672in}{1.981226in}}%
\pgfpathlineto{\pgfqpoint{4.279362in}{1.983656in}}%
\pgfpathlineto{\pgfqpoint{4.266059in}{1.986208in}}%
\pgfpathlineto{\pgfqpoint{4.258469in}{1.976436in}}%
\pgfpathlineto{\pgfqpoint{4.250875in}{1.966683in}}%
\pgfpathlineto{\pgfqpoint{4.243275in}{1.956951in}}%
\pgfpathlineto{\pgfqpoint{4.235671in}{1.947241in}}%
\pgfpathclose%
\pgfusepath{fill}%
\end{pgfscope}%
\begin{pgfscope}%
\pgfpathrectangle{\pgfqpoint{1.254980in}{0.150000in}}{\pgfqpoint{5.490039in}{5.490039in}}%
\pgfusepath{clip}%
\pgfsetbuttcap%
\pgfsetroundjoin%
\definecolor{currentfill}{rgb}{0.283187,0.125848,0.444960}%
\pgfsetfillcolor{currentfill}%
\pgfsetfillopacity{0.700000}%
\pgfsetlinewidth{0.000000pt}%
\definecolor{currentstroke}{rgb}{0.000000,0.000000,0.000000}%
\pgfsetstrokecolor{currentstroke}%
\pgfsetdash{}{0pt}%
\pgfpathmoveto{\pgfqpoint{3.414596in}{2.069213in}}%
\pgfpathlineto{\pgfqpoint{3.427807in}{2.058838in}}%
\pgfpathlineto{\pgfqpoint{3.441020in}{2.048608in}}%
\pgfpathlineto{\pgfqpoint{3.454233in}{2.038521in}}%
\pgfpathlineto{\pgfqpoint{3.467447in}{2.028577in}}%
\pgfpathlineto{\pgfqpoint{3.475361in}{2.034375in}}%
\pgfpathlineto{\pgfqpoint{3.483267in}{2.040279in}}%
\pgfpathlineto{\pgfqpoint{3.491166in}{2.046285in}}%
\pgfpathlineto{\pgfqpoint{3.499057in}{2.052392in}}%
\pgfpathlineto{\pgfqpoint{3.485863in}{2.062071in}}%
\pgfpathlineto{\pgfqpoint{3.472671in}{2.071893in}}%
\pgfpathlineto{\pgfqpoint{3.459480in}{2.081858in}}%
\pgfpathlineto{\pgfqpoint{3.446290in}{2.091967in}}%
\pgfpathlineto{\pgfqpoint{3.438378in}{2.086118in}}%
\pgfpathlineto{\pgfqpoint{3.430459in}{2.080375in}}%
\pgfpathlineto{\pgfqpoint{3.422531in}{2.074739in}}%
\pgfpathlineto{\pgfqpoint{3.414596in}{2.069213in}}%
\pgfpathclose%
\pgfusepath{fill}%
\end{pgfscope}%
\begin{pgfscope}%
\pgfpathrectangle{\pgfqpoint{1.254980in}{0.150000in}}{\pgfqpoint{5.490039in}{5.490039in}}%
\pgfusepath{clip}%
\pgfsetbuttcap%
\pgfsetroundjoin%
\definecolor{currentfill}{rgb}{0.235526,0.309527,0.542944}%
\pgfsetfillcolor{currentfill}%
\pgfsetfillopacity{0.700000}%
\pgfsetlinewidth{0.000000pt}%
\definecolor{currentstroke}{rgb}{0.000000,0.000000,0.000000}%
\pgfsetstrokecolor{currentstroke}%
\pgfsetdash{}{0pt}%
\pgfpathmoveto{\pgfqpoint{2.958778in}{2.469821in}}%
\pgfpathlineto{\pgfqpoint{2.972064in}{2.454148in}}%
\pgfpathlineto{\pgfqpoint{2.985347in}{2.438646in}}%
\pgfpathlineto{\pgfqpoint{2.998626in}{2.423312in}}%
\pgfpathlineto{\pgfqpoint{3.011902in}{2.408146in}}%
\pgfpathlineto{\pgfqpoint{3.020057in}{2.411164in}}%
\pgfpathlineto{\pgfqpoint{3.028201in}{2.414329in}}%
\pgfpathlineto{\pgfqpoint{3.036335in}{2.417639in}}%
\pgfpathlineto{\pgfqpoint{3.044458in}{2.421093in}}%
\pgfpathlineto{\pgfqpoint{3.031212in}{2.435967in}}%
\pgfpathlineto{\pgfqpoint{3.017963in}{2.451008in}}%
\pgfpathlineto{\pgfqpoint{3.004710in}{2.466218in}}%
\pgfpathlineto{\pgfqpoint{2.991454in}{2.481598in}}%
\pgfpathlineto{\pgfqpoint{2.983301in}{2.478430in}}%
\pgfpathlineto{\pgfqpoint{2.975137in}{2.475410in}}%
\pgfpathlineto{\pgfqpoint{2.966963in}{2.472539in}}%
\pgfpathlineto{\pgfqpoint{2.958778in}{2.469821in}}%
\pgfpathclose%
\pgfusepath{fill}%
\end{pgfscope}%
\begin{pgfscope}%
\pgfpathrectangle{\pgfqpoint{1.254980in}{0.150000in}}{\pgfqpoint{5.490039in}{5.490039in}}%
\pgfusepath{clip}%
\pgfsetbuttcap%
\pgfsetroundjoin%
\definecolor{currentfill}{rgb}{0.223925,0.334994,0.548053}%
\pgfsetfillcolor{currentfill}%
\pgfsetfillopacity{0.700000}%
\pgfsetlinewidth{0.000000pt}%
\definecolor{currentstroke}{rgb}{0.000000,0.000000,0.000000}%
\pgfsetstrokecolor{currentstroke}%
\pgfsetdash{}{0pt}%
\pgfpathmoveto{\pgfqpoint{2.905590in}{2.534234in}}%
\pgfpathlineto{\pgfqpoint{2.918893in}{2.517871in}}%
\pgfpathlineto{\pgfqpoint{2.932192in}{2.501682in}}%
\pgfpathlineto{\pgfqpoint{2.945487in}{2.485665in}}%
\pgfpathlineto{\pgfqpoint{2.958778in}{2.469821in}}%
\pgfpathlineto{\pgfqpoint{2.966963in}{2.472539in}}%
\pgfpathlineto{\pgfqpoint{2.975137in}{2.475410in}}%
\pgfpathlineto{\pgfqpoint{2.983301in}{2.478430in}}%
\pgfpathlineto{\pgfqpoint{2.991454in}{2.481598in}}%
\pgfpathlineto{\pgfqpoint{2.978194in}{2.497148in}}%
\pgfpathlineto{\pgfqpoint{2.964930in}{2.512871in}}%
\pgfpathlineto{\pgfqpoint{2.951662in}{2.528766in}}%
\pgfpathlineto{\pgfqpoint{2.938390in}{2.544834in}}%
\pgfpathlineto{\pgfqpoint{2.930207in}{2.541954in}}%
\pgfpathlineto{\pgfqpoint{2.922012in}{2.539226in}}%
\pgfpathlineto{\pgfqpoint{2.913807in}{2.536652in}}%
\pgfpathlineto{\pgfqpoint{2.905590in}{2.534234in}}%
\pgfpathclose%
\pgfusepath{fill}%
\end{pgfscope}%
\begin{pgfscope}%
\pgfpathrectangle{\pgfqpoint{1.254980in}{0.150000in}}{\pgfqpoint{5.490039in}{5.490039in}}%
\pgfusepath{clip}%
\pgfsetbuttcap%
\pgfsetroundjoin%
\definecolor{currentfill}{rgb}{0.246811,0.283237,0.535941}%
\pgfsetfillcolor{currentfill}%
\pgfsetfillopacity{0.700000}%
\pgfsetlinewidth{0.000000pt}%
\definecolor{currentstroke}{rgb}{0.000000,0.000000,0.000000}%
\pgfsetstrokecolor{currentstroke}%
\pgfsetdash{}{0pt}%
\pgfpathmoveto{\pgfqpoint{3.011902in}{2.408146in}}%
\pgfpathlineto{\pgfqpoint{3.025175in}{2.393148in}}%
\pgfpathlineto{\pgfqpoint{3.038444in}{2.378316in}}%
\pgfpathlineto{\pgfqpoint{3.051710in}{2.363650in}}%
\pgfpathlineto{\pgfqpoint{3.064973in}{2.349148in}}%
\pgfpathlineto{\pgfqpoint{3.073098in}{2.352462in}}%
\pgfpathlineto{\pgfqpoint{3.081213in}{2.355920in}}%
\pgfpathlineto{\pgfqpoint{3.089318in}{2.359519in}}%
\pgfpathlineto{\pgfqpoint{3.097413in}{2.363257in}}%
\pgfpathlineto{\pgfqpoint{3.084179in}{2.377469in}}%
\pgfpathlineto{\pgfqpoint{3.070941in}{2.391845in}}%
\pgfpathlineto{\pgfqpoint{3.057701in}{2.406386in}}%
\pgfpathlineto{\pgfqpoint{3.044458in}{2.421093in}}%
\pgfpathlineto{\pgfqpoint{3.036335in}{2.417639in}}%
\pgfpathlineto{\pgfqpoint{3.028201in}{2.414329in}}%
\pgfpathlineto{\pgfqpoint{3.020057in}{2.411164in}}%
\pgfpathlineto{\pgfqpoint{3.011902in}{2.408146in}}%
\pgfpathclose%
\pgfusepath{fill}%
\end{pgfscope}%
\begin{pgfscope}%
\pgfpathrectangle{\pgfqpoint{1.254980in}{0.150000in}}{\pgfqpoint{5.490039in}{5.490039in}}%
\pgfusepath{clip}%
\pgfsetbuttcap%
\pgfsetroundjoin%
\definecolor{currentfill}{rgb}{0.210503,0.363727,0.552206}%
\pgfsetfillcolor{currentfill}%
\pgfsetfillopacity{0.700000}%
\pgfsetlinewidth{0.000000pt}%
\definecolor{currentstroke}{rgb}{0.000000,0.000000,0.000000}%
\pgfsetstrokecolor{currentstroke}%
\pgfsetdash{}{0pt}%
\pgfpathmoveto{\pgfqpoint{2.852328in}{2.601451in}}%
\pgfpathlineto{\pgfqpoint{2.865651in}{2.584380in}}%
\pgfpathlineto{\pgfqpoint{2.878969in}{2.567488in}}%
\pgfpathlineto{\pgfqpoint{2.892282in}{2.550773in}}%
\pgfpathlineto{\pgfqpoint{2.905590in}{2.534234in}}%
\pgfpathlineto{\pgfqpoint{2.913807in}{2.536652in}}%
\pgfpathlineto{\pgfqpoint{2.922012in}{2.539226in}}%
\pgfpathlineto{\pgfqpoint{2.930207in}{2.541954in}}%
\pgfpathlineto{\pgfqpoint{2.938390in}{2.544834in}}%
\pgfpathlineto{\pgfqpoint{2.925113in}{2.561077in}}%
\pgfpathlineto{\pgfqpoint{2.911833in}{2.577497in}}%
\pgfpathlineto{\pgfqpoint{2.898547in}{2.594093in}}%
\pgfpathlineto{\pgfqpoint{2.885257in}{2.610867in}}%
\pgfpathlineto{\pgfqpoint{2.877042in}{2.608276in}}%
\pgfpathlineto{\pgfqpoint{2.868816in}{2.605841in}}%
\pgfpathlineto{\pgfqpoint{2.860578in}{2.603566in}}%
\pgfpathlineto{\pgfqpoint{2.852328in}{2.601451in}}%
\pgfpathclose%
\pgfusepath{fill}%
\end{pgfscope}%
\begin{pgfscope}%
\pgfpathrectangle{\pgfqpoint{1.254980in}{0.150000in}}{\pgfqpoint{5.490039in}{5.490039in}}%
\pgfusepath{clip}%
\pgfsetbuttcap%
\pgfsetroundjoin%
\definecolor{currentfill}{rgb}{0.257322,0.256130,0.526563}%
\pgfsetfillcolor{currentfill}%
\pgfsetfillopacity{0.700000}%
\pgfsetlinewidth{0.000000pt}%
\definecolor{currentstroke}{rgb}{0.000000,0.000000,0.000000}%
\pgfsetstrokecolor{currentstroke}%
\pgfsetdash{}{0pt}%
\pgfpathmoveto{\pgfqpoint{3.064973in}{2.349148in}}%
\pgfpathlineto{\pgfqpoint{3.078233in}{2.334809in}}%
\pgfpathlineto{\pgfqpoint{3.091491in}{2.320633in}}%
\pgfpathlineto{\pgfqpoint{3.104746in}{2.306619in}}%
\pgfpathlineto{\pgfqpoint{3.117999in}{2.292765in}}%
\pgfpathlineto{\pgfqpoint{3.126096in}{2.296376in}}%
\pgfpathlineto{\pgfqpoint{3.134183in}{2.300125in}}%
\pgfpathlineto{\pgfqpoint{3.142260in}{2.304011in}}%
\pgfpathlineto{\pgfqpoint{3.150327in}{2.308032in}}%
\pgfpathlineto{\pgfqpoint{3.137102in}{2.321596in}}%
\pgfpathlineto{\pgfqpoint{3.123875in}{2.335322in}}%
\pgfpathlineto{\pgfqpoint{3.110645in}{2.349208in}}%
\pgfpathlineto{\pgfqpoint{3.097413in}{2.363257in}}%
\pgfpathlineto{\pgfqpoint{3.089318in}{2.359519in}}%
\pgfpathlineto{\pgfqpoint{3.081213in}{2.355920in}}%
\pgfpathlineto{\pgfqpoint{3.073098in}{2.352462in}}%
\pgfpathlineto{\pgfqpoint{3.064973in}{2.349148in}}%
\pgfpathclose%
\pgfusepath{fill}%
\end{pgfscope}%
\begin{pgfscope}%
\pgfpathrectangle{\pgfqpoint{1.254980in}{0.150000in}}{\pgfqpoint{5.490039in}{5.490039in}}%
\pgfusepath{clip}%
\pgfsetbuttcap%
\pgfsetroundjoin%
\definecolor{currentfill}{rgb}{0.151918,0.500685,0.557587}%
\pgfsetfillcolor{currentfill}%
\pgfsetfillopacity{0.700000}%
\pgfsetlinewidth{0.000000pt}%
\definecolor{currentstroke}{rgb}{0.000000,0.000000,0.000000}%
\pgfsetstrokecolor{currentstroke}%
\pgfsetdash{}{0pt}%
\pgfpathmoveto{\pgfqpoint{5.856733in}{2.898801in}}%
\pgfpathlineto{\pgfqpoint{5.870756in}{2.905081in}}%
\pgfpathlineto{\pgfqpoint{5.884795in}{2.911472in}}%
\pgfpathlineto{\pgfqpoint{5.898849in}{2.917974in}}%
\pgfpathlineto{\pgfqpoint{5.912919in}{2.924587in}}%
\pgfpathlineto{\pgfqpoint{5.919911in}{2.932217in}}%
\pgfpathlineto{\pgfqpoint{5.926897in}{2.939807in}}%
\pgfpathlineto{\pgfqpoint{5.933876in}{2.947357in}}%
\pgfpathlineto{\pgfqpoint{5.940848in}{2.954870in}}%
\pgfpathlineto{\pgfqpoint{5.926793in}{2.948419in}}%
\pgfpathlineto{\pgfqpoint{5.912754in}{2.942080in}}%
\pgfpathlineto{\pgfqpoint{5.898730in}{2.935851in}}%
\pgfpathlineto{\pgfqpoint{5.884721in}{2.929733in}}%
\pgfpathlineto{\pgfqpoint{5.877734in}{2.922052in}}%
\pgfpathlineto{\pgfqpoint{5.870740in}{2.914337in}}%
\pgfpathlineto{\pgfqpoint{5.863740in}{2.906588in}}%
\pgfpathlineto{\pgfqpoint{5.856733in}{2.898801in}}%
\pgfpathclose%
\pgfusepath{fill}%
\end{pgfscope}%
\begin{pgfscope}%
\pgfpathrectangle{\pgfqpoint{1.254980in}{0.150000in}}{\pgfqpoint{5.490039in}{5.490039in}}%
\pgfusepath{clip}%
\pgfsetbuttcap%
\pgfsetroundjoin%
\definecolor{currentfill}{rgb}{0.143343,0.522773,0.556295}%
\pgfsetfillcolor{currentfill}%
\pgfsetfillopacity{0.700000}%
\pgfsetlinewidth{0.000000pt}%
\definecolor{currentstroke}{rgb}{0.000000,0.000000,0.000000}%
\pgfsetstrokecolor{currentstroke}%
\pgfsetdash{}{0pt}%
\pgfpathmoveto{\pgfqpoint{5.940848in}{2.954870in}}%
\pgfpathlineto{\pgfqpoint{5.954919in}{2.961432in}}%
\pgfpathlineto{\pgfqpoint{5.969006in}{2.968104in}}%
\pgfpathlineto{\pgfqpoint{5.983109in}{2.974887in}}%
\pgfpathlineto{\pgfqpoint{5.990064in}{2.982235in}}%
\pgfpathlineto{\pgfqpoint{5.997012in}{2.989546in}}%
\pgfpathlineto{\pgfqpoint{6.003954in}{2.996821in}}%
\pgfpathlineto{\pgfqpoint{6.010890in}{3.004063in}}%
\pgfpathlineto{\pgfqpoint{5.996803in}{2.997459in}}%
\pgfpathlineto{\pgfqpoint{5.982732in}{2.990965in}}%
\pgfpathlineto{\pgfqpoint{5.968677in}{2.984582in}}%
\pgfpathlineto{\pgfqpoint{5.961729in}{2.977202in}}%
\pgfpathlineto{\pgfqpoint{5.954775in}{2.969791in}}%
\pgfpathlineto{\pgfqpoint{5.947815in}{2.962347in}}%
\pgfpathlineto{\pgfqpoint{5.940848in}{2.954870in}}%
\pgfpathclose%
\pgfusepath{fill}%
\end{pgfscope}%
\begin{pgfscope}%
\pgfpathrectangle{\pgfqpoint{1.254980in}{0.150000in}}{\pgfqpoint{5.490039in}{5.490039in}}%
\pgfusepath{clip}%
\pgfsetbuttcap%
\pgfsetroundjoin%
\definecolor{currentfill}{rgb}{0.212395,0.359683,0.551710}%
\pgfsetfillcolor{currentfill}%
\pgfsetfillopacity{0.700000}%
\pgfsetlinewidth{0.000000pt}%
\definecolor{currentstroke}{rgb}{0.000000,0.000000,0.000000}%
\pgfsetstrokecolor{currentstroke}%
\pgfsetdash{}{0pt}%
\pgfpathmoveto{\pgfqpoint{5.323934in}{2.527007in}}%
\pgfpathlineto{\pgfqpoint{5.337687in}{2.531259in}}%
\pgfpathlineto{\pgfqpoint{5.351454in}{2.535623in}}%
\pgfpathlineto{\pgfqpoint{5.365234in}{2.540101in}}%
\pgfpathlineto{\pgfqpoint{5.379028in}{2.544691in}}%
\pgfpathlineto{\pgfqpoint{5.386262in}{2.554242in}}%
\pgfpathlineto{\pgfqpoint{5.393490in}{2.563743in}}%
\pgfpathlineto{\pgfqpoint{5.400712in}{2.573197in}}%
\pgfpathlineto{\pgfqpoint{5.407928in}{2.582603in}}%
\pgfpathlineto{\pgfqpoint{5.394143in}{2.578056in}}%
\pgfpathlineto{\pgfqpoint{5.380372in}{2.573623in}}%
\pgfpathlineto{\pgfqpoint{5.366614in}{2.569303in}}%
\pgfpathlineto{\pgfqpoint{5.352869in}{2.565095in}}%
\pgfpathlineto{\pgfqpoint{5.345644in}{2.555639in}}%
\pgfpathlineto{\pgfqpoint{5.338413in}{2.546139in}}%
\pgfpathlineto{\pgfqpoint{5.331177in}{2.536595in}}%
\pgfpathlineto{\pgfqpoint{5.323934in}{2.527007in}}%
\pgfpathclose%
\pgfusepath{fill}%
\end{pgfscope}%
\begin{pgfscope}%
\pgfpathrectangle{\pgfqpoint{1.254980in}{0.150000in}}{\pgfqpoint{5.490039in}{5.490039in}}%
\pgfusepath{clip}%
\pgfsetbuttcap%
\pgfsetroundjoin%
\definecolor{currentfill}{rgb}{0.197636,0.391528,0.554969}%
\pgfsetfillcolor{currentfill}%
\pgfsetfillopacity{0.700000}%
\pgfsetlinewidth{0.000000pt}%
\definecolor{currentstroke}{rgb}{0.000000,0.000000,0.000000}%
\pgfsetstrokecolor{currentstroke}%
\pgfsetdash{}{0pt}%
\pgfpathmoveto{\pgfqpoint{2.798983in}{2.671541in}}%
\pgfpathlineto{\pgfqpoint{2.812328in}{2.653745in}}%
\pgfpathlineto{\pgfqpoint{2.825667in}{2.636132in}}%
\pgfpathlineto{\pgfqpoint{2.839000in}{2.618701in}}%
\pgfpathlineto{\pgfqpoint{2.852328in}{2.601451in}}%
\pgfpathlineto{\pgfqpoint{2.860578in}{2.603566in}}%
\pgfpathlineto{\pgfqpoint{2.868816in}{2.605841in}}%
\pgfpathlineto{\pgfqpoint{2.877042in}{2.608276in}}%
\pgfpathlineto{\pgfqpoint{2.885257in}{2.610867in}}%
\pgfpathlineto{\pgfqpoint{2.871961in}{2.627819in}}%
\pgfpathlineto{\pgfqpoint{2.858661in}{2.644953in}}%
\pgfpathlineto{\pgfqpoint{2.845355in}{2.662267in}}%
\pgfpathlineto{\pgfqpoint{2.832044in}{2.679764in}}%
\pgfpathlineto{\pgfqpoint{2.823797in}{2.677464in}}%
\pgfpathlineto{\pgfqpoint{2.815538in}{2.675326in}}%
\pgfpathlineto{\pgfqpoint{2.807267in}{2.673350in}}%
\pgfpathlineto{\pgfqpoint{2.798983in}{2.671541in}}%
\pgfpathclose%
\pgfusepath{fill}%
\end{pgfscope}%
\begin{pgfscope}%
\pgfpathrectangle{\pgfqpoint{1.254980in}{0.150000in}}{\pgfqpoint{5.490039in}{5.490039in}}%
\pgfusepath{clip}%
\pgfsetbuttcap%
\pgfsetroundjoin%
\definecolor{currentfill}{rgb}{0.265145,0.232956,0.516599}%
\pgfsetfillcolor{currentfill}%
\pgfsetfillopacity{0.700000}%
\pgfsetlinewidth{0.000000pt}%
\definecolor{currentstroke}{rgb}{0.000000,0.000000,0.000000}%
\pgfsetstrokecolor{currentstroke}%
\pgfsetdash{}{0pt}%
\pgfpathmoveto{\pgfqpoint{3.117999in}{2.292765in}}%
\pgfpathlineto{\pgfqpoint{3.131250in}{2.279072in}}%
\pgfpathlineto{\pgfqpoint{3.144498in}{2.265538in}}%
\pgfpathlineto{\pgfqpoint{3.157745in}{2.252162in}}%
\pgfpathlineto{\pgfqpoint{3.170990in}{2.238944in}}%
\pgfpathlineto{\pgfqpoint{3.179059in}{2.242848in}}%
\pgfpathlineto{\pgfqpoint{3.187119in}{2.246888in}}%
\pgfpathlineto{\pgfqpoint{3.195169in}{2.251059in}}%
\pgfpathlineto{\pgfqpoint{3.203210in}{2.255362in}}%
\pgfpathlineto{\pgfqpoint{3.189991in}{2.268293in}}%
\pgfpathlineto{\pgfqpoint{3.176772in}{2.281381in}}%
\pgfpathlineto{\pgfqpoint{3.163550in}{2.294627in}}%
\pgfpathlineto{\pgfqpoint{3.150327in}{2.308032in}}%
\pgfpathlineto{\pgfqpoint{3.142260in}{2.304011in}}%
\pgfpathlineto{\pgfqpoint{3.134183in}{2.300125in}}%
\pgfpathlineto{\pgfqpoint{3.126096in}{2.296376in}}%
\pgfpathlineto{\pgfqpoint{3.117999in}{2.292765in}}%
\pgfpathclose%
\pgfusepath{fill}%
\end{pgfscope}%
\begin{pgfscope}%
\pgfpathrectangle{\pgfqpoint{1.254980in}{0.150000in}}{\pgfqpoint{5.490039in}{5.490039in}}%
\pgfusepath{clip}%
\pgfsetbuttcap%
\pgfsetroundjoin%
\definecolor{currentfill}{rgb}{0.125394,0.574318,0.549086}%
\pgfsetfillcolor{currentfill}%
\pgfsetfillopacity{0.700000}%
\pgfsetlinewidth{0.000000pt}%
\definecolor{currentstroke}{rgb}{0.000000,0.000000,0.000000}%
\pgfsetstrokecolor{currentstroke}%
\pgfsetdash{}{0pt}%
\pgfpathmoveto{\pgfqpoint{2.510492in}{3.157626in}}%
\pgfpathlineto{\pgfqpoint{2.523987in}{3.135370in}}%
\pgfpathlineto{\pgfqpoint{2.537472in}{3.113329in}}%
\pgfpathlineto{\pgfqpoint{2.550947in}{3.091501in}}%
\pgfpathlineto{\pgfqpoint{2.564413in}{3.069884in}}%
\pgfpathlineto{\pgfqpoint{2.572826in}{3.070874in}}%
\pgfpathlineto{\pgfqpoint{2.581225in}{3.072044in}}%
\pgfpathlineto{\pgfqpoint{2.589611in}{3.073390in}}%
\pgfpathlineto{\pgfqpoint{2.597983in}{3.074911in}}%
\pgfpathlineto{\pgfqpoint{2.584555in}{3.096235in}}%
\pgfpathlineto{\pgfqpoint{2.571118in}{3.117769in}}%
\pgfpathlineto{\pgfqpoint{2.557671in}{3.139516in}}%
\pgfpathlineto{\pgfqpoint{2.544215in}{3.161476in}}%
\pgfpathlineto{\pgfqpoint{2.535805in}{3.160242in}}%
\pgfpathlineto{\pgfqpoint{2.527381in}{3.159188in}}%
\pgfpathlineto{\pgfqpoint{2.518944in}{3.158315in}}%
\pgfpathlineto{\pgfqpoint{2.510492in}{3.157626in}}%
\pgfpathclose%
\pgfusepath{fill}%
\end{pgfscope}%
\begin{pgfscope}%
\pgfpathrectangle{\pgfqpoint{1.254980in}{0.150000in}}{\pgfqpoint{5.490039in}{5.490039in}}%
\pgfusepath{clip}%
\pgfsetbuttcap%
\pgfsetroundjoin%
\definecolor{currentfill}{rgb}{0.277018,0.050344,0.375715}%
\pgfsetfillcolor{currentfill}%
\pgfsetfillopacity{0.700000}%
\pgfsetlinewidth{0.000000pt}%
\definecolor{currentstroke}{rgb}{0.000000,0.000000,0.000000}%
\pgfsetstrokecolor{currentstroke}%
\pgfsetdash{}{0pt}%
\pgfpathmoveto{\pgfqpoint{3.794488in}{1.918224in}}%
\pgfpathlineto{\pgfqpoint{3.807718in}{1.911657in}}%
\pgfpathlineto{\pgfqpoint{3.820952in}{1.905221in}}%
\pgfpathlineto{\pgfqpoint{3.834190in}{1.898916in}}%
\pgfpathlineto{\pgfqpoint{3.847432in}{1.892740in}}%
\pgfpathlineto{\pgfqpoint{3.855183in}{1.900742in}}%
\pgfpathlineto{\pgfqpoint{3.862929in}{1.908810in}}%
\pgfpathlineto{\pgfqpoint{3.870668in}{1.916941in}}%
\pgfpathlineto{\pgfqpoint{3.878401in}{1.925133in}}%
\pgfpathlineto{\pgfqpoint{3.865174in}{1.931079in}}%
\pgfpathlineto{\pgfqpoint{3.851951in}{1.937156in}}%
\pgfpathlineto{\pgfqpoint{3.838732in}{1.943362in}}%
\pgfpathlineto{\pgfqpoint{3.825517in}{1.949699in}}%
\pgfpathlineto{\pgfqpoint{3.817769in}{1.941730in}}%
\pgfpathlineto{\pgfqpoint{3.810015in}{1.933827in}}%
\pgfpathlineto{\pgfqpoint{3.802255in}{1.925991in}}%
\pgfpathlineto{\pgfqpoint{3.794488in}{1.918224in}}%
\pgfpathclose%
\pgfusepath{fill}%
\end{pgfscope}%
\begin{pgfscope}%
\pgfpathrectangle{\pgfqpoint{1.254980in}{0.150000in}}{\pgfqpoint{5.490039in}{5.490039in}}%
\pgfusepath{clip}%
\pgfsetbuttcap%
\pgfsetroundjoin%
\definecolor{currentfill}{rgb}{0.277941,0.056324,0.381191}%
\pgfsetfillcolor{currentfill}%
\pgfsetfillopacity{0.700000}%
\pgfsetlinewidth{0.000000pt}%
\definecolor{currentstroke}{rgb}{0.000000,0.000000,0.000000}%
\pgfsetstrokecolor{currentstroke}%
\pgfsetdash{}{0pt}%
\pgfpathmoveto{\pgfqpoint{4.151981in}{1.921456in}}%
\pgfpathlineto{\pgfqpoint{4.165277in}{1.918068in}}%
\pgfpathlineto{\pgfqpoint{4.178579in}{1.914802in}}%
\pgfpathlineto{\pgfqpoint{4.191888in}{1.911659in}}%
\pgfpathlineto{\pgfqpoint{4.205204in}{1.908639in}}%
\pgfpathlineto{\pgfqpoint{4.212828in}{1.918251in}}%
\pgfpathlineto{\pgfqpoint{4.220447in}{1.927889in}}%
\pgfpathlineto{\pgfqpoint{4.228062in}{1.937553in}}%
\pgfpathlineto{\pgfqpoint{4.235671in}{1.947241in}}%
\pgfpathlineto{\pgfqpoint{4.222365in}{1.950080in}}%
\pgfpathlineto{\pgfqpoint{4.209067in}{1.953042in}}%
\pgfpathlineto{\pgfqpoint{4.195775in}{1.956127in}}%
\pgfpathlineto{\pgfqpoint{4.182490in}{1.959334in}}%
\pgfpathlineto{\pgfqpoint{4.174870in}{1.949821in}}%
\pgfpathlineto{\pgfqpoint{4.167246in}{1.940337in}}%
\pgfpathlineto{\pgfqpoint{4.159616in}{1.930881in}}%
\pgfpathlineto{\pgfqpoint{4.151981in}{1.921456in}}%
\pgfpathclose%
\pgfusepath{fill}%
\end{pgfscope}%
\begin{pgfscope}%
\pgfpathrectangle{\pgfqpoint{1.254980in}{0.150000in}}{\pgfqpoint{5.490039in}{5.490039in}}%
\pgfusepath{clip}%
\pgfsetbuttcap%
\pgfsetroundjoin%
\definecolor{currentfill}{rgb}{0.201239,0.383670,0.554294}%
\pgfsetfillcolor{currentfill}%
\pgfsetfillopacity{0.700000}%
\pgfsetlinewidth{0.000000pt}%
\definecolor{currentstroke}{rgb}{0.000000,0.000000,0.000000}%
\pgfsetstrokecolor{currentstroke}%
\pgfsetdash{}{0pt}%
\pgfpathmoveto{\pgfqpoint{5.407928in}{2.582603in}}%
\pgfpathlineto{\pgfqpoint{5.421727in}{2.587261in}}%
\pgfpathlineto{\pgfqpoint{5.435539in}{2.592033in}}%
\pgfpathlineto{\pgfqpoint{5.449365in}{2.596917in}}%
\pgfpathlineto{\pgfqpoint{5.463205in}{2.601914in}}%
\pgfpathlineto{\pgfqpoint{5.470407in}{2.611218in}}%
\pgfpathlineto{\pgfqpoint{5.477602in}{2.620473in}}%
\pgfpathlineto{\pgfqpoint{5.484791in}{2.629679in}}%
\pgfpathlineto{\pgfqpoint{5.491975in}{2.638836in}}%
\pgfpathlineto{\pgfqpoint{5.478144in}{2.633900in}}%
\pgfpathlineto{\pgfqpoint{5.464327in}{2.629077in}}%
\pgfpathlineto{\pgfqpoint{5.450524in}{2.624366in}}%
\pgfpathlineto{\pgfqpoint{5.436734in}{2.619768in}}%
\pgfpathlineto{\pgfqpoint{5.429542in}{2.610544in}}%
\pgfpathlineto{\pgfqpoint{5.422343in}{2.601275in}}%
\pgfpathlineto{\pgfqpoint{5.415139in}{2.591962in}}%
\pgfpathlineto{\pgfqpoint{5.407928in}{2.582603in}}%
\pgfpathclose%
\pgfusepath{fill}%
\end{pgfscope}%
\begin{pgfscope}%
\pgfpathrectangle{\pgfqpoint{1.254980in}{0.150000in}}{\pgfqpoint{5.490039in}{5.490039in}}%
\pgfusepath{clip}%
\pgfsetbuttcap%
\pgfsetroundjoin%
\definecolor{currentfill}{rgb}{0.276022,0.044167,0.370164}%
\pgfsetfillcolor{currentfill}%
\pgfsetfillopacity{0.700000}%
\pgfsetlinewidth{0.000000pt}%
\definecolor{currentstroke}{rgb}{0.000000,0.000000,0.000000}%
\pgfsetstrokecolor{currentstroke}%
\pgfsetdash{}{0pt}%
\pgfpathmoveto{\pgfqpoint{3.931357in}{1.902638in}}%
\pgfpathlineto{\pgfqpoint{3.944607in}{1.897335in}}%
\pgfpathlineto{\pgfqpoint{3.957863in}{1.892160in}}%
\pgfpathlineto{\pgfqpoint{3.971124in}{1.887111in}}%
\pgfpathlineto{\pgfqpoint{3.984390in}{1.882189in}}%
\pgfpathlineto{\pgfqpoint{3.992090in}{1.890880in}}%
\pgfpathlineto{\pgfqpoint{3.999785in}{1.899621in}}%
\pgfpathlineto{\pgfqpoint{4.007475in}{1.908411in}}%
\pgfpathlineto{\pgfqpoint{4.015159in}{1.917248in}}%
\pgfpathlineto{\pgfqpoint{4.001906in}{1.921957in}}%
\pgfpathlineto{\pgfqpoint{3.988658in}{1.926793in}}%
\pgfpathlineto{\pgfqpoint{3.975415in}{1.931755in}}%
\pgfpathlineto{\pgfqpoint{3.962177in}{1.936845in}}%
\pgfpathlineto{\pgfqpoint{3.954481in}{1.928215in}}%
\pgfpathlineto{\pgfqpoint{3.946778in}{1.919636in}}%
\pgfpathlineto{\pgfqpoint{3.939070in}{1.911110in}}%
\pgfpathlineto{\pgfqpoint{3.931357in}{1.902638in}}%
\pgfpathclose%
\pgfusepath{fill}%
\end{pgfscope}%
\begin{pgfscope}%
\pgfpathrectangle{\pgfqpoint{1.254980in}{0.150000in}}{\pgfqpoint{5.490039in}{5.490039in}}%
\pgfusepath{clip}%
\pgfsetbuttcap%
\pgfsetroundjoin%
\definecolor{currentfill}{rgb}{0.183898,0.422383,0.556944}%
\pgfsetfillcolor{currentfill}%
\pgfsetfillopacity{0.700000}%
\pgfsetlinewidth{0.000000pt}%
\definecolor{currentstroke}{rgb}{0.000000,0.000000,0.000000}%
\pgfsetstrokecolor{currentstroke}%
\pgfsetdash{}{0pt}%
\pgfpathmoveto{\pgfqpoint{2.745544in}{2.744576in}}%
\pgfpathlineto{\pgfqpoint{2.758913in}{2.726037in}}%
\pgfpathlineto{\pgfqpoint{2.772276in}{2.707685in}}%
\pgfpathlineto{\pgfqpoint{2.785633in}{2.689520in}}%
\pgfpathlineto{\pgfqpoint{2.798983in}{2.671541in}}%
\pgfpathlineto{\pgfqpoint{2.807267in}{2.673350in}}%
\pgfpathlineto{\pgfqpoint{2.815538in}{2.675326in}}%
\pgfpathlineto{\pgfqpoint{2.823797in}{2.677464in}}%
\pgfpathlineto{\pgfqpoint{2.832044in}{2.679764in}}%
\pgfpathlineto{\pgfqpoint{2.818727in}{2.697444in}}%
\pgfpathlineto{\pgfqpoint{2.805405in}{2.715309in}}%
\pgfpathlineto{\pgfqpoint{2.792076in}{2.733360in}}%
\pgfpathlineto{\pgfqpoint{2.778742in}{2.751598in}}%
\pgfpathlineto{\pgfqpoint{2.770461in}{2.749592in}}%
\pgfpathlineto{\pgfqpoint{2.762168in}{2.747751in}}%
\pgfpathlineto{\pgfqpoint{2.753862in}{2.746078in}}%
\pgfpathlineto{\pgfqpoint{2.745544in}{2.744576in}}%
\pgfpathclose%
\pgfusepath{fill}%
\end{pgfscope}%
\begin{pgfscope}%
\pgfpathrectangle{\pgfqpoint{1.254980in}{0.150000in}}{\pgfqpoint{5.490039in}{5.490039in}}%
\pgfusepath{clip}%
\pgfsetbuttcap%
\pgfsetroundjoin%
\definecolor{currentfill}{rgb}{0.271828,0.209303,0.504434}%
\pgfsetfillcolor{currentfill}%
\pgfsetfillopacity{0.700000}%
\pgfsetlinewidth{0.000000pt}%
\definecolor{currentstroke}{rgb}{0.000000,0.000000,0.000000}%
\pgfsetstrokecolor{currentstroke}%
\pgfsetdash{}{0pt}%
\pgfpathmoveto{\pgfqpoint{3.170990in}{2.238944in}}%
\pgfpathlineto{\pgfqpoint{3.184233in}{2.225882in}}%
\pgfpathlineto{\pgfqpoint{3.197474in}{2.212977in}}%
\pgfpathlineto{\pgfqpoint{3.210715in}{2.200226in}}%
\pgfpathlineto{\pgfqpoint{3.223954in}{2.187630in}}%
\pgfpathlineto{\pgfqpoint{3.231996in}{2.191827in}}%
\pgfpathlineto{\pgfqpoint{3.240030in}{2.196155in}}%
\pgfpathlineto{\pgfqpoint{3.248054in}{2.200611in}}%
\pgfpathlineto{\pgfqpoint{3.256069in}{2.205193in}}%
\pgfpathlineto{\pgfqpoint{3.242856in}{2.217504in}}%
\pgfpathlineto{\pgfqpoint{3.229642in}{2.229968in}}%
\pgfpathlineto{\pgfqpoint{3.216426in}{2.242587in}}%
\pgfpathlineto{\pgfqpoint{3.203210in}{2.255362in}}%
\pgfpathlineto{\pgfqpoint{3.195169in}{2.251059in}}%
\pgfpathlineto{\pgfqpoint{3.187119in}{2.246888in}}%
\pgfpathlineto{\pgfqpoint{3.179059in}{2.242848in}}%
\pgfpathlineto{\pgfqpoint{3.170990in}{2.238944in}}%
\pgfpathclose%
\pgfusepath{fill}%
\end{pgfscope}%
\begin{pgfscope}%
\pgfpathrectangle{\pgfqpoint{1.254980in}{0.150000in}}{\pgfqpoint{5.490039in}{5.490039in}}%
\pgfusepath{clip}%
\pgfsetbuttcap%
\pgfsetroundjoin%
\definecolor{currentfill}{rgb}{0.283091,0.110553,0.431554}%
\pgfsetfillcolor{currentfill}%
\pgfsetfillopacity{0.700000}%
\pgfsetlinewidth{0.000000pt}%
\definecolor{currentstroke}{rgb}{0.000000,0.000000,0.000000}%
\pgfsetstrokecolor{currentstroke}%
\pgfsetdash{}{0pt}%
\pgfpathmoveto{\pgfqpoint{3.467447in}{2.028577in}}%
\pgfpathlineto{\pgfqpoint{3.480663in}{2.018775in}}%
\pgfpathlineto{\pgfqpoint{3.493880in}{2.009115in}}%
\pgfpathlineto{\pgfqpoint{3.507098in}{1.999596in}}%
\pgfpathlineto{\pgfqpoint{3.520317in}{1.990217in}}%
\pgfpathlineto{\pgfqpoint{3.528210in}{1.996286in}}%
\pgfpathlineto{\pgfqpoint{3.536096in}{2.002456in}}%
\pgfpathlineto{\pgfqpoint{3.543974in}{2.008725in}}%
\pgfpathlineto{\pgfqpoint{3.551844in}{2.015091in}}%
\pgfpathlineto{\pgfqpoint{3.538645in}{2.024205in}}%
\pgfpathlineto{\pgfqpoint{3.525447in}{2.033460in}}%
\pgfpathlineto{\pgfqpoint{3.512251in}{2.042855in}}%
\pgfpathlineto{\pgfqpoint{3.499057in}{2.052392in}}%
\pgfpathlineto{\pgfqpoint{3.491166in}{2.046285in}}%
\pgfpathlineto{\pgfqpoint{3.483267in}{2.040279in}}%
\pgfpathlineto{\pgfqpoint{3.475361in}{2.034375in}}%
\pgfpathlineto{\pgfqpoint{3.467447in}{2.028577in}}%
\pgfpathclose%
\pgfusepath{fill}%
\end{pgfscope}%
\begin{pgfscope}%
\pgfpathrectangle{\pgfqpoint{1.254980in}{0.150000in}}{\pgfqpoint{5.490039in}{5.490039in}}%
\pgfusepath{clip}%
\pgfsetbuttcap%
\pgfsetroundjoin%
\definecolor{currentfill}{rgb}{0.279566,0.067836,0.391917}%
\pgfsetfillcolor{currentfill}%
\pgfsetfillopacity{0.700000}%
\pgfsetlinewidth{0.000000pt}%
\definecolor{currentstroke}{rgb}{0.000000,0.000000,0.000000}%
\pgfsetstrokecolor{currentstroke}%
\pgfsetdash{}{0pt}%
\pgfpathmoveto{\pgfqpoint{3.657514in}{1.947157in}}%
\pgfpathlineto{\pgfqpoint{3.670734in}{1.939280in}}%
\pgfpathlineto{\pgfqpoint{3.683957in}{1.931537in}}%
\pgfpathlineto{\pgfqpoint{3.697182in}{1.923929in}}%
\pgfpathlineto{\pgfqpoint{3.710411in}{1.916455in}}%
\pgfpathlineto{\pgfqpoint{3.718220in}{1.923675in}}%
\pgfpathlineto{\pgfqpoint{3.726022in}{1.930975in}}%
\pgfpathlineto{\pgfqpoint{3.733817in}{1.938354in}}%
\pgfpathlineto{\pgfqpoint{3.741606in}{1.945810in}}%
\pgfpathlineto{\pgfqpoint{3.728394in}{1.953038in}}%
\pgfpathlineto{\pgfqpoint{3.715185in}{1.960400in}}%
\pgfpathlineto{\pgfqpoint{3.701980in}{1.967896in}}%
\pgfpathlineto{\pgfqpoint{3.688778in}{1.975526in}}%
\pgfpathlineto{\pgfqpoint{3.680972in}{1.968311in}}%
\pgfpathlineto{\pgfqpoint{3.673159in}{1.961176in}}%
\pgfpathlineto{\pgfqpoint{3.665340in}{1.954124in}}%
\pgfpathlineto{\pgfqpoint{3.657514in}{1.947157in}}%
\pgfpathclose%
\pgfusepath{fill}%
\end{pgfscope}%
\begin{pgfscope}%
\pgfpathrectangle{\pgfqpoint{1.254980in}{0.150000in}}{\pgfqpoint{5.490039in}{5.490039in}}%
\pgfusepath{clip}%
\pgfsetbuttcap%
\pgfsetroundjoin%
\definecolor{currentfill}{rgb}{0.281887,0.150881,0.465405}%
\pgfsetfillcolor{currentfill}%
\pgfsetfillopacity{0.700000}%
\pgfsetlinewidth{0.000000pt}%
\definecolor{currentstroke}{rgb}{0.000000,0.000000,0.000000}%
\pgfsetstrokecolor{currentstroke}%
\pgfsetdash{}{0pt}%
\pgfpathmoveto{\pgfqpoint{4.623830in}{2.084437in}}%
\pgfpathlineto{\pgfqpoint{4.637278in}{2.084675in}}%
\pgfpathlineto{\pgfqpoint{4.650735in}{2.085030in}}%
\pgfpathlineto{\pgfqpoint{4.664202in}{2.085502in}}%
\pgfpathlineto{\pgfqpoint{4.677678in}{2.086091in}}%
\pgfpathlineto{\pgfqpoint{4.685158in}{2.096672in}}%
\pgfpathlineto{\pgfqpoint{4.692633in}{2.107238in}}%
\pgfpathlineto{\pgfqpoint{4.700103in}{2.117786in}}%
\pgfpathlineto{\pgfqpoint{4.707568in}{2.128315in}}%
\pgfpathlineto{\pgfqpoint{4.694098in}{2.127625in}}%
\pgfpathlineto{\pgfqpoint{4.680638in}{2.127051in}}%
\pgfpathlineto{\pgfqpoint{4.667188in}{2.126594in}}%
\pgfpathlineto{\pgfqpoint{4.653747in}{2.126254in}}%
\pgfpathlineto{\pgfqpoint{4.646275in}{2.115820in}}%
\pgfpathlineto{\pgfqpoint{4.638798in}{2.105372in}}%
\pgfpathlineto{\pgfqpoint{4.631317in}{2.094911in}}%
\pgfpathlineto{\pgfqpoint{4.623830in}{2.084437in}}%
\pgfpathclose%
\pgfusepath{fill}%
\end{pgfscope}%
\begin{pgfscope}%
\pgfpathrectangle{\pgfqpoint{1.254980in}{0.150000in}}{\pgfqpoint{5.490039in}{5.490039in}}%
\pgfusepath{clip}%
\pgfsetbuttcap%
\pgfsetroundjoin%
\definecolor{currentfill}{rgb}{0.278826,0.175490,0.483397}%
\pgfsetfillcolor{currentfill}%
\pgfsetfillopacity{0.700000}%
\pgfsetlinewidth{0.000000pt}%
\definecolor{currentstroke}{rgb}{0.000000,0.000000,0.000000}%
\pgfsetstrokecolor{currentstroke}%
\pgfsetdash{}{0pt}%
\pgfpathmoveto{\pgfqpoint{4.707568in}{2.128315in}}%
\pgfpathlineto{\pgfqpoint{4.721048in}{2.129123in}}%
\pgfpathlineto{\pgfqpoint{4.734539in}{2.130046in}}%
\pgfpathlineto{\pgfqpoint{4.748039in}{2.131087in}}%
\pgfpathlineto{\pgfqpoint{4.761550in}{2.132243in}}%
\pgfpathlineto{\pgfqpoint{4.769004in}{2.142846in}}%
\pgfpathlineto{\pgfqpoint{4.776453in}{2.153426in}}%
\pgfpathlineto{\pgfqpoint{4.783897in}{2.163983in}}%
\pgfpathlineto{\pgfqpoint{4.791336in}{2.174517in}}%
\pgfpathlineto{\pgfqpoint{4.777832in}{2.173275in}}%
\pgfpathlineto{\pgfqpoint{4.764338in}{2.172149in}}%
\pgfpathlineto{\pgfqpoint{4.750854in}{2.171139in}}%
\pgfpathlineto{\pgfqpoint{4.737380in}{2.170246in}}%
\pgfpathlineto{\pgfqpoint{4.729935in}{2.159793in}}%
\pgfpathlineto{\pgfqpoint{4.722484in}{2.149319in}}%
\pgfpathlineto{\pgfqpoint{4.715029in}{2.138827in}}%
\pgfpathlineto{\pgfqpoint{4.707568in}{2.128315in}}%
\pgfpathclose%
\pgfusepath{fill}%
\end{pgfscope}%
\begin{pgfscope}%
\pgfpathrectangle{\pgfqpoint{1.254980in}{0.150000in}}{\pgfqpoint{5.490039in}{5.490039in}}%
\pgfusepath{clip}%
\pgfsetbuttcap%
\pgfsetroundjoin%
\definecolor{currentfill}{rgb}{0.283072,0.130895,0.449241}%
\pgfsetfillcolor{currentfill}%
\pgfsetfillopacity{0.700000}%
\pgfsetlinewidth{0.000000pt}%
\definecolor{currentstroke}{rgb}{0.000000,0.000000,0.000000}%
\pgfsetstrokecolor{currentstroke}%
\pgfsetdash{}{0pt}%
\pgfpathmoveto{\pgfqpoint{4.540113in}{2.043121in}}%
\pgfpathlineto{\pgfqpoint{4.553530in}{2.042771in}}%
\pgfpathlineto{\pgfqpoint{4.566956in}{2.042538in}}%
\pgfpathlineto{\pgfqpoint{4.580391in}{2.042424in}}%
\pgfpathlineto{\pgfqpoint{4.593836in}{2.042426in}}%
\pgfpathlineto{\pgfqpoint{4.601342in}{2.052945in}}%
\pgfpathlineto{\pgfqpoint{4.608843in}{2.063453in}}%
\pgfpathlineto{\pgfqpoint{4.616339in}{2.073951in}}%
\pgfpathlineto{\pgfqpoint{4.623830in}{2.084437in}}%
\pgfpathlineto{\pgfqpoint{4.610392in}{2.084316in}}%
\pgfpathlineto{\pgfqpoint{4.596964in}{2.084313in}}%
\pgfpathlineto{\pgfqpoint{4.583545in}{2.084428in}}%
\pgfpathlineto{\pgfqpoint{4.570135in}{2.084661in}}%
\pgfpathlineto{\pgfqpoint{4.562636in}{2.074286in}}%
\pgfpathlineto{\pgfqpoint{4.555133in}{2.063905in}}%
\pgfpathlineto{\pgfqpoint{4.547625in}{2.053516in}}%
\pgfpathlineto{\pgfqpoint{4.540113in}{2.043121in}}%
\pgfpathclose%
\pgfusepath{fill}%
\end{pgfscope}%
\begin{pgfscope}%
\pgfpathrectangle{\pgfqpoint{1.254980in}{0.150000in}}{\pgfqpoint{5.490039in}{5.490039in}}%
\pgfusepath{clip}%
\pgfsetbuttcap%
\pgfsetroundjoin%
\definecolor{currentfill}{rgb}{0.274128,0.199721,0.498911}%
\pgfsetfillcolor{currentfill}%
\pgfsetfillopacity{0.700000}%
\pgfsetlinewidth{0.000000pt}%
\definecolor{currentstroke}{rgb}{0.000000,0.000000,0.000000}%
\pgfsetstrokecolor{currentstroke}%
\pgfsetdash{}{0pt}%
\pgfpathmoveto{\pgfqpoint{4.791336in}{2.174517in}}%
\pgfpathlineto{\pgfqpoint{4.804851in}{2.175875in}}%
\pgfpathlineto{\pgfqpoint{4.818377in}{2.177349in}}%
\pgfpathlineto{\pgfqpoint{4.831913in}{2.178938in}}%
\pgfpathlineto{\pgfqpoint{4.845460in}{2.180643in}}%
\pgfpathlineto{\pgfqpoint{4.852888in}{2.191228in}}%
\pgfpathlineto{\pgfqpoint{4.860311in}{2.201784in}}%
\pgfpathlineto{\pgfqpoint{4.867728in}{2.212311in}}%
\pgfpathlineto{\pgfqpoint{4.875141in}{2.222809in}}%
\pgfpathlineto{\pgfqpoint{4.861600in}{2.221034in}}%
\pgfpathlineto{\pgfqpoint{4.848070in}{2.219375in}}%
\pgfpathlineto{\pgfqpoint{4.834551in}{2.217831in}}%
\pgfpathlineto{\pgfqpoint{4.821043in}{2.216403in}}%
\pgfpathlineto{\pgfqpoint{4.813624in}{2.205969in}}%
\pgfpathlineto{\pgfqpoint{4.806199in}{2.195510in}}%
\pgfpathlineto{\pgfqpoint{4.798770in}{2.185025in}}%
\pgfpathlineto{\pgfqpoint{4.791336in}{2.174517in}}%
\pgfpathclose%
\pgfusepath{fill}%
\end{pgfscope}%
\begin{pgfscope}%
\pgfpathrectangle{\pgfqpoint{1.254980in}{0.150000in}}{\pgfqpoint{5.490039in}{5.490039in}}%
\pgfusepath{clip}%
\pgfsetbuttcap%
\pgfsetroundjoin%
\definecolor{currentfill}{rgb}{0.190631,0.407061,0.556089}%
\pgfsetfillcolor{currentfill}%
\pgfsetfillopacity{0.700000}%
\pgfsetlinewidth{0.000000pt}%
\definecolor{currentstroke}{rgb}{0.000000,0.000000,0.000000}%
\pgfsetstrokecolor{currentstroke}%
\pgfsetdash{}{0pt}%
\pgfpathmoveto{\pgfqpoint{5.491975in}{2.638836in}}%
\pgfpathlineto{\pgfqpoint{5.505820in}{2.643884in}}%
\pgfpathlineto{\pgfqpoint{5.519679in}{2.649044in}}%
\pgfpathlineto{\pgfqpoint{5.533552in}{2.654317in}}%
\pgfpathlineto{\pgfqpoint{5.547439in}{2.659702in}}%
\pgfpathlineto{\pgfqpoint{5.554607in}{2.668740in}}%
\pgfpathlineto{\pgfqpoint{5.561768in}{2.677728in}}%
\pgfpathlineto{\pgfqpoint{5.568923in}{2.686666in}}%
\pgfpathlineto{\pgfqpoint{5.576072in}{2.695556in}}%
\pgfpathlineto{\pgfqpoint{5.562194in}{2.690248in}}%
\pgfpathlineto{\pgfqpoint{5.548331in}{2.685053in}}%
\pgfpathlineto{\pgfqpoint{5.534482in}{2.679970in}}%
\pgfpathlineto{\pgfqpoint{5.520647in}{2.675000in}}%
\pgfpathlineto{\pgfqpoint{5.513488in}{2.666027in}}%
\pgfpathlineto{\pgfqpoint{5.506323in}{2.657009in}}%
\pgfpathlineto{\pgfqpoint{5.499152in}{2.647946in}}%
\pgfpathlineto{\pgfqpoint{5.491975in}{2.638836in}}%
\pgfpathclose%
\pgfusepath{fill}%
\end{pgfscope}%
\begin{pgfscope}%
\pgfpathrectangle{\pgfqpoint{1.254980in}{0.150000in}}{\pgfqpoint{5.490039in}{5.490039in}}%
\pgfusepath{clip}%
\pgfsetbuttcap%
\pgfsetroundjoin%
\definecolor{currentfill}{rgb}{0.267968,0.223549,0.512008}%
\pgfsetfillcolor{currentfill}%
\pgfsetfillopacity{0.700000}%
\pgfsetlinewidth{0.000000pt}%
\definecolor{currentstroke}{rgb}{0.000000,0.000000,0.000000}%
\pgfsetstrokecolor{currentstroke}%
\pgfsetdash{}{0pt}%
\pgfpathmoveto{\pgfqpoint{4.875141in}{2.222809in}}%
\pgfpathlineto{\pgfqpoint{4.888693in}{2.224700in}}%
\pgfpathlineto{\pgfqpoint{4.902256in}{2.226705in}}%
\pgfpathlineto{\pgfqpoint{4.915830in}{2.228826in}}%
\pgfpathlineto{\pgfqpoint{4.929415in}{2.231062in}}%
\pgfpathlineto{\pgfqpoint{4.936817in}{2.241590in}}%
\pgfpathlineto{\pgfqpoint{4.944213in}{2.252084in}}%
\pgfpathlineto{\pgfqpoint{4.951604in}{2.262545in}}%
\pgfpathlineto{\pgfqpoint{4.958990in}{2.272971in}}%
\pgfpathlineto{\pgfqpoint{4.945411in}{2.270682in}}%
\pgfpathlineto{\pgfqpoint{4.931843in}{2.268507in}}%
\pgfpathlineto{\pgfqpoint{4.918287in}{2.266448in}}%
\pgfpathlineto{\pgfqpoint{4.904741in}{2.264504in}}%
\pgfpathlineto{\pgfqpoint{4.897349in}{2.254125in}}%
\pgfpathlineto{\pgfqpoint{4.889951in}{2.243716in}}%
\pgfpathlineto{\pgfqpoint{4.882549in}{2.233278in}}%
\pgfpathlineto{\pgfqpoint{4.875141in}{2.222809in}}%
\pgfpathclose%
\pgfusepath{fill}%
\end{pgfscope}%
\begin{pgfscope}%
\pgfpathrectangle{\pgfqpoint{1.254980in}{0.150000in}}{\pgfqpoint{5.490039in}{5.490039in}}%
\pgfusepath{clip}%
\pgfsetbuttcap%
\pgfsetroundjoin%
\definecolor{currentfill}{rgb}{0.283091,0.110553,0.431554}%
\pgfsetfillcolor{currentfill}%
\pgfsetfillopacity{0.700000}%
\pgfsetlinewidth{0.000000pt}%
\definecolor{currentstroke}{rgb}{0.000000,0.000000,0.000000}%
\pgfsetstrokecolor{currentstroke}%
\pgfsetdash{}{0pt}%
\pgfpathmoveto{\pgfqpoint{4.456404in}{2.004618in}}%
\pgfpathlineto{\pgfqpoint{4.469794in}{2.003661in}}%
\pgfpathlineto{\pgfqpoint{4.483192in}{2.002822in}}%
\pgfpathlineto{\pgfqpoint{4.496598in}{2.002101in}}%
\pgfpathlineto{\pgfqpoint{4.510014in}{2.001499in}}%
\pgfpathlineto{\pgfqpoint{4.517546in}{2.011909in}}%
\pgfpathlineto{\pgfqpoint{4.525073in}{2.022317in}}%
\pgfpathlineto{\pgfqpoint{4.532595in}{2.032721in}}%
\pgfpathlineto{\pgfqpoint{4.540113in}{2.043121in}}%
\pgfpathlineto{\pgfqpoint{4.526704in}{2.043590in}}%
\pgfpathlineto{\pgfqpoint{4.513305in}{2.044177in}}%
\pgfpathlineto{\pgfqpoint{4.499915in}{2.044882in}}%
\pgfpathlineto{\pgfqpoint{4.486533in}{2.045707in}}%
\pgfpathlineto{\pgfqpoint{4.479008in}{2.035434in}}%
\pgfpathlineto{\pgfqpoint{4.471478in}{2.025162in}}%
\pgfpathlineto{\pgfqpoint{4.463944in}{2.014889in}}%
\pgfpathlineto{\pgfqpoint{4.456404in}{2.004618in}}%
\pgfpathclose%
\pgfusepath{fill}%
\end{pgfscope}%
\begin{pgfscope}%
\pgfpathrectangle{\pgfqpoint{1.254980in}{0.150000in}}{\pgfqpoint{5.490039in}{5.490039in}}%
\pgfusepath{clip}%
\pgfsetbuttcap%
\pgfsetroundjoin%
\definecolor{currentfill}{rgb}{0.277134,0.185228,0.489898}%
\pgfsetfillcolor{currentfill}%
\pgfsetfillopacity{0.700000}%
\pgfsetlinewidth{0.000000pt}%
\definecolor{currentstroke}{rgb}{0.000000,0.000000,0.000000}%
\pgfsetstrokecolor{currentstroke}%
\pgfsetdash{}{0pt}%
\pgfpathmoveto{\pgfqpoint{3.223954in}{2.187630in}}%
\pgfpathlineto{\pgfqpoint{3.237191in}{2.175187in}}%
\pgfpathlineto{\pgfqpoint{3.250428in}{2.162897in}}%
\pgfpathlineto{\pgfqpoint{3.263664in}{2.150760in}}%
\pgfpathlineto{\pgfqpoint{3.276899in}{2.138773in}}%
\pgfpathlineto{\pgfqpoint{3.284917in}{2.143262in}}%
\pgfpathlineto{\pgfqpoint{3.292925in}{2.147877in}}%
\pgfpathlineto{\pgfqpoint{3.300924in}{2.152616in}}%
\pgfpathlineto{\pgfqpoint{3.308915in}{2.157477in}}%
\pgfpathlineto{\pgfqpoint{3.295704in}{2.169179in}}%
\pgfpathlineto{\pgfqpoint{3.282493in}{2.181032in}}%
\pgfpathlineto{\pgfqpoint{3.269282in}{2.193036in}}%
\pgfpathlineto{\pgfqpoint{3.256069in}{2.205193in}}%
\pgfpathlineto{\pgfqpoint{3.248054in}{2.200611in}}%
\pgfpathlineto{\pgfqpoint{3.240030in}{2.196155in}}%
\pgfpathlineto{\pgfqpoint{3.231996in}{2.191827in}}%
\pgfpathlineto{\pgfqpoint{3.223954in}{2.187630in}}%
\pgfpathclose%
\pgfusepath{fill}%
\end{pgfscope}%
\begin{pgfscope}%
\pgfpathrectangle{\pgfqpoint{1.254980in}{0.150000in}}{\pgfqpoint{5.490039in}{5.490039in}}%
\pgfusepath{clip}%
\pgfsetbuttcap%
\pgfsetroundjoin%
\definecolor{currentfill}{rgb}{0.171176,0.452530,0.557965}%
\pgfsetfillcolor{currentfill}%
\pgfsetfillopacity{0.700000}%
\pgfsetlinewidth{0.000000pt}%
\definecolor{currentstroke}{rgb}{0.000000,0.000000,0.000000}%
\pgfsetstrokecolor{currentstroke}%
\pgfsetdash{}{0pt}%
\pgfpathmoveto{\pgfqpoint{2.692000in}{2.820633in}}%
\pgfpathlineto{\pgfqpoint{2.705397in}{2.801331in}}%
\pgfpathlineto{\pgfqpoint{2.718786in}{2.782221in}}%
\pgfpathlineto{\pgfqpoint{2.732168in}{2.763304in}}%
\pgfpathlineto{\pgfqpoint{2.745544in}{2.744576in}}%
\pgfpathlineto{\pgfqpoint{2.753862in}{2.746078in}}%
\pgfpathlineto{\pgfqpoint{2.762168in}{2.747751in}}%
\pgfpathlineto{\pgfqpoint{2.770461in}{2.749592in}}%
\pgfpathlineto{\pgfqpoint{2.778742in}{2.751598in}}%
\pgfpathlineto{\pgfqpoint{2.765401in}{2.770024in}}%
\pgfpathlineto{\pgfqpoint{2.752054in}{2.788639in}}%
\pgfpathlineto{\pgfqpoint{2.738700in}{2.807446in}}%
\pgfpathlineto{\pgfqpoint{2.725339in}{2.826444in}}%
\pgfpathlineto{\pgfqpoint{2.717023in}{2.824734in}}%
\pgfpathlineto{\pgfqpoint{2.708695in}{2.823194in}}%
\pgfpathlineto{\pgfqpoint{2.700354in}{2.821826in}}%
\pgfpathlineto{\pgfqpoint{2.692000in}{2.820633in}}%
\pgfpathclose%
\pgfusepath{fill}%
\end{pgfscope}%
\begin{pgfscope}%
\pgfpathrectangle{\pgfqpoint{1.254980in}{0.150000in}}{\pgfqpoint{5.490039in}{5.490039in}}%
\pgfusepath{clip}%
\pgfsetbuttcap%
\pgfsetroundjoin%
\definecolor{currentfill}{rgb}{0.277018,0.050344,0.375715}%
\pgfsetfillcolor{currentfill}%
\pgfsetfillopacity{0.700000}%
\pgfsetlinewidth{0.000000pt}%
\definecolor{currentstroke}{rgb}{0.000000,0.000000,0.000000}%
\pgfsetstrokecolor{currentstroke}%
\pgfsetdash{}{0pt}%
\pgfpathmoveto{\pgfqpoint{4.068228in}{1.899671in}}%
\pgfpathlineto{\pgfqpoint{4.081510in}{1.895589in}}%
\pgfpathlineto{\pgfqpoint{4.094797in}{1.891633in}}%
\pgfpathlineto{\pgfqpoint{4.108091in}{1.887800in}}%
\pgfpathlineto{\pgfqpoint{4.121391in}{1.884091in}}%
\pgfpathlineto{\pgfqpoint{4.129046in}{1.893379in}}%
\pgfpathlineto{\pgfqpoint{4.136697in}{1.902704in}}%
\pgfpathlineto{\pgfqpoint{4.144342in}{1.912063in}}%
\pgfpathlineto{\pgfqpoint{4.151981in}{1.921456in}}%
\pgfpathlineto{\pgfqpoint{4.138692in}{1.924969in}}%
\pgfpathlineto{\pgfqpoint{4.125410in}{1.928605in}}%
\pgfpathlineto{\pgfqpoint{4.112133in}{1.932365in}}%
\pgfpathlineto{\pgfqpoint{4.098863in}{1.936249in}}%
\pgfpathlineto{\pgfqpoint{4.091212in}{1.927047in}}%
\pgfpathlineto{\pgfqpoint{4.083556in}{1.917883in}}%
\pgfpathlineto{\pgfqpoint{4.075895in}{1.908757in}}%
\pgfpathlineto{\pgfqpoint{4.068228in}{1.899671in}}%
\pgfpathclose%
\pgfusepath{fill}%
\end{pgfscope}%
\begin{pgfscope}%
\pgfpathrectangle{\pgfqpoint{1.254980in}{0.150000in}}{\pgfqpoint{5.490039in}{5.490039in}}%
\pgfusepath{clip}%
\pgfsetbuttcap%
\pgfsetroundjoin%
\definecolor{currentfill}{rgb}{0.260571,0.246922,0.522828}%
\pgfsetfillcolor{currentfill}%
\pgfsetfillopacity{0.700000}%
\pgfsetlinewidth{0.000000pt}%
\definecolor{currentstroke}{rgb}{0.000000,0.000000,0.000000}%
\pgfsetstrokecolor{currentstroke}%
\pgfsetdash{}{0pt}%
\pgfpathmoveto{\pgfqpoint{4.958990in}{2.272971in}}%
\pgfpathlineto{\pgfqpoint{4.972581in}{2.275376in}}%
\pgfpathlineto{\pgfqpoint{4.986183in}{2.277895in}}%
\pgfpathlineto{\pgfqpoint{4.999796in}{2.280529in}}%
\pgfpathlineto{\pgfqpoint{5.013422in}{2.283277in}}%
\pgfpathlineto{\pgfqpoint{5.020796in}{2.293713in}}%
\pgfpathlineto{\pgfqpoint{5.028165in}{2.304110in}}%
\pgfpathlineto{\pgfqpoint{5.035529in}{2.314470in}}%
\pgfpathlineto{\pgfqpoint{5.042888in}{2.324791in}}%
\pgfpathlineto{\pgfqpoint{5.029269in}{2.322005in}}%
\pgfpathlineto{\pgfqpoint{5.015661in}{2.319333in}}%
\pgfpathlineto{\pgfqpoint{5.002066in}{2.316777in}}%
\pgfpathlineto{\pgfqpoint{4.988482in}{2.314334in}}%
\pgfpathlineto{\pgfqpoint{4.981117in}{2.304045in}}%
\pgfpathlineto{\pgfqpoint{4.973746in}{2.293722in}}%
\pgfpathlineto{\pgfqpoint{4.966371in}{2.283364in}}%
\pgfpathlineto{\pgfqpoint{4.958990in}{2.272971in}}%
\pgfpathclose%
\pgfusepath{fill}%
\end{pgfscope}%
\begin{pgfscope}%
\pgfpathrectangle{\pgfqpoint{1.254980in}{0.150000in}}{\pgfqpoint{5.490039in}{5.490039in}}%
\pgfusepath{clip}%
\pgfsetbuttcap%
\pgfsetroundjoin%
\definecolor{currentfill}{rgb}{0.282327,0.094955,0.417331}%
\pgfsetfillcolor{currentfill}%
\pgfsetfillopacity{0.700000}%
\pgfsetlinewidth{0.000000pt}%
\definecolor{currentstroke}{rgb}{0.000000,0.000000,0.000000}%
\pgfsetstrokecolor{currentstroke}%
\pgfsetdash{}{0pt}%
\pgfpathmoveto{\pgfqpoint{4.372693in}{1.969189in}}%
\pgfpathlineto{\pgfqpoint{4.386057in}{1.967604in}}%
\pgfpathlineto{\pgfqpoint{4.399430in}{1.966139in}}%
\pgfpathlineto{\pgfqpoint{4.412810in}{1.964794in}}%
\pgfpathlineto{\pgfqpoint{4.426199in}{1.963567in}}%
\pgfpathlineto{\pgfqpoint{4.433757in}{1.973823in}}%
\pgfpathlineto{\pgfqpoint{4.441311in}{1.984084in}}%
\pgfpathlineto{\pgfqpoint{4.448860in}{1.994350in}}%
\pgfpathlineto{\pgfqpoint{4.456404in}{2.004618in}}%
\pgfpathlineto{\pgfqpoint{4.443024in}{2.005695in}}%
\pgfpathlineto{\pgfqpoint{4.429651in}{2.006892in}}%
\pgfpathlineto{\pgfqpoint{4.416287in}{2.008207in}}%
\pgfpathlineto{\pgfqpoint{4.402931in}{2.009643in}}%
\pgfpathlineto{\pgfqpoint{4.395379in}{1.999517in}}%
\pgfpathlineto{\pgfqpoint{4.387822in}{1.989399in}}%
\pgfpathlineto{\pgfqpoint{4.380260in}{1.979289in}}%
\pgfpathlineto{\pgfqpoint{4.372693in}{1.969189in}}%
\pgfpathclose%
\pgfusepath{fill}%
\end{pgfscope}%
\begin{pgfscope}%
\pgfpathrectangle{\pgfqpoint{1.254980in}{0.150000in}}{\pgfqpoint{5.490039in}{5.490039in}}%
\pgfusepath{clip}%
\pgfsetbuttcap%
\pgfsetroundjoin%
\definecolor{currentfill}{rgb}{0.180629,0.429975,0.557282}%
\pgfsetfillcolor{currentfill}%
\pgfsetfillopacity{0.700000}%
\pgfsetlinewidth{0.000000pt}%
\definecolor{currentstroke}{rgb}{0.000000,0.000000,0.000000}%
\pgfsetstrokecolor{currentstroke}%
\pgfsetdash{}{0pt}%
\pgfpathmoveto{\pgfqpoint{5.576072in}{2.695556in}}%
\pgfpathlineto{\pgfqpoint{5.589964in}{2.700975in}}%
\pgfpathlineto{\pgfqpoint{5.603871in}{2.706506in}}%
\pgfpathlineto{\pgfqpoint{5.617792in}{2.712150in}}%
\pgfpathlineto{\pgfqpoint{5.631727in}{2.717905in}}%
\pgfpathlineto{\pgfqpoint{5.638860in}{2.726659in}}%
\pgfpathlineto{\pgfqpoint{5.645985in}{2.735361in}}%
\pgfpathlineto{\pgfqpoint{5.653105in}{2.744015in}}%
\pgfpathlineto{\pgfqpoint{5.660218in}{2.752620in}}%
\pgfpathlineto{\pgfqpoint{5.646293in}{2.746959in}}%
\pgfpathlineto{\pgfqpoint{5.632383in}{2.741410in}}%
\pgfpathlineto{\pgfqpoint{5.618487in}{2.735973in}}%
\pgfpathlineto{\pgfqpoint{5.604606in}{2.730648in}}%
\pgfpathlineto{\pgfqpoint{5.597482in}{2.721943in}}%
\pgfpathlineto{\pgfqpoint{5.590351in}{2.713193in}}%
\pgfpathlineto{\pgfqpoint{5.583215in}{2.704397in}}%
\pgfpathlineto{\pgfqpoint{5.576072in}{2.695556in}}%
\pgfpathclose%
\pgfusepath{fill}%
\end{pgfscope}%
\begin{pgfscope}%
\pgfpathrectangle{\pgfqpoint{1.254980in}{0.150000in}}{\pgfqpoint{5.490039in}{5.490039in}}%
\pgfusepath{clip}%
\pgfsetbuttcap%
\pgfsetroundjoin%
\definecolor{currentfill}{rgb}{0.250425,0.274290,0.533103}%
\pgfsetfillcolor{currentfill}%
\pgfsetfillopacity{0.700000}%
\pgfsetlinewidth{0.000000pt}%
\definecolor{currentstroke}{rgb}{0.000000,0.000000,0.000000}%
\pgfsetstrokecolor{currentstroke}%
\pgfsetdash{}{0pt}%
\pgfpathmoveto{\pgfqpoint{5.042888in}{2.324791in}}%
\pgfpathlineto{\pgfqpoint{5.056519in}{2.327691in}}%
\pgfpathlineto{\pgfqpoint{5.070162in}{2.330705in}}%
\pgfpathlineto{\pgfqpoint{5.083817in}{2.333834in}}%
\pgfpathlineto{\pgfqpoint{5.097484in}{2.337076in}}%
\pgfpathlineto{\pgfqpoint{5.104831in}{2.347386in}}%
\pgfpathlineto{\pgfqpoint{5.112172in}{2.357654in}}%
\pgfpathlineto{\pgfqpoint{5.119508in}{2.367879in}}%
\pgfpathlineto{\pgfqpoint{5.126838in}{2.378063in}}%
\pgfpathlineto{\pgfqpoint{5.113178in}{2.374799in}}%
\pgfpathlineto{\pgfqpoint{5.099529in}{2.371649in}}%
\pgfpathlineto{\pgfqpoint{5.085893in}{2.368613in}}%
\pgfpathlineto{\pgfqpoint{5.072269in}{2.365691in}}%
\pgfpathlineto{\pgfqpoint{5.064932in}{2.355523in}}%
\pgfpathlineto{\pgfqpoint{5.057589in}{2.345317in}}%
\pgfpathlineto{\pgfqpoint{5.050241in}{2.335073in}}%
\pgfpathlineto{\pgfqpoint{5.042888in}{2.324791in}}%
\pgfpathclose%
\pgfusepath{fill}%
\end{pgfscope}%
\begin{pgfscope}%
\pgfpathrectangle{\pgfqpoint{1.254980in}{0.150000in}}{\pgfqpoint{5.490039in}{5.490039in}}%
\pgfusepath{clip}%
\pgfsetbuttcap%
\pgfsetroundjoin%
\definecolor{currentfill}{rgb}{0.282327,0.094955,0.417331}%
\pgfsetfillcolor{currentfill}%
\pgfsetfillopacity{0.700000}%
\pgfsetlinewidth{0.000000pt}%
\definecolor{currentstroke}{rgb}{0.000000,0.000000,0.000000}%
\pgfsetstrokecolor{currentstroke}%
\pgfsetdash{}{0pt}%
\pgfpathmoveto{\pgfqpoint{3.520317in}{1.990217in}}%
\pgfpathlineto{\pgfqpoint{3.533538in}{1.980978in}}%
\pgfpathlineto{\pgfqpoint{3.546761in}{1.971878in}}%
\pgfpathlineto{\pgfqpoint{3.559986in}{1.962918in}}%
\pgfpathlineto{\pgfqpoint{3.573213in}{1.954095in}}%
\pgfpathlineto{\pgfqpoint{3.581086in}{1.960434in}}%
\pgfpathlineto{\pgfqpoint{3.588951in}{1.966870in}}%
\pgfpathlineto{\pgfqpoint{3.596810in}{1.973401in}}%
\pgfpathlineto{\pgfqpoint{3.604661in}{1.980025in}}%
\pgfpathlineto{\pgfqpoint{3.591454in}{1.988584in}}%
\pgfpathlineto{\pgfqpoint{3.578249in}{1.997281in}}%
\pgfpathlineto{\pgfqpoint{3.565045in}{2.006116in}}%
\pgfpathlineto{\pgfqpoint{3.551844in}{2.015091in}}%
\pgfpathlineto{\pgfqpoint{3.543974in}{2.008725in}}%
\pgfpathlineto{\pgfqpoint{3.536096in}{2.002456in}}%
\pgfpathlineto{\pgfqpoint{3.528210in}{1.996286in}}%
\pgfpathlineto{\pgfqpoint{3.520317in}{1.990217in}}%
\pgfpathclose%
\pgfusepath{fill}%
\end{pgfscope}%
\begin{pgfscope}%
\pgfpathrectangle{\pgfqpoint{1.254980in}{0.150000in}}{\pgfqpoint{5.490039in}{5.490039in}}%
\pgfusepath{clip}%
\pgfsetbuttcap%
\pgfsetroundjoin%
\definecolor{currentfill}{rgb}{0.280255,0.165693,0.476498}%
\pgfsetfillcolor{currentfill}%
\pgfsetfillopacity{0.700000}%
\pgfsetlinewidth{0.000000pt}%
\definecolor{currentstroke}{rgb}{0.000000,0.000000,0.000000}%
\pgfsetstrokecolor{currentstroke}%
\pgfsetdash{}{0pt}%
\pgfpathmoveto{\pgfqpoint{3.276899in}{2.138773in}}%
\pgfpathlineto{\pgfqpoint{3.290134in}{2.126937in}}%
\pgfpathlineto{\pgfqpoint{3.303368in}{2.115251in}}%
\pgfpathlineto{\pgfqpoint{3.316602in}{2.103714in}}%
\pgfpathlineto{\pgfqpoint{3.329836in}{2.092325in}}%
\pgfpathlineto{\pgfqpoint{3.337828in}{2.097104in}}%
\pgfpathlineto{\pgfqpoint{3.345812in}{2.102005in}}%
\pgfpathlineto{\pgfqpoint{3.353787in}{2.107026in}}%
\pgfpathlineto{\pgfqpoint{3.361754in}{2.112165in}}%
\pgfpathlineto{\pgfqpoint{3.348544in}{2.123270in}}%
\pgfpathlineto{\pgfqpoint{3.335335in}{2.134523in}}%
\pgfpathlineto{\pgfqpoint{3.322125in}{2.145925in}}%
\pgfpathlineto{\pgfqpoint{3.308915in}{2.157477in}}%
\pgfpathlineto{\pgfqpoint{3.300924in}{2.152616in}}%
\pgfpathlineto{\pgfqpoint{3.292925in}{2.147877in}}%
\pgfpathlineto{\pgfqpoint{3.284917in}{2.143262in}}%
\pgfpathlineto{\pgfqpoint{3.276899in}{2.138773in}}%
\pgfpathclose%
\pgfusepath{fill}%
\end{pgfscope}%
\begin{pgfscope}%
\pgfpathrectangle{\pgfqpoint{1.254980in}{0.150000in}}{\pgfqpoint{5.490039in}{5.490039in}}%
\pgfusepath{clip}%
\pgfsetbuttcap%
\pgfsetroundjoin%
\definecolor{currentfill}{rgb}{0.159194,0.482237,0.558073}%
\pgfsetfillcolor{currentfill}%
\pgfsetfillopacity{0.700000}%
\pgfsetlinewidth{0.000000pt}%
\definecolor{currentstroke}{rgb}{0.000000,0.000000,0.000000}%
\pgfsetstrokecolor{currentstroke}%
\pgfsetdash{}{0pt}%
\pgfpathmoveto{\pgfqpoint{2.638340in}{2.899792in}}%
\pgfpathlineto{\pgfqpoint{2.651766in}{2.879707in}}%
\pgfpathlineto{\pgfqpoint{2.665185in}{2.859819in}}%
\pgfpathlineto{\pgfqpoint{2.678596in}{2.840128in}}%
\pgfpathlineto{\pgfqpoint{2.692000in}{2.820633in}}%
\pgfpathlineto{\pgfqpoint{2.700354in}{2.821826in}}%
\pgfpathlineto{\pgfqpoint{2.708695in}{2.823194in}}%
\pgfpathlineto{\pgfqpoint{2.717023in}{2.824734in}}%
\pgfpathlineto{\pgfqpoint{2.725339in}{2.826444in}}%
\pgfpathlineto{\pgfqpoint{2.711971in}{2.845636in}}%
\pgfpathlineto{\pgfqpoint{2.698596in}{2.865023in}}%
\pgfpathlineto{\pgfqpoint{2.685214in}{2.884605in}}%
\pgfpathlineto{\pgfqpoint{2.671824in}{2.904384in}}%
\pgfpathlineto{\pgfqpoint{2.663473in}{2.902972in}}%
\pgfpathlineto{\pgfqpoint{2.655108in}{2.901734in}}%
\pgfpathlineto{\pgfqpoint{2.646731in}{2.900674in}}%
\pgfpathlineto{\pgfqpoint{2.638340in}{2.899792in}}%
\pgfpathclose%
\pgfusepath{fill}%
\end{pgfscope}%
\begin{pgfscope}%
\pgfpathrectangle{\pgfqpoint{1.254980in}{0.150000in}}{\pgfqpoint{5.490039in}{5.490039in}}%
\pgfusepath{clip}%
\pgfsetbuttcap%
\pgfsetroundjoin%
\definecolor{currentfill}{rgb}{0.280894,0.078907,0.402329}%
\pgfsetfillcolor{currentfill}%
\pgfsetfillopacity{0.700000}%
\pgfsetlinewidth{0.000000pt}%
\definecolor{currentstroke}{rgb}{0.000000,0.000000,0.000000}%
\pgfsetstrokecolor{currentstroke}%
\pgfsetdash{}{0pt}%
\pgfpathmoveto{\pgfqpoint{4.288965in}{1.937102in}}%
\pgfpathlineto{\pgfqpoint{4.302307in}{1.934871in}}%
\pgfpathlineto{\pgfqpoint{4.315656in}{1.932760in}}%
\pgfpathlineto{\pgfqpoint{4.329013in}{1.930770in}}%
\pgfpathlineto{\pgfqpoint{4.342378in}{1.928900in}}%
\pgfpathlineto{\pgfqpoint{4.349964in}{1.938953in}}%
\pgfpathlineto{\pgfqpoint{4.357545in}{1.949019in}}%
\pgfpathlineto{\pgfqpoint{4.365122in}{1.959098in}}%
\pgfpathlineto{\pgfqpoint{4.372693in}{1.969189in}}%
\pgfpathlineto{\pgfqpoint{4.359337in}{1.970893in}}%
\pgfpathlineto{\pgfqpoint{4.345989in}{1.972718in}}%
\pgfpathlineto{\pgfqpoint{4.332648in}{1.974664in}}%
\pgfpathlineto{\pgfqpoint{4.319315in}{1.976730in}}%
\pgfpathlineto{\pgfqpoint{4.311735in}{1.966799in}}%
\pgfpathlineto{\pgfqpoint{4.304150in}{1.956883in}}%
\pgfpathlineto{\pgfqpoint{4.296560in}{1.946984in}}%
\pgfpathlineto{\pgfqpoint{4.288965in}{1.937102in}}%
\pgfpathclose%
\pgfusepath{fill}%
\end{pgfscope}%
\begin{pgfscope}%
\pgfpathrectangle{\pgfqpoint{1.254980in}{0.150000in}}{\pgfqpoint{5.490039in}{5.490039in}}%
\pgfusepath{clip}%
\pgfsetbuttcap%
\pgfsetroundjoin%
\definecolor{currentfill}{rgb}{0.276022,0.044167,0.370164}%
\pgfsetfillcolor{currentfill}%
\pgfsetfillopacity{0.700000}%
\pgfsetlinewidth{0.000000pt}%
\definecolor{currentstroke}{rgb}{0.000000,0.000000,0.000000}%
\pgfsetstrokecolor{currentstroke}%
\pgfsetdash{}{0pt}%
\pgfpathmoveto{\pgfqpoint{3.847432in}{1.892740in}}%
\pgfpathlineto{\pgfqpoint{3.860678in}{1.886694in}}%
\pgfpathlineto{\pgfqpoint{3.873929in}{1.880777in}}%
\pgfpathlineto{\pgfqpoint{3.887184in}{1.874988in}}%
\pgfpathlineto{\pgfqpoint{3.900444in}{1.869328in}}%
\pgfpathlineto{\pgfqpoint{3.908181in}{1.877566in}}%
\pgfpathlineto{\pgfqpoint{3.915912in}{1.885865in}}%
\pgfpathlineto{\pgfqpoint{3.923637in}{1.894223in}}%
\pgfpathlineto{\pgfqpoint{3.931357in}{1.902638in}}%
\pgfpathlineto{\pgfqpoint{3.918111in}{1.908069in}}%
\pgfpathlineto{\pgfqpoint{3.904870in}{1.913628in}}%
\pgfpathlineto{\pgfqpoint{3.891633in}{1.919316in}}%
\pgfpathlineto{\pgfqpoint{3.878401in}{1.925133in}}%
\pgfpathlineto{\pgfqpoint{3.870668in}{1.916941in}}%
\pgfpathlineto{\pgfqpoint{3.862929in}{1.908810in}}%
\pgfpathlineto{\pgfqpoint{3.855183in}{1.900742in}}%
\pgfpathlineto{\pgfqpoint{3.847432in}{1.892740in}}%
\pgfpathclose%
\pgfusepath{fill}%
\end{pgfscope}%
\begin{pgfscope}%
\pgfpathrectangle{\pgfqpoint{1.254980in}{0.150000in}}{\pgfqpoint{5.490039in}{5.490039in}}%
\pgfusepath{clip}%
\pgfsetbuttcap%
\pgfsetroundjoin%
\definecolor{currentfill}{rgb}{0.239346,0.300855,0.540844}%
\pgfsetfillcolor{currentfill}%
\pgfsetfillopacity{0.700000}%
\pgfsetlinewidth{0.000000pt}%
\definecolor{currentstroke}{rgb}{0.000000,0.000000,0.000000}%
\pgfsetstrokecolor{currentstroke}%
\pgfsetdash{}{0pt}%
\pgfpathmoveto{\pgfqpoint{5.126838in}{2.378063in}}%
\pgfpathlineto{\pgfqpoint{5.140511in}{2.381441in}}%
\pgfpathlineto{\pgfqpoint{5.154197in}{2.384932in}}%
\pgfpathlineto{\pgfqpoint{5.167895in}{2.388538in}}%
\pgfpathlineto{\pgfqpoint{5.181605in}{2.392257in}}%
\pgfpathlineto{\pgfqpoint{5.188924in}{2.402409in}}%
\pgfpathlineto{\pgfqpoint{5.196236in}{2.412516in}}%
\pgfpathlineto{\pgfqpoint{5.203543in}{2.422577in}}%
\pgfpathlineto{\pgfqpoint{5.210844in}{2.432593in}}%
\pgfpathlineto{\pgfqpoint{5.197141in}{2.428869in}}%
\pgfpathlineto{\pgfqpoint{5.183450in}{2.425258in}}%
\pgfpathlineto{\pgfqpoint{5.169771in}{2.421761in}}%
\pgfpathlineto{\pgfqpoint{5.156105in}{2.418378in}}%
\pgfpathlineto{\pgfqpoint{5.148797in}{2.408361in}}%
\pgfpathlineto{\pgfqpoint{5.141483in}{2.398303in}}%
\pgfpathlineto{\pgfqpoint{5.134163in}{2.388204in}}%
\pgfpathlineto{\pgfqpoint{5.126838in}{2.378063in}}%
\pgfpathclose%
\pgfusepath{fill}%
\end{pgfscope}%
\begin{pgfscope}%
\pgfpathrectangle{\pgfqpoint{1.254980in}{0.150000in}}{\pgfqpoint{5.490039in}{5.490039in}}%
\pgfusepath{clip}%
\pgfsetbuttcap%
\pgfsetroundjoin%
\definecolor{currentfill}{rgb}{0.171176,0.452530,0.557965}%
\pgfsetfillcolor{currentfill}%
\pgfsetfillopacity{0.700000}%
\pgfsetlinewidth{0.000000pt}%
\definecolor{currentstroke}{rgb}{0.000000,0.000000,0.000000}%
\pgfsetstrokecolor{currentstroke}%
\pgfsetdash{}{0pt}%
\pgfpathmoveto{\pgfqpoint{5.660218in}{2.752620in}}%
\pgfpathlineto{\pgfqpoint{5.674158in}{2.758392in}}%
\pgfpathlineto{\pgfqpoint{5.688113in}{2.764277in}}%
\pgfpathlineto{\pgfqpoint{5.702082in}{2.770273in}}%
\pgfpathlineto{\pgfqpoint{5.716067in}{2.776381in}}%
\pgfpathlineto{\pgfqpoint{5.723162in}{2.784833in}}%
\pgfpathlineto{\pgfqpoint{5.730252in}{2.793235in}}%
\pgfpathlineto{\pgfqpoint{5.737334in}{2.801589in}}%
\pgfpathlineto{\pgfqpoint{5.744411in}{2.809894in}}%
\pgfpathlineto{\pgfqpoint{5.730438in}{2.803898in}}%
\pgfpathlineto{\pgfqpoint{5.716480in}{2.798014in}}%
\pgfpathlineto{\pgfqpoint{5.702537in}{2.792241in}}%
\pgfpathlineto{\pgfqpoint{5.688609in}{2.786579in}}%
\pgfpathlineto{\pgfqpoint{5.681520in}{2.778156in}}%
\pgfpathlineto{\pgfqpoint{5.674426in}{2.769689in}}%
\pgfpathlineto{\pgfqpoint{5.667325in}{2.761177in}}%
\pgfpathlineto{\pgfqpoint{5.660218in}{2.752620in}}%
\pgfpathclose%
\pgfusepath{fill}%
\end{pgfscope}%
\begin{pgfscope}%
\pgfpathrectangle{\pgfqpoint{1.254980in}{0.150000in}}{\pgfqpoint{5.490039in}{5.490039in}}%
\pgfusepath{clip}%
\pgfsetbuttcap%
\pgfsetroundjoin%
\definecolor{currentfill}{rgb}{0.277941,0.056324,0.381191}%
\pgfsetfillcolor{currentfill}%
\pgfsetfillopacity{0.700000}%
\pgfsetlinewidth{0.000000pt}%
\definecolor{currentstroke}{rgb}{0.000000,0.000000,0.000000}%
\pgfsetstrokecolor{currentstroke}%
\pgfsetdash{}{0pt}%
\pgfpathmoveto{\pgfqpoint{3.710411in}{1.916455in}}%
\pgfpathlineto{\pgfqpoint{3.723643in}{1.909114in}}%
\pgfpathlineto{\pgfqpoint{3.736878in}{1.901905in}}%
\pgfpathlineto{\pgfqpoint{3.750117in}{1.894829in}}%
\pgfpathlineto{\pgfqpoint{3.763359in}{1.887885in}}%
\pgfpathlineto{\pgfqpoint{3.771151in}{1.895357in}}%
\pgfpathlineto{\pgfqpoint{3.778936in}{1.902906in}}%
\pgfpathlineto{\pgfqpoint{3.786716in}{1.910529in}}%
\pgfpathlineto{\pgfqpoint{3.794488in}{1.918224in}}%
\pgfpathlineto{\pgfqpoint{3.781262in}{1.924923in}}%
\pgfpathlineto{\pgfqpoint{3.768040in}{1.931753in}}%
\pgfpathlineto{\pgfqpoint{3.754821in}{1.938715in}}%
\pgfpathlineto{\pgfqpoint{3.741606in}{1.945810in}}%
\pgfpathlineto{\pgfqpoint{3.733817in}{1.938354in}}%
\pgfpathlineto{\pgfqpoint{3.726022in}{1.930975in}}%
\pgfpathlineto{\pgfqpoint{3.718220in}{1.923675in}}%
\pgfpathlineto{\pgfqpoint{3.710411in}{1.916455in}}%
\pgfpathclose%
\pgfusepath{fill}%
\end{pgfscope}%
\begin{pgfscope}%
\pgfpathrectangle{\pgfqpoint{1.254980in}{0.150000in}}{\pgfqpoint{5.490039in}{5.490039in}}%
\pgfusepath{clip}%
\pgfsetbuttcap%
\pgfsetroundjoin%
\definecolor{currentfill}{rgb}{0.276022,0.044167,0.370164}%
\pgfsetfillcolor{currentfill}%
\pgfsetfillopacity{0.700000}%
\pgfsetlinewidth{0.000000pt}%
\definecolor{currentstroke}{rgb}{0.000000,0.000000,0.000000}%
\pgfsetstrokecolor{currentstroke}%
\pgfsetdash{}{0pt}%
\pgfpathmoveto{\pgfqpoint{3.984390in}{1.882189in}}%
\pgfpathlineto{\pgfqpoint{3.997661in}{1.877394in}}%
\pgfpathlineto{\pgfqpoint{4.010938in}{1.872724in}}%
\pgfpathlineto{\pgfqpoint{4.024220in}{1.868180in}}%
\pgfpathlineto{\pgfqpoint{4.037508in}{1.863760in}}%
\pgfpathlineto{\pgfqpoint{4.045196in}{1.872670in}}%
\pgfpathlineto{\pgfqpoint{4.052879in}{1.881626in}}%
\pgfpathlineto{\pgfqpoint{4.060556in}{1.890627in}}%
\pgfpathlineto{\pgfqpoint{4.068228in}{1.899671in}}%
\pgfpathlineto{\pgfqpoint{4.054952in}{1.903877in}}%
\pgfpathlineto{\pgfqpoint{4.041682in}{1.908208in}}%
\pgfpathlineto{\pgfqpoint{4.028418in}{1.912665in}}%
\pgfpathlineto{\pgfqpoint{4.015159in}{1.917248in}}%
\pgfpathlineto{\pgfqpoint{4.007475in}{1.908411in}}%
\pgfpathlineto{\pgfqpoint{3.999785in}{1.899621in}}%
\pgfpathlineto{\pgfqpoint{3.992090in}{1.890880in}}%
\pgfpathlineto{\pgfqpoint{3.984390in}{1.882189in}}%
\pgfpathclose%
\pgfusepath{fill}%
\end{pgfscope}%
\begin{pgfscope}%
\pgfpathrectangle{\pgfqpoint{1.254980in}{0.150000in}}{\pgfqpoint{5.490039in}{5.490039in}}%
\pgfusepath{clip}%
\pgfsetbuttcap%
\pgfsetroundjoin%
\definecolor{currentfill}{rgb}{0.282290,0.145912,0.461510}%
\pgfsetfillcolor{currentfill}%
\pgfsetfillopacity{0.700000}%
\pgfsetlinewidth{0.000000pt}%
\definecolor{currentstroke}{rgb}{0.000000,0.000000,0.000000}%
\pgfsetstrokecolor{currentstroke}%
\pgfsetdash{}{0pt}%
\pgfpathmoveto{\pgfqpoint{3.329836in}{2.092325in}}%
\pgfpathlineto{\pgfqpoint{3.343069in}{2.081084in}}%
\pgfpathlineto{\pgfqpoint{3.356303in}{2.069990in}}%
\pgfpathlineto{\pgfqpoint{3.369537in}{2.059042in}}%
\pgfpathlineto{\pgfqpoint{3.382771in}{2.048240in}}%
\pgfpathlineto{\pgfqpoint{3.390739in}{2.053309in}}%
\pgfpathlineto{\pgfqpoint{3.398700in}{2.058495in}}%
\pgfpathlineto{\pgfqpoint{3.406652in}{2.063797in}}%
\pgfpathlineto{\pgfqpoint{3.414596in}{2.069213in}}%
\pgfpathlineto{\pgfqpoint{3.401385in}{2.079732in}}%
\pgfpathlineto{\pgfqpoint{3.388174in}{2.090397in}}%
\pgfpathlineto{\pgfqpoint{3.374964in}{2.101208in}}%
\pgfpathlineto{\pgfqpoint{3.361754in}{2.112165in}}%
\pgfpathlineto{\pgfqpoint{3.353787in}{2.107026in}}%
\pgfpathlineto{\pgfqpoint{3.345812in}{2.102005in}}%
\pgfpathlineto{\pgfqpoint{3.337828in}{2.097104in}}%
\pgfpathlineto{\pgfqpoint{3.329836in}{2.092325in}}%
\pgfpathclose%
\pgfusepath{fill}%
\end{pgfscope}%
\begin{pgfscope}%
\pgfpathrectangle{\pgfqpoint{1.254980in}{0.150000in}}{\pgfqpoint{5.490039in}{5.490039in}}%
\pgfusepath{clip}%
\pgfsetbuttcap%
\pgfsetroundjoin%
\definecolor{currentfill}{rgb}{0.227802,0.326594,0.546532}%
\pgfsetfillcolor{currentfill}%
\pgfsetfillopacity{0.700000}%
\pgfsetlinewidth{0.000000pt}%
\definecolor{currentstroke}{rgb}{0.000000,0.000000,0.000000}%
\pgfsetstrokecolor{currentstroke}%
\pgfsetdash{}{0pt}%
\pgfpathmoveto{\pgfqpoint{5.210844in}{2.432593in}}%
\pgfpathlineto{\pgfqpoint{5.224561in}{2.436430in}}%
\pgfpathlineto{\pgfqpoint{5.238290in}{2.440382in}}%
\pgfpathlineto{\pgfqpoint{5.252032in}{2.444446in}}%
\pgfpathlineto{\pgfqpoint{5.265788in}{2.448624in}}%
\pgfpathlineto{\pgfqpoint{5.273076in}{2.458589in}}%
\pgfpathlineto{\pgfqpoint{5.280359in}{2.468506in}}%
\pgfpathlineto{\pgfqpoint{5.287636in}{2.478374in}}%
\pgfpathlineto{\pgfqpoint{5.294907in}{2.488195in}}%
\pgfpathlineto{\pgfqpoint{5.281159in}{2.484029in}}%
\pgfpathlineto{\pgfqpoint{5.267424in}{2.479976in}}%
\pgfpathlineto{\pgfqpoint{5.253702in}{2.476036in}}%
\pgfpathlineto{\pgfqpoint{5.239993in}{2.472209in}}%
\pgfpathlineto{\pgfqpoint{5.232714in}{2.462371in}}%
\pgfpathlineto{\pgfqpoint{5.225430in}{2.452490in}}%
\pgfpathlineto{\pgfqpoint{5.218140in}{2.442564in}}%
\pgfpathlineto{\pgfqpoint{5.210844in}{2.432593in}}%
\pgfpathclose%
\pgfusepath{fill}%
\end{pgfscope}%
\begin{pgfscope}%
\pgfpathrectangle{\pgfqpoint{1.254980in}{0.150000in}}{\pgfqpoint{5.490039in}{5.490039in}}%
\pgfusepath{clip}%
\pgfsetbuttcap%
\pgfsetroundjoin%
\definecolor{currentfill}{rgb}{0.278791,0.062145,0.386592}%
\pgfsetfillcolor{currentfill}%
\pgfsetfillopacity{0.700000}%
\pgfsetlinewidth{0.000000pt}%
\definecolor{currentstroke}{rgb}{0.000000,0.000000,0.000000}%
\pgfsetstrokecolor{currentstroke}%
\pgfsetdash{}{0pt}%
\pgfpathmoveto{\pgfqpoint{4.205204in}{1.908639in}}%
\pgfpathlineto{\pgfqpoint{4.218526in}{1.905741in}}%
\pgfpathlineto{\pgfqpoint{4.231855in}{1.902964in}}%
\pgfpathlineto{\pgfqpoint{4.245192in}{1.900310in}}%
\pgfpathlineto{\pgfqpoint{4.258536in}{1.897776in}}%
\pgfpathlineto{\pgfqpoint{4.266150in}{1.907575in}}%
\pgfpathlineto{\pgfqpoint{4.273760in}{1.917396in}}%
\pgfpathlineto{\pgfqpoint{4.281365in}{1.927239in}}%
\pgfpathlineto{\pgfqpoint{4.288965in}{1.937102in}}%
\pgfpathlineto{\pgfqpoint{4.275631in}{1.939454in}}%
\pgfpathlineto{\pgfqpoint{4.262304in}{1.941928in}}%
\pgfpathlineto{\pgfqpoint{4.248984in}{1.944523in}}%
\pgfpathlineto{\pgfqpoint{4.235671in}{1.947241in}}%
\pgfpathlineto{\pgfqpoint{4.228062in}{1.937553in}}%
\pgfpathlineto{\pgfqpoint{4.220447in}{1.927889in}}%
\pgfpathlineto{\pgfqpoint{4.212828in}{1.918251in}}%
\pgfpathlineto{\pgfqpoint{4.205204in}{1.908639in}}%
\pgfpathclose%
\pgfusepath{fill}%
\end{pgfscope}%
\begin{pgfscope}%
\pgfpathrectangle{\pgfqpoint{1.254980in}{0.150000in}}{\pgfqpoint{5.490039in}{5.490039in}}%
\pgfusepath{clip}%
\pgfsetbuttcap%
\pgfsetroundjoin%
\definecolor{currentfill}{rgb}{0.146180,0.515413,0.556823}%
\pgfsetfillcolor{currentfill}%
\pgfsetfillopacity{0.700000}%
\pgfsetlinewidth{0.000000pt}%
\definecolor{currentstroke}{rgb}{0.000000,0.000000,0.000000}%
\pgfsetstrokecolor{currentstroke}%
\pgfsetdash{}{0pt}%
\pgfpathmoveto{\pgfqpoint{2.584551in}{2.982139in}}%
\pgfpathlineto{\pgfqpoint{2.598011in}{2.961249in}}%
\pgfpathlineto{\pgfqpoint{2.611462in}{2.940562in}}%
\pgfpathlineto{\pgfqpoint{2.624905in}{2.920077in}}%
\pgfpathlineto{\pgfqpoint{2.638340in}{2.899792in}}%
\pgfpathlineto{\pgfqpoint{2.646731in}{2.900674in}}%
\pgfpathlineto{\pgfqpoint{2.655108in}{2.901734in}}%
\pgfpathlineto{\pgfqpoint{2.663473in}{2.902972in}}%
\pgfpathlineto{\pgfqpoint{2.671824in}{2.904384in}}%
\pgfpathlineto{\pgfqpoint{2.658427in}{2.924362in}}%
\pgfpathlineto{\pgfqpoint{2.645022in}{2.944540in}}%
\pgfpathlineto{\pgfqpoint{2.631608in}{2.964919in}}%
\pgfpathlineto{\pgfqpoint{2.618186in}{2.985501in}}%
\pgfpathlineto{\pgfqpoint{2.609798in}{2.984389in}}%
\pgfpathlineto{\pgfqpoint{2.601396in}{2.983457in}}%
\pgfpathlineto{\pgfqpoint{2.592981in}{2.982706in}}%
\pgfpathlineto{\pgfqpoint{2.584551in}{2.982139in}}%
\pgfpathclose%
\pgfusepath{fill}%
\end{pgfscope}%
\begin{pgfscope}%
\pgfpathrectangle{\pgfqpoint{1.254980in}{0.150000in}}{\pgfqpoint{5.490039in}{5.490039in}}%
\pgfusepath{clip}%
\pgfsetbuttcap%
\pgfsetroundjoin%
\definecolor{currentfill}{rgb}{0.160665,0.478540,0.558115}%
\pgfsetfillcolor{currentfill}%
\pgfsetfillopacity{0.700000}%
\pgfsetlinewidth{0.000000pt}%
\definecolor{currentstroke}{rgb}{0.000000,0.000000,0.000000}%
\pgfsetstrokecolor{currentstroke}%
\pgfsetdash{}{0pt}%
\pgfpathmoveto{\pgfqpoint{5.744411in}{2.809894in}}%
\pgfpathlineto{\pgfqpoint{5.758399in}{2.816002in}}%
\pgfpathlineto{\pgfqpoint{5.772402in}{2.822221in}}%
\pgfpathlineto{\pgfqpoint{5.786420in}{2.828552in}}%
\pgfpathlineto{\pgfqpoint{5.800455in}{2.834995in}}%
\pgfpathlineto{\pgfqpoint{5.807512in}{2.843132in}}%
\pgfpathlineto{\pgfqpoint{5.814563in}{2.851220in}}%
\pgfpathlineto{\pgfqpoint{5.821607in}{2.859261in}}%
\pgfpathlineto{\pgfqpoint{5.828645in}{2.867255in}}%
\pgfpathlineto{\pgfqpoint{5.814624in}{2.860942in}}%
\pgfpathlineto{\pgfqpoint{5.800618in}{2.854739in}}%
\pgfpathlineto{\pgfqpoint{5.786628in}{2.848648in}}%
\pgfpathlineto{\pgfqpoint{5.772652in}{2.842669in}}%
\pgfpathlineto{\pgfqpoint{5.765601in}{2.834540in}}%
\pgfpathlineto{\pgfqpoint{5.758544in}{2.826369in}}%
\pgfpathlineto{\pgfqpoint{5.751480in}{2.818154in}}%
\pgfpathlineto{\pgfqpoint{5.744411in}{2.809894in}}%
\pgfpathclose%
\pgfusepath{fill}%
\end{pgfscope}%
\begin{pgfscope}%
\pgfpathrectangle{\pgfqpoint{1.254980in}{0.150000in}}{\pgfqpoint{5.490039in}{5.490039in}}%
\pgfusepath{clip}%
\pgfsetbuttcap%
\pgfsetroundjoin%
\definecolor{currentfill}{rgb}{0.280894,0.078907,0.402329}%
\pgfsetfillcolor{currentfill}%
\pgfsetfillopacity{0.700000}%
\pgfsetlinewidth{0.000000pt}%
\definecolor{currentstroke}{rgb}{0.000000,0.000000,0.000000}%
\pgfsetstrokecolor{currentstroke}%
\pgfsetdash{}{0pt}%
\pgfpathmoveto{\pgfqpoint{3.573213in}{1.954095in}}%
\pgfpathlineto{\pgfqpoint{3.586441in}{1.945410in}}%
\pgfpathlineto{\pgfqpoint{3.599672in}{1.936862in}}%
\pgfpathlineto{\pgfqpoint{3.612905in}{1.928450in}}%
\pgfpathlineto{\pgfqpoint{3.626141in}{1.920174in}}%
\pgfpathlineto{\pgfqpoint{3.633995in}{1.926783in}}%
\pgfpathlineto{\pgfqpoint{3.641842in}{1.933485in}}%
\pgfpathlineto{\pgfqpoint{3.649681in}{1.940277in}}%
\pgfpathlineto{\pgfqpoint{3.657514in}{1.947157in}}%
\pgfpathlineto{\pgfqpoint{3.644297in}{1.955170in}}%
\pgfpathlineto{\pgfqpoint{3.631083in}{1.963318in}}%
\pgfpathlineto{\pgfqpoint{3.617871in}{1.971603in}}%
\pgfpathlineto{\pgfqpoint{3.604661in}{1.980025in}}%
\pgfpathlineto{\pgfqpoint{3.596810in}{1.973401in}}%
\pgfpathlineto{\pgfqpoint{3.588951in}{1.966870in}}%
\pgfpathlineto{\pgfqpoint{3.581086in}{1.960434in}}%
\pgfpathlineto{\pgfqpoint{3.573213in}{1.954095in}}%
\pgfpathclose%
\pgfusepath{fill}%
\end{pgfscope}%
\begin{pgfscope}%
\pgfpathrectangle{\pgfqpoint{1.254980in}{0.150000in}}{\pgfqpoint{5.490039in}{5.490039in}}%
\pgfusepath{clip}%
\pgfsetbuttcap%
\pgfsetroundjoin%
\definecolor{currentfill}{rgb}{0.216210,0.351535,0.550627}%
\pgfsetfillcolor{currentfill}%
\pgfsetfillopacity{0.700000}%
\pgfsetlinewidth{0.000000pt}%
\definecolor{currentstroke}{rgb}{0.000000,0.000000,0.000000}%
\pgfsetstrokecolor{currentstroke}%
\pgfsetdash{}{0pt}%
\pgfpathmoveto{\pgfqpoint{5.294907in}{2.488195in}}%
\pgfpathlineto{\pgfqpoint{5.308668in}{2.492474in}}%
\pgfpathlineto{\pgfqpoint{5.322443in}{2.496867in}}%
\pgfpathlineto{\pgfqpoint{5.336231in}{2.501372in}}%
\pgfpathlineto{\pgfqpoint{5.350032in}{2.505991in}}%
\pgfpathlineto{\pgfqpoint{5.357290in}{2.515742in}}%
\pgfpathlineto{\pgfqpoint{5.364542in}{2.525442in}}%
\pgfpathlineto{\pgfqpoint{5.371788in}{2.535092in}}%
\pgfpathlineto{\pgfqpoint{5.379028in}{2.544691in}}%
\pgfpathlineto{\pgfqpoint{5.365234in}{2.540101in}}%
\pgfpathlineto{\pgfqpoint{5.351454in}{2.535623in}}%
\pgfpathlineto{\pgfqpoint{5.337687in}{2.531259in}}%
\pgfpathlineto{\pgfqpoint{5.323934in}{2.527007in}}%
\pgfpathlineto{\pgfqpoint{5.316686in}{2.517373in}}%
\pgfpathlineto{\pgfqpoint{5.309432in}{2.507694in}}%
\pgfpathlineto{\pgfqpoint{5.302173in}{2.497968in}}%
\pgfpathlineto{\pgfqpoint{5.294907in}{2.488195in}}%
\pgfpathclose%
\pgfusepath{fill}%
\end{pgfscope}%
\begin{pgfscope}%
\pgfpathrectangle{\pgfqpoint{1.254980in}{0.150000in}}{\pgfqpoint{5.490039in}{5.490039in}}%
\pgfusepath{clip}%
\pgfsetbuttcap%
\pgfsetroundjoin%
\definecolor{currentfill}{rgb}{0.151918,0.500685,0.557587}%
\pgfsetfillcolor{currentfill}%
\pgfsetfillopacity{0.700000}%
\pgfsetlinewidth{0.000000pt}%
\definecolor{currentstroke}{rgb}{0.000000,0.000000,0.000000}%
\pgfsetstrokecolor{currentstroke}%
\pgfsetdash{}{0pt}%
\pgfpathmoveto{\pgfqpoint{5.828645in}{2.867255in}}%
\pgfpathlineto{\pgfqpoint{5.842682in}{2.873680in}}%
\pgfpathlineto{\pgfqpoint{5.856735in}{2.880216in}}%
\pgfpathlineto{\pgfqpoint{5.870803in}{2.886864in}}%
\pgfpathlineto{\pgfqpoint{5.884886in}{2.893623in}}%
\pgfpathlineto{\pgfqpoint{5.891905in}{2.901433in}}%
\pgfpathlineto{\pgfqpoint{5.898916in}{2.909196in}}%
\pgfpathlineto{\pgfqpoint{5.905921in}{2.916913in}}%
\pgfpathlineto{\pgfqpoint{5.912919in}{2.924587in}}%
\pgfpathlineto{\pgfqpoint{5.898849in}{2.917974in}}%
\pgfpathlineto{\pgfqpoint{5.884795in}{2.911472in}}%
\pgfpathlineto{\pgfqpoint{5.870756in}{2.905081in}}%
\pgfpathlineto{\pgfqpoint{5.856733in}{2.898801in}}%
\pgfpathlineto{\pgfqpoint{5.849721in}{2.890977in}}%
\pgfpathlineto{\pgfqpoint{5.842702in}{2.883112in}}%
\pgfpathlineto{\pgfqpoint{5.835677in}{2.875205in}}%
\pgfpathlineto{\pgfqpoint{5.828645in}{2.867255in}}%
\pgfpathclose%
\pgfusepath{fill}%
\end{pgfscope}%
\begin{pgfscope}%
\pgfpathrectangle{\pgfqpoint{1.254980in}{0.150000in}}{\pgfqpoint{5.490039in}{5.490039in}}%
\pgfusepath{clip}%
\pgfsetbuttcap%
\pgfsetroundjoin%
\definecolor{currentfill}{rgb}{0.144759,0.519093,0.556572}%
\pgfsetfillcolor{currentfill}%
\pgfsetfillopacity{0.700000}%
\pgfsetlinewidth{0.000000pt}%
\definecolor{currentstroke}{rgb}{0.000000,0.000000,0.000000}%
\pgfsetstrokecolor{currentstroke}%
\pgfsetdash{}{0pt}%
\pgfpathmoveto{\pgfqpoint{5.912919in}{2.924587in}}%
\pgfpathlineto{\pgfqpoint{5.927005in}{2.931311in}}%
\pgfpathlineto{\pgfqpoint{5.941107in}{2.938146in}}%
\pgfpathlineto{\pgfqpoint{5.955225in}{2.945092in}}%
\pgfpathlineto{\pgfqpoint{5.962205in}{2.952605in}}%
\pgfpathlineto{\pgfqpoint{5.969180in}{2.960074in}}%
\pgfpathlineto{\pgfqpoint{5.976147in}{2.967501in}}%
\pgfpathlineto{\pgfqpoint{5.983109in}{2.974887in}}%
\pgfpathlineto{\pgfqpoint{5.969006in}{2.968104in}}%
\pgfpathlineto{\pgfqpoint{5.954919in}{2.961432in}}%
\pgfpathlineto{\pgfqpoint{5.940848in}{2.954870in}}%
\pgfpathlineto{\pgfqpoint{5.933876in}{2.947357in}}%
\pgfpathlineto{\pgfqpoint{5.926897in}{2.939807in}}%
\pgfpathlineto{\pgfqpoint{5.919911in}{2.932217in}}%
\pgfpathlineto{\pgfqpoint{5.912919in}{2.924587in}}%
\pgfpathclose%
\pgfusepath{fill}%
\end{pgfscope}%
\begin{pgfscope}%
\pgfpathrectangle{\pgfqpoint{1.254980in}{0.150000in}}{\pgfqpoint{5.490039in}{5.490039in}}%
\pgfusepath{clip}%
\pgfsetbuttcap%
\pgfsetroundjoin%
\definecolor{currentfill}{rgb}{0.283072,0.130895,0.449241}%
\pgfsetfillcolor{currentfill}%
\pgfsetfillopacity{0.700000}%
\pgfsetlinewidth{0.000000pt}%
\definecolor{currentstroke}{rgb}{0.000000,0.000000,0.000000}%
\pgfsetstrokecolor{currentstroke}%
\pgfsetdash{}{0pt}%
\pgfpathmoveto{\pgfqpoint{3.382771in}{2.048240in}}%
\pgfpathlineto{\pgfqpoint{3.396005in}{2.037583in}}%
\pgfpathlineto{\pgfqpoint{3.409240in}{2.027071in}}%
\pgfpathlineto{\pgfqpoint{3.422476in}{2.016702in}}%
\pgfpathlineto{\pgfqpoint{3.435712in}{2.006476in}}%
\pgfpathlineto{\pgfqpoint{3.443658in}{2.011833in}}%
\pgfpathlineto{\pgfqpoint{3.451596in}{2.017303in}}%
\pgfpathlineto{\pgfqpoint{3.459526in}{2.022885in}}%
\pgfpathlineto{\pgfqpoint{3.467447in}{2.028577in}}%
\pgfpathlineto{\pgfqpoint{3.454233in}{2.038521in}}%
\pgfpathlineto{\pgfqpoint{3.441020in}{2.048608in}}%
\pgfpathlineto{\pgfqpoint{3.427807in}{2.058838in}}%
\pgfpathlineto{\pgfqpoint{3.414596in}{2.069213in}}%
\pgfpathlineto{\pgfqpoint{3.406652in}{2.063797in}}%
\pgfpathlineto{\pgfqpoint{3.398700in}{2.058495in}}%
\pgfpathlineto{\pgfqpoint{3.390739in}{2.053309in}}%
\pgfpathlineto{\pgfqpoint{3.382771in}{2.048240in}}%
\pgfpathclose%
\pgfusepath{fill}%
\end{pgfscope}%
\begin{pgfscope}%
\pgfpathrectangle{\pgfqpoint{1.254980in}{0.150000in}}{\pgfqpoint{5.490039in}{5.490039in}}%
\pgfusepath{clip}%
\pgfsetbuttcap%
\pgfsetroundjoin%
\definecolor{currentfill}{rgb}{0.277018,0.050344,0.375715}%
\pgfsetfillcolor{currentfill}%
\pgfsetfillopacity{0.700000}%
\pgfsetlinewidth{0.000000pt}%
\definecolor{currentstroke}{rgb}{0.000000,0.000000,0.000000}%
\pgfsetstrokecolor{currentstroke}%
\pgfsetdash{}{0pt}%
\pgfpathmoveto{\pgfqpoint{4.121391in}{1.884091in}}%
\pgfpathlineto{\pgfqpoint{4.134697in}{1.880505in}}%
\pgfpathlineto{\pgfqpoint{4.148010in}{1.877043in}}%
\pgfpathlineto{\pgfqpoint{4.161329in}{1.873703in}}%
\pgfpathlineto{\pgfqpoint{4.174655in}{1.870486in}}%
\pgfpathlineto{\pgfqpoint{4.182300in}{1.879977in}}%
\pgfpathlineto{\pgfqpoint{4.189940in}{1.889501in}}%
\pgfpathlineto{\pgfqpoint{4.197574in}{1.899055in}}%
\pgfpathlineto{\pgfqpoint{4.205204in}{1.908639in}}%
\pgfpathlineto{\pgfqpoint{4.191888in}{1.911659in}}%
\pgfpathlineto{\pgfqpoint{4.178579in}{1.914802in}}%
\pgfpathlineto{\pgfqpoint{4.165277in}{1.918068in}}%
\pgfpathlineto{\pgfqpoint{4.151981in}{1.921456in}}%
\pgfpathlineto{\pgfqpoint{4.144342in}{1.912063in}}%
\pgfpathlineto{\pgfqpoint{4.136697in}{1.902704in}}%
\pgfpathlineto{\pgfqpoint{4.129046in}{1.893379in}}%
\pgfpathlineto{\pgfqpoint{4.121391in}{1.884091in}}%
\pgfpathclose%
\pgfusepath{fill}%
\end{pgfscope}%
\begin{pgfscope}%
\pgfpathrectangle{\pgfqpoint{1.254980in}{0.150000in}}{\pgfqpoint{5.490039in}{5.490039in}}%
\pgfusepath{clip}%
\pgfsetbuttcap%
\pgfsetroundjoin%
\definecolor{currentfill}{rgb}{0.204903,0.375746,0.553533}%
\pgfsetfillcolor{currentfill}%
\pgfsetfillopacity{0.700000}%
\pgfsetlinewidth{0.000000pt}%
\definecolor{currentstroke}{rgb}{0.000000,0.000000,0.000000}%
\pgfsetstrokecolor{currentstroke}%
\pgfsetdash{}{0pt}%
\pgfpathmoveto{\pgfqpoint{5.379028in}{2.544691in}}%
\pgfpathlineto{\pgfqpoint{5.392835in}{2.549395in}}%
\pgfpathlineto{\pgfqpoint{5.406656in}{2.554211in}}%
\pgfpathlineto{\pgfqpoint{5.420490in}{2.559140in}}%
\pgfpathlineto{\pgfqpoint{5.434339in}{2.564181in}}%
\pgfpathlineto{\pgfqpoint{5.441565in}{2.573693in}}%
\pgfpathlineto{\pgfqpoint{5.448784in}{2.583152in}}%
\pgfpathlineto{\pgfqpoint{5.455998in}{2.592559in}}%
\pgfpathlineto{\pgfqpoint{5.463205in}{2.601914in}}%
\pgfpathlineto{\pgfqpoint{5.449365in}{2.596917in}}%
\pgfpathlineto{\pgfqpoint{5.435539in}{2.592033in}}%
\pgfpathlineto{\pgfqpoint{5.421727in}{2.587261in}}%
\pgfpathlineto{\pgfqpoint{5.407928in}{2.582603in}}%
\pgfpathlineto{\pgfqpoint{5.400712in}{2.573197in}}%
\pgfpathlineto{\pgfqpoint{5.393490in}{2.563743in}}%
\pgfpathlineto{\pgfqpoint{5.386262in}{2.554242in}}%
\pgfpathlineto{\pgfqpoint{5.379028in}{2.544691in}}%
\pgfpathclose%
\pgfusepath{fill}%
\end{pgfscope}%
\begin{pgfscope}%
\pgfpathrectangle{\pgfqpoint{1.254980in}{0.150000in}}{\pgfqpoint{5.490039in}{5.490039in}}%
\pgfusepath{clip}%
\pgfsetbuttcap%
\pgfsetroundjoin%
\definecolor{currentfill}{rgb}{0.133743,0.548535,0.553541}%
\pgfsetfillcolor{currentfill}%
\pgfsetfillopacity{0.700000}%
\pgfsetlinewidth{0.000000pt}%
\definecolor{currentstroke}{rgb}{0.000000,0.000000,0.000000}%
\pgfsetstrokecolor{currentstroke}%
\pgfsetdash{}{0pt}%
\pgfpathmoveto{\pgfqpoint{2.530623in}{3.067763in}}%
\pgfpathlineto{\pgfqpoint{2.544119in}{3.046045in}}%
\pgfpathlineto{\pgfqpoint{2.557606in}{3.024536in}}%
\pgfpathlineto{\pgfqpoint{2.571083in}{3.003234in}}%
\pgfpathlineto{\pgfqpoint{2.584551in}{2.982139in}}%
\pgfpathlineto{\pgfqpoint{2.592981in}{2.982706in}}%
\pgfpathlineto{\pgfqpoint{2.601396in}{2.983457in}}%
\pgfpathlineto{\pgfqpoint{2.609798in}{2.984389in}}%
\pgfpathlineto{\pgfqpoint{2.618186in}{2.985501in}}%
\pgfpathlineto{\pgfqpoint{2.604756in}{3.006287in}}%
\pgfpathlineto{\pgfqpoint{2.591317in}{3.027279in}}%
\pgfpathlineto{\pgfqpoint{2.577870in}{3.048477in}}%
\pgfpathlineto{\pgfqpoint{2.564413in}{3.069884in}}%
\pgfpathlineto{\pgfqpoint{2.555987in}{3.069075in}}%
\pgfpathlineto{\pgfqpoint{2.547546in}{3.068451in}}%
\pgfpathlineto{\pgfqpoint{2.539092in}{3.068012in}}%
\pgfpathlineto{\pgfqpoint{2.530623in}{3.067763in}}%
\pgfpathclose%
\pgfusepath{fill}%
\end{pgfscope}%
\begin{pgfscope}%
\pgfpathrectangle{\pgfqpoint{1.254980in}{0.150000in}}{\pgfqpoint{5.490039in}{5.490039in}}%
\pgfusepath{clip}%
\pgfsetbuttcap%
\pgfsetroundjoin%
\definecolor{currentfill}{rgb}{0.280255,0.165693,0.476498}%
\pgfsetfillcolor{currentfill}%
\pgfsetfillopacity{0.700000}%
\pgfsetlinewidth{0.000000pt}%
\definecolor{currentstroke}{rgb}{0.000000,0.000000,0.000000}%
\pgfsetstrokecolor{currentstroke}%
\pgfsetdash{}{0pt}%
\pgfpathmoveto{\pgfqpoint{4.677678in}{2.086091in}}%
\pgfpathlineto{\pgfqpoint{4.691165in}{2.086796in}}%
\pgfpathlineto{\pgfqpoint{4.704661in}{2.087618in}}%
\pgfpathlineto{\pgfqpoint{4.718168in}{2.088556in}}%
\pgfpathlineto{\pgfqpoint{4.731685in}{2.089610in}}%
\pgfpathlineto{\pgfqpoint{4.739159in}{2.100300in}}%
\pgfpathlineto{\pgfqpoint{4.746627in}{2.110969in}}%
\pgfpathlineto{\pgfqpoint{4.754091in}{2.121617in}}%
\pgfpathlineto{\pgfqpoint{4.761550in}{2.132243in}}%
\pgfpathlineto{\pgfqpoint{4.748039in}{2.131087in}}%
\pgfpathlineto{\pgfqpoint{4.734539in}{2.130046in}}%
\pgfpathlineto{\pgfqpoint{4.721048in}{2.129123in}}%
\pgfpathlineto{\pgfqpoint{4.707568in}{2.128315in}}%
\pgfpathlineto{\pgfqpoint{4.700103in}{2.117786in}}%
\pgfpathlineto{\pgfqpoint{4.692633in}{2.107238in}}%
\pgfpathlineto{\pgfqpoint{4.685158in}{2.096672in}}%
\pgfpathlineto{\pgfqpoint{4.677678in}{2.086091in}}%
\pgfpathclose%
\pgfusepath{fill}%
\end{pgfscope}%
\begin{pgfscope}%
\pgfpathrectangle{\pgfqpoint{1.254980in}{0.150000in}}{\pgfqpoint{5.490039in}{5.490039in}}%
\pgfusepath{clip}%
\pgfsetbuttcap%
\pgfsetroundjoin%
\definecolor{currentfill}{rgb}{0.282623,0.140926,0.457517}%
\pgfsetfillcolor{currentfill}%
\pgfsetfillopacity{0.700000}%
\pgfsetlinewidth{0.000000pt}%
\definecolor{currentstroke}{rgb}{0.000000,0.000000,0.000000}%
\pgfsetstrokecolor{currentstroke}%
\pgfsetdash{}{0pt}%
\pgfpathmoveto{\pgfqpoint{4.593836in}{2.042426in}}%
\pgfpathlineto{\pgfqpoint{4.607290in}{2.042546in}}%
\pgfpathlineto{\pgfqpoint{4.620754in}{2.042784in}}%
\pgfpathlineto{\pgfqpoint{4.634227in}{2.043138in}}%
\pgfpathlineto{\pgfqpoint{4.647711in}{2.043609in}}%
\pgfpathlineto{\pgfqpoint{4.655210in}{2.054251in}}%
\pgfpathlineto{\pgfqpoint{4.662704in}{2.064879in}}%
\pgfpathlineto{\pgfqpoint{4.670194in}{2.075492in}}%
\pgfpathlineto{\pgfqpoint{4.677678in}{2.086091in}}%
\pgfpathlineto{\pgfqpoint{4.664202in}{2.085502in}}%
\pgfpathlineto{\pgfqpoint{4.650735in}{2.085030in}}%
\pgfpathlineto{\pgfqpoint{4.637278in}{2.084675in}}%
\pgfpathlineto{\pgfqpoint{4.623830in}{2.084437in}}%
\pgfpathlineto{\pgfqpoint{4.616339in}{2.073951in}}%
\pgfpathlineto{\pgfqpoint{4.608843in}{2.063453in}}%
\pgfpathlineto{\pgfqpoint{4.601342in}{2.052945in}}%
\pgfpathlineto{\pgfqpoint{4.593836in}{2.042426in}}%
\pgfpathclose%
\pgfusepath{fill}%
\end{pgfscope}%
\begin{pgfscope}%
\pgfpathrectangle{\pgfqpoint{1.254980in}{0.150000in}}{\pgfqpoint{5.490039in}{5.490039in}}%
\pgfusepath{clip}%
\pgfsetbuttcap%
\pgfsetroundjoin%
\definecolor{currentfill}{rgb}{0.277018,0.050344,0.375715}%
\pgfsetfillcolor{currentfill}%
\pgfsetfillopacity{0.700000}%
\pgfsetlinewidth{0.000000pt}%
\definecolor{currentstroke}{rgb}{0.000000,0.000000,0.000000}%
\pgfsetstrokecolor{currentstroke}%
\pgfsetdash{}{0pt}%
\pgfpathmoveto{\pgfqpoint{3.763359in}{1.887885in}}%
\pgfpathlineto{\pgfqpoint{3.776605in}{1.881073in}}%
\pgfpathlineto{\pgfqpoint{3.789855in}{1.874391in}}%
\pgfpathlineto{\pgfqpoint{3.803108in}{1.867840in}}%
\pgfpathlineto{\pgfqpoint{3.816366in}{1.861419in}}%
\pgfpathlineto{\pgfqpoint{3.824141in}{1.869142in}}%
\pgfpathlineto{\pgfqpoint{3.831911in}{1.876938in}}%
\pgfpathlineto{\pgfqpoint{3.839675in}{1.884805in}}%
\pgfpathlineto{\pgfqpoint{3.847432in}{1.892740in}}%
\pgfpathlineto{\pgfqpoint{3.834190in}{1.898916in}}%
\pgfpathlineto{\pgfqpoint{3.820952in}{1.905221in}}%
\pgfpathlineto{\pgfqpoint{3.807718in}{1.911657in}}%
\pgfpathlineto{\pgfqpoint{3.794488in}{1.918224in}}%
\pgfpathlineto{\pgfqpoint{3.786716in}{1.910529in}}%
\pgfpathlineto{\pgfqpoint{3.778936in}{1.902906in}}%
\pgfpathlineto{\pgfqpoint{3.771151in}{1.895357in}}%
\pgfpathlineto{\pgfqpoint{3.763359in}{1.887885in}}%
\pgfpathclose%
\pgfusepath{fill}%
\end{pgfscope}%
\begin{pgfscope}%
\pgfpathrectangle{\pgfqpoint{1.254980in}{0.150000in}}{\pgfqpoint{5.490039in}{5.490039in}}%
\pgfusepath{clip}%
\pgfsetbuttcap%
\pgfsetroundjoin%
\definecolor{currentfill}{rgb}{0.276194,0.190074,0.493001}%
\pgfsetfillcolor{currentfill}%
\pgfsetfillopacity{0.700000}%
\pgfsetlinewidth{0.000000pt}%
\definecolor{currentstroke}{rgb}{0.000000,0.000000,0.000000}%
\pgfsetstrokecolor{currentstroke}%
\pgfsetdash{}{0pt}%
\pgfpathmoveto{\pgfqpoint{4.761550in}{2.132243in}}%
\pgfpathlineto{\pgfqpoint{4.775071in}{2.133515in}}%
\pgfpathlineto{\pgfqpoint{4.788603in}{2.134903in}}%
\pgfpathlineto{\pgfqpoint{4.802145in}{2.136406in}}%
\pgfpathlineto{\pgfqpoint{4.815698in}{2.138025in}}%
\pgfpathlineto{\pgfqpoint{4.823146in}{2.148721in}}%
\pgfpathlineto{\pgfqpoint{4.830589in}{2.159389in}}%
\pgfpathlineto{\pgfqpoint{4.838027in}{2.170030in}}%
\pgfpathlineto{\pgfqpoint{4.845460in}{2.180643in}}%
\pgfpathlineto{\pgfqpoint{4.831913in}{2.178938in}}%
\pgfpathlineto{\pgfqpoint{4.818377in}{2.177349in}}%
\pgfpathlineto{\pgfqpoint{4.804851in}{2.175875in}}%
\pgfpathlineto{\pgfqpoint{4.791336in}{2.174517in}}%
\pgfpathlineto{\pgfqpoint{4.783897in}{2.163983in}}%
\pgfpathlineto{\pgfqpoint{4.776453in}{2.153426in}}%
\pgfpathlineto{\pgfqpoint{4.769004in}{2.142846in}}%
\pgfpathlineto{\pgfqpoint{4.761550in}{2.132243in}}%
\pgfpathclose%
\pgfusepath{fill}%
\end{pgfscope}%
\begin{pgfscope}%
\pgfpathrectangle{\pgfqpoint{1.254980in}{0.150000in}}{\pgfqpoint{5.490039in}{5.490039in}}%
\pgfusepath{clip}%
\pgfsetbuttcap%
\pgfsetroundjoin%
\definecolor{currentfill}{rgb}{0.283229,0.120777,0.440584}%
\pgfsetfillcolor{currentfill}%
\pgfsetfillopacity{0.700000}%
\pgfsetlinewidth{0.000000pt}%
\definecolor{currentstroke}{rgb}{0.000000,0.000000,0.000000}%
\pgfsetstrokecolor{currentstroke}%
\pgfsetdash{}{0pt}%
\pgfpathmoveto{\pgfqpoint{4.510014in}{2.001499in}}%
\pgfpathlineto{\pgfqpoint{4.523438in}{2.001015in}}%
\pgfpathlineto{\pgfqpoint{4.536871in}{2.000649in}}%
\pgfpathlineto{\pgfqpoint{4.550314in}{2.000400in}}%
\pgfpathlineto{\pgfqpoint{4.563765in}{2.000270in}}%
\pgfpathlineto{\pgfqpoint{4.571290in}{2.010820in}}%
\pgfpathlineto{\pgfqpoint{4.578810in}{2.021363in}}%
\pgfpathlineto{\pgfqpoint{4.586326in}{2.031899in}}%
\pgfpathlineto{\pgfqpoint{4.593836in}{2.042426in}}%
\pgfpathlineto{\pgfqpoint{4.580391in}{2.042424in}}%
\pgfpathlineto{\pgfqpoint{4.566956in}{2.042538in}}%
\pgfpathlineto{\pgfqpoint{4.553530in}{2.042771in}}%
\pgfpathlineto{\pgfqpoint{4.540113in}{2.043121in}}%
\pgfpathlineto{\pgfqpoint{4.532595in}{2.032721in}}%
\pgfpathlineto{\pgfqpoint{4.525073in}{2.022317in}}%
\pgfpathlineto{\pgfqpoint{4.517546in}{2.011909in}}%
\pgfpathlineto{\pgfqpoint{4.510014in}{2.001499in}}%
\pgfpathclose%
\pgfusepath{fill}%
\end{pgfscope}%
\begin{pgfscope}%
\pgfpathrectangle{\pgfqpoint{1.254980in}{0.150000in}}{\pgfqpoint{5.490039in}{5.490039in}}%
\pgfusepath{clip}%
\pgfsetbuttcap%
\pgfsetroundjoin%
\definecolor{currentfill}{rgb}{0.274952,0.037752,0.364543}%
\pgfsetfillcolor{currentfill}%
\pgfsetfillopacity{0.700000}%
\pgfsetlinewidth{0.000000pt}%
\definecolor{currentstroke}{rgb}{0.000000,0.000000,0.000000}%
\pgfsetstrokecolor{currentstroke}%
\pgfsetdash{}{0pt}%
\pgfpathmoveto{\pgfqpoint{3.900444in}{1.869328in}}%
\pgfpathlineto{\pgfqpoint{3.913708in}{1.863796in}}%
\pgfpathlineto{\pgfqpoint{3.926978in}{1.858392in}}%
\pgfpathlineto{\pgfqpoint{3.940252in}{1.853114in}}%
\pgfpathlineto{\pgfqpoint{3.953531in}{1.847963in}}%
\pgfpathlineto{\pgfqpoint{3.961254in}{1.856436in}}%
\pgfpathlineto{\pgfqpoint{3.968972in}{1.864965in}}%
\pgfpathlineto{\pgfqpoint{3.976683in}{1.873550in}}%
\pgfpathlineto{\pgfqpoint{3.984390in}{1.882189in}}%
\pgfpathlineto{\pgfqpoint{3.971124in}{1.887111in}}%
\pgfpathlineto{\pgfqpoint{3.957863in}{1.892160in}}%
\pgfpathlineto{\pgfqpoint{3.944607in}{1.897335in}}%
\pgfpathlineto{\pgfqpoint{3.931357in}{1.902638in}}%
\pgfpathlineto{\pgfqpoint{3.923637in}{1.894223in}}%
\pgfpathlineto{\pgfqpoint{3.915912in}{1.885865in}}%
\pgfpathlineto{\pgfqpoint{3.908181in}{1.877566in}}%
\pgfpathlineto{\pgfqpoint{3.900444in}{1.869328in}}%
\pgfpathclose%
\pgfusepath{fill}%
\end{pgfscope}%
\begin{pgfscope}%
\pgfpathrectangle{\pgfqpoint{1.254980in}{0.150000in}}{\pgfqpoint{5.490039in}{5.490039in}}%
\pgfusepath{clip}%
\pgfsetbuttcap%
\pgfsetroundjoin%
\definecolor{currentfill}{rgb}{0.231674,0.318106,0.544834}%
\pgfsetfillcolor{currentfill}%
\pgfsetfillopacity{0.700000}%
\pgfsetlinewidth{0.000000pt}%
\definecolor{currentstroke}{rgb}{0.000000,0.000000,0.000000}%
\pgfsetstrokecolor{currentstroke}%
\pgfsetdash{}{0pt}%
\pgfpathmoveto{\pgfqpoint{2.925923in}{2.460519in}}%
\pgfpathlineto{\pgfqpoint{2.939242in}{2.444536in}}%
\pgfpathlineto{\pgfqpoint{2.952556in}{2.428723in}}%
\pgfpathlineto{\pgfqpoint{2.965867in}{2.413079in}}%
\pgfpathlineto{\pgfqpoint{2.979173in}{2.397604in}}%
\pgfpathlineto{\pgfqpoint{2.987372in}{2.400006in}}%
\pgfpathlineto{\pgfqpoint{2.995560in}{2.402565in}}%
\pgfpathlineto{\pgfqpoint{3.003736in}{2.405280in}}%
\pgfpathlineto{\pgfqpoint{3.011902in}{2.408146in}}%
\pgfpathlineto{\pgfqpoint{2.998626in}{2.423312in}}%
\pgfpathlineto{\pgfqpoint{2.985347in}{2.438646in}}%
\pgfpathlineto{\pgfqpoint{2.972064in}{2.454148in}}%
\pgfpathlineto{\pgfqpoint{2.958778in}{2.469821in}}%
\pgfpathlineto{\pgfqpoint{2.950581in}{2.467258in}}%
\pgfpathlineto{\pgfqpoint{2.942373in}{2.464851in}}%
\pgfpathlineto{\pgfqpoint{2.934154in}{2.462604in}}%
\pgfpathlineto{\pgfqpoint{2.925923in}{2.460519in}}%
\pgfpathclose%
\pgfusepath{fill}%
\end{pgfscope}%
\begin{pgfscope}%
\pgfpathrectangle{\pgfqpoint{1.254980in}{0.150000in}}{\pgfqpoint{5.490039in}{5.490039in}}%
\pgfusepath{clip}%
\pgfsetbuttcap%
\pgfsetroundjoin%
\definecolor{currentfill}{rgb}{0.243113,0.292092,0.538516}%
\pgfsetfillcolor{currentfill}%
\pgfsetfillopacity{0.700000}%
\pgfsetlinewidth{0.000000pt}%
\definecolor{currentstroke}{rgb}{0.000000,0.000000,0.000000}%
\pgfsetstrokecolor{currentstroke}%
\pgfsetdash{}{0pt}%
\pgfpathmoveto{\pgfqpoint{2.979173in}{2.397604in}}%
\pgfpathlineto{\pgfqpoint{2.992477in}{2.382297in}}%
\pgfpathlineto{\pgfqpoint{3.005776in}{2.367157in}}%
\pgfpathlineto{\pgfqpoint{3.019073in}{2.352182in}}%
\pgfpathlineto{\pgfqpoint{3.032366in}{2.337372in}}%
\pgfpathlineto{\pgfqpoint{3.040534in}{2.340089in}}%
\pgfpathlineto{\pgfqpoint{3.048691in}{2.342959in}}%
\pgfpathlineto{\pgfqpoint{3.056837in}{2.345979in}}%
\pgfpathlineto{\pgfqpoint{3.064973in}{2.349148in}}%
\pgfpathlineto{\pgfqpoint{3.051710in}{2.363650in}}%
\pgfpathlineto{\pgfqpoint{3.038444in}{2.378316in}}%
\pgfpathlineto{\pgfqpoint{3.025175in}{2.393148in}}%
\pgfpathlineto{\pgfqpoint{3.011902in}{2.408146in}}%
\pgfpathlineto{\pgfqpoint{3.003736in}{2.405280in}}%
\pgfpathlineto{\pgfqpoint{2.995560in}{2.402565in}}%
\pgfpathlineto{\pgfqpoint{2.987372in}{2.400006in}}%
\pgfpathlineto{\pgfqpoint{2.979173in}{2.397604in}}%
\pgfpathclose%
\pgfusepath{fill}%
\end{pgfscope}%
\begin{pgfscope}%
\pgfpathrectangle{\pgfqpoint{1.254980in}{0.150000in}}{\pgfqpoint{5.490039in}{5.490039in}}%
\pgfusepath{clip}%
\pgfsetbuttcap%
\pgfsetroundjoin%
\definecolor{currentfill}{rgb}{0.270595,0.214069,0.507052}%
\pgfsetfillcolor{currentfill}%
\pgfsetfillopacity{0.700000}%
\pgfsetlinewidth{0.000000pt}%
\definecolor{currentstroke}{rgb}{0.000000,0.000000,0.000000}%
\pgfsetstrokecolor{currentstroke}%
\pgfsetdash{}{0pt}%
\pgfpathmoveto{\pgfqpoint{4.845460in}{2.180643in}}%
\pgfpathlineto{\pgfqpoint{4.859018in}{2.182464in}}%
\pgfpathlineto{\pgfqpoint{4.872587in}{2.184399in}}%
\pgfpathlineto{\pgfqpoint{4.886167in}{2.186450in}}%
\pgfpathlineto{\pgfqpoint{4.899758in}{2.188616in}}%
\pgfpathlineto{\pgfqpoint{4.907180in}{2.199276in}}%
\pgfpathlineto{\pgfqpoint{4.914597in}{2.209905in}}%
\pgfpathlineto{\pgfqpoint{4.922009in}{2.220500in}}%
\pgfpathlineto{\pgfqpoint{4.929415in}{2.231062in}}%
\pgfpathlineto{\pgfqpoint{4.915830in}{2.228826in}}%
\pgfpathlineto{\pgfqpoint{4.902256in}{2.226705in}}%
\pgfpathlineto{\pgfqpoint{4.888693in}{2.224700in}}%
\pgfpathlineto{\pgfqpoint{4.875141in}{2.222809in}}%
\pgfpathlineto{\pgfqpoint{4.867728in}{2.212311in}}%
\pgfpathlineto{\pgfqpoint{4.860311in}{2.201784in}}%
\pgfpathlineto{\pgfqpoint{4.852888in}{2.191228in}}%
\pgfpathlineto{\pgfqpoint{4.845460in}{2.180643in}}%
\pgfpathclose%
\pgfusepath{fill}%
\end{pgfscope}%
\begin{pgfscope}%
\pgfpathrectangle{\pgfqpoint{1.254980in}{0.150000in}}{\pgfqpoint{5.490039in}{5.490039in}}%
\pgfusepath{clip}%
\pgfsetbuttcap%
\pgfsetroundjoin%
\definecolor{currentfill}{rgb}{0.218130,0.347432,0.550038}%
\pgfsetfillcolor{currentfill}%
\pgfsetfillopacity{0.700000}%
\pgfsetlinewidth{0.000000pt}%
\definecolor{currentstroke}{rgb}{0.000000,0.000000,0.000000}%
\pgfsetstrokecolor{currentstroke}%
\pgfsetdash{}{0pt}%
\pgfpathmoveto{\pgfqpoint{2.872605in}{2.526180in}}%
\pgfpathlineto{\pgfqpoint{2.885942in}{2.509504in}}%
\pgfpathlineto{\pgfqpoint{2.899273in}{2.493003in}}%
\pgfpathlineto{\pgfqpoint{2.912600in}{2.476675in}}%
\pgfpathlineto{\pgfqpoint{2.925923in}{2.460519in}}%
\pgfpathlineto{\pgfqpoint{2.934154in}{2.462604in}}%
\pgfpathlineto{\pgfqpoint{2.942373in}{2.464851in}}%
\pgfpathlineto{\pgfqpoint{2.950581in}{2.467258in}}%
\pgfpathlineto{\pgfqpoint{2.958778in}{2.469821in}}%
\pgfpathlineto{\pgfqpoint{2.945487in}{2.485665in}}%
\pgfpathlineto{\pgfqpoint{2.932192in}{2.501682in}}%
\pgfpathlineto{\pgfqpoint{2.918893in}{2.517871in}}%
\pgfpathlineto{\pgfqpoint{2.905590in}{2.534234in}}%
\pgfpathlineto{\pgfqpoint{2.897361in}{2.531976in}}%
\pgfpathlineto{\pgfqpoint{2.889121in}{2.529879in}}%
\pgfpathlineto{\pgfqpoint{2.880869in}{2.527947in}}%
\pgfpathlineto{\pgfqpoint{2.872605in}{2.526180in}}%
\pgfpathclose%
\pgfusepath{fill}%
\end{pgfscope}%
\begin{pgfscope}%
\pgfpathrectangle{\pgfqpoint{1.254980in}{0.150000in}}{\pgfqpoint{5.490039in}{5.490039in}}%
\pgfusepath{clip}%
\pgfsetbuttcap%
\pgfsetroundjoin%
\definecolor{currentfill}{rgb}{0.282656,0.100196,0.422160}%
\pgfsetfillcolor{currentfill}%
\pgfsetfillopacity{0.700000}%
\pgfsetlinewidth{0.000000pt}%
\definecolor{currentstroke}{rgb}{0.000000,0.000000,0.000000}%
\pgfsetstrokecolor{currentstroke}%
\pgfsetdash{}{0pt}%
\pgfpathmoveto{\pgfqpoint{4.426199in}{1.963567in}}%
\pgfpathlineto{\pgfqpoint{4.439596in}{1.962460in}}%
\pgfpathlineto{\pgfqpoint{4.453001in}{1.961472in}}%
\pgfpathlineto{\pgfqpoint{4.466415in}{1.960602in}}%
\pgfpathlineto{\pgfqpoint{4.479838in}{1.959850in}}%
\pgfpathlineto{\pgfqpoint{4.487389in}{1.970261in}}%
\pgfpathlineto{\pgfqpoint{4.494935in}{1.980674in}}%
\pgfpathlineto{\pgfqpoint{4.502477in}{1.991087in}}%
\pgfpathlineto{\pgfqpoint{4.510014in}{2.001499in}}%
\pgfpathlineto{\pgfqpoint{4.496598in}{2.002101in}}%
\pgfpathlineto{\pgfqpoint{4.483192in}{2.002822in}}%
\pgfpathlineto{\pgfqpoint{4.469794in}{2.003661in}}%
\pgfpathlineto{\pgfqpoint{4.456404in}{2.004618in}}%
\pgfpathlineto{\pgfqpoint{4.448860in}{1.994350in}}%
\pgfpathlineto{\pgfqpoint{4.441311in}{1.984084in}}%
\pgfpathlineto{\pgfqpoint{4.433757in}{1.973823in}}%
\pgfpathlineto{\pgfqpoint{4.426199in}{1.963567in}}%
\pgfpathclose%
\pgfusepath{fill}%
\end{pgfscope}%
\begin{pgfscope}%
\pgfpathrectangle{\pgfqpoint{1.254980in}{0.150000in}}{\pgfqpoint{5.490039in}{5.490039in}}%
\pgfusepath{clip}%
\pgfsetbuttcap%
\pgfsetroundjoin%
\definecolor{currentfill}{rgb}{0.253935,0.265254,0.529983}%
\pgfsetfillcolor{currentfill}%
\pgfsetfillopacity{0.700000}%
\pgfsetlinewidth{0.000000pt}%
\definecolor{currentstroke}{rgb}{0.000000,0.000000,0.000000}%
\pgfsetstrokecolor{currentstroke}%
\pgfsetdash{}{0pt}%
\pgfpathmoveto{\pgfqpoint{3.032366in}{2.337372in}}%
\pgfpathlineto{\pgfqpoint{3.045656in}{2.322726in}}%
\pgfpathlineto{\pgfqpoint{3.058943in}{2.308244in}}%
\pgfpathlineto{\pgfqpoint{3.072228in}{2.293923in}}%
\pgfpathlineto{\pgfqpoint{3.085509in}{2.279764in}}%
\pgfpathlineto{\pgfqpoint{3.093647in}{2.282793in}}%
\pgfpathlineto{\pgfqpoint{3.101775in}{2.285972in}}%
\pgfpathlineto{\pgfqpoint{3.109892in}{2.289297in}}%
\pgfpathlineto{\pgfqpoint{3.117999in}{2.292765in}}%
\pgfpathlineto{\pgfqpoint{3.104746in}{2.306619in}}%
\pgfpathlineto{\pgfqpoint{3.091491in}{2.320633in}}%
\pgfpathlineto{\pgfqpoint{3.078233in}{2.334809in}}%
\pgfpathlineto{\pgfqpoint{3.064973in}{2.349148in}}%
\pgfpathlineto{\pgfqpoint{3.056837in}{2.345979in}}%
\pgfpathlineto{\pgfqpoint{3.048691in}{2.342959in}}%
\pgfpathlineto{\pgfqpoint{3.040534in}{2.340089in}}%
\pgfpathlineto{\pgfqpoint{3.032366in}{2.337372in}}%
\pgfpathclose%
\pgfusepath{fill}%
\end{pgfscope}%
\begin{pgfscope}%
\pgfpathrectangle{\pgfqpoint{1.254980in}{0.150000in}}{\pgfqpoint{5.490039in}{5.490039in}}%
\pgfusepath{clip}%
\pgfsetbuttcap%
\pgfsetroundjoin%
\definecolor{currentfill}{rgb}{0.192357,0.403199,0.555836}%
\pgfsetfillcolor{currentfill}%
\pgfsetfillopacity{0.700000}%
\pgfsetlinewidth{0.000000pt}%
\definecolor{currentstroke}{rgb}{0.000000,0.000000,0.000000}%
\pgfsetstrokecolor{currentstroke}%
\pgfsetdash{}{0pt}%
\pgfpathmoveto{\pgfqpoint{5.463205in}{2.601914in}}%
\pgfpathlineto{\pgfqpoint{5.477060in}{2.607023in}}%
\pgfpathlineto{\pgfqpoint{5.490928in}{2.612245in}}%
\pgfpathlineto{\pgfqpoint{5.504810in}{2.617579in}}%
\pgfpathlineto{\pgfqpoint{5.518707in}{2.623026in}}%
\pgfpathlineto{\pgfqpoint{5.525900in}{2.632275in}}%
\pgfpathlineto{\pgfqpoint{5.533086in}{2.641470in}}%
\pgfpathlineto{\pgfqpoint{5.540266in}{2.650613in}}%
\pgfpathlineto{\pgfqpoint{5.547439in}{2.659702in}}%
\pgfpathlineto{\pgfqpoint{5.533552in}{2.654317in}}%
\pgfpathlineto{\pgfqpoint{5.519679in}{2.649044in}}%
\pgfpathlineto{\pgfqpoint{5.505820in}{2.643884in}}%
\pgfpathlineto{\pgfqpoint{5.491975in}{2.638836in}}%
\pgfpathlineto{\pgfqpoint{5.484791in}{2.629679in}}%
\pgfpathlineto{\pgfqpoint{5.477602in}{2.620473in}}%
\pgfpathlineto{\pgfqpoint{5.470407in}{2.611218in}}%
\pgfpathlineto{\pgfqpoint{5.463205in}{2.601914in}}%
\pgfpathclose%
\pgfusepath{fill}%
\end{pgfscope}%
\begin{pgfscope}%
\pgfpathrectangle{\pgfqpoint{1.254980in}{0.150000in}}{\pgfqpoint{5.490039in}{5.490039in}}%
\pgfusepath{clip}%
\pgfsetbuttcap%
\pgfsetroundjoin%
\definecolor{currentfill}{rgb}{0.204903,0.375746,0.553533}%
\pgfsetfillcolor{currentfill}%
\pgfsetfillopacity{0.700000}%
\pgfsetlinewidth{0.000000pt}%
\definecolor{currentstroke}{rgb}{0.000000,0.000000,0.000000}%
\pgfsetstrokecolor{currentstroke}%
\pgfsetdash{}{0pt}%
\pgfpathmoveto{\pgfqpoint{2.819209in}{2.594653in}}%
\pgfpathlineto{\pgfqpoint{2.832566in}{2.577268in}}%
\pgfpathlineto{\pgfqpoint{2.845918in}{2.560061in}}%
\pgfpathlineto{\pgfqpoint{2.859264in}{2.543032in}}%
\pgfpathlineto{\pgfqpoint{2.872605in}{2.526180in}}%
\pgfpathlineto{\pgfqpoint{2.880869in}{2.527947in}}%
\pgfpathlineto{\pgfqpoint{2.889121in}{2.529879in}}%
\pgfpathlineto{\pgfqpoint{2.897361in}{2.531976in}}%
\pgfpathlineto{\pgfqpoint{2.905590in}{2.534234in}}%
\pgfpathlineto{\pgfqpoint{2.892282in}{2.550773in}}%
\pgfpathlineto{\pgfqpoint{2.878969in}{2.567488in}}%
\pgfpathlineto{\pgfqpoint{2.865651in}{2.584380in}}%
\pgfpathlineto{\pgfqpoint{2.852328in}{2.601451in}}%
\pgfpathlineto{\pgfqpoint{2.844067in}{2.599500in}}%
\pgfpathlineto{\pgfqpoint{2.835793in}{2.597715in}}%
\pgfpathlineto{\pgfqpoint{2.827507in}{2.596099in}}%
\pgfpathlineto{\pgfqpoint{2.819209in}{2.594653in}}%
\pgfpathclose%
\pgfusepath{fill}%
\end{pgfscope}%
\begin{pgfscope}%
\pgfpathrectangle{\pgfqpoint{1.254980in}{0.150000in}}{\pgfqpoint{5.490039in}{5.490039in}}%
\pgfusepath{clip}%
\pgfsetbuttcap%
\pgfsetroundjoin%
\definecolor{currentfill}{rgb}{0.279566,0.067836,0.391917}%
\pgfsetfillcolor{currentfill}%
\pgfsetfillopacity{0.700000}%
\pgfsetlinewidth{0.000000pt}%
\definecolor{currentstroke}{rgb}{0.000000,0.000000,0.000000}%
\pgfsetstrokecolor{currentstroke}%
\pgfsetdash{}{0pt}%
\pgfpathmoveto{\pgfqpoint{3.626141in}{1.920174in}}%
\pgfpathlineto{\pgfqpoint{3.639379in}{1.912034in}}%
\pgfpathlineto{\pgfqpoint{3.652620in}{1.904029in}}%
\pgfpathlineto{\pgfqpoint{3.665864in}{1.896158in}}%
\pgfpathlineto{\pgfqpoint{3.679110in}{1.888421in}}%
\pgfpathlineto{\pgfqpoint{3.686946in}{1.895299in}}%
\pgfpathlineto{\pgfqpoint{3.694774in}{1.902265in}}%
\pgfpathlineto{\pgfqpoint{3.702596in}{1.909318in}}%
\pgfpathlineto{\pgfqpoint{3.710411in}{1.916455in}}%
\pgfpathlineto{\pgfqpoint{3.697182in}{1.923929in}}%
\pgfpathlineto{\pgfqpoint{3.683957in}{1.931537in}}%
\pgfpathlineto{\pgfqpoint{3.670734in}{1.939280in}}%
\pgfpathlineto{\pgfqpoint{3.657514in}{1.947157in}}%
\pgfpathlineto{\pgfqpoint{3.649681in}{1.940277in}}%
\pgfpathlineto{\pgfqpoint{3.641842in}{1.933485in}}%
\pgfpathlineto{\pgfqpoint{3.633995in}{1.926783in}}%
\pgfpathlineto{\pgfqpoint{3.626141in}{1.920174in}}%
\pgfpathclose%
\pgfusepath{fill}%
\end{pgfscope}%
\begin{pgfscope}%
\pgfpathrectangle{\pgfqpoint{1.254980in}{0.150000in}}{\pgfqpoint{5.490039in}{5.490039in}}%
\pgfusepath{clip}%
\pgfsetbuttcap%
\pgfsetroundjoin%
\definecolor{currentfill}{rgb}{0.262138,0.242286,0.520837}%
\pgfsetfillcolor{currentfill}%
\pgfsetfillopacity{0.700000}%
\pgfsetlinewidth{0.000000pt}%
\definecolor{currentstroke}{rgb}{0.000000,0.000000,0.000000}%
\pgfsetstrokecolor{currentstroke}%
\pgfsetdash{}{0pt}%
\pgfpathmoveto{\pgfqpoint{4.929415in}{2.231062in}}%
\pgfpathlineto{\pgfqpoint{4.943012in}{2.233412in}}%
\pgfpathlineto{\pgfqpoint{4.956620in}{2.235877in}}%
\pgfpathlineto{\pgfqpoint{4.970240in}{2.238457in}}%
\pgfpathlineto{\pgfqpoint{4.983871in}{2.241152in}}%
\pgfpathlineto{\pgfqpoint{4.991267in}{2.251740in}}%
\pgfpathlineto{\pgfqpoint{4.998657in}{2.262290in}}%
\pgfpathlineto{\pgfqpoint{5.006042in}{2.272802in}}%
\pgfpathlineto{\pgfqpoint{5.013422in}{2.283277in}}%
\pgfpathlineto{\pgfqpoint{4.999796in}{2.280529in}}%
\pgfpathlineto{\pgfqpoint{4.986183in}{2.277895in}}%
\pgfpathlineto{\pgfqpoint{4.972581in}{2.275376in}}%
\pgfpathlineto{\pgfqpoint{4.958990in}{2.272971in}}%
\pgfpathlineto{\pgfqpoint{4.951604in}{2.262545in}}%
\pgfpathlineto{\pgfqpoint{4.944213in}{2.252084in}}%
\pgfpathlineto{\pgfqpoint{4.936817in}{2.241590in}}%
\pgfpathlineto{\pgfqpoint{4.929415in}{2.231062in}}%
\pgfpathclose%
\pgfusepath{fill}%
\end{pgfscope}%
\begin{pgfscope}%
\pgfpathrectangle{\pgfqpoint{1.254980in}{0.150000in}}{\pgfqpoint{5.490039in}{5.490039in}}%
\pgfusepath{clip}%
\pgfsetbuttcap%
\pgfsetroundjoin%
\definecolor{currentfill}{rgb}{0.262138,0.242286,0.520837}%
\pgfsetfillcolor{currentfill}%
\pgfsetfillopacity{0.700000}%
\pgfsetlinewidth{0.000000pt}%
\definecolor{currentstroke}{rgb}{0.000000,0.000000,0.000000}%
\pgfsetstrokecolor{currentstroke}%
\pgfsetdash{}{0pt}%
\pgfpathmoveto{\pgfqpoint{3.085509in}{2.279764in}}%
\pgfpathlineto{\pgfqpoint{3.098789in}{2.265765in}}%
\pgfpathlineto{\pgfqpoint{3.112066in}{2.251925in}}%
\pgfpathlineto{\pgfqpoint{3.125341in}{2.238245in}}%
\pgfpathlineto{\pgfqpoint{3.138613in}{2.224722in}}%
\pgfpathlineto{\pgfqpoint{3.146723in}{2.228063in}}%
\pgfpathlineto{\pgfqpoint{3.154822in}{2.231549in}}%
\pgfpathlineto{\pgfqpoint{3.162910in}{2.235177in}}%
\pgfpathlineto{\pgfqpoint{3.170990in}{2.238944in}}%
\pgfpathlineto{\pgfqpoint{3.157745in}{2.252162in}}%
\pgfpathlineto{\pgfqpoint{3.144498in}{2.265538in}}%
\pgfpathlineto{\pgfqpoint{3.131250in}{2.279072in}}%
\pgfpathlineto{\pgfqpoint{3.117999in}{2.292765in}}%
\pgfpathlineto{\pgfqpoint{3.109892in}{2.289297in}}%
\pgfpathlineto{\pgfqpoint{3.101775in}{2.285972in}}%
\pgfpathlineto{\pgfqpoint{3.093647in}{2.282793in}}%
\pgfpathlineto{\pgfqpoint{3.085509in}{2.279764in}}%
\pgfpathclose%
\pgfusepath{fill}%
\end{pgfscope}%
\begin{pgfscope}%
\pgfpathrectangle{\pgfqpoint{1.254980in}{0.150000in}}{\pgfqpoint{5.490039in}{5.490039in}}%
\pgfusepath{clip}%
\pgfsetbuttcap%
\pgfsetroundjoin%
\definecolor{currentfill}{rgb}{0.283091,0.110553,0.431554}%
\pgfsetfillcolor{currentfill}%
\pgfsetfillopacity{0.700000}%
\pgfsetlinewidth{0.000000pt}%
\definecolor{currentstroke}{rgb}{0.000000,0.000000,0.000000}%
\pgfsetstrokecolor{currentstroke}%
\pgfsetdash{}{0pt}%
\pgfpathmoveto{\pgfqpoint{3.435712in}{2.006476in}}%
\pgfpathlineto{\pgfqpoint{3.448949in}{1.996393in}}%
\pgfpathlineto{\pgfqpoint{3.462188in}{1.986451in}}%
\pgfpathlineto{\pgfqpoint{3.475427in}{1.976651in}}%
\pgfpathlineto{\pgfqpoint{3.488668in}{1.966992in}}%
\pgfpathlineto{\pgfqpoint{3.496592in}{1.972636in}}%
\pgfpathlineto{\pgfqpoint{3.504508in}{1.978390in}}%
\pgfpathlineto{\pgfqpoint{3.512416in}{1.984251in}}%
\pgfpathlineto{\pgfqpoint{3.520317in}{1.990217in}}%
\pgfpathlineto{\pgfqpoint{3.507098in}{1.999596in}}%
\pgfpathlineto{\pgfqpoint{3.493880in}{2.009115in}}%
\pgfpathlineto{\pgfqpoint{3.480663in}{2.018775in}}%
\pgfpathlineto{\pgfqpoint{3.467447in}{2.028577in}}%
\pgfpathlineto{\pgfqpoint{3.459526in}{2.022885in}}%
\pgfpathlineto{\pgfqpoint{3.451596in}{2.017303in}}%
\pgfpathlineto{\pgfqpoint{3.443658in}{2.011833in}}%
\pgfpathlineto{\pgfqpoint{3.435712in}{2.006476in}}%
\pgfpathclose%
\pgfusepath{fill}%
\end{pgfscope}%
\begin{pgfscope}%
\pgfpathrectangle{\pgfqpoint{1.254980in}{0.150000in}}{\pgfqpoint{5.490039in}{5.490039in}}%
\pgfusepath{clip}%
\pgfsetbuttcap%
\pgfsetroundjoin%
\definecolor{currentfill}{rgb}{0.281446,0.084320,0.407414}%
\pgfsetfillcolor{currentfill}%
\pgfsetfillopacity{0.700000}%
\pgfsetlinewidth{0.000000pt}%
\definecolor{currentstroke}{rgb}{0.000000,0.000000,0.000000}%
\pgfsetstrokecolor{currentstroke}%
\pgfsetdash{}{0pt}%
\pgfpathmoveto{\pgfqpoint{4.342378in}{1.928900in}}%
\pgfpathlineto{\pgfqpoint{4.355750in}{1.927150in}}%
\pgfpathlineto{\pgfqpoint{4.369131in}{1.925520in}}%
\pgfpathlineto{\pgfqpoint{4.382519in}{1.924009in}}%
\pgfpathlineto{\pgfqpoint{4.395916in}{1.922618in}}%
\pgfpathlineto{\pgfqpoint{4.403494in}{1.932842in}}%
\pgfpathlineto{\pgfqpoint{4.411067in}{1.943075in}}%
\pgfpathlineto{\pgfqpoint{4.418635in}{1.953318in}}%
\pgfpathlineto{\pgfqpoint{4.426199in}{1.963567in}}%
\pgfpathlineto{\pgfqpoint{4.412810in}{1.964794in}}%
\pgfpathlineto{\pgfqpoint{4.399430in}{1.966139in}}%
\pgfpathlineto{\pgfqpoint{4.386057in}{1.967604in}}%
\pgfpathlineto{\pgfqpoint{4.372693in}{1.969189in}}%
\pgfpathlineto{\pgfqpoint{4.365122in}{1.959098in}}%
\pgfpathlineto{\pgfqpoint{4.357545in}{1.949019in}}%
\pgfpathlineto{\pgfqpoint{4.349964in}{1.938953in}}%
\pgfpathlineto{\pgfqpoint{4.342378in}{1.928900in}}%
\pgfpathclose%
\pgfusepath{fill}%
\end{pgfscope}%
\begin{pgfscope}%
\pgfpathrectangle{\pgfqpoint{1.254980in}{0.150000in}}{\pgfqpoint{5.490039in}{5.490039in}}%
\pgfusepath{clip}%
\pgfsetbuttcap%
\pgfsetroundjoin%
\definecolor{currentfill}{rgb}{0.276022,0.044167,0.370164}%
\pgfsetfillcolor{currentfill}%
\pgfsetfillopacity{0.700000}%
\pgfsetlinewidth{0.000000pt}%
\definecolor{currentstroke}{rgb}{0.000000,0.000000,0.000000}%
\pgfsetstrokecolor{currentstroke}%
\pgfsetdash{}{0pt}%
\pgfpathmoveto{\pgfqpoint{4.037508in}{1.863760in}}%
\pgfpathlineto{\pgfqpoint{4.050801in}{1.859466in}}%
\pgfpathlineto{\pgfqpoint{4.064100in}{1.855297in}}%
\pgfpathlineto{\pgfqpoint{4.077406in}{1.851251in}}%
\pgfpathlineto{\pgfqpoint{4.090717in}{1.847329in}}%
\pgfpathlineto{\pgfqpoint{4.098394in}{1.856457in}}%
\pgfpathlineto{\pgfqpoint{4.106065in}{1.865628in}}%
\pgfpathlineto{\pgfqpoint{4.113731in}{1.874840in}}%
\pgfpathlineto{\pgfqpoint{4.121391in}{1.884091in}}%
\pgfpathlineto{\pgfqpoint{4.108091in}{1.887800in}}%
\pgfpathlineto{\pgfqpoint{4.094797in}{1.891633in}}%
\pgfpathlineto{\pgfqpoint{4.081510in}{1.895589in}}%
\pgfpathlineto{\pgfqpoint{4.068228in}{1.899671in}}%
\pgfpathlineto{\pgfqpoint{4.060556in}{1.890627in}}%
\pgfpathlineto{\pgfqpoint{4.052879in}{1.881626in}}%
\pgfpathlineto{\pgfqpoint{4.045196in}{1.872670in}}%
\pgfpathlineto{\pgfqpoint{4.037508in}{1.863760in}}%
\pgfpathclose%
\pgfusepath{fill}%
\end{pgfscope}%
\begin{pgfscope}%
\pgfpathrectangle{\pgfqpoint{1.254980in}{0.150000in}}{\pgfqpoint{5.490039in}{5.490039in}}%
\pgfusepath{clip}%
\pgfsetbuttcap%
\pgfsetroundjoin%
\definecolor{currentfill}{rgb}{0.192357,0.403199,0.555836}%
\pgfsetfillcolor{currentfill}%
\pgfsetfillopacity{0.700000}%
\pgfsetlinewidth{0.000000pt}%
\definecolor{currentstroke}{rgb}{0.000000,0.000000,0.000000}%
\pgfsetstrokecolor{currentstroke}%
\pgfsetdash{}{0pt}%
\pgfpathmoveto{\pgfqpoint{2.765726in}{2.666007in}}%
\pgfpathlineto{\pgfqpoint{2.779105in}{2.647895in}}%
\pgfpathlineto{\pgfqpoint{2.792479in}{2.629966in}}%
\pgfpathlineto{\pgfqpoint{2.805847in}{2.612219in}}%
\pgfpathlineto{\pgfqpoint{2.819209in}{2.594653in}}%
\pgfpathlineto{\pgfqpoint{2.827507in}{2.596099in}}%
\pgfpathlineto{\pgfqpoint{2.835793in}{2.597715in}}%
\pgfpathlineto{\pgfqpoint{2.844067in}{2.599500in}}%
\pgfpathlineto{\pgfqpoint{2.852328in}{2.601451in}}%
\pgfpathlineto{\pgfqpoint{2.839000in}{2.618701in}}%
\pgfpathlineto{\pgfqpoint{2.825667in}{2.636132in}}%
\pgfpathlineto{\pgfqpoint{2.812328in}{2.653745in}}%
\pgfpathlineto{\pgfqpoint{2.798983in}{2.671541in}}%
\pgfpathlineto{\pgfqpoint{2.790688in}{2.669899in}}%
\pgfpathlineto{\pgfqpoint{2.782380in}{2.668428in}}%
\pgfpathlineto{\pgfqpoint{2.774059in}{2.667130in}}%
\pgfpathlineto{\pgfqpoint{2.765726in}{2.666007in}}%
\pgfpathclose%
\pgfusepath{fill}%
\end{pgfscope}%
\begin{pgfscope}%
\pgfpathrectangle{\pgfqpoint{1.254980in}{0.150000in}}{\pgfqpoint{5.490039in}{5.490039in}}%
\pgfusepath{clip}%
\pgfsetbuttcap%
\pgfsetroundjoin%
\definecolor{currentfill}{rgb}{0.253935,0.265254,0.529983}%
\pgfsetfillcolor{currentfill}%
\pgfsetfillopacity{0.700000}%
\pgfsetlinewidth{0.000000pt}%
\definecolor{currentstroke}{rgb}{0.000000,0.000000,0.000000}%
\pgfsetstrokecolor{currentstroke}%
\pgfsetdash{}{0pt}%
\pgfpathmoveto{\pgfqpoint{5.013422in}{2.283277in}}%
\pgfpathlineto{\pgfqpoint{5.027059in}{2.286139in}}%
\pgfpathlineto{\pgfqpoint{5.040708in}{2.289116in}}%
\pgfpathlineto{\pgfqpoint{5.054370in}{2.292207in}}%
\pgfpathlineto{\pgfqpoint{5.068043in}{2.295412in}}%
\pgfpathlineto{\pgfqpoint{5.075411in}{2.305892in}}%
\pgfpathlineto{\pgfqpoint{5.082774in}{2.316329in}}%
\pgfpathlineto{\pgfqpoint{5.090132in}{2.326724in}}%
\pgfpathlineto{\pgfqpoint{5.097484in}{2.337076in}}%
\pgfpathlineto{\pgfqpoint{5.083817in}{2.333834in}}%
\pgfpathlineto{\pgfqpoint{5.070162in}{2.330705in}}%
\pgfpathlineto{\pgfqpoint{5.056519in}{2.327691in}}%
\pgfpathlineto{\pgfqpoint{5.042888in}{2.324791in}}%
\pgfpathlineto{\pgfqpoint{5.035529in}{2.314470in}}%
\pgfpathlineto{\pgfqpoint{5.028165in}{2.304110in}}%
\pgfpathlineto{\pgfqpoint{5.020796in}{2.293713in}}%
\pgfpathlineto{\pgfqpoint{5.013422in}{2.283277in}}%
\pgfpathclose%
\pgfusepath{fill}%
\end{pgfscope}%
\begin{pgfscope}%
\pgfpathrectangle{\pgfqpoint{1.254980in}{0.150000in}}{\pgfqpoint{5.490039in}{5.490039in}}%
\pgfusepath{clip}%
\pgfsetbuttcap%
\pgfsetroundjoin%
\definecolor{currentfill}{rgb}{0.269308,0.218818,0.509577}%
\pgfsetfillcolor{currentfill}%
\pgfsetfillopacity{0.700000}%
\pgfsetlinewidth{0.000000pt}%
\definecolor{currentstroke}{rgb}{0.000000,0.000000,0.000000}%
\pgfsetstrokecolor{currentstroke}%
\pgfsetdash{}{0pt}%
\pgfpathmoveto{\pgfqpoint{3.138613in}{2.224722in}}%
\pgfpathlineto{\pgfqpoint{3.151884in}{2.211356in}}%
\pgfpathlineto{\pgfqpoint{3.165153in}{2.198147in}}%
\pgfpathlineto{\pgfqpoint{3.178421in}{2.185093in}}%
\pgfpathlineto{\pgfqpoint{3.191687in}{2.172194in}}%
\pgfpathlineto{\pgfqpoint{3.199768in}{2.175845in}}%
\pgfpathlineto{\pgfqpoint{3.207840in}{2.179637in}}%
\pgfpathlineto{\pgfqpoint{3.215901in}{2.183566in}}%
\pgfpathlineto{\pgfqpoint{3.223954in}{2.187630in}}%
\pgfpathlineto{\pgfqpoint{3.210715in}{2.200226in}}%
\pgfpathlineto{\pgfqpoint{3.197474in}{2.212977in}}%
\pgfpathlineto{\pgfqpoint{3.184233in}{2.225882in}}%
\pgfpathlineto{\pgfqpoint{3.170990in}{2.238944in}}%
\pgfpathlineto{\pgfqpoint{3.162910in}{2.235177in}}%
\pgfpathlineto{\pgfqpoint{3.154822in}{2.231549in}}%
\pgfpathlineto{\pgfqpoint{3.146723in}{2.228063in}}%
\pgfpathlineto{\pgfqpoint{3.138613in}{2.224722in}}%
\pgfpathclose%
\pgfusepath{fill}%
\end{pgfscope}%
\begin{pgfscope}%
\pgfpathrectangle{\pgfqpoint{1.254980in}{0.150000in}}{\pgfqpoint{5.490039in}{5.490039in}}%
\pgfusepath{clip}%
\pgfsetbuttcap%
\pgfsetroundjoin%
\definecolor{currentfill}{rgb}{0.182256,0.426184,0.557120}%
\pgfsetfillcolor{currentfill}%
\pgfsetfillopacity{0.700000}%
\pgfsetlinewidth{0.000000pt}%
\definecolor{currentstroke}{rgb}{0.000000,0.000000,0.000000}%
\pgfsetstrokecolor{currentstroke}%
\pgfsetdash{}{0pt}%
\pgfpathmoveto{\pgfqpoint{5.547439in}{2.659702in}}%
\pgfpathlineto{\pgfqpoint{5.561341in}{2.665200in}}%
\pgfpathlineto{\pgfqpoint{5.575258in}{2.670809in}}%
\pgfpathlineto{\pgfqpoint{5.589189in}{2.676531in}}%
\pgfpathlineto{\pgfqpoint{5.603135in}{2.682365in}}%
\pgfpathlineto{\pgfqpoint{5.610293in}{2.691331in}}%
\pgfpathlineto{\pgfqpoint{5.617444in}{2.700243in}}%
\pgfpathlineto{\pgfqpoint{5.624589in}{2.709100in}}%
\pgfpathlineto{\pgfqpoint{5.631727in}{2.717905in}}%
\pgfpathlineto{\pgfqpoint{5.617792in}{2.712150in}}%
\pgfpathlineto{\pgfqpoint{5.603871in}{2.706506in}}%
\pgfpathlineto{\pgfqpoint{5.589964in}{2.700975in}}%
\pgfpathlineto{\pgfqpoint{5.576072in}{2.695556in}}%
\pgfpathlineto{\pgfqpoint{5.568923in}{2.686666in}}%
\pgfpathlineto{\pgfqpoint{5.561768in}{2.677728in}}%
\pgfpathlineto{\pgfqpoint{5.554607in}{2.668740in}}%
\pgfpathlineto{\pgfqpoint{5.547439in}{2.659702in}}%
\pgfpathclose%
\pgfusepath{fill}%
\end{pgfscope}%
\begin{pgfscope}%
\pgfpathrectangle{\pgfqpoint{1.254980in}{0.150000in}}{\pgfqpoint{5.490039in}{5.490039in}}%
\pgfusepath{clip}%
\pgfsetbuttcap%
\pgfsetroundjoin%
\definecolor{currentfill}{rgb}{0.122606,0.585371,0.546557}%
\pgfsetfillcolor{currentfill}%
\pgfsetfillopacity{0.700000}%
\pgfsetlinewidth{0.000000pt}%
\definecolor{currentstroke}{rgb}{0.000000,0.000000,0.000000}%
\pgfsetstrokecolor{currentstroke}%
\pgfsetdash{}{0pt}%
\pgfpathmoveto{\pgfqpoint{2.476542in}{3.156757in}}%
\pgfpathlineto{\pgfqpoint{2.490077in}{3.134188in}}%
\pgfpathlineto{\pgfqpoint{2.503603in}{3.111833in}}%
\pgfpathlineto{\pgfqpoint{2.517118in}{3.089692in}}%
\pgfpathlineto{\pgfqpoint{2.530623in}{3.067763in}}%
\pgfpathlineto{\pgfqpoint{2.539092in}{3.068012in}}%
\pgfpathlineto{\pgfqpoint{2.547546in}{3.068451in}}%
\pgfpathlineto{\pgfqpoint{2.555987in}{3.069075in}}%
\pgfpathlineto{\pgfqpoint{2.564413in}{3.069884in}}%
\pgfpathlineto{\pgfqpoint{2.550947in}{3.091501in}}%
\pgfpathlineto{\pgfqpoint{2.537472in}{3.113329in}}%
\pgfpathlineto{\pgfqpoint{2.523987in}{3.135370in}}%
\pgfpathlineto{\pgfqpoint{2.510492in}{3.157626in}}%
\pgfpathlineto{\pgfqpoint{2.502026in}{3.157123in}}%
\pgfpathlineto{\pgfqpoint{2.493546in}{3.156810in}}%
\pgfpathlineto{\pgfqpoint{2.485052in}{3.156687in}}%
\pgfpathlineto{\pgfqpoint{2.476542in}{3.156757in}}%
\pgfpathclose%
\pgfusepath{fill}%
\end{pgfscope}%
\begin{pgfscope}%
\pgfpathrectangle{\pgfqpoint{1.254980in}{0.150000in}}{\pgfqpoint{5.490039in}{5.490039in}}%
\pgfusepath{clip}%
\pgfsetbuttcap%
\pgfsetroundjoin%
\definecolor{currentfill}{rgb}{0.279566,0.067836,0.391917}%
\pgfsetfillcolor{currentfill}%
\pgfsetfillopacity{0.700000}%
\pgfsetlinewidth{0.000000pt}%
\definecolor{currentstroke}{rgb}{0.000000,0.000000,0.000000}%
\pgfsetstrokecolor{currentstroke}%
\pgfsetdash{}{0pt}%
\pgfpathmoveto{\pgfqpoint{4.258536in}{1.897776in}}%
\pgfpathlineto{\pgfqpoint{4.271887in}{1.895364in}}%
\pgfpathlineto{\pgfqpoint{4.285245in}{1.893072in}}%
\pgfpathlineto{\pgfqpoint{4.298611in}{1.890901in}}%
\pgfpathlineto{\pgfqpoint{4.311984in}{1.888850in}}%
\pgfpathlineto{\pgfqpoint{4.319590in}{1.898836in}}%
\pgfpathlineto{\pgfqpoint{4.327191in}{1.908840in}}%
\pgfpathlineto{\pgfqpoint{4.334787in}{1.918862in}}%
\pgfpathlineto{\pgfqpoint{4.342378in}{1.928900in}}%
\pgfpathlineto{\pgfqpoint{4.329013in}{1.930770in}}%
\pgfpathlineto{\pgfqpoint{4.315656in}{1.932760in}}%
\pgfpathlineto{\pgfqpoint{4.302307in}{1.934871in}}%
\pgfpathlineto{\pgfqpoint{4.288965in}{1.937102in}}%
\pgfpathlineto{\pgfqpoint{4.281365in}{1.927239in}}%
\pgfpathlineto{\pgfqpoint{4.273760in}{1.917396in}}%
\pgfpathlineto{\pgfqpoint{4.266150in}{1.907575in}}%
\pgfpathlineto{\pgfqpoint{4.258536in}{1.897776in}}%
\pgfpathclose%
\pgfusepath{fill}%
\end{pgfscope}%
\begin{pgfscope}%
\pgfpathrectangle{\pgfqpoint{1.254980in}{0.150000in}}{\pgfqpoint{5.490039in}{5.490039in}}%
\pgfusepath{clip}%
\pgfsetbuttcap%
\pgfsetroundjoin%
\definecolor{currentfill}{rgb}{0.179019,0.433756,0.557430}%
\pgfsetfillcolor{currentfill}%
\pgfsetfillopacity{0.700000}%
\pgfsetlinewidth{0.000000pt}%
\definecolor{currentstroke}{rgb}{0.000000,0.000000,0.000000}%
\pgfsetstrokecolor{currentstroke}%
\pgfsetdash{}{0pt}%
\pgfpathmoveto{\pgfqpoint{2.712143in}{2.740316in}}%
\pgfpathlineto{\pgfqpoint{2.725549in}{2.721458in}}%
\pgfpathlineto{\pgfqpoint{2.738947in}{2.702788in}}%
\pgfpathlineto{\pgfqpoint{2.752340in}{2.684305in}}%
\pgfpathlineto{\pgfqpoint{2.765726in}{2.666007in}}%
\pgfpathlineto{\pgfqpoint{2.774059in}{2.667130in}}%
\pgfpathlineto{\pgfqpoint{2.782380in}{2.668428in}}%
\pgfpathlineto{\pgfqpoint{2.790688in}{2.669899in}}%
\pgfpathlineto{\pgfqpoint{2.798983in}{2.671541in}}%
\pgfpathlineto{\pgfqpoint{2.785633in}{2.689520in}}%
\pgfpathlineto{\pgfqpoint{2.772276in}{2.707685in}}%
\pgfpathlineto{\pgfqpoint{2.758913in}{2.726037in}}%
\pgfpathlineto{\pgfqpoint{2.745544in}{2.744576in}}%
\pgfpathlineto{\pgfqpoint{2.737213in}{2.743246in}}%
\pgfpathlineto{\pgfqpoint{2.728870in}{2.742091in}}%
\pgfpathlineto{\pgfqpoint{2.720513in}{2.741114in}}%
\pgfpathlineto{\pgfqpoint{2.712143in}{2.740316in}}%
\pgfpathclose%
\pgfusepath{fill}%
\end{pgfscope}%
\begin{pgfscope}%
\pgfpathrectangle{\pgfqpoint{1.254980in}{0.150000in}}{\pgfqpoint{5.490039in}{5.490039in}}%
\pgfusepath{clip}%
\pgfsetbuttcap%
\pgfsetroundjoin%
\definecolor{currentfill}{rgb}{0.243113,0.292092,0.538516}%
\pgfsetfillcolor{currentfill}%
\pgfsetfillopacity{0.700000}%
\pgfsetlinewidth{0.000000pt}%
\definecolor{currentstroke}{rgb}{0.000000,0.000000,0.000000}%
\pgfsetstrokecolor{currentstroke}%
\pgfsetdash{}{0pt}%
\pgfpathmoveto{\pgfqpoint{5.097484in}{2.337076in}}%
\pgfpathlineto{\pgfqpoint{5.111164in}{2.340433in}}%
\pgfpathlineto{\pgfqpoint{5.124856in}{2.343903in}}%
\pgfpathlineto{\pgfqpoint{5.138560in}{2.347488in}}%
\pgfpathlineto{\pgfqpoint{5.152277in}{2.351185in}}%
\pgfpathlineto{\pgfqpoint{5.159618in}{2.361523in}}%
\pgfpathlineto{\pgfqpoint{5.166952in}{2.371814in}}%
\pgfpathlineto{\pgfqpoint{5.174282in}{2.382058in}}%
\pgfpathlineto{\pgfqpoint{5.181605in}{2.392257in}}%
\pgfpathlineto{\pgfqpoint{5.167895in}{2.388538in}}%
\pgfpathlineto{\pgfqpoint{5.154197in}{2.384932in}}%
\pgfpathlineto{\pgfqpoint{5.140511in}{2.381441in}}%
\pgfpathlineto{\pgfqpoint{5.126838in}{2.378063in}}%
\pgfpathlineto{\pgfqpoint{5.119508in}{2.367879in}}%
\pgfpathlineto{\pgfqpoint{5.112172in}{2.357654in}}%
\pgfpathlineto{\pgfqpoint{5.104831in}{2.347386in}}%
\pgfpathlineto{\pgfqpoint{5.097484in}{2.337076in}}%
\pgfpathclose%
\pgfusepath{fill}%
\end{pgfscope}%
\begin{pgfscope}%
\pgfpathrectangle{\pgfqpoint{1.254980in}{0.150000in}}{\pgfqpoint{5.490039in}{5.490039in}}%
\pgfusepath{clip}%
\pgfsetbuttcap%
\pgfsetroundjoin%
\definecolor{currentfill}{rgb}{0.275191,0.194905,0.496005}%
\pgfsetfillcolor{currentfill}%
\pgfsetfillopacity{0.700000}%
\pgfsetlinewidth{0.000000pt}%
\definecolor{currentstroke}{rgb}{0.000000,0.000000,0.000000}%
\pgfsetstrokecolor{currentstroke}%
\pgfsetdash{}{0pt}%
\pgfpathmoveto{\pgfqpoint{3.191687in}{2.172194in}}%
\pgfpathlineto{\pgfqpoint{3.204952in}{2.159448in}}%
\pgfpathlineto{\pgfqpoint{3.218215in}{2.146856in}}%
\pgfpathlineto{\pgfqpoint{3.231477in}{2.134416in}}%
\pgfpathlineto{\pgfqpoint{3.244738in}{2.122128in}}%
\pgfpathlineto{\pgfqpoint{3.252793in}{2.126088in}}%
\pgfpathlineto{\pgfqpoint{3.260838in}{2.130184in}}%
\pgfpathlineto{\pgfqpoint{3.268873in}{2.134413in}}%
\pgfpathlineto{\pgfqpoint{3.276899in}{2.138773in}}%
\pgfpathlineto{\pgfqpoint{3.263664in}{2.150760in}}%
\pgfpathlineto{\pgfqpoint{3.250428in}{2.162897in}}%
\pgfpathlineto{\pgfqpoint{3.237191in}{2.175187in}}%
\pgfpathlineto{\pgfqpoint{3.223954in}{2.187630in}}%
\pgfpathlineto{\pgfqpoint{3.215901in}{2.183566in}}%
\pgfpathlineto{\pgfqpoint{3.207840in}{2.179637in}}%
\pgfpathlineto{\pgfqpoint{3.199768in}{2.175845in}}%
\pgfpathlineto{\pgfqpoint{3.191687in}{2.172194in}}%
\pgfpathclose%
\pgfusepath{fill}%
\end{pgfscope}%
\begin{pgfscope}%
\pgfpathrectangle{\pgfqpoint{1.254980in}{0.150000in}}{\pgfqpoint{5.490039in}{5.490039in}}%
\pgfusepath{clip}%
\pgfsetbuttcap%
\pgfsetroundjoin%
\definecolor{currentfill}{rgb}{0.172719,0.448791,0.557885}%
\pgfsetfillcolor{currentfill}%
\pgfsetfillopacity{0.700000}%
\pgfsetlinewidth{0.000000pt}%
\definecolor{currentstroke}{rgb}{0.000000,0.000000,0.000000}%
\pgfsetstrokecolor{currentstroke}%
\pgfsetdash{}{0pt}%
\pgfpathmoveto{\pgfqpoint{5.631727in}{2.717905in}}%
\pgfpathlineto{\pgfqpoint{5.645678in}{2.723773in}}%
\pgfpathlineto{\pgfqpoint{5.659644in}{2.729752in}}%
\pgfpathlineto{\pgfqpoint{5.673624in}{2.735844in}}%
\pgfpathlineto{\pgfqpoint{5.687620in}{2.742048in}}%
\pgfpathlineto{\pgfqpoint{5.694741in}{2.750712in}}%
\pgfpathlineto{\pgfqpoint{5.701856in}{2.759321in}}%
\pgfpathlineto{\pgfqpoint{5.708965in}{2.767877in}}%
\pgfpathlineto{\pgfqpoint{5.716067in}{2.776381in}}%
\pgfpathlineto{\pgfqpoint{5.702082in}{2.770273in}}%
\pgfpathlineto{\pgfqpoint{5.688113in}{2.764277in}}%
\pgfpathlineto{\pgfqpoint{5.674158in}{2.758392in}}%
\pgfpathlineto{\pgfqpoint{5.660218in}{2.752620in}}%
\pgfpathlineto{\pgfqpoint{5.653105in}{2.744015in}}%
\pgfpathlineto{\pgfqpoint{5.645985in}{2.735361in}}%
\pgfpathlineto{\pgfqpoint{5.638860in}{2.726659in}}%
\pgfpathlineto{\pgfqpoint{5.631727in}{2.717905in}}%
\pgfpathclose%
\pgfusepath{fill}%
\end{pgfscope}%
\begin{pgfscope}%
\pgfpathrectangle{\pgfqpoint{1.254980in}{0.150000in}}{\pgfqpoint{5.490039in}{5.490039in}}%
\pgfusepath{clip}%
\pgfsetbuttcap%
\pgfsetroundjoin%
\definecolor{currentfill}{rgb}{0.276022,0.044167,0.370164}%
\pgfsetfillcolor{currentfill}%
\pgfsetfillopacity{0.700000}%
\pgfsetlinewidth{0.000000pt}%
\definecolor{currentstroke}{rgb}{0.000000,0.000000,0.000000}%
\pgfsetstrokecolor{currentstroke}%
\pgfsetdash{}{0pt}%
\pgfpathmoveto{\pgfqpoint{3.816366in}{1.861419in}}%
\pgfpathlineto{\pgfqpoint{3.829627in}{1.855127in}}%
\pgfpathlineto{\pgfqpoint{3.842893in}{1.848965in}}%
\pgfpathlineto{\pgfqpoint{3.856163in}{1.842931in}}%
\pgfpathlineto{\pgfqpoint{3.869437in}{1.837026in}}%
\pgfpathlineto{\pgfqpoint{3.877198in}{1.845001in}}%
\pgfpathlineto{\pgfqpoint{3.884952in}{1.853044in}}%
\pgfpathlineto{\pgfqpoint{3.892701in}{1.861154in}}%
\pgfpathlineto{\pgfqpoint{3.900444in}{1.869328in}}%
\pgfpathlineto{\pgfqpoint{3.887184in}{1.874988in}}%
\pgfpathlineto{\pgfqpoint{3.873929in}{1.880777in}}%
\pgfpathlineto{\pgfqpoint{3.860678in}{1.886694in}}%
\pgfpathlineto{\pgfqpoint{3.847432in}{1.892740in}}%
\pgfpathlineto{\pgfqpoint{3.839675in}{1.884805in}}%
\pgfpathlineto{\pgfqpoint{3.831911in}{1.876938in}}%
\pgfpathlineto{\pgfqpoint{3.824141in}{1.869142in}}%
\pgfpathlineto{\pgfqpoint{3.816366in}{1.861419in}}%
\pgfpathclose%
\pgfusepath{fill}%
\end{pgfscope}%
\begin{pgfscope}%
\pgfpathrectangle{\pgfqpoint{1.254980in}{0.150000in}}{\pgfqpoint{5.490039in}{5.490039in}}%
\pgfusepath{clip}%
\pgfsetbuttcap%
\pgfsetroundjoin%
\definecolor{currentfill}{rgb}{0.282327,0.094955,0.417331}%
\pgfsetfillcolor{currentfill}%
\pgfsetfillopacity{0.700000}%
\pgfsetlinewidth{0.000000pt}%
\definecolor{currentstroke}{rgb}{0.000000,0.000000,0.000000}%
\pgfsetstrokecolor{currentstroke}%
\pgfsetdash{}{0pt}%
\pgfpathmoveto{\pgfqpoint{3.488668in}{1.966992in}}%
\pgfpathlineto{\pgfqpoint{3.501910in}{1.957472in}}%
\pgfpathlineto{\pgfqpoint{3.515154in}{1.948092in}}%
\pgfpathlineto{\pgfqpoint{3.528399in}{1.938851in}}%
\pgfpathlineto{\pgfqpoint{3.541646in}{1.929748in}}%
\pgfpathlineto{\pgfqpoint{3.549549in}{1.935679in}}%
\pgfpathlineto{\pgfqpoint{3.557444in}{1.941715in}}%
\pgfpathlineto{\pgfqpoint{3.565332in}{1.947854in}}%
\pgfpathlineto{\pgfqpoint{3.573213in}{1.954095in}}%
\pgfpathlineto{\pgfqpoint{3.559986in}{1.962918in}}%
\pgfpathlineto{\pgfqpoint{3.546761in}{1.971878in}}%
\pgfpathlineto{\pgfqpoint{3.533538in}{1.980978in}}%
\pgfpathlineto{\pgfqpoint{3.520317in}{1.990217in}}%
\pgfpathlineto{\pgfqpoint{3.512416in}{1.984251in}}%
\pgfpathlineto{\pgfqpoint{3.504508in}{1.978390in}}%
\pgfpathlineto{\pgfqpoint{3.496592in}{1.972636in}}%
\pgfpathlineto{\pgfqpoint{3.488668in}{1.966992in}}%
\pgfpathclose%
\pgfusepath{fill}%
\end{pgfscope}%
\begin{pgfscope}%
\pgfpathrectangle{\pgfqpoint{1.254980in}{0.150000in}}{\pgfqpoint{5.490039in}{5.490039in}}%
\pgfusepath{clip}%
\pgfsetbuttcap%
\pgfsetroundjoin%
\definecolor{currentfill}{rgb}{0.231674,0.318106,0.544834}%
\pgfsetfillcolor{currentfill}%
\pgfsetfillopacity{0.700000}%
\pgfsetlinewidth{0.000000pt}%
\definecolor{currentstroke}{rgb}{0.000000,0.000000,0.000000}%
\pgfsetstrokecolor{currentstroke}%
\pgfsetdash{}{0pt}%
\pgfpathmoveto{\pgfqpoint{5.181605in}{2.392257in}}%
\pgfpathlineto{\pgfqpoint{5.195329in}{2.396090in}}%
\pgfpathlineto{\pgfqpoint{5.209065in}{2.400036in}}%
\pgfpathlineto{\pgfqpoint{5.222814in}{2.404095in}}%
\pgfpathlineto{\pgfqpoint{5.236576in}{2.408268in}}%
\pgfpathlineto{\pgfqpoint{5.243888in}{2.418432in}}%
\pgfpathlineto{\pgfqpoint{5.251194in}{2.428545in}}%
\pgfpathlineto{\pgfqpoint{5.258494in}{2.438609in}}%
\pgfpathlineto{\pgfqpoint{5.265788in}{2.448624in}}%
\pgfpathlineto{\pgfqpoint{5.252032in}{2.444446in}}%
\pgfpathlineto{\pgfqpoint{5.238290in}{2.440382in}}%
\pgfpathlineto{\pgfqpoint{5.224561in}{2.436430in}}%
\pgfpathlineto{\pgfqpoint{5.210844in}{2.432593in}}%
\pgfpathlineto{\pgfqpoint{5.203543in}{2.422577in}}%
\pgfpathlineto{\pgfqpoint{5.196236in}{2.412516in}}%
\pgfpathlineto{\pgfqpoint{5.188924in}{2.402409in}}%
\pgfpathlineto{\pgfqpoint{5.181605in}{2.392257in}}%
\pgfpathclose%
\pgfusepath{fill}%
\end{pgfscope}%
\begin{pgfscope}%
\pgfpathrectangle{\pgfqpoint{1.254980in}{0.150000in}}{\pgfqpoint{5.490039in}{5.490039in}}%
\pgfusepath{clip}%
\pgfsetbuttcap%
\pgfsetroundjoin%
\definecolor{currentfill}{rgb}{0.166617,0.463708,0.558119}%
\pgfsetfillcolor{currentfill}%
\pgfsetfillopacity{0.700000}%
\pgfsetlinewidth{0.000000pt}%
\definecolor{currentstroke}{rgb}{0.000000,0.000000,0.000000}%
\pgfsetstrokecolor{currentstroke}%
\pgfsetdash{}{0pt}%
\pgfpathmoveto{\pgfqpoint{2.658451in}{2.817657in}}%
\pgfpathlineto{\pgfqpoint{2.671885in}{2.798033in}}%
\pgfpathlineto{\pgfqpoint{2.685312in}{2.778603in}}%
\pgfpathlineto{\pgfqpoint{2.698731in}{2.759364in}}%
\pgfpathlineto{\pgfqpoint{2.712143in}{2.740316in}}%
\pgfpathlineto{\pgfqpoint{2.720513in}{2.741114in}}%
\pgfpathlineto{\pgfqpoint{2.728870in}{2.742091in}}%
\pgfpathlineto{\pgfqpoint{2.737213in}{2.743246in}}%
\pgfpathlineto{\pgfqpoint{2.745544in}{2.744576in}}%
\pgfpathlineto{\pgfqpoint{2.732168in}{2.763304in}}%
\pgfpathlineto{\pgfqpoint{2.718786in}{2.782221in}}%
\pgfpathlineto{\pgfqpoint{2.705397in}{2.801331in}}%
\pgfpathlineto{\pgfqpoint{2.692000in}{2.820633in}}%
\pgfpathlineto{\pgfqpoint{2.683633in}{2.819617in}}%
\pgfpathlineto{\pgfqpoint{2.675253in}{2.818780in}}%
\pgfpathlineto{\pgfqpoint{2.666859in}{2.818126in}}%
\pgfpathlineto{\pgfqpoint{2.658451in}{2.817657in}}%
\pgfpathclose%
\pgfusepath{fill}%
\end{pgfscope}%
\begin{pgfscope}%
\pgfpathrectangle{\pgfqpoint{1.254980in}{0.150000in}}{\pgfqpoint{5.490039in}{5.490039in}}%
\pgfusepath{clip}%
\pgfsetbuttcap%
\pgfsetroundjoin%
\definecolor{currentfill}{rgb}{0.277941,0.056324,0.381191}%
\pgfsetfillcolor{currentfill}%
\pgfsetfillopacity{0.700000}%
\pgfsetlinewidth{0.000000pt}%
\definecolor{currentstroke}{rgb}{0.000000,0.000000,0.000000}%
\pgfsetstrokecolor{currentstroke}%
\pgfsetdash{}{0pt}%
\pgfpathmoveto{\pgfqpoint{3.679110in}{1.888421in}}%
\pgfpathlineto{\pgfqpoint{3.692360in}{1.880818in}}%
\pgfpathlineto{\pgfqpoint{3.705612in}{1.873347in}}%
\pgfpathlineto{\pgfqpoint{3.718868in}{1.866009in}}%
\pgfpathlineto{\pgfqpoint{3.732127in}{1.858803in}}%
\pgfpathlineto{\pgfqpoint{3.739945in}{1.865949in}}%
\pgfpathlineto{\pgfqpoint{3.747756in}{1.873179in}}%
\pgfpathlineto{\pgfqpoint{3.755561in}{1.880492in}}%
\pgfpathlineto{\pgfqpoint{3.763359in}{1.887885in}}%
\pgfpathlineto{\pgfqpoint{3.750117in}{1.894829in}}%
\pgfpathlineto{\pgfqpoint{3.736878in}{1.901905in}}%
\pgfpathlineto{\pgfqpoint{3.723643in}{1.909114in}}%
\pgfpathlineto{\pgfqpoint{3.710411in}{1.916455in}}%
\pgfpathlineto{\pgfqpoint{3.702596in}{1.909318in}}%
\pgfpathlineto{\pgfqpoint{3.694774in}{1.902265in}}%
\pgfpathlineto{\pgfqpoint{3.686946in}{1.895299in}}%
\pgfpathlineto{\pgfqpoint{3.679110in}{1.888421in}}%
\pgfpathclose%
\pgfusepath{fill}%
\end{pgfscope}%
\begin{pgfscope}%
\pgfpathrectangle{\pgfqpoint{1.254980in}{0.150000in}}{\pgfqpoint{5.490039in}{5.490039in}}%
\pgfusepath{clip}%
\pgfsetbuttcap%
\pgfsetroundjoin%
\definecolor{currentfill}{rgb}{0.277941,0.056324,0.381191}%
\pgfsetfillcolor{currentfill}%
\pgfsetfillopacity{0.700000}%
\pgfsetlinewidth{0.000000pt}%
\definecolor{currentstroke}{rgb}{0.000000,0.000000,0.000000}%
\pgfsetstrokecolor{currentstroke}%
\pgfsetdash{}{0pt}%
\pgfpathmoveto{\pgfqpoint{4.174655in}{1.870486in}}%
\pgfpathlineto{\pgfqpoint{4.187988in}{1.867391in}}%
\pgfpathlineto{\pgfqpoint{4.201327in}{1.864418in}}%
\pgfpathlineto{\pgfqpoint{4.214673in}{1.861566in}}%
\pgfpathlineto{\pgfqpoint{4.228027in}{1.858836in}}%
\pgfpathlineto{\pgfqpoint{4.235661in}{1.868530in}}%
\pgfpathlineto{\pgfqpoint{4.243291in}{1.878252in}}%
\pgfpathlineto{\pgfqpoint{4.250916in}{1.888002in}}%
\pgfpathlineto{\pgfqpoint{4.258536in}{1.897776in}}%
\pgfpathlineto{\pgfqpoint{4.245192in}{1.900310in}}%
\pgfpathlineto{\pgfqpoint{4.231855in}{1.902964in}}%
\pgfpathlineto{\pgfqpoint{4.218526in}{1.905741in}}%
\pgfpathlineto{\pgfqpoint{4.205204in}{1.908639in}}%
\pgfpathlineto{\pgfqpoint{4.197574in}{1.899055in}}%
\pgfpathlineto{\pgfqpoint{4.189940in}{1.889501in}}%
\pgfpathlineto{\pgfqpoint{4.182300in}{1.879977in}}%
\pgfpathlineto{\pgfqpoint{4.174655in}{1.870486in}}%
\pgfpathclose%
\pgfusepath{fill}%
\end{pgfscope}%
\begin{pgfscope}%
\pgfpathrectangle{\pgfqpoint{1.254980in}{0.150000in}}{\pgfqpoint{5.490039in}{5.490039in}}%
\pgfusepath{clip}%
\pgfsetbuttcap%
\pgfsetroundjoin%
\definecolor{currentfill}{rgb}{0.274952,0.037752,0.364543}%
\pgfsetfillcolor{currentfill}%
\pgfsetfillopacity{0.700000}%
\pgfsetlinewidth{0.000000pt}%
\definecolor{currentstroke}{rgb}{0.000000,0.000000,0.000000}%
\pgfsetstrokecolor{currentstroke}%
\pgfsetdash{}{0pt}%
\pgfpathmoveto{\pgfqpoint{3.953531in}{1.847963in}}%
\pgfpathlineto{\pgfqpoint{3.966815in}{1.842939in}}%
\pgfpathlineto{\pgfqpoint{3.980105in}{1.838040in}}%
\pgfpathlineto{\pgfqpoint{3.993400in}{1.833267in}}%
\pgfpathlineto{\pgfqpoint{4.006700in}{1.828619in}}%
\pgfpathlineto{\pgfqpoint{4.014410in}{1.837326in}}%
\pgfpathlineto{\pgfqpoint{4.022115in}{1.846087in}}%
\pgfpathlineto{\pgfqpoint{4.029814in}{1.854899in}}%
\pgfpathlineto{\pgfqpoint{4.037508in}{1.863760in}}%
\pgfpathlineto{\pgfqpoint{4.024220in}{1.868180in}}%
\pgfpathlineto{\pgfqpoint{4.010938in}{1.872724in}}%
\pgfpathlineto{\pgfqpoint{3.997661in}{1.877394in}}%
\pgfpathlineto{\pgfqpoint{3.984390in}{1.882189in}}%
\pgfpathlineto{\pgfqpoint{3.976683in}{1.873550in}}%
\pgfpathlineto{\pgfqpoint{3.968972in}{1.864965in}}%
\pgfpathlineto{\pgfqpoint{3.961254in}{1.856436in}}%
\pgfpathlineto{\pgfqpoint{3.953531in}{1.847963in}}%
\pgfpathclose%
\pgfusepath{fill}%
\end{pgfscope}%
\begin{pgfscope}%
\pgfpathrectangle{\pgfqpoint{1.254980in}{0.150000in}}{\pgfqpoint{5.490039in}{5.490039in}}%
\pgfusepath{clip}%
\pgfsetbuttcap%
\pgfsetroundjoin%
\definecolor{currentfill}{rgb}{0.279574,0.170599,0.479997}%
\pgfsetfillcolor{currentfill}%
\pgfsetfillopacity{0.700000}%
\pgfsetlinewidth{0.000000pt}%
\definecolor{currentstroke}{rgb}{0.000000,0.000000,0.000000}%
\pgfsetstrokecolor{currentstroke}%
\pgfsetdash{}{0pt}%
\pgfpathmoveto{\pgfqpoint{3.244738in}{2.122128in}}%
\pgfpathlineto{\pgfqpoint{3.257999in}{2.109990in}}%
\pgfpathlineto{\pgfqpoint{3.271259in}{2.098003in}}%
\pgfpathlineto{\pgfqpoint{3.284518in}{2.086165in}}%
\pgfpathlineto{\pgfqpoint{3.297777in}{2.074476in}}%
\pgfpathlineto{\pgfqpoint{3.305805in}{2.078743in}}%
\pgfpathlineto{\pgfqpoint{3.313824in}{2.083142in}}%
\pgfpathlineto{\pgfqpoint{3.321834in}{2.087670in}}%
\pgfpathlineto{\pgfqpoint{3.329836in}{2.092325in}}%
\pgfpathlineto{\pgfqpoint{3.316602in}{2.103714in}}%
\pgfpathlineto{\pgfqpoint{3.303368in}{2.115251in}}%
\pgfpathlineto{\pgfqpoint{3.290134in}{2.126937in}}%
\pgfpathlineto{\pgfqpoint{3.276899in}{2.138773in}}%
\pgfpathlineto{\pgfqpoint{3.268873in}{2.134413in}}%
\pgfpathlineto{\pgfqpoint{3.260838in}{2.130184in}}%
\pgfpathlineto{\pgfqpoint{3.252793in}{2.126088in}}%
\pgfpathlineto{\pgfqpoint{3.244738in}{2.122128in}}%
\pgfpathclose%
\pgfusepath{fill}%
\end{pgfscope}%
\begin{pgfscope}%
\pgfpathrectangle{\pgfqpoint{1.254980in}{0.150000in}}{\pgfqpoint{5.490039in}{5.490039in}}%
\pgfusepath{clip}%
\pgfsetbuttcap%
\pgfsetroundjoin%
\definecolor{currentfill}{rgb}{0.162142,0.474838,0.558140}%
\pgfsetfillcolor{currentfill}%
\pgfsetfillopacity{0.700000}%
\pgfsetlinewidth{0.000000pt}%
\definecolor{currentstroke}{rgb}{0.000000,0.000000,0.000000}%
\pgfsetstrokecolor{currentstroke}%
\pgfsetdash{}{0pt}%
\pgfpathmoveto{\pgfqpoint{5.716067in}{2.776381in}}%
\pgfpathlineto{\pgfqpoint{5.730067in}{2.782601in}}%
\pgfpathlineto{\pgfqpoint{5.744082in}{2.788932in}}%
\pgfpathlineto{\pgfqpoint{5.758112in}{2.795376in}}%
\pgfpathlineto{\pgfqpoint{5.772158in}{2.801931in}}%
\pgfpathlineto{\pgfqpoint{5.779242in}{2.810277in}}%
\pgfpathlineto{\pgfqpoint{5.786320in}{2.818569in}}%
\pgfpathlineto{\pgfqpoint{5.793390in}{2.826808in}}%
\pgfpathlineto{\pgfqpoint{5.800455in}{2.834995in}}%
\pgfpathlineto{\pgfqpoint{5.786420in}{2.828552in}}%
\pgfpathlineto{\pgfqpoint{5.772402in}{2.822221in}}%
\pgfpathlineto{\pgfqpoint{5.758399in}{2.816002in}}%
\pgfpathlineto{\pgfqpoint{5.744411in}{2.809894in}}%
\pgfpathlineto{\pgfqpoint{5.737334in}{2.801589in}}%
\pgfpathlineto{\pgfqpoint{5.730252in}{2.793235in}}%
\pgfpathlineto{\pgfqpoint{5.723162in}{2.784833in}}%
\pgfpathlineto{\pgfqpoint{5.716067in}{2.776381in}}%
\pgfpathclose%
\pgfusepath{fill}%
\end{pgfscope}%
\begin{pgfscope}%
\pgfpathrectangle{\pgfqpoint{1.254980in}{0.150000in}}{\pgfqpoint{5.490039in}{5.490039in}}%
\pgfusepath{clip}%
\pgfsetbuttcap%
\pgfsetroundjoin%
\definecolor{currentfill}{rgb}{0.220057,0.343307,0.549413}%
\pgfsetfillcolor{currentfill}%
\pgfsetfillopacity{0.700000}%
\pgfsetlinewidth{0.000000pt}%
\definecolor{currentstroke}{rgb}{0.000000,0.000000,0.000000}%
\pgfsetstrokecolor{currentstroke}%
\pgfsetdash{}{0pt}%
\pgfpathmoveto{\pgfqpoint{5.265788in}{2.448624in}}%
\pgfpathlineto{\pgfqpoint{5.279556in}{2.452915in}}%
\pgfpathlineto{\pgfqpoint{5.293338in}{2.457319in}}%
\pgfpathlineto{\pgfqpoint{5.307133in}{2.461836in}}%
\pgfpathlineto{\pgfqpoint{5.320942in}{2.466467in}}%
\pgfpathlineto{\pgfqpoint{5.328224in}{2.476426in}}%
\pgfpathlineto{\pgfqpoint{5.335499in}{2.486334in}}%
\pgfpathlineto{\pgfqpoint{5.342769in}{2.496188in}}%
\pgfpathlineto{\pgfqpoint{5.350032in}{2.505991in}}%
\pgfpathlineto{\pgfqpoint{5.336231in}{2.501372in}}%
\pgfpathlineto{\pgfqpoint{5.322443in}{2.496867in}}%
\pgfpathlineto{\pgfqpoint{5.308668in}{2.492474in}}%
\pgfpathlineto{\pgfqpoint{5.294907in}{2.488195in}}%
\pgfpathlineto{\pgfqpoint{5.287636in}{2.478374in}}%
\pgfpathlineto{\pgfqpoint{5.280359in}{2.468506in}}%
\pgfpathlineto{\pgfqpoint{5.273076in}{2.458589in}}%
\pgfpathlineto{\pgfqpoint{5.265788in}{2.448624in}}%
\pgfpathclose%
\pgfusepath{fill}%
\end{pgfscope}%
\begin{pgfscope}%
\pgfpathrectangle{\pgfqpoint{1.254980in}{0.150000in}}{\pgfqpoint{5.490039in}{5.490039in}}%
\pgfusepath{clip}%
\pgfsetbuttcap%
\pgfsetroundjoin%
\definecolor{currentfill}{rgb}{0.154815,0.493313,0.557840}%
\pgfsetfillcolor{currentfill}%
\pgfsetfillopacity{0.700000}%
\pgfsetlinewidth{0.000000pt}%
\definecolor{currentstroke}{rgb}{0.000000,0.000000,0.000000}%
\pgfsetstrokecolor{currentstroke}%
\pgfsetdash{}{0pt}%
\pgfpathmoveto{\pgfqpoint{2.604638in}{2.898111in}}%
\pgfpathlineto{\pgfqpoint{2.618104in}{2.877701in}}%
\pgfpathlineto{\pgfqpoint{2.631561in}{2.857490in}}%
\pgfpathlineto{\pgfqpoint{2.645010in}{2.837476in}}%
\pgfpathlineto{\pgfqpoint{2.658451in}{2.817657in}}%
\pgfpathlineto{\pgfqpoint{2.666859in}{2.818126in}}%
\pgfpathlineto{\pgfqpoint{2.675253in}{2.818780in}}%
\pgfpathlineto{\pgfqpoint{2.683633in}{2.819617in}}%
\pgfpathlineto{\pgfqpoint{2.692000in}{2.820633in}}%
\pgfpathlineto{\pgfqpoint{2.678596in}{2.840128in}}%
\pgfpathlineto{\pgfqpoint{2.665185in}{2.859819in}}%
\pgfpathlineto{\pgfqpoint{2.651766in}{2.879707in}}%
\pgfpathlineto{\pgfqpoint{2.638340in}{2.899792in}}%
\pgfpathlineto{\pgfqpoint{2.629935in}{2.899093in}}%
\pgfpathlineto{\pgfqpoint{2.621517in}{2.898578in}}%
\pgfpathlineto{\pgfqpoint{2.613085in}{2.898250in}}%
\pgfpathlineto{\pgfqpoint{2.604638in}{2.898111in}}%
\pgfpathclose%
\pgfusepath{fill}%
\end{pgfscope}%
\begin{pgfscope}%
\pgfpathrectangle{\pgfqpoint{1.254980in}{0.150000in}}{\pgfqpoint{5.490039in}{5.490039in}}%
\pgfusepath{clip}%
\pgfsetbuttcap%
\pgfsetroundjoin%
\definecolor{currentfill}{rgb}{0.281887,0.150881,0.465405}%
\pgfsetfillcolor{currentfill}%
\pgfsetfillopacity{0.700000}%
\pgfsetlinewidth{0.000000pt}%
\definecolor{currentstroke}{rgb}{0.000000,0.000000,0.000000}%
\pgfsetstrokecolor{currentstroke}%
\pgfsetdash{}{0pt}%
\pgfpathmoveto{\pgfqpoint{3.297777in}{2.074476in}}%
\pgfpathlineto{\pgfqpoint{3.311035in}{2.062934in}}%
\pgfpathlineto{\pgfqpoint{3.324293in}{2.051540in}}%
\pgfpathlineto{\pgfqpoint{3.337551in}{2.040293in}}%
\pgfpathlineto{\pgfqpoint{3.350809in}{2.029192in}}%
\pgfpathlineto{\pgfqpoint{3.358813in}{2.033765in}}%
\pgfpathlineto{\pgfqpoint{3.366807in}{2.038466in}}%
\pgfpathlineto{\pgfqpoint{3.374793in}{2.043292in}}%
\pgfpathlineto{\pgfqpoint{3.382771in}{2.048240in}}%
\pgfpathlineto{\pgfqpoint{3.369537in}{2.059042in}}%
\pgfpathlineto{\pgfqpoint{3.356303in}{2.069990in}}%
\pgfpathlineto{\pgfqpoint{3.343069in}{2.081084in}}%
\pgfpathlineto{\pgfqpoint{3.329836in}{2.092325in}}%
\pgfpathlineto{\pgfqpoint{3.321834in}{2.087670in}}%
\pgfpathlineto{\pgfqpoint{3.313824in}{2.083142in}}%
\pgfpathlineto{\pgfqpoint{3.305805in}{2.078743in}}%
\pgfpathlineto{\pgfqpoint{3.297777in}{2.074476in}}%
\pgfpathclose%
\pgfusepath{fill}%
\end{pgfscope}%
\begin{pgfscope}%
\pgfpathrectangle{\pgfqpoint{1.254980in}{0.150000in}}{\pgfqpoint{5.490039in}{5.490039in}}%
\pgfusepath{clip}%
\pgfsetbuttcap%
\pgfsetroundjoin%
\definecolor{currentfill}{rgb}{0.281412,0.155834,0.469201}%
\pgfsetfillcolor{currentfill}%
\pgfsetfillopacity{0.700000}%
\pgfsetlinewidth{0.000000pt}%
\definecolor{currentstroke}{rgb}{0.000000,0.000000,0.000000}%
\pgfsetstrokecolor{currentstroke}%
\pgfsetdash{}{0pt}%
\pgfpathmoveto{\pgfqpoint{4.647711in}{2.043609in}}%
\pgfpathlineto{\pgfqpoint{4.661203in}{2.044196in}}%
\pgfpathlineto{\pgfqpoint{4.674706in}{2.044900in}}%
\pgfpathlineto{\pgfqpoint{4.688219in}{2.045720in}}%
\pgfpathlineto{\pgfqpoint{4.701742in}{2.046657in}}%
\pgfpathlineto{\pgfqpoint{4.709235in}{2.057423in}}%
\pgfpathlineto{\pgfqpoint{4.716723in}{2.068171in}}%
\pgfpathlineto{\pgfqpoint{4.724207in}{2.078900in}}%
\pgfpathlineto{\pgfqpoint{4.731685in}{2.089610in}}%
\pgfpathlineto{\pgfqpoint{4.718168in}{2.088556in}}%
\pgfpathlineto{\pgfqpoint{4.704661in}{2.087618in}}%
\pgfpathlineto{\pgfqpoint{4.691165in}{2.086796in}}%
\pgfpathlineto{\pgfqpoint{4.677678in}{2.086091in}}%
\pgfpathlineto{\pgfqpoint{4.670194in}{2.075492in}}%
\pgfpathlineto{\pgfqpoint{4.662704in}{2.064879in}}%
\pgfpathlineto{\pgfqpoint{4.655210in}{2.054251in}}%
\pgfpathlineto{\pgfqpoint{4.647711in}{2.043609in}}%
\pgfpathclose%
\pgfusepath{fill}%
\end{pgfscope}%
\begin{pgfscope}%
\pgfpathrectangle{\pgfqpoint{1.254980in}{0.150000in}}{\pgfqpoint{5.490039in}{5.490039in}}%
\pgfusepath{clip}%
\pgfsetbuttcap%
\pgfsetroundjoin%
\definecolor{currentfill}{rgb}{0.153364,0.497000,0.557724}%
\pgfsetfillcolor{currentfill}%
\pgfsetfillopacity{0.700000}%
\pgfsetlinewidth{0.000000pt}%
\definecolor{currentstroke}{rgb}{0.000000,0.000000,0.000000}%
\pgfsetstrokecolor{currentstroke}%
\pgfsetdash{}{0pt}%
\pgfpathmoveto{\pgfqpoint{5.800455in}{2.834995in}}%
\pgfpathlineto{\pgfqpoint{5.814504in}{2.841549in}}%
\pgfpathlineto{\pgfqpoint{5.828569in}{2.848215in}}%
\pgfpathlineto{\pgfqpoint{5.842650in}{2.854992in}}%
\pgfpathlineto{\pgfqpoint{5.856747in}{2.861881in}}%
\pgfpathlineto{\pgfqpoint{5.863792in}{2.869895in}}%
\pgfpathlineto{\pgfqpoint{5.870830in}{2.877856in}}%
\pgfpathlineto{\pgfqpoint{5.877862in}{2.885765in}}%
\pgfpathlineto{\pgfqpoint{5.884886in}{2.893623in}}%
\pgfpathlineto{\pgfqpoint{5.870803in}{2.886864in}}%
\pgfpathlineto{\pgfqpoint{5.856735in}{2.880216in}}%
\pgfpathlineto{\pgfqpoint{5.842682in}{2.873680in}}%
\pgfpathlineto{\pgfqpoint{5.828645in}{2.867255in}}%
\pgfpathlineto{\pgfqpoint{5.821607in}{2.859261in}}%
\pgfpathlineto{\pgfqpoint{5.814563in}{2.851220in}}%
\pgfpathlineto{\pgfqpoint{5.807512in}{2.843132in}}%
\pgfpathlineto{\pgfqpoint{5.800455in}{2.834995in}}%
\pgfpathclose%
\pgfusepath{fill}%
\end{pgfscope}%
\begin{pgfscope}%
\pgfpathrectangle{\pgfqpoint{1.254980in}{0.150000in}}{\pgfqpoint{5.490039in}{5.490039in}}%
\pgfusepath{clip}%
\pgfsetbuttcap%
\pgfsetroundjoin%
\definecolor{currentfill}{rgb}{0.283072,0.130895,0.449241}%
\pgfsetfillcolor{currentfill}%
\pgfsetfillopacity{0.700000}%
\pgfsetlinewidth{0.000000pt}%
\definecolor{currentstroke}{rgb}{0.000000,0.000000,0.000000}%
\pgfsetstrokecolor{currentstroke}%
\pgfsetdash{}{0pt}%
\pgfpathmoveto{\pgfqpoint{4.563765in}{2.000270in}}%
\pgfpathlineto{\pgfqpoint{4.577226in}{2.000256in}}%
\pgfpathlineto{\pgfqpoint{4.590697in}{2.000359in}}%
\pgfpathlineto{\pgfqpoint{4.604176in}{2.000580in}}%
\pgfpathlineto{\pgfqpoint{4.617666in}{2.000917in}}%
\pgfpathlineto{\pgfqpoint{4.625184in}{2.011607in}}%
\pgfpathlineto{\pgfqpoint{4.632698in}{2.022286in}}%
\pgfpathlineto{\pgfqpoint{4.640207in}{2.032954in}}%
\pgfpathlineto{\pgfqpoint{4.647711in}{2.043609in}}%
\pgfpathlineto{\pgfqpoint{4.634227in}{2.043138in}}%
\pgfpathlineto{\pgfqpoint{4.620754in}{2.042784in}}%
\pgfpathlineto{\pgfqpoint{4.607290in}{2.042546in}}%
\pgfpathlineto{\pgfqpoint{4.593836in}{2.042426in}}%
\pgfpathlineto{\pgfqpoint{4.586326in}{2.031899in}}%
\pgfpathlineto{\pgfqpoint{4.578810in}{2.021363in}}%
\pgfpathlineto{\pgfqpoint{4.571290in}{2.010820in}}%
\pgfpathlineto{\pgfqpoint{4.563765in}{2.000270in}}%
\pgfpathclose%
\pgfusepath{fill}%
\end{pgfscope}%
\begin{pgfscope}%
\pgfpathrectangle{\pgfqpoint{1.254980in}{0.150000in}}{\pgfqpoint{5.490039in}{5.490039in}}%
\pgfusepath{clip}%
\pgfsetbuttcap%
\pgfsetroundjoin%
\definecolor{currentfill}{rgb}{0.276022,0.044167,0.370164}%
\pgfsetfillcolor{currentfill}%
\pgfsetfillopacity{0.700000}%
\pgfsetlinewidth{0.000000pt}%
\definecolor{currentstroke}{rgb}{0.000000,0.000000,0.000000}%
\pgfsetstrokecolor{currentstroke}%
\pgfsetdash{}{0pt}%
\pgfpathmoveto{\pgfqpoint{4.090717in}{1.847329in}}%
\pgfpathlineto{\pgfqpoint{4.104035in}{1.843531in}}%
\pgfpathlineto{\pgfqpoint{4.117358in}{1.839855in}}%
\pgfpathlineto{\pgfqpoint{4.130688in}{1.836303in}}%
\pgfpathlineto{\pgfqpoint{4.144025in}{1.832873in}}%
\pgfpathlineto{\pgfqpoint{4.151690in}{1.842220in}}%
\pgfpathlineto{\pgfqpoint{4.159350in}{1.851606in}}%
\pgfpathlineto{\pgfqpoint{4.167005in}{1.861028in}}%
\pgfpathlineto{\pgfqpoint{4.174655in}{1.870486in}}%
\pgfpathlineto{\pgfqpoint{4.161329in}{1.873703in}}%
\pgfpathlineto{\pgfqpoint{4.148010in}{1.877043in}}%
\pgfpathlineto{\pgfqpoint{4.134697in}{1.880505in}}%
\pgfpathlineto{\pgfqpoint{4.121391in}{1.884091in}}%
\pgfpathlineto{\pgfqpoint{4.113731in}{1.874840in}}%
\pgfpathlineto{\pgfqpoint{4.106065in}{1.865628in}}%
\pgfpathlineto{\pgfqpoint{4.098394in}{1.856457in}}%
\pgfpathlineto{\pgfqpoint{4.090717in}{1.847329in}}%
\pgfpathclose%
\pgfusepath{fill}%
\end{pgfscope}%
\begin{pgfscope}%
\pgfpathrectangle{\pgfqpoint{1.254980in}{0.150000in}}{\pgfqpoint{5.490039in}{5.490039in}}%
\pgfusepath{clip}%
\pgfsetbuttcap%
\pgfsetroundjoin%
\definecolor{currentfill}{rgb}{0.281446,0.084320,0.407414}%
\pgfsetfillcolor{currentfill}%
\pgfsetfillopacity{0.700000}%
\pgfsetlinewidth{0.000000pt}%
\definecolor{currentstroke}{rgb}{0.000000,0.000000,0.000000}%
\pgfsetstrokecolor{currentstroke}%
\pgfsetdash{}{0pt}%
\pgfpathmoveto{\pgfqpoint{3.541646in}{1.929748in}}%
\pgfpathlineto{\pgfqpoint{3.554895in}{1.920783in}}%
\pgfpathlineto{\pgfqpoint{3.568146in}{1.911955in}}%
\pgfpathlineto{\pgfqpoint{3.581399in}{1.903264in}}%
\pgfpathlineto{\pgfqpoint{3.594654in}{1.894709in}}%
\pgfpathlineto{\pgfqpoint{3.602536in}{1.900926in}}%
\pgfpathlineto{\pgfqpoint{3.610412in}{1.907243in}}%
\pgfpathlineto{\pgfqpoint{3.618280in}{1.913660in}}%
\pgfpathlineto{\pgfqpoint{3.626141in}{1.920174in}}%
\pgfpathlineto{\pgfqpoint{3.612905in}{1.928450in}}%
\pgfpathlineto{\pgfqpoint{3.599672in}{1.936862in}}%
\pgfpathlineto{\pgfqpoint{3.586441in}{1.945410in}}%
\pgfpathlineto{\pgfqpoint{3.573213in}{1.954095in}}%
\pgfpathlineto{\pgfqpoint{3.565332in}{1.947854in}}%
\pgfpathlineto{\pgfqpoint{3.557444in}{1.941715in}}%
\pgfpathlineto{\pgfqpoint{3.549549in}{1.935679in}}%
\pgfpathlineto{\pgfqpoint{3.541646in}{1.929748in}}%
\pgfpathclose%
\pgfusepath{fill}%
\end{pgfscope}%
\begin{pgfscope}%
\pgfpathrectangle{\pgfqpoint{1.254980in}{0.150000in}}{\pgfqpoint{5.490039in}{5.490039in}}%
\pgfusepath{clip}%
\pgfsetbuttcap%
\pgfsetroundjoin%
\definecolor{currentfill}{rgb}{0.206756,0.371758,0.553117}%
\pgfsetfillcolor{currentfill}%
\pgfsetfillopacity{0.700000}%
\pgfsetlinewidth{0.000000pt}%
\definecolor{currentstroke}{rgb}{0.000000,0.000000,0.000000}%
\pgfsetstrokecolor{currentstroke}%
\pgfsetdash{}{0pt}%
\pgfpathmoveto{\pgfqpoint{5.350032in}{2.505991in}}%
\pgfpathlineto{\pgfqpoint{5.363847in}{2.510722in}}%
\pgfpathlineto{\pgfqpoint{5.377676in}{2.515567in}}%
\pgfpathlineto{\pgfqpoint{5.391519in}{2.520524in}}%
\pgfpathlineto{\pgfqpoint{5.405375in}{2.525594in}}%
\pgfpathlineto{\pgfqpoint{5.412625in}{2.535323in}}%
\pgfpathlineto{\pgfqpoint{5.419869in}{2.544997in}}%
\pgfpathlineto{\pgfqpoint{5.427107in}{2.554616in}}%
\pgfpathlineto{\pgfqpoint{5.434339in}{2.564181in}}%
\pgfpathlineto{\pgfqpoint{5.420490in}{2.559140in}}%
\pgfpathlineto{\pgfqpoint{5.406656in}{2.554211in}}%
\pgfpathlineto{\pgfqpoint{5.392835in}{2.549395in}}%
\pgfpathlineto{\pgfqpoint{5.379028in}{2.544691in}}%
\pgfpathlineto{\pgfqpoint{5.371788in}{2.535092in}}%
\pgfpathlineto{\pgfqpoint{5.364542in}{2.525442in}}%
\pgfpathlineto{\pgfqpoint{5.357290in}{2.515742in}}%
\pgfpathlineto{\pgfqpoint{5.350032in}{2.505991in}}%
\pgfpathclose%
\pgfusepath{fill}%
\end{pgfscope}%
\begin{pgfscope}%
\pgfpathrectangle{\pgfqpoint{1.254980in}{0.150000in}}{\pgfqpoint{5.490039in}{5.490039in}}%
\pgfusepath{clip}%
\pgfsetbuttcap%
\pgfsetroundjoin%
\definecolor{currentfill}{rgb}{0.278012,0.180367,0.486697}%
\pgfsetfillcolor{currentfill}%
\pgfsetfillopacity{0.700000}%
\pgfsetlinewidth{0.000000pt}%
\definecolor{currentstroke}{rgb}{0.000000,0.000000,0.000000}%
\pgfsetstrokecolor{currentstroke}%
\pgfsetdash{}{0pt}%
\pgfpathmoveto{\pgfqpoint{4.731685in}{2.089610in}}%
\pgfpathlineto{\pgfqpoint{4.745212in}{2.090780in}}%
\pgfpathlineto{\pgfqpoint{4.758750in}{2.092066in}}%
\pgfpathlineto{\pgfqpoint{4.772298in}{2.093468in}}%
\pgfpathlineto{\pgfqpoint{4.785857in}{2.094985in}}%
\pgfpathlineto{\pgfqpoint{4.793325in}{2.105783in}}%
\pgfpathlineto{\pgfqpoint{4.800787in}{2.116556in}}%
\pgfpathlineto{\pgfqpoint{4.808245in}{2.127303in}}%
\pgfpathlineto{\pgfqpoint{4.815698in}{2.138025in}}%
\pgfpathlineto{\pgfqpoint{4.802145in}{2.136406in}}%
\pgfpathlineto{\pgfqpoint{4.788603in}{2.134903in}}%
\pgfpathlineto{\pgfqpoint{4.775071in}{2.133515in}}%
\pgfpathlineto{\pgfqpoint{4.761550in}{2.132243in}}%
\pgfpathlineto{\pgfqpoint{4.754091in}{2.121617in}}%
\pgfpathlineto{\pgfqpoint{4.746627in}{2.110969in}}%
\pgfpathlineto{\pgfqpoint{4.739159in}{2.100300in}}%
\pgfpathlineto{\pgfqpoint{4.731685in}{2.089610in}}%
\pgfpathclose%
\pgfusepath{fill}%
\end{pgfscope}%
\begin{pgfscope}%
\pgfpathrectangle{\pgfqpoint{1.254980in}{0.150000in}}{\pgfqpoint{5.490039in}{5.490039in}}%
\pgfusepath{clip}%
\pgfsetbuttcap%
\pgfsetroundjoin%
\definecolor{currentfill}{rgb}{0.144759,0.519093,0.556572}%
\pgfsetfillcolor{currentfill}%
\pgfsetfillopacity{0.700000}%
\pgfsetlinewidth{0.000000pt}%
\definecolor{currentstroke}{rgb}{0.000000,0.000000,0.000000}%
\pgfsetstrokecolor{currentstroke}%
\pgfsetdash{}{0pt}%
\pgfpathmoveto{\pgfqpoint{5.884886in}{2.893623in}}%
\pgfpathlineto{\pgfqpoint{5.898986in}{2.900494in}}%
\pgfpathlineto{\pgfqpoint{5.913102in}{2.907476in}}%
\pgfpathlineto{\pgfqpoint{5.927234in}{2.914569in}}%
\pgfpathlineto{\pgfqpoint{5.934242in}{2.922274in}}%
\pgfpathlineto{\pgfqpoint{5.941243in}{2.929928in}}%
\pgfpathlineto{\pgfqpoint{5.948237in}{2.937534in}}%
\pgfpathlineto{\pgfqpoint{5.955225in}{2.945092in}}%
\pgfpathlineto{\pgfqpoint{5.941107in}{2.938146in}}%
\pgfpathlineto{\pgfqpoint{5.927005in}{2.931311in}}%
\pgfpathlineto{\pgfqpoint{5.912919in}{2.924587in}}%
\pgfpathlineto{\pgfqpoint{5.905921in}{2.916913in}}%
\pgfpathlineto{\pgfqpoint{5.898916in}{2.909196in}}%
\pgfpathlineto{\pgfqpoint{5.891905in}{2.901433in}}%
\pgfpathlineto{\pgfqpoint{5.884886in}{2.893623in}}%
\pgfpathclose%
\pgfusepath{fill}%
\end{pgfscope}%
\begin{pgfscope}%
\pgfpathrectangle{\pgfqpoint{1.254980in}{0.150000in}}{\pgfqpoint{5.490039in}{5.490039in}}%
\pgfusepath{clip}%
\pgfsetbuttcap%
\pgfsetroundjoin%
\definecolor{currentfill}{rgb}{0.283091,0.110553,0.431554}%
\pgfsetfillcolor{currentfill}%
\pgfsetfillopacity{0.700000}%
\pgfsetlinewidth{0.000000pt}%
\definecolor{currentstroke}{rgb}{0.000000,0.000000,0.000000}%
\pgfsetstrokecolor{currentstroke}%
\pgfsetdash{}{0pt}%
\pgfpathmoveto{\pgfqpoint{4.479838in}{1.959850in}}%
\pgfpathlineto{\pgfqpoint{4.493270in}{1.959217in}}%
\pgfpathlineto{\pgfqpoint{4.506710in}{1.958701in}}%
\pgfpathlineto{\pgfqpoint{4.520159in}{1.958303in}}%
\pgfpathlineto{\pgfqpoint{4.533618in}{1.958022in}}%
\pgfpathlineto{\pgfqpoint{4.541162in}{1.968589in}}%
\pgfpathlineto{\pgfqpoint{4.548701in}{1.979153in}}%
\pgfpathlineto{\pgfqpoint{4.556236in}{1.989714in}}%
\pgfpathlineto{\pgfqpoint{4.563765in}{2.000270in}}%
\pgfpathlineto{\pgfqpoint{4.550314in}{2.000400in}}%
\pgfpathlineto{\pgfqpoint{4.536871in}{2.000649in}}%
\pgfpathlineto{\pgfqpoint{4.523438in}{2.001015in}}%
\pgfpathlineto{\pgfqpoint{4.510014in}{2.001499in}}%
\pgfpathlineto{\pgfqpoint{4.502477in}{1.991087in}}%
\pgfpathlineto{\pgfqpoint{4.494935in}{1.980674in}}%
\pgfpathlineto{\pgfqpoint{4.487389in}{1.970261in}}%
\pgfpathlineto{\pgfqpoint{4.479838in}{1.959850in}}%
\pgfpathclose%
\pgfusepath{fill}%
\end{pgfscope}%
\begin{pgfscope}%
\pgfpathrectangle{\pgfqpoint{1.254980in}{0.150000in}}{\pgfqpoint{5.490039in}{5.490039in}}%
\pgfusepath{clip}%
\pgfsetbuttcap%
\pgfsetroundjoin%
\definecolor{currentfill}{rgb}{0.273006,0.204520,0.501721}%
\pgfsetfillcolor{currentfill}%
\pgfsetfillopacity{0.700000}%
\pgfsetlinewidth{0.000000pt}%
\definecolor{currentstroke}{rgb}{0.000000,0.000000,0.000000}%
\pgfsetstrokecolor{currentstroke}%
\pgfsetdash{}{0pt}%
\pgfpathmoveto{\pgfqpoint{4.815698in}{2.138025in}}%
\pgfpathlineto{\pgfqpoint{4.829262in}{2.139760in}}%
\pgfpathlineto{\pgfqpoint{4.842837in}{2.141609in}}%
\pgfpathlineto{\pgfqpoint{4.856423in}{2.143574in}}%
\pgfpathlineto{\pgfqpoint{4.870019in}{2.145654in}}%
\pgfpathlineto{\pgfqpoint{4.877462in}{2.156441in}}%
\pgfpathlineto{\pgfqpoint{4.884899in}{2.167198in}}%
\pgfpathlineto{\pgfqpoint{4.892331in}{2.177923in}}%
\pgfpathlineto{\pgfqpoint{4.899758in}{2.188616in}}%
\pgfpathlineto{\pgfqpoint{4.886167in}{2.186450in}}%
\pgfpathlineto{\pgfqpoint{4.872587in}{2.184399in}}%
\pgfpathlineto{\pgfqpoint{4.859018in}{2.182464in}}%
\pgfpathlineto{\pgfqpoint{4.845460in}{2.180643in}}%
\pgfpathlineto{\pgfqpoint{4.838027in}{2.170030in}}%
\pgfpathlineto{\pgfqpoint{4.830589in}{2.159389in}}%
\pgfpathlineto{\pgfqpoint{4.823146in}{2.148721in}}%
\pgfpathlineto{\pgfqpoint{4.815698in}{2.138025in}}%
\pgfpathclose%
\pgfusepath{fill}%
\end{pgfscope}%
\begin{pgfscope}%
\pgfpathrectangle{\pgfqpoint{1.254980in}{0.150000in}}{\pgfqpoint{5.490039in}{5.490039in}}%
\pgfusepath{clip}%
\pgfsetbuttcap%
\pgfsetroundjoin%
\definecolor{currentfill}{rgb}{0.282327,0.094955,0.417331}%
\pgfsetfillcolor{currentfill}%
\pgfsetfillopacity{0.700000}%
\pgfsetlinewidth{0.000000pt}%
\definecolor{currentstroke}{rgb}{0.000000,0.000000,0.000000}%
\pgfsetstrokecolor{currentstroke}%
\pgfsetdash{}{0pt}%
\pgfpathmoveto{\pgfqpoint{4.395916in}{1.922618in}}%
\pgfpathlineto{\pgfqpoint{4.409321in}{1.921345in}}%
\pgfpathlineto{\pgfqpoint{4.422734in}{1.920192in}}%
\pgfpathlineto{\pgfqpoint{4.436156in}{1.919156in}}%
\pgfpathlineto{\pgfqpoint{4.449586in}{1.918239in}}%
\pgfpathlineto{\pgfqpoint{4.457156in}{1.928635in}}%
\pgfpathlineto{\pgfqpoint{4.464721in}{1.939036in}}%
\pgfpathlineto{\pgfqpoint{4.472282in}{1.949441in}}%
\pgfpathlineto{\pgfqpoint{4.479838in}{1.959850in}}%
\pgfpathlineto{\pgfqpoint{4.466415in}{1.960602in}}%
\pgfpathlineto{\pgfqpoint{4.453001in}{1.961472in}}%
\pgfpathlineto{\pgfqpoint{4.439596in}{1.962460in}}%
\pgfpathlineto{\pgfqpoint{4.426199in}{1.963567in}}%
\pgfpathlineto{\pgfqpoint{4.418635in}{1.953318in}}%
\pgfpathlineto{\pgfqpoint{4.411067in}{1.943075in}}%
\pgfpathlineto{\pgfqpoint{4.403494in}{1.932842in}}%
\pgfpathlineto{\pgfqpoint{4.395916in}{1.922618in}}%
\pgfpathclose%
\pgfusepath{fill}%
\end{pgfscope}%
\begin{pgfscope}%
\pgfpathrectangle{\pgfqpoint{1.254980in}{0.150000in}}{\pgfqpoint{5.490039in}{5.490039in}}%
\pgfusepath{clip}%
\pgfsetbuttcap%
\pgfsetroundjoin%
\definecolor{currentfill}{rgb}{0.274952,0.037752,0.364543}%
\pgfsetfillcolor{currentfill}%
\pgfsetfillopacity{0.700000}%
\pgfsetlinewidth{0.000000pt}%
\definecolor{currentstroke}{rgb}{0.000000,0.000000,0.000000}%
\pgfsetstrokecolor{currentstroke}%
\pgfsetdash{}{0pt}%
\pgfpathmoveto{\pgfqpoint{3.869437in}{1.837026in}}%
\pgfpathlineto{\pgfqpoint{3.882716in}{1.831249in}}%
\pgfpathlineto{\pgfqpoint{3.895999in}{1.825599in}}%
\pgfpathlineto{\pgfqpoint{3.909287in}{1.820076in}}%
\pgfpathlineto{\pgfqpoint{3.922580in}{1.814681in}}%
\pgfpathlineto{\pgfqpoint{3.930327in}{1.822906in}}%
\pgfpathlineto{\pgfqpoint{3.938067in}{1.831197in}}%
\pgfpathlineto{\pgfqpoint{3.945802in}{1.839550in}}%
\pgfpathlineto{\pgfqpoint{3.953531in}{1.847963in}}%
\pgfpathlineto{\pgfqpoint{3.940252in}{1.853114in}}%
\pgfpathlineto{\pgfqpoint{3.926978in}{1.858392in}}%
\pgfpathlineto{\pgfqpoint{3.913708in}{1.863796in}}%
\pgfpathlineto{\pgfqpoint{3.900444in}{1.869328in}}%
\pgfpathlineto{\pgfqpoint{3.892701in}{1.861154in}}%
\pgfpathlineto{\pgfqpoint{3.884952in}{1.853044in}}%
\pgfpathlineto{\pgfqpoint{3.877198in}{1.845001in}}%
\pgfpathlineto{\pgfqpoint{3.869437in}{1.837026in}}%
\pgfpathclose%
\pgfusepath{fill}%
\end{pgfscope}%
\begin{pgfscope}%
\pgfpathrectangle{\pgfqpoint{1.254980in}{0.150000in}}{\pgfqpoint{5.490039in}{5.490039in}}%
\pgfusepath{clip}%
\pgfsetbuttcap%
\pgfsetroundjoin%
\definecolor{currentfill}{rgb}{0.265145,0.232956,0.516599}%
\pgfsetfillcolor{currentfill}%
\pgfsetfillopacity{0.700000}%
\pgfsetlinewidth{0.000000pt}%
\definecolor{currentstroke}{rgb}{0.000000,0.000000,0.000000}%
\pgfsetstrokecolor{currentstroke}%
\pgfsetdash{}{0pt}%
\pgfpathmoveto{\pgfqpoint{4.899758in}{2.188616in}}%
\pgfpathlineto{\pgfqpoint{4.913361in}{2.190896in}}%
\pgfpathlineto{\pgfqpoint{4.926975in}{2.193292in}}%
\pgfpathlineto{\pgfqpoint{4.940600in}{2.195802in}}%
\pgfpathlineto{\pgfqpoint{4.954237in}{2.198426in}}%
\pgfpathlineto{\pgfqpoint{4.961653in}{2.209163in}}%
\pgfpathlineto{\pgfqpoint{4.969065in}{2.219863in}}%
\pgfpathlineto{\pgfqpoint{4.976471in}{2.230526in}}%
\pgfpathlineto{\pgfqpoint{4.983871in}{2.241152in}}%
\pgfpathlineto{\pgfqpoint{4.970240in}{2.238457in}}%
\pgfpathlineto{\pgfqpoint{4.956620in}{2.235877in}}%
\pgfpathlineto{\pgfqpoint{4.943012in}{2.233412in}}%
\pgfpathlineto{\pgfqpoint{4.929415in}{2.231062in}}%
\pgfpathlineto{\pgfqpoint{4.922009in}{2.220500in}}%
\pgfpathlineto{\pgfqpoint{4.914597in}{2.209905in}}%
\pgfpathlineto{\pgfqpoint{4.907180in}{2.199276in}}%
\pgfpathlineto{\pgfqpoint{4.899758in}{2.188616in}}%
\pgfpathclose%
\pgfusepath{fill}%
\end{pgfscope}%
\begin{pgfscope}%
\pgfpathrectangle{\pgfqpoint{1.254980in}{0.150000in}}{\pgfqpoint{5.490039in}{5.490039in}}%
\pgfusepath{clip}%
\pgfsetbuttcap%
\pgfsetroundjoin%
\definecolor{currentfill}{rgb}{0.276022,0.044167,0.370164}%
\pgfsetfillcolor{currentfill}%
\pgfsetfillopacity{0.700000}%
\pgfsetlinewidth{0.000000pt}%
\definecolor{currentstroke}{rgb}{0.000000,0.000000,0.000000}%
\pgfsetstrokecolor{currentstroke}%
\pgfsetdash{}{0pt}%
\pgfpathmoveto{\pgfqpoint{3.732127in}{1.858803in}}%
\pgfpathlineto{\pgfqpoint{3.745390in}{1.851728in}}%
\pgfpathlineto{\pgfqpoint{3.758656in}{1.844785in}}%
\pgfpathlineto{\pgfqpoint{3.771925in}{1.837972in}}%
\pgfpathlineto{\pgfqpoint{3.785199in}{1.831289in}}%
\pgfpathlineto{\pgfqpoint{3.793000in}{1.838703in}}%
\pgfpathlineto{\pgfqpoint{3.800795in}{1.846197in}}%
\pgfpathlineto{\pgfqpoint{3.808583in}{1.853770in}}%
\pgfpathlineto{\pgfqpoint{3.816366in}{1.861419in}}%
\pgfpathlineto{\pgfqpoint{3.803108in}{1.867840in}}%
\pgfpathlineto{\pgfqpoint{3.789855in}{1.874391in}}%
\pgfpathlineto{\pgfqpoint{3.776605in}{1.881073in}}%
\pgfpathlineto{\pgfqpoint{3.763359in}{1.887885in}}%
\pgfpathlineto{\pgfqpoint{3.755561in}{1.880492in}}%
\pgfpathlineto{\pgfqpoint{3.747756in}{1.873179in}}%
\pgfpathlineto{\pgfqpoint{3.739945in}{1.865949in}}%
\pgfpathlineto{\pgfqpoint{3.732127in}{1.858803in}}%
\pgfpathclose%
\pgfusepath{fill}%
\end{pgfscope}%
\begin{pgfscope}%
\pgfpathrectangle{\pgfqpoint{1.254980in}{0.150000in}}{\pgfqpoint{5.490039in}{5.490039in}}%
\pgfusepath{clip}%
\pgfsetbuttcap%
\pgfsetroundjoin%
\definecolor{currentfill}{rgb}{0.195860,0.395433,0.555276}%
\pgfsetfillcolor{currentfill}%
\pgfsetfillopacity{0.700000}%
\pgfsetlinewidth{0.000000pt}%
\definecolor{currentstroke}{rgb}{0.000000,0.000000,0.000000}%
\pgfsetstrokecolor{currentstroke}%
\pgfsetdash{}{0pt}%
\pgfpathmoveto{\pgfqpoint{5.434339in}{2.564181in}}%
\pgfpathlineto{\pgfqpoint{5.448202in}{2.569335in}}%
\pgfpathlineto{\pgfqpoint{5.462079in}{2.574602in}}%
\pgfpathlineto{\pgfqpoint{5.475969in}{2.579982in}}%
\pgfpathlineto{\pgfqpoint{5.489875in}{2.585474in}}%
\pgfpathlineto{\pgfqpoint{5.497092in}{2.594946in}}%
\pgfpathlineto{\pgfqpoint{5.504304in}{2.604362in}}%
\pgfpathlineto{\pgfqpoint{5.511508in}{2.613722in}}%
\pgfpathlineto{\pgfqpoint{5.518707in}{2.623026in}}%
\pgfpathlineto{\pgfqpoint{5.504810in}{2.617579in}}%
\pgfpathlineto{\pgfqpoint{5.490928in}{2.612245in}}%
\pgfpathlineto{\pgfqpoint{5.477060in}{2.607023in}}%
\pgfpathlineto{\pgfqpoint{5.463205in}{2.601914in}}%
\pgfpathlineto{\pgfqpoint{5.455998in}{2.592559in}}%
\pgfpathlineto{\pgfqpoint{5.448784in}{2.583152in}}%
\pgfpathlineto{\pgfqpoint{5.441565in}{2.573693in}}%
\pgfpathlineto{\pgfqpoint{5.434339in}{2.564181in}}%
\pgfpathclose%
\pgfusepath{fill}%
\end{pgfscope}%
\begin{pgfscope}%
\pgfpathrectangle{\pgfqpoint{1.254980in}{0.150000in}}{\pgfqpoint{5.490039in}{5.490039in}}%
\pgfusepath{clip}%
\pgfsetbuttcap%
\pgfsetroundjoin%
\definecolor{currentfill}{rgb}{0.282884,0.135920,0.453427}%
\pgfsetfillcolor{currentfill}%
\pgfsetfillopacity{0.700000}%
\pgfsetlinewidth{0.000000pt}%
\definecolor{currentstroke}{rgb}{0.000000,0.000000,0.000000}%
\pgfsetstrokecolor{currentstroke}%
\pgfsetdash{}{0pt}%
\pgfpathmoveto{\pgfqpoint{3.350809in}{2.029192in}}%
\pgfpathlineto{\pgfqpoint{3.364068in}{2.018235in}}%
\pgfpathlineto{\pgfqpoint{3.377326in}{2.007424in}}%
\pgfpathlineto{\pgfqpoint{3.390585in}{1.996756in}}%
\pgfpathlineto{\pgfqpoint{3.403845in}{1.986232in}}%
\pgfpathlineto{\pgfqpoint{3.411824in}{1.991111in}}%
\pgfpathlineto{\pgfqpoint{3.419795in}{1.996113in}}%
\pgfpathlineto{\pgfqpoint{3.427758in}{2.001235in}}%
\pgfpathlineto{\pgfqpoint{3.435712in}{2.006476in}}%
\pgfpathlineto{\pgfqpoint{3.422476in}{2.016702in}}%
\pgfpathlineto{\pgfqpoint{3.409240in}{2.027071in}}%
\pgfpathlineto{\pgfqpoint{3.396005in}{2.037583in}}%
\pgfpathlineto{\pgfqpoint{3.382771in}{2.048240in}}%
\pgfpathlineto{\pgfqpoint{3.374793in}{2.043292in}}%
\pgfpathlineto{\pgfqpoint{3.366807in}{2.038466in}}%
\pgfpathlineto{\pgfqpoint{3.358813in}{2.033765in}}%
\pgfpathlineto{\pgfqpoint{3.350809in}{2.029192in}}%
\pgfpathclose%
\pgfusepath{fill}%
\end{pgfscope}%
\begin{pgfscope}%
\pgfpathrectangle{\pgfqpoint{1.254980in}{0.150000in}}{\pgfqpoint{5.490039in}{5.490039in}}%
\pgfusepath{clip}%
\pgfsetbuttcap%
\pgfsetroundjoin%
\definecolor{currentfill}{rgb}{0.141935,0.526453,0.555991}%
\pgfsetfillcolor{currentfill}%
\pgfsetfillopacity{0.700000}%
\pgfsetlinewidth{0.000000pt}%
\definecolor{currentstroke}{rgb}{0.000000,0.000000,0.000000}%
\pgfsetstrokecolor{currentstroke}%
\pgfsetdash{}{0pt}%
\pgfpathmoveto{\pgfqpoint{2.550693in}{2.981763in}}%
\pgfpathlineto{\pgfqpoint{2.564192in}{2.960545in}}%
\pgfpathlineto{\pgfqpoint{2.577683in}{2.939532in}}%
\pgfpathlineto{\pgfqpoint{2.591165in}{2.918721in}}%
\pgfpathlineto{\pgfqpoint{2.604638in}{2.898111in}}%
\pgfpathlineto{\pgfqpoint{2.613085in}{2.898250in}}%
\pgfpathlineto{\pgfqpoint{2.621517in}{2.898578in}}%
\pgfpathlineto{\pgfqpoint{2.629935in}{2.899093in}}%
\pgfpathlineto{\pgfqpoint{2.638340in}{2.899792in}}%
\pgfpathlineto{\pgfqpoint{2.624905in}{2.920077in}}%
\pgfpathlineto{\pgfqpoint{2.611462in}{2.940562in}}%
\pgfpathlineto{\pgfqpoint{2.598011in}{2.961249in}}%
\pgfpathlineto{\pgfqpoint{2.584551in}{2.982139in}}%
\pgfpathlineto{\pgfqpoint{2.576108in}{2.981759in}}%
\pgfpathlineto{\pgfqpoint{2.567651in}{2.981568in}}%
\pgfpathlineto{\pgfqpoint{2.559179in}{2.981568in}}%
\pgfpathlineto{\pgfqpoint{2.550693in}{2.981763in}}%
\pgfpathclose%
\pgfusepath{fill}%
\end{pgfscope}%
\begin{pgfscope}%
\pgfpathrectangle{\pgfqpoint{1.254980in}{0.150000in}}{\pgfqpoint{5.490039in}{5.490039in}}%
\pgfusepath{clip}%
\pgfsetbuttcap%
\pgfsetroundjoin%
\definecolor{currentfill}{rgb}{0.280267,0.073417,0.397163}%
\pgfsetfillcolor{currentfill}%
\pgfsetfillopacity{0.700000}%
\pgfsetlinewidth{0.000000pt}%
\definecolor{currentstroke}{rgb}{0.000000,0.000000,0.000000}%
\pgfsetstrokecolor{currentstroke}%
\pgfsetdash{}{0pt}%
\pgfpathmoveto{\pgfqpoint{4.311984in}{1.888850in}}%
\pgfpathlineto{\pgfqpoint{4.325365in}{1.886919in}}%
\pgfpathlineto{\pgfqpoint{4.338754in}{1.885108in}}%
\pgfpathlineto{\pgfqpoint{4.352151in}{1.883416in}}%
\pgfpathlineto{\pgfqpoint{4.365556in}{1.881844in}}%
\pgfpathlineto{\pgfqpoint{4.373153in}{1.892016in}}%
\pgfpathlineto{\pgfqpoint{4.380746in}{1.902204in}}%
\pgfpathlineto{\pgfqpoint{4.388333in}{1.912405in}}%
\pgfpathlineto{\pgfqpoint{4.395916in}{1.922618in}}%
\pgfpathlineto{\pgfqpoint{4.382519in}{1.924009in}}%
\pgfpathlineto{\pgfqpoint{4.369131in}{1.925520in}}%
\pgfpathlineto{\pgfqpoint{4.355750in}{1.927150in}}%
\pgfpathlineto{\pgfqpoint{4.342378in}{1.928900in}}%
\pgfpathlineto{\pgfqpoint{4.334787in}{1.918862in}}%
\pgfpathlineto{\pgfqpoint{4.327191in}{1.908840in}}%
\pgfpathlineto{\pgfqpoint{4.319590in}{1.898836in}}%
\pgfpathlineto{\pgfqpoint{4.311984in}{1.888850in}}%
\pgfpathclose%
\pgfusepath{fill}%
\end{pgfscope}%
\begin{pgfscope}%
\pgfpathrectangle{\pgfqpoint{1.254980in}{0.150000in}}{\pgfqpoint{5.490039in}{5.490039in}}%
\pgfusepath{clip}%
\pgfsetbuttcap%
\pgfsetroundjoin%
\definecolor{currentfill}{rgb}{0.257322,0.256130,0.526563}%
\pgfsetfillcolor{currentfill}%
\pgfsetfillopacity{0.700000}%
\pgfsetlinewidth{0.000000pt}%
\definecolor{currentstroke}{rgb}{0.000000,0.000000,0.000000}%
\pgfsetstrokecolor{currentstroke}%
\pgfsetdash{}{0pt}%
\pgfpathmoveto{\pgfqpoint{4.983871in}{2.241152in}}%
\pgfpathlineto{\pgfqpoint{4.997514in}{2.243960in}}%
\pgfpathlineto{\pgfqpoint{5.011170in}{2.246883in}}%
\pgfpathlineto{\pgfqpoint{5.024837in}{2.249921in}}%
\pgfpathlineto{\pgfqpoint{5.038516in}{2.253072in}}%
\pgfpathlineto{\pgfqpoint{5.045906in}{2.263720in}}%
\pgfpathlineto{\pgfqpoint{5.053290in}{2.274326in}}%
\pgfpathlineto{\pgfqpoint{5.060669in}{2.284890in}}%
\pgfpathlineto{\pgfqpoint{5.068043in}{2.295412in}}%
\pgfpathlineto{\pgfqpoint{5.054370in}{2.292207in}}%
\pgfpathlineto{\pgfqpoint{5.040708in}{2.289116in}}%
\pgfpathlineto{\pgfqpoint{5.027059in}{2.286139in}}%
\pgfpathlineto{\pgfqpoint{5.013422in}{2.283277in}}%
\pgfpathlineto{\pgfqpoint{5.006042in}{2.272802in}}%
\pgfpathlineto{\pgfqpoint{4.998657in}{2.262290in}}%
\pgfpathlineto{\pgfqpoint{4.991267in}{2.251740in}}%
\pgfpathlineto{\pgfqpoint{4.983871in}{2.241152in}}%
\pgfpathclose%
\pgfusepath{fill}%
\end{pgfscope}%
\begin{pgfscope}%
\pgfpathrectangle{\pgfqpoint{1.254980in}{0.150000in}}{\pgfqpoint{5.490039in}{5.490039in}}%
\pgfusepath{clip}%
\pgfsetbuttcap%
\pgfsetroundjoin%
\definecolor{currentfill}{rgb}{0.274952,0.037752,0.364543}%
\pgfsetfillcolor{currentfill}%
\pgfsetfillopacity{0.700000}%
\pgfsetlinewidth{0.000000pt}%
\definecolor{currentstroke}{rgb}{0.000000,0.000000,0.000000}%
\pgfsetstrokecolor{currentstroke}%
\pgfsetdash{}{0pt}%
\pgfpathmoveto{\pgfqpoint{4.006700in}{1.828619in}}%
\pgfpathlineto{\pgfqpoint{4.020006in}{1.824096in}}%
\pgfpathlineto{\pgfqpoint{4.033317in}{1.819698in}}%
\pgfpathlineto{\pgfqpoint{4.046635in}{1.815423in}}%
\pgfpathlineto{\pgfqpoint{4.059958in}{1.811273in}}%
\pgfpathlineto{\pgfqpoint{4.067656in}{1.820215in}}%
\pgfpathlineto{\pgfqpoint{4.075348in}{1.829206in}}%
\pgfpathlineto{\pgfqpoint{4.083035in}{1.838245in}}%
\pgfpathlineto{\pgfqpoint{4.090717in}{1.847329in}}%
\pgfpathlineto{\pgfqpoint{4.077406in}{1.851251in}}%
\pgfpathlineto{\pgfqpoint{4.064100in}{1.855297in}}%
\pgfpathlineto{\pgfqpoint{4.050801in}{1.859466in}}%
\pgfpathlineto{\pgfqpoint{4.037508in}{1.863760in}}%
\pgfpathlineto{\pgfqpoint{4.029814in}{1.854899in}}%
\pgfpathlineto{\pgfqpoint{4.022115in}{1.846087in}}%
\pgfpathlineto{\pgfqpoint{4.014410in}{1.837326in}}%
\pgfpathlineto{\pgfqpoint{4.006700in}{1.828619in}}%
\pgfpathclose%
\pgfusepath{fill}%
\end{pgfscope}%
\begin{pgfscope}%
\pgfpathrectangle{\pgfqpoint{1.254980in}{0.150000in}}{\pgfqpoint{5.490039in}{5.490039in}}%
\pgfusepath{clip}%
\pgfsetbuttcap%
\pgfsetroundjoin%
\definecolor{currentfill}{rgb}{0.279566,0.067836,0.391917}%
\pgfsetfillcolor{currentfill}%
\pgfsetfillopacity{0.700000}%
\pgfsetlinewidth{0.000000pt}%
\definecolor{currentstroke}{rgb}{0.000000,0.000000,0.000000}%
\pgfsetstrokecolor{currentstroke}%
\pgfsetdash{}{0pt}%
\pgfpathmoveto{\pgfqpoint{3.594654in}{1.894709in}}%
\pgfpathlineto{\pgfqpoint{3.607911in}{1.886289in}}%
\pgfpathlineto{\pgfqpoint{3.621171in}{1.878005in}}%
\pgfpathlineto{\pgfqpoint{3.634433in}{1.869855in}}%
\pgfpathlineto{\pgfqpoint{3.647698in}{1.861840in}}%
\pgfpathlineto{\pgfqpoint{3.655562in}{1.868341in}}%
\pgfpathlineto{\pgfqpoint{3.663418in}{1.874940in}}%
\pgfpathlineto{\pgfqpoint{3.671268in}{1.881634in}}%
\pgfpathlineto{\pgfqpoint{3.679110in}{1.888421in}}%
\pgfpathlineto{\pgfqpoint{3.665864in}{1.896158in}}%
\pgfpathlineto{\pgfqpoint{3.652620in}{1.904029in}}%
\pgfpathlineto{\pgfqpoint{3.639379in}{1.912034in}}%
\pgfpathlineto{\pgfqpoint{3.626141in}{1.920174in}}%
\pgfpathlineto{\pgfqpoint{3.618280in}{1.913660in}}%
\pgfpathlineto{\pgfqpoint{3.610412in}{1.907243in}}%
\pgfpathlineto{\pgfqpoint{3.602536in}{1.900926in}}%
\pgfpathlineto{\pgfqpoint{3.594654in}{1.894709in}}%
\pgfpathclose%
\pgfusepath{fill}%
\end{pgfscope}%
\begin{pgfscope}%
\pgfpathrectangle{\pgfqpoint{1.254980in}{0.150000in}}{\pgfqpoint{5.490039in}{5.490039in}}%
\pgfusepath{clip}%
\pgfsetbuttcap%
\pgfsetroundjoin%
\definecolor{currentfill}{rgb}{0.183898,0.422383,0.556944}%
\pgfsetfillcolor{currentfill}%
\pgfsetfillopacity{0.700000}%
\pgfsetlinewidth{0.000000pt}%
\definecolor{currentstroke}{rgb}{0.000000,0.000000,0.000000}%
\pgfsetstrokecolor{currentstroke}%
\pgfsetdash{}{0pt}%
\pgfpathmoveto{\pgfqpoint{5.518707in}{2.623026in}}%
\pgfpathlineto{\pgfqpoint{5.532618in}{2.628585in}}%
\pgfpathlineto{\pgfqpoint{5.546544in}{2.634257in}}%
\pgfpathlineto{\pgfqpoint{5.560485in}{2.640040in}}%
\pgfpathlineto{\pgfqpoint{5.574440in}{2.645937in}}%
\pgfpathlineto{\pgfqpoint{5.581623in}{2.655130in}}%
\pgfpathlineto{\pgfqpoint{5.588800in}{2.664265in}}%
\pgfpathlineto{\pgfqpoint{5.595971in}{2.673343in}}%
\pgfpathlineto{\pgfqpoint{5.603135in}{2.682365in}}%
\pgfpathlineto{\pgfqpoint{5.589189in}{2.676531in}}%
\pgfpathlineto{\pgfqpoint{5.575258in}{2.670809in}}%
\pgfpathlineto{\pgfqpoint{5.561341in}{2.665200in}}%
\pgfpathlineto{\pgfqpoint{5.547439in}{2.659702in}}%
\pgfpathlineto{\pgfqpoint{5.540266in}{2.650613in}}%
\pgfpathlineto{\pgfqpoint{5.533086in}{2.641470in}}%
\pgfpathlineto{\pgfqpoint{5.525900in}{2.632275in}}%
\pgfpathlineto{\pgfqpoint{5.518707in}{2.623026in}}%
\pgfpathclose%
\pgfusepath{fill}%
\end{pgfscope}%
\begin{pgfscope}%
\pgfpathrectangle{\pgfqpoint{1.254980in}{0.150000in}}{\pgfqpoint{5.490039in}{5.490039in}}%
\pgfusepath{clip}%
\pgfsetbuttcap%
\pgfsetroundjoin%
\definecolor{currentfill}{rgb}{0.246811,0.283237,0.535941}%
\pgfsetfillcolor{currentfill}%
\pgfsetfillopacity{0.700000}%
\pgfsetlinewidth{0.000000pt}%
\definecolor{currentstroke}{rgb}{0.000000,0.000000,0.000000}%
\pgfsetstrokecolor{currentstroke}%
\pgfsetdash{}{0pt}%
\pgfpathmoveto{\pgfqpoint{5.068043in}{2.295412in}}%
\pgfpathlineto{\pgfqpoint{5.081729in}{2.298731in}}%
\pgfpathlineto{\pgfqpoint{5.095427in}{2.302164in}}%
\pgfpathlineto{\pgfqpoint{5.109137in}{2.305711in}}%
\pgfpathlineto{\pgfqpoint{5.122860in}{2.309371in}}%
\pgfpathlineto{\pgfqpoint{5.130223in}{2.319894in}}%
\pgfpathlineto{\pgfqpoint{5.137580in}{2.330371in}}%
\pgfpathlineto{\pgfqpoint{5.144931in}{2.340802in}}%
\pgfpathlineto{\pgfqpoint{5.152277in}{2.351185in}}%
\pgfpathlineto{\pgfqpoint{5.138560in}{2.347488in}}%
\pgfpathlineto{\pgfqpoint{5.124856in}{2.343903in}}%
\pgfpathlineto{\pgfqpoint{5.111164in}{2.340433in}}%
\pgfpathlineto{\pgfqpoint{5.097484in}{2.337076in}}%
\pgfpathlineto{\pgfqpoint{5.090132in}{2.326724in}}%
\pgfpathlineto{\pgfqpoint{5.082774in}{2.316329in}}%
\pgfpathlineto{\pgfqpoint{5.075411in}{2.305892in}}%
\pgfpathlineto{\pgfqpoint{5.068043in}{2.295412in}}%
\pgfpathclose%
\pgfusepath{fill}%
\end{pgfscope}%
\begin{pgfscope}%
\pgfpathrectangle{\pgfqpoint{1.254980in}{0.150000in}}{\pgfqpoint{5.490039in}{5.490039in}}%
\pgfusepath{clip}%
\pgfsetbuttcap%
\pgfsetroundjoin%
\definecolor{currentfill}{rgb}{0.278791,0.062145,0.386592}%
\pgfsetfillcolor{currentfill}%
\pgfsetfillopacity{0.700000}%
\pgfsetlinewidth{0.000000pt}%
\definecolor{currentstroke}{rgb}{0.000000,0.000000,0.000000}%
\pgfsetstrokecolor{currentstroke}%
\pgfsetdash{}{0pt}%
\pgfpathmoveto{\pgfqpoint{4.228027in}{1.858836in}}%
\pgfpathlineto{\pgfqpoint{4.241387in}{1.856227in}}%
\pgfpathlineto{\pgfqpoint{4.254755in}{1.853738in}}%
\pgfpathlineto{\pgfqpoint{4.268130in}{1.851370in}}%
\pgfpathlineto{\pgfqpoint{4.281512in}{1.849122in}}%
\pgfpathlineto{\pgfqpoint{4.289138in}{1.859019in}}%
\pgfpathlineto{\pgfqpoint{4.296758in}{1.868941in}}%
\pgfpathlineto{\pgfqpoint{4.304374in}{1.878885in}}%
\pgfpathlineto{\pgfqpoint{4.311984in}{1.888850in}}%
\pgfpathlineto{\pgfqpoint{4.298611in}{1.890901in}}%
\pgfpathlineto{\pgfqpoint{4.285245in}{1.893072in}}%
\pgfpathlineto{\pgfqpoint{4.271887in}{1.895364in}}%
\pgfpathlineto{\pgfqpoint{4.258536in}{1.897776in}}%
\pgfpathlineto{\pgfqpoint{4.250916in}{1.888002in}}%
\pgfpathlineto{\pgfqpoint{4.243291in}{1.878252in}}%
\pgfpathlineto{\pgfqpoint{4.235661in}{1.868530in}}%
\pgfpathlineto{\pgfqpoint{4.228027in}{1.858836in}}%
\pgfpathclose%
\pgfusepath{fill}%
\end{pgfscope}%
\begin{pgfscope}%
\pgfpathrectangle{\pgfqpoint{1.254980in}{0.150000in}}{\pgfqpoint{5.490039in}{5.490039in}}%
\pgfusepath{clip}%
\pgfsetbuttcap%
\pgfsetroundjoin%
\definecolor{currentfill}{rgb}{0.283197,0.115680,0.436115}%
\pgfsetfillcolor{currentfill}%
\pgfsetfillopacity{0.700000}%
\pgfsetlinewidth{0.000000pt}%
\definecolor{currentstroke}{rgb}{0.000000,0.000000,0.000000}%
\pgfsetstrokecolor{currentstroke}%
\pgfsetdash{}{0pt}%
\pgfpathmoveto{\pgfqpoint{3.403845in}{1.986232in}}%
\pgfpathlineto{\pgfqpoint{3.417105in}{1.975851in}}%
\pgfpathlineto{\pgfqpoint{3.430366in}{1.965611in}}%
\pgfpathlineto{\pgfqpoint{3.443628in}{1.955513in}}%
\pgfpathlineto{\pgfqpoint{3.456891in}{1.945556in}}%
\pgfpathlineto{\pgfqpoint{3.464848in}{1.950739in}}%
\pgfpathlineto{\pgfqpoint{3.472796in}{1.956041in}}%
\pgfpathlineto{\pgfqpoint{3.480736in}{1.961459in}}%
\pgfpathlineto{\pgfqpoint{3.488668in}{1.966992in}}%
\pgfpathlineto{\pgfqpoint{3.475427in}{1.976651in}}%
\pgfpathlineto{\pgfqpoint{3.462188in}{1.986451in}}%
\pgfpathlineto{\pgfqpoint{3.448949in}{1.996393in}}%
\pgfpathlineto{\pgfqpoint{3.435712in}{2.006476in}}%
\pgfpathlineto{\pgfqpoint{3.427758in}{2.001235in}}%
\pgfpathlineto{\pgfqpoint{3.419795in}{1.996113in}}%
\pgfpathlineto{\pgfqpoint{3.411824in}{1.991111in}}%
\pgfpathlineto{\pgfqpoint{3.403845in}{1.986232in}}%
\pgfpathclose%
\pgfusepath{fill}%
\end{pgfscope}%
\begin{pgfscope}%
\pgfpathrectangle{\pgfqpoint{1.254980in}{0.150000in}}{\pgfqpoint{5.490039in}{5.490039in}}%
\pgfusepath{clip}%
\pgfsetbuttcap%
\pgfsetroundjoin%
\definecolor{currentfill}{rgb}{0.239346,0.300855,0.540844}%
\pgfsetfillcolor{currentfill}%
\pgfsetfillopacity{0.700000}%
\pgfsetlinewidth{0.000000pt}%
\definecolor{currentstroke}{rgb}{0.000000,0.000000,0.000000}%
\pgfsetstrokecolor{currentstroke}%
\pgfsetdash{}{0pt}%
\pgfpathmoveto{\pgfqpoint{2.946263in}{2.389624in}}%
\pgfpathlineto{\pgfqpoint{2.959599in}{2.373990in}}%
\pgfpathlineto{\pgfqpoint{2.972930in}{2.358523in}}%
\pgfpathlineto{\pgfqpoint{2.986259in}{2.343223in}}%
\pgfpathlineto{\pgfqpoint{2.999583in}{2.328088in}}%
\pgfpathlineto{\pgfqpoint{3.007796in}{2.330166in}}%
\pgfpathlineto{\pgfqpoint{3.015997in}{2.332408in}}%
\pgfpathlineto{\pgfqpoint{3.024187in}{2.334811in}}%
\pgfpathlineto{\pgfqpoint{3.032366in}{2.337372in}}%
\pgfpathlineto{\pgfqpoint{3.019073in}{2.352182in}}%
\pgfpathlineto{\pgfqpoint{3.005776in}{2.367157in}}%
\pgfpathlineto{\pgfqpoint{2.992477in}{2.382297in}}%
\pgfpathlineto{\pgfqpoint{2.979173in}{2.397604in}}%
\pgfpathlineto{\pgfqpoint{2.970963in}{2.395363in}}%
\pgfpathlineto{\pgfqpoint{2.962742in}{2.393284in}}%
\pgfpathlineto{\pgfqpoint{2.954508in}{2.391370in}}%
\pgfpathlineto{\pgfqpoint{2.946263in}{2.389624in}}%
\pgfpathclose%
\pgfusepath{fill}%
\end{pgfscope}%
\begin{pgfscope}%
\pgfpathrectangle{\pgfqpoint{1.254980in}{0.150000in}}{\pgfqpoint{5.490039in}{5.490039in}}%
\pgfusepath{clip}%
\pgfsetbuttcap%
\pgfsetroundjoin%
\definecolor{currentfill}{rgb}{0.225863,0.330805,0.547314}%
\pgfsetfillcolor{currentfill}%
\pgfsetfillopacity{0.700000}%
\pgfsetlinewidth{0.000000pt}%
\definecolor{currentstroke}{rgb}{0.000000,0.000000,0.000000}%
\pgfsetstrokecolor{currentstroke}%
\pgfsetdash{}{0pt}%
\pgfpathmoveto{\pgfqpoint{2.892881in}{2.453851in}}%
\pgfpathlineto{\pgfqpoint{2.906233in}{2.437538in}}%
\pgfpathlineto{\pgfqpoint{2.919581in}{2.421397in}}%
\pgfpathlineto{\pgfqpoint{2.932924in}{2.405426in}}%
\pgfpathlineto{\pgfqpoint{2.946263in}{2.389624in}}%
\pgfpathlineto{\pgfqpoint{2.954508in}{2.391370in}}%
\pgfpathlineto{\pgfqpoint{2.962742in}{2.393284in}}%
\pgfpathlineto{\pgfqpoint{2.970963in}{2.395363in}}%
\pgfpathlineto{\pgfqpoint{2.979173in}{2.397604in}}%
\pgfpathlineto{\pgfqpoint{2.965867in}{2.413079in}}%
\pgfpathlineto{\pgfqpoint{2.952556in}{2.428723in}}%
\pgfpathlineto{\pgfqpoint{2.939242in}{2.444536in}}%
\pgfpathlineto{\pgfqpoint{2.925923in}{2.460519in}}%
\pgfpathlineto{\pgfqpoint{2.917681in}{2.458599in}}%
\pgfpathlineto{\pgfqpoint{2.909426in}{2.456846in}}%
\pgfpathlineto{\pgfqpoint{2.901160in}{2.455262in}}%
\pgfpathlineto{\pgfqpoint{2.892881in}{2.453851in}}%
\pgfpathclose%
\pgfusepath{fill}%
\end{pgfscope}%
\begin{pgfscope}%
\pgfpathrectangle{\pgfqpoint{1.254980in}{0.150000in}}{\pgfqpoint{5.490039in}{5.490039in}}%
\pgfusepath{clip}%
\pgfsetbuttcap%
\pgfsetroundjoin%
\definecolor{currentfill}{rgb}{0.128729,0.563265,0.551229}%
\pgfsetfillcolor{currentfill}%
\pgfsetfillopacity{0.700000}%
\pgfsetlinewidth{0.000000pt}%
\definecolor{currentstroke}{rgb}{0.000000,0.000000,0.000000}%
\pgfsetstrokecolor{currentstroke}%
\pgfsetdash{}{0pt}%
\pgfpathmoveto{\pgfqpoint{2.496602in}{3.068704in}}%
\pgfpathlineto{\pgfqpoint{2.510139in}{3.046655in}}%
\pgfpathlineto{\pgfqpoint{2.523666in}{3.024817in}}%
\pgfpathlineto{\pgfqpoint{2.537184in}{3.003186in}}%
\pgfpathlineto{\pgfqpoint{2.550693in}{2.981763in}}%
\pgfpathlineto{\pgfqpoint{2.559179in}{2.981568in}}%
\pgfpathlineto{\pgfqpoint{2.567651in}{2.981568in}}%
\pgfpathlineto{\pgfqpoint{2.576108in}{2.981759in}}%
\pgfpathlineto{\pgfqpoint{2.584551in}{2.982139in}}%
\pgfpathlineto{\pgfqpoint{2.571083in}{3.003234in}}%
\pgfpathlineto{\pgfqpoint{2.557606in}{3.024536in}}%
\pgfpathlineto{\pgfqpoint{2.544119in}{3.046045in}}%
\pgfpathlineto{\pgfqpoint{2.530623in}{3.067763in}}%
\pgfpathlineto{\pgfqpoint{2.522140in}{3.067705in}}%
\pgfpathlineto{\pgfqpoint{2.513642in}{3.067840in}}%
\pgfpathlineto{\pgfqpoint{2.505130in}{3.068173in}}%
\pgfpathlineto{\pgfqpoint{2.496602in}{3.068704in}}%
\pgfpathclose%
\pgfusepath{fill}%
\end{pgfscope}%
\begin{pgfscope}%
\pgfpathrectangle{\pgfqpoint{1.254980in}{0.150000in}}{\pgfqpoint{5.490039in}{5.490039in}}%
\pgfusepath{clip}%
\pgfsetbuttcap%
\pgfsetroundjoin%
\definecolor{currentfill}{rgb}{0.250425,0.274290,0.533103}%
\pgfsetfillcolor{currentfill}%
\pgfsetfillopacity{0.700000}%
\pgfsetlinewidth{0.000000pt}%
\definecolor{currentstroke}{rgb}{0.000000,0.000000,0.000000}%
\pgfsetstrokecolor{currentstroke}%
\pgfsetdash{}{0pt}%
\pgfpathmoveto{\pgfqpoint{2.999583in}{2.328088in}}%
\pgfpathlineto{\pgfqpoint{3.012905in}{2.313117in}}%
\pgfpathlineto{\pgfqpoint{3.026223in}{2.298310in}}%
\pgfpathlineto{\pgfqpoint{3.039538in}{2.283665in}}%
\pgfpathlineto{\pgfqpoint{3.052850in}{2.269183in}}%
\pgfpathlineto{\pgfqpoint{3.061031in}{2.271592in}}%
\pgfpathlineto{\pgfqpoint{3.069202in}{2.274160in}}%
\pgfpathlineto{\pgfqpoint{3.077361in}{2.276885in}}%
\pgfpathlineto{\pgfqpoint{3.085509in}{2.279764in}}%
\pgfpathlineto{\pgfqpoint{3.072228in}{2.293923in}}%
\pgfpathlineto{\pgfqpoint{3.058943in}{2.308244in}}%
\pgfpathlineto{\pgfqpoint{3.045656in}{2.322726in}}%
\pgfpathlineto{\pgfqpoint{3.032366in}{2.337372in}}%
\pgfpathlineto{\pgfqpoint{3.024187in}{2.334811in}}%
\pgfpathlineto{\pgfqpoint{3.015997in}{2.332408in}}%
\pgfpathlineto{\pgfqpoint{3.007796in}{2.330166in}}%
\pgfpathlineto{\pgfqpoint{2.999583in}{2.328088in}}%
\pgfpathclose%
\pgfusepath{fill}%
\end{pgfscope}%
\begin{pgfscope}%
\pgfpathrectangle{\pgfqpoint{1.254980in}{0.150000in}}{\pgfqpoint{5.490039in}{5.490039in}}%
\pgfusepath{clip}%
\pgfsetbuttcap%
\pgfsetroundjoin%
\definecolor{currentfill}{rgb}{0.212395,0.359683,0.551710}%
\pgfsetfillcolor{currentfill}%
\pgfsetfillopacity{0.700000}%
\pgfsetlinewidth{0.000000pt}%
\definecolor{currentstroke}{rgb}{0.000000,0.000000,0.000000}%
\pgfsetstrokecolor{currentstroke}%
\pgfsetdash{}{0pt}%
\pgfpathmoveto{\pgfqpoint{2.839427in}{2.520832in}}%
\pgfpathlineto{\pgfqpoint{2.852798in}{2.503825in}}%
\pgfpathlineto{\pgfqpoint{2.866164in}{2.486993in}}%
\pgfpathlineto{\pgfqpoint{2.879525in}{2.470335in}}%
\pgfpathlineto{\pgfqpoint{2.892881in}{2.453851in}}%
\pgfpathlineto{\pgfqpoint{2.901160in}{2.455262in}}%
\pgfpathlineto{\pgfqpoint{2.909426in}{2.456846in}}%
\pgfpathlineto{\pgfqpoint{2.917681in}{2.458599in}}%
\pgfpathlineto{\pgfqpoint{2.925923in}{2.460519in}}%
\pgfpathlineto{\pgfqpoint{2.912600in}{2.476675in}}%
\pgfpathlineto{\pgfqpoint{2.899273in}{2.493003in}}%
\pgfpathlineto{\pgfqpoint{2.885942in}{2.509504in}}%
\pgfpathlineto{\pgfqpoint{2.872605in}{2.526180in}}%
\pgfpathlineto{\pgfqpoint{2.864329in}{2.524583in}}%
\pgfpathlineto{\pgfqpoint{2.856041in}{2.523157in}}%
\pgfpathlineto{\pgfqpoint{2.847740in}{2.521906in}}%
\pgfpathlineto{\pgfqpoint{2.839427in}{2.520832in}}%
\pgfpathclose%
\pgfusepath{fill}%
\end{pgfscope}%
\begin{pgfscope}%
\pgfpathrectangle{\pgfqpoint{1.254980in}{0.150000in}}{\pgfqpoint{5.490039in}{5.490039in}}%
\pgfusepath{clip}%
\pgfsetbuttcap%
\pgfsetroundjoin%
\definecolor{currentfill}{rgb}{0.235526,0.309527,0.542944}%
\pgfsetfillcolor{currentfill}%
\pgfsetfillopacity{0.700000}%
\pgfsetlinewidth{0.000000pt}%
\definecolor{currentstroke}{rgb}{0.000000,0.000000,0.000000}%
\pgfsetstrokecolor{currentstroke}%
\pgfsetdash{}{0pt}%
\pgfpathmoveto{\pgfqpoint{5.152277in}{2.351185in}}%
\pgfpathlineto{\pgfqpoint{5.166007in}{2.354997in}}%
\pgfpathlineto{\pgfqpoint{5.179749in}{2.358922in}}%
\pgfpathlineto{\pgfqpoint{5.193505in}{2.362961in}}%
\pgfpathlineto{\pgfqpoint{5.207273in}{2.367112in}}%
\pgfpathlineto{\pgfqpoint{5.214608in}{2.377477in}}%
\pgfpathlineto{\pgfqpoint{5.221936in}{2.387791in}}%
\pgfpathlineto{\pgfqpoint{5.229259in}{2.398055in}}%
\pgfpathlineto{\pgfqpoint{5.236576in}{2.408268in}}%
\pgfpathlineto{\pgfqpoint{5.222814in}{2.404095in}}%
\pgfpathlineto{\pgfqpoint{5.209065in}{2.400036in}}%
\pgfpathlineto{\pgfqpoint{5.195329in}{2.396090in}}%
\pgfpathlineto{\pgfqpoint{5.181605in}{2.392257in}}%
\pgfpathlineto{\pgfqpoint{5.174282in}{2.382058in}}%
\pgfpathlineto{\pgfqpoint{5.166952in}{2.371814in}}%
\pgfpathlineto{\pgfqpoint{5.159618in}{2.361523in}}%
\pgfpathlineto{\pgfqpoint{5.152277in}{2.351185in}}%
\pgfpathclose%
\pgfusepath{fill}%
\end{pgfscope}%
\begin{pgfscope}%
\pgfpathrectangle{\pgfqpoint{1.254980in}{0.150000in}}{\pgfqpoint{5.490039in}{5.490039in}}%
\pgfusepath{clip}%
\pgfsetbuttcap%
\pgfsetroundjoin%
\definecolor{currentfill}{rgb}{0.258965,0.251537,0.524736}%
\pgfsetfillcolor{currentfill}%
\pgfsetfillopacity{0.700000}%
\pgfsetlinewidth{0.000000pt}%
\definecolor{currentstroke}{rgb}{0.000000,0.000000,0.000000}%
\pgfsetstrokecolor{currentstroke}%
\pgfsetdash{}{0pt}%
\pgfpathmoveto{\pgfqpoint{3.052850in}{2.269183in}}%
\pgfpathlineto{\pgfqpoint{3.066160in}{2.254861in}}%
\pgfpathlineto{\pgfqpoint{3.079467in}{2.240698in}}%
\pgfpathlineto{\pgfqpoint{3.092771in}{2.226696in}}%
\pgfpathlineto{\pgfqpoint{3.106073in}{2.212851in}}%
\pgfpathlineto{\pgfqpoint{3.114224in}{2.215589in}}%
\pgfpathlineto{\pgfqpoint{3.122364in}{2.218483in}}%
\pgfpathlineto{\pgfqpoint{3.130494in}{2.221528in}}%
\pgfpathlineto{\pgfqpoint{3.138613in}{2.224722in}}%
\pgfpathlineto{\pgfqpoint{3.125341in}{2.238245in}}%
\pgfpathlineto{\pgfqpoint{3.112066in}{2.251925in}}%
\pgfpathlineto{\pgfqpoint{3.098789in}{2.265765in}}%
\pgfpathlineto{\pgfqpoint{3.085509in}{2.279764in}}%
\pgfpathlineto{\pgfqpoint{3.077361in}{2.276885in}}%
\pgfpathlineto{\pgfqpoint{3.069202in}{2.274160in}}%
\pgfpathlineto{\pgfqpoint{3.061031in}{2.271592in}}%
\pgfpathlineto{\pgfqpoint{3.052850in}{2.269183in}}%
\pgfpathclose%
\pgfusepath{fill}%
\end{pgfscope}%
\begin{pgfscope}%
\pgfpathrectangle{\pgfqpoint{1.254980in}{0.150000in}}{\pgfqpoint{5.490039in}{5.490039in}}%
\pgfusepath{clip}%
\pgfsetbuttcap%
\pgfsetroundjoin%
\definecolor{currentfill}{rgb}{0.174274,0.445044,0.557792}%
\pgfsetfillcolor{currentfill}%
\pgfsetfillopacity{0.700000}%
\pgfsetlinewidth{0.000000pt}%
\definecolor{currentstroke}{rgb}{0.000000,0.000000,0.000000}%
\pgfsetstrokecolor{currentstroke}%
\pgfsetdash{}{0pt}%
\pgfpathmoveto{\pgfqpoint{5.603135in}{2.682365in}}%
\pgfpathlineto{\pgfqpoint{5.617096in}{2.688311in}}%
\pgfpathlineto{\pgfqpoint{5.631071in}{2.694370in}}%
\pgfpathlineto{\pgfqpoint{5.645062in}{2.700540in}}%
\pgfpathlineto{\pgfqpoint{5.659067in}{2.706823in}}%
\pgfpathlineto{\pgfqpoint{5.666215in}{2.715716in}}%
\pgfpathlineto{\pgfqpoint{5.673357in}{2.724551in}}%
\pgfpathlineto{\pgfqpoint{5.680492in}{2.733328in}}%
\pgfpathlineto{\pgfqpoint{5.687620in}{2.742048in}}%
\pgfpathlineto{\pgfqpoint{5.673624in}{2.735844in}}%
\pgfpathlineto{\pgfqpoint{5.659644in}{2.729752in}}%
\pgfpathlineto{\pgfqpoint{5.645678in}{2.723773in}}%
\pgfpathlineto{\pgfqpoint{5.631727in}{2.717905in}}%
\pgfpathlineto{\pgfqpoint{5.624589in}{2.709100in}}%
\pgfpathlineto{\pgfqpoint{5.617444in}{2.700243in}}%
\pgfpathlineto{\pgfqpoint{5.610293in}{2.691331in}}%
\pgfpathlineto{\pgfqpoint{5.603135in}{2.682365in}}%
\pgfpathclose%
\pgfusepath{fill}%
\end{pgfscope}%
\begin{pgfscope}%
\pgfpathrectangle{\pgfqpoint{1.254980in}{0.150000in}}{\pgfqpoint{5.490039in}{5.490039in}}%
\pgfusepath{clip}%
\pgfsetbuttcap%
\pgfsetroundjoin%
\definecolor{currentfill}{rgb}{0.199430,0.387607,0.554642}%
\pgfsetfillcolor{currentfill}%
\pgfsetfillopacity{0.700000}%
\pgfsetlinewidth{0.000000pt}%
\definecolor{currentstroke}{rgb}{0.000000,0.000000,0.000000}%
\pgfsetstrokecolor{currentstroke}%
\pgfsetdash{}{0pt}%
\pgfpathmoveto{\pgfqpoint{2.785890in}{2.590633in}}%
\pgfpathlineto{\pgfqpoint{2.799282in}{2.572915in}}%
\pgfpathlineto{\pgfqpoint{2.812669in}{2.555376in}}%
\pgfpathlineto{\pgfqpoint{2.826051in}{2.538015in}}%
\pgfpathlineto{\pgfqpoint{2.839427in}{2.520832in}}%
\pgfpathlineto{\pgfqpoint{2.847740in}{2.521906in}}%
\pgfpathlineto{\pgfqpoint{2.856041in}{2.523157in}}%
\pgfpathlineto{\pgfqpoint{2.864329in}{2.524583in}}%
\pgfpathlineto{\pgfqpoint{2.872605in}{2.526180in}}%
\pgfpathlineto{\pgfqpoint{2.859264in}{2.543032in}}%
\pgfpathlineto{\pgfqpoint{2.845918in}{2.560061in}}%
\pgfpathlineto{\pgfqpoint{2.832566in}{2.577268in}}%
\pgfpathlineto{\pgfqpoint{2.819209in}{2.594653in}}%
\pgfpathlineto{\pgfqpoint{2.810899in}{2.593381in}}%
\pgfpathlineto{\pgfqpoint{2.802575in}{2.592285in}}%
\pgfpathlineto{\pgfqpoint{2.794239in}{2.591369in}}%
\pgfpathlineto{\pgfqpoint{2.785890in}{2.590633in}}%
\pgfpathclose%
\pgfusepath{fill}%
\end{pgfscope}%
\begin{pgfscope}%
\pgfpathrectangle{\pgfqpoint{1.254980in}{0.150000in}}{\pgfqpoint{5.490039in}{5.490039in}}%
\pgfusepath{clip}%
\pgfsetbuttcap%
\pgfsetroundjoin%
\definecolor{currentfill}{rgb}{0.276022,0.044167,0.370164}%
\pgfsetfillcolor{currentfill}%
\pgfsetfillopacity{0.700000}%
\pgfsetlinewidth{0.000000pt}%
\definecolor{currentstroke}{rgb}{0.000000,0.000000,0.000000}%
\pgfsetstrokecolor{currentstroke}%
\pgfsetdash{}{0pt}%
\pgfpathmoveto{\pgfqpoint{4.144025in}{1.832873in}}%
\pgfpathlineto{\pgfqpoint{4.157368in}{1.829565in}}%
\pgfpathlineto{\pgfqpoint{4.170717in}{1.826380in}}%
\pgfpathlineto{\pgfqpoint{4.184074in}{1.823315in}}%
\pgfpathlineto{\pgfqpoint{4.197437in}{1.820372in}}%
\pgfpathlineto{\pgfqpoint{4.205092in}{1.829938in}}%
\pgfpathlineto{\pgfqpoint{4.212742in}{1.839538in}}%
\pgfpathlineto{\pgfqpoint{4.220387in}{1.849171in}}%
\pgfpathlineto{\pgfqpoint{4.228027in}{1.858836in}}%
\pgfpathlineto{\pgfqpoint{4.214673in}{1.861566in}}%
\pgfpathlineto{\pgfqpoint{4.201327in}{1.864418in}}%
\pgfpathlineto{\pgfqpoint{4.187988in}{1.867391in}}%
\pgfpathlineto{\pgfqpoint{4.174655in}{1.870486in}}%
\pgfpathlineto{\pgfqpoint{4.167005in}{1.861028in}}%
\pgfpathlineto{\pgfqpoint{4.159350in}{1.851606in}}%
\pgfpathlineto{\pgfqpoint{4.151690in}{1.842220in}}%
\pgfpathlineto{\pgfqpoint{4.144025in}{1.832873in}}%
\pgfpathclose%
\pgfusepath{fill}%
\end{pgfscope}%
\begin{pgfscope}%
\pgfpathrectangle{\pgfqpoint{1.254980in}{0.150000in}}{\pgfqpoint{5.490039in}{5.490039in}}%
\pgfusepath{clip}%
\pgfsetbuttcap%
\pgfsetroundjoin%
\definecolor{currentfill}{rgb}{0.274952,0.037752,0.364543}%
\pgfsetfillcolor{currentfill}%
\pgfsetfillopacity{0.700000}%
\pgfsetlinewidth{0.000000pt}%
\definecolor{currentstroke}{rgb}{0.000000,0.000000,0.000000}%
\pgfsetstrokecolor{currentstroke}%
\pgfsetdash{}{0pt}%
\pgfpathmoveto{\pgfqpoint{3.785199in}{1.831289in}}%
\pgfpathlineto{\pgfqpoint{3.798476in}{1.824736in}}%
\pgfpathlineto{\pgfqpoint{3.811758in}{1.818312in}}%
\pgfpathlineto{\pgfqpoint{3.825043in}{1.812017in}}%
\pgfpathlineto{\pgfqpoint{3.838333in}{1.805851in}}%
\pgfpathlineto{\pgfqpoint{3.846118in}{1.813532in}}%
\pgfpathlineto{\pgfqpoint{3.853897in}{1.821290in}}%
\pgfpathlineto{\pgfqpoint{3.861670in}{1.829122in}}%
\pgfpathlineto{\pgfqpoint{3.869437in}{1.837026in}}%
\pgfpathlineto{\pgfqpoint{3.856163in}{1.842931in}}%
\pgfpathlineto{\pgfqpoint{3.842893in}{1.848965in}}%
\pgfpathlineto{\pgfqpoint{3.829627in}{1.855127in}}%
\pgfpathlineto{\pgfqpoint{3.816366in}{1.861419in}}%
\pgfpathlineto{\pgfqpoint{3.808583in}{1.853770in}}%
\pgfpathlineto{\pgfqpoint{3.800795in}{1.846197in}}%
\pgfpathlineto{\pgfqpoint{3.793000in}{1.838703in}}%
\pgfpathlineto{\pgfqpoint{3.785199in}{1.831289in}}%
\pgfpathclose%
\pgfusepath{fill}%
\end{pgfscope}%
\begin{pgfscope}%
\pgfpathrectangle{\pgfqpoint{1.254980in}{0.150000in}}{\pgfqpoint{5.490039in}{5.490039in}}%
\pgfusepath{clip}%
\pgfsetbuttcap%
\pgfsetroundjoin%
\definecolor{currentfill}{rgb}{0.266580,0.228262,0.514349}%
\pgfsetfillcolor{currentfill}%
\pgfsetfillopacity{0.700000}%
\pgfsetlinewidth{0.000000pt}%
\definecolor{currentstroke}{rgb}{0.000000,0.000000,0.000000}%
\pgfsetstrokecolor{currentstroke}%
\pgfsetdash{}{0pt}%
\pgfpathmoveto{\pgfqpoint{3.106073in}{2.212851in}}%
\pgfpathlineto{\pgfqpoint{3.119373in}{2.199164in}}%
\pgfpathlineto{\pgfqpoint{3.132671in}{2.185633in}}%
\pgfpathlineto{\pgfqpoint{3.145967in}{2.172259in}}%
\pgfpathlineto{\pgfqpoint{3.159262in}{2.159039in}}%
\pgfpathlineto{\pgfqpoint{3.167383in}{2.162105in}}%
\pgfpathlineto{\pgfqpoint{3.175495in}{2.165321in}}%
\pgfpathlineto{\pgfqpoint{3.183596in}{2.168685in}}%
\pgfpathlineto{\pgfqpoint{3.191687in}{2.172194in}}%
\pgfpathlineto{\pgfqpoint{3.178421in}{2.185093in}}%
\pgfpathlineto{\pgfqpoint{3.165153in}{2.198147in}}%
\pgfpathlineto{\pgfqpoint{3.151884in}{2.211356in}}%
\pgfpathlineto{\pgfqpoint{3.138613in}{2.224722in}}%
\pgfpathlineto{\pgfqpoint{3.130494in}{2.221528in}}%
\pgfpathlineto{\pgfqpoint{3.122364in}{2.218483in}}%
\pgfpathlineto{\pgfqpoint{3.114224in}{2.215589in}}%
\pgfpathlineto{\pgfqpoint{3.106073in}{2.212851in}}%
\pgfpathclose%
\pgfusepath{fill}%
\end{pgfscope}%
\begin{pgfscope}%
\pgfpathrectangle{\pgfqpoint{1.254980in}{0.150000in}}{\pgfqpoint{5.490039in}{5.490039in}}%
\pgfusepath{clip}%
\pgfsetbuttcap%
\pgfsetroundjoin%
\definecolor{currentfill}{rgb}{0.273809,0.031497,0.358853}%
\pgfsetfillcolor{currentfill}%
\pgfsetfillopacity{0.700000}%
\pgfsetlinewidth{0.000000pt}%
\definecolor{currentstroke}{rgb}{0.000000,0.000000,0.000000}%
\pgfsetstrokecolor{currentstroke}%
\pgfsetdash{}{0pt}%
\pgfpathmoveto{\pgfqpoint{3.922580in}{1.814681in}}%
\pgfpathlineto{\pgfqpoint{3.935878in}{1.809411in}}%
\pgfpathlineto{\pgfqpoint{3.949181in}{1.804268in}}%
\pgfpathlineto{\pgfqpoint{3.962489in}{1.799250in}}%
\pgfpathlineto{\pgfqpoint{3.975803in}{1.794357in}}%
\pgfpathlineto{\pgfqpoint{3.983536in}{1.802834in}}%
\pgfpathlineto{\pgfqpoint{3.991263in}{1.811371in}}%
\pgfpathlineto{\pgfqpoint{3.998984in}{1.819967in}}%
\pgfpathlineto{\pgfqpoint{4.006700in}{1.828619in}}%
\pgfpathlineto{\pgfqpoint{3.993400in}{1.833267in}}%
\pgfpathlineto{\pgfqpoint{3.980105in}{1.838040in}}%
\pgfpathlineto{\pgfqpoint{3.966815in}{1.842939in}}%
\pgfpathlineto{\pgfqpoint{3.953531in}{1.847963in}}%
\pgfpathlineto{\pgfqpoint{3.945802in}{1.839550in}}%
\pgfpathlineto{\pgfqpoint{3.938067in}{1.831197in}}%
\pgfpathlineto{\pgfqpoint{3.930327in}{1.822906in}}%
\pgfpathlineto{\pgfqpoint{3.922580in}{1.814681in}}%
\pgfpathclose%
\pgfusepath{fill}%
\end{pgfscope}%
\begin{pgfscope}%
\pgfpathrectangle{\pgfqpoint{1.254980in}{0.150000in}}{\pgfqpoint{5.490039in}{5.490039in}}%
\pgfusepath{clip}%
\pgfsetbuttcap%
\pgfsetroundjoin%
\definecolor{currentfill}{rgb}{0.221989,0.339161,0.548752}%
\pgfsetfillcolor{currentfill}%
\pgfsetfillopacity{0.700000}%
\pgfsetlinewidth{0.000000pt}%
\definecolor{currentstroke}{rgb}{0.000000,0.000000,0.000000}%
\pgfsetstrokecolor{currentstroke}%
\pgfsetdash{}{0pt}%
\pgfpathmoveto{\pgfqpoint{5.236576in}{2.408268in}}%
\pgfpathlineto{\pgfqpoint{5.250352in}{2.412555in}}%
\pgfpathlineto{\pgfqpoint{5.264140in}{2.416954in}}%
\pgfpathlineto{\pgfqpoint{5.277942in}{2.421467in}}%
\pgfpathlineto{\pgfqpoint{5.291757in}{2.426093in}}%
\pgfpathlineto{\pgfqpoint{5.299062in}{2.436267in}}%
\pgfpathlineto{\pgfqpoint{5.306362in}{2.446387in}}%
\pgfpathlineto{\pgfqpoint{5.313655in}{2.456454in}}%
\pgfpathlineto{\pgfqpoint{5.320942in}{2.466467in}}%
\pgfpathlineto{\pgfqpoint{5.307133in}{2.461836in}}%
\pgfpathlineto{\pgfqpoint{5.293338in}{2.457319in}}%
\pgfpathlineto{\pgfqpoint{5.279556in}{2.452915in}}%
\pgfpathlineto{\pgfqpoint{5.265788in}{2.448624in}}%
\pgfpathlineto{\pgfqpoint{5.258494in}{2.438609in}}%
\pgfpathlineto{\pgfqpoint{5.251194in}{2.428545in}}%
\pgfpathlineto{\pgfqpoint{5.243888in}{2.418432in}}%
\pgfpathlineto{\pgfqpoint{5.236576in}{2.408268in}}%
\pgfpathclose%
\pgfusepath{fill}%
\end{pgfscope}%
\begin{pgfscope}%
\pgfpathrectangle{\pgfqpoint{1.254980in}{0.150000in}}{\pgfqpoint{5.490039in}{5.490039in}}%
\pgfusepath{clip}%
\pgfsetbuttcap%
\pgfsetroundjoin%
\definecolor{currentfill}{rgb}{0.187231,0.414746,0.556547}%
\pgfsetfillcolor{currentfill}%
\pgfsetfillopacity{0.700000}%
\pgfsetlinewidth{0.000000pt}%
\definecolor{currentstroke}{rgb}{0.000000,0.000000,0.000000}%
\pgfsetstrokecolor{currentstroke}%
\pgfsetdash{}{0pt}%
\pgfpathmoveto{\pgfqpoint{2.732261in}{2.663326in}}%
\pgfpathlineto{\pgfqpoint{2.745677in}{2.644878in}}%
\pgfpathlineto{\pgfqpoint{2.759088in}{2.626614in}}%
\pgfpathlineto{\pgfqpoint{2.772492in}{2.608532in}}%
\pgfpathlineto{\pgfqpoint{2.785890in}{2.590633in}}%
\pgfpathlineto{\pgfqpoint{2.794239in}{2.591369in}}%
\pgfpathlineto{\pgfqpoint{2.802575in}{2.592285in}}%
\pgfpathlineto{\pgfqpoint{2.810899in}{2.593381in}}%
\pgfpathlineto{\pgfqpoint{2.819209in}{2.594653in}}%
\pgfpathlineto{\pgfqpoint{2.805847in}{2.612219in}}%
\pgfpathlineto{\pgfqpoint{2.792479in}{2.629966in}}%
\pgfpathlineto{\pgfqpoint{2.779105in}{2.647895in}}%
\pgfpathlineto{\pgfqpoint{2.765726in}{2.666007in}}%
\pgfpathlineto{\pgfqpoint{2.757379in}{2.665063in}}%
\pgfpathlineto{\pgfqpoint{2.749020in}{2.664300in}}%
\pgfpathlineto{\pgfqpoint{2.740647in}{2.663720in}}%
\pgfpathlineto{\pgfqpoint{2.732261in}{2.663326in}}%
\pgfpathclose%
\pgfusepath{fill}%
\end{pgfscope}%
\begin{pgfscope}%
\pgfpathrectangle{\pgfqpoint{1.254980in}{0.150000in}}{\pgfqpoint{5.490039in}{5.490039in}}%
\pgfusepath{clip}%
\pgfsetbuttcap%
\pgfsetroundjoin%
\definecolor{currentfill}{rgb}{0.163625,0.471133,0.558148}%
\pgfsetfillcolor{currentfill}%
\pgfsetfillopacity{0.700000}%
\pgfsetlinewidth{0.000000pt}%
\definecolor{currentstroke}{rgb}{0.000000,0.000000,0.000000}%
\pgfsetstrokecolor{currentstroke}%
\pgfsetdash{}{0pt}%
\pgfpathmoveto{\pgfqpoint{5.687620in}{2.742048in}}%
\pgfpathlineto{\pgfqpoint{5.701631in}{2.748363in}}%
\pgfpathlineto{\pgfqpoint{5.715657in}{2.754791in}}%
\pgfpathlineto{\pgfqpoint{5.729698in}{2.761330in}}%
\pgfpathlineto{\pgfqpoint{5.743755in}{2.767982in}}%
\pgfpathlineto{\pgfqpoint{5.750866in}{2.776556in}}%
\pgfpathlineto{\pgfqpoint{5.757970in}{2.785072in}}%
\pgfpathlineto{\pgfqpoint{5.765068in}{2.793530in}}%
\pgfpathlineto{\pgfqpoint{5.772158in}{2.801931in}}%
\pgfpathlineto{\pgfqpoint{5.758112in}{2.795376in}}%
\pgfpathlineto{\pgfqpoint{5.744082in}{2.788932in}}%
\pgfpathlineto{\pgfqpoint{5.730067in}{2.782601in}}%
\pgfpathlineto{\pgfqpoint{5.716067in}{2.776381in}}%
\pgfpathlineto{\pgfqpoint{5.708965in}{2.767877in}}%
\pgfpathlineto{\pgfqpoint{5.701856in}{2.759321in}}%
\pgfpathlineto{\pgfqpoint{5.694741in}{2.750712in}}%
\pgfpathlineto{\pgfqpoint{5.687620in}{2.742048in}}%
\pgfpathclose%
\pgfusepath{fill}%
\end{pgfscope}%
\begin{pgfscope}%
\pgfpathrectangle{\pgfqpoint{1.254980in}{0.150000in}}{\pgfqpoint{5.490039in}{5.490039in}}%
\pgfusepath{clip}%
\pgfsetbuttcap%
\pgfsetroundjoin%
\definecolor{currentfill}{rgb}{0.282656,0.100196,0.422160}%
\pgfsetfillcolor{currentfill}%
\pgfsetfillopacity{0.700000}%
\pgfsetlinewidth{0.000000pt}%
\definecolor{currentstroke}{rgb}{0.000000,0.000000,0.000000}%
\pgfsetstrokecolor{currentstroke}%
\pgfsetdash{}{0pt}%
\pgfpathmoveto{\pgfqpoint{3.456891in}{1.945556in}}%
\pgfpathlineto{\pgfqpoint{3.470155in}{1.935739in}}%
\pgfpathlineto{\pgfqpoint{3.483421in}{1.926062in}}%
\pgfpathlineto{\pgfqpoint{3.496688in}{1.916524in}}%
\pgfpathlineto{\pgfqpoint{3.509956in}{1.907125in}}%
\pgfpathlineto{\pgfqpoint{3.517890in}{1.912611in}}%
\pgfpathlineto{\pgfqpoint{3.525817in}{1.918212in}}%
\pgfpathlineto{\pgfqpoint{3.533735in}{1.923925in}}%
\pgfpathlineto{\pgfqpoint{3.541646in}{1.929748in}}%
\pgfpathlineto{\pgfqpoint{3.528399in}{1.938851in}}%
\pgfpathlineto{\pgfqpoint{3.515154in}{1.948092in}}%
\pgfpathlineto{\pgfqpoint{3.501910in}{1.957472in}}%
\pgfpathlineto{\pgfqpoint{3.488668in}{1.966992in}}%
\pgfpathlineto{\pgfqpoint{3.480736in}{1.961459in}}%
\pgfpathlineto{\pgfqpoint{3.472796in}{1.956041in}}%
\pgfpathlineto{\pgfqpoint{3.464848in}{1.950739in}}%
\pgfpathlineto{\pgfqpoint{3.456891in}{1.945556in}}%
\pgfpathclose%
\pgfusepath{fill}%
\end{pgfscope}%
\begin{pgfscope}%
\pgfpathrectangle{\pgfqpoint{1.254980in}{0.150000in}}{\pgfqpoint{5.490039in}{5.490039in}}%
\pgfusepath{clip}%
\pgfsetbuttcap%
\pgfsetroundjoin%
\definecolor{currentfill}{rgb}{0.277941,0.056324,0.381191}%
\pgfsetfillcolor{currentfill}%
\pgfsetfillopacity{0.700000}%
\pgfsetlinewidth{0.000000pt}%
\definecolor{currentstroke}{rgb}{0.000000,0.000000,0.000000}%
\pgfsetstrokecolor{currentstroke}%
\pgfsetdash{}{0pt}%
\pgfpathmoveto{\pgfqpoint{3.647698in}{1.861840in}}%
\pgfpathlineto{\pgfqpoint{3.660966in}{1.853957in}}%
\pgfpathlineto{\pgfqpoint{3.674237in}{1.846208in}}%
\pgfpathlineto{\pgfqpoint{3.687511in}{1.838592in}}%
\pgfpathlineto{\pgfqpoint{3.700787in}{1.831107in}}%
\pgfpathlineto{\pgfqpoint{3.708633in}{1.837894in}}%
\pgfpathlineto{\pgfqpoint{3.716471in}{1.844773in}}%
\pgfpathlineto{\pgfqpoint{3.724302in}{1.851744in}}%
\pgfpathlineto{\pgfqpoint{3.732127in}{1.858803in}}%
\pgfpathlineto{\pgfqpoint{3.718868in}{1.866009in}}%
\pgfpathlineto{\pgfqpoint{3.705612in}{1.873347in}}%
\pgfpathlineto{\pgfqpoint{3.692360in}{1.880818in}}%
\pgfpathlineto{\pgfqpoint{3.679110in}{1.888421in}}%
\pgfpathlineto{\pgfqpoint{3.671268in}{1.881634in}}%
\pgfpathlineto{\pgfqpoint{3.663418in}{1.874940in}}%
\pgfpathlineto{\pgfqpoint{3.655562in}{1.868341in}}%
\pgfpathlineto{\pgfqpoint{3.647698in}{1.861840in}}%
\pgfpathclose%
\pgfusepath{fill}%
\end{pgfscope}%
\begin{pgfscope}%
\pgfpathrectangle{\pgfqpoint{1.254980in}{0.150000in}}{\pgfqpoint{5.490039in}{5.490039in}}%
\pgfusepath{clip}%
\pgfsetbuttcap%
\pgfsetroundjoin%
\definecolor{currentfill}{rgb}{0.273006,0.204520,0.501721}%
\pgfsetfillcolor{currentfill}%
\pgfsetfillopacity{0.700000}%
\pgfsetlinewidth{0.000000pt}%
\definecolor{currentstroke}{rgb}{0.000000,0.000000,0.000000}%
\pgfsetstrokecolor{currentstroke}%
\pgfsetdash{}{0pt}%
\pgfpathmoveto{\pgfqpoint{3.159262in}{2.159039in}}%
\pgfpathlineto{\pgfqpoint{3.172555in}{2.145973in}}%
\pgfpathlineto{\pgfqpoint{3.185846in}{2.133062in}}%
\pgfpathlineto{\pgfqpoint{3.199136in}{2.120302in}}%
\pgfpathlineto{\pgfqpoint{3.212424in}{2.107695in}}%
\pgfpathlineto{\pgfqpoint{3.220518in}{2.111087in}}%
\pgfpathlineto{\pgfqpoint{3.228601in}{2.114625in}}%
\pgfpathlineto{\pgfqpoint{3.236675in}{2.118306in}}%
\pgfpathlineto{\pgfqpoint{3.244738in}{2.122128in}}%
\pgfpathlineto{\pgfqpoint{3.231477in}{2.134416in}}%
\pgfpathlineto{\pgfqpoint{3.218215in}{2.146856in}}%
\pgfpathlineto{\pgfqpoint{3.204952in}{2.159448in}}%
\pgfpathlineto{\pgfqpoint{3.191687in}{2.172194in}}%
\pgfpathlineto{\pgfqpoint{3.183596in}{2.168685in}}%
\pgfpathlineto{\pgfqpoint{3.175495in}{2.165321in}}%
\pgfpathlineto{\pgfqpoint{3.167383in}{2.162105in}}%
\pgfpathlineto{\pgfqpoint{3.159262in}{2.159039in}}%
\pgfpathclose%
\pgfusepath{fill}%
\end{pgfscope}%
\begin{pgfscope}%
\pgfpathrectangle{\pgfqpoint{1.254980in}{0.150000in}}{\pgfqpoint{5.490039in}{5.490039in}}%
\pgfusepath{clip}%
\pgfsetbuttcap%
\pgfsetroundjoin%
\definecolor{currentfill}{rgb}{0.282290,0.145912,0.461510}%
\pgfsetfillcolor{currentfill}%
\pgfsetfillopacity{0.700000}%
\pgfsetlinewidth{0.000000pt}%
\definecolor{currentstroke}{rgb}{0.000000,0.000000,0.000000}%
\pgfsetstrokecolor{currentstroke}%
\pgfsetdash{}{0pt}%
\pgfpathmoveto{\pgfqpoint{4.617666in}{2.000917in}}%
\pgfpathlineto{\pgfqpoint{4.631165in}{2.001371in}}%
\pgfpathlineto{\pgfqpoint{4.644674in}{2.001941in}}%
\pgfpathlineto{\pgfqpoint{4.658193in}{2.002627in}}%
\pgfpathlineto{\pgfqpoint{4.671721in}{2.003429in}}%
\pgfpathlineto{\pgfqpoint{4.679234in}{2.014259in}}%
\pgfpathlineto{\pgfqpoint{4.686741in}{2.025074in}}%
\pgfpathlineto{\pgfqpoint{4.694244in}{2.035874in}}%
\pgfpathlineto{\pgfqpoint{4.701742in}{2.046657in}}%
\pgfpathlineto{\pgfqpoint{4.688219in}{2.045720in}}%
\pgfpathlineto{\pgfqpoint{4.674706in}{2.044900in}}%
\pgfpathlineto{\pgfqpoint{4.661203in}{2.044196in}}%
\pgfpathlineto{\pgfqpoint{4.647711in}{2.043609in}}%
\pgfpathlineto{\pgfqpoint{4.640207in}{2.032954in}}%
\pgfpathlineto{\pgfqpoint{4.632698in}{2.022286in}}%
\pgfpathlineto{\pgfqpoint{4.625184in}{2.011607in}}%
\pgfpathlineto{\pgfqpoint{4.617666in}{2.000917in}}%
\pgfpathclose%
\pgfusepath{fill}%
\end{pgfscope}%
\begin{pgfscope}%
\pgfpathrectangle{\pgfqpoint{1.254980in}{0.150000in}}{\pgfqpoint{5.490039in}{5.490039in}}%
\pgfusepath{clip}%
\pgfsetbuttcap%
\pgfsetroundjoin%
\definecolor{currentfill}{rgb}{0.283229,0.120777,0.440584}%
\pgfsetfillcolor{currentfill}%
\pgfsetfillopacity{0.700000}%
\pgfsetlinewidth{0.000000pt}%
\definecolor{currentstroke}{rgb}{0.000000,0.000000,0.000000}%
\pgfsetstrokecolor{currentstroke}%
\pgfsetdash{}{0pt}%
\pgfpathmoveto{\pgfqpoint{4.533618in}{1.958022in}}%
\pgfpathlineto{\pgfqpoint{4.547085in}{1.957859in}}%
\pgfpathlineto{\pgfqpoint{4.560562in}{1.957813in}}%
\pgfpathlineto{\pgfqpoint{4.574049in}{1.957884in}}%
\pgfpathlineto{\pgfqpoint{4.587544in}{1.958071in}}%
\pgfpathlineto{\pgfqpoint{4.595082in}{1.968794in}}%
\pgfpathlineto{\pgfqpoint{4.602615in}{1.979510in}}%
\pgfpathlineto{\pgfqpoint{4.610143in}{1.990218in}}%
\pgfpathlineto{\pgfqpoint{4.617666in}{2.000917in}}%
\pgfpathlineto{\pgfqpoint{4.604176in}{2.000580in}}%
\pgfpathlineto{\pgfqpoint{4.590697in}{2.000359in}}%
\pgfpathlineto{\pgfqpoint{4.577226in}{2.000256in}}%
\pgfpathlineto{\pgfqpoint{4.563765in}{2.000270in}}%
\pgfpathlineto{\pgfqpoint{4.556236in}{1.989714in}}%
\pgfpathlineto{\pgfqpoint{4.548701in}{1.979153in}}%
\pgfpathlineto{\pgfqpoint{4.541162in}{1.968589in}}%
\pgfpathlineto{\pgfqpoint{4.533618in}{1.958022in}}%
\pgfpathclose%
\pgfusepath{fill}%
\end{pgfscope}%
\begin{pgfscope}%
\pgfpathrectangle{\pgfqpoint{1.254980in}{0.150000in}}{\pgfqpoint{5.490039in}{5.490039in}}%
\pgfusepath{clip}%
\pgfsetbuttcap%
\pgfsetroundjoin%
\definecolor{currentfill}{rgb}{0.120092,0.600104,0.542530}%
\pgfsetfillcolor{currentfill}%
\pgfsetfillopacity{0.700000}%
\pgfsetlinewidth{0.000000pt}%
\definecolor{currentstroke}{rgb}{0.000000,0.000000,0.000000}%
\pgfsetstrokecolor{currentstroke}%
\pgfsetdash{}{0pt}%
\pgfpathmoveto{\pgfqpoint{2.442353in}{3.159028in}}%
\pgfpathlineto{\pgfqpoint{2.455931in}{3.136124in}}%
\pgfpathlineto{\pgfqpoint{2.469498in}{3.113437in}}%
\pgfpathlineto{\pgfqpoint{2.483055in}{3.090964in}}%
\pgfpathlineto{\pgfqpoint{2.496602in}{3.068704in}}%
\pgfpathlineto{\pgfqpoint{2.505130in}{3.068173in}}%
\pgfpathlineto{\pgfqpoint{2.513642in}{3.067840in}}%
\pgfpathlineto{\pgfqpoint{2.522140in}{3.067705in}}%
\pgfpathlineto{\pgfqpoint{2.530623in}{3.067763in}}%
\pgfpathlineto{\pgfqpoint{2.517118in}{3.089692in}}%
\pgfpathlineto{\pgfqpoint{2.503603in}{3.111833in}}%
\pgfpathlineto{\pgfqpoint{2.490077in}{3.134188in}}%
\pgfpathlineto{\pgfqpoint{2.476542in}{3.156757in}}%
\pgfpathlineto{\pgfqpoint{2.468018in}{3.157024in}}%
\pgfpathlineto{\pgfqpoint{2.459478in}{3.157490in}}%
\pgfpathlineto{\pgfqpoint{2.450924in}{3.158157in}}%
\pgfpathlineto{\pgfqpoint{2.442353in}{3.159028in}}%
\pgfpathclose%
\pgfusepath{fill}%
\end{pgfscope}%
\begin{pgfscope}%
\pgfpathrectangle{\pgfqpoint{1.254980in}{0.150000in}}{\pgfqpoint{5.490039in}{5.490039in}}%
\pgfusepath{clip}%
\pgfsetbuttcap%
\pgfsetroundjoin%
\definecolor{currentfill}{rgb}{0.279574,0.170599,0.479997}%
\pgfsetfillcolor{currentfill}%
\pgfsetfillopacity{0.700000}%
\pgfsetlinewidth{0.000000pt}%
\definecolor{currentstroke}{rgb}{0.000000,0.000000,0.000000}%
\pgfsetstrokecolor{currentstroke}%
\pgfsetdash{}{0pt}%
\pgfpathmoveto{\pgfqpoint{4.701742in}{2.046657in}}%
\pgfpathlineto{\pgfqpoint{4.715275in}{2.047709in}}%
\pgfpathlineto{\pgfqpoint{4.728819in}{2.048877in}}%
\pgfpathlineto{\pgfqpoint{4.742372in}{2.050160in}}%
\pgfpathlineto{\pgfqpoint{4.755937in}{2.051559in}}%
\pgfpathlineto{\pgfqpoint{4.763424in}{2.062449in}}%
\pgfpathlineto{\pgfqpoint{4.770907in}{2.073317in}}%
\pgfpathlineto{\pgfqpoint{4.778384in}{2.084163in}}%
\pgfpathlineto{\pgfqpoint{4.785857in}{2.094985in}}%
\pgfpathlineto{\pgfqpoint{4.772298in}{2.093468in}}%
\pgfpathlineto{\pgfqpoint{4.758750in}{2.092066in}}%
\pgfpathlineto{\pgfqpoint{4.745212in}{2.090780in}}%
\pgfpathlineto{\pgfqpoint{4.731685in}{2.089610in}}%
\pgfpathlineto{\pgfqpoint{4.724207in}{2.078900in}}%
\pgfpathlineto{\pgfqpoint{4.716723in}{2.068171in}}%
\pgfpathlineto{\pgfqpoint{4.709235in}{2.057423in}}%
\pgfpathlineto{\pgfqpoint{4.701742in}{2.046657in}}%
\pgfpathclose%
\pgfusepath{fill}%
\end{pgfscope}%
\begin{pgfscope}%
\pgfpathrectangle{\pgfqpoint{1.254980in}{0.150000in}}{\pgfqpoint{5.490039in}{5.490039in}}%
\pgfusepath{clip}%
\pgfsetbuttcap%
\pgfsetroundjoin%
\definecolor{currentfill}{rgb}{0.282656,0.100196,0.422160}%
\pgfsetfillcolor{currentfill}%
\pgfsetfillopacity{0.700000}%
\pgfsetlinewidth{0.000000pt}%
\definecolor{currentstroke}{rgb}{0.000000,0.000000,0.000000}%
\pgfsetstrokecolor{currentstroke}%
\pgfsetdash{}{0pt}%
\pgfpathmoveto{\pgfqpoint{4.449586in}{1.918239in}}%
\pgfpathlineto{\pgfqpoint{4.463025in}{1.917440in}}%
\pgfpathlineto{\pgfqpoint{4.476472in}{1.916759in}}%
\pgfpathlineto{\pgfqpoint{4.489929in}{1.916196in}}%
\pgfpathlineto{\pgfqpoint{4.503394in}{1.915750in}}%
\pgfpathlineto{\pgfqpoint{4.510957in}{1.926316in}}%
\pgfpathlineto{\pgfqpoint{4.518515in}{1.936885in}}%
\pgfpathlineto{\pgfqpoint{4.526069in}{1.947454in}}%
\pgfpathlineto{\pgfqpoint{4.533618in}{1.958022in}}%
\pgfpathlineto{\pgfqpoint{4.520159in}{1.958303in}}%
\pgfpathlineto{\pgfqpoint{4.506710in}{1.958701in}}%
\pgfpathlineto{\pgfqpoint{4.493270in}{1.959217in}}%
\pgfpathlineto{\pgfqpoint{4.479838in}{1.959850in}}%
\pgfpathlineto{\pgfqpoint{4.472282in}{1.949441in}}%
\pgfpathlineto{\pgfqpoint{4.464721in}{1.939036in}}%
\pgfpathlineto{\pgfqpoint{4.457156in}{1.928635in}}%
\pgfpathlineto{\pgfqpoint{4.449586in}{1.918239in}}%
\pgfpathclose%
\pgfusepath{fill}%
\end{pgfscope}%
\begin{pgfscope}%
\pgfpathrectangle{\pgfqpoint{1.254980in}{0.150000in}}{\pgfqpoint{5.490039in}{5.490039in}}%
\pgfusepath{clip}%
\pgfsetbuttcap%
\pgfsetroundjoin%
\definecolor{currentfill}{rgb}{0.174274,0.445044,0.557792}%
\pgfsetfillcolor{currentfill}%
\pgfsetfillopacity{0.700000}%
\pgfsetlinewidth{0.000000pt}%
\definecolor{currentstroke}{rgb}{0.000000,0.000000,0.000000}%
\pgfsetstrokecolor{currentstroke}%
\pgfsetdash{}{0pt}%
\pgfpathmoveto{\pgfqpoint{2.678528in}{2.738983in}}%
\pgfpathlineto{\pgfqpoint{2.691972in}{2.719786in}}%
\pgfpathlineto{\pgfqpoint{2.705408in}{2.700779in}}%
\pgfpathlineto{\pgfqpoint{2.718838in}{2.681959in}}%
\pgfpathlineto{\pgfqpoint{2.732261in}{2.663326in}}%
\pgfpathlineto{\pgfqpoint{2.740647in}{2.663720in}}%
\pgfpathlineto{\pgfqpoint{2.749020in}{2.664300in}}%
\pgfpathlineto{\pgfqpoint{2.757379in}{2.665063in}}%
\pgfpathlineto{\pgfqpoint{2.765726in}{2.666007in}}%
\pgfpathlineto{\pgfqpoint{2.752340in}{2.684305in}}%
\pgfpathlineto{\pgfqpoint{2.738947in}{2.702788in}}%
\pgfpathlineto{\pgfqpoint{2.725549in}{2.721458in}}%
\pgfpathlineto{\pgfqpoint{2.712143in}{2.740316in}}%
\pgfpathlineto{\pgfqpoint{2.703760in}{2.739702in}}%
\pgfpathlineto{\pgfqpoint{2.695363in}{2.739273in}}%
\pgfpathlineto{\pgfqpoint{2.686953in}{2.739033in}}%
\pgfpathlineto{\pgfqpoint{2.678528in}{2.738983in}}%
\pgfpathclose%
\pgfusepath{fill}%
\end{pgfscope}%
\begin{pgfscope}%
\pgfpathrectangle{\pgfqpoint{1.254980in}{0.150000in}}{\pgfqpoint{5.490039in}{5.490039in}}%
\pgfusepath{clip}%
\pgfsetbuttcap%
\pgfsetroundjoin%
\definecolor{currentfill}{rgb}{0.210503,0.363727,0.552206}%
\pgfsetfillcolor{currentfill}%
\pgfsetfillopacity{0.700000}%
\pgfsetlinewidth{0.000000pt}%
\definecolor{currentstroke}{rgb}{0.000000,0.000000,0.000000}%
\pgfsetstrokecolor{currentstroke}%
\pgfsetdash{}{0pt}%
\pgfpathmoveto{\pgfqpoint{5.320942in}{2.466467in}}%
\pgfpathlineto{\pgfqpoint{5.334764in}{2.471210in}}%
\pgfpathlineto{\pgfqpoint{5.348600in}{2.476066in}}%
\pgfpathlineto{\pgfqpoint{5.362450in}{2.481035in}}%
\pgfpathlineto{\pgfqpoint{5.376314in}{2.486118in}}%
\pgfpathlineto{\pgfqpoint{5.383588in}{2.496072in}}%
\pgfpathlineto{\pgfqpoint{5.390857in}{2.505969in}}%
\pgfpathlineto{\pgfqpoint{5.398119in}{2.515809in}}%
\pgfpathlineto{\pgfqpoint{5.405375in}{2.525594in}}%
\pgfpathlineto{\pgfqpoint{5.391519in}{2.520524in}}%
\pgfpathlineto{\pgfqpoint{5.377676in}{2.515567in}}%
\pgfpathlineto{\pgfqpoint{5.363847in}{2.510722in}}%
\pgfpathlineto{\pgfqpoint{5.350032in}{2.505991in}}%
\pgfpathlineto{\pgfqpoint{5.342769in}{2.496188in}}%
\pgfpathlineto{\pgfqpoint{5.335499in}{2.486334in}}%
\pgfpathlineto{\pgfqpoint{5.328224in}{2.476426in}}%
\pgfpathlineto{\pgfqpoint{5.320942in}{2.466467in}}%
\pgfpathclose%
\pgfusepath{fill}%
\end{pgfscope}%
\begin{pgfscope}%
\pgfpathrectangle{\pgfqpoint{1.254980in}{0.150000in}}{\pgfqpoint{5.490039in}{5.490039in}}%
\pgfusepath{clip}%
\pgfsetbuttcap%
\pgfsetroundjoin%
\definecolor{currentfill}{rgb}{0.275191,0.194905,0.496005}%
\pgfsetfillcolor{currentfill}%
\pgfsetfillopacity{0.700000}%
\pgfsetlinewidth{0.000000pt}%
\definecolor{currentstroke}{rgb}{0.000000,0.000000,0.000000}%
\pgfsetstrokecolor{currentstroke}%
\pgfsetdash{}{0pt}%
\pgfpathmoveto{\pgfqpoint{4.785857in}{2.094985in}}%
\pgfpathlineto{\pgfqpoint{4.799426in}{2.096617in}}%
\pgfpathlineto{\pgfqpoint{4.813007in}{2.098365in}}%
\pgfpathlineto{\pgfqpoint{4.826598in}{2.100227in}}%
\pgfpathlineto{\pgfqpoint{4.840200in}{2.102205in}}%
\pgfpathlineto{\pgfqpoint{4.847663in}{2.113111in}}%
\pgfpathlineto{\pgfqpoint{4.855120in}{2.123988in}}%
\pgfpathlineto{\pgfqpoint{4.862572in}{2.134836in}}%
\pgfpathlineto{\pgfqpoint{4.870019in}{2.145654in}}%
\pgfpathlineto{\pgfqpoint{4.856423in}{2.143574in}}%
\pgfpathlineto{\pgfqpoint{4.842837in}{2.141609in}}%
\pgfpathlineto{\pgfqpoint{4.829262in}{2.139760in}}%
\pgfpathlineto{\pgfqpoint{4.815698in}{2.138025in}}%
\pgfpathlineto{\pgfqpoint{4.808245in}{2.127303in}}%
\pgfpathlineto{\pgfqpoint{4.800787in}{2.116556in}}%
\pgfpathlineto{\pgfqpoint{4.793325in}{2.105783in}}%
\pgfpathlineto{\pgfqpoint{4.785857in}{2.094985in}}%
\pgfpathclose%
\pgfusepath{fill}%
\end{pgfscope}%
\begin{pgfscope}%
\pgfpathrectangle{\pgfqpoint{1.254980in}{0.150000in}}{\pgfqpoint{5.490039in}{5.490039in}}%
\pgfusepath{clip}%
\pgfsetbuttcap%
\pgfsetroundjoin%
\definecolor{currentfill}{rgb}{0.274952,0.037752,0.364543}%
\pgfsetfillcolor{currentfill}%
\pgfsetfillopacity{0.700000}%
\pgfsetlinewidth{0.000000pt}%
\definecolor{currentstroke}{rgb}{0.000000,0.000000,0.000000}%
\pgfsetstrokecolor{currentstroke}%
\pgfsetdash{}{0pt}%
\pgfpathmoveto{\pgfqpoint{4.059958in}{1.811273in}}%
\pgfpathlineto{\pgfqpoint{4.073287in}{1.807246in}}%
\pgfpathlineto{\pgfqpoint{4.086622in}{1.803342in}}%
\pgfpathlineto{\pgfqpoint{4.099963in}{1.799561in}}%
\pgfpathlineto{\pgfqpoint{4.113311in}{1.795902in}}%
\pgfpathlineto{\pgfqpoint{4.120997in}{1.805079in}}%
\pgfpathlineto{\pgfqpoint{4.128678in}{1.814301in}}%
\pgfpathlineto{\pgfqpoint{4.136354in}{1.823566in}}%
\pgfpathlineto{\pgfqpoint{4.144025in}{1.832873in}}%
\pgfpathlineto{\pgfqpoint{4.130688in}{1.836303in}}%
\pgfpathlineto{\pgfqpoint{4.117358in}{1.839855in}}%
\pgfpathlineto{\pgfqpoint{4.104035in}{1.843531in}}%
\pgfpathlineto{\pgfqpoint{4.090717in}{1.847329in}}%
\pgfpathlineto{\pgfqpoint{4.083035in}{1.838245in}}%
\pgfpathlineto{\pgfqpoint{4.075348in}{1.829206in}}%
\pgfpathlineto{\pgfqpoint{4.067656in}{1.820215in}}%
\pgfpathlineto{\pgfqpoint{4.059958in}{1.811273in}}%
\pgfpathclose%
\pgfusepath{fill}%
\end{pgfscope}%
\begin{pgfscope}%
\pgfpathrectangle{\pgfqpoint{1.254980in}{0.150000in}}{\pgfqpoint{5.490039in}{5.490039in}}%
\pgfusepath{clip}%
\pgfsetbuttcap%
\pgfsetroundjoin%
\definecolor{currentfill}{rgb}{0.278012,0.180367,0.486697}%
\pgfsetfillcolor{currentfill}%
\pgfsetfillopacity{0.700000}%
\pgfsetlinewidth{0.000000pt}%
\definecolor{currentstroke}{rgb}{0.000000,0.000000,0.000000}%
\pgfsetstrokecolor{currentstroke}%
\pgfsetdash{}{0pt}%
\pgfpathmoveto{\pgfqpoint{3.212424in}{2.107695in}}%
\pgfpathlineto{\pgfqpoint{3.225712in}{2.095239in}}%
\pgfpathlineto{\pgfqpoint{3.238998in}{2.082933in}}%
\pgfpathlineto{\pgfqpoint{3.252284in}{2.070777in}}%
\pgfpathlineto{\pgfqpoint{3.265569in}{2.058770in}}%
\pgfpathlineto{\pgfqpoint{3.273635in}{2.062487in}}%
\pgfpathlineto{\pgfqpoint{3.281692in}{2.066345in}}%
\pgfpathlineto{\pgfqpoint{3.289739in}{2.070342in}}%
\pgfpathlineto{\pgfqpoint{3.297777in}{2.074476in}}%
\pgfpathlineto{\pgfqpoint{3.284518in}{2.086165in}}%
\pgfpathlineto{\pgfqpoint{3.271259in}{2.098003in}}%
\pgfpathlineto{\pgfqpoint{3.257999in}{2.109990in}}%
\pgfpathlineto{\pgfqpoint{3.244738in}{2.122128in}}%
\pgfpathlineto{\pgfqpoint{3.236675in}{2.118306in}}%
\pgfpathlineto{\pgfqpoint{3.228601in}{2.114625in}}%
\pgfpathlineto{\pgfqpoint{3.220518in}{2.111087in}}%
\pgfpathlineto{\pgfqpoint{3.212424in}{2.107695in}}%
\pgfpathclose%
\pgfusepath{fill}%
\end{pgfscope}%
\begin{pgfscope}%
\pgfpathrectangle{\pgfqpoint{1.254980in}{0.150000in}}{\pgfqpoint{5.490039in}{5.490039in}}%
\pgfusepath{clip}%
\pgfsetbuttcap%
\pgfsetroundjoin%
\definecolor{currentfill}{rgb}{0.154815,0.493313,0.557840}%
\pgfsetfillcolor{currentfill}%
\pgfsetfillopacity{0.700000}%
\pgfsetlinewidth{0.000000pt}%
\definecolor{currentstroke}{rgb}{0.000000,0.000000,0.000000}%
\pgfsetstrokecolor{currentstroke}%
\pgfsetdash{}{0pt}%
\pgfpathmoveto{\pgfqpoint{5.772158in}{2.801931in}}%
\pgfpathlineto{\pgfqpoint{5.786220in}{2.808598in}}%
\pgfpathlineto{\pgfqpoint{5.800297in}{2.815377in}}%
\pgfpathlineto{\pgfqpoint{5.814390in}{2.822268in}}%
\pgfpathlineto{\pgfqpoint{5.828499in}{2.829270in}}%
\pgfpathlineto{\pgfqpoint{5.835572in}{2.837509in}}%
\pgfpathlineto{\pgfqpoint{5.842637in}{2.845690in}}%
\pgfpathlineto{\pgfqpoint{5.849696in}{2.853813in}}%
\pgfpathlineto{\pgfqpoint{5.856747in}{2.861881in}}%
\pgfpathlineto{\pgfqpoint{5.842650in}{2.854992in}}%
\pgfpathlineto{\pgfqpoint{5.828569in}{2.848215in}}%
\pgfpathlineto{\pgfqpoint{5.814504in}{2.841549in}}%
\pgfpathlineto{\pgfqpoint{5.800455in}{2.834995in}}%
\pgfpathlineto{\pgfqpoint{5.793390in}{2.826808in}}%
\pgfpathlineto{\pgfqpoint{5.786320in}{2.818569in}}%
\pgfpathlineto{\pgfqpoint{5.779242in}{2.810277in}}%
\pgfpathlineto{\pgfqpoint{5.772158in}{2.801931in}}%
\pgfpathclose%
\pgfusepath{fill}%
\end{pgfscope}%
\begin{pgfscope}%
\pgfpathrectangle{\pgfqpoint{1.254980in}{0.150000in}}{\pgfqpoint{5.490039in}{5.490039in}}%
\pgfusepath{clip}%
\pgfsetbuttcap%
\pgfsetroundjoin%
\definecolor{currentfill}{rgb}{0.281446,0.084320,0.407414}%
\pgfsetfillcolor{currentfill}%
\pgfsetfillopacity{0.700000}%
\pgfsetlinewidth{0.000000pt}%
\definecolor{currentstroke}{rgb}{0.000000,0.000000,0.000000}%
\pgfsetstrokecolor{currentstroke}%
\pgfsetdash{}{0pt}%
\pgfpathmoveto{\pgfqpoint{4.365556in}{1.881844in}}%
\pgfpathlineto{\pgfqpoint{4.378969in}{1.880390in}}%
\pgfpathlineto{\pgfqpoint{4.392390in}{1.879055in}}%
\pgfpathlineto{\pgfqpoint{4.405819in}{1.877838in}}%
\pgfpathlineto{\pgfqpoint{4.419257in}{1.876740in}}%
\pgfpathlineto{\pgfqpoint{4.426846in}{1.887100in}}%
\pgfpathlineto{\pgfqpoint{4.434431in}{1.897471in}}%
\pgfpathlineto{\pgfqpoint{4.442011in}{1.907851in}}%
\pgfpathlineto{\pgfqpoint{4.449586in}{1.918239in}}%
\pgfpathlineto{\pgfqpoint{4.436156in}{1.919156in}}%
\pgfpathlineto{\pgfqpoint{4.422734in}{1.920192in}}%
\pgfpathlineto{\pgfqpoint{4.409321in}{1.921345in}}%
\pgfpathlineto{\pgfqpoint{4.395916in}{1.922618in}}%
\pgfpathlineto{\pgfqpoint{4.388333in}{1.912405in}}%
\pgfpathlineto{\pgfqpoint{4.380746in}{1.902204in}}%
\pgfpathlineto{\pgfqpoint{4.373153in}{1.892016in}}%
\pgfpathlineto{\pgfqpoint{4.365556in}{1.881844in}}%
\pgfpathclose%
\pgfusepath{fill}%
\end{pgfscope}%
\begin{pgfscope}%
\pgfpathrectangle{\pgfqpoint{1.254980in}{0.150000in}}{\pgfqpoint{5.490039in}{5.490039in}}%
\pgfusepath{clip}%
\pgfsetbuttcap%
\pgfsetroundjoin%
\definecolor{currentfill}{rgb}{0.146180,0.515413,0.556823}%
\pgfsetfillcolor{currentfill}%
\pgfsetfillopacity{0.700000}%
\pgfsetlinewidth{0.000000pt}%
\definecolor{currentstroke}{rgb}{0.000000,0.000000,0.000000}%
\pgfsetstrokecolor{currentstroke}%
\pgfsetdash{}{0pt}%
\pgfpathmoveto{\pgfqpoint{5.856747in}{2.861881in}}%
\pgfpathlineto{\pgfqpoint{5.870860in}{2.868882in}}%
\pgfpathlineto{\pgfqpoint{5.884989in}{2.875994in}}%
\pgfpathlineto{\pgfqpoint{5.899134in}{2.883218in}}%
\pgfpathlineto{\pgfqpoint{5.906169in}{2.891139in}}%
\pgfpathlineto{\pgfqpoint{5.913198in}{2.899003in}}%
\pgfpathlineto{\pgfqpoint{5.920219in}{2.906813in}}%
\pgfpathlineto{\pgfqpoint{5.927234in}{2.914569in}}%
\pgfpathlineto{\pgfqpoint{5.913102in}{2.907476in}}%
\pgfpathlineto{\pgfqpoint{5.898986in}{2.900494in}}%
\pgfpathlineto{\pgfqpoint{5.884886in}{2.893623in}}%
\pgfpathlineto{\pgfqpoint{5.877862in}{2.885765in}}%
\pgfpathlineto{\pgfqpoint{5.870830in}{2.877856in}}%
\pgfpathlineto{\pgfqpoint{5.863792in}{2.869895in}}%
\pgfpathlineto{\pgfqpoint{5.856747in}{2.861881in}}%
\pgfpathclose%
\pgfusepath{fill}%
\end{pgfscope}%
\begin{pgfscope}%
\pgfpathrectangle{\pgfqpoint{1.254980in}{0.150000in}}{\pgfqpoint{5.490039in}{5.490039in}}%
\pgfusepath{clip}%
\pgfsetbuttcap%
\pgfsetroundjoin%
\definecolor{currentfill}{rgb}{0.269308,0.218818,0.509577}%
\pgfsetfillcolor{currentfill}%
\pgfsetfillopacity{0.700000}%
\pgfsetlinewidth{0.000000pt}%
\definecolor{currentstroke}{rgb}{0.000000,0.000000,0.000000}%
\pgfsetstrokecolor{currentstroke}%
\pgfsetdash{}{0pt}%
\pgfpathmoveto{\pgfqpoint{4.870019in}{2.145654in}}%
\pgfpathlineto{\pgfqpoint{4.883628in}{2.147848in}}%
\pgfpathlineto{\pgfqpoint{4.897247in}{2.150158in}}%
\pgfpathlineto{\pgfqpoint{4.910878in}{2.152581in}}%
\pgfpathlineto{\pgfqpoint{4.924520in}{2.155120in}}%
\pgfpathlineto{\pgfqpoint{4.931957in}{2.165999in}}%
\pgfpathlineto{\pgfqpoint{4.939389in}{2.176844in}}%
\pgfpathlineto{\pgfqpoint{4.946816in}{2.187653in}}%
\pgfpathlineto{\pgfqpoint{4.954237in}{2.198426in}}%
\pgfpathlineto{\pgfqpoint{4.940600in}{2.195802in}}%
\pgfpathlineto{\pgfqpoint{4.926975in}{2.193292in}}%
\pgfpathlineto{\pgfqpoint{4.913361in}{2.190896in}}%
\pgfpathlineto{\pgfqpoint{4.899758in}{2.188616in}}%
\pgfpathlineto{\pgfqpoint{4.892331in}{2.177923in}}%
\pgfpathlineto{\pgfqpoint{4.884899in}{2.167198in}}%
\pgfpathlineto{\pgfqpoint{4.877462in}{2.156441in}}%
\pgfpathlineto{\pgfqpoint{4.870019in}{2.145654in}}%
\pgfpathclose%
\pgfusepath{fill}%
\end{pgfscope}%
\begin{pgfscope}%
\pgfpathrectangle{\pgfqpoint{1.254980in}{0.150000in}}{\pgfqpoint{5.490039in}{5.490039in}}%
\pgfusepath{clip}%
\pgfsetbuttcap%
\pgfsetroundjoin%
\definecolor{currentfill}{rgb}{0.162142,0.474838,0.558140}%
\pgfsetfillcolor{currentfill}%
\pgfsetfillopacity{0.700000}%
\pgfsetlinewidth{0.000000pt}%
\definecolor{currentstroke}{rgb}{0.000000,0.000000,0.000000}%
\pgfsetstrokecolor{currentstroke}%
\pgfsetdash{}{0pt}%
\pgfpathmoveto{\pgfqpoint{2.624681in}{2.817682in}}%
\pgfpathlineto{\pgfqpoint{2.638154in}{2.797717in}}%
\pgfpathlineto{\pgfqpoint{2.651620in}{2.777947in}}%
\pgfpathlineto{\pgfqpoint{2.665078in}{2.758369in}}%
\pgfpathlineto{\pgfqpoint{2.678528in}{2.738983in}}%
\pgfpathlineto{\pgfqpoint{2.686953in}{2.739033in}}%
\pgfpathlineto{\pgfqpoint{2.695363in}{2.739273in}}%
\pgfpathlineto{\pgfqpoint{2.703760in}{2.739702in}}%
\pgfpathlineto{\pgfqpoint{2.712143in}{2.740316in}}%
\pgfpathlineto{\pgfqpoint{2.698731in}{2.759364in}}%
\pgfpathlineto{\pgfqpoint{2.685312in}{2.778603in}}%
\pgfpathlineto{\pgfqpoint{2.671885in}{2.798033in}}%
\pgfpathlineto{\pgfqpoint{2.658451in}{2.817657in}}%
\pgfpathlineto{\pgfqpoint{2.650030in}{2.817376in}}%
\pgfpathlineto{\pgfqpoint{2.641595in}{2.817284in}}%
\pgfpathlineto{\pgfqpoint{2.633145in}{2.817385in}}%
\pgfpathlineto{\pgfqpoint{2.624681in}{2.817682in}}%
\pgfpathclose%
\pgfusepath{fill}%
\end{pgfscope}%
\begin{pgfscope}%
\pgfpathrectangle{\pgfqpoint{1.254980in}{0.150000in}}{\pgfqpoint{5.490039in}{5.490039in}}%
\pgfusepath{clip}%
\pgfsetbuttcap%
\pgfsetroundjoin%
\definecolor{currentfill}{rgb}{0.197636,0.391528,0.554969}%
\pgfsetfillcolor{currentfill}%
\pgfsetfillopacity{0.700000}%
\pgfsetlinewidth{0.000000pt}%
\definecolor{currentstroke}{rgb}{0.000000,0.000000,0.000000}%
\pgfsetstrokecolor{currentstroke}%
\pgfsetdash{}{0pt}%
\pgfpathmoveto{\pgfqpoint{5.405375in}{2.525594in}}%
\pgfpathlineto{\pgfqpoint{5.419246in}{2.530777in}}%
\pgfpathlineto{\pgfqpoint{5.433130in}{2.536072in}}%
\pgfpathlineto{\pgfqpoint{5.447029in}{2.541480in}}%
\pgfpathlineto{\pgfqpoint{5.460942in}{2.547001in}}%
\pgfpathlineto{\pgfqpoint{5.468185in}{2.556708in}}%
\pgfpathlineto{\pgfqpoint{5.475421in}{2.566355in}}%
\pgfpathlineto{\pgfqpoint{5.482651in}{2.575943in}}%
\pgfpathlineto{\pgfqpoint{5.489875in}{2.585474in}}%
\pgfpathlineto{\pgfqpoint{5.475969in}{2.579982in}}%
\pgfpathlineto{\pgfqpoint{5.462079in}{2.574602in}}%
\pgfpathlineto{\pgfqpoint{5.448202in}{2.569335in}}%
\pgfpathlineto{\pgfqpoint{5.434339in}{2.564181in}}%
\pgfpathlineto{\pgfqpoint{5.427107in}{2.554616in}}%
\pgfpathlineto{\pgfqpoint{5.419869in}{2.544997in}}%
\pgfpathlineto{\pgfqpoint{5.412625in}{2.535323in}}%
\pgfpathlineto{\pgfqpoint{5.405375in}{2.525594in}}%
\pgfpathclose%
\pgfusepath{fill}%
\end{pgfscope}%
\begin{pgfscope}%
\pgfpathrectangle{\pgfqpoint{1.254980in}{0.150000in}}{\pgfqpoint{5.490039in}{5.490039in}}%
\pgfusepath{clip}%
\pgfsetbuttcap%
\pgfsetroundjoin%
\definecolor{currentfill}{rgb}{0.281446,0.084320,0.407414}%
\pgfsetfillcolor{currentfill}%
\pgfsetfillopacity{0.700000}%
\pgfsetlinewidth{0.000000pt}%
\definecolor{currentstroke}{rgb}{0.000000,0.000000,0.000000}%
\pgfsetstrokecolor{currentstroke}%
\pgfsetdash{}{0pt}%
\pgfpathmoveto{\pgfqpoint{3.509956in}{1.907125in}}%
\pgfpathlineto{\pgfqpoint{3.523226in}{1.897863in}}%
\pgfpathlineto{\pgfqpoint{3.536498in}{1.888739in}}%
\pgfpathlineto{\pgfqpoint{3.549771in}{1.879752in}}%
\pgfpathlineto{\pgfqpoint{3.563047in}{1.870901in}}%
\pgfpathlineto{\pgfqpoint{3.570960in}{1.876690in}}%
\pgfpathlineto{\pgfqpoint{3.578865in}{1.882589in}}%
\pgfpathlineto{\pgfqpoint{3.586763in}{1.888596in}}%
\pgfpathlineto{\pgfqpoint{3.594654in}{1.894709in}}%
\pgfpathlineto{\pgfqpoint{3.581399in}{1.903264in}}%
\pgfpathlineto{\pgfqpoint{3.568146in}{1.911955in}}%
\pgfpathlineto{\pgfqpoint{3.554895in}{1.920783in}}%
\pgfpathlineto{\pgfqpoint{3.541646in}{1.929748in}}%
\pgfpathlineto{\pgfqpoint{3.533735in}{1.923925in}}%
\pgfpathlineto{\pgfqpoint{3.525817in}{1.918212in}}%
\pgfpathlineto{\pgfqpoint{3.517890in}{1.912611in}}%
\pgfpathlineto{\pgfqpoint{3.509956in}{1.907125in}}%
\pgfpathclose%
\pgfusepath{fill}%
\end{pgfscope}%
\begin{pgfscope}%
\pgfpathrectangle{\pgfqpoint{1.254980in}{0.150000in}}{\pgfqpoint{5.490039in}{5.490039in}}%
\pgfusepath{clip}%
\pgfsetbuttcap%
\pgfsetroundjoin%
\definecolor{currentfill}{rgb}{0.260571,0.246922,0.522828}%
\pgfsetfillcolor{currentfill}%
\pgfsetfillopacity{0.700000}%
\pgfsetlinewidth{0.000000pt}%
\definecolor{currentstroke}{rgb}{0.000000,0.000000,0.000000}%
\pgfsetstrokecolor{currentstroke}%
\pgfsetdash{}{0pt}%
\pgfpathmoveto{\pgfqpoint{4.954237in}{2.198426in}}%
\pgfpathlineto{\pgfqpoint{4.967886in}{2.201165in}}%
\pgfpathlineto{\pgfqpoint{4.981546in}{2.204018in}}%
\pgfpathlineto{\pgfqpoint{4.995219in}{2.206985in}}%
\pgfpathlineto{\pgfqpoint{5.008903in}{2.210066in}}%
\pgfpathlineto{\pgfqpoint{5.016314in}{2.220879in}}%
\pgfpathlineto{\pgfqpoint{5.023720in}{2.231651in}}%
\pgfpathlineto{\pgfqpoint{5.031121in}{2.242382in}}%
\pgfpathlineto{\pgfqpoint{5.038516in}{2.253072in}}%
\pgfpathlineto{\pgfqpoint{5.024837in}{2.249921in}}%
\pgfpathlineto{\pgfqpoint{5.011170in}{2.246883in}}%
\pgfpathlineto{\pgfqpoint{4.997514in}{2.243960in}}%
\pgfpathlineto{\pgfqpoint{4.983871in}{2.241152in}}%
\pgfpathlineto{\pgfqpoint{4.976471in}{2.230526in}}%
\pgfpathlineto{\pgfqpoint{4.969065in}{2.219863in}}%
\pgfpathlineto{\pgfqpoint{4.961653in}{2.209163in}}%
\pgfpathlineto{\pgfqpoint{4.954237in}{2.198426in}}%
\pgfpathclose%
\pgfusepath{fill}%
\end{pgfscope}%
\begin{pgfscope}%
\pgfpathrectangle{\pgfqpoint{1.254980in}{0.150000in}}{\pgfqpoint{5.490039in}{5.490039in}}%
\pgfusepath{clip}%
\pgfsetbuttcap%
\pgfsetroundjoin%
\definecolor{currentfill}{rgb}{0.279566,0.067836,0.391917}%
\pgfsetfillcolor{currentfill}%
\pgfsetfillopacity{0.700000}%
\pgfsetlinewidth{0.000000pt}%
\definecolor{currentstroke}{rgb}{0.000000,0.000000,0.000000}%
\pgfsetstrokecolor{currentstroke}%
\pgfsetdash{}{0pt}%
\pgfpathmoveto{\pgfqpoint{4.281512in}{1.849122in}}%
\pgfpathlineto{\pgfqpoint{4.294902in}{1.846994in}}%
\pgfpathlineto{\pgfqpoint{4.308300in}{1.844986in}}%
\pgfpathlineto{\pgfqpoint{4.321705in}{1.843097in}}%
\pgfpathlineto{\pgfqpoint{4.335118in}{1.841328in}}%
\pgfpathlineto{\pgfqpoint{4.342735in}{1.851428in}}%
\pgfpathlineto{\pgfqpoint{4.350347in}{1.861548in}}%
\pgfpathlineto{\pgfqpoint{4.357954in}{1.871687in}}%
\pgfpathlineto{\pgfqpoint{4.365556in}{1.881844in}}%
\pgfpathlineto{\pgfqpoint{4.352151in}{1.883416in}}%
\pgfpathlineto{\pgfqpoint{4.338754in}{1.885108in}}%
\pgfpathlineto{\pgfqpoint{4.325365in}{1.886919in}}%
\pgfpathlineto{\pgfqpoint{4.311984in}{1.888850in}}%
\pgfpathlineto{\pgfqpoint{4.304374in}{1.878885in}}%
\pgfpathlineto{\pgfqpoint{4.296758in}{1.868941in}}%
\pgfpathlineto{\pgfqpoint{4.289138in}{1.859019in}}%
\pgfpathlineto{\pgfqpoint{4.281512in}{1.849122in}}%
\pgfpathclose%
\pgfusepath{fill}%
\end{pgfscope}%
\begin{pgfscope}%
\pgfpathrectangle{\pgfqpoint{1.254980in}{0.150000in}}{\pgfqpoint{5.490039in}{5.490039in}}%
\pgfusepath{clip}%
\pgfsetbuttcap%
\pgfsetroundjoin%
\definecolor{currentfill}{rgb}{0.280868,0.160771,0.472899}%
\pgfsetfillcolor{currentfill}%
\pgfsetfillopacity{0.700000}%
\pgfsetlinewidth{0.000000pt}%
\definecolor{currentstroke}{rgb}{0.000000,0.000000,0.000000}%
\pgfsetstrokecolor{currentstroke}%
\pgfsetdash{}{0pt}%
\pgfpathmoveto{\pgfqpoint{3.265569in}{2.058770in}}%
\pgfpathlineto{\pgfqpoint{3.278854in}{2.046912in}}%
\pgfpathlineto{\pgfqpoint{3.292138in}{2.035201in}}%
\pgfpathlineto{\pgfqpoint{3.305421in}{2.023637in}}%
\pgfpathlineto{\pgfqpoint{3.318705in}{2.012219in}}%
\pgfpathlineto{\pgfqpoint{3.326745in}{2.016259in}}%
\pgfpathlineto{\pgfqpoint{3.334775in}{2.020436in}}%
\pgfpathlineto{\pgfqpoint{3.342797in}{2.024748in}}%
\pgfpathlineto{\pgfqpoint{3.350809in}{2.029192in}}%
\pgfpathlineto{\pgfqpoint{3.337551in}{2.040293in}}%
\pgfpathlineto{\pgfqpoint{3.324293in}{2.051540in}}%
\pgfpathlineto{\pgfqpoint{3.311035in}{2.062934in}}%
\pgfpathlineto{\pgfqpoint{3.297777in}{2.074476in}}%
\pgfpathlineto{\pgfqpoint{3.289739in}{2.070342in}}%
\pgfpathlineto{\pgfqpoint{3.281692in}{2.066345in}}%
\pgfpathlineto{\pgfqpoint{3.273635in}{2.062487in}}%
\pgfpathlineto{\pgfqpoint{3.265569in}{2.058770in}}%
\pgfpathclose%
\pgfusepath{fill}%
\end{pgfscope}%
\begin{pgfscope}%
\pgfpathrectangle{\pgfqpoint{1.254980in}{0.150000in}}{\pgfqpoint{5.490039in}{5.490039in}}%
\pgfusepath{clip}%
\pgfsetbuttcap%
\pgfsetroundjoin%
\definecolor{currentfill}{rgb}{0.273809,0.031497,0.358853}%
\pgfsetfillcolor{currentfill}%
\pgfsetfillopacity{0.700000}%
\pgfsetlinewidth{0.000000pt}%
\definecolor{currentstroke}{rgb}{0.000000,0.000000,0.000000}%
\pgfsetstrokecolor{currentstroke}%
\pgfsetdash{}{0pt}%
\pgfpathmoveto{\pgfqpoint{3.838333in}{1.805851in}}%
\pgfpathlineto{\pgfqpoint{3.851627in}{1.799812in}}%
\pgfpathlineto{\pgfqpoint{3.864925in}{1.793902in}}%
\pgfpathlineto{\pgfqpoint{3.878228in}{1.788118in}}%
\pgfpathlineto{\pgfqpoint{3.891535in}{1.782461in}}%
\pgfpathlineto{\pgfqpoint{3.899306in}{1.790409in}}%
\pgfpathlineto{\pgfqpoint{3.907070in}{1.798430in}}%
\pgfpathlineto{\pgfqpoint{3.914828in}{1.806521in}}%
\pgfpathlineto{\pgfqpoint{3.922580in}{1.814681in}}%
\pgfpathlineto{\pgfqpoint{3.909287in}{1.820076in}}%
\pgfpathlineto{\pgfqpoint{3.895999in}{1.825599in}}%
\pgfpathlineto{\pgfqpoint{3.882716in}{1.831249in}}%
\pgfpathlineto{\pgfqpoint{3.869437in}{1.837026in}}%
\pgfpathlineto{\pgfqpoint{3.861670in}{1.829122in}}%
\pgfpathlineto{\pgfqpoint{3.853897in}{1.821290in}}%
\pgfpathlineto{\pgfqpoint{3.846118in}{1.813532in}}%
\pgfpathlineto{\pgfqpoint{3.838333in}{1.805851in}}%
\pgfpathclose%
\pgfusepath{fill}%
\end{pgfscope}%
\begin{pgfscope}%
\pgfpathrectangle{\pgfqpoint{1.254980in}{0.150000in}}{\pgfqpoint{5.490039in}{5.490039in}}%
\pgfusepath{clip}%
\pgfsetbuttcap%
\pgfsetroundjoin%
\definecolor{currentfill}{rgb}{0.276022,0.044167,0.370164}%
\pgfsetfillcolor{currentfill}%
\pgfsetfillopacity{0.700000}%
\pgfsetlinewidth{0.000000pt}%
\definecolor{currentstroke}{rgb}{0.000000,0.000000,0.000000}%
\pgfsetstrokecolor{currentstroke}%
\pgfsetdash{}{0pt}%
\pgfpathmoveto{\pgfqpoint{3.700787in}{1.831107in}}%
\pgfpathlineto{\pgfqpoint{3.714068in}{1.823755in}}%
\pgfpathlineto{\pgfqpoint{3.727351in}{1.816533in}}%
\pgfpathlineto{\pgfqpoint{3.740638in}{1.809442in}}%
\pgfpathlineto{\pgfqpoint{3.753928in}{1.802482in}}%
\pgfpathlineto{\pgfqpoint{3.761756in}{1.809552in}}%
\pgfpathlineto{\pgfqpoint{3.769577in}{1.816712in}}%
\pgfpathlineto{\pgfqpoint{3.777391in}{1.823958in}}%
\pgfpathlineto{\pgfqpoint{3.785199in}{1.831289in}}%
\pgfpathlineto{\pgfqpoint{3.771925in}{1.837972in}}%
\pgfpathlineto{\pgfqpoint{3.758656in}{1.844785in}}%
\pgfpathlineto{\pgfqpoint{3.745390in}{1.851728in}}%
\pgfpathlineto{\pgfqpoint{3.732127in}{1.858803in}}%
\pgfpathlineto{\pgfqpoint{3.724302in}{1.851744in}}%
\pgfpathlineto{\pgfqpoint{3.716471in}{1.844773in}}%
\pgfpathlineto{\pgfqpoint{3.708633in}{1.837894in}}%
\pgfpathlineto{\pgfqpoint{3.700787in}{1.831107in}}%
\pgfpathclose%
\pgfusepath{fill}%
\end{pgfscope}%
\begin{pgfscope}%
\pgfpathrectangle{\pgfqpoint{1.254980in}{0.150000in}}{\pgfqpoint{5.490039in}{5.490039in}}%
\pgfusepath{clip}%
\pgfsetbuttcap%
\pgfsetroundjoin%
\definecolor{currentfill}{rgb}{0.250425,0.274290,0.533103}%
\pgfsetfillcolor{currentfill}%
\pgfsetfillopacity{0.700000}%
\pgfsetlinewidth{0.000000pt}%
\definecolor{currentstroke}{rgb}{0.000000,0.000000,0.000000}%
\pgfsetstrokecolor{currentstroke}%
\pgfsetdash{}{0pt}%
\pgfpathmoveto{\pgfqpoint{5.038516in}{2.253072in}}%
\pgfpathlineto{\pgfqpoint{5.052207in}{2.256337in}}%
\pgfpathlineto{\pgfqpoint{5.065911in}{2.259716in}}%
\pgfpathlineto{\pgfqpoint{5.079627in}{2.263209in}}%
\pgfpathlineto{\pgfqpoint{5.093355in}{2.266816in}}%
\pgfpathlineto{\pgfqpoint{5.100739in}{2.277524in}}%
\pgfpathlineto{\pgfqpoint{5.108118in}{2.288186in}}%
\pgfpathlineto{\pgfqpoint{5.115492in}{2.298802in}}%
\pgfpathlineto{\pgfqpoint{5.122860in}{2.309371in}}%
\pgfpathlineto{\pgfqpoint{5.109137in}{2.305711in}}%
\pgfpathlineto{\pgfqpoint{5.095427in}{2.302164in}}%
\pgfpathlineto{\pgfqpoint{5.081729in}{2.298731in}}%
\pgfpathlineto{\pgfqpoint{5.068043in}{2.295412in}}%
\pgfpathlineto{\pgfqpoint{5.060669in}{2.284890in}}%
\pgfpathlineto{\pgfqpoint{5.053290in}{2.274326in}}%
\pgfpathlineto{\pgfqpoint{5.045906in}{2.263720in}}%
\pgfpathlineto{\pgfqpoint{5.038516in}{2.253072in}}%
\pgfpathclose%
\pgfusepath{fill}%
\end{pgfscope}%
\begin{pgfscope}%
\pgfpathrectangle{\pgfqpoint{1.254980in}{0.150000in}}{\pgfqpoint{5.490039in}{5.490039in}}%
\pgfusepath{clip}%
\pgfsetbuttcap%
\pgfsetroundjoin%
\definecolor{currentfill}{rgb}{0.273809,0.031497,0.358853}%
\pgfsetfillcolor{currentfill}%
\pgfsetfillopacity{0.700000}%
\pgfsetlinewidth{0.000000pt}%
\definecolor{currentstroke}{rgb}{0.000000,0.000000,0.000000}%
\pgfsetstrokecolor{currentstroke}%
\pgfsetdash{}{0pt}%
\pgfpathmoveto{\pgfqpoint{3.975803in}{1.794357in}}%
\pgfpathlineto{\pgfqpoint{3.989122in}{1.789589in}}%
\pgfpathlineto{\pgfqpoint{4.002446in}{1.784946in}}%
\pgfpathlineto{\pgfqpoint{4.015776in}{1.780427in}}%
\pgfpathlineto{\pgfqpoint{4.029111in}{1.776032in}}%
\pgfpathlineto{\pgfqpoint{4.036831in}{1.784759in}}%
\pgfpathlineto{\pgfqpoint{4.044545in}{1.793543in}}%
\pgfpathlineto{\pgfqpoint{4.052254in}{1.802382in}}%
\pgfpathlineto{\pgfqpoint{4.059958in}{1.811273in}}%
\pgfpathlineto{\pgfqpoint{4.046635in}{1.815423in}}%
\pgfpathlineto{\pgfqpoint{4.033317in}{1.819698in}}%
\pgfpathlineto{\pgfqpoint{4.020006in}{1.824096in}}%
\pgfpathlineto{\pgfqpoint{4.006700in}{1.828619in}}%
\pgfpathlineto{\pgfqpoint{3.998984in}{1.819967in}}%
\pgfpathlineto{\pgfqpoint{3.991263in}{1.811371in}}%
\pgfpathlineto{\pgfqpoint{3.983536in}{1.802834in}}%
\pgfpathlineto{\pgfqpoint{3.975803in}{1.794357in}}%
\pgfpathclose%
\pgfusepath{fill}%
\end{pgfscope}%
\begin{pgfscope}%
\pgfpathrectangle{\pgfqpoint{1.254980in}{0.150000in}}{\pgfqpoint{5.490039in}{5.490039in}}%
\pgfusepath{clip}%
\pgfsetbuttcap%
\pgfsetroundjoin%
\definecolor{currentfill}{rgb}{0.187231,0.414746,0.556547}%
\pgfsetfillcolor{currentfill}%
\pgfsetfillopacity{0.700000}%
\pgfsetlinewidth{0.000000pt}%
\definecolor{currentstroke}{rgb}{0.000000,0.000000,0.000000}%
\pgfsetstrokecolor{currentstroke}%
\pgfsetdash{}{0pt}%
\pgfpathmoveto{\pgfqpoint{5.489875in}{2.585474in}}%
\pgfpathlineto{\pgfqpoint{5.503795in}{2.591078in}}%
\pgfpathlineto{\pgfqpoint{5.517729in}{2.596795in}}%
\pgfpathlineto{\pgfqpoint{5.531678in}{2.602625in}}%
\pgfpathlineto{\pgfqpoint{5.545641in}{2.608567in}}%
\pgfpathlineto{\pgfqpoint{5.552850in}{2.618000in}}%
\pgfpathlineto{\pgfqpoint{5.560053in}{2.627372in}}%
\pgfpathlineto{\pgfqpoint{5.567250in}{2.636684in}}%
\pgfpathlineto{\pgfqpoint{5.574440in}{2.645937in}}%
\pgfpathlineto{\pgfqpoint{5.560485in}{2.640040in}}%
\pgfpathlineto{\pgfqpoint{5.546544in}{2.634257in}}%
\pgfpathlineto{\pgfqpoint{5.532618in}{2.628585in}}%
\pgfpathlineto{\pgfqpoint{5.518707in}{2.623026in}}%
\pgfpathlineto{\pgfqpoint{5.511508in}{2.613722in}}%
\pgfpathlineto{\pgfqpoint{5.504304in}{2.604362in}}%
\pgfpathlineto{\pgfqpoint{5.497092in}{2.594946in}}%
\pgfpathlineto{\pgfqpoint{5.489875in}{2.585474in}}%
\pgfpathclose%
\pgfusepath{fill}%
\end{pgfscope}%
\begin{pgfscope}%
\pgfpathrectangle{\pgfqpoint{1.254980in}{0.150000in}}{\pgfqpoint{5.490039in}{5.490039in}}%
\pgfusepath{clip}%
\pgfsetbuttcap%
\pgfsetroundjoin%
\definecolor{currentfill}{rgb}{0.277018,0.050344,0.375715}%
\pgfsetfillcolor{currentfill}%
\pgfsetfillopacity{0.700000}%
\pgfsetlinewidth{0.000000pt}%
\definecolor{currentstroke}{rgb}{0.000000,0.000000,0.000000}%
\pgfsetstrokecolor{currentstroke}%
\pgfsetdash{}{0pt}%
\pgfpathmoveto{\pgfqpoint{4.197437in}{1.820372in}}%
\pgfpathlineto{\pgfqpoint{4.210807in}{1.817550in}}%
\pgfpathlineto{\pgfqpoint{4.224185in}{1.814849in}}%
\pgfpathlineto{\pgfqpoint{4.237569in}{1.812268in}}%
\pgfpathlineto{\pgfqpoint{4.250961in}{1.809808in}}%
\pgfpathlineto{\pgfqpoint{4.258606in}{1.819592in}}%
\pgfpathlineto{\pgfqpoint{4.266247in}{1.829407in}}%
\pgfpathlineto{\pgfqpoint{4.273882in}{1.839251in}}%
\pgfpathlineto{\pgfqpoint{4.281512in}{1.849122in}}%
\pgfpathlineto{\pgfqpoint{4.268130in}{1.851370in}}%
\pgfpathlineto{\pgfqpoint{4.254755in}{1.853738in}}%
\pgfpathlineto{\pgfqpoint{4.241387in}{1.856227in}}%
\pgfpathlineto{\pgfqpoint{4.228027in}{1.858836in}}%
\pgfpathlineto{\pgfqpoint{4.220387in}{1.849171in}}%
\pgfpathlineto{\pgfqpoint{4.212742in}{1.839538in}}%
\pgfpathlineto{\pgfqpoint{4.205092in}{1.829938in}}%
\pgfpathlineto{\pgfqpoint{4.197437in}{1.820372in}}%
\pgfpathclose%
\pgfusepath{fill}%
\end{pgfscope}%
\begin{pgfscope}%
\pgfpathrectangle{\pgfqpoint{1.254980in}{0.150000in}}{\pgfqpoint{5.490039in}{5.490039in}}%
\pgfusepath{clip}%
\pgfsetbuttcap%
\pgfsetroundjoin%
\definecolor{currentfill}{rgb}{0.149039,0.508051,0.557250}%
\pgfsetfillcolor{currentfill}%
\pgfsetfillopacity{0.700000}%
\pgfsetlinewidth{0.000000pt}%
\definecolor{currentstroke}{rgb}{0.000000,0.000000,0.000000}%
\pgfsetstrokecolor{currentstroke}%
\pgfsetdash{}{0pt}%
\pgfpathmoveto{\pgfqpoint{2.570709in}{2.899505in}}%
\pgfpathlineto{\pgfqpoint{2.584214in}{2.878752in}}%
\pgfpathlineto{\pgfqpoint{2.597711in}{2.858198in}}%
\pgfpathlineto{\pgfqpoint{2.611200in}{2.837842in}}%
\pgfpathlineto{\pgfqpoint{2.624681in}{2.817682in}}%
\pgfpathlineto{\pgfqpoint{2.633145in}{2.817385in}}%
\pgfpathlineto{\pgfqpoint{2.641595in}{2.817284in}}%
\pgfpathlineto{\pgfqpoint{2.650030in}{2.817376in}}%
\pgfpathlineto{\pgfqpoint{2.658451in}{2.817657in}}%
\pgfpathlineto{\pgfqpoint{2.645010in}{2.837476in}}%
\pgfpathlineto{\pgfqpoint{2.631561in}{2.857490in}}%
\pgfpathlineto{\pgfqpoint{2.618104in}{2.877701in}}%
\pgfpathlineto{\pgfqpoint{2.604638in}{2.898111in}}%
\pgfpathlineto{\pgfqpoint{2.596178in}{2.898165in}}%
\pgfpathlineto{\pgfqpoint{2.587703in}{2.898413in}}%
\pgfpathlineto{\pgfqpoint{2.579213in}{2.898859in}}%
\pgfpathlineto{\pgfqpoint{2.570709in}{2.899505in}}%
\pgfpathclose%
\pgfusepath{fill}%
\end{pgfscope}%
\begin{pgfscope}%
\pgfpathrectangle{\pgfqpoint{1.254980in}{0.150000in}}{\pgfqpoint{5.490039in}{5.490039in}}%
\pgfusepath{clip}%
\pgfsetbuttcap%
\pgfsetroundjoin%
\definecolor{currentfill}{rgb}{0.282623,0.140926,0.457517}%
\pgfsetfillcolor{currentfill}%
\pgfsetfillopacity{0.700000}%
\pgfsetlinewidth{0.000000pt}%
\definecolor{currentstroke}{rgb}{0.000000,0.000000,0.000000}%
\pgfsetstrokecolor{currentstroke}%
\pgfsetdash{}{0pt}%
\pgfpathmoveto{\pgfqpoint{3.318705in}{2.012219in}}%
\pgfpathlineto{\pgfqpoint{3.331988in}{2.000947in}}%
\pgfpathlineto{\pgfqpoint{3.345272in}{1.989819in}}%
\pgfpathlineto{\pgfqpoint{3.358555in}{1.978836in}}%
\pgfpathlineto{\pgfqpoint{3.371839in}{1.967997in}}%
\pgfpathlineto{\pgfqpoint{3.379854in}{1.972359in}}%
\pgfpathlineto{\pgfqpoint{3.387860in}{1.976854in}}%
\pgfpathlineto{\pgfqpoint{3.395857in}{1.981479in}}%
\pgfpathlineto{\pgfqpoint{3.403845in}{1.986232in}}%
\pgfpathlineto{\pgfqpoint{3.390585in}{1.996756in}}%
\pgfpathlineto{\pgfqpoint{3.377326in}{2.007424in}}%
\pgfpathlineto{\pgfqpoint{3.364068in}{2.018235in}}%
\pgfpathlineto{\pgfqpoint{3.350809in}{2.029192in}}%
\pgfpathlineto{\pgfqpoint{3.342797in}{2.024748in}}%
\pgfpathlineto{\pgfqpoint{3.334775in}{2.020436in}}%
\pgfpathlineto{\pgfqpoint{3.326745in}{2.016259in}}%
\pgfpathlineto{\pgfqpoint{3.318705in}{2.012219in}}%
\pgfpathclose%
\pgfusepath{fill}%
\end{pgfscope}%
\begin{pgfscope}%
\pgfpathrectangle{\pgfqpoint{1.254980in}{0.150000in}}{\pgfqpoint{5.490039in}{5.490039in}}%
\pgfusepath{clip}%
\pgfsetbuttcap%
\pgfsetroundjoin%
\definecolor{currentfill}{rgb}{0.239346,0.300855,0.540844}%
\pgfsetfillcolor{currentfill}%
\pgfsetfillopacity{0.700000}%
\pgfsetlinewidth{0.000000pt}%
\definecolor{currentstroke}{rgb}{0.000000,0.000000,0.000000}%
\pgfsetstrokecolor{currentstroke}%
\pgfsetdash{}{0pt}%
\pgfpathmoveto{\pgfqpoint{5.122860in}{2.309371in}}%
\pgfpathlineto{\pgfqpoint{5.136596in}{2.313145in}}%
\pgfpathlineto{\pgfqpoint{5.150344in}{2.317033in}}%
\pgfpathlineto{\pgfqpoint{5.164105in}{2.321034in}}%
\pgfpathlineto{\pgfqpoint{5.177879in}{2.325148in}}%
\pgfpathlineto{\pgfqpoint{5.185236in}{2.335715in}}%
\pgfpathlineto{\pgfqpoint{5.192587in}{2.346232in}}%
\pgfpathlineto{\pgfqpoint{5.199933in}{2.356697in}}%
\pgfpathlineto{\pgfqpoint{5.207273in}{2.367112in}}%
\pgfpathlineto{\pgfqpoint{5.193505in}{2.362961in}}%
\pgfpathlineto{\pgfqpoint{5.179749in}{2.358922in}}%
\pgfpathlineto{\pgfqpoint{5.166007in}{2.354997in}}%
\pgfpathlineto{\pgfqpoint{5.152277in}{2.351185in}}%
\pgfpathlineto{\pgfqpoint{5.144931in}{2.340802in}}%
\pgfpathlineto{\pgfqpoint{5.137580in}{2.330371in}}%
\pgfpathlineto{\pgfqpoint{5.130223in}{2.319894in}}%
\pgfpathlineto{\pgfqpoint{5.122860in}{2.309371in}}%
\pgfpathclose%
\pgfusepath{fill}%
\end{pgfscope}%
\begin{pgfscope}%
\pgfpathrectangle{\pgfqpoint{1.254980in}{0.150000in}}{\pgfqpoint{5.490039in}{5.490039in}}%
\pgfusepath{clip}%
\pgfsetbuttcap%
\pgfsetroundjoin%
\definecolor{currentfill}{rgb}{0.280267,0.073417,0.397163}%
\pgfsetfillcolor{currentfill}%
\pgfsetfillopacity{0.700000}%
\pgfsetlinewidth{0.000000pt}%
\definecolor{currentstroke}{rgb}{0.000000,0.000000,0.000000}%
\pgfsetstrokecolor{currentstroke}%
\pgfsetdash{}{0pt}%
\pgfpathmoveto{\pgfqpoint{3.563047in}{1.870901in}}%
\pgfpathlineto{\pgfqpoint{3.576324in}{1.862186in}}%
\pgfpathlineto{\pgfqpoint{3.589604in}{1.853606in}}%
\pgfpathlineto{\pgfqpoint{3.602887in}{1.845161in}}%
\pgfpathlineto{\pgfqpoint{3.616171in}{1.836850in}}%
\pgfpathlineto{\pgfqpoint{3.624064in}{1.842940in}}%
\pgfpathlineto{\pgfqpoint{3.631949in}{1.849137in}}%
\pgfpathlineto{\pgfqpoint{3.639828in}{1.855437in}}%
\pgfpathlineto{\pgfqpoint{3.647698in}{1.861840in}}%
\pgfpathlineto{\pgfqpoint{3.634433in}{1.869855in}}%
\pgfpathlineto{\pgfqpoint{3.621171in}{1.878005in}}%
\pgfpathlineto{\pgfqpoint{3.607911in}{1.886289in}}%
\pgfpathlineto{\pgfqpoint{3.594654in}{1.894709in}}%
\pgfpathlineto{\pgfqpoint{3.586763in}{1.888596in}}%
\pgfpathlineto{\pgfqpoint{3.578865in}{1.882589in}}%
\pgfpathlineto{\pgfqpoint{3.570960in}{1.876690in}}%
\pgfpathlineto{\pgfqpoint{3.563047in}{1.870901in}}%
\pgfpathclose%
\pgfusepath{fill}%
\end{pgfscope}%
\begin{pgfscope}%
\pgfpathrectangle{\pgfqpoint{1.254980in}{0.150000in}}{\pgfqpoint{5.490039in}{5.490039in}}%
\pgfusepath{clip}%
\pgfsetbuttcap%
\pgfsetroundjoin%
\definecolor{currentfill}{rgb}{0.175841,0.441290,0.557685}%
\pgfsetfillcolor{currentfill}%
\pgfsetfillopacity{0.700000}%
\pgfsetlinewidth{0.000000pt}%
\definecolor{currentstroke}{rgb}{0.000000,0.000000,0.000000}%
\pgfsetstrokecolor{currentstroke}%
\pgfsetdash{}{0pt}%
\pgfpathmoveto{\pgfqpoint{5.574440in}{2.645937in}}%
\pgfpathlineto{\pgfqpoint{5.588410in}{2.651945in}}%
\pgfpathlineto{\pgfqpoint{5.602394in}{2.658066in}}%
\pgfpathlineto{\pgfqpoint{5.616394in}{2.664299in}}%
\pgfpathlineto{\pgfqpoint{5.630409in}{2.670645in}}%
\pgfpathlineto{\pgfqpoint{5.637584in}{2.679782in}}%
\pgfpathlineto{\pgfqpoint{5.644752in}{2.688857in}}%
\pgfpathlineto{\pgfqpoint{5.651913in}{2.697870in}}%
\pgfpathlineto{\pgfqpoint{5.659067in}{2.706823in}}%
\pgfpathlineto{\pgfqpoint{5.645062in}{2.700540in}}%
\pgfpathlineto{\pgfqpoint{5.631071in}{2.694370in}}%
\pgfpathlineto{\pgfqpoint{5.617096in}{2.688311in}}%
\pgfpathlineto{\pgfqpoint{5.603135in}{2.682365in}}%
\pgfpathlineto{\pgfqpoint{5.595971in}{2.673343in}}%
\pgfpathlineto{\pgfqpoint{5.588800in}{2.664265in}}%
\pgfpathlineto{\pgfqpoint{5.581623in}{2.655130in}}%
\pgfpathlineto{\pgfqpoint{5.574440in}{2.645937in}}%
\pgfpathclose%
\pgfusepath{fill}%
\end{pgfscope}%
\begin{pgfscope}%
\pgfpathrectangle{\pgfqpoint{1.254980in}{0.150000in}}{\pgfqpoint{5.490039in}{5.490039in}}%
\pgfusepath{clip}%
\pgfsetbuttcap%
\pgfsetroundjoin%
\definecolor{currentfill}{rgb}{0.274952,0.037752,0.364543}%
\pgfsetfillcolor{currentfill}%
\pgfsetfillopacity{0.700000}%
\pgfsetlinewidth{0.000000pt}%
\definecolor{currentstroke}{rgb}{0.000000,0.000000,0.000000}%
\pgfsetstrokecolor{currentstroke}%
\pgfsetdash{}{0pt}%
\pgfpathmoveto{\pgfqpoint{4.113311in}{1.795902in}}%
\pgfpathlineto{\pgfqpoint{4.126665in}{1.792366in}}%
\pgfpathlineto{\pgfqpoint{4.140025in}{1.788951in}}%
\pgfpathlineto{\pgfqpoint{4.153392in}{1.785658in}}%
\pgfpathlineto{\pgfqpoint{4.166766in}{1.782487in}}%
\pgfpathlineto{\pgfqpoint{4.174441in}{1.791898in}}%
\pgfpathlineto{\pgfqpoint{4.182112in}{1.801351in}}%
\pgfpathlineto{\pgfqpoint{4.189777in}{1.810843in}}%
\pgfpathlineto{\pgfqpoint{4.197437in}{1.820372in}}%
\pgfpathlineto{\pgfqpoint{4.184074in}{1.823315in}}%
\pgfpathlineto{\pgfqpoint{4.170717in}{1.826380in}}%
\pgfpathlineto{\pgfqpoint{4.157368in}{1.829565in}}%
\pgfpathlineto{\pgfqpoint{4.144025in}{1.832873in}}%
\pgfpathlineto{\pgfqpoint{4.136354in}{1.823566in}}%
\pgfpathlineto{\pgfqpoint{4.128678in}{1.814301in}}%
\pgfpathlineto{\pgfqpoint{4.120997in}{1.805079in}}%
\pgfpathlineto{\pgfqpoint{4.113311in}{1.795902in}}%
\pgfpathclose%
\pgfusepath{fill}%
\end{pgfscope}%
\begin{pgfscope}%
\pgfpathrectangle{\pgfqpoint{1.254980in}{0.150000in}}{\pgfqpoint{5.490039in}{5.490039in}}%
\pgfusepath{clip}%
\pgfsetbuttcap%
\pgfsetroundjoin%
\definecolor{currentfill}{rgb}{0.225863,0.330805,0.547314}%
\pgfsetfillcolor{currentfill}%
\pgfsetfillopacity{0.700000}%
\pgfsetlinewidth{0.000000pt}%
\definecolor{currentstroke}{rgb}{0.000000,0.000000,0.000000}%
\pgfsetstrokecolor{currentstroke}%
\pgfsetdash{}{0pt}%
\pgfpathmoveto{\pgfqpoint{5.207273in}{2.367112in}}%
\pgfpathlineto{\pgfqpoint{5.221055in}{2.371378in}}%
\pgfpathlineto{\pgfqpoint{5.234849in}{2.375756in}}%
\pgfpathlineto{\pgfqpoint{5.248657in}{2.380248in}}%
\pgfpathlineto{\pgfqpoint{5.262479in}{2.384853in}}%
\pgfpathlineto{\pgfqpoint{5.269807in}{2.395245in}}%
\pgfpathlineto{\pgfqpoint{5.277130in}{2.405582in}}%
\pgfpathlineto{\pgfqpoint{5.284447in}{2.415864in}}%
\pgfpathlineto{\pgfqpoint{5.291757in}{2.426093in}}%
\pgfpathlineto{\pgfqpoint{5.277942in}{2.421467in}}%
\pgfpathlineto{\pgfqpoint{5.264140in}{2.416954in}}%
\pgfpathlineto{\pgfqpoint{5.250352in}{2.412555in}}%
\pgfpathlineto{\pgfqpoint{5.236576in}{2.408268in}}%
\pgfpathlineto{\pgfqpoint{5.229259in}{2.398055in}}%
\pgfpathlineto{\pgfqpoint{5.221936in}{2.387791in}}%
\pgfpathlineto{\pgfqpoint{5.214608in}{2.377477in}}%
\pgfpathlineto{\pgfqpoint{5.207273in}{2.367112in}}%
\pgfpathclose%
\pgfusepath{fill}%
\end{pgfscope}%
\begin{pgfscope}%
\pgfpathrectangle{\pgfqpoint{1.254980in}{0.150000in}}{\pgfqpoint{5.490039in}{5.490039in}}%
\pgfusepath{clip}%
\pgfsetbuttcap%
\pgfsetroundjoin%
\definecolor{currentfill}{rgb}{0.136408,0.541173,0.554483}%
\pgfsetfillcolor{currentfill}%
\pgfsetfillopacity{0.700000}%
\pgfsetlinewidth{0.000000pt}%
\definecolor{currentstroke}{rgb}{0.000000,0.000000,0.000000}%
\pgfsetstrokecolor{currentstroke}%
\pgfsetdash{}{0pt}%
\pgfpathmoveto{\pgfqpoint{2.516598in}{2.984539in}}%
\pgfpathlineto{\pgfqpoint{2.530139in}{2.962975in}}%
\pgfpathlineto{\pgfqpoint{2.543671in}{2.941615in}}%
\pgfpathlineto{\pgfqpoint{2.557194in}{2.920459in}}%
\pgfpathlineto{\pgfqpoint{2.570709in}{2.899505in}}%
\pgfpathlineto{\pgfqpoint{2.579213in}{2.898859in}}%
\pgfpathlineto{\pgfqpoint{2.587703in}{2.898413in}}%
\pgfpathlineto{\pgfqpoint{2.596178in}{2.898165in}}%
\pgfpathlineto{\pgfqpoint{2.604638in}{2.898111in}}%
\pgfpathlineto{\pgfqpoint{2.591165in}{2.918721in}}%
\pgfpathlineto{\pgfqpoint{2.577683in}{2.939532in}}%
\pgfpathlineto{\pgfqpoint{2.564192in}{2.960545in}}%
\pgfpathlineto{\pgfqpoint{2.550693in}{2.981763in}}%
\pgfpathlineto{\pgfqpoint{2.542191in}{2.982155in}}%
\pgfpathlineto{\pgfqpoint{2.533675in}{2.982746in}}%
\pgfpathlineto{\pgfqpoint{2.525144in}{2.983540in}}%
\pgfpathlineto{\pgfqpoint{2.516598in}{2.984539in}}%
\pgfpathclose%
\pgfusepath{fill}%
\end{pgfscope}%
\begin{pgfscope}%
\pgfpathrectangle{\pgfqpoint{1.254980in}{0.150000in}}{\pgfqpoint{5.490039in}{5.490039in}}%
\pgfusepath{clip}%
\pgfsetbuttcap%
\pgfsetroundjoin%
\definecolor{currentfill}{rgb}{0.283229,0.120777,0.440584}%
\pgfsetfillcolor{currentfill}%
\pgfsetfillopacity{0.700000}%
\pgfsetlinewidth{0.000000pt}%
\definecolor{currentstroke}{rgb}{0.000000,0.000000,0.000000}%
\pgfsetstrokecolor{currentstroke}%
\pgfsetdash{}{0pt}%
\pgfpathmoveto{\pgfqpoint{3.371839in}{1.967997in}}%
\pgfpathlineto{\pgfqpoint{3.385124in}{1.957300in}}%
\pgfpathlineto{\pgfqpoint{3.398409in}{1.946746in}}%
\pgfpathlineto{\pgfqpoint{3.411694in}{1.936334in}}%
\pgfpathlineto{\pgfqpoint{3.424981in}{1.926062in}}%
\pgfpathlineto{\pgfqpoint{3.432971in}{1.930745in}}%
\pgfpathlineto{\pgfqpoint{3.440953in}{1.935557in}}%
\pgfpathlineto{\pgfqpoint{3.448926in}{1.940495in}}%
\pgfpathlineto{\pgfqpoint{3.456891in}{1.945556in}}%
\pgfpathlineto{\pgfqpoint{3.443628in}{1.955513in}}%
\pgfpathlineto{\pgfqpoint{3.430366in}{1.965611in}}%
\pgfpathlineto{\pgfqpoint{3.417105in}{1.975851in}}%
\pgfpathlineto{\pgfqpoint{3.403845in}{1.986232in}}%
\pgfpathlineto{\pgfqpoint{3.395857in}{1.981479in}}%
\pgfpathlineto{\pgfqpoint{3.387860in}{1.976854in}}%
\pgfpathlineto{\pgfqpoint{3.379854in}{1.972359in}}%
\pgfpathlineto{\pgfqpoint{3.371839in}{1.967997in}}%
\pgfpathclose%
\pgfusepath{fill}%
\end{pgfscope}%
\begin{pgfscope}%
\pgfpathrectangle{\pgfqpoint{1.254980in}{0.150000in}}{\pgfqpoint{5.490039in}{5.490039in}}%
\pgfusepath{clip}%
\pgfsetbuttcap%
\pgfsetroundjoin%
\definecolor{currentfill}{rgb}{0.274952,0.037752,0.364543}%
\pgfsetfillcolor{currentfill}%
\pgfsetfillopacity{0.700000}%
\pgfsetlinewidth{0.000000pt}%
\definecolor{currentstroke}{rgb}{0.000000,0.000000,0.000000}%
\pgfsetstrokecolor{currentstroke}%
\pgfsetdash{}{0pt}%
\pgfpathmoveto{\pgfqpoint{3.753928in}{1.802482in}}%
\pgfpathlineto{\pgfqpoint{3.767222in}{1.795651in}}%
\pgfpathlineto{\pgfqpoint{3.780520in}{1.788949in}}%
\pgfpathlineto{\pgfqpoint{3.793822in}{1.782377in}}%
\pgfpathlineto{\pgfqpoint{3.807127in}{1.775933in}}%
\pgfpathlineto{\pgfqpoint{3.814938in}{1.783287in}}%
\pgfpathlineto{\pgfqpoint{3.822743in}{1.790726in}}%
\pgfpathlineto{\pgfqpoint{3.830541in}{1.798248in}}%
\pgfpathlineto{\pgfqpoint{3.838333in}{1.805851in}}%
\pgfpathlineto{\pgfqpoint{3.825043in}{1.812017in}}%
\pgfpathlineto{\pgfqpoint{3.811758in}{1.818312in}}%
\pgfpathlineto{\pgfqpoint{3.798476in}{1.824736in}}%
\pgfpathlineto{\pgfqpoint{3.785199in}{1.831289in}}%
\pgfpathlineto{\pgfqpoint{3.777391in}{1.823958in}}%
\pgfpathlineto{\pgfqpoint{3.769577in}{1.816712in}}%
\pgfpathlineto{\pgfqpoint{3.761756in}{1.809552in}}%
\pgfpathlineto{\pgfqpoint{3.753928in}{1.802482in}}%
\pgfpathclose%
\pgfusepath{fill}%
\end{pgfscope}%
\begin{pgfscope}%
\pgfpathrectangle{\pgfqpoint{1.254980in}{0.150000in}}{\pgfqpoint{5.490039in}{5.490039in}}%
\pgfusepath{clip}%
\pgfsetbuttcap%
\pgfsetroundjoin%
\definecolor{currentfill}{rgb}{0.282884,0.135920,0.453427}%
\pgfsetfillcolor{currentfill}%
\pgfsetfillopacity{0.700000}%
\pgfsetlinewidth{0.000000pt}%
\definecolor{currentstroke}{rgb}{0.000000,0.000000,0.000000}%
\pgfsetstrokecolor{currentstroke}%
\pgfsetdash{}{0pt}%
\pgfpathmoveto{\pgfqpoint{4.587544in}{1.958071in}}%
\pgfpathlineto{\pgfqpoint{4.601050in}{1.958375in}}%
\pgfpathlineto{\pgfqpoint{4.614565in}{1.958795in}}%
\pgfpathlineto{\pgfqpoint{4.628089in}{1.959332in}}%
\pgfpathlineto{\pgfqpoint{4.641624in}{1.959984in}}%
\pgfpathlineto{\pgfqpoint{4.649156in}{1.970862in}}%
\pgfpathlineto{\pgfqpoint{4.656682in}{1.981730in}}%
\pgfpathlineto{\pgfqpoint{4.664204in}{1.992586in}}%
\pgfpathlineto{\pgfqpoint{4.671721in}{2.003429in}}%
\pgfpathlineto{\pgfqpoint{4.658193in}{2.002627in}}%
\pgfpathlineto{\pgfqpoint{4.644674in}{2.001941in}}%
\pgfpathlineto{\pgfqpoint{4.631165in}{2.001371in}}%
\pgfpathlineto{\pgfqpoint{4.617666in}{2.000917in}}%
\pgfpathlineto{\pgfqpoint{4.610143in}{1.990218in}}%
\pgfpathlineto{\pgfqpoint{4.602615in}{1.979510in}}%
\pgfpathlineto{\pgfqpoint{4.595082in}{1.968794in}}%
\pgfpathlineto{\pgfqpoint{4.587544in}{1.958071in}}%
\pgfpathclose%
\pgfusepath{fill}%
\end{pgfscope}%
\begin{pgfscope}%
\pgfpathrectangle{\pgfqpoint{1.254980in}{0.150000in}}{\pgfqpoint{5.490039in}{5.490039in}}%
\pgfusepath{clip}%
\pgfsetbuttcap%
\pgfsetroundjoin%
\definecolor{currentfill}{rgb}{0.272594,0.025563,0.353093}%
\pgfsetfillcolor{currentfill}%
\pgfsetfillopacity{0.700000}%
\pgfsetlinewidth{0.000000pt}%
\definecolor{currentstroke}{rgb}{0.000000,0.000000,0.000000}%
\pgfsetstrokecolor{currentstroke}%
\pgfsetdash{}{0pt}%
\pgfpathmoveto{\pgfqpoint{3.891535in}{1.782461in}}%
\pgfpathlineto{\pgfqpoint{3.904847in}{1.776931in}}%
\pgfpathlineto{\pgfqpoint{3.918165in}{1.771526in}}%
\pgfpathlineto{\pgfqpoint{3.931487in}{1.766247in}}%
\pgfpathlineto{\pgfqpoint{3.944814in}{1.761094in}}%
\pgfpathlineto{\pgfqpoint{3.952570in}{1.769309in}}%
\pgfpathlineto{\pgfqpoint{3.960320in}{1.777593in}}%
\pgfpathlineto{\pgfqpoint{3.968064in}{1.785943in}}%
\pgfpathlineto{\pgfqpoint{3.975803in}{1.794357in}}%
\pgfpathlineto{\pgfqpoint{3.962489in}{1.799250in}}%
\pgfpathlineto{\pgfqpoint{3.949181in}{1.804268in}}%
\pgfpathlineto{\pgfqpoint{3.935878in}{1.809411in}}%
\pgfpathlineto{\pgfqpoint{3.922580in}{1.814681in}}%
\pgfpathlineto{\pgfqpoint{3.914828in}{1.806521in}}%
\pgfpathlineto{\pgfqpoint{3.907070in}{1.798430in}}%
\pgfpathlineto{\pgfqpoint{3.899306in}{1.790409in}}%
\pgfpathlineto{\pgfqpoint{3.891535in}{1.782461in}}%
\pgfpathclose%
\pgfusepath{fill}%
\end{pgfscope}%
\begin{pgfscope}%
\pgfpathrectangle{\pgfqpoint{1.254980in}{0.150000in}}{\pgfqpoint{5.490039in}{5.490039in}}%
\pgfusepath{clip}%
\pgfsetbuttcap%
\pgfsetroundjoin%
\definecolor{currentfill}{rgb}{0.283091,0.110553,0.431554}%
\pgfsetfillcolor{currentfill}%
\pgfsetfillopacity{0.700000}%
\pgfsetlinewidth{0.000000pt}%
\definecolor{currentstroke}{rgb}{0.000000,0.000000,0.000000}%
\pgfsetstrokecolor{currentstroke}%
\pgfsetdash{}{0pt}%
\pgfpathmoveto{\pgfqpoint{4.503394in}{1.915750in}}%
\pgfpathlineto{\pgfqpoint{4.516868in}{1.915421in}}%
\pgfpathlineto{\pgfqpoint{4.530352in}{1.915209in}}%
\pgfpathlineto{\pgfqpoint{4.543845in}{1.915114in}}%
\pgfpathlineto{\pgfqpoint{4.557347in}{1.915136in}}%
\pgfpathlineto{\pgfqpoint{4.564903in}{1.925874in}}%
\pgfpathlineto{\pgfqpoint{4.572455in}{1.936610in}}%
\pgfpathlineto{\pgfqpoint{4.580002in}{1.947343in}}%
\pgfpathlineto{\pgfqpoint{4.587544in}{1.958071in}}%
\pgfpathlineto{\pgfqpoint{4.574049in}{1.957884in}}%
\pgfpathlineto{\pgfqpoint{4.560562in}{1.957813in}}%
\pgfpathlineto{\pgfqpoint{4.547085in}{1.957859in}}%
\pgfpathlineto{\pgfqpoint{4.533618in}{1.958022in}}%
\pgfpathlineto{\pgfqpoint{4.526069in}{1.947454in}}%
\pgfpathlineto{\pgfqpoint{4.518515in}{1.936885in}}%
\pgfpathlineto{\pgfqpoint{4.510957in}{1.926316in}}%
\pgfpathlineto{\pgfqpoint{4.503394in}{1.915750in}}%
\pgfpathclose%
\pgfusepath{fill}%
\end{pgfscope}%
\begin{pgfscope}%
\pgfpathrectangle{\pgfqpoint{1.254980in}{0.150000in}}{\pgfqpoint{5.490039in}{5.490039in}}%
\pgfusepath{clip}%
\pgfsetbuttcap%
\pgfsetroundjoin%
\definecolor{currentfill}{rgb}{0.280868,0.160771,0.472899}%
\pgfsetfillcolor{currentfill}%
\pgfsetfillopacity{0.700000}%
\pgfsetlinewidth{0.000000pt}%
\definecolor{currentstroke}{rgb}{0.000000,0.000000,0.000000}%
\pgfsetstrokecolor{currentstroke}%
\pgfsetdash{}{0pt}%
\pgfpathmoveto{\pgfqpoint{4.671721in}{2.003429in}}%
\pgfpathlineto{\pgfqpoint{4.685260in}{2.004348in}}%
\pgfpathlineto{\pgfqpoint{4.698809in}{2.005382in}}%
\pgfpathlineto{\pgfqpoint{4.712369in}{2.006531in}}%
\pgfpathlineto{\pgfqpoint{4.725939in}{2.007796in}}%
\pgfpathlineto{\pgfqpoint{4.733445in}{2.018765in}}%
\pgfpathlineto{\pgfqpoint{4.740947in}{2.029717in}}%
\pgfpathlineto{\pgfqpoint{4.748444in}{2.040648in}}%
\pgfpathlineto{\pgfqpoint{4.755937in}{2.051559in}}%
\pgfpathlineto{\pgfqpoint{4.742372in}{2.050160in}}%
\pgfpathlineto{\pgfqpoint{4.728819in}{2.048877in}}%
\pgfpathlineto{\pgfqpoint{4.715275in}{2.047709in}}%
\pgfpathlineto{\pgfqpoint{4.701742in}{2.046657in}}%
\pgfpathlineto{\pgfqpoint{4.694244in}{2.035874in}}%
\pgfpathlineto{\pgfqpoint{4.686741in}{2.025074in}}%
\pgfpathlineto{\pgfqpoint{4.679234in}{2.014259in}}%
\pgfpathlineto{\pgfqpoint{4.671721in}{2.003429in}}%
\pgfpathclose%
\pgfusepath{fill}%
\end{pgfscope}%
\begin{pgfscope}%
\pgfpathrectangle{\pgfqpoint{1.254980in}{0.150000in}}{\pgfqpoint{5.490039in}{5.490039in}}%
\pgfusepath{clip}%
\pgfsetbuttcap%
\pgfsetroundjoin%
\definecolor{currentfill}{rgb}{0.165117,0.467423,0.558141}%
\pgfsetfillcolor{currentfill}%
\pgfsetfillopacity{0.700000}%
\pgfsetlinewidth{0.000000pt}%
\definecolor{currentstroke}{rgb}{0.000000,0.000000,0.000000}%
\pgfsetstrokecolor{currentstroke}%
\pgfsetdash{}{0pt}%
\pgfpathmoveto{\pgfqpoint{5.659067in}{2.706823in}}%
\pgfpathlineto{\pgfqpoint{5.673088in}{2.713218in}}%
\pgfpathlineto{\pgfqpoint{5.687125in}{2.719725in}}%
\pgfpathlineto{\pgfqpoint{5.701176in}{2.726345in}}%
\pgfpathlineto{\pgfqpoint{5.715244in}{2.733076in}}%
\pgfpathlineto{\pgfqpoint{5.722382in}{2.741896in}}%
\pgfpathlineto{\pgfqpoint{5.729513in}{2.750653in}}%
\pgfpathlineto{\pgfqpoint{5.736638in}{2.759348in}}%
\pgfpathlineto{\pgfqpoint{5.743755in}{2.767982in}}%
\pgfpathlineto{\pgfqpoint{5.729698in}{2.761330in}}%
\pgfpathlineto{\pgfqpoint{5.715657in}{2.754791in}}%
\pgfpathlineto{\pgfqpoint{5.701631in}{2.748363in}}%
\pgfpathlineto{\pgfqpoint{5.687620in}{2.742048in}}%
\pgfpathlineto{\pgfqpoint{5.680492in}{2.733328in}}%
\pgfpathlineto{\pgfqpoint{5.673357in}{2.724551in}}%
\pgfpathlineto{\pgfqpoint{5.666215in}{2.715716in}}%
\pgfpathlineto{\pgfqpoint{5.659067in}{2.706823in}}%
\pgfpathclose%
\pgfusepath{fill}%
\end{pgfscope}%
\begin{pgfscope}%
\pgfpathrectangle{\pgfqpoint{1.254980in}{0.150000in}}{\pgfqpoint{5.490039in}{5.490039in}}%
\pgfusepath{clip}%
\pgfsetbuttcap%
\pgfsetroundjoin%
\definecolor{currentfill}{rgb}{0.282327,0.094955,0.417331}%
\pgfsetfillcolor{currentfill}%
\pgfsetfillopacity{0.700000}%
\pgfsetlinewidth{0.000000pt}%
\definecolor{currentstroke}{rgb}{0.000000,0.000000,0.000000}%
\pgfsetstrokecolor{currentstroke}%
\pgfsetdash{}{0pt}%
\pgfpathmoveto{\pgfqpoint{4.419257in}{1.876740in}}%
\pgfpathlineto{\pgfqpoint{4.432703in}{1.875760in}}%
\pgfpathlineto{\pgfqpoint{4.446158in}{1.874897in}}%
\pgfpathlineto{\pgfqpoint{4.459622in}{1.874153in}}%
\pgfpathlineto{\pgfqpoint{4.473094in}{1.873525in}}%
\pgfpathlineto{\pgfqpoint{4.480676in}{1.884072in}}%
\pgfpathlineto{\pgfqpoint{4.488253in}{1.894626in}}%
\pgfpathlineto{\pgfqpoint{4.495826in}{1.905186in}}%
\pgfpathlineto{\pgfqpoint{4.503394in}{1.915750in}}%
\pgfpathlineto{\pgfqpoint{4.489929in}{1.916196in}}%
\pgfpathlineto{\pgfqpoint{4.476472in}{1.916759in}}%
\pgfpathlineto{\pgfqpoint{4.463025in}{1.917440in}}%
\pgfpathlineto{\pgfqpoint{4.449586in}{1.918239in}}%
\pgfpathlineto{\pgfqpoint{4.442011in}{1.907851in}}%
\pgfpathlineto{\pgfqpoint{4.434431in}{1.897471in}}%
\pgfpathlineto{\pgfqpoint{4.426846in}{1.887100in}}%
\pgfpathlineto{\pgfqpoint{4.419257in}{1.876740in}}%
\pgfpathclose%
\pgfusepath{fill}%
\end{pgfscope}%
\begin{pgfscope}%
\pgfpathrectangle{\pgfqpoint{1.254980in}{0.150000in}}{\pgfqpoint{5.490039in}{5.490039in}}%
\pgfusepath{clip}%
\pgfsetbuttcap%
\pgfsetroundjoin%
\definecolor{currentfill}{rgb}{0.277134,0.185228,0.489898}%
\pgfsetfillcolor{currentfill}%
\pgfsetfillopacity{0.700000}%
\pgfsetlinewidth{0.000000pt}%
\definecolor{currentstroke}{rgb}{0.000000,0.000000,0.000000}%
\pgfsetstrokecolor{currentstroke}%
\pgfsetdash{}{0pt}%
\pgfpathmoveto{\pgfqpoint{4.755937in}{2.051559in}}%
\pgfpathlineto{\pgfqpoint{4.769512in}{2.053074in}}%
\pgfpathlineto{\pgfqpoint{4.783097in}{2.054703in}}%
\pgfpathlineto{\pgfqpoint{4.796694in}{2.056447in}}%
\pgfpathlineto{\pgfqpoint{4.810301in}{2.058307in}}%
\pgfpathlineto{\pgfqpoint{4.817783in}{2.069321in}}%
\pgfpathlineto{\pgfqpoint{4.825261in}{2.080309in}}%
\pgfpathlineto{\pgfqpoint{4.832733in}{2.091271in}}%
\pgfpathlineto{\pgfqpoint{4.840200in}{2.102205in}}%
\pgfpathlineto{\pgfqpoint{4.826598in}{2.100227in}}%
\pgfpathlineto{\pgfqpoint{4.813007in}{2.098365in}}%
\pgfpathlineto{\pgfqpoint{4.799426in}{2.096617in}}%
\pgfpathlineto{\pgfqpoint{4.785857in}{2.094985in}}%
\pgfpathlineto{\pgfqpoint{4.778384in}{2.084163in}}%
\pgfpathlineto{\pgfqpoint{4.770907in}{2.073317in}}%
\pgfpathlineto{\pgfqpoint{4.763424in}{2.062449in}}%
\pgfpathlineto{\pgfqpoint{4.755937in}{2.051559in}}%
\pgfpathclose%
\pgfusepath{fill}%
\end{pgfscope}%
\begin{pgfscope}%
\pgfpathrectangle{\pgfqpoint{1.254980in}{0.150000in}}{\pgfqpoint{5.490039in}{5.490039in}}%
\pgfusepath{clip}%
\pgfsetbuttcap%
\pgfsetroundjoin%
\definecolor{currentfill}{rgb}{0.214298,0.355619,0.551184}%
\pgfsetfillcolor{currentfill}%
\pgfsetfillopacity{0.700000}%
\pgfsetlinewidth{0.000000pt}%
\definecolor{currentstroke}{rgb}{0.000000,0.000000,0.000000}%
\pgfsetstrokecolor{currentstroke}%
\pgfsetdash{}{0pt}%
\pgfpathmoveto{\pgfqpoint{5.291757in}{2.426093in}}%
\pgfpathlineto{\pgfqpoint{5.305586in}{2.430832in}}%
\pgfpathlineto{\pgfqpoint{5.319429in}{2.435683in}}%
\pgfpathlineto{\pgfqpoint{5.333285in}{2.440648in}}%
\pgfpathlineto{\pgfqpoint{5.347155in}{2.445726in}}%
\pgfpathlineto{\pgfqpoint{5.354454in}{2.455911in}}%
\pgfpathlineto{\pgfqpoint{5.361747in}{2.466037in}}%
\pgfpathlineto{\pgfqpoint{5.369033in}{2.476106in}}%
\pgfpathlineto{\pgfqpoint{5.376314in}{2.486118in}}%
\pgfpathlineto{\pgfqpoint{5.362450in}{2.481035in}}%
\pgfpathlineto{\pgfqpoint{5.348600in}{2.476066in}}%
\pgfpathlineto{\pgfqpoint{5.334764in}{2.471210in}}%
\pgfpathlineto{\pgfqpoint{5.320942in}{2.466467in}}%
\pgfpathlineto{\pgfqpoint{5.313655in}{2.456454in}}%
\pgfpathlineto{\pgfqpoint{5.306362in}{2.446387in}}%
\pgfpathlineto{\pgfqpoint{5.299062in}{2.436267in}}%
\pgfpathlineto{\pgfqpoint{5.291757in}{2.426093in}}%
\pgfpathclose%
\pgfusepath{fill}%
\end{pgfscope}%
\begin{pgfscope}%
\pgfpathrectangle{\pgfqpoint{1.254980in}{0.150000in}}{\pgfqpoint{5.490039in}{5.490039in}}%
\pgfusepath{clip}%
\pgfsetbuttcap%
\pgfsetroundjoin%
\definecolor{currentfill}{rgb}{0.233603,0.313828,0.543914}%
\pgfsetfillcolor{currentfill}%
\pgfsetfillopacity{0.700000}%
\pgfsetlinewidth{0.000000pt}%
\definecolor{currentstroke}{rgb}{0.000000,0.000000,0.000000}%
\pgfsetstrokecolor{currentstroke}%
\pgfsetdash{}{0pt}%
\pgfpathmoveto{\pgfqpoint{2.913162in}{2.384374in}}%
\pgfpathlineto{\pgfqpoint{2.926531in}{2.368395in}}%
\pgfpathlineto{\pgfqpoint{2.939897in}{2.352584in}}%
\pgfpathlineto{\pgfqpoint{2.953258in}{2.336940in}}%
\pgfpathlineto{\pgfqpoint{2.966616in}{2.321462in}}%
\pgfpathlineto{\pgfqpoint{2.974876in}{2.322860in}}%
\pgfpathlineto{\pgfqpoint{2.983123in}{2.324432in}}%
\pgfpathlineto{\pgfqpoint{2.991359in}{2.326175in}}%
\pgfpathlineto{\pgfqpoint{2.999583in}{2.328088in}}%
\pgfpathlineto{\pgfqpoint{2.986259in}{2.343223in}}%
\pgfpathlineto{\pgfqpoint{2.972930in}{2.358523in}}%
\pgfpathlineto{\pgfqpoint{2.959599in}{2.373990in}}%
\pgfpathlineto{\pgfqpoint{2.946263in}{2.389624in}}%
\pgfpathlineto{\pgfqpoint{2.938006in}{2.388049in}}%
\pgfpathlineto{\pgfqpoint{2.929737in}{2.386647in}}%
\pgfpathlineto{\pgfqpoint{2.921456in}{2.385421in}}%
\pgfpathlineto{\pgfqpoint{2.913162in}{2.384374in}}%
\pgfpathclose%
\pgfusepath{fill}%
\end{pgfscope}%
\begin{pgfscope}%
\pgfpathrectangle{\pgfqpoint{1.254980in}{0.150000in}}{\pgfqpoint{5.490039in}{5.490039in}}%
\pgfusepath{clip}%
\pgfsetbuttcap%
\pgfsetroundjoin%
\definecolor{currentfill}{rgb}{0.244972,0.287675,0.537260}%
\pgfsetfillcolor{currentfill}%
\pgfsetfillopacity{0.700000}%
\pgfsetlinewidth{0.000000pt}%
\definecolor{currentstroke}{rgb}{0.000000,0.000000,0.000000}%
\pgfsetstrokecolor{currentstroke}%
\pgfsetdash{}{0pt}%
\pgfpathmoveto{\pgfqpoint{2.966616in}{2.321462in}}%
\pgfpathlineto{\pgfqpoint{2.979970in}{2.306149in}}%
\pgfpathlineto{\pgfqpoint{2.993321in}{2.290999in}}%
\pgfpathlineto{\pgfqpoint{3.006668in}{2.276013in}}%
\pgfpathlineto{\pgfqpoint{3.020012in}{2.261189in}}%
\pgfpathlineto{\pgfqpoint{3.028239in}{2.262935in}}%
\pgfpathlineto{\pgfqpoint{3.036454in}{2.264851in}}%
\pgfpathlineto{\pgfqpoint{3.044658in}{2.266935in}}%
\pgfpathlineto{\pgfqpoint{3.052850in}{2.269183in}}%
\pgfpathlineto{\pgfqpoint{3.039538in}{2.283665in}}%
\pgfpathlineto{\pgfqpoint{3.026223in}{2.298310in}}%
\pgfpathlineto{\pgfqpoint{3.012905in}{2.313117in}}%
\pgfpathlineto{\pgfqpoint{2.999583in}{2.328088in}}%
\pgfpathlineto{\pgfqpoint{2.991359in}{2.326175in}}%
\pgfpathlineto{\pgfqpoint{2.983123in}{2.324432in}}%
\pgfpathlineto{\pgfqpoint{2.974876in}{2.322860in}}%
\pgfpathlineto{\pgfqpoint{2.966616in}{2.321462in}}%
\pgfpathclose%
\pgfusepath{fill}%
\end{pgfscope}%
\begin{pgfscope}%
\pgfpathrectangle{\pgfqpoint{1.254980in}{0.150000in}}{\pgfqpoint{5.490039in}{5.490039in}}%
\pgfusepath{clip}%
\pgfsetbuttcap%
\pgfsetroundjoin%
\definecolor{currentfill}{rgb}{0.220057,0.343307,0.549413}%
\pgfsetfillcolor{currentfill}%
\pgfsetfillopacity{0.700000}%
\pgfsetlinewidth{0.000000pt}%
\definecolor{currentstroke}{rgb}{0.000000,0.000000,0.000000}%
\pgfsetstrokecolor{currentstroke}%
\pgfsetdash{}{0pt}%
\pgfpathmoveto{\pgfqpoint{2.859641in}{2.449985in}}%
\pgfpathlineto{\pgfqpoint{2.873028in}{2.433325in}}%
\pgfpathlineto{\pgfqpoint{2.886411in}{2.416838in}}%
\pgfpathlineto{\pgfqpoint{2.899789in}{2.400521in}}%
\pgfpathlineto{\pgfqpoint{2.913162in}{2.384374in}}%
\pgfpathlineto{\pgfqpoint{2.921456in}{2.385421in}}%
\pgfpathlineto{\pgfqpoint{2.929737in}{2.386647in}}%
\pgfpathlineto{\pgfqpoint{2.938006in}{2.388049in}}%
\pgfpathlineto{\pgfqpoint{2.946263in}{2.389624in}}%
\pgfpathlineto{\pgfqpoint{2.932924in}{2.405426in}}%
\pgfpathlineto{\pgfqpoint{2.919581in}{2.421397in}}%
\pgfpathlineto{\pgfqpoint{2.906233in}{2.437538in}}%
\pgfpathlineto{\pgfqpoint{2.892881in}{2.453851in}}%
\pgfpathlineto{\pgfqpoint{2.884590in}{2.452615in}}%
\pgfpathlineto{\pgfqpoint{2.876287in}{2.451556in}}%
\pgfpathlineto{\pgfqpoint{2.867970in}{2.450679in}}%
\pgfpathlineto{\pgfqpoint{2.859641in}{2.449985in}}%
\pgfpathclose%
\pgfusepath{fill}%
\end{pgfscope}%
\begin{pgfscope}%
\pgfpathrectangle{\pgfqpoint{1.254980in}{0.150000in}}{\pgfqpoint{5.490039in}{5.490039in}}%
\pgfusepath{clip}%
\pgfsetbuttcap%
\pgfsetroundjoin%
\definecolor{currentfill}{rgb}{0.277941,0.056324,0.381191}%
\pgfsetfillcolor{currentfill}%
\pgfsetfillopacity{0.700000}%
\pgfsetlinewidth{0.000000pt}%
\definecolor{currentstroke}{rgb}{0.000000,0.000000,0.000000}%
\pgfsetstrokecolor{currentstroke}%
\pgfsetdash{}{0pt}%
\pgfpathmoveto{\pgfqpoint{3.616171in}{1.836850in}}%
\pgfpathlineto{\pgfqpoint{3.629458in}{1.828672in}}%
\pgfpathlineto{\pgfqpoint{3.642748in}{1.820628in}}%
\pgfpathlineto{\pgfqpoint{3.656041in}{1.812717in}}%
\pgfpathlineto{\pgfqpoint{3.669336in}{1.804938in}}%
\pgfpathlineto{\pgfqpoint{3.677210in}{1.811329in}}%
\pgfpathlineto{\pgfqpoint{3.685076in}{1.817823in}}%
\pgfpathlineto{\pgfqpoint{3.692935in}{1.824416in}}%
\pgfpathlineto{\pgfqpoint{3.700787in}{1.831107in}}%
\pgfpathlineto{\pgfqpoint{3.687511in}{1.838592in}}%
\pgfpathlineto{\pgfqpoint{3.674237in}{1.846208in}}%
\pgfpathlineto{\pgfqpoint{3.660966in}{1.853957in}}%
\pgfpathlineto{\pgfqpoint{3.647698in}{1.861840in}}%
\pgfpathlineto{\pgfqpoint{3.639828in}{1.855437in}}%
\pgfpathlineto{\pgfqpoint{3.631949in}{1.849137in}}%
\pgfpathlineto{\pgfqpoint{3.624064in}{1.842940in}}%
\pgfpathlineto{\pgfqpoint{3.616171in}{1.836850in}}%
\pgfpathclose%
\pgfusepath{fill}%
\end{pgfscope}%
\begin{pgfscope}%
\pgfpathrectangle{\pgfqpoint{1.254980in}{0.150000in}}{\pgfqpoint{5.490039in}{5.490039in}}%
\pgfusepath{clip}%
\pgfsetbuttcap%
\pgfsetroundjoin%
\definecolor{currentfill}{rgb}{0.271828,0.209303,0.504434}%
\pgfsetfillcolor{currentfill}%
\pgfsetfillopacity{0.700000}%
\pgfsetlinewidth{0.000000pt}%
\definecolor{currentstroke}{rgb}{0.000000,0.000000,0.000000}%
\pgfsetstrokecolor{currentstroke}%
\pgfsetdash{}{0pt}%
\pgfpathmoveto{\pgfqpoint{4.840200in}{2.102205in}}%
\pgfpathlineto{\pgfqpoint{4.853814in}{2.104297in}}%
\pgfpathlineto{\pgfqpoint{4.867438in}{2.106504in}}%
\pgfpathlineto{\pgfqpoint{4.881074in}{2.108826in}}%
\pgfpathlineto{\pgfqpoint{4.894721in}{2.111262in}}%
\pgfpathlineto{\pgfqpoint{4.902179in}{2.122276in}}%
\pgfpathlineto{\pgfqpoint{4.909631in}{2.133258in}}%
\pgfpathlineto{\pgfqpoint{4.917078in}{2.144206in}}%
\pgfpathlineto{\pgfqpoint{4.924520in}{2.155120in}}%
\pgfpathlineto{\pgfqpoint{4.910878in}{2.152581in}}%
\pgfpathlineto{\pgfqpoint{4.897247in}{2.150158in}}%
\pgfpathlineto{\pgfqpoint{4.883628in}{2.147848in}}%
\pgfpathlineto{\pgfqpoint{4.870019in}{2.145654in}}%
\pgfpathlineto{\pgfqpoint{4.862572in}{2.134836in}}%
\pgfpathlineto{\pgfqpoint{4.855120in}{2.123988in}}%
\pgfpathlineto{\pgfqpoint{4.847663in}{2.113111in}}%
\pgfpathlineto{\pgfqpoint{4.840200in}{2.102205in}}%
\pgfpathclose%
\pgfusepath{fill}%
\end{pgfscope}%
\begin{pgfscope}%
\pgfpathrectangle{\pgfqpoint{1.254980in}{0.150000in}}{\pgfqpoint{5.490039in}{5.490039in}}%
\pgfusepath{clip}%
\pgfsetbuttcap%
\pgfsetroundjoin%
\definecolor{currentfill}{rgb}{0.255645,0.260703,0.528312}%
\pgfsetfillcolor{currentfill}%
\pgfsetfillopacity{0.700000}%
\pgfsetlinewidth{0.000000pt}%
\definecolor{currentstroke}{rgb}{0.000000,0.000000,0.000000}%
\pgfsetstrokecolor{currentstroke}%
\pgfsetdash{}{0pt}%
\pgfpathmoveto{\pgfqpoint{3.020012in}{2.261189in}}%
\pgfpathlineto{\pgfqpoint{3.033354in}{2.246526in}}%
\pgfpathlineto{\pgfqpoint{3.046692in}{2.232023in}}%
\pgfpathlineto{\pgfqpoint{3.060028in}{2.217680in}}%
\pgfpathlineto{\pgfqpoint{3.073361in}{2.203496in}}%
\pgfpathlineto{\pgfqpoint{3.081555in}{2.205590in}}%
\pgfpathlineto{\pgfqpoint{3.089739in}{2.207848in}}%
\pgfpathlineto{\pgfqpoint{3.097912in}{2.210270in}}%
\pgfpathlineto{\pgfqpoint{3.106073in}{2.212851in}}%
\pgfpathlineto{\pgfqpoint{3.092771in}{2.226696in}}%
\pgfpathlineto{\pgfqpoint{3.079467in}{2.240698in}}%
\pgfpathlineto{\pgfqpoint{3.066160in}{2.254861in}}%
\pgfpathlineto{\pgfqpoint{3.052850in}{2.269183in}}%
\pgfpathlineto{\pgfqpoint{3.044658in}{2.266935in}}%
\pgfpathlineto{\pgfqpoint{3.036454in}{2.264851in}}%
\pgfpathlineto{\pgfqpoint{3.028239in}{2.262935in}}%
\pgfpathlineto{\pgfqpoint{3.020012in}{2.261189in}}%
\pgfpathclose%
\pgfusepath{fill}%
\end{pgfscope}%
\begin{pgfscope}%
\pgfpathrectangle{\pgfqpoint{1.254980in}{0.150000in}}{\pgfqpoint{5.490039in}{5.490039in}}%
\pgfusepath{clip}%
\pgfsetbuttcap%
\pgfsetroundjoin%
\definecolor{currentfill}{rgb}{0.280267,0.073417,0.397163}%
\pgfsetfillcolor{currentfill}%
\pgfsetfillopacity{0.700000}%
\pgfsetlinewidth{0.000000pt}%
\definecolor{currentstroke}{rgb}{0.000000,0.000000,0.000000}%
\pgfsetstrokecolor{currentstroke}%
\pgfsetdash{}{0pt}%
\pgfpathmoveto{\pgfqpoint{4.335118in}{1.841328in}}%
\pgfpathlineto{\pgfqpoint{4.348539in}{1.839677in}}%
\pgfpathlineto{\pgfqpoint{4.361969in}{1.838145in}}%
\pgfpathlineto{\pgfqpoint{4.375406in}{1.836731in}}%
\pgfpathlineto{\pgfqpoint{4.388851in}{1.835436in}}%
\pgfpathlineto{\pgfqpoint{4.396460in}{1.845739in}}%
\pgfpathlineto{\pgfqpoint{4.404064in}{1.856058in}}%
\pgfpathlineto{\pgfqpoint{4.411663in}{1.866392in}}%
\pgfpathlineto{\pgfqpoint{4.419257in}{1.876740in}}%
\pgfpathlineto{\pgfqpoint{4.405819in}{1.877838in}}%
\pgfpathlineto{\pgfqpoint{4.392390in}{1.879055in}}%
\pgfpathlineto{\pgfqpoint{4.378969in}{1.880390in}}%
\pgfpathlineto{\pgfqpoint{4.365556in}{1.881844in}}%
\pgfpathlineto{\pgfqpoint{4.357954in}{1.871687in}}%
\pgfpathlineto{\pgfqpoint{4.350347in}{1.861548in}}%
\pgfpathlineto{\pgfqpoint{4.342735in}{1.851428in}}%
\pgfpathlineto{\pgfqpoint{4.335118in}{1.841328in}}%
\pgfpathclose%
\pgfusepath{fill}%
\end{pgfscope}%
\begin{pgfscope}%
\pgfpathrectangle{\pgfqpoint{1.254980in}{0.150000in}}{\pgfqpoint{5.490039in}{5.490039in}}%
\pgfusepath{clip}%
\pgfsetbuttcap%
\pgfsetroundjoin%
\definecolor{currentfill}{rgb}{0.273809,0.031497,0.358853}%
\pgfsetfillcolor{currentfill}%
\pgfsetfillopacity{0.700000}%
\pgfsetlinewidth{0.000000pt}%
\definecolor{currentstroke}{rgb}{0.000000,0.000000,0.000000}%
\pgfsetstrokecolor{currentstroke}%
\pgfsetdash{}{0pt}%
\pgfpathmoveto{\pgfqpoint{4.029111in}{1.776032in}}%
\pgfpathlineto{\pgfqpoint{4.042452in}{1.771760in}}%
\pgfpathlineto{\pgfqpoint{4.055799in}{1.767612in}}%
\pgfpathlineto{\pgfqpoint{4.069152in}{1.763586in}}%
\pgfpathlineto{\pgfqpoint{4.082511in}{1.759683in}}%
\pgfpathlineto{\pgfqpoint{4.090219in}{1.768661in}}%
\pgfpathlineto{\pgfqpoint{4.097922in}{1.777691in}}%
\pgfpathlineto{\pgfqpoint{4.105619in}{1.786772in}}%
\pgfpathlineto{\pgfqpoint{4.113311in}{1.795902in}}%
\pgfpathlineto{\pgfqpoint{4.099963in}{1.799561in}}%
\pgfpathlineto{\pgfqpoint{4.086622in}{1.803342in}}%
\pgfpathlineto{\pgfqpoint{4.073287in}{1.807246in}}%
\pgfpathlineto{\pgfqpoint{4.059958in}{1.811273in}}%
\pgfpathlineto{\pgfqpoint{4.052254in}{1.802382in}}%
\pgfpathlineto{\pgfqpoint{4.044545in}{1.793543in}}%
\pgfpathlineto{\pgfqpoint{4.036831in}{1.784759in}}%
\pgfpathlineto{\pgfqpoint{4.029111in}{1.776032in}}%
\pgfpathclose%
\pgfusepath{fill}%
\end{pgfscope}%
\begin{pgfscope}%
\pgfpathrectangle{\pgfqpoint{1.254980in}{0.150000in}}{\pgfqpoint{5.490039in}{5.490039in}}%
\pgfusepath{clip}%
\pgfsetbuttcap%
\pgfsetroundjoin%
\definecolor{currentfill}{rgb}{0.206756,0.371758,0.553117}%
\pgfsetfillcolor{currentfill}%
\pgfsetfillopacity{0.700000}%
\pgfsetlinewidth{0.000000pt}%
\definecolor{currentstroke}{rgb}{0.000000,0.000000,0.000000}%
\pgfsetstrokecolor{currentstroke}%
\pgfsetdash{}{0pt}%
\pgfpathmoveto{\pgfqpoint{2.806044in}{2.518359in}}%
\pgfpathlineto{\pgfqpoint{2.819451in}{2.501002in}}%
\pgfpathlineto{\pgfqpoint{2.832853in}{2.483822in}}%
\pgfpathlineto{\pgfqpoint{2.846250in}{2.466817in}}%
\pgfpathlineto{\pgfqpoint{2.859641in}{2.449985in}}%
\pgfpathlineto{\pgfqpoint{2.867970in}{2.450679in}}%
\pgfpathlineto{\pgfqpoint{2.876287in}{2.451556in}}%
\pgfpathlineto{\pgfqpoint{2.884590in}{2.452615in}}%
\pgfpathlineto{\pgfqpoint{2.892881in}{2.453851in}}%
\pgfpathlineto{\pgfqpoint{2.879525in}{2.470335in}}%
\pgfpathlineto{\pgfqpoint{2.866164in}{2.486993in}}%
\pgfpathlineto{\pgfqpoint{2.852798in}{2.503825in}}%
\pgfpathlineto{\pgfqpoint{2.839427in}{2.520832in}}%
\pgfpathlineto{\pgfqpoint{2.831101in}{2.519937in}}%
\pgfpathlineto{\pgfqpoint{2.822762in}{2.519225in}}%
\pgfpathlineto{\pgfqpoint{2.814409in}{2.518698in}}%
\pgfpathlineto{\pgfqpoint{2.806044in}{2.518359in}}%
\pgfpathclose%
\pgfusepath{fill}%
\end{pgfscope}%
\begin{pgfscope}%
\pgfpathrectangle{\pgfqpoint{1.254980in}{0.150000in}}{\pgfqpoint{5.490039in}{5.490039in}}%
\pgfusepath{clip}%
\pgfsetbuttcap%
\pgfsetroundjoin%
\definecolor{currentfill}{rgb}{0.282910,0.105393,0.426902}%
\pgfsetfillcolor{currentfill}%
\pgfsetfillopacity{0.700000}%
\pgfsetlinewidth{0.000000pt}%
\definecolor{currentstroke}{rgb}{0.000000,0.000000,0.000000}%
\pgfsetstrokecolor{currentstroke}%
\pgfsetdash{}{0pt}%
\pgfpathmoveto{\pgfqpoint{3.424981in}{1.926062in}}%
\pgfpathlineto{\pgfqpoint{3.438268in}{1.915931in}}%
\pgfpathlineto{\pgfqpoint{3.451557in}{1.905941in}}%
\pgfpathlineto{\pgfqpoint{3.464846in}{1.896089in}}%
\pgfpathlineto{\pgfqpoint{3.478137in}{1.886376in}}%
\pgfpathlineto{\pgfqpoint{3.486104in}{1.891379in}}%
\pgfpathlineto{\pgfqpoint{3.494063in}{1.896506in}}%
\pgfpathlineto{\pgfqpoint{3.502014in}{1.901756in}}%
\pgfpathlineto{\pgfqpoint{3.509956in}{1.907125in}}%
\pgfpathlineto{\pgfqpoint{3.496688in}{1.916524in}}%
\pgfpathlineto{\pgfqpoint{3.483421in}{1.926062in}}%
\pgfpathlineto{\pgfqpoint{3.470155in}{1.935739in}}%
\pgfpathlineto{\pgfqpoint{3.456891in}{1.945556in}}%
\pgfpathlineto{\pgfqpoint{3.448926in}{1.940495in}}%
\pgfpathlineto{\pgfqpoint{3.440953in}{1.935557in}}%
\pgfpathlineto{\pgfqpoint{3.432971in}{1.930745in}}%
\pgfpathlineto{\pgfqpoint{3.424981in}{1.926062in}}%
\pgfpathclose%
\pgfusepath{fill}%
\end{pgfscope}%
\begin{pgfscope}%
\pgfpathrectangle{\pgfqpoint{1.254980in}{0.150000in}}{\pgfqpoint{5.490039in}{5.490039in}}%
\pgfusepath{clip}%
\pgfsetbuttcap%
\pgfsetroundjoin%
\definecolor{currentfill}{rgb}{0.154815,0.493313,0.557840}%
\pgfsetfillcolor{currentfill}%
\pgfsetfillopacity{0.700000}%
\pgfsetlinewidth{0.000000pt}%
\definecolor{currentstroke}{rgb}{0.000000,0.000000,0.000000}%
\pgfsetstrokecolor{currentstroke}%
\pgfsetdash{}{0pt}%
\pgfpathmoveto{\pgfqpoint{5.743755in}{2.767982in}}%
\pgfpathlineto{\pgfqpoint{5.757828in}{2.774745in}}%
\pgfpathlineto{\pgfqpoint{5.771916in}{2.781621in}}%
\pgfpathlineto{\pgfqpoint{5.786021in}{2.788609in}}%
\pgfpathlineto{\pgfqpoint{5.800141in}{2.795709in}}%
\pgfpathlineto{\pgfqpoint{5.807241in}{2.804192in}}%
\pgfpathlineto{\pgfqpoint{5.814334in}{2.812613in}}%
\pgfpathlineto{\pgfqpoint{5.821420in}{2.820972in}}%
\pgfpathlineto{\pgfqpoint{5.828499in}{2.829270in}}%
\pgfpathlineto{\pgfqpoint{5.814390in}{2.822268in}}%
\pgfpathlineto{\pgfqpoint{5.800297in}{2.815377in}}%
\pgfpathlineto{\pgfqpoint{5.786220in}{2.808598in}}%
\pgfpathlineto{\pgfqpoint{5.772158in}{2.801931in}}%
\pgfpathlineto{\pgfqpoint{5.765068in}{2.793530in}}%
\pgfpathlineto{\pgfqpoint{5.757970in}{2.785072in}}%
\pgfpathlineto{\pgfqpoint{5.750866in}{2.776556in}}%
\pgfpathlineto{\pgfqpoint{5.743755in}{2.767982in}}%
\pgfpathclose%
\pgfusepath{fill}%
\end{pgfscope}%
\begin{pgfscope}%
\pgfpathrectangle{\pgfqpoint{1.254980in}{0.150000in}}{\pgfqpoint{5.490039in}{5.490039in}}%
\pgfusepath{clip}%
\pgfsetbuttcap%
\pgfsetroundjoin%
\definecolor{currentfill}{rgb}{0.124395,0.578002,0.548287}%
\pgfsetfillcolor{currentfill}%
\pgfsetfillopacity{0.700000}%
\pgfsetlinewidth{0.000000pt}%
\definecolor{currentstroke}{rgb}{0.000000,0.000000,0.000000}%
\pgfsetstrokecolor{currentstroke}%
\pgfsetdash{}{0pt}%
\pgfpathmoveto{\pgfqpoint{2.462337in}{3.072874in}}%
\pgfpathlineto{\pgfqpoint{2.475917in}{3.050476in}}%
\pgfpathlineto{\pgfqpoint{2.489487in}{3.028288in}}%
\pgfpathlineto{\pgfqpoint{2.503047in}{3.006310in}}%
\pgfpathlineto{\pgfqpoint{2.516598in}{2.984539in}}%
\pgfpathlineto{\pgfqpoint{2.525144in}{2.983540in}}%
\pgfpathlineto{\pgfqpoint{2.533675in}{2.982746in}}%
\pgfpathlineto{\pgfqpoint{2.542191in}{2.982155in}}%
\pgfpathlineto{\pgfqpoint{2.550693in}{2.981763in}}%
\pgfpathlineto{\pgfqpoint{2.537184in}{3.003186in}}%
\pgfpathlineto{\pgfqpoint{2.523666in}{3.024817in}}%
\pgfpathlineto{\pgfqpoint{2.510139in}{3.046655in}}%
\pgfpathlineto{\pgfqpoint{2.496602in}{3.068704in}}%
\pgfpathlineto{\pgfqpoint{2.488059in}{3.069437in}}%
\pgfpathlineto{\pgfqpoint{2.479501in}{3.070374in}}%
\pgfpathlineto{\pgfqpoint{2.470927in}{3.071519in}}%
\pgfpathlineto{\pgfqpoint{2.462337in}{3.072874in}}%
\pgfpathclose%
\pgfusepath{fill}%
\end{pgfscope}%
\begin{pgfscope}%
\pgfpathrectangle{\pgfqpoint{1.254980in}{0.150000in}}{\pgfqpoint{5.490039in}{5.490039in}}%
\pgfusepath{clip}%
\pgfsetbuttcap%
\pgfsetroundjoin%
\definecolor{currentfill}{rgb}{0.263663,0.237631,0.518762}%
\pgfsetfillcolor{currentfill}%
\pgfsetfillopacity{0.700000}%
\pgfsetlinewidth{0.000000pt}%
\definecolor{currentstroke}{rgb}{0.000000,0.000000,0.000000}%
\pgfsetstrokecolor{currentstroke}%
\pgfsetdash{}{0pt}%
\pgfpathmoveto{\pgfqpoint{3.073361in}{2.203496in}}%
\pgfpathlineto{\pgfqpoint{3.086691in}{2.189470in}}%
\pgfpathlineto{\pgfqpoint{3.100020in}{2.175601in}}%
\pgfpathlineto{\pgfqpoint{3.113346in}{2.161888in}}%
\pgfpathlineto{\pgfqpoint{3.126670in}{2.148330in}}%
\pgfpathlineto{\pgfqpoint{3.134834in}{2.150769in}}%
\pgfpathlineto{\pgfqpoint{3.142987in}{2.153368in}}%
\pgfpathlineto{\pgfqpoint{3.151130in}{2.156126in}}%
\pgfpathlineto{\pgfqpoint{3.159262in}{2.159039in}}%
\pgfpathlineto{\pgfqpoint{3.145967in}{2.172259in}}%
\pgfpathlineto{\pgfqpoint{3.132671in}{2.185633in}}%
\pgfpathlineto{\pgfqpoint{3.119373in}{2.199164in}}%
\pgfpathlineto{\pgfqpoint{3.106073in}{2.212851in}}%
\pgfpathlineto{\pgfqpoint{3.097912in}{2.210270in}}%
\pgfpathlineto{\pgfqpoint{3.089739in}{2.207848in}}%
\pgfpathlineto{\pgfqpoint{3.081555in}{2.205590in}}%
\pgfpathlineto{\pgfqpoint{3.073361in}{2.203496in}}%
\pgfpathclose%
\pgfusepath{fill}%
\end{pgfscope}%
\begin{pgfscope}%
\pgfpathrectangle{\pgfqpoint{1.254980in}{0.150000in}}{\pgfqpoint{5.490039in}{5.490039in}}%
\pgfusepath{clip}%
\pgfsetbuttcap%
\pgfsetroundjoin%
\definecolor{currentfill}{rgb}{0.263663,0.237631,0.518762}%
\pgfsetfillcolor{currentfill}%
\pgfsetfillopacity{0.700000}%
\pgfsetlinewidth{0.000000pt}%
\definecolor{currentstroke}{rgb}{0.000000,0.000000,0.000000}%
\pgfsetstrokecolor{currentstroke}%
\pgfsetdash{}{0pt}%
\pgfpathmoveto{\pgfqpoint{4.924520in}{2.155120in}}%
\pgfpathlineto{\pgfqpoint{4.938174in}{2.157772in}}%
\pgfpathlineto{\pgfqpoint{4.951840in}{2.160539in}}%
\pgfpathlineto{\pgfqpoint{4.965517in}{2.163420in}}%
\pgfpathlineto{\pgfqpoint{4.979207in}{2.166415in}}%
\pgfpathlineto{\pgfqpoint{4.986639in}{2.177387in}}%
\pgfpathlineto{\pgfqpoint{4.994065in}{2.188320in}}%
\pgfpathlineto{\pgfqpoint{5.001487in}{2.199213in}}%
\pgfpathlineto{\pgfqpoint{5.008903in}{2.210066in}}%
\pgfpathlineto{\pgfqpoint{4.995219in}{2.206985in}}%
\pgfpathlineto{\pgfqpoint{4.981546in}{2.204018in}}%
\pgfpathlineto{\pgfqpoint{4.967886in}{2.201165in}}%
\pgfpathlineto{\pgfqpoint{4.954237in}{2.198426in}}%
\pgfpathlineto{\pgfqpoint{4.946816in}{2.187653in}}%
\pgfpathlineto{\pgfqpoint{4.939389in}{2.176844in}}%
\pgfpathlineto{\pgfqpoint{4.931957in}{2.165999in}}%
\pgfpathlineto{\pgfqpoint{4.924520in}{2.155120in}}%
\pgfpathclose%
\pgfusepath{fill}%
\end{pgfscope}%
\begin{pgfscope}%
\pgfpathrectangle{\pgfqpoint{1.254980in}{0.150000in}}{\pgfqpoint{5.490039in}{5.490039in}}%
\pgfusepath{clip}%
\pgfsetbuttcap%
\pgfsetroundjoin%
\definecolor{currentfill}{rgb}{0.194100,0.399323,0.555565}%
\pgfsetfillcolor{currentfill}%
\pgfsetfillopacity{0.700000}%
\pgfsetlinewidth{0.000000pt}%
\definecolor{currentstroke}{rgb}{0.000000,0.000000,0.000000}%
\pgfsetstrokecolor{currentstroke}%
\pgfsetdash{}{0pt}%
\pgfpathmoveto{\pgfqpoint{2.752360in}{2.589563in}}%
\pgfpathlineto{\pgfqpoint{2.765789in}{2.571493in}}%
\pgfpathlineto{\pgfqpoint{2.779213in}{2.553603in}}%
\pgfpathlineto{\pgfqpoint{2.792631in}{2.535892in}}%
\pgfpathlineto{\pgfqpoint{2.806044in}{2.518359in}}%
\pgfpathlineto{\pgfqpoint{2.814409in}{2.518698in}}%
\pgfpathlineto{\pgfqpoint{2.822762in}{2.519225in}}%
\pgfpathlineto{\pgfqpoint{2.831101in}{2.519937in}}%
\pgfpathlineto{\pgfqpoint{2.839427in}{2.520832in}}%
\pgfpathlineto{\pgfqpoint{2.826051in}{2.538015in}}%
\pgfpathlineto{\pgfqpoint{2.812669in}{2.555376in}}%
\pgfpathlineto{\pgfqpoint{2.799282in}{2.572915in}}%
\pgfpathlineto{\pgfqpoint{2.785890in}{2.590633in}}%
\pgfpathlineto{\pgfqpoint{2.777528in}{2.590082in}}%
\pgfpathlineto{\pgfqpoint{2.769152in}{2.589718in}}%
\pgfpathlineto{\pgfqpoint{2.760763in}{2.589544in}}%
\pgfpathlineto{\pgfqpoint{2.752360in}{2.589563in}}%
\pgfpathclose%
\pgfusepath{fill}%
\end{pgfscope}%
\begin{pgfscope}%
\pgfpathrectangle{\pgfqpoint{1.254980in}{0.150000in}}{\pgfqpoint{5.490039in}{5.490039in}}%
\pgfusepath{clip}%
\pgfsetbuttcap%
\pgfsetroundjoin%
\definecolor{currentfill}{rgb}{0.146180,0.515413,0.556823}%
\pgfsetfillcolor{currentfill}%
\pgfsetfillopacity{0.700000}%
\pgfsetlinewidth{0.000000pt}%
\definecolor{currentstroke}{rgb}{0.000000,0.000000,0.000000}%
\pgfsetstrokecolor{currentstroke}%
\pgfsetdash{}{0pt}%
\pgfpathmoveto{\pgfqpoint{5.828499in}{2.829270in}}%
\pgfpathlineto{\pgfqpoint{5.842624in}{2.836385in}}%
\pgfpathlineto{\pgfqpoint{5.856765in}{2.843611in}}%
\pgfpathlineto{\pgfqpoint{5.870923in}{2.850949in}}%
\pgfpathlineto{\pgfqpoint{5.877986in}{2.859107in}}%
\pgfpathlineto{\pgfqpoint{5.885042in}{2.867204in}}%
\pgfpathlineto{\pgfqpoint{5.892092in}{2.875241in}}%
\pgfpathlineto{\pgfqpoint{5.899134in}{2.883218in}}%
\pgfpathlineto{\pgfqpoint{5.884989in}{2.875994in}}%
\pgfpathlineto{\pgfqpoint{5.870860in}{2.868882in}}%
\pgfpathlineto{\pgfqpoint{5.856747in}{2.861881in}}%
\pgfpathlineto{\pgfqpoint{5.849696in}{2.853813in}}%
\pgfpathlineto{\pgfqpoint{5.842637in}{2.845690in}}%
\pgfpathlineto{\pgfqpoint{5.835572in}{2.837509in}}%
\pgfpathlineto{\pgfqpoint{5.828499in}{2.829270in}}%
\pgfpathclose%
\pgfusepath{fill}%
\end{pgfscope}%
\begin{pgfscope}%
\pgfpathrectangle{\pgfqpoint{1.254980in}{0.150000in}}{\pgfqpoint{5.490039in}{5.490039in}}%
\pgfusepath{clip}%
\pgfsetbuttcap%
\pgfsetroundjoin%
\definecolor{currentfill}{rgb}{0.201239,0.383670,0.554294}%
\pgfsetfillcolor{currentfill}%
\pgfsetfillopacity{0.700000}%
\pgfsetlinewidth{0.000000pt}%
\definecolor{currentstroke}{rgb}{0.000000,0.000000,0.000000}%
\pgfsetstrokecolor{currentstroke}%
\pgfsetdash{}{0pt}%
\pgfpathmoveto{\pgfqpoint{5.376314in}{2.486118in}}%
\pgfpathlineto{\pgfqpoint{5.390191in}{2.491312in}}%
\pgfpathlineto{\pgfqpoint{5.404083in}{2.496620in}}%
\pgfpathlineto{\pgfqpoint{5.417989in}{2.502041in}}%
\pgfpathlineto{\pgfqpoint{5.431909in}{2.507574in}}%
\pgfpathlineto{\pgfqpoint{5.439177in}{2.517522in}}%
\pgfpathlineto{\pgfqpoint{5.446438in}{2.527409in}}%
\pgfpathlineto{\pgfqpoint{5.453693in}{2.537235in}}%
\pgfpathlineto{\pgfqpoint{5.460942in}{2.547001in}}%
\pgfpathlineto{\pgfqpoint{5.447029in}{2.541480in}}%
\pgfpathlineto{\pgfqpoint{5.433130in}{2.536072in}}%
\pgfpathlineto{\pgfqpoint{5.419246in}{2.530777in}}%
\pgfpathlineto{\pgfqpoint{5.405375in}{2.525594in}}%
\pgfpathlineto{\pgfqpoint{5.398119in}{2.515809in}}%
\pgfpathlineto{\pgfqpoint{5.390857in}{2.505969in}}%
\pgfpathlineto{\pgfqpoint{5.383588in}{2.496072in}}%
\pgfpathlineto{\pgfqpoint{5.376314in}{2.486118in}}%
\pgfpathclose%
\pgfusepath{fill}%
\end{pgfscope}%
\begin{pgfscope}%
\pgfpathrectangle{\pgfqpoint{1.254980in}{0.150000in}}{\pgfqpoint{5.490039in}{5.490039in}}%
\pgfusepath{clip}%
\pgfsetbuttcap%
\pgfsetroundjoin%
\definecolor{currentfill}{rgb}{0.277941,0.056324,0.381191}%
\pgfsetfillcolor{currentfill}%
\pgfsetfillopacity{0.700000}%
\pgfsetlinewidth{0.000000pt}%
\definecolor{currentstroke}{rgb}{0.000000,0.000000,0.000000}%
\pgfsetstrokecolor{currentstroke}%
\pgfsetdash{}{0pt}%
\pgfpathmoveto{\pgfqpoint{4.250961in}{1.809808in}}%
\pgfpathlineto{\pgfqpoint{4.264360in}{1.807467in}}%
\pgfpathlineto{\pgfqpoint{4.277767in}{1.805246in}}%
\pgfpathlineto{\pgfqpoint{4.291181in}{1.803144in}}%
\pgfpathlineto{\pgfqpoint{4.304603in}{1.801161in}}%
\pgfpathlineto{\pgfqpoint{4.312239in}{1.811165in}}%
\pgfpathlineto{\pgfqpoint{4.319870in}{1.821195in}}%
\pgfpathlineto{\pgfqpoint{4.327497in}{1.831249in}}%
\pgfpathlineto{\pgfqpoint{4.335118in}{1.841328in}}%
\pgfpathlineto{\pgfqpoint{4.321705in}{1.843097in}}%
\pgfpathlineto{\pgfqpoint{4.308300in}{1.844986in}}%
\pgfpathlineto{\pgfqpoint{4.294902in}{1.846994in}}%
\pgfpathlineto{\pgfqpoint{4.281512in}{1.849122in}}%
\pgfpathlineto{\pgfqpoint{4.273882in}{1.839251in}}%
\pgfpathlineto{\pgfqpoint{4.266247in}{1.829407in}}%
\pgfpathlineto{\pgfqpoint{4.258606in}{1.819592in}}%
\pgfpathlineto{\pgfqpoint{4.250961in}{1.809808in}}%
\pgfpathclose%
\pgfusepath{fill}%
\end{pgfscope}%
\begin{pgfscope}%
\pgfpathrectangle{\pgfqpoint{1.254980in}{0.150000in}}{\pgfqpoint{5.490039in}{5.490039in}}%
\pgfusepath{clip}%
\pgfsetbuttcap%
\pgfsetroundjoin%
\definecolor{currentfill}{rgb}{0.270595,0.214069,0.507052}%
\pgfsetfillcolor{currentfill}%
\pgfsetfillopacity{0.700000}%
\pgfsetlinewidth{0.000000pt}%
\definecolor{currentstroke}{rgb}{0.000000,0.000000,0.000000}%
\pgfsetstrokecolor{currentstroke}%
\pgfsetdash{}{0pt}%
\pgfpathmoveto{\pgfqpoint{3.126670in}{2.148330in}}%
\pgfpathlineto{\pgfqpoint{3.139992in}{2.134928in}}%
\pgfpathlineto{\pgfqpoint{3.153313in}{2.121679in}}%
\pgfpathlineto{\pgfqpoint{3.166631in}{2.108583in}}%
\pgfpathlineto{\pgfqpoint{3.179949in}{2.095639in}}%
\pgfpathlineto{\pgfqpoint{3.188083in}{2.098421in}}%
\pgfpathlineto{\pgfqpoint{3.196207in}{2.101359in}}%
\pgfpathlineto{\pgfqpoint{3.204321in}{2.104452in}}%
\pgfpathlineto{\pgfqpoint{3.212424in}{2.107695in}}%
\pgfpathlineto{\pgfqpoint{3.199136in}{2.120302in}}%
\pgfpathlineto{\pgfqpoint{3.185846in}{2.133062in}}%
\pgfpathlineto{\pgfqpoint{3.172555in}{2.145973in}}%
\pgfpathlineto{\pgfqpoint{3.159262in}{2.159039in}}%
\pgfpathlineto{\pgfqpoint{3.151130in}{2.156126in}}%
\pgfpathlineto{\pgfqpoint{3.142987in}{2.153368in}}%
\pgfpathlineto{\pgfqpoint{3.134834in}{2.150769in}}%
\pgfpathlineto{\pgfqpoint{3.126670in}{2.148330in}}%
\pgfpathclose%
\pgfusepath{fill}%
\end{pgfscope}%
\begin{pgfscope}%
\pgfpathrectangle{\pgfqpoint{1.254980in}{0.150000in}}{\pgfqpoint{5.490039in}{5.490039in}}%
\pgfusepath{clip}%
\pgfsetbuttcap%
\pgfsetroundjoin%
\definecolor{currentfill}{rgb}{0.253935,0.265254,0.529983}%
\pgfsetfillcolor{currentfill}%
\pgfsetfillopacity{0.700000}%
\pgfsetlinewidth{0.000000pt}%
\definecolor{currentstroke}{rgb}{0.000000,0.000000,0.000000}%
\pgfsetstrokecolor{currentstroke}%
\pgfsetdash{}{0pt}%
\pgfpathmoveto{\pgfqpoint{5.008903in}{2.210066in}}%
\pgfpathlineto{\pgfqpoint{5.022600in}{2.213262in}}%
\pgfpathlineto{\pgfqpoint{5.036308in}{2.216571in}}%
\pgfpathlineto{\pgfqpoint{5.050029in}{2.219994in}}%
\pgfpathlineto{\pgfqpoint{5.063763in}{2.223530in}}%
\pgfpathlineto{\pgfqpoint{5.071169in}{2.234419in}}%
\pgfpathlineto{\pgfqpoint{5.078570in}{2.245263in}}%
\pgfpathlineto{\pgfqpoint{5.085965in}{2.256062in}}%
\pgfpathlineto{\pgfqpoint{5.093355in}{2.266816in}}%
\pgfpathlineto{\pgfqpoint{5.079627in}{2.263209in}}%
\pgfpathlineto{\pgfqpoint{5.065911in}{2.259716in}}%
\pgfpathlineto{\pgfqpoint{5.052207in}{2.256337in}}%
\pgfpathlineto{\pgfqpoint{5.038516in}{2.253072in}}%
\pgfpathlineto{\pgfqpoint{5.031121in}{2.242382in}}%
\pgfpathlineto{\pgfqpoint{5.023720in}{2.231651in}}%
\pgfpathlineto{\pgfqpoint{5.016314in}{2.220879in}}%
\pgfpathlineto{\pgfqpoint{5.008903in}{2.210066in}}%
\pgfpathclose%
\pgfusepath{fill}%
\end{pgfscope}%
\begin{pgfscope}%
\pgfpathrectangle{\pgfqpoint{1.254980in}{0.150000in}}{\pgfqpoint{5.490039in}{5.490039in}}%
\pgfusepath{clip}%
\pgfsetbuttcap%
\pgfsetroundjoin%
\definecolor{currentfill}{rgb}{0.180629,0.429975,0.557282}%
\pgfsetfillcolor{currentfill}%
\pgfsetfillopacity{0.700000}%
\pgfsetlinewidth{0.000000pt}%
\definecolor{currentstroke}{rgb}{0.000000,0.000000,0.000000}%
\pgfsetstrokecolor{currentstroke}%
\pgfsetdash{}{0pt}%
\pgfpathmoveto{\pgfqpoint{2.698578in}{2.663668in}}%
\pgfpathlineto{\pgfqpoint{2.712033in}{2.644865in}}%
\pgfpathlineto{\pgfqpoint{2.725482in}{2.626248in}}%
\pgfpathlineto{\pgfqpoint{2.738924in}{2.607814in}}%
\pgfpathlineto{\pgfqpoint{2.752360in}{2.589563in}}%
\pgfpathlineto{\pgfqpoint{2.760763in}{2.589544in}}%
\pgfpathlineto{\pgfqpoint{2.769152in}{2.589718in}}%
\pgfpathlineto{\pgfqpoint{2.777528in}{2.590082in}}%
\pgfpathlineto{\pgfqpoint{2.785890in}{2.590633in}}%
\pgfpathlineto{\pgfqpoint{2.772492in}{2.608532in}}%
\pgfpathlineto{\pgfqpoint{2.759088in}{2.626614in}}%
\pgfpathlineto{\pgfqpoint{2.745677in}{2.644878in}}%
\pgfpathlineto{\pgfqpoint{2.732261in}{2.663326in}}%
\pgfpathlineto{\pgfqpoint{2.723861in}{2.663121in}}%
\pgfpathlineto{\pgfqpoint{2.715447in}{2.663107in}}%
\pgfpathlineto{\pgfqpoint{2.707020in}{2.663289in}}%
\pgfpathlineto{\pgfqpoint{2.698578in}{2.663668in}}%
\pgfpathclose%
\pgfusepath{fill}%
\end{pgfscope}%
\begin{pgfscope}%
\pgfpathrectangle{\pgfqpoint{1.254980in}{0.150000in}}{\pgfqpoint{5.490039in}{5.490039in}}%
\pgfusepath{clip}%
\pgfsetbuttcap%
\pgfsetroundjoin%
\definecolor{currentfill}{rgb}{0.273809,0.031497,0.358853}%
\pgfsetfillcolor{currentfill}%
\pgfsetfillopacity{0.700000}%
\pgfsetlinewidth{0.000000pt}%
\definecolor{currentstroke}{rgb}{0.000000,0.000000,0.000000}%
\pgfsetstrokecolor{currentstroke}%
\pgfsetdash{}{0pt}%
\pgfpathmoveto{\pgfqpoint{3.807127in}{1.775933in}}%
\pgfpathlineto{\pgfqpoint{3.820437in}{1.769617in}}%
\pgfpathlineto{\pgfqpoint{3.833751in}{1.763429in}}%
\pgfpathlineto{\pgfqpoint{3.847070in}{1.757368in}}%
\pgfpathlineto{\pgfqpoint{3.860392in}{1.751434in}}%
\pgfpathlineto{\pgfqpoint{3.868187in}{1.759071in}}%
\pgfpathlineto{\pgfqpoint{3.875976in}{1.766790in}}%
\pgfpathlineto{\pgfqpoint{3.883759in}{1.774587in}}%
\pgfpathlineto{\pgfqpoint{3.891535in}{1.782461in}}%
\pgfpathlineto{\pgfqpoint{3.878228in}{1.788118in}}%
\pgfpathlineto{\pgfqpoint{3.864925in}{1.793902in}}%
\pgfpathlineto{\pgfqpoint{3.851627in}{1.799812in}}%
\pgfpathlineto{\pgfqpoint{3.838333in}{1.805851in}}%
\pgfpathlineto{\pgfqpoint{3.830541in}{1.798248in}}%
\pgfpathlineto{\pgfqpoint{3.822743in}{1.790726in}}%
\pgfpathlineto{\pgfqpoint{3.814938in}{1.783287in}}%
\pgfpathlineto{\pgfqpoint{3.807127in}{1.775933in}}%
\pgfpathclose%
\pgfusepath{fill}%
\end{pgfscope}%
\begin{pgfscope}%
\pgfpathrectangle{\pgfqpoint{1.254980in}{0.150000in}}{\pgfqpoint{5.490039in}{5.490039in}}%
\pgfusepath{clip}%
\pgfsetbuttcap%
\pgfsetroundjoin%
\definecolor{currentfill}{rgb}{0.276194,0.190074,0.493001}%
\pgfsetfillcolor{currentfill}%
\pgfsetfillopacity{0.700000}%
\pgfsetlinewidth{0.000000pt}%
\definecolor{currentstroke}{rgb}{0.000000,0.000000,0.000000}%
\pgfsetstrokecolor{currentstroke}%
\pgfsetdash{}{0pt}%
\pgfpathmoveto{\pgfqpoint{3.179949in}{2.095639in}}%
\pgfpathlineto{\pgfqpoint{3.193265in}{2.082847in}}%
\pgfpathlineto{\pgfqpoint{3.206580in}{2.070206in}}%
\pgfpathlineto{\pgfqpoint{3.219893in}{2.057715in}}%
\pgfpathlineto{\pgfqpoint{3.233206in}{2.045373in}}%
\pgfpathlineto{\pgfqpoint{3.241312in}{2.048497in}}%
\pgfpathlineto{\pgfqpoint{3.249408in}{2.051773in}}%
\pgfpathlineto{\pgfqpoint{3.257493in}{2.055198in}}%
\pgfpathlineto{\pgfqpoint{3.265569in}{2.058770in}}%
\pgfpathlineto{\pgfqpoint{3.252284in}{2.070777in}}%
\pgfpathlineto{\pgfqpoint{3.238998in}{2.082933in}}%
\pgfpathlineto{\pgfqpoint{3.225712in}{2.095239in}}%
\pgfpathlineto{\pgfqpoint{3.212424in}{2.107695in}}%
\pgfpathlineto{\pgfqpoint{3.204321in}{2.104452in}}%
\pgfpathlineto{\pgfqpoint{3.196207in}{2.101359in}}%
\pgfpathlineto{\pgfqpoint{3.188083in}{2.098421in}}%
\pgfpathlineto{\pgfqpoint{3.179949in}{2.095639in}}%
\pgfpathclose%
\pgfusepath{fill}%
\end{pgfscope}%
\begin{pgfscope}%
\pgfpathrectangle{\pgfqpoint{1.254980in}{0.150000in}}{\pgfqpoint{5.490039in}{5.490039in}}%
\pgfusepath{clip}%
\pgfsetbuttcap%
\pgfsetroundjoin%
\definecolor{currentfill}{rgb}{0.188923,0.410910,0.556326}%
\pgfsetfillcolor{currentfill}%
\pgfsetfillopacity{0.700000}%
\pgfsetlinewidth{0.000000pt}%
\definecolor{currentstroke}{rgb}{0.000000,0.000000,0.000000}%
\pgfsetstrokecolor{currentstroke}%
\pgfsetdash{}{0pt}%
\pgfpathmoveto{\pgfqpoint{5.460942in}{2.547001in}}%
\pgfpathlineto{\pgfqpoint{5.474869in}{2.552635in}}%
\pgfpathlineto{\pgfqpoint{5.488811in}{2.558381in}}%
\pgfpathlineto{\pgfqpoint{5.502768in}{2.564240in}}%
\pgfpathlineto{\pgfqpoint{5.516739in}{2.570211in}}%
\pgfpathlineto{\pgfqpoint{5.523974in}{2.579894in}}%
\pgfpathlineto{\pgfqpoint{5.531203in}{2.589514in}}%
\pgfpathlineto{\pgfqpoint{5.538425in}{2.599072in}}%
\pgfpathlineto{\pgfqpoint{5.545641in}{2.608567in}}%
\pgfpathlineto{\pgfqpoint{5.531678in}{2.602625in}}%
\pgfpathlineto{\pgfqpoint{5.517729in}{2.596795in}}%
\pgfpathlineto{\pgfqpoint{5.503795in}{2.591078in}}%
\pgfpathlineto{\pgfqpoint{5.489875in}{2.585474in}}%
\pgfpathlineto{\pgfqpoint{5.482651in}{2.575943in}}%
\pgfpathlineto{\pgfqpoint{5.475421in}{2.566355in}}%
\pgfpathlineto{\pgfqpoint{5.468185in}{2.556708in}}%
\pgfpathlineto{\pgfqpoint{5.460942in}{2.547001in}}%
\pgfpathclose%
\pgfusepath{fill}%
\end{pgfscope}%
\begin{pgfscope}%
\pgfpathrectangle{\pgfqpoint{1.254980in}{0.150000in}}{\pgfqpoint{5.490039in}{5.490039in}}%
\pgfusepath{clip}%
\pgfsetbuttcap%
\pgfsetroundjoin%
\definecolor{currentfill}{rgb}{0.281924,0.089666,0.412415}%
\pgfsetfillcolor{currentfill}%
\pgfsetfillopacity{0.700000}%
\pgfsetlinewidth{0.000000pt}%
\definecolor{currentstroke}{rgb}{0.000000,0.000000,0.000000}%
\pgfsetstrokecolor{currentstroke}%
\pgfsetdash{}{0pt}%
\pgfpathmoveto{\pgfqpoint{3.478137in}{1.886376in}}%
\pgfpathlineto{\pgfqpoint{3.491429in}{1.876801in}}%
\pgfpathlineto{\pgfqpoint{3.504723in}{1.867364in}}%
\pgfpathlineto{\pgfqpoint{3.518018in}{1.858064in}}%
\pgfpathlineto{\pgfqpoint{3.531315in}{1.848901in}}%
\pgfpathlineto{\pgfqpoint{3.539260in}{1.854223in}}%
\pgfpathlineto{\pgfqpoint{3.547197in}{1.859665in}}%
\pgfpathlineto{\pgfqpoint{3.555126in}{1.865225in}}%
\pgfpathlineto{\pgfqpoint{3.563047in}{1.870901in}}%
\pgfpathlineto{\pgfqpoint{3.549771in}{1.879752in}}%
\pgfpathlineto{\pgfqpoint{3.536498in}{1.888739in}}%
\pgfpathlineto{\pgfqpoint{3.523226in}{1.897863in}}%
\pgfpathlineto{\pgfqpoint{3.509956in}{1.907125in}}%
\pgfpathlineto{\pgfqpoint{3.502014in}{1.901756in}}%
\pgfpathlineto{\pgfqpoint{3.494063in}{1.896506in}}%
\pgfpathlineto{\pgfqpoint{3.486104in}{1.891379in}}%
\pgfpathlineto{\pgfqpoint{3.478137in}{1.886376in}}%
\pgfpathclose%
\pgfusepath{fill}%
\end{pgfscope}%
\begin{pgfscope}%
\pgfpathrectangle{\pgfqpoint{1.254980in}{0.150000in}}{\pgfqpoint{5.490039in}{5.490039in}}%
\pgfusepath{clip}%
\pgfsetbuttcap%
\pgfsetroundjoin%
\definecolor{currentfill}{rgb}{0.276022,0.044167,0.370164}%
\pgfsetfillcolor{currentfill}%
\pgfsetfillopacity{0.700000}%
\pgfsetlinewidth{0.000000pt}%
\definecolor{currentstroke}{rgb}{0.000000,0.000000,0.000000}%
\pgfsetstrokecolor{currentstroke}%
\pgfsetdash{}{0pt}%
\pgfpathmoveto{\pgfqpoint{4.166766in}{1.782487in}}%
\pgfpathlineto{\pgfqpoint{4.180146in}{1.779436in}}%
\pgfpathlineto{\pgfqpoint{4.193533in}{1.776506in}}%
\pgfpathlineto{\pgfqpoint{4.206928in}{1.773696in}}%
\pgfpathlineto{\pgfqpoint{4.220329in}{1.771007in}}%
\pgfpathlineto{\pgfqpoint{4.227995in}{1.780653in}}%
\pgfpathlineto{\pgfqpoint{4.235655in}{1.790336in}}%
\pgfpathlineto{\pgfqpoint{4.243311in}{1.800055in}}%
\pgfpathlineto{\pgfqpoint{4.250961in}{1.809808in}}%
\pgfpathlineto{\pgfqpoint{4.237569in}{1.812268in}}%
\pgfpathlineto{\pgfqpoint{4.224185in}{1.814849in}}%
\pgfpathlineto{\pgfqpoint{4.210807in}{1.817550in}}%
\pgfpathlineto{\pgfqpoint{4.197437in}{1.820372in}}%
\pgfpathlineto{\pgfqpoint{4.189777in}{1.810843in}}%
\pgfpathlineto{\pgfqpoint{4.182112in}{1.801351in}}%
\pgfpathlineto{\pgfqpoint{4.174441in}{1.791898in}}%
\pgfpathlineto{\pgfqpoint{4.166766in}{1.782487in}}%
\pgfpathclose%
\pgfusepath{fill}%
\end{pgfscope}%
\begin{pgfscope}%
\pgfpathrectangle{\pgfqpoint{1.254980in}{0.150000in}}{\pgfqpoint{5.490039in}{5.490039in}}%
\pgfusepath{clip}%
\pgfsetbuttcap%
\pgfsetroundjoin%
\definecolor{currentfill}{rgb}{0.272594,0.025563,0.353093}%
\pgfsetfillcolor{currentfill}%
\pgfsetfillopacity{0.700000}%
\pgfsetlinewidth{0.000000pt}%
\definecolor{currentstroke}{rgb}{0.000000,0.000000,0.000000}%
\pgfsetstrokecolor{currentstroke}%
\pgfsetdash{}{0pt}%
\pgfpathmoveto{\pgfqpoint{3.944814in}{1.761094in}}%
\pgfpathlineto{\pgfqpoint{3.958146in}{1.756065in}}%
\pgfpathlineto{\pgfqpoint{3.971483in}{1.751161in}}%
\pgfpathlineto{\pgfqpoint{3.984826in}{1.746382in}}%
\pgfpathlineto{\pgfqpoint{3.998175in}{1.741726in}}%
\pgfpathlineto{\pgfqpoint{4.005917in}{1.750208in}}%
\pgfpathlineto{\pgfqpoint{4.013654in}{1.758754in}}%
\pgfpathlineto{\pgfqpoint{4.021385in}{1.767363in}}%
\pgfpathlineto{\pgfqpoint{4.029111in}{1.776032in}}%
\pgfpathlineto{\pgfqpoint{4.015776in}{1.780427in}}%
\pgfpathlineto{\pgfqpoint{4.002446in}{1.784946in}}%
\pgfpathlineto{\pgfqpoint{3.989122in}{1.789589in}}%
\pgfpathlineto{\pgfqpoint{3.975803in}{1.794357in}}%
\pgfpathlineto{\pgfqpoint{3.968064in}{1.785943in}}%
\pgfpathlineto{\pgfqpoint{3.960320in}{1.777593in}}%
\pgfpathlineto{\pgfqpoint{3.952570in}{1.769309in}}%
\pgfpathlineto{\pgfqpoint{3.944814in}{1.761094in}}%
\pgfpathclose%
\pgfusepath{fill}%
\end{pgfscope}%
\begin{pgfscope}%
\pgfpathrectangle{\pgfqpoint{1.254980in}{0.150000in}}{\pgfqpoint{5.490039in}{5.490039in}}%
\pgfusepath{clip}%
\pgfsetbuttcap%
\pgfsetroundjoin%
\definecolor{currentfill}{rgb}{0.277018,0.050344,0.375715}%
\pgfsetfillcolor{currentfill}%
\pgfsetfillopacity{0.700000}%
\pgfsetlinewidth{0.000000pt}%
\definecolor{currentstroke}{rgb}{0.000000,0.000000,0.000000}%
\pgfsetstrokecolor{currentstroke}%
\pgfsetdash{}{0pt}%
\pgfpathmoveto{\pgfqpoint{3.669336in}{1.804938in}}%
\pgfpathlineto{\pgfqpoint{3.682635in}{1.797291in}}%
\pgfpathlineto{\pgfqpoint{3.695936in}{1.789775in}}%
\pgfpathlineto{\pgfqpoint{3.709241in}{1.782390in}}%
\pgfpathlineto{\pgfqpoint{3.722549in}{1.775135in}}%
\pgfpathlineto{\pgfqpoint{3.730404in}{1.781827in}}%
\pgfpathlineto{\pgfqpoint{3.738253in}{1.788616in}}%
\pgfpathlineto{\pgfqpoint{3.746094in}{1.795502in}}%
\pgfpathlineto{\pgfqpoint{3.753928in}{1.802482in}}%
\pgfpathlineto{\pgfqpoint{3.740638in}{1.809442in}}%
\pgfpathlineto{\pgfqpoint{3.727351in}{1.816533in}}%
\pgfpathlineto{\pgfqpoint{3.714068in}{1.823755in}}%
\pgfpathlineto{\pgfqpoint{3.700787in}{1.831107in}}%
\pgfpathlineto{\pgfqpoint{3.692935in}{1.824416in}}%
\pgfpathlineto{\pgfqpoint{3.685076in}{1.817823in}}%
\pgfpathlineto{\pgfqpoint{3.677210in}{1.811329in}}%
\pgfpathlineto{\pgfqpoint{3.669336in}{1.804938in}}%
\pgfpathclose%
\pgfusepath{fill}%
\end{pgfscope}%
\begin{pgfscope}%
\pgfpathrectangle{\pgfqpoint{1.254980in}{0.150000in}}{\pgfqpoint{5.490039in}{5.490039in}}%
\pgfusepath{clip}%
\pgfsetbuttcap%
\pgfsetroundjoin%
\definecolor{currentfill}{rgb}{0.243113,0.292092,0.538516}%
\pgfsetfillcolor{currentfill}%
\pgfsetfillopacity{0.700000}%
\pgfsetlinewidth{0.000000pt}%
\definecolor{currentstroke}{rgb}{0.000000,0.000000,0.000000}%
\pgfsetstrokecolor{currentstroke}%
\pgfsetdash{}{0pt}%
\pgfpathmoveto{\pgfqpoint{5.093355in}{2.266816in}}%
\pgfpathlineto{\pgfqpoint{5.107096in}{2.270536in}}%
\pgfpathlineto{\pgfqpoint{5.120849in}{2.274370in}}%
\pgfpathlineto{\pgfqpoint{5.134616in}{2.278317in}}%
\pgfpathlineto{\pgfqpoint{5.148395in}{2.282378in}}%
\pgfpathlineto{\pgfqpoint{5.155775in}{2.293146in}}%
\pgfpathlineto{\pgfqpoint{5.163148in}{2.303864in}}%
\pgfpathlineto{\pgfqpoint{5.170517in}{2.314531in}}%
\pgfpathlineto{\pgfqpoint{5.177879in}{2.325148in}}%
\pgfpathlineto{\pgfqpoint{5.164105in}{2.321034in}}%
\pgfpathlineto{\pgfqpoint{5.150344in}{2.317033in}}%
\pgfpathlineto{\pgfqpoint{5.136596in}{2.313145in}}%
\pgfpathlineto{\pgfqpoint{5.122860in}{2.309371in}}%
\pgfpathlineto{\pgfqpoint{5.115492in}{2.298802in}}%
\pgfpathlineto{\pgfqpoint{5.108118in}{2.288186in}}%
\pgfpathlineto{\pgfqpoint{5.100739in}{2.277524in}}%
\pgfpathlineto{\pgfqpoint{5.093355in}{2.266816in}}%
\pgfpathclose%
\pgfusepath{fill}%
\end{pgfscope}%
\begin{pgfscope}%
\pgfpathrectangle{\pgfqpoint{1.254980in}{0.150000in}}{\pgfqpoint{5.490039in}{5.490039in}}%
\pgfusepath{clip}%
\pgfsetbuttcap%
\pgfsetroundjoin%
\definecolor{currentfill}{rgb}{0.119483,0.614817,0.537692}%
\pgfsetfillcolor{currentfill}%
\pgfsetfillopacity{0.700000}%
\pgfsetlinewidth{0.000000pt}%
\definecolor{currentstroke}{rgb}{0.000000,0.000000,0.000000}%
\pgfsetstrokecolor{currentstroke}%
\pgfsetdash{}{0pt}%
\pgfpathmoveto{\pgfqpoint{2.407914in}{3.164605in}}%
\pgfpathlineto{\pgfqpoint{2.421535in}{3.141348in}}%
\pgfpathlineto{\pgfqpoint{2.435147in}{3.118309in}}%
\pgfpathlineto{\pgfqpoint{2.448747in}{3.095484in}}%
\pgfpathlineto{\pgfqpoint{2.462337in}{3.072874in}}%
\pgfpathlineto{\pgfqpoint{2.470927in}{3.071519in}}%
\pgfpathlineto{\pgfqpoint{2.479501in}{3.070374in}}%
\pgfpathlineto{\pgfqpoint{2.488059in}{3.069437in}}%
\pgfpathlineto{\pgfqpoint{2.496602in}{3.068704in}}%
\pgfpathlineto{\pgfqpoint{2.483055in}{3.090964in}}%
\pgfpathlineto{\pgfqpoint{2.469498in}{3.113437in}}%
\pgfpathlineto{\pgfqpoint{2.455931in}{3.136124in}}%
\pgfpathlineto{\pgfqpoint{2.442353in}{3.159028in}}%
\pgfpathlineto{\pgfqpoint{2.433768in}{3.160105in}}%
\pgfpathlineto{\pgfqpoint{2.425166in}{3.161392in}}%
\pgfpathlineto{\pgfqpoint{2.416548in}{3.162891in}}%
\pgfpathlineto{\pgfqpoint{2.407914in}{3.164605in}}%
\pgfpathclose%
\pgfusepath{fill}%
\end{pgfscope}%
\begin{pgfscope}%
\pgfpathrectangle{\pgfqpoint{1.254980in}{0.150000in}}{\pgfqpoint{5.490039in}{5.490039in}}%
\pgfusepath{clip}%
\pgfsetbuttcap%
\pgfsetroundjoin%
\definecolor{currentfill}{rgb}{0.168126,0.459988,0.558082}%
\pgfsetfillcolor{currentfill}%
\pgfsetfillopacity{0.700000}%
\pgfsetlinewidth{0.000000pt}%
\definecolor{currentstroke}{rgb}{0.000000,0.000000,0.000000}%
\pgfsetstrokecolor{currentstroke}%
\pgfsetdash{}{0pt}%
\pgfpathmoveto{\pgfqpoint{2.644688in}{2.740748in}}%
\pgfpathlineto{\pgfqpoint{2.658171in}{2.721195in}}%
\pgfpathlineto{\pgfqpoint{2.671647in}{2.701831in}}%
\pgfpathlineto{\pgfqpoint{2.685116in}{2.682656in}}%
\pgfpathlineto{\pgfqpoint{2.698578in}{2.663668in}}%
\pgfpathlineto{\pgfqpoint{2.707020in}{2.663289in}}%
\pgfpathlineto{\pgfqpoint{2.715447in}{2.663107in}}%
\pgfpathlineto{\pgfqpoint{2.723861in}{2.663121in}}%
\pgfpathlineto{\pgfqpoint{2.732261in}{2.663326in}}%
\pgfpathlineto{\pgfqpoint{2.718838in}{2.681959in}}%
\pgfpathlineto{\pgfqpoint{2.705408in}{2.700779in}}%
\pgfpathlineto{\pgfqpoint{2.691972in}{2.719786in}}%
\pgfpathlineto{\pgfqpoint{2.678528in}{2.738983in}}%
\pgfpathlineto{\pgfqpoint{2.670090in}{2.739126in}}%
\pgfpathlineto{\pgfqpoint{2.661637in}{2.739467in}}%
\pgfpathlineto{\pgfqpoint{2.653170in}{2.740006in}}%
\pgfpathlineto{\pgfqpoint{2.644688in}{2.740748in}}%
\pgfpathclose%
\pgfusepath{fill}%
\end{pgfscope}%
\begin{pgfscope}%
\pgfpathrectangle{\pgfqpoint{1.254980in}{0.150000in}}{\pgfqpoint{5.490039in}{5.490039in}}%
\pgfusepath{clip}%
\pgfsetbuttcap%
\pgfsetroundjoin%
\definecolor{currentfill}{rgb}{0.279574,0.170599,0.479997}%
\pgfsetfillcolor{currentfill}%
\pgfsetfillopacity{0.700000}%
\pgfsetlinewidth{0.000000pt}%
\definecolor{currentstroke}{rgb}{0.000000,0.000000,0.000000}%
\pgfsetstrokecolor{currentstroke}%
\pgfsetdash{}{0pt}%
\pgfpathmoveto{\pgfqpoint{3.233206in}{2.045373in}}%
\pgfpathlineto{\pgfqpoint{3.246518in}{2.033180in}}%
\pgfpathlineto{\pgfqpoint{3.259829in}{2.021135in}}%
\pgfpathlineto{\pgfqpoint{3.273140in}{2.009237in}}%
\pgfpathlineto{\pgfqpoint{3.286450in}{1.997485in}}%
\pgfpathlineto{\pgfqpoint{3.294528in}{2.000950in}}%
\pgfpathlineto{\pgfqpoint{3.302597in}{2.004562in}}%
\pgfpathlineto{\pgfqpoint{3.310656in}{2.008319in}}%
\pgfpathlineto{\pgfqpoint{3.318705in}{2.012219in}}%
\pgfpathlineto{\pgfqpoint{3.305421in}{2.023637in}}%
\pgfpathlineto{\pgfqpoint{3.292138in}{2.035201in}}%
\pgfpathlineto{\pgfqpoint{3.278854in}{2.046912in}}%
\pgfpathlineto{\pgfqpoint{3.265569in}{2.058770in}}%
\pgfpathlineto{\pgfqpoint{3.257493in}{2.055198in}}%
\pgfpathlineto{\pgfqpoint{3.249408in}{2.051773in}}%
\pgfpathlineto{\pgfqpoint{3.241312in}{2.048497in}}%
\pgfpathlineto{\pgfqpoint{3.233206in}{2.045373in}}%
\pgfpathclose%
\pgfusepath{fill}%
\end{pgfscope}%
\begin{pgfscope}%
\pgfpathrectangle{\pgfqpoint{1.254980in}{0.150000in}}{\pgfqpoint{5.490039in}{5.490039in}}%
\pgfusepath{clip}%
\pgfsetbuttcap%
\pgfsetroundjoin%
\definecolor{currentfill}{rgb}{0.177423,0.437527,0.557565}%
\pgfsetfillcolor{currentfill}%
\pgfsetfillopacity{0.700000}%
\pgfsetlinewidth{0.000000pt}%
\definecolor{currentstroke}{rgb}{0.000000,0.000000,0.000000}%
\pgfsetstrokecolor{currentstroke}%
\pgfsetdash{}{0pt}%
\pgfpathmoveto{\pgfqpoint{5.545641in}{2.608567in}}%
\pgfpathlineto{\pgfqpoint{5.559620in}{2.614621in}}%
\pgfpathlineto{\pgfqpoint{5.573613in}{2.620788in}}%
\pgfpathlineto{\pgfqpoint{5.587621in}{2.627067in}}%
\pgfpathlineto{\pgfqpoint{5.601645in}{2.633459in}}%
\pgfpathlineto{\pgfqpoint{5.608846in}{2.642853in}}%
\pgfpathlineto{\pgfqpoint{5.616040in}{2.652181in}}%
\pgfpathlineto{\pgfqpoint{5.623228in}{2.661445in}}%
\pgfpathlineto{\pgfqpoint{5.630409in}{2.670645in}}%
\pgfpathlineto{\pgfqpoint{5.616394in}{2.664299in}}%
\pgfpathlineto{\pgfqpoint{5.602394in}{2.658066in}}%
\pgfpathlineto{\pgfqpoint{5.588410in}{2.651945in}}%
\pgfpathlineto{\pgfqpoint{5.574440in}{2.645937in}}%
\pgfpathlineto{\pgfqpoint{5.567250in}{2.636684in}}%
\pgfpathlineto{\pgfqpoint{5.560053in}{2.627372in}}%
\pgfpathlineto{\pgfqpoint{5.552850in}{2.618000in}}%
\pgfpathlineto{\pgfqpoint{5.545641in}{2.608567in}}%
\pgfpathclose%
\pgfusepath{fill}%
\end{pgfscope}%
\begin{pgfscope}%
\pgfpathrectangle{\pgfqpoint{1.254980in}{0.150000in}}{\pgfqpoint{5.490039in}{5.490039in}}%
\pgfusepath{clip}%
\pgfsetbuttcap%
\pgfsetroundjoin%
\definecolor{currentfill}{rgb}{0.229739,0.322361,0.545706}%
\pgfsetfillcolor{currentfill}%
\pgfsetfillopacity{0.700000}%
\pgfsetlinewidth{0.000000pt}%
\definecolor{currentstroke}{rgb}{0.000000,0.000000,0.000000}%
\pgfsetstrokecolor{currentstroke}%
\pgfsetdash{}{0pt}%
\pgfpathmoveto{\pgfqpoint{5.177879in}{2.325148in}}%
\pgfpathlineto{\pgfqpoint{5.191666in}{2.329376in}}%
\pgfpathlineto{\pgfqpoint{5.205467in}{2.333717in}}%
\pgfpathlineto{\pgfqpoint{5.219280in}{2.338172in}}%
\pgfpathlineto{\pgfqpoint{5.233107in}{2.342739in}}%
\pgfpathlineto{\pgfqpoint{5.240459in}{2.353350in}}%
\pgfpathlineto{\pgfqpoint{5.247804in}{2.363905in}}%
\pgfpathlineto{\pgfqpoint{5.255144in}{2.374406in}}%
\pgfpathlineto{\pgfqpoint{5.262479in}{2.384853in}}%
\pgfpathlineto{\pgfqpoint{5.248657in}{2.380248in}}%
\pgfpathlineto{\pgfqpoint{5.234849in}{2.375756in}}%
\pgfpathlineto{\pgfqpoint{5.221055in}{2.371378in}}%
\pgfpathlineto{\pgfqpoint{5.207273in}{2.367112in}}%
\pgfpathlineto{\pgfqpoint{5.199933in}{2.356697in}}%
\pgfpathlineto{\pgfqpoint{5.192587in}{2.346232in}}%
\pgfpathlineto{\pgfqpoint{5.185236in}{2.335715in}}%
\pgfpathlineto{\pgfqpoint{5.177879in}{2.325148in}}%
\pgfpathclose%
\pgfusepath{fill}%
\end{pgfscope}%
\begin{pgfscope}%
\pgfpathrectangle{\pgfqpoint{1.254980in}{0.150000in}}{\pgfqpoint{5.490039in}{5.490039in}}%
\pgfusepath{clip}%
\pgfsetbuttcap%
\pgfsetroundjoin%
\definecolor{currentfill}{rgb}{0.273809,0.031497,0.358853}%
\pgfsetfillcolor{currentfill}%
\pgfsetfillopacity{0.700000}%
\pgfsetlinewidth{0.000000pt}%
\definecolor{currentstroke}{rgb}{0.000000,0.000000,0.000000}%
\pgfsetstrokecolor{currentstroke}%
\pgfsetdash{}{0pt}%
\pgfpathmoveto{\pgfqpoint{4.082511in}{1.759683in}}%
\pgfpathlineto{\pgfqpoint{4.095877in}{1.755902in}}%
\pgfpathlineto{\pgfqpoint{4.109248in}{1.752242in}}%
\pgfpathlineto{\pgfqpoint{4.122626in}{1.748705in}}%
\pgfpathlineto{\pgfqpoint{4.136011in}{1.745289in}}%
\pgfpathlineto{\pgfqpoint{4.143708in}{1.754517in}}%
\pgfpathlineto{\pgfqpoint{4.151399in}{1.763794in}}%
\pgfpathlineto{\pgfqpoint{4.159085in}{1.773118in}}%
\pgfpathlineto{\pgfqpoint{4.166766in}{1.782487in}}%
\pgfpathlineto{\pgfqpoint{4.153392in}{1.785658in}}%
\pgfpathlineto{\pgfqpoint{4.140025in}{1.788951in}}%
\pgfpathlineto{\pgfqpoint{4.126665in}{1.792366in}}%
\pgfpathlineto{\pgfqpoint{4.113311in}{1.795902in}}%
\pgfpathlineto{\pgfqpoint{4.105619in}{1.786772in}}%
\pgfpathlineto{\pgfqpoint{4.097922in}{1.777691in}}%
\pgfpathlineto{\pgfqpoint{4.090219in}{1.768661in}}%
\pgfpathlineto{\pgfqpoint{4.082511in}{1.759683in}}%
\pgfpathclose%
\pgfusepath{fill}%
\end{pgfscope}%
\begin{pgfscope}%
\pgfpathrectangle{\pgfqpoint{1.254980in}{0.150000in}}{\pgfqpoint{5.490039in}{5.490039in}}%
\pgfusepath{clip}%
\pgfsetbuttcap%
\pgfsetroundjoin%
\definecolor{currentfill}{rgb}{0.156270,0.489624,0.557936}%
\pgfsetfillcolor{currentfill}%
\pgfsetfillopacity{0.700000}%
\pgfsetlinewidth{0.000000pt}%
\definecolor{currentstroke}{rgb}{0.000000,0.000000,0.000000}%
\pgfsetstrokecolor{currentstroke}%
\pgfsetdash{}{0pt}%
\pgfpathmoveto{\pgfqpoint{2.590678in}{2.820882in}}%
\pgfpathlineto{\pgfqpoint{2.604193in}{2.800558in}}%
\pgfpathlineto{\pgfqpoint{2.617699in}{2.780428in}}%
\pgfpathlineto{\pgfqpoint{2.631197in}{2.760492in}}%
\pgfpathlineto{\pgfqpoint{2.644688in}{2.740748in}}%
\pgfpathlineto{\pgfqpoint{2.653170in}{2.740006in}}%
\pgfpathlineto{\pgfqpoint{2.661637in}{2.739467in}}%
\pgfpathlineto{\pgfqpoint{2.670090in}{2.739126in}}%
\pgfpathlineto{\pgfqpoint{2.678528in}{2.738983in}}%
\pgfpathlineto{\pgfqpoint{2.665078in}{2.758369in}}%
\pgfpathlineto{\pgfqpoint{2.651620in}{2.777947in}}%
\pgfpathlineto{\pgfqpoint{2.638154in}{2.797717in}}%
\pgfpathlineto{\pgfqpoint{2.624681in}{2.817682in}}%
\pgfpathlineto{\pgfqpoint{2.616203in}{2.818177in}}%
\pgfpathlineto{\pgfqpoint{2.607710in}{2.818874in}}%
\pgfpathlineto{\pgfqpoint{2.599202in}{2.819774in}}%
\pgfpathlineto{\pgfqpoint{2.590678in}{2.820882in}}%
\pgfpathclose%
\pgfusepath{fill}%
\end{pgfscope}%
\begin{pgfscope}%
\pgfpathrectangle{\pgfqpoint{1.254980in}{0.150000in}}{\pgfqpoint{5.490039in}{5.490039in}}%
\pgfusepath{clip}%
\pgfsetbuttcap%
\pgfsetroundjoin%
\definecolor{currentfill}{rgb}{0.283187,0.125848,0.444960}%
\pgfsetfillcolor{currentfill}%
\pgfsetfillopacity{0.700000}%
\pgfsetlinewidth{0.000000pt}%
\definecolor{currentstroke}{rgb}{0.000000,0.000000,0.000000}%
\pgfsetstrokecolor{currentstroke}%
\pgfsetdash{}{0pt}%
\pgfpathmoveto{\pgfqpoint{4.557347in}{1.915136in}}%
\pgfpathlineto{\pgfqpoint{4.570858in}{1.915274in}}%
\pgfpathlineto{\pgfqpoint{4.584379in}{1.915528in}}%
\pgfpathlineto{\pgfqpoint{4.597910in}{1.915899in}}%
\pgfpathlineto{\pgfqpoint{4.611450in}{1.916385in}}%
\pgfpathlineto{\pgfqpoint{4.619001in}{1.927295in}}%
\pgfpathlineto{\pgfqpoint{4.626547in}{1.938199in}}%
\pgfpathlineto{\pgfqpoint{4.634088in}{1.949096in}}%
\pgfpathlineto{\pgfqpoint{4.641624in}{1.959984in}}%
\pgfpathlineto{\pgfqpoint{4.628089in}{1.959332in}}%
\pgfpathlineto{\pgfqpoint{4.614565in}{1.958795in}}%
\pgfpathlineto{\pgfqpoint{4.601050in}{1.958375in}}%
\pgfpathlineto{\pgfqpoint{4.587544in}{1.958071in}}%
\pgfpathlineto{\pgfqpoint{4.580002in}{1.947343in}}%
\pgfpathlineto{\pgfqpoint{4.572455in}{1.936610in}}%
\pgfpathlineto{\pgfqpoint{4.564903in}{1.925874in}}%
\pgfpathlineto{\pgfqpoint{4.557347in}{1.915136in}}%
\pgfpathclose%
\pgfusepath{fill}%
\end{pgfscope}%
\begin{pgfscope}%
\pgfpathrectangle{\pgfqpoint{1.254980in}{0.150000in}}{\pgfqpoint{5.490039in}{5.490039in}}%
\pgfusepath{clip}%
\pgfsetbuttcap%
\pgfsetroundjoin%
\definecolor{currentfill}{rgb}{0.281887,0.150881,0.465405}%
\pgfsetfillcolor{currentfill}%
\pgfsetfillopacity{0.700000}%
\pgfsetlinewidth{0.000000pt}%
\definecolor{currentstroke}{rgb}{0.000000,0.000000,0.000000}%
\pgfsetstrokecolor{currentstroke}%
\pgfsetdash{}{0pt}%
\pgfpathmoveto{\pgfqpoint{4.641624in}{1.959984in}}%
\pgfpathlineto{\pgfqpoint{4.655169in}{1.960752in}}%
\pgfpathlineto{\pgfqpoint{4.668723in}{1.961636in}}%
\pgfpathlineto{\pgfqpoint{4.682288in}{1.962635in}}%
\pgfpathlineto{\pgfqpoint{4.695863in}{1.963750in}}%
\pgfpathlineto{\pgfqpoint{4.703389in}{1.974784in}}%
\pgfpathlineto{\pgfqpoint{4.710910in}{1.985804in}}%
\pgfpathlineto{\pgfqpoint{4.718427in}{1.996808in}}%
\pgfpathlineto{\pgfqpoint{4.725939in}{2.007796in}}%
\pgfpathlineto{\pgfqpoint{4.712369in}{2.006531in}}%
\pgfpathlineto{\pgfqpoint{4.698809in}{2.005382in}}%
\pgfpathlineto{\pgfqpoint{4.685260in}{2.004348in}}%
\pgfpathlineto{\pgfqpoint{4.671721in}{2.003429in}}%
\pgfpathlineto{\pgfqpoint{4.664204in}{1.992586in}}%
\pgfpathlineto{\pgfqpoint{4.656682in}{1.981730in}}%
\pgfpathlineto{\pgfqpoint{4.649156in}{1.970862in}}%
\pgfpathlineto{\pgfqpoint{4.641624in}{1.959984in}}%
\pgfpathclose%
\pgfusepath{fill}%
\end{pgfscope}%
\begin{pgfscope}%
\pgfpathrectangle{\pgfqpoint{1.254980in}{0.150000in}}{\pgfqpoint{5.490039in}{5.490039in}}%
\pgfusepath{clip}%
\pgfsetbuttcap%
\pgfsetroundjoin%
\definecolor{currentfill}{rgb}{0.282910,0.105393,0.426902}%
\pgfsetfillcolor{currentfill}%
\pgfsetfillopacity{0.700000}%
\pgfsetlinewidth{0.000000pt}%
\definecolor{currentstroke}{rgb}{0.000000,0.000000,0.000000}%
\pgfsetstrokecolor{currentstroke}%
\pgfsetdash{}{0pt}%
\pgfpathmoveto{\pgfqpoint{4.473094in}{1.873525in}}%
\pgfpathlineto{\pgfqpoint{4.486575in}{1.873015in}}%
\pgfpathlineto{\pgfqpoint{4.500065in}{1.872621in}}%
\pgfpathlineto{\pgfqpoint{4.513565in}{1.872345in}}%
\pgfpathlineto{\pgfqpoint{4.527073in}{1.872185in}}%
\pgfpathlineto{\pgfqpoint{4.534649in}{1.882920in}}%
\pgfpathlineto{\pgfqpoint{4.542219in}{1.893657in}}%
\pgfpathlineto{\pgfqpoint{4.549785in}{1.904396in}}%
\pgfpathlineto{\pgfqpoint{4.557347in}{1.915136in}}%
\pgfpathlineto{\pgfqpoint{4.543845in}{1.915114in}}%
\pgfpathlineto{\pgfqpoint{4.530352in}{1.915209in}}%
\pgfpathlineto{\pgfqpoint{4.516868in}{1.915421in}}%
\pgfpathlineto{\pgfqpoint{4.503394in}{1.915750in}}%
\pgfpathlineto{\pgfqpoint{4.495826in}{1.905186in}}%
\pgfpathlineto{\pgfqpoint{4.488253in}{1.894626in}}%
\pgfpathlineto{\pgfqpoint{4.480676in}{1.884072in}}%
\pgfpathlineto{\pgfqpoint{4.473094in}{1.873525in}}%
\pgfpathclose%
\pgfusepath{fill}%
\end{pgfscope}%
\begin{pgfscope}%
\pgfpathrectangle{\pgfqpoint{1.254980in}{0.150000in}}{\pgfqpoint{5.490039in}{5.490039in}}%
\pgfusepath{clip}%
\pgfsetbuttcap%
\pgfsetroundjoin%
\definecolor{currentfill}{rgb}{0.282290,0.145912,0.461510}%
\pgfsetfillcolor{currentfill}%
\pgfsetfillopacity{0.700000}%
\pgfsetlinewidth{0.000000pt}%
\definecolor{currentstroke}{rgb}{0.000000,0.000000,0.000000}%
\pgfsetstrokecolor{currentstroke}%
\pgfsetdash{}{0pt}%
\pgfpathmoveto{\pgfqpoint{3.286450in}{1.997485in}}%
\pgfpathlineto{\pgfqpoint{3.299760in}{1.985880in}}%
\pgfpathlineto{\pgfqpoint{3.313070in}{1.974420in}}%
\pgfpathlineto{\pgfqpoint{3.326379in}{1.963104in}}%
\pgfpathlineto{\pgfqpoint{3.339689in}{1.951932in}}%
\pgfpathlineto{\pgfqpoint{3.347740in}{1.955735in}}%
\pgfpathlineto{\pgfqpoint{3.355783in}{1.959683in}}%
\pgfpathlineto{\pgfqpoint{3.363816in}{1.963771in}}%
\pgfpathlineto{\pgfqpoint{3.371839in}{1.967997in}}%
\pgfpathlineto{\pgfqpoint{3.358555in}{1.978836in}}%
\pgfpathlineto{\pgfqpoint{3.345272in}{1.989819in}}%
\pgfpathlineto{\pgfqpoint{3.331988in}{2.000947in}}%
\pgfpathlineto{\pgfqpoint{3.318705in}{2.012219in}}%
\pgfpathlineto{\pgfqpoint{3.310656in}{2.008319in}}%
\pgfpathlineto{\pgfqpoint{3.302597in}{2.004562in}}%
\pgfpathlineto{\pgfqpoint{3.294528in}{2.000950in}}%
\pgfpathlineto{\pgfqpoint{3.286450in}{1.997485in}}%
\pgfpathclose%
\pgfusepath{fill}%
\end{pgfscope}%
\begin{pgfscope}%
\pgfpathrectangle{\pgfqpoint{1.254980in}{0.150000in}}{\pgfqpoint{5.490039in}{5.490039in}}%
\pgfusepath{clip}%
\pgfsetbuttcap%
\pgfsetroundjoin%
\definecolor{currentfill}{rgb}{0.280267,0.073417,0.397163}%
\pgfsetfillcolor{currentfill}%
\pgfsetfillopacity{0.700000}%
\pgfsetlinewidth{0.000000pt}%
\definecolor{currentstroke}{rgb}{0.000000,0.000000,0.000000}%
\pgfsetstrokecolor{currentstroke}%
\pgfsetdash{}{0pt}%
\pgfpathmoveto{\pgfqpoint{3.531315in}{1.848901in}}%
\pgfpathlineto{\pgfqpoint{3.544614in}{1.839873in}}%
\pgfpathlineto{\pgfqpoint{3.557915in}{1.830981in}}%
\pgfpathlineto{\pgfqpoint{3.571218in}{1.822224in}}%
\pgfpathlineto{\pgfqpoint{3.584523in}{1.813601in}}%
\pgfpathlineto{\pgfqpoint{3.592447in}{1.819242in}}%
\pgfpathlineto{\pgfqpoint{3.600363in}{1.824998in}}%
\pgfpathlineto{\pgfqpoint{3.608271in}{1.830868in}}%
\pgfpathlineto{\pgfqpoint{3.616171in}{1.836850in}}%
\pgfpathlineto{\pgfqpoint{3.602887in}{1.845161in}}%
\pgfpathlineto{\pgfqpoint{3.589604in}{1.853606in}}%
\pgfpathlineto{\pgfqpoint{3.576324in}{1.862186in}}%
\pgfpathlineto{\pgfqpoint{3.563047in}{1.870901in}}%
\pgfpathlineto{\pgfqpoint{3.555126in}{1.865225in}}%
\pgfpathlineto{\pgfqpoint{3.547197in}{1.859665in}}%
\pgfpathlineto{\pgfqpoint{3.539260in}{1.854223in}}%
\pgfpathlineto{\pgfqpoint{3.531315in}{1.848901in}}%
\pgfpathclose%
\pgfusepath{fill}%
\end{pgfscope}%
\begin{pgfscope}%
\pgfpathrectangle{\pgfqpoint{1.254980in}{0.150000in}}{\pgfqpoint{5.490039in}{5.490039in}}%
\pgfusepath{clip}%
\pgfsetbuttcap%
\pgfsetroundjoin%
\definecolor{currentfill}{rgb}{0.278826,0.175490,0.483397}%
\pgfsetfillcolor{currentfill}%
\pgfsetfillopacity{0.700000}%
\pgfsetlinewidth{0.000000pt}%
\definecolor{currentstroke}{rgb}{0.000000,0.000000,0.000000}%
\pgfsetstrokecolor{currentstroke}%
\pgfsetdash{}{0pt}%
\pgfpathmoveto{\pgfqpoint{4.725939in}{2.007796in}}%
\pgfpathlineto{\pgfqpoint{4.739519in}{2.009176in}}%
\pgfpathlineto{\pgfqpoint{4.753110in}{2.010671in}}%
\pgfpathlineto{\pgfqpoint{4.766711in}{2.012281in}}%
\pgfpathlineto{\pgfqpoint{4.780324in}{2.014006in}}%
\pgfpathlineto{\pgfqpoint{4.787825in}{2.025116in}}%
\pgfpathlineto{\pgfqpoint{4.795322in}{2.036203in}}%
\pgfpathlineto{\pgfqpoint{4.802814in}{2.047267in}}%
\pgfpathlineto{\pgfqpoint{4.810301in}{2.058307in}}%
\pgfpathlineto{\pgfqpoint{4.796694in}{2.056447in}}%
\pgfpathlineto{\pgfqpoint{4.783097in}{2.054703in}}%
\pgfpathlineto{\pgfqpoint{4.769512in}{2.053074in}}%
\pgfpathlineto{\pgfqpoint{4.755937in}{2.051559in}}%
\pgfpathlineto{\pgfqpoint{4.748444in}{2.040648in}}%
\pgfpathlineto{\pgfqpoint{4.740947in}{2.029717in}}%
\pgfpathlineto{\pgfqpoint{4.733445in}{2.018765in}}%
\pgfpathlineto{\pgfqpoint{4.725939in}{2.007796in}}%
\pgfpathclose%
\pgfusepath{fill}%
\end{pgfscope}%
\begin{pgfscope}%
\pgfpathrectangle{\pgfqpoint{1.254980in}{0.150000in}}{\pgfqpoint{5.490039in}{5.490039in}}%
\pgfusepath{clip}%
\pgfsetbuttcap%
\pgfsetroundjoin%
\definecolor{currentfill}{rgb}{0.281446,0.084320,0.407414}%
\pgfsetfillcolor{currentfill}%
\pgfsetfillopacity{0.700000}%
\pgfsetlinewidth{0.000000pt}%
\definecolor{currentstroke}{rgb}{0.000000,0.000000,0.000000}%
\pgfsetstrokecolor{currentstroke}%
\pgfsetdash{}{0pt}%
\pgfpathmoveto{\pgfqpoint{4.388851in}{1.835436in}}%
\pgfpathlineto{\pgfqpoint{4.402305in}{1.834258in}}%
\pgfpathlineto{\pgfqpoint{4.415767in}{1.833198in}}%
\pgfpathlineto{\pgfqpoint{4.429238in}{1.832256in}}%
\pgfpathlineto{\pgfqpoint{4.442718in}{1.831431in}}%
\pgfpathlineto{\pgfqpoint{4.450319in}{1.841937in}}%
\pgfpathlineto{\pgfqpoint{4.457915in}{1.852456in}}%
\pgfpathlineto{\pgfqpoint{4.465507in}{1.862986in}}%
\pgfpathlineto{\pgfqpoint{4.473094in}{1.873525in}}%
\pgfpathlineto{\pgfqpoint{4.459622in}{1.874153in}}%
\pgfpathlineto{\pgfqpoint{4.446158in}{1.874897in}}%
\pgfpathlineto{\pgfqpoint{4.432703in}{1.875760in}}%
\pgfpathlineto{\pgfqpoint{4.419257in}{1.876740in}}%
\pgfpathlineto{\pgfqpoint{4.411663in}{1.866392in}}%
\pgfpathlineto{\pgfqpoint{4.404064in}{1.856058in}}%
\pgfpathlineto{\pgfqpoint{4.396460in}{1.845739in}}%
\pgfpathlineto{\pgfqpoint{4.388851in}{1.835436in}}%
\pgfpathclose%
\pgfusepath{fill}%
\end{pgfscope}%
\begin{pgfscope}%
\pgfpathrectangle{\pgfqpoint{1.254980in}{0.150000in}}{\pgfqpoint{5.490039in}{5.490039in}}%
\pgfusepath{clip}%
\pgfsetbuttcap%
\pgfsetroundjoin%
\definecolor{currentfill}{rgb}{0.166617,0.463708,0.558119}%
\pgfsetfillcolor{currentfill}%
\pgfsetfillopacity{0.700000}%
\pgfsetlinewidth{0.000000pt}%
\definecolor{currentstroke}{rgb}{0.000000,0.000000,0.000000}%
\pgfsetstrokecolor{currentstroke}%
\pgfsetdash{}{0pt}%
\pgfpathmoveto{\pgfqpoint{5.630409in}{2.670645in}}%
\pgfpathlineto{\pgfqpoint{5.644440in}{2.677103in}}%
\pgfpathlineto{\pgfqpoint{5.658485in}{2.683673in}}%
\pgfpathlineto{\pgfqpoint{5.672546in}{2.690356in}}%
\pgfpathlineto{\pgfqpoint{5.686623in}{2.697150in}}%
\pgfpathlineto{\pgfqpoint{5.693788in}{2.706231in}}%
\pgfpathlineto{\pgfqpoint{5.700947in}{2.715244in}}%
\pgfpathlineto{\pgfqpoint{5.708099in}{2.724193in}}%
\pgfpathlineto{\pgfqpoint{5.715244in}{2.733076in}}%
\pgfpathlineto{\pgfqpoint{5.701176in}{2.726345in}}%
\pgfpathlineto{\pgfqpoint{5.687125in}{2.719725in}}%
\pgfpathlineto{\pgfqpoint{5.673088in}{2.713218in}}%
\pgfpathlineto{\pgfqpoint{5.659067in}{2.706823in}}%
\pgfpathlineto{\pgfqpoint{5.651913in}{2.697870in}}%
\pgfpathlineto{\pgfqpoint{5.644752in}{2.688857in}}%
\pgfpathlineto{\pgfqpoint{5.637584in}{2.679782in}}%
\pgfpathlineto{\pgfqpoint{5.630409in}{2.670645in}}%
\pgfpathclose%
\pgfusepath{fill}%
\end{pgfscope}%
\begin{pgfscope}%
\pgfpathrectangle{\pgfqpoint{1.254980in}{0.150000in}}{\pgfqpoint{5.490039in}{5.490039in}}%
\pgfusepath{clip}%
\pgfsetbuttcap%
\pgfsetroundjoin%
\definecolor{currentfill}{rgb}{0.216210,0.351535,0.550627}%
\pgfsetfillcolor{currentfill}%
\pgfsetfillopacity{0.700000}%
\pgfsetlinewidth{0.000000pt}%
\definecolor{currentstroke}{rgb}{0.000000,0.000000,0.000000}%
\pgfsetstrokecolor{currentstroke}%
\pgfsetdash{}{0pt}%
\pgfpathmoveto{\pgfqpoint{5.262479in}{2.384853in}}%
\pgfpathlineto{\pgfqpoint{5.276314in}{2.389571in}}%
\pgfpathlineto{\pgfqpoint{5.290162in}{2.394402in}}%
\pgfpathlineto{\pgfqpoint{5.304024in}{2.399346in}}%
\pgfpathlineto{\pgfqpoint{5.317900in}{2.404403in}}%
\pgfpathlineto{\pgfqpoint{5.325223in}{2.414822in}}%
\pgfpathlineto{\pgfqpoint{5.332540in}{2.425182in}}%
\pgfpathlineto{\pgfqpoint{5.339850in}{2.435483in}}%
\pgfpathlineto{\pgfqpoint{5.347155in}{2.445726in}}%
\pgfpathlineto{\pgfqpoint{5.333285in}{2.440648in}}%
\pgfpathlineto{\pgfqpoint{5.319429in}{2.435683in}}%
\pgfpathlineto{\pgfqpoint{5.305586in}{2.430832in}}%
\pgfpathlineto{\pgfqpoint{5.291757in}{2.426093in}}%
\pgfpathlineto{\pgfqpoint{5.284447in}{2.415864in}}%
\pgfpathlineto{\pgfqpoint{5.277130in}{2.405582in}}%
\pgfpathlineto{\pgfqpoint{5.269807in}{2.395245in}}%
\pgfpathlineto{\pgfqpoint{5.262479in}{2.384853in}}%
\pgfpathclose%
\pgfusepath{fill}%
\end{pgfscope}%
\begin{pgfscope}%
\pgfpathrectangle{\pgfqpoint{1.254980in}{0.150000in}}{\pgfqpoint{5.490039in}{5.490039in}}%
\pgfusepath{clip}%
\pgfsetbuttcap%
\pgfsetroundjoin%
\definecolor{currentfill}{rgb}{0.272594,0.025563,0.353093}%
\pgfsetfillcolor{currentfill}%
\pgfsetfillopacity{0.700000}%
\pgfsetlinewidth{0.000000pt}%
\definecolor{currentstroke}{rgb}{0.000000,0.000000,0.000000}%
\pgfsetstrokecolor{currentstroke}%
\pgfsetdash{}{0pt}%
\pgfpathmoveto{\pgfqpoint{3.860392in}{1.751434in}}%
\pgfpathlineto{\pgfqpoint{3.873720in}{1.745626in}}%
\pgfpathlineto{\pgfqpoint{3.887051in}{1.739945in}}%
\pgfpathlineto{\pgfqpoint{3.900388in}{1.734389in}}%
\pgfpathlineto{\pgfqpoint{3.913730in}{1.728959in}}%
\pgfpathlineto{\pgfqpoint{3.921510in}{1.736879in}}%
\pgfpathlineto{\pgfqpoint{3.929284in}{1.744877in}}%
\pgfpathlineto{\pgfqpoint{3.937052in}{1.752949in}}%
\pgfpathlineto{\pgfqpoint{3.944814in}{1.761094in}}%
\pgfpathlineto{\pgfqpoint{3.931487in}{1.766247in}}%
\pgfpathlineto{\pgfqpoint{3.918165in}{1.771526in}}%
\pgfpathlineto{\pgfqpoint{3.904847in}{1.776931in}}%
\pgfpathlineto{\pgfqpoint{3.891535in}{1.782461in}}%
\pgfpathlineto{\pgfqpoint{3.883759in}{1.774587in}}%
\pgfpathlineto{\pgfqpoint{3.875976in}{1.766790in}}%
\pgfpathlineto{\pgfqpoint{3.868187in}{1.759071in}}%
\pgfpathlineto{\pgfqpoint{3.860392in}{1.751434in}}%
\pgfpathclose%
\pgfusepath{fill}%
\end{pgfscope}%
\begin{pgfscope}%
\pgfpathrectangle{\pgfqpoint{1.254980in}{0.150000in}}{\pgfqpoint{5.490039in}{5.490039in}}%
\pgfusepath{clip}%
\pgfsetbuttcap%
\pgfsetroundjoin%
\definecolor{currentfill}{rgb}{0.274128,0.199721,0.498911}%
\pgfsetfillcolor{currentfill}%
\pgfsetfillopacity{0.700000}%
\pgfsetlinewidth{0.000000pt}%
\definecolor{currentstroke}{rgb}{0.000000,0.000000,0.000000}%
\pgfsetstrokecolor{currentstroke}%
\pgfsetdash{}{0pt}%
\pgfpathmoveto{\pgfqpoint{4.810301in}{2.058307in}}%
\pgfpathlineto{\pgfqpoint{4.823920in}{2.060281in}}%
\pgfpathlineto{\pgfqpoint{4.837549in}{2.062369in}}%
\pgfpathlineto{\pgfqpoint{4.851190in}{2.064573in}}%
\pgfpathlineto{\pgfqpoint{4.864842in}{2.066890in}}%
\pgfpathlineto{\pgfqpoint{4.872319in}{2.078029in}}%
\pgfpathlineto{\pgfqpoint{4.879792in}{2.089138in}}%
\pgfpathlineto{\pgfqpoint{4.887259in}{2.100215in}}%
\pgfpathlineto{\pgfqpoint{4.894721in}{2.111262in}}%
\pgfpathlineto{\pgfqpoint{4.881074in}{2.108826in}}%
\pgfpathlineto{\pgfqpoint{4.867438in}{2.106504in}}%
\pgfpathlineto{\pgfqpoint{4.853814in}{2.104297in}}%
\pgfpathlineto{\pgfqpoint{4.840200in}{2.102205in}}%
\pgfpathlineto{\pgfqpoint{4.832733in}{2.091271in}}%
\pgfpathlineto{\pgfqpoint{4.825261in}{2.080309in}}%
\pgfpathlineto{\pgfqpoint{4.817783in}{2.069321in}}%
\pgfpathlineto{\pgfqpoint{4.810301in}{2.058307in}}%
\pgfpathclose%
\pgfusepath{fill}%
\end{pgfscope}%
\begin{pgfscope}%
\pgfpathrectangle{\pgfqpoint{1.254980in}{0.150000in}}{\pgfqpoint{5.490039in}{5.490039in}}%
\pgfusepath{clip}%
\pgfsetbuttcap%
\pgfsetroundjoin%
\definecolor{currentfill}{rgb}{0.274952,0.037752,0.364543}%
\pgfsetfillcolor{currentfill}%
\pgfsetfillopacity{0.700000}%
\pgfsetlinewidth{0.000000pt}%
\definecolor{currentstroke}{rgb}{0.000000,0.000000,0.000000}%
\pgfsetstrokecolor{currentstroke}%
\pgfsetdash{}{0pt}%
\pgfpathmoveto{\pgfqpoint{3.722549in}{1.775135in}}%
\pgfpathlineto{\pgfqpoint{3.735861in}{1.768010in}}%
\pgfpathlineto{\pgfqpoint{3.749176in}{1.761015in}}%
\pgfpathlineto{\pgfqpoint{3.762495in}{1.754148in}}%
\pgfpathlineto{\pgfqpoint{3.775818in}{1.747410in}}%
\pgfpathlineto{\pgfqpoint{3.783655in}{1.754402in}}%
\pgfpathlineto{\pgfqpoint{3.791486in}{1.761488in}}%
\pgfpathlineto{\pgfqpoint{3.799310in}{1.768666in}}%
\pgfpathlineto{\pgfqpoint{3.807127in}{1.775933in}}%
\pgfpathlineto{\pgfqpoint{3.793822in}{1.782377in}}%
\pgfpathlineto{\pgfqpoint{3.780520in}{1.788949in}}%
\pgfpathlineto{\pgfqpoint{3.767222in}{1.795651in}}%
\pgfpathlineto{\pgfqpoint{3.753928in}{1.802482in}}%
\pgfpathlineto{\pgfqpoint{3.746094in}{1.795502in}}%
\pgfpathlineto{\pgfqpoint{3.738253in}{1.788616in}}%
\pgfpathlineto{\pgfqpoint{3.730404in}{1.781827in}}%
\pgfpathlineto{\pgfqpoint{3.722549in}{1.775135in}}%
\pgfpathclose%
\pgfusepath{fill}%
\end{pgfscope}%
\begin{pgfscope}%
\pgfpathrectangle{\pgfqpoint{1.254980in}{0.150000in}}{\pgfqpoint{5.490039in}{5.490039in}}%
\pgfusepath{clip}%
\pgfsetbuttcap%
\pgfsetroundjoin%
\definecolor{currentfill}{rgb}{0.278791,0.062145,0.386592}%
\pgfsetfillcolor{currentfill}%
\pgfsetfillopacity{0.700000}%
\pgfsetlinewidth{0.000000pt}%
\definecolor{currentstroke}{rgb}{0.000000,0.000000,0.000000}%
\pgfsetstrokecolor{currentstroke}%
\pgfsetdash{}{0pt}%
\pgfpathmoveto{\pgfqpoint{4.304603in}{1.801161in}}%
\pgfpathlineto{\pgfqpoint{4.318032in}{1.799297in}}%
\pgfpathlineto{\pgfqpoint{4.331470in}{1.797552in}}%
\pgfpathlineto{\pgfqpoint{4.344915in}{1.795925in}}%
\pgfpathlineto{\pgfqpoint{4.358369in}{1.794417in}}%
\pgfpathlineto{\pgfqpoint{4.365996in}{1.804639in}}%
\pgfpathlineto{\pgfqpoint{4.373620in}{1.814884in}}%
\pgfpathlineto{\pgfqpoint{4.381238in}{1.825150in}}%
\pgfpathlineto{\pgfqpoint{4.388851in}{1.835436in}}%
\pgfpathlineto{\pgfqpoint{4.375406in}{1.836731in}}%
\pgfpathlineto{\pgfqpoint{4.361969in}{1.838145in}}%
\pgfpathlineto{\pgfqpoint{4.348539in}{1.839677in}}%
\pgfpathlineto{\pgfqpoint{4.335118in}{1.841328in}}%
\pgfpathlineto{\pgfqpoint{4.327497in}{1.831249in}}%
\pgfpathlineto{\pgfqpoint{4.319870in}{1.821195in}}%
\pgfpathlineto{\pgfqpoint{4.312239in}{1.811165in}}%
\pgfpathlineto{\pgfqpoint{4.304603in}{1.801161in}}%
\pgfpathclose%
\pgfusepath{fill}%
\end{pgfscope}%
\begin{pgfscope}%
\pgfpathrectangle{\pgfqpoint{1.254980in}{0.150000in}}{\pgfqpoint{5.490039in}{5.490039in}}%
\pgfusepath{clip}%
\pgfsetbuttcap%
\pgfsetroundjoin%
\definecolor{currentfill}{rgb}{0.143343,0.522773,0.556295}%
\pgfsetfillcolor{currentfill}%
\pgfsetfillopacity{0.700000}%
\pgfsetlinewidth{0.000000pt}%
\definecolor{currentstroke}{rgb}{0.000000,0.000000,0.000000}%
\pgfsetstrokecolor{currentstroke}%
\pgfsetdash{}{0pt}%
\pgfpathmoveto{\pgfqpoint{2.536538in}{2.904152in}}%
\pgfpathlineto{\pgfqpoint{2.550086in}{2.883036in}}%
\pgfpathlineto{\pgfqpoint{2.563625in}{2.862119in}}%
\pgfpathlineto{\pgfqpoint{2.577156in}{2.841402in}}%
\pgfpathlineto{\pgfqpoint{2.590678in}{2.820882in}}%
\pgfpathlineto{\pgfqpoint{2.599202in}{2.819774in}}%
\pgfpathlineto{\pgfqpoint{2.607710in}{2.818874in}}%
\pgfpathlineto{\pgfqpoint{2.616203in}{2.818177in}}%
\pgfpathlineto{\pgfqpoint{2.624681in}{2.817682in}}%
\pgfpathlineto{\pgfqpoint{2.611200in}{2.837842in}}%
\pgfpathlineto{\pgfqpoint{2.597711in}{2.858198in}}%
\pgfpathlineto{\pgfqpoint{2.584214in}{2.878752in}}%
\pgfpathlineto{\pgfqpoint{2.570709in}{2.899505in}}%
\pgfpathlineto{\pgfqpoint{2.562189in}{2.900355in}}%
\pgfpathlineto{\pgfqpoint{2.553654in}{2.901411in}}%
\pgfpathlineto{\pgfqpoint{2.545104in}{2.902675in}}%
\pgfpathlineto{\pgfqpoint{2.536538in}{2.904152in}}%
\pgfpathclose%
\pgfusepath{fill}%
\end{pgfscope}%
\begin{pgfscope}%
\pgfpathrectangle{\pgfqpoint{1.254980in}{0.150000in}}{\pgfqpoint{5.490039in}{5.490039in}}%
\pgfusepath{clip}%
\pgfsetbuttcap%
\pgfsetroundjoin%
\definecolor{currentfill}{rgb}{0.266580,0.228262,0.514349}%
\pgfsetfillcolor{currentfill}%
\pgfsetfillopacity{0.700000}%
\pgfsetlinewidth{0.000000pt}%
\definecolor{currentstroke}{rgb}{0.000000,0.000000,0.000000}%
\pgfsetstrokecolor{currentstroke}%
\pgfsetdash{}{0pt}%
\pgfpathmoveto{\pgfqpoint{4.894721in}{2.111262in}}%
\pgfpathlineto{\pgfqpoint{4.908380in}{2.113812in}}%
\pgfpathlineto{\pgfqpoint{4.922051in}{2.116477in}}%
\pgfpathlineto{\pgfqpoint{4.935733in}{2.119255in}}%
\pgfpathlineto{\pgfqpoint{4.949427in}{2.122148in}}%
\pgfpathlineto{\pgfqpoint{4.956879in}{2.133271in}}%
\pgfpathlineto{\pgfqpoint{4.964327in}{2.144357in}}%
\pgfpathlineto{\pgfqpoint{4.971769in}{2.155405in}}%
\pgfpathlineto{\pgfqpoint{4.979207in}{2.166415in}}%
\pgfpathlineto{\pgfqpoint{4.965517in}{2.163420in}}%
\pgfpathlineto{\pgfqpoint{4.951840in}{2.160539in}}%
\pgfpathlineto{\pgfqpoint{4.938174in}{2.157772in}}%
\pgfpathlineto{\pgfqpoint{4.924520in}{2.155120in}}%
\pgfpathlineto{\pgfqpoint{4.917078in}{2.144206in}}%
\pgfpathlineto{\pgfqpoint{4.909631in}{2.133258in}}%
\pgfpathlineto{\pgfqpoint{4.902179in}{2.122276in}}%
\pgfpathlineto{\pgfqpoint{4.894721in}{2.111262in}}%
\pgfpathclose%
\pgfusepath{fill}%
\end{pgfscope}%
\begin{pgfscope}%
\pgfpathrectangle{\pgfqpoint{1.254980in}{0.150000in}}{\pgfqpoint{5.490039in}{5.490039in}}%
\pgfusepath{clip}%
\pgfsetbuttcap%
\pgfsetroundjoin%
\definecolor{currentfill}{rgb}{0.283072,0.130895,0.449241}%
\pgfsetfillcolor{currentfill}%
\pgfsetfillopacity{0.700000}%
\pgfsetlinewidth{0.000000pt}%
\definecolor{currentstroke}{rgb}{0.000000,0.000000,0.000000}%
\pgfsetstrokecolor{currentstroke}%
\pgfsetdash{}{0pt}%
\pgfpathmoveto{\pgfqpoint{3.339689in}{1.951932in}}%
\pgfpathlineto{\pgfqpoint{3.352999in}{1.940903in}}%
\pgfpathlineto{\pgfqpoint{3.366309in}{1.930017in}}%
\pgfpathlineto{\pgfqpoint{3.379619in}{1.919273in}}%
\pgfpathlineto{\pgfqpoint{3.392930in}{1.908671in}}%
\pgfpathlineto{\pgfqpoint{3.400957in}{1.912812in}}%
\pgfpathlineto{\pgfqpoint{3.408974in}{1.917093in}}%
\pgfpathlineto{\pgfqpoint{3.416982in}{1.921511in}}%
\pgfpathlineto{\pgfqpoint{3.424981in}{1.926062in}}%
\pgfpathlineto{\pgfqpoint{3.411694in}{1.936334in}}%
\pgfpathlineto{\pgfqpoint{3.398409in}{1.946746in}}%
\pgfpathlineto{\pgfqpoint{3.385124in}{1.957300in}}%
\pgfpathlineto{\pgfqpoint{3.371839in}{1.967997in}}%
\pgfpathlineto{\pgfqpoint{3.363816in}{1.963771in}}%
\pgfpathlineto{\pgfqpoint{3.355783in}{1.959683in}}%
\pgfpathlineto{\pgfqpoint{3.347740in}{1.955735in}}%
\pgfpathlineto{\pgfqpoint{3.339689in}{1.951932in}}%
\pgfpathclose%
\pgfusepath{fill}%
\end{pgfscope}%
\begin{pgfscope}%
\pgfpathrectangle{\pgfqpoint{1.254980in}{0.150000in}}{\pgfqpoint{5.490039in}{5.490039in}}%
\pgfusepath{clip}%
\pgfsetbuttcap%
\pgfsetroundjoin%
\definecolor{currentfill}{rgb}{0.272594,0.025563,0.353093}%
\pgfsetfillcolor{currentfill}%
\pgfsetfillopacity{0.700000}%
\pgfsetlinewidth{0.000000pt}%
\definecolor{currentstroke}{rgb}{0.000000,0.000000,0.000000}%
\pgfsetstrokecolor{currentstroke}%
\pgfsetdash{}{0pt}%
\pgfpathmoveto{\pgfqpoint{3.998175in}{1.741726in}}%
\pgfpathlineto{\pgfqpoint{4.011529in}{1.737193in}}%
\pgfpathlineto{\pgfqpoint{4.024888in}{1.732784in}}%
\pgfpathlineto{\pgfqpoint{4.038254in}{1.728498in}}%
\pgfpathlineto{\pgfqpoint{4.051625in}{1.724334in}}%
\pgfpathlineto{\pgfqpoint{4.059355in}{1.733083in}}%
\pgfpathlineto{\pgfqpoint{4.067079in}{1.741892in}}%
\pgfpathlineto{\pgfqpoint{4.074798in}{1.750759in}}%
\pgfpathlineto{\pgfqpoint{4.082511in}{1.759683in}}%
\pgfpathlineto{\pgfqpoint{4.069152in}{1.763586in}}%
\pgfpathlineto{\pgfqpoint{4.055799in}{1.767612in}}%
\pgfpathlineto{\pgfqpoint{4.042452in}{1.771760in}}%
\pgfpathlineto{\pgfqpoint{4.029111in}{1.776032in}}%
\pgfpathlineto{\pgfqpoint{4.021385in}{1.767363in}}%
\pgfpathlineto{\pgfqpoint{4.013654in}{1.758754in}}%
\pgfpathlineto{\pgfqpoint{4.005917in}{1.750208in}}%
\pgfpathlineto{\pgfqpoint{3.998175in}{1.741726in}}%
\pgfpathclose%
\pgfusepath{fill}%
\end{pgfscope}%
\begin{pgfscope}%
\pgfpathrectangle{\pgfqpoint{1.254980in}{0.150000in}}{\pgfqpoint{5.490039in}{5.490039in}}%
\pgfusepath{clip}%
\pgfsetbuttcap%
\pgfsetroundjoin%
\definecolor{currentfill}{rgb}{0.204903,0.375746,0.553533}%
\pgfsetfillcolor{currentfill}%
\pgfsetfillopacity{0.700000}%
\pgfsetlinewidth{0.000000pt}%
\definecolor{currentstroke}{rgb}{0.000000,0.000000,0.000000}%
\pgfsetstrokecolor{currentstroke}%
\pgfsetdash{}{0pt}%
\pgfpathmoveto{\pgfqpoint{5.347155in}{2.445726in}}%
\pgfpathlineto{\pgfqpoint{5.361039in}{2.450917in}}%
\pgfpathlineto{\pgfqpoint{5.374937in}{2.456220in}}%
\pgfpathlineto{\pgfqpoint{5.388849in}{2.461637in}}%
\pgfpathlineto{\pgfqpoint{5.402776in}{2.467166in}}%
\pgfpathlineto{\pgfqpoint{5.410068in}{2.477361in}}%
\pgfpathlineto{\pgfqpoint{5.417355in}{2.487494in}}%
\pgfpathlineto{\pgfqpoint{5.424635in}{2.497565in}}%
\pgfpathlineto{\pgfqpoint{5.431909in}{2.507574in}}%
\pgfpathlineto{\pgfqpoint{5.417989in}{2.502041in}}%
\pgfpathlineto{\pgfqpoint{5.404083in}{2.496620in}}%
\pgfpathlineto{\pgfqpoint{5.390191in}{2.491312in}}%
\pgfpathlineto{\pgfqpoint{5.376314in}{2.486118in}}%
\pgfpathlineto{\pgfqpoint{5.369033in}{2.476106in}}%
\pgfpathlineto{\pgfqpoint{5.361747in}{2.466037in}}%
\pgfpathlineto{\pgfqpoint{5.354454in}{2.455911in}}%
\pgfpathlineto{\pgfqpoint{5.347155in}{2.445726in}}%
\pgfpathclose%
\pgfusepath{fill}%
\end{pgfscope}%
\begin{pgfscope}%
\pgfpathrectangle{\pgfqpoint{1.254980in}{0.150000in}}{\pgfqpoint{5.490039in}{5.490039in}}%
\pgfusepath{clip}%
\pgfsetbuttcap%
\pgfsetroundjoin%
\definecolor{currentfill}{rgb}{0.156270,0.489624,0.557936}%
\pgfsetfillcolor{currentfill}%
\pgfsetfillopacity{0.700000}%
\pgfsetlinewidth{0.000000pt}%
\definecolor{currentstroke}{rgb}{0.000000,0.000000,0.000000}%
\pgfsetstrokecolor{currentstroke}%
\pgfsetdash{}{0pt}%
\pgfpathmoveto{\pgfqpoint{5.715244in}{2.733076in}}%
\pgfpathlineto{\pgfqpoint{5.729327in}{2.739920in}}%
\pgfpathlineto{\pgfqpoint{5.743425in}{2.746876in}}%
\pgfpathlineto{\pgfqpoint{5.757540in}{2.753944in}}%
\pgfpathlineto{\pgfqpoint{5.771670in}{2.761124in}}%
\pgfpathlineto{\pgfqpoint{5.778798in}{2.769870in}}%
\pgfpathlineto{\pgfqpoint{5.785919in}{2.778548in}}%
\pgfpathlineto{\pgfqpoint{5.793034in}{2.787161in}}%
\pgfpathlineto{\pgfqpoint{5.800141in}{2.795709in}}%
\pgfpathlineto{\pgfqpoint{5.786021in}{2.788609in}}%
\pgfpathlineto{\pgfqpoint{5.771916in}{2.781621in}}%
\pgfpathlineto{\pgfqpoint{5.757828in}{2.774745in}}%
\pgfpathlineto{\pgfqpoint{5.743755in}{2.767982in}}%
\pgfpathlineto{\pgfqpoint{5.736638in}{2.759348in}}%
\pgfpathlineto{\pgfqpoint{5.729513in}{2.750653in}}%
\pgfpathlineto{\pgfqpoint{5.722382in}{2.741896in}}%
\pgfpathlineto{\pgfqpoint{5.715244in}{2.733076in}}%
\pgfpathclose%
\pgfusepath{fill}%
\end{pgfscope}%
\begin{pgfscope}%
\pgfpathrectangle{\pgfqpoint{1.254980in}{0.150000in}}{\pgfqpoint{5.490039in}{5.490039in}}%
\pgfusepath{clip}%
\pgfsetbuttcap%
\pgfsetroundjoin%
\definecolor{currentfill}{rgb}{0.277018,0.050344,0.375715}%
\pgfsetfillcolor{currentfill}%
\pgfsetfillopacity{0.700000}%
\pgfsetlinewidth{0.000000pt}%
\definecolor{currentstroke}{rgb}{0.000000,0.000000,0.000000}%
\pgfsetstrokecolor{currentstroke}%
\pgfsetdash{}{0pt}%
\pgfpathmoveto{\pgfqpoint{4.220329in}{1.771007in}}%
\pgfpathlineto{\pgfqpoint{4.233738in}{1.768437in}}%
\pgfpathlineto{\pgfqpoint{4.247154in}{1.765987in}}%
\pgfpathlineto{\pgfqpoint{4.260577in}{1.763657in}}%
\pgfpathlineto{\pgfqpoint{4.274008in}{1.761445in}}%
\pgfpathlineto{\pgfqpoint{4.281664in}{1.771326in}}%
\pgfpathlineto{\pgfqpoint{4.289315in}{1.781240in}}%
\pgfpathlineto{\pgfqpoint{4.296961in}{1.791186in}}%
\pgfpathlineto{\pgfqpoint{4.304603in}{1.801161in}}%
\pgfpathlineto{\pgfqpoint{4.291181in}{1.803144in}}%
\pgfpathlineto{\pgfqpoint{4.277767in}{1.805246in}}%
\pgfpathlineto{\pgfqpoint{4.264360in}{1.807467in}}%
\pgfpathlineto{\pgfqpoint{4.250961in}{1.809808in}}%
\pgfpathlineto{\pgfqpoint{4.243311in}{1.800055in}}%
\pgfpathlineto{\pgfqpoint{4.235655in}{1.790336in}}%
\pgfpathlineto{\pgfqpoint{4.227995in}{1.780653in}}%
\pgfpathlineto{\pgfqpoint{4.220329in}{1.771007in}}%
\pgfpathclose%
\pgfusepath{fill}%
\end{pgfscope}%
\begin{pgfscope}%
\pgfpathrectangle{\pgfqpoint{1.254980in}{0.150000in}}{\pgfqpoint{5.490039in}{5.490039in}}%
\pgfusepath{clip}%
\pgfsetbuttcap%
\pgfsetroundjoin%
\definecolor{currentfill}{rgb}{0.278791,0.062145,0.386592}%
\pgfsetfillcolor{currentfill}%
\pgfsetfillopacity{0.700000}%
\pgfsetlinewidth{0.000000pt}%
\definecolor{currentstroke}{rgb}{0.000000,0.000000,0.000000}%
\pgfsetstrokecolor{currentstroke}%
\pgfsetdash{}{0pt}%
\pgfpathmoveto{\pgfqpoint{3.584523in}{1.813601in}}%
\pgfpathlineto{\pgfqpoint{3.597831in}{1.805112in}}%
\pgfpathlineto{\pgfqpoint{3.611141in}{1.796757in}}%
\pgfpathlineto{\pgfqpoint{3.624453in}{1.788534in}}%
\pgfpathlineto{\pgfqpoint{3.637769in}{1.780444in}}%
\pgfpathlineto{\pgfqpoint{3.645672in}{1.786402in}}%
\pgfpathlineto{\pgfqpoint{3.653567in}{1.792472in}}%
\pgfpathlineto{\pgfqpoint{3.661455in}{1.798651in}}%
\pgfpathlineto{\pgfqpoint{3.669336in}{1.804938in}}%
\pgfpathlineto{\pgfqpoint{3.656041in}{1.812717in}}%
\pgfpathlineto{\pgfqpoint{3.642748in}{1.820628in}}%
\pgfpathlineto{\pgfqpoint{3.629458in}{1.828672in}}%
\pgfpathlineto{\pgfqpoint{3.616171in}{1.836850in}}%
\pgfpathlineto{\pgfqpoint{3.608271in}{1.830868in}}%
\pgfpathlineto{\pgfqpoint{3.600363in}{1.824998in}}%
\pgfpathlineto{\pgfqpoint{3.592447in}{1.819242in}}%
\pgfpathlineto{\pgfqpoint{3.584523in}{1.813601in}}%
\pgfpathclose%
\pgfusepath{fill}%
\end{pgfscope}%
\begin{pgfscope}%
\pgfpathrectangle{\pgfqpoint{1.254980in}{0.150000in}}{\pgfqpoint{5.490039in}{5.490039in}}%
\pgfusepath{clip}%
\pgfsetbuttcap%
\pgfsetroundjoin%
\definecolor{currentfill}{rgb}{0.257322,0.256130,0.526563}%
\pgfsetfillcolor{currentfill}%
\pgfsetfillopacity{0.700000}%
\pgfsetlinewidth{0.000000pt}%
\definecolor{currentstroke}{rgb}{0.000000,0.000000,0.000000}%
\pgfsetstrokecolor{currentstroke}%
\pgfsetdash{}{0pt}%
\pgfpathmoveto{\pgfqpoint{4.979207in}{2.166415in}}%
\pgfpathlineto{\pgfqpoint{4.992908in}{2.169524in}}%
\pgfpathlineto{\pgfqpoint{5.006621in}{2.172747in}}%
\pgfpathlineto{\pgfqpoint{5.020347in}{2.176084in}}%
\pgfpathlineto{\pgfqpoint{5.034085in}{2.179534in}}%
\pgfpathlineto{\pgfqpoint{5.041512in}{2.190599in}}%
\pgfpathlineto{\pgfqpoint{5.048935in}{2.201620in}}%
\pgfpathlineto{\pgfqpoint{5.056351in}{2.212597in}}%
\pgfpathlineto{\pgfqpoint{5.063763in}{2.223530in}}%
\pgfpathlineto{\pgfqpoint{5.050029in}{2.219994in}}%
\pgfpathlineto{\pgfqpoint{5.036308in}{2.216571in}}%
\pgfpathlineto{\pgfqpoint{5.022600in}{2.213262in}}%
\pgfpathlineto{\pgfqpoint{5.008903in}{2.210066in}}%
\pgfpathlineto{\pgfqpoint{5.001487in}{2.199213in}}%
\pgfpathlineto{\pgfqpoint{4.994065in}{2.188320in}}%
\pgfpathlineto{\pgfqpoint{4.986639in}{2.177387in}}%
\pgfpathlineto{\pgfqpoint{4.979207in}{2.166415in}}%
\pgfpathclose%
\pgfusepath{fill}%
\end{pgfscope}%
\begin{pgfscope}%
\pgfpathrectangle{\pgfqpoint{1.254980in}{0.150000in}}{\pgfqpoint{5.490039in}{5.490039in}}%
\pgfusepath{clip}%
\pgfsetbuttcap%
\pgfsetroundjoin%
\definecolor{currentfill}{rgb}{0.147607,0.511733,0.557049}%
\pgfsetfillcolor{currentfill}%
\pgfsetfillopacity{0.700000}%
\pgfsetlinewidth{0.000000pt}%
\definecolor{currentstroke}{rgb}{0.000000,0.000000,0.000000}%
\pgfsetstrokecolor{currentstroke}%
\pgfsetdash{}{0pt}%
\pgfpathmoveto{\pgfqpoint{5.800141in}{2.795709in}}%
\pgfpathlineto{\pgfqpoint{5.814277in}{2.802920in}}%
\pgfpathlineto{\pgfqpoint{5.828429in}{2.810244in}}%
\pgfpathlineto{\pgfqpoint{5.842598in}{2.817680in}}%
\pgfpathlineto{\pgfqpoint{5.849690in}{2.826096in}}%
\pgfpathlineto{\pgfqpoint{5.856774in}{2.834445in}}%
\pgfpathlineto{\pgfqpoint{5.863852in}{2.842729in}}%
\pgfpathlineto{\pgfqpoint{5.870923in}{2.850949in}}%
\pgfpathlineto{\pgfqpoint{5.856765in}{2.843611in}}%
\pgfpathlineto{\pgfqpoint{5.842624in}{2.836385in}}%
\pgfpathlineto{\pgfqpoint{5.828499in}{2.829270in}}%
\pgfpathlineto{\pgfqpoint{5.821420in}{2.820972in}}%
\pgfpathlineto{\pgfqpoint{5.814334in}{2.812613in}}%
\pgfpathlineto{\pgfqpoint{5.807241in}{2.804192in}}%
\pgfpathlineto{\pgfqpoint{5.800141in}{2.795709in}}%
\pgfpathclose%
\pgfusepath{fill}%
\end{pgfscope}%
\begin{pgfscope}%
\pgfpathrectangle{\pgfqpoint{1.254980in}{0.150000in}}{\pgfqpoint{5.490039in}{5.490039in}}%
\pgfusepath{clip}%
\pgfsetbuttcap%
\pgfsetroundjoin%
\definecolor{currentfill}{rgb}{0.131172,0.555899,0.552459}%
\pgfsetfillcolor{currentfill}%
\pgfsetfillopacity{0.700000}%
\pgfsetlinewidth{0.000000pt}%
\definecolor{currentstroke}{rgb}{0.000000,0.000000,0.000000}%
\pgfsetstrokecolor{currentstroke}%
\pgfsetdash{}{0pt}%
\pgfpathmoveto{\pgfqpoint{2.482255in}{2.990645in}}%
\pgfpathlineto{\pgfqpoint{2.495840in}{2.968714in}}%
\pgfpathlineto{\pgfqpoint{2.509415in}{2.946990in}}%
\pgfpathlineto{\pgfqpoint{2.522981in}{2.925469in}}%
\pgfpathlineto{\pgfqpoint{2.536538in}{2.904152in}}%
\pgfpathlineto{\pgfqpoint{2.545104in}{2.902675in}}%
\pgfpathlineto{\pgfqpoint{2.553654in}{2.901411in}}%
\pgfpathlineto{\pgfqpoint{2.562189in}{2.900355in}}%
\pgfpathlineto{\pgfqpoint{2.570709in}{2.899505in}}%
\pgfpathlineto{\pgfqpoint{2.557194in}{2.920459in}}%
\pgfpathlineto{\pgfqpoint{2.543671in}{2.941615in}}%
\pgfpathlineto{\pgfqpoint{2.530139in}{2.962975in}}%
\pgfpathlineto{\pgfqpoint{2.516598in}{2.984539in}}%
\pgfpathlineto{\pgfqpoint{2.508036in}{2.985746in}}%
\pgfpathlineto{\pgfqpoint{2.499458in}{2.987164in}}%
\pgfpathlineto{\pgfqpoint{2.490865in}{2.988796in}}%
\pgfpathlineto{\pgfqpoint{2.482255in}{2.990645in}}%
\pgfpathclose%
\pgfusepath{fill}%
\end{pgfscope}%
\begin{pgfscope}%
\pgfpathrectangle{\pgfqpoint{1.254980in}{0.150000in}}{\pgfqpoint{5.490039in}{5.490039in}}%
\pgfusepath{clip}%
\pgfsetbuttcap%
\pgfsetroundjoin%
\definecolor{currentfill}{rgb}{0.283091,0.110553,0.431554}%
\pgfsetfillcolor{currentfill}%
\pgfsetfillopacity{0.700000}%
\pgfsetlinewidth{0.000000pt}%
\definecolor{currentstroke}{rgb}{0.000000,0.000000,0.000000}%
\pgfsetstrokecolor{currentstroke}%
\pgfsetdash{}{0pt}%
\pgfpathmoveto{\pgfqpoint{3.392930in}{1.908671in}}%
\pgfpathlineto{\pgfqpoint{3.406242in}{1.898209in}}%
\pgfpathlineto{\pgfqpoint{3.419555in}{1.887887in}}%
\pgfpathlineto{\pgfqpoint{3.432868in}{1.877705in}}%
\pgfpathlineto{\pgfqpoint{3.446183in}{1.867662in}}%
\pgfpathlineto{\pgfqpoint{3.454184in}{1.872141in}}%
\pgfpathlineto{\pgfqpoint{3.462177in}{1.876754in}}%
\pgfpathlineto{\pgfqpoint{3.470161in}{1.881500in}}%
\pgfpathlineto{\pgfqpoint{3.478137in}{1.886376in}}%
\pgfpathlineto{\pgfqpoint{3.464846in}{1.896089in}}%
\pgfpathlineto{\pgfqpoint{3.451557in}{1.905941in}}%
\pgfpathlineto{\pgfqpoint{3.438268in}{1.915931in}}%
\pgfpathlineto{\pgfqpoint{3.424981in}{1.926062in}}%
\pgfpathlineto{\pgfqpoint{3.416982in}{1.921511in}}%
\pgfpathlineto{\pgfqpoint{3.408974in}{1.917093in}}%
\pgfpathlineto{\pgfqpoint{3.400957in}{1.912812in}}%
\pgfpathlineto{\pgfqpoint{3.392930in}{1.908671in}}%
\pgfpathclose%
\pgfusepath{fill}%
\end{pgfscope}%
\begin{pgfscope}%
\pgfpathrectangle{\pgfqpoint{1.254980in}{0.150000in}}{\pgfqpoint{5.490039in}{5.490039in}}%
\pgfusepath{clip}%
\pgfsetbuttcap%
\pgfsetroundjoin%
\definecolor{currentfill}{rgb}{0.192357,0.403199,0.555836}%
\pgfsetfillcolor{currentfill}%
\pgfsetfillopacity{0.700000}%
\pgfsetlinewidth{0.000000pt}%
\definecolor{currentstroke}{rgb}{0.000000,0.000000,0.000000}%
\pgfsetstrokecolor{currentstroke}%
\pgfsetdash{}{0pt}%
\pgfpathmoveto{\pgfqpoint{5.431909in}{2.507574in}}%
\pgfpathlineto{\pgfqpoint{5.445843in}{2.513220in}}%
\pgfpathlineto{\pgfqpoint{5.459792in}{2.518979in}}%
\pgfpathlineto{\pgfqpoint{5.473756in}{2.524850in}}%
\pgfpathlineto{\pgfqpoint{5.487734in}{2.530834in}}%
\pgfpathlineto{\pgfqpoint{5.494995in}{2.540775in}}%
\pgfpathlineto{\pgfqpoint{5.502250in}{2.550652in}}%
\pgfpathlineto{\pgfqpoint{5.509498in}{2.560463in}}%
\pgfpathlineto{\pgfqpoint{5.516739in}{2.570211in}}%
\pgfpathlineto{\pgfqpoint{5.502768in}{2.564240in}}%
\pgfpathlineto{\pgfqpoint{5.488811in}{2.558381in}}%
\pgfpathlineto{\pgfqpoint{5.474869in}{2.552635in}}%
\pgfpathlineto{\pgfqpoint{5.460942in}{2.547001in}}%
\pgfpathlineto{\pgfqpoint{5.453693in}{2.537235in}}%
\pgfpathlineto{\pgfqpoint{5.446438in}{2.527409in}}%
\pgfpathlineto{\pgfqpoint{5.439177in}{2.517522in}}%
\pgfpathlineto{\pgfqpoint{5.431909in}{2.507574in}}%
\pgfpathclose%
\pgfusepath{fill}%
\end{pgfscope}%
\begin{pgfscope}%
\pgfpathrectangle{\pgfqpoint{1.254980in}{0.150000in}}{\pgfqpoint{5.490039in}{5.490039in}}%
\pgfusepath{clip}%
\pgfsetbuttcap%
\pgfsetroundjoin%
\definecolor{currentfill}{rgb}{0.246811,0.283237,0.535941}%
\pgfsetfillcolor{currentfill}%
\pgfsetfillopacity{0.700000}%
\pgfsetlinewidth{0.000000pt}%
\definecolor{currentstroke}{rgb}{0.000000,0.000000,0.000000}%
\pgfsetstrokecolor{currentstroke}%
\pgfsetdash{}{0pt}%
\pgfpathmoveto{\pgfqpoint{5.063763in}{2.223530in}}%
\pgfpathlineto{\pgfqpoint{5.077509in}{2.227181in}}%
\pgfpathlineto{\pgfqpoint{5.091267in}{2.230944in}}%
\pgfpathlineto{\pgfqpoint{5.105038in}{2.234822in}}%
\pgfpathlineto{\pgfqpoint{5.118822in}{2.238812in}}%
\pgfpathlineto{\pgfqpoint{5.126224in}{2.249777in}}%
\pgfpathlineto{\pgfqpoint{5.133620in}{2.260694in}}%
\pgfpathlineto{\pgfqpoint{5.141010in}{2.271561in}}%
\pgfpathlineto{\pgfqpoint{5.148395in}{2.282378in}}%
\pgfpathlineto{\pgfqpoint{5.134616in}{2.278317in}}%
\pgfpathlineto{\pgfqpoint{5.120849in}{2.274370in}}%
\pgfpathlineto{\pgfqpoint{5.107096in}{2.270536in}}%
\pgfpathlineto{\pgfqpoint{5.093355in}{2.266816in}}%
\pgfpathlineto{\pgfqpoint{5.085965in}{2.256062in}}%
\pgfpathlineto{\pgfqpoint{5.078570in}{2.245263in}}%
\pgfpathlineto{\pgfqpoint{5.071169in}{2.234419in}}%
\pgfpathlineto{\pgfqpoint{5.063763in}{2.223530in}}%
\pgfpathclose%
\pgfusepath{fill}%
\end{pgfscope}%
\begin{pgfscope}%
\pgfpathrectangle{\pgfqpoint{1.254980in}{0.150000in}}{\pgfqpoint{5.490039in}{5.490039in}}%
\pgfusepath{clip}%
\pgfsetbuttcap%
\pgfsetroundjoin%
\definecolor{currentfill}{rgb}{0.239346,0.300855,0.540844}%
\pgfsetfillcolor{currentfill}%
\pgfsetfillopacity{0.700000}%
\pgfsetlinewidth{0.000000pt}%
\definecolor{currentstroke}{rgb}{0.000000,0.000000,0.000000}%
\pgfsetstrokecolor{currentstroke}%
\pgfsetdash{}{0pt}%
\pgfpathmoveto{\pgfqpoint{2.933454in}{2.317673in}}%
\pgfpathlineto{\pgfqpoint{2.946842in}{2.301999in}}%
\pgfpathlineto{\pgfqpoint{2.960227in}{2.286489in}}%
\pgfpathlineto{\pgfqpoint{2.973608in}{2.271143in}}%
\pgfpathlineto{\pgfqpoint{2.986986in}{2.255959in}}%
\pgfpathlineto{\pgfqpoint{2.995261in}{2.256997in}}%
\pgfpathlineto{\pgfqpoint{3.003523in}{2.258217in}}%
\pgfpathlineto{\pgfqpoint{3.011774in}{2.259615in}}%
\pgfpathlineto{\pgfqpoint{3.020012in}{2.261189in}}%
\pgfpathlineto{\pgfqpoint{3.006668in}{2.276013in}}%
\pgfpathlineto{\pgfqpoint{2.993321in}{2.290999in}}%
\pgfpathlineto{\pgfqpoint{2.979970in}{2.306149in}}%
\pgfpathlineto{\pgfqpoint{2.966616in}{2.321462in}}%
\pgfpathlineto{\pgfqpoint{2.958344in}{2.320241in}}%
\pgfpathlineto{\pgfqpoint{2.950060in}{2.319201in}}%
\pgfpathlineto{\pgfqpoint{2.941763in}{2.318344in}}%
\pgfpathlineto{\pgfqpoint{2.933454in}{2.317673in}}%
\pgfpathclose%
\pgfusepath{fill}%
\end{pgfscope}%
\begin{pgfscope}%
\pgfpathrectangle{\pgfqpoint{1.254980in}{0.150000in}}{\pgfqpoint{5.490039in}{5.490039in}}%
\pgfusepath{clip}%
\pgfsetbuttcap%
\pgfsetroundjoin%
\definecolor{currentfill}{rgb}{0.273809,0.031497,0.358853}%
\pgfsetfillcolor{currentfill}%
\pgfsetfillopacity{0.700000}%
\pgfsetlinewidth{0.000000pt}%
\definecolor{currentstroke}{rgb}{0.000000,0.000000,0.000000}%
\pgfsetstrokecolor{currentstroke}%
\pgfsetdash{}{0pt}%
\pgfpathmoveto{\pgfqpoint{4.136011in}{1.745289in}}%
\pgfpathlineto{\pgfqpoint{4.149402in}{1.741993in}}%
\pgfpathlineto{\pgfqpoint{4.162800in}{1.738819in}}%
\pgfpathlineto{\pgfqpoint{4.176205in}{1.735764in}}%
\pgfpathlineto{\pgfqpoint{4.189616in}{1.732830in}}%
\pgfpathlineto{\pgfqpoint{4.197302in}{1.742310in}}%
\pgfpathlineto{\pgfqpoint{4.204983in}{1.751833in}}%
\pgfpathlineto{\pgfqpoint{4.212659in}{1.761400in}}%
\pgfpathlineto{\pgfqpoint{4.220329in}{1.771007in}}%
\pgfpathlineto{\pgfqpoint{4.206928in}{1.773696in}}%
\pgfpathlineto{\pgfqpoint{4.193533in}{1.776506in}}%
\pgfpathlineto{\pgfqpoint{4.180146in}{1.779436in}}%
\pgfpathlineto{\pgfqpoint{4.166766in}{1.782487in}}%
\pgfpathlineto{\pgfqpoint{4.159085in}{1.773118in}}%
\pgfpathlineto{\pgfqpoint{4.151399in}{1.763794in}}%
\pgfpathlineto{\pgfqpoint{4.143708in}{1.754517in}}%
\pgfpathlineto{\pgfqpoint{4.136011in}{1.745289in}}%
\pgfpathclose%
\pgfusepath{fill}%
\end{pgfscope}%
\begin{pgfscope}%
\pgfpathrectangle{\pgfqpoint{1.254980in}{0.150000in}}{\pgfqpoint{5.490039in}{5.490039in}}%
\pgfusepath{clip}%
\pgfsetbuttcap%
\pgfsetroundjoin%
\definecolor{currentfill}{rgb}{0.227802,0.326594,0.546532}%
\pgfsetfillcolor{currentfill}%
\pgfsetfillopacity{0.700000}%
\pgfsetlinewidth{0.000000pt}%
\definecolor{currentstroke}{rgb}{0.000000,0.000000,0.000000}%
\pgfsetstrokecolor{currentstroke}%
\pgfsetdash{}{0pt}%
\pgfpathmoveto{\pgfqpoint{2.879859in}{2.382033in}}%
\pgfpathlineto{\pgfqpoint{2.893264in}{2.365692in}}%
\pgfpathlineto{\pgfqpoint{2.906665in}{2.349518in}}%
\pgfpathlineto{\pgfqpoint{2.920061in}{2.333512in}}%
\pgfpathlineto{\pgfqpoint{2.933454in}{2.317673in}}%
\pgfpathlineto{\pgfqpoint{2.941763in}{2.318344in}}%
\pgfpathlineto{\pgfqpoint{2.950060in}{2.319201in}}%
\pgfpathlineto{\pgfqpoint{2.958344in}{2.320241in}}%
\pgfpathlineto{\pgfqpoint{2.966616in}{2.321462in}}%
\pgfpathlineto{\pgfqpoint{2.953258in}{2.336940in}}%
\pgfpathlineto{\pgfqpoint{2.939897in}{2.352584in}}%
\pgfpathlineto{\pgfqpoint{2.926531in}{2.368395in}}%
\pgfpathlineto{\pgfqpoint{2.913162in}{2.384374in}}%
\pgfpathlineto{\pgfqpoint{2.904856in}{2.383509in}}%
\pgfpathlineto{\pgfqpoint{2.896537in}{2.382828in}}%
\pgfpathlineto{\pgfqpoint{2.888205in}{2.382335in}}%
\pgfpathlineto{\pgfqpoint{2.879859in}{2.382033in}}%
\pgfpathclose%
\pgfusepath{fill}%
\end{pgfscope}%
\begin{pgfscope}%
\pgfpathrectangle{\pgfqpoint{1.254980in}{0.150000in}}{\pgfqpoint{5.490039in}{5.490039in}}%
\pgfusepath{clip}%
\pgfsetbuttcap%
\pgfsetroundjoin%
\definecolor{currentfill}{rgb}{0.250425,0.274290,0.533103}%
\pgfsetfillcolor{currentfill}%
\pgfsetfillopacity{0.700000}%
\pgfsetlinewidth{0.000000pt}%
\definecolor{currentstroke}{rgb}{0.000000,0.000000,0.000000}%
\pgfsetstrokecolor{currentstroke}%
\pgfsetdash{}{0pt}%
\pgfpathmoveto{\pgfqpoint{2.986986in}{2.255959in}}%
\pgfpathlineto{\pgfqpoint{3.000361in}{2.240938in}}%
\pgfpathlineto{\pgfqpoint{3.013732in}{2.226077in}}%
\pgfpathlineto{\pgfqpoint{3.027100in}{2.211376in}}%
\pgfpathlineto{\pgfqpoint{3.040466in}{2.196835in}}%
\pgfpathlineto{\pgfqpoint{3.048707in}{2.198238in}}%
\pgfpathlineto{\pgfqpoint{3.056937in}{2.199817in}}%
\pgfpathlineto{\pgfqpoint{3.065154in}{2.201571in}}%
\pgfpathlineto{\pgfqpoint{3.073361in}{2.203496in}}%
\pgfpathlineto{\pgfqpoint{3.060028in}{2.217680in}}%
\pgfpathlineto{\pgfqpoint{3.046692in}{2.232023in}}%
\pgfpathlineto{\pgfqpoint{3.033354in}{2.246526in}}%
\pgfpathlineto{\pgfqpoint{3.020012in}{2.261189in}}%
\pgfpathlineto{\pgfqpoint{3.011774in}{2.259615in}}%
\pgfpathlineto{\pgfqpoint{3.003523in}{2.258217in}}%
\pgfpathlineto{\pgfqpoint{2.995261in}{2.256997in}}%
\pgfpathlineto{\pgfqpoint{2.986986in}{2.255959in}}%
\pgfpathclose%
\pgfusepath{fill}%
\end{pgfscope}%
\begin{pgfscope}%
\pgfpathrectangle{\pgfqpoint{1.254980in}{0.150000in}}{\pgfqpoint{5.490039in}{5.490039in}}%
\pgfusepath{clip}%
\pgfsetbuttcap%
\pgfsetroundjoin%
\definecolor{currentfill}{rgb}{0.214298,0.355619,0.551184}%
\pgfsetfillcolor{currentfill}%
\pgfsetfillopacity{0.700000}%
\pgfsetlinewidth{0.000000pt}%
\definecolor{currentstroke}{rgb}{0.000000,0.000000,0.000000}%
\pgfsetstrokecolor{currentstroke}%
\pgfsetdash{}{0pt}%
\pgfpathmoveto{\pgfqpoint{2.826193in}{2.449102in}}%
\pgfpathlineto{\pgfqpoint{2.839617in}{2.432077in}}%
\pgfpathlineto{\pgfqpoint{2.853036in}{2.415225in}}%
\pgfpathlineto{\pgfqpoint{2.866450in}{2.398544in}}%
\pgfpathlineto{\pgfqpoint{2.879859in}{2.382033in}}%
\pgfpathlineto{\pgfqpoint{2.888205in}{2.382335in}}%
\pgfpathlineto{\pgfqpoint{2.896537in}{2.382828in}}%
\pgfpathlineto{\pgfqpoint{2.904856in}{2.383509in}}%
\pgfpathlineto{\pgfqpoint{2.913162in}{2.384374in}}%
\pgfpathlineto{\pgfqpoint{2.899789in}{2.400521in}}%
\pgfpathlineto{\pgfqpoint{2.886411in}{2.416838in}}%
\pgfpathlineto{\pgfqpoint{2.873028in}{2.433325in}}%
\pgfpathlineto{\pgfqpoint{2.859641in}{2.449985in}}%
\pgfpathlineto{\pgfqpoint{2.851299in}{2.449477in}}%
\pgfpathlineto{\pgfqpoint{2.842944in}{2.449159in}}%
\pgfpathlineto{\pgfqpoint{2.834576in}{2.449033in}}%
\pgfpathlineto{\pgfqpoint{2.826193in}{2.449102in}}%
\pgfpathclose%
\pgfusepath{fill}%
\end{pgfscope}%
\begin{pgfscope}%
\pgfpathrectangle{\pgfqpoint{1.254980in}{0.150000in}}{\pgfqpoint{5.490039in}{5.490039in}}%
\pgfusepath{clip}%
\pgfsetbuttcap%
\pgfsetroundjoin%
\definecolor{currentfill}{rgb}{0.273809,0.031497,0.358853}%
\pgfsetfillcolor{currentfill}%
\pgfsetfillopacity{0.700000}%
\pgfsetlinewidth{0.000000pt}%
\definecolor{currentstroke}{rgb}{0.000000,0.000000,0.000000}%
\pgfsetstrokecolor{currentstroke}%
\pgfsetdash{}{0pt}%
\pgfpathmoveto{\pgfqpoint{3.775818in}{1.747410in}}%
\pgfpathlineto{\pgfqpoint{3.789144in}{1.740801in}}%
\pgfpathlineto{\pgfqpoint{3.802475in}{1.734319in}}%
\pgfpathlineto{\pgfqpoint{3.815809in}{1.727965in}}%
\pgfpathlineto{\pgfqpoint{3.829148in}{1.721737in}}%
\pgfpathlineto{\pgfqpoint{3.836969in}{1.729029in}}%
\pgfpathlineto{\pgfqpoint{3.844783in}{1.736410in}}%
\pgfpathlineto{\pgfqpoint{3.852591in}{1.743879in}}%
\pgfpathlineto{\pgfqpoint{3.860392in}{1.751434in}}%
\pgfpathlineto{\pgfqpoint{3.847070in}{1.757368in}}%
\pgfpathlineto{\pgfqpoint{3.833751in}{1.763429in}}%
\pgfpathlineto{\pgfqpoint{3.820437in}{1.769617in}}%
\pgfpathlineto{\pgfqpoint{3.807127in}{1.775933in}}%
\pgfpathlineto{\pgfqpoint{3.799310in}{1.768666in}}%
\pgfpathlineto{\pgfqpoint{3.791486in}{1.761488in}}%
\pgfpathlineto{\pgfqpoint{3.783655in}{1.754402in}}%
\pgfpathlineto{\pgfqpoint{3.775818in}{1.747410in}}%
\pgfpathclose%
\pgfusepath{fill}%
\end{pgfscope}%
\begin{pgfscope}%
\pgfpathrectangle{\pgfqpoint{1.254980in}{0.150000in}}{\pgfqpoint{5.490039in}{5.490039in}}%
\pgfusepath{clip}%
\pgfsetbuttcap%
\pgfsetroundjoin%
\definecolor{currentfill}{rgb}{0.271305,0.019942,0.347269}%
\pgfsetfillcolor{currentfill}%
\pgfsetfillopacity{0.700000}%
\pgfsetlinewidth{0.000000pt}%
\definecolor{currentstroke}{rgb}{0.000000,0.000000,0.000000}%
\pgfsetstrokecolor{currentstroke}%
\pgfsetdash{}{0pt}%
\pgfpathmoveto{\pgfqpoint{3.913730in}{1.728959in}}%
\pgfpathlineto{\pgfqpoint{3.927076in}{1.723653in}}%
\pgfpathlineto{\pgfqpoint{3.940428in}{1.718472in}}%
\pgfpathlineto{\pgfqpoint{3.953784in}{1.713416in}}%
\pgfpathlineto{\pgfqpoint{3.967146in}{1.708483in}}%
\pgfpathlineto{\pgfqpoint{3.974912in}{1.716687in}}%
\pgfpathlineto{\pgfqpoint{3.982672in}{1.724963in}}%
\pgfpathlineto{\pgfqpoint{3.990426in}{1.733310in}}%
\pgfpathlineto{\pgfqpoint{3.998175in}{1.741726in}}%
\pgfpathlineto{\pgfqpoint{3.984826in}{1.746382in}}%
\pgfpathlineto{\pgfqpoint{3.971483in}{1.751161in}}%
\pgfpathlineto{\pgfqpoint{3.958146in}{1.756065in}}%
\pgfpathlineto{\pgfqpoint{3.944814in}{1.761094in}}%
\pgfpathlineto{\pgfqpoint{3.937052in}{1.752949in}}%
\pgfpathlineto{\pgfqpoint{3.929284in}{1.744877in}}%
\pgfpathlineto{\pgfqpoint{3.921510in}{1.736879in}}%
\pgfpathlineto{\pgfqpoint{3.913730in}{1.728959in}}%
\pgfpathclose%
\pgfusepath{fill}%
\end{pgfscope}%
\begin{pgfscope}%
\pgfpathrectangle{\pgfqpoint{1.254980in}{0.150000in}}{\pgfqpoint{5.490039in}{5.490039in}}%
\pgfusepath{clip}%
\pgfsetbuttcap%
\pgfsetroundjoin%
\definecolor{currentfill}{rgb}{0.258965,0.251537,0.524736}%
\pgfsetfillcolor{currentfill}%
\pgfsetfillopacity{0.700000}%
\pgfsetlinewidth{0.000000pt}%
\definecolor{currentstroke}{rgb}{0.000000,0.000000,0.000000}%
\pgfsetstrokecolor{currentstroke}%
\pgfsetdash{}{0pt}%
\pgfpathmoveto{\pgfqpoint{3.040466in}{2.196835in}}%
\pgfpathlineto{\pgfqpoint{3.053829in}{2.182452in}}%
\pgfpathlineto{\pgfqpoint{3.067189in}{2.168226in}}%
\pgfpathlineto{\pgfqpoint{3.080547in}{2.154157in}}%
\pgfpathlineto{\pgfqpoint{3.093902in}{2.140244in}}%
\pgfpathlineto{\pgfqpoint{3.102111in}{2.142010in}}%
\pgfpathlineto{\pgfqpoint{3.110309in}{2.143948in}}%
\pgfpathlineto{\pgfqpoint{3.118495in}{2.146056in}}%
\pgfpathlineto{\pgfqpoint{3.126670in}{2.148330in}}%
\pgfpathlineto{\pgfqpoint{3.113346in}{2.161888in}}%
\pgfpathlineto{\pgfqpoint{3.100020in}{2.175601in}}%
\pgfpathlineto{\pgfqpoint{3.086691in}{2.189470in}}%
\pgfpathlineto{\pgfqpoint{3.073361in}{2.203496in}}%
\pgfpathlineto{\pgfqpoint{3.065154in}{2.201571in}}%
\pgfpathlineto{\pgfqpoint{3.056937in}{2.199817in}}%
\pgfpathlineto{\pgfqpoint{3.048707in}{2.198238in}}%
\pgfpathlineto{\pgfqpoint{3.040466in}{2.196835in}}%
\pgfpathclose%
\pgfusepath{fill}%
\end{pgfscope}%
\begin{pgfscope}%
\pgfpathrectangle{\pgfqpoint{1.254980in}{0.150000in}}{\pgfqpoint{5.490039in}{5.490039in}}%
\pgfusepath{clip}%
\pgfsetbuttcap%
\pgfsetroundjoin%
\definecolor{currentfill}{rgb}{0.201239,0.383670,0.554294}%
\pgfsetfillcolor{currentfill}%
\pgfsetfillopacity{0.700000}%
\pgfsetlinewidth{0.000000pt}%
\definecolor{currentstroke}{rgb}{0.000000,0.000000,0.000000}%
\pgfsetstrokecolor{currentstroke}%
\pgfsetdash{}{0pt}%
\pgfpathmoveto{\pgfqpoint{2.772446in}{2.518943in}}%
\pgfpathlineto{\pgfqpoint{2.785891in}{2.501219in}}%
\pgfpathlineto{\pgfqpoint{2.799330in}{2.483672in}}%
\pgfpathlineto{\pgfqpoint{2.812765in}{2.466300in}}%
\pgfpathlineto{\pgfqpoint{2.826193in}{2.449102in}}%
\pgfpathlineto{\pgfqpoint{2.834576in}{2.449033in}}%
\pgfpathlineto{\pgfqpoint{2.842944in}{2.449159in}}%
\pgfpathlineto{\pgfqpoint{2.851299in}{2.449477in}}%
\pgfpathlineto{\pgfqpoint{2.859641in}{2.449985in}}%
\pgfpathlineto{\pgfqpoint{2.846250in}{2.466817in}}%
\pgfpathlineto{\pgfqpoint{2.832853in}{2.483822in}}%
\pgfpathlineto{\pgfqpoint{2.819451in}{2.501002in}}%
\pgfpathlineto{\pgfqpoint{2.806044in}{2.518359in}}%
\pgfpathlineto{\pgfqpoint{2.797665in}{2.518211in}}%
\pgfpathlineto{\pgfqpoint{2.789273in}{2.518257in}}%
\pgfpathlineto{\pgfqpoint{2.780866in}{2.518500in}}%
\pgfpathlineto{\pgfqpoint{2.772446in}{2.518943in}}%
\pgfpathclose%
\pgfusepath{fill}%
\end{pgfscope}%
\begin{pgfscope}%
\pgfpathrectangle{\pgfqpoint{1.254980in}{0.150000in}}{\pgfqpoint{5.490039in}{5.490039in}}%
\pgfusepath{clip}%
\pgfsetbuttcap%
\pgfsetroundjoin%
\definecolor{currentfill}{rgb}{0.233603,0.313828,0.543914}%
\pgfsetfillcolor{currentfill}%
\pgfsetfillopacity{0.700000}%
\pgfsetlinewidth{0.000000pt}%
\definecolor{currentstroke}{rgb}{0.000000,0.000000,0.000000}%
\pgfsetstrokecolor{currentstroke}%
\pgfsetdash{}{0pt}%
\pgfpathmoveto{\pgfqpoint{5.148395in}{2.282378in}}%
\pgfpathlineto{\pgfqpoint{5.162187in}{2.286552in}}%
\pgfpathlineto{\pgfqpoint{5.175993in}{2.290840in}}%
\pgfpathlineto{\pgfqpoint{5.189811in}{2.295240in}}%
\pgfpathlineto{\pgfqpoint{5.203643in}{2.299754in}}%
\pgfpathlineto{\pgfqpoint{5.211017in}{2.310582in}}%
\pgfpathlineto{\pgfqpoint{5.218386in}{2.321355in}}%
\pgfpathlineto{\pgfqpoint{5.225749in}{2.332075in}}%
\pgfpathlineto{\pgfqpoint{5.233107in}{2.342739in}}%
\pgfpathlineto{\pgfqpoint{5.219280in}{2.338172in}}%
\pgfpathlineto{\pgfqpoint{5.205467in}{2.333717in}}%
\pgfpathlineto{\pgfqpoint{5.191666in}{2.329376in}}%
\pgfpathlineto{\pgfqpoint{5.177879in}{2.325148in}}%
\pgfpathlineto{\pgfqpoint{5.170517in}{2.314531in}}%
\pgfpathlineto{\pgfqpoint{5.163148in}{2.303864in}}%
\pgfpathlineto{\pgfqpoint{5.155775in}{2.293146in}}%
\pgfpathlineto{\pgfqpoint{5.148395in}{2.282378in}}%
\pgfpathclose%
\pgfusepath{fill}%
\end{pgfscope}%
\begin{pgfscope}%
\pgfpathrectangle{\pgfqpoint{1.254980in}{0.150000in}}{\pgfqpoint{5.490039in}{5.490039in}}%
\pgfusepath{clip}%
\pgfsetbuttcap%
\pgfsetroundjoin%
\definecolor{currentfill}{rgb}{0.267968,0.223549,0.512008}%
\pgfsetfillcolor{currentfill}%
\pgfsetfillopacity{0.700000}%
\pgfsetlinewidth{0.000000pt}%
\definecolor{currentstroke}{rgb}{0.000000,0.000000,0.000000}%
\pgfsetstrokecolor{currentstroke}%
\pgfsetdash{}{0pt}%
\pgfpathmoveto{\pgfqpoint{3.093902in}{2.140244in}}%
\pgfpathlineto{\pgfqpoint{3.107255in}{2.126486in}}%
\pgfpathlineto{\pgfqpoint{3.120607in}{2.112882in}}%
\pgfpathlineto{\pgfqpoint{3.133956in}{2.099432in}}%
\pgfpathlineto{\pgfqpoint{3.147304in}{2.086134in}}%
\pgfpathlineto{\pgfqpoint{3.155481in}{2.088261in}}%
\pgfpathlineto{\pgfqpoint{3.163648in}{2.090557in}}%
\pgfpathlineto{\pgfqpoint{3.171804in}{2.093017in}}%
\pgfpathlineto{\pgfqpoint{3.179949in}{2.095639in}}%
\pgfpathlineto{\pgfqpoint{3.166631in}{2.108583in}}%
\pgfpathlineto{\pgfqpoint{3.153313in}{2.121679in}}%
\pgfpathlineto{\pgfqpoint{3.139992in}{2.134928in}}%
\pgfpathlineto{\pgfqpoint{3.126670in}{2.148330in}}%
\pgfpathlineto{\pgfqpoint{3.118495in}{2.146056in}}%
\pgfpathlineto{\pgfqpoint{3.110309in}{2.143948in}}%
\pgfpathlineto{\pgfqpoint{3.102111in}{2.142010in}}%
\pgfpathlineto{\pgfqpoint{3.093902in}{2.140244in}}%
\pgfpathclose%
\pgfusepath{fill}%
\end{pgfscope}%
\begin{pgfscope}%
\pgfpathrectangle{\pgfqpoint{1.254980in}{0.150000in}}{\pgfqpoint{5.490039in}{5.490039in}}%
\pgfusepath{clip}%
\pgfsetbuttcap%
\pgfsetroundjoin%
\definecolor{currentfill}{rgb}{0.180629,0.429975,0.557282}%
\pgfsetfillcolor{currentfill}%
\pgfsetfillopacity{0.700000}%
\pgfsetlinewidth{0.000000pt}%
\definecolor{currentstroke}{rgb}{0.000000,0.000000,0.000000}%
\pgfsetstrokecolor{currentstroke}%
\pgfsetdash{}{0pt}%
\pgfpathmoveto{\pgfqpoint{5.516739in}{2.570211in}}%
\pgfpathlineto{\pgfqpoint{5.530725in}{2.576295in}}%
\pgfpathlineto{\pgfqpoint{5.544726in}{2.582491in}}%
\pgfpathlineto{\pgfqpoint{5.558742in}{2.588800in}}%
\pgfpathlineto{\pgfqpoint{5.572774in}{2.595222in}}%
\pgfpathlineto{\pgfqpoint{5.580001in}{2.604882in}}%
\pgfpathlineto{\pgfqpoint{5.587222in}{2.614474in}}%
\pgfpathlineto{\pgfqpoint{5.594437in}{2.624000in}}%
\pgfpathlineto{\pgfqpoint{5.601645in}{2.633459in}}%
\pgfpathlineto{\pgfqpoint{5.587621in}{2.627067in}}%
\pgfpathlineto{\pgfqpoint{5.573613in}{2.620788in}}%
\pgfpathlineto{\pgfqpoint{5.559620in}{2.614621in}}%
\pgfpathlineto{\pgfqpoint{5.545641in}{2.608567in}}%
\pgfpathlineto{\pgfqpoint{5.538425in}{2.599072in}}%
\pgfpathlineto{\pgfqpoint{5.531203in}{2.589514in}}%
\pgfpathlineto{\pgfqpoint{5.523974in}{2.579894in}}%
\pgfpathlineto{\pgfqpoint{5.516739in}{2.570211in}}%
\pgfpathclose%
\pgfusepath{fill}%
\end{pgfscope}%
\begin{pgfscope}%
\pgfpathrectangle{\pgfqpoint{1.254980in}{0.150000in}}{\pgfqpoint{5.490039in}{5.490039in}}%
\pgfusepath{clip}%
\pgfsetbuttcap%
\pgfsetroundjoin%
\definecolor{currentfill}{rgb}{0.277018,0.050344,0.375715}%
\pgfsetfillcolor{currentfill}%
\pgfsetfillopacity{0.700000}%
\pgfsetlinewidth{0.000000pt}%
\definecolor{currentstroke}{rgb}{0.000000,0.000000,0.000000}%
\pgfsetstrokecolor{currentstroke}%
\pgfsetdash{}{0pt}%
\pgfpathmoveto{\pgfqpoint{3.637769in}{1.780444in}}%
\pgfpathlineto{\pgfqpoint{3.651087in}{1.772486in}}%
\pgfpathlineto{\pgfqpoint{3.664407in}{1.764659in}}%
\pgfpathlineto{\pgfqpoint{3.677731in}{1.756963in}}%
\pgfpathlineto{\pgfqpoint{3.691058in}{1.749398in}}%
\pgfpathlineto{\pgfqpoint{3.698942in}{1.755672in}}%
\pgfpathlineto{\pgfqpoint{3.706818in}{1.762055in}}%
\pgfpathlineto{\pgfqpoint{3.714687in}{1.768544in}}%
\pgfpathlineto{\pgfqpoint{3.722549in}{1.775135in}}%
\pgfpathlineto{\pgfqpoint{3.709241in}{1.782390in}}%
\pgfpathlineto{\pgfqpoint{3.695936in}{1.789775in}}%
\pgfpathlineto{\pgfqpoint{3.682635in}{1.797291in}}%
\pgfpathlineto{\pgfqpoint{3.669336in}{1.804938in}}%
\pgfpathlineto{\pgfqpoint{3.661455in}{1.798651in}}%
\pgfpathlineto{\pgfqpoint{3.653567in}{1.792472in}}%
\pgfpathlineto{\pgfqpoint{3.645672in}{1.786402in}}%
\pgfpathlineto{\pgfqpoint{3.637769in}{1.780444in}}%
\pgfpathclose%
\pgfusepath{fill}%
\end{pgfscope}%
\begin{pgfscope}%
\pgfpathrectangle{\pgfqpoint{1.254980in}{0.150000in}}{\pgfqpoint{5.490039in}{5.490039in}}%
\pgfusepath{clip}%
\pgfsetbuttcap%
\pgfsetroundjoin%
\definecolor{currentfill}{rgb}{0.283197,0.115680,0.436115}%
\pgfsetfillcolor{currentfill}%
\pgfsetfillopacity{0.700000}%
\pgfsetlinewidth{0.000000pt}%
\definecolor{currentstroke}{rgb}{0.000000,0.000000,0.000000}%
\pgfsetstrokecolor{currentstroke}%
\pgfsetdash{}{0pt}%
\pgfpathmoveto{\pgfqpoint{4.527073in}{1.872185in}}%
\pgfpathlineto{\pgfqpoint{4.540591in}{1.872141in}}%
\pgfpathlineto{\pgfqpoint{4.554118in}{1.872214in}}%
\pgfpathlineto{\pgfqpoint{4.567655in}{1.872402in}}%
\pgfpathlineto{\pgfqpoint{4.581201in}{1.872707in}}%
\pgfpathlineto{\pgfqpoint{4.588770in}{1.883629in}}%
\pgfpathlineto{\pgfqpoint{4.596335in}{1.894551in}}%
\pgfpathlineto{\pgfqpoint{4.603895in}{1.905470in}}%
\pgfpathlineto{\pgfqpoint{4.611450in}{1.916385in}}%
\pgfpathlineto{\pgfqpoint{4.597910in}{1.915899in}}%
\pgfpathlineto{\pgfqpoint{4.584379in}{1.915528in}}%
\pgfpathlineto{\pgfqpoint{4.570858in}{1.915274in}}%
\pgfpathlineto{\pgfqpoint{4.557347in}{1.915136in}}%
\pgfpathlineto{\pgfqpoint{4.549785in}{1.904396in}}%
\pgfpathlineto{\pgfqpoint{4.542219in}{1.893657in}}%
\pgfpathlineto{\pgfqpoint{4.534649in}{1.882920in}}%
\pgfpathlineto{\pgfqpoint{4.527073in}{1.872185in}}%
\pgfpathclose%
\pgfusepath{fill}%
\end{pgfscope}%
\begin{pgfscope}%
\pgfpathrectangle{\pgfqpoint{1.254980in}{0.150000in}}{\pgfqpoint{5.490039in}{5.490039in}}%
\pgfusepath{clip}%
\pgfsetbuttcap%
\pgfsetroundjoin%
\definecolor{currentfill}{rgb}{0.282327,0.094955,0.417331}%
\pgfsetfillcolor{currentfill}%
\pgfsetfillopacity{0.700000}%
\pgfsetlinewidth{0.000000pt}%
\definecolor{currentstroke}{rgb}{0.000000,0.000000,0.000000}%
\pgfsetstrokecolor{currentstroke}%
\pgfsetdash{}{0pt}%
\pgfpathmoveto{\pgfqpoint{3.446183in}{1.867662in}}%
\pgfpathlineto{\pgfqpoint{3.459498in}{1.857758in}}%
\pgfpathlineto{\pgfqpoint{3.472815in}{1.847991in}}%
\pgfpathlineto{\pgfqpoint{3.486133in}{1.838361in}}%
\pgfpathlineto{\pgfqpoint{3.499453in}{1.828869in}}%
\pgfpathlineto{\pgfqpoint{3.507431in}{1.833683in}}%
\pgfpathlineto{\pgfqpoint{3.515401in}{1.838628in}}%
\pgfpathlineto{\pgfqpoint{3.523362in}{1.843702in}}%
\pgfpathlineto{\pgfqpoint{3.531315in}{1.848901in}}%
\pgfpathlineto{\pgfqpoint{3.518018in}{1.858064in}}%
\pgfpathlineto{\pgfqpoint{3.504723in}{1.867364in}}%
\pgfpathlineto{\pgfqpoint{3.491429in}{1.876801in}}%
\pgfpathlineto{\pgfqpoint{3.478137in}{1.886376in}}%
\pgfpathlineto{\pgfqpoint{3.470161in}{1.881500in}}%
\pgfpathlineto{\pgfqpoint{3.462177in}{1.876754in}}%
\pgfpathlineto{\pgfqpoint{3.454184in}{1.872141in}}%
\pgfpathlineto{\pgfqpoint{3.446183in}{1.867662in}}%
\pgfpathclose%
\pgfusepath{fill}%
\end{pgfscope}%
\begin{pgfscope}%
\pgfpathrectangle{\pgfqpoint{1.254980in}{0.150000in}}{\pgfqpoint{5.490039in}{5.490039in}}%
\pgfusepath{clip}%
\pgfsetbuttcap%
\pgfsetroundjoin%
\definecolor{currentfill}{rgb}{0.121148,0.592739,0.544641}%
\pgfsetfillcolor{currentfill}%
\pgfsetfillopacity{0.700000}%
\pgfsetlinewidth{0.000000pt}%
\definecolor{currentstroke}{rgb}{0.000000,0.000000,0.000000}%
\pgfsetstrokecolor{currentstroke}%
\pgfsetdash{}{0pt}%
\pgfpathmoveto{\pgfqpoint{2.427816in}{3.080452in}}%
\pgfpathlineto{\pgfqpoint{2.441441in}{3.057684in}}%
\pgfpathlineto{\pgfqpoint{2.455056in}{3.035128in}}%
\pgfpathlineto{\pgfqpoint{2.468660in}{3.012782in}}%
\pgfpathlineto{\pgfqpoint{2.482255in}{2.990645in}}%
\pgfpathlineto{\pgfqpoint{2.490865in}{2.988796in}}%
\pgfpathlineto{\pgfqpoint{2.499458in}{2.987164in}}%
\pgfpathlineto{\pgfqpoint{2.508036in}{2.985746in}}%
\pgfpathlineto{\pgfqpoint{2.516598in}{2.984539in}}%
\pgfpathlineto{\pgfqpoint{2.503047in}{3.006310in}}%
\pgfpathlineto{\pgfqpoint{2.489487in}{3.028288in}}%
\pgfpathlineto{\pgfqpoint{2.475917in}{3.050476in}}%
\pgfpathlineto{\pgfqpoint{2.462337in}{3.072874in}}%
\pgfpathlineto{\pgfqpoint{2.453731in}{3.074442in}}%
\pgfpathlineto{\pgfqpoint{2.445109in}{3.076225in}}%
\pgfpathlineto{\pgfqpoint{2.436471in}{3.078228in}}%
\pgfpathlineto{\pgfqpoint{2.427816in}{3.080452in}}%
\pgfpathclose%
\pgfusepath{fill}%
\end{pgfscope}%
\begin{pgfscope}%
\pgfpathrectangle{\pgfqpoint{1.254980in}{0.150000in}}{\pgfqpoint{5.490039in}{5.490039in}}%
\pgfusepath{clip}%
\pgfsetbuttcap%
\pgfsetroundjoin%
\definecolor{currentfill}{rgb}{0.282884,0.135920,0.453427}%
\pgfsetfillcolor{currentfill}%
\pgfsetfillopacity{0.700000}%
\pgfsetlinewidth{0.000000pt}%
\definecolor{currentstroke}{rgb}{0.000000,0.000000,0.000000}%
\pgfsetstrokecolor{currentstroke}%
\pgfsetdash{}{0pt}%
\pgfpathmoveto{\pgfqpoint{4.611450in}{1.916385in}}%
\pgfpathlineto{\pgfqpoint{4.625001in}{1.916987in}}%
\pgfpathlineto{\pgfqpoint{4.638561in}{1.917705in}}%
\pgfpathlineto{\pgfqpoint{4.652131in}{1.918538in}}%
\pgfpathlineto{\pgfqpoint{4.665711in}{1.919486in}}%
\pgfpathlineto{\pgfqpoint{4.673256in}{1.930569in}}%
\pgfpathlineto{\pgfqpoint{4.680797in}{1.941641in}}%
\pgfpathlineto{\pgfqpoint{4.688332in}{1.952702in}}%
\pgfpathlineto{\pgfqpoint{4.695863in}{1.963750in}}%
\pgfpathlineto{\pgfqpoint{4.682288in}{1.962635in}}%
\pgfpathlineto{\pgfqpoint{4.668723in}{1.961636in}}%
\pgfpathlineto{\pgfqpoint{4.655169in}{1.960752in}}%
\pgfpathlineto{\pgfqpoint{4.641624in}{1.959984in}}%
\pgfpathlineto{\pgfqpoint{4.634088in}{1.949096in}}%
\pgfpathlineto{\pgfqpoint{4.626547in}{1.938199in}}%
\pgfpathlineto{\pgfqpoint{4.619001in}{1.927295in}}%
\pgfpathlineto{\pgfqpoint{4.611450in}{1.916385in}}%
\pgfpathclose%
\pgfusepath{fill}%
\end{pgfscope}%
\begin{pgfscope}%
\pgfpathrectangle{\pgfqpoint{1.254980in}{0.150000in}}{\pgfqpoint{5.490039in}{5.490039in}}%
\pgfusepath{clip}%
\pgfsetbuttcap%
\pgfsetroundjoin%
\definecolor{currentfill}{rgb}{0.187231,0.414746,0.556547}%
\pgfsetfillcolor{currentfill}%
\pgfsetfillopacity{0.700000}%
\pgfsetlinewidth{0.000000pt}%
\definecolor{currentstroke}{rgb}{0.000000,0.000000,0.000000}%
\pgfsetstrokecolor{currentstroke}%
\pgfsetdash{}{0pt}%
\pgfpathmoveto{\pgfqpoint{2.718606in}{2.591625in}}%
\pgfpathlineto{\pgfqpoint{2.732075in}{2.573184in}}%
\pgfpathlineto{\pgfqpoint{2.745538in}{2.554924in}}%
\pgfpathlineto{\pgfqpoint{2.758995in}{2.536844in}}%
\pgfpathlineto{\pgfqpoint{2.772446in}{2.518943in}}%
\pgfpathlineto{\pgfqpoint{2.780866in}{2.518500in}}%
\pgfpathlineto{\pgfqpoint{2.789273in}{2.518257in}}%
\pgfpathlineto{\pgfqpoint{2.797665in}{2.518211in}}%
\pgfpathlineto{\pgfqpoint{2.806044in}{2.518359in}}%
\pgfpathlineto{\pgfqpoint{2.792631in}{2.535892in}}%
\pgfpathlineto{\pgfqpoint{2.779213in}{2.553603in}}%
\pgfpathlineto{\pgfqpoint{2.765789in}{2.571493in}}%
\pgfpathlineto{\pgfqpoint{2.752360in}{2.589563in}}%
\pgfpathlineto{\pgfqpoint{2.743943in}{2.589777in}}%
\pgfpathlineto{\pgfqpoint{2.735512in}{2.590190in}}%
\pgfpathlineto{\pgfqpoint{2.727066in}{2.590805in}}%
\pgfpathlineto{\pgfqpoint{2.718606in}{2.591625in}}%
\pgfpathclose%
\pgfusepath{fill}%
\end{pgfscope}%
\begin{pgfscope}%
\pgfpathrectangle{\pgfqpoint{1.254980in}{0.150000in}}{\pgfqpoint{5.490039in}{5.490039in}}%
\pgfusepath{clip}%
\pgfsetbuttcap%
\pgfsetroundjoin%
\definecolor{currentfill}{rgb}{0.282327,0.094955,0.417331}%
\pgfsetfillcolor{currentfill}%
\pgfsetfillopacity{0.700000}%
\pgfsetlinewidth{0.000000pt}%
\definecolor{currentstroke}{rgb}{0.000000,0.000000,0.000000}%
\pgfsetstrokecolor{currentstroke}%
\pgfsetdash{}{0pt}%
\pgfpathmoveto{\pgfqpoint{4.442718in}{1.831431in}}%
\pgfpathlineto{\pgfqpoint{4.456206in}{1.830723in}}%
\pgfpathlineto{\pgfqpoint{4.469703in}{1.830132in}}%
\pgfpathlineto{\pgfqpoint{4.483209in}{1.829658in}}%
\pgfpathlineto{\pgfqpoint{4.496724in}{1.829300in}}%
\pgfpathlineto{\pgfqpoint{4.504318in}{1.840010in}}%
\pgfpathlineto{\pgfqpoint{4.511908in}{1.850729in}}%
\pgfpathlineto{\pgfqpoint{4.519493in}{1.861454in}}%
\pgfpathlineto{\pgfqpoint{4.527073in}{1.872185in}}%
\pgfpathlineto{\pgfqpoint{4.513565in}{1.872345in}}%
\pgfpathlineto{\pgfqpoint{4.500065in}{1.872621in}}%
\pgfpathlineto{\pgfqpoint{4.486575in}{1.873015in}}%
\pgfpathlineto{\pgfqpoint{4.473094in}{1.873525in}}%
\pgfpathlineto{\pgfqpoint{4.465507in}{1.862986in}}%
\pgfpathlineto{\pgfqpoint{4.457915in}{1.852456in}}%
\pgfpathlineto{\pgfqpoint{4.450319in}{1.841937in}}%
\pgfpathlineto{\pgfqpoint{4.442718in}{1.831431in}}%
\pgfpathclose%
\pgfusepath{fill}%
\end{pgfscope}%
\begin{pgfscope}%
\pgfpathrectangle{\pgfqpoint{1.254980in}{0.150000in}}{\pgfqpoint{5.490039in}{5.490039in}}%
\pgfusepath{clip}%
\pgfsetbuttcap%
\pgfsetroundjoin%
\definecolor{currentfill}{rgb}{0.280255,0.165693,0.476498}%
\pgfsetfillcolor{currentfill}%
\pgfsetfillopacity{0.700000}%
\pgfsetlinewidth{0.000000pt}%
\definecolor{currentstroke}{rgb}{0.000000,0.000000,0.000000}%
\pgfsetstrokecolor{currentstroke}%
\pgfsetdash{}{0pt}%
\pgfpathmoveto{\pgfqpoint{4.695863in}{1.963750in}}%
\pgfpathlineto{\pgfqpoint{4.709449in}{1.964980in}}%
\pgfpathlineto{\pgfqpoint{4.723045in}{1.966325in}}%
\pgfpathlineto{\pgfqpoint{4.736651in}{1.967784in}}%
\pgfpathlineto{\pgfqpoint{4.750268in}{1.969359in}}%
\pgfpathlineto{\pgfqpoint{4.757789in}{1.980550in}}%
\pgfpathlineto{\pgfqpoint{4.765306in}{1.991722in}}%
\pgfpathlineto{\pgfqpoint{4.772817in}{2.002874in}}%
\pgfpathlineto{\pgfqpoint{4.780324in}{2.014006in}}%
\pgfpathlineto{\pgfqpoint{4.766711in}{2.012281in}}%
\pgfpathlineto{\pgfqpoint{4.753110in}{2.010671in}}%
\pgfpathlineto{\pgfqpoint{4.739519in}{2.009176in}}%
\pgfpathlineto{\pgfqpoint{4.725939in}{2.007796in}}%
\pgfpathlineto{\pgfqpoint{4.718427in}{1.996808in}}%
\pgfpathlineto{\pgfqpoint{4.710910in}{1.985804in}}%
\pgfpathlineto{\pgfqpoint{4.703389in}{1.974784in}}%
\pgfpathlineto{\pgfqpoint{4.695863in}{1.963750in}}%
\pgfpathclose%
\pgfusepath{fill}%
\end{pgfscope}%
\begin{pgfscope}%
\pgfpathrectangle{\pgfqpoint{1.254980in}{0.150000in}}{\pgfqpoint{5.490039in}{5.490039in}}%
\pgfusepath{clip}%
\pgfsetbuttcap%
\pgfsetroundjoin%
\definecolor{currentfill}{rgb}{0.272594,0.025563,0.353093}%
\pgfsetfillcolor{currentfill}%
\pgfsetfillopacity{0.700000}%
\pgfsetlinewidth{0.000000pt}%
\definecolor{currentstroke}{rgb}{0.000000,0.000000,0.000000}%
\pgfsetstrokecolor{currentstroke}%
\pgfsetdash{}{0pt}%
\pgfpathmoveto{\pgfqpoint{4.051625in}{1.724334in}}%
\pgfpathlineto{\pgfqpoint{4.065002in}{1.720292in}}%
\pgfpathlineto{\pgfqpoint{4.078386in}{1.716373in}}%
\pgfpathlineto{\pgfqpoint{4.091775in}{1.712574in}}%
\pgfpathlineto{\pgfqpoint{4.105171in}{1.708897in}}%
\pgfpathlineto{\pgfqpoint{4.112889in}{1.717913in}}%
\pgfpathlineto{\pgfqpoint{4.120602in}{1.726984in}}%
\pgfpathlineto{\pgfqpoint{4.128309in}{1.736110in}}%
\pgfpathlineto{\pgfqpoint{4.136011in}{1.745289in}}%
\pgfpathlineto{\pgfqpoint{4.122626in}{1.748705in}}%
\pgfpathlineto{\pgfqpoint{4.109248in}{1.752242in}}%
\pgfpathlineto{\pgfqpoint{4.095877in}{1.755902in}}%
\pgfpathlineto{\pgfqpoint{4.082511in}{1.759683in}}%
\pgfpathlineto{\pgfqpoint{4.074798in}{1.750759in}}%
\pgfpathlineto{\pgfqpoint{4.067079in}{1.741892in}}%
\pgfpathlineto{\pgfqpoint{4.059355in}{1.733083in}}%
\pgfpathlineto{\pgfqpoint{4.051625in}{1.724334in}}%
\pgfpathclose%
\pgfusepath{fill}%
\end{pgfscope}%
\begin{pgfscope}%
\pgfpathrectangle{\pgfqpoint{1.254980in}{0.150000in}}{\pgfqpoint{5.490039in}{5.490039in}}%
\pgfusepath{clip}%
\pgfsetbuttcap%
\pgfsetroundjoin%
\definecolor{currentfill}{rgb}{0.274128,0.199721,0.498911}%
\pgfsetfillcolor{currentfill}%
\pgfsetfillopacity{0.700000}%
\pgfsetlinewidth{0.000000pt}%
\definecolor{currentstroke}{rgb}{0.000000,0.000000,0.000000}%
\pgfsetstrokecolor{currentstroke}%
\pgfsetdash{}{0pt}%
\pgfpathmoveto{\pgfqpoint{3.147304in}{2.086134in}}%
\pgfpathlineto{\pgfqpoint{3.160650in}{2.072989in}}%
\pgfpathlineto{\pgfqpoint{3.173994in}{2.059995in}}%
\pgfpathlineto{\pgfqpoint{3.187337in}{2.047151in}}%
\pgfpathlineto{\pgfqpoint{3.200679in}{2.034457in}}%
\pgfpathlineto{\pgfqpoint{3.208827in}{2.036943in}}%
\pgfpathlineto{\pgfqpoint{3.216964in}{2.039593in}}%
\pgfpathlineto{\pgfqpoint{3.225090in}{2.042404in}}%
\pgfpathlineto{\pgfqpoint{3.233206in}{2.045373in}}%
\pgfpathlineto{\pgfqpoint{3.219893in}{2.057715in}}%
\pgfpathlineto{\pgfqpoint{3.206580in}{2.070206in}}%
\pgfpathlineto{\pgfqpoint{3.193265in}{2.082847in}}%
\pgfpathlineto{\pgfqpoint{3.179949in}{2.095639in}}%
\pgfpathlineto{\pgfqpoint{3.171804in}{2.093017in}}%
\pgfpathlineto{\pgfqpoint{3.163648in}{2.090557in}}%
\pgfpathlineto{\pgfqpoint{3.155481in}{2.088261in}}%
\pgfpathlineto{\pgfqpoint{3.147304in}{2.086134in}}%
\pgfpathclose%
\pgfusepath{fill}%
\end{pgfscope}%
\begin{pgfscope}%
\pgfpathrectangle{\pgfqpoint{1.254980in}{0.150000in}}{\pgfqpoint{5.490039in}{5.490039in}}%
\pgfusepath{clip}%
\pgfsetbuttcap%
\pgfsetroundjoin%
\definecolor{currentfill}{rgb}{0.280267,0.073417,0.397163}%
\pgfsetfillcolor{currentfill}%
\pgfsetfillopacity{0.700000}%
\pgfsetlinewidth{0.000000pt}%
\definecolor{currentstroke}{rgb}{0.000000,0.000000,0.000000}%
\pgfsetstrokecolor{currentstroke}%
\pgfsetdash{}{0pt}%
\pgfpathmoveto{\pgfqpoint{4.358369in}{1.794417in}}%
\pgfpathlineto{\pgfqpoint{4.371830in}{1.793026in}}%
\pgfpathlineto{\pgfqpoint{4.385300in}{1.791753in}}%
\pgfpathlineto{\pgfqpoint{4.398778in}{1.790598in}}%
\pgfpathlineto{\pgfqpoint{4.412265in}{1.789559in}}%
\pgfpathlineto{\pgfqpoint{4.419886in}{1.800001in}}%
\pgfpathlineto{\pgfqpoint{4.427501in}{1.810461in}}%
\pgfpathlineto{\pgfqpoint{4.435112in}{1.820939in}}%
\pgfpathlineto{\pgfqpoint{4.442718in}{1.831431in}}%
\pgfpathlineto{\pgfqpoint{4.429238in}{1.832256in}}%
\pgfpathlineto{\pgfqpoint{4.415767in}{1.833198in}}%
\pgfpathlineto{\pgfqpoint{4.402305in}{1.834258in}}%
\pgfpathlineto{\pgfqpoint{4.388851in}{1.835436in}}%
\pgfpathlineto{\pgfqpoint{4.381238in}{1.825150in}}%
\pgfpathlineto{\pgfqpoint{4.373620in}{1.814884in}}%
\pgfpathlineto{\pgfqpoint{4.365996in}{1.804639in}}%
\pgfpathlineto{\pgfqpoint{4.358369in}{1.794417in}}%
\pgfpathclose%
\pgfusepath{fill}%
\end{pgfscope}%
\begin{pgfscope}%
\pgfpathrectangle{\pgfqpoint{1.254980in}{0.150000in}}{\pgfqpoint{5.490039in}{5.490039in}}%
\pgfusepath{clip}%
\pgfsetbuttcap%
\pgfsetroundjoin%
\definecolor{currentfill}{rgb}{0.276194,0.190074,0.493001}%
\pgfsetfillcolor{currentfill}%
\pgfsetfillopacity{0.700000}%
\pgfsetlinewidth{0.000000pt}%
\definecolor{currentstroke}{rgb}{0.000000,0.000000,0.000000}%
\pgfsetstrokecolor{currentstroke}%
\pgfsetdash{}{0pt}%
\pgfpathmoveto{\pgfqpoint{4.780324in}{2.014006in}}%
\pgfpathlineto{\pgfqpoint{4.793947in}{2.015846in}}%
\pgfpathlineto{\pgfqpoint{4.807581in}{2.017800in}}%
\pgfpathlineto{\pgfqpoint{4.821226in}{2.019869in}}%
\pgfpathlineto{\pgfqpoint{4.834883in}{2.022052in}}%
\pgfpathlineto{\pgfqpoint{4.842380in}{2.033302in}}%
\pgfpathlineto{\pgfqpoint{4.849872in}{2.044526in}}%
\pgfpathlineto{\pgfqpoint{4.857360in}{2.055722in}}%
\pgfpathlineto{\pgfqpoint{4.864842in}{2.066890in}}%
\pgfpathlineto{\pgfqpoint{4.851190in}{2.064573in}}%
\pgfpathlineto{\pgfqpoint{4.837549in}{2.062369in}}%
\pgfpathlineto{\pgfqpoint{4.823920in}{2.060281in}}%
\pgfpathlineto{\pgfqpoint{4.810301in}{2.058307in}}%
\pgfpathlineto{\pgfqpoint{4.802814in}{2.047267in}}%
\pgfpathlineto{\pgfqpoint{4.795322in}{2.036203in}}%
\pgfpathlineto{\pgfqpoint{4.787825in}{2.025116in}}%
\pgfpathlineto{\pgfqpoint{4.780324in}{2.014006in}}%
\pgfpathclose%
\pgfusepath{fill}%
\end{pgfscope}%
\begin{pgfscope}%
\pgfpathrectangle{\pgfqpoint{1.254980in}{0.150000in}}{\pgfqpoint{5.490039in}{5.490039in}}%
\pgfusepath{clip}%
\pgfsetbuttcap%
\pgfsetroundjoin%
\definecolor{currentfill}{rgb}{0.220057,0.343307,0.549413}%
\pgfsetfillcolor{currentfill}%
\pgfsetfillopacity{0.700000}%
\pgfsetlinewidth{0.000000pt}%
\definecolor{currentstroke}{rgb}{0.000000,0.000000,0.000000}%
\pgfsetstrokecolor{currentstroke}%
\pgfsetdash{}{0pt}%
\pgfpathmoveto{\pgfqpoint{5.233107in}{2.342739in}}%
\pgfpathlineto{\pgfqpoint{5.246947in}{2.347420in}}%
\pgfpathlineto{\pgfqpoint{5.260801in}{2.352214in}}%
\pgfpathlineto{\pgfqpoint{5.274668in}{2.357121in}}%
\pgfpathlineto{\pgfqpoint{5.288550in}{2.362140in}}%
\pgfpathlineto{\pgfqpoint{5.295896in}{2.372794in}}%
\pgfpathlineto{\pgfqpoint{5.303237in}{2.383389in}}%
\pgfpathlineto{\pgfqpoint{5.310571in}{2.393925in}}%
\pgfpathlineto{\pgfqpoint{5.317900in}{2.404403in}}%
\pgfpathlineto{\pgfqpoint{5.304024in}{2.399346in}}%
\pgfpathlineto{\pgfqpoint{5.290162in}{2.394402in}}%
\pgfpathlineto{\pgfqpoint{5.276314in}{2.389571in}}%
\pgfpathlineto{\pgfqpoint{5.262479in}{2.384853in}}%
\pgfpathlineto{\pgfqpoint{5.255144in}{2.374406in}}%
\pgfpathlineto{\pgfqpoint{5.247804in}{2.363905in}}%
\pgfpathlineto{\pgfqpoint{5.240459in}{2.353350in}}%
\pgfpathlineto{\pgfqpoint{5.233107in}{2.342739in}}%
\pgfpathclose%
\pgfusepath{fill}%
\end{pgfscope}%
\begin{pgfscope}%
\pgfpathrectangle{\pgfqpoint{1.254980in}{0.150000in}}{\pgfqpoint{5.490039in}{5.490039in}}%
\pgfusepath{clip}%
\pgfsetbuttcap%
\pgfsetroundjoin%
\definecolor{currentfill}{rgb}{0.175841,0.441290,0.557685}%
\pgfsetfillcolor{currentfill}%
\pgfsetfillopacity{0.700000}%
\pgfsetlinewidth{0.000000pt}%
\definecolor{currentstroke}{rgb}{0.000000,0.000000,0.000000}%
\pgfsetstrokecolor{currentstroke}%
\pgfsetdash{}{0pt}%
\pgfpathmoveto{\pgfqpoint{2.664665in}{2.667218in}}%
\pgfpathlineto{\pgfqpoint{2.678160in}{2.648042in}}%
\pgfpathlineto{\pgfqpoint{2.691649in}{2.629053in}}%
\pgfpathlineto{\pgfqpoint{2.705131in}{2.610247in}}%
\pgfpathlineto{\pgfqpoint{2.718606in}{2.591625in}}%
\pgfpathlineto{\pgfqpoint{2.727066in}{2.590805in}}%
\pgfpathlineto{\pgfqpoint{2.735512in}{2.590190in}}%
\pgfpathlineto{\pgfqpoint{2.743943in}{2.589777in}}%
\pgfpathlineto{\pgfqpoint{2.752360in}{2.589563in}}%
\pgfpathlineto{\pgfqpoint{2.738924in}{2.607814in}}%
\pgfpathlineto{\pgfqpoint{2.725482in}{2.626248in}}%
\pgfpathlineto{\pgfqpoint{2.712033in}{2.644865in}}%
\pgfpathlineto{\pgfqpoint{2.698578in}{2.663668in}}%
\pgfpathlineto{\pgfqpoint{2.690122in}{2.664247in}}%
\pgfpathlineto{\pgfqpoint{2.681651in}{2.665030in}}%
\pgfpathlineto{\pgfqpoint{2.673165in}{2.666019in}}%
\pgfpathlineto{\pgfqpoint{2.664665in}{2.667218in}}%
\pgfpathclose%
\pgfusepath{fill}%
\end{pgfscope}%
\begin{pgfscope}%
\pgfpathrectangle{\pgfqpoint{1.254980in}{0.150000in}}{\pgfqpoint{5.490039in}{5.490039in}}%
\pgfusepath{clip}%
\pgfsetbuttcap%
\pgfsetroundjoin%
\definecolor{currentfill}{rgb}{0.169646,0.456262,0.558030}%
\pgfsetfillcolor{currentfill}%
\pgfsetfillopacity{0.700000}%
\pgfsetlinewidth{0.000000pt}%
\definecolor{currentstroke}{rgb}{0.000000,0.000000,0.000000}%
\pgfsetstrokecolor{currentstroke}%
\pgfsetdash{}{0pt}%
\pgfpathmoveto{\pgfqpoint{5.601645in}{2.633459in}}%
\pgfpathlineto{\pgfqpoint{5.615683in}{2.639963in}}%
\pgfpathlineto{\pgfqpoint{5.629737in}{2.646580in}}%
\pgfpathlineto{\pgfqpoint{5.643807in}{2.653309in}}%
\pgfpathlineto{\pgfqpoint{5.657892in}{2.660151in}}%
\pgfpathlineto{\pgfqpoint{5.665085in}{2.669504in}}%
\pgfpathlineto{\pgfqpoint{5.672271in}{2.678788in}}%
\pgfpathlineto{\pgfqpoint{5.679450in}{2.688003in}}%
\pgfpathlineto{\pgfqpoint{5.686623in}{2.697150in}}%
\pgfpathlineto{\pgfqpoint{5.672546in}{2.690356in}}%
\pgfpathlineto{\pgfqpoint{5.658485in}{2.683673in}}%
\pgfpathlineto{\pgfqpoint{5.644440in}{2.677103in}}%
\pgfpathlineto{\pgfqpoint{5.630409in}{2.670645in}}%
\pgfpathlineto{\pgfqpoint{5.623228in}{2.661445in}}%
\pgfpathlineto{\pgfqpoint{5.616040in}{2.652181in}}%
\pgfpathlineto{\pgfqpoint{5.608846in}{2.642853in}}%
\pgfpathlineto{\pgfqpoint{5.601645in}{2.633459in}}%
\pgfpathclose%
\pgfusepath{fill}%
\end{pgfscope}%
\begin{pgfscope}%
\pgfpathrectangle{\pgfqpoint{1.254980in}{0.150000in}}{\pgfqpoint{5.490039in}{5.490039in}}%
\pgfusepath{clip}%
\pgfsetbuttcap%
\pgfsetroundjoin%
\definecolor{currentfill}{rgb}{0.278012,0.180367,0.486697}%
\pgfsetfillcolor{currentfill}%
\pgfsetfillopacity{0.700000}%
\pgfsetlinewidth{0.000000pt}%
\definecolor{currentstroke}{rgb}{0.000000,0.000000,0.000000}%
\pgfsetstrokecolor{currentstroke}%
\pgfsetdash{}{0pt}%
\pgfpathmoveto{\pgfqpoint{3.200679in}{2.034457in}}%
\pgfpathlineto{\pgfqpoint{3.214020in}{2.021912in}}%
\pgfpathlineto{\pgfqpoint{3.227360in}{2.009515in}}%
\pgfpathlineto{\pgfqpoint{3.240699in}{1.997266in}}%
\pgfpathlineto{\pgfqpoint{3.254037in}{1.985163in}}%
\pgfpathlineto{\pgfqpoint{3.262155in}{1.988008in}}%
\pgfpathlineto{\pgfqpoint{3.270264in}{1.991012in}}%
\pgfpathlineto{\pgfqpoint{3.278362in}{1.994172in}}%
\pgfpathlineto{\pgfqpoint{3.286450in}{1.997485in}}%
\pgfpathlineto{\pgfqpoint{3.273140in}{2.009237in}}%
\pgfpathlineto{\pgfqpoint{3.259829in}{2.021135in}}%
\pgfpathlineto{\pgfqpoint{3.246518in}{2.033180in}}%
\pgfpathlineto{\pgfqpoint{3.233206in}{2.045373in}}%
\pgfpathlineto{\pgfqpoint{3.225090in}{2.042404in}}%
\pgfpathlineto{\pgfqpoint{3.216964in}{2.039593in}}%
\pgfpathlineto{\pgfqpoint{3.208827in}{2.036943in}}%
\pgfpathlineto{\pgfqpoint{3.200679in}{2.034457in}}%
\pgfpathclose%
\pgfusepath{fill}%
\end{pgfscope}%
\begin{pgfscope}%
\pgfpathrectangle{\pgfqpoint{1.254980in}{0.150000in}}{\pgfqpoint{5.490039in}{5.490039in}}%
\pgfusepath{clip}%
\pgfsetbuttcap%
\pgfsetroundjoin%
\definecolor{currentfill}{rgb}{0.277941,0.056324,0.381191}%
\pgfsetfillcolor{currentfill}%
\pgfsetfillopacity{0.700000}%
\pgfsetlinewidth{0.000000pt}%
\definecolor{currentstroke}{rgb}{0.000000,0.000000,0.000000}%
\pgfsetstrokecolor{currentstroke}%
\pgfsetdash{}{0pt}%
\pgfpathmoveto{\pgfqpoint{4.274008in}{1.761445in}}%
\pgfpathlineto{\pgfqpoint{4.287446in}{1.759353in}}%
\pgfpathlineto{\pgfqpoint{4.300892in}{1.757378in}}%
\pgfpathlineto{\pgfqpoint{4.314346in}{1.755523in}}%
\pgfpathlineto{\pgfqpoint{4.327808in}{1.753785in}}%
\pgfpathlineto{\pgfqpoint{4.335455in}{1.763901in}}%
\pgfpathlineto{\pgfqpoint{4.343098in}{1.774046in}}%
\pgfpathlineto{\pgfqpoint{4.350736in}{1.784218in}}%
\pgfpathlineto{\pgfqpoint{4.358369in}{1.794417in}}%
\pgfpathlineto{\pgfqpoint{4.344915in}{1.795925in}}%
\pgfpathlineto{\pgfqpoint{4.331470in}{1.797552in}}%
\pgfpathlineto{\pgfqpoint{4.318032in}{1.799297in}}%
\pgfpathlineto{\pgfqpoint{4.304603in}{1.801161in}}%
\pgfpathlineto{\pgfqpoint{4.296961in}{1.791186in}}%
\pgfpathlineto{\pgfqpoint{4.289315in}{1.781240in}}%
\pgfpathlineto{\pgfqpoint{4.281664in}{1.771326in}}%
\pgfpathlineto{\pgfqpoint{4.274008in}{1.761445in}}%
\pgfpathclose%
\pgfusepath{fill}%
\end{pgfscope}%
\begin{pgfscope}%
\pgfpathrectangle{\pgfqpoint{1.254980in}{0.150000in}}{\pgfqpoint{5.490039in}{5.490039in}}%
\pgfusepath{clip}%
\pgfsetbuttcap%
\pgfsetroundjoin%
\definecolor{currentfill}{rgb}{0.269308,0.218818,0.509577}%
\pgfsetfillcolor{currentfill}%
\pgfsetfillopacity{0.700000}%
\pgfsetlinewidth{0.000000pt}%
\definecolor{currentstroke}{rgb}{0.000000,0.000000,0.000000}%
\pgfsetstrokecolor{currentstroke}%
\pgfsetdash{}{0pt}%
\pgfpathmoveto{\pgfqpoint{4.864842in}{2.066890in}}%
\pgfpathlineto{\pgfqpoint{4.878505in}{2.069322in}}%
\pgfpathlineto{\pgfqpoint{4.892180in}{2.071868in}}%
\pgfpathlineto{\pgfqpoint{4.905867in}{2.074528in}}%
\pgfpathlineto{\pgfqpoint{4.919565in}{2.077302in}}%
\pgfpathlineto{\pgfqpoint{4.927038in}{2.088565in}}%
\pgfpathlineto{\pgfqpoint{4.934506in}{2.099795in}}%
\pgfpathlineto{\pgfqpoint{4.941969in}{2.110989in}}%
\pgfpathlineto{\pgfqpoint{4.949427in}{2.122148in}}%
\pgfpathlineto{\pgfqpoint{4.935733in}{2.119255in}}%
\pgfpathlineto{\pgfqpoint{4.922051in}{2.116477in}}%
\pgfpathlineto{\pgfqpoint{4.908380in}{2.113812in}}%
\pgfpathlineto{\pgfqpoint{4.894721in}{2.111262in}}%
\pgfpathlineto{\pgfqpoint{4.887259in}{2.100215in}}%
\pgfpathlineto{\pgfqpoint{4.879792in}{2.089138in}}%
\pgfpathlineto{\pgfqpoint{4.872319in}{2.078029in}}%
\pgfpathlineto{\pgfqpoint{4.864842in}{2.066890in}}%
\pgfpathclose%
\pgfusepath{fill}%
\end{pgfscope}%
\begin{pgfscope}%
\pgfpathrectangle{\pgfqpoint{1.254980in}{0.150000in}}{\pgfqpoint{5.490039in}{5.490039in}}%
\pgfusepath{clip}%
\pgfsetbuttcap%
\pgfsetroundjoin%
\definecolor{currentfill}{rgb}{0.272594,0.025563,0.353093}%
\pgfsetfillcolor{currentfill}%
\pgfsetfillopacity{0.700000}%
\pgfsetlinewidth{0.000000pt}%
\definecolor{currentstroke}{rgb}{0.000000,0.000000,0.000000}%
\pgfsetstrokecolor{currentstroke}%
\pgfsetdash{}{0pt}%
\pgfpathmoveto{\pgfqpoint{3.829148in}{1.721737in}}%
\pgfpathlineto{\pgfqpoint{3.842491in}{1.715637in}}%
\pgfpathlineto{\pgfqpoint{3.855839in}{1.709662in}}%
\pgfpathlineto{\pgfqpoint{3.869191in}{1.703813in}}%
\pgfpathlineto{\pgfqpoint{3.882548in}{1.698089in}}%
\pgfpathlineto{\pgfqpoint{3.890352in}{1.705680in}}%
\pgfpathlineto{\pgfqpoint{3.898151in}{1.713357in}}%
\pgfpathlineto{\pgfqpoint{3.905943in}{1.721117in}}%
\pgfpathlineto{\pgfqpoint{3.913730in}{1.728959in}}%
\pgfpathlineto{\pgfqpoint{3.900388in}{1.734389in}}%
\pgfpathlineto{\pgfqpoint{3.887051in}{1.739945in}}%
\pgfpathlineto{\pgfqpoint{3.873720in}{1.745626in}}%
\pgfpathlineto{\pgfqpoint{3.860392in}{1.751434in}}%
\pgfpathlineto{\pgfqpoint{3.852591in}{1.743879in}}%
\pgfpathlineto{\pgfqpoint{3.844783in}{1.736410in}}%
\pgfpathlineto{\pgfqpoint{3.836969in}{1.729029in}}%
\pgfpathlineto{\pgfqpoint{3.829148in}{1.721737in}}%
\pgfpathclose%
\pgfusepath{fill}%
\end{pgfscope}%
\begin{pgfscope}%
\pgfpathrectangle{\pgfqpoint{1.254980in}{0.150000in}}{\pgfqpoint{5.490039in}{5.490039in}}%
\pgfusepath{clip}%
\pgfsetbuttcap%
\pgfsetroundjoin%
\definecolor{currentfill}{rgb}{0.280894,0.078907,0.402329}%
\pgfsetfillcolor{currentfill}%
\pgfsetfillopacity{0.700000}%
\pgfsetlinewidth{0.000000pt}%
\definecolor{currentstroke}{rgb}{0.000000,0.000000,0.000000}%
\pgfsetstrokecolor{currentstroke}%
\pgfsetdash{}{0pt}%
\pgfpathmoveto{\pgfqpoint{3.499453in}{1.828869in}}%
\pgfpathlineto{\pgfqpoint{3.512775in}{1.819512in}}%
\pgfpathlineto{\pgfqpoint{3.526098in}{1.810291in}}%
\pgfpathlineto{\pgfqpoint{3.539423in}{1.801205in}}%
\pgfpathlineto{\pgfqpoint{3.552750in}{1.792254in}}%
\pgfpathlineto{\pgfqpoint{3.560705in}{1.797403in}}%
\pgfpathlineto{\pgfqpoint{3.568653in}{1.802680in}}%
\pgfpathlineto{\pgfqpoint{3.576592in}{1.808080in}}%
\pgfpathlineto{\pgfqpoint{3.584523in}{1.813601in}}%
\pgfpathlineto{\pgfqpoint{3.571218in}{1.822224in}}%
\pgfpathlineto{\pgfqpoint{3.557915in}{1.830981in}}%
\pgfpathlineto{\pgfqpoint{3.544614in}{1.839873in}}%
\pgfpathlineto{\pgfqpoint{3.531315in}{1.848901in}}%
\pgfpathlineto{\pgfqpoint{3.523362in}{1.843702in}}%
\pgfpathlineto{\pgfqpoint{3.515401in}{1.838628in}}%
\pgfpathlineto{\pgfqpoint{3.507431in}{1.833683in}}%
\pgfpathlineto{\pgfqpoint{3.499453in}{1.828869in}}%
\pgfpathclose%
\pgfusepath{fill}%
\end{pgfscope}%
\begin{pgfscope}%
\pgfpathrectangle{\pgfqpoint{1.254980in}{0.150000in}}{\pgfqpoint{5.490039in}{5.490039in}}%
\pgfusepath{clip}%
\pgfsetbuttcap%
\pgfsetroundjoin%
\definecolor{currentfill}{rgb}{0.162142,0.474838,0.558140}%
\pgfsetfillcolor{currentfill}%
\pgfsetfillopacity{0.700000}%
\pgfsetlinewidth{0.000000pt}%
\definecolor{currentstroke}{rgb}{0.000000,0.000000,0.000000}%
\pgfsetstrokecolor{currentstroke}%
\pgfsetdash{}{0pt}%
\pgfpathmoveto{\pgfqpoint{2.610610in}{2.745798in}}%
\pgfpathlineto{\pgfqpoint{2.624135in}{2.725868in}}%
\pgfpathlineto{\pgfqpoint{2.637652in}{2.706129in}}%
\pgfpathlineto{\pgfqpoint{2.651162in}{2.686580in}}%
\pgfpathlineto{\pgfqpoint{2.664665in}{2.667218in}}%
\pgfpathlineto{\pgfqpoint{2.673165in}{2.666019in}}%
\pgfpathlineto{\pgfqpoint{2.681651in}{2.665030in}}%
\pgfpathlineto{\pgfqpoint{2.690122in}{2.664247in}}%
\pgfpathlineto{\pgfqpoint{2.698578in}{2.663668in}}%
\pgfpathlineto{\pgfqpoint{2.685116in}{2.682656in}}%
\pgfpathlineto{\pgfqpoint{2.671647in}{2.701831in}}%
\pgfpathlineto{\pgfqpoint{2.658171in}{2.721195in}}%
\pgfpathlineto{\pgfqpoint{2.644688in}{2.740748in}}%
\pgfpathlineto{\pgfqpoint{2.636191in}{2.741695in}}%
\pgfpathlineto{\pgfqpoint{2.627679in}{2.742850in}}%
\pgfpathlineto{\pgfqpoint{2.619152in}{2.744217in}}%
\pgfpathlineto{\pgfqpoint{2.610610in}{2.745798in}}%
\pgfpathclose%
\pgfusepath{fill}%
\end{pgfscope}%
\begin{pgfscope}%
\pgfpathrectangle{\pgfqpoint{1.254980in}{0.150000in}}{\pgfqpoint{5.490039in}{5.490039in}}%
\pgfusepath{clip}%
\pgfsetbuttcap%
\pgfsetroundjoin%
\definecolor{currentfill}{rgb}{0.206756,0.371758,0.553117}%
\pgfsetfillcolor{currentfill}%
\pgfsetfillopacity{0.700000}%
\pgfsetlinewidth{0.000000pt}%
\definecolor{currentstroke}{rgb}{0.000000,0.000000,0.000000}%
\pgfsetstrokecolor{currentstroke}%
\pgfsetdash{}{0pt}%
\pgfpathmoveto{\pgfqpoint{5.317900in}{2.404403in}}%
\pgfpathlineto{\pgfqpoint{5.331790in}{2.409573in}}%
\pgfpathlineto{\pgfqpoint{5.345694in}{2.414856in}}%
\pgfpathlineto{\pgfqpoint{5.359612in}{2.420251in}}%
\pgfpathlineto{\pgfqpoint{5.373544in}{2.425760in}}%
\pgfpathlineto{\pgfqpoint{5.380861in}{2.436205in}}%
\pgfpathlineto{\pgfqpoint{5.388172in}{2.446588in}}%
\pgfpathlineto{\pgfqpoint{5.395477in}{2.456908in}}%
\pgfpathlineto{\pgfqpoint{5.402776in}{2.467166in}}%
\pgfpathlineto{\pgfqpoint{5.388849in}{2.461637in}}%
\pgfpathlineto{\pgfqpoint{5.374937in}{2.456220in}}%
\pgfpathlineto{\pgfqpoint{5.361039in}{2.450917in}}%
\pgfpathlineto{\pgfqpoint{5.347155in}{2.445726in}}%
\pgfpathlineto{\pgfqpoint{5.339850in}{2.435483in}}%
\pgfpathlineto{\pgfqpoint{5.332540in}{2.425182in}}%
\pgfpathlineto{\pgfqpoint{5.325223in}{2.414822in}}%
\pgfpathlineto{\pgfqpoint{5.317900in}{2.404403in}}%
\pgfpathclose%
\pgfusepath{fill}%
\end{pgfscope}%
\begin{pgfscope}%
\pgfpathrectangle{\pgfqpoint{1.254980in}{0.150000in}}{\pgfqpoint{5.490039in}{5.490039in}}%
\pgfusepath{clip}%
\pgfsetbuttcap%
\pgfsetroundjoin%
\definecolor{currentfill}{rgb}{0.121380,0.629492,0.531973}%
\pgfsetfillcolor{currentfill}%
\pgfsetfillopacity{0.700000}%
\pgfsetlinewidth{0.000000pt}%
\definecolor{currentstroke}{rgb}{0.000000,0.000000,0.000000}%
\pgfsetstrokecolor{currentstroke}%
\pgfsetdash{}{0pt}%
\pgfpathmoveto{\pgfqpoint{2.373209in}{3.173670in}}%
\pgfpathlineto{\pgfqpoint{2.386877in}{3.150041in}}%
\pgfpathlineto{\pgfqpoint{2.400534in}{3.126629in}}%
\pgfpathlineto{\pgfqpoint{2.414180in}{3.103433in}}%
\pgfpathlineto{\pgfqpoint{2.427816in}{3.080452in}}%
\pgfpathlineto{\pgfqpoint{2.436471in}{3.078228in}}%
\pgfpathlineto{\pgfqpoint{2.445109in}{3.076225in}}%
\pgfpathlineto{\pgfqpoint{2.453731in}{3.074442in}}%
\pgfpathlineto{\pgfqpoint{2.462337in}{3.072874in}}%
\pgfpathlineto{\pgfqpoint{2.448747in}{3.095484in}}%
\pgfpathlineto{\pgfqpoint{2.435147in}{3.118309in}}%
\pgfpathlineto{\pgfqpoint{2.421535in}{3.141348in}}%
\pgfpathlineto{\pgfqpoint{2.407914in}{3.164605in}}%
\pgfpathlineto{\pgfqpoint{2.399263in}{3.166537in}}%
\pgfpathlineto{\pgfqpoint{2.390595in}{3.168690in}}%
\pgfpathlineto{\pgfqpoint{2.381911in}{3.171067in}}%
\pgfpathlineto{\pgfqpoint{2.373209in}{3.173670in}}%
\pgfpathclose%
\pgfusepath{fill}%
\end{pgfscope}%
\begin{pgfscope}%
\pgfpathrectangle{\pgfqpoint{1.254980in}{0.150000in}}{\pgfqpoint{5.490039in}{5.490039in}}%
\pgfusepath{clip}%
\pgfsetbuttcap%
\pgfsetroundjoin%
\definecolor{currentfill}{rgb}{0.260571,0.246922,0.522828}%
\pgfsetfillcolor{currentfill}%
\pgfsetfillopacity{0.700000}%
\pgfsetlinewidth{0.000000pt}%
\definecolor{currentstroke}{rgb}{0.000000,0.000000,0.000000}%
\pgfsetstrokecolor{currentstroke}%
\pgfsetdash{}{0pt}%
\pgfpathmoveto{\pgfqpoint{4.949427in}{2.122148in}}%
\pgfpathlineto{\pgfqpoint{4.963133in}{2.125155in}}%
\pgfpathlineto{\pgfqpoint{4.976850in}{2.128275in}}%
\pgfpathlineto{\pgfqpoint{4.990580in}{2.131509in}}%
\pgfpathlineto{\pgfqpoint{5.004323in}{2.134857in}}%
\pgfpathlineto{\pgfqpoint{5.011771in}{2.146088in}}%
\pgfpathlineto{\pgfqpoint{5.019214in}{2.157279in}}%
\pgfpathlineto{\pgfqpoint{5.026652in}{2.168428in}}%
\pgfpathlineto{\pgfqpoint{5.034085in}{2.179534in}}%
\pgfpathlineto{\pgfqpoint{5.020347in}{2.176084in}}%
\pgfpathlineto{\pgfqpoint{5.006621in}{2.172747in}}%
\pgfpathlineto{\pgfqpoint{4.992908in}{2.169524in}}%
\pgfpathlineto{\pgfqpoint{4.979207in}{2.166415in}}%
\pgfpathlineto{\pgfqpoint{4.971769in}{2.155405in}}%
\pgfpathlineto{\pgfqpoint{4.964327in}{2.144357in}}%
\pgfpathlineto{\pgfqpoint{4.956879in}{2.133271in}}%
\pgfpathlineto{\pgfqpoint{4.949427in}{2.122148in}}%
\pgfpathclose%
\pgfusepath{fill}%
\end{pgfscope}%
\begin{pgfscope}%
\pgfpathrectangle{\pgfqpoint{1.254980in}{0.150000in}}{\pgfqpoint{5.490039in}{5.490039in}}%
\pgfusepath{clip}%
\pgfsetbuttcap%
\pgfsetroundjoin%
\definecolor{currentfill}{rgb}{0.274952,0.037752,0.364543}%
\pgfsetfillcolor{currentfill}%
\pgfsetfillopacity{0.700000}%
\pgfsetlinewidth{0.000000pt}%
\definecolor{currentstroke}{rgb}{0.000000,0.000000,0.000000}%
\pgfsetstrokecolor{currentstroke}%
\pgfsetdash{}{0pt}%
\pgfpathmoveto{\pgfqpoint{3.691058in}{1.749398in}}%
\pgfpathlineto{\pgfqpoint{3.704388in}{1.741962in}}%
\pgfpathlineto{\pgfqpoint{3.717722in}{1.734657in}}%
\pgfpathlineto{\pgfqpoint{3.731059in}{1.727480in}}%
\pgfpathlineto{\pgfqpoint{3.744399in}{1.720432in}}%
\pgfpathlineto{\pgfqpoint{3.752264in}{1.727024in}}%
\pgfpathlineto{\pgfqpoint{3.760122in}{1.733719in}}%
\pgfpathlineto{\pgfqpoint{3.767973in}{1.740515in}}%
\pgfpathlineto{\pgfqpoint{3.775818in}{1.747410in}}%
\pgfpathlineto{\pgfqpoint{3.762495in}{1.754148in}}%
\pgfpathlineto{\pgfqpoint{3.749176in}{1.761015in}}%
\pgfpathlineto{\pgfqpoint{3.735861in}{1.768010in}}%
\pgfpathlineto{\pgfqpoint{3.722549in}{1.775135in}}%
\pgfpathlineto{\pgfqpoint{3.714687in}{1.768544in}}%
\pgfpathlineto{\pgfqpoint{3.706818in}{1.762055in}}%
\pgfpathlineto{\pgfqpoint{3.698942in}{1.755672in}}%
\pgfpathlineto{\pgfqpoint{3.691058in}{1.749398in}}%
\pgfpathclose%
\pgfusepath{fill}%
\end{pgfscope}%
\begin{pgfscope}%
\pgfpathrectangle{\pgfqpoint{1.254980in}{0.150000in}}{\pgfqpoint{5.490039in}{5.490039in}}%
\pgfusepath{clip}%
\pgfsetbuttcap%
\pgfsetroundjoin%
\definecolor{currentfill}{rgb}{0.271305,0.019942,0.347269}%
\pgfsetfillcolor{currentfill}%
\pgfsetfillopacity{0.700000}%
\pgfsetlinewidth{0.000000pt}%
\definecolor{currentstroke}{rgb}{0.000000,0.000000,0.000000}%
\pgfsetstrokecolor{currentstroke}%
\pgfsetdash{}{0pt}%
\pgfpathmoveto{\pgfqpoint{3.967146in}{1.708483in}}%
\pgfpathlineto{\pgfqpoint{3.980514in}{1.703674in}}%
\pgfpathlineto{\pgfqpoint{3.993886in}{1.698988in}}%
\pgfpathlineto{\pgfqpoint{4.007265in}{1.694425in}}%
\pgfpathlineto{\pgfqpoint{4.020649in}{1.689984in}}%
\pgfpathlineto{\pgfqpoint{4.028401in}{1.698471in}}%
\pgfpathlineto{\pgfqpoint{4.036148in}{1.707026in}}%
\pgfpathlineto{\pgfqpoint{4.043889in}{1.715648in}}%
\pgfpathlineto{\pgfqpoint{4.051625in}{1.724334in}}%
\pgfpathlineto{\pgfqpoint{4.038254in}{1.728498in}}%
\pgfpathlineto{\pgfqpoint{4.024888in}{1.732784in}}%
\pgfpathlineto{\pgfqpoint{4.011529in}{1.737193in}}%
\pgfpathlineto{\pgfqpoint{3.998175in}{1.741726in}}%
\pgfpathlineto{\pgfqpoint{3.990426in}{1.733310in}}%
\pgfpathlineto{\pgfqpoint{3.982672in}{1.724963in}}%
\pgfpathlineto{\pgfqpoint{3.974912in}{1.716687in}}%
\pgfpathlineto{\pgfqpoint{3.967146in}{1.708483in}}%
\pgfpathclose%
\pgfusepath{fill}%
\end{pgfscope}%
\begin{pgfscope}%
\pgfpathrectangle{\pgfqpoint{1.254980in}{0.150000in}}{\pgfqpoint{5.490039in}{5.490039in}}%
\pgfusepath{clip}%
\pgfsetbuttcap%
\pgfsetroundjoin%
\definecolor{currentfill}{rgb}{0.281412,0.155834,0.469201}%
\pgfsetfillcolor{currentfill}%
\pgfsetfillopacity{0.700000}%
\pgfsetlinewidth{0.000000pt}%
\definecolor{currentstroke}{rgb}{0.000000,0.000000,0.000000}%
\pgfsetstrokecolor{currentstroke}%
\pgfsetdash{}{0pt}%
\pgfpathmoveto{\pgfqpoint{3.254037in}{1.985163in}}%
\pgfpathlineto{\pgfqpoint{3.267375in}{1.973207in}}%
\pgfpathlineto{\pgfqpoint{3.280712in}{1.961397in}}%
\pgfpathlineto{\pgfqpoint{3.294049in}{1.949731in}}%
\pgfpathlineto{\pgfqpoint{3.307385in}{1.938210in}}%
\pgfpathlineto{\pgfqpoint{3.315476in}{1.941411in}}%
\pgfpathlineto{\pgfqpoint{3.323557in}{1.944767in}}%
\pgfpathlineto{\pgfqpoint{3.331628in}{1.948275in}}%
\pgfpathlineto{\pgfqpoint{3.339689in}{1.951932in}}%
\pgfpathlineto{\pgfqpoint{3.326379in}{1.963104in}}%
\pgfpathlineto{\pgfqpoint{3.313070in}{1.974420in}}%
\pgfpathlineto{\pgfqpoint{3.299760in}{1.985880in}}%
\pgfpathlineto{\pgfqpoint{3.286450in}{1.997485in}}%
\pgfpathlineto{\pgfqpoint{3.278362in}{1.994172in}}%
\pgfpathlineto{\pgfqpoint{3.270264in}{1.991012in}}%
\pgfpathlineto{\pgfqpoint{3.262155in}{1.988008in}}%
\pgfpathlineto{\pgfqpoint{3.254037in}{1.985163in}}%
\pgfpathclose%
\pgfusepath{fill}%
\end{pgfscope}%
\begin{pgfscope}%
\pgfpathrectangle{\pgfqpoint{1.254980in}{0.150000in}}{\pgfqpoint{5.490039in}{5.490039in}}%
\pgfusepath{clip}%
\pgfsetbuttcap%
\pgfsetroundjoin%
\definecolor{currentfill}{rgb}{0.159194,0.482237,0.558073}%
\pgfsetfillcolor{currentfill}%
\pgfsetfillopacity{0.700000}%
\pgfsetlinewidth{0.000000pt}%
\definecolor{currentstroke}{rgb}{0.000000,0.000000,0.000000}%
\pgfsetstrokecolor{currentstroke}%
\pgfsetdash{}{0pt}%
\pgfpathmoveto{\pgfqpoint{5.686623in}{2.697150in}}%
\pgfpathlineto{\pgfqpoint{5.700715in}{2.704058in}}%
\pgfpathlineto{\pgfqpoint{5.714823in}{2.711077in}}%
\pgfpathlineto{\pgfqpoint{5.728947in}{2.718209in}}%
\pgfpathlineto{\pgfqpoint{5.743086in}{2.725454in}}%
\pgfpathlineto{\pgfqpoint{5.750243in}{2.734476in}}%
\pgfpathlineto{\pgfqpoint{5.757392in}{2.743428in}}%
\pgfpathlineto{\pgfqpoint{5.764535in}{2.752311in}}%
\pgfpathlineto{\pgfqpoint{5.771670in}{2.761124in}}%
\pgfpathlineto{\pgfqpoint{5.757540in}{2.753944in}}%
\pgfpathlineto{\pgfqpoint{5.743425in}{2.746876in}}%
\pgfpathlineto{\pgfqpoint{5.729327in}{2.739920in}}%
\pgfpathlineto{\pgfqpoint{5.715244in}{2.733076in}}%
\pgfpathlineto{\pgfqpoint{5.708099in}{2.724193in}}%
\pgfpathlineto{\pgfqpoint{5.700947in}{2.715244in}}%
\pgfpathlineto{\pgfqpoint{5.693788in}{2.706231in}}%
\pgfpathlineto{\pgfqpoint{5.686623in}{2.697150in}}%
\pgfpathclose%
\pgfusepath{fill}%
\end{pgfscope}%
\begin{pgfscope}%
\pgfpathrectangle{\pgfqpoint{1.254980in}{0.150000in}}{\pgfqpoint{5.490039in}{5.490039in}}%
\pgfusepath{clip}%
\pgfsetbuttcap%
\pgfsetroundjoin%
\definecolor{currentfill}{rgb}{0.274952,0.037752,0.364543}%
\pgfsetfillcolor{currentfill}%
\pgfsetfillopacity{0.700000}%
\pgfsetlinewidth{0.000000pt}%
\definecolor{currentstroke}{rgb}{0.000000,0.000000,0.000000}%
\pgfsetstrokecolor{currentstroke}%
\pgfsetdash{}{0pt}%
\pgfpathmoveto{\pgfqpoint{4.189616in}{1.732830in}}%
\pgfpathlineto{\pgfqpoint{4.203035in}{1.730016in}}%
\pgfpathlineto{\pgfqpoint{4.216460in}{1.727321in}}%
\pgfpathlineto{\pgfqpoint{4.229893in}{1.724746in}}%
\pgfpathlineto{\pgfqpoint{4.243333in}{1.722289in}}%
\pgfpathlineto{\pgfqpoint{4.251009in}{1.732019in}}%
\pgfpathlineto{\pgfqpoint{4.258681in}{1.741790in}}%
\pgfpathlineto{\pgfqpoint{4.266347in}{1.751599in}}%
\pgfpathlineto{\pgfqpoint{4.274008in}{1.761445in}}%
\pgfpathlineto{\pgfqpoint{4.260577in}{1.763657in}}%
\pgfpathlineto{\pgfqpoint{4.247154in}{1.765987in}}%
\pgfpathlineto{\pgfqpoint{4.233738in}{1.768437in}}%
\pgfpathlineto{\pgfqpoint{4.220329in}{1.771007in}}%
\pgfpathlineto{\pgfqpoint{4.212659in}{1.761400in}}%
\pgfpathlineto{\pgfqpoint{4.204983in}{1.751833in}}%
\pgfpathlineto{\pgfqpoint{4.197302in}{1.742310in}}%
\pgfpathlineto{\pgfqpoint{4.189616in}{1.732830in}}%
\pgfpathclose%
\pgfusepath{fill}%
\end{pgfscope}%
\begin{pgfscope}%
\pgfpathrectangle{\pgfqpoint{1.254980in}{0.150000in}}{\pgfqpoint{5.490039in}{5.490039in}}%
\pgfusepath{clip}%
\pgfsetbuttcap%
\pgfsetroundjoin%
\definecolor{currentfill}{rgb}{0.149039,0.508051,0.557250}%
\pgfsetfillcolor{currentfill}%
\pgfsetfillopacity{0.700000}%
\pgfsetlinewidth{0.000000pt}%
\definecolor{currentstroke}{rgb}{0.000000,0.000000,0.000000}%
\pgfsetstrokecolor{currentstroke}%
\pgfsetdash{}{0pt}%
\pgfpathmoveto{\pgfqpoint{5.771670in}{2.761124in}}%
\pgfpathlineto{\pgfqpoint{5.785817in}{2.768417in}}%
\pgfpathlineto{\pgfqpoint{5.799979in}{2.775822in}}%
\pgfpathlineto{\pgfqpoint{5.814158in}{2.783339in}}%
\pgfpathlineto{\pgfqpoint{5.821279in}{2.792029in}}%
\pgfpathlineto{\pgfqpoint{5.828392in}{2.800648in}}%
\pgfpathlineto{\pgfqpoint{5.835499in}{2.809198in}}%
\pgfpathlineto{\pgfqpoint{5.842598in}{2.817680in}}%
\pgfpathlineto{\pgfqpoint{5.828429in}{2.810244in}}%
\pgfpathlineto{\pgfqpoint{5.814277in}{2.802920in}}%
\pgfpathlineto{\pgfqpoint{5.800141in}{2.795709in}}%
\pgfpathlineto{\pgfqpoint{5.793034in}{2.787161in}}%
\pgfpathlineto{\pgfqpoint{5.785919in}{2.778548in}}%
\pgfpathlineto{\pgfqpoint{5.778798in}{2.769870in}}%
\pgfpathlineto{\pgfqpoint{5.771670in}{2.761124in}}%
\pgfpathclose%
\pgfusepath{fill}%
\end{pgfscope}%
\begin{pgfscope}%
\pgfpathrectangle{\pgfqpoint{1.254980in}{0.150000in}}{\pgfqpoint{5.490039in}{5.490039in}}%
\pgfusepath{clip}%
\pgfsetbuttcap%
\pgfsetroundjoin%
\definecolor{currentfill}{rgb}{0.250425,0.274290,0.533103}%
\pgfsetfillcolor{currentfill}%
\pgfsetfillopacity{0.700000}%
\pgfsetlinewidth{0.000000pt}%
\definecolor{currentstroke}{rgb}{0.000000,0.000000,0.000000}%
\pgfsetstrokecolor{currentstroke}%
\pgfsetdash{}{0pt}%
\pgfpathmoveto{\pgfqpoint{5.034085in}{2.179534in}}%
\pgfpathlineto{\pgfqpoint{5.047835in}{2.183098in}}%
\pgfpathlineto{\pgfqpoint{5.061598in}{2.186776in}}%
\pgfpathlineto{\pgfqpoint{5.075374in}{2.190567in}}%
\pgfpathlineto{\pgfqpoint{5.089162in}{2.194471in}}%
\pgfpathlineto{\pgfqpoint{5.096585in}{2.205627in}}%
\pgfpathlineto{\pgfqpoint{5.104003in}{2.216737in}}%
\pgfpathlineto{\pgfqpoint{5.111415in}{2.227799in}}%
\pgfpathlineto{\pgfqpoint{5.118822in}{2.238812in}}%
\pgfpathlineto{\pgfqpoint{5.105038in}{2.234822in}}%
\pgfpathlineto{\pgfqpoint{5.091267in}{2.230944in}}%
\pgfpathlineto{\pgfqpoint{5.077509in}{2.227181in}}%
\pgfpathlineto{\pgfqpoint{5.063763in}{2.223530in}}%
\pgfpathlineto{\pgfqpoint{5.056351in}{2.212597in}}%
\pgfpathlineto{\pgfqpoint{5.048935in}{2.201620in}}%
\pgfpathlineto{\pgfqpoint{5.041512in}{2.190599in}}%
\pgfpathlineto{\pgfqpoint{5.034085in}{2.179534in}}%
\pgfpathclose%
\pgfusepath{fill}%
\end{pgfscope}%
\begin{pgfscope}%
\pgfpathrectangle{\pgfqpoint{1.254980in}{0.150000in}}{\pgfqpoint{5.490039in}{5.490039in}}%
\pgfusepath{clip}%
\pgfsetbuttcap%
\pgfsetroundjoin%
\definecolor{currentfill}{rgb}{0.150476,0.504369,0.557430}%
\pgfsetfillcolor{currentfill}%
\pgfsetfillopacity{0.700000}%
\pgfsetlinewidth{0.000000pt}%
\definecolor{currentstroke}{rgb}{0.000000,0.000000,0.000000}%
\pgfsetstrokecolor{currentstroke}%
\pgfsetdash{}{0pt}%
\pgfpathmoveto{\pgfqpoint{2.556430in}{2.827443in}}%
\pgfpathlineto{\pgfqpoint{2.569987in}{2.806740in}}%
\pgfpathlineto{\pgfqpoint{2.583536in}{2.786232in}}%
\pgfpathlineto{\pgfqpoint{2.597077in}{2.765918in}}%
\pgfpathlineto{\pgfqpoint{2.610610in}{2.745798in}}%
\pgfpathlineto{\pgfqpoint{2.619152in}{2.744217in}}%
\pgfpathlineto{\pgfqpoint{2.627679in}{2.742850in}}%
\pgfpathlineto{\pgfqpoint{2.636191in}{2.741695in}}%
\pgfpathlineto{\pgfqpoint{2.644688in}{2.740748in}}%
\pgfpathlineto{\pgfqpoint{2.631197in}{2.760492in}}%
\pgfpathlineto{\pgfqpoint{2.617699in}{2.780428in}}%
\pgfpathlineto{\pgfqpoint{2.604193in}{2.800558in}}%
\pgfpathlineto{\pgfqpoint{2.590678in}{2.820882in}}%
\pgfpathlineto{\pgfqpoint{2.582140in}{2.822199in}}%
\pgfpathlineto{\pgfqpoint{2.573586in}{2.823730in}}%
\pgfpathlineto{\pgfqpoint{2.565016in}{2.825477in}}%
\pgfpathlineto{\pgfqpoint{2.556430in}{2.827443in}}%
\pgfpathclose%
\pgfusepath{fill}%
\end{pgfscope}%
\begin{pgfscope}%
\pgfpathrectangle{\pgfqpoint{1.254980in}{0.150000in}}{\pgfqpoint{5.490039in}{5.490039in}}%
\pgfusepath{clip}%
\pgfsetbuttcap%
\pgfsetroundjoin%
\definecolor{currentfill}{rgb}{0.194100,0.399323,0.555565}%
\pgfsetfillcolor{currentfill}%
\pgfsetfillopacity{0.700000}%
\pgfsetlinewidth{0.000000pt}%
\definecolor{currentstroke}{rgb}{0.000000,0.000000,0.000000}%
\pgfsetstrokecolor{currentstroke}%
\pgfsetdash{}{0pt}%
\pgfpathmoveto{\pgfqpoint{5.402776in}{2.467166in}}%
\pgfpathlineto{\pgfqpoint{5.416717in}{2.472808in}}%
\pgfpathlineto{\pgfqpoint{5.430672in}{2.478562in}}%
\pgfpathlineto{\pgfqpoint{5.444642in}{2.484430in}}%
\pgfpathlineto{\pgfqpoint{5.458626in}{2.490410in}}%
\pgfpathlineto{\pgfqpoint{5.465913in}{2.500615in}}%
\pgfpathlineto{\pgfqpoint{5.473193in}{2.510754in}}%
\pgfpathlineto{\pgfqpoint{5.480467in}{2.520827in}}%
\pgfpathlineto{\pgfqpoint{5.487734in}{2.530834in}}%
\pgfpathlineto{\pgfqpoint{5.473756in}{2.524850in}}%
\pgfpathlineto{\pgfqpoint{5.459792in}{2.518979in}}%
\pgfpathlineto{\pgfqpoint{5.445843in}{2.513220in}}%
\pgfpathlineto{\pgfqpoint{5.431909in}{2.507574in}}%
\pgfpathlineto{\pgfqpoint{5.424635in}{2.497565in}}%
\pgfpathlineto{\pgfqpoint{5.417355in}{2.487494in}}%
\pgfpathlineto{\pgfqpoint{5.410068in}{2.477361in}}%
\pgfpathlineto{\pgfqpoint{5.402776in}{2.467166in}}%
\pgfpathclose%
\pgfusepath{fill}%
\end{pgfscope}%
\begin{pgfscope}%
\pgfpathrectangle{\pgfqpoint{1.254980in}{0.150000in}}{\pgfqpoint{5.490039in}{5.490039in}}%
\pgfusepath{clip}%
\pgfsetbuttcap%
\pgfsetroundjoin%
\definecolor{currentfill}{rgb}{0.282884,0.135920,0.453427}%
\pgfsetfillcolor{currentfill}%
\pgfsetfillopacity{0.700000}%
\pgfsetlinewidth{0.000000pt}%
\definecolor{currentstroke}{rgb}{0.000000,0.000000,0.000000}%
\pgfsetstrokecolor{currentstroke}%
\pgfsetdash{}{0pt}%
\pgfpathmoveto{\pgfqpoint{3.307385in}{1.938210in}}%
\pgfpathlineto{\pgfqpoint{3.320722in}{1.926832in}}%
\pgfpathlineto{\pgfqpoint{3.334059in}{1.915597in}}%
\pgfpathlineto{\pgfqpoint{3.347395in}{1.904505in}}%
\pgfpathlineto{\pgfqpoint{3.360732in}{1.893554in}}%
\pgfpathlineto{\pgfqpoint{3.368796in}{1.897110in}}%
\pgfpathlineto{\pgfqpoint{3.376850in}{1.900817in}}%
\pgfpathlineto{\pgfqpoint{3.384895in}{1.904671in}}%
\pgfpathlineto{\pgfqpoint{3.392930in}{1.908671in}}%
\pgfpathlineto{\pgfqpoint{3.379619in}{1.919273in}}%
\pgfpathlineto{\pgfqpoint{3.366309in}{1.930017in}}%
\pgfpathlineto{\pgfqpoint{3.352999in}{1.940903in}}%
\pgfpathlineto{\pgfqpoint{3.339689in}{1.951932in}}%
\pgfpathlineto{\pgfqpoint{3.331628in}{1.948275in}}%
\pgfpathlineto{\pgfqpoint{3.323557in}{1.944767in}}%
\pgfpathlineto{\pgfqpoint{3.315476in}{1.941411in}}%
\pgfpathlineto{\pgfqpoint{3.307385in}{1.938210in}}%
\pgfpathclose%
\pgfusepath{fill}%
\end{pgfscope}%
\begin{pgfscope}%
\pgfpathrectangle{\pgfqpoint{1.254980in}{0.150000in}}{\pgfqpoint{5.490039in}{5.490039in}}%
\pgfusepath{clip}%
\pgfsetbuttcap%
\pgfsetroundjoin%
\definecolor{currentfill}{rgb}{0.279566,0.067836,0.391917}%
\pgfsetfillcolor{currentfill}%
\pgfsetfillopacity{0.700000}%
\pgfsetlinewidth{0.000000pt}%
\definecolor{currentstroke}{rgb}{0.000000,0.000000,0.000000}%
\pgfsetstrokecolor{currentstroke}%
\pgfsetdash{}{0pt}%
\pgfpathmoveto{\pgfqpoint{3.552750in}{1.792254in}}%
\pgfpathlineto{\pgfqpoint{3.566079in}{1.783437in}}%
\pgfpathlineto{\pgfqpoint{3.579410in}{1.774753in}}%
\pgfpathlineto{\pgfqpoint{3.592743in}{1.766203in}}%
\pgfpathlineto{\pgfqpoint{3.606079in}{1.757785in}}%
\pgfpathlineto{\pgfqpoint{3.614013in}{1.763268in}}%
\pgfpathlineto{\pgfqpoint{3.621940in}{1.768875in}}%
\pgfpathlineto{\pgfqpoint{3.629858in}{1.774601in}}%
\pgfpathlineto{\pgfqpoint{3.637769in}{1.780444in}}%
\pgfpathlineto{\pgfqpoint{3.624453in}{1.788534in}}%
\pgfpathlineto{\pgfqpoint{3.611141in}{1.796757in}}%
\pgfpathlineto{\pgfqpoint{3.597831in}{1.805112in}}%
\pgfpathlineto{\pgfqpoint{3.584523in}{1.813601in}}%
\pgfpathlineto{\pgfqpoint{3.576592in}{1.808080in}}%
\pgfpathlineto{\pgfqpoint{3.568653in}{1.802680in}}%
\pgfpathlineto{\pgfqpoint{3.560705in}{1.797403in}}%
\pgfpathlineto{\pgfqpoint{3.552750in}{1.792254in}}%
\pgfpathclose%
\pgfusepath{fill}%
\end{pgfscope}%
\begin{pgfscope}%
\pgfpathrectangle{\pgfqpoint{1.254980in}{0.150000in}}{\pgfqpoint{5.490039in}{5.490039in}}%
\pgfusepath{clip}%
\pgfsetbuttcap%
\pgfsetroundjoin%
\definecolor{currentfill}{rgb}{0.272594,0.025563,0.353093}%
\pgfsetfillcolor{currentfill}%
\pgfsetfillopacity{0.700000}%
\pgfsetlinewidth{0.000000pt}%
\definecolor{currentstroke}{rgb}{0.000000,0.000000,0.000000}%
\pgfsetstrokecolor{currentstroke}%
\pgfsetdash{}{0pt}%
\pgfpathmoveto{\pgfqpoint{4.105171in}{1.708897in}}%
\pgfpathlineto{\pgfqpoint{4.118573in}{1.705341in}}%
\pgfpathlineto{\pgfqpoint{4.131982in}{1.701906in}}%
\pgfpathlineto{\pgfqpoint{4.145398in}{1.698591in}}%
\pgfpathlineto{\pgfqpoint{4.158820in}{1.695396in}}%
\pgfpathlineto{\pgfqpoint{4.166527in}{1.704678in}}%
\pgfpathlineto{\pgfqpoint{4.174228in}{1.714013in}}%
\pgfpathlineto{\pgfqpoint{4.181925in}{1.723397in}}%
\pgfpathlineto{\pgfqpoint{4.189616in}{1.732830in}}%
\pgfpathlineto{\pgfqpoint{4.176205in}{1.735764in}}%
\pgfpathlineto{\pgfqpoint{4.162800in}{1.738819in}}%
\pgfpathlineto{\pgfqpoint{4.149402in}{1.741993in}}%
\pgfpathlineto{\pgfqpoint{4.136011in}{1.745289in}}%
\pgfpathlineto{\pgfqpoint{4.128309in}{1.736110in}}%
\pgfpathlineto{\pgfqpoint{4.120602in}{1.726984in}}%
\pgfpathlineto{\pgfqpoint{4.112889in}{1.717913in}}%
\pgfpathlineto{\pgfqpoint{4.105171in}{1.708897in}}%
\pgfpathclose%
\pgfusepath{fill}%
\end{pgfscope}%
\begin{pgfscope}%
\pgfpathrectangle{\pgfqpoint{1.254980in}{0.150000in}}{\pgfqpoint{5.490039in}{5.490039in}}%
\pgfusepath{clip}%
\pgfsetbuttcap%
\pgfsetroundjoin%
\definecolor{currentfill}{rgb}{0.237441,0.305202,0.541921}%
\pgfsetfillcolor{currentfill}%
\pgfsetfillopacity{0.700000}%
\pgfsetlinewidth{0.000000pt}%
\definecolor{currentstroke}{rgb}{0.000000,0.000000,0.000000}%
\pgfsetstrokecolor{currentstroke}%
\pgfsetdash{}{0pt}%
\pgfpathmoveto{\pgfqpoint{5.118822in}{2.238812in}}%
\pgfpathlineto{\pgfqpoint{5.132619in}{2.242916in}}%
\pgfpathlineto{\pgfqpoint{5.146429in}{2.247134in}}%
\pgfpathlineto{\pgfqpoint{5.160252in}{2.251464in}}%
\pgfpathlineto{\pgfqpoint{5.174088in}{2.255908in}}%
\pgfpathlineto{\pgfqpoint{5.181485in}{2.266949in}}%
\pgfpathlineto{\pgfqpoint{5.188877in}{2.277938in}}%
\pgfpathlineto{\pgfqpoint{5.196263in}{2.288873in}}%
\pgfpathlineto{\pgfqpoint{5.203643in}{2.299754in}}%
\pgfpathlineto{\pgfqpoint{5.189811in}{2.295240in}}%
\pgfpathlineto{\pgfqpoint{5.175993in}{2.290840in}}%
\pgfpathlineto{\pgfqpoint{5.162187in}{2.286552in}}%
\pgfpathlineto{\pgfqpoint{5.148395in}{2.282378in}}%
\pgfpathlineto{\pgfqpoint{5.141010in}{2.271561in}}%
\pgfpathlineto{\pgfqpoint{5.133620in}{2.260694in}}%
\pgfpathlineto{\pgfqpoint{5.126224in}{2.249777in}}%
\pgfpathlineto{\pgfqpoint{5.118822in}{2.238812in}}%
\pgfpathclose%
\pgfusepath{fill}%
\end{pgfscope}%
\begin{pgfscope}%
\pgfpathrectangle{\pgfqpoint{1.254980in}{0.150000in}}{\pgfqpoint{5.490039in}{5.490039in}}%
\pgfusepath{clip}%
\pgfsetbuttcap%
\pgfsetroundjoin%
\definecolor{currentfill}{rgb}{0.283187,0.125848,0.444960}%
\pgfsetfillcolor{currentfill}%
\pgfsetfillopacity{0.700000}%
\pgfsetlinewidth{0.000000pt}%
\definecolor{currentstroke}{rgb}{0.000000,0.000000,0.000000}%
\pgfsetstrokecolor{currentstroke}%
\pgfsetdash{}{0pt}%
\pgfpathmoveto{\pgfqpoint{4.581201in}{1.872707in}}%
\pgfpathlineto{\pgfqpoint{4.594757in}{1.873127in}}%
\pgfpathlineto{\pgfqpoint{4.608323in}{1.873662in}}%
\pgfpathlineto{\pgfqpoint{4.621898in}{1.874313in}}%
\pgfpathlineto{\pgfqpoint{4.635484in}{1.875079in}}%
\pgfpathlineto{\pgfqpoint{4.643048in}{1.886190in}}%
\pgfpathlineto{\pgfqpoint{4.650607in}{1.897296in}}%
\pgfpathlineto{\pgfqpoint{4.658162in}{1.908395in}}%
\pgfpathlineto{\pgfqpoint{4.665711in}{1.919486in}}%
\pgfpathlineto{\pgfqpoint{4.652131in}{1.918538in}}%
\pgfpathlineto{\pgfqpoint{4.638561in}{1.917705in}}%
\pgfpathlineto{\pgfqpoint{4.625001in}{1.916987in}}%
\pgfpathlineto{\pgfqpoint{4.611450in}{1.916385in}}%
\pgfpathlineto{\pgfqpoint{4.603895in}{1.905470in}}%
\pgfpathlineto{\pgfqpoint{4.596335in}{1.894551in}}%
\pgfpathlineto{\pgfqpoint{4.588770in}{1.883629in}}%
\pgfpathlineto{\pgfqpoint{4.581201in}{1.872707in}}%
\pgfpathclose%
\pgfusepath{fill}%
\end{pgfscope}%
\begin{pgfscope}%
\pgfpathrectangle{\pgfqpoint{1.254980in}{0.150000in}}{\pgfqpoint{5.490039in}{5.490039in}}%
\pgfusepath{clip}%
\pgfsetbuttcap%
\pgfsetroundjoin%
\definecolor{currentfill}{rgb}{0.182256,0.426184,0.557120}%
\pgfsetfillcolor{currentfill}%
\pgfsetfillopacity{0.700000}%
\pgfsetlinewidth{0.000000pt}%
\definecolor{currentstroke}{rgb}{0.000000,0.000000,0.000000}%
\pgfsetstrokecolor{currentstroke}%
\pgfsetdash{}{0pt}%
\pgfpathmoveto{\pgfqpoint{5.487734in}{2.530834in}}%
\pgfpathlineto{\pgfqpoint{5.501727in}{2.536931in}}%
\pgfpathlineto{\pgfqpoint{5.515735in}{2.543140in}}%
\pgfpathlineto{\pgfqpoint{5.529758in}{2.549462in}}%
\pgfpathlineto{\pgfqpoint{5.543796in}{2.555896in}}%
\pgfpathlineto{\pgfqpoint{5.551050in}{2.565831in}}%
\pgfpathlineto{\pgfqpoint{5.558298in}{2.575697in}}%
\pgfpathlineto{\pgfqpoint{5.565539in}{2.585493in}}%
\pgfpathlineto{\pgfqpoint{5.572774in}{2.595222in}}%
\pgfpathlineto{\pgfqpoint{5.558742in}{2.588800in}}%
\pgfpathlineto{\pgfqpoint{5.544726in}{2.582491in}}%
\pgfpathlineto{\pgfqpoint{5.530725in}{2.576295in}}%
\pgfpathlineto{\pgfqpoint{5.516739in}{2.570211in}}%
\pgfpathlineto{\pgfqpoint{5.509498in}{2.560463in}}%
\pgfpathlineto{\pgfqpoint{5.502250in}{2.550652in}}%
\pgfpathlineto{\pgfqpoint{5.494995in}{2.540775in}}%
\pgfpathlineto{\pgfqpoint{5.487734in}{2.530834in}}%
\pgfpathclose%
\pgfusepath{fill}%
\end{pgfscope}%
\begin{pgfscope}%
\pgfpathrectangle{\pgfqpoint{1.254980in}{0.150000in}}{\pgfqpoint{5.490039in}{5.490039in}}%
\pgfusepath{clip}%
\pgfsetbuttcap%
\pgfsetroundjoin%
\definecolor{currentfill}{rgb}{0.282910,0.105393,0.426902}%
\pgfsetfillcolor{currentfill}%
\pgfsetfillopacity{0.700000}%
\pgfsetlinewidth{0.000000pt}%
\definecolor{currentstroke}{rgb}{0.000000,0.000000,0.000000}%
\pgfsetstrokecolor{currentstroke}%
\pgfsetdash{}{0pt}%
\pgfpathmoveto{\pgfqpoint{4.496724in}{1.829300in}}%
\pgfpathlineto{\pgfqpoint{4.510248in}{1.829059in}}%
\pgfpathlineto{\pgfqpoint{4.523781in}{1.828934in}}%
\pgfpathlineto{\pgfqpoint{4.537324in}{1.828924in}}%
\pgfpathlineto{\pgfqpoint{4.550876in}{1.829031in}}%
\pgfpathlineto{\pgfqpoint{4.558464in}{1.839945in}}%
\pgfpathlineto{\pgfqpoint{4.566048in}{1.850863in}}%
\pgfpathlineto{\pgfqpoint{4.573627in}{1.861784in}}%
\pgfpathlineto{\pgfqpoint{4.581201in}{1.872707in}}%
\pgfpathlineto{\pgfqpoint{4.567655in}{1.872402in}}%
\pgfpathlineto{\pgfqpoint{4.554118in}{1.872214in}}%
\pgfpathlineto{\pgfqpoint{4.540591in}{1.872141in}}%
\pgfpathlineto{\pgfqpoint{4.527073in}{1.872185in}}%
\pgfpathlineto{\pgfqpoint{4.519493in}{1.861454in}}%
\pgfpathlineto{\pgfqpoint{4.511908in}{1.850729in}}%
\pgfpathlineto{\pgfqpoint{4.504318in}{1.840010in}}%
\pgfpathlineto{\pgfqpoint{4.496724in}{1.829300in}}%
\pgfpathclose%
\pgfusepath{fill}%
\end{pgfscope}%
\begin{pgfscope}%
\pgfpathrectangle{\pgfqpoint{1.254980in}{0.150000in}}{\pgfqpoint{5.490039in}{5.490039in}}%
\pgfusepath{clip}%
\pgfsetbuttcap%
\pgfsetroundjoin%
\definecolor{currentfill}{rgb}{0.137770,0.537492,0.554906}%
\pgfsetfillcolor{currentfill}%
\pgfsetfillopacity{0.700000}%
\pgfsetlinewidth{0.000000pt}%
\definecolor{currentstroke}{rgb}{0.000000,0.000000,0.000000}%
\pgfsetstrokecolor{currentstroke}%
\pgfsetdash{}{0pt}%
\pgfpathmoveto{\pgfqpoint{2.502114in}{2.912238in}}%
\pgfpathlineto{\pgfqpoint{2.515706in}{2.890740in}}%
\pgfpathlineto{\pgfqpoint{2.529290in}{2.869442in}}%
\pgfpathlineto{\pgfqpoint{2.542864in}{2.848344in}}%
\pgfpathlineto{\pgfqpoint{2.556430in}{2.827443in}}%
\pgfpathlineto{\pgfqpoint{2.565016in}{2.825477in}}%
\pgfpathlineto{\pgfqpoint{2.573586in}{2.823730in}}%
\pgfpathlineto{\pgfqpoint{2.582140in}{2.822199in}}%
\pgfpathlineto{\pgfqpoint{2.590678in}{2.820882in}}%
\pgfpathlineto{\pgfqpoint{2.577156in}{2.841402in}}%
\pgfpathlineto{\pgfqpoint{2.563625in}{2.862119in}}%
\pgfpathlineto{\pgfqpoint{2.550086in}{2.883036in}}%
\pgfpathlineto{\pgfqpoint{2.536538in}{2.904152in}}%
\pgfpathlineto{\pgfqpoint{2.527956in}{2.905843in}}%
\pgfpathlineto{\pgfqpoint{2.519359in}{2.907753in}}%
\pgfpathlineto{\pgfqpoint{2.510745in}{2.909883in}}%
\pgfpathlineto{\pgfqpoint{2.502114in}{2.912238in}}%
\pgfpathclose%
\pgfusepath{fill}%
\end{pgfscope}%
\begin{pgfscope}%
\pgfpathrectangle{\pgfqpoint{1.254980in}{0.150000in}}{\pgfqpoint{5.490039in}{5.490039in}}%
\pgfusepath{clip}%
\pgfsetbuttcap%
\pgfsetroundjoin%
\definecolor{currentfill}{rgb}{0.271305,0.019942,0.347269}%
\pgfsetfillcolor{currentfill}%
\pgfsetfillopacity{0.700000}%
\pgfsetlinewidth{0.000000pt}%
\definecolor{currentstroke}{rgb}{0.000000,0.000000,0.000000}%
\pgfsetstrokecolor{currentstroke}%
\pgfsetdash{}{0pt}%
\pgfpathmoveto{\pgfqpoint{3.882548in}{1.698089in}}%
\pgfpathlineto{\pgfqpoint{3.895909in}{1.692491in}}%
\pgfpathlineto{\pgfqpoint{3.909275in}{1.687017in}}%
\pgfpathlineto{\pgfqpoint{3.922647in}{1.681668in}}%
\pgfpathlineto{\pgfqpoint{3.936023in}{1.676442in}}%
\pgfpathlineto{\pgfqpoint{3.943813in}{1.684332in}}%
\pgfpathlineto{\pgfqpoint{3.951597in}{1.692303in}}%
\pgfpathlineto{\pgfqpoint{3.959374in}{1.700355in}}%
\pgfpathlineto{\pgfqpoint{3.967146in}{1.708483in}}%
\pgfpathlineto{\pgfqpoint{3.953784in}{1.713416in}}%
\pgfpathlineto{\pgfqpoint{3.940428in}{1.718472in}}%
\pgfpathlineto{\pgfqpoint{3.927076in}{1.723653in}}%
\pgfpathlineto{\pgfqpoint{3.913730in}{1.728959in}}%
\pgfpathlineto{\pgfqpoint{3.905943in}{1.721117in}}%
\pgfpathlineto{\pgfqpoint{3.898151in}{1.713357in}}%
\pgfpathlineto{\pgfqpoint{3.890352in}{1.705680in}}%
\pgfpathlineto{\pgfqpoint{3.882548in}{1.698089in}}%
\pgfpathclose%
\pgfusepath{fill}%
\end{pgfscope}%
\begin{pgfscope}%
\pgfpathrectangle{\pgfqpoint{1.254980in}{0.150000in}}{\pgfqpoint{5.490039in}{5.490039in}}%
\pgfusepath{clip}%
\pgfsetbuttcap%
\pgfsetroundjoin%
\definecolor{currentfill}{rgb}{0.273809,0.031497,0.358853}%
\pgfsetfillcolor{currentfill}%
\pgfsetfillopacity{0.700000}%
\pgfsetlinewidth{0.000000pt}%
\definecolor{currentstroke}{rgb}{0.000000,0.000000,0.000000}%
\pgfsetstrokecolor{currentstroke}%
\pgfsetdash{}{0pt}%
\pgfpathmoveto{\pgfqpoint{3.744399in}{1.720432in}}%
\pgfpathlineto{\pgfqpoint{3.757743in}{1.713513in}}%
\pgfpathlineto{\pgfqpoint{3.771091in}{1.706721in}}%
\pgfpathlineto{\pgfqpoint{3.784443in}{1.700057in}}%
\pgfpathlineto{\pgfqpoint{3.797799in}{1.693520in}}%
\pgfpathlineto{\pgfqpoint{3.805646in}{1.700427in}}%
\pgfpathlineto{\pgfqpoint{3.813487in}{1.707434in}}%
\pgfpathlineto{\pgfqpoint{3.821321in}{1.714538in}}%
\pgfpathlineto{\pgfqpoint{3.829148in}{1.721737in}}%
\pgfpathlineto{\pgfqpoint{3.815809in}{1.727965in}}%
\pgfpathlineto{\pgfqpoint{3.802475in}{1.734319in}}%
\pgfpathlineto{\pgfqpoint{3.789144in}{1.740801in}}%
\pgfpathlineto{\pgfqpoint{3.775818in}{1.747410in}}%
\pgfpathlineto{\pgfqpoint{3.767973in}{1.740515in}}%
\pgfpathlineto{\pgfqpoint{3.760122in}{1.733719in}}%
\pgfpathlineto{\pgfqpoint{3.752264in}{1.727024in}}%
\pgfpathlineto{\pgfqpoint{3.744399in}{1.720432in}}%
\pgfpathclose%
\pgfusepath{fill}%
\end{pgfscope}%
\begin{pgfscope}%
\pgfpathrectangle{\pgfqpoint{1.254980in}{0.150000in}}{\pgfqpoint{5.490039in}{5.490039in}}%
\pgfusepath{clip}%
\pgfsetbuttcap%
\pgfsetroundjoin%
\definecolor{currentfill}{rgb}{0.281887,0.150881,0.465405}%
\pgfsetfillcolor{currentfill}%
\pgfsetfillopacity{0.700000}%
\pgfsetlinewidth{0.000000pt}%
\definecolor{currentstroke}{rgb}{0.000000,0.000000,0.000000}%
\pgfsetstrokecolor{currentstroke}%
\pgfsetdash{}{0pt}%
\pgfpathmoveto{\pgfqpoint{4.665711in}{1.919486in}}%
\pgfpathlineto{\pgfqpoint{4.679302in}{1.920550in}}%
\pgfpathlineto{\pgfqpoint{4.692903in}{1.921728in}}%
\pgfpathlineto{\pgfqpoint{4.706514in}{1.923022in}}%
\pgfpathlineto{\pgfqpoint{4.720136in}{1.924430in}}%
\pgfpathlineto{\pgfqpoint{4.727676in}{1.935684in}}%
\pgfpathlineto{\pgfqpoint{4.735212in}{1.946925in}}%
\pgfpathlineto{\pgfqpoint{4.742742in}{1.958150in}}%
\pgfpathlineto{\pgfqpoint{4.750268in}{1.969359in}}%
\pgfpathlineto{\pgfqpoint{4.736651in}{1.967784in}}%
\pgfpathlineto{\pgfqpoint{4.723045in}{1.966325in}}%
\pgfpathlineto{\pgfqpoint{4.709449in}{1.964980in}}%
\pgfpathlineto{\pgfqpoint{4.695863in}{1.963750in}}%
\pgfpathlineto{\pgfqpoint{4.688332in}{1.952702in}}%
\pgfpathlineto{\pgfqpoint{4.680797in}{1.941641in}}%
\pgfpathlineto{\pgfqpoint{4.673256in}{1.930569in}}%
\pgfpathlineto{\pgfqpoint{4.665711in}{1.919486in}}%
\pgfpathclose%
\pgfusepath{fill}%
\end{pgfscope}%
\begin{pgfscope}%
\pgfpathrectangle{\pgfqpoint{1.254980in}{0.150000in}}{\pgfqpoint{5.490039in}{5.490039in}}%
\pgfusepath{clip}%
\pgfsetbuttcap%
\pgfsetroundjoin%
\definecolor{currentfill}{rgb}{0.281446,0.084320,0.407414}%
\pgfsetfillcolor{currentfill}%
\pgfsetfillopacity{0.700000}%
\pgfsetlinewidth{0.000000pt}%
\definecolor{currentstroke}{rgb}{0.000000,0.000000,0.000000}%
\pgfsetstrokecolor{currentstroke}%
\pgfsetdash{}{0pt}%
\pgfpathmoveto{\pgfqpoint{4.412265in}{1.789559in}}%
\pgfpathlineto{\pgfqpoint{4.425761in}{1.788638in}}%
\pgfpathlineto{\pgfqpoint{4.439265in}{1.787834in}}%
\pgfpathlineto{\pgfqpoint{4.452777in}{1.787146in}}%
\pgfpathlineto{\pgfqpoint{4.466299in}{1.786575in}}%
\pgfpathlineto{\pgfqpoint{4.473912in}{1.797236in}}%
\pgfpathlineto{\pgfqpoint{4.481521in}{1.807912in}}%
\pgfpathlineto{\pgfqpoint{4.489125in}{1.818600in}}%
\pgfpathlineto{\pgfqpoint{4.496724in}{1.829300in}}%
\pgfpathlineto{\pgfqpoint{4.483209in}{1.829658in}}%
\pgfpathlineto{\pgfqpoint{4.469703in}{1.830132in}}%
\pgfpathlineto{\pgfqpoint{4.456206in}{1.830723in}}%
\pgfpathlineto{\pgfqpoint{4.442718in}{1.831431in}}%
\pgfpathlineto{\pgfqpoint{4.435112in}{1.820939in}}%
\pgfpathlineto{\pgfqpoint{4.427501in}{1.810461in}}%
\pgfpathlineto{\pgfqpoint{4.419886in}{1.800001in}}%
\pgfpathlineto{\pgfqpoint{4.412265in}{1.789559in}}%
\pgfpathclose%
\pgfusepath{fill}%
\end{pgfscope}%
\begin{pgfscope}%
\pgfpathrectangle{\pgfqpoint{1.254980in}{0.150000in}}{\pgfqpoint{5.490039in}{5.490039in}}%
\pgfusepath{clip}%
\pgfsetbuttcap%
\pgfsetroundjoin%
\definecolor{currentfill}{rgb}{0.283229,0.120777,0.440584}%
\pgfsetfillcolor{currentfill}%
\pgfsetfillopacity{0.700000}%
\pgfsetlinewidth{0.000000pt}%
\definecolor{currentstroke}{rgb}{0.000000,0.000000,0.000000}%
\pgfsetstrokecolor{currentstroke}%
\pgfsetdash{}{0pt}%
\pgfpathmoveto{\pgfqpoint{3.360732in}{1.893554in}}%
\pgfpathlineto{\pgfqpoint{3.374070in}{1.882744in}}%
\pgfpathlineto{\pgfqpoint{3.387408in}{1.872074in}}%
\pgfpathlineto{\pgfqpoint{3.400747in}{1.861545in}}%
\pgfpathlineto{\pgfqpoint{3.414086in}{1.851154in}}%
\pgfpathlineto{\pgfqpoint{3.422124in}{1.855065in}}%
\pgfpathlineto{\pgfqpoint{3.430152in}{1.859122in}}%
\pgfpathlineto{\pgfqpoint{3.438172in}{1.863322in}}%
\pgfpathlineto{\pgfqpoint{3.446183in}{1.867662in}}%
\pgfpathlineto{\pgfqpoint{3.432868in}{1.877705in}}%
\pgfpathlineto{\pgfqpoint{3.419555in}{1.887887in}}%
\pgfpathlineto{\pgfqpoint{3.406242in}{1.898209in}}%
\pgfpathlineto{\pgfqpoint{3.392930in}{1.908671in}}%
\pgfpathlineto{\pgfqpoint{3.384895in}{1.904671in}}%
\pgfpathlineto{\pgfqpoint{3.376850in}{1.900817in}}%
\pgfpathlineto{\pgfqpoint{3.368796in}{1.897110in}}%
\pgfpathlineto{\pgfqpoint{3.360732in}{1.893554in}}%
\pgfpathclose%
\pgfusepath{fill}%
\end{pgfscope}%
\begin{pgfscope}%
\pgfpathrectangle{\pgfqpoint{1.254980in}{0.150000in}}{\pgfqpoint{5.490039in}{5.490039in}}%
\pgfusepath{clip}%
\pgfsetbuttcap%
\pgfsetroundjoin%
\definecolor{currentfill}{rgb}{0.278012,0.180367,0.486697}%
\pgfsetfillcolor{currentfill}%
\pgfsetfillopacity{0.700000}%
\pgfsetlinewidth{0.000000pt}%
\definecolor{currentstroke}{rgb}{0.000000,0.000000,0.000000}%
\pgfsetstrokecolor{currentstroke}%
\pgfsetdash{}{0pt}%
\pgfpathmoveto{\pgfqpoint{4.750268in}{1.969359in}}%
\pgfpathlineto{\pgfqpoint{4.763896in}{1.971048in}}%
\pgfpathlineto{\pgfqpoint{4.777535in}{1.972851in}}%
\pgfpathlineto{\pgfqpoint{4.791185in}{1.974769in}}%
\pgfpathlineto{\pgfqpoint{4.804846in}{1.976802in}}%
\pgfpathlineto{\pgfqpoint{4.812362in}{1.988149in}}%
\pgfpathlineto{\pgfqpoint{4.819874in}{1.999474in}}%
\pgfpathlineto{\pgfqpoint{4.827381in}{2.010775in}}%
\pgfpathlineto{\pgfqpoint{4.834883in}{2.022052in}}%
\pgfpathlineto{\pgfqpoint{4.821226in}{2.019869in}}%
\pgfpathlineto{\pgfqpoint{4.807581in}{2.017800in}}%
\pgfpathlineto{\pgfqpoint{4.793947in}{2.015846in}}%
\pgfpathlineto{\pgfqpoint{4.780324in}{2.014006in}}%
\pgfpathlineto{\pgfqpoint{4.772817in}{2.002874in}}%
\pgfpathlineto{\pgfqpoint{4.765306in}{1.991722in}}%
\pgfpathlineto{\pgfqpoint{4.757789in}{1.980550in}}%
\pgfpathlineto{\pgfqpoint{4.750268in}{1.969359in}}%
\pgfpathclose%
\pgfusepath{fill}%
\end{pgfscope}%
\begin{pgfscope}%
\pgfpathrectangle{\pgfqpoint{1.254980in}{0.150000in}}{\pgfqpoint{5.490039in}{5.490039in}}%
\pgfusepath{clip}%
\pgfsetbuttcap%
\pgfsetroundjoin%
\definecolor{currentfill}{rgb}{0.223925,0.334994,0.548053}%
\pgfsetfillcolor{currentfill}%
\pgfsetfillopacity{0.700000}%
\pgfsetlinewidth{0.000000pt}%
\definecolor{currentstroke}{rgb}{0.000000,0.000000,0.000000}%
\pgfsetstrokecolor{currentstroke}%
\pgfsetdash{}{0pt}%
\pgfpathmoveto{\pgfqpoint{5.203643in}{2.299754in}}%
\pgfpathlineto{\pgfqpoint{5.217488in}{2.304381in}}%
\pgfpathlineto{\pgfqpoint{5.231347in}{2.309121in}}%
\pgfpathlineto{\pgfqpoint{5.245219in}{2.313974in}}%
\pgfpathlineto{\pgfqpoint{5.259104in}{2.318940in}}%
\pgfpathlineto{\pgfqpoint{5.266475in}{2.329828in}}%
\pgfpathlineto{\pgfqpoint{5.273839in}{2.340657in}}%
\pgfpathlineto{\pgfqpoint{5.281197in}{2.351428in}}%
\pgfpathlineto{\pgfqpoint{5.288550in}{2.362140in}}%
\pgfpathlineto{\pgfqpoint{5.274668in}{2.357121in}}%
\pgfpathlineto{\pgfqpoint{5.260801in}{2.352214in}}%
\pgfpathlineto{\pgfqpoint{5.246947in}{2.347420in}}%
\pgfpathlineto{\pgfqpoint{5.233107in}{2.342739in}}%
\pgfpathlineto{\pgfqpoint{5.225749in}{2.332075in}}%
\pgfpathlineto{\pgfqpoint{5.218386in}{2.321355in}}%
\pgfpathlineto{\pgfqpoint{5.211017in}{2.310582in}}%
\pgfpathlineto{\pgfqpoint{5.203643in}{2.299754in}}%
\pgfpathclose%
\pgfusepath{fill}%
\end{pgfscope}%
\begin{pgfscope}%
\pgfpathrectangle{\pgfqpoint{1.254980in}{0.150000in}}{\pgfqpoint{5.490039in}{5.490039in}}%
\pgfusepath{clip}%
\pgfsetbuttcap%
\pgfsetroundjoin%
\definecolor{currentfill}{rgb}{0.278791,0.062145,0.386592}%
\pgfsetfillcolor{currentfill}%
\pgfsetfillopacity{0.700000}%
\pgfsetlinewidth{0.000000pt}%
\definecolor{currentstroke}{rgb}{0.000000,0.000000,0.000000}%
\pgfsetstrokecolor{currentstroke}%
\pgfsetdash{}{0pt}%
\pgfpathmoveto{\pgfqpoint{4.327808in}{1.753785in}}%
\pgfpathlineto{\pgfqpoint{4.341278in}{1.752165in}}%
\pgfpathlineto{\pgfqpoint{4.354756in}{1.750663in}}%
\pgfpathlineto{\pgfqpoint{4.368242in}{1.749278in}}%
\pgfpathlineto{\pgfqpoint{4.381736in}{1.748010in}}%
\pgfpathlineto{\pgfqpoint{4.389376in}{1.758362in}}%
\pgfpathlineto{\pgfqpoint{4.397010in}{1.768738in}}%
\pgfpathlineto{\pgfqpoint{4.404640in}{1.779138in}}%
\pgfpathlineto{\pgfqpoint{4.412265in}{1.789559in}}%
\pgfpathlineto{\pgfqpoint{4.398778in}{1.790598in}}%
\pgfpathlineto{\pgfqpoint{4.385300in}{1.791753in}}%
\pgfpathlineto{\pgfqpoint{4.371830in}{1.793026in}}%
\pgfpathlineto{\pgfqpoint{4.358369in}{1.794417in}}%
\pgfpathlineto{\pgfqpoint{4.350736in}{1.784218in}}%
\pgfpathlineto{\pgfqpoint{4.343098in}{1.774046in}}%
\pgfpathlineto{\pgfqpoint{4.335455in}{1.763901in}}%
\pgfpathlineto{\pgfqpoint{4.327808in}{1.753785in}}%
\pgfpathclose%
\pgfusepath{fill}%
\end{pgfscope}%
\begin{pgfscope}%
\pgfpathrectangle{\pgfqpoint{1.254980in}{0.150000in}}{\pgfqpoint{5.490039in}{5.490039in}}%
\pgfusepath{clip}%
\pgfsetbuttcap%
\pgfsetroundjoin%
\definecolor{currentfill}{rgb}{0.271305,0.019942,0.347269}%
\pgfsetfillcolor{currentfill}%
\pgfsetfillopacity{0.700000}%
\pgfsetlinewidth{0.000000pt}%
\definecolor{currentstroke}{rgb}{0.000000,0.000000,0.000000}%
\pgfsetstrokecolor{currentstroke}%
\pgfsetdash{}{0pt}%
\pgfpathmoveto{\pgfqpoint{4.020649in}{1.689984in}}%
\pgfpathlineto{\pgfqpoint{4.034039in}{1.685666in}}%
\pgfpathlineto{\pgfqpoint{4.047435in}{1.681470in}}%
\pgfpathlineto{\pgfqpoint{4.060836in}{1.677395in}}%
\pgfpathlineto{\pgfqpoint{4.074244in}{1.673441in}}%
\pgfpathlineto{\pgfqpoint{4.081984in}{1.682210in}}%
\pgfpathlineto{\pgfqpoint{4.089719in}{1.691044in}}%
\pgfpathlineto{\pgfqpoint{4.097448in}{1.699940in}}%
\pgfpathlineto{\pgfqpoint{4.105171in}{1.708897in}}%
\pgfpathlineto{\pgfqpoint{4.091775in}{1.712574in}}%
\pgfpathlineto{\pgfqpoint{4.078386in}{1.716373in}}%
\pgfpathlineto{\pgfqpoint{4.065002in}{1.720292in}}%
\pgfpathlineto{\pgfqpoint{4.051625in}{1.724334in}}%
\pgfpathlineto{\pgfqpoint{4.043889in}{1.715648in}}%
\pgfpathlineto{\pgfqpoint{4.036148in}{1.707026in}}%
\pgfpathlineto{\pgfqpoint{4.028401in}{1.698471in}}%
\pgfpathlineto{\pgfqpoint{4.020649in}{1.689984in}}%
\pgfpathclose%
\pgfusepath{fill}%
\end{pgfscope}%
\begin{pgfscope}%
\pgfpathrectangle{\pgfqpoint{1.254980in}{0.150000in}}{\pgfqpoint{5.490039in}{5.490039in}}%
\pgfusepath{clip}%
\pgfsetbuttcap%
\pgfsetroundjoin%
\definecolor{currentfill}{rgb}{0.277018,0.050344,0.375715}%
\pgfsetfillcolor{currentfill}%
\pgfsetfillopacity{0.700000}%
\pgfsetlinewidth{0.000000pt}%
\definecolor{currentstroke}{rgb}{0.000000,0.000000,0.000000}%
\pgfsetstrokecolor{currentstroke}%
\pgfsetdash{}{0pt}%
\pgfpathmoveto{\pgfqpoint{3.606079in}{1.757785in}}%
\pgfpathlineto{\pgfqpoint{3.619418in}{1.749499in}}%
\pgfpathlineto{\pgfqpoint{3.632759in}{1.741344in}}%
\pgfpathlineto{\pgfqpoint{3.646103in}{1.733321in}}%
\pgfpathlineto{\pgfqpoint{3.659449in}{1.725429in}}%
\pgfpathlineto{\pgfqpoint{3.667363in}{1.731246in}}%
\pgfpathlineto{\pgfqpoint{3.675269in}{1.737182in}}%
\pgfpathlineto{\pgfqpoint{3.683167in}{1.743233in}}%
\pgfpathlineto{\pgfqpoint{3.691058in}{1.749398in}}%
\pgfpathlineto{\pgfqpoint{3.677731in}{1.756963in}}%
\pgfpathlineto{\pgfqpoint{3.664407in}{1.764659in}}%
\pgfpathlineto{\pgfqpoint{3.651087in}{1.772486in}}%
\pgfpathlineto{\pgfqpoint{3.637769in}{1.780444in}}%
\pgfpathlineto{\pgfqpoint{3.629858in}{1.774601in}}%
\pgfpathlineto{\pgfqpoint{3.621940in}{1.768875in}}%
\pgfpathlineto{\pgfqpoint{3.614013in}{1.763268in}}%
\pgfpathlineto{\pgfqpoint{3.606079in}{1.757785in}}%
\pgfpathclose%
\pgfusepath{fill}%
\end{pgfscope}%
\begin{pgfscope}%
\pgfpathrectangle{\pgfqpoint{1.254980in}{0.150000in}}{\pgfqpoint{5.490039in}{5.490039in}}%
\pgfusepath{clip}%
\pgfsetbuttcap%
\pgfsetroundjoin%
\definecolor{currentfill}{rgb}{0.273006,0.204520,0.501721}%
\pgfsetfillcolor{currentfill}%
\pgfsetfillopacity{0.700000}%
\pgfsetlinewidth{0.000000pt}%
\definecolor{currentstroke}{rgb}{0.000000,0.000000,0.000000}%
\pgfsetstrokecolor{currentstroke}%
\pgfsetdash{}{0pt}%
\pgfpathmoveto{\pgfqpoint{4.834883in}{2.022052in}}%
\pgfpathlineto{\pgfqpoint{4.848551in}{2.024349in}}%
\pgfpathlineto{\pgfqpoint{4.862230in}{2.026760in}}%
\pgfpathlineto{\pgfqpoint{4.875921in}{2.029285in}}%
\pgfpathlineto{\pgfqpoint{4.889623in}{2.031924in}}%
\pgfpathlineto{\pgfqpoint{4.897116in}{2.043316in}}%
\pgfpathlineto{\pgfqpoint{4.904604in}{2.054676in}}%
\pgfpathlineto{\pgfqpoint{4.912087in}{2.066005in}}%
\pgfpathlineto{\pgfqpoint{4.919565in}{2.077302in}}%
\pgfpathlineto{\pgfqpoint{4.905867in}{2.074528in}}%
\pgfpathlineto{\pgfqpoint{4.892180in}{2.071868in}}%
\pgfpathlineto{\pgfqpoint{4.878505in}{2.069322in}}%
\pgfpathlineto{\pgfqpoint{4.864842in}{2.066890in}}%
\pgfpathlineto{\pgfqpoint{4.857360in}{2.055722in}}%
\pgfpathlineto{\pgfqpoint{4.849872in}{2.044526in}}%
\pgfpathlineto{\pgfqpoint{4.842380in}{2.033302in}}%
\pgfpathlineto{\pgfqpoint{4.834883in}{2.022052in}}%
\pgfpathclose%
\pgfusepath{fill}%
\end{pgfscope}%
\begin{pgfscope}%
\pgfpathrectangle{\pgfqpoint{1.254980in}{0.150000in}}{\pgfqpoint{5.490039in}{5.490039in}}%
\pgfusepath{clip}%
\pgfsetbuttcap%
\pgfsetroundjoin%
\definecolor{currentfill}{rgb}{0.171176,0.452530,0.557965}%
\pgfsetfillcolor{currentfill}%
\pgfsetfillopacity{0.700000}%
\pgfsetlinewidth{0.000000pt}%
\definecolor{currentstroke}{rgb}{0.000000,0.000000,0.000000}%
\pgfsetstrokecolor{currentstroke}%
\pgfsetdash{}{0pt}%
\pgfpathmoveto{\pgfqpoint{5.572774in}{2.595222in}}%
\pgfpathlineto{\pgfqpoint{5.586820in}{2.601756in}}%
\pgfpathlineto{\pgfqpoint{5.600882in}{2.608402in}}%
\pgfpathlineto{\pgfqpoint{5.614959in}{2.615161in}}%
\pgfpathlineto{\pgfqpoint{5.629051in}{2.622033in}}%
\pgfpathlineto{\pgfqpoint{5.636272in}{2.631670in}}%
\pgfpathlineto{\pgfqpoint{5.643485in}{2.641234in}}%
\pgfpathlineto{\pgfqpoint{5.650692in}{2.650728in}}%
\pgfpathlineto{\pgfqpoint{5.657892in}{2.660151in}}%
\pgfpathlineto{\pgfqpoint{5.643807in}{2.653309in}}%
\pgfpathlineto{\pgfqpoint{5.629737in}{2.646580in}}%
\pgfpathlineto{\pgfqpoint{5.615683in}{2.639963in}}%
\pgfpathlineto{\pgfqpoint{5.601645in}{2.633459in}}%
\pgfpathlineto{\pgfqpoint{5.594437in}{2.624000in}}%
\pgfpathlineto{\pgfqpoint{5.587222in}{2.614474in}}%
\pgfpathlineto{\pgfqpoint{5.580001in}{2.604882in}}%
\pgfpathlineto{\pgfqpoint{5.572774in}{2.595222in}}%
\pgfpathclose%
\pgfusepath{fill}%
\end{pgfscope}%
\begin{pgfscope}%
\pgfpathrectangle{\pgfqpoint{1.254980in}{0.150000in}}{\pgfqpoint{5.490039in}{5.490039in}}%
\pgfusepath{clip}%
\pgfsetbuttcap%
\pgfsetroundjoin%
\definecolor{currentfill}{rgb}{0.233603,0.313828,0.543914}%
\pgfsetfillcolor{currentfill}%
\pgfsetfillopacity{0.700000}%
\pgfsetlinewidth{0.000000pt}%
\definecolor{currentstroke}{rgb}{0.000000,0.000000,0.000000}%
\pgfsetstrokecolor{currentstroke}%
\pgfsetdash{}{0pt}%
\pgfpathmoveto{\pgfqpoint{2.900086in}{2.316910in}}%
\pgfpathlineto{\pgfqpoint{2.913511in}{2.300857in}}%
\pgfpathlineto{\pgfqpoint{2.926931in}{2.284968in}}%
\pgfpathlineto{\pgfqpoint{2.940348in}{2.269244in}}%
\pgfpathlineto{\pgfqpoint{2.953761in}{2.253683in}}%
\pgfpathlineto{\pgfqpoint{2.962087in}{2.253965in}}%
\pgfpathlineto{\pgfqpoint{2.970399in}{2.254440in}}%
\pgfpathlineto{\pgfqpoint{2.978699in}{2.255106in}}%
\pgfpathlineto{\pgfqpoint{2.986986in}{2.255959in}}%
\pgfpathlineto{\pgfqpoint{2.973608in}{2.271143in}}%
\pgfpathlineto{\pgfqpoint{2.960227in}{2.286489in}}%
\pgfpathlineto{\pgfqpoint{2.946842in}{2.301999in}}%
\pgfpathlineto{\pgfqpoint{2.933454in}{2.317673in}}%
\pgfpathlineto{\pgfqpoint{2.925132in}{2.317191in}}%
\pgfpathlineto{\pgfqpoint{2.916796in}{2.316901in}}%
\pgfpathlineto{\pgfqpoint{2.908448in}{2.316806in}}%
\pgfpathlineto{\pgfqpoint{2.900086in}{2.316910in}}%
\pgfpathclose%
\pgfusepath{fill}%
\end{pgfscope}%
\begin{pgfscope}%
\pgfpathrectangle{\pgfqpoint{1.254980in}{0.150000in}}{\pgfqpoint{5.490039in}{5.490039in}}%
\pgfusepath{clip}%
\pgfsetbuttcap%
\pgfsetroundjoin%
\definecolor{currentfill}{rgb}{0.244972,0.287675,0.537260}%
\pgfsetfillcolor{currentfill}%
\pgfsetfillopacity{0.700000}%
\pgfsetlinewidth{0.000000pt}%
\definecolor{currentstroke}{rgb}{0.000000,0.000000,0.000000}%
\pgfsetstrokecolor{currentstroke}%
\pgfsetdash{}{0pt}%
\pgfpathmoveto{\pgfqpoint{2.953761in}{2.253683in}}%
\pgfpathlineto{\pgfqpoint{2.967171in}{2.238284in}}%
\pgfpathlineto{\pgfqpoint{2.980577in}{2.223047in}}%
\pgfpathlineto{\pgfqpoint{2.993980in}{2.207971in}}%
\pgfpathlineto{\pgfqpoint{3.007379in}{2.193054in}}%
\pgfpathlineto{\pgfqpoint{3.015670in}{2.193718in}}%
\pgfpathlineto{\pgfqpoint{3.023947in}{2.194572in}}%
\pgfpathlineto{\pgfqpoint{3.032213in}{2.195612in}}%
\pgfpathlineto{\pgfqpoint{3.040466in}{2.196835in}}%
\pgfpathlineto{\pgfqpoint{3.027100in}{2.211376in}}%
\pgfpathlineto{\pgfqpoint{3.013732in}{2.226077in}}%
\pgfpathlineto{\pgfqpoint{3.000361in}{2.240938in}}%
\pgfpathlineto{\pgfqpoint{2.986986in}{2.255959in}}%
\pgfpathlineto{\pgfqpoint{2.978699in}{2.255106in}}%
\pgfpathlineto{\pgfqpoint{2.970399in}{2.254440in}}%
\pgfpathlineto{\pgfqpoint{2.962087in}{2.253965in}}%
\pgfpathlineto{\pgfqpoint{2.953761in}{2.253683in}}%
\pgfpathclose%
\pgfusepath{fill}%
\end{pgfscope}%
\begin{pgfscope}%
\pgfpathrectangle{\pgfqpoint{1.254980in}{0.150000in}}{\pgfqpoint{5.490039in}{5.490039in}}%
\pgfusepath{clip}%
\pgfsetbuttcap%
\pgfsetroundjoin%
\definecolor{currentfill}{rgb}{0.276022,0.044167,0.370164}%
\pgfsetfillcolor{currentfill}%
\pgfsetfillopacity{0.700000}%
\pgfsetlinewidth{0.000000pt}%
\definecolor{currentstroke}{rgb}{0.000000,0.000000,0.000000}%
\pgfsetstrokecolor{currentstroke}%
\pgfsetdash{}{0pt}%
\pgfpathmoveto{\pgfqpoint{4.243333in}{1.722289in}}%
\pgfpathlineto{\pgfqpoint{4.256781in}{1.719952in}}%
\pgfpathlineto{\pgfqpoint{4.270236in}{1.717733in}}%
\pgfpathlineto{\pgfqpoint{4.283699in}{1.715632in}}%
\pgfpathlineto{\pgfqpoint{4.297169in}{1.713649in}}%
\pgfpathlineto{\pgfqpoint{4.304836in}{1.723630in}}%
\pgfpathlineto{\pgfqpoint{4.312498in}{1.733648in}}%
\pgfpathlineto{\pgfqpoint{4.320156in}{1.743700in}}%
\pgfpathlineto{\pgfqpoint{4.327808in}{1.753785in}}%
\pgfpathlineto{\pgfqpoint{4.314346in}{1.755523in}}%
\pgfpathlineto{\pgfqpoint{4.300892in}{1.757378in}}%
\pgfpathlineto{\pgfqpoint{4.287446in}{1.759353in}}%
\pgfpathlineto{\pgfqpoint{4.274008in}{1.761445in}}%
\pgfpathlineto{\pgfqpoint{4.266347in}{1.751599in}}%
\pgfpathlineto{\pgfqpoint{4.258681in}{1.741790in}}%
\pgfpathlineto{\pgfqpoint{4.251009in}{1.732019in}}%
\pgfpathlineto{\pgfqpoint{4.243333in}{1.722289in}}%
\pgfpathclose%
\pgfusepath{fill}%
\end{pgfscope}%
\begin{pgfscope}%
\pgfpathrectangle{\pgfqpoint{1.254980in}{0.150000in}}{\pgfqpoint{5.490039in}{5.490039in}}%
\pgfusepath{clip}%
\pgfsetbuttcap%
\pgfsetroundjoin%
\definecolor{currentfill}{rgb}{0.125394,0.574318,0.549086}%
\pgfsetfillcolor{currentfill}%
\pgfsetfillopacity{0.700000}%
\pgfsetlinewidth{0.000000pt}%
\definecolor{currentstroke}{rgb}{0.000000,0.000000,0.000000}%
\pgfsetstrokecolor{currentstroke}%
\pgfsetdash{}{0pt}%
\pgfpathmoveto{\pgfqpoint{2.447650in}{3.000270in}}%
\pgfpathlineto{\pgfqpoint{2.461280in}{2.977953in}}%
\pgfpathlineto{\pgfqpoint{2.474901in}{2.955844in}}%
\pgfpathlineto{\pgfqpoint{2.488512in}{2.933939in}}%
\pgfpathlineto{\pgfqpoint{2.502114in}{2.912238in}}%
\pgfpathlineto{\pgfqpoint{2.510745in}{2.909883in}}%
\pgfpathlineto{\pgfqpoint{2.519359in}{2.907753in}}%
\pgfpathlineto{\pgfqpoint{2.527956in}{2.905843in}}%
\pgfpathlineto{\pgfqpoint{2.536538in}{2.904152in}}%
\pgfpathlineto{\pgfqpoint{2.522981in}{2.925469in}}%
\pgfpathlineto{\pgfqpoint{2.509415in}{2.946990in}}%
\pgfpathlineto{\pgfqpoint{2.495840in}{2.968714in}}%
\pgfpathlineto{\pgfqpoint{2.482255in}{2.990645in}}%
\pgfpathlineto{\pgfqpoint{2.473629in}{2.992713in}}%
\pgfpathlineto{\pgfqpoint{2.464986in}{2.995005in}}%
\pgfpathlineto{\pgfqpoint{2.456326in}{2.997523in}}%
\pgfpathlineto{\pgfqpoint{2.447650in}{3.000270in}}%
\pgfpathclose%
\pgfusepath{fill}%
\end{pgfscope}%
\begin{pgfscope}%
\pgfpathrectangle{\pgfqpoint{1.254980in}{0.150000in}}{\pgfqpoint{5.490039in}{5.490039in}}%
\pgfusepath{clip}%
\pgfsetbuttcap%
\pgfsetroundjoin%
\definecolor{currentfill}{rgb}{0.220057,0.343307,0.549413}%
\pgfsetfillcolor{currentfill}%
\pgfsetfillopacity{0.700000}%
\pgfsetlinewidth{0.000000pt}%
\definecolor{currentstroke}{rgb}{0.000000,0.000000,0.000000}%
\pgfsetstrokecolor{currentstroke}%
\pgfsetdash{}{0pt}%
\pgfpathmoveto{\pgfqpoint{2.846344in}{2.382793in}}%
\pgfpathlineto{\pgfqpoint{2.859786in}{2.366070in}}%
\pgfpathlineto{\pgfqpoint{2.873224in}{2.349516in}}%
\pgfpathlineto{\pgfqpoint{2.886657in}{2.333130in}}%
\pgfpathlineto{\pgfqpoint{2.900086in}{2.316910in}}%
\pgfpathlineto{\pgfqpoint{2.908448in}{2.316806in}}%
\pgfpathlineto{\pgfqpoint{2.916796in}{2.316901in}}%
\pgfpathlineto{\pgfqpoint{2.925132in}{2.317191in}}%
\pgfpathlineto{\pgfqpoint{2.933454in}{2.317673in}}%
\pgfpathlineto{\pgfqpoint{2.920061in}{2.333512in}}%
\pgfpathlineto{\pgfqpoint{2.906665in}{2.349518in}}%
\pgfpathlineto{\pgfqpoint{2.893264in}{2.365692in}}%
\pgfpathlineto{\pgfqpoint{2.879859in}{2.382033in}}%
\pgfpathlineto{\pgfqpoint{2.871501in}{2.381925in}}%
\pgfpathlineto{\pgfqpoint{2.863129in}{2.382013in}}%
\pgfpathlineto{\pgfqpoint{2.854743in}{2.382302in}}%
\pgfpathlineto{\pgfqpoint{2.846344in}{2.382793in}}%
\pgfpathclose%
\pgfusepath{fill}%
\end{pgfscope}%
\begin{pgfscope}%
\pgfpathrectangle{\pgfqpoint{1.254980in}{0.150000in}}{\pgfqpoint{5.490039in}{5.490039in}}%
\pgfusepath{clip}%
\pgfsetbuttcap%
\pgfsetroundjoin%
\definecolor{currentfill}{rgb}{0.282910,0.105393,0.426902}%
\pgfsetfillcolor{currentfill}%
\pgfsetfillopacity{0.700000}%
\pgfsetlinewidth{0.000000pt}%
\definecolor{currentstroke}{rgb}{0.000000,0.000000,0.000000}%
\pgfsetstrokecolor{currentstroke}%
\pgfsetdash{}{0pt}%
\pgfpathmoveto{\pgfqpoint{3.414086in}{1.851154in}}%
\pgfpathlineto{\pgfqpoint{3.427426in}{1.840903in}}%
\pgfpathlineto{\pgfqpoint{3.440768in}{1.830789in}}%
\pgfpathlineto{\pgfqpoint{3.454110in}{1.820813in}}%
\pgfpathlineto{\pgfqpoint{3.467454in}{1.810974in}}%
\pgfpathlineto{\pgfqpoint{3.475467in}{1.815238in}}%
\pgfpathlineto{\pgfqpoint{3.483471in}{1.819643in}}%
\pgfpathlineto{\pgfqpoint{3.491466in}{1.824188in}}%
\pgfpathlineto{\pgfqpoint{3.499453in}{1.828869in}}%
\pgfpathlineto{\pgfqpoint{3.486133in}{1.838361in}}%
\pgfpathlineto{\pgfqpoint{3.472815in}{1.847991in}}%
\pgfpathlineto{\pgfqpoint{3.459498in}{1.857758in}}%
\pgfpathlineto{\pgfqpoint{3.446183in}{1.867662in}}%
\pgfpathlineto{\pgfqpoint{3.438172in}{1.863322in}}%
\pgfpathlineto{\pgfqpoint{3.430152in}{1.859122in}}%
\pgfpathlineto{\pgfqpoint{3.422124in}{1.855065in}}%
\pgfpathlineto{\pgfqpoint{3.414086in}{1.851154in}}%
\pgfpathclose%
\pgfusepath{fill}%
\end{pgfscope}%
\begin{pgfscope}%
\pgfpathrectangle{\pgfqpoint{1.254980in}{0.150000in}}{\pgfqpoint{5.490039in}{5.490039in}}%
\pgfusepath{clip}%
\pgfsetbuttcap%
\pgfsetroundjoin%
\definecolor{currentfill}{rgb}{0.253935,0.265254,0.529983}%
\pgfsetfillcolor{currentfill}%
\pgfsetfillopacity{0.700000}%
\pgfsetlinewidth{0.000000pt}%
\definecolor{currentstroke}{rgb}{0.000000,0.000000,0.000000}%
\pgfsetstrokecolor{currentstroke}%
\pgfsetdash{}{0pt}%
\pgfpathmoveto{\pgfqpoint{3.007379in}{2.193054in}}%
\pgfpathlineto{\pgfqpoint{3.020776in}{2.178295in}}%
\pgfpathlineto{\pgfqpoint{3.034170in}{2.163695in}}%
\pgfpathlineto{\pgfqpoint{3.047561in}{2.149252in}}%
\pgfpathlineto{\pgfqpoint{3.060949in}{2.134966in}}%
\pgfpathlineto{\pgfqpoint{3.069205in}{2.136011in}}%
\pgfpathlineto{\pgfqpoint{3.077449in}{2.137242in}}%
\pgfpathlineto{\pgfqpoint{3.085682in}{2.138654in}}%
\pgfpathlineto{\pgfqpoint{3.093902in}{2.140244in}}%
\pgfpathlineto{\pgfqpoint{3.080547in}{2.154157in}}%
\pgfpathlineto{\pgfqpoint{3.067189in}{2.168226in}}%
\pgfpathlineto{\pgfqpoint{3.053829in}{2.182452in}}%
\pgfpathlineto{\pgfqpoint{3.040466in}{2.196835in}}%
\pgfpathlineto{\pgfqpoint{3.032213in}{2.195612in}}%
\pgfpathlineto{\pgfqpoint{3.023947in}{2.194572in}}%
\pgfpathlineto{\pgfqpoint{3.015670in}{2.193718in}}%
\pgfpathlineto{\pgfqpoint{3.007379in}{2.193054in}}%
\pgfpathclose%
\pgfusepath{fill}%
\end{pgfscope}%
\begin{pgfscope}%
\pgfpathrectangle{\pgfqpoint{1.254980in}{0.150000in}}{\pgfqpoint{5.490039in}{5.490039in}}%
\pgfusepath{clip}%
\pgfsetbuttcap%
\pgfsetroundjoin%
\definecolor{currentfill}{rgb}{0.206756,0.371758,0.553117}%
\pgfsetfillcolor{currentfill}%
\pgfsetfillopacity{0.700000}%
\pgfsetlinewidth{0.000000pt}%
\definecolor{currentstroke}{rgb}{0.000000,0.000000,0.000000}%
\pgfsetstrokecolor{currentstroke}%
\pgfsetdash{}{0pt}%
\pgfpathmoveto{\pgfqpoint{2.792525in}{2.451394in}}%
\pgfpathlineto{\pgfqpoint{2.805988in}{2.433986in}}%
\pgfpathlineto{\pgfqpoint{2.819445in}{2.416750in}}%
\pgfpathlineto{\pgfqpoint{2.832897in}{2.399686in}}%
\pgfpathlineto{\pgfqpoint{2.846344in}{2.382793in}}%
\pgfpathlineto{\pgfqpoint{2.854743in}{2.382302in}}%
\pgfpathlineto{\pgfqpoint{2.863129in}{2.382013in}}%
\pgfpathlineto{\pgfqpoint{2.871501in}{2.381925in}}%
\pgfpathlineto{\pgfqpoint{2.879859in}{2.382033in}}%
\pgfpathlineto{\pgfqpoint{2.866450in}{2.398544in}}%
\pgfpathlineto{\pgfqpoint{2.853036in}{2.415225in}}%
\pgfpathlineto{\pgfqpoint{2.839617in}{2.432077in}}%
\pgfpathlineto{\pgfqpoint{2.826193in}{2.449102in}}%
\pgfpathlineto{\pgfqpoint{2.817797in}{2.449370in}}%
\pgfpathlineto{\pgfqpoint{2.809388in}{2.449839in}}%
\pgfpathlineto{\pgfqpoint{2.800964in}{2.450512in}}%
\pgfpathlineto{\pgfqpoint{2.792525in}{2.451394in}}%
\pgfpathclose%
\pgfusepath{fill}%
\end{pgfscope}%
\begin{pgfscope}%
\pgfpathrectangle{\pgfqpoint{1.254980in}{0.150000in}}{\pgfqpoint{5.490039in}{5.490039in}}%
\pgfusepath{clip}%
\pgfsetbuttcap%
\pgfsetroundjoin%
\definecolor{currentfill}{rgb}{0.210503,0.363727,0.552206}%
\pgfsetfillcolor{currentfill}%
\pgfsetfillopacity{0.700000}%
\pgfsetlinewidth{0.000000pt}%
\definecolor{currentstroke}{rgb}{0.000000,0.000000,0.000000}%
\pgfsetstrokecolor{currentstroke}%
\pgfsetdash{}{0pt}%
\pgfpathmoveto{\pgfqpoint{5.288550in}{2.362140in}}%
\pgfpathlineto{\pgfqpoint{5.302445in}{2.367273in}}%
\pgfpathlineto{\pgfqpoint{5.316353in}{2.372519in}}%
\pgfpathlineto{\pgfqpoint{5.330276in}{2.377877in}}%
\pgfpathlineto{\pgfqpoint{5.344214in}{2.383349in}}%
\pgfpathlineto{\pgfqpoint{5.351555in}{2.394046in}}%
\pgfpathlineto{\pgfqpoint{5.358891in}{2.404680in}}%
\pgfpathlineto{\pgfqpoint{5.366220in}{2.415251in}}%
\pgfpathlineto{\pgfqpoint{5.373544in}{2.425760in}}%
\pgfpathlineto{\pgfqpoint{5.359612in}{2.420251in}}%
\pgfpathlineto{\pgfqpoint{5.345694in}{2.414856in}}%
\pgfpathlineto{\pgfqpoint{5.331790in}{2.409573in}}%
\pgfpathlineto{\pgfqpoint{5.317900in}{2.404403in}}%
\pgfpathlineto{\pgfqpoint{5.310571in}{2.393925in}}%
\pgfpathlineto{\pgfqpoint{5.303237in}{2.383389in}}%
\pgfpathlineto{\pgfqpoint{5.295896in}{2.372794in}}%
\pgfpathlineto{\pgfqpoint{5.288550in}{2.362140in}}%
\pgfpathclose%
\pgfusepath{fill}%
\end{pgfscope}%
\begin{pgfscope}%
\pgfpathrectangle{\pgfqpoint{1.254980in}{0.150000in}}{\pgfqpoint{5.490039in}{5.490039in}}%
\pgfusepath{clip}%
\pgfsetbuttcap%
\pgfsetroundjoin%
\definecolor{currentfill}{rgb}{0.263663,0.237631,0.518762}%
\pgfsetfillcolor{currentfill}%
\pgfsetfillopacity{0.700000}%
\pgfsetlinewidth{0.000000pt}%
\definecolor{currentstroke}{rgb}{0.000000,0.000000,0.000000}%
\pgfsetstrokecolor{currentstroke}%
\pgfsetdash{}{0pt}%
\pgfpathmoveto{\pgfqpoint{4.919565in}{2.077302in}}%
\pgfpathlineto{\pgfqpoint{4.933275in}{2.080190in}}%
\pgfpathlineto{\pgfqpoint{4.946997in}{2.083191in}}%
\pgfpathlineto{\pgfqpoint{4.960731in}{2.086307in}}%
\pgfpathlineto{\pgfqpoint{4.974477in}{2.089536in}}%
\pgfpathlineto{\pgfqpoint{4.981946in}{2.100924in}}%
\pgfpathlineto{\pgfqpoint{4.989410in}{2.112274in}}%
\pgfpathlineto{\pgfqpoint{4.996869in}{2.123585in}}%
\pgfpathlineto{\pgfqpoint{5.004323in}{2.134857in}}%
\pgfpathlineto{\pgfqpoint{4.990580in}{2.131509in}}%
\pgfpathlineto{\pgfqpoint{4.976850in}{2.128275in}}%
\pgfpathlineto{\pgfqpoint{4.963133in}{2.125155in}}%
\pgfpathlineto{\pgfqpoint{4.949427in}{2.122148in}}%
\pgfpathlineto{\pgfqpoint{4.941969in}{2.110989in}}%
\pgfpathlineto{\pgfqpoint{4.934506in}{2.099795in}}%
\pgfpathlineto{\pgfqpoint{4.927038in}{2.088565in}}%
\pgfpathlineto{\pgfqpoint{4.919565in}{2.077302in}}%
\pgfpathclose%
\pgfusepath{fill}%
\end{pgfscope}%
\begin{pgfscope}%
\pgfpathrectangle{\pgfqpoint{1.254980in}{0.150000in}}{\pgfqpoint{5.490039in}{5.490039in}}%
\pgfusepath{clip}%
\pgfsetbuttcap%
\pgfsetroundjoin%
\definecolor{currentfill}{rgb}{0.263663,0.237631,0.518762}%
\pgfsetfillcolor{currentfill}%
\pgfsetfillopacity{0.700000}%
\pgfsetlinewidth{0.000000pt}%
\definecolor{currentstroke}{rgb}{0.000000,0.000000,0.000000}%
\pgfsetstrokecolor{currentstroke}%
\pgfsetdash{}{0pt}%
\pgfpathmoveto{\pgfqpoint{3.060949in}{2.134966in}}%
\pgfpathlineto{\pgfqpoint{3.074335in}{2.120835in}}%
\pgfpathlineto{\pgfqpoint{3.087719in}{2.106858in}}%
\pgfpathlineto{\pgfqpoint{3.101100in}{2.093036in}}%
\pgfpathlineto{\pgfqpoint{3.114479in}{2.079367in}}%
\pgfpathlineto{\pgfqpoint{3.122703in}{2.080792in}}%
\pgfpathlineto{\pgfqpoint{3.130915in}{2.082397in}}%
\pgfpathlineto{\pgfqpoint{3.139115in}{2.084178in}}%
\pgfpathlineto{\pgfqpoint{3.147304in}{2.086134in}}%
\pgfpathlineto{\pgfqpoint{3.133956in}{2.099432in}}%
\pgfpathlineto{\pgfqpoint{3.120607in}{2.112882in}}%
\pgfpathlineto{\pgfqpoint{3.107255in}{2.126486in}}%
\pgfpathlineto{\pgfqpoint{3.093902in}{2.140244in}}%
\pgfpathlineto{\pgfqpoint{3.085682in}{2.138654in}}%
\pgfpathlineto{\pgfqpoint{3.077449in}{2.137242in}}%
\pgfpathlineto{\pgfqpoint{3.069205in}{2.136011in}}%
\pgfpathlineto{\pgfqpoint{3.060949in}{2.134966in}}%
\pgfpathclose%
\pgfusepath{fill}%
\end{pgfscope}%
\begin{pgfscope}%
\pgfpathrectangle{\pgfqpoint{1.254980in}{0.150000in}}{\pgfqpoint{5.490039in}{5.490039in}}%
\pgfusepath{clip}%
\pgfsetbuttcap%
\pgfsetroundjoin%
\definecolor{currentfill}{rgb}{0.194100,0.399323,0.555565}%
\pgfsetfillcolor{currentfill}%
\pgfsetfillopacity{0.700000}%
\pgfsetlinewidth{0.000000pt}%
\definecolor{currentstroke}{rgb}{0.000000,0.000000,0.000000}%
\pgfsetstrokecolor{currentstroke}%
\pgfsetdash{}{0pt}%
\pgfpathmoveto{\pgfqpoint{2.738620in}{2.522778in}}%
\pgfpathlineto{\pgfqpoint{2.752105in}{2.504668in}}%
\pgfpathlineto{\pgfqpoint{2.765584in}{2.486734in}}%
\pgfpathlineto{\pgfqpoint{2.779058in}{2.468977in}}%
\pgfpathlineto{\pgfqpoint{2.792525in}{2.451394in}}%
\pgfpathlineto{\pgfqpoint{2.800964in}{2.450512in}}%
\pgfpathlineto{\pgfqpoint{2.809388in}{2.449839in}}%
\pgfpathlineto{\pgfqpoint{2.817797in}{2.449370in}}%
\pgfpathlineto{\pgfqpoint{2.826193in}{2.449102in}}%
\pgfpathlineto{\pgfqpoint{2.812765in}{2.466300in}}%
\pgfpathlineto{\pgfqpoint{2.799330in}{2.483672in}}%
\pgfpathlineto{\pgfqpoint{2.785891in}{2.501219in}}%
\pgfpathlineto{\pgfqpoint{2.772446in}{2.518943in}}%
\pgfpathlineto{\pgfqpoint{2.764011in}{2.519589in}}%
\pgfpathlineto{\pgfqpoint{2.755562in}{2.520442in}}%
\pgfpathlineto{\pgfqpoint{2.747099in}{2.521504in}}%
\pgfpathlineto{\pgfqpoint{2.738620in}{2.522778in}}%
\pgfpathclose%
\pgfusepath{fill}%
\end{pgfscope}%
\begin{pgfscope}%
\pgfpathrectangle{\pgfqpoint{1.254980in}{0.150000in}}{\pgfqpoint{5.490039in}{5.490039in}}%
\pgfusepath{clip}%
\pgfsetbuttcap%
\pgfsetroundjoin%
\definecolor{currentfill}{rgb}{0.271305,0.019942,0.347269}%
\pgfsetfillcolor{currentfill}%
\pgfsetfillopacity{0.700000}%
\pgfsetlinewidth{0.000000pt}%
\definecolor{currentstroke}{rgb}{0.000000,0.000000,0.000000}%
\pgfsetstrokecolor{currentstroke}%
\pgfsetdash{}{0pt}%
\pgfpathmoveto{\pgfqpoint{3.797799in}{1.693520in}}%
\pgfpathlineto{\pgfqpoint{3.811159in}{1.687109in}}%
\pgfpathlineto{\pgfqpoint{3.824523in}{1.680825in}}%
\pgfpathlineto{\pgfqpoint{3.837891in}{1.674667in}}%
\pgfpathlineto{\pgfqpoint{3.851264in}{1.668634in}}%
\pgfpathlineto{\pgfqpoint{3.859094in}{1.675857in}}%
\pgfpathlineto{\pgfqpoint{3.866919in}{1.683175in}}%
\pgfpathlineto{\pgfqpoint{3.874736in}{1.690587in}}%
\pgfpathlineto{\pgfqpoint{3.882548in}{1.698089in}}%
\pgfpathlineto{\pgfqpoint{3.869191in}{1.703813in}}%
\pgfpathlineto{\pgfqpoint{3.855839in}{1.709662in}}%
\pgfpathlineto{\pgfqpoint{3.842491in}{1.715637in}}%
\pgfpathlineto{\pgfqpoint{3.829148in}{1.721737in}}%
\pgfpathlineto{\pgfqpoint{3.821321in}{1.714538in}}%
\pgfpathlineto{\pgfqpoint{3.813487in}{1.707434in}}%
\pgfpathlineto{\pgfqpoint{3.805646in}{1.700427in}}%
\pgfpathlineto{\pgfqpoint{3.797799in}{1.693520in}}%
\pgfpathclose%
\pgfusepath{fill}%
\end{pgfscope}%
\begin{pgfscope}%
\pgfpathrectangle{\pgfqpoint{1.254980in}{0.150000in}}{\pgfqpoint{5.490039in}{5.490039in}}%
\pgfusepath{clip}%
\pgfsetbuttcap%
\pgfsetroundjoin%
\definecolor{currentfill}{rgb}{0.160665,0.478540,0.558115}%
\pgfsetfillcolor{currentfill}%
\pgfsetfillopacity{0.700000}%
\pgfsetlinewidth{0.000000pt}%
\definecolor{currentstroke}{rgb}{0.000000,0.000000,0.000000}%
\pgfsetstrokecolor{currentstroke}%
\pgfsetdash{}{0pt}%
\pgfpathmoveto{\pgfqpoint{5.657892in}{2.660151in}}%
\pgfpathlineto{\pgfqpoint{5.671993in}{2.667105in}}%
\pgfpathlineto{\pgfqpoint{5.686109in}{2.674172in}}%
\pgfpathlineto{\pgfqpoint{5.700241in}{2.681351in}}%
\pgfpathlineto{\pgfqpoint{5.714389in}{2.688643in}}%
\pgfpathlineto{\pgfqpoint{5.721574in}{2.697955in}}%
\pgfpathlineto{\pgfqpoint{5.728752in}{2.707194in}}%
\pgfpathlineto{\pgfqpoint{5.735923in}{2.716360in}}%
\pgfpathlineto{\pgfqpoint{5.743086in}{2.725454in}}%
\pgfpathlineto{\pgfqpoint{5.728947in}{2.718209in}}%
\pgfpathlineto{\pgfqpoint{5.714823in}{2.711077in}}%
\pgfpathlineto{\pgfqpoint{5.700715in}{2.704058in}}%
\pgfpathlineto{\pgfqpoint{5.686623in}{2.697150in}}%
\pgfpathlineto{\pgfqpoint{5.679450in}{2.688003in}}%
\pgfpathlineto{\pgfqpoint{5.672271in}{2.678788in}}%
\pgfpathlineto{\pgfqpoint{5.665085in}{2.669504in}}%
\pgfpathlineto{\pgfqpoint{5.657892in}{2.660151in}}%
\pgfpathclose%
\pgfusepath{fill}%
\end{pgfscope}%
\begin{pgfscope}%
\pgfpathrectangle{\pgfqpoint{1.254980in}{0.150000in}}{\pgfqpoint{5.490039in}{5.490039in}}%
\pgfusepath{clip}%
\pgfsetbuttcap%
\pgfsetroundjoin%
\definecolor{currentfill}{rgb}{0.273809,0.031497,0.358853}%
\pgfsetfillcolor{currentfill}%
\pgfsetfillopacity{0.700000}%
\pgfsetlinewidth{0.000000pt}%
\definecolor{currentstroke}{rgb}{0.000000,0.000000,0.000000}%
\pgfsetstrokecolor{currentstroke}%
\pgfsetdash{}{0pt}%
\pgfpathmoveto{\pgfqpoint{4.158820in}{1.695396in}}%
\pgfpathlineto{\pgfqpoint{4.172249in}{1.692321in}}%
\pgfpathlineto{\pgfqpoint{4.185685in}{1.689366in}}%
\pgfpathlineto{\pgfqpoint{4.199128in}{1.686530in}}%
\pgfpathlineto{\pgfqpoint{4.212578in}{1.683812in}}%
\pgfpathlineto{\pgfqpoint{4.220274in}{1.693361in}}%
\pgfpathlineto{\pgfqpoint{4.227966in}{1.702958in}}%
\pgfpathlineto{\pgfqpoint{4.235652in}{1.712602in}}%
\pgfpathlineto{\pgfqpoint{4.243333in}{1.722289in}}%
\pgfpathlineto{\pgfqpoint{4.229893in}{1.724746in}}%
\pgfpathlineto{\pgfqpoint{4.216460in}{1.727321in}}%
\pgfpathlineto{\pgfqpoint{4.203035in}{1.730016in}}%
\pgfpathlineto{\pgfqpoint{4.189616in}{1.732830in}}%
\pgfpathlineto{\pgfqpoint{4.181925in}{1.723397in}}%
\pgfpathlineto{\pgfqpoint{4.174228in}{1.714013in}}%
\pgfpathlineto{\pgfqpoint{4.166527in}{1.704678in}}%
\pgfpathlineto{\pgfqpoint{4.158820in}{1.695396in}}%
\pgfpathclose%
\pgfusepath{fill}%
\end{pgfscope}%
\begin{pgfscope}%
\pgfpathrectangle{\pgfqpoint{1.254980in}{0.150000in}}{\pgfqpoint{5.490039in}{5.490039in}}%
\pgfusepath{clip}%
\pgfsetbuttcap%
\pgfsetroundjoin%
\definecolor{currentfill}{rgb}{0.270595,0.214069,0.507052}%
\pgfsetfillcolor{currentfill}%
\pgfsetfillopacity{0.700000}%
\pgfsetlinewidth{0.000000pt}%
\definecolor{currentstroke}{rgb}{0.000000,0.000000,0.000000}%
\pgfsetstrokecolor{currentstroke}%
\pgfsetdash{}{0pt}%
\pgfpathmoveto{\pgfqpoint{3.114479in}{2.079367in}}%
\pgfpathlineto{\pgfqpoint{3.127857in}{2.065850in}}%
\pgfpathlineto{\pgfqpoint{3.141233in}{2.052485in}}%
\pgfpathlineto{\pgfqpoint{3.154607in}{2.039271in}}%
\pgfpathlineto{\pgfqpoint{3.167979in}{2.026206in}}%
\pgfpathlineto{\pgfqpoint{3.176171in}{2.028009in}}%
\pgfpathlineto{\pgfqpoint{3.184351in}{2.029986in}}%
\pgfpathlineto{\pgfqpoint{3.192521in}{2.032137in}}%
\pgfpathlineto{\pgfqpoint{3.200679in}{2.034457in}}%
\pgfpathlineto{\pgfqpoint{3.187337in}{2.047151in}}%
\pgfpathlineto{\pgfqpoint{3.173994in}{2.059995in}}%
\pgfpathlineto{\pgfqpoint{3.160650in}{2.072989in}}%
\pgfpathlineto{\pgfqpoint{3.147304in}{2.086134in}}%
\pgfpathlineto{\pgfqpoint{3.139115in}{2.084178in}}%
\pgfpathlineto{\pgfqpoint{3.130915in}{2.082397in}}%
\pgfpathlineto{\pgfqpoint{3.122703in}{2.080792in}}%
\pgfpathlineto{\pgfqpoint{3.114479in}{2.079367in}}%
\pgfpathclose%
\pgfusepath{fill}%
\end{pgfscope}%
\begin{pgfscope}%
\pgfpathrectangle{\pgfqpoint{1.254980in}{0.150000in}}{\pgfqpoint{5.490039in}{5.490039in}}%
\pgfusepath{clip}%
\pgfsetbuttcap%
\pgfsetroundjoin%
\definecolor{currentfill}{rgb}{0.253935,0.265254,0.529983}%
\pgfsetfillcolor{currentfill}%
\pgfsetfillopacity{0.700000}%
\pgfsetlinewidth{0.000000pt}%
\definecolor{currentstroke}{rgb}{0.000000,0.000000,0.000000}%
\pgfsetstrokecolor{currentstroke}%
\pgfsetdash{}{0pt}%
\pgfpathmoveto{\pgfqpoint{5.004323in}{2.134857in}}%
\pgfpathlineto{\pgfqpoint{5.018077in}{2.138318in}}%
\pgfpathlineto{\pgfqpoint{5.031844in}{2.141893in}}%
\pgfpathlineto{\pgfqpoint{5.045623in}{2.145581in}}%
\pgfpathlineto{\pgfqpoint{5.059415in}{2.149382in}}%
\pgfpathlineto{\pgfqpoint{5.066860in}{2.160723in}}%
\pgfpathlineto{\pgfqpoint{5.074299in}{2.172018in}}%
\pgfpathlineto{\pgfqpoint{5.081733in}{2.183267in}}%
\pgfpathlineto{\pgfqpoint{5.089162in}{2.194471in}}%
\pgfpathlineto{\pgfqpoint{5.075374in}{2.190567in}}%
\pgfpathlineto{\pgfqpoint{5.061598in}{2.186776in}}%
\pgfpathlineto{\pgfqpoint{5.047835in}{2.183098in}}%
\pgfpathlineto{\pgfqpoint{5.034085in}{2.179534in}}%
\pgfpathlineto{\pgfqpoint{5.026652in}{2.168428in}}%
\pgfpathlineto{\pgfqpoint{5.019214in}{2.157279in}}%
\pgfpathlineto{\pgfqpoint{5.011771in}{2.146088in}}%
\pgfpathlineto{\pgfqpoint{5.004323in}{2.134857in}}%
\pgfpathclose%
\pgfusepath{fill}%
\end{pgfscope}%
\begin{pgfscope}%
\pgfpathrectangle{\pgfqpoint{1.254980in}{0.150000in}}{\pgfqpoint{5.490039in}{5.490039in}}%
\pgfusepath{clip}%
\pgfsetbuttcap%
\pgfsetroundjoin%
\definecolor{currentfill}{rgb}{0.269944,0.014625,0.341379}%
\pgfsetfillcolor{currentfill}%
\pgfsetfillopacity{0.700000}%
\pgfsetlinewidth{0.000000pt}%
\definecolor{currentstroke}{rgb}{0.000000,0.000000,0.000000}%
\pgfsetstrokecolor{currentstroke}%
\pgfsetdash{}{0pt}%
\pgfpathmoveto{\pgfqpoint{3.936023in}{1.676442in}}%
\pgfpathlineto{\pgfqpoint{3.949404in}{1.671340in}}%
\pgfpathlineto{\pgfqpoint{3.962791in}{1.666361in}}%
\pgfpathlineto{\pgfqpoint{3.976183in}{1.661505in}}%
\pgfpathlineto{\pgfqpoint{3.989581in}{1.656772in}}%
\pgfpathlineto{\pgfqpoint{3.997357in}{1.664961in}}%
\pgfpathlineto{\pgfqpoint{4.005127in}{1.673227in}}%
\pgfpathlineto{\pgfqpoint{4.012891in}{1.681569in}}%
\pgfpathlineto{\pgfqpoint{4.020649in}{1.689984in}}%
\pgfpathlineto{\pgfqpoint{4.007265in}{1.694425in}}%
\pgfpathlineto{\pgfqpoint{3.993886in}{1.698988in}}%
\pgfpathlineto{\pgfqpoint{3.980514in}{1.703674in}}%
\pgfpathlineto{\pgfqpoint{3.967146in}{1.708483in}}%
\pgfpathlineto{\pgfqpoint{3.959374in}{1.700355in}}%
\pgfpathlineto{\pgfqpoint{3.951597in}{1.692303in}}%
\pgfpathlineto{\pgfqpoint{3.943813in}{1.684332in}}%
\pgfpathlineto{\pgfqpoint{3.936023in}{1.676442in}}%
\pgfpathclose%
\pgfusepath{fill}%
\end{pgfscope}%
\begin{pgfscope}%
\pgfpathrectangle{\pgfqpoint{1.254980in}{0.150000in}}{\pgfqpoint{5.490039in}{5.490039in}}%
\pgfusepath{clip}%
\pgfsetbuttcap%
\pgfsetroundjoin%
\definecolor{currentfill}{rgb}{0.180629,0.429975,0.557282}%
\pgfsetfillcolor{currentfill}%
\pgfsetfillopacity{0.700000}%
\pgfsetlinewidth{0.000000pt}%
\definecolor{currentstroke}{rgb}{0.000000,0.000000,0.000000}%
\pgfsetstrokecolor{currentstroke}%
\pgfsetdash{}{0pt}%
\pgfpathmoveto{\pgfqpoint{2.684618in}{2.597013in}}%
\pgfpathlineto{\pgfqpoint{2.698128in}{2.578183in}}%
\pgfpathlineto{\pgfqpoint{2.711632in}{2.559535in}}%
\pgfpathlineto{\pgfqpoint{2.725129in}{2.541067in}}%
\pgfpathlineto{\pgfqpoint{2.738620in}{2.522778in}}%
\pgfpathlineto{\pgfqpoint{2.747099in}{2.521504in}}%
\pgfpathlineto{\pgfqpoint{2.755562in}{2.520442in}}%
\pgfpathlineto{\pgfqpoint{2.764011in}{2.519589in}}%
\pgfpathlineto{\pgfqpoint{2.772446in}{2.518943in}}%
\pgfpathlineto{\pgfqpoint{2.758995in}{2.536844in}}%
\pgfpathlineto{\pgfqpoint{2.745538in}{2.554924in}}%
\pgfpathlineto{\pgfqpoint{2.732075in}{2.573184in}}%
\pgfpathlineto{\pgfqpoint{2.718606in}{2.591625in}}%
\pgfpathlineto{\pgfqpoint{2.710132in}{2.592652in}}%
\pgfpathlineto{\pgfqpoint{2.701642in}{2.593890in}}%
\pgfpathlineto{\pgfqpoint{2.693138in}{2.595343in}}%
\pgfpathlineto{\pgfqpoint{2.684618in}{2.597013in}}%
\pgfpathclose%
\pgfusepath{fill}%
\end{pgfscope}%
\begin{pgfscope}%
\pgfpathrectangle{\pgfqpoint{1.254980in}{0.150000in}}{\pgfqpoint{5.490039in}{5.490039in}}%
\pgfusepath{clip}%
\pgfsetbuttcap%
\pgfsetroundjoin%
\definecolor{currentfill}{rgb}{0.197636,0.391528,0.554969}%
\pgfsetfillcolor{currentfill}%
\pgfsetfillopacity{0.700000}%
\pgfsetlinewidth{0.000000pt}%
\definecolor{currentstroke}{rgb}{0.000000,0.000000,0.000000}%
\pgfsetstrokecolor{currentstroke}%
\pgfsetdash{}{0pt}%
\pgfpathmoveto{\pgfqpoint{5.373544in}{2.425760in}}%
\pgfpathlineto{\pgfqpoint{5.387490in}{2.431381in}}%
\pgfpathlineto{\pgfqpoint{5.401451in}{2.437115in}}%
\pgfpathlineto{\pgfqpoint{5.415426in}{2.442962in}}%
\pgfpathlineto{\pgfqpoint{5.429416in}{2.448922in}}%
\pgfpathlineto{\pgfqpoint{5.436728in}{2.459394in}}%
\pgfpathlineto{\pgfqpoint{5.444034in}{2.469800in}}%
\pgfpathlineto{\pgfqpoint{5.451333in}{2.480138in}}%
\pgfpathlineto{\pgfqpoint{5.458626in}{2.490410in}}%
\pgfpathlineto{\pgfqpoint{5.444642in}{2.484430in}}%
\pgfpathlineto{\pgfqpoint{5.430672in}{2.478562in}}%
\pgfpathlineto{\pgfqpoint{5.416717in}{2.472808in}}%
\pgfpathlineto{\pgfqpoint{5.402776in}{2.467166in}}%
\pgfpathlineto{\pgfqpoint{5.395477in}{2.456908in}}%
\pgfpathlineto{\pgfqpoint{5.388172in}{2.446588in}}%
\pgfpathlineto{\pgfqpoint{5.380861in}{2.436205in}}%
\pgfpathlineto{\pgfqpoint{5.373544in}{2.425760in}}%
\pgfpathclose%
\pgfusepath{fill}%
\end{pgfscope}%
\begin{pgfscope}%
\pgfpathrectangle{\pgfqpoint{1.254980in}{0.150000in}}{\pgfqpoint{5.490039in}{5.490039in}}%
\pgfusepath{clip}%
\pgfsetbuttcap%
\pgfsetroundjoin%
\definecolor{currentfill}{rgb}{0.150476,0.504369,0.557430}%
\pgfsetfillcolor{currentfill}%
\pgfsetfillopacity{0.700000}%
\pgfsetlinewidth{0.000000pt}%
\definecolor{currentstroke}{rgb}{0.000000,0.000000,0.000000}%
\pgfsetstrokecolor{currentstroke}%
\pgfsetdash{}{0pt}%
\pgfpathmoveto{\pgfqpoint{5.743086in}{2.725454in}}%
\pgfpathlineto{\pgfqpoint{5.757242in}{2.732810in}}%
\pgfpathlineto{\pgfqpoint{5.771414in}{2.740280in}}%
\pgfpathlineto{\pgfqpoint{5.785602in}{2.747861in}}%
\pgfpathlineto{\pgfqpoint{5.792752in}{2.756840in}}%
\pgfpathlineto{\pgfqpoint{5.799895in}{2.765746in}}%
\pgfpathlineto{\pgfqpoint{5.807030in}{2.774578in}}%
\pgfpathlineto{\pgfqpoint{5.814158in}{2.783339in}}%
\pgfpathlineto{\pgfqpoint{5.799979in}{2.775822in}}%
\pgfpathlineto{\pgfqpoint{5.785817in}{2.768417in}}%
\pgfpathlineto{\pgfqpoint{5.771670in}{2.761124in}}%
\pgfpathlineto{\pgfqpoint{5.764535in}{2.752311in}}%
\pgfpathlineto{\pgfqpoint{5.757392in}{2.743428in}}%
\pgfpathlineto{\pgfqpoint{5.750243in}{2.734476in}}%
\pgfpathlineto{\pgfqpoint{5.743086in}{2.725454in}}%
\pgfpathclose%
\pgfusepath{fill}%
\end{pgfscope}%
\begin{pgfscope}%
\pgfpathrectangle{\pgfqpoint{1.254980in}{0.150000in}}{\pgfqpoint{5.490039in}{5.490039in}}%
\pgfusepath{clip}%
\pgfsetbuttcap%
\pgfsetroundjoin%
\definecolor{currentfill}{rgb}{0.276022,0.044167,0.370164}%
\pgfsetfillcolor{currentfill}%
\pgfsetfillopacity{0.700000}%
\pgfsetlinewidth{0.000000pt}%
\definecolor{currentstroke}{rgb}{0.000000,0.000000,0.000000}%
\pgfsetstrokecolor{currentstroke}%
\pgfsetdash{}{0pt}%
\pgfpathmoveto{\pgfqpoint{3.659449in}{1.725429in}}%
\pgfpathlineto{\pgfqpoint{3.672799in}{1.717666in}}%
\pgfpathlineto{\pgfqpoint{3.686152in}{1.710034in}}%
\pgfpathlineto{\pgfqpoint{3.699508in}{1.702531in}}%
\pgfpathlineto{\pgfqpoint{3.712867in}{1.695156in}}%
\pgfpathlineto{\pgfqpoint{3.720761in}{1.701307in}}%
\pgfpathlineto{\pgfqpoint{3.728648in}{1.707571in}}%
\pgfpathlineto{\pgfqpoint{3.736527in}{1.713947in}}%
\pgfpathlineto{\pgfqpoint{3.744399in}{1.720432in}}%
\pgfpathlineto{\pgfqpoint{3.731059in}{1.727480in}}%
\pgfpathlineto{\pgfqpoint{3.717722in}{1.734657in}}%
\pgfpathlineto{\pgfqpoint{3.704388in}{1.741962in}}%
\pgfpathlineto{\pgfqpoint{3.691058in}{1.749398in}}%
\pgfpathlineto{\pgfqpoint{3.683167in}{1.743233in}}%
\pgfpathlineto{\pgfqpoint{3.675269in}{1.737182in}}%
\pgfpathlineto{\pgfqpoint{3.667363in}{1.731246in}}%
\pgfpathlineto{\pgfqpoint{3.659449in}{1.725429in}}%
\pgfpathclose%
\pgfusepath{fill}%
\end{pgfscope}%
\begin{pgfscope}%
\pgfpathrectangle{\pgfqpoint{1.254980in}{0.150000in}}{\pgfqpoint{5.490039in}{5.490039in}}%
\pgfusepath{clip}%
\pgfsetbuttcap%
\pgfsetroundjoin%
\definecolor{currentfill}{rgb}{0.281446,0.084320,0.407414}%
\pgfsetfillcolor{currentfill}%
\pgfsetfillopacity{0.700000}%
\pgfsetlinewidth{0.000000pt}%
\definecolor{currentstroke}{rgb}{0.000000,0.000000,0.000000}%
\pgfsetstrokecolor{currentstroke}%
\pgfsetdash{}{0pt}%
\pgfpathmoveto{\pgfqpoint{3.467454in}{1.810974in}}%
\pgfpathlineto{\pgfqpoint{3.480799in}{1.801272in}}%
\pgfpathlineto{\pgfqpoint{3.494145in}{1.791705in}}%
\pgfpathlineto{\pgfqpoint{3.507493in}{1.782274in}}%
\pgfpathlineto{\pgfqpoint{3.520843in}{1.772977in}}%
\pgfpathlineto{\pgfqpoint{3.528833in}{1.777592in}}%
\pgfpathlineto{\pgfqpoint{3.536813in}{1.782346in}}%
\pgfpathlineto{\pgfqpoint{3.544786in}{1.787234in}}%
\pgfpathlineto{\pgfqpoint{3.552750in}{1.792254in}}%
\pgfpathlineto{\pgfqpoint{3.539423in}{1.801205in}}%
\pgfpathlineto{\pgfqpoint{3.526098in}{1.810291in}}%
\pgfpathlineto{\pgfqpoint{3.512775in}{1.819512in}}%
\pgfpathlineto{\pgfqpoint{3.499453in}{1.828869in}}%
\pgfpathlineto{\pgfqpoint{3.491466in}{1.824188in}}%
\pgfpathlineto{\pgfqpoint{3.483471in}{1.819643in}}%
\pgfpathlineto{\pgfqpoint{3.475467in}{1.815238in}}%
\pgfpathlineto{\pgfqpoint{3.467454in}{1.810974in}}%
\pgfpathclose%
\pgfusepath{fill}%
\end{pgfscope}%
\begin{pgfscope}%
\pgfpathrectangle{\pgfqpoint{1.254980in}{0.150000in}}{\pgfqpoint{5.490039in}{5.490039in}}%
\pgfusepath{clip}%
\pgfsetbuttcap%
\pgfsetroundjoin%
\definecolor{currentfill}{rgb}{0.119423,0.611141,0.538982}%
\pgfsetfillcolor{currentfill}%
\pgfsetfillopacity{0.700000}%
\pgfsetlinewidth{0.000000pt}%
\definecolor{currentstroke}{rgb}{0.000000,0.000000,0.000000}%
\pgfsetstrokecolor{currentstroke}%
\pgfsetdash{}{0pt}%
\pgfpathmoveto{\pgfqpoint{2.393024in}{3.091630in}}%
\pgfpathlineto{\pgfqpoint{2.406696in}{3.068472in}}%
\pgfpathlineto{\pgfqpoint{2.420358in}{3.045528in}}%
\pgfpathlineto{\pgfqpoint{2.434009in}{3.022794in}}%
\pgfpathlineto{\pgfqpoint{2.447650in}{3.000270in}}%
\pgfpathlineto{\pgfqpoint{2.456326in}{2.997523in}}%
\pgfpathlineto{\pgfqpoint{2.464986in}{2.995005in}}%
\pgfpathlineto{\pgfqpoint{2.473629in}{2.992713in}}%
\pgfpathlineto{\pgfqpoint{2.482255in}{2.990645in}}%
\pgfpathlineto{\pgfqpoint{2.468660in}{3.012782in}}%
\pgfpathlineto{\pgfqpoint{2.455056in}{3.035128in}}%
\pgfpathlineto{\pgfqpoint{2.441441in}{3.057684in}}%
\pgfpathlineto{\pgfqpoint{2.427816in}{3.080452in}}%
\pgfpathlineto{\pgfqpoint{2.419144in}{3.082901in}}%
\pgfpathlineto{\pgfqpoint{2.410455in}{3.085579in}}%
\pgfpathlineto{\pgfqpoint{2.401748in}{3.088487in}}%
\pgfpathlineto{\pgfqpoint{2.393024in}{3.091630in}}%
\pgfpathclose%
\pgfusepath{fill}%
\end{pgfscope}%
\begin{pgfscope}%
\pgfpathrectangle{\pgfqpoint{1.254980in}{0.150000in}}{\pgfqpoint{5.490039in}{5.490039in}}%
\pgfusepath{clip}%
\pgfsetbuttcap%
\pgfsetroundjoin%
\definecolor{currentfill}{rgb}{0.276194,0.190074,0.493001}%
\pgfsetfillcolor{currentfill}%
\pgfsetfillopacity{0.700000}%
\pgfsetlinewidth{0.000000pt}%
\definecolor{currentstroke}{rgb}{0.000000,0.000000,0.000000}%
\pgfsetstrokecolor{currentstroke}%
\pgfsetdash{}{0pt}%
\pgfpathmoveto{\pgfqpoint{3.167979in}{2.026206in}}%
\pgfpathlineto{\pgfqpoint{3.181350in}{2.013292in}}%
\pgfpathlineto{\pgfqpoint{3.194720in}{2.000526in}}%
\pgfpathlineto{\pgfqpoint{3.208089in}{1.987908in}}%
\pgfpathlineto{\pgfqpoint{3.221457in}{1.975437in}}%
\pgfpathlineto{\pgfqpoint{3.229618in}{1.977615in}}%
\pgfpathlineto{\pgfqpoint{3.237768in}{1.979964in}}%
\pgfpathlineto{\pgfqpoint{3.245908in}{1.982481in}}%
\pgfpathlineto{\pgfqpoint{3.254037in}{1.985163in}}%
\pgfpathlineto{\pgfqpoint{3.240699in}{1.997266in}}%
\pgfpathlineto{\pgfqpoint{3.227360in}{2.009515in}}%
\pgfpathlineto{\pgfqpoint{3.214020in}{2.021912in}}%
\pgfpathlineto{\pgfqpoint{3.200679in}{2.034457in}}%
\pgfpathlineto{\pgfqpoint{3.192521in}{2.032137in}}%
\pgfpathlineto{\pgfqpoint{3.184351in}{2.029986in}}%
\pgfpathlineto{\pgfqpoint{3.176171in}{2.028009in}}%
\pgfpathlineto{\pgfqpoint{3.167979in}{2.026206in}}%
\pgfpathclose%
\pgfusepath{fill}%
\end{pgfscope}%
\begin{pgfscope}%
\pgfpathrectangle{\pgfqpoint{1.254980in}{0.150000in}}{\pgfqpoint{5.490039in}{5.490039in}}%
\pgfusepath{clip}%
\pgfsetbuttcap%
\pgfsetroundjoin%
\definecolor{currentfill}{rgb}{0.241237,0.296485,0.539709}%
\pgfsetfillcolor{currentfill}%
\pgfsetfillopacity{0.700000}%
\pgfsetlinewidth{0.000000pt}%
\definecolor{currentstroke}{rgb}{0.000000,0.000000,0.000000}%
\pgfsetstrokecolor{currentstroke}%
\pgfsetdash{}{0pt}%
\pgfpathmoveto{\pgfqpoint{5.089162in}{2.194471in}}%
\pgfpathlineto{\pgfqpoint{5.102963in}{2.198488in}}%
\pgfpathlineto{\pgfqpoint{5.116777in}{2.202619in}}%
\pgfpathlineto{\pgfqpoint{5.130604in}{2.206863in}}%
\pgfpathlineto{\pgfqpoint{5.144444in}{2.211220in}}%
\pgfpathlineto{\pgfqpoint{5.151863in}{2.222470in}}%
\pgfpathlineto{\pgfqpoint{5.159277in}{2.233668in}}%
\pgfpathlineto{\pgfqpoint{5.166685in}{2.244814in}}%
\pgfpathlineto{\pgfqpoint{5.174088in}{2.255908in}}%
\pgfpathlineto{\pgfqpoint{5.160252in}{2.251464in}}%
\pgfpathlineto{\pgfqpoint{5.146429in}{2.247134in}}%
\pgfpathlineto{\pgfqpoint{5.132619in}{2.242916in}}%
\pgfpathlineto{\pgfqpoint{5.118822in}{2.238812in}}%
\pgfpathlineto{\pgfqpoint{5.111415in}{2.227799in}}%
\pgfpathlineto{\pgfqpoint{5.104003in}{2.216737in}}%
\pgfpathlineto{\pgfqpoint{5.096585in}{2.205627in}}%
\pgfpathlineto{\pgfqpoint{5.089162in}{2.194471in}}%
\pgfpathclose%
\pgfusepath{fill}%
\end{pgfscope}%
\begin{pgfscope}%
\pgfpathrectangle{\pgfqpoint{1.254980in}{0.150000in}}{\pgfqpoint{5.490039in}{5.490039in}}%
\pgfusepath{clip}%
\pgfsetbuttcap%
\pgfsetroundjoin%
\definecolor{currentfill}{rgb}{0.168126,0.459988,0.558082}%
\pgfsetfillcolor{currentfill}%
\pgfsetfillopacity{0.700000}%
\pgfsetlinewidth{0.000000pt}%
\definecolor{currentstroke}{rgb}{0.000000,0.000000,0.000000}%
\pgfsetstrokecolor{currentstroke}%
\pgfsetdash{}{0pt}%
\pgfpathmoveto{\pgfqpoint{2.630508in}{2.674172in}}%
\pgfpathlineto{\pgfqpoint{2.644047in}{2.654604in}}%
\pgfpathlineto{\pgfqpoint{2.657577in}{2.635222in}}%
\pgfpathlineto{\pgfqpoint{2.671101in}{2.616026in}}%
\pgfpathlineto{\pgfqpoint{2.684618in}{2.597013in}}%
\pgfpathlineto{\pgfqpoint{2.693138in}{2.595343in}}%
\pgfpathlineto{\pgfqpoint{2.701642in}{2.593890in}}%
\pgfpathlineto{\pgfqpoint{2.710132in}{2.592652in}}%
\pgfpathlineto{\pgfqpoint{2.718606in}{2.591625in}}%
\pgfpathlineto{\pgfqpoint{2.705131in}{2.610247in}}%
\pgfpathlineto{\pgfqpoint{2.691649in}{2.629053in}}%
\pgfpathlineto{\pgfqpoint{2.678160in}{2.648042in}}%
\pgfpathlineto{\pgfqpoint{2.664665in}{2.667218in}}%
\pgfpathlineto{\pgfqpoint{2.656149in}{2.668629in}}%
\pgfpathlineto{\pgfqpoint{2.647618in}{2.670256in}}%
\pgfpathlineto{\pgfqpoint{2.639071in}{2.672103in}}%
\pgfpathlineto{\pgfqpoint{2.630508in}{2.674172in}}%
\pgfpathclose%
\pgfusepath{fill}%
\end{pgfscope}%
\begin{pgfscope}%
\pgfpathrectangle{\pgfqpoint{1.254980in}{0.150000in}}{\pgfqpoint{5.490039in}{5.490039in}}%
\pgfusepath{clip}%
\pgfsetbuttcap%
\pgfsetroundjoin%
\definecolor{currentfill}{rgb}{0.271305,0.019942,0.347269}%
\pgfsetfillcolor{currentfill}%
\pgfsetfillopacity{0.700000}%
\pgfsetlinewidth{0.000000pt}%
\definecolor{currentstroke}{rgb}{0.000000,0.000000,0.000000}%
\pgfsetstrokecolor{currentstroke}%
\pgfsetdash{}{0pt}%
\pgfpathmoveto{\pgfqpoint{4.074244in}{1.673441in}}%
\pgfpathlineto{\pgfqpoint{4.087658in}{1.669609in}}%
\pgfpathlineto{\pgfqpoint{4.101079in}{1.665897in}}%
\pgfpathlineto{\pgfqpoint{4.114506in}{1.662305in}}%
\pgfpathlineto{\pgfqpoint{4.127939in}{1.658833in}}%
\pgfpathlineto{\pgfqpoint{4.135667in}{1.667885in}}%
\pgfpathlineto{\pgfqpoint{4.143390in}{1.676998in}}%
\pgfpathlineto{\pgfqpoint{4.151108in}{1.686169in}}%
\pgfpathlineto{\pgfqpoint{4.158820in}{1.695396in}}%
\pgfpathlineto{\pgfqpoint{4.145398in}{1.698591in}}%
\pgfpathlineto{\pgfqpoint{4.131982in}{1.701906in}}%
\pgfpathlineto{\pgfqpoint{4.118573in}{1.705341in}}%
\pgfpathlineto{\pgfqpoint{4.105171in}{1.708897in}}%
\pgfpathlineto{\pgfqpoint{4.097448in}{1.699940in}}%
\pgfpathlineto{\pgfqpoint{4.089719in}{1.691044in}}%
\pgfpathlineto{\pgfqpoint{4.081984in}{1.682210in}}%
\pgfpathlineto{\pgfqpoint{4.074244in}{1.673441in}}%
\pgfpathclose%
\pgfusepath{fill}%
\end{pgfscope}%
\begin{pgfscope}%
\pgfpathrectangle{\pgfqpoint{1.254980in}{0.150000in}}{\pgfqpoint{5.490039in}{5.490039in}}%
\pgfusepath{clip}%
\pgfsetbuttcap%
\pgfsetroundjoin%
\definecolor{currentfill}{rgb}{0.283197,0.115680,0.436115}%
\pgfsetfillcolor{currentfill}%
\pgfsetfillopacity{0.700000}%
\pgfsetlinewidth{0.000000pt}%
\definecolor{currentstroke}{rgb}{0.000000,0.000000,0.000000}%
\pgfsetstrokecolor{currentstroke}%
\pgfsetdash{}{0pt}%
\pgfpathmoveto{\pgfqpoint{4.550876in}{1.829031in}}%
\pgfpathlineto{\pgfqpoint{4.564438in}{1.829253in}}%
\pgfpathlineto{\pgfqpoint{4.578009in}{1.829590in}}%
\pgfpathlineto{\pgfqpoint{4.591590in}{1.830043in}}%
\pgfpathlineto{\pgfqpoint{4.605181in}{1.830611in}}%
\pgfpathlineto{\pgfqpoint{4.612764in}{1.841729in}}%
\pgfpathlineto{\pgfqpoint{4.620342in}{1.852847in}}%
\pgfpathlineto{\pgfqpoint{4.627915in}{1.863964in}}%
\pgfpathlineto{\pgfqpoint{4.635484in}{1.875079in}}%
\pgfpathlineto{\pgfqpoint{4.621898in}{1.874313in}}%
\pgfpathlineto{\pgfqpoint{4.608323in}{1.873662in}}%
\pgfpathlineto{\pgfqpoint{4.594757in}{1.873127in}}%
\pgfpathlineto{\pgfqpoint{4.581201in}{1.872707in}}%
\pgfpathlineto{\pgfqpoint{4.573627in}{1.861784in}}%
\pgfpathlineto{\pgfqpoint{4.566048in}{1.850863in}}%
\pgfpathlineto{\pgfqpoint{4.558464in}{1.839945in}}%
\pgfpathlineto{\pgfqpoint{4.550876in}{1.829031in}}%
\pgfpathclose%
\pgfusepath{fill}%
\end{pgfscope}%
\begin{pgfscope}%
\pgfpathrectangle{\pgfqpoint{1.254980in}{0.150000in}}{\pgfqpoint{5.490039in}{5.490039in}}%
\pgfusepath{clip}%
\pgfsetbuttcap%
\pgfsetroundjoin%
\definecolor{currentfill}{rgb}{0.279574,0.170599,0.479997}%
\pgfsetfillcolor{currentfill}%
\pgfsetfillopacity{0.700000}%
\pgfsetlinewidth{0.000000pt}%
\definecolor{currentstroke}{rgb}{0.000000,0.000000,0.000000}%
\pgfsetstrokecolor{currentstroke}%
\pgfsetdash{}{0pt}%
\pgfpathmoveto{\pgfqpoint{3.221457in}{1.975437in}}%
\pgfpathlineto{\pgfqpoint{3.234824in}{1.963113in}}%
\pgfpathlineto{\pgfqpoint{3.248190in}{1.950935in}}%
\pgfpathlineto{\pgfqpoint{3.261556in}{1.938902in}}%
\pgfpathlineto{\pgfqpoint{3.274921in}{1.927013in}}%
\pgfpathlineto{\pgfqpoint{3.283053in}{1.929565in}}%
\pgfpathlineto{\pgfqpoint{3.291174in}{1.932284in}}%
\pgfpathlineto{\pgfqpoint{3.299285in}{1.935167in}}%
\pgfpathlineto{\pgfqpoint{3.307385in}{1.938210in}}%
\pgfpathlineto{\pgfqpoint{3.294049in}{1.949731in}}%
\pgfpathlineto{\pgfqpoint{3.280712in}{1.961397in}}%
\pgfpathlineto{\pgfqpoint{3.267375in}{1.973207in}}%
\pgfpathlineto{\pgfqpoint{3.254037in}{1.985163in}}%
\pgfpathlineto{\pgfqpoint{3.245908in}{1.982481in}}%
\pgfpathlineto{\pgfqpoint{3.237768in}{1.979964in}}%
\pgfpathlineto{\pgfqpoint{3.229618in}{1.977615in}}%
\pgfpathlineto{\pgfqpoint{3.221457in}{1.975437in}}%
\pgfpathclose%
\pgfusepath{fill}%
\end{pgfscope}%
\begin{pgfscope}%
\pgfpathrectangle{\pgfqpoint{1.254980in}{0.150000in}}{\pgfqpoint{5.490039in}{5.490039in}}%
\pgfusepath{clip}%
\pgfsetbuttcap%
\pgfsetroundjoin%
\definecolor{currentfill}{rgb}{0.282327,0.094955,0.417331}%
\pgfsetfillcolor{currentfill}%
\pgfsetfillopacity{0.700000}%
\pgfsetlinewidth{0.000000pt}%
\definecolor{currentstroke}{rgb}{0.000000,0.000000,0.000000}%
\pgfsetstrokecolor{currentstroke}%
\pgfsetdash{}{0pt}%
\pgfpathmoveto{\pgfqpoint{4.466299in}{1.786575in}}%
\pgfpathlineto{\pgfqpoint{4.479830in}{1.786119in}}%
\pgfpathlineto{\pgfqpoint{4.493370in}{1.785780in}}%
\pgfpathlineto{\pgfqpoint{4.506918in}{1.785557in}}%
\pgfpathlineto{\pgfqpoint{4.520477in}{1.785450in}}%
\pgfpathlineto{\pgfqpoint{4.528084in}{1.796331in}}%
\pgfpathlineto{\pgfqpoint{4.535686in}{1.807222in}}%
\pgfpathlineto{\pgfqpoint{4.543283in}{1.818123in}}%
\pgfpathlineto{\pgfqpoint{4.550876in}{1.829031in}}%
\pgfpathlineto{\pgfqpoint{4.537324in}{1.828924in}}%
\pgfpathlineto{\pgfqpoint{4.523781in}{1.828934in}}%
\pgfpathlineto{\pgfqpoint{4.510248in}{1.829059in}}%
\pgfpathlineto{\pgfqpoint{4.496724in}{1.829300in}}%
\pgfpathlineto{\pgfqpoint{4.489125in}{1.818600in}}%
\pgfpathlineto{\pgfqpoint{4.481521in}{1.807912in}}%
\pgfpathlineto{\pgfqpoint{4.473912in}{1.797236in}}%
\pgfpathlineto{\pgfqpoint{4.466299in}{1.786575in}}%
\pgfpathclose%
\pgfusepath{fill}%
\end{pgfscope}%
\begin{pgfscope}%
\pgfpathrectangle{\pgfqpoint{1.254980in}{0.150000in}}{\pgfqpoint{5.490039in}{5.490039in}}%
\pgfusepath{clip}%
\pgfsetbuttcap%
\pgfsetroundjoin%
\definecolor{currentfill}{rgb}{0.282623,0.140926,0.457517}%
\pgfsetfillcolor{currentfill}%
\pgfsetfillopacity{0.700000}%
\pgfsetlinewidth{0.000000pt}%
\definecolor{currentstroke}{rgb}{0.000000,0.000000,0.000000}%
\pgfsetstrokecolor{currentstroke}%
\pgfsetdash{}{0pt}%
\pgfpathmoveto{\pgfqpoint{4.635484in}{1.875079in}}%
\pgfpathlineto{\pgfqpoint{4.649080in}{1.875960in}}%
\pgfpathlineto{\pgfqpoint{4.662685in}{1.876956in}}%
\pgfpathlineto{\pgfqpoint{4.676302in}{1.878067in}}%
\pgfpathlineto{\pgfqpoint{4.689928in}{1.879293in}}%
\pgfpathlineto{\pgfqpoint{4.697487in}{1.890592in}}%
\pgfpathlineto{\pgfqpoint{4.705042in}{1.901882in}}%
\pgfpathlineto{\pgfqpoint{4.712591in}{1.913162in}}%
\pgfpathlineto{\pgfqpoint{4.720136in}{1.924430in}}%
\pgfpathlineto{\pgfqpoint{4.706514in}{1.923022in}}%
\pgfpathlineto{\pgfqpoint{4.692903in}{1.921728in}}%
\pgfpathlineto{\pgfqpoint{4.679302in}{1.920550in}}%
\pgfpathlineto{\pgfqpoint{4.665711in}{1.919486in}}%
\pgfpathlineto{\pgfqpoint{4.658162in}{1.908395in}}%
\pgfpathlineto{\pgfqpoint{4.650607in}{1.897296in}}%
\pgfpathlineto{\pgfqpoint{4.643048in}{1.886190in}}%
\pgfpathlineto{\pgfqpoint{4.635484in}{1.875079in}}%
\pgfpathclose%
\pgfusepath{fill}%
\end{pgfscope}%
\begin{pgfscope}%
\pgfpathrectangle{\pgfqpoint{1.254980in}{0.150000in}}{\pgfqpoint{5.490039in}{5.490039in}}%
\pgfusepath{clip}%
\pgfsetbuttcap%
\pgfsetroundjoin%
\definecolor{currentfill}{rgb}{0.185556,0.418570,0.556753}%
\pgfsetfillcolor{currentfill}%
\pgfsetfillopacity{0.700000}%
\pgfsetlinewidth{0.000000pt}%
\definecolor{currentstroke}{rgb}{0.000000,0.000000,0.000000}%
\pgfsetstrokecolor{currentstroke}%
\pgfsetdash{}{0pt}%
\pgfpathmoveto{\pgfqpoint{5.458626in}{2.490410in}}%
\pgfpathlineto{\pgfqpoint{5.472625in}{2.496503in}}%
\pgfpathlineto{\pgfqpoint{5.486639in}{2.502709in}}%
\pgfpathlineto{\pgfqpoint{5.500668in}{2.509027in}}%
\pgfpathlineto{\pgfqpoint{5.514712in}{2.515458in}}%
\pgfpathlineto{\pgfqpoint{5.521993in}{2.525673in}}%
\pgfpathlineto{\pgfqpoint{5.529267in}{2.535818in}}%
\pgfpathlineto{\pgfqpoint{5.536535in}{2.545892in}}%
\pgfpathlineto{\pgfqpoint{5.543796in}{2.555896in}}%
\pgfpathlineto{\pgfqpoint{5.529758in}{2.549462in}}%
\pgfpathlineto{\pgfqpoint{5.515735in}{2.543140in}}%
\pgfpathlineto{\pgfqpoint{5.501727in}{2.536931in}}%
\pgfpathlineto{\pgfqpoint{5.487734in}{2.530834in}}%
\pgfpathlineto{\pgfqpoint{5.480467in}{2.520827in}}%
\pgfpathlineto{\pgfqpoint{5.473193in}{2.510754in}}%
\pgfpathlineto{\pgfqpoint{5.465913in}{2.500615in}}%
\pgfpathlineto{\pgfqpoint{5.458626in}{2.490410in}}%
\pgfpathclose%
\pgfusepath{fill}%
\end{pgfscope}%
\begin{pgfscope}%
\pgfpathrectangle{\pgfqpoint{1.254980in}{0.150000in}}{\pgfqpoint{5.490039in}{5.490039in}}%
\pgfusepath{clip}%
\pgfsetbuttcap%
\pgfsetroundjoin%
\definecolor{currentfill}{rgb}{0.280267,0.073417,0.397163}%
\pgfsetfillcolor{currentfill}%
\pgfsetfillopacity{0.700000}%
\pgfsetlinewidth{0.000000pt}%
\definecolor{currentstroke}{rgb}{0.000000,0.000000,0.000000}%
\pgfsetstrokecolor{currentstroke}%
\pgfsetdash{}{0pt}%
\pgfpathmoveto{\pgfqpoint{4.381736in}{1.748010in}}%
\pgfpathlineto{\pgfqpoint{4.395239in}{1.746860in}}%
\pgfpathlineto{\pgfqpoint{4.408750in}{1.745826in}}%
\pgfpathlineto{\pgfqpoint{4.422270in}{1.744909in}}%
\pgfpathlineto{\pgfqpoint{4.435799in}{1.744108in}}%
\pgfpathlineto{\pgfqpoint{4.443431in}{1.754694in}}%
\pgfpathlineto{\pgfqpoint{4.451058in}{1.765302in}}%
\pgfpathlineto{\pgfqpoint{4.458681in}{1.775929in}}%
\pgfpathlineto{\pgfqpoint{4.466299in}{1.786575in}}%
\pgfpathlineto{\pgfqpoint{4.452777in}{1.787146in}}%
\pgfpathlineto{\pgfqpoint{4.439265in}{1.787834in}}%
\pgfpathlineto{\pgfqpoint{4.425761in}{1.788638in}}%
\pgfpathlineto{\pgfqpoint{4.412265in}{1.789559in}}%
\pgfpathlineto{\pgfqpoint{4.404640in}{1.779138in}}%
\pgfpathlineto{\pgfqpoint{4.397010in}{1.768738in}}%
\pgfpathlineto{\pgfqpoint{4.389376in}{1.758362in}}%
\pgfpathlineto{\pgfqpoint{4.381736in}{1.748010in}}%
\pgfpathclose%
\pgfusepath{fill}%
\end{pgfscope}%
\begin{pgfscope}%
\pgfpathrectangle{\pgfqpoint{1.254980in}{0.150000in}}{\pgfqpoint{5.490039in}{5.490039in}}%
\pgfusepath{clip}%
\pgfsetbuttcap%
\pgfsetroundjoin%
\definecolor{currentfill}{rgb}{0.280255,0.165693,0.476498}%
\pgfsetfillcolor{currentfill}%
\pgfsetfillopacity{0.700000}%
\pgfsetlinewidth{0.000000pt}%
\definecolor{currentstroke}{rgb}{0.000000,0.000000,0.000000}%
\pgfsetstrokecolor{currentstroke}%
\pgfsetdash{}{0pt}%
\pgfpathmoveto{\pgfqpoint{4.720136in}{1.924430in}}%
\pgfpathlineto{\pgfqpoint{4.733769in}{1.925952in}}%
\pgfpathlineto{\pgfqpoint{4.747412in}{1.927589in}}%
\pgfpathlineto{\pgfqpoint{4.761066in}{1.929340in}}%
\pgfpathlineto{\pgfqpoint{4.774731in}{1.931206in}}%
\pgfpathlineto{\pgfqpoint{4.782267in}{1.942633in}}%
\pgfpathlineto{\pgfqpoint{4.789798in}{1.954043in}}%
\pgfpathlineto{\pgfqpoint{4.797324in}{1.965432in}}%
\pgfpathlineto{\pgfqpoint{4.804846in}{1.976802in}}%
\pgfpathlineto{\pgfqpoint{4.791185in}{1.974769in}}%
\pgfpathlineto{\pgfqpoint{4.777535in}{1.972851in}}%
\pgfpathlineto{\pgfqpoint{4.763896in}{1.971048in}}%
\pgfpathlineto{\pgfqpoint{4.750268in}{1.969359in}}%
\pgfpathlineto{\pgfqpoint{4.742742in}{1.958150in}}%
\pgfpathlineto{\pgfqpoint{4.735212in}{1.946925in}}%
\pgfpathlineto{\pgfqpoint{4.727676in}{1.935684in}}%
\pgfpathlineto{\pgfqpoint{4.720136in}{1.924430in}}%
\pgfpathclose%
\pgfusepath{fill}%
\end{pgfscope}%
\begin{pgfscope}%
\pgfpathrectangle{\pgfqpoint{1.254980in}{0.150000in}}{\pgfqpoint{5.490039in}{5.490039in}}%
\pgfusepath{clip}%
\pgfsetbuttcap%
\pgfsetroundjoin%
\definecolor{currentfill}{rgb}{0.227802,0.326594,0.546532}%
\pgfsetfillcolor{currentfill}%
\pgfsetfillopacity{0.700000}%
\pgfsetlinewidth{0.000000pt}%
\definecolor{currentstroke}{rgb}{0.000000,0.000000,0.000000}%
\pgfsetstrokecolor{currentstroke}%
\pgfsetdash{}{0pt}%
\pgfpathmoveto{\pgfqpoint{5.174088in}{2.255908in}}%
\pgfpathlineto{\pgfqpoint{5.187937in}{2.260465in}}%
\pgfpathlineto{\pgfqpoint{5.201800in}{2.265134in}}%
\pgfpathlineto{\pgfqpoint{5.215676in}{2.269917in}}%
\pgfpathlineto{\pgfqpoint{5.229566in}{2.274813in}}%
\pgfpathlineto{\pgfqpoint{5.236960in}{2.285931in}}%
\pgfpathlineto{\pgfqpoint{5.244347in}{2.296992in}}%
\pgfpathlineto{\pgfqpoint{5.251729in}{2.307995in}}%
\pgfpathlineto{\pgfqpoint{5.259104in}{2.318940in}}%
\pgfpathlineto{\pgfqpoint{5.245219in}{2.313974in}}%
\pgfpathlineto{\pgfqpoint{5.231347in}{2.309121in}}%
\pgfpathlineto{\pgfqpoint{5.217488in}{2.304381in}}%
\pgfpathlineto{\pgfqpoint{5.203643in}{2.299754in}}%
\pgfpathlineto{\pgfqpoint{5.196263in}{2.288873in}}%
\pgfpathlineto{\pgfqpoint{5.188877in}{2.277938in}}%
\pgfpathlineto{\pgfqpoint{5.181485in}{2.266949in}}%
\pgfpathlineto{\pgfqpoint{5.174088in}{2.255908in}}%
\pgfpathclose%
\pgfusepath{fill}%
\end{pgfscope}%
\begin{pgfscope}%
\pgfpathrectangle{\pgfqpoint{1.254980in}{0.150000in}}{\pgfqpoint{5.490039in}{5.490039in}}%
\pgfusepath{clip}%
\pgfsetbuttcap%
\pgfsetroundjoin%
\definecolor{currentfill}{rgb}{0.156270,0.489624,0.557936}%
\pgfsetfillcolor{currentfill}%
\pgfsetfillopacity{0.700000}%
\pgfsetlinewidth{0.000000pt}%
\definecolor{currentstroke}{rgb}{0.000000,0.000000,0.000000}%
\pgfsetstrokecolor{currentstroke}%
\pgfsetdash{}{0pt}%
\pgfpathmoveto{\pgfqpoint{2.576280in}{2.754329in}}%
\pgfpathlineto{\pgfqpoint{2.589849in}{2.734004in}}%
\pgfpathlineto{\pgfqpoint{2.603410in}{2.713871in}}%
\pgfpathlineto{\pgfqpoint{2.616963in}{2.693927in}}%
\pgfpathlineto{\pgfqpoint{2.630508in}{2.674172in}}%
\pgfpathlineto{\pgfqpoint{2.639071in}{2.672103in}}%
\pgfpathlineto{\pgfqpoint{2.647618in}{2.670256in}}%
\pgfpathlineto{\pgfqpoint{2.656149in}{2.668629in}}%
\pgfpathlineto{\pgfqpoint{2.664665in}{2.667218in}}%
\pgfpathlineto{\pgfqpoint{2.651162in}{2.686580in}}%
\pgfpathlineto{\pgfqpoint{2.637652in}{2.706129in}}%
\pgfpathlineto{\pgfqpoint{2.624135in}{2.725868in}}%
\pgfpathlineto{\pgfqpoint{2.610610in}{2.745798in}}%
\pgfpathlineto{\pgfqpoint{2.602051in}{2.747596in}}%
\pgfpathlineto{\pgfqpoint{2.593477in}{2.749615in}}%
\pgfpathlineto{\pgfqpoint{2.584887in}{2.751859in}}%
\pgfpathlineto{\pgfqpoint{2.576280in}{2.754329in}}%
\pgfpathclose%
\pgfusepath{fill}%
\end{pgfscope}%
\begin{pgfscope}%
\pgfpathrectangle{\pgfqpoint{1.254980in}{0.150000in}}{\pgfqpoint{5.490039in}{5.490039in}}%
\pgfusepath{clip}%
\pgfsetbuttcap%
\pgfsetroundjoin%
\definecolor{currentfill}{rgb}{0.280267,0.073417,0.397163}%
\pgfsetfillcolor{currentfill}%
\pgfsetfillopacity{0.700000}%
\pgfsetlinewidth{0.000000pt}%
\definecolor{currentstroke}{rgb}{0.000000,0.000000,0.000000}%
\pgfsetstrokecolor{currentstroke}%
\pgfsetdash{}{0pt}%
\pgfpathmoveto{\pgfqpoint{3.520843in}{1.772977in}}%
\pgfpathlineto{\pgfqpoint{3.534195in}{1.763815in}}%
\pgfpathlineto{\pgfqpoint{3.547549in}{1.754786in}}%
\pgfpathlineto{\pgfqpoint{3.560904in}{1.745891in}}%
\pgfpathlineto{\pgfqpoint{3.574262in}{1.737129in}}%
\pgfpathlineto{\pgfqpoint{3.582229in}{1.742095in}}%
\pgfpathlineto{\pgfqpoint{3.590187in}{1.747195in}}%
\pgfpathlineto{\pgfqpoint{3.598137in}{1.752426in}}%
\pgfpathlineto{\pgfqpoint{3.606079in}{1.757785in}}%
\pgfpathlineto{\pgfqpoint{3.592743in}{1.766203in}}%
\pgfpathlineto{\pgfqpoint{3.579410in}{1.774753in}}%
\pgfpathlineto{\pgfqpoint{3.566079in}{1.783437in}}%
\pgfpathlineto{\pgfqpoint{3.552750in}{1.792254in}}%
\pgfpathlineto{\pgfqpoint{3.544786in}{1.787234in}}%
\pgfpathlineto{\pgfqpoint{3.536813in}{1.782346in}}%
\pgfpathlineto{\pgfqpoint{3.528833in}{1.777592in}}%
\pgfpathlineto{\pgfqpoint{3.520843in}{1.772977in}}%
\pgfpathclose%
\pgfusepath{fill}%
\end{pgfscope}%
\begin{pgfscope}%
\pgfpathrectangle{\pgfqpoint{1.254980in}{0.150000in}}{\pgfqpoint{5.490039in}{5.490039in}}%
\pgfusepath{clip}%
\pgfsetbuttcap%
\pgfsetroundjoin%
\definecolor{currentfill}{rgb}{0.271305,0.019942,0.347269}%
\pgfsetfillcolor{currentfill}%
\pgfsetfillopacity{0.700000}%
\pgfsetlinewidth{0.000000pt}%
\definecolor{currentstroke}{rgb}{0.000000,0.000000,0.000000}%
\pgfsetstrokecolor{currentstroke}%
\pgfsetdash{}{0pt}%
\pgfpathmoveto{\pgfqpoint{3.851264in}{1.668634in}}%
\pgfpathlineto{\pgfqpoint{3.864641in}{1.662726in}}%
\pgfpathlineto{\pgfqpoint{3.878023in}{1.656943in}}%
\pgfpathlineto{\pgfqpoint{3.891410in}{1.651284in}}%
\pgfpathlineto{\pgfqpoint{3.904801in}{1.645750in}}%
\pgfpathlineto{\pgfqpoint{3.912616in}{1.653288in}}%
\pgfpathlineto{\pgfqpoint{3.920424in}{1.660918in}}%
\pgfpathlineto{\pgfqpoint{3.928227in}{1.668636in}}%
\pgfpathlineto{\pgfqpoint{3.936023in}{1.676442in}}%
\pgfpathlineto{\pgfqpoint{3.922647in}{1.681668in}}%
\pgfpathlineto{\pgfqpoint{3.909275in}{1.687017in}}%
\pgfpathlineto{\pgfqpoint{3.895909in}{1.692491in}}%
\pgfpathlineto{\pgfqpoint{3.882548in}{1.698089in}}%
\pgfpathlineto{\pgfqpoint{3.874736in}{1.690587in}}%
\pgfpathlineto{\pgfqpoint{3.866919in}{1.683175in}}%
\pgfpathlineto{\pgfqpoint{3.859094in}{1.675857in}}%
\pgfpathlineto{\pgfqpoint{3.851264in}{1.668634in}}%
\pgfpathclose%
\pgfusepath{fill}%
\end{pgfscope}%
\begin{pgfscope}%
\pgfpathrectangle{\pgfqpoint{1.254980in}{0.150000in}}{\pgfqpoint{5.490039in}{5.490039in}}%
\pgfusepath{clip}%
\pgfsetbuttcap%
\pgfsetroundjoin%
\definecolor{currentfill}{rgb}{0.277018,0.050344,0.375715}%
\pgfsetfillcolor{currentfill}%
\pgfsetfillopacity{0.700000}%
\pgfsetlinewidth{0.000000pt}%
\definecolor{currentstroke}{rgb}{0.000000,0.000000,0.000000}%
\pgfsetstrokecolor{currentstroke}%
\pgfsetdash{}{0pt}%
\pgfpathmoveto{\pgfqpoint{4.297169in}{1.713649in}}%
\pgfpathlineto{\pgfqpoint{4.310647in}{1.711784in}}%
\pgfpathlineto{\pgfqpoint{4.324134in}{1.710037in}}%
\pgfpathlineto{\pgfqpoint{4.337628in}{1.708407in}}%
\pgfpathlineto{\pgfqpoint{4.351130in}{1.706894in}}%
\pgfpathlineto{\pgfqpoint{4.358789in}{1.717126in}}%
\pgfpathlineto{\pgfqpoint{4.366443in}{1.727391in}}%
\pgfpathlineto{\pgfqpoint{4.374092in}{1.737686in}}%
\pgfpathlineto{\pgfqpoint{4.381736in}{1.748010in}}%
\pgfpathlineto{\pgfqpoint{4.368242in}{1.749278in}}%
\pgfpathlineto{\pgfqpoint{4.354756in}{1.750663in}}%
\pgfpathlineto{\pgfqpoint{4.341278in}{1.752165in}}%
\pgfpathlineto{\pgfqpoint{4.327808in}{1.753785in}}%
\pgfpathlineto{\pgfqpoint{4.320156in}{1.743700in}}%
\pgfpathlineto{\pgfqpoint{4.312498in}{1.733648in}}%
\pgfpathlineto{\pgfqpoint{4.304836in}{1.723630in}}%
\pgfpathlineto{\pgfqpoint{4.297169in}{1.713649in}}%
\pgfpathclose%
\pgfusepath{fill}%
\end{pgfscope}%
\begin{pgfscope}%
\pgfpathrectangle{\pgfqpoint{1.254980in}{0.150000in}}{\pgfqpoint{5.490039in}{5.490039in}}%
\pgfusepath{clip}%
\pgfsetbuttcap%
\pgfsetroundjoin%
\definecolor{currentfill}{rgb}{0.281887,0.150881,0.465405}%
\pgfsetfillcolor{currentfill}%
\pgfsetfillopacity{0.700000}%
\pgfsetlinewidth{0.000000pt}%
\definecolor{currentstroke}{rgb}{0.000000,0.000000,0.000000}%
\pgfsetstrokecolor{currentstroke}%
\pgfsetdash{}{0pt}%
\pgfpathmoveto{\pgfqpoint{3.274921in}{1.927013in}}%
\pgfpathlineto{\pgfqpoint{3.288286in}{1.915269in}}%
\pgfpathlineto{\pgfqpoint{3.301650in}{1.903668in}}%
\pgfpathlineto{\pgfqpoint{3.315015in}{1.892209in}}%
\pgfpathlineto{\pgfqpoint{3.328379in}{1.880892in}}%
\pgfpathlineto{\pgfqpoint{3.336482in}{1.883817in}}%
\pgfpathlineto{\pgfqpoint{3.344576in}{1.886904in}}%
\pgfpathlineto{\pgfqpoint{3.352659in}{1.890151in}}%
\pgfpathlineto{\pgfqpoint{3.360732in}{1.893554in}}%
\pgfpathlineto{\pgfqpoint{3.347395in}{1.904505in}}%
\pgfpathlineto{\pgfqpoint{3.334059in}{1.915597in}}%
\pgfpathlineto{\pgfqpoint{3.320722in}{1.926832in}}%
\pgfpathlineto{\pgfqpoint{3.307385in}{1.938210in}}%
\pgfpathlineto{\pgfqpoint{3.299285in}{1.935167in}}%
\pgfpathlineto{\pgfqpoint{3.291174in}{1.932284in}}%
\pgfpathlineto{\pgfqpoint{3.283053in}{1.929565in}}%
\pgfpathlineto{\pgfqpoint{3.274921in}{1.927013in}}%
\pgfpathclose%
\pgfusepath{fill}%
\end{pgfscope}%
\begin{pgfscope}%
\pgfpathrectangle{\pgfqpoint{1.254980in}{0.150000in}}{\pgfqpoint{5.490039in}{5.490039in}}%
\pgfusepath{clip}%
\pgfsetbuttcap%
\pgfsetroundjoin%
\definecolor{currentfill}{rgb}{0.275191,0.194905,0.496005}%
\pgfsetfillcolor{currentfill}%
\pgfsetfillopacity{0.700000}%
\pgfsetlinewidth{0.000000pt}%
\definecolor{currentstroke}{rgb}{0.000000,0.000000,0.000000}%
\pgfsetstrokecolor{currentstroke}%
\pgfsetdash{}{0pt}%
\pgfpathmoveto{\pgfqpoint{4.804846in}{1.976802in}}%
\pgfpathlineto{\pgfqpoint{4.818518in}{1.978948in}}%
\pgfpathlineto{\pgfqpoint{4.832201in}{1.981208in}}%
\pgfpathlineto{\pgfqpoint{4.845895in}{1.983583in}}%
\pgfpathlineto{\pgfqpoint{4.859601in}{1.986071in}}%
\pgfpathlineto{\pgfqpoint{4.867114in}{1.997575in}}%
\pgfpathlineto{\pgfqpoint{4.874622in}{2.009053in}}%
\pgfpathlineto{\pgfqpoint{4.882125in}{2.020503in}}%
\pgfpathlineto{\pgfqpoint{4.889623in}{2.031924in}}%
\pgfpathlineto{\pgfqpoint{4.875921in}{2.029285in}}%
\pgfpathlineto{\pgfqpoint{4.862230in}{2.026760in}}%
\pgfpathlineto{\pgfqpoint{4.848551in}{2.024349in}}%
\pgfpathlineto{\pgfqpoint{4.834883in}{2.022052in}}%
\pgfpathlineto{\pgfqpoint{4.827381in}{2.010775in}}%
\pgfpathlineto{\pgfqpoint{4.819874in}{1.999474in}}%
\pgfpathlineto{\pgfqpoint{4.812362in}{1.988149in}}%
\pgfpathlineto{\pgfqpoint{4.804846in}{1.976802in}}%
\pgfpathclose%
\pgfusepath{fill}%
\end{pgfscope}%
\begin{pgfscope}%
\pgfpathrectangle{\pgfqpoint{1.254980in}{0.150000in}}{\pgfqpoint{5.490039in}{5.490039in}}%
\pgfusepath{clip}%
\pgfsetbuttcap%
\pgfsetroundjoin%
\definecolor{currentfill}{rgb}{0.128087,0.647749,0.523491}%
\pgfsetfillcolor{currentfill}%
\pgfsetfillopacity{0.700000}%
\pgfsetlinewidth{0.000000pt}%
\definecolor{currentstroke}{rgb}{0.000000,0.000000,0.000000}%
\pgfsetstrokecolor{currentstroke}%
\pgfsetdash{}{0pt}%
\pgfpathmoveto{\pgfqpoint{2.338225in}{3.186416in}}%
\pgfpathlineto{\pgfqpoint{2.351942in}{3.162393in}}%
\pgfpathlineto{\pgfqpoint{2.365647in}{3.138588in}}%
\pgfpathlineto{\pgfqpoint{2.379341in}{3.115001in}}%
\pgfpathlineto{\pgfqpoint{2.393024in}{3.091630in}}%
\pgfpathlineto{\pgfqpoint{2.401748in}{3.088487in}}%
\pgfpathlineto{\pgfqpoint{2.410455in}{3.085579in}}%
\pgfpathlineto{\pgfqpoint{2.419144in}{3.082901in}}%
\pgfpathlineto{\pgfqpoint{2.427816in}{3.080452in}}%
\pgfpathlineto{\pgfqpoint{2.414180in}{3.103433in}}%
\pgfpathlineto{\pgfqpoint{2.400534in}{3.126629in}}%
\pgfpathlineto{\pgfqpoint{2.386877in}{3.150041in}}%
\pgfpathlineto{\pgfqpoint{2.373209in}{3.173670in}}%
\pgfpathlineto{\pgfqpoint{2.364490in}{3.176504in}}%
\pgfpathlineto{\pgfqpoint{2.355753in}{3.179570in}}%
\pgfpathlineto{\pgfqpoint{2.346998in}{3.182873in}}%
\pgfpathlineto{\pgfqpoint{2.338225in}{3.186416in}}%
\pgfpathclose%
\pgfusepath{fill}%
\end{pgfscope}%
\begin{pgfscope}%
\pgfpathrectangle{\pgfqpoint{1.254980in}{0.150000in}}{\pgfqpoint{5.490039in}{5.490039in}}%
\pgfusepath{clip}%
\pgfsetbuttcap%
\pgfsetroundjoin%
\definecolor{currentfill}{rgb}{0.273809,0.031497,0.358853}%
\pgfsetfillcolor{currentfill}%
\pgfsetfillopacity{0.700000}%
\pgfsetlinewidth{0.000000pt}%
\definecolor{currentstroke}{rgb}{0.000000,0.000000,0.000000}%
\pgfsetstrokecolor{currentstroke}%
\pgfsetdash{}{0pt}%
\pgfpathmoveto{\pgfqpoint{3.712867in}{1.695156in}}%
\pgfpathlineto{\pgfqpoint{3.726230in}{1.687910in}}%
\pgfpathlineto{\pgfqpoint{3.739596in}{1.680792in}}%
\pgfpathlineto{\pgfqpoint{3.752966in}{1.673802in}}%
\pgfpathlineto{\pgfqpoint{3.766340in}{1.666939in}}%
\pgfpathlineto{\pgfqpoint{3.774215in}{1.673422in}}%
\pgfpathlineto{\pgfqpoint{3.782083in}{1.680014in}}%
\pgfpathlineto{\pgfqpoint{3.789944in}{1.686715in}}%
\pgfpathlineto{\pgfqpoint{3.797799in}{1.693520in}}%
\pgfpathlineto{\pgfqpoint{3.784443in}{1.700057in}}%
\pgfpathlineto{\pgfqpoint{3.771091in}{1.706721in}}%
\pgfpathlineto{\pgfqpoint{3.757743in}{1.713513in}}%
\pgfpathlineto{\pgfqpoint{3.744399in}{1.720432in}}%
\pgfpathlineto{\pgfqpoint{3.736527in}{1.713947in}}%
\pgfpathlineto{\pgfqpoint{3.728648in}{1.707571in}}%
\pgfpathlineto{\pgfqpoint{3.720761in}{1.701307in}}%
\pgfpathlineto{\pgfqpoint{3.712867in}{1.695156in}}%
\pgfpathclose%
\pgfusepath{fill}%
\end{pgfscope}%
\begin{pgfscope}%
\pgfpathrectangle{\pgfqpoint{1.254980in}{0.150000in}}{\pgfqpoint{5.490039in}{5.490039in}}%
\pgfusepath{clip}%
\pgfsetbuttcap%
\pgfsetroundjoin%
\definecolor{currentfill}{rgb}{0.269944,0.014625,0.341379}%
\pgfsetfillcolor{currentfill}%
\pgfsetfillopacity{0.700000}%
\pgfsetlinewidth{0.000000pt}%
\definecolor{currentstroke}{rgb}{0.000000,0.000000,0.000000}%
\pgfsetstrokecolor{currentstroke}%
\pgfsetdash{}{0pt}%
\pgfpathmoveto{\pgfqpoint{3.989581in}{1.656772in}}%
\pgfpathlineto{\pgfqpoint{4.002984in}{1.652161in}}%
\pgfpathlineto{\pgfqpoint{4.016393in}{1.647672in}}%
\pgfpathlineto{\pgfqpoint{4.029807in}{1.643304in}}%
\pgfpathlineto{\pgfqpoint{4.043228in}{1.639058in}}%
\pgfpathlineto{\pgfqpoint{4.050991in}{1.647545in}}%
\pgfpathlineto{\pgfqpoint{4.058747in}{1.656106in}}%
\pgfpathlineto{\pgfqpoint{4.066499in}{1.664739in}}%
\pgfpathlineto{\pgfqpoint{4.074244in}{1.673441in}}%
\pgfpathlineto{\pgfqpoint{4.060836in}{1.677395in}}%
\pgfpathlineto{\pgfqpoint{4.047435in}{1.681470in}}%
\pgfpathlineto{\pgfqpoint{4.034039in}{1.685666in}}%
\pgfpathlineto{\pgfqpoint{4.020649in}{1.689984in}}%
\pgfpathlineto{\pgfqpoint{4.012891in}{1.681569in}}%
\pgfpathlineto{\pgfqpoint{4.005127in}{1.673227in}}%
\pgfpathlineto{\pgfqpoint{3.997357in}{1.664961in}}%
\pgfpathlineto{\pgfqpoint{3.989581in}{1.656772in}}%
\pgfpathclose%
\pgfusepath{fill}%
\end{pgfscope}%
\begin{pgfscope}%
\pgfpathrectangle{\pgfqpoint{1.254980in}{0.150000in}}{\pgfqpoint{5.490039in}{5.490039in}}%
\pgfusepath{clip}%
\pgfsetbuttcap%
\pgfsetroundjoin%
\definecolor{currentfill}{rgb}{0.174274,0.445044,0.557792}%
\pgfsetfillcolor{currentfill}%
\pgfsetfillopacity{0.700000}%
\pgfsetlinewidth{0.000000pt}%
\definecolor{currentstroke}{rgb}{0.000000,0.000000,0.000000}%
\pgfsetstrokecolor{currentstroke}%
\pgfsetdash{}{0pt}%
\pgfpathmoveto{\pgfqpoint{5.543796in}{2.555896in}}%
\pgfpathlineto{\pgfqpoint{5.557849in}{2.562444in}}%
\pgfpathlineto{\pgfqpoint{5.571918in}{2.569104in}}%
\pgfpathlineto{\pgfqpoint{5.586001in}{2.575876in}}%
\pgfpathlineto{\pgfqpoint{5.600101in}{2.582761in}}%
\pgfpathlineto{\pgfqpoint{5.607349in}{2.592689in}}%
\pgfpathlineto{\pgfqpoint{5.614590in}{2.602543in}}%
\pgfpathlineto{\pgfqpoint{5.621824in}{2.612325in}}%
\pgfpathlineto{\pgfqpoint{5.629051in}{2.622033in}}%
\pgfpathlineto{\pgfqpoint{5.614959in}{2.615161in}}%
\pgfpathlineto{\pgfqpoint{5.600882in}{2.608402in}}%
\pgfpathlineto{\pgfqpoint{5.586820in}{2.601756in}}%
\pgfpathlineto{\pgfqpoint{5.572774in}{2.595222in}}%
\pgfpathlineto{\pgfqpoint{5.565539in}{2.585493in}}%
\pgfpathlineto{\pgfqpoint{5.558298in}{2.575697in}}%
\pgfpathlineto{\pgfqpoint{5.551050in}{2.565831in}}%
\pgfpathlineto{\pgfqpoint{5.543796in}{2.555896in}}%
\pgfpathclose%
\pgfusepath{fill}%
\end{pgfscope}%
\begin{pgfscope}%
\pgfpathrectangle{\pgfqpoint{1.254980in}{0.150000in}}{\pgfqpoint{5.490039in}{5.490039in}}%
\pgfusepath{clip}%
\pgfsetbuttcap%
\pgfsetroundjoin%
\definecolor{currentfill}{rgb}{0.267968,0.223549,0.512008}%
\pgfsetfillcolor{currentfill}%
\pgfsetfillopacity{0.700000}%
\pgfsetlinewidth{0.000000pt}%
\definecolor{currentstroke}{rgb}{0.000000,0.000000,0.000000}%
\pgfsetstrokecolor{currentstroke}%
\pgfsetdash{}{0pt}%
\pgfpathmoveto{\pgfqpoint{4.889623in}{2.031924in}}%
\pgfpathlineto{\pgfqpoint{4.903337in}{2.034677in}}%
\pgfpathlineto{\pgfqpoint{4.917063in}{2.037544in}}%
\pgfpathlineto{\pgfqpoint{4.930800in}{2.040524in}}%
\pgfpathlineto{\pgfqpoint{4.944550in}{2.043618in}}%
\pgfpathlineto{\pgfqpoint{4.952039in}{2.055150in}}%
\pgfpathlineto{\pgfqpoint{4.959524in}{2.066648in}}%
\pgfpathlineto{\pgfqpoint{4.967003in}{2.078110in}}%
\pgfpathlineto{\pgfqpoint{4.974477in}{2.089536in}}%
\pgfpathlineto{\pgfqpoint{4.960731in}{2.086307in}}%
\pgfpathlineto{\pgfqpoint{4.946997in}{2.083191in}}%
\pgfpathlineto{\pgfqpoint{4.933275in}{2.080190in}}%
\pgfpathlineto{\pgfqpoint{4.919565in}{2.077302in}}%
\pgfpathlineto{\pgfqpoint{4.912087in}{2.066005in}}%
\pgfpathlineto{\pgfqpoint{4.904604in}{2.054676in}}%
\pgfpathlineto{\pgfqpoint{4.897116in}{2.043316in}}%
\pgfpathlineto{\pgfqpoint{4.889623in}{2.031924in}}%
\pgfpathclose%
\pgfusepath{fill}%
\end{pgfscope}%
\begin{pgfscope}%
\pgfpathrectangle{\pgfqpoint{1.254980in}{0.150000in}}{\pgfqpoint{5.490039in}{5.490039in}}%
\pgfusepath{clip}%
\pgfsetbuttcap%
\pgfsetroundjoin%
\definecolor{currentfill}{rgb}{0.274952,0.037752,0.364543}%
\pgfsetfillcolor{currentfill}%
\pgfsetfillopacity{0.700000}%
\pgfsetlinewidth{0.000000pt}%
\definecolor{currentstroke}{rgb}{0.000000,0.000000,0.000000}%
\pgfsetstrokecolor{currentstroke}%
\pgfsetdash{}{0pt}%
\pgfpathmoveto{\pgfqpoint{4.212578in}{1.683812in}}%
\pgfpathlineto{\pgfqpoint{4.226035in}{1.681214in}}%
\pgfpathlineto{\pgfqpoint{4.239499in}{1.678734in}}%
\pgfpathlineto{\pgfqpoint{4.252971in}{1.676372in}}%
\pgfpathlineto{\pgfqpoint{4.266451in}{1.674129in}}%
\pgfpathlineto{\pgfqpoint{4.274138in}{1.683944in}}%
\pgfpathlineto{\pgfqpoint{4.281820in}{1.693804in}}%
\pgfpathlineto{\pgfqpoint{4.289497in}{1.703707in}}%
\pgfpathlineto{\pgfqpoint{4.297169in}{1.713649in}}%
\pgfpathlineto{\pgfqpoint{4.283699in}{1.715632in}}%
\pgfpathlineto{\pgfqpoint{4.270236in}{1.717733in}}%
\pgfpathlineto{\pgfqpoint{4.256781in}{1.719952in}}%
\pgfpathlineto{\pgfqpoint{4.243333in}{1.722289in}}%
\pgfpathlineto{\pgfqpoint{4.235652in}{1.712602in}}%
\pgfpathlineto{\pgfqpoint{4.227966in}{1.702958in}}%
\pgfpathlineto{\pgfqpoint{4.220274in}{1.693361in}}%
\pgfpathlineto{\pgfqpoint{4.212578in}{1.683812in}}%
\pgfpathclose%
\pgfusepath{fill}%
\end{pgfscope}%
\begin{pgfscope}%
\pgfpathrectangle{\pgfqpoint{1.254980in}{0.150000in}}{\pgfqpoint{5.490039in}{5.490039in}}%
\pgfusepath{clip}%
\pgfsetbuttcap%
\pgfsetroundjoin%
\definecolor{currentfill}{rgb}{0.214298,0.355619,0.551184}%
\pgfsetfillcolor{currentfill}%
\pgfsetfillopacity{0.700000}%
\pgfsetlinewidth{0.000000pt}%
\definecolor{currentstroke}{rgb}{0.000000,0.000000,0.000000}%
\pgfsetstrokecolor{currentstroke}%
\pgfsetdash{}{0pt}%
\pgfpathmoveto{\pgfqpoint{5.259104in}{2.318940in}}%
\pgfpathlineto{\pgfqpoint{5.273004in}{2.324019in}}%
\pgfpathlineto{\pgfqpoint{5.286917in}{2.329211in}}%
\pgfpathlineto{\pgfqpoint{5.300845in}{2.334516in}}%
\pgfpathlineto{\pgfqpoint{5.314786in}{2.339934in}}%
\pgfpathlineto{\pgfqpoint{5.322152in}{2.350881in}}%
\pgfpathlineto{\pgfqpoint{5.329512in}{2.361766in}}%
\pgfpathlineto{\pgfqpoint{5.336866in}{2.372589in}}%
\pgfpathlineto{\pgfqpoint{5.344214in}{2.383349in}}%
\pgfpathlineto{\pgfqpoint{5.330276in}{2.377877in}}%
\pgfpathlineto{\pgfqpoint{5.316353in}{2.372519in}}%
\pgfpathlineto{\pgfqpoint{5.302445in}{2.367273in}}%
\pgfpathlineto{\pgfqpoint{5.288550in}{2.362140in}}%
\pgfpathlineto{\pgfqpoint{5.281197in}{2.351428in}}%
\pgfpathlineto{\pgfqpoint{5.273839in}{2.340657in}}%
\pgfpathlineto{\pgfqpoint{5.266475in}{2.329828in}}%
\pgfpathlineto{\pgfqpoint{5.259104in}{2.318940in}}%
\pgfpathclose%
\pgfusepath{fill}%
\end{pgfscope}%
\begin{pgfscope}%
\pgfpathrectangle{\pgfqpoint{1.254980in}{0.150000in}}{\pgfqpoint{5.490039in}{5.490039in}}%
\pgfusepath{clip}%
\pgfsetbuttcap%
\pgfsetroundjoin%
\definecolor{currentfill}{rgb}{0.143343,0.522773,0.556295}%
\pgfsetfillcolor{currentfill}%
\pgfsetfillopacity{0.700000}%
\pgfsetlinewidth{0.000000pt}%
\definecolor{currentstroke}{rgb}{0.000000,0.000000,0.000000}%
\pgfsetstrokecolor{currentstroke}%
\pgfsetdash{}{0pt}%
\pgfpathmoveto{\pgfqpoint{2.521922in}{2.837566in}}%
\pgfpathlineto{\pgfqpoint{2.535524in}{2.816463in}}%
\pgfpathlineto{\pgfqpoint{2.549118in}{2.795557in}}%
\pgfpathlineto{\pgfqpoint{2.562703in}{2.774846in}}%
\pgfpathlineto{\pgfqpoint{2.576280in}{2.754329in}}%
\pgfpathlineto{\pgfqpoint{2.584887in}{2.751859in}}%
\pgfpathlineto{\pgfqpoint{2.593477in}{2.749615in}}%
\pgfpathlineto{\pgfqpoint{2.602051in}{2.747596in}}%
\pgfpathlineto{\pgfqpoint{2.610610in}{2.745798in}}%
\pgfpathlineto{\pgfqpoint{2.597077in}{2.765918in}}%
\pgfpathlineto{\pgfqpoint{2.583536in}{2.786232in}}%
\pgfpathlineto{\pgfqpoint{2.569987in}{2.806740in}}%
\pgfpathlineto{\pgfqpoint{2.556430in}{2.827443in}}%
\pgfpathlineto{\pgfqpoint{2.547828in}{2.829632in}}%
\pgfpathlineto{\pgfqpoint{2.539209in}{2.832047in}}%
\pgfpathlineto{\pgfqpoint{2.530574in}{2.834690in}}%
\pgfpathlineto{\pgfqpoint{2.521922in}{2.837566in}}%
\pgfpathclose%
\pgfusepath{fill}%
\end{pgfscope}%
\begin{pgfscope}%
\pgfpathrectangle{\pgfqpoint{1.254980in}{0.150000in}}{\pgfqpoint{5.490039in}{5.490039in}}%
\pgfusepath{clip}%
\pgfsetbuttcap%
\pgfsetroundjoin%
\definecolor{currentfill}{rgb}{0.283072,0.130895,0.449241}%
\pgfsetfillcolor{currentfill}%
\pgfsetfillopacity{0.700000}%
\pgfsetlinewidth{0.000000pt}%
\definecolor{currentstroke}{rgb}{0.000000,0.000000,0.000000}%
\pgfsetstrokecolor{currentstroke}%
\pgfsetdash{}{0pt}%
\pgfpathmoveto{\pgfqpoint{3.328379in}{1.880892in}}%
\pgfpathlineto{\pgfqpoint{3.341744in}{1.869717in}}%
\pgfpathlineto{\pgfqpoint{3.355109in}{1.858683in}}%
\pgfpathlineto{\pgfqpoint{3.368474in}{1.847788in}}%
\pgfpathlineto{\pgfqpoint{3.381840in}{1.837034in}}%
\pgfpathlineto{\pgfqpoint{3.389916in}{1.840330in}}%
\pgfpathlineto{\pgfqpoint{3.397982in}{1.843784in}}%
\pgfpathlineto{\pgfqpoint{3.406039in}{1.847393in}}%
\pgfpathlineto{\pgfqpoint{3.414086in}{1.851154in}}%
\pgfpathlineto{\pgfqpoint{3.400747in}{1.861545in}}%
\pgfpathlineto{\pgfqpoint{3.387408in}{1.872074in}}%
\pgfpathlineto{\pgfqpoint{3.374070in}{1.882744in}}%
\pgfpathlineto{\pgfqpoint{3.360732in}{1.893554in}}%
\pgfpathlineto{\pgfqpoint{3.352659in}{1.890151in}}%
\pgfpathlineto{\pgfqpoint{3.344576in}{1.886904in}}%
\pgfpathlineto{\pgfqpoint{3.336482in}{1.883817in}}%
\pgfpathlineto{\pgfqpoint{3.328379in}{1.880892in}}%
\pgfpathclose%
\pgfusepath{fill}%
\end{pgfscope}%
\begin{pgfscope}%
\pgfpathrectangle{\pgfqpoint{1.254980in}{0.150000in}}{\pgfqpoint{5.490039in}{5.490039in}}%
\pgfusepath{clip}%
\pgfsetbuttcap%
\pgfsetroundjoin%
\definecolor{currentfill}{rgb}{0.277941,0.056324,0.381191}%
\pgfsetfillcolor{currentfill}%
\pgfsetfillopacity{0.700000}%
\pgfsetlinewidth{0.000000pt}%
\definecolor{currentstroke}{rgb}{0.000000,0.000000,0.000000}%
\pgfsetstrokecolor{currentstroke}%
\pgfsetdash{}{0pt}%
\pgfpathmoveto{\pgfqpoint{3.574262in}{1.737129in}}%
\pgfpathlineto{\pgfqpoint{3.587622in}{1.728498in}}%
\pgfpathlineto{\pgfqpoint{3.600985in}{1.720000in}}%
\pgfpathlineto{\pgfqpoint{3.614350in}{1.711633in}}%
\pgfpathlineto{\pgfqpoint{3.627717in}{1.703397in}}%
\pgfpathlineto{\pgfqpoint{3.635662in}{1.708713in}}%
\pgfpathlineto{\pgfqpoint{3.643599in}{1.714160in}}%
\pgfpathlineto{\pgfqpoint{3.651528in}{1.719732in}}%
\pgfpathlineto{\pgfqpoint{3.659449in}{1.725429in}}%
\pgfpathlineto{\pgfqpoint{3.646103in}{1.733321in}}%
\pgfpathlineto{\pgfqpoint{3.632759in}{1.741344in}}%
\pgfpathlineto{\pgfqpoint{3.619418in}{1.749499in}}%
\pgfpathlineto{\pgfqpoint{3.606079in}{1.757785in}}%
\pgfpathlineto{\pgfqpoint{3.598137in}{1.752426in}}%
\pgfpathlineto{\pgfqpoint{3.590187in}{1.747195in}}%
\pgfpathlineto{\pgfqpoint{3.582229in}{1.742095in}}%
\pgfpathlineto{\pgfqpoint{3.574262in}{1.737129in}}%
\pgfpathclose%
\pgfusepath{fill}%
\end{pgfscope}%
\begin{pgfscope}%
\pgfpathrectangle{\pgfqpoint{1.254980in}{0.150000in}}{\pgfqpoint{5.490039in}{5.490039in}}%
\pgfusepath{clip}%
\pgfsetbuttcap%
\pgfsetroundjoin%
\definecolor{currentfill}{rgb}{0.257322,0.256130,0.526563}%
\pgfsetfillcolor{currentfill}%
\pgfsetfillopacity{0.700000}%
\pgfsetlinewidth{0.000000pt}%
\definecolor{currentstroke}{rgb}{0.000000,0.000000,0.000000}%
\pgfsetstrokecolor{currentstroke}%
\pgfsetdash{}{0pt}%
\pgfpathmoveto{\pgfqpoint{4.974477in}{2.089536in}}%
\pgfpathlineto{\pgfqpoint{4.988235in}{2.092878in}}%
\pgfpathlineto{\pgfqpoint{5.002006in}{2.096334in}}%
\pgfpathlineto{\pgfqpoint{5.015789in}{2.099903in}}%
\pgfpathlineto{\pgfqpoint{5.029584in}{2.103585in}}%
\pgfpathlineto{\pgfqpoint{5.037050in}{2.115098in}}%
\pgfpathlineto{\pgfqpoint{5.044510in}{2.126569in}}%
\pgfpathlineto{\pgfqpoint{5.051965in}{2.137998in}}%
\pgfpathlineto{\pgfqpoint{5.059415in}{2.149382in}}%
\pgfpathlineto{\pgfqpoint{5.045623in}{2.145581in}}%
\pgfpathlineto{\pgfqpoint{5.031844in}{2.141893in}}%
\pgfpathlineto{\pgfqpoint{5.018077in}{2.138318in}}%
\pgfpathlineto{\pgfqpoint{5.004323in}{2.134857in}}%
\pgfpathlineto{\pgfqpoint{4.996869in}{2.123585in}}%
\pgfpathlineto{\pgfqpoint{4.989410in}{2.112274in}}%
\pgfpathlineto{\pgfqpoint{4.981946in}{2.100924in}}%
\pgfpathlineto{\pgfqpoint{4.974477in}{2.089536in}}%
\pgfpathclose%
\pgfusepath{fill}%
\end{pgfscope}%
\begin{pgfscope}%
\pgfpathrectangle{\pgfqpoint{1.254980in}{0.150000in}}{\pgfqpoint{5.490039in}{5.490039in}}%
\pgfusepath{clip}%
\pgfsetbuttcap%
\pgfsetroundjoin%
\definecolor{currentfill}{rgb}{0.162142,0.474838,0.558140}%
\pgfsetfillcolor{currentfill}%
\pgfsetfillopacity{0.700000}%
\pgfsetlinewidth{0.000000pt}%
\definecolor{currentstroke}{rgb}{0.000000,0.000000,0.000000}%
\pgfsetstrokecolor{currentstroke}%
\pgfsetdash{}{0pt}%
\pgfpathmoveto{\pgfqpoint{5.629051in}{2.622033in}}%
\pgfpathlineto{\pgfqpoint{5.643159in}{2.629018in}}%
\pgfpathlineto{\pgfqpoint{5.657283in}{2.636115in}}%
\pgfpathlineto{\pgfqpoint{5.671423in}{2.643324in}}%
\pgfpathlineto{\pgfqpoint{5.685579in}{2.650646in}}%
\pgfpathlineto{\pgfqpoint{5.692792in}{2.660259in}}%
\pgfpathlineto{\pgfqpoint{5.699998in}{2.669795in}}%
\pgfpathlineto{\pgfqpoint{5.707197in}{2.679256in}}%
\pgfpathlineto{\pgfqpoint{5.714389in}{2.688643in}}%
\pgfpathlineto{\pgfqpoint{5.700241in}{2.681351in}}%
\pgfpathlineto{\pgfqpoint{5.686109in}{2.674172in}}%
\pgfpathlineto{\pgfqpoint{5.671993in}{2.667105in}}%
\pgfpathlineto{\pgfqpoint{5.657892in}{2.660151in}}%
\pgfpathlineto{\pgfqpoint{5.650692in}{2.650728in}}%
\pgfpathlineto{\pgfqpoint{5.643485in}{2.641234in}}%
\pgfpathlineto{\pgfqpoint{5.636272in}{2.631670in}}%
\pgfpathlineto{\pgfqpoint{5.629051in}{2.622033in}}%
\pgfpathclose%
\pgfusepath{fill}%
\end{pgfscope}%
\begin{pgfscope}%
\pgfpathrectangle{\pgfqpoint{1.254980in}{0.150000in}}{\pgfqpoint{5.490039in}{5.490039in}}%
\pgfusepath{clip}%
\pgfsetbuttcap%
\pgfsetroundjoin%
\definecolor{currentfill}{rgb}{0.272594,0.025563,0.353093}%
\pgfsetfillcolor{currentfill}%
\pgfsetfillopacity{0.700000}%
\pgfsetlinewidth{0.000000pt}%
\definecolor{currentstroke}{rgb}{0.000000,0.000000,0.000000}%
\pgfsetstrokecolor{currentstroke}%
\pgfsetdash{}{0pt}%
\pgfpathmoveto{\pgfqpoint{4.127939in}{1.658833in}}%
\pgfpathlineto{\pgfqpoint{4.141379in}{1.655482in}}%
\pgfpathlineto{\pgfqpoint{4.154825in}{1.652250in}}%
\pgfpathlineto{\pgfqpoint{4.168279in}{1.649137in}}%
\pgfpathlineto{\pgfqpoint{4.181739in}{1.646143in}}%
\pgfpathlineto{\pgfqpoint{4.189457in}{1.655477in}}%
\pgfpathlineto{\pgfqpoint{4.197169in}{1.664868in}}%
\pgfpathlineto{\pgfqpoint{4.204876in}{1.674314in}}%
\pgfpathlineto{\pgfqpoint{4.212578in}{1.683812in}}%
\pgfpathlineto{\pgfqpoint{4.199128in}{1.686530in}}%
\pgfpathlineto{\pgfqpoint{4.185685in}{1.689366in}}%
\pgfpathlineto{\pgfqpoint{4.172249in}{1.692321in}}%
\pgfpathlineto{\pgfqpoint{4.158820in}{1.695396in}}%
\pgfpathlineto{\pgfqpoint{4.151108in}{1.686169in}}%
\pgfpathlineto{\pgfqpoint{4.143390in}{1.676998in}}%
\pgfpathlineto{\pgfqpoint{4.135667in}{1.667885in}}%
\pgfpathlineto{\pgfqpoint{4.127939in}{1.658833in}}%
\pgfpathclose%
\pgfusepath{fill}%
\end{pgfscope}%
\begin{pgfscope}%
\pgfpathrectangle{\pgfqpoint{1.254980in}{0.150000in}}{\pgfqpoint{5.490039in}{5.490039in}}%
\pgfusepath{clip}%
\pgfsetbuttcap%
\pgfsetroundjoin%
\definecolor{currentfill}{rgb}{0.201239,0.383670,0.554294}%
\pgfsetfillcolor{currentfill}%
\pgfsetfillopacity{0.700000}%
\pgfsetlinewidth{0.000000pt}%
\definecolor{currentstroke}{rgb}{0.000000,0.000000,0.000000}%
\pgfsetstrokecolor{currentstroke}%
\pgfsetdash{}{0pt}%
\pgfpathmoveto{\pgfqpoint{5.344214in}{2.383349in}}%
\pgfpathlineto{\pgfqpoint{5.358165in}{2.388933in}}%
\pgfpathlineto{\pgfqpoint{5.372131in}{2.394630in}}%
\pgfpathlineto{\pgfqpoint{5.386111in}{2.400440in}}%
\pgfpathlineto{\pgfqpoint{5.400106in}{2.406362in}}%
\pgfpathlineto{\pgfqpoint{5.407443in}{2.417103in}}%
\pgfpathlineto{\pgfqpoint{5.414773in}{2.427776in}}%
\pgfpathlineto{\pgfqpoint{5.422098in}{2.438383in}}%
\pgfpathlineto{\pgfqpoint{5.429416in}{2.448922in}}%
\pgfpathlineto{\pgfqpoint{5.415426in}{2.442962in}}%
\pgfpathlineto{\pgfqpoint{5.401451in}{2.437115in}}%
\pgfpathlineto{\pgfqpoint{5.387490in}{2.431381in}}%
\pgfpathlineto{\pgfqpoint{5.373544in}{2.425760in}}%
\pgfpathlineto{\pgfqpoint{5.366220in}{2.415251in}}%
\pgfpathlineto{\pgfqpoint{5.358891in}{2.404680in}}%
\pgfpathlineto{\pgfqpoint{5.351555in}{2.394046in}}%
\pgfpathlineto{\pgfqpoint{5.344214in}{2.383349in}}%
\pgfpathclose%
\pgfusepath{fill}%
\end{pgfscope}%
\begin{pgfscope}%
\pgfpathrectangle{\pgfqpoint{1.254980in}{0.150000in}}{\pgfqpoint{5.490039in}{5.490039in}}%
\pgfusepath{clip}%
\pgfsetbuttcap%
\pgfsetroundjoin%
\definecolor{currentfill}{rgb}{0.131172,0.555899,0.552459}%
\pgfsetfillcolor{currentfill}%
\pgfsetfillopacity{0.700000}%
\pgfsetlinewidth{0.000000pt}%
\definecolor{currentstroke}{rgb}{0.000000,0.000000,0.000000}%
\pgfsetstrokecolor{currentstroke}%
\pgfsetdash{}{0pt}%
\pgfpathmoveto{\pgfqpoint{2.467422in}{2.923965in}}%
\pgfpathlineto{\pgfqpoint{2.481061in}{2.902064in}}%
\pgfpathlineto{\pgfqpoint{2.494691in}{2.880364in}}%
\pgfpathlineto{\pgfqpoint{2.508311in}{2.858866in}}%
\pgfpathlineto{\pgfqpoint{2.521922in}{2.837566in}}%
\pgfpathlineto{\pgfqpoint{2.530574in}{2.834690in}}%
\pgfpathlineto{\pgfqpoint{2.539209in}{2.832047in}}%
\pgfpathlineto{\pgfqpoint{2.547828in}{2.829632in}}%
\pgfpathlineto{\pgfqpoint{2.556430in}{2.827443in}}%
\pgfpathlineto{\pgfqpoint{2.542864in}{2.848344in}}%
\pgfpathlineto{\pgfqpoint{2.529290in}{2.869442in}}%
\pgfpathlineto{\pgfqpoint{2.515706in}{2.890740in}}%
\pgfpathlineto{\pgfqpoint{2.502114in}{2.912238in}}%
\pgfpathlineto{\pgfqpoint{2.493467in}{2.914820in}}%
\pgfpathlineto{\pgfqpoint{2.484802in}{2.917633in}}%
\pgfpathlineto{\pgfqpoint{2.476121in}{2.920680in}}%
\pgfpathlineto{\pgfqpoint{2.467422in}{2.923965in}}%
\pgfpathclose%
\pgfusepath{fill}%
\end{pgfscope}%
\begin{pgfscope}%
\pgfpathrectangle{\pgfqpoint{1.254980in}{0.150000in}}{\pgfqpoint{5.490039in}{5.490039in}}%
\pgfusepath{clip}%
\pgfsetbuttcap%
\pgfsetroundjoin%
\definecolor{currentfill}{rgb}{0.283091,0.110553,0.431554}%
\pgfsetfillcolor{currentfill}%
\pgfsetfillopacity{0.700000}%
\pgfsetlinewidth{0.000000pt}%
\definecolor{currentstroke}{rgb}{0.000000,0.000000,0.000000}%
\pgfsetstrokecolor{currentstroke}%
\pgfsetdash{}{0pt}%
\pgfpathmoveto{\pgfqpoint{3.381840in}{1.837034in}}%
\pgfpathlineto{\pgfqpoint{3.395206in}{1.826418in}}%
\pgfpathlineto{\pgfqpoint{3.408573in}{1.815941in}}%
\pgfpathlineto{\pgfqpoint{3.421941in}{1.805601in}}%
\pgfpathlineto{\pgfqpoint{3.435310in}{1.795399in}}%
\pgfpathlineto{\pgfqpoint{3.443360in}{1.799065in}}%
\pgfpathlineto{\pgfqpoint{3.451400in}{1.802885in}}%
\pgfpathlineto{\pgfqpoint{3.459432in}{1.806856in}}%
\pgfpathlineto{\pgfqpoint{3.467454in}{1.810974in}}%
\pgfpathlineto{\pgfqpoint{3.454110in}{1.820813in}}%
\pgfpathlineto{\pgfqpoint{3.440768in}{1.830789in}}%
\pgfpathlineto{\pgfqpoint{3.427426in}{1.840903in}}%
\pgfpathlineto{\pgfqpoint{3.414086in}{1.851154in}}%
\pgfpathlineto{\pgfqpoint{3.406039in}{1.847393in}}%
\pgfpathlineto{\pgfqpoint{3.397982in}{1.843784in}}%
\pgfpathlineto{\pgfqpoint{3.389916in}{1.840330in}}%
\pgfpathlineto{\pgfqpoint{3.381840in}{1.837034in}}%
\pgfpathclose%
\pgfusepath{fill}%
\end{pgfscope}%
\begin{pgfscope}%
\pgfpathrectangle{\pgfqpoint{1.254980in}{0.150000in}}{\pgfqpoint{5.490039in}{5.490039in}}%
\pgfusepath{clip}%
\pgfsetbuttcap%
\pgfsetroundjoin%
\definecolor{currentfill}{rgb}{0.151918,0.500685,0.557587}%
\pgfsetfillcolor{currentfill}%
\pgfsetfillopacity{0.700000}%
\pgfsetlinewidth{0.000000pt}%
\definecolor{currentstroke}{rgb}{0.000000,0.000000,0.000000}%
\pgfsetstrokecolor{currentstroke}%
\pgfsetdash{}{0pt}%
\pgfpathmoveto{\pgfqpoint{5.714389in}{2.688643in}}%
\pgfpathlineto{\pgfqpoint{5.728554in}{2.696047in}}%
\pgfpathlineto{\pgfqpoint{5.742734in}{2.703564in}}%
\pgfpathlineto{\pgfqpoint{5.756931in}{2.711193in}}%
\pgfpathlineto{\pgfqpoint{5.764110in}{2.720475in}}%
\pgfpathlineto{\pgfqpoint{5.771281in}{2.729679in}}%
\pgfpathlineto{\pgfqpoint{5.778445in}{2.738808in}}%
\pgfpathlineto{\pgfqpoint{5.785602in}{2.747861in}}%
\pgfpathlineto{\pgfqpoint{5.771414in}{2.740280in}}%
\pgfpathlineto{\pgfqpoint{5.757242in}{2.732810in}}%
\pgfpathlineto{\pgfqpoint{5.743086in}{2.725454in}}%
\pgfpathlineto{\pgfqpoint{5.735923in}{2.716360in}}%
\pgfpathlineto{\pgfqpoint{5.728752in}{2.707194in}}%
\pgfpathlineto{\pgfqpoint{5.721574in}{2.697955in}}%
\pgfpathlineto{\pgfqpoint{5.714389in}{2.688643in}}%
\pgfpathclose%
\pgfusepath{fill}%
\end{pgfscope}%
\begin{pgfscope}%
\pgfpathrectangle{\pgfqpoint{1.254980in}{0.150000in}}{\pgfqpoint{5.490039in}{5.490039in}}%
\pgfusepath{clip}%
\pgfsetbuttcap%
\pgfsetroundjoin%
\definecolor{currentfill}{rgb}{0.272594,0.025563,0.353093}%
\pgfsetfillcolor{currentfill}%
\pgfsetfillopacity{0.700000}%
\pgfsetlinewidth{0.000000pt}%
\definecolor{currentstroke}{rgb}{0.000000,0.000000,0.000000}%
\pgfsetstrokecolor{currentstroke}%
\pgfsetdash{}{0pt}%
\pgfpathmoveto{\pgfqpoint{3.766340in}{1.666939in}}%
\pgfpathlineto{\pgfqpoint{3.779717in}{1.660202in}}%
\pgfpathlineto{\pgfqpoint{3.793099in}{1.653592in}}%
\pgfpathlineto{\pgfqpoint{3.806484in}{1.647108in}}%
\pgfpathlineto{\pgfqpoint{3.819874in}{1.640749in}}%
\pgfpathlineto{\pgfqpoint{3.827732in}{1.647564in}}%
\pgfpathlineto{\pgfqpoint{3.835582in}{1.654485in}}%
\pgfpathlineto{\pgfqpoint{3.843426in}{1.661509in}}%
\pgfpathlineto{\pgfqpoint{3.851264in}{1.668634in}}%
\pgfpathlineto{\pgfqpoint{3.837891in}{1.674667in}}%
\pgfpathlineto{\pgfqpoint{3.824523in}{1.680825in}}%
\pgfpathlineto{\pgfqpoint{3.811159in}{1.687109in}}%
\pgfpathlineto{\pgfqpoint{3.797799in}{1.693520in}}%
\pgfpathlineto{\pgfqpoint{3.789944in}{1.686715in}}%
\pgfpathlineto{\pgfqpoint{3.782083in}{1.680014in}}%
\pgfpathlineto{\pgfqpoint{3.774215in}{1.673422in}}%
\pgfpathlineto{\pgfqpoint{3.766340in}{1.666939in}}%
\pgfpathclose%
\pgfusepath{fill}%
\end{pgfscope}%
\begin{pgfscope}%
\pgfpathrectangle{\pgfqpoint{1.254980in}{0.150000in}}{\pgfqpoint{5.490039in}{5.490039in}}%
\pgfusepath{clip}%
\pgfsetbuttcap%
\pgfsetroundjoin%
\definecolor{currentfill}{rgb}{0.269944,0.014625,0.341379}%
\pgfsetfillcolor{currentfill}%
\pgfsetfillopacity{0.700000}%
\pgfsetlinewidth{0.000000pt}%
\definecolor{currentstroke}{rgb}{0.000000,0.000000,0.000000}%
\pgfsetstrokecolor{currentstroke}%
\pgfsetdash{}{0pt}%
\pgfpathmoveto{\pgfqpoint{3.904801in}{1.645750in}}%
\pgfpathlineto{\pgfqpoint{3.918197in}{1.640338in}}%
\pgfpathlineto{\pgfqpoint{3.931599in}{1.635051in}}%
\pgfpathlineto{\pgfqpoint{3.945005in}{1.629886in}}%
\pgfpathlineto{\pgfqpoint{3.958417in}{1.624843in}}%
\pgfpathlineto{\pgfqpoint{3.966217in}{1.632697in}}%
\pgfpathlineto{\pgfqpoint{3.974011in}{1.640638in}}%
\pgfpathlineto{\pgfqpoint{3.981799in}{1.648663in}}%
\pgfpathlineto{\pgfqpoint{3.989581in}{1.656772in}}%
\pgfpathlineto{\pgfqpoint{3.976183in}{1.661505in}}%
\pgfpathlineto{\pgfqpoint{3.962791in}{1.666361in}}%
\pgfpathlineto{\pgfqpoint{3.949404in}{1.671340in}}%
\pgfpathlineto{\pgfqpoint{3.936023in}{1.676442in}}%
\pgfpathlineto{\pgfqpoint{3.928227in}{1.668636in}}%
\pgfpathlineto{\pgfqpoint{3.920424in}{1.660918in}}%
\pgfpathlineto{\pgfqpoint{3.912616in}{1.653288in}}%
\pgfpathlineto{\pgfqpoint{3.904801in}{1.645750in}}%
\pgfpathclose%
\pgfusepath{fill}%
\end{pgfscope}%
\begin{pgfscope}%
\pgfpathrectangle{\pgfqpoint{1.254980in}{0.150000in}}{\pgfqpoint{5.490039in}{5.490039in}}%
\pgfusepath{clip}%
\pgfsetbuttcap%
\pgfsetroundjoin%
\definecolor{currentfill}{rgb}{0.246811,0.283237,0.535941}%
\pgfsetfillcolor{currentfill}%
\pgfsetfillopacity{0.700000}%
\pgfsetlinewidth{0.000000pt}%
\definecolor{currentstroke}{rgb}{0.000000,0.000000,0.000000}%
\pgfsetstrokecolor{currentstroke}%
\pgfsetdash{}{0pt}%
\pgfpathmoveto{\pgfqpoint{5.059415in}{2.149382in}}%
\pgfpathlineto{\pgfqpoint{5.073220in}{2.153297in}}%
\pgfpathlineto{\pgfqpoint{5.087038in}{2.157325in}}%
\pgfpathlineto{\pgfqpoint{5.100868in}{2.161466in}}%
\pgfpathlineto{\pgfqpoint{5.114711in}{2.165720in}}%
\pgfpathlineto{\pgfqpoint{5.122153in}{2.177169in}}%
\pgfpathlineto{\pgfqpoint{5.129588in}{2.188569in}}%
\pgfpathlineto{\pgfqpoint{5.137019in}{2.199920in}}%
\pgfpathlineto{\pgfqpoint{5.144444in}{2.211220in}}%
\pgfpathlineto{\pgfqpoint{5.130604in}{2.206863in}}%
\pgfpathlineto{\pgfqpoint{5.116777in}{2.202619in}}%
\pgfpathlineto{\pgfqpoint{5.102963in}{2.198488in}}%
\pgfpathlineto{\pgfqpoint{5.089162in}{2.194471in}}%
\pgfpathlineto{\pgfqpoint{5.081733in}{2.183267in}}%
\pgfpathlineto{\pgfqpoint{5.074299in}{2.172018in}}%
\pgfpathlineto{\pgfqpoint{5.066860in}{2.160723in}}%
\pgfpathlineto{\pgfqpoint{5.059415in}{2.149382in}}%
\pgfpathclose%
\pgfusepath{fill}%
\end{pgfscope}%
\begin{pgfscope}%
\pgfpathrectangle{\pgfqpoint{1.254980in}{0.150000in}}{\pgfqpoint{5.490039in}{5.490039in}}%
\pgfusepath{clip}%
\pgfsetbuttcap%
\pgfsetroundjoin%
\definecolor{currentfill}{rgb}{0.282910,0.105393,0.426902}%
\pgfsetfillcolor{currentfill}%
\pgfsetfillopacity{0.700000}%
\pgfsetlinewidth{0.000000pt}%
\definecolor{currentstroke}{rgb}{0.000000,0.000000,0.000000}%
\pgfsetstrokecolor{currentstroke}%
\pgfsetdash{}{0pt}%
\pgfpathmoveto{\pgfqpoint{4.520477in}{1.785450in}}%
\pgfpathlineto{\pgfqpoint{4.534044in}{1.785458in}}%
\pgfpathlineto{\pgfqpoint{4.547621in}{1.785581in}}%
\pgfpathlineto{\pgfqpoint{4.561208in}{1.785820in}}%
\pgfpathlineto{\pgfqpoint{4.574804in}{1.786173in}}%
\pgfpathlineto{\pgfqpoint{4.582405in}{1.797274in}}%
\pgfpathlineto{\pgfqpoint{4.590002in}{1.808382in}}%
\pgfpathlineto{\pgfqpoint{4.597594in}{1.819495in}}%
\pgfpathlineto{\pgfqpoint{4.605181in}{1.830611in}}%
\pgfpathlineto{\pgfqpoint{4.591590in}{1.830043in}}%
\pgfpathlineto{\pgfqpoint{4.578009in}{1.829590in}}%
\pgfpathlineto{\pgfqpoint{4.564438in}{1.829253in}}%
\pgfpathlineto{\pgfqpoint{4.550876in}{1.829031in}}%
\pgfpathlineto{\pgfqpoint{4.543283in}{1.818123in}}%
\pgfpathlineto{\pgfqpoint{4.535686in}{1.807222in}}%
\pgfpathlineto{\pgfqpoint{4.528084in}{1.796331in}}%
\pgfpathlineto{\pgfqpoint{4.520477in}{1.785450in}}%
\pgfpathclose%
\pgfusepath{fill}%
\end{pgfscope}%
\begin{pgfscope}%
\pgfpathrectangle{\pgfqpoint{1.254980in}{0.150000in}}{\pgfqpoint{5.490039in}{5.490039in}}%
\pgfusepath{clip}%
\pgfsetbuttcap%
\pgfsetroundjoin%
\definecolor{currentfill}{rgb}{0.237441,0.305202,0.541921}%
\pgfsetfillcolor{currentfill}%
\pgfsetfillopacity{0.700000}%
\pgfsetlinewidth{0.000000pt}%
\definecolor{currentstroke}{rgb}{0.000000,0.000000,0.000000}%
\pgfsetstrokecolor{currentstroke}%
\pgfsetdash{}{0pt}%
\pgfpathmoveto{\pgfqpoint{2.920327in}{2.254560in}}%
\pgfpathlineto{\pgfqpoint{2.933773in}{2.238766in}}%
\pgfpathlineto{\pgfqpoint{2.947215in}{2.223133in}}%
\pgfpathlineto{\pgfqpoint{2.960654in}{2.207662in}}%
\pgfpathlineto{\pgfqpoint{2.974090in}{2.192352in}}%
\pgfpathlineto{\pgfqpoint{2.982432in}{2.192227in}}%
\pgfpathlineto{\pgfqpoint{2.990761in}{2.192305in}}%
\pgfpathlineto{\pgfqpoint{2.999076in}{2.192581in}}%
\pgfpathlineto{\pgfqpoint{3.007379in}{2.193054in}}%
\pgfpathlineto{\pgfqpoint{2.993980in}{2.207971in}}%
\pgfpathlineto{\pgfqpoint{2.980577in}{2.223047in}}%
\pgfpathlineto{\pgfqpoint{2.967171in}{2.238284in}}%
\pgfpathlineto{\pgfqpoint{2.953761in}{2.253683in}}%
\pgfpathlineto{\pgfqpoint{2.945423in}{2.253599in}}%
\pgfpathlineto{\pgfqpoint{2.937071in}{2.253714in}}%
\pgfpathlineto{\pgfqpoint{2.928706in}{2.254034in}}%
\pgfpathlineto{\pgfqpoint{2.920327in}{2.254560in}}%
\pgfpathclose%
\pgfusepath{fill}%
\end{pgfscope}%
\begin{pgfscope}%
\pgfpathrectangle{\pgfqpoint{1.254980in}{0.150000in}}{\pgfqpoint{5.490039in}{5.490039in}}%
\pgfusepath{clip}%
\pgfsetbuttcap%
\pgfsetroundjoin%
\definecolor{currentfill}{rgb}{0.283072,0.130895,0.449241}%
\pgfsetfillcolor{currentfill}%
\pgfsetfillopacity{0.700000}%
\pgfsetlinewidth{0.000000pt}%
\definecolor{currentstroke}{rgb}{0.000000,0.000000,0.000000}%
\pgfsetstrokecolor{currentstroke}%
\pgfsetdash{}{0pt}%
\pgfpathmoveto{\pgfqpoint{4.605181in}{1.830611in}}%
\pgfpathlineto{\pgfqpoint{4.618782in}{1.831294in}}%
\pgfpathlineto{\pgfqpoint{4.632393in}{1.832091in}}%
\pgfpathlineto{\pgfqpoint{4.646014in}{1.833003in}}%
\pgfpathlineto{\pgfqpoint{4.659645in}{1.834030in}}%
\pgfpathlineto{\pgfqpoint{4.667223in}{1.845353in}}%
\pgfpathlineto{\pgfqpoint{4.674796in}{1.856671in}}%
\pgfpathlineto{\pgfqpoint{4.682364in}{1.867985in}}%
\pgfpathlineto{\pgfqpoint{4.689928in}{1.879293in}}%
\pgfpathlineto{\pgfqpoint{4.676302in}{1.878067in}}%
\pgfpathlineto{\pgfqpoint{4.662685in}{1.876956in}}%
\pgfpathlineto{\pgfqpoint{4.649080in}{1.875960in}}%
\pgfpathlineto{\pgfqpoint{4.635484in}{1.875079in}}%
\pgfpathlineto{\pgfqpoint{4.627915in}{1.863964in}}%
\pgfpathlineto{\pgfqpoint{4.620342in}{1.852847in}}%
\pgfpathlineto{\pgfqpoint{4.612764in}{1.841729in}}%
\pgfpathlineto{\pgfqpoint{4.605181in}{1.830611in}}%
\pgfpathclose%
\pgfusepath{fill}%
\end{pgfscope}%
\begin{pgfscope}%
\pgfpathrectangle{\pgfqpoint{1.254980in}{0.150000in}}{\pgfqpoint{5.490039in}{5.490039in}}%
\pgfusepath{clip}%
\pgfsetbuttcap%
\pgfsetroundjoin%
\definecolor{currentfill}{rgb}{0.225863,0.330805,0.547314}%
\pgfsetfillcolor{currentfill}%
\pgfsetfillopacity{0.700000}%
\pgfsetlinewidth{0.000000pt}%
\definecolor{currentstroke}{rgb}{0.000000,0.000000,0.000000}%
\pgfsetstrokecolor{currentstroke}%
\pgfsetdash{}{0pt}%
\pgfpathmoveto{\pgfqpoint{2.866501in}{2.319375in}}%
\pgfpathlineto{\pgfqpoint{2.879964in}{2.302924in}}%
\pgfpathlineto{\pgfqpoint{2.893422in}{2.286638in}}%
\pgfpathlineto{\pgfqpoint{2.906876in}{2.270517in}}%
\pgfpathlineto{\pgfqpoint{2.920327in}{2.254560in}}%
\pgfpathlineto{\pgfqpoint{2.928706in}{2.254034in}}%
\pgfpathlineto{\pgfqpoint{2.937071in}{2.253714in}}%
\pgfpathlineto{\pgfqpoint{2.945423in}{2.253599in}}%
\pgfpathlineto{\pgfqpoint{2.953761in}{2.253683in}}%
\pgfpathlineto{\pgfqpoint{2.940348in}{2.269244in}}%
\pgfpathlineto{\pgfqpoint{2.926931in}{2.284968in}}%
\pgfpathlineto{\pgfqpoint{2.913511in}{2.300857in}}%
\pgfpathlineto{\pgfqpoint{2.900086in}{2.316910in}}%
\pgfpathlineto{\pgfqpoint{2.891711in}{2.317216in}}%
\pgfpathlineto{\pgfqpoint{2.883321in}{2.317726in}}%
\pgfpathlineto{\pgfqpoint{2.874918in}{2.318445in}}%
\pgfpathlineto{\pgfqpoint{2.866501in}{2.319375in}}%
\pgfpathclose%
\pgfusepath{fill}%
\end{pgfscope}%
\begin{pgfscope}%
\pgfpathrectangle{\pgfqpoint{1.254980in}{0.150000in}}{\pgfqpoint{5.490039in}{5.490039in}}%
\pgfusepath{clip}%
\pgfsetbuttcap%
\pgfsetroundjoin%
\definecolor{currentfill}{rgb}{0.281446,0.084320,0.407414}%
\pgfsetfillcolor{currentfill}%
\pgfsetfillopacity{0.700000}%
\pgfsetlinewidth{0.000000pt}%
\definecolor{currentstroke}{rgb}{0.000000,0.000000,0.000000}%
\pgfsetstrokecolor{currentstroke}%
\pgfsetdash{}{0pt}%
\pgfpathmoveto{\pgfqpoint{4.435799in}{1.744108in}}%
\pgfpathlineto{\pgfqpoint{4.449336in}{1.743423in}}%
\pgfpathlineto{\pgfqpoint{4.462883in}{1.742854in}}%
\pgfpathlineto{\pgfqpoint{4.476438in}{1.742401in}}%
\pgfpathlineto{\pgfqpoint{4.490002in}{1.742064in}}%
\pgfpathlineto{\pgfqpoint{4.497628in}{1.752886in}}%
\pgfpathlineto{\pgfqpoint{4.505249in}{1.763726in}}%
\pgfpathlineto{\pgfqpoint{4.512865in}{1.774581in}}%
\pgfpathlineto{\pgfqpoint{4.520477in}{1.785450in}}%
\pgfpathlineto{\pgfqpoint{4.506918in}{1.785557in}}%
\pgfpathlineto{\pgfqpoint{4.493370in}{1.785780in}}%
\pgfpathlineto{\pgfqpoint{4.479830in}{1.786119in}}%
\pgfpathlineto{\pgfqpoint{4.466299in}{1.786575in}}%
\pgfpathlineto{\pgfqpoint{4.458681in}{1.775929in}}%
\pgfpathlineto{\pgfqpoint{4.451058in}{1.765302in}}%
\pgfpathlineto{\pgfqpoint{4.443431in}{1.754694in}}%
\pgfpathlineto{\pgfqpoint{4.435799in}{1.744108in}}%
\pgfpathclose%
\pgfusepath{fill}%
\end{pgfscope}%
\begin{pgfscope}%
\pgfpathrectangle{\pgfqpoint{1.254980in}{0.150000in}}{\pgfqpoint{5.490039in}{5.490039in}}%
\pgfusepath{clip}%
\pgfsetbuttcap%
\pgfsetroundjoin%
\definecolor{currentfill}{rgb}{0.248629,0.278775,0.534556}%
\pgfsetfillcolor{currentfill}%
\pgfsetfillopacity{0.700000}%
\pgfsetlinewidth{0.000000pt}%
\definecolor{currentstroke}{rgb}{0.000000,0.000000,0.000000}%
\pgfsetstrokecolor{currentstroke}%
\pgfsetdash{}{0pt}%
\pgfpathmoveto{\pgfqpoint{2.974090in}{2.192352in}}%
\pgfpathlineto{\pgfqpoint{2.987522in}{2.177200in}}%
\pgfpathlineto{\pgfqpoint{3.000951in}{2.162207in}}%
\pgfpathlineto{\pgfqpoint{3.014377in}{2.147372in}}%
\pgfpathlineto{\pgfqpoint{3.027800in}{2.132694in}}%
\pgfpathlineto{\pgfqpoint{3.036106in}{2.132969in}}%
\pgfpathlineto{\pgfqpoint{3.044400in}{2.133441in}}%
\pgfpathlineto{\pgfqpoint{3.052681in}{2.134108in}}%
\pgfpathlineto{\pgfqpoint{3.060949in}{2.134966in}}%
\pgfpathlineto{\pgfqpoint{3.047561in}{2.149252in}}%
\pgfpathlineto{\pgfqpoint{3.034170in}{2.163695in}}%
\pgfpathlineto{\pgfqpoint{3.020776in}{2.178295in}}%
\pgfpathlineto{\pgfqpoint{3.007379in}{2.193054in}}%
\pgfpathlineto{\pgfqpoint{2.999076in}{2.192581in}}%
\pgfpathlineto{\pgfqpoint{2.990761in}{2.192305in}}%
\pgfpathlineto{\pgfqpoint{2.982432in}{2.192227in}}%
\pgfpathlineto{\pgfqpoint{2.974090in}{2.192352in}}%
\pgfpathclose%
\pgfusepath{fill}%
\end{pgfscope}%
\begin{pgfscope}%
\pgfpathrectangle{\pgfqpoint{1.254980in}{0.150000in}}{\pgfqpoint{5.490039in}{5.490039in}}%
\pgfusepath{clip}%
\pgfsetbuttcap%
\pgfsetroundjoin%
\definecolor{currentfill}{rgb}{0.212395,0.359683,0.551710}%
\pgfsetfillcolor{currentfill}%
\pgfsetfillopacity{0.700000}%
\pgfsetlinewidth{0.000000pt}%
\definecolor{currentstroke}{rgb}{0.000000,0.000000,0.000000}%
\pgfsetstrokecolor{currentstroke}%
\pgfsetdash{}{0pt}%
\pgfpathmoveto{\pgfqpoint{2.812604in}{2.386856in}}%
\pgfpathlineto{\pgfqpoint{2.826085in}{2.369733in}}%
\pgfpathlineto{\pgfqpoint{2.839562in}{2.352778in}}%
\pgfpathlineto{\pgfqpoint{2.853034in}{2.335993in}}%
\pgfpathlineto{\pgfqpoint{2.866501in}{2.319375in}}%
\pgfpathlineto{\pgfqpoint{2.874918in}{2.318445in}}%
\pgfpathlineto{\pgfqpoint{2.883321in}{2.317726in}}%
\pgfpathlineto{\pgfqpoint{2.891711in}{2.317216in}}%
\pgfpathlineto{\pgfqpoint{2.900086in}{2.316910in}}%
\pgfpathlineto{\pgfqpoint{2.886657in}{2.333130in}}%
\pgfpathlineto{\pgfqpoint{2.873224in}{2.349516in}}%
\pgfpathlineto{\pgfqpoint{2.859786in}{2.366070in}}%
\pgfpathlineto{\pgfqpoint{2.846344in}{2.382793in}}%
\pgfpathlineto{\pgfqpoint{2.837930in}{2.383491in}}%
\pgfpathlineto{\pgfqpoint{2.829503in}{2.384399in}}%
\pgfpathlineto{\pgfqpoint{2.821061in}{2.385519in}}%
\pgfpathlineto{\pgfqpoint{2.812604in}{2.386856in}}%
\pgfpathclose%
\pgfusepath{fill}%
\end{pgfscope}%
\begin{pgfscope}%
\pgfpathrectangle{\pgfqpoint{1.254980in}{0.150000in}}{\pgfqpoint{5.490039in}{5.490039in}}%
\pgfusepath{clip}%
\pgfsetbuttcap%
\pgfsetroundjoin%
\definecolor{currentfill}{rgb}{0.188923,0.410910,0.556326}%
\pgfsetfillcolor{currentfill}%
\pgfsetfillopacity{0.700000}%
\pgfsetlinewidth{0.000000pt}%
\definecolor{currentstroke}{rgb}{0.000000,0.000000,0.000000}%
\pgfsetstrokecolor{currentstroke}%
\pgfsetdash{}{0pt}%
\pgfpathmoveto{\pgfqpoint{5.429416in}{2.448922in}}%
\pgfpathlineto{\pgfqpoint{5.443421in}{2.454995in}}%
\pgfpathlineto{\pgfqpoint{5.457440in}{2.461180in}}%
\pgfpathlineto{\pgfqpoint{5.471475in}{2.467478in}}%
\pgfpathlineto{\pgfqpoint{5.485524in}{2.473888in}}%
\pgfpathlineto{\pgfqpoint{5.492831in}{2.484387in}}%
\pgfpathlineto{\pgfqpoint{5.500131in}{2.494815in}}%
\pgfpathlineto{\pgfqpoint{5.507425in}{2.505172in}}%
\pgfpathlineto{\pgfqpoint{5.514712in}{2.515458in}}%
\pgfpathlineto{\pgfqpoint{5.500668in}{2.509027in}}%
\pgfpathlineto{\pgfqpoint{5.486639in}{2.502709in}}%
\pgfpathlineto{\pgfqpoint{5.472625in}{2.496503in}}%
\pgfpathlineto{\pgfqpoint{5.458626in}{2.490410in}}%
\pgfpathlineto{\pgfqpoint{5.451333in}{2.480138in}}%
\pgfpathlineto{\pgfqpoint{5.444034in}{2.469800in}}%
\pgfpathlineto{\pgfqpoint{5.436728in}{2.459394in}}%
\pgfpathlineto{\pgfqpoint{5.429416in}{2.448922in}}%
\pgfpathclose%
\pgfusepath{fill}%
\end{pgfscope}%
\begin{pgfscope}%
\pgfpathrectangle{\pgfqpoint{1.254980in}{0.150000in}}{\pgfqpoint{5.490039in}{5.490039in}}%
\pgfusepath{clip}%
\pgfsetbuttcap%
\pgfsetroundjoin%
\definecolor{currentfill}{rgb}{0.269944,0.014625,0.341379}%
\pgfsetfillcolor{currentfill}%
\pgfsetfillopacity{0.700000}%
\pgfsetlinewidth{0.000000pt}%
\definecolor{currentstroke}{rgb}{0.000000,0.000000,0.000000}%
\pgfsetstrokecolor{currentstroke}%
\pgfsetdash{}{0pt}%
\pgfpathmoveto{\pgfqpoint{4.043228in}{1.639058in}}%
\pgfpathlineto{\pgfqpoint{4.056654in}{1.634932in}}%
\pgfpathlineto{\pgfqpoint{4.070087in}{1.630928in}}%
\pgfpathlineto{\pgfqpoint{4.083526in}{1.627043in}}%
\pgfpathlineto{\pgfqpoint{4.096971in}{1.623279in}}%
\pgfpathlineto{\pgfqpoint{4.104721in}{1.632065in}}%
\pgfpathlineto{\pgfqpoint{4.112466in}{1.640921in}}%
\pgfpathlineto{\pgfqpoint{4.120205in}{1.649845in}}%
\pgfpathlineto{\pgfqpoint{4.127939in}{1.658833in}}%
\pgfpathlineto{\pgfqpoint{4.114506in}{1.662305in}}%
\pgfpathlineto{\pgfqpoint{4.101079in}{1.665897in}}%
\pgfpathlineto{\pgfqpoint{4.087658in}{1.669609in}}%
\pgfpathlineto{\pgfqpoint{4.074244in}{1.673441in}}%
\pgfpathlineto{\pgfqpoint{4.066499in}{1.664739in}}%
\pgfpathlineto{\pgfqpoint{4.058747in}{1.656106in}}%
\pgfpathlineto{\pgfqpoint{4.050991in}{1.647545in}}%
\pgfpathlineto{\pgfqpoint{4.043228in}{1.639058in}}%
\pgfpathclose%
\pgfusepath{fill}%
\end{pgfscope}%
\begin{pgfscope}%
\pgfpathrectangle{\pgfqpoint{1.254980in}{0.150000in}}{\pgfqpoint{5.490039in}{5.490039in}}%
\pgfusepath{clip}%
\pgfsetbuttcap%
\pgfsetroundjoin%
\definecolor{currentfill}{rgb}{0.276022,0.044167,0.370164}%
\pgfsetfillcolor{currentfill}%
\pgfsetfillopacity{0.700000}%
\pgfsetlinewidth{0.000000pt}%
\definecolor{currentstroke}{rgb}{0.000000,0.000000,0.000000}%
\pgfsetstrokecolor{currentstroke}%
\pgfsetdash{}{0pt}%
\pgfpathmoveto{\pgfqpoint{3.627717in}{1.703397in}}%
\pgfpathlineto{\pgfqpoint{3.641088in}{1.695291in}}%
\pgfpathlineto{\pgfqpoint{3.654461in}{1.687315in}}%
\pgfpathlineto{\pgfqpoint{3.667837in}{1.679468in}}%
\pgfpathlineto{\pgfqpoint{3.681217in}{1.671751in}}%
\pgfpathlineto{\pgfqpoint{3.689141in}{1.677417in}}%
\pgfpathlineto{\pgfqpoint{3.697057in}{1.683209in}}%
\pgfpathlineto{\pgfqpoint{3.704966in}{1.689123in}}%
\pgfpathlineto{\pgfqpoint{3.712867in}{1.695156in}}%
\pgfpathlineto{\pgfqpoint{3.699508in}{1.702531in}}%
\pgfpathlineto{\pgfqpoint{3.686152in}{1.710034in}}%
\pgfpathlineto{\pgfqpoint{3.672799in}{1.717666in}}%
\pgfpathlineto{\pgfqpoint{3.659449in}{1.725429in}}%
\pgfpathlineto{\pgfqpoint{3.651528in}{1.719732in}}%
\pgfpathlineto{\pgfqpoint{3.643599in}{1.714160in}}%
\pgfpathlineto{\pgfqpoint{3.635662in}{1.708713in}}%
\pgfpathlineto{\pgfqpoint{3.627717in}{1.703397in}}%
\pgfpathclose%
\pgfusepath{fill}%
\end{pgfscope}%
\begin{pgfscope}%
\pgfpathrectangle{\pgfqpoint{1.254980in}{0.150000in}}{\pgfqpoint{5.490039in}{5.490039in}}%
\pgfusepath{clip}%
\pgfsetbuttcap%
\pgfsetroundjoin%
\definecolor{currentfill}{rgb}{0.281412,0.155834,0.469201}%
\pgfsetfillcolor{currentfill}%
\pgfsetfillopacity{0.700000}%
\pgfsetlinewidth{0.000000pt}%
\definecolor{currentstroke}{rgb}{0.000000,0.000000,0.000000}%
\pgfsetstrokecolor{currentstroke}%
\pgfsetdash{}{0pt}%
\pgfpathmoveto{\pgfqpoint{4.689928in}{1.879293in}}%
\pgfpathlineto{\pgfqpoint{4.703565in}{1.880632in}}%
\pgfpathlineto{\pgfqpoint{4.717213in}{1.882086in}}%
\pgfpathlineto{\pgfqpoint{4.730871in}{1.883655in}}%
\pgfpathlineto{\pgfqpoint{4.744540in}{1.885337in}}%
\pgfpathlineto{\pgfqpoint{4.752095in}{1.896825in}}%
\pgfpathlineto{\pgfqpoint{4.759645in}{1.908300in}}%
\pgfpathlineto{\pgfqpoint{4.767190in}{1.919761in}}%
\pgfpathlineto{\pgfqpoint{4.774731in}{1.931206in}}%
\pgfpathlineto{\pgfqpoint{4.761066in}{1.929340in}}%
\pgfpathlineto{\pgfqpoint{4.747412in}{1.927589in}}%
\pgfpathlineto{\pgfqpoint{4.733769in}{1.925952in}}%
\pgfpathlineto{\pgfqpoint{4.720136in}{1.924430in}}%
\pgfpathlineto{\pgfqpoint{4.712591in}{1.913162in}}%
\pgfpathlineto{\pgfqpoint{4.705042in}{1.901882in}}%
\pgfpathlineto{\pgfqpoint{4.697487in}{1.890592in}}%
\pgfpathlineto{\pgfqpoint{4.689928in}{1.879293in}}%
\pgfpathclose%
\pgfusepath{fill}%
\end{pgfscope}%
\begin{pgfscope}%
\pgfpathrectangle{\pgfqpoint{1.254980in}{0.150000in}}{\pgfqpoint{5.490039in}{5.490039in}}%
\pgfusepath{clip}%
\pgfsetbuttcap%
\pgfsetroundjoin%
\definecolor{currentfill}{rgb}{0.258965,0.251537,0.524736}%
\pgfsetfillcolor{currentfill}%
\pgfsetfillopacity{0.700000}%
\pgfsetlinewidth{0.000000pt}%
\definecolor{currentstroke}{rgb}{0.000000,0.000000,0.000000}%
\pgfsetstrokecolor{currentstroke}%
\pgfsetdash{}{0pt}%
\pgfpathmoveto{\pgfqpoint{3.027800in}{2.132694in}}%
\pgfpathlineto{\pgfqpoint{3.041220in}{2.118171in}}%
\pgfpathlineto{\pgfqpoint{3.054638in}{2.103804in}}%
\pgfpathlineto{\pgfqpoint{3.068053in}{2.089591in}}%
\pgfpathlineto{\pgfqpoint{3.081466in}{2.075532in}}%
\pgfpathlineto{\pgfqpoint{3.089738in}{2.076205in}}%
\pgfpathlineto{\pgfqpoint{3.097997in}{2.077070in}}%
\pgfpathlineto{\pgfqpoint{3.106244in}{2.078125in}}%
\pgfpathlineto{\pgfqpoint{3.114479in}{2.079367in}}%
\pgfpathlineto{\pgfqpoint{3.101100in}{2.093036in}}%
\pgfpathlineto{\pgfqpoint{3.087719in}{2.106858in}}%
\pgfpathlineto{\pgfqpoint{3.074335in}{2.120835in}}%
\pgfpathlineto{\pgfqpoint{3.060949in}{2.134966in}}%
\pgfpathlineto{\pgfqpoint{3.052681in}{2.134108in}}%
\pgfpathlineto{\pgfqpoint{3.044400in}{2.133441in}}%
\pgfpathlineto{\pgfqpoint{3.036106in}{2.132969in}}%
\pgfpathlineto{\pgfqpoint{3.027800in}{2.132694in}}%
\pgfpathclose%
\pgfusepath{fill}%
\end{pgfscope}%
\begin{pgfscope}%
\pgfpathrectangle{\pgfqpoint{1.254980in}{0.150000in}}{\pgfqpoint{5.490039in}{5.490039in}}%
\pgfusepath{clip}%
\pgfsetbuttcap%
\pgfsetroundjoin%
\definecolor{currentfill}{rgb}{0.278791,0.062145,0.386592}%
\pgfsetfillcolor{currentfill}%
\pgfsetfillopacity{0.700000}%
\pgfsetlinewidth{0.000000pt}%
\definecolor{currentstroke}{rgb}{0.000000,0.000000,0.000000}%
\pgfsetstrokecolor{currentstroke}%
\pgfsetdash{}{0pt}%
\pgfpathmoveto{\pgfqpoint{4.351130in}{1.706894in}}%
\pgfpathlineto{\pgfqpoint{4.364641in}{1.705498in}}%
\pgfpathlineto{\pgfqpoint{4.378160in}{1.704219in}}%
\pgfpathlineto{\pgfqpoint{4.391687in}{1.703056in}}%
\pgfpathlineto{\pgfqpoint{4.405223in}{1.702010in}}%
\pgfpathlineto{\pgfqpoint{4.412874in}{1.712493in}}%
\pgfpathlineto{\pgfqpoint{4.420520in}{1.723005in}}%
\pgfpathlineto{\pgfqpoint{4.428162in}{1.733544in}}%
\pgfpathlineto{\pgfqpoint{4.435799in}{1.744108in}}%
\pgfpathlineto{\pgfqpoint{4.422270in}{1.744909in}}%
\pgfpathlineto{\pgfqpoint{4.408750in}{1.745826in}}%
\pgfpathlineto{\pgfqpoint{4.395239in}{1.746860in}}%
\pgfpathlineto{\pgfqpoint{4.381736in}{1.748010in}}%
\pgfpathlineto{\pgfqpoint{4.374092in}{1.737686in}}%
\pgfpathlineto{\pgfqpoint{4.366443in}{1.727391in}}%
\pgfpathlineto{\pgfqpoint{4.358789in}{1.717126in}}%
\pgfpathlineto{\pgfqpoint{4.351130in}{1.706894in}}%
\pgfpathclose%
\pgfusepath{fill}%
\end{pgfscope}%
\begin{pgfscope}%
\pgfpathrectangle{\pgfqpoint{1.254980in}{0.150000in}}{\pgfqpoint{5.490039in}{5.490039in}}%
\pgfusepath{clip}%
\pgfsetbuttcap%
\pgfsetroundjoin%
\definecolor{currentfill}{rgb}{0.233603,0.313828,0.543914}%
\pgfsetfillcolor{currentfill}%
\pgfsetfillopacity{0.700000}%
\pgfsetlinewidth{0.000000pt}%
\definecolor{currentstroke}{rgb}{0.000000,0.000000,0.000000}%
\pgfsetstrokecolor{currentstroke}%
\pgfsetdash{}{0pt}%
\pgfpathmoveto{\pgfqpoint{5.144444in}{2.211220in}}%
\pgfpathlineto{\pgfqpoint{5.158297in}{2.215690in}}%
\pgfpathlineto{\pgfqpoint{5.172163in}{2.220273in}}%
\pgfpathlineto{\pgfqpoint{5.186043in}{2.224970in}}%
\pgfpathlineto{\pgfqpoint{5.199937in}{2.229779in}}%
\pgfpathlineto{\pgfqpoint{5.207353in}{2.241121in}}%
\pgfpathlineto{\pgfqpoint{5.214763in}{2.252408in}}%
\pgfpathlineto{\pgfqpoint{5.222167in}{2.263639in}}%
\pgfpathlineto{\pgfqpoint{5.229566in}{2.274813in}}%
\pgfpathlineto{\pgfqpoint{5.215676in}{2.269917in}}%
\pgfpathlineto{\pgfqpoint{5.201800in}{2.265134in}}%
\pgfpathlineto{\pgfqpoint{5.187937in}{2.260465in}}%
\pgfpathlineto{\pgfqpoint{5.174088in}{2.255908in}}%
\pgfpathlineto{\pgfqpoint{5.166685in}{2.244814in}}%
\pgfpathlineto{\pgfqpoint{5.159277in}{2.233668in}}%
\pgfpathlineto{\pgfqpoint{5.151863in}{2.222470in}}%
\pgfpathlineto{\pgfqpoint{5.144444in}{2.211220in}}%
\pgfpathclose%
\pgfusepath{fill}%
\end{pgfscope}%
\begin{pgfscope}%
\pgfpathrectangle{\pgfqpoint{1.254980in}{0.150000in}}{\pgfqpoint{5.490039in}{5.490039in}}%
\pgfusepath{clip}%
\pgfsetbuttcap%
\pgfsetroundjoin%
\definecolor{currentfill}{rgb}{0.199430,0.387607,0.554642}%
\pgfsetfillcolor{currentfill}%
\pgfsetfillopacity{0.700000}%
\pgfsetlinewidth{0.000000pt}%
\definecolor{currentstroke}{rgb}{0.000000,0.000000,0.000000}%
\pgfsetstrokecolor{currentstroke}%
\pgfsetdash{}{0pt}%
\pgfpathmoveto{\pgfqpoint{2.758625in}{2.457065in}}%
\pgfpathlineto{\pgfqpoint{2.772128in}{2.439254in}}%
\pgfpathlineto{\pgfqpoint{2.785625in}{2.421616in}}%
\pgfpathlineto{\pgfqpoint{2.799117in}{2.404150in}}%
\pgfpathlineto{\pgfqpoint{2.812604in}{2.386856in}}%
\pgfpathlineto{\pgfqpoint{2.821061in}{2.385519in}}%
\pgfpathlineto{\pgfqpoint{2.829503in}{2.384399in}}%
\pgfpathlineto{\pgfqpoint{2.837930in}{2.383491in}}%
\pgfpathlineto{\pgfqpoint{2.846344in}{2.382793in}}%
\pgfpathlineto{\pgfqpoint{2.832897in}{2.399686in}}%
\pgfpathlineto{\pgfqpoint{2.819445in}{2.416750in}}%
\pgfpathlineto{\pgfqpoint{2.805988in}{2.433986in}}%
\pgfpathlineto{\pgfqpoint{2.792525in}{2.451394in}}%
\pgfpathlineto{\pgfqpoint{2.784072in}{2.452487in}}%
\pgfpathlineto{\pgfqpoint{2.775605in}{2.453794in}}%
\pgfpathlineto{\pgfqpoint{2.767122in}{2.455319in}}%
\pgfpathlineto{\pgfqpoint{2.758625in}{2.457065in}}%
\pgfpathclose%
\pgfusepath{fill}%
\end{pgfscope}%
\begin{pgfscope}%
\pgfpathrectangle{\pgfqpoint{1.254980in}{0.150000in}}{\pgfqpoint{5.490039in}{5.490039in}}%
\pgfusepath{clip}%
\pgfsetbuttcap%
\pgfsetroundjoin%
\definecolor{currentfill}{rgb}{0.282327,0.094955,0.417331}%
\pgfsetfillcolor{currentfill}%
\pgfsetfillopacity{0.700000}%
\pgfsetlinewidth{0.000000pt}%
\definecolor{currentstroke}{rgb}{0.000000,0.000000,0.000000}%
\pgfsetstrokecolor{currentstroke}%
\pgfsetdash{}{0pt}%
\pgfpathmoveto{\pgfqpoint{3.435310in}{1.795399in}}%
\pgfpathlineto{\pgfqpoint{3.448680in}{1.785333in}}%
\pgfpathlineto{\pgfqpoint{3.462051in}{1.775403in}}%
\pgfpathlineto{\pgfqpoint{3.475424in}{1.765610in}}%
\pgfpathlineto{\pgfqpoint{3.488798in}{1.755951in}}%
\pgfpathlineto{\pgfqpoint{3.496823in}{1.759986in}}%
\pgfpathlineto{\pgfqpoint{3.504838in}{1.764171in}}%
\pgfpathlineto{\pgfqpoint{3.512845in}{1.768502in}}%
\pgfpathlineto{\pgfqpoint{3.520843in}{1.772977in}}%
\pgfpathlineto{\pgfqpoint{3.507493in}{1.782274in}}%
\pgfpathlineto{\pgfqpoint{3.494145in}{1.791705in}}%
\pgfpathlineto{\pgfqpoint{3.480799in}{1.801272in}}%
\pgfpathlineto{\pgfqpoint{3.467454in}{1.810974in}}%
\pgfpathlineto{\pgfqpoint{3.459432in}{1.806856in}}%
\pgfpathlineto{\pgfqpoint{3.451400in}{1.802885in}}%
\pgfpathlineto{\pgfqpoint{3.443360in}{1.799065in}}%
\pgfpathlineto{\pgfqpoint{3.435310in}{1.795399in}}%
\pgfpathclose%
\pgfusepath{fill}%
\end{pgfscope}%
\begin{pgfscope}%
\pgfpathrectangle{\pgfqpoint{1.254980in}{0.150000in}}{\pgfqpoint{5.490039in}{5.490039in}}%
\pgfusepath{clip}%
\pgfsetbuttcap%
\pgfsetroundjoin%
\definecolor{currentfill}{rgb}{0.121148,0.592739,0.544641}%
\pgfsetfillcolor{currentfill}%
\pgfsetfillopacity{0.700000}%
\pgfsetlinewidth{0.000000pt}%
\definecolor{currentstroke}{rgb}{0.000000,0.000000,0.000000}%
\pgfsetstrokecolor{currentstroke}%
\pgfsetdash{}{0pt}%
\pgfpathmoveto{\pgfqpoint{2.412768in}{3.013615in}}%
\pgfpathlineto{\pgfqpoint{2.426447in}{2.990893in}}%
\pgfpathlineto{\pgfqpoint{2.440115in}{2.968378in}}%
\pgfpathlineto{\pgfqpoint{2.453774in}{2.946069in}}%
\pgfpathlineto{\pgfqpoint{2.467422in}{2.923965in}}%
\pgfpathlineto{\pgfqpoint{2.476121in}{2.920680in}}%
\pgfpathlineto{\pgfqpoint{2.484802in}{2.917633in}}%
\pgfpathlineto{\pgfqpoint{2.493467in}{2.914820in}}%
\pgfpathlineto{\pgfqpoint{2.502114in}{2.912238in}}%
\pgfpathlineto{\pgfqpoint{2.488512in}{2.933939in}}%
\pgfpathlineto{\pgfqpoint{2.474901in}{2.955844in}}%
\pgfpathlineto{\pgfqpoint{2.461280in}{2.977953in}}%
\pgfpathlineto{\pgfqpoint{2.447650in}{3.000270in}}%
\pgfpathlineto{\pgfqpoint{2.438956in}{3.003249in}}%
\pgfpathlineto{\pgfqpoint{2.430244in}{3.006464in}}%
\pgfpathlineto{\pgfqpoint{2.421515in}{3.009918in}}%
\pgfpathlineto{\pgfqpoint{2.412768in}{3.013615in}}%
\pgfpathclose%
\pgfusepath{fill}%
\end{pgfscope}%
\begin{pgfscope}%
\pgfpathrectangle{\pgfqpoint{1.254980in}{0.150000in}}{\pgfqpoint{5.490039in}{5.490039in}}%
\pgfusepath{clip}%
\pgfsetbuttcap%
\pgfsetroundjoin%
\definecolor{currentfill}{rgb}{0.266580,0.228262,0.514349}%
\pgfsetfillcolor{currentfill}%
\pgfsetfillopacity{0.700000}%
\pgfsetlinewidth{0.000000pt}%
\definecolor{currentstroke}{rgb}{0.000000,0.000000,0.000000}%
\pgfsetstrokecolor{currentstroke}%
\pgfsetdash{}{0pt}%
\pgfpathmoveto{\pgfqpoint{3.081466in}{2.075532in}}%
\pgfpathlineto{\pgfqpoint{3.094877in}{2.061626in}}%
\pgfpathlineto{\pgfqpoint{3.108285in}{2.047872in}}%
\pgfpathlineto{\pgfqpoint{3.121692in}{2.034270in}}%
\pgfpathlineto{\pgfqpoint{3.135097in}{2.020818in}}%
\pgfpathlineto{\pgfqpoint{3.143335in}{2.021885in}}%
\pgfpathlineto{\pgfqpoint{3.151562in}{2.023142in}}%
\pgfpathlineto{\pgfqpoint{3.159776in}{2.024583in}}%
\pgfpathlineto{\pgfqpoint{3.167979in}{2.026206in}}%
\pgfpathlineto{\pgfqpoint{3.154607in}{2.039271in}}%
\pgfpathlineto{\pgfqpoint{3.141233in}{2.052485in}}%
\pgfpathlineto{\pgfqpoint{3.127857in}{2.065850in}}%
\pgfpathlineto{\pgfqpoint{3.114479in}{2.079367in}}%
\pgfpathlineto{\pgfqpoint{3.106244in}{2.078125in}}%
\pgfpathlineto{\pgfqpoint{3.097997in}{2.077070in}}%
\pgfpathlineto{\pgfqpoint{3.089738in}{2.076205in}}%
\pgfpathlineto{\pgfqpoint{3.081466in}{2.075532in}}%
\pgfpathclose%
\pgfusepath{fill}%
\end{pgfscope}%
\begin{pgfscope}%
\pgfpathrectangle{\pgfqpoint{1.254980in}{0.150000in}}{\pgfqpoint{5.490039in}{5.490039in}}%
\pgfusepath{clip}%
\pgfsetbuttcap%
\pgfsetroundjoin%
\definecolor{currentfill}{rgb}{0.277134,0.185228,0.489898}%
\pgfsetfillcolor{currentfill}%
\pgfsetfillopacity{0.700000}%
\pgfsetlinewidth{0.000000pt}%
\definecolor{currentstroke}{rgb}{0.000000,0.000000,0.000000}%
\pgfsetstrokecolor{currentstroke}%
\pgfsetdash{}{0pt}%
\pgfpathmoveto{\pgfqpoint{4.774731in}{1.931206in}}%
\pgfpathlineto{\pgfqpoint{4.788407in}{1.933185in}}%
\pgfpathlineto{\pgfqpoint{4.802094in}{1.935278in}}%
\pgfpathlineto{\pgfqpoint{4.815793in}{1.937485in}}%
\pgfpathlineto{\pgfqpoint{4.829502in}{1.939806in}}%
\pgfpathlineto{\pgfqpoint{4.837034in}{1.951407in}}%
\pgfpathlineto{\pgfqpoint{4.844562in}{1.962985in}}%
\pgfpathlineto{\pgfqpoint{4.852084in}{1.974540in}}%
\pgfpathlineto{\pgfqpoint{4.859601in}{1.986071in}}%
\pgfpathlineto{\pgfqpoint{4.845895in}{1.983583in}}%
\pgfpathlineto{\pgfqpoint{4.832201in}{1.981208in}}%
\pgfpathlineto{\pgfqpoint{4.818518in}{1.978948in}}%
\pgfpathlineto{\pgfqpoint{4.804846in}{1.976802in}}%
\pgfpathlineto{\pgfqpoint{4.797324in}{1.965432in}}%
\pgfpathlineto{\pgfqpoint{4.789798in}{1.954043in}}%
\pgfpathlineto{\pgfqpoint{4.782267in}{1.942633in}}%
\pgfpathlineto{\pgfqpoint{4.774731in}{1.931206in}}%
\pgfpathclose%
\pgfusepath{fill}%
\end{pgfscope}%
\begin{pgfscope}%
\pgfpathrectangle{\pgfqpoint{1.254980in}{0.150000in}}{\pgfqpoint{5.490039in}{5.490039in}}%
\pgfusepath{clip}%
\pgfsetbuttcap%
\pgfsetroundjoin%
\definecolor{currentfill}{rgb}{0.276022,0.044167,0.370164}%
\pgfsetfillcolor{currentfill}%
\pgfsetfillopacity{0.700000}%
\pgfsetlinewidth{0.000000pt}%
\definecolor{currentstroke}{rgb}{0.000000,0.000000,0.000000}%
\pgfsetstrokecolor{currentstroke}%
\pgfsetdash{}{0pt}%
\pgfpathmoveto{\pgfqpoint{4.266451in}{1.674129in}}%
\pgfpathlineto{\pgfqpoint{4.279938in}{1.672003in}}%
\pgfpathlineto{\pgfqpoint{4.293433in}{1.669994in}}%
\pgfpathlineto{\pgfqpoint{4.306936in}{1.668103in}}%
\pgfpathlineto{\pgfqpoint{4.320446in}{1.666329in}}%
\pgfpathlineto{\pgfqpoint{4.328125in}{1.676412in}}%
\pgfpathlineto{\pgfqpoint{4.335798in}{1.686535in}}%
\pgfpathlineto{\pgfqpoint{4.343467in}{1.696696in}}%
\pgfpathlineto{\pgfqpoint{4.351130in}{1.706894in}}%
\pgfpathlineto{\pgfqpoint{4.337628in}{1.708407in}}%
\pgfpathlineto{\pgfqpoint{4.324134in}{1.710037in}}%
\pgfpathlineto{\pgfqpoint{4.310647in}{1.711784in}}%
\pgfpathlineto{\pgfqpoint{4.297169in}{1.713649in}}%
\pgfpathlineto{\pgfqpoint{4.289497in}{1.703707in}}%
\pgfpathlineto{\pgfqpoint{4.281820in}{1.693804in}}%
\pgfpathlineto{\pgfqpoint{4.274138in}{1.683944in}}%
\pgfpathlineto{\pgfqpoint{4.266451in}{1.674129in}}%
\pgfpathclose%
\pgfusepath{fill}%
\end{pgfscope}%
\begin{pgfscope}%
\pgfpathrectangle{\pgfqpoint{1.254980in}{0.150000in}}{\pgfqpoint{5.490039in}{5.490039in}}%
\pgfusepath{clip}%
\pgfsetbuttcap%
\pgfsetroundjoin%
\definecolor{currentfill}{rgb}{0.185556,0.418570,0.556753}%
\pgfsetfillcolor{currentfill}%
\pgfsetfillopacity{0.700000}%
\pgfsetlinewidth{0.000000pt}%
\definecolor{currentstroke}{rgb}{0.000000,0.000000,0.000000}%
\pgfsetstrokecolor{currentstroke}%
\pgfsetdash{}{0pt}%
\pgfpathmoveto{\pgfqpoint{2.704554in}{2.530069in}}%
\pgfpathlineto{\pgfqpoint{2.718081in}{2.511552in}}%
\pgfpathlineto{\pgfqpoint{2.731602in}{2.493214in}}%
\pgfpathlineto{\pgfqpoint{2.745116in}{2.475052in}}%
\pgfpathlineto{\pgfqpoint{2.758625in}{2.457065in}}%
\pgfpathlineto{\pgfqpoint{2.767122in}{2.455319in}}%
\pgfpathlineto{\pgfqpoint{2.775605in}{2.453794in}}%
\pgfpathlineto{\pgfqpoint{2.784072in}{2.452487in}}%
\pgfpathlineto{\pgfqpoint{2.792525in}{2.451394in}}%
\pgfpathlineto{\pgfqpoint{2.779058in}{2.468977in}}%
\pgfpathlineto{\pgfqpoint{2.765584in}{2.486734in}}%
\pgfpathlineto{\pgfqpoint{2.752105in}{2.504668in}}%
\pgfpathlineto{\pgfqpoint{2.738620in}{2.522778in}}%
\pgfpathlineto{\pgfqpoint{2.730127in}{2.524268in}}%
\pgfpathlineto{\pgfqpoint{2.721618in}{2.525978in}}%
\pgfpathlineto{\pgfqpoint{2.713094in}{2.527910in}}%
\pgfpathlineto{\pgfqpoint{2.704554in}{2.530069in}}%
\pgfpathclose%
\pgfusepath{fill}%
\end{pgfscope}%
\begin{pgfscope}%
\pgfpathrectangle{\pgfqpoint{1.254980in}{0.150000in}}{\pgfqpoint{5.490039in}{5.490039in}}%
\pgfusepath{clip}%
\pgfsetbuttcap%
\pgfsetroundjoin%
\definecolor{currentfill}{rgb}{0.273006,0.204520,0.501721}%
\pgfsetfillcolor{currentfill}%
\pgfsetfillopacity{0.700000}%
\pgfsetlinewidth{0.000000pt}%
\definecolor{currentstroke}{rgb}{0.000000,0.000000,0.000000}%
\pgfsetstrokecolor{currentstroke}%
\pgfsetdash{}{0pt}%
\pgfpathmoveto{\pgfqpoint{3.135097in}{2.020818in}}%
\pgfpathlineto{\pgfqpoint{3.148500in}{2.007515in}}%
\pgfpathlineto{\pgfqpoint{3.161902in}{1.994362in}}%
\pgfpathlineto{\pgfqpoint{3.175302in}{1.981358in}}%
\pgfpathlineto{\pgfqpoint{3.188701in}{1.968501in}}%
\pgfpathlineto{\pgfqpoint{3.196907in}{1.969962in}}%
\pgfpathlineto{\pgfqpoint{3.205102in}{1.971607in}}%
\pgfpathlineto{\pgfqpoint{3.213285in}{1.973434in}}%
\pgfpathlineto{\pgfqpoint{3.221457in}{1.975437in}}%
\pgfpathlineto{\pgfqpoint{3.208089in}{1.987908in}}%
\pgfpathlineto{\pgfqpoint{3.194720in}{2.000526in}}%
\pgfpathlineto{\pgfqpoint{3.181350in}{2.013292in}}%
\pgfpathlineto{\pgfqpoint{3.167979in}{2.026206in}}%
\pgfpathlineto{\pgfqpoint{3.159776in}{2.024583in}}%
\pgfpathlineto{\pgfqpoint{3.151562in}{2.023142in}}%
\pgfpathlineto{\pgfqpoint{3.143335in}{2.021885in}}%
\pgfpathlineto{\pgfqpoint{3.135097in}{2.020818in}}%
\pgfpathclose%
\pgfusepath{fill}%
\end{pgfscope}%
\begin{pgfscope}%
\pgfpathrectangle{\pgfqpoint{1.254980in}{0.150000in}}{\pgfqpoint{5.490039in}{5.490039in}}%
\pgfusepath{clip}%
\pgfsetbuttcap%
\pgfsetroundjoin%
\definecolor{currentfill}{rgb}{0.270595,0.214069,0.507052}%
\pgfsetfillcolor{currentfill}%
\pgfsetfillopacity{0.700000}%
\pgfsetlinewidth{0.000000pt}%
\definecolor{currentstroke}{rgb}{0.000000,0.000000,0.000000}%
\pgfsetstrokecolor{currentstroke}%
\pgfsetdash{}{0pt}%
\pgfpathmoveto{\pgfqpoint{4.859601in}{1.986071in}}%
\pgfpathlineto{\pgfqpoint{4.873319in}{1.988672in}}%
\pgfpathlineto{\pgfqpoint{4.887048in}{1.991388in}}%
\pgfpathlineto{\pgfqpoint{4.900789in}{1.994217in}}%
\pgfpathlineto{\pgfqpoint{4.914542in}{1.997159in}}%
\pgfpathlineto{\pgfqpoint{4.922052in}{2.008821in}}%
\pgfpathlineto{\pgfqpoint{4.929556in}{2.020452in}}%
\pgfpathlineto{\pgfqpoint{4.937055in}{2.032051in}}%
\pgfpathlineto{\pgfqpoint{4.944550in}{2.043618in}}%
\pgfpathlineto{\pgfqpoint{4.930800in}{2.040524in}}%
\pgfpathlineto{\pgfqpoint{4.917063in}{2.037544in}}%
\pgfpathlineto{\pgfqpoint{4.903337in}{2.034677in}}%
\pgfpathlineto{\pgfqpoint{4.889623in}{2.031924in}}%
\pgfpathlineto{\pgfqpoint{4.882125in}{2.020503in}}%
\pgfpathlineto{\pgfqpoint{4.874622in}{2.009053in}}%
\pgfpathlineto{\pgfqpoint{4.867114in}{1.997575in}}%
\pgfpathlineto{\pgfqpoint{4.859601in}{1.986071in}}%
\pgfpathclose%
\pgfusepath{fill}%
\end{pgfscope}%
\begin{pgfscope}%
\pgfpathrectangle{\pgfqpoint{1.254980in}{0.150000in}}{\pgfqpoint{5.490039in}{5.490039in}}%
\pgfusepath{clip}%
\pgfsetbuttcap%
\pgfsetroundjoin%
\definecolor{currentfill}{rgb}{0.175841,0.441290,0.557685}%
\pgfsetfillcolor{currentfill}%
\pgfsetfillopacity{0.700000}%
\pgfsetlinewidth{0.000000pt}%
\definecolor{currentstroke}{rgb}{0.000000,0.000000,0.000000}%
\pgfsetstrokecolor{currentstroke}%
\pgfsetdash{}{0pt}%
\pgfpathmoveto{\pgfqpoint{5.514712in}{2.515458in}}%
\pgfpathlineto{\pgfqpoint{5.528772in}{2.522002in}}%
\pgfpathlineto{\pgfqpoint{5.542846in}{2.528658in}}%
\pgfpathlineto{\pgfqpoint{5.556936in}{2.535427in}}%
\pgfpathlineto{\pgfqpoint{5.571041in}{2.542309in}}%
\pgfpathlineto{\pgfqpoint{5.578316in}{2.552534in}}%
\pgfpathlineto{\pgfqpoint{5.585585in}{2.562684in}}%
\pgfpathlineto{\pgfqpoint{5.592846in}{2.572760in}}%
\pgfpathlineto{\pgfqpoint{5.600101in}{2.582761in}}%
\pgfpathlineto{\pgfqpoint{5.586001in}{2.575876in}}%
\pgfpathlineto{\pgfqpoint{5.571918in}{2.569104in}}%
\pgfpathlineto{\pgfqpoint{5.557849in}{2.562444in}}%
\pgfpathlineto{\pgfqpoint{5.543796in}{2.555896in}}%
\pgfpathlineto{\pgfqpoint{5.536535in}{2.545892in}}%
\pgfpathlineto{\pgfqpoint{5.529267in}{2.535818in}}%
\pgfpathlineto{\pgfqpoint{5.521993in}{2.525673in}}%
\pgfpathlineto{\pgfqpoint{5.514712in}{2.515458in}}%
\pgfpathclose%
\pgfusepath{fill}%
\end{pgfscope}%
\begin{pgfscope}%
\pgfpathrectangle{\pgfqpoint{1.254980in}{0.150000in}}{\pgfqpoint{5.490039in}{5.490039in}}%
\pgfusepath{clip}%
\pgfsetbuttcap%
\pgfsetroundjoin%
\definecolor{currentfill}{rgb}{0.220057,0.343307,0.549413}%
\pgfsetfillcolor{currentfill}%
\pgfsetfillopacity{0.700000}%
\pgfsetlinewidth{0.000000pt}%
\definecolor{currentstroke}{rgb}{0.000000,0.000000,0.000000}%
\pgfsetstrokecolor{currentstroke}%
\pgfsetdash{}{0pt}%
\pgfpathmoveto{\pgfqpoint{5.229566in}{2.274813in}}%
\pgfpathlineto{\pgfqpoint{5.243470in}{2.279822in}}%
\pgfpathlineto{\pgfqpoint{5.257387in}{2.284943in}}%
\pgfpathlineto{\pgfqpoint{5.271319in}{2.290178in}}%
\pgfpathlineto{\pgfqpoint{5.285264in}{2.295525in}}%
\pgfpathlineto{\pgfqpoint{5.292653in}{2.306720in}}%
\pgfpathlineto{\pgfqpoint{5.300037in}{2.317853in}}%
\pgfpathlineto{\pgfqpoint{5.307415in}{2.328924in}}%
\pgfpathlineto{\pgfqpoint{5.314786in}{2.339934in}}%
\pgfpathlineto{\pgfqpoint{5.300845in}{2.334516in}}%
\pgfpathlineto{\pgfqpoint{5.286917in}{2.329211in}}%
\pgfpathlineto{\pgfqpoint{5.273004in}{2.324019in}}%
\pgfpathlineto{\pgfqpoint{5.259104in}{2.318940in}}%
\pgfpathlineto{\pgfqpoint{5.251729in}{2.307995in}}%
\pgfpathlineto{\pgfqpoint{5.244347in}{2.296992in}}%
\pgfpathlineto{\pgfqpoint{5.236960in}{2.285931in}}%
\pgfpathlineto{\pgfqpoint{5.229566in}{2.274813in}}%
\pgfpathclose%
\pgfusepath{fill}%
\end{pgfscope}%
\begin{pgfscope}%
\pgfpathrectangle{\pgfqpoint{1.254980in}{0.150000in}}{\pgfqpoint{5.490039in}{5.490039in}}%
\pgfusepath{clip}%
\pgfsetbuttcap%
\pgfsetroundjoin%
\definecolor{currentfill}{rgb}{0.269944,0.014625,0.341379}%
\pgfsetfillcolor{currentfill}%
\pgfsetfillopacity{0.700000}%
\pgfsetlinewidth{0.000000pt}%
\definecolor{currentstroke}{rgb}{0.000000,0.000000,0.000000}%
\pgfsetstrokecolor{currentstroke}%
\pgfsetdash{}{0pt}%
\pgfpathmoveto{\pgfqpoint{3.819874in}{1.640749in}}%
\pgfpathlineto{\pgfqpoint{3.833268in}{1.634516in}}%
\pgfpathlineto{\pgfqpoint{3.846667in}{1.628407in}}%
\pgfpathlineto{\pgfqpoint{3.860069in}{1.622423in}}%
\pgfpathlineto{\pgfqpoint{3.873477in}{1.616563in}}%
\pgfpathlineto{\pgfqpoint{3.881318in}{1.623709in}}%
\pgfpathlineto{\pgfqpoint{3.889152in}{1.630958in}}%
\pgfpathlineto{\pgfqpoint{3.896980in}{1.638305in}}%
\pgfpathlineto{\pgfqpoint{3.904801in}{1.645750in}}%
\pgfpathlineto{\pgfqpoint{3.891410in}{1.651284in}}%
\pgfpathlineto{\pgfqpoint{3.878023in}{1.656943in}}%
\pgfpathlineto{\pgfqpoint{3.864641in}{1.662726in}}%
\pgfpathlineto{\pgfqpoint{3.851264in}{1.668634in}}%
\pgfpathlineto{\pgfqpoint{3.843426in}{1.661509in}}%
\pgfpathlineto{\pgfqpoint{3.835582in}{1.654485in}}%
\pgfpathlineto{\pgfqpoint{3.827732in}{1.647564in}}%
\pgfpathlineto{\pgfqpoint{3.819874in}{1.640749in}}%
\pgfpathclose%
\pgfusepath{fill}%
\end{pgfscope}%
\begin{pgfscope}%
\pgfpathrectangle{\pgfqpoint{1.254980in}{0.150000in}}{\pgfqpoint{5.490039in}{5.490039in}}%
\pgfusepath{clip}%
\pgfsetbuttcap%
\pgfsetroundjoin%
\definecolor{currentfill}{rgb}{0.174274,0.445044,0.557792}%
\pgfsetfillcolor{currentfill}%
\pgfsetfillopacity{0.700000}%
\pgfsetlinewidth{0.000000pt}%
\definecolor{currentstroke}{rgb}{0.000000,0.000000,0.000000}%
\pgfsetstrokecolor{currentstroke}%
\pgfsetdash{}{0pt}%
\pgfpathmoveto{\pgfqpoint{2.650381in}{2.605935in}}%
\pgfpathlineto{\pgfqpoint{2.663935in}{2.586696in}}%
\pgfpathlineto{\pgfqpoint{2.677481in}{2.567640in}}%
\pgfpathlineto{\pgfqpoint{2.691021in}{2.548764in}}%
\pgfpathlineto{\pgfqpoint{2.704554in}{2.530069in}}%
\pgfpathlineto{\pgfqpoint{2.713094in}{2.527910in}}%
\pgfpathlineto{\pgfqpoint{2.721618in}{2.525978in}}%
\pgfpathlineto{\pgfqpoint{2.730127in}{2.524268in}}%
\pgfpathlineto{\pgfqpoint{2.738620in}{2.522778in}}%
\pgfpathlineto{\pgfqpoint{2.725129in}{2.541067in}}%
\pgfpathlineto{\pgfqpoint{2.711632in}{2.559535in}}%
\pgfpathlineto{\pgfqpoint{2.698128in}{2.578183in}}%
\pgfpathlineto{\pgfqpoint{2.684618in}{2.597013in}}%
\pgfpathlineto{\pgfqpoint{2.676083in}{2.598904in}}%
\pgfpathlineto{\pgfqpoint{2.667532in}{2.601019in}}%
\pgfpathlineto{\pgfqpoint{2.658965in}{2.603362in}}%
\pgfpathlineto{\pgfqpoint{2.650381in}{2.605935in}}%
\pgfpathclose%
\pgfusepath{fill}%
\end{pgfscope}%
\begin{pgfscope}%
\pgfpathrectangle{\pgfqpoint{1.254980in}{0.150000in}}{\pgfqpoint{5.490039in}{5.490039in}}%
\pgfusepath{clip}%
\pgfsetbuttcap%
\pgfsetroundjoin%
\definecolor{currentfill}{rgb}{0.272594,0.025563,0.353093}%
\pgfsetfillcolor{currentfill}%
\pgfsetfillopacity{0.700000}%
\pgfsetlinewidth{0.000000pt}%
\definecolor{currentstroke}{rgb}{0.000000,0.000000,0.000000}%
\pgfsetstrokecolor{currentstroke}%
\pgfsetdash{}{0pt}%
\pgfpathmoveto{\pgfqpoint{4.181739in}{1.646143in}}%
\pgfpathlineto{\pgfqpoint{4.195207in}{1.643267in}}%
\pgfpathlineto{\pgfqpoint{4.208681in}{1.640511in}}%
\pgfpathlineto{\pgfqpoint{4.222163in}{1.637872in}}%
\pgfpathlineto{\pgfqpoint{4.235652in}{1.635351in}}%
\pgfpathlineto{\pgfqpoint{4.243359in}{1.644969in}}%
\pgfpathlineto{\pgfqpoint{4.251062in}{1.654639in}}%
\pgfpathlineto{\pgfqpoint{4.258759in}{1.664359in}}%
\pgfpathlineto{\pgfqpoint{4.266451in}{1.674129in}}%
\pgfpathlineto{\pgfqpoint{4.252971in}{1.676372in}}%
\pgfpathlineto{\pgfqpoint{4.239499in}{1.678734in}}%
\pgfpathlineto{\pgfqpoint{4.226035in}{1.681214in}}%
\pgfpathlineto{\pgfqpoint{4.212578in}{1.683812in}}%
\pgfpathlineto{\pgfqpoint{4.204876in}{1.674314in}}%
\pgfpathlineto{\pgfqpoint{4.197169in}{1.664868in}}%
\pgfpathlineto{\pgfqpoint{4.189457in}{1.655477in}}%
\pgfpathlineto{\pgfqpoint{4.181739in}{1.646143in}}%
\pgfpathclose%
\pgfusepath{fill}%
\end{pgfscope}%
\begin{pgfscope}%
\pgfpathrectangle{\pgfqpoint{1.254980in}{0.150000in}}{\pgfqpoint{5.490039in}{5.490039in}}%
\pgfusepath{clip}%
\pgfsetbuttcap%
\pgfsetroundjoin%
\definecolor{currentfill}{rgb}{0.269944,0.014625,0.341379}%
\pgfsetfillcolor{currentfill}%
\pgfsetfillopacity{0.700000}%
\pgfsetlinewidth{0.000000pt}%
\definecolor{currentstroke}{rgb}{0.000000,0.000000,0.000000}%
\pgfsetstrokecolor{currentstroke}%
\pgfsetdash{}{0pt}%
\pgfpathmoveto{\pgfqpoint{3.958417in}{1.624843in}}%
\pgfpathlineto{\pgfqpoint{3.971834in}{1.619923in}}%
\pgfpathlineto{\pgfqpoint{3.985257in}{1.615125in}}%
\pgfpathlineto{\pgfqpoint{3.998685in}{1.610449in}}%
\pgfpathlineto{\pgfqpoint{4.012119in}{1.605894in}}%
\pgfpathlineto{\pgfqpoint{4.019905in}{1.614062in}}%
\pgfpathlineto{\pgfqpoint{4.027685in}{1.622314in}}%
\pgfpathlineto{\pgfqpoint{4.035460in}{1.630646in}}%
\pgfpathlineto{\pgfqpoint{4.043228in}{1.639058in}}%
\pgfpathlineto{\pgfqpoint{4.029807in}{1.643304in}}%
\pgfpathlineto{\pgfqpoint{4.016393in}{1.647672in}}%
\pgfpathlineto{\pgfqpoint{4.002984in}{1.652161in}}%
\pgfpathlineto{\pgfqpoint{3.989581in}{1.656772in}}%
\pgfpathlineto{\pgfqpoint{3.981799in}{1.648663in}}%
\pgfpathlineto{\pgfqpoint{3.974011in}{1.640638in}}%
\pgfpathlineto{\pgfqpoint{3.966217in}{1.632697in}}%
\pgfpathlineto{\pgfqpoint{3.958417in}{1.624843in}}%
\pgfpathclose%
\pgfusepath{fill}%
\end{pgfscope}%
\begin{pgfscope}%
\pgfpathrectangle{\pgfqpoint{1.254980in}{0.150000in}}{\pgfqpoint{5.490039in}{5.490039in}}%
\pgfusepath{clip}%
\pgfsetbuttcap%
\pgfsetroundjoin%
\definecolor{currentfill}{rgb}{0.278012,0.180367,0.486697}%
\pgfsetfillcolor{currentfill}%
\pgfsetfillopacity{0.700000}%
\pgfsetlinewidth{0.000000pt}%
\definecolor{currentstroke}{rgb}{0.000000,0.000000,0.000000}%
\pgfsetstrokecolor{currentstroke}%
\pgfsetdash{}{0pt}%
\pgfpathmoveto{\pgfqpoint{3.188701in}{1.968501in}}%
\pgfpathlineto{\pgfqpoint{3.202099in}{1.955791in}}%
\pgfpathlineto{\pgfqpoint{3.215496in}{1.943227in}}%
\pgfpathlineto{\pgfqpoint{3.228892in}{1.930809in}}%
\pgfpathlineto{\pgfqpoint{3.242287in}{1.918536in}}%
\pgfpathlineto{\pgfqpoint{3.250462in}{1.920389in}}%
\pgfpathlineto{\pgfqpoint{3.258626in}{1.922422in}}%
\pgfpathlineto{\pgfqpoint{3.266779in}{1.924631in}}%
\pgfpathlineto{\pgfqpoint{3.274921in}{1.927013in}}%
\pgfpathlineto{\pgfqpoint{3.261556in}{1.938902in}}%
\pgfpathlineto{\pgfqpoint{3.248190in}{1.950935in}}%
\pgfpathlineto{\pgfqpoint{3.234824in}{1.963113in}}%
\pgfpathlineto{\pgfqpoint{3.221457in}{1.975437in}}%
\pgfpathlineto{\pgfqpoint{3.213285in}{1.973434in}}%
\pgfpathlineto{\pgfqpoint{3.205102in}{1.971607in}}%
\pgfpathlineto{\pgfqpoint{3.196907in}{1.969962in}}%
\pgfpathlineto{\pgfqpoint{3.188701in}{1.968501in}}%
\pgfpathclose%
\pgfusepath{fill}%
\end{pgfscope}%
\begin{pgfscope}%
\pgfpathrectangle{\pgfqpoint{1.254980in}{0.150000in}}{\pgfqpoint{5.490039in}{5.490039in}}%
\pgfusepath{clip}%
\pgfsetbuttcap%
\pgfsetroundjoin%
\definecolor{currentfill}{rgb}{0.280894,0.078907,0.402329}%
\pgfsetfillcolor{currentfill}%
\pgfsetfillopacity{0.700000}%
\pgfsetlinewidth{0.000000pt}%
\definecolor{currentstroke}{rgb}{0.000000,0.000000,0.000000}%
\pgfsetstrokecolor{currentstroke}%
\pgfsetdash{}{0pt}%
\pgfpathmoveto{\pgfqpoint{3.488798in}{1.755951in}}%
\pgfpathlineto{\pgfqpoint{3.502174in}{1.746426in}}%
\pgfpathlineto{\pgfqpoint{3.515551in}{1.737036in}}%
\pgfpathlineto{\pgfqpoint{3.528930in}{1.727779in}}%
\pgfpathlineto{\pgfqpoint{3.542311in}{1.718655in}}%
\pgfpathlineto{\pgfqpoint{3.550312in}{1.723059in}}%
\pgfpathlineto{\pgfqpoint{3.558304in}{1.727607in}}%
\pgfpathlineto{\pgfqpoint{3.566287in}{1.732298in}}%
\pgfpathlineto{\pgfqpoint{3.574262in}{1.737129in}}%
\pgfpathlineto{\pgfqpoint{3.560904in}{1.745891in}}%
\pgfpathlineto{\pgfqpoint{3.547549in}{1.754786in}}%
\pgfpathlineto{\pgfqpoint{3.534195in}{1.763815in}}%
\pgfpathlineto{\pgfqpoint{3.520843in}{1.772977in}}%
\pgfpathlineto{\pgfqpoint{3.512845in}{1.768502in}}%
\pgfpathlineto{\pgfqpoint{3.504838in}{1.764171in}}%
\pgfpathlineto{\pgfqpoint{3.496823in}{1.759986in}}%
\pgfpathlineto{\pgfqpoint{3.488798in}{1.755951in}}%
\pgfpathclose%
\pgfusepath{fill}%
\end{pgfscope}%
\begin{pgfscope}%
\pgfpathrectangle{\pgfqpoint{1.254980in}{0.150000in}}{\pgfqpoint{5.490039in}{5.490039in}}%
\pgfusepath{clip}%
\pgfsetbuttcap%
\pgfsetroundjoin%
\definecolor{currentfill}{rgb}{0.273809,0.031497,0.358853}%
\pgfsetfillcolor{currentfill}%
\pgfsetfillopacity{0.700000}%
\pgfsetlinewidth{0.000000pt}%
\definecolor{currentstroke}{rgb}{0.000000,0.000000,0.000000}%
\pgfsetstrokecolor{currentstroke}%
\pgfsetdash{}{0pt}%
\pgfpathmoveto{\pgfqpoint{3.681217in}{1.671751in}}%
\pgfpathlineto{\pgfqpoint{3.694599in}{1.664162in}}%
\pgfpathlineto{\pgfqpoint{3.707985in}{1.656701in}}%
\pgfpathlineto{\pgfqpoint{3.721374in}{1.649368in}}%
\pgfpathlineto{\pgfqpoint{3.734766in}{1.642162in}}%
\pgfpathlineto{\pgfqpoint{3.742671in}{1.648177in}}%
\pgfpathlineto{\pgfqpoint{3.750568in}{1.654314in}}%
\pgfpathlineto{\pgfqpoint{3.758457in}{1.660569in}}%
\pgfpathlineto{\pgfqpoint{3.766340in}{1.666939in}}%
\pgfpathlineto{\pgfqpoint{3.752966in}{1.673802in}}%
\pgfpathlineto{\pgfqpoint{3.739596in}{1.680792in}}%
\pgfpathlineto{\pgfqpoint{3.726230in}{1.687910in}}%
\pgfpathlineto{\pgfqpoint{3.712867in}{1.695156in}}%
\pgfpathlineto{\pgfqpoint{3.704966in}{1.689123in}}%
\pgfpathlineto{\pgfqpoint{3.697057in}{1.683209in}}%
\pgfpathlineto{\pgfqpoint{3.689141in}{1.677417in}}%
\pgfpathlineto{\pgfqpoint{3.681217in}{1.671751in}}%
\pgfpathclose%
\pgfusepath{fill}%
\end{pgfscope}%
\begin{pgfscope}%
\pgfpathrectangle{\pgfqpoint{1.254980in}{0.150000in}}{\pgfqpoint{5.490039in}{5.490039in}}%
\pgfusepath{clip}%
\pgfsetbuttcap%
\pgfsetroundjoin%
\definecolor{currentfill}{rgb}{0.262138,0.242286,0.520837}%
\pgfsetfillcolor{currentfill}%
\pgfsetfillopacity{0.700000}%
\pgfsetlinewidth{0.000000pt}%
\definecolor{currentstroke}{rgb}{0.000000,0.000000,0.000000}%
\pgfsetstrokecolor{currentstroke}%
\pgfsetdash{}{0pt}%
\pgfpathmoveto{\pgfqpoint{4.944550in}{2.043618in}}%
\pgfpathlineto{\pgfqpoint{4.958311in}{2.046825in}}%
\pgfpathlineto{\pgfqpoint{4.972085in}{2.050145in}}%
\pgfpathlineto{\pgfqpoint{4.985872in}{2.053579in}}%
\pgfpathlineto{\pgfqpoint{4.999670in}{2.057125in}}%
\pgfpathlineto{\pgfqpoint{5.007156in}{2.068799in}}%
\pgfpathlineto{\pgfqpoint{5.014637in}{2.080435in}}%
\pgfpathlineto{\pgfqpoint{5.022113in}{2.092030in}}%
\pgfpathlineto{\pgfqpoint{5.029584in}{2.103585in}}%
\pgfpathlineto{\pgfqpoint{5.015789in}{2.099903in}}%
\pgfpathlineto{\pgfqpoint{5.002006in}{2.096334in}}%
\pgfpathlineto{\pgfqpoint{4.988235in}{2.092878in}}%
\pgfpathlineto{\pgfqpoint{4.974477in}{2.089536in}}%
\pgfpathlineto{\pgfqpoint{4.967003in}{2.078110in}}%
\pgfpathlineto{\pgfqpoint{4.959524in}{2.066648in}}%
\pgfpathlineto{\pgfqpoint{4.952039in}{2.055150in}}%
\pgfpathlineto{\pgfqpoint{4.944550in}{2.043618in}}%
\pgfpathclose%
\pgfusepath{fill}%
\end{pgfscope}%
\begin{pgfscope}%
\pgfpathrectangle{\pgfqpoint{1.254980in}{0.150000in}}{\pgfqpoint{5.490039in}{5.490039in}}%
\pgfusepath{clip}%
\pgfsetbuttcap%
\pgfsetroundjoin%
\definecolor{currentfill}{rgb}{0.121380,0.629492,0.531973}%
\pgfsetfillcolor{currentfill}%
\pgfsetfillopacity{0.700000}%
\pgfsetlinewidth{0.000000pt}%
\definecolor{currentstroke}{rgb}{0.000000,0.000000,0.000000}%
\pgfsetstrokecolor{currentstroke}%
\pgfsetdash{}{0pt}%
\pgfpathmoveto{\pgfqpoint{2.357947in}{3.106610in}}%
\pgfpathlineto{\pgfqpoint{2.371669in}{3.083042in}}%
\pgfpathlineto{\pgfqpoint{2.385379in}{3.059689in}}%
\pgfpathlineto{\pgfqpoint{2.399079in}{3.036547in}}%
\pgfpathlineto{\pgfqpoint{2.412768in}{3.013615in}}%
\pgfpathlineto{\pgfqpoint{2.421515in}{3.009918in}}%
\pgfpathlineto{\pgfqpoint{2.430244in}{3.006464in}}%
\pgfpathlineto{\pgfqpoint{2.438956in}{3.003249in}}%
\pgfpathlineto{\pgfqpoint{2.447650in}{3.000270in}}%
\pgfpathlineto{\pgfqpoint{2.434009in}{3.022794in}}%
\pgfpathlineto{\pgfqpoint{2.420358in}{3.045528in}}%
\pgfpathlineto{\pgfqpoint{2.406696in}{3.068472in}}%
\pgfpathlineto{\pgfqpoint{2.393024in}{3.091630in}}%
\pgfpathlineto{\pgfqpoint{2.384282in}{3.095010in}}%
\pgfpathlineto{\pgfqpoint{2.375522in}{3.098631in}}%
\pgfpathlineto{\pgfqpoint{2.366744in}{3.102497in}}%
\pgfpathlineto{\pgfqpoint{2.357947in}{3.106610in}}%
\pgfpathclose%
\pgfusepath{fill}%
\end{pgfscope}%
\begin{pgfscope}%
\pgfpathrectangle{\pgfqpoint{1.254980in}{0.150000in}}{\pgfqpoint{5.490039in}{5.490039in}}%
\pgfusepath{clip}%
\pgfsetbuttcap%
\pgfsetroundjoin%
\definecolor{currentfill}{rgb}{0.160665,0.478540,0.558115}%
\pgfsetfillcolor{currentfill}%
\pgfsetfillopacity{0.700000}%
\pgfsetlinewidth{0.000000pt}%
\definecolor{currentstroke}{rgb}{0.000000,0.000000,0.000000}%
\pgfsetstrokecolor{currentstroke}%
\pgfsetdash{}{0pt}%
\pgfpathmoveto{\pgfqpoint{2.596096in}{2.684737in}}%
\pgfpathlineto{\pgfqpoint{2.609678in}{2.664758in}}%
\pgfpathlineto{\pgfqpoint{2.623253in}{2.644965in}}%
\pgfpathlineto{\pgfqpoint{2.636821in}{2.625358in}}%
\pgfpathlineto{\pgfqpoint{2.650381in}{2.605935in}}%
\pgfpathlineto{\pgfqpoint{2.658965in}{2.603362in}}%
\pgfpathlineto{\pgfqpoint{2.667532in}{2.601019in}}%
\pgfpathlineto{\pgfqpoint{2.676083in}{2.598904in}}%
\pgfpathlineto{\pgfqpoint{2.684618in}{2.597013in}}%
\pgfpathlineto{\pgfqpoint{2.671101in}{2.616026in}}%
\pgfpathlineto{\pgfqpoint{2.657577in}{2.635222in}}%
\pgfpathlineto{\pgfqpoint{2.644047in}{2.654604in}}%
\pgfpathlineto{\pgfqpoint{2.630508in}{2.674172in}}%
\pgfpathlineto{\pgfqpoint{2.621930in}{2.676466in}}%
\pgfpathlineto{\pgfqpoint{2.613335in}{2.678990in}}%
\pgfpathlineto{\pgfqpoint{2.604724in}{2.681746in}}%
\pgfpathlineto{\pgfqpoint{2.596096in}{2.684737in}}%
\pgfpathclose%
\pgfusepath{fill}%
\end{pgfscope}%
\begin{pgfscope}%
\pgfpathrectangle{\pgfqpoint{1.254980in}{0.150000in}}{\pgfqpoint{5.490039in}{5.490039in}}%
\pgfusepath{clip}%
\pgfsetbuttcap%
\pgfsetroundjoin%
\definecolor{currentfill}{rgb}{0.165117,0.467423,0.558141}%
\pgfsetfillcolor{currentfill}%
\pgfsetfillopacity{0.700000}%
\pgfsetlinewidth{0.000000pt}%
\definecolor{currentstroke}{rgb}{0.000000,0.000000,0.000000}%
\pgfsetstrokecolor{currentstroke}%
\pgfsetdash{}{0pt}%
\pgfpathmoveto{\pgfqpoint{5.600101in}{2.582761in}}%
\pgfpathlineto{\pgfqpoint{5.614216in}{2.589759in}}%
\pgfpathlineto{\pgfqpoint{5.628346in}{2.596870in}}%
\pgfpathlineto{\pgfqpoint{5.642493in}{2.604093in}}%
\pgfpathlineto{\pgfqpoint{5.656655in}{2.611429in}}%
\pgfpathlineto{\pgfqpoint{5.663897in}{2.621350in}}%
\pgfpathlineto{\pgfqpoint{5.671131in}{2.631192in}}%
\pgfpathlineto{\pgfqpoint{5.678359in}{2.640958in}}%
\pgfpathlineto{\pgfqpoint{5.685579in}{2.650646in}}%
\pgfpathlineto{\pgfqpoint{5.671423in}{2.643324in}}%
\pgfpathlineto{\pgfqpoint{5.657283in}{2.636115in}}%
\pgfpathlineto{\pgfqpoint{5.643159in}{2.629018in}}%
\pgfpathlineto{\pgfqpoint{5.629051in}{2.622033in}}%
\pgfpathlineto{\pgfqpoint{5.621824in}{2.612325in}}%
\pgfpathlineto{\pgfqpoint{5.614590in}{2.602543in}}%
\pgfpathlineto{\pgfqpoint{5.607349in}{2.592689in}}%
\pgfpathlineto{\pgfqpoint{5.600101in}{2.582761in}}%
\pgfpathclose%
\pgfusepath{fill}%
\end{pgfscope}%
\begin{pgfscope}%
\pgfpathrectangle{\pgfqpoint{1.254980in}{0.150000in}}{\pgfqpoint{5.490039in}{5.490039in}}%
\pgfusepath{clip}%
\pgfsetbuttcap%
\pgfsetroundjoin%
\definecolor{currentfill}{rgb}{0.204903,0.375746,0.553533}%
\pgfsetfillcolor{currentfill}%
\pgfsetfillopacity{0.700000}%
\pgfsetlinewidth{0.000000pt}%
\definecolor{currentstroke}{rgb}{0.000000,0.000000,0.000000}%
\pgfsetstrokecolor{currentstroke}%
\pgfsetdash{}{0pt}%
\pgfpathmoveto{\pgfqpoint{5.314786in}{2.339934in}}%
\pgfpathlineto{\pgfqpoint{5.328742in}{2.345464in}}%
\pgfpathlineto{\pgfqpoint{5.342712in}{2.351107in}}%
\pgfpathlineto{\pgfqpoint{5.356696in}{2.356863in}}%
\pgfpathlineto{\pgfqpoint{5.370695in}{2.362732in}}%
\pgfpathlineto{\pgfqpoint{5.378057in}{2.373740in}}%
\pgfpathlineto{\pgfqpoint{5.385413in}{2.384681in}}%
\pgfpathlineto{\pgfqpoint{5.392762in}{2.395555in}}%
\pgfpathlineto{\pgfqpoint{5.400106in}{2.406362in}}%
\pgfpathlineto{\pgfqpoint{5.386111in}{2.400440in}}%
\pgfpathlineto{\pgfqpoint{5.372131in}{2.394630in}}%
\pgfpathlineto{\pgfqpoint{5.358165in}{2.388933in}}%
\pgfpathlineto{\pgfqpoint{5.344214in}{2.383349in}}%
\pgfpathlineto{\pgfqpoint{5.336866in}{2.372589in}}%
\pgfpathlineto{\pgfqpoint{5.329512in}{2.361766in}}%
\pgfpathlineto{\pgfqpoint{5.322152in}{2.350881in}}%
\pgfpathlineto{\pgfqpoint{5.314786in}{2.339934in}}%
\pgfpathclose%
\pgfusepath{fill}%
\end{pgfscope}%
\begin{pgfscope}%
\pgfpathrectangle{\pgfqpoint{1.254980in}{0.150000in}}{\pgfqpoint{5.490039in}{5.490039in}}%
\pgfusepath{clip}%
\pgfsetbuttcap%
\pgfsetroundjoin%
\definecolor{currentfill}{rgb}{0.280868,0.160771,0.472899}%
\pgfsetfillcolor{currentfill}%
\pgfsetfillopacity{0.700000}%
\pgfsetlinewidth{0.000000pt}%
\definecolor{currentstroke}{rgb}{0.000000,0.000000,0.000000}%
\pgfsetstrokecolor{currentstroke}%
\pgfsetdash{}{0pt}%
\pgfpathmoveto{\pgfqpoint{3.242287in}{1.918536in}}%
\pgfpathlineto{\pgfqpoint{3.255681in}{1.906407in}}%
\pgfpathlineto{\pgfqpoint{3.269075in}{1.894422in}}%
\pgfpathlineto{\pgfqpoint{3.282469in}{1.882580in}}%
\pgfpathlineto{\pgfqpoint{3.295862in}{1.870880in}}%
\pgfpathlineto{\pgfqpoint{3.304007in}{1.873124in}}%
\pgfpathlineto{\pgfqpoint{3.312142in}{1.875543in}}%
\pgfpathlineto{\pgfqpoint{3.320266in}{1.878133in}}%
\pgfpathlineto{\pgfqpoint{3.328379in}{1.880892in}}%
\pgfpathlineto{\pgfqpoint{3.315015in}{1.892209in}}%
\pgfpathlineto{\pgfqpoint{3.301650in}{1.903668in}}%
\pgfpathlineto{\pgfqpoint{3.288286in}{1.915269in}}%
\pgfpathlineto{\pgfqpoint{3.274921in}{1.927013in}}%
\pgfpathlineto{\pgfqpoint{3.266779in}{1.924631in}}%
\pgfpathlineto{\pgfqpoint{3.258626in}{1.922422in}}%
\pgfpathlineto{\pgfqpoint{3.250462in}{1.920389in}}%
\pgfpathlineto{\pgfqpoint{3.242287in}{1.918536in}}%
\pgfpathclose%
\pgfusepath{fill}%
\end{pgfscope}%
\begin{pgfscope}%
\pgfpathrectangle{\pgfqpoint{1.254980in}{0.150000in}}{\pgfqpoint{5.490039in}{5.490039in}}%
\pgfusepath{clip}%
\pgfsetbuttcap%
\pgfsetroundjoin%
\definecolor{currentfill}{rgb}{0.271305,0.019942,0.347269}%
\pgfsetfillcolor{currentfill}%
\pgfsetfillopacity{0.700000}%
\pgfsetlinewidth{0.000000pt}%
\definecolor{currentstroke}{rgb}{0.000000,0.000000,0.000000}%
\pgfsetstrokecolor{currentstroke}%
\pgfsetdash{}{0pt}%
\pgfpathmoveto{\pgfqpoint{4.096971in}{1.623279in}}%
\pgfpathlineto{\pgfqpoint{4.110423in}{1.619634in}}%
\pgfpathlineto{\pgfqpoint{4.123881in}{1.616109in}}%
\pgfpathlineto{\pgfqpoint{4.137345in}{1.612704in}}%
\pgfpathlineto{\pgfqpoint{4.150817in}{1.609417in}}%
\pgfpathlineto{\pgfqpoint{4.158555in}{1.618502in}}%
\pgfpathlineto{\pgfqpoint{4.166289in}{1.627653in}}%
\pgfpathlineto{\pgfqpoint{4.174017in}{1.636867in}}%
\pgfpathlineto{\pgfqpoint{4.181739in}{1.646143in}}%
\pgfpathlineto{\pgfqpoint{4.168279in}{1.649137in}}%
\pgfpathlineto{\pgfqpoint{4.154825in}{1.652250in}}%
\pgfpathlineto{\pgfqpoint{4.141379in}{1.655482in}}%
\pgfpathlineto{\pgfqpoint{4.127939in}{1.658833in}}%
\pgfpathlineto{\pgfqpoint{4.120205in}{1.649845in}}%
\pgfpathlineto{\pgfqpoint{4.112466in}{1.640921in}}%
\pgfpathlineto{\pgfqpoint{4.104721in}{1.632065in}}%
\pgfpathlineto{\pgfqpoint{4.096971in}{1.623279in}}%
\pgfpathclose%
\pgfusepath{fill}%
\end{pgfscope}%
\begin{pgfscope}%
\pgfpathrectangle{\pgfqpoint{1.254980in}{0.150000in}}{\pgfqpoint{5.490039in}{5.490039in}}%
\pgfusepath{clip}%
\pgfsetbuttcap%
\pgfsetroundjoin%
\definecolor{currentfill}{rgb}{0.250425,0.274290,0.533103}%
\pgfsetfillcolor{currentfill}%
\pgfsetfillopacity{0.700000}%
\pgfsetlinewidth{0.000000pt}%
\definecolor{currentstroke}{rgb}{0.000000,0.000000,0.000000}%
\pgfsetstrokecolor{currentstroke}%
\pgfsetdash{}{0pt}%
\pgfpathmoveto{\pgfqpoint{5.029584in}{2.103585in}}%
\pgfpathlineto{\pgfqpoint{5.043392in}{2.107380in}}%
\pgfpathlineto{\pgfqpoint{5.057213in}{2.111289in}}%
\pgfpathlineto{\pgfqpoint{5.071047in}{2.115311in}}%
\pgfpathlineto{\pgfqpoint{5.084893in}{2.119445in}}%
\pgfpathlineto{\pgfqpoint{5.092356in}{2.131084in}}%
\pgfpathlineto{\pgfqpoint{5.099813in}{2.142677in}}%
\pgfpathlineto{\pgfqpoint{5.107265in}{2.154222in}}%
\pgfpathlineto{\pgfqpoint{5.114711in}{2.165720in}}%
\pgfpathlineto{\pgfqpoint{5.100868in}{2.161466in}}%
\pgfpathlineto{\pgfqpoint{5.087038in}{2.157325in}}%
\pgfpathlineto{\pgfqpoint{5.073220in}{2.153297in}}%
\pgfpathlineto{\pgfqpoint{5.059415in}{2.149382in}}%
\pgfpathlineto{\pgfqpoint{5.051965in}{2.137998in}}%
\pgfpathlineto{\pgfqpoint{5.044510in}{2.126569in}}%
\pgfpathlineto{\pgfqpoint{5.037050in}{2.115098in}}%
\pgfpathlineto{\pgfqpoint{5.029584in}{2.103585in}}%
\pgfpathclose%
\pgfusepath{fill}%
\end{pgfscope}%
\begin{pgfscope}%
\pgfpathrectangle{\pgfqpoint{1.254980in}{0.150000in}}{\pgfqpoint{5.490039in}{5.490039in}}%
\pgfusepath{clip}%
\pgfsetbuttcap%
\pgfsetroundjoin%
\definecolor{currentfill}{rgb}{0.154815,0.493313,0.557840}%
\pgfsetfillcolor{currentfill}%
\pgfsetfillopacity{0.700000}%
\pgfsetlinewidth{0.000000pt}%
\definecolor{currentstroke}{rgb}{0.000000,0.000000,0.000000}%
\pgfsetstrokecolor{currentstroke}%
\pgfsetdash{}{0pt}%
\pgfpathmoveto{\pgfqpoint{5.685579in}{2.650646in}}%
\pgfpathlineto{\pgfqpoint{5.699751in}{2.658081in}}%
\pgfpathlineto{\pgfqpoint{5.713939in}{2.665629in}}%
\pgfpathlineto{\pgfqpoint{5.728143in}{2.673290in}}%
\pgfpathlineto{\pgfqpoint{5.735351in}{2.682883in}}%
\pgfpathlineto{\pgfqpoint{5.742551in}{2.692398in}}%
\pgfpathlineto{\pgfqpoint{5.749745in}{2.701835in}}%
\pgfpathlineto{\pgfqpoint{5.756931in}{2.711193in}}%
\pgfpathlineto{\pgfqpoint{5.742734in}{2.703564in}}%
\pgfpathlineto{\pgfqpoint{5.728554in}{2.696047in}}%
\pgfpathlineto{\pgfqpoint{5.714389in}{2.688643in}}%
\pgfpathlineto{\pgfqpoint{5.707197in}{2.679256in}}%
\pgfpathlineto{\pgfqpoint{5.699998in}{2.669795in}}%
\pgfpathlineto{\pgfqpoint{5.692792in}{2.660259in}}%
\pgfpathlineto{\pgfqpoint{5.685579in}{2.650646in}}%
\pgfpathclose%
\pgfusepath{fill}%
\end{pgfscope}%
\begin{pgfscope}%
\pgfpathrectangle{\pgfqpoint{1.254980in}{0.150000in}}{\pgfqpoint{5.490039in}{5.490039in}}%
\pgfusepath{clip}%
\pgfsetbuttcap%
\pgfsetroundjoin%
\definecolor{currentfill}{rgb}{0.149039,0.508051,0.557250}%
\pgfsetfillcolor{currentfill}%
\pgfsetfillopacity{0.700000}%
\pgfsetlinewidth{0.000000pt}%
\definecolor{currentstroke}{rgb}{0.000000,0.000000,0.000000}%
\pgfsetstrokecolor{currentstroke}%
\pgfsetdash{}{0pt}%
\pgfpathmoveto{\pgfqpoint{2.541686in}{2.766552in}}%
\pgfpathlineto{\pgfqpoint{2.555300in}{2.745812in}}%
\pgfpathlineto{\pgfqpoint{2.568907in}{2.725263in}}%
\pgfpathlineto{\pgfqpoint{2.582505in}{2.704906in}}%
\pgfpathlineto{\pgfqpoint{2.596096in}{2.684737in}}%
\pgfpathlineto{\pgfqpoint{2.604724in}{2.681746in}}%
\pgfpathlineto{\pgfqpoint{2.613335in}{2.678990in}}%
\pgfpathlineto{\pgfqpoint{2.621930in}{2.676466in}}%
\pgfpathlineto{\pgfqpoint{2.630508in}{2.674172in}}%
\pgfpathlineto{\pgfqpoint{2.616963in}{2.693927in}}%
\pgfpathlineto{\pgfqpoint{2.603410in}{2.713871in}}%
\pgfpathlineto{\pgfqpoint{2.589849in}{2.734004in}}%
\pgfpathlineto{\pgfqpoint{2.576280in}{2.754329in}}%
\pgfpathlineto{\pgfqpoint{2.567657in}{2.757030in}}%
\pgfpathlineto{\pgfqpoint{2.559017in}{2.759966in}}%
\pgfpathlineto{\pgfqpoint{2.550360in}{2.763138in}}%
\pgfpathlineto{\pgfqpoint{2.541686in}{2.766552in}}%
\pgfpathclose%
\pgfusepath{fill}%
\end{pgfscope}%
\begin{pgfscope}%
\pgfpathrectangle{\pgfqpoint{1.254980in}{0.150000in}}{\pgfqpoint{5.490039in}{5.490039in}}%
\pgfusepath{clip}%
\pgfsetbuttcap%
\pgfsetroundjoin%
\definecolor{currentfill}{rgb}{0.279566,0.067836,0.391917}%
\pgfsetfillcolor{currentfill}%
\pgfsetfillopacity{0.700000}%
\pgfsetlinewidth{0.000000pt}%
\definecolor{currentstroke}{rgb}{0.000000,0.000000,0.000000}%
\pgfsetstrokecolor{currentstroke}%
\pgfsetdash{}{0pt}%
\pgfpathmoveto{\pgfqpoint{3.542311in}{1.718655in}}%
\pgfpathlineto{\pgfqpoint{3.555694in}{1.709664in}}%
\pgfpathlineto{\pgfqpoint{3.569079in}{1.700804in}}%
\pgfpathlineto{\pgfqpoint{3.582467in}{1.692076in}}%
\pgfpathlineto{\pgfqpoint{3.595856in}{1.683480in}}%
\pgfpathlineto{\pgfqpoint{3.603834in}{1.688250in}}%
\pgfpathlineto{\pgfqpoint{3.611803in}{1.693162in}}%
\pgfpathlineto{\pgfqpoint{3.619765in}{1.698212in}}%
\pgfpathlineto{\pgfqpoint{3.627717in}{1.703397in}}%
\pgfpathlineto{\pgfqpoint{3.614350in}{1.711633in}}%
\pgfpathlineto{\pgfqpoint{3.600985in}{1.720000in}}%
\pgfpathlineto{\pgfqpoint{3.587622in}{1.728498in}}%
\pgfpathlineto{\pgfqpoint{3.574262in}{1.737129in}}%
\pgfpathlineto{\pgfqpoint{3.566287in}{1.732298in}}%
\pgfpathlineto{\pgfqpoint{3.558304in}{1.727607in}}%
\pgfpathlineto{\pgfqpoint{3.550312in}{1.723059in}}%
\pgfpathlineto{\pgfqpoint{3.542311in}{1.718655in}}%
\pgfpathclose%
\pgfusepath{fill}%
\end{pgfscope}%
\begin{pgfscope}%
\pgfpathrectangle{\pgfqpoint{1.254980in}{0.150000in}}{\pgfqpoint{5.490039in}{5.490039in}}%
\pgfusepath{clip}%
\pgfsetbuttcap%
\pgfsetroundjoin%
\definecolor{currentfill}{rgb}{0.282327,0.094955,0.417331}%
\pgfsetfillcolor{currentfill}%
\pgfsetfillopacity{0.700000}%
\pgfsetlinewidth{0.000000pt}%
\definecolor{currentstroke}{rgb}{0.000000,0.000000,0.000000}%
\pgfsetstrokecolor{currentstroke}%
\pgfsetdash{}{0pt}%
\pgfpathmoveto{\pgfqpoint{4.490002in}{1.742064in}}%
\pgfpathlineto{\pgfqpoint{4.503576in}{1.741842in}}%
\pgfpathlineto{\pgfqpoint{4.517159in}{1.741735in}}%
\pgfpathlineto{\pgfqpoint{4.530751in}{1.741743in}}%
\pgfpathlineto{\pgfqpoint{4.544353in}{1.741867in}}%
\pgfpathlineto{\pgfqpoint{4.551973in}{1.752925in}}%
\pgfpathlineto{\pgfqpoint{4.559588in}{1.763997in}}%
\pgfpathlineto{\pgfqpoint{4.567198in}{1.775080in}}%
\pgfpathlineto{\pgfqpoint{4.574804in}{1.786173in}}%
\pgfpathlineto{\pgfqpoint{4.561208in}{1.785820in}}%
\pgfpathlineto{\pgfqpoint{4.547621in}{1.785581in}}%
\pgfpathlineto{\pgfqpoint{4.534044in}{1.785458in}}%
\pgfpathlineto{\pgfqpoint{4.520477in}{1.785450in}}%
\pgfpathlineto{\pgfqpoint{4.512865in}{1.774581in}}%
\pgfpathlineto{\pgfqpoint{4.505249in}{1.763726in}}%
\pgfpathlineto{\pgfqpoint{4.497628in}{1.752886in}}%
\pgfpathlineto{\pgfqpoint{4.490002in}{1.742064in}}%
\pgfpathclose%
\pgfusepath{fill}%
\end{pgfscope}%
\begin{pgfscope}%
\pgfpathrectangle{\pgfqpoint{1.254980in}{0.150000in}}{\pgfqpoint{5.490039in}{5.490039in}}%
\pgfusepath{clip}%
\pgfsetbuttcap%
\pgfsetroundjoin%
\definecolor{currentfill}{rgb}{0.283229,0.120777,0.440584}%
\pgfsetfillcolor{currentfill}%
\pgfsetfillopacity{0.700000}%
\pgfsetlinewidth{0.000000pt}%
\definecolor{currentstroke}{rgb}{0.000000,0.000000,0.000000}%
\pgfsetstrokecolor{currentstroke}%
\pgfsetdash{}{0pt}%
\pgfpathmoveto{\pgfqpoint{4.574804in}{1.786173in}}%
\pgfpathlineto{\pgfqpoint{4.588410in}{1.786641in}}%
\pgfpathlineto{\pgfqpoint{4.602026in}{1.787224in}}%
\pgfpathlineto{\pgfqpoint{4.615652in}{1.787922in}}%
\pgfpathlineto{\pgfqpoint{4.629288in}{1.788734in}}%
\pgfpathlineto{\pgfqpoint{4.636884in}{1.800056in}}%
\pgfpathlineto{\pgfqpoint{4.644476in}{1.811380in}}%
\pgfpathlineto{\pgfqpoint{4.652063in}{1.822705in}}%
\pgfpathlineto{\pgfqpoint{4.659645in}{1.834030in}}%
\pgfpathlineto{\pgfqpoint{4.646014in}{1.833003in}}%
\pgfpathlineto{\pgfqpoint{4.632393in}{1.832091in}}%
\pgfpathlineto{\pgfqpoint{4.618782in}{1.831294in}}%
\pgfpathlineto{\pgfqpoint{4.605181in}{1.830611in}}%
\pgfpathlineto{\pgfqpoint{4.597594in}{1.819495in}}%
\pgfpathlineto{\pgfqpoint{4.590002in}{1.808382in}}%
\pgfpathlineto{\pgfqpoint{4.582405in}{1.797274in}}%
\pgfpathlineto{\pgfqpoint{4.574804in}{1.786173in}}%
\pgfpathclose%
\pgfusepath{fill}%
\end{pgfscope}%
\begin{pgfscope}%
\pgfpathrectangle{\pgfqpoint{1.254980in}{0.150000in}}{\pgfqpoint{5.490039in}{5.490039in}}%
\pgfusepath{clip}%
\pgfsetbuttcap%
\pgfsetroundjoin%
\definecolor{currentfill}{rgb}{0.282623,0.140926,0.457517}%
\pgfsetfillcolor{currentfill}%
\pgfsetfillopacity{0.700000}%
\pgfsetlinewidth{0.000000pt}%
\definecolor{currentstroke}{rgb}{0.000000,0.000000,0.000000}%
\pgfsetstrokecolor{currentstroke}%
\pgfsetdash{}{0pt}%
\pgfpathmoveto{\pgfqpoint{3.295862in}{1.870880in}}%
\pgfpathlineto{\pgfqpoint{3.309255in}{1.859322in}}%
\pgfpathlineto{\pgfqpoint{3.322649in}{1.847905in}}%
\pgfpathlineto{\pgfqpoint{3.336042in}{1.836629in}}%
\pgfpathlineto{\pgfqpoint{3.349435in}{1.825492in}}%
\pgfpathlineto{\pgfqpoint{3.357552in}{1.828125in}}%
\pgfpathlineto{\pgfqpoint{3.365658in}{1.830928in}}%
\pgfpathlineto{\pgfqpoint{3.373754in}{1.833899in}}%
\pgfpathlineto{\pgfqpoint{3.381840in}{1.837034in}}%
\pgfpathlineto{\pgfqpoint{3.368474in}{1.847788in}}%
\pgfpathlineto{\pgfqpoint{3.355109in}{1.858683in}}%
\pgfpathlineto{\pgfqpoint{3.341744in}{1.869717in}}%
\pgfpathlineto{\pgfqpoint{3.328379in}{1.880892in}}%
\pgfpathlineto{\pgfqpoint{3.320266in}{1.878133in}}%
\pgfpathlineto{\pgfqpoint{3.312142in}{1.875543in}}%
\pgfpathlineto{\pgfqpoint{3.304007in}{1.873124in}}%
\pgfpathlineto{\pgfqpoint{3.295862in}{1.870880in}}%
\pgfpathclose%
\pgfusepath{fill}%
\end{pgfscope}%
\begin{pgfscope}%
\pgfpathrectangle{\pgfqpoint{1.254980in}{0.150000in}}{\pgfqpoint{5.490039in}{5.490039in}}%
\pgfusepath{clip}%
\pgfsetbuttcap%
\pgfsetroundjoin%
\definecolor{currentfill}{rgb}{0.280267,0.073417,0.397163}%
\pgfsetfillcolor{currentfill}%
\pgfsetfillopacity{0.700000}%
\pgfsetlinewidth{0.000000pt}%
\definecolor{currentstroke}{rgb}{0.000000,0.000000,0.000000}%
\pgfsetstrokecolor{currentstroke}%
\pgfsetdash{}{0pt}%
\pgfpathmoveto{\pgfqpoint{4.405223in}{1.702010in}}%
\pgfpathlineto{\pgfqpoint{4.418767in}{1.701079in}}%
\pgfpathlineto{\pgfqpoint{4.432320in}{1.700265in}}%
\pgfpathlineto{\pgfqpoint{4.445882in}{1.699566in}}%
\pgfpathlineto{\pgfqpoint{4.459453in}{1.698983in}}%
\pgfpathlineto{\pgfqpoint{4.467097in}{1.709718in}}%
\pgfpathlineto{\pgfqpoint{4.474737in}{1.720478in}}%
\pgfpathlineto{\pgfqpoint{4.482372in}{1.731260in}}%
\pgfpathlineto{\pgfqpoint{4.490002in}{1.742064in}}%
\pgfpathlineto{\pgfqpoint{4.476438in}{1.742401in}}%
\pgfpathlineto{\pgfqpoint{4.462883in}{1.742854in}}%
\pgfpathlineto{\pgfqpoint{4.449336in}{1.743423in}}%
\pgfpathlineto{\pgfqpoint{4.435799in}{1.744108in}}%
\pgfpathlineto{\pgfqpoint{4.428162in}{1.733544in}}%
\pgfpathlineto{\pgfqpoint{4.420520in}{1.723005in}}%
\pgfpathlineto{\pgfqpoint{4.412874in}{1.712493in}}%
\pgfpathlineto{\pgfqpoint{4.405223in}{1.702010in}}%
\pgfpathclose%
\pgfusepath{fill}%
\end{pgfscope}%
\begin{pgfscope}%
\pgfpathrectangle{\pgfqpoint{1.254980in}{0.150000in}}{\pgfqpoint{5.490039in}{5.490039in}}%
\pgfusepath{clip}%
\pgfsetbuttcap%
\pgfsetroundjoin%
\definecolor{currentfill}{rgb}{0.282290,0.145912,0.461510}%
\pgfsetfillcolor{currentfill}%
\pgfsetfillopacity{0.700000}%
\pgfsetlinewidth{0.000000pt}%
\definecolor{currentstroke}{rgb}{0.000000,0.000000,0.000000}%
\pgfsetstrokecolor{currentstroke}%
\pgfsetdash{}{0pt}%
\pgfpathmoveto{\pgfqpoint{4.659645in}{1.834030in}}%
\pgfpathlineto{\pgfqpoint{4.673287in}{1.835171in}}%
\pgfpathlineto{\pgfqpoint{4.686939in}{1.836426in}}%
\pgfpathlineto{\pgfqpoint{4.700601in}{1.837796in}}%
\pgfpathlineto{\pgfqpoint{4.714274in}{1.839279in}}%
\pgfpathlineto{\pgfqpoint{4.721848in}{1.850806in}}%
\pgfpathlineto{\pgfqpoint{4.729417in}{1.862326in}}%
\pgfpathlineto{\pgfqpoint{4.736981in}{1.873837in}}%
\pgfpathlineto{\pgfqpoint{4.744540in}{1.885337in}}%
\pgfpathlineto{\pgfqpoint{4.730871in}{1.883655in}}%
\pgfpathlineto{\pgfqpoint{4.717213in}{1.882086in}}%
\pgfpathlineto{\pgfqpoint{4.703565in}{1.880632in}}%
\pgfpathlineto{\pgfqpoint{4.689928in}{1.879293in}}%
\pgfpathlineto{\pgfqpoint{4.682364in}{1.867985in}}%
\pgfpathlineto{\pgfqpoint{4.674796in}{1.856671in}}%
\pgfpathlineto{\pgfqpoint{4.667223in}{1.845353in}}%
\pgfpathlineto{\pgfqpoint{4.659645in}{1.834030in}}%
\pgfpathclose%
\pgfusepath{fill}%
\end{pgfscope}%
\begin{pgfscope}%
\pgfpathrectangle{\pgfqpoint{1.254980in}{0.150000in}}{\pgfqpoint{5.490039in}{5.490039in}}%
\pgfusepath{clip}%
\pgfsetbuttcap%
\pgfsetroundjoin%
\definecolor{currentfill}{rgb}{0.269944,0.014625,0.341379}%
\pgfsetfillcolor{currentfill}%
\pgfsetfillopacity{0.700000}%
\pgfsetlinewidth{0.000000pt}%
\definecolor{currentstroke}{rgb}{0.000000,0.000000,0.000000}%
\pgfsetstrokecolor{currentstroke}%
\pgfsetdash{}{0pt}%
\pgfpathmoveto{\pgfqpoint{3.873477in}{1.616563in}}%
\pgfpathlineto{\pgfqpoint{3.886889in}{1.610826in}}%
\pgfpathlineto{\pgfqpoint{3.900306in}{1.605213in}}%
\pgfpathlineto{\pgfqpoint{3.913728in}{1.599723in}}%
\pgfpathlineto{\pgfqpoint{3.927155in}{1.594356in}}%
\pgfpathlineto{\pgfqpoint{3.934980in}{1.601834in}}%
\pgfpathlineto{\pgfqpoint{3.942799in}{1.609409in}}%
\pgfpathlineto{\pgfqpoint{3.950611in}{1.617080in}}%
\pgfpathlineto{\pgfqpoint{3.958417in}{1.624843in}}%
\pgfpathlineto{\pgfqpoint{3.945005in}{1.629886in}}%
\pgfpathlineto{\pgfqpoint{3.931599in}{1.635051in}}%
\pgfpathlineto{\pgfqpoint{3.918197in}{1.640338in}}%
\pgfpathlineto{\pgfqpoint{3.904801in}{1.645750in}}%
\pgfpathlineto{\pgfqpoint{3.896980in}{1.638305in}}%
\pgfpathlineto{\pgfqpoint{3.889152in}{1.630958in}}%
\pgfpathlineto{\pgfqpoint{3.881318in}{1.623709in}}%
\pgfpathlineto{\pgfqpoint{3.873477in}{1.616563in}}%
\pgfpathclose%
\pgfusepath{fill}%
\end{pgfscope}%
\begin{pgfscope}%
\pgfpathrectangle{\pgfqpoint{1.254980in}{0.150000in}}{\pgfqpoint{5.490039in}{5.490039in}}%
\pgfusepath{clip}%
\pgfsetbuttcap%
\pgfsetroundjoin%
\definecolor{currentfill}{rgb}{0.192357,0.403199,0.555836}%
\pgfsetfillcolor{currentfill}%
\pgfsetfillopacity{0.700000}%
\pgfsetlinewidth{0.000000pt}%
\definecolor{currentstroke}{rgb}{0.000000,0.000000,0.000000}%
\pgfsetstrokecolor{currentstroke}%
\pgfsetdash{}{0pt}%
\pgfpathmoveto{\pgfqpoint{5.400106in}{2.406362in}}%
\pgfpathlineto{\pgfqpoint{5.414115in}{2.412398in}}%
\pgfpathlineto{\pgfqpoint{5.428139in}{2.418546in}}%
\pgfpathlineto{\pgfqpoint{5.442178in}{2.424807in}}%
\pgfpathlineto{\pgfqpoint{5.456232in}{2.431181in}}%
\pgfpathlineto{\pgfqpoint{5.463565in}{2.441964in}}%
\pgfpathlineto{\pgfqpoint{5.470891in}{2.452677in}}%
\pgfpathlineto{\pgfqpoint{5.478211in}{2.463318in}}%
\pgfpathlineto{\pgfqpoint{5.485524in}{2.473888in}}%
\pgfpathlineto{\pgfqpoint{5.471475in}{2.467478in}}%
\pgfpathlineto{\pgfqpoint{5.457440in}{2.461180in}}%
\pgfpathlineto{\pgfqpoint{5.443421in}{2.454995in}}%
\pgfpathlineto{\pgfqpoint{5.429416in}{2.448922in}}%
\pgfpathlineto{\pgfqpoint{5.422098in}{2.438383in}}%
\pgfpathlineto{\pgfqpoint{5.414773in}{2.427776in}}%
\pgfpathlineto{\pgfqpoint{5.407443in}{2.417103in}}%
\pgfpathlineto{\pgfqpoint{5.400106in}{2.406362in}}%
\pgfpathclose%
\pgfusepath{fill}%
\end{pgfscope}%
\begin{pgfscope}%
\pgfpathrectangle{\pgfqpoint{1.254980in}{0.150000in}}{\pgfqpoint{5.490039in}{5.490039in}}%
\pgfusepath{clip}%
\pgfsetbuttcap%
\pgfsetroundjoin%
\definecolor{currentfill}{rgb}{0.140210,0.665859,0.513427}%
\pgfsetfillcolor{currentfill}%
\pgfsetfillopacity{0.700000}%
\pgfsetlinewidth{0.000000pt}%
\definecolor{currentstroke}{rgb}{0.000000,0.000000,0.000000}%
\pgfsetstrokecolor{currentstroke}%
\pgfsetdash{}{0pt}%
\pgfpathmoveto{\pgfqpoint{2.302946in}{3.203047in}}%
\pgfpathlineto{\pgfqpoint{2.316714in}{3.178609in}}%
\pgfpathlineto{\pgfqpoint{2.330470in}{3.154392in}}%
\pgfpathlineto{\pgfqpoint{2.344214in}{3.130392in}}%
\pgfpathlineto{\pgfqpoint{2.357947in}{3.106610in}}%
\pgfpathlineto{\pgfqpoint{2.366744in}{3.102497in}}%
\pgfpathlineto{\pgfqpoint{2.375522in}{3.098631in}}%
\pgfpathlineto{\pgfqpoint{2.384282in}{3.095010in}}%
\pgfpathlineto{\pgfqpoint{2.393024in}{3.091630in}}%
\pgfpathlineto{\pgfqpoint{2.379341in}{3.115001in}}%
\pgfpathlineto{\pgfqpoint{2.365647in}{3.138588in}}%
\pgfpathlineto{\pgfqpoint{2.351942in}{3.162393in}}%
\pgfpathlineto{\pgfqpoint{2.338225in}{3.186416in}}%
\pgfpathlineto{\pgfqpoint{2.329433in}{3.190201in}}%
\pgfpathlineto{\pgfqpoint{2.320623in}{3.194232in}}%
\pgfpathlineto{\pgfqpoint{2.311794in}{3.198513in}}%
\pgfpathlineto{\pgfqpoint{2.302946in}{3.203047in}}%
\pgfpathclose%
\pgfusepath{fill}%
\end{pgfscope}%
\begin{pgfscope}%
\pgfpathrectangle{\pgfqpoint{1.254980in}{0.150000in}}{\pgfqpoint{5.490039in}{5.490039in}}%
\pgfusepath{clip}%
\pgfsetbuttcap%
\pgfsetroundjoin%
\definecolor{currentfill}{rgb}{0.237441,0.305202,0.541921}%
\pgfsetfillcolor{currentfill}%
\pgfsetfillopacity{0.700000}%
\pgfsetlinewidth{0.000000pt}%
\definecolor{currentstroke}{rgb}{0.000000,0.000000,0.000000}%
\pgfsetstrokecolor{currentstroke}%
\pgfsetdash{}{0pt}%
\pgfpathmoveto{\pgfqpoint{5.114711in}{2.165720in}}%
\pgfpathlineto{\pgfqpoint{5.128568in}{2.170087in}}%
\pgfpathlineto{\pgfqpoint{5.142438in}{2.174567in}}%
\pgfpathlineto{\pgfqpoint{5.156321in}{2.179160in}}%
\pgfpathlineto{\pgfqpoint{5.170217in}{2.183866in}}%
\pgfpathlineto{\pgfqpoint{5.177656in}{2.195424in}}%
\pgfpathlineto{\pgfqpoint{5.185088in}{2.206930in}}%
\pgfpathlineto{\pgfqpoint{5.192515in}{2.218381in}}%
\pgfpathlineto{\pgfqpoint{5.199937in}{2.229779in}}%
\pgfpathlineto{\pgfqpoint{5.186043in}{2.224970in}}%
\pgfpathlineto{\pgfqpoint{5.172163in}{2.220273in}}%
\pgfpathlineto{\pgfqpoint{5.158297in}{2.215690in}}%
\pgfpathlineto{\pgfqpoint{5.144444in}{2.211220in}}%
\pgfpathlineto{\pgfqpoint{5.137019in}{2.199920in}}%
\pgfpathlineto{\pgfqpoint{5.129588in}{2.188569in}}%
\pgfpathlineto{\pgfqpoint{5.122153in}{2.177169in}}%
\pgfpathlineto{\pgfqpoint{5.114711in}{2.165720in}}%
\pgfpathclose%
\pgfusepath{fill}%
\end{pgfscope}%
\begin{pgfscope}%
\pgfpathrectangle{\pgfqpoint{1.254980in}{0.150000in}}{\pgfqpoint{5.490039in}{5.490039in}}%
\pgfusepath{clip}%
\pgfsetbuttcap%
\pgfsetroundjoin%
\definecolor{currentfill}{rgb}{0.272594,0.025563,0.353093}%
\pgfsetfillcolor{currentfill}%
\pgfsetfillopacity{0.700000}%
\pgfsetlinewidth{0.000000pt}%
\definecolor{currentstroke}{rgb}{0.000000,0.000000,0.000000}%
\pgfsetstrokecolor{currentstroke}%
\pgfsetdash{}{0pt}%
\pgfpathmoveto{\pgfqpoint{3.734766in}{1.642162in}}%
\pgfpathlineto{\pgfqpoint{3.748163in}{1.635083in}}%
\pgfpathlineto{\pgfqpoint{3.761563in}{1.628130in}}%
\pgfpathlineto{\pgfqpoint{3.774966in}{1.621304in}}%
\pgfpathlineto{\pgfqpoint{3.788374in}{1.614603in}}%
\pgfpathlineto{\pgfqpoint{3.796260in}{1.620967in}}%
\pgfpathlineto{\pgfqpoint{3.804138in}{1.627448in}}%
\pgfpathlineto{\pgfqpoint{3.812010in}{1.634043in}}%
\pgfpathlineto{\pgfqpoint{3.819874in}{1.640749in}}%
\pgfpathlineto{\pgfqpoint{3.806484in}{1.647108in}}%
\pgfpathlineto{\pgfqpoint{3.793099in}{1.653592in}}%
\pgfpathlineto{\pgfqpoint{3.779717in}{1.660202in}}%
\pgfpathlineto{\pgfqpoint{3.766340in}{1.666939in}}%
\pgfpathlineto{\pgfqpoint{3.758457in}{1.660569in}}%
\pgfpathlineto{\pgfqpoint{3.750568in}{1.654314in}}%
\pgfpathlineto{\pgfqpoint{3.742671in}{1.648177in}}%
\pgfpathlineto{\pgfqpoint{3.734766in}{1.642162in}}%
\pgfpathclose%
\pgfusepath{fill}%
\end{pgfscope}%
\begin{pgfscope}%
\pgfpathrectangle{\pgfqpoint{1.254980in}{0.150000in}}{\pgfqpoint{5.490039in}{5.490039in}}%
\pgfusepath{clip}%
\pgfsetbuttcap%
\pgfsetroundjoin%
\definecolor{currentfill}{rgb}{0.277018,0.050344,0.375715}%
\pgfsetfillcolor{currentfill}%
\pgfsetfillopacity{0.700000}%
\pgfsetlinewidth{0.000000pt}%
\definecolor{currentstroke}{rgb}{0.000000,0.000000,0.000000}%
\pgfsetstrokecolor{currentstroke}%
\pgfsetdash{}{0pt}%
\pgfpathmoveto{\pgfqpoint{4.320446in}{1.666329in}}%
\pgfpathlineto{\pgfqpoint{4.333965in}{1.664672in}}%
\pgfpathlineto{\pgfqpoint{4.347492in}{1.663131in}}%
\pgfpathlineto{\pgfqpoint{4.361027in}{1.661707in}}%
\pgfpathlineto{\pgfqpoint{4.374570in}{1.660399in}}%
\pgfpathlineto{\pgfqpoint{4.382241in}{1.670749in}}%
\pgfpathlineto{\pgfqpoint{4.389906in}{1.681136in}}%
\pgfpathlineto{\pgfqpoint{4.397567in}{1.691557in}}%
\pgfpathlineto{\pgfqpoint{4.405223in}{1.702010in}}%
\pgfpathlineto{\pgfqpoint{4.391687in}{1.703056in}}%
\pgfpathlineto{\pgfqpoint{4.378160in}{1.704219in}}%
\pgfpathlineto{\pgfqpoint{4.364641in}{1.705498in}}%
\pgfpathlineto{\pgfqpoint{4.351130in}{1.706894in}}%
\pgfpathlineto{\pgfqpoint{4.343467in}{1.696696in}}%
\pgfpathlineto{\pgfqpoint{4.335798in}{1.686535in}}%
\pgfpathlineto{\pgfqpoint{4.328125in}{1.676412in}}%
\pgfpathlineto{\pgfqpoint{4.320446in}{1.666329in}}%
\pgfpathclose%
\pgfusepath{fill}%
\end{pgfscope}%
\begin{pgfscope}%
\pgfpathrectangle{\pgfqpoint{1.254980in}{0.150000in}}{\pgfqpoint{5.490039in}{5.490039in}}%
\pgfusepath{clip}%
\pgfsetbuttcap%
\pgfsetroundjoin%
\definecolor{currentfill}{rgb}{0.279574,0.170599,0.479997}%
\pgfsetfillcolor{currentfill}%
\pgfsetfillopacity{0.700000}%
\pgfsetlinewidth{0.000000pt}%
\definecolor{currentstroke}{rgb}{0.000000,0.000000,0.000000}%
\pgfsetstrokecolor{currentstroke}%
\pgfsetdash{}{0pt}%
\pgfpathmoveto{\pgfqpoint{4.744540in}{1.885337in}}%
\pgfpathlineto{\pgfqpoint{4.758220in}{1.887133in}}%
\pgfpathlineto{\pgfqpoint{4.771911in}{1.889043in}}%
\pgfpathlineto{\pgfqpoint{4.785613in}{1.891067in}}%
\pgfpathlineto{\pgfqpoint{4.799326in}{1.893205in}}%
\pgfpathlineto{\pgfqpoint{4.806878in}{1.904882in}}%
\pgfpathlineto{\pgfqpoint{4.814424in}{1.916542in}}%
\pgfpathlineto{\pgfqpoint{4.821965in}{1.928184in}}%
\pgfpathlineto{\pgfqpoint{4.829502in}{1.939806in}}%
\pgfpathlineto{\pgfqpoint{4.815793in}{1.937485in}}%
\pgfpathlineto{\pgfqpoint{4.802094in}{1.935278in}}%
\pgfpathlineto{\pgfqpoint{4.788407in}{1.933185in}}%
\pgfpathlineto{\pgfqpoint{4.774731in}{1.931206in}}%
\pgfpathlineto{\pgfqpoint{4.767190in}{1.919761in}}%
\pgfpathlineto{\pgfqpoint{4.759645in}{1.908300in}}%
\pgfpathlineto{\pgfqpoint{4.752095in}{1.896825in}}%
\pgfpathlineto{\pgfqpoint{4.744540in}{1.885337in}}%
\pgfpathclose%
\pgfusepath{fill}%
\end{pgfscope}%
\begin{pgfscope}%
\pgfpathrectangle{\pgfqpoint{1.254980in}{0.150000in}}{\pgfqpoint{5.490039in}{5.490039in}}%
\pgfusepath{clip}%
\pgfsetbuttcap%
\pgfsetroundjoin%
\definecolor{currentfill}{rgb}{0.268510,0.009605,0.335427}%
\pgfsetfillcolor{currentfill}%
\pgfsetfillopacity{0.700000}%
\pgfsetlinewidth{0.000000pt}%
\definecolor{currentstroke}{rgb}{0.000000,0.000000,0.000000}%
\pgfsetstrokecolor{currentstroke}%
\pgfsetdash{}{0pt}%
\pgfpathmoveto{\pgfqpoint{4.012119in}{1.605894in}}%
\pgfpathlineto{\pgfqpoint{4.025559in}{1.601459in}}%
\pgfpathlineto{\pgfqpoint{4.039005in}{1.597146in}}%
\pgfpathlineto{\pgfqpoint{4.052456in}{1.592953in}}%
\pgfpathlineto{\pgfqpoint{4.065914in}{1.588879in}}%
\pgfpathlineto{\pgfqpoint{4.073687in}{1.597363in}}%
\pgfpathlineto{\pgfqpoint{4.081454in}{1.605925in}}%
\pgfpathlineto{\pgfqpoint{4.089215in}{1.614565in}}%
\pgfpathlineto{\pgfqpoint{4.096971in}{1.623279in}}%
\pgfpathlineto{\pgfqpoint{4.083526in}{1.627043in}}%
\pgfpathlineto{\pgfqpoint{4.070087in}{1.630928in}}%
\pgfpathlineto{\pgfqpoint{4.056654in}{1.634932in}}%
\pgfpathlineto{\pgfqpoint{4.043228in}{1.639058in}}%
\pgfpathlineto{\pgfqpoint{4.035460in}{1.630646in}}%
\pgfpathlineto{\pgfqpoint{4.027685in}{1.622314in}}%
\pgfpathlineto{\pgfqpoint{4.019905in}{1.614062in}}%
\pgfpathlineto{\pgfqpoint{4.012119in}{1.605894in}}%
\pgfpathclose%
\pgfusepath{fill}%
\end{pgfscope}%
\begin{pgfscope}%
\pgfpathrectangle{\pgfqpoint{1.254980in}{0.150000in}}{\pgfqpoint{5.490039in}{5.490039in}}%
\pgfusepath{clip}%
\pgfsetbuttcap%
\pgfsetroundjoin%
\definecolor{currentfill}{rgb}{0.136408,0.541173,0.554483}%
\pgfsetfillcolor{currentfill}%
\pgfsetfillopacity{0.700000}%
\pgfsetlinewidth{0.000000pt}%
\definecolor{currentstroke}{rgb}{0.000000,0.000000,0.000000}%
\pgfsetstrokecolor{currentstroke}%
\pgfsetdash{}{0pt}%
\pgfpathmoveto{\pgfqpoint{2.487140in}{2.851459in}}%
\pgfpathlineto{\pgfqpoint{2.500790in}{2.829938in}}%
\pgfpathlineto{\pgfqpoint{2.514431in}{2.808614in}}%
\pgfpathlineto{\pgfqpoint{2.528063in}{2.787486in}}%
\pgfpathlineto{\pgfqpoint{2.541686in}{2.766552in}}%
\pgfpathlineto{\pgfqpoint{2.550360in}{2.763138in}}%
\pgfpathlineto{\pgfqpoint{2.559017in}{2.759966in}}%
\pgfpathlineto{\pgfqpoint{2.567657in}{2.757030in}}%
\pgfpathlineto{\pgfqpoint{2.576280in}{2.754329in}}%
\pgfpathlineto{\pgfqpoint{2.562703in}{2.774846in}}%
\pgfpathlineto{\pgfqpoint{2.549118in}{2.795557in}}%
\pgfpathlineto{\pgfqpoint{2.535524in}{2.816463in}}%
\pgfpathlineto{\pgfqpoint{2.521922in}{2.837566in}}%
\pgfpathlineto{\pgfqpoint{2.513253in}{2.840677in}}%
\pgfpathlineto{\pgfqpoint{2.504566in}{2.844027in}}%
\pgfpathlineto{\pgfqpoint{2.495862in}{2.847620in}}%
\pgfpathlineto{\pgfqpoint{2.487140in}{2.851459in}}%
\pgfpathclose%
\pgfusepath{fill}%
\end{pgfscope}%
\begin{pgfscope}%
\pgfpathrectangle{\pgfqpoint{1.254980in}{0.150000in}}{\pgfqpoint{5.490039in}{5.490039in}}%
\pgfusepath{clip}%
\pgfsetbuttcap%
\pgfsetroundjoin%
\definecolor{currentfill}{rgb}{0.283229,0.120777,0.440584}%
\pgfsetfillcolor{currentfill}%
\pgfsetfillopacity{0.700000}%
\pgfsetlinewidth{0.000000pt}%
\definecolor{currentstroke}{rgb}{0.000000,0.000000,0.000000}%
\pgfsetstrokecolor{currentstroke}%
\pgfsetdash{}{0pt}%
\pgfpathmoveto{\pgfqpoint{3.349435in}{1.825492in}}%
\pgfpathlineto{\pgfqpoint{3.362829in}{1.814495in}}%
\pgfpathlineto{\pgfqpoint{3.376224in}{1.803636in}}%
\pgfpathlineto{\pgfqpoint{3.389618in}{1.792916in}}%
\pgfpathlineto{\pgfqpoint{3.403014in}{1.782333in}}%
\pgfpathlineto{\pgfqpoint{3.411103in}{1.785353in}}%
\pgfpathlineto{\pgfqpoint{3.419182in}{1.788540in}}%
\pgfpathlineto{\pgfqpoint{3.427251in}{1.791889in}}%
\pgfpathlineto{\pgfqpoint{3.435310in}{1.795399in}}%
\pgfpathlineto{\pgfqpoint{3.421941in}{1.805601in}}%
\pgfpathlineto{\pgfqpoint{3.408573in}{1.815941in}}%
\pgfpathlineto{\pgfqpoint{3.395206in}{1.826418in}}%
\pgfpathlineto{\pgfqpoint{3.381840in}{1.837034in}}%
\pgfpathlineto{\pgfqpoint{3.373754in}{1.833899in}}%
\pgfpathlineto{\pgfqpoint{3.365658in}{1.830928in}}%
\pgfpathlineto{\pgfqpoint{3.357552in}{1.828125in}}%
\pgfpathlineto{\pgfqpoint{3.349435in}{1.825492in}}%
\pgfpathclose%
\pgfusepath{fill}%
\end{pgfscope}%
\begin{pgfscope}%
\pgfpathrectangle{\pgfqpoint{1.254980in}{0.150000in}}{\pgfqpoint{5.490039in}{5.490039in}}%
\pgfusepath{clip}%
\pgfsetbuttcap%
\pgfsetroundjoin%
\definecolor{currentfill}{rgb}{0.274128,0.199721,0.498911}%
\pgfsetfillcolor{currentfill}%
\pgfsetfillopacity{0.700000}%
\pgfsetlinewidth{0.000000pt}%
\definecolor{currentstroke}{rgb}{0.000000,0.000000,0.000000}%
\pgfsetstrokecolor{currentstroke}%
\pgfsetdash{}{0pt}%
\pgfpathmoveto{\pgfqpoint{4.829502in}{1.939806in}}%
\pgfpathlineto{\pgfqpoint{4.843223in}{1.942240in}}%
\pgfpathlineto{\pgfqpoint{4.856956in}{1.944788in}}%
\pgfpathlineto{\pgfqpoint{4.870700in}{1.947450in}}%
\pgfpathlineto{\pgfqpoint{4.884456in}{1.950224in}}%
\pgfpathlineto{\pgfqpoint{4.891985in}{1.961999in}}%
\pgfpathlineto{\pgfqpoint{4.899509in}{1.973747in}}%
\pgfpathlineto{\pgfqpoint{4.907028in}{1.985467in}}%
\pgfpathlineto{\pgfqpoint{4.914542in}{1.997159in}}%
\pgfpathlineto{\pgfqpoint{4.900789in}{1.994217in}}%
\pgfpathlineto{\pgfqpoint{4.887048in}{1.991388in}}%
\pgfpathlineto{\pgfqpoint{4.873319in}{1.988672in}}%
\pgfpathlineto{\pgfqpoint{4.859601in}{1.986071in}}%
\pgfpathlineto{\pgfqpoint{4.852084in}{1.974540in}}%
\pgfpathlineto{\pgfqpoint{4.844562in}{1.962985in}}%
\pgfpathlineto{\pgfqpoint{4.837034in}{1.951407in}}%
\pgfpathlineto{\pgfqpoint{4.829502in}{1.939806in}}%
\pgfpathclose%
\pgfusepath{fill}%
\end{pgfscope}%
\begin{pgfscope}%
\pgfpathrectangle{\pgfqpoint{1.254980in}{0.150000in}}{\pgfqpoint{5.490039in}{5.490039in}}%
\pgfusepath{clip}%
\pgfsetbuttcap%
\pgfsetroundjoin%
\definecolor{currentfill}{rgb}{0.273809,0.031497,0.358853}%
\pgfsetfillcolor{currentfill}%
\pgfsetfillopacity{0.700000}%
\pgfsetlinewidth{0.000000pt}%
\definecolor{currentstroke}{rgb}{0.000000,0.000000,0.000000}%
\pgfsetstrokecolor{currentstroke}%
\pgfsetdash{}{0pt}%
\pgfpathmoveto{\pgfqpoint{4.235652in}{1.635351in}}%
\pgfpathlineto{\pgfqpoint{4.249149in}{1.632948in}}%
\pgfpathlineto{\pgfqpoint{4.262653in}{1.630663in}}%
\pgfpathlineto{\pgfqpoint{4.276164in}{1.628494in}}%
\pgfpathlineto{\pgfqpoint{4.289684in}{1.626443in}}%
\pgfpathlineto{\pgfqpoint{4.297382in}{1.636344in}}%
\pgfpathlineto{\pgfqpoint{4.305075in}{1.646293in}}%
\pgfpathlineto{\pgfqpoint{4.312763in}{1.656289in}}%
\pgfpathlineto{\pgfqpoint{4.320446in}{1.666329in}}%
\pgfpathlineto{\pgfqpoint{4.306936in}{1.668103in}}%
\pgfpathlineto{\pgfqpoint{4.293433in}{1.669994in}}%
\pgfpathlineto{\pgfqpoint{4.279938in}{1.672003in}}%
\pgfpathlineto{\pgfqpoint{4.266451in}{1.674129in}}%
\pgfpathlineto{\pgfqpoint{4.258759in}{1.664359in}}%
\pgfpathlineto{\pgfqpoint{4.251062in}{1.654639in}}%
\pgfpathlineto{\pgfqpoint{4.243359in}{1.644969in}}%
\pgfpathlineto{\pgfqpoint{4.235652in}{1.635351in}}%
\pgfpathclose%
\pgfusepath{fill}%
\end{pgfscope}%
\begin{pgfscope}%
\pgfpathrectangle{\pgfqpoint{1.254980in}{0.150000in}}{\pgfqpoint{5.490039in}{5.490039in}}%
\pgfusepath{clip}%
\pgfsetbuttcap%
\pgfsetroundjoin%
\definecolor{currentfill}{rgb}{0.223925,0.334994,0.548053}%
\pgfsetfillcolor{currentfill}%
\pgfsetfillopacity{0.700000}%
\pgfsetlinewidth{0.000000pt}%
\definecolor{currentstroke}{rgb}{0.000000,0.000000,0.000000}%
\pgfsetstrokecolor{currentstroke}%
\pgfsetdash{}{0pt}%
\pgfpathmoveto{\pgfqpoint{5.199937in}{2.229779in}}%
\pgfpathlineto{\pgfqpoint{5.213844in}{2.234701in}}%
\pgfpathlineto{\pgfqpoint{5.227765in}{2.239735in}}%
\pgfpathlineto{\pgfqpoint{5.241699in}{2.244883in}}%
\pgfpathlineto{\pgfqpoint{5.255648in}{2.250143in}}%
\pgfpathlineto{\pgfqpoint{5.263060in}{2.261579in}}%
\pgfpathlineto{\pgfqpoint{5.270467in}{2.272954in}}%
\pgfpathlineto{\pgfqpoint{5.277869in}{2.284270in}}%
\pgfpathlineto{\pgfqpoint{5.285264in}{2.295525in}}%
\pgfpathlineto{\pgfqpoint{5.271319in}{2.290178in}}%
\pgfpathlineto{\pgfqpoint{5.257387in}{2.284943in}}%
\pgfpathlineto{\pgfqpoint{5.243470in}{2.279822in}}%
\pgfpathlineto{\pgfqpoint{5.229566in}{2.274813in}}%
\pgfpathlineto{\pgfqpoint{5.222167in}{2.263639in}}%
\pgfpathlineto{\pgfqpoint{5.214763in}{2.252408in}}%
\pgfpathlineto{\pgfqpoint{5.207353in}{2.241121in}}%
\pgfpathlineto{\pgfqpoint{5.199937in}{2.229779in}}%
\pgfpathclose%
\pgfusepath{fill}%
\end{pgfscope}%
\begin{pgfscope}%
\pgfpathrectangle{\pgfqpoint{1.254980in}{0.150000in}}{\pgfqpoint{5.490039in}{5.490039in}}%
\pgfusepath{clip}%
\pgfsetbuttcap%
\pgfsetroundjoin%
\definecolor{currentfill}{rgb}{0.277018,0.050344,0.375715}%
\pgfsetfillcolor{currentfill}%
\pgfsetfillopacity{0.700000}%
\pgfsetlinewidth{0.000000pt}%
\definecolor{currentstroke}{rgb}{0.000000,0.000000,0.000000}%
\pgfsetstrokecolor{currentstroke}%
\pgfsetdash{}{0pt}%
\pgfpathmoveto{\pgfqpoint{3.595856in}{1.683480in}}%
\pgfpathlineto{\pgfqpoint{3.609249in}{1.675013in}}%
\pgfpathlineto{\pgfqpoint{3.622643in}{1.666677in}}%
\pgfpathlineto{\pgfqpoint{3.636041in}{1.658470in}}%
\pgfpathlineto{\pgfqpoint{3.649441in}{1.650393in}}%
\pgfpathlineto{\pgfqpoint{3.657397in}{1.655530in}}%
\pgfpathlineto{\pgfqpoint{3.665345in}{1.660804in}}%
\pgfpathlineto{\pgfqpoint{3.673285in}{1.666212in}}%
\pgfpathlineto{\pgfqpoint{3.681217in}{1.671751in}}%
\pgfpathlineto{\pgfqpoint{3.667837in}{1.679468in}}%
\pgfpathlineto{\pgfqpoint{3.654461in}{1.687315in}}%
\pgfpathlineto{\pgfqpoint{3.641088in}{1.695291in}}%
\pgfpathlineto{\pgfqpoint{3.627717in}{1.703397in}}%
\pgfpathlineto{\pgfqpoint{3.619765in}{1.698212in}}%
\pgfpathlineto{\pgfqpoint{3.611803in}{1.693162in}}%
\pgfpathlineto{\pgfqpoint{3.603834in}{1.688250in}}%
\pgfpathlineto{\pgfqpoint{3.595856in}{1.683480in}}%
\pgfpathclose%
\pgfusepath{fill}%
\end{pgfscope}%
\begin{pgfscope}%
\pgfpathrectangle{\pgfqpoint{1.254980in}{0.150000in}}{\pgfqpoint{5.490039in}{5.490039in}}%
\pgfusepath{clip}%
\pgfsetbuttcap%
\pgfsetroundjoin%
\definecolor{currentfill}{rgb}{0.179019,0.433756,0.557430}%
\pgfsetfillcolor{currentfill}%
\pgfsetfillopacity{0.700000}%
\pgfsetlinewidth{0.000000pt}%
\definecolor{currentstroke}{rgb}{0.000000,0.000000,0.000000}%
\pgfsetstrokecolor{currentstroke}%
\pgfsetdash{}{0pt}%
\pgfpathmoveto{\pgfqpoint{5.485524in}{2.473888in}}%
\pgfpathlineto{\pgfqpoint{5.499589in}{2.480412in}}%
\pgfpathlineto{\pgfqpoint{5.513668in}{2.487048in}}%
\pgfpathlineto{\pgfqpoint{5.527763in}{2.493797in}}%
\pgfpathlineto{\pgfqpoint{5.541874in}{2.500659in}}%
\pgfpathlineto{\pgfqpoint{5.549176in}{2.511184in}}%
\pgfpathlineto{\pgfqpoint{5.556471in}{2.521634in}}%
\pgfpathlineto{\pgfqpoint{5.563759in}{2.532009in}}%
\pgfpathlineto{\pgfqpoint{5.571041in}{2.542309in}}%
\pgfpathlineto{\pgfqpoint{5.556936in}{2.535427in}}%
\pgfpathlineto{\pgfqpoint{5.542846in}{2.528658in}}%
\pgfpathlineto{\pgfqpoint{5.528772in}{2.522002in}}%
\pgfpathlineto{\pgfqpoint{5.514712in}{2.515458in}}%
\pgfpathlineto{\pgfqpoint{5.507425in}{2.505172in}}%
\pgfpathlineto{\pgfqpoint{5.500131in}{2.494815in}}%
\pgfpathlineto{\pgfqpoint{5.492831in}{2.484387in}}%
\pgfpathlineto{\pgfqpoint{5.485524in}{2.473888in}}%
\pgfpathclose%
\pgfusepath{fill}%
\end{pgfscope}%
\begin{pgfscope}%
\pgfpathrectangle{\pgfqpoint{1.254980in}{0.150000in}}{\pgfqpoint{5.490039in}{5.490039in}}%
\pgfusepath{clip}%
\pgfsetbuttcap%
\pgfsetroundjoin%
\definecolor{currentfill}{rgb}{0.265145,0.232956,0.516599}%
\pgfsetfillcolor{currentfill}%
\pgfsetfillopacity{0.700000}%
\pgfsetlinewidth{0.000000pt}%
\definecolor{currentstroke}{rgb}{0.000000,0.000000,0.000000}%
\pgfsetstrokecolor{currentstroke}%
\pgfsetdash{}{0pt}%
\pgfpathmoveto{\pgfqpoint{4.914542in}{1.997159in}}%
\pgfpathlineto{\pgfqpoint{4.928307in}{2.000214in}}%
\pgfpathlineto{\pgfqpoint{4.942084in}{2.003383in}}%
\pgfpathlineto{\pgfqpoint{4.955873in}{2.006665in}}%
\pgfpathlineto{\pgfqpoint{4.969675in}{2.010060in}}%
\pgfpathlineto{\pgfqpoint{4.977181in}{2.021880in}}%
\pgfpathlineto{\pgfqpoint{4.984682in}{2.033665in}}%
\pgfpathlineto{\pgfqpoint{4.992179in}{2.045413in}}%
\pgfpathlineto{\pgfqpoint{4.999670in}{2.057125in}}%
\pgfpathlineto{\pgfqpoint{4.985872in}{2.053579in}}%
\pgfpathlineto{\pgfqpoint{4.972085in}{2.050145in}}%
\pgfpathlineto{\pgfqpoint{4.958311in}{2.046825in}}%
\pgfpathlineto{\pgfqpoint{4.944550in}{2.043618in}}%
\pgfpathlineto{\pgfqpoint{4.937055in}{2.032051in}}%
\pgfpathlineto{\pgfqpoint{4.929556in}{2.020452in}}%
\pgfpathlineto{\pgfqpoint{4.922052in}{2.008821in}}%
\pgfpathlineto{\pgfqpoint{4.914542in}{1.997159in}}%
\pgfpathclose%
\pgfusepath{fill}%
\end{pgfscope}%
\begin{pgfscope}%
\pgfpathrectangle{\pgfqpoint{1.254980in}{0.150000in}}{\pgfqpoint{5.490039in}{5.490039in}}%
\pgfusepath{clip}%
\pgfsetbuttcap%
\pgfsetroundjoin%
\definecolor{currentfill}{rgb}{0.271305,0.019942,0.347269}%
\pgfsetfillcolor{currentfill}%
\pgfsetfillopacity{0.700000}%
\pgfsetlinewidth{0.000000pt}%
\definecolor{currentstroke}{rgb}{0.000000,0.000000,0.000000}%
\pgfsetstrokecolor{currentstroke}%
\pgfsetdash{}{0pt}%
\pgfpathmoveto{\pgfqpoint{4.150817in}{1.609417in}}%
\pgfpathlineto{\pgfqpoint{4.164295in}{1.606249in}}%
\pgfpathlineto{\pgfqpoint{4.177780in}{1.603199in}}%
\pgfpathlineto{\pgfqpoint{4.191272in}{1.600268in}}%
\pgfpathlineto{\pgfqpoint{4.204771in}{1.597454in}}%
\pgfpathlineto{\pgfqpoint{4.212499in}{1.606838in}}%
\pgfpathlineto{\pgfqpoint{4.220222in}{1.616284in}}%
\pgfpathlineto{\pgfqpoint{4.227940in}{1.625789in}}%
\pgfpathlineto{\pgfqpoint{4.235652in}{1.635351in}}%
\pgfpathlineto{\pgfqpoint{4.222163in}{1.637872in}}%
\pgfpathlineto{\pgfqpoint{4.208681in}{1.640511in}}%
\pgfpathlineto{\pgfqpoint{4.195207in}{1.643267in}}%
\pgfpathlineto{\pgfqpoint{4.181739in}{1.646143in}}%
\pgfpathlineto{\pgfqpoint{4.174017in}{1.636867in}}%
\pgfpathlineto{\pgfqpoint{4.166289in}{1.627653in}}%
\pgfpathlineto{\pgfqpoint{4.158555in}{1.618502in}}%
\pgfpathlineto{\pgfqpoint{4.150817in}{1.609417in}}%
\pgfpathclose%
\pgfusepath{fill}%
\end{pgfscope}%
\begin{pgfscope}%
\pgfpathrectangle{\pgfqpoint{1.254980in}{0.150000in}}{\pgfqpoint{5.490039in}{5.490039in}}%
\pgfusepath{clip}%
\pgfsetbuttcap%
\pgfsetroundjoin%
\definecolor{currentfill}{rgb}{0.282910,0.105393,0.426902}%
\pgfsetfillcolor{currentfill}%
\pgfsetfillopacity{0.700000}%
\pgfsetlinewidth{0.000000pt}%
\definecolor{currentstroke}{rgb}{0.000000,0.000000,0.000000}%
\pgfsetstrokecolor{currentstroke}%
\pgfsetdash{}{0pt}%
\pgfpathmoveto{\pgfqpoint{3.403014in}{1.782333in}}%
\pgfpathlineto{\pgfqpoint{3.416410in}{1.771887in}}%
\pgfpathlineto{\pgfqpoint{3.429808in}{1.761577in}}%
\pgfpathlineto{\pgfqpoint{3.443206in}{1.751403in}}%
\pgfpathlineto{\pgfqpoint{3.456606in}{1.741365in}}%
\pgfpathlineto{\pgfqpoint{3.464668in}{1.744772in}}%
\pgfpathlineto{\pgfqpoint{3.472721in}{1.748340in}}%
\pgfpathlineto{\pgfqpoint{3.480764in}{1.752068in}}%
\pgfpathlineto{\pgfqpoint{3.488798in}{1.755951in}}%
\pgfpathlineto{\pgfqpoint{3.475424in}{1.765610in}}%
\pgfpathlineto{\pgfqpoint{3.462051in}{1.775403in}}%
\pgfpathlineto{\pgfqpoint{3.448680in}{1.785333in}}%
\pgfpathlineto{\pgfqpoint{3.435310in}{1.795399in}}%
\pgfpathlineto{\pgfqpoint{3.427251in}{1.791889in}}%
\pgfpathlineto{\pgfqpoint{3.419182in}{1.788540in}}%
\pgfpathlineto{\pgfqpoint{3.411103in}{1.785353in}}%
\pgfpathlineto{\pgfqpoint{3.403014in}{1.782333in}}%
\pgfpathclose%
\pgfusepath{fill}%
\end{pgfscope}%
\begin{pgfscope}%
\pgfpathrectangle{\pgfqpoint{1.254980in}{0.150000in}}{\pgfqpoint{5.490039in}{5.490039in}}%
\pgfusepath{clip}%
\pgfsetbuttcap%
\pgfsetroundjoin%
\definecolor{currentfill}{rgb}{0.124395,0.578002,0.548287}%
\pgfsetfillcolor{currentfill}%
\pgfsetfillopacity{0.700000}%
\pgfsetlinewidth{0.000000pt}%
\definecolor{currentstroke}{rgb}{0.000000,0.000000,0.000000}%
\pgfsetstrokecolor{currentstroke}%
\pgfsetdash{}{0pt}%
\pgfpathmoveto{\pgfqpoint{2.432447in}{2.939544in}}%
\pgfpathlineto{\pgfqpoint{2.446135in}{2.917220in}}%
\pgfpathlineto{\pgfqpoint{2.459813in}{2.895099in}}%
\pgfpathlineto{\pgfqpoint{2.473481in}{2.873179in}}%
\pgfpathlineto{\pgfqpoint{2.487140in}{2.851459in}}%
\pgfpathlineto{\pgfqpoint{2.495862in}{2.847620in}}%
\pgfpathlineto{\pgfqpoint{2.504566in}{2.844027in}}%
\pgfpathlineto{\pgfqpoint{2.513253in}{2.840677in}}%
\pgfpathlineto{\pgfqpoint{2.521922in}{2.837566in}}%
\pgfpathlineto{\pgfqpoint{2.508311in}{2.858866in}}%
\pgfpathlineto{\pgfqpoint{2.494691in}{2.880364in}}%
\pgfpathlineto{\pgfqpoint{2.481061in}{2.902064in}}%
\pgfpathlineto{\pgfqpoint{2.467422in}{2.923965in}}%
\pgfpathlineto{\pgfqpoint{2.458706in}{2.927490in}}%
\pgfpathlineto{\pgfqpoint{2.449971in}{2.931259in}}%
\pgfpathlineto{\pgfqpoint{2.441218in}{2.935276in}}%
\pgfpathlineto{\pgfqpoint{2.432447in}{2.939544in}}%
\pgfpathclose%
\pgfusepath{fill}%
\end{pgfscope}%
\begin{pgfscope}%
\pgfpathrectangle{\pgfqpoint{1.254980in}{0.150000in}}{\pgfqpoint{5.490039in}{5.490039in}}%
\pgfusepath{clip}%
\pgfsetbuttcap%
\pgfsetroundjoin%
\definecolor{currentfill}{rgb}{0.271305,0.019942,0.347269}%
\pgfsetfillcolor{currentfill}%
\pgfsetfillopacity{0.700000}%
\pgfsetlinewidth{0.000000pt}%
\definecolor{currentstroke}{rgb}{0.000000,0.000000,0.000000}%
\pgfsetstrokecolor{currentstroke}%
\pgfsetdash{}{0pt}%
\pgfpathmoveto{\pgfqpoint{3.788374in}{1.614603in}}%
\pgfpathlineto{\pgfqpoint{3.801786in}{1.608027in}}%
\pgfpathlineto{\pgfqpoint{3.815202in}{1.601577in}}%
\pgfpathlineto{\pgfqpoint{3.828622in}{1.595251in}}%
\pgfpathlineto{\pgfqpoint{3.842046in}{1.589049in}}%
\pgfpathlineto{\pgfqpoint{3.849914in}{1.595761in}}%
\pgfpathlineto{\pgfqpoint{3.857775in}{1.602586in}}%
\pgfpathlineto{\pgfqpoint{3.865629in}{1.609521in}}%
\pgfpathlineto{\pgfqpoint{3.873477in}{1.616563in}}%
\pgfpathlineto{\pgfqpoint{3.860069in}{1.622423in}}%
\pgfpathlineto{\pgfqpoint{3.846667in}{1.628407in}}%
\pgfpathlineto{\pgfqpoint{3.833268in}{1.634516in}}%
\pgfpathlineto{\pgfqpoint{3.819874in}{1.640749in}}%
\pgfpathlineto{\pgfqpoint{3.812010in}{1.634043in}}%
\pgfpathlineto{\pgfqpoint{3.804138in}{1.627448in}}%
\pgfpathlineto{\pgfqpoint{3.796260in}{1.620967in}}%
\pgfpathlineto{\pgfqpoint{3.788374in}{1.614603in}}%
\pgfpathclose%
\pgfusepath{fill}%
\end{pgfscope}%
\begin{pgfscope}%
\pgfpathrectangle{\pgfqpoint{1.254980in}{0.150000in}}{\pgfqpoint{5.490039in}{5.490039in}}%
\pgfusepath{clip}%
\pgfsetbuttcap%
\pgfsetroundjoin%
\definecolor{currentfill}{rgb}{0.229739,0.322361,0.545706}%
\pgfsetfillcolor{currentfill}%
\pgfsetfillopacity{0.700000}%
\pgfsetlinewidth{0.000000pt}%
\definecolor{currentstroke}{rgb}{0.000000,0.000000,0.000000}%
\pgfsetstrokecolor{currentstroke}%
\pgfsetdash{}{0pt}%
\pgfpathmoveto{\pgfqpoint{2.886670in}{2.258801in}}%
\pgfpathlineto{\pgfqpoint{2.900155in}{2.242593in}}%
\pgfpathlineto{\pgfqpoint{2.913636in}{2.226547in}}%
\pgfpathlineto{\pgfqpoint{2.927113in}{2.210663in}}%
\pgfpathlineto{\pgfqpoint{2.940586in}{2.194939in}}%
\pgfpathlineto{\pgfqpoint{2.948983in}{2.193972in}}%
\pgfpathlineto{\pgfqpoint{2.957365in}{2.193221in}}%
\pgfpathlineto{\pgfqpoint{2.965734in}{2.192682in}}%
\pgfpathlineto{\pgfqpoint{2.974090in}{2.192352in}}%
\pgfpathlineto{\pgfqpoint{2.960654in}{2.207662in}}%
\pgfpathlineto{\pgfqpoint{2.947215in}{2.223133in}}%
\pgfpathlineto{\pgfqpoint{2.933773in}{2.238766in}}%
\pgfpathlineto{\pgfqpoint{2.920327in}{2.254560in}}%
\pgfpathlineto{\pgfqpoint{2.911934in}{2.255296in}}%
\pgfpathlineto{\pgfqpoint{2.903527in}{2.256246in}}%
\pgfpathlineto{\pgfqpoint{2.895105in}{2.257414in}}%
\pgfpathlineto{\pgfqpoint{2.886670in}{2.258801in}}%
\pgfpathclose%
\pgfusepath{fill}%
\end{pgfscope}%
\begin{pgfscope}%
\pgfpathrectangle{\pgfqpoint{1.254980in}{0.150000in}}{\pgfqpoint{5.490039in}{5.490039in}}%
\pgfusepath{clip}%
\pgfsetbuttcap%
\pgfsetroundjoin%
\definecolor{currentfill}{rgb}{0.241237,0.296485,0.539709}%
\pgfsetfillcolor{currentfill}%
\pgfsetfillopacity{0.700000}%
\pgfsetlinewidth{0.000000pt}%
\definecolor{currentstroke}{rgb}{0.000000,0.000000,0.000000}%
\pgfsetstrokecolor{currentstroke}%
\pgfsetdash{}{0pt}%
\pgfpathmoveto{\pgfqpoint{2.940586in}{2.194939in}}%
\pgfpathlineto{\pgfqpoint{2.954055in}{2.179376in}}%
\pgfpathlineto{\pgfqpoint{2.967521in}{2.163971in}}%
\pgfpathlineto{\pgfqpoint{2.980984in}{2.148725in}}%
\pgfpathlineto{\pgfqpoint{2.994444in}{2.133636in}}%
\pgfpathlineto{\pgfqpoint{3.002803in}{2.133087in}}%
\pgfpathlineto{\pgfqpoint{3.011148in}{2.132750in}}%
\pgfpathlineto{\pgfqpoint{3.019481in}{2.132619in}}%
\pgfpathlineto{\pgfqpoint{3.027800in}{2.132694in}}%
\pgfpathlineto{\pgfqpoint{3.014377in}{2.147372in}}%
\pgfpathlineto{\pgfqpoint{3.000951in}{2.162207in}}%
\pgfpathlineto{\pgfqpoint{2.987522in}{2.177200in}}%
\pgfpathlineto{\pgfqpoint{2.974090in}{2.192352in}}%
\pgfpathlineto{\pgfqpoint{2.965734in}{2.192682in}}%
\pgfpathlineto{\pgfqpoint{2.957365in}{2.193221in}}%
\pgfpathlineto{\pgfqpoint{2.948983in}{2.193972in}}%
\pgfpathlineto{\pgfqpoint{2.940586in}{2.194939in}}%
\pgfpathclose%
\pgfusepath{fill}%
\end{pgfscope}%
\begin{pgfscope}%
\pgfpathrectangle{\pgfqpoint{1.254980in}{0.150000in}}{\pgfqpoint{5.490039in}{5.490039in}}%
\pgfusepath{clip}%
\pgfsetbuttcap%
\pgfsetroundjoin%
\definecolor{currentfill}{rgb}{0.268510,0.009605,0.335427}%
\pgfsetfillcolor{currentfill}%
\pgfsetfillopacity{0.700000}%
\pgfsetlinewidth{0.000000pt}%
\definecolor{currentstroke}{rgb}{0.000000,0.000000,0.000000}%
\pgfsetstrokecolor{currentstroke}%
\pgfsetdash{}{0pt}%
\pgfpathmoveto{\pgfqpoint{3.927155in}{1.594356in}}%
\pgfpathlineto{\pgfqpoint{3.940587in}{1.589110in}}%
\pgfpathlineto{\pgfqpoint{3.954025in}{1.583987in}}%
\pgfpathlineto{\pgfqpoint{3.967467in}{1.578986in}}%
\pgfpathlineto{\pgfqpoint{3.980915in}{1.574105in}}%
\pgfpathlineto{\pgfqpoint{3.988725in}{1.581915in}}%
\pgfpathlineto{\pgfqpoint{3.996529in}{1.589817in}}%
\pgfpathlineto{\pgfqpoint{4.004327in}{1.597811in}}%
\pgfpathlineto{\pgfqpoint{4.012119in}{1.605894in}}%
\pgfpathlineto{\pgfqpoint{3.998685in}{1.610449in}}%
\pgfpathlineto{\pgfqpoint{3.985257in}{1.615125in}}%
\pgfpathlineto{\pgfqpoint{3.971834in}{1.619923in}}%
\pgfpathlineto{\pgfqpoint{3.958417in}{1.624843in}}%
\pgfpathlineto{\pgfqpoint{3.950611in}{1.617080in}}%
\pgfpathlineto{\pgfqpoint{3.942799in}{1.609409in}}%
\pgfpathlineto{\pgfqpoint{3.934980in}{1.601834in}}%
\pgfpathlineto{\pgfqpoint{3.927155in}{1.594356in}}%
\pgfpathclose%
\pgfusepath{fill}%
\end{pgfscope}%
\begin{pgfscope}%
\pgfpathrectangle{\pgfqpoint{1.254980in}{0.150000in}}{\pgfqpoint{5.490039in}{5.490039in}}%
\pgfusepath{clip}%
\pgfsetbuttcap%
\pgfsetroundjoin%
\definecolor{currentfill}{rgb}{0.208623,0.367752,0.552675}%
\pgfsetfillcolor{currentfill}%
\pgfsetfillopacity{0.700000}%
\pgfsetlinewidth{0.000000pt}%
\definecolor{currentstroke}{rgb}{0.000000,0.000000,0.000000}%
\pgfsetstrokecolor{currentstroke}%
\pgfsetdash{}{0pt}%
\pgfpathmoveto{\pgfqpoint{5.285264in}{2.295525in}}%
\pgfpathlineto{\pgfqpoint{5.299223in}{2.300985in}}%
\pgfpathlineto{\pgfqpoint{5.313197in}{2.306558in}}%
\pgfpathlineto{\pgfqpoint{5.327185in}{2.312244in}}%
\pgfpathlineto{\pgfqpoint{5.341187in}{2.318042in}}%
\pgfpathlineto{\pgfqpoint{5.348573in}{2.329313in}}%
\pgfpathlineto{\pgfqpoint{5.355953in}{2.340519in}}%
\pgfpathlineto{\pgfqpoint{5.363327in}{2.351659in}}%
\pgfpathlineto{\pgfqpoint{5.370695in}{2.362732in}}%
\pgfpathlineto{\pgfqpoint{5.356696in}{2.356863in}}%
\pgfpathlineto{\pgfqpoint{5.342712in}{2.351107in}}%
\pgfpathlineto{\pgfqpoint{5.328742in}{2.345464in}}%
\pgfpathlineto{\pgfqpoint{5.314786in}{2.339934in}}%
\pgfpathlineto{\pgfqpoint{5.307415in}{2.328924in}}%
\pgfpathlineto{\pgfqpoint{5.300037in}{2.317853in}}%
\pgfpathlineto{\pgfqpoint{5.292653in}{2.306720in}}%
\pgfpathlineto{\pgfqpoint{5.285264in}{2.295525in}}%
\pgfpathclose%
\pgfusepath{fill}%
\end{pgfscope}%
\begin{pgfscope}%
\pgfpathrectangle{\pgfqpoint{1.254980in}{0.150000in}}{\pgfqpoint{5.490039in}{5.490039in}}%
\pgfusepath{clip}%
\pgfsetbuttcap%
\pgfsetroundjoin%
\definecolor{currentfill}{rgb}{0.216210,0.351535,0.550627}%
\pgfsetfillcolor{currentfill}%
\pgfsetfillopacity{0.700000}%
\pgfsetlinewidth{0.000000pt}%
\definecolor{currentstroke}{rgb}{0.000000,0.000000,0.000000}%
\pgfsetstrokecolor{currentstroke}%
\pgfsetdash{}{0pt}%
\pgfpathmoveto{\pgfqpoint{2.832686in}{2.325281in}}%
\pgfpathlineto{\pgfqpoint{2.846189in}{2.308412in}}%
\pgfpathlineto{\pgfqpoint{2.859687in}{2.291710in}}%
\pgfpathlineto{\pgfqpoint{2.873181in}{2.275174in}}%
\pgfpathlineto{\pgfqpoint{2.886670in}{2.258801in}}%
\pgfpathlineto{\pgfqpoint{2.895105in}{2.257414in}}%
\pgfpathlineto{\pgfqpoint{2.903527in}{2.256246in}}%
\pgfpathlineto{\pgfqpoint{2.911934in}{2.255296in}}%
\pgfpathlineto{\pgfqpoint{2.920327in}{2.254560in}}%
\pgfpathlineto{\pgfqpoint{2.906876in}{2.270517in}}%
\pgfpathlineto{\pgfqpoint{2.893422in}{2.286638in}}%
\pgfpathlineto{\pgfqpoint{2.879964in}{2.302924in}}%
\pgfpathlineto{\pgfqpoint{2.866501in}{2.319375in}}%
\pgfpathlineto{\pgfqpoint{2.858069in}{2.320520in}}%
\pgfpathlineto{\pgfqpoint{2.849623in}{2.321884in}}%
\pgfpathlineto{\pgfqpoint{2.841162in}{2.323470in}}%
\pgfpathlineto{\pgfqpoint{2.832686in}{2.325281in}}%
\pgfpathclose%
\pgfusepath{fill}%
\end{pgfscope}%
\begin{pgfscope}%
\pgfpathrectangle{\pgfqpoint{1.254980in}{0.150000in}}{\pgfqpoint{5.490039in}{5.490039in}}%
\pgfusepath{clip}%
\pgfsetbuttcap%
\pgfsetroundjoin%
\definecolor{currentfill}{rgb}{0.252194,0.269783,0.531579}%
\pgfsetfillcolor{currentfill}%
\pgfsetfillopacity{0.700000}%
\pgfsetlinewidth{0.000000pt}%
\definecolor{currentstroke}{rgb}{0.000000,0.000000,0.000000}%
\pgfsetstrokecolor{currentstroke}%
\pgfsetdash{}{0pt}%
\pgfpathmoveto{\pgfqpoint{2.994444in}{2.133636in}}%
\pgfpathlineto{\pgfqpoint{3.007900in}{2.118704in}}%
\pgfpathlineto{\pgfqpoint{3.021354in}{2.103928in}}%
\pgfpathlineto{\pgfqpoint{3.034805in}{2.089307in}}%
\pgfpathlineto{\pgfqpoint{3.048253in}{2.074840in}}%
\pgfpathlineto{\pgfqpoint{3.056575in}{2.074706in}}%
\pgfpathlineto{\pgfqpoint{3.064885in}{2.074780in}}%
\pgfpathlineto{\pgfqpoint{3.073182in}{2.075056in}}%
\pgfpathlineto{\pgfqpoint{3.081466in}{2.075532in}}%
\pgfpathlineto{\pgfqpoint{3.068053in}{2.089591in}}%
\pgfpathlineto{\pgfqpoint{3.054638in}{2.103804in}}%
\pgfpathlineto{\pgfqpoint{3.041220in}{2.118171in}}%
\pgfpathlineto{\pgfqpoint{3.027800in}{2.132694in}}%
\pgfpathlineto{\pgfqpoint{3.019481in}{2.132619in}}%
\pgfpathlineto{\pgfqpoint{3.011148in}{2.132750in}}%
\pgfpathlineto{\pgfqpoint{3.002803in}{2.133087in}}%
\pgfpathlineto{\pgfqpoint{2.994444in}{2.133636in}}%
\pgfpathclose%
\pgfusepath{fill}%
\end{pgfscope}%
\begin{pgfscope}%
\pgfpathrectangle{\pgfqpoint{1.254980in}{0.150000in}}{\pgfqpoint{5.490039in}{5.490039in}}%
\pgfusepath{clip}%
\pgfsetbuttcap%
\pgfsetroundjoin%
\definecolor{currentfill}{rgb}{0.255645,0.260703,0.528312}%
\pgfsetfillcolor{currentfill}%
\pgfsetfillopacity{0.700000}%
\pgfsetlinewidth{0.000000pt}%
\definecolor{currentstroke}{rgb}{0.000000,0.000000,0.000000}%
\pgfsetstrokecolor{currentstroke}%
\pgfsetdash{}{0pt}%
\pgfpathmoveto{\pgfqpoint{4.999670in}{2.057125in}}%
\pgfpathlineto{\pgfqpoint{5.013481in}{2.060785in}}%
\pgfpathlineto{\pgfqpoint{5.027305in}{2.064558in}}%
\pgfpathlineto{\pgfqpoint{5.041141in}{2.068444in}}%
\pgfpathlineto{\pgfqpoint{5.054990in}{2.072443in}}%
\pgfpathlineto{\pgfqpoint{5.062474in}{2.084259in}}%
\pgfpathlineto{\pgfqpoint{5.069952in}{2.096032in}}%
\pgfpathlineto{\pgfqpoint{5.077425in}{2.107761in}}%
\pgfpathlineto{\pgfqpoint{5.084893in}{2.119445in}}%
\pgfpathlineto{\pgfqpoint{5.071047in}{2.115311in}}%
\pgfpathlineto{\pgfqpoint{5.057213in}{2.111289in}}%
\pgfpathlineto{\pgfqpoint{5.043392in}{2.107380in}}%
\pgfpathlineto{\pgfqpoint{5.029584in}{2.103585in}}%
\pgfpathlineto{\pgfqpoint{5.022113in}{2.092030in}}%
\pgfpathlineto{\pgfqpoint{5.014637in}{2.080435in}}%
\pgfpathlineto{\pgfqpoint{5.007156in}{2.068799in}}%
\pgfpathlineto{\pgfqpoint{4.999670in}{2.057125in}}%
\pgfpathclose%
\pgfusepath{fill}%
\end{pgfscope}%
\begin{pgfscope}%
\pgfpathrectangle{\pgfqpoint{1.254980in}{0.150000in}}{\pgfqpoint{5.490039in}{5.490039in}}%
\pgfusepath{clip}%
\pgfsetbuttcap%
\pgfsetroundjoin%
\definecolor{currentfill}{rgb}{0.166617,0.463708,0.558119}%
\pgfsetfillcolor{currentfill}%
\pgfsetfillopacity{0.700000}%
\pgfsetlinewidth{0.000000pt}%
\definecolor{currentstroke}{rgb}{0.000000,0.000000,0.000000}%
\pgfsetstrokecolor{currentstroke}%
\pgfsetdash{}{0pt}%
\pgfpathmoveto{\pgfqpoint{5.571041in}{2.542309in}}%
\pgfpathlineto{\pgfqpoint{5.585162in}{2.549303in}}%
\pgfpathlineto{\pgfqpoint{5.599299in}{2.556411in}}%
\pgfpathlineto{\pgfqpoint{5.613451in}{2.563631in}}%
\pgfpathlineto{\pgfqpoint{5.627619in}{2.570964in}}%
\pgfpathlineto{\pgfqpoint{5.634889in}{2.581198in}}%
\pgfpathlineto{\pgfqpoint{5.642151in}{2.591354in}}%
\pgfpathlineto{\pgfqpoint{5.649407in}{2.601431in}}%
\pgfpathlineto{\pgfqpoint{5.656655in}{2.611429in}}%
\pgfpathlineto{\pgfqpoint{5.642493in}{2.604093in}}%
\pgfpathlineto{\pgfqpoint{5.628346in}{2.596870in}}%
\pgfpathlineto{\pgfqpoint{5.614216in}{2.589759in}}%
\pgfpathlineto{\pgfqpoint{5.600101in}{2.582761in}}%
\pgfpathlineto{\pgfqpoint{5.592846in}{2.572760in}}%
\pgfpathlineto{\pgfqpoint{5.585585in}{2.562684in}}%
\pgfpathlineto{\pgfqpoint{5.578316in}{2.552534in}}%
\pgfpathlineto{\pgfqpoint{5.571041in}{2.542309in}}%
\pgfpathclose%
\pgfusepath{fill}%
\end{pgfscope}%
\begin{pgfscope}%
\pgfpathrectangle{\pgfqpoint{1.254980in}{0.150000in}}{\pgfqpoint{5.490039in}{5.490039in}}%
\pgfusepath{clip}%
\pgfsetbuttcap%
\pgfsetroundjoin%
\definecolor{currentfill}{rgb}{0.204903,0.375746,0.553533}%
\pgfsetfillcolor{currentfill}%
\pgfsetfillopacity{0.700000}%
\pgfsetlinewidth{0.000000pt}%
\definecolor{currentstroke}{rgb}{0.000000,0.000000,0.000000}%
\pgfsetstrokecolor{currentstroke}%
\pgfsetdash{}{0pt}%
\pgfpathmoveto{\pgfqpoint{2.778626in}{2.394436in}}%
\pgfpathlineto{\pgfqpoint{2.792149in}{2.376893in}}%
\pgfpathlineto{\pgfqpoint{2.805666in}{2.359520in}}%
\pgfpathlineto{\pgfqpoint{2.819179in}{2.342316in}}%
\pgfpathlineto{\pgfqpoint{2.832686in}{2.325281in}}%
\pgfpathlineto{\pgfqpoint{2.841162in}{2.323470in}}%
\pgfpathlineto{\pgfqpoint{2.849623in}{2.321884in}}%
\pgfpathlineto{\pgfqpoint{2.858069in}{2.320520in}}%
\pgfpathlineto{\pgfqpoint{2.866501in}{2.319375in}}%
\pgfpathlineto{\pgfqpoint{2.853034in}{2.335993in}}%
\pgfpathlineto{\pgfqpoint{2.839562in}{2.352778in}}%
\pgfpathlineto{\pgfqpoint{2.826085in}{2.369733in}}%
\pgfpathlineto{\pgfqpoint{2.812604in}{2.386856in}}%
\pgfpathlineto{\pgfqpoint{2.804132in}{2.388413in}}%
\pgfpathlineto{\pgfqpoint{2.795645in}{2.390193in}}%
\pgfpathlineto{\pgfqpoint{2.787143in}{2.392199in}}%
\pgfpathlineto{\pgfqpoint{2.778626in}{2.394436in}}%
\pgfpathclose%
\pgfusepath{fill}%
\end{pgfscope}%
\begin{pgfscope}%
\pgfpathrectangle{\pgfqpoint{1.254980in}{0.150000in}}{\pgfqpoint{5.490039in}{5.490039in}}%
\pgfusepath{clip}%
\pgfsetbuttcap%
\pgfsetroundjoin%
\definecolor{currentfill}{rgb}{0.262138,0.242286,0.520837}%
\pgfsetfillcolor{currentfill}%
\pgfsetfillopacity{0.700000}%
\pgfsetlinewidth{0.000000pt}%
\definecolor{currentstroke}{rgb}{0.000000,0.000000,0.000000}%
\pgfsetstrokecolor{currentstroke}%
\pgfsetdash{}{0pt}%
\pgfpathmoveto{\pgfqpoint{3.048253in}{2.074840in}}%
\pgfpathlineto{\pgfqpoint{3.061698in}{2.060526in}}%
\pgfpathlineto{\pgfqpoint{3.075142in}{2.046365in}}%
\pgfpathlineto{\pgfqpoint{3.088583in}{2.032355in}}%
\pgfpathlineto{\pgfqpoint{3.102022in}{2.018497in}}%
\pgfpathlineto{\pgfqpoint{3.110309in}{2.018778in}}%
\pgfpathlineto{\pgfqpoint{3.118584in}{2.019260in}}%
\pgfpathlineto{\pgfqpoint{3.126847in}{2.019941in}}%
\pgfpathlineto{\pgfqpoint{3.135097in}{2.020818in}}%
\pgfpathlineto{\pgfqpoint{3.121692in}{2.034270in}}%
\pgfpathlineto{\pgfqpoint{3.108285in}{2.047872in}}%
\pgfpathlineto{\pgfqpoint{3.094877in}{2.061626in}}%
\pgfpathlineto{\pgfqpoint{3.081466in}{2.075532in}}%
\pgfpathlineto{\pgfqpoint{3.073182in}{2.075056in}}%
\pgfpathlineto{\pgfqpoint{3.064885in}{2.074780in}}%
\pgfpathlineto{\pgfqpoint{3.056575in}{2.074706in}}%
\pgfpathlineto{\pgfqpoint{3.048253in}{2.074840in}}%
\pgfpathclose%
\pgfusepath{fill}%
\end{pgfscope}%
\begin{pgfscope}%
\pgfpathrectangle{\pgfqpoint{1.254980in}{0.150000in}}{\pgfqpoint{5.490039in}{5.490039in}}%
\pgfusepath{clip}%
\pgfsetbuttcap%
\pgfsetroundjoin%
\definecolor{currentfill}{rgb}{0.274952,0.037752,0.364543}%
\pgfsetfillcolor{currentfill}%
\pgfsetfillopacity{0.700000}%
\pgfsetlinewidth{0.000000pt}%
\definecolor{currentstroke}{rgb}{0.000000,0.000000,0.000000}%
\pgfsetstrokecolor{currentstroke}%
\pgfsetdash{}{0pt}%
\pgfpathmoveto{\pgfqpoint{3.649441in}{1.650393in}}%
\pgfpathlineto{\pgfqpoint{3.662845in}{1.642444in}}%
\pgfpathlineto{\pgfqpoint{3.676251in}{1.634624in}}%
\pgfpathlineto{\pgfqpoint{3.689660in}{1.626931in}}%
\pgfpathlineto{\pgfqpoint{3.703073in}{1.619366in}}%
\pgfpathlineto{\pgfqpoint{3.711008in}{1.624869in}}%
\pgfpathlineto{\pgfqpoint{3.718935in}{1.630505in}}%
\pgfpathlineto{\pgfqpoint{3.726855in}{1.636270in}}%
\pgfpathlineto{\pgfqpoint{3.734766in}{1.642162in}}%
\pgfpathlineto{\pgfqpoint{3.721374in}{1.649368in}}%
\pgfpathlineto{\pgfqpoint{3.707985in}{1.656701in}}%
\pgfpathlineto{\pgfqpoint{3.694599in}{1.664162in}}%
\pgfpathlineto{\pgfqpoint{3.681217in}{1.671751in}}%
\pgfpathlineto{\pgfqpoint{3.673285in}{1.666212in}}%
\pgfpathlineto{\pgfqpoint{3.665345in}{1.660804in}}%
\pgfpathlineto{\pgfqpoint{3.657397in}{1.655530in}}%
\pgfpathlineto{\pgfqpoint{3.649441in}{1.650393in}}%
\pgfpathclose%
\pgfusepath{fill}%
\end{pgfscope}%
\begin{pgfscope}%
\pgfpathrectangle{\pgfqpoint{1.254980in}{0.150000in}}{\pgfqpoint{5.490039in}{5.490039in}}%
\pgfusepath{clip}%
\pgfsetbuttcap%
\pgfsetroundjoin%
\definecolor{currentfill}{rgb}{0.190631,0.407061,0.556089}%
\pgfsetfillcolor{currentfill}%
\pgfsetfillopacity{0.700000}%
\pgfsetlinewidth{0.000000pt}%
\definecolor{currentstroke}{rgb}{0.000000,0.000000,0.000000}%
\pgfsetstrokecolor{currentstroke}%
\pgfsetdash{}{0pt}%
\pgfpathmoveto{\pgfqpoint{2.724478in}{2.466331in}}%
\pgfpathlineto{\pgfqpoint{2.738024in}{2.448097in}}%
\pgfpathlineto{\pgfqpoint{2.751564in}{2.430037in}}%
\pgfpathlineto{\pgfqpoint{2.765097in}{2.412151in}}%
\pgfpathlineto{\pgfqpoint{2.778626in}{2.394436in}}%
\pgfpathlineto{\pgfqpoint{2.787143in}{2.392199in}}%
\pgfpathlineto{\pgfqpoint{2.795645in}{2.390193in}}%
\pgfpathlineto{\pgfqpoint{2.804132in}{2.388413in}}%
\pgfpathlineto{\pgfqpoint{2.812604in}{2.386856in}}%
\pgfpathlineto{\pgfqpoint{2.799117in}{2.404150in}}%
\pgfpathlineto{\pgfqpoint{2.785625in}{2.421616in}}%
\pgfpathlineto{\pgfqpoint{2.772128in}{2.439254in}}%
\pgfpathlineto{\pgfqpoint{2.758625in}{2.457065in}}%
\pgfpathlineto{\pgfqpoint{2.750112in}{2.459036in}}%
\pgfpathlineto{\pgfqpoint{2.741583in}{2.461235in}}%
\pgfpathlineto{\pgfqpoint{2.733039in}{2.463666in}}%
\pgfpathlineto{\pgfqpoint{2.724478in}{2.466331in}}%
\pgfpathclose%
\pgfusepath{fill}%
\end{pgfscope}%
\begin{pgfscope}%
\pgfpathrectangle{\pgfqpoint{1.254980in}{0.150000in}}{\pgfqpoint{5.490039in}{5.490039in}}%
\pgfusepath{clip}%
\pgfsetbuttcap%
\pgfsetroundjoin%
\definecolor{currentfill}{rgb}{0.269944,0.014625,0.341379}%
\pgfsetfillcolor{currentfill}%
\pgfsetfillopacity{0.700000}%
\pgfsetlinewidth{0.000000pt}%
\definecolor{currentstroke}{rgb}{0.000000,0.000000,0.000000}%
\pgfsetstrokecolor{currentstroke}%
\pgfsetdash{}{0pt}%
\pgfpathmoveto{\pgfqpoint{4.065914in}{1.588879in}}%
\pgfpathlineto{\pgfqpoint{4.079378in}{1.584926in}}%
\pgfpathlineto{\pgfqpoint{4.092848in}{1.581092in}}%
\pgfpathlineto{\pgfqpoint{4.106324in}{1.577378in}}%
\pgfpathlineto{\pgfqpoint{4.119807in}{1.573782in}}%
\pgfpathlineto{\pgfqpoint{4.127568in}{1.582580in}}%
\pgfpathlineto{\pgfqpoint{4.135323in}{1.591454in}}%
\pgfpathlineto{\pgfqpoint{4.143072in}{1.600400in}}%
\pgfpathlineto{\pgfqpoint{4.150817in}{1.609417in}}%
\pgfpathlineto{\pgfqpoint{4.137345in}{1.612704in}}%
\pgfpathlineto{\pgfqpoint{4.123881in}{1.616109in}}%
\pgfpathlineto{\pgfqpoint{4.110423in}{1.619634in}}%
\pgfpathlineto{\pgfqpoint{4.096971in}{1.623279in}}%
\pgfpathlineto{\pgfqpoint{4.089215in}{1.614565in}}%
\pgfpathlineto{\pgfqpoint{4.081454in}{1.605925in}}%
\pgfpathlineto{\pgfqpoint{4.073687in}{1.597363in}}%
\pgfpathlineto{\pgfqpoint{4.065914in}{1.588879in}}%
\pgfpathclose%
\pgfusepath{fill}%
\end{pgfscope}%
\begin{pgfscope}%
\pgfpathrectangle{\pgfqpoint{1.254980in}{0.150000in}}{\pgfqpoint{5.490039in}{5.490039in}}%
\pgfusepath{clip}%
\pgfsetbuttcap%
\pgfsetroundjoin%
\definecolor{currentfill}{rgb}{0.269308,0.218818,0.509577}%
\pgfsetfillcolor{currentfill}%
\pgfsetfillopacity{0.700000}%
\pgfsetlinewidth{0.000000pt}%
\definecolor{currentstroke}{rgb}{0.000000,0.000000,0.000000}%
\pgfsetstrokecolor{currentstroke}%
\pgfsetdash{}{0pt}%
\pgfpathmoveto{\pgfqpoint{3.102022in}{2.018497in}}%
\pgfpathlineto{\pgfqpoint{3.115458in}{2.004789in}}%
\pgfpathlineto{\pgfqpoint{3.128894in}{1.991231in}}%
\pgfpathlineto{\pgfqpoint{3.142327in}{1.977821in}}%
\pgfpathlineto{\pgfqpoint{3.155759in}{1.964560in}}%
\pgfpathlineto{\pgfqpoint{3.164012in}{1.965253in}}%
\pgfpathlineto{\pgfqpoint{3.172254in}{1.966143in}}%
\pgfpathlineto{\pgfqpoint{3.180483in}{1.967226in}}%
\pgfpathlineto{\pgfqpoint{3.188701in}{1.968501in}}%
\pgfpathlineto{\pgfqpoint{3.175302in}{1.981358in}}%
\pgfpathlineto{\pgfqpoint{3.161902in}{1.994362in}}%
\pgfpathlineto{\pgfqpoint{3.148500in}{2.007515in}}%
\pgfpathlineto{\pgfqpoint{3.135097in}{2.020818in}}%
\pgfpathlineto{\pgfqpoint{3.126847in}{2.019941in}}%
\pgfpathlineto{\pgfqpoint{3.118584in}{2.019260in}}%
\pgfpathlineto{\pgfqpoint{3.110309in}{2.018778in}}%
\pgfpathlineto{\pgfqpoint{3.102022in}{2.018497in}}%
\pgfpathclose%
\pgfusepath{fill}%
\end{pgfscope}%
\begin{pgfscope}%
\pgfpathrectangle{\pgfqpoint{1.254980in}{0.150000in}}{\pgfqpoint{5.490039in}{5.490039in}}%
\pgfusepath{clip}%
\pgfsetbuttcap%
\pgfsetroundjoin%
\definecolor{currentfill}{rgb}{0.282910,0.105393,0.426902}%
\pgfsetfillcolor{currentfill}%
\pgfsetfillopacity{0.700000}%
\pgfsetlinewidth{0.000000pt}%
\definecolor{currentstroke}{rgb}{0.000000,0.000000,0.000000}%
\pgfsetstrokecolor{currentstroke}%
\pgfsetdash{}{0pt}%
\pgfpathmoveto{\pgfqpoint{4.544353in}{1.741867in}}%
\pgfpathlineto{\pgfqpoint{4.557964in}{1.742104in}}%
\pgfpathlineto{\pgfqpoint{4.571586in}{1.742457in}}%
\pgfpathlineto{\pgfqpoint{4.585216in}{1.742924in}}%
\pgfpathlineto{\pgfqpoint{4.598857in}{1.743505in}}%
\pgfpathlineto{\pgfqpoint{4.606472in}{1.754800in}}%
\pgfpathlineto{\pgfqpoint{4.614082in}{1.766104in}}%
\pgfpathlineto{\pgfqpoint{4.621687in}{1.777416in}}%
\pgfpathlineto{\pgfqpoint{4.629288in}{1.788734in}}%
\pgfpathlineto{\pgfqpoint{4.615652in}{1.787922in}}%
\pgfpathlineto{\pgfqpoint{4.602026in}{1.787224in}}%
\pgfpathlineto{\pgfqpoint{4.588410in}{1.786641in}}%
\pgfpathlineto{\pgfqpoint{4.574804in}{1.786173in}}%
\pgfpathlineto{\pgfqpoint{4.567198in}{1.775080in}}%
\pgfpathlineto{\pgfqpoint{4.559588in}{1.763997in}}%
\pgfpathlineto{\pgfqpoint{4.551973in}{1.752925in}}%
\pgfpathlineto{\pgfqpoint{4.544353in}{1.741867in}}%
\pgfpathclose%
\pgfusepath{fill}%
\end{pgfscope}%
\begin{pgfscope}%
\pgfpathrectangle{\pgfqpoint{1.254980in}{0.150000in}}{\pgfqpoint{5.490039in}{5.490039in}}%
\pgfusepath{clip}%
\pgfsetbuttcap%
\pgfsetroundjoin%
\definecolor{currentfill}{rgb}{0.156270,0.489624,0.557936}%
\pgfsetfillcolor{currentfill}%
\pgfsetfillopacity{0.700000}%
\pgfsetlinewidth{0.000000pt}%
\definecolor{currentstroke}{rgb}{0.000000,0.000000,0.000000}%
\pgfsetstrokecolor{currentstroke}%
\pgfsetdash{}{0pt}%
\pgfpathmoveto{\pgfqpoint{5.656655in}{2.611429in}}%
\pgfpathlineto{\pgfqpoint{5.670834in}{2.618878in}}%
\pgfpathlineto{\pgfqpoint{5.685028in}{2.626439in}}%
\pgfpathlineto{\pgfqpoint{5.699239in}{2.634114in}}%
\pgfpathlineto{\pgfqpoint{5.706476in}{2.644029in}}%
\pgfpathlineto{\pgfqpoint{5.713706in}{2.653863in}}%
\pgfpathlineto{\pgfqpoint{5.720928in}{2.663616in}}%
\pgfpathlineto{\pgfqpoint{5.728143in}{2.673290in}}%
\pgfpathlineto{\pgfqpoint{5.713939in}{2.665629in}}%
\pgfpathlineto{\pgfqpoint{5.699751in}{2.658081in}}%
\pgfpathlineto{\pgfqpoint{5.685579in}{2.650646in}}%
\pgfpathlineto{\pgfqpoint{5.678359in}{2.640958in}}%
\pgfpathlineto{\pgfqpoint{5.671131in}{2.631192in}}%
\pgfpathlineto{\pgfqpoint{5.663897in}{2.621350in}}%
\pgfpathlineto{\pgfqpoint{5.656655in}{2.611429in}}%
\pgfpathclose%
\pgfusepath{fill}%
\end{pgfscope}%
\begin{pgfscope}%
\pgfpathrectangle{\pgfqpoint{1.254980in}{0.150000in}}{\pgfqpoint{5.490039in}{5.490039in}}%
\pgfusepath{clip}%
\pgfsetbuttcap%
\pgfsetroundjoin%
\definecolor{currentfill}{rgb}{0.281446,0.084320,0.407414}%
\pgfsetfillcolor{currentfill}%
\pgfsetfillopacity{0.700000}%
\pgfsetlinewidth{0.000000pt}%
\definecolor{currentstroke}{rgb}{0.000000,0.000000,0.000000}%
\pgfsetstrokecolor{currentstroke}%
\pgfsetdash{}{0pt}%
\pgfpathmoveto{\pgfqpoint{4.459453in}{1.698983in}}%
\pgfpathlineto{\pgfqpoint{4.473033in}{1.698515in}}%
\pgfpathlineto{\pgfqpoint{4.486622in}{1.698162in}}%
\pgfpathlineto{\pgfqpoint{4.500220in}{1.697924in}}%
\pgfpathlineto{\pgfqpoint{4.513828in}{1.697801in}}%
\pgfpathlineto{\pgfqpoint{4.521466in}{1.708788in}}%
\pgfpathlineto{\pgfqpoint{4.529100in}{1.719796in}}%
\pgfpathlineto{\pgfqpoint{4.536729in}{1.730823in}}%
\pgfpathlineto{\pgfqpoint{4.544353in}{1.741867in}}%
\pgfpathlineto{\pgfqpoint{4.530751in}{1.741743in}}%
\pgfpathlineto{\pgfqpoint{4.517159in}{1.741735in}}%
\pgfpathlineto{\pgfqpoint{4.503576in}{1.741842in}}%
\pgfpathlineto{\pgfqpoint{4.490002in}{1.742064in}}%
\pgfpathlineto{\pgfqpoint{4.482372in}{1.731260in}}%
\pgfpathlineto{\pgfqpoint{4.474737in}{1.720478in}}%
\pgfpathlineto{\pgfqpoint{4.467097in}{1.709718in}}%
\pgfpathlineto{\pgfqpoint{4.459453in}{1.698983in}}%
\pgfpathclose%
\pgfusepath{fill}%
\end{pgfscope}%
\begin{pgfscope}%
\pgfpathrectangle{\pgfqpoint{1.254980in}{0.150000in}}{\pgfqpoint{5.490039in}{5.490039in}}%
\pgfusepath{clip}%
\pgfsetbuttcap%
\pgfsetroundjoin%
\definecolor{currentfill}{rgb}{0.281924,0.089666,0.412415}%
\pgfsetfillcolor{currentfill}%
\pgfsetfillopacity{0.700000}%
\pgfsetlinewidth{0.000000pt}%
\definecolor{currentstroke}{rgb}{0.000000,0.000000,0.000000}%
\pgfsetstrokecolor{currentstroke}%
\pgfsetdash{}{0pt}%
\pgfpathmoveto{\pgfqpoint{3.456606in}{1.741365in}}%
\pgfpathlineto{\pgfqpoint{3.470007in}{1.731461in}}%
\pgfpathlineto{\pgfqpoint{3.483409in}{1.721692in}}%
\pgfpathlineto{\pgfqpoint{3.496813in}{1.712056in}}%
\pgfpathlineto{\pgfqpoint{3.510219in}{1.702554in}}%
\pgfpathlineto{\pgfqpoint{3.518255in}{1.706346in}}%
\pgfpathlineto{\pgfqpoint{3.526283in}{1.710296in}}%
\pgfpathlineto{\pgfqpoint{3.534301in}{1.714400in}}%
\pgfpathlineto{\pgfqpoint{3.542311in}{1.718655in}}%
\pgfpathlineto{\pgfqpoint{3.528930in}{1.727779in}}%
\pgfpathlineto{\pgfqpoint{3.515551in}{1.737036in}}%
\pgfpathlineto{\pgfqpoint{3.502174in}{1.746426in}}%
\pgfpathlineto{\pgfqpoint{3.488798in}{1.755951in}}%
\pgfpathlineto{\pgfqpoint{3.480764in}{1.752068in}}%
\pgfpathlineto{\pgfqpoint{3.472721in}{1.748340in}}%
\pgfpathlineto{\pgfqpoint{3.464668in}{1.744772in}}%
\pgfpathlineto{\pgfqpoint{3.456606in}{1.741365in}}%
\pgfpathclose%
\pgfusepath{fill}%
\end{pgfscope}%
\begin{pgfscope}%
\pgfpathrectangle{\pgfqpoint{1.254980in}{0.150000in}}{\pgfqpoint{5.490039in}{5.490039in}}%
\pgfusepath{clip}%
\pgfsetbuttcap%
\pgfsetroundjoin%
\definecolor{currentfill}{rgb}{0.283072,0.130895,0.449241}%
\pgfsetfillcolor{currentfill}%
\pgfsetfillopacity{0.700000}%
\pgfsetlinewidth{0.000000pt}%
\definecolor{currentstroke}{rgb}{0.000000,0.000000,0.000000}%
\pgfsetstrokecolor{currentstroke}%
\pgfsetdash{}{0pt}%
\pgfpathmoveto{\pgfqpoint{4.629288in}{1.788734in}}%
\pgfpathlineto{\pgfqpoint{4.642934in}{1.789660in}}%
\pgfpathlineto{\pgfqpoint{4.656591in}{1.790700in}}%
\pgfpathlineto{\pgfqpoint{4.670257in}{1.791855in}}%
\pgfpathlineto{\pgfqpoint{4.683935in}{1.793123in}}%
\pgfpathlineto{\pgfqpoint{4.691526in}{1.804666in}}%
\pgfpathlineto{\pgfqpoint{4.699114in}{1.816207in}}%
\pgfpathlineto{\pgfqpoint{4.706696in}{1.827745in}}%
\pgfpathlineto{\pgfqpoint{4.714274in}{1.839279in}}%
\pgfpathlineto{\pgfqpoint{4.700601in}{1.837796in}}%
\pgfpathlineto{\pgfqpoint{4.686939in}{1.836426in}}%
\pgfpathlineto{\pgfqpoint{4.673287in}{1.835171in}}%
\pgfpathlineto{\pgfqpoint{4.659645in}{1.834030in}}%
\pgfpathlineto{\pgfqpoint{4.652063in}{1.822705in}}%
\pgfpathlineto{\pgfqpoint{4.644476in}{1.811380in}}%
\pgfpathlineto{\pgfqpoint{4.636884in}{1.800056in}}%
\pgfpathlineto{\pgfqpoint{4.629288in}{1.788734in}}%
\pgfpathclose%
\pgfusepath{fill}%
\end{pgfscope}%
\begin{pgfscope}%
\pgfpathrectangle{\pgfqpoint{1.254980in}{0.150000in}}{\pgfqpoint{5.490039in}{5.490039in}}%
\pgfusepath{clip}%
\pgfsetbuttcap%
\pgfsetroundjoin%
\definecolor{currentfill}{rgb}{0.119423,0.611141,0.538982}%
\pgfsetfillcolor{currentfill}%
\pgfsetfillopacity{0.700000}%
\pgfsetlinewidth{0.000000pt}%
\definecolor{currentstroke}{rgb}{0.000000,0.000000,0.000000}%
\pgfsetstrokecolor{currentstroke}%
\pgfsetdash{}{0pt}%
\pgfpathmoveto{\pgfqpoint{2.377594in}{3.030896in}}%
\pgfpathlineto{\pgfqpoint{2.391323in}{3.007747in}}%
\pgfpathlineto{\pgfqpoint{2.405042in}{2.984806in}}%
\pgfpathlineto{\pgfqpoint{2.418750in}{2.962073in}}%
\pgfpathlineto{\pgfqpoint{2.432447in}{2.939544in}}%
\pgfpathlineto{\pgfqpoint{2.441218in}{2.935276in}}%
\pgfpathlineto{\pgfqpoint{2.449971in}{2.931259in}}%
\pgfpathlineto{\pgfqpoint{2.458706in}{2.927490in}}%
\pgfpathlineto{\pgfqpoint{2.467422in}{2.923965in}}%
\pgfpathlineto{\pgfqpoint{2.453774in}{2.946069in}}%
\pgfpathlineto{\pgfqpoint{2.440115in}{2.968378in}}%
\pgfpathlineto{\pgfqpoint{2.426447in}{2.990893in}}%
\pgfpathlineto{\pgfqpoint{2.412768in}{3.013615in}}%
\pgfpathlineto{\pgfqpoint{2.404003in}{3.017558in}}%
\pgfpathlineto{\pgfqpoint{2.395219in}{3.021750in}}%
\pgfpathlineto{\pgfqpoint{2.386416in}{3.026195in}}%
\pgfpathlineto{\pgfqpoint{2.377594in}{3.030896in}}%
\pgfpathclose%
\pgfusepath{fill}%
\end{pgfscope}%
\begin{pgfscope}%
\pgfpathrectangle{\pgfqpoint{1.254980in}{0.150000in}}{\pgfqpoint{5.490039in}{5.490039in}}%
\pgfusepath{clip}%
\pgfsetbuttcap%
\pgfsetroundjoin%
\definecolor{currentfill}{rgb}{0.278791,0.062145,0.386592}%
\pgfsetfillcolor{currentfill}%
\pgfsetfillopacity{0.700000}%
\pgfsetlinewidth{0.000000pt}%
\definecolor{currentstroke}{rgb}{0.000000,0.000000,0.000000}%
\pgfsetstrokecolor{currentstroke}%
\pgfsetdash{}{0pt}%
\pgfpathmoveto{\pgfqpoint{4.374570in}{1.660399in}}%
\pgfpathlineto{\pgfqpoint{4.388122in}{1.659207in}}%
\pgfpathlineto{\pgfqpoint{4.401682in}{1.658131in}}%
\pgfpathlineto{\pgfqpoint{4.415251in}{1.657170in}}%
\pgfpathlineto{\pgfqpoint{4.428829in}{1.656325in}}%
\pgfpathlineto{\pgfqpoint{4.436492in}{1.666943in}}%
\pgfpathlineto{\pgfqpoint{4.444150in}{1.677593in}}%
\pgfpathlineto{\pgfqpoint{4.451804in}{1.688274in}}%
\pgfpathlineto{\pgfqpoint{4.459453in}{1.698983in}}%
\pgfpathlineto{\pgfqpoint{4.445882in}{1.699566in}}%
\pgfpathlineto{\pgfqpoint{4.432320in}{1.700265in}}%
\pgfpathlineto{\pgfqpoint{4.418767in}{1.701079in}}%
\pgfpathlineto{\pgfqpoint{4.405223in}{1.702010in}}%
\pgfpathlineto{\pgfqpoint{4.397567in}{1.691557in}}%
\pgfpathlineto{\pgfqpoint{4.389906in}{1.681136in}}%
\pgfpathlineto{\pgfqpoint{4.382241in}{1.670749in}}%
\pgfpathlineto{\pgfqpoint{4.374570in}{1.660399in}}%
\pgfpathclose%
\pgfusepath{fill}%
\end{pgfscope}%
\begin{pgfscope}%
\pgfpathrectangle{\pgfqpoint{1.254980in}{0.150000in}}{\pgfqpoint{5.490039in}{5.490039in}}%
\pgfusepath{clip}%
\pgfsetbuttcap%
\pgfsetroundjoin%
\definecolor{currentfill}{rgb}{0.243113,0.292092,0.538516}%
\pgfsetfillcolor{currentfill}%
\pgfsetfillopacity{0.700000}%
\pgfsetlinewidth{0.000000pt}%
\definecolor{currentstroke}{rgb}{0.000000,0.000000,0.000000}%
\pgfsetstrokecolor{currentstroke}%
\pgfsetdash{}{0pt}%
\pgfpathmoveto{\pgfqpoint{5.084893in}{2.119445in}}%
\pgfpathlineto{\pgfqpoint{5.098753in}{2.123693in}}%
\pgfpathlineto{\pgfqpoint{5.112625in}{2.128053in}}%
\pgfpathlineto{\pgfqpoint{5.126511in}{2.132526in}}%
\pgfpathlineto{\pgfqpoint{5.140410in}{2.137112in}}%
\pgfpathlineto{\pgfqpoint{5.147870in}{2.148877in}}%
\pgfpathlineto{\pgfqpoint{5.155325in}{2.160591in}}%
\pgfpathlineto{\pgfqpoint{5.162774in}{2.172255in}}%
\pgfpathlineto{\pgfqpoint{5.170217in}{2.183866in}}%
\pgfpathlineto{\pgfqpoint{5.156321in}{2.179160in}}%
\pgfpathlineto{\pgfqpoint{5.142438in}{2.174567in}}%
\pgfpathlineto{\pgfqpoint{5.128568in}{2.170087in}}%
\pgfpathlineto{\pgfqpoint{5.114711in}{2.165720in}}%
\pgfpathlineto{\pgfqpoint{5.107265in}{2.154222in}}%
\pgfpathlineto{\pgfqpoint{5.099813in}{2.142677in}}%
\pgfpathlineto{\pgfqpoint{5.092356in}{2.131084in}}%
\pgfpathlineto{\pgfqpoint{5.084893in}{2.119445in}}%
\pgfpathclose%
\pgfusepath{fill}%
\end{pgfscope}%
\begin{pgfscope}%
\pgfpathrectangle{\pgfqpoint{1.254980in}{0.150000in}}{\pgfqpoint{5.490039in}{5.490039in}}%
\pgfusepath{clip}%
\pgfsetbuttcap%
\pgfsetroundjoin%
\definecolor{currentfill}{rgb}{0.195860,0.395433,0.555276}%
\pgfsetfillcolor{currentfill}%
\pgfsetfillopacity{0.700000}%
\pgfsetlinewidth{0.000000pt}%
\definecolor{currentstroke}{rgb}{0.000000,0.000000,0.000000}%
\pgfsetstrokecolor{currentstroke}%
\pgfsetdash{}{0pt}%
\pgfpathmoveto{\pgfqpoint{5.370695in}{2.362732in}}%
\pgfpathlineto{\pgfqpoint{5.384709in}{2.368714in}}%
\pgfpathlineto{\pgfqpoint{5.398737in}{2.374808in}}%
\pgfpathlineto{\pgfqpoint{5.412780in}{2.381015in}}%
\pgfpathlineto{\pgfqpoint{5.426838in}{2.387335in}}%
\pgfpathlineto{\pgfqpoint{5.434196in}{2.398403in}}%
\pgfpathlineto{\pgfqpoint{5.441547in}{2.409400in}}%
\pgfpathlineto{\pgfqpoint{5.448893in}{2.420326in}}%
\pgfpathlineto{\pgfqpoint{5.456232in}{2.431181in}}%
\pgfpathlineto{\pgfqpoint{5.442178in}{2.424807in}}%
\pgfpathlineto{\pgfqpoint{5.428139in}{2.418546in}}%
\pgfpathlineto{\pgfqpoint{5.414115in}{2.412398in}}%
\pgfpathlineto{\pgfqpoint{5.400106in}{2.406362in}}%
\pgfpathlineto{\pgfqpoint{5.392762in}{2.395555in}}%
\pgfpathlineto{\pgfqpoint{5.385413in}{2.384681in}}%
\pgfpathlineto{\pgfqpoint{5.378057in}{2.373740in}}%
\pgfpathlineto{\pgfqpoint{5.370695in}{2.362732in}}%
\pgfpathclose%
\pgfusepath{fill}%
\end{pgfscope}%
\begin{pgfscope}%
\pgfpathrectangle{\pgfqpoint{1.254980in}{0.150000in}}{\pgfqpoint{5.490039in}{5.490039in}}%
\pgfusepath{clip}%
\pgfsetbuttcap%
\pgfsetroundjoin%
\definecolor{currentfill}{rgb}{0.179019,0.433756,0.557430}%
\pgfsetfillcolor{currentfill}%
\pgfsetfillopacity{0.700000}%
\pgfsetlinewidth{0.000000pt}%
\definecolor{currentstroke}{rgb}{0.000000,0.000000,0.000000}%
\pgfsetstrokecolor{currentstroke}%
\pgfsetdash{}{0pt}%
\pgfpathmoveto{\pgfqpoint{2.670234in}{2.541033in}}%
\pgfpathlineto{\pgfqpoint{2.683805in}{2.522090in}}%
\pgfpathlineto{\pgfqpoint{2.697369in}{2.503327in}}%
\pgfpathlineto{\pgfqpoint{2.710927in}{2.484741in}}%
\pgfpathlineto{\pgfqpoint{2.724478in}{2.466331in}}%
\pgfpathlineto{\pgfqpoint{2.733039in}{2.463666in}}%
\pgfpathlineto{\pgfqpoint{2.741583in}{2.461235in}}%
\pgfpathlineto{\pgfqpoint{2.750112in}{2.459036in}}%
\pgfpathlineto{\pgfqpoint{2.758625in}{2.457065in}}%
\pgfpathlineto{\pgfqpoint{2.745116in}{2.475052in}}%
\pgfpathlineto{\pgfqpoint{2.731602in}{2.493214in}}%
\pgfpathlineto{\pgfqpoint{2.718081in}{2.511552in}}%
\pgfpathlineto{\pgfqpoint{2.704554in}{2.530069in}}%
\pgfpathlineto{\pgfqpoint{2.695998in}{2.532457in}}%
\pgfpathlineto{\pgfqpoint{2.687427in}{2.535077in}}%
\pgfpathlineto{\pgfqpoint{2.678839in}{2.537935in}}%
\pgfpathlineto{\pgfqpoint{2.670234in}{2.541033in}}%
\pgfpathclose%
\pgfusepath{fill}%
\end{pgfscope}%
\begin{pgfscope}%
\pgfpathrectangle{\pgfqpoint{1.254980in}{0.150000in}}{\pgfqpoint{5.490039in}{5.490039in}}%
\pgfusepath{clip}%
\pgfsetbuttcap%
\pgfsetroundjoin%
\definecolor{currentfill}{rgb}{0.275191,0.194905,0.496005}%
\pgfsetfillcolor{currentfill}%
\pgfsetfillopacity{0.700000}%
\pgfsetlinewidth{0.000000pt}%
\definecolor{currentstroke}{rgb}{0.000000,0.000000,0.000000}%
\pgfsetstrokecolor{currentstroke}%
\pgfsetdash{}{0pt}%
\pgfpathmoveto{\pgfqpoint{3.155759in}{1.964560in}}%
\pgfpathlineto{\pgfqpoint{3.169189in}{1.951446in}}%
\pgfpathlineto{\pgfqpoint{3.182618in}{1.938479in}}%
\pgfpathlineto{\pgfqpoint{3.196046in}{1.925658in}}%
\pgfpathlineto{\pgfqpoint{3.209473in}{1.912983in}}%
\pgfpathlineto{\pgfqpoint{3.217694in}{1.914085in}}%
\pgfpathlineto{\pgfqpoint{3.225903in}{1.915381in}}%
\pgfpathlineto{\pgfqpoint{3.234100in}{1.916865in}}%
\pgfpathlineto{\pgfqpoint{3.242287in}{1.918536in}}%
\pgfpathlineto{\pgfqpoint{3.228892in}{1.930809in}}%
\pgfpathlineto{\pgfqpoint{3.215496in}{1.943227in}}%
\pgfpathlineto{\pgfqpoint{3.202099in}{1.955791in}}%
\pgfpathlineto{\pgfqpoint{3.188701in}{1.968501in}}%
\pgfpathlineto{\pgfqpoint{3.180483in}{1.967226in}}%
\pgfpathlineto{\pgfqpoint{3.172254in}{1.966143in}}%
\pgfpathlineto{\pgfqpoint{3.164012in}{1.965253in}}%
\pgfpathlineto{\pgfqpoint{3.155759in}{1.964560in}}%
\pgfpathclose%
\pgfusepath{fill}%
\end{pgfscope}%
\begin{pgfscope}%
\pgfpathrectangle{\pgfqpoint{1.254980in}{0.150000in}}{\pgfqpoint{5.490039in}{5.490039in}}%
\pgfusepath{clip}%
\pgfsetbuttcap%
\pgfsetroundjoin%
\definecolor{currentfill}{rgb}{0.280868,0.160771,0.472899}%
\pgfsetfillcolor{currentfill}%
\pgfsetfillopacity{0.700000}%
\pgfsetlinewidth{0.000000pt}%
\definecolor{currentstroke}{rgb}{0.000000,0.000000,0.000000}%
\pgfsetstrokecolor{currentstroke}%
\pgfsetdash{}{0pt}%
\pgfpathmoveto{\pgfqpoint{4.714274in}{1.839279in}}%
\pgfpathlineto{\pgfqpoint{4.727958in}{1.840876in}}%
\pgfpathlineto{\pgfqpoint{4.741653in}{1.842587in}}%
\pgfpathlineto{\pgfqpoint{4.755359in}{1.844411in}}%
\pgfpathlineto{\pgfqpoint{4.769075in}{1.846349in}}%
\pgfpathlineto{\pgfqpoint{4.776645in}{1.858082in}}%
\pgfpathlineto{\pgfqpoint{4.784210in}{1.869803in}}%
\pgfpathlineto{\pgfqpoint{4.791771in}{1.881511in}}%
\pgfpathlineto{\pgfqpoint{4.799326in}{1.893205in}}%
\pgfpathlineto{\pgfqpoint{4.785613in}{1.891067in}}%
\pgfpathlineto{\pgfqpoint{4.771911in}{1.889043in}}%
\pgfpathlineto{\pgfqpoint{4.758220in}{1.887133in}}%
\pgfpathlineto{\pgfqpoint{4.744540in}{1.885337in}}%
\pgfpathlineto{\pgfqpoint{4.736981in}{1.873837in}}%
\pgfpathlineto{\pgfqpoint{4.729417in}{1.862326in}}%
\pgfpathlineto{\pgfqpoint{4.721848in}{1.850806in}}%
\pgfpathlineto{\pgfqpoint{4.714274in}{1.839279in}}%
\pgfpathclose%
\pgfusepath{fill}%
\end{pgfscope}%
\begin{pgfscope}%
\pgfpathrectangle{\pgfqpoint{1.254980in}{0.150000in}}{\pgfqpoint{5.490039in}{5.490039in}}%
\pgfusepath{clip}%
\pgfsetbuttcap%
\pgfsetroundjoin%
\definecolor{currentfill}{rgb}{0.276022,0.044167,0.370164}%
\pgfsetfillcolor{currentfill}%
\pgfsetfillopacity{0.700000}%
\pgfsetlinewidth{0.000000pt}%
\definecolor{currentstroke}{rgb}{0.000000,0.000000,0.000000}%
\pgfsetstrokecolor{currentstroke}%
\pgfsetdash{}{0pt}%
\pgfpathmoveto{\pgfqpoint{4.289684in}{1.626443in}}%
\pgfpathlineto{\pgfqpoint{4.303211in}{1.624509in}}%
\pgfpathlineto{\pgfqpoint{4.316746in}{1.622691in}}%
\pgfpathlineto{\pgfqpoint{4.330289in}{1.620989in}}%
\pgfpathlineto{\pgfqpoint{4.343841in}{1.619404in}}%
\pgfpathlineto{\pgfqpoint{4.351530in}{1.629588in}}%
\pgfpathlineto{\pgfqpoint{4.359215in}{1.639816in}}%
\pgfpathlineto{\pgfqpoint{4.366895in}{1.650087in}}%
\pgfpathlineto{\pgfqpoint{4.374570in}{1.660399in}}%
\pgfpathlineto{\pgfqpoint{4.361027in}{1.661707in}}%
\pgfpathlineto{\pgfqpoint{4.347492in}{1.663131in}}%
\pgfpathlineto{\pgfqpoint{4.333965in}{1.664672in}}%
\pgfpathlineto{\pgfqpoint{4.320446in}{1.666329in}}%
\pgfpathlineto{\pgfqpoint{4.312763in}{1.656289in}}%
\pgfpathlineto{\pgfqpoint{4.305075in}{1.646293in}}%
\pgfpathlineto{\pgfqpoint{4.297382in}{1.636344in}}%
\pgfpathlineto{\pgfqpoint{4.289684in}{1.626443in}}%
\pgfpathclose%
\pgfusepath{fill}%
\end{pgfscope}%
\begin{pgfscope}%
\pgfpathrectangle{\pgfqpoint{1.254980in}{0.150000in}}{\pgfqpoint{5.490039in}{5.490039in}}%
\pgfusepath{clip}%
\pgfsetbuttcap%
\pgfsetroundjoin%
\definecolor{currentfill}{rgb}{0.276194,0.190074,0.493001}%
\pgfsetfillcolor{currentfill}%
\pgfsetfillopacity{0.700000}%
\pgfsetlinewidth{0.000000pt}%
\definecolor{currentstroke}{rgb}{0.000000,0.000000,0.000000}%
\pgfsetstrokecolor{currentstroke}%
\pgfsetdash{}{0pt}%
\pgfpathmoveto{\pgfqpoint{4.799326in}{1.893205in}}%
\pgfpathlineto{\pgfqpoint{4.813051in}{1.895455in}}%
\pgfpathlineto{\pgfqpoint{4.826787in}{1.897820in}}%
\pgfpathlineto{\pgfqpoint{4.840534in}{1.900297in}}%
\pgfpathlineto{\pgfqpoint{4.854293in}{1.902888in}}%
\pgfpathlineto{\pgfqpoint{4.861841in}{1.914755in}}%
\pgfpathlineto{\pgfqpoint{4.869384in}{1.926601in}}%
\pgfpathlineto{\pgfqpoint{4.876923in}{1.938425in}}%
\pgfpathlineto{\pgfqpoint{4.884456in}{1.950224in}}%
\pgfpathlineto{\pgfqpoint{4.870700in}{1.947450in}}%
\pgfpathlineto{\pgfqpoint{4.856956in}{1.944788in}}%
\pgfpathlineto{\pgfqpoint{4.843223in}{1.942240in}}%
\pgfpathlineto{\pgfqpoint{4.829502in}{1.939806in}}%
\pgfpathlineto{\pgfqpoint{4.821965in}{1.928184in}}%
\pgfpathlineto{\pgfqpoint{4.814424in}{1.916542in}}%
\pgfpathlineto{\pgfqpoint{4.806878in}{1.904882in}}%
\pgfpathlineto{\pgfqpoint{4.799326in}{1.893205in}}%
\pgfpathclose%
\pgfusepath{fill}%
\end{pgfscope}%
\begin{pgfscope}%
\pgfpathrectangle{\pgfqpoint{1.254980in}{0.150000in}}{\pgfqpoint{5.490039in}{5.490039in}}%
\pgfusepath{clip}%
\pgfsetbuttcap%
\pgfsetroundjoin%
\definecolor{currentfill}{rgb}{0.166617,0.463708,0.558119}%
\pgfsetfillcolor{currentfill}%
\pgfsetfillopacity{0.700000}%
\pgfsetlinewidth{0.000000pt}%
\definecolor{currentstroke}{rgb}{0.000000,0.000000,0.000000}%
\pgfsetstrokecolor{currentstroke}%
\pgfsetdash{}{0pt}%
\pgfpathmoveto{\pgfqpoint{2.615882in}{2.618609in}}%
\pgfpathlineto{\pgfqpoint{2.629480in}{2.598942in}}%
\pgfpathlineto{\pgfqpoint{2.643072in}{2.579457in}}%
\pgfpathlineto{\pgfqpoint{2.656656in}{2.560154in}}%
\pgfpathlineto{\pgfqpoint{2.670234in}{2.541033in}}%
\pgfpathlineto{\pgfqpoint{2.678839in}{2.537935in}}%
\pgfpathlineto{\pgfqpoint{2.687427in}{2.535077in}}%
\pgfpathlineto{\pgfqpoint{2.695998in}{2.532457in}}%
\pgfpathlineto{\pgfqpoint{2.704554in}{2.530069in}}%
\pgfpathlineto{\pgfqpoint{2.691021in}{2.548764in}}%
\pgfpathlineto{\pgfqpoint{2.677481in}{2.567640in}}%
\pgfpathlineto{\pgfqpoint{2.663935in}{2.586696in}}%
\pgfpathlineto{\pgfqpoint{2.650381in}{2.605935in}}%
\pgfpathlineto{\pgfqpoint{2.641782in}{2.608743in}}%
\pgfpathlineto{\pgfqpoint{2.633165in}{2.611789in}}%
\pgfpathlineto{\pgfqpoint{2.624532in}{2.615077in}}%
\pgfpathlineto{\pgfqpoint{2.615882in}{2.618609in}}%
\pgfpathclose%
\pgfusepath{fill}%
\end{pgfscope}%
\begin{pgfscope}%
\pgfpathrectangle{\pgfqpoint{1.254980in}{0.150000in}}{\pgfqpoint{5.490039in}{5.490039in}}%
\pgfusepath{clip}%
\pgfsetbuttcap%
\pgfsetroundjoin%
\definecolor{currentfill}{rgb}{0.269944,0.014625,0.341379}%
\pgfsetfillcolor{currentfill}%
\pgfsetfillopacity{0.700000}%
\pgfsetlinewidth{0.000000pt}%
\definecolor{currentstroke}{rgb}{0.000000,0.000000,0.000000}%
\pgfsetstrokecolor{currentstroke}%
\pgfsetdash{}{0pt}%
\pgfpathmoveto{\pgfqpoint{3.842046in}{1.589049in}}%
\pgfpathlineto{\pgfqpoint{3.855475in}{1.582971in}}%
\pgfpathlineto{\pgfqpoint{3.868909in}{1.577016in}}%
\pgfpathlineto{\pgfqpoint{3.882347in}{1.571184in}}%
\pgfpathlineto{\pgfqpoint{3.895790in}{1.565475in}}%
\pgfpathlineto{\pgfqpoint{3.903641in}{1.572535in}}%
\pgfpathlineto{\pgfqpoint{3.911486in}{1.579704in}}%
\pgfpathlineto{\pgfqpoint{3.919324in}{1.586978in}}%
\pgfpathlineto{\pgfqpoint{3.927155in}{1.594356in}}%
\pgfpathlineto{\pgfqpoint{3.913728in}{1.599723in}}%
\pgfpathlineto{\pgfqpoint{3.900306in}{1.605213in}}%
\pgfpathlineto{\pgfqpoint{3.886889in}{1.610826in}}%
\pgfpathlineto{\pgfqpoint{3.873477in}{1.616563in}}%
\pgfpathlineto{\pgfqpoint{3.865629in}{1.609521in}}%
\pgfpathlineto{\pgfqpoint{3.857775in}{1.602586in}}%
\pgfpathlineto{\pgfqpoint{3.849914in}{1.595761in}}%
\pgfpathlineto{\pgfqpoint{3.842046in}{1.589049in}}%
\pgfpathclose%
\pgfusepath{fill}%
\end{pgfscope}%
\begin{pgfscope}%
\pgfpathrectangle{\pgfqpoint{1.254980in}{0.150000in}}{\pgfqpoint{5.490039in}{5.490039in}}%
\pgfusepath{clip}%
\pgfsetbuttcap%
\pgfsetroundjoin%
\definecolor{currentfill}{rgb}{0.278826,0.175490,0.483397}%
\pgfsetfillcolor{currentfill}%
\pgfsetfillopacity{0.700000}%
\pgfsetlinewidth{0.000000pt}%
\definecolor{currentstroke}{rgb}{0.000000,0.000000,0.000000}%
\pgfsetstrokecolor{currentstroke}%
\pgfsetdash{}{0pt}%
\pgfpathmoveto{\pgfqpoint{3.209473in}{1.912983in}}%
\pgfpathlineto{\pgfqpoint{3.222899in}{1.900452in}}%
\pgfpathlineto{\pgfqpoint{3.236324in}{1.888065in}}%
\pgfpathlineto{\pgfqpoint{3.249748in}{1.875821in}}%
\pgfpathlineto{\pgfqpoint{3.263172in}{1.863720in}}%
\pgfpathlineto{\pgfqpoint{3.271361in}{1.865231in}}%
\pgfpathlineto{\pgfqpoint{3.279539in}{1.866931in}}%
\pgfpathlineto{\pgfqpoint{3.287706in}{1.868815in}}%
\pgfpathlineto{\pgfqpoint{3.295862in}{1.870880in}}%
\pgfpathlineto{\pgfqpoint{3.282469in}{1.882580in}}%
\pgfpathlineto{\pgfqpoint{3.269075in}{1.894422in}}%
\pgfpathlineto{\pgfqpoint{3.255681in}{1.906407in}}%
\pgfpathlineto{\pgfqpoint{3.242287in}{1.918536in}}%
\pgfpathlineto{\pgfqpoint{3.234100in}{1.916865in}}%
\pgfpathlineto{\pgfqpoint{3.225903in}{1.915381in}}%
\pgfpathlineto{\pgfqpoint{3.217694in}{1.914085in}}%
\pgfpathlineto{\pgfqpoint{3.209473in}{1.912983in}}%
\pgfpathclose%
\pgfusepath{fill}%
\end{pgfscope}%
\begin{pgfscope}%
\pgfpathrectangle{\pgfqpoint{1.254980in}{0.150000in}}{\pgfqpoint{5.490039in}{5.490039in}}%
\pgfusepath{clip}%
\pgfsetbuttcap%
\pgfsetroundjoin%
\definecolor{currentfill}{rgb}{0.227802,0.326594,0.546532}%
\pgfsetfillcolor{currentfill}%
\pgfsetfillopacity{0.700000}%
\pgfsetlinewidth{0.000000pt}%
\definecolor{currentstroke}{rgb}{0.000000,0.000000,0.000000}%
\pgfsetstrokecolor{currentstroke}%
\pgfsetdash{}{0pt}%
\pgfpathmoveto{\pgfqpoint{5.170217in}{2.183866in}}%
\pgfpathlineto{\pgfqpoint{5.184127in}{2.188684in}}%
\pgfpathlineto{\pgfqpoint{5.198051in}{2.193616in}}%
\pgfpathlineto{\pgfqpoint{5.211988in}{2.198660in}}%
\pgfpathlineto{\pgfqpoint{5.225940in}{2.203817in}}%
\pgfpathlineto{\pgfqpoint{5.233375in}{2.215485in}}%
\pgfpathlineto{\pgfqpoint{5.240805in}{2.227096in}}%
\pgfpathlineto{\pgfqpoint{5.248229in}{2.238649in}}%
\pgfpathlineto{\pgfqpoint{5.255648in}{2.250143in}}%
\pgfpathlineto{\pgfqpoint{5.241699in}{2.244883in}}%
\pgfpathlineto{\pgfqpoint{5.227765in}{2.239735in}}%
\pgfpathlineto{\pgfqpoint{5.213844in}{2.234701in}}%
\pgfpathlineto{\pgfqpoint{5.199937in}{2.229779in}}%
\pgfpathlineto{\pgfqpoint{5.192515in}{2.218381in}}%
\pgfpathlineto{\pgfqpoint{5.185088in}{2.206930in}}%
\pgfpathlineto{\pgfqpoint{5.177656in}{2.195424in}}%
\pgfpathlineto{\pgfqpoint{5.170217in}{2.183866in}}%
\pgfpathclose%
\pgfusepath{fill}%
\end{pgfscope}%
\begin{pgfscope}%
\pgfpathrectangle{\pgfqpoint{1.254980in}{0.150000in}}{\pgfqpoint{5.490039in}{5.490039in}}%
\pgfusepath{clip}%
\pgfsetbuttcap%
\pgfsetroundjoin%
\definecolor{currentfill}{rgb}{0.280267,0.073417,0.397163}%
\pgfsetfillcolor{currentfill}%
\pgfsetfillopacity{0.700000}%
\pgfsetlinewidth{0.000000pt}%
\definecolor{currentstroke}{rgb}{0.000000,0.000000,0.000000}%
\pgfsetstrokecolor{currentstroke}%
\pgfsetdash{}{0pt}%
\pgfpathmoveto{\pgfqpoint{3.510219in}{1.702554in}}%
\pgfpathlineto{\pgfqpoint{3.523626in}{1.693184in}}%
\pgfpathlineto{\pgfqpoint{3.537035in}{1.683947in}}%
\pgfpathlineto{\pgfqpoint{3.550446in}{1.674841in}}%
\pgfpathlineto{\pgfqpoint{3.563859in}{1.665867in}}%
\pgfpathlineto{\pgfqpoint{3.571872in}{1.670043in}}%
\pgfpathlineto{\pgfqpoint{3.579875in}{1.674373in}}%
\pgfpathlineto{\pgfqpoint{3.587870in}{1.678853in}}%
\pgfpathlineto{\pgfqpoint{3.595856in}{1.683480in}}%
\pgfpathlineto{\pgfqpoint{3.582467in}{1.692076in}}%
\pgfpathlineto{\pgfqpoint{3.569079in}{1.700804in}}%
\pgfpathlineto{\pgfqpoint{3.555694in}{1.709664in}}%
\pgfpathlineto{\pgfqpoint{3.542311in}{1.718655in}}%
\pgfpathlineto{\pgfqpoint{3.534301in}{1.714400in}}%
\pgfpathlineto{\pgfqpoint{3.526283in}{1.710296in}}%
\pgfpathlineto{\pgfqpoint{3.518255in}{1.706346in}}%
\pgfpathlineto{\pgfqpoint{3.510219in}{1.702554in}}%
\pgfpathclose%
\pgfusepath{fill}%
\end{pgfscope}%
\begin{pgfscope}%
\pgfpathrectangle{\pgfqpoint{1.254980in}{0.150000in}}{\pgfqpoint{5.490039in}{5.490039in}}%
\pgfusepath{clip}%
\pgfsetbuttcap%
\pgfsetroundjoin%
\definecolor{currentfill}{rgb}{0.268510,0.009605,0.335427}%
\pgfsetfillcolor{currentfill}%
\pgfsetfillopacity{0.700000}%
\pgfsetlinewidth{0.000000pt}%
\definecolor{currentstroke}{rgb}{0.000000,0.000000,0.000000}%
\pgfsetstrokecolor{currentstroke}%
\pgfsetdash{}{0pt}%
\pgfpathmoveto{\pgfqpoint{3.980915in}{1.574105in}}%
\pgfpathlineto{\pgfqpoint{3.994369in}{1.569346in}}%
\pgfpathlineto{\pgfqpoint{4.007828in}{1.564708in}}%
\pgfpathlineto{\pgfqpoint{4.021293in}{1.560189in}}%
\pgfpathlineto{\pgfqpoint{4.034764in}{1.555791in}}%
\pgfpathlineto{\pgfqpoint{4.042560in}{1.563931in}}%
\pgfpathlineto{\pgfqpoint{4.050351in}{1.572161in}}%
\pgfpathlineto{\pgfqpoint{4.058135in}{1.580478in}}%
\pgfpathlineto{\pgfqpoint{4.065914in}{1.588879in}}%
\pgfpathlineto{\pgfqpoint{4.052456in}{1.592953in}}%
\pgfpathlineto{\pgfqpoint{4.039005in}{1.597146in}}%
\pgfpathlineto{\pgfqpoint{4.025559in}{1.601459in}}%
\pgfpathlineto{\pgfqpoint{4.012119in}{1.605894in}}%
\pgfpathlineto{\pgfqpoint{4.004327in}{1.597811in}}%
\pgfpathlineto{\pgfqpoint{3.996529in}{1.589817in}}%
\pgfpathlineto{\pgfqpoint{3.988725in}{1.581915in}}%
\pgfpathlineto{\pgfqpoint{3.980915in}{1.574105in}}%
\pgfpathclose%
\pgfusepath{fill}%
\end{pgfscope}%
\begin{pgfscope}%
\pgfpathrectangle{\pgfqpoint{1.254980in}{0.150000in}}{\pgfqpoint{5.490039in}{5.490039in}}%
\pgfusepath{clip}%
\pgfsetbuttcap%
\pgfsetroundjoin%
\definecolor{currentfill}{rgb}{0.272594,0.025563,0.353093}%
\pgfsetfillcolor{currentfill}%
\pgfsetfillopacity{0.700000}%
\pgfsetlinewidth{0.000000pt}%
\definecolor{currentstroke}{rgb}{0.000000,0.000000,0.000000}%
\pgfsetstrokecolor{currentstroke}%
\pgfsetdash{}{0pt}%
\pgfpathmoveto{\pgfqpoint{4.204771in}{1.597454in}}%
\pgfpathlineto{\pgfqpoint{4.218278in}{1.594758in}}%
\pgfpathlineto{\pgfqpoint{4.231791in}{1.592179in}}%
\pgfpathlineto{\pgfqpoint{4.245313in}{1.589718in}}%
\pgfpathlineto{\pgfqpoint{4.258841in}{1.587373in}}%
\pgfpathlineto{\pgfqpoint{4.266559in}{1.597057in}}%
\pgfpathlineto{\pgfqpoint{4.274273in}{1.606797in}}%
\pgfpathlineto{\pgfqpoint{4.281981in}{1.616594in}}%
\pgfpathlineto{\pgfqpoint{4.289684in}{1.626443in}}%
\pgfpathlineto{\pgfqpoint{4.276164in}{1.628494in}}%
\pgfpathlineto{\pgfqpoint{4.262653in}{1.630663in}}%
\pgfpathlineto{\pgfqpoint{4.249149in}{1.632948in}}%
\pgfpathlineto{\pgfqpoint{4.235652in}{1.635351in}}%
\pgfpathlineto{\pgfqpoint{4.227940in}{1.625789in}}%
\pgfpathlineto{\pgfqpoint{4.220222in}{1.616284in}}%
\pgfpathlineto{\pgfqpoint{4.212499in}{1.606838in}}%
\pgfpathlineto{\pgfqpoint{4.204771in}{1.597454in}}%
\pgfpathclose%
\pgfusepath{fill}%
\end{pgfscope}%
\begin{pgfscope}%
\pgfpathrectangle{\pgfqpoint{1.254980in}{0.150000in}}{\pgfqpoint{5.490039in}{5.490039in}}%
\pgfusepath{clip}%
\pgfsetbuttcap%
\pgfsetroundjoin%
\definecolor{currentfill}{rgb}{0.273809,0.031497,0.358853}%
\pgfsetfillcolor{currentfill}%
\pgfsetfillopacity{0.700000}%
\pgfsetlinewidth{0.000000pt}%
\definecolor{currentstroke}{rgb}{0.000000,0.000000,0.000000}%
\pgfsetstrokecolor{currentstroke}%
\pgfsetdash{}{0pt}%
\pgfpathmoveto{\pgfqpoint{3.703073in}{1.619366in}}%
\pgfpathlineto{\pgfqpoint{3.716489in}{1.611928in}}%
\pgfpathlineto{\pgfqpoint{3.729909in}{1.604617in}}%
\pgfpathlineto{\pgfqpoint{3.743332in}{1.597431in}}%
\pgfpathlineto{\pgfqpoint{3.756759in}{1.590372in}}%
\pgfpathlineto{\pgfqpoint{3.764674in}{1.596240in}}%
\pgfpathlineto{\pgfqpoint{3.772581in}{1.602237in}}%
\pgfpathlineto{\pgfqpoint{3.780481in}{1.608359in}}%
\pgfpathlineto{\pgfqpoint{3.788374in}{1.614603in}}%
\pgfpathlineto{\pgfqpoint{3.774966in}{1.621304in}}%
\pgfpathlineto{\pgfqpoint{3.761563in}{1.628130in}}%
\pgfpathlineto{\pgfqpoint{3.748163in}{1.635083in}}%
\pgfpathlineto{\pgfqpoint{3.734766in}{1.642162in}}%
\pgfpathlineto{\pgfqpoint{3.726855in}{1.636270in}}%
\pgfpathlineto{\pgfqpoint{3.718935in}{1.630505in}}%
\pgfpathlineto{\pgfqpoint{3.711008in}{1.624869in}}%
\pgfpathlineto{\pgfqpoint{3.703073in}{1.619366in}}%
\pgfpathclose%
\pgfusepath{fill}%
\end{pgfscope}%
\begin{pgfscope}%
\pgfpathrectangle{\pgfqpoint{1.254980in}{0.150000in}}{\pgfqpoint{5.490039in}{5.490039in}}%
\pgfusepath{clip}%
\pgfsetbuttcap%
\pgfsetroundjoin%
\definecolor{currentfill}{rgb}{0.182256,0.426184,0.557120}%
\pgfsetfillcolor{currentfill}%
\pgfsetfillopacity{0.700000}%
\pgfsetlinewidth{0.000000pt}%
\definecolor{currentstroke}{rgb}{0.000000,0.000000,0.000000}%
\pgfsetstrokecolor{currentstroke}%
\pgfsetdash{}{0pt}%
\pgfpathmoveto{\pgfqpoint{5.456232in}{2.431181in}}%
\pgfpathlineto{\pgfqpoint{5.470301in}{2.437667in}}%
\pgfpathlineto{\pgfqpoint{5.484385in}{2.444266in}}%
\pgfpathlineto{\pgfqpoint{5.498485in}{2.450978in}}%
\pgfpathlineto{\pgfqpoint{5.512599in}{2.457803in}}%
\pgfpathlineto{\pgfqpoint{5.519928in}{2.468630in}}%
\pgfpathlineto{\pgfqpoint{5.527250in}{2.479382in}}%
\pgfpathlineto{\pgfqpoint{5.534565in}{2.490058in}}%
\pgfpathlineto{\pgfqpoint{5.541874in}{2.500659in}}%
\pgfpathlineto{\pgfqpoint{5.527763in}{2.493797in}}%
\pgfpathlineto{\pgfqpoint{5.513668in}{2.487048in}}%
\pgfpathlineto{\pgfqpoint{5.499589in}{2.480412in}}%
\pgfpathlineto{\pgfqpoint{5.485524in}{2.473888in}}%
\pgfpathlineto{\pgfqpoint{5.478211in}{2.463318in}}%
\pgfpathlineto{\pgfqpoint{5.470891in}{2.452677in}}%
\pgfpathlineto{\pgfqpoint{5.463565in}{2.441964in}}%
\pgfpathlineto{\pgfqpoint{5.456232in}{2.431181in}}%
\pgfpathclose%
\pgfusepath{fill}%
\end{pgfscope}%
\begin{pgfscope}%
\pgfpathrectangle{\pgfqpoint{1.254980in}{0.150000in}}{\pgfqpoint{5.490039in}{5.490039in}}%
\pgfusepath{clip}%
\pgfsetbuttcap%
\pgfsetroundjoin%
\definecolor{currentfill}{rgb}{0.269308,0.218818,0.509577}%
\pgfsetfillcolor{currentfill}%
\pgfsetfillopacity{0.700000}%
\pgfsetlinewidth{0.000000pt}%
\definecolor{currentstroke}{rgb}{0.000000,0.000000,0.000000}%
\pgfsetstrokecolor{currentstroke}%
\pgfsetdash{}{0pt}%
\pgfpathmoveto{\pgfqpoint{4.884456in}{1.950224in}}%
\pgfpathlineto{\pgfqpoint{4.898224in}{1.953112in}}%
\pgfpathlineto{\pgfqpoint{4.912004in}{1.956113in}}%
\pgfpathlineto{\pgfqpoint{4.925796in}{1.959227in}}%
\pgfpathlineto{\pgfqpoint{4.939600in}{1.962454in}}%
\pgfpathlineto{\pgfqpoint{4.947126in}{1.974402in}}%
\pgfpathlineto{\pgfqpoint{4.954647in}{1.986320in}}%
\pgfpathlineto{\pgfqpoint{4.962163in}{1.998206in}}%
\pgfpathlineto{\pgfqpoint{4.969675in}{2.010060in}}%
\pgfpathlineto{\pgfqpoint{4.955873in}{2.006665in}}%
\pgfpathlineto{\pgfqpoint{4.942084in}{2.003383in}}%
\pgfpathlineto{\pgfqpoint{4.928307in}{2.000214in}}%
\pgfpathlineto{\pgfqpoint{4.914542in}{1.997159in}}%
\pgfpathlineto{\pgfqpoint{4.907028in}{1.985467in}}%
\pgfpathlineto{\pgfqpoint{4.899509in}{1.973747in}}%
\pgfpathlineto{\pgfqpoint{4.891985in}{1.961999in}}%
\pgfpathlineto{\pgfqpoint{4.884456in}{1.950224in}}%
\pgfpathclose%
\pgfusepath{fill}%
\end{pgfscope}%
\begin{pgfscope}%
\pgfpathrectangle{\pgfqpoint{1.254980in}{0.150000in}}{\pgfqpoint{5.490039in}{5.490039in}}%
\pgfusepath{clip}%
\pgfsetbuttcap%
\pgfsetroundjoin%
\definecolor{currentfill}{rgb}{0.130067,0.651384,0.521608}%
\pgfsetfillcolor{currentfill}%
\pgfsetfillopacity{0.700000}%
\pgfsetlinewidth{0.000000pt}%
\definecolor{currentstroke}{rgb}{0.000000,0.000000,0.000000}%
\pgfsetstrokecolor{currentstroke}%
\pgfsetdash{}{0pt}%
\pgfpathmoveto{\pgfqpoint{2.322569in}{3.125609in}}%
\pgfpathlineto{\pgfqpoint{2.336342in}{3.101610in}}%
\pgfpathlineto{\pgfqpoint{2.350104in}{3.077826in}}%
\pgfpathlineto{\pgfqpoint{2.363855in}{3.054256in}}%
\pgfpathlineto{\pgfqpoint{2.377594in}{3.030896in}}%
\pgfpathlineto{\pgfqpoint{2.386416in}{3.026195in}}%
\pgfpathlineto{\pgfqpoint{2.395219in}{3.021750in}}%
\pgfpathlineto{\pgfqpoint{2.404003in}{3.017558in}}%
\pgfpathlineto{\pgfqpoint{2.412768in}{3.013615in}}%
\pgfpathlineto{\pgfqpoint{2.399079in}{3.036547in}}%
\pgfpathlineto{\pgfqpoint{2.385379in}{3.059689in}}%
\pgfpathlineto{\pgfqpoint{2.371669in}{3.083042in}}%
\pgfpathlineto{\pgfqpoint{2.357947in}{3.106610in}}%
\pgfpathlineto{\pgfqpoint{2.349131in}{3.110974in}}%
\pgfpathlineto{\pgfqpoint{2.340297in}{3.115593in}}%
\pgfpathlineto{\pgfqpoint{2.331442in}{3.120470in}}%
\pgfpathlineto{\pgfqpoint{2.322569in}{3.125609in}}%
\pgfpathclose%
\pgfusepath{fill}%
\end{pgfscope}%
\begin{pgfscope}%
\pgfpathrectangle{\pgfqpoint{1.254980in}{0.150000in}}{\pgfqpoint{5.490039in}{5.490039in}}%
\pgfusepath{clip}%
\pgfsetbuttcap%
\pgfsetroundjoin%
\definecolor{currentfill}{rgb}{0.153364,0.497000,0.557724}%
\pgfsetfillcolor{currentfill}%
\pgfsetfillopacity{0.700000}%
\pgfsetlinewidth{0.000000pt}%
\definecolor{currentstroke}{rgb}{0.000000,0.000000,0.000000}%
\pgfsetstrokecolor{currentstroke}%
\pgfsetdash{}{0pt}%
\pgfpathmoveto{\pgfqpoint{2.561411in}{2.699136in}}%
\pgfpathlineto{\pgfqpoint{2.575040in}{2.678723in}}%
\pgfpathlineto{\pgfqpoint{2.588662in}{2.658499in}}%
\pgfpathlineto{\pgfqpoint{2.602276in}{2.638462in}}%
\pgfpathlineto{\pgfqpoint{2.615882in}{2.618609in}}%
\pgfpathlineto{\pgfqpoint{2.624532in}{2.615077in}}%
\pgfpathlineto{\pgfqpoint{2.633165in}{2.611789in}}%
\pgfpathlineto{\pgfqpoint{2.641782in}{2.608743in}}%
\pgfpathlineto{\pgfqpoint{2.650381in}{2.605935in}}%
\pgfpathlineto{\pgfqpoint{2.636821in}{2.625358in}}%
\pgfpathlineto{\pgfqpoint{2.623253in}{2.644965in}}%
\pgfpathlineto{\pgfqpoint{2.609678in}{2.664758in}}%
\pgfpathlineto{\pgfqpoint{2.596096in}{2.684737in}}%
\pgfpathlineto{\pgfqpoint{2.587450in}{2.687969in}}%
\pgfpathlineto{\pgfqpoint{2.578788in}{2.691443in}}%
\pgfpathlineto{\pgfqpoint{2.570108in}{2.695164in}}%
\pgfpathlineto{\pgfqpoint{2.561411in}{2.699136in}}%
\pgfpathclose%
\pgfusepath{fill}%
\end{pgfscope}%
\begin{pgfscope}%
\pgfpathrectangle{\pgfqpoint{1.254980in}{0.150000in}}{\pgfqpoint{5.490039in}{5.490039in}}%
\pgfusepath{clip}%
\pgfsetbuttcap%
\pgfsetroundjoin%
\definecolor{currentfill}{rgb}{0.281412,0.155834,0.469201}%
\pgfsetfillcolor{currentfill}%
\pgfsetfillopacity{0.700000}%
\pgfsetlinewidth{0.000000pt}%
\definecolor{currentstroke}{rgb}{0.000000,0.000000,0.000000}%
\pgfsetstrokecolor{currentstroke}%
\pgfsetdash{}{0pt}%
\pgfpathmoveto{\pgfqpoint{3.263172in}{1.863720in}}%
\pgfpathlineto{\pgfqpoint{3.276595in}{1.851762in}}%
\pgfpathlineto{\pgfqpoint{3.290019in}{1.839945in}}%
\pgfpathlineto{\pgfqpoint{3.303441in}{1.828268in}}%
\pgfpathlineto{\pgfqpoint{3.316864in}{1.816732in}}%
\pgfpathlineto{\pgfqpoint{3.325023in}{1.818650in}}%
\pgfpathlineto{\pgfqpoint{3.333171in}{1.820751in}}%
\pgfpathlineto{\pgfqpoint{3.341309in}{1.823033in}}%
\pgfpathlineto{\pgfqpoint{3.349435in}{1.825492in}}%
\pgfpathlineto{\pgfqpoint{3.336042in}{1.836629in}}%
\pgfpathlineto{\pgfqpoint{3.322649in}{1.847905in}}%
\pgfpathlineto{\pgfqpoint{3.309255in}{1.859322in}}%
\pgfpathlineto{\pgfqpoint{3.295862in}{1.870880in}}%
\pgfpathlineto{\pgfqpoint{3.287706in}{1.868815in}}%
\pgfpathlineto{\pgfqpoint{3.279539in}{1.866931in}}%
\pgfpathlineto{\pgfqpoint{3.271361in}{1.865231in}}%
\pgfpathlineto{\pgfqpoint{3.263172in}{1.863720in}}%
\pgfpathclose%
\pgfusepath{fill}%
\end{pgfscope}%
\begin{pgfscope}%
\pgfpathrectangle{\pgfqpoint{1.254980in}{0.150000in}}{\pgfqpoint{5.490039in}{5.490039in}}%
\pgfusepath{clip}%
\pgfsetbuttcap%
\pgfsetroundjoin%
\definecolor{currentfill}{rgb}{0.214298,0.355619,0.551184}%
\pgfsetfillcolor{currentfill}%
\pgfsetfillopacity{0.700000}%
\pgfsetlinewidth{0.000000pt}%
\definecolor{currentstroke}{rgb}{0.000000,0.000000,0.000000}%
\pgfsetstrokecolor{currentstroke}%
\pgfsetdash{}{0pt}%
\pgfpathmoveto{\pgfqpoint{5.255648in}{2.250143in}}%
\pgfpathlineto{\pgfqpoint{5.269610in}{2.255517in}}%
\pgfpathlineto{\pgfqpoint{5.283587in}{2.261002in}}%
\pgfpathlineto{\pgfqpoint{5.297578in}{2.266601in}}%
\pgfpathlineto{\pgfqpoint{5.311583in}{2.272312in}}%
\pgfpathlineto{\pgfqpoint{5.318993in}{2.283841in}}%
\pgfpathlineto{\pgfqpoint{5.326397in}{2.295306in}}%
\pgfpathlineto{\pgfqpoint{5.333795in}{2.306706in}}%
\pgfpathlineto{\pgfqpoint{5.341187in}{2.318042in}}%
\pgfpathlineto{\pgfqpoint{5.327185in}{2.312244in}}%
\pgfpathlineto{\pgfqpoint{5.313197in}{2.306558in}}%
\pgfpathlineto{\pgfqpoint{5.299223in}{2.300985in}}%
\pgfpathlineto{\pgfqpoint{5.285264in}{2.295525in}}%
\pgfpathlineto{\pgfqpoint{5.277869in}{2.284270in}}%
\pgfpathlineto{\pgfqpoint{5.270467in}{2.272954in}}%
\pgfpathlineto{\pgfqpoint{5.263060in}{2.261579in}}%
\pgfpathlineto{\pgfqpoint{5.255648in}{2.250143in}}%
\pgfpathclose%
\pgfusepath{fill}%
\end{pgfscope}%
\begin{pgfscope}%
\pgfpathrectangle{\pgfqpoint{1.254980in}{0.150000in}}{\pgfqpoint{5.490039in}{5.490039in}}%
\pgfusepath{clip}%
\pgfsetbuttcap%
\pgfsetroundjoin%
\definecolor{currentfill}{rgb}{0.269944,0.014625,0.341379}%
\pgfsetfillcolor{currentfill}%
\pgfsetfillopacity{0.700000}%
\pgfsetlinewidth{0.000000pt}%
\definecolor{currentstroke}{rgb}{0.000000,0.000000,0.000000}%
\pgfsetstrokecolor{currentstroke}%
\pgfsetdash{}{0pt}%
\pgfpathmoveto{\pgfqpoint{4.119807in}{1.573782in}}%
\pgfpathlineto{\pgfqpoint{4.133297in}{1.570305in}}%
\pgfpathlineto{\pgfqpoint{4.146793in}{1.566946in}}%
\pgfpathlineto{\pgfqpoint{4.160296in}{1.563706in}}%
\pgfpathlineto{\pgfqpoint{4.173806in}{1.560583in}}%
\pgfpathlineto{\pgfqpoint{4.181555in}{1.569696in}}%
\pgfpathlineto{\pgfqpoint{4.189299in}{1.578881in}}%
\pgfpathlineto{\pgfqpoint{4.197038in}{1.588134in}}%
\pgfpathlineto{\pgfqpoint{4.204771in}{1.597454in}}%
\pgfpathlineto{\pgfqpoint{4.191272in}{1.600268in}}%
\pgfpathlineto{\pgfqpoint{4.177780in}{1.603199in}}%
\pgfpathlineto{\pgfqpoint{4.164295in}{1.606249in}}%
\pgfpathlineto{\pgfqpoint{4.150817in}{1.609417in}}%
\pgfpathlineto{\pgfqpoint{4.143072in}{1.600400in}}%
\pgfpathlineto{\pgfqpoint{4.135323in}{1.591454in}}%
\pgfpathlineto{\pgfqpoint{4.127568in}{1.582580in}}%
\pgfpathlineto{\pgfqpoint{4.119807in}{1.573782in}}%
\pgfpathclose%
\pgfusepath{fill}%
\end{pgfscope}%
\begin{pgfscope}%
\pgfpathrectangle{\pgfqpoint{1.254980in}{0.150000in}}{\pgfqpoint{5.490039in}{5.490039in}}%
\pgfusepath{clip}%
\pgfsetbuttcap%
\pgfsetroundjoin%
\definecolor{currentfill}{rgb}{0.258965,0.251537,0.524736}%
\pgfsetfillcolor{currentfill}%
\pgfsetfillopacity{0.700000}%
\pgfsetlinewidth{0.000000pt}%
\definecolor{currentstroke}{rgb}{0.000000,0.000000,0.000000}%
\pgfsetstrokecolor{currentstroke}%
\pgfsetdash{}{0pt}%
\pgfpathmoveto{\pgfqpoint{4.969675in}{2.010060in}}%
\pgfpathlineto{\pgfqpoint{4.983488in}{2.013568in}}%
\pgfpathlineto{\pgfqpoint{4.997315in}{2.017189in}}%
\pgfpathlineto{\pgfqpoint{5.011153in}{2.020923in}}%
\pgfpathlineto{\pgfqpoint{5.025005in}{2.024769in}}%
\pgfpathlineto{\pgfqpoint{5.032509in}{2.036747in}}%
\pgfpathlineto{\pgfqpoint{5.040008in}{2.048686in}}%
\pgfpathlineto{\pgfqpoint{5.047502in}{2.060585in}}%
\pgfpathlineto{\pgfqpoint{5.054990in}{2.072443in}}%
\pgfpathlineto{\pgfqpoint{5.041141in}{2.068444in}}%
\pgfpathlineto{\pgfqpoint{5.027305in}{2.064558in}}%
\pgfpathlineto{\pgfqpoint{5.013481in}{2.060785in}}%
\pgfpathlineto{\pgfqpoint{4.999670in}{2.057125in}}%
\pgfpathlineto{\pgfqpoint{4.992179in}{2.045413in}}%
\pgfpathlineto{\pgfqpoint{4.984682in}{2.033665in}}%
\pgfpathlineto{\pgfqpoint{4.977181in}{2.021880in}}%
\pgfpathlineto{\pgfqpoint{4.969675in}{2.010060in}}%
\pgfpathclose%
\pgfusepath{fill}%
\end{pgfscope}%
\begin{pgfscope}%
\pgfpathrectangle{\pgfqpoint{1.254980in}{0.150000in}}{\pgfqpoint{5.490039in}{5.490039in}}%
\pgfusepath{clip}%
\pgfsetbuttcap%
\pgfsetroundjoin%
\definecolor{currentfill}{rgb}{0.278791,0.062145,0.386592}%
\pgfsetfillcolor{currentfill}%
\pgfsetfillopacity{0.700000}%
\pgfsetlinewidth{0.000000pt}%
\definecolor{currentstroke}{rgb}{0.000000,0.000000,0.000000}%
\pgfsetstrokecolor{currentstroke}%
\pgfsetdash{}{0pt}%
\pgfpathmoveto{\pgfqpoint{3.563859in}{1.665867in}}%
\pgfpathlineto{\pgfqpoint{3.577275in}{1.657023in}}%
\pgfpathlineto{\pgfqpoint{3.590693in}{1.648310in}}%
\pgfpathlineto{\pgfqpoint{3.604113in}{1.639726in}}%
\pgfpathlineto{\pgfqpoint{3.617535in}{1.631272in}}%
\pgfpathlineto{\pgfqpoint{3.625525in}{1.635832in}}%
\pgfpathlineto{\pgfqpoint{3.633505in}{1.640541in}}%
\pgfpathlineto{\pgfqpoint{3.641477in}{1.645396in}}%
\pgfpathlineto{\pgfqpoint{3.649441in}{1.650393in}}%
\pgfpathlineto{\pgfqpoint{3.636041in}{1.658470in}}%
\pgfpathlineto{\pgfqpoint{3.622643in}{1.666677in}}%
\pgfpathlineto{\pgfqpoint{3.609249in}{1.675013in}}%
\pgfpathlineto{\pgfqpoint{3.595856in}{1.683480in}}%
\pgfpathlineto{\pgfqpoint{3.587870in}{1.678853in}}%
\pgfpathlineto{\pgfqpoint{3.579875in}{1.674373in}}%
\pgfpathlineto{\pgfqpoint{3.571872in}{1.670043in}}%
\pgfpathlineto{\pgfqpoint{3.563859in}{1.665867in}}%
\pgfpathclose%
\pgfusepath{fill}%
\end{pgfscope}%
\begin{pgfscope}%
\pgfpathrectangle{\pgfqpoint{1.254980in}{0.150000in}}{\pgfqpoint{5.490039in}{5.490039in}}%
\pgfusepath{clip}%
\pgfsetbuttcap%
\pgfsetroundjoin%
\definecolor{currentfill}{rgb}{0.282884,0.135920,0.453427}%
\pgfsetfillcolor{currentfill}%
\pgfsetfillopacity{0.700000}%
\pgfsetlinewidth{0.000000pt}%
\definecolor{currentstroke}{rgb}{0.000000,0.000000,0.000000}%
\pgfsetstrokecolor{currentstroke}%
\pgfsetdash{}{0pt}%
\pgfpathmoveto{\pgfqpoint{3.316864in}{1.816732in}}%
\pgfpathlineto{\pgfqpoint{3.330287in}{1.805336in}}%
\pgfpathlineto{\pgfqpoint{3.343710in}{1.794078in}}%
\pgfpathlineto{\pgfqpoint{3.357134in}{1.782959in}}%
\pgfpathlineto{\pgfqpoint{3.370557in}{1.771978in}}%
\pgfpathlineto{\pgfqpoint{3.378687in}{1.774301in}}%
\pgfpathlineto{\pgfqpoint{3.386806in}{1.776803in}}%
\pgfpathlineto{\pgfqpoint{3.394915in}{1.779482in}}%
\pgfpathlineto{\pgfqpoint{3.403014in}{1.782333in}}%
\pgfpathlineto{\pgfqpoint{3.389618in}{1.792916in}}%
\pgfpathlineto{\pgfqpoint{3.376224in}{1.803636in}}%
\pgfpathlineto{\pgfqpoint{3.362829in}{1.814495in}}%
\pgfpathlineto{\pgfqpoint{3.349435in}{1.825492in}}%
\pgfpathlineto{\pgfqpoint{3.341309in}{1.823033in}}%
\pgfpathlineto{\pgfqpoint{3.333171in}{1.820751in}}%
\pgfpathlineto{\pgfqpoint{3.325023in}{1.818650in}}%
\pgfpathlineto{\pgfqpoint{3.316864in}{1.816732in}}%
\pgfpathclose%
\pgfusepath{fill}%
\end{pgfscope}%
\begin{pgfscope}%
\pgfpathrectangle{\pgfqpoint{1.254980in}{0.150000in}}{\pgfqpoint{5.490039in}{5.490039in}}%
\pgfusepath{clip}%
\pgfsetbuttcap%
\pgfsetroundjoin%
\definecolor{currentfill}{rgb}{0.169646,0.456262,0.558030}%
\pgfsetfillcolor{currentfill}%
\pgfsetfillopacity{0.700000}%
\pgfsetlinewidth{0.000000pt}%
\definecolor{currentstroke}{rgb}{0.000000,0.000000,0.000000}%
\pgfsetstrokecolor{currentstroke}%
\pgfsetdash{}{0pt}%
\pgfpathmoveto{\pgfqpoint{5.541874in}{2.500659in}}%
\pgfpathlineto{\pgfqpoint{5.556000in}{2.507633in}}%
\pgfpathlineto{\pgfqpoint{5.570141in}{2.514720in}}%
\pgfpathlineto{\pgfqpoint{5.584299in}{2.521921in}}%
\pgfpathlineto{\pgfqpoint{5.598472in}{2.529233in}}%
\pgfpathlineto{\pgfqpoint{5.605769in}{2.539785in}}%
\pgfpathlineto{\pgfqpoint{5.613059in}{2.550258in}}%
\pgfpathlineto{\pgfqpoint{5.620343in}{2.560650in}}%
\pgfpathlineto{\pgfqpoint{5.627619in}{2.570964in}}%
\pgfpathlineto{\pgfqpoint{5.613451in}{2.563631in}}%
\pgfpathlineto{\pgfqpoint{5.599299in}{2.556411in}}%
\pgfpathlineto{\pgfqpoint{5.585162in}{2.549303in}}%
\pgfpathlineto{\pgfqpoint{5.571041in}{2.542309in}}%
\pgfpathlineto{\pgfqpoint{5.563759in}{2.532009in}}%
\pgfpathlineto{\pgfqpoint{5.556471in}{2.521634in}}%
\pgfpathlineto{\pgfqpoint{5.549176in}{2.511184in}}%
\pgfpathlineto{\pgfqpoint{5.541874in}{2.500659in}}%
\pgfpathclose%
\pgfusepath{fill}%
\end{pgfscope}%
\begin{pgfscope}%
\pgfpathrectangle{\pgfqpoint{1.254980in}{0.150000in}}{\pgfqpoint{5.490039in}{5.490039in}}%
\pgfusepath{clip}%
\pgfsetbuttcap%
\pgfsetroundjoin%
\definecolor{currentfill}{rgb}{0.140536,0.530132,0.555659}%
\pgfsetfillcolor{currentfill}%
\pgfsetfillopacity{0.700000}%
\pgfsetlinewidth{0.000000pt}%
\definecolor{currentstroke}{rgb}{0.000000,0.000000,0.000000}%
\pgfsetstrokecolor{currentstroke}%
\pgfsetdash{}{0pt}%
\pgfpathmoveto{\pgfqpoint{2.506811in}{2.782688in}}%
\pgfpathlineto{\pgfqpoint{2.520474in}{2.761512in}}%
\pgfpathlineto{\pgfqpoint{2.534128in}{2.740529in}}%
\pgfpathlineto{\pgfqpoint{2.547774in}{2.719737in}}%
\pgfpathlineto{\pgfqpoint{2.561411in}{2.699136in}}%
\pgfpathlineto{\pgfqpoint{2.570108in}{2.695164in}}%
\pgfpathlineto{\pgfqpoint{2.578788in}{2.691443in}}%
\pgfpathlineto{\pgfqpoint{2.587450in}{2.687969in}}%
\pgfpathlineto{\pgfqpoint{2.596096in}{2.684737in}}%
\pgfpathlineto{\pgfqpoint{2.582505in}{2.704906in}}%
\pgfpathlineto{\pgfqpoint{2.568907in}{2.725263in}}%
\pgfpathlineto{\pgfqpoint{2.555300in}{2.745812in}}%
\pgfpathlineto{\pgfqpoint{2.541686in}{2.766552in}}%
\pgfpathlineto{\pgfqpoint{2.532994in}{2.770210in}}%
\pgfpathlineto{\pgfqpoint{2.524284in}{2.774117in}}%
\pgfpathlineto{\pgfqpoint{2.515557in}{2.778275in}}%
\pgfpathlineto{\pgfqpoint{2.506811in}{2.782688in}}%
\pgfpathclose%
\pgfusepath{fill}%
\end{pgfscope}%
\begin{pgfscope}%
\pgfpathrectangle{\pgfqpoint{1.254980in}{0.150000in}}{\pgfqpoint{5.490039in}{5.490039in}}%
\pgfusepath{clip}%
\pgfsetbuttcap%
\pgfsetroundjoin%
\definecolor{currentfill}{rgb}{0.268510,0.009605,0.335427}%
\pgfsetfillcolor{currentfill}%
\pgfsetfillopacity{0.700000}%
\pgfsetlinewidth{0.000000pt}%
\definecolor{currentstroke}{rgb}{0.000000,0.000000,0.000000}%
\pgfsetstrokecolor{currentstroke}%
\pgfsetdash{}{0pt}%
\pgfpathmoveto{\pgfqpoint{3.895790in}{1.565475in}}%
\pgfpathlineto{\pgfqpoint{3.909238in}{1.559888in}}%
\pgfpathlineto{\pgfqpoint{3.922691in}{1.554424in}}%
\pgfpathlineto{\pgfqpoint{3.936149in}{1.549081in}}%
\pgfpathlineto{\pgfqpoint{3.949612in}{1.543859in}}%
\pgfpathlineto{\pgfqpoint{3.957448in}{1.551267in}}%
\pgfpathlineto{\pgfqpoint{3.965276in}{1.558779in}}%
\pgfpathlineto{\pgfqpoint{3.973099in}{1.566393in}}%
\pgfpathlineto{\pgfqpoint{3.980915in}{1.574105in}}%
\pgfpathlineto{\pgfqpoint{3.967467in}{1.578986in}}%
\pgfpathlineto{\pgfqpoint{3.954025in}{1.583987in}}%
\pgfpathlineto{\pgfqpoint{3.940587in}{1.589110in}}%
\pgfpathlineto{\pgfqpoint{3.927155in}{1.594356in}}%
\pgfpathlineto{\pgfqpoint{3.919324in}{1.586978in}}%
\pgfpathlineto{\pgfqpoint{3.911486in}{1.579704in}}%
\pgfpathlineto{\pgfqpoint{3.903641in}{1.572535in}}%
\pgfpathlineto{\pgfqpoint{3.895790in}{1.565475in}}%
\pgfpathclose%
\pgfusepath{fill}%
\end{pgfscope}%
\begin{pgfscope}%
\pgfpathrectangle{\pgfqpoint{1.254980in}{0.150000in}}{\pgfqpoint{5.490039in}{5.490039in}}%
\pgfusepath{clip}%
\pgfsetbuttcap%
\pgfsetroundjoin%
\definecolor{currentfill}{rgb}{0.282327,0.094955,0.417331}%
\pgfsetfillcolor{currentfill}%
\pgfsetfillopacity{0.700000}%
\pgfsetlinewidth{0.000000pt}%
\definecolor{currentstroke}{rgb}{0.000000,0.000000,0.000000}%
\pgfsetstrokecolor{currentstroke}%
\pgfsetdash{}{0pt}%
\pgfpathmoveto{\pgfqpoint{4.513828in}{1.697801in}}%
\pgfpathlineto{\pgfqpoint{4.527445in}{1.697792in}}%
\pgfpathlineto{\pgfqpoint{4.541072in}{1.697898in}}%
\pgfpathlineto{\pgfqpoint{4.554708in}{1.698119in}}%
\pgfpathlineto{\pgfqpoint{4.568354in}{1.698453in}}%
\pgfpathlineto{\pgfqpoint{4.575986in}{1.709693in}}%
\pgfpathlineto{\pgfqpoint{4.583615in}{1.720950in}}%
\pgfpathlineto{\pgfqpoint{4.591238in}{1.732221in}}%
\pgfpathlineto{\pgfqpoint{4.598857in}{1.743505in}}%
\pgfpathlineto{\pgfqpoint{4.585216in}{1.742924in}}%
\pgfpathlineto{\pgfqpoint{4.571586in}{1.742457in}}%
\pgfpathlineto{\pgfqpoint{4.557964in}{1.742104in}}%
\pgfpathlineto{\pgfqpoint{4.544353in}{1.741867in}}%
\pgfpathlineto{\pgfqpoint{4.536729in}{1.730823in}}%
\pgfpathlineto{\pgfqpoint{4.529100in}{1.719796in}}%
\pgfpathlineto{\pgfqpoint{4.521466in}{1.708788in}}%
\pgfpathlineto{\pgfqpoint{4.513828in}{1.697801in}}%
\pgfpathclose%
\pgfusepath{fill}%
\end{pgfscope}%
\begin{pgfscope}%
\pgfpathrectangle{\pgfqpoint{1.254980in}{0.150000in}}{\pgfqpoint{5.490039in}{5.490039in}}%
\pgfusepath{clip}%
\pgfsetbuttcap%
\pgfsetroundjoin%
\definecolor{currentfill}{rgb}{0.271305,0.019942,0.347269}%
\pgfsetfillcolor{currentfill}%
\pgfsetfillopacity{0.700000}%
\pgfsetlinewidth{0.000000pt}%
\definecolor{currentstroke}{rgb}{0.000000,0.000000,0.000000}%
\pgfsetstrokecolor{currentstroke}%
\pgfsetdash{}{0pt}%
\pgfpathmoveto{\pgfqpoint{3.756759in}{1.590372in}}%
\pgfpathlineto{\pgfqpoint{3.770189in}{1.583438in}}%
\pgfpathlineto{\pgfqpoint{3.783624in}{1.576628in}}%
\pgfpathlineto{\pgfqpoint{3.797062in}{1.569944in}}%
\pgfpathlineto{\pgfqpoint{3.810505in}{1.563384in}}%
\pgfpathlineto{\pgfqpoint{3.818401in}{1.569617in}}%
\pgfpathlineto{\pgfqpoint{3.826290in}{1.575974in}}%
\pgfpathlineto{\pgfqpoint{3.834172in}{1.582452in}}%
\pgfpathlineto{\pgfqpoint{3.842046in}{1.589049in}}%
\pgfpathlineto{\pgfqpoint{3.828622in}{1.595251in}}%
\pgfpathlineto{\pgfqpoint{3.815202in}{1.601577in}}%
\pgfpathlineto{\pgfqpoint{3.801786in}{1.608027in}}%
\pgfpathlineto{\pgfqpoint{3.788374in}{1.614603in}}%
\pgfpathlineto{\pgfqpoint{3.780481in}{1.608359in}}%
\pgfpathlineto{\pgfqpoint{3.772581in}{1.602237in}}%
\pgfpathlineto{\pgfqpoint{3.764674in}{1.596240in}}%
\pgfpathlineto{\pgfqpoint{3.756759in}{1.590372in}}%
\pgfpathclose%
\pgfusepath{fill}%
\end{pgfscope}%
\begin{pgfscope}%
\pgfpathrectangle{\pgfqpoint{1.254980in}{0.150000in}}{\pgfqpoint{5.490039in}{5.490039in}}%
\pgfusepath{clip}%
\pgfsetbuttcap%
\pgfsetroundjoin%
\definecolor{currentfill}{rgb}{0.280267,0.073417,0.397163}%
\pgfsetfillcolor{currentfill}%
\pgfsetfillopacity{0.700000}%
\pgfsetlinewidth{0.000000pt}%
\definecolor{currentstroke}{rgb}{0.000000,0.000000,0.000000}%
\pgfsetstrokecolor{currentstroke}%
\pgfsetdash{}{0pt}%
\pgfpathmoveto{\pgfqpoint{4.428829in}{1.656325in}}%
\pgfpathlineto{\pgfqpoint{4.442416in}{1.655595in}}%
\pgfpathlineto{\pgfqpoint{4.456011in}{1.654980in}}%
\pgfpathlineto{\pgfqpoint{4.469616in}{1.654480in}}%
\pgfpathlineto{\pgfqpoint{4.483229in}{1.654095in}}%
\pgfpathlineto{\pgfqpoint{4.490886in}{1.664981in}}%
\pgfpathlineto{\pgfqpoint{4.498538in}{1.675895in}}%
\pgfpathlineto{\pgfqpoint{4.506185in}{1.686836in}}%
\pgfpathlineto{\pgfqpoint{4.513828in}{1.697801in}}%
\pgfpathlineto{\pgfqpoint{4.500220in}{1.697924in}}%
\pgfpathlineto{\pgfqpoint{4.486622in}{1.698162in}}%
\pgfpathlineto{\pgfqpoint{4.473033in}{1.698515in}}%
\pgfpathlineto{\pgfqpoint{4.459453in}{1.698983in}}%
\pgfpathlineto{\pgfqpoint{4.451804in}{1.688274in}}%
\pgfpathlineto{\pgfqpoint{4.444150in}{1.677593in}}%
\pgfpathlineto{\pgfqpoint{4.436492in}{1.666943in}}%
\pgfpathlineto{\pgfqpoint{4.428829in}{1.656325in}}%
\pgfpathclose%
\pgfusepath{fill}%
\end{pgfscope}%
\begin{pgfscope}%
\pgfpathrectangle{\pgfqpoint{1.254980in}{0.150000in}}{\pgfqpoint{5.490039in}{5.490039in}}%
\pgfusepath{clip}%
\pgfsetbuttcap%
\pgfsetroundjoin%
\definecolor{currentfill}{rgb}{0.283229,0.120777,0.440584}%
\pgfsetfillcolor{currentfill}%
\pgfsetfillopacity{0.700000}%
\pgfsetlinewidth{0.000000pt}%
\definecolor{currentstroke}{rgb}{0.000000,0.000000,0.000000}%
\pgfsetstrokecolor{currentstroke}%
\pgfsetdash{}{0pt}%
\pgfpathmoveto{\pgfqpoint{4.598857in}{1.743505in}}%
\pgfpathlineto{\pgfqpoint{4.612508in}{1.744200in}}%
\pgfpathlineto{\pgfqpoint{4.626169in}{1.745010in}}%
\pgfpathlineto{\pgfqpoint{4.639840in}{1.745933in}}%
\pgfpathlineto{\pgfqpoint{4.653522in}{1.746970in}}%
\pgfpathlineto{\pgfqpoint{4.661132in}{1.758502in}}%
\pgfpathlineto{\pgfqpoint{4.668737in}{1.770039in}}%
\pgfpathlineto{\pgfqpoint{4.676338in}{1.781580in}}%
\pgfpathlineto{\pgfqpoint{4.683935in}{1.793123in}}%
\pgfpathlineto{\pgfqpoint{4.670257in}{1.791855in}}%
\pgfpathlineto{\pgfqpoint{4.656591in}{1.790700in}}%
\pgfpathlineto{\pgfqpoint{4.642934in}{1.789660in}}%
\pgfpathlineto{\pgfqpoint{4.629288in}{1.788734in}}%
\pgfpathlineto{\pgfqpoint{4.621687in}{1.777416in}}%
\pgfpathlineto{\pgfqpoint{4.614082in}{1.766104in}}%
\pgfpathlineto{\pgfqpoint{4.606472in}{1.754800in}}%
\pgfpathlineto{\pgfqpoint{4.598857in}{1.743505in}}%
\pgfpathclose%
\pgfusepath{fill}%
\end{pgfscope}%
\begin{pgfscope}%
\pgfpathrectangle{\pgfqpoint{1.254980in}{0.150000in}}{\pgfqpoint{5.490039in}{5.490039in}}%
\pgfusepath{clip}%
\pgfsetbuttcap%
\pgfsetroundjoin%
\definecolor{currentfill}{rgb}{0.246811,0.283237,0.535941}%
\pgfsetfillcolor{currentfill}%
\pgfsetfillopacity{0.700000}%
\pgfsetlinewidth{0.000000pt}%
\definecolor{currentstroke}{rgb}{0.000000,0.000000,0.000000}%
\pgfsetstrokecolor{currentstroke}%
\pgfsetdash{}{0pt}%
\pgfpathmoveto{\pgfqpoint{5.054990in}{2.072443in}}%
\pgfpathlineto{\pgfqpoint{5.068852in}{2.076554in}}%
\pgfpathlineto{\pgfqpoint{5.082727in}{2.080779in}}%
\pgfpathlineto{\pgfqpoint{5.096615in}{2.085116in}}%
\pgfpathlineto{\pgfqpoint{5.110517in}{2.089566in}}%
\pgfpathlineto{\pgfqpoint{5.117998in}{2.101524in}}%
\pgfpathlineto{\pgfqpoint{5.125474in}{2.113435in}}%
\pgfpathlineto{\pgfqpoint{5.132945in}{2.125298in}}%
\pgfpathlineto{\pgfqpoint{5.140410in}{2.137112in}}%
\pgfpathlineto{\pgfqpoint{5.126511in}{2.132526in}}%
\pgfpathlineto{\pgfqpoint{5.112625in}{2.128053in}}%
\pgfpathlineto{\pgfqpoint{5.098753in}{2.123693in}}%
\pgfpathlineto{\pgfqpoint{5.084893in}{2.119445in}}%
\pgfpathlineto{\pgfqpoint{5.077425in}{2.107761in}}%
\pgfpathlineto{\pgfqpoint{5.069952in}{2.096032in}}%
\pgfpathlineto{\pgfqpoint{5.062474in}{2.084259in}}%
\pgfpathlineto{\pgfqpoint{5.054990in}{2.072443in}}%
\pgfpathclose%
\pgfusepath{fill}%
\end{pgfscope}%
\begin{pgfscope}%
\pgfpathrectangle{\pgfqpoint{1.254980in}{0.150000in}}{\pgfqpoint{5.490039in}{5.490039in}}%
\pgfusepath{clip}%
\pgfsetbuttcap%
\pgfsetroundjoin%
\definecolor{currentfill}{rgb}{0.162016,0.687316,0.499129}%
\pgfsetfillcolor{currentfill}%
\pgfsetfillopacity{0.700000}%
\pgfsetlinewidth{0.000000pt}%
\definecolor{currentstroke}{rgb}{0.000000,0.000000,0.000000}%
\pgfsetstrokecolor{currentstroke}%
\pgfsetdash{}{0pt}%
\pgfpathmoveto{\pgfqpoint{2.267357in}{3.223781in}}%
\pgfpathlineto{\pgfqpoint{2.281178in}{3.198908in}}%
\pgfpathlineto{\pgfqpoint{2.294987in}{3.174256in}}%
\pgfpathlineto{\pgfqpoint{2.308784in}{3.149824in}}%
\pgfpathlineto{\pgfqpoint{2.322569in}{3.125609in}}%
\pgfpathlineto{\pgfqpoint{2.331442in}{3.120470in}}%
\pgfpathlineto{\pgfqpoint{2.340297in}{3.115593in}}%
\pgfpathlineto{\pgfqpoint{2.349131in}{3.110974in}}%
\pgfpathlineto{\pgfqpoint{2.357947in}{3.106610in}}%
\pgfpathlineto{\pgfqpoint{2.344214in}{3.130392in}}%
\pgfpathlineto{\pgfqpoint{2.330470in}{3.154392in}}%
\pgfpathlineto{\pgfqpoint{2.316714in}{3.178609in}}%
\pgfpathlineto{\pgfqpoint{2.302946in}{3.203047in}}%
\pgfpathlineto{\pgfqpoint{2.294079in}{3.207837in}}%
\pgfpathlineto{\pgfqpoint{2.285191in}{3.212887in}}%
\pgfpathlineto{\pgfqpoint{2.276284in}{3.218200in}}%
\pgfpathlineto{\pgfqpoint{2.267357in}{3.223781in}}%
\pgfpathclose%
\pgfusepath{fill}%
\end{pgfscope}%
\begin{pgfscope}%
\pgfpathrectangle{\pgfqpoint{1.254980in}{0.150000in}}{\pgfqpoint{5.490039in}{5.490039in}}%
\pgfusepath{clip}%
\pgfsetbuttcap%
\pgfsetroundjoin%
\definecolor{currentfill}{rgb}{0.159194,0.482237,0.558073}%
\pgfsetfillcolor{currentfill}%
\pgfsetfillopacity{0.700000}%
\pgfsetlinewidth{0.000000pt}%
\definecolor{currentstroke}{rgb}{0.000000,0.000000,0.000000}%
\pgfsetstrokecolor{currentstroke}%
\pgfsetdash{}{0pt}%
\pgfpathmoveto{\pgfqpoint{5.627619in}{2.570964in}}%
\pgfpathlineto{\pgfqpoint{5.641804in}{2.578410in}}%
\pgfpathlineto{\pgfqpoint{5.656004in}{2.585968in}}%
\pgfpathlineto{\pgfqpoint{5.670220in}{2.593640in}}%
\pgfpathlineto{\pgfqpoint{5.677486in}{2.603881in}}%
\pgfpathlineto{\pgfqpoint{5.684744in}{2.614040in}}%
\pgfpathlineto{\pgfqpoint{5.691995in}{2.624118in}}%
\pgfpathlineto{\pgfqpoint{5.699239in}{2.634114in}}%
\pgfpathlineto{\pgfqpoint{5.685028in}{2.626439in}}%
\pgfpathlineto{\pgfqpoint{5.670834in}{2.618878in}}%
\pgfpathlineto{\pgfqpoint{5.656655in}{2.611429in}}%
\pgfpathlineto{\pgfqpoint{5.649407in}{2.601431in}}%
\pgfpathlineto{\pgfqpoint{5.642151in}{2.591354in}}%
\pgfpathlineto{\pgfqpoint{5.634889in}{2.581198in}}%
\pgfpathlineto{\pgfqpoint{5.627619in}{2.570964in}}%
\pgfpathclose%
\pgfusepath{fill}%
\end{pgfscope}%
\begin{pgfscope}%
\pgfpathrectangle{\pgfqpoint{1.254980in}{0.150000in}}{\pgfqpoint{5.490039in}{5.490039in}}%
\pgfusepath{clip}%
\pgfsetbuttcap%
\pgfsetroundjoin%
\definecolor{currentfill}{rgb}{0.199430,0.387607,0.554642}%
\pgfsetfillcolor{currentfill}%
\pgfsetfillopacity{0.700000}%
\pgfsetlinewidth{0.000000pt}%
\definecolor{currentstroke}{rgb}{0.000000,0.000000,0.000000}%
\pgfsetstrokecolor{currentstroke}%
\pgfsetdash{}{0pt}%
\pgfpathmoveto{\pgfqpoint{5.341187in}{2.318042in}}%
\pgfpathlineto{\pgfqpoint{5.355204in}{2.323954in}}%
\pgfpathlineto{\pgfqpoint{5.369235in}{2.329977in}}%
\pgfpathlineto{\pgfqpoint{5.383282in}{2.336114in}}%
\pgfpathlineto{\pgfqpoint{5.397343in}{2.342363in}}%
\pgfpathlineto{\pgfqpoint{5.404726in}{2.353711in}}%
\pgfpathlineto{\pgfqpoint{5.412102in}{2.364989in}}%
\pgfpathlineto{\pgfqpoint{5.419473in}{2.376197in}}%
\pgfpathlineto{\pgfqpoint{5.426838in}{2.387335in}}%
\pgfpathlineto{\pgfqpoint{5.412780in}{2.381015in}}%
\pgfpathlineto{\pgfqpoint{5.398737in}{2.374808in}}%
\pgfpathlineto{\pgfqpoint{5.384709in}{2.368714in}}%
\pgfpathlineto{\pgfqpoint{5.370695in}{2.362732in}}%
\pgfpathlineto{\pgfqpoint{5.363327in}{2.351659in}}%
\pgfpathlineto{\pgfqpoint{5.355953in}{2.340519in}}%
\pgfpathlineto{\pgfqpoint{5.348573in}{2.329313in}}%
\pgfpathlineto{\pgfqpoint{5.341187in}{2.318042in}}%
\pgfpathclose%
\pgfusepath{fill}%
\end{pgfscope}%
\begin{pgfscope}%
\pgfpathrectangle{\pgfqpoint{1.254980in}{0.150000in}}{\pgfqpoint{5.490039in}{5.490039in}}%
\pgfusepath{clip}%
\pgfsetbuttcap%
\pgfsetroundjoin%
\definecolor{currentfill}{rgb}{0.282290,0.145912,0.461510}%
\pgfsetfillcolor{currentfill}%
\pgfsetfillopacity{0.700000}%
\pgfsetlinewidth{0.000000pt}%
\definecolor{currentstroke}{rgb}{0.000000,0.000000,0.000000}%
\pgfsetstrokecolor{currentstroke}%
\pgfsetdash{}{0pt}%
\pgfpathmoveto{\pgfqpoint{4.683935in}{1.793123in}}%
\pgfpathlineto{\pgfqpoint{4.697622in}{1.794504in}}%
\pgfpathlineto{\pgfqpoint{4.711321in}{1.796000in}}%
\pgfpathlineto{\pgfqpoint{4.725030in}{1.797609in}}%
\pgfpathlineto{\pgfqpoint{4.738750in}{1.799331in}}%
\pgfpathlineto{\pgfqpoint{4.746339in}{1.811095in}}%
\pgfpathlineto{\pgfqpoint{4.753922in}{1.822854in}}%
\pgfpathlineto{\pgfqpoint{4.761501in}{1.834606in}}%
\pgfpathlineto{\pgfqpoint{4.769075in}{1.846349in}}%
\pgfpathlineto{\pgfqpoint{4.755359in}{1.844411in}}%
\pgfpathlineto{\pgfqpoint{4.741653in}{1.842587in}}%
\pgfpathlineto{\pgfqpoint{4.727958in}{1.840876in}}%
\pgfpathlineto{\pgfqpoint{4.714274in}{1.839279in}}%
\pgfpathlineto{\pgfqpoint{4.706696in}{1.827745in}}%
\pgfpathlineto{\pgfqpoint{4.699114in}{1.816207in}}%
\pgfpathlineto{\pgfqpoint{4.691526in}{1.804666in}}%
\pgfpathlineto{\pgfqpoint{4.683935in}{1.793123in}}%
\pgfpathclose%
\pgfusepath{fill}%
\end{pgfscope}%
\begin{pgfscope}%
\pgfpathrectangle{\pgfqpoint{1.254980in}{0.150000in}}{\pgfqpoint{5.490039in}{5.490039in}}%
\pgfusepath{clip}%
\pgfsetbuttcap%
\pgfsetroundjoin%
\definecolor{currentfill}{rgb}{0.277018,0.050344,0.375715}%
\pgfsetfillcolor{currentfill}%
\pgfsetfillopacity{0.700000}%
\pgfsetlinewidth{0.000000pt}%
\definecolor{currentstroke}{rgb}{0.000000,0.000000,0.000000}%
\pgfsetstrokecolor{currentstroke}%
\pgfsetdash{}{0pt}%
\pgfpathmoveto{\pgfqpoint{4.343841in}{1.619404in}}%
\pgfpathlineto{\pgfqpoint{4.357400in}{1.617934in}}%
\pgfpathlineto{\pgfqpoint{4.370968in}{1.616580in}}%
\pgfpathlineto{\pgfqpoint{4.384545in}{1.615342in}}%
\pgfpathlineto{\pgfqpoint{4.398129in}{1.614219in}}%
\pgfpathlineto{\pgfqpoint{4.405811in}{1.624686in}}%
\pgfpathlineto{\pgfqpoint{4.413489in}{1.635195in}}%
\pgfpathlineto{\pgfqpoint{4.421161in}{1.645742in}}%
\pgfpathlineto{\pgfqpoint{4.428829in}{1.656325in}}%
\pgfpathlineto{\pgfqpoint{4.415251in}{1.657170in}}%
\pgfpathlineto{\pgfqpoint{4.401682in}{1.658131in}}%
\pgfpathlineto{\pgfqpoint{4.388122in}{1.659207in}}%
\pgfpathlineto{\pgfqpoint{4.374570in}{1.660399in}}%
\pgfpathlineto{\pgfqpoint{4.366895in}{1.650087in}}%
\pgfpathlineto{\pgfqpoint{4.359215in}{1.639816in}}%
\pgfpathlineto{\pgfqpoint{4.351530in}{1.629588in}}%
\pgfpathlineto{\pgfqpoint{4.343841in}{1.619404in}}%
\pgfpathclose%
\pgfusepath{fill}%
\end{pgfscope}%
\begin{pgfscope}%
\pgfpathrectangle{\pgfqpoint{1.254980in}{0.150000in}}{\pgfqpoint{5.490039in}{5.490039in}}%
\pgfusepath{clip}%
\pgfsetbuttcap%
\pgfsetroundjoin%
\definecolor{currentfill}{rgb}{0.268510,0.009605,0.335427}%
\pgfsetfillcolor{currentfill}%
\pgfsetfillopacity{0.700000}%
\pgfsetlinewidth{0.000000pt}%
\definecolor{currentstroke}{rgb}{0.000000,0.000000,0.000000}%
\pgfsetstrokecolor{currentstroke}%
\pgfsetdash{}{0pt}%
\pgfpathmoveto{\pgfqpoint{4.034764in}{1.555791in}}%
\pgfpathlineto{\pgfqpoint{4.048241in}{1.551513in}}%
\pgfpathlineto{\pgfqpoint{4.061724in}{1.547354in}}%
\pgfpathlineto{\pgfqpoint{4.075213in}{1.543314in}}%
\pgfpathlineto{\pgfqpoint{4.088709in}{1.539394in}}%
\pgfpathlineto{\pgfqpoint{4.096492in}{1.547865in}}%
\pgfpathlineto{\pgfqpoint{4.104269in}{1.556422in}}%
\pgfpathlineto{\pgfqpoint{4.112041in}{1.565062in}}%
\pgfpathlineto{\pgfqpoint{4.119807in}{1.573782in}}%
\pgfpathlineto{\pgfqpoint{4.106324in}{1.577378in}}%
\pgfpathlineto{\pgfqpoint{4.092848in}{1.581092in}}%
\pgfpathlineto{\pgfqpoint{4.079378in}{1.584926in}}%
\pgfpathlineto{\pgfqpoint{4.065914in}{1.588879in}}%
\pgfpathlineto{\pgfqpoint{4.058135in}{1.580478in}}%
\pgfpathlineto{\pgfqpoint{4.050351in}{1.572161in}}%
\pgfpathlineto{\pgfqpoint{4.042560in}{1.563931in}}%
\pgfpathlineto{\pgfqpoint{4.034764in}{1.555791in}}%
\pgfpathclose%
\pgfusepath{fill}%
\end{pgfscope}%
\begin{pgfscope}%
\pgfpathrectangle{\pgfqpoint{1.254980in}{0.150000in}}{\pgfqpoint{5.490039in}{5.490039in}}%
\pgfusepath{clip}%
\pgfsetbuttcap%
\pgfsetroundjoin%
\definecolor{currentfill}{rgb}{0.283197,0.115680,0.436115}%
\pgfsetfillcolor{currentfill}%
\pgfsetfillopacity{0.700000}%
\pgfsetlinewidth{0.000000pt}%
\definecolor{currentstroke}{rgb}{0.000000,0.000000,0.000000}%
\pgfsetstrokecolor{currentstroke}%
\pgfsetdash{}{0pt}%
\pgfpathmoveto{\pgfqpoint{3.370557in}{1.771978in}}%
\pgfpathlineto{\pgfqpoint{3.383982in}{1.761134in}}%
\pgfpathlineto{\pgfqpoint{3.397407in}{1.750427in}}%
\pgfpathlineto{\pgfqpoint{3.410833in}{1.739856in}}%
\pgfpathlineto{\pgfqpoint{3.424259in}{1.729421in}}%
\pgfpathlineto{\pgfqpoint{3.432361in}{1.732148in}}%
\pgfpathlineto{\pgfqpoint{3.440453in}{1.735050in}}%
\pgfpathlineto{\pgfqpoint{3.448534in}{1.738123in}}%
\pgfpathlineto{\pgfqpoint{3.456606in}{1.741365in}}%
\pgfpathlineto{\pgfqpoint{3.443206in}{1.751403in}}%
\pgfpathlineto{\pgfqpoint{3.429808in}{1.761577in}}%
\pgfpathlineto{\pgfqpoint{3.416410in}{1.771887in}}%
\pgfpathlineto{\pgfqpoint{3.403014in}{1.782333in}}%
\pgfpathlineto{\pgfqpoint{3.394915in}{1.779482in}}%
\pgfpathlineto{\pgfqpoint{3.386806in}{1.776803in}}%
\pgfpathlineto{\pgfqpoint{3.378687in}{1.774301in}}%
\pgfpathlineto{\pgfqpoint{3.370557in}{1.771978in}}%
\pgfpathclose%
\pgfusepath{fill}%
\end{pgfscope}%
\begin{pgfscope}%
\pgfpathrectangle{\pgfqpoint{1.254980in}{0.150000in}}{\pgfqpoint{5.490039in}{5.490039in}}%
\pgfusepath{clip}%
\pgfsetbuttcap%
\pgfsetroundjoin%
\definecolor{currentfill}{rgb}{0.128729,0.563265,0.551229}%
\pgfsetfillcolor{currentfill}%
\pgfsetfillopacity{0.700000}%
\pgfsetlinewidth{0.000000pt}%
\definecolor{currentstroke}{rgb}{0.000000,0.000000,0.000000}%
\pgfsetstrokecolor{currentstroke}%
\pgfsetdash{}{0pt}%
\pgfpathmoveto{\pgfqpoint{2.452069in}{2.869349in}}%
\pgfpathlineto{\pgfqpoint{2.465768in}{2.847388in}}%
\pgfpathlineto{\pgfqpoint{2.479458in}{2.825625in}}%
\pgfpathlineto{\pgfqpoint{2.493139in}{2.804059in}}%
\pgfpathlineto{\pgfqpoint{2.506811in}{2.782688in}}%
\pgfpathlineto{\pgfqpoint{2.515557in}{2.778275in}}%
\pgfpathlineto{\pgfqpoint{2.524284in}{2.774117in}}%
\pgfpathlineto{\pgfqpoint{2.532994in}{2.770210in}}%
\pgfpathlineto{\pgfqpoint{2.541686in}{2.766552in}}%
\pgfpathlineto{\pgfqpoint{2.528063in}{2.787486in}}%
\pgfpathlineto{\pgfqpoint{2.514431in}{2.808614in}}%
\pgfpathlineto{\pgfqpoint{2.500790in}{2.829938in}}%
\pgfpathlineto{\pgfqpoint{2.487140in}{2.851459in}}%
\pgfpathlineto{\pgfqpoint{2.478400in}{2.855548in}}%
\pgfpathlineto{\pgfqpoint{2.469642in}{2.859890in}}%
\pgfpathlineto{\pgfqpoint{2.460865in}{2.864489in}}%
\pgfpathlineto{\pgfqpoint{2.452069in}{2.869349in}}%
\pgfpathclose%
\pgfusepath{fill}%
\end{pgfscope}%
\begin{pgfscope}%
\pgfpathrectangle{\pgfqpoint{1.254980in}{0.150000in}}{\pgfqpoint{5.490039in}{5.490039in}}%
\pgfusepath{clip}%
\pgfsetbuttcap%
\pgfsetroundjoin%
\definecolor{currentfill}{rgb}{0.278826,0.175490,0.483397}%
\pgfsetfillcolor{currentfill}%
\pgfsetfillopacity{0.700000}%
\pgfsetlinewidth{0.000000pt}%
\definecolor{currentstroke}{rgb}{0.000000,0.000000,0.000000}%
\pgfsetstrokecolor{currentstroke}%
\pgfsetdash{}{0pt}%
\pgfpathmoveto{\pgfqpoint{4.769075in}{1.846349in}}%
\pgfpathlineto{\pgfqpoint{4.782803in}{1.848400in}}%
\pgfpathlineto{\pgfqpoint{4.796542in}{1.850564in}}%
\pgfpathlineto{\pgfqpoint{4.810293in}{1.852842in}}%
\pgfpathlineto{\pgfqpoint{4.824055in}{1.855233in}}%
\pgfpathlineto{\pgfqpoint{4.831621in}{1.867171in}}%
\pgfpathlineto{\pgfqpoint{4.839183in}{1.879094in}}%
\pgfpathlineto{\pgfqpoint{4.846741in}{1.891000in}}%
\pgfpathlineto{\pgfqpoint{4.854293in}{1.902888in}}%
\pgfpathlineto{\pgfqpoint{4.840534in}{1.900297in}}%
\pgfpathlineto{\pgfqpoint{4.826787in}{1.897820in}}%
\pgfpathlineto{\pgfqpoint{4.813051in}{1.895455in}}%
\pgfpathlineto{\pgfqpoint{4.799326in}{1.893205in}}%
\pgfpathlineto{\pgfqpoint{4.791771in}{1.881511in}}%
\pgfpathlineto{\pgfqpoint{4.784210in}{1.869803in}}%
\pgfpathlineto{\pgfqpoint{4.776645in}{1.858082in}}%
\pgfpathlineto{\pgfqpoint{4.769075in}{1.846349in}}%
\pgfpathclose%
\pgfusepath{fill}%
\end{pgfscope}%
\begin{pgfscope}%
\pgfpathrectangle{\pgfqpoint{1.254980in}{0.150000in}}{\pgfqpoint{5.490039in}{5.490039in}}%
\pgfusepath{clip}%
\pgfsetbuttcap%
\pgfsetroundjoin%
\definecolor{currentfill}{rgb}{0.273809,0.031497,0.358853}%
\pgfsetfillcolor{currentfill}%
\pgfsetfillopacity{0.700000}%
\pgfsetlinewidth{0.000000pt}%
\definecolor{currentstroke}{rgb}{0.000000,0.000000,0.000000}%
\pgfsetstrokecolor{currentstroke}%
\pgfsetdash{}{0pt}%
\pgfpathmoveto{\pgfqpoint{4.258841in}{1.587373in}}%
\pgfpathlineto{\pgfqpoint{4.272378in}{1.585146in}}%
\pgfpathlineto{\pgfqpoint{4.285922in}{1.583034in}}%
\pgfpathlineto{\pgfqpoint{4.299474in}{1.581039in}}%
\pgfpathlineto{\pgfqpoint{4.313033in}{1.579160in}}%
\pgfpathlineto{\pgfqpoint{4.320743in}{1.589143in}}%
\pgfpathlineto{\pgfqpoint{4.328447in}{1.599179in}}%
\pgfpathlineto{\pgfqpoint{4.336146in}{1.609267in}}%
\pgfpathlineto{\pgfqpoint{4.343841in}{1.619404in}}%
\pgfpathlineto{\pgfqpoint{4.330289in}{1.620989in}}%
\pgfpathlineto{\pgfqpoint{4.316746in}{1.622691in}}%
\pgfpathlineto{\pgfqpoint{4.303211in}{1.624509in}}%
\pgfpathlineto{\pgfqpoint{4.289684in}{1.626443in}}%
\pgfpathlineto{\pgfqpoint{4.281981in}{1.616594in}}%
\pgfpathlineto{\pgfqpoint{4.274273in}{1.606797in}}%
\pgfpathlineto{\pgfqpoint{4.266559in}{1.597057in}}%
\pgfpathlineto{\pgfqpoint{4.258841in}{1.587373in}}%
\pgfpathclose%
\pgfusepath{fill}%
\end{pgfscope}%
\begin{pgfscope}%
\pgfpathrectangle{\pgfqpoint{1.254980in}{0.150000in}}{\pgfqpoint{5.490039in}{5.490039in}}%
\pgfusepath{clip}%
\pgfsetbuttcap%
\pgfsetroundjoin%
\definecolor{currentfill}{rgb}{0.276022,0.044167,0.370164}%
\pgfsetfillcolor{currentfill}%
\pgfsetfillopacity{0.700000}%
\pgfsetlinewidth{0.000000pt}%
\definecolor{currentstroke}{rgb}{0.000000,0.000000,0.000000}%
\pgfsetstrokecolor{currentstroke}%
\pgfsetdash{}{0pt}%
\pgfpathmoveto{\pgfqpoint{3.617535in}{1.631272in}}%
\pgfpathlineto{\pgfqpoint{3.630961in}{1.622947in}}%
\pgfpathlineto{\pgfqpoint{3.644389in}{1.614750in}}%
\pgfpathlineto{\pgfqpoint{3.657820in}{1.606681in}}%
\pgfpathlineto{\pgfqpoint{3.671254in}{1.598740in}}%
\pgfpathlineto{\pgfqpoint{3.679221in}{1.603682in}}%
\pgfpathlineto{\pgfqpoint{3.687180in}{1.608770in}}%
\pgfpathlineto{\pgfqpoint{3.695130in}{1.613999in}}%
\pgfpathlineto{\pgfqpoint{3.703073in}{1.619366in}}%
\pgfpathlineto{\pgfqpoint{3.689660in}{1.626931in}}%
\pgfpathlineto{\pgfqpoint{3.676251in}{1.634624in}}%
\pgfpathlineto{\pgfqpoint{3.662845in}{1.642444in}}%
\pgfpathlineto{\pgfqpoint{3.649441in}{1.650393in}}%
\pgfpathlineto{\pgfqpoint{3.641477in}{1.645396in}}%
\pgfpathlineto{\pgfqpoint{3.633505in}{1.640541in}}%
\pgfpathlineto{\pgfqpoint{3.625525in}{1.635832in}}%
\pgfpathlineto{\pgfqpoint{3.617535in}{1.631272in}}%
\pgfpathclose%
\pgfusepath{fill}%
\end{pgfscope}%
\begin{pgfscope}%
\pgfpathrectangle{\pgfqpoint{1.254980in}{0.150000in}}{\pgfqpoint{5.490039in}{5.490039in}}%
\pgfusepath{clip}%
\pgfsetbuttcap%
\pgfsetroundjoin%
\definecolor{currentfill}{rgb}{0.233603,0.313828,0.543914}%
\pgfsetfillcolor{currentfill}%
\pgfsetfillopacity{0.700000}%
\pgfsetlinewidth{0.000000pt}%
\definecolor{currentstroke}{rgb}{0.000000,0.000000,0.000000}%
\pgfsetstrokecolor{currentstroke}%
\pgfsetdash{}{0pt}%
\pgfpathmoveto{\pgfqpoint{5.140410in}{2.137112in}}%
\pgfpathlineto{\pgfqpoint{5.154323in}{2.141811in}}%
\pgfpathlineto{\pgfqpoint{5.168249in}{2.146623in}}%
\pgfpathlineto{\pgfqpoint{5.182188in}{2.151547in}}%
\pgfpathlineto{\pgfqpoint{5.196142in}{2.156584in}}%
\pgfpathlineto{\pgfqpoint{5.203599in}{2.168475in}}%
\pgfpathlineto{\pgfqpoint{5.211052in}{2.180311in}}%
\pgfpathlineto{\pgfqpoint{5.218498in}{2.192092in}}%
\pgfpathlineto{\pgfqpoint{5.225940in}{2.203817in}}%
\pgfpathlineto{\pgfqpoint{5.211988in}{2.198660in}}%
\pgfpathlineto{\pgfqpoint{5.198051in}{2.193616in}}%
\pgfpathlineto{\pgfqpoint{5.184127in}{2.188684in}}%
\pgfpathlineto{\pgfqpoint{5.170217in}{2.183866in}}%
\pgfpathlineto{\pgfqpoint{5.162774in}{2.172255in}}%
\pgfpathlineto{\pgfqpoint{5.155325in}{2.160591in}}%
\pgfpathlineto{\pgfqpoint{5.147870in}{2.148877in}}%
\pgfpathlineto{\pgfqpoint{5.140410in}{2.137112in}}%
\pgfpathclose%
\pgfusepath{fill}%
\end{pgfscope}%
\begin{pgfscope}%
\pgfpathrectangle{\pgfqpoint{1.254980in}{0.150000in}}{\pgfqpoint{5.490039in}{5.490039in}}%
\pgfusepath{clip}%
\pgfsetbuttcap%
\pgfsetroundjoin%
\definecolor{currentfill}{rgb}{0.271828,0.209303,0.504434}%
\pgfsetfillcolor{currentfill}%
\pgfsetfillopacity{0.700000}%
\pgfsetlinewidth{0.000000pt}%
\definecolor{currentstroke}{rgb}{0.000000,0.000000,0.000000}%
\pgfsetstrokecolor{currentstroke}%
\pgfsetdash{}{0pt}%
\pgfpathmoveto{\pgfqpoint{4.854293in}{1.902888in}}%
\pgfpathlineto{\pgfqpoint{4.868064in}{1.905592in}}%
\pgfpathlineto{\pgfqpoint{4.881846in}{1.908408in}}%
\pgfpathlineto{\pgfqpoint{4.895641in}{1.911338in}}%
\pgfpathlineto{\pgfqpoint{4.909447in}{1.914380in}}%
\pgfpathlineto{\pgfqpoint{4.916992in}{1.926438in}}%
\pgfpathlineto{\pgfqpoint{4.924533in}{1.938470in}}%
\pgfpathlineto{\pgfqpoint{4.932069in}{1.950476in}}%
\pgfpathlineto{\pgfqpoint{4.939600in}{1.962454in}}%
\pgfpathlineto{\pgfqpoint{4.925796in}{1.959227in}}%
\pgfpathlineto{\pgfqpoint{4.912004in}{1.956113in}}%
\pgfpathlineto{\pgfqpoint{4.898224in}{1.953112in}}%
\pgfpathlineto{\pgfqpoint{4.884456in}{1.950224in}}%
\pgfpathlineto{\pgfqpoint{4.876923in}{1.938425in}}%
\pgfpathlineto{\pgfqpoint{4.869384in}{1.926601in}}%
\pgfpathlineto{\pgfqpoint{4.861841in}{1.914755in}}%
\pgfpathlineto{\pgfqpoint{4.854293in}{1.902888in}}%
\pgfpathclose%
\pgfusepath{fill}%
\end{pgfscope}%
\begin{pgfscope}%
\pgfpathrectangle{\pgfqpoint{1.254980in}{0.150000in}}{\pgfqpoint{5.490039in}{5.490039in}}%
\pgfusepath{clip}%
\pgfsetbuttcap%
\pgfsetroundjoin%
\definecolor{currentfill}{rgb}{0.233603,0.313828,0.543914}%
\pgfsetfillcolor{currentfill}%
\pgfsetfillopacity{0.700000}%
\pgfsetlinewidth{0.000000pt}%
\definecolor{currentstroke}{rgb}{0.000000,0.000000,0.000000}%
\pgfsetstrokecolor{currentstroke}%
\pgfsetdash{}{0pt}%
\pgfpathmoveto{\pgfqpoint{2.906855in}{2.201039in}}%
\pgfpathlineto{\pgfqpoint{2.920364in}{2.185044in}}%
\pgfpathlineto{\pgfqpoint{2.933869in}{2.169209in}}%
\pgfpathlineto{\pgfqpoint{2.947370in}{2.153534in}}%
\pgfpathlineto{\pgfqpoint{2.960868in}{2.138016in}}%
\pgfpathlineto{\pgfqpoint{2.969283in}{2.136586in}}%
\pgfpathlineto{\pgfqpoint{2.977684in}{2.135382in}}%
\pgfpathlineto{\pgfqpoint{2.986071in}{2.134400in}}%
\pgfpathlineto{\pgfqpoint{2.994444in}{2.133636in}}%
\pgfpathlineto{\pgfqpoint{2.980984in}{2.148725in}}%
\pgfpathlineto{\pgfqpoint{2.967521in}{2.163971in}}%
\pgfpathlineto{\pgfqpoint{2.954055in}{2.179376in}}%
\pgfpathlineto{\pgfqpoint{2.940586in}{2.194939in}}%
\pgfpathlineto{\pgfqpoint{2.932175in}{2.196126in}}%
\pgfpathlineto{\pgfqpoint{2.923749in}{2.197536in}}%
\pgfpathlineto{\pgfqpoint{2.915310in}{2.199172in}}%
\pgfpathlineto{\pgfqpoint{2.906855in}{2.201039in}}%
\pgfpathclose%
\pgfusepath{fill}%
\end{pgfscope}%
\begin{pgfscope}%
\pgfpathrectangle{\pgfqpoint{1.254980in}{0.150000in}}{\pgfqpoint{5.490039in}{5.490039in}}%
\pgfusepath{clip}%
\pgfsetbuttcap%
\pgfsetroundjoin%
\definecolor{currentfill}{rgb}{0.185556,0.418570,0.556753}%
\pgfsetfillcolor{currentfill}%
\pgfsetfillopacity{0.700000}%
\pgfsetlinewidth{0.000000pt}%
\definecolor{currentstroke}{rgb}{0.000000,0.000000,0.000000}%
\pgfsetstrokecolor{currentstroke}%
\pgfsetdash{}{0pt}%
\pgfpathmoveto{\pgfqpoint{5.426838in}{2.387335in}}%
\pgfpathlineto{\pgfqpoint{5.440910in}{2.393768in}}%
\pgfpathlineto{\pgfqpoint{5.454998in}{2.400313in}}%
\pgfpathlineto{\pgfqpoint{5.469101in}{2.406971in}}%
\pgfpathlineto{\pgfqpoint{5.483220in}{2.413742in}}%
\pgfpathlineto{\pgfqpoint{5.490574in}{2.424870in}}%
\pgfpathlineto{\pgfqpoint{5.497923in}{2.435923in}}%
\pgfpathlineto{\pgfqpoint{5.505264in}{2.446900in}}%
\pgfpathlineto{\pgfqpoint{5.512599in}{2.457803in}}%
\pgfpathlineto{\pgfqpoint{5.498485in}{2.450978in}}%
\pgfpathlineto{\pgfqpoint{5.484385in}{2.444266in}}%
\pgfpathlineto{\pgfqpoint{5.470301in}{2.437667in}}%
\pgfpathlineto{\pgfqpoint{5.456232in}{2.431181in}}%
\pgfpathlineto{\pgfqpoint{5.448893in}{2.420326in}}%
\pgfpathlineto{\pgfqpoint{5.441547in}{2.409400in}}%
\pgfpathlineto{\pgfqpoint{5.434196in}{2.398403in}}%
\pgfpathlineto{\pgfqpoint{5.426838in}{2.387335in}}%
\pgfpathclose%
\pgfusepath{fill}%
\end{pgfscope}%
\begin{pgfscope}%
\pgfpathrectangle{\pgfqpoint{1.254980in}{0.150000in}}{\pgfqpoint{5.490039in}{5.490039in}}%
\pgfusepath{clip}%
\pgfsetbuttcap%
\pgfsetroundjoin%
\definecolor{currentfill}{rgb}{0.220057,0.343307,0.549413}%
\pgfsetfillcolor{currentfill}%
\pgfsetfillopacity{0.700000}%
\pgfsetlinewidth{0.000000pt}%
\definecolor{currentstroke}{rgb}{0.000000,0.000000,0.000000}%
\pgfsetstrokecolor{currentstroke}%
\pgfsetdash{}{0pt}%
\pgfpathmoveto{\pgfqpoint{2.852778in}{2.266632in}}%
\pgfpathlineto{\pgfqpoint{2.866304in}{2.249989in}}%
\pgfpathlineto{\pgfqpoint{2.879825in}{2.233510in}}%
\pgfpathlineto{\pgfqpoint{2.893342in}{2.217194in}}%
\pgfpathlineto{\pgfqpoint{2.906855in}{2.201039in}}%
\pgfpathlineto{\pgfqpoint{2.915310in}{2.199172in}}%
\pgfpathlineto{\pgfqpoint{2.923749in}{2.197536in}}%
\pgfpathlineto{\pgfqpoint{2.932175in}{2.196126in}}%
\pgfpathlineto{\pgfqpoint{2.940586in}{2.194939in}}%
\pgfpathlineto{\pgfqpoint{2.927113in}{2.210663in}}%
\pgfpathlineto{\pgfqpoint{2.913636in}{2.226547in}}%
\pgfpathlineto{\pgfqpoint{2.900155in}{2.242593in}}%
\pgfpathlineto{\pgfqpoint{2.886670in}{2.258801in}}%
\pgfpathlineto{\pgfqpoint{2.878219in}{2.260414in}}%
\pgfpathlineto{\pgfqpoint{2.869754in}{2.262253in}}%
\pgfpathlineto{\pgfqpoint{2.861274in}{2.264325in}}%
\pgfpathlineto{\pgfqpoint{2.852778in}{2.266632in}}%
\pgfpathclose%
\pgfusepath{fill}%
\end{pgfscope}%
\begin{pgfscope}%
\pgfpathrectangle{\pgfqpoint{1.254980in}{0.150000in}}{\pgfqpoint{5.490039in}{5.490039in}}%
\pgfusepath{clip}%
\pgfsetbuttcap%
\pgfsetroundjoin%
\definecolor{currentfill}{rgb}{0.244972,0.287675,0.537260}%
\pgfsetfillcolor{currentfill}%
\pgfsetfillopacity{0.700000}%
\pgfsetlinewidth{0.000000pt}%
\definecolor{currentstroke}{rgb}{0.000000,0.000000,0.000000}%
\pgfsetstrokecolor{currentstroke}%
\pgfsetdash{}{0pt}%
\pgfpathmoveto{\pgfqpoint{2.960868in}{2.138016in}}%
\pgfpathlineto{\pgfqpoint{2.974363in}{2.122655in}}%
\pgfpathlineto{\pgfqpoint{2.987854in}{2.107451in}}%
\pgfpathlineto{\pgfqpoint{3.001342in}{2.092402in}}%
\pgfpathlineto{\pgfqpoint{3.014828in}{2.077508in}}%
\pgfpathlineto{\pgfqpoint{3.023205in}{2.076513in}}%
\pgfpathlineto{\pgfqpoint{3.031567in}{2.075739in}}%
\pgfpathlineto{\pgfqpoint{3.039917in}{2.075183in}}%
\pgfpathlineto{\pgfqpoint{3.048253in}{2.074840in}}%
\pgfpathlineto{\pgfqpoint{3.034805in}{2.089307in}}%
\pgfpathlineto{\pgfqpoint{3.021354in}{2.103928in}}%
\pgfpathlineto{\pgfqpoint{3.007900in}{2.118704in}}%
\pgfpathlineto{\pgfqpoint{2.994444in}{2.133636in}}%
\pgfpathlineto{\pgfqpoint{2.986071in}{2.134400in}}%
\pgfpathlineto{\pgfqpoint{2.977684in}{2.135382in}}%
\pgfpathlineto{\pgfqpoint{2.969283in}{2.136586in}}%
\pgfpathlineto{\pgfqpoint{2.960868in}{2.138016in}}%
\pgfpathclose%
\pgfusepath{fill}%
\end{pgfscope}%
\begin{pgfscope}%
\pgfpathrectangle{\pgfqpoint{1.254980in}{0.150000in}}{\pgfqpoint{5.490039in}{5.490039in}}%
\pgfusepath{clip}%
\pgfsetbuttcap%
\pgfsetroundjoin%
\definecolor{currentfill}{rgb}{0.282656,0.100196,0.422160}%
\pgfsetfillcolor{currentfill}%
\pgfsetfillopacity{0.700000}%
\pgfsetlinewidth{0.000000pt}%
\definecolor{currentstroke}{rgb}{0.000000,0.000000,0.000000}%
\pgfsetstrokecolor{currentstroke}%
\pgfsetdash{}{0pt}%
\pgfpathmoveto{\pgfqpoint{3.424259in}{1.729421in}}%
\pgfpathlineto{\pgfqpoint{3.437687in}{1.719121in}}%
\pgfpathlineto{\pgfqpoint{3.451116in}{1.708955in}}%
\pgfpathlineto{\pgfqpoint{3.464546in}{1.698924in}}%
\pgfpathlineto{\pgfqpoint{3.477977in}{1.689026in}}%
\pgfpathlineto{\pgfqpoint{3.486052in}{1.692155in}}%
\pgfpathlineto{\pgfqpoint{3.494117in}{1.695455in}}%
\pgfpathlineto{\pgfqpoint{3.502173in}{1.698923in}}%
\pgfpathlineto{\pgfqpoint{3.510219in}{1.702554in}}%
\pgfpathlineto{\pgfqpoint{3.496813in}{1.712056in}}%
\pgfpathlineto{\pgfqpoint{3.483409in}{1.721692in}}%
\pgfpathlineto{\pgfqpoint{3.470007in}{1.731461in}}%
\pgfpathlineto{\pgfqpoint{3.456606in}{1.741365in}}%
\pgfpathlineto{\pgfqpoint{3.448534in}{1.738123in}}%
\pgfpathlineto{\pgfqpoint{3.440453in}{1.735050in}}%
\pgfpathlineto{\pgfqpoint{3.432361in}{1.732148in}}%
\pgfpathlineto{\pgfqpoint{3.424259in}{1.729421in}}%
\pgfpathclose%
\pgfusepath{fill}%
\end{pgfscope}%
\begin{pgfscope}%
\pgfpathrectangle{\pgfqpoint{1.254980in}{0.150000in}}{\pgfqpoint{5.490039in}{5.490039in}}%
\pgfusepath{clip}%
\pgfsetbuttcap%
\pgfsetroundjoin%
\definecolor{currentfill}{rgb}{0.208623,0.367752,0.552675}%
\pgfsetfillcolor{currentfill}%
\pgfsetfillopacity{0.700000}%
\pgfsetlinewidth{0.000000pt}%
\definecolor{currentstroke}{rgb}{0.000000,0.000000,0.000000}%
\pgfsetstrokecolor{currentstroke}%
\pgfsetdash{}{0pt}%
\pgfpathmoveto{\pgfqpoint{2.798628in}{2.334852in}}%
\pgfpathlineto{\pgfqpoint{2.812173in}{2.317547in}}%
\pgfpathlineto{\pgfqpoint{2.825713in}{2.300410in}}%
\pgfpathlineto{\pgfqpoint{2.839248in}{2.283438in}}%
\pgfpathlineto{\pgfqpoint{2.852778in}{2.266632in}}%
\pgfpathlineto{\pgfqpoint{2.861274in}{2.264325in}}%
\pgfpathlineto{\pgfqpoint{2.869754in}{2.262253in}}%
\pgfpathlineto{\pgfqpoint{2.878219in}{2.260414in}}%
\pgfpathlineto{\pgfqpoint{2.886670in}{2.258801in}}%
\pgfpathlineto{\pgfqpoint{2.873181in}{2.275174in}}%
\pgfpathlineto{\pgfqpoint{2.859687in}{2.291710in}}%
\pgfpathlineto{\pgfqpoint{2.846189in}{2.308412in}}%
\pgfpathlineto{\pgfqpoint{2.832686in}{2.325281in}}%
\pgfpathlineto{\pgfqpoint{2.824195in}{2.327321in}}%
\pgfpathlineto{\pgfqpoint{2.815689in}{2.329593in}}%
\pgfpathlineto{\pgfqpoint{2.807166in}{2.332103in}}%
\pgfpathlineto{\pgfqpoint{2.798628in}{2.334852in}}%
\pgfpathclose%
\pgfusepath{fill}%
\end{pgfscope}%
\begin{pgfscope}%
\pgfpathrectangle{\pgfqpoint{1.254980in}{0.150000in}}{\pgfqpoint{5.490039in}{5.490039in}}%
\pgfusepath{clip}%
\pgfsetbuttcap%
\pgfsetroundjoin%
\definecolor{currentfill}{rgb}{0.255645,0.260703,0.528312}%
\pgfsetfillcolor{currentfill}%
\pgfsetfillopacity{0.700000}%
\pgfsetlinewidth{0.000000pt}%
\definecolor{currentstroke}{rgb}{0.000000,0.000000,0.000000}%
\pgfsetstrokecolor{currentstroke}%
\pgfsetdash{}{0pt}%
\pgfpathmoveto{\pgfqpoint{3.014828in}{2.077508in}}%
\pgfpathlineto{\pgfqpoint{3.028310in}{2.062768in}}%
\pgfpathlineto{\pgfqpoint{3.041790in}{2.048181in}}%
\pgfpathlineto{\pgfqpoint{3.055267in}{2.033746in}}%
\pgfpathlineto{\pgfqpoint{3.068742in}{2.019463in}}%
\pgfpathlineto{\pgfqpoint{3.077082in}{2.018901in}}%
\pgfpathlineto{\pgfqpoint{3.085408in}{2.018555in}}%
\pgfpathlineto{\pgfqpoint{3.093721in}{2.018422in}}%
\pgfpathlineto{\pgfqpoint{3.102022in}{2.018497in}}%
\pgfpathlineto{\pgfqpoint{3.088583in}{2.032355in}}%
\pgfpathlineto{\pgfqpoint{3.075142in}{2.046365in}}%
\pgfpathlineto{\pgfqpoint{3.061698in}{2.060526in}}%
\pgfpathlineto{\pgfqpoint{3.048253in}{2.074840in}}%
\pgfpathlineto{\pgfqpoint{3.039917in}{2.075183in}}%
\pgfpathlineto{\pgfqpoint{3.031567in}{2.075739in}}%
\pgfpathlineto{\pgfqpoint{3.023205in}{2.076513in}}%
\pgfpathlineto{\pgfqpoint{3.014828in}{2.077508in}}%
\pgfpathclose%
\pgfusepath{fill}%
\end{pgfscope}%
\begin{pgfscope}%
\pgfpathrectangle{\pgfqpoint{1.254980in}{0.150000in}}{\pgfqpoint{5.490039in}{5.490039in}}%
\pgfusepath{clip}%
\pgfsetbuttcap%
\pgfsetroundjoin%
\definecolor{currentfill}{rgb}{0.271305,0.019942,0.347269}%
\pgfsetfillcolor{currentfill}%
\pgfsetfillopacity{0.700000}%
\pgfsetlinewidth{0.000000pt}%
\definecolor{currentstroke}{rgb}{0.000000,0.000000,0.000000}%
\pgfsetstrokecolor{currentstroke}%
\pgfsetdash{}{0pt}%
\pgfpathmoveto{\pgfqpoint{4.173806in}{1.560583in}}%
\pgfpathlineto{\pgfqpoint{4.187323in}{1.557578in}}%
\pgfpathlineto{\pgfqpoint{4.200847in}{1.554690in}}%
\pgfpathlineto{\pgfqpoint{4.214378in}{1.551920in}}%
\pgfpathlineto{\pgfqpoint{4.227917in}{1.549266in}}%
\pgfpathlineto{\pgfqpoint{4.235656in}{1.558694in}}%
\pgfpathlineto{\pgfqpoint{4.243389in}{1.568190in}}%
\pgfpathlineto{\pgfqpoint{4.251118in}{1.577750in}}%
\pgfpathlineto{\pgfqpoint{4.258841in}{1.587373in}}%
\pgfpathlineto{\pgfqpoint{4.245313in}{1.589718in}}%
\pgfpathlineto{\pgfqpoint{4.231791in}{1.592179in}}%
\pgfpathlineto{\pgfqpoint{4.218278in}{1.594758in}}%
\pgfpathlineto{\pgfqpoint{4.204771in}{1.597454in}}%
\pgfpathlineto{\pgfqpoint{4.197038in}{1.588134in}}%
\pgfpathlineto{\pgfqpoint{4.189299in}{1.578881in}}%
\pgfpathlineto{\pgfqpoint{4.181555in}{1.569696in}}%
\pgfpathlineto{\pgfqpoint{4.173806in}{1.560583in}}%
\pgfpathclose%
\pgfusepath{fill}%
\end{pgfscope}%
\begin{pgfscope}%
\pgfpathrectangle{\pgfqpoint{1.254980in}{0.150000in}}{\pgfqpoint{5.490039in}{5.490039in}}%
\pgfusepath{clip}%
\pgfsetbuttcap%
\pgfsetroundjoin%
\definecolor{currentfill}{rgb}{0.120565,0.596422,0.543611}%
\pgfsetfillcolor{currentfill}%
\pgfsetfillopacity{0.700000}%
\pgfsetlinewidth{0.000000pt}%
\definecolor{currentstroke}{rgb}{0.000000,0.000000,0.000000}%
\pgfsetstrokecolor{currentstroke}%
\pgfsetdash{}{0pt}%
\pgfpathmoveto{\pgfqpoint{2.397173in}{2.959203in}}%
\pgfpathlineto{\pgfqpoint{2.410912in}{2.936435in}}%
\pgfpathlineto{\pgfqpoint{2.424641in}{2.913871in}}%
\pgfpathlineto{\pgfqpoint{2.438360in}{2.891509in}}%
\pgfpathlineto{\pgfqpoint{2.452069in}{2.869349in}}%
\pgfpathlineto{\pgfqpoint{2.460865in}{2.864489in}}%
\pgfpathlineto{\pgfqpoint{2.469642in}{2.859890in}}%
\pgfpathlineto{\pgfqpoint{2.478400in}{2.855548in}}%
\pgfpathlineto{\pgfqpoint{2.487140in}{2.851459in}}%
\pgfpathlineto{\pgfqpoint{2.473481in}{2.873179in}}%
\pgfpathlineto{\pgfqpoint{2.459813in}{2.895099in}}%
\pgfpathlineto{\pgfqpoint{2.446135in}{2.917220in}}%
\pgfpathlineto{\pgfqpoint{2.432447in}{2.939544in}}%
\pgfpathlineto{\pgfqpoint{2.423657in}{2.944068in}}%
\pgfpathlineto{\pgfqpoint{2.414849in}{2.948849in}}%
\pgfpathlineto{\pgfqpoint{2.406021in}{2.953893in}}%
\pgfpathlineto{\pgfqpoint{2.397173in}{2.959203in}}%
\pgfpathclose%
\pgfusepath{fill}%
\end{pgfscope}%
\begin{pgfscope}%
\pgfpathrectangle{\pgfqpoint{1.254980in}{0.150000in}}{\pgfqpoint{5.490039in}{5.490039in}}%
\pgfusepath{clip}%
\pgfsetbuttcap%
\pgfsetroundjoin%
\definecolor{currentfill}{rgb}{0.269944,0.014625,0.341379}%
\pgfsetfillcolor{currentfill}%
\pgfsetfillopacity{0.700000}%
\pgfsetlinewidth{0.000000pt}%
\definecolor{currentstroke}{rgb}{0.000000,0.000000,0.000000}%
\pgfsetstrokecolor{currentstroke}%
\pgfsetdash{}{0pt}%
\pgfpathmoveto{\pgfqpoint{3.810505in}{1.563384in}}%
\pgfpathlineto{\pgfqpoint{3.823952in}{1.556947in}}%
\pgfpathlineto{\pgfqpoint{3.837403in}{1.550634in}}%
\pgfpathlineto{\pgfqpoint{3.850858in}{1.544444in}}%
\pgfpathlineto{\pgfqpoint{3.864318in}{1.538377in}}%
\pgfpathlineto{\pgfqpoint{3.872197in}{1.544975in}}%
\pgfpathlineto{\pgfqpoint{3.880068in}{1.551692in}}%
\pgfpathlineto{\pgfqpoint{3.887933in}{1.558526in}}%
\pgfpathlineto{\pgfqpoint{3.895790in}{1.565475in}}%
\pgfpathlineto{\pgfqpoint{3.882347in}{1.571184in}}%
\pgfpathlineto{\pgfqpoint{3.868909in}{1.577016in}}%
\pgfpathlineto{\pgfqpoint{3.855475in}{1.582971in}}%
\pgfpathlineto{\pgfqpoint{3.842046in}{1.589049in}}%
\pgfpathlineto{\pgfqpoint{3.834172in}{1.582452in}}%
\pgfpathlineto{\pgfqpoint{3.826290in}{1.575974in}}%
\pgfpathlineto{\pgfqpoint{3.818401in}{1.569617in}}%
\pgfpathlineto{\pgfqpoint{3.810505in}{1.563384in}}%
\pgfpathclose%
\pgfusepath{fill}%
\end{pgfscope}%
\begin{pgfscope}%
\pgfpathrectangle{\pgfqpoint{1.254980in}{0.150000in}}{\pgfqpoint{5.490039in}{5.490039in}}%
\pgfusepath{clip}%
\pgfsetbuttcap%
\pgfsetroundjoin%
\definecolor{currentfill}{rgb}{0.268510,0.009605,0.335427}%
\pgfsetfillcolor{currentfill}%
\pgfsetfillopacity{0.700000}%
\pgfsetlinewidth{0.000000pt}%
\definecolor{currentstroke}{rgb}{0.000000,0.000000,0.000000}%
\pgfsetstrokecolor{currentstroke}%
\pgfsetdash{}{0pt}%
\pgfpathmoveto{\pgfqpoint{3.949612in}{1.543859in}}%
\pgfpathlineto{\pgfqpoint{3.963081in}{1.538758in}}%
\pgfpathlineto{\pgfqpoint{3.976555in}{1.533779in}}%
\pgfpathlineto{\pgfqpoint{3.990034in}{1.528919in}}%
\pgfpathlineto{\pgfqpoint{4.003519in}{1.524180in}}%
\pgfpathlineto{\pgfqpoint{4.011340in}{1.531935in}}%
\pgfpathlineto{\pgfqpoint{4.019154in}{1.539790in}}%
\pgfpathlineto{\pgfqpoint{4.026962in}{1.547743in}}%
\pgfpathlineto{\pgfqpoint{4.034764in}{1.555791in}}%
\pgfpathlineto{\pgfqpoint{4.021293in}{1.560189in}}%
\pgfpathlineto{\pgfqpoint{4.007828in}{1.564708in}}%
\pgfpathlineto{\pgfqpoint{3.994369in}{1.569346in}}%
\pgfpathlineto{\pgfqpoint{3.980915in}{1.574105in}}%
\pgfpathlineto{\pgfqpoint{3.973099in}{1.566393in}}%
\pgfpathlineto{\pgfqpoint{3.965276in}{1.558779in}}%
\pgfpathlineto{\pgfqpoint{3.957448in}{1.551267in}}%
\pgfpathlineto{\pgfqpoint{3.949612in}{1.543859in}}%
\pgfpathclose%
\pgfusepath{fill}%
\end{pgfscope}%
\begin{pgfscope}%
\pgfpathrectangle{\pgfqpoint{1.254980in}{0.150000in}}{\pgfqpoint{5.490039in}{5.490039in}}%
\pgfusepath{clip}%
\pgfsetbuttcap%
\pgfsetroundjoin%
\definecolor{currentfill}{rgb}{0.195860,0.395433,0.555276}%
\pgfsetfillcolor{currentfill}%
\pgfsetfillopacity{0.700000}%
\pgfsetlinewidth{0.000000pt}%
\definecolor{currentstroke}{rgb}{0.000000,0.000000,0.000000}%
\pgfsetstrokecolor{currentstroke}%
\pgfsetdash{}{0pt}%
\pgfpathmoveto{\pgfqpoint{2.744396in}{2.405760in}}%
\pgfpathlineto{\pgfqpoint{2.757962in}{2.387778in}}%
\pgfpathlineto{\pgfqpoint{2.771523in}{2.369966in}}%
\pgfpathlineto{\pgfqpoint{2.785078in}{2.352325in}}%
\pgfpathlineto{\pgfqpoint{2.798628in}{2.334852in}}%
\pgfpathlineto{\pgfqpoint{2.807166in}{2.332103in}}%
\pgfpathlineto{\pgfqpoint{2.815689in}{2.329593in}}%
\pgfpathlineto{\pgfqpoint{2.824195in}{2.327321in}}%
\pgfpathlineto{\pgfqpoint{2.832686in}{2.325281in}}%
\pgfpathlineto{\pgfqpoint{2.819179in}{2.342316in}}%
\pgfpathlineto{\pgfqpoint{2.805666in}{2.359520in}}%
\pgfpathlineto{\pgfqpoint{2.792149in}{2.376893in}}%
\pgfpathlineto{\pgfqpoint{2.778626in}{2.394436in}}%
\pgfpathlineto{\pgfqpoint{2.770093in}{2.396907in}}%
\pgfpathlineto{\pgfqpoint{2.761543in}{2.399616in}}%
\pgfpathlineto{\pgfqpoint{2.752978in}{2.402565in}}%
\pgfpathlineto{\pgfqpoint{2.744396in}{2.405760in}}%
\pgfpathclose%
\pgfusepath{fill}%
\end{pgfscope}%
\begin{pgfscope}%
\pgfpathrectangle{\pgfqpoint{1.254980in}{0.150000in}}{\pgfqpoint{5.490039in}{5.490039in}}%
\pgfusepath{clip}%
\pgfsetbuttcap%
\pgfsetroundjoin%
\definecolor{currentfill}{rgb}{0.263663,0.237631,0.518762}%
\pgfsetfillcolor{currentfill}%
\pgfsetfillopacity{0.700000}%
\pgfsetlinewidth{0.000000pt}%
\definecolor{currentstroke}{rgb}{0.000000,0.000000,0.000000}%
\pgfsetstrokecolor{currentstroke}%
\pgfsetdash{}{0pt}%
\pgfpathmoveto{\pgfqpoint{3.068742in}{2.019463in}}%
\pgfpathlineto{\pgfqpoint{3.082214in}{2.005331in}}%
\pgfpathlineto{\pgfqpoint{3.095685in}{1.991349in}}%
\pgfpathlineto{\pgfqpoint{3.109153in}{1.977517in}}%
\pgfpathlineto{\pgfqpoint{3.122620in}{1.963833in}}%
\pgfpathlineto{\pgfqpoint{3.130923in}{1.963701in}}%
\pgfpathlineto{\pgfqpoint{3.139215in}{1.963781in}}%
\pgfpathlineto{\pgfqpoint{3.147493in}{1.964068in}}%
\pgfpathlineto{\pgfqpoint{3.155759in}{1.964560in}}%
\pgfpathlineto{\pgfqpoint{3.142327in}{1.977821in}}%
\pgfpathlineto{\pgfqpoint{3.128894in}{1.991231in}}%
\pgfpathlineto{\pgfqpoint{3.115458in}{2.004789in}}%
\pgfpathlineto{\pgfqpoint{3.102022in}{2.018497in}}%
\pgfpathlineto{\pgfqpoint{3.093721in}{2.018422in}}%
\pgfpathlineto{\pgfqpoint{3.085408in}{2.018555in}}%
\pgfpathlineto{\pgfqpoint{3.077082in}{2.018901in}}%
\pgfpathlineto{\pgfqpoint{3.068742in}{2.019463in}}%
\pgfpathclose%
\pgfusepath{fill}%
\end{pgfscope}%
\begin{pgfscope}%
\pgfpathrectangle{\pgfqpoint{1.254980in}{0.150000in}}{\pgfqpoint{5.490039in}{5.490039in}}%
\pgfusepath{clip}%
\pgfsetbuttcap%
\pgfsetroundjoin%
\definecolor{currentfill}{rgb}{0.218130,0.347432,0.550038}%
\pgfsetfillcolor{currentfill}%
\pgfsetfillopacity{0.700000}%
\pgfsetlinewidth{0.000000pt}%
\definecolor{currentstroke}{rgb}{0.000000,0.000000,0.000000}%
\pgfsetstrokecolor{currentstroke}%
\pgfsetdash{}{0pt}%
\pgfpathmoveto{\pgfqpoint{5.225940in}{2.203817in}}%
\pgfpathlineto{\pgfqpoint{5.239905in}{2.209086in}}%
\pgfpathlineto{\pgfqpoint{5.253884in}{2.214469in}}%
\pgfpathlineto{\pgfqpoint{5.267877in}{2.219964in}}%
\pgfpathlineto{\pgfqpoint{5.281885in}{2.225571in}}%
\pgfpathlineto{\pgfqpoint{5.289318in}{2.237349in}}%
\pgfpathlineto{\pgfqpoint{5.296745in}{2.249066in}}%
\pgfpathlineto{\pgfqpoint{5.304167in}{2.260720in}}%
\pgfpathlineto{\pgfqpoint{5.311583in}{2.272312in}}%
\pgfpathlineto{\pgfqpoint{5.297578in}{2.266601in}}%
\pgfpathlineto{\pgfqpoint{5.283587in}{2.261002in}}%
\pgfpathlineto{\pgfqpoint{5.269610in}{2.255517in}}%
\pgfpathlineto{\pgfqpoint{5.255648in}{2.250143in}}%
\pgfpathlineto{\pgfqpoint{5.248229in}{2.238649in}}%
\pgfpathlineto{\pgfqpoint{5.240805in}{2.227096in}}%
\pgfpathlineto{\pgfqpoint{5.233375in}{2.215485in}}%
\pgfpathlineto{\pgfqpoint{5.225940in}{2.203817in}}%
\pgfpathclose%
\pgfusepath{fill}%
\end{pgfscope}%
\begin{pgfscope}%
\pgfpathrectangle{\pgfqpoint{1.254980in}{0.150000in}}{\pgfqpoint{5.490039in}{5.490039in}}%
\pgfusepath{clip}%
\pgfsetbuttcap%
\pgfsetroundjoin%
\definecolor{currentfill}{rgb}{0.263663,0.237631,0.518762}%
\pgfsetfillcolor{currentfill}%
\pgfsetfillopacity{0.700000}%
\pgfsetlinewidth{0.000000pt}%
\definecolor{currentstroke}{rgb}{0.000000,0.000000,0.000000}%
\pgfsetstrokecolor{currentstroke}%
\pgfsetdash{}{0pt}%
\pgfpathmoveto{\pgfqpoint{4.939600in}{1.962454in}}%
\pgfpathlineto{\pgfqpoint{4.953416in}{1.965793in}}%
\pgfpathlineto{\pgfqpoint{4.967245in}{1.969246in}}%
\pgfpathlineto{\pgfqpoint{4.981085in}{1.972811in}}%
\pgfpathlineto{\pgfqpoint{4.994939in}{1.976489in}}%
\pgfpathlineto{\pgfqpoint{5.002463in}{1.988612in}}%
\pgfpathlineto{\pgfqpoint{5.009982in}{2.000700in}}%
\pgfpathlineto{\pgfqpoint{5.017496in}{2.012753in}}%
\pgfpathlineto{\pgfqpoint{5.025005in}{2.024769in}}%
\pgfpathlineto{\pgfqpoint{5.011153in}{2.020923in}}%
\pgfpathlineto{\pgfqpoint{4.997315in}{2.017189in}}%
\pgfpathlineto{\pgfqpoint{4.983488in}{2.013568in}}%
\pgfpathlineto{\pgfqpoint{4.969675in}{2.010060in}}%
\pgfpathlineto{\pgfqpoint{4.962163in}{1.998206in}}%
\pgfpathlineto{\pgfqpoint{4.954647in}{1.986320in}}%
\pgfpathlineto{\pgfqpoint{4.947126in}{1.974402in}}%
\pgfpathlineto{\pgfqpoint{4.939600in}{1.962454in}}%
\pgfpathclose%
\pgfusepath{fill}%
\end{pgfscope}%
\begin{pgfscope}%
\pgfpathrectangle{\pgfqpoint{1.254980in}{0.150000in}}{\pgfqpoint{5.490039in}{5.490039in}}%
\pgfusepath{clip}%
\pgfsetbuttcap%
\pgfsetroundjoin%
\definecolor{currentfill}{rgb}{0.270595,0.214069,0.507052}%
\pgfsetfillcolor{currentfill}%
\pgfsetfillopacity{0.700000}%
\pgfsetlinewidth{0.000000pt}%
\definecolor{currentstroke}{rgb}{0.000000,0.000000,0.000000}%
\pgfsetstrokecolor{currentstroke}%
\pgfsetdash{}{0pt}%
\pgfpathmoveto{\pgfqpoint{3.122620in}{1.963833in}}%
\pgfpathlineto{\pgfqpoint{3.136084in}{1.950297in}}%
\pgfpathlineto{\pgfqpoint{3.149547in}{1.936908in}}%
\pgfpathlineto{\pgfqpoint{3.163009in}{1.923666in}}%
\pgfpathlineto{\pgfqpoint{3.176469in}{1.910569in}}%
\pgfpathlineto{\pgfqpoint{3.184738in}{1.910866in}}%
\pgfpathlineto{\pgfqpoint{3.192995in}{1.911369in}}%
\pgfpathlineto{\pgfqpoint{3.201240in}{1.912076in}}%
\pgfpathlineto{\pgfqpoint{3.209473in}{1.912983in}}%
\pgfpathlineto{\pgfqpoint{3.196046in}{1.925658in}}%
\pgfpathlineto{\pgfqpoint{3.182618in}{1.938479in}}%
\pgfpathlineto{\pgfqpoint{3.169189in}{1.951446in}}%
\pgfpathlineto{\pgfqpoint{3.155759in}{1.964560in}}%
\pgfpathlineto{\pgfqpoint{3.147493in}{1.964068in}}%
\pgfpathlineto{\pgfqpoint{3.139215in}{1.963781in}}%
\pgfpathlineto{\pgfqpoint{3.130923in}{1.963701in}}%
\pgfpathlineto{\pgfqpoint{3.122620in}{1.963833in}}%
\pgfpathclose%
\pgfusepath{fill}%
\end{pgfscope}%
\begin{pgfscope}%
\pgfpathrectangle{\pgfqpoint{1.254980in}{0.150000in}}{\pgfqpoint{5.490039in}{5.490039in}}%
\pgfusepath{clip}%
\pgfsetbuttcap%
\pgfsetroundjoin%
\definecolor{currentfill}{rgb}{0.182256,0.426184,0.557120}%
\pgfsetfillcolor{currentfill}%
\pgfsetfillopacity{0.700000}%
\pgfsetlinewidth{0.000000pt}%
\definecolor{currentstroke}{rgb}{0.000000,0.000000,0.000000}%
\pgfsetstrokecolor{currentstroke}%
\pgfsetdash{}{0pt}%
\pgfpathmoveto{\pgfqpoint{2.690072in}{2.479420in}}%
\pgfpathlineto{\pgfqpoint{2.703662in}{2.460744in}}%
\pgfpathlineto{\pgfqpoint{2.717246in}{2.442242in}}%
\pgfpathlineto{\pgfqpoint{2.730824in}{2.423915in}}%
\pgfpathlineto{\pgfqpoint{2.744396in}{2.405760in}}%
\pgfpathlineto{\pgfqpoint{2.752978in}{2.402565in}}%
\pgfpathlineto{\pgfqpoint{2.761543in}{2.399616in}}%
\pgfpathlineto{\pgfqpoint{2.770093in}{2.396907in}}%
\pgfpathlineto{\pgfqpoint{2.778626in}{2.394436in}}%
\pgfpathlineto{\pgfqpoint{2.765097in}{2.412151in}}%
\pgfpathlineto{\pgfqpoint{2.751564in}{2.430037in}}%
\pgfpathlineto{\pgfqpoint{2.738024in}{2.448097in}}%
\pgfpathlineto{\pgfqpoint{2.724478in}{2.466331in}}%
\pgfpathlineto{\pgfqpoint{2.715902in}{2.469236in}}%
\pgfpathlineto{\pgfqpoint{2.707309in}{2.472383in}}%
\pgfpathlineto{\pgfqpoint{2.698699in}{2.475777in}}%
\pgfpathlineto{\pgfqpoint{2.690072in}{2.479420in}}%
\pgfpathclose%
\pgfusepath{fill}%
\end{pgfscope}%
\begin{pgfscope}%
\pgfpathrectangle{\pgfqpoint{1.254980in}{0.150000in}}{\pgfqpoint{5.490039in}{5.490039in}}%
\pgfusepath{clip}%
\pgfsetbuttcap%
\pgfsetroundjoin%
\definecolor{currentfill}{rgb}{0.273809,0.031497,0.358853}%
\pgfsetfillcolor{currentfill}%
\pgfsetfillopacity{0.700000}%
\pgfsetlinewidth{0.000000pt}%
\definecolor{currentstroke}{rgb}{0.000000,0.000000,0.000000}%
\pgfsetstrokecolor{currentstroke}%
\pgfsetdash{}{0pt}%
\pgfpathmoveto{\pgfqpoint{3.671254in}{1.598740in}}%
\pgfpathlineto{\pgfqpoint{3.684691in}{1.590926in}}%
\pgfpathlineto{\pgfqpoint{3.698131in}{1.583239in}}%
\pgfpathlineto{\pgfqpoint{3.711575in}{1.575678in}}%
\pgfpathlineto{\pgfqpoint{3.725022in}{1.568243in}}%
\pgfpathlineto{\pgfqpoint{3.732968in}{1.573567in}}%
\pgfpathlineto{\pgfqpoint{3.740906in}{1.579032in}}%
\pgfpathlineto{\pgfqpoint{3.748836in}{1.584635in}}%
\pgfpathlineto{\pgfqpoint{3.756759in}{1.590372in}}%
\pgfpathlineto{\pgfqpoint{3.743332in}{1.597431in}}%
\pgfpathlineto{\pgfqpoint{3.729909in}{1.604617in}}%
\pgfpathlineto{\pgfqpoint{3.716489in}{1.611928in}}%
\pgfpathlineto{\pgfqpoint{3.703073in}{1.619366in}}%
\pgfpathlineto{\pgfqpoint{3.695130in}{1.613999in}}%
\pgfpathlineto{\pgfqpoint{3.687180in}{1.608770in}}%
\pgfpathlineto{\pgfqpoint{3.679221in}{1.603682in}}%
\pgfpathlineto{\pgfqpoint{3.671254in}{1.598740in}}%
\pgfpathclose%
\pgfusepath{fill}%
\end{pgfscope}%
\begin{pgfscope}%
\pgfpathrectangle{\pgfqpoint{1.254980in}{0.150000in}}{\pgfqpoint{5.490039in}{5.490039in}}%
\pgfusepath{clip}%
\pgfsetbuttcap%
\pgfsetroundjoin%
\definecolor{currentfill}{rgb}{0.172719,0.448791,0.557885}%
\pgfsetfillcolor{currentfill}%
\pgfsetfillopacity{0.700000}%
\pgfsetlinewidth{0.000000pt}%
\definecolor{currentstroke}{rgb}{0.000000,0.000000,0.000000}%
\pgfsetstrokecolor{currentstroke}%
\pgfsetdash{}{0pt}%
\pgfpathmoveto{\pgfqpoint{5.512599in}{2.457803in}}%
\pgfpathlineto{\pgfqpoint{5.526730in}{2.464740in}}%
\pgfpathlineto{\pgfqpoint{5.540876in}{2.471791in}}%
\pgfpathlineto{\pgfqpoint{5.555037in}{2.478954in}}%
\pgfpathlineto{\pgfqpoint{5.569214in}{2.486230in}}%
\pgfpathlineto{\pgfqpoint{5.576539in}{2.497100in}}%
\pgfpathlineto{\pgfqpoint{5.583857in}{2.507891in}}%
\pgfpathlineto{\pgfqpoint{5.591168in}{2.518602in}}%
\pgfpathlineto{\pgfqpoint{5.598472in}{2.529233in}}%
\pgfpathlineto{\pgfqpoint{5.584299in}{2.521921in}}%
\pgfpathlineto{\pgfqpoint{5.570141in}{2.514720in}}%
\pgfpathlineto{\pgfqpoint{5.556000in}{2.507633in}}%
\pgfpathlineto{\pgfqpoint{5.541874in}{2.500659in}}%
\pgfpathlineto{\pgfqpoint{5.534565in}{2.490058in}}%
\pgfpathlineto{\pgfqpoint{5.527250in}{2.479382in}}%
\pgfpathlineto{\pgfqpoint{5.519928in}{2.468630in}}%
\pgfpathlineto{\pgfqpoint{5.512599in}{2.457803in}}%
\pgfpathclose%
\pgfusepath{fill}%
\end{pgfscope}%
\begin{pgfscope}%
\pgfpathrectangle{\pgfqpoint{1.254980in}{0.150000in}}{\pgfqpoint{5.490039in}{5.490039in}}%
\pgfusepath{clip}%
\pgfsetbuttcap%
\pgfsetroundjoin%
\definecolor{currentfill}{rgb}{0.281446,0.084320,0.407414}%
\pgfsetfillcolor{currentfill}%
\pgfsetfillopacity{0.700000}%
\pgfsetlinewidth{0.000000pt}%
\definecolor{currentstroke}{rgb}{0.000000,0.000000,0.000000}%
\pgfsetstrokecolor{currentstroke}%
\pgfsetdash{}{0pt}%
\pgfpathmoveto{\pgfqpoint{3.477977in}{1.689026in}}%
\pgfpathlineto{\pgfqpoint{3.491410in}{1.679260in}}%
\pgfpathlineto{\pgfqpoint{3.504845in}{1.669628in}}%
\pgfpathlineto{\pgfqpoint{3.518281in}{1.660127in}}%
\pgfpathlineto{\pgfqpoint{3.531719in}{1.650758in}}%
\pgfpathlineto{\pgfqpoint{3.539768in}{1.654289in}}%
\pgfpathlineto{\pgfqpoint{3.547808in}{1.657986in}}%
\pgfpathlineto{\pgfqpoint{3.555838in}{1.661847in}}%
\pgfpathlineto{\pgfqpoint{3.563859in}{1.665867in}}%
\pgfpathlineto{\pgfqpoint{3.550446in}{1.674841in}}%
\pgfpathlineto{\pgfqpoint{3.537035in}{1.683947in}}%
\pgfpathlineto{\pgfqpoint{3.523626in}{1.693184in}}%
\pgfpathlineto{\pgfqpoint{3.510219in}{1.702554in}}%
\pgfpathlineto{\pgfqpoint{3.502173in}{1.698923in}}%
\pgfpathlineto{\pgfqpoint{3.494117in}{1.695455in}}%
\pgfpathlineto{\pgfqpoint{3.486052in}{1.692155in}}%
\pgfpathlineto{\pgfqpoint{3.477977in}{1.689026in}}%
\pgfpathclose%
\pgfusepath{fill}%
\end{pgfscope}%
\begin{pgfscope}%
\pgfpathrectangle{\pgfqpoint{1.254980in}{0.150000in}}{\pgfqpoint{5.490039in}{5.490039in}}%
\pgfusepath{clip}%
\pgfsetbuttcap%
\pgfsetroundjoin%
\definecolor{currentfill}{rgb}{0.268510,0.009605,0.335427}%
\pgfsetfillcolor{currentfill}%
\pgfsetfillopacity{0.700000}%
\pgfsetlinewidth{0.000000pt}%
\definecolor{currentstroke}{rgb}{0.000000,0.000000,0.000000}%
\pgfsetstrokecolor{currentstroke}%
\pgfsetdash{}{0pt}%
\pgfpathmoveto{\pgfqpoint{4.088709in}{1.539394in}}%
\pgfpathlineto{\pgfqpoint{4.102210in}{1.535591in}}%
\pgfpathlineto{\pgfqpoint{4.115719in}{1.531908in}}%
\pgfpathlineto{\pgfqpoint{4.129233in}{1.528342in}}%
\pgfpathlineto{\pgfqpoint{4.142755in}{1.524894in}}%
\pgfpathlineto{\pgfqpoint{4.150526in}{1.533697in}}%
\pgfpathlineto{\pgfqpoint{4.158291in}{1.542581in}}%
\pgfpathlineto{\pgfqpoint{4.166051in}{1.551544in}}%
\pgfpathlineto{\pgfqpoint{4.173806in}{1.560583in}}%
\pgfpathlineto{\pgfqpoint{4.160296in}{1.563706in}}%
\pgfpathlineto{\pgfqpoint{4.146793in}{1.566946in}}%
\pgfpathlineto{\pgfqpoint{4.133297in}{1.570305in}}%
\pgfpathlineto{\pgfqpoint{4.119807in}{1.573782in}}%
\pgfpathlineto{\pgfqpoint{4.112041in}{1.565062in}}%
\pgfpathlineto{\pgfqpoint{4.104269in}{1.556422in}}%
\pgfpathlineto{\pgfqpoint{4.096492in}{1.547865in}}%
\pgfpathlineto{\pgfqpoint{4.088709in}{1.539394in}}%
\pgfpathclose%
\pgfusepath{fill}%
\end{pgfscope}%
\begin{pgfscope}%
\pgfpathrectangle{\pgfqpoint{1.254980in}{0.150000in}}{\pgfqpoint{5.490039in}{5.490039in}}%
\pgfusepath{clip}%
\pgfsetbuttcap%
\pgfsetroundjoin%
\definecolor{currentfill}{rgb}{0.276194,0.190074,0.493001}%
\pgfsetfillcolor{currentfill}%
\pgfsetfillopacity{0.700000}%
\pgfsetlinewidth{0.000000pt}%
\definecolor{currentstroke}{rgb}{0.000000,0.000000,0.000000}%
\pgfsetstrokecolor{currentstroke}%
\pgfsetdash{}{0pt}%
\pgfpathmoveto{\pgfqpoint{3.176469in}{1.910569in}}%
\pgfpathlineto{\pgfqpoint{3.189928in}{1.897618in}}%
\pgfpathlineto{\pgfqpoint{3.203386in}{1.884811in}}%
\pgfpathlineto{\pgfqpoint{3.216843in}{1.872148in}}%
\pgfpathlineto{\pgfqpoint{3.230299in}{1.859628in}}%
\pgfpathlineto{\pgfqpoint{3.238535in}{1.860352in}}%
\pgfpathlineto{\pgfqpoint{3.246759in}{1.861277in}}%
\pgfpathlineto{\pgfqpoint{3.254971in}{1.862401in}}%
\pgfpathlineto{\pgfqpoint{3.263172in}{1.863720in}}%
\pgfpathlineto{\pgfqpoint{3.249748in}{1.875821in}}%
\pgfpathlineto{\pgfqpoint{3.236324in}{1.888065in}}%
\pgfpathlineto{\pgfqpoint{3.222899in}{1.900452in}}%
\pgfpathlineto{\pgfqpoint{3.209473in}{1.912983in}}%
\pgfpathlineto{\pgfqpoint{3.201240in}{1.912076in}}%
\pgfpathlineto{\pgfqpoint{3.192995in}{1.911369in}}%
\pgfpathlineto{\pgfqpoint{3.184738in}{1.910866in}}%
\pgfpathlineto{\pgfqpoint{3.176469in}{1.910569in}}%
\pgfpathclose%
\pgfusepath{fill}%
\end{pgfscope}%
\begin{pgfscope}%
\pgfpathrectangle{\pgfqpoint{1.254980in}{0.150000in}}{\pgfqpoint{5.490039in}{5.490039in}}%
\pgfusepath{clip}%
\pgfsetbuttcap%
\pgfsetroundjoin%
\definecolor{currentfill}{rgb}{0.169646,0.456262,0.558030}%
\pgfsetfillcolor{currentfill}%
\pgfsetfillopacity{0.700000}%
\pgfsetlinewidth{0.000000pt}%
\definecolor{currentstroke}{rgb}{0.000000,0.000000,0.000000}%
\pgfsetstrokecolor{currentstroke}%
\pgfsetdash{}{0pt}%
\pgfpathmoveto{\pgfqpoint{2.635645in}{2.555899in}}%
\pgfpathlineto{\pgfqpoint{2.649262in}{2.536512in}}%
\pgfpathlineto{\pgfqpoint{2.662872in}{2.517303in}}%
\pgfpathlineto{\pgfqpoint{2.676475in}{2.498273in}}%
\pgfpathlineto{\pgfqpoint{2.690072in}{2.479420in}}%
\pgfpathlineto{\pgfqpoint{2.698699in}{2.475777in}}%
\pgfpathlineto{\pgfqpoint{2.707309in}{2.472383in}}%
\pgfpathlineto{\pgfqpoint{2.715902in}{2.469236in}}%
\pgfpathlineto{\pgfqpoint{2.724478in}{2.466331in}}%
\pgfpathlineto{\pgfqpoint{2.710927in}{2.484741in}}%
\pgfpathlineto{\pgfqpoint{2.697369in}{2.503327in}}%
\pgfpathlineto{\pgfqpoint{2.683805in}{2.522090in}}%
\pgfpathlineto{\pgfqpoint{2.670234in}{2.541033in}}%
\pgfpathlineto{\pgfqpoint{2.661612in}{2.544374in}}%
\pgfpathlineto{\pgfqpoint{2.652974in}{2.547963in}}%
\pgfpathlineto{\pgfqpoint{2.644318in}{2.551804in}}%
\pgfpathlineto{\pgfqpoint{2.635645in}{2.555899in}}%
\pgfpathclose%
\pgfusepath{fill}%
\end{pgfscope}%
\begin{pgfscope}%
\pgfpathrectangle{\pgfqpoint{1.254980in}{0.150000in}}{\pgfqpoint{5.490039in}{5.490039in}}%
\pgfusepath{clip}%
\pgfsetbuttcap%
\pgfsetroundjoin%
\definecolor{currentfill}{rgb}{0.252194,0.269783,0.531579}%
\pgfsetfillcolor{currentfill}%
\pgfsetfillopacity{0.700000}%
\pgfsetlinewidth{0.000000pt}%
\definecolor{currentstroke}{rgb}{0.000000,0.000000,0.000000}%
\pgfsetstrokecolor{currentstroke}%
\pgfsetdash{}{0pt}%
\pgfpathmoveto{\pgfqpoint{5.025005in}{2.024769in}}%
\pgfpathlineto{\pgfqpoint{5.038869in}{2.028728in}}%
\pgfpathlineto{\pgfqpoint{5.052746in}{2.032800in}}%
\pgfpathlineto{\pgfqpoint{5.066636in}{2.036985in}}%
\pgfpathlineto{\pgfqpoint{5.080539in}{2.041282in}}%
\pgfpathlineto{\pgfqpoint{5.088041in}{2.053419in}}%
\pgfpathlineto{\pgfqpoint{5.095538in}{2.065512in}}%
\pgfpathlineto{\pgfqpoint{5.103030in}{2.077562in}}%
\pgfpathlineto{\pgfqpoint{5.110517in}{2.089566in}}%
\pgfpathlineto{\pgfqpoint{5.096615in}{2.085116in}}%
\pgfpathlineto{\pgfqpoint{5.082727in}{2.080779in}}%
\pgfpathlineto{\pgfqpoint{5.068852in}{2.076554in}}%
\pgfpathlineto{\pgfqpoint{5.054990in}{2.072443in}}%
\pgfpathlineto{\pgfqpoint{5.047502in}{2.060585in}}%
\pgfpathlineto{\pgfqpoint{5.040008in}{2.048686in}}%
\pgfpathlineto{\pgfqpoint{5.032509in}{2.036747in}}%
\pgfpathlineto{\pgfqpoint{5.025005in}{2.024769in}}%
\pgfpathclose%
\pgfusepath{fill}%
\end{pgfscope}%
\begin{pgfscope}%
\pgfpathrectangle{\pgfqpoint{1.254980in}{0.150000in}}{\pgfqpoint{5.490039in}{5.490039in}}%
\pgfusepath{clip}%
\pgfsetbuttcap%
\pgfsetroundjoin%
\definecolor{currentfill}{rgb}{0.281446,0.084320,0.407414}%
\pgfsetfillcolor{currentfill}%
\pgfsetfillopacity{0.700000}%
\pgfsetlinewidth{0.000000pt}%
\definecolor{currentstroke}{rgb}{0.000000,0.000000,0.000000}%
\pgfsetstrokecolor{currentstroke}%
\pgfsetdash{}{0pt}%
\pgfpathmoveto{\pgfqpoint{4.483229in}{1.654095in}}%
\pgfpathlineto{\pgfqpoint{4.496852in}{1.653824in}}%
\pgfpathlineto{\pgfqpoint{4.510484in}{1.653667in}}%
\pgfpathlineto{\pgfqpoint{4.524126in}{1.653625in}}%
\pgfpathlineto{\pgfqpoint{4.537777in}{1.653697in}}%
\pgfpathlineto{\pgfqpoint{4.545428in}{1.664851in}}%
\pgfpathlineto{\pgfqpoint{4.553074in}{1.676030in}}%
\pgfpathlineto{\pgfqpoint{4.560716in}{1.687231in}}%
\pgfpathlineto{\pgfqpoint{4.568354in}{1.698453in}}%
\pgfpathlineto{\pgfqpoint{4.554708in}{1.698119in}}%
\pgfpathlineto{\pgfqpoint{4.541072in}{1.697898in}}%
\pgfpathlineto{\pgfqpoint{4.527445in}{1.697792in}}%
\pgfpathlineto{\pgfqpoint{4.513828in}{1.697801in}}%
\pgfpathlineto{\pgfqpoint{4.506185in}{1.686836in}}%
\pgfpathlineto{\pgfqpoint{4.498538in}{1.675895in}}%
\pgfpathlineto{\pgfqpoint{4.490886in}{1.664981in}}%
\pgfpathlineto{\pgfqpoint{4.483229in}{1.654095in}}%
\pgfpathclose%
\pgfusepath{fill}%
\end{pgfscope}%
\begin{pgfscope}%
\pgfpathrectangle{\pgfqpoint{1.254980in}{0.150000in}}{\pgfqpoint{5.490039in}{5.490039in}}%
\pgfusepath{clip}%
\pgfsetbuttcap%
\pgfsetroundjoin%
\definecolor{currentfill}{rgb}{0.283091,0.110553,0.431554}%
\pgfsetfillcolor{currentfill}%
\pgfsetfillopacity{0.700000}%
\pgfsetlinewidth{0.000000pt}%
\definecolor{currentstroke}{rgb}{0.000000,0.000000,0.000000}%
\pgfsetstrokecolor{currentstroke}%
\pgfsetdash{}{0pt}%
\pgfpathmoveto{\pgfqpoint{4.568354in}{1.698453in}}%
\pgfpathlineto{\pgfqpoint{4.582009in}{1.698901in}}%
\pgfpathlineto{\pgfqpoint{4.595675in}{1.699464in}}%
\pgfpathlineto{\pgfqpoint{4.609350in}{1.700140in}}%
\pgfpathlineto{\pgfqpoint{4.623036in}{1.700929in}}%
\pgfpathlineto{\pgfqpoint{4.630664in}{1.712422in}}%
\pgfpathlineto{\pgfqpoint{4.638288in}{1.723928in}}%
\pgfpathlineto{\pgfqpoint{4.645907in}{1.735444in}}%
\pgfpathlineto{\pgfqpoint{4.653522in}{1.746970in}}%
\pgfpathlineto{\pgfqpoint{4.639840in}{1.745933in}}%
\pgfpathlineto{\pgfqpoint{4.626169in}{1.745010in}}%
\pgfpathlineto{\pgfqpoint{4.612508in}{1.744200in}}%
\pgfpathlineto{\pgfqpoint{4.598857in}{1.743505in}}%
\pgfpathlineto{\pgfqpoint{4.591238in}{1.732221in}}%
\pgfpathlineto{\pgfqpoint{4.583615in}{1.720950in}}%
\pgfpathlineto{\pgfqpoint{4.575986in}{1.709693in}}%
\pgfpathlineto{\pgfqpoint{4.568354in}{1.698453in}}%
\pgfpathclose%
\pgfusepath{fill}%
\end{pgfscope}%
\begin{pgfscope}%
\pgfpathrectangle{\pgfqpoint{1.254980in}{0.150000in}}{\pgfqpoint{5.490039in}{5.490039in}}%
\pgfusepath{clip}%
\pgfsetbuttcap%
\pgfsetroundjoin%
\definecolor{currentfill}{rgb}{0.123444,0.636809,0.528763}%
\pgfsetfillcolor{currentfill}%
\pgfsetfillopacity{0.700000}%
\pgfsetlinewidth{0.000000pt}%
\definecolor{currentstroke}{rgb}{0.000000,0.000000,0.000000}%
\pgfsetstrokecolor{currentstroke}%
\pgfsetdash{}{0pt}%
\pgfpathmoveto{\pgfqpoint{2.342112in}{3.052340in}}%
\pgfpathlineto{\pgfqpoint{2.355894in}{3.028743in}}%
\pgfpathlineto{\pgfqpoint{2.369664in}{3.005355in}}%
\pgfpathlineto{\pgfqpoint{2.383424in}{2.982176in}}%
\pgfpathlineto{\pgfqpoint{2.397173in}{2.959203in}}%
\pgfpathlineto{\pgfqpoint{2.406021in}{2.953893in}}%
\pgfpathlineto{\pgfqpoint{2.414849in}{2.948849in}}%
\pgfpathlineto{\pgfqpoint{2.423657in}{2.944068in}}%
\pgfpathlineto{\pgfqpoint{2.432447in}{2.939544in}}%
\pgfpathlineto{\pgfqpoint{2.418750in}{2.962073in}}%
\pgfpathlineto{\pgfqpoint{2.405042in}{2.984806in}}%
\pgfpathlineto{\pgfqpoint{2.391323in}{3.007747in}}%
\pgfpathlineto{\pgfqpoint{2.377594in}{3.030896in}}%
\pgfpathlineto{\pgfqpoint{2.368753in}{3.035858in}}%
\pgfpathlineto{\pgfqpoint{2.359893in}{3.041083in}}%
\pgfpathlineto{\pgfqpoint{2.351012in}{3.046576in}}%
\pgfpathlineto{\pgfqpoint{2.342112in}{3.052340in}}%
\pgfpathclose%
\pgfusepath{fill}%
\end{pgfscope}%
\begin{pgfscope}%
\pgfpathrectangle{\pgfqpoint{1.254980in}{0.150000in}}{\pgfqpoint{5.490039in}{5.490039in}}%
\pgfusepath{clip}%
\pgfsetbuttcap%
\pgfsetroundjoin%
\definecolor{currentfill}{rgb}{0.278791,0.062145,0.386592}%
\pgfsetfillcolor{currentfill}%
\pgfsetfillopacity{0.700000}%
\pgfsetlinewidth{0.000000pt}%
\definecolor{currentstroke}{rgb}{0.000000,0.000000,0.000000}%
\pgfsetstrokecolor{currentstroke}%
\pgfsetdash{}{0pt}%
\pgfpathmoveto{\pgfqpoint{4.398129in}{1.614219in}}%
\pgfpathlineto{\pgfqpoint{4.411723in}{1.613211in}}%
\pgfpathlineto{\pgfqpoint{4.425325in}{1.612318in}}%
\pgfpathlineto{\pgfqpoint{4.438936in}{1.611540in}}%
\pgfpathlineto{\pgfqpoint{4.452556in}{1.610876in}}%
\pgfpathlineto{\pgfqpoint{4.460231in}{1.621628in}}%
\pgfpathlineto{\pgfqpoint{4.467902in}{1.632416in}}%
\pgfpathlineto{\pgfqpoint{4.475568in}{1.643239in}}%
\pgfpathlineto{\pgfqpoint{4.483229in}{1.654095in}}%
\pgfpathlineto{\pgfqpoint{4.469616in}{1.654480in}}%
\pgfpathlineto{\pgfqpoint{4.456011in}{1.654980in}}%
\pgfpathlineto{\pgfqpoint{4.442416in}{1.655595in}}%
\pgfpathlineto{\pgfqpoint{4.428829in}{1.656325in}}%
\pgfpathlineto{\pgfqpoint{4.421161in}{1.645742in}}%
\pgfpathlineto{\pgfqpoint{4.413489in}{1.635195in}}%
\pgfpathlineto{\pgfqpoint{4.405811in}{1.624686in}}%
\pgfpathlineto{\pgfqpoint{4.398129in}{1.614219in}}%
\pgfpathclose%
\pgfusepath{fill}%
\end{pgfscope}%
\begin{pgfscope}%
\pgfpathrectangle{\pgfqpoint{1.254980in}{0.150000in}}{\pgfqpoint{5.490039in}{5.490039in}}%
\pgfusepath{clip}%
\pgfsetbuttcap%
\pgfsetroundjoin%
\definecolor{currentfill}{rgb}{0.203063,0.379716,0.553925}%
\pgfsetfillcolor{currentfill}%
\pgfsetfillopacity{0.700000}%
\pgfsetlinewidth{0.000000pt}%
\definecolor{currentstroke}{rgb}{0.000000,0.000000,0.000000}%
\pgfsetstrokecolor{currentstroke}%
\pgfsetdash{}{0pt}%
\pgfpathmoveto{\pgfqpoint{5.311583in}{2.272312in}}%
\pgfpathlineto{\pgfqpoint{5.325602in}{2.278136in}}%
\pgfpathlineto{\pgfqpoint{5.339637in}{2.284073in}}%
\pgfpathlineto{\pgfqpoint{5.353685in}{2.290122in}}%
\pgfpathlineto{\pgfqpoint{5.367749in}{2.296284in}}%
\pgfpathlineto{\pgfqpoint{5.375157in}{2.307906in}}%
\pgfpathlineto{\pgfqpoint{5.382558in}{2.319461in}}%
\pgfpathlineto{\pgfqpoint{5.389953in}{2.330946in}}%
\pgfpathlineto{\pgfqpoint{5.397343in}{2.342363in}}%
\pgfpathlineto{\pgfqpoint{5.383282in}{2.336114in}}%
\pgfpathlineto{\pgfqpoint{5.369235in}{2.329977in}}%
\pgfpathlineto{\pgfqpoint{5.355204in}{2.323954in}}%
\pgfpathlineto{\pgfqpoint{5.341187in}{2.318042in}}%
\pgfpathlineto{\pgfqpoint{5.333795in}{2.306706in}}%
\pgfpathlineto{\pgfqpoint{5.326397in}{2.295306in}}%
\pgfpathlineto{\pgfqpoint{5.318993in}{2.283841in}}%
\pgfpathlineto{\pgfqpoint{5.311583in}{2.272312in}}%
\pgfpathclose%
\pgfusepath{fill}%
\end{pgfscope}%
\begin{pgfscope}%
\pgfpathrectangle{\pgfqpoint{1.254980in}{0.150000in}}{\pgfqpoint{5.490039in}{5.490039in}}%
\pgfusepath{clip}%
\pgfsetbuttcap%
\pgfsetroundjoin%
\definecolor{currentfill}{rgb}{0.282884,0.135920,0.453427}%
\pgfsetfillcolor{currentfill}%
\pgfsetfillopacity{0.700000}%
\pgfsetlinewidth{0.000000pt}%
\definecolor{currentstroke}{rgb}{0.000000,0.000000,0.000000}%
\pgfsetstrokecolor{currentstroke}%
\pgfsetdash{}{0pt}%
\pgfpathmoveto{\pgfqpoint{4.653522in}{1.746970in}}%
\pgfpathlineto{\pgfqpoint{4.667213in}{1.748120in}}%
\pgfpathlineto{\pgfqpoint{4.680916in}{1.749384in}}%
\pgfpathlineto{\pgfqpoint{4.694629in}{1.750761in}}%
\pgfpathlineto{\pgfqpoint{4.708352in}{1.752251in}}%
\pgfpathlineto{\pgfqpoint{4.715959in}{1.764021in}}%
\pgfpathlineto{\pgfqpoint{4.723560in}{1.775792in}}%
\pgfpathlineto{\pgfqpoint{4.731158in}{1.787563in}}%
\pgfpathlineto{\pgfqpoint{4.738750in}{1.799331in}}%
\pgfpathlineto{\pgfqpoint{4.725030in}{1.797609in}}%
\pgfpathlineto{\pgfqpoint{4.711321in}{1.796000in}}%
\pgfpathlineto{\pgfqpoint{4.697622in}{1.794504in}}%
\pgfpathlineto{\pgfqpoint{4.683935in}{1.793123in}}%
\pgfpathlineto{\pgfqpoint{4.676338in}{1.781580in}}%
\pgfpathlineto{\pgfqpoint{4.668737in}{1.770039in}}%
\pgfpathlineto{\pgfqpoint{4.661132in}{1.758502in}}%
\pgfpathlineto{\pgfqpoint{4.653522in}{1.746970in}}%
\pgfpathclose%
\pgfusepath{fill}%
\end{pgfscope}%
\begin{pgfscope}%
\pgfpathrectangle{\pgfqpoint{1.254980in}{0.150000in}}{\pgfqpoint{5.490039in}{5.490039in}}%
\pgfusepath{clip}%
\pgfsetbuttcap%
\pgfsetroundjoin%
\definecolor{currentfill}{rgb}{0.162142,0.474838,0.558140}%
\pgfsetfillcolor{currentfill}%
\pgfsetfillopacity{0.700000}%
\pgfsetlinewidth{0.000000pt}%
\definecolor{currentstroke}{rgb}{0.000000,0.000000,0.000000}%
\pgfsetstrokecolor{currentstroke}%
\pgfsetdash{}{0pt}%
\pgfpathmoveto{\pgfqpoint{5.598472in}{2.529233in}}%
\pgfpathlineto{\pgfqpoint{5.612661in}{2.536659in}}%
\pgfpathlineto{\pgfqpoint{5.626866in}{2.544198in}}%
\pgfpathlineto{\pgfqpoint{5.641088in}{2.551850in}}%
\pgfpathlineto{\pgfqpoint{5.648381in}{2.562421in}}%
\pgfpathlineto{\pgfqpoint{5.655668in}{2.572910in}}%
\pgfpathlineto{\pgfqpoint{5.662948in}{2.583316in}}%
\pgfpathlineto{\pgfqpoint{5.670220in}{2.593640in}}%
\pgfpathlineto{\pgfqpoint{5.656004in}{2.585968in}}%
\pgfpathlineto{\pgfqpoint{5.641804in}{2.578410in}}%
\pgfpathlineto{\pgfqpoint{5.627619in}{2.570964in}}%
\pgfpathlineto{\pgfqpoint{5.620343in}{2.560650in}}%
\pgfpathlineto{\pgfqpoint{5.613059in}{2.550258in}}%
\pgfpathlineto{\pgfqpoint{5.605769in}{2.539785in}}%
\pgfpathlineto{\pgfqpoint{5.598472in}{2.529233in}}%
\pgfpathclose%
\pgfusepath{fill}%
\end{pgfscope}%
\begin{pgfscope}%
\pgfpathrectangle{\pgfqpoint{1.254980in}{0.150000in}}{\pgfqpoint{5.490039in}{5.490039in}}%
\pgfusepath{clip}%
\pgfsetbuttcap%
\pgfsetroundjoin%
\definecolor{currentfill}{rgb}{0.279574,0.170599,0.479997}%
\pgfsetfillcolor{currentfill}%
\pgfsetfillopacity{0.700000}%
\pgfsetlinewidth{0.000000pt}%
\definecolor{currentstroke}{rgb}{0.000000,0.000000,0.000000}%
\pgfsetstrokecolor{currentstroke}%
\pgfsetdash{}{0pt}%
\pgfpathmoveto{\pgfqpoint{3.230299in}{1.859628in}}%
\pgfpathlineto{\pgfqpoint{3.243754in}{1.847251in}}%
\pgfpathlineto{\pgfqpoint{3.257209in}{1.835016in}}%
\pgfpathlineto{\pgfqpoint{3.270663in}{1.822922in}}%
\pgfpathlineto{\pgfqpoint{3.284117in}{1.810968in}}%
\pgfpathlineto{\pgfqpoint{3.292321in}{1.812116in}}%
\pgfpathlineto{\pgfqpoint{3.300513in}{1.813462in}}%
\pgfpathlineto{\pgfqpoint{3.308694in}{1.815002in}}%
\pgfpathlineto{\pgfqpoint{3.316864in}{1.816732in}}%
\pgfpathlineto{\pgfqpoint{3.303441in}{1.828268in}}%
\pgfpathlineto{\pgfqpoint{3.290019in}{1.839945in}}%
\pgfpathlineto{\pgfqpoint{3.276595in}{1.851762in}}%
\pgfpathlineto{\pgfqpoint{3.263172in}{1.863720in}}%
\pgfpathlineto{\pgfqpoint{3.254971in}{1.862401in}}%
\pgfpathlineto{\pgfqpoint{3.246759in}{1.861277in}}%
\pgfpathlineto{\pgfqpoint{3.238535in}{1.860352in}}%
\pgfpathlineto{\pgfqpoint{3.230299in}{1.859628in}}%
\pgfpathclose%
\pgfusepath{fill}%
\end{pgfscope}%
\begin{pgfscope}%
\pgfpathrectangle{\pgfqpoint{1.254980in}{0.150000in}}{\pgfqpoint{5.490039in}{5.490039in}}%
\pgfusepath{clip}%
\pgfsetbuttcap%
\pgfsetroundjoin%
\definecolor{currentfill}{rgb}{0.276022,0.044167,0.370164}%
\pgfsetfillcolor{currentfill}%
\pgfsetfillopacity{0.700000}%
\pgfsetlinewidth{0.000000pt}%
\definecolor{currentstroke}{rgb}{0.000000,0.000000,0.000000}%
\pgfsetstrokecolor{currentstroke}%
\pgfsetdash{}{0pt}%
\pgfpathmoveto{\pgfqpoint{4.313033in}{1.579160in}}%
\pgfpathlineto{\pgfqpoint{4.326601in}{1.577397in}}%
\pgfpathlineto{\pgfqpoint{4.340177in}{1.575750in}}%
\pgfpathlineto{\pgfqpoint{4.353761in}{1.574218in}}%
\pgfpathlineto{\pgfqpoint{4.367354in}{1.572801in}}%
\pgfpathlineto{\pgfqpoint{4.375055in}{1.583083in}}%
\pgfpathlineto{\pgfqpoint{4.382751in}{1.593415in}}%
\pgfpathlineto{\pgfqpoint{4.390443in}{1.603794in}}%
\pgfpathlineto{\pgfqpoint{4.398129in}{1.614219in}}%
\pgfpathlineto{\pgfqpoint{4.384545in}{1.615342in}}%
\pgfpathlineto{\pgfqpoint{4.370968in}{1.616580in}}%
\pgfpathlineto{\pgfqpoint{4.357400in}{1.617934in}}%
\pgfpathlineto{\pgfqpoint{4.343841in}{1.619404in}}%
\pgfpathlineto{\pgfqpoint{4.336146in}{1.609267in}}%
\pgfpathlineto{\pgfqpoint{4.328447in}{1.599179in}}%
\pgfpathlineto{\pgfqpoint{4.320743in}{1.589143in}}%
\pgfpathlineto{\pgfqpoint{4.313033in}{1.579160in}}%
\pgfpathclose%
\pgfusepath{fill}%
\end{pgfscope}%
\begin{pgfscope}%
\pgfpathrectangle{\pgfqpoint{1.254980in}{0.150000in}}{\pgfqpoint{5.490039in}{5.490039in}}%
\pgfusepath{clip}%
\pgfsetbuttcap%
\pgfsetroundjoin%
\definecolor{currentfill}{rgb}{0.157729,0.485932,0.558013}%
\pgfsetfillcolor{currentfill}%
\pgfsetfillopacity{0.700000}%
\pgfsetlinewidth{0.000000pt}%
\definecolor{currentstroke}{rgb}{0.000000,0.000000,0.000000}%
\pgfsetstrokecolor{currentstroke}%
\pgfsetdash{}{0pt}%
\pgfpathmoveto{\pgfqpoint{2.581104in}{2.635268in}}%
\pgfpathlineto{\pgfqpoint{2.594751in}{2.615151in}}%
\pgfpathlineto{\pgfqpoint{2.608389in}{2.595218in}}%
\pgfpathlineto{\pgfqpoint{2.622021in}{2.575468in}}%
\pgfpathlineto{\pgfqpoint{2.635645in}{2.555899in}}%
\pgfpathlineto{\pgfqpoint{2.644318in}{2.551804in}}%
\pgfpathlineto{\pgfqpoint{2.652974in}{2.547963in}}%
\pgfpathlineto{\pgfqpoint{2.661612in}{2.544374in}}%
\pgfpathlineto{\pgfqpoint{2.670234in}{2.541033in}}%
\pgfpathlineto{\pgfqpoint{2.656656in}{2.560154in}}%
\pgfpathlineto{\pgfqpoint{2.643072in}{2.579457in}}%
\pgfpathlineto{\pgfqpoint{2.629480in}{2.598942in}}%
\pgfpathlineto{\pgfqpoint{2.615882in}{2.618609in}}%
\pgfpathlineto{\pgfqpoint{2.607214in}{2.622391in}}%
\pgfpathlineto{\pgfqpoint{2.598529in}{2.626426in}}%
\pgfpathlineto{\pgfqpoint{2.589826in}{2.630716in}}%
\pgfpathlineto{\pgfqpoint{2.581104in}{2.635268in}}%
\pgfpathclose%
\pgfusepath{fill}%
\end{pgfscope}%
\begin{pgfscope}%
\pgfpathrectangle{\pgfqpoint{1.254980in}{0.150000in}}{\pgfqpoint{5.490039in}{5.490039in}}%
\pgfusepath{clip}%
\pgfsetbuttcap%
\pgfsetroundjoin%
\definecolor{currentfill}{rgb}{0.268510,0.009605,0.335427}%
\pgfsetfillcolor{currentfill}%
\pgfsetfillopacity{0.700000}%
\pgfsetlinewidth{0.000000pt}%
\definecolor{currentstroke}{rgb}{0.000000,0.000000,0.000000}%
\pgfsetstrokecolor{currentstroke}%
\pgfsetdash{}{0pt}%
\pgfpathmoveto{\pgfqpoint{3.864318in}{1.538377in}}%
\pgfpathlineto{\pgfqpoint{3.877783in}{1.532433in}}%
\pgfpathlineto{\pgfqpoint{3.891253in}{1.526610in}}%
\pgfpathlineto{\pgfqpoint{3.904727in}{1.520909in}}%
\pgfpathlineto{\pgfqpoint{3.918206in}{1.515330in}}%
\pgfpathlineto{\pgfqpoint{3.926068in}{1.522291in}}%
\pgfpathlineto{\pgfqpoint{3.933923in}{1.529368in}}%
\pgfpathlineto{\pgfqpoint{3.941771in}{1.536559in}}%
\pgfpathlineto{\pgfqpoint{3.949612in}{1.543859in}}%
\pgfpathlineto{\pgfqpoint{3.936149in}{1.549081in}}%
\pgfpathlineto{\pgfqpoint{3.922691in}{1.554424in}}%
\pgfpathlineto{\pgfqpoint{3.909238in}{1.559888in}}%
\pgfpathlineto{\pgfqpoint{3.895790in}{1.565475in}}%
\pgfpathlineto{\pgfqpoint{3.887933in}{1.558526in}}%
\pgfpathlineto{\pgfqpoint{3.880068in}{1.551692in}}%
\pgfpathlineto{\pgfqpoint{3.872197in}{1.544975in}}%
\pgfpathlineto{\pgfqpoint{3.864318in}{1.538377in}}%
\pgfpathclose%
\pgfusepath{fill}%
\end{pgfscope}%
\begin{pgfscope}%
\pgfpathrectangle{\pgfqpoint{1.254980in}{0.150000in}}{\pgfqpoint{5.490039in}{5.490039in}}%
\pgfusepath{clip}%
\pgfsetbuttcap%
\pgfsetroundjoin%
\definecolor{currentfill}{rgb}{0.280255,0.165693,0.476498}%
\pgfsetfillcolor{currentfill}%
\pgfsetfillopacity{0.700000}%
\pgfsetlinewidth{0.000000pt}%
\definecolor{currentstroke}{rgb}{0.000000,0.000000,0.000000}%
\pgfsetstrokecolor{currentstroke}%
\pgfsetdash{}{0pt}%
\pgfpathmoveto{\pgfqpoint{4.738750in}{1.799331in}}%
\pgfpathlineto{\pgfqpoint{4.752481in}{1.801166in}}%
\pgfpathlineto{\pgfqpoint{4.766224in}{1.803115in}}%
\pgfpathlineto{\pgfqpoint{4.779977in}{1.805176in}}%
\pgfpathlineto{\pgfqpoint{4.793742in}{1.807350in}}%
\pgfpathlineto{\pgfqpoint{4.801327in}{1.819337in}}%
\pgfpathlineto{\pgfqpoint{4.808907in}{1.831314in}}%
\pgfpathlineto{\pgfqpoint{4.816483in}{1.843279in}}%
\pgfpathlineto{\pgfqpoint{4.824055in}{1.855233in}}%
\pgfpathlineto{\pgfqpoint{4.810293in}{1.852842in}}%
\pgfpathlineto{\pgfqpoint{4.796542in}{1.850564in}}%
\pgfpathlineto{\pgfqpoint{4.782803in}{1.848400in}}%
\pgfpathlineto{\pgfqpoint{4.769075in}{1.846349in}}%
\pgfpathlineto{\pgfqpoint{4.761501in}{1.834606in}}%
\pgfpathlineto{\pgfqpoint{4.753922in}{1.822854in}}%
\pgfpathlineto{\pgfqpoint{4.746339in}{1.811095in}}%
\pgfpathlineto{\pgfqpoint{4.738750in}{1.799331in}}%
\pgfpathclose%
\pgfusepath{fill}%
\end{pgfscope}%
\begin{pgfscope}%
\pgfpathrectangle{\pgfqpoint{1.254980in}{0.150000in}}{\pgfqpoint{5.490039in}{5.490039in}}%
\pgfusepath{clip}%
\pgfsetbuttcap%
\pgfsetroundjoin%
\definecolor{currentfill}{rgb}{0.279566,0.067836,0.391917}%
\pgfsetfillcolor{currentfill}%
\pgfsetfillopacity{0.700000}%
\pgfsetlinewidth{0.000000pt}%
\definecolor{currentstroke}{rgb}{0.000000,0.000000,0.000000}%
\pgfsetstrokecolor{currentstroke}%
\pgfsetdash{}{0pt}%
\pgfpathmoveto{\pgfqpoint{3.531719in}{1.650758in}}%
\pgfpathlineto{\pgfqpoint{3.545159in}{1.641520in}}%
\pgfpathlineto{\pgfqpoint{3.558601in}{1.632412in}}%
\pgfpathlineto{\pgfqpoint{3.572045in}{1.623434in}}%
\pgfpathlineto{\pgfqpoint{3.585492in}{1.614586in}}%
\pgfpathlineto{\pgfqpoint{3.593516in}{1.618518in}}%
\pgfpathlineto{\pgfqpoint{3.601531in}{1.622612in}}%
\pgfpathlineto{\pgfqpoint{3.609538in}{1.626864in}}%
\pgfpathlineto{\pgfqpoint{3.617535in}{1.631272in}}%
\pgfpathlineto{\pgfqpoint{3.604113in}{1.639726in}}%
\pgfpathlineto{\pgfqpoint{3.590693in}{1.648310in}}%
\pgfpathlineto{\pgfqpoint{3.577275in}{1.657023in}}%
\pgfpathlineto{\pgfqpoint{3.563859in}{1.665867in}}%
\pgfpathlineto{\pgfqpoint{3.555838in}{1.661847in}}%
\pgfpathlineto{\pgfqpoint{3.547808in}{1.657986in}}%
\pgfpathlineto{\pgfqpoint{3.539768in}{1.654289in}}%
\pgfpathlineto{\pgfqpoint{3.531719in}{1.650758in}}%
\pgfpathclose%
\pgfusepath{fill}%
\end{pgfscope}%
\begin{pgfscope}%
\pgfpathrectangle{\pgfqpoint{1.254980in}{0.150000in}}{\pgfqpoint{5.490039in}{5.490039in}}%
\pgfusepath{clip}%
\pgfsetbuttcap%
\pgfsetroundjoin%
\definecolor{currentfill}{rgb}{0.237441,0.305202,0.541921}%
\pgfsetfillcolor{currentfill}%
\pgfsetfillopacity{0.700000}%
\pgfsetlinewidth{0.000000pt}%
\definecolor{currentstroke}{rgb}{0.000000,0.000000,0.000000}%
\pgfsetstrokecolor{currentstroke}%
\pgfsetdash{}{0pt}%
\pgfpathmoveto{\pgfqpoint{5.110517in}{2.089566in}}%
\pgfpathlineto{\pgfqpoint{5.124431in}{2.094128in}}%
\pgfpathlineto{\pgfqpoint{5.138359in}{2.098803in}}%
\pgfpathlineto{\pgfqpoint{5.152301in}{2.103591in}}%
\pgfpathlineto{\pgfqpoint{5.166256in}{2.108491in}}%
\pgfpathlineto{\pgfqpoint{5.173735in}{2.120592in}}%
\pgfpathlineto{\pgfqpoint{5.181210in}{2.132642in}}%
\pgfpathlineto{\pgfqpoint{5.188678in}{2.144639in}}%
\pgfpathlineto{\pgfqpoint{5.196142in}{2.156584in}}%
\pgfpathlineto{\pgfqpoint{5.182188in}{2.151547in}}%
\pgfpathlineto{\pgfqpoint{5.168249in}{2.146623in}}%
\pgfpathlineto{\pgfqpoint{5.154323in}{2.141811in}}%
\pgfpathlineto{\pgfqpoint{5.140410in}{2.137112in}}%
\pgfpathlineto{\pgfqpoint{5.132945in}{2.125298in}}%
\pgfpathlineto{\pgfqpoint{5.125474in}{2.113435in}}%
\pgfpathlineto{\pgfqpoint{5.117998in}{2.101524in}}%
\pgfpathlineto{\pgfqpoint{5.110517in}{2.089566in}}%
\pgfpathclose%
\pgfusepath{fill}%
\end{pgfscope}%
\begin{pgfscope}%
\pgfpathrectangle{\pgfqpoint{1.254980in}{0.150000in}}{\pgfqpoint{5.490039in}{5.490039in}}%
\pgfusepath{clip}%
\pgfsetbuttcap%
\pgfsetroundjoin%
\definecolor{currentfill}{rgb}{0.267004,0.004874,0.329415}%
\pgfsetfillcolor{currentfill}%
\pgfsetfillopacity{0.700000}%
\pgfsetlinewidth{0.000000pt}%
\definecolor{currentstroke}{rgb}{0.000000,0.000000,0.000000}%
\pgfsetstrokecolor{currentstroke}%
\pgfsetdash{}{0pt}%
\pgfpathmoveto{\pgfqpoint{4.003519in}{1.524180in}}%
\pgfpathlineto{\pgfqpoint{4.017010in}{1.519560in}}%
\pgfpathlineto{\pgfqpoint{4.030507in}{1.515060in}}%
\pgfpathlineto{\pgfqpoint{4.044009in}{1.510679in}}%
\pgfpathlineto{\pgfqpoint{4.057518in}{1.506417in}}%
\pgfpathlineto{\pgfqpoint{4.065324in}{1.514519in}}%
\pgfpathlineto{\pgfqpoint{4.073125in}{1.522718in}}%
\pgfpathlineto{\pgfqpoint{4.080920in}{1.531010in}}%
\pgfpathlineto{\pgfqpoint{4.088709in}{1.539394in}}%
\pgfpathlineto{\pgfqpoint{4.075213in}{1.543314in}}%
\pgfpathlineto{\pgfqpoint{4.061724in}{1.547354in}}%
\pgfpathlineto{\pgfqpoint{4.048241in}{1.551513in}}%
\pgfpathlineto{\pgfqpoint{4.034764in}{1.555791in}}%
\pgfpathlineto{\pgfqpoint{4.026962in}{1.547743in}}%
\pgfpathlineto{\pgfqpoint{4.019154in}{1.539790in}}%
\pgfpathlineto{\pgfqpoint{4.011340in}{1.531935in}}%
\pgfpathlineto{\pgfqpoint{4.003519in}{1.524180in}}%
\pgfpathclose%
\pgfusepath{fill}%
\end{pgfscope}%
\begin{pgfscope}%
\pgfpathrectangle{\pgfqpoint{1.254980in}{0.150000in}}{\pgfqpoint{5.490039in}{5.490039in}}%
\pgfusepath{clip}%
\pgfsetbuttcap%
\pgfsetroundjoin%
\definecolor{currentfill}{rgb}{0.272594,0.025563,0.353093}%
\pgfsetfillcolor{currentfill}%
\pgfsetfillopacity{0.700000}%
\pgfsetlinewidth{0.000000pt}%
\definecolor{currentstroke}{rgb}{0.000000,0.000000,0.000000}%
\pgfsetstrokecolor{currentstroke}%
\pgfsetdash{}{0pt}%
\pgfpathmoveto{\pgfqpoint{3.725022in}{1.568243in}}%
\pgfpathlineto{\pgfqpoint{3.738473in}{1.560933in}}%
\pgfpathlineto{\pgfqpoint{3.751927in}{1.553749in}}%
\pgfpathlineto{\pgfqpoint{3.765385in}{1.546689in}}%
\pgfpathlineto{\pgfqpoint{3.778847in}{1.539754in}}%
\pgfpathlineto{\pgfqpoint{3.786772in}{1.545460in}}%
\pgfpathlineto{\pgfqpoint{3.794691in}{1.551302in}}%
\pgfpathlineto{\pgfqpoint{3.802601in}{1.557278in}}%
\pgfpathlineto{\pgfqpoint{3.810505in}{1.563384in}}%
\pgfpathlineto{\pgfqpoint{3.797062in}{1.569944in}}%
\pgfpathlineto{\pgfqpoint{3.783624in}{1.576628in}}%
\pgfpathlineto{\pgfqpoint{3.770189in}{1.583438in}}%
\pgfpathlineto{\pgfqpoint{3.756759in}{1.590372in}}%
\pgfpathlineto{\pgfqpoint{3.748836in}{1.584635in}}%
\pgfpathlineto{\pgfqpoint{3.740906in}{1.579032in}}%
\pgfpathlineto{\pgfqpoint{3.732968in}{1.573567in}}%
\pgfpathlineto{\pgfqpoint{3.725022in}{1.568243in}}%
\pgfpathclose%
\pgfusepath{fill}%
\end{pgfscope}%
\begin{pgfscope}%
\pgfpathrectangle{\pgfqpoint{1.254980in}{0.150000in}}{\pgfqpoint{5.490039in}{5.490039in}}%
\pgfusepath{clip}%
\pgfsetbuttcap%
\pgfsetroundjoin%
\definecolor{currentfill}{rgb}{0.272594,0.025563,0.353093}%
\pgfsetfillcolor{currentfill}%
\pgfsetfillopacity{0.700000}%
\pgfsetlinewidth{0.000000pt}%
\definecolor{currentstroke}{rgb}{0.000000,0.000000,0.000000}%
\pgfsetstrokecolor{currentstroke}%
\pgfsetdash{}{0pt}%
\pgfpathmoveto{\pgfqpoint{4.227917in}{1.549266in}}%
\pgfpathlineto{\pgfqpoint{4.241463in}{1.546729in}}%
\pgfpathlineto{\pgfqpoint{4.255017in}{1.544308in}}%
\pgfpathlineto{\pgfqpoint{4.268578in}{1.542004in}}%
\pgfpathlineto{\pgfqpoint{4.282147in}{1.539815in}}%
\pgfpathlineto{\pgfqpoint{4.289876in}{1.549559in}}%
\pgfpathlineto{\pgfqpoint{4.297600in}{1.559366in}}%
\pgfpathlineto{\pgfqpoint{4.305319in}{1.569234in}}%
\pgfpathlineto{\pgfqpoint{4.313033in}{1.579160in}}%
\pgfpathlineto{\pgfqpoint{4.299474in}{1.581039in}}%
\pgfpathlineto{\pgfqpoint{4.285922in}{1.583034in}}%
\pgfpathlineto{\pgfqpoint{4.272378in}{1.585146in}}%
\pgfpathlineto{\pgfqpoint{4.258841in}{1.587373in}}%
\pgfpathlineto{\pgfqpoint{4.251118in}{1.577750in}}%
\pgfpathlineto{\pgfqpoint{4.243389in}{1.568190in}}%
\pgfpathlineto{\pgfqpoint{4.235656in}{1.558694in}}%
\pgfpathlineto{\pgfqpoint{4.227917in}{1.549266in}}%
\pgfpathclose%
\pgfusepath{fill}%
\end{pgfscope}%
\begin{pgfscope}%
\pgfpathrectangle{\pgfqpoint{1.254980in}{0.150000in}}{\pgfqpoint{5.490039in}{5.490039in}}%
\pgfusepath{clip}%
\pgfsetbuttcap%
\pgfsetroundjoin%
\definecolor{currentfill}{rgb}{0.275191,0.194905,0.496005}%
\pgfsetfillcolor{currentfill}%
\pgfsetfillopacity{0.700000}%
\pgfsetlinewidth{0.000000pt}%
\definecolor{currentstroke}{rgb}{0.000000,0.000000,0.000000}%
\pgfsetstrokecolor{currentstroke}%
\pgfsetdash{}{0pt}%
\pgfpathmoveto{\pgfqpoint{4.824055in}{1.855233in}}%
\pgfpathlineto{\pgfqpoint{4.837828in}{1.857736in}}%
\pgfpathlineto{\pgfqpoint{4.851613in}{1.860352in}}%
\pgfpathlineto{\pgfqpoint{4.865410in}{1.863082in}}%
\pgfpathlineto{\pgfqpoint{4.879218in}{1.865924in}}%
\pgfpathlineto{\pgfqpoint{4.886782in}{1.878069in}}%
\pgfpathlineto{\pgfqpoint{4.894342in}{1.890194in}}%
\pgfpathlineto{\pgfqpoint{4.901897in}{1.902299in}}%
\pgfpathlineto{\pgfqpoint{4.909447in}{1.914380in}}%
\pgfpathlineto{\pgfqpoint{4.895641in}{1.911338in}}%
\pgfpathlineto{\pgfqpoint{4.881846in}{1.908408in}}%
\pgfpathlineto{\pgfqpoint{4.868064in}{1.905592in}}%
\pgfpathlineto{\pgfqpoint{4.854293in}{1.902888in}}%
\pgfpathlineto{\pgfqpoint{4.846741in}{1.891000in}}%
\pgfpathlineto{\pgfqpoint{4.839183in}{1.879094in}}%
\pgfpathlineto{\pgfqpoint{4.831621in}{1.867171in}}%
\pgfpathlineto{\pgfqpoint{4.824055in}{1.855233in}}%
\pgfpathclose%
\pgfusepath{fill}%
\end{pgfscope}%
\begin{pgfscope}%
\pgfpathrectangle{\pgfqpoint{1.254980in}{0.150000in}}{\pgfqpoint{5.490039in}{5.490039in}}%
\pgfusepath{clip}%
\pgfsetbuttcap%
\pgfsetroundjoin%
\definecolor{currentfill}{rgb}{0.281887,0.150881,0.465405}%
\pgfsetfillcolor{currentfill}%
\pgfsetfillopacity{0.700000}%
\pgfsetlinewidth{0.000000pt}%
\definecolor{currentstroke}{rgb}{0.000000,0.000000,0.000000}%
\pgfsetstrokecolor{currentstroke}%
\pgfsetdash{}{0pt}%
\pgfpathmoveto{\pgfqpoint{3.284117in}{1.810968in}}%
\pgfpathlineto{\pgfqpoint{3.297570in}{1.799155in}}%
\pgfpathlineto{\pgfqpoint{3.311024in}{1.787481in}}%
\pgfpathlineto{\pgfqpoint{3.324477in}{1.775945in}}%
\pgfpathlineto{\pgfqpoint{3.337931in}{1.764548in}}%
\pgfpathlineto{\pgfqpoint{3.346104in}{1.766120in}}%
\pgfpathlineto{\pgfqpoint{3.354266in}{1.767884in}}%
\pgfpathlineto{\pgfqpoint{3.362417in}{1.769838in}}%
\pgfpathlineto{\pgfqpoint{3.370557in}{1.771978in}}%
\pgfpathlineto{\pgfqpoint{3.357134in}{1.782959in}}%
\pgfpathlineto{\pgfqpoint{3.343710in}{1.794078in}}%
\pgfpathlineto{\pgfqpoint{3.330287in}{1.805336in}}%
\pgfpathlineto{\pgfqpoint{3.316864in}{1.816732in}}%
\pgfpathlineto{\pgfqpoint{3.308694in}{1.815002in}}%
\pgfpathlineto{\pgfqpoint{3.300513in}{1.813462in}}%
\pgfpathlineto{\pgfqpoint{3.292321in}{1.812116in}}%
\pgfpathlineto{\pgfqpoint{3.284117in}{1.810968in}}%
\pgfpathclose%
\pgfusepath{fill}%
\end{pgfscope}%
\begin{pgfscope}%
\pgfpathrectangle{\pgfqpoint{1.254980in}{0.150000in}}{\pgfqpoint{5.490039in}{5.490039in}}%
\pgfusepath{clip}%
\pgfsetbuttcap%
\pgfsetroundjoin%
\definecolor{currentfill}{rgb}{0.188923,0.410910,0.556326}%
\pgfsetfillcolor{currentfill}%
\pgfsetfillopacity{0.700000}%
\pgfsetlinewidth{0.000000pt}%
\definecolor{currentstroke}{rgb}{0.000000,0.000000,0.000000}%
\pgfsetstrokecolor{currentstroke}%
\pgfsetdash{}{0pt}%
\pgfpathmoveto{\pgfqpoint{5.397343in}{2.342363in}}%
\pgfpathlineto{\pgfqpoint{5.411418in}{2.348725in}}%
\pgfpathlineto{\pgfqpoint{5.425509in}{2.355200in}}%
\pgfpathlineto{\pgfqpoint{5.439616in}{2.361788in}}%
\pgfpathlineto{\pgfqpoint{5.453737in}{2.368488in}}%
\pgfpathlineto{\pgfqpoint{5.461117in}{2.379912in}}%
\pgfpathlineto{\pgfqpoint{5.468491in}{2.391263in}}%
\pgfpathlineto{\pgfqpoint{5.475859in}{2.402540in}}%
\pgfpathlineto{\pgfqpoint{5.483220in}{2.413742in}}%
\pgfpathlineto{\pgfqpoint{5.469101in}{2.406971in}}%
\pgfpathlineto{\pgfqpoint{5.454998in}{2.400313in}}%
\pgfpathlineto{\pgfqpoint{5.440910in}{2.393768in}}%
\pgfpathlineto{\pgfqpoint{5.426838in}{2.387335in}}%
\pgfpathlineto{\pgfqpoint{5.419473in}{2.376197in}}%
\pgfpathlineto{\pgfqpoint{5.412102in}{2.364989in}}%
\pgfpathlineto{\pgfqpoint{5.404726in}{2.353711in}}%
\pgfpathlineto{\pgfqpoint{5.397343in}{2.342363in}}%
\pgfpathclose%
\pgfusepath{fill}%
\end{pgfscope}%
\begin{pgfscope}%
\pgfpathrectangle{\pgfqpoint{1.254980in}{0.150000in}}{\pgfqpoint{5.490039in}{5.490039in}}%
\pgfusepath{clip}%
\pgfsetbuttcap%
\pgfsetroundjoin%
\definecolor{currentfill}{rgb}{0.144759,0.519093,0.556572}%
\pgfsetfillcolor{currentfill}%
\pgfsetfillopacity{0.700000}%
\pgfsetlinewidth{0.000000pt}%
\definecolor{currentstroke}{rgb}{0.000000,0.000000,0.000000}%
\pgfsetstrokecolor{currentstroke}%
\pgfsetdash{}{0pt}%
\pgfpathmoveto{\pgfqpoint{2.526439in}{2.717600in}}%
\pgfpathlineto{\pgfqpoint{2.540118in}{2.696735in}}%
\pgfpathlineto{\pgfqpoint{2.553788in}{2.676058in}}%
\pgfpathlineto{\pgfqpoint{2.567450in}{2.655570in}}%
\pgfpathlineto{\pgfqpoint{2.581104in}{2.635268in}}%
\pgfpathlineto{\pgfqpoint{2.589826in}{2.630716in}}%
\pgfpathlineto{\pgfqpoint{2.598529in}{2.626426in}}%
\pgfpathlineto{\pgfqpoint{2.607214in}{2.622391in}}%
\pgfpathlineto{\pgfqpoint{2.615882in}{2.618609in}}%
\pgfpathlineto{\pgfqpoint{2.602276in}{2.638462in}}%
\pgfpathlineto{\pgfqpoint{2.588662in}{2.658499in}}%
\pgfpathlineto{\pgfqpoint{2.575040in}{2.678723in}}%
\pgfpathlineto{\pgfqpoint{2.561411in}{2.699136in}}%
\pgfpathlineto{\pgfqpoint{2.552696in}{2.703361in}}%
\pgfpathlineto{\pgfqpoint{2.543962in}{2.707844in}}%
\pgfpathlineto{\pgfqpoint{2.535210in}{2.712589in}}%
\pgfpathlineto{\pgfqpoint{2.526439in}{2.717600in}}%
\pgfpathclose%
\pgfusepath{fill}%
\end{pgfscope}%
\begin{pgfscope}%
\pgfpathrectangle{\pgfqpoint{1.254980in}{0.150000in}}{\pgfqpoint{5.490039in}{5.490039in}}%
\pgfusepath{clip}%
\pgfsetbuttcap%
\pgfsetroundjoin%
\definecolor{currentfill}{rgb}{0.146616,0.673050,0.508936}%
\pgfsetfillcolor{currentfill}%
\pgfsetfillopacity{0.700000}%
\pgfsetlinewidth{0.000000pt}%
\definecolor{currentstroke}{rgb}{0.000000,0.000000,0.000000}%
\pgfsetstrokecolor{currentstroke}%
\pgfsetdash{}{0pt}%
\pgfpathmoveto{\pgfqpoint{2.286871in}{3.148856in}}%
\pgfpathlineto{\pgfqpoint{2.300699in}{3.124405in}}%
\pgfpathlineto{\pgfqpoint{2.314515in}{3.100170in}}%
\pgfpathlineto{\pgfqpoint{2.328319in}{3.076149in}}%
\pgfpathlineto{\pgfqpoint{2.342112in}{3.052340in}}%
\pgfpathlineto{\pgfqpoint{2.351012in}{3.046576in}}%
\pgfpathlineto{\pgfqpoint{2.359893in}{3.041083in}}%
\pgfpathlineto{\pgfqpoint{2.368753in}{3.035858in}}%
\pgfpathlineto{\pgfqpoint{2.377594in}{3.030896in}}%
\pgfpathlineto{\pgfqpoint{2.363855in}{3.054256in}}%
\pgfpathlineto{\pgfqpoint{2.350104in}{3.077826in}}%
\pgfpathlineto{\pgfqpoint{2.336342in}{3.101610in}}%
\pgfpathlineto{\pgfqpoint{2.322569in}{3.125609in}}%
\pgfpathlineto{\pgfqpoint{2.313675in}{3.131013in}}%
\pgfpathlineto{\pgfqpoint{2.304761in}{3.136687in}}%
\pgfpathlineto{\pgfqpoint{2.295827in}{3.142633in}}%
\pgfpathlineto{\pgfqpoint{2.286871in}{3.148856in}}%
\pgfpathclose%
\pgfusepath{fill}%
\end{pgfscope}%
\begin{pgfscope}%
\pgfpathrectangle{\pgfqpoint{1.254980in}{0.150000in}}{\pgfqpoint{5.490039in}{5.490039in}}%
\pgfusepath{clip}%
\pgfsetbuttcap%
\pgfsetroundjoin%
\definecolor{currentfill}{rgb}{0.266580,0.228262,0.514349}%
\pgfsetfillcolor{currentfill}%
\pgfsetfillopacity{0.700000}%
\pgfsetlinewidth{0.000000pt}%
\definecolor{currentstroke}{rgb}{0.000000,0.000000,0.000000}%
\pgfsetstrokecolor{currentstroke}%
\pgfsetdash{}{0pt}%
\pgfpathmoveto{\pgfqpoint{4.909447in}{1.914380in}}%
\pgfpathlineto{\pgfqpoint{4.923266in}{1.917536in}}%
\pgfpathlineto{\pgfqpoint{4.937096in}{1.920804in}}%
\pgfpathlineto{\pgfqpoint{4.950939in}{1.924184in}}%
\pgfpathlineto{\pgfqpoint{4.964794in}{1.927677in}}%
\pgfpathlineto{\pgfqpoint{4.972338in}{1.939926in}}%
\pgfpathlineto{\pgfqpoint{4.979876in}{1.952145in}}%
\pgfpathlineto{\pgfqpoint{4.987410in}{1.964333in}}%
\pgfpathlineto{\pgfqpoint{4.994939in}{1.976489in}}%
\pgfpathlineto{\pgfqpoint{4.981085in}{1.972811in}}%
\pgfpathlineto{\pgfqpoint{4.967245in}{1.969246in}}%
\pgfpathlineto{\pgfqpoint{4.953416in}{1.965793in}}%
\pgfpathlineto{\pgfqpoint{4.939600in}{1.962454in}}%
\pgfpathlineto{\pgfqpoint{4.932069in}{1.950476in}}%
\pgfpathlineto{\pgfqpoint{4.924533in}{1.938470in}}%
\pgfpathlineto{\pgfqpoint{4.916992in}{1.926438in}}%
\pgfpathlineto{\pgfqpoint{4.909447in}{1.914380in}}%
\pgfpathclose%
\pgfusepath{fill}%
\end{pgfscope}%
\begin{pgfscope}%
\pgfpathrectangle{\pgfqpoint{1.254980in}{0.150000in}}{\pgfqpoint{5.490039in}{5.490039in}}%
\pgfusepath{clip}%
\pgfsetbuttcap%
\pgfsetroundjoin%
\definecolor{currentfill}{rgb}{0.223925,0.334994,0.548053}%
\pgfsetfillcolor{currentfill}%
\pgfsetfillopacity{0.700000}%
\pgfsetlinewidth{0.000000pt}%
\definecolor{currentstroke}{rgb}{0.000000,0.000000,0.000000}%
\pgfsetstrokecolor{currentstroke}%
\pgfsetdash{}{0pt}%
\pgfpathmoveto{\pgfqpoint{5.196142in}{2.156584in}}%
\pgfpathlineto{\pgfqpoint{5.210109in}{2.161733in}}%
\pgfpathlineto{\pgfqpoint{5.224090in}{2.166995in}}%
\pgfpathlineto{\pgfqpoint{5.238085in}{2.172370in}}%
\pgfpathlineto{\pgfqpoint{5.252094in}{2.177857in}}%
\pgfpathlineto{\pgfqpoint{5.259550in}{2.189874in}}%
\pgfpathlineto{\pgfqpoint{5.267001in}{2.201833in}}%
\pgfpathlineto{\pgfqpoint{5.274446in}{2.213732in}}%
\pgfpathlineto{\pgfqpoint{5.281885in}{2.225571in}}%
\pgfpathlineto{\pgfqpoint{5.267877in}{2.219964in}}%
\pgfpathlineto{\pgfqpoint{5.253884in}{2.214469in}}%
\pgfpathlineto{\pgfqpoint{5.239905in}{2.209086in}}%
\pgfpathlineto{\pgfqpoint{5.225940in}{2.203817in}}%
\pgfpathlineto{\pgfqpoint{5.218498in}{2.192092in}}%
\pgfpathlineto{\pgfqpoint{5.211052in}{2.180311in}}%
\pgfpathlineto{\pgfqpoint{5.203599in}{2.168475in}}%
\pgfpathlineto{\pgfqpoint{5.196142in}{2.156584in}}%
\pgfpathclose%
\pgfusepath{fill}%
\end{pgfscope}%
\begin{pgfscope}%
\pgfpathrectangle{\pgfqpoint{1.254980in}{0.150000in}}{\pgfqpoint{5.490039in}{5.490039in}}%
\pgfusepath{clip}%
\pgfsetbuttcap%
\pgfsetroundjoin%
\definecolor{currentfill}{rgb}{0.269944,0.014625,0.341379}%
\pgfsetfillcolor{currentfill}%
\pgfsetfillopacity{0.700000}%
\pgfsetlinewidth{0.000000pt}%
\definecolor{currentstroke}{rgb}{0.000000,0.000000,0.000000}%
\pgfsetstrokecolor{currentstroke}%
\pgfsetdash{}{0pt}%
\pgfpathmoveto{\pgfqpoint{4.142755in}{1.524894in}}%
\pgfpathlineto{\pgfqpoint{4.156283in}{1.521564in}}%
\pgfpathlineto{\pgfqpoint{4.169818in}{1.518351in}}%
\pgfpathlineto{\pgfqpoint{4.183360in}{1.515255in}}%
\pgfpathlineto{\pgfqpoint{4.196909in}{1.512276in}}%
\pgfpathlineto{\pgfqpoint{4.204669in}{1.521410in}}%
\pgfpathlineto{\pgfqpoint{4.212424in}{1.530621in}}%
\pgfpathlineto{\pgfqpoint{4.220173in}{1.539907in}}%
\pgfpathlineto{\pgfqpoint{4.227917in}{1.549266in}}%
\pgfpathlineto{\pgfqpoint{4.214378in}{1.551920in}}%
\pgfpathlineto{\pgfqpoint{4.200847in}{1.554690in}}%
\pgfpathlineto{\pgfqpoint{4.187323in}{1.557578in}}%
\pgfpathlineto{\pgfqpoint{4.173806in}{1.560583in}}%
\pgfpathlineto{\pgfqpoint{4.166051in}{1.551544in}}%
\pgfpathlineto{\pgfqpoint{4.158291in}{1.542581in}}%
\pgfpathlineto{\pgfqpoint{4.150526in}{1.533697in}}%
\pgfpathlineto{\pgfqpoint{4.142755in}{1.524894in}}%
\pgfpathclose%
\pgfusepath{fill}%
\end{pgfscope}%
\begin{pgfscope}%
\pgfpathrectangle{\pgfqpoint{1.254980in}{0.150000in}}{\pgfqpoint{5.490039in}{5.490039in}}%
\pgfusepath{clip}%
\pgfsetbuttcap%
\pgfsetroundjoin%
\definecolor{currentfill}{rgb}{0.283072,0.130895,0.449241}%
\pgfsetfillcolor{currentfill}%
\pgfsetfillopacity{0.700000}%
\pgfsetlinewidth{0.000000pt}%
\definecolor{currentstroke}{rgb}{0.000000,0.000000,0.000000}%
\pgfsetstrokecolor{currentstroke}%
\pgfsetdash{}{0pt}%
\pgfpathmoveto{\pgfqpoint{3.337931in}{1.764548in}}%
\pgfpathlineto{\pgfqpoint{3.351385in}{1.753289in}}%
\pgfpathlineto{\pgfqpoint{3.364839in}{1.742167in}}%
\pgfpathlineto{\pgfqpoint{3.378294in}{1.731181in}}%
\pgfpathlineto{\pgfqpoint{3.391749in}{1.720331in}}%
\pgfpathlineto{\pgfqpoint{3.399893in}{1.722324in}}%
\pgfpathlineto{\pgfqpoint{3.408025in}{1.724506in}}%
\pgfpathlineto{\pgfqpoint{3.416147in}{1.726873in}}%
\pgfpathlineto{\pgfqpoint{3.424259in}{1.729421in}}%
\pgfpathlineto{\pgfqpoint{3.410833in}{1.739856in}}%
\pgfpathlineto{\pgfqpoint{3.397407in}{1.750427in}}%
\pgfpathlineto{\pgfqpoint{3.383982in}{1.761134in}}%
\pgfpathlineto{\pgfqpoint{3.370557in}{1.771978in}}%
\pgfpathlineto{\pgfqpoint{3.362417in}{1.769838in}}%
\pgfpathlineto{\pgfqpoint{3.354266in}{1.767884in}}%
\pgfpathlineto{\pgfqpoint{3.346104in}{1.766120in}}%
\pgfpathlineto{\pgfqpoint{3.337931in}{1.764548in}}%
\pgfpathclose%
\pgfusepath{fill}%
\end{pgfscope}%
\begin{pgfscope}%
\pgfpathrectangle{\pgfqpoint{1.254980in}{0.150000in}}{\pgfqpoint{5.490039in}{5.490039in}}%
\pgfusepath{clip}%
\pgfsetbuttcap%
\pgfsetroundjoin%
\definecolor{currentfill}{rgb}{0.277941,0.056324,0.381191}%
\pgfsetfillcolor{currentfill}%
\pgfsetfillopacity{0.700000}%
\pgfsetlinewidth{0.000000pt}%
\definecolor{currentstroke}{rgb}{0.000000,0.000000,0.000000}%
\pgfsetstrokecolor{currentstroke}%
\pgfsetdash{}{0pt}%
\pgfpathmoveto{\pgfqpoint{3.585492in}{1.614586in}}%
\pgfpathlineto{\pgfqpoint{3.598941in}{1.605867in}}%
\pgfpathlineto{\pgfqpoint{3.612392in}{1.597277in}}%
\pgfpathlineto{\pgfqpoint{3.625846in}{1.588815in}}%
\pgfpathlineto{\pgfqpoint{3.639302in}{1.580481in}}%
\pgfpathlineto{\pgfqpoint{3.647303in}{1.584812in}}%
\pgfpathlineto{\pgfqpoint{3.655295in}{1.589302in}}%
\pgfpathlineto{\pgfqpoint{3.663279in}{1.593945in}}%
\pgfpathlineto{\pgfqpoint{3.671254in}{1.598740in}}%
\pgfpathlineto{\pgfqpoint{3.657820in}{1.606681in}}%
\pgfpathlineto{\pgfqpoint{3.644389in}{1.614750in}}%
\pgfpathlineto{\pgfqpoint{3.630961in}{1.622947in}}%
\pgfpathlineto{\pgfqpoint{3.617535in}{1.631272in}}%
\pgfpathlineto{\pgfqpoint{3.609538in}{1.626864in}}%
\pgfpathlineto{\pgfqpoint{3.601531in}{1.622612in}}%
\pgfpathlineto{\pgfqpoint{3.593516in}{1.618518in}}%
\pgfpathlineto{\pgfqpoint{3.585492in}{1.614586in}}%
\pgfpathclose%
\pgfusepath{fill}%
\end{pgfscope}%
\begin{pgfscope}%
\pgfpathrectangle{\pgfqpoint{1.254980in}{0.150000in}}{\pgfqpoint{5.490039in}{5.490039in}}%
\pgfusepath{clip}%
\pgfsetbuttcap%
\pgfsetroundjoin%
\definecolor{currentfill}{rgb}{0.132444,0.552216,0.553018}%
\pgfsetfillcolor{currentfill}%
\pgfsetfillopacity{0.700000}%
\pgfsetlinewidth{0.000000pt}%
\definecolor{currentstroke}{rgb}{0.000000,0.000000,0.000000}%
\pgfsetstrokecolor{currentstroke}%
\pgfsetdash{}{0pt}%
\pgfpathmoveto{\pgfqpoint{2.471639in}{2.802973in}}%
\pgfpathlineto{\pgfqpoint{2.485352in}{2.781340in}}%
\pgfpathlineto{\pgfqpoint{2.499057in}{2.759901in}}%
\pgfpathlineto{\pgfqpoint{2.512752in}{2.738655in}}%
\pgfpathlineto{\pgfqpoint{2.526439in}{2.717600in}}%
\pgfpathlineto{\pgfqpoint{2.535210in}{2.712589in}}%
\pgfpathlineto{\pgfqpoint{2.543962in}{2.707844in}}%
\pgfpathlineto{\pgfqpoint{2.552696in}{2.703361in}}%
\pgfpathlineto{\pgfqpoint{2.561411in}{2.699136in}}%
\pgfpathlineto{\pgfqpoint{2.547774in}{2.719737in}}%
\pgfpathlineto{\pgfqpoint{2.534128in}{2.740529in}}%
\pgfpathlineto{\pgfqpoint{2.520474in}{2.761512in}}%
\pgfpathlineto{\pgfqpoint{2.506811in}{2.782688in}}%
\pgfpathlineto{\pgfqpoint{2.498046in}{2.787361in}}%
\pgfpathlineto{\pgfqpoint{2.489263in}{2.792297in}}%
\pgfpathlineto{\pgfqpoint{2.480461in}{2.797500in}}%
\pgfpathlineto{\pgfqpoint{2.471639in}{2.802973in}}%
\pgfpathclose%
\pgfusepath{fill}%
\end{pgfscope}%
\begin{pgfscope}%
\pgfpathrectangle{\pgfqpoint{1.254980in}{0.150000in}}{\pgfqpoint{5.490039in}{5.490039in}}%
\pgfusepath{clip}%
\pgfsetbuttcap%
\pgfsetroundjoin%
\definecolor{currentfill}{rgb}{0.267004,0.004874,0.329415}%
\pgfsetfillcolor{currentfill}%
\pgfsetfillopacity{0.700000}%
\pgfsetlinewidth{0.000000pt}%
\definecolor{currentstroke}{rgb}{0.000000,0.000000,0.000000}%
\pgfsetstrokecolor{currentstroke}%
\pgfsetdash{}{0pt}%
\pgfpathmoveto{\pgfqpoint{3.918206in}{1.515330in}}%
\pgfpathlineto{\pgfqpoint{3.931691in}{1.509872in}}%
\pgfpathlineto{\pgfqpoint{3.945180in}{1.504534in}}%
\pgfpathlineto{\pgfqpoint{3.958675in}{1.499317in}}%
\pgfpathlineto{\pgfqpoint{3.972175in}{1.494220in}}%
\pgfpathlineto{\pgfqpoint{3.980021in}{1.501545in}}%
\pgfpathlineto{\pgfqpoint{3.987860in}{1.508982in}}%
\pgfpathlineto{\pgfqpoint{3.995693in}{1.516528in}}%
\pgfpathlineto{\pgfqpoint{4.003519in}{1.524180in}}%
\pgfpathlineto{\pgfqpoint{3.990034in}{1.528919in}}%
\pgfpathlineto{\pgfqpoint{3.976555in}{1.533779in}}%
\pgfpathlineto{\pgfqpoint{3.963081in}{1.538758in}}%
\pgfpathlineto{\pgfqpoint{3.949612in}{1.543859in}}%
\pgfpathlineto{\pgfqpoint{3.941771in}{1.536559in}}%
\pgfpathlineto{\pgfqpoint{3.933923in}{1.529368in}}%
\pgfpathlineto{\pgfqpoint{3.926068in}{1.522291in}}%
\pgfpathlineto{\pgfqpoint{3.918206in}{1.515330in}}%
\pgfpathclose%
\pgfusepath{fill}%
\end{pgfscope}%
\begin{pgfscope}%
\pgfpathrectangle{\pgfqpoint{1.254980in}{0.150000in}}{\pgfqpoint{5.490039in}{5.490039in}}%
\pgfusepath{clip}%
\pgfsetbuttcap%
\pgfsetroundjoin%
\definecolor{currentfill}{rgb}{0.175841,0.441290,0.557685}%
\pgfsetfillcolor{currentfill}%
\pgfsetfillopacity{0.700000}%
\pgfsetlinewidth{0.000000pt}%
\definecolor{currentstroke}{rgb}{0.000000,0.000000,0.000000}%
\pgfsetstrokecolor{currentstroke}%
\pgfsetdash{}{0pt}%
\pgfpathmoveto{\pgfqpoint{5.483220in}{2.413742in}}%
\pgfpathlineto{\pgfqpoint{5.497354in}{2.420626in}}%
\pgfpathlineto{\pgfqpoint{5.511503in}{2.427622in}}%
\pgfpathlineto{\pgfqpoint{5.525668in}{2.434732in}}%
\pgfpathlineto{\pgfqpoint{5.539849in}{2.441954in}}%
\pgfpathlineto{\pgfqpoint{5.547200in}{2.453142in}}%
\pgfpathlineto{\pgfqpoint{5.554545in}{2.464250in}}%
\pgfpathlineto{\pgfqpoint{5.561883in}{2.475280in}}%
\pgfpathlineto{\pgfqpoint{5.569214in}{2.486230in}}%
\pgfpathlineto{\pgfqpoint{5.555037in}{2.478954in}}%
\pgfpathlineto{\pgfqpoint{5.540876in}{2.471791in}}%
\pgfpathlineto{\pgfqpoint{5.526730in}{2.464740in}}%
\pgfpathlineto{\pgfqpoint{5.512599in}{2.457803in}}%
\pgfpathlineto{\pgfqpoint{5.505264in}{2.446900in}}%
\pgfpathlineto{\pgfqpoint{5.497923in}{2.435923in}}%
\pgfpathlineto{\pgfqpoint{5.490574in}{2.424870in}}%
\pgfpathlineto{\pgfqpoint{5.483220in}{2.413742in}}%
\pgfpathclose%
\pgfusepath{fill}%
\end{pgfscope}%
\begin{pgfscope}%
\pgfpathrectangle{\pgfqpoint{1.254980in}{0.150000in}}{\pgfqpoint{5.490039in}{5.490039in}}%
\pgfusepath{clip}%
\pgfsetbuttcap%
\pgfsetroundjoin%
\definecolor{currentfill}{rgb}{0.269944,0.014625,0.341379}%
\pgfsetfillcolor{currentfill}%
\pgfsetfillopacity{0.700000}%
\pgfsetlinewidth{0.000000pt}%
\definecolor{currentstroke}{rgb}{0.000000,0.000000,0.000000}%
\pgfsetstrokecolor{currentstroke}%
\pgfsetdash{}{0pt}%
\pgfpathmoveto{\pgfqpoint{3.778847in}{1.539754in}}%
\pgfpathlineto{\pgfqpoint{3.792312in}{1.532943in}}%
\pgfpathlineto{\pgfqpoint{3.805782in}{1.526255in}}%
\pgfpathlineto{\pgfqpoint{3.819256in}{1.519690in}}%
\pgfpathlineto{\pgfqpoint{3.832735in}{1.513249in}}%
\pgfpathlineto{\pgfqpoint{3.840641in}{1.519336in}}%
\pgfpathlineto{\pgfqpoint{3.848541in}{1.525555in}}%
\pgfpathlineto{\pgfqpoint{3.856433in}{1.531903in}}%
\pgfpathlineto{\pgfqpoint{3.864318in}{1.538377in}}%
\pgfpathlineto{\pgfqpoint{3.850858in}{1.544444in}}%
\pgfpathlineto{\pgfqpoint{3.837403in}{1.550634in}}%
\pgfpathlineto{\pgfqpoint{3.823952in}{1.556947in}}%
\pgfpathlineto{\pgfqpoint{3.810505in}{1.563384in}}%
\pgfpathlineto{\pgfqpoint{3.802601in}{1.557278in}}%
\pgfpathlineto{\pgfqpoint{3.794691in}{1.551302in}}%
\pgfpathlineto{\pgfqpoint{3.786772in}{1.545460in}}%
\pgfpathlineto{\pgfqpoint{3.778847in}{1.539754in}}%
\pgfpathclose%
\pgfusepath{fill}%
\end{pgfscope}%
\begin{pgfscope}%
\pgfpathrectangle{\pgfqpoint{1.254980in}{0.150000in}}{\pgfqpoint{5.490039in}{5.490039in}}%
\pgfusepath{clip}%
\pgfsetbuttcap%
\pgfsetroundjoin%
\definecolor{currentfill}{rgb}{0.255645,0.260703,0.528312}%
\pgfsetfillcolor{currentfill}%
\pgfsetfillopacity{0.700000}%
\pgfsetlinewidth{0.000000pt}%
\definecolor{currentstroke}{rgb}{0.000000,0.000000,0.000000}%
\pgfsetstrokecolor{currentstroke}%
\pgfsetdash{}{0pt}%
\pgfpathmoveto{\pgfqpoint{4.994939in}{1.976489in}}%
\pgfpathlineto{\pgfqpoint{5.008805in}{1.980280in}}%
\pgfpathlineto{\pgfqpoint{5.022684in}{1.984183in}}%
\pgfpathlineto{\pgfqpoint{5.036576in}{1.988198in}}%
\pgfpathlineto{\pgfqpoint{5.050480in}{1.992326in}}%
\pgfpathlineto{\pgfqpoint{5.058003in}{2.004624in}}%
\pgfpathlineto{\pgfqpoint{5.065520in}{2.016884in}}%
\pgfpathlineto{\pgfqpoint{5.073032in}{2.029103in}}%
\pgfpathlineto{\pgfqpoint{5.080539in}{2.041282in}}%
\pgfpathlineto{\pgfqpoint{5.066636in}{2.036985in}}%
\pgfpathlineto{\pgfqpoint{5.052746in}{2.032800in}}%
\pgfpathlineto{\pgfqpoint{5.038869in}{2.028728in}}%
\pgfpathlineto{\pgfqpoint{5.025005in}{2.024769in}}%
\pgfpathlineto{\pgfqpoint{5.017496in}{2.012753in}}%
\pgfpathlineto{\pgfqpoint{5.009982in}{2.000700in}}%
\pgfpathlineto{\pgfqpoint{5.002463in}{1.988612in}}%
\pgfpathlineto{\pgfqpoint{4.994939in}{1.976489in}}%
\pgfpathclose%
\pgfusepath{fill}%
\end{pgfscope}%
\begin{pgfscope}%
\pgfpathrectangle{\pgfqpoint{1.254980in}{0.150000in}}{\pgfqpoint{5.490039in}{5.490039in}}%
\pgfusepath{clip}%
\pgfsetbuttcap%
\pgfsetroundjoin%
\definecolor{currentfill}{rgb}{0.282656,0.100196,0.422160}%
\pgfsetfillcolor{currentfill}%
\pgfsetfillopacity{0.700000}%
\pgfsetlinewidth{0.000000pt}%
\definecolor{currentstroke}{rgb}{0.000000,0.000000,0.000000}%
\pgfsetstrokecolor{currentstroke}%
\pgfsetdash{}{0pt}%
\pgfpathmoveto{\pgfqpoint{4.537777in}{1.653697in}}%
\pgfpathlineto{\pgfqpoint{4.551438in}{1.653882in}}%
\pgfpathlineto{\pgfqpoint{4.565108in}{1.654181in}}%
\pgfpathlineto{\pgfqpoint{4.578789in}{1.654594in}}%
\pgfpathlineto{\pgfqpoint{4.592479in}{1.655121in}}%
\pgfpathlineto{\pgfqpoint{4.600125in}{1.666544in}}%
\pgfpathlineto{\pgfqpoint{4.607766in}{1.677988in}}%
\pgfpathlineto{\pgfqpoint{4.615404in}{1.689451in}}%
\pgfpathlineto{\pgfqpoint{4.623036in}{1.700929in}}%
\pgfpathlineto{\pgfqpoint{4.609350in}{1.700140in}}%
\pgfpathlineto{\pgfqpoint{4.595675in}{1.699464in}}%
\pgfpathlineto{\pgfqpoint{4.582009in}{1.698901in}}%
\pgfpathlineto{\pgfqpoint{4.568354in}{1.698453in}}%
\pgfpathlineto{\pgfqpoint{4.560716in}{1.687231in}}%
\pgfpathlineto{\pgfqpoint{4.553074in}{1.676030in}}%
\pgfpathlineto{\pgfqpoint{4.545428in}{1.664851in}}%
\pgfpathlineto{\pgfqpoint{4.537777in}{1.653697in}}%
\pgfpathclose%
\pgfusepath{fill}%
\end{pgfscope}%
\begin{pgfscope}%
\pgfpathrectangle{\pgfqpoint{1.254980in}{0.150000in}}{\pgfqpoint{5.490039in}{5.490039in}}%
\pgfusepath{clip}%
\pgfsetbuttcap%
\pgfsetroundjoin%
\definecolor{currentfill}{rgb}{0.280267,0.073417,0.397163}%
\pgfsetfillcolor{currentfill}%
\pgfsetfillopacity{0.700000}%
\pgfsetlinewidth{0.000000pt}%
\definecolor{currentstroke}{rgb}{0.000000,0.000000,0.000000}%
\pgfsetstrokecolor{currentstroke}%
\pgfsetdash{}{0pt}%
\pgfpathmoveto{\pgfqpoint{4.452556in}{1.610876in}}%
\pgfpathlineto{\pgfqpoint{4.466185in}{1.610327in}}%
\pgfpathlineto{\pgfqpoint{4.479824in}{1.609892in}}%
\pgfpathlineto{\pgfqpoint{4.493471in}{1.609571in}}%
\pgfpathlineto{\pgfqpoint{4.507128in}{1.609364in}}%
\pgfpathlineto{\pgfqpoint{4.514797in}{1.620400in}}%
\pgfpathlineto{\pgfqpoint{4.522461in}{1.631469in}}%
\pgfpathlineto{\pgfqpoint{4.530121in}{1.642568in}}%
\pgfpathlineto{\pgfqpoint{4.537777in}{1.653697in}}%
\pgfpathlineto{\pgfqpoint{4.524126in}{1.653625in}}%
\pgfpathlineto{\pgfqpoint{4.510484in}{1.653667in}}%
\pgfpathlineto{\pgfqpoint{4.496852in}{1.653824in}}%
\pgfpathlineto{\pgfqpoint{4.483229in}{1.654095in}}%
\pgfpathlineto{\pgfqpoint{4.475568in}{1.643239in}}%
\pgfpathlineto{\pgfqpoint{4.467902in}{1.632416in}}%
\pgfpathlineto{\pgfqpoint{4.460231in}{1.621628in}}%
\pgfpathlineto{\pgfqpoint{4.452556in}{1.610876in}}%
\pgfpathclose%
\pgfusepath{fill}%
\end{pgfscope}%
\begin{pgfscope}%
\pgfpathrectangle{\pgfqpoint{1.254980in}{0.150000in}}{\pgfqpoint{5.490039in}{5.490039in}}%
\pgfusepath{clip}%
\pgfsetbuttcap%
\pgfsetroundjoin%
\definecolor{currentfill}{rgb}{0.283091,0.110553,0.431554}%
\pgfsetfillcolor{currentfill}%
\pgfsetfillopacity{0.700000}%
\pgfsetlinewidth{0.000000pt}%
\definecolor{currentstroke}{rgb}{0.000000,0.000000,0.000000}%
\pgfsetstrokecolor{currentstroke}%
\pgfsetdash{}{0pt}%
\pgfpathmoveto{\pgfqpoint{3.391749in}{1.720331in}}%
\pgfpathlineto{\pgfqpoint{3.405205in}{1.709617in}}%
\pgfpathlineto{\pgfqpoint{3.418662in}{1.699037in}}%
\pgfpathlineto{\pgfqpoint{3.432120in}{1.688592in}}%
\pgfpathlineto{\pgfqpoint{3.445579in}{1.678280in}}%
\pgfpathlineto{\pgfqpoint{3.453694in}{1.680694in}}%
\pgfpathlineto{\pgfqpoint{3.461798in}{1.683291in}}%
\pgfpathlineto{\pgfqpoint{3.469893in}{1.686070in}}%
\pgfpathlineto{\pgfqpoint{3.477977in}{1.689026in}}%
\pgfpathlineto{\pgfqpoint{3.464546in}{1.698924in}}%
\pgfpathlineto{\pgfqpoint{3.451116in}{1.708955in}}%
\pgfpathlineto{\pgfqpoint{3.437687in}{1.719121in}}%
\pgfpathlineto{\pgfqpoint{3.424259in}{1.729421in}}%
\pgfpathlineto{\pgfqpoint{3.416147in}{1.726873in}}%
\pgfpathlineto{\pgfqpoint{3.408025in}{1.724506in}}%
\pgfpathlineto{\pgfqpoint{3.399893in}{1.722324in}}%
\pgfpathlineto{\pgfqpoint{3.391749in}{1.720331in}}%
\pgfpathclose%
\pgfusepath{fill}%
\end{pgfscope}%
\begin{pgfscope}%
\pgfpathrectangle{\pgfqpoint{1.254980in}{0.150000in}}{\pgfqpoint{5.490039in}{5.490039in}}%
\pgfusepath{clip}%
\pgfsetbuttcap%
\pgfsetroundjoin%
\definecolor{currentfill}{rgb}{0.208623,0.367752,0.552675}%
\pgfsetfillcolor{currentfill}%
\pgfsetfillopacity{0.700000}%
\pgfsetlinewidth{0.000000pt}%
\definecolor{currentstroke}{rgb}{0.000000,0.000000,0.000000}%
\pgfsetstrokecolor{currentstroke}%
\pgfsetdash{}{0pt}%
\pgfpathmoveto{\pgfqpoint{5.281885in}{2.225571in}}%
\pgfpathlineto{\pgfqpoint{5.295906in}{2.231291in}}%
\pgfpathlineto{\pgfqpoint{5.309943in}{2.237124in}}%
\pgfpathlineto{\pgfqpoint{5.323993in}{2.243069in}}%
\pgfpathlineto{\pgfqpoint{5.338059in}{2.249127in}}%
\pgfpathlineto{\pgfqpoint{5.345490in}{2.261016in}}%
\pgfpathlineto{\pgfqpoint{5.352916in}{2.272838in}}%
\pgfpathlineto{\pgfqpoint{5.360335in}{2.284595in}}%
\pgfpathlineto{\pgfqpoint{5.367749in}{2.296284in}}%
\pgfpathlineto{\pgfqpoint{5.353685in}{2.290122in}}%
\pgfpathlineto{\pgfqpoint{5.339637in}{2.284073in}}%
\pgfpathlineto{\pgfqpoint{5.325602in}{2.278136in}}%
\pgfpathlineto{\pgfqpoint{5.311583in}{2.272312in}}%
\pgfpathlineto{\pgfqpoint{5.304167in}{2.260720in}}%
\pgfpathlineto{\pgfqpoint{5.296745in}{2.249066in}}%
\pgfpathlineto{\pgfqpoint{5.289318in}{2.237349in}}%
\pgfpathlineto{\pgfqpoint{5.281885in}{2.225571in}}%
\pgfpathclose%
\pgfusepath{fill}%
\end{pgfscope}%
\begin{pgfscope}%
\pgfpathrectangle{\pgfqpoint{1.254980in}{0.150000in}}{\pgfqpoint{5.490039in}{5.490039in}}%
\pgfusepath{clip}%
\pgfsetbuttcap%
\pgfsetroundjoin%
\definecolor{currentfill}{rgb}{0.196571,0.711827,0.479221}%
\pgfsetfillcolor{currentfill}%
\pgfsetfillopacity{0.700000}%
\pgfsetlinewidth{0.000000pt}%
\definecolor{currentstroke}{rgb}{0.000000,0.000000,0.000000}%
\pgfsetstrokecolor{currentstroke}%
\pgfsetdash{}{0pt}%
\pgfpathmoveto{\pgfqpoint{2.231438in}{3.248850in}}%
\pgfpathlineto{\pgfqpoint{2.245315in}{3.223520in}}%
\pgfpathlineto{\pgfqpoint{2.259180in}{3.198412in}}%
\pgfpathlineto{\pgfqpoint{2.273032in}{3.173525in}}%
\pgfpathlineto{\pgfqpoint{2.286871in}{3.148856in}}%
\pgfpathlineto{\pgfqpoint{2.295827in}{3.142633in}}%
\pgfpathlineto{\pgfqpoint{2.304761in}{3.136687in}}%
\pgfpathlineto{\pgfqpoint{2.313675in}{3.131013in}}%
\pgfpathlineto{\pgfqpoint{2.322569in}{3.125609in}}%
\pgfpathlineto{\pgfqpoint{2.308784in}{3.149824in}}%
\pgfpathlineto{\pgfqpoint{2.294987in}{3.174256in}}%
\pgfpathlineto{\pgfqpoint{2.281178in}{3.198908in}}%
\pgfpathlineto{\pgfqpoint{2.267357in}{3.223781in}}%
\pgfpathlineto{\pgfqpoint{2.258409in}{3.229633in}}%
\pgfpathlineto{\pgfqpoint{2.249440in}{3.235759in}}%
\pgfpathlineto{\pgfqpoint{2.240450in}{3.242164in}}%
\pgfpathlineto{\pgfqpoint{2.231438in}{3.248850in}}%
\pgfpathclose%
\pgfusepath{fill}%
\end{pgfscope}%
\begin{pgfscope}%
\pgfpathrectangle{\pgfqpoint{1.254980in}{0.150000in}}{\pgfqpoint{5.490039in}{5.490039in}}%
\pgfusepath{clip}%
\pgfsetbuttcap%
\pgfsetroundjoin%
\definecolor{currentfill}{rgb}{0.283187,0.125848,0.444960}%
\pgfsetfillcolor{currentfill}%
\pgfsetfillopacity{0.700000}%
\pgfsetlinewidth{0.000000pt}%
\definecolor{currentstroke}{rgb}{0.000000,0.000000,0.000000}%
\pgfsetstrokecolor{currentstroke}%
\pgfsetdash{}{0pt}%
\pgfpathmoveto{\pgfqpoint{4.623036in}{1.700929in}}%
\pgfpathlineto{\pgfqpoint{4.636732in}{1.701832in}}%
\pgfpathlineto{\pgfqpoint{4.650438in}{1.702849in}}%
\pgfpathlineto{\pgfqpoint{4.664155in}{1.703978in}}%
\pgfpathlineto{\pgfqpoint{4.677882in}{1.705221in}}%
\pgfpathlineto{\pgfqpoint{4.685507in}{1.716967in}}%
\pgfpathlineto{\pgfqpoint{4.693126in}{1.728723in}}%
\pgfpathlineto{\pgfqpoint{4.700742in}{1.740485in}}%
\pgfpathlineto{\pgfqpoint{4.708352in}{1.752251in}}%
\pgfpathlineto{\pgfqpoint{4.694629in}{1.750761in}}%
\pgfpathlineto{\pgfqpoint{4.680916in}{1.749384in}}%
\pgfpathlineto{\pgfqpoint{4.667213in}{1.748120in}}%
\pgfpathlineto{\pgfqpoint{4.653522in}{1.746970in}}%
\pgfpathlineto{\pgfqpoint{4.645907in}{1.735444in}}%
\pgfpathlineto{\pgfqpoint{4.638288in}{1.723928in}}%
\pgfpathlineto{\pgfqpoint{4.630664in}{1.712422in}}%
\pgfpathlineto{\pgfqpoint{4.623036in}{1.700929in}}%
\pgfpathclose%
\pgfusepath{fill}%
\end{pgfscope}%
\begin{pgfscope}%
\pgfpathrectangle{\pgfqpoint{1.254980in}{0.150000in}}{\pgfqpoint{5.490039in}{5.490039in}}%
\pgfusepath{clip}%
\pgfsetbuttcap%
\pgfsetroundjoin%
\definecolor{currentfill}{rgb}{0.267004,0.004874,0.329415}%
\pgfsetfillcolor{currentfill}%
\pgfsetfillopacity{0.700000}%
\pgfsetlinewidth{0.000000pt}%
\definecolor{currentstroke}{rgb}{0.000000,0.000000,0.000000}%
\pgfsetstrokecolor{currentstroke}%
\pgfsetdash{}{0pt}%
\pgfpathmoveto{\pgfqpoint{4.057518in}{1.506417in}}%
\pgfpathlineto{\pgfqpoint{4.071033in}{1.502274in}}%
\pgfpathlineto{\pgfqpoint{4.084554in}{1.498249in}}%
\pgfpathlineto{\pgfqpoint{4.098081in}{1.494342in}}%
\pgfpathlineto{\pgfqpoint{4.111615in}{1.490553in}}%
\pgfpathlineto{\pgfqpoint{4.119408in}{1.499002in}}%
\pgfpathlineto{\pgfqpoint{4.127196in}{1.507544in}}%
\pgfpathlineto{\pgfqpoint{4.134978in}{1.516176in}}%
\pgfpathlineto{\pgfqpoint{4.142755in}{1.524894in}}%
\pgfpathlineto{\pgfqpoint{4.129233in}{1.528342in}}%
\pgfpathlineto{\pgfqpoint{4.115719in}{1.531908in}}%
\pgfpathlineto{\pgfqpoint{4.102210in}{1.535591in}}%
\pgfpathlineto{\pgfqpoint{4.088709in}{1.539394in}}%
\pgfpathlineto{\pgfqpoint{4.080920in}{1.531010in}}%
\pgfpathlineto{\pgfqpoint{4.073125in}{1.522718in}}%
\pgfpathlineto{\pgfqpoint{4.065324in}{1.514519in}}%
\pgfpathlineto{\pgfqpoint{4.057518in}{1.506417in}}%
\pgfpathclose%
\pgfusepath{fill}%
\end{pgfscope}%
\begin{pgfscope}%
\pgfpathrectangle{\pgfqpoint{1.254980in}{0.150000in}}{\pgfqpoint{5.490039in}{5.490039in}}%
\pgfusepath{clip}%
\pgfsetbuttcap%
\pgfsetroundjoin%
\definecolor{currentfill}{rgb}{0.277018,0.050344,0.375715}%
\pgfsetfillcolor{currentfill}%
\pgfsetfillopacity{0.700000}%
\pgfsetlinewidth{0.000000pt}%
\definecolor{currentstroke}{rgb}{0.000000,0.000000,0.000000}%
\pgfsetstrokecolor{currentstroke}%
\pgfsetdash{}{0pt}%
\pgfpathmoveto{\pgfqpoint{4.367354in}{1.572801in}}%
\pgfpathlineto{\pgfqpoint{4.380955in}{1.571499in}}%
\pgfpathlineto{\pgfqpoint{4.394564in}{1.570312in}}%
\pgfpathlineto{\pgfqpoint{4.408182in}{1.569239in}}%
\pgfpathlineto{\pgfqpoint{4.421809in}{1.568281in}}%
\pgfpathlineto{\pgfqpoint{4.429503in}{1.578864in}}%
\pgfpathlineto{\pgfqpoint{4.437192in}{1.589492in}}%
\pgfpathlineto{\pgfqpoint{4.444876in}{1.600163in}}%
\pgfpathlineto{\pgfqpoint{4.452556in}{1.610876in}}%
\pgfpathlineto{\pgfqpoint{4.438936in}{1.611540in}}%
\pgfpathlineto{\pgfqpoint{4.425325in}{1.612318in}}%
\pgfpathlineto{\pgfqpoint{4.411723in}{1.613211in}}%
\pgfpathlineto{\pgfqpoint{4.398129in}{1.614219in}}%
\pgfpathlineto{\pgfqpoint{4.390443in}{1.603794in}}%
\pgfpathlineto{\pgfqpoint{4.382751in}{1.593415in}}%
\pgfpathlineto{\pgfqpoint{4.375055in}{1.583083in}}%
\pgfpathlineto{\pgfqpoint{4.367354in}{1.572801in}}%
\pgfpathclose%
\pgfusepath{fill}%
\end{pgfscope}%
\begin{pgfscope}%
\pgfpathrectangle{\pgfqpoint{1.254980in}{0.150000in}}{\pgfqpoint{5.490039in}{5.490039in}}%
\pgfusepath{clip}%
\pgfsetbuttcap%
\pgfsetroundjoin%
\definecolor{currentfill}{rgb}{0.165117,0.467423,0.558141}%
\pgfsetfillcolor{currentfill}%
\pgfsetfillopacity{0.700000}%
\pgfsetlinewidth{0.000000pt}%
\definecolor{currentstroke}{rgb}{0.000000,0.000000,0.000000}%
\pgfsetstrokecolor{currentstroke}%
\pgfsetdash{}{0pt}%
\pgfpathmoveto{\pgfqpoint{5.569214in}{2.486230in}}%
\pgfpathlineto{\pgfqpoint{5.583408in}{2.493619in}}%
\pgfpathlineto{\pgfqpoint{5.597617in}{2.501120in}}%
\pgfpathlineto{\pgfqpoint{5.611842in}{2.508735in}}%
\pgfpathlineto{\pgfqpoint{5.619164in}{2.519638in}}%
\pgfpathlineto{\pgfqpoint{5.626479in}{2.530458in}}%
\pgfpathlineto{\pgfqpoint{5.633787in}{2.541195in}}%
\pgfpathlineto{\pgfqpoint{5.641088in}{2.551850in}}%
\pgfpathlineto{\pgfqpoint{5.626866in}{2.544198in}}%
\pgfpathlineto{\pgfqpoint{5.612661in}{2.536659in}}%
\pgfpathlineto{\pgfqpoint{5.598472in}{2.529233in}}%
\pgfpathlineto{\pgfqpoint{5.591168in}{2.518602in}}%
\pgfpathlineto{\pgfqpoint{5.583857in}{2.507891in}}%
\pgfpathlineto{\pgfqpoint{5.576539in}{2.497100in}}%
\pgfpathlineto{\pgfqpoint{5.569214in}{2.486230in}}%
\pgfpathclose%
\pgfusepath{fill}%
\end{pgfscope}%
\begin{pgfscope}%
\pgfpathrectangle{\pgfqpoint{1.254980in}{0.150000in}}{\pgfqpoint{5.490039in}{5.490039in}}%
\pgfusepath{clip}%
\pgfsetbuttcap%
\pgfsetroundjoin%
\definecolor{currentfill}{rgb}{0.122606,0.585371,0.546557}%
\pgfsetfillcolor{currentfill}%
\pgfsetfillopacity{0.700000}%
\pgfsetlinewidth{0.000000pt}%
\definecolor{currentstroke}{rgb}{0.000000,0.000000,0.000000}%
\pgfsetstrokecolor{currentstroke}%
\pgfsetdash{}{0pt}%
\pgfpathmoveto{\pgfqpoint{2.416691in}{2.891471in}}%
\pgfpathlineto{\pgfqpoint{2.430442in}{2.869049in}}%
\pgfpathlineto{\pgfqpoint{2.444184in}{2.846826in}}%
\pgfpathlineto{\pgfqpoint{2.457916in}{2.824801in}}%
\pgfpathlineto{\pgfqpoint{2.471639in}{2.802973in}}%
\pgfpathlineto{\pgfqpoint{2.480461in}{2.797500in}}%
\pgfpathlineto{\pgfqpoint{2.489263in}{2.792297in}}%
\pgfpathlineto{\pgfqpoint{2.498046in}{2.787361in}}%
\pgfpathlineto{\pgfqpoint{2.506811in}{2.782688in}}%
\pgfpathlineto{\pgfqpoint{2.493139in}{2.804059in}}%
\pgfpathlineto{\pgfqpoint{2.479458in}{2.825625in}}%
\pgfpathlineto{\pgfqpoint{2.465768in}{2.847388in}}%
\pgfpathlineto{\pgfqpoint{2.452069in}{2.869349in}}%
\pgfpathlineto{\pgfqpoint{2.443254in}{2.874473in}}%
\pgfpathlineto{\pgfqpoint{2.434419in}{2.879865in}}%
\pgfpathlineto{\pgfqpoint{2.425565in}{2.885530in}}%
\pgfpathlineto{\pgfqpoint{2.416691in}{2.891471in}}%
\pgfpathclose%
\pgfusepath{fill}%
\end{pgfscope}%
\begin{pgfscope}%
\pgfpathrectangle{\pgfqpoint{1.254980in}{0.150000in}}{\pgfqpoint{5.490039in}{5.490039in}}%
\pgfusepath{clip}%
\pgfsetbuttcap%
\pgfsetroundjoin%
\definecolor{currentfill}{rgb}{0.281887,0.150881,0.465405}%
\pgfsetfillcolor{currentfill}%
\pgfsetfillopacity{0.700000}%
\pgfsetlinewidth{0.000000pt}%
\definecolor{currentstroke}{rgb}{0.000000,0.000000,0.000000}%
\pgfsetstrokecolor{currentstroke}%
\pgfsetdash{}{0pt}%
\pgfpathmoveto{\pgfqpoint{4.708352in}{1.752251in}}%
\pgfpathlineto{\pgfqpoint{4.722087in}{1.753855in}}%
\pgfpathlineto{\pgfqpoint{4.735832in}{1.755571in}}%
\pgfpathlineto{\pgfqpoint{4.749588in}{1.757401in}}%
\pgfpathlineto{\pgfqpoint{4.763356in}{1.759343in}}%
\pgfpathlineto{\pgfqpoint{4.770959in}{1.771350in}}%
\pgfpathlineto{\pgfqpoint{4.778558in}{1.783356in}}%
\pgfpathlineto{\pgfqpoint{4.786152in}{1.795356in}}%
\pgfpathlineto{\pgfqpoint{4.793742in}{1.807350in}}%
\pgfpathlineto{\pgfqpoint{4.779977in}{1.805176in}}%
\pgfpathlineto{\pgfqpoint{4.766224in}{1.803115in}}%
\pgfpathlineto{\pgfqpoint{4.752481in}{1.801166in}}%
\pgfpathlineto{\pgfqpoint{4.738750in}{1.799331in}}%
\pgfpathlineto{\pgfqpoint{4.731158in}{1.787563in}}%
\pgfpathlineto{\pgfqpoint{4.723560in}{1.775792in}}%
\pgfpathlineto{\pgfqpoint{4.715959in}{1.764021in}}%
\pgfpathlineto{\pgfqpoint{4.708352in}{1.752251in}}%
\pgfpathclose%
\pgfusepath{fill}%
\end{pgfscope}%
\begin{pgfscope}%
\pgfpathrectangle{\pgfqpoint{1.254980in}{0.150000in}}{\pgfqpoint{5.490039in}{5.490039in}}%
\pgfusepath{clip}%
\pgfsetbuttcap%
\pgfsetroundjoin%
\definecolor{currentfill}{rgb}{0.276022,0.044167,0.370164}%
\pgfsetfillcolor{currentfill}%
\pgfsetfillopacity{0.700000}%
\pgfsetlinewidth{0.000000pt}%
\definecolor{currentstroke}{rgb}{0.000000,0.000000,0.000000}%
\pgfsetstrokecolor{currentstroke}%
\pgfsetdash{}{0pt}%
\pgfpathmoveto{\pgfqpoint{3.639302in}{1.580481in}}%
\pgfpathlineto{\pgfqpoint{3.652762in}{1.572274in}}%
\pgfpathlineto{\pgfqpoint{3.666224in}{1.564194in}}%
\pgfpathlineto{\pgfqpoint{3.679690in}{1.556240in}}%
\pgfpathlineto{\pgfqpoint{3.693158in}{1.548413in}}%
\pgfpathlineto{\pgfqpoint{3.701137in}{1.553144in}}%
\pgfpathlineto{\pgfqpoint{3.709107in}{1.558028in}}%
\pgfpathlineto{\pgfqpoint{3.717068in}{1.563062in}}%
\pgfpathlineto{\pgfqpoint{3.725022in}{1.568243in}}%
\pgfpathlineto{\pgfqpoint{3.711575in}{1.575678in}}%
\pgfpathlineto{\pgfqpoint{3.698131in}{1.583239in}}%
\pgfpathlineto{\pgfqpoint{3.684691in}{1.590926in}}%
\pgfpathlineto{\pgfqpoint{3.671254in}{1.598740in}}%
\pgfpathlineto{\pgfqpoint{3.663279in}{1.593945in}}%
\pgfpathlineto{\pgfqpoint{3.655295in}{1.589302in}}%
\pgfpathlineto{\pgfqpoint{3.647303in}{1.584812in}}%
\pgfpathlineto{\pgfqpoint{3.639302in}{1.580481in}}%
\pgfpathclose%
\pgfusepath{fill}%
\end{pgfscope}%
\begin{pgfscope}%
\pgfpathrectangle{\pgfqpoint{1.254980in}{0.150000in}}{\pgfqpoint{5.490039in}{5.490039in}}%
\pgfusepath{clip}%
\pgfsetbuttcap%
\pgfsetroundjoin%
\definecolor{currentfill}{rgb}{0.243113,0.292092,0.538516}%
\pgfsetfillcolor{currentfill}%
\pgfsetfillopacity{0.700000}%
\pgfsetlinewidth{0.000000pt}%
\definecolor{currentstroke}{rgb}{0.000000,0.000000,0.000000}%
\pgfsetstrokecolor{currentstroke}%
\pgfsetdash{}{0pt}%
\pgfpathmoveto{\pgfqpoint{5.080539in}{2.041282in}}%
\pgfpathlineto{\pgfqpoint{5.094456in}{2.045692in}}%
\pgfpathlineto{\pgfqpoint{5.108385in}{2.050214in}}%
\pgfpathlineto{\pgfqpoint{5.122328in}{2.054849in}}%
\pgfpathlineto{\pgfqpoint{5.136285in}{2.059596in}}%
\pgfpathlineto{\pgfqpoint{5.143785in}{2.071891in}}%
\pgfpathlineto{\pgfqpoint{5.151281in}{2.084140in}}%
\pgfpathlineto{\pgfqpoint{5.158771in}{2.096340in}}%
\pgfpathlineto{\pgfqpoint{5.166256in}{2.108491in}}%
\pgfpathlineto{\pgfqpoint{5.152301in}{2.103591in}}%
\pgfpathlineto{\pgfqpoint{5.138359in}{2.098803in}}%
\pgfpathlineto{\pgfqpoint{5.124431in}{2.094128in}}%
\pgfpathlineto{\pgfqpoint{5.110517in}{2.089566in}}%
\pgfpathlineto{\pgfqpoint{5.103030in}{2.077562in}}%
\pgfpathlineto{\pgfqpoint{5.095538in}{2.065512in}}%
\pgfpathlineto{\pgfqpoint{5.088041in}{2.053419in}}%
\pgfpathlineto{\pgfqpoint{5.080539in}{2.041282in}}%
\pgfpathclose%
\pgfusepath{fill}%
\end{pgfscope}%
\begin{pgfscope}%
\pgfpathrectangle{\pgfqpoint{1.254980in}{0.150000in}}{\pgfqpoint{5.490039in}{5.490039in}}%
\pgfusepath{clip}%
\pgfsetbuttcap%
\pgfsetroundjoin%
\definecolor{currentfill}{rgb}{0.273809,0.031497,0.358853}%
\pgfsetfillcolor{currentfill}%
\pgfsetfillopacity{0.700000}%
\pgfsetlinewidth{0.000000pt}%
\definecolor{currentstroke}{rgb}{0.000000,0.000000,0.000000}%
\pgfsetstrokecolor{currentstroke}%
\pgfsetdash{}{0pt}%
\pgfpathmoveto{\pgfqpoint{4.282147in}{1.539815in}}%
\pgfpathlineto{\pgfqpoint{4.295723in}{1.537743in}}%
\pgfpathlineto{\pgfqpoint{4.309308in}{1.535785in}}%
\pgfpathlineto{\pgfqpoint{4.322900in}{1.533943in}}%
\pgfpathlineto{\pgfqpoint{4.336501in}{1.532217in}}%
\pgfpathlineto{\pgfqpoint{4.344221in}{1.542276in}}%
\pgfpathlineto{\pgfqpoint{4.351937in}{1.552395in}}%
\pgfpathlineto{\pgfqpoint{4.359648in}{1.562571in}}%
\pgfpathlineto{\pgfqpoint{4.367354in}{1.572801in}}%
\pgfpathlineto{\pgfqpoint{4.353761in}{1.574218in}}%
\pgfpathlineto{\pgfqpoint{4.340177in}{1.575750in}}%
\pgfpathlineto{\pgfqpoint{4.326601in}{1.577397in}}%
\pgfpathlineto{\pgfqpoint{4.313033in}{1.579160in}}%
\pgfpathlineto{\pgfqpoint{4.305319in}{1.569234in}}%
\pgfpathlineto{\pgfqpoint{4.297600in}{1.559366in}}%
\pgfpathlineto{\pgfqpoint{4.289876in}{1.549559in}}%
\pgfpathlineto{\pgfqpoint{4.282147in}{1.539815in}}%
\pgfpathclose%
\pgfusepath{fill}%
\end{pgfscope}%
\begin{pgfscope}%
\pgfpathrectangle{\pgfqpoint{1.254980in}{0.150000in}}{\pgfqpoint{5.490039in}{5.490039in}}%
\pgfusepath{clip}%
\pgfsetbuttcap%
\pgfsetroundjoin%
\definecolor{currentfill}{rgb}{0.235526,0.309527,0.542944}%
\pgfsetfillcolor{currentfill}%
\pgfsetfillopacity{0.700000}%
\pgfsetlinewidth{0.000000pt}%
\definecolor{currentstroke}{rgb}{0.000000,0.000000,0.000000}%
\pgfsetstrokecolor{currentstroke}%
\pgfsetdash{}{0pt}%
\pgfpathmoveto{\pgfqpoint{2.927060in}{2.146064in}}%
\pgfpathlineto{\pgfqpoint{2.940595in}{2.130256in}}%
\pgfpathlineto{\pgfqpoint{2.954126in}{2.114604in}}%
\pgfpathlineto{\pgfqpoint{2.967654in}{2.099109in}}%
\pgfpathlineto{\pgfqpoint{2.981178in}{2.083769in}}%
\pgfpathlineto{\pgfqpoint{2.989613in}{2.081854in}}%
\pgfpathlineto{\pgfqpoint{2.998032in}{2.080174in}}%
\pgfpathlineto{\pgfqpoint{3.006437in}{2.078727in}}%
\pgfpathlineto{\pgfqpoint{3.014828in}{2.077508in}}%
\pgfpathlineto{\pgfqpoint{3.001342in}{2.092402in}}%
\pgfpathlineto{\pgfqpoint{2.987854in}{2.107451in}}%
\pgfpathlineto{\pgfqpoint{2.974363in}{2.122655in}}%
\pgfpathlineto{\pgfqpoint{2.960868in}{2.138016in}}%
\pgfpathlineto{\pgfqpoint{2.952439in}{2.139674in}}%
\pgfpathlineto{\pgfqpoint{2.943994in}{2.141566in}}%
\pgfpathlineto{\pgfqpoint{2.935535in}{2.143695in}}%
\pgfpathlineto{\pgfqpoint{2.927060in}{2.146064in}}%
\pgfpathclose%
\pgfusepath{fill}%
\end{pgfscope}%
\begin{pgfscope}%
\pgfpathrectangle{\pgfqpoint{1.254980in}{0.150000in}}{\pgfqpoint{5.490039in}{5.490039in}}%
\pgfusepath{clip}%
\pgfsetbuttcap%
\pgfsetroundjoin%
\definecolor{currentfill}{rgb}{0.223925,0.334994,0.548053}%
\pgfsetfillcolor{currentfill}%
\pgfsetfillopacity{0.700000}%
\pgfsetlinewidth{0.000000pt}%
\definecolor{currentstroke}{rgb}{0.000000,0.000000,0.000000}%
\pgfsetstrokecolor{currentstroke}%
\pgfsetdash{}{0pt}%
\pgfpathmoveto{\pgfqpoint{2.872883in}{2.210885in}}%
\pgfpathlineto{\pgfqpoint{2.886433in}{2.194440in}}%
\pgfpathlineto{\pgfqpoint{2.899980in}{2.178156in}}%
\pgfpathlineto{\pgfqpoint{2.913522in}{2.162031in}}%
\pgfpathlineto{\pgfqpoint{2.927060in}{2.146064in}}%
\pgfpathlineto{\pgfqpoint{2.935535in}{2.143695in}}%
\pgfpathlineto{\pgfqpoint{2.943994in}{2.141566in}}%
\pgfpathlineto{\pgfqpoint{2.952439in}{2.139674in}}%
\pgfpathlineto{\pgfqpoint{2.960868in}{2.138016in}}%
\pgfpathlineto{\pgfqpoint{2.947370in}{2.153534in}}%
\pgfpathlineto{\pgfqpoint{2.933869in}{2.169209in}}%
\pgfpathlineto{\pgfqpoint{2.920364in}{2.185044in}}%
\pgfpathlineto{\pgfqpoint{2.906855in}{2.201039in}}%
\pgfpathlineto{\pgfqpoint{2.898385in}{2.203140in}}%
\pgfpathlineto{\pgfqpoint{2.889900in}{2.205478in}}%
\pgfpathlineto{\pgfqpoint{2.881399in}{2.208059in}}%
\pgfpathlineto{\pgfqpoint{2.872883in}{2.210885in}}%
\pgfpathclose%
\pgfusepath{fill}%
\end{pgfscope}%
\begin{pgfscope}%
\pgfpathrectangle{\pgfqpoint{1.254980in}{0.150000in}}{\pgfqpoint{5.490039in}{5.490039in}}%
\pgfusepath{clip}%
\pgfsetbuttcap%
\pgfsetroundjoin%
\definecolor{currentfill}{rgb}{0.246811,0.283237,0.535941}%
\pgfsetfillcolor{currentfill}%
\pgfsetfillopacity{0.700000}%
\pgfsetlinewidth{0.000000pt}%
\definecolor{currentstroke}{rgb}{0.000000,0.000000,0.000000}%
\pgfsetstrokecolor{currentstroke}%
\pgfsetdash{}{0pt}%
\pgfpathmoveto{\pgfqpoint{2.981178in}{2.083769in}}%
\pgfpathlineto{\pgfqpoint{2.994700in}{2.068584in}}%
\pgfpathlineto{\pgfqpoint{3.008218in}{2.053552in}}%
\pgfpathlineto{\pgfqpoint{3.021733in}{2.038673in}}%
\pgfpathlineto{\pgfqpoint{3.035246in}{2.023947in}}%
\pgfpathlineto{\pgfqpoint{3.043641in}{2.022483in}}%
\pgfpathlineto{\pgfqpoint{3.052022in}{2.021251in}}%
\pgfpathlineto{\pgfqpoint{3.060389in}{2.020245in}}%
\pgfpathlineto{\pgfqpoint{3.068742in}{2.019463in}}%
\pgfpathlineto{\pgfqpoint{3.055267in}{2.033746in}}%
\pgfpathlineto{\pgfqpoint{3.041790in}{2.048181in}}%
\pgfpathlineto{\pgfqpoint{3.028310in}{2.062768in}}%
\pgfpathlineto{\pgfqpoint{3.014828in}{2.077508in}}%
\pgfpathlineto{\pgfqpoint{3.006437in}{2.078727in}}%
\pgfpathlineto{\pgfqpoint{2.998032in}{2.080174in}}%
\pgfpathlineto{\pgfqpoint{2.989613in}{2.081854in}}%
\pgfpathlineto{\pgfqpoint{2.981178in}{2.083769in}}%
\pgfpathclose%
\pgfusepath{fill}%
\end{pgfscope}%
\begin{pgfscope}%
\pgfpathrectangle{\pgfqpoint{1.254980in}{0.150000in}}{\pgfqpoint{5.490039in}{5.490039in}}%
\pgfusepath{clip}%
\pgfsetbuttcap%
\pgfsetroundjoin%
\definecolor{currentfill}{rgb}{0.210503,0.363727,0.552206}%
\pgfsetfillcolor{currentfill}%
\pgfsetfillopacity{0.700000}%
\pgfsetlinewidth{0.000000pt}%
\definecolor{currentstroke}{rgb}{0.000000,0.000000,0.000000}%
\pgfsetstrokecolor{currentstroke}%
\pgfsetdash{}{0pt}%
\pgfpathmoveto{\pgfqpoint{2.818637in}{2.278287in}}%
\pgfpathlineto{\pgfqpoint{2.832205in}{2.261191in}}%
\pgfpathlineto{\pgfqpoint{2.845769in}{2.244260in}}%
\pgfpathlineto{\pgfqpoint{2.859328in}{2.227491in}}%
\pgfpathlineto{\pgfqpoint{2.872883in}{2.210885in}}%
\pgfpathlineto{\pgfqpoint{2.881399in}{2.208059in}}%
\pgfpathlineto{\pgfqpoint{2.889900in}{2.205478in}}%
\pgfpathlineto{\pgfqpoint{2.898385in}{2.203140in}}%
\pgfpathlineto{\pgfqpoint{2.906855in}{2.201039in}}%
\pgfpathlineto{\pgfqpoint{2.893342in}{2.217194in}}%
\pgfpathlineto{\pgfqpoint{2.879825in}{2.233510in}}%
\pgfpathlineto{\pgfqpoint{2.866304in}{2.249989in}}%
\pgfpathlineto{\pgfqpoint{2.852778in}{2.266632in}}%
\pgfpathlineto{\pgfqpoint{2.844267in}{2.269177in}}%
\pgfpathlineto{\pgfqpoint{2.835739in}{2.271966in}}%
\pgfpathlineto{\pgfqpoint{2.827196in}{2.275001in}}%
\pgfpathlineto{\pgfqpoint{2.818637in}{2.278287in}}%
\pgfpathclose%
\pgfusepath{fill}%
\end{pgfscope}%
\begin{pgfscope}%
\pgfpathrectangle{\pgfqpoint{1.254980in}{0.150000in}}{\pgfqpoint{5.490039in}{5.490039in}}%
\pgfusepath{clip}%
\pgfsetbuttcap%
\pgfsetroundjoin%
\definecolor{currentfill}{rgb}{0.278012,0.180367,0.486697}%
\pgfsetfillcolor{currentfill}%
\pgfsetfillopacity{0.700000}%
\pgfsetlinewidth{0.000000pt}%
\definecolor{currentstroke}{rgb}{0.000000,0.000000,0.000000}%
\pgfsetstrokecolor{currentstroke}%
\pgfsetdash{}{0pt}%
\pgfpathmoveto{\pgfqpoint{4.793742in}{1.807350in}}%
\pgfpathlineto{\pgfqpoint{4.807518in}{1.809638in}}%
\pgfpathlineto{\pgfqpoint{4.821305in}{1.812038in}}%
\pgfpathlineto{\pgfqpoint{4.835104in}{1.814550in}}%
\pgfpathlineto{\pgfqpoint{4.848915in}{1.817175in}}%
\pgfpathlineto{\pgfqpoint{4.856498in}{1.829384in}}%
\pgfpathlineto{\pgfqpoint{4.864076in}{1.841580in}}%
\pgfpathlineto{\pgfqpoint{4.871649in}{1.853760in}}%
\pgfpathlineto{\pgfqpoint{4.879218in}{1.865924in}}%
\pgfpathlineto{\pgfqpoint{4.865410in}{1.863082in}}%
\pgfpathlineto{\pgfqpoint{4.851613in}{1.860352in}}%
\pgfpathlineto{\pgfqpoint{4.837828in}{1.857736in}}%
\pgfpathlineto{\pgfqpoint{4.824055in}{1.855233in}}%
\pgfpathlineto{\pgfqpoint{4.816483in}{1.843279in}}%
\pgfpathlineto{\pgfqpoint{4.808907in}{1.831314in}}%
\pgfpathlineto{\pgfqpoint{4.801327in}{1.819337in}}%
\pgfpathlineto{\pgfqpoint{4.793742in}{1.807350in}}%
\pgfpathclose%
\pgfusepath{fill}%
\end{pgfscope}%
\begin{pgfscope}%
\pgfpathrectangle{\pgfqpoint{1.254980in}{0.150000in}}{\pgfqpoint{5.490039in}{5.490039in}}%
\pgfusepath{clip}%
\pgfsetbuttcap%
\pgfsetroundjoin%
\definecolor{currentfill}{rgb}{0.282327,0.094955,0.417331}%
\pgfsetfillcolor{currentfill}%
\pgfsetfillopacity{0.700000}%
\pgfsetlinewidth{0.000000pt}%
\definecolor{currentstroke}{rgb}{0.000000,0.000000,0.000000}%
\pgfsetstrokecolor{currentstroke}%
\pgfsetdash{}{0pt}%
\pgfpathmoveto{\pgfqpoint{3.445579in}{1.678280in}}%
\pgfpathlineto{\pgfqpoint{3.459039in}{1.668102in}}%
\pgfpathlineto{\pgfqpoint{3.472500in}{1.658057in}}%
\pgfpathlineto{\pgfqpoint{3.485963in}{1.648144in}}%
\pgfpathlineto{\pgfqpoint{3.499428in}{1.638363in}}%
\pgfpathlineto{\pgfqpoint{3.507515in}{1.641195in}}%
\pgfpathlineto{\pgfqpoint{3.515593in}{1.644207in}}%
\pgfpathlineto{\pgfqpoint{3.523661in}{1.647396in}}%
\pgfpathlineto{\pgfqpoint{3.531719in}{1.650758in}}%
\pgfpathlineto{\pgfqpoint{3.518281in}{1.660127in}}%
\pgfpathlineto{\pgfqpoint{3.504845in}{1.669628in}}%
\pgfpathlineto{\pgfqpoint{3.491410in}{1.679260in}}%
\pgfpathlineto{\pgfqpoint{3.477977in}{1.689026in}}%
\pgfpathlineto{\pgfqpoint{3.469893in}{1.686070in}}%
\pgfpathlineto{\pgfqpoint{3.461798in}{1.683291in}}%
\pgfpathlineto{\pgfqpoint{3.453694in}{1.680694in}}%
\pgfpathlineto{\pgfqpoint{3.445579in}{1.678280in}}%
\pgfpathclose%
\pgfusepath{fill}%
\end{pgfscope}%
\begin{pgfscope}%
\pgfpathrectangle{\pgfqpoint{1.254980in}{0.150000in}}{\pgfqpoint{5.490039in}{5.490039in}}%
\pgfusepath{clip}%
\pgfsetbuttcap%
\pgfsetroundjoin%
\definecolor{currentfill}{rgb}{0.257322,0.256130,0.526563}%
\pgfsetfillcolor{currentfill}%
\pgfsetfillopacity{0.700000}%
\pgfsetlinewidth{0.000000pt}%
\definecolor{currentstroke}{rgb}{0.000000,0.000000,0.000000}%
\pgfsetstrokecolor{currentstroke}%
\pgfsetdash{}{0pt}%
\pgfpathmoveto{\pgfqpoint{3.035246in}{2.023947in}}%
\pgfpathlineto{\pgfqpoint{3.048756in}{2.009372in}}%
\pgfpathlineto{\pgfqpoint{3.062263in}{1.994947in}}%
\pgfpathlineto{\pgfqpoint{3.075769in}{1.980673in}}%
\pgfpathlineto{\pgfqpoint{3.089272in}{1.966548in}}%
\pgfpathlineto{\pgfqpoint{3.097629in}{1.965533in}}%
\pgfpathlineto{\pgfqpoint{3.105972in}{1.964745in}}%
\pgfpathlineto{\pgfqpoint{3.114303in}{1.964179in}}%
\pgfpathlineto{\pgfqpoint{3.122620in}{1.963833in}}%
\pgfpathlineto{\pgfqpoint{3.109153in}{1.977517in}}%
\pgfpathlineto{\pgfqpoint{3.095685in}{1.991349in}}%
\pgfpathlineto{\pgfqpoint{3.082214in}{2.005331in}}%
\pgfpathlineto{\pgfqpoint{3.068742in}{2.019463in}}%
\pgfpathlineto{\pgfqpoint{3.060389in}{2.020245in}}%
\pgfpathlineto{\pgfqpoint{3.052022in}{2.021251in}}%
\pgfpathlineto{\pgfqpoint{3.043641in}{2.022483in}}%
\pgfpathlineto{\pgfqpoint{3.035246in}{2.023947in}}%
\pgfpathclose%
\pgfusepath{fill}%
\end{pgfscope}%
\begin{pgfscope}%
\pgfpathrectangle{\pgfqpoint{1.254980in}{0.150000in}}{\pgfqpoint{5.490039in}{5.490039in}}%
\pgfusepath{clip}%
\pgfsetbuttcap%
\pgfsetroundjoin%
\definecolor{currentfill}{rgb}{0.197636,0.391528,0.554969}%
\pgfsetfillcolor{currentfill}%
\pgfsetfillopacity{0.700000}%
\pgfsetlinewidth{0.000000pt}%
\definecolor{currentstroke}{rgb}{0.000000,0.000000,0.000000}%
\pgfsetstrokecolor{currentstroke}%
\pgfsetdash{}{0pt}%
\pgfpathmoveto{\pgfqpoint{2.764312in}{2.348327in}}%
\pgfpathlineto{\pgfqpoint{2.777901in}{2.330566in}}%
\pgfpathlineto{\pgfqpoint{2.791485in}{2.312973in}}%
\pgfpathlineto{\pgfqpoint{2.805063in}{2.295547in}}%
\pgfpathlineto{\pgfqpoint{2.818637in}{2.278287in}}%
\pgfpathlineto{\pgfqpoint{2.827196in}{2.275001in}}%
\pgfpathlineto{\pgfqpoint{2.835739in}{2.271966in}}%
\pgfpathlineto{\pgfqpoint{2.844267in}{2.269177in}}%
\pgfpathlineto{\pgfqpoint{2.852778in}{2.266632in}}%
\pgfpathlineto{\pgfqpoint{2.839248in}{2.283438in}}%
\pgfpathlineto{\pgfqpoint{2.825713in}{2.300410in}}%
\pgfpathlineto{\pgfqpoint{2.812173in}{2.317547in}}%
\pgfpathlineto{\pgfqpoint{2.798628in}{2.334852in}}%
\pgfpathlineto{\pgfqpoint{2.790074in}{2.337845in}}%
\pgfpathlineto{\pgfqpoint{2.781503in}{2.341086in}}%
\pgfpathlineto{\pgfqpoint{2.772916in}{2.344579in}}%
\pgfpathlineto{\pgfqpoint{2.764312in}{2.348327in}}%
\pgfpathclose%
\pgfusepath{fill}%
\end{pgfscope}%
\begin{pgfscope}%
\pgfpathrectangle{\pgfqpoint{1.254980in}{0.150000in}}{\pgfqpoint{5.490039in}{5.490039in}}%
\pgfusepath{clip}%
\pgfsetbuttcap%
\pgfsetroundjoin%
\definecolor{currentfill}{rgb}{0.194100,0.399323,0.555565}%
\pgfsetfillcolor{currentfill}%
\pgfsetfillopacity{0.700000}%
\pgfsetlinewidth{0.000000pt}%
\definecolor{currentstroke}{rgb}{0.000000,0.000000,0.000000}%
\pgfsetstrokecolor{currentstroke}%
\pgfsetdash{}{0pt}%
\pgfpathmoveto{\pgfqpoint{5.367749in}{2.296284in}}%
\pgfpathlineto{\pgfqpoint{5.381827in}{2.302559in}}%
\pgfpathlineto{\pgfqpoint{5.395921in}{2.308946in}}%
\pgfpathlineto{\pgfqpoint{5.410029in}{2.315446in}}%
\pgfpathlineto{\pgfqpoint{5.424153in}{2.322059in}}%
\pgfpathlineto{\pgfqpoint{5.431558in}{2.333775in}}%
\pgfpathlineto{\pgfqpoint{5.438957in}{2.345419in}}%
\pgfpathlineto{\pgfqpoint{5.446350in}{2.356990in}}%
\pgfpathlineto{\pgfqpoint{5.453737in}{2.368488in}}%
\pgfpathlineto{\pgfqpoint{5.439616in}{2.361788in}}%
\pgfpathlineto{\pgfqpoint{5.425509in}{2.355200in}}%
\pgfpathlineto{\pgfqpoint{5.411418in}{2.348725in}}%
\pgfpathlineto{\pgfqpoint{5.397343in}{2.342363in}}%
\pgfpathlineto{\pgfqpoint{5.389953in}{2.330946in}}%
\pgfpathlineto{\pgfqpoint{5.382558in}{2.319461in}}%
\pgfpathlineto{\pgfqpoint{5.375157in}{2.307906in}}%
\pgfpathlineto{\pgfqpoint{5.367749in}{2.296284in}}%
\pgfpathclose%
\pgfusepath{fill}%
\end{pgfscope}%
\begin{pgfscope}%
\pgfpathrectangle{\pgfqpoint{1.254980in}{0.150000in}}{\pgfqpoint{5.490039in}{5.490039in}}%
\pgfusepath{clip}%
\pgfsetbuttcap%
\pgfsetroundjoin%
\definecolor{currentfill}{rgb}{0.265145,0.232956,0.516599}%
\pgfsetfillcolor{currentfill}%
\pgfsetfillopacity{0.700000}%
\pgfsetlinewidth{0.000000pt}%
\definecolor{currentstroke}{rgb}{0.000000,0.000000,0.000000}%
\pgfsetstrokecolor{currentstroke}%
\pgfsetdash{}{0pt}%
\pgfpathmoveto{\pgfqpoint{3.089272in}{1.966548in}}%
\pgfpathlineto{\pgfqpoint{3.102773in}{1.952571in}}%
\pgfpathlineto{\pgfqpoint{3.116272in}{1.938742in}}%
\pgfpathlineto{\pgfqpoint{3.129769in}{1.925060in}}%
\pgfpathlineto{\pgfqpoint{3.143264in}{1.911524in}}%
\pgfpathlineto{\pgfqpoint{3.151585in}{1.910957in}}%
\pgfpathlineto{\pgfqpoint{3.159893in}{1.910611in}}%
\pgfpathlineto{\pgfqpoint{3.168187in}{1.910483in}}%
\pgfpathlineto{\pgfqpoint{3.176469in}{1.910569in}}%
\pgfpathlineto{\pgfqpoint{3.163009in}{1.923666in}}%
\pgfpathlineto{\pgfqpoint{3.149547in}{1.936908in}}%
\pgfpathlineto{\pgfqpoint{3.136084in}{1.950297in}}%
\pgfpathlineto{\pgfqpoint{3.122620in}{1.963833in}}%
\pgfpathlineto{\pgfqpoint{3.114303in}{1.964179in}}%
\pgfpathlineto{\pgfqpoint{3.105972in}{1.964745in}}%
\pgfpathlineto{\pgfqpoint{3.097629in}{1.965533in}}%
\pgfpathlineto{\pgfqpoint{3.089272in}{1.966548in}}%
\pgfpathclose%
\pgfusepath{fill}%
\end{pgfscope}%
\begin{pgfscope}%
\pgfpathrectangle{\pgfqpoint{1.254980in}{0.150000in}}{\pgfqpoint{5.490039in}{5.490039in}}%
\pgfusepath{clip}%
\pgfsetbuttcap%
\pgfsetroundjoin%
\definecolor{currentfill}{rgb}{0.271305,0.019942,0.347269}%
\pgfsetfillcolor{currentfill}%
\pgfsetfillopacity{0.700000}%
\pgfsetlinewidth{0.000000pt}%
\definecolor{currentstroke}{rgb}{0.000000,0.000000,0.000000}%
\pgfsetstrokecolor{currentstroke}%
\pgfsetdash{}{0pt}%
\pgfpathmoveto{\pgfqpoint{4.196909in}{1.512276in}}%
\pgfpathlineto{\pgfqpoint{4.210466in}{1.509414in}}%
\pgfpathlineto{\pgfqpoint{4.224029in}{1.506668in}}%
\pgfpathlineto{\pgfqpoint{4.237600in}{1.504038in}}%
\pgfpathlineto{\pgfqpoint{4.251179in}{1.501524in}}%
\pgfpathlineto{\pgfqpoint{4.258928in}{1.510989in}}%
\pgfpathlineto{\pgfqpoint{4.266673in}{1.520528in}}%
\pgfpathlineto{\pgfqpoint{4.274412in}{1.530137in}}%
\pgfpathlineto{\pgfqpoint{4.282147in}{1.539815in}}%
\pgfpathlineto{\pgfqpoint{4.268578in}{1.542004in}}%
\pgfpathlineto{\pgfqpoint{4.255017in}{1.544308in}}%
\pgfpathlineto{\pgfqpoint{4.241463in}{1.546729in}}%
\pgfpathlineto{\pgfqpoint{4.227917in}{1.549266in}}%
\pgfpathlineto{\pgfqpoint{4.220173in}{1.539907in}}%
\pgfpathlineto{\pgfqpoint{4.212424in}{1.530621in}}%
\pgfpathlineto{\pgfqpoint{4.204669in}{1.521410in}}%
\pgfpathlineto{\pgfqpoint{4.196909in}{1.512276in}}%
\pgfpathclose%
\pgfusepath{fill}%
\end{pgfscope}%
\begin{pgfscope}%
\pgfpathrectangle{\pgfqpoint{1.254980in}{0.150000in}}{\pgfqpoint{5.490039in}{5.490039in}}%
\pgfusepath{clip}%
\pgfsetbuttcap%
\pgfsetroundjoin%
\definecolor{currentfill}{rgb}{0.268510,0.009605,0.335427}%
\pgfsetfillcolor{currentfill}%
\pgfsetfillopacity{0.700000}%
\pgfsetlinewidth{0.000000pt}%
\definecolor{currentstroke}{rgb}{0.000000,0.000000,0.000000}%
\pgfsetstrokecolor{currentstroke}%
\pgfsetdash{}{0pt}%
\pgfpathmoveto{\pgfqpoint{3.832735in}{1.513249in}}%
\pgfpathlineto{\pgfqpoint{3.846217in}{1.506930in}}%
\pgfpathlineto{\pgfqpoint{3.859704in}{1.500733in}}%
\pgfpathlineto{\pgfqpoint{3.873196in}{1.494658in}}%
\pgfpathlineto{\pgfqpoint{3.886693in}{1.488704in}}%
\pgfpathlineto{\pgfqpoint{3.894581in}{1.495172in}}%
\pgfpathlineto{\pgfqpoint{3.902463in}{1.501767in}}%
\pgfpathlineto{\pgfqpoint{3.910338in}{1.508488in}}%
\pgfpathlineto{\pgfqpoint{3.918206in}{1.515330in}}%
\pgfpathlineto{\pgfqpoint{3.904727in}{1.520909in}}%
\pgfpathlineto{\pgfqpoint{3.891253in}{1.526610in}}%
\pgfpathlineto{\pgfqpoint{3.877783in}{1.532433in}}%
\pgfpathlineto{\pgfqpoint{3.864318in}{1.538377in}}%
\pgfpathlineto{\pgfqpoint{3.856433in}{1.531903in}}%
\pgfpathlineto{\pgfqpoint{3.848541in}{1.525555in}}%
\pgfpathlineto{\pgfqpoint{3.840641in}{1.519336in}}%
\pgfpathlineto{\pgfqpoint{3.832735in}{1.513249in}}%
\pgfpathclose%
\pgfusepath{fill}%
\end{pgfscope}%
\begin{pgfscope}%
\pgfpathrectangle{\pgfqpoint{1.254980in}{0.150000in}}{\pgfqpoint{5.490039in}{5.490039in}}%
\pgfusepath{clip}%
\pgfsetbuttcap%
\pgfsetroundjoin%
\definecolor{currentfill}{rgb}{0.185556,0.418570,0.556753}%
\pgfsetfillcolor{currentfill}%
\pgfsetfillopacity{0.700000}%
\pgfsetlinewidth{0.000000pt}%
\definecolor{currentstroke}{rgb}{0.000000,0.000000,0.000000}%
\pgfsetstrokecolor{currentstroke}%
\pgfsetdash{}{0pt}%
\pgfpathmoveto{\pgfqpoint{2.709900in}{2.421068in}}%
\pgfpathlineto{\pgfqpoint{2.723512in}{2.402626in}}%
\pgfpathlineto{\pgfqpoint{2.737118in}{2.384356in}}%
\pgfpathlineto{\pgfqpoint{2.750718in}{2.366257in}}%
\pgfpathlineto{\pgfqpoint{2.764312in}{2.348327in}}%
\pgfpathlineto{\pgfqpoint{2.772916in}{2.344579in}}%
\pgfpathlineto{\pgfqpoint{2.781503in}{2.341086in}}%
\pgfpathlineto{\pgfqpoint{2.790074in}{2.337845in}}%
\pgfpathlineto{\pgfqpoint{2.798628in}{2.334852in}}%
\pgfpathlineto{\pgfqpoint{2.785078in}{2.352325in}}%
\pgfpathlineto{\pgfqpoint{2.771523in}{2.369966in}}%
\pgfpathlineto{\pgfqpoint{2.757962in}{2.387778in}}%
\pgfpathlineto{\pgfqpoint{2.744396in}{2.405760in}}%
\pgfpathlineto{\pgfqpoint{2.735798in}{2.409204in}}%
\pgfpathlineto{\pgfqpoint{2.727182in}{2.412901in}}%
\pgfpathlineto{\pgfqpoint{2.718550in}{2.416854in}}%
\pgfpathlineto{\pgfqpoint{2.709900in}{2.421068in}}%
\pgfpathclose%
\pgfusepath{fill}%
\end{pgfscope}%
\begin{pgfscope}%
\pgfpathrectangle{\pgfqpoint{1.254980in}{0.150000in}}{\pgfqpoint{5.490039in}{5.490039in}}%
\pgfusepath{clip}%
\pgfsetbuttcap%
\pgfsetroundjoin%
\definecolor{currentfill}{rgb}{0.267004,0.004874,0.329415}%
\pgfsetfillcolor{currentfill}%
\pgfsetfillopacity{0.700000}%
\pgfsetlinewidth{0.000000pt}%
\definecolor{currentstroke}{rgb}{0.000000,0.000000,0.000000}%
\pgfsetstrokecolor{currentstroke}%
\pgfsetdash{}{0pt}%
\pgfpathmoveto{\pgfqpoint{3.972175in}{1.494220in}}%
\pgfpathlineto{\pgfqpoint{3.985681in}{1.489243in}}%
\pgfpathlineto{\pgfqpoint{3.999192in}{1.484385in}}%
\pgfpathlineto{\pgfqpoint{4.012709in}{1.479647in}}%
\pgfpathlineto{\pgfqpoint{4.026232in}{1.475028in}}%
\pgfpathlineto{\pgfqpoint{4.034062in}{1.482716in}}%
\pgfpathlineto{\pgfqpoint{4.041887in}{1.490513in}}%
\pgfpathlineto{\pgfqpoint{4.049705in}{1.498414in}}%
\pgfpathlineto{\pgfqpoint{4.057518in}{1.506417in}}%
\pgfpathlineto{\pgfqpoint{4.044009in}{1.510679in}}%
\pgfpathlineto{\pgfqpoint{4.030507in}{1.515060in}}%
\pgfpathlineto{\pgfqpoint{4.017010in}{1.519560in}}%
\pgfpathlineto{\pgfqpoint{4.003519in}{1.524180in}}%
\pgfpathlineto{\pgfqpoint{3.995693in}{1.516528in}}%
\pgfpathlineto{\pgfqpoint{3.987860in}{1.508982in}}%
\pgfpathlineto{\pgfqpoint{3.980021in}{1.501545in}}%
\pgfpathlineto{\pgfqpoint{3.972175in}{1.494220in}}%
\pgfpathclose%
\pgfusepath{fill}%
\end{pgfscope}%
\begin{pgfscope}%
\pgfpathrectangle{\pgfqpoint{1.254980in}{0.150000in}}{\pgfqpoint{5.490039in}{5.490039in}}%
\pgfusepath{clip}%
\pgfsetbuttcap%
\pgfsetroundjoin%
\definecolor{currentfill}{rgb}{0.120081,0.622161,0.534946}%
\pgfsetfillcolor{currentfill}%
\pgfsetfillopacity{0.700000}%
\pgfsetlinewidth{0.000000pt}%
\definecolor{currentstroke}{rgb}{0.000000,0.000000,0.000000}%
\pgfsetstrokecolor{currentstroke}%
\pgfsetdash{}{0pt}%
\pgfpathmoveto{\pgfqpoint{2.361583in}{2.983179in}}%
\pgfpathlineto{\pgfqpoint{2.375376in}{2.959946in}}%
\pgfpathlineto{\pgfqpoint{2.389158in}{2.936918in}}%
\pgfpathlineto{\pgfqpoint{2.402929in}{2.914093in}}%
\pgfpathlineto{\pgfqpoint{2.416691in}{2.891471in}}%
\pgfpathlineto{\pgfqpoint{2.425565in}{2.885530in}}%
\pgfpathlineto{\pgfqpoint{2.434419in}{2.879865in}}%
\pgfpathlineto{\pgfqpoint{2.443254in}{2.874473in}}%
\pgfpathlineto{\pgfqpoint{2.452069in}{2.869349in}}%
\pgfpathlineto{\pgfqpoint{2.438360in}{2.891509in}}%
\pgfpathlineto{\pgfqpoint{2.424641in}{2.913871in}}%
\pgfpathlineto{\pgfqpoint{2.410912in}{2.936435in}}%
\pgfpathlineto{\pgfqpoint{2.397173in}{2.959203in}}%
\pgfpathlineto{\pgfqpoint{2.388306in}{2.964782in}}%
\pgfpathlineto{\pgfqpoint{2.379419in}{2.970636in}}%
\pgfpathlineto{\pgfqpoint{2.370511in}{2.976766in}}%
\pgfpathlineto{\pgfqpoint{2.361583in}{2.983179in}}%
\pgfpathclose%
\pgfusepath{fill}%
\end{pgfscope}%
\begin{pgfscope}%
\pgfpathrectangle{\pgfqpoint{1.254980in}{0.150000in}}{\pgfqpoint{5.490039in}{5.490039in}}%
\pgfusepath{clip}%
\pgfsetbuttcap%
\pgfsetroundjoin%
\definecolor{currentfill}{rgb}{0.270595,0.214069,0.507052}%
\pgfsetfillcolor{currentfill}%
\pgfsetfillopacity{0.700000}%
\pgfsetlinewidth{0.000000pt}%
\definecolor{currentstroke}{rgb}{0.000000,0.000000,0.000000}%
\pgfsetstrokecolor{currentstroke}%
\pgfsetdash{}{0pt}%
\pgfpathmoveto{\pgfqpoint{4.879218in}{1.865924in}}%
\pgfpathlineto{\pgfqpoint{4.893039in}{1.868878in}}%
\pgfpathlineto{\pgfqpoint{4.906871in}{1.871945in}}%
\pgfpathlineto{\pgfqpoint{4.920716in}{1.875124in}}%
\pgfpathlineto{\pgfqpoint{4.934573in}{1.878416in}}%
\pgfpathlineto{\pgfqpoint{4.942135in}{1.890769in}}%
\pgfpathlineto{\pgfqpoint{4.949693in}{1.903097in}}%
\pgfpathlineto{\pgfqpoint{4.957246in}{1.915401in}}%
\pgfpathlineto{\pgfqpoint{4.964794in}{1.927677in}}%
\pgfpathlineto{\pgfqpoint{4.950939in}{1.924184in}}%
\pgfpathlineto{\pgfqpoint{4.937096in}{1.920804in}}%
\pgfpathlineto{\pgfqpoint{4.923266in}{1.917536in}}%
\pgfpathlineto{\pgfqpoint{4.909447in}{1.914380in}}%
\pgfpathlineto{\pgfqpoint{4.901897in}{1.902299in}}%
\pgfpathlineto{\pgfqpoint{4.894342in}{1.890194in}}%
\pgfpathlineto{\pgfqpoint{4.886782in}{1.878069in}}%
\pgfpathlineto{\pgfqpoint{4.879218in}{1.865924in}}%
\pgfpathclose%
\pgfusepath{fill}%
\end{pgfscope}%
\begin{pgfscope}%
\pgfpathrectangle{\pgfqpoint{1.254980in}{0.150000in}}{\pgfqpoint{5.490039in}{5.490039in}}%
\pgfusepath{clip}%
\pgfsetbuttcap%
\pgfsetroundjoin%
\definecolor{currentfill}{rgb}{0.227802,0.326594,0.546532}%
\pgfsetfillcolor{currentfill}%
\pgfsetfillopacity{0.700000}%
\pgfsetlinewidth{0.000000pt}%
\definecolor{currentstroke}{rgb}{0.000000,0.000000,0.000000}%
\pgfsetstrokecolor{currentstroke}%
\pgfsetdash{}{0pt}%
\pgfpathmoveto{\pgfqpoint{5.166256in}{2.108491in}}%
\pgfpathlineto{\pgfqpoint{5.180225in}{2.113504in}}%
\pgfpathlineto{\pgfqpoint{5.194207in}{2.118629in}}%
\pgfpathlineto{\pgfqpoint{5.208204in}{2.123867in}}%
\pgfpathlineto{\pgfqpoint{5.222214in}{2.129217in}}%
\pgfpathlineto{\pgfqpoint{5.229693in}{2.141461in}}%
\pgfpathlineto{\pgfqpoint{5.237165in}{2.153649in}}%
\pgfpathlineto{\pgfqpoint{5.244633in}{2.165781in}}%
\pgfpathlineto{\pgfqpoint{5.252094in}{2.177857in}}%
\pgfpathlineto{\pgfqpoint{5.238085in}{2.172370in}}%
\pgfpathlineto{\pgfqpoint{5.224090in}{2.166995in}}%
\pgfpathlineto{\pgfqpoint{5.210109in}{2.161733in}}%
\pgfpathlineto{\pgfqpoint{5.196142in}{2.156584in}}%
\pgfpathlineto{\pgfqpoint{5.188678in}{2.144639in}}%
\pgfpathlineto{\pgfqpoint{5.181210in}{2.132642in}}%
\pgfpathlineto{\pgfqpoint{5.173735in}{2.120592in}}%
\pgfpathlineto{\pgfqpoint{5.166256in}{2.108491in}}%
\pgfpathclose%
\pgfusepath{fill}%
\end{pgfscope}%
\begin{pgfscope}%
\pgfpathrectangle{\pgfqpoint{1.254980in}{0.150000in}}{\pgfqpoint{5.490039in}{5.490039in}}%
\pgfusepath{clip}%
\pgfsetbuttcap%
\pgfsetroundjoin%
\definecolor{currentfill}{rgb}{0.271828,0.209303,0.504434}%
\pgfsetfillcolor{currentfill}%
\pgfsetfillopacity{0.700000}%
\pgfsetlinewidth{0.000000pt}%
\definecolor{currentstroke}{rgb}{0.000000,0.000000,0.000000}%
\pgfsetstrokecolor{currentstroke}%
\pgfsetdash{}{0pt}%
\pgfpathmoveto{\pgfqpoint{3.143264in}{1.911524in}}%
\pgfpathlineto{\pgfqpoint{3.156758in}{1.898134in}}%
\pgfpathlineto{\pgfqpoint{3.170251in}{1.884889in}}%
\pgfpathlineto{\pgfqpoint{3.183742in}{1.871788in}}%
\pgfpathlineto{\pgfqpoint{3.197232in}{1.858831in}}%
\pgfpathlineto{\pgfqpoint{3.205517in}{1.858708in}}%
\pgfpathlineto{\pgfqpoint{3.213790in}{1.858803in}}%
\pgfpathlineto{\pgfqpoint{3.222051in}{1.859111in}}%
\pgfpathlineto{\pgfqpoint{3.230299in}{1.859628in}}%
\pgfpathlineto{\pgfqpoint{3.216843in}{1.872148in}}%
\pgfpathlineto{\pgfqpoint{3.203386in}{1.884811in}}%
\pgfpathlineto{\pgfqpoint{3.189928in}{1.897618in}}%
\pgfpathlineto{\pgfqpoint{3.176469in}{1.910569in}}%
\pgfpathlineto{\pgfqpoint{3.168187in}{1.910483in}}%
\pgfpathlineto{\pgfqpoint{3.159893in}{1.910611in}}%
\pgfpathlineto{\pgfqpoint{3.151585in}{1.910957in}}%
\pgfpathlineto{\pgfqpoint{3.143264in}{1.911524in}}%
\pgfpathclose%
\pgfusepath{fill}%
\end{pgfscope}%
\begin{pgfscope}%
\pgfpathrectangle{\pgfqpoint{1.254980in}{0.150000in}}{\pgfqpoint{5.490039in}{5.490039in}}%
\pgfusepath{clip}%
\pgfsetbuttcap%
\pgfsetroundjoin%
\definecolor{currentfill}{rgb}{0.172719,0.448791,0.557885}%
\pgfsetfillcolor{currentfill}%
\pgfsetfillopacity{0.700000}%
\pgfsetlinewidth{0.000000pt}%
\definecolor{currentstroke}{rgb}{0.000000,0.000000,0.000000}%
\pgfsetstrokecolor{currentstroke}%
\pgfsetdash{}{0pt}%
\pgfpathmoveto{\pgfqpoint{2.655389in}{2.496574in}}%
\pgfpathlineto{\pgfqpoint{2.669027in}{2.477435in}}%
\pgfpathlineto{\pgfqpoint{2.682658in}{2.458471in}}%
\pgfpathlineto{\pgfqpoint{2.696282in}{2.439683in}}%
\pgfpathlineto{\pgfqpoint{2.709900in}{2.421068in}}%
\pgfpathlineto{\pgfqpoint{2.718550in}{2.416854in}}%
\pgfpathlineto{\pgfqpoint{2.727182in}{2.412901in}}%
\pgfpathlineto{\pgfqpoint{2.735798in}{2.409204in}}%
\pgfpathlineto{\pgfqpoint{2.744396in}{2.405760in}}%
\pgfpathlineto{\pgfqpoint{2.730824in}{2.423915in}}%
\pgfpathlineto{\pgfqpoint{2.717246in}{2.442242in}}%
\pgfpathlineto{\pgfqpoint{2.703662in}{2.460744in}}%
\pgfpathlineto{\pgfqpoint{2.690072in}{2.479420in}}%
\pgfpathlineto{\pgfqpoint{2.681428in}{2.483318in}}%
\pgfpathlineto{\pgfqpoint{2.672766in}{2.487474in}}%
\pgfpathlineto{\pgfqpoint{2.664087in}{2.491891in}}%
\pgfpathlineto{\pgfqpoint{2.655389in}{2.496574in}}%
\pgfpathclose%
\pgfusepath{fill}%
\end{pgfscope}%
\begin{pgfscope}%
\pgfpathrectangle{\pgfqpoint{1.254980in}{0.150000in}}{\pgfqpoint{5.490039in}{5.490039in}}%
\pgfusepath{clip}%
\pgfsetbuttcap%
\pgfsetroundjoin%
\definecolor{currentfill}{rgb}{0.273809,0.031497,0.358853}%
\pgfsetfillcolor{currentfill}%
\pgfsetfillopacity{0.700000}%
\pgfsetlinewidth{0.000000pt}%
\definecolor{currentstroke}{rgb}{0.000000,0.000000,0.000000}%
\pgfsetstrokecolor{currentstroke}%
\pgfsetdash{}{0pt}%
\pgfpathmoveto{\pgfqpoint{3.693158in}{1.548413in}}%
\pgfpathlineto{\pgfqpoint{3.706630in}{1.540711in}}%
\pgfpathlineto{\pgfqpoint{3.720105in}{1.533135in}}%
\pgfpathlineto{\pgfqpoint{3.733584in}{1.525683in}}%
\pgfpathlineto{\pgfqpoint{3.747066in}{1.518356in}}%
\pgfpathlineto{\pgfqpoint{3.755023in}{1.523485in}}%
\pgfpathlineto{\pgfqpoint{3.762972in}{1.528763in}}%
\pgfpathlineto{\pgfqpoint{3.770913in}{1.534187in}}%
\pgfpathlineto{\pgfqpoint{3.778847in}{1.539754in}}%
\pgfpathlineto{\pgfqpoint{3.765385in}{1.546689in}}%
\pgfpathlineto{\pgfqpoint{3.751927in}{1.553749in}}%
\pgfpathlineto{\pgfqpoint{3.738473in}{1.560933in}}%
\pgfpathlineto{\pgfqpoint{3.725022in}{1.568243in}}%
\pgfpathlineto{\pgfqpoint{3.717068in}{1.563062in}}%
\pgfpathlineto{\pgfqpoint{3.709107in}{1.558028in}}%
\pgfpathlineto{\pgfqpoint{3.701137in}{1.553144in}}%
\pgfpathlineto{\pgfqpoint{3.693158in}{1.548413in}}%
\pgfpathclose%
\pgfusepath{fill}%
\end{pgfscope}%
\begin{pgfscope}%
\pgfpathrectangle{\pgfqpoint{1.254980in}{0.150000in}}{\pgfqpoint{5.490039in}{5.490039in}}%
\pgfusepath{clip}%
\pgfsetbuttcap%
\pgfsetroundjoin%
\definecolor{currentfill}{rgb}{0.280894,0.078907,0.402329}%
\pgfsetfillcolor{currentfill}%
\pgfsetfillopacity{0.700000}%
\pgfsetlinewidth{0.000000pt}%
\definecolor{currentstroke}{rgb}{0.000000,0.000000,0.000000}%
\pgfsetstrokecolor{currentstroke}%
\pgfsetdash{}{0pt}%
\pgfpathmoveto{\pgfqpoint{3.499428in}{1.638363in}}%
\pgfpathlineto{\pgfqpoint{3.512894in}{1.628712in}}%
\pgfpathlineto{\pgfqpoint{3.526361in}{1.619193in}}%
\pgfpathlineto{\pgfqpoint{3.539831in}{1.609804in}}%
\pgfpathlineto{\pgfqpoint{3.553302in}{1.600545in}}%
\pgfpathlineto{\pgfqpoint{3.561364in}{1.603795in}}%
\pgfpathlineto{\pgfqpoint{3.569416in}{1.607221in}}%
\pgfpathlineto{\pgfqpoint{3.577458in}{1.610820in}}%
\pgfpathlineto{\pgfqpoint{3.585492in}{1.614586in}}%
\pgfpathlineto{\pgfqpoint{3.572045in}{1.623434in}}%
\pgfpathlineto{\pgfqpoint{3.558601in}{1.632412in}}%
\pgfpathlineto{\pgfqpoint{3.545159in}{1.641520in}}%
\pgfpathlineto{\pgfqpoint{3.531719in}{1.650758in}}%
\pgfpathlineto{\pgfqpoint{3.523661in}{1.647396in}}%
\pgfpathlineto{\pgfqpoint{3.515593in}{1.644207in}}%
\pgfpathlineto{\pgfqpoint{3.507515in}{1.641195in}}%
\pgfpathlineto{\pgfqpoint{3.499428in}{1.638363in}}%
\pgfpathclose%
\pgfusepath{fill}%
\end{pgfscope}%
\begin{pgfscope}%
\pgfpathrectangle{\pgfqpoint{1.254980in}{0.150000in}}{\pgfqpoint{5.490039in}{5.490039in}}%
\pgfusepath{clip}%
\pgfsetbuttcap%
\pgfsetroundjoin%
\definecolor{currentfill}{rgb}{0.277134,0.185228,0.489898}%
\pgfsetfillcolor{currentfill}%
\pgfsetfillopacity{0.700000}%
\pgfsetlinewidth{0.000000pt}%
\definecolor{currentstroke}{rgb}{0.000000,0.000000,0.000000}%
\pgfsetstrokecolor{currentstroke}%
\pgfsetdash{}{0pt}%
\pgfpathmoveto{\pgfqpoint{3.197232in}{1.858831in}}%
\pgfpathlineto{\pgfqpoint{3.210721in}{1.846017in}}%
\pgfpathlineto{\pgfqpoint{3.224209in}{1.833345in}}%
\pgfpathlineto{\pgfqpoint{3.237696in}{1.820815in}}%
\pgfpathlineto{\pgfqpoint{3.251183in}{1.808425in}}%
\pgfpathlineto{\pgfqpoint{3.259434in}{1.808746in}}%
\pgfpathlineto{\pgfqpoint{3.267674in}{1.809280in}}%
\pgfpathlineto{\pgfqpoint{3.275901in}{1.810021in}}%
\pgfpathlineto{\pgfqpoint{3.284117in}{1.810968in}}%
\pgfpathlineto{\pgfqpoint{3.270663in}{1.822922in}}%
\pgfpathlineto{\pgfqpoint{3.257209in}{1.835016in}}%
\pgfpathlineto{\pgfqpoint{3.243754in}{1.847251in}}%
\pgfpathlineto{\pgfqpoint{3.230299in}{1.859628in}}%
\pgfpathlineto{\pgfqpoint{3.222051in}{1.859111in}}%
\pgfpathlineto{\pgfqpoint{3.213790in}{1.858803in}}%
\pgfpathlineto{\pgfqpoint{3.205517in}{1.858708in}}%
\pgfpathlineto{\pgfqpoint{3.197232in}{1.858831in}}%
\pgfpathclose%
\pgfusepath{fill}%
\end{pgfscope}%
\begin{pgfscope}%
\pgfpathrectangle{\pgfqpoint{1.254980in}{0.150000in}}{\pgfqpoint{5.490039in}{5.490039in}}%
\pgfusepath{clip}%
\pgfsetbuttcap%
\pgfsetroundjoin%
\definecolor{currentfill}{rgb}{0.268510,0.009605,0.335427}%
\pgfsetfillcolor{currentfill}%
\pgfsetfillopacity{0.700000}%
\pgfsetlinewidth{0.000000pt}%
\definecolor{currentstroke}{rgb}{0.000000,0.000000,0.000000}%
\pgfsetstrokecolor{currentstroke}%
\pgfsetdash{}{0pt}%
\pgfpathmoveto{\pgfqpoint{4.111615in}{1.490553in}}%
\pgfpathlineto{\pgfqpoint{4.125155in}{1.486881in}}%
\pgfpathlineto{\pgfqpoint{4.138702in}{1.483327in}}%
\pgfpathlineto{\pgfqpoint{4.152255in}{1.479890in}}%
\pgfpathlineto{\pgfqpoint{4.165816in}{1.476569in}}%
\pgfpathlineto{\pgfqpoint{4.173597in}{1.485366in}}%
\pgfpathlineto{\pgfqpoint{4.181373in}{1.494252in}}%
\pgfpathlineto{\pgfqpoint{4.189144in}{1.503223in}}%
\pgfpathlineto{\pgfqpoint{4.196909in}{1.512276in}}%
\pgfpathlineto{\pgfqpoint{4.183360in}{1.515255in}}%
\pgfpathlineto{\pgfqpoint{4.169818in}{1.518351in}}%
\pgfpathlineto{\pgfqpoint{4.156283in}{1.521564in}}%
\pgfpathlineto{\pgfqpoint{4.142755in}{1.524894in}}%
\pgfpathlineto{\pgfqpoint{4.134978in}{1.516176in}}%
\pgfpathlineto{\pgfqpoint{4.127196in}{1.507544in}}%
\pgfpathlineto{\pgfqpoint{4.119408in}{1.499002in}}%
\pgfpathlineto{\pgfqpoint{4.111615in}{1.490553in}}%
\pgfpathclose%
\pgfusepath{fill}%
\end{pgfscope}%
\begin{pgfscope}%
\pgfpathrectangle{\pgfqpoint{1.254980in}{0.150000in}}{\pgfqpoint{5.490039in}{5.490039in}}%
\pgfusepath{clip}%
\pgfsetbuttcap%
\pgfsetroundjoin%
\definecolor{currentfill}{rgb}{0.260571,0.246922,0.522828}%
\pgfsetfillcolor{currentfill}%
\pgfsetfillopacity{0.700000}%
\pgfsetlinewidth{0.000000pt}%
\definecolor{currentstroke}{rgb}{0.000000,0.000000,0.000000}%
\pgfsetstrokecolor{currentstroke}%
\pgfsetdash{}{0pt}%
\pgfpathmoveto{\pgfqpoint{4.964794in}{1.927677in}}%
\pgfpathlineto{\pgfqpoint{4.978662in}{1.931283in}}%
\pgfpathlineto{\pgfqpoint{4.992542in}{1.935001in}}%
\pgfpathlineto{\pgfqpoint{5.006436in}{1.938831in}}%
\pgfpathlineto{\pgfqpoint{5.020341in}{1.942773in}}%
\pgfpathlineto{\pgfqpoint{5.027883in}{1.955213in}}%
\pgfpathlineto{\pgfqpoint{5.035421in}{1.967620in}}%
\pgfpathlineto{\pgfqpoint{5.042953in}{1.979991in}}%
\pgfpathlineto{\pgfqpoint{5.050480in}{1.992326in}}%
\pgfpathlineto{\pgfqpoint{5.036576in}{1.988198in}}%
\pgfpathlineto{\pgfqpoint{5.022684in}{1.984183in}}%
\pgfpathlineto{\pgfqpoint{5.008805in}{1.980280in}}%
\pgfpathlineto{\pgfqpoint{4.994939in}{1.976489in}}%
\pgfpathlineto{\pgfqpoint{4.987410in}{1.964333in}}%
\pgfpathlineto{\pgfqpoint{4.979876in}{1.952145in}}%
\pgfpathlineto{\pgfqpoint{4.972338in}{1.939926in}}%
\pgfpathlineto{\pgfqpoint{4.964794in}{1.927677in}}%
\pgfpathclose%
\pgfusepath{fill}%
\end{pgfscope}%
\begin{pgfscope}%
\pgfpathrectangle{\pgfqpoint{1.254980in}{0.150000in}}{\pgfqpoint{5.490039in}{5.490039in}}%
\pgfusepath{clip}%
\pgfsetbuttcap%
\pgfsetroundjoin%
\definecolor{currentfill}{rgb}{0.180629,0.429975,0.557282}%
\pgfsetfillcolor{currentfill}%
\pgfsetfillopacity{0.700000}%
\pgfsetlinewidth{0.000000pt}%
\definecolor{currentstroke}{rgb}{0.000000,0.000000,0.000000}%
\pgfsetstrokecolor{currentstroke}%
\pgfsetdash{}{0pt}%
\pgfpathmoveto{\pgfqpoint{5.453737in}{2.368488in}}%
\pgfpathlineto{\pgfqpoint{5.467874in}{2.375301in}}%
\pgfpathlineto{\pgfqpoint{5.482026in}{2.382226in}}%
\pgfpathlineto{\pgfqpoint{5.496193in}{2.389265in}}%
\pgfpathlineto{\pgfqpoint{5.510377in}{2.396416in}}%
\pgfpathlineto{\pgfqpoint{5.517755in}{2.407918in}}%
\pgfpathlineto{\pgfqpoint{5.525126in}{2.419342in}}%
\pgfpathlineto{\pgfqpoint{5.532491in}{2.430687in}}%
\pgfpathlineto{\pgfqpoint{5.539849in}{2.441954in}}%
\pgfpathlineto{\pgfqpoint{5.525668in}{2.434732in}}%
\pgfpathlineto{\pgfqpoint{5.511503in}{2.427622in}}%
\pgfpathlineto{\pgfqpoint{5.497354in}{2.420626in}}%
\pgfpathlineto{\pgfqpoint{5.483220in}{2.413742in}}%
\pgfpathlineto{\pgfqpoint{5.475859in}{2.402540in}}%
\pgfpathlineto{\pgfqpoint{5.468491in}{2.391263in}}%
\pgfpathlineto{\pgfqpoint{5.461117in}{2.379912in}}%
\pgfpathlineto{\pgfqpoint{5.453737in}{2.368488in}}%
\pgfpathclose%
\pgfusepath{fill}%
\end{pgfscope}%
\begin{pgfscope}%
\pgfpathrectangle{\pgfqpoint{1.254980in}{0.150000in}}{\pgfqpoint{5.490039in}{5.490039in}}%
\pgfusepath{clip}%
\pgfsetbuttcap%
\pgfsetroundjoin%
\definecolor{currentfill}{rgb}{0.160665,0.478540,0.558115}%
\pgfsetfillcolor{currentfill}%
\pgfsetfillopacity{0.700000}%
\pgfsetlinewidth{0.000000pt}%
\definecolor{currentstroke}{rgb}{0.000000,0.000000,0.000000}%
\pgfsetstrokecolor{currentstroke}%
\pgfsetdash{}{0pt}%
\pgfpathmoveto{\pgfqpoint{2.600770in}{2.574912in}}%
\pgfpathlineto{\pgfqpoint{2.614436in}{2.555058in}}%
\pgfpathlineto{\pgfqpoint{2.628094in}{2.535385in}}%
\pgfpathlineto{\pgfqpoint{2.641745in}{2.515890in}}%
\pgfpathlineto{\pgfqpoint{2.655389in}{2.496574in}}%
\pgfpathlineto{\pgfqpoint{2.664087in}{2.491891in}}%
\pgfpathlineto{\pgfqpoint{2.672766in}{2.487474in}}%
\pgfpathlineto{\pgfqpoint{2.681428in}{2.483318in}}%
\pgfpathlineto{\pgfqpoint{2.690072in}{2.479420in}}%
\pgfpathlineto{\pgfqpoint{2.676475in}{2.498273in}}%
\pgfpathlineto{\pgfqpoint{2.662872in}{2.517303in}}%
\pgfpathlineto{\pgfqpoint{2.649262in}{2.536512in}}%
\pgfpathlineto{\pgfqpoint{2.635645in}{2.555899in}}%
\pgfpathlineto{\pgfqpoint{2.626953in}{2.560254in}}%
\pgfpathlineto{\pgfqpoint{2.618244in}{2.564872in}}%
\pgfpathlineto{\pgfqpoint{2.609517in}{2.569757in}}%
\pgfpathlineto{\pgfqpoint{2.600770in}{2.574912in}}%
\pgfpathclose%
\pgfusepath{fill}%
\end{pgfscope}%
\begin{pgfscope}%
\pgfpathrectangle{\pgfqpoint{1.254980in}{0.150000in}}{\pgfqpoint{5.490039in}{5.490039in}}%
\pgfusepath{clip}%
\pgfsetbuttcap%
\pgfsetroundjoin%
\definecolor{currentfill}{rgb}{0.212395,0.359683,0.551710}%
\pgfsetfillcolor{currentfill}%
\pgfsetfillopacity{0.700000}%
\pgfsetlinewidth{0.000000pt}%
\definecolor{currentstroke}{rgb}{0.000000,0.000000,0.000000}%
\pgfsetstrokecolor{currentstroke}%
\pgfsetdash{}{0pt}%
\pgfpathmoveto{\pgfqpoint{5.252094in}{2.177857in}}%
\pgfpathlineto{\pgfqpoint{5.266118in}{2.183456in}}%
\pgfpathlineto{\pgfqpoint{5.280156in}{2.189168in}}%
\pgfpathlineto{\pgfqpoint{5.294208in}{2.194993in}}%
\pgfpathlineto{\pgfqpoint{5.308275in}{2.200930in}}%
\pgfpathlineto{\pgfqpoint{5.315729in}{2.213075in}}%
\pgfpathlineto{\pgfqpoint{5.323178in}{2.225156in}}%
\pgfpathlineto{\pgfqpoint{5.330622in}{2.237174in}}%
\pgfpathlineto{\pgfqpoint{5.338059in}{2.249127in}}%
\pgfpathlineto{\pgfqpoint{5.323993in}{2.243069in}}%
\pgfpathlineto{\pgfqpoint{5.309943in}{2.237124in}}%
\pgfpathlineto{\pgfqpoint{5.295906in}{2.231291in}}%
\pgfpathlineto{\pgfqpoint{5.281885in}{2.225571in}}%
\pgfpathlineto{\pgfqpoint{5.274446in}{2.213732in}}%
\pgfpathlineto{\pgfqpoint{5.267001in}{2.201833in}}%
\pgfpathlineto{\pgfqpoint{5.259550in}{2.189874in}}%
\pgfpathlineto{\pgfqpoint{5.252094in}{2.177857in}}%
\pgfpathclose%
\pgfusepath{fill}%
\end{pgfscope}%
\begin{pgfscope}%
\pgfpathrectangle{\pgfqpoint{1.254980in}{0.150000in}}{\pgfqpoint{5.490039in}{5.490039in}}%
\pgfusepath{clip}%
\pgfsetbuttcap%
\pgfsetroundjoin%
\definecolor{currentfill}{rgb}{0.281446,0.084320,0.407414}%
\pgfsetfillcolor{currentfill}%
\pgfsetfillopacity{0.700000}%
\pgfsetlinewidth{0.000000pt}%
\definecolor{currentstroke}{rgb}{0.000000,0.000000,0.000000}%
\pgfsetstrokecolor{currentstroke}%
\pgfsetdash{}{0pt}%
\pgfpathmoveto{\pgfqpoint{4.507128in}{1.609364in}}%
\pgfpathlineto{\pgfqpoint{4.520794in}{1.609270in}}%
\pgfpathlineto{\pgfqpoint{4.534470in}{1.609291in}}%
\pgfpathlineto{\pgfqpoint{4.548155in}{1.609424in}}%
\pgfpathlineto{\pgfqpoint{4.561850in}{1.609671in}}%
\pgfpathlineto{\pgfqpoint{4.569514in}{1.620993in}}%
\pgfpathlineto{\pgfqpoint{4.577173in}{1.632343in}}%
\pgfpathlineto{\pgfqpoint{4.584828in}{1.643719in}}%
\pgfpathlineto{\pgfqpoint{4.592479in}{1.655121in}}%
\pgfpathlineto{\pgfqpoint{4.578789in}{1.654594in}}%
\pgfpathlineto{\pgfqpoint{4.565108in}{1.654181in}}%
\pgfpathlineto{\pgfqpoint{4.551438in}{1.653882in}}%
\pgfpathlineto{\pgfqpoint{4.537777in}{1.653697in}}%
\pgfpathlineto{\pgfqpoint{4.530121in}{1.642568in}}%
\pgfpathlineto{\pgfqpoint{4.522461in}{1.631469in}}%
\pgfpathlineto{\pgfqpoint{4.514797in}{1.620400in}}%
\pgfpathlineto{\pgfqpoint{4.507128in}{1.609364in}}%
\pgfpathclose%
\pgfusepath{fill}%
\end{pgfscope}%
\begin{pgfscope}%
\pgfpathrectangle{\pgfqpoint{1.254980in}{0.150000in}}{\pgfqpoint{5.490039in}{5.490039in}}%
\pgfusepath{clip}%
\pgfsetbuttcap%
\pgfsetroundjoin%
\definecolor{currentfill}{rgb}{0.283091,0.110553,0.431554}%
\pgfsetfillcolor{currentfill}%
\pgfsetfillopacity{0.700000}%
\pgfsetlinewidth{0.000000pt}%
\definecolor{currentstroke}{rgb}{0.000000,0.000000,0.000000}%
\pgfsetstrokecolor{currentstroke}%
\pgfsetdash{}{0pt}%
\pgfpathmoveto{\pgfqpoint{4.592479in}{1.655121in}}%
\pgfpathlineto{\pgfqpoint{4.606179in}{1.655760in}}%
\pgfpathlineto{\pgfqpoint{4.619890in}{1.656513in}}%
\pgfpathlineto{\pgfqpoint{4.633610in}{1.657379in}}%
\pgfpathlineto{\pgfqpoint{4.647341in}{1.658357in}}%
\pgfpathlineto{\pgfqpoint{4.654983in}{1.670051in}}%
\pgfpathlineto{\pgfqpoint{4.662621in}{1.681760in}}%
\pgfpathlineto{\pgfqpoint{4.670254in}{1.693484in}}%
\pgfpathlineto{\pgfqpoint{4.677882in}{1.705221in}}%
\pgfpathlineto{\pgfqpoint{4.664155in}{1.703978in}}%
\pgfpathlineto{\pgfqpoint{4.650438in}{1.702849in}}%
\pgfpathlineto{\pgfqpoint{4.636732in}{1.701832in}}%
\pgfpathlineto{\pgfqpoint{4.623036in}{1.700929in}}%
\pgfpathlineto{\pgfqpoint{4.615404in}{1.689451in}}%
\pgfpathlineto{\pgfqpoint{4.607766in}{1.677988in}}%
\pgfpathlineto{\pgfqpoint{4.600125in}{1.666544in}}%
\pgfpathlineto{\pgfqpoint{4.592479in}{1.655121in}}%
\pgfpathclose%
\pgfusepath{fill}%
\end{pgfscope}%
\begin{pgfscope}%
\pgfpathrectangle{\pgfqpoint{1.254980in}{0.150000in}}{\pgfqpoint{5.490039in}{5.490039in}}%
\pgfusepath{clip}%
\pgfsetbuttcap%
\pgfsetroundjoin%
\definecolor{currentfill}{rgb}{0.278791,0.062145,0.386592}%
\pgfsetfillcolor{currentfill}%
\pgfsetfillopacity{0.700000}%
\pgfsetlinewidth{0.000000pt}%
\definecolor{currentstroke}{rgb}{0.000000,0.000000,0.000000}%
\pgfsetstrokecolor{currentstroke}%
\pgfsetdash{}{0pt}%
\pgfpathmoveto{\pgfqpoint{4.421809in}{1.568281in}}%
\pgfpathlineto{\pgfqpoint{4.435445in}{1.567438in}}%
\pgfpathlineto{\pgfqpoint{4.449089in}{1.566708in}}%
\pgfpathlineto{\pgfqpoint{4.462743in}{1.566093in}}%
\pgfpathlineto{\pgfqpoint{4.476406in}{1.565591in}}%
\pgfpathlineto{\pgfqpoint{4.484093in}{1.576474in}}%
\pgfpathlineto{\pgfqpoint{4.491776in}{1.587398in}}%
\pgfpathlineto{\pgfqpoint{4.499454in}{1.598362in}}%
\pgfpathlineto{\pgfqpoint{4.507128in}{1.609364in}}%
\pgfpathlineto{\pgfqpoint{4.493471in}{1.609571in}}%
\pgfpathlineto{\pgfqpoint{4.479824in}{1.609892in}}%
\pgfpathlineto{\pgfqpoint{4.466185in}{1.610327in}}%
\pgfpathlineto{\pgfqpoint{4.452556in}{1.610876in}}%
\pgfpathlineto{\pgfqpoint{4.444876in}{1.600163in}}%
\pgfpathlineto{\pgfqpoint{4.437192in}{1.589492in}}%
\pgfpathlineto{\pgfqpoint{4.429503in}{1.578864in}}%
\pgfpathlineto{\pgfqpoint{4.421809in}{1.568281in}}%
\pgfpathclose%
\pgfusepath{fill}%
\end{pgfscope}%
\begin{pgfscope}%
\pgfpathrectangle{\pgfqpoint{1.254980in}{0.150000in}}{\pgfqpoint{5.490039in}{5.490039in}}%
\pgfusepath{clip}%
\pgfsetbuttcap%
\pgfsetroundjoin%
\definecolor{currentfill}{rgb}{0.134692,0.658636,0.517649}%
\pgfsetfillcolor{currentfill}%
\pgfsetfillopacity{0.700000}%
\pgfsetlinewidth{0.000000pt}%
\definecolor{currentstroke}{rgb}{0.000000,0.000000,0.000000}%
\pgfsetstrokecolor{currentstroke}%
\pgfsetdash{}{0pt}%
\pgfpathmoveto{\pgfqpoint{2.306303in}{3.078188in}}%
\pgfpathlineto{\pgfqpoint{2.320140in}{3.054121in}}%
\pgfpathlineto{\pgfqpoint{2.333965in}{3.030265in}}%
\pgfpathlineto{\pgfqpoint{2.347780in}{3.006618in}}%
\pgfpathlineto{\pgfqpoint{2.361583in}{2.983179in}}%
\pgfpathlineto{\pgfqpoint{2.370511in}{2.976766in}}%
\pgfpathlineto{\pgfqpoint{2.379419in}{2.970636in}}%
\pgfpathlineto{\pgfqpoint{2.388306in}{2.964782in}}%
\pgfpathlineto{\pgfqpoint{2.397173in}{2.959203in}}%
\pgfpathlineto{\pgfqpoint{2.383424in}{2.982176in}}%
\pgfpathlineto{\pgfqpoint{2.369664in}{3.005355in}}%
\pgfpathlineto{\pgfqpoint{2.355894in}{3.028743in}}%
\pgfpathlineto{\pgfqpoint{2.342112in}{3.052340in}}%
\pgfpathlineto{\pgfqpoint{2.333191in}{3.058380in}}%
\pgfpathlineto{\pgfqpoint{2.324249in}{3.064698in}}%
\pgfpathlineto{\pgfqpoint{2.315287in}{3.071300in}}%
\pgfpathlineto{\pgfqpoint{2.306303in}{3.078188in}}%
\pgfpathclose%
\pgfusepath{fill}%
\end{pgfscope}%
\begin{pgfscope}%
\pgfpathrectangle{\pgfqpoint{1.254980in}{0.150000in}}{\pgfqpoint{5.490039in}{5.490039in}}%
\pgfusepath{clip}%
\pgfsetbuttcap%
\pgfsetroundjoin%
\definecolor{currentfill}{rgb}{0.280255,0.165693,0.476498}%
\pgfsetfillcolor{currentfill}%
\pgfsetfillopacity{0.700000}%
\pgfsetlinewidth{0.000000pt}%
\definecolor{currentstroke}{rgb}{0.000000,0.000000,0.000000}%
\pgfsetstrokecolor{currentstroke}%
\pgfsetdash{}{0pt}%
\pgfpathmoveto{\pgfqpoint{3.251183in}{1.808425in}}%
\pgfpathlineto{\pgfqpoint{3.264669in}{1.796177in}}%
\pgfpathlineto{\pgfqpoint{3.278154in}{1.784068in}}%
\pgfpathlineto{\pgfqpoint{3.291640in}{1.772098in}}%
\pgfpathlineto{\pgfqpoint{3.305125in}{1.760267in}}%
\pgfpathlineto{\pgfqpoint{3.313344in}{1.761030in}}%
\pgfpathlineto{\pgfqpoint{3.321551in}{1.762000in}}%
\pgfpathlineto{\pgfqpoint{3.329747in}{1.763174in}}%
\pgfpathlineto{\pgfqpoint{3.337931in}{1.764548in}}%
\pgfpathlineto{\pgfqpoint{3.324477in}{1.775945in}}%
\pgfpathlineto{\pgfqpoint{3.311024in}{1.787481in}}%
\pgfpathlineto{\pgfqpoint{3.297570in}{1.799155in}}%
\pgfpathlineto{\pgfqpoint{3.284117in}{1.810968in}}%
\pgfpathlineto{\pgfqpoint{3.275901in}{1.810021in}}%
\pgfpathlineto{\pgfqpoint{3.267674in}{1.809280in}}%
\pgfpathlineto{\pgfqpoint{3.259434in}{1.808746in}}%
\pgfpathlineto{\pgfqpoint{3.251183in}{1.808425in}}%
\pgfpathclose%
\pgfusepath{fill}%
\end{pgfscope}%
\begin{pgfscope}%
\pgfpathrectangle{\pgfqpoint{1.254980in}{0.150000in}}{\pgfqpoint{5.490039in}{5.490039in}}%
\pgfusepath{clip}%
\pgfsetbuttcap%
\pgfsetroundjoin%
\definecolor{currentfill}{rgb}{0.168126,0.459988,0.558082}%
\pgfsetfillcolor{currentfill}%
\pgfsetfillopacity{0.700000}%
\pgfsetlinewidth{0.000000pt}%
\definecolor{currentstroke}{rgb}{0.000000,0.000000,0.000000}%
\pgfsetstrokecolor{currentstroke}%
\pgfsetdash{}{0pt}%
\pgfpathmoveto{\pgfqpoint{5.539849in}{2.441954in}}%
\pgfpathlineto{\pgfqpoint{5.554045in}{2.449289in}}%
\pgfpathlineto{\pgfqpoint{5.568258in}{2.456737in}}%
\pgfpathlineto{\pgfqpoint{5.582486in}{2.464298in}}%
\pgfpathlineto{\pgfqpoint{5.589836in}{2.475531in}}%
\pgfpathlineto{\pgfqpoint{5.597178in}{2.486681in}}%
\pgfpathlineto{\pgfqpoint{5.604514in}{2.497750in}}%
\pgfpathlineto{\pgfqpoint{5.611842in}{2.508735in}}%
\pgfpathlineto{\pgfqpoint{5.597617in}{2.501120in}}%
\pgfpathlineto{\pgfqpoint{5.583408in}{2.493619in}}%
\pgfpathlineto{\pgfqpoint{5.569214in}{2.486230in}}%
\pgfpathlineto{\pgfqpoint{5.561883in}{2.475280in}}%
\pgfpathlineto{\pgfqpoint{5.554545in}{2.464250in}}%
\pgfpathlineto{\pgfqpoint{5.547200in}{2.453142in}}%
\pgfpathlineto{\pgfqpoint{5.539849in}{2.441954in}}%
\pgfpathclose%
\pgfusepath{fill}%
\end{pgfscope}%
\begin{pgfscope}%
\pgfpathrectangle{\pgfqpoint{1.254980in}{0.150000in}}{\pgfqpoint{5.490039in}{5.490039in}}%
\pgfusepath{clip}%
\pgfsetbuttcap%
\pgfsetroundjoin%
\definecolor{currentfill}{rgb}{0.268510,0.009605,0.335427}%
\pgfsetfillcolor{currentfill}%
\pgfsetfillopacity{0.700000}%
\pgfsetlinewidth{0.000000pt}%
\definecolor{currentstroke}{rgb}{0.000000,0.000000,0.000000}%
\pgfsetstrokecolor{currentstroke}%
\pgfsetdash{}{0pt}%
\pgfpathmoveto{\pgfqpoint{3.886693in}{1.488704in}}%
\pgfpathlineto{\pgfqpoint{3.900194in}{1.482872in}}%
\pgfpathlineto{\pgfqpoint{3.913700in}{1.477160in}}%
\pgfpathlineto{\pgfqpoint{3.927211in}{1.471569in}}%
\pgfpathlineto{\pgfqpoint{3.940728in}{1.466098in}}%
\pgfpathlineto{\pgfqpoint{3.948599in}{1.472946in}}%
\pgfpathlineto{\pgfqpoint{3.956465in}{1.479918in}}%
\pgfpathlineto{\pgfqpoint{3.964323in}{1.487010in}}%
\pgfpathlineto{\pgfqpoint{3.972175in}{1.494220in}}%
\pgfpathlineto{\pgfqpoint{3.958675in}{1.499317in}}%
\pgfpathlineto{\pgfqpoint{3.945180in}{1.504534in}}%
\pgfpathlineto{\pgfqpoint{3.931691in}{1.509872in}}%
\pgfpathlineto{\pgfqpoint{3.918206in}{1.515330in}}%
\pgfpathlineto{\pgfqpoint{3.910338in}{1.508488in}}%
\pgfpathlineto{\pgfqpoint{3.902463in}{1.501767in}}%
\pgfpathlineto{\pgfqpoint{3.894581in}{1.495172in}}%
\pgfpathlineto{\pgfqpoint{3.886693in}{1.488704in}}%
\pgfpathclose%
\pgfusepath{fill}%
\end{pgfscope}%
\begin{pgfscope}%
\pgfpathrectangle{\pgfqpoint{1.254980in}{0.150000in}}{\pgfqpoint{5.490039in}{5.490039in}}%
\pgfusepath{clip}%
\pgfsetbuttcap%
\pgfsetroundjoin%
\definecolor{currentfill}{rgb}{0.282623,0.140926,0.457517}%
\pgfsetfillcolor{currentfill}%
\pgfsetfillopacity{0.700000}%
\pgfsetlinewidth{0.000000pt}%
\definecolor{currentstroke}{rgb}{0.000000,0.000000,0.000000}%
\pgfsetstrokecolor{currentstroke}%
\pgfsetdash{}{0pt}%
\pgfpathmoveto{\pgfqpoint{4.677882in}{1.705221in}}%
\pgfpathlineto{\pgfqpoint{4.691620in}{1.706576in}}%
\pgfpathlineto{\pgfqpoint{4.705369in}{1.708044in}}%
\pgfpathlineto{\pgfqpoint{4.719128in}{1.709625in}}%
\pgfpathlineto{\pgfqpoint{4.732898in}{1.711319in}}%
\pgfpathlineto{\pgfqpoint{4.740519in}{1.723320in}}%
\pgfpathlineto{\pgfqpoint{4.748136in}{1.735326in}}%
\pgfpathlineto{\pgfqpoint{4.755748in}{1.747334in}}%
\pgfpathlineto{\pgfqpoint{4.763356in}{1.759343in}}%
\pgfpathlineto{\pgfqpoint{4.749588in}{1.757401in}}%
\pgfpathlineto{\pgfqpoint{4.735832in}{1.755571in}}%
\pgfpathlineto{\pgfqpoint{4.722087in}{1.753855in}}%
\pgfpathlineto{\pgfqpoint{4.708352in}{1.752251in}}%
\pgfpathlineto{\pgfqpoint{4.700742in}{1.740485in}}%
\pgfpathlineto{\pgfqpoint{4.693126in}{1.728723in}}%
\pgfpathlineto{\pgfqpoint{4.685507in}{1.716967in}}%
\pgfpathlineto{\pgfqpoint{4.677882in}{1.705221in}}%
\pgfpathclose%
\pgfusepath{fill}%
\end{pgfscope}%
\begin{pgfscope}%
\pgfpathrectangle{\pgfqpoint{1.254980in}{0.150000in}}{\pgfqpoint{5.490039in}{5.490039in}}%
\pgfusepath{clip}%
\pgfsetbuttcap%
\pgfsetroundjoin%
\definecolor{currentfill}{rgb}{0.274952,0.037752,0.364543}%
\pgfsetfillcolor{currentfill}%
\pgfsetfillopacity{0.700000}%
\pgfsetlinewidth{0.000000pt}%
\definecolor{currentstroke}{rgb}{0.000000,0.000000,0.000000}%
\pgfsetstrokecolor{currentstroke}%
\pgfsetdash{}{0pt}%
\pgfpathmoveto{\pgfqpoint{4.336501in}{1.532217in}}%
\pgfpathlineto{\pgfqpoint{4.350110in}{1.530605in}}%
\pgfpathlineto{\pgfqpoint{4.363727in}{1.529108in}}%
\pgfpathlineto{\pgfqpoint{4.377353in}{1.527725in}}%
\pgfpathlineto{\pgfqpoint{4.390987in}{1.526457in}}%
\pgfpathlineto{\pgfqpoint{4.398699in}{1.536833in}}%
\pgfpathlineto{\pgfqpoint{4.406407in}{1.547264in}}%
\pgfpathlineto{\pgfqpoint{4.414111in}{1.557747in}}%
\pgfpathlineto{\pgfqpoint{4.421809in}{1.568281in}}%
\pgfpathlineto{\pgfqpoint{4.408182in}{1.569239in}}%
\pgfpathlineto{\pgfqpoint{4.394564in}{1.570312in}}%
\pgfpathlineto{\pgfqpoint{4.380955in}{1.571499in}}%
\pgfpathlineto{\pgfqpoint{4.367354in}{1.572801in}}%
\pgfpathlineto{\pgfqpoint{4.359648in}{1.562571in}}%
\pgfpathlineto{\pgfqpoint{4.351937in}{1.552395in}}%
\pgfpathlineto{\pgfqpoint{4.344221in}{1.542276in}}%
\pgfpathlineto{\pgfqpoint{4.336501in}{1.532217in}}%
\pgfpathclose%
\pgfusepath{fill}%
\end{pgfscope}%
\begin{pgfscope}%
\pgfpathrectangle{\pgfqpoint{1.254980in}{0.150000in}}{\pgfqpoint{5.490039in}{5.490039in}}%
\pgfusepath{clip}%
\pgfsetbuttcap%
\pgfsetroundjoin%
\definecolor{currentfill}{rgb}{0.149039,0.508051,0.557250}%
\pgfsetfillcolor{currentfill}%
\pgfsetfillopacity{0.700000}%
\pgfsetlinewidth{0.000000pt}%
\definecolor{currentstroke}{rgb}{0.000000,0.000000,0.000000}%
\pgfsetstrokecolor{currentstroke}%
\pgfsetdash{}{0pt}%
\pgfpathmoveto{\pgfqpoint{2.546032in}{2.656155in}}%
\pgfpathlineto{\pgfqpoint{2.559728in}{2.635568in}}%
\pgfpathlineto{\pgfqpoint{2.573417in}{2.615166in}}%
\pgfpathlineto{\pgfqpoint{2.587097in}{2.594948in}}%
\pgfpathlineto{\pgfqpoint{2.600770in}{2.574912in}}%
\pgfpathlineto{\pgfqpoint{2.609517in}{2.569757in}}%
\pgfpathlineto{\pgfqpoint{2.618244in}{2.564872in}}%
\pgfpathlineto{\pgfqpoint{2.626953in}{2.560254in}}%
\pgfpathlineto{\pgfqpoint{2.635645in}{2.555899in}}%
\pgfpathlineto{\pgfqpoint{2.622021in}{2.575468in}}%
\pgfpathlineto{\pgfqpoint{2.608389in}{2.595218in}}%
\pgfpathlineto{\pgfqpoint{2.594751in}{2.615151in}}%
\pgfpathlineto{\pgfqpoint{2.581104in}{2.635268in}}%
\pgfpathlineto{\pgfqpoint{2.572364in}{2.640083in}}%
\pgfpathlineto{\pgfqpoint{2.563606in}{2.645167in}}%
\pgfpathlineto{\pgfqpoint{2.554829in}{2.650523in}}%
\pgfpathlineto{\pgfqpoint{2.546032in}{2.656155in}}%
\pgfpathclose%
\pgfusepath{fill}%
\end{pgfscope}%
\begin{pgfscope}%
\pgfpathrectangle{\pgfqpoint{1.254980in}{0.150000in}}{\pgfqpoint{5.490039in}{5.490039in}}%
\pgfusepath{clip}%
\pgfsetbuttcap%
\pgfsetroundjoin%
\definecolor{currentfill}{rgb}{0.248629,0.278775,0.534556}%
\pgfsetfillcolor{currentfill}%
\pgfsetfillopacity{0.700000}%
\pgfsetlinewidth{0.000000pt}%
\definecolor{currentstroke}{rgb}{0.000000,0.000000,0.000000}%
\pgfsetstrokecolor{currentstroke}%
\pgfsetdash{}{0pt}%
\pgfpathmoveto{\pgfqpoint{5.050480in}{1.992326in}}%
\pgfpathlineto{\pgfqpoint{5.064398in}{1.996567in}}%
\pgfpathlineto{\pgfqpoint{5.078329in}{2.000920in}}%
\pgfpathlineto{\pgfqpoint{5.092273in}{2.005385in}}%
\pgfpathlineto{\pgfqpoint{5.106230in}{2.009962in}}%
\pgfpathlineto{\pgfqpoint{5.113751in}{2.022436in}}%
\pgfpathlineto{\pgfqpoint{5.121268in}{2.034867in}}%
\pgfpathlineto{\pgfqpoint{5.128779in}{2.047254in}}%
\pgfpathlineto{\pgfqpoint{5.136285in}{2.059596in}}%
\pgfpathlineto{\pgfqpoint{5.122328in}{2.054849in}}%
\pgfpathlineto{\pgfqpoint{5.108385in}{2.050214in}}%
\pgfpathlineto{\pgfqpoint{5.094456in}{2.045692in}}%
\pgfpathlineto{\pgfqpoint{5.080539in}{2.041282in}}%
\pgfpathlineto{\pgfqpoint{5.073032in}{2.029103in}}%
\pgfpathlineto{\pgfqpoint{5.065520in}{2.016884in}}%
\pgfpathlineto{\pgfqpoint{5.058003in}{2.004624in}}%
\pgfpathlineto{\pgfqpoint{5.050480in}{1.992326in}}%
\pgfpathclose%
\pgfusepath{fill}%
\end{pgfscope}%
\begin{pgfscope}%
\pgfpathrectangle{\pgfqpoint{1.254980in}{0.150000in}}{\pgfqpoint{5.490039in}{5.490039in}}%
\pgfusepath{clip}%
\pgfsetbuttcap%
\pgfsetroundjoin%
\definecolor{currentfill}{rgb}{0.279566,0.067836,0.391917}%
\pgfsetfillcolor{currentfill}%
\pgfsetfillopacity{0.700000}%
\pgfsetlinewidth{0.000000pt}%
\definecolor{currentstroke}{rgb}{0.000000,0.000000,0.000000}%
\pgfsetstrokecolor{currentstroke}%
\pgfsetdash{}{0pt}%
\pgfpathmoveto{\pgfqpoint{3.553302in}{1.600545in}}%
\pgfpathlineto{\pgfqpoint{3.566776in}{1.591415in}}%
\pgfpathlineto{\pgfqpoint{3.580252in}{1.582414in}}%
\pgfpathlineto{\pgfqpoint{3.593730in}{1.573542in}}%
\pgfpathlineto{\pgfqpoint{3.607211in}{1.564797in}}%
\pgfpathlineto{\pgfqpoint{3.615247in}{1.568465in}}%
\pgfpathlineto{\pgfqpoint{3.623275in}{1.572304in}}%
\pgfpathlineto{\pgfqpoint{3.631293in}{1.576310in}}%
\pgfpathlineto{\pgfqpoint{3.639302in}{1.580481in}}%
\pgfpathlineto{\pgfqpoint{3.625846in}{1.588815in}}%
\pgfpathlineto{\pgfqpoint{3.612392in}{1.597277in}}%
\pgfpathlineto{\pgfqpoint{3.598941in}{1.605867in}}%
\pgfpathlineto{\pgfqpoint{3.585492in}{1.614586in}}%
\pgfpathlineto{\pgfqpoint{3.577458in}{1.610820in}}%
\pgfpathlineto{\pgfqpoint{3.569416in}{1.607221in}}%
\pgfpathlineto{\pgfqpoint{3.561364in}{1.603795in}}%
\pgfpathlineto{\pgfqpoint{3.553302in}{1.600545in}}%
\pgfpathclose%
\pgfusepath{fill}%
\end{pgfscope}%
\begin{pgfscope}%
\pgfpathrectangle{\pgfqpoint{1.254980in}{0.150000in}}{\pgfqpoint{5.490039in}{5.490039in}}%
\pgfusepath{clip}%
\pgfsetbuttcap%
\pgfsetroundjoin%
\definecolor{currentfill}{rgb}{0.271305,0.019942,0.347269}%
\pgfsetfillcolor{currentfill}%
\pgfsetfillopacity{0.700000}%
\pgfsetlinewidth{0.000000pt}%
\definecolor{currentstroke}{rgb}{0.000000,0.000000,0.000000}%
\pgfsetstrokecolor{currentstroke}%
\pgfsetdash{}{0pt}%
\pgfpathmoveto{\pgfqpoint{3.747066in}{1.518356in}}%
\pgfpathlineto{\pgfqpoint{3.760552in}{1.511154in}}%
\pgfpathlineto{\pgfqpoint{3.774042in}{1.504074in}}%
\pgfpathlineto{\pgfqpoint{3.787536in}{1.497119in}}%
\pgfpathlineto{\pgfqpoint{3.801033in}{1.490286in}}%
\pgfpathlineto{\pgfqpoint{3.808970in}{1.495812in}}%
\pgfpathlineto{\pgfqpoint{3.816899in}{1.501484in}}%
\pgfpathlineto{\pgfqpoint{3.824821in}{1.507297in}}%
\pgfpathlineto{\pgfqpoint{3.832735in}{1.513249in}}%
\pgfpathlineto{\pgfqpoint{3.819256in}{1.519690in}}%
\pgfpathlineto{\pgfqpoint{3.805782in}{1.526255in}}%
\pgfpathlineto{\pgfqpoint{3.792312in}{1.532943in}}%
\pgfpathlineto{\pgfqpoint{3.778847in}{1.539754in}}%
\pgfpathlineto{\pgfqpoint{3.770913in}{1.534187in}}%
\pgfpathlineto{\pgfqpoint{3.762972in}{1.528763in}}%
\pgfpathlineto{\pgfqpoint{3.755023in}{1.523485in}}%
\pgfpathlineto{\pgfqpoint{3.747066in}{1.518356in}}%
\pgfpathclose%
\pgfusepath{fill}%
\end{pgfscope}%
\begin{pgfscope}%
\pgfpathrectangle{\pgfqpoint{1.254980in}{0.150000in}}{\pgfqpoint{5.490039in}{5.490039in}}%
\pgfusepath{clip}%
\pgfsetbuttcap%
\pgfsetroundjoin%
\definecolor{currentfill}{rgb}{0.267004,0.004874,0.329415}%
\pgfsetfillcolor{currentfill}%
\pgfsetfillopacity{0.700000}%
\pgfsetlinewidth{0.000000pt}%
\definecolor{currentstroke}{rgb}{0.000000,0.000000,0.000000}%
\pgfsetstrokecolor{currentstroke}%
\pgfsetdash{}{0pt}%
\pgfpathmoveto{\pgfqpoint{4.026232in}{1.475028in}}%
\pgfpathlineto{\pgfqpoint{4.039760in}{1.470527in}}%
\pgfpathlineto{\pgfqpoint{4.053295in}{1.466144in}}%
\pgfpathlineto{\pgfqpoint{4.066835in}{1.461880in}}%
\pgfpathlineto{\pgfqpoint{4.080382in}{1.457733in}}%
\pgfpathlineto{\pgfqpoint{4.088199in}{1.465786in}}%
\pgfpathlineto{\pgfqpoint{4.096010in}{1.473941in}}%
\pgfpathlineto{\pgfqpoint{4.103815in}{1.482198in}}%
\pgfpathlineto{\pgfqpoint{4.111615in}{1.490553in}}%
\pgfpathlineto{\pgfqpoint{4.098081in}{1.494342in}}%
\pgfpathlineto{\pgfqpoint{4.084554in}{1.498249in}}%
\pgfpathlineto{\pgfqpoint{4.071033in}{1.502274in}}%
\pgfpathlineto{\pgfqpoint{4.057518in}{1.506417in}}%
\pgfpathlineto{\pgfqpoint{4.049705in}{1.498414in}}%
\pgfpathlineto{\pgfqpoint{4.041887in}{1.490513in}}%
\pgfpathlineto{\pgfqpoint{4.034062in}{1.482716in}}%
\pgfpathlineto{\pgfqpoint{4.026232in}{1.475028in}}%
\pgfpathclose%
\pgfusepath{fill}%
\end{pgfscope}%
\begin{pgfscope}%
\pgfpathrectangle{\pgfqpoint{1.254980in}{0.150000in}}{\pgfqpoint{5.490039in}{5.490039in}}%
\pgfusepath{clip}%
\pgfsetbuttcap%
\pgfsetroundjoin%
\definecolor{currentfill}{rgb}{0.279574,0.170599,0.479997}%
\pgfsetfillcolor{currentfill}%
\pgfsetfillopacity{0.700000}%
\pgfsetlinewidth{0.000000pt}%
\definecolor{currentstroke}{rgb}{0.000000,0.000000,0.000000}%
\pgfsetstrokecolor{currentstroke}%
\pgfsetdash{}{0pt}%
\pgfpathmoveto{\pgfqpoint{4.763356in}{1.759343in}}%
\pgfpathlineto{\pgfqpoint{4.777135in}{1.761397in}}%
\pgfpathlineto{\pgfqpoint{4.790924in}{1.763565in}}%
\pgfpathlineto{\pgfqpoint{4.804726in}{1.765844in}}%
\pgfpathlineto{\pgfqpoint{4.818539in}{1.768236in}}%
\pgfpathlineto{\pgfqpoint{4.826139in}{1.780483in}}%
\pgfpathlineto{\pgfqpoint{4.833736in}{1.792723in}}%
\pgfpathlineto{\pgfqpoint{4.841328in}{1.804954in}}%
\pgfpathlineto{\pgfqpoint{4.848915in}{1.817175in}}%
\pgfpathlineto{\pgfqpoint{4.835104in}{1.814550in}}%
\pgfpathlineto{\pgfqpoint{4.821305in}{1.812038in}}%
\pgfpathlineto{\pgfqpoint{4.807518in}{1.809638in}}%
\pgfpathlineto{\pgfqpoint{4.793742in}{1.807350in}}%
\pgfpathlineto{\pgfqpoint{4.786152in}{1.795356in}}%
\pgfpathlineto{\pgfqpoint{4.778558in}{1.783356in}}%
\pgfpathlineto{\pgfqpoint{4.770959in}{1.771350in}}%
\pgfpathlineto{\pgfqpoint{4.763356in}{1.759343in}}%
\pgfpathclose%
\pgfusepath{fill}%
\end{pgfscope}%
\begin{pgfscope}%
\pgfpathrectangle{\pgfqpoint{1.254980in}{0.150000in}}{\pgfqpoint{5.490039in}{5.490039in}}%
\pgfusepath{clip}%
\pgfsetbuttcap%
\pgfsetroundjoin%
\definecolor{currentfill}{rgb}{0.282290,0.145912,0.461510}%
\pgfsetfillcolor{currentfill}%
\pgfsetfillopacity{0.700000}%
\pgfsetlinewidth{0.000000pt}%
\definecolor{currentstroke}{rgb}{0.000000,0.000000,0.000000}%
\pgfsetstrokecolor{currentstroke}%
\pgfsetdash{}{0pt}%
\pgfpathmoveto{\pgfqpoint{3.305125in}{1.760267in}}%
\pgfpathlineto{\pgfqpoint{3.318610in}{1.748574in}}%
\pgfpathlineto{\pgfqpoint{3.332095in}{1.737019in}}%
\pgfpathlineto{\pgfqpoint{3.345580in}{1.725600in}}%
\pgfpathlineto{\pgfqpoint{3.359066in}{1.714318in}}%
\pgfpathlineto{\pgfqpoint{3.367254in}{1.715520in}}%
\pgfpathlineto{\pgfqpoint{3.375430in}{1.716926in}}%
\pgfpathlineto{\pgfqpoint{3.383595in}{1.718530in}}%
\pgfpathlineto{\pgfqpoint{3.391749in}{1.720331in}}%
\pgfpathlineto{\pgfqpoint{3.378294in}{1.731181in}}%
\pgfpathlineto{\pgfqpoint{3.364839in}{1.742167in}}%
\pgfpathlineto{\pgfqpoint{3.351385in}{1.753289in}}%
\pgfpathlineto{\pgfqpoint{3.337931in}{1.764548in}}%
\pgfpathlineto{\pgfqpoint{3.329747in}{1.763174in}}%
\pgfpathlineto{\pgfqpoint{3.321551in}{1.762000in}}%
\pgfpathlineto{\pgfqpoint{3.313344in}{1.761030in}}%
\pgfpathlineto{\pgfqpoint{3.305125in}{1.760267in}}%
\pgfpathclose%
\pgfusepath{fill}%
\end{pgfscope}%
\begin{pgfscope}%
\pgfpathrectangle{\pgfqpoint{1.254980in}{0.150000in}}{\pgfqpoint{5.490039in}{5.490039in}}%
\pgfusepath{clip}%
\pgfsetbuttcap%
\pgfsetroundjoin%
\definecolor{currentfill}{rgb}{0.272594,0.025563,0.353093}%
\pgfsetfillcolor{currentfill}%
\pgfsetfillopacity{0.700000}%
\pgfsetlinewidth{0.000000pt}%
\definecolor{currentstroke}{rgb}{0.000000,0.000000,0.000000}%
\pgfsetstrokecolor{currentstroke}%
\pgfsetdash{}{0pt}%
\pgfpathmoveto{\pgfqpoint{4.251179in}{1.501524in}}%
\pgfpathlineto{\pgfqpoint{4.264765in}{1.499125in}}%
\pgfpathlineto{\pgfqpoint{4.278359in}{1.496843in}}%
\pgfpathlineto{\pgfqpoint{4.291960in}{1.494675in}}%
\pgfpathlineto{\pgfqpoint{4.305570in}{1.492622in}}%
\pgfpathlineto{\pgfqpoint{4.313310in}{1.502419in}}%
\pgfpathlineto{\pgfqpoint{4.321045in}{1.512286in}}%
\pgfpathlineto{\pgfqpoint{4.328775in}{1.522219in}}%
\pgfpathlineto{\pgfqpoint{4.336501in}{1.532217in}}%
\pgfpathlineto{\pgfqpoint{4.322900in}{1.533943in}}%
\pgfpathlineto{\pgfqpoint{4.309308in}{1.535785in}}%
\pgfpathlineto{\pgfqpoint{4.295723in}{1.537743in}}%
\pgfpathlineto{\pgfqpoint{4.282147in}{1.539815in}}%
\pgfpathlineto{\pgfqpoint{4.274412in}{1.530137in}}%
\pgfpathlineto{\pgfqpoint{4.266673in}{1.520528in}}%
\pgfpathlineto{\pgfqpoint{4.258928in}{1.510989in}}%
\pgfpathlineto{\pgfqpoint{4.251179in}{1.501524in}}%
\pgfpathclose%
\pgfusepath{fill}%
\end{pgfscope}%
\begin{pgfscope}%
\pgfpathrectangle{\pgfqpoint{1.254980in}{0.150000in}}{\pgfqpoint{5.490039in}{5.490039in}}%
\pgfusepath{clip}%
\pgfsetbuttcap%
\pgfsetroundjoin%
\definecolor{currentfill}{rgb}{0.197636,0.391528,0.554969}%
\pgfsetfillcolor{currentfill}%
\pgfsetfillopacity{0.700000}%
\pgfsetlinewidth{0.000000pt}%
\definecolor{currentstroke}{rgb}{0.000000,0.000000,0.000000}%
\pgfsetstrokecolor{currentstroke}%
\pgfsetdash{}{0pt}%
\pgfpathmoveto{\pgfqpoint{5.338059in}{2.249127in}}%
\pgfpathlineto{\pgfqpoint{5.352139in}{2.255298in}}%
\pgfpathlineto{\pgfqpoint{5.366234in}{2.261581in}}%
\pgfpathlineto{\pgfqpoint{5.380344in}{2.267976in}}%
\pgfpathlineto{\pgfqpoint{5.394469in}{2.274485in}}%
\pgfpathlineto{\pgfqpoint{5.401899in}{2.286484in}}%
\pgfpathlineto{\pgfqpoint{5.409323in}{2.298413in}}%
\pgfpathlineto{\pgfqpoint{5.416741in}{2.310271in}}%
\pgfpathlineto{\pgfqpoint{5.424153in}{2.322059in}}%
\pgfpathlineto{\pgfqpoint{5.410029in}{2.315446in}}%
\pgfpathlineto{\pgfqpoint{5.395921in}{2.308946in}}%
\pgfpathlineto{\pgfqpoint{5.381827in}{2.302559in}}%
\pgfpathlineto{\pgfqpoint{5.367749in}{2.296284in}}%
\pgfpathlineto{\pgfqpoint{5.360335in}{2.284595in}}%
\pgfpathlineto{\pgfqpoint{5.352916in}{2.272838in}}%
\pgfpathlineto{\pgfqpoint{5.345490in}{2.261016in}}%
\pgfpathlineto{\pgfqpoint{5.338059in}{2.249127in}}%
\pgfpathclose%
\pgfusepath{fill}%
\end{pgfscope}%
\begin{pgfscope}%
\pgfpathrectangle{\pgfqpoint{1.254980in}{0.150000in}}{\pgfqpoint{5.490039in}{5.490039in}}%
\pgfusepath{clip}%
\pgfsetbuttcap%
\pgfsetroundjoin%
\definecolor{currentfill}{rgb}{0.136408,0.541173,0.554483}%
\pgfsetfillcolor{currentfill}%
\pgfsetfillopacity{0.700000}%
\pgfsetlinewidth{0.000000pt}%
\definecolor{currentstroke}{rgb}{0.000000,0.000000,0.000000}%
\pgfsetstrokecolor{currentstroke}%
\pgfsetdash{}{0pt}%
\pgfpathmoveto{\pgfqpoint{2.491163in}{2.740376in}}%
\pgfpathlineto{\pgfqpoint{2.504893in}{2.719037in}}%
\pgfpathlineto{\pgfqpoint{2.518615in}{2.697888in}}%
\pgfpathlineto{\pgfqpoint{2.532328in}{2.676928in}}%
\pgfpathlineto{\pgfqpoint{2.546032in}{2.656155in}}%
\pgfpathlineto{\pgfqpoint{2.554829in}{2.650523in}}%
\pgfpathlineto{\pgfqpoint{2.563606in}{2.645167in}}%
\pgfpathlineto{\pgfqpoint{2.572364in}{2.640083in}}%
\pgfpathlineto{\pgfqpoint{2.581104in}{2.635268in}}%
\pgfpathlineto{\pgfqpoint{2.567450in}{2.655570in}}%
\pgfpathlineto{\pgfqpoint{2.553788in}{2.676058in}}%
\pgfpathlineto{\pgfqpoint{2.540118in}{2.696735in}}%
\pgfpathlineto{\pgfqpoint{2.526439in}{2.717600in}}%
\pgfpathlineto{\pgfqpoint{2.517650in}{2.722879in}}%
\pgfpathlineto{\pgfqpoint{2.508840in}{2.728433in}}%
\pgfpathlineto{\pgfqpoint{2.500012in}{2.734264in}}%
\pgfpathlineto{\pgfqpoint{2.491163in}{2.740376in}}%
\pgfpathclose%
\pgfusepath{fill}%
\end{pgfscope}%
\begin{pgfscope}%
\pgfpathrectangle{\pgfqpoint{1.254980in}{0.150000in}}{\pgfqpoint{5.490039in}{5.490039in}}%
\pgfusepath{clip}%
\pgfsetbuttcap%
\pgfsetroundjoin%
\definecolor{currentfill}{rgb}{0.274128,0.199721,0.498911}%
\pgfsetfillcolor{currentfill}%
\pgfsetfillopacity{0.700000}%
\pgfsetlinewidth{0.000000pt}%
\definecolor{currentstroke}{rgb}{0.000000,0.000000,0.000000}%
\pgfsetstrokecolor{currentstroke}%
\pgfsetdash{}{0pt}%
\pgfpathmoveto{\pgfqpoint{4.848915in}{1.817175in}}%
\pgfpathlineto{\pgfqpoint{4.862737in}{1.819913in}}%
\pgfpathlineto{\pgfqpoint{4.876572in}{1.822763in}}%
\pgfpathlineto{\pgfqpoint{4.890418in}{1.825725in}}%
\pgfpathlineto{\pgfqpoint{4.904276in}{1.828799in}}%
\pgfpathlineto{\pgfqpoint{4.911857in}{1.841231in}}%
\pgfpathlineto{\pgfqpoint{4.919434in}{1.853646in}}%
\pgfpathlineto{\pgfqpoint{4.927006in}{1.866042in}}%
\pgfpathlineto{\pgfqpoint{4.934573in}{1.878416in}}%
\pgfpathlineto{\pgfqpoint{4.920716in}{1.875124in}}%
\pgfpathlineto{\pgfqpoint{4.906871in}{1.871945in}}%
\pgfpathlineto{\pgfqpoint{4.893039in}{1.868878in}}%
\pgfpathlineto{\pgfqpoint{4.879218in}{1.865924in}}%
\pgfpathlineto{\pgfqpoint{4.871649in}{1.853760in}}%
\pgfpathlineto{\pgfqpoint{4.864076in}{1.841580in}}%
\pgfpathlineto{\pgfqpoint{4.856498in}{1.829384in}}%
\pgfpathlineto{\pgfqpoint{4.848915in}{1.817175in}}%
\pgfpathclose%
\pgfusepath{fill}%
\end{pgfscope}%
\begin{pgfscope}%
\pgfpathrectangle{\pgfqpoint{1.254980in}{0.150000in}}{\pgfqpoint{5.490039in}{5.490039in}}%
\pgfusepath{clip}%
\pgfsetbuttcap%
\pgfsetroundjoin%
\definecolor{currentfill}{rgb}{0.233603,0.313828,0.543914}%
\pgfsetfillcolor{currentfill}%
\pgfsetfillopacity{0.700000}%
\pgfsetlinewidth{0.000000pt}%
\definecolor{currentstroke}{rgb}{0.000000,0.000000,0.000000}%
\pgfsetstrokecolor{currentstroke}%
\pgfsetdash{}{0pt}%
\pgfpathmoveto{\pgfqpoint{5.136285in}{2.059596in}}%
\pgfpathlineto{\pgfqpoint{5.150255in}{2.064455in}}%
\pgfpathlineto{\pgfqpoint{5.164238in}{2.069427in}}%
\pgfpathlineto{\pgfqpoint{5.178236in}{2.074511in}}%
\pgfpathlineto{\pgfqpoint{5.192247in}{2.079707in}}%
\pgfpathlineto{\pgfqpoint{5.199747in}{2.092163in}}%
\pgfpathlineto{\pgfqpoint{5.207242in}{2.104567in}}%
\pgfpathlineto{\pgfqpoint{5.214731in}{2.116918in}}%
\pgfpathlineto{\pgfqpoint{5.222214in}{2.129217in}}%
\pgfpathlineto{\pgfqpoint{5.208204in}{2.123867in}}%
\pgfpathlineto{\pgfqpoint{5.194207in}{2.118629in}}%
\pgfpathlineto{\pgfqpoint{5.180225in}{2.113504in}}%
\pgfpathlineto{\pgfqpoint{5.166256in}{2.108491in}}%
\pgfpathlineto{\pgfqpoint{5.158771in}{2.096340in}}%
\pgfpathlineto{\pgfqpoint{5.151281in}{2.084140in}}%
\pgfpathlineto{\pgfqpoint{5.143785in}{2.071891in}}%
\pgfpathlineto{\pgfqpoint{5.136285in}{2.059596in}}%
\pgfpathclose%
\pgfusepath{fill}%
\end{pgfscope}%
\begin{pgfscope}%
\pgfpathrectangle{\pgfqpoint{1.254980in}{0.150000in}}{\pgfqpoint{5.490039in}{5.490039in}}%
\pgfusepath{clip}%
\pgfsetbuttcap%
\pgfsetroundjoin%
\definecolor{currentfill}{rgb}{0.175707,0.697900,0.491033}%
\pgfsetfillcolor{currentfill}%
\pgfsetfillopacity{0.700000}%
\pgfsetlinewidth{0.000000pt}%
\definecolor{currentstroke}{rgb}{0.000000,0.000000,0.000000}%
\pgfsetstrokecolor{currentstroke}%
\pgfsetdash{}{0pt}%
\pgfpathmoveto{\pgfqpoint{2.250837in}{3.176595in}}%
\pgfpathlineto{\pgfqpoint{2.264722in}{3.151670in}}%
\pgfpathlineto{\pgfqpoint{2.278594in}{3.126961in}}%
\pgfpathlineto{\pgfqpoint{2.292454in}{3.102468in}}%
\pgfpathlineto{\pgfqpoint{2.306303in}{3.078188in}}%
\pgfpathlineto{\pgfqpoint{2.315287in}{3.071300in}}%
\pgfpathlineto{\pgfqpoint{2.324249in}{3.064698in}}%
\pgfpathlineto{\pgfqpoint{2.333191in}{3.058380in}}%
\pgfpathlineto{\pgfqpoint{2.342112in}{3.052340in}}%
\pgfpathlineto{\pgfqpoint{2.328319in}{3.076149in}}%
\pgfpathlineto{\pgfqpoint{2.314515in}{3.100170in}}%
\pgfpathlineto{\pgfqpoint{2.300699in}{3.124405in}}%
\pgfpathlineto{\pgfqpoint{2.286871in}{3.148856in}}%
\pgfpathlineto{\pgfqpoint{2.277895in}{3.155360in}}%
\pgfpathlineto{\pgfqpoint{2.268898in}{3.162149in}}%
\pgfpathlineto{\pgfqpoint{2.259878in}{3.169226in}}%
\pgfpathlineto{\pgfqpoint{2.250837in}{3.176595in}}%
\pgfpathclose%
\pgfusepath{fill}%
\end{pgfscope}%
\begin{pgfscope}%
\pgfpathrectangle{\pgfqpoint{1.254980in}{0.150000in}}{\pgfqpoint{5.490039in}{5.490039in}}%
\pgfusepath{clip}%
\pgfsetbuttcap%
\pgfsetroundjoin%
\definecolor{currentfill}{rgb}{0.283187,0.125848,0.444960}%
\pgfsetfillcolor{currentfill}%
\pgfsetfillopacity{0.700000}%
\pgfsetlinewidth{0.000000pt}%
\definecolor{currentstroke}{rgb}{0.000000,0.000000,0.000000}%
\pgfsetstrokecolor{currentstroke}%
\pgfsetdash{}{0pt}%
\pgfpathmoveto{\pgfqpoint{3.359066in}{1.714318in}}%
\pgfpathlineto{\pgfqpoint{3.372552in}{1.703172in}}%
\pgfpathlineto{\pgfqpoint{3.386039in}{1.692160in}}%
\pgfpathlineto{\pgfqpoint{3.399526in}{1.681284in}}%
\pgfpathlineto{\pgfqpoint{3.413014in}{1.670541in}}%
\pgfpathlineto{\pgfqpoint{3.421171in}{1.672182in}}%
\pgfpathlineto{\pgfqpoint{3.429318in}{1.674021in}}%
\pgfpathlineto{\pgfqpoint{3.437454in}{1.676055in}}%
\pgfpathlineto{\pgfqpoint{3.445579in}{1.678280in}}%
\pgfpathlineto{\pgfqpoint{3.432120in}{1.688592in}}%
\pgfpathlineto{\pgfqpoint{3.418662in}{1.699037in}}%
\pgfpathlineto{\pgfqpoint{3.405205in}{1.709617in}}%
\pgfpathlineto{\pgfqpoint{3.391749in}{1.720331in}}%
\pgfpathlineto{\pgfqpoint{3.383595in}{1.718530in}}%
\pgfpathlineto{\pgfqpoint{3.375430in}{1.716926in}}%
\pgfpathlineto{\pgfqpoint{3.367254in}{1.715520in}}%
\pgfpathlineto{\pgfqpoint{3.359066in}{1.714318in}}%
\pgfpathclose%
\pgfusepath{fill}%
\end{pgfscope}%
\begin{pgfscope}%
\pgfpathrectangle{\pgfqpoint{1.254980in}{0.150000in}}{\pgfqpoint{5.490039in}{5.490039in}}%
\pgfusepath{clip}%
\pgfsetbuttcap%
\pgfsetroundjoin%
\definecolor{currentfill}{rgb}{0.268510,0.009605,0.335427}%
\pgfsetfillcolor{currentfill}%
\pgfsetfillopacity{0.700000}%
\pgfsetlinewidth{0.000000pt}%
\definecolor{currentstroke}{rgb}{0.000000,0.000000,0.000000}%
\pgfsetstrokecolor{currentstroke}%
\pgfsetdash{}{0pt}%
\pgfpathmoveto{\pgfqpoint{4.165816in}{1.476569in}}%
\pgfpathlineto{\pgfqpoint{4.179383in}{1.473365in}}%
\pgfpathlineto{\pgfqpoint{4.192958in}{1.470278in}}%
\pgfpathlineto{\pgfqpoint{4.206539in}{1.467306in}}%
\pgfpathlineto{\pgfqpoint{4.220128in}{1.464451in}}%
\pgfpathlineto{\pgfqpoint{4.227899in}{1.473595in}}%
\pgfpathlineto{\pgfqpoint{4.235664in}{1.482824in}}%
\pgfpathlineto{\pgfqpoint{4.243424in}{1.492135in}}%
\pgfpathlineto{\pgfqpoint{4.251179in}{1.501524in}}%
\pgfpathlineto{\pgfqpoint{4.237600in}{1.504038in}}%
\pgfpathlineto{\pgfqpoint{4.224029in}{1.506668in}}%
\pgfpathlineto{\pgfqpoint{4.210466in}{1.509414in}}%
\pgfpathlineto{\pgfqpoint{4.196909in}{1.512276in}}%
\pgfpathlineto{\pgfqpoint{4.189144in}{1.503223in}}%
\pgfpathlineto{\pgfqpoint{4.181373in}{1.494252in}}%
\pgfpathlineto{\pgfqpoint{4.173597in}{1.485366in}}%
\pgfpathlineto{\pgfqpoint{4.165816in}{1.476569in}}%
\pgfpathclose%
\pgfusepath{fill}%
\end{pgfscope}%
\begin{pgfscope}%
\pgfpathrectangle{\pgfqpoint{1.254980in}{0.150000in}}{\pgfqpoint{5.490039in}{5.490039in}}%
\pgfusepath{clip}%
\pgfsetbuttcap%
\pgfsetroundjoin%
\definecolor{currentfill}{rgb}{0.277018,0.050344,0.375715}%
\pgfsetfillcolor{currentfill}%
\pgfsetfillopacity{0.700000}%
\pgfsetlinewidth{0.000000pt}%
\definecolor{currentstroke}{rgb}{0.000000,0.000000,0.000000}%
\pgfsetstrokecolor{currentstroke}%
\pgfsetdash{}{0pt}%
\pgfpathmoveto{\pgfqpoint{3.607211in}{1.564797in}}%
\pgfpathlineto{\pgfqpoint{3.620694in}{1.556180in}}%
\pgfpathlineto{\pgfqpoint{3.634180in}{1.547691in}}%
\pgfpathlineto{\pgfqpoint{3.647669in}{1.539328in}}%
\pgfpathlineto{\pgfqpoint{3.661160in}{1.531091in}}%
\pgfpathlineto{\pgfqpoint{3.669173in}{1.535175in}}%
\pgfpathlineto{\pgfqpoint{3.677176in}{1.539425in}}%
\pgfpathlineto{\pgfqpoint{3.685172in}{1.543839in}}%
\pgfpathlineto{\pgfqpoint{3.693158in}{1.548413in}}%
\pgfpathlineto{\pgfqpoint{3.679690in}{1.556240in}}%
\pgfpathlineto{\pgfqpoint{3.666224in}{1.564194in}}%
\pgfpathlineto{\pgfqpoint{3.652762in}{1.572274in}}%
\pgfpathlineto{\pgfqpoint{3.639302in}{1.580481in}}%
\pgfpathlineto{\pgfqpoint{3.631293in}{1.576310in}}%
\pgfpathlineto{\pgfqpoint{3.623275in}{1.572304in}}%
\pgfpathlineto{\pgfqpoint{3.615247in}{1.568465in}}%
\pgfpathlineto{\pgfqpoint{3.607211in}{1.564797in}}%
\pgfpathclose%
\pgfusepath{fill}%
\end{pgfscope}%
\begin{pgfscope}%
\pgfpathrectangle{\pgfqpoint{1.254980in}{0.150000in}}{\pgfqpoint{5.490039in}{5.490039in}}%
\pgfusepath{clip}%
\pgfsetbuttcap%
\pgfsetroundjoin%
\definecolor{currentfill}{rgb}{0.265145,0.232956,0.516599}%
\pgfsetfillcolor{currentfill}%
\pgfsetfillopacity{0.700000}%
\pgfsetlinewidth{0.000000pt}%
\definecolor{currentstroke}{rgb}{0.000000,0.000000,0.000000}%
\pgfsetstrokecolor{currentstroke}%
\pgfsetdash{}{0pt}%
\pgfpathmoveto{\pgfqpoint{4.934573in}{1.878416in}}%
\pgfpathlineto{\pgfqpoint{4.948442in}{1.881820in}}%
\pgfpathlineto{\pgfqpoint{4.962324in}{1.885337in}}%
\pgfpathlineto{\pgfqpoint{4.976218in}{1.888966in}}%
\pgfpathlineto{\pgfqpoint{4.990125in}{1.892706in}}%
\pgfpathlineto{\pgfqpoint{4.997686in}{1.905266in}}%
\pgfpathlineto{\pgfqpoint{5.005243in}{1.917799in}}%
\pgfpathlineto{\pgfqpoint{5.012794in}{1.930301in}}%
\pgfpathlineto{\pgfqpoint{5.020341in}{1.942773in}}%
\pgfpathlineto{\pgfqpoint{5.006436in}{1.938831in}}%
\pgfpathlineto{\pgfqpoint{4.992542in}{1.935001in}}%
\pgfpathlineto{\pgfqpoint{4.978662in}{1.931283in}}%
\pgfpathlineto{\pgfqpoint{4.964794in}{1.927677in}}%
\pgfpathlineto{\pgfqpoint{4.957246in}{1.915401in}}%
\pgfpathlineto{\pgfqpoint{4.949693in}{1.903097in}}%
\pgfpathlineto{\pgfqpoint{4.942135in}{1.890769in}}%
\pgfpathlineto{\pgfqpoint{4.934573in}{1.878416in}}%
\pgfpathclose%
\pgfusepath{fill}%
\end{pgfscope}%
\begin{pgfscope}%
\pgfpathrectangle{\pgfqpoint{1.254980in}{0.150000in}}{\pgfqpoint{5.490039in}{5.490039in}}%
\pgfusepath{clip}%
\pgfsetbuttcap%
\pgfsetroundjoin%
\definecolor{currentfill}{rgb}{0.267004,0.004874,0.329415}%
\pgfsetfillcolor{currentfill}%
\pgfsetfillopacity{0.700000}%
\pgfsetlinewidth{0.000000pt}%
\definecolor{currentstroke}{rgb}{0.000000,0.000000,0.000000}%
\pgfsetstrokecolor{currentstroke}%
\pgfsetdash{}{0pt}%
\pgfpathmoveto{\pgfqpoint{3.940728in}{1.466098in}}%
\pgfpathlineto{\pgfqpoint{3.954249in}{1.460747in}}%
\pgfpathlineto{\pgfqpoint{3.967776in}{1.455515in}}%
\pgfpathlineto{\pgfqpoint{3.981308in}{1.450403in}}%
\pgfpathlineto{\pgfqpoint{3.994846in}{1.445410in}}%
\pgfpathlineto{\pgfqpoint{4.002702in}{1.452638in}}%
\pgfpathlineto{\pgfqpoint{4.010551in}{1.459985in}}%
\pgfpathlineto{\pgfqpoint{4.018395in}{1.467450in}}%
\pgfpathlineto{\pgfqpoint{4.026232in}{1.475028in}}%
\pgfpathlineto{\pgfqpoint{4.012709in}{1.479647in}}%
\pgfpathlineto{\pgfqpoint{3.999192in}{1.484385in}}%
\pgfpathlineto{\pgfqpoint{3.985681in}{1.489243in}}%
\pgfpathlineto{\pgfqpoint{3.972175in}{1.494220in}}%
\pgfpathlineto{\pgfqpoint{3.964323in}{1.487010in}}%
\pgfpathlineto{\pgfqpoint{3.956465in}{1.479918in}}%
\pgfpathlineto{\pgfqpoint{3.948599in}{1.472946in}}%
\pgfpathlineto{\pgfqpoint{3.940728in}{1.466098in}}%
\pgfpathclose%
\pgfusepath{fill}%
\end{pgfscope}%
\begin{pgfscope}%
\pgfpathrectangle{\pgfqpoint{1.254980in}{0.150000in}}{\pgfqpoint{5.490039in}{5.490039in}}%
\pgfusepath{clip}%
\pgfsetbuttcap%
\pgfsetroundjoin%
\definecolor{currentfill}{rgb}{0.125394,0.574318,0.549086}%
\pgfsetfillcolor{currentfill}%
\pgfsetfillopacity{0.700000}%
\pgfsetlinewidth{0.000000pt}%
\definecolor{currentstroke}{rgb}{0.000000,0.000000,0.000000}%
\pgfsetstrokecolor{currentstroke}%
\pgfsetdash{}{0pt}%
\pgfpathmoveto{\pgfqpoint{2.436153in}{2.827656in}}%
\pgfpathlineto{\pgfqpoint{2.449919in}{2.805545in}}%
\pgfpathlineto{\pgfqpoint{2.463677in}{2.783629in}}%
\pgfpathlineto{\pgfqpoint{2.477424in}{2.761906in}}%
\pgfpathlineto{\pgfqpoint{2.491163in}{2.740376in}}%
\pgfpathlineto{\pgfqpoint{2.500012in}{2.734264in}}%
\pgfpathlineto{\pgfqpoint{2.508840in}{2.728433in}}%
\pgfpathlineto{\pgfqpoint{2.517650in}{2.722879in}}%
\pgfpathlineto{\pgfqpoint{2.526439in}{2.717600in}}%
\pgfpathlineto{\pgfqpoint{2.512752in}{2.738655in}}%
\pgfpathlineto{\pgfqpoint{2.499057in}{2.759901in}}%
\pgfpathlineto{\pgfqpoint{2.485352in}{2.781340in}}%
\pgfpathlineto{\pgfqpoint{2.471639in}{2.802973in}}%
\pgfpathlineto{\pgfqpoint{2.462797in}{2.808722in}}%
\pgfpathlineto{\pgfqpoint{2.453936in}{2.814749in}}%
\pgfpathlineto{\pgfqpoint{2.445055in}{2.821059in}}%
\pgfpathlineto{\pgfqpoint{2.436153in}{2.827656in}}%
\pgfpathclose%
\pgfusepath{fill}%
\end{pgfscope}%
\begin{pgfscope}%
\pgfpathrectangle{\pgfqpoint{1.254980in}{0.150000in}}{\pgfqpoint{5.490039in}{5.490039in}}%
\pgfusepath{clip}%
\pgfsetbuttcap%
\pgfsetroundjoin%
\definecolor{currentfill}{rgb}{0.269944,0.014625,0.341379}%
\pgfsetfillcolor{currentfill}%
\pgfsetfillopacity{0.700000}%
\pgfsetlinewidth{0.000000pt}%
\definecolor{currentstroke}{rgb}{0.000000,0.000000,0.000000}%
\pgfsetstrokecolor{currentstroke}%
\pgfsetdash{}{0pt}%
\pgfpathmoveto{\pgfqpoint{3.801033in}{1.490286in}}%
\pgfpathlineto{\pgfqpoint{3.814535in}{1.483575in}}%
\pgfpathlineto{\pgfqpoint{3.828041in}{1.476988in}}%
\pgfpathlineto{\pgfqpoint{3.841551in}{1.470522in}}%
\pgfpathlineto{\pgfqpoint{3.855066in}{1.464177in}}%
\pgfpathlineto{\pgfqpoint{3.862984in}{1.470101in}}%
\pgfpathlineto{\pgfqpoint{3.870894in}{1.476166in}}%
\pgfpathlineto{\pgfqpoint{3.878797in}{1.482368in}}%
\pgfpathlineto{\pgfqpoint{3.886693in}{1.488704in}}%
\pgfpathlineto{\pgfqpoint{3.873196in}{1.494658in}}%
\pgfpathlineto{\pgfqpoint{3.859704in}{1.500733in}}%
\pgfpathlineto{\pgfqpoint{3.846217in}{1.506930in}}%
\pgfpathlineto{\pgfqpoint{3.832735in}{1.513249in}}%
\pgfpathlineto{\pgfqpoint{3.824821in}{1.507297in}}%
\pgfpathlineto{\pgfqpoint{3.816899in}{1.501484in}}%
\pgfpathlineto{\pgfqpoint{3.808970in}{1.495812in}}%
\pgfpathlineto{\pgfqpoint{3.801033in}{1.490286in}}%
\pgfpathclose%
\pgfusepath{fill}%
\end{pgfscope}%
\begin{pgfscope}%
\pgfpathrectangle{\pgfqpoint{1.254980in}{0.150000in}}{\pgfqpoint{5.490039in}{5.490039in}}%
\pgfusepath{clip}%
\pgfsetbuttcap%
\pgfsetroundjoin%
\definecolor{currentfill}{rgb}{0.183898,0.422383,0.556944}%
\pgfsetfillcolor{currentfill}%
\pgfsetfillopacity{0.700000}%
\pgfsetlinewidth{0.000000pt}%
\definecolor{currentstroke}{rgb}{0.000000,0.000000,0.000000}%
\pgfsetstrokecolor{currentstroke}%
\pgfsetdash{}{0pt}%
\pgfpathmoveto{\pgfqpoint{5.424153in}{2.322059in}}%
\pgfpathlineto{\pgfqpoint{5.438291in}{2.328784in}}%
\pgfpathlineto{\pgfqpoint{5.452446in}{2.335622in}}%
\pgfpathlineto{\pgfqpoint{5.466615in}{2.342573in}}%
\pgfpathlineto{\pgfqpoint{5.480800in}{2.349637in}}%
\pgfpathlineto{\pgfqpoint{5.488204in}{2.361447in}}%
\pgfpathlineto{\pgfqpoint{5.495601in}{2.373180in}}%
\pgfpathlineto{\pgfqpoint{5.502992in}{2.384837in}}%
\pgfpathlineto{\pgfqpoint{5.510377in}{2.396416in}}%
\pgfpathlineto{\pgfqpoint{5.496193in}{2.389265in}}%
\pgfpathlineto{\pgfqpoint{5.482026in}{2.382226in}}%
\pgfpathlineto{\pgfqpoint{5.467874in}{2.375301in}}%
\pgfpathlineto{\pgfqpoint{5.453737in}{2.368488in}}%
\pgfpathlineto{\pgfqpoint{5.446350in}{2.356990in}}%
\pgfpathlineto{\pgfqpoint{5.438957in}{2.345419in}}%
\pgfpathlineto{\pgfqpoint{5.431558in}{2.333775in}}%
\pgfpathlineto{\pgfqpoint{5.424153in}{2.322059in}}%
\pgfpathclose%
\pgfusepath{fill}%
\end{pgfscope}%
\begin{pgfscope}%
\pgfpathrectangle{\pgfqpoint{1.254980in}{0.150000in}}{\pgfqpoint{5.490039in}{5.490039in}}%
\pgfusepath{clip}%
\pgfsetbuttcap%
\pgfsetroundjoin%
\definecolor{currentfill}{rgb}{0.218130,0.347432,0.550038}%
\pgfsetfillcolor{currentfill}%
\pgfsetfillopacity{0.700000}%
\pgfsetlinewidth{0.000000pt}%
\definecolor{currentstroke}{rgb}{0.000000,0.000000,0.000000}%
\pgfsetstrokecolor{currentstroke}%
\pgfsetdash{}{0pt}%
\pgfpathmoveto{\pgfqpoint{5.222214in}{2.129217in}}%
\pgfpathlineto{\pgfqpoint{5.236239in}{2.134679in}}%
\pgfpathlineto{\pgfqpoint{5.250278in}{2.140254in}}%
\pgfpathlineto{\pgfqpoint{5.264331in}{2.145941in}}%
\pgfpathlineto{\pgfqpoint{5.278399in}{2.151741in}}%
\pgfpathlineto{\pgfqpoint{5.285876in}{2.164128in}}%
\pgfpathlineto{\pgfqpoint{5.293348in}{2.176456in}}%
\pgfpathlineto{\pgfqpoint{5.300814in}{2.188724in}}%
\pgfpathlineto{\pgfqpoint{5.308275in}{2.200930in}}%
\pgfpathlineto{\pgfqpoint{5.294208in}{2.194993in}}%
\pgfpathlineto{\pgfqpoint{5.280156in}{2.189168in}}%
\pgfpathlineto{\pgfqpoint{5.266118in}{2.183456in}}%
\pgfpathlineto{\pgfqpoint{5.252094in}{2.177857in}}%
\pgfpathlineto{\pgfqpoint{5.244633in}{2.165781in}}%
\pgfpathlineto{\pgfqpoint{5.237165in}{2.153649in}}%
\pgfpathlineto{\pgfqpoint{5.229693in}{2.141461in}}%
\pgfpathlineto{\pgfqpoint{5.222214in}{2.129217in}}%
\pgfpathclose%
\pgfusepath{fill}%
\end{pgfscope}%
\begin{pgfscope}%
\pgfpathrectangle{\pgfqpoint{1.254980in}{0.150000in}}{\pgfqpoint{5.490039in}{5.490039in}}%
\pgfusepath{clip}%
\pgfsetbuttcap%
\pgfsetroundjoin%
\definecolor{currentfill}{rgb}{0.283091,0.110553,0.431554}%
\pgfsetfillcolor{currentfill}%
\pgfsetfillopacity{0.700000}%
\pgfsetlinewidth{0.000000pt}%
\definecolor{currentstroke}{rgb}{0.000000,0.000000,0.000000}%
\pgfsetstrokecolor{currentstroke}%
\pgfsetdash{}{0pt}%
\pgfpathmoveto{\pgfqpoint{3.413014in}{1.670541in}}%
\pgfpathlineto{\pgfqpoint{3.426503in}{1.659933in}}%
\pgfpathlineto{\pgfqpoint{3.439993in}{1.649457in}}%
\pgfpathlineto{\pgfqpoint{3.453484in}{1.639114in}}%
\pgfpathlineto{\pgfqpoint{3.466976in}{1.628902in}}%
\pgfpathlineto{\pgfqpoint{3.475104in}{1.630980in}}%
\pgfpathlineto{\pgfqpoint{3.483222in}{1.633251in}}%
\pgfpathlineto{\pgfqpoint{3.491330in}{1.635713in}}%
\pgfpathlineto{\pgfqpoint{3.499428in}{1.638363in}}%
\pgfpathlineto{\pgfqpoint{3.485963in}{1.648144in}}%
\pgfpathlineto{\pgfqpoint{3.472500in}{1.658057in}}%
\pgfpathlineto{\pgfqpoint{3.459039in}{1.668102in}}%
\pgfpathlineto{\pgfqpoint{3.445579in}{1.678280in}}%
\pgfpathlineto{\pgfqpoint{3.437454in}{1.676055in}}%
\pgfpathlineto{\pgfqpoint{3.429318in}{1.674021in}}%
\pgfpathlineto{\pgfqpoint{3.421171in}{1.672182in}}%
\pgfpathlineto{\pgfqpoint{3.413014in}{1.670541in}}%
\pgfpathclose%
\pgfusepath{fill}%
\end{pgfscope}%
\begin{pgfscope}%
\pgfpathrectangle{\pgfqpoint{1.254980in}{0.150000in}}{\pgfqpoint{5.490039in}{5.490039in}}%
\pgfusepath{clip}%
\pgfsetbuttcap%
\pgfsetroundjoin%
\definecolor{currentfill}{rgb}{0.280267,0.073417,0.397163}%
\pgfsetfillcolor{currentfill}%
\pgfsetfillopacity{0.700000}%
\pgfsetlinewidth{0.000000pt}%
\definecolor{currentstroke}{rgb}{0.000000,0.000000,0.000000}%
\pgfsetstrokecolor{currentstroke}%
\pgfsetdash{}{0pt}%
\pgfpathmoveto{\pgfqpoint{4.476406in}{1.565591in}}%
\pgfpathlineto{\pgfqpoint{4.490077in}{1.565203in}}%
\pgfpathlineto{\pgfqpoint{4.503759in}{1.564928in}}%
\pgfpathlineto{\pgfqpoint{4.517449in}{1.564767in}}%
\pgfpathlineto{\pgfqpoint{4.531150in}{1.564718in}}%
\pgfpathlineto{\pgfqpoint{4.538831in}{1.575902in}}%
\pgfpathlineto{\pgfqpoint{4.546509in}{1.587124in}}%
\pgfpathlineto{\pgfqpoint{4.554182in}{1.598381in}}%
\pgfpathlineto{\pgfqpoint{4.561850in}{1.609671in}}%
\pgfpathlineto{\pgfqpoint{4.548155in}{1.609424in}}%
\pgfpathlineto{\pgfqpoint{4.534470in}{1.609291in}}%
\pgfpathlineto{\pgfqpoint{4.520794in}{1.609270in}}%
\pgfpathlineto{\pgfqpoint{4.507128in}{1.609364in}}%
\pgfpathlineto{\pgfqpoint{4.499454in}{1.598362in}}%
\pgfpathlineto{\pgfqpoint{4.491776in}{1.587398in}}%
\pgfpathlineto{\pgfqpoint{4.484093in}{1.576474in}}%
\pgfpathlineto{\pgfqpoint{4.476406in}{1.565591in}}%
\pgfpathclose%
\pgfusepath{fill}%
\end{pgfscope}%
\begin{pgfscope}%
\pgfpathrectangle{\pgfqpoint{1.254980in}{0.150000in}}{\pgfqpoint{5.490039in}{5.490039in}}%
\pgfusepath{clip}%
\pgfsetbuttcap%
\pgfsetroundjoin%
\definecolor{currentfill}{rgb}{0.267004,0.004874,0.329415}%
\pgfsetfillcolor{currentfill}%
\pgfsetfillopacity{0.700000}%
\pgfsetlinewidth{0.000000pt}%
\definecolor{currentstroke}{rgb}{0.000000,0.000000,0.000000}%
\pgfsetstrokecolor{currentstroke}%
\pgfsetdash{}{0pt}%
\pgfpathmoveto{\pgfqpoint{4.080382in}{1.457733in}}%
\pgfpathlineto{\pgfqpoint{4.093935in}{1.453704in}}%
\pgfpathlineto{\pgfqpoint{4.107495in}{1.449792in}}%
\pgfpathlineto{\pgfqpoint{4.121061in}{1.445998in}}%
\pgfpathlineto{\pgfqpoint{4.134633in}{1.442320in}}%
\pgfpathlineto{\pgfqpoint{4.142438in}{1.450736in}}%
\pgfpathlineto{\pgfqpoint{4.150236in}{1.459251in}}%
\pgfpathlineto{\pgfqpoint{4.158029in}{1.467863in}}%
\pgfpathlineto{\pgfqpoint{4.165816in}{1.476569in}}%
\pgfpathlineto{\pgfqpoint{4.152255in}{1.479890in}}%
\pgfpathlineto{\pgfqpoint{4.138702in}{1.483327in}}%
\pgfpathlineto{\pgfqpoint{4.125155in}{1.486881in}}%
\pgfpathlineto{\pgfqpoint{4.111615in}{1.490553in}}%
\pgfpathlineto{\pgfqpoint{4.103815in}{1.482198in}}%
\pgfpathlineto{\pgfqpoint{4.096010in}{1.473941in}}%
\pgfpathlineto{\pgfqpoint{4.088199in}{1.465786in}}%
\pgfpathlineto{\pgfqpoint{4.080382in}{1.457733in}}%
\pgfpathclose%
\pgfusepath{fill}%
\end{pgfscope}%
\begin{pgfscope}%
\pgfpathrectangle{\pgfqpoint{1.254980in}{0.150000in}}{\pgfqpoint{5.490039in}{5.490039in}}%
\pgfusepath{clip}%
\pgfsetbuttcap%
\pgfsetroundjoin%
\definecolor{currentfill}{rgb}{0.282656,0.100196,0.422160}%
\pgfsetfillcolor{currentfill}%
\pgfsetfillopacity{0.700000}%
\pgfsetlinewidth{0.000000pt}%
\definecolor{currentstroke}{rgb}{0.000000,0.000000,0.000000}%
\pgfsetstrokecolor{currentstroke}%
\pgfsetdash{}{0pt}%
\pgfpathmoveto{\pgfqpoint{4.561850in}{1.609671in}}%
\pgfpathlineto{\pgfqpoint{4.575555in}{1.610031in}}%
\pgfpathlineto{\pgfqpoint{4.589270in}{1.610505in}}%
\pgfpathlineto{\pgfqpoint{4.602994in}{1.611091in}}%
\pgfpathlineto{\pgfqpoint{4.616729in}{1.611789in}}%
\pgfpathlineto{\pgfqpoint{4.624389in}{1.623396in}}%
\pgfpathlineto{\pgfqpoint{4.632044in}{1.635028in}}%
\pgfpathlineto{\pgfqpoint{4.639695in}{1.646682in}}%
\pgfpathlineto{\pgfqpoint{4.647341in}{1.658357in}}%
\pgfpathlineto{\pgfqpoint{4.633610in}{1.657379in}}%
\pgfpathlineto{\pgfqpoint{4.619890in}{1.656513in}}%
\pgfpathlineto{\pgfqpoint{4.606179in}{1.655760in}}%
\pgfpathlineto{\pgfqpoint{4.592479in}{1.655121in}}%
\pgfpathlineto{\pgfqpoint{4.584828in}{1.643719in}}%
\pgfpathlineto{\pgfqpoint{4.577173in}{1.632343in}}%
\pgfpathlineto{\pgfqpoint{4.569514in}{1.620993in}}%
\pgfpathlineto{\pgfqpoint{4.561850in}{1.609671in}}%
\pgfpathclose%
\pgfusepath{fill}%
\end{pgfscope}%
\begin{pgfscope}%
\pgfpathrectangle{\pgfqpoint{1.254980in}{0.150000in}}{\pgfqpoint{5.490039in}{5.490039in}}%
\pgfusepath{clip}%
\pgfsetbuttcap%
\pgfsetroundjoin%
\definecolor{currentfill}{rgb}{0.277018,0.050344,0.375715}%
\pgfsetfillcolor{currentfill}%
\pgfsetfillopacity{0.700000}%
\pgfsetlinewidth{0.000000pt}%
\definecolor{currentstroke}{rgb}{0.000000,0.000000,0.000000}%
\pgfsetstrokecolor{currentstroke}%
\pgfsetdash{}{0pt}%
\pgfpathmoveto{\pgfqpoint{4.390987in}{1.526457in}}%
\pgfpathlineto{\pgfqpoint{4.404629in}{1.525303in}}%
\pgfpathlineto{\pgfqpoint{4.418281in}{1.524263in}}%
\pgfpathlineto{\pgfqpoint{4.431941in}{1.523337in}}%
\pgfpathlineto{\pgfqpoint{4.445610in}{1.522525in}}%
\pgfpathlineto{\pgfqpoint{4.453316in}{1.533217in}}%
\pgfpathlineto{\pgfqpoint{4.461017in}{1.543960in}}%
\pgfpathlineto{\pgfqpoint{4.468714in}{1.554752in}}%
\pgfpathlineto{\pgfqpoint{4.476406in}{1.565591in}}%
\pgfpathlineto{\pgfqpoint{4.462743in}{1.566093in}}%
\pgfpathlineto{\pgfqpoint{4.449089in}{1.566708in}}%
\pgfpathlineto{\pgfqpoint{4.435445in}{1.567438in}}%
\pgfpathlineto{\pgfqpoint{4.421809in}{1.568281in}}%
\pgfpathlineto{\pgfqpoint{4.414111in}{1.557747in}}%
\pgfpathlineto{\pgfqpoint{4.406407in}{1.547264in}}%
\pgfpathlineto{\pgfqpoint{4.398699in}{1.536833in}}%
\pgfpathlineto{\pgfqpoint{4.390987in}{1.526457in}}%
\pgfpathclose%
\pgfusepath{fill}%
\end{pgfscope}%
\begin{pgfscope}%
\pgfpathrectangle{\pgfqpoint{1.254980in}{0.150000in}}{\pgfqpoint{5.490039in}{5.490039in}}%
\pgfusepath{clip}%
\pgfsetbuttcap%
\pgfsetroundjoin%
\definecolor{currentfill}{rgb}{0.283187,0.125848,0.444960}%
\pgfsetfillcolor{currentfill}%
\pgfsetfillopacity{0.700000}%
\pgfsetlinewidth{0.000000pt}%
\definecolor{currentstroke}{rgb}{0.000000,0.000000,0.000000}%
\pgfsetstrokecolor{currentstroke}%
\pgfsetdash{}{0pt}%
\pgfpathmoveto{\pgfqpoint{4.647341in}{1.658357in}}%
\pgfpathlineto{\pgfqpoint{4.661083in}{1.659449in}}%
\pgfpathlineto{\pgfqpoint{4.674835in}{1.660653in}}%
\pgfpathlineto{\pgfqpoint{4.688597in}{1.661969in}}%
\pgfpathlineto{\pgfqpoint{4.702370in}{1.663399in}}%
\pgfpathlineto{\pgfqpoint{4.710009in}{1.675362in}}%
\pgfpathlineto{\pgfqpoint{4.717643in}{1.687338in}}%
\pgfpathlineto{\pgfqpoint{4.725273in}{1.699324in}}%
\pgfpathlineto{\pgfqpoint{4.732898in}{1.711319in}}%
\pgfpathlineto{\pgfqpoint{4.719128in}{1.709625in}}%
\pgfpathlineto{\pgfqpoint{4.705369in}{1.708044in}}%
\pgfpathlineto{\pgfqpoint{4.691620in}{1.706576in}}%
\pgfpathlineto{\pgfqpoint{4.677882in}{1.705221in}}%
\pgfpathlineto{\pgfqpoint{4.670254in}{1.693484in}}%
\pgfpathlineto{\pgfqpoint{4.662621in}{1.681760in}}%
\pgfpathlineto{\pgfqpoint{4.654983in}{1.670051in}}%
\pgfpathlineto{\pgfqpoint{4.647341in}{1.658357in}}%
\pgfpathclose%
\pgfusepath{fill}%
\end{pgfscope}%
\begin{pgfscope}%
\pgfpathrectangle{\pgfqpoint{1.254980in}{0.150000in}}{\pgfqpoint{5.490039in}{5.490039in}}%
\pgfusepath{clip}%
\pgfsetbuttcap%
\pgfsetroundjoin%
\definecolor{currentfill}{rgb}{0.253935,0.265254,0.529983}%
\pgfsetfillcolor{currentfill}%
\pgfsetfillopacity{0.700000}%
\pgfsetlinewidth{0.000000pt}%
\definecolor{currentstroke}{rgb}{0.000000,0.000000,0.000000}%
\pgfsetstrokecolor{currentstroke}%
\pgfsetdash{}{0pt}%
\pgfpathmoveto{\pgfqpoint{5.020341in}{1.942773in}}%
\pgfpathlineto{\pgfqpoint{5.034260in}{1.946828in}}%
\pgfpathlineto{\pgfqpoint{5.048192in}{1.950995in}}%
\pgfpathlineto{\pgfqpoint{5.062137in}{1.955274in}}%
\pgfpathlineto{\pgfqpoint{5.076095in}{1.959666in}}%
\pgfpathlineto{\pgfqpoint{5.083636in}{1.972298in}}%
\pgfpathlineto{\pgfqpoint{5.091172in}{1.984892in}}%
\pgfpathlineto{\pgfqpoint{5.098704in}{1.997447in}}%
\pgfpathlineto{\pgfqpoint{5.106230in}{2.009962in}}%
\pgfpathlineto{\pgfqpoint{5.092273in}{2.005385in}}%
\pgfpathlineto{\pgfqpoint{5.078329in}{2.000920in}}%
\pgfpathlineto{\pgfqpoint{5.064398in}{1.996567in}}%
\pgfpathlineto{\pgfqpoint{5.050480in}{1.992326in}}%
\pgfpathlineto{\pgfqpoint{5.042953in}{1.979991in}}%
\pgfpathlineto{\pgfqpoint{5.035421in}{1.967620in}}%
\pgfpathlineto{\pgfqpoint{5.027883in}{1.955213in}}%
\pgfpathlineto{\pgfqpoint{5.020341in}{1.942773in}}%
\pgfpathclose%
\pgfusepath{fill}%
\end{pgfscope}%
\begin{pgfscope}%
\pgfpathrectangle{\pgfqpoint{1.254980in}{0.150000in}}{\pgfqpoint{5.490039in}{5.490039in}}%
\pgfusepath{clip}%
\pgfsetbuttcap%
\pgfsetroundjoin%
\definecolor{currentfill}{rgb}{0.171176,0.452530,0.557965}%
\pgfsetfillcolor{currentfill}%
\pgfsetfillopacity{0.700000}%
\pgfsetlinewidth{0.000000pt}%
\definecolor{currentstroke}{rgb}{0.000000,0.000000,0.000000}%
\pgfsetstrokecolor{currentstroke}%
\pgfsetdash{}{0pt}%
\pgfpathmoveto{\pgfqpoint{5.510377in}{2.396416in}}%
\pgfpathlineto{\pgfqpoint{5.524576in}{2.403680in}}%
\pgfpathlineto{\pgfqpoint{5.538791in}{2.411057in}}%
\pgfpathlineto{\pgfqpoint{5.553022in}{2.418547in}}%
\pgfpathlineto{\pgfqpoint{5.560398in}{2.430107in}}%
\pgfpathlineto{\pgfqpoint{5.567767in}{2.441586in}}%
\pgfpathlineto{\pgfqpoint{5.575130in}{2.452983in}}%
\pgfpathlineto{\pgfqpoint{5.582486in}{2.464298in}}%
\pgfpathlineto{\pgfqpoint{5.568258in}{2.456737in}}%
\pgfpathlineto{\pgfqpoint{5.554045in}{2.449289in}}%
\pgfpathlineto{\pgfqpoint{5.539849in}{2.441954in}}%
\pgfpathlineto{\pgfqpoint{5.532491in}{2.430687in}}%
\pgfpathlineto{\pgfqpoint{5.525126in}{2.419342in}}%
\pgfpathlineto{\pgfqpoint{5.517755in}{2.407918in}}%
\pgfpathlineto{\pgfqpoint{5.510377in}{2.396416in}}%
\pgfpathclose%
\pgfusepath{fill}%
\end{pgfscope}%
\begin{pgfscope}%
\pgfpathrectangle{\pgfqpoint{1.254980in}{0.150000in}}{\pgfqpoint{5.490039in}{5.490039in}}%
\pgfusepath{clip}%
\pgfsetbuttcap%
\pgfsetroundjoin%
\definecolor{currentfill}{rgb}{0.239374,0.735588,0.455688}%
\pgfsetfillcolor{currentfill}%
\pgfsetfillopacity{0.700000}%
\pgfsetlinewidth{0.000000pt}%
\definecolor{currentstroke}{rgb}{0.000000,0.000000,0.000000}%
\pgfsetstrokecolor{currentstroke}%
\pgfsetdash{}{0pt}%
\pgfpathmoveto{\pgfqpoint{2.195172in}{3.278500in}}%
\pgfpathlineto{\pgfqpoint{2.209108in}{3.252690in}}%
\pgfpathlineto{\pgfqpoint{2.223030in}{3.227104in}}%
\pgfpathlineto{\pgfqpoint{2.236940in}{3.201740in}}%
\pgfpathlineto{\pgfqpoint{2.250837in}{3.176595in}}%
\pgfpathlineto{\pgfqpoint{2.259878in}{3.169226in}}%
\pgfpathlineto{\pgfqpoint{2.268898in}{3.162149in}}%
\pgfpathlineto{\pgfqpoint{2.277895in}{3.155360in}}%
\pgfpathlineto{\pgfqpoint{2.286871in}{3.148856in}}%
\pgfpathlineto{\pgfqpoint{2.273032in}{3.173525in}}%
\pgfpathlineto{\pgfqpoint{2.259180in}{3.198412in}}%
\pgfpathlineto{\pgfqpoint{2.245315in}{3.223520in}}%
\pgfpathlineto{\pgfqpoint{2.231438in}{3.248850in}}%
\pgfpathlineto{\pgfqpoint{2.222405in}{3.255824in}}%
\pgfpathlineto{\pgfqpoint{2.213350in}{3.263087in}}%
\pgfpathlineto{\pgfqpoint{2.204272in}{3.270645in}}%
\pgfpathlineto{\pgfqpoint{2.195172in}{3.278500in}}%
\pgfpathclose%
\pgfusepath{fill}%
\end{pgfscope}%
\begin{pgfscope}%
\pgfpathrectangle{\pgfqpoint{1.254980in}{0.150000in}}{\pgfqpoint{5.490039in}{5.490039in}}%
\pgfusepath{clip}%
\pgfsetbuttcap%
\pgfsetroundjoin%
\definecolor{currentfill}{rgb}{0.225863,0.330805,0.547314}%
\pgfsetfillcolor{currentfill}%
\pgfsetfillopacity{0.700000}%
\pgfsetlinewidth{0.000000pt}%
\definecolor{currentstroke}{rgb}{0.000000,0.000000,0.000000}%
\pgfsetstrokecolor{currentstroke}%
\pgfsetdash{}{0pt}%
\pgfpathmoveto{\pgfqpoint{2.893006in}{2.158029in}}%
\pgfpathlineto{\pgfqpoint{2.906583in}{2.141753in}}%
\pgfpathlineto{\pgfqpoint{2.920156in}{2.125635in}}%
\pgfpathlineto{\pgfqpoint{2.933726in}{2.109673in}}%
\pgfpathlineto{\pgfqpoint{2.947291in}{2.093868in}}%
\pgfpathlineto{\pgfqpoint{2.955786in}{2.090970in}}%
\pgfpathlineto{\pgfqpoint{2.964266in}{2.088323in}}%
\pgfpathlineto{\pgfqpoint{2.972730in}{2.085924in}}%
\pgfpathlineto{\pgfqpoint{2.981178in}{2.083769in}}%
\pgfpathlineto{\pgfqpoint{2.967654in}{2.099109in}}%
\pgfpathlineto{\pgfqpoint{2.954126in}{2.114604in}}%
\pgfpathlineto{\pgfqpoint{2.940595in}{2.130256in}}%
\pgfpathlineto{\pgfqpoint{2.927060in}{2.146064in}}%
\pgfpathlineto{\pgfqpoint{2.918571in}{2.148678in}}%
\pgfpathlineto{\pgfqpoint{2.910065in}{2.151541in}}%
\pgfpathlineto{\pgfqpoint{2.901544in}{2.154656in}}%
\pgfpathlineto{\pgfqpoint{2.893006in}{2.158029in}}%
\pgfpathclose%
\pgfusepath{fill}%
\end{pgfscope}%
\begin{pgfscope}%
\pgfpathrectangle{\pgfqpoint{1.254980in}{0.150000in}}{\pgfqpoint{5.490039in}{5.490039in}}%
\pgfusepath{clip}%
\pgfsetbuttcap%
\pgfsetroundjoin%
\definecolor{currentfill}{rgb}{0.237441,0.305202,0.541921}%
\pgfsetfillcolor{currentfill}%
\pgfsetfillopacity{0.700000}%
\pgfsetlinewidth{0.000000pt}%
\definecolor{currentstroke}{rgb}{0.000000,0.000000,0.000000}%
\pgfsetstrokecolor{currentstroke}%
\pgfsetdash{}{0pt}%
\pgfpathmoveto{\pgfqpoint{2.947291in}{2.093868in}}%
\pgfpathlineto{\pgfqpoint{2.960853in}{2.078218in}}%
\pgfpathlineto{\pgfqpoint{2.974412in}{2.062722in}}%
\pgfpathlineto{\pgfqpoint{2.987968in}{2.047380in}}%
\pgfpathlineto{\pgfqpoint{3.001520in}{2.032191in}}%
\pgfpathlineto{\pgfqpoint{3.009974in}{2.029763in}}%
\pgfpathlineto{\pgfqpoint{3.018413in}{2.027583in}}%
\pgfpathlineto{\pgfqpoint{3.026837in}{2.025646in}}%
\pgfpathlineto{\pgfqpoint{3.035246in}{2.023947in}}%
\pgfpathlineto{\pgfqpoint{3.021733in}{2.038673in}}%
\pgfpathlineto{\pgfqpoint{3.008218in}{2.053552in}}%
\pgfpathlineto{\pgfqpoint{2.994700in}{2.068584in}}%
\pgfpathlineto{\pgfqpoint{2.981178in}{2.083769in}}%
\pgfpathlineto{\pgfqpoint{2.972730in}{2.085924in}}%
\pgfpathlineto{\pgfqpoint{2.964266in}{2.088323in}}%
\pgfpathlineto{\pgfqpoint{2.955786in}{2.090970in}}%
\pgfpathlineto{\pgfqpoint{2.947291in}{2.093868in}}%
\pgfpathclose%
\pgfusepath{fill}%
\end{pgfscope}%
\begin{pgfscope}%
\pgfpathrectangle{\pgfqpoint{1.254980in}{0.150000in}}{\pgfqpoint{5.490039in}{5.490039in}}%
\pgfusepath{clip}%
\pgfsetbuttcap%
\pgfsetroundjoin%
\definecolor{currentfill}{rgb}{0.274952,0.037752,0.364543}%
\pgfsetfillcolor{currentfill}%
\pgfsetfillopacity{0.700000}%
\pgfsetlinewidth{0.000000pt}%
\definecolor{currentstroke}{rgb}{0.000000,0.000000,0.000000}%
\pgfsetstrokecolor{currentstroke}%
\pgfsetdash{}{0pt}%
\pgfpathmoveto{\pgfqpoint{3.661160in}{1.531091in}}%
\pgfpathlineto{\pgfqpoint{3.674655in}{1.522980in}}%
\pgfpathlineto{\pgfqpoint{3.688152in}{1.514995in}}%
\pgfpathlineto{\pgfqpoint{3.701653in}{1.507134in}}%
\pgfpathlineto{\pgfqpoint{3.715157in}{1.499398in}}%
\pgfpathlineto{\pgfqpoint{3.723147in}{1.503897in}}%
\pgfpathlineto{\pgfqpoint{3.731128in}{1.508559in}}%
\pgfpathlineto{\pgfqpoint{3.739101in}{1.513380in}}%
\pgfpathlineto{\pgfqpoint{3.747066in}{1.518356in}}%
\pgfpathlineto{\pgfqpoint{3.733584in}{1.525683in}}%
\pgfpathlineto{\pgfqpoint{3.720105in}{1.533135in}}%
\pgfpathlineto{\pgfqpoint{3.706630in}{1.540711in}}%
\pgfpathlineto{\pgfqpoint{3.693158in}{1.548413in}}%
\pgfpathlineto{\pgfqpoint{3.685172in}{1.543839in}}%
\pgfpathlineto{\pgfqpoint{3.677176in}{1.539425in}}%
\pgfpathlineto{\pgfqpoint{3.669173in}{1.535175in}}%
\pgfpathlineto{\pgfqpoint{3.661160in}{1.531091in}}%
\pgfpathclose%
\pgfusepath{fill}%
\end{pgfscope}%
\begin{pgfscope}%
\pgfpathrectangle{\pgfqpoint{1.254980in}{0.150000in}}{\pgfqpoint{5.490039in}{5.490039in}}%
\pgfusepath{clip}%
\pgfsetbuttcap%
\pgfsetroundjoin%
\definecolor{currentfill}{rgb}{0.214298,0.355619,0.551184}%
\pgfsetfillcolor{currentfill}%
\pgfsetfillopacity{0.700000}%
\pgfsetlinewidth{0.000000pt}%
\definecolor{currentstroke}{rgb}{0.000000,0.000000,0.000000}%
\pgfsetstrokecolor{currentstroke}%
\pgfsetdash{}{0pt}%
\pgfpathmoveto{\pgfqpoint{2.838656in}{2.224726in}}%
\pgfpathlineto{\pgfqpoint{2.852250in}{2.207811in}}%
\pgfpathlineto{\pgfqpoint{2.865840in}{2.191057in}}%
\pgfpathlineto{\pgfqpoint{2.879425in}{2.174463in}}%
\pgfpathlineto{\pgfqpoint{2.893006in}{2.158029in}}%
\pgfpathlineto{\pgfqpoint{2.901544in}{2.154656in}}%
\pgfpathlineto{\pgfqpoint{2.910065in}{2.151541in}}%
\pgfpathlineto{\pgfqpoint{2.918571in}{2.148678in}}%
\pgfpathlineto{\pgfqpoint{2.927060in}{2.146064in}}%
\pgfpathlineto{\pgfqpoint{2.913522in}{2.162031in}}%
\pgfpathlineto{\pgfqpoint{2.899980in}{2.178156in}}%
\pgfpathlineto{\pgfqpoint{2.886433in}{2.194440in}}%
\pgfpathlineto{\pgfqpoint{2.872883in}{2.210885in}}%
\pgfpathlineto{\pgfqpoint{2.864351in}{2.213961in}}%
\pgfpathlineto{\pgfqpoint{2.855802in}{2.217290in}}%
\pgfpathlineto{\pgfqpoint{2.847238in}{2.220877in}}%
\pgfpathlineto{\pgfqpoint{2.838656in}{2.224726in}}%
\pgfpathclose%
\pgfusepath{fill}%
\end{pgfscope}%
\begin{pgfscope}%
\pgfpathrectangle{\pgfqpoint{1.254980in}{0.150000in}}{\pgfqpoint{5.490039in}{5.490039in}}%
\pgfusepath{clip}%
\pgfsetbuttcap%
\pgfsetroundjoin%
\definecolor{currentfill}{rgb}{0.248629,0.278775,0.534556}%
\pgfsetfillcolor{currentfill}%
\pgfsetfillopacity{0.700000}%
\pgfsetlinewidth{0.000000pt}%
\definecolor{currentstroke}{rgb}{0.000000,0.000000,0.000000}%
\pgfsetstrokecolor{currentstroke}%
\pgfsetdash{}{0pt}%
\pgfpathmoveto{\pgfqpoint{3.001520in}{2.032191in}}%
\pgfpathlineto{\pgfqpoint{3.015070in}{2.017153in}}%
\pgfpathlineto{\pgfqpoint{3.028617in}{2.002267in}}%
\pgfpathlineto{\pgfqpoint{3.042161in}{1.987532in}}%
\pgfpathlineto{\pgfqpoint{3.055702in}{1.972946in}}%
\pgfpathlineto{\pgfqpoint{3.064116in}{1.970988in}}%
\pgfpathlineto{\pgfqpoint{3.072515in}{1.969271in}}%
\pgfpathlineto{\pgfqpoint{3.080901in}{1.967792in}}%
\pgfpathlineto{\pgfqpoint{3.089272in}{1.966548in}}%
\pgfpathlineto{\pgfqpoint{3.075769in}{1.980673in}}%
\pgfpathlineto{\pgfqpoint{3.062263in}{1.994947in}}%
\pgfpathlineto{\pgfqpoint{3.048756in}{2.009372in}}%
\pgfpathlineto{\pgfqpoint{3.035246in}{2.023947in}}%
\pgfpathlineto{\pgfqpoint{3.026837in}{2.025646in}}%
\pgfpathlineto{\pgfqpoint{3.018413in}{2.027583in}}%
\pgfpathlineto{\pgfqpoint{3.009974in}{2.029763in}}%
\pgfpathlineto{\pgfqpoint{3.001520in}{2.032191in}}%
\pgfpathclose%
\pgfusepath{fill}%
\end{pgfscope}%
\begin{pgfscope}%
\pgfpathrectangle{\pgfqpoint{1.254980in}{0.150000in}}{\pgfqpoint{5.490039in}{5.490039in}}%
\pgfusepath{clip}%
\pgfsetbuttcap%
\pgfsetroundjoin%
\definecolor{currentfill}{rgb}{0.119423,0.611141,0.538982}%
\pgfsetfillcolor{currentfill}%
\pgfsetfillopacity{0.700000}%
\pgfsetlinewidth{0.000000pt}%
\definecolor{currentstroke}{rgb}{0.000000,0.000000,0.000000}%
\pgfsetstrokecolor{currentstroke}%
\pgfsetdash{}{0pt}%
\pgfpathmoveto{\pgfqpoint{2.380988in}{2.918077in}}%
\pgfpathlineto{\pgfqpoint{2.394795in}{2.895172in}}%
\pgfpathlineto{\pgfqpoint{2.408591in}{2.872468in}}%
\pgfpathlineto{\pgfqpoint{2.422377in}{2.849963in}}%
\pgfpathlineto{\pgfqpoint{2.436153in}{2.827656in}}%
\pgfpathlineto{\pgfqpoint{2.445055in}{2.821059in}}%
\pgfpathlineto{\pgfqpoint{2.453936in}{2.814749in}}%
\pgfpathlineto{\pgfqpoint{2.462797in}{2.808722in}}%
\pgfpathlineto{\pgfqpoint{2.471639in}{2.802973in}}%
\pgfpathlineto{\pgfqpoint{2.457916in}{2.824801in}}%
\pgfpathlineto{\pgfqpoint{2.444184in}{2.846826in}}%
\pgfpathlineto{\pgfqpoint{2.430442in}{2.869049in}}%
\pgfpathlineto{\pgfqpoint{2.416691in}{2.891471in}}%
\pgfpathlineto{\pgfqpoint{2.407796in}{2.897692in}}%
\pgfpathlineto{\pgfqpoint{2.398881in}{2.904197in}}%
\pgfpathlineto{\pgfqpoint{2.389945in}{2.910991in}}%
\pgfpathlineto{\pgfqpoint{2.380988in}{2.918077in}}%
\pgfpathclose%
\pgfusepath{fill}%
\end{pgfscope}%
\begin{pgfscope}%
\pgfpathrectangle{\pgfqpoint{1.254980in}{0.150000in}}{\pgfqpoint{5.490039in}{5.490039in}}%
\pgfusepath{clip}%
\pgfsetbuttcap%
\pgfsetroundjoin%
\definecolor{currentfill}{rgb}{0.201239,0.383670,0.554294}%
\pgfsetfillcolor{currentfill}%
\pgfsetfillopacity{0.700000}%
\pgfsetlinewidth{0.000000pt}%
\definecolor{currentstroke}{rgb}{0.000000,0.000000,0.000000}%
\pgfsetstrokecolor{currentstroke}%
\pgfsetdash{}{0pt}%
\pgfpathmoveto{\pgfqpoint{2.784232in}{2.294016in}}%
\pgfpathlineto{\pgfqpoint{2.797845in}{2.276447in}}%
\pgfpathlineto{\pgfqpoint{2.811454in}{2.259043in}}%
\pgfpathlineto{\pgfqpoint{2.825057in}{2.241803in}}%
\pgfpathlineto{\pgfqpoint{2.838656in}{2.224726in}}%
\pgfpathlineto{\pgfqpoint{2.847238in}{2.220877in}}%
\pgfpathlineto{\pgfqpoint{2.855802in}{2.217290in}}%
\pgfpathlineto{\pgfqpoint{2.864351in}{2.213961in}}%
\pgfpathlineto{\pgfqpoint{2.872883in}{2.210885in}}%
\pgfpathlineto{\pgfqpoint{2.859328in}{2.227491in}}%
\pgfpathlineto{\pgfqpoint{2.845769in}{2.244260in}}%
\pgfpathlineto{\pgfqpoint{2.832205in}{2.261191in}}%
\pgfpathlineto{\pgfqpoint{2.818637in}{2.278287in}}%
\pgfpathlineto{\pgfqpoint{2.810061in}{2.281827in}}%
\pgfpathlineto{\pgfqpoint{2.801468in}{2.285626in}}%
\pgfpathlineto{\pgfqpoint{2.792859in}{2.289687in}}%
\pgfpathlineto{\pgfqpoint{2.784232in}{2.294016in}}%
\pgfpathclose%
\pgfusepath{fill}%
\end{pgfscope}%
\begin{pgfscope}%
\pgfpathrectangle{\pgfqpoint{1.254980in}{0.150000in}}{\pgfqpoint{5.490039in}{5.490039in}}%
\pgfusepath{clip}%
\pgfsetbuttcap%
\pgfsetroundjoin%
\definecolor{currentfill}{rgb}{0.273809,0.031497,0.358853}%
\pgfsetfillcolor{currentfill}%
\pgfsetfillopacity{0.700000}%
\pgfsetlinewidth{0.000000pt}%
\definecolor{currentstroke}{rgb}{0.000000,0.000000,0.000000}%
\pgfsetstrokecolor{currentstroke}%
\pgfsetdash{}{0pt}%
\pgfpathmoveto{\pgfqpoint{4.305570in}{1.492622in}}%
\pgfpathlineto{\pgfqpoint{4.319187in}{1.490685in}}%
\pgfpathlineto{\pgfqpoint{4.332812in}{1.488861in}}%
\pgfpathlineto{\pgfqpoint{4.346446in}{1.487153in}}%
\pgfpathlineto{\pgfqpoint{4.360088in}{1.485558in}}%
\pgfpathlineto{\pgfqpoint{4.367820in}{1.495688in}}%
\pgfpathlineto{\pgfqpoint{4.375547in}{1.505882in}}%
\pgfpathlineto{\pgfqpoint{4.383269in}{1.516139in}}%
\pgfpathlineto{\pgfqpoint{4.390987in}{1.526457in}}%
\pgfpathlineto{\pgfqpoint{4.377353in}{1.527725in}}%
\pgfpathlineto{\pgfqpoint{4.363727in}{1.529108in}}%
\pgfpathlineto{\pgfqpoint{4.350110in}{1.530605in}}%
\pgfpathlineto{\pgfqpoint{4.336501in}{1.532217in}}%
\pgfpathlineto{\pgfqpoint{4.328775in}{1.522219in}}%
\pgfpathlineto{\pgfqpoint{4.321045in}{1.512286in}}%
\pgfpathlineto{\pgfqpoint{4.313310in}{1.502419in}}%
\pgfpathlineto{\pgfqpoint{4.305570in}{1.492622in}}%
\pgfpathclose%
\pgfusepath{fill}%
\end{pgfscope}%
\begin{pgfscope}%
\pgfpathrectangle{\pgfqpoint{1.254980in}{0.150000in}}{\pgfqpoint{5.490039in}{5.490039in}}%
\pgfusepath{clip}%
\pgfsetbuttcap%
\pgfsetroundjoin%
\definecolor{currentfill}{rgb}{0.281412,0.155834,0.469201}%
\pgfsetfillcolor{currentfill}%
\pgfsetfillopacity{0.700000}%
\pgfsetlinewidth{0.000000pt}%
\definecolor{currentstroke}{rgb}{0.000000,0.000000,0.000000}%
\pgfsetstrokecolor{currentstroke}%
\pgfsetdash{}{0pt}%
\pgfpathmoveto{\pgfqpoint{4.732898in}{1.711319in}}%
\pgfpathlineto{\pgfqpoint{4.746680in}{1.713125in}}%
\pgfpathlineto{\pgfqpoint{4.760472in}{1.715044in}}%
\pgfpathlineto{\pgfqpoint{4.774276in}{1.717074in}}%
\pgfpathlineto{\pgfqpoint{4.788091in}{1.719217in}}%
\pgfpathlineto{\pgfqpoint{4.795709in}{1.731473in}}%
\pgfpathlineto{\pgfqpoint{4.803324in}{1.743730in}}%
\pgfpathlineto{\pgfqpoint{4.810933in}{1.755985in}}%
\pgfpathlineto{\pgfqpoint{4.818539in}{1.768236in}}%
\pgfpathlineto{\pgfqpoint{4.804726in}{1.765844in}}%
\pgfpathlineto{\pgfqpoint{4.790924in}{1.763565in}}%
\pgfpathlineto{\pgfqpoint{4.777135in}{1.761397in}}%
\pgfpathlineto{\pgfqpoint{4.763356in}{1.759343in}}%
\pgfpathlineto{\pgfqpoint{4.755748in}{1.747334in}}%
\pgfpathlineto{\pgfqpoint{4.748136in}{1.735326in}}%
\pgfpathlineto{\pgfqpoint{4.740519in}{1.723320in}}%
\pgfpathlineto{\pgfqpoint{4.732898in}{1.711319in}}%
\pgfpathclose%
\pgfusepath{fill}%
\end{pgfscope}%
\begin{pgfscope}%
\pgfpathrectangle{\pgfqpoint{1.254980in}{0.150000in}}{\pgfqpoint{5.490039in}{5.490039in}}%
\pgfusepath{clip}%
\pgfsetbuttcap%
\pgfsetroundjoin%
\definecolor{currentfill}{rgb}{0.258965,0.251537,0.524736}%
\pgfsetfillcolor{currentfill}%
\pgfsetfillopacity{0.700000}%
\pgfsetlinewidth{0.000000pt}%
\definecolor{currentstroke}{rgb}{0.000000,0.000000,0.000000}%
\pgfsetstrokecolor{currentstroke}%
\pgfsetdash{}{0pt}%
\pgfpathmoveto{\pgfqpoint{3.055702in}{1.972946in}}%
\pgfpathlineto{\pgfqpoint{3.069241in}{1.958510in}}%
\pgfpathlineto{\pgfqpoint{3.082778in}{1.944221in}}%
\pgfpathlineto{\pgfqpoint{3.096313in}{1.930080in}}%
\pgfpathlineto{\pgfqpoint{3.109846in}{1.916087in}}%
\pgfpathlineto{\pgfqpoint{3.118221in}{1.914594in}}%
\pgfpathlineto{\pgfqpoint{3.126582in}{1.913339in}}%
\pgfpathlineto{\pgfqpoint{3.134930in}{1.912317in}}%
\pgfpathlineto{\pgfqpoint{3.143264in}{1.911524in}}%
\pgfpathlineto{\pgfqpoint{3.129769in}{1.925060in}}%
\pgfpathlineto{\pgfqpoint{3.116272in}{1.938742in}}%
\pgfpathlineto{\pgfqpoint{3.102773in}{1.952571in}}%
\pgfpathlineto{\pgfqpoint{3.089272in}{1.966548in}}%
\pgfpathlineto{\pgfqpoint{3.080901in}{1.967792in}}%
\pgfpathlineto{\pgfqpoint{3.072515in}{1.969271in}}%
\pgfpathlineto{\pgfqpoint{3.064116in}{1.970988in}}%
\pgfpathlineto{\pgfqpoint{3.055702in}{1.972946in}}%
\pgfpathclose%
\pgfusepath{fill}%
\end{pgfscope}%
\begin{pgfscope}%
\pgfpathrectangle{\pgfqpoint{1.254980in}{0.150000in}}{\pgfqpoint{5.490039in}{5.490039in}}%
\pgfusepath{clip}%
\pgfsetbuttcap%
\pgfsetroundjoin%
\definecolor{currentfill}{rgb}{0.282327,0.094955,0.417331}%
\pgfsetfillcolor{currentfill}%
\pgfsetfillopacity{0.700000}%
\pgfsetlinewidth{0.000000pt}%
\definecolor{currentstroke}{rgb}{0.000000,0.000000,0.000000}%
\pgfsetstrokecolor{currentstroke}%
\pgfsetdash{}{0pt}%
\pgfpathmoveto{\pgfqpoint{3.466976in}{1.628902in}}%
\pgfpathlineto{\pgfqpoint{3.480470in}{1.618823in}}%
\pgfpathlineto{\pgfqpoint{3.493965in}{1.608874in}}%
\pgfpathlineto{\pgfqpoint{3.507461in}{1.599056in}}%
\pgfpathlineto{\pgfqpoint{3.520960in}{1.589369in}}%
\pgfpathlineto{\pgfqpoint{3.529060in}{1.591882in}}%
\pgfpathlineto{\pgfqpoint{3.537151in}{1.594584in}}%
\pgfpathlineto{\pgfqpoint{3.545232in}{1.597473in}}%
\pgfpathlineto{\pgfqpoint{3.553302in}{1.600545in}}%
\pgfpathlineto{\pgfqpoint{3.539831in}{1.609804in}}%
\pgfpathlineto{\pgfqpoint{3.526361in}{1.619193in}}%
\pgfpathlineto{\pgfqpoint{3.512894in}{1.628712in}}%
\pgfpathlineto{\pgfqpoint{3.499428in}{1.638363in}}%
\pgfpathlineto{\pgfqpoint{3.491330in}{1.635713in}}%
\pgfpathlineto{\pgfqpoint{3.483222in}{1.633251in}}%
\pgfpathlineto{\pgfqpoint{3.475104in}{1.630980in}}%
\pgfpathlineto{\pgfqpoint{3.466976in}{1.628902in}}%
\pgfpathclose%
\pgfusepath{fill}%
\end{pgfscope}%
\begin{pgfscope}%
\pgfpathrectangle{\pgfqpoint{1.254980in}{0.150000in}}{\pgfqpoint{5.490039in}{5.490039in}}%
\pgfusepath{clip}%
\pgfsetbuttcap%
\pgfsetroundjoin%
\definecolor{currentfill}{rgb}{0.188923,0.410910,0.556326}%
\pgfsetfillcolor{currentfill}%
\pgfsetfillopacity{0.700000}%
\pgfsetlinewidth{0.000000pt}%
\definecolor{currentstroke}{rgb}{0.000000,0.000000,0.000000}%
\pgfsetstrokecolor{currentstroke}%
\pgfsetdash{}{0pt}%
\pgfpathmoveto{\pgfqpoint{2.729723in}{2.365957in}}%
\pgfpathlineto{\pgfqpoint{2.743359in}{2.347720in}}%
\pgfpathlineto{\pgfqpoint{2.756989in}{2.329651in}}%
\pgfpathlineto{\pgfqpoint{2.770613in}{2.311750in}}%
\pgfpathlineto{\pgfqpoint{2.784232in}{2.294016in}}%
\pgfpathlineto{\pgfqpoint{2.792859in}{2.289687in}}%
\pgfpathlineto{\pgfqpoint{2.801468in}{2.285626in}}%
\pgfpathlineto{\pgfqpoint{2.810061in}{2.281827in}}%
\pgfpathlineto{\pgfqpoint{2.818637in}{2.278287in}}%
\pgfpathlineto{\pgfqpoint{2.805063in}{2.295547in}}%
\pgfpathlineto{\pgfqpoint{2.791485in}{2.312973in}}%
\pgfpathlineto{\pgfqpoint{2.777901in}{2.330566in}}%
\pgfpathlineto{\pgfqpoint{2.764312in}{2.348327in}}%
\pgfpathlineto{\pgfqpoint{2.755691in}{2.352335in}}%
\pgfpathlineto{\pgfqpoint{2.747053in}{2.356607in}}%
\pgfpathlineto{\pgfqpoint{2.738397in}{2.361146in}}%
\pgfpathlineto{\pgfqpoint{2.729723in}{2.365957in}}%
\pgfpathclose%
\pgfusepath{fill}%
\end{pgfscope}%
\begin{pgfscope}%
\pgfpathrectangle{\pgfqpoint{1.254980in}{0.150000in}}{\pgfqpoint{5.490039in}{5.490039in}}%
\pgfusepath{clip}%
\pgfsetbuttcap%
\pgfsetroundjoin%
\definecolor{currentfill}{rgb}{0.203063,0.379716,0.553925}%
\pgfsetfillcolor{currentfill}%
\pgfsetfillopacity{0.700000}%
\pgfsetlinewidth{0.000000pt}%
\definecolor{currentstroke}{rgb}{0.000000,0.000000,0.000000}%
\pgfsetstrokecolor{currentstroke}%
\pgfsetdash{}{0pt}%
\pgfpathmoveto{\pgfqpoint{5.308275in}{2.200930in}}%
\pgfpathlineto{\pgfqpoint{5.322356in}{2.206980in}}%
\pgfpathlineto{\pgfqpoint{5.336452in}{2.213142in}}%
\pgfpathlineto{\pgfqpoint{5.350563in}{2.219416in}}%
\pgfpathlineto{\pgfqpoint{5.364689in}{2.225804in}}%
\pgfpathlineto{\pgfqpoint{5.372143in}{2.238075in}}%
\pgfpathlineto{\pgfqpoint{5.379591in}{2.250280in}}%
\pgfpathlineto{\pgfqpoint{5.387033in}{2.262417in}}%
\pgfpathlineto{\pgfqpoint{5.394469in}{2.274485in}}%
\pgfpathlineto{\pgfqpoint{5.380344in}{2.267976in}}%
\pgfpathlineto{\pgfqpoint{5.366234in}{2.261581in}}%
\pgfpathlineto{\pgfqpoint{5.352139in}{2.255298in}}%
\pgfpathlineto{\pgfqpoint{5.338059in}{2.249127in}}%
\pgfpathlineto{\pgfqpoint{5.330622in}{2.237174in}}%
\pgfpathlineto{\pgfqpoint{5.323178in}{2.225156in}}%
\pgfpathlineto{\pgfqpoint{5.315729in}{2.213075in}}%
\pgfpathlineto{\pgfqpoint{5.308275in}{2.200930in}}%
\pgfpathclose%
\pgfusepath{fill}%
\end{pgfscope}%
\begin{pgfscope}%
\pgfpathrectangle{\pgfqpoint{1.254980in}{0.150000in}}{\pgfqpoint{5.490039in}{5.490039in}}%
\pgfusepath{clip}%
\pgfsetbuttcap%
\pgfsetroundjoin%
\definecolor{currentfill}{rgb}{0.266580,0.228262,0.514349}%
\pgfsetfillcolor{currentfill}%
\pgfsetfillopacity{0.700000}%
\pgfsetlinewidth{0.000000pt}%
\definecolor{currentstroke}{rgb}{0.000000,0.000000,0.000000}%
\pgfsetstrokecolor{currentstroke}%
\pgfsetdash{}{0pt}%
\pgfpathmoveto{\pgfqpoint{3.109846in}{1.916087in}}%
\pgfpathlineto{\pgfqpoint{3.123377in}{1.902239in}}%
\pgfpathlineto{\pgfqpoint{3.136906in}{1.888537in}}%
\pgfpathlineto{\pgfqpoint{3.150433in}{1.874979in}}%
\pgfpathlineto{\pgfqpoint{3.163959in}{1.861566in}}%
\pgfpathlineto{\pgfqpoint{3.172297in}{1.860538in}}%
\pgfpathlineto{\pgfqpoint{3.180622in}{1.859741in}}%
\pgfpathlineto{\pgfqpoint{3.188933in}{1.859174in}}%
\pgfpathlineto{\pgfqpoint{3.197232in}{1.858831in}}%
\pgfpathlineto{\pgfqpoint{3.183742in}{1.871788in}}%
\pgfpathlineto{\pgfqpoint{3.170251in}{1.884889in}}%
\pgfpathlineto{\pgfqpoint{3.156758in}{1.898134in}}%
\pgfpathlineto{\pgfqpoint{3.143264in}{1.911524in}}%
\pgfpathlineto{\pgfqpoint{3.134930in}{1.912317in}}%
\pgfpathlineto{\pgfqpoint{3.126582in}{1.913339in}}%
\pgfpathlineto{\pgfqpoint{3.118221in}{1.914594in}}%
\pgfpathlineto{\pgfqpoint{3.109846in}{1.916087in}}%
\pgfpathclose%
\pgfusepath{fill}%
\end{pgfscope}%
\begin{pgfscope}%
\pgfpathrectangle{\pgfqpoint{1.254980in}{0.150000in}}{\pgfqpoint{5.490039in}{5.490039in}}%
\pgfusepath{clip}%
\pgfsetbuttcap%
\pgfsetroundjoin%
\definecolor{currentfill}{rgb}{0.268510,0.009605,0.335427}%
\pgfsetfillcolor{currentfill}%
\pgfsetfillopacity{0.700000}%
\pgfsetlinewidth{0.000000pt}%
\definecolor{currentstroke}{rgb}{0.000000,0.000000,0.000000}%
\pgfsetstrokecolor{currentstroke}%
\pgfsetdash{}{0pt}%
\pgfpathmoveto{\pgfqpoint{3.855066in}{1.464177in}}%
\pgfpathlineto{\pgfqpoint{3.868585in}{1.457954in}}%
\pgfpathlineto{\pgfqpoint{3.882109in}{1.451852in}}%
\pgfpathlineto{\pgfqpoint{3.895638in}{1.445870in}}%
\pgfpathlineto{\pgfqpoint{3.909172in}{1.440008in}}%
\pgfpathlineto{\pgfqpoint{3.917071in}{1.446329in}}%
\pgfpathlineto{\pgfqpoint{3.924963in}{1.452787in}}%
\pgfpathlineto{\pgfqpoint{3.932849in}{1.459377in}}%
\pgfpathlineto{\pgfqpoint{3.940728in}{1.466098in}}%
\pgfpathlineto{\pgfqpoint{3.927211in}{1.471569in}}%
\pgfpathlineto{\pgfqpoint{3.913700in}{1.477160in}}%
\pgfpathlineto{\pgfqpoint{3.900194in}{1.482872in}}%
\pgfpathlineto{\pgfqpoint{3.886693in}{1.488704in}}%
\pgfpathlineto{\pgfqpoint{3.878797in}{1.482368in}}%
\pgfpathlineto{\pgfqpoint{3.870894in}{1.476166in}}%
\pgfpathlineto{\pgfqpoint{3.862984in}{1.470101in}}%
\pgfpathlineto{\pgfqpoint{3.855066in}{1.464177in}}%
\pgfpathclose%
\pgfusepath{fill}%
\end{pgfscope}%
\begin{pgfscope}%
\pgfpathrectangle{\pgfqpoint{1.254980in}{0.150000in}}{\pgfqpoint{5.490039in}{5.490039in}}%
\pgfusepath{clip}%
\pgfsetbuttcap%
\pgfsetroundjoin%
\definecolor{currentfill}{rgb}{0.277134,0.185228,0.489898}%
\pgfsetfillcolor{currentfill}%
\pgfsetfillopacity{0.700000}%
\pgfsetlinewidth{0.000000pt}%
\definecolor{currentstroke}{rgb}{0.000000,0.000000,0.000000}%
\pgfsetstrokecolor{currentstroke}%
\pgfsetdash{}{0pt}%
\pgfpathmoveto{\pgfqpoint{4.818539in}{1.768236in}}%
\pgfpathlineto{\pgfqpoint{4.832363in}{1.770741in}}%
\pgfpathlineto{\pgfqpoint{4.846199in}{1.773357in}}%
\pgfpathlineto{\pgfqpoint{4.860047in}{1.776086in}}%
\pgfpathlineto{\pgfqpoint{4.873907in}{1.778927in}}%
\pgfpathlineto{\pgfqpoint{4.881506in}{1.791413in}}%
\pgfpathlineto{\pgfqpoint{4.889101in}{1.803888in}}%
\pgfpathlineto{\pgfqpoint{4.896691in}{1.816351in}}%
\pgfpathlineto{\pgfqpoint{4.904276in}{1.828799in}}%
\pgfpathlineto{\pgfqpoint{4.890418in}{1.825725in}}%
\pgfpathlineto{\pgfqpoint{4.876572in}{1.822763in}}%
\pgfpathlineto{\pgfqpoint{4.862737in}{1.819913in}}%
\pgfpathlineto{\pgfqpoint{4.848915in}{1.817175in}}%
\pgfpathlineto{\pgfqpoint{4.841328in}{1.804954in}}%
\pgfpathlineto{\pgfqpoint{4.833736in}{1.792723in}}%
\pgfpathlineto{\pgfqpoint{4.826139in}{1.780483in}}%
\pgfpathlineto{\pgfqpoint{4.818539in}{1.768236in}}%
\pgfpathclose%
\pgfusepath{fill}%
\end{pgfscope}%
\begin{pgfscope}%
\pgfpathrectangle{\pgfqpoint{1.254980in}{0.150000in}}{\pgfqpoint{5.490039in}{5.490039in}}%
\pgfusepath{clip}%
\pgfsetbuttcap%
\pgfsetroundjoin%
\definecolor{currentfill}{rgb}{0.239346,0.300855,0.540844}%
\pgfsetfillcolor{currentfill}%
\pgfsetfillopacity{0.700000}%
\pgfsetlinewidth{0.000000pt}%
\definecolor{currentstroke}{rgb}{0.000000,0.000000,0.000000}%
\pgfsetstrokecolor{currentstroke}%
\pgfsetdash{}{0pt}%
\pgfpathmoveto{\pgfqpoint{5.106230in}{2.009962in}}%
\pgfpathlineto{\pgfqpoint{5.120201in}{2.014652in}}%
\pgfpathlineto{\pgfqpoint{5.134185in}{2.019454in}}%
\pgfpathlineto{\pgfqpoint{5.148183in}{2.024368in}}%
\pgfpathlineto{\pgfqpoint{5.162195in}{2.029394in}}%
\pgfpathlineto{\pgfqpoint{5.169716in}{2.042044in}}%
\pgfpathlineto{\pgfqpoint{5.177232in}{2.054646in}}%
\pgfpathlineto{\pgfqpoint{5.184742in}{2.067201in}}%
\pgfpathlineto{\pgfqpoint{5.192247in}{2.079707in}}%
\pgfpathlineto{\pgfqpoint{5.178236in}{2.074511in}}%
\pgfpathlineto{\pgfqpoint{5.164238in}{2.069427in}}%
\pgfpathlineto{\pgfqpoint{5.150255in}{2.064455in}}%
\pgfpathlineto{\pgfqpoint{5.136285in}{2.059596in}}%
\pgfpathlineto{\pgfqpoint{5.128779in}{2.047254in}}%
\pgfpathlineto{\pgfqpoint{5.121268in}{2.034867in}}%
\pgfpathlineto{\pgfqpoint{5.113751in}{2.022436in}}%
\pgfpathlineto{\pgfqpoint{5.106230in}{2.009962in}}%
\pgfpathclose%
\pgfusepath{fill}%
\end{pgfscope}%
\begin{pgfscope}%
\pgfpathrectangle{\pgfqpoint{1.254980in}{0.150000in}}{\pgfqpoint{5.490039in}{5.490039in}}%
\pgfusepath{clip}%
\pgfsetbuttcap%
\pgfsetroundjoin%
\definecolor{currentfill}{rgb}{0.269944,0.014625,0.341379}%
\pgfsetfillcolor{currentfill}%
\pgfsetfillopacity{0.700000}%
\pgfsetlinewidth{0.000000pt}%
\definecolor{currentstroke}{rgb}{0.000000,0.000000,0.000000}%
\pgfsetstrokecolor{currentstroke}%
\pgfsetdash{}{0pt}%
\pgfpathmoveto{\pgfqpoint{4.220128in}{1.464451in}}%
\pgfpathlineto{\pgfqpoint{4.233724in}{1.461711in}}%
\pgfpathlineto{\pgfqpoint{4.247328in}{1.459086in}}%
\pgfpathlineto{\pgfqpoint{4.260939in}{1.456577in}}%
\pgfpathlineto{\pgfqpoint{4.274558in}{1.454182in}}%
\pgfpathlineto{\pgfqpoint{4.282318in}{1.463675in}}%
\pgfpathlineto{\pgfqpoint{4.290074in}{1.473247in}}%
\pgfpathlineto{\pgfqpoint{4.297824in}{1.482897in}}%
\pgfpathlineto{\pgfqpoint{4.305570in}{1.492622in}}%
\pgfpathlineto{\pgfqpoint{4.291960in}{1.494675in}}%
\pgfpathlineto{\pgfqpoint{4.278359in}{1.496843in}}%
\pgfpathlineto{\pgfqpoint{4.264765in}{1.499125in}}%
\pgfpathlineto{\pgfqpoint{4.251179in}{1.501524in}}%
\pgfpathlineto{\pgfqpoint{4.243424in}{1.492135in}}%
\pgfpathlineto{\pgfqpoint{4.235664in}{1.482824in}}%
\pgfpathlineto{\pgfqpoint{4.227899in}{1.473595in}}%
\pgfpathlineto{\pgfqpoint{4.220128in}{1.464451in}}%
\pgfpathclose%
\pgfusepath{fill}%
\end{pgfscope}%
\begin{pgfscope}%
\pgfpathrectangle{\pgfqpoint{1.254980in}{0.150000in}}{\pgfqpoint{5.490039in}{5.490039in}}%
\pgfusepath{clip}%
\pgfsetbuttcap%
\pgfsetroundjoin%
\definecolor{currentfill}{rgb}{0.175841,0.441290,0.557685}%
\pgfsetfillcolor{currentfill}%
\pgfsetfillopacity{0.700000}%
\pgfsetlinewidth{0.000000pt}%
\definecolor{currentstroke}{rgb}{0.000000,0.000000,0.000000}%
\pgfsetstrokecolor{currentstroke}%
\pgfsetdash{}{0pt}%
\pgfpathmoveto{\pgfqpoint{2.675121in}{2.440613in}}%
\pgfpathlineto{\pgfqpoint{2.688781in}{2.421691in}}%
\pgfpathlineto{\pgfqpoint{2.702435in}{2.402942in}}%
\pgfpathlineto{\pgfqpoint{2.716082in}{2.384364in}}%
\pgfpathlineto{\pgfqpoint{2.729723in}{2.365957in}}%
\pgfpathlineto{\pgfqpoint{2.738397in}{2.361146in}}%
\pgfpathlineto{\pgfqpoint{2.747053in}{2.356607in}}%
\pgfpathlineto{\pgfqpoint{2.755691in}{2.352335in}}%
\pgfpathlineto{\pgfqpoint{2.764312in}{2.348327in}}%
\pgfpathlineto{\pgfqpoint{2.750718in}{2.366257in}}%
\pgfpathlineto{\pgfqpoint{2.737118in}{2.384356in}}%
\pgfpathlineto{\pgfqpoint{2.723512in}{2.402626in}}%
\pgfpathlineto{\pgfqpoint{2.709900in}{2.421068in}}%
\pgfpathlineto{\pgfqpoint{2.701232in}{2.425547in}}%
\pgfpathlineto{\pgfqpoint{2.692547in}{2.430295in}}%
\pgfpathlineto{\pgfqpoint{2.683843in}{2.435315in}}%
\pgfpathlineto{\pgfqpoint{2.675121in}{2.440613in}}%
\pgfpathclose%
\pgfusepath{fill}%
\end{pgfscope}%
\begin{pgfscope}%
\pgfpathrectangle{\pgfqpoint{1.254980in}{0.150000in}}{\pgfqpoint{5.490039in}{5.490039in}}%
\pgfusepath{clip}%
\pgfsetbuttcap%
\pgfsetroundjoin%
\definecolor{currentfill}{rgb}{0.267004,0.004874,0.329415}%
\pgfsetfillcolor{currentfill}%
\pgfsetfillopacity{0.700000}%
\pgfsetlinewidth{0.000000pt}%
\definecolor{currentstroke}{rgb}{0.000000,0.000000,0.000000}%
\pgfsetstrokecolor{currentstroke}%
\pgfsetdash{}{0pt}%
\pgfpathmoveto{\pgfqpoint{3.994846in}{1.445410in}}%
\pgfpathlineto{\pgfqpoint{4.008389in}{1.440535in}}%
\pgfpathlineto{\pgfqpoint{4.021938in}{1.435779in}}%
\pgfpathlineto{\pgfqpoint{4.035493in}{1.431141in}}%
\pgfpathlineto{\pgfqpoint{4.049054in}{1.426620in}}%
\pgfpathlineto{\pgfqpoint{4.056895in}{1.434228in}}%
\pgfpathlineto{\pgfqpoint{4.064730in}{1.441951in}}%
\pgfpathlineto{\pgfqpoint{4.072559in}{1.449788in}}%
\pgfpathlineto{\pgfqpoint{4.080382in}{1.457733in}}%
\pgfpathlineto{\pgfqpoint{4.066835in}{1.461880in}}%
\pgfpathlineto{\pgfqpoint{4.053295in}{1.466144in}}%
\pgfpathlineto{\pgfqpoint{4.039760in}{1.470527in}}%
\pgfpathlineto{\pgfqpoint{4.026232in}{1.475028in}}%
\pgfpathlineto{\pgfqpoint{4.018395in}{1.467450in}}%
\pgfpathlineto{\pgfqpoint{4.010551in}{1.459985in}}%
\pgfpathlineto{\pgfqpoint{4.002702in}{1.452638in}}%
\pgfpathlineto{\pgfqpoint{3.994846in}{1.445410in}}%
\pgfpathclose%
\pgfusepath{fill}%
\end{pgfscope}%
\begin{pgfscope}%
\pgfpathrectangle{\pgfqpoint{1.254980in}{0.150000in}}{\pgfqpoint{5.490039in}{5.490039in}}%
\pgfusepath{clip}%
\pgfsetbuttcap%
\pgfsetroundjoin%
\definecolor{currentfill}{rgb}{0.273006,0.204520,0.501721}%
\pgfsetfillcolor{currentfill}%
\pgfsetfillopacity{0.700000}%
\pgfsetlinewidth{0.000000pt}%
\definecolor{currentstroke}{rgb}{0.000000,0.000000,0.000000}%
\pgfsetstrokecolor{currentstroke}%
\pgfsetdash{}{0pt}%
\pgfpathmoveto{\pgfqpoint{3.163959in}{1.861566in}}%
\pgfpathlineto{\pgfqpoint{3.177484in}{1.848296in}}%
\pgfpathlineto{\pgfqpoint{3.191007in}{1.835169in}}%
\pgfpathlineto{\pgfqpoint{3.204529in}{1.822185in}}%
\pgfpathlineto{\pgfqpoint{3.218050in}{1.809341in}}%
\pgfpathlineto{\pgfqpoint{3.226353in}{1.808775in}}%
\pgfpathlineto{\pgfqpoint{3.234642in}{1.808436in}}%
\pgfpathlineto{\pgfqpoint{3.242919in}{1.808321in}}%
\pgfpathlineto{\pgfqpoint{3.251183in}{1.808425in}}%
\pgfpathlineto{\pgfqpoint{3.237696in}{1.820815in}}%
\pgfpathlineto{\pgfqpoint{3.224209in}{1.833345in}}%
\pgfpathlineto{\pgfqpoint{3.210721in}{1.846017in}}%
\pgfpathlineto{\pgfqpoint{3.197232in}{1.858831in}}%
\pgfpathlineto{\pgfqpoint{3.188933in}{1.859174in}}%
\pgfpathlineto{\pgfqpoint{3.180622in}{1.859741in}}%
\pgfpathlineto{\pgfqpoint{3.172297in}{1.860538in}}%
\pgfpathlineto{\pgfqpoint{3.163959in}{1.861566in}}%
\pgfpathclose%
\pgfusepath{fill}%
\end{pgfscope}%
\begin{pgfscope}%
\pgfpathrectangle{\pgfqpoint{1.254980in}{0.150000in}}{\pgfqpoint{5.490039in}{5.490039in}}%
\pgfusepath{clip}%
\pgfsetbuttcap%
\pgfsetroundjoin%
\definecolor{currentfill}{rgb}{0.128087,0.647749,0.523491}%
\pgfsetfillcolor{currentfill}%
\pgfsetfillopacity{0.700000}%
\pgfsetlinewidth{0.000000pt}%
\definecolor{currentstroke}{rgb}{0.000000,0.000000,0.000000}%
\pgfsetstrokecolor{currentstroke}%
\pgfsetdash{}{0pt}%
\pgfpathmoveto{\pgfqpoint{2.325658in}{3.011726in}}%
\pgfpathlineto{\pgfqpoint{2.339507in}{2.988006in}}%
\pgfpathlineto{\pgfqpoint{2.353345in}{2.964492in}}%
\pgfpathlineto{\pgfqpoint{2.367172in}{2.941183in}}%
\pgfpathlineto{\pgfqpoint{2.380988in}{2.918077in}}%
\pgfpathlineto{\pgfqpoint{2.389945in}{2.910991in}}%
\pgfpathlineto{\pgfqpoint{2.398881in}{2.904197in}}%
\pgfpathlineto{\pgfqpoint{2.407796in}{2.897692in}}%
\pgfpathlineto{\pgfqpoint{2.416691in}{2.891471in}}%
\pgfpathlineto{\pgfqpoint{2.402929in}{2.914093in}}%
\pgfpathlineto{\pgfqpoint{2.389158in}{2.936918in}}%
\pgfpathlineto{\pgfqpoint{2.375376in}{2.959946in}}%
\pgfpathlineto{\pgfqpoint{2.361583in}{2.983179in}}%
\pgfpathlineto{\pgfqpoint{2.352634in}{2.989877in}}%
\pgfpathlineto{\pgfqpoint{2.343663in}{2.996865in}}%
\pgfpathlineto{\pgfqpoint{2.334672in}{3.004146in}}%
\pgfpathlineto{\pgfqpoint{2.325658in}{3.011726in}}%
\pgfpathclose%
\pgfusepath{fill}%
\end{pgfscope}%
\begin{pgfscope}%
\pgfpathrectangle{\pgfqpoint{1.254980in}{0.150000in}}{\pgfqpoint{5.490039in}{5.490039in}}%
\pgfusepath{clip}%
\pgfsetbuttcap%
\pgfsetroundjoin%
\definecolor{currentfill}{rgb}{0.272594,0.025563,0.353093}%
\pgfsetfillcolor{currentfill}%
\pgfsetfillopacity{0.700000}%
\pgfsetlinewidth{0.000000pt}%
\definecolor{currentstroke}{rgb}{0.000000,0.000000,0.000000}%
\pgfsetstrokecolor{currentstroke}%
\pgfsetdash{}{0pt}%
\pgfpathmoveto{\pgfqpoint{3.715157in}{1.499398in}}%
\pgfpathlineto{\pgfqpoint{3.728664in}{1.491787in}}%
\pgfpathlineto{\pgfqpoint{3.742175in}{1.484299in}}%
\pgfpathlineto{\pgfqpoint{3.755690in}{1.476935in}}%
\pgfpathlineto{\pgfqpoint{3.769208in}{1.469694in}}%
\pgfpathlineto{\pgfqpoint{3.777177in}{1.474608in}}%
\pgfpathlineto{\pgfqpoint{3.785137in}{1.479680in}}%
\pgfpathlineto{\pgfqpoint{3.793089in}{1.484907in}}%
\pgfpathlineto{\pgfqpoint{3.801033in}{1.490286in}}%
\pgfpathlineto{\pgfqpoint{3.787536in}{1.497119in}}%
\pgfpathlineto{\pgfqpoint{3.774042in}{1.504074in}}%
\pgfpathlineto{\pgfqpoint{3.760552in}{1.511154in}}%
\pgfpathlineto{\pgfqpoint{3.747066in}{1.518356in}}%
\pgfpathlineto{\pgfqpoint{3.739101in}{1.513380in}}%
\pgfpathlineto{\pgfqpoint{3.731128in}{1.508559in}}%
\pgfpathlineto{\pgfqpoint{3.723147in}{1.503897in}}%
\pgfpathlineto{\pgfqpoint{3.715157in}{1.499398in}}%
\pgfpathclose%
\pgfusepath{fill}%
\end{pgfscope}%
\begin{pgfscope}%
\pgfpathrectangle{\pgfqpoint{1.254980in}{0.150000in}}{\pgfqpoint{5.490039in}{5.490039in}}%
\pgfusepath{clip}%
\pgfsetbuttcap%
\pgfsetroundjoin%
\definecolor{currentfill}{rgb}{0.163625,0.471133,0.558148}%
\pgfsetfillcolor{currentfill}%
\pgfsetfillopacity{0.700000}%
\pgfsetlinewidth{0.000000pt}%
\definecolor{currentstroke}{rgb}{0.000000,0.000000,0.000000}%
\pgfsetstrokecolor{currentstroke}%
\pgfsetdash{}{0pt}%
\pgfpathmoveto{\pgfqpoint{2.620415in}{2.518047in}}%
\pgfpathlineto{\pgfqpoint{2.634102in}{2.498424in}}%
\pgfpathlineto{\pgfqpoint{2.647782in}{2.478978in}}%
\pgfpathlineto{\pgfqpoint{2.661455in}{2.459708in}}%
\pgfpathlineto{\pgfqpoint{2.675121in}{2.440613in}}%
\pgfpathlineto{\pgfqpoint{2.683843in}{2.435315in}}%
\pgfpathlineto{\pgfqpoint{2.692547in}{2.430295in}}%
\pgfpathlineto{\pgfqpoint{2.701232in}{2.425547in}}%
\pgfpathlineto{\pgfqpoint{2.709900in}{2.421068in}}%
\pgfpathlineto{\pgfqpoint{2.696282in}{2.439683in}}%
\pgfpathlineto{\pgfqpoint{2.682658in}{2.458471in}}%
\pgfpathlineto{\pgfqpoint{2.669027in}{2.477435in}}%
\pgfpathlineto{\pgfqpoint{2.655389in}{2.496574in}}%
\pgfpathlineto{\pgfqpoint{2.646674in}{2.501527in}}%
\pgfpathlineto{\pgfqpoint{2.637940in}{2.506754in}}%
\pgfpathlineto{\pgfqpoint{2.629187in}{2.512259in}}%
\pgfpathlineto{\pgfqpoint{2.620415in}{2.518047in}}%
\pgfpathclose%
\pgfusepath{fill}%
\end{pgfscope}%
\begin{pgfscope}%
\pgfpathrectangle{\pgfqpoint{1.254980in}{0.150000in}}{\pgfqpoint{5.490039in}{5.490039in}}%
\pgfusepath{clip}%
\pgfsetbuttcap%
\pgfsetroundjoin%
\definecolor{currentfill}{rgb}{0.280894,0.078907,0.402329}%
\pgfsetfillcolor{currentfill}%
\pgfsetfillopacity{0.700000}%
\pgfsetlinewidth{0.000000pt}%
\definecolor{currentstroke}{rgb}{0.000000,0.000000,0.000000}%
\pgfsetstrokecolor{currentstroke}%
\pgfsetdash{}{0pt}%
\pgfpathmoveto{\pgfqpoint{3.520960in}{1.589369in}}%
\pgfpathlineto{\pgfqpoint{3.534460in}{1.579811in}}%
\pgfpathlineto{\pgfqpoint{3.547962in}{1.570382in}}%
\pgfpathlineto{\pgfqpoint{3.561466in}{1.561081in}}%
\pgfpathlineto{\pgfqpoint{3.574972in}{1.551909in}}%
\pgfpathlineto{\pgfqpoint{3.583046in}{1.554857in}}%
\pgfpathlineto{\pgfqpoint{3.591110in}{1.557990in}}%
\pgfpathlineto{\pgfqpoint{3.599165in}{1.561304in}}%
\pgfpathlineto{\pgfqpoint{3.607211in}{1.564797in}}%
\pgfpathlineto{\pgfqpoint{3.593730in}{1.573542in}}%
\pgfpathlineto{\pgfqpoint{3.580252in}{1.582414in}}%
\pgfpathlineto{\pgfqpoint{3.566776in}{1.591415in}}%
\pgfpathlineto{\pgfqpoint{3.553302in}{1.600545in}}%
\pgfpathlineto{\pgfqpoint{3.545232in}{1.597473in}}%
\pgfpathlineto{\pgfqpoint{3.537151in}{1.594584in}}%
\pgfpathlineto{\pgfqpoint{3.529060in}{1.591882in}}%
\pgfpathlineto{\pgfqpoint{3.520960in}{1.589369in}}%
\pgfpathclose%
\pgfusepath{fill}%
\end{pgfscope}%
\begin{pgfscope}%
\pgfpathrectangle{\pgfqpoint{1.254980in}{0.150000in}}{\pgfqpoint{5.490039in}{5.490039in}}%
\pgfusepath{clip}%
\pgfsetbuttcap%
\pgfsetroundjoin%
\definecolor{currentfill}{rgb}{0.269308,0.218818,0.509577}%
\pgfsetfillcolor{currentfill}%
\pgfsetfillopacity{0.700000}%
\pgfsetlinewidth{0.000000pt}%
\definecolor{currentstroke}{rgb}{0.000000,0.000000,0.000000}%
\pgfsetstrokecolor{currentstroke}%
\pgfsetdash{}{0pt}%
\pgfpathmoveto{\pgfqpoint{4.904276in}{1.828799in}}%
\pgfpathlineto{\pgfqpoint{4.918147in}{1.831986in}}%
\pgfpathlineto{\pgfqpoint{4.932030in}{1.835284in}}%
\pgfpathlineto{\pgfqpoint{4.945925in}{1.838695in}}%
\pgfpathlineto{\pgfqpoint{4.959833in}{1.842217in}}%
\pgfpathlineto{\pgfqpoint{4.967413in}{1.854874in}}%
\pgfpathlineto{\pgfqpoint{4.974988in}{1.867508in}}%
\pgfpathlineto{\pgfqpoint{4.982559in}{1.880120in}}%
\pgfpathlineto{\pgfqpoint{4.990125in}{1.892706in}}%
\pgfpathlineto{\pgfqpoint{4.976218in}{1.888966in}}%
\pgfpathlineto{\pgfqpoint{4.962324in}{1.885337in}}%
\pgfpathlineto{\pgfqpoint{4.948442in}{1.881820in}}%
\pgfpathlineto{\pgfqpoint{4.934573in}{1.878416in}}%
\pgfpathlineto{\pgfqpoint{4.927006in}{1.866042in}}%
\pgfpathlineto{\pgfqpoint{4.919434in}{1.853646in}}%
\pgfpathlineto{\pgfqpoint{4.911857in}{1.841231in}}%
\pgfpathlineto{\pgfqpoint{4.904276in}{1.828799in}}%
\pgfpathclose%
\pgfusepath{fill}%
\end{pgfscope}%
\begin{pgfscope}%
\pgfpathrectangle{\pgfqpoint{1.254980in}{0.150000in}}{\pgfqpoint{5.490039in}{5.490039in}}%
\pgfusepath{clip}%
\pgfsetbuttcap%
\pgfsetroundjoin%
\definecolor{currentfill}{rgb}{0.277134,0.185228,0.489898}%
\pgfsetfillcolor{currentfill}%
\pgfsetfillopacity{0.700000}%
\pgfsetlinewidth{0.000000pt}%
\definecolor{currentstroke}{rgb}{0.000000,0.000000,0.000000}%
\pgfsetstrokecolor{currentstroke}%
\pgfsetdash{}{0pt}%
\pgfpathmoveto{\pgfqpoint{3.218050in}{1.809341in}}%
\pgfpathlineto{\pgfqpoint{3.231571in}{1.796639in}}%
\pgfpathlineto{\pgfqpoint{3.245090in}{1.784077in}}%
\pgfpathlineto{\pgfqpoint{3.258609in}{1.771655in}}%
\pgfpathlineto{\pgfqpoint{3.272128in}{1.759372in}}%
\pgfpathlineto{\pgfqpoint{3.280396in}{1.759265in}}%
\pgfpathlineto{\pgfqpoint{3.288651in}{1.759381in}}%
\pgfpathlineto{\pgfqpoint{3.296894in}{1.759717in}}%
\pgfpathlineto{\pgfqpoint{3.305125in}{1.760267in}}%
\pgfpathlineto{\pgfqpoint{3.291640in}{1.772098in}}%
\pgfpathlineto{\pgfqpoint{3.278154in}{1.784068in}}%
\pgfpathlineto{\pgfqpoint{3.264669in}{1.796177in}}%
\pgfpathlineto{\pgfqpoint{3.251183in}{1.808425in}}%
\pgfpathlineto{\pgfqpoint{3.242919in}{1.808321in}}%
\pgfpathlineto{\pgfqpoint{3.234642in}{1.808436in}}%
\pgfpathlineto{\pgfqpoint{3.226353in}{1.808775in}}%
\pgfpathlineto{\pgfqpoint{3.218050in}{1.809341in}}%
\pgfpathclose%
\pgfusepath{fill}%
\end{pgfscope}%
\begin{pgfscope}%
\pgfpathrectangle{\pgfqpoint{1.254980in}{0.150000in}}{\pgfqpoint{5.490039in}{5.490039in}}%
\pgfusepath{clip}%
\pgfsetbuttcap%
\pgfsetroundjoin%
\definecolor{currentfill}{rgb}{0.268510,0.009605,0.335427}%
\pgfsetfillcolor{currentfill}%
\pgfsetfillopacity{0.700000}%
\pgfsetlinewidth{0.000000pt}%
\definecolor{currentstroke}{rgb}{0.000000,0.000000,0.000000}%
\pgfsetstrokecolor{currentstroke}%
\pgfsetdash{}{0pt}%
\pgfpathmoveto{\pgfqpoint{4.134633in}{1.442320in}}%
\pgfpathlineto{\pgfqpoint{4.148213in}{1.438758in}}%
\pgfpathlineto{\pgfqpoint{4.161799in}{1.435313in}}%
\pgfpathlineto{\pgfqpoint{4.175392in}{1.431984in}}%
\pgfpathlineto{\pgfqpoint{4.188992in}{1.428770in}}%
\pgfpathlineto{\pgfqpoint{4.196784in}{1.437550in}}%
\pgfpathlineto{\pgfqpoint{4.204571in}{1.446425in}}%
\pgfpathlineto{\pgfqpoint{4.212352in}{1.455393in}}%
\pgfpathlineto{\pgfqpoint{4.220128in}{1.464451in}}%
\pgfpathlineto{\pgfqpoint{4.206539in}{1.467306in}}%
\pgfpathlineto{\pgfqpoint{4.192958in}{1.470278in}}%
\pgfpathlineto{\pgfqpoint{4.179383in}{1.473365in}}%
\pgfpathlineto{\pgfqpoint{4.165816in}{1.476569in}}%
\pgfpathlineto{\pgfqpoint{4.158029in}{1.467863in}}%
\pgfpathlineto{\pgfqpoint{4.150236in}{1.459251in}}%
\pgfpathlineto{\pgfqpoint{4.142438in}{1.450736in}}%
\pgfpathlineto{\pgfqpoint{4.134633in}{1.442320in}}%
\pgfpathclose%
\pgfusepath{fill}%
\end{pgfscope}%
\begin{pgfscope}%
\pgfpathrectangle{\pgfqpoint{1.254980in}{0.150000in}}{\pgfqpoint{5.490039in}{5.490039in}}%
\pgfusepath{clip}%
\pgfsetbuttcap%
\pgfsetroundjoin%
\definecolor{currentfill}{rgb}{0.188923,0.410910,0.556326}%
\pgfsetfillcolor{currentfill}%
\pgfsetfillopacity{0.700000}%
\pgfsetlinewidth{0.000000pt}%
\definecolor{currentstroke}{rgb}{0.000000,0.000000,0.000000}%
\pgfsetstrokecolor{currentstroke}%
\pgfsetdash{}{0pt}%
\pgfpathmoveto{\pgfqpoint{5.394469in}{2.274485in}}%
\pgfpathlineto{\pgfqpoint{5.408610in}{2.281106in}}%
\pgfpathlineto{\pgfqpoint{5.422765in}{2.287839in}}%
\pgfpathlineto{\pgfqpoint{5.436936in}{2.294685in}}%
\pgfpathlineto{\pgfqpoint{5.451122in}{2.301644in}}%
\pgfpathlineto{\pgfqpoint{5.458551in}{2.313754in}}%
\pgfpathlineto{\pgfqpoint{5.465974in}{2.325790in}}%
\pgfpathlineto{\pgfqpoint{5.473390in}{2.337751in}}%
\pgfpathlineto{\pgfqpoint{5.480800in}{2.349637in}}%
\pgfpathlineto{\pgfqpoint{5.466615in}{2.342573in}}%
\pgfpathlineto{\pgfqpoint{5.452446in}{2.335622in}}%
\pgfpathlineto{\pgfqpoint{5.438291in}{2.328784in}}%
\pgfpathlineto{\pgfqpoint{5.424153in}{2.322059in}}%
\pgfpathlineto{\pgfqpoint{5.416741in}{2.310271in}}%
\pgfpathlineto{\pgfqpoint{5.409323in}{2.298413in}}%
\pgfpathlineto{\pgfqpoint{5.401899in}{2.286484in}}%
\pgfpathlineto{\pgfqpoint{5.394469in}{2.274485in}}%
\pgfpathclose%
\pgfusepath{fill}%
\end{pgfscope}%
\begin{pgfscope}%
\pgfpathrectangle{\pgfqpoint{1.254980in}{0.150000in}}{\pgfqpoint{5.490039in}{5.490039in}}%
\pgfusepath{clip}%
\pgfsetbuttcap%
\pgfsetroundjoin%
\definecolor{currentfill}{rgb}{0.223925,0.334994,0.548053}%
\pgfsetfillcolor{currentfill}%
\pgfsetfillopacity{0.700000}%
\pgfsetlinewidth{0.000000pt}%
\definecolor{currentstroke}{rgb}{0.000000,0.000000,0.000000}%
\pgfsetstrokecolor{currentstroke}%
\pgfsetdash{}{0pt}%
\pgfpathmoveto{\pgfqpoint{5.192247in}{2.079707in}}%
\pgfpathlineto{\pgfqpoint{5.206273in}{2.085016in}}%
\pgfpathlineto{\pgfqpoint{5.220312in}{2.090437in}}%
\pgfpathlineto{\pgfqpoint{5.234366in}{2.095970in}}%
\pgfpathlineto{\pgfqpoint{5.248434in}{2.101615in}}%
\pgfpathlineto{\pgfqpoint{5.255933in}{2.114231in}}%
\pgfpathlineto{\pgfqpoint{5.263427in}{2.126791in}}%
\pgfpathlineto{\pgfqpoint{5.270916in}{2.139295in}}%
\pgfpathlineto{\pgfqpoint{5.278399in}{2.151741in}}%
\pgfpathlineto{\pgfqpoint{5.264331in}{2.145941in}}%
\pgfpathlineto{\pgfqpoint{5.250278in}{2.140254in}}%
\pgfpathlineto{\pgfqpoint{5.236239in}{2.134679in}}%
\pgfpathlineto{\pgfqpoint{5.222214in}{2.129217in}}%
\pgfpathlineto{\pgfqpoint{5.214731in}{2.116918in}}%
\pgfpathlineto{\pgfqpoint{5.207242in}{2.104567in}}%
\pgfpathlineto{\pgfqpoint{5.199747in}{2.092163in}}%
\pgfpathlineto{\pgfqpoint{5.192247in}{2.079707in}}%
\pgfpathclose%
\pgfusepath{fill}%
\end{pgfscope}%
\begin{pgfscope}%
\pgfpathrectangle{\pgfqpoint{1.254980in}{0.150000in}}{\pgfqpoint{5.490039in}{5.490039in}}%
\pgfusepath{clip}%
\pgfsetbuttcap%
\pgfsetroundjoin%
\definecolor{currentfill}{rgb}{0.151918,0.500685,0.557587}%
\pgfsetfillcolor{currentfill}%
\pgfsetfillopacity{0.700000}%
\pgfsetlinewidth{0.000000pt}%
\definecolor{currentstroke}{rgb}{0.000000,0.000000,0.000000}%
\pgfsetstrokecolor{currentstroke}%
\pgfsetdash{}{0pt}%
\pgfpathmoveto{\pgfqpoint{2.565594in}{2.598328in}}%
\pgfpathlineto{\pgfqpoint{2.579311in}{2.577987in}}%
\pgfpathlineto{\pgfqpoint{2.593020in}{2.557827in}}%
\pgfpathlineto{\pgfqpoint{2.606721in}{2.537848in}}%
\pgfpathlineto{\pgfqpoint{2.620415in}{2.518047in}}%
\pgfpathlineto{\pgfqpoint{2.629187in}{2.512259in}}%
\pgfpathlineto{\pgfqpoint{2.637940in}{2.506754in}}%
\pgfpathlineto{\pgfqpoint{2.646674in}{2.501527in}}%
\pgfpathlineto{\pgfqpoint{2.655389in}{2.496574in}}%
\pgfpathlineto{\pgfqpoint{2.641745in}{2.515890in}}%
\pgfpathlineto{\pgfqpoint{2.628094in}{2.535385in}}%
\pgfpathlineto{\pgfqpoint{2.614436in}{2.555058in}}%
\pgfpathlineto{\pgfqpoint{2.600770in}{2.574912in}}%
\pgfpathlineto{\pgfqpoint{2.592005in}{2.580343in}}%
\pgfpathlineto{\pgfqpoint{2.583221in}{2.586053in}}%
\pgfpathlineto{\pgfqpoint{2.574417in}{2.592047in}}%
\pgfpathlineto{\pgfqpoint{2.565594in}{2.598328in}}%
\pgfpathclose%
\pgfusepath{fill}%
\end{pgfscope}%
\begin{pgfscope}%
\pgfpathrectangle{\pgfqpoint{1.254980in}{0.150000in}}{\pgfqpoint{5.490039in}{5.490039in}}%
\pgfusepath{clip}%
\pgfsetbuttcap%
\pgfsetroundjoin%
\definecolor{currentfill}{rgb}{0.280868,0.160771,0.472899}%
\pgfsetfillcolor{currentfill}%
\pgfsetfillopacity{0.700000}%
\pgfsetlinewidth{0.000000pt}%
\definecolor{currentstroke}{rgb}{0.000000,0.000000,0.000000}%
\pgfsetstrokecolor{currentstroke}%
\pgfsetdash{}{0pt}%
\pgfpathmoveto{\pgfqpoint{3.272128in}{1.759372in}}%
\pgfpathlineto{\pgfqpoint{3.285646in}{1.747227in}}%
\pgfpathlineto{\pgfqpoint{3.299164in}{1.735220in}}%
\pgfpathlineto{\pgfqpoint{3.312682in}{1.723350in}}%
\pgfpathlineto{\pgfqpoint{3.326199in}{1.711617in}}%
\pgfpathlineto{\pgfqpoint{3.334434in}{1.711969in}}%
\pgfpathlineto{\pgfqpoint{3.342656in}{1.712539in}}%
\pgfpathlineto{\pgfqpoint{3.350867in}{1.713323in}}%
\pgfpathlineto{\pgfqpoint{3.359066in}{1.714318in}}%
\pgfpathlineto{\pgfqpoint{3.345580in}{1.725600in}}%
\pgfpathlineto{\pgfqpoint{3.332095in}{1.737019in}}%
\pgfpathlineto{\pgfqpoint{3.318610in}{1.748574in}}%
\pgfpathlineto{\pgfqpoint{3.305125in}{1.760267in}}%
\pgfpathlineto{\pgfqpoint{3.296894in}{1.759717in}}%
\pgfpathlineto{\pgfqpoint{3.288651in}{1.759381in}}%
\pgfpathlineto{\pgfqpoint{3.280396in}{1.759265in}}%
\pgfpathlineto{\pgfqpoint{3.272128in}{1.759372in}}%
\pgfpathclose%
\pgfusepath{fill}%
\end{pgfscope}%
\begin{pgfscope}%
\pgfpathrectangle{\pgfqpoint{1.254980in}{0.150000in}}{\pgfqpoint{5.490039in}{5.490039in}}%
\pgfusepath{clip}%
\pgfsetbuttcap%
\pgfsetroundjoin%
\definecolor{currentfill}{rgb}{0.281924,0.089666,0.412415}%
\pgfsetfillcolor{currentfill}%
\pgfsetfillopacity{0.700000}%
\pgfsetlinewidth{0.000000pt}%
\definecolor{currentstroke}{rgb}{0.000000,0.000000,0.000000}%
\pgfsetstrokecolor{currentstroke}%
\pgfsetdash{}{0pt}%
\pgfpathmoveto{\pgfqpoint{4.531150in}{1.564718in}}%
\pgfpathlineto{\pgfqpoint{4.544859in}{1.564783in}}%
\pgfpathlineto{\pgfqpoint{4.558579in}{1.564961in}}%
\pgfpathlineto{\pgfqpoint{4.572308in}{1.565251in}}%
\pgfpathlineto{\pgfqpoint{4.586048in}{1.565654in}}%
\pgfpathlineto{\pgfqpoint{4.593725in}{1.577139in}}%
\pgfpathlineto{\pgfqpoint{4.601397in}{1.588659in}}%
\pgfpathlineto{\pgfqpoint{4.609066in}{1.600209in}}%
\pgfpathlineto{\pgfqpoint{4.616729in}{1.611789in}}%
\pgfpathlineto{\pgfqpoint{4.602994in}{1.611091in}}%
\pgfpathlineto{\pgfqpoint{4.589270in}{1.610505in}}%
\pgfpathlineto{\pgfqpoint{4.575555in}{1.610031in}}%
\pgfpathlineto{\pgfqpoint{4.561850in}{1.609671in}}%
\pgfpathlineto{\pgfqpoint{4.554182in}{1.598381in}}%
\pgfpathlineto{\pgfqpoint{4.546509in}{1.587124in}}%
\pgfpathlineto{\pgfqpoint{4.538831in}{1.575902in}}%
\pgfpathlineto{\pgfqpoint{4.531150in}{1.564718in}}%
\pgfpathclose%
\pgfusepath{fill}%
\end{pgfscope}%
\begin{pgfscope}%
\pgfpathrectangle{\pgfqpoint{1.254980in}{0.150000in}}{\pgfqpoint{5.490039in}{5.490039in}}%
\pgfusepath{clip}%
\pgfsetbuttcap%
\pgfsetroundjoin%
\definecolor{currentfill}{rgb}{0.258965,0.251537,0.524736}%
\pgfsetfillcolor{currentfill}%
\pgfsetfillopacity{0.700000}%
\pgfsetlinewidth{0.000000pt}%
\definecolor{currentstroke}{rgb}{0.000000,0.000000,0.000000}%
\pgfsetstrokecolor{currentstroke}%
\pgfsetdash{}{0pt}%
\pgfpathmoveto{\pgfqpoint{4.990125in}{1.892706in}}%
\pgfpathlineto{\pgfqpoint{5.004044in}{1.896559in}}%
\pgfpathlineto{\pgfqpoint{5.017977in}{1.900524in}}%
\pgfpathlineto{\pgfqpoint{5.031922in}{1.904601in}}%
\pgfpathlineto{\pgfqpoint{5.045881in}{1.908789in}}%
\pgfpathlineto{\pgfqpoint{5.053442in}{1.921558in}}%
\pgfpathlineto{\pgfqpoint{5.060997in}{1.934294in}}%
\pgfpathlineto{\pgfqpoint{5.068549in}{1.946997in}}%
\pgfpathlineto{\pgfqpoint{5.076095in}{1.959666in}}%
\pgfpathlineto{\pgfqpoint{5.062137in}{1.955274in}}%
\pgfpathlineto{\pgfqpoint{5.048192in}{1.950995in}}%
\pgfpathlineto{\pgfqpoint{5.034260in}{1.946828in}}%
\pgfpathlineto{\pgfqpoint{5.020341in}{1.942773in}}%
\pgfpathlineto{\pgfqpoint{5.012794in}{1.930301in}}%
\pgfpathlineto{\pgfqpoint{5.005243in}{1.917799in}}%
\pgfpathlineto{\pgfqpoint{4.997686in}{1.905266in}}%
\pgfpathlineto{\pgfqpoint{4.990125in}{1.892706in}}%
\pgfpathclose%
\pgfusepath{fill}%
\end{pgfscope}%
\begin{pgfscope}%
\pgfpathrectangle{\pgfqpoint{1.254980in}{0.150000in}}{\pgfqpoint{5.490039in}{5.490039in}}%
\pgfusepath{clip}%
\pgfsetbuttcap%
\pgfsetroundjoin%
\definecolor{currentfill}{rgb}{0.278791,0.062145,0.386592}%
\pgfsetfillcolor{currentfill}%
\pgfsetfillopacity{0.700000}%
\pgfsetlinewidth{0.000000pt}%
\definecolor{currentstroke}{rgb}{0.000000,0.000000,0.000000}%
\pgfsetstrokecolor{currentstroke}%
\pgfsetdash{}{0pt}%
\pgfpathmoveto{\pgfqpoint{4.445610in}{1.522525in}}%
\pgfpathlineto{\pgfqpoint{4.459288in}{1.521825in}}%
\pgfpathlineto{\pgfqpoint{4.472975in}{1.521240in}}%
\pgfpathlineto{\pgfqpoint{4.486672in}{1.520767in}}%
\pgfpathlineto{\pgfqpoint{4.500378in}{1.520408in}}%
\pgfpathlineto{\pgfqpoint{4.508077in}{1.531417in}}%
\pgfpathlineto{\pgfqpoint{4.515773in}{1.542473in}}%
\pgfpathlineto{\pgfqpoint{4.523463in}{1.553574in}}%
\pgfpathlineto{\pgfqpoint{4.531150in}{1.564718in}}%
\pgfpathlineto{\pgfqpoint{4.517449in}{1.564767in}}%
\pgfpathlineto{\pgfqpoint{4.503759in}{1.564928in}}%
\pgfpathlineto{\pgfqpoint{4.490077in}{1.565203in}}%
\pgfpathlineto{\pgfqpoint{4.476406in}{1.565591in}}%
\pgfpathlineto{\pgfqpoint{4.468714in}{1.554752in}}%
\pgfpathlineto{\pgfqpoint{4.461017in}{1.543960in}}%
\pgfpathlineto{\pgfqpoint{4.453316in}{1.533217in}}%
\pgfpathlineto{\pgfqpoint{4.445610in}{1.522525in}}%
\pgfpathclose%
\pgfusepath{fill}%
\end{pgfscope}%
\begin{pgfscope}%
\pgfpathrectangle{\pgfqpoint{1.254980in}{0.150000in}}{\pgfqpoint{5.490039in}{5.490039in}}%
\pgfusepath{clip}%
\pgfsetbuttcap%
\pgfsetroundjoin%
\definecolor{currentfill}{rgb}{0.283197,0.115680,0.436115}%
\pgfsetfillcolor{currentfill}%
\pgfsetfillopacity{0.700000}%
\pgfsetlinewidth{0.000000pt}%
\definecolor{currentstroke}{rgb}{0.000000,0.000000,0.000000}%
\pgfsetstrokecolor{currentstroke}%
\pgfsetdash{}{0pt}%
\pgfpathmoveto{\pgfqpoint{4.616729in}{1.611789in}}%
\pgfpathlineto{\pgfqpoint{4.630475in}{1.612601in}}%
\pgfpathlineto{\pgfqpoint{4.644230in}{1.613525in}}%
\pgfpathlineto{\pgfqpoint{4.657996in}{1.614561in}}%
\pgfpathlineto{\pgfqpoint{4.671773in}{1.615709in}}%
\pgfpathlineto{\pgfqpoint{4.679429in}{1.627603in}}%
\pgfpathlineto{\pgfqpoint{4.687080in}{1.639517in}}%
\pgfpathlineto{\pgfqpoint{4.694728in}{1.651449in}}%
\pgfpathlineto{\pgfqpoint{4.702370in}{1.663399in}}%
\pgfpathlineto{\pgfqpoint{4.688597in}{1.661969in}}%
\pgfpathlineto{\pgfqpoint{4.674835in}{1.660653in}}%
\pgfpathlineto{\pgfqpoint{4.661083in}{1.659449in}}%
\pgfpathlineto{\pgfqpoint{4.647341in}{1.658357in}}%
\pgfpathlineto{\pgfqpoint{4.639695in}{1.646682in}}%
\pgfpathlineto{\pgfqpoint{4.632044in}{1.635028in}}%
\pgfpathlineto{\pgfqpoint{4.624389in}{1.623396in}}%
\pgfpathlineto{\pgfqpoint{4.616729in}{1.611789in}}%
\pgfpathclose%
\pgfusepath{fill}%
\end{pgfscope}%
\begin{pgfscope}%
\pgfpathrectangle{\pgfqpoint{1.254980in}{0.150000in}}{\pgfqpoint{5.490039in}{5.490039in}}%
\pgfusepath{clip}%
\pgfsetbuttcap%
\pgfsetroundjoin%
\definecolor{currentfill}{rgb}{0.267004,0.004874,0.329415}%
\pgfsetfillcolor{currentfill}%
\pgfsetfillopacity{0.700000}%
\pgfsetlinewidth{0.000000pt}%
\definecolor{currentstroke}{rgb}{0.000000,0.000000,0.000000}%
\pgfsetstrokecolor{currentstroke}%
\pgfsetdash{}{0pt}%
\pgfpathmoveto{\pgfqpoint{3.909172in}{1.440008in}}%
\pgfpathlineto{\pgfqpoint{3.922710in}{1.434267in}}%
\pgfpathlineto{\pgfqpoint{3.936254in}{1.428645in}}%
\pgfpathlineto{\pgfqpoint{3.949802in}{1.423142in}}%
\pgfpathlineto{\pgfqpoint{3.963356in}{1.417758in}}%
\pgfpathlineto{\pgfqpoint{3.971239in}{1.424476in}}%
\pgfpathlineto{\pgfqpoint{3.979114in}{1.431326in}}%
\pgfpathlineto{\pgfqpoint{3.986983in}{1.438305in}}%
\pgfpathlineto{\pgfqpoint{3.994846in}{1.445410in}}%
\pgfpathlineto{\pgfqpoint{3.981308in}{1.450403in}}%
\pgfpathlineto{\pgfqpoint{3.967776in}{1.455515in}}%
\pgfpathlineto{\pgfqpoint{3.954249in}{1.460747in}}%
\pgfpathlineto{\pgfqpoint{3.940728in}{1.466098in}}%
\pgfpathlineto{\pgfqpoint{3.932849in}{1.459377in}}%
\pgfpathlineto{\pgfqpoint{3.924963in}{1.452787in}}%
\pgfpathlineto{\pgfqpoint{3.917071in}{1.446329in}}%
\pgfpathlineto{\pgfqpoint{3.909172in}{1.440008in}}%
\pgfpathclose%
\pgfusepath{fill}%
\end{pgfscope}%
\begin{pgfscope}%
\pgfpathrectangle{\pgfqpoint{1.254980in}{0.150000in}}{\pgfqpoint{5.490039in}{5.490039in}}%
\pgfusepath{clip}%
\pgfsetbuttcap%
\pgfsetroundjoin%
\definecolor{currentfill}{rgb}{0.175841,0.441290,0.557685}%
\pgfsetfillcolor{currentfill}%
\pgfsetfillopacity{0.700000}%
\pgfsetlinewidth{0.000000pt}%
\definecolor{currentstroke}{rgb}{0.000000,0.000000,0.000000}%
\pgfsetstrokecolor{currentstroke}%
\pgfsetdash{}{0pt}%
\pgfpathmoveto{\pgfqpoint{5.480800in}{2.349637in}}%
\pgfpathlineto{\pgfqpoint{5.495001in}{2.356813in}}%
\pgfpathlineto{\pgfqpoint{5.509218in}{2.364102in}}%
\pgfpathlineto{\pgfqpoint{5.523450in}{2.371504in}}%
\pgfpathlineto{\pgfqpoint{5.530853in}{2.383385in}}%
\pgfpathlineto{\pgfqpoint{5.538249in}{2.395186in}}%
\pgfpathlineto{\pgfqpoint{5.545639in}{2.406907in}}%
\pgfpathlineto{\pgfqpoint{5.553022in}{2.418547in}}%
\pgfpathlineto{\pgfqpoint{5.538791in}{2.411057in}}%
\pgfpathlineto{\pgfqpoint{5.524576in}{2.403680in}}%
\pgfpathlineto{\pgfqpoint{5.510377in}{2.396416in}}%
\pgfpathlineto{\pgfqpoint{5.502992in}{2.384837in}}%
\pgfpathlineto{\pgfqpoint{5.495601in}{2.373180in}}%
\pgfpathlineto{\pgfqpoint{5.488204in}{2.361447in}}%
\pgfpathlineto{\pgfqpoint{5.480800in}{2.349637in}}%
\pgfpathclose%
\pgfusepath{fill}%
\end{pgfscope}%
\begin{pgfscope}%
\pgfpathrectangle{\pgfqpoint{1.254980in}{0.150000in}}{\pgfqpoint{5.490039in}{5.490039in}}%
\pgfusepath{clip}%
\pgfsetbuttcap%
\pgfsetroundjoin%
\definecolor{currentfill}{rgb}{0.278791,0.062145,0.386592}%
\pgfsetfillcolor{currentfill}%
\pgfsetfillopacity{0.700000}%
\pgfsetlinewidth{0.000000pt}%
\definecolor{currentstroke}{rgb}{0.000000,0.000000,0.000000}%
\pgfsetstrokecolor{currentstroke}%
\pgfsetdash{}{0pt}%
\pgfpathmoveto{\pgfqpoint{3.574972in}{1.551909in}}%
\pgfpathlineto{\pgfqpoint{3.588480in}{1.542865in}}%
\pgfpathlineto{\pgfqpoint{3.601991in}{1.533948in}}%
\pgfpathlineto{\pgfqpoint{3.615505in}{1.525158in}}%
\pgfpathlineto{\pgfqpoint{3.629020in}{1.516495in}}%
\pgfpathlineto{\pgfqpoint{3.637069in}{1.519876in}}%
\pgfpathlineto{\pgfqpoint{3.645108in}{1.523438in}}%
\pgfpathlineto{\pgfqpoint{3.653139in}{1.527177in}}%
\pgfpathlineto{\pgfqpoint{3.661160in}{1.531091in}}%
\pgfpathlineto{\pgfqpoint{3.647669in}{1.539328in}}%
\pgfpathlineto{\pgfqpoint{3.634180in}{1.547691in}}%
\pgfpathlineto{\pgfqpoint{3.620694in}{1.556180in}}%
\pgfpathlineto{\pgfqpoint{3.607211in}{1.564797in}}%
\pgfpathlineto{\pgfqpoint{3.599165in}{1.561304in}}%
\pgfpathlineto{\pgfqpoint{3.591110in}{1.557990in}}%
\pgfpathlineto{\pgfqpoint{3.583046in}{1.554857in}}%
\pgfpathlineto{\pgfqpoint{3.574972in}{1.551909in}}%
\pgfpathclose%
\pgfusepath{fill}%
\end{pgfscope}%
\begin{pgfscope}%
\pgfpathrectangle{\pgfqpoint{1.254980in}{0.150000in}}{\pgfqpoint{5.490039in}{5.490039in}}%
\pgfusepath{clip}%
\pgfsetbuttcap%
\pgfsetroundjoin%
\definecolor{currentfill}{rgb}{0.157851,0.683765,0.501686}%
\pgfsetfillcolor{currentfill}%
\pgfsetfillopacity{0.700000}%
\pgfsetlinewidth{0.000000pt}%
\definecolor{currentstroke}{rgb}{0.000000,0.000000,0.000000}%
\pgfsetstrokecolor{currentstroke}%
\pgfsetdash{}{0pt}%
\pgfpathmoveto{\pgfqpoint{2.270148in}{3.108696in}}%
\pgfpathlineto{\pgfqpoint{2.284043in}{3.084137in}}%
\pgfpathlineto{\pgfqpoint{2.297926in}{3.059790in}}%
\pgfpathlineto{\pgfqpoint{2.311798in}{3.035654in}}%
\pgfpathlineto{\pgfqpoint{2.325658in}{3.011726in}}%
\pgfpathlineto{\pgfqpoint{2.334672in}{3.004146in}}%
\pgfpathlineto{\pgfqpoint{2.343663in}{2.996865in}}%
\pgfpathlineto{\pgfqpoint{2.352634in}{2.989877in}}%
\pgfpathlineto{\pgfqpoint{2.361583in}{2.983179in}}%
\pgfpathlineto{\pgfqpoint{2.347780in}{3.006618in}}%
\pgfpathlineto{\pgfqpoint{2.333965in}{3.030265in}}%
\pgfpathlineto{\pgfqpoint{2.320140in}{3.054121in}}%
\pgfpathlineto{\pgfqpoint{2.306303in}{3.078188in}}%
\pgfpathlineto{\pgfqpoint{2.297297in}{3.085368in}}%
\pgfpathlineto{\pgfqpoint{2.288270in}{3.092843in}}%
\pgfpathlineto{\pgfqpoint{2.279220in}{3.100618in}}%
\pgfpathlineto{\pgfqpoint{2.270148in}{3.108696in}}%
\pgfpathclose%
\pgfusepath{fill}%
\end{pgfscope}%
\begin{pgfscope}%
\pgfpathrectangle{\pgfqpoint{1.254980in}{0.150000in}}{\pgfqpoint{5.490039in}{5.490039in}}%
\pgfusepath{clip}%
\pgfsetbuttcap%
\pgfsetroundjoin%
\definecolor{currentfill}{rgb}{0.274952,0.037752,0.364543}%
\pgfsetfillcolor{currentfill}%
\pgfsetfillopacity{0.700000}%
\pgfsetlinewidth{0.000000pt}%
\definecolor{currentstroke}{rgb}{0.000000,0.000000,0.000000}%
\pgfsetstrokecolor{currentstroke}%
\pgfsetdash{}{0pt}%
\pgfpathmoveto{\pgfqpoint{4.360088in}{1.485558in}}%
\pgfpathlineto{\pgfqpoint{4.373738in}{1.484078in}}%
\pgfpathlineto{\pgfqpoint{4.387397in}{1.482712in}}%
\pgfpathlineto{\pgfqpoint{4.401064in}{1.481459in}}%
\pgfpathlineto{\pgfqpoint{4.414740in}{1.480320in}}%
\pgfpathlineto{\pgfqpoint{4.422465in}{1.490781in}}%
\pgfpathlineto{\pgfqpoint{4.430185in}{1.501304in}}%
\pgfpathlineto{\pgfqpoint{4.437900in}{1.511886in}}%
\pgfpathlineto{\pgfqpoint{4.445610in}{1.522525in}}%
\pgfpathlineto{\pgfqpoint{4.431941in}{1.523337in}}%
\pgfpathlineto{\pgfqpoint{4.418281in}{1.524263in}}%
\pgfpathlineto{\pgfqpoint{4.404629in}{1.525303in}}%
\pgfpathlineto{\pgfqpoint{4.390987in}{1.526457in}}%
\pgfpathlineto{\pgfqpoint{4.383269in}{1.516139in}}%
\pgfpathlineto{\pgfqpoint{4.375547in}{1.505882in}}%
\pgfpathlineto{\pgfqpoint{4.367820in}{1.495688in}}%
\pgfpathlineto{\pgfqpoint{4.360088in}{1.485558in}}%
\pgfpathclose%
\pgfusepath{fill}%
\end{pgfscope}%
\begin{pgfscope}%
\pgfpathrectangle{\pgfqpoint{1.254980in}{0.150000in}}{\pgfqpoint{5.490039in}{5.490039in}}%
\pgfusepath{clip}%
\pgfsetbuttcap%
\pgfsetroundjoin%
\definecolor{currentfill}{rgb}{0.139147,0.533812,0.555298}%
\pgfsetfillcolor{currentfill}%
\pgfsetfillopacity{0.700000}%
\pgfsetlinewidth{0.000000pt}%
\definecolor{currentstroke}{rgb}{0.000000,0.000000,0.000000}%
\pgfsetstrokecolor{currentstroke}%
\pgfsetdash{}{0pt}%
\pgfpathmoveto{\pgfqpoint{2.510648in}{2.681530in}}%
\pgfpathlineto{\pgfqpoint{2.524397in}{2.660452in}}%
\pgfpathlineto{\pgfqpoint{2.538137in}{2.639560in}}%
\pgfpathlineto{\pgfqpoint{2.551870in}{2.618852in}}%
\pgfpathlineto{\pgfqpoint{2.565594in}{2.598328in}}%
\pgfpathlineto{\pgfqpoint{2.574417in}{2.592047in}}%
\pgfpathlineto{\pgfqpoint{2.583221in}{2.586053in}}%
\pgfpathlineto{\pgfqpoint{2.592005in}{2.580343in}}%
\pgfpathlineto{\pgfqpoint{2.600770in}{2.574912in}}%
\pgfpathlineto{\pgfqpoint{2.587097in}{2.594948in}}%
\pgfpathlineto{\pgfqpoint{2.573417in}{2.615166in}}%
\pgfpathlineto{\pgfqpoint{2.559728in}{2.635568in}}%
\pgfpathlineto{\pgfqpoint{2.546032in}{2.656155in}}%
\pgfpathlineto{\pgfqpoint{2.537216in}{2.662067in}}%
\pgfpathlineto{\pgfqpoint{2.528380in}{2.668265in}}%
\pgfpathlineto{\pgfqpoint{2.519524in}{2.674751in}}%
\pgfpathlineto{\pgfqpoint{2.510648in}{2.681530in}}%
\pgfpathclose%
\pgfusepath{fill}%
\end{pgfscope}%
\begin{pgfscope}%
\pgfpathrectangle{\pgfqpoint{1.254980in}{0.150000in}}{\pgfqpoint{5.490039in}{5.490039in}}%
\pgfusepath{clip}%
\pgfsetbuttcap%
\pgfsetroundjoin%
\definecolor{currentfill}{rgb}{0.282623,0.140926,0.457517}%
\pgfsetfillcolor{currentfill}%
\pgfsetfillopacity{0.700000}%
\pgfsetlinewidth{0.000000pt}%
\definecolor{currentstroke}{rgb}{0.000000,0.000000,0.000000}%
\pgfsetstrokecolor{currentstroke}%
\pgfsetdash{}{0pt}%
\pgfpathmoveto{\pgfqpoint{4.702370in}{1.663399in}}%
\pgfpathlineto{\pgfqpoint{4.716155in}{1.664940in}}%
\pgfpathlineto{\pgfqpoint{4.729950in}{1.666594in}}%
\pgfpathlineto{\pgfqpoint{4.743756in}{1.668359in}}%
\pgfpathlineto{\pgfqpoint{4.757573in}{1.670237in}}%
\pgfpathlineto{\pgfqpoint{4.765209in}{1.682472in}}%
\pgfpathlineto{\pgfqpoint{4.772841in}{1.694714in}}%
\pgfpathlineto{\pgfqpoint{4.780468in}{1.706964in}}%
\pgfpathlineto{\pgfqpoint{4.788091in}{1.719217in}}%
\pgfpathlineto{\pgfqpoint{4.774276in}{1.717074in}}%
\pgfpathlineto{\pgfqpoint{4.760472in}{1.715044in}}%
\pgfpathlineto{\pgfqpoint{4.746680in}{1.713125in}}%
\pgfpathlineto{\pgfqpoint{4.732898in}{1.711319in}}%
\pgfpathlineto{\pgfqpoint{4.725273in}{1.699324in}}%
\pgfpathlineto{\pgfqpoint{4.717643in}{1.687338in}}%
\pgfpathlineto{\pgfqpoint{4.710009in}{1.675362in}}%
\pgfpathlineto{\pgfqpoint{4.702370in}{1.663399in}}%
\pgfpathclose%
\pgfusepath{fill}%
\end{pgfscope}%
\begin{pgfscope}%
\pgfpathrectangle{\pgfqpoint{1.254980in}{0.150000in}}{\pgfqpoint{5.490039in}{5.490039in}}%
\pgfusepath{clip}%
\pgfsetbuttcap%
\pgfsetroundjoin%
\definecolor{currentfill}{rgb}{0.271305,0.019942,0.347269}%
\pgfsetfillcolor{currentfill}%
\pgfsetfillopacity{0.700000}%
\pgfsetlinewidth{0.000000pt}%
\definecolor{currentstroke}{rgb}{0.000000,0.000000,0.000000}%
\pgfsetstrokecolor{currentstroke}%
\pgfsetdash{}{0pt}%
\pgfpathmoveto{\pgfqpoint{3.769208in}{1.469694in}}%
\pgfpathlineto{\pgfqpoint{3.782731in}{1.462576in}}%
\pgfpathlineto{\pgfqpoint{3.796257in}{1.455580in}}%
\pgfpathlineto{\pgfqpoint{3.809787in}{1.448707in}}%
\pgfpathlineto{\pgfqpoint{3.823321in}{1.441955in}}%
\pgfpathlineto{\pgfqpoint{3.831269in}{1.447282in}}%
\pgfpathlineto{\pgfqpoint{3.839209in}{1.452764in}}%
\pgfpathlineto{\pgfqpoint{3.847141in}{1.458397in}}%
\pgfpathlineto{\pgfqpoint{3.855066in}{1.464177in}}%
\pgfpathlineto{\pgfqpoint{3.841551in}{1.470522in}}%
\pgfpathlineto{\pgfqpoint{3.828041in}{1.476988in}}%
\pgfpathlineto{\pgfqpoint{3.814535in}{1.483575in}}%
\pgfpathlineto{\pgfqpoint{3.801033in}{1.490286in}}%
\pgfpathlineto{\pgfqpoint{3.793089in}{1.484907in}}%
\pgfpathlineto{\pgfqpoint{3.785137in}{1.479680in}}%
\pgfpathlineto{\pgfqpoint{3.777177in}{1.474608in}}%
\pgfpathlineto{\pgfqpoint{3.769208in}{1.469694in}}%
\pgfpathclose%
\pgfusepath{fill}%
\end{pgfscope}%
\begin{pgfscope}%
\pgfpathrectangle{\pgfqpoint{1.254980in}{0.150000in}}{\pgfqpoint{5.490039in}{5.490039in}}%
\pgfusepath{clip}%
\pgfsetbuttcap%
\pgfsetroundjoin%
\definecolor{currentfill}{rgb}{0.267004,0.004874,0.329415}%
\pgfsetfillcolor{currentfill}%
\pgfsetfillopacity{0.700000}%
\pgfsetlinewidth{0.000000pt}%
\definecolor{currentstroke}{rgb}{0.000000,0.000000,0.000000}%
\pgfsetstrokecolor{currentstroke}%
\pgfsetdash{}{0pt}%
\pgfpathmoveto{\pgfqpoint{4.049054in}{1.426620in}}%
\pgfpathlineto{\pgfqpoint{4.062621in}{1.422217in}}%
\pgfpathlineto{\pgfqpoint{4.076194in}{1.417932in}}%
\pgfpathlineto{\pgfqpoint{4.089774in}{1.413763in}}%
\pgfpathlineto{\pgfqpoint{4.103359in}{1.409711in}}%
\pgfpathlineto{\pgfqpoint{4.111187in}{1.417699in}}%
\pgfpathlineto{\pgfqpoint{4.119008in}{1.425798in}}%
\pgfpathlineto{\pgfqpoint{4.126824in}{1.434006in}}%
\pgfpathlineto{\pgfqpoint{4.134633in}{1.442320in}}%
\pgfpathlineto{\pgfqpoint{4.121061in}{1.445998in}}%
\pgfpathlineto{\pgfqpoint{4.107495in}{1.449792in}}%
\pgfpathlineto{\pgfqpoint{4.093935in}{1.453704in}}%
\pgfpathlineto{\pgfqpoint{4.080382in}{1.457733in}}%
\pgfpathlineto{\pgfqpoint{4.072559in}{1.449788in}}%
\pgfpathlineto{\pgfqpoint{4.064730in}{1.441951in}}%
\pgfpathlineto{\pgfqpoint{4.056895in}{1.434228in}}%
\pgfpathlineto{\pgfqpoint{4.049054in}{1.426620in}}%
\pgfpathclose%
\pgfusepath{fill}%
\end{pgfscope}%
\begin{pgfscope}%
\pgfpathrectangle{\pgfqpoint{1.254980in}{0.150000in}}{\pgfqpoint{5.490039in}{5.490039in}}%
\pgfusepath{clip}%
\pgfsetbuttcap%
\pgfsetroundjoin%
\definecolor{currentfill}{rgb}{0.282623,0.140926,0.457517}%
\pgfsetfillcolor{currentfill}%
\pgfsetfillopacity{0.700000}%
\pgfsetlinewidth{0.000000pt}%
\definecolor{currentstroke}{rgb}{0.000000,0.000000,0.000000}%
\pgfsetstrokecolor{currentstroke}%
\pgfsetdash{}{0pt}%
\pgfpathmoveto{\pgfqpoint{3.326199in}{1.711617in}}%
\pgfpathlineto{\pgfqpoint{3.339717in}{1.700021in}}%
\pgfpathlineto{\pgfqpoint{3.353235in}{1.688560in}}%
\pgfpathlineto{\pgfqpoint{3.366754in}{1.677234in}}%
\pgfpathlineto{\pgfqpoint{3.380272in}{1.666042in}}%
\pgfpathlineto{\pgfqpoint{3.388475in}{1.666850in}}%
\pgfpathlineto{\pgfqpoint{3.396666in}{1.667872in}}%
\pgfpathlineto{\pgfqpoint{3.404845in}{1.669103in}}%
\pgfpathlineto{\pgfqpoint{3.413014in}{1.670541in}}%
\pgfpathlineto{\pgfqpoint{3.399526in}{1.681284in}}%
\pgfpathlineto{\pgfqpoint{3.386039in}{1.692160in}}%
\pgfpathlineto{\pgfqpoint{3.372552in}{1.703172in}}%
\pgfpathlineto{\pgfqpoint{3.359066in}{1.714318in}}%
\pgfpathlineto{\pgfqpoint{3.350867in}{1.713323in}}%
\pgfpathlineto{\pgfqpoint{3.342656in}{1.712539in}}%
\pgfpathlineto{\pgfqpoint{3.334434in}{1.711969in}}%
\pgfpathlineto{\pgfqpoint{3.326199in}{1.711617in}}%
\pgfpathclose%
\pgfusepath{fill}%
\end{pgfscope}%
\begin{pgfscope}%
\pgfpathrectangle{\pgfqpoint{1.254980in}{0.150000in}}{\pgfqpoint{5.490039in}{5.490039in}}%
\pgfusepath{clip}%
\pgfsetbuttcap%
\pgfsetroundjoin%
\definecolor{currentfill}{rgb}{0.208623,0.367752,0.552675}%
\pgfsetfillcolor{currentfill}%
\pgfsetfillopacity{0.700000}%
\pgfsetlinewidth{0.000000pt}%
\definecolor{currentstroke}{rgb}{0.000000,0.000000,0.000000}%
\pgfsetstrokecolor{currentstroke}%
\pgfsetdash{}{0pt}%
\pgfpathmoveto{\pgfqpoint{5.278399in}{2.151741in}}%
\pgfpathlineto{\pgfqpoint{5.292481in}{2.157653in}}%
\pgfpathlineto{\pgfqpoint{5.306578in}{2.163677in}}%
\pgfpathlineto{\pgfqpoint{5.320689in}{2.169814in}}%
\pgfpathlineto{\pgfqpoint{5.334816in}{2.176063in}}%
\pgfpathlineto{\pgfqpoint{5.342293in}{2.188594in}}%
\pgfpathlineto{\pgfqpoint{5.349764in}{2.201062in}}%
\pgfpathlineto{\pgfqpoint{5.357230in}{2.213466in}}%
\pgfpathlineto{\pgfqpoint{5.364689in}{2.225804in}}%
\pgfpathlineto{\pgfqpoint{5.350563in}{2.219416in}}%
\pgfpathlineto{\pgfqpoint{5.336452in}{2.213142in}}%
\pgfpathlineto{\pgfqpoint{5.322356in}{2.206980in}}%
\pgfpathlineto{\pgfqpoint{5.308275in}{2.200930in}}%
\pgfpathlineto{\pgfqpoint{5.300814in}{2.188724in}}%
\pgfpathlineto{\pgfqpoint{5.293348in}{2.176456in}}%
\pgfpathlineto{\pgfqpoint{5.285876in}{2.164128in}}%
\pgfpathlineto{\pgfqpoint{5.278399in}{2.151741in}}%
\pgfpathclose%
\pgfusepath{fill}%
\end{pgfscope}%
\begin{pgfscope}%
\pgfpathrectangle{\pgfqpoint{1.254980in}{0.150000in}}{\pgfqpoint{5.490039in}{5.490039in}}%
\pgfusepath{clip}%
\pgfsetbuttcap%
\pgfsetroundjoin%
\definecolor{currentfill}{rgb}{0.272594,0.025563,0.353093}%
\pgfsetfillcolor{currentfill}%
\pgfsetfillopacity{0.700000}%
\pgfsetlinewidth{0.000000pt}%
\definecolor{currentstroke}{rgb}{0.000000,0.000000,0.000000}%
\pgfsetstrokecolor{currentstroke}%
\pgfsetdash{}{0pt}%
\pgfpathmoveto{\pgfqpoint{4.274558in}{1.454182in}}%
\pgfpathlineto{\pgfqpoint{4.288184in}{1.451902in}}%
\pgfpathlineto{\pgfqpoint{4.301819in}{1.449737in}}%
\pgfpathlineto{\pgfqpoint{4.315461in}{1.447686in}}%
\pgfpathlineto{\pgfqpoint{4.329112in}{1.445750in}}%
\pgfpathlineto{\pgfqpoint{4.336863in}{1.455590in}}%
\pgfpathlineto{\pgfqpoint{4.344610in}{1.465507in}}%
\pgfpathlineto{\pgfqpoint{4.352351in}{1.475497in}}%
\pgfpathlineto{\pgfqpoint{4.360088in}{1.485558in}}%
\pgfpathlineto{\pgfqpoint{4.346446in}{1.487153in}}%
\pgfpathlineto{\pgfqpoint{4.332812in}{1.488861in}}%
\pgfpathlineto{\pgfqpoint{4.319187in}{1.490685in}}%
\pgfpathlineto{\pgfqpoint{4.305570in}{1.492622in}}%
\pgfpathlineto{\pgfqpoint{4.297824in}{1.482897in}}%
\pgfpathlineto{\pgfqpoint{4.290074in}{1.473247in}}%
\pgfpathlineto{\pgfqpoint{4.282318in}{1.463675in}}%
\pgfpathlineto{\pgfqpoint{4.274558in}{1.454182in}}%
\pgfpathclose%
\pgfusepath{fill}%
\end{pgfscope}%
\begin{pgfscope}%
\pgfpathrectangle{\pgfqpoint{1.254980in}{0.150000in}}{\pgfqpoint{5.490039in}{5.490039in}}%
\pgfusepath{clip}%
\pgfsetbuttcap%
\pgfsetroundjoin%
\definecolor{currentfill}{rgb}{0.244972,0.287675,0.537260}%
\pgfsetfillcolor{currentfill}%
\pgfsetfillopacity{0.700000}%
\pgfsetlinewidth{0.000000pt}%
\definecolor{currentstroke}{rgb}{0.000000,0.000000,0.000000}%
\pgfsetstrokecolor{currentstroke}%
\pgfsetdash{}{0pt}%
\pgfpathmoveto{\pgfqpoint{5.076095in}{1.959666in}}%
\pgfpathlineto{\pgfqpoint{5.090066in}{1.964169in}}%
\pgfpathlineto{\pgfqpoint{5.104051in}{1.968784in}}%
\pgfpathlineto{\pgfqpoint{5.118049in}{1.973512in}}%
\pgfpathlineto{\pgfqpoint{5.132061in}{1.978351in}}%
\pgfpathlineto{\pgfqpoint{5.139602in}{1.991176in}}%
\pgfpathlineto{\pgfqpoint{5.147138in}{2.003959in}}%
\pgfpathlineto{\pgfqpoint{5.154669in}{2.016698in}}%
\pgfpathlineto{\pgfqpoint{5.162195in}{2.029394in}}%
\pgfpathlineto{\pgfqpoint{5.148183in}{2.024368in}}%
\pgfpathlineto{\pgfqpoint{5.134185in}{2.019454in}}%
\pgfpathlineto{\pgfqpoint{5.120201in}{2.014652in}}%
\pgfpathlineto{\pgfqpoint{5.106230in}{2.009962in}}%
\pgfpathlineto{\pgfqpoint{5.098704in}{1.997447in}}%
\pgfpathlineto{\pgfqpoint{5.091172in}{1.984892in}}%
\pgfpathlineto{\pgfqpoint{5.083636in}{1.972298in}}%
\pgfpathlineto{\pgfqpoint{5.076095in}{1.959666in}}%
\pgfpathclose%
\pgfusepath{fill}%
\end{pgfscope}%
\begin{pgfscope}%
\pgfpathrectangle{\pgfqpoint{1.254980in}{0.150000in}}{\pgfqpoint{5.490039in}{5.490039in}}%
\pgfusepath{clip}%
\pgfsetbuttcap%
\pgfsetroundjoin%
\definecolor{currentfill}{rgb}{0.278826,0.175490,0.483397}%
\pgfsetfillcolor{currentfill}%
\pgfsetfillopacity{0.700000}%
\pgfsetlinewidth{0.000000pt}%
\definecolor{currentstroke}{rgb}{0.000000,0.000000,0.000000}%
\pgfsetstrokecolor{currentstroke}%
\pgfsetdash{}{0pt}%
\pgfpathmoveto{\pgfqpoint{4.788091in}{1.719217in}}%
\pgfpathlineto{\pgfqpoint{4.801917in}{1.721472in}}%
\pgfpathlineto{\pgfqpoint{4.815755in}{1.723839in}}%
\pgfpathlineto{\pgfqpoint{4.829605in}{1.726318in}}%
\pgfpathlineto{\pgfqpoint{4.843466in}{1.728909in}}%
\pgfpathlineto{\pgfqpoint{4.851083in}{1.741421in}}%
\pgfpathlineto{\pgfqpoint{4.858695in}{1.753929in}}%
\pgfpathlineto{\pgfqpoint{4.866303in}{1.766431in}}%
\pgfpathlineto{\pgfqpoint{4.873907in}{1.778927in}}%
\pgfpathlineto{\pgfqpoint{4.860047in}{1.776086in}}%
\pgfpathlineto{\pgfqpoint{4.846199in}{1.773357in}}%
\pgfpathlineto{\pgfqpoint{4.832363in}{1.770741in}}%
\pgfpathlineto{\pgfqpoint{4.818539in}{1.768236in}}%
\pgfpathlineto{\pgfqpoint{4.810933in}{1.755985in}}%
\pgfpathlineto{\pgfqpoint{4.803324in}{1.743730in}}%
\pgfpathlineto{\pgfqpoint{4.795709in}{1.731473in}}%
\pgfpathlineto{\pgfqpoint{4.788091in}{1.719217in}}%
\pgfpathclose%
\pgfusepath{fill}%
\end{pgfscope}%
\begin{pgfscope}%
\pgfpathrectangle{\pgfqpoint{1.254980in}{0.150000in}}{\pgfqpoint{5.490039in}{5.490039in}}%
\pgfusepath{clip}%
\pgfsetbuttcap%
\pgfsetroundjoin%
\definecolor{currentfill}{rgb}{0.127568,0.566949,0.550556}%
\pgfsetfillcolor{currentfill}%
\pgfsetfillopacity{0.700000}%
\pgfsetlinewidth{0.000000pt}%
\definecolor{currentstroke}{rgb}{0.000000,0.000000,0.000000}%
\pgfsetstrokecolor{currentstroke}%
\pgfsetdash{}{0pt}%
\pgfpathmoveto{\pgfqpoint{2.455565in}{2.767727in}}%
\pgfpathlineto{\pgfqpoint{2.469349in}{2.745892in}}%
\pgfpathlineto{\pgfqpoint{2.483124in}{2.724249in}}%
\pgfpathlineto{\pgfqpoint{2.496890in}{2.702795in}}%
\pgfpathlineto{\pgfqpoint{2.510648in}{2.681530in}}%
\pgfpathlineto{\pgfqpoint{2.519524in}{2.674751in}}%
\pgfpathlineto{\pgfqpoint{2.528380in}{2.668265in}}%
\pgfpathlineto{\pgfqpoint{2.537216in}{2.662067in}}%
\pgfpathlineto{\pgfqpoint{2.546032in}{2.656155in}}%
\pgfpathlineto{\pgfqpoint{2.532328in}{2.676928in}}%
\pgfpathlineto{\pgfqpoint{2.518615in}{2.697888in}}%
\pgfpathlineto{\pgfqpoint{2.504893in}{2.719037in}}%
\pgfpathlineto{\pgfqpoint{2.491163in}{2.740376in}}%
\pgfpathlineto{\pgfqpoint{2.482295in}{2.746775in}}%
\pgfpathlineto{\pgfqpoint{2.473406in}{2.753463in}}%
\pgfpathlineto{\pgfqpoint{2.464496in}{2.760446in}}%
\pgfpathlineto{\pgfqpoint{2.455565in}{2.767727in}}%
\pgfpathclose%
\pgfusepath{fill}%
\end{pgfscope}%
\begin{pgfscope}%
\pgfpathrectangle{\pgfqpoint{1.254980in}{0.150000in}}{\pgfqpoint{5.490039in}{5.490039in}}%
\pgfusepath{clip}%
\pgfsetbuttcap%
\pgfsetroundjoin%
\definecolor{currentfill}{rgb}{0.283187,0.125848,0.444960}%
\pgfsetfillcolor{currentfill}%
\pgfsetfillopacity{0.700000}%
\pgfsetlinewidth{0.000000pt}%
\definecolor{currentstroke}{rgb}{0.000000,0.000000,0.000000}%
\pgfsetstrokecolor{currentstroke}%
\pgfsetdash{}{0pt}%
\pgfpathmoveto{\pgfqpoint{3.380272in}{1.666042in}}%
\pgfpathlineto{\pgfqpoint{3.393792in}{1.654984in}}%
\pgfpathlineto{\pgfqpoint{3.407312in}{1.644060in}}%
\pgfpathlineto{\pgfqpoint{3.420833in}{1.633269in}}%
\pgfpathlineto{\pgfqpoint{3.434355in}{1.622610in}}%
\pgfpathlineto{\pgfqpoint{3.442526in}{1.623873in}}%
\pgfpathlineto{\pgfqpoint{3.450687in}{1.625345in}}%
\pgfpathlineto{\pgfqpoint{3.458837in}{1.627023in}}%
\pgfpathlineto{\pgfqpoint{3.466976in}{1.628902in}}%
\pgfpathlineto{\pgfqpoint{3.453484in}{1.639114in}}%
\pgfpathlineto{\pgfqpoint{3.439993in}{1.649457in}}%
\pgfpathlineto{\pgfqpoint{3.426503in}{1.659933in}}%
\pgfpathlineto{\pgfqpoint{3.413014in}{1.670541in}}%
\pgfpathlineto{\pgfqpoint{3.404845in}{1.669103in}}%
\pgfpathlineto{\pgfqpoint{3.396666in}{1.667872in}}%
\pgfpathlineto{\pgfqpoint{3.388475in}{1.666850in}}%
\pgfpathlineto{\pgfqpoint{3.380272in}{1.666042in}}%
\pgfpathclose%
\pgfusepath{fill}%
\end{pgfscope}%
\begin{pgfscope}%
\pgfpathrectangle{\pgfqpoint{1.254980in}{0.150000in}}{\pgfqpoint{5.490039in}{5.490039in}}%
\pgfusepath{clip}%
\pgfsetbuttcap%
\pgfsetroundjoin%
\definecolor{currentfill}{rgb}{0.277018,0.050344,0.375715}%
\pgfsetfillcolor{currentfill}%
\pgfsetfillopacity{0.700000}%
\pgfsetlinewidth{0.000000pt}%
\definecolor{currentstroke}{rgb}{0.000000,0.000000,0.000000}%
\pgfsetstrokecolor{currentstroke}%
\pgfsetdash{}{0pt}%
\pgfpathmoveto{\pgfqpoint{3.629020in}{1.516495in}}%
\pgfpathlineto{\pgfqpoint{3.642539in}{1.507957in}}%
\pgfpathlineto{\pgfqpoint{3.656060in}{1.499546in}}%
\pgfpathlineto{\pgfqpoint{3.669585in}{1.491259in}}%
\pgfpathlineto{\pgfqpoint{3.683112in}{1.483098in}}%
\pgfpathlineto{\pgfqpoint{3.691136in}{1.486911in}}%
\pgfpathlineto{\pgfqpoint{3.699152in}{1.490902in}}%
\pgfpathlineto{\pgfqpoint{3.707159in}{1.495065in}}%
\pgfpathlineto{\pgfqpoint{3.715157in}{1.499398in}}%
\pgfpathlineto{\pgfqpoint{3.701653in}{1.507134in}}%
\pgfpathlineto{\pgfqpoint{3.688152in}{1.514995in}}%
\pgfpathlineto{\pgfqpoint{3.674655in}{1.522980in}}%
\pgfpathlineto{\pgfqpoint{3.661160in}{1.531091in}}%
\pgfpathlineto{\pgfqpoint{3.653139in}{1.527177in}}%
\pgfpathlineto{\pgfqpoint{3.645108in}{1.523438in}}%
\pgfpathlineto{\pgfqpoint{3.637069in}{1.519876in}}%
\pgfpathlineto{\pgfqpoint{3.629020in}{1.516495in}}%
\pgfpathclose%
\pgfusepath{fill}%
\end{pgfscope}%
\begin{pgfscope}%
\pgfpathrectangle{\pgfqpoint{1.254980in}{0.150000in}}{\pgfqpoint{5.490039in}{5.490039in}}%
\pgfusepath{clip}%
\pgfsetbuttcap%
\pgfsetroundjoin%
\definecolor{currentfill}{rgb}{0.268510,0.009605,0.335427}%
\pgfsetfillcolor{currentfill}%
\pgfsetfillopacity{0.700000}%
\pgfsetlinewidth{0.000000pt}%
\definecolor{currentstroke}{rgb}{0.000000,0.000000,0.000000}%
\pgfsetstrokecolor{currentstroke}%
\pgfsetdash{}{0pt}%
\pgfpathmoveto{\pgfqpoint{4.188992in}{1.428770in}}%
\pgfpathlineto{\pgfqpoint{4.202599in}{1.425672in}}%
\pgfpathlineto{\pgfqpoint{4.216213in}{1.422690in}}%
\pgfpathlineto{\pgfqpoint{4.229835in}{1.419822in}}%
\pgfpathlineto{\pgfqpoint{4.243464in}{1.417070in}}%
\pgfpathlineto{\pgfqpoint{4.251245in}{1.426214in}}%
\pgfpathlineto{\pgfqpoint{4.259021in}{1.435449in}}%
\pgfpathlineto{\pgfqpoint{4.266792in}{1.444773in}}%
\pgfpathlineto{\pgfqpoint{4.274558in}{1.454182in}}%
\pgfpathlineto{\pgfqpoint{4.260939in}{1.456577in}}%
\pgfpathlineto{\pgfqpoint{4.247328in}{1.459086in}}%
\pgfpathlineto{\pgfqpoint{4.233724in}{1.461711in}}%
\pgfpathlineto{\pgfqpoint{4.220128in}{1.464451in}}%
\pgfpathlineto{\pgfqpoint{4.212352in}{1.455393in}}%
\pgfpathlineto{\pgfqpoint{4.204571in}{1.446425in}}%
\pgfpathlineto{\pgfqpoint{4.196784in}{1.437550in}}%
\pgfpathlineto{\pgfqpoint{4.188992in}{1.428770in}}%
\pgfpathclose%
\pgfusepath{fill}%
\end{pgfscope}%
\begin{pgfscope}%
\pgfpathrectangle{\pgfqpoint{1.254980in}{0.150000in}}{\pgfqpoint{5.490039in}{5.490039in}}%
\pgfusepath{clip}%
\pgfsetbuttcap%
\pgfsetroundjoin%
\definecolor{currentfill}{rgb}{0.273006,0.204520,0.501721}%
\pgfsetfillcolor{currentfill}%
\pgfsetfillopacity{0.700000}%
\pgfsetlinewidth{0.000000pt}%
\definecolor{currentstroke}{rgb}{0.000000,0.000000,0.000000}%
\pgfsetstrokecolor{currentstroke}%
\pgfsetdash{}{0pt}%
\pgfpathmoveto{\pgfqpoint{4.873907in}{1.778927in}}%
\pgfpathlineto{\pgfqpoint{4.887779in}{1.781879in}}%
\pgfpathlineto{\pgfqpoint{4.901663in}{1.784944in}}%
\pgfpathlineto{\pgfqpoint{4.915559in}{1.788120in}}%
\pgfpathlineto{\pgfqpoint{4.929467in}{1.791409in}}%
\pgfpathlineto{\pgfqpoint{4.937065in}{1.804135in}}%
\pgfpathlineto{\pgfqpoint{4.944659in}{1.816846in}}%
\pgfpathlineto{\pgfqpoint{4.952248in}{1.829541in}}%
\pgfpathlineto{\pgfqpoint{4.959833in}{1.842217in}}%
\pgfpathlineto{\pgfqpoint{4.945925in}{1.838695in}}%
\pgfpathlineto{\pgfqpoint{4.932030in}{1.835284in}}%
\pgfpathlineto{\pgfqpoint{4.918147in}{1.831986in}}%
\pgfpathlineto{\pgfqpoint{4.904276in}{1.828799in}}%
\pgfpathlineto{\pgfqpoint{4.896691in}{1.816351in}}%
\pgfpathlineto{\pgfqpoint{4.889101in}{1.803888in}}%
\pgfpathlineto{\pgfqpoint{4.881506in}{1.791413in}}%
\pgfpathlineto{\pgfqpoint{4.873907in}{1.778927in}}%
\pgfpathclose%
\pgfusepath{fill}%
\end{pgfscope}%
\begin{pgfscope}%
\pgfpathrectangle{\pgfqpoint{1.254980in}{0.150000in}}{\pgfqpoint{5.490039in}{5.490039in}}%
\pgfusepath{clip}%
\pgfsetbuttcap%
\pgfsetroundjoin%
\definecolor{currentfill}{rgb}{0.214000,0.722114,0.469588}%
\pgfsetfillcolor{currentfill}%
\pgfsetfillopacity{0.700000}%
\pgfsetlinewidth{0.000000pt}%
\definecolor{currentstroke}{rgb}{0.000000,0.000000,0.000000}%
\pgfsetstrokecolor{currentstroke}%
\pgfsetdash{}{0pt}%
\pgfpathmoveto{\pgfqpoint{2.214446in}{3.209084in}}%
\pgfpathlineto{\pgfqpoint{2.228391in}{3.183661in}}%
\pgfpathlineto{\pgfqpoint{2.242322in}{3.158457in}}%
\pgfpathlineto{\pgfqpoint{2.256241in}{3.133469in}}%
\pgfpathlineto{\pgfqpoint{2.270148in}{3.108696in}}%
\pgfpathlineto{\pgfqpoint{2.279220in}{3.100618in}}%
\pgfpathlineto{\pgfqpoint{2.288270in}{3.092843in}}%
\pgfpathlineto{\pgfqpoint{2.297297in}{3.085368in}}%
\pgfpathlineto{\pgfqpoint{2.306303in}{3.078188in}}%
\pgfpathlineto{\pgfqpoint{2.292454in}{3.102468in}}%
\pgfpathlineto{\pgfqpoint{2.278594in}{3.126961in}}%
\pgfpathlineto{\pgfqpoint{2.264722in}{3.151670in}}%
\pgfpathlineto{\pgfqpoint{2.250837in}{3.176595in}}%
\pgfpathlineto{\pgfqpoint{2.241774in}{3.184262in}}%
\pgfpathlineto{\pgfqpoint{2.232688in}{3.192229in}}%
\pgfpathlineto{\pgfqpoint{2.223579in}{3.200502in}}%
\pgfpathlineto{\pgfqpoint{2.214446in}{3.209084in}}%
\pgfpathclose%
\pgfusepath{fill}%
\end{pgfscope}%
\begin{pgfscope}%
\pgfpathrectangle{\pgfqpoint{1.254980in}{0.150000in}}{\pgfqpoint{5.490039in}{5.490039in}}%
\pgfusepath{clip}%
\pgfsetbuttcap%
\pgfsetroundjoin%
\definecolor{currentfill}{rgb}{0.192357,0.403199,0.555836}%
\pgfsetfillcolor{currentfill}%
\pgfsetfillopacity{0.700000}%
\pgfsetlinewidth{0.000000pt}%
\definecolor{currentstroke}{rgb}{0.000000,0.000000,0.000000}%
\pgfsetstrokecolor{currentstroke}%
\pgfsetdash{}{0pt}%
\pgfpathmoveto{\pgfqpoint{5.364689in}{2.225804in}}%
\pgfpathlineto{\pgfqpoint{5.378831in}{2.232303in}}%
\pgfpathlineto{\pgfqpoint{5.392987in}{2.238915in}}%
\pgfpathlineto{\pgfqpoint{5.407158in}{2.245640in}}%
\pgfpathlineto{\pgfqpoint{5.421345in}{2.252477in}}%
\pgfpathlineto{\pgfqpoint{5.428799in}{2.264877in}}%
\pgfpathlineto{\pgfqpoint{5.436246in}{2.277205in}}%
\pgfpathlineto{\pgfqpoint{5.443687in}{2.289461in}}%
\pgfpathlineto{\pgfqpoint{5.451122in}{2.301644in}}%
\pgfpathlineto{\pgfqpoint{5.436936in}{2.294685in}}%
\pgfpathlineto{\pgfqpoint{5.422765in}{2.287839in}}%
\pgfpathlineto{\pgfqpoint{5.408610in}{2.281106in}}%
\pgfpathlineto{\pgfqpoint{5.394469in}{2.274485in}}%
\pgfpathlineto{\pgfqpoint{5.387033in}{2.262417in}}%
\pgfpathlineto{\pgfqpoint{5.379591in}{2.250280in}}%
\pgfpathlineto{\pgfqpoint{5.372143in}{2.238075in}}%
\pgfpathlineto{\pgfqpoint{5.364689in}{2.225804in}}%
\pgfpathclose%
\pgfusepath{fill}%
\end{pgfscope}%
\begin{pgfscope}%
\pgfpathrectangle{\pgfqpoint{1.254980in}{0.150000in}}{\pgfqpoint{5.490039in}{5.490039in}}%
\pgfusepath{clip}%
\pgfsetbuttcap%
\pgfsetroundjoin%
\definecolor{currentfill}{rgb}{0.267004,0.004874,0.329415}%
\pgfsetfillcolor{currentfill}%
\pgfsetfillopacity{0.700000}%
\pgfsetlinewidth{0.000000pt}%
\definecolor{currentstroke}{rgb}{0.000000,0.000000,0.000000}%
\pgfsetstrokecolor{currentstroke}%
\pgfsetdash{}{0pt}%
\pgfpathmoveto{\pgfqpoint{3.963356in}{1.417758in}}%
\pgfpathlineto{\pgfqpoint{3.976916in}{1.412493in}}%
\pgfpathlineto{\pgfqpoint{3.990480in}{1.407347in}}%
\pgfpathlineto{\pgfqpoint{4.004051in}{1.402318in}}%
\pgfpathlineto{\pgfqpoint{4.017627in}{1.397408in}}%
\pgfpathlineto{\pgfqpoint{4.025493in}{1.404522in}}%
\pgfpathlineto{\pgfqpoint{4.033353in}{1.411764in}}%
\pgfpathlineto{\pgfqpoint{4.041207in}{1.419131in}}%
\pgfpathlineto{\pgfqpoint{4.049054in}{1.426620in}}%
\pgfpathlineto{\pgfqpoint{4.035493in}{1.431141in}}%
\pgfpathlineto{\pgfqpoint{4.021938in}{1.435779in}}%
\pgfpathlineto{\pgfqpoint{4.008389in}{1.440535in}}%
\pgfpathlineto{\pgfqpoint{3.994846in}{1.445410in}}%
\pgfpathlineto{\pgfqpoint{3.986983in}{1.438305in}}%
\pgfpathlineto{\pgfqpoint{3.979114in}{1.431326in}}%
\pgfpathlineto{\pgfqpoint{3.971239in}{1.424476in}}%
\pgfpathlineto{\pgfqpoint{3.963356in}{1.417758in}}%
\pgfpathclose%
\pgfusepath{fill}%
\end{pgfscope}%
\begin{pgfscope}%
\pgfpathrectangle{\pgfqpoint{1.254980in}{0.150000in}}{\pgfqpoint{5.490039in}{5.490039in}}%
\pgfusepath{clip}%
\pgfsetbuttcap%
\pgfsetroundjoin%
\definecolor{currentfill}{rgb}{0.229739,0.322361,0.545706}%
\pgfsetfillcolor{currentfill}%
\pgfsetfillopacity{0.700000}%
\pgfsetlinewidth{0.000000pt}%
\definecolor{currentstroke}{rgb}{0.000000,0.000000,0.000000}%
\pgfsetstrokecolor{currentstroke}%
\pgfsetdash{}{0pt}%
\pgfpathmoveto{\pgfqpoint{5.162195in}{2.029394in}}%
\pgfpathlineto{\pgfqpoint{5.176221in}{2.034532in}}%
\pgfpathlineto{\pgfqpoint{5.190261in}{2.039782in}}%
\pgfpathlineto{\pgfqpoint{5.204314in}{2.045145in}}%
\pgfpathlineto{\pgfqpoint{5.218382in}{2.050619in}}%
\pgfpathlineto{\pgfqpoint{5.225903in}{2.063446in}}%
\pgfpathlineto{\pgfqpoint{5.233419in}{2.076222in}}%
\pgfpathlineto{\pgfqpoint{5.240929in}{2.088945in}}%
\pgfpathlineto{\pgfqpoint{5.248434in}{2.101615in}}%
\pgfpathlineto{\pgfqpoint{5.234366in}{2.095970in}}%
\pgfpathlineto{\pgfqpoint{5.220312in}{2.090437in}}%
\pgfpathlineto{\pgfqpoint{5.206273in}{2.085016in}}%
\pgfpathlineto{\pgfqpoint{5.192247in}{2.079707in}}%
\pgfpathlineto{\pgfqpoint{5.184742in}{2.067201in}}%
\pgfpathlineto{\pgfqpoint{5.177232in}{2.054646in}}%
\pgfpathlineto{\pgfqpoint{5.169716in}{2.042044in}}%
\pgfpathlineto{\pgfqpoint{5.162195in}{2.029394in}}%
\pgfpathclose%
\pgfusepath{fill}%
\end{pgfscope}%
\begin{pgfscope}%
\pgfpathrectangle{\pgfqpoint{1.254980in}{0.150000in}}{\pgfqpoint{5.490039in}{5.490039in}}%
\pgfusepath{clip}%
\pgfsetbuttcap%
\pgfsetroundjoin%
\definecolor{currentfill}{rgb}{0.269944,0.014625,0.341379}%
\pgfsetfillcolor{currentfill}%
\pgfsetfillopacity{0.700000}%
\pgfsetlinewidth{0.000000pt}%
\definecolor{currentstroke}{rgb}{0.000000,0.000000,0.000000}%
\pgfsetstrokecolor{currentstroke}%
\pgfsetdash{}{0pt}%
\pgfpathmoveto{\pgfqpoint{3.823321in}{1.441955in}}%
\pgfpathlineto{\pgfqpoint{3.836860in}{1.435324in}}%
\pgfpathlineto{\pgfqpoint{3.850403in}{1.428814in}}%
\pgfpathlineto{\pgfqpoint{3.863950in}{1.422425in}}%
\pgfpathlineto{\pgfqpoint{3.877502in}{1.416156in}}%
\pgfpathlineto{\pgfqpoint{3.885430in}{1.421898in}}%
\pgfpathlineto{\pgfqpoint{3.893351in}{1.427789in}}%
\pgfpathlineto{\pgfqpoint{3.901265in}{1.433827in}}%
\pgfpathlineto{\pgfqpoint{3.909172in}{1.440008in}}%
\pgfpathlineto{\pgfqpoint{3.895638in}{1.445870in}}%
\pgfpathlineto{\pgfqpoint{3.882109in}{1.451852in}}%
\pgfpathlineto{\pgfqpoint{3.868585in}{1.457954in}}%
\pgfpathlineto{\pgfqpoint{3.855066in}{1.464177in}}%
\pgfpathlineto{\pgfqpoint{3.847141in}{1.458397in}}%
\pgfpathlineto{\pgfqpoint{3.839209in}{1.452764in}}%
\pgfpathlineto{\pgfqpoint{3.831269in}{1.447282in}}%
\pgfpathlineto{\pgfqpoint{3.823321in}{1.441955in}}%
\pgfpathclose%
\pgfusepath{fill}%
\end{pgfscope}%
\begin{pgfscope}%
\pgfpathrectangle{\pgfqpoint{1.254980in}{0.150000in}}{\pgfqpoint{5.490039in}{5.490039in}}%
\pgfusepath{clip}%
\pgfsetbuttcap%
\pgfsetroundjoin%
\definecolor{currentfill}{rgb}{0.282910,0.105393,0.426902}%
\pgfsetfillcolor{currentfill}%
\pgfsetfillopacity{0.700000}%
\pgfsetlinewidth{0.000000pt}%
\definecolor{currentstroke}{rgb}{0.000000,0.000000,0.000000}%
\pgfsetstrokecolor{currentstroke}%
\pgfsetdash{}{0pt}%
\pgfpathmoveto{\pgfqpoint{3.434355in}{1.622610in}}%
\pgfpathlineto{\pgfqpoint{3.447878in}{1.612083in}}%
\pgfpathlineto{\pgfqpoint{3.461402in}{1.601688in}}%
\pgfpathlineto{\pgfqpoint{3.474927in}{1.591423in}}%
\pgfpathlineto{\pgfqpoint{3.488454in}{1.581289in}}%
\pgfpathlineto{\pgfqpoint{3.496596in}{1.583006in}}%
\pgfpathlineto{\pgfqpoint{3.504728in}{1.584927in}}%
\pgfpathlineto{\pgfqpoint{3.512849in}{1.587049in}}%
\pgfpathlineto{\pgfqpoint{3.520960in}{1.589369in}}%
\pgfpathlineto{\pgfqpoint{3.507461in}{1.599056in}}%
\pgfpathlineto{\pgfqpoint{3.493965in}{1.608874in}}%
\pgfpathlineto{\pgfqpoint{3.480470in}{1.618823in}}%
\pgfpathlineto{\pgfqpoint{3.466976in}{1.628902in}}%
\pgfpathlineto{\pgfqpoint{3.458837in}{1.627023in}}%
\pgfpathlineto{\pgfqpoint{3.450687in}{1.625345in}}%
\pgfpathlineto{\pgfqpoint{3.442526in}{1.623873in}}%
\pgfpathlineto{\pgfqpoint{3.434355in}{1.622610in}}%
\pgfpathclose%
\pgfusepath{fill}%
\end{pgfscope}%
\begin{pgfscope}%
\pgfpathrectangle{\pgfqpoint{1.254980in}{0.150000in}}{\pgfqpoint{5.490039in}{5.490039in}}%
\pgfusepath{clip}%
\pgfsetbuttcap%
\pgfsetroundjoin%
\definecolor{currentfill}{rgb}{0.120092,0.600104,0.542530}%
\pgfsetfillcolor{currentfill}%
\pgfsetfillopacity{0.700000}%
\pgfsetlinewidth{0.000000pt}%
\definecolor{currentstroke}{rgb}{0.000000,0.000000,0.000000}%
\pgfsetstrokecolor{currentstroke}%
\pgfsetdash{}{0pt}%
\pgfpathmoveto{\pgfqpoint{2.400334in}{2.857000in}}%
\pgfpathlineto{\pgfqpoint{2.414156in}{2.834388in}}%
\pgfpathlineto{\pgfqpoint{2.427969in}{2.811974in}}%
\pgfpathlineto{\pgfqpoint{2.441772in}{2.789753in}}%
\pgfpathlineto{\pgfqpoint{2.455565in}{2.767727in}}%
\pgfpathlineto{\pgfqpoint{2.464496in}{2.760446in}}%
\pgfpathlineto{\pgfqpoint{2.473406in}{2.753463in}}%
\pgfpathlineto{\pgfqpoint{2.482295in}{2.746775in}}%
\pgfpathlineto{\pgfqpoint{2.491163in}{2.740376in}}%
\pgfpathlineto{\pgfqpoint{2.477424in}{2.761906in}}%
\pgfpathlineto{\pgfqpoint{2.463677in}{2.783629in}}%
\pgfpathlineto{\pgfqpoint{2.449919in}{2.805545in}}%
\pgfpathlineto{\pgfqpoint{2.436153in}{2.827656in}}%
\pgfpathlineto{\pgfqpoint{2.427230in}{2.834544in}}%
\pgfpathlineto{\pgfqpoint{2.418286in}{2.841728in}}%
\pgfpathlineto{\pgfqpoint{2.409321in}{2.849212in}}%
\pgfpathlineto{\pgfqpoint{2.400334in}{2.857000in}}%
\pgfpathclose%
\pgfusepath{fill}%
\end{pgfscope}%
\begin{pgfscope}%
\pgfpathrectangle{\pgfqpoint{1.254980in}{0.150000in}}{\pgfqpoint{5.490039in}{5.490039in}}%
\pgfusepath{clip}%
\pgfsetbuttcap%
\pgfsetroundjoin%
\definecolor{currentfill}{rgb}{0.263663,0.237631,0.518762}%
\pgfsetfillcolor{currentfill}%
\pgfsetfillopacity{0.700000}%
\pgfsetlinewidth{0.000000pt}%
\definecolor{currentstroke}{rgb}{0.000000,0.000000,0.000000}%
\pgfsetstrokecolor{currentstroke}%
\pgfsetdash{}{0pt}%
\pgfpathmoveto{\pgfqpoint{4.959833in}{1.842217in}}%
\pgfpathlineto{\pgfqpoint{4.973753in}{1.845852in}}%
\pgfpathlineto{\pgfqpoint{4.987686in}{1.849598in}}%
\pgfpathlineto{\pgfqpoint{5.001632in}{1.853456in}}%
\pgfpathlineto{\pgfqpoint{5.015591in}{1.857426in}}%
\pgfpathlineto{\pgfqpoint{5.023170in}{1.870307in}}%
\pgfpathlineto{\pgfqpoint{5.030745in}{1.883163in}}%
\pgfpathlineto{\pgfqpoint{5.038315in}{1.895991in}}%
\pgfpathlineto{\pgfqpoint{5.045881in}{1.908789in}}%
\pgfpathlineto{\pgfqpoint{5.031922in}{1.904601in}}%
\pgfpathlineto{\pgfqpoint{5.017977in}{1.900524in}}%
\pgfpathlineto{\pgfqpoint{5.004044in}{1.896559in}}%
\pgfpathlineto{\pgfqpoint{4.990125in}{1.892706in}}%
\pgfpathlineto{\pgfqpoint{4.982559in}{1.880120in}}%
\pgfpathlineto{\pgfqpoint{4.974988in}{1.867508in}}%
\pgfpathlineto{\pgfqpoint{4.967413in}{1.854874in}}%
\pgfpathlineto{\pgfqpoint{4.959833in}{1.842217in}}%
\pgfpathclose%
\pgfusepath{fill}%
\end{pgfscope}%
\begin{pgfscope}%
\pgfpathrectangle{\pgfqpoint{1.254980in}{0.150000in}}{\pgfqpoint{5.490039in}{5.490039in}}%
\pgfusepath{clip}%
\pgfsetbuttcap%
\pgfsetroundjoin%
\definecolor{currentfill}{rgb}{0.267004,0.004874,0.329415}%
\pgfsetfillcolor{currentfill}%
\pgfsetfillopacity{0.700000}%
\pgfsetlinewidth{0.000000pt}%
\definecolor{currentstroke}{rgb}{0.000000,0.000000,0.000000}%
\pgfsetstrokecolor{currentstroke}%
\pgfsetdash{}{0pt}%
\pgfpathmoveto{\pgfqpoint{4.103359in}{1.409711in}}%
\pgfpathlineto{\pgfqpoint{4.116951in}{1.405776in}}%
\pgfpathlineto{\pgfqpoint{4.130550in}{1.401957in}}%
\pgfpathlineto{\pgfqpoint{4.144155in}{1.398254in}}%
\pgfpathlineto{\pgfqpoint{4.157767in}{1.394666in}}%
\pgfpathlineto{\pgfqpoint{4.165582in}{1.403034in}}%
\pgfpathlineto{\pgfqpoint{4.173391in}{1.411509in}}%
\pgfpathlineto{\pgfqpoint{4.181194in}{1.420089in}}%
\pgfpathlineto{\pgfqpoint{4.188992in}{1.428770in}}%
\pgfpathlineto{\pgfqpoint{4.175392in}{1.431984in}}%
\pgfpathlineto{\pgfqpoint{4.161799in}{1.435313in}}%
\pgfpathlineto{\pgfqpoint{4.148213in}{1.438758in}}%
\pgfpathlineto{\pgfqpoint{4.134633in}{1.442320in}}%
\pgfpathlineto{\pgfqpoint{4.126824in}{1.434006in}}%
\pgfpathlineto{\pgfqpoint{4.119008in}{1.425798in}}%
\pgfpathlineto{\pgfqpoint{4.111187in}{1.417699in}}%
\pgfpathlineto{\pgfqpoint{4.103359in}{1.409711in}}%
\pgfpathclose%
\pgfusepath{fill}%
\end{pgfscope}%
\begin{pgfscope}%
\pgfpathrectangle{\pgfqpoint{1.254980in}{0.150000in}}{\pgfqpoint{5.490039in}{5.490039in}}%
\pgfusepath{clip}%
\pgfsetbuttcap%
\pgfsetroundjoin%
\definecolor{currentfill}{rgb}{0.227802,0.326594,0.546532}%
\pgfsetfillcolor{currentfill}%
\pgfsetfillopacity{0.700000}%
\pgfsetlinewidth{0.000000pt}%
\definecolor{currentstroke}{rgb}{0.000000,0.000000,0.000000}%
\pgfsetstrokecolor{currentstroke}%
\pgfsetdash{}{0pt}%
\pgfpathmoveto{\pgfqpoint{2.913152in}{2.108062in}}%
\pgfpathlineto{\pgfqpoint{2.926757in}{2.091927in}}%
\pgfpathlineto{\pgfqpoint{2.940359in}{2.075948in}}%
\pgfpathlineto{\pgfqpoint{2.953957in}{2.060122in}}%
\pgfpathlineto{\pgfqpoint{2.967551in}{2.044451in}}%
\pgfpathlineto{\pgfqpoint{2.976067in}{2.040995in}}%
\pgfpathlineto{\pgfqpoint{2.984567in}{2.037802in}}%
\pgfpathlineto{\pgfqpoint{2.993051in}{2.034869in}}%
\pgfpathlineto{\pgfqpoint{3.001520in}{2.032191in}}%
\pgfpathlineto{\pgfqpoint{2.987968in}{2.047380in}}%
\pgfpathlineto{\pgfqpoint{2.974412in}{2.062722in}}%
\pgfpathlineto{\pgfqpoint{2.960853in}{2.078218in}}%
\pgfpathlineto{\pgfqpoint{2.947291in}{2.093868in}}%
\pgfpathlineto{\pgfqpoint{2.938781in}{2.097022in}}%
\pgfpathlineto{\pgfqpoint{2.930254in}{2.100436in}}%
\pgfpathlineto{\pgfqpoint{2.921711in}{2.104115in}}%
\pgfpathlineto{\pgfqpoint{2.913152in}{2.108062in}}%
\pgfpathclose%
\pgfusepath{fill}%
\end{pgfscope}%
\begin{pgfscope}%
\pgfpathrectangle{\pgfqpoint{1.254980in}{0.150000in}}{\pgfqpoint{5.490039in}{5.490039in}}%
\pgfusepath{clip}%
\pgfsetbuttcap%
\pgfsetroundjoin%
\definecolor{currentfill}{rgb}{0.280267,0.073417,0.397163}%
\pgfsetfillcolor{currentfill}%
\pgfsetfillopacity{0.700000}%
\pgfsetlinewidth{0.000000pt}%
\definecolor{currentstroke}{rgb}{0.000000,0.000000,0.000000}%
\pgfsetstrokecolor{currentstroke}%
\pgfsetdash{}{0pt}%
\pgfpathmoveto{\pgfqpoint{4.500378in}{1.520408in}}%
\pgfpathlineto{\pgfqpoint{4.514093in}{1.520161in}}%
\pgfpathlineto{\pgfqpoint{4.527817in}{1.520027in}}%
\pgfpathlineto{\pgfqpoint{4.541552in}{1.520006in}}%
\pgfpathlineto{\pgfqpoint{4.555296in}{1.520097in}}%
\pgfpathlineto{\pgfqpoint{4.562990in}{1.531423in}}%
\pgfpathlineto{\pgfqpoint{4.570680in}{1.542793in}}%
\pgfpathlineto{\pgfqpoint{4.578366in}{1.554204in}}%
\pgfpathlineto{\pgfqpoint{4.586048in}{1.565654in}}%
\pgfpathlineto{\pgfqpoint{4.572308in}{1.565251in}}%
\pgfpathlineto{\pgfqpoint{4.558579in}{1.564961in}}%
\pgfpathlineto{\pgfqpoint{4.544859in}{1.564783in}}%
\pgfpathlineto{\pgfqpoint{4.531150in}{1.564718in}}%
\pgfpathlineto{\pgfqpoint{4.523463in}{1.553574in}}%
\pgfpathlineto{\pgfqpoint{4.515773in}{1.542473in}}%
\pgfpathlineto{\pgfqpoint{4.508077in}{1.531417in}}%
\pgfpathlineto{\pgfqpoint{4.500378in}{1.520408in}}%
\pgfpathclose%
\pgfusepath{fill}%
\end{pgfscope}%
\begin{pgfscope}%
\pgfpathrectangle{\pgfqpoint{1.254980in}{0.150000in}}{\pgfqpoint{5.490039in}{5.490039in}}%
\pgfusepath{clip}%
\pgfsetbuttcap%
\pgfsetroundjoin%
\definecolor{currentfill}{rgb}{0.216210,0.351535,0.550627}%
\pgfsetfillcolor{currentfill}%
\pgfsetfillopacity{0.700000}%
\pgfsetlinewidth{0.000000pt}%
\definecolor{currentstroke}{rgb}{0.000000,0.000000,0.000000}%
\pgfsetstrokecolor{currentstroke}%
\pgfsetdash{}{0pt}%
\pgfpathmoveto{\pgfqpoint{2.858691in}{2.174168in}}%
\pgfpathlineto{\pgfqpoint{2.872312in}{2.157405in}}%
\pgfpathlineto{\pgfqpoint{2.885929in}{2.140800in}}%
\pgfpathlineto{\pgfqpoint{2.899543in}{2.124352in}}%
\pgfpathlineto{\pgfqpoint{2.913152in}{2.108062in}}%
\pgfpathlineto{\pgfqpoint{2.921711in}{2.104115in}}%
\pgfpathlineto{\pgfqpoint{2.930254in}{2.100436in}}%
\pgfpathlineto{\pgfqpoint{2.938781in}{2.097022in}}%
\pgfpathlineto{\pgfqpoint{2.947291in}{2.093868in}}%
\pgfpathlineto{\pgfqpoint{2.933726in}{2.109673in}}%
\pgfpathlineto{\pgfqpoint{2.920156in}{2.125635in}}%
\pgfpathlineto{\pgfqpoint{2.906583in}{2.141753in}}%
\pgfpathlineto{\pgfqpoint{2.893006in}{2.158029in}}%
\pgfpathlineto{\pgfqpoint{2.884452in}{2.161662in}}%
\pgfpathlineto{\pgfqpoint{2.875882in}{2.165560in}}%
\pgfpathlineto{\pgfqpoint{2.867295in}{2.169727in}}%
\pgfpathlineto{\pgfqpoint{2.858691in}{2.174168in}}%
\pgfpathclose%
\pgfusepath{fill}%
\end{pgfscope}%
\begin{pgfscope}%
\pgfpathrectangle{\pgfqpoint{1.254980in}{0.150000in}}{\pgfqpoint{5.490039in}{5.490039in}}%
\pgfusepath{clip}%
\pgfsetbuttcap%
\pgfsetroundjoin%
\definecolor{currentfill}{rgb}{0.239346,0.300855,0.540844}%
\pgfsetfillcolor{currentfill}%
\pgfsetfillopacity{0.700000}%
\pgfsetlinewidth{0.000000pt}%
\definecolor{currentstroke}{rgb}{0.000000,0.000000,0.000000}%
\pgfsetstrokecolor{currentstroke}%
\pgfsetdash{}{0pt}%
\pgfpathmoveto{\pgfqpoint{2.967551in}{2.044451in}}%
\pgfpathlineto{\pgfqpoint{2.981142in}{2.028932in}}%
\pgfpathlineto{\pgfqpoint{2.994731in}{2.013565in}}%
\pgfpathlineto{\pgfqpoint{3.008316in}{1.998349in}}%
\pgfpathlineto{\pgfqpoint{3.021898in}{1.983283in}}%
\pgfpathlineto{\pgfqpoint{3.030372in}{1.980315in}}%
\pgfpathlineto{\pgfqpoint{3.038830in}{1.977606in}}%
\pgfpathlineto{\pgfqpoint{3.047274in}{1.975151in}}%
\pgfpathlineto{\pgfqpoint{3.055702in}{1.972946in}}%
\pgfpathlineto{\pgfqpoint{3.042161in}{1.987532in}}%
\pgfpathlineto{\pgfqpoint{3.028617in}{2.002267in}}%
\pgfpathlineto{\pgfqpoint{3.015070in}{2.017153in}}%
\pgfpathlineto{\pgfqpoint{3.001520in}{2.032191in}}%
\pgfpathlineto{\pgfqpoint{2.993051in}{2.034869in}}%
\pgfpathlineto{\pgfqpoint{2.984567in}{2.037802in}}%
\pgfpathlineto{\pgfqpoint{2.976067in}{2.040995in}}%
\pgfpathlineto{\pgfqpoint{2.967551in}{2.044451in}}%
\pgfpathclose%
\pgfusepath{fill}%
\end{pgfscope}%
\begin{pgfscope}%
\pgfpathrectangle{\pgfqpoint{1.254980in}{0.150000in}}{\pgfqpoint{5.490039in}{5.490039in}}%
\pgfusepath{clip}%
\pgfsetbuttcap%
\pgfsetroundjoin%
\definecolor{currentfill}{rgb}{0.179019,0.433756,0.557430}%
\pgfsetfillcolor{currentfill}%
\pgfsetfillopacity{0.700000}%
\pgfsetlinewidth{0.000000pt}%
\definecolor{currentstroke}{rgb}{0.000000,0.000000,0.000000}%
\pgfsetstrokecolor{currentstroke}%
\pgfsetdash{}{0pt}%
\pgfpathmoveto{\pgfqpoint{5.451122in}{2.301644in}}%
\pgfpathlineto{\pgfqpoint{5.465324in}{2.308716in}}%
\pgfpathlineto{\pgfqpoint{5.479542in}{2.315900in}}%
\pgfpathlineto{\pgfqpoint{5.493775in}{2.323197in}}%
\pgfpathlineto{\pgfqpoint{5.501204in}{2.335390in}}%
\pgfpathlineto{\pgfqpoint{5.508626in}{2.347506in}}%
\pgfpathlineto{\pgfqpoint{5.516041in}{2.359544in}}%
\pgfpathlineto{\pgfqpoint{5.523450in}{2.371504in}}%
\pgfpathlineto{\pgfqpoint{5.509218in}{2.364102in}}%
\pgfpathlineto{\pgfqpoint{5.495001in}{2.356813in}}%
\pgfpathlineto{\pgfqpoint{5.480800in}{2.349637in}}%
\pgfpathlineto{\pgfqpoint{5.473390in}{2.337751in}}%
\pgfpathlineto{\pgfqpoint{5.465974in}{2.325790in}}%
\pgfpathlineto{\pgfqpoint{5.458551in}{2.313754in}}%
\pgfpathlineto{\pgfqpoint{5.451122in}{2.301644in}}%
\pgfpathclose%
\pgfusepath{fill}%
\end{pgfscope}%
\begin{pgfscope}%
\pgfpathrectangle{\pgfqpoint{1.254980in}{0.150000in}}{\pgfqpoint{5.490039in}{5.490039in}}%
\pgfusepath{clip}%
\pgfsetbuttcap%
\pgfsetroundjoin%
\definecolor{currentfill}{rgb}{0.282656,0.100196,0.422160}%
\pgfsetfillcolor{currentfill}%
\pgfsetfillopacity{0.700000}%
\pgfsetlinewidth{0.000000pt}%
\definecolor{currentstroke}{rgb}{0.000000,0.000000,0.000000}%
\pgfsetstrokecolor{currentstroke}%
\pgfsetdash{}{0pt}%
\pgfpathmoveto{\pgfqpoint{4.586048in}{1.565654in}}%
\pgfpathlineto{\pgfqpoint{4.599797in}{1.566169in}}%
\pgfpathlineto{\pgfqpoint{4.613556in}{1.566797in}}%
\pgfpathlineto{\pgfqpoint{4.627326in}{1.567537in}}%
\pgfpathlineto{\pgfqpoint{4.641106in}{1.568389in}}%
\pgfpathlineto{\pgfqpoint{4.648779in}{1.580176in}}%
\pgfpathlineto{\pgfqpoint{4.656448in}{1.591994in}}%
\pgfpathlineto{\pgfqpoint{4.664112in}{1.603839in}}%
\pgfpathlineto{\pgfqpoint{4.671773in}{1.615709in}}%
\pgfpathlineto{\pgfqpoint{4.657996in}{1.614561in}}%
\pgfpathlineto{\pgfqpoint{4.644230in}{1.613525in}}%
\pgfpathlineto{\pgfqpoint{4.630475in}{1.612601in}}%
\pgfpathlineto{\pgfqpoint{4.616729in}{1.611789in}}%
\pgfpathlineto{\pgfqpoint{4.609066in}{1.600209in}}%
\pgfpathlineto{\pgfqpoint{4.601397in}{1.588659in}}%
\pgfpathlineto{\pgfqpoint{4.593725in}{1.577139in}}%
\pgfpathlineto{\pgfqpoint{4.586048in}{1.565654in}}%
\pgfpathclose%
\pgfusepath{fill}%
\end{pgfscope}%
\begin{pgfscope}%
\pgfpathrectangle{\pgfqpoint{1.254980in}{0.150000in}}{\pgfqpoint{5.490039in}{5.490039in}}%
\pgfusepath{clip}%
\pgfsetbuttcap%
\pgfsetroundjoin%
\definecolor{currentfill}{rgb}{0.277018,0.050344,0.375715}%
\pgfsetfillcolor{currentfill}%
\pgfsetfillopacity{0.700000}%
\pgfsetlinewidth{0.000000pt}%
\definecolor{currentstroke}{rgb}{0.000000,0.000000,0.000000}%
\pgfsetstrokecolor{currentstroke}%
\pgfsetdash{}{0pt}%
\pgfpathmoveto{\pgfqpoint{4.414740in}{1.480320in}}%
\pgfpathlineto{\pgfqpoint{4.428425in}{1.479294in}}%
\pgfpathlineto{\pgfqpoint{4.442119in}{1.478381in}}%
\pgfpathlineto{\pgfqpoint{4.455822in}{1.477581in}}%
\pgfpathlineto{\pgfqpoint{4.469534in}{1.476894in}}%
\pgfpathlineto{\pgfqpoint{4.477251in}{1.487689in}}%
\pgfpathlineto{\pgfqpoint{4.484965in}{1.498541in}}%
\pgfpathlineto{\pgfqpoint{4.492673in}{1.509448in}}%
\pgfpathlineto{\pgfqpoint{4.500378in}{1.520408in}}%
\pgfpathlineto{\pgfqpoint{4.486672in}{1.520767in}}%
\pgfpathlineto{\pgfqpoint{4.472975in}{1.521240in}}%
\pgfpathlineto{\pgfqpoint{4.459288in}{1.521825in}}%
\pgfpathlineto{\pgfqpoint{4.445610in}{1.522525in}}%
\pgfpathlineto{\pgfqpoint{4.437900in}{1.511886in}}%
\pgfpathlineto{\pgfqpoint{4.430185in}{1.501304in}}%
\pgfpathlineto{\pgfqpoint{4.422465in}{1.490781in}}%
\pgfpathlineto{\pgfqpoint{4.414740in}{1.480320in}}%
\pgfpathclose%
\pgfusepath{fill}%
\end{pgfscope}%
\begin{pgfscope}%
\pgfpathrectangle{\pgfqpoint{1.254980in}{0.150000in}}{\pgfqpoint{5.490039in}{5.490039in}}%
\pgfusepath{clip}%
\pgfsetbuttcap%
\pgfsetroundjoin%
\definecolor{currentfill}{rgb}{0.274952,0.037752,0.364543}%
\pgfsetfillcolor{currentfill}%
\pgfsetfillopacity{0.700000}%
\pgfsetlinewidth{0.000000pt}%
\definecolor{currentstroke}{rgb}{0.000000,0.000000,0.000000}%
\pgfsetstrokecolor{currentstroke}%
\pgfsetdash{}{0pt}%
\pgfpathmoveto{\pgfqpoint{3.683112in}{1.483098in}}%
\pgfpathlineto{\pgfqpoint{3.696642in}{1.475061in}}%
\pgfpathlineto{\pgfqpoint{3.710176in}{1.467147in}}%
\pgfpathlineto{\pgfqpoint{3.723713in}{1.459358in}}%
\pgfpathlineto{\pgfqpoint{3.737253in}{1.451692in}}%
\pgfpathlineto{\pgfqpoint{3.745255in}{1.455938in}}%
\pgfpathlineto{\pgfqpoint{3.753247in}{1.460356in}}%
\pgfpathlineto{\pgfqpoint{3.761232in}{1.464942in}}%
\pgfpathlineto{\pgfqpoint{3.769208in}{1.469694in}}%
\pgfpathlineto{\pgfqpoint{3.755690in}{1.476935in}}%
\pgfpathlineto{\pgfqpoint{3.742175in}{1.484299in}}%
\pgfpathlineto{\pgfqpoint{3.728664in}{1.491787in}}%
\pgfpathlineto{\pgfqpoint{3.715157in}{1.499398in}}%
\pgfpathlineto{\pgfqpoint{3.707159in}{1.495065in}}%
\pgfpathlineto{\pgfqpoint{3.699152in}{1.490902in}}%
\pgfpathlineto{\pgfqpoint{3.691136in}{1.486911in}}%
\pgfpathlineto{\pgfqpoint{3.683112in}{1.483098in}}%
\pgfpathclose%
\pgfusepath{fill}%
\end{pgfscope}%
\begin{pgfscope}%
\pgfpathrectangle{\pgfqpoint{1.254980in}{0.150000in}}{\pgfqpoint{5.490039in}{5.490039in}}%
\pgfusepath{clip}%
\pgfsetbuttcap%
\pgfsetroundjoin%
\definecolor{currentfill}{rgb}{0.203063,0.379716,0.553925}%
\pgfsetfillcolor{currentfill}%
\pgfsetfillopacity{0.700000}%
\pgfsetlinewidth{0.000000pt}%
\definecolor{currentstroke}{rgb}{0.000000,0.000000,0.000000}%
\pgfsetstrokecolor{currentstroke}%
\pgfsetdash{}{0pt}%
\pgfpathmoveto{\pgfqpoint{2.804159in}{2.242822in}}%
\pgfpathlineto{\pgfqpoint{2.817799in}{2.225417in}}%
\pgfpathlineto{\pgfqpoint{2.831434in}{2.208173in}}%
\pgfpathlineto{\pgfqpoint{2.845065in}{2.191090in}}%
\pgfpathlineto{\pgfqpoint{2.858691in}{2.174168in}}%
\pgfpathlineto{\pgfqpoint{2.867295in}{2.169727in}}%
\pgfpathlineto{\pgfqpoint{2.875882in}{2.165560in}}%
\pgfpathlineto{\pgfqpoint{2.884452in}{2.161662in}}%
\pgfpathlineto{\pgfqpoint{2.893006in}{2.158029in}}%
\pgfpathlineto{\pgfqpoint{2.879425in}{2.174463in}}%
\pgfpathlineto{\pgfqpoint{2.865840in}{2.191057in}}%
\pgfpathlineto{\pgfqpoint{2.852250in}{2.207811in}}%
\pgfpathlineto{\pgfqpoint{2.838656in}{2.224726in}}%
\pgfpathlineto{\pgfqpoint{2.830058in}{2.228841in}}%
\pgfpathlineto{\pgfqpoint{2.821442in}{2.233225in}}%
\pgfpathlineto{\pgfqpoint{2.812809in}{2.237885in}}%
\pgfpathlineto{\pgfqpoint{2.804159in}{2.242822in}}%
\pgfpathclose%
\pgfusepath{fill}%
\end{pgfscope}%
\begin{pgfscope}%
\pgfpathrectangle{\pgfqpoint{1.254980in}{0.150000in}}{\pgfqpoint{5.490039in}{5.490039in}}%
\pgfusepath{clip}%
\pgfsetbuttcap%
\pgfsetroundjoin%
\definecolor{currentfill}{rgb}{0.250425,0.274290,0.533103}%
\pgfsetfillcolor{currentfill}%
\pgfsetfillopacity{0.700000}%
\pgfsetlinewidth{0.000000pt}%
\definecolor{currentstroke}{rgb}{0.000000,0.000000,0.000000}%
\pgfsetstrokecolor{currentstroke}%
\pgfsetdash{}{0pt}%
\pgfpathmoveto{\pgfqpoint{3.021898in}{1.983283in}}%
\pgfpathlineto{\pgfqpoint{3.035477in}{1.968367in}}%
\pgfpathlineto{\pgfqpoint{3.049054in}{1.953600in}}%
\pgfpathlineto{\pgfqpoint{3.062629in}{1.938981in}}%
\pgfpathlineto{\pgfqpoint{3.076201in}{1.924510in}}%
\pgfpathlineto{\pgfqpoint{3.084634in}{1.922028in}}%
\pgfpathlineto{\pgfqpoint{3.093052in}{1.919799in}}%
\pgfpathlineto{\pgfqpoint{3.101456in}{1.917820in}}%
\pgfpathlineto{\pgfqpoint{3.109846in}{1.916087in}}%
\pgfpathlineto{\pgfqpoint{3.096313in}{1.930080in}}%
\pgfpathlineto{\pgfqpoint{3.082778in}{1.944221in}}%
\pgfpathlineto{\pgfqpoint{3.069241in}{1.958510in}}%
\pgfpathlineto{\pgfqpoint{3.055702in}{1.972946in}}%
\pgfpathlineto{\pgfqpoint{3.047274in}{1.975151in}}%
\pgfpathlineto{\pgfqpoint{3.038830in}{1.977606in}}%
\pgfpathlineto{\pgfqpoint{3.030372in}{1.980315in}}%
\pgfpathlineto{\pgfqpoint{3.021898in}{1.983283in}}%
\pgfpathclose%
\pgfusepath{fill}%
\end{pgfscope}%
\begin{pgfscope}%
\pgfpathrectangle{\pgfqpoint{1.254980in}{0.150000in}}{\pgfqpoint{5.490039in}{5.490039in}}%
\pgfusepath{clip}%
\pgfsetbuttcap%
\pgfsetroundjoin%
\definecolor{currentfill}{rgb}{0.283072,0.130895,0.449241}%
\pgfsetfillcolor{currentfill}%
\pgfsetfillopacity{0.700000}%
\pgfsetlinewidth{0.000000pt}%
\definecolor{currentstroke}{rgb}{0.000000,0.000000,0.000000}%
\pgfsetstrokecolor{currentstroke}%
\pgfsetdash{}{0pt}%
\pgfpathmoveto{\pgfqpoint{4.671773in}{1.615709in}}%
\pgfpathlineto{\pgfqpoint{4.685560in}{1.616970in}}%
\pgfpathlineto{\pgfqpoint{4.699358in}{1.618343in}}%
\pgfpathlineto{\pgfqpoint{4.713166in}{1.619827in}}%
\pgfpathlineto{\pgfqpoint{4.726986in}{1.621424in}}%
\pgfpathlineto{\pgfqpoint{4.734639in}{1.633604in}}%
\pgfpathlineto{\pgfqpoint{4.742288in}{1.645801in}}%
\pgfpathlineto{\pgfqpoint{4.749933in}{1.658013in}}%
\pgfpathlineto{\pgfqpoint{4.757573in}{1.670237in}}%
\pgfpathlineto{\pgfqpoint{4.743756in}{1.668359in}}%
\pgfpathlineto{\pgfqpoint{4.729950in}{1.666594in}}%
\pgfpathlineto{\pgfqpoint{4.716155in}{1.664940in}}%
\pgfpathlineto{\pgfqpoint{4.702370in}{1.663399in}}%
\pgfpathlineto{\pgfqpoint{4.694728in}{1.651449in}}%
\pgfpathlineto{\pgfqpoint{4.687080in}{1.639517in}}%
\pgfpathlineto{\pgfqpoint{4.679429in}{1.627603in}}%
\pgfpathlineto{\pgfqpoint{4.671773in}{1.615709in}}%
\pgfpathclose%
\pgfusepath{fill}%
\end{pgfscope}%
\begin{pgfscope}%
\pgfpathrectangle{\pgfqpoint{1.254980in}{0.150000in}}{\pgfqpoint{5.490039in}{5.490039in}}%
\pgfusepath{clip}%
\pgfsetbuttcap%
\pgfsetroundjoin%
\definecolor{currentfill}{rgb}{0.190631,0.407061,0.556089}%
\pgfsetfillcolor{currentfill}%
\pgfsetfillopacity{0.700000}%
\pgfsetlinewidth{0.000000pt}%
\definecolor{currentstroke}{rgb}{0.000000,0.000000,0.000000}%
\pgfsetstrokecolor{currentstroke}%
\pgfsetdash{}{0pt}%
\pgfpathmoveto{\pgfqpoint{2.749546in}{2.314082in}}%
\pgfpathlineto{\pgfqpoint{2.763208in}{2.296020in}}%
\pgfpathlineto{\pgfqpoint{2.776863in}{2.278123in}}%
\pgfpathlineto{\pgfqpoint{2.790513in}{2.260391in}}%
\pgfpathlineto{\pgfqpoint{2.804159in}{2.242822in}}%
\pgfpathlineto{\pgfqpoint{2.812809in}{2.237885in}}%
\pgfpathlineto{\pgfqpoint{2.821442in}{2.233225in}}%
\pgfpathlineto{\pgfqpoint{2.830058in}{2.228841in}}%
\pgfpathlineto{\pgfqpoint{2.838656in}{2.224726in}}%
\pgfpathlineto{\pgfqpoint{2.825057in}{2.241803in}}%
\pgfpathlineto{\pgfqpoint{2.811454in}{2.259043in}}%
\pgfpathlineto{\pgfqpoint{2.797845in}{2.276447in}}%
\pgfpathlineto{\pgfqpoint{2.784232in}{2.294016in}}%
\pgfpathlineto{\pgfqpoint{2.775587in}{2.298615in}}%
\pgfpathlineto{\pgfqpoint{2.766925in}{2.303490in}}%
\pgfpathlineto{\pgfqpoint{2.758245in}{2.308644in}}%
\pgfpathlineto{\pgfqpoint{2.749546in}{2.314082in}}%
\pgfpathclose%
\pgfusepath{fill}%
\end{pgfscope}%
\begin{pgfscope}%
\pgfpathrectangle{\pgfqpoint{1.254980in}{0.150000in}}{\pgfqpoint{5.490039in}{5.490039in}}%
\pgfusepath{clip}%
\pgfsetbuttcap%
\pgfsetroundjoin%
\definecolor{currentfill}{rgb}{0.258965,0.251537,0.524736}%
\pgfsetfillcolor{currentfill}%
\pgfsetfillopacity{0.700000}%
\pgfsetlinewidth{0.000000pt}%
\definecolor{currentstroke}{rgb}{0.000000,0.000000,0.000000}%
\pgfsetstrokecolor{currentstroke}%
\pgfsetdash{}{0pt}%
\pgfpathmoveto{\pgfqpoint{3.076201in}{1.924510in}}%
\pgfpathlineto{\pgfqpoint{3.089770in}{1.910186in}}%
\pgfpathlineto{\pgfqpoint{3.103338in}{1.896007in}}%
\pgfpathlineto{\pgfqpoint{3.116904in}{1.881974in}}%
\pgfpathlineto{\pgfqpoint{3.130468in}{1.868086in}}%
\pgfpathlineto{\pgfqpoint{3.138862in}{1.866087in}}%
\pgfpathlineto{\pgfqpoint{3.147242in}{1.864337in}}%
\pgfpathlineto{\pgfqpoint{3.155607in}{1.862831in}}%
\pgfpathlineto{\pgfqpoint{3.163959in}{1.861566in}}%
\pgfpathlineto{\pgfqpoint{3.150433in}{1.874979in}}%
\pgfpathlineto{\pgfqpoint{3.136906in}{1.888537in}}%
\pgfpathlineto{\pgfqpoint{3.123377in}{1.902239in}}%
\pgfpathlineto{\pgfqpoint{3.109846in}{1.916087in}}%
\pgfpathlineto{\pgfqpoint{3.101456in}{1.917820in}}%
\pgfpathlineto{\pgfqpoint{3.093052in}{1.919799in}}%
\pgfpathlineto{\pgfqpoint{3.084634in}{1.922028in}}%
\pgfpathlineto{\pgfqpoint{3.076201in}{1.924510in}}%
\pgfpathclose%
\pgfusepath{fill}%
\end{pgfscope}%
\begin{pgfscope}%
\pgfpathrectangle{\pgfqpoint{1.254980in}{0.150000in}}{\pgfqpoint{5.490039in}{5.490039in}}%
\pgfusepath{clip}%
\pgfsetbuttcap%
\pgfsetroundjoin%
\definecolor{currentfill}{rgb}{0.214298,0.355619,0.551184}%
\pgfsetfillcolor{currentfill}%
\pgfsetfillopacity{0.700000}%
\pgfsetlinewidth{0.000000pt}%
\definecolor{currentstroke}{rgb}{0.000000,0.000000,0.000000}%
\pgfsetstrokecolor{currentstroke}%
\pgfsetdash{}{0pt}%
\pgfpathmoveto{\pgfqpoint{5.248434in}{2.101615in}}%
\pgfpathlineto{\pgfqpoint{5.262516in}{2.107373in}}%
\pgfpathlineto{\pgfqpoint{5.276613in}{2.113243in}}%
\pgfpathlineto{\pgfqpoint{5.290724in}{2.119225in}}%
\pgfpathlineto{\pgfqpoint{5.304851in}{2.125319in}}%
\pgfpathlineto{\pgfqpoint{5.312350in}{2.138096in}}%
\pgfpathlineto{\pgfqpoint{5.319844in}{2.150812in}}%
\pgfpathlineto{\pgfqpoint{5.327333in}{2.163468in}}%
\pgfpathlineto{\pgfqpoint{5.334816in}{2.176063in}}%
\pgfpathlineto{\pgfqpoint{5.320689in}{2.169814in}}%
\pgfpathlineto{\pgfqpoint{5.306578in}{2.163677in}}%
\pgfpathlineto{\pgfqpoint{5.292481in}{2.157653in}}%
\pgfpathlineto{\pgfqpoint{5.278399in}{2.151741in}}%
\pgfpathlineto{\pgfqpoint{5.270916in}{2.139295in}}%
\pgfpathlineto{\pgfqpoint{5.263427in}{2.126791in}}%
\pgfpathlineto{\pgfqpoint{5.255933in}{2.114231in}}%
\pgfpathlineto{\pgfqpoint{5.248434in}{2.101615in}}%
\pgfpathclose%
\pgfusepath{fill}%
\end{pgfscope}%
\begin{pgfscope}%
\pgfpathrectangle{\pgfqpoint{1.254980in}{0.150000in}}{\pgfqpoint{5.490039in}{5.490039in}}%
\pgfusepath{clip}%
\pgfsetbuttcap%
\pgfsetroundjoin%
\definecolor{currentfill}{rgb}{0.273809,0.031497,0.358853}%
\pgfsetfillcolor{currentfill}%
\pgfsetfillopacity{0.700000}%
\pgfsetlinewidth{0.000000pt}%
\definecolor{currentstroke}{rgb}{0.000000,0.000000,0.000000}%
\pgfsetstrokecolor{currentstroke}%
\pgfsetdash{}{0pt}%
\pgfpathmoveto{\pgfqpoint{4.329112in}{1.445750in}}%
\pgfpathlineto{\pgfqpoint{4.342770in}{1.443927in}}%
\pgfpathlineto{\pgfqpoint{4.356437in}{1.442218in}}%
\pgfpathlineto{\pgfqpoint{4.370112in}{1.440622in}}%
\pgfpathlineto{\pgfqpoint{4.383796in}{1.439140in}}%
\pgfpathlineto{\pgfqpoint{4.391539in}{1.449330in}}%
\pgfpathlineto{\pgfqpoint{4.399278in}{1.459591in}}%
\pgfpathlineto{\pgfqpoint{4.407011in}{1.469922in}}%
\pgfpathlineto{\pgfqpoint{4.414740in}{1.480320in}}%
\pgfpathlineto{\pgfqpoint{4.401064in}{1.481459in}}%
\pgfpathlineto{\pgfqpoint{4.387397in}{1.482712in}}%
\pgfpathlineto{\pgfqpoint{4.373738in}{1.484078in}}%
\pgfpathlineto{\pgfqpoint{4.360088in}{1.485558in}}%
\pgfpathlineto{\pgfqpoint{4.352351in}{1.475497in}}%
\pgfpathlineto{\pgfqpoint{4.344610in}{1.465507in}}%
\pgfpathlineto{\pgfqpoint{4.336863in}{1.455590in}}%
\pgfpathlineto{\pgfqpoint{4.329112in}{1.445750in}}%
\pgfpathclose%
\pgfusepath{fill}%
\end{pgfscope}%
\begin{pgfscope}%
\pgfpathrectangle{\pgfqpoint{1.254980in}{0.150000in}}{\pgfqpoint{5.490039in}{5.490039in}}%
\pgfusepath{clip}%
\pgfsetbuttcap%
\pgfsetroundjoin%
\definecolor{currentfill}{rgb}{0.281924,0.089666,0.412415}%
\pgfsetfillcolor{currentfill}%
\pgfsetfillopacity{0.700000}%
\pgfsetlinewidth{0.000000pt}%
\definecolor{currentstroke}{rgb}{0.000000,0.000000,0.000000}%
\pgfsetstrokecolor{currentstroke}%
\pgfsetdash{}{0pt}%
\pgfpathmoveto{\pgfqpoint{3.488454in}{1.581289in}}%
\pgfpathlineto{\pgfqpoint{3.501982in}{1.571285in}}%
\pgfpathlineto{\pgfqpoint{3.515512in}{1.561410in}}%
\pgfpathlineto{\pgfqpoint{3.529043in}{1.551664in}}%
\pgfpathlineto{\pgfqpoint{3.542577in}{1.542047in}}%
\pgfpathlineto{\pgfqpoint{3.550691in}{1.544216in}}%
\pgfpathlineto{\pgfqpoint{3.558794in}{1.546585in}}%
\pgfpathlineto{\pgfqpoint{3.566888in}{1.549151in}}%
\pgfpathlineto{\pgfqpoint{3.574972in}{1.551909in}}%
\pgfpathlineto{\pgfqpoint{3.561466in}{1.561081in}}%
\pgfpathlineto{\pgfqpoint{3.547962in}{1.570382in}}%
\pgfpathlineto{\pgfqpoint{3.534460in}{1.579811in}}%
\pgfpathlineto{\pgfqpoint{3.520960in}{1.589369in}}%
\pgfpathlineto{\pgfqpoint{3.512849in}{1.587049in}}%
\pgfpathlineto{\pgfqpoint{3.504728in}{1.584927in}}%
\pgfpathlineto{\pgfqpoint{3.496596in}{1.583006in}}%
\pgfpathlineto{\pgfqpoint{3.488454in}{1.581289in}}%
\pgfpathclose%
\pgfusepath{fill}%
\end{pgfscope}%
\begin{pgfscope}%
\pgfpathrectangle{\pgfqpoint{1.254980in}{0.150000in}}{\pgfqpoint{5.490039in}{5.490039in}}%
\pgfusepath{clip}%
\pgfsetbuttcap%
\pgfsetroundjoin%
\definecolor{currentfill}{rgb}{0.296479,0.761561,0.424223}%
\pgfsetfillcolor{currentfill}%
\pgfsetfillopacity{0.700000}%
\pgfsetlinewidth{0.000000pt}%
\definecolor{currentstroke}{rgb}{0.000000,0.000000,0.000000}%
\pgfsetstrokecolor{currentstroke}%
\pgfsetdash{}{0pt}%
\pgfpathmoveto{\pgfqpoint{2.158538in}{3.312991in}}%
\pgfpathlineto{\pgfqpoint{2.172535in}{3.286678in}}%
\pgfpathlineto{\pgfqpoint{2.186519in}{3.260591in}}%
\pgfpathlineto{\pgfqpoint{2.200489in}{3.234727in}}%
\pgfpathlineto{\pgfqpoint{2.214446in}{3.209084in}}%
\pgfpathlineto{\pgfqpoint{2.223579in}{3.200502in}}%
\pgfpathlineto{\pgfqpoint{2.232688in}{3.192229in}}%
\pgfpathlineto{\pgfqpoint{2.241774in}{3.184262in}}%
\pgfpathlineto{\pgfqpoint{2.250837in}{3.176595in}}%
\pgfpathlineto{\pgfqpoint{2.236940in}{3.201740in}}%
\pgfpathlineto{\pgfqpoint{2.223030in}{3.227104in}}%
\pgfpathlineto{\pgfqpoint{2.209108in}{3.252690in}}%
\pgfpathlineto{\pgfqpoint{2.195172in}{3.278500in}}%
\pgfpathlineto{\pgfqpoint{2.186049in}{3.286659in}}%
\pgfpathlineto{\pgfqpoint{2.176903in}{3.295124in}}%
\pgfpathlineto{\pgfqpoint{2.167732in}{3.303900in}}%
\pgfpathlineto{\pgfqpoint{2.158538in}{3.312991in}}%
\pgfpathclose%
\pgfusepath{fill}%
\end{pgfscope}%
\begin{pgfscope}%
\pgfpathrectangle{\pgfqpoint{1.254980in}{0.150000in}}{\pgfqpoint{5.490039in}{5.490039in}}%
\pgfusepath{clip}%
\pgfsetbuttcap%
\pgfsetroundjoin%
\definecolor{currentfill}{rgb}{0.250425,0.274290,0.533103}%
\pgfsetfillcolor{currentfill}%
\pgfsetfillopacity{0.700000}%
\pgfsetlinewidth{0.000000pt}%
\definecolor{currentstroke}{rgb}{0.000000,0.000000,0.000000}%
\pgfsetstrokecolor{currentstroke}%
\pgfsetdash{}{0pt}%
\pgfpathmoveto{\pgfqpoint{5.045881in}{1.908789in}}%
\pgfpathlineto{\pgfqpoint{5.059852in}{1.913090in}}%
\pgfpathlineto{\pgfqpoint{5.073837in}{1.917503in}}%
\pgfpathlineto{\pgfqpoint{5.087835in}{1.922027in}}%
\pgfpathlineto{\pgfqpoint{5.101847in}{1.926663in}}%
\pgfpathlineto{\pgfqpoint{5.109408in}{1.939641in}}%
\pgfpathlineto{\pgfqpoint{5.116964in}{1.952582in}}%
\pgfpathlineto{\pgfqpoint{5.124515in}{1.965486in}}%
\pgfpathlineto{\pgfqpoint{5.132061in}{1.978351in}}%
\pgfpathlineto{\pgfqpoint{5.118049in}{1.973512in}}%
\pgfpathlineto{\pgfqpoint{5.104051in}{1.968784in}}%
\pgfpathlineto{\pgfqpoint{5.090066in}{1.964169in}}%
\pgfpathlineto{\pgfqpoint{5.076095in}{1.959666in}}%
\pgfpathlineto{\pgfqpoint{5.068549in}{1.946997in}}%
\pgfpathlineto{\pgfqpoint{5.060997in}{1.934294in}}%
\pgfpathlineto{\pgfqpoint{5.053442in}{1.921558in}}%
\pgfpathlineto{\pgfqpoint{5.045881in}{1.908789in}}%
\pgfpathclose%
\pgfusepath{fill}%
\end{pgfscope}%
\begin{pgfscope}%
\pgfpathrectangle{\pgfqpoint{1.254980in}{0.150000in}}{\pgfqpoint{5.490039in}{5.490039in}}%
\pgfusepath{clip}%
\pgfsetbuttcap%
\pgfsetroundjoin%
\definecolor{currentfill}{rgb}{0.123444,0.636809,0.528763}%
\pgfsetfillcolor{currentfill}%
\pgfsetfillopacity{0.700000}%
\pgfsetlinewidth{0.000000pt}%
\definecolor{currentstroke}{rgb}{0.000000,0.000000,0.000000}%
\pgfsetstrokecolor{currentstroke}%
\pgfsetdash{}{0pt}%
\pgfpathmoveto{\pgfqpoint{2.344942in}{2.949432in}}%
\pgfpathlineto{\pgfqpoint{2.358806in}{2.926023in}}%
\pgfpathlineto{\pgfqpoint{2.372659in}{2.902815in}}%
\pgfpathlineto{\pgfqpoint{2.386501in}{2.879808in}}%
\pgfpathlineto{\pgfqpoint{2.400334in}{2.857000in}}%
\pgfpathlineto{\pgfqpoint{2.409321in}{2.849212in}}%
\pgfpathlineto{\pgfqpoint{2.418286in}{2.841728in}}%
\pgfpathlineto{\pgfqpoint{2.427230in}{2.834544in}}%
\pgfpathlineto{\pgfqpoint{2.436153in}{2.827656in}}%
\pgfpathlineto{\pgfqpoint{2.422377in}{2.849963in}}%
\pgfpathlineto{\pgfqpoint{2.408591in}{2.872468in}}%
\pgfpathlineto{\pgfqpoint{2.394795in}{2.895172in}}%
\pgfpathlineto{\pgfqpoint{2.380988in}{2.918077in}}%
\pgfpathlineto{\pgfqpoint{2.372010in}{2.925460in}}%
\pgfpathlineto{\pgfqpoint{2.363010in}{2.933143in}}%
\pgfpathlineto{\pgfqpoint{2.353987in}{2.941133in}}%
\pgfpathlineto{\pgfqpoint{2.344942in}{2.949432in}}%
\pgfpathclose%
\pgfusepath{fill}%
\end{pgfscope}%
\begin{pgfscope}%
\pgfpathrectangle{\pgfqpoint{1.254980in}{0.150000in}}{\pgfqpoint{5.490039in}{5.490039in}}%
\pgfusepath{clip}%
\pgfsetbuttcap%
\pgfsetroundjoin%
\definecolor{currentfill}{rgb}{0.266580,0.228262,0.514349}%
\pgfsetfillcolor{currentfill}%
\pgfsetfillopacity{0.700000}%
\pgfsetlinewidth{0.000000pt}%
\definecolor{currentstroke}{rgb}{0.000000,0.000000,0.000000}%
\pgfsetstrokecolor{currentstroke}%
\pgfsetdash{}{0pt}%
\pgfpathmoveto{\pgfqpoint{3.130468in}{1.868086in}}%
\pgfpathlineto{\pgfqpoint{3.144030in}{1.854342in}}%
\pgfpathlineto{\pgfqpoint{3.157591in}{1.840741in}}%
\pgfpathlineto{\pgfqpoint{3.171150in}{1.827282in}}%
\pgfpathlineto{\pgfqpoint{3.184708in}{1.813966in}}%
\pgfpathlineto{\pgfqpoint{3.193064in}{1.812448in}}%
\pgfpathlineto{\pgfqpoint{3.201406in}{1.811174in}}%
\pgfpathlineto{\pgfqpoint{3.209735in}{1.810140in}}%
\pgfpathlineto{\pgfqpoint{3.218050in}{1.809341in}}%
\pgfpathlineto{\pgfqpoint{3.204529in}{1.822185in}}%
\pgfpathlineto{\pgfqpoint{3.191007in}{1.835169in}}%
\pgfpathlineto{\pgfqpoint{3.177484in}{1.848296in}}%
\pgfpathlineto{\pgfqpoint{3.163959in}{1.861566in}}%
\pgfpathlineto{\pgfqpoint{3.155607in}{1.862831in}}%
\pgfpathlineto{\pgfqpoint{3.147242in}{1.864337in}}%
\pgfpathlineto{\pgfqpoint{3.138862in}{1.866087in}}%
\pgfpathlineto{\pgfqpoint{3.130468in}{1.868086in}}%
\pgfpathclose%
\pgfusepath{fill}%
\end{pgfscope}%
\begin{pgfscope}%
\pgfpathrectangle{\pgfqpoint{1.254980in}{0.150000in}}{\pgfqpoint{5.490039in}{5.490039in}}%
\pgfusepath{clip}%
\pgfsetbuttcap%
\pgfsetroundjoin%
\definecolor{currentfill}{rgb}{0.280868,0.160771,0.472899}%
\pgfsetfillcolor{currentfill}%
\pgfsetfillopacity{0.700000}%
\pgfsetlinewidth{0.000000pt}%
\definecolor{currentstroke}{rgb}{0.000000,0.000000,0.000000}%
\pgfsetstrokecolor{currentstroke}%
\pgfsetdash{}{0pt}%
\pgfpathmoveto{\pgfqpoint{4.757573in}{1.670237in}}%
\pgfpathlineto{\pgfqpoint{4.771401in}{1.672227in}}%
\pgfpathlineto{\pgfqpoint{4.785241in}{1.674328in}}%
\pgfpathlineto{\pgfqpoint{4.799092in}{1.676541in}}%
\pgfpathlineto{\pgfqpoint{4.812955in}{1.678866in}}%
\pgfpathlineto{\pgfqpoint{4.820589in}{1.691373in}}%
\pgfpathlineto{\pgfqpoint{4.828219in}{1.703883in}}%
\pgfpathlineto{\pgfqpoint{4.835845in}{1.716396in}}%
\pgfpathlineto{\pgfqpoint{4.843466in}{1.728909in}}%
\pgfpathlineto{\pgfqpoint{4.829605in}{1.726318in}}%
\pgfpathlineto{\pgfqpoint{4.815755in}{1.723839in}}%
\pgfpathlineto{\pgfqpoint{4.801917in}{1.721472in}}%
\pgfpathlineto{\pgfqpoint{4.788091in}{1.719217in}}%
\pgfpathlineto{\pgfqpoint{4.780468in}{1.706964in}}%
\pgfpathlineto{\pgfqpoint{4.772841in}{1.694714in}}%
\pgfpathlineto{\pgfqpoint{4.765209in}{1.682472in}}%
\pgfpathlineto{\pgfqpoint{4.757573in}{1.670237in}}%
\pgfpathclose%
\pgfusepath{fill}%
\end{pgfscope}%
\begin{pgfscope}%
\pgfpathrectangle{\pgfqpoint{1.254980in}{0.150000in}}{\pgfqpoint{5.490039in}{5.490039in}}%
\pgfusepath{clip}%
\pgfsetbuttcap%
\pgfsetroundjoin%
\definecolor{currentfill}{rgb}{0.177423,0.437527,0.557565}%
\pgfsetfillcolor{currentfill}%
\pgfsetfillopacity{0.700000}%
\pgfsetlinewidth{0.000000pt}%
\definecolor{currentstroke}{rgb}{0.000000,0.000000,0.000000}%
\pgfsetstrokecolor{currentstroke}%
\pgfsetdash{}{0pt}%
\pgfpathmoveto{\pgfqpoint{2.694844in}{2.388008in}}%
\pgfpathlineto{\pgfqpoint{2.708529in}{2.369273in}}%
\pgfpathlineto{\pgfqpoint{2.722207in}{2.350708in}}%
\pgfpathlineto{\pgfqpoint{2.735880in}{2.332312in}}%
\pgfpathlineto{\pgfqpoint{2.749546in}{2.314082in}}%
\pgfpathlineto{\pgfqpoint{2.758245in}{2.308644in}}%
\pgfpathlineto{\pgfqpoint{2.766925in}{2.303490in}}%
\pgfpathlineto{\pgfqpoint{2.775587in}{2.298615in}}%
\pgfpathlineto{\pgfqpoint{2.784232in}{2.294016in}}%
\pgfpathlineto{\pgfqpoint{2.770613in}{2.311750in}}%
\pgfpathlineto{\pgfqpoint{2.756989in}{2.329651in}}%
\pgfpathlineto{\pgfqpoint{2.743359in}{2.347720in}}%
\pgfpathlineto{\pgfqpoint{2.729723in}{2.365957in}}%
\pgfpathlineto{\pgfqpoint{2.721032in}{2.371045in}}%
\pgfpathlineto{\pgfqpoint{2.712321in}{2.376413in}}%
\pgfpathlineto{\pgfqpoint{2.703592in}{2.382066in}}%
\pgfpathlineto{\pgfqpoint{2.694844in}{2.388008in}}%
\pgfpathclose%
\pgfusepath{fill}%
\end{pgfscope}%
\begin{pgfscope}%
\pgfpathrectangle{\pgfqpoint{1.254980in}{0.150000in}}{\pgfqpoint{5.490039in}{5.490039in}}%
\pgfusepath{clip}%
\pgfsetbuttcap%
\pgfsetroundjoin%
\definecolor{currentfill}{rgb}{0.268510,0.009605,0.335427}%
\pgfsetfillcolor{currentfill}%
\pgfsetfillopacity{0.700000}%
\pgfsetlinewidth{0.000000pt}%
\definecolor{currentstroke}{rgb}{0.000000,0.000000,0.000000}%
\pgfsetstrokecolor{currentstroke}%
\pgfsetdash{}{0pt}%
\pgfpathmoveto{\pgfqpoint{3.877502in}{1.416156in}}%
\pgfpathlineto{\pgfqpoint{3.891059in}{1.410007in}}%
\pgfpathlineto{\pgfqpoint{3.904620in}{1.403978in}}%
\pgfpathlineto{\pgfqpoint{3.918186in}{1.398069in}}%
\pgfpathlineto{\pgfqpoint{3.931758in}{1.392278in}}%
\pgfpathlineto{\pgfqpoint{3.939668in}{1.398433in}}%
\pgfpathlineto{\pgfqpoint{3.947571in}{1.404734in}}%
\pgfpathlineto{\pgfqpoint{3.955467in}{1.411177in}}%
\pgfpathlineto{\pgfqpoint{3.963356in}{1.417758in}}%
\pgfpathlineto{\pgfqpoint{3.949802in}{1.423142in}}%
\pgfpathlineto{\pgfqpoint{3.936254in}{1.428645in}}%
\pgfpathlineto{\pgfqpoint{3.922710in}{1.434267in}}%
\pgfpathlineto{\pgfqpoint{3.909172in}{1.440008in}}%
\pgfpathlineto{\pgfqpoint{3.901265in}{1.433827in}}%
\pgfpathlineto{\pgfqpoint{3.893351in}{1.427789in}}%
\pgfpathlineto{\pgfqpoint{3.885430in}{1.421898in}}%
\pgfpathlineto{\pgfqpoint{3.877502in}{1.416156in}}%
\pgfpathclose%
\pgfusepath{fill}%
\end{pgfscope}%
\begin{pgfscope}%
\pgfpathrectangle{\pgfqpoint{1.254980in}{0.150000in}}{\pgfqpoint{5.490039in}{5.490039in}}%
\pgfusepath{clip}%
\pgfsetbuttcap%
\pgfsetroundjoin%
\definecolor{currentfill}{rgb}{0.269944,0.014625,0.341379}%
\pgfsetfillcolor{currentfill}%
\pgfsetfillopacity{0.700000}%
\pgfsetlinewidth{0.000000pt}%
\definecolor{currentstroke}{rgb}{0.000000,0.000000,0.000000}%
\pgfsetstrokecolor{currentstroke}%
\pgfsetdash{}{0pt}%
\pgfpathmoveto{\pgfqpoint{4.243464in}{1.417070in}}%
\pgfpathlineto{\pgfqpoint{4.257100in}{1.414432in}}%
\pgfpathlineto{\pgfqpoint{4.270745in}{1.411909in}}%
\pgfpathlineto{\pgfqpoint{4.284396in}{1.409500in}}%
\pgfpathlineto{\pgfqpoint{4.298056in}{1.407204in}}%
\pgfpathlineto{\pgfqpoint{4.305827in}{1.416712in}}%
\pgfpathlineto{\pgfqpoint{4.313594in}{1.426308in}}%
\pgfpathlineto{\pgfqpoint{4.321355in}{1.435988in}}%
\pgfpathlineto{\pgfqpoint{4.329112in}{1.445750in}}%
\pgfpathlineto{\pgfqpoint{4.315461in}{1.447686in}}%
\pgfpathlineto{\pgfqpoint{4.301819in}{1.449737in}}%
\pgfpathlineto{\pgfqpoint{4.288184in}{1.451902in}}%
\pgfpathlineto{\pgfqpoint{4.274558in}{1.454182in}}%
\pgfpathlineto{\pgfqpoint{4.266792in}{1.444773in}}%
\pgfpathlineto{\pgfqpoint{4.259021in}{1.435449in}}%
\pgfpathlineto{\pgfqpoint{4.251245in}{1.426214in}}%
\pgfpathlineto{\pgfqpoint{4.243464in}{1.417070in}}%
\pgfpathclose%
\pgfusepath{fill}%
\end{pgfscope}%
\begin{pgfscope}%
\pgfpathrectangle{\pgfqpoint{1.254980in}{0.150000in}}{\pgfqpoint{5.490039in}{5.490039in}}%
\pgfusepath{clip}%
\pgfsetbuttcap%
\pgfsetroundjoin%
\definecolor{currentfill}{rgb}{0.267004,0.004874,0.329415}%
\pgfsetfillcolor{currentfill}%
\pgfsetfillopacity{0.700000}%
\pgfsetlinewidth{0.000000pt}%
\definecolor{currentstroke}{rgb}{0.000000,0.000000,0.000000}%
\pgfsetstrokecolor{currentstroke}%
\pgfsetdash{}{0pt}%
\pgfpathmoveto{\pgfqpoint{4.017627in}{1.397408in}}%
\pgfpathlineto{\pgfqpoint{4.031208in}{1.392615in}}%
\pgfpathlineto{\pgfqpoint{4.044796in}{1.387939in}}%
\pgfpathlineto{\pgfqpoint{4.058390in}{1.383380in}}%
\pgfpathlineto{\pgfqpoint{4.071989in}{1.378938in}}%
\pgfpathlineto{\pgfqpoint{4.079841in}{1.386448in}}%
\pgfpathlineto{\pgfqpoint{4.087687in}{1.394083in}}%
\pgfpathlineto{\pgfqpoint{4.095526in}{1.401838in}}%
\pgfpathlineto{\pgfqpoint{4.103359in}{1.409711in}}%
\pgfpathlineto{\pgfqpoint{4.089774in}{1.413763in}}%
\pgfpathlineto{\pgfqpoint{4.076194in}{1.417932in}}%
\pgfpathlineto{\pgfqpoint{4.062621in}{1.422217in}}%
\pgfpathlineto{\pgfqpoint{4.049054in}{1.426620in}}%
\pgfpathlineto{\pgfqpoint{4.041207in}{1.419131in}}%
\pgfpathlineto{\pgfqpoint{4.033353in}{1.411764in}}%
\pgfpathlineto{\pgfqpoint{4.025493in}{1.404522in}}%
\pgfpathlineto{\pgfqpoint{4.017627in}{1.397408in}}%
\pgfpathclose%
\pgfusepath{fill}%
\end{pgfscope}%
\begin{pgfscope}%
\pgfpathrectangle{\pgfqpoint{1.254980in}{0.150000in}}{\pgfqpoint{5.490039in}{5.490039in}}%
\pgfusepath{clip}%
\pgfsetbuttcap%
\pgfsetroundjoin%
\definecolor{currentfill}{rgb}{0.273006,0.204520,0.501721}%
\pgfsetfillcolor{currentfill}%
\pgfsetfillopacity{0.700000}%
\pgfsetlinewidth{0.000000pt}%
\definecolor{currentstroke}{rgb}{0.000000,0.000000,0.000000}%
\pgfsetstrokecolor{currentstroke}%
\pgfsetdash{}{0pt}%
\pgfpathmoveto{\pgfqpoint{3.184708in}{1.813966in}}%
\pgfpathlineto{\pgfqpoint{3.198264in}{1.800791in}}%
\pgfpathlineto{\pgfqpoint{3.211820in}{1.787757in}}%
\pgfpathlineto{\pgfqpoint{3.225375in}{1.774864in}}%
\pgfpathlineto{\pgfqpoint{3.238928in}{1.762109in}}%
\pgfpathlineto{\pgfqpoint{3.247248in}{1.761070in}}%
\pgfpathlineto{\pgfqpoint{3.255554in}{1.760270in}}%
\pgfpathlineto{\pgfqpoint{3.263847in}{1.759705in}}%
\pgfpathlineto{\pgfqpoint{3.272128in}{1.759372in}}%
\pgfpathlineto{\pgfqpoint{3.258609in}{1.771655in}}%
\pgfpathlineto{\pgfqpoint{3.245090in}{1.784077in}}%
\pgfpathlineto{\pgfqpoint{3.231571in}{1.796639in}}%
\pgfpathlineto{\pgfqpoint{3.218050in}{1.809341in}}%
\pgfpathlineto{\pgfqpoint{3.209735in}{1.810140in}}%
\pgfpathlineto{\pgfqpoint{3.201406in}{1.811174in}}%
\pgfpathlineto{\pgfqpoint{3.193064in}{1.812448in}}%
\pgfpathlineto{\pgfqpoint{3.184708in}{1.813966in}}%
\pgfpathclose%
\pgfusepath{fill}%
\end{pgfscope}%
\begin{pgfscope}%
\pgfpathrectangle{\pgfqpoint{1.254980in}{0.150000in}}{\pgfqpoint{5.490039in}{5.490039in}}%
\pgfusepath{clip}%
\pgfsetbuttcap%
\pgfsetroundjoin%
\definecolor{currentfill}{rgb}{0.166617,0.463708,0.558119}%
\pgfsetfillcolor{currentfill}%
\pgfsetfillopacity{0.700000}%
\pgfsetlinewidth{0.000000pt}%
\definecolor{currentstroke}{rgb}{0.000000,0.000000,0.000000}%
\pgfsetstrokecolor{currentstroke}%
\pgfsetdash{}{0pt}%
\pgfpathmoveto{\pgfqpoint{2.640043in}{2.464661in}}%
\pgfpathlineto{\pgfqpoint{2.653753in}{2.445238in}}%
\pgfpathlineto{\pgfqpoint{2.667457in}{2.425989in}}%
\pgfpathlineto{\pgfqpoint{2.681154in}{2.406913in}}%
\pgfpathlineto{\pgfqpoint{2.694844in}{2.388008in}}%
\pgfpathlineto{\pgfqpoint{2.703592in}{2.382066in}}%
\pgfpathlineto{\pgfqpoint{2.712321in}{2.376413in}}%
\pgfpathlineto{\pgfqpoint{2.721032in}{2.371045in}}%
\pgfpathlineto{\pgfqpoint{2.729723in}{2.365957in}}%
\pgfpathlineto{\pgfqpoint{2.716082in}{2.384364in}}%
\pgfpathlineto{\pgfqpoint{2.702435in}{2.402942in}}%
\pgfpathlineto{\pgfqpoint{2.688781in}{2.421691in}}%
\pgfpathlineto{\pgfqpoint{2.675121in}{2.440613in}}%
\pgfpathlineto{\pgfqpoint{2.666380in}{2.446192in}}%
\pgfpathlineto{\pgfqpoint{2.657620in}{2.452057in}}%
\pgfpathlineto{\pgfqpoint{2.648841in}{2.458211in}}%
\pgfpathlineto{\pgfqpoint{2.640043in}{2.464661in}}%
\pgfpathclose%
\pgfusepath{fill}%
\end{pgfscope}%
\begin{pgfscope}%
\pgfpathrectangle{\pgfqpoint{1.254980in}{0.150000in}}{\pgfqpoint{5.490039in}{5.490039in}}%
\pgfusepath{clip}%
\pgfsetbuttcap%
\pgfsetroundjoin%
\definecolor{currentfill}{rgb}{0.276194,0.190074,0.493001}%
\pgfsetfillcolor{currentfill}%
\pgfsetfillopacity{0.700000}%
\pgfsetlinewidth{0.000000pt}%
\definecolor{currentstroke}{rgb}{0.000000,0.000000,0.000000}%
\pgfsetstrokecolor{currentstroke}%
\pgfsetdash{}{0pt}%
\pgfpathmoveto{\pgfqpoint{4.843466in}{1.728909in}}%
\pgfpathlineto{\pgfqpoint{4.857339in}{1.731612in}}%
\pgfpathlineto{\pgfqpoint{4.871224in}{1.734426in}}%
\pgfpathlineto{\pgfqpoint{4.885121in}{1.737352in}}%
\pgfpathlineto{\pgfqpoint{4.899030in}{1.740390in}}%
\pgfpathlineto{\pgfqpoint{4.906646in}{1.753158in}}%
\pgfpathlineto{\pgfqpoint{4.914258in}{1.765918in}}%
\pgfpathlineto{\pgfqpoint{4.921865in}{1.778669in}}%
\pgfpathlineto{\pgfqpoint{4.929467in}{1.791409in}}%
\pgfpathlineto{\pgfqpoint{4.915559in}{1.788120in}}%
\pgfpathlineto{\pgfqpoint{4.901663in}{1.784944in}}%
\pgfpathlineto{\pgfqpoint{4.887779in}{1.781879in}}%
\pgfpathlineto{\pgfqpoint{4.873907in}{1.778927in}}%
\pgfpathlineto{\pgfqpoint{4.866303in}{1.766431in}}%
\pgfpathlineto{\pgfqpoint{4.858695in}{1.753929in}}%
\pgfpathlineto{\pgfqpoint{4.851083in}{1.741421in}}%
\pgfpathlineto{\pgfqpoint{4.843466in}{1.728909in}}%
\pgfpathclose%
\pgfusepath{fill}%
\end{pgfscope}%
\begin{pgfscope}%
\pgfpathrectangle{\pgfqpoint{1.254980in}{0.150000in}}{\pgfqpoint{5.490039in}{5.490039in}}%
\pgfusepath{clip}%
\pgfsetbuttcap%
\pgfsetroundjoin%
\definecolor{currentfill}{rgb}{0.272594,0.025563,0.353093}%
\pgfsetfillcolor{currentfill}%
\pgfsetfillopacity{0.700000}%
\pgfsetlinewidth{0.000000pt}%
\definecolor{currentstroke}{rgb}{0.000000,0.000000,0.000000}%
\pgfsetstrokecolor{currentstroke}%
\pgfsetdash{}{0pt}%
\pgfpathmoveto{\pgfqpoint{3.737253in}{1.451692in}}%
\pgfpathlineto{\pgfqpoint{3.750797in}{1.444149in}}%
\pgfpathlineto{\pgfqpoint{3.764345in}{1.436728in}}%
\pgfpathlineto{\pgfqpoint{3.777896in}{1.429430in}}%
\pgfpathlineto{\pgfqpoint{3.791451in}{1.422253in}}%
\pgfpathlineto{\pgfqpoint{3.799431in}{1.426930in}}%
\pgfpathlineto{\pgfqpoint{3.807402in}{1.431775in}}%
\pgfpathlineto{\pgfqpoint{3.815366in}{1.436784in}}%
\pgfpathlineto{\pgfqpoint{3.823321in}{1.441955in}}%
\pgfpathlineto{\pgfqpoint{3.809787in}{1.448707in}}%
\pgfpathlineto{\pgfqpoint{3.796257in}{1.455580in}}%
\pgfpathlineto{\pgfqpoint{3.782731in}{1.462576in}}%
\pgfpathlineto{\pgfqpoint{3.769208in}{1.469694in}}%
\pgfpathlineto{\pgfqpoint{3.761232in}{1.464942in}}%
\pgfpathlineto{\pgfqpoint{3.753247in}{1.460356in}}%
\pgfpathlineto{\pgfqpoint{3.745255in}{1.455938in}}%
\pgfpathlineto{\pgfqpoint{3.737253in}{1.451692in}}%
\pgfpathclose%
\pgfusepath{fill}%
\end{pgfscope}%
\begin{pgfscope}%
\pgfpathrectangle{\pgfqpoint{1.254980in}{0.150000in}}{\pgfqpoint{5.490039in}{5.490039in}}%
\pgfusepath{clip}%
\pgfsetbuttcap%
\pgfsetroundjoin%
\definecolor{currentfill}{rgb}{0.280894,0.078907,0.402329}%
\pgfsetfillcolor{currentfill}%
\pgfsetfillopacity{0.700000}%
\pgfsetlinewidth{0.000000pt}%
\definecolor{currentstroke}{rgb}{0.000000,0.000000,0.000000}%
\pgfsetstrokecolor{currentstroke}%
\pgfsetdash{}{0pt}%
\pgfpathmoveto{\pgfqpoint{3.542577in}{1.542047in}}%
\pgfpathlineto{\pgfqpoint{3.556112in}{1.532558in}}%
\pgfpathlineto{\pgfqpoint{3.569649in}{1.523196in}}%
\pgfpathlineto{\pgfqpoint{3.583189in}{1.513962in}}%
\pgfpathlineto{\pgfqpoint{3.596731in}{1.504854in}}%
\pgfpathlineto{\pgfqpoint{3.604818in}{1.507474in}}%
\pgfpathlineto{\pgfqpoint{3.612895in}{1.510290in}}%
\pgfpathlineto{\pgfqpoint{3.620962in}{1.513298in}}%
\pgfpathlineto{\pgfqpoint{3.629020in}{1.516495in}}%
\pgfpathlineto{\pgfqpoint{3.615505in}{1.525158in}}%
\pgfpathlineto{\pgfqpoint{3.601991in}{1.533948in}}%
\pgfpathlineto{\pgfqpoint{3.588480in}{1.542865in}}%
\pgfpathlineto{\pgfqpoint{3.574972in}{1.551909in}}%
\pgfpathlineto{\pgfqpoint{3.566888in}{1.549151in}}%
\pgfpathlineto{\pgfqpoint{3.558794in}{1.546585in}}%
\pgfpathlineto{\pgfqpoint{3.550691in}{1.544216in}}%
\pgfpathlineto{\pgfqpoint{3.542577in}{1.542047in}}%
\pgfpathclose%
\pgfusepath{fill}%
\end{pgfscope}%
\begin{pgfscope}%
\pgfpathrectangle{\pgfqpoint{1.254980in}{0.150000in}}{\pgfqpoint{5.490039in}{5.490039in}}%
\pgfusepath{clip}%
\pgfsetbuttcap%
\pgfsetroundjoin%
\definecolor{currentfill}{rgb}{0.197636,0.391528,0.554969}%
\pgfsetfillcolor{currentfill}%
\pgfsetfillopacity{0.700000}%
\pgfsetlinewidth{0.000000pt}%
\definecolor{currentstroke}{rgb}{0.000000,0.000000,0.000000}%
\pgfsetstrokecolor{currentstroke}%
\pgfsetdash{}{0pt}%
\pgfpathmoveto{\pgfqpoint{5.334816in}{2.176063in}}%
\pgfpathlineto{\pgfqpoint{5.348957in}{2.182424in}}%
\pgfpathlineto{\pgfqpoint{5.363113in}{2.188898in}}%
\pgfpathlineto{\pgfqpoint{5.377285in}{2.195484in}}%
\pgfpathlineto{\pgfqpoint{5.391472in}{2.202183in}}%
\pgfpathlineto{\pgfqpoint{5.398949in}{2.214859in}}%
\pgfpathlineto{\pgfqpoint{5.406420in}{2.227468in}}%
\pgfpathlineto{\pgfqpoint{5.413886in}{2.240007in}}%
\pgfpathlineto{\pgfqpoint{5.421345in}{2.252477in}}%
\pgfpathlineto{\pgfqpoint{5.407158in}{2.245640in}}%
\pgfpathlineto{\pgfqpoint{5.392987in}{2.238915in}}%
\pgfpathlineto{\pgfqpoint{5.378831in}{2.232303in}}%
\pgfpathlineto{\pgfqpoint{5.364689in}{2.225804in}}%
\pgfpathlineto{\pgfqpoint{5.357230in}{2.213466in}}%
\pgfpathlineto{\pgfqpoint{5.349764in}{2.201062in}}%
\pgfpathlineto{\pgfqpoint{5.342293in}{2.188594in}}%
\pgfpathlineto{\pgfqpoint{5.334816in}{2.176063in}}%
\pgfpathclose%
\pgfusepath{fill}%
\end{pgfscope}%
\begin{pgfscope}%
\pgfpathrectangle{\pgfqpoint{1.254980in}{0.150000in}}{\pgfqpoint{5.490039in}{5.490039in}}%
\pgfusepath{clip}%
\pgfsetbuttcap%
\pgfsetroundjoin%
\definecolor{currentfill}{rgb}{0.235526,0.309527,0.542944}%
\pgfsetfillcolor{currentfill}%
\pgfsetfillopacity{0.700000}%
\pgfsetlinewidth{0.000000pt}%
\definecolor{currentstroke}{rgb}{0.000000,0.000000,0.000000}%
\pgfsetstrokecolor{currentstroke}%
\pgfsetdash{}{0pt}%
\pgfpathmoveto{\pgfqpoint{5.132061in}{1.978351in}}%
\pgfpathlineto{\pgfqpoint{5.146087in}{1.983303in}}%
\pgfpathlineto{\pgfqpoint{5.160126in}{1.988366in}}%
\pgfpathlineto{\pgfqpoint{5.174179in}{1.993541in}}%
\pgfpathlineto{\pgfqpoint{5.188247in}{1.998828in}}%
\pgfpathlineto{\pgfqpoint{5.195788in}{2.011846in}}%
\pgfpathlineto{\pgfqpoint{5.203325in}{2.024818in}}%
\pgfpathlineto{\pgfqpoint{5.210856in}{2.037743in}}%
\pgfpathlineto{\pgfqpoint{5.218382in}{2.050619in}}%
\pgfpathlineto{\pgfqpoint{5.204314in}{2.045145in}}%
\pgfpathlineto{\pgfqpoint{5.190261in}{2.039782in}}%
\pgfpathlineto{\pgfqpoint{5.176221in}{2.034532in}}%
\pgfpathlineto{\pgfqpoint{5.162195in}{2.029394in}}%
\pgfpathlineto{\pgfqpoint{5.154669in}{2.016698in}}%
\pgfpathlineto{\pgfqpoint{5.147138in}{2.003959in}}%
\pgfpathlineto{\pgfqpoint{5.139602in}{1.991176in}}%
\pgfpathlineto{\pgfqpoint{5.132061in}{1.978351in}}%
\pgfpathclose%
\pgfusepath{fill}%
\end{pgfscope}%
\begin{pgfscope}%
\pgfpathrectangle{\pgfqpoint{1.254980in}{0.150000in}}{\pgfqpoint{5.490039in}{5.490039in}}%
\pgfusepath{clip}%
\pgfsetbuttcap%
\pgfsetroundjoin%
\definecolor{currentfill}{rgb}{0.278012,0.180367,0.486697}%
\pgfsetfillcolor{currentfill}%
\pgfsetfillopacity{0.700000}%
\pgfsetlinewidth{0.000000pt}%
\definecolor{currentstroke}{rgb}{0.000000,0.000000,0.000000}%
\pgfsetstrokecolor{currentstroke}%
\pgfsetdash{}{0pt}%
\pgfpathmoveto{\pgfqpoint{3.238928in}{1.762109in}}%
\pgfpathlineto{\pgfqpoint{3.252482in}{1.749494in}}%
\pgfpathlineto{\pgfqpoint{3.266034in}{1.737017in}}%
\pgfpathlineto{\pgfqpoint{3.279586in}{1.724678in}}%
\pgfpathlineto{\pgfqpoint{3.293138in}{1.712476in}}%
\pgfpathlineto{\pgfqpoint{3.301422in}{1.711914in}}%
\pgfpathlineto{\pgfqpoint{3.309694in}{1.711586in}}%
\pgfpathlineto{\pgfqpoint{3.317953in}{1.711488in}}%
\pgfpathlineto{\pgfqpoint{3.326199in}{1.711617in}}%
\pgfpathlineto{\pgfqpoint{3.312682in}{1.723350in}}%
\pgfpathlineto{\pgfqpoint{3.299164in}{1.735220in}}%
\pgfpathlineto{\pgfqpoint{3.285646in}{1.747227in}}%
\pgfpathlineto{\pgfqpoint{3.272128in}{1.759372in}}%
\pgfpathlineto{\pgfqpoint{3.263847in}{1.759705in}}%
\pgfpathlineto{\pgfqpoint{3.255554in}{1.760270in}}%
\pgfpathlineto{\pgfqpoint{3.247248in}{1.761070in}}%
\pgfpathlineto{\pgfqpoint{3.238928in}{1.762109in}}%
\pgfpathclose%
\pgfusepath{fill}%
\end{pgfscope}%
\begin{pgfscope}%
\pgfpathrectangle{\pgfqpoint{1.254980in}{0.150000in}}{\pgfqpoint{5.490039in}{5.490039in}}%
\pgfusepath{clip}%
\pgfsetbuttcap%
\pgfsetroundjoin%
\definecolor{currentfill}{rgb}{0.153364,0.497000,0.557724}%
\pgfsetfillcolor{currentfill}%
\pgfsetfillopacity{0.700000}%
\pgfsetlinewidth{0.000000pt}%
\definecolor{currentstroke}{rgb}{0.000000,0.000000,0.000000}%
\pgfsetstrokecolor{currentstroke}%
\pgfsetdash{}{0pt}%
\pgfpathmoveto{\pgfqpoint{2.585130in}{2.544108in}}%
\pgfpathlineto{\pgfqpoint{2.598869in}{2.523981in}}%
\pgfpathlineto{\pgfqpoint{2.612601in}{2.504031in}}%
\pgfpathlineto{\pgfqpoint{2.626325in}{2.484258in}}%
\pgfpathlineto{\pgfqpoint{2.640043in}{2.464661in}}%
\pgfpathlineto{\pgfqpoint{2.648841in}{2.458211in}}%
\pgfpathlineto{\pgfqpoint{2.657620in}{2.452057in}}%
\pgfpathlineto{\pgfqpoint{2.666380in}{2.446192in}}%
\pgfpathlineto{\pgfqpoint{2.675121in}{2.440613in}}%
\pgfpathlineto{\pgfqpoint{2.661455in}{2.459708in}}%
\pgfpathlineto{\pgfqpoint{2.647782in}{2.478978in}}%
\pgfpathlineto{\pgfqpoint{2.634102in}{2.498424in}}%
\pgfpathlineto{\pgfqpoint{2.620415in}{2.518047in}}%
\pgfpathlineto{\pgfqpoint{2.611624in}{2.524121in}}%
\pgfpathlineto{\pgfqpoint{2.602813in}{2.530487in}}%
\pgfpathlineto{\pgfqpoint{2.593982in}{2.537147in}}%
\pgfpathlineto{\pgfqpoint{2.585130in}{2.544108in}}%
\pgfpathclose%
\pgfusepath{fill}%
\end{pgfscope}%
\begin{pgfscope}%
\pgfpathrectangle{\pgfqpoint{1.254980in}{0.150000in}}{\pgfqpoint{5.490039in}{5.490039in}}%
\pgfusepath{clip}%
\pgfsetbuttcap%
\pgfsetroundjoin%
\definecolor{currentfill}{rgb}{0.268510,0.009605,0.335427}%
\pgfsetfillcolor{currentfill}%
\pgfsetfillopacity{0.700000}%
\pgfsetlinewidth{0.000000pt}%
\definecolor{currentstroke}{rgb}{0.000000,0.000000,0.000000}%
\pgfsetstrokecolor{currentstroke}%
\pgfsetdash{}{0pt}%
\pgfpathmoveto{\pgfqpoint{4.157767in}{1.394666in}}%
\pgfpathlineto{\pgfqpoint{4.171386in}{1.391194in}}%
\pgfpathlineto{\pgfqpoint{4.185012in}{1.387838in}}%
\pgfpathlineto{\pgfqpoint{4.198645in}{1.384596in}}%
\pgfpathlineto{\pgfqpoint{4.212285in}{1.381469in}}%
\pgfpathlineto{\pgfqpoint{4.220088in}{1.390217in}}%
\pgfpathlineto{\pgfqpoint{4.227885in}{1.399069in}}%
\pgfpathlineto{\pgfqpoint{4.235677in}{1.408021in}}%
\pgfpathlineto{\pgfqpoint{4.243464in}{1.417070in}}%
\pgfpathlineto{\pgfqpoint{4.229835in}{1.419822in}}%
\pgfpathlineto{\pgfqpoint{4.216213in}{1.422690in}}%
\pgfpathlineto{\pgfqpoint{4.202599in}{1.425672in}}%
\pgfpathlineto{\pgfqpoint{4.188992in}{1.428770in}}%
\pgfpathlineto{\pgfqpoint{4.181194in}{1.420089in}}%
\pgfpathlineto{\pgfqpoint{4.173391in}{1.411509in}}%
\pgfpathlineto{\pgfqpoint{4.165582in}{1.403034in}}%
\pgfpathlineto{\pgfqpoint{4.157767in}{1.394666in}}%
\pgfpathclose%
\pgfusepath{fill}%
\end{pgfscope}%
\begin{pgfscope}%
\pgfpathrectangle{\pgfqpoint{1.254980in}{0.150000in}}{\pgfqpoint{5.490039in}{5.490039in}}%
\pgfusepath{clip}%
\pgfsetbuttcap%
\pgfsetroundjoin%
\definecolor{currentfill}{rgb}{0.146616,0.673050,0.508936}%
\pgfsetfillcolor{currentfill}%
\pgfsetfillopacity{0.700000}%
\pgfsetlinewidth{0.000000pt}%
\definecolor{currentstroke}{rgb}{0.000000,0.000000,0.000000}%
\pgfsetstrokecolor{currentstroke}%
\pgfsetdash{}{0pt}%
\pgfpathmoveto{\pgfqpoint{2.289378in}{3.045112in}}%
\pgfpathlineto{\pgfqpoint{2.303286in}{3.020883in}}%
\pgfpathlineto{\pgfqpoint{2.317182in}{2.996860in}}%
\pgfpathlineto{\pgfqpoint{2.331068in}{2.973044in}}%
\pgfpathlineto{\pgfqpoint{2.344942in}{2.949432in}}%
\pgfpathlineto{\pgfqpoint{2.353987in}{2.941133in}}%
\pgfpathlineto{\pgfqpoint{2.363010in}{2.933143in}}%
\pgfpathlineto{\pgfqpoint{2.372010in}{2.925460in}}%
\pgfpathlineto{\pgfqpoint{2.380988in}{2.918077in}}%
\pgfpathlineto{\pgfqpoint{2.367172in}{2.941183in}}%
\pgfpathlineto{\pgfqpoint{2.353345in}{2.964492in}}%
\pgfpathlineto{\pgfqpoint{2.339507in}{2.988006in}}%
\pgfpathlineto{\pgfqpoint{2.325658in}{3.011726in}}%
\pgfpathlineto{\pgfqpoint{2.316622in}{3.019608in}}%
\pgfpathlineto{\pgfqpoint{2.307563in}{3.027797in}}%
\pgfpathlineto{\pgfqpoint{2.298482in}{3.036297in}}%
\pgfpathlineto{\pgfqpoint{2.289378in}{3.045112in}}%
\pgfpathclose%
\pgfusepath{fill}%
\end{pgfscope}%
\begin{pgfscope}%
\pgfpathrectangle{\pgfqpoint{1.254980in}{0.150000in}}{\pgfqpoint{5.490039in}{5.490039in}}%
\pgfusepath{clip}%
\pgfsetbuttcap%
\pgfsetroundjoin%
\definecolor{currentfill}{rgb}{0.267968,0.223549,0.512008}%
\pgfsetfillcolor{currentfill}%
\pgfsetfillopacity{0.700000}%
\pgfsetlinewidth{0.000000pt}%
\definecolor{currentstroke}{rgb}{0.000000,0.000000,0.000000}%
\pgfsetstrokecolor{currentstroke}%
\pgfsetdash{}{0pt}%
\pgfpathmoveto{\pgfqpoint{4.929467in}{1.791409in}}%
\pgfpathlineto{\pgfqpoint{4.943388in}{1.794808in}}%
\pgfpathlineto{\pgfqpoint{4.957322in}{1.798320in}}%
\pgfpathlineto{\pgfqpoint{4.971268in}{1.801943in}}%
\pgfpathlineto{\pgfqpoint{4.985226in}{1.805678in}}%
\pgfpathlineto{\pgfqpoint{4.992824in}{1.818645in}}%
\pgfpathlineto{\pgfqpoint{5.000417in}{1.831594in}}%
\pgfpathlineto{\pgfqpoint{5.008006in}{1.844521in}}%
\pgfpathlineto{\pgfqpoint{5.015591in}{1.857426in}}%
\pgfpathlineto{\pgfqpoint{5.001632in}{1.853456in}}%
\pgfpathlineto{\pgfqpoint{4.987686in}{1.849598in}}%
\pgfpathlineto{\pgfqpoint{4.973753in}{1.845852in}}%
\pgfpathlineto{\pgfqpoint{4.959833in}{1.842217in}}%
\pgfpathlineto{\pgfqpoint{4.952248in}{1.829541in}}%
\pgfpathlineto{\pgfqpoint{4.944659in}{1.816846in}}%
\pgfpathlineto{\pgfqpoint{4.937065in}{1.804135in}}%
\pgfpathlineto{\pgfqpoint{4.929467in}{1.791409in}}%
\pgfpathclose%
\pgfusepath{fill}%
\end{pgfscope}%
\begin{pgfscope}%
\pgfpathrectangle{\pgfqpoint{1.254980in}{0.150000in}}{\pgfqpoint{5.490039in}{5.490039in}}%
\pgfusepath{clip}%
\pgfsetbuttcap%
\pgfsetroundjoin%
\definecolor{currentfill}{rgb}{0.280868,0.160771,0.472899}%
\pgfsetfillcolor{currentfill}%
\pgfsetfillopacity{0.700000}%
\pgfsetlinewidth{0.000000pt}%
\definecolor{currentstroke}{rgb}{0.000000,0.000000,0.000000}%
\pgfsetstrokecolor{currentstroke}%
\pgfsetdash{}{0pt}%
\pgfpathmoveto{\pgfqpoint{3.293138in}{1.712476in}}%
\pgfpathlineto{\pgfqpoint{3.306689in}{1.700410in}}%
\pgfpathlineto{\pgfqpoint{3.320241in}{1.688481in}}%
\pgfpathlineto{\pgfqpoint{3.333792in}{1.676687in}}%
\pgfpathlineto{\pgfqpoint{3.347344in}{1.665028in}}%
\pgfpathlineto{\pgfqpoint{3.355594in}{1.664941in}}%
\pgfpathlineto{\pgfqpoint{3.363832in}{1.665084in}}%
\pgfpathlineto{\pgfqpoint{3.372058in}{1.665452in}}%
\pgfpathlineto{\pgfqpoint{3.380272in}{1.666042in}}%
\pgfpathlineto{\pgfqpoint{3.366754in}{1.677234in}}%
\pgfpathlineto{\pgfqpoint{3.353235in}{1.688560in}}%
\pgfpathlineto{\pgfqpoint{3.339717in}{1.700021in}}%
\pgfpathlineto{\pgfqpoint{3.326199in}{1.711617in}}%
\pgfpathlineto{\pgfqpoint{3.317953in}{1.711488in}}%
\pgfpathlineto{\pgfqpoint{3.309694in}{1.711586in}}%
\pgfpathlineto{\pgfqpoint{3.301422in}{1.711914in}}%
\pgfpathlineto{\pgfqpoint{3.293138in}{1.712476in}}%
\pgfpathclose%
\pgfusepath{fill}%
\end{pgfscope}%
\begin{pgfscope}%
\pgfpathrectangle{\pgfqpoint{1.254980in}{0.150000in}}{\pgfqpoint{5.490039in}{5.490039in}}%
\pgfusepath{clip}%
\pgfsetbuttcap%
\pgfsetroundjoin%
\definecolor{currentfill}{rgb}{0.183898,0.422383,0.556944}%
\pgfsetfillcolor{currentfill}%
\pgfsetfillopacity{0.700000}%
\pgfsetlinewidth{0.000000pt}%
\definecolor{currentstroke}{rgb}{0.000000,0.000000,0.000000}%
\pgfsetstrokecolor{currentstroke}%
\pgfsetdash{}{0pt}%
\pgfpathmoveto{\pgfqpoint{5.421345in}{2.252477in}}%
\pgfpathlineto{\pgfqpoint{5.435548in}{2.259427in}}%
\pgfpathlineto{\pgfqpoint{5.449766in}{2.266489in}}%
\pgfpathlineto{\pgfqpoint{5.463999in}{2.273664in}}%
\pgfpathlineto{\pgfqpoint{5.471453in}{2.286160in}}%
\pgfpathlineto{\pgfqpoint{5.478900in}{2.298581in}}%
\pgfpathlineto{\pgfqpoint{5.486341in}{2.310927in}}%
\pgfpathlineto{\pgfqpoint{5.493775in}{2.323197in}}%
\pgfpathlineto{\pgfqpoint{5.479542in}{2.315900in}}%
\pgfpathlineto{\pgfqpoint{5.465324in}{2.308716in}}%
\pgfpathlineto{\pgfqpoint{5.451122in}{2.301644in}}%
\pgfpathlineto{\pgfqpoint{5.443687in}{2.289461in}}%
\pgfpathlineto{\pgfqpoint{5.436246in}{2.277205in}}%
\pgfpathlineto{\pgfqpoint{5.428799in}{2.264877in}}%
\pgfpathlineto{\pgfqpoint{5.421345in}{2.252477in}}%
\pgfpathclose%
\pgfusepath{fill}%
\end{pgfscope}%
\begin{pgfscope}%
\pgfpathrectangle{\pgfqpoint{1.254980in}{0.150000in}}{\pgfqpoint{5.490039in}{5.490039in}}%
\pgfusepath{clip}%
\pgfsetbuttcap%
\pgfsetroundjoin%
\definecolor{currentfill}{rgb}{0.141935,0.526453,0.555991}%
\pgfsetfillcolor{currentfill}%
\pgfsetfillopacity{0.700000}%
\pgfsetlinewidth{0.000000pt}%
\definecolor{currentstroke}{rgb}{0.000000,0.000000,0.000000}%
\pgfsetstrokecolor{currentstroke}%
\pgfsetdash{}{0pt}%
\pgfpathmoveto{\pgfqpoint{2.530098in}{2.626419in}}%
\pgfpathlineto{\pgfqpoint{2.543868in}{2.605569in}}%
\pgfpathlineto{\pgfqpoint{2.557630in}{2.584901in}}%
\pgfpathlineto{\pgfqpoint{2.571384in}{2.564414in}}%
\pgfpathlineto{\pgfqpoint{2.585130in}{2.544108in}}%
\pgfpathlineto{\pgfqpoint{2.593982in}{2.537147in}}%
\pgfpathlineto{\pgfqpoint{2.602813in}{2.530487in}}%
\pgfpathlineto{\pgfqpoint{2.611624in}{2.524121in}}%
\pgfpathlineto{\pgfqpoint{2.620415in}{2.518047in}}%
\pgfpathlineto{\pgfqpoint{2.606721in}{2.537848in}}%
\pgfpathlineto{\pgfqpoint{2.593020in}{2.557827in}}%
\pgfpathlineto{\pgfqpoint{2.579311in}{2.577987in}}%
\pgfpathlineto{\pgfqpoint{2.565594in}{2.598328in}}%
\pgfpathlineto{\pgfqpoint{2.556751in}{2.604902in}}%
\pgfpathlineto{\pgfqpoint{2.547887in}{2.611772in}}%
\pgfpathlineto{\pgfqpoint{2.539003in}{2.618943in}}%
\pgfpathlineto{\pgfqpoint{2.530098in}{2.626419in}}%
\pgfpathclose%
\pgfusepath{fill}%
\end{pgfscope}%
\begin{pgfscope}%
\pgfpathrectangle{\pgfqpoint{1.254980in}{0.150000in}}{\pgfqpoint{5.490039in}{5.490039in}}%
\pgfusepath{clip}%
\pgfsetbuttcap%
\pgfsetroundjoin%
\definecolor{currentfill}{rgb}{0.281924,0.089666,0.412415}%
\pgfsetfillcolor{currentfill}%
\pgfsetfillopacity{0.700000}%
\pgfsetlinewidth{0.000000pt}%
\definecolor{currentstroke}{rgb}{0.000000,0.000000,0.000000}%
\pgfsetstrokecolor{currentstroke}%
\pgfsetdash{}{0pt}%
\pgfpathmoveto{\pgfqpoint{4.555296in}{1.520097in}}%
\pgfpathlineto{\pgfqpoint{4.569049in}{1.520300in}}%
\pgfpathlineto{\pgfqpoint{4.582813in}{1.520615in}}%
\pgfpathlineto{\pgfqpoint{4.596587in}{1.521043in}}%
\pgfpathlineto{\pgfqpoint{4.610370in}{1.521582in}}%
\pgfpathlineto{\pgfqpoint{4.618061in}{1.533227in}}%
\pgfpathlineto{\pgfqpoint{4.625747in}{1.544912in}}%
\pgfpathlineto{\pgfqpoint{4.633428in}{1.556633in}}%
\pgfpathlineto{\pgfqpoint{4.641106in}{1.568389in}}%
\pgfpathlineto{\pgfqpoint{4.627326in}{1.567537in}}%
\pgfpathlineto{\pgfqpoint{4.613556in}{1.566797in}}%
\pgfpathlineto{\pgfqpoint{4.599797in}{1.566169in}}%
\pgfpathlineto{\pgfqpoint{4.586048in}{1.565654in}}%
\pgfpathlineto{\pgfqpoint{4.578366in}{1.554204in}}%
\pgfpathlineto{\pgfqpoint{4.570680in}{1.542793in}}%
\pgfpathlineto{\pgfqpoint{4.562990in}{1.531423in}}%
\pgfpathlineto{\pgfqpoint{4.555296in}{1.520097in}}%
\pgfpathclose%
\pgfusepath{fill}%
\end{pgfscope}%
\begin{pgfscope}%
\pgfpathrectangle{\pgfqpoint{1.254980in}{0.150000in}}{\pgfqpoint{5.490039in}{5.490039in}}%
\pgfusepath{clip}%
\pgfsetbuttcap%
\pgfsetroundjoin%
\definecolor{currentfill}{rgb}{0.278791,0.062145,0.386592}%
\pgfsetfillcolor{currentfill}%
\pgfsetfillopacity{0.700000}%
\pgfsetlinewidth{0.000000pt}%
\definecolor{currentstroke}{rgb}{0.000000,0.000000,0.000000}%
\pgfsetstrokecolor{currentstroke}%
\pgfsetdash{}{0pt}%
\pgfpathmoveto{\pgfqpoint{4.469534in}{1.476894in}}%
\pgfpathlineto{\pgfqpoint{4.483254in}{1.476320in}}%
\pgfpathlineto{\pgfqpoint{4.496985in}{1.475859in}}%
\pgfpathlineto{\pgfqpoint{4.510724in}{1.475510in}}%
\pgfpathlineto{\pgfqpoint{4.524473in}{1.475273in}}%
\pgfpathlineto{\pgfqpoint{4.532186in}{1.486401in}}%
\pgfpathlineto{\pgfqpoint{4.539893in}{1.497583in}}%
\pgfpathlineto{\pgfqpoint{4.547597in}{1.508816in}}%
\pgfpathlineto{\pgfqpoint{4.555296in}{1.520097in}}%
\pgfpathlineto{\pgfqpoint{4.541552in}{1.520006in}}%
\pgfpathlineto{\pgfqpoint{4.527817in}{1.520027in}}%
\pgfpathlineto{\pgfqpoint{4.514093in}{1.520161in}}%
\pgfpathlineto{\pgfqpoint{4.500378in}{1.520408in}}%
\pgfpathlineto{\pgfqpoint{4.492673in}{1.509448in}}%
\pgfpathlineto{\pgfqpoint{4.484965in}{1.498541in}}%
\pgfpathlineto{\pgfqpoint{4.477251in}{1.487689in}}%
\pgfpathlineto{\pgfqpoint{4.469534in}{1.476894in}}%
\pgfpathclose%
\pgfusepath{fill}%
\end{pgfscope}%
\begin{pgfscope}%
\pgfpathrectangle{\pgfqpoint{1.254980in}{0.150000in}}{\pgfqpoint{5.490039in}{5.490039in}}%
\pgfusepath{clip}%
\pgfsetbuttcap%
\pgfsetroundjoin%
\definecolor{currentfill}{rgb}{0.267004,0.004874,0.329415}%
\pgfsetfillcolor{currentfill}%
\pgfsetfillopacity{0.700000}%
\pgfsetlinewidth{0.000000pt}%
\definecolor{currentstroke}{rgb}{0.000000,0.000000,0.000000}%
\pgfsetstrokecolor{currentstroke}%
\pgfsetdash{}{0pt}%
\pgfpathmoveto{\pgfqpoint{3.931758in}{1.392278in}}%
\pgfpathlineto{\pgfqpoint{3.945334in}{1.386606in}}%
\pgfpathlineto{\pgfqpoint{3.958916in}{1.381053in}}%
\pgfpathlineto{\pgfqpoint{3.972502in}{1.375618in}}%
\pgfpathlineto{\pgfqpoint{3.986095in}{1.370300in}}%
\pgfpathlineto{\pgfqpoint{3.993988in}{1.376868in}}%
\pgfpathlineto{\pgfqpoint{4.001874in}{1.383578in}}%
\pgfpathlineto{\pgfqpoint{4.009754in}{1.390425in}}%
\pgfpathlineto{\pgfqpoint{4.017627in}{1.397408in}}%
\pgfpathlineto{\pgfqpoint{4.004051in}{1.402318in}}%
\pgfpathlineto{\pgfqpoint{3.990480in}{1.407347in}}%
\pgfpathlineto{\pgfqpoint{3.976916in}{1.412493in}}%
\pgfpathlineto{\pgfqpoint{3.963356in}{1.417758in}}%
\pgfpathlineto{\pgfqpoint{3.955467in}{1.411177in}}%
\pgfpathlineto{\pgfqpoint{3.947571in}{1.404734in}}%
\pgfpathlineto{\pgfqpoint{3.939668in}{1.398433in}}%
\pgfpathlineto{\pgfqpoint{3.931758in}{1.392278in}}%
\pgfpathclose%
\pgfusepath{fill}%
\end{pgfscope}%
\begin{pgfscope}%
\pgfpathrectangle{\pgfqpoint{1.254980in}{0.150000in}}{\pgfqpoint{5.490039in}{5.490039in}}%
\pgfusepath{clip}%
\pgfsetbuttcap%
\pgfsetroundjoin%
\definecolor{currentfill}{rgb}{0.278791,0.062145,0.386592}%
\pgfsetfillcolor{currentfill}%
\pgfsetfillopacity{0.700000}%
\pgfsetlinewidth{0.000000pt}%
\definecolor{currentstroke}{rgb}{0.000000,0.000000,0.000000}%
\pgfsetstrokecolor{currentstroke}%
\pgfsetdash{}{0pt}%
\pgfpathmoveto{\pgfqpoint{3.596731in}{1.504854in}}%
\pgfpathlineto{\pgfqpoint{3.610275in}{1.495873in}}%
\pgfpathlineto{\pgfqpoint{3.623821in}{1.487018in}}%
\pgfpathlineto{\pgfqpoint{3.637371in}{1.478288in}}%
\pgfpathlineto{\pgfqpoint{3.650923in}{1.469683in}}%
\pgfpathlineto{\pgfqpoint{3.658984in}{1.472753in}}%
\pgfpathlineto{\pgfqpoint{3.667036in}{1.476015in}}%
\pgfpathlineto{\pgfqpoint{3.675078in}{1.479464in}}%
\pgfpathlineto{\pgfqpoint{3.683112in}{1.483098in}}%
\pgfpathlineto{\pgfqpoint{3.669585in}{1.491259in}}%
\pgfpathlineto{\pgfqpoint{3.656060in}{1.499546in}}%
\pgfpathlineto{\pgfqpoint{3.642539in}{1.507957in}}%
\pgfpathlineto{\pgfqpoint{3.629020in}{1.516495in}}%
\pgfpathlineto{\pgfqpoint{3.620962in}{1.513298in}}%
\pgfpathlineto{\pgfqpoint{3.612895in}{1.510290in}}%
\pgfpathlineto{\pgfqpoint{3.604818in}{1.507474in}}%
\pgfpathlineto{\pgfqpoint{3.596731in}{1.504854in}}%
\pgfpathclose%
\pgfusepath{fill}%
\end{pgfscope}%
\begin{pgfscope}%
\pgfpathrectangle{\pgfqpoint{1.254980in}{0.150000in}}{\pgfqpoint{5.490039in}{5.490039in}}%
\pgfusepath{clip}%
\pgfsetbuttcap%
\pgfsetroundjoin%
\definecolor{currentfill}{rgb}{0.220057,0.343307,0.549413}%
\pgfsetfillcolor{currentfill}%
\pgfsetfillopacity{0.700000}%
\pgfsetlinewidth{0.000000pt}%
\definecolor{currentstroke}{rgb}{0.000000,0.000000,0.000000}%
\pgfsetstrokecolor{currentstroke}%
\pgfsetdash{}{0pt}%
\pgfpathmoveto{\pgfqpoint{5.218382in}{2.050619in}}%
\pgfpathlineto{\pgfqpoint{5.232464in}{2.056206in}}%
\pgfpathlineto{\pgfqpoint{5.246561in}{2.061905in}}%
\pgfpathlineto{\pgfqpoint{5.260672in}{2.067715in}}%
\pgfpathlineto{\pgfqpoint{5.274797in}{2.073638in}}%
\pgfpathlineto{\pgfqpoint{5.282319in}{2.086642in}}%
\pgfpathlineto{\pgfqpoint{5.289835in}{2.099591in}}%
\pgfpathlineto{\pgfqpoint{5.297345in}{2.112484in}}%
\pgfpathlineto{\pgfqpoint{5.304851in}{2.125319in}}%
\pgfpathlineto{\pgfqpoint{5.290724in}{2.119225in}}%
\pgfpathlineto{\pgfqpoint{5.276613in}{2.113243in}}%
\pgfpathlineto{\pgfqpoint{5.262516in}{2.107373in}}%
\pgfpathlineto{\pgfqpoint{5.248434in}{2.101615in}}%
\pgfpathlineto{\pgfqpoint{5.240929in}{2.088945in}}%
\pgfpathlineto{\pgfqpoint{5.233419in}{2.076222in}}%
\pgfpathlineto{\pgfqpoint{5.225903in}{2.063446in}}%
\pgfpathlineto{\pgfqpoint{5.218382in}{2.050619in}}%
\pgfpathclose%
\pgfusepath{fill}%
\end{pgfscope}%
\begin{pgfscope}%
\pgfpathrectangle{\pgfqpoint{1.254980in}{0.150000in}}{\pgfqpoint{5.490039in}{5.490039in}}%
\pgfusepath{clip}%
\pgfsetbuttcap%
\pgfsetroundjoin%
\definecolor{currentfill}{rgb}{0.283197,0.115680,0.436115}%
\pgfsetfillcolor{currentfill}%
\pgfsetfillopacity{0.700000}%
\pgfsetlinewidth{0.000000pt}%
\definecolor{currentstroke}{rgb}{0.000000,0.000000,0.000000}%
\pgfsetstrokecolor{currentstroke}%
\pgfsetdash{}{0pt}%
\pgfpathmoveto{\pgfqpoint{4.641106in}{1.568389in}}%
\pgfpathlineto{\pgfqpoint{4.654896in}{1.569352in}}%
\pgfpathlineto{\pgfqpoint{4.668697in}{1.570428in}}%
\pgfpathlineto{\pgfqpoint{4.682509in}{1.571616in}}%
\pgfpathlineto{\pgfqpoint{4.696331in}{1.572915in}}%
\pgfpathlineto{\pgfqpoint{4.704001in}{1.585006in}}%
\pgfpathlineto{\pgfqpoint{4.711667in}{1.597122in}}%
\pgfpathlineto{\pgfqpoint{4.719329in}{1.609263in}}%
\pgfpathlineto{\pgfqpoint{4.726986in}{1.621424in}}%
\pgfpathlineto{\pgfqpoint{4.713166in}{1.619827in}}%
\pgfpathlineto{\pgfqpoint{4.699358in}{1.618343in}}%
\pgfpathlineto{\pgfqpoint{4.685560in}{1.616970in}}%
\pgfpathlineto{\pgfqpoint{4.671773in}{1.615709in}}%
\pgfpathlineto{\pgfqpoint{4.664112in}{1.603839in}}%
\pgfpathlineto{\pgfqpoint{4.656448in}{1.591994in}}%
\pgfpathlineto{\pgfqpoint{4.648779in}{1.580176in}}%
\pgfpathlineto{\pgfqpoint{4.641106in}{1.568389in}}%
\pgfpathclose%
\pgfusepath{fill}%
\end{pgfscope}%
\begin{pgfscope}%
\pgfpathrectangle{\pgfqpoint{1.254980in}{0.150000in}}{\pgfqpoint{5.490039in}{5.490039in}}%
\pgfusepath{clip}%
\pgfsetbuttcap%
\pgfsetroundjoin%
\definecolor{currentfill}{rgb}{0.276022,0.044167,0.370164}%
\pgfsetfillcolor{currentfill}%
\pgfsetfillopacity{0.700000}%
\pgfsetlinewidth{0.000000pt}%
\definecolor{currentstroke}{rgb}{0.000000,0.000000,0.000000}%
\pgfsetstrokecolor{currentstroke}%
\pgfsetdash{}{0pt}%
\pgfpathmoveto{\pgfqpoint{4.383796in}{1.439140in}}%
\pgfpathlineto{\pgfqpoint{4.397488in}{1.437772in}}%
\pgfpathlineto{\pgfqpoint{4.411189in}{1.436516in}}%
\pgfpathlineto{\pgfqpoint{4.424898in}{1.435373in}}%
\pgfpathlineto{\pgfqpoint{4.438617in}{1.434343in}}%
\pgfpathlineto{\pgfqpoint{4.446353in}{1.444882in}}%
\pgfpathlineto{\pgfqpoint{4.454084in}{1.455488in}}%
\pgfpathlineto{\pgfqpoint{4.461811in}{1.466160in}}%
\pgfpathlineto{\pgfqpoint{4.469534in}{1.476894in}}%
\pgfpathlineto{\pgfqpoint{4.455822in}{1.477581in}}%
\pgfpathlineto{\pgfqpoint{4.442119in}{1.478381in}}%
\pgfpathlineto{\pgfqpoint{4.428425in}{1.479294in}}%
\pgfpathlineto{\pgfqpoint{4.414740in}{1.480320in}}%
\pgfpathlineto{\pgfqpoint{4.407011in}{1.469922in}}%
\pgfpathlineto{\pgfqpoint{4.399278in}{1.459591in}}%
\pgfpathlineto{\pgfqpoint{4.391539in}{1.449330in}}%
\pgfpathlineto{\pgfqpoint{4.383796in}{1.439140in}}%
\pgfpathclose%
\pgfusepath{fill}%
\end{pgfscope}%
\begin{pgfscope}%
\pgfpathrectangle{\pgfqpoint{1.254980in}{0.150000in}}{\pgfqpoint{5.490039in}{5.490039in}}%
\pgfusepath{clip}%
\pgfsetbuttcap%
\pgfsetroundjoin%
\definecolor{currentfill}{rgb}{0.271305,0.019942,0.347269}%
\pgfsetfillcolor{currentfill}%
\pgfsetfillopacity{0.700000}%
\pgfsetlinewidth{0.000000pt}%
\definecolor{currentstroke}{rgb}{0.000000,0.000000,0.000000}%
\pgfsetstrokecolor{currentstroke}%
\pgfsetdash{}{0pt}%
\pgfpathmoveto{\pgfqpoint{3.791451in}{1.422253in}}%
\pgfpathlineto{\pgfqpoint{3.805010in}{1.415198in}}%
\pgfpathlineto{\pgfqpoint{3.818574in}{1.408264in}}%
\pgfpathlineto{\pgfqpoint{3.832141in}{1.401450in}}%
\pgfpathlineto{\pgfqpoint{3.845713in}{1.394758in}}%
\pgfpathlineto{\pgfqpoint{3.853672in}{1.399865in}}%
\pgfpathlineto{\pgfqpoint{3.861623in}{1.405136in}}%
\pgfpathlineto{\pgfqpoint{3.869566in}{1.410568in}}%
\pgfpathlineto{\pgfqpoint{3.877502in}{1.416156in}}%
\pgfpathlineto{\pgfqpoint{3.863950in}{1.422425in}}%
\pgfpathlineto{\pgfqpoint{3.850403in}{1.428814in}}%
\pgfpathlineto{\pgfqpoint{3.836860in}{1.435324in}}%
\pgfpathlineto{\pgfqpoint{3.823321in}{1.441955in}}%
\pgfpathlineto{\pgfqpoint{3.815366in}{1.436784in}}%
\pgfpathlineto{\pgfqpoint{3.807402in}{1.431775in}}%
\pgfpathlineto{\pgfqpoint{3.799431in}{1.426930in}}%
\pgfpathlineto{\pgfqpoint{3.791451in}{1.422253in}}%
\pgfpathclose%
\pgfusepath{fill}%
\end{pgfscope}%
\begin{pgfscope}%
\pgfpathrectangle{\pgfqpoint{1.254980in}{0.150000in}}{\pgfqpoint{5.490039in}{5.490039in}}%
\pgfusepath{clip}%
\pgfsetbuttcap%
\pgfsetroundjoin%
\definecolor{currentfill}{rgb}{0.255645,0.260703,0.528312}%
\pgfsetfillcolor{currentfill}%
\pgfsetfillopacity{0.700000}%
\pgfsetlinewidth{0.000000pt}%
\definecolor{currentstroke}{rgb}{0.000000,0.000000,0.000000}%
\pgfsetstrokecolor{currentstroke}%
\pgfsetdash{}{0pt}%
\pgfpathmoveto{\pgfqpoint{5.015591in}{1.857426in}}%
\pgfpathlineto{\pgfqpoint{5.029562in}{1.861508in}}%
\pgfpathlineto{\pgfqpoint{5.043547in}{1.865701in}}%
\pgfpathlineto{\pgfqpoint{5.057545in}{1.870006in}}%
\pgfpathlineto{\pgfqpoint{5.071556in}{1.874423in}}%
\pgfpathlineto{\pgfqpoint{5.079136in}{1.887529in}}%
\pgfpathlineto{\pgfqpoint{5.086711in}{1.900606in}}%
\pgfpathlineto{\pgfqpoint{5.094281in}{1.913651in}}%
\pgfpathlineto{\pgfqpoint{5.101847in}{1.926663in}}%
\pgfpathlineto{\pgfqpoint{5.087835in}{1.922027in}}%
\pgfpathlineto{\pgfqpoint{5.073837in}{1.917503in}}%
\pgfpathlineto{\pgfqpoint{5.059852in}{1.913090in}}%
\pgfpathlineto{\pgfqpoint{5.045881in}{1.908789in}}%
\pgfpathlineto{\pgfqpoint{5.038315in}{1.895991in}}%
\pgfpathlineto{\pgfqpoint{5.030745in}{1.883163in}}%
\pgfpathlineto{\pgfqpoint{5.023170in}{1.870307in}}%
\pgfpathlineto{\pgfqpoint{5.015591in}{1.857426in}}%
\pgfpathclose%
\pgfusepath{fill}%
\end{pgfscope}%
\begin{pgfscope}%
\pgfpathrectangle{\pgfqpoint{1.254980in}{0.150000in}}{\pgfqpoint{5.490039in}{5.490039in}}%
\pgfusepath{clip}%
\pgfsetbuttcap%
\pgfsetroundjoin%
\definecolor{currentfill}{rgb}{0.267004,0.004874,0.329415}%
\pgfsetfillcolor{currentfill}%
\pgfsetfillopacity{0.700000}%
\pgfsetlinewidth{0.000000pt}%
\definecolor{currentstroke}{rgb}{0.000000,0.000000,0.000000}%
\pgfsetstrokecolor{currentstroke}%
\pgfsetdash{}{0pt}%
\pgfpathmoveto{\pgfqpoint{4.071989in}{1.378938in}}%
\pgfpathlineto{\pgfqpoint{4.085595in}{1.374612in}}%
\pgfpathlineto{\pgfqpoint{4.099208in}{1.370403in}}%
\pgfpathlineto{\pgfqpoint{4.112826in}{1.366310in}}%
\pgfpathlineto{\pgfqpoint{4.126451in}{1.362332in}}%
\pgfpathlineto{\pgfqpoint{4.134289in}{1.370239in}}%
\pgfpathlineto{\pgfqpoint{4.142121in}{1.378265in}}%
\pgfpathlineto{\pgfqpoint{4.149947in}{1.386409in}}%
\pgfpathlineto{\pgfqpoint{4.157767in}{1.394666in}}%
\pgfpathlineto{\pgfqpoint{4.144155in}{1.398254in}}%
\pgfpathlineto{\pgfqpoint{4.130550in}{1.401957in}}%
\pgfpathlineto{\pgfqpoint{4.116951in}{1.405776in}}%
\pgfpathlineto{\pgfqpoint{4.103359in}{1.409711in}}%
\pgfpathlineto{\pgfqpoint{4.095526in}{1.401838in}}%
\pgfpathlineto{\pgfqpoint{4.087687in}{1.394083in}}%
\pgfpathlineto{\pgfqpoint{4.079841in}{1.386448in}}%
\pgfpathlineto{\pgfqpoint{4.071989in}{1.378938in}}%
\pgfpathclose%
\pgfusepath{fill}%
\end{pgfscope}%
\begin{pgfscope}%
\pgfpathrectangle{\pgfqpoint{1.254980in}{0.150000in}}{\pgfqpoint{5.490039in}{5.490039in}}%
\pgfusepath{clip}%
\pgfsetbuttcap%
\pgfsetroundjoin%
\definecolor{currentfill}{rgb}{0.282623,0.140926,0.457517}%
\pgfsetfillcolor{currentfill}%
\pgfsetfillopacity{0.700000}%
\pgfsetlinewidth{0.000000pt}%
\definecolor{currentstroke}{rgb}{0.000000,0.000000,0.000000}%
\pgfsetstrokecolor{currentstroke}%
\pgfsetdash{}{0pt}%
\pgfpathmoveto{\pgfqpoint{3.347344in}{1.665028in}}%
\pgfpathlineto{\pgfqpoint{3.360895in}{1.653504in}}%
\pgfpathlineto{\pgfqpoint{3.374448in}{1.642113in}}%
\pgfpathlineto{\pgfqpoint{3.388000in}{1.630855in}}%
\pgfpathlineto{\pgfqpoint{3.401554in}{1.619731in}}%
\pgfpathlineto{\pgfqpoint{3.409771in}{1.620117in}}%
\pgfpathlineto{\pgfqpoint{3.417977in}{1.620728in}}%
\pgfpathlineto{\pgfqpoint{3.426172in}{1.621561in}}%
\pgfpathlineto{\pgfqpoint{3.434355in}{1.622610in}}%
\pgfpathlineto{\pgfqpoint{3.420833in}{1.633269in}}%
\pgfpathlineto{\pgfqpoint{3.407312in}{1.644060in}}%
\pgfpathlineto{\pgfqpoint{3.393792in}{1.654984in}}%
\pgfpathlineto{\pgfqpoint{3.380272in}{1.666042in}}%
\pgfpathlineto{\pgfqpoint{3.372058in}{1.665452in}}%
\pgfpathlineto{\pgfqpoint{3.363832in}{1.665084in}}%
\pgfpathlineto{\pgfqpoint{3.355594in}{1.664941in}}%
\pgfpathlineto{\pgfqpoint{3.347344in}{1.665028in}}%
\pgfpathclose%
\pgfusepath{fill}%
\end{pgfscope}%
\begin{pgfscope}%
\pgfpathrectangle{\pgfqpoint{1.254980in}{0.150000in}}{\pgfqpoint{5.490039in}{5.490039in}}%
\pgfusepath{clip}%
\pgfsetbuttcap%
\pgfsetroundjoin%
\definecolor{currentfill}{rgb}{0.282290,0.145912,0.461510}%
\pgfsetfillcolor{currentfill}%
\pgfsetfillopacity{0.700000}%
\pgfsetlinewidth{0.000000pt}%
\definecolor{currentstroke}{rgb}{0.000000,0.000000,0.000000}%
\pgfsetstrokecolor{currentstroke}%
\pgfsetdash{}{0pt}%
\pgfpathmoveto{\pgfqpoint{4.726986in}{1.621424in}}%
\pgfpathlineto{\pgfqpoint{4.740817in}{1.623132in}}%
\pgfpathlineto{\pgfqpoint{4.754658in}{1.624952in}}%
\pgfpathlineto{\pgfqpoint{4.768511in}{1.626883in}}%
\pgfpathlineto{\pgfqpoint{4.782376in}{1.628926in}}%
\pgfpathlineto{\pgfqpoint{4.790027in}{1.641394in}}%
\pgfpathlineto{\pgfqpoint{4.797674in}{1.653875in}}%
\pgfpathlineto{\pgfqpoint{4.805317in}{1.666367in}}%
\pgfpathlineto{\pgfqpoint{4.812955in}{1.678866in}}%
\pgfpathlineto{\pgfqpoint{4.799092in}{1.676541in}}%
\pgfpathlineto{\pgfqpoint{4.785241in}{1.674328in}}%
\pgfpathlineto{\pgfqpoint{4.771401in}{1.672227in}}%
\pgfpathlineto{\pgfqpoint{4.757573in}{1.670237in}}%
\pgfpathlineto{\pgfqpoint{4.749933in}{1.658013in}}%
\pgfpathlineto{\pgfqpoint{4.742288in}{1.645801in}}%
\pgfpathlineto{\pgfqpoint{4.734639in}{1.633604in}}%
\pgfpathlineto{\pgfqpoint{4.726986in}{1.621424in}}%
\pgfpathclose%
\pgfusepath{fill}%
\end{pgfscope}%
\begin{pgfscope}%
\pgfpathrectangle{\pgfqpoint{1.254980in}{0.150000in}}{\pgfqpoint{5.490039in}{5.490039in}}%
\pgfusepath{clip}%
\pgfsetbuttcap%
\pgfsetroundjoin%
\definecolor{currentfill}{rgb}{0.196571,0.711827,0.479221}%
\pgfsetfillcolor{currentfill}%
\pgfsetfillopacity{0.700000}%
\pgfsetlinewidth{0.000000pt}%
\definecolor{currentstroke}{rgb}{0.000000,0.000000,0.000000}%
\pgfsetstrokecolor{currentstroke}%
\pgfsetdash{}{0pt}%
\pgfpathmoveto{\pgfqpoint{2.233628in}{3.144134in}}%
\pgfpathlineto{\pgfqpoint{2.247583in}{3.119060in}}%
\pgfpathlineto{\pgfqpoint{2.261527in}{3.094199in}}%
\pgfpathlineto{\pgfqpoint{2.275458in}{3.069551in}}%
\pgfpathlineto{\pgfqpoint{2.289378in}{3.045112in}}%
\pgfpathlineto{\pgfqpoint{2.298482in}{3.036297in}}%
\pgfpathlineto{\pgfqpoint{2.307563in}{3.027797in}}%
\pgfpathlineto{\pgfqpoint{2.316622in}{3.019608in}}%
\pgfpathlineto{\pgfqpoint{2.325658in}{3.011726in}}%
\pgfpathlineto{\pgfqpoint{2.311798in}{3.035654in}}%
\pgfpathlineto{\pgfqpoint{2.297926in}{3.059790in}}%
\pgfpathlineto{\pgfqpoint{2.284043in}{3.084137in}}%
\pgfpathlineto{\pgfqpoint{2.270148in}{3.108696in}}%
\pgfpathlineto{\pgfqpoint{2.261053in}{3.117083in}}%
\pgfpathlineto{\pgfqpoint{2.251935in}{3.125781in}}%
\pgfpathlineto{\pgfqpoint{2.242793in}{3.134797in}}%
\pgfpathlineto{\pgfqpoint{2.233628in}{3.144134in}}%
\pgfpathclose%
\pgfusepath{fill}%
\end{pgfscope}%
\begin{pgfscope}%
\pgfpathrectangle{\pgfqpoint{1.254980in}{0.150000in}}{\pgfqpoint{5.490039in}{5.490039in}}%
\pgfusepath{clip}%
\pgfsetbuttcap%
\pgfsetroundjoin%
\definecolor{currentfill}{rgb}{0.129933,0.559582,0.551864}%
\pgfsetfillcolor{currentfill}%
\pgfsetfillopacity{0.700000}%
\pgfsetlinewidth{0.000000pt}%
\definecolor{currentstroke}{rgb}{0.000000,0.000000,0.000000}%
\pgfsetstrokecolor{currentstroke}%
\pgfsetdash{}{0pt}%
\pgfpathmoveto{\pgfqpoint{2.474933in}{2.711666in}}%
\pgfpathlineto{\pgfqpoint{2.488737in}{2.690075in}}%
\pgfpathlineto{\pgfqpoint{2.502532in}{2.668671in}}%
\pgfpathlineto{\pgfqpoint{2.516319in}{2.647452in}}%
\pgfpathlineto{\pgfqpoint{2.530098in}{2.626419in}}%
\pgfpathlineto{\pgfqpoint{2.539003in}{2.618943in}}%
\pgfpathlineto{\pgfqpoint{2.547887in}{2.611772in}}%
\pgfpathlineto{\pgfqpoint{2.556751in}{2.604902in}}%
\pgfpathlineto{\pgfqpoint{2.565594in}{2.598328in}}%
\pgfpathlineto{\pgfqpoint{2.551870in}{2.618852in}}%
\pgfpathlineto{\pgfqpoint{2.538137in}{2.639560in}}%
\pgfpathlineto{\pgfqpoint{2.524397in}{2.660452in}}%
\pgfpathlineto{\pgfqpoint{2.510648in}{2.681530in}}%
\pgfpathlineto{\pgfqpoint{2.501751in}{2.688607in}}%
\pgfpathlineto{\pgfqpoint{2.492833in}{2.695985in}}%
\pgfpathlineto{\pgfqpoint{2.483894in}{2.703670in}}%
\pgfpathlineto{\pgfqpoint{2.474933in}{2.711666in}}%
\pgfpathclose%
\pgfusepath{fill}%
\end{pgfscope}%
\begin{pgfscope}%
\pgfpathrectangle{\pgfqpoint{1.254980in}{0.150000in}}{\pgfqpoint{5.490039in}{5.490039in}}%
\pgfusepath{clip}%
\pgfsetbuttcap%
\pgfsetroundjoin%
\definecolor{currentfill}{rgb}{0.272594,0.025563,0.353093}%
\pgfsetfillcolor{currentfill}%
\pgfsetfillopacity{0.700000}%
\pgfsetlinewidth{0.000000pt}%
\definecolor{currentstroke}{rgb}{0.000000,0.000000,0.000000}%
\pgfsetstrokecolor{currentstroke}%
\pgfsetdash{}{0pt}%
\pgfpathmoveto{\pgfqpoint{4.298056in}{1.407204in}}%
\pgfpathlineto{\pgfqpoint{4.311723in}{1.405023in}}%
\pgfpathlineto{\pgfqpoint{4.325399in}{1.402956in}}%
\pgfpathlineto{\pgfqpoint{4.339083in}{1.401002in}}%
\pgfpathlineto{\pgfqpoint{4.352774in}{1.399161in}}%
\pgfpathlineto{\pgfqpoint{4.360537in}{1.409034in}}%
\pgfpathlineto{\pgfqpoint{4.368295in}{1.418990in}}%
\pgfpathlineto{\pgfqpoint{4.376048in}{1.429026in}}%
\pgfpathlineto{\pgfqpoint{4.383796in}{1.439140in}}%
\pgfpathlineto{\pgfqpoint{4.370112in}{1.440622in}}%
\pgfpathlineto{\pgfqpoint{4.356437in}{1.442218in}}%
\pgfpathlineto{\pgfqpoint{4.342770in}{1.443927in}}%
\pgfpathlineto{\pgfqpoint{4.329112in}{1.445750in}}%
\pgfpathlineto{\pgfqpoint{4.321355in}{1.435988in}}%
\pgfpathlineto{\pgfqpoint{4.313594in}{1.426308in}}%
\pgfpathlineto{\pgfqpoint{4.305827in}{1.416712in}}%
\pgfpathlineto{\pgfqpoint{4.298056in}{1.407204in}}%
\pgfpathclose%
\pgfusepath{fill}%
\end{pgfscope}%
\begin{pgfscope}%
\pgfpathrectangle{\pgfqpoint{1.254980in}{0.150000in}}{\pgfqpoint{5.490039in}{5.490039in}}%
\pgfusepath{clip}%
\pgfsetbuttcap%
\pgfsetroundjoin%
\definecolor{currentfill}{rgb}{0.278826,0.175490,0.483397}%
\pgfsetfillcolor{currentfill}%
\pgfsetfillopacity{0.700000}%
\pgfsetlinewidth{0.000000pt}%
\definecolor{currentstroke}{rgb}{0.000000,0.000000,0.000000}%
\pgfsetstrokecolor{currentstroke}%
\pgfsetdash{}{0pt}%
\pgfpathmoveto{\pgfqpoint{4.812955in}{1.678866in}}%
\pgfpathlineto{\pgfqpoint{4.826829in}{1.681303in}}%
\pgfpathlineto{\pgfqpoint{4.840716in}{1.683851in}}%
\pgfpathlineto{\pgfqpoint{4.854614in}{1.686510in}}%
\pgfpathlineto{\pgfqpoint{4.868523in}{1.689281in}}%
\pgfpathlineto{\pgfqpoint{4.876157in}{1.702060in}}%
\pgfpathlineto{\pgfqpoint{4.883785in}{1.714839in}}%
\pgfpathlineto{\pgfqpoint{4.891410in}{1.727616in}}%
\pgfpathlineto{\pgfqpoint{4.899030in}{1.740390in}}%
\pgfpathlineto{\pgfqpoint{4.885121in}{1.737352in}}%
\pgfpathlineto{\pgfqpoint{4.871224in}{1.734426in}}%
\pgfpathlineto{\pgfqpoint{4.857339in}{1.731612in}}%
\pgfpathlineto{\pgfqpoint{4.843466in}{1.728909in}}%
\pgfpathlineto{\pgfqpoint{4.835845in}{1.716396in}}%
\pgfpathlineto{\pgfqpoint{4.828219in}{1.703883in}}%
\pgfpathlineto{\pgfqpoint{4.820589in}{1.691373in}}%
\pgfpathlineto{\pgfqpoint{4.812955in}{1.678866in}}%
\pgfpathclose%
\pgfusepath{fill}%
\end{pgfscope}%
\begin{pgfscope}%
\pgfpathrectangle{\pgfqpoint{1.254980in}{0.150000in}}{\pgfqpoint{5.490039in}{5.490039in}}%
\pgfusepath{clip}%
\pgfsetbuttcap%
\pgfsetroundjoin%
\definecolor{currentfill}{rgb}{0.283187,0.125848,0.444960}%
\pgfsetfillcolor{currentfill}%
\pgfsetfillopacity{0.700000}%
\pgfsetlinewidth{0.000000pt}%
\definecolor{currentstroke}{rgb}{0.000000,0.000000,0.000000}%
\pgfsetstrokecolor{currentstroke}%
\pgfsetdash{}{0pt}%
\pgfpathmoveto{\pgfqpoint{3.401554in}{1.619731in}}%
\pgfpathlineto{\pgfqpoint{3.415108in}{1.608738in}}%
\pgfpathlineto{\pgfqpoint{3.428662in}{1.597878in}}%
\pgfpathlineto{\pgfqpoint{3.442218in}{1.587148in}}%
\pgfpathlineto{\pgfqpoint{3.455775in}{1.576550in}}%
\pgfpathlineto{\pgfqpoint{3.463961in}{1.577408in}}%
\pgfpathlineto{\pgfqpoint{3.472137in}{1.578486in}}%
\pgfpathlineto{\pgfqpoint{3.480301in}{1.579781in}}%
\pgfpathlineto{\pgfqpoint{3.488454in}{1.581289in}}%
\pgfpathlineto{\pgfqpoint{3.474927in}{1.591423in}}%
\pgfpathlineto{\pgfqpoint{3.461402in}{1.601688in}}%
\pgfpathlineto{\pgfqpoint{3.447878in}{1.612083in}}%
\pgfpathlineto{\pgfqpoint{3.434355in}{1.622610in}}%
\pgfpathlineto{\pgfqpoint{3.426172in}{1.621561in}}%
\pgfpathlineto{\pgfqpoint{3.417977in}{1.620728in}}%
\pgfpathlineto{\pgfqpoint{3.409771in}{1.620117in}}%
\pgfpathlineto{\pgfqpoint{3.401554in}{1.619731in}}%
\pgfpathclose%
\pgfusepath{fill}%
\end{pgfscope}%
\begin{pgfscope}%
\pgfpathrectangle{\pgfqpoint{1.254980in}{0.150000in}}{\pgfqpoint{5.490039in}{5.490039in}}%
\pgfusepath{clip}%
\pgfsetbuttcap%
\pgfsetroundjoin%
\definecolor{currentfill}{rgb}{0.277018,0.050344,0.375715}%
\pgfsetfillcolor{currentfill}%
\pgfsetfillopacity{0.700000}%
\pgfsetlinewidth{0.000000pt}%
\definecolor{currentstroke}{rgb}{0.000000,0.000000,0.000000}%
\pgfsetstrokecolor{currentstroke}%
\pgfsetdash{}{0pt}%
\pgfpathmoveto{\pgfqpoint{3.650923in}{1.469683in}}%
\pgfpathlineto{\pgfqpoint{3.664478in}{1.461203in}}%
\pgfpathlineto{\pgfqpoint{3.678035in}{1.452847in}}%
\pgfpathlineto{\pgfqpoint{3.691596in}{1.444615in}}%
\pgfpathlineto{\pgfqpoint{3.705160in}{1.436507in}}%
\pgfpathlineto{\pgfqpoint{3.713197in}{1.440026in}}%
\pgfpathlineto{\pgfqpoint{3.721224in}{1.443732in}}%
\pgfpathlineto{\pgfqpoint{3.729243in}{1.447622in}}%
\pgfpathlineto{\pgfqpoint{3.737253in}{1.451692in}}%
\pgfpathlineto{\pgfqpoint{3.723713in}{1.459358in}}%
\pgfpathlineto{\pgfqpoint{3.710176in}{1.467147in}}%
\pgfpathlineto{\pgfqpoint{3.696642in}{1.475061in}}%
\pgfpathlineto{\pgfqpoint{3.683112in}{1.483098in}}%
\pgfpathlineto{\pgfqpoint{3.675078in}{1.479464in}}%
\pgfpathlineto{\pgfqpoint{3.667036in}{1.476015in}}%
\pgfpathlineto{\pgfqpoint{3.658984in}{1.472753in}}%
\pgfpathlineto{\pgfqpoint{3.650923in}{1.469683in}}%
\pgfpathclose%
\pgfusepath{fill}%
\end{pgfscope}%
\begin{pgfscope}%
\pgfpathrectangle{\pgfqpoint{1.254980in}{0.150000in}}{\pgfqpoint{5.490039in}{5.490039in}}%
\pgfusepath{clip}%
\pgfsetbuttcap%
\pgfsetroundjoin%
\definecolor{currentfill}{rgb}{0.241237,0.296485,0.539709}%
\pgfsetfillcolor{currentfill}%
\pgfsetfillopacity{0.700000}%
\pgfsetlinewidth{0.000000pt}%
\definecolor{currentstroke}{rgb}{0.000000,0.000000,0.000000}%
\pgfsetstrokecolor{currentstroke}%
\pgfsetdash{}{0pt}%
\pgfpathmoveto{\pgfqpoint{5.101847in}{1.926663in}}%
\pgfpathlineto{\pgfqpoint{5.115872in}{1.931411in}}%
\pgfpathlineto{\pgfqpoint{5.129911in}{1.936271in}}%
\pgfpathlineto{\pgfqpoint{5.143964in}{1.941242in}}%
\pgfpathlineto{\pgfqpoint{5.158030in}{1.946325in}}%
\pgfpathlineto{\pgfqpoint{5.165592in}{1.959513in}}%
\pgfpathlineto{\pgfqpoint{5.173149in}{1.972660in}}%
\pgfpathlineto{\pgfqpoint{5.180700in}{1.985766in}}%
\pgfpathlineto{\pgfqpoint{5.188247in}{1.998828in}}%
\pgfpathlineto{\pgfqpoint{5.174179in}{1.993541in}}%
\pgfpathlineto{\pgfqpoint{5.160126in}{1.988366in}}%
\pgfpathlineto{\pgfqpoint{5.146087in}{1.983303in}}%
\pgfpathlineto{\pgfqpoint{5.132061in}{1.978351in}}%
\pgfpathlineto{\pgfqpoint{5.124515in}{1.965486in}}%
\pgfpathlineto{\pgfqpoint{5.116964in}{1.952582in}}%
\pgfpathlineto{\pgfqpoint{5.109408in}{1.939641in}}%
\pgfpathlineto{\pgfqpoint{5.101847in}{1.926663in}}%
\pgfpathclose%
\pgfusepath{fill}%
\end{pgfscope}%
\begin{pgfscope}%
\pgfpathrectangle{\pgfqpoint{1.254980in}{0.150000in}}{\pgfqpoint{5.490039in}{5.490039in}}%
\pgfusepath{clip}%
\pgfsetbuttcap%
\pgfsetroundjoin%
\definecolor{currentfill}{rgb}{0.268510,0.009605,0.335427}%
\pgfsetfillcolor{currentfill}%
\pgfsetfillopacity{0.700000}%
\pgfsetlinewidth{0.000000pt}%
\definecolor{currentstroke}{rgb}{0.000000,0.000000,0.000000}%
\pgfsetstrokecolor{currentstroke}%
\pgfsetdash{}{0pt}%
\pgfpathmoveto{\pgfqpoint{4.212285in}{1.381469in}}%
\pgfpathlineto{\pgfqpoint{4.225932in}{1.378457in}}%
\pgfpathlineto{\pgfqpoint{4.239587in}{1.375560in}}%
\pgfpathlineto{\pgfqpoint{4.253249in}{1.372776in}}%
\pgfpathlineto{\pgfqpoint{4.266919in}{1.370107in}}%
\pgfpathlineto{\pgfqpoint{4.274711in}{1.379235in}}%
\pgfpathlineto{\pgfqpoint{4.282498in}{1.388463in}}%
\pgfpathlineto{\pgfqpoint{4.290279in}{1.397787in}}%
\pgfpathlineto{\pgfqpoint{4.298056in}{1.407204in}}%
\pgfpathlineto{\pgfqpoint{4.284396in}{1.409500in}}%
\pgfpathlineto{\pgfqpoint{4.270745in}{1.411909in}}%
\pgfpathlineto{\pgfqpoint{4.257100in}{1.414432in}}%
\pgfpathlineto{\pgfqpoint{4.243464in}{1.417070in}}%
\pgfpathlineto{\pgfqpoint{4.235677in}{1.408021in}}%
\pgfpathlineto{\pgfqpoint{4.227885in}{1.399069in}}%
\pgfpathlineto{\pgfqpoint{4.220088in}{1.390217in}}%
\pgfpathlineto{\pgfqpoint{4.212285in}{1.381469in}}%
\pgfpathclose%
\pgfusepath{fill}%
\end{pgfscope}%
\begin{pgfscope}%
\pgfpathrectangle{\pgfqpoint{1.254980in}{0.150000in}}{\pgfqpoint{5.490039in}{5.490039in}}%
\pgfusepath{clip}%
\pgfsetbuttcap%
\pgfsetroundjoin%
\definecolor{currentfill}{rgb}{0.203063,0.379716,0.553925}%
\pgfsetfillcolor{currentfill}%
\pgfsetfillopacity{0.700000}%
\pgfsetlinewidth{0.000000pt}%
\definecolor{currentstroke}{rgb}{0.000000,0.000000,0.000000}%
\pgfsetstrokecolor{currentstroke}%
\pgfsetdash{}{0pt}%
\pgfpathmoveto{\pgfqpoint{5.304851in}{2.125319in}}%
\pgfpathlineto{\pgfqpoint{5.318992in}{2.131526in}}%
\pgfpathlineto{\pgfqpoint{5.333148in}{2.137844in}}%
\pgfpathlineto{\pgfqpoint{5.347319in}{2.144275in}}%
\pgfpathlineto{\pgfqpoint{5.361505in}{2.150819in}}%
\pgfpathlineto{\pgfqpoint{5.369005in}{2.163757in}}%
\pgfpathlineto{\pgfqpoint{5.376500in}{2.176631in}}%
\pgfpathlineto{\pgfqpoint{5.383989in}{2.189440in}}%
\pgfpathlineto{\pgfqpoint{5.391472in}{2.202183in}}%
\pgfpathlineto{\pgfqpoint{5.377285in}{2.195484in}}%
\pgfpathlineto{\pgfqpoint{5.363113in}{2.188898in}}%
\pgfpathlineto{\pgfqpoint{5.348957in}{2.182424in}}%
\pgfpathlineto{\pgfqpoint{5.334816in}{2.176063in}}%
\pgfpathlineto{\pgfqpoint{5.327333in}{2.163468in}}%
\pgfpathlineto{\pgfqpoint{5.319844in}{2.150812in}}%
\pgfpathlineto{\pgfqpoint{5.312350in}{2.138096in}}%
\pgfpathlineto{\pgfqpoint{5.304851in}{2.125319in}}%
\pgfpathclose%
\pgfusepath{fill}%
\end{pgfscope}%
\begin{pgfscope}%
\pgfpathrectangle{\pgfqpoint{1.254980in}{0.150000in}}{\pgfqpoint{5.490039in}{5.490039in}}%
\pgfusepath{clip}%
\pgfsetbuttcap%
\pgfsetroundjoin%
\definecolor{currentfill}{rgb}{0.121148,0.592739,0.544641}%
\pgfsetfillcolor{currentfill}%
\pgfsetfillopacity{0.700000}%
\pgfsetlinewidth{0.000000pt}%
\definecolor{currentstroke}{rgb}{0.000000,0.000000,0.000000}%
\pgfsetstrokecolor{currentstroke}%
\pgfsetdash{}{0pt}%
\pgfpathmoveto{\pgfqpoint{2.419625in}{2.799927in}}%
\pgfpathlineto{\pgfqpoint{2.433466in}{2.777575in}}%
\pgfpathlineto{\pgfqpoint{2.447298in}{2.755415in}}%
\pgfpathlineto{\pgfqpoint{2.461120in}{2.733446in}}%
\pgfpathlineto{\pgfqpoint{2.474933in}{2.711666in}}%
\pgfpathlineto{\pgfqpoint{2.483894in}{2.703670in}}%
\pgfpathlineto{\pgfqpoint{2.492833in}{2.695985in}}%
\pgfpathlineto{\pgfqpoint{2.501751in}{2.688607in}}%
\pgfpathlineto{\pgfqpoint{2.510648in}{2.681530in}}%
\pgfpathlineto{\pgfqpoint{2.496890in}{2.702795in}}%
\pgfpathlineto{\pgfqpoint{2.483124in}{2.724249in}}%
\pgfpathlineto{\pgfqpoint{2.469349in}{2.745892in}}%
\pgfpathlineto{\pgfqpoint{2.455565in}{2.767727in}}%
\pgfpathlineto{\pgfqpoint{2.446613in}{2.775311in}}%
\pgfpathlineto{\pgfqpoint{2.437639in}{2.783203in}}%
\pgfpathlineto{\pgfqpoint{2.428643in}{2.791407in}}%
\pgfpathlineto{\pgfqpoint{2.419625in}{2.799927in}}%
\pgfpathclose%
\pgfusepath{fill}%
\end{pgfscope}%
\begin{pgfscope}%
\pgfpathrectangle{\pgfqpoint{1.254980in}{0.150000in}}{\pgfqpoint{5.490039in}{5.490039in}}%
\pgfusepath{clip}%
\pgfsetbuttcap%
\pgfsetroundjoin%
\definecolor{currentfill}{rgb}{0.267004,0.004874,0.329415}%
\pgfsetfillcolor{currentfill}%
\pgfsetfillopacity{0.700000}%
\pgfsetlinewidth{0.000000pt}%
\definecolor{currentstroke}{rgb}{0.000000,0.000000,0.000000}%
\pgfsetstrokecolor{currentstroke}%
\pgfsetdash{}{0pt}%
\pgfpathmoveto{\pgfqpoint{3.986095in}{1.370300in}}%
\pgfpathlineto{\pgfqpoint{3.999693in}{1.365100in}}%
\pgfpathlineto{\pgfqpoint{4.013296in}{1.360018in}}%
\pgfpathlineto{\pgfqpoint{4.026905in}{1.355052in}}%
\pgfpathlineto{\pgfqpoint{4.040520in}{1.350204in}}%
\pgfpathlineto{\pgfqpoint{4.048397in}{1.357185in}}%
\pgfpathlineto{\pgfqpoint{4.056267in}{1.364303in}}%
\pgfpathlineto{\pgfqpoint{4.064132in}{1.371555in}}%
\pgfpathlineto{\pgfqpoint{4.071989in}{1.378938in}}%
\pgfpathlineto{\pgfqpoint{4.058390in}{1.383380in}}%
\pgfpathlineto{\pgfqpoint{4.044796in}{1.387939in}}%
\pgfpathlineto{\pgfqpoint{4.031208in}{1.392615in}}%
\pgfpathlineto{\pgfqpoint{4.017627in}{1.397408in}}%
\pgfpathlineto{\pgfqpoint{4.009754in}{1.390425in}}%
\pgfpathlineto{\pgfqpoint{4.001874in}{1.383578in}}%
\pgfpathlineto{\pgfqpoint{3.993988in}{1.376868in}}%
\pgfpathlineto{\pgfqpoint{3.986095in}{1.370300in}}%
\pgfpathclose%
\pgfusepath{fill}%
\end{pgfscope}%
\begin{pgfscope}%
\pgfpathrectangle{\pgfqpoint{1.254980in}{0.150000in}}{\pgfqpoint{5.490039in}{5.490039in}}%
\pgfusepath{clip}%
\pgfsetbuttcap%
\pgfsetroundjoin%
\definecolor{currentfill}{rgb}{0.271828,0.209303,0.504434}%
\pgfsetfillcolor{currentfill}%
\pgfsetfillopacity{0.700000}%
\pgfsetlinewidth{0.000000pt}%
\definecolor{currentstroke}{rgb}{0.000000,0.000000,0.000000}%
\pgfsetstrokecolor{currentstroke}%
\pgfsetdash{}{0pt}%
\pgfpathmoveto{\pgfqpoint{4.899030in}{1.740390in}}%
\pgfpathlineto{\pgfqpoint{4.912952in}{1.743539in}}%
\pgfpathlineto{\pgfqpoint{4.926885in}{1.746800in}}%
\pgfpathlineto{\pgfqpoint{4.940831in}{1.750172in}}%
\pgfpathlineto{\pgfqpoint{4.954790in}{1.753655in}}%
\pgfpathlineto{\pgfqpoint{4.962406in}{1.766680in}}%
\pgfpathlineto{\pgfqpoint{4.970017in}{1.779693in}}%
\pgfpathlineto{\pgfqpoint{4.977624in}{1.792693in}}%
\pgfpathlineto{\pgfqpoint{4.985226in}{1.805678in}}%
\pgfpathlineto{\pgfqpoint{4.971268in}{1.801943in}}%
\pgfpathlineto{\pgfqpoint{4.957322in}{1.798320in}}%
\pgfpathlineto{\pgfqpoint{4.943388in}{1.794808in}}%
\pgfpathlineto{\pgfqpoint{4.929467in}{1.791409in}}%
\pgfpathlineto{\pgfqpoint{4.921865in}{1.778669in}}%
\pgfpathlineto{\pgfqpoint{4.914258in}{1.765918in}}%
\pgfpathlineto{\pgfqpoint{4.906646in}{1.753158in}}%
\pgfpathlineto{\pgfqpoint{4.899030in}{1.740390in}}%
\pgfpathclose%
\pgfusepath{fill}%
\end{pgfscope}%
\begin{pgfscope}%
\pgfpathrectangle{\pgfqpoint{1.254980in}{0.150000in}}{\pgfqpoint{5.490039in}{5.490039in}}%
\pgfusepath{clip}%
\pgfsetbuttcap%
\pgfsetroundjoin%
\definecolor{currentfill}{rgb}{0.269944,0.014625,0.341379}%
\pgfsetfillcolor{currentfill}%
\pgfsetfillopacity{0.700000}%
\pgfsetlinewidth{0.000000pt}%
\definecolor{currentstroke}{rgb}{0.000000,0.000000,0.000000}%
\pgfsetstrokecolor{currentstroke}%
\pgfsetdash{}{0pt}%
\pgfpathmoveto{\pgfqpoint{3.845713in}{1.394758in}}%
\pgfpathlineto{\pgfqpoint{3.859289in}{1.388185in}}%
\pgfpathlineto{\pgfqpoint{3.872870in}{1.381732in}}%
\pgfpathlineto{\pgfqpoint{3.886455in}{1.375398in}}%
\pgfpathlineto{\pgfqpoint{3.900045in}{1.369184in}}%
\pgfpathlineto{\pgfqpoint{3.907984in}{1.374722in}}%
\pgfpathlineto{\pgfqpoint{3.915916in}{1.380419in}}%
\pgfpathlineto{\pgfqpoint{3.923840in}{1.386272in}}%
\pgfpathlineto{\pgfqpoint{3.931758in}{1.392278in}}%
\pgfpathlineto{\pgfqpoint{3.918186in}{1.398069in}}%
\pgfpathlineto{\pgfqpoint{3.904620in}{1.403978in}}%
\pgfpathlineto{\pgfqpoint{3.891059in}{1.410007in}}%
\pgfpathlineto{\pgfqpoint{3.877502in}{1.416156in}}%
\pgfpathlineto{\pgfqpoint{3.869566in}{1.410568in}}%
\pgfpathlineto{\pgfqpoint{3.861623in}{1.405136in}}%
\pgfpathlineto{\pgfqpoint{3.853672in}{1.399865in}}%
\pgfpathlineto{\pgfqpoint{3.845713in}{1.394758in}}%
\pgfpathclose%
\pgfusepath{fill}%
\end{pgfscope}%
\begin{pgfscope}%
\pgfpathrectangle{\pgfqpoint{1.254980in}{0.150000in}}{\pgfqpoint{5.490039in}{5.490039in}}%
\pgfusepath{clip}%
\pgfsetbuttcap%
\pgfsetroundjoin%
\definecolor{currentfill}{rgb}{0.282910,0.105393,0.426902}%
\pgfsetfillcolor{currentfill}%
\pgfsetfillopacity{0.700000}%
\pgfsetlinewidth{0.000000pt}%
\definecolor{currentstroke}{rgb}{0.000000,0.000000,0.000000}%
\pgfsetstrokecolor{currentstroke}%
\pgfsetdash{}{0pt}%
\pgfpathmoveto{\pgfqpoint{3.455775in}{1.576550in}}%
\pgfpathlineto{\pgfqpoint{3.469333in}{1.566082in}}%
\pgfpathlineto{\pgfqpoint{3.482892in}{1.555743in}}%
\pgfpathlineto{\pgfqpoint{3.496453in}{1.545534in}}%
\pgfpathlineto{\pgfqpoint{3.510015in}{1.535453in}}%
\pgfpathlineto{\pgfqpoint{3.518172in}{1.536782in}}%
\pgfpathlineto{\pgfqpoint{3.526317in}{1.538326in}}%
\pgfpathlineto{\pgfqpoint{3.534452in}{1.540083in}}%
\pgfpathlineto{\pgfqpoint{3.542577in}{1.542047in}}%
\pgfpathlineto{\pgfqpoint{3.529043in}{1.551664in}}%
\pgfpathlineto{\pgfqpoint{3.515512in}{1.561410in}}%
\pgfpathlineto{\pgfqpoint{3.501982in}{1.571285in}}%
\pgfpathlineto{\pgfqpoint{3.488454in}{1.581289in}}%
\pgfpathlineto{\pgfqpoint{3.480301in}{1.579781in}}%
\pgfpathlineto{\pgfqpoint{3.472137in}{1.578486in}}%
\pgfpathlineto{\pgfqpoint{3.463961in}{1.577408in}}%
\pgfpathlineto{\pgfqpoint{3.455775in}{1.576550in}}%
\pgfpathclose%
\pgfusepath{fill}%
\end{pgfscope}%
\begin{pgfscope}%
\pgfpathrectangle{\pgfqpoint{1.254980in}{0.150000in}}{\pgfqpoint{5.490039in}{5.490039in}}%
\pgfusepath{clip}%
\pgfsetbuttcap%
\pgfsetroundjoin%
\definecolor{currentfill}{rgb}{0.266941,0.748751,0.440573}%
\pgfsetfillcolor{currentfill}%
\pgfsetfillopacity{0.700000}%
\pgfsetlinewidth{0.000000pt}%
\definecolor{currentstroke}{rgb}{0.000000,0.000000,0.000000}%
\pgfsetstrokecolor{currentstroke}%
\pgfsetdash{}{0pt}%
\pgfpathmoveto{\pgfqpoint{2.177678in}{3.246594in}}%
\pgfpathlineto{\pgfqpoint{2.191685in}{3.220651in}}%
\pgfpathlineto{\pgfqpoint{2.205679in}{3.194928in}}%
\pgfpathlineto{\pgfqpoint{2.219659in}{3.169423in}}%
\pgfpathlineto{\pgfqpoint{2.233628in}{3.144134in}}%
\pgfpathlineto{\pgfqpoint{2.242793in}{3.134797in}}%
\pgfpathlineto{\pgfqpoint{2.251935in}{3.125781in}}%
\pgfpathlineto{\pgfqpoint{2.261053in}{3.117083in}}%
\pgfpathlineto{\pgfqpoint{2.270148in}{3.108696in}}%
\pgfpathlineto{\pgfqpoint{2.256241in}{3.133469in}}%
\pgfpathlineto{\pgfqpoint{2.242322in}{3.158457in}}%
\pgfpathlineto{\pgfqpoint{2.228391in}{3.183661in}}%
\pgfpathlineto{\pgfqpoint{2.214446in}{3.209084in}}%
\pgfpathlineto{\pgfqpoint{2.205291in}{3.217980in}}%
\pgfpathlineto{\pgfqpoint{2.196111in}{3.227194in}}%
\pgfpathlineto{\pgfqpoint{2.186907in}{3.236730in}}%
\pgfpathlineto{\pgfqpoint{2.177678in}{3.246594in}}%
\pgfpathclose%
\pgfusepath{fill}%
\end{pgfscope}%
\begin{pgfscope}%
\pgfpathrectangle{\pgfqpoint{1.254980in}{0.150000in}}{\pgfqpoint{5.490039in}{5.490039in}}%
\pgfusepath{clip}%
\pgfsetbuttcap%
\pgfsetroundjoin%
\definecolor{currentfill}{rgb}{0.188923,0.410910,0.556326}%
\pgfsetfillcolor{currentfill}%
\pgfsetfillopacity{0.700000}%
\pgfsetlinewidth{0.000000pt}%
\definecolor{currentstroke}{rgb}{0.000000,0.000000,0.000000}%
\pgfsetstrokecolor{currentstroke}%
\pgfsetdash{}{0pt}%
\pgfpathmoveto{\pgfqpoint{5.391472in}{2.202183in}}%
\pgfpathlineto{\pgfqpoint{5.405674in}{2.208994in}}%
\pgfpathlineto{\pgfqpoint{5.419892in}{2.215918in}}%
\pgfpathlineto{\pgfqpoint{5.434125in}{2.222954in}}%
\pgfpathlineto{\pgfqpoint{5.441602in}{2.235739in}}%
\pgfpathlineto{\pgfqpoint{5.449074in}{2.248453in}}%
\pgfpathlineto{\pgfqpoint{5.456540in}{2.261095in}}%
\pgfpathlineto{\pgfqpoint{5.463999in}{2.273664in}}%
\pgfpathlineto{\pgfqpoint{5.449766in}{2.266489in}}%
\pgfpathlineto{\pgfqpoint{5.435548in}{2.259427in}}%
\pgfpathlineto{\pgfqpoint{5.421345in}{2.252477in}}%
\pgfpathlineto{\pgfqpoint{5.413886in}{2.240007in}}%
\pgfpathlineto{\pgfqpoint{5.406420in}{2.227468in}}%
\pgfpathlineto{\pgfqpoint{5.398949in}{2.214859in}}%
\pgfpathlineto{\pgfqpoint{5.391472in}{2.202183in}}%
\pgfpathclose%
\pgfusepath{fill}%
\end{pgfscope}%
\begin{pgfscope}%
\pgfpathrectangle{\pgfqpoint{1.254980in}{0.150000in}}{\pgfqpoint{5.490039in}{5.490039in}}%
\pgfusepath{clip}%
\pgfsetbuttcap%
\pgfsetroundjoin%
\definecolor{currentfill}{rgb}{0.227802,0.326594,0.546532}%
\pgfsetfillcolor{currentfill}%
\pgfsetfillopacity{0.700000}%
\pgfsetlinewidth{0.000000pt}%
\definecolor{currentstroke}{rgb}{0.000000,0.000000,0.000000}%
\pgfsetstrokecolor{currentstroke}%
\pgfsetdash{}{0pt}%
\pgfpathmoveto{\pgfqpoint{2.933323in}{2.060996in}}%
\pgfpathlineto{\pgfqpoint{2.946958in}{2.044975in}}%
\pgfpathlineto{\pgfqpoint{2.960590in}{2.029107in}}%
\pgfpathlineto{\pgfqpoint{2.974219in}{2.013391in}}%
\pgfpathlineto{\pgfqpoint{2.987844in}{1.997826in}}%
\pgfpathlineto{\pgfqpoint{2.996382in}{1.993780in}}%
\pgfpathlineto{\pgfqpoint{3.004903in}{1.990011in}}%
\pgfpathlineto{\pgfqpoint{3.013408in}{1.986514in}}%
\pgfpathlineto{\pgfqpoint{3.021898in}{1.983283in}}%
\pgfpathlineto{\pgfqpoint{3.008316in}{1.998349in}}%
\pgfpathlineto{\pgfqpoint{2.994731in}{2.013565in}}%
\pgfpathlineto{\pgfqpoint{2.981142in}{2.028932in}}%
\pgfpathlineto{\pgfqpoint{2.967551in}{2.044451in}}%
\pgfpathlineto{\pgfqpoint{2.959019in}{2.048175in}}%
\pgfpathlineto{\pgfqpoint{2.950470in}{2.052170in}}%
\pgfpathlineto{\pgfqpoint{2.941905in}{2.056443in}}%
\pgfpathlineto{\pgfqpoint{2.933323in}{2.060996in}}%
\pgfpathclose%
\pgfusepath{fill}%
\end{pgfscope}%
\begin{pgfscope}%
\pgfpathrectangle{\pgfqpoint{1.254980in}{0.150000in}}{\pgfqpoint{5.490039in}{5.490039in}}%
\pgfusepath{clip}%
\pgfsetbuttcap%
\pgfsetroundjoin%
\definecolor{currentfill}{rgb}{0.267004,0.004874,0.329415}%
\pgfsetfillcolor{currentfill}%
\pgfsetfillopacity{0.700000}%
\pgfsetlinewidth{0.000000pt}%
\definecolor{currentstroke}{rgb}{0.000000,0.000000,0.000000}%
\pgfsetstrokecolor{currentstroke}%
\pgfsetdash{}{0pt}%
\pgfpathmoveto{\pgfqpoint{4.126451in}{1.362332in}}%
\pgfpathlineto{\pgfqpoint{4.140083in}{1.358470in}}%
\pgfpathlineto{\pgfqpoint{4.153721in}{1.354723in}}%
\pgfpathlineto{\pgfqpoint{4.167366in}{1.351091in}}%
\pgfpathlineto{\pgfqpoint{4.181018in}{1.347574in}}%
\pgfpathlineto{\pgfqpoint{4.188843in}{1.355877in}}%
\pgfpathlineto{\pgfqpoint{4.196663in}{1.364296in}}%
\pgfpathlineto{\pgfqpoint{4.204477in}{1.372828in}}%
\pgfpathlineto{\pgfqpoint{4.212285in}{1.381469in}}%
\pgfpathlineto{\pgfqpoint{4.198645in}{1.384596in}}%
\pgfpathlineto{\pgfqpoint{4.185012in}{1.387838in}}%
\pgfpathlineto{\pgfqpoint{4.171386in}{1.391194in}}%
\pgfpathlineto{\pgfqpoint{4.157767in}{1.394666in}}%
\pgfpathlineto{\pgfqpoint{4.149947in}{1.386409in}}%
\pgfpathlineto{\pgfqpoint{4.142121in}{1.378265in}}%
\pgfpathlineto{\pgfqpoint{4.134289in}{1.370239in}}%
\pgfpathlineto{\pgfqpoint{4.126451in}{1.362332in}}%
\pgfpathclose%
\pgfusepath{fill}%
\end{pgfscope}%
\begin{pgfscope}%
\pgfpathrectangle{\pgfqpoint{1.254980in}{0.150000in}}{\pgfqpoint{5.490039in}{5.490039in}}%
\pgfusepath{clip}%
\pgfsetbuttcap%
\pgfsetroundjoin%
\definecolor{currentfill}{rgb}{0.216210,0.351535,0.550627}%
\pgfsetfillcolor{currentfill}%
\pgfsetfillopacity{0.700000}%
\pgfsetlinewidth{0.000000pt}%
\definecolor{currentstroke}{rgb}{0.000000,0.000000,0.000000}%
\pgfsetstrokecolor{currentstroke}%
\pgfsetdash{}{0pt}%
\pgfpathmoveto{\pgfqpoint{2.878744in}{2.126622in}}%
\pgfpathlineto{\pgfqpoint{2.892395in}{2.109983in}}%
\pgfpathlineto{\pgfqpoint{2.906041in}{2.093499in}}%
\pgfpathlineto{\pgfqpoint{2.919684in}{2.077171in}}%
\pgfpathlineto{\pgfqpoint{2.933323in}{2.060996in}}%
\pgfpathlineto{\pgfqpoint{2.941905in}{2.056443in}}%
\pgfpathlineto{\pgfqpoint{2.950470in}{2.052170in}}%
\pgfpathlineto{\pgfqpoint{2.959019in}{2.048175in}}%
\pgfpathlineto{\pgfqpoint{2.967551in}{2.044451in}}%
\pgfpathlineto{\pgfqpoint{2.953957in}{2.060122in}}%
\pgfpathlineto{\pgfqpoint{2.940359in}{2.075948in}}%
\pgfpathlineto{\pgfqpoint{2.926757in}{2.091927in}}%
\pgfpathlineto{\pgfqpoint{2.913152in}{2.108062in}}%
\pgfpathlineto{\pgfqpoint{2.904575in}{2.112282in}}%
\pgfpathlineto{\pgfqpoint{2.895982in}{2.116779in}}%
\pgfpathlineto{\pgfqpoint{2.887372in}{2.121557in}}%
\pgfpathlineto{\pgfqpoint{2.878744in}{2.126622in}}%
\pgfpathclose%
\pgfusepath{fill}%
\end{pgfscope}%
\begin{pgfscope}%
\pgfpathrectangle{\pgfqpoint{1.254980in}{0.150000in}}{\pgfqpoint{5.490039in}{5.490039in}}%
\pgfusepath{clip}%
\pgfsetbuttcap%
\pgfsetroundjoin%
\definecolor{currentfill}{rgb}{0.239346,0.300855,0.540844}%
\pgfsetfillcolor{currentfill}%
\pgfsetfillopacity{0.700000}%
\pgfsetlinewidth{0.000000pt}%
\definecolor{currentstroke}{rgb}{0.000000,0.000000,0.000000}%
\pgfsetstrokecolor{currentstroke}%
\pgfsetdash{}{0pt}%
\pgfpathmoveto{\pgfqpoint{2.987844in}{1.997826in}}%
\pgfpathlineto{\pgfqpoint{3.001466in}{1.982411in}}%
\pgfpathlineto{\pgfqpoint{3.015085in}{1.967145in}}%
\pgfpathlineto{\pgfqpoint{3.028701in}{1.952029in}}%
\pgfpathlineto{\pgfqpoint{3.042315in}{1.937060in}}%
\pgfpathlineto{\pgfqpoint{3.050810in}{1.933521in}}%
\pgfpathlineto{\pgfqpoint{3.059289in}{1.930252in}}%
\pgfpathlineto{\pgfqpoint{3.067752in}{1.927250in}}%
\pgfpathlineto{\pgfqpoint{3.076201in}{1.924510in}}%
\pgfpathlineto{\pgfqpoint{3.062629in}{1.938981in}}%
\pgfpathlineto{\pgfqpoint{3.049054in}{1.953600in}}%
\pgfpathlineto{\pgfqpoint{3.035477in}{1.968367in}}%
\pgfpathlineto{\pgfqpoint{3.021898in}{1.983283in}}%
\pgfpathlineto{\pgfqpoint{3.013408in}{1.986514in}}%
\pgfpathlineto{\pgfqpoint{3.004903in}{1.990011in}}%
\pgfpathlineto{\pgfqpoint{2.996382in}{1.993780in}}%
\pgfpathlineto{\pgfqpoint{2.987844in}{1.997826in}}%
\pgfpathclose%
\pgfusepath{fill}%
\end{pgfscope}%
\begin{pgfscope}%
\pgfpathrectangle{\pgfqpoint{1.254980in}{0.150000in}}{\pgfqpoint{5.490039in}{5.490039in}}%
\pgfusepath{clip}%
\pgfsetbuttcap%
\pgfsetroundjoin%
\definecolor{currentfill}{rgb}{0.203063,0.379716,0.553925}%
\pgfsetfillcolor{currentfill}%
\pgfsetfillopacity{0.700000}%
\pgfsetlinewidth{0.000000pt}%
\definecolor{currentstroke}{rgb}{0.000000,0.000000,0.000000}%
\pgfsetstrokecolor{currentstroke}%
\pgfsetdash{}{0pt}%
\pgfpathmoveto{\pgfqpoint{2.824097in}{2.194755in}}%
\pgfpathlineto{\pgfqpoint{2.837766in}{2.177484in}}%
\pgfpathlineto{\pgfqpoint{2.851429in}{2.160372in}}%
\pgfpathlineto{\pgfqpoint{2.865089in}{2.143418in}}%
\pgfpathlineto{\pgfqpoint{2.878744in}{2.126622in}}%
\pgfpathlineto{\pgfqpoint{2.887372in}{2.121557in}}%
\pgfpathlineto{\pgfqpoint{2.895982in}{2.116779in}}%
\pgfpathlineto{\pgfqpoint{2.904575in}{2.112282in}}%
\pgfpathlineto{\pgfqpoint{2.913152in}{2.108062in}}%
\pgfpathlineto{\pgfqpoint{2.899543in}{2.124352in}}%
\pgfpathlineto{\pgfqpoint{2.885929in}{2.140800in}}%
\pgfpathlineto{\pgfqpoint{2.872312in}{2.157405in}}%
\pgfpathlineto{\pgfqpoint{2.858691in}{2.174168in}}%
\pgfpathlineto{\pgfqpoint{2.850069in}{2.178887in}}%
\pgfpathlineto{\pgfqpoint{2.841430in}{2.183888in}}%
\pgfpathlineto{\pgfqpoint{2.832772in}{2.189176in}}%
\pgfpathlineto{\pgfqpoint{2.824097in}{2.194755in}}%
\pgfpathclose%
\pgfusepath{fill}%
\end{pgfscope}%
\begin{pgfscope}%
\pgfpathrectangle{\pgfqpoint{1.254980in}{0.150000in}}{\pgfqpoint{5.490039in}{5.490039in}}%
\pgfusepath{clip}%
\pgfsetbuttcap%
\pgfsetroundjoin%
\definecolor{currentfill}{rgb}{0.225863,0.330805,0.547314}%
\pgfsetfillcolor{currentfill}%
\pgfsetfillopacity{0.700000}%
\pgfsetlinewidth{0.000000pt}%
\definecolor{currentstroke}{rgb}{0.000000,0.000000,0.000000}%
\pgfsetstrokecolor{currentstroke}%
\pgfsetdash{}{0pt}%
\pgfpathmoveto{\pgfqpoint{5.188247in}{1.998828in}}%
\pgfpathlineto{\pgfqpoint{5.202328in}{2.004227in}}%
\pgfpathlineto{\pgfqpoint{5.216424in}{2.009738in}}%
\pgfpathlineto{\pgfqpoint{5.230534in}{2.015361in}}%
\pgfpathlineto{\pgfqpoint{5.244659in}{2.021095in}}%
\pgfpathlineto{\pgfqpoint{5.252201in}{2.034307in}}%
\pgfpathlineto{\pgfqpoint{5.259738in}{2.047470in}}%
\pgfpathlineto{\pgfqpoint{5.267271in}{2.060580in}}%
\pgfpathlineto{\pgfqpoint{5.274797in}{2.073638in}}%
\pgfpathlineto{\pgfqpoint{5.260672in}{2.067715in}}%
\pgfpathlineto{\pgfqpoint{5.246561in}{2.061905in}}%
\pgfpathlineto{\pgfqpoint{5.232464in}{2.056206in}}%
\pgfpathlineto{\pgfqpoint{5.218382in}{2.050619in}}%
\pgfpathlineto{\pgfqpoint{5.210856in}{2.037743in}}%
\pgfpathlineto{\pgfqpoint{5.203325in}{2.024818in}}%
\pgfpathlineto{\pgfqpoint{5.195788in}{2.011846in}}%
\pgfpathlineto{\pgfqpoint{5.188247in}{1.998828in}}%
\pgfpathclose%
\pgfusepath{fill}%
\end{pgfscope}%
\begin{pgfscope}%
\pgfpathrectangle{\pgfqpoint{1.254980in}{0.150000in}}{\pgfqpoint{5.490039in}{5.490039in}}%
\pgfusepath{clip}%
\pgfsetbuttcap%
\pgfsetroundjoin%
\definecolor{currentfill}{rgb}{0.280894,0.078907,0.402329}%
\pgfsetfillcolor{currentfill}%
\pgfsetfillopacity{0.700000}%
\pgfsetlinewidth{0.000000pt}%
\definecolor{currentstroke}{rgb}{0.000000,0.000000,0.000000}%
\pgfsetstrokecolor{currentstroke}%
\pgfsetdash{}{0pt}%
\pgfpathmoveto{\pgfqpoint{4.524473in}{1.475273in}}%
\pgfpathlineto{\pgfqpoint{4.538232in}{1.475148in}}%
\pgfpathlineto{\pgfqpoint{4.552000in}{1.475135in}}%
\pgfpathlineto{\pgfqpoint{4.565779in}{1.475235in}}%
\pgfpathlineto{\pgfqpoint{4.579567in}{1.475445in}}%
\pgfpathlineto{\pgfqpoint{4.587274in}{1.486908in}}%
\pgfpathlineto{\pgfqpoint{4.594977in}{1.498420in}}%
\pgfpathlineto{\pgfqpoint{4.602676in}{1.509979in}}%
\pgfpathlineto{\pgfqpoint{4.610370in}{1.521582in}}%
\pgfpathlineto{\pgfqpoint{4.596587in}{1.521043in}}%
\pgfpathlineto{\pgfqpoint{4.582813in}{1.520615in}}%
\pgfpathlineto{\pgfqpoint{4.569049in}{1.520300in}}%
\pgfpathlineto{\pgfqpoint{4.555296in}{1.520097in}}%
\pgfpathlineto{\pgfqpoint{4.547597in}{1.508816in}}%
\pgfpathlineto{\pgfqpoint{4.539893in}{1.497583in}}%
\pgfpathlineto{\pgfqpoint{4.532186in}{1.486401in}}%
\pgfpathlineto{\pgfqpoint{4.524473in}{1.475273in}}%
\pgfpathclose%
\pgfusepath{fill}%
\end{pgfscope}%
\begin{pgfscope}%
\pgfpathrectangle{\pgfqpoint{1.254980in}{0.150000in}}{\pgfqpoint{5.490039in}{5.490039in}}%
\pgfusepath{clip}%
\pgfsetbuttcap%
\pgfsetroundjoin%
\definecolor{currentfill}{rgb}{0.250425,0.274290,0.533103}%
\pgfsetfillcolor{currentfill}%
\pgfsetfillopacity{0.700000}%
\pgfsetlinewidth{0.000000pt}%
\definecolor{currentstroke}{rgb}{0.000000,0.000000,0.000000}%
\pgfsetstrokecolor{currentstroke}%
\pgfsetdash{}{0pt}%
\pgfpathmoveto{\pgfqpoint{3.042315in}{1.937060in}}%
\pgfpathlineto{\pgfqpoint{3.055926in}{1.922240in}}%
\pgfpathlineto{\pgfqpoint{3.069534in}{1.907565in}}%
\pgfpathlineto{\pgfqpoint{3.083141in}{1.893037in}}%
\pgfpathlineto{\pgfqpoint{3.096745in}{1.878654in}}%
\pgfpathlineto{\pgfqpoint{3.105198in}{1.875618in}}%
\pgfpathlineto{\pgfqpoint{3.113636in}{1.872847in}}%
\pgfpathlineto{\pgfqpoint{3.122059in}{1.870338in}}%
\pgfpathlineto{\pgfqpoint{3.130468in}{1.868086in}}%
\pgfpathlineto{\pgfqpoint{3.116904in}{1.881974in}}%
\pgfpathlineto{\pgfqpoint{3.103338in}{1.896007in}}%
\pgfpathlineto{\pgfqpoint{3.089770in}{1.910186in}}%
\pgfpathlineto{\pgfqpoint{3.076201in}{1.924510in}}%
\pgfpathlineto{\pgfqpoint{3.067752in}{1.927250in}}%
\pgfpathlineto{\pgfqpoint{3.059289in}{1.930252in}}%
\pgfpathlineto{\pgfqpoint{3.050810in}{1.933521in}}%
\pgfpathlineto{\pgfqpoint{3.042315in}{1.937060in}}%
\pgfpathclose%
\pgfusepath{fill}%
\end{pgfscope}%
\begin{pgfscope}%
\pgfpathrectangle{\pgfqpoint{1.254980in}{0.150000in}}{\pgfqpoint{5.490039in}{5.490039in}}%
\pgfusepath{clip}%
\pgfsetbuttcap%
\pgfsetroundjoin%
\definecolor{currentfill}{rgb}{0.121380,0.629492,0.531973}%
\pgfsetfillcolor{currentfill}%
\pgfsetfillopacity{0.700000}%
\pgfsetlinewidth{0.000000pt}%
\definecolor{currentstroke}{rgb}{0.000000,0.000000,0.000000}%
\pgfsetstrokecolor{currentstroke}%
\pgfsetdash{}{0pt}%
\pgfpathmoveto{\pgfqpoint{2.364162in}{2.891282in}}%
\pgfpathlineto{\pgfqpoint{2.378043in}{2.868149in}}%
\pgfpathlineto{\pgfqpoint{2.391914in}{2.845212in}}%
\pgfpathlineto{\pgfqpoint{2.405774in}{2.822472in}}%
\pgfpathlineto{\pgfqpoint{2.419625in}{2.799927in}}%
\pgfpathlineto{\pgfqpoint{2.428643in}{2.791407in}}%
\pgfpathlineto{\pgfqpoint{2.437639in}{2.783203in}}%
\pgfpathlineto{\pgfqpoint{2.446613in}{2.775311in}}%
\pgfpathlineto{\pgfqpoint{2.455565in}{2.767727in}}%
\pgfpathlineto{\pgfqpoint{2.441772in}{2.789753in}}%
\pgfpathlineto{\pgfqpoint{2.427969in}{2.811974in}}%
\pgfpathlineto{\pgfqpoint{2.414156in}{2.834388in}}%
\pgfpathlineto{\pgfqpoint{2.400334in}{2.857000in}}%
\pgfpathlineto{\pgfqpoint{2.391325in}{2.865096in}}%
\pgfpathlineto{\pgfqpoint{2.382293in}{2.873505in}}%
\pgfpathlineto{\pgfqpoint{2.373239in}{2.882233in}}%
\pgfpathlineto{\pgfqpoint{2.364162in}{2.891282in}}%
\pgfpathclose%
\pgfusepath{fill}%
\end{pgfscope}%
\begin{pgfscope}%
\pgfpathrectangle{\pgfqpoint{1.254980in}{0.150000in}}{\pgfqpoint{5.490039in}{5.490039in}}%
\pgfusepath{clip}%
\pgfsetbuttcap%
\pgfsetroundjoin%
\definecolor{currentfill}{rgb}{0.274952,0.037752,0.364543}%
\pgfsetfillcolor{currentfill}%
\pgfsetfillopacity{0.700000}%
\pgfsetlinewidth{0.000000pt}%
\definecolor{currentstroke}{rgb}{0.000000,0.000000,0.000000}%
\pgfsetstrokecolor{currentstroke}%
\pgfsetdash{}{0pt}%
\pgfpathmoveto{\pgfqpoint{3.705160in}{1.436507in}}%
\pgfpathlineto{\pgfqpoint{3.718727in}{1.428521in}}%
\pgfpathlineto{\pgfqpoint{3.732298in}{1.420658in}}%
\pgfpathlineto{\pgfqpoint{3.745872in}{1.412918in}}%
\pgfpathlineto{\pgfqpoint{3.759449in}{1.405300in}}%
\pgfpathlineto{\pgfqpoint{3.767462in}{1.409267in}}%
\pgfpathlineto{\pgfqpoint{3.775467in}{1.413418in}}%
\pgfpathlineto{\pgfqpoint{3.783463in}{1.417748in}}%
\pgfpathlineto{\pgfqpoint{3.791451in}{1.422253in}}%
\pgfpathlineto{\pgfqpoint{3.777896in}{1.429430in}}%
\pgfpathlineto{\pgfqpoint{3.764345in}{1.436728in}}%
\pgfpathlineto{\pgfqpoint{3.750797in}{1.444149in}}%
\pgfpathlineto{\pgfqpoint{3.737253in}{1.451692in}}%
\pgfpathlineto{\pgfqpoint{3.729243in}{1.447622in}}%
\pgfpathlineto{\pgfqpoint{3.721224in}{1.443732in}}%
\pgfpathlineto{\pgfqpoint{3.713197in}{1.440026in}}%
\pgfpathlineto{\pgfqpoint{3.705160in}{1.436507in}}%
\pgfpathclose%
\pgfusepath{fill}%
\end{pgfscope}%
\begin{pgfscope}%
\pgfpathrectangle{\pgfqpoint{1.254980in}{0.150000in}}{\pgfqpoint{5.490039in}{5.490039in}}%
\pgfusepath{clip}%
\pgfsetbuttcap%
\pgfsetroundjoin%
\definecolor{currentfill}{rgb}{0.277018,0.050344,0.375715}%
\pgfsetfillcolor{currentfill}%
\pgfsetfillopacity{0.700000}%
\pgfsetlinewidth{0.000000pt}%
\definecolor{currentstroke}{rgb}{0.000000,0.000000,0.000000}%
\pgfsetstrokecolor{currentstroke}%
\pgfsetdash{}{0pt}%
\pgfpathmoveto{\pgfqpoint{4.438617in}{1.434343in}}%
\pgfpathlineto{\pgfqpoint{4.452344in}{1.433426in}}%
\pgfpathlineto{\pgfqpoint{4.466080in}{1.432621in}}%
\pgfpathlineto{\pgfqpoint{4.479826in}{1.431928in}}%
\pgfpathlineto{\pgfqpoint{4.493581in}{1.431347in}}%
\pgfpathlineto{\pgfqpoint{4.501310in}{1.442235in}}%
\pgfpathlineto{\pgfqpoint{4.509036in}{1.453187in}}%
\pgfpathlineto{\pgfqpoint{4.516757in}{1.464201in}}%
\pgfpathlineto{\pgfqpoint{4.524473in}{1.475273in}}%
\pgfpathlineto{\pgfqpoint{4.510724in}{1.475510in}}%
\pgfpathlineto{\pgfqpoint{4.496985in}{1.475859in}}%
\pgfpathlineto{\pgfqpoint{4.483254in}{1.476320in}}%
\pgfpathlineto{\pgfqpoint{4.469534in}{1.476894in}}%
\pgfpathlineto{\pgfqpoint{4.461811in}{1.466160in}}%
\pgfpathlineto{\pgfqpoint{4.454084in}{1.455488in}}%
\pgfpathlineto{\pgfqpoint{4.446353in}{1.444882in}}%
\pgfpathlineto{\pgfqpoint{4.438617in}{1.434343in}}%
\pgfpathclose%
\pgfusepath{fill}%
\end{pgfscope}%
\begin{pgfscope}%
\pgfpathrectangle{\pgfqpoint{1.254980in}{0.150000in}}{\pgfqpoint{5.490039in}{5.490039in}}%
\pgfusepath{clip}%
\pgfsetbuttcap%
\pgfsetroundjoin%
\definecolor{currentfill}{rgb}{0.260571,0.246922,0.522828}%
\pgfsetfillcolor{currentfill}%
\pgfsetfillopacity{0.700000}%
\pgfsetlinewidth{0.000000pt}%
\definecolor{currentstroke}{rgb}{0.000000,0.000000,0.000000}%
\pgfsetstrokecolor{currentstroke}%
\pgfsetdash{}{0pt}%
\pgfpathmoveto{\pgfqpoint{4.985226in}{1.805678in}}%
\pgfpathlineto{\pgfqpoint{4.999198in}{1.809524in}}%
\pgfpathlineto{\pgfqpoint{5.013182in}{1.813482in}}%
\pgfpathlineto{\pgfqpoint{5.027180in}{1.817551in}}%
\pgfpathlineto{\pgfqpoint{5.041190in}{1.821731in}}%
\pgfpathlineto{\pgfqpoint{5.048788in}{1.834941in}}%
\pgfpathlineto{\pgfqpoint{5.056382in}{1.848127in}}%
\pgfpathlineto{\pgfqpoint{5.063971in}{1.861288in}}%
\pgfpathlineto{\pgfqpoint{5.071556in}{1.874423in}}%
\pgfpathlineto{\pgfqpoint{5.057545in}{1.870006in}}%
\pgfpathlineto{\pgfqpoint{5.043547in}{1.865701in}}%
\pgfpathlineto{\pgfqpoint{5.029562in}{1.861508in}}%
\pgfpathlineto{\pgfqpoint{5.015591in}{1.857426in}}%
\pgfpathlineto{\pgfqpoint{5.008006in}{1.844521in}}%
\pgfpathlineto{\pgfqpoint{5.000417in}{1.831594in}}%
\pgfpathlineto{\pgfqpoint{4.992824in}{1.818645in}}%
\pgfpathlineto{\pgfqpoint{4.985226in}{1.805678in}}%
\pgfpathclose%
\pgfusepath{fill}%
\end{pgfscope}%
\begin{pgfscope}%
\pgfpathrectangle{\pgfqpoint{1.254980in}{0.150000in}}{\pgfqpoint{5.490039in}{5.490039in}}%
\pgfusepath{clip}%
\pgfsetbuttcap%
\pgfsetroundjoin%
\definecolor{currentfill}{rgb}{0.282910,0.105393,0.426902}%
\pgfsetfillcolor{currentfill}%
\pgfsetfillopacity{0.700000}%
\pgfsetlinewidth{0.000000pt}%
\definecolor{currentstroke}{rgb}{0.000000,0.000000,0.000000}%
\pgfsetstrokecolor{currentstroke}%
\pgfsetdash{}{0pt}%
\pgfpathmoveto{\pgfqpoint{4.610370in}{1.521582in}}%
\pgfpathlineto{\pgfqpoint{4.624164in}{1.522233in}}%
\pgfpathlineto{\pgfqpoint{4.637969in}{1.522996in}}%
\pgfpathlineto{\pgfqpoint{4.651783in}{1.523871in}}%
\pgfpathlineto{\pgfqpoint{4.665609in}{1.524856in}}%
\pgfpathlineto{\pgfqpoint{4.673296in}{1.536821in}}%
\pgfpathlineto{\pgfqpoint{4.680978in}{1.548820in}}%
\pgfpathlineto{\pgfqpoint{4.688657in}{1.560852in}}%
\pgfpathlineto{\pgfqpoint{4.696331in}{1.572915in}}%
\pgfpathlineto{\pgfqpoint{4.682509in}{1.571616in}}%
\pgfpathlineto{\pgfqpoint{4.668697in}{1.570428in}}%
\pgfpathlineto{\pgfqpoint{4.654896in}{1.569352in}}%
\pgfpathlineto{\pgfqpoint{4.641106in}{1.568389in}}%
\pgfpathlineto{\pgfqpoint{4.633428in}{1.556633in}}%
\pgfpathlineto{\pgfqpoint{4.625747in}{1.544912in}}%
\pgfpathlineto{\pgfqpoint{4.618061in}{1.533227in}}%
\pgfpathlineto{\pgfqpoint{4.610370in}{1.521582in}}%
\pgfpathclose%
\pgfusepath{fill}%
\end{pgfscope}%
\begin{pgfscope}%
\pgfpathrectangle{\pgfqpoint{1.254980in}{0.150000in}}{\pgfqpoint{5.490039in}{5.490039in}}%
\pgfusepath{clip}%
\pgfsetbuttcap%
\pgfsetroundjoin%
\definecolor{currentfill}{rgb}{0.190631,0.407061,0.556089}%
\pgfsetfillcolor{currentfill}%
\pgfsetfillopacity{0.700000}%
\pgfsetlinewidth{0.000000pt}%
\definecolor{currentstroke}{rgb}{0.000000,0.000000,0.000000}%
\pgfsetstrokecolor{currentstroke}%
\pgfsetdash{}{0pt}%
\pgfpathmoveto{\pgfqpoint{2.769373in}{2.265450in}}%
\pgfpathlineto{\pgfqpoint{2.783062in}{2.247533in}}%
\pgfpathlineto{\pgfqpoint{2.796745in}{2.229779in}}%
\pgfpathlineto{\pgfqpoint{2.810424in}{2.212186in}}%
\pgfpathlineto{\pgfqpoint{2.824097in}{2.194755in}}%
\pgfpathlineto{\pgfqpoint{2.832772in}{2.189176in}}%
\pgfpathlineto{\pgfqpoint{2.841430in}{2.183888in}}%
\pgfpathlineto{\pgfqpoint{2.850069in}{2.178887in}}%
\pgfpathlineto{\pgfqpoint{2.858691in}{2.174168in}}%
\pgfpathlineto{\pgfqpoint{2.845065in}{2.191090in}}%
\pgfpathlineto{\pgfqpoint{2.831434in}{2.208173in}}%
\pgfpathlineto{\pgfqpoint{2.817799in}{2.225417in}}%
\pgfpathlineto{\pgfqpoint{2.804159in}{2.242822in}}%
\pgfpathlineto{\pgfqpoint{2.795490in}{2.248043in}}%
\pgfpathlineto{\pgfqpoint{2.786803in}{2.253552in}}%
\pgfpathlineto{\pgfqpoint{2.778098in}{2.259352in}}%
\pgfpathlineto{\pgfqpoint{2.769373in}{2.265450in}}%
\pgfpathclose%
\pgfusepath{fill}%
\end{pgfscope}%
\begin{pgfscope}%
\pgfpathrectangle{\pgfqpoint{1.254980in}{0.150000in}}{\pgfqpoint{5.490039in}{5.490039in}}%
\pgfusepath{clip}%
\pgfsetbuttcap%
\pgfsetroundjoin%
\definecolor{currentfill}{rgb}{0.258965,0.251537,0.524736}%
\pgfsetfillcolor{currentfill}%
\pgfsetfillopacity{0.700000}%
\pgfsetlinewidth{0.000000pt}%
\definecolor{currentstroke}{rgb}{0.000000,0.000000,0.000000}%
\pgfsetstrokecolor{currentstroke}%
\pgfsetdash{}{0pt}%
\pgfpathmoveto{\pgfqpoint{3.096745in}{1.878654in}}%
\pgfpathlineto{\pgfqpoint{3.110347in}{1.864416in}}%
\pgfpathlineto{\pgfqpoint{3.123947in}{1.850322in}}%
\pgfpathlineto{\pgfqpoint{3.137545in}{1.836370in}}%
\pgfpathlineto{\pgfqpoint{3.151142in}{1.822562in}}%
\pgfpathlineto{\pgfqpoint{3.159555in}{1.820026in}}%
\pgfpathlineto{\pgfqpoint{3.167953in}{1.817751in}}%
\pgfpathlineto{\pgfqpoint{3.176338in}{1.815732in}}%
\pgfpathlineto{\pgfqpoint{3.184708in}{1.813966in}}%
\pgfpathlineto{\pgfqpoint{3.171150in}{1.827282in}}%
\pgfpathlineto{\pgfqpoint{3.157591in}{1.840741in}}%
\pgfpathlineto{\pgfqpoint{3.144030in}{1.854342in}}%
\pgfpathlineto{\pgfqpoint{3.130468in}{1.868086in}}%
\pgfpathlineto{\pgfqpoint{3.122059in}{1.870338in}}%
\pgfpathlineto{\pgfqpoint{3.113636in}{1.872847in}}%
\pgfpathlineto{\pgfqpoint{3.105198in}{1.875618in}}%
\pgfpathlineto{\pgfqpoint{3.096745in}{1.878654in}}%
\pgfpathclose%
\pgfusepath{fill}%
\end{pgfscope}%
\begin{pgfscope}%
\pgfpathrectangle{\pgfqpoint{1.254980in}{0.150000in}}{\pgfqpoint{5.490039in}{5.490039in}}%
\pgfusepath{clip}%
\pgfsetbuttcap%
\pgfsetroundjoin%
\definecolor{currentfill}{rgb}{0.273809,0.031497,0.358853}%
\pgfsetfillcolor{currentfill}%
\pgfsetfillopacity{0.700000}%
\pgfsetlinewidth{0.000000pt}%
\definecolor{currentstroke}{rgb}{0.000000,0.000000,0.000000}%
\pgfsetstrokecolor{currentstroke}%
\pgfsetdash{}{0pt}%
\pgfpathmoveto{\pgfqpoint{4.352774in}{1.399161in}}%
\pgfpathlineto{\pgfqpoint{4.366475in}{1.397433in}}%
\pgfpathlineto{\pgfqpoint{4.380183in}{1.395819in}}%
\pgfpathlineto{\pgfqpoint{4.393900in}{1.394317in}}%
\pgfpathlineto{\pgfqpoint{4.407626in}{1.392927in}}%
\pgfpathlineto{\pgfqpoint{4.415380in}{1.403165in}}%
\pgfpathlineto{\pgfqpoint{4.423131in}{1.413482in}}%
\pgfpathlineto{\pgfqpoint{4.430876in}{1.423876in}}%
\pgfpathlineto{\pgfqpoint{4.438617in}{1.434343in}}%
\pgfpathlineto{\pgfqpoint{4.424898in}{1.435373in}}%
\pgfpathlineto{\pgfqpoint{4.411189in}{1.436516in}}%
\pgfpathlineto{\pgfqpoint{4.397488in}{1.437772in}}%
\pgfpathlineto{\pgfqpoint{4.383796in}{1.439140in}}%
\pgfpathlineto{\pgfqpoint{4.376048in}{1.429026in}}%
\pgfpathlineto{\pgfqpoint{4.368295in}{1.418990in}}%
\pgfpathlineto{\pgfqpoint{4.360537in}{1.409034in}}%
\pgfpathlineto{\pgfqpoint{4.352774in}{1.399161in}}%
\pgfpathclose%
\pgfusepath{fill}%
\end{pgfscope}%
\begin{pgfscope}%
\pgfpathrectangle{\pgfqpoint{1.254980in}{0.150000in}}{\pgfqpoint{5.490039in}{5.490039in}}%
\pgfusepath{clip}%
\pgfsetbuttcap%
\pgfsetroundjoin%
\definecolor{currentfill}{rgb}{0.283072,0.130895,0.449241}%
\pgfsetfillcolor{currentfill}%
\pgfsetfillopacity{0.700000}%
\pgfsetlinewidth{0.000000pt}%
\definecolor{currentstroke}{rgb}{0.000000,0.000000,0.000000}%
\pgfsetstrokecolor{currentstroke}%
\pgfsetdash{}{0pt}%
\pgfpathmoveto{\pgfqpoint{4.696331in}{1.572915in}}%
\pgfpathlineto{\pgfqpoint{4.710164in}{1.574326in}}%
\pgfpathlineto{\pgfqpoint{4.724008in}{1.575848in}}%
\pgfpathlineto{\pgfqpoint{4.737863in}{1.577481in}}%
\pgfpathlineto{\pgfqpoint{4.751729in}{1.579226in}}%
\pgfpathlineto{\pgfqpoint{4.759397in}{1.591621in}}%
\pgfpathlineto{\pgfqpoint{4.767061in}{1.604037in}}%
\pgfpathlineto{\pgfqpoint{4.774720in}{1.616473in}}%
\pgfpathlineto{\pgfqpoint{4.782376in}{1.628926in}}%
\pgfpathlineto{\pgfqpoint{4.768511in}{1.626883in}}%
\pgfpathlineto{\pgfqpoint{4.754658in}{1.624952in}}%
\pgfpathlineto{\pgfqpoint{4.740817in}{1.623132in}}%
\pgfpathlineto{\pgfqpoint{4.726986in}{1.621424in}}%
\pgfpathlineto{\pgfqpoint{4.719329in}{1.609263in}}%
\pgfpathlineto{\pgfqpoint{4.711667in}{1.597122in}}%
\pgfpathlineto{\pgfqpoint{4.704001in}{1.585006in}}%
\pgfpathlineto{\pgfqpoint{4.696331in}{1.572915in}}%
\pgfpathclose%
\pgfusepath{fill}%
\end{pgfscope}%
\begin{pgfscope}%
\pgfpathrectangle{\pgfqpoint{1.254980in}{0.150000in}}{\pgfqpoint{5.490039in}{5.490039in}}%
\pgfusepath{clip}%
\pgfsetbuttcap%
\pgfsetroundjoin%
\definecolor{currentfill}{rgb}{0.179019,0.433756,0.557430}%
\pgfsetfillcolor{currentfill}%
\pgfsetfillopacity{0.700000}%
\pgfsetlinewidth{0.000000pt}%
\definecolor{currentstroke}{rgb}{0.000000,0.000000,0.000000}%
\pgfsetstrokecolor{currentstroke}%
\pgfsetdash{}{0pt}%
\pgfpathmoveto{\pgfqpoint{2.714564in}{2.338763in}}%
\pgfpathlineto{\pgfqpoint{2.728275in}{2.320186in}}%
\pgfpathlineto{\pgfqpoint{2.741980in}{2.301775in}}%
\pgfpathlineto{\pgfqpoint{2.755679in}{2.283530in}}%
\pgfpathlineto{\pgfqpoint{2.769373in}{2.265450in}}%
\pgfpathlineto{\pgfqpoint{2.778098in}{2.259352in}}%
\pgfpathlineto{\pgfqpoint{2.786803in}{2.253552in}}%
\pgfpathlineto{\pgfqpoint{2.795490in}{2.248043in}}%
\pgfpathlineto{\pgfqpoint{2.804159in}{2.242822in}}%
\pgfpathlineto{\pgfqpoint{2.790513in}{2.260391in}}%
\pgfpathlineto{\pgfqpoint{2.776863in}{2.278123in}}%
\pgfpathlineto{\pgfqpoint{2.763208in}{2.296020in}}%
\pgfpathlineto{\pgfqpoint{2.749546in}{2.314082in}}%
\pgfpathlineto{\pgfqpoint{2.740829in}{2.319809in}}%
\pgfpathlineto{\pgfqpoint{2.732093in}{2.325828in}}%
\pgfpathlineto{\pgfqpoint{2.723338in}{2.332145in}}%
\pgfpathlineto{\pgfqpoint{2.714564in}{2.338763in}}%
\pgfpathclose%
\pgfusepath{fill}%
\end{pgfscope}%
\begin{pgfscope}%
\pgfpathrectangle{\pgfqpoint{1.254980in}{0.150000in}}{\pgfqpoint{5.490039in}{5.490039in}}%
\pgfusepath{clip}%
\pgfsetbuttcap%
\pgfsetroundjoin%
\definecolor{currentfill}{rgb}{0.281924,0.089666,0.412415}%
\pgfsetfillcolor{currentfill}%
\pgfsetfillopacity{0.700000}%
\pgfsetlinewidth{0.000000pt}%
\definecolor{currentstroke}{rgb}{0.000000,0.000000,0.000000}%
\pgfsetstrokecolor{currentstroke}%
\pgfsetdash{}{0pt}%
\pgfpathmoveto{\pgfqpoint{3.510015in}{1.535453in}}%
\pgfpathlineto{\pgfqpoint{3.523579in}{1.525501in}}%
\pgfpathlineto{\pgfqpoint{3.537145in}{1.515677in}}%
\pgfpathlineto{\pgfqpoint{3.550712in}{1.505981in}}%
\pgfpathlineto{\pgfqpoint{3.564282in}{1.496411in}}%
\pgfpathlineto{\pgfqpoint{3.572409in}{1.498209in}}%
\pgfpathlineto{\pgfqpoint{3.580527in}{1.500218in}}%
\pgfpathlineto{\pgfqpoint{3.588634in}{1.502434in}}%
\pgfpathlineto{\pgfqpoint{3.596731in}{1.504854in}}%
\pgfpathlineto{\pgfqpoint{3.583189in}{1.513962in}}%
\pgfpathlineto{\pgfqpoint{3.569649in}{1.523196in}}%
\pgfpathlineto{\pgfqpoint{3.556112in}{1.532558in}}%
\pgfpathlineto{\pgfqpoint{3.542577in}{1.542047in}}%
\pgfpathlineto{\pgfqpoint{3.534452in}{1.540083in}}%
\pgfpathlineto{\pgfqpoint{3.526317in}{1.538326in}}%
\pgfpathlineto{\pgfqpoint{3.518172in}{1.536782in}}%
\pgfpathlineto{\pgfqpoint{3.510015in}{1.535453in}}%
\pgfpathclose%
\pgfusepath{fill}%
\end{pgfscope}%
\begin{pgfscope}%
\pgfpathrectangle{\pgfqpoint{1.254980in}{0.150000in}}{\pgfqpoint{5.490039in}{5.490039in}}%
\pgfusepath{clip}%
\pgfsetbuttcap%
\pgfsetroundjoin%
\definecolor{currentfill}{rgb}{0.267968,0.223549,0.512008}%
\pgfsetfillcolor{currentfill}%
\pgfsetfillopacity{0.700000}%
\pgfsetlinewidth{0.000000pt}%
\definecolor{currentstroke}{rgb}{0.000000,0.000000,0.000000}%
\pgfsetstrokecolor{currentstroke}%
\pgfsetdash{}{0pt}%
\pgfpathmoveto{\pgfqpoint{3.151142in}{1.822562in}}%
\pgfpathlineto{\pgfqpoint{3.164737in}{1.808896in}}%
\pgfpathlineto{\pgfqpoint{3.178330in}{1.795370in}}%
\pgfpathlineto{\pgfqpoint{3.191923in}{1.781986in}}%
\pgfpathlineto{\pgfqpoint{3.205514in}{1.768742in}}%
\pgfpathlineto{\pgfqpoint{3.213888in}{1.766704in}}%
\pgfpathlineto{\pgfqpoint{3.222249in}{1.764922in}}%
\pgfpathlineto{\pgfqpoint{3.230595in}{1.763392in}}%
\pgfpathlineto{\pgfqpoint{3.238928in}{1.762109in}}%
\pgfpathlineto{\pgfqpoint{3.225375in}{1.774864in}}%
\pgfpathlineto{\pgfqpoint{3.211820in}{1.787757in}}%
\pgfpathlineto{\pgfqpoint{3.198264in}{1.800791in}}%
\pgfpathlineto{\pgfqpoint{3.184708in}{1.813966in}}%
\pgfpathlineto{\pgfqpoint{3.176338in}{1.815732in}}%
\pgfpathlineto{\pgfqpoint{3.167953in}{1.817751in}}%
\pgfpathlineto{\pgfqpoint{3.159555in}{1.820026in}}%
\pgfpathlineto{\pgfqpoint{3.151142in}{1.822562in}}%
\pgfpathclose%
\pgfusepath{fill}%
\end{pgfscope}%
\begin{pgfscope}%
\pgfpathrectangle{\pgfqpoint{1.254980in}{0.150000in}}{\pgfqpoint{5.490039in}{5.490039in}}%
\pgfusepath{clip}%
\pgfsetbuttcap%
\pgfsetroundjoin%
\definecolor{currentfill}{rgb}{0.268510,0.009605,0.335427}%
\pgfsetfillcolor{currentfill}%
\pgfsetfillopacity{0.700000}%
\pgfsetlinewidth{0.000000pt}%
\definecolor{currentstroke}{rgb}{0.000000,0.000000,0.000000}%
\pgfsetstrokecolor{currentstroke}%
\pgfsetdash{}{0pt}%
\pgfpathmoveto{\pgfqpoint{3.900045in}{1.369184in}}%
\pgfpathlineto{\pgfqpoint{3.913639in}{1.363089in}}%
\pgfpathlineto{\pgfqpoint{3.927239in}{1.357112in}}%
\pgfpathlineto{\pgfqpoint{3.940844in}{1.351253in}}%
\pgfpathlineto{\pgfqpoint{3.954453in}{1.345512in}}%
\pgfpathlineto{\pgfqpoint{3.962374in}{1.351480in}}%
\pgfpathlineto{\pgfqpoint{3.970288in}{1.357602in}}%
\pgfpathlineto{\pgfqpoint{3.978195in}{1.363877in}}%
\pgfpathlineto{\pgfqpoint{3.986095in}{1.370300in}}%
\pgfpathlineto{\pgfqpoint{3.972502in}{1.375618in}}%
\pgfpathlineto{\pgfqpoint{3.958916in}{1.381053in}}%
\pgfpathlineto{\pgfqpoint{3.945334in}{1.386606in}}%
\pgfpathlineto{\pgfqpoint{3.931758in}{1.392278in}}%
\pgfpathlineto{\pgfqpoint{3.923840in}{1.386272in}}%
\pgfpathlineto{\pgfqpoint{3.915916in}{1.380419in}}%
\pgfpathlineto{\pgfqpoint{3.907984in}{1.374722in}}%
\pgfpathlineto{\pgfqpoint{3.900045in}{1.369184in}}%
\pgfpathclose%
\pgfusepath{fill}%
\end{pgfscope}%
\begin{pgfscope}%
\pgfpathrectangle{\pgfqpoint{1.254980in}{0.150000in}}{\pgfqpoint{5.490039in}{5.490039in}}%
\pgfusepath{clip}%
\pgfsetbuttcap%
\pgfsetroundjoin%
\definecolor{currentfill}{rgb}{0.166617,0.463708,0.558119}%
\pgfsetfillcolor{currentfill}%
\pgfsetfillopacity{0.700000}%
\pgfsetlinewidth{0.000000pt}%
\definecolor{currentstroke}{rgb}{0.000000,0.000000,0.000000}%
\pgfsetstrokecolor{currentstroke}%
\pgfsetdash{}{0pt}%
\pgfpathmoveto{\pgfqpoint{2.659658in}{2.414757in}}%
\pgfpathlineto{\pgfqpoint{2.673394in}{2.395504in}}%
\pgfpathlineto{\pgfqpoint{2.687123in}{2.376421in}}%
\pgfpathlineto{\pgfqpoint{2.700847in}{2.357508in}}%
\pgfpathlineto{\pgfqpoint{2.714564in}{2.338763in}}%
\pgfpathlineto{\pgfqpoint{2.723338in}{2.332145in}}%
\pgfpathlineto{\pgfqpoint{2.732093in}{2.325828in}}%
\pgfpathlineto{\pgfqpoint{2.740829in}{2.319809in}}%
\pgfpathlineto{\pgfqpoint{2.749546in}{2.314082in}}%
\pgfpathlineto{\pgfqpoint{2.735880in}{2.332312in}}%
\pgfpathlineto{\pgfqpoint{2.722207in}{2.350708in}}%
\pgfpathlineto{\pgfqpoint{2.708529in}{2.369273in}}%
\pgfpathlineto{\pgfqpoint{2.694844in}{2.388008in}}%
\pgfpathlineto{\pgfqpoint{2.686077in}{2.394243in}}%
\pgfpathlineto{\pgfqpoint{2.677291in}{2.400777in}}%
\pgfpathlineto{\pgfqpoint{2.668484in}{2.407613in}}%
\pgfpathlineto{\pgfqpoint{2.659658in}{2.414757in}}%
\pgfpathclose%
\pgfusepath{fill}%
\end{pgfscope}%
\begin{pgfscope}%
\pgfpathrectangle{\pgfqpoint{1.254980in}{0.150000in}}{\pgfqpoint{5.490039in}{5.490039in}}%
\pgfusepath{clip}%
\pgfsetbuttcap%
\pgfsetroundjoin%
\definecolor{currentfill}{rgb}{0.280255,0.165693,0.476498}%
\pgfsetfillcolor{currentfill}%
\pgfsetfillopacity{0.700000}%
\pgfsetlinewidth{0.000000pt}%
\definecolor{currentstroke}{rgb}{0.000000,0.000000,0.000000}%
\pgfsetstrokecolor{currentstroke}%
\pgfsetdash{}{0pt}%
\pgfpathmoveto{\pgfqpoint{4.782376in}{1.628926in}}%
\pgfpathlineto{\pgfqpoint{4.796252in}{1.631080in}}%
\pgfpathlineto{\pgfqpoint{4.810139in}{1.633346in}}%
\pgfpathlineto{\pgfqpoint{4.824038in}{1.635723in}}%
\pgfpathlineto{\pgfqpoint{4.837948in}{1.638211in}}%
\pgfpathlineto{\pgfqpoint{4.845598in}{1.650967in}}%
\pgfpathlineto{\pgfqpoint{4.853244in}{1.663733in}}%
\pgfpathlineto{\pgfqpoint{4.860886in}{1.676505in}}%
\pgfpathlineto{\pgfqpoint{4.868523in}{1.689281in}}%
\pgfpathlineto{\pgfqpoint{4.854614in}{1.686510in}}%
\pgfpathlineto{\pgfqpoint{4.840716in}{1.683851in}}%
\pgfpathlineto{\pgfqpoint{4.826829in}{1.681303in}}%
\pgfpathlineto{\pgfqpoint{4.812955in}{1.678866in}}%
\pgfpathlineto{\pgfqpoint{4.805317in}{1.666367in}}%
\pgfpathlineto{\pgfqpoint{4.797674in}{1.653875in}}%
\pgfpathlineto{\pgfqpoint{4.790027in}{1.641394in}}%
\pgfpathlineto{\pgfqpoint{4.782376in}{1.628926in}}%
\pgfpathclose%
\pgfusepath{fill}%
\end{pgfscope}%
\begin{pgfscope}%
\pgfpathrectangle{\pgfqpoint{1.254980in}{0.150000in}}{\pgfqpoint{5.490039in}{5.490039in}}%
\pgfusepath{clip}%
\pgfsetbuttcap%
\pgfsetroundjoin%
\definecolor{currentfill}{rgb}{0.269944,0.014625,0.341379}%
\pgfsetfillcolor{currentfill}%
\pgfsetfillopacity{0.700000}%
\pgfsetlinewidth{0.000000pt}%
\definecolor{currentstroke}{rgb}{0.000000,0.000000,0.000000}%
\pgfsetstrokecolor{currentstroke}%
\pgfsetdash{}{0pt}%
\pgfpathmoveto{\pgfqpoint{4.266919in}{1.370107in}}%
\pgfpathlineto{\pgfqpoint{4.280596in}{1.367551in}}%
\pgfpathlineto{\pgfqpoint{4.294281in}{1.365109in}}%
\pgfpathlineto{\pgfqpoint{4.307974in}{1.362780in}}%
\pgfpathlineto{\pgfqpoint{4.321675in}{1.360564in}}%
\pgfpathlineto{\pgfqpoint{4.329457in}{1.370073in}}%
\pgfpathlineto{\pgfqpoint{4.337234in}{1.379678in}}%
\pgfpathlineto{\pgfqpoint{4.345007in}{1.389375in}}%
\pgfpathlineto{\pgfqpoint{4.352774in}{1.399161in}}%
\pgfpathlineto{\pgfqpoint{4.339083in}{1.401002in}}%
\pgfpathlineto{\pgfqpoint{4.325399in}{1.402956in}}%
\pgfpathlineto{\pgfqpoint{4.311723in}{1.405023in}}%
\pgfpathlineto{\pgfqpoint{4.298056in}{1.407204in}}%
\pgfpathlineto{\pgfqpoint{4.290279in}{1.397787in}}%
\pgfpathlineto{\pgfqpoint{4.282498in}{1.388463in}}%
\pgfpathlineto{\pgfqpoint{4.274711in}{1.379235in}}%
\pgfpathlineto{\pgfqpoint{4.266919in}{1.370107in}}%
\pgfpathclose%
\pgfusepath{fill}%
\end{pgfscope}%
\begin{pgfscope}%
\pgfpathrectangle{\pgfqpoint{1.254980in}{0.150000in}}{\pgfqpoint{5.490039in}{5.490039in}}%
\pgfusepath{clip}%
\pgfsetbuttcap%
\pgfsetroundjoin%
\definecolor{currentfill}{rgb}{0.267004,0.004874,0.329415}%
\pgfsetfillcolor{currentfill}%
\pgfsetfillopacity{0.700000}%
\pgfsetlinewidth{0.000000pt}%
\definecolor{currentstroke}{rgb}{0.000000,0.000000,0.000000}%
\pgfsetstrokecolor{currentstroke}%
\pgfsetdash{}{0pt}%
\pgfpathmoveto{\pgfqpoint{4.040520in}{1.350204in}}%
\pgfpathlineto{\pgfqpoint{4.054141in}{1.345471in}}%
\pgfpathlineto{\pgfqpoint{4.067767in}{1.340855in}}%
\pgfpathlineto{\pgfqpoint{4.081400in}{1.336355in}}%
\pgfpathlineto{\pgfqpoint{4.095039in}{1.331971in}}%
\pgfpathlineto{\pgfqpoint{4.102902in}{1.339365in}}%
\pgfpathlineto{\pgfqpoint{4.110758in}{1.346892in}}%
\pgfpathlineto{\pgfqpoint{4.118607in}{1.354549in}}%
\pgfpathlineto{\pgfqpoint{4.126451in}{1.362332in}}%
\pgfpathlineto{\pgfqpoint{4.112826in}{1.366310in}}%
\pgfpathlineto{\pgfqpoint{4.099208in}{1.370403in}}%
\pgfpathlineto{\pgfqpoint{4.085595in}{1.374612in}}%
\pgfpathlineto{\pgfqpoint{4.071989in}{1.378938in}}%
\pgfpathlineto{\pgfqpoint{4.064132in}{1.371555in}}%
\pgfpathlineto{\pgfqpoint{4.056267in}{1.364303in}}%
\pgfpathlineto{\pgfqpoint{4.048397in}{1.357185in}}%
\pgfpathlineto{\pgfqpoint{4.040520in}{1.350204in}}%
\pgfpathclose%
\pgfusepath{fill}%
\end{pgfscope}%
\begin{pgfscope}%
\pgfpathrectangle{\pgfqpoint{1.254980in}{0.150000in}}{\pgfqpoint{5.490039in}{5.490039in}}%
\pgfusepath{clip}%
\pgfsetbuttcap%
\pgfsetroundjoin%
\definecolor{currentfill}{rgb}{0.246811,0.283237,0.535941}%
\pgfsetfillcolor{currentfill}%
\pgfsetfillopacity{0.700000}%
\pgfsetlinewidth{0.000000pt}%
\definecolor{currentstroke}{rgb}{0.000000,0.000000,0.000000}%
\pgfsetstrokecolor{currentstroke}%
\pgfsetdash{}{0pt}%
\pgfpathmoveto{\pgfqpoint{5.071556in}{1.874423in}}%
\pgfpathlineto{\pgfqpoint{5.085581in}{1.878951in}}%
\pgfpathlineto{\pgfqpoint{5.099619in}{1.883590in}}%
\pgfpathlineto{\pgfqpoint{5.113670in}{1.888342in}}%
\pgfpathlineto{\pgfqpoint{5.127736in}{1.893204in}}%
\pgfpathlineto{\pgfqpoint{5.135317in}{1.906537in}}%
\pgfpathlineto{\pgfqpoint{5.142893in}{1.919836in}}%
\pgfpathlineto{\pgfqpoint{5.150464in}{1.933099in}}%
\pgfpathlineto{\pgfqpoint{5.158030in}{1.946325in}}%
\pgfpathlineto{\pgfqpoint{5.143964in}{1.941242in}}%
\pgfpathlineto{\pgfqpoint{5.129911in}{1.936271in}}%
\pgfpathlineto{\pgfqpoint{5.115872in}{1.931411in}}%
\pgfpathlineto{\pgfqpoint{5.101847in}{1.926663in}}%
\pgfpathlineto{\pgfqpoint{5.094281in}{1.913651in}}%
\pgfpathlineto{\pgfqpoint{5.086711in}{1.900606in}}%
\pgfpathlineto{\pgfqpoint{5.079136in}{1.887529in}}%
\pgfpathlineto{\pgfqpoint{5.071556in}{1.874423in}}%
\pgfpathclose%
\pgfusepath{fill}%
\end{pgfscope}%
\begin{pgfscope}%
\pgfpathrectangle{\pgfqpoint{1.254980in}{0.150000in}}{\pgfqpoint{5.490039in}{5.490039in}}%
\pgfusepath{clip}%
\pgfsetbuttcap%
\pgfsetroundjoin%
\definecolor{currentfill}{rgb}{0.273006,0.204520,0.501721}%
\pgfsetfillcolor{currentfill}%
\pgfsetfillopacity{0.700000}%
\pgfsetlinewidth{0.000000pt}%
\definecolor{currentstroke}{rgb}{0.000000,0.000000,0.000000}%
\pgfsetstrokecolor{currentstroke}%
\pgfsetdash{}{0pt}%
\pgfpathmoveto{\pgfqpoint{3.205514in}{1.768742in}}%
\pgfpathlineto{\pgfqpoint{3.219104in}{1.755637in}}%
\pgfpathlineto{\pgfqpoint{3.232693in}{1.742671in}}%
\pgfpathlineto{\pgfqpoint{3.246282in}{1.729843in}}%
\pgfpathlineto{\pgfqpoint{3.259869in}{1.717153in}}%
\pgfpathlineto{\pgfqpoint{3.268207in}{1.715611in}}%
\pgfpathlineto{\pgfqpoint{3.276530in}{1.714320in}}%
\pgfpathlineto{\pgfqpoint{3.284841in}{1.713277in}}%
\pgfpathlineto{\pgfqpoint{3.293138in}{1.712476in}}%
\pgfpathlineto{\pgfqpoint{3.279586in}{1.724678in}}%
\pgfpathlineto{\pgfqpoint{3.266034in}{1.737017in}}%
\pgfpathlineto{\pgfqpoint{3.252482in}{1.749494in}}%
\pgfpathlineto{\pgfqpoint{3.238928in}{1.762109in}}%
\pgfpathlineto{\pgfqpoint{3.230595in}{1.763392in}}%
\pgfpathlineto{\pgfqpoint{3.222249in}{1.764922in}}%
\pgfpathlineto{\pgfqpoint{3.213888in}{1.766704in}}%
\pgfpathlineto{\pgfqpoint{3.205514in}{1.768742in}}%
\pgfpathclose%
\pgfusepath{fill}%
\end{pgfscope}%
\begin{pgfscope}%
\pgfpathrectangle{\pgfqpoint{1.254980in}{0.150000in}}{\pgfqpoint{5.490039in}{5.490039in}}%
\pgfusepath{clip}%
\pgfsetbuttcap%
\pgfsetroundjoin%
\definecolor{currentfill}{rgb}{0.208623,0.367752,0.552675}%
\pgfsetfillcolor{currentfill}%
\pgfsetfillopacity{0.700000}%
\pgfsetlinewidth{0.000000pt}%
\definecolor{currentstroke}{rgb}{0.000000,0.000000,0.000000}%
\pgfsetstrokecolor{currentstroke}%
\pgfsetdash{}{0pt}%
\pgfpathmoveto{\pgfqpoint{5.274797in}{2.073638in}}%
\pgfpathlineto{\pgfqpoint{5.288938in}{2.079673in}}%
\pgfpathlineto{\pgfqpoint{5.303093in}{2.085820in}}%
\pgfpathlineto{\pgfqpoint{5.317263in}{2.092079in}}%
\pgfpathlineto{\pgfqpoint{5.331448in}{2.098450in}}%
\pgfpathlineto{\pgfqpoint{5.338970in}{2.111633in}}%
\pgfpathlineto{\pgfqpoint{5.346487in}{2.124756in}}%
\pgfpathlineto{\pgfqpoint{5.353999in}{2.137818in}}%
\pgfpathlineto{\pgfqpoint{5.361505in}{2.150819in}}%
\pgfpathlineto{\pgfqpoint{5.347319in}{2.144275in}}%
\pgfpathlineto{\pgfqpoint{5.333148in}{2.137844in}}%
\pgfpathlineto{\pgfqpoint{5.318992in}{2.131526in}}%
\pgfpathlineto{\pgfqpoint{5.304851in}{2.125319in}}%
\pgfpathlineto{\pgfqpoint{5.297345in}{2.112484in}}%
\pgfpathlineto{\pgfqpoint{5.289835in}{2.099591in}}%
\pgfpathlineto{\pgfqpoint{5.282319in}{2.086642in}}%
\pgfpathlineto{\pgfqpoint{5.274797in}{2.073638in}}%
\pgfpathclose%
\pgfusepath{fill}%
\end{pgfscope}%
\begin{pgfscope}%
\pgfpathrectangle{\pgfqpoint{1.254980in}{0.150000in}}{\pgfqpoint{5.490039in}{5.490039in}}%
\pgfusepath{clip}%
\pgfsetbuttcap%
\pgfsetroundjoin%
\definecolor{currentfill}{rgb}{0.360741,0.785964,0.387814}%
\pgfsetfillcolor{currentfill}%
\pgfsetfillopacity{0.700000}%
\pgfsetlinewidth{0.000000pt}%
\definecolor{currentstroke}{rgb}{0.000000,0.000000,0.000000}%
\pgfsetstrokecolor{currentstroke}%
\pgfsetdash{}{0pt}%
\pgfpathmoveto{\pgfqpoint{2.121515in}{3.352597in}}%
\pgfpathlineto{\pgfqpoint{2.135577in}{3.325759in}}%
\pgfpathlineto{\pgfqpoint{2.149624in}{3.299146in}}%
\pgfpathlineto{\pgfqpoint{2.163658in}{3.272759in}}%
\pgfpathlineto{\pgfqpoint{2.177678in}{3.246594in}}%
\pgfpathlineto{\pgfqpoint{2.186907in}{3.236730in}}%
\pgfpathlineto{\pgfqpoint{2.196111in}{3.227194in}}%
\pgfpathlineto{\pgfqpoint{2.205291in}{3.217980in}}%
\pgfpathlineto{\pgfqpoint{2.214446in}{3.209084in}}%
\pgfpathlineto{\pgfqpoint{2.200489in}{3.234727in}}%
\pgfpathlineto{\pgfqpoint{2.186519in}{3.260591in}}%
\pgfpathlineto{\pgfqpoint{2.172535in}{3.286678in}}%
\pgfpathlineto{\pgfqpoint{2.158538in}{3.312991in}}%
\pgfpathlineto{\pgfqpoint{2.149320in}{3.322402in}}%
\pgfpathlineto{\pgfqpoint{2.140077in}{3.332137in}}%
\pgfpathlineto{\pgfqpoint{2.130809in}{3.342200in}}%
\pgfpathlineto{\pgfqpoint{2.121515in}{3.352597in}}%
\pgfpathclose%
\pgfusepath{fill}%
\end{pgfscope}%
\begin{pgfscope}%
\pgfpathrectangle{\pgfqpoint{1.254980in}{0.150000in}}{\pgfqpoint{5.490039in}{5.490039in}}%
\pgfusepath{clip}%
\pgfsetbuttcap%
\pgfsetroundjoin%
\definecolor{currentfill}{rgb}{0.140210,0.665859,0.513427}%
\pgfsetfillcolor{currentfill}%
\pgfsetfillopacity{0.700000}%
\pgfsetlinewidth{0.000000pt}%
\definecolor{currentstroke}{rgb}{0.000000,0.000000,0.000000}%
\pgfsetstrokecolor{currentstroke}%
\pgfsetdash{}{0pt}%
\pgfpathmoveto{\pgfqpoint{2.308532in}{2.985817in}}%
\pgfpathlineto{\pgfqpoint{2.322456in}{2.961880in}}%
\pgfpathlineto{\pgfqpoint{2.336369in}{2.938146in}}%
\pgfpathlineto{\pgfqpoint{2.350271in}{2.914614in}}%
\pgfpathlineto{\pgfqpoint{2.364162in}{2.891282in}}%
\pgfpathlineto{\pgfqpoint{2.373239in}{2.882233in}}%
\pgfpathlineto{\pgfqpoint{2.382293in}{2.873505in}}%
\pgfpathlineto{\pgfqpoint{2.391325in}{2.865096in}}%
\pgfpathlineto{\pgfqpoint{2.400334in}{2.857000in}}%
\pgfpathlineto{\pgfqpoint{2.386501in}{2.879808in}}%
\pgfpathlineto{\pgfqpoint{2.372659in}{2.902815in}}%
\pgfpathlineto{\pgfqpoint{2.358806in}{2.926023in}}%
\pgfpathlineto{\pgfqpoint{2.344942in}{2.949432in}}%
\pgfpathlineto{\pgfqpoint{2.335875in}{2.958045in}}%
\pgfpathlineto{\pgfqpoint{2.326784in}{2.966977in}}%
\pgfpathlineto{\pgfqpoint{2.317670in}{2.976233in}}%
\pgfpathlineto{\pgfqpoint{2.308532in}{2.985817in}}%
\pgfpathclose%
\pgfusepath{fill}%
\end{pgfscope}%
\begin{pgfscope}%
\pgfpathrectangle{\pgfqpoint{1.254980in}{0.150000in}}{\pgfqpoint{5.490039in}{5.490039in}}%
\pgfusepath{clip}%
\pgfsetbuttcap%
\pgfsetroundjoin%
\definecolor{currentfill}{rgb}{0.272594,0.025563,0.353093}%
\pgfsetfillcolor{currentfill}%
\pgfsetfillopacity{0.700000}%
\pgfsetlinewidth{0.000000pt}%
\definecolor{currentstroke}{rgb}{0.000000,0.000000,0.000000}%
\pgfsetstrokecolor{currentstroke}%
\pgfsetdash{}{0pt}%
\pgfpathmoveto{\pgfqpoint{3.759449in}{1.405300in}}%
\pgfpathlineto{\pgfqpoint{3.773030in}{1.397803in}}%
\pgfpathlineto{\pgfqpoint{3.786616in}{1.390427in}}%
\pgfpathlineto{\pgfqpoint{3.800204in}{1.383173in}}%
\pgfpathlineto{\pgfqpoint{3.813797in}{1.376039in}}%
\pgfpathlineto{\pgfqpoint{3.821788in}{1.380455in}}%
\pgfpathlineto{\pgfqpoint{3.829771in}{1.385049in}}%
\pgfpathlineto{\pgfqpoint{3.837746in}{1.389818in}}%
\pgfpathlineto{\pgfqpoint{3.845713in}{1.394758in}}%
\pgfpathlineto{\pgfqpoint{3.832141in}{1.401450in}}%
\pgfpathlineto{\pgfqpoint{3.818574in}{1.408264in}}%
\pgfpathlineto{\pgfqpoint{3.805010in}{1.415198in}}%
\pgfpathlineto{\pgfqpoint{3.791451in}{1.422253in}}%
\pgfpathlineto{\pgfqpoint{3.783463in}{1.417748in}}%
\pgfpathlineto{\pgfqpoint{3.775467in}{1.413418in}}%
\pgfpathlineto{\pgfqpoint{3.767462in}{1.409267in}}%
\pgfpathlineto{\pgfqpoint{3.759449in}{1.405300in}}%
\pgfpathclose%
\pgfusepath{fill}%
\end{pgfscope}%
\begin{pgfscope}%
\pgfpathrectangle{\pgfqpoint{1.254980in}{0.150000in}}{\pgfqpoint{5.490039in}{5.490039in}}%
\pgfusepath{clip}%
\pgfsetbuttcap%
\pgfsetroundjoin%
\definecolor{currentfill}{rgb}{0.154815,0.493313,0.557840}%
\pgfsetfillcolor{currentfill}%
\pgfsetfillopacity{0.700000}%
\pgfsetlinewidth{0.000000pt}%
\definecolor{currentstroke}{rgb}{0.000000,0.000000,0.000000}%
\pgfsetstrokecolor{currentstroke}%
\pgfsetdash{}{0pt}%
\pgfpathmoveto{\pgfqpoint{2.604646in}{2.493494in}}%
\pgfpathlineto{\pgfqpoint{2.618409in}{2.473549in}}%
\pgfpathlineto{\pgfqpoint{2.632166in}{2.453779in}}%
\pgfpathlineto{\pgfqpoint{2.645915in}{2.434182in}}%
\pgfpathlineto{\pgfqpoint{2.659658in}{2.414757in}}%
\pgfpathlineto{\pgfqpoint{2.668484in}{2.407613in}}%
\pgfpathlineto{\pgfqpoint{2.677291in}{2.400777in}}%
\pgfpathlineto{\pgfqpoint{2.686077in}{2.394243in}}%
\pgfpathlineto{\pgfqpoint{2.694844in}{2.388008in}}%
\pgfpathlineto{\pgfqpoint{2.681154in}{2.406913in}}%
\pgfpathlineto{\pgfqpoint{2.667457in}{2.425989in}}%
\pgfpathlineto{\pgfqpoint{2.653753in}{2.445238in}}%
\pgfpathlineto{\pgfqpoint{2.640043in}{2.464661in}}%
\pgfpathlineto{\pgfqpoint{2.631224in}{2.471409in}}%
\pgfpathlineto{\pgfqpoint{2.622385in}{2.478461in}}%
\pgfpathlineto{\pgfqpoint{2.613526in}{2.485821in}}%
\pgfpathlineto{\pgfqpoint{2.604646in}{2.493494in}}%
\pgfpathclose%
\pgfusepath{fill}%
\end{pgfscope}%
\begin{pgfscope}%
\pgfpathrectangle{\pgfqpoint{1.254980in}{0.150000in}}{\pgfqpoint{5.490039in}{5.490039in}}%
\pgfusepath{clip}%
\pgfsetbuttcap%
\pgfsetroundjoin%
\definecolor{currentfill}{rgb}{0.280894,0.078907,0.402329}%
\pgfsetfillcolor{currentfill}%
\pgfsetfillopacity{0.700000}%
\pgfsetlinewidth{0.000000pt}%
\definecolor{currentstroke}{rgb}{0.000000,0.000000,0.000000}%
\pgfsetstrokecolor{currentstroke}%
\pgfsetdash{}{0pt}%
\pgfpathmoveto{\pgfqpoint{3.564282in}{1.496411in}}%
\pgfpathlineto{\pgfqpoint{3.577853in}{1.486968in}}%
\pgfpathlineto{\pgfqpoint{3.591427in}{1.477652in}}%
\pgfpathlineto{\pgfqpoint{3.605003in}{1.468461in}}%
\pgfpathlineto{\pgfqpoint{3.618581in}{1.459395in}}%
\pgfpathlineto{\pgfqpoint{3.626681in}{1.461660in}}%
\pgfpathlineto{\pgfqpoint{3.634772in}{1.464133in}}%
\pgfpathlineto{\pgfqpoint{3.642852in}{1.466808in}}%
\pgfpathlineto{\pgfqpoint{3.650923in}{1.469683in}}%
\pgfpathlineto{\pgfqpoint{3.637371in}{1.478288in}}%
\pgfpathlineto{\pgfqpoint{3.623821in}{1.487018in}}%
\pgfpathlineto{\pgfqpoint{3.610275in}{1.495873in}}%
\pgfpathlineto{\pgfqpoint{3.596731in}{1.504854in}}%
\pgfpathlineto{\pgfqpoint{3.588634in}{1.502434in}}%
\pgfpathlineto{\pgfqpoint{3.580527in}{1.500218in}}%
\pgfpathlineto{\pgfqpoint{3.572409in}{1.498209in}}%
\pgfpathlineto{\pgfqpoint{3.564282in}{1.496411in}}%
\pgfpathclose%
\pgfusepath{fill}%
\end{pgfscope}%
\begin{pgfscope}%
\pgfpathrectangle{\pgfqpoint{1.254980in}{0.150000in}}{\pgfqpoint{5.490039in}{5.490039in}}%
\pgfusepath{clip}%
\pgfsetbuttcap%
\pgfsetroundjoin%
\definecolor{currentfill}{rgb}{0.275191,0.194905,0.496005}%
\pgfsetfillcolor{currentfill}%
\pgfsetfillopacity{0.700000}%
\pgfsetlinewidth{0.000000pt}%
\definecolor{currentstroke}{rgb}{0.000000,0.000000,0.000000}%
\pgfsetstrokecolor{currentstroke}%
\pgfsetdash{}{0pt}%
\pgfpathmoveto{\pgfqpoint{4.868523in}{1.689281in}}%
\pgfpathlineto{\pgfqpoint{4.882445in}{1.692163in}}%
\pgfpathlineto{\pgfqpoint{4.896379in}{1.695157in}}%
\pgfpathlineto{\pgfqpoint{4.910326in}{1.698261in}}%
\pgfpathlineto{\pgfqpoint{4.924284in}{1.701477in}}%
\pgfpathlineto{\pgfqpoint{4.931917in}{1.714529in}}%
\pgfpathlineto{\pgfqpoint{4.939546in}{1.727577in}}%
\pgfpathlineto{\pgfqpoint{4.947170in}{1.740620in}}%
\pgfpathlineto{\pgfqpoint{4.954790in}{1.753655in}}%
\pgfpathlineto{\pgfqpoint{4.940831in}{1.750172in}}%
\pgfpathlineto{\pgfqpoint{4.926885in}{1.746800in}}%
\pgfpathlineto{\pgfqpoint{4.912952in}{1.743539in}}%
\pgfpathlineto{\pgfqpoint{4.899030in}{1.740390in}}%
\pgfpathlineto{\pgfqpoint{4.891410in}{1.727616in}}%
\pgfpathlineto{\pgfqpoint{4.883785in}{1.714839in}}%
\pgfpathlineto{\pgfqpoint{4.876157in}{1.702060in}}%
\pgfpathlineto{\pgfqpoint{4.868523in}{1.689281in}}%
\pgfpathclose%
\pgfusepath{fill}%
\end{pgfscope}%
\begin{pgfscope}%
\pgfpathrectangle{\pgfqpoint{1.254980in}{0.150000in}}{\pgfqpoint{5.490039in}{5.490039in}}%
\pgfusepath{clip}%
\pgfsetbuttcap%
\pgfsetroundjoin%
\definecolor{currentfill}{rgb}{0.278012,0.180367,0.486697}%
\pgfsetfillcolor{currentfill}%
\pgfsetfillopacity{0.700000}%
\pgfsetlinewidth{0.000000pt}%
\definecolor{currentstroke}{rgb}{0.000000,0.000000,0.000000}%
\pgfsetstrokecolor{currentstroke}%
\pgfsetdash{}{0pt}%
\pgfpathmoveto{\pgfqpoint{3.259869in}{1.717153in}}%
\pgfpathlineto{\pgfqpoint{3.273457in}{1.704600in}}%
\pgfpathlineto{\pgfqpoint{3.287043in}{1.692184in}}%
\pgfpathlineto{\pgfqpoint{3.300630in}{1.679903in}}%
\pgfpathlineto{\pgfqpoint{3.314216in}{1.667758in}}%
\pgfpathlineto{\pgfqpoint{3.322517in}{1.666710in}}%
\pgfpathlineto{\pgfqpoint{3.330805in}{1.665909in}}%
\pgfpathlineto{\pgfqpoint{3.339081in}{1.665349in}}%
\pgfpathlineto{\pgfqpoint{3.347344in}{1.665028in}}%
\pgfpathlineto{\pgfqpoint{3.333792in}{1.676687in}}%
\pgfpathlineto{\pgfqpoint{3.320241in}{1.688481in}}%
\pgfpathlineto{\pgfqpoint{3.306689in}{1.700410in}}%
\pgfpathlineto{\pgfqpoint{3.293138in}{1.712476in}}%
\pgfpathlineto{\pgfqpoint{3.284841in}{1.713277in}}%
\pgfpathlineto{\pgfqpoint{3.276530in}{1.714320in}}%
\pgfpathlineto{\pgfqpoint{3.268207in}{1.715611in}}%
\pgfpathlineto{\pgfqpoint{3.259869in}{1.717153in}}%
\pgfpathclose%
\pgfusepath{fill}%
\end{pgfscope}%
\begin{pgfscope}%
\pgfpathrectangle{\pgfqpoint{1.254980in}{0.150000in}}{\pgfqpoint{5.490039in}{5.490039in}}%
\pgfusepath{clip}%
\pgfsetbuttcap%
\pgfsetroundjoin%
\definecolor{currentfill}{rgb}{0.267004,0.004874,0.329415}%
\pgfsetfillcolor{currentfill}%
\pgfsetfillopacity{0.700000}%
\pgfsetlinewidth{0.000000pt}%
\definecolor{currentstroke}{rgb}{0.000000,0.000000,0.000000}%
\pgfsetstrokecolor{currentstroke}%
\pgfsetdash{}{0pt}%
\pgfpathmoveto{\pgfqpoint{4.181018in}{1.347574in}}%
\pgfpathlineto{\pgfqpoint{4.194677in}{1.344171in}}%
\pgfpathlineto{\pgfqpoint{4.208343in}{1.340883in}}%
\pgfpathlineto{\pgfqpoint{4.222016in}{1.337709in}}%
\pgfpathlineto{\pgfqpoint{4.235697in}{1.334649in}}%
\pgfpathlineto{\pgfqpoint{4.243510in}{1.343349in}}%
\pgfpathlineto{\pgfqpoint{4.251319in}{1.352160in}}%
\pgfpathlineto{\pgfqpoint{4.259121in}{1.361081in}}%
\pgfpathlineto{\pgfqpoint{4.266919in}{1.370107in}}%
\pgfpathlineto{\pgfqpoint{4.253249in}{1.372776in}}%
\pgfpathlineto{\pgfqpoint{4.239587in}{1.375560in}}%
\pgfpathlineto{\pgfqpoint{4.225932in}{1.378457in}}%
\pgfpathlineto{\pgfqpoint{4.212285in}{1.381469in}}%
\pgfpathlineto{\pgfqpoint{4.204477in}{1.372828in}}%
\pgfpathlineto{\pgfqpoint{4.196663in}{1.364296in}}%
\pgfpathlineto{\pgfqpoint{4.188843in}{1.355877in}}%
\pgfpathlineto{\pgfqpoint{4.181018in}{1.347574in}}%
\pgfpathclose%
\pgfusepath{fill}%
\end{pgfscope}%
\begin{pgfscope}%
\pgfpathrectangle{\pgfqpoint{1.254980in}{0.150000in}}{\pgfqpoint{5.490039in}{5.490039in}}%
\pgfusepath{clip}%
\pgfsetbuttcap%
\pgfsetroundjoin%
\definecolor{currentfill}{rgb}{0.194100,0.399323,0.555565}%
\pgfsetfillcolor{currentfill}%
\pgfsetfillopacity{0.700000}%
\pgfsetlinewidth{0.000000pt}%
\definecolor{currentstroke}{rgb}{0.000000,0.000000,0.000000}%
\pgfsetstrokecolor{currentstroke}%
\pgfsetdash{}{0pt}%
\pgfpathmoveto{\pgfqpoint{5.361505in}{2.150819in}}%
\pgfpathlineto{\pgfqpoint{5.375706in}{2.157474in}}%
\pgfpathlineto{\pgfqpoint{5.389923in}{2.164242in}}%
\pgfpathlineto{\pgfqpoint{5.404155in}{2.171122in}}%
\pgfpathlineto{\pgfqpoint{5.411656in}{2.184182in}}%
\pgfpathlineto{\pgfqpoint{5.419152in}{2.197174in}}%
\pgfpathlineto{\pgfqpoint{5.426641in}{2.210099in}}%
\pgfpathlineto{\pgfqpoint{5.434125in}{2.222954in}}%
\pgfpathlineto{\pgfqpoint{5.419892in}{2.215918in}}%
\pgfpathlineto{\pgfqpoint{5.405674in}{2.208994in}}%
\pgfpathlineto{\pgfqpoint{5.391472in}{2.202183in}}%
\pgfpathlineto{\pgfqpoint{5.383989in}{2.189440in}}%
\pgfpathlineto{\pgfqpoint{5.376500in}{2.176631in}}%
\pgfpathlineto{\pgfqpoint{5.369005in}{2.163757in}}%
\pgfpathlineto{\pgfqpoint{5.361505in}{2.150819in}}%
\pgfpathclose%
\pgfusepath{fill}%
\end{pgfscope}%
\begin{pgfscope}%
\pgfpathrectangle{\pgfqpoint{1.254980in}{0.150000in}}{\pgfqpoint{5.490039in}{5.490039in}}%
\pgfusepath{clip}%
\pgfsetbuttcap%
\pgfsetroundjoin%
\definecolor{currentfill}{rgb}{0.143343,0.522773,0.556295}%
\pgfsetfillcolor{currentfill}%
\pgfsetfillopacity{0.700000}%
\pgfsetlinewidth{0.000000pt}%
\definecolor{currentstroke}{rgb}{0.000000,0.000000,0.000000}%
\pgfsetstrokecolor{currentstroke}%
\pgfsetdash{}{0pt}%
\pgfpathmoveto{\pgfqpoint{2.549517in}{2.575041in}}%
\pgfpathlineto{\pgfqpoint{2.563311in}{2.554387in}}%
\pgfpathlineto{\pgfqpoint{2.577097in}{2.533912in}}%
\pgfpathlineto{\pgfqpoint{2.590875in}{2.513615in}}%
\pgfpathlineto{\pgfqpoint{2.604646in}{2.493494in}}%
\pgfpathlineto{\pgfqpoint{2.613526in}{2.485821in}}%
\pgfpathlineto{\pgfqpoint{2.622385in}{2.478461in}}%
\pgfpathlineto{\pgfqpoint{2.631224in}{2.471409in}}%
\pgfpathlineto{\pgfqpoint{2.640043in}{2.464661in}}%
\pgfpathlineto{\pgfqpoint{2.626325in}{2.484258in}}%
\pgfpathlineto{\pgfqpoint{2.612601in}{2.504031in}}%
\pgfpathlineto{\pgfqpoint{2.598869in}{2.523981in}}%
\pgfpathlineto{\pgfqpoint{2.585130in}{2.544108in}}%
\pgfpathlineto{\pgfqpoint{2.576259in}{2.551373in}}%
\pgfpathlineto{\pgfqpoint{2.567366in}{2.558947in}}%
\pgfpathlineto{\pgfqpoint{2.558452in}{2.566835in}}%
\pgfpathlineto{\pgfqpoint{2.549517in}{2.575041in}}%
\pgfpathclose%
\pgfusepath{fill}%
\end{pgfscope}%
\begin{pgfscope}%
\pgfpathrectangle{\pgfqpoint{1.254980in}{0.150000in}}{\pgfqpoint{5.490039in}{5.490039in}}%
\pgfusepath{clip}%
\pgfsetbuttcap%
\pgfsetroundjoin%
\definecolor{currentfill}{rgb}{0.231674,0.318106,0.544834}%
\pgfsetfillcolor{currentfill}%
\pgfsetfillopacity{0.700000}%
\pgfsetlinewidth{0.000000pt}%
\definecolor{currentstroke}{rgb}{0.000000,0.000000,0.000000}%
\pgfsetstrokecolor{currentstroke}%
\pgfsetdash{}{0pt}%
\pgfpathmoveto{\pgfqpoint{5.158030in}{1.946325in}}%
\pgfpathlineto{\pgfqpoint{5.172111in}{1.951520in}}%
\pgfpathlineto{\pgfqpoint{5.186205in}{1.956827in}}%
\pgfpathlineto{\pgfqpoint{5.200314in}{1.962245in}}%
\pgfpathlineto{\pgfqpoint{5.214437in}{1.967775in}}%
\pgfpathlineto{\pgfqpoint{5.222000in}{1.981173in}}%
\pgfpathlineto{\pgfqpoint{5.229558in}{1.994527in}}%
\pgfpathlineto{\pgfqpoint{5.237111in}{2.007835in}}%
\pgfpathlineto{\pgfqpoint{5.244659in}{2.021095in}}%
\pgfpathlineto{\pgfqpoint{5.230534in}{2.015361in}}%
\pgfpathlineto{\pgfqpoint{5.216424in}{2.009738in}}%
\pgfpathlineto{\pgfqpoint{5.202328in}{2.004227in}}%
\pgfpathlineto{\pgfqpoint{5.188247in}{1.998828in}}%
\pgfpathlineto{\pgfqpoint{5.180700in}{1.985766in}}%
\pgfpathlineto{\pgfqpoint{5.173149in}{1.972660in}}%
\pgfpathlineto{\pgfqpoint{5.165592in}{1.959513in}}%
\pgfpathlineto{\pgfqpoint{5.158030in}{1.946325in}}%
\pgfpathclose%
\pgfusepath{fill}%
\end{pgfscope}%
\begin{pgfscope}%
\pgfpathrectangle{\pgfqpoint{1.254980in}{0.150000in}}{\pgfqpoint{5.490039in}{5.490039in}}%
\pgfusepath{clip}%
\pgfsetbuttcap%
\pgfsetroundjoin%
\definecolor{currentfill}{rgb}{0.280868,0.160771,0.472899}%
\pgfsetfillcolor{currentfill}%
\pgfsetfillopacity{0.700000}%
\pgfsetlinewidth{0.000000pt}%
\definecolor{currentstroke}{rgb}{0.000000,0.000000,0.000000}%
\pgfsetstrokecolor{currentstroke}%
\pgfsetdash{}{0pt}%
\pgfpathmoveto{\pgfqpoint{3.314216in}{1.667758in}}%
\pgfpathlineto{\pgfqpoint{3.327802in}{1.655748in}}%
\pgfpathlineto{\pgfqpoint{3.341388in}{1.643872in}}%
\pgfpathlineto{\pgfqpoint{3.354975in}{1.632130in}}%
\pgfpathlineto{\pgfqpoint{3.368561in}{1.620521in}}%
\pgfpathlineto{\pgfqpoint{3.376828in}{1.619965in}}%
\pgfpathlineto{\pgfqpoint{3.385082in}{1.619650in}}%
\pgfpathlineto{\pgfqpoint{3.393324in}{1.619574in}}%
\pgfpathlineto{\pgfqpoint{3.401554in}{1.619731in}}%
\pgfpathlineto{\pgfqpoint{3.388000in}{1.630855in}}%
\pgfpathlineto{\pgfqpoint{3.374448in}{1.642113in}}%
\pgfpathlineto{\pgfqpoint{3.360895in}{1.653504in}}%
\pgfpathlineto{\pgfqpoint{3.347344in}{1.665028in}}%
\pgfpathlineto{\pgfqpoint{3.339081in}{1.665349in}}%
\pgfpathlineto{\pgfqpoint{3.330805in}{1.665909in}}%
\pgfpathlineto{\pgfqpoint{3.322517in}{1.666710in}}%
\pgfpathlineto{\pgfqpoint{3.314216in}{1.667758in}}%
\pgfpathclose%
\pgfusepath{fill}%
\end{pgfscope}%
\begin{pgfscope}%
\pgfpathrectangle{\pgfqpoint{1.254980in}{0.150000in}}{\pgfqpoint{5.490039in}{5.490039in}}%
\pgfusepath{clip}%
\pgfsetbuttcap%
\pgfsetroundjoin%
\definecolor{currentfill}{rgb}{0.265145,0.232956,0.516599}%
\pgfsetfillcolor{currentfill}%
\pgfsetfillopacity{0.700000}%
\pgfsetlinewidth{0.000000pt}%
\definecolor{currentstroke}{rgb}{0.000000,0.000000,0.000000}%
\pgfsetstrokecolor{currentstroke}%
\pgfsetdash{}{0pt}%
\pgfpathmoveto{\pgfqpoint{4.954790in}{1.753655in}}%
\pgfpathlineto{\pgfqpoint{4.968761in}{1.757249in}}%
\pgfpathlineto{\pgfqpoint{4.982745in}{1.760955in}}%
\pgfpathlineto{\pgfqpoint{4.996742in}{1.764772in}}%
\pgfpathlineto{\pgfqpoint{5.010752in}{1.768700in}}%
\pgfpathlineto{\pgfqpoint{5.018368in}{1.781984in}}%
\pgfpathlineto{\pgfqpoint{5.025980in}{1.795251in}}%
\pgfpathlineto{\pgfqpoint{5.033587in}{1.808501in}}%
\pgfpathlineto{\pgfqpoint{5.041190in}{1.821731in}}%
\pgfpathlineto{\pgfqpoint{5.027180in}{1.817551in}}%
\pgfpathlineto{\pgfqpoint{5.013182in}{1.813482in}}%
\pgfpathlineto{\pgfqpoint{4.999198in}{1.809524in}}%
\pgfpathlineto{\pgfqpoint{4.985226in}{1.805678in}}%
\pgfpathlineto{\pgfqpoint{4.977624in}{1.792693in}}%
\pgfpathlineto{\pgfqpoint{4.970017in}{1.779693in}}%
\pgfpathlineto{\pgfqpoint{4.962406in}{1.766680in}}%
\pgfpathlineto{\pgfqpoint{4.954790in}{1.753655in}}%
\pgfpathclose%
\pgfusepath{fill}%
\end{pgfscope}%
\begin{pgfscope}%
\pgfpathrectangle{\pgfqpoint{1.254980in}{0.150000in}}{\pgfqpoint{5.490039in}{5.490039in}}%
\pgfusepath{clip}%
\pgfsetbuttcap%
\pgfsetroundjoin%
\definecolor{currentfill}{rgb}{0.185783,0.704891,0.485273}%
\pgfsetfillcolor{currentfill}%
\pgfsetfillopacity{0.700000}%
\pgfsetlinewidth{0.000000pt}%
\definecolor{currentstroke}{rgb}{0.000000,0.000000,0.000000}%
\pgfsetstrokecolor{currentstroke}%
\pgfsetdash{}{0pt}%
\pgfpathmoveto{\pgfqpoint{2.252722in}{3.083620in}}%
\pgfpathlineto{\pgfqpoint{2.266692in}{3.058858in}}%
\pgfpathlineto{\pgfqpoint{2.280650in}{3.034304in}}%
\pgfpathlineto{\pgfqpoint{2.294597in}{3.009957in}}%
\pgfpathlineto{\pgfqpoint{2.308532in}{2.985817in}}%
\pgfpathlineto{\pgfqpoint{2.317670in}{2.976233in}}%
\pgfpathlineto{\pgfqpoint{2.326784in}{2.966977in}}%
\pgfpathlineto{\pgfqpoint{2.335875in}{2.958045in}}%
\pgfpathlineto{\pgfqpoint{2.344942in}{2.949432in}}%
\pgfpathlineto{\pgfqpoint{2.331068in}{2.973044in}}%
\pgfpathlineto{\pgfqpoint{2.317182in}{2.996860in}}%
\pgfpathlineto{\pgfqpoint{2.303286in}{3.020883in}}%
\pgfpathlineto{\pgfqpoint{2.289378in}{3.045112in}}%
\pgfpathlineto{\pgfqpoint{2.280250in}{3.054248in}}%
\pgfpathlineto{\pgfqpoint{2.271098in}{3.063708in}}%
\pgfpathlineto{\pgfqpoint{2.261923in}{3.073497in}}%
\pgfpathlineto{\pgfqpoint{2.252722in}{3.083620in}}%
\pgfpathclose%
\pgfusepath{fill}%
\end{pgfscope}%
\begin{pgfscope}%
\pgfpathrectangle{\pgfqpoint{1.254980in}{0.150000in}}{\pgfqpoint{5.490039in}{5.490039in}}%
\pgfusepath{clip}%
\pgfsetbuttcap%
\pgfsetroundjoin%
\definecolor{currentfill}{rgb}{0.267004,0.004874,0.329415}%
\pgfsetfillcolor{currentfill}%
\pgfsetfillopacity{0.700000}%
\pgfsetlinewidth{0.000000pt}%
\definecolor{currentstroke}{rgb}{0.000000,0.000000,0.000000}%
\pgfsetstrokecolor{currentstroke}%
\pgfsetdash{}{0pt}%
\pgfpathmoveto{\pgfqpoint{3.954453in}{1.345512in}}%
\pgfpathlineto{\pgfqpoint{3.968068in}{1.339889in}}%
\pgfpathlineto{\pgfqpoint{3.981689in}{1.334383in}}%
\pgfpathlineto{\pgfqpoint{3.995314in}{1.328994in}}%
\pgfpathlineto{\pgfqpoint{4.008946in}{1.323722in}}%
\pgfpathlineto{\pgfqpoint{4.016849in}{1.330119in}}%
\pgfpathlineto{\pgfqpoint{4.024746in}{1.336668in}}%
\pgfpathlineto{\pgfqpoint{4.032636in}{1.343364in}}%
\pgfpathlineto{\pgfqpoint{4.040520in}{1.350204in}}%
\pgfpathlineto{\pgfqpoint{4.026905in}{1.355052in}}%
\pgfpathlineto{\pgfqpoint{4.013296in}{1.360018in}}%
\pgfpathlineto{\pgfqpoint{3.999693in}{1.365100in}}%
\pgfpathlineto{\pgfqpoint{3.986095in}{1.370300in}}%
\pgfpathlineto{\pgfqpoint{3.978195in}{1.363877in}}%
\pgfpathlineto{\pgfqpoint{3.970288in}{1.357602in}}%
\pgfpathlineto{\pgfqpoint{3.962374in}{1.351480in}}%
\pgfpathlineto{\pgfqpoint{3.954453in}{1.345512in}}%
\pgfpathclose%
\pgfusepath{fill}%
\end{pgfscope}%
\begin{pgfscope}%
\pgfpathrectangle{\pgfqpoint{1.254980in}{0.150000in}}{\pgfqpoint{5.490039in}{5.490039in}}%
\pgfusepath{clip}%
\pgfsetbuttcap%
\pgfsetroundjoin%
\definecolor{currentfill}{rgb}{0.279566,0.067836,0.391917}%
\pgfsetfillcolor{currentfill}%
\pgfsetfillopacity{0.700000}%
\pgfsetlinewidth{0.000000pt}%
\definecolor{currentstroke}{rgb}{0.000000,0.000000,0.000000}%
\pgfsetstrokecolor{currentstroke}%
\pgfsetdash{}{0pt}%
\pgfpathmoveto{\pgfqpoint{4.493581in}{1.431347in}}%
\pgfpathlineto{\pgfqpoint{4.507345in}{1.430878in}}%
\pgfpathlineto{\pgfqpoint{4.521118in}{1.430522in}}%
\pgfpathlineto{\pgfqpoint{4.534901in}{1.430276in}}%
\pgfpathlineto{\pgfqpoint{4.548694in}{1.430143in}}%
\pgfpathlineto{\pgfqpoint{4.556419in}{1.441381in}}%
\pgfpathlineto{\pgfqpoint{4.564139in}{1.452679in}}%
\pgfpathlineto{\pgfqpoint{4.571855in}{1.464035in}}%
\pgfpathlineto{\pgfqpoint{4.579567in}{1.475445in}}%
\pgfpathlineto{\pgfqpoint{4.565779in}{1.475235in}}%
\pgfpathlineto{\pgfqpoint{4.552000in}{1.475135in}}%
\pgfpathlineto{\pgfqpoint{4.538232in}{1.475148in}}%
\pgfpathlineto{\pgfqpoint{4.524473in}{1.475273in}}%
\pgfpathlineto{\pgfqpoint{4.516757in}{1.464201in}}%
\pgfpathlineto{\pgfqpoint{4.509036in}{1.453187in}}%
\pgfpathlineto{\pgfqpoint{4.501310in}{1.442235in}}%
\pgfpathlineto{\pgfqpoint{4.493581in}{1.431347in}}%
\pgfpathclose%
\pgfusepath{fill}%
\end{pgfscope}%
\begin{pgfscope}%
\pgfpathrectangle{\pgfqpoint{1.254980in}{0.150000in}}{\pgfqpoint{5.490039in}{5.490039in}}%
\pgfusepath{clip}%
\pgfsetbuttcap%
\pgfsetroundjoin%
\definecolor{currentfill}{rgb}{0.281924,0.089666,0.412415}%
\pgfsetfillcolor{currentfill}%
\pgfsetfillopacity{0.700000}%
\pgfsetlinewidth{0.000000pt}%
\definecolor{currentstroke}{rgb}{0.000000,0.000000,0.000000}%
\pgfsetstrokecolor{currentstroke}%
\pgfsetdash{}{0pt}%
\pgfpathmoveto{\pgfqpoint{4.579567in}{1.475445in}}%
\pgfpathlineto{\pgfqpoint{4.593365in}{1.475768in}}%
\pgfpathlineto{\pgfqpoint{4.607173in}{1.476202in}}%
\pgfpathlineto{\pgfqpoint{4.620991in}{1.476747in}}%
\pgfpathlineto{\pgfqpoint{4.634819in}{1.477404in}}%
\pgfpathlineto{\pgfqpoint{4.642523in}{1.489201in}}%
\pgfpathlineto{\pgfqpoint{4.650222in}{1.501044in}}%
\pgfpathlineto{\pgfqpoint{4.657918in}{1.512930in}}%
\pgfpathlineto{\pgfqpoint{4.665609in}{1.524856in}}%
\pgfpathlineto{\pgfqpoint{4.651783in}{1.523871in}}%
\pgfpathlineto{\pgfqpoint{4.637969in}{1.522996in}}%
\pgfpathlineto{\pgfqpoint{4.624164in}{1.522233in}}%
\pgfpathlineto{\pgfqpoint{4.610370in}{1.521582in}}%
\pgfpathlineto{\pgfqpoint{4.602676in}{1.509979in}}%
\pgfpathlineto{\pgfqpoint{4.594977in}{1.498420in}}%
\pgfpathlineto{\pgfqpoint{4.587274in}{1.486908in}}%
\pgfpathlineto{\pgfqpoint{4.579567in}{1.475445in}}%
\pgfpathclose%
\pgfusepath{fill}%
\end{pgfscope}%
\begin{pgfscope}%
\pgfpathrectangle{\pgfqpoint{1.254980in}{0.150000in}}{\pgfqpoint{5.490039in}{5.490039in}}%
\pgfusepath{clip}%
\pgfsetbuttcap%
\pgfsetroundjoin%
\definecolor{currentfill}{rgb}{0.278791,0.062145,0.386592}%
\pgfsetfillcolor{currentfill}%
\pgfsetfillopacity{0.700000}%
\pgfsetlinewidth{0.000000pt}%
\definecolor{currentstroke}{rgb}{0.000000,0.000000,0.000000}%
\pgfsetstrokecolor{currentstroke}%
\pgfsetdash{}{0pt}%
\pgfpathmoveto{\pgfqpoint{3.618581in}{1.459395in}}%
\pgfpathlineto{\pgfqpoint{3.632162in}{1.450454in}}%
\pgfpathlineto{\pgfqpoint{3.645746in}{1.441638in}}%
\pgfpathlineto{\pgfqpoint{3.659332in}{1.432946in}}%
\pgfpathlineto{\pgfqpoint{3.672921in}{1.424377in}}%
\pgfpathlineto{\pgfqpoint{3.680995in}{1.427110in}}%
\pgfpathlineto{\pgfqpoint{3.689059in}{1.430044in}}%
\pgfpathlineto{\pgfqpoint{3.697114in}{1.433178in}}%
\pgfpathlineto{\pgfqpoint{3.705160in}{1.436507in}}%
\pgfpathlineto{\pgfqpoint{3.691596in}{1.444615in}}%
\pgfpathlineto{\pgfqpoint{3.678035in}{1.452847in}}%
\pgfpathlineto{\pgfqpoint{3.664478in}{1.461203in}}%
\pgfpathlineto{\pgfqpoint{3.650923in}{1.469683in}}%
\pgfpathlineto{\pgfqpoint{3.642852in}{1.466808in}}%
\pgfpathlineto{\pgfqpoint{3.634772in}{1.464133in}}%
\pgfpathlineto{\pgfqpoint{3.626681in}{1.461660in}}%
\pgfpathlineto{\pgfqpoint{3.618581in}{1.459395in}}%
\pgfpathclose%
\pgfusepath{fill}%
\end{pgfscope}%
\begin{pgfscope}%
\pgfpathrectangle{\pgfqpoint{1.254980in}{0.150000in}}{\pgfqpoint{5.490039in}{5.490039in}}%
\pgfusepath{clip}%
\pgfsetbuttcap%
\pgfsetroundjoin%
\definecolor{currentfill}{rgb}{0.276022,0.044167,0.370164}%
\pgfsetfillcolor{currentfill}%
\pgfsetfillopacity{0.700000}%
\pgfsetlinewidth{0.000000pt}%
\definecolor{currentstroke}{rgb}{0.000000,0.000000,0.000000}%
\pgfsetstrokecolor{currentstroke}%
\pgfsetdash{}{0pt}%
\pgfpathmoveto{\pgfqpoint{4.407626in}{1.392927in}}%
\pgfpathlineto{\pgfqpoint{4.421360in}{1.391651in}}%
\pgfpathlineto{\pgfqpoint{4.435103in}{1.390486in}}%
\pgfpathlineto{\pgfqpoint{4.448855in}{1.389433in}}%
\pgfpathlineto{\pgfqpoint{4.462616in}{1.388493in}}%
\pgfpathlineto{\pgfqpoint{4.470364in}{1.399096in}}%
\pgfpathlineto{\pgfqpoint{4.478107in}{1.409775in}}%
\pgfpathlineto{\pgfqpoint{4.485846in}{1.420526in}}%
\pgfpathlineto{\pgfqpoint{4.493581in}{1.431347in}}%
\pgfpathlineto{\pgfqpoint{4.479826in}{1.431928in}}%
\pgfpathlineto{\pgfqpoint{4.466080in}{1.432621in}}%
\pgfpathlineto{\pgfqpoint{4.452344in}{1.433426in}}%
\pgfpathlineto{\pgfqpoint{4.438617in}{1.434343in}}%
\pgfpathlineto{\pgfqpoint{4.430876in}{1.423876in}}%
\pgfpathlineto{\pgfqpoint{4.423131in}{1.413482in}}%
\pgfpathlineto{\pgfqpoint{4.415380in}{1.403165in}}%
\pgfpathlineto{\pgfqpoint{4.407626in}{1.392927in}}%
\pgfpathclose%
\pgfusepath{fill}%
\end{pgfscope}%
\begin{pgfscope}%
\pgfpathrectangle{\pgfqpoint{1.254980in}{0.150000in}}{\pgfqpoint{5.490039in}{5.490039in}}%
\pgfusepath{clip}%
\pgfsetbuttcap%
\pgfsetroundjoin%
\definecolor{currentfill}{rgb}{0.131172,0.555899,0.552459}%
\pgfsetfillcolor{currentfill}%
\pgfsetfillopacity{0.700000}%
\pgfsetlinewidth{0.000000pt}%
\definecolor{currentstroke}{rgb}{0.000000,0.000000,0.000000}%
\pgfsetstrokecolor{currentstroke}%
\pgfsetdash{}{0pt}%
\pgfpathmoveto{\pgfqpoint{2.494261in}{2.659469in}}%
\pgfpathlineto{\pgfqpoint{2.508088in}{2.638088in}}%
\pgfpathlineto{\pgfqpoint{2.521906in}{2.616891in}}%
\pgfpathlineto{\pgfqpoint{2.535715in}{2.595875in}}%
\pgfpathlineto{\pgfqpoint{2.549517in}{2.575041in}}%
\pgfpathlineto{\pgfqpoint{2.558452in}{2.566835in}}%
\pgfpathlineto{\pgfqpoint{2.567366in}{2.558947in}}%
\pgfpathlineto{\pgfqpoint{2.576259in}{2.551373in}}%
\pgfpathlineto{\pgfqpoint{2.585130in}{2.544108in}}%
\pgfpathlineto{\pgfqpoint{2.571384in}{2.564414in}}%
\pgfpathlineto{\pgfqpoint{2.557630in}{2.584901in}}%
\pgfpathlineto{\pgfqpoint{2.543868in}{2.605569in}}%
\pgfpathlineto{\pgfqpoint{2.530098in}{2.626419in}}%
\pgfpathlineto{\pgfqpoint{2.521171in}{2.634205in}}%
\pgfpathlineto{\pgfqpoint{2.512223in}{2.642306in}}%
\pgfpathlineto{\pgfqpoint{2.503253in}{2.650725in}}%
\pgfpathlineto{\pgfqpoint{2.494261in}{2.659469in}}%
\pgfpathclose%
\pgfusepath{fill}%
\end{pgfscope}%
\begin{pgfscope}%
\pgfpathrectangle{\pgfqpoint{1.254980in}{0.150000in}}{\pgfqpoint{5.490039in}{5.490039in}}%
\pgfusepath{clip}%
\pgfsetbuttcap%
\pgfsetroundjoin%
\definecolor{currentfill}{rgb}{0.283229,0.120777,0.440584}%
\pgfsetfillcolor{currentfill}%
\pgfsetfillopacity{0.700000}%
\pgfsetlinewidth{0.000000pt}%
\definecolor{currentstroke}{rgb}{0.000000,0.000000,0.000000}%
\pgfsetstrokecolor{currentstroke}%
\pgfsetdash{}{0pt}%
\pgfpathmoveto{\pgfqpoint{4.665609in}{1.524856in}}%
\pgfpathlineto{\pgfqpoint{4.679444in}{1.525954in}}%
\pgfpathlineto{\pgfqpoint{4.693291in}{1.527162in}}%
\pgfpathlineto{\pgfqpoint{4.707148in}{1.528482in}}%
\pgfpathlineto{\pgfqpoint{4.721016in}{1.529912in}}%
\pgfpathlineto{\pgfqpoint{4.728701in}{1.542196in}}%
\pgfpathlineto{\pgfqpoint{4.736381in}{1.554511in}}%
\pgfpathlineto{\pgfqpoint{4.744057in}{1.566856in}}%
\pgfpathlineto{\pgfqpoint{4.751729in}{1.579226in}}%
\pgfpathlineto{\pgfqpoint{4.737863in}{1.577481in}}%
\pgfpathlineto{\pgfqpoint{4.724008in}{1.575848in}}%
\pgfpathlineto{\pgfqpoint{4.710164in}{1.574326in}}%
\pgfpathlineto{\pgfqpoint{4.696331in}{1.572915in}}%
\pgfpathlineto{\pgfqpoint{4.688657in}{1.560852in}}%
\pgfpathlineto{\pgfqpoint{4.680978in}{1.548820in}}%
\pgfpathlineto{\pgfqpoint{4.673296in}{1.536821in}}%
\pgfpathlineto{\pgfqpoint{4.665609in}{1.524856in}}%
\pgfpathclose%
\pgfusepath{fill}%
\end{pgfscope}%
\begin{pgfscope}%
\pgfpathrectangle{\pgfqpoint{1.254980in}{0.150000in}}{\pgfqpoint{5.490039in}{5.490039in}}%
\pgfusepath{clip}%
\pgfsetbuttcap%
\pgfsetroundjoin%
\definecolor{currentfill}{rgb}{0.271305,0.019942,0.347269}%
\pgfsetfillcolor{currentfill}%
\pgfsetfillopacity{0.700000}%
\pgfsetlinewidth{0.000000pt}%
\definecolor{currentstroke}{rgb}{0.000000,0.000000,0.000000}%
\pgfsetstrokecolor{currentstroke}%
\pgfsetdash{}{0pt}%
\pgfpathmoveto{\pgfqpoint{3.813797in}{1.376039in}}%
\pgfpathlineto{\pgfqpoint{3.827394in}{1.369025in}}%
\pgfpathlineto{\pgfqpoint{3.840996in}{1.362131in}}%
\pgfpathlineto{\pgfqpoint{3.854601in}{1.355357in}}%
\pgfpathlineto{\pgfqpoint{3.868211in}{1.348702in}}%
\pgfpathlineto{\pgfqpoint{3.876181in}{1.353565in}}%
\pgfpathlineto{\pgfqpoint{3.884143in}{1.358602in}}%
\pgfpathlineto{\pgfqpoint{3.892098in}{1.363810in}}%
\pgfpathlineto{\pgfqpoint{3.900045in}{1.369184in}}%
\pgfpathlineto{\pgfqpoint{3.886455in}{1.375398in}}%
\pgfpathlineto{\pgfqpoint{3.872870in}{1.381732in}}%
\pgfpathlineto{\pgfqpoint{3.859289in}{1.388185in}}%
\pgfpathlineto{\pgfqpoint{3.845713in}{1.394758in}}%
\pgfpathlineto{\pgfqpoint{3.837746in}{1.389818in}}%
\pgfpathlineto{\pgfqpoint{3.829771in}{1.385049in}}%
\pgfpathlineto{\pgfqpoint{3.821788in}{1.380455in}}%
\pgfpathlineto{\pgfqpoint{3.813797in}{1.376039in}}%
\pgfpathclose%
\pgfusepath{fill}%
\end{pgfscope}%
\begin{pgfscope}%
\pgfpathrectangle{\pgfqpoint{1.254980in}{0.150000in}}{\pgfqpoint{5.490039in}{5.490039in}}%
\pgfusepath{clip}%
\pgfsetbuttcap%
\pgfsetroundjoin%
\definecolor{currentfill}{rgb}{0.267004,0.004874,0.329415}%
\pgfsetfillcolor{currentfill}%
\pgfsetfillopacity{0.700000}%
\pgfsetlinewidth{0.000000pt}%
\definecolor{currentstroke}{rgb}{0.000000,0.000000,0.000000}%
\pgfsetstrokecolor{currentstroke}%
\pgfsetdash{}{0pt}%
\pgfpathmoveto{\pgfqpoint{4.095039in}{1.331971in}}%
\pgfpathlineto{\pgfqpoint{4.108685in}{1.327702in}}%
\pgfpathlineto{\pgfqpoint{4.122337in}{1.323549in}}%
\pgfpathlineto{\pgfqpoint{4.135995in}{1.319510in}}%
\pgfpathlineto{\pgfqpoint{4.149660in}{1.315586in}}%
\pgfpathlineto{\pgfqpoint{4.157508in}{1.323393in}}%
\pgfpathlineto{\pgfqpoint{4.165351in}{1.331328in}}%
\pgfpathlineto{\pgfqpoint{4.173187in}{1.339390in}}%
\pgfpathlineto{\pgfqpoint{4.181018in}{1.347574in}}%
\pgfpathlineto{\pgfqpoint{4.167366in}{1.351091in}}%
\pgfpathlineto{\pgfqpoint{4.153721in}{1.354723in}}%
\pgfpathlineto{\pgfqpoint{4.140083in}{1.358470in}}%
\pgfpathlineto{\pgfqpoint{4.126451in}{1.362332in}}%
\pgfpathlineto{\pgfqpoint{4.118607in}{1.354549in}}%
\pgfpathlineto{\pgfqpoint{4.110758in}{1.346892in}}%
\pgfpathlineto{\pgfqpoint{4.102902in}{1.339365in}}%
\pgfpathlineto{\pgfqpoint{4.095039in}{1.331971in}}%
\pgfpathclose%
\pgfusepath{fill}%
\end{pgfscope}%
\begin{pgfscope}%
\pgfpathrectangle{\pgfqpoint{1.254980in}{0.150000in}}{\pgfqpoint{5.490039in}{5.490039in}}%
\pgfusepath{clip}%
\pgfsetbuttcap%
\pgfsetroundjoin%
\definecolor{currentfill}{rgb}{0.282623,0.140926,0.457517}%
\pgfsetfillcolor{currentfill}%
\pgfsetfillopacity{0.700000}%
\pgfsetlinewidth{0.000000pt}%
\definecolor{currentstroke}{rgb}{0.000000,0.000000,0.000000}%
\pgfsetstrokecolor{currentstroke}%
\pgfsetdash{}{0pt}%
\pgfpathmoveto{\pgfqpoint{3.368561in}{1.620521in}}%
\pgfpathlineto{\pgfqpoint{3.382148in}{1.609044in}}%
\pgfpathlineto{\pgfqpoint{3.395736in}{1.597700in}}%
\pgfpathlineto{\pgfqpoint{3.409324in}{1.586488in}}%
\pgfpathlineto{\pgfqpoint{3.422913in}{1.575406in}}%
\pgfpathlineto{\pgfqpoint{3.431146in}{1.575341in}}%
\pgfpathlineto{\pgfqpoint{3.439368in}{1.575512in}}%
\pgfpathlineto{\pgfqpoint{3.447577in}{1.575917in}}%
\pgfpathlineto{\pgfqpoint{3.455775in}{1.576550in}}%
\pgfpathlineto{\pgfqpoint{3.442218in}{1.587148in}}%
\pgfpathlineto{\pgfqpoint{3.428662in}{1.597878in}}%
\pgfpathlineto{\pgfqpoint{3.415108in}{1.608738in}}%
\pgfpathlineto{\pgfqpoint{3.401554in}{1.619731in}}%
\pgfpathlineto{\pgfqpoint{3.393324in}{1.619574in}}%
\pgfpathlineto{\pgfqpoint{3.385082in}{1.619650in}}%
\pgfpathlineto{\pgfqpoint{3.376828in}{1.619965in}}%
\pgfpathlineto{\pgfqpoint{3.368561in}{1.620521in}}%
\pgfpathclose%
\pgfusepath{fill}%
\end{pgfscope}%
\begin{pgfscope}%
\pgfpathrectangle{\pgfqpoint{1.254980in}{0.150000in}}{\pgfqpoint{5.490039in}{5.490039in}}%
\pgfusepath{clip}%
\pgfsetbuttcap%
\pgfsetroundjoin%
\definecolor{currentfill}{rgb}{0.272594,0.025563,0.353093}%
\pgfsetfillcolor{currentfill}%
\pgfsetfillopacity{0.700000}%
\pgfsetlinewidth{0.000000pt}%
\definecolor{currentstroke}{rgb}{0.000000,0.000000,0.000000}%
\pgfsetstrokecolor{currentstroke}%
\pgfsetdash{}{0pt}%
\pgfpathmoveto{\pgfqpoint{4.321675in}{1.360564in}}%
\pgfpathlineto{\pgfqpoint{4.335383in}{1.358462in}}%
\pgfpathlineto{\pgfqpoint{4.349101in}{1.356472in}}%
\pgfpathlineto{\pgfqpoint{4.362826in}{1.354595in}}%
\pgfpathlineto{\pgfqpoint{4.376559in}{1.352830in}}%
\pgfpathlineto{\pgfqpoint{4.384333in}{1.362720in}}%
\pgfpathlineto{\pgfqpoint{4.392102in}{1.372702in}}%
\pgfpathlineto{\pgfqpoint{4.399866in}{1.382772in}}%
\pgfpathlineto{\pgfqpoint{4.407626in}{1.392927in}}%
\pgfpathlineto{\pgfqpoint{4.393900in}{1.394317in}}%
\pgfpathlineto{\pgfqpoint{4.380183in}{1.395819in}}%
\pgfpathlineto{\pgfqpoint{4.366475in}{1.397433in}}%
\pgfpathlineto{\pgfqpoint{4.352774in}{1.399161in}}%
\pgfpathlineto{\pgfqpoint{4.345007in}{1.389375in}}%
\pgfpathlineto{\pgfqpoint{4.337234in}{1.379678in}}%
\pgfpathlineto{\pgfqpoint{4.329457in}{1.370073in}}%
\pgfpathlineto{\pgfqpoint{4.321675in}{1.360564in}}%
\pgfpathclose%
\pgfusepath{fill}%
\end{pgfscope}%
\begin{pgfscope}%
\pgfpathrectangle{\pgfqpoint{1.254980in}{0.150000in}}{\pgfqpoint{5.490039in}{5.490039in}}%
\pgfusepath{clip}%
\pgfsetbuttcap%
\pgfsetroundjoin%
\definecolor{currentfill}{rgb}{0.281887,0.150881,0.465405}%
\pgfsetfillcolor{currentfill}%
\pgfsetfillopacity{0.700000}%
\pgfsetlinewidth{0.000000pt}%
\definecolor{currentstroke}{rgb}{0.000000,0.000000,0.000000}%
\pgfsetstrokecolor{currentstroke}%
\pgfsetdash{}{0pt}%
\pgfpathmoveto{\pgfqpoint{4.751729in}{1.579226in}}%
\pgfpathlineto{\pgfqpoint{4.765607in}{1.581082in}}%
\pgfpathlineto{\pgfqpoint{4.779495in}{1.583049in}}%
\pgfpathlineto{\pgfqpoint{4.793395in}{1.585127in}}%
\pgfpathlineto{\pgfqpoint{4.807307in}{1.587316in}}%
\pgfpathlineto{\pgfqpoint{4.814974in}{1.600015in}}%
\pgfpathlineto{\pgfqpoint{4.822636in}{1.612732in}}%
\pgfpathlineto{\pgfqpoint{4.830294in}{1.625465in}}%
\pgfpathlineto{\pgfqpoint{4.837948in}{1.638211in}}%
\pgfpathlineto{\pgfqpoint{4.824038in}{1.635723in}}%
\pgfpathlineto{\pgfqpoint{4.810139in}{1.633346in}}%
\pgfpathlineto{\pgfqpoint{4.796252in}{1.631080in}}%
\pgfpathlineto{\pgfqpoint{4.782376in}{1.628926in}}%
\pgfpathlineto{\pgfqpoint{4.774720in}{1.616473in}}%
\pgfpathlineto{\pgfqpoint{4.767061in}{1.604037in}}%
\pgfpathlineto{\pgfqpoint{4.759397in}{1.591621in}}%
\pgfpathlineto{\pgfqpoint{4.751729in}{1.579226in}}%
\pgfpathclose%
\pgfusepath{fill}%
\end{pgfscope}%
\begin{pgfscope}%
\pgfpathrectangle{\pgfqpoint{1.254980in}{0.150000in}}{\pgfqpoint{5.490039in}{5.490039in}}%
\pgfusepath{clip}%
\pgfsetbuttcap%
\pgfsetroundjoin%
\definecolor{currentfill}{rgb}{0.253935,0.265254,0.529983}%
\pgfsetfillcolor{currentfill}%
\pgfsetfillopacity{0.700000}%
\pgfsetlinewidth{0.000000pt}%
\definecolor{currentstroke}{rgb}{0.000000,0.000000,0.000000}%
\pgfsetstrokecolor{currentstroke}%
\pgfsetdash{}{0pt}%
\pgfpathmoveto{\pgfqpoint{5.041190in}{1.821731in}}%
\pgfpathlineto{\pgfqpoint{5.055214in}{1.826023in}}%
\pgfpathlineto{\pgfqpoint{5.069251in}{1.830427in}}%
\pgfpathlineto{\pgfqpoint{5.083302in}{1.834941in}}%
\pgfpathlineto{\pgfqpoint{5.097366in}{1.839567in}}%
\pgfpathlineto{\pgfqpoint{5.104965in}{1.853019in}}%
\pgfpathlineto{\pgfqpoint{5.112560in}{1.866443in}}%
\pgfpathlineto{\pgfqpoint{5.120150in}{1.879839in}}%
\pgfpathlineto{\pgfqpoint{5.127736in}{1.893204in}}%
\pgfpathlineto{\pgfqpoint{5.113670in}{1.888342in}}%
\pgfpathlineto{\pgfqpoint{5.099619in}{1.883590in}}%
\pgfpathlineto{\pgfqpoint{5.085581in}{1.878951in}}%
\pgfpathlineto{\pgfqpoint{5.071556in}{1.874423in}}%
\pgfpathlineto{\pgfqpoint{5.063971in}{1.861288in}}%
\pgfpathlineto{\pgfqpoint{5.056382in}{1.848127in}}%
\pgfpathlineto{\pgfqpoint{5.048788in}{1.834941in}}%
\pgfpathlineto{\pgfqpoint{5.041190in}{1.821731in}}%
\pgfpathclose%
\pgfusepath{fill}%
\end{pgfscope}%
\begin{pgfscope}%
\pgfpathrectangle{\pgfqpoint{1.254980in}{0.150000in}}{\pgfqpoint{5.490039in}{5.490039in}}%
\pgfusepath{clip}%
\pgfsetbuttcap%
\pgfsetroundjoin%
\definecolor{currentfill}{rgb}{0.214298,0.355619,0.551184}%
\pgfsetfillcolor{currentfill}%
\pgfsetfillopacity{0.700000}%
\pgfsetlinewidth{0.000000pt}%
\definecolor{currentstroke}{rgb}{0.000000,0.000000,0.000000}%
\pgfsetstrokecolor{currentstroke}%
\pgfsetdash{}{0pt}%
\pgfpathmoveto{\pgfqpoint{5.244659in}{2.021095in}}%
\pgfpathlineto{\pgfqpoint{5.258798in}{2.026942in}}%
\pgfpathlineto{\pgfqpoint{5.272951in}{2.032900in}}%
\pgfpathlineto{\pgfqpoint{5.287120in}{2.038971in}}%
\pgfpathlineto{\pgfqpoint{5.301303in}{2.045153in}}%
\pgfpathlineto{\pgfqpoint{5.308847in}{2.058560in}}%
\pgfpathlineto{\pgfqpoint{5.316386in}{2.071913in}}%
\pgfpathlineto{\pgfqpoint{5.323919in}{2.085210in}}%
\pgfpathlineto{\pgfqpoint{5.331448in}{2.098450in}}%
\pgfpathlineto{\pgfqpoint{5.317263in}{2.092079in}}%
\pgfpathlineto{\pgfqpoint{5.303093in}{2.085820in}}%
\pgfpathlineto{\pgfqpoint{5.288938in}{2.079673in}}%
\pgfpathlineto{\pgfqpoint{5.274797in}{2.073638in}}%
\pgfpathlineto{\pgfqpoint{5.267271in}{2.060580in}}%
\pgfpathlineto{\pgfqpoint{5.259738in}{2.047470in}}%
\pgfpathlineto{\pgfqpoint{5.252201in}{2.034307in}}%
\pgfpathlineto{\pgfqpoint{5.244659in}{2.021095in}}%
\pgfpathclose%
\pgfusepath{fill}%
\end{pgfscope}%
\begin{pgfscope}%
\pgfpathrectangle{\pgfqpoint{1.254980in}{0.150000in}}{\pgfqpoint{5.490039in}{5.490039in}}%
\pgfusepath{clip}%
\pgfsetbuttcap%
\pgfsetroundjoin%
\definecolor{currentfill}{rgb}{0.121831,0.589055,0.545623}%
\pgfsetfillcolor{currentfill}%
\pgfsetfillopacity{0.700000}%
\pgfsetlinewidth{0.000000pt}%
\definecolor{currentstroke}{rgb}{0.000000,0.000000,0.000000}%
\pgfsetstrokecolor{currentstroke}%
\pgfsetdash{}{0pt}%
\pgfpathmoveto{\pgfqpoint{2.438867in}{2.746852in}}%
\pgfpathlineto{\pgfqpoint{2.452729in}{2.724725in}}%
\pgfpathlineto{\pgfqpoint{2.466582in}{2.702787in}}%
\pgfpathlineto{\pgfqpoint{2.480426in}{2.681035in}}%
\pgfpathlineto{\pgfqpoint{2.494261in}{2.659469in}}%
\pgfpathlineto{\pgfqpoint{2.503253in}{2.650725in}}%
\pgfpathlineto{\pgfqpoint{2.512223in}{2.642306in}}%
\pgfpathlineto{\pgfqpoint{2.521171in}{2.634205in}}%
\pgfpathlineto{\pgfqpoint{2.530098in}{2.626419in}}%
\pgfpathlineto{\pgfqpoint{2.516319in}{2.647452in}}%
\pgfpathlineto{\pgfqpoint{2.502532in}{2.668671in}}%
\pgfpathlineto{\pgfqpoint{2.488737in}{2.690075in}}%
\pgfpathlineto{\pgfqpoint{2.474933in}{2.711666in}}%
\pgfpathlineto{\pgfqpoint{2.465950in}{2.719978in}}%
\pgfpathlineto{\pgfqpoint{2.456945in}{2.728609in}}%
\pgfpathlineto{\pgfqpoint{2.447917in}{2.737566in}}%
\pgfpathlineto{\pgfqpoint{2.438867in}{2.746852in}}%
\pgfpathclose%
\pgfusepath{fill}%
\end{pgfscope}%
\begin{pgfscope}%
\pgfpathrectangle{\pgfqpoint{1.254980in}{0.150000in}}{\pgfqpoint{5.490039in}{5.490039in}}%
\pgfusepath{clip}%
\pgfsetbuttcap%
\pgfsetroundjoin%
\definecolor{currentfill}{rgb}{0.283187,0.125848,0.444960}%
\pgfsetfillcolor{currentfill}%
\pgfsetfillopacity{0.700000}%
\pgfsetlinewidth{0.000000pt}%
\definecolor{currentstroke}{rgb}{0.000000,0.000000,0.000000}%
\pgfsetstrokecolor{currentstroke}%
\pgfsetdash{}{0pt}%
\pgfpathmoveto{\pgfqpoint{3.422913in}{1.575406in}}%
\pgfpathlineto{\pgfqpoint{3.436502in}{1.564455in}}%
\pgfpathlineto{\pgfqpoint{3.450093in}{1.553635in}}%
\pgfpathlineto{\pgfqpoint{3.463685in}{1.542944in}}%
\pgfpathlineto{\pgfqpoint{3.477278in}{1.532382in}}%
\pgfpathlineto{\pgfqpoint{3.485479in}{1.532805in}}%
\pgfpathlineto{\pgfqpoint{3.493669in}{1.533461in}}%
\pgfpathlineto{\pgfqpoint{3.501848in}{1.534345in}}%
\pgfpathlineto{\pgfqpoint{3.510015in}{1.535453in}}%
\pgfpathlineto{\pgfqpoint{3.496453in}{1.545534in}}%
\pgfpathlineto{\pgfqpoint{3.482892in}{1.555743in}}%
\pgfpathlineto{\pgfqpoint{3.469333in}{1.566082in}}%
\pgfpathlineto{\pgfqpoint{3.455775in}{1.576550in}}%
\pgfpathlineto{\pgfqpoint{3.447577in}{1.575917in}}%
\pgfpathlineto{\pgfqpoint{3.439368in}{1.575512in}}%
\pgfpathlineto{\pgfqpoint{3.431146in}{1.575341in}}%
\pgfpathlineto{\pgfqpoint{3.422913in}{1.575406in}}%
\pgfpathclose%
\pgfusepath{fill}%
\end{pgfscope}%
\begin{pgfscope}%
\pgfpathrectangle{\pgfqpoint{1.254980in}{0.150000in}}{\pgfqpoint{5.490039in}{5.490039in}}%
\pgfusepath{clip}%
\pgfsetbuttcap%
\pgfsetroundjoin%
\definecolor{currentfill}{rgb}{0.252899,0.742211,0.448284}%
\pgfsetfillcolor{currentfill}%
\pgfsetfillopacity{0.700000}%
\pgfsetlinewidth{0.000000pt}%
\definecolor{currentstroke}{rgb}{0.000000,0.000000,0.000000}%
\pgfsetstrokecolor{currentstroke}%
\pgfsetdash{}{0pt}%
\pgfpathmoveto{\pgfqpoint{2.196719in}{3.184787in}}%
\pgfpathlineto{\pgfqpoint{2.210739in}{3.159174in}}%
\pgfpathlineto{\pgfqpoint{2.224746in}{3.133777in}}%
\pgfpathlineto{\pgfqpoint{2.238740in}{3.108593in}}%
\pgfpathlineto{\pgfqpoint{2.252722in}{3.083620in}}%
\pgfpathlineto{\pgfqpoint{2.261923in}{3.073497in}}%
\pgfpathlineto{\pgfqpoint{2.271098in}{3.063708in}}%
\pgfpathlineto{\pgfqpoint{2.280250in}{3.054248in}}%
\pgfpathlineto{\pgfqpoint{2.289378in}{3.045112in}}%
\pgfpathlineto{\pgfqpoint{2.275458in}{3.069551in}}%
\pgfpathlineto{\pgfqpoint{2.261527in}{3.094199in}}%
\pgfpathlineto{\pgfqpoint{2.247583in}{3.119060in}}%
\pgfpathlineto{\pgfqpoint{2.233628in}{3.144134in}}%
\pgfpathlineto{\pgfqpoint{2.224438in}{3.153796in}}%
\pgfpathlineto{\pgfqpoint{2.215223in}{3.163790in}}%
\pgfpathlineto{\pgfqpoint{2.205984in}{3.174118in}}%
\pgfpathlineto{\pgfqpoint{2.196719in}{3.184787in}}%
\pgfpathclose%
\pgfusepath{fill}%
\end{pgfscope}%
\begin{pgfscope}%
\pgfpathrectangle{\pgfqpoint{1.254980in}{0.150000in}}{\pgfqpoint{5.490039in}{5.490039in}}%
\pgfusepath{clip}%
\pgfsetbuttcap%
\pgfsetroundjoin%
\definecolor{currentfill}{rgb}{0.278012,0.180367,0.486697}%
\pgfsetfillcolor{currentfill}%
\pgfsetfillopacity{0.700000}%
\pgfsetlinewidth{0.000000pt}%
\definecolor{currentstroke}{rgb}{0.000000,0.000000,0.000000}%
\pgfsetstrokecolor{currentstroke}%
\pgfsetdash{}{0pt}%
\pgfpathmoveto{\pgfqpoint{4.837948in}{1.638211in}}%
\pgfpathlineto{\pgfqpoint{4.851871in}{1.640810in}}%
\pgfpathlineto{\pgfqpoint{4.865805in}{1.643520in}}%
\pgfpathlineto{\pgfqpoint{4.879752in}{1.646341in}}%
\pgfpathlineto{\pgfqpoint{4.893710in}{1.649273in}}%
\pgfpathlineto{\pgfqpoint{4.901360in}{1.662319in}}%
\pgfpathlineto{\pgfqpoint{4.909006in}{1.675370in}}%
\pgfpathlineto{\pgfqpoint{4.916647in}{1.688423in}}%
\pgfpathlineto{\pgfqpoint{4.924284in}{1.701477in}}%
\pgfpathlineto{\pgfqpoint{4.910326in}{1.698261in}}%
\pgfpathlineto{\pgfqpoint{4.896379in}{1.695157in}}%
\pgfpathlineto{\pgfqpoint{4.882445in}{1.692163in}}%
\pgfpathlineto{\pgfqpoint{4.868523in}{1.689281in}}%
\pgfpathlineto{\pgfqpoint{4.860886in}{1.676505in}}%
\pgfpathlineto{\pgfqpoint{4.853244in}{1.663733in}}%
\pgfpathlineto{\pgfqpoint{4.845598in}{1.650967in}}%
\pgfpathlineto{\pgfqpoint{4.837948in}{1.638211in}}%
\pgfpathclose%
\pgfusepath{fill}%
\end{pgfscope}%
\begin{pgfscope}%
\pgfpathrectangle{\pgfqpoint{1.254980in}{0.150000in}}{\pgfqpoint{5.490039in}{5.490039in}}%
\pgfusepath{clip}%
\pgfsetbuttcap%
\pgfsetroundjoin%
\definecolor{currentfill}{rgb}{0.277018,0.050344,0.375715}%
\pgfsetfillcolor{currentfill}%
\pgfsetfillopacity{0.700000}%
\pgfsetlinewidth{0.000000pt}%
\definecolor{currentstroke}{rgb}{0.000000,0.000000,0.000000}%
\pgfsetstrokecolor{currentstroke}%
\pgfsetdash{}{0pt}%
\pgfpathmoveto{\pgfqpoint{3.672921in}{1.424377in}}%
\pgfpathlineto{\pgfqpoint{3.686513in}{1.415932in}}%
\pgfpathlineto{\pgfqpoint{3.700108in}{1.407610in}}%
\pgfpathlineto{\pgfqpoint{3.713706in}{1.399410in}}%
\pgfpathlineto{\pgfqpoint{3.727308in}{1.391333in}}%
\pgfpathlineto{\pgfqpoint{3.735357in}{1.394531in}}%
\pgfpathlineto{\pgfqpoint{3.743396in}{1.397928in}}%
\pgfpathlineto{\pgfqpoint{3.751427in}{1.401518in}}%
\pgfpathlineto{\pgfqpoint{3.759449in}{1.405300in}}%
\pgfpathlineto{\pgfqpoint{3.745872in}{1.412918in}}%
\pgfpathlineto{\pgfqpoint{3.732298in}{1.420658in}}%
\pgfpathlineto{\pgfqpoint{3.718727in}{1.428521in}}%
\pgfpathlineto{\pgfqpoint{3.705160in}{1.436507in}}%
\pgfpathlineto{\pgfqpoint{3.697114in}{1.433178in}}%
\pgfpathlineto{\pgfqpoint{3.689059in}{1.430044in}}%
\pgfpathlineto{\pgfqpoint{3.680995in}{1.427110in}}%
\pgfpathlineto{\pgfqpoint{3.672921in}{1.424377in}}%
\pgfpathclose%
\pgfusepath{fill}%
\end{pgfscope}%
\begin{pgfscope}%
\pgfpathrectangle{\pgfqpoint{1.254980in}{0.150000in}}{\pgfqpoint{5.490039in}{5.490039in}}%
\pgfusepath{clip}%
\pgfsetbuttcap%
\pgfsetroundjoin%
\definecolor{currentfill}{rgb}{0.268510,0.009605,0.335427}%
\pgfsetfillcolor{currentfill}%
\pgfsetfillopacity{0.700000}%
\pgfsetlinewidth{0.000000pt}%
\definecolor{currentstroke}{rgb}{0.000000,0.000000,0.000000}%
\pgfsetstrokecolor{currentstroke}%
\pgfsetdash{}{0pt}%
\pgfpathmoveto{\pgfqpoint{4.235697in}{1.334649in}}%
\pgfpathlineto{\pgfqpoint{4.249385in}{1.331702in}}%
\pgfpathlineto{\pgfqpoint{4.263080in}{1.328869in}}%
\pgfpathlineto{\pgfqpoint{4.276783in}{1.326150in}}%
\pgfpathlineto{\pgfqpoint{4.290494in}{1.323543in}}%
\pgfpathlineto{\pgfqpoint{4.298297in}{1.332640in}}%
\pgfpathlineto{\pgfqpoint{4.306095in}{1.341844in}}%
\pgfpathlineto{\pgfqpoint{4.313887in}{1.351153in}}%
\pgfpathlineto{\pgfqpoint{4.321675in}{1.360564in}}%
\pgfpathlineto{\pgfqpoint{4.307974in}{1.362780in}}%
\pgfpathlineto{\pgfqpoint{4.294281in}{1.365109in}}%
\pgfpathlineto{\pgfqpoint{4.280596in}{1.367551in}}%
\pgfpathlineto{\pgfqpoint{4.266919in}{1.370107in}}%
\pgfpathlineto{\pgfqpoint{4.259121in}{1.361081in}}%
\pgfpathlineto{\pgfqpoint{4.251319in}{1.352160in}}%
\pgfpathlineto{\pgfqpoint{4.243510in}{1.343349in}}%
\pgfpathlineto{\pgfqpoint{4.235697in}{1.334649in}}%
\pgfpathclose%
\pgfusepath{fill}%
\end{pgfscope}%
\begin{pgfscope}%
\pgfpathrectangle{\pgfqpoint{1.254980in}{0.150000in}}{\pgfqpoint{5.490039in}{5.490039in}}%
\pgfusepath{clip}%
\pgfsetbuttcap%
\pgfsetroundjoin%
\definecolor{currentfill}{rgb}{0.199430,0.387607,0.554642}%
\pgfsetfillcolor{currentfill}%
\pgfsetfillopacity{0.700000}%
\pgfsetlinewidth{0.000000pt}%
\definecolor{currentstroke}{rgb}{0.000000,0.000000,0.000000}%
\pgfsetstrokecolor{currentstroke}%
\pgfsetdash{}{0pt}%
\pgfpathmoveto{\pgfqpoint{5.331448in}{2.098450in}}%
\pgfpathlineto{\pgfqpoint{5.345648in}{2.104934in}}%
\pgfpathlineto{\pgfqpoint{5.359863in}{2.111529in}}%
\pgfpathlineto{\pgfqpoint{5.374093in}{2.118236in}}%
\pgfpathlineto{\pgfqpoint{5.381617in}{2.131553in}}%
\pgfpathlineto{\pgfqpoint{5.389136in}{2.144807in}}%
\pgfpathlineto{\pgfqpoint{5.396648in}{2.157997in}}%
\pgfpathlineto{\pgfqpoint{5.404155in}{2.171122in}}%
\pgfpathlineto{\pgfqpoint{5.389923in}{2.164242in}}%
\pgfpathlineto{\pgfqpoint{5.375706in}{2.157474in}}%
\pgfpathlineto{\pgfqpoint{5.361505in}{2.150819in}}%
\pgfpathlineto{\pgfqpoint{5.353999in}{2.137818in}}%
\pgfpathlineto{\pgfqpoint{5.346487in}{2.124756in}}%
\pgfpathlineto{\pgfqpoint{5.338970in}{2.111633in}}%
\pgfpathlineto{\pgfqpoint{5.331448in}{2.098450in}}%
\pgfpathclose%
\pgfusepath{fill}%
\end{pgfscope}%
\begin{pgfscope}%
\pgfpathrectangle{\pgfqpoint{1.254980in}{0.150000in}}{\pgfqpoint{5.490039in}{5.490039in}}%
\pgfusepath{clip}%
\pgfsetbuttcap%
\pgfsetroundjoin%
\definecolor{currentfill}{rgb}{0.267004,0.004874,0.329415}%
\pgfsetfillcolor{currentfill}%
\pgfsetfillopacity{0.700000}%
\pgfsetlinewidth{0.000000pt}%
\definecolor{currentstroke}{rgb}{0.000000,0.000000,0.000000}%
\pgfsetstrokecolor{currentstroke}%
\pgfsetdash{}{0pt}%
\pgfpathmoveto{\pgfqpoint{4.008946in}{1.323722in}}%
\pgfpathlineto{\pgfqpoint{4.022582in}{1.318567in}}%
\pgfpathlineto{\pgfqpoint{4.036225in}{1.313528in}}%
\pgfpathlineto{\pgfqpoint{4.049873in}{1.308605in}}%
\pgfpathlineto{\pgfqpoint{4.063528in}{1.303797in}}%
\pgfpathlineto{\pgfqpoint{4.071415in}{1.310624in}}%
\pgfpathlineto{\pgfqpoint{4.079296in}{1.317597in}}%
\pgfpathlineto{\pgfqpoint{4.087171in}{1.324714in}}%
\pgfpathlineto{\pgfqpoint{4.095039in}{1.331971in}}%
\pgfpathlineto{\pgfqpoint{4.081400in}{1.336355in}}%
\pgfpathlineto{\pgfqpoint{4.067767in}{1.340855in}}%
\pgfpathlineto{\pgfqpoint{4.054141in}{1.345471in}}%
\pgfpathlineto{\pgfqpoint{4.040520in}{1.350204in}}%
\pgfpathlineto{\pgfqpoint{4.032636in}{1.343364in}}%
\pgfpathlineto{\pgfqpoint{4.024746in}{1.336668in}}%
\pgfpathlineto{\pgfqpoint{4.016849in}{1.330119in}}%
\pgfpathlineto{\pgfqpoint{4.008946in}{1.323722in}}%
\pgfpathclose%
\pgfusepath{fill}%
\end{pgfscope}%
\begin{pgfscope}%
\pgfpathrectangle{\pgfqpoint{1.254980in}{0.150000in}}{\pgfqpoint{5.490039in}{5.490039in}}%
\pgfusepath{clip}%
\pgfsetbuttcap%
\pgfsetroundjoin%
\definecolor{currentfill}{rgb}{0.269944,0.014625,0.341379}%
\pgfsetfillcolor{currentfill}%
\pgfsetfillopacity{0.700000}%
\pgfsetlinewidth{0.000000pt}%
\definecolor{currentstroke}{rgb}{0.000000,0.000000,0.000000}%
\pgfsetstrokecolor{currentstroke}%
\pgfsetdash{}{0pt}%
\pgfpathmoveto{\pgfqpoint{3.868211in}{1.348702in}}%
\pgfpathlineto{\pgfqpoint{3.881825in}{1.342166in}}%
\pgfpathlineto{\pgfqpoint{3.895444in}{1.335749in}}%
\pgfpathlineto{\pgfqpoint{3.909068in}{1.329450in}}%
\pgfpathlineto{\pgfqpoint{3.922697in}{1.323269in}}%
\pgfpathlineto{\pgfqpoint{3.930647in}{1.328578in}}%
\pgfpathlineto{\pgfqpoint{3.938590in}{1.334058in}}%
\pgfpathlineto{\pgfqpoint{3.946525in}{1.339704in}}%
\pgfpathlineto{\pgfqpoint{3.954453in}{1.345512in}}%
\pgfpathlineto{\pgfqpoint{3.940844in}{1.351253in}}%
\pgfpathlineto{\pgfqpoint{3.927239in}{1.357112in}}%
\pgfpathlineto{\pgfqpoint{3.913639in}{1.363089in}}%
\pgfpathlineto{\pgfqpoint{3.900045in}{1.369184in}}%
\pgfpathlineto{\pgfqpoint{3.892098in}{1.363810in}}%
\pgfpathlineto{\pgfqpoint{3.884143in}{1.358602in}}%
\pgfpathlineto{\pgfqpoint{3.876181in}{1.353565in}}%
\pgfpathlineto{\pgfqpoint{3.868211in}{1.348702in}}%
\pgfpathclose%
\pgfusepath{fill}%
\end{pgfscope}%
\begin{pgfscope}%
\pgfpathrectangle{\pgfqpoint{1.254980in}{0.150000in}}{\pgfqpoint{5.490039in}{5.490039in}}%
\pgfusepath{clip}%
\pgfsetbuttcap%
\pgfsetroundjoin%
\definecolor{currentfill}{rgb}{0.237441,0.305202,0.541921}%
\pgfsetfillcolor{currentfill}%
\pgfsetfillopacity{0.700000}%
\pgfsetlinewidth{0.000000pt}%
\definecolor{currentstroke}{rgb}{0.000000,0.000000,0.000000}%
\pgfsetstrokecolor{currentstroke}%
\pgfsetdash{}{0pt}%
\pgfpathmoveto{\pgfqpoint{5.127736in}{1.893204in}}%
\pgfpathlineto{\pgfqpoint{5.141815in}{1.898178in}}%
\pgfpathlineto{\pgfqpoint{5.155908in}{1.903264in}}%
\pgfpathlineto{\pgfqpoint{5.170015in}{1.908461in}}%
\pgfpathlineto{\pgfqpoint{5.184137in}{1.913770in}}%
\pgfpathlineto{\pgfqpoint{5.191719in}{1.927330in}}%
\pgfpathlineto{\pgfqpoint{5.199297in}{1.940852in}}%
\pgfpathlineto{\pgfqpoint{5.206869in}{1.954334in}}%
\pgfpathlineto{\pgfqpoint{5.214437in}{1.967775in}}%
\pgfpathlineto{\pgfqpoint{5.200314in}{1.962245in}}%
\pgfpathlineto{\pgfqpoint{5.186205in}{1.956827in}}%
\pgfpathlineto{\pgfqpoint{5.172111in}{1.951520in}}%
\pgfpathlineto{\pgfqpoint{5.158030in}{1.946325in}}%
\pgfpathlineto{\pgfqpoint{5.150464in}{1.933099in}}%
\pgfpathlineto{\pgfqpoint{5.142893in}{1.919836in}}%
\pgfpathlineto{\pgfqpoint{5.135317in}{1.906537in}}%
\pgfpathlineto{\pgfqpoint{5.127736in}{1.893204in}}%
\pgfpathclose%
\pgfusepath{fill}%
\end{pgfscope}%
\begin{pgfscope}%
\pgfpathrectangle{\pgfqpoint{1.254980in}{0.150000in}}{\pgfqpoint{5.490039in}{5.490039in}}%
\pgfusepath{clip}%
\pgfsetbuttcap%
\pgfsetroundjoin%
\definecolor{currentfill}{rgb}{0.270595,0.214069,0.507052}%
\pgfsetfillcolor{currentfill}%
\pgfsetfillopacity{0.700000}%
\pgfsetlinewidth{0.000000pt}%
\definecolor{currentstroke}{rgb}{0.000000,0.000000,0.000000}%
\pgfsetstrokecolor{currentstroke}%
\pgfsetdash{}{0pt}%
\pgfpathmoveto{\pgfqpoint{4.924284in}{1.701477in}}%
\pgfpathlineto{\pgfqpoint{4.938255in}{1.704803in}}%
\pgfpathlineto{\pgfqpoint{4.952239in}{1.708241in}}%
\pgfpathlineto{\pgfqpoint{4.966235in}{1.711790in}}%
\pgfpathlineto{\pgfqpoint{4.980244in}{1.715450in}}%
\pgfpathlineto{\pgfqpoint{4.987877in}{1.728776in}}%
\pgfpathlineto{\pgfqpoint{4.995507in}{1.742095in}}%
\pgfpathlineto{\pgfqpoint{5.003132in}{1.755403in}}%
\pgfpathlineto{\pgfqpoint{5.010752in}{1.768700in}}%
\pgfpathlineto{\pgfqpoint{4.996742in}{1.764772in}}%
\pgfpathlineto{\pgfqpoint{4.982745in}{1.760955in}}%
\pgfpathlineto{\pgfqpoint{4.968761in}{1.757249in}}%
\pgfpathlineto{\pgfqpoint{4.954790in}{1.753655in}}%
\pgfpathlineto{\pgfqpoint{4.947170in}{1.740620in}}%
\pgfpathlineto{\pgfqpoint{4.939546in}{1.727577in}}%
\pgfpathlineto{\pgfqpoint{4.931917in}{1.714529in}}%
\pgfpathlineto{\pgfqpoint{4.924284in}{1.701477in}}%
\pgfpathclose%
\pgfusepath{fill}%
\end{pgfscope}%
\begin{pgfscope}%
\pgfpathrectangle{\pgfqpoint{1.254980in}{0.150000in}}{\pgfqpoint{5.490039in}{5.490039in}}%
\pgfusepath{clip}%
\pgfsetbuttcap%
\pgfsetroundjoin%
\definecolor{currentfill}{rgb}{0.120638,0.625828,0.533488}%
\pgfsetfillcolor{currentfill}%
\pgfsetfillopacity{0.700000}%
\pgfsetlinewidth{0.000000pt}%
\definecolor{currentstroke}{rgb}{0.000000,0.000000,0.000000}%
\pgfsetstrokecolor{currentstroke}%
\pgfsetdash{}{0pt}%
\pgfpathmoveto{\pgfqpoint{2.383322in}{2.837267in}}%
\pgfpathlineto{\pgfqpoint{2.397223in}{2.814375in}}%
\pgfpathlineto{\pgfqpoint{2.411114in}{2.791675in}}%
\pgfpathlineto{\pgfqpoint{2.424995in}{2.769168in}}%
\pgfpathlineto{\pgfqpoint{2.438867in}{2.746852in}}%
\pgfpathlineto{\pgfqpoint{2.447917in}{2.737566in}}%
\pgfpathlineto{\pgfqpoint{2.456945in}{2.728609in}}%
\pgfpathlineto{\pgfqpoint{2.465950in}{2.719978in}}%
\pgfpathlineto{\pgfqpoint{2.474933in}{2.711666in}}%
\pgfpathlineto{\pgfqpoint{2.461120in}{2.733446in}}%
\pgfpathlineto{\pgfqpoint{2.447298in}{2.755415in}}%
\pgfpathlineto{\pgfqpoint{2.433466in}{2.777575in}}%
\pgfpathlineto{\pgfqpoint{2.419625in}{2.799927in}}%
\pgfpathlineto{\pgfqpoint{2.410584in}{2.808768in}}%
\pgfpathlineto{\pgfqpoint{2.401520in}{2.817936in}}%
\pgfpathlineto{\pgfqpoint{2.392433in}{2.827434in}}%
\pgfpathlineto{\pgfqpoint{2.383322in}{2.837267in}}%
\pgfpathclose%
\pgfusepath{fill}%
\end{pgfscope}%
\begin{pgfscope}%
\pgfpathrectangle{\pgfqpoint{1.254980in}{0.150000in}}{\pgfqpoint{5.490039in}{5.490039in}}%
\pgfusepath{clip}%
\pgfsetbuttcap%
\pgfsetroundjoin%
\definecolor{currentfill}{rgb}{0.283091,0.110553,0.431554}%
\pgfsetfillcolor{currentfill}%
\pgfsetfillopacity{0.700000}%
\pgfsetlinewidth{0.000000pt}%
\definecolor{currentstroke}{rgb}{0.000000,0.000000,0.000000}%
\pgfsetstrokecolor{currentstroke}%
\pgfsetdash{}{0pt}%
\pgfpathmoveto{\pgfqpoint{3.477278in}{1.532382in}}%
\pgfpathlineto{\pgfqpoint{3.490872in}{1.521949in}}%
\pgfpathlineto{\pgfqpoint{3.504468in}{1.511645in}}%
\pgfpathlineto{\pgfqpoint{3.518065in}{1.501468in}}%
\pgfpathlineto{\pgfqpoint{3.531664in}{1.491418in}}%
\pgfpathlineto{\pgfqpoint{3.539835in}{1.492329in}}%
\pgfpathlineto{\pgfqpoint{3.547995in}{1.493467in}}%
\pgfpathlineto{\pgfqpoint{3.556143in}{1.494829in}}%
\pgfpathlineto{\pgfqpoint{3.564282in}{1.496411in}}%
\pgfpathlineto{\pgfqpoint{3.550712in}{1.505981in}}%
\pgfpathlineto{\pgfqpoint{3.537145in}{1.515677in}}%
\pgfpathlineto{\pgfqpoint{3.523579in}{1.525501in}}%
\pgfpathlineto{\pgfqpoint{3.510015in}{1.535453in}}%
\pgfpathlineto{\pgfqpoint{3.501848in}{1.534345in}}%
\pgfpathlineto{\pgfqpoint{3.493669in}{1.533461in}}%
\pgfpathlineto{\pgfqpoint{3.485479in}{1.532805in}}%
\pgfpathlineto{\pgfqpoint{3.477278in}{1.532382in}}%
\pgfpathclose%
\pgfusepath{fill}%
\end{pgfscope}%
\begin{pgfscope}%
\pgfpathrectangle{\pgfqpoint{1.254980in}{0.150000in}}{\pgfqpoint{5.490039in}{5.490039in}}%
\pgfusepath{clip}%
\pgfsetbuttcap%
\pgfsetroundjoin%
\definecolor{currentfill}{rgb}{0.216210,0.351535,0.550627}%
\pgfsetfillcolor{currentfill}%
\pgfsetfillopacity{0.700000}%
\pgfsetlinewidth{0.000000pt}%
\definecolor{currentstroke}{rgb}{0.000000,0.000000,0.000000}%
\pgfsetstrokecolor{currentstroke}%
\pgfsetdash{}{0pt}%
\pgfpathmoveto{\pgfqpoint{2.898820in}{2.082111in}}%
\pgfpathlineto{\pgfqpoint{2.912502in}{2.065567in}}%
\pgfpathlineto{\pgfqpoint{2.926180in}{2.049178in}}%
\pgfpathlineto{\pgfqpoint{2.939854in}{2.032940in}}%
\pgfpathlineto{\pgfqpoint{2.953524in}{2.016855in}}%
\pgfpathlineto{\pgfqpoint{2.962130in}{2.011661in}}%
\pgfpathlineto{\pgfqpoint{2.970718in}{2.006761in}}%
\pgfpathlineto{\pgfqpoint{2.979290in}{2.002151in}}%
\pgfpathlineto{\pgfqpoint{2.987844in}{1.997826in}}%
\pgfpathlineto{\pgfqpoint{2.974219in}{2.013391in}}%
\pgfpathlineto{\pgfqpoint{2.960590in}{2.029107in}}%
\pgfpathlineto{\pgfqpoint{2.946958in}{2.044975in}}%
\pgfpathlineto{\pgfqpoint{2.933323in}{2.060996in}}%
\pgfpathlineto{\pgfqpoint{2.924724in}{2.065835in}}%
\pgfpathlineto{\pgfqpoint{2.916107in}{2.070964in}}%
\pgfpathlineto{\pgfqpoint{2.907472in}{2.076388in}}%
\pgfpathlineto{\pgfqpoint{2.898820in}{2.082111in}}%
\pgfpathclose%
\pgfusepath{fill}%
\end{pgfscope}%
\begin{pgfscope}%
\pgfpathrectangle{\pgfqpoint{1.254980in}{0.150000in}}{\pgfqpoint{5.490039in}{5.490039in}}%
\pgfusepath{clip}%
\pgfsetbuttcap%
\pgfsetroundjoin%
\definecolor{currentfill}{rgb}{0.227802,0.326594,0.546532}%
\pgfsetfillcolor{currentfill}%
\pgfsetfillopacity{0.700000}%
\pgfsetlinewidth{0.000000pt}%
\definecolor{currentstroke}{rgb}{0.000000,0.000000,0.000000}%
\pgfsetstrokecolor{currentstroke}%
\pgfsetdash{}{0pt}%
\pgfpathmoveto{\pgfqpoint{2.953524in}{2.016855in}}%
\pgfpathlineto{\pgfqpoint{2.967191in}{2.000921in}}%
\pgfpathlineto{\pgfqpoint{2.980855in}{1.985137in}}%
\pgfpathlineto{\pgfqpoint{2.994516in}{1.969502in}}%
\pgfpathlineto{\pgfqpoint{3.008173in}{1.954017in}}%
\pgfpathlineto{\pgfqpoint{3.016733in}{1.949349in}}%
\pgfpathlineto{\pgfqpoint{3.025277in}{1.944970in}}%
\pgfpathlineto{\pgfqpoint{3.033804in}{1.940875in}}%
\pgfpathlineto{\pgfqpoint{3.042315in}{1.937060in}}%
\pgfpathlineto{\pgfqpoint{3.028701in}{1.952029in}}%
\pgfpathlineto{\pgfqpoint{3.015085in}{1.967145in}}%
\pgfpathlineto{\pgfqpoint{3.001466in}{1.982411in}}%
\pgfpathlineto{\pgfqpoint{2.987844in}{1.997826in}}%
\pgfpathlineto{\pgfqpoint{2.979290in}{2.002151in}}%
\pgfpathlineto{\pgfqpoint{2.970718in}{2.006761in}}%
\pgfpathlineto{\pgfqpoint{2.962130in}{2.011661in}}%
\pgfpathlineto{\pgfqpoint{2.953524in}{2.016855in}}%
\pgfpathclose%
\pgfusepath{fill}%
\end{pgfscope}%
\begin{pgfscope}%
\pgfpathrectangle{\pgfqpoint{1.254980in}{0.150000in}}{\pgfqpoint{5.490039in}{5.490039in}}%
\pgfusepath{clip}%
\pgfsetbuttcap%
\pgfsetroundjoin%
\definecolor{currentfill}{rgb}{0.267004,0.004874,0.329415}%
\pgfsetfillcolor{currentfill}%
\pgfsetfillopacity{0.700000}%
\pgfsetlinewidth{0.000000pt}%
\definecolor{currentstroke}{rgb}{0.000000,0.000000,0.000000}%
\pgfsetstrokecolor{currentstroke}%
\pgfsetdash{}{0pt}%
\pgfpathmoveto{\pgfqpoint{4.149660in}{1.315586in}}%
\pgfpathlineto{\pgfqpoint{4.163332in}{1.311777in}}%
\pgfpathlineto{\pgfqpoint{4.177010in}{1.308082in}}%
\pgfpathlineto{\pgfqpoint{4.190696in}{1.304501in}}%
\pgfpathlineto{\pgfqpoint{4.204388in}{1.301034in}}%
\pgfpathlineto{\pgfqpoint{4.212224in}{1.309253in}}%
\pgfpathlineto{\pgfqpoint{4.220054in}{1.317598in}}%
\pgfpathlineto{\pgfqpoint{4.227878in}{1.326064in}}%
\pgfpathlineto{\pgfqpoint{4.235697in}{1.334649in}}%
\pgfpathlineto{\pgfqpoint{4.222016in}{1.337709in}}%
\pgfpathlineto{\pgfqpoint{4.208343in}{1.340883in}}%
\pgfpathlineto{\pgfqpoint{4.194677in}{1.344171in}}%
\pgfpathlineto{\pgfqpoint{4.181018in}{1.347574in}}%
\pgfpathlineto{\pgfqpoint{4.173187in}{1.339390in}}%
\pgfpathlineto{\pgfqpoint{4.165351in}{1.331328in}}%
\pgfpathlineto{\pgfqpoint{4.157508in}{1.323393in}}%
\pgfpathlineto{\pgfqpoint{4.149660in}{1.315586in}}%
\pgfpathclose%
\pgfusepath{fill}%
\end{pgfscope}%
\begin{pgfscope}%
\pgfpathrectangle{\pgfqpoint{1.254980in}{0.150000in}}{\pgfqpoint{5.490039in}{5.490039in}}%
\pgfusepath{clip}%
\pgfsetbuttcap%
\pgfsetroundjoin%
\definecolor{currentfill}{rgb}{0.204903,0.375746,0.553533}%
\pgfsetfillcolor{currentfill}%
\pgfsetfillopacity{0.700000}%
\pgfsetlinewidth{0.000000pt}%
\definecolor{currentstroke}{rgb}{0.000000,0.000000,0.000000}%
\pgfsetstrokecolor{currentstroke}%
\pgfsetdash{}{0pt}%
\pgfpathmoveto{\pgfqpoint{2.844051in}{2.149834in}}%
\pgfpathlineto{\pgfqpoint{2.857750in}{2.132669in}}%
\pgfpathlineto{\pgfqpoint{2.871444in}{2.115661in}}%
\pgfpathlineto{\pgfqpoint{2.885134in}{2.098808in}}%
\pgfpathlineto{\pgfqpoint{2.898820in}{2.082111in}}%
\pgfpathlineto{\pgfqpoint{2.907472in}{2.076388in}}%
\pgfpathlineto{\pgfqpoint{2.916107in}{2.070964in}}%
\pgfpathlineto{\pgfqpoint{2.924724in}{2.065835in}}%
\pgfpathlineto{\pgfqpoint{2.933323in}{2.060996in}}%
\pgfpathlineto{\pgfqpoint{2.919684in}{2.077171in}}%
\pgfpathlineto{\pgfqpoint{2.906041in}{2.093499in}}%
\pgfpathlineto{\pgfqpoint{2.892395in}{2.109983in}}%
\pgfpathlineto{\pgfqpoint{2.878744in}{2.126622in}}%
\pgfpathlineto{\pgfqpoint{2.870098in}{2.131978in}}%
\pgfpathlineto{\pgfqpoint{2.861434in}{2.137628in}}%
\pgfpathlineto{\pgfqpoint{2.852752in}{2.143579in}}%
\pgfpathlineto{\pgfqpoint{2.844051in}{2.149834in}}%
\pgfpathclose%
\pgfusepath{fill}%
\end{pgfscope}%
\begin{pgfscope}%
\pgfpathrectangle{\pgfqpoint{1.254980in}{0.150000in}}{\pgfqpoint{5.490039in}{5.490039in}}%
\pgfusepath{clip}%
\pgfsetbuttcap%
\pgfsetroundjoin%
\definecolor{currentfill}{rgb}{0.239346,0.300855,0.540844}%
\pgfsetfillcolor{currentfill}%
\pgfsetfillopacity{0.700000}%
\pgfsetlinewidth{0.000000pt}%
\definecolor{currentstroke}{rgb}{0.000000,0.000000,0.000000}%
\pgfsetstrokecolor{currentstroke}%
\pgfsetdash{}{0pt}%
\pgfpathmoveto{\pgfqpoint{3.008173in}{1.954017in}}%
\pgfpathlineto{\pgfqpoint{3.021828in}{1.938679in}}%
\pgfpathlineto{\pgfqpoint{3.035479in}{1.923489in}}%
\pgfpathlineto{\pgfqpoint{3.049128in}{1.908446in}}%
\pgfpathlineto{\pgfqpoint{3.062775in}{1.893549in}}%
\pgfpathlineto{\pgfqpoint{3.071291in}{1.889404in}}%
\pgfpathlineto{\pgfqpoint{3.079792in}{1.885543in}}%
\pgfpathlineto{\pgfqpoint{3.088276in}{1.881961in}}%
\pgfpathlineto{\pgfqpoint{3.096745in}{1.878654in}}%
\pgfpathlineto{\pgfqpoint{3.083141in}{1.893037in}}%
\pgfpathlineto{\pgfqpoint{3.069534in}{1.907565in}}%
\pgfpathlineto{\pgfqpoint{3.055926in}{1.922240in}}%
\pgfpathlineto{\pgfqpoint{3.042315in}{1.937060in}}%
\pgfpathlineto{\pgfqpoint{3.033804in}{1.940875in}}%
\pgfpathlineto{\pgfqpoint{3.025277in}{1.944970in}}%
\pgfpathlineto{\pgfqpoint{3.016733in}{1.949349in}}%
\pgfpathlineto{\pgfqpoint{3.008173in}{1.954017in}}%
\pgfpathclose%
\pgfusepath{fill}%
\end{pgfscope}%
\begin{pgfscope}%
\pgfpathrectangle{\pgfqpoint{1.254980in}{0.150000in}}{\pgfqpoint{5.490039in}{5.490039in}}%
\pgfusepath{clip}%
\pgfsetbuttcap%
\pgfsetroundjoin%
\definecolor{currentfill}{rgb}{0.250425,0.274290,0.533103}%
\pgfsetfillcolor{currentfill}%
\pgfsetfillopacity{0.700000}%
\pgfsetlinewidth{0.000000pt}%
\definecolor{currentstroke}{rgb}{0.000000,0.000000,0.000000}%
\pgfsetstrokecolor{currentstroke}%
\pgfsetdash{}{0pt}%
\pgfpathmoveto{\pgfqpoint{3.062775in}{1.893549in}}%
\pgfpathlineto{\pgfqpoint{3.076419in}{1.878797in}}%
\pgfpathlineto{\pgfqpoint{3.090061in}{1.864189in}}%
\pgfpathlineto{\pgfqpoint{3.103700in}{1.849725in}}%
\pgfpathlineto{\pgfqpoint{3.117338in}{1.835405in}}%
\pgfpathlineto{\pgfqpoint{3.125812in}{1.831781in}}%
\pgfpathlineto{\pgfqpoint{3.134270in}{1.828435in}}%
\pgfpathlineto{\pgfqpoint{3.142714in}{1.825364in}}%
\pgfpathlineto{\pgfqpoint{3.151142in}{1.822562in}}%
\pgfpathlineto{\pgfqpoint{3.137545in}{1.836370in}}%
\pgfpathlineto{\pgfqpoint{3.123947in}{1.850322in}}%
\pgfpathlineto{\pgfqpoint{3.110347in}{1.864416in}}%
\pgfpathlineto{\pgfqpoint{3.096745in}{1.878654in}}%
\pgfpathlineto{\pgfqpoint{3.088276in}{1.881961in}}%
\pgfpathlineto{\pgfqpoint{3.079792in}{1.885543in}}%
\pgfpathlineto{\pgfqpoint{3.071291in}{1.889404in}}%
\pgfpathlineto{\pgfqpoint{3.062775in}{1.893549in}}%
\pgfpathclose%
\pgfusepath{fill}%
\end{pgfscope}%
\begin{pgfscope}%
\pgfpathrectangle{\pgfqpoint{1.254980in}{0.150000in}}{\pgfqpoint{5.490039in}{5.490039in}}%
\pgfusepath{clip}%
\pgfsetbuttcap%
\pgfsetroundjoin%
\definecolor{currentfill}{rgb}{0.192357,0.403199,0.555836}%
\pgfsetfillcolor{currentfill}%
\pgfsetfillopacity{0.700000}%
\pgfsetlinewidth{0.000000pt}%
\definecolor{currentstroke}{rgb}{0.000000,0.000000,0.000000}%
\pgfsetstrokecolor{currentstroke}%
\pgfsetdash{}{0pt}%
\pgfpathmoveto{\pgfqpoint{2.789208in}{2.220078in}}%
\pgfpathlineto{\pgfqpoint{2.802926in}{2.202277in}}%
\pgfpathlineto{\pgfqpoint{2.816639in}{2.184637in}}%
\pgfpathlineto{\pgfqpoint{2.830348in}{2.167157in}}%
\pgfpathlineto{\pgfqpoint{2.844051in}{2.149834in}}%
\pgfpathlineto{\pgfqpoint{2.852752in}{2.143579in}}%
\pgfpathlineto{\pgfqpoint{2.861434in}{2.137628in}}%
\pgfpathlineto{\pgfqpoint{2.870098in}{2.131978in}}%
\pgfpathlineto{\pgfqpoint{2.878744in}{2.126622in}}%
\pgfpathlineto{\pgfqpoint{2.865089in}{2.143418in}}%
\pgfpathlineto{\pgfqpoint{2.851429in}{2.160372in}}%
\pgfpathlineto{\pgfqpoint{2.837766in}{2.177484in}}%
\pgfpathlineto{\pgfqpoint{2.824097in}{2.194755in}}%
\pgfpathlineto{\pgfqpoint{2.815403in}{2.200630in}}%
\pgfpathlineto{\pgfqpoint{2.806691in}{2.206806in}}%
\pgfpathlineto{\pgfqpoint{2.797959in}{2.213287in}}%
\pgfpathlineto{\pgfqpoint{2.789208in}{2.220078in}}%
\pgfpathclose%
\pgfusepath{fill}%
\end{pgfscope}%
\begin{pgfscope}%
\pgfpathrectangle{\pgfqpoint{1.254980in}{0.150000in}}{\pgfqpoint{5.490039in}{5.490039in}}%
\pgfusepath{clip}%
\pgfsetbuttcap%
\pgfsetroundjoin%
\definecolor{currentfill}{rgb}{0.280894,0.078907,0.402329}%
\pgfsetfillcolor{currentfill}%
\pgfsetfillopacity{0.700000}%
\pgfsetlinewidth{0.000000pt}%
\definecolor{currentstroke}{rgb}{0.000000,0.000000,0.000000}%
\pgfsetstrokecolor{currentstroke}%
\pgfsetdash{}{0pt}%
\pgfpathmoveto{\pgfqpoint{4.548694in}{1.430143in}}%
\pgfpathlineto{\pgfqpoint{4.562496in}{1.430120in}}%
\pgfpathlineto{\pgfqpoint{4.576309in}{1.430210in}}%
\pgfpathlineto{\pgfqpoint{4.590131in}{1.430410in}}%
\pgfpathlineto{\pgfqpoint{4.603963in}{1.430721in}}%
\pgfpathlineto{\pgfqpoint{4.611684in}{1.442310in}}%
\pgfpathlineto{\pgfqpoint{4.619400in}{1.453956in}}%
\pgfpathlineto{\pgfqpoint{4.627112in}{1.465654in}}%
\pgfpathlineto{\pgfqpoint{4.634819in}{1.477404in}}%
\pgfpathlineto{\pgfqpoint{4.620991in}{1.476747in}}%
\pgfpathlineto{\pgfqpoint{4.607173in}{1.476202in}}%
\pgfpathlineto{\pgfqpoint{4.593365in}{1.475768in}}%
\pgfpathlineto{\pgfqpoint{4.579567in}{1.475445in}}%
\pgfpathlineto{\pgfqpoint{4.571855in}{1.464035in}}%
\pgfpathlineto{\pgfqpoint{4.564139in}{1.452679in}}%
\pgfpathlineto{\pgfqpoint{4.556419in}{1.441381in}}%
\pgfpathlineto{\pgfqpoint{4.548694in}{1.430143in}}%
\pgfpathclose%
\pgfusepath{fill}%
\end{pgfscope}%
\begin{pgfscope}%
\pgfpathrectangle{\pgfqpoint{1.254980in}{0.150000in}}{\pgfqpoint{5.490039in}{5.490039in}}%
\pgfusepath{clip}%
\pgfsetbuttcap%
\pgfsetroundjoin%
\definecolor{currentfill}{rgb}{0.274952,0.037752,0.364543}%
\pgfsetfillcolor{currentfill}%
\pgfsetfillopacity{0.700000}%
\pgfsetlinewidth{0.000000pt}%
\definecolor{currentstroke}{rgb}{0.000000,0.000000,0.000000}%
\pgfsetstrokecolor{currentstroke}%
\pgfsetdash{}{0pt}%
\pgfpathmoveto{\pgfqpoint{3.727308in}{1.391333in}}%
\pgfpathlineto{\pgfqpoint{3.740913in}{1.383377in}}%
\pgfpathlineto{\pgfqpoint{3.754521in}{1.375543in}}%
\pgfpathlineto{\pgfqpoint{3.768133in}{1.367830in}}%
\pgfpathlineto{\pgfqpoint{3.781748in}{1.360238in}}%
\pgfpathlineto{\pgfqpoint{3.789774in}{1.363901in}}%
\pgfpathlineto{\pgfqpoint{3.797790in}{1.367758in}}%
\pgfpathlineto{\pgfqpoint{3.805798in}{1.371806in}}%
\pgfpathlineto{\pgfqpoint{3.813797in}{1.376039in}}%
\pgfpathlineto{\pgfqpoint{3.800204in}{1.383173in}}%
\pgfpathlineto{\pgfqpoint{3.786616in}{1.390427in}}%
\pgfpathlineto{\pgfqpoint{3.773030in}{1.397803in}}%
\pgfpathlineto{\pgfqpoint{3.759449in}{1.405300in}}%
\pgfpathlineto{\pgfqpoint{3.751427in}{1.401518in}}%
\pgfpathlineto{\pgfqpoint{3.743396in}{1.397928in}}%
\pgfpathlineto{\pgfqpoint{3.735357in}{1.394531in}}%
\pgfpathlineto{\pgfqpoint{3.727308in}{1.391333in}}%
\pgfpathclose%
\pgfusepath{fill}%
\end{pgfscope}%
\begin{pgfscope}%
\pgfpathrectangle{\pgfqpoint{1.254980in}{0.150000in}}{\pgfqpoint{5.490039in}{5.490039in}}%
\pgfusepath{clip}%
\pgfsetbuttcap%
\pgfsetroundjoin%
\definecolor{currentfill}{rgb}{0.277941,0.056324,0.381191}%
\pgfsetfillcolor{currentfill}%
\pgfsetfillopacity{0.700000}%
\pgfsetlinewidth{0.000000pt}%
\definecolor{currentstroke}{rgb}{0.000000,0.000000,0.000000}%
\pgfsetstrokecolor{currentstroke}%
\pgfsetdash{}{0pt}%
\pgfpathmoveto{\pgfqpoint{4.462616in}{1.388493in}}%
\pgfpathlineto{\pgfqpoint{4.476386in}{1.387664in}}%
\pgfpathlineto{\pgfqpoint{4.490166in}{1.386947in}}%
\pgfpathlineto{\pgfqpoint{4.503954in}{1.386342in}}%
\pgfpathlineto{\pgfqpoint{4.517752in}{1.385847in}}%
\pgfpathlineto{\pgfqpoint{4.525494in}{1.396817in}}%
\pgfpathlineto{\pgfqpoint{4.533232in}{1.407858in}}%
\pgfpathlineto{\pgfqpoint{4.540965in}{1.418967in}}%
\pgfpathlineto{\pgfqpoint{4.548694in}{1.430143in}}%
\pgfpathlineto{\pgfqpoint{4.534901in}{1.430276in}}%
\pgfpathlineto{\pgfqpoint{4.521118in}{1.430522in}}%
\pgfpathlineto{\pgfqpoint{4.507345in}{1.430878in}}%
\pgfpathlineto{\pgfqpoint{4.493581in}{1.431347in}}%
\pgfpathlineto{\pgfqpoint{4.485846in}{1.420526in}}%
\pgfpathlineto{\pgfqpoint{4.478107in}{1.409775in}}%
\pgfpathlineto{\pgfqpoint{4.470364in}{1.399096in}}%
\pgfpathlineto{\pgfqpoint{4.462616in}{1.388493in}}%
\pgfpathclose%
\pgfusepath{fill}%
\end{pgfscope}%
\begin{pgfscope}%
\pgfpathrectangle{\pgfqpoint{1.254980in}{0.150000in}}{\pgfqpoint{5.490039in}{5.490039in}}%
\pgfusepath{clip}%
\pgfsetbuttcap%
\pgfsetroundjoin%
\definecolor{currentfill}{rgb}{0.282910,0.105393,0.426902}%
\pgfsetfillcolor{currentfill}%
\pgfsetfillopacity{0.700000}%
\pgfsetlinewidth{0.000000pt}%
\definecolor{currentstroke}{rgb}{0.000000,0.000000,0.000000}%
\pgfsetstrokecolor{currentstroke}%
\pgfsetdash{}{0pt}%
\pgfpathmoveto{\pgfqpoint{4.634819in}{1.477404in}}%
\pgfpathlineto{\pgfqpoint{4.648658in}{1.478171in}}%
\pgfpathlineto{\pgfqpoint{4.662508in}{1.479050in}}%
\pgfpathlineto{\pgfqpoint{4.676368in}{1.480040in}}%
\pgfpathlineto{\pgfqpoint{4.690238in}{1.481140in}}%
\pgfpathlineto{\pgfqpoint{4.697939in}{1.493274in}}%
\pgfpathlineto{\pgfqpoint{4.705635in}{1.505449in}}%
\pgfpathlineto{\pgfqpoint{4.713328in}{1.517662in}}%
\pgfpathlineto{\pgfqpoint{4.721016in}{1.529912in}}%
\pgfpathlineto{\pgfqpoint{4.707148in}{1.528482in}}%
\pgfpathlineto{\pgfqpoint{4.693291in}{1.527162in}}%
\pgfpathlineto{\pgfqpoint{4.679444in}{1.525954in}}%
\pgfpathlineto{\pgfqpoint{4.665609in}{1.524856in}}%
\pgfpathlineto{\pgfqpoint{4.657918in}{1.512930in}}%
\pgfpathlineto{\pgfqpoint{4.650222in}{1.501044in}}%
\pgfpathlineto{\pgfqpoint{4.642523in}{1.489201in}}%
\pgfpathlineto{\pgfqpoint{4.634819in}{1.477404in}}%
\pgfpathclose%
\pgfusepath{fill}%
\end{pgfscope}%
\begin{pgfscope}%
\pgfpathrectangle{\pgfqpoint{1.254980in}{0.150000in}}{\pgfqpoint{5.490039in}{5.490039in}}%
\pgfusepath{clip}%
\pgfsetbuttcap%
\pgfsetroundjoin%
\definecolor{currentfill}{rgb}{0.335885,0.777018,0.402049}%
\pgfsetfillcolor{currentfill}%
\pgfsetfillopacity{0.700000}%
\pgfsetlinewidth{0.000000pt}%
\definecolor{currentstroke}{rgb}{0.000000,0.000000,0.000000}%
\pgfsetstrokecolor{currentstroke}%
\pgfsetdash{}{0pt}%
\pgfpathmoveto{\pgfqpoint{2.140510in}{3.289415in}}%
\pgfpathlineto{\pgfqpoint{2.154582in}{3.262928in}}%
\pgfpathlineto{\pgfqpoint{2.168641in}{3.236662in}}%
\pgfpathlineto{\pgfqpoint{2.182687in}{3.210615in}}%
\pgfpathlineto{\pgfqpoint{2.196719in}{3.184787in}}%
\pgfpathlineto{\pgfqpoint{2.205984in}{3.174118in}}%
\pgfpathlineto{\pgfqpoint{2.215223in}{3.163790in}}%
\pgfpathlineto{\pgfqpoint{2.224438in}{3.153796in}}%
\pgfpathlineto{\pgfqpoint{2.233628in}{3.144134in}}%
\pgfpathlineto{\pgfqpoint{2.219659in}{3.169423in}}%
\pgfpathlineto{\pgfqpoint{2.205679in}{3.194928in}}%
\pgfpathlineto{\pgfqpoint{2.191685in}{3.220651in}}%
\pgfpathlineto{\pgfqpoint{2.177678in}{3.246594in}}%
\pgfpathlineto{\pgfqpoint{2.168424in}{3.256790in}}%
\pgfpathlineto{\pgfqpoint{2.159145in}{3.267322in}}%
\pgfpathlineto{\pgfqpoint{2.149841in}{3.278196in}}%
\pgfpathlineto{\pgfqpoint{2.140510in}{3.289415in}}%
\pgfpathclose%
\pgfusepath{fill}%
\end{pgfscope}%
\begin{pgfscope}%
\pgfpathrectangle{\pgfqpoint{1.254980in}{0.150000in}}{\pgfqpoint{5.490039in}{5.490039in}}%
\pgfusepath{clip}%
\pgfsetbuttcap%
\pgfsetroundjoin%
\definecolor{currentfill}{rgb}{0.258965,0.251537,0.524736}%
\pgfsetfillcolor{currentfill}%
\pgfsetfillopacity{0.700000}%
\pgfsetlinewidth{0.000000pt}%
\definecolor{currentstroke}{rgb}{0.000000,0.000000,0.000000}%
\pgfsetstrokecolor{currentstroke}%
\pgfsetdash{}{0pt}%
\pgfpathmoveto{\pgfqpoint{3.117338in}{1.835405in}}%
\pgfpathlineto{\pgfqpoint{3.130974in}{1.821227in}}%
\pgfpathlineto{\pgfqpoint{3.144608in}{1.807191in}}%
\pgfpathlineto{\pgfqpoint{3.158240in}{1.793297in}}%
\pgfpathlineto{\pgfqpoint{3.171871in}{1.779543in}}%
\pgfpathlineto{\pgfqpoint{3.180304in}{1.776436in}}%
\pgfpathlineto{\pgfqpoint{3.188722in}{1.773604in}}%
\pgfpathlineto{\pgfqpoint{3.197125in}{1.771040in}}%
\pgfpathlineto{\pgfqpoint{3.205514in}{1.768742in}}%
\pgfpathlineto{\pgfqpoint{3.191923in}{1.781986in}}%
\pgfpathlineto{\pgfqpoint{3.178330in}{1.795370in}}%
\pgfpathlineto{\pgfqpoint{3.164737in}{1.808896in}}%
\pgfpathlineto{\pgfqpoint{3.151142in}{1.822562in}}%
\pgfpathlineto{\pgfqpoint{3.142714in}{1.825364in}}%
\pgfpathlineto{\pgfqpoint{3.134270in}{1.828435in}}%
\pgfpathlineto{\pgfqpoint{3.125812in}{1.831781in}}%
\pgfpathlineto{\pgfqpoint{3.117338in}{1.835405in}}%
\pgfpathclose%
\pgfusepath{fill}%
\end{pgfscope}%
\begin{pgfscope}%
\pgfpathrectangle{\pgfqpoint{1.254980in}{0.150000in}}{\pgfqpoint{5.490039in}{5.490039in}}%
\pgfusepath{clip}%
\pgfsetbuttcap%
\pgfsetroundjoin%
\definecolor{currentfill}{rgb}{0.179019,0.433756,0.557430}%
\pgfsetfillcolor{currentfill}%
\pgfsetfillopacity{0.700000}%
\pgfsetlinewidth{0.000000pt}%
\definecolor{currentstroke}{rgb}{0.000000,0.000000,0.000000}%
\pgfsetstrokecolor{currentstroke}%
\pgfsetdash{}{0pt}%
\pgfpathmoveto{\pgfqpoint{2.734283in}{2.292897in}}%
\pgfpathlineto{\pgfqpoint{2.748023in}{2.274448in}}%
\pgfpathlineto{\pgfqpoint{2.761757in}{2.256162in}}%
\pgfpathlineto{\pgfqpoint{2.775485in}{2.238039in}}%
\pgfpathlineto{\pgfqpoint{2.789208in}{2.220078in}}%
\pgfpathlineto{\pgfqpoint{2.797959in}{2.213287in}}%
\pgfpathlineto{\pgfqpoint{2.806691in}{2.206806in}}%
\pgfpathlineto{\pgfqpoint{2.815403in}{2.200630in}}%
\pgfpathlineto{\pgfqpoint{2.824097in}{2.194755in}}%
\pgfpathlineto{\pgfqpoint{2.810424in}{2.212186in}}%
\pgfpathlineto{\pgfqpoint{2.796745in}{2.229779in}}%
\pgfpathlineto{\pgfqpoint{2.783062in}{2.247533in}}%
\pgfpathlineto{\pgfqpoint{2.769373in}{2.265450in}}%
\pgfpathlineto{\pgfqpoint{2.760630in}{2.271848in}}%
\pgfpathlineto{\pgfqpoint{2.751868in}{2.278552in}}%
\pgfpathlineto{\pgfqpoint{2.743085in}{2.285567in}}%
\pgfpathlineto{\pgfqpoint{2.734283in}{2.292897in}}%
\pgfpathclose%
\pgfusepath{fill}%
\end{pgfscope}%
\begin{pgfscope}%
\pgfpathrectangle{\pgfqpoint{1.254980in}{0.150000in}}{\pgfqpoint{5.490039in}{5.490039in}}%
\pgfusepath{clip}%
\pgfsetbuttcap%
\pgfsetroundjoin%
\definecolor{currentfill}{rgb}{0.273809,0.031497,0.358853}%
\pgfsetfillcolor{currentfill}%
\pgfsetfillopacity{0.700000}%
\pgfsetlinewidth{0.000000pt}%
\definecolor{currentstroke}{rgb}{0.000000,0.000000,0.000000}%
\pgfsetstrokecolor{currentstroke}%
\pgfsetdash{}{0pt}%
\pgfpathmoveto{\pgfqpoint{4.376559in}{1.352830in}}%
\pgfpathlineto{\pgfqpoint{4.390301in}{1.351178in}}%
\pgfpathlineto{\pgfqpoint{4.404052in}{1.349638in}}%
\pgfpathlineto{\pgfqpoint{4.417811in}{1.348209in}}%
\pgfpathlineto{\pgfqpoint{4.431579in}{1.346893in}}%
\pgfpathlineto{\pgfqpoint{4.439345in}{1.357165in}}%
\pgfpathlineto{\pgfqpoint{4.447107in}{1.367524in}}%
\pgfpathlineto{\pgfqpoint{4.454864in}{1.377968in}}%
\pgfpathlineto{\pgfqpoint{4.462616in}{1.388493in}}%
\pgfpathlineto{\pgfqpoint{4.448855in}{1.389433in}}%
\pgfpathlineto{\pgfqpoint{4.435103in}{1.390486in}}%
\pgfpathlineto{\pgfqpoint{4.421360in}{1.391651in}}%
\pgfpathlineto{\pgfqpoint{4.407626in}{1.392927in}}%
\pgfpathlineto{\pgfqpoint{4.399866in}{1.382772in}}%
\pgfpathlineto{\pgfqpoint{4.392102in}{1.372702in}}%
\pgfpathlineto{\pgfqpoint{4.384333in}{1.362720in}}%
\pgfpathlineto{\pgfqpoint{4.376559in}{1.352830in}}%
\pgfpathclose%
\pgfusepath{fill}%
\end{pgfscope}%
\begin{pgfscope}%
\pgfpathrectangle{\pgfqpoint{1.254980in}{0.150000in}}{\pgfqpoint{5.490039in}{5.490039in}}%
\pgfusepath{clip}%
\pgfsetbuttcap%
\pgfsetroundjoin%
\definecolor{currentfill}{rgb}{0.258965,0.251537,0.524736}%
\pgfsetfillcolor{currentfill}%
\pgfsetfillopacity{0.700000}%
\pgfsetlinewidth{0.000000pt}%
\definecolor{currentstroke}{rgb}{0.000000,0.000000,0.000000}%
\pgfsetstrokecolor{currentstroke}%
\pgfsetdash{}{0pt}%
\pgfpathmoveto{\pgfqpoint{5.010752in}{1.768700in}}%
\pgfpathlineto{\pgfqpoint{5.024775in}{1.772740in}}%
\pgfpathlineto{\pgfqpoint{5.038811in}{1.776890in}}%
\pgfpathlineto{\pgfqpoint{5.052860in}{1.781152in}}%
\pgfpathlineto{\pgfqpoint{5.066923in}{1.785524in}}%
\pgfpathlineto{\pgfqpoint{5.074540in}{1.799066in}}%
\pgfpathlineto{\pgfqpoint{5.082153in}{1.812589in}}%
\pgfpathlineto{\pgfqpoint{5.089762in}{1.826090in}}%
\pgfpathlineto{\pgfqpoint{5.097366in}{1.839567in}}%
\pgfpathlineto{\pgfqpoint{5.083302in}{1.834941in}}%
\pgfpathlineto{\pgfqpoint{5.069251in}{1.830427in}}%
\pgfpathlineto{\pgfqpoint{5.055214in}{1.826023in}}%
\pgfpathlineto{\pgfqpoint{5.041190in}{1.821731in}}%
\pgfpathlineto{\pgfqpoint{5.033587in}{1.808501in}}%
\pgfpathlineto{\pgfqpoint{5.025980in}{1.795251in}}%
\pgfpathlineto{\pgfqpoint{5.018368in}{1.781984in}}%
\pgfpathlineto{\pgfqpoint{5.010752in}{1.768700in}}%
\pgfpathclose%
\pgfusepath{fill}%
\end{pgfscope}%
\begin{pgfscope}%
\pgfpathrectangle{\pgfqpoint{1.254980in}{0.150000in}}{\pgfqpoint{5.490039in}{5.490039in}}%
\pgfusepath{clip}%
\pgfsetbuttcap%
\pgfsetroundjoin%
\definecolor{currentfill}{rgb}{0.221989,0.339161,0.548752}%
\pgfsetfillcolor{currentfill}%
\pgfsetfillopacity{0.700000}%
\pgfsetlinewidth{0.000000pt}%
\definecolor{currentstroke}{rgb}{0.000000,0.000000,0.000000}%
\pgfsetstrokecolor{currentstroke}%
\pgfsetdash{}{0pt}%
\pgfpathmoveto{\pgfqpoint{5.214437in}{1.967775in}}%
\pgfpathlineto{\pgfqpoint{5.228575in}{1.973416in}}%
\pgfpathlineto{\pgfqpoint{5.242727in}{1.979170in}}%
\pgfpathlineto{\pgfqpoint{5.256893in}{1.985034in}}%
\pgfpathlineto{\pgfqpoint{5.271074in}{1.991011in}}%
\pgfpathlineto{\pgfqpoint{5.278639in}{2.004621in}}%
\pgfpathlineto{\pgfqpoint{5.286199in}{2.018182in}}%
\pgfpathlineto{\pgfqpoint{5.293754in}{2.031693in}}%
\pgfpathlineto{\pgfqpoint{5.301303in}{2.045153in}}%
\pgfpathlineto{\pgfqpoint{5.287120in}{2.038971in}}%
\pgfpathlineto{\pgfqpoint{5.272951in}{2.032900in}}%
\pgfpathlineto{\pgfqpoint{5.258798in}{2.026942in}}%
\pgfpathlineto{\pgfqpoint{5.244659in}{2.021095in}}%
\pgfpathlineto{\pgfqpoint{5.237111in}{2.007835in}}%
\pgfpathlineto{\pgfqpoint{5.229558in}{1.994527in}}%
\pgfpathlineto{\pgfqpoint{5.222000in}{1.981173in}}%
\pgfpathlineto{\pgfqpoint{5.214437in}{1.967775in}}%
\pgfpathclose%
\pgfusepath{fill}%
\end{pgfscope}%
\begin{pgfscope}%
\pgfpathrectangle{\pgfqpoint{1.254980in}{0.150000in}}{\pgfqpoint{5.490039in}{5.490039in}}%
\pgfusepath{clip}%
\pgfsetbuttcap%
\pgfsetroundjoin%
\definecolor{currentfill}{rgb}{0.282327,0.094955,0.417331}%
\pgfsetfillcolor{currentfill}%
\pgfsetfillopacity{0.700000}%
\pgfsetlinewidth{0.000000pt}%
\definecolor{currentstroke}{rgb}{0.000000,0.000000,0.000000}%
\pgfsetstrokecolor{currentstroke}%
\pgfsetdash{}{0pt}%
\pgfpathmoveto{\pgfqpoint{3.531664in}{1.491418in}}%
\pgfpathlineto{\pgfqpoint{3.545264in}{1.481496in}}%
\pgfpathlineto{\pgfqpoint{3.558867in}{1.471700in}}%
\pgfpathlineto{\pgfqpoint{3.572471in}{1.462030in}}%
\pgfpathlineto{\pgfqpoint{3.586078in}{1.452485in}}%
\pgfpathlineto{\pgfqpoint{3.594219in}{1.453882in}}%
\pgfpathlineto{\pgfqpoint{3.602350in}{1.455502in}}%
\pgfpathlineto{\pgfqpoint{3.610471in}{1.457341in}}%
\pgfpathlineto{\pgfqpoint{3.618581in}{1.459395in}}%
\pgfpathlineto{\pgfqpoint{3.605003in}{1.468461in}}%
\pgfpathlineto{\pgfqpoint{3.591427in}{1.477652in}}%
\pgfpathlineto{\pgfqpoint{3.577853in}{1.486968in}}%
\pgfpathlineto{\pgfqpoint{3.564282in}{1.496411in}}%
\pgfpathlineto{\pgfqpoint{3.556143in}{1.494829in}}%
\pgfpathlineto{\pgfqpoint{3.547995in}{1.493467in}}%
\pgfpathlineto{\pgfqpoint{3.539835in}{1.492329in}}%
\pgfpathlineto{\pgfqpoint{3.531664in}{1.491418in}}%
\pgfpathclose%
\pgfusepath{fill}%
\end{pgfscope}%
\begin{pgfscope}%
\pgfpathrectangle{\pgfqpoint{1.254980in}{0.150000in}}{\pgfqpoint{5.490039in}{5.490039in}}%
\pgfusepath{clip}%
\pgfsetbuttcap%
\pgfsetroundjoin%
\definecolor{currentfill}{rgb}{0.282884,0.135920,0.453427}%
\pgfsetfillcolor{currentfill}%
\pgfsetfillopacity{0.700000}%
\pgfsetlinewidth{0.000000pt}%
\definecolor{currentstroke}{rgb}{0.000000,0.000000,0.000000}%
\pgfsetstrokecolor{currentstroke}%
\pgfsetdash{}{0pt}%
\pgfpathmoveto{\pgfqpoint{4.721016in}{1.529912in}}%
\pgfpathlineto{\pgfqpoint{4.734896in}{1.531454in}}%
\pgfpathlineto{\pgfqpoint{4.748786in}{1.533106in}}%
\pgfpathlineto{\pgfqpoint{4.762687in}{1.534870in}}%
\pgfpathlineto{\pgfqpoint{4.776600in}{1.536744in}}%
\pgfpathlineto{\pgfqpoint{4.784283in}{1.549348in}}%
\pgfpathlineto{\pgfqpoint{4.791962in}{1.561980in}}%
\pgfpathlineto{\pgfqpoint{4.799636in}{1.574637in}}%
\pgfpathlineto{\pgfqpoint{4.807307in}{1.587316in}}%
\pgfpathlineto{\pgfqpoint{4.793395in}{1.585127in}}%
\pgfpathlineto{\pgfqpoint{4.779495in}{1.583049in}}%
\pgfpathlineto{\pgfqpoint{4.765607in}{1.581082in}}%
\pgfpathlineto{\pgfqpoint{4.751729in}{1.579226in}}%
\pgfpathlineto{\pgfqpoint{4.744057in}{1.566856in}}%
\pgfpathlineto{\pgfqpoint{4.736381in}{1.554511in}}%
\pgfpathlineto{\pgfqpoint{4.728701in}{1.542196in}}%
\pgfpathlineto{\pgfqpoint{4.721016in}{1.529912in}}%
\pgfpathclose%
\pgfusepath{fill}%
\end{pgfscope}%
\begin{pgfscope}%
\pgfpathrectangle{\pgfqpoint{1.254980in}{0.150000in}}{\pgfqpoint{5.490039in}{5.490039in}}%
\pgfusepath{clip}%
\pgfsetbuttcap%
\pgfsetroundjoin%
\definecolor{currentfill}{rgb}{0.134692,0.658636,0.517649}%
\pgfsetfillcolor{currentfill}%
\pgfsetfillopacity{0.700000}%
\pgfsetlinewidth{0.000000pt}%
\definecolor{currentstroke}{rgb}{0.000000,0.000000,0.000000}%
\pgfsetstrokecolor{currentstroke}%
\pgfsetdash{}{0pt}%
\pgfpathmoveto{\pgfqpoint{2.327616in}{2.930797in}}%
\pgfpathlineto{\pgfqpoint{2.341558in}{2.907118in}}%
\pgfpathlineto{\pgfqpoint{2.355490in}{2.883637in}}%
\pgfpathlineto{\pgfqpoint{2.369411in}{2.860354in}}%
\pgfpathlineto{\pgfqpoint{2.383322in}{2.837267in}}%
\pgfpathlineto{\pgfqpoint{2.392433in}{2.827434in}}%
\pgfpathlineto{\pgfqpoint{2.401520in}{2.817936in}}%
\pgfpathlineto{\pgfqpoint{2.410584in}{2.808768in}}%
\pgfpathlineto{\pgfqpoint{2.419625in}{2.799927in}}%
\pgfpathlineto{\pgfqpoint{2.405774in}{2.822472in}}%
\pgfpathlineto{\pgfqpoint{2.391914in}{2.845212in}}%
\pgfpathlineto{\pgfqpoint{2.378043in}{2.868149in}}%
\pgfpathlineto{\pgfqpoint{2.364162in}{2.891282in}}%
\pgfpathlineto{\pgfqpoint{2.355062in}{2.900658in}}%
\pgfpathlineto{\pgfqpoint{2.345937in}{2.910366in}}%
\pgfpathlineto{\pgfqpoint{2.336789in}{2.920411in}}%
\pgfpathlineto{\pgfqpoint{2.327616in}{2.930797in}}%
\pgfpathclose%
\pgfusepath{fill}%
\end{pgfscope}%
\begin{pgfscope}%
\pgfpathrectangle{\pgfqpoint{1.254980in}{0.150000in}}{\pgfqpoint{5.490039in}{5.490039in}}%
\pgfusepath{clip}%
\pgfsetbuttcap%
\pgfsetroundjoin%
\definecolor{currentfill}{rgb}{0.266580,0.228262,0.514349}%
\pgfsetfillcolor{currentfill}%
\pgfsetfillopacity{0.700000}%
\pgfsetlinewidth{0.000000pt}%
\definecolor{currentstroke}{rgb}{0.000000,0.000000,0.000000}%
\pgfsetstrokecolor{currentstroke}%
\pgfsetdash{}{0pt}%
\pgfpathmoveto{\pgfqpoint{3.171871in}{1.779543in}}%
\pgfpathlineto{\pgfqpoint{3.185500in}{1.765929in}}%
\pgfpathlineto{\pgfqpoint{3.199128in}{1.752455in}}%
\pgfpathlineto{\pgfqpoint{3.212755in}{1.739119in}}%
\pgfpathlineto{\pgfqpoint{3.226381in}{1.725922in}}%
\pgfpathlineto{\pgfqpoint{3.234774in}{1.723331in}}%
\pgfpathlineto{\pgfqpoint{3.243153in}{1.721009in}}%
\pgfpathlineto{\pgfqpoint{3.251518in}{1.718951in}}%
\pgfpathlineto{\pgfqpoint{3.259869in}{1.717153in}}%
\pgfpathlineto{\pgfqpoint{3.246282in}{1.729843in}}%
\pgfpathlineto{\pgfqpoint{3.232693in}{1.742671in}}%
\pgfpathlineto{\pgfqpoint{3.219104in}{1.755637in}}%
\pgfpathlineto{\pgfqpoint{3.205514in}{1.768742in}}%
\pgfpathlineto{\pgfqpoint{3.197125in}{1.771040in}}%
\pgfpathlineto{\pgfqpoint{3.188722in}{1.773604in}}%
\pgfpathlineto{\pgfqpoint{3.180304in}{1.776436in}}%
\pgfpathlineto{\pgfqpoint{3.171871in}{1.779543in}}%
\pgfpathclose%
\pgfusepath{fill}%
\end{pgfscope}%
\begin{pgfscope}%
\pgfpathrectangle{\pgfqpoint{1.254980in}{0.150000in}}{\pgfqpoint{5.490039in}{5.490039in}}%
\pgfusepath{clip}%
\pgfsetbuttcap%
\pgfsetroundjoin%
\definecolor{currentfill}{rgb}{0.168126,0.459988,0.558082}%
\pgfsetfillcolor{currentfill}%
\pgfsetfillopacity{0.700000}%
\pgfsetlinewidth{0.000000pt}%
\definecolor{currentstroke}{rgb}{0.000000,0.000000,0.000000}%
\pgfsetstrokecolor{currentstroke}%
\pgfsetdash{}{0pt}%
\pgfpathmoveto{\pgfqpoint{2.679265in}{2.368351in}}%
\pgfpathlineto{\pgfqpoint{2.693029in}{2.349237in}}%
\pgfpathlineto{\pgfqpoint{2.706786in}{2.330291in}}%
\pgfpathlineto{\pgfqpoint{2.720537in}{2.311511in}}%
\pgfpathlineto{\pgfqpoint{2.734283in}{2.292897in}}%
\pgfpathlineto{\pgfqpoint{2.743085in}{2.285567in}}%
\pgfpathlineto{\pgfqpoint{2.751868in}{2.278552in}}%
\pgfpathlineto{\pgfqpoint{2.760630in}{2.271848in}}%
\pgfpathlineto{\pgfqpoint{2.769373in}{2.265450in}}%
\pgfpathlineto{\pgfqpoint{2.755679in}{2.283530in}}%
\pgfpathlineto{\pgfqpoint{2.741980in}{2.301775in}}%
\pgfpathlineto{\pgfqpoint{2.728275in}{2.320186in}}%
\pgfpathlineto{\pgfqpoint{2.714564in}{2.338763in}}%
\pgfpathlineto{\pgfqpoint{2.705769in}{2.345689in}}%
\pgfpathlineto{\pgfqpoint{2.696955in}{2.352925in}}%
\pgfpathlineto{\pgfqpoint{2.688120in}{2.360477in}}%
\pgfpathlineto{\pgfqpoint{2.679265in}{2.368351in}}%
\pgfpathclose%
\pgfusepath{fill}%
\end{pgfscope}%
\begin{pgfscope}%
\pgfpathrectangle{\pgfqpoint{1.254980in}{0.150000in}}{\pgfqpoint{5.490039in}{5.490039in}}%
\pgfusepath{clip}%
\pgfsetbuttcap%
\pgfsetroundjoin%
\definecolor{currentfill}{rgb}{0.268510,0.009605,0.335427}%
\pgfsetfillcolor{currentfill}%
\pgfsetfillopacity{0.700000}%
\pgfsetlinewidth{0.000000pt}%
\definecolor{currentstroke}{rgb}{0.000000,0.000000,0.000000}%
\pgfsetstrokecolor{currentstroke}%
\pgfsetdash{}{0pt}%
\pgfpathmoveto{\pgfqpoint{3.922697in}{1.323269in}}%
\pgfpathlineto{\pgfqpoint{3.936330in}{1.317205in}}%
\pgfpathlineto{\pgfqpoint{3.949969in}{1.311259in}}%
\pgfpathlineto{\pgfqpoint{3.963612in}{1.305431in}}%
\pgfpathlineto{\pgfqpoint{3.977261in}{1.299719in}}%
\pgfpathlineto{\pgfqpoint{3.985193in}{1.305475in}}%
\pgfpathlineto{\pgfqpoint{3.993117in}{1.311396in}}%
\pgfpathlineto{\pgfqpoint{4.001035in}{1.317480in}}%
\pgfpathlineto{\pgfqpoint{4.008946in}{1.323722in}}%
\pgfpathlineto{\pgfqpoint{3.995314in}{1.328994in}}%
\pgfpathlineto{\pgfqpoint{3.981689in}{1.334383in}}%
\pgfpathlineto{\pgfqpoint{3.968068in}{1.339889in}}%
\pgfpathlineto{\pgfqpoint{3.954453in}{1.345512in}}%
\pgfpathlineto{\pgfqpoint{3.946525in}{1.339704in}}%
\pgfpathlineto{\pgfqpoint{3.938590in}{1.334058in}}%
\pgfpathlineto{\pgfqpoint{3.930647in}{1.328578in}}%
\pgfpathlineto{\pgfqpoint{3.922697in}{1.323269in}}%
\pgfpathclose%
\pgfusepath{fill}%
\end{pgfscope}%
\begin{pgfscope}%
\pgfpathrectangle{\pgfqpoint{1.254980in}{0.150000in}}{\pgfqpoint{5.490039in}{5.490039in}}%
\pgfusepath{clip}%
\pgfsetbuttcap%
\pgfsetroundjoin%
\definecolor{currentfill}{rgb}{0.269944,0.014625,0.341379}%
\pgfsetfillcolor{currentfill}%
\pgfsetfillopacity{0.700000}%
\pgfsetlinewidth{0.000000pt}%
\definecolor{currentstroke}{rgb}{0.000000,0.000000,0.000000}%
\pgfsetstrokecolor{currentstroke}%
\pgfsetdash{}{0pt}%
\pgfpathmoveto{\pgfqpoint{4.290494in}{1.323543in}}%
\pgfpathlineto{\pgfqpoint{4.304212in}{1.321049in}}%
\pgfpathlineto{\pgfqpoint{4.317939in}{1.318668in}}%
\pgfpathlineto{\pgfqpoint{4.331673in}{1.316400in}}%
\pgfpathlineto{\pgfqpoint{4.345416in}{1.314244in}}%
\pgfpathlineto{\pgfqpoint{4.353209in}{1.323738in}}%
\pgfpathlineto{\pgfqpoint{4.360997in}{1.333336in}}%
\pgfpathlineto{\pgfqpoint{4.368781in}{1.343034in}}%
\pgfpathlineto{\pgfqpoint{4.376559in}{1.352830in}}%
\pgfpathlineto{\pgfqpoint{4.362826in}{1.354595in}}%
\pgfpathlineto{\pgfqpoint{4.349101in}{1.356472in}}%
\pgfpathlineto{\pgfqpoint{4.335383in}{1.358462in}}%
\pgfpathlineto{\pgfqpoint{4.321675in}{1.360564in}}%
\pgfpathlineto{\pgfqpoint{4.313887in}{1.351153in}}%
\pgfpathlineto{\pgfqpoint{4.306095in}{1.341844in}}%
\pgfpathlineto{\pgfqpoint{4.298297in}{1.332640in}}%
\pgfpathlineto{\pgfqpoint{4.290494in}{1.323543in}}%
\pgfpathclose%
\pgfusepath{fill}%
\end{pgfscope}%
\begin{pgfscope}%
\pgfpathrectangle{\pgfqpoint{1.254980in}{0.150000in}}{\pgfqpoint{5.490039in}{5.490039in}}%
\pgfusepath{clip}%
\pgfsetbuttcap%
\pgfsetroundjoin%
\definecolor{currentfill}{rgb}{0.267004,0.004874,0.329415}%
\pgfsetfillcolor{currentfill}%
\pgfsetfillopacity{0.700000}%
\pgfsetlinewidth{0.000000pt}%
\definecolor{currentstroke}{rgb}{0.000000,0.000000,0.000000}%
\pgfsetstrokecolor{currentstroke}%
\pgfsetdash{}{0pt}%
\pgfpathmoveto{\pgfqpoint{4.063528in}{1.303797in}}%
\pgfpathlineto{\pgfqpoint{4.077188in}{1.299105in}}%
\pgfpathlineto{\pgfqpoint{4.090855in}{1.294528in}}%
\pgfpathlineto{\pgfqpoint{4.104527in}{1.290067in}}%
\pgfpathlineto{\pgfqpoint{4.118206in}{1.285720in}}%
\pgfpathlineto{\pgfqpoint{4.126079in}{1.292975in}}%
\pgfpathlineto{\pgfqpoint{4.133945in}{1.300374in}}%
\pgfpathlineto{\pgfqpoint{4.141806in}{1.307912in}}%
\pgfpathlineto{\pgfqpoint{4.149660in}{1.315586in}}%
\pgfpathlineto{\pgfqpoint{4.135995in}{1.319510in}}%
\pgfpathlineto{\pgfqpoint{4.122337in}{1.323549in}}%
\pgfpathlineto{\pgfqpoint{4.108685in}{1.327702in}}%
\pgfpathlineto{\pgfqpoint{4.095039in}{1.331971in}}%
\pgfpathlineto{\pgfqpoint{4.087171in}{1.324714in}}%
\pgfpathlineto{\pgfqpoint{4.079296in}{1.317597in}}%
\pgfpathlineto{\pgfqpoint{4.071415in}{1.310624in}}%
\pgfpathlineto{\pgfqpoint{4.063528in}{1.303797in}}%
\pgfpathclose%
\pgfusepath{fill}%
\end{pgfscope}%
\begin{pgfscope}%
\pgfpathrectangle{\pgfqpoint{1.254980in}{0.150000in}}{\pgfqpoint{5.490039in}{5.490039in}}%
\pgfusepath{clip}%
\pgfsetbuttcap%
\pgfsetroundjoin%
\definecolor{currentfill}{rgb}{0.280255,0.165693,0.476498}%
\pgfsetfillcolor{currentfill}%
\pgfsetfillopacity{0.700000}%
\pgfsetlinewidth{0.000000pt}%
\definecolor{currentstroke}{rgb}{0.000000,0.000000,0.000000}%
\pgfsetstrokecolor{currentstroke}%
\pgfsetdash{}{0pt}%
\pgfpathmoveto{\pgfqpoint{4.807307in}{1.587316in}}%
\pgfpathlineto{\pgfqpoint{4.821230in}{1.589616in}}%
\pgfpathlineto{\pgfqpoint{4.835165in}{1.592026in}}%
\pgfpathlineto{\pgfqpoint{4.849112in}{1.594548in}}%
\pgfpathlineto{\pgfqpoint{4.863071in}{1.597180in}}%
\pgfpathlineto{\pgfqpoint{4.870737in}{1.610185in}}%
\pgfpathlineto{\pgfqpoint{4.878399in}{1.623203in}}%
\pgfpathlineto{\pgfqpoint{4.886057in}{1.636233in}}%
\pgfpathlineto{\pgfqpoint{4.893710in}{1.649273in}}%
\pgfpathlineto{\pgfqpoint{4.879752in}{1.646341in}}%
\pgfpathlineto{\pgfqpoint{4.865805in}{1.643520in}}%
\pgfpathlineto{\pgfqpoint{4.851871in}{1.640810in}}%
\pgfpathlineto{\pgfqpoint{4.837948in}{1.638211in}}%
\pgfpathlineto{\pgfqpoint{4.830294in}{1.625465in}}%
\pgfpathlineto{\pgfqpoint{4.822636in}{1.612732in}}%
\pgfpathlineto{\pgfqpoint{4.814974in}{1.600015in}}%
\pgfpathlineto{\pgfqpoint{4.807307in}{1.587316in}}%
\pgfpathclose%
\pgfusepath{fill}%
\end{pgfscope}%
\begin{pgfscope}%
\pgfpathrectangle{\pgfqpoint{1.254980in}{0.150000in}}{\pgfqpoint{5.490039in}{5.490039in}}%
\pgfusepath{clip}%
\pgfsetbuttcap%
\pgfsetroundjoin%
\definecolor{currentfill}{rgb}{0.273006,0.204520,0.501721}%
\pgfsetfillcolor{currentfill}%
\pgfsetfillopacity{0.700000}%
\pgfsetlinewidth{0.000000pt}%
\definecolor{currentstroke}{rgb}{0.000000,0.000000,0.000000}%
\pgfsetstrokecolor{currentstroke}%
\pgfsetdash{}{0pt}%
\pgfpathmoveto{\pgfqpoint{3.226381in}{1.725922in}}%
\pgfpathlineto{\pgfqpoint{3.240006in}{1.712862in}}%
\pgfpathlineto{\pgfqpoint{3.253630in}{1.699940in}}%
\pgfpathlineto{\pgfqpoint{3.267254in}{1.687153in}}%
\pgfpathlineto{\pgfqpoint{3.280877in}{1.674503in}}%
\pgfpathlineto{\pgfqpoint{3.289232in}{1.672425in}}%
\pgfpathlineto{\pgfqpoint{3.297574in}{1.670611in}}%
\pgfpathlineto{\pgfqpoint{3.305901in}{1.669057in}}%
\pgfpathlineto{\pgfqpoint{3.314216in}{1.667758in}}%
\pgfpathlineto{\pgfqpoint{3.300630in}{1.679903in}}%
\pgfpathlineto{\pgfqpoint{3.287043in}{1.692184in}}%
\pgfpathlineto{\pgfqpoint{3.273457in}{1.704600in}}%
\pgfpathlineto{\pgfqpoint{3.259869in}{1.717153in}}%
\pgfpathlineto{\pgfqpoint{3.251518in}{1.718951in}}%
\pgfpathlineto{\pgfqpoint{3.243153in}{1.721009in}}%
\pgfpathlineto{\pgfqpoint{3.234774in}{1.723331in}}%
\pgfpathlineto{\pgfqpoint{3.226381in}{1.725922in}}%
\pgfpathclose%
\pgfusepath{fill}%
\end{pgfscope}%
\begin{pgfscope}%
\pgfpathrectangle{\pgfqpoint{1.254980in}{0.150000in}}{\pgfqpoint{5.490039in}{5.490039in}}%
\pgfusepath{clip}%
\pgfsetbuttcap%
\pgfsetroundjoin%
\definecolor{currentfill}{rgb}{0.156270,0.489624,0.557936}%
\pgfsetfillcolor{currentfill}%
\pgfsetfillopacity{0.700000}%
\pgfsetlinewidth{0.000000pt}%
\definecolor{currentstroke}{rgb}{0.000000,0.000000,0.000000}%
\pgfsetstrokecolor{currentstroke}%
\pgfsetdash{}{0pt}%
\pgfpathmoveto{\pgfqpoint{2.624144in}{2.446500in}}%
\pgfpathlineto{\pgfqpoint{2.637934in}{2.426706in}}%
\pgfpathlineto{\pgfqpoint{2.651718in}{2.407084in}}%
\pgfpathlineto{\pgfqpoint{2.665495in}{2.387633in}}%
\pgfpathlineto{\pgfqpoint{2.679265in}{2.368351in}}%
\pgfpathlineto{\pgfqpoint{2.688120in}{2.360477in}}%
\pgfpathlineto{\pgfqpoint{2.696955in}{2.352925in}}%
\pgfpathlineto{\pgfqpoint{2.705769in}{2.345689in}}%
\pgfpathlineto{\pgfqpoint{2.714564in}{2.338763in}}%
\pgfpathlineto{\pgfqpoint{2.700847in}{2.357508in}}%
\pgfpathlineto{\pgfqpoint{2.687123in}{2.376421in}}%
\pgfpathlineto{\pgfqpoint{2.673394in}{2.395504in}}%
\pgfpathlineto{\pgfqpoint{2.659658in}{2.414757in}}%
\pgfpathlineto{\pgfqpoint{2.650811in}{2.422213in}}%
\pgfpathlineto{\pgfqpoint{2.641943in}{2.429985in}}%
\pgfpathlineto{\pgfqpoint{2.633054in}{2.438079in}}%
\pgfpathlineto{\pgfqpoint{2.624144in}{2.446500in}}%
\pgfpathclose%
\pgfusepath{fill}%
\end{pgfscope}%
\begin{pgfscope}%
\pgfpathrectangle{\pgfqpoint{1.254980in}{0.150000in}}{\pgfqpoint{5.490039in}{5.490039in}}%
\pgfusepath{clip}%
\pgfsetbuttcap%
\pgfsetroundjoin%
\definecolor{currentfill}{rgb}{0.206756,0.371758,0.553117}%
\pgfsetfillcolor{currentfill}%
\pgfsetfillopacity{0.700000}%
\pgfsetlinewidth{0.000000pt}%
\definecolor{currentstroke}{rgb}{0.000000,0.000000,0.000000}%
\pgfsetstrokecolor{currentstroke}%
\pgfsetdash{}{0pt}%
\pgfpathmoveto{\pgfqpoint{5.301303in}{2.045153in}}%
\pgfpathlineto{\pgfqpoint{5.315501in}{2.051447in}}%
\pgfpathlineto{\pgfqpoint{5.329715in}{2.057853in}}%
\pgfpathlineto{\pgfqpoint{5.343943in}{2.064371in}}%
\pgfpathlineto{\pgfqpoint{5.351489in}{2.077925in}}%
\pgfpathlineto{\pgfqpoint{5.359029in}{2.091421in}}%
\pgfpathlineto{\pgfqpoint{5.366564in}{2.104859in}}%
\pgfpathlineto{\pgfqpoint{5.374093in}{2.118236in}}%
\pgfpathlineto{\pgfqpoint{5.359863in}{2.111529in}}%
\pgfpathlineto{\pgfqpoint{5.345648in}{2.104934in}}%
\pgfpathlineto{\pgfqpoint{5.331448in}{2.098450in}}%
\pgfpathlineto{\pgfqpoint{5.323919in}{2.085210in}}%
\pgfpathlineto{\pgfqpoint{5.316386in}{2.071913in}}%
\pgfpathlineto{\pgfqpoint{5.308847in}{2.058560in}}%
\pgfpathlineto{\pgfqpoint{5.301303in}{2.045153in}}%
\pgfpathclose%
\pgfusepath{fill}%
\end{pgfscope}%
\begin{pgfscope}%
\pgfpathrectangle{\pgfqpoint{1.254980in}{0.150000in}}{\pgfqpoint{5.490039in}{5.490039in}}%
\pgfusepath{clip}%
\pgfsetbuttcap%
\pgfsetroundjoin%
\definecolor{currentfill}{rgb}{0.272594,0.025563,0.353093}%
\pgfsetfillcolor{currentfill}%
\pgfsetfillopacity{0.700000}%
\pgfsetlinewidth{0.000000pt}%
\definecolor{currentstroke}{rgb}{0.000000,0.000000,0.000000}%
\pgfsetstrokecolor{currentstroke}%
\pgfsetdash{}{0pt}%
\pgfpathmoveto{\pgfqpoint{3.781748in}{1.360238in}}%
\pgfpathlineto{\pgfqpoint{3.795368in}{1.352766in}}%
\pgfpathlineto{\pgfqpoint{3.808991in}{1.345414in}}%
\pgfpathlineto{\pgfqpoint{3.822618in}{1.338182in}}%
\pgfpathlineto{\pgfqpoint{3.836250in}{1.331069in}}%
\pgfpathlineto{\pgfqpoint{3.844252in}{1.335197in}}%
\pgfpathlineto{\pgfqpoint{3.852247in}{1.339514in}}%
\pgfpathlineto{\pgfqpoint{3.860233in}{1.344017in}}%
\pgfpathlineto{\pgfqpoint{3.868211in}{1.348702in}}%
\pgfpathlineto{\pgfqpoint{3.854601in}{1.355357in}}%
\pgfpathlineto{\pgfqpoint{3.840996in}{1.362131in}}%
\pgfpathlineto{\pgfqpoint{3.827394in}{1.369025in}}%
\pgfpathlineto{\pgfqpoint{3.813797in}{1.376039in}}%
\pgfpathlineto{\pgfqpoint{3.805798in}{1.371806in}}%
\pgfpathlineto{\pgfqpoint{3.797790in}{1.367758in}}%
\pgfpathlineto{\pgfqpoint{3.789774in}{1.363901in}}%
\pgfpathlineto{\pgfqpoint{3.781748in}{1.360238in}}%
\pgfpathclose%
\pgfusepath{fill}%
\end{pgfscope}%
\begin{pgfscope}%
\pgfpathrectangle{\pgfqpoint{1.254980in}{0.150000in}}{\pgfqpoint{5.490039in}{5.490039in}}%
\pgfusepath{clip}%
\pgfsetbuttcap%
\pgfsetroundjoin%
\definecolor{currentfill}{rgb}{0.244972,0.287675,0.537260}%
\pgfsetfillcolor{currentfill}%
\pgfsetfillopacity{0.700000}%
\pgfsetlinewidth{0.000000pt}%
\definecolor{currentstroke}{rgb}{0.000000,0.000000,0.000000}%
\pgfsetstrokecolor{currentstroke}%
\pgfsetdash{}{0pt}%
\pgfpathmoveto{\pgfqpoint{5.097366in}{1.839567in}}%
\pgfpathlineto{\pgfqpoint{5.111443in}{1.844304in}}%
\pgfpathlineto{\pgfqpoint{5.125535in}{1.849152in}}%
\pgfpathlineto{\pgfqpoint{5.139640in}{1.854112in}}%
\pgfpathlineto{\pgfqpoint{5.153759in}{1.859182in}}%
\pgfpathlineto{\pgfqpoint{5.161361in}{1.872878in}}%
\pgfpathlineto{\pgfqpoint{5.168957in}{1.886542in}}%
\pgfpathlineto{\pgfqpoint{5.176549in}{1.900173in}}%
\pgfpathlineto{\pgfqpoint{5.184137in}{1.913770in}}%
\pgfpathlineto{\pgfqpoint{5.170015in}{1.908461in}}%
\pgfpathlineto{\pgfqpoint{5.155908in}{1.903264in}}%
\pgfpathlineto{\pgfqpoint{5.141815in}{1.898178in}}%
\pgfpathlineto{\pgfqpoint{5.127736in}{1.893204in}}%
\pgfpathlineto{\pgfqpoint{5.120150in}{1.879839in}}%
\pgfpathlineto{\pgfqpoint{5.112560in}{1.866443in}}%
\pgfpathlineto{\pgfqpoint{5.104965in}{1.853019in}}%
\pgfpathlineto{\pgfqpoint{5.097366in}{1.839567in}}%
\pgfpathclose%
\pgfusepath{fill}%
\end{pgfscope}%
\begin{pgfscope}%
\pgfpathrectangle{\pgfqpoint{1.254980in}{0.150000in}}{\pgfqpoint{5.490039in}{5.490039in}}%
\pgfusepath{clip}%
\pgfsetbuttcap%
\pgfsetroundjoin%
\definecolor{currentfill}{rgb}{0.280894,0.078907,0.402329}%
\pgfsetfillcolor{currentfill}%
\pgfsetfillopacity{0.700000}%
\pgfsetlinewidth{0.000000pt}%
\definecolor{currentstroke}{rgb}{0.000000,0.000000,0.000000}%
\pgfsetstrokecolor{currentstroke}%
\pgfsetdash{}{0pt}%
\pgfpathmoveto{\pgfqpoint{3.586078in}{1.452485in}}%
\pgfpathlineto{\pgfqpoint{3.599687in}{1.443066in}}%
\pgfpathlineto{\pgfqpoint{3.613298in}{1.433772in}}%
\pgfpathlineto{\pgfqpoint{3.626911in}{1.424601in}}%
\pgfpathlineto{\pgfqpoint{3.640527in}{1.415555in}}%
\pgfpathlineto{\pgfqpoint{3.648640in}{1.417437in}}%
\pgfpathlineto{\pgfqpoint{3.656744in}{1.419537in}}%
\pgfpathlineto{\pgfqpoint{3.664837in}{1.421852in}}%
\pgfpathlineto{\pgfqpoint{3.672921in}{1.424377in}}%
\pgfpathlineto{\pgfqpoint{3.659332in}{1.432946in}}%
\pgfpathlineto{\pgfqpoint{3.645746in}{1.441638in}}%
\pgfpathlineto{\pgfqpoint{3.632162in}{1.450454in}}%
\pgfpathlineto{\pgfqpoint{3.618581in}{1.459395in}}%
\pgfpathlineto{\pgfqpoint{3.610471in}{1.457341in}}%
\pgfpathlineto{\pgfqpoint{3.602350in}{1.455502in}}%
\pgfpathlineto{\pgfqpoint{3.594219in}{1.453882in}}%
\pgfpathlineto{\pgfqpoint{3.586078in}{1.452485in}}%
\pgfpathclose%
\pgfusepath{fill}%
\end{pgfscope}%
\begin{pgfscope}%
\pgfpathrectangle{\pgfqpoint{1.254980in}{0.150000in}}{\pgfqpoint{5.490039in}{5.490039in}}%
\pgfusepath{clip}%
\pgfsetbuttcap%
\pgfsetroundjoin%
\definecolor{currentfill}{rgb}{0.277134,0.185228,0.489898}%
\pgfsetfillcolor{currentfill}%
\pgfsetfillopacity{0.700000}%
\pgfsetlinewidth{0.000000pt}%
\definecolor{currentstroke}{rgb}{0.000000,0.000000,0.000000}%
\pgfsetstrokecolor{currentstroke}%
\pgfsetdash{}{0pt}%
\pgfpathmoveto{\pgfqpoint{3.280877in}{1.674503in}}%
\pgfpathlineto{\pgfqpoint{3.294499in}{1.661988in}}%
\pgfpathlineto{\pgfqpoint{3.308121in}{1.649608in}}%
\pgfpathlineto{\pgfqpoint{3.321744in}{1.637362in}}%
\pgfpathlineto{\pgfqpoint{3.335366in}{1.625249in}}%
\pgfpathlineto{\pgfqpoint{3.343684in}{1.623683in}}%
\pgfpathlineto{\pgfqpoint{3.351990in}{1.622375in}}%
\pgfpathlineto{\pgfqpoint{3.360282in}{1.621323in}}%
\pgfpathlineto{\pgfqpoint{3.368561in}{1.620521in}}%
\pgfpathlineto{\pgfqpoint{3.354975in}{1.632130in}}%
\pgfpathlineto{\pgfqpoint{3.341388in}{1.643872in}}%
\pgfpathlineto{\pgfqpoint{3.327802in}{1.655748in}}%
\pgfpathlineto{\pgfqpoint{3.314216in}{1.667758in}}%
\pgfpathlineto{\pgfqpoint{3.305901in}{1.669057in}}%
\pgfpathlineto{\pgfqpoint{3.297574in}{1.670611in}}%
\pgfpathlineto{\pgfqpoint{3.289232in}{1.672425in}}%
\pgfpathlineto{\pgfqpoint{3.280877in}{1.674503in}}%
\pgfpathclose%
\pgfusepath{fill}%
\end{pgfscope}%
\begin{pgfscope}%
\pgfpathrectangle{\pgfqpoint{1.254980in}{0.150000in}}{\pgfqpoint{5.490039in}{5.490039in}}%
\pgfusepath{clip}%
\pgfsetbuttcap%
\pgfsetroundjoin%
\definecolor{currentfill}{rgb}{0.268510,0.009605,0.335427}%
\pgfsetfillcolor{currentfill}%
\pgfsetfillopacity{0.700000}%
\pgfsetlinewidth{0.000000pt}%
\definecolor{currentstroke}{rgb}{0.000000,0.000000,0.000000}%
\pgfsetstrokecolor{currentstroke}%
\pgfsetdash{}{0pt}%
\pgfpathmoveto{\pgfqpoint{4.204388in}{1.301034in}}%
\pgfpathlineto{\pgfqpoint{4.218088in}{1.297680in}}%
\pgfpathlineto{\pgfqpoint{4.231794in}{1.294440in}}%
\pgfpathlineto{\pgfqpoint{4.245508in}{1.291313in}}%
\pgfpathlineto{\pgfqpoint{4.259230in}{1.288300in}}%
\pgfpathlineto{\pgfqpoint{4.267054in}{1.296932in}}%
\pgfpathlineto{\pgfqpoint{4.274873in}{1.305686in}}%
\pgfpathlineto{\pgfqpoint{4.282686in}{1.314557in}}%
\pgfpathlineto{\pgfqpoint{4.290494in}{1.323543in}}%
\pgfpathlineto{\pgfqpoint{4.276783in}{1.326150in}}%
\pgfpathlineto{\pgfqpoint{4.263080in}{1.328869in}}%
\pgfpathlineto{\pgfqpoint{4.249385in}{1.331702in}}%
\pgfpathlineto{\pgfqpoint{4.235697in}{1.334649in}}%
\pgfpathlineto{\pgfqpoint{4.227878in}{1.326064in}}%
\pgfpathlineto{\pgfqpoint{4.220054in}{1.317598in}}%
\pgfpathlineto{\pgfqpoint{4.212224in}{1.309253in}}%
\pgfpathlineto{\pgfqpoint{4.204388in}{1.301034in}}%
\pgfpathclose%
\pgfusepath{fill}%
\end{pgfscope}%
\begin{pgfscope}%
\pgfpathrectangle{\pgfqpoint{1.254980in}{0.150000in}}{\pgfqpoint{5.490039in}{5.490039in}}%
\pgfusepath{clip}%
\pgfsetbuttcap%
\pgfsetroundjoin%
\definecolor{currentfill}{rgb}{0.274128,0.199721,0.498911}%
\pgfsetfillcolor{currentfill}%
\pgfsetfillopacity{0.700000}%
\pgfsetlinewidth{0.000000pt}%
\definecolor{currentstroke}{rgb}{0.000000,0.000000,0.000000}%
\pgfsetstrokecolor{currentstroke}%
\pgfsetdash{}{0pt}%
\pgfpathmoveto{\pgfqpoint{4.893710in}{1.649273in}}%
\pgfpathlineto{\pgfqpoint{4.907681in}{1.652315in}}%
\pgfpathlineto{\pgfqpoint{4.921665in}{1.655469in}}%
\pgfpathlineto{\pgfqpoint{4.935660in}{1.658733in}}%
\pgfpathlineto{\pgfqpoint{4.949668in}{1.662108in}}%
\pgfpathlineto{\pgfqpoint{4.957318in}{1.675444in}}%
\pgfpathlineto{\pgfqpoint{4.964965in}{1.688782in}}%
\pgfpathlineto{\pgfqpoint{4.972606in}{1.702117in}}%
\pgfpathlineto{\pgfqpoint{4.980244in}{1.715450in}}%
\pgfpathlineto{\pgfqpoint{4.966235in}{1.711790in}}%
\pgfpathlineto{\pgfqpoint{4.952239in}{1.708241in}}%
\pgfpathlineto{\pgfqpoint{4.938255in}{1.704803in}}%
\pgfpathlineto{\pgfqpoint{4.924284in}{1.701477in}}%
\pgfpathlineto{\pgfqpoint{4.916647in}{1.688423in}}%
\pgfpathlineto{\pgfqpoint{4.909006in}{1.675370in}}%
\pgfpathlineto{\pgfqpoint{4.901360in}{1.662319in}}%
\pgfpathlineto{\pgfqpoint{4.893710in}{1.649273in}}%
\pgfpathclose%
\pgfusepath{fill}%
\end{pgfscope}%
\begin{pgfscope}%
\pgfpathrectangle{\pgfqpoint{1.254980in}{0.150000in}}{\pgfqpoint{5.490039in}{5.490039in}}%
\pgfusepath{clip}%
\pgfsetbuttcap%
\pgfsetroundjoin%
\definecolor{currentfill}{rgb}{0.144759,0.519093,0.556572}%
\pgfsetfillcolor{currentfill}%
\pgfsetfillopacity{0.700000}%
\pgfsetlinewidth{0.000000pt}%
\definecolor{currentstroke}{rgb}{0.000000,0.000000,0.000000}%
\pgfsetstrokecolor{currentstroke}%
\pgfsetdash{}{0pt}%
\pgfpathmoveto{\pgfqpoint{2.568911in}{2.527409in}}%
\pgfpathlineto{\pgfqpoint{2.582730in}{2.506919in}}%
\pgfpathlineto{\pgfqpoint{2.596542in}{2.486605in}}%
\pgfpathlineto{\pgfqpoint{2.610347in}{2.466466in}}%
\pgfpathlineto{\pgfqpoint{2.624144in}{2.446500in}}%
\pgfpathlineto{\pgfqpoint{2.633054in}{2.438079in}}%
\pgfpathlineto{\pgfqpoint{2.641943in}{2.429985in}}%
\pgfpathlineto{\pgfqpoint{2.650811in}{2.422213in}}%
\pgfpathlineto{\pgfqpoint{2.659658in}{2.414757in}}%
\pgfpathlineto{\pgfqpoint{2.645915in}{2.434182in}}%
\pgfpathlineto{\pgfqpoint{2.632166in}{2.453779in}}%
\pgfpathlineto{\pgfqpoint{2.618409in}{2.473549in}}%
\pgfpathlineto{\pgfqpoint{2.604646in}{2.493494in}}%
\pgfpathlineto{\pgfqpoint{2.595745in}{2.501484in}}%
\pgfpathlineto{\pgfqpoint{2.586822in}{2.509797in}}%
\pgfpathlineto{\pgfqpoint{2.577877in}{2.518437in}}%
\pgfpathlineto{\pgfqpoint{2.568911in}{2.527409in}}%
\pgfpathclose%
\pgfusepath{fill}%
\end{pgfscope}%
\begin{pgfscope}%
\pgfpathrectangle{\pgfqpoint{1.254980in}{0.150000in}}{\pgfqpoint{5.490039in}{5.490039in}}%
\pgfusepath{clip}%
\pgfsetbuttcap%
\pgfsetroundjoin%
\definecolor{currentfill}{rgb}{0.175707,0.697900,0.491033}%
\pgfsetfillcolor{currentfill}%
\pgfsetfillopacity{0.700000}%
\pgfsetlinewidth{0.000000pt}%
\definecolor{currentstroke}{rgb}{0.000000,0.000000,0.000000}%
\pgfsetstrokecolor{currentstroke}%
\pgfsetdash{}{0pt}%
\pgfpathmoveto{\pgfqpoint{2.271736in}{3.027527in}}%
\pgfpathlineto{\pgfqpoint{2.285723in}{3.003040in}}%
\pgfpathlineto{\pgfqpoint{2.299698in}{2.978756in}}%
\pgfpathlineto{\pgfqpoint{2.313663in}{2.954676in}}%
\pgfpathlineto{\pgfqpoint{2.327616in}{2.930797in}}%
\pgfpathlineto{\pgfqpoint{2.336789in}{2.920411in}}%
\pgfpathlineto{\pgfqpoint{2.345937in}{2.910366in}}%
\pgfpathlineto{\pgfqpoint{2.355062in}{2.900658in}}%
\pgfpathlineto{\pgfqpoint{2.364162in}{2.891282in}}%
\pgfpathlineto{\pgfqpoint{2.350271in}{2.914614in}}%
\pgfpathlineto{\pgfqpoint{2.336369in}{2.938146in}}%
\pgfpathlineto{\pgfqpoint{2.322456in}{2.961880in}}%
\pgfpathlineto{\pgfqpoint{2.308532in}{2.985817in}}%
\pgfpathlineto{\pgfqpoint{2.299370in}{2.995733in}}%
\pgfpathlineto{\pgfqpoint{2.290183in}{3.005987in}}%
\pgfpathlineto{\pgfqpoint{2.280972in}{3.016583in}}%
\pgfpathlineto{\pgfqpoint{2.271736in}{3.027527in}}%
\pgfpathclose%
\pgfusepath{fill}%
\end{pgfscope}%
\begin{pgfscope}%
\pgfpathrectangle{\pgfqpoint{1.254980in}{0.150000in}}{\pgfqpoint{5.490039in}{5.490039in}}%
\pgfusepath{clip}%
\pgfsetbuttcap%
\pgfsetroundjoin%
\definecolor{currentfill}{rgb}{0.440137,0.811138,0.340967}%
\pgfsetfillcolor{currentfill}%
\pgfsetfillopacity{0.700000}%
\pgfsetlinewidth{0.000000pt}%
\definecolor{currentstroke}{rgb}{0.000000,0.000000,0.000000}%
\pgfsetstrokecolor{currentstroke}%
\pgfsetdash{}{0pt}%
\pgfpathmoveto{\pgfqpoint{2.084079in}{3.397611in}}%
\pgfpathlineto{\pgfqpoint{2.098209in}{3.370221in}}%
\pgfpathlineto{\pgfqpoint{2.112323in}{3.343060in}}%
\pgfpathlineto{\pgfqpoint{2.126423in}{3.316126in}}%
\pgfpathlineto{\pgfqpoint{2.140510in}{3.289415in}}%
\pgfpathlineto{\pgfqpoint{2.149841in}{3.278196in}}%
\pgfpathlineto{\pgfqpoint{2.159145in}{3.267322in}}%
\pgfpathlineto{\pgfqpoint{2.168424in}{3.256790in}}%
\pgfpathlineto{\pgfqpoint{2.177678in}{3.246594in}}%
\pgfpathlineto{\pgfqpoint{2.163658in}{3.272759in}}%
\pgfpathlineto{\pgfqpoint{2.149624in}{3.299146in}}%
\pgfpathlineto{\pgfqpoint{2.135577in}{3.325759in}}%
\pgfpathlineto{\pgfqpoint{2.121515in}{3.352597in}}%
\pgfpathlineto{\pgfqpoint{2.112196in}{3.363332in}}%
\pgfpathlineto{\pgfqpoint{2.102850in}{3.374409in}}%
\pgfpathlineto{\pgfqpoint{2.093478in}{3.385834in}}%
\pgfpathlineto{\pgfqpoint{2.084079in}{3.397611in}}%
\pgfpathclose%
\pgfusepath{fill}%
\end{pgfscope}%
\begin{pgfscope}%
\pgfpathrectangle{\pgfqpoint{1.254980in}{0.150000in}}{\pgfqpoint{5.490039in}{5.490039in}}%
\pgfusepath{clip}%
\pgfsetbuttcap%
\pgfsetroundjoin%
\definecolor{currentfill}{rgb}{0.280255,0.165693,0.476498}%
\pgfsetfillcolor{currentfill}%
\pgfsetfillopacity{0.700000}%
\pgfsetlinewidth{0.000000pt}%
\definecolor{currentstroke}{rgb}{0.000000,0.000000,0.000000}%
\pgfsetstrokecolor{currentstroke}%
\pgfsetdash{}{0pt}%
\pgfpathmoveto{\pgfqpoint{3.335366in}{1.625249in}}%
\pgfpathlineto{\pgfqpoint{3.348988in}{1.613270in}}%
\pgfpathlineto{\pgfqpoint{3.362610in}{1.601424in}}%
\pgfpathlineto{\pgfqpoint{3.376232in}{1.589709in}}%
\pgfpathlineto{\pgfqpoint{3.389855in}{1.578126in}}%
\pgfpathlineto{\pgfqpoint{3.398139in}{1.577069in}}%
\pgfpathlineto{\pgfqpoint{3.406409in}{1.576266in}}%
\pgfpathlineto{\pgfqpoint{3.414667in}{1.575713in}}%
\pgfpathlineto{\pgfqpoint{3.422913in}{1.575406in}}%
\pgfpathlineto{\pgfqpoint{3.409324in}{1.586488in}}%
\pgfpathlineto{\pgfqpoint{3.395736in}{1.597700in}}%
\pgfpathlineto{\pgfqpoint{3.382148in}{1.609044in}}%
\pgfpathlineto{\pgfqpoint{3.368561in}{1.620521in}}%
\pgfpathlineto{\pgfqpoint{3.360282in}{1.621323in}}%
\pgfpathlineto{\pgfqpoint{3.351990in}{1.622375in}}%
\pgfpathlineto{\pgfqpoint{3.343684in}{1.623683in}}%
\pgfpathlineto{\pgfqpoint{3.335366in}{1.625249in}}%
\pgfpathclose%
\pgfusepath{fill}%
\end{pgfscope}%
\begin{pgfscope}%
\pgfpathrectangle{\pgfqpoint{1.254980in}{0.150000in}}{\pgfqpoint{5.490039in}{5.490039in}}%
\pgfusepath{clip}%
\pgfsetbuttcap%
\pgfsetroundjoin%
\definecolor{currentfill}{rgb}{0.267004,0.004874,0.329415}%
\pgfsetfillcolor{currentfill}%
\pgfsetfillopacity{0.700000}%
\pgfsetlinewidth{0.000000pt}%
\definecolor{currentstroke}{rgb}{0.000000,0.000000,0.000000}%
\pgfsetstrokecolor{currentstroke}%
\pgfsetdash{}{0pt}%
\pgfpathmoveto{\pgfqpoint{3.977261in}{1.299719in}}%
\pgfpathlineto{\pgfqpoint{3.990915in}{1.294123in}}%
\pgfpathlineto{\pgfqpoint{4.004575in}{1.288644in}}%
\pgfpathlineto{\pgfqpoint{4.018240in}{1.283281in}}%
\pgfpathlineto{\pgfqpoint{4.031911in}{1.278034in}}%
\pgfpathlineto{\pgfqpoint{4.039825in}{1.284236in}}%
\pgfpathlineto{\pgfqpoint{4.047733in}{1.290600in}}%
\pgfpathlineto{\pgfqpoint{4.055634in}{1.297121in}}%
\pgfpathlineto{\pgfqpoint{4.063528in}{1.303797in}}%
\pgfpathlineto{\pgfqpoint{4.049873in}{1.308605in}}%
\pgfpathlineto{\pgfqpoint{4.036225in}{1.313528in}}%
\pgfpathlineto{\pgfqpoint{4.022582in}{1.318567in}}%
\pgfpathlineto{\pgfqpoint{4.008946in}{1.323722in}}%
\pgfpathlineto{\pgfqpoint{4.001035in}{1.317480in}}%
\pgfpathlineto{\pgfqpoint{3.993117in}{1.311396in}}%
\pgfpathlineto{\pgfqpoint{3.985193in}{1.305475in}}%
\pgfpathlineto{\pgfqpoint{3.977261in}{1.299719in}}%
\pgfpathclose%
\pgfusepath{fill}%
\end{pgfscope}%
\begin{pgfscope}%
\pgfpathrectangle{\pgfqpoint{1.254980in}{0.150000in}}{\pgfqpoint{5.490039in}{5.490039in}}%
\pgfusepath{clip}%
\pgfsetbuttcap%
\pgfsetroundjoin%
\definecolor{currentfill}{rgb}{0.132444,0.552216,0.553018}%
\pgfsetfillcolor{currentfill}%
\pgfsetfillopacity{0.700000}%
\pgfsetlinewidth{0.000000pt}%
\definecolor{currentstroke}{rgb}{0.000000,0.000000,0.000000}%
\pgfsetstrokecolor{currentstroke}%
\pgfsetdash{}{0pt}%
\pgfpathmoveto{\pgfqpoint{2.513554in}{2.611146in}}%
\pgfpathlineto{\pgfqpoint{2.527406in}{2.589943in}}%
\pgfpathlineto{\pgfqpoint{2.541249in}{2.568919in}}%
\pgfpathlineto{\pgfqpoint{2.555084in}{2.548075in}}%
\pgfpathlineto{\pgfqpoint{2.568911in}{2.527409in}}%
\pgfpathlineto{\pgfqpoint{2.577877in}{2.518437in}}%
\pgfpathlineto{\pgfqpoint{2.586822in}{2.509797in}}%
\pgfpathlineto{\pgfqpoint{2.595745in}{2.501484in}}%
\pgfpathlineto{\pgfqpoint{2.604646in}{2.493494in}}%
\pgfpathlineto{\pgfqpoint{2.590875in}{2.513615in}}%
\pgfpathlineto{\pgfqpoint{2.577097in}{2.533912in}}%
\pgfpathlineto{\pgfqpoint{2.563311in}{2.554387in}}%
\pgfpathlineto{\pgfqpoint{2.549517in}{2.575041in}}%
\pgfpathlineto{\pgfqpoint{2.540560in}{2.583571in}}%
\pgfpathlineto{\pgfqpoint{2.531581in}{2.592428in}}%
\pgfpathlineto{\pgfqpoint{2.522579in}{2.601618in}}%
\pgfpathlineto{\pgfqpoint{2.513554in}{2.611146in}}%
\pgfpathclose%
\pgfusepath{fill}%
\end{pgfscope}%
\begin{pgfscope}%
\pgfpathrectangle{\pgfqpoint{1.254980in}{0.150000in}}{\pgfqpoint{5.490039in}{5.490039in}}%
\pgfusepath{clip}%
\pgfsetbuttcap%
\pgfsetroundjoin%
\definecolor{currentfill}{rgb}{0.279566,0.067836,0.391917}%
\pgfsetfillcolor{currentfill}%
\pgfsetfillopacity{0.700000}%
\pgfsetlinewidth{0.000000pt}%
\definecolor{currentstroke}{rgb}{0.000000,0.000000,0.000000}%
\pgfsetstrokecolor{currentstroke}%
\pgfsetdash{}{0pt}%
\pgfpathmoveto{\pgfqpoint{4.517752in}{1.385847in}}%
\pgfpathlineto{\pgfqpoint{4.531560in}{1.385465in}}%
\pgfpathlineto{\pgfqpoint{4.545377in}{1.385193in}}%
\pgfpathlineto{\pgfqpoint{4.559204in}{1.385032in}}%
\pgfpathlineto{\pgfqpoint{4.573040in}{1.384982in}}%
\pgfpathlineto{\pgfqpoint{4.580777in}{1.396319in}}%
\pgfpathlineto{\pgfqpoint{4.588510in}{1.407722in}}%
\pgfpathlineto{\pgfqpoint{4.596239in}{1.419191in}}%
\pgfpathlineto{\pgfqpoint{4.603963in}{1.430721in}}%
\pgfpathlineto{\pgfqpoint{4.590131in}{1.430410in}}%
\pgfpathlineto{\pgfqpoint{4.576309in}{1.430210in}}%
\pgfpathlineto{\pgfqpoint{4.562496in}{1.430120in}}%
\pgfpathlineto{\pgfqpoint{4.548694in}{1.430143in}}%
\pgfpathlineto{\pgfqpoint{4.540965in}{1.418967in}}%
\pgfpathlineto{\pgfqpoint{4.533232in}{1.407858in}}%
\pgfpathlineto{\pgfqpoint{4.525494in}{1.396817in}}%
\pgfpathlineto{\pgfqpoint{4.517752in}{1.385847in}}%
\pgfpathclose%
\pgfusepath{fill}%
\end{pgfscope}%
\begin{pgfscope}%
\pgfpathrectangle{\pgfqpoint{1.254980in}{0.150000in}}{\pgfqpoint{5.490039in}{5.490039in}}%
\pgfusepath{clip}%
\pgfsetbuttcap%
\pgfsetroundjoin%
\definecolor{currentfill}{rgb}{0.282327,0.094955,0.417331}%
\pgfsetfillcolor{currentfill}%
\pgfsetfillopacity{0.700000}%
\pgfsetlinewidth{0.000000pt}%
\definecolor{currentstroke}{rgb}{0.000000,0.000000,0.000000}%
\pgfsetstrokecolor{currentstroke}%
\pgfsetdash{}{0pt}%
\pgfpathmoveto{\pgfqpoint{4.603963in}{1.430721in}}%
\pgfpathlineto{\pgfqpoint{4.617806in}{1.431143in}}%
\pgfpathlineto{\pgfqpoint{4.631658in}{1.431677in}}%
\pgfpathlineto{\pgfqpoint{4.645521in}{1.432320in}}%
\pgfpathlineto{\pgfqpoint{4.659395in}{1.433075in}}%
\pgfpathlineto{\pgfqpoint{4.667112in}{1.445016in}}%
\pgfpathlineto{\pgfqpoint{4.674825in}{1.457009in}}%
\pgfpathlineto{\pgfqpoint{4.682533in}{1.469051in}}%
\pgfpathlineto{\pgfqpoint{4.690238in}{1.481140in}}%
\pgfpathlineto{\pgfqpoint{4.676368in}{1.480040in}}%
\pgfpathlineto{\pgfqpoint{4.662508in}{1.479050in}}%
\pgfpathlineto{\pgfqpoint{4.648658in}{1.478171in}}%
\pgfpathlineto{\pgfqpoint{4.634819in}{1.477404in}}%
\pgfpathlineto{\pgfqpoint{4.627112in}{1.465654in}}%
\pgfpathlineto{\pgfqpoint{4.619400in}{1.453956in}}%
\pgfpathlineto{\pgfqpoint{4.611684in}{1.442310in}}%
\pgfpathlineto{\pgfqpoint{4.603963in}{1.430721in}}%
\pgfpathclose%
\pgfusepath{fill}%
\end{pgfscope}%
\begin{pgfscope}%
\pgfpathrectangle{\pgfqpoint{1.254980in}{0.150000in}}{\pgfqpoint{5.490039in}{5.490039in}}%
\pgfusepath{clip}%
\pgfsetbuttcap%
\pgfsetroundjoin%
\definecolor{currentfill}{rgb}{0.263663,0.237631,0.518762}%
\pgfsetfillcolor{currentfill}%
\pgfsetfillopacity{0.700000}%
\pgfsetlinewidth{0.000000pt}%
\definecolor{currentstroke}{rgb}{0.000000,0.000000,0.000000}%
\pgfsetstrokecolor{currentstroke}%
\pgfsetdash{}{0pt}%
\pgfpathmoveto{\pgfqpoint{4.980244in}{1.715450in}}%
\pgfpathlineto{\pgfqpoint{4.994266in}{1.719220in}}%
\pgfpathlineto{\pgfqpoint{5.008301in}{1.723101in}}%
\pgfpathlineto{\pgfqpoint{5.022348in}{1.727094in}}%
\pgfpathlineto{\pgfqpoint{5.036409in}{1.731197in}}%
\pgfpathlineto{\pgfqpoint{5.044044in}{1.744798in}}%
\pgfpathlineto{\pgfqpoint{5.051675in}{1.758388in}}%
\pgfpathlineto{\pgfqpoint{5.059301in}{1.771964in}}%
\pgfpathlineto{\pgfqpoint{5.066923in}{1.785524in}}%
\pgfpathlineto{\pgfqpoint{5.052860in}{1.781152in}}%
\pgfpathlineto{\pgfqpoint{5.038811in}{1.776890in}}%
\pgfpathlineto{\pgfqpoint{5.024775in}{1.772740in}}%
\pgfpathlineto{\pgfqpoint{5.010752in}{1.768700in}}%
\pgfpathlineto{\pgfqpoint{5.003132in}{1.755403in}}%
\pgfpathlineto{\pgfqpoint{4.995507in}{1.742095in}}%
\pgfpathlineto{\pgfqpoint{4.987877in}{1.728776in}}%
\pgfpathlineto{\pgfqpoint{4.980244in}{1.715450in}}%
\pgfpathclose%
\pgfusepath{fill}%
\end{pgfscope}%
\begin{pgfscope}%
\pgfpathrectangle{\pgfqpoint{1.254980in}{0.150000in}}{\pgfqpoint{5.490039in}{5.490039in}}%
\pgfusepath{clip}%
\pgfsetbuttcap%
\pgfsetroundjoin%
\definecolor{currentfill}{rgb}{0.227802,0.326594,0.546532}%
\pgfsetfillcolor{currentfill}%
\pgfsetfillopacity{0.700000}%
\pgfsetlinewidth{0.000000pt}%
\definecolor{currentstroke}{rgb}{0.000000,0.000000,0.000000}%
\pgfsetstrokecolor{currentstroke}%
\pgfsetdash{}{0pt}%
\pgfpathmoveto{\pgfqpoint{5.184137in}{1.913770in}}%
\pgfpathlineto{\pgfqpoint{5.198272in}{1.919190in}}%
\pgfpathlineto{\pgfqpoint{5.212422in}{1.924721in}}%
\pgfpathlineto{\pgfqpoint{5.226586in}{1.930364in}}%
\pgfpathlineto{\pgfqpoint{5.240765in}{1.936118in}}%
\pgfpathlineto{\pgfqpoint{5.248350in}{1.949906in}}%
\pgfpathlineto{\pgfqpoint{5.255930in}{1.963652in}}%
\pgfpathlineto{\pgfqpoint{5.263504in}{1.977354in}}%
\pgfpathlineto{\pgfqpoint{5.271074in}{1.991011in}}%
\pgfpathlineto{\pgfqpoint{5.256893in}{1.985034in}}%
\pgfpathlineto{\pgfqpoint{5.242727in}{1.979170in}}%
\pgfpathlineto{\pgfqpoint{5.228575in}{1.973416in}}%
\pgfpathlineto{\pgfqpoint{5.214437in}{1.967775in}}%
\pgfpathlineto{\pgfqpoint{5.206869in}{1.954334in}}%
\pgfpathlineto{\pgfqpoint{5.199297in}{1.940852in}}%
\pgfpathlineto{\pgfqpoint{5.191719in}{1.927330in}}%
\pgfpathlineto{\pgfqpoint{5.184137in}{1.913770in}}%
\pgfpathclose%
\pgfusepath{fill}%
\end{pgfscope}%
\begin{pgfscope}%
\pgfpathrectangle{\pgfqpoint{1.254980in}{0.150000in}}{\pgfqpoint{5.490039in}{5.490039in}}%
\pgfusepath{clip}%
\pgfsetbuttcap%
\pgfsetroundjoin%
\definecolor{currentfill}{rgb}{0.278791,0.062145,0.386592}%
\pgfsetfillcolor{currentfill}%
\pgfsetfillopacity{0.700000}%
\pgfsetlinewidth{0.000000pt}%
\definecolor{currentstroke}{rgb}{0.000000,0.000000,0.000000}%
\pgfsetstrokecolor{currentstroke}%
\pgfsetdash{}{0pt}%
\pgfpathmoveto{\pgfqpoint{3.640527in}{1.415555in}}%
\pgfpathlineto{\pgfqpoint{3.654145in}{1.406633in}}%
\pgfpathlineto{\pgfqpoint{3.667767in}{1.397833in}}%
\pgfpathlineto{\pgfqpoint{3.681391in}{1.389157in}}%
\pgfpathlineto{\pgfqpoint{3.695018in}{1.380603in}}%
\pgfpathlineto{\pgfqpoint{3.703105in}{1.382968in}}%
\pgfpathlineto{\pgfqpoint{3.711182in}{1.385547in}}%
\pgfpathlineto{\pgfqpoint{3.719249in}{1.388337in}}%
\pgfpathlineto{\pgfqpoint{3.727308in}{1.391333in}}%
\pgfpathlineto{\pgfqpoint{3.713706in}{1.399410in}}%
\pgfpathlineto{\pgfqpoint{3.700108in}{1.407610in}}%
\pgfpathlineto{\pgfqpoint{3.686513in}{1.415932in}}%
\pgfpathlineto{\pgfqpoint{3.672921in}{1.424377in}}%
\pgfpathlineto{\pgfqpoint{3.664837in}{1.421852in}}%
\pgfpathlineto{\pgfqpoint{3.656744in}{1.419537in}}%
\pgfpathlineto{\pgfqpoint{3.648640in}{1.417437in}}%
\pgfpathlineto{\pgfqpoint{3.640527in}{1.415555in}}%
\pgfpathclose%
\pgfusepath{fill}%
\end{pgfscope}%
\begin{pgfscope}%
\pgfpathrectangle{\pgfqpoint{1.254980in}{0.150000in}}{\pgfqpoint{5.490039in}{5.490039in}}%
\pgfusepath{clip}%
\pgfsetbuttcap%
\pgfsetroundjoin%
\definecolor{currentfill}{rgb}{0.276022,0.044167,0.370164}%
\pgfsetfillcolor{currentfill}%
\pgfsetfillopacity{0.700000}%
\pgfsetlinewidth{0.000000pt}%
\definecolor{currentstroke}{rgb}{0.000000,0.000000,0.000000}%
\pgfsetstrokecolor{currentstroke}%
\pgfsetdash{}{0pt}%
\pgfpathmoveto{\pgfqpoint{4.431579in}{1.346893in}}%
\pgfpathlineto{\pgfqpoint{4.445356in}{1.345688in}}%
\pgfpathlineto{\pgfqpoint{4.459142in}{1.344595in}}%
\pgfpathlineto{\pgfqpoint{4.472937in}{1.343613in}}%
\pgfpathlineto{\pgfqpoint{4.486741in}{1.342743in}}%
\pgfpathlineto{\pgfqpoint{4.494500in}{1.353397in}}%
\pgfpathlineto{\pgfqpoint{4.502255in}{1.364134in}}%
\pgfpathlineto{\pgfqpoint{4.510006in}{1.374952in}}%
\pgfpathlineto{\pgfqpoint{4.517752in}{1.385847in}}%
\pgfpathlineto{\pgfqpoint{4.503954in}{1.386342in}}%
\pgfpathlineto{\pgfqpoint{4.490166in}{1.386947in}}%
\pgfpathlineto{\pgfqpoint{4.476386in}{1.387664in}}%
\pgfpathlineto{\pgfqpoint{4.462616in}{1.388493in}}%
\pgfpathlineto{\pgfqpoint{4.454864in}{1.377968in}}%
\pgfpathlineto{\pgfqpoint{4.447107in}{1.367524in}}%
\pgfpathlineto{\pgfqpoint{4.439345in}{1.357165in}}%
\pgfpathlineto{\pgfqpoint{4.431579in}{1.346893in}}%
\pgfpathclose%
\pgfusepath{fill}%
\end{pgfscope}%
\begin{pgfscope}%
\pgfpathrectangle{\pgfqpoint{1.254980in}{0.150000in}}{\pgfqpoint{5.490039in}{5.490039in}}%
\pgfusepath{clip}%
\pgfsetbuttcap%
\pgfsetroundjoin%
\definecolor{currentfill}{rgb}{0.271305,0.019942,0.347269}%
\pgfsetfillcolor{currentfill}%
\pgfsetfillopacity{0.700000}%
\pgfsetlinewidth{0.000000pt}%
\definecolor{currentstroke}{rgb}{0.000000,0.000000,0.000000}%
\pgfsetstrokecolor{currentstroke}%
\pgfsetdash{}{0pt}%
\pgfpathmoveto{\pgfqpoint{3.836250in}{1.331069in}}%
\pgfpathlineto{\pgfqpoint{3.849885in}{1.324075in}}%
\pgfpathlineto{\pgfqpoint{3.863525in}{1.317201in}}%
\pgfpathlineto{\pgfqpoint{3.877169in}{1.310444in}}%
\pgfpathlineto{\pgfqpoint{3.890818in}{1.303806in}}%
\pgfpathlineto{\pgfqpoint{3.898800in}{1.308398in}}%
\pgfpathlineto{\pgfqpoint{3.906773in}{1.313175in}}%
\pgfpathlineto{\pgfqpoint{3.914739in}{1.318133in}}%
\pgfpathlineto{\pgfqpoint{3.922697in}{1.323269in}}%
\pgfpathlineto{\pgfqpoint{3.909068in}{1.329450in}}%
\pgfpathlineto{\pgfqpoint{3.895444in}{1.335749in}}%
\pgfpathlineto{\pgfqpoint{3.881825in}{1.342166in}}%
\pgfpathlineto{\pgfqpoint{3.868211in}{1.348702in}}%
\pgfpathlineto{\pgfqpoint{3.860233in}{1.344017in}}%
\pgfpathlineto{\pgfqpoint{3.852247in}{1.339514in}}%
\pgfpathlineto{\pgfqpoint{3.844252in}{1.335197in}}%
\pgfpathlineto{\pgfqpoint{3.836250in}{1.331069in}}%
\pgfpathclose%
\pgfusepath{fill}%
\end{pgfscope}%
\begin{pgfscope}%
\pgfpathrectangle{\pgfqpoint{1.254980in}{0.150000in}}{\pgfqpoint{5.490039in}{5.490039in}}%
\pgfusepath{clip}%
\pgfsetbuttcap%
\pgfsetroundjoin%
\definecolor{currentfill}{rgb}{0.267004,0.004874,0.329415}%
\pgfsetfillcolor{currentfill}%
\pgfsetfillopacity{0.700000}%
\pgfsetlinewidth{0.000000pt}%
\definecolor{currentstroke}{rgb}{0.000000,0.000000,0.000000}%
\pgfsetstrokecolor{currentstroke}%
\pgfsetdash{}{0pt}%
\pgfpathmoveto{\pgfqpoint{4.118206in}{1.285720in}}%
\pgfpathlineto{\pgfqpoint{4.131892in}{1.281487in}}%
\pgfpathlineto{\pgfqpoint{4.145584in}{1.277369in}}%
\pgfpathlineto{\pgfqpoint{4.159282in}{1.273365in}}%
\pgfpathlineto{\pgfqpoint{4.172988in}{1.269474in}}%
\pgfpathlineto{\pgfqpoint{4.180847in}{1.277159in}}%
\pgfpathlineto{\pgfqpoint{4.188700in}{1.284983in}}%
\pgfpathlineto{\pgfqpoint{4.196547in}{1.292943in}}%
\pgfpathlineto{\pgfqpoint{4.204388in}{1.301034in}}%
\pgfpathlineto{\pgfqpoint{4.190696in}{1.304501in}}%
\pgfpathlineto{\pgfqpoint{4.177010in}{1.308082in}}%
\pgfpathlineto{\pgfqpoint{4.163332in}{1.311777in}}%
\pgfpathlineto{\pgfqpoint{4.149660in}{1.315586in}}%
\pgfpathlineto{\pgfqpoint{4.141806in}{1.307912in}}%
\pgfpathlineto{\pgfqpoint{4.133945in}{1.300374in}}%
\pgfpathlineto{\pgfqpoint{4.126079in}{1.292975in}}%
\pgfpathlineto{\pgfqpoint{4.118206in}{1.285720in}}%
\pgfpathclose%
\pgfusepath{fill}%
\end{pgfscope}%
\begin{pgfscope}%
\pgfpathrectangle{\pgfqpoint{1.254980in}{0.150000in}}{\pgfqpoint{5.490039in}{5.490039in}}%
\pgfusepath{clip}%
\pgfsetbuttcap%
\pgfsetroundjoin%
\definecolor{currentfill}{rgb}{0.283229,0.120777,0.440584}%
\pgfsetfillcolor{currentfill}%
\pgfsetfillopacity{0.700000}%
\pgfsetlinewidth{0.000000pt}%
\definecolor{currentstroke}{rgb}{0.000000,0.000000,0.000000}%
\pgfsetstrokecolor{currentstroke}%
\pgfsetdash{}{0pt}%
\pgfpathmoveto{\pgfqpoint{4.690238in}{1.481140in}}%
\pgfpathlineto{\pgfqpoint{4.704119in}{1.482352in}}%
\pgfpathlineto{\pgfqpoint{4.718012in}{1.483674in}}%
\pgfpathlineto{\pgfqpoint{4.731915in}{1.485106in}}%
\pgfpathlineto{\pgfqpoint{4.745829in}{1.486649in}}%
\pgfpathlineto{\pgfqpoint{4.753528in}{1.499119in}}%
\pgfpathlineto{\pgfqpoint{4.761223in}{1.511627in}}%
\pgfpathlineto{\pgfqpoint{4.768913in}{1.524169in}}%
\pgfpathlineto{\pgfqpoint{4.776600in}{1.536744in}}%
\pgfpathlineto{\pgfqpoint{4.762687in}{1.534870in}}%
\pgfpathlineto{\pgfqpoint{4.748786in}{1.533106in}}%
\pgfpathlineto{\pgfqpoint{4.734896in}{1.531454in}}%
\pgfpathlineto{\pgfqpoint{4.721016in}{1.529912in}}%
\pgfpathlineto{\pgfqpoint{4.713328in}{1.517662in}}%
\pgfpathlineto{\pgfqpoint{4.705635in}{1.505449in}}%
\pgfpathlineto{\pgfqpoint{4.697939in}{1.493274in}}%
\pgfpathlineto{\pgfqpoint{4.690238in}{1.481140in}}%
\pgfpathclose%
\pgfusepath{fill}%
\end{pgfscope}%
\begin{pgfscope}%
\pgfpathrectangle{\pgfqpoint{1.254980in}{0.150000in}}{\pgfqpoint{5.490039in}{5.490039in}}%
\pgfusepath{clip}%
\pgfsetbuttcap%
\pgfsetroundjoin%
\definecolor{currentfill}{rgb}{0.282290,0.145912,0.461510}%
\pgfsetfillcolor{currentfill}%
\pgfsetfillopacity{0.700000}%
\pgfsetlinewidth{0.000000pt}%
\definecolor{currentstroke}{rgb}{0.000000,0.000000,0.000000}%
\pgfsetstrokecolor{currentstroke}%
\pgfsetdash{}{0pt}%
\pgfpathmoveto{\pgfqpoint{3.389855in}{1.578126in}}%
\pgfpathlineto{\pgfqpoint{3.403479in}{1.566674in}}%
\pgfpathlineto{\pgfqpoint{3.417103in}{1.555353in}}%
\pgfpathlineto{\pgfqpoint{3.430727in}{1.544162in}}%
\pgfpathlineto{\pgfqpoint{3.444353in}{1.533101in}}%
\pgfpathlineto{\pgfqpoint{3.452603in}{1.532551in}}%
\pgfpathlineto{\pgfqpoint{3.460840in}{1.532251in}}%
\pgfpathlineto{\pgfqpoint{3.469065in}{1.532196in}}%
\pgfpathlineto{\pgfqpoint{3.477278in}{1.532382in}}%
\pgfpathlineto{\pgfqpoint{3.463685in}{1.542944in}}%
\pgfpathlineto{\pgfqpoint{3.450093in}{1.553635in}}%
\pgfpathlineto{\pgfqpoint{3.436502in}{1.564455in}}%
\pgfpathlineto{\pgfqpoint{3.422913in}{1.575406in}}%
\pgfpathlineto{\pgfqpoint{3.414667in}{1.575713in}}%
\pgfpathlineto{\pgfqpoint{3.406409in}{1.576266in}}%
\pgfpathlineto{\pgfqpoint{3.398139in}{1.577069in}}%
\pgfpathlineto{\pgfqpoint{3.389855in}{1.578126in}}%
\pgfpathclose%
\pgfusepath{fill}%
\end{pgfscope}%
\begin{pgfscope}%
\pgfpathrectangle{\pgfqpoint{1.254980in}{0.150000in}}{\pgfqpoint{5.490039in}{5.490039in}}%
\pgfusepath{clip}%
\pgfsetbuttcap%
\pgfsetroundjoin%
\definecolor{currentfill}{rgb}{0.239374,0.735588,0.455688}%
\pgfsetfillcolor{currentfill}%
\pgfsetfillopacity{0.700000}%
\pgfsetlinewidth{0.000000pt}%
\definecolor{currentstroke}{rgb}{0.000000,0.000000,0.000000}%
\pgfsetstrokecolor{currentstroke}%
\pgfsetdash{}{0pt}%
\pgfpathmoveto{\pgfqpoint{2.215668in}{3.127549in}}%
\pgfpathlineto{\pgfqpoint{2.229703in}{3.102230in}}%
\pgfpathlineto{\pgfqpoint{2.243726in}{3.077121in}}%
\pgfpathlineto{\pgfqpoint{2.257737in}{3.052220in}}%
\pgfpathlineto{\pgfqpoint{2.271736in}{3.027527in}}%
\pgfpathlineto{\pgfqpoint{2.280972in}{3.016583in}}%
\pgfpathlineto{\pgfqpoint{2.290183in}{3.005987in}}%
\pgfpathlineto{\pgfqpoint{2.299370in}{2.995733in}}%
\pgfpathlineto{\pgfqpoint{2.308532in}{2.985817in}}%
\pgfpathlineto{\pgfqpoint{2.294597in}{3.009957in}}%
\pgfpathlineto{\pgfqpoint{2.280650in}{3.034304in}}%
\pgfpathlineto{\pgfqpoint{2.266692in}{3.058858in}}%
\pgfpathlineto{\pgfqpoint{2.252722in}{3.083620in}}%
\pgfpathlineto{\pgfqpoint{2.243497in}{3.094082in}}%
\pgfpathlineto{\pgfqpoint{2.234246in}{3.104887in}}%
\pgfpathlineto{\pgfqpoint{2.224970in}{3.116041in}}%
\pgfpathlineto{\pgfqpoint{2.215668in}{3.127549in}}%
\pgfpathclose%
\pgfusepath{fill}%
\end{pgfscope}%
\begin{pgfscope}%
\pgfpathrectangle{\pgfqpoint{1.254980in}{0.150000in}}{\pgfqpoint{5.490039in}{5.490039in}}%
\pgfusepath{clip}%
\pgfsetbuttcap%
\pgfsetroundjoin%
\definecolor{currentfill}{rgb}{0.272594,0.025563,0.353093}%
\pgfsetfillcolor{currentfill}%
\pgfsetfillopacity{0.700000}%
\pgfsetlinewidth{0.000000pt}%
\definecolor{currentstroke}{rgb}{0.000000,0.000000,0.000000}%
\pgfsetstrokecolor{currentstroke}%
\pgfsetdash{}{0pt}%
\pgfpathmoveto{\pgfqpoint{4.345416in}{1.314244in}}%
\pgfpathlineto{\pgfqpoint{4.359166in}{1.312200in}}%
\pgfpathlineto{\pgfqpoint{4.372925in}{1.310268in}}%
\pgfpathlineto{\pgfqpoint{4.386692in}{1.308448in}}%
\pgfpathlineto{\pgfqpoint{4.400468in}{1.306740in}}%
\pgfpathlineto{\pgfqpoint{4.408253in}{1.316632in}}%
\pgfpathlineto{\pgfqpoint{4.416033in}{1.326623in}}%
\pgfpathlineto{\pgfqpoint{4.423808in}{1.336711in}}%
\pgfpathlineto{\pgfqpoint{4.431579in}{1.346893in}}%
\pgfpathlineto{\pgfqpoint{4.417811in}{1.348209in}}%
\pgfpathlineto{\pgfqpoint{4.404052in}{1.349638in}}%
\pgfpathlineto{\pgfqpoint{4.390301in}{1.351178in}}%
\pgfpathlineto{\pgfqpoint{4.376559in}{1.352830in}}%
\pgfpathlineto{\pgfqpoint{4.368781in}{1.343034in}}%
\pgfpathlineto{\pgfqpoint{4.360997in}{1.333336in}}%
\pgfpathlineto{\pgfqpoint{4.353209in}{1.323738in}}%
\pgfpathlineto{\pgfqpoint{4.345416in}{1.314244in}}%
\pgfpathclose%
\pgfusepath{fill}%
\end{pgfscope}%
\begin{pgfscope}%
\pgfpathrectangle{\pgfqpoint{1.254980in}{0.150000in}}{\pgfqpoint{5.490039in}{5.490039in}}%
\pgfusepath{clip}%
\pgfsetbuttcap%
\pgfsetroundjoin%
\definecolor{currentfill}{rgb}{0.122606,0.585371,0.546557}%
\pgfsetfillcolor{currentfill}%
\pgfsetfillopacity{0.700000}%
\pgfsetlinewidth{0.000000pt}%
\definecolor{currentstroke}{rgb}{0.000000,0.000000,0.000000}%
\pgfsetstrokecolor{currentstroke}%
\pgfsetdash{}{0pt}%
\pgfpathmoveto{\pgfqpoint{2.458064in}{2.697782in}}%
\pgfpathlineto{\pgfqpoint{2.471950in}{2.675847in}}%
\pgfpathlineto{\pgfqpoint{2.485827in}{2.654097in}}%
\pgfpathlineto{\pgfqpoint{2.499695in}{2.632530in}}%
\pgfpathlineto{\pgfqpoint{2.513554in}{2.611146in}}%
\pgfpathlineto{\pgfqpoint{2.522579in}{2.601618in}}%
\pgfpathlineto{\pgfqpoint{2.531581in}{2.592428in}}%
\pgfpathlineto{\pgfqpoint{2.540560in}{2.583571in}}%
\pgfpathlineto{\pgfqpoint{2.549517in}{2.575041in}}%
\pgfpathlineto{\pgfqpoint{2.535715in}{2.595875in}}%
\pgfpathlineto{\pgfqpoint{2.521906in}{2.616891in}}%
\pgfpathlineto{\pgfqpoint{2.508088in}{2.638088in}}%
\pgfpathlineto{\pgfqpoint{2.494261in}{2.659469in}}%
\pgfpathlineto{\pgfqpoint{2.485247in}{2.668542in}}%
\pgfpathlineto{\pgfqpoint{2.476209in}{2.677948in}}%
\pgfpathlineto{\pgfqpoint{2.467148in}{2.687693in}}%
\pgfpathlineto{\pgfqpoint{2.458064in}{2.697782in}}%
\pgfpathclose%
\pgfusepath{fill}%
\end{pgfscope}%
\begin{pgfscope}%
\pgfpathrectangle{\pgfqpoint{1.254980in}{0.150000in}}{\pgfqpoint{5.490039in}{5.490039in}}%
\pgfusepath{clip}%
\pgfsetbuttcap%
\pgfsetroundjoin%
\definecolor{currentfill}{rgb}{0.281887,0.150881,0.465405}%
\pgfsetfillcolor{currentfill}%
\pgfsetfillopacity{0.700000}%
\pgfsetlinewidth{0.000000pt}%
\definecolor{currentstroke}{rgb}{0.000000,0.000000,0.000000}%
\pgfsetstrokecolor{currentstroke}%
\pgfsetdash{}{0pt}%
\pgfpathmoveto{\pgfqpoint{4.776600in}{1.536744in}}%
\pgfpathlineto{\pgfqpoint{4.790524in}{1.538728in}}%
\pgfpathlineto{\pgfqpoint{4.804460in}{1.540823in}}%
\pgfpathlineto{\pgfqpoint{4.818408in}{1.543029in}}%
\pgfpathlineto{\pgfqpoint{4.832367in}{1.545345in}}%
\pgfpathlineto{\pgfqpoint{4.840049in}{1.558271in}}%
\pgfpathlineto{\pgfqpoint{4.847727in}{1.571221in}}%
\pgfpathlineto{\pgfqpoint{4.855401in}{1.584191in}}%
\pgfpathlineto{\pgfqpoint{4.863071in}{1.597180in}}%
\pgfpathlineto{\pgfqpoint{4.849112in}{1.594548in}}%
\pgfpathlineto{\pgfqpoint{4.835165in}{1.592026in}}%
\pgfpathlineto{\pgfqpoint{4.821230in}{1.589616in}}%
\pgfpathlineto{\pgfqpoint{4.807307in}{1.587316in}}%
\pgfpathlineto{\pgfqpoint{4.799636in}{1.574637in}}%
\pgfpathlineto{\pgfqpoint{4.791962in}{1.561980in}}%
\pgfpathlineto{\pgfqpoint{4.784283in}{1.549348in}}%
\pgfpathlineto{\pgfqpoint{4.776600in}{1.536744in}}%
\pgfpathclose%
\pgfusepath{fill}%
\end{pgfscope}%
\begin{pgfscope}%
\pgfpathrectangle{\pgfqpoint{1.254980in}{0.150000in}}{\pgfqpoint{5.490039in}{5.490039in}}%
\pgfusepath{clip}%
\pgfsetbuttcap%
\pgfsetroundjoin%
\definecolor{currentfill}{rgb}{0.212395,0.359683,0.551710}%
\pgfsetfillcolor{currentfill}%
\pgfsetfillopacity{0.700000}%
\pgfsetlinewidth{0.000000pt}%
\definecolor{currentstroke}{rgb}{0.000000,0.000000,0.000000}%
\pgfsetstrokecolor{currentstroke}%
\pgfsetdash{}{0pt}%
\pgfpathmoveto{\pgfqpoint{5.271074in}{1.991011in}}%
\pgfpathlineto{\pgfqpoint{5.285270in}{1.997099in}}%
\pgfpathlineto{\pgfqpoint{5.299481in}{2.003299in}}%
\pgfpathlineto{\pgfqpoint{5.313707in}{2.009611in}}%
\pgfpathlineto{\pgfqpoint{5.321274in}{2.023380in}}%
\pgfpathlineto{\pgfqpoint{5.328836in}{2.037097in}}%
\pgfpathlineto{\pgfqpoint{5.336392in}{2.050761in}}%
\pgfpathlineto{\pgfqpoint{5.343943in}{2.064371in}}%
\pgfpathlineto{\pgfqpoint{5.329715in}{2.057853in}}%
\pgfpathlineto{\pgfqpoint{5.315501in}{2.051447in}}%
\pgfpathlineto{\pgfqpoint{5.301303in}{2.045153in}}%
\pgfpathlineto{\pgfqpoint{5.293754in}{2.031693in}}%
\pgfpathlineto{\pgfqpoint{5.286199in}{2.018182in}}%
\pgfpathlineto{\pgfqpoint{5.278639in}{2.004621in}}%
\pgfpathlineto{\pgfqpoint{5.271074in}{1.991011in}}%
\pgfpathclose%
\pgfusepath{fill}%
\end{pgfscope}%
\begin{pgfscope}%
\pgfpathrectangle{\pgfqpoint{1.254980in}{0.150000in}}{\pgfqpoint{5.490039in}{5.490039in}}%
\pgfusepath{clip}%
\pgfsetbuttcap%
\pgfsetroundjoin%
\definecolor{currentfill}{rgb}{0.250425,0.274290,0.533103}%
\pgfsetfillcolor{currentfill}%
\pgfsetfillopacity{0.700000}%
\pgfsetlinewidth{0.000000pt}%
\definecolor{currentstroke}{rgb}{0.000000,0.000000,0.000000}%
\pgfsetstrokecolor{currentstroke}%
\pgfsetdash{}{0pt}%
\pgfpathmoveto{\pgfqpoint{5.066923in}{1.785524in}}%
\pgfpathlineto{\pgfqpoint{5.080999in}{1.790008in}}%
\pgfpathlineto{\pgfqpoint{5.095088in}{1.794602in}}%
\pgfpathlineto{\pgfqpoint{5.109192in}{1.799308in}}%
\pgfpathlineto{\pgfqpoint{5.123309in}{1.804124in}}%
\pgfpathlineto{\pgfqpoint{5.130928in}{1.817927in}}%
\pgfpathlineto{\pgfqpoint{5.138543in}{1.831705in}}%
\pgfpathlineto{\pgfqpoint{5.146153in}{1.845457in}}%
\pgfpathlineto{\pgfqpoint{5.153759in}{1.859182in}}%
\pgfpathlineto{\pgfqpoint{5.139640in}{1.854112in}}%
\pgfpathlineto{\pgfqpoint{5.125535in}{1.849152in}}%
\pgfpathlineto{\pgfqpoint{5.111443in}{1.844304in}}%
\pgfpathlineto{\pgfqpoint{5.097366in}{1.839567in}}%
\pgfpathlineto{\pgfqpoint{5.089762in}{1.826090in}}%
\pgfpathlineto{\pgfqpoint{5.082153in}{1.812589in}}%
\pgfpathlineto{\pgfqpoint{5.074540in}{1.799066in}}%
\pgfpathlineto{\pgfqpoint{5.066923in}{1.785524in}}%
\pgfpathclose%
\pgfusepath{fill}%
\end{pgfscope}%
\begin{pgfscope}%
\pgfpathrectangle{\pgfqpoint{1.254980in}{0.150000in}}{\pgfqpoint{5.490039in}{5.490039in}}%
\pgfusepath{clip}%
\pgfsetbuttcap%
\pgfsetroundjoin%
\definecolor{currentfill}{rgb}{0.283187,0.125848,0.444960}%
\pgfsetfillcolor{currentfill}%
\pgfsetfillopacity{0.700000}%
\pgfsetlinewidth{0.000000pt}%
\definecolor{currentstroke}{rgb}{0.000000,0.000000,0.000000}%
\pgfsetstrokecolor{currentstroke}%
\pgfsetdash{}{0pt}%
\pgfpathmoveto{\pgfqpoint{3.444353in}{1.533101in}}%
\pgfpathlineto{\pgfqpoint{3.457980in}{1.522168in}}%
\pgfpathlineto{\pgfqpoint{3.471607in}{1.511364in}}%
\pgfpathlineto{\pgfqpoint{3.485236in}{1.500689in}}%
\pgfpathlineto{\pgfqpoint{3.498866in}{1.490141in}}%
\pgfpathlineto{\pgfqpoint{3.507083in}{1.490097in}}%
\pgfpathlineto{\pgfqpoint{3.515288in}{1.490298in}}%
\pgfpathlineto{\pgfqpoint{3.523482in}{1.490740in}}%
\pgfpathlineto{\pgfqpoint{3.531664in}{1.491418in}}%
\pgfpathlineto{\pgfqpoint{3.518065in}{1.501468in}}%
\pgfpathlineto{\pgfqpoint{3.504468in}{1.511645in}}%
\pgfpathlineto{\pgfqpoint{3.490872in}{1.521949in}}%
\pgfpathlineto{\pgfqpoint{3.477278in}{1.532382in}}%
\pgfpathlineto{\pgfqpoint{3.469065in}{1.532196in}}%
\pgfpathlineto{\pgfqpoint{3.460840in}{1.532251in}}%
\pgfpathlineto{\pgfqpoint{3.452603in}{1.532551in}}%
\pgfpathlineto{\pgfqpoint{3.444353in}{1.533101in}}%
\pgfpathclose%
\pgfusepath{fill}%
\end{pgfscope}%
\begin{pgfscope}%
\pgfpathrectangle{\pgfqpoint{1.254980in}{0.150000in}}{\pgfqpoint{5.490039in}{5.490039in}}%
\pgfusepath{clip}%
\pgfsetbuttcap%
\pgfsetroundjoin%
\definecolor{currentfill}{rgb}{0.268510,0.009605,0.335427}%
\pgfsetfillcolor{currentfill}%
\pgfsetfillopacity{0.700000}%
\pgfsetlinewidth{0.000000pt}%
\definecolor{currentstroke}{rgb}{0.000000,0.000000,0.000000}%
\pgfsetstrokecolor{currentstroke}%
\pgfsetdash{}{0pt}%
\pgfpathmoveto{\pgfqpoint{4.259230in}{1.288300in}}%
\pgfpathlineto{\pgfqpoint{4.272959in}{1.285399in}}%
\pgfpathlineto{\pgfqpoint{4.286695in}{1.282610in}}%
\pgfpathlineto{\pgfqpoint{4.300440in}{1.279935in}}%
\pgfpathlineto{\pgfqpoint{4.314192in}{1.277371in}}%
\pgfpathlineto{\pgfqpoint{4.322005in}{1.286417in}}%
\pgfpathlineto{\pgfqpoint{4.329814in}{1.295580in}}%
\pgfpathlineto{\pgfqpoint{4.337617in}{1.304857in}}%
\pgfpathlineto{\pgfqpoint{4.345416in}{1.314244in}}%
\pgfpathlineto{\pgfqpoint{4.331673in}{1.316400in}}%
\pgfpathlineto{\pgfqpoint{4.317939in}{1.318668in}}%
\pgfpathlineto{\pgfqpoint{4.304212in}{1.321049in}}%
\pgfpathlineto{\pgfqpoint{4.290494in}{1.323543in}}%
\pgfpathlineto{\pgfqpoint{4.282686in}{1.314557in}}%
\pgfpathlineto{\pgfqpoint{4.274873in}{1.305686in}}%
\pgfpathlineto{\pgfqpoint{4.267054in}{1.296932in}}%
\pgfpathlineto{\pgfqpoint{4.259230in}{1.288300in}}%
\pgfpathclose%
\pgfusepath{fill}%
\end{pgfscope}%
\begin{pgfscope}%
\pgfpathrectangle{\pgfqpoint{1.254980in}{0.150000in}}{\pgfqpoint{5.490039in}{5.490039in}}%
\pgfusepath{clip}%
\pgfsetbuttcap%
\pgfsetroundjoin%
\definecolor{currentfill}{rgb}{0.277018,0.050344,0.375715}%
\pgfsetfillcolor{currentfill}%
\pgfsetfillopacity{0.700000}%
\pgfsetlinewidth{0.000000pt}%
\definecolor{currentstroke}{rgb}{0.000000,0.000000,0.000000}%
\pgfsetstrokecolor{currentstroke}%
\pgfsetdash{}{0pt}%
\pgfpathmoveto{\pgfqpoint{3.695018in}{1.380603in}}%
\pgfpathlineto{\pgfqpoint{3.708648in}{1.372171in}}%
\pgfpathlineto{\pgfqpoint{3.722281in}{1.363860in}}%
\pgfpathlineto{\pgfqpoint{3.735918in}{1.355671in}}%
\pgfpathlineto{\pgfqpoint{3.749558in}{1.347603in}}%
\pgfpathlineto{\pgfqpoint{3.757619in}{1.350451in}}%
\pgfpathlineto{\pgfqpoint{3.765671in}{1.353509in}}%
\pgfpathlineto{\pgfqpoint{3.773714in}{1.356772in}}%
\pgfpathlineto{\pgfqpoint{3.781748in}{1.360238in}}%
\pgfpathlineto{\pgfqpoint{3.768133in}{1.367830in}}%
\pgfpathlineto{\pgfqpoint{3.754521in}{1.375543in}}%
\pgfpathlineto{\pgfqpoint{3.740913in}{1.383377in}}%
\pgfpathlineto{\pgfqpoint{3.727308in}{1.391333in}}%
\pgfpathlineto{\pgfqpoint{3.719249in}{1.388337in}}%
\pgfpathlineto{\pgfqpoint{3.711182in}{1.385547in}}%
\pgfpathlineto{\pgfqpoint{3.703105in}{1.382968in}}%
\pgfpathlineto{\pgfqpoint{3.695018in}{1.380603in}}%
\pgfpathclose%
\pgfusepath{fill}%
\end{pgfscope}%
\begin{pgfscope}%
\pgfpathrectangle{\pgfqpoint{1.254980in}{0.150000in}}{\pgfqpoint{5.490039in}{5.490039in}}%
\pgfusepath{clip}%
\pgfsetbuttcap%
\pgfsetroundjoin%
\definecolor{currentfill}{rgb}{0.277134,0.185228,0.489898}%
\pgfsetfillcolor{currentfill}%
\pgfsetfillopacity{0.700000}%
\pgfsetlinewidth{0.000000pt}%
\definecolor{currentstroke}{rgb}{0.000000,0.000000,0.000000}%
\pgfsetstrokecolor{currentstroke}%
\pgfsetdash{}{0pt}%
\pgfpathmoveto{\pgfqpoint{4.863071in}{1.597180in}}%
\pgfpathlineto{\pgfqpoint{4.877042in}{1.599922in}}%
\pgfpathlineto{\pgfqpoint{4.891024in}{1.602775in}}%
\pgfpathlineto{\pgfqpoint{4.905019in}{1.605739in}}%
\pgfpathlineto{\pgfqpoint{4.919027in}{1.608813in}}%
\pgfpathlineto{\pgfqpoint{4.926693in}{1.622124in}}%
\pgfpathlineto{\pgfqpoint{4.934356in}{1.635445in}}%
\pgfpathlineto{\pgfqpoint{4.942014in}{1.648774in}}%
\pgfpathlineto{\pgfqpoint{4.949668in}{1.662108in}}%
\pgfpathlineto{\pgfqpoint{4.935660in}{1.658733in}}%
\pgfpathlineto{\pgfqpoint{4.921665in}{1.655469in}}%
\pgfpathlineto{\pgfqpoint{4.907681in}{1.652315in}}%
\pgfpathlineto{\pgfqpoint{4.893710in}{1.649273in}}%
\pgfpathlineto{\pgfqpoint{4.886057in}{1.636233in}}%
\pgfpathlineto{\pgfqpoint{4.878399in}{1.623203in}}%
\pgfpathlineto{\pgfqpoint{4.870737in}{1.610185in}}%
\pgfpathlineto{\pgfqpoint{4.863071in}{1.597180in}}%
\pgfpathclose%
\pgfusepath{fill}%
\end{pgfscope}%
\begin{pgfscope}%
\pgfpathrectangle{\pgfqpoint{1.254980in}{0.150000in}}{\pgfqpoint{5.490039in}{5.490039in}}%
\pgfusepath{clip}%
\pgfsetbuttcap%
\pgfsetroundjoin%
\definecolor{currentfill}{rgb}{0.267004,0.004874,0.329415}%
\pgfsetfillcolor{currentfill}%
\pgfsetfillopacity{0.700000}%
\pgfsetlinewidth{0.000000pt}%
\definecolor{currentstroke}{rgb}{0.000000,0.000000,0.000000}%
\pgfsetstrokecolor{currentstroke}%
\pgfsetdash{}{0pt}%
\pgfpathmoveto{\pgfqpoint{4.031911in}{1.278034in}}%
\pgfpathlineto{\pgfqpoint{4.045588in}{1.272902in}}%
\pgfpathlineto{\pgfqpoint{4.059270in}{1.267886in}}%
\pgfpathlineto{\pgfqpoint{4.072958in}{1.262984in}}%
\pgfpathlineto{\pgfqpoint{4.086653in}{1.258197in}}%
\pgfpathlineto{\pgfqpoint{4.094551in}{1.264846in}}%
\pgfpathlineto{\pgfqpoint{4.102442in}{1.271651in}}%
\pgfpathlineto{\pgfqpoint{4.110327in}{1.278610in}}%
\pgfpathlineto{\pgfqpoint{4.118206in}{1.285720in}}%
\pgfpathlineto{\pgfqpoint{4.104527in}{1.290067in}}%
\pgfpathlineto{\pgfqpoint{4.090855in}{1.294528in}}%
\pgfpathlineto{\pgfqpoint{4.077188in}{1.299105in}}%
\pgfpathlineto{\pgfqpoint{4.063528in}{1.303797in}}%
\pgfpathlineto{\pgfqpoint{4.055634in}{1.297121in}}%
\pgfpathlineto{\pgfqpoint{4.047733in}{1.290600in}}%
\pgfpathlineto{\pgfqpoint{4.039825in}{1.284236in}}%
\pgfpathlineto{\pgfqpoint{4.031911in}{1.278034in}}%
\pgfpathclose%
\pgfusepath{fill}%
\end{pgfscope}%
\begin{pgfscope}%
\pgfpathrectangle{\pgfqpoint{1.254980in}{0.150000in}}{\pgfqpoint{5.490039in}{5.490039in}}%
\pgfusepath{clip}%
\pgfsetbuttcap%
\pgfsetroundjoin%
\definecolor{currentfill}{rgb}{0.120081,0.622161,0.534946}%
\pgfsetfillcolor{currentfill}%
\pgfsetfillopacity{0.700000}%
\pgfsetlinewidth{0.000000pt}%
\definecolor{currentstroke}{rgb}{0.000000,0.000000,0.000000}%
\pgfsetstrokecolor{currentstroke}%
\pgfsetdash{}{0pt}%
\pgfpathmoveto{\pgfqpoint{2.402428in}{2.787392in}}%
\pgfpathlineto{\pgfqpoint{2.416351in}{2.764706in}}%
\pgfpathlineto{\pgfqpoint{2.430265in}{2.742210in}}%
\pgfpathlineto{\pgfqpoint{2.444169in}{2.719902in}}%
\pgfpathlineto{\pgfqpoint{2.458064in}{2.697782in}}%
\pgfpathlineto{\pgfqpoint{2.467148in}{2.687693in}}%
\pgfpathlineto{\pgfqpoint{2.476209in}{2.677948in}}%
\pgfpathlineto{\pgfqpoint{2.485247in}{2.668542in}}%
\pgfpathlineto{\pgfqpoint{2.494261in}{2.659469in}}%
\pgfpathlineto{\pgfqpoint{2.480426in}{2.681035in}}%
\pgfpathlineto{\pgfqpoint{2.466582in}{2.702787in}}%
\pgfpathlineto{\pgfqpoint{2.452729in}{2.724725in}}%
\pgfpathlineto{\pgfqpoint{2.438867in}{2.746852in}}%
\pgfpathlineto{\pgfqpoint{2.429793in}{2.756473in}}%
\pgfpathlineto{\pgfqpoint{2.420695in}{2.766433in}}%
\pgfpathlineto{\pgfqpoint{2.411574in}{2.776738in}}%
\pgfpathlineto{\pgfqpoint{2.402428in}{2.787392in}}%
\pgfpathclose%
\pgfusepath{fill}%
\end{pgfscope}%
\begin{pgfscope}%
\pgfpathrectangle{\pgfqpoint{1.254980in}{0.150000in}}{\pgfqpoint{5.490039in}{5.490039in}}%
\pgfusepath{clip}%
\pgfsetbuttcap%
\pgfsetroundjoin%
\definecolor{currentfill}{rgb}{0.269944,0.014625,0.341379}%
\pgfsetfillcolor{currentfill}%
\pgfsetfillopacity{0.700000}%
\pgfsetlinewidth{0.000000pt}%
\definecolor{currentstroke}{rgb}{0.000000,0.000000,0.000000}%
\pgfsetstrokecolor{currentstroke}%
\pgfsetdash{}{0pt}%
\pgfpathmoveto{\pgfqpoint{3.890818in}{1.303806in}}%
\pgfpathlineto{\pgfqpoint{3.904472in}{1.297285in}}%
\pgfpathlineto{\pgfqpoint{3.918130in}{1.290882in}}%
\pgfpathlineto{\pgfqpoint{3.931793in}{1.284596in}}%
\pgfpathlineto{\pgfqpoint{3.945461in}{1.278428in}}%
\pgfpathlineto{\pgfqpoint{3.953422in}{1.283483in}}%
\pgfpathlineto{\pgfqpoint{3.961376in}{1.288719in}}%
\pgfpathlineto{\pgfqpoint{3.969322in}{1.294132in}}%
\pgfpathlineto{\pgfqpoint{3.977261in}{1.299719in}}%
\pgfpathlineto{\pgfqpoint{3.963612in}{1.305431in}}%
\pgfpathlineto{\pgfqpoint{3.949969in}{1.311259in}}%
\pgfpathlineto{\pgfqpoint{3.936330in}{1.317205in}}%
\pgfpathlineto{\pgfqpoint{3.922697in}{1.323269in}}%
\pgfpathlineto{\pgfqpoint{3.914739in}{1.318133in}}%
\pgfpathlineto{\pgfqpoint{3.906773in}{1.313175in}}%
\pgfpathlineto{\pgfqpoint{3.898800in}{1.308398in}}%
\pgfpathlineto{\pgfqpoint{3.890818in}{1.303806in}}%
\pgfpathclose%
\pgfusepath{fill}%
\end{pgfscope}%
\begin{pgfscope}%
\pgfpathrectangle{\pgfqpoint{1.254980in}{0.150000in}}{\pgfqpoint{5.490039in}{5.490039in}}%
\pgfusepath{clip}%
\pgfsetbuttcap%
\pgfsetroundjoin%
\definecolor{currentfill}{rgb}{0.319809,0.770914,0.411152}%
\pgfsetfillcolor{currentfill}%
\pgfsetfillopacity{0.700000}%
\pgfsetlinewidth{0.000000pt}%
\definecolor{currentstroke}{rgb}{0.000000,0.000000,0.000000}%
\pgfsetstrokecolor{currentstroke}%
\pgfsetdash{}{0pt}%
\pgfpathmoveto{\pgfqpoint{2.159400in}{3.230957in}}%
\pgfpathlineto{\pgfqpoint{2.173487in}{3.204782in}}%
\pgfpathlineto{\pgfqpoint{2.187560in}{3.178823in}}%
\pgfpathlineto{\pgfqpoint{2.201620in}{3.153079in}}%
\pgfpathlineto{\pgfqpoint{2.215668in}{3.127549in}}%
\pgfpathlineto{\pgfqpoint{2.224970in}{3.116041in}}%
\pgfpathlineto{\pgfqpoint{2.234246in}{3.104887in}}%
\pgfpathlineto{\pgfqpoint{2.243497in}{3.094082in}}%
\pgfpathlineto{\pgfqpoint{2.252722in}{3.083620in}}%
\pgfpathlineto{\pgfqpoint{2.238740in}{3.108593in}}%
\pgfpathlineto{\pgfqpoint{2.224746in}{3.133777in}}%
\pgfpathlineto{\pgfqpoint{2.210739in}{3.159174in}}%
\pgfpathlineto{\pgfqpoint{2.196719in}{3.184787in}}%
\pgfpathlineto{\pgfqpoint{2.187429in}{3.195800in}}%
\pgfpathlineto{\pgfqpoint{2.178113in}{3.207162in}}%
\pgfpathlineto{\pgfqpoint{2.168770in}{3.218880in}}%
\pgfpathlineto{\pgfqpoint{2.159400in}{3.230957in}}%
\pgfpathclose%
\pgfusepath{fill}%
\end{pgfscope}%
\begin{pgfscope}%
\pgfpathrectangle{\pgfqpoint{1.254980in}{0.150000in}}{\pgfqpoint{5.490039in}{5.490039in}}%
\pgfusepath{clip}%
\pgfsetbuttcap%
\pgfsetroundjoin%
\definecolor{currentfill}{rgb}{0.283091,0.110553,0.431554}%
\pgfsetfillcolor{currentfill}%
\pgfsetfillopacity{0.700000}%
\pgfsetlinewidth{0.000000pt}%
\definecolor{currentstroke}{rgb}{0.000000,0.000000,0.000000}%
\pgfsetstrokecolor{currentstroke}%
\pgfsetdash{}{0pt}%
\pgfpathmoveto{\pgfqpoint{3.498866in}{1.490141in}}%
\pgfpathlineto{\pgfqpoint{3.512498in}{1.479720in}}%
\pgfpathlineto{\pgfqpoint{3.526131in}{1.469427in}}%
\pgfpathlineto{\pgfqpoint{3.539766in}{1.459259in}}%
\pgfpathlineto{\pgfqpoint{3.553402in}{1.449218in}}%
\pgfpathlineto{\pgfqpoint{3.561588in}{1.449678in}}%
\pgfpathlineto{\pgfqpoint{3.569762in}{1.450379in}}%
\pgfpathlineto{\pgfqpoint{3.577926in}{1.451316in}}%
\pgfpathlineto{\pgfqpoint{3.586078in}{1.452485in}}%
\pgfpathlineto{\pgfqpoint{3.572471in}{1.462030in}}%
\pgfpathlineto{\pgfqpoint{3.558867in}{1.471700in}}%
\pgfpathlineto{\pgfqpoint{3.545264in}{1.481496in}}%
\pgfpathlineto{\pgfqpoint{3.531664in}{1.491418in}}%
\pgfpathlineto{\pgfqpoint{3.523482in}{1.490740in}}%
\pgfpathlineto{\pgfqpoint{3.515288in}{1.490298in}}%
\pgfpathlineto{\pgfqpoint{3.507083in}{1.490097in}}%
\pgfpathlineto{\pgfqpoint{3.498866in}{1.490141in}}%
\pgfpathclose%
\pgfusepath{fill}%
\end{pgfscope}%
\begin{pgfscope}%
\pgfpathrectangle{\pgfqpoint{1.254980in}{0.150000in}}{\pgfqpoint{5.490039in}{5.490039in}}%
\pgfusepath{clip}%
\pgfsetbuttcap%
\pgfsetroundjoin%
\definecolor{currentfill}{rgb}{0.267004,0.004874,0.329415}%
\pgfsetfillcolor{currentfill}%
\pgfsetfillopacity{0.700000}%
\pgfsetlinewidth{0.000000pt}%
\definecolor{currentstroke}{rgb}{0.000000,0.000000,0.000000}%
\pgfsetstrokecolor{currentstroke}%
\pgfsetdash{}{0pt}%
\pgfpathmoveto{\pgfqpoint{4.172988in}{1.269474in}}%
\pgfpathlineto{\pgfqpoint{4.186700in}{1.265697in}}%
\pgfpathlineto{\pgfqpoint{4.200419in}{1.262034in}}%
\pgfpathlineto{\pgfqpoint{4.214145in}{1.258484in}}%
\pgfpathlineto{\pgfqpoint{4.227879in}{1.255047in}}%
\pgfpathlineto{\pgfqpoint{4.235725in}{1.263161in}}%
\pgfpathlineto{\pgfqpoint{4.243565in}{1.271411in}}%
\pgfpathlineto{\pgfqpoint{4.251400in}{1.279791in}}%
\pgfpathlineto{\pgfqpoint{4.259230in}{1.288300in}}%
\pgfpathlineto{\pgfqpoint{4.245508in}{1.291313in}}%
\pgfpathlineto{\pgfqpoint{4.231794in}{1.294440in}}%
\pgfpathlineto{\pgfqpoint{4.218088in}{1.297680in}}%
\pgfpathlineto{\pgfqpoint{4.204388in}{1.301034in}}%
\pgfpathlineto{\pgfqpoint{4.196547in}{1.292943in}}%
\pgfpathlineto{\pgfqpoint{4.188700in}{1.284983in}}%
\pgfpathlineto{\pgfqpoint{4.180847in}{1.277159in}}%
\pgfpathlineto{\pgfqpoint{4.172988in}{1.269474in}}%
\pgfpathclose%
\pgfusepath{fill}%
\end{pgfscope}%
\begin{pgfscope}%
\pgfpathrectangle{\pgfqpoint{1.254980in}{0.150000in}}{\pgfqpoint{5.490039in}{5.490039in}}%
\pgfusepath{clip}%
\pgfsetbuttcap%
\pgfsetroundjoin%
\definecolor{currentfill}{rgb}{0.235526,0.309527,0.542944}%
\pgfsetfillcolor{currentfill}%
\pgfsetfillopacity{0.700000}%
\pgfsetlinewidth{0.000000pt}%
\definecolor{currentstroke}{rgb}{0.000000,0.000000,0.000000}%
\pgfsetstrokecolor{currentstroke}%
\pgfsetdash{}{0pt}%
\pgfpathmoveto{\pgfqpoint{5.153759in}{1.859182in}}%
\pgfpathlineto{\pgfqpoint{5.167893in}{1.864364in}}%
\pgfpathlineto{\pgfqpoint{5.182040in}{1.869657in}}%
\pgfpathlineto{\pgfqpoint{5.196202in}{1.875062in}}%
\pgfpathlineto{\pgfqpoint{5.210378in}{1.880577in}}%
\pgfpathlineto{\pgfqpoint{5.217982in}{1.894517in}}%
\pgfpathlineto{\pgfqpoint{5.225581in}{1.908422in}}%
\pgfpathlineto{\pgfqpoint{5.233175in}{1.922290in}}%
\pgfpathlineto{\pgfqpoint{5.240765in}{1.936118in}}%
\pgfpathlineto{\pgfqpoint{5.226586in}{1.930364in}}%
\pgfpathlineto{\pgfqpoint{5.212422in}{1.924721in}}%
\pgfpathlineto{\pgfqpoint{5.198272in}{1.919190in}}%
\pgfpathlineto{\pgfqpoint{5.184137in}{1.913770in}}%
\pgfpathlineto{\pgfqpoint{5.176549in}{1.900173in}}%
\pgfpathlineto{\pgfqpoint{5.168957in}{1.886542in}}%
\pgfpathlineto{\pgfqpoint{5.161361in}{1.872878in}}%
\pgfpathlineto{\pgfqpoint{5.153759in}{1.859182in}}%
\pgfpathclose%
\pgfusepath{fill}%
\end{pgfscope}%
\begin{pgfscope}%
\pgfpathrectangle{\pgfqpoint{1.254980in}{0.150000in}}{\pgfqpoint{5.490039in}{5.490039in}}%
\pgfusepath{clip}%
\pgfsetbuttcap%
\pgfsetroundjoin%
\definecolor{currentfill}{rgb}{0.216210,0.351535,0.550627}%
\pgfsetfillcolor{currentfill}%
\pgfsetfillopacity{0.700000}%
\pgfsetlinewidth{0.000000pt}%
\definecolor{currentstroke}{rgb}{0.000000,0.000000,0.000000}%
\pgfsetstrokecolor{currentstroke}%
\pgfsetdash{}{0pt}%
\pgfpathmoveto{\pgfqpoint{2.918923in}{2.040668in}}%
\pgfpathlineto{\pgfqpoint{2.932637in}{2.024193in}}%
\pgfpathlineto{\pgfqpoint{2.946348in}{2.007870in}}%
\pgfpathlineto{\pgfqpoint{2.960055in}{1.991697in}}%
\pgfpathlineto{\pgfqpoint{2.973759in}{1.975673in}}%
\pgfpathlineto{\pgfqpoint{2.982389in}{1.969802in}}%
\pgfpathlineto{\pgfqpoint{2.991001in}{1.964239in}}%
\pgfpathlineto{\pgfqpoint{2.999596in}{1.958979in}}%
\pgfpathlineto{\pgfqpoint{3.008173in}{1.954017in}}%
\pgfpathlineto{\pgfqpoint{2.994516in}{1.969502in}}%
\pgfpathlineto{\pgfqpoint{2.980855in}{1.985137in}}%
\pgfpathlineto{\pgfqpoint{2.967191in}{2.000921in}}%
\pgfpathlineto{\pgfqpoint{2.953524in}{2.016855in}}%
\pgfpathlineto{\pgfqpoint{2.944901in}{2.022348in}}%
\pgfpathlineto{\pgfqpoint{2.936260in}{2.028144in}}%
\pgfpathlineto{\pgfqpoint{2.927600in}{2.034249in}}%
\pgfpathlineto{\pgfqpoint{2.918923in}{2.040668in}}%
\pgfpathclose%
\pgfusepath{fill}%
\end{pgfscope}%
\begin{pgfscope}%
\pgfpathrectangle{\pgfqpoint{1.254980in}{0.150000in}}{\pgfqpoint{5.490039in}{5.490039in}}%
\pgfusepath{clip}%
\pgfsetbuttcap%
\pgfsetroundjoin%
\definecolor{currentfill}{rgb}{0.267968,0.223549,0.512008}%
\pgfsetfillcolor{currentfill}%
\pgfsetfillopacity{0.700000}%
\pgfsetlinewidth{0.000000pt}%
\definecolor{currentstroke}{rgb}{0.000000,0.000000,0.000000}%
\pgfsetstrokecolor{currentstroke}%
\pgfsetdash{}{0pt}%
\pgfpathmoveto{\pgfqpoint{4.949668in}{1.662108in}}%
\pgfpathlineto{\pgfqpoint{4.963689in}{1.665593in}}%
\pgfpathlineto{\pgfqpoint{4.977723in}{1.669189in}}%
\pgfpathlineto{\pgfqpoint{4.991769in}{1.672896in}}%
\pgfpathlineto{\pgfqpoint{5.005828in}{1.676713in}}%
\pgfpathlineto{\pgfqpoint{5.013480in}{1.690341in}}%
\pgfpathlineto{\pgfqpoint{5.021127in}{1.703966in}}%
\pgfpathlineto{\pgfqpoint{5.028770in}{1.717585in}}%
\pgfpathlineto{\pgfqpoint{5.036409in}{1.731197in}}%
\pgfpathlineto{\pgfqpoint{5.022348in}{1.727094in}}%
\pgfpathlineto{\pgfqpoint{5.008301in}{1.723101in}}%
\pgfpathlineto{\pgfqpoint{4.994266in}{1.719220in}}%
\pgfpathlineto{\pgfqpoint{4.980244in}{1.715450in}}%
\pgfpathlineto{\pgfqpoint{4.972606in}{1.702117in}}%
\pgfpathlineto{\pgfqpoint{4.964965in}{1.688782in}}%
\pgfpathlineto{\pgfqpoint{4.957318in}{1.675444in}}%
\pgfpathlineto{\pgfqpoint{4.949668in}{1.662108in}}%
\pgfpathclose%
\pgfusepath{fill}%
\end{pgfscope}%
\begin{pgfscope}%
\pgfpathrectangle{\pgfqpoint{1.254980in}{0.150000in}}{\pgfqpoint{5.490039in}{5.490039in}}%
\pgfusepath{clip}%
\pgfsetbuttcap%
\pgfsetroundjoin%
\definecolor{currentfill}{rgb}{0.227802,0.326594,0.546532}%
\pgfsetfillcolor{currentfill}%
\pgfsetfillopacity{0.700000}%
\pgfsetlinewidth{0.000000pt}%
\definecolor{currentstroke}{rgb}{0.000000,0.000000,0.000000}%
\pgfsetstrokecolor{currentstroke}%
\pgfsetdash{}{0pt}%
\pgfpathmoveto{\pgfqpoint{2.973759in}{1.975673in}}%
\pgfpathlineto{\pgfqpoint{2.987460in}{1.959799in}}%
\pgfpathlineto{\pgfqpoint{3.001157in}{1.944073in}}%
\pgfpathlineto{\pgfqpoint{3.014851in}{1.928494in}}%
\pgfpathlineto{\pgfqpoint{3.028542in}{1.913061in}}%
\pgfpathlineto{\pgfqpoint{3.037126in}{1.907734in}}%
\pgfpathlineto{\pgfqpoint{3.045693in}{1.902709in}}%
\pgfpathlineto{\pgfqpoint{3.054242in}{1.897982in}}%
\pgfpathlineto{\pgfqpoint{3.062775in}{1.893549in}}%
\pgfpathlineto{\pgfqpoint{3.049128in}{1.908446in}}%
\pgfpathlineto{\pgfqpoint{3.035479in}{1.923489in}}%
\pgfpathlineto{\pgfqpoint{3.021828in}{1.938679in}}%
\pgfpathlineto{\pgfqpoint{3.008173in}{1.954017in}}%
\pgfpathlineto{\pgfqpoint{2.999596in}{1.958979in}}%
\pgfpathlineto{\pgfqpoint{2.991001in}{1.964239in}}%
\pgfpathlineto{\pgfqpoint{2.982389in}{1.969802in}}%
\pgfpathlineto{\pgfqpoint{2.973759in}{1.975673in}}%
\pgfpathclose%
\pgfusepath{fill}%
\end{pgfscope}%
\begin{pgfscope}%
\pgfpathrectangle{\pgfqpoint{1.254980in}{0.150000in}}{\pgfqpoint{5.490039in}{5.490039in}}%
\pgfusepath{clip}%
\pgfsetbuttcap%
\pgfsetroundjoin%
\definecolor{currentfill}{rgb}{0.203063,0.379716,0.553925}%
\pgfsetfillcolor{currentfill}%
\pgfsetfillopacity{0.700000}%
\pgfsetlinewidth{0.000000pt}%
\definecolor{currentstroke}{rgb}{0.000000,0.000000,0.000000}%
\pgfsetstrokecolor{currentstroke}%
\pgfsetdash{}{0pt}%
\pgfpathmoveto{\pgfqpoint{2.864024in}{2.108092in}}%
\pgfpathlineto{\pgfqpoint{2.877755in}{2.091005in}}%
\pgfpathlineto{\pgfqpoint{2.891482in}{2.074073in}}%
\pgfpathlineto{\pgfqpoint{2.905204in}{2.057294in}}%
\pgfpathlineto{\pgfqpoint{2.918923in}{2.040668in}}%
\pgfpathlineto{\pgfqpoint{2.927600in}{2.034249in}}%
\pgfpathlineto{\pgfqpoint{2.936260in}{2.028144in}}%
\pgfpathlineto{\pgfqpoint{2.944901in}{2.022348in}}%
\pgfpathlineto{\pgfqpoint{2.953524in}{2.016855in}}%
\pgfpathlineto{\pgfqpoint{2.939854in}{2.032940in}}%
\pgfpathlineto{\pgfqpoint{2.926180in}{2.049178in}}%
\pgfpathlineto{\pgfqpoint{2.912502in}{2.065567in}}%
\pgfpathlineto{\pgfqpoint{2.898820in}{2.082111in}}%
\pgfpathlineto{\pgfqpoint{2.890149in}{2.088138in}}%
\pgfpathlineto{\pgfqpoint{2.881460in}{2.094474in}}%
\pgfpathlineto{\pgfqpoint{2.872752in}{2.101124in}}%
\pgfpathlineto{\pgfqpoint{2.864024in}{2.108092in}}%
\pgfpathclose%
\pgfusepath{fill}%
\end{pgfscope}%
\begin{pgfscope}%
\pgfpathrectangle{\pgfqpoint{1.254980in}{0.150000in}}{\pgfqpoint{5.490039in}{5.490039in}}%
\pgfusepath{clip}%
\pgfsetbuttcap%
\pgfsetroundjoin%
\definecolor{currentfill}{rgb}{0.239346,0.300855,0.540844}%
\pgfsetfillcolor{currentfill}%
\pgfsetfillopacity{0.700000}%
\pgfsetlinewidth{0.000000pt}%
\definecolor{currentstroke}{rgb}{0.000000,0.000000,0.000000}%
\pgfsetstrokecolor{currentstroke}%
\pgfsetdash{}{0pt}%
\pgfpathmoveto{\pgfqpoint{3.028542in}{1.913061in}}%
\pgfpathlineto{\pgfqpoint{3.042231in}{1.897775in}}%
\pgfpathlineto{\pgfqpoint{3.055917in}{1.882634in}}%
\pgfpathlineto{\pgfqpoint{3.069600in}{1.867638in}}%
\pgfpathlineto{\pgfqpoint{3.083281in}{1.852785in}}%
\pgfpathlineto{\pgfqpoint{3.091820in}{1.847998in}}%
\pgfpathlineto{\pgfqpoint{3.100342in}{1.843509in}}%
\pgfpathlineto{\pgfqpoint{3.108848in}{1.839313in}}%
\pgfpathlineto{\pgfqpoint{3.117338in}{1.835405in}}%
\pgfpathlineto{\pgfqpoint{3.103700in}{1.849725in}}%
\pgfpathlineto{\pgfqpoint{3.090061in}{1.864189in}}%
\pgfpathlineto{\pgfqpoint{3.076419in}{1.878797in}}%
\pgfpathlineto{\pgfqpoint{3.062775in}{1.893549in}}%
\pgfpathlineto{\pgfqpoint{3.054242in}{1.897982in}}%
\pgfpathlineto{\pgfqpoint{3.045693in}{1.902709in}}%
\pgfpathlineto{\pgfqpoint{3.037126in}{1.907734in}}%
\pgfpathlineto{\pgfqpoint{3.028542in}{1.913061in}}%
\pgfpathclose%
\pgfusepath{fill}%
\end{pgfscope}%
\begin{pgfscope}%
\pgfpathrectangle{\pgfqpoint{1.254980in}{0.150000in}}{\pgfqpoint{5.490039in}{5.490039in}}%
\pgfusepath{clip}%
\pgfsetbuttcap%
\pgfsetroundjoin%
\definecolor{currentfill}{rgb}{0.190631,0.407061,0.556089}%
\pgfsetfillcolor{currentfill}%
\pgfsetfillopacity{0.700000}%
\pgfsetlinewidth{0.000000pt}%
\definecolor{currentstroke}{rgb}{0.000000,0.000000,0.000000}%
\pgfsetstrokecolor{currentstroke}%
\pgfsetdash{}{0pt}%
\pgfpathmoveto{\pgfqpoint{2.809055in}{2.177998in}}%
\pgfpathlineto{\pgfqpoint{2.822804in}{2.160286in}}%
\pgfpathlineto{\pgfqpoint{2.836549in}{2.142732in}}%
\pgfpathlineto{\pgfqpoint{2.850289in}{2.125334in}}%
\pgfpathlineto{\pgfqpoint{2.864024in}{2.108092in}}%
\pgfpathlineto{\pgfqpoint{2.872752in}{2.101124in}}%
\pgfpathlineto{\pgfqpoint{2.881460in}{2.094474in}}%
\pgfpathlineto{\pgfqpoint{2.890149in}{2.088138in}}%
\pgfpathlineto{\pgfqpoint{2.898820in}{2.082111in}}%
\pgfpathlineto{\pgfqpoint{2.885134in}{2.098808in}}%
\pgfpathlineto{\pgfqpoint{2.871444in}{2.115661in}}%
\pgfpathlineto{\pgfqpoint{2.857750in}{2.132669in}}%
\pgfpathlineto{\pgfqpoint{2.844051in}{2.149834in}}%
\pgfpathlineto{\pgfqpoint{2.835331in}{2.156399in}}%
\pgfpathlineto{\pgfqpoint{2.826592in}{2.163278in}}%
\pgfpathlineto{\pgfqpoint{2.817833in}{2.170476in}}%
\pgfpathlineto{\pgfqpoint{2.809055in}{2.177998in}}%
\pgfpathclose%
\pgfusepath{fill}%
\end{pgfscope}%
\begin{pgfscope}%
\pgfpathrectangle{\pgfqpoint{1.254980in}{0.150000in}}{\pgfqpoint{5.490039in}{5.490039in}}%
\pgfusepath{clip}%
\pgfsetbuttcap%
\pgfsetroundjoin%
\definecolor{currentfill}{rgb}{0.280894,0.078907,0.402329}%
\pgfsetfillcolor{currentfill}%
\pgfsetfillopacity{0.700000}%
\pgfsetlinewidth{0.000000pt}%
\definecolor{currentstroke}{rgb}{0.000000,0.000000,0.000000}%
\pgfsetstrokecolor{currentstroke}%
\pgfsetdash{}{0pt}%
\pgfpathmoveto{\pgfqpoint{4.573040in}{1.384982in}}%
\pgfpathlineto{\pgfqpoint{4.586887in}{1.385043in}}%
\pgfpathlineto{\pgfqpoint{4.600743in}{1.385215in}}%
\pgfpathlineto{\pgfqpoint{4.614610in}{1.385497in}}%
\pgfpathlineto{\pgfqpoint{4.628487in}{1.385889in}}%
\pgfpathlineto{\pgfqpoint{4.636220in}{1.397593in}}%
\pgfpathlineto{\pgfqpoint{4.643949in}{1.409361in}}%
\pgfpathlineto{\pgfqpoint{4.651674in}{1.421189in}}%
\pgfpathlineto{\pgfqpoint{4.659395in}{1.433075in}}%
\pgfpathlineto{\pgfqpoint{4.645521in}{1.432320in}}%
\pgfpathlineto{\pgfqpoint{4.631658in}{1.431677in}}%
\pgfpathlineto{\pgfqpoint{4.617806in}{1.431143in}}%
\pgfpathlineto{\pgfqpoint{4.603963in}{1.430721in}}%
\pgfpathlineto{\pgfqpoint{4.596239in}{1.419191in}}%
\pgfpathlineto{\pgfqpoint{4.588510in}{1.407722in}}%
\pgfpathlineto{\pgfqpoint{4.580777in}{1.396319in}}%
\pgfpathlineto{\pgfqpoint{4.573040in}{1.384982in}}%
\pgfpathclose%
\pgfusepath{fill}%
\end{pgfscope}%
\begin{pgfscope}%
\pgfpathrectangle{\pgfqpoint{1.254980in}{0.150000in}}{\pgfqpoint{5.490039in}{5.490039in}}%
\pgfusepath{clip}%
\pgfsetbuttcap%
\pgfsetroundjoin%
\definecolor{currentfill}{rgb}{0.277941,0.056324,0.381191}%
\pgfsetfillcolor{currentfill}%
\pgfsetfillopacity{0.700000}%
\pgfsetlinewidth{0.000000pt}%
\definecolor{currentstroke}{rgb}{0.000000,0.000000,0.000000}%
\pgfsetstrokecolor{currentstroke}%
\pgfsetdash{}{0pt}%
\pgfpathmoveto{\pgfqpoint{4.486741in}{1.342743in}}%
\pgfpathlineto{\pgfqpoint{4.500554in}{1.341983in}}%
\pgfpathlineto{\pgfqpoint{4.514376in}{1.341335in}}%
\pgfpathlineto{\pgfqpoint{4.528208in}{1.340797in}}%
\pgfpathlineto{\pgfqpoint{4.542050in}{1.340370in}}%
\pgfpathlineto{\pgfqpoint{4.549804in}{1.351407in}}%
\pgfpathlineto{\pgfqpoint{4.557554in}{1.362523in}}%
\pgfpathlineto{\pgfqpoint{4.565299in}{1.373716in}}%
\pgfpathlineto{\pgfqpoint{4.573040in}{1.384982in}}%
\pgfpathlineto{\pgfqpoint{4.559204in}{1.385032in}}%
\pgfpathlineto{\pgfqpoint{4.545377in}{1.385193in}}%
\pgfpathlineto{\pgfqpoint{4.531560in}{1.385465in}}%
\pgfpathlineto{\pgfqpoint{4.517752in}{1.385847in}}%
\pgfpathlineto{\pgfqpoint{4.510006in}{1.374952in}}%
\pgfpathlineto{\pgfqpoint{4.502255in}{1.364134in}}%
\pgfpathlineto{\pgfqpoint{4.494500in}{1.353397in}}%
\pgfpathlineto{\pgfqpoint{4.486741in}{1.342743in}}%
\pgfpathclose%
\pgfusepath{fill}%
\end{pgfscope}%
\begin{pgfscope}%
\pgfpathrectangle{\pgfqpoint{1.254980in}{0.150000in}}{\pgfqpoint{5.490039in}{5.490039in}}%
\pgfusepath{clip}%
\pgfsetbuttcap%
\pgfsetroundjoin%
\definecolor{currentfill}{rgb}{0.274952,0.037752,0.364543}%
\pgfsetfillcolor{currentfill}%
\pgfsetfillopacity{0.700000}%
\pgfsetlinewidth{0.000000pt}%
\definecolor{currentstroke}{rgb}{0.000000,0.000000,0.000000}%
\pgfsetstrokecolor{currentstroke}%
\pgfsetdash{}{0pt}%
\pgfpathmoveto{\pgfqpoint{3.749558in}{1.347603in}}%
\pgfpathlineto{\pgfqpoint{3.763201in}{1.339655in}}%
\pgfpathlineto{\pgfqpoint{3.776848in}{1.331828in}}%
\pgfpathlineto{\pgfqpoint{3.790499in}{1.324120in}}%
\pgfpathlineto{\pgfqpoint{3.804153in}{1.316533in}}%
\pgfpathlineto{\pgfqpoint{3.812190in}{1.319862in}}%
\pgfpathlineto{\pgfqpoint{3.820219in}{1.323398in}}%
\pgfpathlineto{\pgfqpoint{3.828238in}{1.327135in}}%
\pgfpathlineto{\pgfqpoint{3.836250in}{1.331069in}}%
\pgfpathlineto{\pgfqpoint{3.822618in}{1.338182in}}%
\pgfpathlineto{\pgfqpoint{3.808991in}{1.345414in}}%
\pgfpathlineto{\pgfqpoint{3.795368in}{1.352766in}}%
\pgfpathlineto{\pgfqpoint{3.781748in}{1.360238in}}%
\pgfpathlineto{\pgfqpoint{3.773714in}{1.356772in}}%
\pgfpathlineto{\pgfqpoint{3.765671in}{1.353509in}}%
\pgfpathlineto{\pgfqpoint{3.757619in}{1.350451in}}%
\pgfpathlineto{\pgfqpoint{3.749558in}{1.347603in}}%
\pgfpathclose%
\pgfusepath{fill}%
\end{pgfscope}%
\begin{pgfscope}%
\pgfpathrectangle{\pgfqpoint{1.254980in}{0.150000in}}{\pgfqpoint{5.490039in}{5.490039in}}%
\pgfusepath{clip}%
\pgfsetbuttcap%
\pgfsetroundjoin%
\definecolor{currentfill}{rgb}{0.248629,0.278775,0.534556}%
\pgfsetfillcolor{currentfill}%
\pgfsetfillopacity{0.700000}%
\pgfsetlinewidth{0.000000pt}%
\definecolor{currentstroke}{rgb}{0.000000,0.000000,0.000000}%
\pgfsetstrokecolor{currentstroke}%
\pgfsetdash{}{0pt}%
\pgfpathmoveto{\pgfqpoint{3.083281in}{1.852785in}}%
\pgfpathlineto{\pgfqpoint{3.096960in}{1.838076in}}%
\pgfpathlineto{\pgfqpoint{3.110637in}{1.823509in}}%
\pgfpathlineto{\pgfqpoint{3.124311in}{1.809085in}}%
\pgfpathlineto{\pgfqpoint{3.137984in}{1.794801in}}%
\pgfpathlineto{\pgfqpoint{3.146479in}{1.790553in}}%
\pgfpathlineto{\pgfqpoint{3.154959in}{1.786597in}}%
\pgfpathlineto{\pgfqpoint{3.163422in}{1.782928in}}%
\pgfpathlineto{\pgfqpoint{3.171871in}{1.779543in}}%
\pgfpathlineto{\pgfqpoint{3.158240in}{1.793297in}}%
\pgfpathlineto{\pgfqpoint{3.144608in}{1.807191in}}%
\pgfpathlineto{\pgfqpoint{3.130974in}{1.821227in}}%
\pgfpathlineto{\pgfqpoint{3.117338in}{1.835405in}}%
\pgfpathlineto{\pgfqpoint{3.108848in}{1.839313in}}%
\pgfpathlineto{\pgfqpoint{3.100342in}{1.843509in}}%
\pgfpathlineto{\pgfqpoint{3.091820in}{1.847998in}}%
\pgfpathlineto{\pgfqpoint{3.083281in}{1.852785in}}%
\pgfpathclose%
\pgfusepath{fill}%
\end{pgfscope}%
\begin{pgfscope}%
\pgfpathrectangle{\pgfqpoint{1.254980in}{0.150000in}}{\pgfqpoint{5.490039in}{5.490039in}}%
\pgfusepath{clip}%
\pgfsetbuttcap%
\pgfsetroundjoin%
\definecolor{currentfill}{rgb}{0.132268,0.655014,0.519661}%
\pgfsetfillcolor{currentfill}%
\pgfsetfillopacity{0.700000}%
\pgfsetlinewidth{0.000000pt}%
\definecolor{currentstroke}{rgb}{0.000000,0.000000,0.000000}%
\pgfsetstrokecolor{currentstroke}%
\pgfsetdash{}{0pt}%
\pgfpathmoveto{\pgfqpoint{2.346635in}{2.880055in}}%
\pgfpathlineto{\pgfqpoint{2.360599in}{2.856598in}}%
\pgfpathlineto{\pgfqpoint{2.374552in}{2.833337in}}%
\pgfpathlineto{\pgfqpoint{2.388495in}{2.810268in}}%
\pgfpathlineto{\pgfqpoint{2.402428in}{2.787392in}}%
\pgfpathlineto{\pgfqpoint{2.411574in}{2.776738in}}%
\pgfpathlineto{\pgfqpoint{2.420695in}{2.766433in}}%
\pgfpathlineto{\pgfqpoint{2.429793in}{2.756473in}}%
\pgfpathlineto{\pgfqpoint{2.438867in}{2.746852in}}%
\pgfpathlineto{\pgfqpoint{2.424995in}{2.769168in}}%
\pgfpathlineto{\pgfqpoint{2.411114in}{2.791675in}}%
\pgfpathlineto{\pgfqpoint{2.397223in}{2.814375in}}%
\pgfpathlineto{\pgfqpoint{2.383322in}{2.837267in}}%
\pgfpathlineto{\pgfqpoint{2.374187in}{2.847441in}}%
\pgfpathlineto{\pgfqpoint{2.365028in}{2.857960in}}%
\pgfpathlineto{\pgfqpoint{2.355844in}{2.868830in}}%
\pgfpathlineto{\pgfqpoint{2.346635in}{2.880055in}}%
\pgfpathclose%
\pgfusepath{fill}%
\end{pgfscope}%
\begin{pgfscope}%
\pgfpathrectangle{\pgfqpoint{1.254980in}{0.150000in}}{\pgfqpoint{5.490039in}{5.490039in}}%
\pgfusepath{clip}%
\pgfsetbuttcap%
\pgfsetroundjoin%
\definecolor{currentfill}{rgb}{0.283091,0.110553,0.431554}%
\pgfsetfillcolor{currentfill}%
\pgfsetfillopacity{0.700000}%
\pgfsetlinewidth{0.000000pt}%
\definecolor{currentstroke}{rgb}{0.000000,0.000000,0.000000}%
\pgfsetstrokecolor{currentstroke}%
\pgfsetdash{}{0pt}%
\pgfpathmoveto{\pgfqpoint{4.659395in}{1.433075in}}%
\pgfpathlineto{\pgfqpoint{4.673279in}{1.433940in}}%
\pgfpathlineto{\pgfqpoint{4.687173in}{1.434915in}}%
\pgfpathlineto{\pgfqpoint{4.701079in}{1.436001in}}%
\pgfpathlineto{\pgfqpoint{4.714995in}{1.437197in}}%
\pgfpathlineto{\pgfqpoint{4.722709in}{1.449490in}}%
\pgfpathlineto{\pgfqpoint{4.730420in}{1.461832in}}%
\pgfpathlineto{\pgfqpoint{4.738127in}{1.474219in}}%
\pgfpathlineto{\pgfqpoint{4.745829in}{1.486649in}}%
\pgfpathlineto{\pgfqpoint{4.731915in}{1.485106in}}%
\pgfpathlineto{\pgfqpoint{4.718012in}{1.483674in}}%
\pgfpathlineto{\pgfqpoint{4.704119in}{1.482352in}}%
\pgfpathlineto{\pgfqpoint{4.690238in}{1.481140in}}%
\pgfpathlineto{\pgfqpoint{4.682533in}{1.469051in}}%
\pgfpathlineto{\pgfqpoint{4.674825in}{1.457009in}}%
\pgfpathlineto{\pgfqpoint{4.667112in}{1.445016in}}%
\pgfpathlineto{\pgfqpoint{4.659395in}{1.433075in}}%
\pgfpathclose%
\pgfusepath{fill}%
\end{pgfscope}%
\begin{pgfscope}%
\pgfpathrectangle{\pgfqpoint{1.254980in}{0.150000in}}{\pgfqpoint{5.490039in}{5.490039in}}%
\pgfusepath{clip}%
\pgfsetbuttcap%
\pgfsetroundjoin%
\definecolor{currentfill}{rgb}{0.179019,0.433756,0.557430}%
\pgfsetfillcolor{currentfill}%
\pgfsetfillopacity{0.700000}%
\pgfsetlinewidth{0.000000pt}%
\definecolor{currentstroke}{rgb}{0.000000,0.000000,0.000000}%
\pgfsetstrokecolor{currentstroke}%
\pgfsetdash{}{0pt}%
\pgfpathmoveto{\pgfqpoint{2.754006in}{2.250438in}}%
\pgfpathlineto{\pgfqpoint{2.767776in}{2.232088in}}%
\pgfpathlineto{\pgfqpoint{2.781541in}{2.213898in}}%
\pgfpathlineto{\pgfqpoint{2.795301in}{2.195868in}}%
\pgfpathlineto{\pgfqpoint{2.809055in}{2.177998in}}%
\pgfpathlineto{\pgfqpoint{2.817833in}{2.170476in}}%
\pgfpathlineto{\pgfqpoint{2.826592in}{2.163278in}}%
\pgfpathlineto{\pgfqpoint{2.835331in}{2.156399in}}%
\pgfpathlineto{\pgfqpoint{2.844051in}{2.149834in}}%
\pgfpathlineto{\pgfqpoint{2.830348in}{2.167157in}}%
\pgfpathlineto{\pgfqpoint{2.816639in}{2.184637in}}%
\pgfpathlineto{\pgfqpoint{2.802926in}{2.202277in}}%
\pgfpathlineto{\pgfqpoint{2.789208in}{2.220078in}}%
\pgfpathlineto{\pgfqpoint{2.780438in}{2.227183in}}%
\pgfpathlineto{\pgfqpoint{2.771648in}{2.234609in}}%
\pgfpathlineto{\pgfqpoint{2.762837in}{2.242359in}}%
\pgfpathlineto{\pgfqpoint{2.754006in}{2.250438in}}%
\pgfpathclose%
\pgfusepath{fill}%
\end{pgfscope}%
\begin{pgfscope}%
\pgfpathrectangle{\pgfqpoint{1.254980in}{0.150000in}}{\pgfqpoint{5.490039in}{5.490039in}}%
\pgfusepath{clip}%
\pgfsetbuttcap%
\pgfsetroundjoin%
\definecolor{currentfill}{rgb}{0.273809,0.031497,0.358853}%
\pgfsetfillcolor{currentfill}%
\pgfsetfillopacity{0.700000}%
\pgfsetlinewidth{0.000000pt}%
\definecolor{currentstroke}{rgb}{0.000000,0.000000,0.000000}%
\pgfsetstrokecolor{currentstroke}%
\pgfsetdash{}{0pt}%
\pgfpathmoveto{\pgfqpoint{4.400468in}{1.306740in}}%
\pgfpathlineto{\pgfqpoint{4.414252in}{1.305143in}}%
\pgfpathlineto{\pgfqpoint{4.428045in}{1.303658in}}%
\pgfpathlineto{\pgfqpoint{4.441847in}{1.302284in}}%
\pgfpathlineto{\pgfqpoint{4.455658in}{1.301021in}}%
\pgfpathlineto{\pgfqpoint{4.463435in}{1.311311in}}%
\pgfpathlineto{\pgfqpoint{4.471208in}{1.321696in}}%
\pgfpathlineto{\pgfqpoint{4.478977in}{1.332175in}}%
\pgfpathlineto{\pgfqpoint{4.486741in}{1.342743in}}%
\pgfpathlineto{\pgfqpoint{4.472937in}{1.343613in}}%
\pgfpathlineto{\pgfqpoint{4.459142in}{1.344595in}}%
\pgfpathlineto{\pgfqpoint{4.445356in}{1.345688in}}%
\pgfpathlineto{\pgfqpoint{4.431579in}{1.346893in}}%
\pgfpathlineto{\pgfqpoint{4.423808in}{1.336711in}}%
\pgfpathlineto{\pgfqpoint{4.416033in}{1.326623in}}%
\pgfpathlineto{\pgfqpoint{4.408253in}{1.316632in}}%
\pgfpathlineto{\pgfqpoint{4.400468in}{1.306740in}}%
\pgfpathclose%
\pgfusepath{fill}%
\end{pgfscope}%
\begin{pgfscope}%
\pgfpathrectangle{\pgfqpoint{1.254980in}{0.150000in}}{\pgfqpoint{5.490039in}{5.490039in}}%
\pgfusepath{clip}%
\pgfsetbuttcap%
\pgfsetroundjoin%
\definecolor{currentfill}{rgb}{0.258965,0.251537,0.524736}%
\pgfsetfillcolor{currentfill}%
\pgfsetfillopacity{0.700000}%
\pgfsetlinewidth{0.000000pt}%
\definecolor{currentstroke}{rgb}{0.000000,0.000000,0.000000}%
\pgfsetstrokecolor{currentstroke}%
\pgfsetdash{}{0pt}%
\pgfpathmoveto{\pgfqpoint{3.137984in}{1.794801in}}%
\pgfpathlineto{\pgfqpoint{3.151655in}{1.780659in}}%
\pgfpathlineto{\pgfqpoint{3.165324in}{1.766656in}}%
\pgfpathlineto{\pgfqpoint{3.178992in}{1.752793in}}%
\pgfpathlineto{\pgfqpoint{3.192658in}{1.739069in}}%
\pgfpathlineto{\pgfqpoint{3.201111in}{1.735355in}}%
\pgfpathlineto{\pgfqpoint{3.209550in}{1.731930in}}%
\pgfpathlineto{\pgfqpoint{3.217973in}{1.728787in}}%
\pgfpathlineto{\pgfqpoint{3.226381in}{1.725922in}}%
\pgfpathlineto{\pgfqpoint{3.212755in}{1.739119in}}%
\pgfpathlineto{\pgfqpoint{3.199128in}{1.752455in}}%
\pgfpathlineto{\pgfqpoint{3.185500in}{1.765929in}}%
\pgfpathlineto{\pgfqpoint{3.171871in}{1.779543in}}%
\pgfpathlineto{\pgfqpoint{3.163422in}{1.782928in}}%
\pgfpathlineto{\pgfqpoint{3.154959in}{1.786597in}}%
\pgfpathlineto{\pgfqpoint{3.146479in}{1.790553in}}%
\pgfpathlineto{\pgfqpoint{3.137984in}{1.794801in}}%
\pgfpathclose%
\pgfusepath{fill}%
\end{pgfscope}%
\begin{pgfscope}%
\pgfpathrectangle{\pgfqpoint{1.254980in}{0.150000in}}{\pgfqpoint{5.490039in}{5.490039in}}%
\pgfusepath{clip}%
\pgfsetbuttcap%
\pgfsetroundjoin%
\definecolor{currentfill}{rgb}{0.218130,0.347432,0.550038}%
\pgfsetfillcolor{currentfill}%
\pgfsetfillopacity{0.700000}%
\pgfsetlinewidth{0.000000pt}%
\definecolor{currentstroke}{rgb}{0.000000,0.000000,0.000000}%
\pgfsetstrokecolor{currentstroke}%
\pgfsetdash{}{0pt}%
\pgfpathmoveto{\pgfqpoint{5.240765in}{1.936118in}}%
\pgfpathlineto{\pgfqpoint{5.254958in}{1.941984in}}%
\pgfpathlineto{\pgfqpoint{5.269167in}{1.947961in}}%
\pgfpathlineto{\pgfqpoint{5.283390in}{1.954050in}}%
\pgfpathlineto{\pgfqpoint{5.290976in}{1.968010in}}%
\pgfpathlineto{\pgfqpoint{5.298558in}{1.981925in}}%
\pgfpathlineto{\pgfqpoint{5.306135in}{1.995792in}}%
\pgfpathlineto{\pgfqpoint{5.313707in}{2.009611in}}%
\pgfpathlineto{\pgfqpoint{5.299481in}{2.003299in}}%
\pgfpathlineto{\pgfqpoint{5.285270in}{1.997099in}}%
\pgfpathlineto{\pgfqpoint{5.271074in}{1.991011in}}%
\pgfpathlineto{\pgfqpoint{5.263504in}{1.977354in}}%
\pgfpathlineto{\pgfqpoint{5.255930in}{1.963652in}}%
\pgfpathlineto{\pgfqpoint{5.248350in}{1.949906in}}%
\pgfpathlineto{\pgfqpoint{5.240765in}{1.936118in}}%
\pgfpathclose%
\pgfusepath{fill}%
\end{pgfscope}%
\begin{pgfscope}%
\pgfpathrectangle{\pgfqpoint{1.254980in}{0.150000in}}{\pgfqpoint{5.490039in}{5.490039in}}%
\pgfusepath{clip}%
\pgfsetbuttcap%
\pgfsetroundjoin%
\definecolor{currentfill}{rgb}{0.282327,0.094955,0.417331}%
\pgfsetfillcolor{currentfill}%
\pgfsetfillopacity{0.700000}%
\pgfsetlinewidth{0.000000pt}%
\definecolor{currentstroke}{rgb}{0.000000,0.000000,0.000000}%
\pgfsetstrokecolor{currentstroke}%
\pgfsetdash{}{0pt}%
\pgfpathmoveto{\pgfqpoint{3.553402in}{1.449218in}}%
\pgfpathlineto{\pgfqpoint{3.567041in}{1.439302in}}%
\pgfpathlineto{\pgfqpoint{3.580681in}{1.429511in}}%
\pgfpathlineto{\pgfqpoint{3.594324in}{1.419845in}}%
\pgfpathlineto{\pgfqpoint{3.607968in}{1.410303in}}%
\pgfpathlineto{\pgfqpoint{3.616124in}{1.411267in}}%
\pgfpathlineto{\pgfqpoint{3.624269in}{1.412466in}}%
\pgfpathlineto{\pgfqpoint{3.632403in}{1.413897in}}%
\pgfpathlineto{\pgfqpoint{3.640527in}{1.415555in}}%
\pgfpathlineto{\pgfqpoint{3.626911in}{1.424601in}}%
\pgfpathlineto{\pgfqpoint{3.613298in}{1.433772in}}%
\pgfpathlineto{\pgfqpoint{3.599687in}{1.443066in}}%
\pgfpathlineto{\pgfqpoint{3.586078in}{1.452485in}}%
\pgfpathlineto{\pgfqpoint{3.577926in}{1.451316in}}%
\pgfpathlineto{\pgfqpoint{3.569762in}{1.450379in}}%
\pgfpathlineto{\pgfqpoint{3.561588in}{1.449678in}}%
\pgfpathlineto{\pgfqpoint{3.553402in}{1.449218in}}%
\pgfpathclose%
\pgfusepath{fill}%
\end{pgfscope}%
\begin{pgfscope}%
\pgfpathrectangle{\pgfqpoint{1.254980in}{0.150000in}}{\pgfqpoint{5.490039in}{5.490039in}}%
\pgfusepath{clip}%
\pgfsetbuttcap%
\pgfsetroundjoin%
\definecolor{currentfill}{rgb}{0.282884,0.135920,0.453427}%
\pgfsetfillcolor{currentfill}%
\pgfsetfillopacity{0.700000}%
\pgfsetlinewidth{0.000000pt}%
\definecolor{currentstroke}{rgb}{0.000000,0.000000,0.000000}%
\pgfsetstrokecolor{currentstroke}%
\pgfsetdash{}{0pt}%
\pgfpathmoveto{\pgfqpoint{4.745829in}{1.486649in}}%
\pgfpathlineto{\pgfqpoint{4.759755in}{1.488302in}}%
\pgfpathlineto{\pgfqpoint{4.773691in}{1.490066in}}%
\pgfpathlineto{\pgfqpoint{4.787640in}{1.491940in}}%
\pgfpathlineto{\pgfqpoint{4.801599in}{1.493924in}}%
\pgfpathlineto{\pgfqpoint{4.809297in}{1.506732in}}%
\pgfpathlineto{\pgfqpoint{4.816991in}{1.519573in}}%
\pgfpathlineto{\pgfqpoint{4.824681in}{1.532445in}}%
\pgfpathlineto{\pgfqpoint{4.832367in}{1.545345in}}%
\pgfpathlineto{\pgfqpoint{4.818408in}{1.543029in}}%
\pgfpathlineto{\pgfqpoint{4.804460in}{1.540823in}}%
\pgfpathlineto{\pgfqpoint{4.790524in}{1.538728in}}%
\pgfpathlineto{\pgfqpoint{4.776600in}{1.536744in}}%
\pgfpathlineto{\pgfqpoint{4.768913in}{1.524169in}}%
\pgfpathlineto{\pgfqpoint{4.761223in}{1.511627in}}%
\pgfpathlineto{\pgfqpoint{4.753528in}{1.499119in}}%
\pgfpathlineto{\pgfqpoint{4.745829in}{1.486649in}}%
\pgfpathclose%
\pgfusepath{fill}%
\end{pgfscope}%
\begin{pgfscope}%
\pgfpathrectangle{\pgfqpoint{1.254980in}{0.150000in}}{\pgfqpoint{5.490039in}{5.490039in}}%
\pgfusepath{clip}%
\pgfsetbuttcap%
\pgfsetroundjoin%
\definecolor{currentfill}{rgb}{0.168126,0.459988,0.558082}%
\pgfsetfillcolor{currentfill}%
\pgfsetfillopacity{0.700000}%
\pgfsetlinewidth{0.000000pt}%
\definecolor{currentstroke}{rgb}{0.000000,0.000000,0.000000}%
\pgfsetstrokecolor{currentstroke}%
\pgfsetdash{}{0pt}%
\pgfpathmoveto{\pgfqpoint{2.698868in}{2.325470in}}%
\pgfpathlineto{\pgfqpoint{2.712661in}{2.306466in}}%
\pgfpathlineto{\pgfqpoint{2.726449in}{2.287627in}}%
\pgfpathlineto{\pgfqpoint{2.740230in}{2.268951in}}%
\pgfpathlineto{\pgfqpoint{2.754006in}{2.250438in}}%
\pgfpathlineto{\pgfqpoint{2.762837in}{2.242359in}}%
\pgfpathlineto{\pgfqpoint{2.771648in}{2.234609in}}%
\pgfpathlineto{\pgfqpoint{2.780438in}{2.227183in}}%
\pgfpathlineto{\pgfqpoint{2.789208in}{2.220078in}}%
\pgfpathlineto{\pgfqpoint{2.775485in}{2.238039in}}%
\pgfpathlineto{\pgfqpoint{2.761757in}{2.256162in}}%
\pgfpathlineto{\pgfqpoint{2.748023in}{2.274448in}}%
\pgfpathlineto{\pgfqpoint{2.734283in}{2.292897in}}%
\pgfpathlineto{\pgfqpoint{2.725460in}{2.300548in}}%
\pgfpathlineto{\pgfqpoint{2.716617in}{2.308523in}}%
\pgfpathlineto{\pgfqpoint{2.707753in}{2.316829in}}%
\pgfpathlineto{\pgfqpoint{2.698868in}{2.325470in}}%
\pgfpathclose%
\pgfusepath{fill}%
\end{pgfscope}%
\begin{pgfscope}%
\pgfpathrectangle{\pgfqpoint{1.254980in}{0.150000in}}{\pgfqpoint{5.490039in}{5.490039in}}%
\pgfusepath{clip}%
\pgfsetbuttcap%
\pgfsetroundjoin%
\definecolor{currentfill}{rgb}{0.257322,0.256130,0.526563}%
\pgfsetfillcolor{currentfill}%
\pgfsetfillopacity{0.700000}%
\pgfsetlinewidth{0.000000pt}%
\definecolor{currentstroke}{rgb}{0.000000,0.000000,0.000000}%
\pgfsetstrokecolor{currentstroke}%
\pgfsetdash{}{0pt}%
\pgfpathmoveto{\pgfqpoint{5.036409in}{1.731197in}}%
\pgfpathlineto{\pgfqpoint{5.050484in}{1.735410in}}%
\pgfpathlineto{\pgfqpoint{5.064571in}{1.739735in}}%
\pgfpathlineto{\pgfqpoint{5.078672in}{1.744170in}}%
\pgfpathlineto{\pgfqpoint{5.092787in}{1.748716in}}%
\pgfpathlineto{\pgfqpoint{5.100424in}{1.762594in}}%
\pgfpathlineto{\pgfqpoint{5.108056in}{1.776456in}}%
\pgfpathlineto{\pgfqpoint{5.115685in}{1.790300in}}%
\pgfpathlineto{\pgfqpoint{5.123309in}{1.804124in}}%
\pgfpathlineto{\pgfqpoint{5.109192in}{1.799308in}}%
\pgfpathlineto{\pgfqpoint{5.095088in}{1.794602in}}%
\pgfpathlineto{\pgfqpoint{5.080999in}{1.790008in}}%
\pgfpathlineto{\pgfqpoint{5.066923in}{1.785524in}}%
\pgfpathlineto{\pgfqpoint{5.059301in}{1.771964in}}%
\pgfpathlineto{\pgfqpoint{5.051675in}{1.758388in}}%
\pgfpathlineto{\pgfqpoint{5.044044in}{1.744798in}}%
\pgfpathlineto{\pgfqpoint{5.036409in}{1.731197in}}%
\pgfpathclose%
\pgfusepath{fill}%
\end{pgfscope}%
\begin{pgfscope}%
\pgfpathrectangle{\pgfqpoint{1.254980in}{0.150000in}}{\pgfqpoint{5.490039in}{5.490039in}}%
\pgfusepath{clip}%
\pgfsetbuttcap%
\pgfsetroundjoin%
\definecolor{currentfill}{rgb}{0.266580,0.228262,0.514349}%
\pgfsetfillcolor{currentfill}%
\pgfsetfillopacity{0.700000}%
\pgfsetlinewidth{0.000000pt}%
\definecolor{currentstroke}{rgb}{0.000000,0.000000,0.000000}%
\pgfsetstrokecolor{currentstroke}%
\pgfsetdash{}{0pt}%
\pgfpathmoveto{\pgfqpoint{3.192658in}{1.739069in}}%
\pgfpathlineto{\pgfqpoint{3.206323in}{1.725482in}}%
\pgfpathlineto{\pgfqpoint{3.219987in}{1.712034in}}%
\pgfpathlineto{\pgfqpoint{3.233650in}{1.698722in}}%
\pgfpathlineto{\pgfqpoint{3.247312in}{1.685547in}}%
\pgfpathlineto{\pgfqpoint{3.255725in}{1.682367in}}%
\pgfpathlineto{\pgfqpoint{3.264123in}{1.679469in}}%
\pgfpathlineto{\pgfqpoint{3.272507in}{1.676850in}}%
\pgfpathlineto{\pgfqpoint{3.280877in}{1.674503in}}%
\pgfpathlineto{\pgfqpoint{3.267254in}{1.687153in}}%
\pgfpathlineto{\pgfqpoint{3.253630in}{1.699940in}}%
\pgfpathlineto{\pgfqpoint{3.240006in}{1.712862in}}%
\pgfpathlineto{\pgfqpoint{3.226381in}{1.725922in}}%
\pgfpathlineto{\pgfqpoint{3.217973in}{1.728787in}}%
\pgfpathlineto{\pgfqpoint{3.209550in}{1.731930in}}%
\pgfpathlineto{\pgfqpoint{3.201111in}{1.735355in}}%
\pgfpathlineto{\pgfqpoint{3.192658in}{1.739069in}}%
\pgfpathclose%
\pgfusepath{fill}%
\end{pgfscope}%
\begin{pgfscope}%
\pgfpathrectangle{\pgfqpoint{1.254980in}{0.150000in}}{\pgfqpoint{5.490039in}{5.490039in}}%
\pgfusepath{clip}%
\pgfsetbuttcap%
\pgfsetroundjoin%
\definecolor{currentfill}{rgb}{0.268510,0.009605,0.335427}%
\pgfsetfillcolor{currentfill}%
\pgfsetfillopacity{0.700000}%
\pgfsetlinewidth{0.000000pt}%
\definecolor{currentstroke}{rgb}{0.000000,0.000000,0.000000}%
\pgfsetstrokecolor{currentstroke}%
\pgfsetdash{}{0pt}%
\pgfpathmoveto{\pgfqpoint{3.945461in}{1.278428in}}%
\pgfpathlineto{\pgfqpoint{3.959134in}{1.272375in}}%
\pgfpathlineto{\pgfqpoint{3.972812in}{1.266440in}}%
\pgfpathlineto{\pgfqpoint{3.986495in}{1.260620in}}%
\pgfpathlineto{\pgfqpoint{4.000184in}{1.254916in}}%
\pgfpathlineto{\pgfqpoint{4.008126in}{1.260434in}}%
\pgfpathlineto{\pgfqpoint{4.016061in}{1.266129in}}%
\pgfpathlineto{\pgfqpoint{4.023990in}{1.271997in}}%
\pgfpathlineto{\pgfqpoint{4.031911in}{1.278034in}}%
\pgfpathlineto{\pgfqpoint{4.018240in}{1.283281in}}%
\pgfpathlineto{\pgfqpoint{4.004575in}{1.288644in}}%
\pgfpathlineto{\pgfqpoint{3.990915in}{1.294123in}}%
\pgfpathlineto{\pgfqpoint{3.977261in}{1.299719in}}%
\pgfpathlineto{\pgfqpoint{3.969322in}{1.294132in}}%
\pgfpathlineto{\pgfqpoint{3.961376in}{1.288719in}}%
\pgfpathlineto{\pgfqpoint{3.953422in}{1.283483in}}%
\pgfpathlineto{\pgfqpoint{3.945461in}{1.278428in}}%
\pgfpathclose%
\pgfusepath{fill}%
\end{pgfscope}%
\begin{pgfscope}%
\pgfpathrectangle{\pgfqpoint{1.254980in}{0.150000in}}{\pgfqpoint{5.490039in}{5.490039in}}%
\pgfusepath{clip}%
\pgfsetbuttcap%
\pgfsetroundjoin%
\definecolor{currentfill}{rgb}{0.271305,0.019942,0.347269}%
\pgfsetfillcolor{currentfill}%
\pgfsetfillopacity{0.700000}%
\pgfsetlinewidth{0.000000pt}%
\definecolor{currentstroke}{rgb}{0.000000,0.000000,0.000000}%
\pgfsetstrokecolor{currentstroke}%
\pgfsetdash{}{0pt}%
\pgfpathmoveto{\pgfqpoint{4.314192in}{1.277371in}}%
\pgfpathlineto{\pgfqpoint{4.327952in}{1.274919in}}%
\pgfpathlineto{\pgfqpoint{4.341720in}{1.272580in}}%
\pgfpathlineto{\pgfqpoint{4.355496in}{1.270352in}}%
\pgfpathlineto{\pgfqpoint{4.369280in}{1.268236in}}%
\pgfpathlineto{\pgfqpoint{4.377085in}{1.277696in}}%
\pgfpathlineto{\pgfqpoint{4.384884in}{1.287269in}}%
\pgfpathlineto{\pgfqpoint{4.392678in}{1.296951in}}%
\pgfpathlineto{\pgfqpoint{4.400468in}{1.306740in}}%
\pgfpathlineto{\pgfqpoint{4.386692in}{1.308448in}}%
\pgfpathlineto{\pgfqpoint{4.372925in}{1.310268in}}%
\pgfpathlineto{\pgfqpoint{4.359166in}{1.312200in}}%
\pgfpathlineto{\pgfqpoint{4.345416in}{1.314244in}}%
\pgfpathlineto{\pgfqpoint{4.337617in}{1.304857in}}%
\pgfpathlineto{\pgfqpoint{4.329814in}{1.295580in}}%
\pgfpathlineto{\pgfqpoint{4.322005in}{1.286417in}}%
\pgfpathlineto{\pgfqpoint{4.314192in}{1.277371in}}%
\pgfpathclose%
\pgfusepath{fill}%
\end{pgfscope}%
\begin{pgfscope}%
\pgfpathrectangle{\pgfqpoint{1.254980in}{0.150000in}}{\pgfqpoint{5.490039in}{5.490039in}}%
\pgfusepath{clip}%
\pgfsetbuttcap%
\pgfsetroundjoin%
\definecolor{currentfill}{rgb}{0.267004,0.004874,0.329415}%
\pgfsetfillcolor{currentfill}%
\pgfsetfillopacity{0.700000}%
\pgfsetlinewidth{0.000000pt}%
\definecolor{currentstroke}{rgb}{0.000000,0.000000,0.000000}%
\pgfsetstrokecolor{currentstroke}%
\pgfsetdash{}{0pt}%
\pgfpathmoveto{\pgfqpoint{4.086653in}{1.258197in}}%
\pgfpathlineto{\pgfqpoint{4.100353in}{1.253525in}}%
\pgfpathlineto{\pgfqpoint{4.114060in}{1.248967in}}%
\pgfpathlineto{\pgfqpoint{4.127773in}{1.244523in}}%
\pgfpathlineto{\pgfqpoint{4.141492in}{1.240193in}}%
\pgfpathlineto{\pgfqpoint{4.149375in}{1.247287in}}%
\pgfpathlineto{\pgfqpoint{4.157252in}{1.254535in}}%
\pgfpathlineto{\pgfqpoint{4.165123in}{1.261931in}}%
\pgfpathlineto{\pgfqpoint{4.172988in}{1.269474in}}%
\pgfpathlineto{\pgfqpoint{4.159282in}{1.273365in}}%
\pgfpathlineto{\pgfqpoint{4.145584in}{1.277369in}}%
\pgfpathlineto{\pgfqpoint{4.131892in}{1.281487in}}%
\pgfpathlineto{\pgfqpoint{4.118206in}{1.285720in}}%
\pgfpathlineto{\pgfqpoint{4.110327in}{1.278610in}}%
\pgfpathlineto{\pgfqpoint{4.102442in}{1.271651in}}%
\pgfpathlineto{\pgfqpoint{4.094551in}{1.264846in}}%
\pgfpathlineto{\pgfqpoint{4.086653in}{1.258197in}}%
\pgfpathclose%
\pgfusepath{fill}%
\end{pgfscope}%
\begin{pgfscope}%
\pgfpathrectangle{\pgfqpoint{1.254980in}{0.150000in}}{\pgfqpoint{5.490039in}{5.490039in}}%
\pgfusepath{clip}%
\pgfsetbuttcap%
\pgfsetroundjoin%
\definecolor{currentfill}{rgb}{0.412913,0.803041,0.357269}%
\pgfsetfillcolor{currentfill}%
\pgfsetfillopacity{0.700000}%
\pgfsetlinewidth{0.000000pt}%
\definecolor{currentstroke}{rgb}{0.000000,0.000000,0.000000}%
\pgfsetstrokecolor{currentstroke}%
\pgfsetdash{}{0pt}%
\pgfpathmoveto{\pgfqpoint{2.102918in}{3.337851in}}%
\pgfpathlineto{\pgfqpoint{2.117059in}{3.310795in}}%
\pgfpathlineto{\pgfqpoint{2.131187in}{3.283962in}}%
\pgfpathlineto{\pgfqpoint{2.145300in}{3.257349in}}%
\pgfpathlineto{\pgfqpoint{2.159400in}{3.230957in}}%
\pgfpathlineto{\pgfqpoint{2.168770in}{3.218880in}}%
\pgfpathlineto{\pgfqpoint{2.178113in}{3.207162in}}%
\pgfpathlineto{\pgfqpoint{2.187429in}{3.195800in}}%
\pgfpathlineto{\pgfqpoint{2.196719in}{3.184787in}}%
\pgfpathlineto{\pgfqpoint{2.182687in}{3.210615in}}%
\pgfpathlineto{\pgfqpoint{2.168641in}{3.236662in}}%
\pgfpathlineto{\pgfqpoint{2.154582in}{3.262928in}}%
\pgfpathlineto{\pgfqpoint{2.140510in}{3.289415in}}%
\pgfpathlineto{\pgfqpoint{2.131153in}{3.300986in}}%
\pgfpathlineto{\pgfqpoint{2.121768in}{3.312912in}}%
\pgfpathlineto{\pgfqpoint{2.112357in}{3.325199in}}%
\pgfpathlineto{\pgfqpoint{2.102918in}{3.337851in}}%
\pgfpathclose%
\pgfusepath{fill}%
\end{pgfscope}%
\begin{pgfscope}%
\pgfpathrectangle{\pgfqpoint{1.254980in}{0.150000in}}{\pgfqpoint{5.490039in}{5.490039in}}%
\pgfusepath{clip}%
\pgfsetbuttcap%
\pgfsetroundjoin%
\definecolor{currentfill}{rgb}{0.156270,0.489624,0.557936}%
\pgfsetfillcolor{currentfill}%
\pgfsetfillopacity{0.700000}%
\pgfsetlinewidth{0.000000pt}%
\definecolor{currentstroke}{rgb}{0.000000,0.000000,0.000000}%
\pgfsetstrokecolor{currentstroke}%
\pgfsetdash{}{0pt}%
\pgfpathmoveto{\pgfqpoint{2.643630in}{2.403152in}}%
\pgfpathlineto{\pgfqpoint{2.657449in}{2.383479in}}%
\pgfpathlineto{\pgfqpoint{2.671262in}{2.363976in}}%
\pgfpathlineto{\pgfqpoint{2.685068in}{2.344640in}}%
\pgfpathlineto{\pgfqpoint{2.698868in}{2.325470in}}%
\pgfpathlineto{\pgfqpoint{2.707753in}{2.316829in}}%
\pgfpathlineto{\pgfqpoint{2.716617in}{2.308523in}}%
\pgfpathlineto{\pgfqpoint{2.725460in}{2.300548in}}%
\pgfpathlineto{\pgfqpoint{2.734283in}{2.292897in}}%
\pgfpathlineto{\pgfqpoint{2.720537in}{2.311511in}}%
\pgfpathlineto{\pgfqpoint{2.706786in}{2.330291in}}%
\pgfpathlineto{\pgfqpoint{2.693029in}{2.349237in}}%
\pgfpathlineto{\pgfqpoint{2.679265in}{2.368351in}}%
\pgfpathlineto{\pgfqpoint{2.670388in}{2.376550in}}%
\pgfpathlineto{\pgfqpoint{2.661491in}{2.385080in}}%
\pgfpathlineto{\pgfqpoint{2.652571in}{2.393945in}}%
\pgfpathlineto{\pgfqpoint{2.643630in}{2.403152in}}%
\pgfpathclose%
\pgfusepath{fill}%
\end{pgfscope}%
\begin{pgfscope}%
\pgfpathrectangle{\pgfqpoint{1.254980in}{0.150000in}}{\pgfqpoint{5.490039in}{5.490039in}}%
\pgfusepath{clip}%
\pgfsetbuttcap%
\pgfsetroundjoin%
\definecolor{currentfill}{rgb}{0.279574,0.170599,0.479997}%
\pgfsetfillcolor{currentfill}%
\pgfsetfillopacity{0.700000}%
\pgfsetlinewidth{0.000000pt}%
\definecolor{currentstroke}{rgb}{0.000000,0.000000,0.000000}%
\pgfsetstrokecolor{currentstroke}%
\pgfsetdash{}{0pt}%
\pgfpathmoveto{\pgfqpoint{4.832367in}{1.545345in}}%
\pgfpathlineto{\pgfqpoint{4.846337in}{1.547771in}}%
\pgfpathlineto{\pgfqpoint{4.860320in}{1.550308in}}%
\pgfpathlineto{\pgfqpoint{4.874315in}{1.552955in}}%
\pgfpathlineto{\pgfqpoint{4.888322in}{1.555712in}}%
\pgfpathlineto{\pgfqpoint{4.896004in}{1.568961in}}%
\pgfpathlineto{\pgfqpoint{4.903682in}{1.582229in}}%
\pgfpathlineto{\pgfqpoint{4.911357in}{1.595514in}}%
\pgfpathlineto{\pgfqpoint{4.919027in}{1.608813in}}%
\pgfpathlineto{\pgfqpoint{4.905019in}{1.605739in}}%
\pgfpathlineto{\pgfqpoint{4.891024in}{1.602775in}}%
\pgfpathlineto{\pgfqpoint{4.877042in}{1.599922in}}%
\pgfpathlineto{\pgfqpoint{4.863071in}{1.597180in}}%
\pgfpathlineto{\pgfqpoint{4.855401in}{1.584191in}}%
\pgfpathlineto{\pgfqpoint{4.847727in}{1.571221in}}%
\pgfpathlineto{\pgfqpoint{4.840049in}{1.558271in}}%
\pgfpathlineto{\pgfqpoint{4.832367in}{1.545345in}}%
\pgfpathclose%
\pgfusepath{fill}%
\end{pgfscope}%
\begin{pgfscope}%
\pgfpathrectangle{\pgfqpoint{1.254980in}{0.150000in}}{\pgfqpoint{5.490039in}{5.490039in}}%
\pgfusepath{clip}%
\pgfsetbuttcap%
\pgfsetroundjoin%
\definecolor{currentfill}{rgb}{0.271828,0.209303,0.504434}%
\pgfsetfillcolor{currentfill}%
\pgfsetfillopacity{0.700000}%
\pgfsetlinewidth{0.000000pt}%
\definecolor{currentstroke}{rgb}{0.000000,0.000000,0.000000}%
\pgfsetstrokecolor{currentstroke}%
\pgfsetdash{}{0pt}%
\pgfpathmoveto{\pgfqpoint{3.247312in}{1.685547in}}%
\pgfpathlineto{\pgfqpoint{3.260973in}{1.672508in}}%
\pgfpathlineto{\pgfqpoint{3.274634in}{1.659604in}}%
\pgfpathlineto{\pgfqpoint{3.288294in}{1.646835in}}%
\pgfpathlineto{\pgfqpoint{3.301953in}{1.634200in}}%
\pgfpathlineto{\pgfqpoint{3.310327in}{1.631551in}}%
\pgfpathlineto{\pgfqpoint{3.318687in}{1.629179in}}%
\pgfpathlineto{\pgfqpoint{3.327033in}{1.627080in}}%
\pgfpathlineto{\pgfqpoint{3.335366in}{1.625249in}}%
\pgfpathlineto{\pgfqpoint{3.321744in}{1.637362in}}%
\pgfpathlineto{\pgfqpoint{3.308121in}{1.649608in}}%
\pgfpathlineto{\pgfqpoint{3.294499in}{1.661988in}}%
\pgfpathlineto{\pgfqpoint{3.280877in}{1.674503in}}%
\pgfpathlineto{\pgfqpoint{3.272507in}{1.676850in}}%
\pgfpathlineto{\pgfqpoint{3.264123in}{1.679469in}}%
\pgfpathlineto{\pgfqpoint{3.255725in}{1.682367in}}%
\pgfpathlineto{\pgfqpoint{3.247312in}{1.685547in}}%
\pgfpathclose%
\pgfusepath{fill}%
\end{pgfscope}%
\begin{pgfscope}%
\pgfpathrectangle{\pgfqpoint{1.254980in}{0.150000in}}{\pgfqpoint{5.490039in}{5.490039in}}%
\pgfusepath{clip}%
\pgfsetbuttcap%
\pgfsetroundjoin%
\definecolor{currentfill}{rgb}{0.170948,0.694384,0.493803}%
\pgfsetfillcolor{currentfill}%
\pgfsetfillopacity{0.700000}%
\pgfsetlinewidth{0.000000pt}%
\definecolor{currentstroke}{rgb}{0.000000,0.000000,0.000000}%
\pgfsetstrokecolor{currentstroke}%
\pgfsetdash{}{0pt}%
\pgfpathmoveto{\pgfqpoint{2.290673in}{2.975854in}}%
\pgfpathlineto{\pgfqpoint{2.304680in}{2.951606in}}%
\pgfpathlineto{\pgfqpoint{2.318676in}{2.927557in}}%
\pgfpathlineto{\pgfqpoint{2.332661in}{2.903707in}}%
\pgfpathlineto{\pgfqpoint{2.346635in}{2.880055in}}%
\pgfpathlineto{\pgfqpoint{2.355844in}{2.868830in}}%
\pgfpathlineto{\pgfqpoint{2.365028in}{2.857960in}}%
\pgfpathlineto{\pgfqpoint{2.374187in}{2.847441in}}%
\pgfpathlineto{\pgfqpoint{2.383322in}{2.837267in}}%
\pgfpathlineto{\pgfqpoint{2.369411in}{2.860354in}}%
\pgfpathlineto{\pgfqpoint{2.355490in}{2.883637in}}%
\pgfpathlineto{\pgfqpoint{2.341558in}{2.907118in}}%
\pgfpathlineto{\pgfqpoint{2.327616in}{2.930797in}}%
\pgfpathlineto{\pgfqpoint{2.318418in}{2.941529in}}%
\pgfpathlineto{\pgfqpoint{2.309196in}{2.952613in}}%
\pgfpathlineto{\pgfqpoint{2.299947in}{2.964053in}}%
\pgfpathlineto{\pgfqpoint{2.290673in}{2.975854in}}%
\pgfpathclose%
\pgfusepath{fill}%
\end{pgfscope}%
\begin{pgfscope}%
\pgfpathrectangle{\pgfqpoint{1.254980in}{0.150000in}}{\pgfqpoint{5.490039in}{5.490039in}}%
\pgfusepath{clip}%
\pgfsetbuttcap%
\pgfsetroundjoin%
\definecolor{currentfill}{rgb}{0.273809,0.031497,0.358853}%
\pgfsetfillcolor{currentfill}%
\pgfsetfillopacity{0.700000}%
\pgfsetlinewidth{0.000000pt}%
\definecolor{currentstroke}{rgb}{0.000000,0.000000,0.000000}%
\pgfsetstrokecolor{currentstroke}%
\pgfsetdash{}{0pt}%
\pgfpathmoveto{\pgfqpoint{3.804153in}{1.316533in}}%
\pgfpathlineto{\pgfqpoint{3.817811in}{1.309064in}}%
\pgfpathlineto{\pgfqpoint{3.831474in}{1.301714in}}%
\pgfpathlineto{\pgfqpoint{3.845140in}{1.294483in}}%
\pgfpathlineto{\pgfqpoint{3.858811in}{1.287370in}}%
\pgfpathlineto{\pgfqpoint{3.866825in}{1.291181in}}%
\pgfpathlineto{\pgfqpoint{3.874831in}{1.295194in}}%
\pgfpathlineto{\pgfqpoint{3.882829in}{1.299403in}}%
\pgfpathlineto{\pgfqpoint{3.890818in}{1.303806in}}%
\pgfpathlineto{\pgfqpoint{3.877169in}{1.310444in}}%
\pgfpathlineto{\pgfqpoint{3.863525in}{1.317201in}}%
\pgfpathlineto{\pgfqpoint{3.849885in}{1.324075in}}%
\pgfpathlineto{\pgfqpoint{3.836250in}{1.331069in}}%
\pgfpathlineto{\pgfqpoint{3.828238in}{1.327135in}}%
\pgfpathlineto{\pgfqpoint{3.820219in}{1.323398in}}%
\pgfpathlineto{\pgfqpoint{3.812190in}{1.319862in}}%
\pgfpathlineto{\pgfqpoint{3.804153in}{1.316533in}}%
\pgfpathclose%
\pgfusepath{fill}%
\end{pgfscope}%
\begin{pgfscope}%
\pgfpathrectangle{\pgfqpoint{1.254980in}{0.150000in}}{\pgfqpoint{5.490039in}{5.490039in}}%
\pgfusepath{clip}%
\pgfsetbuttcap%
\pgfsetroundjoin%
\definecolor{currentfill}{rgb}{0.280894,0.078907,0.402329}%
\pgfsetfillcolor{currentfill}%
\pgfsetfillopacity{0.700000}%
\pgfsetlinewidth{0.000000pt}%
\definecolor{currentstroke}{rgb}{0.000000,0.000000,0.000000}%
\pgfsetstrokecolor{currentstroke}%
\pgfsetdash{}{0pt}%
\pgfpathmoveto{\pgfqpoint{3.607968in}{1.410303in}}%
\pgfpathlineto{\pgfqpoint{3.621615in}{1.400885in}}%
\pgfpathlineto{\pgfqpoint{3.635264in}{1.391591in}}%
\pgfpathlineto{\pgfqpoint{3.648916in}{1.382419in}}%
\pgfpathlineto{\pgfqpoint{3.662571in}{1.373370in}}%
\pgfpathlineto{\pgfqpoint{3.670698in}{1.374836in}}%
\pgfpathlineto{\pgfqpoint{3.678815in}{1.376532in}}%
\pgfpathlineto{\pgfqpoint{3.686921in}{1.378456in}}%
\pgfpathlineto{\pgfqpoint{3.695018in}{1.380603in}}%
\pgfpathlineto{\pgfqpoint{3.681391in}{1.389157in}}%
\pgfpathlineto{\pgfqpoint{3.667767in}{1.397833in}}%
\pgfpathlineto{\pgfqpoint{3.654145in}{1.406633in}}%
\pgfpathlineto{\pgfqpoint{3.640527in}{1.415555in}}%
\pgfpathlineto{\pgfqpoint{3.632403in}{1.413897in}}%
\pgfpathlineto{\pgfqpoint{3.624269in}{1.412466in}}%
\pgfpathlineto{\pgfqpoint{3.616124in}{1.411267in}}%
\pgfpathlineto{\pgfqpoint{3.607968in}{1.410303in}}%
\pgfpathclose%
\pgfusepath{fill}%
\end{pgfscope}%
\begin{pgfscope}%
\pgfpathrectangle{\pgfqpoint{1.254980in}{0.150000in}}{\pgfqpoint{5.490039in}{5.490039in}}%
\pgfusepath{clip}%
\pgfsetbuttcap%
\pgfsetroundjoin%
\definecolor{currentfill}{rgb}{0.144759,0.519093,0.556572}%
\pgfsetfillcolor{currentfill}%
\pgfsetfillopacity{0.700000}%
\pgfsetlinewidth{0.000000pt}%
\definecolor{currentstroke}{rgb}{0.000000,0.000000,0.000000}%
\pgfsetstrokecolor{currentstroke}%
\pgfsetdash{}{0pt}%
\pgfpathmoveto{\pgfqpoint{2.588283in}{2.483546in}}%
\pgfpathlineto{\pgfqpoint{2.602131in}{2.463189in}}%
\pgfpathlineto{\pgfqpoint{2.615971in}{2.443006in}}%
\pgfpathlineto{\pgfqpoint{2.629804in}{2.422993in}}%
\pgfpathlineto{\pgfqpoint{2.643630in}{2.403152in}}%
\pgfpathlineto{\pgfqpoint{2.652571in}{2.393945in}}%
\pgfpathlineto{\pgfqpoint{2.661491in}{2.385080in}}%
\pgfpathlineto{\pgfqpoint{2.670388in}{2.376550in}}%
\pgfpathlineto{\pgfqpoint{2.679265in}{2.368351in}}%
\pgfpathlineto{\pgfqpoint{2.665495in}{2.387633in}}%
\pgfpathlineto{\pgfqpoint{2.651718in}{2.407084in}}%
\pgfpathlineto{\pgfqpoint{2.637934in}{2.426706in}}%
\pgfpathlineto{\pgfqpoint{2.624144in}{2.446500in}}%
\pgfpathlineto{\pgfqpoint{2.615212in}{2.455252in}}%
\pgfpathlineto{\pgfqpoint{2.606258in}{2.464340in}}%
\pgfpathlineto{\pgfqpoint{2.597282in}{2.473770in}}%
\pgfpathlineto{\pgfqpoint{2.588283in}{2.483546in}}%
\pgfpathclose%
\pgfusepath{fill}%
\end{pgfscope}%
\begin{pgfscope}%
\pgfpathrectangle{\pgfqpoint{1.254980in}{0.150000in}}{\pgfqpoint{5.490039in}{5.490039in}}%
\pgfusepath{clip}%
\pgfsetbuttcap%
\pgfsetroundjoin%
\definecolor{currentfill}{rgb}{0.268510,0.009605,0.335427}%
\pgfsetfillcolor{currentfill}%
\pgfsetfillopacity{0.700000}%
\pgfsetlinewidth{0.000000pt}%
\definecolor{currentstroke}{rgb}{0.000000,0.000000,0.000000}%
\pgfsetstrokecolor{currentstroke}%
\pgfsetdash{}{0pt}%
\pgfpathmoveto{\pgfqpoint{4.227879in}{1.255047in}}%
\pgfpathlineto{\pgfqpoint{4.241619in}{1.251722in}}%
\pgfpathlineto{\pgfqpoint{4.255367in}{1.248510in}}%
\pgfpathlineto{\pgfqpoint{4.269122in}{1.245411in}}%
\pgfpathlineto{\pgfqpoint{4.282885in}{1.242423in}}%
\pgfpathlineto{\pgfqpoint{4.290720in}{1.250968in}}%
\pgfpathlineto{\pgfqpoint{4.298549in}{1.259643in}}%
\pgfpathlineto{\pgfqpoint{4.306373in}{1.268445in}}%
\pgfpathlineto{\pgfqpoint{4.314192in}{1.277371in}}%
\pgfpathlineto{\pgfqpoint{4.300440in}{1.279935in}}%
\pgfpathlineto{\pgfqpoint{4.286695in}{1.282610in}}%
\pgfpathlineto{\pgfqpoint{4.272959in}{1.285399in}}%
\pgfpathlineto{\pgfqpoint{4.259230in}{1.288300in}}%
\pgfpathlineto{\pgfqpoint{4.251400in}{1.279791in}}%
\pgfpathlineto{\pgfqpoint{4.243565in}{1.271411in}}%
\pgfpathlineto{\pgfqpoint{4.235725in}{1.263161in}}%
\pgfpathlineto{\pgfqpoint{4.227879in}{1.255047in}}%
\pgfpathclose%
\pgfusepath{fill}%
\end{pgfscope}%
\begin{pgfscope}%
\pgfpathrectangle{\pgfqpoint{1.254980in}{0.150000in}}{\pgfqpoint{5.490039in}{5.490039in}}%
\pgfusepath{clip}%
\pgfsetbuttcap%
\pgfsetroundjoin%
\definecolor{currentfill}{rgb}{0.277134,0.185228,0.489898}%
\pgfsetfillcolor{currentfill}%
\pgfsetfillopacity{0.700000}%
\pgfsetlinewidth{0.000000pt}%
\definecolor{currentstroke}{rgb}{0.000000,0.000000,0.000000}%
\pgfsetstrokecolor{currentstroke}%
\pgfsetdash{}{0pt}%
\pgfpathmoveto{\pgfqpoint{3.301953in}{1.634200in}}%
\pgfpathlineto{\pgfqpoint{3.315613in}{1.621699in}}%
\pgfpathlineto{\pgfqpoint{3.329272in}{1.609331in}}%
\pgfpathlineto{\pgfqpoint{3.342931in}{1.597095in}}%
\pgfpathlineto{\pgfqpoint{3.356590in}{1.584991in}}%
\pgfpathlineto{\pgfqpoint{3.364926in}{1.582871in}}%
\pgfpathlineto{\pgfqpoint{3.373249in}{1.581023in}}%
\pgfpathlineto{\pgfqpoint{3.381559in}{1.579443in}}%
\pgfpathlineto{\pgfqpoint{3.389855in}{1.578126in}}%
\pgfpathlineto{\pgfqpoint{3.376232in}{1.589709in}}%
\pgfpathlineto{\pgfqpoint{3.362610in}{1.601424in}}%
\pgfpathlineto{\pgfqpoint{3.348988in}{1.613270in}}%
\pgfpathlineto{\pgfqpoint{3.335366in}{1.625249in}}%
\pgfpathlineto{\pgfqpoint{3.327033in}{1.627080in}}%
\pgfpathlineto{\pgfqpoint{3.318687in}{1.629179in}}%
\pgfpathlineto{\pgfqpoint{3.310327in}{1.631551in}}%
\pgfpathlineto{\pgfqpoint{3.301953in}{1.634200in}}%
\pgfpathclose%
\pgfusepath{fill}%
\end{pgfscope}%
\begin{pgfscope}%
\pgfpathrectangle{\pgfqpoint{1.254980in}{0.150000in}}{\pgfqpoint{5.490039in}{5.490039in}}%
\pgfusepath{clip}%
\pgfsetbuttcap%
\pgfsetroundjoin%
\definecolor{currentfill}{rgb}{0.241237,0.296485,0.539709}%
\pgfsetfillcolor{currentfill}%
\pgfsetfillopacity{0.700000}%
\pgfsetlinewidth{0.000000pt}%
\definecolor{currentstroke}{rgb}{0.000000,0.000000,0.000000}%
\pgfsetstrokecolor{currentstroke}%
\pgfsetdash{}{0pt}%
\pgfpathmoveto{\pgfqpoint{5.123309in}{1.804124in}}%
\pgfpathlineto{\pgfqpoint{5.137439in}{1.809052in}}%
\pgfpathlineto{\pgfqpoint{5.151584in}{1.814090in}}%
\pgfpathlineto{\pgfqpoint{5.165743in}{1.819240in}}%
\pgfpathlineto{\pgfqpoint{5.179916in}{1.824500in}}%
\pgfpathlineto{\pgfqpoint{5.187538in}{1.838563in}}%
\pgfpathlineto{\pgfqpoint{5.195156in}{1.852598in}}%
\pgfpathlineto{\pgfqpoint{5.202769in}{1.866604in}}%
\pgfpathlineto{\pgfqpoint{5.210378in}{1.880577in}}%
\pgfpathlineto{\pgfqpoint{5.196202in}{1.875062in}}%
\pgfpathlineto{\pgfqpoint{5.182040in}{1.869657in}}%
\pgfpathlineto{\pgfqpoint{5.167893in}{1.864364in}}%
\pgfpathlineto{\pgfqpoint{5.153759in}{1.859182in}}%
\pgfpathlineto{\pgfqpoint{5.146153in}{1.845457in}}%
\pgfpathlineto{\pgfqpoint{5.138543in}{1.831705in}}%
\pgfpathlineto{\pgfqpoint{5.130928in}{1.817927in}}%
\pgfpathlineto{\pgfqpoint{5.123309in}{1.804124in}}%
\pgfpathclose%
\pgfusepath{fill}%
\end{pgfscope}%
\begin{pgfscope}%
\pgfpathrectangle{\pgfqpoint{1.254980in}{0.150000in}}{\pgfqpoint{5.490039in}{5.490039in}}%
\pgfusepath{clip}%
\pgfsetbuttcap%
\pgfsetroundjoin%
\definecolor{currentfill}{rgb}{0.273006,0.204520,0.501721}%
\pgfsetfillcolor{currentfill}%
\pgfsetfillopacity{0.700000}%
\pgfsetlinewidth{0.000000pt}%
\definecolor{currentstroke}{rgb}{0.000000,0.000000,0.000000}%
\pgfsetstrokecolor{currentstroke}%
\pgfsetdash{}{0pt}%
\pgfpathmoveto{\pgfqpoint{4.919027in}{1.608813in}}%
\pgfpathlineto{\pgfqpoint{4.933047in}{1.611997in}}%
\pgfpathlineto{\pgfqpoint{4.947079in}{1.615292in}}%
\pgfpathlineto{\pgfqpoint{4.961124in}{1.618697in}}%
\pgfpathlineto{\pgfqpoint{4.975182in}{1.622212in}}%
\pgfpathlineto{\pgfqpoint{4.982850in}{1.635831in}}%
\pgfpathlineto{\pgfqpoint{4.990513in}{1.649456in}}%
\pgfpathlineto{\pgfqpoint{4.998173in}{1.663084in}}%
\pgfpathlineto{\pgfqpoint{5.005828in}{1.676713in}}%
\pgfpathlineto{\pgfqpoint{4.991769in}{1.672896in}}%
\pgfpathlineto{\pgfqpoint{4.977723in}{1.669189in}}%
\pgfpathlineto{\pgfqpoint{4.963689in}{1.665593in}}%
\pgfpathlineto{\pgfqpoint{4.949668in}{1.662108in}}%
\pgfpathlineto{\pgfqpoint{4.942014in}{1.648774in}}%
\pgfpathlineto{\pgfqpoint{4.934356in}{1.635445in}}%
\pgfpathlineto{\pgfqpoint{4.926693in}{1.622124in}}%
\pgfpathlineto{\pgfqpoint{4.919027in}{1.608813in}}%
\pgfpathclose%
\pgfusepath{fill}%
\end{pgfscope}%
\begin{pgfscope}%
\pgfpathrectangle{\pgfqpoint{1.254980in}{0.150000in}}{\pgfqpoint{5.490039in}{5.490039in}}%
\pgfusepath{clip}%
\pgfsetbuttcap%
\pgfsetroundjoin%
\definecolor{currentfill}{rgb}{0.132444,0.552216,0.553018}%
\pgfsetfillcolor{currentfill}%
\pgfsetfillopacity{0.700000}%
\pgfsetlinewidth{0.000000pt}%
\definecolor{currentstroke}{rgb}{0.000000,0.000000,0.000000}%
\pgfsetstrokecolor{currentstroke}%
\pgfsetdash{}{0pt}%
\pgfpathmoveto{\pgfqpoint{2.532817in}{2.566718in}}%
\pgfpathlineto{\pgfqpoint{2.546695in}{2.545661in}}%
\pgfpathlineto{\pgfqpoint{2.560566in}{2.524780in}}%
\pgfpathlineto{\pgfqpoint{2.574428in}{2.504076in}}%
\pgfpathlineto{\pgfqpoint{2.588283in}{2.483546in}}%
\pgfpathlineto{\pgfqpoint{2.597282in}{2.473770in}}%
\pgfpathlineto{\pgfqpoint{2.606258in}{2.464340in}}%
\pgfpathlineto{\pgfqpoint{2.615212in}{2.455252in}}%
\pgfpathlineto{\pgfqpoint{2.624144in}{2.446500in}}%
\pgfpathlineto{\pgfqpoint{2.610347in}{2.466466in}}%
\pgfpathlineto{\pgfqpoint{2.596542in}{2.486605in}}%
\pgfpathlineto{\pgfqpoint{2.582730in}{2.506919in}}%
\pgfpathlineto{\pgfqpoint{2.568911in}{2.527409in}}%
\pgfpathlineto{\pgfqpoint{2.559922in}{2.536718in}}%
\pgfpathlineto{\pgfqpoint{2.550910in}{2.546369in}}%
\pgfpathlineto{\pgfqpoint{2.541875in}{2.556367in}}%
\pgfpathlineto{\pgfqpoint{2.532817in}{2.566718in}}%
\pgfpathclose%
\pgfusepath{fill}%
\end{pgfscope}%
\begin{pgfscope}%
\pgfpathrectangle{\pgfqpoint{1.254980in}{0.150000in}}{\pgfqpoint{5.490039in}{5.490039in}}%
\pgfusepath{clip}%
\pgfsetbuttcap%
\pgfsetroundjoin%
\definecolor{currentfill}{rgb}{0.280255,0.165693,0.476498}%
\pgfsetfillcolor{currentfill}%
\pgfsetfillopacity{0.700000}%
\pgfsetlinewidth{0.000000pt}%
\definecolor{currentstroke}{rgb}{0.000000,0.000000,0.000000}%
\pgfsetstrokecolor{currentstroke}%
\pgfsetdash{}{0pt}%
\pgfpathmoveto{\pgfqpoint{3.356590in}{1.584991in}}%
\pgfpathlineto{\pgfqpoint{3.370249in}{1.573019in}}%
\pgfpathlineto{\pgfqpoint{3.383909in}{1.561179in}}%
\pgfpathlineto{\pgfqpoint{3.397568in}{1.549468in}}%
\pgfpathlineto{\pgfqpoint{3.411229in}{1.537888in}}%
\pgfpathlineto{\pgfqpoint{3.419529in}{1.536294in}}%
\pgfpathlineto{\pgfqpoint{3.427817in}{1.534968in}}%
\pgfpathlineto{\pgfqpoint{3.436091in}{1.533905in}}%
\pgfpathlineto{\pgfqpoint{3.444353in}{1.533101in}}%
\pgfpathlineto{\pgfqpoint{3.430727in}{1.544162in}}%
\pgfpathlineto{\pgfqpoint{3.417103in}{1.555353in}}%
\pgfpathlineto{\pgfqpoint{3.403479in}{1.566674in}}%
\pgfpathlineto{\pgfqpoint{3.389855in}{1.578126in}}%
\pgfpathlineto{\pgfqpoint{3.381559in}{1.579443in}}%
\pgfpathlineto{\pgfqpoint{3.373249in}{1.581023in}}%
\pgfpathlineto{\pgfqpoint{3.364926in}{1.582871in}}%
\pgfpathlineto{\pgfqpoint{3.356590in}{1.584991in}}%
\pgfpathclose%
\pgfusepath{fill}%
\end{pgfscope}%
\begin{pgfscope}%
\pgfpathrectangle{\pgfqpoint{1.254980in}{0.150000in}}{\pgfqpoint{5.490039in}{5.490039in}}%
\pgfusepath{clip}%
\pgfsetbuttcap%
\pgfsetroundjoin%
\definecolor{currentfill}{rgb}{0.268510,0.009605,0.335427}%
\pgfsetfillcolor{currentfill}%
\pgfsetfillopacity{0.700000}%
\pgfsetlinewidth{0.000000pt}%
\definecolor{currentstroke}{rgb}{0.000000,0.000000,0.000000}%
\pgfsetstrokecolor{currentstroke}%
\pgfsetdash{}{0pt}%
\pgfpathmoveto{\pgfqpoint{4.000184in}{1.254916in}}%
\pgfpathlineto{\pgfqpoint{4.013878in}{1.249328in}}%
\pgfpathlineto{\pgfqpoint{4.027577in}{1.243855in}}%
\pgfpathlineto{\pgfqpoint{4.041282in}{1.238497in}}%
\pgfpathlineto{\pgfqpoint{4.054993in}{1.233253in}}%
\pgfpathlineto{\pgfqpoint{4.062918in}{1.239235in}}%
\pgfpathlineto{\pgfqpoint{4.070837in}{1.245388in}}%
\pgfpathlineto{\pgfqpoint{4.078748in}{1.251710in}}%
\pgfpathlineto{\pgfqpoint{4.086653in}{1.258197in}}%
\pgfpathlineto{\pgfqpoint{4.072958in}{1.262984in}}%
\pgfpathlineto{\pgfqpoint{4.059270in}{1.267886in}}%
\pgfpathlineto{\pgfqpoint{4.045588in}{1.272902in}}%
\pgfpathlineto{\pgfqpoint{4.031911in}{1.278034in}}%
\pgfpathlineto{\pgfqpoint{4.023990in}{1.271997in}}%
\pgfpathlineto{\pgfqpoint{4.016061in}{1.266129in}}%
\pgfpathlineto{\pgfqpoint{4.008126in}{1.260434in}}%
\pgfpathlineto{\pgfqpoint{4.000184in}{1.254916in}}%
\pgfpathclose%
\pgfusepath{fill}%
\end{pgfscope}%
\begin{pgfscope}%
\pgfpathrectangle{\pgfqpoint{1.254980in}{0.150000in}}{\pgfqpoint{5.490039in}{5.490039in}}%
\pgfusepath{clip}%
\pgfsetbuttcap%
\pgfsetroundjoin%
\definecolor{currentfill}{rgb}{0.279566,0.067836,0.391917}%
\pgfsetfillcolor{currentfill}%
\pgfsetfillopacity{0.700000}%
\pgfsetlinewidth{0.000000pt}%
\definecolor{currentstroke}{rgb}{0.000000,0.000000,0.000000}%
\pgfsetstrokecolor{currentstroke}%
\pgfsetdash{}{0pt}%
\pgfpathmoveto{\pgfqpoint{4.542050in}{1.340370in}}%
\pgfpathlineto{\pgfqpoint{4.555901in}{1.340053in}}%
\pgfpathlineto{\pgfqpoint{4.569762in}{1.339847in}}%
\pgfpathlineto{\pgfqpoint{4.583633in}{1.339751in}}%
\pgfpathlineto{\pgfqpoint{4.597513in}{1.339766in}}%
\pgfpathlineto{\pgfqpoint{4.605263in}{1.351187in}}%
\pgfpathlineto{\pgfqpoint{4.613008in}{1.362683in}}%
\pgfpathlineto{\pgfqpoint{4.620749in}{1.374251in}}%
\pgfpathlineto{\pgfqpoint{4.628487in}{1.385889in}}%
\pgfpathlineto{\pgfqpoint{4.614610in}{1.385497in}}%
\pgfpathlineto{\pgfqpoint{4.600743in}{1.385215in}}%
\pgfpathlineto{\pgfqpoint{4.586887in}{1.385043in}}%
\pgfpathlineto{\pgfqpoint{4.573040in}{1.384982in}}%
\pgfpathlineto{\pgfqpoint{4.565299in}{1.373716in}}%
\pgfpathlineto{\pgfqpoint{4.557554in}{1.362523in}}%
\pgfpathlineto{\pgfqpoint{4.549804in}{1.351407in}}%
\pgfpathlineto{\pgfqpoint{4.542050in}{1.340370in}}%
\pgfpathclose%
\pgfusepath{fill}%
\end{pgfscope}%
\begin{pgfscope}%
\pgfpathrectangle{\pgfqpoint{1.254980in}{0.150000in}}{\pgfqpoint{5.490039in}{5.490039in}}%
\pgfusepath{clip}%
\pgfsetbuttcap%
\pgfsetroundjoin%
\definecolor{currentfill}{rgb}{0.226397,0.728888,0.462789}%
\pgfsetfillcolor{currentfill}%
\pgfsetfillopacity{0.700000}%
\pgfsetlinewidth{0.000000pt}%
\definecolor{currentstroke}{rgb}{0.000000,0.000000,0.000000}%
\pgfsetstrokecolor{currentstroke}%
\pgfsetdash{}{0pt}%
\pgfpathmoveto{\pgfqpoint{2.234529in}{3.074876in}}%
\pgfpathlineto{\pgfqpoint{2.248583in}{3.049814in}}%
\pgfpathlineto{\pgfqpoint{2.262625in}{3.024957in}}%
\pgfpathlineto{\pgfqpoint{2.276655in}{3.000304in}}%
\pgfpathlineto{\pgfqpoint{2.290673in}{2.975854in}}%
\pgfpathlineto{\pgfqpoint{2.299947in}{2.964053in}}%
\pgfpathlineto{\pgfqpoint{2.309196in}{2.952613in}}%
\pgfpathlineto{\pgfqpoint{2.318418in}{2.941529in}}%
\pgfpathlineto{\pgfqpoint{2.327616in}{2.930797in}}%
\pgfpathlineto{\pgfqpoint{2.313663in}{2.954676in}}%
\pgfpathlineto{\pgfqpoint{2.299698in}{2.978756in}}%
\pgfpathlineto{\pgfqpoint{2.285723in}{3.003040in}}%
\pgfpathlineto{\pgfqpoint{2.271736in}{3.027527in}}%
\pgfpathlineto{\pgfqpoint{2.262473in}{3.038823in}}%
\pgfpathlineto{\pgfqpoint{2.253185in}{3.050477in}}%
\pgfpathlineto{\pgfqpoint{2.243871in}{3.062493in}}%
\pgfpathlineto{\pgfqpoint{2.234529in}{3.074876in}}%
\pgfpathclose%
\pgfusepath{fill}%
\end{pgfscope}%
\begin{pgfscope}%
\pgfpathrectangle{\pgfqpoint{1.254980in}{0.150000in}}{\pgfqpoint{5.490039in}{5.490039in}}%
\pgfusepath{clip}%
\pgfsetbuttcap%
\pgfsetroundjoin%
\definecolor{currentfill}{rgb}{0.282327,0.094955,0.417331}%
\pgfsetfillcolor{currentfill}%
\pgfsetfillopacity{0.700000}%
\pgfsetlinewidth{0.000000pt}%
\definecolor{currentstroke}{rgb}{0.000000,0.000000,0.000000}%
\pgfsetstrokecolor{currentstroke}%
\pgfsetdash{}{0pt}%
\pgfpathmoveto{\pgfqpoint{4.628487in}{1.385889in}}%
\pgfpathlineto{\pgfqpoint{4.642374in}{1.386392in}}%
\pgfpathlineto{\pgfqpoint{4.656271in}{1.387005in}}%
\pgfpathlineto{\pgfqpoint{4.670179in}{1.387728in}}%
\pgfpathlineto{\pgfqpoint{4.684097in}{1.388561in}}%
\pgfpathlineto{\pgfqpoint{4.691828in}{1.400634in}}%
\pgfpathlineto{\pgfqpoint{4.699554in}{1.412766in}}%
\pgfpathlineto{\pgfqpoint{4.707276in}{1.424955in}}%
\pgfpathlineto{\pgfqpoint{4.714995in}{1.437197in}}%
\pgfpathlineto{\pgfqpoint{4.701079in}{1.436001in}}%
\pgfpathlineto{\pgfqpoint{4.687173in}{1.434915in}}%
\pgfpathlineto{\pgfqpoint{4.673279in}{1.433940in}}%
\pgfpathlineto{\pgfqpoint{4.659395in}{1.433075in}}%
\pgfpathlineto{\pgfqpoint{4.651674in}{1.421189in}}%
\pgfpathlineto{\pgfqpoint{4.643949in}{1.409361in}}%
\pgfpathlineto{\pgfqpoint{4.636220in}{1.397593in}}%
\pgfpathlineto{\pgfqpoint{4.628487in}{1.385889in}}%
\pgfpathclose%
\pgfusepath{fill}%
\end{pgfscope}%
\begin{pgfscope}%
\pgfpathrectangle{\pgfqpoint{1.254980in}{0.150000in}}{\pgfqpoint{5.490039in}{5.490039in}}%
\pgfusepath{clip}%
\pgfsetbuttcap%
\pgfsetroundjoin%
\definecolor{currentfill}{rgb}{0.225863,0.330805,0.547314}%
\pgfsetfillcolor{currentfill}%
\pgfsetfillopacity{0.700000}%
\pgfsetlinewidth{0.000000pt}%
\definecolor{currentstroke}{rgb}{0.000000,0.000000,0.000000}%
\pgfsetstrokecolor{currentstroke}%
\pgfsetdash{}{0pt}%
\pgfpathmoveto{\pgfqpoint{5.210378in}{1.880577in}}%
\pgfpathlineto{\pgfqpoint{5.224568in}{1.886204in}}%
\pgfpathlineto{\pgfqpoint{5.238773in}{1.891942in}}%
\pgfpathlineto{\pgfqpoint{5.252993in}{1.897791in}}%
\pgfpathlineto{\pgfqpoint{5.260599in}{1.911916in}}%
\pgfpathlineto{\pgfqpoint{5.268201in}{1.926001in}}%
\pgfpathlineto{\pgfqpoint{5.275798in}{1.940047in}}%
\pgfpathlineto{\pgfqpoint{5.283390in}{1.954050in}}%
\pgfpathlineto{\pgfqpoint{5.269167in}{1.947961in}}%
\pgfpathlineto{\pgfqpoint{5.254958in}{1.941984in}}%
\pgfpathlineto{\pgfqpoint{5.240765in}{1.936118in}}%
\pgfpathlineto{\pgfqpoint{5.233175in}{1.922290in}}%
\pgfpathlineto{\pgfqpoint{5.225581in}{1.908422in}}%
\pgfpathlineto{\pgfqpoint{5.217982in}{1.894517in}}%
\pgfpathlineto{\pgfqpoint{5.210378in}{1.880577in}}%
\pgfpathclose%
\pgfusepath{fill}%
\end{pgfscope}%
\begin{pgfscope}%
\pgfpathrectangle{\pgfqpoint{1.254980in}{0.150000in}}{\pgfqpoint{5.490039in}{5.490039in}}%
\pgfusepath{clip}%
\pgfsetbuttcap%
\pgfsetroundjoin%
\definecolor{currentfill}{rgb}{0.276022,0.044167,0.370164}%
\pgfsetfillcolor{currentfill}%
\pgfsetfillopacity{0.700000}%
\pgfsetlinewidth{0.000000pt}%
\definecolor{currentstroke}{rgb}{0.000000,0.000000,0.000000}%
\pgfsetstrokecolor{currentstroke}%
\pgfsetdash{}{0pt}%
\pgfpathmoveto{\pgfqpoint{4.455658in}{1.301021in}}%
\pgfpathlineto{\pgfqpoint{4.469477in}{1.299868in}}%
\pgfpathlineto{\pgfqpoint{4.483306in}{1.298827in}}%
\pgfpathlineto{\pgfqpoint{4.497144in}{1.297896in}}%
\pgfpathlineto{\pgfqpoint{4.510991in}{1.297076in}}%
\pgfpathlineto{\pgfqpoint{4.518762in}{1.307765in}}%
\pgfpathlineto{\pgfqpoint{4.526529in}{1.318546in}}%
\pgfpathlineto{\pgfqpoint{4.534292in}{1.329415in}}%
\pgfpathlineto{\pgfqpoint{4.542050in}{1.340370in}}%
\pgfpathlineto{\pgfqpoint{4.528208in}{1.340797in}}%
\pgfpathlineto{\pgfqpoint{4.514376in}{1.341335in}}%
\pgfpathlineto{\pgfqpoint{4.500554in}{1.341983in}}%
\pgfpathlineto{\pgfqpoint{4.486741in}{1.342743in}}%
\pgfpathlineto{\pgfqpoint{4.478977in}{1.332175in}}%
\pgfpathlineto{\pgfqpoint{4.471208in}{1.321696in}}%
\pgfpathlineto{\pgfqpoint{4.463435in}{1.311311in}}%
\pgfpathlineto{\pgfqpoint{4.455658in}{1.301021in}}%
\pgfpathclose%
\pgfusepath{fill}%
\end{pgfscope}%
\begin{pgfscope}%
\pgfpathrectangle{\pgfqpoint{1.254980in}{0.150000in}}{\pgfqpoint{5.490039in}{5.490039in}}%
\pgfusepath{clip}%
\pgfsetbuttcap%
\pgfsetroundjoin%
\definecolor{currentfill}{rgb}{0.525776,0.833491,0.288127}%
\pgfsetfillcolor{currentfill}%
\pgfsetfillopacity{0.700000}%
\pgfsetlinewidth{0.000000pt}%
\definecolor{currentstroke}{rgb}{0.000000,0.000000,0.000000}%
\pgfsetstrokecolor{currentstroke}%
\pgfsetdash{}{0pt}%
\pgfpathmoveto{\pgfqpoint{2.046207in}{3.448339in}}%
\pgfpathlineto{\pgfqpoint{2.060407in}{3.420374in}}%
\pgfpathlineto{\pgfqpoint{2.074592in}{3.392639in}}%
\pgfpathlineto{\pgfqpoint{2.088762in}{3.365132in}}%
\pgfpathlineto{\pgfqpoint{2.102918in}{3.337851in}}%
\pgfpathlineto{\pgfqpoint{2.112357in}{3.325199in}}%
\pgfpathlineto{\pgfqpoint{2.121768in}{3.312912in}}%
\pgfpathlineto{\pgfqpoint{2.131153in}{3.300986in}}%
\pgfpathlineto{\pgfqpoint{2.140510in}{3.289415in}}%
\pgfpathlineto{\pgfqpoint{2.126423in}{3.316126in}}%
\pgfpathlineto{\pgfqpoint{2.112323in}{3.343060in}}%
\pgfpathlineto{\pgfqpoint{2.098209in}{3.370221in}}%
\pgfpathlineto{\pgfqpoint{2.084079in}{3.397611in}}%
\pgfpathlineto{\pgfqpoint{2.074653in}{3.409744in}}%
\pgfpathlineto{\pgfqpoint{2.065199in}{3.422240in}}%
\pgfpathlineto{\pgfqpoint{2.055717in}{3.435103in}}%
\pgfpathlineto{\pgfqpoint{2.046207in}{3.448339in}}%
\pgfpathclose%
\pgfusepath{fill}%
\end{pgfscope}%
\begin{pgfscope}%
\pgfpathrectangle{\pgfqpoint{1.254980in}{0.150000in}}{\pgfqpoint{5.490039in}{5.490039in}}%
\pgfusepath{clip}%
\pgfsetbuttcap%
\pgfsetroundjoin%
\definecolor{currentfill}{rgb}{0.279566,0.067836,0.391917}%
\pgfsetfillcolor{currentfill}%
\pgfsetfillopacity{0.700000}%
\pgfsetlinewidth{0.000000pt}%
\definecolor{currentstroke}{rgb}{0.000000,0.000000,0.000000}%
\pgfsetstrokecolor{currentstroke}%
\pgfsetdash{}{0pt}%
\pgfpathmoveto{\pgfqpoint{3.662571in}{1.373370in}}%
\pgfpathlineto{\pgfqpoint{3.676228in}{1.364444in}}%
\pgfpathlineto{\pgfqpoint{3.689888in}{1.355639in}}%
\pgfpathlineto{\pgfqpoint{3.703551in}{1.346956in}}%
\pgfpathlineto{\pgfqpoint{3.717217in}{1.338395in}}%
\pgfpathlineto{\pgfqpoint{3.725316in}{1.340361in}}%
\pgfpathlineto{\pgfqpoint{3.733407in}{1.342554in}}%
\pgfpathlineto{\pgfqpoint{3.741487in}{1.344969in}}%
\pgfpathlineto{\pgfqpoint{3.749558in}{1.347603in}}%
\pgfpathlineto{\pgfqpoint{3.735918in}{1.355671in}}%
\pgfpathlineto{\pgfqpoint{3.722281in}{1.363860in}}%
\pgfpathlineto{\pgfqpoint{3.708648in}{1.372171in}}%
\pgfpathlineto{\pgfqpoint{3.695018in}{1.380603in}}%
\pgfpathlineto{\pgfqpoint{3.686921in}{1.378456in}}%
\pgfpathlineto{\pgfqpoint{3.678815in}{1.376532in}}%
\pgfpathlineto{\pgfqpoint{3.670698in}{1.374836in}}%
\pgfpathlineto{\pgfqpoint{3.662571in}{1.373370in}}%
\pgfpathclose%
\pgfusepath{fill}%
\end{pgfscope}%
\begin{pgfscope}%
\pgfpathrectangle{\pgfqpoint{1.254980in}{0.150000in}}{\pgfqpoint{5.490039in}{5.490039in}}%
\pgfusepath{clip}%
\pgfsetbuttcap%
\pgfsetroundjoin%
\definecolor{currentfill}{rgb}{0.267004,0.004874,0.329415}%
\pgfsetfillcolor{currentfill}%
\pgfsetfillopacity{0.700000}%
\pgfsetlinewidth{0.000000pt}%
\definecolor{currentstroke}{rgb}{0.000000,0.000000,0.000000}%
\pgfsetstrokecolor{currentstroke}%
\pgfsetdash{}{0pt}%
\pgfpathmoveto{\pgfqpoint{4.141492in}{1.240193in}}%
\pgfpathlineto{\pgfqpoint{4.155218in}{1.235976in}}%
\pgfpathlineto{\pgfqpoint{4.168951in}{1.231873in}}%
\pgfpathlineto{\pgfqpoint{4.182691in}{1.227883in}}%
\pgfpathlineto{\pgfqpoint{4.196437in}{1.224006in}}%
\pgfpathlineto{\pgfqpoint{4.204306in}{1.231546in}}%
\pgfpathlineto{\pgfqpoint{4.212169in}{1.239235in}}%
\pgfpathlineto{\pgfqpoint{4.220027in}{1.247070in}}%
\pgfpathlineto{\pgfqpoint{4.227879in}{1.255047in}}%
\pgfpathlineto{\pgfqpoint{4.214145in}{1.258484in}}%
\pgfpathlineto{\pgfqpoint{4.200419in}{1.262034in}}%
\pgfpathlineto{\pgfqpoint{4.186700in}{1.265697in}}%
\pgfpathlineto{\pgfqpoint{4.172988in}{1.269474in}}%
\pgfpathlineto{\pgfqpoint{4.165123in}{1.261931in}}%
\pgfpathlineto{\pgfqpoint{4.157252in}{1.254535in}}%
\pgfpathlineto{\pgfqpoint{4.149375in}{1.247287in}}%
\pgfpathlineto{\pgfqpoint{4.141492in}{1.240193in}}%
\pgfpathclose%
\pgfusepath{fill}%
\end{pgfscope}%
\begin{pgfscope}%
\pgfpathrectangle{\pgfqpoint{1.254980in}{0.150000in}}{\pgfqpoint{5.490039in}{5.490039in}}%
\pgfusepath{clip}%
\pgfsetbuttcap%
\pgfsetroundjoin%
\definecolor{currentfill}{rgb}{0.271305,0.019942,0.347269}%
\pgfsetfillcolor{currentfill}%
\pgfsetfillopacity{0.700000}%
\pgfsetlinewidth{0.000000pt}%
\definecolor{currentstroke}{rgb}{0.000000,0.000000,0.000000}%
\pgfsetstrokecolor{currentstroke}%
\pgfsetdash{}{0pt}%
\pgfpathmoveto{\pgfqpoint{3.858811in}{1.287370in}}%
\pgfpathlineto{\pgfqpoint{3.872486in}{1.280375in}}%
\pgfpathlineto{\pgfqpoint{3.886165in}{1.273498in}}%
\pgfpathlineto{\pgfqpoint{3.899849in}{1.266738in}}%
\pgfpathlineto{\pgfqpoint{3.913537in}{1.260095in}}%
\pgfpathlineto{\pgfqpoint{3.921530in}{1.264387in}}%
\pgfpathlineto{\pgfqpoint{3.929515in}{1.268876in}}%
\pgfpathlineto{\pgfqpoint{3.937492in}{1.273557in}}%
\pgfpathlineto{\pgfqpoint{3.945461in}{1.278428in}}%
\pgfpathlineto{\pgfqpoint{3.931793in}{1.284596in}}%
\pgfpathlineto{\pgfqpoint{3.918130in}{1.290882in}}%
\pgfpathlineto{\pgfqpoint{3.904472in}{1.297285in}}%
\pgfpathlineto{\pgfqpoint{3.890818in}{1.303806in}}%
\pgfpathlineto{\pgfqpoint{3.882829in}{1.299403in}}%
\pgfpathlineto{\pgfqpoint{3.874831in}{1.295194in}}%
\pgfpathlineto{\pgfqpoint{3.866825in}{1.291181in}}%
\pgfpathlineto{\pgfqpoint{3.858811in}{1.287370in}}%
\pgfpathclose%
\pgfusepath{fill}%
\end{pgfscope}%
\begin{pgfscope}%
\pgfpathrectangle{\pgfqpoint{1.254980in}{0.150000in}}{\pgfqpoint{5.490039in}{5.490039in}}%
\pgfusepath{clip}%
\pgfsetbuttcap%
\pgfsetroundjoin%
\definecolor{currentfill}{rgb}{0.262138,0.242286,0.520837}%
\pgfsetfillcolor{currentfill}%
\pgfsetfillopacity{0.700000}%
\pgfsetlinewidth{0.000000pt}%
\definecolor{currentstroke}{rgb}{0.000000,0.000000,0.000000}%
\pgfsetstrokecolor{currentstroke}%
\pgfsetdash{}{0pt}%
\pgfpathmoveto{\pgfqpoint{5.005828in}{1.676713in}}%
\pgfpathlineto{\pgfqpoint{5.019901in}{1.680641in}}%
\pgfpathlineto{\pgfqpoint{5.033986in}{1.684679in}}%
\pgfpathlineto{\pgfqpoint{5.048085in}{1.688828in}}%
\pgfpathlineto{\pgfqpoint{5.062197in}{1.693087in}}%
\pgfpathlineto{\pgfqpoint{5.069851in}{1.707007in}}%
\pgfpathlineto{\pgfqpoint{5.077501in}{1.720921in}}%
\pgfpathlineto{\pgfqpoint{5.085146in}{1.734824in}}%
\pgfpathlineto{\pgfqpoint{5.092787in}{1.748716in}}%
\pgfpathlineto{\pgfqpoint{5.078672in}{1.744170in}}%
\pgfpathlineto{\pgfqpoint{5.064571in}{1.739735in}}%
\pgfpathlineto{\pgfqpoint{5.050484in}{1.735410in}}%
\pgfpathlineto{\pgfqpoint{5.036409in}{1.731197in}}%
\pgfpathlineto{\pgfqpoint{5.028770in}{1.717585in}}%
\pgfpathlineto{\pgfqpoint{5.021127in}{1.703966in}}%
\pgfpathlineto{\pgfqpoint{5.013480in}{1.690341in}}%
\pgfpathlineto{\pgfqpoint{5.005828in}{1.676713in}}%
\pgfpathclose%
\pgfusepath{fill}%
\end{pgfscope}%
\begin{pgfscope}%
\pgfpathrectangle{\pgfqpoint{1.254980in}{0.150000in}}{\pgfqpoint{5.490039in}{5.490039in}}%
\pgfusepath{clip}%
\pgfsetbuttcap%
\pgfsetroundjoin%
\definecolor{currentfill}{rgb}{0.283187,0.125848,0.444960}%
\pgfsetfillcolor{currentfill}%
\pgfsetfillopacity{0.700000}%
\pgfsetlinewidth{0.000000pt}%
\definecolor{currentstroke}{rgb}{0.000000,0.000000,0.000000}%
\pgfsetstrokecolor{currentstroke}%
\pgfsetdash{}{0pt}%
\pgfpathmoveto{\pgfqpoint{4.714995in}{1.437197in}}%
\pgfpathlineto{\pgfqpoint{4.728922in}{1.438503in}}%
\pgfpathlineto{\pgfqpoint{4.742860in}{1.439919in}}%
\pgfpathlineto{\pgfqpoint{4.756809in}{1.441446in}}%
\pgfpathlineto{\pgfqpoint{4.770770in}{1.443082in}}%
\pgfpathlineto{\pgfqpoint{4.778483in}{1.455729in}}%
\pgfpathlineto{\pgfqpoint{4.786192in}{1.468420in}}%
\pgfpathlineto{\pgfqpoint{4.793898in}{1.481153in}}%
\pgfpathlineto{\pgfqpoint{4.801599in}{1.493924in}}%
\pgfpathlineto{\pgfqpoint{4.787640in}{1.491940in}}%
\pgfpathlineto{\pgfqpoint{4.773691in}{1.490066in}}%
\pgfpathlineto{\pgfqpoint{4.759755in}{1.488302in}}%
\pgfpathlineto{\pgfqpoint{4.745829in}{1.486649in}}%
\pgfpathlineto{\pgfqpoint{4.738127in}{1.474219in}}%
\pgfpathlineto{\pgfqpoint{4.730420in}{1.461832in}}%
\pgfpathlineto{\pgfqpoint{4.722709in}{1.449490in}}%
\pgfpathlineto{\pgfqpoint{4.714995in}{1.437197in}}%
\pgfpathclose%
\pgfusepath{fill}%
\end{pgfscope}%
\begin{pgfscope}%
\pgfpathrectangle{\pgfqpoint{1.254980in}{0.150000in}}{\pgfqpoint{5.490039in}{5.490039in}}%
\pgfusepath{clip}%
\pgfsetbuttcap%
\pgfsetroundjoin%
\definecolor{currentfill}{rgb}{0.272594,0.025563,0.353093}%
\pgfsetfillcolor{currentfill}%
\pgfsetfillopacity{0.700000}%
\pgfsetlinewidth{0.000000pt}%
\definecolor{currentstroke}{rgb}{0.000000,0.000000,0.000000}%
\pgfsetstrokecolor{currentstroke}%
\pgfsetdash{}{0pt}%
\pgfpathmoveto{\pgfqpoint{4.369280in}{1.268236in}}%
\pgfpathlineto{\pgfqpoint{4.383073in}{1.266231in}}%
\pgfpathlineto{\pgfqpoint{4.396874in}{1.264337in}}%
\pgfpathlineto{\pgfqpoint{4.410684in}{1.262555in}}%
\pgfpathlineto{\pgfqpoint{4.424502in}{1.260883in}}%
\pgfpathlineto{\pgfqpoint{4.432298in}{1.270757in}}%
\pgfpathlineto{\pgfqpoint{4.440089in}{1.280741in}}%
\pgfpathlineto{\pgfqpoint{4.447876in}{1.290830in}}%
\pgfpathlineto{\pgfqpoint{4.455658in}{1.301021in}}%
\pgfpathlineto{\pgfqpoint{4.441847in}{1.302284in}}%
\pgfpathlineto{\pgfqpoint{4.428045in}{1.303658in}}%
\pgfpathlineto{\pgfqpoint{4.414252in}{1.305143in}}%
\pgfpathlineto{\pgfqpoint{4.400468in}{1.306740in}}%
\pgfpathlineto{\pgfqpoint{4.392678in}{1.296951in}}%
\pgfpathlineto{\pgfqpoint{4.384884in}{1.287269in}}%
\pgfpathlineto{\pgfqpoint{4.377085in}{1.277696in}}%
\pgfpathlineto{\pgfqpoint{4.369280in}{1.268236in}}%
\pgfpathclose%
\pgfusepath{fill}%
\end{pgfscope}%
\begin{pgfscope}%
\pgfpathrectangle{\pgfqpoint{1.254980in}{0.150000in}}{\pgfqpoint{5.490039in}{5.490039in}}%
\pgfusepath{clip}%
\pgfsetbuttcap%
\pgfsetroundjoin%
\definecolor{currentfill}{rgb}{0.122606,0.585371,0.546557}%
\pgfsetfillcolor{currentfill}%
\pgfsetfillopacity{0.700000}%
\pgfsetlinewidth{0.000000pt}%
\definecolor{currentstroke}{rgb}{0.000000,0.000000,0.000000}%
\pgfsetstrokecolor{currentstroke}%
\pgfsetdash{}{0pt}%
\pgfpathmoveto{\pgfqpoint{2.477221in}{2.652736in}}%
\pgfpathlineto{\pgfqpoint{2.491133in}{2.630961in}}%
\pgfpathlineto{\pgfqpoint{2.505036in}{2.609367in}}%
\pgfpathlineto{\pgfqpoint{2.518931in}{2.587953in}}%
\pgfpathlineto{\pgfqpoint{2.532817in}{2.566718in}}%
\pgfpathlineto{\pgfqpoint{2.541875in}{2.556367in}}%
\pgfpathlineto{\pgfqpoint{2.550910in}{2.546369in}}%
\pgfpathlineto{\pgfqpoint{2.559922in}{2.536718in}}%
\pgfpathlineto{\pgfqpoint{2.568911in}{2.527409in}}%
\pgfpathlineto{\pgfqpoint{2.555084in}{2.548075in}}%
\pgfpathlineto{\pgfqpoint{2.541249in}{2.568919in}}%
\pgfpathlineto{\pgfqpoint{2.527406in}{2.589943in}}%
\pgfpathlineto{\pgfqpoint{2.513554in}{2.611146in}}%
\pgfpathlineto{\pgfqpoint{2.504507in}{2.621016in}}%
\pgfpathlineto{\pgfqpoint{2.495435in}{2.631235in}}%
\pgfpathlineto{\pgfqpoint{2.486340in}{2.641806in}}%
\pgfpathlineto{\pgfqpoint{2.477221in}{2.652736in}}%
\pgfpathclose%
\pgfusepath{fill}%
\end{pgfscope}%
\begin{pgfscope}%
\pgfpathrectangle{\pgfqpoint{1.254980in}{0.150000in}}{\pgfqpoint{5.490039in}{5.490039in}}%
\pgfusepath{clip}%
\pgfsetbuttcap%
\pgfsetroundjoin%
\definecolor{currentfill}{rgb}{0.281887,0.150881,0.465405}%
\pgfsetfillcolor{currentfill}%
\pgfsetfillopacity{0.700000}%
\pgfsetlinewidth{0.000000pt}%
\definecolor{currentstroke}{rgb}{0.000000,0.000000,0.000000}%
\pgfsetstrokecolor{currentstroke}%
\pgfsetdash{}{0pt}%
\pgfpathmoveto{\pgfqpoint{3.411229in}{1.537888in}}%
\pgfpathlineto{\pgfqpoint{3.424890in}{1.526437in}}%
\pgfpathlineto{\pgfqpoint{3.438552in}{1.515115in}}%
\pgfpathlineto{\pgfqpoint{3.452214in}{1.503922in}}%
\pgfpathlineto{\pgfqpoint{3.465878in}{1.492857in}}%
\pgfpathlineto{\pgfqpoint{3.474144in}{1.491788in}}%
\pgfpathlineto{\pgfqpoint{3.482397in}{1.490982in}}%
\pgfpathlineto{\pgfqpoint{3.490638in}{1.490435in}}%
\pgfpathlineto{\pgfqpoint{3.498866in}{1.490141in}}%
\pgfpathlineto{\pgfqpoint{3.485236in}{1.500689in}}%
\pgfpathlineto{\pgfqpoint{3.471607in}{1.511364in}}%
\pgfpathlineto{\pgfqpoint{3.457980in}{1.522168in}}%
\pgfpathlineto{\pgfqpoint{3.444353in}{1.533101in}}%
\pgfpathlineto{\pgfqpoint{3.436091in}{1.533905in}}%
\pgfpathlineto{\pgfqpoint{3.427817in}{1.534968in}}%
\pgfpathlineto{\pgfqpoint{3.419529in}{1.536294in}}%
\pgfpathlineto{\pgfqpoint{3.411229in}{1.537888in}}%
\pgfpathclose%
\pgfusepath{fill}%
\end{pgfscope}%
\begin{pgfscope}%
\pgfpathrectangle{\pgfqpoint{1.254980in}{0.150000in}}{\pgfqpoint{5.490039in}{5.490039in}}%
\pgfusepath{clip}%
\pgfsetbuttcap%
\pgfsetroundjoin%
\definecolor{currentfill}{rgb}{0.281412,0.155834,0.469201}%
\pgfsetfillcolor{currentfill}%
\pgfsetfillopacity{0.700000}%
\pgfsetlinewidth{0.000000pt}%
\definecolor{currentstroke}{rgb}{0.000000,0.000000,0.000000}%
\pgfsetstrokecolor{currentstroke}%
\pgfsetdash{}{0pt}%
\pgfpathmoveto{\pgfqpoint{4.801599in}{1.493924in}}%
\pgfpathlineto{\pgfqpoint{4.815570in}{1.496018in}}%
\pgfpathlineto{\pgfqpoint{4.829553in}{1.498222in}}%
\pgfpathlineto{\pgfqpoint{4.843548in}{1.500537in}}%
\pgfpathlineto{\pgfqpoint{4.857554in}{1.502961in}}%
\pgfpathlineto{\pgfqpoint{4.865252in}{1.516107in}}%
\pgfpathlineto{\pgfqpoint{4.872946in}{1.529282in}}%
\pgfpathlineto{\pgfqpoint{4.880636in}{1.542485in}}%
\pgfpathlineto{\pgfqpoint{4.888322in}{1.555712in}}%
\pgfpathlineto{\pgfqpoint{4.874315in}{1.552955in}}%
\pgfpathlineto{\pgfqpoint{4.860320in}{1.550308in}}%
\pgfpathlineto{\pgfqpoint{4.846337in}{1.547771in}}%
\pgfpathlineto{\pgfqpoint{4.832367in}{1.545345in}}%
\pgfpathlineto{\pgfqpoint{4.824681in}{1.532445in}}%
\pgfpathlineto{\pgfqpoint{4.816991in}{1.519573in}}%
\pgfpathlineto{\pgfqpoint{4.809297in}{1.506732in}}%
\pgfpathlineto{\pgfqpoint{4.801599in}{1.493924in}}%
\pgfpathclose%
\pgfusepath{fill}%
\end{pgfscope}%
\begin{pgfscope}%
\pgfpathrectangle{\pgfqpoint{1.254980in}{0.150000in}}{\pgfqpoint{5.490039in}{5.490039in}}%
\pgfusepath{clip}%
\pgfsetbuttcap%
\pgfsetroundjoin%
\definecolor{currentfill}{rgb}{0.269944,0.014625,0.341379}%
\pgfsetfillcolor{currentfill}%
\pgfsetfillopacity{0.700000}%
\pgfsetlinewidth{0.000000pt}%
\definecolor{currentstroke}{rgb}{0.000000,0.000000,0.000000}%
\pgfsetstrokecolor{currentstroke}%
\pgfsetdash{}{0pt}%
\pgfpathmoveto{\pgfqpoint{4.282885in}{1.242423in}}%
\pgfpathlineto{\pgfqpoint{4.296656in}{1.239548in}}%
\pgfpathlineto{\pgfqpoint{4.310434in}{1.236784in}}%
\pgfpathlineto{\pgfqpoint{4.324220in}{1.234132in}}%
\pgfpathlineto{\pgfqpoint{4.338014in}{1.231592in}}%
\pgfpathlineto{\pgfqpoint{4.345838in}{1.240566in}}%
\pgfpathlineto{\pgfqpoint{4.353657in}{1.249668in}}%
\pgfpathlineto{\pgfqpoint{4.361471in}{1.258892in}}%
\pgfpathlineto{\pgfqpoint{4.369280in}{1.268236in}}%
\pgfpathlineto{\pgfqpoint{4.355496in}{1.270352in}}%
\pgfpathlineto{\pgfqpoint{4.341720in}{1.272580in}}%
\pgfpathlineto{\pgfqpoint{4.327952in}{1.274919in}}%
\pgfpathlineto{\pgfqpoint{4.314192in}{1.277371in}}%
\pgfpathlineto{\pgfqpoint{4.306373in}{1.268445in}}%
\pgfpathlineto{\pgfqpoint{4.298549in}{1.259643in}}%
\pgfpathlineto{\pgfqpoint{4.290720in}{1.250968in}}%
\pgfpathlineto{\pgfqpoint{4.282885in}{1.242423in}}%
\pgfpathclose%
\pgfusepath{fill}%
\end{pgfscope}%
\begin{pgfscope}%
\pgfpathrectangle{\pgfqpoint{1.254980in}{0.150000in}}{\pgfqpoint{5.490039in}{5.490039in}}%
\pgfusepath{clip}%
\pgfsetbuttcap%
\pgfsetroundjoin%
\definecolor{currentfill}{rgb}{0.283072,0.130895,0.449241}%
\pgfsetfillcolor{currentfill}%
\pgfsetfillopacity{0.700000}%
\pgfsetlinewidth{0.000000pt}%
\definecolor{currentstroke}{rgb}{0.000000,0.000000,0.000000}%
\pgfsetstrokecolor{currentstroke}%
\pgfsetdash{}{0pt}%
\pgfpathmoveto{\pgfqpoint{3.465878in}{1.492857in}}%
\pgfpathlineto{\pgfqpoint{3.479542in}{1.481920in}}%
\pgfpathlineto{\pgfqpoint{3.493208in}{1.471110in}}%
\pgfpathlineto{\pgfqpoint{3.506875in}{1.460427in}}%
\pgfpathlineto{\pgfqpoint{3.520544in}{1.449870in}}%
\pgfpathlineto{\pgfqpoint{3.528776in}{1.449324in}}%
\pgfpathlineto{\pgfqpoint{3.536997in}{1.449036in}}%
\pgfpathlineto{\pgfqpoint{3.545205in}{1.449002in}}%
\pgfpathlineto{\pgfqpoint{3.553402in}{1.449218in}}%
\pgfpathlineto{\pgfqpoint{3.539766in}{1.459259in}}%
\pgfpathlineto{\pgfqpoint{3.526131in}{1.469427in}}%
\pgfpathlineto{\pgfqpoint{3.512498in}{1.479720in}}%
\pgfpathlineto{\pgfqpoint{3.498866in}{1.490141in}}%
\pgfpathlineto{\pgfqpoint{3.490638in}{1.490435in}}%
\pgfpathlineto{\pgfqpoint{3.482397in}{1.490982in}}%
\pgfpathlineto{\pgfqpoint{3.474144in}{1.491788in}}%
\pgfpathlineto{\pgfqpoint{3.465878in}{1.492857in}}%
\pgfpathclose%
\pgfusepath{fill}%
\end{pgfscope}%
\begin{pgfscope}%
\pgfpathrectangle{\pgfqpoint{1.254980in}{0.150000in}}{\pgfqpoint{5.490039in}{5.490039in}}%
\pgfusepath{clip}%
\pgfsetbuttcap%
\pgfsetroundjoin%
\definecolor{currentfill}{rgb}{0.277941,0.056324,0.381191}%
\pgfsetfillcolor{currentfill}%
\pgfsetfillopacity{0.700000}%
\pgfsetlinewidth{0.000000pt}%
\definecolor{currentstroke}{rgb}{0.000000,0.000000,0.000000}%
\pgfsetstrokecolor{currentstroke}%
\pgfsetdash{}{0pt}%
\pgfpathmoveto{\pgfqpoint{3.717217in}{1.338395in}}%
\pgfpathlineto{\pgfqpoint{3.730886in}{1.329954in}}%
\pgfpathlineto{\pgfqpoint{3.744558in}{1.321633in}}%
\pgfpathlineto{\pgfqpoint{3.758234in}{1.313433in}}%
\pgfpathlineto{\pgfqpoint{3.771913in}{1.305352in}}%
\pgfpathlineto{\pgfqpoint{3.779987in}{1.307818in}}%
\pgfpathlineto{\pgfqpoint{3.788051in}{1.310506in}}%
\pgfpathlineto{\pgfqpoint{3.796107in}{1.313412in}}%
\pgfpathlineto{\pgfqpoint{3.804153in}{1.316533in}}%
\pgfpathlineto{\pgfqpoint{3.790499in}{1.324120in}}%
\pgfpathlineto{\pgfqpoint{3.776848in}{1.331828in}}%
\pgfpathlineto{\pgfqpoint{3.763201in}{1.339655in}}%
\pgfpathlineto{\pgfqpoint{3.749558in}{1.347603in}}%
\pgfpathlineto{\pgfqpoint{3.741487in}{1.344969in}}%
\pgfpathlineto{\pgfqpoint{3.733407in}{1.342554in}}%
\pgfpathlineto{\pgfqpoint{3.725316in}{1.340361in}}%
\pgfpathlineto{\pgfqpoint{3.717217in}{1.338395in}}%
\pgfpathclose%
\pgfusepath{fill}%
\end{pgfscope}%
\begin{pgfscope}%
\pgfpathrectangle{\pgfqpoint{1.254980in}{0.150000in}}{\pgfqpoint{5.490039in}{5.490039in}}%
\pgfusepath{clip}%
\pgfsetbuttcap%
\pgfsetroundjoin%
\definecolor{currentfill}{rgb}{0.248629,0.278775,0.534556}%
\pgfsetfillcolor{currentfill}%
\pgfsetfillopacity{0.700000}%
\pgfsetlinewidth{0.000000pt}%
\definecolor{currentstroke}{rgb}{0.000000,0.000000,0.000000}%
\pgfsetstrokecolor{currentstroke}%
\pgfsetdash{}{0pt}%
\pgfpathmoveto{\pgfqpoint{5.092787in}{1.748716in}}%
\pgfpathlineto{\pgfqpoint{5.106915in}{1.753373in}}%
\pgfpathlineto{\pgfqpoint{5.121057in}{1.758140in}}%
\pgfpathlineto{\pgfqpoint{5.135213in}{1.763018in}}%
\pgfpathlineto{\pgfqpoint{5.149383in}{1.768006in}}%
\pgfpathlineto{\pgfqpoint{5.157023in}{1.782162in}}%
\pgfpathlineto{\pgfqpoint{5.164658in}{1.796297in}}%
\pgfpathlineto{\pgfqpoint{5.172290in}{1.810411in}}%
\pgfpathlineto{\pgfqpoint{5.179916in}{1.824500in}}%
\pgfpathlineto{\pgfqpoint{5.165743in}{1.819240in}}%
\pgfpathlineto{\pgfqpoint{5.151584in}{1.814090in}}%
\pgfpathlineto{\pgfqpoint{5.137439in}{1.809052in}}%
\pgfpathlineto{\pgfqpoint{5.123309in}{1.804124in}}%
\pgfpathlineto{\pgfqpoint{5.115685in}{1.790300in}}%
\pgfpathlineto{\pgfqpoint{5.108056in}{1.776456in}}%
\pgfpathlineto{\pgfqpoint{5.100424in}{1.762594in}}%
\pgfpathlineto{\pgfqpoint{5.092787in}{1.748716in}}%
\pgfpathclose%
\pgfusepath{fill}%
\end{pgfscope}%
\begin{pgfscope}%
\pgfpathrectangle{\pgfqpoint{1.254980in}{0.150000in}}{\pgfqpoint{5.490039in}{5.490039in}}%
\pgfusepath{clip}%
\pgfsetbuttcap%
\pgfsetroundjoin%
\definecolor{currentfill}{rgb}{0.304148,0.764704,0.419943}%
\pgfsetfillcolor{currentfill}%
\pgfsetfillopacity{0.700000}%
\pgfsetlinewidth{0.000000pt}%
\definecolor{currentstroke}{rgb}{0.000000,0.000000,0.000000}%
\pgfsetstrokecolor{currentstroke}%
\pgfsetdash{}{0pt}%
\pgfpathmoveto{\pgfqpoint{2.178191in}{3.177214in}}%
\pgfpathlineto{\pgfqpoint{2.192295in}{3.151313in}}%
\pgfpathlineto{\pgfqpoint{2.206386in}{3.125625in}}%
\pgfpathlineto{\pgfqpoint{2.220464in}{3.100146in}}%
\pgfpathlineto{\pgfqpoint{2.234529in}{3.074876in}}%
\pgfpathlineto{\pgfqpoint{2.243871in}{3.062493in}}%
\pgfpathlineto{\pgfqpoint{2.253185in}{3.050477in}}%
\pgfpathlineto{\pgfqpoint{2.262473in}{3.038823in}}%
\pgfpathlineto{\pgfqpoint{2.271736in}{3.027527in}}%
\pgfpathlineto{\pgfqpoint{2.257737in}{3.052220in}}%
\pgfpathlineto{\pgfqpoint{2.243726in}{3.077121in}}%
\pgfpathlineto{\pgfqpoint{2.229703in}{3.102230in}}%
\pgfpathlineto{\pgfqpoint{2.215668in}{3.127549in}}%
\pgfpathlineto{\pgfqpoint{2.206339in}{3.139414in}}%
\pgfpathlineto{\pgfqpoint{2.196984in}{3.151644in}}%
\pgfpathlineto{\pgfqpoint{2.187601in}{3.164242in}}%
\pgfpathlineto{\pgfqpoint{2.178191in}{3.177214in}}%
\pgfpathclose%
\pgfusepath{fill}%
\end{pgfscope}%
\begin{pgfscope}%
\pgfpathrectangle{\pgfqpoint{1.254980in}{0.150000in}}{\pgfqpoint{5.490039in}{5.490039in}}%
\pgfusepath{clip}%
\pgfsetbuttcap%
\pgfsetroundjoin%
\definecolor{currentfill}{rgb}{0.119699,0.618490,0.536347}%
\pgfsetfillcolor{currentfill}%
\pgfsetfillopacity{0.700000}%
\pgfsetlinewidth{0.000000pt}%
\definecolor{currentstroke}{rgb}{0.000000,0.000000,0.000000}%
\pgfsetstrokecolor{currentstroke}%
\pgfsetdash{}{0pt}%
\pgfpathmoveto{\pgfqpoint{2.421484in}{2.741674in}}%
\pgfpathlineto{\pgfqpoint{2.435432in}{2.719162in}}%
\pgfpathlineto{\pgfqpoint{2.449371in}{2.696836in}}%
\pgfpathlineto{\pgfqpoint{2.463300in}{2.674694in}}%
\pgfpathlineto{\pgfqpoint{2.477221in}{2.652736in}}%
\pgfpathlineto{\pgfqpoint{2.486340in}{2.641806in}}%
\pgfpathlineto{\pgfqpoint{2.495435in}{2.631235in}}%
\pgfpathlineto{\pgfqpoint{2.504507in}{2.621016in}}%
\pgfpathlineto{\pgfqpoint{2.513554in}{2.611146in}}%
\pgfpathlineto{\pgfqpoint{2.499695in}{2.632530in}}%
\pgfpathlineto{\pgfqpoint{2.485827in}{2.654097in}}%
\pgfpathlineto{\pgfqpoint{2.471950in}{2.675847in}}%
\pgfpathlineto{\pgfqpoint{2.458064in}{2.697782in}}%
\pgfpathlineto{\pgfqpoint{2.448956in}{2.708219in}}%
\pgfpathlineto{\pgfqpoint{2.439823in}{2.719010in}}%
\pgfpathlineto{\pgfqpoint{2.430666in}{2.730160in}}%
\pgfpathlineto{\pgfqpoint{2.421484in}{2.741674in}}%
\pgfpathclose%
\pgfusepath{fill}%
\end{pgfscope}%
\begin{pgfscope}%
\pgfpathrectangle{\pgfqpoint{1.254980in}{0.150000in}}{\pgfqpoint{5.490039in}{5.490039in}}%
\pgfusepath{clip}%
\pgfsetbuttcap%
\pgfsetroundjoin%
\definecolor{currentfill}{rgb}{0.276194,0.190074,0.493001}%
\pgfsetfillcolor{currentfill}%
\pgfsetfillopacity{0.700000}%
\pgfsetlinewidth{0.000000pt}%
\definecolor{currentstroke}{rgb}{0.000000,0.000000,0.000000}%
\pgfsetstrokecolor{currentstroke}%
\pgfsetdash{}{0pt}%
\pgfpathmoveto{\pgfqpoint{4.888322in}{1.555712in}}%
\pgfpathlineto{\pgfqpoint{4.902341in}{1.558579in}}%
\pgfpathlineto{\pgfqpoint{4.916372in}{1.561556in}}%
\pgfpathlineto{\pgfqpoint{4.930416in}{1.564644in}}%
\pgfpathlineto{\pgfqpoint{4.944472in}{1.567841in}}%
\pgfpathlineto{\pgfqpoint{4.952156in}{1.581413in}}%
\pgfpathlineto{\pgfqpoint{4.959835in}{1.595001in}}%
\pgfpathlineto{\pgfqpoint{4.967510in}{1.608601in}}%
\pgfpathlineto{\pgfqpoint{4.975182in}{1.622212in}}%
\pgfpathlineto{\pgfqpoint{4.961124in}{1.618697in}}%
\pgfpathlineto{\pgfqpoint{4.947079in}{1.615292in}}%
\pgfpathlineto{\pgfqpoint{4.933047in}{1.611997in}}%
\pgfpathlineto{\pgfqpoint{4.919027in}{1.608813in}}%
\pgfpathlineto{\pgfqpoint{4.911357in}{1.595514in}}%
\pgfpathlineto{\pgfqpoint{4.903682in}{1.582229in}}%
\pgfpathlineto{\pgfqpoint{4.896004in}{1.568961in}}%
\pgfpathlineto{\pgfqpoint{4.888322in}{1.555712in}}%
\pgfpathclose%
\pgfusepath{fill}%
\end{pgfscope}%
\begin{pgfscope}%
\pgfpathrectangle{\pgfqpoint{1.254980in}{0.150000in}}{\pgfqpoint{5.490039in}{5.490039in}}%
\pgfusepath{clip}%
\pgfsetbuttcap%
\pgfsetroundjoin%
\definecolor{currentfill}{rgb}{0.267004,0.004874,0.329415}%
\pgfsetfillcolor{currentfill}%
\pgfsetfillopacity{0.700000}%
\pgfsetlinewidth{0.000000pt}%
\definecolor{currentstroke}{rgb}{0.000000,0.000000,0.000000}%
\pgfsetstrokecolor{currentstroke}%
\pgfsetdash{}{0pt}%
\pgfpathmoveto{\pgfqpoint{4.054993in}{1.233253in}}%
\pgfpathlineto{\pgfqpoint{4.068710in}{1.228125in}}%
\pgfpathlineto{\pgfqpoint{4.082433in}{1.223110in}}%
\pgfpathlineto{\pgfqpoint{4.096162in}{1.218210in}}%
\pgfpathlineto{\pgfqpoint{4.109897in}{1.213423in}}%
\pgfpathlineto{\pgfqpoint{4.117805in}{1.219867in}}%
\pgfpathlineto{\pgfqpoint{4.125707in}{1.226479in}}%
\pgfpathlineto{\pgfqpoint{4.133603in}{1.233256in}}%
\pgfpathlineto{\pgfqpoint{4.141492in}{1.240193in}}%
\pgfpathlineto{\pgfqpoint{4.127773in}{1.244523in}}%
\pgfpathlineto{\pgfqpoint{4.114060in}{1.248967in}}%
\pgfpathlineto{\pgfqpoint{4.100353in}{1.253525in}}%
\pgfpathlineto{\pgfqpoint{4.086653in}{1.258197in}}%
\pgfpathlineto{\pgfqpoint{4.078748in}{1.251710in}}%
\pgfpathlineto{\pgfqpoint{4.070837in}{1.245388in}}%
\pgfpathlineto{\pgfqpoint{4.062918in}{1.239235in}}%
\pgfpathlineto{\pgfqpoint{4.054993in}{1.233253in}}%
\pgfpathclose%
\pgfusepath{fill}%
\end{pgfscope}%
\begin{pgfscope}%
\pgfpathrectangle{\pgfqpoint{1.254980in}{0.150000in}}{\pgfqpoint{5.490039in}{5.490039in}}%
\pgfusepath{clip}%
\pgfsetbuttcap%
\pgfsetroundjoin%
\definecolor{currentfill}{rgb}{0.269944,0.014625,0.341379}%
\pgfsetfillcolor{currentfill}%
\pgfsetfillopacity{0.700000}%
\pgfsetlinewidth{0.000000pt}%
\definecolor{currentstroke}{rgb}{0.000000,0.000000,0.000000}%
\pgfsetstrokecolor{currentstroke}%
\pgfsetdash{}{0pt}%
\pgfpathmoveto{\pgfqpoint{3.913537in}{1.260095in}}%
\pgfpathlineto{\pgfqpoint{3.927230in}{1.253569in}}%
\pgfpathlineto{\pgfqpoint{3.940928in}{1.247160in}}%
\pgfpathlineto{\pgfqpoint{3.954631in}{1.240866in}}%
\pgfpathlineto{\pgfqpoint{3.968339in}{1.234689in}}%
\pgfpathlineto{\pgfqpoint{3.976312in}{1.239461in}}%
\pgfpathlineto{\pgfqpoint{3.984276in}{1.244425in}}%
\pgfpathlineto{\pgfqpoint{3.992234in}{1.249578in}}%
\pgfpathlineto{\pgfqpoint{4.000184in}{1.254916in}}%
\pgfpathlineto{\pgfqpoint{3.986495in}{1.260620in}}%
\pgfpathlineto{\pgfqpoint{3.972812in}{1.266440in}}%
\pgfpathlineto{\pgfqpoint{3.959134in}{1.272375in}}%
\pgfpathlineto{\pgfqpoint{3.945461in}{1.278428in}}%
\pgfpathlineto{\pgfqpoint{3.937492in}{1.273557in}}%
\pgfpathlineto{\pgfqpoint{3.929515in}{1.268876in}}%
\pgfpathlineto{\pgfqpoint{3.921530in}{1.264387in}}%
\pgfpathlineto{\pgfqpoint{3.913537in}{1.260095in}}%
\pgfpathclose%
\pgfusepath{fill}%
\end{pgfscope}%
\begin{pgfscope}%
\pgfpathrectangle{\pgfqpoint{1.254980in}{0.150000in}}{\pgfqpoint{5.490039in}{5.490039in}}%
\pgfusepath{clip}%
\pgfsetbuttcap%
\pgfsetroundjoin%
\definecolor{currentfill}{rgb}{0.267004,0.004874,0.329415}%
\pgfsetfillcolor{currentfill}%
\pgfsetfillopacity{0.700000}%
\pgfsetlinewidth{0.000000pt}%
\definecolor{currentstroke}{rgb}{0.000000,0.000000,0.000000}%
\pgfsetstrokecolor{currentstroke}%
\pgfsetdash{}{0pt}%
\pgfpathmoveto{\pgfqpoint{4.196437in}{1.224006in}}%
\pgfpathlineto{\pgfqpoint{4.210190in}{1.220242in}}%
\pgfpathlineto{\pgfqpoint{4.223950in}{1.216590in}}%
\pgfpathlineto{\pgfqpoint{4.237718in}{1.213050in}}%
\pgfpathlineto{\pgfqpoint{4.251492in}{1.209622in}}%
\pgfpathlineto{\pgfqpoint{4.259349in}{1.217609in}}%
\pgfpathlineto{\pgfqpoint{4.267200in}{1.225740in}}%
\pgfpathlineto{\pgfqpoint{4.275045in}{1.234013in}}%
\pgfpathlineto{\pgfqpoint{4.282885in}{1.242423in}}%
\pgfpathlineto{\pgfqpoint{4.269122in}{1.245411in}}%
\pgfpathlineto{\pgfqpoint{4.255367in}{1.248510in}}%
\pgfpathlineto{\pgfqpoint{4.241619in}{1.251722in}}%
\pgfpathlineto{\pgfqpoint{4.227879in}{1.255047in}}%
\pgfpathlineto{\pgfqpoint{4.220027in}{1.247070in}}%
\pgfpathlineto{\pgfqpoint{4.212169in}{1.239235in}}%
\pgfpathlineto{\pgfqpoint{4.204306in}{1.231546in}}%
\pgfpathlineto{\pgfqpoint{4.196437in}{1.224006in}}%
\pgfpathclose%
\pgfusepath{fill}%
\end{pgfscope}%
\begin{pgfscope}%
\pgfpathrectangle{\pgfqpoint{1.254980in}{0.150000in}}{\pgfqpoint{5.490039in}{5.490039in}}%
\pgfusepath{clip}%
\pgfsetbuttcap%
\pgfsetroundjoin%
\definecolor{currentfill}{rgb}{0.233603,0.313828,0.543914}%
\pgfsetfillcolor{currentfill}%
\pgfsetfillopacity{0.700000}%
\pgfsetlinewidth{0.000000pt}%
\definecolor{currentstroke}{rgb}{0.000000,0.000000,0.000000}%
\pgfsetstrokecolor{currentstroke}%
\pgfsetdash{}{0pt}%
\pgfpathmoveto{\pgfqpoint{5.179916in}{1.824500in}}%
\pgfpathlineto{\pgfqpoint{5.194104in}{1.829871in}}%
\pgfpathlineto{\pgfqpoint{5.208305in}{1.835353in}}%
\pgfpathlineto{\pgfqpoint{5.222522in}{1.840946in}}%
\pgfpathlineto{\pgfqpoint{5.230146in}{1.855206in}}%
\pgfpathlineto{\pgfqpoint{5.237767in}{1.869435in}}%
\pgfpathlineto{\pgfqpoint{5.245382in}{1.883631in}}%
\pgfpathlineto{\pgfqpoint{5.252993in}{1.897791in}}%
\pgfpathlineto{\pgfqpoint{5.238773in}{1.891942in}}%
\pgfpathlineto{\pgfqpoint{5.224568in}{1.886204in}}%
\pgfpathlineto{\pgfqpoint{5.210378in}{1.880577in}}%
\pgfpathlineto{\pgfqpoint{5.202769in}{1.866604in}}%
\pgfpathlineto{\pgfqpoint{5.195156in}{1.852598in}}%
\pgfpathlineto{\pgfqpoint{5.187538in}{1.838563in}}%
\pgfpathlineto{\pgfqpoint{5.179916in}{1.824500in}}%
\pgfpathclose%
\pgfusepath{fill}%
\end{pgfscope}%
\begin{pgfscope}%
\pgfpathrectangle{\pgfqpoint{1.254980in}{0.150000in}}{\pgfqpoint{5.490039in}{5.490039in}}%
\pgfusepath{clip}%
\pgfsetbuttcap%
\pgfsetroundjoin%
\definecolor{currentfill}{rgb}{0.283197,0.115680,0.436115}%
\pgfsetfillcolor{currentfill}%
\pgfsetfillopacity{0.700000}%
\pgfsetlinewidth{0.000000pt}%
\definecolor{currentstroke}{rgb}{0.000000,0.000000,0.000000}%
\pgfsetstrokecolor{currentstroke}%
\pgfsetdash{}{0pt}%
\pgfpathmoveto{\pgfqpoint{3.520544in}{1.449870in}}%
\pgfpathlineto{\pgfqpoint{3.534214in}{1.439439in}}%
\pgfpathlineto{\pgfqpoint{3.547886in}{1.429133in}}%
\pgfpathlineto{\pgfqpoint{3.561559in}{1.418952in}}%
\pgfpathlineto{\pgfqpoint{3.575234in}{1.408896in}}%
\pgfpathlineto{\pgfqpoint{3.583435in}{1.408872in}}%
\pgfpathlineto{\pgfqpoint{3.591624in}{1.409101in}}%
\pgfpathlineto{\pgfqpoint{3.599802in}{1.409580in}}%
\pgfpathlineto{\pgfqpoint{3.607968in}{1.410303in}}%
\pgfpathlineto{\pgfqpoint{3.594324in}{1.419845in}}%
\pgfpathlineto{\pgfqpoint{3.580681in}{1.429511in}}%
\pgfpathlineto{\pgfqpoint{3.567041in}{1.439302in}}%
\pgfpathlineto{\pgfqpoint{3.553402in}{1.449218in}}%
\pgfpathlineto{\pgfqpoint{3.545205in}{1.449002in}}%
\pgfpathlineto{\pgfqpoint{3.536997in}{1.449036in}}%
\pgfpathlineto{\pgfqpoint{3.528776in}{1.449324in}}%
\pgfpathlineto{\pgfqpoint{3.520544in}{1.449870in}}%
\pgfpathclose%
\pgfusepath{fill}%
\end{pgfscope}%
\begin{pgfscope}%
\pgfpathrectangle{\pgfqpoint{1.254980in}{0.150000in}}{\pgfqpoint{5.490039in}{5.490039in}}%
\pgfusepath{clip}%
\pgfsetbuttcap%
\pgfsetroundjoin%
\definecolor{currentfill}{rgb}{0.214298,0.355619,0.551184}%
\pgfsetfillcolor{currentfill}%
\pgfsetfillopacity{0.700000}%
\pgfsetlinewidth{0.000000pt}%
\definecolor{currentstroke}{rgb}{0.000000,0.000000,0.000000}%
\pgfsetstrokecolor{currentstroke}%
\pgfsetdash{}{0pt}%
\pgfpathmoveto{\pgfqpoint{2.939055in}{2.002339in}}%
\pgfpathlineto{\pgfqpoint{2.952804in}{1.985906in}}%
\pgfpathlineto{\pgfqpoint{2.966550in}{1.969622in}}%
\pgfpathlineto{\pgfqpoint{2.980292in}{1.953487in}}%
\pgfpathlineto{\pgfqpoint{2.994031in}{1.937499in}}%
\pgfpathlineto{\pgfqpoint{3.002686in}{1.930910in}}%
\pgfpathlineto{\pgfqpoint{3.011323in}{1.924645in}}%
\pgfpathlineto{\pgfqpoint{3.019941in}{1.918697in}}%
\pgfpathlineto{\pgfqpoint{3.028542in}{1.913061in}}%
\pgfpathlineto{\pgfqpoint{3.014851in}{1.928494in}}%
\pgfpathlineto{\pgfqpoint{3.001157in}{1.944073in}}%
\pgfpathlineto{\pgfqpoint{2.987460in}{1.959799in}}%
\pgfpathlineto{\pgfqpoint{2.973759in}{1.975673in}}%
\pgfpathlineto{\pgfqpoint{2.965111in}{1.981858in}}%
\pgfpathlineto{\pgfqpoint{2.956444in}{1.988360in}}%
\pgfpathlineto{\pgfqpoint{2.947759in}{1.995185in}}%
\pgfpathlineto{\pgfqpoint{2.939055in}{2.002339in}}%
\pgfpathclose%
\pgfusepath{fill}%
\end{pgfscope}%
\begin{pgfscope}%
\pgfpathrectangle{\pgfqpoint{1.254980in}{0.150000in}}{\pgfqpoint{5.490039in}{5.490039in}}%
\pgfusepath{clip}%
\pgfsetbuttcap%
\pgfsetroundjoin%
\definecolor{currentfill}{rgb}{0.225863,0.330805,0.547314}%
\pgfsetfillcolor{currentfill}%
\pgfsetfillopacity{0.700000}%
\pgfsetlinewidth{0.000000pt}%
\definecolor{currentstroke}{rgb}{0.000000,0.000000,0.000000}%
\pgfsetstrokecolor{currentstroke}%
\pgfsetdash{}{0pt}%
\pgfpathmoveto{\pgfqpoint{2.994031in}{1.937499in}}%
\pgfpathlineto{\pgfqpoint{3.007766in}{1.921657in}}%
\pgfpathlineto{\pgfqpoint{3.021499in}{1.905962in}}%
\pgfpathlineto{\pgfqpoint{3.035228in}{1.890412in}}%
\pgfpathlineto{\pgfqpoint{3.048955in}{1.875007in}}%
\pgfpathlineto{\pgfqpoint{3.057563in}{1.868980in}}%
\pgfpathlineto{\pgfqpoint{3.066153in}{1.863271in}}%
\pgfpathlineto{\pgfqpoint{3.074726in}{1.857874in}}%
\pgfpathlineto{\pgfqpoint{3.083281in}{1.852785in}}%
\pgfpathlineto{\pgfqpoint{3.069600in}{1.867638in}}%
\pgfpathlineto{\pgfqpoint{3.055917in}{1.882634in}}%
\pgfpathlineto{\pgfqpoint{3.042231in}{1.897775in}}%
\pgfpathlineto{\pgfqpoint{3.028542in}{1.913061in}}%
\pgfpathlineto{\pgfqpoint{3.019941in}{1.918697in}}%
\pgfpathlineto{\pgfqpoint{3.011323in}{1.924645in}}%
\pgfpathlineto{\pgfqpoint{3.002686in}{1.930910in}}%
\pgfpathlineto{\pgfqpoint{2.994031in}{1.937499in}}%
\pgfpathclose%
\pgfusepath{fill}%
\end{pgfscope}%
\begin{pgfscope}%
\pgfpathrectangle{\pgfqpoint{1.254980in}{0.150000in}}{\pgfqpoint{5.490039in}{5.490039in}}%
\pgfusepath{clip}%
\pgfsetbuttcap%
\pgfsetroundjoin%
\definecolor{currentfill}{rgb}{0.132268,0.655014,0.519661}%
\pgfsetfillcolor{currentfill}%
\pgfsetfillopacity{0.700000}%
\pgfsetlinewidth{0.000000pt}%
\definecolor{currentstroke}{rgb}{0.000000,0.000000,0.000000}%
\pgfsetstrokecolor{currentstroke}%
\pgfsetdash{}{0pt}%
\pgfpathmoveto{\pgfqpoint{2.365594in}{2.833606in}}%
\pgfpathlineto{\pgfqpoint{2.379581in}{2.810338in}}%
\pgfpathlineto{\pgfqpoint{2.393558in}{2.787261in}}%
\pgfpathlineto{\pgfqpoint{2.407526in}{2.764373in}}%
\pgfpathlineto{\pgfqpoint{2.421484in}{2.741674in}}%
\pgfpathlineto{\pgfqpoint{2.430666in}{2.730160in}}%
\pgfpathlineto{\pgfqpoint{2.439823in}{2.719010in}}%
\pgfpathlineto{\pgfqpoint{2.448956in}{2.708219in}}%
\pgfpathlineto{\pgfqpoint{2.458064in}{2.697782in}}%
\pgfpathlineto{\pgfqpoint{2.444169in}{2.719902in}}%
\pgfpathlineto{\pgfqpoint{2.430265in}{2.742210in}}%
\pgfpathlineto{\pgfqpoint{2.416351in}{2.764706in}}%
\pgfpathlineto{\pgfqpoint{2.402428in}{2.787392in}}%
\pgfpathlineto{\pgfqpoint{2.393257in}{2.798400in}}%
\pgfpathlineto{\pgfqpoint{2.384062in}{2.809769in}}%
\pgfpathlineto{\pgfqpoint{2.374841in}{2.821503in}}%
\pgfpathlineto{\pgfqpoint{2.365594in}{2.833606in}}%
\pgfpathclose%
\pgfusepath{fill}%
\end{pgfscope}%
\begin{pgfscope}%
\pgfpathrectangle{\pgfqpoint{1.254980in}{0.150000in}}{\pgfqpoint{5.490039in}{5.490039in}}%
\pgfusepath{clip}%
\pgfsetbuttcap%
\pgfsetroundjoin%
\definecolor{currentfill}{rgb}{0.266580,0.228262,0.514349}%
\pgfsetfillcolor{currentfill}%
\pgfsetfillopacity{0.700000}%
\pgfsetlinewidth{0.000000pt}%
\definecolor{currentstroke}{rgb}{0.000000,0.000000,0.000000}%
\pgfsetstrokecolor{currentstroke}%
\pgfsetdash{}{0pt}%
\pgfpathmoveto{\pgfqpoint{4.975182in}{1.622212in}}%
\pgfpathlineto{\pgfqpoint{4.989253in}{1.625838in}}%
\pgfpathlineto{\pgfqpoint{5.003336in}{1.629573in}}%
\pgfpathlineto{\pgfqpoint{5.017433in}{1.633419in}}%
\pgfpathlineto{\pgfqpoint{5.031543in}{1.637375in}}%
\pgfpathlineto{\pgfqpoint{5.039212in}{1.651303in}}%
\pgfpathlineto{\pgfqpoint{5.046878in}{1.665232in}}%
\pgfpathlineto{\pgfqpoint{5.054540in}{1.679161in}}%
\pgfpathlineto{\pgfqpoint{5.062197in}{1.693087in}}%
\pgfpathlineto{\pgfqpoint{5.048085in}{1.688828in}}%
\pgfpathlineto{\pgfqpoint{5.033986in}{1.684679in}}%
\pgfpathlineto{\pgfqpoint{5.019901in}{1.680641in}}%
\pgfpathlineto{\pgfqpoint{5.005828in}{1.676713in}}%
\pgfpathlineto{\pgfqpoint{4.998173in}{1.663084in}}%
\pgfpathlineto{\pgfqpoint{4.990513in}{1.649456in}}%
\pgfpathlineto{\pgfqpoint{4.982850in}{1.635831in}}%
\pgfpathlineto{\pgfqpoint{4.975182in}{1.622212in}}%
\pgfpathclose%
\pgfusepath{fill}%
\end{pgfscope}%
\begin{pgfscope}%
\pgfpathrectangle{\pgfqpoint{1.254980in}{0.150000in}}{\pgfqpoint{5.490039in}{5.490039in}}%
\pgfusepath{clip}%
\pgfsetbuttcap%
\pgfsetroundjoin%
\definecolor{currentfill}{rgb}{0.201239,0.383670,0.554294}%
\pgfsetfillcolor{currentfill}%
\pgfsetfillopacity{0.700000}%
\pgfsetlinewidth{0.000000pt}%
\definecolor{currentstroke}{rgb}{0.000000,0.000000,0.000000}%
\pgfsetstrokecolor{currentstroke}%
\pgfsetdash{}{0pt}%
\pgfpathmoveto{\pgfqpoint{2.884020in}{2.069574in}}%
\pgfpathlineto{\pgfqpoint{2.897785in}{2.052538in}}%
\pgfpathlineto{\pgfqpoint{2.911546in}{2.035654in}}%
\pgfpathlineto{\pgfqpoint{2.925302in}{2.018921in}}%
\pgfpathlineto{\pgfqpoint{2.939055in}{2.002339in}}%
\pgfpathlineto{\pgfqpoint{2.947759in}{1.995185in}}%
\pgfpathlineto{\pgfqpoint{2.956444in}{1.988360in}}%
\pgfpathlineto{\pgfqpoint{2.965111in}{1.981858in}}%
\pgfpathlineto{\pgfqpoint{2.973759in}{1.975673in}}%
\pgfpathlineto{\pgfqpoint{2.960055in}{1.991697in}}%
\pgfpathlineto{\pgfqpoint{2.946348in}{2.007870in}}%
\pgfpathlineto{\pgfqpoint{2.932637in}{2.024193in}}%
\pgfpathlineto{\pgfqpoint{2.918923in}{2.040668in}}%
\pgfpathlineto{\pgfqpoint{2.910226in}{2.047404in}}%
\pgfpathlineto{\pgfqpoint{2.901510in}{2.054464in}}%
\pgfpathlineto{\pgfqpoint{2.892775in}{2.061852in}}%
\pgfpathlineto{\pgfqpoint{2.884020in}{2.069574in}}%
\pgfpathclose%
\pgfusepath{fill}%
\end{pgfscope}%
\begin{pgfscope}%
\pgfpathrectangle{\pgfqpoint{1.254980in}{0.150000in}}{\pgfqpoint{5.490039in}{5.490039in}}%
\pgfusepath{clip}%
\pgfsetbuttcap%
\pgfsetroundjoin%
\definecolor{currentfill}{rgb}{0.281446,0.084320,0.407414}%
\pgfsetfillcolor{currentfill}%
\pgfsetfillopacity{0.700000}%
\pgfsetlinewidth{0.000000pt}%
\definecolor{currentstroke}{rgb}{0.000000,0.000000,0.000000}%
\pgfsetstrokecolor{currentstroke}%
\pgfsetdash{}{0pt}%
\pgfpathmoveto{\pgfqpoint{4.597513in}{1.339766in}}%
\pgfpathlineto{\pgfqpoint{4.611404in}{1.339891in}}%
\pgfpathlineto{\pgfqpoint{4.625305in}{1.340125in}}%
\pgfpathlineto{\pgfqpoint{4.639216in}{1.340470in}}%
\pgfpathlineto{\pgfqpoint{4.653137in}{1.340924in}}%
\pgfpathlineto{\pgfqpoint{4.660883in}{1.352729in}}%
\pgfpathlineto{\pgfqpoint{4.668625in}{1.364606in}}%
\pgfpathlineto{\pgfqpoint{4.676363in}{1.376551in}}%
\pgfpathlineto{\pgfqpoint{4.684097in}{1.388561in}}%
\pgfpathlineto{\pgfqpoint{4.670179in}{1.387728in}}%
\pgfpathlineto{\pgfqpoint{4.656271in}{1.387005in}}%
\pgfpathlineto{\pgfqpoint{4.642374in}{1.386392in}}%
\pgfpathlineto{\pgfqpoint{4.628487in}{1.385889in}}%
\pgfpathlineto{\pgfqpoint{4.620749in}{1.374251in}}%
\pgfpathlineto{\pgfqpoint{4.613008in}{1.362683in}}%
\pgfpathlineto{\pgfqpoint{4.605263in}{1.351187in}}%
\pgfpathlineto{\pgfqpoint{4.597513in}{1.339766in}}%
\pgfpathclose%
\pgfusepath{fill}%
\end{pgfscope}%
\begin{pgfscope}%
\pgfpathrectangle{\pgfqpoint{1.254980in}{0.150000in}}{\pgfqpoint{5.490039in}{5.490039in}}%
\pgfusepath{clip}%
\pgfsetbuttcap%
\pgfsetroundjoin%
\definecolor{currentfill}{rgb}{0.277941,0.056324,0.381191}%
\pgfsetfillcolor{currentfill}%
\pgfsetfillopacity{0.700000}%
\pgfsetlinewidth{0.000000pt}%
\definecolor{currentstroke}{rgb}{0.000000,0.000000,0.000000}%
\pgfsetstrokecolor{currentstroke}%
\pgfsetdash{}{0pt}%
\pgfpathmoveto{\pgfqpoint{4.510991in}{1.297076in}}%
\pgfpathlineto{\pgfqpoint{4.524848in}{1.296366in}}%
\pgfpathlineto{\pgfqpoint{4.538714in}{1.295767in}}%
\pgfpathlineto{\pgfqpoint{4.552589in}{1.295277in}}%
\pgfpathlineto{\pgfqpoint{4.566474in}{1.294898in}}%
\pgfpathlineto{\pgfqpoint{4.574240in}{1.305986in}}%
\pgfpathlineto{\pgfqpoint{4.582002in}{1.317163in}}%
\pgfpathlineto{\pgfqpoint{4.589760in}{1.328424in}}%
\pgfpathlineto{\pgfqpoint{4.597513in}{1.339766in}}%
\pgfpathlineto{\pgfqpoint{4.583633in}{1.339751in}}%
\pgfpathlineto{\pgfqpoint{4.569762in}{1.339847in}}%
\pgfpathlineto{\pgfqpoint{4.555901in}{1.340053in}}%
\pgfpathlineto{\pgfqpoint{4.542050in}{1.340370in}}%
\pgfpathlineto{\pgfqpoint{4.534292in}{1.329415in}}%
\pgfpathlineto{\pgfqpoint{4.526529in}{1.318546in}}%
\pgfpathlineto{\pgfqpoint{4.518762in}{1.307765in}}%
\pgfpathlineto{\pgfqpoint{4.510991in}{1.297076in}}%
\pgfpathclose%
\pgfusepath{fill}%
\end{pgfscope}%
\begin{pgfscope}%
\pgfpathrectangle{\pgfqpoint{1.254980in}{0.150000in}}{\pgfqpoint{5.490039in}{5.490039in}}%
\pgfusepath{clip}%
\pgfsetbuttcap%
\pgfsetroundjoin%
\definecolor{currentfill}{rgb}{0.237441,0.305202,0.541921}%
\pgfsetfillcolor{currentfill}%
\pgfsetfillopacity{0.700000}%
\pgfsetlinewidth{0.000000pt}%
\definecolor{currentstroke}{rgb}{0.000000,0.000000,0.000000}%
\pgfsetstrokecolor{currentstroke}%
\pgfsetdash{}{0pt}%
\pgfpathmoveto{\pgfqpoint{3.048955in}{1.875007in}}%
\pgfpathlineto{\pgfqpoint{3.062679in}{1.859746in}}%
\pgfpathlineto{\pgfqpoint{3.076401in}{1.844628in}}%
\pgfpathlineto{\pgfqpoint{3.090120in}{1.829652in}}%
\pgfpathlineto{\pgfqpoint{3.103837in}{1.814819in}}%
\pgfpathlineto{\pgfqpoint{3.112399in}{1.809351in}}%
\pgfpathlineto{\pgfqpoint{3.120944in}{1.804196in}}%
\pgfpathlineto{\pgfqpoint{3.129472in}{1.799347in}}%
\pgfpathlineto{\pgfqpoint{3.137984in}{1.794801in}}%
\pgfpathlineto{\pgfqpoint{3.124311in}{1.809085in}}%
\pgfpathlineto{\pgfqpoint{3.110637in}{1.823509in}}%
\pgfpathlineto{\pgfqpoint{3.096960in}{1.838076in}}%
\pgfpathlineto{\pgfqpoint{3.083281in}{1.852785in}}%
\pgfpathlineto{\pgfqpoint{3.074726in}{1.857874in}}%
\pgfpathlineto{\pgfqpoint{3.066153in}{1.863271in}}%
\pgfpathlineto{\pgfqpoint{3.057563in}{1.868980in}}%
\pgfpathlineto{\pgfqpoint{3.048955in}{1.875007in}}%
\pgfpathclose%
\pgfusepath{fill}%
\end{pgfscope}%
\begin{pgfscope}%
\pgfpathrectangle{\pgfqpoint{1.254980in}{0.150000in}}{\pgfqpoint{5.490039in}{5.490039in}}%
\pgfusepath{clip}%
\pgfsetbuttcap%
\pgfsetroundjoin%
\definecolor{currentfill}{rgb}{0.190631,0.407061,0.556089}%
\pgfsetfillcolor{currentfill}%
\pgfsetfillopacity{0.700000}%
\pgfsetlinewidth{0.000000pt}%
\definecolor{currentstroke}{rgb}{0.000000,0.000000,0.000000}%
\pgfsetstrokecolor{currentstroke}%
\pgfsetdash{}{0pt}%
\pgfpathmoveto{\pgfqpoint{2.828917in}{2.139254in}}%
\pgfpathlineto{\pgfqpoint{2.842700in}{2.121602in}}%
\pgfpathlineto{\pgfqpoint{2.856478in}{2.104105in}}%
\pgfpathlineto{\pgfqpoint{2.870251in}{2.086763in}}%
\pgfpathlineto{\pgfqpoint{2.884020in}{2.069574in}}%
\pgfpathlineto{\pgfqpoint{2.892775in}{2.061852in}}%
\pgfpathlineto{\pgfqpoint{2.901510in}{2.054464in}}%
\pgfpathlineto{\pgfqpoint{2.910226in}{2.047404in}}%
\pgfpathlineto{\pgfqpoint{2.918923in}{2.040668in}}%
\pgfpathlineto{\pgfqpoint{2.905204in}{2.057294in}}%
\pgfpathlineto{\pgfqpoint{2.891482in}{2.074073in}}%
\pgfpathlineto{\pgfqpoint{2.877755in}{2.091005in}}%
\pgfpathlineto{\pgfqpoint{2.864024in}{2.108092in}}%
\pgfpathlineto{\pgfqpoint{2.855277in}{2.115384in}}%
\pgfpathlineto{\pgfqpoint{2.846511in}{2.123005in}}%
\pgfpathlineto{\pgfqpoint{2.837724in}{2.130960in}}%
\pgfpathlineto{\pgfqpoint{2.828917in}{2.139254in}}%
\pgfpathclose%
\pgfusepath{fill}%
\end{pgfscope}%
\begin{pgfscope}%
\pgfpathrectangle{\pgfqpoint{1.254980in}{0.150000in}}{\pgfqpoint{5.490039in}{5.490039in}}%
\pgfusepath{clip}%
\pgfsetbuttcap%
\pgfsetroundjoin%
\definecolor{currentfill}{rgb}{0.276022,0.044167,0.370164}%
\pgfsetfillcolor{currentfill}%
\pgfsetfillopacity{0.700000}%
\pgfsetlinewidth{0.000000pt}%
\definecolor{currentstroke}{rgb}{0.000000,0.000000,0.000000}%
\pgfsetstrokecolor{currentstroke}%
\pgfsetdash{}{0pt}%
\pgfpathmoveto{\pgfqpoint{3.771913in}{1.305352in}}%
\pgfpathlineto{\pgfqpoint{3.785596in}{1.297391in}}%
\pgfpathlineto{\pgfqpoint{3.799282in}{1.289549in}}%
\pgfpathlineto{\pgfqpoint{3.812972in}{1.281826in}}%
\pgfpathlineto{\pgfqpoint{3.826666in}{1.274221in}}%
\pgfpathlineto{\pgfqpoint{3.834716in}{1.277185in}}%
\pgfpathlineto{\pgfqpoint{3.842756in}{1.280368in}}%
\pgfpathlineto{\pgfqpoint{3.850788in}{1.283764in}}%
\pgfpathlineto{\pgfqpoint{3.858811in}{1.287370in}}%
\pgfpathlineto{\pgfqpoint{3.845140in}{1.294483in}}%
\pgfpathlineto{\pgfqpoint{3.831474in}{1.301714in}}%
\pgfpathlineto{\pgfqpoint{3.817811in}{1.309064in}}%
\pgfpathlineto{\pgfqpoint{3.804153in}{1.316533in}}%
\pgfpathlineto{\pgfqpoint{3.796107in}{1.313412in}}%
\pgfpathlineto{\pgfqpoint{3.788051in}{1.310506in}}%
\pgfpathlineto{\pgfqpoint{3.779987in}{1.307818in}}%
\pgfpathlineto{\pgfqpoint{3.771913in}{1.305352in}}%
\pgfpathclose%
\pgfusepath{fill}%
\end{pgfscope}%
\begin{pgfscope}%
\pgfpathrectangle{\pgfqpoint{1.254980in}{0.150000in}}{\pgfqpoint{5.490039in}{5.490039in}}%
\pgfusepath{clip}%
\pgfsetbuttcap%
\pgfsetroundjoin%
\definecolor{currentfill}{rgb}{0.283091,0.110553,0.431554}%
\pgfsetfillcolor{currentfill}%
\pgfsetfillopacity{0.700000}%
\pgfsetlinewidth{0.000000pt}%
\definecolor{currentstroke}{rgb}{0.000000,0.000000,0.000000}%
\pgfsetstrokecolor{currentstroke}%
\pgfsetdash{}{0pt}%
\pgfpathmoveto{\pgfqpoint{4.684097in}{1.388561in}}%
\pgfpathlineto{\pgfqpoint{4.698027in}{1.389504in}}%
\pgfpathlineto{\pgfqpoint{4.711967in}{1.390557in}}%
\pgfpathlineto{\pgfqpoint{4.725918in}{1.391719in}}%
\pgfpathlineto{\pgfqpoint{4.739879in}{1.392992in}}%
\pgfpathlineto{\pgfqpoint{4.747608in}{1.405434in}}%
\pgfpathlineto{\pgfqpoint{4.755332in}{1.417931in}}%
\pgfpathlineto{\pgfqpoint{4.763053in}{1.430482in}}%
\pgfpathlineto{\pgfqpoint{4.770770in}{1.443082in}}%
\pgfpathlineto{\pgfqpoint{4.756809in}{1.441446in}}%
\pgfpathlineto{\pgfqpoint{4.742860in}{1.439919in}}%
\pgfpathlineto{\pgfqpoint{4.728922in}{1.438503in}}%
\pgfpathlineto{\pgfqpoint{4.714995in}{1.437197in}}%
\pgfpathlineto{\pgfqpoint{4.707276in}{1.424955in}}%
\pgfpathlineto{\pgfqpoint{4.699554in}{1.412766in}}%
\pgfpathlineto{\pgfqpoint{4.691828in}{1.400634in}}%
\pgfpathlineto{\pgfqpoint{4.684097in}{1.388561in}}%
\pgfpathclose%
\pgfusepath{fill}%
\end{pgfscope}%
\begin{pgfscope}%
\pgfpathrectangle{\pgfqpoint{1.254980in}{0.150000in}}{\pgfqpoint{5.490039in}{5.490039in}}%
\pgfusepath{clip}%
\pgfsetbuttcap%
\pgfsetroundjoin%
\definecolor{currentfill}{rgb}{0.246811,0.283237,0.535941}%
\pgfsetfillcolor{currentfill}%
\pgfsetfillopacity{0.700000}%
\pgfsetlinewidth{0.000000pt}%
\definecolor{currentstroke}{rgb}{0.000000,0.000000,0.000000}%
\pgfsetstrokecolor{currentstroke}%
\pgfsetdash{}{0pt}%
\pgfpathmoveto{\pgfqpoint{3.103837in}{1.814819in}}%
\pgfpathlineto{\pgfqpoint{3.117552in}{1.800127in}}%
\pgfpathlineto{\pgfqpoint{3.131265in}{1.785576in}}%
\pgfpathlineto{\pgfqpoint{3.144976in}{1.771165in}}%
\pgfpathlineto{\pgfqpoint{3.158685in}{1.756893in}}%
\pgfpathlineto{\pgfqpoint{3.167203in}{1.751981in}}%
\pgfpathlineto{\pgfqpoint{3.175704in}{1.747377in}}%
\pgfpathlineto{\pgfqpoint{3.184189in}{1.743074in}}%
\pgfpathlineto{\pgfqpoint{3.192658in}{1.739069in}}%
\pgfpathlineto{\pgfqpoint{3.178992in}{1.752793in}}%
\pgfpathlineto{\pgfqpoint{3.165324in}{1.766656in}}%
\pgfpathlineto{\pgfqpoint{3.151655in}{1.780659in}}%
\pgfpathlineto{\pgfqpoint{3.137984in}{1.794801in}}%
\pgfpathlineto{\pgfqpoint{3.129472in}{1.799347in}}%
\pgfpathlineto{\pgfqpoint{3.120944in}{1.804196in}}%
\pgfpathlineto{\pgfqpoint{3.112399in}{1.809351in}}%
\pgfpathlineto{\pgfqpoint{3.103837in}{1.814819in}}%
\pgfpathclose%
\pgfusepath{fill}%
\end{pgfscope}%
\begin{pgfscope}%
\pgfpathrectangle{\pgfqpoint{1.254980in}{0.150000in}}{\pgfqpoint{5.490039in}{5.490039in}}%
\pgfusepath{clip}%
\pgfsetbuttcap%
\pgfsetroundjoin%
\definecolor{currentfill}{rgb}{0.273809,0.031497,0.358853}%
\pgfsetfillcolor{currentfill}%
\pgfsetfillopacity{0.700000}%
\pgfsetlinewidth{0.000000pt}%
\definecolor{currentstroke}{rgb}{0.000000,0.000000,0.000000}%
\pgfsetstrokecolor{currentstroke}%
\pgfsetdash{}{0pt}%
\pgfpathmoveto{\pgfqpoint{4.424502in}{1.260883in}}%
\pgfpathlineto{\pgfqpoint{4.438329in}{1.259322in}}%
\pgfpathlineto{\pgfqpoint{4.452164in}{1.257872in}}%
\pgfpathlineto{\pgfqpoint{4.466009in}{1.256532in}}%
\pgfpathlineto{\pgfqpoint{4.479863in}{1.255303in}}%
\pgfpathlineto{\pgfqpoint{4.487651in}{1.265592in}}%
\pgfpathlineto{\pgfqpoint{4.495436in}{1.275986in}}%
\pgfpathlineto{\pgfqpoint{4.503216in}{1.286482in}}%
\pgfpathlineto{\pgfqpoint{4.510991in}{1.297076in}}%
\pgfpathlineto{\pgfqpoint{4.497144in}{1.297896in}}%
\pgfpathlineto{\pgfqpoint{4.483306in}{1.298827in}}%
\pgfpathlineto{\pgfqpoint{4.469477in}{1.299868in}}%
\pgfpathlineto{\pgfqpoint{4.455658in}{1.301021in}}%
\pgfpathlineto{\pgfqpoint{4.447876in}{1.290830in}}%
\pgfpathlineto{\pgfqpoint{4.440089in}{1.280741in}}%
\pgfpathlineto{\pgfqpoint{4.432298in}{1.270757in}}%
\pgfpathlineto{\pgfqpoint{4.424502in}{1.260883in}}%
\pgfpathclose%
\pgfusepath{fill}%
\end{pgfscope}%
\begin{pgfscope}%
\pgfpathrectangle{\pgfqpoint{1.254980in}{0.150000in}}{\pgfqpoint{5.490039in}{5.490039in}}%
\pgfusepath{clip}%
\pgfsetbuttcap%
\pgfsetroundjoin%
\definecolor{currentfill}{rgb}{0.179019,0.433756,0.557430}%
\pgfsetfillcolor{currentfill}%
\pgfsetfillopacity{0.700000}%
\pgfsetlinewidth{0.000000pt}%
\definecolor{currentstroke}{rgb}{0.000000,0.000000,0.000000}%
\pgfsetstrokecolor{currentstroke}%
\pgfsetdash{}{0pt}%
\pgfpathmoveto{\pgfqpoint{2.773737in}{2.211430in}}%
\pgfpathlineto{\pgfqpoint{2.787539in}{2.193149in}}%
\pgfpathlineto{\pgfqpoint{2.801337in}{2.175026in}}%
\pgfpathlineto{\pgfqpoint{2.815129in}{2.157062in}}%
\pgfpathlineto{\pgfqpoint{2.828917in}{2.139254in}}%
\pgfpathlineto{\pgfqpoint{2.837724in}{2.130960in}}%
\pgfpathlineto{\pgfqpoint{2.846511in}{2.123005in}}%
\pgfpathlineto{\pgfqpoint{2.855277in}{2.115384in}}%
\pgfpathlineto{\pgfqpoint{2.864024in}{2.108092in}}%
\pgfpathlineto{\pgfqpoint{2.850289in}{2.125334in}}%
\pgfpathlineto{\pgfqpoint{2.836549in}{2.142732in}}%
\pgfpathlineto{\pgfqpoint{2.822804in}{2.160286in}}%
\pgfpathlineto{\pgfqpoint{2.809055in}{2.177998in}}%
\pgfpathlineto{\pgfqpoint{2.800256in}{2.185849in}}%
\pgfpathlineto{\pgfqpoint{2.791437in}{2.194035in}}%
\pgfpathlineto{\pgfqpoint{2.782598in}{2.202560in}}%
\pgfpathlineto{\pgfqpoint{2.773737in}{2.211430in}}%
\pgfpathclose%
\pgfusepath{fill}%
\end{pgfscope}%
\begin{pgfscope}%
\pgfpathrectangle{\pgfqpoint{1.254980in}{0.150000in}}{\pgfqpoint{5.490039in}{5.490039in}}%
\pgfusepath{clip}%
\pgfsetbuttcap%
\pgfsetroundjoin%
\definecolor{currentfill}{rgb}{0.404001,0.800275,0.362552}%
\pgfsetfillcolor{currentfill}%
\pgfsetfillopacity{0.700000}%
\pgfsetlinewidth{0.000000pt}%
\definecolor{currentstroke}{rgb}{0.000000,0.000000,0.000000}%
\pgfsetstrokecolor{currentstroke}%
\pgfsetdash{}{0pt}%
\pgfpathmoveto{\pgfqpoint{2.121645in}{3.282963in}}%
\pgfpathlineto{\pgfqpoint{2.135802in}{3.256200in}}%
\pgfpathlineto{\pgfqpoint{2.149945in}{3.229656in}}%
\pgfpathlineto{\pgfqpoint{2.164075in}{3.203327in}}%
\pgfpathlineto{\pgfqpoint{2.178191in}{3.177214in}}%
\pgfpathlineto{\pgfqpoint{2.187601in}{3.164242in}}%
\pgfpathlineto{\pgfqpoint{2.196984in}{3.151644in}}%
\pgfpathlineto{\pgfqpoint{2.206339in}{3.139414in}}%
\pgfpathlineto{\pgfqpoint{2.215668in}{3.127549in}}%
\pgfpathlineto{\pgfqpoint{2.201620in}{3.153079in}}%
\pgfpathlineto{\pgfqpoint{2.187560in}{3.178823in}}%
\pgfpathlineto{\pgfqpoint{2.173487in}{3.204782in}}%
\pgfpathlineto{\pgfqpoint{2.159400in}{3.230957in}}%
\pgfpathlineto{\pgfqpoint{2.150003in}{3.243398in}}%
\pgfpathlineto{\pgfqpoint{2.140579in}{3.256210in}}%
\pgfpathlineto{\pgfqpoint{2.131126in}{3.269396in}}%
\pgfpathlineto{\pgfqpoint{2.121645in}{3.282963in}}%
\pgfpathclose%
\pgfusepath{fill}%
\end{pgfscope}%
\begin{pgfscope}%
\pgfpathrectangle{\pgfqpoint{1.254980in}{0.150000in}}{\pgfqpoint{5.490039in}{5.490039in}}%
\pgfusepath{clip}%
\pgfsetbuttcap%
\pgfsetroundjoin%
\definecolor{currentfill}{rgb}{0.257322,0.256130,0.526563}%
\pgfsetfillcolor{currentfill}%
\pgfsetfillopacity{0.700000}%
\pgfsetlinewidth{0.000000pt}%
\definecolor{currentstroke}{rgb}{0.000000,0.000000,0.000000}%
\pgfsetstrokecolor{currentstroke}%
\pgfsetdash{}{0pt}%
\pgfpathmoveto{\pgfqpoint{3.158685in}{1.756893in}}%
\pgfpathlineto{\pgfqpoint{3.172393in}{1.742760in}}%
\pgfpathlineto{\pgfqpoint{3.186099in}{1.728766in}}%
\pgfpathlineto{\pgfqpoint{3.199804in}{1.714909in}}%
\pgfpathlineto{\pgfqpoint{3.213507in}{1.701189in}}%
\pgfpathlineto{\pgfqpoint{3.221982in}{1.696831in}}%
\pgfpathlineto{\pgfqpoint{3.230440in}{1.692774in}}%
\pgfpathlineto{\pgfqpoint{3.238884in}{1.689015in}}%
\pgfpathlineto{\pgfqpoint{3.247312in}{1.685547in}}%
\pgfpathlineto{\pgfqpoint{3.233650in}{1.698722in}}%
\pgfpathlineto{\pgfqpoint{3.219987in}{1.712034in}}%
\pgfpathlineto{\pgfqpoint{3.206323in}{1.725482in}}%
\pgfpathlineto{\pgfqpoint{3.192658in}{1.739069in}}%
\pgfpathlineto{\pgfqpoint{3.184189in}{1.743074in}}%
\pgfpathlineto{\pgfqpoint{3.175704in}{1.747377in}}%
\pgfpathlineto{\pgfqpoint{3.167203in}{1.751981in}}%
\pgfpathlineto{\pgfqpoint{3.158685in}{1.756893in}}%
\pgfpathclose%
\pgfusepath{fill}%
\end{pgfscope}%
\begin{pgfscope}%
\pgfpathrectangle{\pgfqpoint{1.254980in}{0.150000in}}{\pgfqpoint{5.490039in}{5.490039in}}%
\pgfusepath{clip}%
\pgfsetbuttcap%
\pgfsetroundjoin%
\definecolor{currentfill}{rgb}{0.282623,0.140926,0.457517}%
\pgfsetfillcolor{currentfill}%
\pgfsetfillopacity{0.700000}%
\pgfsetlinewidth{0.000000pt}%
\definecolor{currentstroke}{rgb}{0.000000,0.000000,0.000000}%
\pgfsetstrokecolor{currentstroke}%
\pgfsetdash{}{0pt}%
\pgfpathmoveto{\pgfqpoint{4.770770in}{1.443082in}}%
\pgfpathlineto{\pgfqpoint{4.784742in}{1.444828in}}%
\pgfpathlineto{\pgfqpoint{4.798725in}{1.446683in}}%
\pgfpathlineto{\pgfqpoint{4.812720in}{1.448649in}}%
\pgfpathlineto{\pgfqpoint{4.826726in}{1.450724in}}%
\pgfpathlineto{\pgfqpoint{4.834439in}{1.463726in}}%
\pgfpathlineto{\pgfqpoint{4.842148in}{1.476767in}}%
\pgfpathlineto{\pgfqpoint{4.849853in}{1.489847in}}%
\pgfpathlineto{\pgfqpoint{4.857554in}{1.502961in}}%
\pgfpathlineto{\pgfqpoint{4.843548in}{1.500537in}}%
\pgfpathlineto{\pgfqpoint{4.829553in}{1.498222in}}%
\pgfpathlineto{\pgfqpoint{4.815570in}{1.496018in}}%
\pgfpathlineto{\pgfqpoint{4.801599in}{1.493924in}}%
\pgfpathlineto{\pgfqpoint{4.793898in}{1.481153in}}%
\pgfpathlineto{\pgfqpoint{4.786192in}{1.468420in}}%
\pgfpathlineto{\pgfqpoint{4.778483in}{1.455729in}}%
\pgfpathlineto{\pgfqpoint{4.770770in}{1.443082in}}%
\pgfpathclose%
\pgfusepath{fill}%
\end{pgfscope}%
\begin{pgfscope}%
\pgfpathrectangle{\pgfqpoint{1.254980in}{0.150000in}}{\pgfqpoint{5.490039in}{5.490039in}}%
\pgfusepath{clip}%
\pgfsetbuttcap%
\pgfsetroundjoin%
\definecolor{currentfill}{rgb}{0.166617,0.463708,0.558119}%
\pgfsetfillcolor{currentfill}%
\pgfsetfillopacity{0.700000}%
\pgfsetlinewidth{0.000000pt}%
\definecolor{currentstroke}{rgb}{0.000000,0.000000,0.000000}%
\pgfsetstrokecolor{currentstroke}%
\pgfsetdash{}{0pt}%
\pgfpathmoveto{\pgfqpoint{2.718470in}{2.286156in}}%
\pgfpathlineto{\pgfqpoint{2.732295in}{2.267232in}}%
\pgfpathlineto{\pgfqpoint{2.746115in}{2.248470in}}%
\pgfpathlineto{\pgfqpoint{2.759928in}{2.229870in}}%
\pgfpathlineto{\pgfqpoint{2.773737in}{2.211430in}}%
\pgfpathlineto{\pgfqpoint{2.782598in}{2.202560in}}%
\pgfpathlineto{\pgfqpoint{2.791437in}{2.194035in}}%
\pgfpathlineto{\pgfqpoint{2.800256in}{2.185849in}}%
\pgfpathlineto{\pgfqpoint{2.809055in}{2.177998in}}%
\pgfpathlineto{\pgfqpoint{2.795301in}{2.195868in}}%
\pgfpathlineto{\pgfqpoint{2.781541in}{2.213898in}}%
\pgfpathlineto{\pgfqpoint{2.767776in}{2.232088in}}%
\pgfpathlineto{\pgfqpoint{2.754006in}{2.250438in}}%
\pgfpathlineto{\pgfqpoint{2.745154in}{2.258853in}}%
\pgfpathlineto{\pgfqpoint{2.736281in}{2.267607in}}%
\pgfpathlineto{\pgfqpoint{2.727386in}{2.276706in}}%
\pgfpathlineto{\pgfqpoint{2.718470in}{2.286156in}}%
\pgfpathclose%
\pgfusepath{fill}%
\end{pgfscope}%
\begin{pgfscope}%
\pgfpathrectangle{\pgfqpoint{1.254980in}{0.150000in}}{\pgfqpoint{5.490039in}{5.490039in}}%
\pgfusepath{clip}%
\pgfsetbuttcap%
\pgfsetroundjoin%
\definecolor{currentfill}{rgb}{0.282656,0.100196,0.422160}%
\pgfsetfillcolor{currentfill}%
\pgfsetfillopacity{0.700000}%
\pgfsetlinewidth{0.000000pt}%
\definecolor{currentstroke}{rgb}{0.000000,0.000000,0.000000}%
\pgfsetstrokecolor{currentstroke}%
\pgfsetdash{}{0pt}%
\pgfpathmoveto{\pgfqpoint{3.575234in}{1.408896in}}%
\pgfpathlineto{\pgfqpoint{3.588911in}{1.398965in}}%
\pgfpathlineto{\pgfqpoint{3.602591in}{1.389157in}}%
\pgfpathlineto{\pgfqpoint{3.616272in}{1.379472in}}%
\pgfpathlineto{\pgfqpoint{3.629956in}{1.369910in}}%
\pgfpathlineto{\pgfqpoint{3.638126in}{1.370406in}}%
\pgfpathlineto{\pgfqpoint{3.646285in}{1.371151in}}%
\pgfpathlineto{\pgfqpoint{3.654433in}{1.372141in}}%
\pgfpathlineto{\pgfqpoint{3.662571in}{1.373370in}}%
\pgfpathlineto{\pgfqpoint{3.648916in}{1.382419in}}%
\pgfpathlineto{\pgfqpoint{3.635264in}{1.391591in}}%
\pgfpathlineto{\pgfqpoint{3.621615in}{1.400885in}}%
\pgfpathlineto{\pgfqpoint{3.607968in}{1.410303in}}%
\pgfpathlineto{\pgfqpoint{3.599802in}{1.409580in}}%
\pgfpathlineto{\pgfqpoint{3.591624in}{1.409101in}}%
\pgfpathlineto{\pgfqpoint{3.583435in}{1.408872in}}%
\pgfpathlineto{\pgfqpoint{3.575234in}{1.408896in}}%
\pgfpathclose%
\pgfusepath{fill}%
\end{pgfscope}%
\begin{pgfscope}%
\pgfpathrectangle{\pgfqpoint{1.254980in}{0.150000in}}{\pgfqpoint{5.490039in}{5.490039in}}%
\pgfusepath{clip}%
\pgfsetbuttcap%
\pgfsetroundjoin%
\definecolor{currentfill}{rgb}{0.271305,0.019942,0.347269}%
\pgfsetfillcolor{currentfill}%
\pgfsetfillopacity{0.700000}%
\pgfsetlinewidth{0.000000pt}%
\definecolor{currentstroke}{rgb}{0.000000,0.000000,0.000000}%
\pgfsetstrokecolor{currentstroke}%
\pgfsetdash{}{0pt}%
\pgfpathmoveto{\pgfqpoint{4.338014in}{1.231592in}}%
\pgfpathlineto{\pgfqpoint{4.351816in}{1.229163in}}%
\pgfpathlineto{\pgfqpoint{4.365626in}{1.226844in}}%
\pgfpathlineto{\pgfqpoint{4.379444in}{1.224637in}}%
\pgfpathlineto{\pgfqpoint{4.393271in}{1.222541in}}%
\pgfpathlineto{\pgfqpoint{4.401086in}{1.231946in}}%
\pgfpathlineto{\pgfqpoint{4.408896in}{1.241474in}}%
\pgfpathlineto{\pgfqpoint{4.416701in}{1.251121in}}%
\pgfpathlineto{\pgfqpoint{4.424502in}{1.260883in}}%
\pgfpathlineto{\pgfqpoint{4.410684in}{1.262555in}}%
\pgfpathlineto{\pgfqpoint{4.396874in}{1.264337in}}%
\pgfpathlineto{\pgfqpoint{4.383073in}{1.266231in}}%
\pgfpathlineto{\pgfqpoint{4.369280in}{1.268236in}}%
\pgfpathlineto{\pgfqpoint{4.361471in}{1.258892in}}%
\pgfpathlineto{\pgfqpoint{4.353657in}{1.249668in}}%
\pgfpathlineto{\pgfqpoint{4.345838in}{1.240566in}}%
\pgfpathlineto{\pgfqpoint{4.338014in}{1.231592in}}%
\pgfpathclose%
\pgfusepath{fill}%
\end{pgfscope}%
\begin{pgfscope}%
\pgfpathrectangle{\pgfqpoint{1.254980in}{0.150000in}}{\pgfqpoint{5.490039in}{5.490039in}}%
\pgfusepath{clip}%
\pgfsetbuttcap%
\pgfsetroundjoin%
\definecolor{currentfill}{rgb}{0.269944,0.014625,0.341379}%
\pgfsetfillcolor{currentfill}%
\pgfsetfillopacity{0.700000}%
\pgfsetlinewidth{0.000000pt}%
\definecolor{currentstroke}{rgb}{0.000000,0.000000,0.000000}%
\pgfsetstrokecolor{currentstroke}%
\pgfsetdash{}{0pt}%
\pgfpathmoveto{\pgfqpoint{3.968339in}{1.234689in}}%
\pgfpathlineto{\pgfqpoint{3.982052in}{1.228627in}}%
\pgfpathlineto{\pgfqpoint{3.995771in}{1.222680in}}%
\pgfpathlineto{\pgfqpoint{4.009494in}{1.216849in}}%
\pgfpathlineto{\pgfqpoint{4.023223in}{1.211132in}}%
\pgfpathlineto{\pgfqpoint{4.031177in}{1.216384in}}%
\pgfpathlineto{\pgfqpoint{4.039123in}{1.221824in}}%
\pgfpathlineto{\pgfqpoint{4.047062in}{1.227449in}}%
\pgfpathlineto{\pgfqpoint{4.054993in}{1.233253in}}%
\pgfpathlineto{\pgfqpoint{4.041282in}{1.238497in}}%
\pgfpathlineto{\pgfqpoint{4.027577in}{1.243855in}}%
\pgfpathlineto{\pgfqpoint{4.013878in}{1.249328in}}%
\pgfpathlineto{\pgfqpoint{4.000184in}{1.254916in}}%
\pgfpathlineto{\pgfqpoint{3.992234in}{1.249578in}}%
\pgfpathlineto{\pgfqpoint{3.984276in}{1.244425in}}%
\pgfpathlineto{\pgfqpoint{3.976312in}{1.239461in}}%
\pgfpathlineto{\pgfqpoint{3.968339in}{1.234689in}}%
\pgfpathclose%
\pgfusepath{fill}%
\end{pgfscope}%
\begin{pgfscope}%
\pgfpathrectangle{\pgfqpoint{1.254980in}{0.150000in}}{\pgfqpoint{5.490039in}{5.490039in}}%
\pgfusepath{clip}%
\pgfsetbuttcap%
\pgfsetroundjoin%
\definecolor{currentfill}{rgb}{0.267004,0.004874,0.329415}%
\pgfsetfillcolor{currentfill}%
\pgfsetfillopacity{0.700000}%
\pgfsetlinewidth{0.000000pt}%
\definecolor{currentstroke}{rgb}{0.000000,0.000000,0.000000}%
\pgfsetstrokecolor{currentstroke}%
\pgfsetdash{}{0pt}%
\pgfpathmoveto{\pgfqpoint{4.109897in}{1.213423in}}%
\pgfpathlineto{\pgfqpoint{4.123638in}{1.208750in}}%
\pgfpathlineto{\pgfqpoint{4.137385in}{1.204191in}}%
\pgfpathlineto{\pgfqpoint{4.151139in}{1.199744in}}%
\pgfpathlineto{\pgfqpoint{4.164900in}{1.195411in}}%
\pgfpathlineto{\pgfqpoint{4.172793in}{1.202317in}}%
\pgfpathlineto{\pgfqpoint{4.180681in}{1.209388in}}%
\pgfpathlineto{\pgfqpoint{4.188562in}{1.216619in}}%
\pgfpathlineto{\pgfqpoint{4.196437in}{1.224006in}}%
\pgfpathlineto{\pgfqpoint{4.182691in}{1.227883in}}%
\pgfpathlineto{\pgfqpoint{4.168951in}{1.231873in}}%
\pgfpathlineto{\pgfqpoint{4.155218in}{1.235976in}}%
\pgfpathlineto{\pgfqpoint{4.141492in}{1.240193in}}%
\pgfpathlineto{\pgfqpoint{4.133603in}{1.233256in}}%
\pgfpathlineto{\pgfqpoint{4.125707in}{1.226479in}}%
\pgfpathlineto{\pgfqpoint{4.117805in}{1.219867in}}%
\pgfpathlineto{\pgfqpoint{4.109897in}{1.213423in}}%
\pgfpathclose%
\pgfusepath{fill}%
\end{pgfscope}%
\begin{pgfscope}%
\pgfpathrectangle{\pgfqpoint{1.254980in}{0.150000in}}{\pgfqpoint{5.490039in}{5.490039in}}%
\pgfusepath{clip}%
\pgfsetbuttcap%
\pgfsetroundjoin%
\definecolor{currentfill}{rgb}{0.265145,0.232956,0.516599}%
\pgfsetfillcolor{currentfill}%
\pgfsetfillopacity{0.700000}%
\pgfsetlinewidth{0.000000pt}%
\definecolor{currentstroke}{rgb}{0.000000,0.000000,0.000000}%
\pgfsetstrokecolor{currentstroke}%
\pgfsetdash{}{0pt}%
\pgfpathmoveto{\pgfqpoint{3.213507in}{1.701189in}}%
\pgfpathlineto{\pgfqpoint{3.227209in}{1.687606in}}%
\pgfpathlineto{\pgfqpoint{3.240911in}{1.674158in}}%
\pgfpathlineto{\pgfqpoint{3.254611in}{1.660846in}}%
\pgfpathlineto{\pgfqpoint{3.268310in}{1.647669in}}%
\pgfpathlineto{\pgfqpoint{3.276743in}{1.643861in}}%
\pgfpathlineto{\pgfqpoint{3.285162in}{1.640351in}}%
\pgfpathlineto{\pgfqpoint{3.293565in}{1.637132in}}%
\pgfpathlineto{\pgfqpoint{3.301953in}{1.634200in}}%
\pgfpathlineto{\pgfqpoint{3.288294in}{1.646835in}}%
\pgfpathlineto{\pgfqpoint{3.274634in}{1.659604in}}%
\pgfpathlineto{\pgfqpoint{3.260973in}{1.672508in}}%
\pgfpathlineto{\pgfqpoint{3.247312in}{1.685547in}}%
\pgfpathlineto{\pgfqpoint{3.238884in}{1.689015in}}%
\pgfpathlineto{\pgfqpoint{3.230440in}{1.692774in}}%
\pgfpathlineto{\pgfqpoint{3.221982in}{1.696831in}}%
\pgfpathlineto{\pgfqpoint{3.213507in}{1.701189in}}%
\pgfpathclose%
\pgfusepath{fill}%
\end{pgfscope}%
\begin{pgfscope}%
\pgfpathrectangle{\pgfqpoint{1.254980in}{0.150000in}}{\pgfqpoint{5.490039in}{5.490039in}}%
\pgfusepath{clip}%
\pgfsetbuttcap%
\pgfsetroundjoin%
\definecolor{currentfill}{rgb}{0.253935,0.265254,0.529983}%
\pgfsetfillcolor{currentfill}%
\pgfsetfillopacity{0.700000}%
\pgfsetlinewidth{0.000000pt}%
\definecolor{currentstroke}{rgb}{0.000000,0.000000,0.000000}%
\pgfsetstrokecolor{currentstroke}%
\pgfsetdash{}{0pt}%
\pgfpathmoveto{\pgfqpoint{5.062197in}{1.693087in}}%
\pgfpathlineto{\pgfqpoint{5.076323in}{1.697456in}}%
\pgfpathlineto{\pgfqpoint{5.090462in}{1.701936in}}%
\pgfpathlineto{\pgfqpoint{5.104615in}{1.706526in}}%
\pgfpathlineto{\pgfqpoint{5.118782in}{1.711227in}}%
\pgfpathlineto{\pgfqpoint{5.126439in}{1.725441in}}%
\pgfpathlineto{\pgfqpoint{5.134091in}{1.739644in}}%
\pgfpathlineto{\pgfqpoint{5.141739in}{1.753833in}}%
\pgfpathlineto{\pgfqpoint{5.149383in}{1.768006in}}%
\pgfpathlineto{\pgfqpoint{5.135213in}{1.763018in}}%
\pgfpathlineto{\pgfqpoint{5.121057in}{1.758140in}}%
\pgfpathlineto{\pgfqpoint{5.106915in}{1.753373in}}%
\pgfpathlineto{\pgfqpoint{5.092787in}{1.748716in}}%
\pgfpathlineto{\pgfqpoint{5.085146in}{1.734824in}}%
\pgfpathlineto{\pgfqpoint{5.077501in}{1.720921in}}%
\pgfpathlineto{\pgfqpoint{5.069851in}{1.707007in}}%
\pgfpathlineto{\pgfqpoint{5.062197in}{1.693087in}}%
\pgfpathclose%
\pgfusepath{fill}%
\end{pgfscope}%
\begin{pgfscope}%
\pgfpathrectangle{\pgfqpoint{1.254980in}{0.150000in}}{\pgfqpoint{5.490039in}{5.490039in}}%
\pgfusepath{clip}%
\pgfsetbuttcap%
\pgfsetroundjoin%
\definecolor{currentfill}{rgb}{0.166383,0.690856,0.496502}%
\pgfsetfillcolor{currentfill}%
\pgfsetfillopacity{0.700000}%
\pgfsetlinewidth{0.000000pt}%
\definecolor{currentstroke}{rgb}{0.000000,0.000000,0.000000}%
\pgfsetstrokecolor{currentstroke}%
\pgfsetdash{}{0pt}%
\pgfpathmoveto{\pgfqpoint{2.309539in}{2.928614in}}%
\pgfpathlineto{\pgfqpoint{2.323569in}{2.904569in}}%
\pgfpathlineto{\pgfqpoint{2.337588in}{2.880721in}}%
\pgfpathlineto{\pgfqpoint{2.351596in}{2.857067in}}%
\pgfpathlineto{\pgfqpoint{2.365594in}{2.833606in}}%
\pgfpathlineto{\pgfqpoint{2.374841in}{2.821503in}}%
\pgfpathlineto{\pgfqpoint{2.384062in}{2.809769in}}%
\pgfpathlineto{\pgfqpoint{2.393257in}{2.798400in}}%
\pgfpathlineto{\pgfqpoint{2.402428in}{2.787392in}}%
\pgfpathlineto{\pgfqpoint{2.388495in}{2.810268in}}%
\pgfpathlineto{\pgfqpoint{2.374552in}{2.833337in}}%
\pgfpathlineto{\pgfqpoint{2.360599in}{2.856598in}}%
\pgfpathlineto{\pgfqpoint{2.346635in}{2.880055in}}%
\pgfpathlineto{\pgfqpoint{2.337400in}{2.891641in}}%
\pgfpathlineto{\pgfqpoint{2.328140in}{2.903592in}}%
\pgfpathlineto{\pgfqpoint{2.318853in}{2.915915in}}%
\pgfpathlineto{\pgfqpoint{2.309539in}{2.928614in}}%
\pgfpathclose%
\pgfusepath{fill}%
\end{pgfscope}%
\begin{pgfscope}%
\pgfpathrectangle{\pgfqpoint{1.254980in}{0.150000in}}{\pgfqpoint{5.490039in}{5.490039in}}%
\pgfusepath{clip}%
\pgfsetbuttcap%
\pgfsetroundjoin%
\definecolor{currentfill}{rgb}{0.154815,0.493313,0.557840}%
\pgfsetfillcolor{currentfill}%
\pgfsetfillopacity{0.700000}%
\pgfsetlinewidth{0.000000pt}%
\definecolor{currentstroke}{rgb}{0.000000,0.000000,0.000000}%
\pgfsetstrokecolor{currentstroke}%
\pgfsetdash{}{0pt}%
\pgfpathmoveto{\pgfqpoint{2.663107in}{2.363489in}}%
\pgfpathlineto{\pgfqpoint{2.676957in}{2.343908in}}%
\pgfpathlineto{\pgfqpoint{2.690801in}{2.324493in}}%
\pgfpathlineto{\pgfqpoint{2.704638in}{2.305242in}}%
\pgfpathlineto{\pgfqpoint{2.718470in}{2.286156in}}%
\pgfpathlineto{\pgfqpoint{2.727386in}{2.276706in}}%
\pgfpathlineto{\pgfqpoint{2.736281in}{2.267607in}}%
\pgfpathlineto{\pgfqpoint{2.745154in}{2.258853in}}%
\pgfpathlineto{\pgfqpoint{2.754006in}{2.250438in}}%
\pgfpathlineto{\pgfqpoint{2.740230in}{2.268951in}}%
\pgfpathlineto{\pgfqpoint{2.726449in}{2.287627in}}%
\pgfpathlineto{\pgfqpoint{2.712661in}{2.306466in}}%
\pgfpathlineto{\pgfqpoint{2.698868in}{2.325470in}}%
\pgfpathlineto{\pgfqpoint{2.689961in}{2.334451in}}%
\pgfpathlineto{\pgfqpoint{2.681032in}{2.343778in}}%
\pgfpathlineto{\pgfqpoint{2.672080in}{2.353455in}}%
\pgfpathlineto{\pgfqpoint{2.663107in}{2.363489in}}%
\pgfpathclose%
\pgfusepath{fill}%
\end{pgfscope}%
\begin{pgfscope}%
\pgfpathrectangle{\pgfqpoint{1.254980in}{0.150000in}}{\pgfqpoint{5.490039in}{5.490039in}}%
\pgfusepath{clip}%
\pgfsetbuttcap%
\pgfsetroundjoin%
\definecolor{currentfill}{rgb}{0.278826,0.175490,0.483397}%
\pgfsetfillcolor{currentfill}%
\pgfsetfillopacity{0.700000}%
\pgfsetlinewidth{0.000000pt}%
\definecolor{currentstroke}{rgb}{0.000000,0.000000,0.000000}%
\pgfsetstrokecolor{currentstroke}%
\pgfsetdash{}{0pt}%
\pgfpathmoveto{\pgfqpoint{4.857554in}{1.502961in}}%
\pgfpathlineto{\pgfqpoint{4.871573in}{1.505495in}}%
\pgfpathlineto{\pgfqpoint{4.885604in}{1.508138in}}%
\pgfpathlineto{\pgfqpoint{4.899646in}{1.510892in}}%
\pgfpathlineto{\pgfqpoint{4.913701in}{1.513755in}}%
\pgfpathlineto{\pgfqpoint{4.921400in}{1.527241in}}%
\pgfpathlineto{\pgfqpoint{4.929094in}{1.540752in}}%
\pgfpathlineto{\pgfqpoint{4.936785in}{1.554286in}}%
\pgfpathlineto{\pgfqpoint{4.944472in}{1.567841in}}%
\pgfpathlineto{\pgfqpoint{4.930416in}{1.564644in}}%
\pgfpathlineto{\pgfqpoint{4.916372in}{1.561556in}}%
\pgfpathlineto{\pgfqpoint{4.902341in}{1.558579in}}%
\pgfpathlineto{\pgfqpoint{4.888322in}{1.555712in}}%
\pgfpathlineto{\pgfqpoint{4.880636in}{1.542485in}}%
\pgfpathlineto{\pgfqpoint{4.872946in}{1.529282in}}%
\pgfpathlineto{\pgfqpoint{4.865252in}{1.516107in}}%
\pgfpathlineto{\pgfqpoint{4.857554in}{1.502961in}}%
\pgfpathclose%
\pgfusepath{fill}%
\end{pgfscope}%
\begin{pgfscope}%
\pgfpathrectangle{\pgfqpoint{1.254980in}{0.150000in}}{\pgfqpoint{5.490039in}{5.490039in}}%
\pgfusepath{clip}%
\pgfsetbuttcap%
\pgfsetroundjoin%
\definecolor{currentfill}{rgb}{0.270595,0.214069,0.507052}%
\pgfsetfillcolor{currentfill}%
\pgfsetfillopacity{0.700000}%
\pgfsetlinewidth{0.000000pt}%
\definecolor{currentstroke}{rgb}{0.000000,0.000000,0.000000}%
\pgfsetstrokecolor{currentstroke}%
\pgfsetdash{}{0pt}%
\pgfpathmoveto{\pgfqpoint{3.268310in}{1.647669in}}%
\pgfpathlineto{\pgfqpoint{3.282009in}{1.634626in}}%
\pgfpathlineto{\pgfqpoint{3.295707in}{1.621716in}}%
\pgfpathlineto{\pgfqpoint{3.309405in}{1.608940in}}%
\pgfpathlineto{\pgfqpoint{3.323103in}{1.596296in}}%
\pgfpathlineto{\pgfqpoint{3.331496in}{1.593037in}}%
\pgfpathlineto{\pgfqpoint{3.339875in}{1.590070in}}%
\pgfpathlineto{\pgfqpoint{3.348239in}{1.587390in}}%
\pgfpathlineto{\pgfqpoint{3.356590in}{1.584991in}}%
\pgfpathlineto{\pgfqpoint{3.342931in}{1.597095in}}%
\pgfpathlineto{\pgfqpoint{3.329272in}{1.609331in}}%
\pgfpathlineto{\pgfqpoint{3.315613in}{1.621699in}}%
\pgfpathlineto{\pgfqpoint{3.301953in}{1.634200in}}%
\pgfpathlineto{\pgfqpoint{3.293565in}{1.637132in}}%
\pgfpathlineto{\pgfqpoint{3.285162in}{1.640351in}}%
\pgfpathlineto{\pgfqpoint{3.276743in}{1.643861in}}%
\pgfpathlineto{\pgfqpoint{3.268310in}{1.647669in}}%
\pgfpathclose%
\pgfusepath{fill}%
\end{pgfscope}%
\begin{pgfscope}%
\pgfpathrectangle{\pgfqpoint{1.254980in}{0.150000in}}{\pgfqpoint{5.490039in}{5.490039in}}%
\pgfusepath{clip}%
\pgfsetbuttcap%
\pgfsetroundjoin%
\definecolor{currentfill}{rgb}{0.273809,0.031497,0.358853}%
\pgfsetfillcolor{currentfill}%
\pgfsetfillopacity{0.700000}%
\pgfsetlinewidth{0.000000pt}%
\definecolor{currentstroke}{rgb}{0.000000,0.000000,0.000000}%
\pgfsetstrokecolor{currentstroke}%
\pgfsetdash{}{0pt}%
\pgfpathmoveto{\pgfqpoint{3.826666in}{1.274221in}}%
\pgfpathlineto{\pgfqpoint{3.840364in}{1.266734in}}%
\pgfpathlineto{\pgfqpoint{3.854066in}{1.259365in}}%
\pgfpathlineto{\pgfqpoint{3.867773in}{1.252114in}}%
\pgfpathlineto{\pgfqpoint{3.881483in}{1.244979in}}%
\pgfpathlineto{\pgfqpoint{3.889509in}{1.248442in}}%
\pgfpathlineto{\pgfqpoint{3.897527in}{1.252119in}}%
\pgfpathlineto{\pgfqpoint{3.905536in}{1.256004in}}%
\pgfpathlineto{\pgfqpoint{3.913537in}{1.260095in}}%
\pgfpathlineto{\pgfqpoint{3.899849in}{1.266738in}}%
\pgfpathlineto{\pgfqpoint{3.886165in}{1.273498in}}%
\pgfpathlineto{\pgfqpoint{3.872486in}{1.280375in}}%
\pgfpathlineto{\pgfqpoint{3.858811in}{1.287370in}}%
\pgfpathlineto{\pgfqpoint{3.850788in}{1.283764in}}%
\pgfpathlineto{\pgfqpoint{3.842756in}{1.280368in}}%
\pgfpathlineto{\pgfqpoint{3.834716in}{1.277185in}}%
\pgfpathlineto{\pgfqpoint{3.826666in}{1.274221in}}%
\pgfpathclose%
\pgfusepath{fill}%
\end{pgfscope}%
\begin{pgfscope}%
\pgfpathrectangle{\pgfqpoint{1.254980in}{0.150000in}}{\pgfqpoint{5.490039in}{5.490039in}}%
\pgfusepath{clip}%
\pgfsetbuttcap%
\pgfsetroundjoin%
\definecolor{currentfill}{rgb}{0.268510,0.009605,0.335427}%
\pgfsetfillcolor{currentfill}%
\pgfsetfillopacity{0.700000}%
\pgfsetlinewidth{0.000000pt}%
\definecolor{currentstroke}{rgb}{0.000000,0.000000,0.000000}%
\pgfsetstrokecolor{currentstroke}%
\pgfsetdash{}{0pt}%
\pgfpathmoveto{\pgfqpoint{4.251492in}{1.209622in}}%
\pgfpathlineto{\pgfqpoint{4.265274in}{1.206307in}}%
\pgfpathlineto{\pgfqpoint{4.279064in}{1.203103in}}%
\pgfpathlineto{\pgfqpoint{4.292861in}{1.200011in}}%
\pgfpathlineto{\pgfqpoint{4.306665in}{1.197030in}}%
\pgfpathlineto{\pgfqpoint{4.314510in}{1.205463in}}%
\pgfpathlineto{\pgfqpoint{4.322350in}{1.214036in}}%
\pgfpathlineto{\pgfqpoint{4.330185in}{1.222747in}}%
\pgfpathlineto{\pgfqpoint{4.338014in}{1.231592in}}%
\pgfpathlineto{\pgfqpoint{4.324220in}{1.234132in}}%
\pgfpathlineto{\pgfqpoint{4.310434in}{1.236784in}}%
\pgfpathlineto{\pgfqpoint{4.296656in}{1.239548in}}%
\pgfpathlineto{\pgfqpoint{4.282885in}{1.242423in}}%
\pgfpathlineto{\pgfqpoint{4.275045in}{1.234013in}}%
\pgfpathlineto{\pgfqpoint{4.267200in}{1.225740in}}%
\pgfpathlineto{\pgfqpoint{4.259349in}{1.217609in}}%
\pgfpathlineto{\pgfqpoint{4.251492in}{1.209622in}}%
\pgfpathclose%
\pgfusepath{fill}%
\end{pgfscope}%
\begin{pgfscope}%
\pgfpathrectangle{\pgfqpoint{1.254980in}{0.150000in}}{\pgfqpoint{5.490039in}{5.490039in}}%
\pgfusepath{clip}%
\pgfsetbuttcap%
\pgfsetroundjoin%
\definecolor{currentfill}{rgb}{0.143343,0.522773,0.556295}%
\pgfsetfillcolor{currentfill}%
\pgfsetfillopacity{0.700000}%
\pgfsetlinewidth{0.000000pt}%
\definecolor{currentstroke}{rgb}{0.000000,0.000000,0.000000}%
\pgfsetstrokecolor{currentstroke}%
\pgfsetdash{}{0pt}%
\pgfpathmoveto{\pgfqpoint{2.607638in}{2.443490in}}%
\pgfpathlineto{\pgfqpoint{2.621516in}{2.423236in}}%
\pgfpathlineto{\pgfqpoint{2.635386in}{2.403152in}}%
\pgfpathlineto{\pgfqpoint{2.649250in}{2.383237in}}%
\pgfpathlineto{\pgfqpoint{2.663107in}{2.363489in}}%
\pgfpathlineto{\pgfqpoint{2.672080in}{2.353455in}}%
\pgfpathlineto{\pgfqpoint{2.681032in}{2.343778in}}%
\pgfpathlineto{\pgfqpoint{2.689961in}{2.334451in}}%
\pgfpathlineto{\pgfqpoint{2.698868in}{2.325470in}}%
\pgfpathlineto{\pgfqpoint{2.685068in}{2.344640in}}%
\pgfpathlineto{\pgfqpoint{2.671262in}{2.363976in}}%
\pgfpathlineto{\pgfqpoint{2.657449in}{2.383479in}}%
\pgfpathlineto{\pgfqpoint{2.643630in}{2.403152in}}%
\pgfpathlineto{\pgfqpoint{2.634666in}{2.412704in}}%
\pgfpathlineto{\pgfqpoint{2.625680in}{2.422608in}}%
\pgfpathlineto{\pgfqpoint{2.616671in}{2.432868in}}%
\pgfpathlineto{\pgfqpoint{2.607638in}{2.443490in}}%
\pgfpathclose%
\pgfusepath{fill}%
\end{pgfscope}%
\begin{pgfscope}%
\pgfpathrectangle{\pgfqpoint{1.254980in}{0.150000in}}{\pgfqpoint{5.490039in}{5.490039in}}%
\pgfusepath{clip}%
\pgfsetbuttcap%
\pgfsetroundjoin%
\definecolor{currentfill}{rgb}{0.281446,0.084320,0.407414}%
\pgfsetfillcolor{currentfill}%
\pgfsetfillopacity{0.700000}%
\pgfsetlinewidth{0.000000pt}%
\definecolor{currentstroke}{rgb}{0.000000,0.000000,0.000000}%
\pgfsetstrokecolor{currentstroke}%
\pgfsetdash{}{0pt}%
\pgfpathmoveto{\pgfqpoint{3.629956in}{1.369910in}}%
\pgfpathlineto{\pgfqpoint{3.643642in}{1.360471in}}%
\pgfpathlineto{\pgfqpoint{3.657330in}{1.351155in}}%
\pgfpathlineto{\pgfqpoint{3.671021in}{1.341960in}}%
\pgfpathlineto{\pgfqpoint{3.684715in}{1.332886in}}%
\pgfpathlineto{\pgfqpoint{3.692856in}{1.333901in}}%
\pgfpathlineto{\pgfqpoint{3.700987in}{1.335160in}}%
\pgfpathlineto{\pgfqpoint{3.709107in}{1.336660in}}%
\pgfpathlineto{\pgfqpoint{3.717217in}{1.338395in}}%
\pgfpathlineto{\pgfqpoint{3.703551in}{1.346956in}}%
\pgfpathlineto{\pgfqpoint{3.689888in}{1.355639in}}%
\pgfpathlineto{\pgfqpoint{3.676228in}{1.364444in}}%
\pgfpathlineto{\pgfqpoint{3.662571in}{1.373370in}}%
\pgfpathlineto{\pgfqpoint{3.654433in}{1.372141in}}%
\pgfpathlineto{\pgfqpoint{3.646285in}{1.371151in}}%
\pgfpathlineto{\pgfqpoint{3.638126in}{1.370406in}}%
\pgfpathlineto{\pgfqpoint{3.629956in}{1.369910in}}%
\pgfpathclose%
\pgfusepath{fill}%
\end{pgfscope}%
\begin{pgfscope}%
\pgfpathrectangle{\pgfqpoint{1.254980in}{0.150000in}}{\pgfqpoint{5.490039in}{5.490039in}}%
\pgfusepath{clip}%
\pgfsetbuttcap%
\pgfsetroundjoin%
\definecolor{currentfill}{rgb}{0.239346,0.300855,0.540844}%
\pgfsetfillcolor{currentfill}%
\pgfsetfillopacity{0.700000}%
\pgfsetlinewidth{0.000000pt}%
\definecolor{currentstroke}{rgb}{0.000000,0.000000,0.000000}%
\pgfsetstrokecolor{currentstroke}%
\pgfsetdash{}{0pt}%
\pgfpathmoveto{\pgfqpoint{5.149383in}{1.768006in}}%
\pgfpathlineto{\pgfqpoint{5.163567in}{1.773106in}}%
\pgfpathlineto{\pgfqpoint{5.177765in}{1.778316in}}%
\pgfpathlineto{\pgfqpoint{5.191978in}{1.783636in}}%
\pgfpathlineto{\pgfqpoint{5.199620in}{1.798001in}}%
\pgfpathlineto{\pgfqpoint{5.207259in}{1.812342in}}%
\pgfpathlineto{\pgfqpoint{5.214892in}{1.826658in}}%
\pgfpathlineto{\pgfqpoint{5.222522in}{1.840946in}}%
\pgfpathlineto{\pgfqpoint{5.208305in}{1.835353in}}%
\pgfpathlineto{\pgfqpoint{5.194104in}{1.829871in}}%
\pgfpathlineto{\pgfqpoint{5.179916in}{1.824500in}}%
\pgfpathlineto{\pgfqpoint{5.172290in}{1.810411in}}%
\pgfpathlineto{\pgfqpoint{5.164658in}{1.796297in}}%
\pgfpathlineto{\pgfqpoint{5.157023in}{1.782162in}}%
\pgfpathlineto{\pgfqpoint{5.149383in}{1.768006in}}%
\pgfpathclose%
\pgfusepath{fill}%
\end{pgfscope}%
\begin{pgfscope}%
\pgfpathrectangle{\pgfqpoint{1.254980in}{0.150000in}}{\pgfqpoint{5.490039in}{5.490039in}}%
\pgfusepath{clip}%
\pgfsetbuttcap%
\pgfsetroundjoin%
\definecolor{currentfill}{rgb}{0.276194,0.190074,0.493001}%
\pgfsetfillcolor{currentfill}%
\pgfsetfillopacity{0.700000}%
\pgfsetlinewidth{0.000000pt}%
\definecolor{currentstroke}{rgb}{0.000000,0.000000,0.000000}%
\pgfsetstrokecolor{currentstroke}%
\pgfsetdash{}{0pt}%
\pgfpathmoveto{\pgfqpoint{3.323103in}{1.596296in}}%
\pgfpathlineto{\pgfqpoint{3.336800in}{1.583785in}}%
\pgfpathlineto{\pgfqpoint{3.350497in}{1.571405in}}%
\pgfpathlineto{\pgfqpoint{3.364194in}{1.559156in}}%
\pgfpathlineto{\pgfqpoint{3.377892in}{1.547037in}}%
\pgfpathlineto{\pgfqpoint{3.386247in}{1.544324in}}%
\pgfpathlineto{\pgfqpoint{3.394588in}{1.541898in}}%
\pgfpathlineto{\pgfqpoint{3.402915in}{1.539754in}}%
\pgfpathlineto{\pgfqpoint{3.411229in}{1.537888in}}%
\pgfpathlineto{\pgfqpoint{3.397568in}{1.549468in}}%
\pgfpathlineto{\pgfqpoint{3.383909in}{1.561179in}}%
\pgfpathlineto{\pgfqpoint{3.370249in}{1.573019in}}%
\pgfpathlineto{\pgfqpoint{3.356590in}{1.584991in}}%
\pgfpathlineto{\pgfqpoint{3.348239in}{1.587390in}}%
\pgfpathlineto{\pgfqpoint{3.339875in}{1.590070in}}%
\pgfpathlineto{\pgfqpoint{3.331496in}{1.593037in}}%
\pgfpathlineto{\pgfqpoint{3.323103in}{1.596296in}}%
\pgfpathclose%
\pgfusepath{fill}%
\end{pgfscope}%
\begin{pgfscope}%
\pgfpathrectangle{\pgfqpoint{1.254980in}{0.150000in}}{\pgfqpoint{5.490039in}{5.490039in}}%
\pgfusepath{clip}%
\pgfsetbuttcap%
\pgfsetroundjoin%
\definecolor{currentfill}{rgb}{0.506271,0.828786,0.300362}%
\pgfsetfillcolor{currentfill}%
\pgfsetfillopacity{0.700000}%
\pgfsetlinewidth{0.000000pt}%
\definecolor{currentstroke}{rgb}{0.000000,0.000000,0.000000}%
\pgfsetstrokecolor{currentstroke}%
\pgfsetdash{}{0pt}%
\pgfpathmoveto{\pgfqpoint{2.064876in}{3.392225in}}%
\pgfpathlineto{\pgfqpoint{2.079090in}{3.364574in}}%
\pgfpathlineto{\pgfqpoint{2.093289in}{3.337148in}}%
\pgfpathlineto{\pgfqpoint{2.107474in}{3.309945in}}%
\pgfpathlineto{\pgfqpoint{2.121645in}{3.282963in}}%
\pgfpathlineto{\pgfqpoint{2.131126in}{3.269396in}}%
\pgfpathlineto{\pgfqpoint{2.140579in}{3.256210in}}%
\pgfpathlineto{\pgfqpoint{2.150003in}{3.243398in}}%
\pgfpathlineto{\pgfqpoint{2.159400in}{3.230957in}}%
\pgfpathlineto{\pgfqpoint{2.145300in}{3.257349in}}%
\pgfpathlineto{\pgfqpoint{2.131187in}{3.283962in}}%
\pgfpathlineto{\pgfqpoint{2.117059in}{3.310795in}}%
\pgfpathlineto{\pgfqpoint{2.102918in}{3.337851in}}%
\pgfpathlineto{\pgfqpoint{2.093451in}{3.350875in}}%
\pgfpathlineto{\pgfqpoint{2.083955in}{3.364275in}}%
\pgfpathlineto{\pgfqpoint{2.074430in}{3.378057in}}%
\pgfpathlineto{\pgfqpoint{2.064876in}{3.392225in}}%
\pgfpathclose%
\pgfusepath{fill}%
\end{pgfscope}%
\begin{pgfscope}%
\pgfpathrectangle{\pgfqpoint{1.254980in}{0.150000in}}{\pgfqpoint{5.490039in}{5.490039in}}%
\pgfusepath{clip}%
\pgfsetbuttcap%
\pgfsetroundjoin%
\definecolor{currentfill}{rgb}{0.271828,0.209303,0.504434}%
\pgfsetfillcolor{currentfill}%
\pgfsetfillopacity{0.700000}%
\pgfsetlinewidth{0.000000pt}%
\definecolor{currentstroke}{rgb}{0.000000,0.000000,0.000000}%
\pgfsetstrokecolor{currentstroke}%
\pgfsetdash{}{0pt}%
\pgfpathmoveto{\pgfqpoint{4.944472in}{1.567841in}}%
\pgfpathlineto{\pgfqpoint{4.958541in}{1.571148in}}%
\pgfpathlineto{\pgfqpoint{4.972623in}{1.574565in}}%
\pgfpathlineto{\pgfqpoint{4.986717in}{1.578092in}}%
\pgfpathlineto{\pgfqpoint{5.000825in}{1.581729in}}%
\pgfpathlineto{\pgfqpoint{5.008510in}{1.595626in}}%
\pgfpathlineto{\pgfqpoint{5.016191in}{1.609534in}}%
\pgfpathlineto{\pgfqpoint{5.023869in}{1.623452in}}%
\pgfpathlineto{\pgfqpoint{5.031543in}{1.637375in}}%
\pgfpathlineto{\pgfqpoint{5.017433in}{1.633419in}}%
\pgfpathlineto{\pgfqpoint{5.003336in}{1.629573in}}%
\pgfpathlineto{\pgfqpoint{4.989253in}{1.625838in}}%
\pgfpathlineto{\pgfqpoint{4.975182in}{1.622212in}}%
\pgfpathlineto{\pgfqpoint{4.967510in}{1.608601in}}%
\pgfpathlineto{\pgfqpoint{4.959835in}{1.595001in}}%
\pgfpathlineto{\pgfqpoint{4.952156in}{1.581413in}}%
\pgfpathlineto{\pgfqpoint{4.944472in}{1.567841in}}%
\pgfpathclose%
\pgfusepath{fill}%
\end{pgfscope}%
\begin{pgfscope}%
\pgfpathrectangle{\pgfqpoint{1.254980in}{0.150000in}}{\pgfqpoint{5.490039in}{5.490039in}}%
\pgfusepath{clip}%
\pgfsetbuttcap%
\pgfsetroundjoin%
\definecolor{currentfill}{rgb}{0.220124,0.725509,0.466226}%
\pgfsetfillcolor{currentfill}%
\pgfsetfillopacity{0.700000}%
\pgfsetlinewidth{0.000000pt}%
\definecolor{currentstroke}{rgb}{0.000000,0.000000,0.000000}%
\pgfsetstrokecolor{currentstroke}%
\pgfsetdash{}{0pt}%
\pgfpathmoveto{\pgfqpoint{2.253309in}{3.026780in}}%
\pgfpathlineto{\pgfqpoint{2.267384in}{3.001938in}}%
\pgfpathlineto{\pgfqpoint{2.281447in}{2.977297in}}%
\pgfpathlineto{\pgfqpoint{2.295499in}{2.952856in}}%
\pgfpathlineto{\pgfqpoint{2.309539in}{2.928614in}}%
\pgfpathlineto{\pgfqpoint{2.318853in}{2.915915in}}%
\pgfpathlineto{\pgfqpoint{2.328140in}{2.903592in}}%
\pgfpathlineto{\pgfqpoint{2.337400in}{2.891641in}}%
\pgfpathlineto{\pgfqpoint{2.346635in}{2.880055in}}%
\pgfpathlineto{\pgfqpoint{2.332661in}{2.903707in}}%
\pgfpathlineto{\pgfqpoint{2.318676in}{2.927557in}}%
\pgfpathlineto{\pgfqpoint{2.304680in}{2.951606in}}%
\pgfpathlineto{\pgfqpoint{2.290673in}{2.975854in}}%
\pgfpathlineto{\pgfqpoint{2.281372in}{2.988022in}}%
\pgfpathlineto{\pgfqpoint{2.272045in}{3.000562in}}%
\pgfpathlineto{\pgfqpoint{2.262691in}{3.013480in}}%
\pgfpathlineto{\pgfqpoint{2.253309in}{3.026780in}}%
\pgfpathclose%
\pgfusepath{fill}%
\end{pgfscope}%
\begin{pgfscope}%
\pgfpathrectangle{\pgfqpoint{1.254980in}{0.150000in}}{\pgfqpoint{5.490039in}{5.490039in}}%
\pgfusepath{clip}%
\pgfsetbuttcap%
\pgfsetroundjoin%
\definecolor{currentfill}{rgb}{0.132444,0.552216,0.553018}%
\pgfsetfillcolor{currentfill}%
\pgfsetfillopacity{0.700000}%
\pgfsetlinewidth{0.000000pt}%
\definecolor{currentstroke}{rgb}{0.000000,0.000000,0.000000}%
\pgfsetstrokecolor{currentstroke}%
\pgfsetdash{}{0pt}%
\pgfpathmoveto{\pgfqpoint{2.552053in}{2.526222in}}%
\pgfpathlineto{\pgfqpoint{2.565961in}{2.505279in}}%
\pgfpathlineto{\pgfqpoint{2.579861in}{2.484510in}}%
\pgfpathlineto{\pgfqpoint{2.593753in}{2.463914in}}%
\pgfpathlineto{\pgfqpoint{2.607638in}{2.443490in}}%
\pgfpathlineto{\pgfqpoint{2.616671in}{2.432868in}}%
\pgfpathlineto{\pgfqpoint{2.625680in}{2.422608in}}%
\pgfpathlineto{\pgfqpoint{2.634666in}{2.412704in}}%
\pgfpathlineto{\pgfqpoint{2.643630in}{2.403152in}}%
\pgfpathlineto{\pgfqpoint{2.629804in}{2.422993in}}%
\pgfpathlineto{\pgfqpoint{2.615971in}{2.443006in}}%
\pgfpathlineto{\pgfqpoint{2.602131in}{2.463189in}}%
\pgfpathlineto{\pgfqpoint{2.588283in}{2.483546in}}%
\pgfpathlineto{\pgfqpoint{2.579261in}{2.493674in}}%
\pgfpathlineto{\pgfqpoint{2.570216in}{2.504159in}}%
\pgfpathlineto{\pgfqpoint{2.561146in}{2.515006in}}%
\pgfpathlineto{\pgfqpoint{2.552053in}{2.526222in}}%
\pgfpathclose%
\pgfusepath{fill}%
\end{pgfscope}%
\begin{pgfscope}%
\pgfpathrectangle{\pgfqpoint{1.254980in}{0.150000in}}{\pgfqpoint{5.490039in}{5.490039in}}%
\pgfusepath{clip}%
\pgfsetbuttcap%
\pgfsetroundjoin%
\definecolor{currentfill}{rgb}{0.268510,0.009605,0.335427}%
\pgfsetfillcolor{currentfill}%
\pgfsetfillopacity{0.700000}%
\pgfsetlinewidth{0.000000pt}%
\definecolor{currentstroke}{rgb}{0.000000,0.000000,0.000000}%
\pgfsetstrokecolor{currentstroke}%
\pgfsetdash{}{0pt}%
\pgfpathmoveto{\pgfqpoint{4.023223in}{1.211132in}}%
\pgfpathlineto{\pgfqpoint{4.036958in}{1.205530in}}%
\pgfpathlineto{\pgfqpoint{4.050698in}{1.200042in}}%
\pgfpathlineto{\pgfqpoint{4.064444in}{1.194669in}}%
\pgfpathlineto{\pgfqpoint{4.078196in}{1.189409in}}%
\pgfpathlineto{\pgfqpoint{4.086131in}{1.195141in}}%
\pgfpathlineto{\pgfqpoint{4.094060in}{1.201056in}}%
\pgfpathlineto{\pgfqpoint{4.101982in}{1.207152in}}%
\pgfpathlineto{\pgfqpoint{4.109897in}{1.213423in}}%
\pgfpathlineto{\pgfqpoint{4.096162in}{1.218210in}}%
\pgfpathlineto{\pgfqpoint{4.082433in}{1.223110in}}%
\pgfpathlineto{\pgfqpoint{4.068710in}{1.228125in}}%
\pgfpathlineto{\pgfqpoint{4.054993in}{1.233253in}}%
\pgfpathlineto{\pgfqpoint{4.047062in}{1.227449in}}%
\pgfpathlineto{\pgfqpoint{4.039123in}{1.221824in}}%
\pgfpathlineto{\pgfqpoint{4.031177in}{1.216384in}}%
\pgfpathlineto{\pgfqpoint{4.023223in}{1.211132in}}%
\pgfpathclose%
\pgfusepath{fill}%
\end{pgfscope}%
\begin{pgfscope}%
\pgfpathrectangle{\pgfqpoint{1.254980in}{0.150000in}}{\pgfqpoint{5.490039in}{5.490039in}}%
\pgfusepath{clip}%
\pgfsetbuttcap%
\pgfsetroundjoin%
\definecolor{currentfill}{rgb}{0.279566,0.067836,0.391917}%
\pgfsetfillcolor{currentfill}%
\pgfsetfillopacity{0.700000}%
\pgfsetlinewidth{0.000000pt}%
\definecolor{currentstroke}{rgb}{0.000000,0.000000,0.000000}%
\pgfsetstrokecolor{currentstroke}%
\pgfsetdash{}{0pt}%
\pgfpathmoveto{\pgfqpoint{4.566474in}{1.294898in}}%
\pgfpathlineto{\pgfqpoint{4.580369in}{1.294628in}}%
\pgfpathlineto{\pgfqpoint{4.594274in}{1.294468in}}%
\pgfpathlineto{\pgfqpoint{4.608189in}{1.294418in}}%
\pgfpathlineto{\pgfqpoint{4.622114in}{1.294478in}}%
\pgfpathlineto{\pgfqpoint{4.629876in}{1.305967in}}%
\pgfpathlineto{\pgfqpoint{4.637634in}{1.317540in}}%
\pgfpathlineto{\pgfqpoint{4.645387in}{1.329193in}}%
\pgfpathlineto{\pgfqpoint{4.653137in}{1.340924in}}%
\pgfpathlineto{\pgfqpoint{4.639216in}{1.340470in}}%
\pgfpathlineto{\pgfqpoint{4.625305in}{1.340125in}}%
\pgfpathlineto{\pgfqpoint{4.611404in}{1.339891in}}%
\pgfpathlineto{\pgfqpoint{4.597513in}{1.339766in}}%
\pgfpathlineto{\pgfqpoint{4.589760in}{1.328424in}}%
\pgfpathlineto{\pgfqpoint{4.582002in}{1.317163in}}%
\pgfpathlineto{\pgfqpoint{4.574240in}{1.305986in}}%
\pgfpathlineto{\pgfqpoint{4.566474in}{1.294898in}}%
\pgfpathclose%
\pgfusepath{fill}%
\end{pgfscope}%
\begin{pgfscope}%
\pgfpathrectangle{\pgfqpoint{1.254980in}{0.150000in}}{\pgfqpoint{5.490039in}{5.490039in}}%
\pgfusepath{clip}%
\pgfsetbuttcap%
\pgfsetroundjoin%
\definecolor{currentfill}{rgb}{0.279574,0.170599,0.479997}%
\pgfsetfillcolor{currentfill}%
\pgfsetfillopacity{0.700000}%
\pgfsetlinewidth{0.000000pt}%
\definecolor{currentstroke}{rgb}{0.000000,0.000000,0.000000}%
\pgfsetstrokecolor{currentstroke}%
\pgfsetdash{}{0pt}%
\pgfpathmoveto{\pgfqpoint{3.377892in}{1.547037in}}%
\pgfpathlineto{\pgfqpoint{3.391590in}{1.535049in}}%
\pgfpathlineto{\pgfqpoint{3.405288in}{1.523190in}}%
\pgfpathlineto{\pgfqpoint{3.418986in}{1.511460in}}%
\pgfpathlineto{\pgfqpoint{3.432685in}{1.499859in}}%
\pgfpathlineto{\pgfqpoint{3.441003in}{1.497690in}}%
\pgfpathlineto{\pgfqpoint{3.449308in}{1.495804in}}%
\pgfpathlineto{\pgfqpoint{3.457599in}{1.494194in}}%
\pgfpathlineto{\pgfqpoint{3.465878in}{1.492857in}}%
\pgfpathlineto{\pgfqpoint{3.452214in}{1.503922in}}%
\pgfpathlineto{\pgfqpoint{3.438552in}{1.515115in}}%
\pgfpathlineto{\pgfqpoint{3.424890in}{1.526437in}}%
\pgfpathlineto{\pgfqpoint{3.411229in}{1.537888in}}%
\pgfpathlineto{\pgfqpoint{3.402915in}{1.539754in}}%
\pgfpathlineto{\pgfqpoint{3.394588in}{1.541898in}}%
\pgfpathlineto{\pgfqpoint{3.386247in}{1.544324in}}%
\pgfpathlineto{\pgfqpoint{3.377892in}{1.547037in}}%
\pgfpathclose%
\pgfusepath{fill}%
\end{pgfscope}%
\begin{pgfscope}%
\pgfpathrectangle{\pgfqpoint{1.254980in}{0.150000in}}{\pgfqpoint{5.490039in}{5.490039in}}%
\pgfusepath{clip}%
\pgfsetbuttcap%
\pgfsetroundjoin%
\definecolor{currentfill}{rgb}{0.282327,0.094955,0.417331}%
\pgfsetfillcolor{currentfill}%
\pgfsetfillopacity{0.700000}%
\pgfsetlinewidth{0.000000pt}%
\definecolor{currentstroke}{rgb}{0.000000,0.000000,0.000000}%
\pgfsetstrokecolor{currentstroke}%
\pgfsetdash{}{0pt}%
\pgfpathmoveto{\pgfqpoint{4.653137in}{1.340924in}}%
\pgfpathlineto{\pgfqpoint{4.667069in}{1.341488in}}%
\pgfpathlineto{\pgfqpoint{4.681011in}{1.342161in}}%
\pgfpathlineto{\pgfqpoint{4.694964in}{1.342944in}}%
\pgfpathlineto{\pgfqpoint{4.708928in}{1.343837in}}%
\pgfpathlineto{\pgfqpoint{4.716672in}{1.356028in}}%
\pgfpathlineto{\pgfqpoint{4.724411in}{1.368286in}}%
\pgfpathlineto{\pgfqpoint{4.732147in}{1.380608in}}%
\pgfpathlineto{\pgfqpoint{4.739879in}{1.392992in}}%
\pgfpathlineto{\pgfqpoint{4.725918in}{1.391719in}}%
\pgfpathlineto{\pgfqpoint{4.711967in}{1.390557in}}%
\pgfpathlineto{\pgfqpoint{4.698027in}{1.389504in}}%
\pgfpathlineto{\pgfqpoint{4.684097in}{1.388561in}}%
\pgfpathlineto{\pgfqpoint{4.676363in}{1.376551in}}%
\pgfpathlineto{\pgfqpoint{4.668625in}{1.364606in}}%
\pgfpathlineto{\pgfqpoint{4.660883in}{1.352729in}}%
\pgfpathlineto{\pgfqpoint{4.653137in}{1.340924in}}%
\pgfpathclose%
\pgfusepath{fill}%
\end{pgfscope}%
\begin{pgfscope}%
\pgfpathrectangle{\pgfqpoint{1.254980in}{0.150000in}}{\pgfqpoint{5.490039in}{5.490039in}}%
\pgfusepath{clip}%
\pgfsetbuttcap%
\pgfsetroundjoin%
\definecolor{currentfill}{rgb}{0.276022,0.044167,0.370164}%
\pgfsetfillcolor{currentfill}%
\pgfsetfillopacity{0.700000}%
\pgfsetlinewidth{0.000000pt}%
\definecolor{currentstroke}{rgb}{0.000000,0.000000,0.000000}%
\pgfsetstrokecolor{currentstroke}%
\pgfsetdash{}{0pt}%
\pgfpathmoveto{\pgfqpoint{4.479863in}{1.255303in}}%
\pgfpathlineto{\pgfqpoint{4.493725in}{1.254183in}}%
\pgfpathlineto{\pgfqpoint{4.507597in}{1.253174in}}%
\pgfpathlineto{\pgfqpoint{4.521478in}{1.252275in}}%
\pgfpathlineto{\pgfqpoint{4.535369in}{1.251486in}}%
\pgfpathlineto{\pgfqpoint{4.543152in}{1.262191in}}%
\pgfpathlineto{\pgfqpoint{4.550930in}{1.272997in}}%
\pgfpathlineto{\pgfqpoint{4.558704in}{1.283900in}}%
\pgfpathlineto{\pgfqpoint{4.566474in}{1.294898in}}%
\pgfpathlineto{\pgfqpoint{4.552589in}{1.295277in}}%
\pgfpathlineto{\pgfqpoint{4.538714in}{1.295767in}}%
\pgfpathlineto{\pgfqpoint{4.524848in}{1.296366in}}%
\pgfpathlineto{\pgfqpoint{4.510991in}{1.297076in}}%
\pgfpathlineto{\pgfqpoint{4.503216in}{1.286482in}}%
\pgfpathlineto{\pgfqpoint{4.495436in}{1.275986in}}%
\pgfpathlineto{\pgfqpoint{4.487651in}{1.265592in}}%
\pgfpathlineto{\pgfqpoint{4.479863in}{1.255303in}}%
\pgfpathclose%
\pgfusepath{fill}%
\end{pgfscope}%
\begin{pgfscope}%
\pgfpathrectangle{\pgfqpoint{1.254980in}{0.150000in}}{\pgfqpoint{5.490039in}{5.490039in}}%
\pgfusepath{clip}%
\pgfsetbuttcap%
\pgfsetroundjoin%
\definecolor{currentfill}{rgb}{0.267004,0.004874,0.329415}%
\pgfsetfillcolor{currentfill}%
\pgfsetfillopacity{0.700000}%
\pgfsetlinewidth{0.000000pt}%
\definecolor{currentstroke}{rgb}{0.000000,0.000000,0.000000}%
\pgfsetstrokecolor{currentstroke}%
\pgfsetdash{}{0pt}%
\pgfpathmoveto{\pgfqpoint{4.164900in}{1.195411in}}%
\pgfpathlineto{\pgfqpoint{4.178667in}{1.191190in}}%
\pgfpathlineto{\pgfqpoint{4.192441in}{1.187081in}}%
\pgfpathlineto{\pgfqpoint{4.206222in}{1.183085in}}%
\pgfpathlineto{\pgfqpoint{4.220009in}{1.179201in}}%
\pgfpathlineto{\pgfqpoint{4.227889in}{1.186570in}}%
\pgfpathlineto{\pgfqpoint{4.235762in}{1.194099in}}%
\pgfpathlineto{\pgfqpoint{4.243630in}{1.201785in}}%
\pgfpathlineto{\pgfqpoint{4.251492in}{1.209622in}}%
\pgfpathlineto{\pgfqpoint{4.237718in}{1.213050in}}%
\pgfpathlineto{\pgfqpoint{4.223950in}{1.216590in}}%
\pgfpathlineto{\pgfqpoint{4.210190in}{1.220242in}}%
\pgfpathlineto{\pgfqpoint{4.196437in}{1.224006in}}%
\pgfpathlineto{\pgfqpoint{4.188562in}{1.216619in}}%
\pgfpathlineto{\pgfqpoint{4.180681in}{1.209388in}}%
\pgfpathlineto{\pgfqpoint{4.172793in}{1.202317in}}%
\pgfpathlineto{\pgfqpoint{4.164900in}{1.195411in}}%
\pgfpathclose%
\pgfusepath{fill}%
\end{pgfscope}%
\begin{pgfscope}%
\pgfpathrectangle{\pgfqpoint{1.254980in}{0.150000in}}{\pgfqpoint{5.490039in}{5.490039in}}%
\pgfusepath{clip}%
\pgfsetbuttcap%
\pgfsetroundjoin%
\definecolor{currentfill}{rgb}{0.280267,0.073417,0.397163}%
\pgfsetfillcolor{currentfill}%
\pgfsetfillopacity{0.700000}%
\pgfsetlinewidth{0.000000pt}%
\definecolor{currentstroke}{rgb}{0.000000,0.000000,0.000000}%
\pgfsetstrokecolor{currentstroke}%
\pgfsetdash{}{0pt}%
\pgfpathmoveto{\pgfqpoint{3.684715in}{1.332886in}}%
\pgfpathlineto{\pgfqpoint{3.698412in}{1.323934in}}%
\pgfpathlineto{\pgfqpoint{3.712112in}{1.315102in}}%
\pgfpathlineto{\pgfqpoint{3.725814in}{1.306391in}}%
\pgfpathlineto{\pgfqpoint{3.739520in}{1.297799in}}%
\pgfpathlineto{\pgfqpoint{3.747633in}{1.299332in}}%
\pgfpathlineto{\pgfqpoint{3.755736in}{1.301105in}}%
\pgfpathlineto{\pgfqpoint{3.763829in}{1.303113in}}%
\pgfpathlineto{\pgfqpoint{3.771913in}{1.305352in}}%
\pgfpathlineto{\pgfqpoint{3.758234in}{1.313433in}}%
\pgfpathlineto{\pgfqpoint{3.744558in}{1.321633in}}%
\pgfpathlineto{\pgfqpoint{3.730886in}{1.329954in}}%
\pgfpathlineto{\pgfqpoint{3.717217in}{1.338395in}}%
\pgfpathlineto{\pgfqpoint{3.709107in}{1.336660in}}%
\pgfpathlineto{\pgfqpoint{3.700987in}{1.335160in}}%
\pgfpathlineto{\pgfqpoint{3.692856in}{1.333901in}}%
\pgfpathlineto{\pgfqpoint{3.684715in}{1.332886in}}%
\pgfpathclose%
\pgfusepath{fill}%
\end{pgfscope}%
\begin{pgfscope}%
\pgfpathrectangle{\pgfqpoint{1.254980in}{0.150000in}}{\pgfqpoint{5.490039in}{5.490039in}}%
\pgfusepath{clip}%
\pgfsetbuttcap%
\pgfsetroundjoin%
\definecolor{currentfill}{rgb}{0.272594,0.025563,0.353093}%
\pgfsetfillcolor{currentfill}%
\pgfsetfillopacity{0.700000}%
\pgfsetlinewidth{0.000000pt}%
\definecolor{currentstroke}{rgb}{0.000000,0.000000,0.000000}%
\pgfsetstrokecolor{currentstroke}%
\pgfsetdash{}{0pt}%
\pgfpathmoveto{\pgfqpoint{3.881483in}{1.244979in}}%
\pgfpathlineto{\pgfqpoint{3.895198in}{1.237962in}}%
\pgfpathlineto{\pgfqpoint{3.908918in}{1.231061in}}%
\pgfpathlineto{\pgfqpoint{3.922642in}{1.224277in}}%
\pgfpathlineto{\pgfqpoint{3.936371in}{1.217608in}}%
\pgfpathlineto{\pgfqpoint{3.944375in}{1.221569in}}%
\pgfpathlineto{\pgfqpoint{3.952371in}{1.225739in}}%
\pgfpathlineto{\pgfqpoint{3.960359in}{1.230113in}}%
\pgfpathlineto{\pgfqpoint{3.968339in}{1.234689in}}%
\pgfpathlineto{\pgfqpoint{3.954631in}{1.240866in}}%
\pgfpathlineto{\pgfqpoint{3.940928in}{1.247160in}}%
\pgfpathlineto{\pgfqpoint{3.927230in}{1.253569in}}%
\pgfpathlineto{\pgfqpoint{3.913537in}{1.260095in}}%
\pgfpathlineto{\pgfqpoint{3.905536in}{1.256004in}}%
\pgfpathlineto{\pgfqpoint{3.897527in}{1.252119in}}%
\pgfpathlineto{\pgfqpoint{3.889509in}{1.248442in}}%
\pgfpathlineto{\pgfqpoint{3.881483in}{1.244979in}}%
\pgfpathclose%
\pgfusepath{fill}%
\end{pgfscope}%
\begin{pgfscope}%
\pgfpathrectangle{\pgfqpoint{1.254980in}{0.150000in}}{\pgfqpoint{5.490039in}{5.490039in}}%
\pgfusepath{clip}%
\pgfsetbuttcap%
\pgfsetroundjoin%
\definecolor{currentfill}{rgb}{0.283187,0.125848,0.444960}%
\pgfsetfillcolor{currentfill}%
\pgfsetfillopacity{0.700000}%
\pgfsetlinewidth{0.000000pt}%
\definecolor{currentstroke}{rgb}{0.000000,0.000000,0.000000}%
\pgfsetstrokecolor{currentstroke}%
\pgfsetdash{}{0pt}%
\pgfpathmoveto{\pgfqpoint{4.739879in}{1.392992in}}%
\pgfpathlineto{\pgfqpoint{4.753852in}{1.394374in}}%
\pgfpathlineto{\pgfqpoint{4.767836in}{1.395865in}}%
\pgfpathlineto{\pgfqpoint{4.781832in}{1.397466in}}%
\pgfpathlineto{\pgfqpoint{4.795839in}{1.399176in}}%
\pgfpathlineto{\pgfqpoint{4.803566in}{1.411988in}}%
\pgfpathlineto{\pgfqpoint{4.811290in}{1.424853in}}%
\pgfpathlineto{\pgfqpoint{4.819010in}{1.437765in}}%
\pgfpathlineto{\pgfqpoint{4.826726in}{1.450724in}}%
\pgfpathlineto{\pgfqpoint{4.812720in}{1.448649in}}%
\pgfpathlineto{\pgfqpoint{4.798725in}{1.446683in}}%
\pgfpathlineto{\pgfqpoint{4.784742in}{1.444828in}}%
\pgfpathlineto{\pgfqpoint{4.770770in}{1.443082in}}%
\pgfpathlineto{\pgfqpoint{4.763053in}{1.430482in}}%
\pgfpathlineto{\pgfqpoint{4.755332in}{1.417931in}}%
\pgfpathlineto{\pgfqpoint{4.747608in}{1.405434in}}%
\pgfpathlineto{\pgfqpoint{4.739879in}{1.392992in}}%
\pgfpathclose%
\pgfusepath{fill}%
\end{pgfscope}%
\begin{pgfscope}%
\pgfpathrectangle{\pgfqpoint{1.254980in}{0.150000in}}{\pgfqpoint{5.490039in}{5.490039in}}%
\pgfusepath{clip}%
\pgfsetbuttcap%
\pgfsetroundjoin%
\definecolor{currentfill}{rgb}{0.272594,0.025563,0.353093}%
\pgfsetfillcolor{currentfill}%
\pgfsetfillopacity{0.700000}%
\pgfsetlinewidth{0.000000pt}%
\definecolor{currentstroke}{rgb}{0.000000,0.000000,0.000000}%
\pgfsetstrokecolor{currentstroke}%
\pgfsetdash{}{0pt}%
\pgfpathmoveto{\pgfqpoint{4.393271in}{1.222541in}}%
\pgfpathlineto{\pgfqpoint{4.407106in}{1.220555in}}%
\pgfpathlineto{\pgfqpoint{4.420949in}{1.218680in}}%
\pgfpathlineto{\pgfqpoint{4.434802in}{1.216915in}}%
\pgfpathlineto{\pgfqpoint{4.448663in}{1.215260in}}%
\pgfpathlineto{\pgfqpoint{4.456469in}{1.225097in}}%
\pgfpathlineto{\pgfqpoint{4.464272in}{1.235051in}}%
\pgfpathlineto{\pgfqpoint{4.472069in}{1.245121in}}%
\pgfpathlineto{\pgfqpoint{4.479863in}{1.255303in}}%
\pgfpathlineto{\pgfqpoint{4.466009in}{1.256532in}}%
\pgfpathlineto{\pgfqpoint{4.452164in}{1.257872in}}%
\pgfpathlineto{\pgfqpoint{4.438329in}{1.259322in}}%
\pgfpathlineto{\pgfqpoint{4.424502in}{1.260883in}}%
\pgfpathlineto{\pgfqpoint{4.416701in}{1.251121in}}%
\pgfpathlineto{\pgfqpoint{4.408896in}{1.241474in}}%
\pgfpathlineto{\pgfqpoint{4.401086in}{1.231946in}}%
\pgfpathlineto{\pgfqpoint{4.393271in}{1.222541in}}%
\pgfpathclose%
\pgfusepath{fill}%
\end{pgfscope}%
\begin{pgfscope}%
\pgfpathrectangle{\pgfqpoint{1.254980in}{0.150000in}}{\pgfqpoint{5.490039in}{5.490039in}}%
\pgfusepath{clip}%
\pgfsetbuttcap%
\pgfsetroundjoin%
\definecolor{currentfill}{rgb}{0.260571,0.246922,0.522828}%
\pgfsetfillcolor{currentfill}%
\pgfsetfillopacity{0.700000}%
\pgfsetlinewidth{0.000000pt}%
\definecolor{currentstroke}{rgb}{0.000000,0.000000,0.000000}%
\pgfsetstrokecolor{currentstroke}%
\pgfsetdash{}{0pt}%
\pgfpathmoveto{\pgfqpoint{5.031543in}{1.637375in}}%
\pgfpathlineto{\pgfqpoint{5.045666in}{1.641441in}}%
\pgfpathlineto{\pgfqpoint{5.059802in}{1.645618in}}%
\pgfpathlineto{\pgfqpoint{5.073952in}{1.649904in}}%
\pgfpathlineto{\pgfqpoint{5.088115in}{1.654300in}}%
\pgfpathlineto{\pgfqpoint{5.095788in}{1.668538in}}%
\pgfpathlineto{\pgfqpoint{5.103457in}{1.682773in}}%
\pgfpathlineto{\pgfqpoint{5.111121in}{1.697004in}}%
\pgfpathlineto{\pgfqpoint{5.118782in}{1.711227in}}%
\pgfpathlineto{\pgfqpoint{5.104615in}{1.706526in}}%
\pgfpathlineto{\pgfqpoint{5.090462in}{1.701936in}}%
\pgfpathlineto{\pgfqpoint{5.076323in}{1.697456in}}%
\pgfpathlineto{\pgfqpoint{5.062197in}{1.693087in}}%
\pgfpathlineto{\pgfqpoint{5.054540in}{1.679161in}}%
\pgfpathlineto{\pgfqpoint{5.046878in}{1.665232in}}%
\pgfpathlineto{\pgfqpoint{5.039212in}{1.651303in}}%
\pgfpathlineto{\pgfqpoint{5.031543in}{1.637375in}}%
\pgfpathclose%
\pgfusepath{fill}%
\end{pgfscope}%
\begin{pgfscope}%
\pgfpathrectangle{\pgfqpoint{1.254980in}{0.150000in}}{\pgfqpoint{5.490039in}{5.490039in}}%
\pgfusepath{clip}%
\pgfsetbuttcap%
\pgfsetroundjoin%
\definecolor{currentfill}{rgb}{0.122606,0.585371,0.546557}%
\pgfsetfillcolor{currentfill}%
\pgfsetfillopacity{0.700000}%
\pgfsetlinewidth{0.000000pt}%
\definecolor{currentstroke}{rgb}{0.000000,0.000000,0.000000}%
\pgfsetstrokecolor{currentstroke}%
\pgfsetdash{}{0pt}%
\pgfpathmoveto{\pgfqpoint{2.496342in}{2.611750in}}%
\pgfpathlineto{\pgfqpoint{2.510282in}{2.590102in}}%
\pgfpathlineto{\pgfqpoint{2.524214in}{2.568632in}}%
\pgfpathlineto{\pgfqpoint{2.538138in}{2.547339in}}%
\pgfpathlineto{\pgfqpoint{2.552053in}{2.526222in}}%
\pgfpathlineto{\pgfqpoint{2.561146in}{2.515006in}}%
\pgfpathlineto{\pgfqpoint{2.570216in}{2.504159in}}%
\pgfpathlineto{\pgfqpoint{2.579261in}{2.493674in}}%
\pgfpathlineto{\pgfqpoint{2.588283in}{2.483546in}}%
\pgfpathlineto{\pgfqpoint{2.574428in}{2.504076in}}%
\pgfpathlineto{\pgfqpoint{2.560566in}{2.524780in}}%
\pgfpathlineto{\pgfqpoint{2.546695in}{2.545661in}}%
\pgfpathlineto{\pgfqpoint{2.532817in}{2.566718in}}%
\pgfpathlineto{\pgfqpoint{2.523735in}{2.577426in}}%
\pgfpathlineto{\pgfqpoint{2.514629in}{2.588497in}}%
\pgfpathlineto{\pgfqpoint{2.505498in}{2.599937in}}%
\pgfpathlineto{\pgfqpoint{2.496342in}{2.611750in}}%
\pgfpathclose%
\pgfusepath{fill}%
\end{pgfscope}%
\begin{pgfscope}%
\pgfpathrectangle{\pgfqpoint{1.254980in}{0.150000in}}{\pgfqpoint{5.490039in}{5.490039in}}%
\pgfusepath{clip}%
\pgfsetbuttcap%
\pgfsetroundjoin%
\definecolor{currentfill}{rgb}{0.281412,0.155834,0.469201}%
\pgfsetfillcolor{currentfill}%
\pgfsetfillopacity{0.700000}%
\pgfsetlinewidth{0.000000pt}%
\definecolor{currentstroke}{rgb}{0.000000,0.000000,0.000000}%
\pgfsetstrokecolor{currentstroke}%
\pgfsetdash{}{0pt}%
\pgfpathmoveto{\pgfqpoint{3.432685in}{1.499859in}}%
\pgfpathlineto{\pgfqpoint{3.446385in}{1.488386in}}%
\pgfpathlineto{\pgfqpoint{3.460086in}{1.477040in}}%
\pgfpathlineto{\pgfqpoint{3.473787in}{1.465822in}}%
\pgfpathlineto{\pgfqpoint{3.487490in}{1.454731in}}%
\pgfpathlineto{\pgfqpoint{3.495772in}{1.453104in}}%
\pgfpathlineto{\pgfqpoint{3.504042in}{1.451755in}}%
\pgfpathlineto{\pgfqpoint{3.512299in}{1.450679in}}%
\pgfpathlineto{\pgfqpoint{3.520544in}{1.449870in}}%
\pgfpathlineto{\pgfqpoint{3.506875in}{1.460427in}}%
\pgfpathlineto{\pgfqpoint{3.493208in}{1.471110in}}%
\pgfpathlineto{\pgfqpoint{3.479542in}{1.481920in}}%
\pgfpathlineto{\pgfqpoint{3.465878in}{1.492857in}}%
\pgfpathlineto{\pgfqpoint{3.457599in}{1.494194in}}%
\pgfpathlineto{\pgfqpoint{3.449308in}{1.495804in}}%
\pgfpathlineto{\pgfqpoint{3.441003in}{1.497690in}}%
\pgfpathlineto{\pgfqpoint{3.432685in}{1.499859in}}%
\pgfpathclose%
\pgfusepath{fill}%
\end{pgfscope}%
\begin{pgfscope}%
\pgfpathrectangle{\pgfqpoint{1.254980in}{0.150000in}}{\pgfqpoint{5.490039in}{5.490039in}}%
\pgfusepath{clip}%
\pgfsetbuttcap%
\pgfsetroundjoin%
\definecolor{currentfill}{rgb}{0.280868,0.160771,0.472899}%
\pgfsetfillcolor{currentfill}%
\pgfsetfillopacity{0.700000}%
\pgfsetlinewidth{0.000000pt}%
\definecolor{currentstroke}{rgb}{0.000000,0.000000,0.000000}%
\pgfsetstrokecolor{currentstroke}%
\pgfsetdash{}{0pt}%
\pgfpathmoveto{\pgfqpoint{4.826726in}{1.450724in}}%
\pgfpathlineto{\pgfqpoint{4.840745in}{1.452909in}}%
\pgfpathlineto{\pgfqpoint{4.854775in}{1.455203in}}%
\pgfpathlineto{\pgfqpoint{4.868817in}{1.457606in}}%
\pgfpathlineto{\pgfqpoint{4.882871in}{1.460120in}}%
\pgfpathlineto{\pgfqpoint{4.890584in}{1.473477in}}%
\pgfpathlineto{\pgfqpoint{4.898293in}{1.486870in}}%
\pgfpathlineto{\pgfqpoint{4.905999in}{1.500297in}}%
\pgfpathlineto{\pgfqpoint{4.913701in}{1.513755in}}%
\pgfpathlineto{\pgfqpoint{4.899646in}{1.510892in}}%
\pgfpathlineto{\pgfqpoint{4.885604in}{1.508138in}}%
\pgfpathlineto{\pgfqpoint{4.871573in}{1.505495in}}%
\pgfpathlineto{\pgfqpoint{4.857554in}{1.502961in}}%
\pgfpathlineto{\pgfqpoint{4.849853in}{1.489847in}}%
\pgfpathlineto{\pgfqpoint{4.842148in}{1.476767in}}%
\pgfpathlineto{\pgfqpoint{4.834439in}{1.463726in}}%
\pgfpathlineto{\pgfqpoint{4.826726in}{1.450724in}}%
\pgfpathclose%
\pgfusepath{fill}%
\end{pgfscope}%
\begin{pgfscope}%
\pgfpathrectangle{\pgfqpoint{1.254980in}{0.150000in}}{\pgfqpoint{5.490039in}{5.490039in}}%
\pgfusepath{clip}%
\pgfsetbuttcap%
\pgfsetroundjoin%
\definecolor{currentfill}{rgb}{0.296479,0.761561,0.424223}%
\pgfsetfillcolor{currentfill}%
\pgfsetfillopacity{0.700000}%
\pgfsetlinewidth{0.000000pt}%
\definecolor{currentstroke}{rgb}{0.000000,0.000000,0.000000}%
\pgfsetstrokecolor{currentstroke}%
\pgfsetdash{}{0pt}%
\pgfpathmoveto{\pgfqpoint{2.196889in}{3.128194in}}%
\pgfpathlineto{\pgfqpoint{2.211012in}{3.102531in}}%
\pgfpathlineto{\pgfqpoint{2.225123in}{3.077076in}}%
\pgfpathlineto{\pgfqpoint{2.239222in}{3.051826in}}%
\pgfpathlineto{\pgfqpoint{2.253309in}{3.026780in}}%
\pgfpathlineto{\pgfqpoint{2.262691in}{3.013480in}}%
\pgfpathlineto{\pgfqpoint{2.272045in}{3.000562in}}%
\pgfpathlineto{\pgfqpoint{2.281372in}{2.988022in}}%
\pgfpathlineto{\pgfqpoint{2.290673in}{2.975854in}}%
\pgfpathlineto{\pgfqpoint{2.276655in}{3.000304in}}%
\pgfpathlineto{\pgfqpoint{2.262625in}{3.024957in}}%
\pgfpathlineto{\pgfqpoint{2.248583in}{3.049814in}}%
\pgfpathlineto{\pgfqpoint{2.234529in}{3.074876in}}%
\pgfpathlineto{\pgfqpoint{2.225161in}{3.087633in}}%
\pgfpathlineto{\pgfqpoint{2.215765in}{3.100768in}}%
\pgfpathlineto{\pgfqpoint{2.206341in}{3.114287in}}%
\pgfpathlineto{\pgfqpoint{2.196889in}{3.128194in}}%
\pgfpathclose%
\pgfusepath{fill}%
\end{pgfscope}%
\begin{pgfscope}%
\pgfpathrectangle{\pgfqpoint{1.254980in}{0.150000in}}{\pgfqpoint{5.490039in}{5.490039in}}%
\pgfusepath{clip}%
\pgfsetbuttcap%
\pgfsetroundjoin%
\definecolor{currentfill}{rgb}{0.269944,0.014625,0.341379}%
\pgfsetfillcolor{currentfill}%
\pgfsetfillopacity{0.700000}%
\pgfsetlinewidth{0.000000pt}%
\definecolor{currentstroke}{rgb}{0.000000,0.000000,0.000000}%
\pgfsetstrokecolor{currentstroke}%
\pgfsetdash{}{0pt}%
\pgfpathmoveto{\pgfqpoint{4.306665in}{1.197030in}}%
\pgfpathlineto{\pgfqpoint{4.320478in}{1.194160in}}%
\pgfpathlineto{\pgfqpoint{4.334298in}{1.191401in}}%
\pgfpathlineto{\pgfqpoint{4.348126in}{1.188753in}}%
\pgfpathlineto{\pgfqpoint{4.361962in}{1.186216in}}%
\pgfpathlineto{\pgfqpoint{4.369797in}{1.195096in}}%
\pgfpathlineto{\pgfqpoint{4.377626in}{1.204112in}}%
\pgfpathlineto{\pgfqpoint{4.385451in}{1.213262in}}%
\pgfpathlineto{\pgfqpoint{4.393271in}{1.222541in}}%
\pgfpathlineto{\pgfqpoint{4.379444in}{1.224637in}}%
\pgfpathlineto{\pgfqpoint{4.365626in}{1.226844in}}%
\pgfpathlineto{\pgfqpoint{4.351816in}{1.229163in}}%
\pgfpathlineto{\pgfqpoint{4.338014in}{1.231592in}}%
\pgfpathlineto{\pgfqpoint{4.330185in}{1.222747in}}%
\pgfpathlineto{\pgfqpoint{4.322350in}{1.214036in}}%
\pgfpathlineto{\pgfqpoint{4.314510in}{1.205463in}}%
\pgfpathlineto{\pgfqpoint{4.306665in}{1.197030in}}%
\pgfpathclose%
\pgfusepath{fill}%
\end{pgfscope}%
\begin{pgfscope}%
\pgfpathrectangle{\pgfqpoint{1.254980in}{0.150000in}}{\pgfqpoint{5.490039in}{5.490039in}}%
\pgfusepath{clip}%
\pgfsetbuttcap%
\pgfsetroundjoin%
\definecolor{currentfill}{rgb}{0.636902,0.856542,0.216620}%
\pgfsetfillcolor{currentfill}%
\pgfsetfillopacity{0.700000}%
\pgfsetlinewidth{0.000000pt}%
\definecolor{currentstroke}{rgb}{0.000000,0.000000,0.000000}%
\pgfsetstrokecolor{currentstroke}%
\pgfsetdash{}{0pt}%
\pgfpathmoveto{\pgfqpoint{2.007871in}{3.505107in}}%
\pgfpathlineto{\pgfqpoint{2.022146in}{3.476541in}}%
\pgfpathlineto{\pgfqpoint{2.036404in}{3.448207in}}%
\pgfpathlineto{\pgfqpoint{2.050648in}{3.420102in}}%
\pgfpathlineto{\pgfqpoint{2.064876in}{3.392225in}}%
\pgfpathlineto{\pgfqpoint{2.074430in}{3.378057in}}%
\pgfpathlineto{\pgfqpoint{2.083955in}{3.364275in}}%
\pgfpathlineto{\pgfqpoint{2.093451in}{3.350875in}}%
\pgfpathlineto{\pgfqpoint{2.102918in}{3.337851in}}%
\pgfpathlineto{\pgfqpoint{2.088762in}{3.365132in}}%
\pgfpathlineto{\pgfqpoint{2.074592in}{3.392639in}}%
\pgfpathlineto{\pgfqpoint{2.060407in}{3.420374in}}%
\pgfpathlineto{\pgfqpoint{2.046207in}{3.448339in}}%
\pgfpathlineto{\pgfqpoint{2.036667in}{3.461951in}}%
\pgfpathlineto{\pgfqpoint{2.027099in}{3.475947in}}%
\pgfpathlineto{\pgfqpoint{2.017500in}{3.490330in}}%
\pgfpathlineto{\pgfqpoint{2.007871in}{3.505107in}}%
\pgfpathclose%
\pgfusepath{fill}%
\end{pgfscope}%
\begin{pgfscope}%
\pgfpathrectangle{\pgfqpoint{1.254980in}{0.150000in}}{\pgfqpoint{5.490039in}{5.490039in}}%
\pgfusepath{clip}%
\pgfsetbuttcap%
\pgfsetroundjoin%
\definecolor{currentfill}{rgb}{0.246811,0.283237,0.535941}%
\pgfsetfillcolor{currentfill}%
\pgfsetfillopacity{0.700000}%
\pgfsetlinewidth{0.000000pt}%
\definecolor{currentstroke}{rgb}{0.000000,0.000000,0.000000}%
\pgfsetstrokecolor{currentstroke}%
\pgfsetdash{}{0pt}%
\pgfpathmoveto{\pgfqpoint{5.118782in}{1.711227in}}%
\pgfpathlineto{\pgfqpoint{5.132962in}{1.716038in}}%
\pgfpathlineto{\pgfqpoint{5.147157in}{1.720959in}}%
\pgfpathlineto{\pgfqpoint{5.161366in}{1.725991in}}%
\pgfpathlineto{\pgfqpoint{5.169025in}{1.740426in}}%
\pgfpathlineto{\pgfqpoint{5.176680in}{1.754847in}}%
\pgfpathlineto{\pgfqpoint{5.184331in}{1.769251in}}%
\pgfpathlineto{\pgfqpoint{5.191978in}{1.783636in}}%
\pgfpathlineto{\pgfqpoint{5.177765in}{1.778316in}}%
\pgfpathlineto{\pgfqpoint{5.163567in}{1.773106in}}%
\pgfpathlineto{\pgfqpoint{5.149383in}{1.768006in}}%
\pgfpathlineto{\pgfqpoint{5.141739in}{1.753833in}}%
\pgfpathlineto{\pgfqpoint{5.134091in}{1.739644in}}%
\pgfpathlineto{\pgfqpoint{5.126439in}{1.725441in}}%
\pgfpathlineto{\pgfqpoint{5.118782in}{1.711227in}}%
\pgfpathclose%
\pgfusepath{fill}%
\end{pgfscope}%
\begin{pgfscope}%
\pgfpathrectangle{\pgfqpoint{1.254980in}{0.150000in}}{\pgfqpoint{5.490039in}{5.490039in}}%
\pgfusepath{clip}%
\pgfsetbuttcap%
\pgfsetroundjoin%
\definecolor{currentfill}{rgb}{0.119699,0.618490,0.536347}%
\pgfsetfillcolor{currentfill}%
\pgfsetfillopacity{0.700000}%
\pgfsetlinewidth{0.000000pt}%
\definecolor{currentstroke}{rgb}{0.000000,0.000000,0.000000}%
\pgfsetstrokecolor{currentstroke}%
\pgfsetdash{}{0pt}%
\pgfpathmoveto{\pgfqpoint{2.440493in}{2.700146in}}%
\pgfpathlineto{\pgfqpoint{2.454469in}{2.677774in}}%
\pgfpathlineto{\pgfqpoint{2.468435in}{2.655585in}}%
\pgfpathlineto{\pgfqpoint{2.482393in}{2.633578in}}%
\pgfpathlineto{\pgfqpoint{2.496342in}{2.611750in}}%
\pgfpathlineto{\pgfqpoint{2.505498in}{2.599937in}}%
\pgfpathlineto{\pgfqpoint{2.514629in}{2.588497in}}%
\pgfpathlineto{\pgfqpoint{2.523735in}{2.577426in}}%
\pgfpathlineto{\pgfqpoint{2.532817in}{2.566718in}}%
\pgfpathlineto{\pgfqpoint{2.518931in}{2.587953in}}%
\pgfpathlineto{\pgfqpoint{2.505036in}{2.609367in}}%
\pgfpathlineto{\pgfqpoint{2.491133in}{2.630961in}}%
\pgfpathlineto{\pgfqpoint{2.477221in}{2.652736in}}%
\pgfpathlineto{\pgfqpoint{2.468077in}{2.664030in}}%
\pgfpathlineto{\pgfqpoint{2.458908in}{2.675692in}}%
\pgfpathlineto{\pgfqpoint{2.449713in}{2.687729in}}%
\pgfpathlineto{\pgfqpoint{2.440493in}{2.700146in}}%
\pgfpathclose%
\pgfusepath{fill}%
\end{pgfscope}%
\begin{pgfscope}%
\pgfpathrectangle{\pgfqpoint{1.254980in}{0.150000in}}{\pgfqpoint{5.490039in}{5.490039in}}%
\pgfusepath{clip}%
\pgfsetbuttcap%
\pgfsetroundjoin%
\definecolor{currentfill}{rgb}{0.278791,0.062145,0.386592}%
\pgfsetfillcolor{currentfill}%
\pgfsetfillopacity{0.700000}%
\pgfsetlinewidth{0.000000pt}%
\definecolor{currentstroke}{rgb}{0.000000,0.000000,0.000000}%
\pgfsetstrokecolor{currentstroke}%
\pgfsetdash{}{0pt}%
\pgfpathmoveto{\pgfqpoint{3.739520in}{1.297799in}}%
\pgfpathlineto{\pgfqpoint{3.753229in}{1.289328in}}%
\pgfpathlineto{\pgfqpoint{3.766941in}{1.280975in}}%
\pgfpathlineto{\pgfqpoint{3.780657in}{1.272742in}}%
\pgfpathlineto{\pgfqpoint{3.794376in}{1.264627in}}%
\pgfpathlineto{\pgfqpoint{3.802463in}{1.266677in}}%
\pgfpathlineto{\pgfqpoint{3.810540in}{1.268962in}}%
\pgfpathlineto{\pgfqpoint{3.818608in}{1.271478in}}%
\pgfpathlineto{\pgfqpoint{3.826666in}{1.274221in}}%
\pgfpathlineto{\pgfqpoint{3.812972in}{1.281826in}}%
\pgfpathlineto{\pgfqpoint{3.799282in}{1.289549in}}%
\pgfpathlineto{\pgfqpoint{3.785596in}{1.297391in}}%
\pgfpathlineto{\pgfqpoint{3.771913in}{1.305352in}}%
\pgfpathlineto{\pgfqpoint{3.763829in}{1.303113in}}%
\pgfpathlineto{\pgfqpoint{3.755736in}{1.301105in}}%
\pgfpathlineto{\pgfqpoint{3.747633in}{1.299332in}}%
\pgfpathlineto{\pgfqpoint{3.739520in}{1.297799in}}%
\pgfpathclose%
\pgfusepath{fill}%
\end{pgfscope}%
\begin{pgfscope}%
\pgfpathrectangle{\pgfqpoint{1.254980in}{0.150000in}}{\pgfqpoint{5.490039in}{5.490039in}}%
\pgfusepath{clip}%
\pgfsetbuttcap%
\pgfsetroundjoin%
\definecolor{currentfill}{rgb}{0.282884,0.135920,0.453427}%
\pgfsetfillcolor{currentfill}%
\pgfsetfillopacity{0.700000}%
\pgfsetlinewidth{0.000000pt}%
\definecolor{currentstroke}{rgb}{0.000000,0.000000,0.000000}%
\pgfsetstrokecolor{currentstroke}%
\pgfsetdash{}{0pt}%
\pgfpathmoveto{\pgfqpoint{3.487490in}{1.454731in}}%
\pgfpathlineto{\pgfqpoint{3.501194in}{1.443765in}}%
\pgfpathlineto{\pgfqpoint{3.514899in}{1.432926in}}%
\pgfpathlineto{\pgfqpoint{3.528605in}{1.422212in}}%
\pgfpathlineto{\pgfqpoint{3.542313in}{1.411623in}}%
\pgfpathlineto{\pgfqpoint{3.550562in}{1.410538in}}%
\pgfpathlineto{\pgfqpoint{3.558798in}{1.409725in}}%
\pgfpathlineto{\pgfqpoint{3.567022in}{1.409179in}}%
\pgfpathlineto{\pgfqpoint{3.575234in}{1.408896in}}%
\pgfpathlineto{\pgfqpoint{3.561559in}{1.418952in}}%
\pgfpathlineto{\pgfqpoint{3.547886in}{1.429133in}}%
\pgfpathlineto{\pgfqpoint{3.534214in}{1.439439in}}%
\pgfpathlineto{\pgfqpoint{3.520544in}{1.449870in}}%
\pgfpathlineto{\pgfqpoint{3.512299in}{1.450679in}}%
\pgfpathlineto{\pgfqpoint{3.504042in}{1.451755in}}%
\pgfpathlineto{\pgfqpoint{3.495772in}{1.453104in}}%
\pgfpathlineto{\pgfqpoint{3.487490in}{1.454731in}}%
\pgfpathclose%
\pgfusepath{fill}%
\end{pgfscope}%
\begin{pgfscope}%
\pgfpathrectangle{\pgfqpoint{1.254980in}{0.150000in}}{\pgfqpoint{5.490039in}{5.490039in}}%
\pgfusepath{clip}%
\pgfsetbuttcap%
\pgfsetroundjoin%
\definecolor{currentfill}{rgb}{0.268510,0.009605,0.335427}%
\pgfsetfillcolor{currentfill}%
\pgfsetfillopacity{0.700000}%
\pgfsetlinewidth{0.000000pt}%
\definecolor{currentstroke}{rgb}{0.000000,0.000000,0.000000}%
\pgfsetstrokecolor{currentstroke}%
\pgfsetdash{}{0pt}%
\pgfpathmoveto{\pgfqpoint{4.078196in}{1.189409in}}%
\pgfpathlineto{\pgfqpoint{4.091953in}{1.184263in}}%
\pgfpathlineto{\pgfqpoint{4.105717in}{1.179230in}}%
\pgfpathlineto{\pgfqpoint{4.119487in}{1.174311in}}%
\pgfpathlineto{\pgfqpoint{4.133263in}{1.169504in}}%
\pgfpathlineto{\pgfqpoint{4.141182in}{1.175715in}}%
\pgfpathlineto{\pgfqpoint{4.149094in}{1.182106in}}%
\pgfpathlineto{\pgfqpoint{4.157000in}{1.188672in}}%
\pgfpathlineto{\pgfqpoint{4.164900in}{1.195411in}}%
\pgfpathlineto{\pgfqpoint{4.151139in}{1.199744in}}%
\pgfpathlineto{\pgfqpoint{4.137385in}{1.204191in}}%
\pgfpathlineto{\pgfqpoint{4.123638in}{1.208750in}}%
\pgfpathlineto{\pgfqpoint{4.109897in}{1.213423in}}%
\pgfpathlineto{\pgfqpoint{4.101982in}{1.207152in}}%
\pgfpathlineto{\pgfqpoint{4.094060in}{1.201056in}}%
\pgfpathlineto{\pgfqpoint{4.086131in}{1.195141in}}%
\pgfpathlineto{\pgfqpoint{4.078196in}{1.189409in}}%
\pgfpathclose%
\pgfusepath{fill}%
\end{pgfscope}%
\begin{pgfscope}%
\pgfpathrectangle{\pgfqpoint{1.254980in}{0.150000in}}{\pgfqpoint{5.490039in}{5.490039in}}%
\pgfusepath{clip}%
\pgfsetbuttcap%
\pgfsetroundjoin%
\definecolor{currentfill}{rgb}{0.275191,0.194905,0.496005}%
\pgfsetfillcolor{currentfill}%
\pgfsetfillopacity{0.700000}%
\pgfsetlinewidth{0.000000pt}%
\definecolor{currentstroke}{rgb}{0.000000,0.000000,0.000000}%
\pgfsetstrokecolor{currentstroke}%
\pgfsetdash{}{0pt}%
\pgfpathmoveto{\pgfqpoint{4.913701in}{1.513755in}}%
\pgfpathlineto{\pgfqpoint{4.927769in}{1.516728in}}%
\pgfpathlineto{\pgfqpoint{4.941849in}{1.519810in}}%
\pgfpathlineto{\pgfqpoint{4.955941in}{1.523002in}}%
\pgfpathlineto{\pgfqpoint{4.970046in}{1.526304in}}%
\pgfpathlineto{\pgfqpoint{4.977747in}{1.540130in}}%
\pgfpathlineto{\pgfqpoint{4.985443in}{1.553978in}}%
\pgfpathlineto{\pgfqpoint{4.993136in}{1.567845in}}%
\pgfpathlineto{\pgfqpoint{5.000825in}{1.581729in}}%
\pgfpathlineto{\pgfqpoint{4.986717in}{1.578092in}}%
\pgfpathlineto{\pgfqpoint{4.972623in}{1.574565in}}%
\pgfpathlineto{\pgfqpoint{4.958541in}{1.571148in}}%
\pgfpathlineto{\pgfqpoint{4.944472in}{1.567841in}}%
\pgfpathlineto{\pgfqpoint{4.936785in}{1.554286in}}%
\pgfpathlineto{\pgfqpoint{4.929094in}{1.540752in}}%
\pgfpathlineto{\pgfqpoint{4.921400in}{1.527241in}}%
\pgfpathlineto{\pgfqpoint{4.913701in}{1.513755in}}%
\pgfpathclose%
\pgfusepath{fill}%
\end{pgfscope}%
\begin{pgfscope}%
\pgfpathrectangle{\pgfqpoint{1.254980in}{0.150000in}}{\pgfqpoint{5.490039in}{5.490039in}}%
\pgfusepath{clip}%
\pgfsetbuttcap%
\pgfsetroundjoin%
\definecolor{currentfill}{rgb}{0.271305,0.019942,0.347269}%
\pgfsetfillcolor{currentfill}%
\pgfsetfillopacity{0.700000}%
\pgfsetlinewidth{0.000000pt}%
\definecolor{currentstroke}{rgb}{0.000000,0.000000,0.000000}%
\pgfsetstrokecolor{currentstroke}%
\pgfsetdash{}{0pt}%
\pgfpathmoveto{\pgfqpoint{3.936371in}{1.217608in}}%
\pgfpathlineto{\pgfqpoint{3.950104in}{1.211056in}}%
\pgfpathlineto{\pgfqpoint{3.963843in}{1.204619in}}%
\pgfpathlineto{\pgfqpoint{3.977586in}{1.198297in}}%
\pgfpathlineto{\pgfqpoint{3.991335in}{1.192089in}}%
\pgfpathlineto{\pgfqpoint{3.999318in}{1.196547in}}%
\pgfpathlineto{\pgfqpoint{4.007294in}{1.201210in}}%
\pgfpathlineto{\pgfqpoint{4.015262in}{1.206073in}}%
\pgfpathlineto{\pgfqpoint{4.023223in}{1.211132in}}%
\pgfpathlineto{\pgfqpoint{4.009494in}{1.216849in}}%
\pgfpathlineto{\pgfqpoint{3.995771in}{1.222680in}}%
\pgfpathlineto{\pgfqpoint{3.982052in}{1.228627in}}%
\pgfpathlineto{\pgfqpoint{3.968339in}{1.234689in}}%
\pgfpathlineto{\pgfqpoint{3.960359in}{1.230113in}}%
\pgfpathlineto{\pgfqpoint{3.952371in}{1.225739in}}%
\pgfpathlineto{\pgfqpoint{3.944375in}{1.221569in}}%
\pgfpathlineto{\pgfqpoint{3.936371in}{1.217608in}}%
\pgfpathclose%
\pgfusepath{fill}%
\end{pgfscope}%
\begin{pgfscope}%
\pgfpathrectangle{\pgfqpoint{1.254980in}{0.150000in}}{\pgfqpoint{5.490039in}{5.490039in}}%
\pgfusepath{clip}%
\pgfsetbuttcap%
\pgfsetroundjoin%
\definecolor{currentfill}{rgb}{0.268510,0.009605,0.335427}%
\pgfsetfillcolor{currentfill}%
\pgfsetfillopacity{0.700000}%
\pgfsetlinewidth{0.000000pt}%
\definecolor{currentstroke}{rgb}{0.000000,0.000000,0.000000}%
\pgfsetstrokecolor{currentstroke}%
\pgfsetdash{}{0pt}%
\pgfpathmoveto{\pgfqpoint{4.220009in}{1.179201in}}%
\pgfpathlineto{\pgfqpoint{4.233804in}{1.175429in}}%
\pgfpathlineto{\pgfqpoint{4.247606in}{1.171768in}}%
\pgfpathlineto{\pgfqpoint{4.261415in}{1.168219in}}%
\pgfpathlineto{\pgfqpoint{4.275231in}{1.164781in}}%
\pgfpathlineto{\pgfqpoint{4.283098in}{1.172613in}}%
\pgfpathlineto{\pgfqpoint{4.290959in}{1.180601in}}%
\pgfpathlineto{\pgfqpoint{4.298815in}{1.188741in}}%
\pgfpathlineto{\pgfqpoint{4.306665in}{1.197030in}}%
\pgfpathlineto{\pgfqpoint{4.292861in}{1.200011in}}%
\pgfpathlineto{\pgfqpoint{4.279064in}{1.203103in}}%
\pgfpathlineto{\pgfqpoint{4.265274in}{1.206307in}}%
\pgfpathlineto{\pgfqpoint{4.251492in}{1.209622in}}%
\pgfpathlineto{\pgfqpoint{4.243630in}{1.201785in}}%
\pgfpathlineto{\pgfqpoint{4.235762in}{1.194099in}}%
\pgfpathlineto{\pgfqpoint{4.227889in}{1.186570in}}%
\pgfpathlineto{\pgfqpoint{4.220009in}{1.179201in}}%
\pgfpathclose%
\pgfusepath{fill}%
\end{pgfscope}%
\begin{pgfscope}%
\pgfpathrectangle{\pgfqpoint{1.254980in}{0.150000in}}{\pgfqpoint{5.490039in}{5.490039in}}%
\pgfusepath{clip}%
\pgfsetbuttcap%
\pgfsetroundjoin%
\definecolor{currentfill}{rgb}{0.132268,0.655014,0.519661}%
\pgfsetfillcolor{currentfill}%
\pgfsetfillopacity{0.700000}%
\pgfsetlinewidth{0.000000pt}%
\definecolor{currentstroke}{rgb}{0.000000,0.000000,0.000000}%
\pgfsetstrokecolor{currentstroke}%
\pgfsetdash{}{0pt}%
\pgfpathmoveto{\pgfqpoint{2.384496in}{2.791481in}}%
\pgfpathlineto{\pgfqpoint{2.398510in}{2.768368in}}%
\pgfpathlineto{\pgfqpoint{2.412514in}{2.745441in}}%
\pgfpathlineto{\pgfqpoint{2.426508in}{2.722701in}}%
\pgfpathlineto{\pgfqpoint{2.440493in}{2.700146in}}%
\pgfpathlineto{\pgfqpoint{2.449713in}{2.687729in}}%
\pgfpathlineto{\pgfqpoint{2.458908in}{2.675692in}}%
\pgfpathlineto{\pgfqpoint{2.468077in}{2.664030in}}%
\pgfpathlineto{\pgfqpoint{2.477221in}{2.652736in}}%
\pgfpathlineto{\pgfqpoint{2.463300in}{2.674694in}}%
\pgfpathlineto{\pgfqpoint{2.449371in}{2.696836in}}%
\pgfpathlineto{\pgfqpoint{2.435432in}{2.719162in}}%
\pgfpathlineto{\pgfqpoint{2.421484in}{2.741674in}}%
\pgfpathlineto{\pgfqpoint{2.412276in}{2.753558in}}%
\pgfpathlineto{\pgfqpoint{2.403042in}{2.765817in}}%
\pgfpathlineto{\pgfqpoint{2.393783in}{2.778456in}}%
\pgfpathlineto{\pgfqpoint{2.384496in}{2.791481in}}%
\pgfpathclose%
\pgfusepath{fill}%
\end{pgfscope}%
\begin{pgfscope}%
\pgfpathrectangle{\pgfqpoint{1.254980in}{0.150000in}}{\pgfqpoint{5.490039in}{5.490039in}}%
\pgfusepath{clip}%
\pgfsetbuttcap%
\pgfsetroundjoin%
\definecolor{currentfill}{rgb}{0.283229,0.120777,0.440584}%
\pgfsetfillcolor{currentfill}%
\pgfsetfillopacity{0.700000}%
\pgfsetlinewidth{0.000000pt}%
\definecolor{currentstroke}{rgb}{0.000000,0.000000,0.000000}%
\pgfsetstrokecolor{currentstroke}%
\pgfsetdash{}{0pt}%
\pgfpathmoveto{\pgfqpoint{3.542313in}{1.411623in}}%
\pgfpathlineto{\pgfqpoint{3.556023in}{1.401159in}}%
\pgfpathlineto{\pgfqpoint{3.569734in}{1.390819in}}%
\pgfpathlineto{\pgfqpoint{3.583447in}{1.380602in}}%
\pgfpathlineto{\pgfqpoint{3.597162in}{1.370509in}}%
\pgfpathlineto{\pgfqpoint{3.605378in}{1.369963in}}%
\pgfpathlineto{\pgfqpoint{3.613582in}{1.369684in}}%
\pgfpathlineto{\pgfqpoint{3.621774in}{1.369668in}}%
\pgfpathlineto{\pgfqpoint{3.629956in}{1.369910in}}%
\pgfpathlineto{\pgfqpoint{3.616272in}{1.379472in}}%
\pgfpathlineto{\pgfqpoint{3.602591in}{1.389157in}}%
\pgfpathlineto{\pgfqpoint{3.588911in}{1.398965in}}%
\pgfpathlineto{\pgfqpoint{3.575234in}{1.408896in}}%
\pgfpathlineto{\pgfqpoint{3.567022in}{1.409179in}}%
\pgfpathlineto{\pgfqpoint{3.558798in}{1.409725in}}%
\pgfpathlineto{\pgfqpoint{3.550562in}{1.410538in}}%
\pgfpathlineto{\pgfqpoint{3.542313in}{1.411623in}}%
\pgfpathclose%
\pgfusepath{fill}%
\end{pgfscope}%
\begin{pgfscope}%
\pgfpathrectangle{\pgfqpoint{1.254980in}{0.150000in}}{\pgfqpoint{5.490039in}{5.490039in}}%
\pgfusepath{clip}%
\pgfsetbuttcap%
\pgfsetroundjoin%
\definecolor{currentfill}{rgb}{0.386433,0.794644,0.372886}%
\pgfsetfillcolor{currentfill}%
\pgfsetfillopacity{0.700000}%
\pgfsetlinewidth{0.000000pt}%
\definecolor{currentstroke}{rgb}{0.000000,0.000000,0.000000}%
\pgfsetstrokecolor{currentstroke}%
\pgfsetdash{}{0pt}%
\pgfpathmoveto{\pgfqpoint{2.140266in}{3.232949in}}%
\pgfpathlineto{\pgfqpoint{2.154441in}{3.206442in}}%
\pgfpathlineto{\pgfqpoint{2.168604in}{3.180148in}}%
\pgfpathlineto{\pgfqpoint{2.182753in}{3.154066in}}%
\pgfpathlineto{\pgfqpoint{2.196889in}{3.128194in}}%
\pgfpathlineto{\pgfqpoint{2.206341in}{3.114287in}}%
\pgfpathlineto{\pgfqpoint{2.215765in}{3.100768in}}%
\pgfpathlineto{\pgfqpoint{2.225161in}{3.087633in}}%
\pgfpathlineto{\pgfqpoint{2.234529in}{3.074876in}}%
\pgfpathlineto{\pgfqpoint{2.220464in}{3.100146in}}%
\pgfpathlineto{\pgfqpoint{2.206386in}{3.125625in}}%
\pgfpathlineto{\pgfqpoint{2.192295in}{3.151313in}}%
\pgfpathlineto{\pgfqpoint{2.178191in}{3.177214in}}%
\pgfpathlineto{\pgfqpoint{2.168753in}{3.190565in}}%
\pgfpathlineto{\pgfqpoint{2.159286in}{3.204301in}}%
\pgfpathlineto{\pgfqpoint{2.149791in}{3.218427in}}%
\pgfpathlineto{\pgfqpoint{2.140266in}{3.232949in}}%
\pgfpathclose%
\pgfusepath{fill}%
\end{pgfscope}%
\begin{pgfscope}%
\pgfpathrectangle{\pgfqpoint{1.254980in}{0.150000in}}{\pgfqpoint{5.490039in}{5.490039in}}%
\pgfusepath{clip}%
\pgfsetbuttcap%
\pgfsetroundjoin%
\definecolor{currentfill}{rgb}{0.281446,0.084320,0.407414}%
\pgfsetfillcolor{currentfill}%
\pgfsetfillopacity{0.700000}%
\pgfsetlinewidth{0.000000pt}%
\definecolor{currentstroke}{rgb}{0.000000,0.000000,0.000000}%
\pgfsetstrokecolor{currentstroke}%
\pgfsetdash{}{0pt}%
\pgfpathmoveto{\pgfqpoint{4.622114in}{1.294478in}}%
\pgfpathlineto{\pgfqpoint{4.636049in}{1.294647in}}%
\pgfpathlineto{\pgfqpoint{4.649994in}{1.294925in}}%
\pgfpathlineto{\pgfqpoint{4.663950in}{1.295313in}}%
\pgfpathlineto{\pgfqpoint{4.677916in}{1.295809in}}%
\pgfpathlineto{\pgfqpoint{4.685675in}{1.307700in}}%
\pgfpathlineto{\pgfqpoint{4.693430in}{1.319670in}}%
\pgfpathlineto{\pgfqpoint{4.701181in}{1.331717in}}%
\pgfpathlineto{\pgfqpoint{4.708928in}{1.343837in}}%
\pgfpathlineto{\pgfqpoint{4.694964in}{1.342944in}}%
\pgfpathlineto{\pgfqpoint{4.681011in}{1.342161in}}%
\pgfpathlineto{\pgfqpoint{4.667069in}{1.341488in}}%
\pgfpathlineto{\pgfqpoint{4.653137in}{1.340924in}}%
\pgfpathlineto{\pgfqpoint{4.645387in}{1.329193in}}%
\pgfpathlineto{\pgfqpoint{4.637634in}{1.317540in}}%
\pgfpathlineto{\pgfqpoint{4.629876in}{1.305967in}}%
\pgfpathlineto{\pgfqpoint{4.622114in}{1.294478in}}%
\pgfpathclose%
\pgfusepath{fill}%
\end{pgfscope}%
\begin{pgfscope}%
\pgfpathrectangle{\pgfqpoint{1.254980in}{0.150000in}}{\pgfqpoint{5.490039in}{5.490039in}}%
\pgfusepath{clip}%
\pgfsetbuttcap%
\pgfsetroundjoin%
\definecolor{currentfill}{rgb}{0.277941,0.056324,0.381191}%
\pgfsetfillcolor{currentfill}%
\pgfsetfillopacity{0.700000}%
\pgfsetlinewidth{0.000000pt}%
\definecolor{currentstroke}{rgb}{0.000000,0.000000,0.000000}%
\pgfsetstrokecolor{currentstroke}%
\pgfsetdash{}{0pt}%
\pgfpathmoveto{\pgfqpoint{4.535369in}{1.251486in}}%
\pgfpathlineto{\pgfqpoint{4.549269in}{1.250806in}}%
\pgfpathlineto{\pgfqpoint{4.563179in}{1.250236in}}%
\pgfpathlineto{\pgfqpoint{4.577098in}{1.249775in}}%
\pgfpathlineto{\pgfqpoint{4.591027in}{1.249424in}}%
\pgfpathlineto{\pgfqpoint{4.598805in}{1.260546in}}%
\pgfpathlineto{\pgfqpoint{4.606579in}{1.271764in}}%
\pgfpathlineto{\pgfqpoint{4.614348in}{1.283076in}}%
\pgfpathlineto{\pgfqpoint{4.622114in}{1.294478in}}%
\pgfpathlineto{\pgfqpoint{4.608189in}{1.294418in}}%
\pgfpathlineto{\pgfqpoint{4.594274in}{1.294468in}}%
\pgfpathlineto{\pgfqpoint{4.580369in}{1.294628in}}%
\pgfpathlineto{\pgfqpoint{4.566474in}{1.294898in}}%
\pgfpathlineto{\pgfqpoint{4.558704in}{1.283900in}}%
\pgfpathlineto{\pgfqpoint{4.550930in}{1.272997in}}%
\pgfpathlineto{\pgfqpoint{4.543152in}{1.262191in}}%
\pgfpathlineto{\pgfqpoint{4.535369in}{1.251486in}}%
\pgfpathclose%
\pgfusepath{fill}%
\end{pgfscope}%
\begin{pgfscope}%
\pgfpathrectangle{\pgfqpoint{1.254980in}{0.150000in}}{\pgfqpoint{5.490039in}{5.490039in}}%
\pgfusepath{clip}%
\pgfsetbuttcap%
\pgfsetroundjoin%
\definecolor{currentfill}{rgb}{0.266580,0.228262,0.514349}%
\pgfsetfillcolor{currentfill}%
\pgfsetfillopacity{0.700000}%
\pgfsetlinewidth{0.000000pt}%
\definecolor{currentstroke}{rgb}{0.000000,0.000000,0.000000}%
\pgfsetstrokecolor{currentstroke}%
\pgfsetdash{}{0pt}%
\pgfpathmoveto{\pgfqpoint{5.000825in}{1.581729in}}%
\pgfpathlineto{\pgfqpoint{5.014945in}{1.585475in}}%
\pgfpathlineto{\pgfqpoint{5.029079in}{1.589332in}}%
\pgfpathlineto{\pgfqpoint{5.043226in}{1.593298in}}%
\pgfpathlineto{\pgfqpoint{5.057386in}{1.597374in}}%
\pgfpathlineto{\pgfqpoint{5.065074in}{1.611597in}}%
\pgfpathlineto{\pgfqpoint{5.072758in}{1.625828in}}%
\pgfpathlineto{\pgfqpoint{5.080439in}{1.640063in}}%
\pgfpathlineto{\pgfqpoint{5.088115in}{1.654300in}}%
\pgfpathlineto{\pgfqpoint{5.073952in}{1.649904in}}%
\pgfpathlineto{\pgfqpoint{5.059802in}{1.645618in}}%
\pgfpathlineto{\pgfqpoint{5.045666in}{1.641441in}}%
\pgfpathlineto{\pgfqpoint{5.031543in}{1.637375in}}%
\pgfpathlineto{\pgfqpoint{5.023869in}{1.623452in}}%
\pgfpathlineto{\pgfqpoint{5.016191in}{1.609534in}}%
\pgfpathlineto{\pgfqpoint{5.008510in}{1.595626in}}%
\pgfpathlineto{\pgfqpoint{5.000825in}{1.581729in}}%
\pgfpathclose%
\pgfusepath{fill}%
\end{pgfscope}%
\begin{pgfscope}%
\pgfpathrectangle{\pgfqpoint{1.254980in}{0.150000in}}{\pgfqpoint{5.490039in}{5.490039in}}%
\pgfusepath{clip}%
\pgfsetbuttcap%
\pgfsetroundjoin%
\definecolor{currentfill}{rgb}{0.212395,0.359683,0.551710}%
\pgfsetfillcolor{currentfill}%
\pgfsetfillopacity{0.700000}%
\pgfsetlinewidth{0.000000pt}%
\definecolor{currentstroke}{rgb}{0.000000,0.000000,0.000000}%
\pgfsetstrokecolor{currentstroke}%
\pgfsetdash{}{0pt}%
\pgfpathmoveto{\pgfqpoint{2.959221in}{1.967183in}}%
\pgfpathlineto{\pgfqpoint{2.973006in}{1.950765in}}%
\pgfpathlineto{\pgfqpoint{2.986788in}{1.934494in}}%
\pgfpathlineto{\pgfqpoint{3.000567in}{1.918369in}}%
\pgfpathlineto{\pgfqpoint{3.014342in}{1.902390in}}%
\pgfpathlineto{\pgfqpoint{3.023023in}{1.895042in}}%
\pgfpathlineto{\pgfqpoint{3.031686in}{1.888032in}}%
\pgfpathlineto{\pgfqpoint{3.040330in}{1.881356in}}%
\pgfpathlineto{\pgfqpoint{3.048955in}{1.875007in}}%
\pgfpathlineto{\pgfqpoint{3.035228in}{1.890412in}}%
\pgfpathlineto{\pgfqpoint{3.021499in}{1.905962in}}%
\pgfpathlineto{\pgfqpoint{3.007766in}{1.921657in}}%
\pgfpathlineto{\pgfqpoint{2.994031in}{1.937499in}}%
\pgfpathlineto{\pgfqpoint{2.985357in}{1.944415in}}%
\pgfpathlineto{\pgfqpoint{2.976664in}{1.951664in}}%
\pgfpathlineto{\pgfqpoint{2.967952in}{1.959251in}}%
\pgfpathlineto{\pgfqpoint{2.959221in}{1.967183in}}%
\pgfpathclose%
\pgfusepath{fill}%
\end{pgfscope}%
\begin{pgfscope}%
\pgfpathrectangle{\pgfqpoint{1.254980in}{0.150000in}}{\pgfqpoint{5.490039in}{5.490039in}}%
\pgfusepath{clip}%
\pgfsetbuttcap%
\pgfsetroundjoin%
\definecolor{currentfill}{rgb}{0.223925,0.334994,0.548053}%
\pgfsetfillcolor{currentfill}%
\pgfsetfillopacity{0.700000}%
\pgfsetlinewidth{0.000000pt}%
\definecolor{currentstroke}{rgb}{0.000000,0.000000,0.000000}%
\pgfsetstrokecolor{currentstroke}%
\pgfsetdash{}{0pt}%
\pgfpathmoveto{\pgfqpoint{3.014342in}{1.902390in}}%
\pgfpathlineto{\pgfqpoint{3.028115in}{1.886555in}}%
\pgfpathlineto{\pgfqpoint{3.041884in}{1.870865in}}%
\pgfpathlineto{\pgfqpoint{3.055651in}{1.855318in}}%
\pgfpathlineto{\pgfqpoint{3.069415in}{1.839914in}}%
\pgfpathlineto{\pgfqpoint{3.078047in}{1.833146in}}%
\pgfpathlineto{\pgfqpoint{3.086662in}{1.826711in}}%
\pgfpathlineto{\pgfqpoint{3.095258in}{1.820604in}}%
\pgfpathlineto{\pgfqpoint{3.103837in}{1.814819in}}%
\pgfpathlineto{\pgfqpoint{3.090120in}{1.829652in}}%
\pgfpathlineto{\pgfqpoint{3.076401in}{1.844628in}}%
\pgfpathlineto{\pgfqpoint{3.062679in}{1.859746in}}%
\pgfpathlineto{\pgfqpoint{3.048955in}{1.875007in}}%
\pgfpathlineto{\pgfqpoint{3.040330in}{1.881356in}}%
\pgfpathlineto{\pgfqpoint{3.031686in}{1.888032in}}%
\pgfpathlineto{\pgfqpoint{3.023023in}{1.895042in}}%
\pgfpathlineto{\pgfqpoint{3.014342in}{1.902390in}}%
\pgfpathclose%
\pgfusepath{fill}%
\end{pgfscope}%
\begin{pgfscope}%
\pgfpathrectangle{\pgfqpoint{1.254980in}{0.150000in}}{\pgfqpoint{5.490039in}{5.490039in}}%
\pgfusepath{clip}%
\pgfsetbuttcap%
\pgfsetroundjoin%
\definecolor{currentfill}{rgb}{0.199430,0.387607,0.554642}%
\pgfsetfillcolor{currentfill}%
\pgfsetfillopacity{0.700000}%
\pgfsetlinewidth{0.000000pt}%
\definecolor{currentstroke}{rgb}{0.000000,0.000000,0.000000}%
\pgfsetstrokecolor{currentstroke}%
\pgfsetdash{}{0pt}%
\pgfpathmoveto{\pgfqpoint{2.904042in}{2.034338in}}%
\pgfpathlineto{\pgfqpoint{2.917843in}{2.017325in}}%
\pgfpathlineto{\pgfqpoint{2.931639in}{2.000462in}}%
\pgfpathlineto{\pgfqpoint{2.945432in}{1.983748in}}%
\pgfpathlineto{\pgfqpoint{2.959221in}{1.967183in}}%
\pgfpathlineto{\pgfqpoint{2.967952in}{1.959251in}}%
\pgfpathlineto{\pgfqpoint{2.976664in}{1.951664in}}%
\pgfpathlineto{\pgfqpoint{2.985357in}{1.944415in}}%
\pgfpathlineto{\pgfqpoint{2.994031in}{1.937499in}}%
\pgfpathlineto{\pgfqpoint{2.980292in}{1.953487in}}%
\pgfpathlineto{\pgfqpoint{2.966550in}{1.969622in}}%
\pgfpathlineto{\pgfqpoint{2.952804in}{1.985906in}}%
\pgfpathlineto{\pgfqpoint{2.939055in}{2.002339in}}%
\pgfpathlineto{\pgfqpoint{2.930332in}{2.009825in}}%
\pgfpathlineto{\pgfqpoint{2.921588in}{2.017651in}}%
\pgfpathlineto{\pgfqpoint{2.912825in}{2.025819in}}%
\pgfpathlineto{\pgfqpoint{2.904042in}{2.034338in}}%
\pgfpathclose%
\pgfusepath{fill}%
\end{pgfscope}%
\begin{pgfscope}%
\pgfpathrectangle{\pgfqpoint{1.254980in}{0.150000in}}{\pgfqpoint{5.490039in}{5.490039in}}%
\pgfusepath{clip}%
\pgfsetbuttcap%
\pgfsetroundjoin%
\definecolor{currentfill}{rgb}{0.283091,0.110553,0.431554}%
\pgfsetfillcolor{currentfill}%
\pgfsetfillopacity{0.700000}%
\pgfsetlinewidth{0.000000pt}%
\definecolor{currentstroke}{rgb}{0.000000,0.000000,0.000000}%
\pgfsetstrokecolor{currentstroke}%
\pgfsetdash{}{0pt}%
\pgfpathmoveto{\pgfqpoint{4.708928in}{1.343837in}}%
\pgfpathlineto{\pgfqpoint{4.722903in}{1.344839in}}%
\pgfpathlineto{\pgfqpoint{4.736888in}{1.345950in}}%
\pgfpathlineto{\pgfqpoint{4.750885in}{1.347170in}}%
\pgfpathlineto{\pgfqpoint{4.764892in}{1.348499in}}%
\pgfpathlineto{\pgfqpoint{4.772634in}{1.361076in}}%
\pgfpathlineto{\pgfqpoint{4.780373in}{1.373717in}}%
\pgfpathlineto{\pgfqpoint{4.788108in}{1.386418in}}%
\pgfpathlineto{\pgfqpoint{4.795839in}{1.399176in}}%
\pgfpathlineto{\pgfqpoint{4.781832in}{1.397466in}}%
\pgfpathlineto{\pgfqpoint{4.767836in}{1.395865in}}%
\pgfpathlineto{\pgfqpoint{4.753852in}{1.394374in}}%
\pgfpathlineto{\pgfqpoint{4.739879in}{1.392992in}}%
\pgfpathlineto{\pgfqpoint{4.732147in}{1.380608in}}%
\pgfpathlineto{\pgfqpoint{4.724411in}{1.368286in}}%
\pgfpathlineto{\pgfqpoint{4.716672in}{1.356028in}}%
\pgfpathlineto{\pgfqpoint{4.708928in}{1.343837in}}%
\pgfpathclose%
\pgfusepath{fill}%
\end{pgfscope}%
\begin{pgfscope}%
\pgfpathrectangle{\pgfqpoint{1.254980in}{0.150000in}}{\pgfqpoint{5.490039in}{5.490039in}}%
\pgfusepath{clip}%
\pgfsetbuttcap%
\pgfsetroundjoin%
\definecolor{currentfill}{rgb}{0.277018,0.050344,0.375715}%
\pgfsetfillcolor{currentfill}%
\pgfsetfillopacity{0.700000}%
\pgfsetlinewidth{0.000000pt}%
\definecolor{currentstroke}{rgb}{0.000000,0.000000,0.000000}%
\pgfsetstrokecolor{currentstroke}%
\pgfsetdash{}{0pt}%
\pgfpathmoveto{\pgfqpoint{3.794376in}{1.264627in}}%
\pgfpathlineto{\pgfqpoint{3.808099in}{1.256631in}}%
\pgfpathlineto{\pgfqpoint{3.821826in}{1.248753in}}%
\pgfpathlineto{\pgfqpoint{3.835556in}{1.240992in}}%
\pgfpathlineto{\pgfqpoint{3.849291in}{1.233349in}}%
\pgfpathlineto{\pgfqpoint{3.857352in}{1.235914in}}%
\pgfpathlineto{\pgfqpoint{3.865405in}{1.238711in}}%
\pgfpathlineto{\pgfqpoint{3.873449in}{1.241734in}}%
\pgfpathlineto{\pgfqpoint{3.881483in}{1.244979in}}%
\pgfpathlineto{\pgfqpoint{3.867773in}{1.252114in}}%
\pgfpathlineto{\pgfqpoint{3.854066in}{1.259365in}}%
\pgfpathlineto{\pgfqpoint{3.840364in}{1.266734in}}%
\pgfpathlineto{\pgfqpoint{3.826666in}{1.274221in}}%
\pgfpathlineto{\pgfqpoint{3.818608in}{1.271478in}}%
\pgfpathlineto{\pgfqpoint{3.810540in}{1.268962in}}%
\pgfpathlineto{\pgfqpoint{3.802463in}{1.266677in}}%
\pgfpathlineto{\pgfqpoint{3.794376in}{1.264627in}}%
\pgfpathclose%
\pgfusepath{fill}%
\end{pgfscope}%
\begin{pgfscope}%
\pgfpathrectangle{\pgfqpoint{1.254980in}{0.150000in}}{\pgfqpoint{5.490039in}{5.490039in}}%
\pgfusepath{clip}%
\pgfsetbuttcap%
\pgfsetroundjoin%
\definecolor{currentfill}{rgb}{0.274952,0.037752,0.364543}%
\pgfsetfillcolor{currentfill}%
\pgfsetfillopacity{0.700000}%
\pgfsetlinewidth{0.000000pt}%
\definecolor{currentstroke}{rgb}{0.000000,0.000000,0.000000}%
\pgfsetstrokecolor{currentstroke}%
\pgfsetdash{}{0pt}%
\pgfpathmoveto{\pgfqpoint{4.448663in}{1.215260in}}%
\pgfpathlineto{\pgfqpoint{4.462532in}{1.213716in}}%
\pgfpathlineto{\pgfqpoint{4.476411in}{1.212281in}}%
\pgfpathlineto{\pgfqpoint{4.490299in}{1.210956in}}%
\pgfpathlineto{\pgfqpoint{4.504195in}{1.209741in}}%
\pgfpathlineto{\pgfqpoint{4.511995in}{1.220009in}}%
\pgfpathlineto{\pgfqpoint{4.519791in}{1.230391in}}%
\pgfpathlineto{\pgfqpoint{4.527582in}{1.240885in}}%
\pgfpathlineto{\pgfqpoint{4.535369in}{1.251486in}}%
\pgfpathlineto{\pgfqpoint{4.521478in}{1.252275in}}%
\pgfpathlineto{\pgfqpoint{4.507597in}{1.253174in}}%
\pgfpathlineto{\pgfqpoint{4.493725in}{1.254183in}}%
\pgfpathlineto{\pgfqpoint{4.479863in}{1.255303in}}%
\pgfpathlineto{\pgfqpoint{4.472069in}{1.245121in}}%
\pgfpathlineto{\pgfqpoint{4.464272in}{1.235051in}}%
\pgfpathlineto{\pgfqpoint{4.456469in}{1.225097in}}%
\pgfpathlineto{\pgfqpoint{4.448663in}{1.215260in}}%
\pgfpathclose%
\pgfusepath{fill}%
\end{pgfscope}%
\begin{pgfscope}%
\pgfpathrectangle{\pgfqpoint{1.254980in}{0.150000in}}{\pgfqpoint{5.490039in}{5.490039in}}%
\pgfusepath{clip}%
\pgfsetbuttcap%
\pgfsetroundjoin%
\definecolor{currentfill}{rgb}{0.235526,0.309527,0.542944}%
\pgfsetfillcolor{currentfill}%
\pgfsetfillopacity{0.700000}%
\pgfsetlinewidth{0.000000pt}%
\definecolor{currentstroke}{rgb}{0.000000,0.000000,0.000000}%
\pgfsetstrokecolor{currentstroke}%
\pgfsetdash{}{0pt}%
\pgfpathmoveto{\pgfqpoint{3.069415in}{1.839914in}}%
\pgfpathlineto{\pgfqpoint{3.083176in}{1.824651in}}%
\pgfpathlineto{\pgfqpoint{3.096935in}{1.809531in}}%
\pgfpathlineto{\pgfqpoint{3.110692in}{1.794551in}}%
\pgfpathlineto{\pgfqpoint{3.124447in}{1.779711in}}%
\pgfpathlineto{\pgfqpoint{3.133032in}{1.773521in}}%
\pgfpathlineto{\pgfqpoint{3.141600in}{1.767658in}}%
\pgfpathlineto{\pgfqpoint{3.150151in}{1.762117in}}%
\pgfpathlineto{\pgfqpoint{3.158685in}{1.756893in}}%
\pgfpathlineto{\pgfqpoint{3.144976in}{1.771165in}}%
\pgfpathlineto{\pgfqpoint{3.131265in}{1.785576in}}%
\pgfpathlineto{\pgfqpoint{3.117552in}{1.800127in}}%
\pgfpathlineto{\pgfqpoint{3.103837in}{1.814819in}}%
\pgfpathlineto{\pgfqpoint{3.095258in}{1.820604in}}%
\pgfpathlineto{\pgfqpoint{3.086662in}{1.826711in}}%
\pgfpathlineto{\pgfqpoint{3.078047in}{1.833146in}}%
\pgfpathlineto{\pgfqpoint{3.069415in}{1.839914in}}%
\pgfpathclose%
\pgfusepath{fill}%
\end{pgfscope}%
\begin{pgfscope}%
\pgfpathrectangle{\pgfqpoint{1.254980in}{0.150000in}}{\pgfqpoint{5.490039in}{5.490039in}}%
\pgfusepath{clip}%
\pgfsetbuttcap%
\pgfsetroundjoin%
\definecolor{currentfill}{rgb}{0.188923,0.410910,0.556326}%
\pgfsetfillcolor{currentfill}%
\pgfsetfillopacity{0.700000}%
\pgfsetlinewidth{0.000000pt}%
\definecolor{currentstroke}{rgb}{0.000000,0.000000,0.000000}%
\pgfsetstrokecolor{currentstroke}%
\pgfsetdash{}{0pt}%
\pgfpathmoveto{\pgfqpoint{2.848798in}{2.103902in}}%
\pgfpathlineto{\pgfqpoint{2.862615in}{2.086282in}}%
\pgfpathlineto{\pgfqpoint{2.876429in}{2.068816in}}%
\pgfpathlineto{\pgfqpoint{2.890237in}{2.051501in}}%
\pgfpathlineto{\pgfqpoint{2.904042in}{2.034338in}}%
\pgfpathlineto{\pgfqpoint{2.912825in}{2.025819in}}%
\pgfpathlineto{\pgfqpoint{2.921588in}{2.017651in}}%
\pgfpathlineto{\pgfqpoint{2.930332in}{2.009825in}}%
\pgfpathlineto{\pgfqpoint{2.939055in}{2.002339in}}%
\pgfpathlineto{\pgfqpoint{2.925302in}{2.018921in}}%
\pgfpathlineto{\pgfqpoint{2.911546in}{2.035654in}}%
\pgfpathlineto{\pgfqpoint{2.897785in}{2.052538in}}%
\pgfpathlineto{\pgfqpoint{2.884020in}{2.069574in}}%
\pgfpathlineto{\pgfqpoint{2.875245in}{2.077635in}}%
\pgfpathlineto{\pgfqpoint{2.866450in}{2.086039in}}%
\pgfpathlineto{\pgfqpoint{2.857634in}{2.094793in}}%
\pgfpathlineto{\pgfqpoint{2.848798in}{2.103902in}}%
\pgfpathclose%
\pgfusepath{fill}%
\end{pgfscope}%
\begin{pgfscope}%
\pgfpathrectangle{\pgfqpoint{1.254980in}{0.150000in}}{\pgfqpoint{5.490039in}{5.490039in}}%
\pgfusepath{clip}%
\pgfsetbuttcap%
\pgfsetroundjoin%
\definecolor{currentfill}{rgb}{0.244972,0.287675,0.537260}%
\pgfsetfillcolor{currentfill}%
\pgfsetfillopacity{0.700000}%
\pgfsetlinewidth{0.000000pt}%
\definecolor{currentstroke}{rgb}{0.000000,0.000000,0.000000}%
\pgfsetstrokecolor{currentstroke}%
\pgfsetdash{}{0pt}%
\pgfpathmoveto{\pgfqpoint{3.124447in}{1.779711in}}%
\pgfpathlineto{\pgfqpoint{3.138199in}{1.765011in}}%
\pgfpathlineto{\pgfqpoint{3.151950in}{1.750450in}}%
\pgfpathlineto{\pgfqpoint{3.165699in}{1.736027in}}%
\pgfpathlineto{\pgfqpoint{3.179446in}{1.721742in}}%
\pgfpathlineto{\pgfqpoint{3.187986in}{1.716126in}}%
\pgfpathlineto{\pgfqpoint{3.196510in}{1.710832in}}%
\pgfpathlineto{\pgfqpoint{3.205017in}{1.705854in}}%
\pgfpathlineto{\pgfqpoint{3.213507in}{1.701189in}}%
\pgfpathlineto{\pgfqpoint{3.199804in}{1.714909in}}%
\pgfpathlineto{\pgfqpoint{3.186099in}{1.728766in}}%
\pgfpathlineto{\pgfqpoint{3.172393in}{1.742760in}}%
\pgfpathlineto{\pgfqpoint{3.158685in}{1.756893in}}%
\pgfpathlineto{\pgfqpoint{3.150151in}{1.762117in}}%
\pgfpathlineto{\pgfqpoint{3.141600in}{1.767658in}}%
\pgfpathlineto{\pgfqpoint{3.133032in}{1.773521in}}%
\pgfpathlineto{\pgfqpoint{3.124447in}{1.779711in}}%
\pgfpathclose%
\pgfusepath{fill}%
\end{pgfscope}%
\begin{pgfscope}%
\pgfpathrectangle{\pgfqpoint{1.254980in}{0.150000in}}{\pgfqpoint{5.490039in}{5.490039in}}%
\pgfusepath{clip}%
\pgfsetbuttcap%
\pgfsetroundjoin%
\definecolor{currentfill}{rgb}{0.175841,0.441290,0.557685}%
\pgfsetfillcolor{currentfill}%
\pgfsetfillopacity{0.700000}%
\pgfsetlinewidth{0.000000pt}%
\definecolor{currentstroke}{rgb}{0.000000,0.000000,0.000000}%
\pgfsetstrokecolor{currentstroke}%
\pgfsetdash{}{0pt}%
\pgfpathmoveto{\pgfqpoint{2.793478in}{2.175926in}}%
\pgfpathlineto{\pgfqpoint{2.807316in}{2.157686in}}%
\pgfpathlineto{\pgfqpoint{2.821148in}{2.139603in}}%
\pgfpathlineto{\pgfqpoint{2.834975in}{2.121675in}}%
\pgfpathlineto{\pgfqpoint{2.848798in}{2.103902in}}%
\pgfpathlineto{\pgfqpoint{2.857634in}{2.094793in}}%
\pgfpathlineto{\pgfqpoint{2.866450in}{2.086039in}}%
\pgfpathlineto{\pgfqpoint{2.875245in}{2.077635in}}%
\pgfpathlineto{\pgfqpoint{2.884020in}{2.069574in}}%
\pgfpathlineto{\pgfqpoint{2.870251in}{2.086763in}}%
\pgfpathlineto{\pgfqpoint{2.856478in}{2.104105in}}%
\pgfpathlineto{\pgfqpoint{2.842700in}{2.121602in}}%
\pgfpathlineto{\pgfqpoint{2.828917in}{2.139254in}}%
\pgfpathlineto{\pgfqpoint{2.820089in}{2.147892in}}%
\pgfpathlineto{\pgfqpoint{2.811240in}{2.156880in}}%
\pgfpathlineto{\pgfqpoint{2.802370in}{2.166223in}}%
\pgfpathlineto{\pgfqpoint{2.793478in}{2.175926in}}%
\pgfpathclose%
\pgfusepath{fill}%
\end{pgfscope}%
\begin{pgfscope}%
\pgfpathrectangle{\pgfqpoint{1.254980in}{0.150000in}}{\pgfqpoint{5.490039in}{5.490039in}}%
\pgfusepath{clip}%
\pgfsetbuttcap%
\pgfsetroundjoin%
\definecolor{currentfill}{rgb}{0.282623,0.140926,0.457517}%
\pgfsetfillcolor{currentfill}%
\pgfsetfillopacity{0.700000}%
\pgfsetlinewidth{0.000000pt}%
\definecolor{currentstroke}{rgb}{0.000000,0.000000,0.000000}%
\pgfsetstrokecolor{currentstroke}%
\pgfsetdash{}{0pt}%
\pgfpathmoveto{\pgfqpoint{4.795839in}{1.399176in}}%
\pgfpathlineto{\pgfqpoint{4.809857in}{1.400995in}}%
\pgfpathlineto{\pgfqpoint{4.823887in}{1.402924in}}%
\pgfpathlineto{\pgfqpoint{4.837929in}{1.404962in}}%
\pgfpathlineto{\pgfqpoint{4.851982in}{1.407109in}}%
\pgfpathlineto{\pgfqpoint{4.859710in}{1.420293in}}%
\pgfpathlineto{\pgfqpoint{4.867434in}{1.433525in}}%
\pgfpathlineto{\pgfqpoint{4.875154in}{1.446801in}}%
\pgfpathlineto{\pgfqpoint{4.882871in}{1.460120in}}%
\pgfpathlineto{\pgfqpoint{4.868817in}{1.457606in}}%
\pgfpathlineto{\pgfqpoint{4.854775in}{1.455203in}}%
\pgfpathlineto{\pgfqpoint{4.840745in}{1.452909in}}%
\pgfpathlineto{\pgfqpoint{4.826726in}{1.450724in}}%
\pgfpathlineto{\pgfqpoint{4.819010in}{1.437765in}}%
\pgfpathlineto{\pgfqpoint{4.811290in}{1.424853in}}%
\pgfpathlineto{\pgfqpoint{4.803566in}{1.411988in}}%
\pgfpathlineto{\pgfqpoint{4.795839in}{1.399176in}}%
\pgfpathclose%
\pgfusepath{fill}%
\end{pgfscope}%
\begin{pgfscope}%
\pgfpathrectangle{\pgfqpoint{1.254980in}{0.150000in}}{\pgfqpoint{5.490039in}{5.490039in}}%
\pgfusepath{clip}%
\pgfsetbuttcap%
\pgfsetroundjoin%
\definecolor{currentfill}{rgb}{0.253935,0.265254,0.529983}%
\pgfsetfillcolor{currentfill}%
\pgfsetfillopacity{0.700000}%
\pgfsetlinewidth{0.000000pt}%
\definecolor{currentstroke}{rgb}{0.000000,0.000000,0.000000}%
\pgfsetstrokecolor{currentstroke}%
\pgfsetdash{}{0pt}%
\pgfpathmoveto{\pgfqpoint{5.088115in}{1.654300in}}%
\pgfpathlineto{\pgfqpoint{5.102292in}{1.658807in}}%
\pgfpathlineto{\pgfqpoint{5.116483in}{1.663423in}}%
\pgfpathlineto{\pgfqpoint{5.130688in}{1.668150in}}%
\pgfpathlineto{\pgfqpoint{5.138363in}{1.682621in}}%
\pgfpathlineto{\pgfqpoint{5.146035in}{1.697086in}}%
\pgfpathlineto{\pgfqpoint{5.153702in}{1.711544in}}%
\pgfpathlineto{\pgfqpoint{5.161366in}{1.725991in}}%
\pgfpathlineto{\pgfqpoint{5.147157in}{1.720959in}}%
\pgfpathlineto{\pgfqpoint{5.132962in}{1.716038in}}%
\pgfpathlineto{\pgfqpoint{5.118782in}{1.711227in}}%
\pgfpathlineto{\pgfqpoint{5.111121in}{1.697004in}}%
\pgfpathlineto{\pgfqpoint{5.103457in}{1.682773in}}%
\pgfpathlineto{\pgfqpoint{5.095788in}{1.668538in}}%
\pgfpathlineto{\pgfqpoint{5.088115in}{1.654300in}}%
\pgfpathclose%
\pgfusepath{fill}%
\end{pgfscope}%
\begin{pgfscope}%
\pgfpathrectangle{\pgfqpoint{1.254980in}{0.150000in}}{\pgfqpoint{5.490039in}{5.490039in}}%
\pgfusepath{clip}%
\pgfsetbuttcap%
\pgfsetroundjoin%
\definecolor{currentfill}{rgb}{0.271305,0.019942,0.347269}%
\pgfsetfillcolor{currentfill}%
\pgfsetfillopacity{0.700000}%
\pgfsetlinewidth{0.000000pt}%
\definecolor{currentstroke}{rgb}{0.000000,0.000000,0.000000}%
\pgfsetstrokecolor{currentstroke}%
\pgfsetdash{}{0pt}%
\pgfpathmoveto{\pgfqpoint{4.361962in}{1.186216in}}%
\pgfpathlineto{\pgfqpoint{4.375806in}{1.183789in}}%
\pgfpathlineto{\pgfqpoint{4.389659in}{1.181473in}}%
\pgfpathlineto{\pgfqpoint{4.403520in}{1.179267in}}%
\pgfpathlineto{\pgfqpoint{4.417389in}{1.177171in}}%
\pgfpathlineto{\pgfqpoint{4.425214in}{1.186498in}}%
\pgfpathlineto{\pgfqpoint{4.433035in}{1.195957in}}%
\pgfpathlineto{\pgfqpoint{4.440851in}{1.205546in}}%
\pgfpathlineto{\pgfqpoint{4.448663in}{1.215260in}}%
\pgfpathlineto{\pgfqpoint{4.434802in}{1.216915in}}%
\pgfpathlineto{\pgfqpoint{4.420949in}{1.218680in}}%
\pgfpathlineto{\pgfqpoint{4.407106in}{1.220555in}}%
\pgfpathlineto{\pgfqpoint{4.393271in}{1.222541in}}%
\pgfpathlineto{\pgfqpoint{4.385451in}{1.213262in}}%
\pgfpathlineto{\pgfqpoint{4.377626in}{1.204112in}}%
\pgfpathlineto{\pgfqpoint{4.369797in}{1.195096in}}%
\pgfpathlineto{\pgfqpoint{4.361962in}{1.186216in}}%
\pgfpathclose%
\pgfusepath{fill}%
\end{pgfscope}%
\begin{pgfscope}%
\pgfpathrectangle{\pgfqpoint{1.254980in}{0.150000in}}{\pgfqpoint{5.490039in}{5.490039in}}%
\pgfusepath{clip}%
\pgfsetbuttcap%
\pgfsetroundjoin%
\definecolor{currentfill}{rgb}{0.253935,0.265254,0.529983}%
\pgfsetfillcolor{currentfill}%
\pgfsetfillopacity{0.700000}%
\pgfsetlinewidth{0.000000pt}%
\definecolor{currentstroke}{rgb}{0.000000,0.000000,0.000000}%
\pgfsetstrokecolor{currentstroke}%
\pgfsetdash{}{0pt}%
\pgfpathmoveto{\pgfqpoint{3.179446in}{1.721742in}}%
\pgfpathlineto{\pgfqpoint{3.193192in}{1.707594in}}%
\pgfpathlineto{\pgfqpoint{3.206936in}{1.693583in}}%
\pgfpathlineto{\pgfqpoint{3.220679in}{1.679707in}}%
\pgfpathlineto{\pgfqpoint{3.234421in}{1.665967in}}%
\pgfpathlineto{\pgfqpoint{3.242917in}{1.660922in}}%
\pgfpathlineto{\pgfqpoint{3.251397in}{1.656194in}}%
\pgfpathlineto{\pgfqpoint{3.259862in}{1.651778in}}%
\pgfpathlineto{\pgfqpoint{3.268310in}{1.647669in}}%
\pgfpathlineto{\pgfqpoint{3.254611in}{1.660846in}}%
\pgfpathlineto{\pgfqpoint{3.240911in}{1.674158in}}%
\pgfpathlineto{\pgfqpoint{3.227209in}{1.687606in}}%
\pgfpathlineto{\pgfqpoint{3.213507in}{1.701189in}}%
\pgfpathlineto{\pgfqpoint{3.205017in}{1.705854in}}%
\pgfpathlineto{\pgfqpoint{3.196510in}{1.710832in}}%
\pgfpathlineto{\pgfqpoint{3.187986in}{1.716126in}}%
\pgfpathlineto{\pgfqpoint{3.179446in}{1.721742in}}%
\pgfpathclose%
\pgfusepath{fill}%
\end{pgfscope}%
\begin{pgfscope}%
\pgfpathrectangle{\pgfqpoint{1.254980in}{0.150000in}}{\pgfqpoint{5.490039in}{5.490039in}}%
\pgfusepath{clip}%
\pgfsetbuttcap%
\pgfsetroundjoin%
\definecolor{currentfill}{rgb}{0.165117,0.467423,0.558141}%
\pgfsetfillcolor{currentfill}%
\pgfsetfillopacity{0.700000}%
\pgfsetlinewidth{0.000000pt}%
\definecolor{currentstroke}{rgb}{0.000000,0.000000,0.000000}%
\pgfsetstrokecolor{currentstroke}%
\pgfsetdash{}{0pt}%
\pgfpathmoveto{\pgfqpoint{2.738075in}{2.250461in}}%
\pgfpathlineto{\pgfqpoint{2.751934in}{2.231589in}}%
\pgfpathlineto{\pgfqpoint{2.765788in}{2.212876in}}%
\pgfpathlineto{\pgfqpoint{2.779636in}{2.194322in}}%
\pgfpathlineto{\pgfqpoint{2.793478in}{2.175926in}}%
\pgfpathlineto{\pgfqpoint{2.802370in}{2.166223in}}%
\pgfpathlineto{\pgfqpoint{2.811240in}{2.156880in}}%
\pgfpathlineto{\pgfqpoint{2.820089in}{2.147892in}}%
\pgfpathlineto{\pgfqpoint{2.828917in}{2.139254in}}%
\pgfpathlineto{\pgfqpoint{2.815129in}{2.157062in}}%
\pgfpathlineto{\pgfqpoint{2.801337in}{2.175026in}}%
\pgfpathlineto{\pgfqpoint{2.787539in}{2.193149in}}%
\pgfpathlineto{\pgfqpoint{2.773737in}{2.211430in}}%
\pgfpathlineto{\pgfqpoint{2.764854in}{2.220649in}}%
\pgfpathlineto{\pgfqpoint{2.755950in}{2.230224in}}%
\pgfpathlineto{\pgfqpoint{2.747024in}{2.240160in}}%
\pgfpathlineto{\pgfqpoint{2.738075in}{2.250461in}}%
\pgfpathclose%
\pgfusepath{fill}%
\end{pgfscope}%
\begin{pgfscope}%
\pgfpathrectangle{\pgfqpoint{1.254980in}{0.150000in}}{\pgfqpoint{5.490039in}{5.490039in}}%
\pgfusepath{clip}%
\pgfsetbuttcap%
\pgfsetroundjoin%
\definecolor{currentfill}{rgb}{0.282910,0.105393,0.426902}%
\pgfsetfillcolor{currentfill}%
\pgfsetfillopacity{0.700000}%
\pgfsetlinewidth{0.000000pt}%
\definecolor{currentstroke}{rgb}{0.000000,0.000000,0.000000}%
\pgfsetstrokecolor{currentstroke}%
\pgfsetdash{}{0pt}%
\pgfpathmoveto{\pgfqpoint{3.597162in}{1.370509in}}%
\pgfpathlineto{\pgfqpoint{3.610879in}{1.360539in}}%
\pgfpathlineto{\pgfqpoint{3.624598in}{1.350691in}}%
\pgfpathlineto{\pgfqpoint{3.638319in}{1.340966in}}%
\pgfpathlineto{\pgfqpoint{3.652043in}{1.331362in}}%
\pgfpathlineto{\pgfqpoint{3.660228in}{1.331354in}}%
\pgfpathlineto{\pgfqpoint{3.668401in}{1.331608in}}%
\pgfpathlineto{\pgfqpoint{3.676564in}{1.332120in}}%
\pgfpathlineto{\pgfqpoint{3.684715in}{1.332886in}}%
\pgfpathlineto{\pgfqpoint{3.671021in}{1.341960in}}%
\pgfpathlineto{\pgfqpoint{3.657330in}{1.351155in}}%
\pgfpathlineto{\pgfqpoint{3.643642in}{1.360471in}}%
\pgfpathlineto{\pgfqpoint{3.629956in}{1.369910in}}%
\pgfpathlineto{\pgfqpoint{3.621774in}{1.369668in}}%
\pgfpathlineto{\pgfqpoint{3.613582in}{1.369684in}}%
\pgfpathlineto{\pgfqpoint{3.605378in}{1.369963in}}%
\pgfpathlineto{\pgfqpoint{3.597162in}{1.370509in}}%
\pgfpathclose%
\pgfusepath{fill}%
\end{pgfscope}%
\begin{pgfscope}%
\pgfpathrectangle{\pgfqpoint{1.254980in}{0.150000in}}{\pgfqpoint{5.490039in}{5.490039in}}%
\pgfusepath{clip}%
\pgfsetbuttcap%
\pgfsetroundjoin%
\definecolor{currentfill}{rgb}{0.166383,0.690856,0.496502}%
\pgfsetfillcolor{currentfill}%
\pgfsetfillopacity{0.700000}%
\pgfsetlinewidth{0.000000pt}%
\definecolor{currentstroke}{rgb}{0.000000,0.000000,0.000000}%
\pgfsetstrokecolor{currentstroke}%
\pgfsetdash{}{0pt}%
\pgfpathmoveto{\pgfqpoint{2.328339in}{2.885834in}}%
\pgfpathlineto{\pgfqpoint{2.342394in}{2.861959in}}%
\pgfpathlineto{\pgfqpoint{2.356438in}{2.838276in}}%
\pgfpathlineto{\pgfqpoint{2.370472in}{2.814784in}}%
\pgfpathlineto{\pgfqpoint{2.384496in}{2.791481in}}%
\pgfpathlineto{\pgfqpoint{2.393783in}{2.778456in}}%
\pgfpathlineto{\pgfqpoint{2.403042in}{2.765817in}}%
\pgfpathlineto{\pgfqpoint{2.412276in}{2.753558in}}%
\pgfpathlineto{\pgfqpoint{2.421484in}{2.741674in}}%
\pgfpathlineto{\pgfqpoint{2.407526in}{2.764373in}}%
\pgfpathlineto{\pgfqpoint{2.393558in}{2.787261in}}%
\pgfpathlineto{\pgfqpoint{2.379581in}{2.810338in}}%
\pgfpathlineto{\pgfqpoint{2.365594in}{2.833606in}}%
\pgfpathlineto{\pgfqpoint{2.356320in}{2.846086in}}%
\pgfpathlineto{\pgfqpoint{2.347021in}{2.858947in}}%
\pgfpathlineto{\pgfqpoint{2.337694in}{2.872195in}}%
\pgfpathlineto{\pgfqpoint{2.328339in}{2.885834in}}%
\pgfpathclose%
\pgfusepath{fill}%
\end{pgfscope}%
\begin{pgfscope}%
\pgfpathrectangle{\pgfqpoint{1.254980in}{0.150000in}}{\pgfqpoint{5.490039in}{5.490039in}}%
\pgfusepath{clip}%
\pgfsetbuttcap%
\pgfsetroundjoin%
\definecolor{currentfill}{rgb}{0.269944,0.014625,0.341379}%
\pgfsetfillcolor{currentfill}%
\pgfsetfillopacity{0.700000}%
\pgfsetlinewidth{0.000000pt}%
\definecolor{currentstroke}{rgb}{0.000000,0.000000,0.000000}%
\pgfsetstrokecolor{currentstroke}%
\pgfsetdash{}{0pt}%
\pgfpathmoveto{\pgfqpoint{3.991335in}{1.192089in}}%
\pgfpathlineto{\pgfqpoint{4.005089in}{1.185997in}}%
\pgfpathlineto{\pgfqpoint{4.018848in}{1.180019in}}%
\pgfpathlineto{\pgfqpoint{4.032612in}{1.174155in}}%
\pgfpathlineto{\pgfqpoint{4.046383in}{1.168405in}}%
\pgfpathlineto{\pgfqpoint{4.054347in}{1.173360in}}%
\pgfpathlineto{\pgfqpoint{4.062303in}{1.178514in}}%
\pgfpathlineto{\pgfqpoint{4.070253in}{1.183866in}}%
\pgfpathlineto{\pgfqpoint{4.078196in}{1.189409in}}%
\pgfpathlineto{\pgfqpoint{4.064444in}{1.194669in}}%
\pgfpathlineto{\pgfqpoint{4.050698in}{1.200042in}}%
\pgfpathlineto{\pgfqpoint{4.036958in}{1.205530in}}%
\pgfpathlineto{\pgfqpoint{4.023223in}{1.211132in}}%
\pgfpathlineto{\pgfqpoint{4.015262in}{1.206073in}}%
\pgfpathlineto{\pgfqpoint{4.007294in}{1.201210in}}%
\pgfpathlineto{\pgfqpoint{3.999318in}{1.196547in}}%
\pgfpathlineto{\pgfqpoint{3.991335in}{1.192089in}}%
\pgfpathclose%
\pgfusepath{fill}%
\end{pgfscope}%
\begin{pgfscope}%
\pgfpathrectangle{\pgfqpoint{1.254980in}{0.150000in}}{\pgfqpoint{5.490039in}{5.490039in}}%
\pgfusepath{clip}%
\pgfsetbuttcap%
\pgfsetroundjoin%
\definecolor{currentfill}{rgb}{0.268510,0.009605,0.335427}%
\pgfsetfillcolor{currentfill}%
\pgfsetfillopacity{0.700000}%
\pgfsetlinewidth{0.000000pt}%
\definecolor{currentstroke}{rgb}{0.000000,0.000000,0.000000}%
\pgfsetstrokecolor{currentstroke}%
\pgfsetdash{}{0pt}%
\pgfpathmoveto{\pgfqpoint{4.133263in}{1.169504in}}%
\pgfpathlineto{\pgfqpoint{4.147045in}{1.164810in}}%
\pgfpathlineto{\pgfqpoint{4.160834in}{1.160228in}}%
\pgfpathlineto{\pgfqpoint{4.174629in}{1.155759in}}%
\pgfpathlineto{\pgfqpoint{4.188431in}{1.151401in}}%
\pgfpathlineto{\pgfqpoint{4.196335in}{1.158092in}}%
\pgfpathlineto{\pgfqpoint{4.204233in}{1.164958in}}%
\pgfpathlineto{\pgfqpoint{4.212124in}{1.171996in}}%
\pgfpathlineto{\pgfqpoint{4.220009in}{1.179201in}}%
\pgfpathlineto{\pgfqpoint{4.206222in}{1.183085in}}%
\pgfpathlineto{\pgfqpoint{4.192441in}{1.187081in}}%
\pgfpathlineto{\pgfqpoint{4.178667in}{1.191190in}}%
\pgfpathlineto{\pgfqpoint{4.164900in}{1.195411in}}%
\pgfpathlineto{\pgfqpoint{4.157000in}{1.188672in}}%
\pgfpathlineto{\pgfqpoint{4.149094in}{1.182106in}}%
\pgfpathlineto{\pgfqpoint{4.141182in}{1.175715in}}%
\pgfpathlineto{\pgfqpoint{4.133263in}{1.169504in}}%
\pgfpathclose%
\pgfusepath{fill}%
\end{pgfscope}%
\begin{pgfscope}%
\pgfpathrectangle{\pgfqpoint{1.254980in}{0.150000in}}{\pgfqpoint{5.490039in}{5.490039in}}%
\pgfusepath{clip}%
\pgfsetbuttcap%
\pgfsetroundjoin%
\definecolor{currentfill}{rgb}{0.262138,0.242286,0.520837}%
\pgfsetfillcolor{currentfill}%
\pgfsetfillopacity{0.700000}%
\pgfsetlinewidth{0.000000pt}%
\definecolor{currentstroke}{rgb}{0.000000,0.000000,0.000000}%
\pgfsetstrokecolor{currentstroke}%
\pgfsetdash{}{0pt}%
\pgfpathmoveto{\pgfqpoint{3.234421in}{1.665967in}}%
\pgfpathlineto{\pgfqpoint{3.248162in}{1.652362in}}%
\pgfpathlineto{\pgfqpoint{3.261902in}{1.638891in}}%
\pgfpathlineto{\pgfqpoint{3.275641in}{1.625554in}}%
\pgfpathlineto{\pgfqpoint{3.289379in}{1.612350in}}%
\pgfpathlineto{\pgfqpoint{3.297833in}{1.607874in}}%
\pgfpathlineto{\pgfqpoint{3.306271in}{1.603710in}}%
\pgfpathlineto{\pgfqpoint{3.314695in}{1.599852in}}%
\pgfpathlineto{\pgfqpoint{3.323103in}{1.596296in}}%
\pgfpathlineto{\pgfqpoint{3.309405in}{1.608940in}}%
\pgfpathlineto{\pgfqpoint{3.295707in}{1.621716in}}%
\pgfpathlineto{\pgfqpoint{3.282009in}{1.634626in}}%
\pgfpathlineto{\pgfqpoint{3.268310in}{1.647669in}}%
\pgfpathlineto{\pgfqpoint{3.259862in}{1.651778in}}%
\pgfpathlineto{\pgfqpoint{3.251397in}{1.656194in}}%
\pgfpathlineto{\pgfqpoint{3.242917in}{1.660922in}}%
\pgfpathlineto{\pgfqpoint{3.234421in}{1.665967in}}%
\pgfpathclose%
\pgfusepath{fill}%
\end{pgfscope}%
\begin{pgfscope}%
\pgfpathrectangle{\pgfqpoint{1.254980in}{0.150000in}}{\pgfqpoint{5.490039in}{5.490039in}}%
\pgfusepath{clip}%
\pgfsetbuttcap%
\pgfsetroundjoin%
\definecolor{currentfill}{rgb}{0.153364,0.497000,0.557724}%
\pgfsetfillcolor{currentfill}%
\pgfsetfillopacity{0.700000}%
\pgfsetlinewidth{0.000000pt}%
\definecolor{currentstroke}{rgb}{0.000000,0.000000,0.000000}%
\pgfsetstrokecolor{currentstroke}%
\pgfsetdash{}{0pt}%
\pgfpathmoveto{\pgfqpoint{2.682578in}{2.327564in}}%
\pgfpathlineto{\pgfqpoint{2.696462in}{2.308045in}}%
\pgfpathlineto{\pgfqpoint{2.710339in}{2.288688in}}%
\pgfpathlineto{\pgfqpoint{2.724210in}{2.269494in}}%
\pgfpathlineto{\pgfqpoint{2.738075in}{2.250461in}}%
\pgfpathlineto{\pgfqpoint{2.747024in}{2.240160in}}%
\pgfpathlineto{\pgfqpoint{2.755950in}{2.230224in}}%
\pgfpathlineto{\pgfqpoint{2.764854in}{2.220649in}}%
\pgfpathlineto{\pgfqpoint{2.773737in}{2.211430in}}%
\pgfpathlineto{\pgfqpoint{2.759928in}{2.229870in}}%
\pgfpathlineto{\pgfqpoint{2.746115in}{2.248470in}}%
\pgfpathlineto{\pgfqpoint{2.732295in}{2.267232in}}%
\pgfpathlineto{\pgfqpoint{2.718470in}{2.286156in}}%
\pgfpathlineto{\pgfqpoint{2.709531in}{2.295961in}}%
\pgfpathlineto{\pgfqpoint{2.700570in}{2.306127in}}%
\pgfpathlineto{\pgfqpoint{2.691586in}{2.316660in}}%
\pgfpathlineto{\pgfqpoint{2.682578in}{2.327564in}}%
\pgfpathclose%
\pgfusepath{fill}%
\end{pgfscope}%
\begin{pgfscope}%
\pgfpathrectangle{\pgfqpoint{1.254980in}{0.150000in}}{\pgfqpoint{5.490039in}{5.490039in}}%
\pgfusepath{clip}%
\pgfsetbuttcap%
\pgfsetroundjoin%
\definecolor{currentfill}{rgb}{0.278826,0.175490,0.483397}%
\pgfsetfillcolor{currentfill}%
\pgfsetfillopacity{0.700000}%
\pgfsetlinewidth{0.000000pt}%
\definecolor{currentstroke}{rgb}{0.000000,0.000000,0.000000}%
\pgfsetstrokecolor{currentstroke}%
\pgfsetdash{}{0pt}%
\pgfpathmoveto{\pgfqpoint{4.882871in}{1.460120in}}%
\pgfpathlineto{\pgfqpoint{4.896937in}{1.462742in}}%
\pgfpathlineto{\pgfqpoint{4.911015in}{1.465474in}}%
\pgfpathlineto{\pgfqpoint{4.925106in}{1.468315in}}%
\pgfpathlineto{\pgfqpoint{4.939209in}{1.471265in}}%
\pgfpathlineto{\pgfqpoint{4.946924in}{1.484979in}}%
\pgfpathlineto{\pgfqpoint{4.954635in}{1.498725in}}%
\pgfpathlineto{\pgfqpoint{4.962343in}{1.512501in}}%
\pgfpathlineto{\pgfqpoint{4.970046in}{1.526304in}}%
\pgfpathlineto{\pgfqpoint{4.955941in}{1.523002in}}%
\pgfpathlineto{\pgfqpoint{4.941849in}{1.519810in}}%
\pgfpathlineto{\pgfqpoint{4.927769in}{1.516728in}}%
\pgfpathlineto{\pgfqpoint{4.913701in}{1.513755in}}%
\pgfpathlineto{\pgfqpoint{4.905999in}{1.500297in}}%
\pgfpathlineto{\pgfqpoint{4.898293in}{1.486870in}}%
\pgfpathlineto{\pgfqpoint{4.890584in}{1.473477in}}%
\pgfpathlineto{\pgfqpoint{4.882871in}{1.460120in}}%
\pgfpathclose%
\pgfusepath{fill}%
\end{pgfscope}%
\begin{pgfscope}%
\pgfpathrectangle{\pgfqpoint{1.254980in}{0.150000in}}{\pgfqpoint{5.490039in}{5.490039in}}%
\pgfusepath{clip}%
\pgfsetbuttcap%
\pgfsetroundjoin%
\definecolor{currentfill}{rgb}{0.496615,0.826376,0.306377}%
\pgfsetfillcolor{currentfill}%
\pgfsetfillopacity{0.700000}%
\pgfsetlinewidth{0.000000pt}%
\definecolor{currentstroke}{rgb}{0.000000,0.000000,0.000000}%
\pgfsetstrokecolor{currentstroke}%
\pgfsetdash{}{0pt}%
\pgfpathmoveto{\pgfqpoint{2.083428in}{3.341142in}}%
\pgfpathlineto{\pgfqpoint{2.097658in}{3.313766in}}%
\pgfpathlineto{\pgfqpoint{2.111875in}{3.286609in}}%
\pgfpathlineto{\pgfqpoint{2.126077in}{3.259671in}}%
\pgfpathlineto{\pgfqpoint{2.140266in}{3.232949in}}%
\pgfpathlineto{\pgfqpoint{2.149791in}{3.218427in}}%
\pgfpathlineto{\pgfqpoint{2.159286in}{3.204301in}}%
\pgfpathlineto{\pgfqpoint{2.168753in}{3.190565in}}%
\pgfpathlineto{\pgfqpoint{2.178191in}{3.177214in}}%
\pgfpathlineto{\pgfqpoint{2.164075in}{3.203327in}}%
\pgfpathlineto{\pgfqpoint{2.149945in}{3.229656in}}%
\pgfpathlineto{\pgfqpoint{2.135802in}{3.256200in}}%
\pgfpathlineto{\pgfqpoint{2.121645in}{3.282963in}}%
\pgfpathlineto{\pgfqpoint{2.112135in}{3.296915in}}%
\pgfpathlineto{\pgfqpoint{2.102596in}{3.311259in}}%
\pgfpathlineto{\pgfqpoint{2.093027in}{3.325999in}}%
\pgfpathlineto{\pgfqpoint{2.083428in}{3.341142in}}%
\pgfpathclose%
\pgfusepath{fill}%
\end{pgfscope}%
\begin{pgfscope}%
\pgfpathrectangle{\pgfqpoint{1.254980in}{0.150000in}}{\pgfqpoint{5.490039in}{5.490039in}}%
\pgfusepath{clip}%
\pgfsetbuttcap%
\pgfsetroundjoin%
\definecolor{currentfill}{rgb}{0.274952,0.037752,0.364543}%
\pgfsetfillcolor{currentfill}%
\pgfsetfillopacity{0.700000}%
\pgfsetlinewidth{0.000000pt}%
\definecolor{currentstroke}{rgb}{0.000000,0.000000,0.000000}%
\pgfsetstrokecolor{currentstroke}%
\pgfsetdash{}{0pt}%
\pgfpathmoveto{\pgfqpoint{3.849291in}{1.233349in}}%
\pgfpathlineto{\pgfqpoint{3.863029in}{1.225823in}}%
\pgfpathlineto{\pgfqpoint{3.876772in}{1.218413in}}%
\pgfpathlineto{\pgfqpoint{3.890519in}{1.211120in}}%
\pgfpathlineto{\pgfqpoint{3.904271in}{1.203943in}}%
\pgfpathlineto{\pgfqpoint{3.912308in}{1.207024in}}%
\pgfpathlineto{\pgfqpoint{3.920338in}{1.210332in}}%
\pgfpathlineto{\pgfqpoint{3.928358in}{1.213861in}}%
\pgfpathlineto{\pgfqpoint{3.936371in}{1.217608in}}%
\pgfpathlineto{\pgfqpoint{3.922642in}{1.224277in}}%
\pgfpathlineto{\pgfqpoint{3.908918in}{1.231061in}}%
\pgfpathlineto{\pgfqpoint{3.895198in}{1.237962in}}%
\pgfpathlineto{\pgfqpoint{3.881483in}{1.244979in}}%
\pgfpathlineto{\pgfqpoint{3.873449in}{1.241734in}}%
\pgfpathlineto{\pgfqpoint{3.865405in}{1.238711in}}%
\pgfpathlineto{\pgfqpoint{3.857352in}{1.235914in}}%
\pgfpathlineto{\pgfqpoint{3.849291in}{1.233349in}}%
\pgfpathclose%
\pgfusepath{fill}%
\end{pgfscope}%
\begin{pgfscope}%
\pgfpathrectangle{\pgfqpoint{1.254980in}{0.150000in}}{\pgfqpoint{5.490039in}{5.490039in}}%
\pgfusepath{clip}%
\pgfsetbuttcap%
\pgfsetroundjoin%
\definecolor{currentfill}{rgb}{0.268510,0.009605,0.335427}%
\pgfsetfillcolor{currentfill}%
\pgfsetfillopacity{0.700000}%
\pgfsetlinewidth{0.000000pt}%
\definecolor{currentstroke}{rgb}{0.000000,0.000000,0.000000}%
\pgfsetstrokecolor{currentstroke}%
\pgfsetdash{}{0pt}%
\pgfpathmoveto{\pgfqpoint{4.275231in}{1.164781in}}%
\pgfpathlineto{\pgfqpoint{4.289055in}{1.161455in}}%
\pgfpathlineto{\pgfqpoint{4.302887in}{1.158239in}}%
\pgfpathlineto{\pgfqpoint{4.316726in}{1.155134in}}%
\pgfpathlineto{\pgfqpoint{4.330572in}{1.152140in}}%
\pgfpathlineto{\pgfqpoint{4.338427in}{1.160435in}}%
\pgfpathlineto{\pgfqpoint{4.346277in}{1.168882in}}%
\pgfpathlineto{\pgfqpoint{4.354122in}{1.177477in}}%
\pgfpathlineto{\pgfqpoint{4.361962in}{1.186216in}}%
\pgfpathlineto{\pgfqpoint{4.348126in}{1.188753in}}%
\pgfpathlineto{\pgfqpoint{4.334298in}{1.191401in}}%
\pgfpathlineto{\pgfqpoint{4.320478in}{1.194160in}}%
\pgfpathlineto{\pgfqpoint{4.306665in}{1.197030in}}%
\pgfpathlineto{\pgfqpoint{4.298815in}{1.188741in}}%
\pgfpathlineto{\pgfqpoint{4.290959in}{1.180601in}}%
\pgfpathlineto{\pgfqpoint{4.283098in}{1.172613in}}%
\pgfpathlineto{\pgfqpoint{4.275231in}{1.164781in}}%
\pgfpathclose%
\pgfusepath{fill}%
\end{pgfscope}%
\begin{pgfscope}%
\pgfpathrectangle{\pgfqpoint{1.254980in}{0.150000in}}{\pgfqpoint{5.490039in}{5.490039in}}%
\pgfusepath{clip}%
\pgfsetbuttcap%
\pgfsetroundjoin%
\definecolor{currentfill}{rgb}{0.269308,0.218818,0.509577}%
\pgfsetfillcolor{currentfill}%
\pgfsetfillopacity{0.700000}%
\pgfsetlinewidth{0.000000pt}%
\definecolor{currentstroke}{rgb}{0.000000,0.000000,0.000000}%
\pgfsetstrokecolor{currentstroke}%
\pgfsetdash{}{0pt}%
\pgfpathmoveto{\pgfqpoint{3.289379in}{1.612350in}}%
\pgfpathlineto{\pgfqpoint{3.303117in}{1.599279in}}%
\pgfpathlineto{\pgfqpoint{3.316854in}{1.586340in}}%
\pgfpathlineto{\pgfqpoint{3.330591in}{1.573532in}}%
\pgfpathlineto{\pgfqpoint{3.344328in}{1.560856in}}%
\pgfpathlineto{\pgfqpoint{3.352741in}{1.556946in}}%
\pgfpathlineto{\pgfqpoint{3.361139in}{1.553343in}}%
\pgfpathlineto{\pgfqpoint{3.369523in}{1.550042in}}%
\pgfpathlineto{\pgfqpoint{3.377892in}{1.547037in}}%
\pgfpathlineto{\pgfqpoint{3.364194in}{1.559156in}}%
\pgfpathlineto{\pgfqpoint{3.350497in}{1.571405in}}%
\pgfpathlineto{\pgfqpoint{3.336800in}{1.583785in}}%
\pgfpathlineto{\pgfqpoint{3.323103in}{1.596296in}}%
\pgfpathlineto{\pgfqpoint{3.314695in}{1.599852in}}%
\pgfpathlineto{\pgfqpoint{3.306271in}{1.603710in}}%
\pgfpathlineto{\pgfqpoint{3.297833in}{1.607874in}}%
\pgfpathlineto{\pgfqpoint{3.289379in}{1.612350in}}%
\pgfpathclose%
\pgfusepath{fill}%
\end{pgfscope}%
\begin{pgfscope}%
\pgfpathrectangle{\pgfqpoint{1.254980in}{0.150000in}}{\pgfqpoint{5.490039in}{5.490039in}}%
\pgfusepath{clip}%
\pgfsetbuttcap%
\pgfsetroundjoin%
\definecolor{currentfill}{rgb}{0.141935,0.526453,0.555991}%
\pgfsetfillcolor{currentfill}%
\pgfsetfillopacity{0.700000}%
\pgfsetlinewidth{0.000000pt}%
\definecolor{currentstroke}{rgb}{0.000000,0.000000,0.000000}%
\pgfsetstrokecolor{currentstroke}%
\pgfsetdash{}{0pt}%
\pgfpathmoveto{\pgfqpoint{2.626978in}{2.407292in}}%
\pgfpathlineto{\pgfqpoint{2.640889in}{2.387111in}}%
\pgfpathlineto{\pgfqpoint{2.654792in}{2.367096in}}%
\pgfpathlineto{\pgfqpoint{2.668688in}{2.347248in}}%
\pgfpathlineto{\pgfqpoint{2.682578in}{2.327564in}}%
\pgfpathlineto{\pgfqpoint{2.691586in}{2.316660in}}%
\pgfpathlineto{\pgfqpoint{2.700570in}{2.306127in}}%
\pgfpathlineto{\pgfqpoint{2.709531in}{2.295961in}}%
\pgfpathlineto{\pgfqpoint{2.718470in}{2.286156in}}%
\pgfpathlineto{\pgfqpoint{2.704638in}{2.305242in}}%
\pgfpathlineto{\pgfqpoint{2.690801in}{2.324493in}}%
\pgfpathlineto{\pgfqpoint{2.676957in}{2.343908in}}%
\pgfpathlineto{\pgfqpoint{2.663107in}{2.363489in}}%
\pgfpathlineto{\pgfqpoint{2.654110in}{2.373884in}}%
\pgfpathlineto{\pgfqpoint{2.645090in}{2.384646in}}%
\pgfpathlineto{\pgfqpoint{2.636046in}{2.395780in}}%
\pgfpathlineto{\pgfqpoint{2.626978in}{2.407292in}}%
\pgfpathclose%
\pgfusepath{fill}%
\end{pgfscope}%
\begin{pgfscope}%
\pgfpathrectangle{\pgfqpoint{1.254980in}{0.150000in}}{\pgfqpoint{5.490039in}{5.490039in}}%
\pgfusepath{clip}%
\pgfsetbuttcap%
\pgfsetroundjoin%
\definecolor{currentfill}{rgb}{0.281924,0.089666,0.412415}%
\pgfsetfillcolor{currentfill}%
\pgfsetfillopacity{0.700000}%
\pgfsetlinewidth{0.000000pt}%
\definecolor{currentstroke}{rgb}{0.000000,0.000000,0.000000}%
\pgfsetstrokecolor{currentstroke}%
\pgfsetdash{}{0pt}%
\pgfpathmoveto{\pgfqpoint{3.652043in}{1.331362in}}%
\pgfpathlineto{\pgfqpoint{3.665769in}{1.321880in}}%
\pgfpathlineto{\pgfqpoint{3.679498in}{1.312519in}}%
\pgfpathlineto{\pgfqpoint{3.693230in}{1.303278in}}%
\pgfpathlineto{\pgfqpoint{3.706964in}{1.294158in}}%
\pgfpathlineto{\pgfqpoint{3.715119in}{1.294686in}}%
\pgfpathlineto{\pgfqpoint{3.723263in}{1.295471in}}%
\pgfpathlineto{\pgfqpoint{3.731397in}{1.296511in}}%
\pgfpathlineto{\pgfqpoint{3.739520in}{1.297799in}}%
\pgfpathlineto{\pgfqpoint{3.725814in}{1.306391in}}%
\pgfpathlineto{\pgfqpoint{3.712112in}{1.315102in}}%
\pgfpathlineto{\pgfqpoint{3.698412in}{1.323934in}}%
\pgfpathlineto{\pgfqpoint{3.684715in}{1.332886in}}%
\pgfpathlineto{\pgfqpoint{3.676564in}{1.332120in}}%
\pgfpathlineto{\pgfqpoint{3.668401in}{1.331608in}}%
\pgfpathlineto{\pgfqpoint{3.660228in}{1.331354in}}%
\pgfpathlineto{\pgfqpoint{3.652043in}{1.331362in}}%
\pgfpathclose%
\pgfusepath{fill}%
\end{pgfscope}%
\begin{pgfscope}%
\pgfpathrectangle{\pgfqpoint{1.254980in}{0.150000in}}{\pgfqpoint{5.490039in}{5.490039in}}%
\pgfusepath{clip}%
\pgfsetbuttcap%
\pgfsetroundjoin%
\definecolor{currentfill}{rgb}{0.220124,0.725509,0.466226}%
\pgfsetfillcolor{currentfill}%
\pgfsetfillopacity{0.700000}%
\pgfsetlinewidth{0.000000pt}%
\definecolor{currentstroke}{rgb}{0.000000,0.000000,0.000000}%
\pgfsetstrokecolor{currentstroke}%
\pgfsetdash{}{0pt}%
\pgfpathmoveto{\pgfqpoint{2.272010in}{2.983286in}}%
\pgfpathlineto{\pgfqpoint{2.286109in}{2.958628in}}%
\pgfpathlineto{\pgfqpoint{2.300197in}{2.934168in}}%
\pgfpathlineto{\pgfqpoint{2.314273in}{2.909904in}}%
\pgfpathlineto{\pgfqpoint{2.328339in}{2.885834in}}%
\pgfpathlineto{\pgfqpoint{2.337694in}{2.872195in}}%
\pgfpathlineto{\pgfqpoint{2.347021in}{2.858947in}}%
\pgfpathlineto{\pgfqpoint{2.356320in}{2.846086in}}%
\pgfpathlineto{\pgfqpoint{2.365594in}{2.833606in}}%
\pgfpathlineto{\pgfqpoint{2.351596in}{2.857067in}}%
\pgfpathlineto{\pgfqpoint{2.337588in}{2.880721in}}%
\pgfpathlineto{\pgfqpoint{2.323569in}{2.904569in}}%
\pgfpathlineto{\pgfqpoint{2.309539in}{2.928614in}}%
\pgfpathlineto{\pgfqpoint{2.300199in}{2.941695in}}%
\pgfpathlineto{\pgfqpoint{2.290831in}{2.955164in}}%
\pgfpathlineto{\pgfqpoint{2.281435in}{2.969025in}}%
\pgfpathlineto{\pgfqpoint{2.272010in}{2.983286in}}%
\pgfpathclose%
\pgfusepath{fill}%
\end{pgfscope}%
\begin{pgfscope}%
\pgfpathrectangle{\pgfqpoint{1.254980in}{0.150000in}}{\pgfqpoint{5.490039in}{5.490039in}}%
\pgfusepath{clip}%
\pgfsetbuttcap%
\pgfsetroundjoin%
\definecolor{currentfill}{rgb}{0.270595,0.214069,0.507052}%
\pgfsetfillcolor{currentfill}%
\pgfsetfillopacity{0.700000}%
\pgfsetlinewidth{0.000000pt}%
\definecolor{currentstroke}{rgb}{0.000000,0.000000,0.000000}%
\pgfsetstrokecolor{currentstroke}%
\pgfsetdash{}{0pt}%
\pgfpathmoveto{\pgfqpoint{4.970046in}{1.526304in}}%
\pgfpathlineto{\pgfqpoint{4.984164in}{1.529715in}}%
\pgfpathlineto{\pgfqpoint{4.998295in}{1.533235in}}%
\pgfpathlineto{\pgfqpoint{5.012439in}{1.536865in}}%
\pgfpathlineto{\pgfqpoint{5.026596in}{1.540605in}}%
\pgfpathlineto{\pgfqpoint{5.034299in}{1.554773in}}%
\pgfpathlineto{\pgfqpoint{5.041998in}{1.568960in}}%
\pgfpathlineto{\pgfqpoint{5.049694in}{1.583161in}}%
\pgfpathlineto{\pgfqpoint{5.057386in}{1.597374in}}%
\pgfpathlineto{\pgfqpoint{5.043226in}{1.593298in}}%
\pgfpathlineto{\pgfqpoint{5.029079in}{1.589332in}}%
\pgfpathlineto{\pgfqpoint{5.014945in}{1.585475in}}%
\pgfpathlineto{\pgfqpoint{5.000825in}{1.581729in}}%
\pgfpathlineto{\pgfqpoint{4.993136in}{1.567845in}}%
\pgfpathlineto{\pgfqpoint{4.985443in}{1.553978in}}%
\pgfpathlineto{\pgfqpoint{4.977747in}{1.540130in}}%
\pgfpathlineto{\pgfqpoint{4.970046in}{1.526304in}}%
\pgfpathclose%
\pgfusepath{fill}%
\end{pgfscope}%
\begin{pgfscope}%
\pgfpathrectangle{\pgfqpoint{1.254980in}{0.150000in}}{\pgfqpoint{5.490039in}{5.490039in}}%
\pgfusepath{clip}%
\pgfsetbuttcap%
\pgfsetroundjoin%
\definecolor{currentfill}{rgb}{0.274128,0.199721,0.498911}%
\pgfsetfillcolor{currentfill}%
\pgfsetfillopacity{0.700000}%
\pgfsetlinewidth{0.000000pt}%
\definecolor{currentstroke}{rgb}{0.000000,0.000000,0.000000}%
\pgfsetstrokecolor{currentstroke}%
\pgfsetdash{}{0pt}%
\pgfpathmoveto{\pgfqpoint{3.344328in}{1.560856in}}%
\pgfpathlineto{\pgfqpoint{3.358065in}{1.548310in}}%
\pgfpathlineto{\pgfqpoint{3.371801in}{1.535894in}}%
\pgfpathlineto{\pgfqpoint{3.385538in}{1.523608in}}%
\pgfpathlineto{\pgfqpoint{3.399275in}{1.511451in}}%
\pgfpathlineto{\pgfqpoint{3.407649in}{1.508105in}}%
\pgfpathlineto{\pgfqpoint{3.416008in}{1.505061in}}%
\pgfpathlineto{\pgfqpoint{3.424354in}{1.502314in}}%
\pgfpathlineto{\pgfqpoint{3.432685in}{1.499859in}}%
\pgfpathlineto{\pgfqpoint{3.418986in}{1.511460in}}%
\pgfpathlineto{\pgfqpoint{3.405288in}{1.523190in}}%
\pgfpathlineto{\pgfqpoint{3.391590in}{1.535049in}}%
\pgfpathlineto{\pgfqpoint{3.377892in}{1.547037in}}%
\pgfpathlineto{\pgfqpoint{3.369523in}{1.550042in}}%
\pgfpathlineto{\pgfqpoint{3.361139in}{1.553343in}}%
\pgfpathlineto{\pgfqpoint{3.352741in}{1.556946in}}%
\pgfpathlineto{\pgfqpoint{3.344328in}{1.560856in}}%
\pgfpathclose%
\pgfusepath{fill}%
\end{pgfscope}%
\begin{pgfscope}%
\pgfpathrectangle{\pgfqpoint{1.254980in}{0.150000in}}{\pgfqpoint{5.490039in}{5.490039in}}%
\pgfusepath{clip}%
\pgfsetbuttcap%
\pgfsetroundjoin%
\definecolor{currentfill}{rgb}{0.131172,0.555899,0.552459}%
\pgfsetfillcolor{currentfill}%
\pgfsetfillopacity{0.700000}%
\pgfsetlinewidth{0.000000pt}%
\definecolor{currentstroke}{rgb}{0.000000,0.000000,0.000000}%
\pgfsetstrokecolor{currentstroke}%
\pgfsetdash{}{0pt}%
\pgfpathmoveto{\pgfqpoint{2.571266in}{2.489707in}}%
\pgfpathlineto{\pgfqpoint{2.585205in}{2.468848in}}%
\pgfpathlineto{\pgfqpoint{2.599137in}{2.448160in}}%
\pgfpathlineto{\pgfqpoint{2.613061in}{2.427642in}}%
\pgfpathlineto{\pgfqpoint{2.626978in}{2.407292in}}%
\pgfpathlineto{\pgfqpoint{2.636046in}{2.395780in}}%
\pgfpathlineto{\pgfqpoint{2.645090in}{2.384646in}}%
\pgfpathlineto{\pgfqpoint{2.654110in}{2.373884in}}%
\pgfpathlineto{\pgfqpoint{2.663107in}{2.363489in}}%
\pgfpathlineto{\pgfqpoint{2.649250in}{2.383237in}}%
\pgfpathlineto{\pgfqpoint{2.635386in}{2.403152in}}%
\pgfpathlineto{\pgfqpoint{2.621516in}{2.423236in}}%
\pgfpathlineto{\pgfqpoint{2.607638in}{2.443490in}}%
\pgfpathlineto{\pgfqpoint{2.598581in}{2.454479in}}%
\pgfpathlineto{\pgfqpoint{2.589501in}{2.465841in}}%
\pgfpathlineto{\pgfqpoint{2.580396in}{2.477582in}}%
\pgfpathlineto{\pgfqpoint{2.571266in}{2.489707in}}%
\pgfpathclose%
\pgfusepath{fill}%
\end{pgfscope}%
\begin{pgfscope}%
\pgfpathrectangle{\pgfqpoint{1.254980in}{0.150000in}}{\pgfqpoint{5.490039in}{5.490039in}}%
\pgfusepath{clip}%
\pgfsetbuttcap%
\pgfsetroundjoin%
\definecolor{currentfill}{rgb}{0.280267,0.073417,0.397163}%
\pgfsetfillcolor{currentfill}%
\pgfsetfillopacity{0.700000}%
\pgfsetlinewidth{0.000000pt}%
\definecolor{currentstroke}{rgb}{0.000000,0.000000,0.000000}%
\pgfsetstrokecolor{currentstroke}%
\pgfsetdash{}{0pt}%
\pgfpathmoveto{\pgfqpoint{4.591027in}{1.249424in}}%
\pgfpathlineto{\pgfqpoint{4.604966in}{1.249182in}}%
\pgfpathlineto{\pgfqpoint{4.618915in}{1.249049in}}%
\pgfpathlineto{\pgfqpoint{4.632874in}{1.249025in}}%
\pgfpathlineto{\pgfqpoint{4.646844in}{1.249111in}}%
\pgfpathlineto{\pgfqpoint{4.654618in}{1.260649in}}%
\pgfpathlineto{\pgfqpoint{4.662388in}{1.272281in}}%
\pgfpathlineto{\pgfqpoint{4.670154in}{1.284002in}}%
\pgfpathlineto{\pgfqpoint{4.677916in}{1.295809in}}%
\pgfpathlineto{\pgfqpoint{4.663950in}{1.295313in}}%
\pgfpathlineto{\pgfqpoint{4.649994in}{1.294925in}}%
\pgfpathlineto{\pgfqpoint{4.636049in}{1.294647in}}%
\pgfpathlineto{\pgfqpoint{4.622114in}{1.294478in}}%
\pgfpathlineto{\pgfqpoint{4.614348in}{1.283076in}}%
\pgfpathlineto{\pgfqpoint{4.606579in}{1.271764in}}%
\pgfpathlineto{\pgfqpoint{4.598805in}{1.260546in}}%
\pgfpathlineto{\pgfqpoint{4.591027in}{1.249424in}}%
\pgfpathclose%
\pgfusepath{fill}%
\end{pgfscope}%
\begin{pgfscope}%
\pgfpathrectangle{\pgfqpoint{1.254980in}{0.150000in}}{\pgfqpoint{5.490039in}{5.490039in}}%
\pgfusepath{clip}%
\pgfsetbuttcap%
\pgfsetroundjoin%
\definecolor{currentfill}{rgb}{0.269944,0.014625,0.341379}%
\pgfsetfillcolor{currentfill}%
\pgfsetfillopacity{0.700000}%
\pgfsetlinewidth{0.000000pt}%
\definecolor{currentstroke}{rgb}{0.000000,0.000000,0.000000}%
\pgfsetstrokecolor{currentstroke}%
\pgfsetdash{}{0pt}%
\pgfpathmoveto{\pgfqpoint{4.046383in}{1.168405in}}%
\pgfpathlineto{\pgfqpoint{4.060158in}{1.162769in}}%
\pgfpathlineto{\pgfqpoint{4.073940in}{1.157246in}}%
\pgfpathlineto{\pgfqpoint{4.087727in}{1.151836in}}%
\pgfpathlineto{\pgfqpoint{4.101520in}{1.146540in}}%
\pgfpathlineto{\pgfqpoint{4.109466in}{1.151990in}}%
\pgfpathlineto{\pgfqpoint{4.117405in}{1.157637in}}%
\pgfpathlineto{\pgfqpoint{4.125337in}{1.163477in}}%
\pgfpathlineto{\pgfqpoint{4.133263in}{1.169504in}}%
\pgfpathlineto{\pgfqpoint{4.119487in}{1.174311in}}%
\pgfpathlineto{\pgfqpoint{4.105717in}{1.179230in}}%
\pgfpathlineto{\pgfqpoint{4.091953in}{1.184263in}}%
\pgfpathlineto{\pgfqpoint{4.078196in}{1.189409in}}%
\pgfpathlineto{\pgfqpoint{4.070253in}{1.183866in}}%
\pgfpathlineto{\pgfqpoint{4.062303in}{1.178514in}}%
\pgfpathlineto{\pgfqpoint{4.054347in}{1.173360in}}%
\pgfpathlineto{\pgfqpoint{4.046383in}{1.168405in}}%
\pgfpathclose%
\pgfusepath{fill}%
\end{pgfscope}%
\begin{pgfscope}%
\pgfpathrectangle{\pgfqpoint{1.254980in}{0.150000in}}{\pgfqpoint{5.490039in}{5.490039in}}%
\pgfusepath{clip}%
\pgfsetbuttcap%
\pgfsetroundjoin%
\definecolor{currentfill}{rgb}{0.282656,0.100196,0.422160}%
\pgfsetfillcolor{currentfill}%
\pgfsetfillopacity{0.700000}%
\pgfsetlinewidth{0.000000pt}%
\definecolor{currentstroke}{rgb}{0.000000,0.000000,0.000000}%
\pgfsetstrokecolor{currentstroke}%
\pgfsetdash{}{0pt}%
\pgfpathmoveto{\pgfqpoint{4.677916in}{1.295809in}}%
\pgfpathlineto{\pgfqpoint{4.691893in}{1.296415in}}%
\pgfpathlineto{\pgfqpoint{4.705880in}{1.297130in}}%
\pgfpathlineto{\pgfqpoint{4.719879in}{1.297954in}}%
\pgfpathlineto{\pgfqpoint{4.733888in}{1.298887in}}%
\pgfpathlineto{\pgfqpoint{4.741644in}{1.311179in}}%
\pgfpathlineto{\pgfqpoint{4.749397in}{1.323548in}}%
\pgfpathlineto{\pgfqpoint{4.757147in}{1.335989in}}%
\pgfpathlineto{\pgfqpoint{4.764892in}{1.348499in}}%
\pgfpathlineto{\pgfqpoint{4.750885in}{1.347170in}}%
\pgfpathlineto{\pgfqpoint{4.736888in}{1.345950in}}%
\pgfpathlineto{\pgfqpoint{4.722903in}{1.344839in}}%
\pgfpathlineto{\pgfqpoint{4.708928in}{1.343837in}}%
\pgfpathlineto{\pgfqpoint{4.701181in}{1.331717in}}%
\pgfpathlineto{\pgfqpoint{4.693430in}{1.319670in}}%
\pgfpathlineto{\pgfqpoint{4.685675in}{1.307700in}}%
\pgfpathlineto{\pgfqpoint{4.677916in}{1.295809in}}%
\pgfpathclose%
\pgfusepath{fill}%
\end{pgfscope}%
\begin{pgfscope}%
\pgfpathrectangle{\pgfqpoint{1.254980in}{0.150000in}}{\pgfqpoint{5.490039in}{5.490039in}}%
\pgfusepath{clip}%
\pgfsetbuttcap%
\pgfsetroundjoin%
\definecolor{currentfill}{rgb}{0.277018,0.050344,0.375715}%
\pgfsetfillcolor{currentfill}%
\pgfsetfillopacity{0.700000}%
\pgfsetlinewidth{0.000000pt}%
\definecolor{currentstroke}{rgb}{0.000000,0.000000,0.000000}%
\pgfsetstrokecolor{currentstroke}%
\pgfsetdash{}{0pt}%
\pgfpathmoveto{\pgfqpoint{4.504195in}{1.209741in}}%
\pgfpathlineto{\pgfqpoint{4.518101in}{1.208635in}}%
\pgfpathlineto{\pgfqpoint{4.532017in}{1.207639in}}%
\pgfpathlineto{\pgfqpoint{4.545941in}{1.206752in}}%
\pgfpathlineto{\pgfqpoint{4.559875in}{1.205974in}}%
\pgfpathlineto{\pgfqpoint{4.567670in}{1.216674in}}%
\pgfpathlineto{\pgfqpoint{4.575459in}{1.227485in}}%
\pgfpathlineto{\pgfqpoint{4.583245in}{1.238403in}}%
\pgfpathlineto{\pgfqpoint{4.591027in}{1.249424in}}%
\pgfpathlineto{\pgfqpoint{4.577098in}{1.249775in}}%
\pgfpathlineto{\pgfqpoint{4.563179in}{1.250236in}}%
\pgfpathlineto{\pgfqpoint{4.549269in}{1.250806in}}%
\pgfpathlineto{\pgfqpoint{4.535369in}{1.251486in}}%
\pgfpathlineto{\pgfqpoint{4.527582in}{1.240885in}}%
\pgfpathlineto{\pgfqpoint{4.519791in}{1.230391in}}%
\pgfpathlineto{\pgfqpoint{4.511995in}{1.220009in}}%
\pgfpathlineto{\pgfqpoint{4.504195in}{1.209741in}}%
\pgfpathclose%
\pgfusepath{fill}%
\end{pgfscope}%
\begin{pgfscope}%
\pgfpathrectangle{\pgfqpoint{1.254980in}{0.150000in}}{\pgfqpoint{5.490039in}{5.490039in}}%
\pgfusepath{clip}%
\pgfsetbuttcap%
\pgfsetroundjoin%
\definecolor{currentfill}{rgb}{0.268510,0.009605,0.335427}%
\pgfsetfillcolor{currentfill}%
\pgfsetfillopacity{0.700000}%
\pgfsetlinewidth{0.000000pt}%
\definecolor{currentstroke}{rgb}{0.000000,0.000000,0.000000}%
\pgfsetstrokecolor{currentstroke}%
\pgfsetdash{}{0pt}%
\pgfpathmoveto{\pgfqpoint{4.188431in}{1.151401in}}%
\pgfpathlineto{\pgfqpoint{4.202240in}{1.147156in}}%
\pgfpathlineto{\pgfqpoint{4.216056in}{1.143022in}}%
\pgfpathlineto{\pgfqpoint{4.229878in}{1.139000in}}%
\pgfpathlineto{\pgfqpoint{4.243708in}{1.135089in}}%
\pgfpathlineto{\pgfqpoint{4.251597in}{1.142259in}}%
\pgfpathlineto{\pgfqpoint{4.259481in}{1.149600in}}%
\pgfpathlineto{\pgfqpoint{4.267359in}{1.157109in}}%
\pgfpathlineto{\pgfqpoint{4.275231in}{1.164781in}}%
\pgfpathlineto{\pgfqpoint{4.261415in}{1.168219in}}%
\pgfpathlineto{\pgfqpoint{4.247606in}{1.171768in}}%
\pgfpathlineto{\pgfqpoint{4.233804in}{1.175429in}}%
\pgfpathlineto{\pgfqpoint{4.220009in}{1.179201in}}%
\pgfpathlineto{\pgfqpoint{4.212124in}{1.171996in}}%
\pgfpathlineto{\pgfqpoint{4.204233in}{1.164958in}}%
\pgfpathlineto{\pgfqpoint{4.196335in}{1.158092in}}%
\pgfpathlineto{\pgfqpoint{4.188431in}{1.151401in}}%
\pgfpathclose%
\pgfusepath{fill}%
\end{pgfscope}%
\begin{pgfscope}%
\pgfpathrectangle{\pgfqpoint{1.254980in}{0.150000in}}{\pgfqpoint{5.490039in}{5.490039in}}%
\pgfusepath{clip}%
\pgfsetbuttcap%
\pgfsetroundjoin%
\definecolor{currentfill}{rgb}{0.258965,0.251537,0.524736}%
\pgfsetfillcolor{currentfill}%
\pgfsetfillopacity{0.700000}%
\pgfsetlinewidth{0.000000pt}%
\definecolor{currentstroke}{rgb}{0.000000,0.000000,0.000000}%
\pgfsetstrokecolor{currentstroke}%
\pgfsetdash{}{0pt}%
\pgfpathmoveto{\pgfqpoint{5.057386in}{1.597374in}}%
\pgfpathlineto{\pgfqpoint{5.071559in}{1.601560in}}%
\pgfpathlineto{\pgfqpoint{5.085746in}{1.605855in}}%
\pgfpathlineto{\pgfqpoint{5.099947in}{1.610260in}}%
\pgfpathlineto{\pgfqpoint{5.107638in}{1.624729in}}%
\pgfpathlineto{\pgfqpoint{5.115325in}{1.639202in}}%
\pgfpathlineto{\pgfqpoint{5.123008in}{1.653676in}}%
\pgfpathlineto{\pgfqpoint{5.130688in}{1.668150in}}%
\pgfpathlineto{\pgfqpoint{5.116483in}{1.663423in}}%
\pgfpathlineto{\pgfqpoint{5.102292in}{1.658807in}}%
\pgfpathlineto{\pgfqpoint{5.088115in}{1.654300in}}%
\pgfpathlineto{\pgfqpoint{5.080439in}{1.640063in}}%
\pgfpathlineto{\pgfqpoint{5.072758in}{1.625828in}}%
\pgfpathlineto{\pgfqpoint{5.065074in}{1.611597in}}%
\pgfpathlineto{\pgfqpoint{5.057386in}{1.597374in}}%
\pgfpathclose%
\pgfusepath{fill}%
\end{pgfscope}%
\begin{pgfscope}%
\pgfpathrectangle{\pgfqpoint{1.254980in}{0.150000in}}{\pgfqpoint{5.490039in}{5.490039in}}%
\pgfusepath{clip}%
\pgfsetbuttcap%
\pgfsetroundjoin%
\definecolor{currentfill}{rgb}{0.278012,0.180367,0.486697}%
\pgfsetfillcolor{currentfill}%
\pgfsetfillopacity{0.700000}%
\pgfsetlinewidth{0.000000pt}%
\definecolor{currentstroke}{rgb}{0.000000,0.000000,0.000000}%
\pgfsetstrokecolor{currentstroke}%
\pgfsetdash{}{0pt}%
\pgfpathmoveto{\pgfqpoint{3.399275in}{1.511451in}}%
\pgfpathlineto{\pgfqpoint{3.413012in}{1.499422in}}%
\pgfpathlineto{\pgfqpoint{3.426750in}{1.487522in}}%
\pgfpathlineto{\pgfqpoint{3.440488in}{1.475749in}}%
\pgfpathlineto{\pgfqpoint{3.454227in}{1.464104in}}%
\pgfpathlineto{\pgfqpoint{3.462563in}{1.461320in}}%
\pgfpathlineto{\pgfqpoint{3.470886in}{1.458834in}}%
\pgfpathlineto{\pgfqpoint{3.479194in}{1.456639in}}%
\pgfpathlineto{\pgfqpoint{3.487490in}{1.454731in}}%
\pgfpathlineto{\pgfqpoint{3.473787in}{1.465822in}}%
\pgfpathlineto{\pgfqpoint{3.460086in}{1.477040in}}%
\pgfpathlineto{\pgfqpoint{3.446385in}{1.488386in}}%
\pgfpathlineto{\pgfqpoint{3.432685in}{1.499859in}}%
\pgfpathlineto{\pgfqpoint{3.424354in}{1.502314in}}%
\pgfpathlineto{\pgfqpoint{3.416008in}{1.505061in}}%
\pgfpathlineto{\pgfqpoint{3.407649in}{1.508105in}}%
\pgfpathlineto{\pgfqpoint{3.399275in}{1.511451in}}%
\pgfpathclose%
\pgfusepath{fill}%
\end{pgfscope}%
\begin{pgfscope}%
\pgfpathrectangle{\pgfqpoint{1.254980in}{0.150000in}}{\pgfqpoint{5.490039in}{5.490039in}}%
\pgfusepath{clip}%
\pgfsetbuttcap%
\pgfsetroundjoin%
\definecolor{currentfill}{rgb}{0.283072,0.130895,0.449241}%
\pgfsetfillcolor{currentfill}%
\pgfsetfillopacity{0.700000}%
\pgfsetlinewidth{0.000000pt}%
\definecolor{currentstroke}{rgb}{0.000000,0.000000,0.000000}%
\pgfsetstrokecolor{currentstroke}%
\pgfsetdash{}{0pt}%
\pgfpathmoveto{\pgfqpoint{4.764892in}{1.348499in}}%
\pgfpathlineto{\pgfqpoint{4.778911in}{1.349938in}}%
\pgfpathlineto{\pgfqpoint{4.792941in}{1.351485in}}%
\pgfpathlineto{\pgfqpoint{4.806983in}{1.353141in}}%
\pgfpathlineto{\pgfqpoint{4.821036in}{1.354907in}}%
\pgfpathlineto{\pgfqpoint{4.828778in}{1.367871in}}%
\pgfpathlineto{\pgfqpoint{4.836516in}{1.380895in}}%
\pgfpathlineto{\pgfqpoint{4.844251in}{1.393975in}}%
\pgfpathlineto{\pgfqpoint{4.851982in}{1.407109in}}%
\pgfpathlineto{\pgfqpoint{4.837929in}{1.404962in}}%
\pgfpathlineto{\pgfqpoint{4.823887in}{1.402924in}}%
\pgfpathlineto{\pgfqpoint{4.809857in}{1.400995in}}%
\pgfpathlineto{\pgfqpoint{4.795839in}{1.399176in}}%
\pgfpathlineto{\pgfqpoint{4.788108in}{1.386418in}}%
\pgfpathlineto{\pgfqpoint{4.780373in}{1.373717in}}%
\pgfpathlineto{\pgfqpoint{4.772634in}{1.361076in}}%
\pgfpathlineto{\pgfqpoint{4.764892in}{1.348499in}}%
\pgfpathclose%
\pgfusepath{fill}%
\end{pgfscope}%
\begin{pgfscope}%
\pgfpathrectangle{\pgfqpoint{1.254980in}{0.150000in}}{\pgfqpoint{5.490039in}{5.490039in}}%
\pgfusepath{clip}%
\pgfsetbuttcap%
\pgfsetroundjoin%
\definecolor{currentfill}{rgb}{0.273809,0.031497,0.358853}%
\pgfsetfillcolor{currentfill}%
\pgfsetfillopacity{0.700000}%
\pgfsetlinewidth{0.000000pt}%
\definecolor{currentstroke}{rgb}{0.000000,0.000000,0.000000}%
\pgfsetstrokecolor{currentstroke}%
\pgfsetdash{}{0pt}%
\pgfpathmoveto{\pgfqpoint{3.904271in}{1.203943in}}%
\pgfpathlineto{\pgfqpoint{3.918026in}{1.196883in}}%
\pgfpathlineto{\pgfqpoint{3.931787in}{1.189937in}}%
\pgfpathlineto{\pgfqpoint{3.945552in}{1.183107in}}%
\pgfpathlineto{\pgfqpoint{3.959322in}{1.176392in}}%
\pgfpathlineto{\pgfqpoint{3.967337in}{1.179988in}}%
\pgfpathlineto{\pgfqpoint{3.975344in}{1.183806in}}%
\pgfpathlineto{\pgfqpoint{3.983344in}{1.187841in}}%
\pgfpathlineto{\pgfqpoint{3.991335in}{1.192089in}}%
\pgfpathlineto{\pgfqpoint{3.977586in}{1.198297in}}%
\pgfpathlineto{\pgfqpoint{3.963843in}{1.204619in}}%
\pgfpathlineto{\pgfqpoint{3.950104in}{1.211056in}}%
\pgfpathlineto{\pgfqpoint{3.936371in}{1.217608in}}%
\pgfpathlineto{\pgfqpoint{3.928358in}{1.213861in}}%
\pgfpathlineto{\pgfqpoint{3.920338in}{1.210332in}}%
\pgfpathlineto{\pgfqpoint{3.912308in}{1.207024in}}%
\pgfpathlineto{\pgfqpoint{3.904271in}{1.203943in}}%
\pgfpathclose%
\pgfusepath{fill}%
\end{pgfscope}%
\begin{pgfscope}%
\pgfpathrectangle{\pgfqpoint{1.254980in}{0.150000in}}{\pgfqpoint{5.490039in}{5.490039in}}%
\pgfusepath{clip}%
\pgfsetbuttcap%
\pgfsetroundjoin%
\definecolor{currentfill}{rgb}{0.272594,0.025563,0.353093}%
\pgfsetfillcolor{currentfill}%
\pgfsetfillopacity{0.700000}%
\pgfsetlinewidth{0.000000pt}%
\definecolor{currentstroke}{rgb}{0.000000,0.000000,0.000000}%
\pgfsetstrokecolor{currentstroke}%
\pgfsetdash{}{0pt}%
\pgfpathmoveto{\pgfqpoint{4.417389in}{1.177171in}}%
\pgfpathlineto{\pgfqpoint{4.431266in}{1.175185in}}%
\pgfpathlineto{\pgfqpoint{4.445153in}{1.173308in}}%
\pgfpathlineto{\pgfqpoint{4.459048in}{1.171542in}}%
\pgfpathlineto{\pgfqpoint{4.472952in}{1.169884in}}%
\pgfpathlineto{\pgfqpoint{4.480769in}{1.179659in}}%
\pgfpathlineto{\pgfqpoint{4.488582in}{1.189562in}}%
\pgfpathlineto{\pgfqpoint{4.496391in}{1.199591in}}%
\pgfpathlineto{\pgfqpoint{4.504195in}{1.209741in}}%
\pgfpathlineto{\pgfqpoint{4.490299in}{1.210956in}}%
\pgfpathlineto{\pgfqpoint{4.476411in}{1.212281in}}%
\pgfpathlineto{\pgfqpoint{4.462532in}{1.213716in}}%
\pgfpathlineto{\pgfqpoint{4.448663in}{1.215260in}}%
\pgfpathlineto{\pgfqpoint{4.440851in}{1.205546in}}%
\pgfpathlineto{\pgfqpoint{4.433035in}{1.195957in}}%
\pgfpathlineto{\pgfqpoint{4.425214in}{1.186498in}}%
\pgfpathlineto{\pgfqpoint{4.417389in}{1.177171in}}%
\pgfpathclose%
\pgfusepath{fill}%
\end{pgfscope}%
\begin{pgfscope}%
\pgfpathrectangle{\pgfqpoint{1.254980in}{0.150000in}}{\pgfqpoint{5.490039in}{5.490039in}}%
\pgfusepath{clip}%
\pgfsetbuttcap%
\pgfsetroundjoin%
\definecolor{currentfill}{rgb}{0.280894,0.078907,0.402329}%
\pgfsetfillcolor{currentfill}%
\pgfsetfillopacity{0.700000}%
\pgfsetlinewidth{0.000000pt}%
\definecolor{currentstroke}{rgb}{0.000000,0.000000,0.000000}%
\pgfsetstrokecolor{currentstroke}%
\pgfsetdash{}{0pt}%
\pgfpathmoveto{\pgfqpoint{3.706964in}{1.294158in}}%
\pgfpathlineto{\pgfqpoint{3.720701in}{1.285158in}}%
\pgfpathlineto{\pgfqpoint{3.734441in}{1.276277in}}%
\pgfpathlineto{\pgfqpoint{3.748184in}{1.267516in}}%
\pgfpathlineto{\pgfqpoint{3.761931in}{1.258873in}}%
\pgfpathlineto{\pgfqpoint{3.770057in}{1.259936in}}%
\pgfpathlineto{\pgfqpoint{3.778174in}{1.261252in}}%
\pgfpathlineto{\pgfqpoint{3.786280in}{1.262818in}}%
\pgfpathlineto{\pgfqpoint{3.794376in}{1.264627in}}%
\pgfpathlineto{\pgfqpoint{3.780657in}{1.272742in}}%
\pgfpathlineto{\pgfqpoint{3.766941in}{1.280975in}}%
\pgfpathlineto{\pgfqpoint{3.753229in}{1.289328in}}%
\pgfpathlineto{\pgfqpoint{3.739520in}{1.297799in}}%
\pgfpathlineto{\pgfqpoint{3.731397in}{1.296511in}}%
\pgfpathlineto{\pgfqpoint{3.723263in}{1.295471in}}%
\pgfpathlineto{\pgfqpoint{3.715119in}{1.294686in}}%
\pgfpathlineto{\pgfqpoint{3.706964in}{1.294158in}}%
\pgfpathclose%
\pgfusepath{fill}%
\end{pgfscope}%
\begin{pgfscope}%
\pgfpathrectangle{\pgfqpoint{1.254980in}{0.150000in}}{\pgfqpoint{5.490039in}{5.490039in}}%
\pgfusepath{clip}%
\pgfsetbuttcap%
\pgfsetroundjoin%
\definecolor{currentfill}{rgb}{0.121831,0.589055,0.545623}%
\pgfsetfillcolor{currentfill}%
\pgfsetfillopacity{0.700000}%
\pgfsetlinewidth{0.000000pt}%
\definecolor{currentstroke}{rgb}{0.000000,0.000000,0.000000}%
\pgfsetstrokecolor{currentstroke}%
\pgfsetdash{}{0pt}%
\pgfpathmoveto{\pgfqpoint{2.515430in}{2.574871in}}%
\pgfpathlineto{\pgfqpoint{2.529401in}{2.553318in}}%
\pgfpathlineto{\pgfqpoint{2.543364in}{2.531941in}}%
\pgfpathlineto{\pgfqpoint{2.557319in}{2.510737in}}%
\pgfpathlineto{\pgfqpoint{2.571266in}{2.489707in}}%
\pgfpathlineto{\pgfqpoint{2.580396in}{2.477582in}}%
\pgfpathlineto{\pgfqpoint{2.589501in}{2.465841in}}%
\pgfpathlineto{\pgfqpoint{2.598581in}{2.454479in}}%
\pgfpathlineto{\pgfqpoint{2.607638in}{2.443490in}}%
\pgfpathlineto{\pgfqpoint{2.593753in}{2.463914in}}%
\pgfpathlineto{\pgfqpoint{2.579861in}{2.484510in}}%
\pgfpathlineto{\pgfqpoint{2.565961in}{2.505279in}}%
\pgfpathlineto{\pgfqpoint{2.552053in}{2.526222in}}%
\pgfpathlineto{\pgfqpoint{2.542935in}{2.537810in}}%
\pgfpathlineto{\pgfqpoint{2.533792in}{2.549777in}}%
\pgfpathlineto{\pgfqpoint{2.524624in}{2.562129in}}%
\pgfpathlineto{\pgfqpoint{2.515430in}{2.574871in}}%
\pgfpathclose%
\pgfusepath{fill}%
\end{pgfscope}%
\begin{pgfscope}%
\pgfpathrectangle{\pgfqpoint{1.254980in}{0.150000in}}{\pgfqpoint{5.490039in}{5.490039in}}%
\pgfusepath{clip}%
\pgfsetbuttcap%
\pgfsetroundjoin%
\definecolor{currentfill}{rgb}{0.616293,0.852709,0.230052}%
\pgfsetfillcolor{currentfill}%
\pgfsetfillopacity{0.700000}%
\pgfsetlinewidth{0.000000pt}%
\definecolor{currentstroke}{rgb}{0.000000,0.000000,0.000000}%
\pgfsetstrokecolor{currentstroke}%
\pgfsetdash{}{0pt}%
\pgfpathmoveto{\pgfqpoint{2.026359in}{3.452876in}}%
\pgfpathlineto{\pgfqpoint{2.040648in}{3.424605in}}%
\pgfpathlineto{\pgfqpoint{2.054923in}{3.396560in}}%
\pgfpathlineto{\pgfqpoint{2.069182in}{3.368739in}}%
\pgfpathlineto{\pgfqpoint{2.083428in}{3.341142in}}%
\pgfpathlineto{\pgfqpoint{2.093027in}{3.325999in}}%
\pgfpathlineto{\pgfqpoint{2.102596in}{3.311259in}}%
\pgfpathlineto{\pgfqpoint{2.112135in}{3.296915in}}%
\pgfpathlineto{\pgfqpoint{2.121645in}{3.282963in}}%
\pgfpathlineto{\pgfqpoint{2.107474in}{3.309945in}}%
\pgfpathlineto{\pgfqpoint{2.093289in}{3.337148in}}%
\pgfpathlineto{\pgfqpoint{2.079090in}{3.364574in}}%
\pgfpathlineto{\pgfqpoint{2.064876in}{3.392225in}}%
\pgfpathlineto{\pgfqpoint{2.055293in}{3.406785in}}%
\pgfpathlineto{\pgfqpoint{2.045679in}{3.421744in}}%
\pgfpathlineto{\pgfqpoint{2.036034in}{3.437105in}}%
\pgfpathlineto{\pgfqpoint{2.026359in}{3.452876in}}%
\pgfpathclose%
\pgfusepath{fill}%
\end{pgfscope}%
\begin{pgfscope}%
\pgfpathrectangle{\pgfqpoint{1.254980in}{0.150000in}}{\pgfqpoint{5.490039in}{5.490039in}}%
\pgfusepath{clip}%
\pgfsetbuttcap%
\pgfsetroundjoin%
\definecolor{currentfill}{rgb}{0.296479,0.761561,0.424223}%
\pgfsetfillcolor{currentfill}%
\pgfsetfillopacity{0.700000}%
\pgfsetlinewidth{0.000000pt}%
\definecolor{currentstroke}{rgb}{0.000000,0.000000,0.000000}%
\pgfsetstrokecolor{currentstroke}%
\pgfsetdash{}{0pt}%
\pgfpathmoveto{\pgfqpoint{2.215497in}{3.083921in}}%
\pgfpathlineto{\pgfqpoint{2.229643in}{3.058459in}}%
\pgfpathlineto{\pgfqpoint{2.243777in}{3.033200in}}%
\pgfpathlineto{\pgfqpoint{2.257900in}{3.008143in}}%
\pgfpathlineto{\pgfqpoint{2.272010in}{2.983286in}}%
\pgfpathlineto{\pgfqpoint{2.281435in}{2.969025in}}%
\pgfpathlineto{\pgfqpoint{2.290831in}{2.955164in}}%
\pgfpathlineto{\pgfqpoint{2.300199in}{2.941695in}}%
\pgfpathlineto{\pgfqpoint{2.309539in}{2.928614in}}%
\pgfpathlineto{\pgfqpoint{2.295499in}{2.952856in}}%
\pgfpathlineto{\pgfqpoint{2.281447in}{2.977297in}}%
\pgfpathlineto{\pgfqpoint{2.267384in}{3.001938in}}%
\pgfpathlineto{\pgfqpoint{2.253309in}{3.026780in}}%
\pgfpathlineto{\pgfqpoint{2.243899in}{3.040469in}}%
\pgfpathlineto{\pgfqpoint{2.234460in}{3.054552in}}%
\pgfpathlineto{\pgfqpoint{2.224993in}{3.069034in}}%
\pgfpathlineto{\pgfqpoint{2.215497in}{3.083921in}}%
\pgfpathclose%
\pgfusepath{fill}%
\end{pgfscope}%
\begin{pgfscope}%
\pgfpathrectangle{\pgfqpoint{1.254980in}{0.150000in}}{\pgfqpoint{5.490039in}{5.490039in}}%
\pgfusepath{clip}%
\pgfsetbuttcap%
\pgfsetroundjoin%
\definecolor{currentfill}{rgb}{0.280868,0.160771,0.472899}%
\pgfsetfillcolor{currentfill}%
\pgfsetfillopacity{0.700000}%
\pgfsetlinewidth{0.000000pt}%
\definecolor{currentstroke}{rgb}{0.000000,0.000000,0.000000}%
\pgfsetstrokecolor{currentstroke}%
\pgfsetdash{}{0pt}%
\pgfpathmoveto{\pgfqpoint{4.851982in}{1.407109in}}%
\pgfpathlineto{\pgfqpoint{4.866047in}{1.409365in}}%
\pgfpathlineto{\pgfqpoint{4.880125in}{1.411730in}}%
\pgfpathlineto{\pgfqpoint{4.894214in}{1.414205in}}%
\pgfpathlineto{\pgfqpoint{4.908316in}{1.416788in}}%
\pgfpathlineto{\pgfqpoint{4.916044in}{1.430344in}}%
\pgfpathlineto{\pgfqpoint{4.923769in}{1.443945in}}%
\pgfpathlineto{\pgfqpoint{4.931491in}{1.457586in}}%
\pgfpathlineto{\pgfqpoint{4.939209in}{1.471265in}}%
\pgfpathlineto{\pgfqpoint{4.925106in}{1.468315in}}%
\pgfpathlineto{\pgfqpoint{4.911015in}{1.465474in}}%
\pgfpathlineto{\pgfqpoint{4.896937in}{1.462742in}}%
\pgfpathlineto{\pgfqpoint{4.882871in}{1.460120in}}%
\pgfpathlineto{\pgfqpoint{4.875154in}{1.446801in}}%
\pgfpathlineto{\pgfqpoint{4.867434in}{1.433525in}}%
\pgfpathlineto{\pgfqpoint{4.859710in}{1.420293in}}%
\pgfpathlineto{\pgfqpoint{4.851982in}{1.407109in}}%
\pgfpathclose%
\pgfusepath{fill}%
\end{pgfscope}%
\begin{pgfscope}%
\pgfpathrectangle{\pgfqpoint{1.254980in}{0.150000in}}{\pgfqpoint{5.490039in}{5.490039in}}%
\pgfusepath{clip}%
\pgfsetbuttcap%
\pgfsetroundjoin%
\definecolor{currentfill}{rgb}{0.280868,0.160771,0.472899}%
\pgfsetfillcolor{currentfill}%
\pgfsetfillopacity{0.700000}%
\pgfsetlinewidth{0.000000pt}%
\definecolor{currentstroke}{rgb}{0.000000,0.000000,0.000000}%
\pgfsetstrokecolor{currentstroke}%
\pgfsetdash{}{0pt}%
\pgfpathmoveto{\pgfqpoint{3.454227in}{1.464104in}}%
\pgfpathlineto{\pgfqpoint{3.467967in}{1.452585in}}%
\pgfpathlineto{\pgfqpoint{3.481708in}{1.441193in}}%
\pgfpathlineto{\pgfqpoint{3.495450in}{1.429926in}}%
\pgfpathlineto{\pgfqpoint{3.509192in}{1.418785in}}%
\pgfpathlineto{\pgfqpoint{3.517492in}{1.416562in}}%
\pgfpathlineto{\pgfqpoint{3.525779in}{1.414630in}}%
\pgfpathlineto{\pgfqpoint{3.534052in}{1.412986in}}%
\pgfpathlineto{\pgfqpoint{3.542313in}{1.411623in}}%
\pgfpathlineto{\pgfqpoint{3.528605in}{1.422212in}}%
\pgfpathlineto{\pgfqpoint{3.514899in}{1.432926in}}%
\pgfpathlineto{\pgfqpoint{3.501194in}{1.443765in}}%
\pgfpathlineto{\pgfqpoint{3.487490in}{1.454731in}}%
\pgfpathlineto{\pgfqpoint{3.479194in}{1.456639in}}%
\pgfpathlineto{\pgfqpoint{3.470886in}{1.458834in}}%
\pgfpathlineto{\pgfqpoint{3.462563in}{1.461320in}}%
\pgfpathlineto{\pgfqpoint{3.454227in}{1.464104in}}%
\pgfpathclose%
\pgfusepath{fill}%
\end{pgfscope}%
\begin{pgfscope}%
\pgfpathrectangle{\pgfqpoint{1.254980in}{0.150000in}}{\pgfqpoint{5.490039in}{5.490039in}}%
\pgfusepath{clip}%
\pgfsetbuttcap%
\pgfsetroundjoin%
\definecolor{currentfill}{rgb}{0.269944,0.014625,0.341379}%
\pgfsetfillcolor{currentfill}%
\pgfsetfillopacity{0.700000}%
\pgfsetlinewidth{0.000000pt}%
\definecolor{currentstroke}{rgb}{0.000000,0.000000,0.000000}%
\pgfsetstrokecolor{currentstroke}%
\pgfsetdash{}{0pt}%
\pgfpathmoveto{\pgfqpoint{4.330572in}{1.152140in}}%
\pgfpathlineto{\pgfqpoint{4.344427in}{1.149256in}}%
\pgfpathlineto{\pgfqpoint{4.358289in}{1.146482in}}%
\pgfpathlineto{\pgfqpoint{4.372160in}{1.143818in}}%
\pgfpathlineto{\pgfqpoint{4.386038in}{1.141265in}}%
\pgfpathlineto{\pgfqpoint{4.393883in}{1.150023in}}%
\pgfpathlineto{\pgfqpoint{4.401723in}{1.158930in}}%
\pgfpathlineto{\pgfqpoint{4.409558in}{1.167980in}}%
\pgfpathlineto{\pgfqpoint{4.417389in}{1.177171in}}%
\pgfpathlineto{\pgfqpoint{4.403520in}{1.179267in}}%
\pgfpathlineto{\pgfqpoint{4.389659in}{1.181473in}}%
\pgfpathlineto{\pgfqpoint{4.375806in}{1.183789in}}%
\pgfpathlineto{\pgfqpoint{4.361962in}{1.186216in}}%
\pgfpathlineto{\pgfqpoint{4.354122in}{1.177477in}}%
\pgfpathlineto{\pgfqpoint{4.346277in}{1.168882in}}%
\pgfpathlineto{\pgfqpoint{4.338427in}{1.160435in}}%
\pgfpathlineto{\pgfqpoint{4.330572in}{1.152140in}}%
\pgfpathclose%
\pgfusepath{fill}%
\end{pgfscope}%
\begin{pgfscope}%
\pgfpathrectangle{\pgfqpoint{1.254980in}{0.150000in}}{\pgfqpoint{5.490039in}{5.490039in}}%
\pgfusepath{clip}%
\pgfsetbuttcap%
\pgfsetroundjoin%
\definecolor{currentfill}{rgb}{0.120081,0.622161,0.534946}%
\pgfsetfillcolor{currentfill}%
\pgfsetfillopacity{0.700000}%
\pgfsetlinewidth{0.000000pt}%
\definecolor{currentstroke}{rgb}{0.000000,0.000000,0.000000}%
\pgfsetstrokecolor{currentstroke}%
\pgfsetdash{}{0pt}%
\pgfpathmoveto{\pgfqpoint{2.459461in}{2.662853in}}%
\pgfpathlineto{\pgfqpoint{2.473466in}{2.640590in}}%
\pgfpathlineto{\pgfqpoint{2.487463in}{2.618506in}}%
\pgfpathlineto{\pgfqpoint{2.501451in}{2.596600in}}%
\pgfpathlineto{\pgfqpoint{2.515430in}{2.574871in}}%
\pgfpathlineto{\pgfqpoint{2.524624in}{2.562129in}}%
\pgfpathlineto{\pgfqpoint{2.533792in}{2.549777in}}%
\pgfpathlineto{\pgfqpoint{2.542935in}{2.537810in}}%
\pgfpathlineto{\pgfqpoint{2.552053in}{2.526222in}}%
\pgfpathlineto{\pgfqpoint{2.538138in}{2.547339in}}%
\pgfpathlineto{\pgfqpoint{2.524214in}{2.568632in}}%
\pgfpathlineto{\pgfqpoint{2.510282in}{2.590102in}}%
\pgfpathlineto{\pgfqpoint{2.496342in}{2.611750in}}%
\pgfpathlineto{\pgfqpoint{2.487160in}{2.623943in}}%
\pgfpathlineto{\pgfqpoint{2.477953in}{2.636521in}}%
\pgfpathlineto{\pgfqpoint{2.468720in}{2.649489in}}%
\pgfpathlineto{\pgfqpoint{2.459461in}{2.662853in}}%
\pgfpathclose%
\pgfusepath{fill}%
\end{pgfscope}%
\begin{pgfscope}%
\pgfpathrectangle{\pgfqpoint{1.254980in}{0.150000in}}{\pgfqpoint{5.490039in}{5.490039in}}%
\pgfusepath{clip}%
\pgfsetbuttcap%
\pgfsetroundjoin%
\definecolor{currentfill}{rgb}{0.279566,0.067836,0.391917}%
\pgfsetfillcolor{currentfill}%
\pgfsetfillopacity{0.700000}%
\pgfsetlinewidth{0.000000pt}%
\definecolor{currentstroke}{rgb}{0.000000,0.000000,0.000000}%
\pgfsetstrokecolor{currentstroke}%
\pgfsetdash{}{0pt}%
\pgfpathmoveto{\pgfqpoint{3.761931in}{1.258873in}}%
\pgfpathlineto{\pgfqpoint{3.775681in}{1.250349in}}%
\pgfpathlineto{\pgfqpoint{3.789434in}{1.241943in}}%
\pgfpathlineto{\pgfqpoint{3.803190in}{1.233656in}}%
\pgfpathlineto{\pgfqpoint{3.816951in}{1.225485in}}%
\pgfpathlineto{\pgfqpoint{3.825050in}{1.227082in}}%
\pgfpathlineto{\pgfqpoint{3.833140in}{1.228928in}}%
\pgfpathlineto{\pgfqpoint{3.841220in}{1.231018in}}%
\pgfpathlineto{\pgfqpoint{3.849291in}{1.233349in}}%
\pgfpathlineto{\pgfqpoint{3.835556in}{1.240992in}}%
\pgfpathlineto{\pgfqpoint{3.821826in}{1.248753in}}%
\pgfpathlineto{\pgfqpoint{3.808099in}{1.256631in}}%
\pgfpathlineto{\pgfqpoint{3.794376in}{1.264627in}}%
\pgfpathlineto{\pgfqpoint{3.786280in}{1.262818in}}%
\pgfpathlineto{\pgfqpoint{3.778174in}{1.261252in}}%
\pgfpathlineto{\pgfqpoint{3.770057in}{1.259936in}}%
\pgfpathlineto{\pgfqpoint{3.761931in}{1.258873in}}%
\pgfpathclose%
\pgfusepath{fill}%
\end{pgfscope}%
\begin{pgfscope}%
\pgfpathrectangle{\pgfqpoint{1.254980in}{0.150000in}}{\pgfqpoint{5.490039in}{5.490039in}}%
\pgfusepath{clip}%
\pgfsetbuttcap%
\pgfsetroundjoin%
\definecolor{currentfill}{rgb}{0.268510,0.009605,0.335427}%
\pgfsetfillcolor{currentfill}%
\pgfsetfillopacity{0.700000}%
\pgfsetlinewidth{0.000000pt}%
\definecolor{currentstroke}{rgb}{0.000000,0.000000,0.000000}%
\pgfsetstrokecolor{currentstroke}%
\pgfsetdash{}{0pt}%
\pgfpathmoveto{\pgfqpoint{4.101520in}{1.146540in}}%
\pgfpathlineto{\pgfqpoint{4.115319in}{1.141355in}}%
\pgfpathlineto{\pgfqpoint{4.129124in}{1.136284in}}%
\pgfpathlineto{\pgfqpoint{4.142936in}{1.131325in}}%
\pgfpathlineto{\pgfqpoint{4.156753in}{1.126477in}}%
\pgfpathlineto{\pgfqpoint{4.164682in}{1.132425in}}%
\pgfpathlineto{\pgfqpoint{4.172605in}{1.138564in}}%
\pgfpathlineto{\pgfqpoint{4.180521in}{1.144891in}}%
\pgfpathlineto{\pgfqpoint{4.188431in}{1.151401in}}%
\pgfpathlineto{\pgfqpoint{4.174629in}{1.155759in}}%
\pgfpathlineto{\pgfqpoint{4.160834in}{1.160228in}}%
\pgfpathlineto{\pgfqpoint{4.147045in}{1.164810in}}%
\pgfpathlineto{\pgfqpoint{4.133263in}{1.169504in}}%
\pgfpathlineto{\pgfqpoint{4.125337in}{1.163477in}}%
\pgfpathlineto{\pgfqpoint{4.117405in}{1.157637in}}%
\pgfpathlineto{\pgfqpoint{4.109466in}{1.151990in}}%
\pgfpathlineto{\pgfqpoint{4.101520in}{1.146540in}}%
\pgfpathclose%
\pgfusepath{fill}%
\end{pgfscope}%
\begin{pgfscope}%
\pgfpathrectangle{\pgfqpoint{1.254980in}{0.150000in}}{\pgfqpoint{5.490039in}{5.490039in}}%
\pgfusepath{clip}%
\pgfsetbuttcap%
\pgfsetroundjoin%
\definecolor{currentfill}{rgb}{0.275191,0.194905,0.496005}%
\pgfsetfillcolor{currentfill}%
\pgfsetfillopacity{0.700000}%
\pgfsetlinewidth{0.000000pt}%
\definecolor{currentstroke}{rgb}{0.000000,0.000000,0.000000}%
\pgfsetstrokecolor{currentstroke}%
\pgfsetdash{}{0pt}%
\pgfpathmoveto{\pgfqpoint{4.939209in}{1.471265in}}%
\pgfpathlineto{\pgfqpoint{4.953325in}{1.474325in}}%
\pgfpathlineto{\pgfqpoint{4.967454in}{1.477493in}}%
\pgfpathlineto{\pgfqpoint{4.981595in}{1.480771in}}%
\pgfpathlineto{\pgfqpoint{4.995749in}{1.484158in}}%
\pgfpathlineto{\pgfqpoint{5.003466in}{1.498230in}}%
\pgfpathlineto{\pgfqpoint{5.011180in}{1.512330in}}%
\pgfpathlineto{\pgfqpoint{5.018890in}{1.526456in}}%
\pgfpathlineto{\pgfqpoint{5.026596in}{1.540605in}}%
\pgfpathlineto{\pgfqpoint{5.012439in}{1.536865in}}%
\pgfpathlineto{\pgfqpoint{4.998295in}{1.533235in}}%
\pgfpathlineto{\pgfqpoint{4.984164in}{1.529715in}}%
\pgfpathlineto{\pgfqpoint{4.970046in}{1.526304in}}%
\pgfpathlineto{\pgfqpoint{4.962343in}{1.512501in}}%
\pgfpathlineto{\pgfqpoint{4.954635in}{1.498725in}}%
\pgfpathlineto{\pgfqpoint{4.946924in}{1.484979in}}%
\pgfpathlineto{\pgfqpoint{4.939209in}{1.471265in}}%
\pgfpathclose%
\pgfusepath{fill}%
\end{pgfscope}%
\begin{pgfscope}%
\pgfpathrectangle{\pgfqpoint{1.254980in}{0.150000in}}{\pgfqpoint{5.490039in}{5.490039in}}%
\pgfusepath{clip}%
\pgfsetbuttcap%
\pgfsetroundjoin%
\definecolor{currentfill}{rgb}{0.282290,0.145912,0.461510}%
\pgfsetfillcolor{currentfill}%
\pgfsetfillopacity{0.700000}%
\pgfsetlinewidth{0.000000pt}%
\definecolor{currentstroke}{rgb}{0.000000,0.000000,0.000000}%
\pgfsetstrokecolor{currentstroke}%
\pgfsetdash{}{0pt}%
\pgfpathmoveto{\pgfqpoint{3.509192in}{1.418785in}}%
\pgfpathlineto{\pgfqpoint{3.522937in}{1.407769in}}%
\pgfpathlineto{\pgfqpoint{3.536682in}{1.396878in}}%
\pgfpathlineto{\pgfqpoint{3.550429in}{1.386111in}}%
\pgfpathlineto{\pgfqpoint{3.564177in}{1.375467in}}%
\pgfpathlineto{\pgfqpoint{3.572442in}{1.373802in}}%
\pgfpathlineto{\pgfqpoint{3.580694in}{1.372424in}}%
\pgfpathlineto{\pgfqpoint{3.588934in}{1.371328in}}%
\pgfpathlineto{\pgfqpoint{3.597162in}{1.370509in}}%
\pgfpathlineto{\pgfqpoint{3.583447in}{1.380602in}}%
\pgfpathlineto{\pgfqpoint{3.569734in}{1.390819in}}%
\pgfpathlineto{\pgfqpoint{3.556023in}{1.401159in}}%
\pgfpathlineto{\pgfqpoint{3.542313in}{1.411623in}}%
\pgfpathlineto{\pgfqpoint{3.534052in}{1.412986in}}%
\pgfpathlineto{\pgfqpoint{3.525779in}{1.414630in}}%
\pgfpathlineto{\pgfqpoint{3.517492in}{1.416562in}}%
\pgfpathlineto{\pgfqpoint{3.509192in}{1.418785in}}%
\pgfpathclose%
\pgfusepath{fill}%
\end{pgfscope}%
\begin{pgfscope}%
\pgfpathrectangle{\pgfqpoint{1.254980in}{0.150000in}}{\pgfqpoint{5.490039in}{5.490039in}}%
\pgfusepath{clip}%
\pgfsetbuttcap%
\pgfsetroundjoin%
\definecolor{currentfill}{rgb}{0.272594,0.025563,0.353093}%
\pgfsetfillcolor{currentfill}%
\pgfsetfillopacity{0.700000}%
\pgfsetlinewidth{0.000000pt}%
\definecolor{currentstroke}{rgb}{0.000000,0.000000,0.000000}%
\pgfsetstrokecolor{currentstroke}%
\pgfsetdash{}{0pt}%
\pgfpathmoveto{\pgfqpoint{3.959322in}{1.176392in}}%
\pgfpathlineto{\pgfqpoint{3.973096in}{1.169792in}}%
\pgfpathlineto{\pgfqpoint{3.986876in}{1.163307in}}%
\pgfpathlineto{\pgfqpoint{4.000661in}{1.156935in}}%
\pgfpathlineto{\pgfqpoint{4.014451in}{1.150678in}}%
\pgfpathlineto{\pgfqpoint{4.022445in}{1.154788in}}%
\pgfpathlineto{\pgfqpoint{4.030432in}{1.159115in}}%
\pgfpathlineto{\pgfqpoint{4.038411in}{1.163656in}}%
\pgfpathlineto{\pgfqpoint{4.046383in}{1.168405in}}%
\pgfpathlineto{\pgfqpoint{4.032612in}{1.174155in}}%
\pgfpathlineto{\pgfqpoint{4.018848in}{1.180019in}}%
\pgfpathlineto{\pgfqpoint{4.005089in}{1.185997in}}%
\pgfpathlineto{\pgfqpoint{3.991335in}{1.192089in}}%
\pgfpathlineto{\pgfqpoint{3.983344in}{1.187841in}}%
\pgfpathlineto{\pgfqpoint{3.975344in}{1.183806in}}%
\pgfpathlineto{\pgfqpoint{3.967337in}{1.179988in}}%
\pgfpathlineto{\pgfqpoint{3.959322in}{1.176392in}}%
\pgfpathclose%
\pgfusepath{fill}%
\end{pgfscope}%
\begin{pgfscope}%
\pgfpathrectangle{\pgfqpoint{1.254980in}{0.150000in}}{\pgfqpoint{5.490039in}{5.490039in}}%
\pgfusepath{clip}%
\pgfsetbuttcap%
\pgfsetroundjoin%
\definecolor{currentfill}{rgb}{0.268510,0.009605,0.335427}%
\pgfsetfillcolor{currentfill}%
\pgfsetfillopacity{0.700000}%
\pgfsetlinewidth{0.000000pt}%
\definecolor{currentstroke}{rgb}{0.000000,0.000000,0.000000}%
\pgfsetstrokecolor{currentstroke}%
\pgfsetdash{}{0pt}%
\pgfpathmoveto{\pgfqpoint{4.243708in}{1.135089in}}%
\pgfpathlineto{\pgfqpoint{4.257544in}{1.131289in}}%
\pgfpathlineto{\pgfqpoint{4.271388in}{1.127600in}}%
\pgfpathlineto{\pgfqpoint{4.285239in}{1.124021in}}%
\pgfpathlineto{\pgfqpoint{4.299098in}{1.120553in}}%
\pgfpathlineto{\pgfqpoint{4.306974in}{1.128203in}}%
\pgfpathlineto{\pgfqpoint{4.314846in}{1.136020in}}%
\pgfpathlineto{\pgfqpoint{4.322712in}{1.144000in}}%
\pgfpathlineto{\pgfqpoint{4.330572in}{1.152140in}}%
\pgfpathlineto{\pgfqpoint{4.316726in}{1.155134in}}%
\pgfpathlineto{\pgfqpoint{4.302887in}{1.158239in}}%
\pgfpathlineto{\pgfqpoint{4.289055in}{1.161455in}}%
\pgfpathlineto{\pgfqpoint{4.275231in}{1.164781in}}%
\pgfpathlineto{\pgfqpoint{4.267359in}{1.157109in}}%
\pgfpathlineto{\pgfqpoint{4.259481in}{1.149600in}}%
\pgfpathlineto{\pgfqpoint{4.251597in}{1.142259in}}%
\pgfpathlineto{\pgfqpoint{4.243708in}{1.135089in}}%
\pgfpathclose%
\pgfusepath{fill}%
\end{pgfscope}%
\begin{pgfscope}%
\pgfpathrectangle{\pgfqpoint{1.254980in}{0.150000in}}{\pgfqpoint{5.490039in}{5.490039in}}%
\pgfusepath{clip}%
\pgfsetbuttcap%
\pgfsetroundjoin%
\definecolor{currentfill}{rgb}{0.386433,0.794644,0.372886}%
\pgfsetfillcolor{currentfill}%
\pgfsetfillopacity{0.700000}%
\pgfsetlinewidth{0.000000pt}%
\definecolor{currentstroke}{rgb}{0.000000,0.000000,0.000000}%
\pgfsetstrokecolor{currentstroke}%
\pgfsetdash{}{0pt}%
\pgfpathmoveto{\pgfqpoint{2.158786in}{3.187830in}}%
\pgfpathlineto{\pgfqpoint{2.172983in}{3.161541in}}%
\pgfpathlineto{\pgfqpoint{2.187167in}{3.135461in}}%
\pgfpathlineto{\pgfqpoint{2.201338in}{3.109588in}}%
\pgfpathlineto{\pgfqpoint{2.215497in}{3.083921in}}%
\pgfpathlineto{\pgfqpoint{2.224993in}{3.069034in}}%
\pgfpathlineto{\pgfqpoint{2.234460in}{3.054552in}}%
\pgfpathlineto{\pgfqpoint{2.243899in}{3.040469in}}%
\pgfpathlineto{\pgfqpoint{2.253309in}{3.026780in}}%
\pgfpathlineto{\pgfqpoint{2.239222in}{3.051826in}}%
\pgfpathlineto{\pgfqpoint{2.225123in}{3.077076in}}%
\pgfpathlineto{\pgfqpoint{2.211012in}{3.102531in}}%
\pgfpathlineto{\pgfqpoint{2.196889in}{3.128194in}}%
\pgfpathlineto{\pgfqpoint{2.187407in}{3.142497in}}%
\pgfpathlineto{\pgfqpoint{2.177897in}{3.157200in}}%
\pgfpathlineto{\pgfqpoint{2.168357in}{3.172309in}}%
\pgfpathlineto{\pgfqpoint{2.158786in}{3.187830in}}%
\pgfpathclose%
\pgfusepath{fill}%
\end{pgfscope}%
\begin{pgfscope}%
\pgfpathrectangle{\pgfqpoint{1.254980in}{0.150000in}}{\pgfqpoint{5.490039in}{5.490039in}}%
\pgfusepath{clip}%
\pgfsetbuttcap%
\pgfsetroundjoin%
\definecolor{currentfill}{rgb}{0.265145,0.232956,0.516599}%
\pgfsetfillcolor{currentfill}%
\pgfsetfillopacity{0.700000}%
\pgfsetlinewidth{0.000000pt}%
\definecolor{currentstroke}{rgb}{0.000000,0.000000,0.000000}%
\pgfsetstrokecolor{currentstroke}%
\pgfsetdash{}{0pt}%
\pgfpathmoveto{\pgfqpoint{5.026596in}{1.540605in}}%
\pgfpathlineto{\pgfqpoint{5.040766in}{1.544454in}}%
\pgfpathlineto{\pgfqpoint{5.054950in}{1.548412in}}%
\pgfpathlineto{\pgfqpoint{5.069147in}{1.552480in}}%
\pgfpathlineto{\pgfqpoint{5.076852in}{1.566906in}}%
\pgfpathlineto{\pgfqpoint{5.084554in}{1.581346in}}%
\pgfpathlineto{\pgfqpoint{5.092253in}{1.595799in}}%
\pgfpathlineto{\pgfqpoint{5.099947in}{1.610260in}}%
\pgfpathlineto{\pgfqpoint{5.085746in}{1.605855in}}%
\pgfpathlineto{\pgfqpoint{5.071559in}{1.601560in}}%
\pgfpathlineto{\pgfqpoint{5.057386in}{1.597374in}}%
\pgfpathlineto{\pgfqpoint{5.049694in}{1.583161in}}%
\pgfpathlineto{\pgfqpoint{5.041998in}{1.568960in}}%
\pgfpathlineto{\pgfqpoint{5.034299in}{1.554773in}}%
\pgfpathlineto{\pgfqpoint{5.026596in}{1.540605in}}%
\pgfpathclose%
\pgfusepath{fill}%
\end{pgfscope}%
\begin{pgfscope}%
\pgfpathrectangle{\pgfqpoint{1.254980in}{0.150000in}}{\pgfqpoint{5.490039in}{5.490039in}}%
\pgfusepath{clip}%
\pgfsetbuttcap%
\pgfsetroundjoin%
\definecolor{currentfill}{rgb}{0.132268,0.655014,0.519661}%
\pgfsetfillcolor{currentfill}%
\pgfsetfillopacity{0.700000}%
\pgfsetlinewidth{0.000000pt}%
\definecolor{currentstroke}{rgb}{0.000000,0.000000,0.000000}%
\pgfsetstrokecolor{currentstroke}%
\pgfsetdash{}{0pt}%
\pgfpathmoveto{\pgfqpoint{2.403346in}{2.753724in}}%
\pgfpathlineto{\pgfqpoint{2.417389in}{2.730731in}}%
\pgfpathlineto{\pgfqpoint{2.431422in}{2.707923in}}%
\pgfpathlineto{\pgfqpoint{2.445446in}{2.685297in}}%
\pgfpathlineto{\pgfqpoint{2.459461in}{2.662853in}}%
\pgfpathlineto{\pgfqpoint{2.468720in}{2.649489in}}%
\pgfpathlineto{\pgfqpoint{2.477953in}{2.636521in}}%
\pgfpathlineto{\pgfqpoint{2.487160in}{2.623943in}}%
\pgfpathlineto{\pgfqpoint{2.496342in}{2.611750in}}%
\pgfpathlineto{\pgfqpoint{2.482393in}{2.633578in}}%
\pgfpathlineto{\pgfqpoint{2.468435in}{2.655585in}}%
\pgfpathlineto{\pgfqpoint{2.454469in}{2.677774in}}%
\pgfpathlineto{\pgfqpoint{2.440493in}{2.700146in}}%
\pgfpathlineto{\pgfqpoint{2.431247in}{2.712948in}}%
\pgfpathlineto{\pgfqpoint{2.421974in}{2.726141in}}%
\pgfpathlineto{\pgfqpoint{2.412674in}{2.739731in}}%
\pgfpathlineto{\pgfqpoint{2.403346in}{2.753724in}}%
\pgfpathclose%
\pgfusepath{fill}%
\end{pgfscope}%
\begin{pgfscope}%
\pgfpathrectangle{\pgfqpoint{1.254980in}{0.150000in}}{\pgfqpoint{5.490039in}{5.490039in}}%
\pgfusepath{clip}%
\pgfsetbuttcap%
\pgfsetroundjoin%
\definecolor{currentfill}{rgb}{0.281446,0.084320,0.407414}%
\pgfsetfillcolor{currentfill}%
\pgfsetfillopacity{0.700000}%
\pgfsetlinewidth{0.000000pt}%
\definecolor{currentstroke}{rgb}{0.000000,0.000000,0.000000}%
\pgfsetstrokecolor{currentstroke}%
\pgfsetdash{}{0pt}%
\pgfpathmoveto{\pgfqpoint{4.646844in}{1.249111in}}%
\pgfpathlineto{\pgfqpoint{4.660823in}{1.249305in}}%
\pgfpathlineto{\pgfqpoint{4.674813in}{1.249607in}}%
\pgfpathlineto{\pgfqpoint{4.688814in}{1.250019in}}%
\pgfpathlineto{\pgfqpoint{4.702825in}{1.250539in}}%
\pgfpathlineto{\pgfqpoint{4.710596in}{1.262496in}}%
\pgfpathlineto{\pgfqpoint{4.718363in}{1.274542in}}%
\pgfpathlineto{\pgfqpoint{4.726127in}{1.286673in}}%
\pgfpathlineto{\pgfqpoint{4.733888in}{1.298887in}}%
\pgfpathlineto{\pgfqpoint{4.719879in}{1.297954in}}%
\pgfpathlineto{\pgfqpoint{4.705880in}{1.297130in}}%
\pgfpathlineto{\pgfqpoint{4.691893in}{1.296415in}}%
\pgfpathlineto{\pgfqpoint{4.677916in}{1.295809in}}%
\pgfpathlineto{\pgfqpoint{4.670154in}{1.284002in}}%
\pgfpathlineto{\pgfqpoint{4.662388in}{1.272281in}}%
\pgfpathlineto{\pgfqpoint{4.654618in}{1.260649in}}%
\pgfpathlineto{\pgfqpoint{4.646844in}{1.249111in}}%
\pgfpathclose%
\pgfusepath{fill}%
\end{pgfscope}%
\begin{pgfscope}%
\pgfpathrectangle{\pgfqpoint{1.254980in}{0.150000in}}{\pgfqpoint{5.490039in}{5.490039in}}%
\pgfusepath{clip}%
\pgfsetbuttcap%
\pgfsetroundjoin%
\definecolor{currentfill}{rgb}{0.278791,0.062145,0.386592}%
\pgfsetfillcolor{currentfill}%
\pgfsetfillopacity{0.700000}%
\pgfsetlinewidth{0.000000pt}%
\definecolor{currentstroke}{rgb}{0.000000,0.000000,0.000000}%
\pgfsetstrokecolor{currentstroke}%
\pgfsetdash{}{0pt}%
\pgfpathmoveto{\pgfqpoint{4.559875in}{1.205974in}}%
\pgfpathlineto{\pgfqpoint{4.573819in}{1.205305in}}%
\pgfpathlineto{\pgfqpoint{4.587773in}{1.204745in}}%
\pgfpathlineto{\pgfqpoint{4.601736in}{1.204294in}}%
\pgfpathlineto{\pgfqpoint{4.615709in}{1.203951in}}%
\pgfpathlineto{\pgfqpoint{4.623499in}{1.215085in}}%
\pgfpathlineto{\pgfqpoint{4.631284in}{1.226325in}}%
\pgfpathlineto{\pgfqpoint{4.639066in}{1.237668in}}%
\pgfpathlineto{\pgfqpoint{4.646844in}{1.249111in}}%
\pgfpathlineto{\pgfqpoint{4.632874in}{1.249025in}}%
\pgfpathlineto{\pgfqpoint{4.618915in}{1.249049in}}%
\pgfpathlineto{\pgfqpoint{4.604966in}{1.249182in}}%
\pgfpathlineto{\pgfqpoint{4.591027in}{1.249424in}}%
\pgfpathlineto{\pgfqpoint{4.583245in}{1.238403in}}%
\pgfpathlineto{\pgfqpoint{4.575459in}{1.227485in}}%
\pgfpathlineto{\pgfqpoint{4.567670in}{1.216674in}}%
\pgfpathlineto{\pgfqpoint{4.559875in}{1.205974in}}%
\pgfpathclose%
\pgfusepath{fill}%
\end{pgfscope}%
\begin{pgfscope}%
\pgfpathrectangle{\pgfqpoint{1.254980in}{0.150000in}}{\pgfqpoint{5.490039in}{5.490039in}}%
\pgfusepath{clip}%
\pgfsetbuttcap%
\pgfsetroundjoin%
\definecolor{currentfill}{rgb}{0.751884,0.874951,0.143228}%
\pgfsetfillcolor{currentfill}%
\pgfsetfillopacity{0.700000}%
\pgfsetlinewidth{0.000000pt}%
\definecolor{currentstroke}{rgb}{0.000000,0.000000,0.000000}%
\pgfsetstrokecolor{currentstroke}%
\pgfsetdash{}{0pt}%
\pgfpathmoveto{\pgfqpoint{1.969044in}{3.568260in}}%
\pgfpathlineto{\pgfqpoint{1.983397in}{3.539066in}}%
\pgfpathlineto{\pgfqpoint{1.997733in}{3.510105in}}%
\pgfpathlineto{\pgfqpoint{2.012054in}{3.481376in}}%
\pgfpathlineto{\pgfqpoint{2.026359in}{3.452876in}}%
\pgfpathlineto{\pgfqpoint{2.036034in}{3.437105in}}%
\pgfpathlineto{\pgfqpoint{2.045679in}{3.421744in}}%
\pgfpathlineto{\pgfqpoint{2.055293in}{3.406785in}}%
\pgfpathlineto{\pgfqpoint{2.064876in}{3.392225in}}%
\pgfpathlineto{\pgfqpoint{2.050648in}{3.420102in}}%
\pgfpathlineto{\pgfqpoint{2.036404in}{3.448207in}}%
\pgfpathlineto{\pgfqpoint{2.022146in}{3.476541in}}%
\pgfpathlineto{\pgfqpoint{2.007871in}{3.505107in}}%
\pgfpathlineto{\pgfqpoint{1.998212in}{3.520283in}}%
\pgfpathlineto{\pgfqpoint{1.988521in}{3.535863in}}%
\pgfpathlineto{\pgfqpoint{1.978799in}{3.551854in}}%
\pgfpathlineto{\pgfqpoint{1.969044in}{3.568260in}}%
\pgfpathclose%
\pgfusepath{fill}%
\end{pgfscope}%
\begin{pgfscope}%
\pgfpathrectangle{\pgfqpoint{1.254980in}{0.150000in}}{\pgfqpoint{5.490039in}{5.490039in}}%
\pgfusepath{clip}%
\pgfsetbuttcap%
\pgfsetroundjoin%
\definecolor{currentfill}{rgb}{0.283197,0.115680,0.436115}%
\pgfsetfillcolor{currentfill}%
\pgfsetfillopacity{0.700000}%
\pgfsetlinewidth{0.000000pt}%
\definecolor{currentstroke}{rgb}{0.000000,0.000000,0.000000}%
\pgfsetstrokecolor{currentstroke}%
\pgfsetdash{}{0pt}%
\pgfpathmoveto{\pgfqpoint{4.733888in}{1.298887in}}%
\pgfpathlineto{\pgfqpoint{4.747908in}{1.299928in}}%
\pgfpathlineto{\pgfqpoint{4.761939in}{1.301078in}}%
\pgfpathlineto{\pgfqpoint{4.775981in}{1.302337in}}%
\pgfpathlineto{\pgfqpoint{4.790034in}{1.303704in}}%
\pgfpathlineto{\pgfqpoint{4.797790in}{1.316400in}}%
\pgfpathlineto{\pgfqpoint{4.805542in}{1.329168in}}%
\pgfpathlineto{\pgfqpoint{4.813291in}{1.342004in}}%
\pgfpathlineto{\pgfqpoint{4.821036in}{1.354907in}}%
\pgfpathlineto{\pgfqpoint{4.806983in}{1.353141in}}%
\pgfpathlineto{\pgfqpoint{4.792941in}{1.351485in}}%
\pgfpathlineto{\pgfqpoint{4.778911in}{1.349938in}}%
\pgfpathlineto{\pgfqpoint{4.764892in}{1.348499in}}%
\pgfpathlineto{\pgfqpoint{4.757147in}{1.335989in}}%
\pgfpathlineto{\pgfqpoint{4.749397in}{1.323548in}}%
\pgfpathlineto{\pgfqpoint{4.741644in}{1.311179in}}%
\pgfpathlineto{\pgfqpoint{4.733888in}{1.298887in}}%
\pgfpathclose%
\pgfusepath{fill}%
\end{pgfscope}%
\begin{pgfscope}%
\pgfpathrectangle{\pgfqpoint{1.254980in}{0.150000in}}{\pgfqpoint{5.490039in}{5.490039in}}%
\pgfusepath{clip}%
\pgfsetbuttcap%
\pgfsetroundjoin%
\definecolor{currentfill}{rgb}{0.283187,0.125848,0.444960}%
\pgfsetfillcolor{currentfill}%
\pgfsetfillopacity{0.700000}%
\pgfsetlinewidth{0.000000pt}%
\definecolor{currentstroke}{rgb}{0.000000,0.000000,0.000000}%
\pgfsetstrokecolor{currentstroke}%
\pgfsetdash{}{0pt}%
\pgfpathmoveto{\pgfqpoint{3.564177in}{1.375467in}}%
\pgfpathlineto{\pgfqpoint{3.577927in}{1.364947in}}%
\pgfpathlineto{\pgfqpoint{3.591679in}{1.354550in}}%
\pgfpathlineto{\pgfqpoint{3.605433in}{1.344275in}}%
\pgfpathlineto{\pgfqpoint{3.619189in}{1.334122in}}%
\pgfpathlineto{\pgfqpoint{3.627420in}{1.333014in}}%
\pgfpathlineto{\pgfqpoint{3.635640in}{1.332188in}}%
\pgfpathlineto{\pgfqpoint{3.643847in}{1.331639in}}%
\pgfpathlineto{\pgfqpoint{3.652043in}{1.331362in}}%
\pgfpathlineto{\pgfqpoint{3.638319in}{1.340966in}}%
\pgfpathlineto{\pgfqpoint{3.624598in}{1.350691in}}%
\pgfpathlineto{\pgfqpoint{3.610879in}{1.360539in}}%
\pgfpathlineto{\pgfqpoint{3.597162in}{1.370509in}}%
\pgfpathlineto{\pgfqpoint{3.588934in}{1.371328in}}%
\pgfpathlineto{\pgfqpoint{3.580694in}{1.372424in}}%
\pgfpathlineto{\pgfqpoint{3.572442in}{1.373802in}}%
\pgfpathlineto{\pgfqpoint{3.564177in}{1.375467in}}%
\pgfpathclose%
\pgfusepath{fill}%
\end{pgfscope}%
\begin{pgfscope}%
\pgfpathrectangle{\pgfqpoint{1.254980in}{0.150000in}}{\pgfqpoint{5.490039in}{5.490039in}}%
\pgfusepath{clip}%
\pgfsetbuttcap%
\pgfsetroundjoin%
\definecolor{currentfill}{rgb}{0.274952,0.037752,0.364543}%
\pgfsetfillcolor{currentfill}%
\pgfsetfillopacity{0.700000}%
\pgfsetlinewidth{0.000000pt}%
\definecolor{currentstroke}{rgb}{0.000000,0.000000,0.000000}%
\pgfsetstrokecolor{currentstroke}%
\pgfsetdash{}{0pt}%
\pgfpathmoveto{\pgfqpoint{4.472952in}{1.169884in}}%
\pgfpathlineto{\pgfqpoint{4.486865in}{1.168336in}}%
\pgfpathlineto{\pgfqpoint{4.500787in}{1.166898in}}%
\pgfpathlineto{\pgfqpoint{4.514718in}{1.165568in}}%
\pgfpathlineto{\pgfqpoint{4.528658in}{1.164347in}}%
\pgfpathlineto{\pgfqpoint{4.536468in}{1.174571in}}%
\pgfpathlineto{\pgfqpoint{4.544275in}{1.184918in}}%
\pgfpathlineto{\pgfqpoint{4.552077in}{1.195387in}}%
\pgfpathlineto{\pgfqpoint{4.559875in}{1.205974in}}%
\pgfpathlineto{\pgfqpoint{4.545941in}{1.206752in}}%
\pgfpathlineto{\pgfqpoint{4.532017in}{1.207639in}}%
\pgfpathlineto{\pgfqpoint{4.518101in}{1.208635in}}%
\pgfpathlineto{\pgfqpoint{4.504195in}{1.209741in}}%
\pgfpathlineto{\pgfqpoint{4.496391in}{1.199591in}}%
\pgfpathlineto{\pgfqpoint{4.488582in}{1.189562in}}%
\pgfpathlineto{\pgfqpoint{4.480769in}{1.179659in}}%
\pgfpathlineto{\pgfqpoint{4.472952in}{1.169884in}}%
\pgfpathclose%
\pgfusepath{fill}%
\end{pgfscope}%
\begin{pgfscope}%
\pgfpathrectangle{\pgfqpoint{1.254980in}{0.150000in}}{\pgfqpoint{5.490039in}{5.490039in}}%
\pgfusepath{clip}%
\pgfsetbuttcap%
\pgfsetroundjoin%
\definecolor{currentfill}{rgb}{0.277941,0.056324,0.381191}%
\pgfsetfillcolor{currentfill}%
\pgfsetfillopacity{0.700000}%
\pgfsetlinewidth{0.000000pt}%
\definecolor{currentstroke}{rgb}{0.000000,0.000000,0.000000}%
\pgfsetstrokecolor{currentstroke}%
\pgfsetdash{}{0pt}%
\pgfpathmoveto{\pgfqpoint{3.816951in}{1.225485in}}%
\pgfpathlineto{\pgfqpoint{3.830715in}{1.217432in}}%
\pgfpathlineto{\pgfqpoint{3.844482in}{1.209496in}}%
\pgfpathlineto{\pgfqpoint{3.858254in}{1.201677in}}%
\pgfpathlineto{\pgfqpoint{3.872030in}{1.193974in}}%
\pgfpathlineto{\pgfqpoint{3.880104in}{1.196104in}}%
\pgfpathlineto{\pgfqpoint{3.888168in}{1.198479in}}%
\pgfpathlineto{\pgfqpoint{3.896224in}{1.201093in}}%
\pgfpathlineto{\pgfqpoint{3.904271in}{1.203943in}}%
\pgfpathlineto{\pgfqpoint{3.890519in}{1.211120in}}%
\pgfpathlineto{\pgfqpoint{3.876772in}{1.218413in}}%
\pgfpathlineto{\pgfqpoint{3.863029in}{1.225823in}}%
\pgfpathlineto{\pgfqpoint{3.849291in}{1.233349in}}%
\pgfpathlineto{\pgfqpoint{3.841220in}{1.231018in}}%
\pgfpathlineto{\pgfqpoint{3.833140in}{1.228928in}}%
\pgfpathlineto{\pgfqpoint{3.825050in}{1.227082in}}%
\pgfpathlineto{\pgfqpoint{3.816951in}{1.225485in}}%
\pgfpathclose%
\pgfusepath{fill}%
\end{pgfscope}%
\begin{pgfscope}%
\pgfpathrectangle{\pgfqpoint{1.254980in}{0.150000in}}{\pgfqpoint{5.490039in}{5.490039in}}%
\pgfusepath{clip}%
\pgfsetbuttcap%
\pgfsetroundjoin%
\definecolor{currentfill}{rgb}{0.282290,0.145912,0.461510}%
\pgfsetfillcolor{currentfill}%
\pgfsetfillopacity{0.700000}%
\pgfsetlinewidth{0.000000pt}%
\definecolor{currentstroke}{rgb}{0.000000,0.000000,0.000000}%
\pgfsetstrokecolor{currentstroke}%
\pgfsetdash{}{0pt}%
\pgfpathmoveto{\pgfqpoint{4.821036in}{1.354907in}}%
\pgfpathlineto{\pgfqpoint{4.835101in}{1.356780in}}%
\pgfpathlineto{\pgfqpoint{4.849178in}{1.358763in}}%
\pgfpathlineto{\pgfqpoint{4.863266in}{1.360854in}}%
\pgfpathlineto{\pgfqpoint{4.877366in}{1.363054in}}%
\pgfpathlineto{\pgfqpoint{4.885109in}{1.376408in}}%
\pgfpathlineto{\pgfqpoint{4.892848in}{1.389816in}}%
\pgfpathlineto{\pgfqpoint{4.900583in}{1.403277in}}%
\pgfpathlineto{\pgfqpoint{4.908316in}{1.416788in}}%
\pgfpathlineto{\pgfqpoint{4.894214in}{1.414205in}}%
\pgfpathlineto{\pgfqpoint{4.880125in}{1.411730in}}%
\pgfpathlineto{\pgfqpoint{4.866047in}{1.409365in}}%
\pgfpathlineto{\pgfqpoint{4.851982in}{1.407109in}}%
\pgfpathlineto{\pgfqpoint{4.844251in}{1.393975in}}%
\pgfpathlineto{\pgfqpoint{4.836516in}{1.380895in}}%
\pgfpathlineto{\pgfqpoint{4.828778in}{1.367871in}}%
\pgfpathlineto{\pgfqpoint{4.821036in}{1.354907in}}%
\pgfpathclose%
\pgfusepath{fill}%
\end{pgfscope}%
\begin{pgfscope}%
\pgfpathrectangle{\pgfqpoint{1.254980in}{0.150000in}}{\pgfqpoint{5.490039in}{5.490039in}}%
\pgfusepath{clip}%
\pgfsetbuttcap%
\pgfsetroundjoin%
\definecolor{currentfill}{rgb}{0.271305,0.019942,0.347269}%
\pgfsetfillcolor{currentfill}%
\pgfsetfillopacity{0.700000}%
\pgfsetlinewidth{0.000000pt}%
\definecolor{currentstroke}{rgb}{0.000000,0.000000,0.000000}%
\pgfsetstrokecolor{currentstroke}%
\pgfsetdash{}{0pt}%
\pgfpathmoveto{\pgfqpoint{4.386038in}{1.141265in}}%
\pgfpathlineto{\pgfqpoint{4.399925in}{1.138821in}}%
\pgfpathlineto{\pgfqpoint{4.413820in}{1.136486in}}%
\pgfpathlineto{\pgfqpoint{4.427724in}{1.134262in}}%
\pgfpathlineto{\pgfqpoint{4.441636in}{1.132146in}}%
\pgfpathlineto{\pgfqpoint{4.449472in}{1.141369in}}%
\pgfpathlineto{\pgfqpoint{4.457303in}{1.150736in}}%
\pgfpathlineto{\pgfqpoint{4.465130in}{1.160242in}}%
\pgfpathlineto{\pgfqpoint{4.472952in}{1.169884in}}%
\pgfpathlineto{\pgfqpoint{4.459048in}{1.171542in}}%
\pgfpathlineto{\pgfqpoint{4.445153in}{1.173308in}}%
\pgfpathlineto{\pgfqpoint{4.431266in}{1.175185in}}%
\pgfpathlineto{\pgfqpoint{4.417389in}{1.177171in}}%
\pgfpathlineto{\pgfqpoint{4.409558in}{1.167980in}}%
\pgfpathlineto{\pgfqpoint{4.401723in}{1.158930in}}%
\pgfpathlineto{\pgfqpoint{4.393883in}{1.150023in}}%
\pgfpathlineto{\pgfqpoint{4.386038in}{1.141265in}}%
\pgfpathclose%
\pgfusepath{fill}%
\end{pgfscope}%
\begin{pgfscope}%
\pgfpathrectangle{\pgfqpoint{1.254980in}{0.150000in}}{\pgfqpoint{5.490039in}{5.490039in}}%
\pgfusepath{clip}%
\pgfsetbuttcap%
\pgfsetroundjoin%
\definecolor{currentfill}{rgb}{0.166383,0.690856,0.496502}%
\pgfsetfillcolor{currentfill}%
\pgfsetfillopacity{0.700000}%
\pgfsetlinewidth{0.000000pt}%
\definecolor{currentstroke}{rgb}{0.000000,0.000000,0.000000}%
\pgfsetstrokecolor{currentstroke}%
\pgfsetdash{}{0pt}%
\pgfpathmoveto{\pgfqpoint{2.347076in}{2.847558in}}%
\pgfpathlineto{\pgfqpoint{2.361159in}{2.823817in}}%
\pgfpathlineto{\pgfqpoint{2.375231in}{2.800266in}}%
\pgfpathlineto{\pgfqpoint{2.389294in}{2.776902in}}%
\pgfpathlineto{\pgfqpoint{2.403346in}{2.753724in}}%
\pgfpathlineto{\pgfqpoint{2.412674in}{2.739731in}}%
\pgfpathlineto{\pgfqpoint{2.421974in}{2.726141in}}%
\pgfpathlineto{\pgfqpoint{2.431247in}{2.712948in}}%
\pgfpathlineto{\pgfqpoint{2.440493in}{2.700146in}}%
\pgfpathlineto{\pgfqpoint{2.426508in}{2.722701in}}%
\pgfpathlineto{\pgfqpoint{2.412514in}{2.745441in}}%
\pgfpathlineto{\pgfqpoint{2.398510in}{2.768368in}}%
\pgfpathlineto{\pgfqpoint{2.384496in}{2.791481in}}%
\pgfpathlineto{\pgfqpoint{2.375183in}{2.804899in}}%
\pgfpathlineto{\pgfqpoint{2.365842in}{2.818713in}}%
\pgfpathlineto{\pgfqpoint{2.356473in}{2.832931in}}%
\pgfpathlineto{\pgfqpoint{2.347076in}{2.847558in}}%
\pgfpathclose%
\pgfusepath{fill}%
\end{pgfscope}%
\begin{pgfscope}%
\pgfpathrectangle{\pgfqpoint{1.254980in}{0.150000in}}{\pgfqpoint{5.490039in}{5.490039in}}%
\pgfusepath{clip}%
\pgfsetbuttcap%
\pgfsetroundjoin%
\definecolor{currentfill}{rgb}{0.268510,0.009605,0.335427}%
\pgfsetfillcolor{currentfill}%
\pgfsetfillopacity{0.700000}%
\pgfsetlinewidth{0.000000pt}%
\definecolor{currentstroke}{rgb}{0.000000,0.000000,0.000000}%
\pgfsetstrokecolor{currentstroke}%
\pgfsetdash{}{0pt}%
\pgfpathmoveto{\pgfqpoint{4.156753in}{1.126477in}}%
\pgfpathlineto{\pgfqpoint{4.170577in}{1.121742in}}%
\pgfpathlineto{\pgfqpoint{4.184408in}{1.117118in}}%
\pgfpathlineto{\pgfqpoint{4.198245in}{1.112606in}}%
\pgfpathlineto{\pgfqpoint{4.212089in}{1.108205in}}%
\pgfpathlineto{\pgfqpoint{4.220003in}{1.114648in}}%
\pgfpathlineto{\pgfqpoint{4.227911in}{1.121279in}}%
\pgfpathlineto{\pgfqpoint{4.235812in}{1.128094in}}%
\pgfpathlineto{\pgfqpoint{4.243708in}{1.135089in}}%
\pgfpathlineto{\pgfqpoint{4.229878in}{1.139000in}}%
\pgfpathlineto{\pgfqpoint{4.216056in}{1.143022in}}%
\pgfpathlineto{\pgfqpoint{4.202240in}{1.147156in}}%
\pgfpathlineto{\pgfqpoint{4.188431in}{1.151401in}}%
\pgfpathlineto{\pgfqpoint{4.180521in}{1.144891in}}%
\pgfpathlineto{\pgfqpoint{4.172605in}{1.138564in}}%
\pgfpathlineto{\pgfqpoint{4.164682in}{1.132425in}}%
\pgfpathlineto{\pgfqpoint{4.156753in}{1.126477in}}%
\pgfpathclose%
\pgfusepath{fill}%
\end{pgfscope}%
\begin{pgfscope}%
\pgfpathrectangle{\pgfqpoint{1.254980in}{0.150000in}}{\pgfqpoint{5.490039in}{5.490039in}}%
\pgfusepath{clip}%
\pgfsetbuttcap%
\pgfsetroundjoin%
\definecolor{currentfill}{rgb}{0.271305,0.019942,0.347269}%
\pgfsetfillcolor{currentfill}%
\pgfsetfillopacity{0.700000}%
\pgfsetlinewidth{0.000000pt}%
\definecolor{currentstroke}{rgb}{0.000000,0.000000,0.000000}%
\pgfsetstrokecolor{currentstroke}%
\pgfsetdash{}{0pt}%
\pgfpathmoveto{\pgfqpoint{4.014451in}{1.150678in}}%
\pgfpathlineto{\pgfqpoint{4.028246in}{1.144534in}}%
\pgfpathlineto{\pgfqpoint{4.042047in}{1.138504in}}%
\pgfpathlineto{\pgfqpoint{4.055853in}{1.132587in}}%
\pgfpathlineto{\pgfqpoint{4.069664in}{1.126783in}}%
\pgfpathlineto{\pgfqpoint{4.077639in}{1.131407in}}%
\pgfpathlineto{\pgfqpoint{4.085607in}{1.136243in}}%
\pgfpathlineto{\pgfqpoint{4.093567in}{1.141289in}}%
\pgfpathlineto{\pgfqpoint{4.101520in}{1.146540in}}%
\pgfpathlineto{\pgfqpoint{4.087727in}{1.151836in}}%
\pgfpathlineto{\pgfqpoint{4.073940in}{1.157246in}}%
\pgfpathlineto{\pgfqpoint{4.060158in}{1.162769in}}%
\pgfpathlineto{\pgfqpoint{4.046383in}{1.168405in}}%
\pgfpathlineto{\pgfqpoint{4.038411in}{1.163656in}}%
\pgfpathlineto{\pgfqpoint{4.030432in}{1.159115in}}%
\pgfpathlineto{\pgfqpoint{4.022445in}{1.154788in}}%
\pgfpathlineto{\pgfqpoint{4.014451in}{1.150678in}}%
\pgfpathclose%
\pgfusepath{fill}%
\end{pgfscope}%
\begin{pgfscope}%
\pgfpathrectangle{\pgfqpoint{1.254980in}{0.150000in}}{\pgfqpoint{5.490039in}{5.490039in}}%
\pgfusepath{clip}%
\pgfsetbuttcap%
\pgfsetroundjoin%
\definecolor{currentfill}{rgb}{0.283091,0.110553,0.431554}%
\pgfsetfillcolor{currentfill}%
\pgfsetfillopacity{0.700000}%
\pgfsetlinewidth{0.000000pt}%
\definecolor{currentstroke}{rgb}{0.000000,0.000000,0.000000}%
\pgfsetstrokecolor{currentstroke}%
\pgfsetdash{}{0pt}%
\pgfpathmoveto{\pgfqpoint{3.619189in}{1.334122in}}%
\pgfpathlineto{\pgfqpoint{3.632947in}{1.324091in}}%
\pgfpathlineto{\pgfqpoint{3.646707in}{1.314182in}}%
\pgfpathlineto{\pgfqpoint{3.660469in}{1.304393in}}%
\pgfpathlineto{\pgfqpoint{3.674234in}{1.294726in}}%
\pgfpathlineto{\pgfqpoint{3.682434in}{1.294172in}}%
\pgfpathlineto{\pgfqpoint{3.690622in}{1.293897in}}%
\pgfpathlineto{\pgfqpoint{3.698798in}{1.293894in}}%
\pgfpathlineto{\pgfqpoint{3.706964in}{1.294158in}}%
\pgfpathlineto{\pgfqpoint{3.693230in}{1.303278in}}%
\pgfpathlineto{\pgfqpoint{3.679498in}{1.312519in}}%
\pgfpathlineto{\pgfqpoint{3.665769in}{1.321880in}}%
\pgfpathlineto{\pgfqpoint{3.652043in}{1.331362in}}%
\pgfpathlineto{\pgfqpoint{3.643847in}{1.331639in}}%
\pgfpathlineto{\pgfqpoint{3.635640in}{1.332188in}}%
\pgfpathlineto{\pgfqpoint{3.627420in}{1.333014in}}%
\pgfpathlineto{\pgfqpoint{3.619189in}{1.334122in}}%
\pgfpathclose%
\pgfusepath{fill}%
\end{pgfscope}%
\begin{pgfscope}%
\pgfpathrectangle{\pgfqpoint{1.254980in}{0.150000in}}{\pgfqpoint{5.490039in}{5.490039in}}%
\pgfusepath{clip}%
\pgfsetbuttcap%
\pgfsetroundjoin%
\definecolor{currentfill}{rgb}{0.496615,0.826376,0.306377}%
\pgfsetfillcolor{currentfill}%
\pgfsetfillopacity{0.700000}%
\pgfsetlinewidth{0.000000pt}%
\definecolor{currentstroke}{rgb}{0.000000,0.000000,0.000000}%
\pgfsetstrokecolor{currentstroke}%
\pgfsetdash{}{0pt}%
\pgfpathmoveto{\pgfqpoint{2.101865in}{3.295106in}}%
\pgfpathlineto{\pgfqpoint{2.116116in}{3.267966in}}%
\pgfpathlineto{\pgfqpoint{2.130353in}{3.241041in}}%
\pgfpathlineto{\pgfqpoint{2.144576in}{3.214329in}}%
\pgfpathlineto{\pgfqpoint{2.158786in}{3.187830in}}%
\pgfpathlineto{\pgfqpoint{2.168357in}{3.172309in}}%
\pgfpathlineto{\pgfqpoint{2.177897in}{3.157200in}}%
\pgfpathlineto{\pgfqpoint{2.187407in}{3.142497in}}%
\pgfpathlineto{\pgfqpoint{2.196889in}{3.128194in}}%
\pgfpathlineto{\pgfqpoint{2.182753in}{3.154066in}}%
\pgfpathlineto{\pgfqpoint{2.168604in}{3.180148in}}%
\pgfpathlineto{\pgfqpoint{2.154441in}{3.206442in}}%
\pgfpathlineto{\pgfqpoint{2.140266in}{3.232949in}}%
\pgfpathlineto{\pgfqpoint{2.130712in}{3.247872in}}%
\pgfpathlineto{\pgfqpoint{2.121127in}{3.263202in}}%
\pgfpathlineto{\pgfqpoint{2.111512in}{3.278945in}}%
\pgfpathlineto{\pgfqpoint{2.101865in}{3.295106in}}%
\pgfpathclose%
\pgfusepath{fill}%
\end{pgfscope}%
\begin{pgfscope}%
\pgfpathrectangle{\pgfqpoint{1.254980in}{0.150000in}}{\pgfqpoint{5.490039in}{5.490039in}}%
\pgfusepath{clip}%
\pgfsetbuttcap%
\pgfsetroundjoin%
\definecolor{currentfill}{rgb}{0.278012,0.180367,0.486697}%
\pgfsetfillcolor{currentfill}%
\pgfsetfillopacity{0.700000}%
\pgfsetlinewidth{0.000000pt}%
\definecolor{currentstroke}{rgb}{0.000000,0.000000,0.000000}%
\pgfsetstrokecolor{currentstroke}%
\pgfsetdash{}{0pt}%
\pgfpathmoveto{\pgfqpoint{4.908316in}{1.416788in}}%
\pgfpathlineto{\pgfqpoint{4.922429in}{1.419480in}}%
\pgfpathlineto{\pgfqpoint{4.936556in}{1.422280in}}%
\pgfpathlineto{\pgfqpoint{4.950695in}{1.425190in}}%
\pgfpathlineto{\pgfqpoint{4.964846in}{1.428209in}}%
\pgfpathlineto{\pgfqpoint{4.972577in}{1.442139in}}%
\pgfpathlineto{\pgfqpoint{4.980304in}{1.456110in}}%
\pgfpathlineto{\pgfqpoint{4.988028in}{1.470117in}}%
\pgfpathlineto{\pgfqpoint{4.995749in}{1.484158in}}%
\pgfpathlineto{\pgfqpoint{4.981595in}{1.480771in}}%
\pgfpathlineto{\pgfqpoint{4.967454in}{1.477493in}}%
\pgfpathlineto{\pgfqpoint{4.953325in}{1.474325in}}%
\pgfpathlineto{\pgfqpoint{4.939209in}{1.471265in}}%
\pgfpathlineto{\pgfqpoint{4.931491in}{1.457586in}}%
\pgfpathlineto{\pgfqpoint{4.923769in}{1.443945in}}%
\pgfpathlineto{\pgfqpoint{4.916044in}{1.430344in}}%
\pgfpathlineto{\pgfqpoint{4.908316in}{1.416788in}}%
\pgfpathclose%
\pgfusepath{fill}%
\end{pgfscope}%
\begin{pgfscope}%
\pgfpathrectangle{\pgfqpoint{1.254980in}{0.150000in}}{\pgfqpoint{5.490039in}{5.490039in}}%
\pgfusepath{clip}%
\pgfsetbuttcap%
\pgfsetroundjoin%
\definecolor{currentfill}{rgb}{0.269944,0.014625,0.341379}%
\pgfsetfillcolor{currentfill}%
\pgfsetfillopacity{0.700000}%
\pgfsetlinewidth{0.000000pt}%
\definecolor{currentstroke}{rgb}{0.000000,0.000000,0.000000}%
\pgfsetstrokecolor{currentstroke}%
\pgfsetdash{}{0pt}%
\pgfpathmoveto{\pgfqpoint{4.299098in}{1.120553in}}%
\pgfpathlineto{\pgfqpoint{4.312964in}{1.117195in}}%
\pgfpathlineto{\pgfqpoint{4.326837in}{1.113948in}}%
\pgfpathlineto{\pgfqpoint{4.340719in}{1.110810in}}%
\pgfpathlineto{\pgfqpoint{4.354608in}{1.107783in}}%
\pgfpathlineto{\pgfqpoint{4.362473in}{1.115912in}}%
\pgfpathlineto{\pgfqpoint{4.370333in}{1.124205in}}%
\pgfpathlineto{\pgfqpoint{4.378188in}{1.132657in}}%
\pgfpathlineto{\pgfqpoint{4.386038in}{1.141265in}}%
\pgfpathlineto{\pgfqpoint{4.372160in}{1.143818in}}%
\pgfpathlineto{\pgfqpoint{4.358289in}{1.146482in}}%
\pgfpathlineto{\pgfqpoint{4.344427in}{1.149256in}}%
\pgfpathlineto{\pgfqpoint{4.330572in}{1.152140in}}%
\pgfpathlineto{\pgfqpoint{4.322712in}{1.144000in}}%
\pgfpathlineto{\pgfqpoint{4.314846in}{1.136020in}}%
\pgfpathlineto{\pgfqpoint{4.306974in}{1.128203in}}%
\pgfpathlineto{\pgfqpoint{4.299098in}{1.120553in}}%
\pgfpathclose%
\pgfusepath{fill}%
\end{pgfscope}%
\begin{pgfscope}%
\pgfpathrectangle{\pgfqpoint{1.254980in}{0.150000in}}{\pgfqpoint{5.490039in}{5.490039in}}%
\pgfusepath{clip}%
\pgfsetbuttcap%
\pgfsetroundjoin%
\definecolor{currentfill}{rgb}{0.276022,0.044167,0.370164}%
\pgfsetfillcolor{currentfill}%
\pgfsetfillopacity{0.700000}%
\pgfsetlinewidth{0.000000pt}%
\definecolor{currentstroke}{rgb}{0.000000,0.000000,0.000000}%
\pgfsetstrokecolor{currentstroke}%
\pgfsetdash{}{0pt}%
\pgfpathmoveto{\pgfqpoint{3.872030in}{1.193974in}}%
\pgfpathlineto{\pgfqpoint{3.885810in}{1.186387in}}%
\pgfpathlineto{\pgfqpoint{3.899594in}{1.178916in}}%
\pgfpathlineto{\pgfqpoint{3.913382in}{1.171561in}}%
\pgfpathlineto{\pgfqpoint{3.927175in}{1.164320in}}%
\pgfpathlineto{\pgfqpoint{3.935225in}{1.166983in}}%
\pgfpathlineto{\pgfqpoint{3.943266in}{1.169885in}}%
\pgfpathlineto{\pgfqpoint{3.951298in}{1.173023in}}%
\pgfpathlineto{\pgfqpoint{3.959322in}{1.176392in}}%
\pgfpathlineto{\pgfqpoint{3.945552in}{1.183107in}}%
\pgfpathlineto{\pgfqpoint{3.931787in}{1.189937in}}%
\pgfpathlineto{\pgfqpoint{3.918026in}{1.196883in}}%
\pgfpathlineto{\pgfqpoint{3.904271in}{1.203943in}}%
\pgfpathlineto{\pgfqpoint{3.896224in}{1.201093in}}%
\pgfpathlineto{\pgfqpoint{3.888168in}{1.198479in}}%
\pgfpathlineto{\pgfqpoint{3.880104in}{1.196104in}}%
\pgfpathlineto{\pgfqpoint{3.872030in}{1.193974in}}%
\pgfpathclose%
\pgfusepath{fill}%
\end{pgfscope}%
\begin{pgfscope}%
\pgfpathrectangle{\pgfqpoint{1.254980in}{0.150000in}}{\pgfqpoint{5.490039in}{5.490039in}}%
\pgfusepath{clip}%
\pgfsetbuttcap%
\pgfsetroundjoin%
\definecolor{currentfill}{rgb}{0.270595,0.214069,0.507052}%
\pgfsetfillcolor{currentfill}%
\pgfsetfillopacity{0.700000}%
\pgfsetlinewidth{0.000000pt}%
\definecolor{currentstroke}{rgb}{0.000000,0.000000,0.000000}%
\pgfsetstrokecolor{currentstroke}%
\pgfsetdash{}{0pt}%
\pgfpathmoveto{\pgfqpoint{4.995749in}{1.484158in}}%
\pgfpathlineto{\pgfqpoint{5.009916in}{1.487654in}}%
\pgfpathlineto{\pgfqpoint{5.024096in}{1.491259in}}%
\pgfpathlineto{\pgfqpoint{5.038289in}{1.494973in}}%
\pgfpathlineto{\pgfqpoint{5.046009in}{1.509315in}}%
\pgfpathlineto{\pgfqpoint{5.053725in}{1.523682in}}%
\pgfpathlineto{\pgfqpoint{5.061438in}{1.538071in}}%
\pgfpathlineto{\pgfqpoint{5.069147in}{1.552480in}}%
\pgfpathlineto{\pgfqpoint{5.054950in}{1.548412in}}%
\pgfpathlineto{\pgfqpoint{5.040766in}{1.544454in}}%
\pgfpathlineto{\pgfqpoint{5.026596in}{1.540605in}}%
\pgfpathlineto{\pgfqpoint{5.018890in}{1.526456in}}%
\pgfpathlineto{\pgfqpoint{5.011180in}{1.512330in}}%
\pgfpathlineto{\pgfqpoint{5.003466in}{1.498230in}}%
\pgfpathlineto{\pgfqpoint{4.995749in}{1.484158in}}%
\pgfpathclose%
\pgfusepath{fill}%
\end{pgfscope}%
\begin{pgfscope}%
\pgfpathrectangle{\pgfqpoint{1.254980in}{0.150000in}}{\pgfqpoint{5.490039in}{5.490039in}}%
\pgfusepath{clip}%
\pgfsetbuttcap%
\pgfsetroundjoin%
\definecolor{currentfill}{rgb}{0.226397,0.728888,0.462789}%
\pgfsetfillcolor{currentfill}%
\pgfsetfillopacity{0.700000}%
\pgfsetlinewidth{0.000000pt}%
\definecolor{currentstroke}{rgb}{0.000000,0.000000,0.000000}%
\pgfsetstrokecolor{currentstroke}%
\pgfsetdash{}{0pt}%
\pgfpathmoveto{\pgfqpoint{2.290638in}{2.944433in}}%
\pgfpathlineto{\pgfqpoint{2.304764in}{2.919925in}}%
\pgfpathlineto{\pgfqpoint{2.318879in}{2.895610in}}%
\pgfpathlineto{\pgfqpoint{2.332983in}{2.871488in}}%
\pgfpathlineto{\pgfqpoint{2.347076in}{2.847558in}}%
\pgfpathlineto{\pgfqpoint{2.356473in}{2.832931in}}%
\pgfpathlineto{\pgfqpoint{2.365842in}{2.818713in}}%
\pgfpathlineto{\pgfqpoint{2.375183in}{2.804899in}}%
\pgfpathlineto{\pgfqpoint{2.384496in}{2.791481in}}%
\pgfpathlineto{\pgfqpoint{2.370472in}{2.814784in}}%
\pgfpathlineto{\pgfqpoint{2.356438in}{2.838276in}}%
\pgfpathlineto{\pgfqpoint{2.342394in}{2.861959in}}%
\pgfpathlineto{\pgfqpoint{2.328339in}{2.885834in}}%
\pgfpathlineto{\pgfqpoint{2.318957in}{2.899872in}}%
\pgfpathlineto{\pgfqpoint{2.309546in}{2.914314in}}%
\pgfpathlineto{\pgfqpoint{2.300107in}{2.929166in}}%
\pgfpathlineto{\pgfqpoint{2.290638in}{2.944433in}}%
\pgfpathclose%
\pgfusepath{fill}%
\end{pgfscope}%
\begin{pgfscope}%
\pgfpathrectangle{\pgfqpoint{1.254980in}{0.150000in}}{\pgfqpoint{5.490039in}{5.490039in}}%
\pgfusepath{clip}%
\pgfsetbuttcap%
\pgfsetroundjoin%
\definecolor{currentfill}{rgb}{0.282656,0.100196,0.422160}%
\pgfsetfillcolor{currentfill}%
\pgfsetfillopacity{0.700000}%
\pgfsetlinewidth{0.000000pt}%
\definecolor{currentstroke}{rgb}{0.000000,0.000000,0.000000}%
\pgfsetstrokecolor{currentstroke}%
\pgfsetdash{}{0pt}%
\pgfpathmoveto{\pgfqpoint{3.674234in}{1.294726in}}%
\pgfpathlineto{\pgfqpoint{3.688002in}{1.285178in}}%
\pgfpathlineto{\pgfqpoint{3.701772in}{1.275750in}}%
\pgfpathlineto{\pgfqpoint{3.715544in}{1.266442in}}%
\pgfpathlineto{\pgfqpoint{3.729320in}{1.257253in}}%
\pgfpathlineto{\pgfqpoint{3.737489in}{1.257254in}}%
\pgfpathlineto{\pgfqpoint{3.745647in}{1.257527in}}%
\pgfpathlineto{\pgfqpoint{3.753794in}{1.258069in}}%
\pgfpathlineto{\pgfqpoint{3.761931in}{1.258873in}}%
\pgfpathlineto{\pgfqpoint{3.748184in}{1.267516in}}%
\pgfpathlineto{\pgfqpoint{3.734441in}{1.276277in}}%
\pgfpathlineto{\pgfqpoint{3.720701in}{1.285158in}}%
\pgfpathlineto{\pgfqpoint{3.706964in}{1.294158in}}%
\pgfpathlineto{\pgfqpoint{3.698798in}{1.293894in}}%
\pgfpathlineto{\pgfqpoint{3.690622in}{1.293897in}}%
\pgfpathlineto{\pgfqpoint{3.682434in}{1.294172in}}%
\pgfpathlineto{\pgfqpoint{3.674234in}{1.294726in}}%
\pgfpathclose%
\pgfusepath{fill}%
\end{pgfscope}%
\begin{pgfscope}%
\pgfpathrectangle{\pgfqpoint{1.254980in}{0.150000in}}{\pgfqpoint{5.490039in}{5.490039in}}%
\pgfusepath{clip}%
\pgfsetbuttcap%
\pgfsetroundjoin%
\definecolor{currentfill}{rgb}{0.280267,0.073417,0.397163}%
\pgfsetfillcolor{currentfill}%
\pgfsetfillopacity{0.700000}%
\pgfsetlinewidth{0.000000pt}%
\definecolor{currentstroke}{rgb}{0.000000,0.000000,0.000000}%
\pgfsetstrokecolor{currentstroke}%
\pgfsetdash{}{0pt}%
\pgfpathmoveto{\pgfqpoint{4.615709in}{1.203951in}}%
\pgfpathlineto{\pgfqpoint{4.629693in}{1.203718in}}%
\pgfpathlineto{\pgfqpoint{4.643686in}{1.203593in}}%
\pgfpathlineto{\pgfqpoint{4.657690in}{1.203576in}}%
\pgfpathlineto{\pgfqpoint{4.671703in}{1.203667in}}%
\pgfpathlineto{\pgfqpoint{4.679489in}{1.215235in}}%
\pgfpathlineto{\pgfqpoint{4.687271in}{1.226905in}}%
\pgfpathlineto{\pgfqpoint{4.695050in}{1.238674in}}%
\pgfpathlineto{\pgfqpoint{4.702825in}{1.250539in}}%
\pgfpathlineto{\pgfqpoint{4.688814in}{1.250019in}}%
\pgfpathlineto{\pgfqpoint{4.674813in}{1.249607in}}%
\pgfpathlineto{\pgfqpoint{4.660823in}{1.249305in}}%
\pgfpathlineto{\pgfqpoint{4.646844in}{1.249111in}}%
\pgfpathlineto{\pgfqpoint{4.639066in}{1.237668in}}%
\pgfpathlineto{\pgfqpoint{4.631284in}{1.226325in}}%
\pgfpathlineto{\pgfqpoint{4.623499in}{1.215085in}}%
\pgfpathlineto{\pgfqpoint{4.615709in}{1.203951in}}%
\pgfpathclose%
\pgfusepath{fill}%
\end{pgfscope}%
\begin{pgfscope}%
\pgfpathrectangle{\pgfqpoint{1.254980in}{0.150000in}}{\pgfqpoint{5.490039in}{5.490039in}}%
\pgfusepath{clip}%
\pgfsetbuttcap%
\pgfsetroundjoin%
\definecolor{currentfill}{rgb}{0.282656,0.100196,0.422160}%
\pgfsetfillcolor{currentfill}%
\pgfsetfillopacity{0.700000}%
\pgfsetlinewidth{0.000000pt}%
\definecolor{currentstroke}{rgb}{0.000000,0.000000,0.000000}%
\pgfsetstrokecolor{currentstroke}%
\pgfsetdash{}{0pt}%
\pgfpathmoveto{\pgfqpoint{4.702825in}{1.250539in}}%
\pgfpathlineto{\pgfqpoint{4.716847in}{1.251168in}}%
\pgfpathlineto{\pgfqpoint{4.730879in}{1.251905in}}%
\pgfpathlineto{\pgfqpoint{4.744923in}{1.252750in}}%
\pgfpathlineto{\pgfqpoint{4.758977in}{1.253704in}}%
\pgfpathlineto{\pgfqpoint{4.766747in}{1.266080in}}%
\pgfpathlineto{\pgfqpoint{4.774513in}{1.278541in}}%
\pgfpathlineto{\pgfqpoint{4.782275in}{1.291083in}}%
\pgfpathlineto{\pgfqpoint{4.790034in}{1.303704in}}%
\pgfpathlineto{\pgfqpoint{4.775981in}{1.302337in}}%
\pgfpathlineto{\pgfqpoint{4.761939in}{1.301078in}}%
\pgfpathlineto{\pgfqpoint{4.747908in}{1.299928in}}%
\pgfpathlineto{\pgfqpoint{4.733888in}{1.298887in}}%
\pgfpathlineto{\pgfqpoint{4.726127in}{1.286673in}}%
\pgfpathlineto{\pgfqpoint{4.718363in}{1.274542in}}%
\pgfpathlineto{\pgfqpoint{4.710596in}{1.262496in}}%
\pgfpathlineto{\pgfqpoint{4.702825in}{1.250539in}}%
\pgfpathclose%
\pgfusepath{fill}%
\end{pgfscope}%
\begin{pgfscope}%
\pgfpathrectangle{\pgfqpoint{1.254980in}{0.150000in}}{\pgfqpoint{5.490039in}{5.490039in}}%
\pgfusepath{clip}%
\pgfsetbuttcap%
\pgfsetroundjoin%
\definecolor{currentfill}{rgb}{0.277018,0.050344,0.375715}%
\pgfsetfillcolor{currentfill}%
\pgfsetfillopacity{0.700000}%
\pgfsetlinewidth{0.000000pt}%
\definecolor{currentstroke}{rgb}{0.000000,0.000000,0.000000}%
\pgfsetstrokecolor{currentstroke}%
\pgfsetdash{}{0pt}%
\pgfpathmoveto{\pgfqpoint{4.528658in}{1.164347in}}%
\pgfpathlineto{\pgfqpoint{4.542607in}{1.163236in}}%
\pgfpathlineto{\pgfqpoint{4.556566in}{1.162232in}}%
\pgfpathlineto{\pgfqpoint{4.570535in}{1.161338in}}%
\pgfpathlineto{\pgfqpoint{4.584513in}{1.160552in}}%
\pgfpathlineto{\pgfqpoint{4.592318in}{1.171225in}}%
\pgfpathlineto{\pgfqpoint{4.600119in}{1.182018in}}%
\pgfpathlineto{\pgfqpoint{4.607916in}{1.192928in}}%
\pgfpathlineto{\pgfqpoint{4.615709in}{1.203951in}}%
\pgfpathlineto{\pgfqpoint{4.601736in}{1.204294in}}%
\pgfpathlineto{\pgfqpoint{4.587773in}{1.204745in}}%
\pgfpathlineto{\pgfqpoint{4.573819in}{1.205305in}}%
\pgfpathlineto{\pgfqpoint{4.559875in}{1.205974in}}%
\pgfpathlineto{\pgfqpoint{4.552077in}{1.195387in}}%
\pgfpathlineto{\pgfqpoint{4.544275in}{1.184918in}}%
\pgfpathlineto{\pgfqpoint{4.536468in}{1.174571in}}%
\pgfpathlineto{\pgfqpoint{4.528658in}{1.164347in}}%
\pgfpathclose%
\pgfusepath{fill}%
\end{pgfscope}%
\begin{pgfscope}%
\pgfpathrectangle{\pgfqpoint{1.254980in}{0.150000in}}{\pgfqpoint{5.490039in}{5.490039in}}%
\pgfusepath{clip}%
\pgfsetbuttcap%
\pgfsetroundjoin%
\definecolor{currentfill}{rgb}{0.269944,0.014625,0.341379}%
\pgfsetfillcolor{currentfill}%
\pgfsetfillopacity{0.700000}%
\pgfsetlinewidth{0.000000pt}%
\definecolor{currentstroke}{rgb}{0.000000,0.000000,0.000000}%
\pgfsetstrokecolor{currentstroke}%
\pgfsetdash{}{0pt}%
\pgfpathmoveto{\pgfqpoint{4.069664in}{1.126783in}}%
\pgfpathlineto{\pgfqpoint{4.083482in}{1.121092in}}%
\pgfpathlineto{\pgfqpoint{4.097305in}{1.115514in}}%
\pgfpathlineto{\pgfqpoint{4.111134in}{1.110047in}}%
\pgfpathlineto{\pgfqpoint{4.124969in}{1.104693in}}%
\pgfpathlineto{\pgfqpoint{4.132925in}{1.109830in}}%
\pgfpathlineto{\pgfqpoint{4.140875in}{1.115176in}}%
\pgfpathlineto{\pgfqpoint{4.148817in}{1.120726in}}%
\pgfpathlineto{\pgfqpoint{4.156753in}{1.126477in}}%
\pgfpathlineto{\pgfqpoint{4.142936in}{1.131325in}}%
\pgfpathlineto{\pgfqpoint{4.129124in}{1.136284in}}%
\pgfpathlineto{\pgfqpoint{4.115319in}{1.141355in}}%
\pgfpathlineto{\pgfqpoint{4.101520in}{1.146540in}}%
\pgfpathlineto{\pgfqpoint{4.093567in}{1.141289in}}%
\pgfpathlineto{\pgfqpoint{4.085607in}{1.136243in}}%
\pgfpathlineto{\pgfqpoint{4.077639in}{1.131407in}}%
\pgfpathlineto{\pgfqpoint{4.069664in}{1.126783in}}%
\pgfpathclose%
\pgfusepath{fill}%
\end{pgfscope}%
\begin{pgfscope}%
\pgfpathrectangle{\pgfqpoint{1.254980in}{0.150000in}}{\pgfqpoint{5.490039in}{5.490039in}}%
\pgfusepath{clip}%
\pgfsetbuttcap%
\pgfsetroundjoin%
\definecolor{currentfill}{rgb}{0.268510,0.009605,0.335427}%
\pgfsetfillcolor{currentfill}%
\pgfsetfillopacity{0.700000}%
\pgfsetlinewidth{0.000000pt}%
\definecolor{currentstroke}{rgb}{0.000000,0.000000,0.000000}%
\pgfsetstrokecolor{currentstroke}%
\pgfsetdash{}{0pt}%
\pgfpathmoveto{\pgfqpoint{4.212089in}{1.108205in}}%
\pgfpathlineto{\pgfqpoint{4.225940in}{1.103915in}}%
\pgfpathlineto{\pgfqpoint{4.239798in}{1.099735in}}%
\pgfpathlineto{\pgfqpoint{4.253662in}{1.095667in}}%
\pgfpathlineto{\pgfqpoint{4.267534in}{1.091709in}}%
\pgfpathlineto{\pgfqpoint{4.275434in}{1.098648in}}%
\pgfpathlineto{\pgfqpoint{4.283327in}{1.105772in}}%
\pgfpathlineto{\pgfqpoint{4.291215in}{1.113075in}}%
\pgfpathlineto{\pgfqpoint{4.299098in}{1.120553in}}%
\pgfpathlineto{\pgfqpoint{4.285239in}{1.124021in}}%
\pgfpathlineto{\pgfqpoint{4.271388in}{1.127600in}}%
\pgfpathlineto{\pgfqpoint{4.257544in}{1.131289in}}%
\pgfpathlineto{\pgfqpoint{4.243708in}{1.135089in}}%
\pgfpathlineto{\pgfqpoint{4.235812in}{1.128094in}}%
\pgfpathlineto{\pgfqpoint{4.227911in}{1.121279in}}%
\pgfpathlineto{\pgfqpoint{4.220003in}{1.114648in}}%
\pgfpathlineto{\pgfqpoint{4.212089in}{1.108205in}}%
\pgfpathclose%
\pgfusepath{fill}%
\end{pgfscope}%
\begin{pgfscope}%
\pgfpathrectangle{\pgfqpoint{1.254980in}{0.150000in}}{\pgfqpoint{5.490039in}{5.490039in}}%
\pgfusepath{clip}%
\pgfsetbuttcap%
\pgfsetroundjoin%
\definecolor{currentfill}{rgb}{0.616293,0.852709,0.230052}%
\pgfsetfillcolor{currentfill}%
\pgfsetfillopacity{0.700000}%
\pgfsetlinewidth{0.000000pt}%
\definecolor{currentstroke}{rgb}{0.000000,0.000000,0.000000}%
\pgfsetstrokecolor{currentstroke}%
\pgfsetdash{}{0pt}%
\pgfpathmoveto{\pgfqpoint{2.044720in}{3.405850in}}%
\pgfpathlineto{\pgfqpoint{2.059028in}{3.377833in}}%
\pgfpathlineto{\pgfqpoint{2.073321in}{3.350038in}}%
\pgfpathlineto{\pgfqpoint{2.087600in}{3.322463in}}%
\pgfpathlineto{\pgfqpoint{2.101865in}{3.295106in}}%
\pgfpathlineto{\pgfqpoint{2.111512in}{3.278945in}}%
\pgfpathlineto{\pgfqpoint{2.121127in}{3.263202in}}%
\pgfpathlineto{\pgfqpoint{2.130712in}{3.247872in}}%
\pgfpathlineto{\pgfqpoint{2.140266in}{3.232949in}}%
\pgfpathlineto{\pgfqpoint{2.126077in}{3.259671in}}%
\pgfpathlineto{\pgfqpoint{2.111875in}{3.286609in}}%
\pgfpathlineto{\pgfqpoint{2.097658in}{3.313766in}}%
\pgfpathlineto{\pgfqpoint{2.083428in}{3.341142in}}%
\pgfpathlineto{\pgfqpoint{2.073798in}{3.356692in}}%
\pgfpathlineto{\pgfqpoint{2.064137in}{3.372657in}}%
\pgfpathlineto{\pgfqpoint{2.054444in}{3.389041in}}%
\pgfpathlineto{\pgfqpoint{2.044720in}{3.405850in}}%
\pgfpathclose%
\pgfusepath{fill}%
\end{pgfscope}%
\begin{pgfscope}%
\pgfpathrectangle{\pgfqpoint{1.254980in}{0.150000in}}{\pgfqpoint{5.490039in}{5.490039in}}%
\pgfusepath{clip}%
\pgfsetbuttcap%
\pgfsetroundjoin%
\definecolor{currentfill}{rgb}{0.283072,0.130895,0.449241}%
\pgfsetfillcolor{currentfill}%
\pgfsetfillopacity{0.700000}%
\pgfsetlinewidth{0.000000pt}%
\definecolor{currentstroke}{rgb}{0.000000,0.000000,0.000000}%
\pgfsetstrokecolor{currentstroke}%
\pgfsetdash{}{0pt}%
\pgfpathmoveto{\pgfqpoint{4.790034in}{1.303704in}}%
\pgfpathlineto{\pgfqpoint{4.804099in}{1.305180in}}%
\pgfpathlineto{\pgfqpoint{4.818176in}{1.306764in}}%
\pgfpathlineto{\pgfqpoint{4.832264in}{1.308457in}}%
\pgfpathlineto{\pgfqpoint{4.846363in}{1.310258in}}%
\pgfpathlineto{\pgfqpoint{4.854119in}{1.323358in}}%
\pgfpathlineto{\pgfqpoint{4.861871in}{1.336527in}}%
\pgfpathlineto{\pgfqpoint{4.869621in}{1.349760in}}%
\pgfpathlineto{\pgfqpoint{4.877366in}{1.363054in}}%
\pgfpathlineto{\pgfqpoint{4.863266in}{1.360854in}}%
\pgfpathlineto{\pgfqpoint{4.849178in}{1.358763in}}%
\pgfpathlineto{\pgfqpoint{4.835101in}{1.356780in}}%
\pgfpathlineto{\pgfqpoint{4.821036in}{1.354907in}}%
\pgfpathlineto{\pgfqpoint{4.813291in}{1.342004in}}%
\pgfpathlineto{\pgfqpoint{4.805542in}{1.329168in}}%
\pgfpathlineto{\pgfqpoint{4.797790in}{1.316400in}}%
\pgfpathlineto{\pgfqpoint{4.790034in}{1.303704in}}%
\pgfpathclose%
\pgfusepath{fill}%
\end{pgfscope}%
\begin{pgfscope}%
\pgfpathrectangle{\pgfqpoint{1.254980in}{0.150000in}}{\pgfqpoint{5.490039in}{5.490039in}}%
\pgfusepath{clip}%
\pgfsetbuttcap%
\pgfsetroundjoin%
\definecolor{currentfill}{rgb}{0.273809,0.031497,0.358853}%
\pgfsetfillcolor{currentfill}%
\pgfsetfillopacity{0.700000}%
\pgfsetlinewidth{0.000000pt}%
\definecolor{currentstroke}{rgb}{0.000000,0.000000,0.000000}%
\pgfsetstrokecolor{currentstroke}%
\pgfsetdash{}{0pt}%
\pgfpathmoveto{\pgfqpoint{4.441636in}{1.132146in}}%
\pgfpathlineto{\pgfqpoint{4.455557in}{1.130140in}}%
\pgfpathlineto{\pgfqpoint{4.469486in}{1.128243in}}%
\pgfpathlineto{\pgfqpoint{4.483425in}{1.126454in}}%
\pgfpathlineto{\pgfqpoint{4.497372in}{1.124775in}}%
\pgfpathlineto{\pgfqpoint{4.505200in}{1.134463in}}%
\pgfpathlineto{\pgfqpoint{4.513023in}{1.144290in}}%
\pgfpathlineto{\pgfqpoint{4.520843in}{1.154252in}}%
\pgfpathlineto{\pgfqpoint{4.528658in}{1.164347in}}%
\pgfpathlineto{\pgfqpoint{4.514718in}{1.165568in}}%
\pgfpathlineto{\pgfqpoint{4.500787in}{1.166898in}}%
\pgfpathlineto{\pgfqpoint{4.486865in}{1.168336in}}%
\pgfpathlineto{\pgfqpoint{4.472952in}{1.169884in}}%
\pgfpathlineto{\pgfqpoint{4.465130in}{1.160242in}}%
\pgfpathlineto{\pgfqpoint{4.457303in}{1.150736in}}%
\pgfpathlineto{\pgfqpoint{4.449472in}{1.141369in}}%
\pgfpathlineto{\pgfqpoint{4.441636in}{1.132146in}}%
\pgfpathclose%
\pgfusepath{fill}%
\end{pgfscope}%
\begin{pgfscope}%
\pgfpathrectangle{\pgfqpoint{1.254980in}{0.150000in}}{\pgfqpoint{5.490039in}{5.490039in}}%
\pgfusepath{clip}%
\pgfsetbuttcap%
\pgfsetroundjoin%
\definecolor{currentfill}{rgb}{0.274952,0.037752,0.364543}%
\pgfsetfillcolor{currentfill}%
\pgfsetfillopacity{0.700000}%
\pgfsetlinewidth{0.000000pt}%
\definecolor{currentstroke}{rgb}{0.000000,0.000000,0.000000}%
\pgfsetstrokecolor{currentstroke}%
\pgfsetdash{}{0pt}%
\pgfpathmoveto{\pgfqpoint{3.927175in}{1.164320in}}%
\pgfpathlineto{\pgfqpoint{3.940972in}{1.157195in}}%
\pgfpathlineto{\pgfqpoint{3.954775in}{1.150184in}}%
\pgfpathlineto{\pgfqpoint{3.968581in}{1.143288in}}%
\pgfpathlineto{\pgfqpoint{3.982393in}{1.136505in}}%
\pgfpathlineto{\pgfqpoint{3.990420in}{1.139700in}}%
\pgfpathlineto{\pgfqpoint{3.998438in}{1.143129in}}%
\pgfpathlineto{\pgfqpoint{4.006449in}{1.146790in}}%
\pgfpathlineto{\pgfqpoint{4.014451in}{1.150678in}}%
\pgfpathlineto{\pgfqpoint{4.000661in}{1.156935in}}%
\pgfpathlineto{\pgfqpoint{3.986876in}{1.163307in}}%
\pgfpathlineto{\pgfqpoint{3.973096in}{1.169792in}}%
\pgfpathlineto{\pgfqpoint{3.959322in}{1.176392in}}%
\pgfpathlineto{\pgfqpoint{3.951298in}{1.173023in}}%
\pgfpathlineto{\pgfqpoint{3.943266in}{1.169885in}}%
\pgfpathlineto{\pgfqpoint{3.935225in}{1.166983in}}%
\pgfpathlineto{\pgfqpoint{3.927175in}{1.164320in}}%
\pgfpathclose%
\pgfusepath{fill}%
\end{pgfscope}%
\begin{pgfscope}%
\pgfpathrectangle{\pgfqpoint{1.254980in}{0.150000in}}{\pgfqpoint{5.490039in}{5.490039in}}%
\pgfusepath{clip}%
\pgfsetbuttcap%
\pgfsetroundjoin%
\definecolor{currentfill}{rgb}{0.281446,0.084320,0.407414}%
\pgfsetfillcolor{currentfill}%
\pgfsetfillopacity{0.700000}%
\pgfsetlinewidth{0.000000pt}%
\definecolor{currentstroke}{rgb}{0.000000,0.000000,0.000000}%
\pgfsetstrokecolor{currentstroke}%
\pgfsetdash{}{0pt}%
\pgfpathmoveto{\pgfqpoint{3.729320in}{1.257253in}}%
\pgfpathlineto{\pgfqpoint{3.743098in}{1.248183in}}%
\pgfpathlineto{\pgfqpoint{3.756880in}{1.239232in}}%
\pgfpathlineto{\pgfqpoint{3.770665in}{1.230399in}}%
\pgfpathlineto{\pgfqpoint{3.784453in}{1.221683in}}%
\pgfpathlineto{\pgfqpoint{3.792593in}{1.222236in}}%
\pgfpathlineto{\pgfqpoint{3.800722in}{1.223058in}}%
\pgfpathlineto{\pgfqpoint{3.808841in}{1.224142in}}%
\pgfpathlineto{\pgfqpoint{3.816951in}{1.225485in}}%
\pgfpathlineto{\pgfqpoint{3.803190in}{1.233656in}}%
\pgfpathlineto{\pgfqpoint{3.789434in}{1.241943in}}%
\pgfpathlineto{\pgfqpoint{3.775681in}{1.250349in}}%
\pgfpathlineto{\pgfqpoint{3.761931in}{1.258873in}}%
\pgfpathlineto{\pgfqpoint{3.753794in}{1.258069in}}%
\pgfpathlineto{\pgfqpoint{3.745647in}{1.257527in}}%
\pgfpathlineto{\pgfqpoint{3.737489in}{1.257254in}}%
\pgfpathlineto{\pgfqpoint{3.729320in}{1.257253in}}%
\pgfpathclose%
\pgfusepath{fill}%
\end{pgfscope}%
\begin{pgfscope}%
\pgfpathrectangle{\pgfqpoint{1.254980in}{0.150000in}}{\pgfqpoint{5.490039in}{5.490039in}}%
\pgfusepath{clip}%
\pgfsetbuttcap%
\pgfsetroundjoin%
\definecolor{currentfill}{rgb}{0.296479,0.761561,0.424223}%
\pgfsetfillcolor{currentfill}%
\pgfsetfillopacity{0.700000}%
\pgfsetlinewidth{0.000000pt}%
\definecolor{currentstroke}{rgb}{0.000000,0.000000,0.000000}%
\pgfsetstrokecolor{currentstroke}%
\pgfsetdash{}{0pt}%
\pgfpathmoveto{\pgfqpoint{2.234020in}{3.044432in}}%
\pgfpathlineto{\pgfqpoint{2.248192in}{3.019134in}}%
\pgfpathlineto{\pgfqpoint{2.262352in}{2.994036in}}%
\pgfpathlineto{\pgfqpoint{2.276501in}{2.969136in}}%
\pgfpathlineto{\pgfqpoint{2.290638in}{2.944433in}}%
\pgfpathlineto{\pgfqpoint{2.300107in}{2.929166in}}%
\pgfpathlineto{\pgfqpoint{2.309546in}{2.914314in}}%
\pgfpathlineto{\pgfqpoint{2.318957in}{2.899872in}}%
\pgfpathlineto{\pgfqpoint{2.328339in}{2.885834in}}%
\pgfpathlineto{\pgfqpoint{2.314273in}{2.909904in}}%
\pgfpathlineto{\pgfqpoint{2.300197in}{2.934168in}}%
\pgfpathlineto{\pgfqpoint{2.286109in}{2.958628in}}%
\pgfpathlineto{\pgfqpoint{2.272010in}{2.983286in}}%
\pgfpathlineto{\pgfqpoint{2.262557in}{2.997951in}}%
\pgfpathlineto{\pgfqpoint{2.253074in}{3.013026in}}%
\pgfpathlineto{\pgfqpoint{2.243562in}{3.028518in}}%
\pgfpathlineto{\pgfqpoint{2.234020in}{3.044432in}}%
\pgfpathclose%
\pgfusepath{fill}%
\end{pgfscope}%
\begin{pgfscope}%
\pgfpathrectangle{\pgfqpoint{1.254980in}{0.150000in}}{\pgfqpoint{5.490039in}{5.490039in}}%
\pgfusepath{clip}%
\pgfsetbuttcap%
\pgfsetroundjoin%
\definecolor{currentfill}{rgb}{0.280255,0.165693,0.476498}%
\pgfsetfillcolor{currentfill}%
\pgfsetfillopacity{0.700000}%
\pgfsetlinewidth{0.000000pt}%
\definecolor{currentstroke}{rgb}{0.000000,0.000000,0.000000}%
\pgfsetstrokecolor{currentstroke}%
\pgfsetdash{}{0pt}%
\pgfpathmoveto{\pgfqpoint{4.877366in}{1.363054in}}%
\pgfpathlineto{\pgfqpoint{4.891479in}{1.365363in}}%
\pgfpathlineto{\pgfqpoint{4.905603in}{1.367780in}}%
\pgfpathlineto{\pgfqpoint{4.919740in}{1.370306in}}%
\pgfpathlineto{\pgfqpoint{4.933889in}{1.372940in}}%
\pgfpathlineto{\pgfqpoint{4.941633in}{1.386683in}}%
\pgfpathlineto{\pgfqpoint{4.949374in}{1.400477in}}%
\pgfpathlineto{\pgfqpoint{4.957112in}{1.414320in}}%
\pgfpathlineto{\pgfqpoint{4.964846in}{1.428209in}}%
\pgfpathlineto{\pgfqpoint{4.950695in}{1.425190in}}%
\pgfpathlineto{\pgfqpoint{4.936556in}{1.422280in}}%
\pgfpathlineto{\pgfqpoint{4.922429in}{1.419480in}}%
\pgfpathlineto{\pgfqpoint{4.908316in}{1.416788in}}%
\pgfpathlineto{\pgfqpoint{4.900583in}{1.403277in}}%
\pgfpathlineto{\pgfqpoint{4.892848in}{1.389816in}}%
\pgfpathlineto{\pgfqpoint{4.885109in}{1.376408in}}%
\pgfpathlineto{\pgfqpoint{4.877366in}{1.363054in}}%
\pgfpathclose%
\pgfusepath{fill}%
\end{pgfscope}%
\begin{pgfscope}%
\pgfpathrectangle{\pgfqpoint{1.254980in}{0.150000in}}{\pgfqpoint{5.490039in}{5.490039in}}%
\pgfusepath{clip}%
\pgfsetbuttcap%
\pgfsetroundjoin%
\definecolor{currentfill}{rgb}{0.271305,0.019942,0.347269}%
\pgfsetfillcolor{currentfill}%
\pgfsetfillopacity{0.700000}%
\pgfsetlinewidth{0.000000pt}%
\definecolor{currentstroke}{rgb}{0.000000,0.000000,0.000000}%
\pgfsetstrokecolor{currentstroke}%
\pgfsetdash{}{0pt}%
\pgfpathmoveto{\pgfqpoint{4.354608in}{1.107783in}}%
\pgfpathlineto{\pgfqpoint{4.368505in}{1.104865in}}%
\pgfpathlineto{\pgfqpoint{4.382410in}{1.102056in}}%
\pgfpathlineto{\pgfqpoint{4.396324in}{1.099357in}}%
\pgfpathlineto{\pgfqpoint{4.410245in}{1.096767in}}%
\pgfpathlineto{\pgfqpoint{4.418100in}{1.105377in}}%
\pgfpathlineto{\pgfqpoint{4.425950in}{1.114146in}}%
\pgfpathlineto{\pgfqpoint{4.433796in}{1.123071in}}%
\pgfpathlineto{\pgfqpoint{4.441636in}{1.132146in}}%
\pgfpathlineto{\pgfqpoint{4.427724in}{1.134262in}}%
\pgfpathlineto{\pgfqpoint{4.413820in}{1.136486in}}%
\pgfpathlineto{\pgfqpoint{4.399925in}{1.138821in}}%
\pgfpathlineto{\pgfqpoint{4.386038in}{1.141265in}}%
\pgfpathlineto{\pgfqpoint{4.378188in}{1.132657in}}%
\pgfpathlineto{\pgfqpoint{4.370333in}{1.124205in}}%
\pgfpathlineto{\pgfqpoint{4.362473in}{1.115912in}}%
\pgfpathlineto{\pgfqpoint{4.354608in}{1.107783in}}%
\pgfpathclose%
\pgfusepath{fill}%
\end{pgfscope}%
\begin{pgfscope}%
\pgfpathrectangle{\pgfqpoint{1.254980in}{0.150000in}}{\pgfqpoint{5.490039in}{5.490039in}}%
\pgfusepath{clip}%
\pgfsetbuttcap%
\pgfsetroundjoin%
\definecolor{currentfill}{rgb}{0.274128,0.199721,0.498911}%
\pgfsetfillcolor{currentfill}%
\pgfsetfillopacity{0.700000}%
\pgfsetlinewidth{0.000000pt}%
\definecolor{currentstroke}{rgb}{0.000000,0.000000,0.000000}%
\pgfsetstrokecolor{currentstroke}%
\pgfsetdash{}{0pt}%
\pgfpathmoveto{\pgfqpoint{4.964846in}{1.428209in}}%
\pgfpathlineto{\pgfqpoint{4.979010in}{1.431336in}}%
\pgfpathlineto{\pgfqpoint{4.993187in}{1.434572in}}%
\pgfpathlineto{\pgfqpoint{5.007377in}{1.437916in}}%
\pgfpathlineto{\pgfqpoint{5.015110in}{1.452129in}}%
\pgfpathlineto{\pgfqpoint{5.022840in}{1.466378in}}%
\pgfpathlineto{\pgfqpoint{5.030566in}{1.480660in}}%
\pgfpathlineto{\pgfqpoint{5.038289in}{1.494973in}}%
\pgfpathlineto{\pgfqpoint{5.024096in}{1.491259in}}%
\pgfpathlineto{\pgfqpoint{5.009916in}{1.487654in}}%
\pgfpathlineto{\pgfqpoint{4.995749in}{1.484158in}}%
\pgfpathlineto{\pgfqpoint{4.988028in}{1.470117in}}%
\pgfpathlineto{\pgfqpoint{4.980304in}{1.456110in}}%
\pgfpathlineto{\pgfqpoint{4.972577in}{1.442139in}}%
\pgfpathlineto{\pgfqpoint{4.964846in}{1.428209in}}%
\pgfpathclose%
\pgfusepath{fill}%
\end{pgfscope}%
\begin{pgfscope}%
\pgfpathrectangle{\pgfqpoint{1.254980in}{0.150000in}}{\pgfqpoint{5.490039in}{5.490039in}}%
\pgfusepath{clip}%
\pgfsetbuttcap%
\pgfsetroundjoin%
\definecolor{currentfill}{rgb}{0.269944,0.014625,0.341379}%
\pgfsetfillcolor{currentfill}%
\pgfsetfillopacity{0.700000}%
\pgfsetlinewidth{0.000000pt}%
\definecolor{currentstroke}{rgb}{0.000000,0.000000,0.000000}%
\pgfsetstrokecolor{currentstroke}%
\pgfsetdash{}{0pt}%
\pgfpathmoveto{\pgfqpoint{4.124969in}{1.104693in}}%
\pgfpathlineto{\pgfqpoint{4.138810in}{1.099451in}}%
\pgfpathlineto{\pgfqpoint{4.152657in}{1.094320in}}%
\pgfpathlineto{\pgfqpoint{4.166511in}{1.089301in}}%
\pgfpathlineto{\pgfqpoint{4.180371in}{1.084393in}}%
\pgfpathlineto{\pgfqpoint{4.188310in}{1.090043in}}%
\pgfpathlineto{\pgfqpoint{4.196243in}{1.095898in}}%
\pgfpathlineto{\pgfqpoint{4.204169in}{1.101953in}}%
\pgfpathlineto{\pgfqpoint{4.212089in}{1.108205in}}%
\pgfpathlineto{\pgfqpoint{4.198245in}{1.112606in}}%
\pgfpathlineto{\pgfqpoint{4.184408in}{1.117118in}}%
\pgfpathlineto{\pgfqpoint{4.170577in}{1.121742in}}%
\pgfpathlineto{\pgfqpoint{4.156753in}{1.126477in}}%
\pgfpathlineto{\pgfqpoint{4.148817in}{1.120726in}}%
\pgfpathlineto{\pgfqpoint{4.140875in}{1.115176in}}%
\pgfpathlineto{\pgfqpoint{4.132925in}{1.109830in}}%
\pgfpathlineto{\pgfqpoint{4.124969in}{1.104693in}}%
\pgfpathclose%
\pgfusepath{fill}%
\end{pgfscope}%
\begin{pgfscope}%
\pgfpathrectangle{\pgfqpoint{1.254980in}{0.150000in}}{\pgfqpoint{5.490039in}{5.490039in}}%
\pgfusepath{clip}%
\pgfsetbuttcap%
\pgfsetroundjoin%
\definecolor{currentfill}{rgb}{0.280267,0.073417,0.397163}%
\pgfsetfillcolor{currentfill}%
\pgfsetfillopacity{0.700000}%
\pgfsetlinewidth{0.000000pt}%
\definecolor{currentstroke}{rgb}{0.000000,0.000000,0.000000}%
\pgfsetstrokecolor{currentstroke}%
\pgfsetdash{}{0pt}%
\pgfpathmoveto{\pgfqpoint{3.784453in}{1.221683in}}%
\pgfpathlineto{\pgfqpoint{3.798244in}{1.213085in}}%
\pgfpathlineto{\pgfqpoint{3.812039in}{1.204604in}}%
\pgfpathlineto{\pgfqpoint{3.825837in}{1.196241in}}%
\pgfpathlineto{\pgfqpoint{3.839639in}{1.187993in}}%
\pgfpathlineto{\pgfqpoint{3.847752in}{1.189098in}}%
\pgfpathlineto{\pgfqpoint{3.855854in}{1.190466in}}%
\pgfpathlineto{\pgfqpoint{3.863947in}{1.192093in}}%
\pgfpathlineto{\pgfqpoint{3.872030in}{1.193974in}}%
\pgfpathlineto{\pgfqpoint{3.858254in}{1.201677in}}%
\pgfpathlineto{\pgfqpoint{3.844482in}{1.209496in}}%
\pgfpathlineto{\pgfqpoint{3.830715in}{1.217432in}}%
\pgfpathlineto{\pgfqpoint{3.816951in}{1.225485in}}%
\pgfpathlineto{\pgfqpoint{3.808841in}{1.224142in}}%
\pgfpathlineto{\pgfqpoint{3.800722in}{1.223058in}}%
\pgfpathlineto{\pgfqpoint{3.792593in}{1.222236in}}%
\pgfpathlineto{\pgfqpoint{3.784453in}{1.221683in}}%
\pgfpathclose%
\pgfusepath{fill}%
\end{pgfscope}%
\begin{pgfscope}%
\pgfpathrectangle{\pgfqpoint{1.254980in}{0.150000in}}{\pgfqpoint{5.490039in}{5.490039in}}%
\pgfusepath{clip}%
\pgfsetbuttcap%
\pgfsetroundjoin%
\definecolor{currentfill}{rgb}{0.273809,0.031497,0.358853}%
\pgfsetfillcolor{currentfill}%
\pgfsetfillopacity{0.700000}%
\pgfsetlinewidth{0.000000pt}%
\definecolor{currentstroke}{rgb}{0.000000,0.000000,0.000000}%
\pgfsetstrokecolor{currentstroke}%
\pgfsetdash{}{0pt}%
\pgfpathmoveto{\pgfqpoint{3.982393in}{1.136505in}}%
\pgfpathlineto{\pgfqpoint{3.996209in}{1.129837in}}%
\pgfpathlineto{\pgfqpoint{4.010031in}{1.123282in}}%
\pgfpathlineto{\pgfqpoint{4.023858in}{1.116840in}}%
\pgfpathlineto{\pgfqpoint{4.037690in}{1.110512in}}%
\pgfpathlineto{\pgfqpoint{4.045695in}{1.114238in}}%
\pgfpathlineto{\pgfqpoint{4.053692in}{1.118194in}}%
\pgfpathlineto{\pgfqpoint{4.061682in}{1.122378in}}%
\pgfpathlineto{\pgfqpoint{4.069664in}{1.126783in}}%
\pgfpathlineto{\pgfqpoint{4.055853in}{1.132587in}}%
\pgfpathlineto{\pgfqpoint{4.042047in}{1.138504in}}%
\pgfpathlineto{\pgfqpoint{4.028246in}{1.144534in}}%
\pgfpathlineto{\pgfqpoint{4.014451in}{1.150678in}}%
\pgfpathlineto{\pgfqpoint{4.006449in}{1.146790in}}%
\pgfpathlineto{\pgfqpoint{3.998438in}{1.143129in}}%
\pgfpathlineto{\pgfqpoint{3.990420in}{1.139700in}}%
\pgfpathlineto{\pgfqpoint{3.982393in}{1.136505in}}%
\pgfpathclose%
\pgfusepath{fill}%
\end{pgfscope}%
\begin{pgfscope}%
\pgfpathrectangle{\pgfqpoint{1.254980in}{0.150000in}}{\pgfqpoint{5.490039in}{5.490039in}}%
\pgfusepath{clip}%
\pgfsetbuttcap%
\pgfsetroundjoin%
\definecolor{currentfill}{rgb}{0.741388,0.873449,0.149561}%
\pgfsetfillcolor{currentfill}%
\pgfsetfillopacity{0.700000}%
\pgfsetlinewidth{0.000000pt}%
\definecolor{currentstroke}{rgb}{0.000000,0.000000,0.000000}%
\pgfsetstrokecolor{currentstroke}%
\pgfsetdash{}{0pt}%
\pgfpathmoveto{\pgfqpoint{1.987335in}{3.520165in}}%
\pgfpathlineto{\pgfqpoint{2.001704in}{3.491245in}}%
\pgfpathlineto{\pgfqpoint{2.016058in}{3.462554in}}%
\pgfpathlineto{\pgfqpoint{2.030396in}{3.434089in}}%
\pgfpathlineto{\pgfqpoint{2.044720in}{3.405850in}}%
\pgfpathlineto{\pgfqpoint{2.054444in}{3.389041in}}%
\pgfpathlineto{\pgfqpoint{2.064137in}{3.372657in}}%
\pgfpathlineto{\pgfqpoint{2.073798in}{3.356692in}}%
\pgfpathlineto{\pgfqpoint{2.083428in}{3.341142in}}%
\pgfpathlineto{\pgfqpoint{2.069182in}{3.368739in}}%
\pgfpathlineto{\pgfqpoint{2.054923in}{3.396560in}}%
\pgfpathlineto{\pgfqpoint{2.040648in}{3.424605in}}%
\pgfpathlineto{\pgfqpoint{2.026359in}{3.452876in}}%
\pgfpathlineto{\pgfqpoint{2.016651in}{3.469062in}}%
\pgfpathlineto{\pgfqpoint{2.006912in}{3.485667in}}%
\pgfpathlineto{\pgfqpoint{1.997140in}{3.502700in}}%
\pgfpathlineto{\pgfqpoint{1.987335in}{3.520165in}}%
\pgfpathclose%
\pgfusepath{fill}%
\end{pgfscope}%
\begin{pgfscope}%
\pgfpathrectangle{\pgfqpoint{1.254980in}{0.150000in}}{\pgfqpoint{5.490039in}{5.490039in}}%
\pgfusepath{clip}%
\pgfsetbuttcap%
\pgfsetroundjoin%
\definecolor{currentfill}{rgb}{0.386433,0.794644,0.372886}%
\pgfsetfillcolor{currentfill}%
\pgfsetfillopacity{0.700000}%
\pgfsetlinewidth{0.000000pt}%
\definecolor{currentstroke}{rgb}{0.000000,0.000000,0.000000}%
\pgfsetstrokecolor{currentstroke}%
\pgfsetdash{}{0pt}%
\pgfpathmoveto{\pgfqpoint{2.177210in}{3.147641in}}%
\pgfpathlineto{\pgfqpoint{2.191431in}{3.121532in}}%
\pgfpathlineto{\pgfqpoint{2.205640in}{3.095629in}}%
\pgfpathlineto{\pgfqpoint{2.219836in}{3.069929in}}%
\pgfpathlineto{\pgfqpoint{2.234020in}{3.044432in}}%
\pgfpathlineto{\pgfqpoint{2.243562in}{3.028518in}}%
\pgfpathlineto{\pgfqpoint{2.253074in}{3.013026in}}%
\pgfpathlineto{\pgfqpoint{2.262557in}{2.997951in}}%
\pgfpathlineto{\pgfqpoint{2.272010in}{2.983286in}}%
\pgfpathlineto{\pgfqpoint{2.257900in}{3.008143in}}%
\pgfpathlineto{\pgfqpoint{2.243777in}{3.033200in}}%
\pgfpathlineto{\pgfqpoint{2.229643in}{3.058459in}}%
\pgfpathlineto{\pgfqpoint{2.215497in}{3.083921in}}%
\pgfpathlineto{\pgfqpoint{2.205971in}{3.099220in}}%
\pgfpathlineto{\pgfqpoint{2.196414in}{3.114935in}}%
\pgfpathlineto{\pgfqpoint{2.186828in}{3.131073in}}%
\pgfpathlineto{\pgfqpoint{2.177210in}{3.147641in}}%
\pgfpathclose%
\pgfusepath{fill}%
\end{pgfscope}%
\begin{pgfscope}%
\pgfpathrectangle{\pgfqpoint{1.254980in}{0.150000in}}{\pgfqpoint{5.490039in}{5.490039in}}%
\pgfusepath{clip}%
\pgfsetbuttcap%
\pgfsetroundjoin%
\definecolor{currentfill}{rgb}{0.269944,0.014625,0.341379}%
\pgfsetfillcolor{currentfill}%
\pgfsetfillopacity{0.700000}%
\pgfsetlinewidth{0.000000pt}%
\definecolor{currentstroke}{rgb}{0.000000,0.000000,0.000000}%
\pgfsetstrokecolor{currentstroke}%
\pgfsetdash{}{0pt}%
\pgfpathmoveto{\pgfqpoint{4.267534in}{1.091709in}}%
\pgfpathlineto{\pgfqpoint{4.281413in}{1.087861in}}%
\pgfpathlineto{\pgfqpoint{4.295299in}{1.084123in}}%
\pgfpathlineto{\pgfqpoint{4.309193in}{1.080495in}}%
\pgfpathlineto{\pgfqpoint{4.323094in}{1.076977in}}%
\pgfpathlineto{\pgfqpoint{4.330981in}{1.084413in}}%
\pgfpathlineto{\pgfqpoint{4.338862in}{1.092029in}}%
\pgfpathlineto{\pgfqpoint{4.346738in}{1.099820in}}%
\pgfpathlineto{\pgfqpoint{4.354608in}{1.107783in}}%
\pgfpathlineto{\pgfqpoint{4.340719in}{1.110810in}}%
\pgfpathlineto{\pgfqpoint{4.326837in}{1.113948in}}%
\pgfpathlineto{\pgfqpoint{4.312964in}{1.117195in}}%
\pgfpathlineto{\pgfqpoint{4.299098in}{1.120553in}}%
\pgfpathlineto{\pgfqpoint{4.291215in}{1.113075in}}%
\pgfpathlineto{\pgfqpoint{4.283327in}{1.105772in}}%
\pgfpathlineto{\pgfqpoint{4.275434in}{1.098648in}}%
\pgfpathlineto{\pgfqpoint{4.267534in}{1.091709in}}%
\pgfpathclose%
\pgfusepath{fill}%
\end{pgfscope}%
\begin{pgfscope}%
\pgfpathrectangle{\pgfqpoint{1.254980in}{0.150000in}}{\pgfqpoint{5.490039in}{5.490039in}}%
\pgfusepath{clip}%
\pgfsetbuttcap%
\pgfsetroundjoin%
\definecolor{currentfill}{rgb}{0.281924,0.089666,0.412415}%
\pgfsetfillcolor{currentfill}%
\pgfsetfillopacity{0.700000}%
\pgfsetlinewidth{0.000000pt}%
\definecolor{currentstroke}{rgb}{0.000000,0.000000,0.000000}%
\pgfsetstrokecolor{currentstroke}%
\pgfsetdash{}{0pt}%
\pgfpathmoveto{\pgfqpoint{4.671703in}{1.203667in}}%
\pgfpathlineto{\pgfqpoint{4.685728in}{1.203867in}}%
\pgfpathlineto{\pgfqpoint{4.699762in}{1.204175in}}%
\pgfpathlineto{\pgfqpoint{4.713808in}{1.204591in}}%
\pgfpathlineto{\pgfqpoint{4.727864in}{1.205115in}}%
\pgfpathlineto{\pgfqpoint{4.735647in}{1.217118in}}%
\pgfpathlineto{\pgfqpoint{4.743427in}{1.229219in}}%
\pgfpathlineto{\pgfqpoint{4.751204in}{1.241416in}}%
\pgfpathlineto{\pgfqpoint{4.758977in}{1.253704in}}%
\pgfpathlineto{\pgfqpoint{4.744923in}{1.252750in}}%
\pgfpathlineto{\pgfqpoint{4.730879in}{1.251905in}}%
\pgfpathlineto{\pgfqpoint{4.716847in}{1.251168in}}%
\pgfpathlineto{\pgfqpoint{4.702825in}{1.250539in}}%
\pgfpathlineto{\pgfqpoint{4.695050in}{1.238674in}}%
\pgfpathlineto{\pgfqpoint{4.687271in}{1.226905in}}%
\pgfpathlineto{\pgfqpoint{4.679489in}{1.215235in}}%
\pgfpathlineto{\pgfqpoint{4.671703in}{1.203667in}}%
\pgfpathclose%
\pgfusepath{fill}%
\end{pgfscope}%
\begin{pgfscope}%
\pgfpathrectangle{\pgfqpoint{1.254980in}{0.150000in}}{\pgfqpoint{5.490039in}{5.490039in}}%
\pgfusepath{clip}%
\pgfsetbuttcap%
\pgfsetroundjoin%
\definecolor{currentfill}{rgb}{0.278791,0.062145,0.386592}%
\pgfsetfillcolor{currentfill}%
\pgfsetfillopacity{0.700000}%
\pgfsetlinewidth{0.000000pt}%
\definecolor{currentstroke}{rgb}{0.000000,0.000000,0.000000}%
\pgfsetstrokecolor{currentstroke}%
\pgfsetdash{}{0pt}%
\pgfpathmoveto{\pgfqpoint{4.584513in}{1.160552in}}%
\pgfpathlineto{\pgfqpoint{4.598500in}{1.159875in}}%
\pgfpathlineto{\pgfqpoint{4.612498in}{1.159306in}}%
\pgfpathlineto{\pgfqpoint{4.626505in}{1.158845in}}%
\pgfpathlineto{\pgfqpoint{4.640523in}{1.158492in}}%
\pgfpathlineto{\pgfqpoint{4.648324in}{1.169614in}}%
\pgfpathlineto{\pgfqpoint{4.656121in}{1.180853in}}%
\pgfpathlineto{\pgfqpoint{4.663914in}{1.192206in}}%
\pgfpathlineto{\pgfqpoint{4.671703in}{1.203667in}}%
\pgfpathlineto{\pgfqpoint{4.657690in}{1.203576in}}%
\pgfpathlineto{\pgfqpoint{4.643686in}{1.203593in}}%
\pgfpathlineto{\pgfqpoint{4.629693in}{1.203718in}}%
\pgfpathlineto{\pgfqpoint{4.615709in}{1.203951in}}%
\pgfpathlineto{\pgfqpoint{4.607916in}{1.192928in}}%
\pgfpathlineto{\pgfqpoint{4.600119in}{1.182018in}}%
\pgfpathlineto{\pgfqpoint{4.592318in}{1.171225in}}%
\pgfpathlineto{\pgfqpoint{4.584513in}{1.160552in}}%
\pgfpathclose%
\pgfusepath{fill}%
\end{pgfscope}%
\begin{pgfscope}%
\pgfpathrectangle{\pgfqpoint{1.254980in}{0.150000in}}{\pgfqpoint{5.490039in}{5.490039in}}%
\pgfusepath{clip}%
\pgfsetbuttcap%
\pgfsetroundjoin%
\definecolor{currentfill}{rgb}{0.283197,0.115680,0.436115}%
\pgfsetfillcolor{currentfill}%
\pgfsetfillopacity{0.700000}%
\pgfsetlinewidth{0.000000pt}%
\definecolor{currentstroke}{rgb}{0.000000,0.000000,0.000000}%
\pgfsetstrokecolor{currentstroke}%
\pgfsetdash{}{0pt}%
\pgfpathmoveto{\pgfqpoint{4.758977in}{1.253704in}}%
\pgfpathlineto{\pgfqpoint{4.773042in}{1.254765in}}%
\pgfpathlineto{\pgfqpoint{4.787119in}{1.255935in}}%
\pgfpathlineto{\pgfqpoint{4.801207in}{1.257214in}}%
\pgfpathlineto{\pgfqpoint{4.815307in}{1.258600in}}%
\pgfpathlineto{\pgfqpoint{4.823076in}{1.271396in}}%
\pgfpathlineto{\pgfqpoint{4.830841in}{1.284274in}}%
\pgfpathlineto{\pgfqpoint{4.838604in}{1.297229in}}%
\pgfpathlineto{\pgfqpoint{4.846363in}{1.310258in}}%
\pgfpathlineto{\pgfqpoint{4.832264in}{1.308457in}}%
\pgfpathlineto{\pgfqpoint{4.818176in}{1.306764in}}%
\pgfpathlineto{\pgfqpoint{4.804099in}{1.305180in}}%
\pgfpathlineto{\pgfqpoint{4.790034in}{1.303704in}}%
\pgfpathlineto{\pgfqpoint{4.782275in}{1.291083in}}%
\pgfpathlineto{\pgfqpoint{4.774513in}{1.278541in}}%
\pgfpathlineto{\pgfqpoint{4.766747in}{1.266080in}}%
\pgfpathlineto{\pgfqpoint{4.758977in}{1.253704in}}%
\pgfpathclose%
\pgfusepath{fill}%
\end{pgfscope}%
\begin{pgfscope}%
\pgfpathrectangle{\pgfqpoint{1.254980in}{0.150000in}}{\pgfqpoint{5.490039in}{5.490039in}}%
\pgfusepath{clip}%
\pgfsetbuttcap%
\pgfsetroundjoin%
\definecolor{currentfill}{rgb}{0.274952,0.037752,0.364543}%
\pgfsetfillcolor{currentfill}%
\pgfsetfillopacity{0.700000}%
\pgfsetlinewidth{0.000000pt}%
\definecolor{currentstroke}{rgb}{0.000000,0.000000,0.000000}%
\pgfsetstrokecolor{currentstroke}%
\pgfsetdash{}{0pt}%
\pgfpathmoveto{\pgfqpoint{4.497372in}{1.124775in}}%
\pgfpathlineto{\pgfqpoint{4.511328in}{1.123204in}}%
\pgfpathlineto{\pgfqpoint{4.525293in}{1.121742in}}%
\pgfpathlineto{\pgfqpoint{4.539268in}{1.120388in}}%
\pgfpathlineto{\pgfqpoint{4.553252in}{1.119143in}}%
\pgfpathlineto{\pgfqpoint{4.561073in}{1.129296in}}%
\pgfpathlineto{\pgfqpoint{4.568890in}{1.139584in}}%
\pgfpathlineto{\pgfqpoint{4.576703in}{1.150004in}}%
\pgfpathlineto{\pgfqpoint{4.584513in}{1.160552in}}%
\pgfpathlineto{\pgfqpoint{4.570535in}{1.161338in}}%
\pgfpathlineto{\pgfqpoint{4.556566in}{1.162232in}}%
\pgfpathlineto{\pgfqpoint{4.542607in}{1.163236in}}%
\pgfpathlineto{\pgfqpoint{4.528658in}{1.164347in}}%
\pgfpathlineto{\pgfqpoint{4.520843in}{1.154252in}}%
\pgfpathlineto{\pgfqpoint{4.513023in}{1.144290in}}%
\pgfpathlineto{\pgfqpoint{4.505200in}{1.134463in}}%
\pgfpathlineto{\pgfqpoint{4.497372in}{1.124775in}}%
\pgfpathclose%
\pgfusepath{fill}%
\end{pgfscope}%
\begin{pgfscope}%
\pgfpathrectangle{\pgfqpoint{1.254980in}{0.150000in}}{\pgfqpoint{5.490039in}{5.490039in}}%
\pgfusepath{clip}%
\pgfsetbuttcap%
\pgfsetroundjoin%
\definecolor{currentfill}{rgb}{0.278791,0.062145,0.386592}%
\pgfsetfillcolor{currentfill}%
\pgfsetfillopacity{0.700000}%
\pgfsetlinewidth{0.000000pt}%
\definecolor{currentstroke}{rgb}{0.000000,0.000000,0.000000}%
\pgfsetstrokecolor{currentstroke}%
\pgfsetdash{}{0pt}%
\pgfpathmoveto{\pgfqpoint{3.839639in}{1.187993in}}%
\pgfpathlineto{\pgfqpoint{3.853445in}{1.179863in}}%
\pgfpathlineto{\pgfqpoint{3.867255in}{1.171848in}}%
\pgfpathlineto{\pgfqpoint{3.881068in}{1.163948in}}%
\pgfpathlineto{\pgfqpoint{3.894886in}{1.156165in}}%
\pgfpathlineto{\pgfqpoint{3.902972in}{1.157820in}}%
\pgfpathlineto{\pgfqpoint{3.911049in}{1.159734in}}%
\pgfpathlineto{\pgfqpoint{3.919117in}{1.161903in}}%
\pgfpathlineto{\pgfqpoint{3.927175in}{1.164320in}}%
\pgfpathlineto{\pgfqpoint{3.913382in}{1.171561in}}%
\pgfpathlineto{\pgfqpoint{3.899594in}{1.178916in}}%
\pgfpathlineto{\pgfqpoint{3.885810in}{1.186387in}}%
\pgfpathlineto{\pgfqpoint{3.872030in}{1.193974in}}%
\pgfpathlineto{\pgfqpoint{3.863947in}{1.192093in}}%
\pgfpathlineto{\pgfqpoint{3.855854in}{1.190466in}}%
\pgfpathlineto{\pgfqpoint{3.847752in}{1.189098in}}%
\pgfpathlineto{\pgfqpoint{3.839639in}{1.187993in}}%
\pgfpathclose%
\pgfusepath{fill}%
\end{pgfscope}%
\begin{pgfscope}%
\pgfpathrectangle{\pgfqpoint{1.254980in}{0.150000in}}{\pgfqpoint{5.490039in}{5.490039in}}%
\pgfusepath{clip}%
\pgfsetbuttcap%
\pgfsetroundjoin%
\definecolor{currentfill}{rgb}{0.282290,0.145912,0.461510}%
\pgfsetfillcolor{currentfill}%
\pgfsetfillopacity{0.700000}%
\pgfsetlinewidth{0.000000pt}%
\definecolor{currentstroke}{rgb}{0.000000,0.000000,0.000000}%
\pgfsetstrokecolor{currentstroke}%
\pgfsetdash{}{0pt}%
\pgfpathmoveto{\pgfqpoint{4.846363in}{1.310258in}}%
\pgfpathlineto{\pgfqpoint{4.860474in}{1.312167in}}%
\pgfpathlineto{\pgfqpoint{4.874598in}{1.314185in}}%
\pgfpathlineto{\pgfqpoint{4.888733in}{1.316310in}}%
\pgfpathlineto{\pgfqpoint{4.902880in}{1.318544in}}%
\pgfpathlineto{\pgfqpoint{4.910637in}{1.332050in}}%
\pgfpathlineto{\pgfqpoint{4.918391in}{1.345620in}}%
\pgfpathlineto{\pgfqpoint{4.926142in}{1.359251in}}%
\pgfpathlineto{\pgfqpoint{4.933889in}{1.372940in}}%
\pgfpathlineto{\pgfqpoint{4.919740in}{1.370306in}}%
\pgfpathlineto{\pgfqpoint{4.905603in}{1.367780in}}%
\pgfpathlineto{\pgfqpoint{4.891479in}{1.365363in}}%
\pgfpathlineto{\pgfqpoint{4.877366in}{1.363054in}}%
\pgfpathlineto{\pgfqpoint{4.869621in}{1.349760in}}%
\pgfpathlineto{\pgfqpoint{4.861871in}{1.336527in}}%
\pgfpathlineto{\pgfqpoint{4.854119in}{1.323358in}}%
\pgfpathlineto{\pgfqpoint{4.846363in}{1.310258in}}%
\pgfpathclose%
\pgfusepath{fill}%
\end{pgfscope}%
\begin{pgfscope}%
\pgfpathrectangle{\pgfqpoint{1.254980in}{0.150000in}}{\pgfqpoint{5.490039in}{5.490039in}}%
\pgfusepath{clip}%
\pgfsetbuttcap%
\pgfsetroundjoin%
\definecolor{currentfill}{rgb}{0.272594,0.025563,0.353093}%
\pgfsetfillcolor{currentfill}%
\pgfsetfillopacity{0.700000}%
\pgfsetlinewidth{0.000000pt}%
\definecolor{currentstroke}{rgb}{0.000000,0.000000,0.000000}%
\pgfsetstrokecolor{currentstroke}%
\pgfsetdash{}{0pt}%
\pgfpathmoveto{\pgfqpoint{4.410245in}{1.096767in}}%
\pgfpathlineto{\pgfqpoint{4.424175in}{1.094287in}}%
\pgfpathlineto{\pgfqpoint{4.438113in}{1.091915in}}%
\pgfpathlineto{\pgfqpoint{4.452060in}{1.089652in}}%
\pgfpathlineto{\pgfqpoint{4.466015in}{1.087497in}}%
\pgfpathlineto{\pgfqpoint{4.473861in}{1.096588in}}%
\pgfpathlineto{\pgfqpoint{4.481702in}{1.105834in}}%
\pgfpathlineto{\pgfqpoint{4.489539in}{1.115231in}}%
\pgfpathlineto{\pgfqpoint{4.497372in}{1.124775in}}%
\pgfpathlineto{\pgfqpoint{4.483425in}{1.126454in}}%
\pgfpathlineto{\pgfqpoint{4.469486in}{1.128243in}}%
\pgfpathlineto{\pgfqpoint{4.455557in}{1.130140in}}%
\pgfpathlineto{\pgfqpoint{4.441636in}{1.132146in}}%
\pgfpathlineto{\pgfqpoint{4.433796in}{1.123071in}}%
\pgfpathlineto{\pgfqpoint{4.425950in}{1.114146in}}%
\pgfpathlineto{\pgfqpoint{4.418100in}{1.105377in}}%
\pgfpathlineto{\pgfqpoint{4.410245in}{1.096767in}}%
\pgfpathclose%
\pgfusepath{fill}%
\end{pgfscope}%
\begin{pgfscope}%
\pgfpathrectangle{\pgfqpoint{1.254980in}{0.150000in}}{\pgfqpoint{5.490039in}{5.490039in}}%
\pgfusepath{clip}%
\pgfsetbuttcap%
\pgfsetroundjoin%
\definecolor{currentfill}{rgb}{0.278012,0.180367,0.486697}%
\pgfsetfillcolor{currentfill}%
\pgfsetfillopacity{0.700000}%
\pgfsetlinewidth{0.000000pt}%
\definecolor{currentstroke}{rgb}{0.000000,0.000000,0.000000}%
\pgfsetstrokecolor{currentstroke}%
\pgfsetdash{}{0pt}%
\pgfpathmoveto{\pgfqpoint{4.933889in}{1.372940in}}%
\pgfpathlineto{\pgfqpoint{4.948051in}{1.375682in}}%
\pgfpathlineto{\pgfqpoint{4.962225in}{1.378533in}}%
\pgfpathlineto{\pgfqpoint{4.976411in}{1.381493in}}%
\pgfpathlineto{\pgfqpoint{4.984158in}{1.395529in}}%
\pgfpathlineto{\pgfqpoint{4.991901in}{1.409614in}}%
\pgfpathlineto{\pgfqpoint{4.999640in}{1.423744in}}%
\pgfpathlineto{\pgfqpoint{5.007377in}{1.437916in}}%
\pgfpathlineto{\pgfqpoint{4.993187in}{1.434572in}}%
\pgfpathlineto{\pgfqpoint{4.979010in}{1.431336in}}%
\pgfpathlineto{\pgfqpoint{4.964846in}{1.428209in}}%
\pgfpathlineto{\pgfqpoint{4.957112in}{1.414320in}}%
\pgfpathlineto{\pgfqpoint{4.949374in}{1.400477in}}%
\pgfpathlineto{\pgfqpoint{4.941633in}{1.386683in}}%
\pgfpathlineto{\pgfqpoint{4.933889in}{1.372940in}}%
\pgfpathclose%
\pgfusepath{fill}%
\end{pgfscope}%
\begin{pgfscope}%
\pgfpathrectangle{\pgfqpoint{1.254980in}{0.150000in}}{\pgfqpoint{5.490039in}{5.490039in}}%
\pgfusepath{clip}%
\pgfsetbuttcap%
\pgfsetroundjoin%
\definecolor{currentfill}{rgb}{0.269944,0.014625,0.341379}%
\pgfsetfillcolor{currentfill}%
\pgfsetfillopacity{0.700000}%
\pgfsetlinewidth{0.000000pt}%
\definecolor{currentstroke}{rgb}{0.000000,0.000000,0.000000}%
\pgfsetstrokecolor{currentstroke}%
\pgfsetdash{}{0pt}%
\pgfpathmoveto{\pgfqpoint{4.180371in}{1.084393in}}%
\pgfpathlineto{\pgfqpoint{4.194237in}{1.079596in}}%
\pgfpathlineto{\pgfqpoint{4.208110in}{1.074910in}}%
\pgfpathlineto{\pgfqpoint{4.221990in}{1.070334in}}%
\pgfpathlineto{\pgfqpoint{4.235876in}{1.065869in}}%
\pgfpathlineto{\pgfqpoint{4.243800in}{1.072032in}}%
\pgfpathlineto{\pgfqpoint{4.251717in}{1.078396in}}%
\pgfpathlineto{\pgfqpoint{4.259629in}{1.084956in}}%
\pgfpathlineto{\pgfqpoint{4.267534in}{1.091709in}}%
\pgfpathlineto{\pgfqpoint{4.253662in}{1.095667in}}%
\pgfpathlineto{\pgfqpoint{4.239798in}{1.099735in}}%
\pgfpathlineto{\pgfqpoint{4.225940in}{1.103915in}}%
\pgfpathlineto{\pgfqpoint{4.212089in}{1.108205in}}%
\pgfpathlineto{\pgfqpoint{4.204169in}{1.101953in}}%
\pgfpathlineto{\pgfqpoint{4.196243in}{1.095898in}}%
\pgfpathlineto{\pgfqpoint{4.188310in}{1.090043in}}%
\pgfpathlineto{\pgfqpoint{4.180371in}{1.084393in}}%
\pgfpathclose%
\pgfusepath{fill}%
\end{pgfscope}%
\begin{pgfscope}%
\pgfpathrectangle{\pgfqpoint{1.254980in}{0.150000in}}{\pgfqpoint{5.490039in}{5.490039in}}%
\pgfusepath{clip}%
\pgfsetbuttcap%
\pgfsetroundjoin%
\definecolor{currentfill}{rgb}{0.496615,0.826376,0.306377}%
\pgfsetfillcolor{currentfill}%
\pgfsetfillopacity{0.700000}%
\pgfsetlinewidth{0.000000pt}%
\definecolor{currentstroke}{rgb}{0.000000,0.000000,0.000000}%
\pgfsetstrokecolor{currentstroke}%
\pgfsetdash{}{0pt}%
\pgfpathmoveto{\pgfqpoint{2.120194in}{3.254151in}}%
\pgfpathlineto{\pgfqpoint{2.134468in}{3.227209in}}%
\pgfpathlineto{\pgfqpoint{2.148728in}{3.200477in}}%
\pgfpathlineto{\pgfqpoint{2.162975in}{3.173955in}}%
\pgfpathlineto{\pgfqpoint{2.177210in}{3.147641in}}%
\pgfpathlineto{\pgfqpoint{2.186828in}{3.131073in}}%
\pgfpathlineto{\pgfqpoint{2.196414in}{3.114935in}}%
\pgfpathlineto{\pgfqpoint{2.205971in}{3.099220in}}%
\pgfpathlineto{\pgfqpoint{2.215497in}{3.083921in}}%
\pgfpathlineto{\pgfqpoint{2.201338in}{3.109588in}}%
\pgfpathlineto{\pgfqpoint{2.187167in}{3.135461in}}%
\pgfpathlineto{\pgfqpoint{2.172983in}{3.161541in}}%
\pgfpathlineto{\pgfqpoint{2.158786in}{3.187830in}}%
\pgfpathlineto{\pgfqpoint{2.149185in}{3.203769in}}%
\pgfpathlineto{\pgfqpoint{2.139553in}{3.220131in}}%
\pgfpathlineto{\pgfqpoint{2.129889in}{3.236923in}}%
\pgfpathlineto{\pgfqpoint{2.120194in}{3.254151in}}%
\pgfpathclose%
\pgfusepath{fill}%
\end{pgfscope}%
\begin{pgfscope}%
\pgfpathrectangle{\pgfqpoint{1.254980in}{0.150000in}}{\pgfqpoint{5.490039in}{5.490039in}}%
\pgfusepath{clip}%
\pgfsetbuttcap%
\pgfsetroundjoin%
\definecolor{currentfill}{rgb}{0.272594,0.025563,0.353093}%
\pgfsetfillcolor{currentfill}%
\pgfsetfillopacity{0.700000}%
\pgfsetlinewidth{0.000000pt}%
\definecolor{currentstroke}{rgb}{0.000000,0.000000,0.000000}%
\pgfsetstrokecolor{currentstroke}%
\pgfsetdash{}{0pt}%
\pgfpathmoveto{\pgfqpoint{4.037690in}{1.110512in}}%
\pgfpathlineto{\pgfqpoint{4.051527in}{1.104297in}}%
\pgfpathlineto{\pgfqpoint{4.065370in}{1.098194in}}%
\pgfpathlineto{\pgfqpoint{4.079218in}{1.092203in}}%
\pgfpathlineto{\pgfqpoint{4.093072in}{1.086325in}}%
\pgfpathlineto{\pgfqpoint{4.101057in}{1.090581in}}%
\pgfpathlineto{\pgfqpoint{4.109035in}{1.095064in}}%
\pgfpathlineto{\pgfqpoint{4.117006in}{1.099770in}}%
\pgfpathlineto{\pgfqpoint{4.124969in}{1.104693in}}%
\pgfpathlineto{\pgfqpoint{4.111134in}{1.110047in}}%
\pgfpathlineto{\pgfqpoint{4.097305in}{1.115514in}}%
\pgfpathlineto{\pgfqpoint{4.083482in}{1.121092in}}%
\pgfpathlineto{\pgfqpoint{4.069664in}{1.126783in}}%
\pgfpathlineto{\pgfqpoint{4.061682in}{1.122378in}}%
\pgfpathlineto{\pgfqpoint{4.053692in}{1.118194in}}%
\pgfpathlineto{\pgfqpoint{4.045695in}{1.114238in}}%
\pgfpathlineto{\pgfqpoint{4.037690in}{1.110512in}}%
\pgfpathclose%
\pgfusepath{fill}%
\end{pgfscope}%
\begin{pgfscope}%
\pgfpathrectangle{\pgfqpoint{1.254980in}{0.150000in}}{\pgfqpoint{5.490039in}{5.490039in}}%
\pgfusepath{clip}%
\pgfsetbuttcap%
\pgfsetroundjoin%
\definecolor{currentfill}{rgb}{0.876168,0.891125,0.095250}%
\pgfsetfillcolor{currentfill}%
\pgfsetfillopacity{0.700000}%
\pgfsetlinewidth{0.000000pt}%
\definecolor{currentstroke}{rgb}{0.000000,0.000000,0.000000}%
\pgfsetstrokecolor{currentstroke}%
\pgfsetdash{}{0pt}%
\pgfpathmoveto{\pgfqpoint{1.929696in}{3.638161in}}%
\pgfpathlineto{\pgfqpoint{1.944131in}{3.608310in}}%
\pgfpathlineto{\pgfqpoint{1.958548in}{3.578695in}}%
\pgfpathlineto{\pgfqpoint{1.972950in}{3.549314in}}%
\pgfpathlineto{\pgfqpoint{1.987335in}{3.520165in}}%
\pgfpathlineto{\pgfqpoint{1.997140in}{3.502700in}}%
\pgfpathlineto{\pgfqpoint{2.006912in}{3.485667in}}%
\pgfpathlineto{\pgfqpoint{2.016651in}{3.469062in}}%
\pgfpathlineto{\pgfqpoint{2.026359in}{3.452876in}}%
\pgfpathlineto{\pgfqpoint{2.012054in}{3.481376in}}%
\pgfpathlineto{\pgfqpoint{1.997733in}{3.510105in}}%
\pgfpathlineto{\pgfqpoint{1.983397in}{3.539066in}}%
\pgfpathlineto{\pgfqpoint{1.969044in}{3.568260in}}%
\pgfpathlineto{\pgfqpoint{1.959257in}{3.585088in}}%
\pgfpathlineto{\pgfqpoint{1.949437in}{3.602343in}}%
\pgfpathlineto{\pgfqpoint{1.939584in}{3.620032in}}%
\pgfpathlineto{\pgfqpoint{1.929696in}{3.638161in}}%
\pgfpathclose%
\pgfusepath{fill}%
\end{pgfscope}%
\begin{pgfscope}%
\pgfpathrectangle{\pgfqpoint{1.254980in}{0.150000in}}{\pgfqpoint{5.490039in}{5.490039in}}%
\pgfusepath{clip}%
\pgfsetbuttcap%
\pgfsetroundjoin%
\definecolor{currentfill}{rgb}{0.269944,0.014625,0.341379}%
\pgfsetfillcolor{currentfill}%
\pgfsetfillopacity{0.700000}%
\pgfsetlinewidth{0.000000pt}%
\definecolor{currentstroke}{rgb}{0.000000,0.000000,0.000000}%
\pgfsetstrokecolor{currentstroke}%
\pgfsetdash{}{0pt}%
\pgfpathmoveto{\pgfqpoint{4.323094in}{1.076977in}}%
\pgfpathlineto{\pgfqpoint{4.337003in}{1.073568in}}%
\pgfpathlineto{\pgfqpoint{4.350919in}{1.070269in}}%
\pgfpathlineto{\pgfqpoint{4.364843in}{1.067080in}}%
\pgfpathlineto{\pgfqpoint{4.378775in}{1.063999in}}%
\pgfpathlineto{\pgfqpoint{4.386650in}{1.071932in}}%
\pgfpathlineto{\pgfqpoint{4.394520in}{1.080041in}}%
\pgfpathlineto{\pgfqpoint{4.402385in}{1.088321in}}%
\pgfpathlineto{\pgfqpoint{4.410245in}{1.096767in}}%
\pgfpathlineto{\pgfqpoint{4.396324in}{1.099357in}}%
\pgfpathlineto{\pgfqpoint{4.382410in}{1.102056in}}%
\pgfpathlineto{\pgfqpoint{4.368505in}{1.104865in}}%
\pgfpathlineto{\pgfqpoint{4.354608in}{1.107783in}}%
\pgfpathlineto{\pgfqpoint{4.346738in}{1.099820in}}%
\pgfpathlineto{\pgfqpoint{4.338862in}{1.092029in}}%
\pgfpathlineto{\pgfqpoint{4.330981in}{1.084413in}}%
\pgfpathlineto{\pgfqpoint{4.323094in}{1.076977in}}%
\pgfpathclose%
\pgfusepath{fill}%
\end{pgfscope}%
\begin{pgfscope}%
\pgfpathrectangle{\pgfqpoint{1.254980in}{0.150000in}}{\pgfqpoint{5.490039in}{5.490039in}}%
\pgfusepath{clip}%
\pgfsetbuttcap%
\pgfsetroundjoin%
\definecolor{currentfill}{rgb}{0.277941,0.056324,0.381191}%
\pgfsetfillcolor{currentfill}%
\pgfsetfillopacity{0.700000}%
\pgfsetlinewidth{0.000000pt}%
\definecolor{currentstroke}{rgb}{0.000000,0.000000,0.000000}%
\pgfsetstrokecolor{currentstroke}%
\pgfsetdash{}{0pt}%
\pgfpathmoveto{\pgfqpoint{3.894886in}{1.156165in}}%
\pgfpathlineto{\pgfqpoint{3.908708in}{1.148496in}}%
\pgfpathlineto{\pgfqpoint{3.922534in}{1.140942in}}%
\pgfpathlineto{\pgfqpoint{3.936365in}{1.133503in}}%
\pgfpathlineto{\pgfqpoint{3.950200in}{1.126178in}}%
\pgfpathlineto{\pgfqpoint{3.958261in}{1.128383in}}%
\pgfpathlineto{\pgfqpoint{3.966314in}{1.130842in}}%
\pgfpathlineto{\pgfqpoint{3.974358in}{1.133551in}}%
\pgfpathlineto{\pgfqpoint{3.982393in}{1.136505in}}%
\pgfpathlineto{\pgfqpoint{3.968581in}{1.143288in}}%
\pgfpathlineto{\pgfqpoint{3.954775in}{1.150184in}}%
\pgfpathlineto{\pgfqpoint{3.940972in}{1.157195in}}%
\pgfpathlineto{\pgfqpoint{3.927175in}{1.164320in}}%
\pgfpathlineto{\pgfqpoint{3.919117in}{1.161903in}}%
\pgfpathlineto{\pgfqpoint{3.911049in}{1.159734in}}%
\pgfpathlineto{\pgfqpoint{3.902972in}{1.157820in}}%
\pgfpathlineto{\pgfqpoint{3.894886in}{1.156165in}}%
\pgfpathclose%
\pgfusepath{fill}%
\end{pgfscope}%
\begin{pgfscope}%
\pgfpathrectangle{\pgfqpoint{1.254980in}{0.150000in}}{\pgfqpoint{5.490039in}{5.490039in}}%
\pgfusepath{clip}%
\pgfsetbuttcap%
\pgfsetroundjoin%
\definecolor{currentfill}{rgb}{0.280267,0.073417,0.397163}%
\pgfsetfillcolor{currentfill}%
\pgfsetfillopacity{0.700000}%
\pgfsetlinewidth{0.000000pt}%
\definecolor{currentstroke}{rgb}{0.000000,0.000000,0.000000}%
\pgfsetstrokecolor{currentstroke}%
\pgfsetdash{}{0pt}%
\pgfpathmoveto{\pgfqpoint{4.640523in}{1.158492in}}%
\pgfpathlineto{\pgfqpoint{4.654550in}{1.158247in}}%
\pgfpathlineto{\pgfqpoint{4.668588in}{1.158110in}}%
\pgfpathlineto{\pgfqpoint{4.682636in}{1.158081in}}%
\pgfpathlineto{\pgfqpoint{4.696695in}{1.158160in}}%
\pgfpathlineto{\pgfqpoint{4.704492in}{1.169733in}}%
\pgfpathlineto{\pgfqpoint{4.712286in}{1.181419in}}%
\pgfpathlineto{\pgfqpoint{4.720077in}{1.193215in}}%
\pgfpathlineto{\pgfqpoint{4.727864in}{1.205115in}}%
\pgfpathlineto{\pgfqpoint{4.713808in}{1.204591in}}%
\pgfpathlineto{\pgfqpoint{4.699762in}{1.204175in}}%
\pgfpathlineto{\pgfqpoint{4.685728in}{1.203867in}}%
\pgfpathlineto{\pgfqpoint{4.671703in}{1.203667in}}%
\pgfpathlineto{\pgfqpoint{4.663914in}{1.192206in}}%
\pgfpathlineto{\pgfqpoint{4.656121in}{1.180853in}}%
\pgfpathlineto{\pgfqpoint{4.648324in}{1.169614in}}%
\pgfpathlineto{\pgfqpoint{4.640523in}{1.158492in}}%
\pgfpathclose%
\pgfusepath{fill}%
\end{pgfscope}%
\begin{pgfscope}%
\pgfpathrectangle{\pgfqpoint{1.254980in}{0.150000in}}{\pgfqpoint{5.490039in}{5.490039in}}%
\pgfusepath{clip}%
\pgfsetbuttcap%
\pgfsetroundjoin%
\definecolor{currentfill}{rgb}{0.282656,0.100196,0.422160}%
\pgfsetfillcolor{currentfill}%
\pgfsetfillopacity{0.700000}%
\pgfsetlinewidth{0.000000pt}%
\definecolor{currentstroke}{rgb}{0.000000,0.000000,0.000000}%
\pgfsetstrokecolor{currentstroke}%
\pgfsetdash{}{0pt}%
\pgfpathmoveto{\pgfqpoint{4.727864in}{1.205115in}}%
\pgfpathlineto{\pgfqpoint{4.741931in}{1.205747in}}%
\pgfpathlineto{\pgfqpoint{4.756008in}{1.206487in}}%
\pgfpathlineto{\pgfqpoint{4.770097in}{1.207335in}}%
\pgfpathlineto{\pgfqpoint{4.784197in}{1.208291in}}%
\pgfpathlineto{\pgfqpoint{4.791979in}{1.220730in}}%
\pgfpathlineto{\pgfqpoint{4.799758in}{1.233263in}}%
\pgfpathlineto{\pgfqpoint{4.807534in}{1.245888in}}%
\pgfpathlineto{\pgfqpoint{4.815307in}{1.258600in}}%
\pgfpathlineto{\pgfqpoint{4.801207in}{1.257214in}}%
\pgfpathlineto{\pgfqpoint{4.787119in}{1.255935in}}%
\pgfpathlineto{\pgfqpoint{4.773042in}{1.254765in}}%
\pgfpathlineto{\pgfqpoint{4.758977in}{1.253704in}}%
\pgfpathlineto{\pgfqpoint{4.751204in}{1.241416in}}%
\pgfpathlineto{\pgfqpoint{4.743427in}{1.229219in}}%
\pgfpathlineto{\pgfqpoint{4.735647in}{1.217118in}}%
\pgfpathlineto{\pgfqpoint{4.727864in}{1.205115in}}%
\pgfpathclose%
\pgfusepath{fill}%
\end{pgfscope}%
\begin{pgfscope}%
\pgfpathrectangle{\pgfqpoint{1.254980in}{0.150000in}}{\pgfqpoint{5.490039in}{5.490039in}}%
\pgfusepath{clip}%
\pgfsetbuttcap%
\pgfsetroundjoin%
\definecolor{currentfill}{rgb}{0.277018,0.050344,0.375715}%
\pgfsetfillcolor{currentfill}%
\pgfsetfillopacity{0.700000}%
\pgfsetlinewidth{0.000000pt}%
\definecolor{currentstroke}{rgb}{0.000000,0.000000,0.000000}%
\pgfsetstrokecolor{currentstroke}%
\pgfsetdash{}{0pt}%
\pgfpathmoveto{\pgfqpoint{4.553252in}{1.119143in}}%
\pgfpathlineto{\pgfqpoint{4.567245in}{1.118006in}}%
\pgfpathlineto{\pgfqpoint{4.581248in}{1.116976in}}%
\pgfpathlineto{\pgfqpoint{4.595260in}{1.116055in}}%
\pgfpathlineto{\pgfqpoint{4.609282in}{1.115242in}}%
\pgfpathlineto{\pgfqpoint{4.617098in}{1.125861in}}%
\pgfpathlineto{\pgfqpoint{4.624910in}{1.136611in}}%
\pgfpathlineto{\pgfqpoint{4.632718in}{1.147490in}}%
\pgfpathlineto{\pgfqpoint{4.640523in}{1.158492in}}%
\pgfpathlineto{\pgfqpoint{4.626505in}{1.158845in}}%
\pgfpathlineto{\pgfqpoint{4.612498in}{1.159306in}}%
\pgfpathlineto{\pgfqpoint{4.598500in}{1.159875in}}%
\pgfpathlineto{\pgfqpoint{4.584513in}{1.160552in}}%
\pgfpathlineto{\pgfqpoint{4.576703in}{1.150004in}}%
\pgfpathlineto{\pgfqpoint{4.568890in}{1.139584in}}%
\pgfpathlineto{\pgfqpoint{4.561073in}{1.129296in}}%
\pgfpathlineto{\pgfqpoint{4.553252in}{1.119143in}}%
\pgfpathclose%
\pgfusepath{fill}%
\end{pgfscope}%
\begin{pgfscope}%
\pgfpathrectangle{\pgfqpoint{1.254980in}{0.150000in}}{\pgfqpoint{5.490039in}{5.490039in}}%
\pgfusepath{clip}%
\pgfsetbuttcap%
\pgfsetroundjoin%
\definecolor{currentfill}{rgb}{0.606045,0.850733,0.236712}%
\pgfsetfillcolor{currentfill}%
\pgfsetfillopacity{0.700000}%
\pgfsetlinewidth{0.000000pt}%
\definecolor{currentstroke}{rgb}{0.000000,0.000000,0.000000}%
\pgfsetstrokecolor{currentstroke}%
\pgfsetdash{}{0pt}%
\pgfpathmoveto{\pgfqpoint{2.062959in}{3.364058in}}%
\pgfpathlineto{\pgfqpoint{2.077289in}{3.336257in}}%
\pgfpathlineto{\pgfqpoint{2.091605in}{3.308673in}}%
\pgfpathlineto{\pgfqpoint{2.105906in}{3.281305in}}%
\pgfpathlineto{\pgfqpoint{2.120194in}{3.254151in}}%
\pgfpathlineto{\pgfqpoint{2.129889in}{3.236923in}}%
\pgfpathlineto{\pgfqpoint{2.139553in}{3.220131in}}%
\pgfpathlineto{\pgfqpoint{2.149185in}{3.203769in}}%
\pgfpathlineto{\pgfqpoint{2.158786in}{3.187830in}}%
\pgfpathlineto{\pgfqpoint{2.144576in}{3.214329in}}%
\pgfpathlineto{\pgfqpoint{2.130353in}{3.241041in}}%
\pgfpathlineto{\pgfqpoint{2.116116in}{3.267966in}}%
\pgfpathlineto{\pgfqpoint{2.101865in}{3.295106in}}%
\pgfpathlineto{\pgfqpoint{2.092187in}{3.311692in}}%
\pgfpathlineto{\pgfqpoint{2.082477in}{3.328709in}}%
\pgfpathlineto{\pgfqpoint{2.072734in}{3.346162in}}%
\pgfpathlineto{\pgfqpoint{2.062959in}{3.364058in}}%
\pgfpathclose%
\pgfusepath{fill}%
\end{pgfscope}%
\begin{pgfscope}%
\pgfpathrectangle{\pgfqpoint{1.254980in}{0.150000in}}{\pgfqpoint{5.490039in}{5.490039in}}%
\pgfusepath{clip}%
\pgfsetbuttcap%
\pgfsetroundjoin%
\definecolor{currentfill}{rgb}{0.271305,0.019942,0.347269}%
\pgfsetfillcolor{currentfill}%
\pgfsetfillopacity{0.700000}%
\pgfsetlinewidth{0.000000pt}%
\definecolor{currentstroke}{rgb}{0.000000,0.000000,0.000000}%
\pgfsetstrokecolor{currentstroke}%
\pgfsetdash{}{0pt}%
\pgfpathmoveto{\pgfqpoint{4.093072in}{1.086325in}}%
\pgfpathlineto{\pgfqpoint{4.106932in}{1.080558in}}%
\pgfpathlineto{\pgfqpoint{4.120797in}{1.074904in}}%
\pgfpathlineto{\pgfqpoint{4.134668in}{1.069360in}}%
\pgfpathlineto{\pgfqpoint{4.148546in}{1.063928in}}%
\pgfpathlineto{\pgfqpoint{4.156512in}{1.068715in}}%
\pgfpathlineto{\pgfqpoint{4.164472in}{1.073725in}}%
\pgfpathlineto{\pgfqpoint{4.172425in}{1.078952in}}%
\pgfpathlineto{\pgfqpoint{4.180371in}{1.084393in}}%
\pgfpathlineto{\pgfqpoint{4.166511in}{1.089301in}}%
\pgfpathlineto{\pgfqpoint{4.152657in}{1.094320in}}%
\pgfpathlineto{\pgfqpoint{4.138810in}{1.099451in}}%
\pgfpathlineto{\pgfqpoint{4.124969in}{1.104693in}}%
\pgfpathlineto{\pgfqpoint{4.117006in}{1.099770in}}%
\pgfpathlineto{\pgfqpoint{4.109035in}{1.095064in}}%
\pgfpathlineto{\pgfqpoint{4.101057in}{1.090581in}}%
\pgfpathlineto{\pgfqpoint{4.093072in}{1.086325in}}%
\pgfpathclose%
\pgfusepath{fill}%
\end{pgfscope}%
\begin{pgfscope}%
\pgfpathrectangle{\pgfqpoint{1.254980in}{0.150000in}}{\pgfqpoint{5.490039in}{5.490039in}}%
\pgfusepath{clip}%
\pgfsetbuttcap%
\pgfsetroundjoin%
\definecolor{currentfill}{rgb}{0.269944,0.014625,0.341379}%
\pgfsetfillcolor{currentfill}%
\pgfsetfillopacity{0.700000}%
\pgfsetlinewidth{0.000000pt}%
\definecolor{currentstroke}{rgb}{0.000000,0.000000,0.000000}%
\pgfsetstrokecolor{currentstroke}%
\pgfsetdash{}{0pt}%
\pgfpathmoveto{\pgfqpoint{4.235876in}{1.065869in}}%
\pgfpathlineto{\pgfqpoint{4.249769in}{1.061514in}}%
\pgfpathlineto{\pgfqpoint{4.263669in}{1.057269in}}%
\pgfpathlineto{\pgfqpoint{4.277577in}{1.053135in}}%
\pgfpathlineto{\pgfqpoint{4.291491in}{1.049109in}}%
\pgfpathlineto{\pgfqpoint{4.299400in}{1.055786in}}%
\pgfpathlineto{\pgfqpoint{4.307304in}{1.062659in}}%
\pgfpathlineto{\pgfqpoint{4.315202in}{1.069724in}}%
\pgfpathlineto{\pgfqpoint{4.323094in}{1.076977in}}%
\pgfpathlineto{\pgfqpoint{4.309193in}{1.080495in}}%
\pgfpathlineto{\pgfqpoint{4.295299in}{1.084123in}}%
\pgfpathlineto{\pgfqpoint{4.281413in}{1.087861in}}%
\pgfpathlineto{\pgfqpoint{4.267534in}{1.091709in}}%
\pgfpathlineto{\pgfqpoint{4.259629in}{1.084956in}}%
\pgfpathlineto{\pgfqpoint{4.251717in}{1.078396in}}%
\pgfpathlineto{\pgfqpoint{4.243800in}{1.072032in}}%
\pgfpathlineto{\pgfqpoint{4.235876in}{1.065869in}}%
\pgfpathclose%
\pgfusepath{fill}%
\end{pgfscope}%
\begin{pgfscope}%
\pgfpathrectangle{\pgfqpoint{1.254980in}{0.150000in}}{\pgfqpoint{5.490039in}{5.490039in}}%
\pgfusepath{clip}%
\pgfsetbuttcap%
\pgfsetroundjoin%
\definecolor{currentfill}{rgb}{0.283072,0.130895,0.449241}%
\pgfsetfillcolor{currentfill}%
\pgfsetfillopacity{0.700000}%
\pgfsetlinewidth{0.000000pt}%
\definecolor{currentstroke}{rgb}{0.000000,0.000000,0.000000}%
\pgfsetstrokecolor{currentstroke}%
\pgfsetdash{}{0pt}%
\pgfpathmoveto{\pgfqpoint{4.815307in}{1.258600in}}%
\pgfpathlineto{\pgfqpoint{4.829417in}{1.260094in}}%
\pgfpathlineto{\pgfqpoint{4.843540in}{1.261696in}}%
\pgfpathlineto{\pgfqpoint{4.857674in}{1.263406in}}%
\pgfpathlineto{\pgfqpoint{4.871820in}{1.265224in}}%
\pgfpathlineto{\pgfqpoint{4.879590in}{1.278442in}}%
\pgfpathlineto{\pgfqpoint{4.887356in}{1.291737in}}%
\pgfpathlineto{\pgfqpoint{4.895120in}{1.305105in}}%
\pgfpathlineto{\pgfqpoint{4.902880in}{1.318544in}}%
\pgfpathlineto{\pgfqpoint{4.888733in}{1.316310in}}%
\pgfpathlineto{\pgfqpoint{4.874598in}{1.314185in}}%
\pgfpathlineto{\pgfqpoint{4.860474in}{1.312167in}}%
\pgfpathlineto{\pgfqpoint{4.846363in}{1.310258in}}%
\pgfpathlineto{\pgfqpoint{4.838604in}{1.297229in}}%
\pgfpathlineto{\pgfqpoint{4.830841in}{1.284274in}}%
\pgfpathlineto{\pgfqpoint{4.823076in}{1.271396in}}%
\pgfpathlineto{\pgfqpoint{4.815307in}{1.258600in}}%
\pgfpathclose%
\pgfusepath{fill}%
\end{pgfscope}%
\begin{pgfscope}%
\pgfpathrectangle{\pgfqpoint{1.254980in}{0.150000in}}{\pgfqpoint{5.490039in}{5.490039in}}%
\pgfusepath{clip}%
\pgfsetbuttcap%
\pgfsetroundjoin%
\definecolor{currentfill}{rgb}{0.273809,0.031497,0.358853}%
\pgfsetfillcolor{currentfill}%
\pgfsetfillopacity{0.700000}%
\pgfsetlinewidth{0.000000pt}%
\definecolor{currentstroke}{rgb}{0.000000,0.000000,0.000000}%
\pgfsetstrokecolor{currentstroke}%
\pgfsetdash{}{0pt}%
\pgfpathmoveto{\pgfqpoint{4.466015in}{1.087497in}}%
\pgfpathlineto{\pgfqpoint{4.479979in}{1.085451in}}%
\pgfpathlineto{\pgfqpoint{4.493952in}{1.083514in}}%
\pgfpathlineto{\pgfqpoint{4.507934in}{1.081684in}}%
\pgfpathlineto{\pgfqpoint{4.521924in}{1.079963in}}%
\pgfpathlineto{\pgfqpoint{4.529762in}{1.089536in}}%
\pgfpathlineto{\pgfqpoint{4.537596in}{1.099259in}}%
\pgfpathlineto{\pgfqpoint{4.545426in}{1.109129in}}%
\pgfpathlineto{\pgfqpoint{4.553252in}{1.119143in}}%
\pgfpathlineto{\pgfqpoint{4.539268in}{1.120388in}}%
\pgfpathlineto{\pgfqpoint{4.525293in}{1.121742in}}%
\pgfpathlineto{\pgfqpoint{4.511328in}{1.123204in}}%
\pgfpathlineto{\pgfqpoint{4.497372in}{1.124775in}}%
\pgfpathlineto{\pgfqpoint{4.489539in}{1.115231in}}%
\pgfpathlineto{\pgfqpoint{4.481702in}{1.105834in}}%
\pgfpathlineto{\pgfqpoint{4.473861in}{1.096588in}}%
\pgfpathlineto{\pgfqpoint{4.466015in}{1.087497in}}%
\pgfpathclose%
\pgfusepath{fill}%
\end{pgfscope}%
\begin{pgfscope}%
\pgfpathrectangle{\pgfqpoint{1.254980in}{0.150000in}}{\pgfqpoint{5.490039in}{5.490039in}}%
\pgfusepath{clip}%
\pgfsetbuttcap%
\pgfsetroundjoin%
\definecolor{currentfill}{rgb}{0.280255,0.165693,0.476498}%
\pgfsetfillcolor{currentfill}%
\pgfsetfillopacity{0.700000}%
\pgfsetlinewidth{0.000000pt}%
\definecolor{currentstroke}{rgb}{0.000000,0.000000,0.000000}%
\pgfsetstrokecolor{currentstroke}%
\pgfsetdash{}{0pt}%
\pgfpathmoveto{\pgfqpoint{4.902880in}{1.318544in}}%
\pgfpathlineto{\pgfqpoint{4.917040in}{1.320886in}}%
\pgfpathlineto{\pgfqpoint{4.931211in}{1.323336in}}%
\pgfpathlineto{\pgfqpoint{4.945395in}{1.325894in}}%
\pgfpathlineto{\pgfqpoint{4.953154in}{1.339706in}}%
\pgfpathlineto{\pgfqpoint{4.960910in}{1.353578in}}%
\pgfpathlineto{\pgfqpoint{4.968662in}{1.367508in}}%
\pgfpathlineto{\pgfqpoint{4.976411in}{1.381493in}}%
\pgfpathlineto{\pgfqpoint{4.962225in}{1.378533in}}%
\pgfpathlineto{\pgfqpoint{4.948051in}{1.375682in}}%
\pgfpathlineto{\pgfqpoint{4.933889in}{1.372940in}}%
\pgfpathlineto{\pgfqpoint{4.926142in}{1.359251in}}%
\pgfpathlineto{\pgfqpoint{4.918391in}{1.345620in}}%
\pgfpathlineto{\pgfqpoint{4.910637in}{1.332050in}}%
\pgfpathlineto{\pgfqpoint{4.902880in}{1.318544in}}%
\pgfpathclose%
\pgfusepath{fill}%
\end{pgfscope}%
\begin{pgfscope}%
\pgfpathrectangle{\pgfqpoint{1.254980in}{0.150000in}}{\pgfqpoint{5.490039in}{5.490039in}}%
\pgfusepath{clip}%
\pgfsetbuttcap%
\pgfsetroundjoin%
\definecolor{currentfill}{rgb}{0.276022,0.044167,0.370164}%
\pgfsetfillcolor{currentfill}%
\pgfsetfillopacity{0.700000}%
\pgfsetlinewidth{0.000000pt}%
\definecolor{currentstroke}{rgb}{0.000000,0.000000,0.000000}%
\pgfsetstrokecolor{currentstroke}%
\pgfsetdash{}{0pt}%
\pgfpathmoveto{\pgfqpoint{3.950200in}{1.126178in}}%
\pgfpathlineto{\pgfqpoint{3.964040in}{1.118967in}}%
\pgfpathlineto{\pgfqpoint{3.977884in}{1.111869in}}%
\pgfpathlineto{\pgfqpoint{3.991733in}{1.104886in}}%
\pgfpathlineto{\pgfqpoint{4.005587in}{1.098015in}}%
\pgfpathlineto{\pgfqpoint{4.013625in}{1.100769in}}%
\pgfpathlineto{\pgfqpoint{4.021655in}{1.103773in}}%
\pgfpathlineto{\pgfqpoint{4.029676in}{1.107022in}}%
\pgfpathlineto{\pgfqpoint{4.037690in}{1.110512in}}%
\pgfpathlineto{\pgfqpoint{4.023858in}{1.116840in}}%
\pgfpathlineto{\pgfqpoint{4.010031in}{1.123282in}}%
\pgfpathlineto{\pgfqpoint{3.996209in}{1.129837in}}%
\pgfpathlineto{\pgfqpoint{3.982393in}{1.136505in}}%
\pgfpathlineto{\pgfqpoint{3.974358in}{1.133551in}}%
\pgfpathlineto{\pgfqpoint{3.966314in}{1.130842in}}%
\pgfpathlineto{\pgfqpoint{3.958261in}{1.128383in}}%
\pgfpathlineto{\pgfqpoint{3.950200in}{1.126178in}}%
\pgfpathclose%
\pgfusepath{fill}%
\end{pgfscope}%
\begin{pgfscope}%
\pgfpathrectangle{\pgfqpoint{1.254980in}{0.150000in}}{\pgfqpoint{5.490039in}{5.490039in}}%
\pgfusepath{clip}%
\pgfsetbuttcap%
\pgfsetroundjoin%
\definecolor{currentfill}{rgb}{0.271305,0.019942,0.347269}%
\pgfsetfillcolor{currentfill}%
\pgfsetfillopacity{0.700000}%
\pgfsetlinewidth{0.000000pt}%
\definecolor{currentstroke}{rgb}{0.000000,0.000000,0.000000}%
\pgfsetstrokecolor{currentstroke}%
\pgfsetdash{}{0pt}%
\pgfpathmoveto{\pgfqpoint{4.378775in}{1.063999in}}%
\pgfpathlineto{\pgfqpoint{4.392715in}{1.061027in}}%
\pgfpathlineto{\pgfqpoint{4.406664in}{1.058164in}}%
\pgfpathlineto{\pgfqpoint{4.420620in}{1.055410in}}%
\pgfpathlineto{\pgfqpoint{4.434585in}{1.052764in}}%
\pgfpathlineto{\pgfqpoint{4.442449in}{1.061195in}}%
\pgfpathlineto{\pgfqpoint{4.450309in}{1.069797in}}%
\pgfpathlineto{\pgfqpoint{4.458165in}{1.078565in}}%
\pgfpathlineto{\pgfqpoint{4.466015in}{1.087497in}}%
\pgfpathlineto{\pgfqpoint{4.452060in}{1.089652in}}%
\pgfpathlineto{\pgfqpoint{4.438113in}{1.091915in}}%
\pgfpathlineto{\pgfqpoint{4.424175in}{1.094287in}}%
\pgfpathlineto{\pgfqpoint{4.410245in}{1.096767in}}%
\pgfpathlineto{\pgfqpoint{4.402385in}{1.088321in}}%
\pgfpathlineto{\pgfqpoint{4.394520in}{1.080041in}}%
\pgfpathlineto{\pgfqpoint{4.386650in}{1.071932in}}%
\pgfpathlineto{\pgfqpoint{4.378775in}{1.063999in}}%
\pgfpathclose%
\pgfusepath{fill}%
\end{pgfscope}%
\begin{pgfscope}%
\pgfpathrectangle{\pgfqpoint{1.254980in}{0.150000in}}{\pgfqpoint{5.490039in}{5.490039in}}%
\pgfusepath{clip}%
\pgfsetbuttcap%
\pgfsetroundjoin%
\definecolor{currentfill}{rgb}{0.741388,0.873449,0.149561}%
\pgfsetfillcolor{currentfill}%
\pgfsetfillopacity{0.700000}%
\pgfsetlinewidth{0.000000pt}%
\definecolor{currentstroke}{rgb}{0.000000,0.000000,0.000000}%
\pgfsetstrokecolor{currentstroke}%
\pgfsetdash{}{0pt}%
\pgfpathmoveto{\pgfqpoint{2.005491in}{3.477462in}}%
\pgfpathlineto{\pgfqpoint{2.019880in}{3.448778in}}%
\pgfpathlineto{\pgfqpoint{2.034255in}{3.420316in}}%
\pgfpathlineto{\pgfqpoint{2.048614in}{3.392077in}}%
\pgfpathlineto{\pgfqpoint{2.062959in}{3.364058in}}%
\pgfpathlineto{\pgfqpoint{2.072734in}{3.346162in}}%
\pgfpathlineto{\pgfqpoint{2.082477in}{3.328709in}}%
\pgfpathlineto{\pgfqpoint{2.092187in}{3.311692in}}%
\pgfpathlineto{\pgfqpoint{2.101865in}{3.295106in}}%
\pgfpathlineto{\pgfqpoint{2.087600in}{3.322463in}}%
\pgfpathlineto{\pgfqpoint{2.073321in}{3.350038in}}%
\pgfpathlineto{\pgfqpoint{2.059028in}{3.377833in}}%
\pgfpathlineto{\pgfqpoint{2.044720in}{3.405850in}}%
\pgfpathlineto{\pgfqpoint{2.034962in}{3.423090in}}%
\pgfpathlineto{\pgfqpoint{2.025172in}{3.440769in}}%
\pgfpathlineto{\pgfqpoint{2.015348in}{3.458891in}}%
\pgfpathlineto{\pgfqpoint{2.005491in}{3.477462in}}%
\pgfpathclose%
\pgfusepath{fill}%
\end{pgfscope}%
\begin{pgfscope}%
\pgfpathrectangle{\pgfqpoint{1.254980in}{0.150000in}}{\pgfqpoint{5.490039in}{5.490039in}}%
\pgfusepath{clip}%
\pgfsetbuttcap%
\pgfsetroundjoin%
\definecolor{currentfill}{rgb}{0.271305,0.019942,0.347269}%
\pgfsetfillcolor{currentfill}%
\pgfsetfillopacity{0.700000}%
\pgfsetlinewidth{0.000000pt}%
\definecolor{currentstroke}{rgb}{0.000000,0.000000,0.000000}%
\pgfsetstrokecolor{currentstroke}%
\pgfsetdash{}{0pt}%
\pgfpathmoveto{\pgfqpoint{4.148546in}{1.063928in}}%
\pgfpathlineto{\pgfqpoint{4.162429in}{1.058607in}}%
\pgfpathlineto{\pgfqpoint{4.176319in}{1.053397in}}%
\pgfpathlineto{\pgfqpoint{4.190215in}{1.048298in}}%
\pgfpathlineto{\pgfqpoint{4.204118in}{1.043309in}}%
\pgfpathlineto{\pgfqpoint{4.212067in}{1.048626in}}%
\pgfpathlineto{\pgfqpoint{4.220010in}{1.054162in}}%
\pgfpathlineto{\pgfqpoint{4.227946in}{1.059911in}}%
\pgfpathlineto{\pgfqpoint{4.235876in}{1.065869in}}%
\pgfpathlineto{\pgfqpoint{4.221990in}{1.070334in}}%
\pgfpathlineto{\pgfqpoint{4.208110in}{1.074910in}}%
\pgfpathlineto{\pgfqpoint{4.194237in}{1.079596in}}%
\pgfpathlineto{\pgfqpoint{4.180371in}{1.084393in}}%
\pgfpathlineto{\pgfqpoint{4.172425in}{1.078952in}}%
\pgfpathlineto{\pgfqpoint{4.164472in}{1.073725in}}%
\pgfpathlineto{\pgfqpoint{4.156512in}{1.068715in}}%
\pgfpathlineto{\pgfqpoint{4.148546in}{1.063928in}}%
\pgfpathclose%
\pgfusepath{fill}%
\end{pgfscope}%
\begin{pgfscope}%
\pgfpathrectangle{\pgfqpoint{1.254980in}{0.150000in}}{\pgfqpoint{5.490039in}{5.490039in}}%
\pgfusepath{clip}%
\pgfsetbuttcap%
\pgfsetroundjoin%
\definecolor{currentfill}{rgb}{0.269944,0.014625,0.341379}%
\pgfsetfillcolor{currentfill}%
\pgfsetfillopacity{0.700000}%
\pgfsetlinewidth{0.000000pt}%
\definecolor{currentstroke}{rgb}{0.000000,0.000000,0.000000}%
\pgfsetstrokecolor{currentstroke}%
\pgfsetdash{}{0pt}%
\pgfpathmoveto{\pgfqpoint{4.291491in}{1.049109in}}%
\pgfpathlineto{\pgfqpoint{4.305413in}{1.045194in}}%
\pgfpathlineto{\pgfqpoint{4.319342in}{1.041388in}}%
\pgfpathlineto{\pgfqpoint{4.333278in}{1.037690in}}%
\pgfpathlineto{\pgfqpoint{4.347223in}{1.034102in}}%
\pgfpathlineto{\pgfqpoint{4.355119in}{1.041293in}}%
\pgfpathlineto{\pgfqpoint{4.363010in}{1.048675in}}%
\pgfpathlineto{\pgfqpoint{4.370895in}{1.056245in}}%
\pgfpathlineto{\pgfqpoint{4.378775in}{1.063999in}}%
\pgfpathlineto{\pgfqpoint{4.364843in}{1.067080in}}%
\pgfpathlineto{\pgfqpoint{4.350919in}{1.070269in}}%
\pgfpathlineto{\pgfqpoint{4.337003in}{1.073568in}}%
\pgfpathlineto{\pgfqpoint{4.323094in}{1.076977in}}%
\pgfpathlineto{\pgfqpoint{4.315202in}{1.069724in}}%
\pgfpathlineto{\pgfqpoint{4.307304in}{1.062659in}}%
\pgfpathlineto{\pgfqpoint{4.299400in}{1.055786in}}%
\pgfpathlineto{\pgfqpoint{4.291491in}{1.049109in}}%
\pgfpathclose%
\pgfusepath{fill}%
\end{pgfscope}%
\begin{pgfscope}%
\pgfpathrectangle{\pgfqpoint{1.254980in}{0.150000in}}{\pgfqpoint{5.490039in}{5.490039in}}%
\pgfusepath{clip}%
\pgfsetbuttcap%
\pgfsetroundjoin%
\definecolor{currentfill}{rgb}{0.274952,0.037752,0.364543}%
\pgfsetfillcolor{currentfill}%
\pgfsetfillopacity{0.700000}%
\pgfsetlinewidth{0.000000pt}%
\definecolor{currentstroke}{rgb}{0.000000,0.000000,0.000000}%
\pgfsetstrokecolor{currentstroke}%
\pgfsetdash{}{0pt}%
\pgfpathmoveto{\pgfqpoint{4.005587in}{1.098015in}}%
\pgfpathlineto{\pgfqpoint{4.019446in}{1.091258in}}%
\pgfpathlineto{\pgfqpoint{4.033311in}{1.084613in}}%
\pgfpathlineto{\pgfqpoint{4.047180in}{1.078080in}}%
\pgfpathlineto{\pgfqpoint{4.061055in}{1.071660in}}%
\pgfpathlineto{\pgfqpoint{4.069071in}{1.074963in}}%
\pgfpathlineto{\pgfqpoint{4.077079in}{1.078511in}}%
\pgfpathlineto{\pgfqpoint{4.085079in}{1.082300in}}%
\pgfpathlineto{\pgfqpoint{4.093072in}{1.086325in}}%
\pgfpathlineto{\pgfqpoint{4.079218in}{1.092203in}}%
\pgfpathlineto{\pgfqpoint{4.065370in}{1.098194in}}%
\pgfpathlineto{\pgfqpoint{4.051527in}{1.104297in}}%
\pgfpathlineto{\pgfqpoint{4.037690in}{1.110512in}}%
\pgfpathlineto{\pgfqpoint{4.029676in}{1.107022in}}%
\pgfpathlineto{\pgfqpoint{4.021655in}{1.103773in}}%
\pgfpathlineto{\pgfqpoint{4.013625in}{1.100769in}}%
\pgfpathlineto{\pgfqpoint{4.005587in}{1.098015in}}%
\pgfpathclose%
\pgfusepath{fill}%
\end{pgfscope}%
\begin{pgfscope}%
\pgfpathrectangle{\pgfqpoint{1.254980in}{0.150000in}}{\pgfqpoint{5.490039in}{5.490039in}}%
\pgfusepath{clip}%
\pgfsetbuttcap%
\pgfsetroundjoin%
\definecolor{currentfill}{rgb}{0.281924,0.089666,0.412415}%
\pgfsetfillcolor{currentfill}%
\pgfsetfillopacity{0.700000}%
\pgfsetlinewidth{0.000000pt}%
\definecolor{currentstroke}{rgb}{0.000000,0.000000,0.000000}%
\pgfsetstrokecolor{currentstroke}%
\pgfsetdash{}{0pt}%
\pgfpathmoveto{\pgfqpoint{4.696695in}{1.158160in}}%
\pgfpathlineto{\pgfqpoint{4.710764in}{1.158346in}}%
\pgfpathlineto{\pgfqpoint{4.724843in}{1.158640in}}%
\pgfpathlineto{\pgfqpoint{4.738934in}{1.159042in}}%
\pgfpathlineto{\pgfqpoint{4.753035in}{1.159551in}}%
\pgfpathlineto{\pgfqpoint{4.760830in}{1.171576in}}%
\pgfpathlineto{\pgfqpoint{4.768623in}{1.183710in}}%
\pgfpathlineto{\pgfqpoint{4.776412in}{1.195950in}}%
\pgfpathlineto{\pgfqpoint{4.784197in}{1.208291in}}%
\pgfpathlineto{\pgfqpoint{4.770097in}{1.207335in}}%
\pgfpathlineto{\pgfqpoint{4.756008in}{1.206487in}}%
\pgfpathlineto{\pgfqpoint{4.741931in}{1.205747in}}%
\pgfpathlineto{\pgfqpoint{4.727864in}{1.205115in}}%
\pgfpathlineto{\pgfqpoint{4.720077in}{1.193215in}}%
\pgfpathlineto{\pgfqpoint{4.712286in}{1.181419in}}%
\pgfpathlineto{\pgfqpoint{4.704492in}{1.169733in}}%
\pgfpathlineto{\pgfqpoint{4.696695in}{1.158160in}}%
\pgfpathclose%
\pgfusepath{fill}%
\end{pgfscope}%
\begin{pgfscope}%
\pgfpathrectangle{\pgfqpoint{1.254980in}{0.150000in}}{\pgfqpoint{5.490039in}{5.490039in}}%
\pgfusepath{clip}%
\pgfsetbuttcap%
\pgfsetroundjoin%
\definecolor{currentfill}{rgb}{0.278791,0.062145,0.386592}%
\pgfsetfillcolor{currentfill}%
\pgfsetfillopacity{0.700000}%
\pgfsetlinewidth{0.000000pt}%
\definecolor{currentstroke}{rgb}{0.000000,0.000000,0.000000}%
\pgfsetstrokecolor{currentstroke}%
\pgfsetdash{}{0pt}%
\pgfpathmoveto{\pgfqpoint{4.609282in}{1.115242in}}%
\pgfpathlineto{\pgfqpoint{4.623314in}{1.114537in}}%
\pgfpathlineto{\pgfqpoint{4.637356in}{1.113939in}}%
\pgfpathlineto{\pgfqpoint{4.651407in}{1.113449in}}%
\pgfpathlineto{\pgfqpoint{4.665469in}{1.113066in}}%
\pgfpathlineto{\pgfqpoint{4.673281in}{1.124152in}}%
\pgfpathlineto{\pgfqpoint{4.681089in}{1.135365in}}%
\pgfpathlineto{\pgfqpoint{4.688894in}{1.146702in}}%
\pgfpathlineto{\pgfqpoint{4.696695in}{1.158160in}}%
\pgfpathlineto{\pgfqpoint{4.682636in}{1.158081in}}%
\pgfpathlineto{\pgfqpoint{4.668588in}{1.158110in}}%
\pgfpathlineto{\pgfqpoint{4.654550in}{1.158247in}}%
\pgfpathlineto{\pgfqpoint{4.640523in}{1.158492in}}%
\pgfpathlineto{\pgfqpoint{4.632718in}{1.147490in}}%
\pgfpathlineto{\pgfqpoint{4.624910in}{1.136611in}}%
\pgfpathlineto{\pgfqpoint{4.617098in}{1.125861in}}%
\pgfpathlineto{\pgfqpoint{4.609282in}{1.115242in}}%
\pgfpathclose%
\pgfusepath{fill}%
\end{pgfscope}%
\begin{pgfscope}%
\pgfpathrectangle{\pgfqpoint{1.254980in}{0.150000in}}{\pgfqpoint{5.490039in}{5.490039in}}%
\pgfusepath{clip}%
\pgfsetbuttcap%
\pgfsetroundjoin%
\definecolor{currentfill}{rgb}{0.283197,0.115680,0.436115}%
\pgfsetfillcolor{currentfill}%
\pgfsetfillopacity{0.700000}%
\pgfsetlinewidth{0.000000pt}%
\definecolor{currentstroke}{rgb}{0.000000,0.000000,0.000000}%
\pgfsetstrokecolor{currentstroke}%
\pgfsetdash{}{0pt}%
\pgfpathmoveto{\pgfqpoint{4.784197in}{1.208291in}}%
\pgfpathlineto{\pgfqpoint{4.798308in}{1.209354in}}%
\pgfpathlineto{\pgfqpoint{4.812431in}{1.210525in}}%
\pgfpathlineto{\pgfqpoint{4.826564in}{1.211803in}}%
\pgfpathlineto{\pgfqpoint{4.840710in}{1.213189in}}%
\pgfpathlineto{\pgfqpoint{4.848492in}{1.226065in}}%
\pgfpathlineto{\pgfqpoint{4.856271in}{1.239032in}}%
\pgfpathlineto{\pgfqpoint{4.864047in}{1.252086in}}%
\pgfpathlineto{\pgfqpoint{4.871820in}{1.265224in}}%
\pgfpathlineto{\pgfqpoint{4.857674in}{1.263406in}}%
\pgfpathlineto{\pgfqpoint{4.843540in}{1.261696in}}%
\pgfpathlineto{\pgfqpoint{4.829417in}{1.260094in}}%
\pgfpathlineto{\pgfqpoint{4.815307in}{1.258600in}}%
\pgfpathlineto{\pgfqpoint{4.807534in}{1.245888in}}%
\pgfpathlineto{\pgfqpoint{4.799758in}{1.233263in}}%
\pgfpathlineto{\pgfqpoint{4.791979in}{1.220730in}}%
\pgfpathlineto{\pgfqpoint{4.784197in}{1.208291in}}%
\pgfpathclose%
\pgfusepath{fill}%
\end{pgfscope}%
\begin{pgfscope}%
\pgfpathrectangle{\pgfqpoint{1.254980in}{0.150000in}}{\pgfqpoint{5.490039in}{5.490039in}}%
\pgfusepath{clip}%
\pgfsetbuttcap%
\pgfsetroundjoin%
\definecolor{currentfill}{rgb}{0.276022,0.044167,0.370164}%
\pgfsetfillcolor{currentfill}%
\pgfsetfillopacity{0.700000}%
\pgfsetlinewidth{0.000000pt}%
\definecolor{currentstroke}{rgb}{0.000000,0.000000,0.000000}%
\pgfsetstrokecolor{currentstroke}%
\pgfsetdash{}{0pt}%
\pgfpathmoveto{\pgfqpoint{4.521924in}{1.079963in}}%
\pgfpathlineto{\pgfqpoint{4.535924in}{1.078350in}}%
\pgfpathlineto{\pgfqpoint{4.549933in}{1.076845in}}%
\pgfpathlineto{\pgfqpoint{4.563951in}{1.075448in}}%
\pgfpathlineto{\pgfqpoint{4.577979in}{1.074158in}}%
\pgfpathlineto{\pgfqpoint{4.585811in}{1.084213in}}%
\pgfpathlineto{\pgfqpoint{4.593638in}{1.094414in}}%
\pgfpathlineto{\pgfqpoint{4.601462in}{1.104758in}}%
\pgfpathlineto{\pgfqpoint{4.609282in}{1.115242in}}%
\pgfpathlineto{\pgfqpoint{4.595260in}{1.116055in}}%
\pgfpathlineto{\pgfqpoint{4.581248in}{1.116976in}}%
\pgfpathlineto{\pgfqpoint{4.567245in}{1.118006in}}%
\pgfpathlineto{\pgfqpoint{4.553252in}{1.119143in}}%
\pgfpathlineto{\pgfqpoint{4.545426in}{1.109129in}}%
\pgfpathlineto{\pgfqpoint{4.537596in}{1.099259in}}%
\pgfpathlineto{\pgfqpoint{4.529762in}{1.089536in}}%
\pgfpathlineto{\pgfqpoint{4.521924in}{1.079963in}}%
\pgfpathclose%
\pgfusepath{fill}%
\end{pgfscope}%
\begin{pgfscope}%
\pgfpathrectangle{\pgfqpoint{1.254980in}{0.150000in}}{\pgfqpoint{5.490039in}{5.490039in}}%
\pgfusepath{clip}%
\pgfsetbuttcap%
\pgfsetroundjoin%
\definecolor{currentfill}{rgb}{0.282290,0.145912,0.461510}%
\pgfsetfillcolor{currentfill}%
\pgfsetfillopacity{0.700000}%
\pgfsetlinewidth{0.000000pt}%
\definecolor{currentstroke}{rgb}{0.000000,0.000000,0.000000}%
\pgfsetstrokecolor{currentstroke}%
\pgfsetdash{}{0pt}%
\pgfpathmoveto{\pgfqpoint{4.871820in}{1.265224in}}%
\pgfpathlineto{\pgfqpoint{4.885978in}{1.267149in}}%
\pgfpathlineto{\pgfqpoint{4.900148in}{1.269182in}}%
\pgfpathlineto{\pgfqpoint{4.914330in}{1.271323in}}%
\pgfpathlineto{\pgfqpoint{4.922101in}{1.284858in}}%
\pgfpathlineto{\pgfqpoint{4.929869in}{1.298467in}}%
\pgfpathlineto{\pgfqpoint{4.937633in}{1.312147in}}%
\pgfpathlineto{\pgfqpoint{4.945395in}{1.325894in}}%
\pgfpathlineto{\pgfqpoint{4.931211in}{1.323336in}}%
\pgfpathlineto{\pgfqpoint{4.917040in}{1.320886in}}%
\pgfpathlineto{\pgfqpoint{4.902880in}{1.318544in}}%
\pgfpathlineto{\pgfqpoint{4.895120in}{1.305105in}}%
\pgfpathlineto{\pgfqpoint{4.887356in}{1.291737in}}%
\pgfpathlineto{\pgfqpoint{4.879590in}{1.278442in}}%
\pgfpathlineto{\pgfqpoint{4.871820in}{1.265224in}}%
\pgfpathclose%
\pgfusepath{fill}%
\end{pgfscope}%
\begin{pgfscope}%
\pgfpathrectangle{\pgfqpoint{1.254980in}{0.150000in}}{\pgfqpoint{5.490039in}{5.490039in}}%
\pgfusepath{clip}%
\pgfsetbuttcap%
\pgfsetroundjoin%
\definecolor{currentfill}{rgb}{0.272594,0.025563,0.353093}%
\pgfsetfillcolor{currentfill}%
\pgfsetfillopacity{0.700000}%
\pgfsetlinewidth{0.000000pt}%
\definecolor{currentstroke}{rgb}{0.000000,0.000000,0.000000}%
\pgfsetstrokecolor{currentstroke}%
\pgfsetdash{}{0pt}%
\pgfpathmoveto{\pgfqpoint{4.434585in}{1.052764in}}%
\pgfpathlineto{\pgfqpoint{4.448558in}{1.050227in}}%
\pgfpathlineto{\pgfqpoint{4.462539in}{1.047798in}}%
\pgfpathlineto{\pgfqpoint{4.476529in}{1.045477in}}%
\pgfpathlineto{\pgfqpoint{4.490528in}{1.043264in}}%
\pgfpathlineto{\pgfqpoint{4.498384in}{1.052192in}}%
\pgfpathlineto{\pgfqpoint{4.506235in}{1.061288in}}%
\pgfpathlineto{\pgfqpoint{4.514082in}{1.070546in}}%
\pgfpathlineto{\pgfqpoint{4.521924in}{1.079963in}}%
\pgfpathlineto{\pgfqpoint{4.507934in}{1.081684in}}%
\pgfpathlineto{\pgfqpoint{4.493952in}{1.083514in}}%
\pgfpathlineto{\pgfqpoint{4.479979in}{1.085451in}}%
\pgfpathlineto{\pgfqpoint{4.466015in}{1.087497in}}%
\pgfpathlineto{\pgfqpoint{4.458165in}{1.078565in}}%
\pgfpathlineto{\pgfqpoint{4.450309in}{1.069797in}}%
\pgfpathlineto{\pgfqpoint{4.442449in}{1.061195in}}%
\pgfpathlineto{\pgfqpoint{4.434585in}{1.052764in}}%
\pgfpathclose%
\pgfusepath{fill}%
\end{pgfscope}%
\begin{pgfscope}%
\pgfpathrectangle{\pgfqpoint{1.254980in}{0.150000in}}{\pgfqpoint{5.490039in}{5.490039in}}%
\pgfusepath{clip}%
\pgfsetbuttcap%
\pgfsetroundjoin%
\definecolor{currentfill}{rgb}{0.271305,0.019942,0.347269}%
\pgfsetfillcolor{currentfill}%
\pgfsetfillopacity{0.700000}%
\pgfsetlinewidth{0.000000pt}%
\definecolor{currentstroke}{rgb}{0.000000,0.000000,0.000000}%
\pgfsetstrokecolor{currentstroke}%
\pgfsetdash{}{0pt}%
\pgfpathmoveto{\pgfqpoint{4.204118in}{1.043309in}}%
\pgfpathlineto{\pgfqpoint{4.218027in}{1.038430in}}%
\pgfpathlineto{\pgfqpoint{4.231943in}{1.033662in}}%
\pgfpathlineto{\pgfqpoint{4.245865in}{1.029003in}}%
\pgfpathlineto{\pgfqpoint{4.259795in}{1.024454in}}%
\pgfpathlineto{\pgfqpoint{4.267728in}{1.030301in}}%
\pgfpathlineto{\pgfqpoint{4.275655in}{1.036363in}}%
\pgfpathlineto{\pgfqpoint{4.283576in}{1.042633in}}%
\pgfpathlineto{\pgfqpoint{4.291491in}{1.049109in}}%
\pgfpathlineto{\pgfqpoint{4.277577in}{1.053135in}}%
\pgfpathlineto{\pgfqpoint{4.263669in}{1.057269in}}%
\pgfpathlineto{\pgfqpoint{4.249769in}{1.061514in}}%
\pgfpathlineto{\pgfqpoint{4.235876in}{1.065869in}}%
\pgfpathlineto{\pgfqpoint{4.227946in}{1.059911in}}%
\pgfpathlineto{\pgfqpoint{4.220010in}{1.054162in}}%
\pgfpathlineto{\pgfqpoint{4.212067in}{1.048626in}}%
\pgfpathlineto{\pgfqpoint{4.204118in}{1.043309in}}%
\pgfpathclose%
\pgfusepath{fill}%
\end{pgfscope}%
\begin{pgfscope}%
\pgfpathrectangle{\pgfqpoint{1.254980in}{0.150000in}}{\pgfqpoint{5.490039in}{5.490039in}}%
\pgfusepath{clip}%
\pgfsetbuttcap%
\pgfsetroundjoin%
\definecolor{currentfill}{rgb}{0.866013,0.889868,0.095953}%
\pgfsetfillcolor{currentfill}%
\pgfsetfillopacity{0.700000}%
\pgfsetlinewidth{0.000000pt}%
\definecolor{currentstroke}{rgb}{0.000000,0.000000,0.000000}%
\pgfsetstrokecolor{currentstroke}%
\pgfsetdash{}{0pt}%
\pgfpathmoveto{\pgfqpoint{1.947775in}{3.594471in}}%
\pgfpathlineto{\pgfqpoint{1.962228in}{3.564875in}}%
\pgfpathlineto{\pgfqpoint{1.976665in}{3.535509in}}%
\pgfpathlineto{\pgfqpoint{1.991086in}{3.506372in}}%
\pgfpathlineto{\pgfqpoint{2.005491in}{3.477462in}}%
\pgfpathlineto{\pgfqpoint{2.015348in}{3.458891in}}%
\pgfpathlineto{\pgfqpoint{2.025172in}{3.440769in}}%
\pgfpathlineto{\pgfqpoint{2.034962in}{3.423090in}}%
\pgfpathlineto{\pgfqpoint{2.044720in}{3.405850in}}%
\pgfpathlineto{\pgfqpoint{2.030396in}{3.434089in}}%
\pgfpathlineto{\pgfqpoint{2.016058in}{3.462554in}}%
\pgfpathlineto{\pgfqpoint{2.001704in}{3.491245in}}%
\pgfpathlineto{\pgfqpoint{1.987335in}{3.520165in}}%
\pgfpathlineto{\pgfqpoint{1.977496in}{3.538068in}}%
\pgfpathlineto{\pgfqpoint{1.967624in}{3.556416in}}%
\pgfpathlineto{\pgfqpoint{1.957717in}{3.575215in}}%
\pgfpathlineto{\pgfqpoint{1.947775in}{3.594471in}}%
\pgfpathclose%
\pgfusepath{fill}%
\end{pgfscope}%
\begin{pgfscope}%
\pgfpathrectangle{\pgfqpoint{1.254980in}{0.150000in}}{\pgfqpoint{5.490039in}{5.490039in}}%
\pgfusepath{clip}%
\pgfsetbuttcap%
\pgfsetroundjoin%
\definecolor{currentfill}{rgb}{0.273809,0.031497,0.358853}%
\pgfsetfillcolor{currentfill}%
\pgfsetfillopacity{0.700000}%
\pgfsetlinewidth{0.000000pt}%
\definecolor{currentstroke}{rgb}{0.000000,0.000000,0.000000}%
\pgfsetstrokecolor{currentstroke}%
\pgfsetdash{}{0pt}%
\pgfpathmoveto{\pgfqpoint{4.061055in}{1.071660in}}%
\pgfpathlineto{\pgfqpoint{4.074934in}{1.065352in}}%
\pgfpathlineto{\pgfqpoint{4.088820in}{1.059156in}}%
\pgfpathlineto{\pgfqpoint{4.102711in}{1.053071in}}%
\pgfpathlineto{\pgfqpoint{4.116608in}{1.047098in}}%
\pgfpathlineto{\pgfqpoint{4.124603in}{1.050949in}}%
\pgfpathlineto{\pgfqpoint{4.132591in}{1.055041in}}%
\pgfpathlineto{\pgfqpoint{4.140572in}{1.059369in}}%
\pgfpathlineto{\pgfqpoint{4.148546in}{1.063928in}}%
\pgfpathlineto{\pgfqpoint{4.134668in}{1.069360in}}%
\pgfpathlineto{\pgfqpoint{4.120797in}{1.074904in}}%
\pgfpathlineto{\pgfqpoint{4.106932in}{1.080558in}}%
\pgfpathlineto{\pgfqpoint{4.093072in}{1.086325in}}%
\pgfpathlineto{\pgfqpoint{4.085079in}{1.082300in}}%
\pgfpathlineto{\pgfqpoint{4.077079in}{1.078511in}}%
\pgfpathlineto{\pgfqpoint{4.069071in}{1.074963in}}%
\pgfpathlineto{\pgfqpoint{4.061055in}{1.071660in}}%
\pgfpathclose%
\pgfusepath{fill}%
\end{pgfscope}%
\begin{pgfscope}%
\pgfpathrectangle{\pgfqpoint{1.254980in}{0.150000in}}{\pgfqpoint{5.490039in}{5.490039in}}%
\pgfusepath{clip}%
\pgfsetbuttcap%
\pgfsetroundjoin%
\definecolor{currentfill}{rgb}{0.271305,0.019942,0.347269}%
\pgfsetfillcolor{currentfill}%
\pgfsetfillopacity{0.700000}%
\pgfsetlinewidth{0.000000pt}%
\definecolor{currentstroke}{rgb}{0.000000,0.000000,0.000000}%
\pgfsetstrokecolor{currentstroke}%
\pgfsetdash{}{0pt}%
\pgfpathmoveto{\pgfqpoint{4.347223in}{1.034102in}}%
\pgfpathlineto{\pgfqpoint{4.361174in}{1.030623in}}%
\pgfpathlineto{\pgfqpoint{4.375134in}{1.027253in}}%
\pgfpathlineto{\pgfqpoint{4.389101in}{1.023991in}}%
\pgfpathlineto{\pgfqpoint{4.403076in}{1.020837in}}%
\pgfpathlineto{\pgfqpoint{4.410961in}{1.028541in}}%
\pgfpathlineto{\pgfqpoint{4.418840in}{1.036433in}}%
\pgfpathlineto{\pgfqpoint{4.426715in}{1.044509in}}%
\pgfpathlineto{\pgfqpoint{4.434585in}{1.052764in}}%
\pgfpathlineto{\pgfqpoint{4.420620in}{1.055410in}}%
\pgfpathlineto{\pgfqpoint{4.406664in}{1.058164in}}%
\pgfpathlineto{\pgfqpoint{4.392715in}{1.061027in}}%
\pgfpathlineto{\pgfqpoint{4.378775in}{1.063999in}}%
\pgfpathlineto{\pgfqpoint{4.370895in}{1.056245in}}%
\pgfpathlineto{\pgfqpoint{4.363010in}{1.048675in}}%
\pgfpathlineto{\pgfqpoint{4.355119in}{1.041293in}}%
\pgfpathlineto{\pgfqpoint{4.347223in}{1.034102in}}%
\pgfpathclose%
\pgfusepath{fill}%
\end{pgfscope}%
\begin{pgfscope}%
\pgfpathrectangle{\pgfqpoint{1.254980in}{0.150000in}}{\pgfqpoint{5.490039in}{5.490039in}}%
\pgfusepath{clip}%
\pgfsetbuttcap%
\pgfsetroundjoin%
\definecolor{currentfill}{rgb}{0.280894,0.078907,0.402329}%
\pgfsetfillcolor{currentfill}%
\pgfsetfillopacity{0.700000}%
\pgfsetlinewidth{0.000000pt}%
\definecolor{currentstroke}{rgb}{0.000000,0.000000,0.000000}%
\pgfsetstrokecolor{currentstroke}%
\pgfsetdash{}{0pt}%
\pgfpathmoveto{\pgfqpoint{4.665469in}{1.113066in}}%
\pgfpathlineto{\pgfqpoint{4.679541in}{1.112791in}}%
\pgfpathlineto{\pgfqpoint{4.693624in}{1.112623in}}%
\pgfpathlineto{\pgfqpoint{4.707716in}{1.112563in}}%
\pgfpathlineto{\pgfqpoint{4.721820in}{1.112609in}}%
\pgfpathlineto{\pgfqpoint{4.729628in}{1.124163in}}%
\pgfpathlineto{\pgfqpoint{4.737434in}{1.135840in}}%
\pgfpathlineto{\pgfqpoint{4.745236in}{1.147637in}}%
\pgfpathlineto{\pgfqpoint{4.753035in}{1.159551in}}%
\pgfpathlineto{\pgfqpoint{4.738934in}{1.159042in}}%
\pgfpathlineto{\pgfqpoint{4.724843in}{1.158640in}}%
\pgfpathlineto{\pgfqpoint{4.710764in}{1.158346in}}%
\pgfpathlineto{\pgfqpoint{4.696695in}{1.158160in}}%
\pgfpathlineto{\pgfqpoint{4.688894in}{1.146702in}}%
\pgfpathlineto{\pgfqpoint{4.681089in}{1.135365in}}%
\pgfpathlineto{\pgfqpoint{4.673281in}{1.124152in}}%
\pgfpathlineto{\pgfqpoint{4.665469in}{1.113066in}}%
\pgfpathclose%
\pgfusepath{fill}%
\end{pgfscope}%
\begin{pgfscope}%
\pgfpathrectangle{\pgfqpoint{1.254980in}{0.150000in}}{\pgfqpoint{5.490039in}{5.490039in}}%
\pgfusepath{clip}%
\pgfsetbuttcap%
\pgfsetroundjoin%
\definecolor{currentfill}{rgb}{0.282910,0.105393,0.426902}%
\pgfsetfillcolor{currentfill}%
\pgfsetfillopacity{0.700000}%
\pgfsetlinewidth{0.000000pt}%
\definecolor{currentstroke}{rgb}{0.000000,0.000000,0.000000}%
\pgfsetstrokecolor{currentstroke}%
\pgfsetdash{}{0pt}%
\pgfpathmoveto{\pgfqpoint{4.753035in}{1.159551in}}%
\pgfpathlineto{\pgfqpoint{4.767147in}{1.160167in}}%
\pgfpathlineto{\pgfqpoint{4.781270in}{1.160891in}}%
\pgfpathlineto{\pgfqpoint{4.795404in}{1.161722in}}%
\pgfpathlineto{\pgfqpoint{4.809549in}{1.162660in}}%
\pgfpathlineto{\pgfqpoint{4.817344in}{1.175139in}}%
\pgfpathlineto{\pgfqpoint{4.825136in}{1.187722in}}%
\pgfpathlineto{\pgfqpoint{4.832924in}{1.200407in}}%
\pgfpathlineto{\pgfqpoint{4.840710in}{1.213189in}}%
\pgfpathlineto{\pgfqpoint{4.826564in}{1.211803in}}%
\pgfpathlineto{\pgfqpoint{4.812431in}{1.210525in}}%
\pgfpathlineto{\pgfqpoint{4.798308in}{1.209354in}}%
\pgfpathlineto{\pgfqpoint{4.784197in}{1.208291in}}%
\pgfpathlineto{\pgfqpoint{4.776412in}{1.195950in}}%
\pgfpathlineto{\pgfqpoint{4.768623in}{1.183710in}}%
\pgfpathlineto{\pgfqpoint{4.760830in}{1.171576in}}%
\pgfpathlineto{\pgfqpoint{4.753035in}{1.159551in}}%
\pgfpathclose%
\pgfusepath{fill}%
\end{pgfscope}%
\begin{pgfscope}%
\pgfpathrectangle{\pgfqpoint{1.254980in}{0.150000in}}{\pgfqpoint{5.490039in}{5.490039in}}%
\pgfusepath{clip}%
\pgfsetbuttcap%
\pgfsetroundjoin%
\definecolor{currentfill}{rgb}{0.277941,0.056324,0.381191}%
\pgfsetfillcolor{currentfill}%
\pgfsetfillopacity{0.700000}%
\pgfsetlinewidth{0.000000pt}%
\definecolor{currentstroke}{rgb}{0.000000,0.000000,0.000000}%
\pgfsetstrokecolor{currentstroke}%
\pgfsetdash{}{0pt}%
\pgfpathmoveto{\pgfqpoint{4.577979in}{1.074158in}}%
\pgfpathlineto{\pgfqpoint{4.592016in}{1.072976in}}%
\pgfpathlineto{\pgfqpoint{4.606063in}{1.071901in}}%
\pgfpathlineto{\pgfqpoint{4.620119in}{1.070934in}}%
\pgfpathlineto{\pgfqpoint{4.634185in}{1.070074in}}%
\pgfpathlineto{\pgfqpoint{4.642012in}{1.080612in}}%
\pgfpathlineto{\pgfqpoint{4.649835in}{1.091292in}}%
\pgfpathlineto{\pgfqpoint{4.657654in}{1.102112in}}%
\pgfpathlineto{\pgfqpoint{4.665469in}{1.113066in}}%
\pgfpathlineto{\pgfqpoint{4.651407in}{1.113449in}}%
\pgfpathlineto{\pgfqpoint{4.637356in}{1.113939in}}%
\pgfpathlineto{\pgfqpoint{4.623314in}{1.114537in}}%
\pgfpathlineto{\pgfqpoint{4.609282in}{1.115242in}}%
\pgfpathlineto{\pgfqpoint{4.601462in}{1.104758in}}%
\pgfpathlineto{\pgfqpoint{4.593638in}{1.094414in}}%
\pgfpathlineto{\pgfqpoint{4.585811in}{1.084213in}}%
\pgfpathlineto{\pgfqpoint{4.577979in}{1.074158in}}%
\pgfpathclose%
\pgfusepath{fill}%
\end{pgfscope}%
\begin{pgfscope}%
\pgfpathrectangle{\pgfqpoint{1.254980in}{0.150000in}}{\pgfqpoint{5.490039in}{5.490039in}}%
\pgfusepath{clip}%
\pgfsetbuttcap%
\pgfsetroundjoin%
\definecolor{currentfill}{rgb}{0.283072,0.130895,0.449241}%
\pgfsetfillcolor{currentfill}%
\pgfsetfillopacity{0.700000}%
\pgfsetlinewidth{0.000000pt}%
\definecolor{currentstroke}{rgb}{0.000000,0.000000,0.000000}%
\pgfsetstrokecolor{currentstroke}%
\pgfsetdash{}{0pt}%
\pgfpathmoveto{\pgfqpoint{4.840710in}{1.213189in}}%
\pgfpathlineto{\pgfqpoint{4.854866in}{1.214682in}}%
\pgfpathlineto{\pgfqpoint{4.869035in}{1.216283in}}%
\pgfpathlineto{\pgfqpoint{4.883215in}{1.217991in}}%
\pgfpathlineto{\pgfqpoint{4.890998in}{1.231196in}}%
\pgfpathlineto{\pgfqpoint{4.898778in}{1.244489in}}%
\pgfpathlineto{\pgfqpoint{4.906556in}{1.257866in}}%
\pgfpathlineto{\pgfqpoint{4.914330in}{1.271323in}}%
\pgfpathlineto{\pgfqpoint{4.900148in}{1.269182in}}%
\pgfpathlineto{\pgfqpoint{4.885978in}{1.267149in}}%
\pgfpathlineto{\pgfqpoint{4.871820in}{1.265224in}}%
\pgfpathlineto{\pgfqpoint{4.864047in}{1.252086in}}%
\pgfpathlineto{\pgfqpoint{4.856271in}{1.239032in}}%
\pgfpathlineto{\pgfqpoint{4.848492in}{1.226065in}}%
\pgfpathlineto{\pgfqpoint{4.840710in}{1.213189in}}%
\pgfpathclose%
\pgfusepath{fill}%
\end{pgfscope}%
\begin{pgfscope}%
\pgfpathrectangle{\pgfqpoint{1.254980in}{0.150000in}}{\pgfqpoint{5.490039in}{5.490039in}}%
\pgfusepath{clip}%
\pgfsetbuttcap%
\pgfsetroundjoin%
\definecolor{currentfill}{rgb}{0.272594,0.025563,0.353093}%
\pgfsetfillcolor{currentfill}%
\pgfsetfillopacity{0.700000}%
\pgfsetlinewidth{0.000000pt}%
\definecolor{currentstroke}{rgb}{0.000000,0.000000,0.000000}%
\pgfsetstrokecolor{currentstroke}%
\pgfsetdash{}{0pt}%
\pgfpathmoveto{\pgfqpoint{4.116608in}{1.047098in}}%
\pgfpathlineto{\pgfqpoint{4.130510in}{1.041236in}}%
\pgfpathlineto{\pgfqpoint{4.144419in}{1.035485in}}%
\pgfpathlineto{\pgfqpoint{4.158333in}{1.029844in}}%
\pgfpathlineto{\pgfqpoint{4.172254in}{1.024314in}}%
\pgfpathlineto{\pgfqpoint{4.180230in}{1.028712in}}%
\pgfpathlineto{\pgfqpoint{4.188200in}{1.033347in}}%
\pgfpathlineto{\pgfqpoint{4.196162in}{1.038215in}}%
\pgfpathlineto{\pgfqpoint{4.204118in}{1.043309in}}%
\pgfpathlineto{\pgfqpoint{4.190215in}{1.048298in}}%
\pgfpathlineto{\pgfqpoint{4.176319in}{1.053397in}}%
\pgfpathlineto{\pgfqpoint{4.162429in}{1.058607in}}%
\pgfpathlineto{\pgfqpoint{4.148546in}{1.063928in}}%
\pgfpathlineto{\pgfqpoint{4.140572in}{1.059369in}}%
\pgfpathlineto{\pgfqpoint{4.132591in}{1.055041in}}%
\pgfpathlineto{\pgfqpoint{4.124603in}{1.050949in}}%
\pgfpathlineto{\pgfqpoint{4.116608in}{1.047098in}}%
\pgfpathclose%
\pgfusepath{fill}%
\end{pgfscope}%
\begin{pgfscope}%
\pgfpathrectangle{\pgfqpoint{1.254980in}{0.150000in}}{\pgfqpoint{5.490039in}{5.490039in}}%
\pgfusepath{clip}%
\pgfsetbuttcap%
\pgfsetroundjoin%
\definecolor{currentfill}{rgb}{0.274952,0.037752,0.364543}%
\pgfsetfillcolor{currentfill}%
\pgfsetfillopacity{0.700000}%
\pgfsetlinewidth{0.000000pt}%
\definecolor{currentstroke}{rgb}{0.000000,0.000000,0.000000}%
\pgfsetstrokecolor{currentstroke}%
\pgfsetdash{}{0pt}%
\pgfpathmoveto{\pgfqpoint{4.490528in}{1.043264in}}%
\pgfpathlineto{\pgfqpoint{4.504535in}{1.041159in}}%
\pgfpathlineto{\pgfqpoint{4.518552in}{1.039161in}}%
\pgfpathlineto{\pgfqpoint{4.532577in}{1.037272in}}%
\pgfpathlineto{\pgfqpoint{4.546611in}{1.035489in}}%
\pgfpathlineto{\pgfqpoint{4.554460in}{1.044916in}}%
\pgfpathlineto{\pgfqpoint{4.562303in}{1.054506in}}%
\pgfpathlineto{\pgfqpoint{4.570143in}{1.064254in}}%
\pgfpathlineto{\pgfqpoint{4.577979in}{1.074158in}}%
\pgfpathlineto{\pgfqpoint{4.563951in}{1.075448in}}%
\pgfpathlineto{\pgfqpoint{4.549933in}{1.076845in}}%
\pgfpathlineto{\pgfqpoint{4.535924in}{1.078350in}}%
\pgfpathlineto{\pgfqpoint{4.521924in}{1.079963in}}%
\pgfpathlineto{\pgfqpoint{4.514082in}{1.070546in}}%
\pgfpathlineto{\pgfqpoint{4.506235in}{1.061288in}}%
\pgfpathlineto{\pgfqpoint{4.498384in}{1.052192in}}%
\pgfpathlineto{\pgfqpoint{4.490528in}{1.043264in}}%
\pgfpathclose%
\pgfusepath{fill}%
\end{pgfscope}%
\begin{pgfscope}%
\pgfpathrectangle{\pgfqpoint{1.254980in}{0.150000in}}{\pgfqpoint{5.490039in}{5.490039in}}%
\pgfusepath{clip}%
\pgfsetbuttcap%
\pgfsetroundjoin%
\definecolor{currentfill}{rgb}{0.271305,0.019942,0.347269}%
\pgfsetfillcolor{currentfill}%
\pgfsetfillopacity{0.700000}%
\pgfsetlinewidth{0.000000pt}%
\definecolor{currentstroke}{rgb}{0.000000,0.000000,0.000000}%
\pgfsetstrokecolor{currentstroke}%
\pgfsetdash{}{0pt}%
\pgfpathmoveto{\pgfqpoint{4.259795in}{1.024454in}}%
\pgfpathlineto{\pgfqpoint{4.273731in}{1.020014in}}%
\pgfpathlineto{\pgfqpoint{4.287674in}{1.015684in}}%
\pgfpathlineto{\pgfqpoint{4.301624in}{1.011463in}}%
\pgfpathlineto{\pgfqpoint{4.315582in}{1.007351in}}%
\pgfpathlineto{\pgfqpoint{4.323501in}{1.013728in}}%
\pgfpathlineto{\pgfqpoint{4.331414in}{1.020316in}}%
\pgfpathlineto{\pgfqpoint{4.339321in}{1.027109in}}%
\pgfpathlineto{\pgfqpoint{4.347223in}{1.034102in}}%
\pgfpathlineto{\pgfqpoint{4.333278in}{1.037690in}}%
\pgfpathlineto{\pgfqpoint{4.319342in}{1.041388in}}%
\pgfpathlineto{\pgfqpoint{4.305413in}{1.045194in}}%
\pgfpathlineto{\pgfqpoint{4.291491in}{1.049109in}}%
\pgfpathlineto{\pgfqpoint{4.283576in}{1.042633in}}%
\pgfpathlineto{\pgfqpoint{4.275655in}{1.036363in}}%
\pgfpathlineto{\pgfqpoint{4.267728in}{1.030301in}}%
\pgfpathlineto{\pgfqpoint{4.259795in}{1.024454in}}%
\pgfpathclose%
\pgfusepath{fill}%
\end{pgfscope}%
\begin{pgfscope}%
\pgfpathrectangle{\pgfqpoint{1.254980in}{0.150000in}}{\pgfqpoint{5.490039in}{5.490039in}}%
\pgfusepath{clip}%
\pgfsetbuttcap%
\pgfsetroundjoin%
\definecolor{currentfill}{rgb}{0.993248,0.906157,0.143936}%
\pgfsetfillcolor{currentfill}%
\pgfsetfillopacity{0.700000}%
\pgfsetlinewidth{0.000000pt}%
\definecolor{currentstroke}{rgb}{0.000000,0.000000,0.000000}%
\pgfsetstrokecolor{currentstroke}%
\pgfsetdash{}{0pt}%
\pgfpathmoveto{\pgfqpoint{1.889796in}{3.715194in}}%
\pgfpathlineto{\pgfqpoint{1.904316in}{3.684658in}}%
\pgfpathlineto{\pgfqpoint{1.918819in}{3.654361in}}%
\pgfpathlineto{\pgfqpoint{1.933305in}{3.624299in}}%
\pgfpathlineto{\pgfqpoint{1.947775in}{3.594471in}}%
\pgfpathlineto{\pgfqpoint{1.957717in}{3.575215in}}%
\pgfpathlineto{\pgfqpoint{1.967624in}{3.556416in}}%
\pgfpathlineto{\pgfqpoint{1.977496in}{3.538068in}}%
\pgfpathlineto{\pgfqpoint{1.987335in}{3.520165in}}%
\pgfpathlineto{\pgfqpoint{1.972950in}{3.549314in}}%
\pgfpathlineto{\pgfqpoint{1.958548in}{3.578695in}}%
\pgfpathlineto{\pgfqpoint{1.944131in}{3.608310in}}%
\pgfpathlineto{\pgfqpoint{1.929696in}{3.638161in}}%
\pgfpathlineto{\pgfqpoint{1.919774in}{3.656735in}}%
\pgfpathlineto{\pgfqpoint{1.909817in}{3.675761in}}%
\pgfpathlineto{\pgfqpoint{1.899824in}{3.695245in}}%
\pgfpathlineto{\pgfqpoint{1.889796in}{3.715194in}}%
\pgfpathclose%
\pgfusepath{fill}%
\end{pgfscope}%
\begin{pgfscope}%
\pgfpathrectangle{\pgfqpoint{1.254980in}{0.150000in}}{\pgfqpoint{5.490039in}{5.490039in}}%
\pgfusepath{clip}%
\pgfsetbuttcap%
\pgfsetroundjoin%
\definecolor{currentfill}{rgb}{0.272594,0.025563,0.353093}%
\pgfsetfillcolor{currentfill}%
\pgfsetfillopacity{0.700000}%
\pgfsetlinewidth{0.000000pt}%
\definecolor{currentstroke}{rgb}{0.000000,0.000000,0.000000}%
\pgfsetstrokecolor{currentstroke}%
\pgfsetdash{}{0pt}%
\pgfpathmoveto{\pgfqpoint{4.403076in}{1.020837in}}%
\pgfpathlineto{\pgfqpoint{4.417060in}{1.017792in}}%
\pgfpathlineto{\pgfqpoint{4.431051in}{1.014855in}}%
\pgfpathlineto{\pgfqpoint{4.445051in}{1.012026in}}%
\pgfpathlineto{\pgfqpoint{4.459059in}{1.009305in}}%
\pgfpathlineto{\pgfqpoint{4.466933in}{1.017523in}}%
\pgfpathlineto{\pgfqpoint{4.474803in}{1.025925in}}%
\pgfpathlineto{\pgfqpoint{4.482668in}{1.034507in}}%
\pgfpathlineto{\pgfqpoint{4.490528in}{1.043264in}}%
\pgfpathlineto{\pgfqpoint{4.476529in}{1.045477in}}%
\pgfpathlineto{\pgfqpoint{4.462539in}{1.047798in}}%
\pgfpathlineto{\pgfqpoint{4.448558in}{1.050227in}}%
\pgfpathlineto{\pgfqpoint{4.434585in}{1.052764in}}%
\pgfpathlineto{\pgfqpoint{4.426715in}{1.044509in}}%
\pgfpathlineto{\pgfqpoint{4.418840in}{1.036433in}}%
\pgfpathlineto{\pgfqpoint{4.410961in}{1.028541in}}%
\pgfpathlineto{\pgfqpoint{4.403076in}{1.020837in}}%
\pgfpathclose%
\pgfusepath{fill}%
\end{pgfscope}%
\begin{pgfscope}%
\pgfpathrectangle{\pgfqpoint{1.254980in}{0.150000in}}{\pgfqpoint{5.490039in}{5.490039in}}%
\pgfusepath{clip}%
\pgfsetbuttcap%
\pgfsetroundjoin%
\definecolor{currentfill}{rgb}{0.272594,0.025563,0.353093}%
\pgfsetfillcolor{currentfill}%
\pgfsetfillopacity{0.700000}%
\pgfsetlinewidth{0.000000pt}%
\definecolor{currentstroke}{rgb}{0.000000,0.000000,0.000000}%
\pgfsetstrokecolor{currentstroke}%
\pgfsetdash{}{0pt}%
\pgfpathmoveto{\pgfqpoint{4.172254in}{1.024314in}}%
\pgfpathlineto{\pgfqpoint{4.186180in}{1.018894in}}%
\pgfpathlineto{\pgfqpoint{4.200113in}{1.013585in}}%
\pgfpathlineto{\pgfqpoint{4.214053in}{1.008385in}}%
\pgfpathlineto{\pgfqpoint{4.227998in}{1.003295in}}%
\pgfpathlineto{\pgfqpoint{4.235957in}{1.008241in}}%
\pgfpathlineto{\pgfqpoint{4.243909in}{1.013419in}}%
\pgfpathlineto{\pgfqpoint{4.251855in}{1.018825in}}%
\pgfpathlineto{\pgfqpoint{4.259795in}{1.024454in}}%
\pgfpathlineto{\pgfqpoint{4.245865in}{1.029003in}}%
\pgfpathlineto{\pgfqpoint{4.231943in}{1.033662in}}%
\pgfpathlineto{\pgfqpoint{4.218027in}{1.038430in}}%
\pgfpathlineto{\pgfqpoint{4.204118in}{1.043309in}}%
\pgfpathlineto{\pgfqpoint{4.196162in}{1.038215in}}%
\pgfpathlineto{\pgfqpoint{4.188200in}{1.033347in}}%
\pgfpathlineto{\pgfqpoint{4.180230in}{1.028712in}}%
\pgfpathlineto{\pgfqpoint{4.172254in}{1.024314in}}%
\pgfpathclose%
\pgfusepath{fill}%
\end{pgfscope}%
\begin{pgfscope}%
\pgfpathrectangle{\pgfqpoint{1.254980in}{0.150000in}}{\pgfqpoint{5.490039in}{5.490039in}}%
\pgfusepath{clip}%
\pgfsetbuttcap%
\pgfsetroundjoin%
\definecolor{currentfill}{rgb}{0.283197,0.115680,0.436115}%
\pgfsetfillcolor{currentfill}%
\pgfsetfillopacity{0.700000}%
\pgfsetlinewidth{0.000000pt}%
\definecolor{currentstroke}{rgb}{0.000000,0.000000,0.000000}%
\pgfsetstrokecolor{currentstroke}%
\pgfsetdash{}{0pt}%
\pgfpathmoveto{\pgfqpoint{4.809549in}{1.162660in}}%
\pgfpathlineto{\pgfqpoint{4.823706in}{1.163705in}}%
\pgfpathlineto{\pgfqpoint{4.837874in}{1.164858in}}%
\pgfpathlineto{\pgfqpoint{4.852054in}{1.166117in}}%
\pgfpathlineto{\pgfqpoint{4.859849in}{1.178937in}}%
\pgfpathlineto{\pgfqpoint{4.867640in}{1.191858in}}%
\pgfpathlineto{\pgfqpoint{4.875429in}{1.204877in}}%
\pgfpathlineto{\pgfqpoint{4.883215in}{1.217991in}}%
\pgfpathlineto{\pgfqpoint{4.869035in}{1.216283in}}%
\pgfpathlineto{\pgfqpoint{4.854866in}{1.214682in}}%
\pgfpathlineto{\pgfqpoint{4.840710in}{1.213189in}}%
\pgfpathlineto{\pgfqpoint{4.832924in}{1.200407in}}%
\pgfpathlineto{\pgfqpoint{4.825136in}{1.187722in}}%
\pgfpathlineto{\pgfqpoint{4.817344in}{1.175139in}}%
\pgfpathlineto{\pgfqpoint{4.809549in}{1.162660in}}%
\pgfpathclose%
\pgfusepath{fill}%
\end{pgfscope}%
\begin{pgfscope}%
\pgfpathrectangle{\pgfqpoint{1.254980in}{0.150000in}}{\pgfqpoint{5.490039in}{5.490039in}}%
\pgfusepath{clip}%
\pgfsetbuttcap%
\pgfsetroundjoin%
\definecolor{currentfill}{rgb}{0.281924,0.089666,0.412415}%
\pgfsetfillcolor{currentfill}%
\pgfsetfillopacity{0.700000}%
\pgfsetlinewidth{0.000000pt}%
\definecolor{currentstroke}{rgb}{0.000000,0.000000,0.000000}%
\pgfsetstrokecolor{currentstroke}%
\pgfsetdash{}{0pt}%
\pgfpathmoveto{\pgfqpoint{4.721820in}{1.112609in}}%
\pgfpathlineto{\pgfqpoint{4.735933in}{1.112763in}}%
\pgfpathlineto{\pgfqpoint{4.750058in}{1.113024in}}%
\pgfpathlineto{\pgfqpoint{4.764193in}{1.113392in}}%
\pgfpathlineto{\pgfqpoint{4.778339in}{1.113866in}}%
\pgfpathlineto{\pgfqpoint{4.786147in}{1.125889in}}%
\pgfpathlineto{\pgfqpoint{4.793951in}{1.138031in}}%
\pgfpathlineto{\pgfqpoint{4.801752in}{1.150290in}}%
\pgfpathlineto{\pgfqpoint{4.809549in}{1.162660in}}%
\pgfpathlineto{\pgfqpoint{4.795404in}{1.161722in}}%
\pgfpathlineto{\pgfqpoint{4.781270in}{1.160891in}}%
\pgfpathlineto{\pgfqpoint{4.767147in}{1.160167in}}%
\pgfpathlineto{\pgfqpoint{4.753035in}{1.159551in}}%
\pgfpathlineto{\pgfqpoint{4.745236in}{1.147637in}}%
\pgfpathlineto{\pgfqpoint{4.737434in}{1.135840in}}%
\pgfpathlineto{\pgfqpoint{4.729628in}{1.124163in}}%
\pgfpathlineto{\pgfqpoint{4.721820in}{1.112609in}}%
\pgfpathclose%
\pgfusepath{fill}%
\end{pgfscope}%
\begin{pgfscope}%
\pgfpathrectangle{\pgfqpoint{1.254980in}{0.150000in}}{\pgfqpoint{5.490039in}{5.490039in}}%
\pgfusepath{clip}%
\pgfsetbuttcap%
\pgfsetroundjoin%
\definecolor{currentfill}{rgb}{0.271305,0.019942,0.347269}%
\pgfsetfillcolor{currentfill}%
\pgfsetfillopacity{0.700000}%
\pgfsetlinewidth{0.000000pt}%
\definecolor{currentstroke}{rgb}{0.000000,0.000000,0.000000}%
\pgfsetstrokecolor{currentstroke}%
\pgfsetdash{}{0pt}%
\pgfpathmoveto{\pgfqpoint{4.315582in}{1.007351in}}%
\pgfpathlineto{\pgfqpoint{4.329547in}{1.003347in}}%
\pgfpathlineto{\pgfqpoint{4.343519in}{0.999453in}}%
\pgfpathlineto{\pgfqpoint{4.357499in}{0.995666in}}%
\pgfpathlineto{\pgfqpoint{4.371486in}{0.991989in}}%
\pgfpathlineto{\pgfqpoint{4.379392in}{0.998897in}}%
\pgfpathlineto{\pgfqpoint{4.387292in}{1.006011in}}%
\pgfpathlineto{\pgfqpoint{4.395187in}{1.013326in}}%
\pgfpathlineto{\pgfqpoint{4.403076in}{1.020837in}}%
\pgfpathlineto{\pgfqpoint{4.389101in}{1.023991in}}%
\pgfpathlineto{\pgfqpoint{4.375134in}{1.027253in}}%
\pgfpathlineto{\pgfqpoint{4.361174in}{1.030623in}}%
\pgfpathlineto{\pgfqpoint{4.347223in}{1.034102in}}%
\pgfpathlineto{\pgfqpoint{4.339321in}{1.027109in}}%
\pgfpathlineto{\pgfqpoint{4.331414in}{1.020316in}}%
\pgfpathlineto{\pgfqpoint{4.323501in}{1.013728in}}%
\pgfpathlineto{\pgfqpoint{4.315582in}{1.007351in}}%
\pgfpathclose%
\pgfusepath{fill}%
\end{pgfscope}%
\begin{pgfscope}%
\pgfpathrectangle{\pgfqpoint{1.254980in}{0.150000in}}{\pgfqpoint{5.490039in}{5.490039in}}%
\pgfusepath{clip}%
\pgfsetbuttcap%
\pgfsetroundjoin%
\definecolor{currentfill}{rgb}{0.279566,0.067836,0.391917}%
\pgfsetfillcolor{currentfill}%
\pgfsetfillopacity{0.700000}%
\pgfsetlinewidth{0.000000pt}%
\definecolor{currentstroke}{rgb}{0.000000,0.000000,0.000000}%
\pgfsetstrokecolor{currentstroke}%
\pgfsetdash{}{0pt}%
\pgfpathmoveto{\pgfqpoint{4.634185in}{1.070074in}}%
\pgfpathlineto{\pgfqpoint{4.648261in}{1.069322in}}%
\pgfpathlineto{\pgfqpoint{4.662348in}{1.068676in}}%
\pgfpathlineto{\pgfqpoint{4.676444in}{1.068137in}}%
\pgfpathlineto{\pgfqpoint{4.690550in}{1.067706in}}%
\pgfpathlineto{\pgfqpoint{4.698373in}{1.078727in}}%
\pgfpathlineto{\pgfqpoint{4.706192in}{1.089887in}}%
\pgfpathlineto{\pgfqpoint{4.714007in}{1.101183in}}%
\pgfpathlineto{\pgfqpoint{4.721820in}{1.112609in}}%
\pgfpathlineto{\pgfqpoint{4.707716in}{1.112563in}}%
\pgfpathlineto{\pgfqpoint{4.693624in}{1.112623in}}%
\pgfpathlineto{\pgfqpoint{4.679541in}{1.112791in}}%
\pgfpathlineto{\pgfqpoint{4.665469in}{1.113066in}}%
\pgfpathlineto{\pgfqpoint{4.657654in}{1.102112in}}%
\pgfpathlineto{\pgfqpoint{4.649835in}{1.091292in}}%
\pgfpathlineto{\pgfqpoint{4.642012in}{1.080612in}}%
\pgfpathlineto{\pgfqpoint{4.634185in}{1.070074in}}%
\pgfpathclose%
\pgfusepath{fill}%
\end{pgfscope}%
\begin{pgfscope}%
\pgfpathrectangle{\pgfqpoint{1.254980in}{0.150000in}}{\pgfqpoint{5.490039in}{5.490039in}}%
\pgfusepath{clip}%
\pgfsetbuttcap%
\pgfsetroundjoin%
\definecolor{currentfill}{rgb}{0.276022,0.044167,0.370164}%
\pgfsetfillcolor{currentfill}%
\pgfsetfillopacity{0.700000}%
\pgfsetlinewidth{0.000000pt}%
\definecolor{currentstroke}{rgb}{0.000000,0.000000,0.000000}%
\pgfsetstrokecolor{currentstroke}%
\pgfsetdash{}{0pt}%
\pgfpathmoveto{\pgfqpoint{4.546611in}{1.035489in}}%
\pgfpathlineto{\pgfqpoint{4.560655in}{1.033815in}}%
\pgfpathlineto{\pgfqpoint{4.574708in}{1.032247in}}%
\pgfpathlineto{\pgfqpoint{4.588770in}{1.030787in}}%
\pgfpathlineto{\pgfqpoint{4.602842in}{1.029433in}}%
\pgfpathlineto{\pgfqpoint{4.610683in}{1.039359in}}%
\pgfpathlineto{\pgfqpoint{4.618521in}{1.049444in}}%
\pgfpathlineto{\pgfqpoint{4.626355in}{1.059684in}}%
\pgfpathlineto{\pgfqpoint{4.634185in}{1.070074in}}%
\pgfpathlineto{\pgfqpoint{4.620119in}{1.070934in}}%
\pgfpathlineto{\pgfqpoint{4.606063in}{1.071901in}}%
\pgfpathlineto{\pgfqpoint{4.592016in}{1.072976in}}%
\pgfpathlineto{\pgfqpoint{4.577979in}{1.074158in}}%
\pgfpathlineto{\pgfqpoint{4.570143in}{1.064254in}}%
\pgfpathlineto{\pgfqpoint{4.562303in}{1.054506in}}%
\pgfpathlineto{\pgfqpoint{4.554460in}{1.044916in}}%
\pgfpathlineto{\pgfqpoint{4.546611in}{1.035489in}}%
\pgfpathclose%
\pgfusepath{fill}%
\end{pgfscope}%
\begin{pgfscope}%
\pgfpathrectangle{\pgfqpoint{1.254980in}{0.150000in}}{\pgfqpoint{5.490039in}{5.490039in}}%
\pgfusepath{clip}%
\pgfsetbuttcap%
\pgfsetroundjoin%
\definecolor{currentfill}{rgb}{0.273809,0.031497,0.358853}%
\pgfsetfillcolor{currentfill}%
\pgfsetfillopacity{0.700000}%
\pgfsetlinewidth{0.000000pt}%
\definecolor{currentstroke}{rgb}{0.000000,0.000000,0.000000}%
\pgfsetstrokecolor{currentstroke}%
\pgfsetdash{}{0pt}%
\pgfpathmoveto{\pgfqpoint{4.459059in}{1.009305in}}%
\pgfpathlineto{\pgfqpoint{4.473076in}{1.006692in}}%
\pgfpathlineto{\pgfqpoint{4.487101in}{1.004186in}}%
\pgfpathlineto{\pgfqpoint{4.501134in}{1.001787in}}%
\pgfpathlineto{\pgfqpoint{4.515177in}{0.999496in}}%
\pgfpathlineto{\pgfqpoint{4.523042in}{1.008229in}}%
\pgfpathlineto{\pgfqpoint{4.530903in}{1.017142in}}%
\pgfpathlineto{\pgfqpoint{4.538759in}{1.026230in}}%
\pgfpathlineto{\pgfqpoint{4.546611in}{1.035489in}}%
\pgfpathlineto{\pgfqpoint{4.532577in}{1.037272in}}%
\pgfpathlineto{\pgfqpoint{4.518552in}{1.039161in}}%
\pgfpathlineto{\pgfqpoint{4.504535in}{1.041159in}}%
\pgfpathlineto{\pgfqpoint{4.490528in}{1.043264in}}%
\pgfpathlineto{\pgfqpoint{4.482668in}{1.034507in}}%
\pgfpathlineto{\pgfqpoint{4.474803in}{1.025925in}}%
\pgfpathlineto{\pgfqpoint{4.466933in}{1.017523in}}%
\pgfpathlineto{\pgfqpoint{4.459059in}{1.009305in}}%
\pgfpathclose%
\pgfusepath{fill}%
\end{pgfscope}%
\begin{pgfscope}%
\pgfpathrectangle{\pgfqpoint{1.254980in}{0.150000in}}{\pgfqpoint{5.490039in}{5.490039in}}%
\pgfusepath{clip}%
\pgfsetbuttcap%
\pgfsetroundjoin%
\definecolor{currentfill}{rgb}{0.272594,0.025563,0.353093}%
\pgfsetfillcolor{currentfill}%
\pgfsetfillopacity{0.700000}%
\pgfsetlinewidth{0.000000pt}%
\definecolor{currentstroke}{rgb}{0.000000,0.000000,0.000000}%
\pgfsetstrokecolor{currentstroke}%
\pgfsetdash{}{0pt}%
\pgfpathmoveto{\pgfqpoint{4.227998in}{1.003295in}}%
\pgfpathlineto{\pgfqpoint{4.241951in}{0.998314in}}%
\pgfpathlineto{\pgfqpoint{4.255910in}{0.993443in}}%
\pgfpathlineto{\pgfqpoint{4.269876in}{0.988681in}}%
\pgfpathlineto{\pgfqpoint{4.283848in}{0.984027in}}%
\pgfpathlineto{\pgfqpoint{4.291791in}{0.989521in}}%
\pgfpathlineto{\pgfqpoint{4.299727in}{0.995242in}}%
\pgfpathlineto{\pgfqpoint{4.307658in}{1.001187in}}%
\pgfpathlineto{\pgfqpoint{4.315582in}{1.007351in}}%
\pgfpathlineto{\pgfqpoint{4.301624in}{1.011463in}}%
\pgfpathlineto{\pgfqpoint{4.287674in}{1.015684in}}%
\pgfpathlineto{\pgfqpoint{4.273731in}{1.020014in}}%
\pgfpathlineto{\pgfqpoint{4.259795in}{1.024454in}}%
\pgfpathlineto{\pgfqpoint{4.251855in}{1.018825in}}%
\pgfpathlineto{\pgfqpoint{4.243909in}{1.013419in}}%
\pgfpathlineto{\pgfqpoint{4.235957in}{1.008241in}}%
\pgfpathlineto{\pgfqpoint{4.227998in}{1.003295in}}%
\pgfpathclose%
\pgfusepath{fill}%
\end{pgfscope}%
\begin{pgfscope}%
\pgfpathrectangle{\pgfqpoint{1.254980in}{0.150000in}}{\pgfqpoint{5.490039in}{5.490039in}}%
\pgfusepath{clip}%
\pgfsetbuttcap%
\pgfsetroundjoin%
\definecolor{currentfill}{rgb}{0.272594,0.025563,0.353093}%
\pgfsetfillcolor{currentfill}%
\pgfsetfillopacity{0.700000}%
\pgfsetlinewidth{0.000000pt}%
\definecolor{currentstroke}{rgb}{0.000000,0.000000,0.000000}%
\pgfsetstrokecolor{currentstroke}%
\pgfsetdash{}{0pt}%
\pgfpathmoveto{\pgfqpoint{4.371486in}{0.991989in}}%
\pgfpathlineto{\pgfqpoint{4.385482in}{0.988419in}}%
\pgfpathlineto{\pgfqpoint{4.399485in}{0.984957in}}%
\pgfpathlineto{\pgfqpoint{4.413495in}{0.981604in}}%
\pgfpathlineto{\pgfqpoint{4.427514in}{0.978358in}}%
\pgfpathlineto{\pgfqpoint{4.435408in}{0.985797in}}%
\pgfpathlineto{\pgfqpoint{4.443297in}{0.993438in}}%
\pgfpathlineto{\pgfqpoint{4.451180in}{1.001275in}}%
\pgfpathlineto{\pgfqpoint{4.459059in}{1.009305in}}%
\pgfpathlineto{\pgfqpoint{4.445051in}{1.012026in}}%
\pgfpathlineto{\pgfqpoint{4.431051in}{1.014855in}}%
\pgfpathlineto{\pgfqpoint{4.417060in}{1.017792in}}%
\pgfpathlineto{\pgfqpoint{4.403076in}{1.020837in}}%
\pgfpathlineto{\pgfqpoint{4.395187in}{1.013326in}}%
\pgfpathlineto{\pgfqpoint{4.387292in}{1.006011in}}%
\pgfpathlineto{\pgfqpoint{4.379392in}{0.998897in}}%
\pgfpathlineto{\pgfqpoint{4.371486in}{0.991989in}}%
\pgfpathclose%
\pgfusepath{fill}%
\end{pgfscope}%
\begin{pgfscope}%
\pgfpathrectangle{\pgfqpoint{1.254980in}{0.150000in}}{\pgfqpoint{5.490039in}{5.490039in}}%
\pgfusepath{clip}%
\pgfsetbuttcap%
\pgfsetroundjoin%
\definecolor{currentfill}{rgb}{0.282910,0.105393,0.426902}%
\pgfsetfillcolor{currentfill}%
\pgfsetfillopacity{0.700000}%
\pgfsetlinewidth{0.000000pt}%
\definecolor{currentstroke}{rgb}{0.000000,0.000000,0.000000}%
\pgfsetstrokecolor{currentstroke}%
\pgfsetdash{}{0pt}%
\pgfpathmoveto{\pgfqpoint{4.778339in}{1.113866in}}%
\pgfpathlineto{\pgfqpoint{4.792497in}{1.114448in}}%
\pgfpathlineto{\pgfqpoint{4.806665in}{1.115136in}}%
\pgfpathlineto{\pgfqpoint{4.820844in}{1.115931in}}%
\pgfpathlineto{\pgfqpoint{4.828651in}{1.128307in}}%
\pgfpathlineto{\pgfqpoint{4.836455in}{1.140799in}}%
\pgfpathlineto{\pgfqpoint{4.844256in}{1.153403in}}%
\pgfpathlineto{\pgfqpoint{4.852054in}{1.166117in}}%
\pgfpathlineto{\pgfqpoint{4.837874in}{1.164858in}}%
\pgfpathlineto{\pgfqpoint{4.823706in}{1.163705in}}%
\pgfpathlineto{\pgfqpoint{4.809549in}{1.162660in}}%
\pgfpathlineto{\pgfqpoint{4.801752in}{1.150290in}}%
\pgfpathlineto{\pgfqpoint{4.793951in}{1.138031in}}%
\pgfpathlineto{\pgfqpoint{4.786147in}{1.125889in}}%
\pgfpathlineto{\pgfqpoint{4.778339in}{1.113866in}}%
\pgfpathclose%
\pgfusepath{fill}%
\end{pgfscope}%
\begin{pgfscope}%
\pgfpathrectangle{\pgfqpoint{1.254980in}{0.150000in}}{\pgfqpoint{5.490039in}{5.490039in}}%
\pgfusepath{clip}%
\pgfsetbuttcap%
\pgfsetroundjoin%
\definecolor{currentfill}{rgb}{0.280894,0.078907,0.402329}%
\pgfsetfillcolor{currentfill}%
\pgfsetfillopacity{0.700000}%
\pgfsetlinewidth{0.000000pt}%
\definecolor{currentstroke}{rgb}{0.000000,0.000000,0.000000}%
\pgfsetstrokecolor{currentstroke}%
\pgfsetdash{}{0pt}%
\pgfpathmoveto{\pgfqpoint{4.690550in}{1.067706in}}%
\pgfpathlineto{\pgfqpoint{4.704667in}{1.067381in}}%
\pgfpathlineto{\pgfqpoint{4.718794in}{1.067163in}}%
\pgfpathlineto{\pgfqpoint{4.732931in}{1.067052in}}%
\pgfpathlineto{\pgfqpoint{4.747079in}{1.067047in}}%
\pgfpathlineto{\pgfqpoint{4.754899in}{1.078553in}}%
\pgfpathlineto{\pgfqpoint{4.762716in}{1.090195in}}%
\pgfpathlineto{\pgfqpoint{4.770529in}{1.101967in}}%
\pgfpathlineto{\pgfqpoint{4.778339in}{1.113866in}}%
\pgfpathlineto{\pgfqpoint{4.764193in}{1.113392in}}%
\pgfpathlineto{\pgfqpoint{4.750058in}{1.113024in}}%
\pgfpathlineto{\pgfqpoint{4.735933in}{1.112763in}}%
\pgfpathlineto{\pgfqpoint{4.721820in}{1.112609in}}%
\pgfpathlineto{\pgfqpoint{4.714007in}{1.101183in}}%
\pgfpathlineto{\pgfqpoint{4.706192in}{1.089887in}}%
\pgfpathlineto{\pgfqpoint{4.698373in}{1.078727in}}%
\pgfpathlineto{\pgfqpoint{4.690550in}{1.067706in}}%
\pgfpathclose%
\pgfusepath{fill}%
\end{pgfscope}%
\begin{pgfscope}%
\pgfpathrectangle{\pgfqpoint{1.254980in}{0.150000in}}{\pgfqpoint{5.490039in}{5.490039in}}%
\pgfusepath{clip}%
\pgfsetbuttcap%
\pgfsetroundjoin%
\definecolor{currentfill}{rgb}{0.277941,0.056324,0.381191}%
\pgfsetfillcolor{currentfill}%
\pgfsetfillopacity{0.700000}%
\pgfsetlinewidth{0.000000pt}%
\definecolor{currentstroke}{rgb}{0.000000,0.000000,0.000000}%
\pgfsetstrokecolor{currentstroke}%
\pgfsetdash{}{0pt}%
\pgfpathmoveto{\pgfqpoint{4.602842in}{1.029433in}}%
\pgfpathlineto{\pgfqpoint{4.616923in}{1.028187in}}%
\pgfpathlineto{\pgfqpoint{4.631014in}{1.027048in}}%
\pgfpathlineto{\pgfqpoint{4.645115in}{1.026015in}}%
\pgfpathlineto{\pgfqpoint{4.659225in}{1.025089in}}%
\pgfpathlineto{\pgfqpoint{4.667062in}{1.035515in}}%
\pgfpathlineto{\pgfqpoint{4.674895in}{1.046096in}}%
\pgfpathlineto{\pgfqpoint{4.682724in}{1.056827in}}%
\pgfpathlineto{\pgfqpoint{4.690550in}{1.067706in}}%
\pgfpathlineto{\pgfqpoint{4.676444in}{1.068137in}}%
\pgfpathlineto{\pgfqpoint{4.662348in}{1.068676in}}%
\pgfpathlineto{\pgfqpoint{4.648261in}{1.069322in}}%
\pgfpathlineto{\pgfqpoint{4.634185in}{1.070074in}}%
\pgfpathlineto{\pgfqpoint{4.626355in}{1.059684in}}%
\pgfpathlineto{\pgfqpoint{4.618521in}{1.049444in}}%
\pgfpathlineto{\pgfqpoint{4.610683in}{1.039359in}}%
\pgfpathlineto{\pgfqpoint{4.602842in}{1.029433in}}%
\pgfpathclose%
\pgfusepath{fill}%
\end{pgfscope}%
\begin{pgfscope}%
\pgfpathrectangle{\pgfqpoint{1.254980in}{0.150000in}}{\pgfqpoint{5.490039in}{5.490039in}}%
\pgfusepath{clip}%
\pgfsetbuttcap%
\pgfsetroundjoin%
\definecolor{currentfill}{rgb}{0.274952,0.037752,0.364543}%
\pgfsetfillcolor{currentfill}%
\pgfsetfillopacity{0.700000}%
\pgfsetlinewidth{0.000000pt}%
\definecolor{currentstroke}{rgb}{0.000000,0.000000,0.000000}%
\pgfsetstrokecolor{currentstroke}%
\pgfsetdash{}{0pt}%
\pgfpathmoveto{\pgfqpoint{4.515177in}{0.999496in}}%
\pgfpathlineto{\pgfqpoint{4.529228in}{0.997312in}}%
\pgfpathlineto{\pgfqpoint{4.543288in}{0.995236in}}%
\pgfpathlineto{\pgfqpoint{4.557358in}{0.993266in}}%
\pgfpathlineto{\pgfqpoint{4.571436in}{0.991403in}}%
\pgfpathlineto{\pgfqpoint{4.579293in}{1.000651in}}%
\pgfpathlineto{\pgfqpoint{4.587147in}{1.010075in}}%
\pgfpathlineto{\pgfqpoint{4.594996in}{1.019671in}}%
\pgfpathlineto{\pgfqpoint{4.602842in}{1.029433in}}%
\pgfpathlineto{\pgfqpoint{4.588770in}{1.030787in}}%
\pgfpathlineto{\pgfqpoint{4.574708in}{1.032247in}}%
\pgfpathlineto{\pgfqpoint{4.560655in}{1.033815in}}%
\pgfpathlineto{\pgfqpoint{4.546611in}{1.035489in}}%
\pgfpathlineto{\pgfqpoint{4.538759in}{1.026230in}}%
\pgfpathlineto{\pgfqpoint{4.530903in}{1.017142in}}%
\pgfpathlineto{\pgfqpoint{4.523042in}{1.008229in}}%
\pgfpathlineto{\pgfqpoint{4.515177in}{0.999496in}}%
\pgfpathclose%
\pgfusepath{fill}%
\end{pgfscope}%
\begin{pgfscope}%
\pgfpathrectangle{\pgfqpoint{1.254980in}{0.150000in}}{\pgfqpoint{5.490039in}{5.490039in}}%
\pgfusepath{clip}%
\pgfsetbuttcap%
\pgfsetroundjoin%
\definecolor{currentfill}{rgb}{0.272594,0.025563,0.353093}%
\pgfsetfillcolor{currentfill}%
\pgfsetfillopacity{0.700000}%
\pgfsetlinewidth{0.000000pt}%
\definecolor{currentstroke}{rgb}{0.000000,0.000000,0.000000}%
\pgfsetstrokecolor{currentstroke}%
\pgfsetdash{}{0pt}%
\pgfpathmoveto{\pgfqpoint{4.283848in}{0.984027in}}%
\pgfpathlineto{\pgfqpoint{4.297828in}{0.979483in}}%
\pgfpathlineto{\pgfqpoint{4.311815in}{0.975047in}}%
\pgfpathlineto{\pgfqpoint{4.325808in}{0.970720in}}%
\pgfpathlineto{\pgfqpoint{4.339810in}{0.966501in}}%
\pgfpathlineto{\pgfqpoint{4.347737in}{0.972542in}}%
\pgfpathlineto{\pgfqpoint{4.355659in}{0.978807in}}%
\pgfpathlineto{\pgfqpoint{4.363576in}{0.985290in}}%
\pgfpathlineto{\pgfqpoint{4.371486in}{0.991989in}}%
\pgfpathlineto{\pgfqpoint{4.357499in}{0.995666in}}%
\pgfpathlineto{\pgfqpoint{4.343519in}{0.999453in}}%
\pgfpathlineto{\pgfqpoint{4.329547in}{1.003347in}}%
\pgfpathlineto{\pgfqpoint{4.315582in}{1.007351in}}%
\pgfpathlineto{\pgfqpoint{4.307658in}{1.001187in}}%
\pgfpathlineto{\pgfqpoint{4.299727in}{0.995242in}}%
\pgfpathlineto{\pgfqpoint{4.291791in}{0.989521in}}%
\pgfpathlineto{\pgfqpoint{4.283848in}{0.984027in}}%
\pgfpathclose%
\pgfusepath{fill}%
\end{pgfscope}%
\begin{pgfscope}%
\pgfpathrectangle{\pgfqpoint{1.254980in}{0.150000in}}{\pgfqpoint{5.490039in}{5.490039in}}%
\pgfusepath{clip}%
\pgfsetbuttcap%
\pgfsetroundjoin%
\definecolor{currentfill}{rgb}{0.273809,0.031497,0.358853}%
\pgfsetfillcolor{currentfill}%
\pgfsetfillopacity{0.700000}%
\pgfsetlinewidth{0.000000pt}%
\definecolor{currentstroke}{rgb}{0.000000,0.000000,0.000000}%
\pgfsetstrokecolor{currentstroke}%
\pgfsetdash{}{0pt}%
\pgfpathmoveto{\pgfqpoint{4.427514in}{0.978358in}}%
\pgfpathlineto{\pgfqpoint{4.441541in}{0.975220in}}%
\pgfpathlineto{\pgfqpoint{4.455576in}{0.972189in}}%
\pgfpathlineto{\pgfqpoint{4.469620in}{0.969265in}}%
\pgfpathlineto{\pgfqpoint{4.483672in}{0.966449in}}%
\pgfpathlineto{\pgfqpoint{4.491555in}{0.974419in}}%
\pgfpathlineto{\pgfqpoint{4.499433in}{0.982587in}}%
\pgfpathlineto{\pgfqpoint{4.507307in}{0.990947in}}%
\pgfpathlineto{\pgfqpoint{4.515177in}{0.999496in}}%
\pgfpathlineto{\pgfqpoint{4.501134in}{1.001787in}}%
\pgfpathlineto{\pgfqpoint{4.487101in}{1.004186in}}%
\pgfpathlineto{\pgfqpoint{4.473076in}{1.006692in}}%
\pgfpathlineto{\pgfqpoint{4.459059in}{1.009305in}}%
\pgfpathlineto{\pgfqpoint{4.451180in}{1.001275in}}%
\pgfpathlineto{\pgfqpoint{4.443297in}{0.993438in}}%
\pgfpathlineto{\pgfqpoint{4.435408in}{0.985797in}}%
\pgfpathlineto{\pgfqpoint{4.427514in}{0.978358in}}%
\pgfpathclose%
\pgfusepath{fill}%
\end{pgfscope}%
\begin{pgfscope}%
\pgfpathrectangle{\pgfqpoint{1.254980in}{0.150000in}}{\pgfqpoint{5.490039in}{5.490039in}}%
\pgfusepath{clip}%
\pgfsetbuttcap%
\pgfsetroundjoin%
\definecolor{currentfill}{rgb}{0.281924,0.089666,0.412415}%
\pgfsetfillcolor{currentfill}%
\pgfsetfillopacity{0.700000}%
\pgfsetlinewidth{0.000000pt}%
\definecolor{currentstroke}{rgb}{0.000000,0.000000,0.000000}%
\pgfsetstrokecolor{currentstroke}%
\pgfsetdash{}{0pt}%
\pgfpathmoveto{\pgfqpoint{4.747079in}{1.067047in}}%
\pgfpathlineto{\pgfqpoint{4.761238in}{1.067149in}}%
\pgfpathlineto{\pgfqpoint{4.775407in}{1.067357in}}%
\pgfpathlineto{\pgfqpoint{4.789588in}{1.067672in}}%
\pgfpathlineto{\pgfqpoint{4.797406in}{1.079543in}}%
\pgfpathlineto{\pgfqpoint{4.805222in}{1.091546in}}%
\pgfpathlineto{\pgfqpoint{4.813035in}{1.103676in}}%
\pgfpathlineto{\pgfqpoint{4.820844in}{1.115931in}}%
\pgfpathlineto{\pgfqpoint{4.806665in}{1.115136in}}%
\pgfpathlineto{\pgfqpoint{4.792497in}{1.114448in}}%
\pgfpathlineto{\pgfqpoint{4.778339in}{1.113866in}}%
\pgfpathlineto{\pgfqpoint{4.770529in}{1.101967in}}%
\pgfpathlineto{\pgfqpoint{4.762716in}{1.090195in}}%
\pgfpathlineto{\pgfqpoint{4.754899in}{1.078553in}}%
\pgfpathlineto{\pgfqpoint{4.747079in}{1.067047in}}%
\pgfpathclose%
\pgfusepath{fill}%
\end{pgfscope}%
\begin{pgfscope}%
\pgfpathrectangle{\pgfqpoint{1.254980in}{0.150000in}}{\pgfqpoint{5.490039in}{5.490039in}}%
\pgfusepath{clip}%
\pgfsetbuttcap%
\pgfsetroundjoin%
\definecolor{currentfill}{rgb}{0.279566,0.067836,0.391917}%
\pgfsetfillcolor{currentfill}%
\pgfsetfillopacity{0.700000}%
\pgfsetlinewidth{0.000000pt}%
\definecolor{currentstroke}{rgb}{0.000000,0.000000,0.000000}%
\pgfsetstrokecolor{currentstroke}%
\pgfsetdash{}{0pt}%
\pgfpathmoveto{\pgfqpoint{4.659225in}{1.025089in}}%
\pgfpathlineto{\pgfqpoint{4.673346in}{1.024270in}}%
\pgfpathlineto{\pgfqpoint{4.687476in}{1.023557in}}%
\pgfpathlineto{\pgfqpoint{4.701617in}{1.022951in}}%
\pgfpathlineto{\pgfqpoint{4.715768in}{1.022451in}}%
\pgfpathlineto{\pgfqpoint{4.723600in}{1.033378in}}%
\pgfpathlineto{\pgfqpoint{4.731430in}{1.044455in}}%
\pgfpathlineto{\pgfqpoint{4.739256in}{1.055680in}}%
\pgfpathlineto{\pgfqpoint{4.747079in}{1.067047in}}%
\pgfpathlineto{\pgfqpoint{4.732931in}{1.067052in}}%
\pgfpathlineto{\pgfqpoint{4.718794in}{1.067163in}}%
\pgfpathlineto{\pgfqpoint{4.704667in}{1.067381in}}%
\pgfpathlineto{\pgfqpoint{4.690550in}{1.067706in}}%
\pgfpathlineto{\pgfqpoint{4.682724in}{1.056827in}}%
\pgfpathlineto{\pgfqpoint{4.674895in}{1.046096in}}%
\pgfpathlineto{\pgfqpoint{4.667062in}{1.035515in}}%
\pgfpathlineto{\pgfqpoint{4.659225in}{1.025089in}}%
\pgfpathclose%
\pgfusepath{fill}%
\end{pgfscope}%
\begin{pgfscope}%
\pgfpathrectangle{\pgfqpoint{1.254980in}{0.150000in}}{\pgfqpoint{5.490039in}{5.490039in}}%
\pgfusepath{clip}%
\pgfsetbuttcap%
\pgfsetroundjoin%
\definecolor{currentfill}{rgb}{0.272594,0.025563,0.353093}%
\pgfsetfillcolor{currentfill}%
\pgfsetfillopacity{0.700000}%
\pgfsetlinewidth{0.000000pt}%
\definecolor{currentstroke}{rgb}{0.000000,0.000000,0.000000}%
\pgfsetstrokecolor{currentstroke}%
\pgfsetdash{}{0pt}%
\pgfpathmoveto{\pgfqpoint{4.339810in}{0.966501in}}%
\pgfpathlineto{\pgfqpoint{4.353818in}{0.962390in}}%
\pgfpathlineto{\pgfqpoint{4.367834in}{0.958387in}}%
\pgfpathlineto{\pgfqpoint{4.381857in}{0.954492in}}%
\pgfpathlineto{\pgfqpoint{4.395889in}{0.950705in}}%
\pgfpathlineto{\pgfqpoint{4.403803in}{0.957294in}}%
\pgfpathlineto{\pgfqpoint{4.411712in}{0.964102in}}%
\pgfpathlineto{\pgfqpoint{4.419616in}{0.971125in}}%
\pgfpathlineto{\pgfqpoint{4.427514in}{0.978358in}}%
\pgfpathlineto{\pgfqpoint{4.413495in}{0.981604in}}%
\pgfpathlineto{\pgfqpoint{4.399485in}{0.984957in}}%
\pgfpathlineto{\pgfqpoint{4.385482in}{0.988419in}}%
\pgfpathlineto{\pgfqpoint{4.371486in}{0.991989in}}%
\pgfpathlineto{\pgfqpoint{4.363576in}{0.985290in}}%
\pgfpathlineto{\pgfqpoint{4.355659in}{0.978807in}}%
\pgfpathlineto{\pgfqpoint{4.347737in}{0.972542in}}%
\pgfpathlineto{\pgfqpoint{4.339810in}{0.966501in}}%
\pgfpathclose%
\pgfusepath{fill}%
\end{pgfscope}%
\begin{pgfscope}%
\pgfpathrectangle{\pgfqpoint{1.254980in}{0.150000in}}{\pgfqpoint{5.490039in}{5.490039in}}%
\pgfusepath{clip}%
\pgfsetbuttcap%
\pgfsetroundjoin%
\definecolor{currentfill}{rgb}{0.277018,0.050344,0.375715}%
\pgfsetfillcolor{currentfill}%
\pgfsetfillopacity{0.700000}%
\pgfsetlinewidth{0.000000pt}%
\definecolor{currentstroke}{rgb}{0.000000,0.000000,0.000000}%
\pgfsetstrokecolor{currentstroke}%
\pgfsetdash{}{0pt}%
\pgfpathmoveto{\pgfqpoint{4.571436in}{0.991403in}}%
\pgfpathlineto{\pgfqpoint{4.585524in}{0.989647in}}%
\pgfpathlineto{\pgfqpoint{4.599620in}{0.987998in}}%
\pgfpathlineto{\pgfqpoint{4.613727in}{0.986455in}}%
\pgfpathlineto{\pgfqpoint{4.627843in}{0.985019in}}%
\pgfpathlineto{\pgfqpoint{4.635694in}{0.994783in}}%
\pgfpathlineto{\pgfqpoint{4.643541in}{1.004719in}}%
\pgfpathlineto{\pgfqpoint{4.651385in}{1.014823in}}%
\pgfpathlineto{\pgfqpoint{4.659225in}{1.025089in}}%
\pgfpathlineto{\pgfqpoint{4.645115in}{1.026015in}}%
\pgfpathlineto{\pgfqpoint{4.631014in}{1.027048in}}%
\pgfpathlineto{\pgfqpoint{4.616923in}{1.028187in}}%
\pgfpathlineto{\pgfqpoint{4.602842in}{1.029433in}}%
\pgfpathlineto{\pgfqpoint{4.594996in}{1.019671in}}%
\pgfpathlineto{\pgfqpoint{4.587147in}{1.010075in}}%
\pgfpathlineto{\pgfqpoint{4.579293in}{1.000651in}}%
\pgfpathlineto{\pgfqpoint{4.571436in}{0.991403in}}%
\pgfpathclose%
\pgfusepath{fill}%
\end{pgfscope}%
\begin{pgfscope}%
\pgfpathrectangle{\pgfqpoint{1.254980in}{0.150000in}}{\pgfqpoint{5.490039in}{5.490039in}}%
\pgfusepath{clip}%
\pgfsetbuttcap%
\pgfsetroundjoin%
\definecolor{currentfill}{rgb}{0.274952,0.037752,0.364543}%
\pgfsetfillcolor{currentfill}%
\pgfsetfillopacity{0.700000}%
\pgfsetlinewidth{0.000000pt}%
\definecolor{currentstroke}{rgb}{0.000000,0.000000,0.000000}%
\pgfsetstrokecolor{currentstroke}%
\pgfsetdash{}{0pt}%
\pgfpathmoveto{\pgfqpoint{4.483672in}{0.966449in}}%
\pgfpathlineto{\pgfqpoint{4.497732in}{0.963740in}}%
\pgfpathlineto{\pgfqpoint{4.511801in}{0.961138in}}%
\pgfpathlineto{\pgfqpoint{4.525879in}{0.958642in}}%
\pgfpathlineto{\pgfqpoint{4.539965in}{0.956254in}}%
\pgfpathlineto{\pgfqpoint{4.547839in}{0.964756in}}%
\pgfpathlineto{\pgfqpoint{4.555709in}{0.973451in}}%
\pgfpathlineto{\pgfqpoint{4.563574in}{0.982335in}}%
\pgfpathlineto{\pgfqpoint{4.571436in}{0.991403in}}%
\pgfpathlineto{\pgfqpoint{4.557358in}{0.993266in}}%
\pgfpathlineto{\pgfqpoint{4.543288in}{0.995236in}}%
\pgfpathlineto{\pgfqpoint{4.529228in}{0.997312in}}%
\pgfpathlineto{\pgfqpoint{4.515177in}{0.999496in}}%
\pgfpathlineto{\pgfqpoint{4.507307in}{0.990947in}}%
\pgfpathlineto{\pgfqpoint{4.499433in}{0.982587in}}%
\pgfpathlineto{\pgfqpoint{4.491555in}{0.974419in}}%
\pgfpathlineto{\pgfqpoint{4.483672in}{0.966449in}}%
\pgfpathclose%
\pgfusepath{fill}%
\end{pgfscope}%
\begin{pgfscope}%
\pgfpathrectangle{\pgfqpoint{1.254980in}{0.150000in}}{\pgfqpoint{5.490039in}{5.490039in}}%
\pgfusepath{clip}%
\pgfsetbuttcap%
\pgfsetroundjoin%
\definecolor{currentfill}{rgb}{0.273809,0.031497,0.358853}%
\pgfsetfillcolor{currentfill}%
\pgfsetfillopacity{0.700000}%
\pgfsetlinewidth{0.000000pt}%
\definecolor{currentstroke}{rgb}{0.000000,0.000000,0.000000}%
\pgfsetstrokecolor{currentstroke}%
\pgfsetdash{}{0pt}%
\pgfpathmoveto{\pgfqpoint{4.395889in}{0.950705in}}%
\pgfpathlineto{\pgfqpoint{4.409928in}{0.947025in}}%
\pgfpathlineto{\pgfqpoint{4.423974in}{0.943453in}}%
\pgfpathlineto{\pgfqpoint{4.438029in}{0.939988in}}%
\pgfpathlineto{\pgfqpoint{4.452092in}{0.936630in}}%
\pgfpathlineto{\pgfqpoint{4.459994in}{0.943766in}}%
\pgfpathlineto{\pgfqpoint{4.467891in}{0.951118in}}%
\pgfpathlineto{\pgfqpoint{4.475784in}{0.958680in}}%
\pgfpathlineto{\pgfqpoint{4.483672in}{0.966449in}}%
\pgfpathlineto{\pgfqpoint{4.469620in}{0.969265in}}%
\pgfpathlineto{\pgfqpoint{4.455576in}{0.972189in}}%
\pgfpathlineto{\pgfqpoint{4.441541in}{0.975220in}}%
\pgfpathlineto{\pgfqpoint{4.427514in}{0.978358in}}%
\pgfpathlineto{\pgfqpoint{4.419616in}{0.971125in}}%
\pgfpathlineto{\pgfqpoint{4.411712in}{0.964102in}}%
\pgfpathlineto{\pgfqpoint{4.403803in}{0.957294in}}%
\pgfpathlineto{\pgfqpoint{4.395889in}{0.950705in}}%
\pgfpathclose%
\pgfusepath{fill}%
\end{pgfscope}%
\begin{pgfscope}%
\pgfpathrectangle{\pgfqpoint{1.254980in}{0.150000in}}{\pgfqpoint{5.490039in}{5.490039in}}%
\pgfusepath{clip}%
\pgfsetbuttcap%
\pgfsetroundjoin%
\definecolor{currentfill}{rgb}{0.280894,0.078907,0.402329}%
\pgfsetfillcolor{currentfill}%
\pgfsetfillopacity{0.700000}%
\pgfsetlinewidth{0.000000pt}%
\definecolor{currentstroke}{rgb}{0.000000,0.000000,0.000000}%
\pgfsetstrokecolor{currentstroke}%
\pgfsetdash{}{0pt}%
\pgfpathmoveto{\pgfqpoint{4.715768in}{1.022451in}}%
\pgfpathlineto{\pgfqpoint{4.729929in}{1.022058in}}%
\pgfpathlineto{\pgfqpoint{4.744101in}{1.021770in}}%
\pgfpathlineto{\pgfqpoint{4.758283in}{1.021589in}}%
\pgfpathlineto{\pgfqpoint{4.766114in}{1.032892in}}%
\pgfpathlineto{\pgfqpoint{4.773941in}{1.044343in}}%
\pgfpathlineto{\pgfqpoint{4.781766in}{1.055938in}}%
\pgfpathlineto{\pgfqpoint{4.789588in}{1.067672in}}%
\pgfpathlineto{\pgfqpoint{4.775407in}{1.067357in}}%
\pgfpathlineto{\pgfqpoint{4.761238in}{1.067149in}}%
\pgfpathlineto{\pgfqpoint{4.747079in}{1.067047in}}%
\pgfpathlineto{\pgfqpoint{4.739256in}{1.055680in}}%
\pgfpathlineto{\pgfqpoint{4.731430in}{1.044455in}}%
\pgfpathlineto{\pgfqpoint{4.723600in}{1.033378in}}%
\pgfpathlineto{\pgfqpoint{4.715768in}{1.022451in}}%
\pgfpathclose%
\pgfusepath{fill}%
\end{pgfscope}%
\begin{pgfscope}%
\pgfpathrectangle{\pgfqpoint{1.254980in}{0.150000in}}{\pgfqpoint{5.490039in}{5.490039in}}%
\pgfusepath{clip}%
\pgfsetbuttcap%
\pgfsetroundjoin%
\definecolor{currentfill}{rgb}{0.278791,0.062145,0.386592}%
\pgfsetfillcolor{currentfill}%
\pgfsetfillopacity{0.700000}%
\pgfsetlinewidth{0.000000pt}%
\definecolor{currentstroke}{rgb}{0.000000,0.000000,0.000000}%
\pgfsetstrokecolor{currentstroke}%
\pgfsetdash{}{0pt}%
\pgfpathmoveto{\pgfqpoint{4.627843in}{0.985019in}}%
\pgfpathlineto{\pgfqpoint{4.641968in}{0.983689in}}%
\pgfpathlineto{\pgfqpoint{4.656103in}{0.982466in}}%
\pgfpathlineto{\pgfqpoint{4.670248in}{0.981348in}}%
\pgfpathlineto{\pgfqpoint{4.684403in}{0.980337in}}%
\pgfpathlineto{\pgfqpoint{4.692249in}{0.990619in}}%
\pgfpathlineto{\pgfqpoint{4.700092in}{1.001067in}}%
\pgfpathlineto{\pgfqpoint{4.707932in}{1.011680in}}%
\pgfpathlineto{\pgfqpoint{4.715768in}{1.022451in}}%
\pgfpathlineto{\pgfqpoint{4.701617in}{1.022951in}}%
\pgfpathlineto{\pgfqpoint{4.687476in}{1.023557in}}%
\pgfpathlineto{\pgfqpoint{4.673346in}{1.024270in}}%
\pgfpathlineto{\pgfqpoint{4.659225in}{1.025089in}}%
\pgfpathlineto{\pgfqpoint{4.651385in}{1.014823in}}%
\pgfpathlineto{\pgfqpoint{4.643541in}{1.004719in}}%
\pgfpathlineto{\pgfqpoint{4.635694in}{0.994783in}}%
\pgfpathlineto{\pgfqpoint{4.627843in}{0.985019in}}%
\pgfpathclose%
\pgfusepath{fill}%
\end{pgfscope}%
\begin{pgfscope}%
\pgfpathrectangle{\pgfqpoint{1.254980in}{0.150000in}}{\pgfqpoint{5.490039in}{5.490039in}}%
\pgfusepath{clip}%
\pgfsetbuttcap%
\pgfsetroundjoin%
\definecolor{currentfill}{rgb}{0.276022,0.044167,0.370164}%
\pgfsetfillcolor{currentfill}%
\pgfsetfillopacity{0.700000}%
\pgfsetlinewidth{0.000000pt}%
\definecolor{currentstroke}{rgb}{0.000000,0.000000,0.000000}%
\pgfsetstrokecolor{currentstroke}%
\pgfsetdash{}{0pt}%
\pgfpathmoveto{\pgfqpoint{4.539965in}{0.956254in}}%
\pgfpathlineto{\pgfqpoint{4.554060in}{0.953972in}}%
\pgfpathlineto{\pgfqpoint{4.568164in}{0.951796in}}%
\pgfpathlineto{\pgfqpoint{4.582278in}{0.949727in}}%
\pgfpathlineto{\pgfqpoint{4.596400in}{0.947765in}}%
\pgfpathlineto{\pgfqpoint{4.604267in}{0.956799in}}%
\pgfpathlineto{\pgfqpoint{4.612129in}{0.966023in}}%
\pgfpathlineto{\pgfqpoint{4.619988in}{0.975431in}}%
\pgfpathlineto{\pgfqpoint{4.627843in}{0.985019in}}%
\pgfpathlineto{\pgfqpoint{4.613727in}{0.986455in}}%
\pgfpathlineto{\pgfqpoint{4.599620in}{0.987998in}}%
\pgfpathlineto{\pgfqpoint{4.585524in}{0.989647in}}%
\pgfpathlineto{\pgfqpoint{4.571436in}{0.991403in}}%
\pgfpathlineto{\pgfqpoint{4.563574in}{0.982335in}}%
\pgfpathlineto{\pgfqpoint{4.555709in}{0.973451in}}%
\pgfpathlineto{\pgfqpoint{4.547839in}{0.964756in}}%
\pgfpathlineto{\pgfqpoint{4.539965in}{0.956254in}}%
\pgfpathclose%
\pgfusepath{fill}%
\end{pgfscope}%
\begin{pgfscope}%
\pgfpathrectangle{\pgfqpoint{1.254980in}{0.150000in}}{\pgfqpoint{5.490039in}{5.490039in}}%
\pgfusepath{clip}%
\pgfsetbuttcap%
\pgfsetroundjoin%
\definecolor{currentfill}{rgb}{0.274952,0.037752,0.364543}%
\pgfsetfillcolor{currentfill}%
\pgfsetfillopacity{0.700000}%
\pgfsetlinewidth{0.000000pt}%
\definecolor{currentstroke}{rgb}{0.000000,0.000000,0.000000}%
\pgfsetstrokecolor{currentstroke}%
\pgfsetdash{}{0pt}%
\pgfpathmoveto{\pgfqpoint{4.452092in}{0.936630in}}%
\pgfpathlineto{\pgfqpoint{4.466163in}{0.933379in}}%
\pgfpathlineto{\pgfqpoint{4.480242in}{0.930234in}}%
\pgfpathlineto{\pgfqpoint{4.494329in}{0.927197in}}%
\pgfpathlineto{\pgfqpoint{4.508425in}{0.924266in}}%
\pgfpathlineto{\pgfqpoint{4.516317in}{0.931951in}}%
\pgfpathlineto{\pgfqpoint{4.524204in}{0.939847in}}%
\pgfpathlineto{\pgfqpoint{4.532087in}{0.947949in}}%
\pgfpathlineto{\pgfqpoint{4.539965in}{0.956254in}}%
\pgfpathlineto{\pgfqpoint{4.525879in}{0.958642in}}%
\pgfpathlineto{\pgfqpoint{4.511801in}{0.961138in}}%
\pgfpathlineto{\pgfqpoint{4.497732in}{0.963740in}}%
\pgfpathlineto{\pgfqpoint{4.483672in}{0.966449in}}%
\pgfpathlineto{\pgfqpoint{4.475784in}{0.958680in}}%
\pgfpathlineto{\pgfqpoint{4.467891in}{0.951118in}}%
\pgfpathlineto{\pgfqpoint{4.459994in}{0.943766in}}%
\pgfpathlineto{\pgfqpoint{4.452092in}{0.936630in}}%
\pgfpathclose%
\pgfusepath{fill}%
\end{pgfscope}%
\begin{pgfscope}%
\pgfpathrectangle{\pgfqpoint{1.254980in}{0.150000in}}{\pgfqpoint{5.490039in}{5.490039in}}%
\pgfusepath{clip}%
\pgfsetbuttcap%
\pgfsetroundjoin%
\definecolor{currentfill}{rgb}{0.280267,0.073417,0.397163}%
\pgfsetfillcolor{currentfill}%
\pgfsetfillopacity{0.700000}%
\pgfsetlinewidth{0.000000pt}%
\definecolor{currentstroke}{rgb}{0.000000,0.000000,0.000000}%
\pgfsetstrokecolor{currentstroke}%
\pgfsetdash{}{0pt}%
\pgfpathmoveto{\pgfqpoint{4.684403in}{0.980337in}}%
\pgfpathlineto{\pgfqpoint{4.698568in}{0.979432in}}%
\pgfpathlineto{\pgfqpoint{4.712743in}{0.978633in}}%
\pgfpathlineto{\pgfqpoint{4.726929in}{0.977940in}}%
\pgfpathlineto{\pgfqpoint{4.734772in}{0.988610in}}%
\pgfpathlineto{\pgfqpoint{4.742612in}{0.999444in}}%
\pgfpathlineto{\pgfqpoint{4.750449in}{1.010438in}}%
\pgfpathlineto{\pgfqpoint{4.758283in}{1.021589in}}%
\pgfpathlineto{\pgfqpoint{4.744101in}{1.021770in}}%
\pgfpathlineto{\pgfqpoint{4.729929in}{1.022058in}}%
\pgfpathlineto{\pgfqpoint{4.715768in}{1.022451in}}%
\pgfpathlineto{\pgfqpoint{4.707932in}{1.011680in}}%
\pgfpathlineto{\pgfqpoint{4.700092in}{1.001067in}}%
\pgfpathlineto{\pgfqpoint{4.692249in}{0.990619in}}%
\pgfpathlineto{\pgfqpoint{4.684403in}{0.980337in}}%
\pgfpathclose%
\pgfusepath{fill}%
\end{pgfscope}%
\begin{pgfscope}%
\pgfpathrectangle{\pgfqpoint{1.254980in}{0.150000in}}{\pgfqpoint{5.490039in}{5.490039in}}%
\pgfusepath{clip}%
\pgfsetbuttcap%
\pgfsetroundjoin%
\definecolor{currentfill}{rgb}{0.277941,0.056324,0.381191}%
\pgfsetfillcolor{currentfill}%
\pgfsetfillopacity{0.700000}%
\pgfsetlinewidth{0.000000pt}%
\definecolor{currentstroke}{rgb}{0.000000,0.000000,0.000000}%
\pgfsetstrokecolor{currentstroke}%
\pgfsetdash{}{0pt}%
\pgfpathmoveto{\pgfqpoint{4.596400in}{0.947765in}}%
\pgfpathlineto{\pgfqpoint{4.610532in}{0.945908in}}%
\pgfpathlineto{\pgfqpoint{4.624673in}{0.944158in}}%
\pgfpathlineto{\pgfqpoint{4.638824in}{0.942513in}}%
\pgfpathlineto{\pgfqpoint{4.652984in}{0.940975in}}%
\pgfpathlineto{\pgfqpoint{4.660844in}{0.950542in}}%
\pgfpathlineto{\pgfqpoint{4.668701in}{0.960295in}}%
\pgfpathlineto{\pgfqpoint{4.676554in}{0.970228in}}%
\pgfpathlineto{\pgfqpoint{4.684403in}{0.980337in}}%
\pgfpathlineto{\pgfqpoint{4.670248in}{0.981348in}}%
\pgfpathlineto{\pgfqpoint{4.656103in}{0.982466in}}%
\pgfpathlineto{\pgfqpoint{4.641968in}{0.983689in}}%
\pgfpathlineto{\pgfqpoint{4.627843in}{0.985019in}}%
\pgfpathlineto{\pgfqpoint{4.619988in}{0.975431in}}%
\pgfpathlineto{\pgfqpoint{4.612129in}{0.966023in}}%
\pgfpathlineto{\pgfqpoint{4.604267in}{0.956799in}}%
\pgfpathlineto{\pgfqpoint{4.596400in}{0.947765in}}%
\pgfpathclose%
\pgfusepath{fill}%
\end{pgfscope}%
\begin{pgfscope}%
\pgfpathrectangle{\pgfqpoint{1.254980in}{0.150000in}}{\pgfqpoint{5.490039in}{5.490039in}}%
\pgfusepath{clip}%
\pgfsetbuttcap%
\pgfsetroundjoin%
\definecolor{currentfill}{rgb}{0.276022,0.044167,0.370164}%
\pgfsetfillcolor{currentfill}%
\pgfsetfillopacity{0.700000}%
\pgfsetlinewidth{0.000000pt}%
\definecolor{currentstroke}{rgb}{0.000000,0.000000,0.000000}%
\pgfsetstrokecolor{currentstroke}%
\pgfsetdash{}{0pt}%
\pgfpathmoveto{\pgfqpoint{4.508425in}{0.924266in}}%
\pgfpathlineto{\pgfqpoint{4.522529in}{0.921442in}}%
\pgfpathlineto{\pgfqpoint{4.536642in}{0.918724in}}%
\pgfpathlineto{\pgfqpoint{4.550764in}{0.916112in}}%
\pgfpathlineto{\pgfqpoint{4.564895in}{0.913606in}}%
\pgfpathlineto{\pgfqpoint{4.572777in}{0.921840in}}%
\pgfpathlineto{\pgfqpoint{4.580656in}{0.930281in}}%
\pgfpathlineto{\pgfqpoint{4.588530in}{0.938924in}}%
\pgfpathlineto{\pgfqpoint{4.596400in}{0.947765in}}%
\pgfpathlineto{\pgfqpoint{4.582278in}{0.949727in}}%
\pgfpathlineto{\pgfqpoint{4.568164in}{0.951796in}}%
\pgfpathlineto{\pgfqpoint{4.554060in}{0.953972in}}%
\pgfpathlineto{\pgfqpoint{4.539965in}{0.956254in}}%
\pgfpathlineto{\pgfqpoint{4.532087in}{0.947949in}}%
\pgfpathlineto{\pgfqpoint{4.524204in}{0.939847in}}%
\pgfpathlineto{\pgfqpoint{4.516317in}{0.931951in}}%
\pgfpathlineto{\pgfqpoint{4.508425in}{0.924266in}}%
\pgfpathclose%
\pgfusepath{fill}%
\end{pgfscope}%
\begin{pgfscope}%
\pgfpathrectangle{\pgfqpoint{1.254980in}{0.150000in}}{\pgfqpoint{5.490039in}{5.490039in}}%
\pgfusepath{clip}%
\pgfsetbuttcap%
\pgfsetroundjoin%
\definecolor{currentfill}{rgb}{0.278791,0.062145,0.386592}%
\pgfsetfillcolor{currentfill}%
\pgfsetfillopacity{0.700000}%
\pgfsetlinewidth{0.000000pt}%
\definecolor{currentstroke}{rgb}{0.000000,0.000000,0.000000}%
\pgfsetstrokecolor{currentstroke}%
\pgfsetdash{}{0pt}%
\pgfpathmoveto{\pgfqpoint{4.652984in}{0.940975in}}%
\pgfpathlineto{\pgfqpoint{4.667154in}{0.939542in}}%
\pgfpathlineto{\pgfqpoint{4.681334in}{0.938215in}}%
\pgfpathlineto{\pgfqpoint{4.695523in}{0.936994in}}%
\pgfpathlineto{\pgfqpoint{4.703379in}{0.946962in}}%
\pgfpathlineto{\pgfqpoint{4.711232in}{0.957112in}}%
\pgfpathlineto{\pgfqpoint{4.719082in}{0.967439in}}%
\pgfpathlineto{\pgfqpoint{4.726929in}{0.977940in}}%
\pgfpathlineto{\pgfqpoint{4.712743in}{0.978633in}}%
\pgfpathlineto{\pgfqpoint{4.698568in}{0.979432in}}%
\pgfpathlineto{\pgfqpoint{4.684403in}{0.980337in}}%
\pgfpathlineto{\pgfqpoint{4.676554in}{0.970228in}}%
\pgfpathlineto{\pgfqpoint{4.668701in}{0.960295in}}%
\pgfpathlineto{\pgfqpoint{4.660844in}{0.950542in}}%
\pgfpathlineto{\pgfqpoint{4.652984in}{0.940975in}}%
\pgfpathclose%
\pgfusepath{fill}%
\end{pgfscope}%
\begin{pgfscope}%
\pgfpathrectangle{\pgfqpoint{1.254980in}{0.150000in}}{\pgfqpoint{5.490039in}{5.490039in}}%
\pgfusepath{clip}%
\pgfsetbuttcap%
\pgfsetroundjoin%
\definecolor{currentfill}{rgb}{0.277018,0.050344,0.375715}%
\pgfsetfillcolor{currentfill}%
\pgfsetfillopacity{0.700000}%
\pgfsetlinewidth{0.000000pt}%
\definecolor{currentstroke}{rgb}{0.000000,0.000000,0.000000}%
\pgfsetstrokecolor{currentstroke}%
\pgfsetdash{}{0pt}%
\pgfpathmoveto{\pgfqpoint{4.564895in}{0.913606in}}%
\pgfpathlineto{\pgfqpoint{4.579034in}{0.911207in}}%
\pgfpathlineto{\pgfqpoint{4.593183in}{0.908913in}}%
\pgfpathlineto{\pgfqpoint{4.607341in}{0.906725in}}%
\pgfpathlineto{\pgfqpoint{4.621507in}{0.904643in}}%
\pgfpathlineto{\pgfqpoint{4.629382in}{0.913426in}}%
\pgfpathlineto{\pgfqpoint{4.637253in}{0.922412in}}%
\pgfpathlineto{\pgfqpoint{4.645120in}{0.931597in}}%
\pgfpathlineto{\pgfqpoint{4.652984in}{0.940975in}}%
\pgfpathlineto{\pgfqpoint{4.638824in}{0.942513in}}%
\pgfpathlineto{\pgfqpoint{4.624673in}{0.944158in}}%
\pgfpathlineto{\pgfqpoint{4.610532in}{0.945908in}}%
\pgfpathlineto{\pgfqpoint{4.596400in}{0.947765in}}%
\pgfpathlineto{\pgfqpoint{4.588530in}{0.938924in}}%
\pgfpathlineto{\pgfqpoint{4.580656in}{0.930281in}}%
\pgfpathlineto{\pgfqpoint{4.572777in}{0.921840in}}%
\pgfpathlineto{\pgfqpoint{4.564895in}{0.913606in}}%
\pgfpathclose%
\pgfusepath{fill}%
\end{pgfscope}%
\begin{pgfscope}%
\pgfpathrectangle{\pgfqpoint{1.254980in}{0.150000in}}{\pgfqpoint{5.490039in}{5.490039in}}%
\pgfusepath{clip}%
\pgfsetbuttcap%
\pgfsetroundjoin%
\definecolor{currentfill}{rgb}{0.277941,0.056324,0.381191}%
\pgfsetfillcolor{currentfill}%
\pgfsetfillopacity{0.700000}%
\pgfsetlinewidth{0.000000pt}%
\definecolor{currentstroke}{rgb}{0.000000,0.000000,0.000000}%
\pgfsetstrokecolor{currentstroke}%
\pgfsetdash{}{0pt}%
\pgfpathmoveto{\pgfqpoint{4.621507in}{0.904643in}}%
\pgfpathlineto{\pgfqpoint{4.635684in}{0.902667in}}%
\pgfpathlineto{\pgfqpoint{4.649869in}{0.900796in}}%
\pgfpathlineto{\pgfqpoint{4.664064in}{0.899031in}}%
\pgfpathlineto{\pgfqpoint{4.671934in}{0.908226in}}%
\pgfpathlineto{\pgfqpoint{4.679801in}{0.917622in}}%
\pgfpathlineto{\pgfqpoint{4.687664in}{0.927212in}}%
\pgfpathlineto{\pgfqpoint{4.695523in}{0.936994in}}%
\pgfpathlineto{\pgfqpoint{4.681334in}{0.938215in}}%
\pgfpathlineto{\pgfqpoint{4.667154in}{0.939542in}}%
\pgfpathlineto{\pgfqpoint{4.652984in}{0.940975in}}%
\pgfpathlineto{\pgfqpoint{4.645120in}{0.931597in}}%
\pgfpathlineto{\pgfqpoint{4.637253in}{0.922412in}}%
\pgfpathlineto{\pgfqpoint{4.629382in}{0.913426in}}%
\pgfpathlineto{\pgfqpoint{4.621507in}{0.904643in}}%
\pgfpathclose%
\pgfusepath{fill}%
\end{pgfscope}%
\begin{pgfscope}%
\pgfsetbuttcap%
\pgfsetmiterjoin%
\definecolor{currentfill}{rgb}{1.000000,1.000000,1.000000}%
\pgfsetfillcolor{currentfill}%
\pgfsetfillopacity{0.800000}%
\pgfsetlinewidth{1.003750pt}%
\definecolor{currentstroke}{rgb}{0.800000,0.800000,0.800000}%
\pgfsetstrokecolor{currentstroke}%
\pgfsetstrokeopacity{0.800000}%
\pgfsetdash{}{0pt}%
\pgfpathmoveto{\pgfqpoint{5.541867in}{5.121213in}}%
\pgfpathlineto{\pgfqpoint{6.647797in}{5.121213in}}%
\pgfpathquadraticcurveto{\pgfqpoint{6.675575in}{5.121213in}}{\pgfqpoint{6.675575in}{5.148991in}}%
\pgfpathlineto{\pgfqpoint{6.675575in}{5.542817in}}%
\pgfpathquadraticcurveto{\pgfqpoint{6.675575in}{5.570595in}}{\pgfqpoint{6.647797in}{5.570595in}}%
\pgfpathlineto{\pgfqpoint{5.541867in}{5.570595in}}%
\pgfpathquadraticcurveto{\pgfqpoint{5.514090in}{5.570595in}}{\pgfqpoint{5.514090in}{5.542817in}}%
\pgfpathlineto{\pgfqpoint{5.514090in}{5.148991in}}%
\pgfpathquadraticcurveto{\pgfqpoint{5.514090in}{5.121213in}}{\pgfqpoint{5.541867in}{5.121213in}}%
\pgfpathlineto{\pgfqpoint{5.541867in}{5.121213in}}%
\pgfpathclose%
\pgfusepath{stroke,fill}%
\end{pgfscope}%
\begin{pgfscope}%
\pgfsetrectcap%
\pgfsetroundjoin%
\pgfsetlinewidth{1.505625pt}%
\definecolor{currentstroke}{rgb}{1.000000,0.000000,0.000000}%
\pgfsetstrokecolor{currentstroke}%
\pgfsetdash{}{0pt}%
\pgfpathmoveto{\pgfqpoint{5.569645in}{5.458127in}}%
\pgfpathlineto{\pgfqpoint{5.708534in}{5.458127in}}%
\pgfpathlineto{\pgfqpoint{5.847423in}{5.458127in}}%
\pgfusepath{stroke}%
\end{pgfscope}%
\begin{pgfscope}%
\pgfsetbuttcap%
\pgfsetroundjoin%
\definecolor{currentfill}{rgb}{1.000000,0.000000,0.000000}%
\pgfsetfillcolor{currentfill}%
\pgfsetlinewidth{1.003750pt}%
\definecolor{currentstroke}{rgb}{1.000000,0.000000,0.000000}%
\pgfsetstrokecolor{currentstroke}%
\pgfsetdash{}{0pt}%
\pgfsys@defobject{currentmarker}{\pgfqpoint{-0.041667in}{-0.041667in}}{\pgfqpoint{0.041667in}{0.041667in}}{%
\pgfpathmoveto{\pgfqpoint{0.000000in}{-0.041667in}}%
\pgfpathcurveto{\pgfqpoint{0.011050in}{-0.041667in}}{\pgfqpoint{0.021649in}{-0.037276in}}{\pgfqpoint{0.029463in}{-0.029463in}}%
\pgfpathcurveto{\pgfqpoint{0.037276in}{-0.021649in}}{\pgfqpoint{0.041667in}{-0.011050in}}{\pgfqpoint{0.041667in}{0.000000in}}%
\pgfpathcurveto{\pgfqpoint{0.041667in}{0.011050in}}{\pgfqpoint{0.037276in}{0.021649in}}{\pgfqpoint{0.029463in}{0.029463in}}%
\pgfpathcurveto{\pgfqpoint{0.021649in}{0.037276in}}{\pgfqpoint{0.011050in}{0.041667in}}{\pgfqpoint{0.000000in}{0.041667in}}%
\pgfpathcurveto{\pgfqpoint{-0.011050in}{0.041667in}}{\pgfqpoint{-0.021649in}{0.037276in}}{\pgfqpoint{-0.029463in}{0.029463in}}%
\pgfpathcurveto{\pgfqpoint{-0.037276in}{0.021649in}}{\pgfqpoint{-0.041667in}{0.011050in}}{\pgfqpoint{-0.041667in}{0.000000in}}%
\pgfpathcurveto{\pgfqpoint{-0.041667in}{-0.011050in}}{\pgfqpoint{-0.037276in}{-0.021649in}}{\pgfqpoint{-0.029463in}{-0.029463in}}%
\pgfpathcurveto{\pgfqpoint{-0.021649in}{-0.037276in}}{\pgfqpoint{-0.011050in}{-0.041667in}}{\pgfqpoint{0.000000in}{-0.041667in}}%
\pgfpathlineto{\pgfqpoint{0.000000in}{-0.041667in}}%
\pgfpathclose%
\pgfusepath{stroke,fill}%
}%
\begin{pgfscope}%
\pgfsys@transformshift{5.708534in}{5.458127in}%
\pgfsys@useobject{currentmarker}{}%
\end{pgfscope}%
\end{pgfscope}%
\begin{pgfscope}%
\definecolor{textcolor}{rgb}{0.000000,0.000000,0.000000}%
\pgfsetstrokecolor{textcolor}%
\pgfsetfillcolor{textcolor}%
\pgftext[x=5.958534in,y=5.409516in,left,base]{\color{textcolor}\sffamily\fontsize{10.000000}{12.000000}\selectfont Iterations}%
\end{pgfscope}%
\begin{pgfscope}%
\pgfsetbuttcap%
\pgfsetroundjoin%
\definecolor{currentfill}{rgb}{0.000000,0.000000,1.000000}%
\pgfsetfillcolor{currentfill}%
\pgfsetlinewidth{1.003750pt}%
\definecolor{currentstroke}{rgb}{0.000000,0.000000,1.000000}%
\pgfsetstrokecolor{currentstroke}%
\pgfsetdash{}{0pt}%
\pgfsys@defobject{currentmarker}{\pgfqpoint{-0.069444in}{-0.069444in}}{\pgfqpoint{0.069444in}{0.069444in}}{%
\pgfpathmoveto{\pgfqpoint{0.000000in}{-0.069444in}}%
\pgfpathcurveto{\pgfqpoint{0.018417in}{-0.069444in}}{\pgfqpoint{0.036082in}{-0.062127in}}{\pgfqpoint{0.049105in}{-0.049105in}}%
\pgfpathcurveto{\pgfqpoint{0.062127in}{-0.036082in}}{\pgfqpoint{0.069444in}{-0.018417in}}{\pgfqpoint{0.069444in}{0.000000in}}%
\pgfpathcurveto{\pgfqpoint{0.069444in}{0.018417in}}{\pgfqpoint{0.062127in}{0.036082in}}{\pgfqpoint{0.049105in}{0.049105in}}%
\pgfpathcurveto{\pgfqpoint{0.036082in}{0.062127in}}{\pgfqpoint{0.018417in}{0.069444in}}{\pgfqpoint{0.000000in}{0.069444in}}%
\pgfpathcurveto{\pgfqpoint{-0.018417in}{0.069444in}}{\pgfqpoint{-0.036082in}{0.062127in}}{\pgfqpoint{-0.049105in}{0.049105in}}%
\pgfpathcurveto{\pgfqpoint{-0.062127in}{0.036082in}}{\pgfqpoint{-0.069444in}{0.018417in}}{\pgfqpoint{-0.069444in}{0.000000in}}%
\pgfpathcurveto{\pgfqpoint{-0.069444in}{-0.018417in}}{\pgfqpoint{-0.062127in}{-0.036082in}}{\pgfqpoint{-0.049105in}{-0.049105in}}%
\pgfpathcurveto{\pgfqpoint{-0.036082in}{-0.062127in}}{\pgfqpoint{-0.018417in}{-0.069444in}}{\pgfqpoint{0.000000in}{-0.069444in}}%
\pgfpathlineto{\pgfqpoint{0.000000in}{-0.069444in}}%
\pgfpathclose%
\pgfusepath{stroke,fill}%
}%
\begin{pgfscope}%
\pgfsys@transformshift{5.708534in}{5.242117in}%
\pgfsys@useobject{currentmarker}{}%
\end{pgfscope}%
\end{pgfscope}%
\begin{pgfscope}%
\definecolor{textcolor}{rgb}{0.000000,0.000000,0.000000}%
\pgfsetstrokecolor{textcolor}%
\pgfsetfillcolor{textcolor}%
\pgftext[x=5.958534in,y=5.205659in,left,base]{\color{textcolor}\sffamily\fontsize{10.000000}{12.000000}\selectfont Minimum}%
\end{pgfscope}%
\end{pgfpicture}%
\makeatother%
\endgroup%
}
        \caption{3D graf funkcie}
        \label{fig:newton_vpravo}
    \end{subfigure}

    \label{fig:newton_komplet}
\end{figure}



\newpage
\noindent \textbf{Počiatočný bod} $x^{[0]} = [1.5; 0.5]$ \\

\begin{table}[H]
    \centering
    \begin{tabular}{cccc}
        \toprule
        \textbf{Iterácia} & \textbf{Bod } $x^{[k]} = [x;y]$ & \textbf{Hodnota } $f(x^{[k]})$ & \textbf{Norma } $\|\nabla f\|$ \\
        \midrule
        0  & $[1.500000;\; 0.500000]$   & 4.943351 & -- \\
        1  & $[-0.138435;\; 0.925639]$  & 4.524030 & 3.385639 \\
        2  & $[0.396305;\; -0.626467]$  & 1.603598 & 6.566560 \\
        \dots & \dots & \dots & \dots \\
        18 & $[0.389757;\; -0.742198]$ & 1.568998 & 0.006963 \\
        19 & $[0.389926;\; -0.741981]$ & 1.568998 & 0.002284 \\
        20 & $[0.389487;\; -0.741218]$ & 1.568998 & 0.001749 \\
        \dots & \dots & \dots & \dots \\
        25 & $[0.388632;\; -0.741332]$ & 1.568997 & 0.002571 \\
        \bottomrule
    \end{tabular}
    \caption{Priebeh MSG pre $x^{[0]} = [1.5;\;0.5]$.}
\end{table}

Pri tomto počiatočnom bode sa výrazne prejavila nevýhoda absencie reštartov, keďže počet iterácií vzrástol až na 25. Ide o viac než dvojnásobok v porovnaní s verziou s resetovaním, ktorá dosiahla konvergenciu už po 11 krokoch. Bez vynulovania parametra $\beta$ sa smery vyhľadávania stávajú menej efektívnymi, čo vedie k spomaleniu konvergencie v záverečnej fáze v okolí minima, čo je zároveň viditeľné aj na zvýšenej hustote bodov na grafe.


\begin{figure}[H]
    \centering

    \begin{subfigure}{0.48\textwidth}
        \centering
        \resizebox{\linewidth}{!}{\input{grafy/cg_contour_11.pgf}}
        \caption{Pohľad zhora (Vrstevnice)}
        \label{fig:newton_vlavo}
    \end{subfigure}
    \hfill
    \begin{subfigure}{0.48\textwidth}
        \centering
        \resizebox{\linewidth}{!}{\input{grafy/cg_surface_11.pgf}}
        \caption{3D graf funkcie}
        \label{fig:newton_vpravo}
    \end{subfigure}

    \label{fig:newton_komplet}
\end{figure}


\newpage

\begin{table}[htbp]
\centering
\footnotesize
\begin{tabular}{|c|c|c|c|c|c|}
\hline
\textbf{Poč. bod} $x^{[0]}$ & \textbf{Iter (Reset)} & \textbf{Iter (Bez)} & \textbf{Minimum} $\tilde{x}$ & \textbf{Hodnota} & \textbf{Rozdiel} \\ \hline
$[0.0; 0.0]$     & 15 & 20 & $[0.3880; -0.7409]$ & $1,568996$ & $0,001887$ \\ \hline
$[1.5; 0.5]$     & 11 & 25 & $[0.3882; -0.7410]$ & $1,568996$ & $0,001725$ \\ \hline
$[2.0; -2.0]$    & 13 & 22 & $[0.3885; -0.7405]$ & $1,568998$ & $0,001540$ \\ \hline
$[-1.0; -1.0]$   & 12 & 19 & $[0.3879; -0.7412]$ & $1,568999$ & $0,001220$ \\ \hline
$[-1.0; 1.0]$    & 14 & 26 & $[0.3883; -0.7408]$ & $1,568997$ & $0,001650$ \\ \hline
$[2.0; 2.0]$     & 16 & 35 & $[0.3878; -0.7415]$ & $1,569002$ & $0,001910$ \\ \hline
$[-2.0; -2.0]$   & 14 & 21 & $[0.3884; -0.7406]$ & $1,568998$ & $0,001480$ \\ \hline
$[0.5; -1.0]$    & 7  & 9  & $[0.3881; -0.7409]$ & $1,568996$ & $0,000950$ \\ \hline
$[3.0; 0.0]$     & 15 & 28 & $[0.3886; -0.7404]$ & $1,569005$ & $0,001820$ \\ \hline
$[0.0; 2.0]$     & 18 & 31 & $[0.3875; -0.7418]$ & $1,569010$ & $0,001950$ \\ \hline
\end{tabular}
\caption{Porovnanie konvergencie MSG}
\end{table}

Vo všetkých testovaných prípadoch dosiahla verzia s resetom výsledok rýchlejšie. Najvýraznejší rozdiel vidíme pri bode $[2.0; 2.0]$ (16 vs. 35 iterácií) a $[1.5; 0.5]$ (11 vs. 25 iterácií). Bez resetu má metóda tendenciu generovať smery, ktoré po mnohých krokoch strácajú vlastnosť združenosti (kvôli nekvadratickej povahe funkcie), čo spomaľuje postup k minimu.

Zatiaľ čo Newtonova metóda potrebovala na nájdenie minima len cca 3 až 6 krokov, MSG s resetom potrebuje priemerne 11 až 18 krokov. To potvrdzuje teóriu, že MSG (metóda prvého rádu) je pomalšia než Newtonova metóda (metóda druhého rádu), avšak jej výhodou je, že nevyžaduje výpočet a inverziu Hessovej matice.

\end{document}
